%
% Copyright � 2013 Peeter Joot.  All Rights Reserved.
% Licenced as described in the file LICENSE under the root directory of this GIT repository.
%
\makeoproblem{Debye calculation in two dimensions}{condensedMatter:problemSet5:3}{2013 ps5 p3}{
Repeat the Debye theory calculation that we did in class, but for a
two-dimensional lattice.  Assume (quite artificially) that the atoms
are free to move only within the plane, so that there are \(2rN\) degrees
of freedom,  and there is only one transverse acoustic
phonon mode, instead of two as in the three-dimensional calculation.

Show that the low temperature limit of the specific heat at constant
area, per unit area, is:
\begin{eqnarray*}
\cA(T) = 7.213\, \frac{4rN}{A}\,\kB\,\frac{T^2}{\Theta^2},
\end{eqnarray*}
where \(A\) is the area of the crystal, \(rN\) is the number of atoms
in the crystal,
\(\Theta\) is defined by \(\kB\Theta = \hbar\omega_{\txtD}\),
and
\begin{eqnarray*}
\int_{0}^{\infty} \frac{y^3 e^y}{(e^y-1)^2}\,dy \simeq 7.213.
\end{eqnarray*}

} % makeproblem

\makeanswer{condensedMatter:problemSet5:3}{

We first setup the 2D density of states construction as we did for 3D, also employing the periodic relations

\begin{dmath}\label{eqn:condensedMatterProblemSet5Problem3:20}
\begin{aligned}
2 \pi n_x &= L_x q_x \\
2 \pi n_y &= L_y q_y,
\end{aligned}
\end{dmath}

so that a sum over the quantum numbers \(\Bn\) can be approximated as

\begin{dmath}\label{eqn:condensedMatterProblemSet5Problem3:40}
\sum_\Bn
\approx
\int dn_x dn_y
=
\frac{A}{(2\pi)^2 } \int d^2\Bq
=
\frac{A}{(2\pi)^2 } \int d f_\omega d q_\perp
=
\frac{A}{(2\pi)^2 } \int
\frac{df_\omega}{\Abs{\spacegrad_\Bq \omega(\Bq) }} d \omega
=
\int Z(\omega) d\omega
\end{dmath}

This Debye model we have

\begin{dmath}\label{eqn:condensedMatterProblemSet5Problem3:60}
\omega =
\left\{
\begin{array}{l l}
C_{\txtL} q & \quad \mbox{longitudinal acoustic} \\
C_{\txtT} q & \quad \mbox{transverse acoustic}
\end{array}
\right.
\end{dmath}

This gives

\begin{dmath}\label{eqn:condensedMatterProblemSet5Problem3:80}
\int Z(\omega) d\omega
=
\sum_{LA, TA} \frac{A}{(2\pi)^2} \int \frac{d f_\omega }{\Abs{ \spacegrad_\Bq \omega(\Bq)}} d\omega
=
\frac{A}{(2\pi)^2}
\int
\sum_{LA, TA}
\frac{
\mathLabelBox
[
   labelstyle={xshift=2cm},
   linestyle={out=270,in=90, latex-}
]
{
d f_\omega
}
{\(q\) space surface ``area'' element}
}{\frac{d\omega}{dq}}
d\omega
=
\int
\frac{A}{(2\pi)^2}
\lr{
\inv{C_{\txtL}} + \frac{1}{C_{\txtT}}
}
\mathLabelBox
[
   labelstyle={xshift=2cm},
   linestyle={out=270,in=90, latex-}
]
{
2 \pi q
}
{\(= \int df_\omega\)}
d\omega
=
\int
\frac{A}{2\pi}
\lr{
\frac{q}{C_{\txtL}} + \frac{q}{C_{\txtT}}
}
d \omega
=
\int
\frac{A}{2\pi}
\lr{
\frac{1}{C_{\txtL}^2} + \frac{1}{C_{\txtT}^2}
}
\omega d \omega,
\end{dmath}

or
\begin{dmath}\label{eqn:condensedMatterProblemSet5Problem3:160}
Z(\omega) =
\frac{A}{2\pi}
\lr{
\frac{1}{C_{\txtL}^2} + \frac{1}{C_{\txtT}^2}
}
\omega.
\end{dmath}

Define the Debye frequency \(\omega_{\txtD}\) by

\begin{dmath}\label{eqn:condensedMatterProblemSet5Problem3:100}
\int_0^{\omega_{\txtD}} Z(\omega) d\omega = 2 r N
\end{dmath}

\begin{dmath}\label{eqn:condensedMatterProblemSet5Problem3:120}
2 r N
=
\frac{A}{ 2 \pi} \lr{ \inv{C_{\txtL}^2} + \frac{1}{C_{\txtT}^2} } \int_0^{\omega_{\txtD}} \omega d\omega
=
\frac{A}{ 2 \pi} \lr{ \inv{C_{\txtL}^2} + \frac{1}{C_{\txtT}^2} } \inv{2} \omega_{\txtD}^2,
\end{dmath}

or

\begin{dmath}\label{eqn:condensedMatterProblemSet5Problem3:140}
\frac{A}{ 2 \pi} \lr{ \inv{C_{\txtL}^2} + \frac{1}{C_{\txtT}^2} } \omega_{\txtD}^2 = 4 r N.
\end{dmath}

Inserting this Debye frequency into the density of states gives

\begin{dmath}\label{eqn:condensedMatterProblemSet5Problem3:180}
Z(\omega) =
\frac{4 r N \omega}{ \omega_{\txtD}^2 }.
\end{dmath}

We can now start the \textAndIndex{specific heat} calculation

\begin{dmath}\label{eqn:condensedMatterProblemSet5Problem3:200}
\cA(T)
=
\frac{dU}{dT}
=
\frac{d}{dT}
\inv{A}
 \int_0^{\omega_{\txtD}}
\calE(\omega, T)
Z(\omega)
d\omega
=
\frac{4 r N }{ A \omega_{\txtD}^2 }
 \int_0^{\omega_{\txtD}} \frac{d}{dT} \calE(\omega, T) \omega d\omega
=
\frac{4 r N }{ A \omega_{\txtD}^2 }
\frac{d}{dT}
\int_0^{\omega_{\txtD}}
\Hbar \omega
\lr{
\inv{2}
+
\inv
{
e^{\Hbar \omega/\kB T} - 1
}
}
\omega d\omega
=
\frac{4 r N }{ A \omega_{\txtD}^2 }
\int_0^{\omega_{\txtD}}
-\Hbar \omega^2
\inv
{
\lr{ e^{\Hbar \omega/\kB T} - 1 }^2
}
e^{\Hbar \omega/\kB T} \lr{ -\frac{\Hbar \omega}{\kB T^2} }
d\omega
=
\frac{4 r N }{ A \omega_{\txtD}^2 }
\int_0^{\omega_{\txtD}}
\kB^2 T \frac{\Hbar^2 \omega^3}{\kB^3 T^3}
\inv
{
\lr{ e^{\Hbar \omega/\kB T} - 1 }^2
}
e^{\Hbar \omega/\kB T}
d\omega
\end{dmath}

As in class we make substitutions
\begin{subequations}
\begin{dmath}\label{eqn:condensedMatterProblemSet5Problem3:220}
y = \frac{\Hbar \omega}{\kB T}
\end{dmath}
\begin{dmath}\label{eqn:condensedMatterProblemSet5Problem3:240}
d\omega = \frac{\kB T}{\Hbar} dy
\end{dmath}
\begin{dmath}\label{eqn:condensedMatterProblemSet5Problem3:280}
y(\omega_{\txtD}) = \frac{\Hbar \omega_{\txtD}}{ \kB T} = \frac{\kB \Theta}{\kB T} = \frac{\Theta}{T}.
\end{dmath}
\end{subequations}

Inserting these we have

\begin{dmath}\label{eqn:condensedMatterProblemSet5Problem3:300}
\cA(T)
=
\frac{4 r N }{ A \omega_{\txtD}^2 }
\frac{\kB^3 T^2}{\Hbar^2}
\int_0^{\Theta/T}
\frac{
y^3
e^{y}
dy
}
{
\lr{ e^{y} - 1 }^2
}
=
\frac{4 r N \kB T^2}{ A \Theta^2 }
\int_0^{\Theta/T}
\frac{
y^3
e^{y}
dy
}
{
\lr{ e^{y} - 1 }^2
}
\end{dmath}

In the \(\kB T \ll \Hbar \omega_{\txtD} = \kB \Theta\) limit where the integrand is small, the integral limit can be approximated by extension to \(\infty\).  This produces the desired result

\begin{dmath}\label{eqn:condensedMatterProblemSet5Problem3:320}
\cA(T) = 7.213 \frac{4 r N}{A} \kB \frac{T^2}{ \Theta^2 }.
\end{dmath}
}
