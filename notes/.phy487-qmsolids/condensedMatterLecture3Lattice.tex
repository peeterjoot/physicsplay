%
% Copyright © 2013 Peeter Joot.  All Rights Reserved.
% Licenced as described in the file LICENSE under the root directory of this GIT repository.
%
%\section{Crystal structures}
\index{crystal structure}

\section{Periodicity}
\index{periodicity}

\reading \S 2.0 \citep{ibach2009solid}.  Also very helpful is \citep{tung:bravais}.

Most solids prefer a periodic arrangement of their atoms.  This is due to directional bonding, and is easy to see in some cases, as in our diamond tetrahedral pattern of \cref{fig:qmSolidsL3:qmSolidsL3Fig7}.

\imageFigure{../../figures/phy487/qmSolidsL3Fig7}{Diamond tetrahedron.}{fig:qmSolidsL3:qmSolidsL3Fig7}{0.2}

Unproven theorem of no name: lowest energy configuration of atoms in a solid is periodic.

Minimum Coulomb energy for integer ratios of atoms
\ce{Na1Cl1},
\ce{Fe2O3},
\ce{Fe3O4},
\ce{PrOs4Sb12},
\ce{YbCO2Zn20}.

There's a lot of info on ``amorphous/glassy'' materials are \textunderline{not} periodic.  We won't consider these.

\paragraph{Mathematical description of periodicity}
\index{periodicity}

Starting with a 2D lattice as in \cref{fig:qmSolidsL3:qmSolidsL3Fig8}.  Two vectors can generate a lattice (or 2 lengths and 1 angle)

\imageFigure{../../figures/phy487/qmSolidsL3Fig8}{2D lattice}{fig:qmSolidsL3:qmSolidsL3Fig8}{0.2}

We'll assign each atomic center a vector

\begin{dmath}\label{eqn:condensedMatterLecture3:100}
\Br_\Bn = n_1 \Ba + n_2 \Bb
\end{dmath}

Only five cases (in 2D) that are symmetrically distinct that leave no spaces

%\cref{fig:qmSolidsL3:qmSolidsL3Fig9}.
\imageFigure{../../figures/phy487/qmSolidsL3Fig9}{square lattice.  \(a = b\), \(\gamma = \pi/2\).}{fig:qmSolidsL3:qmSolidsL3Fig9}{0.2}

%\cref{fig:qmSolidsL3:qmSolidsL3Fig10}.
\imageFigure{../../figures/phy487/qmSolidsL3Fig10}{
Rectangular.  \(a \ne b\), \(\gamma = \pi/2\).
}{fig:qmSolidsL3:qmSolidsL3Fig10}{0.2}

%\cref{fig:qmSolidsL3:qmSolidsL3Fig11}.
\imageFigure{../../figures/phy487/qmSolidsL3Fig11}{Hexagonal close packed}{fig:qmSolidsL3:qmSolidsL3Fig11}{0.2}

%\cref{fig:qmSolidsL3:qmSolidsL3Fig12}.
\imageFigure{../../figures/phy487/qmSolidsL3Fig12}{
Rhombic, or centered rectangular.  \(a = b\), \(\gamma \ne \pi/2, \pi/3\).
}{fig:qmSolidsL3:qmSolidsL3Fig12}{0.2}

FIXME: No figure -- too fast with the eraser.  \(a \ne b\), \(\gamma = \pi/2\)?

%\EndArticle
