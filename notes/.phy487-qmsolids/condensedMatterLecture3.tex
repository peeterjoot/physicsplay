%
% Copyright � 2013 Peeter Joot.  All Rights Reserved.
% Licenced as described in the file LICENSE under the root directory of this GIT repository.
%
%\newcommand{\authorname}{Peeter Joot}
\newcommand{\email}{peeterjoot@protonmail.com}
\newcommand{\basename}{FIXMEbasenameUndefined}
\newcommand{\dirname}{notes/FIXMEdirnameUndefined/}

%\renewcommand{\basename}{condensedMatterLecture3}
%\renewcommand{\dirname}{notes/phy487/}
%\newcommand{\keywords}{Condensed matter physics, PHY487H1F}
%\newcommand{\authorname}{Peeter Joot}
\newcommand{\onlineurl}{http://sites.google.com/site/peeterjoot2/math2013/\basename.pdf}
\newcommand{\sourcepath}{\dirname\basename.tex}
\newcommand{\generatetitle}[1]{\chapter{#1}}

\newcommand{\vcsinfo}{%
\section*{}
\noindent{\color{DarkOliveGreen}{\rule{\linewidth}{0.1mm}}}
\paragraph{Document version}
%\paragraph{\color{Maroon}{Document version}}
{
\small
\begin{itemize}
\item Available online at:\\ 
\href{\onlineurl}{\onlineurl}
\item Git Repository: \input{./.revinfo/gitRepo.tex}
\item Source: \sourcepath
\item last commit: \input{./.revinfo/gitCommitString.tex}
\item commit date: \input{./.revinfo/gitCommitDate.tex}
\end{itemize}
}
}

%\PassOptionsToPackage{dvipsnames,svgnames}{xcolor}
\PassOptionsToPackage{square,numbers}{natbib}
\documentclass{scrreprt}

\usepackage[left=2cm,right=2cm]{geometry}
\usepackage[svgnames]{xcolor}
\usepackage{peeters_layout}

\usepackage{natbib}

\usepackage[
colorlinks=true,
bookmarks=false,
pdfauthor={\authorname, \email},
backref 
]{hyperref}

% http://tex.stackexchange.com/questions/75773/how-to-reference-problems-by-the-text-label-in-an-exercise-envioronment
\usepackage[english]{cleveref}
\crefname{Exercise}{exercise}{exercises}
\Crefname{Exercise}{Exercise}{Exercises}

\RequirePackage{titlesec}
\RequirePackage{ifthen}

% http://stackoverflow.com/questions/4932910/date-in-the-tabular-environment
\makeatletter
\let\insertdate\@date
\makeatother

\titleformat{\chapter}[display]
{\bfseries\Large}
{\color{DarkSlateGrey}\filleft \authorname
\ifthenelse{\isundefined{\studentnumber}}{}{\\ \studentnumber}
\ifthenelse{\isundefined{\email}}{}{\\ \email}
\ifthenelse{\isundefined{\dateintitle}}{}{\\ \insertdate}
%\ifthenelse{\isundefined{\coursename}}{}{\\ \coursename} % put in title instead.
}
{4ex}
{\color{DarkOliveGreen}{\titlerule}\color{Maroon}
\vspace{2ex}%
\filright}
[\vspace{2ex}%
\color{DarkOliveGreen}\titlerule
]

\newcommand{\beginArtWithToc}[0]{\begin{document}\tableofcontents}
\newcommand{\beginArtNoToc}[0]{\begin{document}}
\newcommand{\EndNoBibArticle}[0]{\end{document}}
\newcommand{\EndArticle}[0]{\bibliography{Bibliography}\bibliographystyle{plainnat}\end{document}}

% 
%\newcommand{\citep}[1]{\cite{#1}}

\colorSectionsForArticle


%
%%\citep{harald2003solid} \S x.y
%%\citep{ibach2009solid} \S x.y
%
%\usepackage{mhchem}
%\usepackage{amssymb}
%
%\beginArtNoToc
%\generatetitle{PHY487H1F Condensed Matter Physics.  Lecture 3: Bonding and lattice structure.  Taught by Prof.\ Stephen Julian}
%%\chapter{Bonding and lattice structure}
%\label{chap:condensedMatterLecture3}
%
%\section{Disclaimer}
%
%Peeter's lecture notes from class.  May not be entirely coherent.
%
%\section{Bonding (cont.)}

%\section{Ionic bonding}
\index{ionic bonding}
\index{lattice structure}

We introduce the \underlineAndIndex{Madelung constant} \(A\) for the potential energy of the solid configuration

\begin{dmath}\label{eqn:condensedMatterLecture3:20}
\Phi_{tot} = \sum_i \phi_i = \inv{2} \sum_{i \ne j} \phi_{ij}
= \inv{2}
\mathLabelBox
[
   labelstyle={xshift=-2cm},
   linestyle={out=270,in=90, latex-}
]
{N}{number of ions in solid}
\Bigl(
-\frac{e^2}{4 \pi \epsilon_\nought r}
\mathLabelBox
[
   labelstyle={yshift=0.3cm},
   linestyle={out=270,in=90, latex-}
]
{
\sum_{i \ne j} \frac{\lr{\pm 1}}{ p_{ij} }
}{A}
+ \frac{B}{r^n}
\sum_{i \ne j} \inv{p_{ij}^n}
\Bigr)
\end{dmath}

Note that \(r\) is the nn separation (center to center), and \(r_{ij} = p_{ij} r\), as illustrated in \cref{fig:qmSolidsL3:qmSolidsL3Fig1}.

\imageFigure{../../figures/phy487/qmSolidsL3Fig1}{Madelung separation}{fig:qmSolidsL3:qmSolidsL3Fig1}{0.1}

As an additional illustration, we have the \ce{NaCl} configuration in \cref{fig:qmSolidsL3:qmSolidsL3Fig2}.

\imageFigure{../../figures/phy487/qmSolidsL3Fig2}{\ce{NaCl} lattice separation}{fig:qmSolidsL3:qmSolidsL3Fig2}{0.2}

\examhint{Eminently examinable material (since it can be calculated).}

Examples
\begin{itemize}
\item \ce{NaCl} structure \(A = 1.748\)
\item \ce{CsCl} structure \(A = 1.763\)
\end{itemize}

Ionic bonds are weaker than covalent, non-directional.  One indicator of this is the melting points

\begin{subequations}
\begin{dmath}\label{eqn:condensedMatterLecture3:40}
T_m = 1074 K \quad \mbox{\ce{NaCl}}
\end{dmath}
\begin{dmath}\label{eqn:condensedMatterLecture3:60}
T_m = 918 K \quad \mbox{\ce{CsCl}}
\end{dmath}
\end{subequations}

\section{Metallic bonding}
\index{metallic bonding}

We now focus on the regions of the periodic table illustrated in \cref{fig:qmSolidsL3:qmSolidsL3Fig3}.

\imageFigure{../../figures/phy487/qmSolidsL3Fig3}{Metallic bonding regions in the periodic table}{fig:qmSolidsL3:qmSolidsL3Fig3}{0.2}

Curiously, the name is somewhat misleading.  Just because something is a metal doesn't mean it is metallic bonded.

\begin{itemize}
\item s-orbitals from \(2s\) to \(5s\)
\item p-orbitals from \(n = 4, 5, 6, 7\)
\end{itemize}

These are big orbitals that extend beyond the nn, as illustrated in \cref{fig:qmSolidsL3:qmSolidsL3Fig4}.

\imageFigure{../../figures/phy487/qmSolidsL3Fig4}{Extensive wave function}{fig:qmSolidsL3:qmSolidsL3Fig4}{0.2}

Weakly bound electrons overlap many nearby potential wells.  This lowers the Coulomb energy.  This is like a non-directional covalent bond.  This non-directionality results in malleability.

Pure metallic bonds are weak.  Melting points are correspondingly low, where for column 1 elements we have \(T_m = \mbox{room temperature to \(200^\circ\) C}\).

\section{Transition metals}
\index{transition metals}

Here we have both metallically bonded s-orbitals and covalent d-orbitals \cref{fig:qmSolidsL3:qmSolidsL3Fig6}.

\imageFigure{../../figures/phy487/qmSolidsL3Fig6}{Overlapping \(s\) and \(p\) orbitals}{fig:qmSolidsL3:qmSolidsL3Fig6}{0.2}

%\cref{fig:qmSolidsL3:qmSolidsL3Fig5}.
%\imageFigure{../../figures/phy487/qmSolidsL3Fig5}{Overlapping orbitals}{fig:qmSolidsL3:qmSolidsL3Fig5}{0.2}

\begin{dmath}\label{eqn:condensedMatterLecture3:80}
T_m \gtrsim 2000^\circ C.
\end{dmath}

\paragraph{On sign conventions.}  The \(+, -\)'s assume real representation of wave functions.  Here bonding is matching signs (constructive interference), and antibonding is when the signs are in opposition (destructive interference).

\reading \S 1.5, 1.6 \citep{ibach2009solid}.

