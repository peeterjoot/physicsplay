%
% Copyright © 2016 Peeter Joot.  All Rights Reserved.
% Licenced as described in the file LICENSE under the root directory of this GIT repository.
%

\makeproblem{Derive the Thomas-Fermi screening length}{pr:condensedMatterLecture13:1}{
Find the solution to \eqnref{eqn:condensedMatterLecture13:460}, thus finding the Thomas-Fermi screening length expression of
\eqnref{eqn:condensedMatterLecture13:500}.
} % makeproblem

\makeanswer{pr:condensedMatterLecture13:1}{

We wish to solve an equation of the form

\begin{dmath}\label{eqn:condensedMatterLecture13P1:560}
\inv{r^2} \frac{d}{dr} \lr{ r^2 \frac{df}{dr} } = a f.
\end{dmath}

We can make this more tractable with a substitution

\begin{dmath}\label{eqn:condensedMatterLecture13P1:580}
g = r f,
\end{dmath}

for which we have
\begin{dmath}\label{eqn:condensedMatterLecture13P1:600}
r \frac{dg}{dr} = r \lr{ f + r \frac{df}{dr} },
\end{dmath}

or

\begin{equation}\label{eqn:condensedMatterLecture13P1:620}
r^2 \frac{df}{dr} =
r \frac{dg}{dr} - r f
=
r \frac{dg}{dr} - g.
\end{equation}

This reduces our equation to

\begin{dmath}\label{eqn:condensedMatterLecture13P1:640}
\begin{aligned}
\frac{d}{dr} \lr{
r \frac{dg}{dr} - g } &= a r g \\
\cancel{\frac{dg}{dr}} + r \frac{d^2 g}{dr^2}  - \cancel{\frac{dg}{dr}} &=
\end{aligned}
\end{dmath}

or just

\begin{dmath}\label{eqn:condensedMatterLecture13P1:660}
\frac{d^2 g}{dr^2} = a g.
\end{dmath}

This has the solution

\begin{dmath}\label{eqn:condensedMatterLecture13P1:680}
g = \sum_{\pm} A_{\pm} e^{\pm \sqrt{a} r},
\end{dmath}

Picking the non-hyperbolic solution, that is

\begin{dmath}\label{eqn:condensedMatterLecture13P1:700}
f \propto \frac{e^{-\sqrt{a} r}}{r}.
\end{dmath}

With \(a = e^2 D(\EF)/\epsilon_\nought = 1/r_{\mathrm{TF}}^2\), we find \eqnref{eqn:condensedMatterLecture13:500} as desired.
} % makeanswer
