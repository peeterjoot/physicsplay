%
% Copyright � 2012 Peeter Joot.  All Rights Reserved.
% Licenced as described in the file LICENSE under the root directory of this GIT repository.
%

% 
% 
%%
% Copyright � 2015 Peeter Joot.  All Rights Reserved.
% Licenced as described in the file LICENSE under the root directory of this GIT repository.
%
\documentclass[]{eliblog}

\usepackage{amsmath}
\usepackage{mathpazo}

%
% shorthand for bold symbols, convenient for vectors and matrices
%
\newcommand{\Ba}[0]{\mathbf{a}}
\newcommand{\Bb}[0]{\mathbf{b}}
\newcommand{\Bc}[0]{\mathbf{c}}
\newcommand{\Bd}[0]{\mathbf{d}}
\newcommand{\Be}[0]{\mathbf{e}}
\newcommand{\Bf}[0]{\mathbf{f}}
\newcommand{\Bg}[0]{\mathbf{g}}
\newcommand{\Bh}[0]{\mathbf{h}}
\newcommand{\Bi}[0]{\mathbf{i}}
\newcommand{\Bj}[0]{\mathbf{j}}
\newcommand{\Bk}[0]{\mathbf{k}}
\newcommand{\Bl}[0]{\mathbf{l}}
\newcommand{\Bm}[0]{\mathbf{m}}
\newcommand{\Bn}[0]{\mathbf{n}}
\newcommand{\Bo}[0]{\mathbf{o}}
\newcommand{\Bp}[0]{\mathbf{p}}
\newcommand{\Bq}[0]{\mathbf{q}}
\newcommand{\Br}[0]{\mathbf{r}}
\newcommand{\Bs}[0]{\mathbf{s}}
\newcommand{\Bt}[0]{\mathbf{t}}
\newcommand{\Bu}[0]{\mathbf{u}}
\newcommand{\Bv}[0]{\mathbf{v}}
\newcommand{\Bw}[0]{\mathbf{w}}
\newcommand{\Bx}[0]{\mathbf{x}}
\newcommand{\By}[0]{\mathbf{y}}
\newcommand{\Bz}[0]{\mathbf{z}}
\newcommand{\BA}[0]{\mathbf{A}}
\newcommand{\BB}[0]{\mathbf{B}}
\newcommand{\BC}[0]{\mathbf{C}}
\newcommand{\BD}[0]{\mathbf{D}}
\newcommand{\BE}[0]{\mathbf{E}}
\newcommand{\BF}[0]{\mathbf{F}}
\newcommand{\BG}[0]{\mathbf{G}}
\newcommand{\BH}[0]{\mathbf{H}}
\newcommand{\BI}[0]{\mathbf{I}}
\newcommand{\BJ}[0]{\mathbf{J}}
\newcommand{\BK}[0]{\mathbf{K}}
\newcommand{\BL}[0]{\mathbf{L}}
\newcommand{\BM}[0]{\mathbf{M}}
\newcommand{\BN}[0]{\mathbf{N}}
\newcommand{\BO}[0]{\mathbf{O}}
\newcommand{\BP}[0]{\mathbf{P}}
\newcommand{\BQ}[0]{\mathbf{Q}}
\newcommand{\BR}[0]{\mathbf{R}}
\newcommand{\BS}[0]{\mathbf{S}}
\newcommand{\BT}[0]{\mathbf{T}}
\newcommand{\BU}[0]{\mathbf{U}}
\newcommand{\BV}[0]{\mathbf{V}}
\newcommand{\BW}[0]{\mathbf{W}}
\newcommand{\BX}[0]{\mathbf{X}}
\newcommand{\BY}[0]{\mathbf{Y}}
\newcommand{\BZ}[0]{\mathbf{Z}}

\newcommand{\Bzero}[0]{\mathbf{0}}
\newcommand{\Btheta}[0]{\boldsymbol{\theta}}
\newcommand{\Btau}[0]{\boldsymbol{\tau}}
\newcommand{\Bomega}[0]{\boldsymbol{\omega}}

%
% shorthand for unit vectors
%
\newcommand{\acap}[0]{\hat{\Ba}}
\newcommand{\bcap}[0]{\hat{\Bb}}
\newcommand{\ccap}[0]{\hat{\Bc}}
\newcommand{\dcap}[0]{\hat{\Bd}}
\newcommand{\ecap}[0]{\hat{\Be}}
\newcommand{\fcap}[0]{\hat{\Bf}}
\newcommand{\gcap}[0]{\hat{\Bg}}
\newcommand{\hcap}[0]{\hat{\Bh}}
\newcommand{\icap}[0]{\hat{\Bi}}
\newcommand{\jcap}[0]{\hat{\Bj}}
\newcommand{\kcap}[0]{\hat{\Bk}}
\newcommand{\lcap}[0]{\hat{\Bl}}
\newcommand{\mcap}[0]{\hat{\Bm}}
\newcommand{\ncap}[0]{\hat{\Bn}}
\newcommand{\ocap}[0]{\hat{\Bo}}
\newcommand{\pcap}[0]{\hat{\Bp}}
\newcommand{\qcap}[0]{\hat{\Bq}}
\newcommand{\rcap}[0]{\hat{\Br}}
\newcommand{\scap}[0]{\hat{\Bs}}
\newcommand{\tcap}[0]{\hat{\Bt}}
\newcommand{\ucap}[0]{\hat{\Bu}}
\newcommand{\vcap}[0]{\hat{\Bv}}
\newcommand{\wcap}[0]{\hat{\Bw}}
\newcommand{\xcap}[0]{\hat{\Bx}}
\newcommand{\ycap}[0]{\hat{\By}}
\newcommand{\zcap}[0]{\hat{\Bz}}
\newcommand{\thetacap}[0]{\hat{\Btheta}}

%
% to write R^n and C^n in a distinguishable fashion.  Perhaps change this
% to the double lined characters upon figuring out how to do so.
%
\newcommand{\C}[1]{$\mathbb{C}^{#1}$}
\newcommand{\R}[1]{$\mathbb{R}^{#1}$}

%
% various generally useful helpers
%

% derivative of #1 wrt. #2:
\newcommand{\D}[2] {\frac {d#2} {d#1}}

\newcommand{\inv}[1]{\frac{1}{#1}}
\newcommand{\cross}[0]{\times}

\newcommand{\abs}[1]{\lvert{#1}\rvert}
\newcommand{\norm}[1]{\lVert{#1}\rVert}
\newcommand{\innerprod}[2]{\langle{#1}, {#2}\rangle}
\newcommand{\dotprod}[2]{{#1} \cdot {#2}}
\newcommand{\bdotprod}[2]{\left({#1} \cdot {#2}\right)}
\newcommand{\crossprod}[2]{{#1} \cross {#2}}
\newcommand{\tripleprod}[3]{\dotprod{\left(\crossprod{#1}{#2}\right)}{#3}}

\DeclareMathOperator{\Proj}{Proj}
\DeclareMathOperator{\Span}{span}
\DeclareMathOperator{\Sgn}{sgn}
\DeclareMathOperator{\Area}{Area}
\DeclareMathOperator{\Volume}{Volume}

%
% A few miscellaneous things specific to this document
%
\newcommand{\crossop}[1]{\crossprod{#1}{}}

% R2 vector.
\newcommand{\VectorTwo}[2]{
\begin{bmatrix}
 {#1} \\
 {#2}
\end{bmatrix}
}

\newcommand{\VectorN}[1]{
\begin{bmatrix}
{#1}_1 \\
{#1}_2 \\
\vdots \\
{#1}_N \\
\end{bmatrix}
}

\newcommand{\DETuvij}[4]{
\begin{vmatrix}
 {#1}_{#3} & {#1}_{#4} \\
 {#2}_{#3} & {#2}_{#4}
\end{vmatrix}
}

\newcommand{\DETuvwijk}[6]{
\begin{vmatrix}
 {#1}_{#4} & {#1}_{#5} & {#1}_{#6} \\
 {#2}_{#4} & {#2}_{#5} & {#2}_{#6} \\
 {#3}_{#4} & {#3}_{#5} & {#3}_{#6}
\end{vmatrix}
}

\newcommand{\DETuvwxijkl}[8]{
\begin{vmatrix}
 {#1}_{#5} & {#1}_{#6} & {#1}_{#7} & {#1}_{#8} \\
 {#2}_{#5} & {#2}_{#6} & {#2}_{#7} & {#2}_{#8} \\
 {#3}_{#5} & {#3}_{#6} & {#3}_{#7} & {#3}_{#8} \\
 {#4}_{#5} & {#4}_{#6} & {#4}_{#7} & {#4}_{#8} \\
\end{vmatrix}
}

%\newcommand{\DETuvwxyijklm}[10]{
%\begin{vmatrix}
% {#1}_{#6} & {#1}_{#7} & {#1}_{#8} & {#1}_{#9} & {#1}_{#10} \\
% {#2}_{#6} & {#2}_{#7} & {#2}_{#8} & {#2}_{#9} & {#2}_{#10} \\
% {#3}_{#6} & {#3}_{#7} & {#3}_{#8} & {#3}_{#9} & {#3}_{#10} \\
% {#4}_{#6} & {#4}_{#7} & {#4}_{#8} & {#4}_{#9} & {#4}_{#10} \\
% {#5}_{#6} & {#5}_{#7} & {#5}_{#8} & {#5}_{#9} & {#5}_{#10}
%\end{vmatrix}
%}

% R3 vector.
\newcommand{\VectorThree}[3]{
\begin{bmatrix}
 {#1} \\
 {#2} \\
 {#3}
\end{bmatrix}
}



\author{Peeter Joot}
\email{peeter.joot@gmail.com}

%\documentclass[]{eliblogwidescreen}

\usepackage{amsmath}
\usepackage{mathpazo}

%
% shorthand for bold symbols, convenient for vectors and matrices
%
\newcommand{\Ba}[0]{\mathbf{a}}
\newcommand{\Bb}[0]{\mathbf{b}}
\newcommand{\Bc}[0]{\mathbf{c}}
\newcommand{\Bd}[0]{\mathbf{d}}
\newcommand{\Be}[0]{\mathbf{e}}
\newcommand{\Bf}[0]{\mathbf{f}}
\newcommand{\Bg}[0]{\mathbf{g}}
\newcommand{\Bh}[0]{\mathbf{h}}
\newcommand{\Bi}[0]{\mathbf{i}}
\newcommand{\Bj}[0]{\mathbf{j}}
\newcommand{\Bk}[0]{\mathbf{k}}
\newcommand{\Bl}[0]{\mathbf{l}}
\newcommand{\Bm}[0]{\mathbf{m}}
\newcommand{\Bn}[0]{\mathbf{n}}
\newcommand{\Bo}[0]{\mathbf{o}}
\newcommand{\Bp}[0]{\mathbf{p}}
\newcommand{\Bq}[0]{\mathbf{q}}
\newcommand{\Br}[0]{\mathbf{r}}
\newcommand{\Bs}[0]{\mathbf{s}}
\newcommand{\Bt}[0]{\mathbf{t}}
\newcommand{\Bu}[0]{\mathbf{u}}
\newcommand{\Bv}[0]{\mathbf{v}}
\newcommand{\Bw}[0]{\mathbf{w}}
\newcommand{\Bx}[0]{\mathbf{x}}
\newcommand{\By}[0]{\mathbf{y}}
\newcommand{\Bz}[0]{\mathbf{z}}
\newcommand{\BA}[0]{\mathbf{A}}
\newcommand{\BB}[0]{\mathbf{B}}
\newcommand{\BC}[0]{\mathbf{C}}
\newcommand{\BD}[0]{\mathbf{D}}
\newcommand{\BE}[0]{\mathbf{E}}
\newcommand{\BF}[0]{\mathbf{F}}
\newcommand{\BG}[0]{\mathbf{G}}
\newcommand{\BH}[0]{\mathbf{H}}
\newcommand{\BI}[0]{\mathbf{I}}
\newcommand{\BJ}[0]{\mathbf{J}}
\newcommand{\BK}[0]{\mathbf{K}}
\newcommand{\BL}[0]{\mathbf{L}}
\newcommand{\BM}[0]{\mathbf{M}}
\newcommand{\BN}[0]{\mathbf{N}}
\newcommand{\BO}[0]{\mathbf{O}}
\newcommand{\BP}[0]{\mathbf{P}}
\newcommand{\BQ}[0]{\mathbf{Q}}
\newcommand{\BR}[0]{\mathbf{R}}
\newcommand{\BS}[0]{\mathbf{S}}
\newcommand{\BT}[0]{\mathbf{T}}
\newcommand{\BU}[0]{\mathbf{U}}
\newcommand{\BV}[0]{\mathbf{V}}
\newcommand{\BW}[0]{\mathbf{W}}
\newcommand{\BX}[0]{\mathbf{X}}
\newcommand{\BY}[0]{\mathbf{Y}}
\newcommand{\BZ}[0]{\mathbf{Z}}

\newcommand{\Bzero}[0]{\mathbf{0}}
\newcommand{\Btheta}[0]{\boldsymbol{\theta}}
\newcommand{\Btau}[0]{\boldsymbol{\tau}}
\newcommand{\Bomega}[0]{\boldsymbol{\omega}}

%
% shorthand for unit vectors
%
\newcommand{\acap}[0]{\hat{\Ba}}
\newcommand{\bcap}[0]{\hat{\Bb}}
\newcommand{\ccap}[0]{\hat{\Bc}}
\newcommand{\dcap}[0]{\hat{\Bd}}
\newcommand{\ecap}[0]{\hat{\Be}}
\newcommand{\fcap}[0]{\hat{\Bf}}
\newcommand{\gcap}[0]{\hat{\Bg}}
\newcommand{\hcap}[0]{\hat{\Bh}}
\newcommand{\icap}[0]{\hat{\Bi}}
\newcommand{\jcap}[0]{\hat{\Bj}}
\newcommand{\kcap}[0]{\hat{\Bk}}
\newcommand{\lcap}[0]{\hat{\Bl}}
\newcommand{\mcap}[0]{\hat{\Bm}}
\newcommand{\ncap}[0]{\hat{\Bn}}
\newcommand{\ocap}[0]{\hat{\Bo}}
\newcommand{\pcap}[0]{\hat{\Bp}}
\newcommand{\qcap}[0]{\hat{\Bq}}
\newcommand{\rcap}[0]{\hat{\Br}}
\newcommand{\scap}[0]{\hat{\Bs}}
\newcommand{\tcap}[0]{\hat{\Bt}}
\newcommand{\ucap}[0]{\hat{\Bu}}
\newcommand{\vcap}[0]{\hat{\Bv}}
\newcommand{\wcap}[0]{\hat{\Bw}}
\newcommand{\xcap}[0]{\hat{\Bx}}
\newcommand{\ycap}[0]{\hat{\By}}
\newcommand{\zcap}[0]{\hat{\Bz}}
\newcommand{\thetacap}[0]{\hat{\Btheta}}

%
% to write R^n and C^n in a distinguishable fashion.  Perhaps change this
% to the double lined characters upon figuring out how to do so.
%
\newcommand{\C}[1]{$\mathbb{C}^{#1}$}
\newcommand{\R}[1]{$\mathbb{R}^{#1}$}

%
% various generally useful helpers
%

% derivative of #1 wrt. #2:
\newcommand{\D}[2] {\frac {d#2} {d#1}}

\newcommand{\inv}[1]{\frac{1}{#1}}
\newcommand{\cross}[0]{\times}

\newcommand{\abs}[1]{\lvert{#1}\rvert}
\newcommand{\norm}[1]{\lVert{#1}\rVert}
\newcommand{\innerprod}[2]{\langle{#1}, {#2}\rangle}
\newcommand{\dotprod}[2]{{#1} \cdot {#2}}
\newcommand{\bdotprod}[2]{\left({#1} \cdot {#2}\right)}
\newcommand{\crossprod}[2]{{#1} \cross {#2}}
\newcommand{\tripleprod}[3]{\dotprod{\left(\crossprod{#1}{#2}\right)}{#3}}

\DeclareMathOperator{\Proj}{Proj}
\DeclareMathOperator{\Span}{span}
\DeclareMathOperator{\Sgn}{sgn}
\DeclareMathOperator{\Area}{Area}
\DeclareMathOperator{\Volume}{Volume}

%
% A few miscellaneous things specific to this document
%
\newcommand{\crossop}[1]{\crossprod{#1}{}}

% R2 vector.
\newcommand{\VectorTwo}[2]{
\begin{bmatrix}
 {#1} \\
 {#2}
\end{bmatrix}
}

\newcommand{\VectorN}[1]{
\begin{bmatrix}
{#1}_1 \\
{#1}_2 \\
\vdots \\
{#1}_N \\
\end{bmatrix}
}

\newcommand{\DETuvij}[4]{
\begin{vmatrix}
 {#1}_{#3} & {#1}_{#4} \\
 {#2}_{#3} & {#2}_{#4}
\end{vmatrix}
}

\newcommand{\DETuvwijk}[6]{
\begin{vmatrix}
 {#1}_{#4} & {#1}_{#5} & {#1}_{#6} \\
 {#2}_{#4} & {#2}_{#5} & {#2}_{#6} \\
 {#3}_{#4} & {#3}_{#5} & {#3}_{#6}
\end{vmatrix}
}

\newcommand{\DETuvwxijkl}[8]{
\begin{vmatrix}
 {#1}_{#5} & {#1}_{#6} & {#1}_{#7} & {#1}_{#8} \\
 {#2}_{#5} & {#2}_{#6} & {#2}_{#7} & {#2}_{#8} \\
 {#3}_{#5} & {#3}_{#6} & {#3}_{#7} & {#3}_{#8} \\
 {#4}_{#5} & {#4}_{#6} & {#4}_{#7} & {#4}_{#8} \\
\end{vmatrix}
}

%\newcommand{\DETuvwxyijklm}[10]{
%\begin{vmatrix}
% {#1}_{#6} & {#1}_{#7} & {#1}_{#8} & {#1}_{#9} & {#1}_{#10} \\
% {#2}_{#6} & {#2}_{#7} & {#2}_{#8} & {#2}_{#9} & {#2}_{#10} \\
% {#3}_{#6} & {#3}_{#7} & {#3}_{#8} & {#3}_{#9} & {#3}_{#10} \\
% {#4}_{#6} & {#4}_{#7} & {#4}_{#8} & {#4}_{#9} & {#4}_{#10} \\
% {#5}_{#6} & {#5}_{#7} & {#5}_{#8} & {#5}_{#9} & {#5}_{#10}
%\end{vmatrix}
%}

% R3 vector.
\newcommand{\VectorThree}[3]{
\begin{bmatrix}
 {#1} \\
 {#2} \\
 {#3}
\end{bmatrix}
}



\author{Peeter Joot}
\email{peeter.joot@gmail.com}


\chapter{WKB method and Stark shift.}
%\chapter{PHY456H1F: Quantum Mechanics II.  Recitation 3 (Taught by Mr. Federico Duque Gomez).  WKB method and Stark shift.}
\label{chap:qmTwoR3}
%\useCCL
\blogpage{http://sites.google.com/site/peeterjoot/math2011/qmTwoR3.pdf}
\date{Oct 28, 2011}
\revisionInfo{qmTwoR3.tex}

\beginArtWithToc
%\beginArtNoToc

%\section{Disclaimer.}
%
%Peeter's lecture notes from class.  May not be entirely coherent.
%
\section{WKB method.}

Consider the potential

\begin{equation}\label{eqn:qmTwoR3:10}
V(x) = 
\left\{
\begin{array}{l l}
v(x) & \quad \mbox{if $x \in [0,a]$} \\
\infty & \quad \mbox{otherwise} \\
\end{array}
\right.
\end{equation}

as illustrated in figure (\ref{fig:qmTwoR3:qmTwoR3fig1})
\imageFigure{figures/qmTwoR3fig1}{Arbitrary potential in an infinite well.}{fig:qmTwoR3:qmTwoR3fig1}{0.4}

Inside the well, we have

\begin{equation}\label{eqn:qmTwoR3:30}
\psi(x) = \inv{\sqrt{k(x)}} \left( 
C_{+} e^{i \int_0^x k(x') dx'}
+C_{-} e^{-i \int_0^x k(x') dx'}
\right)
\end{equation}

where

\begin{equation}\label{eqn:qmTwoR3:50}
k(x) = \inv{\hbar} \sqrt{ 2m( E - v(x) }
\end{equation}

With
\begin{equation}\label{eqn:qmTwoR3:70}
\phi(x) = e^{\int_0^x k(x') dx'}
\end{equation}

We have

\begin{align*}
\psi(x) 
&= \inv{\sqrt{k(x)}} \left( 
C_{+}(\cos \phi + i\sin\phi) + C_{-}(\cos\phi - i \sin\phi)
\right) \\
&= \inv{\sqrt{k(x)}} \left( 
(C_{+} + C_{-})\cos \phi + i(C_{+} - C_{-}) \sin\phi
\right) \\
&= \inv{\sqrt{k(x)}} \left( 
(C_{+} + C_{-})\cos \phi + i(C_{+} - C_{-}) \sin\phi
\right) \\
&\equiv 
\inv{\sqrt{k(x)}} \left( 
C_2 \cos \phi + C_1 \sin\phi
\right),
\end{align*}

Where
\begin{align}\label{eqn:qmTwoR3:85}
C_2 &= C_{+} + C_{-} \\
C_1 &= i( C_{+} - C_{-})
\end{align}

Setting boundary conditions we have

\begin{equation}\label{eqn:qmTwoR3:90}
\phi(0) = 0
\end{equation}

Noting that we have $\phi(0) = 0$, we have

\begin{equation}\label{eqn:qmTwoR3:110}
\inv{\sqrt{k(0)}} C_2 = 0
\end{equation}

So

\begin{equation}\label{eqn:qmTwoR3:130}
\psi(x) 
\sim
\inv{\sqrt{k(x)}} \sin\phi
\end{equation}

At the other boundary

\begin{equation}\label{eqn:qmTwoR3:150}
\psi(a) = 0
\end{equation}

So we require

\begin{equation}\label{eqn:qmTwoR3:170}
\sin \phi(a) = \sin(n \pi)
\end{equation}

or

\begin{equation}\label{eqn:qmTwoR3:190}
\inv{\hbar} \int_0^a \sqrt{2 m (E - v(x')} dx' = n \pi
\end{equation}

This is called the Bohr-Sommerfeld condition.

\paragraph{Check} with $v(x) = 0$.

We have

\begin{equation}\label{eqn:qmTwoR3:210}
\inv{\hbar} \sqrt{2m E} a = n \pi
\end{equation}

or

\begin{equation}\label{eqn:qmTwoR3:230}
E = \inv{2m} \left(\frac{n \pi \hbar}{a}\right)^2
\end{equation}

\section{Stark Shift}

Reading: \S 16.5 of \cite{desai2009quantum}.

Time independent perturbation theory

\begin{equation}\label{eqn:qmTwoR3:250}
H = H_0 + \lambda H'
\end{equation}

\begin{equation}\label{eqn:qmTwoR3:270}
H' = e \mathcal{E}_z \hat{Z}
\end{equation}

where $\mathcal{E}_z$ is the electric field.

To first order we have

%{(1)} = {(1)}
\begin{equation}\label{eqn:qmTwoR3:290}
\ket{\psi_\alpha^{(1)}} = \ket{\psi_\alpha^{(0)}} 
+ 
\sum_{\beta \ne \alpha} \frac{ 
\ket{\psi_\beta^{(0)}} \bra{\psi_\beta^{(0)}} H' \ket{\psi_\alpha^{(0)}} 
}{
E_\alpha^{(0)} 
-E_\beta^{(0)} 
}
\end{equation}

and 

\begin{equation}\label{eqn:qmTwoR3:310}
E_\alpha^{(1)} = 
\bra{\psi_\alpha^{(0)}} H' \ket{\psi_\alpha^{(0)}} 
\end{equation}

With the default basis $\{\ket{\psi_\beta^{(0)}}\}$, and $n=2$ we have a 4 fold degeneracy

\begin{align*}
l,m &= 0,0 \\
l,m &= 1,-1 \\
l,m &= 1,0 \\
l,m &= 1,+1
\end{align*}

but can diagonalize as follows

\begin{equation}\label{eqn:qmTwoR3:570}
\begin{bmatrix}
\text{nlm} & 200 & 210 & 211 & 21\,-1 \\
200    & 0 & \Delta & 0 & 0 \\
210    & \Delta & 0 & 0 & 0 \\
211    & 0 & 0 & 0 & 0 \\
21\,-1 & 0 & 0 & 0 & 0
\end{bmatrix}
\end{equation}

FIXME: show.

where
\begin{equation}\label{eqn:qmTwoR3:590}
\Delta = -3 e \mathcal{E}_z a_0
\end{equation}

We have a split of energy levels as illustrated in figure (\ref{fig:qmTwoR3:qmTwoR3fig2})

\imageFigure{figures/qmTwoR3fig2}{Energy level splitting}{fig:qmTwoR3:qmTwoR3fig2}{0.4}

Observe the embedded Pauli matrix (FIXME: missed the point of this?)

\begin{equation}\label{eqn:qmTwoR3:330}
\sigma_x = \PauliX
\end{equation}

Proper basis for perturbation (FIXME:check) is then

\begin{equation}\label{eqn:qmTwoR3:350}
\left\{
\inv{\sqrt{2}}
( 
\ket{2,0,0} 
\pm 
\ket{2,1,0} 
), 
\ket{2, 1, \pm 1}
\right\}
\end{equation}

and our result is

\begin{equation}\label{eqn:qmTwoR3:370}
\ket{\psi_{\alpha, n=2}^{(1)}} = 
\ket{\psi_{\alpha}^{(0)}} 
+\sum_{\beta \notin \text{degenerate subspace}} \frac{ 
\ket{\psi_\beta^{(0)}} \bra{\psi_\beta^{(0)}} H' \ket{\psi_\alpha^{(0)}} 
}{
E_\alpha^{(0)} 
-E_\beta^{(0)} 
}
\end{equation}

\section{Adiabatic perturbation theory}

Utilizing instantaneous eigenstates

\begin{equation}\label{eqn:qmTwoR3:390}
\ket{\psi(t)} = \sum_{\alpha} b_\alpha(t) \ket{\hat{\psi}_\alpha(t)}
\end{equation}

where
\begin{equation}\label{eqn:qmTwoR3:410}
H(t) \ket{\hat{\psi}_\alpha(t)}
= E_\alpha(t) \ket{\hat{\psi}_\alpha(t)}
\end{equation}

We found

\begin{equation}\label{eqn:qmTwoR3:430}
b_\alpha(t) = \bar{b}_\alpha(t) e^{-\frac{i}{\hbar} \int_0^t (E_\alpha(t') - \hbar \Gamma_\alpha(t')) dt'}
\end{equation}

where
\begin{equation}\label{eqn:qmTwoR3:450}
\Gamma_\alpha = i
\bra{\hat{\psi}_\alpha(t)}
\ddt{} \ket{\hat{\psi}_\alpha(t)}
\end{equation}

and

\begin{equation}\label{eqn:qmTwoR3:470}
\ddt{}\bar{b}_\alpha(t)
=
-\sum_{\beta \ne \alpha} \bar{b}_\beta(t)
e^{-\frac{i}{\hbar} \int_0^t (E_{\beta\alpha}(t') - \hbar \Gamma_{\beta\alpha}(t')) dt'}
\bra{\hat{\psi}_\alpha(t)}
\ddt{} \ket{\hat{\psi}_\beta(t)}
\end{equation}

Suppose we start in a subspace 

\begin{equation}\label{eqn:qmTwoR3:490}
\Span \left\{
\inv{\sqrt{2}}( \ket{2,0,0} \pm \ket{2,1,0} ), \ket{2, 1, \pm 1}
\right\}
\end{equation}

Now expand the bra derivative kets

\begin{equation}\label{eqn:qmTwoR3:510}
\bra{\hat{\psi}_\alpha(t)}
\ddt{} \ket{\hat{\psi}_\beta(t)}
=
\left(
\bra{\psi_{\alpha}^{(0)}} 
+\sum_{\gamma} \frac{ 
\bra{\psi_\gamma^{(0)}} H' \ket{\psi_\alpha^{(0)}} 
\bra{\psi_\gamma^{(0)}} 
}{
E_\alpha^{(0)} 
-E_\gamma^{(0)} 
}
\right)
\ddt{}
\left(
\ket{\psi_{\beta}^{(0)}} 
+\sum_{\gamma'} \frac{ 
\ket{\psi_{\gamma'}^{(0)}} 
\bra{\psi_{\gamma'}^{(0)}} H' \ket{\psi_\beta^{(0)}} 
}{
E_\beta^{(0)} 
-E_{\gamma'}^{(0)} 
}
\right)
\end{equation}

To first order we can drop the quadratic terms in $\gamma,\gamma'$ leaving

\begin{equation}\label{eqn:qmTwoR3:530}
\begin{aligned}
\bra{\hat{\psi}_\alpha(t)}
\ddt{} \ket{\hat{\psi}_\beta(t)}
&\sim
\sum_{\gamma'} 
\braket{\psi_{\alpha}^{(0)}}{\psi_{\gamma'}^{(0)}} 
\frac{ 
\bra{\psi_{\gamma'}^{(0)}} \ddt{H'(t)} \ket{\psi_\beta^{(0)}} 
}{
E_\beta^{(0)} 
-E_{\gamma'}^{(0)} 
}
&=
\frac{ 
\bra{\psi_{\alpha}^{(0)}} \ddt{H'(t)} \ket{\psi_\beta^{(0)}} 
}{
E_\beta^{(0)} 
-E_{\alpha}^{(0)} 
}
\end{aligned}
\end{equation}

so

\begin{equation}\label{eqn:qmTwoR3:550}
\ddt{}\bar{b}_\alpha(t)
=
-\sum_{\beta \ne \alpha} \bar{b}_\beta(t)
e^{-\frac{i}{\hbar} \int_0^t (E_{\beta\alpha}(t') - \hbar \Gamma_{\beta\alpha}(t')) dt'}
\frac{ 
\bra{\psi_{\alpha}^{(0)}} \ddt{H'(t)} \ket{\psi_\beta^{(0)}} 
}{
E_\beta^{(0)} 
-E_{\alpha}^{(0)} 
}
\end{equation}

\subsection{A different way to this end result.}

A result of this form is also derived in \cite{bohm1989qt} \S 20.1, but with a different approach.  There he takes derivatives of

\begin{equation}\label{eqn:qmTwoR3:610}
H(t) \ket{\hat{\psi}_\beta(t)} = E_\beta(t) \ket{\hat{\psi}_\beta(t)},
\end{equation}

\begin{equation}\label{eqn:qmTwoR3:630}
\ddt{H(t)} \ket{\hat{\psi}_\beta(t)} + H(t) \ddt{}\ket{\hat{\psi}_\beta(t)} = \ddt{E_\beta(t)} \ket{\hat{\psi}_\beta(t)}
+ E_\beta(t) \ddt{} \ket{\hat{\psi}_\beta(t)}
\end{equation}

Bra'ing $\bra{\hat{\psi}_\alpha(t)}$ into this we have, for $\alpha \ne \beta$

\begin{align*}
\bra{\hat{\psi}_\alpha(t)}
\ddt{H(t)} \ket{\hat{\psi}_\beta(t)} 
+ 
\bra{\hat{\psi}_\alpha(t)}
H(t) \ddt{}\ket{\hat{\psi}_\beta(t)} 
&= 
\cancel{
\bra{\hat{\psi}_\alpha(t)}
\ddt{E_\beta(t)} \ket{\hat{\psi}_\beta(t)}
}
+ 
\bra{\hat{\psi}_\alpha(t)}
E_\beta(t) \ddt{} \ket{\hat{\psi}_\beta(t)}
 \\
\bra{\hat{\psi}_\alpha(t)}
\ddt{H(t)} \ket{\hat{\psi}_\beta(t)} 
+ 
E_\alpha(t) \bra{\hat{\psi}_\alpha(t)}
\ddt{}\ket{\hat{\psi}_\beta(t)} 
&=
\end{align*}

or
\begin{equation}\label{eqn:qmTwoR3:650}
\bra{\hat{\psi}_\alpha(t)}
\ddt{}\ket{\hat{\psi}_\beta(t)} 
=
\frac{
\bra{\hat{\psi}_\alpha(t)}
\ddt{H(t)} \ket{\hat{\psi}_\beta(t)} 
}
{E_\beta(t) - E_\alpha(t)}
\end{equation}

so without the implied $\lambda$ perturbation of $\ket{\hat{\psi}_\alpha(t)}$ we can from \ref{eqn:qmTwoR3:470} write the \underline{exact} generalization of \ref{eqn:qmTwoR3:550} as

\begin{equation}\label{eqn:qmTwoR3:470b}
\ddt{}\bar{b}_\alpha(t)
=
-\sum_{\beta \ne \alpha} \bar{b}_\beta(t)
e^{-\frac{i}{\hbar} \int_0^t (E_{\beta\alpha}(t') - \hbar \Gamma_{\beta\alpha}(t')) dt'}
%\bra{\hat{\psi}_\alpha(t)}
%\ddt{} \ket{\hat{\psi}_\beta(t)}
\frac{
\bra{\hat{\psi}_\alpha(t)}
\ddt{H(t)} \ket{\hat{\psi}_\beta(t)} 
}
{E_\beta(t) - E_\alpha(t)}
\end{equation}

\EndArticle
