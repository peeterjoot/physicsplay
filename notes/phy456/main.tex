% mp134: mon/wed/fri: 11-12
\documentclass[12pt,leqno]{book}

\usepackage{amsmath,amssymb,amsfonts} % Typical maths resource packages
\usepackage{graphicx}
\usepackage{color}                   % For creating coloured text and background
\usepackage{txfonts} 
\usepackage{listings}
\usepackage[bookmarks=true,plainpages=false]{hyperref}

\parindent 1cm
\parskip 0.2cm
\topmargin 0.2cm
\oddsidemargin 1cm
\evensidemargin 0.5cm
\textwidth 15cm
\textheight 21cm

\usepackage{amsmath}
\usepackage{mathpazo}

%
% shorthand for bold symbols, convenient for vectors and matrices
%
\newcommand{\Ba}[0]{\mathbf{a}}
\newcommand{\Bb}[0]{\mathbf{b}}
\newcommand{\Bc}[0]{\mathbf{c}}
\newcommand{\Bd}[0]{\mathbf{d}}
\newcommand{\Be}[0]{\mathbf{e}}
\newcommand{\Bf}[0]{\mathbf{f}}
\newcommand{\Bg}[0]{\mathbf{g}}
\newcommand{\Bh}[0]{\mathbf{h}}
\newcommand{\Bi}[0]{\mathbf{i}}
\newcommand{\Bj}[0]{\mathbf{j}}
\newcommand{\Bk}[0]{\mathbf{k}}
\newcommand{\Bl}[0]{\mathbf{l}}
\newcommand{\Bm}[0]{\mathbf{m}}
\newcommand{\Bn}[0]{\mathbf{n}}
\newcommand{\Bo}[0]{\mathbf{o}}
\newcommand{\Bp}[0]{\mathbf{p}}
\newcommand{\Bq}[0]{\mathbf{q}}
\newcommand{\Br}[0]{\mathbf{r}}
\newcommand{\Bs}[0]{\mathbf{s}}
\newcommand{\Bt}[0]{\mathbf{t}}
\newcommand{\Bu}[0]{\mathbf{u}}
\newcommand{\Bv}[0]{\mathbf{v}}
\newcommand{\Bw}[0]{\mathbf{w}}
\newcommand{\Bx}[0]{\mathbf{x}}
\newcommand{\By}[0]{\mathbf{y}}
\newcommand{\Bz}[0]{\mathbf{z}}
\newcommand{\BA}[0]{\mathbf{A}}
\newcommand{\BB}[0]{\mathbf{B}}
\newcommand{\BC}[0]{\mathbf{C}}
\newcommand{\BD}[0]{\mathbf{D}}
\newcommand{\BE}[0]{\mathbf{E}}
\newcommand{\BF}[0]{\mathbf{F}}
\newcommand{\BG}[0]{\mathbf{G}}
\newcommand{\BH}[0]{\mathbf{H}}
\newcommand{\BI}[0]{\mathbf{I}}
\newcommand{\BJ}[0]{\mathbf{J}}
\newcommand{\BK}[0]{\mathbf{K}}
\newcommand{\BL}[0]{\mathbf{L}}
\newcommand{\BM}[0]{\mathbf{M}}
\newcommand{\BN}[0]{\mathbf{N}}
\newcommand{\BO}[0]{\mathbf{O}}
\newcommand{\BP}[0]{\mathbf{P}}
\newcommand{\BQ}[0]{\mathbf{Q}}
\newcommand{\BR}[0]{\mathbf{R}}
\newcommand{\BS}[0]{\mathbf{S}}
\newcommand{\BT}[0]{\mathbf{T}}
\newcommand{\BU}[0]{\mathbf{U}}
\newcommand{\BV}[0]{\mathbf{V}}
\newcommand{\BW}[0]{\mathbf{W}}
\newcommand{\BX}[0]{\mathbf{X}}
\newcommand{\BY}[0]{\mathbf{Y}}
\newcommand{\BZ}[0]{\mathbf{Z}}

\newcommand{\Bzero}[0]{\mathbf{0}}
\newcommand{\Btheta}[0]{\boldsymbol{\theta}}
\newcommand{\Btau}[0]{\boldsymbol{\tau}}
\newcommand{\Bomega}[0]{\boldsymbol{\omega}}

%
% shorthand for unit vectors
%
\newcommand{\acap}[0]{\hat{\Ba}}
\newcommand{\bcap}[0]{\hat{\Bb}}
\newcommand{\ccap}[0]{\hat{\Bc}}
\newcommand{\dcap}[0]{\hat{\Bd}}
\newcommand{\ecap}[0]{\hat{\Be}}
\newcommand{\fcap}[0]{\hat{\Bf}}
\newcommand{\gcap}[0]{\hat{\Bg}}
\newcommand{\hcap}[0]{\hat{\Bh}}
\newcommand{\icap}[0]{\hat{\Bi}}
\newcommand{\jcap}[0]{\hat{\Bj}}
\newcommand{\kcap}[0]{\hat{\Bk}}
\newcommand{\lcap}[0]{\hat{\Bl}}
\newcommand{\mcap}[0]{\hat{\Bm}}
\newcommand{\ncap}[0]{\hat{\Bn}}
\newcommand{\ocap}[0]{\hat{\Bo}}
\newcommand{\pcap}[0]{\hat{\Bp}}
\newcommand{\qcap}[0]{\hat{\Bq}}
\newcommand{\rcap}[0]{\hat{\Br}}
\newcommand{\scap}[0]{\hat{\Bs}}
\newcommand{\tcap}[0]{\hat{\Bt}}
\newcommand{\ucap}[0]{\hat{\Bu}}
\newcommand{\vcap}[0]{\hat{\Bv}}
\newcommand{\wcap}[0]{\hat{\Bw}}
\newcommand{\xcap}[0]{\hat{\Bx}}
\newcommand{\ycap}[0]{\hat{\By}}
\newcommand{\zcap}[0]{\hat{\Bz}}
\newcommand{\thetacap}[0]{\hat{\Btheta}}

%
% to write R^n and C^n in a distinguishable fashion.  Perhaps change this
% to the double lined characters upon figuring out how to do so.
%
\newcommand{\C}[1]{$\mathbb{C}^{#1}$}
\newcommand{\R}[1]{$\mathbb{R}^{#1}$}

%
% various generally useful helpers
%

% derivative of #1 wrt. #2:
\newcommand{\D}[2] {\frac {d#2} {d#1}}

\newcommand{\inv}[1]{\frac{1}{#1}}
\newcommand{\cross}[0]{\times}

\newcommand{\abs}[1]{\lvert{#1}\rvert}
\newcommand{\norm}[1]{\lVert{#1}\rVert}
\newcommand{\innerprod}[2]{\langle{#1}, {#2}\rangle}
\newcommand{\dotprod}[2]{{#1} \cdot {#2}}
\newcommand{\bdotprod}[2]{\left({#1} \cdot {#2}\right)}
\newcommand{\crossprod}[2]{{#1} \cross {#2}}
\newcommand{\tripleprod}[3]{\dotprod{\left(\crossprod{#1}{#2}\right)}{#3}}

\DeclareMathOperator{\Proj}{Proj}
\DeclareMathOperator{\Span}{span}
\DeclareMathOperator{\Sgn}{sgn}
\DeclareMathOperator{\Area}{Area}
\DeclareMathOperator{\Volume}{Volume}

%
% A few miscellaneous things specific to this document
%
\newcommand{\crossop}[1]{\crossprod{#1}{}}

% R2 vector.
\newcommand{\VectorTwo}[2]{
\begin{bmatrix}
 {#1} \\
 {#2}
\end{bmatrix}
}

\newcommand{\VectorN}[1]{
\begin{bmatrix}
{#1}_1 \\
{#1}_2 \\
\vdots \\
{#1}_N \\
\end{bmatrix}
}

\newcommand{\DETuvij}[4]{
\begin{vmatrix}
 {#1}_{#3} & {#1}_{#4} \\
 {#2}_{#3} & {#2}_{#4}
\end{vmatrix}
}

\newcommand{\DETuvwijk}[6]{
\begin{vmatrix}
 {#1}_{#4} & {#1}_{#5} & {#1}_{#6} \\
 {#2}_{#4} & {#2}_{#5} & {#2}_{#6} \\
 {#3}_{#4} & {#3}_{#5} & {#3}_{#6}
\end{vmatrix}
}

\newcommand{\DETuvwxijkl}[8]{
\begin{vmatrix}
 {#1}_{#5} & {#1}_{#6} & {#1}_{#7} & {#1}_{#8} \\
 {#2}_{#5} & {#2}_{#6} & {#2}_{#7} & {#2}_{#8} \\
 {#3}_{#5} & {#3}_{#6} & {#3}_{#7} & {#3}_{#8} \\
 {#4}_{#5} & {#4}_{#6} & {#4}_{#7} & {#4}_{#8} \\
\end{vmatrix}
}

%\newcommand{\DETuvwxyijklm}[10]{
%\begin{vmatrix}
% {#1}_{#6} & {#1}_{#7} & {#1}_{#8} & {#1}_{#9} & {#1}_{#10} \\
% {#2}_{#6} & {#2}_{#7} & {#2}_{#8} & {#2}_{#9} & {#2}_{#10} \\
% {#3}_{#6} & {#3}_{#7} & {#3}_{#8} & {#3}_{#9} & {#3}_{#10} \\
% {#4}_{#6} & {#4}_{#7} & {#4}_{#8} & {#4}_{#9} & {#4}_{#10} \\
% {#5}_{#6} & {#5}_{#7} & {#5}_{#8} & {#5}_{#9} & {#5}_{#10}
%\end{vmatrix}
%}

% R3 vector.
\newcommand{\VectorThree}[3]{
\begin{bmatrix}
 {#1} \\
 {#2} \\
 {#3}
\end{bmatrix}
}



\newcommand{\citep}[1]{\cite{#1}}
\newcommand{\citet}[1]{\cite{#1}}
\newcommand{\chapcite}[1]{\ref{chap:#1}}

%-----------------------------------------
%
% stubs for article class.
%
\newcommand{\blogpage}[1]{}
\newcommand{\email}[1]{}
\newcommand{\beginArtWithToc}[0]{}
\newcommand{\beginArtNoToc}[0]{}
\newcommand{\EndArticle}[0]{}
\newcommand{\EndNoBibArticle}[0]{}
\newcommand{\revisionInfo}[1]{}
%-----------------------------------------
\DeclareMathOperator{\Atan}{atan}

%%
% Copyright � 2012 Peeter Joot.  All Rights Reserved.
% Licenced as described in the file LICENSE under the root directory of this GIT repository.
%

% 
% 
\DeclareMathOperator{\Div}{div}
\DeclareMathOperator{\Mod}{mod}
\DeclareMathOperator{\PV}{PV}
\DeclareMathOperator{\Prob}{Prob}
\DeclareMathOperator{\rank}{rank}
\DeclareMathOperator{\sgn}{sgn}
\DeclareMathOperator{\sinc}{sinc}
%\DeclareMathOperator{\Atan2}{atan2}
\DeclareMathOperator{\atan}{atan}


\newcommand{\expectation}[1]{\langle{#1}\rangle}
%\newcommand{\gpgradefour}[1] {\gpgrade{#1}{4}}
%\newcommand{\gpgradeone}[1] {\gpgrade{#1}{1}}
%\newcommand{\gpgradethree}[1] {\gpgrade{#1}{3}}
%\newcommand{\gpgradetwo}[1] {\gpgrade{#1}{2}}
%\newcommand{\gpgradezero}[1] {\gpgrade{#1}{}}
%\newcommand{\gpgrade}[2] {{\left\langle{{#1}}\right\rangle}_{#2}}
%\newcommand{\grad}[0]{\boldsymbol{\nabla}}
%\newcommand{\grad}[0]{\nabla}


\newcommand{\ketbra}[2]{\ket{#1}\bra{#2}}
\newcommand{\ket}[1]{\lvert {#1} \rangle}
%\newcommand{\norm}[1]{\lVert#1\rVert}
\newcommand{\questionEquals}[0]{\stackrel{?}{=}}
\newcommand{\rightshift}[0]{\gg}
%\newcommand{\spacegrad}[0]{\boldsymbol{\nabla}}
\newcommand{\symmetric}[2]{{\left\{{#1},{#2}\right\}}}
\newcommand{\antisymmetric}[2]{\left[{#1},{#2}\right]}

%\newcommand{\Abs}[1]{\left\lvert{#1}\right\rvert}

%\newcommand{\BB}[0]{\mathbf{B}}
%\newcommand{\BE}[0]{\mathbf{E}}
%\newcommand{\BF}[0]{\mathbf{F}}
%\newcommand{\BS}[0]{\mathbf{S}}
%\newcommand{\BV}[0]{\mathbf{V}}
%\newcommand{\Bj}[0]{\mathbf{j}}

\newcommand{\BraOpKet}[3]{\bra{#1} \hat{#2} \ket{#3} }
%\newcommand{\Brho}[0]{\boldsymbol{\rho}}
\newcommand{\CC}[0]{c^2}
\newcommand{\Cos}[1]{\cos{\left({#1}\right)}}

% not working anymore.  think it's a conflicting macro for \not.
% compared to original usage in klien_gordon.ltx
%
%\newcommand{\Dslash}[0]{{\not}D}
%\newcommand{\Dslash}[0]{{\not{}}D}
% switched to cancel in macros.tex
%\newcommand{\Dslash}[0]{D\!\!\!/}

\newcommand{\Expectation}[1]{\left\langle {#1} \right\rangle}
\newcommand{\Exp}[1]{\exp{\left({#1}\right)}}
\newcommand{\FF}[0]{\mathcal{F}}
\newcommand{\FM}[0]{\inv{\sqrt{2\pi\hbar}}}
\newcommand{\IIinf}[0]{ \int_{-\infty}^\infty }
\newcommand{\Innerprod}[2]{\left\langle{#1}, {#2}\right\rangle}
%\newcommand{\LL}[0]{\mathcal{L}}

%\newcommand{\PD}[2] {\frac {\partial #2} {\partial #1}}

% backwards from ../peeterj_macros2:
\newcommand{\PDb}[2]{ \frac{\partial{#1}}{\partial {#2}} }

%\newcommand{\PDD}[3]{\frac{\partial^2 {#3}}{\partial {#1}\partial {#2}}}
\newcommand{\PDN}[3]{\frac{\partial^{#3} {#2}}{\partial {#1}^{#3}}}

\newcommand{\PDSq}[2]{\frac{\partial^2 {#2}}{\partial {#1}^2}}
\newcommand{\PDsQ}[2]{\frac{\partial^2 {#2}}{\partial^2 {#1}}}

\newcommand{\Sch}[0]{{Schr\"{o}dinger} }
\newcommand{\Sin}[1]{\sin{\left({#1}\right)}}
\newcommand{\Sw}[0]{\mathcal{S}}
%\newcommand{\T}[0]{\text{T}}
\newcommand{\T}[0]{{\text{T}}}

\newcommand{\braket}[2]{\langle{#1} \vert {#2}\rangle}
\newcommand{\bra}[1]{\langle {#1} \rvert}
\newcommand{\curl}[0]{\grad \times}
\newcommand{\delambert}[0]{\sum_{\alpha = 1}^4{\PDSq{x_\alpha}{}}}
\newcommand{\delsquared}[0]{\nabla^2}
\newcommand{\diverg}[0]{\grad \cdot}

\newcommand{\halfPhi}[0]{\frac{\phi}{2}}
\newcommand{\hatH}[0]{\hat{H}}
\newcommand{\hatS}[0]{\hat{S}}
\newcommand{\hatk}[0]{\hat{k}}
\newcommand{\hatp}[0]{\hat{p}}
\newcommand{\hatx}[0]{\hat{x}}


\newcommand{\Rdot}[0]{\dot{R}}
%\newcommand{\addot}[0]{\ddot{a}}
%\newcommand{\adot}[0]{\dot{a}}
%\newcommand{\fddot}[0]{\ddot{f}}
%\newcommand{\fdot}[0]{\dot{f}}
%\newcommand{\bddot}[0]{\ddot{b}}
%\newcommand{\bdot}[0]{\dot{b}}
\newcommand{\ddotOmega}[0]{\ddot{\Omega}}
\newcommand{\ddotalpha}[0]{\ddot{\alpha}}
\newcommand{\ddotomega}[0]{\ddot{\omega}}
\newcommand{\ddotphi}[0]{\ddot{\phi}}
\newcommand{\ddotpsi}[0]{\ddot{\psi}}
\newcommand{\ddottheta}[0]{\ddot{\theta}}
\newcommand{\dotOmega}[0]{\dot{\Omega}}
\newcommand{\dotalpha}[0]{\dot{\alpha}}
\newcommand{\dotomega}[0]{\dot{\omega}}
\newcommand{\dotphi}[0]{\dot{\phi}}
\newcommand{\dotpsi}[0]{\dot{\psi}}
\newcommand{\dottheta}[0]{\dot{\theta}}
%\newcommand{\pddot}[0]{\ddot{p}}
%\newcommand{\pdot}[0]{\dot{p}}
%\newcommand{\qddot}[0]{\ddot{q}}
%\newcommand{\qdot}[0]{\dot{q}}
%\newcommand{\rddot}[0]{\ddot{r}}
%\newcommand{\rdot}[0]{\dot{r}}
%\newcommand{\tddot}[0]{\ddot{t}}
%\newcommand{\tdot}[0]{\dot{t}}
%\newcommand{\uddot}[0]{\ddot{u}}
%\newcommand{\udot}[0]{\dot{u}}
%\newcommand{\xddot}[0]{\ddot{x}}
%\newcommand{\xdot}[0]{\dot{x}}
%\newcommand{\yddot}[0]{\ddot{y}}
%\newcommand{\ydot}[0]{\dot{y}}
%\newcommand{\zddot}[0]{\ddot{z}}
%\newcommand{\zdot}[0]{\dot{z}}








%-------------------------------------------------------------------
% ORIGINS:
%
% bohm11.tex

%\DeclareMathOperator{\sgn}{sgn}
%\newcommand{\PDSq}[2]{\frac{\partial^2 {#2}}{\partial {#1}^2}}
%\newcommand{\PDN}[3]{\frac{\partial^{#3} {#2}}{\partial {#1}^{#3}}}
%\DeclareMathOperator{\sinc}{sinc}
%\DeclareMathOperator{\PV}{PV}
%\newcommand{\FF}[0]{\mathcal{F}}
%\newcommand{\Sw}[0]{\mathcal{S}}
%\newcommand{\IIinf}[0]{ \int_{-\infty}^\infty }
%\newcommand{\FM}[0]{\inv{\sqrt{2\pi\hbar}}}
%\newcommand{\expectation}[1]{\langle{#1}\rangle}
%
%

% bohm_ch10.tex

%\DeclareMathOperator{\sgn}{sgn}
%\newcommand{\expectation}[1]{\langle{#1}\rangle}
%\newcommand{\IIinf}[0]{ \int_{-\infty}^\infty }
%\DeclareMathOperator{\PV}{PV}
%
%

% bohm_ch9.tex

%\newcommand{\PDSq}[2]{\frac{\partial^2 {#2}}{\partial {#1}^2}}
%\newcommand{\PDN}[3]{\frac{\partial^{#3} {#2}}{\partial {#1}^{#3}}}
%\DeclareMathOperator{\sinc}{sinc}
%\DeclareMathOperator{\PV}{PV}
%\newcommand{\FF}[0]{\mathcal{F}}
%\newcommand{\Sw}[0]{\mathcal{S}}
%\newcommand{\IIinf}[0]{ \int_{-\infty}^\infty }
%\newcommand{\FM}[0]{\inv{\sqrt{2\pi\hbar}}}
%\newcommand{\expectation}[1]{\langle{#1}\rangle}
%
%

% commutator_herm.tex

%\newcommand{\symmetric}[2]{{\left\{{#1},{#2}\right\}}}
%\newcommand{\antisymmetric}[2]{\left[{#1},{#2}\right]}
%
%%\newcommand{\ket}[1]{\lvert {#1} \rangle}
%%\newcommand{\bra}[1]{\langle {#1} \rvert}
%%\newcommand{\braket}[2]{\langle{#1} \vert {#2}\rangle}
%%\newcommand{\ketbra}[2]{\ket{#1}\bra{#2}}
%%\newcommand{\BraOpKet}[3]{\bra{#1} \hat{#2} \ket{#3} }
%%\newcommand{\Innerprod}[2]{\left\langle{#1}, {#2}\right\rangle}
%\newcommand{\Expectation}[1]{\left\langle {#1} \right\rangle}
%
%

% delta_ortho_series.tex

%\newcommand{\IIinf}[0]{ \int_{-\infty}^\infty }
%\newcommand{\ket}[1]{\lvert {#1} \rangle}
%\newcommand{\bra}[1]{\langle {#1} \rvert}
%\newcommand{\braket}[2]{\langle{#1} \vert {#2}\rangle}
%\newcommand{\ketbra}[2]{\ket{#1}\bra{#2}}
%\newcommand{\BraOpKet}[3]{\bra{#1} \hat{#2} \ket{#3} }
%\newcommand{\Innerprod}[2]{\left\langle{#1}, {#2}\right\rangle}
%
%

% distributions.tex

%\newcommand{\PDSq}[2]{\frac{\partial^2 {#2}}{\partial {#1}^2}}
%\DeclareMathOperator{\sinc}{sinc}
%\DeclareMathOperator{\PV}{PV}
%\newcommand{\FF}[0]{\mathcal{F}}
%\newcommand{\Sw}[0]{\mathcal{S}}
%\newcommand{\IIinf}[0]{ \int_{-\infty}^\infty }
%
%

% ehrenfest.tex

%\newcommand{\PDSq}[2]{\frac{\partial^2 {#2}}{\partial {#1}^2}}
%
%

% fletcher.tex

%\DeclareMathOperator{\Div}{div}
%\DeclareMathOperator{\Mod}{mod}
%\newcommand{\rightshift}[0]{\gg}
%\newcommand{\questionEquals}[0]{\stackrel{?}{=}}
%
%

% fvec.tex

%\newcommand{\grad}[0]{\nabla}
%\newcommand{\PD}[2]{ \frac{\partial{#1}}{\partial {#2}} }
%
%

% gacs_q8_8.tex

%\newcommand{\halfPhi}[0]{\frac{\phi}{2}}
%\newcommand{\Sin}[1]{\sin{\left({#1}\right)}}
%\newcommand{\Cos}[1]{\cos{\left({#1}\right)}}
%\newcommand{\Exp}[1]{\exp{\left({#1}\right)}}
%
%

% goldstein_ch1_2.tex

%\newcommand{\spacegrad}[0]{\boldsymbol{\nabla}}
%\newcommand{\Brho}[0]{\boldsymbol{\rho}}
%\newcommand{\LL}[0]{\mathcal{L}}
%\newcommand{\Abs}[1]{\left\lvert{#1}\right\rvert}
%\newcommand{\qdot}[0]{\dot{q}}
%\newcommand{\qddot}[0]{\ddot{q}}
%\newcommand{\xdot}[0]{\dot{x}}
%\newcommand{\xddot}[0]{\ddot{x}}
%\newcommand{\ydot}[0]{\dot{y}}
%\newcommand{\yddot}[0]{\ddot{y}}
%\newcommand{\dotalpha}[0]{\dot{\alpha}}
%\newcommand{\ddotalpha}[0]{\ddot{\alpha}}
%\newcommand{\dottheta}[0]{\dot{\theta}}
%\newcommand{\ddottheta}[0]{\ddot{\theta}}
%\newcommand{\dotphi}[0]{\dot{\phi}}
%\newcommand{\ddotphi}[0]{\ddot{\phi}}
%% == \partial_{#1} {#2}
%\newcommand{\PD}[2]{\frac{\partial {#2}}{\partial {#1}}}
%\newcommand{\PDD}[3]{\frac{\partial^2 {#3}}{\partial {#1}\partial {#2}}}
%
%% <grade selection>
%%
%\newcommand{\gpgrade}[2] {{\left\langle{{#1}}\right\rangle}_{#2}}
%
%\newcommand{\gpgradezero}[1] {\gpgrade{#1}{}}
%%\newcommand{\gpscalargrade}[1] {{\left\langle{{#1}}\right\rangle}}
%%\newcommand{\gpgradezero}[1] {\gpgrade{#1}{0}}
%
%%\newcommand{\gpgradeone}[1] {{\left\langle{{#1}}\right\rangle}_{1}}
%\newcommand{\gpgradeone}[1] {\gpgrade{#1}{1}}
%
%\newcommand{\gpgradetwo}[1] {\gpgrade{#1}{2}}
%\newcommand{\gpgradethree}[1] {\gpgrade{#1}{3}}
%\newcommand{\gpgradefour}[1] {\gpgrade{#1}{4}}
%%
%% </grade selection>
%
%
%

% harmonic_osc.tex

%\newcommand{\IIinf}[0]{ \int_{-\infty}^\infty }
%
%

% klein_gordon.tex

%\newcommand{\PDSq}[2]{\frac{\partial^2 {#2}}{\partial {#1}^2}}
%%\newcommand{\Dslash}[0]{D\!\!\!/}
%\newcommand{\Dslash}[0]{{\not}D}
%
%

% matrix_to_operator.tex

%\newcommand{\T}[0]{{\text{T}}}
%
%

% maxwell.tex

%\newcommand{\norm}[1]{\lVert#1\rVert}
%\newcommand{\grad}[0]{\boldsymbol{\nabla}}
%\newcommand{\curl}[0]{\grad \times}
%\newcommand{\diverg}[0]{\grad \cdot}
%\newcommand{\delsquared}[0]{\nabla^2}
%\newcommand{\delambert}[0]{\sum_{\alpha = 1}^4{\PDSq{x_\alpha}{}}}
%
%% partial derivative of #1 wrt. #2:
%\newcommand{\PD}[2] {\frac {\partial #2} {\partial #1}}
%% second partial derivative of #1 wrt. #2:
%\newcommand{\PDSq}[2] {\frac {\partial^2 #2} {\partial {#1}^2}}
%
%%
%% shorthand for bold symbols:
%%
%\newcommand{\Bj}[0]{\mathbf{j}}
%\newcommand{\BB}[0]{\mathbf{B}}
%\newcommand{\BE}[0]{\mathbf{E}}
%\newcommand{\BF}[0]{\mathbf{F}}
%\newcommand{\BS}[0]{\mathbf{S}}
%\newcommand{\BV}[0]{\mathbf{V}}
%
%

% mp_inverse_svd_rough_notes.tex

%\newcommand{\T}[0]{\text{T}}
%\DeclareMathOperator{\rank}{rank}
%
%

% outermorphism_det.tex

%\newcommand{\gpgrade}[2] {{\left\langle{{#1}}\right\rangle}_{#2}}
%\newcommand{\gpgradeone}[1] {\gpgrade{#1}{1}}
%\newcommand{\gpgradetwo}[1] {\gpgrade{#1}{2}}
%
%

% pauli_qm_relativity_intro.tex

%\newcommand{\Sch}[0]{{Schr\"{o}dinger} }
%
%

% pe.tex

%
%\newcommand{\grad}[0]{\nabla}

% qm_susskind.tex

%\newcommand{\ket}[1]{\lvert {#1} \rangle}
%\newcommand{\bra}[1]{\langle {#1} \rvert}
%\newcommand{\braket}[2]{\langle{#1} \vert {#2}\rangle}
%\newcommand{\ketbra}[2]{\ket{#1}\bra{#2}}
%\newcommand{\BraOpKet}[3]{\bra{#1} \hat{#2} \ket{#3} }
%\newcommand{\hatH}[0]{\hat{H}}
%\newcommand{\hatS}[0]{\hat{S}}
%\newcommand{\hatk}[0]{\hat{k}}
%\newcommand{\hatx}[0]{\hat{x}}
%\newcommand{\hatp}[0]{\hat{p}}
%\DeclareMathOperator{\Prob}{Prob}
%
%

% schwartzchild_metric.tex

%\newcommand{\grad}[0]{\nabla}
%\newcommand{\Abs}[1]{\left\lvert{#1}\right\rvert}
%\newcommand{\spacegrad}[0]{\boldsymbol{\nabla}}
%\newcommand{\LL}[0]{\mathcal{L}}
%\newcommand{\PD}[2]{\frac{\partial {#2}}{\partial {#1}}}
%\newcommand{\PDsQ}[2]{\frac{\partial^2 {#2}}{\partial^2 {#1}}}
%\newcommand{\dotalpha}[0]{\dot{\alpha}}
%\newcommand{\ddotalpha}[0]{\ddot{\alpha}}
%
%\newcommand{\dotomega}[0]{\dot{\omega}}
%\newcommand{\ddotomega}[0]{\ddot{\omega}}
%
%\newcommand{\dotOmega}[0]{\dot{\Omega}}
%\newcommand{\ddotOmega}[0]{\ddot{\Omega}}
%
%\newcommand{\CC}[0]{c^2}
%
%\newcommand{\dottheta}[0]{\dot{\theta}}
%\newcommand{\ddottheta}[0]{\ddot{\theta}}
%
%\newcommand{\dotpsi}[0]{\dot{\psi}}
%\newcommand{\ddotpsi}[0]{\ddot{\psi}}
%
%\newcommand{\adot}[0]{\dot{a}}
%\newcommand{\addot}[0]{\ddot{a}}
%\newcommand{\udot}[0]{\dot{u}}
%\newcommand{\uddot}[0]{\ddot{u}}
%\newcommand{\fdot}[0]{\dot{f}}
%\newcommand{\fddot}[0]{\ddot{f}}
%\newcommand{\bdot}[0]{\dot{b}}
%\newcommand{\bddot}[0]{\ddot{b}}
%\newcommand{\qdot}[0]{\dot{q}}
%\newcommand{\qddot}[0]{\ddot{q}}
%\newcommand{\tdot}[0]{\dot{t}}
%\newcommand{\tddot}[0]{\ddot{t}}
%
%\newcommand{\Rdot}[0]{\dot{R}}
%
%\newcommand{\pdot}[0]{\dot{p}}
%\newcommand{\pddot}[0]{\ddot{p}}
%
%\newcommand{\xdot}[0]{\dot{x}}
%\newcommand{\xddot}[0]{\ddot{x}}
%
%\newcommand{\zdot}[0]{\dot{z}}
%\newcommand{\zddot}[0]{\ddot{z}}
%
%\newcommand{\rdot}[0]{\dot{r}}
%\newcommand{\rddot}[0]{\ddot{r}}
%
%

% shear.tex

%\newcommand{\gpgrade}[2] {{\left\langle{{#1}}\right\rangle}_{#2}}
%
%

% tong_mf1.tex

%\newcommand{\Abs}[1]{\left\lvert{#1}\right\rvert}
%\newcommand{\grad}[0]{\nabla}
%\newcommand{\LL}[0]{\mathcal{L}}
%
%\newcommand{\dotalpha}[0]{\dot{\alpha}}
%\newcommand{\ddotalpha}[0]{\ddot{\alpha}}
%
%\newcommand{\dotomega}[0]{\dot{\omega}}
%\newcommand{\ddotomega}[0]{\ddot{\omega}}
%
%\newcommand{\dottheta}[0]{\dot{\theta}}
%\newcommand{\ddottheta}[0]{\ddot{\theta}}
%
%\newcommand{\dotpsi}[0]{\dot{\psi}}
%\newcommand{\ddotpsi}[0]{\ddot{\psi}}
%
%\newcommand{\qdot}[0]{\dot{q}}
%\newcommand{\qddot}[0]{\ddot{q}}
%
%\newcommand{\Rdot}[0]{\dot{R}}
%
%\newcommand{\pdot}[0]{\dot{p}}
%\newcommand{\pddot}[0]{\ddot{p}}
%
%\newcommand{\xdot}[0]{\dot{x}}
%\newcommand{\xddot}[0]{\ddot{x}}
%
%\newcommand{\zdot}[0]{\dot{z}}
%\newcommand{\zddot}[0]{\ddot{z}}
%
%\newcommand{\rdot}[0]{\dot{r}}
%\newcommand{\rddot}[0]{\ddot{r}}
%
%% == \partial_{#1} {#2}
%\newcommand{\PD}[2]{\frac{\partial {#2}}{\partial {#1}}}
%\newcommand{\PDD}[3]{\frac{\partial^2 {#3}}{\partial {#1}\partial {#2}}}
%
%

% wavepacket.tex

%\newcommand{\PDSq}[2]{\frac{\partial^2 {#2}}{\partial {#1}^2}}
%\newcommand{\IIinf}[0]{ \int_{-\infty}^\infty }
%
%

% wavevariation.tex

%\newcommand{\PDSq}[2]{\frac{\partial^2 {#2}}{\partial {#1}^2}}
%
%

% qm_barrier
%\DeclareMathOperator{\Atan2}{atan2}
%\DeclareMathOperator{\atan}{atan}

% twobodies.tex
%\DeclareMathOperator{\sgn}{sgn}


% sr_lagrangian_q.tex

%\newcommand{\PD}[2]{\frac{\partial {#2}}{\partial {#1}}}
%\newcommand{\xdot}[0]{\dot{x}}
%\newcommand{\xddot}[0]{\ddot{x}}

% stub_em_fields.tex

%\newcommand{\EE}[0]{\boldsymbol{\mathcal{E}}}
%\newcommand{\HH}[0]{\boldsymbol{\mathcal{H}}}
%\newcommand{\PDSq}[2]{\frac{\partial^2 {#2}}{\partial {#1}^2}}

% long_wire_q.tex

%\newcommand{\grad}[0]{\nabla}

% lorentz_tx_em_potential.tex
%\newcommand{\LL}[0]{\mathcal{L}}
%\newcommand{\grad}[0]{\nabla}
%\newcommand{\pdot}[0]{\dot{p}}
%\newcommand{\pddot}[0]{\ddot{p}}

%------------------------------------------------------
% cross_old.tex

%%
%% shorthand for bold symbols, convenient for vectors and matrices
%%
%\newcommand{\Ba}[0]{\mathbf{a}}
%\newcommand{\Bb}[0]{\mathbf{b}}
%\newcommand{\Bc}[0]{\mathbf{c}}
%\newcommand{\Bd}[0]{\mathbf{d}}
%\newcommand{\Be}[0]{\mathbf{e}}
%\newcommand{\Bf}[0]{\mathbf{f}}
%\newcommand{\Bg}[0]{\mathbf{g}}
%\newcommand{\Bh}[0]{\mathbf{h}}
%\newcommand{\Bi}[0]{\mathbf{i}}
%\newcommand{\Bj}[0]{\mathbf{j}}
%\newcommand{\Bk}[0]{\mathbf{k}}
%\newcommand{\Bl}[0]{\mathbf{l}}
%\newcommand{\Bm}[0]{\mathbf{m}}
%\newcommand{\Bn}[0]{\mathbf{n}}
%\newcommand{\Bo}[0]{\mathbf{o}}
%\newcommand{\Bp}[0]{\mathbf{p}}
%\newcommand{\Bq}[0]{\mathbf{q}}
%\newcommand{\Br}[0]{\mathbf{r}}
%\newcommand{\Bs}[0]{\mathbf{s}}
%\newcommand{\Bt}[0]{\mathbf{t}}
%\newcommand{\Bu}[0]{\mathbf{u}}
%\newcommand{\Bv}[0]{\mathbf{v}}
%\newcommand{\Bw}[0]{\mathbf{w}}
%\newcommand{\Bx}[0]{\mathbf{x}}
%\newcommand{\By}[0]{\mathbf{y}}
%\newcommand{\Bz}[0]{\mathbf{z}}
%\newcommand{\BA}[0]{\mathbf{A}}
%\newcommand{\BB}[0]{\mathbf{B}}
%\newcommand{\BC}[0]{\mathbf{C}}
%\newcommand{\BD}[0]{\mathbf{D}}
%\newcommand{\BE}[0]{\mathbf{E}}
%\newcommand{\BF}[0]{\mathbf{F}}
%\newcommand{\BG}[0]{\mathbf{G}}
%\newcommand{\BH}[0]{\mathbf{H}}
%\newcommand{\BI}[0]{\mathbf{I}}
%\newcommand{\BJ}[0]{\mathbf{J}}
%\newcommand{\BK}[0]{\mathbf{K}}
%\newcommand{\BL}[0]{\mathbf{L}}
%\newcommand{\BM}[0]{\mathbf{M}}
%\newcommand{\BN}[0]{\mathbf{N}}
%\newcommand{\BO}[0]{\mathbf{O}}
%\newcommand{\BP}[0]{\mathbf{P}}
%\newcommand{\BQ}[0]{\mathbf{Q}}
%\newcommand{\BR}[0]{\mathbf{R}}
%\newcommand{\BS}[0]{\mathbf{S}}
%\newcommand{\BT}[0]{\mathbf{T}}
%\newcommand{\BU}[0]{\mathbf{U}}
%\newcommand{\BV}[0]{\mathbf{V}}
%\newcommand{\BW}[0]{\mathbf{W}}
%\newcommand{\BX}[0]{\mathbf{X}}
%\newcommand{\BY}[0]{\mathbf{Y}}
%\newcommand{\BZ}[0]{\mathbf{Z}}
%
%\newcommand{\Bzero}[0]{\mathbf{0}}
%\newcommand{\Btheta}[0]{\boldsymbol{\theta}}
%\newcommand{\Btau}[0]{\boldsymbol{\tau}}
%\newcommand{\Bomega}[0]{\boldsymbol{\omega}}
%
%%
%% shorthand for unit vectors
%%
%\newcommand{\acap}[0]{\hat{\Ba}}
%\newcommand{\bcap}[0]{\hat{\Bb}}
%\newcommand{\ccap}[0]{\hat{\Bc}}
%\newcommand{\dcap}[0]{\hat{\Bd}}
%\newcommand{\ecap}[0]{\hat{\Be}}
%\newcommand{\fcap}[0]{\hat{\Bf}}
%\newcommand{\gcap}[0]{\hat{\Bg}}
%\newcommand{\hcap}[0]{\hat{\Bh}}
%\newcommand{\icap}[0]{\hat{\Bi}}
%\newcommand{\jCap}[0]{\hat{\Bj}}
%\newcommand{\kcap}[0]{\hat{\Bk}}
%\newcommand{\lcap}[0]{\hat{\Bl}}
%\newcommand{\mcap}[0]{\hat{\Bm}}
%\newcommand{\ncap}[0]{\hat{\Bn}}
%\newcommand{\ocap}[0]{\hat{\Bo}}
%\newcommand{\pcap}[0]{\hat{\Bp}}
%\newcommand{\qcap}[0]{\hat{\Bq}}
%\newcommand{\rcap}[0]{\hat{\Br}}
%\newcommand{\scap}[0]{\hat{\Bs}}
%\newcommand{\tcap}[0]{\hat{\Bt}}
%\newcommand{\ucap}[0]{\hat{\Bu}}
%\newcommand{\vcap}[0]{\hat{\Bv}}
%\newcommand{\wcap}[0]{\hat{\Bw}}
%\newcommand{\xcap}[0]{\hat{\Bx}}
%\newcommand{\ycap}[0]{\hat{\By}}
%\newcommand{\zcap}[0]{\hat{\Bz}}
%\newcommand{\thetacap}[0]{\hat{\Btheta}}
%
%%
%% to write R^n and C^n in a distinguishable fashion.  Perhaps change this
%% to the double lined characters upon figuring out how to do so.
%%
%\newcommand{\C}[1]{${\BC}^{#1}$}
%\newcommand{\R}[1]{${\BR}^{#1}$}
%
%%
%% various generally useful helpers
%%
%
%% derivative of #1 wrt. #2:
%\newcommand{\D}[2] {\frac {d#2} {d#1}}

%\newcommand{\inv}[1]{\frac{1}{#1}}
%\newcommand{\cross}[0]{\times}

%\newcommand{\abs}[1]{\lvert#1\rvert}
%\newcommand{\norm}[1]{\lVert#1\rVert}
%\newcommand{\innerprod}[2]{\langle{#1}, {#2}\rangle}
%\newcommand{\dotprod}[2]{#1 \cdot #2}
%\newcommand{\crossprod}[2]{#1 \cross #2}
%\newcommand{\tripleprod}[3]{\dotprod{\crossprod{#1}{#2}}{#3}}

%
% A few miscellaneous things specific to this document
%
%\newcommand{\crossop}[1]{\crossprod{#1}{}}

\newcommand{\PDP}[2]{\BP^{#1}\BD{\BP^{#2}}}
\newcommand{\PDPDP}[3]{\Bv^T\BP^{#1}\BD\BP^{#2}\BD\BP^{#3}\Bv}

\newcommand{\Mp}[0]{
\begin{bmatrix}
0 & 1 & 0 & 0 \\
0 & 0 & 1 & 0 \\
0 & 0 & 0 & 1 \\
1 & 0 & 0 & 0
\end{bmatrix}
}
\newcommand{\Mpp}[0]{
\begin{bmatrix}
0 & 0 & 1 & 0 \\
0 & 0 & 0 & 1 \\
1 & 0 & 0 & 0 \\
0 & 1 & 0 & 0
\end{bmatrix}
}
\newcommand{\Mppp}[0]{
\begin{bmatrix}
0 & 0 & 0 & 1 \\
1 & 0 & 0 & 0 \\
0 & 1 & 0 & 0 \\
0 & 0 & 1 & 0
\end{bmatrix}
}
\newcommand{\Mpu}[0]{
\begin{bmatrix}
u_1 & 0 & 0 & 0 \\
0 & u_2 & 0 & 0 \\
0 & 0 & u_3 & 0 \\
0 & 0 & 0 & u_4
\end{bmatrix}
}

%------------------------------------------------------



%\makeindex

\begin{document}
\pagenumbering{alph}

\title{Course notes and problems from\\University of Toronto PHY456H1F\\Quantum Physics II.}
\author{Peeter Joot \quad peeter.joot@gmail.com}

\maketitle

\clearpage\pagenumbering{roman}
\tableofcontents

\clearpage\pagenumbering{arabic}

\pagestyle{plain}

%
% Copyright � 2015 Peeter Joot.  All Rights Reserved.
% Licenced as described in the file LICENSE under the root directory of this GIT repository.
%

% 
%\chapter{Preface}
% this suppresses an explicit chapter number for the preface.
\chapter*{Preface}%\normalsize
  \addcontentsline{toc}{chapter}{Preface}

This document was produced while taking the Spring 2016, University of Toronto Microwave Circuits course (ECE1236H), taught by Prof.\ G. V. Eleftheriades.

\paragraph{Course Syllabus}

This course outlines the principles of designing modern microwave and RF circuits.  Signal-integrity issues in high-speed digital circuits are also examined.

\begin{itemize}
\item The wave equation.
\item Ideal transmission lines.
\item Transients on transmission-lines.
\item Planar transmission lines and introduction to MMIC's.
\item Designing with scattering parameters.
\item Planar power dividers.
\item Directional couplers.
\item Microwave filters.
\item Solid-state microwave amplifiers.
\item Noise.
\item Diode-mixers.
\item RF receiver chains.
\item Oscillators.
\end{itemize}

\withproblemsetsMessage{
\textcolor{Maroon}{
\textit{THIS DOCUMENT IS REDACTED.  THE PROBLEM SET SOLUTIONS AND ASSOCIATED MATHEMATICA CODE IS NOT VISIBLE.  PLEASE EMAIL ME FOR THE FULL VERSION IF YOU ARE NOT TAKING ECE1236.}
}
}

\paragraph{This document contains:}

\begin{itemize}
\item Lecture notes.
\item Personal notes exploring auxiliary details.
\item Worked practice problems.

\ifthenelse{\boolean{redacted}}%
{%
\item Links to Mathematica notebooks associated with the course material and problems (but not problem sets).
}%
{
\item Assigned problems.%
\item Links to Mathematica notebooks associated with problems and course material.%
}
\end{itemize}

%This set of notes is significantly different from my notes for many other classes.  With the class taught on slides (and some of those slides mirroring the text closely), I did not take live notes in class.
%These notes fill in details that I felt deserved clarification, contain problem sets solutions, as well as a number of loosely related musings on Geometric Algebra equivalents to some of the generalized concepts of electromagnetic theory encountered in this class (i.e. magnetic sources).
%
My thanks go to Professor Eleftheriades for teaching this course.

Peeter Joot  \quad peeterjoot@protonmail.com 


\part{Readings}

\chapter{Reading and problem status}

\begin{itemize}
\item \S 24.2 Variational method.  -- reading done.  problems 3-6 done.  problem 7 would be good to do.
\item \S 16.1 - \S 16.2 Time independent pertubation, theory and HO example.  Reading done.
\item \S 16.4 stark effect.  Not covered in class.  I'm having trouble confirming 16.72 using 16.66?  See 'desai attempt to verify section 16.3.nb'.  Revisit after course.
\item \S 16.5 degeneracy.  Excellent description (helpful to also read \S 13.1.1 - \S 13.1.2 on two level problems.)  Revisit degeneracy problem set question having read this, and derive the pertubation expansion for two simple cases: 1 two fold degneracy + 1 non-degenerate state.  1 two fold degeneracy + 2 independent non-degenerate states.  Then do the general derivation.
\item \S 16.6: Problems: All look like good ones.  Haven't done any yet.
\item \S 3.3 Reread.  Prof Sipe's coverage was much clearer.

\item \S 17.1 Time dependent perturbation theory.  Reading: done.
\item \S 18.3 Coloumb excitation.  Given as an example in lecture 6 when starting time dependent pertubation.  Rather than working with the Taylor expansion and identifying the first term as the electric field the text works with the operators directly, and goes a lot deeper than we did in class (class goes to about 18.53 and then skips the rest of the section).  This would be good to revisit after the course.
\item \S 17.5.1 Adiabatic perturbation.  Reading: done.
\item \S 17.5.2 Berry phase. Reading: todo.
\item \S 17.2 Fermi's golden rule. Reading: todo.
\item SKIP: \S 17.3 - \S 17.4 scattering cross section, resonance and decay.  Not yet covered in class.  Skip for now, and revisit after or later in course.
\item \S 17.6 Problems.  Most look appropriate.
\end{itemize}
%-------------------------------------------------------

\part{Lecture Notes.}
%%
% Copyright � 2015 Peeter Joot.  All Rights Reserved.
% Licenced as described in the file LICENSE under the root directory of this GIT repository.
%
\documentclass[]{eliblog}

\usepackage{amsmath}
\usepackage{mathpazo}

%
% shorthand for bold symbols, convenient for vectors and matrices
%
\newcommand{\Ba}[0]{\mathbf{a}}
\newcommand{\Bb}[0]{\mathbf{b}}
\newcommand{\Bc}[0]{\mathbf{c}}
\newcommand{\Bd}[0]{\mathbf{d}}
\newcommand{\Be}[0]{\mathbf{e}}
\newcommand{\Bf}[0]{\mathbf{f}}
\newcommand{\Bg}[0]{\mathbf{g}}
\newcommand{\Bh}[0]{\mathbf{h}}
\newcommand{\Bi}[0]{\mathbf{i}}
\newcommand{\Bj}[0]{\mathbf{j}}
\newcommand{\Bk}[0]{\mathbf{k}}
\newcommand{\Bl}[0]{\mathbf{l}}
\newcommand{\Bm}[0]{\mathbf{m}}
\newcommand{\Bn}[0]{\mathbf{n}}
\newcommand{\Bo}[0]{\mathbf{o}}
\newcommand{\Bp}[0]{\mathbf{p}}
\newcommand{\Bq}[0]{\mathbf{q}}
\newcommand{\Br}[0]{\mathbf{r}}
\newcommand{\Bs}[0]{\mathbf{s}}
\newcommand{\Bt}[0]{\mathbf{t}}
\newcommand{\Bu}[0]{\mathbf{u}}
\newcommand{\Bv}[0]{\mathbf{v}}
\newcommand{\Bw}[0]{\mathbf{w}}
\newcommand{\Bx}[0]{\mathbf{x}}
\newcommand{\By}[0]{\mathbf{y}}
\newcommand{\Bz}[0]{\mathbf{z}}
\newcommand{\BA}[0]{\mathbf{A}}
\newcommand{\BB}[0]{\mathbf{B}}
\newcommand{\BC}[0]{\mathbf{C}}
\newcommand{\BD}[0]{\mathbf{D}}
\newcommand{\BE}[0]{\mathbf{E}}
\newcommand{\BF}[0]{\mathbf{F}}
\newcommand{\BG}[0]{\mathbf{G}}
\newcommand{\BH}[0]{\mathbf{H}}
\newcommand{\BI}[0]{\mathbf{I}}
\newcommand{\BJ}[0]{\mathbf{J}}
\newcommand{\BK}[0]{\mathbf{K}}
\newcommand{\BL}[0]{\mathbf{L}}
\newcommand{\BM}[0]{\mathbf{M}}
\newcommand{\BN}[0]{\mathbf{N}}
\newcommand{\BO}[0]{\mathbf{O}}
\newcommand{\BP}[0]{\mathbf{P}}
\newcommand{\BQ}[0]{\mathbf{Q}}
\newcommand{\BR}[0]{\mathbf{R}}
\newcommand{\BS}[0]{\mathbf{S}}
\newcommand{\BT}[0]{\mathbf{T}}
\newcommand{\BU}[0]{\mathbf{U}}
\newcommand{\BV}[0]{\mathbf{V}}
\newcommand{\BW}[0]{\mathbf{W}}
\newcommand{\BX}[0]{\mathbf{X}}
\newcommand{\BY}[0]{\mathbf{Y}}
\newcommand{\BZ}[0]{\mathbf{Z}}

\newcommand{\Bzero}[0]{\mathbf{0}}
\newcommand{\Btheta}[0]{\boldsymbol{\theta}}
\newcommand{\Btau}[0]{\boldsymbol{\tau}}
\newcommand{\Bomega}[0]{\boldsymbol{\omega}}

%
% shorthand for unit vectors
%
\newcommand{\acap}[0]{\hat{\Ba}}
\newcommand{\bcap}[0]{\hat{\Bb}}
\newcommand{\ccap}[0]{\hat{\Bc}}
\newcommand{\dcap}[0]{\hat{\Bd}}
\newcommand{\ecap}[0]{\hat{\Be}}
\newcommand{\fcap}[0]{\hat{\Bf}}
\newcommand{\gcap}[0]{\hat{\Bg}}
\newcommand{\hcap}[0]{\hat{\Bh}}
\newcommand{\icap}[0]{\hat{\Bi}}
\newcommand{\jcap}[0]{\hat{\Bj}}
\newcommand{\kcap}[0]{\hat{\Bk}}
\newcommand{\lcap}[0]{\hat{\Bl}}
\newcommand{\mcap}[0]{\hat{\Bm}}
\newcommand{\ncap}[0]{\hat{\Bn}}
\newcommand{\ocap}[0]{\hat{\Bo}}
\newcommand{\pcap}[0]{\hat{\Bp}}
\newcommand{\qcap}[0]{\hat{\Bq}}
\newcommand{\rcap}[0]{\hat{\Br}}
\newcommand{\scap}[0]{\hat{\Bs}}
\newcommand{\tcap}[0]{\hat{\Bt}}
\newcommand{\ucap}[0]{\hat{\Bu}}
\newcommand{\vcap}[0]{\hat{\Bv}}
\newcommand{\wcap}[0]{\hat{\Bw}}
\newcommand{\xcap}[0]{\hat{\Bx}}
\newcommand{\ycap}[0]{\hat{\By}}
\newcommand{\zcap}[0]{\hat{\Bz}}
\newcommand{\thetacap}[0]{\hat{\Btheta}}

%
% to write R^n and C^n in a distinguishable fashion.  Perhaps change this
% to the double lined characters upon figuring out how to do so.
%
\newcommand{\C}[1]{$\mathbb{C}^{#1}$}
\newcommand{\R}[1]{$\mathbb{R}^{#1}$}

%
% various generally useful helpers
%

% derivative of #1 wrt. #2:
\newcommand{\D}[2] {\frac {d#2} {d#1}}

\newcommand{\inv}[1]{\frac{1}{#1}}
\newcommand{\cross}[0]{\times}

\newcommand{\abs}[1]{\lvert{#1}\rvert}
\newcommand{\norm}[1]{\lVert{#1}\rVert}
\newcommand{\innerprod}[2]{\langle{#1}, {#2}\rangle}
\newcommand{\dotprod}[2]{{#1} \cdot {#2}}
\newcommand{\bdotprod}[2]{\left({#1} \cdot {#2}\right)}
\newcommand{\crossprod}[2]{{#1} \cross {#2}}
\newcommand{\tripleprod}[3]{\dotprod{\left(\crossprod{#1}{#2}\right)}{#3}}

\DeclareMathOperator{\Proj}{Proj}
\DeclareMathOperator{\Span}{span}
\DeclareMathOperator{\Sgn}{sgn}
\DeclareMathOperator{\Area}{Area}
\DeclareMathOperator{\Volume}{Volume}

%
% A few miscellaneous things specific to this document
%
\newcommand{\crossop}[1]{\crossprod{#1}{}}

% R2 vector.
\newcommand{\VectorTwo}[2]{
\begin{bmatrix}
 {#1} \\
 {#2}
\end{bmatrix}
}

\newcommand{\VectorN}[1]{
\begin{bmatrix}
{#1}_1 \\
{#1}_2 \\
\vdots \\
{#1}_N \\
\end{bmatrix}
}

\newcommand{\DETuvij}[4]{
\begin{vmatrix}
 {#1}_{#3} & {#1}_{#4} \\
 {#2}_{#3} & {#2}_{#4}
\end{vmatrix}
}

\newcommand{\DETuvwijk}[6]{
\begin{vmatrix}
 {#1}_{#4} & {#1}_{#5} & {#1}_{#6} \\
 {#2}_{#4} & {#2}_{#5} & {#2}_{#6} \\
 {#3}_{#4} & {#3}_{#5} & {#3}_{#6}
\end{vmatrix}
}

\newcommand{\DETuvwxijkl}[8]{
\begin{vmatrix}
 {#1}_{#5} & {#1}_{#6} & {#1}_{#7} & {#1}_{#8} \\
 {#2}_{#5} & {#2}_{#6} & {#2}_{#7} & {#2}_{#8} \\
 {#3}_{#5} & {#3}_{#6} & {#3}_{#7} & {#3}_{#8} \\
 {#4}_{#5} & {#4}_{#6} & {#4}_{#7} & {#4}_{#8} \\
\end{vmatrix}
}

%\newcommand{\DETuvwxyijklm}[10]{
%\begin{vmatrix}
% {#1}_{#6} & {#1}_{#7} & {#1}_{#8} & {#1}_{#9} & {#1}_{#10} \\
% {#2}_{#6} & {#2}_{#7} & {#2}_{#8} & {#2}_{#9} & {#2}_{#10} \\
% {#3}_{#6} & {#3}_{#7} & {#3}_{#8} & {#3}_{#9} & {#3}_{#10} \\
% {#4}_{#6} & {#4}_{#7} & {#4}_{#8} & {#4}_{#9} & {#4}_{#10} \\
% {#5}_{#6} & {#5}_{#7} & {#5}_{#8} & {#5}_{#9} & {#5}_{#10}
%\end{vmatrix}
%}

% R3 vector.
\newcommand{\VectorThree}[3]{
\begin{bmatrix}
 {#1} \\
 {#2} \\
 {#3}
\end{bmatrix}
}



\author{Peeter Joot}
\email{peeter.joot@gmail.com}

%\documentclass[]{eliblogwidescreen}

\usepackage{amsmath}
\usepackage{mathpazo}

%
% shorthand for bold symbols, convenient for vectors and matrices
%
\newcommand{\Ba}[0]{\mathbf{a}}
\newcommand{\Bb}[0]{\mathbf{b}}
\newcommand{\Bc}[0]{\mathbf{c}}
\newcommand{\Bd}[0]{\mathbf{d}}
\newcommand{\Be}[0]{\mathbf{e}}
\newcommand{\Bf}[0]{\mathbf{f}}
\newcommand{\Bg}[0]{\mathbf{g}}
\newcommand{\Bh}[0]{\mathbf{h}}
\newcommand{\Bi}[0]{\mathbf{i}}
\newcommand{\Bj}[0]{\mathbf{j}}
\newcommand{\Bk}[0]{\mathbf{k}}
\newcommand{\Bl}[0]{\mathbf{l}}
\newcommand{\Bm}[0]{\mathbf{m}}
\newcommand{\Bn}[0]{\mathbf{n}}
\newcommand{\Bo}[0]{\mathbf{o}}
\newcommand{\Bp}[0]{\mathbf{p}}
\newcommand{\Bq}[0]{\mathbf{q}}
\newcommand{\Br}[0]{\mathbf{r}}
\newcommand{\Bs}[0]{\mathbf{s}}
\newcommand{\Bt}[0]{\mathbf{t}}
\newcommand{\Bu}[0]{\mathbf{u}}
\newcommand{\Bv}[0]{\mathbf{v}}
\newcommand{\Bw}[0]{\mathbf{w}}
\newcommand{\Bx}[0]{\mathbf{x}}
\newcommand{\By}[0]{\mathbf{y}}
\newcommand{\Bz}[0]{\mathbf{z}}
\newcommand{\BA}[0]{\mathbf{A}}
\newcommand{\BB}[0]{\mathbf{B}}
\newcommand{\BC}[0]{\mathbf{C}}
\newcommand{\BD}[0]{\mathbf{D}}
\newcommand{\BE}[0]{\mathbf{E}}
\newcommand{\BF}[0]{\mathbf{F}}
\newcommand{\BG}[0]{\mathbf{G}}
\newcommand{\BH}[0]{\mathbf{H}}
\newcommand{\BI}[0]{\mathbf{I}}
\newcommand{\BJ}[0]{\mathbf{J}}
\newcommand{\BK}[0]{\mathbf{K}}
\newcommand{\BL}[0]{\mathbf{L}}
\newcommand{\BM}[0]{\mathbf{M}}
\newcommand{\BN}[0]{\mathbf{N}}
\newcommand{\BO}[0]{\mathbf{O}}
\newcommand{\BP}[0]{\mathbf{P}}
\newcommand{\BQ}[0]{\mathbf{Q}}
\newcommand{\BR}[0]{\mathbf{R}}
\newcommand{\BS}[0]{\mathbf{S}}
\newcommand{\BT}[0]{\mathbf{T}}
\newcommand{\BU}[0]{\mathbf{U}}
\newcommand{\BV}[0]{\mathbf{V}}
\newcommand{\BW}[0]{\mathbf{W}}
\newcommand{\BX}[0]{\mathbf{X}}
\newcommand{\BY}[0]{\mathbf{Y}}
\newcommand{\BZ}[0]{\mathbf{Z}}

\newcommand{\Bzero}[0]{\mathbf{0}}
\newcommand{\Btheta}[0]{\boldsymbol{\theta}}
\newcommand{\Btau}[0]{\boldsymbol{\tau}}
\newcommand{\Bomega}[0]{\boldsymbol{\omega}}

%
% shorthand for unit vectors
%
\newcommand{\acap}[0]{\hat{\Ba}}
\newcommand{\bcap}[0]{\hat{\Bb}}
\newcommand{\ccap}[0]{\hat{\Bc}}
\newcommand{\dcap}[0]{\hat{\Bd}}
\newcommand{\ecap}[0]{\hat{\Be}}
\newcommand{\fcap}[0]{\hat{\Bf}}
\newcommand{\gcap}[0]{\hat{\Bg}}
\newcommand{\hcap}[0]{\hat{\Bh}}
\newcommand{\icap}[0]{\hat{\Bi}}
\newcommand{\jcap}[0]{\hat{\Bj}}
\newcommand{\kcap}[0]{\hat{\Bk}}
\newcommand{\lcap}[0]{\hat{\Bl}}
\newcommand{\mcap}[0]{\hat{\Bm}}
\newcommand{\ncap}[0]{\hat{\Bn}}
\newcommand{\ocap}[0]{\hat{\Bo}}
\newcommand{\pcap}[0]{\hat{\Bp}}
\newcommand{\qcap}[0]{\hat{\Bq}}
\newcommand{\rcap}[0]{\hat{\Br}}
\newcommand{\scap}[0]{\hat{\Bs}}
\newcommand{\tcap}[0]{\hat{\Bt}}
\newcommand{\ucap}[0]{\hat{\Bu}}
\newcommand{\vcap}[0]{\hat{\Bv}}
\newcommand{\wcap}[0]{\hat{\Bw}}
\newcommand{\xcap}[0]{\hat{\Bx}}
\newcommand{\ycap}[0]{\hat{\By}}
\newcommand{\zcap}[0]{\hat{\Bz}}
\newcommand{\thetacap}[0]{\hat{\Btheta}}

%
% to write R^n and C^n in a distinguishable fashion.  Perhaps change this
% to the double lined characters upon figuring out how to do so.
%
\newcommand{\C}[1]{$\mathbb{C}^{#1}$}
\newcommand{\R}[1]{$\mathbb{R}^{#1}$}

%
% various generally useful helpers
%

% derivative of #1 wrt. #2:
\newcommand{\D}[2] {\frac {d#2} {d#1}}

\newcommand{\inv}[1]{\frac{1}{#1}}
\newcommand{\cross}[0]{\times}

\newcommand{\abs}[1]{\lvert{#1}\rvert}
\newcommand{\norm}[1]{\lVert{#1}\rVert}
\newcommand{\innerprod}[2]{\langle{#1}, {#2}\rangle}
\newcommand{\dotprod}[2]{{#1} \cdot {#2}}
\newcommand{\bdotprod}[2]{\left({#1} \cdot {#2}\right)}
\newcommand{\crossprod}[2]{{#1} \cross {#2}}
\newcommand{\tripleprod}[3]{\dotprod{\left(\crossprod{#1}{#2}\right)}{#3}}

\DeclareMathOperator{\Proj}{Proj}
\DeclareMathOperator{\Span}{span}
\DeclareMathOperator{\Sgn}{sgn}
\DeclareMathOperator{\Area}{Area}
\DeclareMathOperator{\Volume}{Volume}

%
% A few miscellaneous things specific to this document
%
\newcommand{\crossop}[1]{\crossprod{#1}{}}

% R2 vector.
\newcommand{\VectorTwo}[2]{
\begin{bmatrix}
 {#1} \\
 {#2}
\end{bmatrix}
}

\newcommand{\VectorN}[1]{
\begin{bmatrix}
{#1}_1 \\
{#1}_2 \\
\vdots \\
{#1}_N \\
\end{bmatrix}
}

\newcommand{\DETuvij}[4]{
\begin{vmatrix}
 {#1}_{#3} & {#1}_{#4} \\
 {#2}_{#3} & {#2}_{#4}
\end{vmatrix}
}

\newcommand{\DETuvwijk}[6]{
\begin{vmatrix}
 {#1}_{#4} & {#1}_{#5} & {#1}_{#6} \\
 {#2}_{#4} & {#2}_{#5} & {#2}_{#6} \\
 {#3}_{#4} & {#3}_{#5} & {#3}_{#6}
\end{vmatrix}
}

\newcommand{\DETuvwxijkl}[8]{
\begin{vmatrix}
 {#1}_{#5} & {#1}_{#6} & {#1}_{#7} & {#1}_{#8} \\
 {#2}_{#5} & {#2}_{#6} & {#2}_{#7} & {#2}_{#8} \\
 {#3}_{#5} & {#3}_{#6} & {#3}_{#7} & {#3}_{#8} \\
 {#4}_{#5} & {#4}_{#6} & {#4}_{#7} & {#4}_{#8} \\
\end{vmatrix}
}

%\newcommand{\DETuvwxyijklm}[10]{
%\begin{vmatrix}
% {#1}_{#6} & {#1}_{#7} & {#1}_{#8} & {#1}_{#9} & {#1}_{#10} \\
% {#2}_{#6} & {#2}_{#7} & {#2}_{#8} & {#2}_{#9} & {#2}_{#10} \\
% {#3}_{#6} & {#3}_{#7} & {#3}_{#8} & {#3}_{#9} & {#3}_{#10} \\
% {#4}_{#6} & {#4}_{#7} & {#4}_{#8} & {#4}_{#9} & {#4}_{#10} \\
% {#5}_{#6} & {#5}_{#7} & {#5}_{#8} & {#5}_{#9} & {#5}_{#10}
%\end{vmatrix}
%}

% R3 vector.
\newcommand{\VectorThree}[3]{
\begin{bmatrix}
 {#1} \\
 {#2} \\
 {#3}
\end{bmatrix}
}



\author{Peeter Joot}
\email{peeter.joot@gmail.com}


\chapter{Review: Composite systems}
\label{chap:qmTwoL1}
%\useCCL
\blogpage{http://sites.google.com/site/peeterjoot/math2011/qmTwoL1.pdf}
\date{Sept 12, 2011}
\revisionInfo{qmTwoL1.tex}

\beginArtWithToc
%\beginArtNoToc

\section{Composite systems.}

This is apparently covered as a side effect in the text \cite{desai2009quantum} in one of the advanced material sections.  FIXME: what section?

Example, one spin one half particle and one spin one particle.  We can describe either quantum mechanically, described by a pair of Hilbert spaces

\begin{equation}\label{eqn:qmTwoL1:10}
H_1,
\end{equation}

of dimension $D_1$

\begin{equation}\label{eqn:qmTwoL1:30}
H_2,
\end{equation}

of dimension $D_2$

Recall that a Hilbert space (finite or infinite dimensional) is the set of states that describe the system.  There were some additional details (completeness, normalizable, $L2$ integrable, ...) not really covered in the physics curriculum, but available in mathematical descriptions.

We form the composite (Hilbert) space

\begin{equation}\label{eqn:qmTwoL1:50}
H = H_1 \otimes H_2
\end{equation}

\begin{equation}\label{eqn:qmTwoL1:70}
H_1 : { \ket{\phi_1^{(i)}} }
\end{equation}

for any ket in $H_1$

\begin{equation}\label{eqn:qmTwoL1:90}
\ket{I} = \sum_{i=1}^{D_1} c_i \ket{\phi_1^{(i)}} 
\end{equation}

where

\begin{equation}\label{eqn:qmTwoL1:110}
\braket{ \phi_1^{(i)}}{ \phi_1^{(j)}} = \delta^{i j}
\end{equation}

Similarly
\begin{equation}\label{eqn:qmTwoL1:130}
H_2 : { \ket{\phi_2^{(i)}} }
\end{equation}

for any ket in $H_2$

\begin{equation}\label{eqn:qmTwoL1:150}
\ket{II} = \sum_{i=1}^{D_2} d_i \ket{\phi_2^{(i)}} 
\end{equation}

where

\begin{equation}\label{eqn:qmTwoL1:170}
\braket{ \phi_2^{(i)}}{ \phi_2^{(j)}} = \delta^{i j}
\end{equation}

The composite Hilbert space has dimension $D_1 D_2$

basis kets:

\begin{equation}\label{eqn:qmTwoL1:190}
\ket{ \phi_1^{(i)}} \otimes \ket{ \phi_2^{(j)}}  = \ket{ \phi^{(ij)}},
\end{equation}

where
\begin{equation}\label{eqn:qmTwoL1:210}
\braket{ \phi^{(ij)}}{ \phi^{(kl)}} = \delta^{ik} \delta^{jl}.
\end{equation}

Any ket in $H$ can be written

\begin{align*}
\ket{\psi} 
&= 
\sum_{i = 1}^{D_1}
\sum_{j = 1}^{D_2}
f_{ij}
\ket{ \phi_1^{(i)}} \otimes \ket{ \phi_2^{(j)}}  \\
&= 
\sum_{i = 1}^{D_1}
\sum_{j = 1}^{D_2}
f_{ij}
\ket{ \phi^{(ij)}}.
\end{align*}

\paragraph{Direct product of kets:}

\begin{align*}
\ket{I} \otimes \ket{II} 
&\equiv
\sum_{i = 1}^{D_1}
\sum_{j = 1}^{D_2}
c_i d_j
\ket{ \phi_1^{(i)}} \otimes \ket{ \phi_2^{(j)}} \\
&=
\sum_{i = 1}^{D_1}
\sum_{j = 1}^{D_2}
c_i d_j
\ket{ \phi^{(ij)}} 
\end{align*}

If $\ket{\psi}$ in $H$ cannot be written as $\ket{I} \otimes \ket{II}$, then $\ket{\psi}$ is said to be ``entangled''.

FIXME: insert a concrete example of this, with some low dimension.

\subsection{Operators.}

With operators $\mathcal{O}_1$ and $\mathcal{O}_2$ on the respective Hilbert spaces.  We'd now like to build 

\begin{equation}\label{eqn:qmTwoL1:230}
\mathcal{O}_1 \otimes \mathcal{O}_2
\end{equation}

If one defines
\begin{equation}\label{eqn:qmTwoL1:250}
\mathcal{O}_1 \otimes \mathcal{O}_2
\equiv
\sum_{i = 1}^{D_1}
\sum_{j = 1}^{D_2}
f_{ij}
\ket{ \mathcal{O}_1 \phi_1^{(i)}} \otimes \ket{ \mathcal{O}_2 \phi_2^{(j)}} 
\end{equation}

\paragraph{Q:Can every operator that can be defined on the composite space have a representation of this form?}

No.

Special cases.  The identity operators.  Suppose that 

\begin{equation}\label{eqn:qmTwoL1:270}
\ket{\psi}
=
\sum_{i = 1}^{D_1}
\sum_{j = 1}^{D_2}
f_{ij}
\ket{ \phi_1^{(i)}} \otimes \ket{ \phi_2^{(j)}} 
\end{equation}

then

\begin{equation}\label{eqn:qmTwoL1:290}
(\mathcal{O}_1 \otimes \mathcal{I}_2) \ket{\psi}
=
\sum_{i = 1}^{D_1}
\sum_{j = 1}^{D_2}
f_{ij}
\ket{ \mathcal{O}_1 \phi_1^{(i)}} \otimes \ket{ \phi_2^{(j)}} 
\end{equation}

\subsubsection{Example commutator.}

Can do other operations.  Example:

\begin{equation}\label{eqn:qmTwoL1:310}
\antisymmetric{ \mathcal{O}_1 \otimes \mathcal{I}_2 }{ \mathcal{I}_1 \otimes \mathcal{O}_2 } = 0
\end{equation}

Let's verify this one.  Suppose that our state has the representation

\begin{equation}\label{eqn:qmTwoL1:330}
\ket{\psi} 
= 
\sum_{i = 1}^{D_1}
\sum_{j = 1}^{D_2}
f_{ij}
\ket{ \phi_1^{(i)}} \otimes \ket{ \phi_2^{(j)}}
\end{equation}

so that the action on this ket from the composite operations are
\begin{align}\label{eqn:qmTwoL1:350}
(\mathcal{O}_1 \otimes \mathcal{I}_2)
\ket{\psi} 
&= 
\sum_{i = 1}^{D_1}
\sum_{j = 1}^{D_2}
f_{ij}
\ket{ \mathcal{O}_1 \phi_1^{(i)}} \otimes \ket{ \phi_2^{(j)}} \\
(\mathcal{I}_1 \otimes \mathcal{O}_2)
\ket{\psi} 
&= 
\sum_{i = 1}^{D_1}
\sum_{j = 1}^{D_2}
f_{ij}
\ket{ \phi_1^{(i)}} \otimes \ket{ \mathcal{O}_2 \phi_2^{(j)}}
\end{align}

Our commutator is
\begin{align*}
&\antisymmetric{(\mathcal{O}_1 \otimes \mathcal{I}_2)}{(\mathcal{I}_1 \otimes \mathcal{O}_2)}
\ket{\psi} \\
&=
(\mathcal{O}_1 \otimes \mathcal{I}_2)(\mathcal{I}_1 \otimes \mathcal{O}_2) 
\ket{\psi} 
-(\mathcal{I}_1 \otimes \mathcal{O}_2)(\mathcal{O}_1 \otimes \mathcal{I}_2)
\ket{\psi}  \\
&=
(\mathcal{O}_1 \otimes \mathcal{I}_2)
\sum_{i = 1}^{D_1}
\sum_{j = 1}^{D_2}
f_{ij}
\ket{ \phi_1^{(i)}} \otimes \ket{ \mathcal{O}_2 \phi_2^{(j)}}
-(\mathcal{I}_1 \otimes \mathcal{O}_2)
\sum_{i = 1}^{D_1}
\sum_{j = 1}^{D_2}
f_{ij}
\ket{ \mathcal{O}_1 \phi_1^{(i)}} \otimes \ket{ \phi_2^{(j)}} \\
&=
\sum_{i = 1}^{D_1}
\sum_{j = 1}^{D_2}
f_{ij}
\ket{ \mathcal{O}_1 \phi_1^{(i)}} \otimes \ket{ \mathcal{O}_2 \phi_2^{(j)}}
-
\sum_{i = 1}^{D_1}
\sum_{j = 1}^{D_2}
f_{ij}
\ket{ \mathcal{O}_1 \phi_1^{(i)}} \otimes \ket{ \mathcal{O}_2 \phi_2^{(j)}} \\
&=
0 \qquad \square
\end{align*}

\subsubsection{Generalizations.}

Can generalize to 

\begin{equation}\label{eqn:qmTwoL1:370}
H_1 \otimes H_2 \otimes H_3 \otimes \cdots
\end{equation}

Can also start with $H$ and seek factor spaces.  If $H$ is not prime there are, in general, many ways to find factor spaces

\begin{equation}\label{eqn:qmTwoL1:390}
H = 
H_1 \otimes H_2 =
H_1' \otimes H_2'
\end{equation}

A ket $\ket{\psi}$, if unentangled in the first factor space, then it will be in general entangled in a second space.  Thus ket entanglement is not a property of the ket itself, but instead is intrinsically related to the space in which it is represented.

\section{An example}

We had one example of a composite system in phy356 that I recall.
It was related to states of the silver atoms in a Stern Gerlach
apparatus, where we had one state from the Hamiltonian that governs
position and momentum and another from the Hamiltonian for the spin,
where each of these states was considered separately.

This makes me wonder what would the Hamiltonian for a system (say a single
electron) that includes both spin and position/momentum would look
like, and how is it that one can solve this taking spin and non-spin
states separately?

Professor Sipe, when asked said of this

``It is complicated because not only would the spin of the electron interact with the magnetic field, but its translational motion would respond to the magnetic field too.  A simpler case is a neutral atom with an electron with an unpaired spin.  Then there is no Lorentz force on the atom itself.  The Hamiltonian is just the sum of a free particle Hamiltonian and a Zeeman term due to the spin interacting with the magnetic field.  This is precisely the Stern-Gerlach problem''

I didn't remember what the Zeeman term looked like, but wikipedia does \cite{wiki:zeeman}, and it is the magnetic field interaction

\begin{equation}\label{eqn:qmTwoL1:n}
-\Bmu \cdot \BB
\end{equation}

that we get when we gauge transform the Dirac equation for the electron as covered in \S 36.4 of the text (also introduced in chapter 6, which wasn't covered in class).  That doesn't look too much like how we studied the Stern-Gerlach problem?  I thought that for that problem we had a Hamiltonian of the form

\begin{equation}\label{eqn:qmTwoL1:n}
H = a_{i j} \ket{i} \bra{j}
\end{equation}

It's not clear to me how this ket-bra Hamiltonian and the Zeeman Hamiltonian are related (ie: the spin Hamiltonians that we used in 356 and were on old 356 exams were all pulled out of magic hats and it wasn't obvious where these came from).

\EndArticle

%
% Copyright � 2012 Peeter Joot.  All Rights Reserved.
% Licenced as described in the file LICENSE under the root directory of this GIT repository.
%

% 
% 
%%
% Copyright � 2015 Peeter Joot.  All Rights Reserved.
% Licenced as described in the file LICENSE under the root directory of this GIT repository.
%
\documentclass[]{eliblog}

\usepackage{amsmath}
\usepackage{mathpazo}

%
% shorthand for bold symbols, convenient for vectors and matrices
%
\newcommand{\Ba}[0]{\mathbf{a}}
\newcommand{\Bb}[0]{\mathbf{b}}
\newcommand{\Bc}[0]{\mathbf{c}}
\newcommand{\Bd}[0]{\mathbf{d}}
\newcommand{\Be}[0]{\mathbf{e}}
\newcommand{\Bf}[0]{\mathbf{f}}
\newcommand{\Bg}[0]{\mathbf{g}}
\newcommand{\Bh}[0]{\mathbf{h}}
\newcommand{\Bi}[0]{\mathbf{i}}
\newcommand{\Bj}[0]{\mathbf{j}}
\newcommand{\Bk}[0]{\mathbf{k}}
\newcommand{\Bl}[0]{\mathbf{l}}
\newcommand{\Bm}[0]{\mathbf{m}}
\newcommand{\Bn}[0]{\mathbf{n}}
\newcommand{\Bo}[0]{\mathbf{o}}
\newcommand{\Bp}[0]{\mathbf{p}}
\newcommand{\Bq}[0]{\mathbf{q}}
\newcommand{\Br}[0]{\mathbf{r}}
\newcommand{\Bs}[0]{\mathbf{s}}
\newcommand{\Bt}[0]{\mathbf{t}}
\newcommand{\Bu}[0]{\mathbf{u}}
\newcommand{\Bv}[0]{\mathbf{v}}
\newcommand{\Bw}[0]{\mathbf{w}}
\newcommand{\Bx}[0]{\mathbf{x}}
\newcommand{\By}[0]{\mathbf{y}}
\newcommand{\Bz}[0]{\mathbf{z}}
\newcommand{\BA}[0]{\mathbf{A}}
\newcommand{\BB}[0]{\mathbf{B}}
\newcommand{\BC}[0]{\mathbf{C}}
\newcommand{\BD}[0]{\mathbf{D}}
\newcommand{\BE}[0]{\mathbf{E}}
\newcommand{\BF}[0]{\mathbf{F}}
\newcommand{\BG}[0]{\mathbf{G}}
\newcommand{\BH}[0]{\mathbf{H}}
\newcommand{\BI}[0]{\mathbf{I}}
\newcommand{\BJ}[0]{\mathbf{J}}
\newcommand{\BK}[0]{\mathbf{K}}
\newcommand{\BL}[0]{\mathbf{L}}
\newcommand{\BM}[0]{\mathbf{M}}
\newcommand{\BN}[0]{\mathbf{N}}
\newcommand{\BO}[0]{\mathbf{O}}
\newcommand{\BP}[0]{\mathbf{P}}
\newcommand{\BQ}[0]{\mathbf{Q}}
\newcommand{\BR}[0]{\mathbf{R}}
\newcommand{\BS}[0]{\mathbf{S}}
\newcommand{\BT}[0]{\mathbf{T}}
\newcommand{\BU}[0]{\mathbf{U}}
\newcommand{\BV}[0]{\mathbf{V}}
\newcommand{\BW}[0]{\mathbf{W}}
\newcommand{\BX}[0]{\mathbf{X}}
\newcommand{\BY}[0]{\mathbf{Y}}
\newcommand{\BZ}[0]{\mathbf{Z}}

\newcommand{\Bzero}[0]{\mathbf{0}}
\newcommand{\Btheta}[0]{\boldsymbol{\theta}}
\newcommand{\Btau}[0]{\boldsymbol{\tau}}
\newcommand{\Bomega}[0]{\boldsymbol{\omega}}

%
% shorthand for unit vectors
%
\newcommand{\acap}[0]{\hat{\Ba}}
\newcommand{\bcap}[0]{\hat{\Bb}}
\newcommand{\ccap}[0]{\hat{\Bc}}
\newcommand{\dcap}[0]{\hat{\Bd}}
\newcommand{\ecap}[0]{\hat{\Be}}
\newcommand{\fcap}[0]{\hat{\Bf}}
\newcommand{\gcap}[0]{\hat{\Bg}}
\newcommand{\hcap}[0]{\hat{\Bh}}
\newcommand{\icap}[0]{\hat{\Bi}}
\newcommand{\jcap}[0]{\hat{\Bj}}
\newcommand{\kcap}[0]{\hat{\Bk}}
\newcommand{\lcap}[0]{\hat{\Bl}}
\newcommand{\mcap}[0]{\hat{\Bm}}
\newcommand{\ncap}[0]{\hat{\Bn}}
\newcommand{\ocap}[0]{\hat{\Bo}}
\newcommand{\pcap}[0]{\hat{\Bp}}
\newcommand{\qcap}[0]{\hat{\Bq}}
\newcommand{\rcap}[0]{\hat{\Br}}
\newcommand{\scap}[0]{\hat{\Bs}}
\newcommand{\tcap}[0]{\hat{\Bt}}
\newcommand{\ucap}[0]{\hat{\Bu}}
\newcommand{\vcap}[0]{\hat{\Bv}}
\newcommand{\wcap}[0]{\hat{\Bw}}
\newcommand{\xcap}[0]{\hat{\Bx}}
\newcommand{\ycap}[0]{\hat{\By}}
\newcommand{\zcap}[0]{\hat{\Bz}}
\newcommand{\thetacap}[0]{\hat{\Btheta}}

%
% to write R^n and C^n in a distinguishable fashion.  Perhaps change this
% to the double lined characters upon figuring out how to do so.
%
\newcommand{\C}[1]{$\mathbb{C}^{#1}$}
\newcommand{\R}[1]{$\mathbb{R}^{#1}$}

%
% various generally useful helpers
%

% derivative of #1 wrt. #2:
\newcommand{\D}[2] {\frac {d#2} {d#1}}

\newcommand{\inv}[1]{\frac{1}{#1}}
\newcommand{\cross}[0]{\times}

\newcommand{\abs}[1]{\lvert{#1}\rvert}
\newcommand{\norm}[1]{\lVert{#1}\rVert}
\newcommand{\innerprod}[2]{\langle{#1}, {#2}\rangle}
\newcommand{\dotprod}[2]{{#1} \cdot {#2}}
\newcommand{\bdotprod}[2]{\left({#1} \cdot {#2}\right)}
\newcommand{\crossprod}[2]{{#1} \cross {#2}}
\newcommand{\tripleprod}[3]{\dotprod{\left(\crossprod{#1}{#2}\right)}{#3}}

\DeclareMathOperator{\Proj}{Proj}
\DeclareMathOperator{\Span}{span}
\DeclareMathOperator{\Sgn}{sgn}
\DeclareMathOperator{\Area}{Area}
\DeclareMathOperator{\Volume}{Volume}

%
% A few miscellaneous things specific to this document
%
\newcommand{\crossop}[1]{\crossprod{#1}{}}

% R2 vector.
\newcommand{\VectorTwo}[2]{
\begin{bmatrix}
 {#1} \\
 {#2}
\end{bmatrix}
}

\newcommand{\VectorN}[1]{
\begin{bmatrix}
{#1}_1 \\
{#1}_2 \\
\vdots \\
{#1}_N \\
\end{bmatrix}
}

\newcommand{\DETuvij}[4]{
\begin{vmatrix}
 {#1}_{#3} & {#1}_{#4} \\
 {#2}_{#3} & {#2}_{#4}
\end{vmatrix}
}

\newcommand{\DETuvwijk}[6]{
\begin{vmatrix}
 {#1}_{#4} & {#1}_{#5} & {#1}_{#6} \\
 {#2}_{#4} & {#2}_{#5} & {#2}_{#6} \\
 {#3}_{#4} & {#3}_{#5} & {#3}_{#6}
\end{vmatrix}
}

\newcommand{\DETuvwxijkl}[8]{
\begin{vmatrix}
 {#1}_{#5} & {#1}_{#6} & {#1}_{#7} & {#1}_{#8} \\
 {#2}_{#5} & {#2}_{#6} & {#2}_{#7} & {#2}_{#8} \\
 {#3}_{#5} & {#3}_{#6} & {#3}_{#7} & {#3}_{#8} \\
 {#4}_{#5} & {#4}_{#6} & {#4}_{#7} & {#4}_{#8} \\
\end{vmatrix}
}

%\newcommand{\DETuvwxyijklm}[10]{
%\begin{vmatrix}
% {#1}_{#6} & {#1}_{#7} & {#1}_{#8} & {#1}_{#9} & {#1}_{#10} \\
% {#2}_{#6} & {#2}_{#7} & {#2}_{#8} & {#2}_{#9} & {#2}_{#10} \\
% {#3}_{#6} & {#3}_{#7} & {#3}_{#8} & {#3}_{#9} & {#3}_{#10} \\
% {#4}_{#6} & {#4}_{#7} & {#4}_{#8} & {#4}_{#9} & {#4}_{#10} \\
% {#5}_{#6} & {#5}_{#7} & {#5}_{#8} & {#5}_{#9} & {#5}_{#10}
%\end{vmatrix}
%}

% R3 vector.
\newcommand{\VectorThree}[3]{
\begin{bmatrix}
 {#1} \\
 {#2} \\
 {#3}
\end{bmatrix}
}



\author{Peeter Joot}
\email{peeter.joot@gmail.com}

%\documentclass[]{eliblogwidescreen}

\usepackage{amsmath}
\usepackage{mathpazo}

%
% shorthand for bold symbols, convenient for vectors and matrices
%
\newcommand{\Ba}[0]{\mathbf{a}}
\newcommand{\Bb}[0]{\mathbf{b}}
\newcommand{\Bc}[0]{\mathbf{c}}
\newcommand{\Bd}[0]{\mathbf{d}}
\newcommand{\Be}[0]{\mathbf{e}}
\newcommand{\Bf}[0]{\mathbf{f}}
\newcommand{\Bg}[0]{\mathbf{g}}
\newcommand{\Bh}[0]{\mathbf{h}}
\newcommand{\Bi}[0]{\mathbf{i}}
\newcommand{\Bj}[0]{\mathbf{j}}
\newcommand{\Bk}[0]{\mathbf{k}}
\newcommand{\Bl}[0]{\mathbf{l}}
\newcommand{\Bm}[0]{\mathbf{m}}
\newcommand{\Bn}[0]{\mathbf{n}}
\newcommand{\Bo}[0]{\mathbf{o}}
\newcommand{\Bp}[0]{\mathbf{p}}
\newcommand{\Bq}[0]{\mathbf{q}}
\newcommand{\Br}[0]{\mathbf{r}}
\newcommand{\Bs}[0]{\mathbf{s}}
\newcommand{\Bt}[0]{\mathbf{t}}
\newcommand{\Bu}[0]{\mathbf{u}}
\newcommand{\Bv}[0]{\mathbf{v}}
\newcommand{\Bw}[0]{\mathbf{w}}
\newcommand{\Bx}[0]{\mathbf{x}}
\newcommand{\By}[0]{\mathbf{y}}
\newcommand{\Bz}[0]{\mathbf{z}}
\newcommand{\BA}[0]{\mathbf{A}}
\newcommand{\BB}[0]{\mathbf{B}}
\newcommand{\BC}[0]{\mathbf{C}}
\newcommand{\BD}[0]{\mathbf{D}}
\newcommand{\BE}[0]{\mathbf{E}}
\newcommand{\BF}[0]{\mathbf{F}}
\newcommand{\BG}[0]{\mathbf{G}}
\newcommand{\BH}[0]{\mathbf{H}}
\newcommand{\BI}[0]{\mathbf{I}}
\newcommand{\BJ}[0]{\mathbf{J}}
\newcommand{\BK}[0]{\mathbf{K}}
\newcommand{\BL}[0]{\mathbf{L}}
\newcommand{\BM}[0]{\mathbf{M}}
\newcommand{\BN}[0]{\mathbf{N}}
\newcommand{\BO}[0]{\mathbf{O}}
\newcommand{\BP}[0]{\mathbf{P}}
\newcommand{\BQ}[0]{\mathbf{Q}}
\newcommand{\BR}[0]{\mathbf{R}}
\newcommand{\BS}[0]{\mathbf{S}}
\newcommand{\BT}[0]{\mathbf{T}}
\newcommand{\BU}[0]{\mathbf{U}}
\newcommand{\BV}[0]{\mathbf{V}}
\newcommand{\BW}[0]{\mathbf{W}}
\newcommand{\BX}[0]{\mathbf{X}}
\newcommand{\BY}[0]{\mathbf{Y}}
\newcommand{\BZ}[0]{\mathbf{Z}}

\newcommand{\Bzero}[0]{\mathbf{0}}
\newcommand{\Btheta}[0]{\boldsymbol{\theta}}
\newcommand{\Btau}[0]{\boldsymbol{\tau}}
\newcommand{\Bomega}[0]{\boldsymbol{\omega}}

%
% shorthand for unit vectors
%
\newcommand{\acap}[0]{\hat{\Ba}}
\newcommand{\bcap}[0]{\hat{\Bb}}
\newcommand{\ccap}[0]{\hat{\Bc}}
\newcommand{\dcap}[0]{\hat{\Bd}}
\newcommand{\ecap}[0]{\hat{\Be}}
\newcommand{\fcap}[0]{\hat{\Bf}}
\newcommand{\gcap}[0]{\hat{\Bg}}
\newcommand{\hcap}[0]{\hat{\Bh}}
\newcommand{\icap}[0]{\hat{\Bi}}
\newcommand{\jcap}[0]{\hat{\Bj}}
\newcommand{\kcap}[0]{\hat{\Bk}}
\newcommand{\lcap}[0]{\hat{\Bl}}
\newcommand{\mcap}[0]{\hat{\Bm}}
\newcommand{\ncap}[0]{\hat{\Bn}}
\newcommand{\ocap}[0]{\hat{\Bo}}
\newcommand{\pcap}[0]{\hat{\Bp}}
\newcommand{\qcap}[0]{\hat{\Bq}}
\newcommand{\rcap}[0]{\hat{\Br}}
\newcommand{\scap}[0]{\hat{\Bs}}
\newcommand{\tcap}[0]{\hat{\Bt}}
\newcommand{\ucap}[0]{\hat{\Bu}}
\newcommand{\vcap}[0]{\hat{\Bv}}
\newcommand{\wcap}[0]{\hat{\Bw}}
\newcommand{\xcap}[0]{\hat{\Bx}}
\newcommand{\ycap}[0]{\hat{\By}}
\newcommand{\zcap}[0]{\hat{\Bz}}
\newcommand{\thetacap}[0]{\hat{\Btheta}}

%
% to write R^n and C^n in a distinguishable fashion.  Perhaps change this
% to the double lined characters upon figuring out how to do so.
%
\newcommand{\C}[1]{$\mathbb{C}^{#1}$}
\newcommand{\R}[1]{$\mathbb{R}^{#1}$}

%
% various generally useful helpers
%

% derivative of #1 wrt. #2:
\newcommand{\D}[2] {\frac {d#2} {d#1}}

\newcommand{\inv}[1]{\frac{1}{#1}}
\newcommand{\cross}[0]{\times}

\newcommand{\abs}[1]{\lvert{#1}\rvert}
\newcommand{\norm}[1]{\lVert{#1}\rVert}
\newcommand{\innerprod}[2]{\langle{#1}, {#2}\rangle}
\newcommand{\dotprod}[2]{{#1} \cdot {#2}}
\newcommand{\bdotprod}[2]{\left({#1} \cdot {#2}\right)}
\newcommand{\crossprod}[2]{{#1} \cross {#2}}
\newcommand{\tripleprod}[3]{\dotprod{\left(\crossprod{#1}{#2}\right)}{#3}}

\DeclareMathOperator{\Proj}{Proj}
\DeclareMathOperator{\Span}{span}
\DeclareMathOperator{\Sgn}{sgn}
\DeclareMathOperator{\Area}{Area}
\DeclareMathOperator{\Volume}{Volume}

%
% A few miscellaneous things specific to this document
%
\newcommand{\crossop}[1]{\crossprod{#1}{}}

% R2 vector.
\newcommand{\VectorTwo}[2]{
\begin{bmatrix}
 {#1} \\
 {#2}
\end{bmatrix}
}

\newcommand{\VectorN}[1]{
\begin{bmatrix}
{#1}_1 \\
{#1}_2 \\
\vdots \\
{#1}_N \\
\end{bmatrix}
}

\newcommand{\DETuvij}[4]{
\begin{vmatrix}
 {#1}_{#3} & {#1}_{#4} \\
 {#2}_{#3} & {#2}_{#4}
\end{vmatrix}
}

\newcommand{\DETuvwijk}[6]{
\begin{vmatrix}
 {#1}_{#4} & {#1}_{#5} & {#1}_{#6} \\
 {#2}_{#4} & {#2}_{#5} & {#2}_{#6} \\
 {#3}_{#4} & {#3}_{#5} & {#3}_{#6}
\end{vmatrix}
}

\newcommand{\DETuvwxijkl}[8]{
\begin{vmatrix}
 {#1}_{#5} & {#1}_{#6} & {#1}_{#7} & {#1}_{#8} \\
 {#2}_{#5} & {#2}_{#6} & {#2}_{#7} & {#2}_{#8} \\
 {#3}_{#5} & {#3}_{#6} & {#3}_{#7} & {#3}_{#8} \\
 {#4}_{#5} & {#4}_{#6} & {#4}_{#7} & {#4}_{#8} \\
\end{vmatrix}
}

%\newcommand{\DETuvwxyijklm}[10]{
%\begin{vmatrix}
% {#1}_{#6} & {#1}_{#7} & {#1}_{#8} & {#1}_{#9} & {#1}_{#10} \\
% {#2}_{#6} & {#2}_{#7} & {#2}_{#8} & {#2}_{#9} & {#2}_{#10} \\
% {#3}_{#6} & {#3}_{#7} & {#3}_{#8} & {#3}_{#9} & {#3}_{#10} \\
% {#4}_{#6} & {#4}_{#7} & {#4}_{#8} & {#4}_{#9} & {#4}_{#10} \\
% {#5}_{#6} & {#5}_{#7} & {#5}_{#8} & {#5}_{#9} & {#5}_{#10}
%\end{vmatrix}
%}

% R3 vector.
\newcommand{\VectorThree}[3]{
\begin{bmatrix}
 {#1} \\
 {#2} \\
 {#3}
\end{bmatrix}
}



\author{Peeter Joot}
\email{peeter.joot@gmail.com}


\chapter{Approximate methods.}
%\chapter{PHY456H1F: Quantum Mechanics II.  Lecture 2 (Taught by Prof J.E. Sipe).  Approximate methods.}
\label{chap:qmTwoL2}
%\useCCL
\blogpage{http://sites.google.com/site/peeterjoot/math2011/qmTwoL2.pdf}
\date{Sept 14, 2011}
\revisionInfo{qmTwoL2.tex}

\beginArtWithToc
%\beginArtNoToc

%\section{Disclaimer.}
%
%Peeter's lecture notes from class.  May not be entirely coherent.

\section{Approximate methods for finding energy eigenvalues and eigenkets.}

In many situations one has a Hamiltonian $H$

\begin{equation}\label{eqn:qmTwoL2:10}
H \ket{\Psi_{n \alpha}} = E_n \ket{\Psi_{n \alpha}}
\end{equation}

Here $\alpha$ is a ``degeneracy index'' (example: as in Hydrogen atom).

\paragraph{Why?}

\begin{itemize}
\item Simplifies dynamics

take

\begin{align*}
\ket{\Psi(0)}
= \sum_{n\alpha}
\ket{\Psi_{n \alpha}}
\braket{\Psi_{n \alpha}}{\Psi(0)}
&
= \sum_{n\alpha} c_{n \alpha} \ket{\Psi_{n \alpha}}
\end{align*}

Then
\begin{align*}
\ket{\Psi(t)}
&=
e^{-i H t/\hbar}
\ket{\Psi(0)} \\
&=
\sum_{n\alpha} c_{n \alpha}
e^{-i H t/\hbar}
\ket{\Psi_{n \alpha}}  \\
&=
\sum_{n\alpha} c_{n \alpha}
e^{-i E_n t/\hbar}
\ket{\Psi_{n \alpha}}
\end{align*}

\item ``Applied  field"' can often be thought of a driving the system from one eigenstate to another.

\begin{figure}[htp]
\centering
\includegraphics[totalheight=0.4\textheight]{qmTwoL2fig1}
\caption{qmTwoL2fig1}\label{fig:qmTwoL2:1}
\end{figure}

\item Stat mech.

In thermal equilibrium

\begin{equation}\label{eqn:qmTwoL2:30}
\expectation{\mathcal{O}} =
\frac{\sum_{n \alpha} \bra{\Psi_{n\alpha}} \mathcal{O} \ket{\Psi_{n \alpha}}  e^{-\beta E_n}}{
Z
}
\end{equation}

where

\begin{equation}\label{eqn:qmTwoL2:50}
\beta = \inv{k_B T},
\end{equation}

and

\begin{equation}\label{eqn:qmTwoL2:70}
Z = \sum_{n \alpha} e^{-\beta E_n}
\end{equation}
\end{itemize}

\section{Variational principle}

Consider any ket

\begin{equation}\label{eqn:qmTwoL2:90}
\ket{\Psi} = \sum_{n \alpha} c_{n \alpha} \ket{\Psi_{n \alpha}}
\end{equation}

(perhaps not even normalized), and where

\begin{equation}\label{eqn:qmTwoL2:110}
c_{n \alpha} = \braket{\Psi_{n \alpha}}{\Psi}
\end{equation}

but we don't know these.

\begin{equation}\label{eqn:qmTwoL2:130}
\braket{\Psi}{\Psi} = \sum_{n \alpha} \Abs{c_{n \alpha}}^2
\end{equation}

\begin{align*}
\frac{
\bra{\Psi} H \ket{\Psi}
}{
\braket{\Psi}{\Psi}
}
&=
\frac{
\sum_{n \alpha} \Abs{c_{n \alpha}}^2 E_n
}{
\sum_{m \beta} \Abs{c_{m \beta}}^2
} \\
&\ge
\frac{
\sum_{n \alpha} \Abs{c_{n \alpha}}^2 E_0
}{
\sum_{m \beta} \Abs{c_{m \beta}}^2
}  \\
&=
E_0
\end{align*}

So for any ket we can form the upper bound for the ground state energy

\begin{equation}\label{eqn:qmTwoL2:150}
\frac{
\bra{\Psi} H \ket{\Psi}
}{
\braket{\Psi}{\Psi}
}
\ge E_0
\end{equation}

There's a whole set of strategies based on estimating the ground state energy.  This is called the Variational principle for ground state.  See \S 24.2 in the text \cite{desai2009quantum}.

We define the functional

\begin{equation}\label{eqn:qmTwoL2:170}
E[\Psi] =
\frac{
\bra{\Psi} H \ket{\Psi}
}{
\braket{\Psi}{\Psi}
}
\ge E_0
\end{equation}

If $\ket{\Psi} = c \ket{\Psi_0}$ where $\ket{\Psi_0}$ is the normalized ground state, then

\begin{equation}\label{eqn:qmTwoL2:190}
E[ c \Psi_0 ] = E_0
\end{equation}

\subsection{Examples.}

\subsubsection{Hydrogen atom}

\begin{equation}\label{eqn:qmTwoL2:210}
\bra{\Br} H \ket{\Br'} = \mathcal{H} \delta^3(\Br - \Br')
\end{equation}

where

\begin{equation}\label{eqn:qmTwoL2:230}
\mathcal{H} = -\frac{\hbar^2}{2 \mu} \spacegrad^2 - \frac{e^2}{r}
\end{equation}

Here $\mu$ is the reduced mass.

We know the exact solution:

\begin{equation}\label{eqn:qmTwoL2:250}
H \ket{\Psi_0}
\end{equation}

\begin{equation}\label{eqn:qmTwoL2:270}
E_0 = -R_y
\end{equation}

\begin{equation}\label{eqn:qmTwoL2:290}
R_y = \frac{\mu e^4}{2 \hbar^2} \approx 13.6 \text{eV}
\end{equation}

\begin{equation}\label{eqn:qmTwoL2:310}
\braket{\Br}{\Psi_0} = \Phi_{100}(\Br) = \left( \inv{\pi a_0^3}\right)^{1/2} e^{-r/a_0}
\end{equation}

\begin{equation}\label{eqn:qmTwoL2:330}
a_0 = \frac{\hbar^2}{\mu e^2} \approx 0.53 \text{\AA}
\end{equation}

\begin{figure}[htp]
\centering
\includegraphics[totalheight=0.4\textheight]{qmTwoL2fig2}
\caption{qmTwoL2fig2}\label{fig:qmTwoL2:2}
\end{figure}

\begin{figure}[htp]
\centering
\includegraphics[totalheight=0.4\textheight]{qmTwoL2fig3}
\caption{qmTwoL2fig3}\label{fig:qmTwoL2:3}
\end{figure}

estimate

\begin{align}\label{eqn:qmTwoL2:350}
\bra{\Psi} H \ket{\Psi} &= \int d^3 \Br \Psi^\conj(\Br) \left( -\frac{\hbar^2}{2 \mu} \spacegrad^2 - \frac{e^2}{r} \right) \Psi(\Br) \\
\braket{\Psi}{\Psi} &= \int d^3 \Br \Abs{\Psi(\Br)}^2
\end{align}

Or guess shape

\begin{figure}[htp]
\centering
\includegraphics[totalheight=0.4\textheight]{qmTwoL2fig4}
\caption{qmTwoL2fig4}\label{fig:qmTwoL2:4}
\end{figure}

Using the trial wave function $e^{-\alpha r^2}$

\begin{equation}\label{eqn:qmTwoL2:370}
E[\Psi] \rightarrow E(\alpha)
\end{equation}

\begin{equation}\label{eqn:qmTwoL2:390}
E(\alpha) =
\frac{\int d^3 \Br e^{-\alpha r^2} \left( -\frac{\hbar^2}{2 \mu} \spacegrad^2 - \frac{e^2}{r} \right) e^{-\alpha r^2}}{
\int d^3\Br e^{-2 \alpha r^2}
}
\end{equation}

find
\begin{equation}\label{eqn:qmTwoL2:410}
E(\alpha) = A \alpha - B \alpha^{1/2}
\end{equation}

\begin{align}\label{eqn:qmTwoL2:430}
A &= \frac{3 \hbar^2}{2\mu} \\
B &= 2 e^2 \left( \frac{2}{\pi} \right)^{1/2}
\end{align}

\begin{figure}[htp]
\centering
\includegraphics[totalheight=0.4\textheight]{qmTwoL2fig5}
\caption{qmTwoL2fig5}\label{fig:qmTwoL2:5}
\end{figure}

Minimum at

\begin{equation}\label{eqn:qmTwoL2:450}
\alpha_0 =
\left( \frac{\mu e^2}{ \hbar^2 } \right) \frac{8 }{9 \pi}
\end{equation}

So

\begin{equation}\label{eqn:qmTwoL2:470}
E(\alpha_0) =
- \frac{\mu e^4 }{2 \hbar^2} \frac{8 }{3 \pi} = -0.85 R_y
\end{equation}

maybe not too bad...

\subsubsection{Helium atom}

Assume an infinite nuclear mass with nucleus charge $2 e$

\begin{figure}[htp]
\centering
\includegraphics[totalheight=0.4\textheight]{qmTwoL2fig6}
\caption{qmTwoL2fig6}\label{fig:qmTwoL2:6}
\end{figure}

ground state wavefunction

\begin{equation}\label{eqn:qmTwoL2:490}
\Psi_0(\Br_1, \Br_2)
\end{equation}

The problem that we want to solve is

\begin{equation}\label{eqn:qmTwoL2:510}
\left(
-\frac{\hbar^2}{2 m} \spacegrad_1^2
-\frac{\hbar^2}{2 m} \spacegrad_2^2
- \frac{2 e}{r}
+
\frac{e^2}{\Abs{\Br_1 - \Br_2}}
\right)
\Psi_0(\Br_1, \Br_2) = E_0 \Psi_0(\Br_1, \Br_2)
\end{equation}

Nobody can solve this problem.  It is one of the simplest real problems in QM that cannot be solved exactly.

Suppose that we neglected the electron, electron repulsion.  Then

\begin{equation}\label{eqn:qmTwoL2:530}
\Psi_0(\Br_1, \Br_2)
=
\bar{\Phi}_{100}(\Br_1)
\bar{\Phi}_{100}(\Br_2)
\end{equation}

where

\begin{equation}\label{eqn:qmTwoL2:550}
\left( -\frac{\hbar^2}{2 m} \spacegrad^2
- \frac{2 e}{r} \right)
\bar{\Phi}_{100}(\Br) = \epsilon \bar{\Phi}_{100}(\Br)
\end{equation}

with

\begin{equation}\label{eqn:qmTwoL2:570}
\epsilon = - 4 R_y
\end{equation}

\begin{equation}\label{eqn:qmTwoL2:590}
R_y = \frac{m e^4}{2 \hbar^2}
\end{equation}

This is the solution to

\begin{equation}\label{eqn:qmTwoL2:610}
\left(
-\frac{\hbar^2}{2 m} \spacegrad_1^2
-\frac{\hbar^2}{2 m} \spacegrad_2^2
- \frac{2 e}{r}
\right)
\Psi_0^{(0)}(\Br_1, \Br_2) = E_0 \Psi_0(\Br_1, \Br_2)
=
E_0^{(0)} \Psi_0^{(0)}(\Br_1, \Br_2)
\end{equation}

\begin{equation}\label{eqn:qmTwoL2:630}
E_0^{(0)} = - 8 R_y.
\end{equation}

Now we want to put back in the electron electron repulsion, and make an estimate.

Trial wavefunction

\begin{equation}\label{eqn:qmTwoL2:650}
\Psi(\Br_1, \Br_2, Z) =
\left(
\left(\frac{Z^3}{ \pi a_0^3 }\right)^{1/2} e^{-Z r_1/a_0}
\right)
\left(
\left(\frac{Z^3}{ \pi a_0^3 }\right)^{1/2} e^{-Z r_2/a_0}
\right)
\end{equation}

expect that the best estimate is for $Z \in [1,2]$.

This can be calculated numerically, and we find

\begin{equation}\label{eqn:qmTwoL2:670}
E(Z) = 2 R_Y \left( Z^2 - 4 Z + \frac{5}{8} Z \right)
\end{equation}

The $Z^2$ comes from the kinetic energy.  The $-4 Z$ is the electron nuclear attraction, and the final term is from the electron-electron repulsion.

The actual minimum is

\begin{equation}\label{eqn:qmTwoL2:690}
Z = 2 - \frac{5}{16}
\end{equation}

\begin{equation}\label{eqn:qmTwoL2:710}
E(2 - 5/16) = -77.5 \text{eV}
\end{equation}

Whereas the measured value is $-78.6 \text{eV}$.

\EndArticle

%
% Copyright � 2015 Peeter Joot.  All Rights Reserved.
% Licenced as described in the file LICENSE under the root directory of this GIT repository.
%
\documentclass[]{eliblog}

\usepackage{amsmath}
\usepackage{mathpazo}

%
% shorthand for bold symbols, convenient for vectors and matrices
%
\newcommand{\Ba}[0]{\mathbf{a}}
\newcommand{\Bb}[0]{\mathbf{b}}
\newcommand{\Bc}[0]{\mathbf{c}}
\newcommand{\Bd}[0]{\mathbf{d}}
\newcommand{\Be}[0]{\mathbf{e}}
\newcommand{\Bf}[0]{\mathbf{f}}
\newcommand{\Bg}[0]{\mathbf{g}}
\newcommand{\Bh}[0]{\mathbf{h}}
\newcommand{\Bi}[0]{\mathbf{i}}
\newcommand{\Bj}[0]{\mathbf{j}}
\newcommand{\Bk}[0]{\mathbf{k}}
\newcommand{\Bl}[0]{\mathbf{l}}
\newcommand{\Bm}[0]{\mathbf{m}}
\newcommand{\Bn}[0]{\mathbf{n}}
\newcommand{\Bo}[0]{\mathbf{o}}
\newcommand{\Bp}[0]{\mathbf{p}}
\newcommand{\Bq}[0]{\mathbf{q}}
\newcommand{\Br}[0]{\mathbf{r}}
\newcommand{\Bs}[0]{\mathbf{s}}
\newcommand{\Bt}[0]{\mathbf{t}}
\newcommand{\Bu}[0]{\mathbf{u}}
\newcommand{\Bv}[0]{\mathbf{v}}
\newcommand{\Bw}[0]{\mathbf{w}}
\newcommand{\Bx}[0]{\mathbf{x}}
\newcommand{\By}[0]{\mathbf{y}}
\newcommand{\Bz}[0]{\mathbf{z}}
\newcommand{\BA}[0]{\mathbf{A}}
\newcommand{\BB}[0]{\mathbf{B}}
\newcommand{\BC}[0]{\mathbf{C}}
\newcommand{\BD}[0]{\mathbf{D}}
\newcommand{\BE}[0]{\mathbf{E}}
\newcommand{\BF}[0]{\mathbf{F}}
\newcommand{\BG}[0]{\mathbf{G}}
\newcommand{\BH}[0]{\mathbf{H}}
\newcommand{\BI}[0]{\mathbf{I}}
\newcommand{\BJ}[0]{\mathbf{J}}
\newcommand{\BK}[0]{\mathbf{K}}
\newcommand{\BL}[0]{\mathbf{L}}
\newcommand{\BM}[0]{\mathbf{M}}
\newcommand{\BN}[0]{\mathbf{N}}
\newcommand{\BO}[0]{\mathbf{O}}
\newcommand{\BP}[0]{\mathbf{P}}
\newcommand{\BQ}[0]{\mathbf{Q}}
\newcommand{\BR}[0]{\mathbf{R}}
\newcommand{\BS}[0]{\mathbf{S}}
\newcommand{\BT}[0]{\mathbf{T}}
\newcommand{\BU}[0]{\mathbf{U}}
\newcommand{\BV}[0]{\mathbf{V}}
\newcommand{\BW}[0]{\mathbf{W}}
\newcommand{\BX}[0]{\mathbf{X}}
\newcommand{\BY}[0]{\mathbf{Y}}
\newcommand{\BZ}[0]{\mathbf{Z}}

\newcommand{\Bzero}[0]{\mathbf{0}}
\newcommand{\Btheta}[0]{\boldsymbol{\theta}}
\newcommand{\Btau}[0]{\boldsymbol{\tau}}
\newcommand{\Bomega}[0]{\boldsymbol{\omega}}

%
% shorthand for unit vectors
%
\newcommand{\acap}[0]{\hat{\Ba}}
\newcommand{\bcap}[0]{\hat{\Bb}}
\newcommand{\ccap}[0]{\hat{\Bc}}
\newcommand{\dcap}[0]{\hat{\Bd}}
\newcommand{\ecap}[0]{\hat{\Be}}
\newcommand{\fcap}[0]{\hat{\Bf}}
\newcommand{\gcap}[0]{\hat{\Bg}}
\newcommand{\hcap}[0]{\hat{\Bh}}
\newcommand{\icap}[0]{\hat{\Bi}}
\newcommand{\jcap}[0]{\hat{\Bj}}
\newcommand{\kcap}[0]{\hat{\Bk}}
\newcommand{\lcap}[0]{\hat{\Bl}}
\newcommand{\mcap}[0]{\hat{\Bm}}
\newcommand{\ncap}[0]{\hat{\Bn}}
\newcommand{\ocap}[0]{\hat{\Bo}}
\newcommand{\pcap}[0]{\hat{\Bp}}
\newcommand{\qcap}[0]{\hat{\Bq}}
\newcommand{\rcap}[0]{\hat{\Br}}
\newcommand{\scap}[0]{\hat{\Bs}}
\newcommand{\tcap}[0]{\hat{\Bt}}
\newcommand{\ucap}[0]{\hat{\Bu}}
\newcommand{\vcap}[0]{\hat{\Bv}}
\newcommand{\wcap}[0]{\hat{\Bw}}
\newcommand{\xcap}[0]{\hat{\Bx}}
\newcommand{\ycap}[0]{\hat{\By}}
\newcommand{\zcap}[0]{\hat{\Bz}}
\newcommand{\thetacap}[0]{\hat{\Btheta}}

%
% to write R^n and C^n in a distinguishable fashion.  Perhaps change this
% to the double lined characters upon figuring out how to do so.
%
\newcommand{\C}[1]{$\mathbb{C}^{#1}$}
\newcommand{\R}[1]{$\mathbb{R}^{#1}$}

%
% various generally useful helpers
%

% derivative of #1 wrt. #2:
\newcommand{\D}[2] {\frac {d#2} {d#1}}

\newcommand{\inv}[1]{\frac{1}{#1}}
\newcommand{\cross}[0]{\times}

\newcommand{\abs}[1]{\lvert{#1}\rvert}
\newcommand{\norm}[1]{\lVert{#1}\rVert}
\newcommand{\innerprod}[2]{\langle{#1}, {#2}\rangle}
\newcommand{\dotprod}[2]{{#1} \cdot {#2}}
\newcommand{\bdotprod}[2]{\left({#1} \cdot {#2}\right)}
\newcommand{\crossprod}[2]{{#1} \cross {#2}}
\newcommand{\tripleprod}[3]{\dotprod{\left(\crossprod{#1}{#2}\right)}{#3}}

\DeclareMathOperator{\Proj}{Proj}
\DeclareMathOperator{\Span}{span}
\DeclareMathOperator{\Sgn}{sgn}
\DeclareMathOperator{\Area}{Area}
\DeclareMathOperator{\Volume}{Volume}

%
% A few miscellaneous things specific to this document
%
\newcommand{\crossop}[1]{\crossprod{#1}{}}

% R2 vector.
\newcommand{\VectorTwo}[2]{
\begin{bmatrix}
 {#1} \\
 {#2}
\end{bmatrix}
}

\newcommand{\VectorN}[1]{
\begin{bmatrix}
{#1}_1 \\
{#1}_2 \\
\vdots \\
{#1}_N \\
\end{bmatrix}
}

\newcommand{\DETuvij}[4]{
\begin{vmatrix}
 {#1}_{#3} & {#1}_{#4} \\
 {#2}_{#3} & {#2}_{#4}
\end{vmatrix}
}

\newcommand{\DETuvwijk}[6]{
\begin{vmatrix}
 {#1}_{#4} & {#1}_{#5} & {#1}_{#6} \\
 {#2}_{#4} & {#2}_{#5} & {#2}_{#6} \\
 {#3}_{#4} & {#3}_{#5} & {#3}_{#6}
\end{vmatrix}
}

\newcommand{\DETuvwxijkl}[8]{
\begin{vmatrix}
 {#1}_{#5} & {#1}_{#6} & {#1}_{#7} & {#1}_{#8} \\
 {#2}_{#5} & {#2}_{#6} & {#2}_{#7} & {#2}_{#8} \\
 {#3}_{#5} & {#3}_{#6} & {#3}_{#7} & {#3}_{#8} \\
 {#4}_{#5} & {#4}_{#6} & {#4}_{#7} & {#4}_{#8} \\
\end{vmatrix}
}

%\newcommand{\DETuvwxyijklm}[10]{
%\begin{vmatrix}
% {#1}_{#6} & {#1}_{#7} & {#1}_{#8} & {#1}_{#9} & {#1}_{#10} \\
% {#2}_{#6} & {#2}_{#7} & {#2}_{#8} & {#2}_{#9} & {#2}_{#10} \\
% {#3}_{#6} & {#3}_{#7} & {#3}_{#8} & {#3}_{#9} & {#3}_{#10} \\
% {#4}_{#6} & {#4}_{#7} & {#4}_{#8} & {#4}_{#9} & {#4}_{#10} \\
% {#5}_{#6} & {#5}_{#7} & {#5}_{#8} & {#5}_{#9} & {#5}_{#10}
%\end{vmatrix}
%}

% R3 vector.
\newcommand{\VectorThree}[3]{
\begin{bmatrix}
 {#1} \\
 {#2} \\
 {#3}
\end{bmatrix}
}



\author{Peeter Joot}
\email{peeter.joot@gmail.com}

%\documentclass[]{eliblogwidescreen}

\usepackage{amsmath}
\usepackage{mathpazo}

%
% shorthand for bold symbols, convenient for vectors and matrices
%
\newcommand{\Ba}[0]{\mathbf{a}}
\newcommand{\Bb}[0]{\mathbf{b}}
\newcommand{\Bc}[0]{\mathbf{c}}
\newcommand{\Bd}[0]{\mathbf{d}}
\newcommand{\Be}[0]{\mathbf{e}}
\newcommand{\Bf}[0]{\mathbf{f}}
\newcommand{\Bg}[0]{\mathbf{g}}
\newcommand{\Bh}[0]{\mathbf{h}}
\newcommand{\Bi}[0]{\mathbf{i}}
\newcommand{\Bj}[0]{\mathbf{j}}
\newcommand{\Bk}[0]{\mathbf{k}}
\newcommand{\Bl}[0]{\mathbf{l}}
\newcommand{\Bm}[0]{\mathbf{m}}
\newcommand{\Bn}[0]{\mathbf{n}}
\newcommand{\Bo}[0]{\mathbf{o}}
\newcommand{\Bp}[0]{\mathbf{p}}
\newcommand{\Bq}[0]{\mathbf{q}}
\newcommand{\Br}[0]{\mathbf{r}}
\newcommand{\Bs}[0]{\mathbf{s}}
\newcommand{\Bt}[0]{\mathbf{t}}
\newcommand{\Bu}[0]{\mathbf{u}}
\newcommand{\Bv}[0]{\mathbf{v}}
\newcommand{\Bw}[0]{\mathbf{w}}
\newcommand{\Bx}[0]{\mathbf{x}}
\newcommand{\By}[0]{\mathbf{y}}
\newcommand{\Bz}[0]{\mathbf{z}}
\newcommand{\BA}[0]{\mathbf{A}}
\newcommand{\BB}[0]{\mathbf{B}}
\newcommand{\BC}[0]{\mathbf{C}}
\newcommand{\BD}[0]{\mathbf{D}}
\newcommand{\BE}[0]{\mathbf{E}}
\newcommand{\BF}[0]{\mathbf{F}}
\newcommand{\BG}[0]{\mathbf{G}}
\newcommand{\BH}[0]{\mathbf{H}}
\newcommand{\BI}[0]{\mathbf{I}}
\newcommand{\BJ}[0]{\mathbf{J}}
\newcommand{\BK}[0]{\mathbf{K}}
\newcommand{\BL}[0]{\mathbf{L}}
\newcommand{\BM}[0]{\mathbf{M}}
\newcommand{\BN}[0]{\mathbf{N}}
\newcommand{\BO}[0]{\mathbf{O}}
\newcommand{\BP}[0]{\mathbf{P}}
\newcommand{\BQ}[0]{\mathbf{Q}}
\newcommand{\BR}[0]{\mathbf{R}}
\newcommand{\BS}[0]{\mathbf{S}}
\newcommand{\BT}[0]{\mathbf{T}}
\newcommand{\BU}[0]{\mathbf{U}}
\newcommand{\BV}[0]{\mathbf{V}}
\newcommand{\BW}[0]{\mathbf{W}}
\newcommand{\BX}[0]{\mathbf{X}}
\newcommand{\BY}[0]{\mathbf{Y}}
\newcommand{\BZ}[0]{\mathbf{Z}}

\newcommand{\Bzero}[0]{\mathbf{0}}
\newcommand{\Btheta}[0]{\boldsymbol{\theta}}
\newcommand{\Btau}[0]{\boldsymbol{\tau}}
\newcommand{\Bomega}[0]{\boldsymbol{\omega}}

%
% shorthand for unit vectors
%
\newcommand{\acap}[0]{\hat{\Ba}}
\newcommand{\bcap}[0]{\hat{\Bb}}
\newcommand{\ccap}[0]{\hat{\Bc}}
\newcommand{\dcap}[0]{\hat{\Bd}}
\newcommand{\ecap}[0]{\hat{\Be}}
\newcommand{\fcap}[0]{\hat{\Bf}}
\newcommand{\gcap}[0]{\hat{\Bg}}
\newcommand{\hcap}[0]{\hat{\Bh}}
\newcommand{\icap}[0]{\hat{\Bi}}
\newcommand{\jcap}[0]{\hat{\Bj}}
\newcommand{\kcap}[0]{\hat{\Bk}}
\newcommand{\lcap}[0]{\hat{\Bl}}
\newcommand{\mcap}[0]{\hat{\Bm}}
\newcommand{\ncap}[0]{\hat{\Bn}}
\newcommand{\ocap}[0]{\hat{\Bo}}
\newcommand{\pcap}[0]{\hat{\Bp}}
\newcommand{\qcap}[0]{\hat{\Bq}}
\newcommand{\rcap}[0]{\hat{\Br}}
\newcommand{\scap}[0]{\hat{\Bs}}
\newcommand{\tcap}[0]{\hat{\Bt}}
\newcommand{\ucap}[0]{\hat{\Bu}}
\newcommand{\vcap}[0]{\hat{\Bv}}
\newcommand{\wcap}[0]{\hat{\Bw}}
\newcommand{\xcap}[0]{\hat{\Bx}}
\newcommand{\ycap}[0]{\hat{\By}}
\newcommand{\zcap}[0]{\hat{\Bz}}
\newcommand{\thetacap}[0]{\hat{\Btheta}}

%
% to write R^n and C^n in a distinguishable fashion.  Perhaps change this
% to the double lined characters upon figuring out how to do so.
%
\newcommand{\C}[1]{$\mathbb{C}^{#1}$}
\newcommand{\R}[1]{$\mathbb{R}^{#1}$}

%
% various generally useful helpers
%

% derivative of #1 wrt. #2:
\newcommand{\D}[2] {\frac {d#2} {d#1}}

\newcommand{\inv}[1]{\frac{1}{#1}}
\newcommand{\cross}[0]{\times}

\newcommand{\abs}[1]{\lvert{#1}\rvert}
\newcommand{\norm}[1]{\lVert{#1}\rVert}
\newcommand{\innerprod}[2]{\langle{#1}, {#2}\rangle}
\newcommand{\dotprod}[2]{{#1} \cdot {#2}}
\newcommand{\bdotprod}[2]{\left({#1} \cdot {#2}\right)}
\newcommand{\crossprod}[2]{{#1} \cross {#2}}
\newcommand{\tripleprod}[3]{\dotprod{\left(\crossprod{#1}{#2}\right)}{#3}}

\DeclareMathOperator{\Proj}{Proj}
\DeclareMathOperator{\Span}{span}
\DeclareMathOperator{\Sgn}{sgn}
\DeclareMathOperator{\Area}{Area}
\DeclareMathOperator{\Volume}{Volume}

%
% A few miscellaneous things specific to this document
%
\newcommand{\crossop}[1]{\crossprod{#1}{}}

% R2 vector.
\newcommand{\VectorTwo}[2]{
\begin{bmatrix}
 {#1} \\
 {#2}
\end{bmatrix}
}

\newcommand{\VectorN}[1]{
\begin{bmatrix}
{#1}_1 \\
{#1}_2 \\
\vdots \\
{#1}_N \\
\end{bmatrix}
}

\newcommand{\DETuvij}[4]{
\begin{vmatrix}
 {#1}_{#3} & {#1}_{#4} \\
 {#2}_{#3} & {#2}_{#4}
\end{vmatrix}
}

\newcommand{\DETuvwijk}[6]{
\begin{vmatrix}
 {#1}_{#4} & {#1}_{#5} & {#1}_{#6} \\
 {#2}_{#4} & {#2}_{#5} & {#2}_{#6} \\
 {#3}_{#4} & {#3}_{#5} & {#3}_{#6}
\end{vmatrix}
}

\newcommand{\DETuvwxijkl}[8]{
\begin{vmatrix}
 {#1}_{#5} & {#1}_{#6} & {#1}_{#7} & {#1}_{#8} \\
 {#2}_{#5} & {#2}_{#6} & {#2}_{#7} & {#2}_{#8} \\
 {#3}_{#5} & {#3}_{#6} & {#3}_{#7} & {#3}_{#8} \\
 {#4}_{#5} & {#4}_{#6} & {#4}_{#7} & {#4}_{#8} \\
\end{vmatrix}
}

%\newcommand{\DETuvwxyijklm}[10]{
%\begin{vmatrix}
% {#1}_{#6} & {#1}_{#7} & {#1}_{#8} & {#1}_{#9} & {#1}_{#10} \\
% {#2}_{#6} & {#2}_{#7} & {#2}_{#8} & {#2}_{#9} & {#2}_{#10} \\
% {#3}_{#6} & {#3}_{#7} & {#3}_{#8} & {#3}_{#9} & {#3}_{#10} \\
% {#4}_{#6} & {#4}_{#7} & {#4}_{#8} & {#4}_{#9} & {#4}_{#10} \\
% {#5}_{#6} & {#5}_{#7} & {#5}_{#8} & {#5}_{#9} & {#5}_{#10}
%\end{vmatrix}
%}

% R3 vector.
\newcommand{\VectorThree}[3]{
\begin{bmatrix}
 {#1} \\
 {#2} \\
 {#3}
\end{bmatrix}
}



\author{Peeter Joot}
\email{peeter.joot@gmail.com}


\chapter{PHY450H1F: Quantum Mechanics II.  Lecture 3 (Taught by Prof J.E. Sipe).  Perturbation methods}
\label{chap:qmTwoL3}
%\useCCL
\blogpage{http://sites.google.com/site/peeterjoot/math2011/qmTwoL3.pdf}
\date{Sept 19, 2011}
\revisionInfo{qmTwoL3.tex}

\beginArtWithToc
%\beginArtNoToc

Peeter's lecture notes from class.  May not be entirely coherent.

\section{States and wave functions}

Suppose we have the following non-degenerate energy eigenstates

\begin{align*}
&\vdots \\
E_1 &\sim \ket{\Psi_1} \\
E_0 &\sim \ket{\Psi_0}
\end{align*}

and consider a state that is ``very close'' to $\ket{\Psi_n}$.

\begin{equation}\label{eqn:qmTwoL3:10}
\ket{\Psi} = \ket{\Psi_n} + \ket{\delta \Psi_n}
\end{equation}

We form projections onto $\ket{\Psi_n}$ ``direction''.  The difference from this projection will be written $\ket{\Psi_{n \perp}}$, as depicted in figure (\ref{fig:qmTwoL3fig1}).  This illustration cannot not be interpreted literally, but illustrates the idea nicely.

\begin{figure}[htp]
\centering
\includegraphics[totalheight=0.4\textheight]{qmTwoL3fig1}
\caption{Pictorial illustration of ket projections}\label{fig:qmTwoL3fig1}
\end{figure}

For the amount along the projection onto $\ket{\Psi_n}$ we write

\begin{equation}\label{eqn:qmTwoL3:30}
\braket{\Psi_n}{\delta \Psi_n} = \delta \alpha
\end{equation}

so that the total deviation from the original state is

\begin{equation}\label{eqn:qmTwoL3:50}
\ket{\delta \Psi_n} 
= \delta \alpha \ket{\Psi_n} 
+ \ket{\delta \Psi_{n \perp}} .
\end{equation}

The varied ket is then
\begin{equation}\label{eqn:qmTwoL3:70}
\ket{\Psi} 
= (1 + \delta \alpha )\ket{\Psi_n} + \ket{\delta \Psi_{n \perp}} 
\end{equation}

where

\begin{equation}\label{eqn:qmTwoL3:90}
(\delta \alpha)^2, \braket{\delta \Psi_{n \perp}}{\delta \Psi_{n \perp}}  \ll 1
\end{equation}

In terms of these projections our kets magnitude is

\begin{align*}
\braket{\Psi}{\Psi} 
&= 
\Bigl(
(1 + {\delta \alpha}^\conj )\bra{\Psi_n} + \bra{\delta \Psi_{n \perp}} 
\Bigr)
\Bigl(
(1 + \delta \alpha )\ket{\Psi_n} + \ket{\delta \Psi_{n \perp}} 
\Bigr) \\
&=
\Abs{1 + \delta \alpha}^2 \braket{\Psi_n}{\Psi_n}
+ 
\braket{\delta \Psi_{n \perp}}{\delta \Psi_{n \perp}}  \\
&\quad +
(1 + {\delta \alpha}^\conj )\braket{\Psi_n}{\delta \Psi_{n \perp}} 
+
(1 + \delta \alpha )\braket{\delta \Psi_{n \perp}}{\delta \Psi_n} 
\end{align*}

Because $\braket{\delta \Psi_{n \perp}}{\delta \Psi_n} = 0$ this is

\begin{equation}\label{eqn:qmTwoL3:110}
\braket{\Psi}{\Psi}
= 
\Abs{1 + \delta \alpha }^2
\braket{\delta \Psi_{n \perp}}{\delta \Psi_{n \perp}}.
\end{equation}

Similarly for the energy expectation we have

\begin{align*}
\braket{\Psi}{\Psi} 
&= 
\Bigl(
(1 + {\delta \alpha}^\conj )\bra{\Psi_n} + \bra{\delta \Psi_{n \perp}} 
\Bigr)
H
\Bigl(
(1 + \delta \alpha )\ket{\Psi_n} + \ket{\delta \Psi_{n \perp}} 
\Bigr) \\
&=
\Abs{1 + \delta \alpha}^2 E_n \braket{\Psi_n}{\Psi_n}
+ 
\braket{\delta \Psi_{n \perp}} H {\delta \Psi_{n \perp}}  \\
&\quad + 
(1 + {\delta \alpha}^\conj ) E_n \braket{\Psi_n}{\delta \Psi_{n \perp}} 
+
(1 + \delta \alpha ) E_n \braket{\delta \Psi_{n \perp}}{\delta \Psi_n} 
\end{align*}

Or
\begin{equation}\label{eqn:qmTwoL3:130}
\bra{\Psi} H \ket{\Psi}
= 
E_n \Abs{1 + \delta \alpha }^2
+
\bra{\delta \Psi_{n \perp}} H \ket{\delta \Psi_{n \perp}}.
\end{equation}

This gives

\begin{align*}
E[\Psi] 
&= 
\frac{
\bra{\Psi} H \ket{\Psi}
}
{
\braket{\Psi}{\Psi}
} \\
&=
\frac{
E_n \Abs{1 + \delta \alpha }^2 + 
\bra{\delta \Psi_{n \perp}} H \ket{\delta \Psi_{n \perp}}
}
{
\Abs{1 + \delta \alpha }^2
\braket{\delta \Psi_{n \perp}}{\delta \Psi_{n \perp}} 
} \\
&=
\frac{
E_n 
+ 
\frac{\bra{\delta \Psi_{n \perp}} H \ket{\delta \Psi_{n \perp}} }
{\Abs{1 + \delta \alpha }^2}
}
{
1
+\frac{\braket{\delta \Psi_{n \perp}}{\delta \Psi_{n \perp}} }
{\Abs{1 + \delta \alpha }^2}
} \\
&=
E_n \left( 1 - 
\frac{\braket{\delta \Psi_{n \perp}}{\delta \Psi_{n \perp}} }
{\Abs{1 + \delta \alpha }^2}
+ \cdots \right) + \cdots \\
&=
E_n\left[1 + \mathcal{O}\left((\delta \Psi_{n \perp})^2\right)\right]
\end{align*}

where
\begin{equation}\label{eqn:qmTwoL3:150}
(\delta \Psi_{n \perp})^2
\sim
\braket{\delta \Psi_{n \perp}}{\delta \Psi_{n \perp}}
\end{equation}

\begin{figure}[htp]
\centering
\includegraphics[totalheight=0.4\textheight]{qmTwoL3fig2}
\caption{Illustration of variation of energy with variation of Hamiltonian}\label{fig:qmTwoL3fig2}
\end{figure}
%figure (\ref{fig:qmTwoL3fig2})

``small errors'' in $\ket{\Psi}$ don't lead to large errors in $E[\Psi]$

It is reasonably easy to get a good estimate and $E_0$, although it is reasonably hard to get a good estimate of $\ket{\Psi_0}$.  This is for the same reason, because $E[]$ is not terribly sensitive.

\section{Excited states.}

\begin{align*}
&\vdots \\
E_2 &\sim \ket{\Psi_2} \\
E_1 &\sim \ket{\Psi_1} \\
E_0 &\sim \ket{\Psi_0}
\end{align*}

Suppose we wanted an estimate of $E_1$.  If we knew the ground state $\ket{\Psi_0}$.  For any trial $\ket{\Psi}$ form

\begin{equation}\label{eqn:qmTwoL3:170}
\ket{\Psi'} = 
\ket{\Psi} - 
\ket{\Psi_0}  \braket{\Psi_0}{\Psi}
\end{equation}

We are taking out the projection of the ground state from an arbitrary trial function.

For a state written in terms of the basis states, allowing for an $\alpha$ degeneracy

\begin{equation}\label{eqn:qmTwoL3:190}
\ket{\Psi} = 
c_0 \ket{\Psi_0}  
+
\sum_{n> 0, \alpha} c_{n \alpha} \ket{\Psi_{n \alpha}}
\end{equation}

\begin{equation}\label{eqn:qmTwoL3:210}
\braket{\Psi_0}{\Psi} = 
c_0 
\end{equation}

and

\begin{equation}\label{eqn:qmTwoL3:230}
\ket{\Psi'} = 
\sum_{n> 0, \alpha} c_{n \alpha} \ket{\Psi_{n \alpha}}
\end{equation}

(note that there are some theorems that tell us that the ground state is generally non-degenerate).

\begin{align*}
E[\Psi'] 
&= 
\frac{
\bra{\Psi'} H \ket{\Psi'}
}
{
\braket{\Psi'}{\Psi'}
}  \\
&=
\frac{
\sum_{n> 0, \alpha} \Abs{c_{n \alpha}}^2 E_n
}
{
\sum_{m> 0, \beta} \Abs{c_{m \beta}}^2 
}
\ge E_1
\end{align*}

Often don't know the exact ground state, although we might have a guess $\ket{\tilde{\Psi}_0}$.

for

\begin{equation}\label{eqn:qmTwoL3:250}
\ket{\Psi''} = \ket{\Psi} - 
\ket{\tilde{\Psi}_0}
\braket{\tilde{\Psi}_0}{\Psi}
\end{equation}

but cannot prove that
\begin{equation}\label{eqn:qmTwoL3:270}
\frac{
\bra{\Psi''} H \ket{\Psi''}
}
{
\braket{\Psi''}{\Psi''}
} 
\ge E_1
\end{equation}

%But sometimes, even if you don't know the ground state $\ket{\Psi_0}$, can choose trial kets $\ket{\Psi'''}$ such that $\braket{\Psi_0}{\Psi'''} = 0$.

Then

FIXME: missed something here.

\begin{equation}\label{eqn:qmTwoL3:290}
\frac{
\bra{\Psi'''} H \ket{\Psi'''}
}
{
\braket{\Psi'''}{\Psi'''}
} 
\ge E_1
\end{equation}

Somewhat remarkably, this is often possible.  We talked last time about the Hydrogen atom.  In that case, you can guess that the excited state is in the $2s$ orbital and and therefore orthogonal to the $1s$ (?) orbital.  

\section{Time independent perturbation theory.}

See \S 16.1 of the text \cite{desai2009quantum}.

We can sometimes use this sort of physical insight to help construct a good approximation.  This is provided that we have some of this physical insight, or that it is good insight in the first place.

This is the no-think (turn the crank) approach.

Here we split our Hamiltonian into two parts

\begin{equation}\label{eqn:qmTwoL3:310}
H = H_0 + H'
\end{equation}

where $H_0$ is a Hamiltonian for which we know the energy eigenstates and the eigenkets.  The $H'$ is the ``perturbation'' that is supposed to be small ``in some sense''.

Prof Sipe will provide some references later that provide a more specific meaning to this ``smallness''.  From some ad-hoc discussion in the class it sounds like one has to consider sequences of operators, and look at the convergence of those sequences (is this L2 measure theory?)

\begin{figure}[htp]
\centering
\includegraphics[totalheight=0.4\textheight]{qmTwoL3fig3}
\caption{Example of small perturbation from known Hamiltonian}\label{fig:qmTwoL3fig3}
\end{figure}
%figure (\ref{fig:qmTwoL3fig3})

We'd like to consider a range of problems of the form

\begin{equation}\label{eqn:qmTwoL3:330}
H = H_0 + \lambda H'
\end{equation}

where

\begin{equation}\label{eqn:qmTwoL3:350}
\lambda \in [0,1]
\end{equation}

So that when $\lambda \rightarrow 0$ we have

\begin{equation}\label{eqn:qmTwoL3:370}
H \rightarrow H_0
\end{equation}

the problem that we already know, but for $\lambda \rightarrow 1$ we have

\begin{equation}\label{eqn:qmTwoL3:390}
H = H_0 + H'
\end{equation}

the problem that we'd like to solve.

We are assuming that we know the eigenstates and eigenvalues for $H_0$.  Assuming no degeneracy

\begin{equation}\label{eqn:qmTwoL3:410}
H_0 \ket{\Psi_s^{(0)}} = 
E_s^{(0)}
\ket{\Psi_s^{(0)}} 
\end{equation}

We seek
\begin{equation}\label{eqn:qmTwoL3:430}
(H_0 + H')\ket{\Psi_s} = 
E_s
\ket{\Psi_s} 
\end{equation}

(this is the $\lambda = 1$ case).

Once (if) found, when $\lambda \rightarrow 0$ we will have

\begin{align*}
E_s &\rightarrow E_s^{(0)} \\
\ket{\Psi_s} &\rightarrow \ket{\Psi_s^{(0)}}
\end{align*}

\begin{equation}\label{eqn:qmTwoL3:450}
E_s = E_s^{(0)}  + \lambda E_s^{(1)} + \frac{\lambda^2}{2} E_s^{(2)}
\end{equation}

\begin{equation}\label{eqn:qmTwoL3:470}
\Psi_s = \sum_n c_{ns} \ket{\Psi_n^{(0)}}
\end{equation}

This we know we can do because we are assumed to have a complete set of states.

with
\begin{equation}\label{eqn:qmTwoL3:490}
c_{ns} = c_{ns}^{(0)}  + \lambda c_{ns}^{(1)} + \frac{\lambda^2}{2} c_{ns}^{(2)}
\end{equation}

where

\begin{equation}\label{eqn:qmTwoL3:510}
c_{ns}^{(0)} = \delta_{ns}
\end{equation}

There's a subtlety here that will be treated differently from the text.  We write

\begin{align*}
\ket{\Psi_s}
&=
\ket{\Psi_s^{(0)}}
+ 
\lambda 
\sum_n
c_{ns}^{(1)} 
\ket{\Psi_n^{(0)}}
+ 
\frac{\lambda^2}{2} 
\sum_n
c_{ns}^{(2)}
\ket{\Psi_n^{(0)}}
+ \cdots \\
&=
\left(
1 + \lambda c_{ss}^{(1)} + \cdots
\right)
\ket{\Psi_s^{(0)}}
+ \lambda 
\sum_{n \ne s} c_{ns}^{(1)} 
\ket{\Psi_n^{(0)}}
+ \cdots
\end{align*}

Take
\begin{align*}
\ket{\bar{\Psi}_s}
&=
\ket{\bar{\Psi}_s^{(0)}}
+ 
\lambda
\frac{
\sum_{n \ne s} c_{ns}^{(1)} 
\ket{\Psi_n^{(0)}}
}
{
1 + \lambda c_{ss}^{(1)}
} 
+ \cdots
\\
&=
\ket{\bar{\Psi}_s^{(0)}}
+ 
\lambda
\sum_{n \ne s} \bar{c}_{ns}^{(1)} 
\ket{\Psi_n^{(0)}} + \cdots
\end{align*}

where 

\begin{equation}\label{eqn:qmTwoL3:n}
\bar{c}_{ns}^{(1)}  =
\frac{c_{ns}^{(1)} }
{
1 + \lambda c_{ss}^{(1)}
} 
\end{equation}

We have: 

\begin{align*}
\bar{c}_{ns}^{(1)} &= c_{ns}^{(1)} \\
\bar{c}_{ns}^{(2)} &\ne c_{ns}^{(2)} 
\end{align*}

FIXME: I missed something here.

Note that this is no longer normalized.

\begin{equation}\label{eqn:qmTwoL3:530}
\braket{\bar{\Psi}_s}{\bar{\Psi}_s} \ne 1
\end{equation}

\EndArticle

%
% Copyright � 2015 Peeter Joot.  All Rights Reserved.
% Licenced as described in the file LICENSE under the root directory of this GIT repository.
%
\documentclass[]{eliblog}

\usepackage{amsmath}
\usepackage{mathpazo}

%
% shorthand for bold symbols, convenient for vectors and matrices
%
\newcommand{\Ba}[0]{\mathbf{a}}
\newcommand{\Bb}[0]{\mathbf{b}}
\newcommand{\Bc}[0]{\mathbf{c}}
\newcommand{\Bd}[0]{\mathbf{d}}
\newcommand{\Be}[0]{\mathbf{e}}
\newcommand{\Bf}[0]{\mathbf{f}}
\newcommand{\Bg}[0]{\mathbf{g}}
\newcommand{\Bh}[0]{\mathbf{h}}
\newcommand{\Bi}[0]{\mathbf{i}}
\newcommand{\Bj}[0]{\mathbf{j}}
\newcommand{\Bk}[0]{\mathbf{k}}
\newcommand{\Bl}[0]{\mathbf{l}}
\newcommand{\Bm}[0]{\mathbf{m}}
\newcommand{\Bn}[0]{\mathbf{n}}
\newcommand{\Bo}[0]{\mathbf{o}}
\newcommand{\Bp}[0]{\mathbf{p}}
\newcommand{\Bq}[0]{\mathbf{q}}
\newcommand{\Br}[0]{\mathbf{r}}
\newcommand{\Bs}[0]{\mathbf{s}}
\newcommand{\Bt}[0]{\mathbf{t}}
\newcommand{\Bu}[0]{\mathbf{u}}
\newcommand{\Bv}[0]{\mathbf{v}}
\newcommand{\Bw}[0]{\mathbf{w}}
\newcommand{\Bx}[0]{\mathbf{x}}
\newcommand{\By}[0]{\mathbf{y}}
\newcommand{\Bz}[0]{\mathbf{z}}
\newcommand{\BA}[0]{\mathbf{A}}
\newcommand{\BB}[0]{\mathbf{B}}
\newcommand{\BC}[0]{\mathbf{C}}
\newcommand{\BD}[0]{\mathbf{D}}
\newcommand{\BE}[0]{\mathbf{E}}
\newcommand{\BF}[0]{\mathbf{F}}
\newcommand{\BG}[0]{\mathbf{G}}
\newcommand{\BH}[0]{\mathbf{H}}
\newcommand{\BI}[0]{\mathbf{I}}
\newcommand{\BJ}[0]{\mathbf{J}}
\newcommand{\BK}[0]{\mathbf{K}}
\newcommand{\BL}[0]{\mathbf{L}}
\newcommand{\BM}[0]{\mathbf{M}}
\newcommand{\BN}[0]{\mathbf{N}}
\newcommand{\BO}[0]{\mathbf{O}}
\newcommand{\BP}[0]{\mathbf{P}}
\newcommand{\BQ}[0]{\mathbf{Q}}
\newcommand{\BR}[0]{\mathbf{R}}
\newcommand{\BS}[0]{\mathbf{S}}
\newcommand{\BT}[0]{\mathbf{T}}
\newcommand{\BU}[0]{\mathbf{U}}
\newcommand{\BV}[0]{\mathbf{V}}
\newcommand{\BW}[0]{\mathbf{W}}
\newcommand{\BX}[0]{\mathbf{X}}
\newcommand{\BY}[0]{\mathbf{Y}}
\newcommand{\BZ}[0]{\mathbf{Z}}

\newcommand{\Bzero}[0]{\mathbf{0}}
\newcommand{\Btheta}[0]{\boldsymbol{\theta}}
\newcommand{\Btau}[0]{\boldsymbol{\tau}}
\newcommand{\Bomega}[0]{\boldsymbol{\omega}}

%
% shorthand for unit vectors
%
\newcommand{\acap}[0]{\hat{\Ba}}
\newcommand{\bcap}[0]{\hat{\Bb}}
\newcommand{\ccap}[0]{\hat{\Bc}}
\newcommand{\dcap}[0]{\hat{\Bd}}
\newcommand{\ecap}[0]{\hat{\Be}}
\newcommand{\fcap}[0]{\hat{\Bf}}
\newcommand{\gcap}[0]{\hat{\Bg}}
\newcommand{\hcap}[0]{\hat{\Bh}}
\newcommand{\icap}[0]{\hat{\Bi}}
\newcommand{\jcap}[0]{\hat{\Bj}}
\newcommand{\kcap}[0]{\hat{\Bk}}
\newcommand{\lcap}[0]{\hat{\Bl}}
\newcommand{\mcap}[0]{\hat{\Bm}}
\newcommand{\ncap}[0]{\hat{\Bn}}
\newcommand{\ocap}[0]{\hat{\Bo}}
\newcommand{\pcap}[0]{\hat{\Bp}}
\newcommand{\qcap}[0]{\hat{\Bq}}
\newcommand{\rcap}[0]{\hat{\Br}}
\newcommand{\scap}[0]{\hat{\Bs}}
\newcommand{\tcap}[0]{\hat{\Bt}}
\newcommand{\ucap}[0]{\hat{\Bu}}
\newcommand{\vcap}[0]{\hat{\Bv}}
\newcommand{\wcap}[0]{\hat{\Bw}}
\newcommand{\xcap}[0]{\hat{\Bx}}
\newcommand{\ycap}[0]{\hat{\By}}
\newcommand{\zcap}[0]{\hat{\Bz}}
\newcommand{\thetacap}[0]{\hat{\Btheta}}

%
% to write R^n and C^n in a distinguishable fashion.  Perhaps change this
% to the double lined characters upon figuring out how to do so.
%
\newcommand{\C}[1]{$\mathbb{C}^{#1}$}
\newcommand{\R}[1]{$\mathbb{R}^{#1}$}

%
% various generally useful helpers
%

% derivative of #1 wrt. #2:
\newcommand{\D}[2] {\frac {d#2} {d#1}}

\newcommand{\inv}[1]{\frac{1}{#1}}
\newcommand{\cross}[0]{\times}

\newcommand{\abs}[1]{\lvert{#1}\rvert}
\newcommand{\norm}[1]{\lVert{#1}\rVert}
\newcommand{\innerprod}[2]{\langle{#1}, {#2}\rangle}
\newcommand{\dotprod}[2]{{#1} \cdot {#2}}
\newcommand{\bdotprod}[2]{\left({#1} \cdot {#2}\right)}
\newcommand{\crossprod}[2]{{#1} \cross {#2}}
\newcommand{\tripleprod}[3]{\dotprod{\left(\crossprod{#1}{#2}\right)}{#3}}

\DeclareMathOperator{\Proj}{Proj}
\DeclareMathOperator{\Span}{span}
\DeclareMathOperator{\Sgn}{sgn}
\DeclareMathOperator{\Area}{Area}
\DeclareMathOperator{\Volume}{Volume}

%
% A few miscellaneous things specific to this document
%
\newcommand{\crossop}[1]{\crossprod{#1}{}}

% R2 vector.
\newcommand{\VectorTwo}[2]{
\begin{bmatrix}
 {#1} \\
 {#2}
\end{bmatrix}
}

\newcommand{\VectorN}[1]{
\begin{bmatrix}
{#1}_1 \\
{#1}_2 \\
\vdots \\
{#1}_N \\
\end{bmatrix}
}

\newcommand{\DETuvij}[4]{
\begin{vmatrix}
 {#1}_{#3} & {#1}_{#4} \\
 {#2}_{#3} & {#2}_{#4}
\end{vmatrix}
}

\newcommand{\DETuvwijk}[6]{
\begin{vmatrix}
 {#1}_{#4} & {#1}_{#5} & {#1}_{#6} \\
 {#2}_{#4} & {#2}_{#5} & {#2}_{#6} \\
 {#3}_{#4} & {#3}_{#5} & {#3}_{#6}
\end{vmatrix}
}

\newcommand{\DETuvwxijkl}[8]{
\begin{vmatrix}
 {#1}_{#5} & {#1}_{#6} & {#1}_{#7} & {#1}_{#8} \\
 {#2}_{#5} & {#2}_{#6} & {#2}_{#7} & {#2}_{#8} \\
 {#3}_{#5} & {#3}_{#6} & {#3}_{#7} & {#3}_{#8} \\
 {#4}_{#5} & {#4}_{#6} & {#4}_{#7} & {#4}_{#8} \\
\end{vmatrix}
}

%\newcommand{\DETuvwxyijklm}[10]{
%\begin{vmatrix}
% {#1}_{#6} & {#1}_{#7} & {#1}_{#8} & {#1}_{#9} & {#1}_{#10} \\
% {#2}_{#6} & {#2}_{#7} & {#2}_{#8} & {#2}_{#9} & {#2}_{#10} \\
% {#3}_{#6} & {#3}_{#7} & {#3}_{#8} & {#3}_{#9} & {#3}_{#10} \\
% {#4}_{#6} & {#4}_{#7} & {#4}_{#8} & {#4}_{#9} & {#4}_{#10} \\
% {#5}_{#6} & {#5}_{#7} & {#5}_{#8} & {#5}_{#9} & {#5}_{#10}
%\end{vmatrix}
%}

% R3 vector.
\newcommand{\VectorThree}[3]{
\begin{bmatrix}
 {#1} \\
 {#2} \\
 {#3}
\end{bmatrix}
}



\author{Peeter Joot}
\email{peeter.joot@gmail.com}

%\documentclass[]{eliblogwidescreen}

\usepackage{amsmath}
\usepackage{mathpazo}

%
% shorthand for bold symbols, convenient for vectors and matrices
%
\newcommand{\Ba}[0]{\mathbf{a}}
\newcommand{\Bb}[0]{\mathbf{b}}
\newcommand{\Bc}[0]{\mathbf{c}}
\newcommand{\Bd}[0]{\mathbf{d}}
\newcommand{\Be}[0]{\mathbf{e}}
\newcommand{\Bf}[0]{\mathbf{f}}
\newcommand{\Bg}[0]{\mathbf{g}}
\newcommand{\Bh}[0]{\mathbf{h}}
\newcommand{\Bi}[0]{\mathbf{i}}
\newcommand{\Bj}[0]{\mathbf{j}}
\newcommand{\Bk}[0]{\mathbf{k}}
\newcommand{\Bl}[0]{\mathbf{l}}
\newcommand{\Bm}[0]{\mathbf{m}}
\newcommand{\Bn}[0]{\mathbf{n}}
\newcommand{\Bo}[0]{\mathbf{o}}
\newcommand{\Bp}[0]{\mathbf{p}}
\newcommand{\Bq}[0]{\mathbf{q}}
\newcommand{\Br}[0]{\mathbf{r}}
\newcommand{\Bs}[0]{\mathbf{s}}
\newcommand{\Bt}[0]{\mathbf{t}}
\newcommand{\Bu}[0]{\mathbf{u}}
\newcommand{\Bv}[0]{\mathbf{v}}
\newcommand{\Bw}[0]{\mathbf{w}}
\newcommand{\Bx}[0]{\mathbf{x}}
\newcommand{\By}[0]{\mathbf{y}}
\newcommand{\Bz}[0]{\mathbf{z}}
\newcommand{\BA}[0]{\mathbf{A}}
\newcommand{\BB}[0]{\mathbf{B}}
\newcommand{\BC}[0]{\mathbf{C}}
\newcommand{\BD}[0]{\mathbf{D}}
\newcommand{\BE}[0]{\mathbf{E}}
\newcommand{\BF}[0]{\mathbf{F}}
\newcommand{\BG}[0]{\mathbf{G}}
\newcommand{\BH}[0]{\mathbf{H}}
\newcommand{\BI}[0]{\mathbf{I}}
\newcommand{\BJ}[0]{\mathbf{J}}
\newcommand{\BK}[0]{\mathbf{K}}
\newcommand{\BL}[0]{\mathbf{L}}
\newcommand{\BM}[0]{\mathbf{M}}
\newcommand{\BN}[0]{\mathbf{N}}
\newcommand{\BO}[0]{\mathbf{O}}
\newcommand{\BP}[0]{\mathbf{P}}
\newcommand{\BQ}[0]{\mathbf{Q}}
\newcommand{\BR}[0]{\mathbf{R}}
\newcommand{\BS}[0]{\mathbf{S}}
\newcommand{\BT}[0]{\mathbf{T}}
\newcommand{\BU}[0]{\mathbf{U}}
\newcommand{\BV}[0]{\mathbf{V}}
\newcommand{\BW}[0]{\mathbf{W}}
\newcommand{\BX}[0]{\mathbf{X}}
\newcommand{\BY}[0]{\mathbf{Y}}
\newcommand{\BZ}[0]{\mathbf{Z}}

\newcommand{\Bzero}[0]{\mathbf{0}}
\newcommand{\Btheta}[0]{\boldsymbol{\theta}}
\newcommand{\Btau}[0]{\boldsymbol{\tau}}
\newcommand{\Bomega}[0]{\boldsymbol{\omega}}

%
% shorthand for unit vectors
%
\newcommand{\acap}[0]{\hat{\Ba}}
\newcommand{\bcap}[0]{\hat{\Bb}}
\newcommand{\ccap}[0]{\hat{\Bc}}
\newcommand{\dcap}[0]{\hat{\Bd}}
\newcommand{\ecap}[0]{\hat{\Be}}
\newcommand{\fcap}[0]{\hat{\Bf}}
\newcommand{\gcap}[0]{\hat{\Bg}}
\newcommand{\hcap}[0]{\hat{\Bh}}
\newcommand{\icap}[0]{\hat{\Bi}}
\newcommand{\jcap}[0]{\hat{\Bj}}
\newcommand{\kcap}[0]{\hat{\Bk}}
\newcommand{\lcap}[0]{\hat{\Bl}}
\newcommand{\mcap}[0]{\hat{\Bm}}
\newcommand{\ncap}[0]{\hat{\Bn}}
\newcommand{\ocap}[0]{\hat{\Bo}}
\newcommand{\pcap}[0]{\hat{\Bp}}
\newcommand{\qcap}[0]{\hat{\Bq}}
\newcommand{\rcap}[0]{\hat{\Br}}
\newcommand{\scap}[0]{\hat{\Bs}}
\newcommand{\tcap}[0]{\hat{\Bt}}
\newcommand{\ucap}[0]{\hat{\Bu}}
\newcommand{\vcap}[0]{\hat{\Bv}}
\newcommand{\wcap}[0]{\hat{\Bw}}
\newcommand{\xcap}[0]{\hat{\Bx}}
\newcommand{\ycap}[0]{\hat{\By}}
\newcommand{\zcap}[0]{\hat{\Bz}}
\newcommand{\thetacap}[0]{\hat{\Btheta}}

%
% to write R^n and C^n in a distinguishable fashion.  Perhaps change this
% to the double lined characters upon figuring out how to do so.
%
\newcommand{\C}[1]{$\mathbb{C}^{#1}$}
\newcommand{\R}[1]{$\mathbb{R}^{#1}$}

%
% various generally useful helpers
%

% derivative of #1 wrt. #2:
\newcommand{\D}[2] {\frac {d#2} {d#1}}

\newcommand{\inv}[1]{\frac{1}{#1}}
\newcommand{\cross}[0]{\times}

\newcommand{\abs}[1]{\lvert{#1}\rvert}
\newcommand{\norm}[1]{\lVert{#1}\rVert}
\newcommand{\innerprod}[2]{\langle{#1}, {#2}\rangle}
\newcommand{\dotprod}[2]{{#1} \cdot {#2}}
\newcommand{\bdotprod}[2]{\left({#1} \cdot {#2}\right)}
\newcommand{\crossprod}[2]{{#1} \cross {#2}}
\newcommand{\tripleprod}[3]{\dotprod{\left(\crossprod{#1}{#2}\right)}{#3}}

\DeclareMathOperator{\Proj}{Proj}
\DeclareMathOperator{\Span}{span}
\DeclareMathOperator{\Sgn}{sgn}
\DeclareMathOperator{\Area}{Area}
\DeclareMathOperator{\Volume}{Volume}

%
% A few miscellaneous things specific to this document
%
\newcommand{\crossop}[1]{\crossprod{#1}{}}

% R2 vector.
\newcommand{\VectorTwo}[2]{
\begin{bmatrix}
 {#1} \\
 {#2}
\end{bmatrix}
}

\newcommand{\VectorN}[1]{
\begin{bmatrix}
{#1}_1 \\
{#1}_2 \\
\vdots \\
{#1}_N \\
\end{bmatrix}
}

\newcommand{\DETuvij}[4]{
\begin{vmatrix}
 {#1}_{#3} & {#1}_{#4} \\
 {#2}_{#3} & {#2}_{#4}
\end{vmatrix}
}

\newcommand{\DETuvwijk}[6]{
\begin{vmatrix}
 {#1}_{#4} & {#1}_{#5} & {#1}_{#6} \\
 {#2}_{#4} & {#2}_{#5} & {#2}_{#6} \\
 {#3}_{#4} & {#3}_{#5} & {#3}_{#6}
\end{vmatrix}
}

\newcommand{\DETuvwxijkl}[8]{
\begin{vmatrix}
 {#1}_{#5} & {#1}_{#6} & {#1}_{#7} & {#1}_{#8} \\
 {#2}_{#5} & {#2}_{#6} & {#2}_{#7} & {#2}_{#8} \\
 {#3}_{#5} & {#3}_{#6} & {#3}_{#7} & {#3}_{#8} \\
 {#4}_{#5} & {#4}_{#6} & {#4}_{#7} & {#4}_{#8} \\
\end{vmatrix}
}

%\newcommand{\DETuvwxyijklm}[10]{
%\begin{vmatrix}
% {#1}_{#6} & {#1}_{#7} & {#1}_{#8} & {#1}_{#9} & {#1}_{#10} \\
% {#2}_{#6} & {#2}_{#7} & {#2}_{#8} & {#2}_{#9} & {#2}_{#10} \\
% {#3}_{#6} & {#3}_{#7} & {#3}_{#8} & {#3}_{#9} & {#3}_{#10} \\
% {#4}_{#6} & {#4}_{#7} & {#4}_{#8} & {#4}_{#9} & {#4}_{#10} \\
% {#5}_{#6} & {#5}_{#7} & {#5}_{#8} & {#5}_{#9} & {#5}_{#10}
%\end{vmatrix}
%}

% R3 vector.
\newcommand{\VectorThree}[3]{
\begin{bmatrix}
 {#1} \\
 {#2} \\
 {#3}
\end{bmatrix}
}



\author{Peeter Joot}
\email{peeter.joot@gmail.com}


\chapter{PHY450H1F: Quantum Mechanics II.  Lecture 4 (Taught by Prof J.E. Sipe).  Time independent pertubation theory (continued)}
\label{chap:qmTwoL4}
%\useCCL
\blogpage{http://sites.google.com/site/peeterjoot/math2011/qmTwoL4.pdf}
\date{Sept 21, 2011}
\revisionInfo{qmTwoL4.tex}

% subst: \Esn => {E_s}^{(n)}
% subst: \kpsi_s^n => \ket{{\psi_s}^{(n)}}
% subst: \kpsi_s => \ket{\psi_s}
% subst: \cns\d => {c_{ns}}^{(d)}
% subst: \cns => c_{ns}
%:,$ s/\\E\(.\)\(.\)/E_\1^{(\2)}/cg
%:,$ s/\\kpsi_\(.\)^\(.\)/\\ket{\\psi_\1^{(\2)}}/cg
%:,$ s/\\kpsi_\(.\)/\\ket{\\psi_\1}/cg
%:,$ s/\\cns\([0-9]\)/{c_{ns}}^{(\1)}/cg
%:,$ s/\\cns\>/c_{ns}/cg
\beginArtWithToc
%\beginArtNoToc

\section{Disclaimer.}

Peeter's lecture notes from class.  May not be entirely coherent.

\section{Time independent pertubation.}
\subsection{The setup}

To recap, we were covering the time independent pertubation methods from \S 16.1 of the text \cite{desai2009quantum}.  We start with a known Hamiltonian $H_0$, and alter it with the addition of a ``small'' pertubation

\begin{equation}\label{eqn:qmTwoL4:10}
H = H_0 + \lambda H', \qquad \lambda \in [0,1]
\end{equation}

For the original operator, we assume that a complete set of eigenvectors and eigenkets is known

\begin{equation}\label{eqn:qmTwoL4:30}
H_0 \ket{{\psi_0}^{(0)}} = {E_s}^{(0)} \ket{{\psi_s}^{(0)}}
\end{equation}

We seek the perturbed eigensolution

\begin{equation}\label{eqn:qmTwoL4:50}
H \ket{\psi_s} = E_s \ket{\psi_s}
\end{equation}

and assumed a pertubative series representation for the energy eigenvalues in the new system

\begin{equation}\label{eqn:qmTwoL4:70}
E_s = {E_s}^{(0)} + \lambda {E_s}^{(1)} + \lambda^2 {E_s}^{(2)} + \cdots
\end{equation}

Given an assumed representation for the new eigenkets in terms of the known basis

\begin{equation}\label{eqn:qmTwoL4:90}
\ket{\psi_s} = \sum_n c_{ns} \ket{{\psi_n}^{(0)}} 
\end{equation}

and a pertubative series representation for the probability coeffecients

\begin{equation}\label{eqn:qmTwoL4:110}
c_{ns} = {c_{ns}}^{(0)} + \lambda {c_{ns}}^{(1)} + \lambda^2 {c_{ns}}^{(2)},
\end{equation}

so that 

\begin{equation}\label{eqn:qmTwoL4:130}
\ket{\psi_s} = 
\sum_n {c_{ns}}^{(0)} \ket{{\psi_n}^{(0)}} 
+
\lambda
\sum_n {c_{ns}}^{(1)} \ket{{\psi_n}^{(0)}} 
+ 
\lambda^2
\sum_n {c_{ns}}^{(2)} \ket{{\psi_n}^{(0)}} 
+ \cdots
\end{equation}

Setting $\lambda = 0$ requires 

\begin{equation}\label{eqn:qmTwoL4:150}
{c_{ns}}^{(0)} = \delta_{ns},
\end{equation}

for

\begin{equation}\label{eqn:qmTwoL4:170}
\begin{aligned}
\ket{\psi_s} 
&= 
\ket{{\psi_s}^{(0)}} 
+
\lambda
\sum_n {c_{ns}}^{(1)} \ket{{\psi_n}^{(0)}} 
+ 
\lambda^2
\sum_n {c_{ns}}^{(2)} \ket{{\psi_n}^{(0)}} 
+ \cdots \\
&=
\left(
1 
+ \lambda {c_{ns}}^{(1)} 
+ \lambda^2 {c_{ns}}^{(2)} 
+ \cdots
\right)
\ket{{\psi_s}^{(0)}} 
+ 
\lambda
\sum_{n \ne s} {c_{ns}}^{(1)} \ket{{\psi_n}^{(0)}} 
+
\lambda^2
\sum_{n \ne s} {c_{ns}}^{(2)} \ket{{\psi_n}^{(0)}} 
+ \cdots
\end{aligned}
\end{equation}

We rescaled our kets 

\begin{equation}\label{eqn:qmTwoL4:190}
\ket{\bar{\psi}_s} 
=
\ket{{\psi_s}^{(0)}} 
+ 
\lambda
\sum_{n \ne s} {\bar{c}_{ns}}^{(1)} \ket{{\psi_n}^{(0)}} 
+
\lambda^2
\sum_{n \ne s} {\bar{c}_{ns}}^{(2)} \ket{{\psi_n}^{(0)}} 
+ \cdots
\end{equation}

where
\begin{equation}\label{eqn:qmTwoL4:210}
{\bar{c}_{ns}}^{(j)} = 
\frac{{c_{ns}}^{(j)}}
{
1 
+ \lambda {c_{ns}}^{(1)} 
+ \lambda^2 {c_{ns}}^{(2)} 
+ \cdots
}
\end{equation}

The normalization of the rescaled kets is then

\begin{equation}\label{eqn:qmTwoL4:230}
\braket{\bar{\psi}_s}{\bar{\psi}_s} 
=
1
+ 
\lambda^2
\sum_{n \ne s} \Abs{{\bar{c}_{ns}}^{(1)}}^2
+
\cdots
\equiv \inv{Z_s},
\end{equation}

One can then construct a renormalized ket if desired

\begin{equation}\label{eqn:qmTwoL4:250}
\ket{\bar{\psi}_s}_R = Z_s^{1/2} \ket{\bar{\psi}_s},
\end{equation}

so that
\begin{equation}\label{eqn:qmTwoL4:270}
(\ket{\bar{\psi}_s}_R)^\dagger \ket{\bar{\psi}_s}_R = Z_s \braket{\bar{\psi}_s}{\bar{\psi}_s} = 1.
\end{equation}

\subsection{The meat.}

That's as far as we got last time.  We continue by renaming terms in \ref{eqn:qmTwoL4:190}

\begin{equation}\label{eqn:qmTwoL4:300}
\ket{\bar{\psi}_s} 
=
\ket{{\psi_s}^{(0)}} 
+ 
\lambda \ket{{\psi_s}^{(1)}} 
+ 
\lambda^2 \ket{{\psi_s}^{(2)}} 
+ \cdots
\end{equation}

where

\begin{equation}\label{eqn:qmTwoL4:320}
\ket{{\psi_n}^{(j)}} = \sum_{n \ne s} {\bar{c}_{ns}}^{(j)} \ket{{\psi_s}^{(0)}}.
\end{equation}

Now we act on this with the Hamiltonian

\begin{equation}\label{eqn:qmTwoL4:340}
H \ket{\bar{\psi}_s} = E_s \ket{\bar{\psi}_s},
\end{equation}

or

\begin{equation}\label{eqn:qmTwoL4:360}
H \ket{\bar{\psi}_s} - E_s \ket{\bar{\psi}_s} = 0.
\end{equation}

Expanding this, we have
\begin{equation}\label{eqn:qmTwoL4:380}
\begin{aligned}
&(H_0 + \lambda H') 
\left(
\ket{{\psi_s}^{(0)}} 
+ 
\lambda \ket{{\psi_s}^{(1)}} 
+ 
\lambda^2 \ket{{\psi_s}^{(2)}} 
+ \cdots
\right) \\
&\quad - 
\left( {E_s}^{(0)} + \lambda {E_s}^{(1)} + \lambda^2 {E_s}^{(2)} + \cdots \right)
\left(
\ket{{\psi_s}^{(0)}} 
+ 
\lambda \ket{{\psi_s}^{(1)}} 
+ 
\lambda^2 \ket{{\psi_s}^{(2)}} 
+ \cdots
\right)
= 0.
\end{aligned}
\end{equation}

We want to write this as

\begin{equation}\label{eqn:qmTwoL4:400}
\ket{A} + \lambda \ket{B} + \lambda^2 \ket{C} + \cdots = 0.
\end{equation}

This is

\begin{equation}\label{eqn:qmTwoL4:420}
\begin{aligned}
0 &=
\lambda^0
(H_0 - E_s^{(0)}) \ket{{\psi_s}^{(0)}}  \\
&+ \lambda
\left(
(H_0 - E_s^{(0)}) \ket{{\psi_s}^{(1)}} 
+(H' - E_s^{(1)}) \ket{{\psi_s}^{(0)}} 
\right) \\
&+ \lambda^2
\left(
(H_0 - E_s^{(0)}) \ket{{\psi_s}^{(2)}} 
+(H' - E_s^{(1)}) \ket{{\psi_s}^{(1)}} 
-E_s^{(2)} \ket{{\psi_s}^{(0)}} 
\right) \\
&\cdots
\end{aligned}
\end{equation}

So we form

\begin{align}\label{eqn:qmTwoL4:440}
\ket{A} &=
(H_0 - E_s^{(0)}) \ket{{\psi_s}^{(0)}} \\
\ket{B} &=
(H_0 - E_s^{(0)}) \ket{{\psi_s}^{(1)}} 
+(H' - E_s^{(1)}) \ket{{\psi_s}^{(0)}} \\
\ket{C} &=
(H_0 - E_s^{(0)}) \ket{{\psi_s}^{(2)}} 
+(H' - E_s^{(1)}) \ket{{\psi_s}^{(1)}} 
-E_s^{(2)} \ket{{\psi_s}^{(0)}}  \\
&\vdots
\end{align}

\paragraph{Zeroth order in $\lambda$}

Since $H_0 \ket{{\psi_s}^{(0)}} = E_s^{(0)} \ket{{\psi_s}^{(0)}}$, this first condition on $\ket{A}$ is not much more than a statement that $0 - 0 = 0$.  

\paragraph{First order in $\lambda$}

How about $\ket{B} = 0$?  For this to be zero we require that both of the following are simulataneously zero

\begin{align}\label{eqn:qmTwoL4:460}
\braket{{\psi_s}^{(0)}}{B} &= 0 \\
\braket{{\psi_m}^{(0)}}{B} &= 0, \qquad m \ne s
\end{align}

This first condition is
\begin{equation}\label{eqn:qmTwoL4:480}
\bra{{\psi_s}^{(0)}} (H' - E_s^{(1)}) \ket{{\psi_s}^{(0)}} = 0.
\end{equation}

With
\begin{equation}\label{eqn:qmTwoL4:500}
\bra{{\psi_m}^{(0)}} H' \ket{{\psi_s}^{(0)}} \equiv {H_{ms}}',
\end{equation}

or
\begin{equation}\label{eqn:qmTwoL4:520}
{H_{ss}}' = E_s^{(1)}.
\end{equation}

From the second condition we have
\begin{equation}\label{eqn:qmTwoL4:540}
0 = \bra{{\psi_m}^{(0)}} 
(H_0 - E_s^{(0)}) \ket{{\psi_s}^{(1)}} 
+\bra{{\psi_m}^{(0)}} 
(H' - E_s^{(1)}) \ket{{\psi_s}^{(0)}} 
\end{equation}

Utilizing the Hermitian nature of $H_0$ we can act backwards on $\bra{{\psi_m}^{(0)}}$ 

\begin{equation}\label{eqn:qmTwoL4:560}
\bra{{\psi_m}^{(0)}} H_0
=
E_m^{(0)} \bra{{\psi_m}^{(0)}}.
\end{equation}

We note that $\braket{{\psi_m}^{(0)}}{{\psi_s}^{(0)}} = 0, m \ne s$.  We can also expand the $\braket{{\psi_m}^{(0)}}{{\psi_s}^{(1)}}$, which is

\begin{align*}
\braket{{\psi_m}^{(0)}}{{\psi_s}^{(1)}} 
&=
\bra{{\psi_m}^{(0)}}
\left(
\sum_{n \ne s} {\bar{c}_{ns}}^{(1)} \ket{{\psi_n}^{(0)}}
\right) \\
\end{align*}

I found that reducing this sum wasn't obvious until some actual integers were plugged in.  Suppose that $s = 3$, and $m = 5$, then this is

\begin{align*}
\braket{{\psi_5}^{(0)}}{{\psi_3}^{(1)}} 
&=
\bra{{\psi_5}^{(0)}}
\left(
\sum_{n = 0, 1, 2, 4, 5, \cdots} {\bar{c}_{n3}}^{(1)} \ket{{\psi_n}^{(0)}}
\right) \\
&=
\bra{{\psi_5}^{(0)}}
{\bar{c}_{53}}^{(1)} \braket{{\psi_5}^{(0)}}{{\psi_5}^{(0)}} \\
&=
{\bar{c}_{53}}^{(1)}.
\end{align*}

More generally that is
\begin{equation}\label{eqn:qmTwoL4:580}
\braket{{\psi_m}^{(0)}}{{\psi_s}^{(1)}} 
=
{\bar{c}_{ms}}^{(1)}.
\end{equation}

Utilizing this gives us
\begin{equation}\label{eqn:qmTwoL4:600}
0 = 
( E_m^{(0)} - E_s^{(0)}) 
{\bar{c}_{ms}}^{(1)}
+
{H_{ms}}' 
\end{equation}

And summarizing what we learn from our $\ket{B} = 0$ conditions we have

\begin{align}\label{eqn:qmTwoL4:620}
E_s^{(1)} &= {H_{ss}}' \\
{\bar{c}_{ms}}^{(1)}
&=
\frac{{H_{ms}}' }
{ E_s^{(0)} - E_m^{(0)} }
\end{align}

\paragraph{Second order in $\lambda$}

Doing the same thing for $\ket{C} = 0$ we form (or assume)

\begin{align}\label{eqn:qmTwoL4:640}
\braket{{\psi_s}^{(0)}}{C} &= 0 \\
\braket{{\psi_m}^{(0)}}{C} &= 0, \qquad m \ne s
\end{align}

\begin{align*}
0 
&= \braket{{\psi_s}^{(0)}}{C}  \\
&=
\bra{{\psi_s}^{(0)}}
\left(
(H_0 - E_s^{(0)}) \ket{{\psi_s}^{(2)}} 
+(H' - E_s^{(1)}) \ket{{\psi_s}^{(1)}} 
-E_s^{(2)} \ket{{\psi_s}^{(0)}}  
\right) \\
&=
(E_s^{(0)} - E_s^{(0)}) 
\braket{{\psi_s}^{(0)}}{{\psi_s}^{(2)}} 
+
\bra{{\psi_s}^{(0)}}
(H' - E_s^{(1)}) \ket{{\psi_s}^{(1)}} 
-E_s^{(2)} \braket{{\psi_s}^{(0)}}{{\psi_s}^{(0)}} 
\end{align*}

We need to know what the $\braket{{\psi_s}^{(0)}}{{\psi_s}^{(1)}}$ is, and find that it is zero

\begin{equation}\label{eqn:qmTwoL4:660}
\braket{{\psi_s}^{(0)}}{{\psi_s}^{(1)}}
=
\bra{{\psi_s}^{(0)}}
\sum_{n \ne s} {\bar{c}_{ns}}^{(1)} \ket{{\psi_n}^{(0)}}
\end{equation}

Again, suppose that $s = 3$.  Our sum ranges over all $n \ne 3$, so all the brakets are zero.  Utilizing that we have

\begin{align*}
E_s^{(2)} 
&=
\bra{{\psi_s}^{(0)}} H' \ket{{\psi_s}^{(1)}}  \\
&=
\bra{{\psi_s}^{(0)}} H' \sum_{m \ne s} {\bar{c}_{ms}}^{(1)} \ket{{\psi_m}^{(0)}} \\
&=
\sum_{m \ne s} {\bar{c}_{ms}}^{(1)} {H_{sm}}'
\end{align*}

From \ref{eqn:qmTwoL4:620} we have

\begin{equation}\label{eqn:qmTwoL4:n}
E_s^{(2)} 
=
\sum_{m \ne s} 
\frac{{H_{ms}}' }
{ E_s^{(0)} - E_m^{(0)} }
{H_{sm}}'
=
\sum_{m \ne s} 
\frac{\Abs{{H_{ms}}'}^2 }
{ E_s^{(0)} - E_m^{(0)} }
\end{equation}

\EndArticle

%
% Copyright � 2015 Peeter Joot.  All Rights Reserved.
% Licenced as described in the file LICENSE under the root directory of this GIT repository.
%
\documentclass[]{eliblog}

\usepackage{amsmath}
\usepackage{mathpazo}

%
% shorthand for bold symbols, convenient for vectors and matrices
%
\newcommand{\Ba}[0]{\mathbf{a}}
\newcommand{\Bb}[0]{\mathbf{b}}
\newcommand{\Bc}[0]{\mathbf{c}}
\newcommand{\Bd}[0]{\mathbf{d}}
\newcommand{\Be}[0]{\mathbf{e}}
\newcommand{\Bf}[0]{\mathbf{f}}
\newcommand{\Bg}[0]{\mathbf{g}}
\newcommand{\Bh}[0]{\mathbf{h}}
\newcommand{\Bi}[0]{\mathbf{i}}
\newcommand{\Bj}[0]{\mathbf{j}}
\newcommand{\Bk}[0]{\mathbf{k}}
\newcommand{\Bl}[0]{\mathbf{l}}
\newcommand{\Bm}[0]{\mathbf{m}}
\newcommand{\Bn}[0]{\mathbf{n}}
\newcommand{\Bo}[0]{\mathbf{o}}
\newcommand{\Bp}[0]{\mathbf{p}}
\newcommand{\Bq}[0]{\mathbf{q}}
\newcommand{\Br}[0]{\mathbf{r}}
\newcommand{\Bs}[0]{\mathbf{s}}
\newcommand{\Bt}[0]{\mathbf{t}}
\newcommand{\Bu}[0]{\mathbf{u}}
\newcommand{\Bv}[0]{\mathbf{v}}
\newcommand{\Bw}[0]{\mathbf{w}}
\newcommand{\Bx}[0]{\mathbf{x}}
\newcommand{\By}[0]{\mathbf{y}}
\newcommand{\Bz}[0]{\mathbf{z}}
\newcommand{\BA}[0]{\mathbf{A}}
\newcommand{\BB}[0]{\mathbf{B}}
\newcommand{\BC}[0]{\mathbf{C}}
\newcommand{\BD}[0]{\mathbf{D}}
\newcommand{\BE}[0]{\mathbf{E}}
\newcommand{\BF}[0]{\mathbf{F}}
\newcommand{\BG}[0]{\mathbf{G}}
\newcommand{\BH}[0]{\mathbf{H}}
\newcommand{\BI}[0]{\mathbf{I}}
\newcommand{\BJ}[0]{\mathbf{J}}
\newcommand{\BK}[0]{\mathbf{K}}
\newcommand{\BL}[0]{\mathbf{L}}
\newcommand{\BM}[0]{\mathbf{M}}
\newcommand{\BN}[0]{\mathbf{N}}
\newcommand{\BO}[0]{\mathbf{O}}
\newcommand{\BP}[0]{\mathbf{P}}
\newcommand{\BQ}[0]{\mathbf{Q}}
\newcommand{\BR}[0]{\mathbf{R}}
\newcommand{\BS}[0]{\mathbf{S}}
\newcommand{\BT}[0]{\mathbf{T}}
\newcommand{\BU}[0]{\mathbf{U}}
\newcommand{\BV}[0]{\mathbf{V}}
\newcommand{\BW}[0]{\mathbf{W}}
\newcommand{\BX}[0]{\mathbf{X}}
\newcommand{\BY}[0]{\mathbf{Y}}
\newcommand{\BZ}[0]{\mathbf{Z}}

\newcommand{\Bzero}[0]{\mathbf{0}}
\newcommand{\Btheta}[0]{\boldsymbol{\theta}}
\newcommand{\Btau}[0]{\boldsymbol{\tau}}
\newcommand{\Bomega}[0]{\boldsymbol{\omega}}

%
% shorthand for unit vectors
%
\newcommand{\acap}[0]{\hat{\Ba}}
\newcommand{\bcap}[0]{\hat{\Bb}}
\newcommand{\ccap}[0]{\hat{\Bc}}
\newcommand{\dcap}[0]{\hat{\Bd}}
\newcommand{\ecap}[0]{\hat{\Be}}
\newcommand{\fcap}[0]{\hat{\Bf}}
\newcommand{\gcap}[0]{\hat{\Bg}}
\newcommand{\hcap}[0]{\hat{\Bh}}
\newcommand{\icap}[0]{\hat{\Bi}}
\newcommand{\jcap}[0]{\hat{\Bj}}
\newcommand{\kcap}[0]{\hat{\Bk}}
\newcommand{\lcap}[0]{\hat{\Bl}}
\newcommand{\mcap}[0]{\hat{\Bm}}
\newcommand{\ncap}[0]{\hat{\Bn}}
\newcommand{\ocap}[0]{\hat{\Bo}}
\newcommand{\pcap}[0]{\hat{\Bp}}
\newcommand{\qcap}[0]{\hat{\Bq}}
\newcommand{\rcap}[0]{\hat{\Br}}
\newcommand{\scap}[0]{\hat{\Bs}}
\newcommand{\tcap}[0]{\hat{\Bt}}
\newcommand{\ucap}[0]{\hat{\Bu}}
\newcommand{\vcap}[0]{\hat{\Bv}}
\newcommand{\wcap}[0]{\hat{\Bw}}
\newcommand{\xcap}[0]{\hat{\Bx}}
\newcommand{\ycap}[0]{\hat{\By}}
\newcommand{\zcap}[0]{\hat{\Bz}}
\newcommand{\thetacap}[0]{\hat{\Btheta}}

%
% to write R^n and C^n in a distinguishable fashion.  Perhaps change this
% to the double lined characters upon figuring out how to do so.
%
\newcommand{\C}[1]{$\mathbb{C}^{#1}$}
\newcommand{\R}[1]{$\mathbb{R}^{#1}$}

%
% various generally useful helpers
%

% derivative of #1 wrt. #2:
\newcommand{\D}[2] {\frac {d#2} {d#1}}

\newcommand{\inv}[1]{\frac{1}{#1}}
\newcommand{\cross}[0]{\times}

\newcommand{\abs}[1]{\lvert{#1}\rvert}
\newcommand{\norm}[1]{\lVert{#1}\rVert}
\newcommand{\innerprod}[2]{\langle{#1}, {#2}\rangle}
\newcommand{\dotprod}[2]{{#1} \cdot {#2}}
\newcommand{\bdotprod}[2]{\left({#1} \cdot {#2}\right)}
\newcommand{\crossprod}[2]{{#1} \cross {#2}}
\newcommand{\tripleprod}[3]{\dotprod{\left(\crossprod{#1}{#2}\right)}{#3}}

\DeclareMathOperator{\Proj}{Proj}
\DeclareMathOperator{\Span}{span}
\DeclareMathOperator{\Sgn}{sgn}
\DeclareMathOperator{\Area}{Area}
\DeclareMathOperator{\Volume}{Volume}

%
% A few miscellaneous things specific to this document
%
\newcommand{\crossop}[1]{\crossprod{#1}{}}

% R2 vector.
\newcommand{\VectorTwo}[2]{
\begin{bmatrix}
 {#1} \\
 {#2}
\end{bmatrix}
}

\newcommand{\VectorN}[1]{
\begin{bmatrix}
{#1}_1 \\
{#1}_2 \\
\vdots \\
{#1}_N \\
\end{bmatrix}
}

\newcommand{\DETuvij}[4]{
\begin{vmatrix}
 {#1}_{#3} & {#1}_{#4} \\
 {#2}_{#3} & {#2}_{#4}
\end{vmatrix}
}

\newcommand{\DETuvwijk}[6]{
\begin{vmatrix}
 {#1}_{#4} & {#1}_{#5} & {#1}_{#6} \\
 {#2}_{#4} & {#2}_{#5} & {#2}_{#6} \\
 {#3}_{#4} & {#3}_{#5} & {#3}_{#6}
\end{vmatrix}
}

\newcommand{\DETuvwxijkl}[8]{
\begin{vmatrix}
 {#1}_{#5} & {#1}_{#6} & {#1}_{#7} & {#1}_{#8} \\
 {#2}_{#5} & {#2}_{#6} & {#2}_{#7} & {#2}_{#8} \\
 {#3}_{#5} & {#3}_{#6} & {#3}_{#7} & {#3}_{#8} \\
 {#4}_{#5} & {#4}_{#6} & {#4}_{#7} & {#4}_{#8} \\
\end{vmatrix}
}

%\newcommand{\DETuvwxyijklm}[10]{
%\begin{vmatrix}
% {#1}_{#6} & {#1}_{#7} & {#1}_{#8} & {#1}_{#9} & {#1}_{#10} \\
% {#2}_{#6} & {#2}_{#7} & {#2}_{#8} & {#2}_{#9} & {#2}_{#10} \\
% {#3}_{#6} & {#3}_{#7} & {#3}_{#8} & {#3}_{#9} & {#3}_{#10} \\
% {#4}_{#6} & {#4}_{#7} & {#4}_{#8} & {#4}_{#9} & {#4}_{#10} \\
% {#5}_{#6} & {#5}_{#7} & {#5}_{#8} & {#5}_{#9} & {#5}_{#10}
%\end{vmatrix}
%}

% R3 vector.
\newcommand{\VectorThree}[3]{
\begin{bmatrix}
 {#1} \\
 {#2} \\
 {#3}
\end{bmatrix}
}



\author{Peeter Joot}
\email{peeter.joot@gmail.com}

%\documentclass[]{eliblogwidescreen}

\usepackage{amsmath}
\usepackage{mathpazo}

%
% shorthand for bold symbols, convenient for vectors and matrices
%
\newcommand{\Ba}[0]{\mathbf{a}}
\newcommand{\Bb}[0]{\mathbf{b}}
\newcommand{\Bc}[0]{\mathbf{c}}
\newcommand{\Bd}[0]{\mathbf{d}}
\newcommand{\Be}[0]{\mathbf{e}}
\newcommand{\Bf}[0]{\mathbf{f}}
\newcommand{\Bg}[0]{\mathbf{g}}
\newcommand{\Bh}[0]{\mathbf{h}}
\newcommand{\Bi}[0]{\mathbf{i}}
\newcommand{\Bj}[0]{\mathbf{j}}
\newcommand{\Bk}[0]{\mathbf{k}}
\newcommand{\Bl}[0]{\mathbf{l}}
\newcommand{\Bm}[0]{\mathbf{m}}
\newcommand{\Bn}[0]{\mathbf{n}}
\newcommand{\Bo}[0]{\mathbf{o}}
\newcommand{\Bp}[0]{\mathbf{p}}
\newcommand{\Bq}[0]{\mathbf{q}}
\newcommand{\Br}[0]{\mathbf{r}}
\newcommand{\Bs}[0]{\mathbf{s}}
\newcommand{\Bt}[0]{\mathbf{t}}
\newcommand{\Bu}[0]{\mathbf{u}}
\newcommand{\Bv}[0]{\mathbf{v}}
\newcommand{\Bw}[0]{\mathbf{w}}
\newcommand{\Bx}[0]{\mathbf{x}}
\newcommand{\By}[0]{\mathbf{y}}
\newcommand{\Bz}[0]{\mathbf{z}}
\newcommand{\BA}[0]{\mathbf{A}}
\newcommand{\BB}[0]{\mathbf{B}}
\newcommand{\BC}[0]{\mathbf{C}}
\newcommand{\BD}[0]{\mathbf{D}}
\newcommand{\BE}[0]{\mathbf{E}}
\newcommand{\BF}[0]{\mathbf{F}}
\newcommand{\BG}[0]{\mathbf{G}}
\newcommand{\BH}[0]{\mathbf{H}}
\newcommand{\BI}[0]{\mathbf{I}}
\newcommand{\BJ}[0]{\mathbf{J}}
\newcommand{\BK}[0]{\mathbf{K}}
\newcommand{\BL}[0]{\mathbf{L}}
\newcommand{\BM}[0]{\mathbf{M}}
\newcommand{\BN}[0]{\mathbf{N}}
\newcommand{\BO}[0]{\mathbf{O}}
\newcommand{\BP}[0]{\mathbf{P}}
\newcommand{\BQ}[0]{\mathbf{Q}}
\newcommand{\BR}[0]{\mathbf{R}}
\newcommand{\BS}[0]{\mathbf{S}}
\newcommand{\BT}[0]{\mathbf{T}}
\newcommand{\BU}[0]{\mathbf{U}}
\newcommand{\BV}[0]{\mathbf{V}}
\newcommand{\BW}[0]{\mathbf{W}}
\newcommand{\BX}[0]{\mathbf{X}}
\newcommand{\BY}[0]{\mathbf{Y}}
\newcommand{\BZ}[0]{\mathbf{Z}}

\newcommand{\Bzero}[0]{\mathbf{0}}
\newcommand{\Btheta}[0]{\boldsymbol{\theta}}
\newcommand{\Btau}[0]{\boldsymbol{\tau}}
\newcommand{\Bomega}[0]{\boldsymbol{\omega}}

%
% shorthand for unit vectors
%
\newcommand{\acap}[0]{\hat{\Ba}}
\newcommand{\bcap}[0]{\hat{\Bb}}
\newcommand{\ccap}[0]{\hat{\Bc}}
\newcommand{\dcap}[0]{\hat{\Bd}}
\newcommand{\ecap}[0]{\hat{\Be}}
\newcommand{\fcap}[0]{\hat{\Bf}}
\newcommand{\gcap}[0]{\hat{\Bg}}
\newcommand{\hcap}[0]{\hat{\Bh}}
\newcommand{\icap}[0]{\hat{\Bi}}
\newcommand{\jcap}[0]{\hat{\Bj}}
\newcommand{\kcap}[0]{\hat{\Bk}}
\newcommand{\lcap}[0]{\hat{\Bl}}
\newcommand{\mcap}[0]{\hat{\Bm}}
\newcommand{\ncap}[0]{\hat{\Bn}}
\newcommand{\ocap}[0]{\hat{\Bo}}
\newcommand{\pcap}[0]{\hat{\Bp}}
\newcommand{\qcap}[0]{\hat{\Bq}}
\newcommand{\rcap}[0]{\hat{\Br}}
\newcommand{\scap}[0]{\hat{\Bs}}
\newcommand{\tcap}[0]{\hat{\Bt}}
\newcommand{\ucap}[0]{\hat{\Bu}}
\newcommand{\vcap}[0]{\hat{\Bv}}
\newcommand{\wcap}[0]{\hat{\Bw}}
\newcommand{\xcap}[0]{\hat{\Bx}}
\newcommand{\ycap}[0]{\hat{\By}}
\newcommand{\zcap}[0]{\hat{\Bz}}
\newcommand{\thetacap}[0]{\hat{\Btheta}}

%
% to write R^n and C^n in a distinguishable fashion.  Perhaps change this
% to the double lined characters upon figuring out how to do so.
%
\newcommand{\C}[1]{$\mathbb{C}^{#1}$}
\newcommand{\R}[1]{$\mathbb{R}^{#1}$}

%
% various generally useful helpers
%

% derivative of #1 wrt. #2:
\newcommand{\D}[2] {\frac {d#2} {d#1}}

\newcommand{\inv}[1]{\frac{1}{#1}}
\newcommand{\cross}[0]{\times}

\newcommand{\abs}[1]{\lvert{#1}\rvert}
\newcommand{\norm}[1]{\lVert{#1}\rVert}
\newcommand{\innerprod}[2]{\langle{#1}, {#2}\rangle}
\newcommand{\dotprod}[2]{{#1} \cdot {#2}}
\newcommand{\bdotprod}[2]{\left({#1} \cdot {#2}\right)}
\newcommand{\crossprod}[2]{{#1} \cross {#2}}
\newcommand{\tripleprod}[3]{\dotprod{\left(\crossprod{#1}{#2}\right)}{#3}}

\DeclareMathOperator{\Proj}{Proj}
\DeclareMathOperator{\Span}{span}
\DeclareMathOperator{\Sgn}{sgn}
\DeclareMathOperator{\Area}{Area}
\DeclareMathOperator{\Volume}{Volume}

%
% A few miscellaneous things specific to this document
%
\newcommand{\crossop}[1]{\crossprod{#1}{}}

% R2 vector.
\newcommand{\VectorTwo}[2]{
\begin{bmatrix}
 {#1} \\
 {#2}
\end{bmatrix}
}

\newcommand{\VectorN}[1]{
\begin{bmatrix}
{#1}_1 \\
{#1}_2 \\
\vdots \\
{#1}_N \\
\end{bmatrix}
}

\newcommand{\DETuvij}[4]{
\begin{vmatrix}
 {#1}_{#3} & {#1}_{#4} \\
 {#2}_{#3} & {#2}_{#4}
\end{vmatrix}
}

\newcommand{\DETuvwijk}[6]{
\begin{vmatrix}
 {#1}_{#4} & {#1}_{#5} & {#1}_{#6} \\
 {#2}_{#4} & {#2}_{#5} & {#2}_{#6} \\
 {#3}_{#4} & {#3}_{#5} & {#3}_{#6}
\end{vmatrix}
}

\newcommand{\DETuvwxijkl}[8]{
\begin{vmatrix}
 {#1}_{#5} & {#1}_{#6} & {#1}_{#7} & {#1}_{#8} \\
 {#2}_{#5} & {#2}_{#6} & {#2}_{#7} & {#2}_{#8} \\
 {#3}_{#5} & {#3}_{#6} & {#3}_{#7} & {#3}_{#8} \\
 {#4}_{#5} & {#4}_{#6} & {#4}_{#7} & {#4}_{#8} \\
\end{vmatrix}
}

%\newcommand{\DETuvwxyijklm}[10]{
%\begin{vmatrix}
% {#1}_{#6} & {#1}_{#7} & {#1}_{#8} & {#1}_{#9} & {#1}_{#10} \\
% {#2}_{#6} & {#2}_{#7} & {#2}_{#8} & {#2}_{#9} & {#2}_{#10} \\
% {#3}_{#6} & {#3}_{#7} & {#3}_{#8} & {#3}_{#9} & {#3}_{#10} \\
% {#4}_{#6} & {#4}_{#7} & {#4}_{#8} & {#4}_{#9} & {#4}_{#10} \\
% {#5}_{#6} & {#5}_{#7} & {#5}_{#8} & {#5}_{#9} & {#5}_{#10}
%\end{vmatrix}
%}

% R3 vector.
\newcommand{\VectorThree}[3]{
\begin{bmatrix}
 {#1} \\
 {#2} \\
 {#3}
\end{bmatrix}
}



\author{Peeter Joot}
\email{peeter.joot@gmail.com}


\chapter{PHY456H1F: Quantum Mechanics II.  Lecture L5 (Taught by Prof J.E. Sipe).  Pertubation theory and degeneracy}
\label{chap:qmTwoL5}
%\useCCL
\blogpage{http://sites.google.com/site/peeterjoot/math2011/qmTwoL5.pdf}
\date{Sept 23, 2011}
\revisionInfo{qmTwoL5.tex}

\beginArtWithToc
%\beginArtNoToc

\section{Disclaimer.}

Peeter's lecture notes from class.  May not be entirely coherent.

\section{Issues concerning degeneracy.}

\section{Review of dynamics.}

This is covered in \S 3 of the text \cite{desai2009quantum}.

\EndArticle

%
% Copyright � 2012 Peeter Joot.  All Rights Reserved.
% Licenced as described in the file LICENSE under the root directory of this GIT repository.
%

%
%
%%
% Copyright � 2015 Peeter Joot.  All Rights Reserved.
% Licenced as described in the file LICENSE under the root directory of this GIT repository.
%
\documentclass[]{eliblog}

\usepackage{amsmath}
\usepackage{mathpazo}

%
% shorthand for bold symbols, convenient for vectors and matrices
%
\newcommand{\Ba}[0]{\mathbf{a}}
\newcommand{\Bb}[0]{\mathbf{b}}
\newcommand{\Bc}[0]{\mathbf{c}}
\newcommand{\Bd}[0]{\mathbf{d}}
\newcommand{\Be}[0]{\mathbf{e}}
\newcommand{\Bf}[0]{\mathbf{f}}
\newcommand{\Bg}[0]{\mathbf{g}}
\newcommand{\Bh}[0]{\mathbf{h}}
\newcommand{\Bi}[0]{\mathbf{i}}
\newcommand{\Bj}[0]{\mathbf{j}}
\newcommand{\Bk}[0]{\mathbf{k}}
\newcommand{\Bl}[0]{\mathbf{l}}
\newcommand{\Bm}[0]{\mathbf{m}}
\newcommand{\Bn}[0]{\mathbf{n}}
\newcommand{\Bo}[0]{\mathbf{o}}
\newcommand{\Bp}[0]{\mathbf{p}}
\newcommand{\Bq}[0]{\mathbf{q}}
\newcommand{\Br}[0]{\mathbf{r}}
\newcommand{\Bs}[0]{\mathbf{s}}
\newcommand{\Bt}[0]{\mathbf{t}}
\newcommand{\Bu}[0]{\mathbf{u}}
\newcommand{\Bv}[0]{\mathbf{v}}
\newcommand{\Bw}[0]{\mathbf{w}}
\newcommand{\Bx}[0]{\mathbf{x}}
\newcommand{\By}[0]{\mathbf{y}}
\newcommand{\Bz}[0]{\mathbf{z}}
\newcommand{\BA}[0]{\mathbf{A}}
\newcommand{\BB}[0]{\mathbf{B}}
\newcommand{\BC}[0]{\mathbf{C}}
\newcommand{\BD}[0]{\mathbf{D}}
\newcommand{\BE}[0]{\mathbf{E}}
\newcommand{\BF}[0]{\mathbf{F}}
\newcommand{\BG}[0]{\mathbf{G}}
\newcommand{\BH}[0]{\mathbf{H}}
\newcommand{\BI}[0]{\mathbf{I}}
\newcommand{\BJ}[0]{\mathbf{J}}
\newcommand{\BK}[0]{\mathbf{K}}
\newcommand{\BL}[0]{\mathbf{L}}
\newcommand{\BM}[0]{\mathbf{M}}
\newcommand{\BN}[0]{\mathbf{N}}
\newcommand{\BO}[0]{\mathbf{O}}
\newcommand{\BP}[0]{\mathbf{P}}
\newcommand{\BQ}[0]{\mathbf{Q}}
\newcommand{\BR}[0]{\mathbf{R}}
\newcommand{\BS}[0]{\mathbf{S}}
\newcommand{\BT}[0]{\mathbf{T}}
\newcommand{\BU}[0]{\mathbf{U}}
\newcommand{\BV}[0]{\mathbf{V}}
\newcommand{\BW}[0]{\mathbf{W}}
\newcommand{\BX}[0]{\mathbf{X}}
\newcommand{\BY}[0]{\mathbf{Y}}
\newcommand{\BZ}[0]{\mathbf{Z}}

\newcommand{\Bzero}[0]{\mathbf{0}}
\newcommand{\Btheta}[0]{\boldsymbol{\theta}}
\newcommand{\Btau}[0]{\boldsymbol{\tau}}
\newcommand{\Bomega}[0]{\boldsymbol{\omega}}

%
% shorthand for unit vectors
%
\newcommand{\acap}[0]{\hat{\Ba}}
\newcommand{\bcap}[0]{\hat{\Bb}}
\newcommand{\ccap}[0]{\hat{\Bc}}
\newcommand{\dcap}[0]{\hat{\Bd}}
\newcommand{\ecap}[0]{\hat{\Be}}
\newcommand{\fcap}[0]{\hat{\Bf}}
\newcommand{\gcap}[0]{\hat{\Bg}}
\newcommand{\hcap}[0]{\hat{\Bh}}
\newcommand{\icap}[0]{\hat{\Bi}}
\newcommand{\jcap}[0]{\hat{\Bj}}
\newcommand{\kcap}[0]{\hat{\Bk}}
\newcommand{\lcap}[0]{\hat{\Bl}}
\newcommand{\mcap}[0]{\hat{\Bm}}
\newcommand{\ncap}[0]{\hat{\Bn}}
\newcommand{\ocap}[0]{\hat{\Bo}}
\newcommand{\pcap}[0]{\hat{\Bp}}
\newcommand{\qcap}[0]{\hat{\Bq}}
\newcommand{\rcap}[0]{\hat{\Br}}
\newcommand{\scap}[0]{\hat{\Bs}}
\newcommand{\tcap}[0]{\hat{\Bt}}
\newcommand{\ucap}[0]{\hat{\Bu}}
\newcommand{\vcap}[0]{\hat{\Bv}}
\newcommand{\wcap}[0]{\hat{\Bw}}
\newcommand{\xcap}[0]{\hat{\Bx}}
\newcommand{\ycap}[0]{\hat{\By}}
\newcommand{\zcap}[0]{\hat{\Bz}}
\newcommand{\thetacap}[0]{\hat{\Btheta}}

%
% to write R^n and C^n in a distinguishable fashion.  Perhaps change this
% to the double lined characters upon figuring out how to do so.
%
\newcommand{\C}[1]{$\mathbb{C}^{#1}$}
\newcommand{\R}[1]{$\mathbb{R}^{#1}$}

%
% various generally useful helpers
%

% derivative of #1 wrt. #2:
\newcommand{\D}[2] {\frac {d#2} {d#1}}

\newcommand{\inv}[1]{\frac{1}{#1}}
\newcommand{\cross}[0]{\times}

\newcommand{\abs}[1]{\lvert{#1}\rvert}
\newcommand{\norm}[1]{\lVert{#1}\rVert}
\newcommand{\innerprod}[2]{\langle{#1}, {#2}\rangle}
\newcommand{\dotprod}[2]{{#1} \cdot {#2}}
\newcommand{\bdotprod}[2]{\left({#1} \cdot {#2}\right)}
\newcommand{\crossprod}[2]{{#1} \cross {#2}}
\newcommand{\tripleprod}[3]{\dotprod{\left(\crossprod{#1}{#2}\right)}{#3}}

\DeclareMathOperator{\Proj}{Proj}
\DeclareMathOperator{\Span}{span}
\DeclareMathOperator{\Sgn}{sgn}
\DeclareMathOperator{\Area}{Area}
\DeclareMathOperator{\Volume}{Volume}

%
% A few miscellaneous things specific to this document
%
\newcommand{\crossop}[1]{\crossprod{#1}{}}

% R2 vector.
\newcommand{\VectorTwo}[2]{
\begin{bmatrix}
 {#1} \\
 {#2}
\end{bmatrix}
}

\newcommand{\VectorN}[1]{
\begin{bmatrix}
{#1}_1 \\
{#1}_2 \\
\vdots \\
{#1}_N \\
\end{bmatrix}
}

\newcommand{\DETuvij}[4]{
\begin{vmatrix}
 {#1}_{#3} & {#1}_{#4} \\
 {#2}_{#3} & {#2}_{#4}
\end{vmatrix}
}

\newcommand{\DETuvwijk}[6]{
\begin{vmatrix}
 {#1}_{#4} & {#1}_{#5} & {#1}_{#6} \\
 {#2}_{#4} & {#2}_{#5} & {#2}_{#6} \\
 {#3}_{#4} & {#3}_{#5} & {#3}_{#6}
\end{vmatrix}
}

\newcommand{\DETuvwxijkl}[8]{
\begin{vmatrix}
 {#1}_{#5} & {#1}_{#6} & {#1}_{#7} & {#1}_{#8} \\
 {#2}_{#5} & {#2}_{#6} & {#2}_{#7} & {#2}_{#8} \\
 {#3}_{#5} & {#3}_{#6} & {#3}_{#7} & {#3}_{#8} \\
 {#4}_{#5} & {#4}_{#6} & {#4}_{#7} & {#4}_{#8} \\
\end{vmatrix}
}

%\newcommand{\DETuvwxyijklm}[10]{
%\begin{vmatrix}
% {#1}_{#6} & {#1}_{#7} & {#1}_{#8} & {#1}_{#9} & {#1}_{#10} \\
% {#2}_{#6} & {#2}_{#7} & {#2}_{#8} & {#2}_{#9} & {#2}_{#10} \\
% {#3}_{#6} & {#3}_{#7} & {#3}_{#8} & {#3}_{#9} & {#3}_{#10} \\
% {#4}_{#6} & {#4}_{#7} & {#4}_{#8} & {#4}_{#9} & {#4}_{#10} \\
% {#5}_{#6} & {#5}_{#7} & {#5}_{#8} & {#5}_{#9} & {#5}_{#10}
%\end{vmatrix}
%}

% R3 vector.
\newcommand{\VectorThree}[3]{
\begin{bmatrix}
 {#1} \\
 {#2} \\
 {#3}
\end{bmatrix}
}



\author{Peeter Joot}
\email{peeter.joot@gmail.com}

%\documentclass[]{eliblogwidescreen}

\usepackage{amsmath}
\usepackage{mathpazo}

%
% shorthand for bold symbols, convenient for vectors and matrices
%
\newcommand{\Ba}[0]{\mathbf{a}}
\newcommand{\Bb}[0]{\mathbf{b}}
\newcommand{\Bc}[0]{\mathbf{c}}
\newcommand{\Bd}[0]{\mathbf{d}}
\newcommand{\Be}[0]{\mathbf{e}}
\newcommand{\Bf}[0]{\mathbf{f}}
\newcommand{\Bg}[0]{\mathbf{g}}
\newcommand{\Bh}[0]{\mathbf{h}}
\newcommand{\Bi}[0]{\mathbf{i}}
\newcommand{\Bj}[0]{\mathbf{j}}
\newcommand{\Bk}[0]{\mathbf{k}}
\newcommand{\Bl}[0]{\mathbf{l}}
\newcommand{\Bm}[0]{\mathbf{m}}
\newcommand{\Bn}[0]{\mathbf{n}}
\newcommand{\Bo}[0]{\mathbf{o}}
\newcommand{\Bp}[0]{\mathbf{p}}
\newcommand{\Bq}[0]{\mathbf{q}}
\newcommand{\Br}[0]{\mathbf{r}}
\newcommand{\Bs}[0]{\mathbf{s}}
\newcommand{\Bt}[0]{\mathbf{t}}
\newcommand{\Bu}[0]{\mathbf{u}}
\newcommand{\Bv}[0]{\mathbf{v}}
\newcommand{\Bw}[0]{\mathbf{w}}
\newcommand{\Bx}[0]{\mathbf{x}}
\newcommand{\By}[0]{\mathbf{y}}
\newcommand{\Bz}[0]{\mathbf{z}}
\newcommand{\BA}[0]{\mathbf{A}}
\newcommand{\BB}[0]{\mathbf{B}}
\newcommand{\BC}[0]{\mathbf{C}}
\newcommand{\BD}[0]{\mathbf{D}}
\newcommand{\BE}[0]{\mathbf{E}}
\newcommand{\BF}[0]{\mathbf{F}}
\newcommand{\BG}[0]{\mathbf{G}}
\newcommand{\BH}[0]{\mathbf{H}}
\newcommand{\BI}[0]{\mathbf{I}}
\newcommand{\BJ}[0]{\mathbf{J}}
\newcommand{\BK}[0]{\mathbf{K}}
\newcommand{\BL}[0]{\mathbf{L}}
\newcommand{\BM}[0]{\mathbf{M}}
\newcommand{\BN}[0]{\mathbf{N}}
\newcommand{\BO}[0]{\mathbf{O}}
\newcommand{\BP}[0]{\mathbf{P}}
\newcommand{\BQ}[0]{\mathbf{Q}}
\newcommand{\BR}[0]{\mathbf{R}}
\newcommand{\BS}[0]{\mathbf{S}}
\newcommand{\BT}[0]{\mathbf{T}}
\newcommand{\BU}[0]{\mathbf{U}}
\newcommand{\BV}[0]{\mathbf{V}}
\newcommand{\BW}[0]{\mathbf{W}}
\newcommand{\BX}[0]{\mathbf{X}}
\newcommand{\BY}[0]{\mathbf{Y}}
\newcommand{\BZ}[0]{\mathbf{Z}}

\newcommand{\Bzero}[0]{\mathbf{0}}
\newcommand{\Btheta}[0]{\boldsymbol{\theta}}
\newcommand{\Btau}[0]{\boldsymbol{\tau}}
\newcommand{\Bomega}[0]{\boldsymbol{\omega}}

%
% shorthand for unit vectors
%
\newcommand{\acap}[0]{\hat{\Ba}}
\newcommand{\bcap}[0]{\hat{\Bb}}
\newcommand{\ccap}[0]{\hat{\Bc}}
\newcommand{\dcap}[0]{\hat{\Bd}}
\newcommand{\ecap}[0]{\hat{\Be}}
\newcommand{\fcap}[0]{\hat{\Bf}}
\newcommand{\gcap}[0]{\hat{\Bg}}
\newcommand{\hcap}[0]{\hat{\Bh}}
\newcommand{\icap}[0]{\hat{\Bi}}
\newcommand{\jcap}[0]{\hat{\Bj}}
\newcommand{\kcap}[0]{\hat{\Bk}}
\newcommand{\lcap}[0]{\hat{\Bl}}
\newcommand{\mcap}[0]{\hat{\Bm}}
\newcommand{\ncap}[0]{\hat{\Bn}}
\newcommand{\ocap}[0]{\hat{\Bo}}
\newcommand{\pcap}[0]{\hat{\Bp}}
\newcommand{\qcap}[0]{\hat{\Bq}}
\newcommand{\rcap}[0]{\hat{\Br}}
\newcommand{\scap}[0]{\hat{\Bs}}
\newcommand{\tcap}[0]{\hat{\Bt}}
\newcommand{\ucap}[0]{\hat{\Bu}}
\newcommand{\vcap}[0]{\hat{\Bv}}
\newcommand{\wcap}[0]{\hat{\Bw}}
\newcommand{\xcap}[0]{\hat{\Bx}}
\newcommand{\ycap}[0]{\hat{\By}}
\newcommand{\zcap}[0]{\hat{\Bz}}
\newcommand{\thetacap}[0]{\hat{\Btheta}}

%
% to write R^n and C^n in a distinguishable fashion.  Perhaps change this
% to the double lined characters upon figuring out how to do so.
%
\newcommand{\C}[1]{$\mathbb{C}^{#1}$}
\newcommand{\R}[1]{$\mathbb{R}^{#1}$}

%
% various generally useful helpers
%

% derivative of #1 wrt. #2:
\newcommand{\D}[2] {\frac {d#2} {d#1}}

\newcommand{\inv}[1]{\frac{1}{#1}}
\newcommand{\cross}[0]{\times}

\newcommand{\abs}[1]{\lvert{#1}\rvert}
\newcommand{\norm}[1]{\lVert{#1}\rVert}
\newcommand{\innerprod}[2]{\langle{#1}, {#2}\rangle}
\newcommand{\dotprod}[2]{{#1} \cdot {#2}}
\newcommand{\bdotprod}[2]{\left({#1} \cdot {#2}\right)}
\newcommand{\crossprod}[2]{{#1} \cross {#2}}
\newcommand{\tripleprod}[3]{\dotprod{\left(\crossprod{#1}{#2}\right)}{#3}}

\DeclareMathOperator{\Proj}{Proj}
\DeclareMathOperator{\Span}{span}
\DeclareMathOperator{\Sgn}{sgn}
\DeclareMathOperator{\Area}{Area}
\DeclareMathOperator{\Volume}{Volume}

%
% A few miscellaneous things specific to this document
%
\newcommand{\crossop}[1]{\crossprod{#1}{}}

% R2 vector.
\newcommand{\VectorTwo}[2]{
\begin{bmatrix}
 {#1} \\
 {#2}
\end{bmatrix}
}

\newcommand{\VectorN}[1]{
\begin{bmatrix}
{#1}_1 \\
{#1}_2 \\
\vdots \\
{#1}_N \\
\end{bmatrix}
}

\newcommand{\DETuvij}[4]{
\begin{vmatrix}
 {#1}_{#3} & {#1}_{#4} \\
 {#2}_{#3} & {#2}_{#4}
\end{vmatrix}
}

\newcommand{\DETuvwijk}[6]{
\begin{vmatrix}
 {#1}_{#4} & {#1}_{#5} & {#1}_{#6} \\
 {#2}_{#4} & {#2}_{#5} & {#2}_{#6} \\
 {#3}_{#4} & {#3}_{#5} & {#3}_{#6}
\end{vmatrix}
}

\newcommand{\DETuvwxijkl}[8]{
\begin{vmatrix}
 {#1}_{#5} & {#1}_{#6} & {#1}_{#7} & {#1}_{#8} \\
 {#2}_{#5} & {#2}_{#6} & {#2}_{#7} & {#2}_{#8} \\
 {#3}_{#5} & {#3}_{#6} & {#3}_{#7} & {#3}_{#8} \\
 {#4}_{#5} & {#4}_{#6} & {#4}_{#7} & {#4}_{#8} \\
\end{vmatrix}
}

%\newcommand{\DETuvwxyijklm}[10]{
%\begin{vmatrix}
% {#1}_{#6} & {#1}_{#7} & {#1}_{#8} & {#1}_{#9} & {#1}_{#10} \\
% {#2}_{#6} & {#2}_{#7} & {#2}_{#8} & {#2}_{#9} & {#2}_{#10} \\
% {#3}_{#6} & {#3}_{#7} & {#3}_{#8} & {#3}_{#9} & {#3}_{#10} \\
% {#4}_{#6} & {#4}_{#7} & {#4}_{#8} & {#4}_{#9} & {#4}_{#10} \\
% {#5}_{#6} & {#5}_{#7} & {#5}_{#8} & {#5}_{#9} & {#5}_{#10}
%\end{vmatrix}
%}

% R3 vector.
\newcommand{\VectorThree}[3]{
\begin{bmatrix}
 {#1} \\
 {#2} \\
 {#3}
\end{bmatrix}
}



\author{Peeter Joot}
\email{peeter.joot@gmail.com}


%\chapter{PHY456H1F: Quantum Mechanics II.  Lecture 6 (Taught by Prof J.E. Sipe).  Interaction picture}
%\chapter{Interaction picture}
\index{interaction picture}
\label{chap:qmTwoL6}
\blogpage{http://sites.google.com/site/peeterjoot/math2011/qmTwoL6.pdf}
%\date{Sept 26, 2011}





\section{Interaction picture}
\paragraph{Recap}

Recall our table comparing our two interaction pictures

\begin{equation}\label{eqn:qmTwoL6:970}
\begin{aligned}
\text{Schr\"{o}dinger picture} &\qquad \text{Heisenberg picture} \\
i \Hbar \frac{d}{dt} \ket{\psi_s(t)} = H \ket{\psi_s(t)} &\qquad i \Hbar \frac{d}{dt} O_H(t) = \antisymmetric{O_H}{H} \\
\bra{\psi_s(t)} O_S \ket{\psi_s(t)} &= \bra{\psi_H} O_H \ket{\psi_H} \\
\ket{\psi_s(0)} &= \ket{\psi_H} \\
O_S &= O_H(0)
\end{aligned}
\end{equation}

\paragraph{A motivating example}

While fundamental Hamiltonians are independent of time, in a number of common cases, we can form approximate Hamiltonians that are time dependent.  One such example is that of Coulomb excitations of an atom, as covered in \S 18.3 of the text \citep{desai2009quantum}, and shown in \cref{fig:qmTwoL6:1}.

\imageFigure{../../figures/phy456/qmTwoL6fig1}{Coulomb interaction of a nucleus and heavy atom}{fig:qmTwoL6:1}{0.4}

We consider the interaction of a nucleus with a neutral atom, heavy enough that it can be considered classically.  From the atoms point of view, the effects of the heavy nucleus barreling by can be described using a time dependent Hamiltonian.  For the atom, that interaction Hamiltonian is

\begin{equation}\label{eqn:qmTwoL6:10}
H' = \sum_i \frac{ Z e q_i }{\Abs{\Br_N(t) - \BR_i}}.
\end{equation}

Here and \(\Br_N\) is the position vector for the heavy nucleus, and \(\BR_i\) is the position to each charge within the atom, where \(i\) ranges over all the internal charges, positive and negative, within the atom.

Placing the origin close to the atom, we can write this interaction Hamiltonian as

\begin{equation}\label{eqn:qmTwoL6:30}
H'(t) = \cancel{\sum_i \frac{Z e q_i}{\Abs{\Br_N(t)}}}
+ \sum_i Z e q_i \BR_i \cdot
\evalbar{\left(
\PD{\Br}{} \inv{\Abs{ \Br_N(t) - \Br}}
\right)}{\Br = 0}
\end{equation}

The first term vanishes because the total charge in our neutral atom is zero.  This leaves us with

\begin{equation}\label{eqn:qmTwoL6:50}
\begin{aligned}
H'(t)
&=
-\sum_i q_i \BR_i \cdot \evalbar{\left(
-\PD{\Br}{} \frac{ Z e}{\Abs{ \Br_N(t) - \Br}}
\right)}{\Br = 0} \\
&= - \sum_i q_i \BR_i \cdot \BE(t),
\end{aligned}
\end{equation}

where \(\BE(t)\) is the electric field at the origin due to the nucleus.

Introducing a dipole moment \textunderline{operator} for the atom

\begin{equation}\label{eqn:qmTwoL6:70}
\Bmu = \sum_i q_i \BR_i,
\end{equation}

the interaction takes the form

\begin{equation}\label{eqn:qmTwoL6:90}
H'(t) = -\Bmu \cdot \BE(t).
\end{equation}

Here we have a quantum mechanical operator, and a classical field taken together.  This sort of dipole interaction also occurs when we treat a atom placed into an electromagnetic field, treated classically as depicted in \cref{fig:qmTwoL6:2}

\imageFigure{../../figures/phy456/qmTwoL6fig2}{atom in a field}{fig:qmTwoL6:2}{0.4}

In the figure, we can use the dipole interaction, provided \(\lambda \gg a\), where \(a\) is the ``width'' of the atom.

Because it is great for examples, we will see this dipole interaction a lot.

\paragraph{The interaction picture}

Having talked about both the Schr\"{o}dinger and Heisenberg pictures, we can now move on to describe a hybrid, one where our Hamiltonian has been split into static and time dependent parts

\begin{equation}\label{eqn:qmTwoL6:110}
H(t) = H_0 + H'(t)
\end{equation}

We will formulate an approach for dealing with problems of this sort called the interaction picture.

This is also covered in \S 3.3 of the text, albeit in a much harder to understand fashion (the text appears to try to not pull the result from a magic hat, but the steps to get to the end result are messy).  It would probably have been nicer to see it this way instead.

In the Schr\"{o}dinger picture our dynamics have the form

\begin{equation}\label{eqn:qmTwoL6:130}
i \Hbar \frac{d}{dt} \ket{\psi_s(t)} = H \ket{\psi_s(t)}
\end{equation}

How about the Heisenberg picture?  We look for a solution

\begin{equation}\label{eqn:qmTwoL6:400}
\ket{\psi_s(t)} = U(t, t_0) \ket{\psi_s(t_0)}.
\end{equation}

We want to find this operator that evolves the state from the state as some initial time \(t_0\), to the arbitrary later state found at time \(t\).  Plugging in we have

\begin{equation}\label{eqn:qmTwoL6:420}
i \Hbar \ddt{} U(t, t_0) \ket{\psi_s(t_0)}
=
H(t) U(t, t_0) \ket{\psi_s(t_0)}.
\end{equation}

This has to hold for all \(\ket{\psi_s(t_0)}\), and we can equivalently seek a solution of the operator equation

\begin{equation}\label{eqn:qmTwoL6:440}
i \Hbar \ddt{} U(t, t_0) = H(t) U(t, t_0),
\end{equation}

where

\begin{equation}\label{eqn:qmTwoL6:460}
U(t_0, t_0) = I,
\end{equation}

the identity for the Hilbert space.

Suppose that \(H(t)\) was independent of time.  We could find that

\begin{equation}\label{eqn:qmTwoL6:480}
U(t, t_0) = e^{-i H(t - t_0)/\Hbar}.
\end{equation}

If \(H(t)\) depends on time could you guess that

\begin{equation}\label{eqn:qmTwoL6:500}
U(t, t_0) = e^{-\frac{i}{\Hbar} \int_{t_0}^t H(\tau) d\tau}
\end{equation}

holds?  No.  This may be true when \(H(t)\) is a number, but when it is an operator, the Hamiltonian does not necessarily commute with itself at different times

\begin{equation}\label{eqn:qmTwoL6:520}
\antisymmetric{H(t')}{H(t'')} \ne 0.
\end{equation}

So this is \textunderline{wrong} in general.  As an aside, for numbers, \eqnref{eqn:qmTwoL6:500} can be verified easily.  We have

\begin{equation}\label{eqn:qmTwoL6:990}
\begin{aligned}
i \Hbar \left( e^{-\frac{i}{\Hbar} \int_{t_0}^t H(\tau) d\tau} \right)'
&=
i \Hbar \left( -\frac{i}{\Hbar} \right) \left( \int_{t_0}^t H(\tau) d\tau \right)'
e^{-\frac{i}{\Hbar} \int_{t_0}^t H(\tau) d\tau } \\
&=
\left( H(t) \frac{dt}{dt} - H(t_0) \frac{dt_0}{dt} \right)
e^{-\frac{i}{\Hbar} \int_{t_0}^t H(\tau) d\tau}  \\
&=
H(t) U(t, t_0)
\end{aligned}
\end{equation}

\paragraph{Expectations}

Suppose that we do find \(U(t, t_0)\).  Then our expectation takes the form

\begin{equation}\label{eqn:qmTwoL6:600}
\bra{\psi_s(t)} O_s \ket{\psi_s(t)}
=
\bra{\psi_s(t_0)} U^\dagger(t, t_0) O_s U(t, t_0) \ket{\psi_s(t_0)}
\end{equation}

Put

\begin{equation}\label{eqn:qmTwoL6:620}
\ket{\psi_H} = \ket{\psi_s(t_0)},
\end{equation}

and form

\begin{equation}\label{eqn:qmTwoL6:640}
O_H = U^\dagger(t, t_0) O_s U(t, t_0)
\end{equation}

so that our expectation has the familiar representations

\begin{equation}\label{eqn:qmTwoL6:660}
\bra{\psi_s(t)} O_s \ket{\psi_s(t)}
=
\bra{\psi_H} O_H \ket{\psi_H}
\end{equation}

\paragraph{New strategy.  Interaction picture}

Let us define

\begin{equation}\label{eqn:qmTwoL6:680}
U_I(t, t_0) = e^{\frac{i}{\Hbar} H_0(t - t_0)} U(t, t_0)
\end{equation}

or
\begin{equation}\label{eqn:qmTwoL6:700}
U(t, t_0) = e^{-\frac{i}{\Hbar} H_0(t - t_0)} U_I(t, t_0).
\end{equation}

Let us see how this works.  We have

\begin{equation}\label{eqn:qmTwoL6:1010}
\begin{aligned}
i \Hbar \ddt{U_I}
&=
i \Hbar \ddt{} \left(
e^{\frac{i}{\Hbar} H_0(t - t_0)} U(t, t_0)
\right) \\
&=
-H_0 U(t, t_0)
+
e^{\frac{i}{\Hbar} H_0(t - t_0)} \left( i \Hbar \ddt{} U(t, t_0) \right) \\
&=
-H_0 U(t, t_0)
+
e^{\frac{i}{\Hbar} H_0(t - t_0)} \left( (H + H'(t)) U(t, t_0) \right) \\
&=
e^{\frac{i}{\Hbar} H_0(t - t_0)} H'(t) U(t, t_0) \\
&=
e^{\frac{i}{\Hbar} H_0(t - t_0)} H'(t) e^{-\frac{i}{\Hbar} H_0(t - t_0)} U_I(t, t_0).
\end{aligned}
\end{equation}

Define

\begin{equation}\label{eqn:qmTwoL6:720}
\overbar{H}'(t) =
e^{\frac{i}{\Hbar} H_0(t - t_0)} H'(t) e^{-\frac{i}{\Hbar} H_0(t - t_0)},
\end{equation}

so that our operator equation takes the form

\begin{equation}\label{eqn:qmTwoL6:740}
i \Hbar \ddt{} U_I(t, t_0) = \overbar{H}'(t) U_I(t, t_0).
\end{equation}

Note that we also have the required identity at the initial time

\begin{equation}\label{eqn:qmTwoL6:760}
U_I(t_0, t_0) = I.
\end{equation}

Without requiring us to actually find \(U(t, t_0)\) all of the dynamics of the time dependent interaction are now embedded in our operator equation for \(\overbar{H}'(t)\), with all of the simple interaction related to the non time dependent portions of the Hamiltonian left separate.

\paragraph{Connection with the Schr\"{o}dinger picture}

In the Schr\"{o}dinger picture we have

\begin{equation}\label{eqn:qmTwoL6:1030}
\begin{aligned}
\ket{\psi_s(t)}
&= U(t, t_0)
\ket{\psi_s(t_0)}  \\
&=
e^{-\frac{i}{\Hbar} H_0(t - t_0)} U_I(t, t_0)
\ket{\psi_s(t_0)}.
\end{aligned}
\end{equation}

With a definition of the interaction picture ket as

\begin{equation}\label{eqn:qmTwoL6:780}
\ket{\psi_I}
= U_I(t, t_0) \ket{\psi_s(t_0)} = U_I(t, t_0) \ket{\psi_H},
\end{equation}

the Schr\"{o}dinger picture is then related to the interaction picture by

\begin{equation}\label{eqn:qmTwoL6:800}
\ket{\psi_s(t)} = e^{-\frac{i}{\Hbar} H_0(t - t_0)} \ket{\psi_I}.
\end{equation}

Also, by multiplying \eqnref{eqn:qmTwoL6:740} by our Schr\"{o}dinger ket, we remove the last vestiges of \(U_I\) and \(U\) from the dynamical equation for our time dependent interaction

\begin{equation}\label{eqn:qmTwoL6:820}
i \Hbar \ddt{}
\ket{\psi_I}
= \overbar{H}'(t)
\ket{\psi_I}.
\end{equation}

\paragraph{Interaction picture expectation}

Inverting \eqnref{eqn:qmTwoL6:800}, we can form an operator expectation, and relate it the interaction and Schr\"{o}dinger pictures

\begin{equation}\label{eqn:qmTwoL6:840}
\bra{\psi_s(t)} O_s \ket{\psi_s(t)} =
\bra{\psi_I}
e^{\frac{i}{\Hbar} H_0(t - t_0)}
O_s
e^{-\frac{i}{\Hbar} H_0(t - t_0)}
\ket{\psi_I} .
\end{equation}

With a definition

\begin{equation}\label{eqn:qmTwoL6:860}
O_I =
e^{\frac{i}{\Hbar} H_0(t - t_0)}
O_s
e^{-\frac{i}{\Hbar} H_0(t - t_0)},
\end{equation}

we have
\begin{equation}\label{eqn:qmTwoL6:880}
\bra{\psi_s(t)} O_s \ket{\psi_s(t)} =
\bra{\psi_I}
O_I
\ket{\psi_I}.
\end{equation}

As before, the time evolution of our interaction picture operator, can be found by taking derivatives of \eqnref{eqn:qmTwoL6:860}, for which we find

\begin{equation}\label{eqn:qmTwoL6:900}
i \Hbar \ddt{O_I(t)} = \antisymmetric{O_I(t)}{H_0}
\end{equation}

\paragraph{Summarizing the interaction picture}

Given

\begin{equation}\label{eqn:qmTwoL6:910}
H(t) = H_0 + H'(t),
\end{equation}

and initial time states
\begin{equation}\label{eqn:qmTwoL6:950}
\ket{\psi_I(t_0)} =
\ket{\psi_s(t_0)} = \ket{\psi_H},
\end{equation}

we have
\begin{equation}\label{eqn:qmTwoL6:880b}
\bra{\psi_s(t)} O_s \ket{\psi_s(t)} =
\bra{\psi_I}
O_I
\ket{\psi_I},
\end{equation}

where

\begin{equation}\label{eqn:qmTwoL6:920}
\ket{\psi_I}
= U_I(t, t_0) \ket{\psi_s(t_0)},
\end{equation}

and

\begin{equation}\label{eqn:qmTwoL6:820b}
i \Hbar \ddt{}
\ket{\psi_I}
= \overbar{H}'(t)
\ket{\psi_I},
\end{equation}

or

\begin{equation}\label{eqn:qmTwoL6:740b}
\begin{aligned}
i \Hbar \ddt{} U_I(t, t_0) &= \overbar{H}'(t) U_I(t, t_0) \\
U_I(t_0, t_0) &= I.
\end{aligned}
\end{equation}

Our interaction picture Hamiltonian is

\begin{equation}\label{eqn:qmTwoL6:720b}
\overbar{H}'(t) =
e^{\frac{i}{\Hbar} H_0(t - t_0)} H'(t) e^{-\frac{i}{\Hbar} H_0(t - t_0)},
\end{equation}

and for Schr\"{o}dinger operators, independent of time, we have the dynamical equation

\begin{equation}\label{eqn:qmTwoL6:900b}
i \Hbar \ddt{O_I(t)} = \antisymmetric{O_I(t)}{H_0}
\end{equation}

\section{Justifying the Taylor expansion above (not class notes)}

\paragraph{Multivariable Taylor series}
\index{Taylor series!multivariable}

As outlined in \S 2.8 (\(8.10\)) of \citep{hestenes1999nfc}, we want to derive the multi-variable Taylor expansion for a scalar valued function of some number of variables

\begin{equation}\label{eqn:qmTwoL6:210}
f(\Bu) = f(u^1, u^2, \cdots),
\end{equation}

consider the displacement operation applied to the vector argument

\begin{equation}\label{eqn:qmTwoL6:230}
f(\Ba + \Bx) = \evalbar{f(\Ba + t \Bx)}{t=1}.
\end{equation}

We can Taylor expand a single variable function without any trouble, so introduce

\begin{equation}\label{eqn:qmTwoL6:250}
g(t) = f(\Ba + t \Bx),
\end{equation}

where

\begin{equation}\label{eqn:qmTwoL6:270}
g(1) = f(\Ba + \Bx).
\end{equation}

We have

\begin{equation}\label{eqn:qmTwoL6:290}
g(t) = g(0)
+ t \evalbar{ \PD{t}{g} }{t = 0}
+ \frac{t^2}{2!} \evalbar{ \PD{t}{g} }{t = 0}
+ \cdots,
\end{equation}

so that

\begin{equation}\label{eqn:qmTwoL6:310}
g(1) = g(0) +
+ \evalbar{ \PD{t}{g} }{t = 0}
+ \frac{1}{2!} \evalbar{ \PD{t}{g} }{t = 0}
+ \cdots.
\end{equation}

The multivariable Taylor series now becomes a plain old application of the chain rule, where we have to evaluate

\begin{equation}\label{eqn:qmTwoL6:1050}
\begin{aligned}
\frac{dg}{dt}
&= \ddt{} f(a^1 + t x^1, a^2 + t x^2, \cdots) \\
&= \sum_i \PD{(a^i + t x^i)}{} f(\Ba + t \Bx) \PD{t}{a^i + t x^i},
\end{aligned}
\end{equation}

so that

\begin{equation}\label{eqn:qmTwoL6:330}
\evalbar{\frac{dg}{dt} }{t=0}
= \sum_i x^i \left(
\evalbar{ \PD{x^i}{f}}{x^i = a^i}
\right).
\end{equation}

Assuming an Euclidean space we can write this in the notationally more pleasant fashion using a gradient operator for the space

\begin{equation}\label{eqn:qmTwoL6:350}
\evalbar{\frac{dg}{dt} }{t=0} = \evalbar{\Bx \cdot \spacegrad_{\Bu} f(\Bu)}{\Bu = \Ba}.
\end{equation}

To handle the higher order terms, we repeat the chain rule application, yielding for example

\begin{equation}\label{eqn:qmTwoL6:1070}
\begin{aligned}
\evalbar{\frac{d^2 f(\Ba + t \Bx)}{dt^2} }{t=0}
&=
\evalbar{\ddt{}
\sum_i x^i
\PD{(a^i + t x^i)}{f(\Ba + t \Bx)} }{t=0}\\
&=
\evalbar{\sum_i x^i
\PD{(a^i + t x^i)}{} \ddt{f(\Ba + t \Bx)}}{t=0} \\
&=
\evalbar{(\Bx \cdot \spacegrad_{\Bu})^2 f(\Bu)}{\Bu = \Ba}.
\end{aligned}
\end{equation}

Thus the Taylor series associated with a vector displacement takes the tidy form

\begin{equation}\label{eqn:qmTwoL6:370}
f(\Ba + \Bx) = \sum_{k=0}^\infty \inv{k!} \evalbar{(\Bx \cdot \spacegrad_{\Bu})^k f(\Bu)}{\Bu = \Ba}.
\end{equation}

Even more fancy, we can form the operator equation

\begin{equation}\label{eqn:qmTwoL6:390}
f(\Ba + \Bx) = \evalbar{e^{ \Bx \cdot \spacegrad_{\Bu} } f(\Bu)}{\Bu = \Ba}
\end{equation}

Here a dummy variable \(\Bu\) has been retained as an instruction not to differentiate the \(\Bx\) part of the directional derivative in any repeated applications of the \(\Bx \cdot \spacegrad\) operator.

That notational cludge can be removed by swapping \(\Ba\) and \(\Bx\)

\begin{equation}\label{eqn:qmTwoL6:390b}
f(\Ba + \Bx) =
%\sum_{k=0}^\infty \inv{k!} \evalbar{(\Ba \cdot \spacegrad_{\Bu})^k f(\Bu)}{\Bu = \Bx}
\sum_{k=0}^\infty \inv{k!} (\Ba \cdot \spacegrad)^k f(\Bx)
=
e^{ \Ba \cdot \spacegrad } f(\Bx),
\end{equation}

where \(\spacegrad = \spacegrad_{\Bx} = (\PDi{x^1}{}, \PDi{x^2}{}, ...)\).

Having derived this (or for those with lesser degrees of amnesia, recall it), we can see that \eqnref{eqn:qmTwoL6:30} was a direct application of this, retaining no second order or higher terms.

Our expression used in the interaction Hamiltonian discussion was

\begin{equation}\label{eqn:qmTwoL6:170}
\inv{\Abs{\Br - \BR}} \approx \inv{\Abs{\Br}}
+
\BR \cdot \evalbar{\left(
\PD{\BR}{} \inv{\Abs{ \Br - \BR}}
\right)}{\BR = 0}.
\end{equation}

which we can see has the same structure as above with some variable substitutions.  Evaluating it we have

\begin{equation}\label{eqn:qmTwoL6:1090}
\begin{aligned}
\PD{\BR}{} \inv{\Abs{ \Br - \BR}}
&=
\Be_i \PD{R^i}{} ((x^j - R^j)^2)^{-1/2} \\
&=
\Be_i \left(-\inv{2}\right) 2 (x^j - R^j) \PD{R^i}{(x^j - R^j)} \inv{\Abs{\Br - \BR}^3} \\
&= \frac{\Br - \BR}{
\Abs{\Br - \BR}^3} ,
\end{aligned}
\end{equation}

and at \(\BR = 0\) we have

\begin{equation}\label{eqn:qmTwoL6:190}
\inv{\Abs{\Br - \BR}} \approx \inv{\Abs{\Br}}
+
\BR \cdot
\frac{\Br}{\Abs{\Br}^3}.
\end{equation}

We see in this direction derivative produces the classical electric Coulomb field expression for an electrostatic distribution, once we take the \(\Br/\Abs{\Br}^3\) and multiply it with the \(- Z e\) factor.

\paragraph{With algebra}

A different way to justify the expansion of \eqnref{eqn:qmTwoL6:30} is to consider a Clifford algebra factorization (following notation from \citep{doran2003gap}) of the absolute vector difference, where \(\BR\) is considered small.

\begin{equation}\label{eqn:qmTwoL6:1110}
\begin{aligned}
\Abs{\Br - \BR}
&= \sqrt{ \left(\Br - \BR\right) \left(\Br - \BR\right) } \\
&= \sqrt{ \gpgradezero{\Br \left(1 - \inv{\Br} \BR\right) \left(1 - \BR \inv{\Br}\right) \Br} } \\
&= \sqrt{ \gpgradezero{\Br^2 \left(1 - \inv{\Br} \BR\right) \left(1 - \BR \inv{\Br}\right) } } \\
&= \Abs{\Br} \sqrt{ 1 - 2 \inv{\Br} \cdot \BR + \gpgradezero{\inv{\Br} \BR \BR \inv{\Br}}} \\
&= \Abs{\Br} \sqrt{ 1 - 2 \inv{\Br} \cdot \BR + \frac{\BR^2}{\Br^2}}
\end{aligned}
\end{equation}

Neglecting the \(\BR^2\) term, we can then Taylor series expand this scalar expression

\begin{equation}\label{eqn:qmTwoL6:150}
\inv{\Abs{\Br - \BR}}
\approx
\inv{\Abs{\Br}} \left(
1 + \inv{\Br} \cdot \BR
\right)
=
\inv{\Abs{\Br}}
+ \frac{\rcap}{\Br^2} \cdot \BR
=
\inv{\Abs{\Br}}
+ \frac{\Br}{\Abs{\Br}^3} \cdot \BR.
\end{equation}

Observe this is what was found with the multivariable Taylor series expansion too.



%
% Copyright � 2012 Peeter Joot.  All Rights Reserved.
% Licenced as described in the file LICENSE under the root directory of this GIT repository.
%

% 
% 
%%
% Copyright � 2015 Peeter Joot.  All Rights Reserved.
% Licenced as described in the file LICENSE under the root directory of this GIT repository.
%
\documentclass[]{eliblog}

\usepackage{amsmath}
\usepackage{mathpazo}

%
% shorthand for bold symbols, convenient for vectors and matrices
%
\newcommand{\Ba}[0]{\mathbf{a}}
\newcommand{\Bb}[0]{\mathbf{b}}
\newcommand{\Bc}[0]{\mathbf{c}}
\newcommand{\Bd}[0]{\mathbf{d}}
\newcommand{\Be}[0]{\mathbf{e}}
\newcommand{\Bf}[0]{\mathbf{f}}
\newcommand{\Bg}[0]{\mathbf{g}}
\newcommand{\Bh}[0]{\mathbf{h}}
\newcommand{\Bi}[0]{\mathbf{i}}
\newcommand{\Bj}[0]{\mathbf{j}}
\newcommand{\Bk}[0]{\mathbf{k}}
\newcommand{\Bl}[0]{\mathbf{l}}
\newcommand{\Bm}[0]{\mathbf{m}}
\newcommand{\Bn}[0]{\mathbf{n}}
\newcommand{\Bo}[0]{\mathbf{o}}
\newcommand{\Bp}[0]{\mathbf{p}}
\newcommand{\Bq}[0]{\mathbf{q}}
\newcommand{\Br}[0]{\mathbf{r}}
\newcommand{\Bs}[0]{\mathbf{s}}
\newcommand{\Bt}[0]{\mathbf{t}}
\newcommand{\Bu}[0]{\mathbf{u}}
\newcommand{\Bv}[0]{\mathbf{v}}
\newcommand{\Bw}[0]{\mathbf{w}}
\newcommand{\Bx}[0]{\mathbf{x}}
\newcommand{\By}[0]{\mathbf{y}}
\newcommand{\Bz}[0]{\mathbf{z}}
\newcommand{\BA}[0]{\mathbf{A}}
\newcommand{\BB}[0]{\mathbf{B}}
\newcommand{\BC}[0]{\mathbf{C}}
\newcommand{\BD}[0]{\mathbf{D}}
\newcommand{\BE}[0]{\mathbf{E}}
\newcommand{\BF}[0]{\mathbf{F}}
\newcommand{\BG}[0]{\mathbf{G}}
\newcommand{\BH}[0]{\mathbf{H}}
\newcommand{\BI}[0]{\mathbf{I}}
\newcommand{\BJ}[0]{\mathbf{J}}
\newcommand{\BK}[0]{\mathbf{K}}
\newcommand{\BL}[0]{\mathbf{L}}
\newcommand{\BM}[0]{\mathbf{M}}
\newcommand{\BN}[0]{\mathbf{N}}
\newcommand{\BO}[0]{\mathbf{O}}
\newcommand{\BP}[0]{\mathbf{P}}
\newcommand{\BQ}[0]{\mathbf{Q}}
\newcommand{\BR}[0]{\mathbf{R}}
\newcommand{\BS}[0]{\mathbf{S}}
\newcommand{\BT}[0]{\mathbf{T}}
\newcommand{\BU}[0]{\mathbf{U}}
\newcommand{\BV}[0]{\mathbf{V}}
\newcommand{\BW}[0]{\mathbf{W}}
\newcommand{\BX}[0]{\mathbf{X}}
\newcommand{\BY}[0]{\mathbf{Y}}
\newcommand{\BZ}[0]{\mathbf{Z}}

\newcommand{\Bzero}[0]{\mathbf{0}}
\newcommand{\Btheta}[0]{\boldsymbol{\theta}}
\newcommand{\Btau}[0]{\boldsymbol{\tau}}
\newcommand{\Bomega}[0]{\boldsymbol{\omega}}

%
% shorthand for unit vectors
%
\newcommand{\acap}[0]{\hat{\Ba}}
\newcommand{\bcap}[0]{\hat{\Bb}}
\newcommand{\ccap}[0]{\hat{\Bc}}
\newcommand{\dcap}[0]{\hat{\Bd}}
\newcommand{\ecap}[0]{\hat{\Be}}
\newcommand{\fcap}[0]{\hat{\Bf}}
\newcommand{\gcap}[0]{\hat{\Bg}}
\newcommand{\hcap}[0]{\hat{\Bh}}
\newcommand{\icap}[0]{\hat{\Bi}}
\newcommand{\jcap}[0]{\hat{\Bj}}
\newcommand{\kcap}[0]{\hat{\Bk}}
\newcommand{\lcap}[0]{\hat{\Bl}}
\newcommand{\mcap}[0]{\hat{\Bm}}
\newcommand{\ncap}[0]{\hat{\Bn}}
\newcommand{\ocap}[0]{\hat{\Bo}}
\newcommand{\pcap}[0]{\hat{\Bp}}
\newcommand{\qcap}[0]{\hat{\Bq}}
\newcommand{\rcap}[0]{\hat{\Br}}
\newcommand{\scap}[0]{\hat{\Bs}}
\newcommand{\tcap}[0]{\hat{\Bt}}
\newcommand{\ucap}[0]{\hat{\Bu}}
\newcommand{\vcap}[0]{\hat{\Bv}}
\newcommand{\wcap}[0]{\hat{\Bw}}
\newcommand{\xcap}[0]{\hat{\Bx}}
\newcommand{\ycap}[0]{\hat{\By}}
\newcommand{\zcap}[0]{\hat{\Bz}}
\newcommand{\thetacap}[0]{\hat{\Btheta}}

%
% to write R^n and C^n in a distinguishable fashion.  Perhaps change this
% to the double lined characters upon figuring out how to do so.
%
\newcommand{\C}[1]{$\mathbb{C}^{#1}$}
\newcommand{\R}[1]{$\mathbb{R}^{#1}$}

%
% various generally useful helpers
%

% derivative of #1 wrt. #2:
\newcommand{\D}[2] {\frac {d#2} {d#1}}

\newcommand{\inv}[1]{\frac{1}{#1}}
\newcommand{\cross}[0]{\times}

\newcommand{\abs}[1]{\lvert{#1}\rvert}
\newcommand{\norm}[1]{\lVert{#1}\rVert}
\newcommand{\innerprod}[2]{\langle{#1}, {#2}\rangle}
\newcommand{\dotprod}[2]{{#1} \cdot {#2}}
\newcommand{\bdotprod}[2]{\left({#1} \cdot {#2}\right)}
\newcommand{\crossprod}[2]{{#1} \cross {#2}}
\newcommand{\tripleprod}[3]{\dotprod{\left(\crossprod{#1}{#2}\right)}{#3}}

\DeclareMathOperator{\Proj}{Proj}
\DeclareMathOperator{\Span}{span}
\DeclareMathOperator{\Sgn}{sgn}
\DeclareMathOperator{\Area}{Area}
\DeclareMathOperator{\Volume}{Volume}

%
% A few miscellaneous things specific to this document
%
\newcommand{\crossop}[1]{\crossprod{#1}{}}

% R2 vector.
\newcommand{\VectorTwo}[2]{
\begin{bmatrix}
 {#1} \\
 {#2}
\end{bmatrix}
}

\newcommand{\VectorN}[1]{
\begin{bmatrix}
{#1}_1 \\
{#1}_2 \\
\vdots \\
{#1}_N \\
\end{bmatrix}
}

\newcommand{\DETuvij}[4]{
\begin{vmatrix}
 {#1}_{#3} & {#1}_{#4} \\
 {#2}_{#3} & {#2}_{#4}
\end{vmatrix}
}

\newcommand{\DETuvwijk}[6]{
\begin{vmatrix}
 {#1}_{#4} & {#1}_{#5} & {#1}_{#6} \\
 {#2}_{#4} & {#2}_{#5} & {#2}_{#6} \\
 {#3}_{#4} & {#3}_{#5} & {#3}_{#6}
\end{vmatrix}
}

\newcommand{\DETuvwxijkl}[8]{
\begin{vmatrix}
 {#1}_{#5} & {#1}_{#6} & {#1}_{#7} & {#1}_{#8} \\
 {#2}_{#5} & {#2}_{#6} & {#2}_{#7} & {#2}_{#8} \\
 {#3}_{#5} & {#3}_{#6} & {#3}_{#7} & {#3}_{#8} \\
 {#4}_{#5} & {#4}_{#6} & {#4}_{#7} & {#4}_{#8} \\
\end{vmatrix}
}

%\newcommand{\DETuvwxyijklm}[10]{
%\begin{vmatrix}
% {#1}_{#6} & {#1}_{#7} & {#1}_{#8} & {#1}_{#9} & {#1}_{#10} \\
% {#2}_{#6} & {#2}_{#7} & {#2}_{#8} & {#2}_{#9} & {#2}_{#10} \\
% {#3}_{#6} & {#3}_{#7} & {#3}_{#8} & {#3}_{#9} & {#3}_{#10} \\
% {#4}_{#6} & {#4}_{#7} & {#4}_{#8} & {#4}_{#9} & {#4}_{#10} \\
% {#5}_{#6} & {#5}_{#7} & {#5}_{#8} & {#5}_{#9} & {#5}_{#10}
%\end{vmatrix}
%}

% R3 vector.
\newcommand{\VectorThree}[3]{
\begin{bmatrix}
 {#1} \\
 {#2} \\
 {#3}
\end{bmatrix}
}



\author{Peeter Joot}
\email{peeter.joot@gmail.com}

%\documentclass[]{eliblogwidescreen}

\usepackage{amsmath}
\usepackage{mathpazo}

%
% shorthand for bold symbols, convenient for vectors and matrices
%
\newcommand{\Ba}[0]{\mathbf{a}}
\newcommand{\Bb}[0]{\mathbf{b}}
\newcommand{\Bc}[0]{\mathbf{c}}
\newcommand{\Bd}[0]{\mathbf{d}}
\newcommand{\Be}[0]{\mathbf{e}}
\newcommand{\Bf}[0]{\mathbf{f}}
\newcommand{\Bg}[0]{\mathbf{g}}
\newcommand{\Bh}[0]{\mathbf{h}}
\newcommand{\Bi}[0]{\mathbf{i}}
\newcommand{\Bj}[0]{\mathbf{j}}
\newcommand{\Bk}[0]{\mathbf{k}}
\newcommand{\Bl}[0]{\mathbf{l}}
\newcommand{\Bm}[0]{\mathbf{m}}
\newcommand{\Bn}[0]{\mathbf{n}}
\newcommand{\Bo}[0]{\mathbf{o}}
\newcommand{\Bp}[0]{\mathbf{p}}
\newcommand{\Bq}[0]{\mathbf{q}}
\newcommand{\Br}[0]{\mathbf{r}}
\newcommand{\Bs}[0]{\mathbf{s}}
\newcommand{\Bt}[0]{\mathbf{t}}
\newcommand{\Bu}[0]{\mathbf{u}}
\newcommand{\Bv}[0]{\mathbf{v}}
\newcommand{\Bw}[0]{\mathbf{w}}
\newcommand{\Bx}[0]{\mathbf{x}}
\newcommand{\By}[0]{\mathbf{y}}
\newcommand{\Bz}[0]{\mathbf{z}}
\newcommand{\BA}[0]{\mathbf{A}}
\newcommand{\BB}[0]{\mathbf{B}}
\newcommand{\BC}[0]{\mathbf{C}}
\newcommand{\BD}[0]{\mathbf{D}}
\newcommand{\BE}[0]{\mathbf{E}}
\newcommand{\BF}[0]{\mathbf{F}}
\newcommand{\BG}[0]{\mathbf{G}}
\newcommand{\BH}[0]{\mathbf{H}}
\newcommand{\BI}[0]{\mathbf{I}}
\newcommand{\BJ}[0]{\mathbf{J}}
\newcommand{\BK}[0]{\mathbf{K}}
\newcommand{\BL}[0]{\mathbf{L}}
\newcommand{\BM}[0]{\mathbf{M}}
\newcommand{\BN}[0]{\mathbf{N}}
\newcommand{\BO}[0]{\mathbf{O}}
\newcommand{\BP}[0]{\mathbf{P}}
\newcommand{\BQ}[0]{\mathbf{Q}}
\newcommand{\BR}[0]{\mathbf{R}}
\newcommand{\BS}[0]{\mathbf{S}}
\newcommand{\BT}[0]{\mathbf{T}}
\newcommand{\BU}[0]{\mathbf{U}}
\newcommand{\BV}[0]{\mathbf{V}}
\newcommand{\BW}[0]{\mathbf{W}}
\newcommand{\BX}[0]{\mathbf{X}}
\newcommand{\BY}[0]{\mathbf{Y}}
\newcommand{\BZ}[0]{\mathbf{Z}}

\newcommand{\Bzero}[0]{\mathbf{0}}
\newcommand{\Btheta}[0]{\boldsymbol{\theta}}
\newcommand{\Btau}[0]{\boldsymbol{\tau}}
\newcommand{\Bomega}[0]{\boldsymbol{\omega}}

%
% shorthand for unit vectors
%
\newcommand{\acap}[0]{\hat{\Ba}}
\newcommand{\bcap}[0]{\hat{\Bb}}
\newcommand{\ccap}[0]{\hat{\Bc}}
\newcommand{\dcap}[0]{\hat{\Bd}}
\newcommand{\ecap}[0]{\hat{\Be}}
\newcommand{\fcap}[0]{\hat{\Bf}}
\newcommand{\gcap}[0]{\hat{\Bg}}
\newcommand{\hcap}[0]{\hat{\Bh}}
\newcommand{\icap}[0]{\hat{\Bi}}
\newcommand{\jcap}[0]{\hat{\Bj}}
\newcommand{\kcap}[0]{\hat{\Bk}}
\newcommand{\lcap}[0]{\hat{\Bl}}
\newcommand{\mcap}[0]{\hat{\Bm}}
\newcommand{\ncap}[0]{\hat{\Bn}}
\newcommand{\ocap}[0]{\hat{\Bo}}
\newcommand{\pcap}[0]{\hat{\Bp}}
\newcommand{\qcap}[0]{\hat{\Bq}}
\newcommand{\rcap}[0]{\hat{\Br}}
\newcommand{\scap}[0]{\hat{\Bs}}
\newcommand{\tcap}[0]{\hat{\Bt}}
\newcommand{\ucap}[0]{\hat{\Bu}}
\newcommand{\vcap}[0]{\hat{\Bv}}
\newcommand{\wcap}[0]{\hat{\Bw}}
\newcommand{\xcap}[0]{\hat{\Bx}}
\newcommand{\ycap}[0]{\hat{\By}}
\newcommand{\zcap}[0]{\hat{\Bz}}
\newcommand{\thetacap}[0]{\hat{\Btheta}}

%
% to write R^n and C^n in a distinguishable fashion.  Perhaps change this
% to the double lined characters upon figuring out how to do so.
%
\newcommand{\C}[1]{$\mathbb{C}^{#1}$}
\newcommand{\R}[1]{$\mathbb{R}^{#1}$}

%
% various generally useful helpers
%

% derivative of #1 wrt. #2:
\newcommand{\D}[2] {\frac {d#2} {d#1}}

\newcommand{\inv}[1]{\frac{1}{#1}}
\newcommand{\cross}[0]{\times}

\newcommand{\abs}[1]{\lvert{#1}\rvert}
\newcommand{\norm}[1]{\lVert{#1}\rVert}
\newcommand{\innerprod}[2]{\langle{#1}, {#2}\rangle}
\newcommand{\dotprod}[2]{{#1} \cdot {#2}}
\newcommand{\bdotprod}[2]{\left({#1} \cdot {#2}\right)}
\newcommand{\crossprod}[2]{{#1} \cross {#2}}
\newcommand{\tripleprod}[3]{\dotprod{\left(\crossprod{#1}{#2}\right)}{#3}}

\DeclareMathOperator{\Proj}{Proj}
\DeclareMathOperator{\Span}{span}
\DeclareMathOperator{\Sgn}{sgn}
\DeclareMathOperator{\Area}{Area}
\DeclareMathOperator{\Volume}{Volume}

%
% A few miscellaneous things specific to this document
%
\newcommand{\crossop}[1]{\crossprod{#1}{}}

% R2 vector.
\newcommand{\VectorTwo}[2]{
\begin{bmatrix}
 {#1} \\
 {#2}
\end{bmatrix}
}

\newcommand{\VectorN}[1]{
\begin{bmatrix}
{#1}_1 \\
{#1}_2 \\
\vdots \\
{#1}_N \\
\end{bmatrix}
}

\newcommand{\DETuvij}[4]{
\begin{vmatrix}
 {#1}_{#3} & {#1}_{#4} \\
 {#2}_{#3} & {#2}_{#4}
\end{vmatrix}
}

\newcommand{\DETuvwijk}[6]{
\begin{vmatrix}
 {#1}_{#4} & {#1}_{#5} & {#1}_{#6} \\
 {#2}_{#4} & {#2}_{#5} & {#2}_{#6} \\
 {#3}_{#4} & {#3}_{#5} & {#3}_{#6}
\end{vmatrix}
}

\newcommand{\DETuvwxijkl}[8]{
\begin{vmatrix}
 {#1}_{#5} & {#1}_{#6} & {#1}_{#7} & {#1}_{#8} \\
 {#2}_{#5} & {#2}_{#6} & {#2}_{#7} & {#2}_{#8} \\
 {#3}_{#5} & {#3}_{#6} & {#3}_{#7} & {#3}_{#8} \\
 {#4}_{#5} & {#4}_{#6} & {#4}_{#7} & {#4}_{#8} \\
\end{vmatrix}
}

%\newcommand{\DETuvwxyijklm}[10]{
%\begin{vmatrix}
% {#1}_{#6} & {#1}_{#7} & {#1}_{#8} & {#1}_{#9} & {#1}_{#10} \\
% {#2}_{#6} & {#2}_{#7} & {#2}_{#8} & {#2}_{#9} & {#2}_{#10} \\
% {#3}_{#6} & {#3}_{#7} & {#3}_{#8} & {#3}_{#9} & {#3}_{#10} \\
% {#4}_{#6} & {#4}_{#7} & {#4}_{#8} & {#4}_{#9} & {#4}_{#10} \\
% {#5}_{#6} & {#5}_{#7} & {#5}_{#8} & {#5}_{#9} & {#5}_{#10}
%\end{vmatrix}
%}

% R3 vector.
\newcommand{\VectorThree}[3]{
\begin{bmatrix}
 {#1} \\
 {#2} \\
 {#3}
\end{bmatrix}
}



\author{Peeter Joot}
\email{peeter.joot@gmail.com}


\chapter{Time dependent pertubation}
%\chapter{PHY456H1F: Quantum Mechanics II.  Lecture 7 (Taught by Prof J.E. Sipe).  Time dependent perturbation}
\label{chap:qmTwoL7}
\blogpage{http://sites.google.com/site/peeterjoot/math2011/qmTwoL7.pdf}
%\date{Sept 25, 2011}





%\section{Disclaimer.}
%
%Peeter's lecture notes from class.  May not be entirely coherent.

\section{Recap: Interaction picture}

We'll use the interaction picture to examine time dependent perturbations.  We wrote our Schr\"{o}dinger ket in terms of the interaction ket

\begin{equation}\label{eqn:qmTwoL7:10}
\ket{\psi}
= e^{-i H_0 (t - t_0)/\hbar}
\ket{\psi_I(t)},
\end{equation}

where
\begin{equation}\label{eqn:qmTwoL7:30}
\ket{\psi_I}
= U_I(t, t_0) \ket{\psi_I(t_0)}.
\end{equation}

Our dynamics is given by the operator equation
\begin{equation}\label{eqn:qmTwoL7:50}
i \hbar \ddt{} U_I(t, t_0) = \bar{H}'(t) U_I(t, t_0),
\end{equation}

where

\begin{equation}\label{eqn:qmTwoL7:70}
\bar{H}'(t) =
e^{\frac{i}{\hbar} H_0(t - t_0)} H'(t) e^{-\frac{i}{\hbar} H_0(t - t_0)}.
\end{equation}

We can formally solve \ref{eqn:qmTwoL7:50} by writing

\begin{equation}\label{eqn:qmTwoL7:90}
U_I(t, t_0) = I - \frac{i}{\hbar} \int_{t_0}^t dt' \bar{H}'(t') U_I(t', t_0).
\end{equation}

This is easy enough to verify by direct differentiation

\begin{align*}
i \hbar \ddt{} U_I
&=
\left(\int_{t_0}^t dt' \bar{H}'(t') U_I(t', t_0) \right)' \\
&=
\bar{H}'(t) U_I(t, t_0) \frac{dt}{dt}
-
\bar{H}'(t) U_I(t, t_0) \frac{dt_0}{dt} \\
&=
\bar{H}'(t) U_I(t, t_0)
\end{align*}

This is a bit of a chicken and an egg expression, since it is cyclic with a dependency on unknown $U_I(t', t_0)$ factors.

We start with an initial estimate of the operator to be determined, and iterate.  This can seem like an odd thing to do, but one can find books on just this integral kernel iteration method (like the nice little Dover book \cite{tricomi1985integral} that has sat on my (Peeter's) shelf all lonely so many years).

Suppose for $t$ near $t_0$, try

\begin{equation}\label{eqn:qmTwoL7:110}
U_I(t, t_0) \approx
I - \frac{i}{\hbar} \int_{t_0}^t dt' \bar{H}'(t').
\end{equation}

A second order iteration is now possible

\begin{equation}\label{eqn:qmTwoL7:130}
\begin{aligned}
U_I(t, t_0)
&\approx
I - \frac{i}{\hbar} \int_{t_0}^t dt' \bar{H}'(t') \left(
I - \frac{i}{\hbar} \int_{t_0}^{t'} dt'' \bar{H}'(t'').
\right) \\
&=
I - \frac{i}{\hbar} \int_{t_0}^t dt' \bar{H}'(t') + \left(\frac{-i}{\hbar}\right)^2
\int_{t_0}^t dt' \bar{H}'(t') \int_{t_0}^{t'} dt'' \bar{H}'(t'')
\end{aligned}
\end{equation}

It is possible to continue this iteration, and this approach is considered in some detail in \S 3.3 of the text \cite{desai2009quantum}, and is apparently also the basis for Feynman diagrams.

\section{Time dependent perturbation theory.}

As covered in \S 17 of the text, we'll split the interaction into time independent and time dependent terms

\begin{equation}\label{eqn:qmTwoL7:150}
H(t) = H_0 + H'(t),
\end{equation}

and work in the interaction picture with

\begin{equation}\label{eqn:qmTwoL7:170}
\ket{\psi_I(t)} = \sum_n \tilde{c}_n(t) \ket{\psi_n^{(0)} }.
\end{equation}

Our Schr\"{o}dinger ket is then

\begin{equation}\label{eqn:qmTwoL7:190}
\begin{aligned}
\ket{\psi(t)}
&=
e^{-i H_0^{(0)}(t- t_0)/\hbar}
\ket{\psi_I(t_0) } \\
&=
\sum_n \tilde{c}_n(t)
e^{-i E_n^{(0)}(t- t_0)/\hbar}
\ket{\psi_n^{(0)} }.
\end{aligned}
\end{equation}

With a definition

\begin{equation}\label{eqn:qmTwoL7:210}
c_n(t) = \tilde{c}_n(t) e^{i E_n t_0/\hbar},
\end{equation}

(where we leave off the zero superscript for the unperturbed state), our time evolved ket becomes

\begin{equation}\label{eqn:qmTwoL7:230}
\ket{\psi(t)}
=
\sum_n c_n(t)
e^{-i E_n t/\hbar}
\ket{\psi_n^{(0)} }.
\end{equation}

We can now plug \ref{eqn:qmTwoL7:170} into our evolution equation

\begin{align*}
i\hbar \ddt{} \ket{\psi_I(t)}
&=
\bar{H}'(t) \ket{\psi_I(t)} \\
&=
e^{\frac{i}{\hbar} H_0(t - t_0)} H'(t) e^{-\frac{i}{\hbar} H_0(t - t_0)}
\ket{\psi_I(t)},
\end{align*}

which gives us
%\ket{\psi_I(t)} = \sum_n
%\tilde{c}_n(t) \ket{\psi_n^{(0)} }.

\begin{equation}\label{eqn:qmTwoL7:250}
i \hbar \sum_p \PD{t}{}
\tilde{c}_p(t) \ket{\psi_p^{(0)} }
=
e^{\frac{i}{\hbar} H_0(t - t_0)} H'(t) e^{-\frac{i}{\hbar} H_0(t - t_0)}
\sum_n
\tilde{c}_n(t) \ket{\psi_n^{(0)} }.
\end{equation}

We can apply the bra $\bra{\psi_m^{(0)}}$ to this equation, yielding

\begin{equation}\label{eqn:qmTwoL7:270}
i \hbar \PD{t}{}
\tilde{c}_m(t)
=
\sum_n
\tilde{c}_n(t)
e^{\frac{i}{\hbar} E_m(t - t_0)}
\bra{\psi_m^{(0)}} H'(t)
\ket{\psi_n^{(0)} }
e^{-\frac{i}{\hbar} E_n(t - t_0)}.
\end{equation}

With

\begin{align}\label{eqn:qmTwoL7:290}
\omega_m &= \frac{E_m}{\hbar} \\
\omega_{mn} &= \omega_m - \omega_n \\
H_{mn}'(t) &= \bra{\psi_m^{(0)}} H'(t) \ket{\psi_n^{(0)} },
\end{align}

this is
\begin{equation}\label{eqn:qmTwoL7:310}
i \hbar \PD{t}{
\tilde{c}_m(t) }
=
\sum_n
\tilde{c}_n(t)
e^{
i \omega_{mn}(t - t_0)}
H_{mn}'(t)
\end{equation}

Inverting \ref{eqn:qmTwoL7:210} and plugging in

\begin{equation}\label{eqn:qmTwoL7:330}
\tilde{c}_n(t) = c_n(t) e^{-i \omega_n t_0},
\end{equation}

yields

\begin{equation}\label{eqn:qmTwoL7:350}
i \hbar \PD{t}{
c_m(t)
}
e^{-i \omega_m t_0}
=
\sum_n
c_n(t) e^{-i \omega_n t_0}
e^{i\omega_{mn}t}
e^{-i(\omega_m -\omega_n) t_0}
H_{mn}'(t),
\end{equation}

from which we can cancel the exponentials on both sides yielding
\begin{equation}\label{eqn:qmTwoL7:370}
i \hbar \PD{t}{
c_m(t)
}
=
\sum_n
c_n(t)
e^{i\omega_{mn}t}
H_{mn}'(t)
\end{equation}

We are now left with all of our time dependence nicely separated out, with the coefficients $c_n(t)$ encoding all the non-oscillatory time evolution information

\begin{align}\label{eqn:qmTwoL7:390}
H &= H_0 + H'(t) \\
\ket{\psi(t)} &= \sum_n c_n(t) e^{-i\omega_n t} \ket{\psi_n^{(0)}} \\
i \hbar \dot{c}_m &= \sum_n H_{mn}'(t) e^{i \omega_{mn} t} c_n(t)
\end{align}

\section{Perturbation expansion.}

We now introduce our $\lambda$ parametrization

\begin{equation}\label{eqn:qmTwoL7:410}
H'(t) \rightarrow \lambda H'(t),
\end{equation}

and hope for convergence, or at least something that at least has well defined asymptotic behavior.  We have

\begin{equation}\label{eqn:qmTwoL7:430}
i \hbar \dot{c}_m = \lambda \sum_n H_{mn}'(t) e^{i \omega_{mn} t} c_n(t),
\end{equation}

and try

\begin{equation}\label{eqn:qmTwoL7:450}
c_m(t) = c_m^{(0)}(t) + \lambda c_m^{(1)}(t) + \lambda^2 c_m^{(2)}(t) + \cdots
\end{equation}

Plugging in, we have

\begin{equation}\label{eqn:qmTwoL7:470}
i \hbar
\sum_k
\lambda^k \dot{c}_m^{(k)}(t)
=
\sum_{n,p} H_{mn}'(t) e^{i \omega_{mn} t}
\lambda^{p+1} c_n^{(p)}(t).
\end{equation}

As before, for equality, we treat this as an equation for each $\lambda^k$.  Expanding explicitly for the first few powers, gives us

\begin{align*}
0
&= \lambda^0 \left( i \hbar \dot{c}_m^{(0)}(t) - 0 \right) \\
&+ \lambda^1 \left( i \hbar \dot{c}_m^{(1)}(t) -
\sum_{n} H_{mn}'(t) e^{i \omega_{mn} t}
c_n^{(0)}(t)
\right) \\
&+ \lambda^2 \left( i \hbar \dot{c}_m^{(2)}(t) -
\sum_{n} H_{mn}'(t) e^{i \omega_{mn} t}
c_n^{(1)}(t)
\right) \\
&\vdots
\end{align*}

Suppose we have a set of energy levels as depicted in figure (\ref{fig:qmTwoL7:1})

\imageFigure{figures/qmTwoL7fig1}{Perturbation around energy level s}{fig:qmTwoL7:1}{0.4}

With $c_n^{(i)} = 0$ before the perturbation for all $i \ge 1, n$ and $c_m^{(0)} = \delta_{ms}$, we can proceed iteratively, solving each equation, starting with

\begin{equation}\label{eqn:qmTwoL7:490}
i \hbar \dot{c}_m^{(1)} = H_{ms}'(t) e^{i \omega_{ms} t}
\end{equation}

\subsection{Example: Slow nucleus passing an atom.}

\begin{equation}\label{eqn:qmTwoL7:510}
H'(t) = - \Bmu \cdot \BE(t)
\end{equation}

with
\begin{equation}\label{eqn:qmTwoL7:530}
H_{ms}' = -\Bmu_{ms} \cdot \BE(t),
\end{equation}

where
\begin{equation}\label{eqn:qmTwoL7:550}
\Bmu_{ms} =
\bra{\psi_m^{(0)}}
\Bmu
\ket{\psi_s^{(0)}}.
\end{equation}

Using our previous nucleus passing an atom example, as depicted in figure (\ref{fig:qmTwoL7:2})

\imageFigure{figures/qmTwoL7fig2}{Slow nucleus passing an atom}{fig:qmTwoL7:2}{0.4}

We have

\begin{equation}\label{eqn:qmTwoL7:570}
\Bmu = \sum_i q_i \BR_i,
\end{equation}

the dipole moment for each of the charges in the atom.  We will have fields as depicted in figure (\ref{fig:qmTwoL7:3})

\imageFigure{figures/qmTwoL7fig3}{Fields for nucleus atom example}{fig:qmTwoL7:3}{0.4}

FIXME: think through.

\subsection{Example: Electromagnetic wave pulse interacting with an atom.}

Consider a EM wave pulse, perhaps Gaussian, of the form depicted in figure (\ref{fig:qmTwoL7:4})

\imageFigure{figures/qmTwoL7fig4}{Atom interacting with an EM pulse}{fig:qmTwoL7:4}{0.4}

\begin{equation}\label{eqn:qmTwoL7:590}
E_y(t) = e^{-t^2/T^2} \cos(\omega_0 t).
\end{equation}

As we learned very early, perhaps sitting on our mother's knee, we can solve the differential equation \ref{eqn:qmTwoL7:490} for the first order perturbation, by direct integration

\begin{equation}\label{eqn:qmTwoL7:510b}
c_m^{(1)}(t) =
\inv{i \hbar} \int_{-\infty}^t
H_{ms}'(t') e^{i \omega_{ms} t'} dt'.
\end{equation}

Here the perturbation is assumed equal to zero at $-\infty$.  Suppose our electric field is specified in terms of a Fourier transform

\begin{equation}\label{eqn:qmTwoL7:530b}
\BE(t) = \int_{-\infty}^\infty \frac{d \omega}{2\pi} \BE(\omega) e^{-i \omega t},
\end{equation}

so

\begin{equation}\label{eqn:qmTwoL7:550b}
c_m^{(1)}(t) =
\frac{\Bmu_{ms}}{2 \pi i \hbar} \cdot
\int_{-\infty}^\infty
\int_{-\infty}^t
\BE(\omega)
e^{i (\omega_{ms} -\omega) t'} dt' d\omega.
\end{equation}

From this, ``after the perturbation'', as $t \rightarrow \infty$ we find

\begin{align*}
c_m^{(1)}(\infty)
&=
\frac{\Bmu_{ms}}{2 \pi i \hbar} \cdot
\int_{-\infty}^\infty
\int_{-\infty}^\infty
\BE(\omega)
e^{i (\omega_{ms} -\omega) t'} dt' d\omega \\
&=
\frac{\Bmu_{ms}}{i \hbar} \cdot
\int_{-\infty}^\infty
\BE(\omega)
\delta(\omega_{ms} - \omega)
d\omega
\end{align*}

since we identify

\begin{equation}\label{eqn:qmTwoL7:570b}
\inv{2 \pi}
\int_{-\infty}^\infty
e^{i (\omega_{ms} -\omega) t'} dt' \equiv \delta(\omega_{ms} - \omega)
\end{equation}

Thus the steady state first order perturbation coefficient is

\begin{equation}\label{eqn:qmTwoL7:590b}
c_m^{(1)}(\infty)
=
\frac{\Bmu_{ms}}{i \hbar} \cdot
\BE(\omega_{ms}).
\end{equation}

\subsubsection{Frequency symmetry for the Fourier spectrum of a real field.}

We will look further at this next week, but we first require an intermediate result from transform theory.  Because our field is real, we have

\begin{equation}\label{eqn:qmTwoL7:610}
\BE^\conj(t) = \BE(t)
\end{equation}

so

\begin{align*}
\BE^\conj(t)
&= \int \frac{d\omega}{2 \pi} \BE^\conj(\omega) e^{i \omega t} \\
&= \int \frac{d\omega}{2 \pi} \BE^\conj(-\omega) e^{-i \omega t} \\
\end{align*}

and thus

\begin{equation}\label{eqn:qmTwoL7:630}
\BE(\omega) = \BE^\conj(-\omega),
\end{equation}

and
\begin{equation}\label{eqn:qmTwoL7:650}
\Abs{\BE(\omega)}^2 = \Abs{\BE(-\omega)}^2.
\end{equation}

We will see shortly what the point of this aside is.



%
% Copyright � 2012 Peeter Joot.  All Rights Reserved.
% Licenced as described in the file LICENSE under the root directory of this GIT repository.
%

% 
% 
%%
% Copyright � 2015 Peeter Joot.  All Rights Reserved.
% Licenced as described in the file LICENSE under the root directory of this GIT repository.
%
\documentclass[]{eliblog}

\usepackage{amsmath}
\usepackage{mathpazo}

%
% shorthand for bold symbols, convenient for vectors and matrices
%
\newcommand{\Ba}[0]{\mathbf{a}}
\newcommand{\Bb}[0]{\mathbf{b}}
\newcommand{\Bc}[0]{\mathbf{c}}
\newcommand{\Bd}[0]{\mathbf{d}}
\newcommand{\Be}[0]{\mathbf{e}}
\newcommand{\Bf}[0]{\mathbf{f}}
\newcommand{\Bg}[0]{\mathbf{g}}
\newcommand{\Bh}[0]{\mathbf{h}}
\newcommand{\Bi}[0]{\mathbf{i}}
\newcommand{\Bj}[0]{\mathbf{j}}
\newcommand{\Bk}[0]{\mathbf{k}}
\newcommand{\Bl}[0]{\mathbf{l}}
\newcommand{\Bm}[0]{\mathbf{m}}
\newcommand{\Bn}[0]{\mathbf{n}}
\newcommand{\Bo}[0]{\mathbf{o}}
\newcommand{\Bp}[0]{\mathbf{p}}
\newcommand{\Bq}[0]{\mathbf{q}}
\newcommand{\Br}[0]{\mathbf{r}}
\newcommand{\Bs}[0]{\mathbf{s}}
\newcommand{\Bt}[0]{\mathbf{t}}
\newcommand{\Bu}[0]{\mathbf{u}}
\newcommand{\Bv}[0]{\mathbf{v}}
\newcommand{\Bw}[0]{\mathbf{w}}
\newcommand{\Bx}[0]{\mathbf{x}}
\newcommand{\By}[0]{\mathbf{y}}
\newcommand{\Bz}[0]{\mathbf{z}}
\newcommand{\BA}[0]{\mathbf{A}}
\newcommand{\BB}[0]{\mathbf{B}}
\newcommand{\BC}[0]{\mathbf{C}}
\newcommand{\BD}[0]{\mathbf{D}}
\newcommand{\BE}[0]{\mathbf{E}}
\newcommand{\BF}[0]{\mathbf{F}}
\newcommand{\BG}[0]{\mathbf{G}}
\newcommand{\BH}[0]{\mathbf{H}}
\newcommand{\BI}[0]{\mathbf{I}}
\newcommand{\BJ}[0]{\mathbf{J}}
\newcommand{\BK}[0]{\mathbf{K}}
\newcommand{\BL}[0]{\mathbf{L}}
\newcommand{\BM}[0]{\mathbf{M}}
\newcommand{\BN}[0]{\mathbf{N}}
\newcommand{\BO}[0]{\mathbf{O}}
\newcommand{\BP}[0]{\mathbf{P}}
\newcommand{\BQ}[0]{\mathbf{Q}}
\newcommand{\BR}[0]{\mathbf{R}}
\newcommand{\BS}[0]{\mathbf{S}}
\newcommand{\BT}[0]{\mathbf{T}}
\newcommand{\BU}[0]{\mathbf{U}}
\newcommand{\BV}[0]{\mathbf{V}}
\newcommand{\BW}[0]{\mathbf{W}}
\newcommand{\BX}[0]{\mathbf{X}}
\newcommand{\BY}[0]{\mathbf{Y}}
\newcommand{\BZ}[0]{\mathbf{Z}}

\newcommand{\Bzero}[0]{\mathbf{0}}
\newcommand{\Btheta}[0]{\boldsymbol{\theta}}
\newcommand{\Btau}[0]{\boldsymbol{\tau}}
\newcommand{\Bomega}[0]{\boldsymbol{\omega}}

%
% shorthand for unit vectors
%
\newcommand{\acap}[0]{\hat{\Ba}}
\newcommand{\bcap}[0]{\hat{\Bb}}
\newcommand{\ccap}[0]{\hat{\Bc}}
\newcommand{\dcap}[0]{\hat{\Bd}}
\newcommand{\ecap}[0]{\hat{\Be}}
\newcommand{\fcap}[0]{\hat{\Bf}}
\newcommand{\gcap}[0]{\hat{\Bg}}
\newcommand{\hcap}[0]{\hat{\Bh}}
\newcommand{\icap}[0]{\hat{\Bi}}
\newcommand{\jcap}[0]{\hat{\Bj}}
\newcommand{\kcap}[0]{\hat{\Bk}}
\newcommand{\lcap}[0]{\hat{\Bl}}
\newcommand{\mcap}[0]{\hat{\Bm}}
\newcommand{\ncap}[0]{\hat{\Bn}}
\newcommand{\ocap}[0]{\hat{\Bo}}
\newcommand{\pcap}[0]{\hat{\Bp}}
\newcommand{\qcap}[0]{\hat{\Bq}}
\newcommand{\rcap}[0]{\hat{\Br}}
\newcommand{\scap}[0]{\hat{\Bs}}
\newcommand{\tcap}[0]{\hat{\Bt}}
\newcommand{\ucap}[0]{\hat{\Bu}}
\newcommand{\vcap}[0]{\hat{\Bv}}
\newcommand{\wcap}[0]{\hat{\Bw}}
\newcommand{\xcap}[0]{\hat{\Bx}}
\newcommand{\ycap}[0]{\hat{\By}}
\newcommand{\zcap}[0]{\hat{\Bz}}
\newcommand{\thetacap}[0]{\hat{\Btheta}}

%
% to write R^n and C^n in a distinguishable fashion.  Perhaps change this
% to the double lined characters upon figuring out how to do so.
%
\newcommand{\C}[1]{$\mathbb{C}^{#1}$}
\newcommand{\R}[1]{$\mathbb{R}^{#1}$}

%
% various generally useful helpers
%

% derivative of #1 wrt. #2:
\newcommand{\D}[2] {\frac {d#2} {d#1}}

\newcommand{\inv}[1]{\frac{1}{#1}}
\newcommand{\cross}[0]{\times}

\newcommand{\abs}[1]{\lvert{#1}\rvert}
\newcommand{\norm}[1]{\lVert{#1}\rVert}
\newcommand{\innerprod}[2]{\langle{#1}, {#2}\rangle}
\newcommand{\dotprod}[2]{{#1} \cdot {#2}}
\newcommand{\bdotprod}[2]{\left({#1} \cdot {#2}\right)}
\newcommand{\crossprod}[2]{{#1} \cross {#2}}
\newcommand{\tripleprod}[3]{\dotprod{\left(\crossprod{#1}{#2}\right)}{#3}}

\DeclareMathOperator{\Proj}{Proj}
\DeclareMathOperator{\Span}{span}
\DeclareMathOperator{\Sgn}{sgn}
\DeclareMathOperator{\Area}{Area}
\DeclareMathOperator{\Volume}{Volume}

%
% A few miscellaneous things specific to this document
%
\newcommand{\crossop}[1]{\crossprod{#1}{}}

% R2 vector.
\newcommand{\VectorTwo}[2]{
\begin{bmatrix}
 {#1} \\
 {#2}
\end{bmatrix}
}

\newcommand{\VectorN}[1]{
\begin{bmatrix}
{#1}_1 \\
{#1}_2 \\
\vdots \\
{#1}_N \\
\end{bmatrix}
}

\newcommand{\DETuvij}[4]{
\begin{vmatrix}
 {#1}_{#3} & {#1}_{#4} \\
 {#2}_{#3} & {#2}_{#4}
\end{vmatrix}
}

\newcommand{\DETuvwijk}[6]{
\begin{vmatrix}
 {#1}_{#4} & {#1}_{#5} & {#1}_{#6} \\
 {#2}_{#4} & {#2}_{#5} & {#2}_{#6} \\
 {#3}_{#4} & {#3}_{#5} & {#3}_{#6}
\end{vmatrix}
}

\newcommand{\DETuvwxijkl}[8]{
\begin{vmatrix}
 {#1}_{#5} & {#1}_{#6} & {#1}_{#7} & {#1}_{#8} \\
 {#2}_{#5} & {#2}_{#6} & {#2}_{#7} & {#2}_{#8} \\
 {#3}_{#5} & {#3}_{#6} & {#3}_{#7} & {#3}_{#8} \\
 {#4}_{#5} & {#4}_{#6} & {#4}_{#7} & {#4}_{#8} \\
\end{vmatrix}
}

%\newcommand{\DETuvwxyijklm}[10]{
%\begin{vmatrix}
% {#1}_{#6} & {#1}_{#7} & {#1}_{#8} & {#1}_{#9} & {#1}_{#10} \\
% {#2}_{#6} & {#2}_{#7} & {#2}_{#8} & {#2}_{#9} & {#2}_{#10} \\
% {#3}_{#6} & {#3}_{#7} & {#3}_{#8} & {#3}_{#9} & {#3}_{#10} \\
% {#4}_{#6} & {#4}_{#7} & {#4}_{#8} & {#4}_{#9} & {#4}_{#10} \\
% {#5}_{#6} & {#5}_{#7} & {#5}_{#8} & {#5}_{#9} & {#5}_{#10}
%\end{vmatrix}
%}

% R3 vector.
\newcommand{\VectorThree}[3]{
\begin{bmatrix}
 {#1} \\
 {#2} \\
 {#3}
\end{bmatrix}
}



\author{Peeter Joot}
\email{peeter.joot@gmail.com}

%\documentclass[]{eliblogwidescreen}

\usepackage{amsmath}
\usepackage{mathpazo}

%
% shorthand for bold symbols, convenient for vectors and matrices
%
\newcommand{\Ba}[0]{\mathbf{a}}
\newcommand{\Bb}[0]{\mathbf{b}}
\newcommand{\Bc}[0]{\mathbf{c}}
\newcommand{\Bd}[0]{\mathbf{d}}
\newcommand{\Be}[0]{\mathbf{e}}
\newcommand{\Bf}[0]{\mathbf{f}}
\newcommand{\Bg}[0]{\mathbf{g}}
\newcommand{\Bh}[0]{\mathbf{h}}
\newcommand{\Bi}[0]{\mathbf{i}}
\newcommand{\Bj}[0]{\mathbf{j}}
\newcommand{\Bk}[0]{\mathbf{k}}
\newcommand{\Bl}[0]{\mathbf{l}}
\newcommand{\Bm}[0]{\mathbf{m}}
\newcommand{\Bn}[0]{\mathbf{n}}
\newcommand{\Bo}[0]{\mathbf{o}}
\newcommand{\Bp}[0]{\mathbf{p}}
\newcommand{\Bq}[0]{\mathbf{q}}
\newcommand{\Br}[0]{\mathbf{r}}
\newcommand{\Bs}[0]{\mathbf{s}}
\newcommand{\Bt}[0]{\mathbf{t}}
\newcommand{\Bu}[0]{\mathbf{u}}
\newcommand{\Bv}[0]{\mathbf{v}}
\newcommand{\Bw}[0]{\mathbf{w}}
\newcommand{\Bx}[0]{\mathbf{x}}
\newcommand{\By}[0]{\mathbf{y}}
\newcommand{\Bz}[0]{\mathbf{z}}
\newcommand{\BA}[0]{\mathbf{A}}
\newcommand{\BB}[0]{\mathbf{B}}
\newcommand{\BC}[0]{\mathbf{C}}
\newcommand{\BD}[0]{\mathbf{D}}
\newcommand{\BE}[0]{\mathbf{E}}
\newcommand{\BF}[0]{\mathbf{F}}
\newcommand{\BG}[0]{\mathbf{G}}
\newcommand{\BH}[0]{\mathbf{H}}
\newcommand{\BI}[0]{\mathbf{I}}
\newcommand{\BJ}[0]{\mathbf{J}}
\newcommand{\BK}[0]{\mathbf{K}}
\newcommand{\BL}[0]{\mathbf{L}}
\newcommand{\BM}[0]{\mathbf{M}}
\newcommand{\BN}[0]{\mathbf{N}}
\newcommand{\BO}[0]{\mathbf{O}}
\newcommand{\BP}[0]{\mathbf{P}}
\newcommand{\BQ}[0]{\mathbf{Q}}
\newcommand{\BR}[0]{\mathbf{R}}
\newcommand{\BS}[0]{\mathbf{S}}
\newcommand{\BT}[0]{\mathbf{T}}
\newcommand{\BU}[0]{\mathbf{U}}
\newcommand{\BV}[0]{\mathbf{V}}
\newcommand{\BW}[0]{\mathbf{W}}
\newcommand{\BX}[0]{\mathbf{X}}
\newcommand{\BY}[0]{\mathbf{Y}}
\newcommand{\BZ}[0]{\mathbf{Z}}

\newcommand{\Bzero}[0]{\mathbf{0}}
\newcommand{\Btheta}[0]{\boldsymbol{\theta}}
\newcommand{\Btau}[0]{\boldsymbol{\tau}}
\newcommand{\Bomega}[0]{\boldsymbol{\omega}}

%
% shorthand for unit vectors
%
\newcommand{\acap}[0]{\hat{\Ba}}
\newcommand{\bcap}[0]{\hat{\Bb}}
\newcommand{\ccap}[0]{\hat{\Bc}}
\newcommand{\dcap}[0]{\hat{\Bd}}
\newcommand{\ecap}[0]{\hat{\Be}}
\newcommand{\fcap}[0]{\hat{\Bf}}
\newcommand{\gcap}[0]{\hat{\Bg}}
\newcommand{\hcap}[0]{\hat{\Bh}}
\newcommand{\icap}[0]{\hat{\Bi}}
\newcommand{\jcap}[0]{\hat{\Bj}}
\newcommand{\kcap}[0]{\hat{\Bk}}
\newcommand{\lcap}[0]{\hat{\Bl}}
\newcommand{\mcap}[0]{\hat{\Bm}}
\newcommand{\ncap}[0]{\hat{\Bn}}
\newcommand{\ocap}[0]{\hat{\Bo}}
\newcommand{\pcap}[0]{\hat{\Bp}}
\newcommand{\qcap}[0]{\hat{\Bq}}
\newcommand{\rcap}[0]{\hat{\Br}}
\newcommand{\scap}[0]{\hat{\Bs}}
\newcommand{\tcap}[0]{\hat{\Bt}}
\newcommand{\ucap}[0]{\hat{\Bu}}
\newcommand{\vcap}[0]{\hat{\Bv}}
\newcommand{\wcap}[0]{\hat{\Bw}}
\newcommand{\xcap}[0]{\hat{\Bx}}
\newcommand{\ycap}[0]{\hat{\By}}
\newcommand{\zcap}[0]{\hat{\Bz}}
\newcommand{\thetacap}[0]{\hat{\Btheta}}

%
% to write R^n and C^n in a distinguishable fashion.  Perhaps change this
% to the double lined characters upon figuring out how to do so.
%
\newcommand{\C}[1]{$\mathbb{C}^{#1}$}
\newcommand{\R}[1]{$\mathbb{R}^{#1}$}

%
% various generally useful helpers
%

% derivative of #1 wrt. #2:
\newcommand{\D}[2] {\frac {d#2} {d#1}}

\newcommand{\inv}[1]{\frac{1}{#1}}
\newcommand{\cross}[0]{\times}

\newcommand{\abs}[1]{\lvert{#1}\rvert}
\newcommand{\norm}[1]{\lVert{#1}\rVert}
\newcommand{\innerprod}[2]{\langle{#1}, {#2}\rangle}
\newcommand{\dotprod}[2]{{#1} \cdot {#2}}
\newcommand{\bdotprod}[2]{\left({#1} \cdot {#2}\right)}
\newcommand{\crossprod}[2]{{#1} \cross {#2}}
\newcommand{\tripleprod}[3]{\dotprod{\left(\crossprod{#1}{#2}\right)}{#3}}

\DeclareMathOperator{\Proj}{Proj}
\DeclareMathOperator{\Span}{span}
\DeclareMathOperator{\Sgn}{sgn}
\DeclareMathOperator{\Area}{Area}
\DeclareMathOperator{\Volume}{Volume}

%
% A few miscellaneous things specific to this document
%
\newcommand{\crossop}[1]{\crossprod{#1}{}}

% R2 vector.
\newcommand{\VectorTwo}[2]{
\begin{bmatrix}
 {#1} \\
 {#2}
\end{bmatrix}
}

\newcommand{\VectorN}[1]{
\begin{bmatrix}
{#1}_1 \\
{#1}_2 \\
\vdots \\
{#1}_N \\
\end{bmatrix}
}

\newcommand{\DETuvij}[4]{
\begin{vmatrix}
 {#1}_{#3} & {#1}_{#4} \\
 {#2}_{#3} & {#2}_{#4}
\end{vmatrix}
}

\newcommand{\DETuvwijk}[6]{
\begin{vmatrix}
 {#1}_{#4} & {#1}_{#5} & {#1}_{#6} \\
 {#2}_{#4} & {#2}_{#5} & {#2}_{#6} \\
 {#3}_{#4} & {#3}_{#5} & {#3}_{#6}
\end{vmatrix}
}

\newcommand{\DETuvwxijkl}[8]{
\begin{vmatrix}
 {#1}_{#5} & {#1}_{#6} & {#1}_{#7} & {#1}_{#8} \\
 {#2}_{#5} & {#2}_{#6} & {#2}_{#7} & {#2}_{#8} \\
 {#3}_{#5} & {#3}_{#6} & {#3}_{#7} & {#3}_{#8} \\
 {#4}_{#5} & {#4}_{#6} & {#4}_{#7} & {#4}_{#8} \\
\end{vmatrix}
}

%\newcommand{\DETuvwxyijklm}[10]{
%\begin{vmatrix}
% {#1}_{#6} & {#1}_{#7} & {#1}_{#8} & {#1}_{#9} & {#1}_{#10} \\
% {#2}_{#6} & {#2}_{#7} & {#2}_{#8} & {#2}_{#9} & {#2}_{#10} \\
% {#3}_{#6} & {#3}_{#7} & {#3}_{#8} & {#3}_{#9} & {#3}_{#10} \\
% {#4}_{#6} & {#4}_{#7} & {#4}_{#8} & {#4}_{#9} & {#4}_{#10} \\
% {#5}_{#6} & {#5}_{#7} & {#5}_{#8} & {#5}_{#9} & {#5}_{#10}
%\end{vmatrix}
%}

% R3 vector.
\newcommand{\VectorThree}[3]{
\begin{bmatrix}
 {#1} \\
 {#2} \\
 {#3}
\end{bmatrix}
}



\author{Peeter Joot}
\email{peeter.joot@gmail.com}


%\usepackage{pstricks}
%\usepackage[off]{auto-pst-pdf}
% for:
%\input{qmTwoL8fig0FrequenciesAbsorbtionAndEmission}
% as generated by Inkscape.
%
% not sure how to make this work?

%\chapter{PHY456H1F: Quantum Mechanics II.  Lecture 8 (Taught by Prof J.E. Sipe).  Time dependent perturbation (cont.)}
\chapter{Time dependent perturbation (cont.)}
\label{chap:qmTwoL8}
%\useCCL
\blogpage{http://sites.google.com/site/peeterjoot/math2011/qmTwoL8.pdf}
\date{Oct 3, 2011}
\revisionInfo{qmTwoL8.tex}

\beginArtWithToc
%\beginArtNoToc

%\section{Disclaimer.}
%
%Peeter's lecture notes from class.  May not be entirely coherent.

\section{Time dependent perturbation.}

We'd gotten as far as calculating

\begin{equation}\label{eqn:qmTwoL8:10}
c_m^{(1)}(\infty) = \inv{i \hbar} \Bmu_{ms} \cdot \BE(\omega_{ms})
\end{equation}

where

\begin{equation}\label{eqn:qmTwoL8:30}
\BE(t) = \int \frac{d\omega}{2 \pi} \BE(\omega) e^{-i \omega t},
\end{equation}

and

\begin{equation}\label{eqn:qmTwoL8:50}
\omega_{ms} = \frac{E_m - E_s}{\hbar}.
\end{equation}

Graphically, these frequencies are illustrated in figure (\ref{fig:qmTwoL8fig0FrequenciesAbsorbtionAndEmission})

%\begin{figure}

\pdfTexFigure{figures/qmTwoL8fig0FrequenciesAbsorbtionAndEmission.pdf_tex}{Positive and negative frequencies.}{fig:qmTwoL8fig0FrequenciesAbsorbtionAndEmission}{0.3}
%\includegraphics[totalheight=0.3\textheight]{qmTwoL8fig0FrequenciesAbsorbtionAndEmission}

The probability for a transition from $m$ to $s$ is therefore

\begin{equation}\label{eqn:qmTwoL8:70}
\rho_{m \rightarrow s} = \Abs{ c_m^{(1)}(\infty) }^2
= \inv{\hbar}^2 \Abs{\Bmu_{ms} \cdot \BE(\omega_{ms})}^2
\end{equation}

Recall that because the electric field is real we had

\begin{equation}\label{eqn:qmTwoL8:90}
\Abs{\BE(\omega)}^2 = \Abs{\BE(-\omega)}^2.
\end{equation}

Suppose that we have a wave pulse, where our field magnitude is perhaps of the form

\begin{equation}\label{eqn:qmTwoL8:110}
E(t) = e^{-t^2/T^2} \cos(\omega_0 t),
\end{equation}

as illustrated with $\omega = 10, T = 1$ in figure (\ref{fig:qmTwoL8:gaussianWavePacket}).

\imageFigure{figures/gaussianWavePacket}{Gaussian wave packet}{fig:qmTwoL8:gaussianWavePacket}{0.2}

We expect this to have a two lobe Fourier spectrum, with the lobes centered at $\omega = \pm 10$, and width proportional to $1/T$.

For reference, as calculated using \href{https://github.com/peeterjoot/physicsplay/tree/master/notes/phy456/qmTwoL8figures.nb}{Mathematica} this Fourier transform is

\begin{equation}\label{eqn:qmTwoL8:130}
E(\omega) = \frac{e^{-\frac{1}{4} T^2 (\omega_0+\omega )^2}}{2 \sqrt{\frac{2}{T^2}}}+\frac{e^{\omega_0 T^2 \omega -\frac{1}{4} T^2 (\omega_0+\omega )^2}}{2 \sqrt{\frac{2}{T^2}}}
\end{equation}

This is illustrated, again for $\omega_0 = 10$, and $T=1$, in figure (\ref{fig:qmTwoL8:FTgaussianWavePacket})

\imageFigure{figures/FTgaussianWavePacket}{FTgaussianWavePacket}{fig:qmTwoL8:FTgaussianWavePacket}{0.2}

where we see the expected Gaussian result, since the Fourier transform of a Gaussian is a Gaussian.

FIXME: not sure what the point of this was?

\section{Sudden perturbations.}

Given our wave equation

\begin{equation}\label{eqn:qmTwoL8:150}
i \hbar \ddt{} \ket{\psi(t)} = H(t) \ket{\psi(t)}
\end{equation}

and a sudden perturbation in the Hamiltonian, as illustrated in figure (\ref{fig:qmTwoL8:suddenStepHamiltonian})

\imageFigure{figures/suddenStepHamiltonian}{Sudden step Hamiltonian.}{fig:qmTwoL8:suddenStepHamiltonian}{0.2}

Consider $H_0$ and $H_F$ fixed, and decrease $\Delta t \rightarrow 0$.  We can formally integrate \ref{eqn:qmTwoL8:150}

\begin{equation}\label{eqn:qmTwoL8:150b}
\ddt{} \ket{\psi(t)} = \inv{i \hbar} H(t) \ket{\psi(t)}
\end{equation}

For
\begin{equation}\label{eqn:qmTwoL8:150e}
\ket{\psi(t)} -\ket{\psi(t_0)}
 = \inv{i \hbar} \int_{t_0}^t H(t') \ket{\psi(t')} dt'.
\end{equation}

While this is an exact solution, it is also not terribly useful since we don't know $\ket{\psi(t)}$.  However, we can select the small interval $\Delta t$, and write

\begin{equation}\label{eqn:qmTwoL8:150c}
\ket{\psi(\Delta t/2)} =
\ket{\psi(-\Delta t/2)}
+ \inv{i \hbar} \int_{t_0}^t H(t') \ket{\psi(t')} dt'.
\end{equation}

Note that we could use the integral kernel iteration technique here and substitute $\ket{\psi(t')} = \ket{\psi(-\Delta t/2)}$ and then develop this, to generate a power series with $(\Delta t/2)^k$ dependence.  However, we note that \ref{eqn:qmTwoL8:150c} is still an exact relation, and if $\Delta t \rightarrow 0$, with the integration limits narrowing (provided $H(t')$ is well behaved) we are left with just

\begin{equation}\label{eqn:qmTwoL8:180}
\ket{\psi(\Delta t/2)} = \ket{\psi(-\Delta t/2)}
\end{equation}

Or
\begin{equation}\label{eqn:qmTwoL8:200}
\ket{\psi_{\text{after}}} = \ket{\psi_{\text{before}}},
\end{equation}

provided that we change the Hamiltonian fast enough.  On the surface there appears to be no consequences, but there are some very serious ones!

\subsection{Example: Harmonic oscillator.}

Consider our harmonic oscillator Hamiltonian, with

\begin{align}\label{eqn:qmTwoL8:220}
H_0 &= \frac{P^2}{2m} + \inv{2} m \omega_0^2 X^2 \\
H_F &= \frac{P^2}{2m} + \inv{2} m \omega_F^2 X^2
\end{align}

Here $\omega_0 \rightarrow \omega_F$ continuously, but \underline{very quickly}.  In effect, we have tightened the spring constant.  Note that there are cases in linear optics when you can actually do exactly that.

Imagine that $\ket{\psi_{\text{before}}}$ is in the ground state of the harmonic oscillator as in figure (\ref{fig:qmTwoL8:suddenHamiltonianPertubationHO})

\imageFigure{figures/suddenHamiltonianPertubationHO}{Harmonic oscillator sudden Hamiltonian perturbation.}{fig:qmTwoL8:suddenHamiltonianPertubationHO}{0.2}

and we suddenly change the Hamiltonian with potential $V_0 \rightarrow V_F$ (weakening the ``spring'').  Professor Sipe gives us a graphical demo of this, by impersonating a constrained wavefunction with his arms, doing weak chicken-flapping of them.  Now with the potential weakened, he wiggles and flaps his arms with more freedom and somewhat chaotically.  His ``wave function'' arms are now bouncing around in the new limiting potential (initially over doing it and then bouncing back).

We had in this case the exact relation

\begin{equation}\label{eqn:qmTwoL8:240}
H_0 \ket{\psi_0^{(0)}} = \inv{2} \hbar \omega_0 \ket{\psi_0^{(0)}}
\end{equation}

but we also have
\begin{equation}\label{eqn:qmTwoL8:260}
\ket{\psi_{\text{after}}} = \ket{\psi_{\text{before}}} = \ket{\psi_0^{(0)}}
\end{equation}

and
\begin{equation}\label{eqn:qmTwoL8:280}
H_F \ket{\psi_n^{(f)}} = \inv{2} \hbar \omega_F \left( n + \inv{2} \right) \ket{\psi_n^{(f)}}
\end{equation}

So
\begin{align*}
\ket{\psi_{\text{after}}}
&=
\ket{\psi_0^{(0)}} \\
&=
\sum_n \ket{\psi_n^{(f)}}
\underbrace{\braket{\psi_n^{(f)}}{\psi_0^{(0)}} }_{c_n} \\
&=
\sum_n c_n \ket{\psi_n^{(f)}}
\end{align*}

and at later times

\begin{align*}
\ket{\psi(t)^{(f)}}
&=
\ket{\psi_0^{(0)}} \\
&=
\sum_n c_n e^{i \omega_n^{(f)} t} \ket{\psi_n^{(f)}},
\end{align*}

whereas

\begin{align*}
\ket{\psi(t)^{(o)}}
&=
e^{i \omega_0^{(0)} t} \ket{\psi_0^{(0)}},
\end{align*}

So, while the wave functions may be exactly the same after such a sudden change in Hamiltonian, the dynamics of the situation change for all future times, since we now have a wavefunction that has a different set of components in the basis for the new Hamiltonian.  In particular, the evolution of the wave function is now significantly more complex.

FIXME: plot an example of this.

\section{Adiabatic perturbations.}

This is treated in \S 17.5.2 of the text \cite{desai2009quantum}.

I wondered what Adiabatic meant in this context.  The usage in class sounds like it was just ``really slow and gradual'', yet this has a definition ``\href{http://www.thefreedictionary.com/adiabatic}{Of, relating to, or being a reversible thermodynamic process that occurs without gain or loss of heat and without a change in entropy}''.  Wikipedia \cite{wiki:AdiabaticTheorem} appears to confirm that the QM meaning of this term is just ``slow'' changing.

This is the reverse case, and we now vary the Hamiltonian $H(t)$ \underline{very slowly}.

\begin{equation}\label{eqn:qmTwoL8:150f}
\ddt{} \ket{\psi(t)} = \inv{i \hbar} H(t) \ket{\psi(t)}
\end{equation}

We first consider only non-degenerate states, and at $t = 0$ write

\begin{equation}\label{eqn:qmTwoL8:300}
H(0) = H_0,
\end{equation}

and
\begin{equation}\label{eqn:qmTwoL8:320}
H_0 \ket{\psi_s^{(0)}} = E_s^{(0)} \ket{\psi_s^{(0)}}
\end{equation}

Imagine that at each time $t$ we can find the ``instantaneous'' energy eigenstates

\begin{equation}\label{eqn:qmTwoL8:340}
H(t) \ket{\hat{\psi}_s(t)} = E_s(t) \ket{\hat{\psi}_s(t)} 
\end{equation}

These states do not satisfy Schr\"{o}dinger's equation, but are simply solutions to the eigen problem.  Our standard strategy in perturbation is based on analysis of

\begin{equation}\label{eqn:qmTwoL8:360}
\ket{\psi(t)} = \sum_n c_n(t) e^{- i \omega_n^{(0)} t} \ket{\psi_n^{(0)} },
\end{equation}

Here instead

\begin{equation}\label{eqn:qmTwoL8:380}
\ket{\psi(t)} = 
\sum_n b_n(t) \ket{\hat{\psi}_n(t)}
,
\end{equation}

we will expand, not using our initial basis, but instead using the instantaneous kets.  Plugging into Schr\"{o}dinger's equation we have

\begin{align*}
H(t) \ket{\psi(t)} 
&= H(t) \sum_n b_n(t) \ket{\hat{\psi}_n(t)} \\
&= \sum_n b_n(t) E_n(t) \ket{\hat{\psi}_n(t)} 
\end{align*}

This was complicated before with matrix elements all over the place.  Now it is easy, however, the time derivative becomes harder.  Doing that we find

\begin{align*}
i \hbar \ddt{} \ket{\psi(t)}
&=
i \hbar
\ddt{} \sum_n b_n(t) \ket{\hat{\psi}_n(t)} \\
&=
i \hbar
\sum_n 
\ddt{b_n(t)} \ket{\hat{\psi}_n(t)} 
+ \sum_n b_n(t) \ddt{} \ket{\hat{\psi}_n(t)} \\
&= \sum_n b_n(t) E_n(t) \ket{\hat{\psi}_n(t)} 
\end{align*}

We bra $\bra{\hat{\psi}_m(t)}$ into this

\begin{equation}\label{eqn:qmTwoL8:400}
i \hbar
\sum_n 
\ddt{b_n(t)} 
\braket{\hat{\psi}_m(t)}{\hat{\psi}_n(t)}
+ \sum_n b_n(t) 
\bra{\hat{\psi}_m(t)}
\ddt{} \ket{\hat{\psi}_n(t)} 
= \sum_n b_n(t) E_n(t) \braket{\hat{\psi}_m(t)}{\hat{\psi}_n(t)} ,
\end{equation}

and find

\begin{equation}\label{eqn:qmTwoL8:410}
i \hbar
\ddt{b_m(t)} 
+ 
\sum_n b_n(t) 
\bra{\hat{\psi}_m(t)}
\ddt{} \ket{\hat{\psi}_n(t)} 
= b_m(t) E_m(t) 
\end{equation}

If the Hamiltonian is changed \underline{very very} slowly in time, we can imagine that $\ket{\hat{\psi}_n(t)}'$ is also changing very very slowly, but we are not quite there yet.  Let's first split our sum of bra and ket products 

\begin{equation}\label{eqn:qmTwoL8:430}
\sum_n b_n(t) 
\bra{\hat{\psi}_m(t)}
\ddt{} \ket{\hat{\psi}_n(t)} 
\end{equation}

into $n \ne m$ and $n = m$ terms.  Looking at just the $n = m$ term 

\begin{equation}\label{eqn:qmTwoL8:450}
\bra{\hat{\psi}_m(t)}
\ddt{} \ket{\hat{\psi}_m(t)} 
\end{equation}

we note

\begin{align*}
0 
&=
\ddt{} \braket{\hat{\psi}_m(t)}{\hat{\psi}_m(t)} \\
&=
\left( \ddt{} \bra{\hat{\psi}_m(t)} \right) \ket{\hat{\psi}_m(t)} 
+ \bra{\hat{\psi}_m(t)} \ddt{} \ket{\hat{\psi}_m(t)} \\
\end{align*}

Something plus its complex conjugate equals 0

\begin{equation}\label{eqn:qmTwoL8:470}
a + i b + (a + i b)^\conj = 2 a = 0 \implies a = 0,
\end{equation}

so $\bra{\hat{\psi}_m(t)} \ddt{} \ket{\hat{\psi}_m(t)}$ must be purely imaginary.  We write

\begin{equation}\label{eqn:qmTwoL8:490}
\bra{\hat{\psi}_m(t)} \ddt{} \ket{\hat{\psi}_m(t)} = -i \Gamma_s(t),
\end{equation}

where $\Gamma_s$ is real.


\EndArticle

%
% Copyright � 2012 Peeter Joot.  All Rights Reserved.
% Licenced as described in the file LICENSE under the root directory of this GIT repository.
%

% 
% 
%%
% Copyright � 2015 Peeter Joot.  All Rights Reserved.
% Licenced as described in the file LICENSE under the root directory of this GIT repository.
%
\documentclass[]{eliblog}

\usepackage{amsmath}
\usepackage{mathpazo}

%
% shorthand for bold symbols, convenient for vectors and matrices
%
\newcommand{\Ba}[0]{\mathbf{a}}
\newcommand{\Bb}[0]{\mathbf{b}}
\newcommand{\Bc}[0]{\mathbf{c}}
\newcommand{\Bd}[0]{\mathbf{d}}
\newcommand{\Be}[0]{\mathbf{e}}
\newcommand{\Bf}[0]{\mathbf{f}}
\newcommand{\Bg}[0]{\mathbf{g}}
\newcommand{\Bh}[0]{\mathbf{h}}
\newcommand{\Bi}[0]{\mathbf{i}}
\newcommand{\Bj}[0]{\mathbf{j}}
\newcommand{\Bk}[0]{\mathbf{k}}
\newcommand{\Bl}[0]{\mathbf{l}}
\newcommand{\Bm}[0]{\mathbf{m}}
\newcommand{\Bn}[0]{\mathbf{n}}
\newcommand{\Bo}[0]{\mathbf{o}}
\newcommand{\Bp}[0]{\mathbf{p}}
\newcommand{\Bq}[0]{\mathbf{q}}
\newcommand{\Br}[0]{\mathbf{r}}
\newcommand{\Bs}[0]{\mathbf{s}}
\newcommand{\Bt}[0]{\mathbf{t}}
\newcommand{\Bu}[0]{\mathbf{u}}
\newcommand{\Bv}[0]{\mathbf{v}}
\newcommand{\Bw}[0]{\mathbf{w}}
\newcommand{\Bx}[0]{\mathbf{x}}
\newcommand{\By}[0]{\mathbf{y}}
\newcommand{\Bz}[0]{\mathbf{z}}
\newcommand{\BA}[0]{\mathbf{A}}
\newcommand{\BB}[0]{\mathbf{B}}
\newcommand{\BC}[0]{\mathbf{C}}
\newcommand{\BD}[0]{\mathbf{D}}
\newcommand{\BE}[0]{\mathbf{E}}
\newcommand{\BF}[0]{\mathbf{F}}
\newcommand{\BG}[0]{\mathbf{G}}
\newcommand{\BH}[0]{\mathbf{H}}
\newcommand{\BI}[0]{\mathbf{I}}
\newcommand{\BJ}[0]{\mathbf{J}}
\newcommand{\BK}[0]{\mathbf{K}}
\newcommand{\BL}[0]{\mathbf{L}}
\newcommand{\BM}[0]{\mathbf{M}}
\newcommand{\BN}[0]{\mathbf{N}}
\newcommand{\BO}[0]{\mathbf{O}}
\newcommand{\BP}[0]{\mathbf{P}}
\newcommand{\BQ}[0]{\mathbf{Q}}
\newcommand{\BR}[0]{\mathbf{R}}
\newcommand{\BS}[0]{\mathbf{S}}
\newcommand{\BT}[0]{\mathbf{T}}
\newcommand{\BU}[0]{\mathbf{U}}
\newcommand{\BV}[0]{\mathbf{V}}
\newcommand{\BW}[0]{\mathbf{W}}
\newcommand{\BX}[0]{\mathbf{X}}
\newcommand{\BY}[0]{\mathbf{Y}}
\newcommand{\BZ}[0]{\mathbf{Z}}

\newcommand{\Bzero}[0]{\mathbf{0}}
\newcommand{\Btheta}[0]{\boldsymbol{\theta}}
\newcommand{\Btau}[0]{\boldsymbol{\tau}}
\newcommand{\Bomega}[0]{\boldsymbol{\omega}}

%
% shorthand for unit vectors
%
\newcommand{\acap}[0]{\hat{\Ba}}
\newcommand{\bcap}[0]{\hat{\Bb}}
\newcommand{\ccap}[0]{\hat{\Bc}}
\newcommand{\dcap}[0]{\hat{\Bd}}
\newcommand{\ecap}[0]{\hat{\Be}}
\newcommand{\fcap}[0]{\hat{\Bf}}
\newcommand{\gcap}[0]{\hat{\Bg}}
\newcommand{\hcap}[0]{\hat{\Bh}}
\newcommand{\icap}[0]{\hat{\Bi}}
\newcommand{\jcap}[0]{\hat{\Bj}}
\newcommand{\kcap}[0]{\hat{\Bk}}
\newcommand{\lcap}[0]{\hat{\Bl}}
\newcommand{\mcap}[0]{\hat{\Bm}}
\newcommand{\ncap}[0]{\hat{\Bn}}
\newcommand{\ocap}[0]{\hat{\Bo}}
\newcommand{\pcap}[0]{\hat{\Bp}}
\newcommand{\qcap}[0]{\hat{\Bq}}
\newcommand{\rcap}[0]{\hat{\Br}}
\newcommand{\scap}[0]{\hat{\Bs}}
\newcommand{\tcap}[0]{\hat{\Bt}}
\newcommand{\ucap}[0]{\hat{\Bu}}
\newcommand{\vcap}[0]{\hat{\Bv}}
\newcommand{\wcap}[0]{\hat{\Bw}}
\newcommand{\xcap}[0]{\hat{\Bx}}
\newcommand{\ycap}[0]{\hat{\By}}
\newcommand{\zcap}[0]{\hat{\Bz}}
\newcommand{\thetacap}[0]{\hat{\Btheta}}

%
% to write R^n and C^n in a distinguishable fashion.  Perhaps change this
% to the double lined characters upon figuring out how to do so.
%
\newcommand{\C}[1]{$\mathbb{C}^{#1}$}
\newcommand{\R}[1]{$\mathbb{R}^{#1}$}

%
% various generally useful helpers
%

% derivative of #1 wrt. #2:
\newcommand{\D}[2] {\frac {d#2} {d#1}}

\newcommand{\inv}[1]{\frac{1}{#1}}
\newcommand{\cross}[0]{\times}

\newcommand{\abs}[1]{\lvert{#1}\rvert}
\newcommand{\norm}[1]{\lVert{#1}\rVert}
\newcommand{\innerprod}[2]{\langle{#1}, {#2}\rangle}
\newcommand{\dotprod}[2]{{#1} \cdot {#2}}
\newcommand{\bdotprod}[2]{\left({#1} \cdot {#2}\right)}
\newcommand{\crossprod}[2]{{#1} \cross {#2}}
\newcommand{\tripleprod}[3]{\dotprod{\left(\crossprod{#1}{#2}\right)}{#3}}

\DeclareMathOperator{\Proj}{Proj}
\DeclareMathOperator{\Span}{span}
\DeclareMathOperator{\Sgn}{sgn}
\DeclareMathOperator{\Area}{Area}
\DeclareMathOperator{\Volume}{Volume}

%
% A few miscellaneous things specific to this document
%
\newcommand{\crossop}[1]{\crossprod{#1}{}}

% R2 vector.
\newcommand{\VectorTwo}[2]{
\begin{bmatrix}
 {#1} \\
 {#2}
\end{bmatrix}
}

\newcommand{\VectorN}[1]{
\begin{bmatrix}
{#1}_1 \\
{#1}_2 \\
\vdots \\
{#1}_N \\
\end{bmatrix}
}

\newcommand{\DETuvij}[4]{
\begin{vmatrix}
 {#1}_{#3} & {#1}_{#4} \\
 {#2}_{#3} & {#2}_{#4}
\end{vmatrix}
}

\newcommand{\DETuvwijk}[6]{
\begin{vmatrix}
 {#1}_{#4} & {#1}_{#5} & {#1}_{#6} \\
 {#2}_{#4} & {#2}_{#5} & {#2}_{#6} \\
 {#3}_{#4} & {#3}_{#5} & {#3}_{#6}
\end{vmatrix}
}

\newcommand{\DETuvwxijkl}[8]{
\begin{vmatrix}
 {#1}_{#5} & {#1}_{#6} & {#1}_{#7} & {#1}_{#8} \\
 {#2}_{#5} & {#2}_{#6} & {#2}_{#7} & {#2}_{#8} \\
 {#3}_{#5} & {#3}_{#6} & {#3}_{#7} & {#3}_{#8} \\
 {#4}_{#5} & {#4}_{#6} & {#4}_{#7} & {#4}_{#8} \\
\end{vmatrix}
}

%\newcommand{\DETuvwxyijklm}[10]{
%\begin{vmatrix}
% {#1}_{#6} & {#1}_{#7} & {#1}_{#8} & {#1}_{#9} & {#1}_{#10} \\
% {#2}_{#6} & {#2}_{#7} & {#2}_{#8} & {#2}_{#9} & {#2}_{#10} \\
% {#3}_{#6} & {#3}_{#7} & {#3}_{#8} & {#3}_{#9} & {#3}_{#10} \\
% {#4}_{#6} & {#4}_{#7} & {#4}_{#8} & {#4}_{#9} & {#4}_{#10} \\
% {#5}_{#6} & {#5}_{#7} & {#5}_{#8} & {#5}_{#9} & {#5}_{#10}
%\end{vmatrix}
%}

% R3 vector.
\newcommand{\VectorThree}[3]{
\begin{bmatrix}
 {#1} \\
 {#2} \\
 {#3}
\end{bmatrix}
}



\author{Peeter Joot}
\email{peeter.joot@gmail.com}

%\documentclass[]{eliblogwidescreen}

\usepackage{amsmath}
\usepackage{mathpazo}

%
% shorthand for bold symbols, convenient for vectors and matrices
%
\newcommand{\Ba}[0]{\mathbf{a}}
\newcommand{\Bb}[0]{\mathbf{b}}
\newcommand{\Bc}[0]{\mathbf{c}}
\newcommand{\Bd}[0]{\mathbf{d}}
\newcommand{\Be}[0]{\mathbf{e}}
\newcommand{\Bf}[0]{\mathbf{f}}
\newcommand{\Bg}[0]{\mathbf{g}}
\newcommand{\Bh}[0]{\mathbf{h}}
\newcommand{\Bi}[0]{\mathbf{i}}
\newcommand{\Bj}[0]{\mathbf{j}}
\newcommand{\Bk}[0]{\mathbf{k}}
\newcommand{\Bl}[0]{\mathbf{l}}
\newcommand{\Bm}[0]{\mathbf{m}}
\newcommand{\Bn}[0]{\mathbf{n}}
\newcommand{\Bo}[0]{\mathbf{o}}
\newcommand{\Bp}[0]{\mathbf{p}}
\newcommand{\Bq}[0]{\mathbf{q}}
\newcommand{\Br}[0]{\mathbf{r}}
\newcommand{\Bs}[0]{\mathbf{s}}
\newcommand{\Bt}[0]{\mathbf{t}}
\newcommand{\Bu}[0]{\mathbf{u}}
\newcommand{\Bv}[0]{\mathbf{v}}
\newcommand{\Bw}[0]{\mathbf{w}}
\newcommand{\Bx}[0]{\mathbf{x}}
\newcommand{\By}[0]{\mathbf{y}}
\newcommand{\Bz}[0]{\mathbf{z}}
\newcommand{\BA}[0]{\mathbf{A}}
\newcommand{\BB}[0]{\mathbf{B}}
\newcommand{\BC}[0]{\mathbf{C}}
\newcommand{\BD}[0]{\mathbf{D}}
\newcommand{\BE}[0]{\mathbf{E}}
\newcommand{\BF}[0]{\mathbf{F}}
\newcommand{\BG}[0]{\mathbf{G}}
\newcommand{\BH}[0]{\mathbf{H}}
\newcommand{\BI}[0]{\mathbf{I}}
\newcommand{\BJ}[0]{\mathbf{J}}
\newcommand{\BK}[0]{\mathbf{K}}
\newcommand{\BL}[0]{\mathbf{L}}
\newcommand{\BM}[0]{\mathbf{M}}
\newcommand{\BN}[0]{\mathbf{N}}
\newcommand{\BO}[0]{\mathbf{O}}
\newcommand{\BP}[0]{\mathbf{P}}
\newcommand{\BQ}[0]{\mathbf{Q}}
\newcommand{\BR}[0]{\mathbf{R}}
\newcommand{\BS}[0]{\mathbf{S}}
\newcommand{\BT}[0]{\mathbf{T}}
\newcommand{\BU}[0]{\mathbf{U}}
\newcommand{\BV}[0]{\mathbf{V}}
\newcommand{\BW}[0]{\mathbf{W}}
\newcommand{\BX}[0]{\mathbf{X}}
\newcommand{\BY}[0]{\mathbf{Y}}
\newcommand{\BZ}[0]{\mathbf{Z}}

\newcommand{\Bzero}[0]{\mathbf{0}}
\newcommand{\Btheta}[0]{\boldsymbol{\theta}}
\newcommand{\Btau}[0]{\boldsymbol{\tau}}
\newcommand{\Bomega}[0]{\boldsymbol{\omega}}

%
% shorthand for unit vectors
%
\newcommand{\acap}[0]{\hat{\Ba}}
\newcommand{\bcap}[0]{\hat{\Bb}}
\newcommand{\ccap}[0]{\hat{\Bc}}
\newcommand{\dcap}[0]{\hat{\Bd}}
\newcommand{\ecap}[0]{\hat{\Be}}
\newcommand{\fcap}[0]{\hat{\Bf}}
\newcommand{\gcap}[0]{\hat{\Bg}}
\newcommand{\hcap}[0]{\hat{\Bh}}
\newcommand{\icap}[0]{\hat{\Bi}}
\newcommand{\jcap}[0]{\hat{\Bj}}
\newcommand{\kcap}[0]{\hat{\Bk}}
\newcommand{\lcap}[0]{\hat{\Bl}}
\newcommand{\mcap}[0]{\hat{\Bm}}
\newcommand{\ncap}[0]{\hat{\Bn}}
\newcommand{\ocap}[0]{\hat{\Bo}}
\newcommand{\pcap}[0]{\hat{\Bp}}
\newcommand{\qcap}[0]{\hat{\Bq}}
\newcommand{\rcap}[0]{\hat{\Br}}
\newcommand{\scap}[0]{\hat{\Bs}}
\newcommand{\tcap}[0]{\hat{\Bt}}
\newcommand{\ucap}[0]{\hat{\Bu}}
\newcommand{\vcap}[0]{\hat{\Bv}}
\newcommand{\wcap}[0]{\hat{\Bw}}
\newcommand{\xcap}[0]{\hat{\Bx}}
\newcommand{\ycap}[0]{\hat{\By}}
\newcommand{\zcap}[0]{\hat{\Bz}}
\newcommand{\thetacap}[0]{\hat{\Btheta}}

%
% to write R^n and C^n in a distinguishable fashion.  Perhaps change this
% to the double lined characters upon figuring out how to do so.
%
\newcommand{\C}[1]{$\mathbb{C}^{#1}$}
\newcommand{\R}[1]{$\mathbb{R}^{#1}$}

%
% various generally useful helpers
%

% derivative of #1 wrt. #2:
\newcommand{\D}[2] {\frac {d#2} {d#1}}

\newcommand{\inv}[1]{\frac{1}{#1}}
\newcommand{\cross}[0]{\times}

\newcommand{\abs}[1]{\lvert{#1}\rvert}
\newcommand{\norm}[1]{\lVert{#1}\rVert}
\newcommand{\innerprod}[2]{\langle{#1}, {#2}\rangle}
\newcommand{\dotprod}[2]{{#1} \cdot {#2}}
\newcommand{\bdotprod}[2]{\left({#1} \cdot {#2}\right)}
\newcommand{\crossprod}[2]{{#1} \cross {#2}}
\newcommand{\tripleprod}[3]{\dotprod{\left(\crossprod{#1}{#2}\right)}{#3}}

\DeclareMathOperator{\Proj}{Proj}
\DeclareMathOperator{\Span}{span}
\DeclareMathOperator{\Sgn}{sgn}
\DeclareMathOperator{\Area}{Area}
\DeclareMathOperator{\Volume}{Volume}

%
% A few miscellaneous things specific to this document
%
\newcommand{\crossop}[1]{\crossprod{#1}{}}

% R2 vector.
\newcommand{\VectorTwo}[2]{
\begin{bmatrix}
 {#1} \\
 {#2}
\end{bmatrix}
}

\newcommand{\VectorN}[1]{
\begin{bmatrix}
{#1}_1 \\
{#1}_2 \\
\vdots \\
{#1}_N \\
\end{bmatrix}
}

\newcommand{\DETuvij}[4]{
\begin{vmatrix}
 {#1}_{#3} & {#1}_{#4} \\
 {#2}_{#3} & {#2}_{#4}
\end{vmatrix}
}

\newcommand{\DETuvwijk}[6]{
\begin{vmatrix}
 {#1}_{#4} & {#1}_{#5} & {#1}_{#6} \\
 {#2}_{#4} & {#2}_{#5} & {#2}_{#6} \\
 {#3}_{#4} & {#3}_{#5} & {#3}_{#6}
\end{vmatrix}
}

\newcommand{\DETuvwxijkl}[8]{
\begin{vmatrix}
 {#1}_{#5} & {#1}_{#6} & {#1}_{#7} & {#1}_{#8} \\
 {#2}_{#5} & {#2}_{#6} & {#2}_{#7} & {#2}_{#8} \\
 {#3}_{#5} & {#3}_{#6} & {#3}_{#7} & {#3}_{#8} \\
 {#4}_{#5} & {#4}_{#6} & {#4}_{#7} & {#4}_{#8} \\
\end{vmatrix}
}

%\newcommand{\DETuvwxyijklm}[10]{
%\begin{vmatrix}
% {#1}_{#6} & {#1}_{#7} & {#1}_{#8} & {#1}_{#9} & {#1}_{#10} \\
% {#2}_{#6} & {#2}_{#7} & {#2}_{#8} & {#2}_{#9} & {#2}_{#10} \\
% {#3}_{#6} & {#3}_{#7} & {#3}_{#8} & {#3}_{#9} & {#3}_{#10} \\
% {#4}_{#6} & {#4}_{#7} & {#4}_{#8} & {#4}_{#9} & {#4}_{#10} \\
% {#5}_{#6} & {#5}_{#7} & {#5}_{#8} & {#5}_{#9} & {#5}_{#10}
%\end{vmatrix}
%}

% R3 vector.
\newcommand{\VectorThree}[3]{
\begin{bmatrix}
 {#1} \\
 {#2} \\
 {#3}
\end{bmatrix}
}



\author{Peeter Joot}
\email{peeter.joot@gmail.com}


%\chapter{PHY456H1F: Quantum Mechanics II.  Lecture 9 (Taught by Prof J.E. Sipe).  Adiabatic perturbation theory (cont.)}
\chapter{Adiabatic perturbation theory (cont.), and Fermi's golden rule}
\label{chap:qmTwoL9}
\blogpage{http://sites.google.com/site/peeterjoot/math2011/qmTwoL9.pdf}
%\date{Oct 5, 2011}





%\section{Disclaimer}
%
%Peeter's lecture notes from class.  May not be entirely coherent.

\section{Adiabatic perturbation theory (cont.)}

We were working through Adiabatic time dependent perturbation (as also covered in \S 17.5.2 of the text \cite{desai2009quantum}.)

Utilizing an expansion

\begin{align}\label{eqn:qmTwoL9:10}
\ket{\psi(t)} &= \sum_n c_n(t) e^{- i \omega_n^{(0)} t} \ket{\psi_n^{(0)} } \\
&= \sum_n b_n(t) \ket{\hat{\psi}_n(t)},
\end{align}

where

\begin{equation}\label{eqn:qmTwoL9:30}
H(t) \ket{\hat{\psi}_s(t)} = E_s(t) \ket{\hat{\psi}_s(t)} 
\end{equation}

and found

\begin{equation}\label{eqn:qmTwoL9:50}
\ddt{b_s(t)} = 
-i \left( 
\omega_s(t) - \Gamma_s(t)
\right) b_s(t)
- 
\sum_{n \ne s} b_n(t) 
\bra{\hat{\psi}_s(t)}
\ddt{} \ket{\hat{\psi}_n(t)}
\end{equation}

where

\begin{equation}\label{eqn:qmTwoL9:70}
\Gamma_s(t) =
i \bra{\hat{\psi}_s(t)} \ddt{} \ket{\hat{\psi}_s(t)} 
\end{equation}

Look for a solution of the form

\begin{equation}\label{eqn:qmTwoL9:90}
\begin{aligned}
b_s(t) &= \bar{b}_s(t) e^{-i \int_0^t dt' (\omega_s(t') - \Gamma_s(t'))} \\
&= 
\bar{b}_s(t) e^{-i \gamma_s(t)}
\end{aligned}
\end{equation}

where
\begin{equation}\label{eqn:qmTwoL9:110}
\gamma_s(t) = 
\int_0^t dt' (\omega_s(t') - \Gamma_s(t')).
\end{equation}

Taking derivatives of $\bar{b}_s$ and after a bit of manipulation we find that things conveniently cancel

\begin{align*}
\ddt{\bar{b}_s(t)} 
&= \ddt{} \left( b_s(t) e^{i \gamma_s(t) } \right) \\
&= 
\ddt{b_s(t)} e^{i \gamma_s(t) } +
b_s(t) \ddt{} e^{i \gamma_s(t) }  \\
&= 
\ddt{b_s(t)} e^{i \gamma_s(t) } +
b_s(t) i (\omega_s(t) - \Gamma_s(t)) e^{i \gamma_s(t) }.
\end{align*}

We find

\begin{align*}
\ddt{\bar{b}_s(t)} 
e^{-i \gamma_s(t)} 
&= 
\ddt{b_s(t)} + i b_s(t) (\omega_s(t) - \Gamma_s(t))  \\
&=
\cancel{i b_s(t) (\omega_s(t) - \Gamma_s(t)) }
-\cancel{i \left( 
\omega_s(t) - \Gamma_s(t)
\right) b_s(t)}
- 
\sum_{n \ne s} b_n(t) 
\bra{\hat{\psi}_s(t)}
\ddt{} \ket{\hat{\psi}_n(t)},
\end{align*}

so

\begin{align*}
\ddt{\bar{b}_s(t)} 
&=
- 
\sum_{n \ne s} b_n(t) 
e^{i \gamma_s(t)} 
\bra{\hat{\psi}_s(t)}
\ddt{} \ket{\hat{\psi}_n(t)} \\
&=
- 
\sum_{n \ne s} \bar{b}_n(t) 
e^{i (\gamma_s(t) - \gamma_n(t))} 
\bra{\hat{\psi}_s(t)}
\ddt{} \ket{\hat{\psi}_n(t)}.
\end{align*}

With a last bit of notation

\begin{equation}\label{eqn:qmTwoL9:130}
\gamma_{sn}(t) = \gamma_s(t) - \gamma_n(t)),
\end{equation}

the problem is reduced to one involving only the sums over the $n \ne s$ terms, and where all the dependence on $\bra{\hat{\psi}_s(t)} \ddt{} \ket{\hat{\psi}_s(t)}$ has been nicely isolated in a phase term

\begin{equation}\label{eqn:qmTwoL9:150}
\ddt{\bar{b}_s(t)} 
=
- 
\sum_{n \ne s} \bar{b}_n(t) 
e^{i \gamma_{sn}(t) }
\bra{\hat{\psi}_s(t)}
\ddt{} \ket{\hat{\psi}_n(t)}.
\end{equation}

\subsection{Looking for an approximate solution}

\paragraph{Try}: An approximate solution

\begin{equation}\label{eqn:qmTwoL9:170}
\bar{b}_n(t) = 
\delta_{nm}
\end{equation}

For $s = m$ this is okay, since we have $\ddt{\delta_{ns}} = 0$ which is consistent with

\begin{equation}\label{eqn:qmTwoL9:190}
\sum_{n \ne s} \delta_{ns} ( \cdots ) = 0
\end{equation}

However, for $s \ne m$ we get

\begin{align*}
\ddt{\bar{b}_s(t)} 
&=
- 
\sum_{n \ne s} 
\delta_{nm}
e^{i \gamma_{sn}(t) }
\bra{\hat{\psi}_s(t)}
\ddt{} \ket{\hat{\psi}_n(t)} \\
&=
- 
e^{i \gamma_{sm}(t) }
\bra{\hat{\psi}_s(t)}
\ddt{} \ket{\hat{\psi}_m(t)} \\
\end{align*}

But 

\begin{equation}\label{eqn:qmTwoL9:210}
\gamma_{sm}(t) = \int_0^t dt' \left( \inv{\hbar}( E_s(t') - E_m(t') ) - \Gamma_s(t') + \Gamma_m(t') \right)
\end{equation}

FIXME: I think we argued in class that the $\Gamma$ contributions are negligible.  Why was that?

Now, are energy levels will have variation with time, as illustrated in figure (\ref{fig:qmTwoL9:1})

\pdfTexFigure{figures/qmTwoL9fig1.pdf_tex}{Energy level variation with time}{fig:qmTwoL9:1}{0.2}

Perhaps unrealistically, suppose that our energy levels have some ``typical'' energy difference $\Delta E$, so that

\begin{equation}\label{eqn:qmTwoL9:230}
\gamma_{sm}(t) \approx \frac{\Delta E}{\hbar} t \equiv \frac{t}{\tau},
\end{equation}

or

\begin{equation}\label{eqn:qmTwoL9:250}
\tau = \frac{\hbar}{\Delta E}
\end{equation}

Suppose that $\tau$ is much less than a typical time $T$ over which instantaneous quantities (wavefunctions and brakets) change.  After a large time $T$

\begin{equation}\label{eqn:qmTwoL9:270}
e^{i \gamma_{sm}(t)} \approx e^{i T/\tau}
\end{equation}

so we have our phase term whipping around really fast, as illustrated in figure (\ref{fig:qmTwoL9:2}).

\pdfTexFigure{figures/qmTwoL9fig2.pdf_tex}{Phase whipping around}{fig:qmTwoL9:2}{0.3}

So, while $\bra{\hat{\psi}_s(t)} \ddt{} \ket{\hat{\psi}_m(t)}$ is moving really slow, but our phase space portion is changing really fast.  The key to the approximate solution is factoring out this quickly changing phase term.

\paragraph{Note} $\Gamma_s(t)$ is called the ``Berry'' phase \cite{wiki:GeometricPhase}, whereas the $E_s(t')/\hbar$ part is called the geometric phase, and can be shown to have a geometric interpretation.

To proceed we can introduce $\lambda$ terms, perhaps

\begin{equation}\label{eqn:qmTwoL9:290}
\bar{b}_s(t) = \delta_{ms} + \lambda \bar{b}^{(1)}_s(t) + \cdots
\end{equation}

and 
\begin{equation}\label{eqn:qmTwoL9:310}
- \sum_{n \ne s} e^{i \gamma_{sn}(t)} \lambda (\cdots)
\end{equation}

This $\lambda$ approximation and a similar Taylor series expansion in time have been explored further in \ref{chap:adiabaticApprox}.

\subsection{Degeneracy}

Suppose we have some branching of energy levels that were initially degenerate, as illustrated in figure (\ref{fig:qmTwoL9:3})

\pdfTexFigure{figures/qmTwoL9fig3.pdf_tex}{Degenerate energy level splitting}{fig:qmTwoL9:3}{0.3}

We have a necessity to choose states properly so there is a continuous evolution in the instantaneous eigenvalues as $H(t)$ changes.

\paragraph{Question: A physical  example?}

FIXME: Prof Sipe to ponder and revisit.

\section{Fermi's golden rule}

See \S 17.2 of the text \cite{desai2009quantum}.

Fermi originally had two golden rules, but his first one has mostly been forgotten.  This refers to his second.

This is really important, and probably the \underline{single most} important thing to learn in this course.  You'll find this falls out of many complex calculations.

Returning to general time dependent equations with

\begin{equation}\label{eqn:qmTwoL9:330}
H = H_0 + H'(t)
\end{equation}

\begin{equation}\label{eqn:qmTwoL9:350}
\ket{\psi(t)} = \sum_n c_n(t) e^{-i\omega_n t} \ket{\psi_n}
\end{equation}

and 

\begin{equation}\label{eqn:qmTwoL9:370}
i \hbar \dot{c}_n = \sum_n H_{mn}' e^{i \omega_{mn} t} c_n(t)
\end{equation}

where

\begin{align}\label{eqn:qmTwoL9:390}
H_{mn}'(t) &= \bra{\psi_m} H'(t) \ket{\psi_n} \\
\omega_n &= \frac{E_n}{\hbar} \\
\omega_{mn} &= \omega_m - \omega_n
\end{align}

Example:

\begin{equation}\label{eqn:qmTwoL9:410}
H'(t) = - \Bmu \cdot \BE(t).
\end{equation}

If $c_m^{(0)} = \delta_{mi}$, then to first order

\begin{equation}\label{eqn:qmTwoL9:430}
i \hbar \dot{c}^{(1)}(t) = H_{mi}'(t) e^{i \omega_{mi} t},
\end{equation}

and

\begin{equation}\label{eqn:qmTwoL9:450}
c_m^{(1)}(t) = \inv{i\hbar} \int_{t_0}^t H_{mi}'(t') e^{i \omega_{mi} t'} dt'.
\end{equation}

Assume the perturbation vanishes before time $t_0$.

\paragraph{Reminder}.  Have considered this using \ref{eqn:qmTwoL9:450} for a pulse as in figure (\ref{fig:qmTwoL9:gaussianWavePacket})

\imageFigure{figures/gaussianWavePacket}{Gaussian wave packet}{fig:qmTwoL9:gaussianWavePacket}{0.2}

Now we want to consider instead a non-terminating signal, that was zero before some initial time as illustrated in figure (\ref{fig:qmTwoL9:unitStepSine}), where the separation between two peaks is $\Delta t = 2\pi/\omega_0$.

\imageFigure{figures/unitStepSine}{Sine only after an initial time}{fig:qmTwoL9:unitStepSine}{0.2}

Our matrix element is 

\begin{equation}\label{eqn:qmTwoL9:470}
H_{mi}'(t) = - \bra{\psi_m} \Bmu \ket{\psi_i} \cdot \BE(t) = 
\left\{
\begin{array}{l l}
2 A_{mi} \sin(\omega_0 t) & \quad \mbox{if $t > 0$} \\
0 & \quad \mbox{if $t < 0$} \\
\end{array}
\right.
\end{equation}

Here the factor of 2 has been included for consistency with the text.

\begin{equation}\label{eqn:qmTwoL9:490}
H_{mi}'(t) = i A_{mi} 
\left( 
e^{-i \omega_0 t}
-e^{i \omega_0 t} 
\right)
\end{equation}

Plug this into the perturbation

\begin{equation}\label{eqn:qmTwoL9:510}
c_m^{(1)}(t) = 
\frac{A_{mi}}{\hbar} \int_{t_0}^t dt' 
\left( 
e^{i (\omega_{mi} - \omega_0 t) }
-e^{i (\omega_{mi} + \omega_0 t) }
\right)
\end{equation}

\pdfTexFigure{figures/qmTwoL9fig6.pdf_tex}{$\omega_{mi}$ illustrated}{fig:qmTwoL9:6}{0.3}
%figure (\ref{fig:qmTwoL9:6})

Suppose that

\begin{equation}\label{eqn:qmTwoL9:530}
\omega_0 \approx \omega_{mi},
\end{equation}

then

\begin{equation}\label{eqn:qmTwoL9:550}
c_m^{(1)}(t) \approx
\frac{A_{mi}}{\hbar} \int_{t_0}^t dt' 
\left( 
1
-e^{2 i \omega_0 t }
\right),
\end{equation}

but the exponential has essentially no contribution

\begin{align*}
\Abs{\int_0^t e^{2 i \omega_0 t'} dt' } 
&= 
\Abs{\frac{e^{2 i \omega_0 t} -1 }{2 i \omega_0}}  \\
&= 
\frac{\sin(\omega_0 t)}{\omega_0} \\
&\sim \inv{\omega_0}
\end{align*}

so for $t \gg \inv{\omega_0}$ and $\omega_0 \approx \omega_{mi}$ we have 

\begin{equation}\label{eqn:qmTwoL9:570}
c_m^{(1)}(t) \approx \frac{A_{mi}}{\hbar} t
\end{equation}

Similarly for $\omega_0 \approx \omega_{im}$ as in figure (\ref{fig:qmTwoL9:7}) 

\pdfTexFigure{figures/qmTwoL9fig7.pdf_tex}{FIXME: qmTwoL9fig7}{fig:qmTwoL9:7}{0.3}

then

\begin{equation}\label{eqn:qmTwoL9:590}
c_m^{(1)}(t) \approx
\frac{A_{mi}}{\hbar} \int_{t_0}^t dt' 
\left( 
e^{-2 i \omega_0 t }
-1
\right),
\end{equation}

and we have

\begin{equation}\label{eqn:qmTwoL9:610}
c_m^{(1)}(t) \approx -\frac{A_{mi}}{\hbar} t
\end{equation}



%
% Copyright � 2012 Peeter Joot.  All Rights Reserved.
% Licenced as described in the file LICENSE under the root directory of this GIT repository.
%

%\chapter{PHY456H1F: Quantum Mechanics II.  Lecture 10 (Taught by Prof J.E. Sipe).  Fermi's golden rule (cont.)}
%\chapter{Fermi's golden rule (cont.)}
\label{chap:qmTwoL10}
%\blogpage{http://sites.google.com/site/peeterjoot/math2011/qmTwoL10.pdf}
%\date{Oct 10, 2011}

\section{Recap. Where we got to on Fermi's golden rule}

We are continuing on the topic of Fermi golden rule, as also covered in \S 17.2 of the text \citep{desai2009quantum}.  Utilizing a wave train with peaks separation \(\Delta t = 2\pi/\omega_0\), zero before some initial time \cref{fig:qmTwoL10:unitStepSine}.

\imageFigure{../../figures/phy456/unitStepSine}{Sine only after an initial time}{fig:qmTwoL10:unitStepSine}{0.2}

Perturbing a state in the \(i\)th energy level, and looking at the states for the \(m\)th energy level as illustrated in \cref{fig:qmTwoL10:2}


\pdfTexFigure{../../figures/phy456/qmTwoL10fig2.pdf_tex}{Perturbation from \(i\) to \(m\)th energy levels}{fig:qmTwoL10:2}{0.3}

Our matrix element was

\begin{equation}\label{eqn:qmTwoL10:10}
\begin{aligned}
H_{mi}'(t) 
&= 2 A_mi \sin(\omega_0 t) \theta(t) \\
&= i A_{mi} ( e^{-i \omega_0 t} - e^{i \omega_0 t} ) \theta(t),
\end{aligned}
\end{equation}

and we found

\begin{equation}\label{eqn:qmTwoL10:30}
c_m^{(1)}(t) = \frac{A_{mi}}{\Hbar} \int_0^t dt' \left( 
e^{i (\omega_{mi} - \omega_0) t'} 
-
e^{i (\omega_{mi} + \omega_0) t'} 
\right),
\end{equation}

and argued that 

\begin{equation}\label{eqn:qmTwoL10:50}
\Abs{ c_m^{(1)}(t) }^2  \sim \left( \frac{A_{mi}}{\Hbar} \right)^2 t^2 + \cdots
\end{equation}

where \(\omega_0 t \gg 1\) for \(\omega_{mi} \sim \pm \omega_0\).

We can also just integrate \eqnref{eqn:qmTwoL10:30} directly

\begin{equation}\label{eqn:qmTwoL10:30b}
\begin{aligned}
c_m^{(1)}(t) 
&= 
\frac{A_{mi}}{\Hbar} \left( 
\frac{e^{i (\omega_{mi} - \omega_0) t} - 1}
{ i (\omega_{mi} - \omega_0) }
-
\frac{e^{i (\omega_{mi} + \omega_0) t} - 1}{ i (\omega_{mi} + \omega_0) }
\right) \\
&\equiv 
A_{mi}(\omega_0, t) - A_{mi}(-\omega_0, t),
\end{aligned}
\end{equation}

where 

\begin{equation}\label{eqn:qmTwoL10:70}
A_{mi}(\omega_0, t) =
\frac{A_{mi}}{\Hbar} 
\frac{e^{i (\omega_{mi} - \omega_0) t} - 1}
{ i (\omega_{mi} - \omega_0) }
\end{equation}

Factoring out the phase term, we have

\begin{equation}\label{eqn:qmTwoL10:90}
A_{mi}(\omega_0, t) =
\frac{2 A_{mi}}{\Hbar} 
e^{i (\omega_{mi} - \omega_0) t/2}
\frac{
\sin(
(\omega_{mi} - \omega_0) t/2
)
}
{ (\omega_{mi} - \omega_0) }
\end{equation}

We we will have two lobes, centered on \(\pm \omega_0\), as illustrated in \cref{fig:qmTwoL10:qmTwoL10fig3}
\imageFigure{../../figures/phy456/qmTwoL10fig3}{Two sinc lobes}{fig:qmTwoL10:qmTwoL10fig3}{0.2}

\section{Fermi's Golden rule}

Fermi's Golden rule applies to a continuum of states (there are other forms of Fermi's golden rule, but this is the one we will talk about, and is the one in the book).  One example is the ionized states of an atom, where the energy level separation becomes so small that we can consider it continuous

\imageFigure{../../figures/phy456/continuumEnergyLevels}{Continuum of energy levels for ionized states of an atom}{fig:qmTwoL10:continuumEnergyLevels}{0.2}
%\cref{fig:qmTwoL10:continuumEnergyLevels}

Another example are the unbound states in a semiconductor well as illustrated in \cref{fig:qmTwoL10:semiConductorWell}
\imageFigure{../../figures/phy456/semiConductorWell}{Semi-conductor well}{fig:qmTwoL10:semiConductorWell}{0.2}

Note that we can have reflection from the well even in the continuum states where we would have no such reflection classically.  However, with enough energy, states are approximately plane waves.  In one dimension

\begin{equation}\label{eqn:qmTwoL10:110}
\begin{aligned}
\braket{x}{\psi_p} &\approx \frac{e^{i p x/\Hbar}}{\sqrt{2 \pi \Hbar}} \\
\braket{\psi_p}{\psi_p'} &= \delta(p - p')
\end{aligned}
\end{equation}

or in 3d

\begin{equation}\label{eqn:qmTwoL10:130}
\begin{aligned}
\braket{\Br}{\psi_{\Bp}} &\approx \frac{e^{i \Bp \cdot \Br/\Hbar}}{(2 \pi \Hbar)^{3/2} } \\
\braket{\psi_{\Bp}}{\psi_{\Bp'}} &= \delta^3(\Bp - \Bp')
\end{aligned}
\end{equation}

Let us consider the 1d model for the quantum well in more detail.  Including both discrete and continuous states we have

\begin{equation}\label{eqn:qmTwoL10:150}
\ket{\psi(t)} = 
\sum_n c_n(t) e^{-i \omega_n t} \ket{\psi_n} + 
\int dp c_p(t) e^{-i \omega_p t} \ket{\psi_p} 
\end{equation}

Imagine at \(t=0\) that the wave function started in some discrete state, and look at the probability that we ``kick the electron out of the well''.  Calculate

\begin{equation}\label{eqn:qmTwoL10:170}
\calP = \int dp \Abs{c_p^{(1)}(t)}^2
\end{equation}

Now, we assume that our matrix element has the following form

\begin{equation}\label{eqn:qmTwoL10:190}
H_{pi}'(t) = \left( 
\overbar{A}_{pi} e^{-i \omega_0 t}
+\overbar{B}_{pi} e^{i \omega_0 t} \right) \theta(t)
\end{equation}

generalizing the wave train matrix element that we had previously

\begin{equation}\label{eqn:qmTwoL10:210}
H_{mi}'(t) = i A_{mi} \left( 
e^{-i \omega_0 t}
- e^{i \omega_0 t} \right) \theta(t)
\end{equation}

Doing the perturbation we have

\begin{equation}\label{eqn:qmTwoL10:230}
\calP = \int dp \Abs{
A_{pi}(\omega_0, t)
+ B_{pi}(-\omega_0, t)
}^2
\end{equation}

where

\begin{equation}\label{eqn:qmTwoL10:250}
A_{pi}(\omega_0, t) = 
\frac{2 \overbar{A}_{pi}}{i \Hbar } 
e^{i (\omega_{pi} - \omega_0) t/2}
\frac{\sin((\omega_{pi} - \omega_0) t/2)}{
\omega_{pi} - \omega_0
}
\end{equation}

which is peaked at \(\omega_{pi} = \omega_0\), and

\begin{equation}\label{eqn:qmTwoL10:270}
B_{pi}(\omega_0, t) = 
\frac{2 \overbar{B}_{pi}}{i \Hbar } 
e^{i (\omega_{pi} + \omega_0) t/2}
\frac{\sin((\omega_{pi} + \omega_0) t/2)}{
\omega_{pi} + \omega_0
}
\end{equation}

which is peaked at \(\omega_{pi} = -\omega_0\).

FIXME: show that this is the perturbation result.

In \eqnref{eqn:qmTwoL10:230} at \(t \gg 0\) the only significant contribution is from the \(A\) portion as illustrated in \cref{fig:qmTwoL10:qmTwoL10fig6} where we are down in the wiggles of \(A_{pi}\).

\imageFigure{../../figures/phy456/qmTwoL10fig6}{}{fig:qmTwoL10:qmTwoL10fig6}{0.4}

Our probability to find the particle in the continuum range is now approximately

\begin{equation}\label{eqn:qmTwoL10:290}
\calP = \int dp \Abs{
A_{pi}(\omega_0, t)
}^2
\end{equation}

With 

\begin{equation}\label{eqn:qmTwoL10:310}
\omega_{pi} - \omega_0 = \inv{\Hbar}\left( \frac{p^2}{2m} - E_i \right) - \omega_0,
\end{equation}

define \(\overbar{p}\) so that

\begin{equation}\label{eqn:qmTwoL10:330}
0 = \inv{\Hbar}\left( \frac{\overbar{p}^2}{2m} - E_i \right) - \omega_0.
\end{equation}

In momentum space, we know have the sinc functions peaked at \(\pm \overbar{p}\) as in \cref{fig:qmTwoL10:qmTwoL10fig7}

\imageFigure{../../figures/phy456/qmTwoL10fig7}{Momentum space view}{fig:qmTwoL10:qmTwoL10fig7}{0.2}

The probability that the electron goes to the right is then

\begin{equation}\label{eqn:qmTwoL10:350}
\begin{aligned}
\calP_{+} 
&= 
\int_0^\infty dp 
\Abs{
c_p^{(1)}(t)
}^2 \\
&=
\int_0^\infty dp 
\Abs{
\overbar{A}_{pi}
}^2 
\frac{
\sin^2\left( (\omega_{pi} - \omega_0) t/2 \right)
}{
\left( \omega_{pi} - \omega_0 \right)^2
},
\end{aligned}
\end{equation}

with 

\begin{equation}\label{eqn:qmTwoL10:370}
\omega_{pi} = \inv{\Hbar}\left( \frac{p^2}{2m} - E_i
\right)
\end{equation}

we have with a change of variables

\begin{equation}\label{eqn:qmTwoL10:390}
\calP_{+} 
=
\frac{4}{\Hbar^2}
\int_{-E_i/\Hbar}^\infty d\omega_{pi}
\Abs{
\overbar{A}_{pi}
}^2 
\frac{dp}{d\omega_{pi}}
\frac{
\sin^2\left( (\omega_{pi} - \omega_0) t/2 \right)
}{
\left( \omega_{pi} - \omega_0 \right)^2
}.
\end{equation}

Now suppose we have \(t\) small enough so that \(\calP_{+} \ll 1\) and \(t\) large enough so 

\begin{equation}\label{eqn:qmTwoL10:410}
\Abs{
\overbar{A}_{pi}
}^2 
\frac{dp}{d\omega_{pi}}
\end{equation}

is roughly constant over \(\Delta \omega\).  This is a sort of ``Goldilocks condition'', a time that can not be too small, and can not be too large, but instead has to be ``just right''.  Given such a condition

\begin{equation}\label{eqn:qmTwoL10:430}
\calP_{+} 
=
\frac{4}{\Hbar^2}
\Abs{
\overbar{A}_{pi}
}^2 
\frac{dp}{d\omega_{pi}}
\int_{-E_i/\Hbar}^\infty d\omega_{pi}
\frac{
\sin^2\left( (\omega_{pi} - \omega_0) t/2 \right)
}{
\left( \omega_{pi} - \omega_0 \right)^2
},
\end{equation}

where we can pull stuff out of the integral since the main contribution is at the peak.  Provided \(\overbar{p}\) is large enough, using \eqnref{eqn:sincIntegral:50}, then 

\begin{equation}\label{eqn:qmTwoL10:450}
\begin{aligned}
\int_{-E_i/\Hbar}^\infty d\omega_{pi}
\frac{
\sin^2\left( (\omega_{pi} - \omega_0) t/2 \right)
}{
\left( \omega_{pi} - \omega_0 \right)^2
}
&\approx 
\int_{-\infty}^\infty d\omega_{pi}
\frac{
\sin^2\left( (\omega_{pi} - \omega_0) t/2 \right)
}{
\left( \omega_{pi} - \omega_0 \right)^2
} \\
&=
\frac{t}{2} \pi,
\end{aligned}
\end{equation}

leaving the probability of the electron with a going right continuum state as

\begin{equation}\label{eqn:qmTwoL10:470}
\calP_{+} 
=
\frac{4}{\Hbar^2}
\mathLabelBox{
\Abs{
\overbar{A}_{pi}
}^2 
}{matrix element}
\mathLabelBox
[
   labelstyle={below of=m\themathLableNode, below of=m\themathLableNode}
]
{\evalbar{\frac{dp}{d\omega_{pi}}}{\overbar{p}}}{density of states}
\frac{t}{2} \pi.
\end{equation}

The \(dp/d\omega_{pi}\) is something like ``how many continuous states are associated with a transition from a discrete frequency interval.''

We can also get this formally from \eqnref{eqn:qmTwoL10:430} with

\begin{equation}\label{eqn:qmTwoL10:490}
\frac{
\sin^2\left( (\omega_{pi} - \omega_0) t/2 \right)
}{
\left( \omega_{pi} - \omega_0 \right)^2
}
\rightarrow 
\frac{t}{2} \pi \delta(\omega_{pi} - \omega_0),
\end{equation}

so

\begin{equation}\label{eqn:qmTwoL10:510}
\begin{aligned}
c_p^{(1)}(t) 
&\rightarrow \frac{2 \pi t}{\Hbar^2} 
\Abs{
\overbar{A}_{pi}
}^2 
\delta(\omega_{pi} - \omega_0) \\
&=
\frac{2 \pi t}{\Hbar} 
\Abs{
\overbar{A}_{pi}
}^2 
\delta(E_{pi} - \Hbar \omega_0)
\end{aligned}
\end{equation}

where \(\delta(ax) = \delta(x)/\Abs{a}\) has been used to pull in a factor of \(\Hbar\) into the delta.

The ratio of the coefficient to time is then

\begin{equation}\label{eqn:qmTwoL10:530}
\frac{c_p^{(1)}(t) }{t}
=
\frac{2 \pi}{\Hbar} 
\Abs{
\overbar{A}_{pi}
}^2 
\delta(E_{pi} - \Hbar \omega_0).
\end{equation}

or ``between friends''

\begin{equation}\label{eqn:qmTwoL10:550}
''\frac{dc_p^{(1)}(t) }{dt}''
=
\frac{2 \pi}{\Hbar} 
\Abs{
\overbar{A}_{pi}
}^2 
\delta(E_{pi} - \Hbar \omega_0),
\end{equation}

roughly speaking we have a ``rate'' or transitions from the discrete into the continuous.  Here ``rate'' is in quotes since it does not hold for small t.

This has been worked out for \(\calP_{+}\).  This can also be done for \(\calP_{-}\), the probability that the electron will end up in a left trending continuum state.

While the above is not a formal derivation, but illustrates the form of what is called Fermi's golden rule.  Namely that such a rate has the structure

\begin{equation}\label{eqn:qmTwoL10:570}
\frac{2 \pi}{\Hbar} \times (\text{matrix element})^2 \times \text{energy conservation}
\end{equation}

\part{Notes and Problems.}
%
% Copyright � 2012 Peeter Joot.  All Rights Reserved.
% Licenced as described in the file LICENSE under the root directory of this GIT repository.
%

\label{chap:phy456ps1}
%\blogpage{http://sites.google.com/site/peeterjoot/math2011/phy456ps1.pdf}
%%\date{Sept 12, 2011}

\makeoproblem{Harmonic oscillator}{pr:phy456ps1:1}{2011 ps1/p1}{
Let \(H_o\) indicate the Hamiltonian of a 1D harmonic oscillator with mass \(m\) and frequency \(\omega\)

\begin{equation}\label{eqn:phy456ps1:99}
H_o = \frac{P^2}{2m} + \inv{2} m \omega^2 X^2
\end{equation}

and denote the energy eigenstates by \(\ket{n}\), where \(n\) is the eigenvalue of the number operator.

\makesubproblem{Find \(\bra{n} X^4 \ket{n}\)}{pr:phy456ps1:1:a}
\makesubproblem{Quadratic pertubation}{pr:phy456ps1:1:b}

Find the ground state energy of the Hamiltonian \(H = H_o + \gamma X^2\).  You may assume \(\gamma > 0\). [Hint: This is not a trick question.]

\makesubproblem{linear pertubation}{pr:phy456ps1:1:c}

Find the ground state energy of the Hamiltonian \(H = H_o - \alpha X\). [Hint: This is a bit harder than part \ref{pr:phy456ps1:1:b} but not much. Try "completing the square."]
} % makeoproblem

\makeanswer{pr:phy456ps1:1}{ 

\makeSubAnswer{\(X^4\)}{pr:phy456ps1:1:a}

Working through \ref{chap:phy456ps1SHO} we have now got enough context to attempt the first part of the question, calculation of

\begin{equation}\label{eqn:phy456ps1:320}
\bra{n} X^4 \ket{n}
\end{equation}

We have calculated things like this before, such as
\begin{equation}\label{eqn:phy456ps1:860}
\begin{aligned}
\bra{n} X^2 \ket{n}
&=
\frac{\Hbar}{2 m \omega} \bra{n} (a + a^\dagger)^2 \ket{n}
\end{aligned}
\end{equation}

To continue we need an exact relation between \(\ket{n}\) and \(\ket{n \pm 1}\).  Recall that \(a \ket{n}\) was an eigenstate of \(a^\dagger a\) with eigenvalue \(n - 1\).  This implies that the eigenstates \(a \ket{n}\) and \(\ket{n-1}\) are proportional

\begin{equation}\label{eqn:phy456ps1:340}
a \ket{n} = c_n \ket{n - 1},
\end{equation}

or
\begin{equation}\label{eqn:phy456ps1:880}
\begin{aligned}
\bra{n} a^\dagger a \ket{n} &= \Abs{c_n}^2 \braket{n - 1}{n-1} = \Abs{c_n}^2 \\
n \braket{n}{n} &= \\
n &=
\end{aligned}
\end{equation}

so that

\begin{equation}\label{eqn:phy456ps1:380}
a \ket{n} = \sqrt{n} \ket{n - 1}.
\end{equation}

Similarly let

\begin{equation}\label{eqn:phy456ps1:400}
a^\dagger \ket{n} = b_n \ket{n + 1},
\end{equation}

or
\begin{equation}\label{eqn:phy456ps1:900}
\begin{aligned}
\bra{n} a a^\dagger \ket{n} &= \Abs{b_n}^2 \braket{n - 1}{n-1} = \Abs{b_n}^2 \\
\bra{n} (1 + a^\dagger a) \ket{n} &= \\
1 + n &=
\end{aligned}
\end{equation}

so that

\begin{equation}\label{eqn:phy456ps1:440}
a^\dagger \ket{n} = \sqrt{n+1} \ket{n + 1}.
\end{equation}

We can now return to \eqnref{eqn:phy456ps1:320}, and find

\begin{equation}\label{eqn:phy456ps1:920}
\begin{aligned}
\bra{n} X^4 \ket{n}
&=
\frac{\Hbar^2}{4 m^2 \omega^2} \bra{n} (a + a^\dagger)^4 \ket{n}
\end{aligned}
\end{equation}

Consider half of this braket

\begin{equation}\label{eqn:phy456ps1:940}
\begin{aligned}
(a + a^\dagger)^2 \ket{n}
&=
\left( a^2 + (a^\dagger)^2 + a^\dagger a + a a^\dagger \right) \ket{n} \\
&=
\left( a^2 + (a^\dagger)^2 + a^\dagger a + (1 + a^\dagger a) \right) \ket{n} \\
&=
\left( a^2 + (a^\dagger)^2 + 1 + 2 a^\dagger a \right) \ket{n} \\
&=
\sqrt{n-1}\sqrt{n-2} \ket{n-2}
+
\sqrt{n+1}\sqrt{n+2} \ket{n + 2}
+
\ket{n}
+  2 n \ket{n}
\end{aligned}
\end{equation}

Squaring, utilizing the Hermitian nature of the \(X\) operator %, we have for \(n > 2\)

\begin{equation}\label{eqn:phy456ps1:500}
\bra{n} X^4 \ket{n}
=
\frac{\Hbar^2}{4 m^2 \omega^2}
\left(
(n-1)(n-2) + (n+1)(n+2) + (1 + 2n)^2
\right)
=
\frac{\Hbar^2}{4 m^2 \omega^2}
\left( 6 n^2 + 4 n + 5 \right)
\end{equation}

%It also looks like we can drop the \(n > 2\) restriction since the \(\sqrt{n-1}\) and \(\sqrt{n-2}\) factors kill off the
\makeSubAnswer{Quadratic ground state}{pr:phy456ps1:1:b}

Find the ground state energy of the Hamiltonian \(H = H_0 + \gamma X^2\) for \(\gamma > 0\).

The new Hamiltonian has the form

\begin{equation}\label{eqn:phy456ps1:520}
H = \frac{P^2}{2m} + \inv{2} m \left(\omega^2 + \frac{2 \gamma}{m} \right) X^2 =
\frac{P^2}{2m} + \inv{2} m {\omega'}^2 X^2,
\end{equation}

where
\begin{equation}\label{eqn:phy456ps1:540}
\omega' = \sqrt{ \omega^2 + \frac{2 \gamma}{m} }
\end{equation}

The energy states of the Hamiltonian are thus

\begin{equation}\label{eqn:phy456ps1:560}
E_n = \Hbar \sqrt{ \omega^2 + \frac{2 \gamma}{m} } \left( n + \inv{2} \right)
\end{equation}

and the ground state of the modified Hamiltonian \(H\) is thus

\begin{equation}\label{eqn:phy456ps1:580}
E_0 = \frac{\Hbar}{2} \sqrt{ \omega^2 + \frac{2 \gamma}{m} }
\end{equation}

\makeSubAnswer{Linear ground state}{pr:phy456ps1:1:c}

Find the ground state energy of the Hamiltonian \(H = H_0 - \alpha X\).

With a bit of play, this new Hamiltonian can be factored into

\begin{equation}\label{eqn:phy456ps1:590}
H
= \Hbar \omega \left( b^\dagger b + \inv{2} \right) - \frac{\alpha^2}{2 m \omega^2}
= \Hbar \omega \left( b b^\dagger - \inv{2} \right) - \frac{\alpha^2}{2 m \omega^2},
\end{equation}

where

\begin{equation}\label{eqn:phy456ps1:600}
\begin{aligned}
b &= \sqrt{\frac{m \omega}{2\Hbar}} X + \frac{i P}{\sqrt{2 m \Hbar \omega}} - \frac{\alpha}{\omega \sqrt{ 2 m \Hbar \omega }} \\
b^\dagger &= \sqrt{\frac{m \omega}{2\Hbar}} X - \frac{i P}{\sqrt{2 m \Hbar \omega}} - \frac{\alpha}{\omega \sqrt{ 2 m \Hbar \omega }}.
\end{aligned}
\end{equation}

From \eqnref{eqn:phy456ps1:590} we see that we have the same sort of commutator relationship as in the original Hamiltonian

\begin{equation}\label{eqn:phy456ps1:610}
\antisymmetric{b}{b^\dagger} = 1,
\end{equation}

and because of this, all the preceding arguments follow unchanged with the exception that the energy eigenstates of this Hamiltonian are shifted by a constant

\begin{equation}\label{eqn:phy456ps1:620}
H \ket{n} = \left( \Hbar \omega \left( n + \inv{2} \right) - \frac{\alpha^2}{2 m \omega^2} \right) \ket{n},
\end{equation}

where the \(\ket{n}\) states are simultaneous eigenstates of the \(b^\dagger b\) operator

\begin{equation}\label{eqn:phy456ps1:630}
b^\dagger b \ket{n} = n \ket{n}.
\end{equation}

The ground state energy is then
\begin{equation}\label{eqn:phy456ps1:640}
E_0 = \frac{\Hbar \omega }{2} - \frac{\alpha^2}{2 m \omega^2}.
\end{equation}

This makes sense.  A translation of the entire position of the system should not effect the energy level distribution of the system, but we have set our reference potential differently, and have this constant energy adjustment to the entire system.
} % makeanswer

\makeoproblem{Expectation values for position operators for spinless hydrogen}{pr:phy456ps1:2}{2011 ps1/p2}{ 
Show that for all energy eigenstates \(\ket{\Phi_{nlm}}\) of the (spinless) hydrogen atom, where as usual \(n\), \(l\), and \(m\) are respectively the principal, azimuthal, and magnetic quantum numbers, we have

\begin{equation}\label{eqn:phy456ps1:98}
\bra{\Phi_{nlm}}
X
\ket{\Phi_{nlm}}
=
\bra{\Phi_{nlm}}
Y
\ket{\Phi_{nlm}}
=
\bra{\Phi_{nlm}}
Z
\ket{\Phi_{nlm}}
= 0
\end{equation}

[Hint: Take note of the parity of the spherical harmonics (see "quick summary" notes on the spherical harmonics).]
} % makeoproblem

\makeanswer{pr:phy456ps1:2}{
%We are asked to show that for any eigenkets of the hydrogen atom \(\ket{\Phi_{nlm}}\) we have
%
%\begin{equation}\label{eqn:phy456ps1:700}
%\bra{\Phi_{nlm}} X \ket{\Phi_{nlm}} 
%=
%\bra{\Phi_{nlm}} Y \ket{\Phi_{nlm}} 
%=
%\bra{\Phi_{nlm}} Z \ket{\Phi_{nlm}}.
%\end{equation}
%
The summary sheet provides us with the wavefunction
\begin{equation}\label{eqn:phy456ps1:720}
\braket{\Br}{\Phi_{nlm}} = 
\frac{2}{n^2 a_0^{3/2}} \sqrt{\frac{(n-l-1)!}{(n+l)!)^3}} F_{nl}\left( \frac{2r}{n a_0} \right) Y_l^m(\theta, \phi),
\end{equation}

where \(F_{nl}\) is a real valued function defined in terms of Lagueere polynomials.  Working with the expectation of the \(X\) operator to start with we have

\begin{equation}\label{eqn:phy456ps1:960}
\begin{aligned}
\bra{\Phi_{nlm}} X \ket{\Phi_{nlm}} 
&=
\int 
\braket{\Phi_{nlm}}{\Br'} \bra{\Br'} X \ket{\Br} \braket{\Br}{\Phi_{nlm}} d^3 \Br d^3 \Br' \\
&=
\int 
\braket{\Phi_{nlm}}{\Br'} \delta(\Br - \Br') r \sin\theta \cos\phi \braket{\Br}{\Phi_{nlm}} d^3 \Br d^3 \Br' \\
&=
\int 
\Phi_{nlm}^\conj(\Br) r \sin\theta \cos\phi \Phi_{nlm}(\Br) d^3 \Br \\
&\sim
\int r^2 dr \Abs{ F_{nl}\left(\frac{2 r}{ n a_0} \right)}^2 r 
\int \sin\theta d\theta d\phi
{Y_l^m}^\conj(\theta, \phi) \sin\theta \cos\phi Y_l^m(\theta, \phi) \\
\end{aligned}
\end{equation}

Recalling that the only \(\phi\) dependence in \(Y_l^m\) is \(e^{i m \phi}\) we can perform the \(d\phi\) integration directly, which is

\begin{equation}\label{eqn:phy456ps1:740}
\int_{\phi=0}^{2\pi} \cos\phi d\phi e^{-i m \phi} e^{i m \phi} = 0.
\end{equation}

We have the same story for the \(Y\) expectation which is

\begin{equation}\label{eqn:phy456ps1:760}
\bra{\Phi_{nlm}} X \ket{\Phi_{nlm}} 
\sim
\int r^2 dr \Abs{F_{nl}\left( \frac{2 r}{ n a_0} \right)}^2 r 
\int \sin\theta d\theta d\phi
{Y_l^m}^\conj(\theta, \phi) \sin\theta \sin\phi Y_l^m(\theta, \phi).
\end{equation}

Our \(\phi\) integral is then just

\begin{equation}\label{eqn:phy456ps1:780}
\int_{\phi=0}^{2\pi} \sin\phi d\phi e^{-i m \phi} e^{i m \phi} = 0,
\end{equation}

also zero.  The \(Z\) expectation is a slightly different story.  There we have

\begin{equation}\label{eqn:phy456ps1:800}
\begin{aligned}
\bra{\Phi_{nlm}} Z \ket{\Phi_{nlm}} 
&\sim
\int dr \Abs{F_{nl}\left( \frac{2 r}{ n a_0} \right)}^2 r^3  \\
&\quad \int_0^{2\pi} d\phi
\int_0^\pi \sin \theta d\theta
\left( \sin\theta \right)^{-2m}
\left( \frac{d^{l - m}}{d (\cos\theta)^{l-m}} \sin^{2l}\theta \right)^2
\cos\theta.
\end{aligned}
\end{equation}

Within this last integral we can make the substitution

\begin{equation}\label{eqn:phy456ps1:820}
\begin{aligned}
u &= \cos\theta \\
\sin\theta d\theta &= - d(\cos\theta) = -du \\
u &\in [1, -1],
\end{aligned}
\end{equation}

and the integral takes the form
\begin{equation}\label{eqn:phy456ps1:840}
-\int_{-1}^1 
(-du) 
\inv{(1 - u^2)^m} 
\left( \frac{d^{l-m}}{d u^{l -m }} (1 - u^2)^l
\right)^2 u.
\end{equation}

Here we have the product of two even functions, times one odd function (\(u\)), over a symmetric interval, so the end result is zero, completing the problem.

I was not able to see how to exploit the parity result suggested in the problem, but it was not so bad to show these directly.
} % makeanswer

\makeproblem{Angular momentum operator}{pr:phy456ps1:3}{2011 ps1/p3}{ 
Working with the appropriate expressions in \it{Cartesian components}, confirm that \(L_i \ket{\psi} = 0\) for each component of angular momentum \(L_i\), if \(\braket{\Br}{\psi} = \psi(\Br)\) is in fact only a function of \(r = \Abs{\Br}\).
} % makeoproblem

\makeanswer{pr:phy456ps1:3}{ 
In order to proceed, we will have to consider a matrix element, so that we can operate on \(\ket{\psi}\) in position space.  For that matrix element, we can proceed to insert complete states, and reduce the problem to a question of wavefunctions.  That is

\begin{equation}\label{eqn:phy456ps1:980}
\begin{aligned}
\bra{\Br} L_i \ket{\psi}
&=
\int d^3 \Br' \bra{\Br} L_i \ket{\Br'} \braket{\Br'}{\psi} \\
&=
\int d^3 \Br' \bra{\Br} \epsilon_{i a b} X_a P_b \ket{\Br'} \braket{\Br'}{\psi} \\
&=
-i \Hbar \epsilon_{i a b} \int d^3 \Br' x_a \bra{\Br} \PD{X_b}{\psi(\Br')} \ket{\Br'}  \\
&=
-i \Hbar \epsilon_{i a b} \int d^3 \Br' x_a \PD{x_b}{\psi(\Br')} \braket{\Br}{\Br'}  \\
&=
-i \Hbar \epsilon_{i a b} \int d^3 \Br' x_a \PD{x_b}{\psi(\Br')} \delta^3(\Br - \Br') \\
&=
-i \Hbar \epsilon_{i a b} x_a \PD{x_b}{\psi(\Br)} 
\end{aligned}
\end{equation}

With \(\psi(\Br) = \psi(r)\) we have

\begin{equation}\label{eqn:phy456ps1:1000}
\begin{aligned}
\bra{\Br} L_i \ket{\psi}
&=
-i \Hbar \epsilon_{i a b} x_a \PD{x_b}{\psi(r)}  \\
&=
-i \Hbar \epsilon_{i a b} x_a \PD{x_b}{r} \frac{d\psi(r)}{dr}  \\
&=
-i \Hbar \epsilon_{i a b} x_a \inv{2} 2 x_b \inv{r} \frac{d\psi(r)}{dr}  \\
\end{aligned}
\end{equation}

We are left with an sum of a symmetric product \(x_a x_b\) with the antisymmetric tensor \(\epsilon_{i a b}\) so this is zero for all \(i \in [1,3]\).
} % makeanswer

%
% Copyright � 2012 Peeter Joot.  All Rights Reserved.
% Licenced as described in the file LICENSE under the root directory of this GIT repository.
%

% 
% 
%%
% Copyright � 2015 Peeter Joot.  All Rights Reserved.
% Licenced as described in the file LICENSE under the root directory of this GIT repository.
%
\documentclass[]{eliblog}

\usepackage{amsmath}
\usepackage{mathpazo}

%
% shorthand for bold symbols, convenient for vectors and matrices
%
\newcommand{\Ba}[0]{\mathbf{a}}
\newcommand{\Bb}[0]{\mathbf{b}}
\newcommand{\Bc}[0]{\mathbf{c}}
\newcommand{\Bd}[0]{\mathbf{d}}
\newcommand{\Be}[0]{\mathbf{e}}
\newcommand{\Bf}[0]{\mathbf{f}}
\newcommand{\Bg}[0]{\mathbf{g}}
\newcommand{\Bh}[0]{\mathbf{h}}
\newcommand{\Bi}[0]{\mathbf{i}}
\newcommand{\Bj}[0]{\mathbf{j}}
\newcommand{\Bk}[0]{\mathbf{k}}
\newcommand{\Bl}[0]{\mathbf{l}}
\newcommand{\Bm}[0]{\mathbf{m}}
\newcommand{\Bn}[0]{\mathbf{n}}
\newcommand{\Bo}[0]{\mathbf{o}}
\newcommand{\Bp}[0]{\mathbf{p}}
\newcommand{\Bq}[0]{\mathbf{q}}
\newcommand{\Br}[0]{\mathbf{r}}
\newcommand{\Bs}[0]{\mathbf{s}}
\newcommand{\Bt}[0]{\mathbf{t}}
\newcommand{\Bu}[0]{\mathbf{u}}
\newcommand{\Bv}[0]{\mathbf{v}}
\newcommand{\Bw}[0]{\mathbf{w}}
\newcommand{\Bx}[0]{\mathbf{x}}
\newcommand{\By}[0]{\mathbf{y}}
\newcommand{\Bz}[0]{\mathbf{z}}
\newcommand{\BA}[0]{\mathbf{A}}
\newcommand{\BB}[0]{\mathbf{B}}
\newcommand{\BC}[0]{\mathbf{C}}
\newcommand{\BD}[0]{\mathbf{D}}
\newcommand{\BE}[0]{\mathbf{E}}
\newcommand{\BF}[0]{\mathbf{F}}
\newcommand{\BG}[0]{\mathbf{G}}
\newcommand{\BH}[0]{\mathbf{H}}
\newcommand{\BI}[0]{\mathbf{I}}
\newcommand{\BJ}[0]{\mathbf{J}}
\newcommand{\BK}[0]{\mathbf{K}}
\newcommand{\BL}[0]{\mathbf{L}}
\newcommand{\BM}[0]{\mathbf{M}}
\newcommand{\BN}[0]{\mathbf{N}}
\newcommand{\BO}[0]{\mathbf{O}}
\newcommand{\BP}[0]{\mathbf{P}}
\newcommand{\BQ}[0]{\mathbf{Q}}
\newcommand{\BR}[0]{\mathbf{R}}
\newcommand{\BS}[0]{\mathbf{S}}
\newcommand{\BT}[0]{\mathbf{T}}
\newcommand{\BU}[0]{\mathbf{U}}
\newcommand{\BV}[0]{\mathbf{V}}
\newcommand{\BW}[0]{\mathbf{W}}
\newcommand{\BX}[0]{\mathbf{X}}
\newcommand{\BY}[0]{\mathbf{Y}}
\newcommand{\BZ}[0]{\mathbf{Z}}

\newcommand{\Bzero}[0]{\mathbf{0}}
\newcommand{\Btheta}[0]{\boldsymbol{\theta}}
\newcommand{\Btau}[0]{\boldsymbol{\tau}}
\newcommand{\Bomega}[0]{\boldsymbol{\omega}}

%
% shorthand for unit vectors
%
\newcommand{\acap}[0]{\hat{\Ba}}
\newcommand{\bcap}[0]{\hat{\Bb}}
\newcommand{\ccap}[0]{\hat{\Bc}}
\newcommand{\dcap}[0]{\hat{\Bd}}
\newcommand{\ecap}[0]{\hat{\Be}}
\newcommand{\fcap}[0]{\hat{\Bf}}
\newcommand{\gcap}[0]{\hat{\Bg}}
\newcommand{\hcap}[0]{\hat{\Bh}}
\newcommand{\icap}[0]{\hat{\Bi}}
\newcommand{\jcap}[0]{\hat{\Bj}}
\newcommand{\kcap}[0]{\hat{\Bk}}
\newcommand{\lcap}[0]{\hat{\Bl}}
\newcommand{\mcap}[0]{\hat{\Bm}}
\newcommand{\ncap}[0]{\hat{\Bn}}
\newcommand{\ocap}[0]{\hat{\Bo}}
\newcommand{\pcap}[0]{\hat{\Bp}}
\newcommand{\qcap}[0]{\hat{\Bq}}
\newcommand{\rcap}[0]{\hat{\Br}}
\newcommand{\scap}[0]{\hat{\Bs}}
\newcommand{\tcap}[0]{\hat{\Bt}}
\newcommand{\ucap}[0]{\hat{\Bu}}
\newcommand{\vcap}[0]{\hat{\Bv}}
\newcommand{\wcap}[0]{\hat{\Bw}}
\newcommand{\xcap}[0]{\hat{\Bx}}
\newcommand{\ycap}[0]{\hat{\By}}
\newcommand{\zcap}[0]{\hat{\Bz}}
\newcommand{\thetacap}[0]{\hat{\Btheta}}

%
% to write R^n and C^n in a distinguishable fashion.  Perhaps change this
% to the double lined characters upon figuring out how to do so.
%
\newcommand{\C}[1]{$\mathbb{C}^{#1}$}
\newcommand{\R}[1]{$\mathbb{R}^{#1}$}

%
% various generally useful helpers
%

% derivative of #1 wrt. #2:
\newcommand{\D}[2] {\frac {d#2} {d#1}}

\newcommand{\inv}[1]{\frac{1}{#1}}
\newcommand{\cross}[0]{\times}

\newcommand{\abs}[1]{\lvert{#1}\rvert}
\newcommand{\norm}[1]{\lVert{#1}\rVert}
\newcommand{\innerprod}[2]{\langle{#1}, {#2}\rangle}
\newcommand{\dotprod}[2]{{#1} \cdot {#2}}
\newcommand{\bdotprod}[2]{\left({#1} \cdot {#2}\right)}
\newcommand{\crossprod}[2]{{#1} \cross {#2}}
\newcommand{\tripleprod}[3]{\dotprod{\left(\crossprod{#1}{#2}\right)}{#3}}

\DeclareMathOperator{\Proj}{Proj}
\DeclareMathOperator{\Span}{span}
\DeclareMathOperator{\Sgn}{sgn}
\DeclareMathOperator{\Area}{Area}
\DeclareMathOperator{\Volume}{Volume}

%
% A few miscellaneous things specific to this document
%
\newcommand{\crossop}[1]{\crossprod{#1}{}}

% R2 vector.
\newcommand{\VectorTwo}[2]{
\begin{bmatrix}
 {#1} \\
 {#2}
\end{bmatrix}
}

\newcommand{\VectorN}[1]{
\begin{bmatrix}
{#1}_1 \\
{#1}_2 \\
\vdots \\
{#1}_N \\
\end{bmatrix}
}

\newcommand{\DETuvij}[4]{
\begin{vmatrix}
 {#1}_{#3} & {#1}_{#4} \\
 {#2}_{#3} & {#2}_{#4}
\end{vmatrix}
}

\newcommand{\DETuvwijk}[6]{
\begin{vmatrix}
 {#1}_{#4} & {#1}_{#5} & {#1}_{#6} \\
 {#2}_{#4} & {#2}_{#5} & {#2}_{#6} \\
 {#3}_{#4} & {#3}_{#5} & {#3}_{#6}
\end{vmatrix}
}

\newcommand{\DETuvwxijkl}[8]{
\begin{vmatrix}
 {#1}_{#5} & {#1}_{#6} & {#1}_{#7} & {#1}_{#8} \\
 {#2}_{#5} & {#2}_{#6} & {#2}_{#7} & {#2}_{#8} \\
 {#3}_{#5} & {#3}_{#6} & {#3}_{#7} & {#3}_{#8} \\
 {#4}_{#5} & {#4}_{#6} & {#4}_{#7} & {#4}_{#8} \\
\end{vmatrix}
}

%\newcommand{\DETuvwxyijklm}[10]{
%\begin{vmatrix}
% {#1}_{#6} & {#1}_{#7} & {#1}_{#8} & {#1}_{#9} & {#1}_{#10} \\
% {#2}_{#6} & {#2}_{#7} & {#2}_{#8} & {#2}_{#9} & {#2}_{#10} \\
% {#3}_{#6} & {#3}_{#7} & {#3}_{#8} & {#3}_{#9} & {#3}_{#10} \\
% {#4}_{#6} & {#4}_{#7} & {#4}_{#8} & {#4}_{#9} & {#4}_{#10} \\
% {#5}_{#6} & {#5}_{#7} & {#5}_{#8} & {#5}_{#9} & {#5}_{#10}
%\end{vmatrix}
%}

% R3 vector.
\newcommand{\VectorThree}[3]{
\begin{bmatrix}
 {#1} \\
 {#2} \\
 {#3}
\end{bmatrix}
}



\author{Peeter Joot}
\email{peeter.joot@gmail.com}

%\documentclass[]{eliblogwidescreen}

\usepackage{amsmath}
\usepackage{mathpazo}

%
% shorthand for bold symbols, convenient for vectors and matrices
%
\newcommand{\Ba}[0]{\mathbf{a}}
\newcommand{\Bb}[0]{\mathbf{b}}
\newcommand{\Bc}[0]{\mathbf{c}}
\newcommand{\Bd}[0]{\mathbf{d}}
\newcommand{\Be}[0]{\mathbf{e}}
\newcommand{\Bf}[0]{\mathbf{f}}
\newcommand{\Bg}[0]{\mathbf{g}}
\newcommand{\Bh}[0]{\mathbf{h}}
\newcommand{\Bi}[0]{\mathbf{i}}
\newcommand{\Bj}[0]{\mathbf{j}}
\newcommand{\Bk}[0]{\mathbf{k}}
\newcommand{\Bl}[0]{\mathbf{l}}
\newcommand{\Bm}[0]{\mathbf{m}}
\newcommand{\Bn}[0]{\mathbf{n}}
\newcommand{\Bo}[0]{\mathbf{o}}
\newcommand{\Bp}[0]{\mathbf{p}}
\newcommand{\Bq}[0]{\mathbf{q}}
\newcommand{\Br}[0]{\mathbf{r}}
\newcommand{\Bs}[0]{\mathbf{s}}
\newcommand{\Bt}[0]{\mathbf{t}}
\newcommand{\Bu}[0]{\mathbf{u}}
\newcommand{\Bv}[0]{\mathbf{v}}
\newcommand{\Bw}[0]{\mathbf{w}}
\newcommand{\Bx}[0]{\mathbf{x}}
\newcommand{\By}[0]{\mathbf{y}}
\newcommand{\Bz}[0]{\mathbf{z}}
\newcommand{\BA}[0]{\mathbf{A}}
\newcommand{\BB}[0]{\mathbf{B}}
\newcommand{\BC}[0]{\mathbf{C}}
\newcommand{\BD}[0]{\mathbf{D}}
\newcommand{\BE}[0]{\mathbf{E}}
\newcommand{\BF}[0]{\mathbf{F}}
\newcommand{\BG}[0]{\mathbf{G}}
\newcommand{\BH}[0]{\mathbf{H}}
\newcommand{\BI}[0]{\mathbf{I}}
\newcommand{\BJ}[0]{\mathbf{J}}
\newcommand{\BK}[0]{\mathbf{K}}
\newcommand{\BL}[0]{\mathbf{L}}
\newcommand{\BM}[0]{\mathbf{M}}
\newcommand{\BN}[0]{\mathbf{N}}
\newcommand{\BO}[0]{\mathbf{O}}
\newcommand{\BP}[0]{\mathbf{P}}
\newcommand{\BQ}[0]{\mathbf{Q}}
\newcommand{\BR}[0]{\mathbf{R}}
\newcommand{\BS}[0]{\mathbf{S}}
\newcommand{\BT}[0]{\mathbf{T}}
\newcommand{\BU}[0]{\mathbf{U}}
\newcommand{\BV}[0]{\mathbf{V}}
\newcommand{\BW}[0]{\mathbf{W}}
\newcommand{\BX}[0]{\mathbf{X}}
\newcommand{\BY}[0]{\mathbf{Y}}
\newcommand{\BZ}[0]{\mathbf{Z}}

\newcommand{\Bzero}[0]{\mathbf{0}}
\newcommand{\Btheta}[0]{\boldsymbol{\theta}}
\newcommand{\Btau}[0]{\boldsymbol{\tau}}
\newcommand{\Bomega}[0]{\boldsymbol{\omega}}

%
% shorthand for unit vectors
%
\newcommand{\acap}[0]{\hat{\Ba}}
\newcommand{\bcap}[0]{\hat{\Bb}}
\newcommand{\ccap}[0]{\hat{\Bc}}
\newcommand{\dcap}[0]{\hat{\Bd}}
\newcommand{\ecap}[0]{\hat{\Be}}
\newcommand{\fcap}[0]{\hat{\Bf}}
\newcommand{\gcap}[0]{\hat{\Bg}}
\newcommand{\hcap}[0]{\hat{\Bh}}
\newcommand{\icap}[0]{\hat{\Bi}}
\newcommand{\jcap}[0]{\hat{\Bj}}
\newcommand{\kcap}[0]{\hat{\Bk}}
\newcommand{\lcap}[0]{\hat{\Bl}}
\newcommand{\mcap}[0]{\hat{\Bm}}
\newcommand{\ncap}[0]{\hat{\Bn}}
\newcommand{\ocap}[0]{\hat{\Bo}}
\newcommand{\pcap}[0]{\hat{\Bp}}
\newcommand{\qcap}[0]{\hat{\Bq}}
\newcommand{\rcap}[0]{\hat{\Br}}
\newcommand{\scap}[0]{\hat{\Bs}}
\newcommand{\tcap}[0]{\hat{\Bt}}
\newcommand{\ucap}[0]{\hat{\Bu}}
\newcommand{\vcap}[0]{\hat{\Bv}}
\newcommand{\wcap}[0]{\hat{\Bw}}
\newcommand{\xcap}[0]{\hat{\Bx}}
\newcommand{\ycap}[0]{\hat{\By}}
\newcommand{\zcap}[0]{\hat{\Bz}}
\newcommand{\thetacap}[0]{\hat{\Btheta}}

%
% to write R^n and C^n in a distinguishable fashion.  Perhaps change this
% to the double lined characters upon figuring out how to do so.
%
\newcommand{\C}[1]{$\mathbb{C}^{#1}$}
\newcommand{\R}[1]{$\mathbb{R}^{#1}$}

%
% various generally useful helpers
%

% derivative of #1 wrt. #2:
\newcommand{\D}[2] {\frac {d#2} {d#1}}

\newcommand{\inv}[1]{\frac{1}{#1}}
\newcommand{\cross}[0]{\times}

\newcommand{\abs}[1]{\lvert{#1}\rvert}
\newcommand{\norm}[1]{\lVert{#1}\rVert}
\newcommand{\innerprod}[2]{\langle{#1}, {#2}\rangle}
\newcommand{\dotprod}[2]{{#1} \cdot {#2}}
\newcommand{\bdotprod}[2]{\left({#1} \cdot {#2}\right)}
\newcommand{\crossprod}[2]{{#1} \cross {#2}}
\newcommand{\tripleprod}[3]{\dotprod{\left(\crossprod{#1}{#2}\right)}{#3}}

\DeclareMathOperator{\Proj}{Proj}
\DeclareMathOperator{\Span}{span}
\DeclareMathOperator{\Sgn}{sgn}
\DeclareMathOperator{\Area}{Area}
\DeclareMathOperator{\Volume}{Volume}

%
% A few miscellaneous things specific to this document
%
\newcommand{\crossop}[1]{\crossprod{#1}{}}

% R2 vector.
\newcommand{\VectorTwo}[2]{
\begin{bmatrix}
 {#1} \\
 {#2}
\end{bmatrix}
}

\newcommand{\VectorN}[1]{
\begin{bmatrix}
{#1}_1 \\
{#1}_2 \\
\vdots \\
{#1}_N \\
\end{bmatrix}
}

\newcommand{\DETuvij}[4]{
\begin{vmatrix}
 {#1}_{#3} & {#1}_{#4} \\
 {#2}_{#3} & {#2}_{#4}
\end{vmatrix}
}

\newcommand{\DETuvwijk}[6]{
\begin{vmatrix}
 {#1}_{#4} & {#1}_{#5} & {#1}_{#6} \\
 {#2}_{#4} & {#2}_{#5} & {#2}_{#6} \\
 {#3}_{#4} & {#3}_{#5} & {#3}_{#6}
\end{vmatrix}
}

\newcommand{\DETuvwxijkl}[8]{
\begin{vmatrix}
 {#1}_{#5} & {#1}_{#6} & {#1}_{#7} & {#1}_{#8} \\
 {#2}_{#5} & {#2}_{#6} & {#2}_{#7} & {#2}_{#8} \\
 {#3}_{#5} & {#3}_{#6} & {#3}_{#7} & {#3}_{#8} \\
 {#4}_{#5} & {#4}_{#6} & {#4}_{#7} & {#4}_{#8} \\
\end{vmatrix}
}

%\newcommand{\DETuvwxyijklm}[10]{
%\begin{vmatrix}
% {#1}_{#6} & {#1}_{#7} & {#1}_{#8} & {#1}_{#9} & {#1}_{#10} \\
% {#2}_{#6} & {#2}_{#7} & {#2}_{#8} & {#2}_{#9} & {#2}_{#10} \\
% {#3}_{#6} & {#3}_{#7} & {#3}_{#8} & {#3}_{#9} & {#3}_{#10} \\
% {#4}_{#6} & {#4}_{#7} & {#4}_{#8} & {#4}_{#9} & {#4}_{#10} \\
% {#5}_{#6} & {#5}_{#7} & {#5}_{#8} & {#5}_{#9} & {#5}_{#10}
%\end{vmatrix}
%}

% R3 vector.
\newcommand{\VectorThree}[3]{
\begin{bmatrix}
 {#1} \\
 {#2} \\
 {#3}
\end{bmatrix}
}



\author{Peeter Joot}
\email{peeter.joot@gmail.com}


\chapter{Helium atom ground state energy estimation notes.}
\label{chap:variationalHelium}
%\useCCL
\blogpage{http://sites.google.com/site/peeterjoot/math2011/variationalHelium.pdf}
\date{Sept 29, 2011}
\revisionInfo{variationalHelium.tex}

%\beginArtWithToc
\beginArtNoToc

\section{Motivation.}

In \S 24.2.1 of the text \cite{desai2009quantum} is an expectation value calculation associated with the Helium atom.  One of the equations (24.76) seems wrong (according to hand calculation as shown below and according to Mathematica).  Is there another compensating error somewhere?  Here I work the entire calculation in detail to attempt to find this.

\section{Guts}

We start with

\begin{align*}
\expectation{\frac{e^2}{\Abs{\Br_1 - \Br_2}}}
&=
\left( \frac{Z^3}{\pi a_0^3}\right)^2 e^2
\int d^3 k d^3 r_1 d^3 r_2 \inv{2 \pi^2 k^2} e^{i \Bk \cdot (\Br_1 - \Br_2) } e^{ -2 Z (r_1 + r_2)/a_0} \\
&= 
\left( \frac{Z^3}{\pi a_0^3}\right)^2 e^2
\inv{2 \pi^2} 
\int d^3 k \inv{k^2}
\int d^3 r_1 
e^{i \Bk \cdot \Br_1 } e^{ -2 Z r_1 /a_0} 
\int 
d^3 r_2 
e^{-i \Bk \cdot \Br_2 } e^{ -2 Z r_2/a_0} \\
\end{align*}

To evaluate the two last integrals, I figure the author has aligned the axis for the $d^3 r_1$ volume elements to make the integrals easier.  Specifically, for the first so that $\Bk \cdot \Br_1 = k r_1 \cos\theta$, so the integral takes the form

\begin{align*}
\int 
d^3 r_1 
e^{i \Bk \cdot \Br_1 } e^{ -2 Z r_1 /a_0} 
&=
-\int 
r_1^2 d r_1 d\phi d(\cos\theta)
e^{i k r_1 \cos\theta } e^{ -2 Z r_1 /a_0} \\
&=
- 2 \pi \int_{r=0}^\infty \int_{u=1}^{-1}
r^2 dr du
e^{i k r u } e^{ -2 Z r /a_0} \\
&=
- 2 \pi \int_{r=0}^\infty 
r^2 dr 
\inv{i k r} \left( e^{-i k r } - e^{i k r} \right) e^{ -2 Z r /a_0} \\
&=
\frac{4 \pi}{k } \int_{r=0}^\infty 
r dr 
\inv{2i} \left( e^{i k r } - e^{-i k r} \right) e^{ -2 Z r /a_0} \\
&=
\frac{4 \pi}{k } \int_{r=0}^\infty r dr \sin(k r) e^{ -2 Z r /a_0} \\
\end{align*}

For this last, Mathematica gives me (24.75) from the text

\begin{equation}\label{eqn:variationalHelium:10}
\int 
d^3 r_1 
e^{i \Bk \cdot \Br_1 } e^{ -2 Z r_1 /a_0} 
=
\frac{ 16 \pi Z a_0^3 }{(k^2 a_0^2 + 4 Z^2)^2}
\end{equation}

For the second integral, if we align the axis so that $-\Bk \cdot \Br_2 = k r \cos\theta$ and repeat, then we have

\begin{align*}
\expectation{\frac{e^2}{\Abs{\Br_1 - \Br_2}}}
&=
\left( \frac{Z^3}{\pi a_0^3}\right)^2 e^2
\inv{2 \pi^2} 
16^2 \pi^2 Z^2 a_0^6 
\int d^3 k \inv{k^2}
\frac{ 1 }{(k^2 a_0^2 + 4 Z^2)^4} \\
&=
\frac{128 Z^8}{\pi^2 } e^2
\int dk d\Omega 
\frac{ 1 }{(k^2 a_0^2 + 4 Z^2)^4} \\
&=
\frac{512 Z^8}{\pi} e^2
\int dk 
\frac{ 1 }{(k^2 a_0^2 + 4 Z^2)^4} \\
&=
\frac{512 Z^8}{\pi} e^2
\int dk 
\frac{ 1 }{(k^2 a_0^2 + 4 Z^2)^4} \\
\end{align*}

With $k a_0 = 2 Z \kappa$ this is

\begin{align*}
\expectation{\frac{e^2}{\Abs{\Br_1 - \Br_2}}}
&=
\frac{512 Z^8}{\pi} e^2
\int d\kappa 
\frac{ 2 Z }{a_0}
\frac{ 1 }{(2 Z)^8 (\kappa^2 + 1)^4} \\
&=
\frac{4 Z}{\pi a_0} e^2
\int d\kappa 
\frac{ 1 }{(\kappa^2 + 1)^4} \\
\end{align*}

Here I note that 

\begin{align*}
\frac{d}{d\kappa}
\frac{-1}{ 3 (1+x)^3 }
=
\inv{ (1+x)^4 }
\end{align*}

so the definite integral has the value
\begin{align*}
\int_0^\infty d\kappa 
\frac{ 1 }{ (\kappa^2 + 1)^4} 
=
\frac{-1}{ 3 (1+\infty)^3 }
-
\frac{-1}{ 3 (1+0)^3 }
= \inv{3}
\end{align*}

(not $5 \pi/3$ as claimed in the text).  This gives us

\begin{align*}
\expectation{\frac{e^2}{\Abs{\Br_1 - \Br_2}}}
&=
\frac{4 Z e^2} { \pi a_0 } 
\int d\kappa 
\frac{ 1 }{(\kappa^2 + 1)^4} \\
&=
\frac{4 Z e^2} { 3 \pi a_0 }  \\
&\approx 0.424 \frac{Z e^2 }{a_0} \ne \frac{5}{8} \frac{Z e^2 }{a_0}
\end{align*}

So, no compensating error is found, yet the end result of the calculation, which requires the $5/8$ result matches with the results obtained other ways (as in problem set III).  How would this be accounted for?  Is there an error above?

\EndArticle
%\EndNoBibArticle

%
% Copyright � 2012 Peeter Joot.  All Rights Reserved.
% Licenced as described in the file LICENSE under the root directory of this GIT repository.
%

\label{chap:variationHarmonicOscillator}

%\blogpage{http://sites.google.com/site/peeterjoot/math2011/variationHarmonicOscillator.pdf}
%\date{Oct 3, 2011}

\subsubsection{Recap.  Variational method to find the ground state energy}

Problem 3 of \S 24.4 in the text \citep{desai2009quantum} is an interesting one.  It asks to use the variational method to find the ground state energy of a one dimensional harmonic oscillator Hamiltonian.

Somewhat unexpectedly, once I take derivatives equate to zero, I find that the variational parameter beta becomes imaginary?

I tried this twice on paper and pencil, both times getting the same thing.  This seems like a noteworthy problem, and one worth reflecting on a bit.


\subsubsection{Recap.  The variational method}

Given any, not necessarily normalized wavefunction, with a series representation specified using the energy eigenvectors for the space

\begin{equation}\label{eqn:variationHarmonicOscillator:10}
\ket{\psi} = \sum_m c_{m} \ket{\psi_m},
\end{equation}

where

\begin{equation}\label{eqn:variationHarmonicOscillator:30}
H \ket{\psi_m} = E_m \ket{\psi_m},
\end{equation}

and

\begin{equation}\label{eqn:variationHarmonicOscillator:50}
\braket{\psi_m}{\psi_n} = \delta_{mn}.
\end{equation}

We can perform an energy expectation calculation with respect to this more general state

\begin{equation}\label{eqn:variationHarmonicOscillator:491}
\begin{aligned}
\bra{\psi} H \ket{\psi}
&=
\sum_m c_{m}^\conj \bra{\psi_m}
H
\sum_n c_{n} \ket{\psi_n} \\
&=
\sum_m c_{m}^\conj \bra{\psi_m}
\sum_n c_{n} E_n \ket{\psi_n} \\
&=
\sum_{m,n} c_{m}^\conj c_n E_n \braket{\psi_m}{\psi_n} \\
&
\sum_{m} \Abs{c_{m}}^2 E_m \\
&\ge
\sum_{m} \Abs{c_{m}}^2 E_0 \\
&=
E_0 \braket{\psi}{\psi}
\end{aligned}
\end{equation}

This allows us to form an estimate of the ground state energy for the system, by using any state vector formed from a superposition of energy eigenstates, by simply calculating

\begin{equation}\label{eqn:variationHarmonicOscillator:70}
E_0 \le \frac{\bra{\psi} H \ket{\psi}}{ \braket{\psi}{\psi} }.
\end{equation}

One of the examples in the text is to use this to find an approximation of the ground state energy for the Helium atom Hamiltonian

\begin{equation}\label{eqn:variationHarmonicOscillator:90}
H =
-\frac{\Hbar^2}{2m} \left(
\spacegrad_1^2
+\spacegrad_1^2\right) - 2 e^2 \left( \inv{r_1} + \inv{r_2} \right) + \frac{e^2}{\Abs{\Br_1 - \Br_2}}.
\end{equation}

This calculation is performed using a trial function that was a solution of the interaction free Hamiltonian

\begin{equation}\label{eqn:variationHarmonicOscillator:110}
\phi = \frac{Z^3}{\pi a_0^3} e^{-Z (r_1 + r_2)/a_0 }.
\end{equation}

This is despite the fact that this is not a solution to the interaction Hamiltonian.  The end result ends up being pretty close to the measured value (although there is a pesky error in the book that appears to require a compensating error somewhere else).

Part of the variational technique used in that problem, is to allow Z to vary, and then once the normalized expectation is computed, set the derivative of that equal to zero to calculate the trial wavefunction as a parameter of Z that has the lowest energy eigenstate for a function of that form.  We find considering the Harmonic oscillator that this final variation does not necessarily produce meaningful results.

\subsubsection{The Harmonic oscillator variational problem}

The problem asks for the use of the trial wavefunction

\begin{equation}\label{eqn:variationHarmonicOscillator:130}
\phi = e^{-\beta \Abs{x}},
\end{equation}

to perform the variational calculation above for the Harmonic oscillator Hamiltonian, which has the one dimensional position space representation

\begin{equation}\label{eqn:variationHarmonicOscillator:150}
H = -\frac{\Hbar^2}{2m} \frac{d^2}{dx^2} + \inv{2} m \omega^2 x^2.
\end{equation}

We can find the normalization easily

\begin{equation}\label{eqn:variationHarmonicOscillator:511}
\begin{aligned}
\braket{\phi}{\phi}
&= \int_{-\infty}^\infty e^{- 2 \beta \Abs{x}} dx \\
&= 2 \inv{2 \beta} \int_{0}^\infty e^{- 2 \beta x} 2 \beta dx \\
&= 2 \inv{2 \beta} \int_{0}^\infty e^{- u} du \\
&= \inv{\beta}
\end{aligned}
\end{equation}

Using integration by parts, we find for the energy expectation

\begin{equation}\label{eqn:variationHarmonicOscillator:531}
\begin{aligned}
\bra{\phi} H \ket{\phi}
&=
\int_{-\infty}^\infty dx
e^{- \beta \Abs{x}}
\left( -\frac{\Hbar^2}{2m} \frac{d^2}{dx^2} + \inv{2} m \omega^2 x^2 \right)
e^{- \beta \Abs{x}}  \\
&=
\lim_{\epsilon \rightarrow 0}
\left(
\int_{-\infty}^{-\epsilon}
+
\int_{-\epsilon}^\epsilon
+
\int_{\epsilon}^\infty
\right)
dx
e^{- \beta \Abs{x}}
\left( -\frac{\Hbar^2}{2m} \frac{d^2}{dx^2} + \inv{2} m \omega^2 x^2 \right)
e^{- \beta \Abs{x}}  \\
&=
2 \int_{0}^\infty dx
e^{ - 2 \beta x }
\left( -\frac{\Hbar^2 \beta^2}{2m} + \inv{2} m \omega^2 x^2 \right)
-
\frac{\Hbar^2}{2m}
\lim_{\epsilon \rightarrow 0}
\int_{-\epsilon}^\epsilon
dx
e^{- \beta \Abs{x}}
\frac{d^2}{dx^2}
e^{- \beta \Abs{x}}
\end{aligned}
\end{equation}

The first integral we can do

\begin{equation}\label{eqn:variationHarmonicOscillator:551}
\begin{aligned}
2 \int_{0}^\infty dx
e^{- 2 \beta x}
\left( -\frac{\Hbar^2 \beta^2}{2m} + \inv{2} m \omega^2 x^2 \right)
&=
-\frac{\Hbar^2 \beta^2}{m}
\int_{0}^\infty dx e^{- 2 \beta x}
+ m \omega^2
 \int_{0}^\infty dx x^2 e^{- 2 \beta x}  \\
&=
-\frac{\Hbar^2 \beta}{2 m}
\int_{0}^\infty du e^{- u}
+ \frac{m \omega^2 }{8 \beta^3}
 \int_{0}^\infty du u^2 e^{- u}  \\
&=
-\frac{\beta \Hbar^2}{2m} + \frac{m \omega^2}{4 \beta^3}
\end{aligned}
\end{equation}

A naive evaluation of this integral requires the origin to be avoided where the derivative of \(\Abs{x}\) becomes undefined.  This also provides a nice way to evaluate this integral because we can double the integral and half the range, eliminating the absolute value.

However, can we assume that the remaining integral is zero?

I thought that we could, but the end result is curious.  I also verified my calculation symbolically in \nbref{24.4.3_attempt_with_mathematica.nb}, but found that Mathematica required some special hand holding to deal with the origin.  Initially I coded this by avoiding the origin as above, but later switched to \(\Abs{x} = \sqrt{x^2}\) which Mathematica treats more gracefully.
%\href{https://github.com/peeterjoot/physicsplay/tree/master/notes/phy456/24.4.3 attempt with mathematica.nb}{using Mathematica}.

Without that last integral, involving our singular \(\Abs{x}'\) and \(\Abs{x}''\) terms, our ground state energy estimation, parameterized by \(\beta\) is

\begin{equation}\label{eqn:variationHarmonicOscillator:190}
E[\beta] = -\frac{\beta^2 \Hbar^2}{2m} + \frac{m \omega^2}{4 \beta^2}.
\end{equation}

Observe that if we set the derivative of this equal to zero to find the ``best'' beta associated with this trial function

\begin{equation}\label{eqn:variationHarmonicOscillator:210}
0 = \PD{\beta}{E} = -\frac{\beta \Hbar^2}{2m} - \frac{m \omega^2}{2 \beta^3}
\end{equation}

we find that the parameter beta that best minimizes this ground state energy function is complex with value

\begin{equation}\label{eqn:variationHarmonicOscillator:230}
\beta^2 = \pm \frac{i m \omega}{\sqrt{2} \Hbar}.
\end{equation}

It appears at first glance that we can not minimize \eqnref{eqn:variationHarmonicOscillator:190} to find a best ground state energy estimate associated with the trial function \eqnref{eqn:variationHarmonicOscillator:130}.  We do however, know the exact ground state energy \(\Hbar \omega/2\) for the Harmonic oscillator.  Is is possible to show that for all \(\beta^2\) we have

\begin{equation}\label{eqn:variationHarmonicOscillator:250}
\frac{\Hbar \omega}{2} \le -\frac{\beta^2 \Hbar^2}{2m} + \frac{m \omega^2}{4 \beta^2}
\end{equation}

?  This inequality would be expected if we can assume that the trial wavefunction has a Fourier series representation utilizing the actual energy eigenfunctions for the system.

The resolution to this question is avoided once we include the singularity.  This is explored in the last part of these notes.

\subsubsection{Is our trial function representable?}

I thought perhaps that since the trial wave function for this problem lies outside the span of the Hilbert space that describes the solutions to the Harmonic oscillator.  Another thing of possible interest is the trouble near the origin for this wave function, when operated on by \(P^2/2m\), and this has been (incorrectly assumed to have zero contribution above).

I had initially thought that part of the value of this variational method was that we can use it despite not even knowing what the exact solution is (and in the case of the Helium atom, I believe it was stated in class that an exact closed form solution is not even known).  This makes me wonder what restrictions must be imposed on the trial solutions to get a meaningful answer from the variational calculation?

Suppose that the trial wavefunction is not representable in the solution space.  If that is the case, we need to adjust the treatment to account for that.  Suppose we have

\begin{equation}\label{eqn:variationHarmonicOscillator:270}
\ket{\phi} = \sum_n c_n \ket{\psi_n} + c_\perp \ket{\psi_\perp}.
\end{equation}

where \(\ket{\psi_\perp}\) is unknown, and presumed not orthogonal to any of the energy eigenkets.  We can still calculate the norm of the trial function

\begin{equation}\label{eqn:variationHarmonicOscillator:571}
\begin{aligned}
\braket{\phi}{\phi}
&=
\sum_{n,m} \braket{ c_n \psi_n + c_\perp \psi_\perp}{ c_m \psi_m + c_\perp \psi_\perp} \\
&=
\sum_n \Abs{c_n}^2
+ c_n^\conj c_\perp
\braket{\psi_n}{\psi_\perp}
+ c_n c_\perp^\conj \braket{\psi_\perp}{\psi_n}
+ \Abs{c_\perp}^2
\braket{\psi_\perp}{\psi_\perp} \\
&=
\braket{\psi_\perp}{\psi_\perp} +
\sum_n \Abs{c_n}^2 + 2 \Real \left(c_n^\conj c_\perp \braket{\psi_n}{\psi_\perp} \right).
\end{aligned}
\end{equation}

Similarly we can calculate the energy expectation for this unnormalized state and find

\begin{equation}\label{eqn:variationHarmonicOscillator:591}
\begin{aligned}
\bra{\phi} H \ket{\phi}
&=
\sum_{n,m} \bra{ c_n \psi_n + c_\perp \psi_\perp} H \ket{ c_m \psi_m + c_\perp \psi_\perp} \\
&=
\sum_n \Abs{c_n}^2 E_n
+ c_n^\conj c_\perp E_n
\braket{\psi_n}{\psi_\perp}
+ c_n c_\perp^\conj E_n \braket{\psi_\perp}{\psi_n}
+ \Abs{c_\perp}^2
\bra{\psi_\perp} H \ket{\psi_\perp}
%&=
%\braket{\psi_\perp} H {\psi_\perp} +
%\sum_n \Abs{c_n}^2 + 2 \Real \left(c_n^\conj c_\perp \braket{\psi_n}{\psi_\perp} \right).
\end{aligned}
\end{equation}

Our normalized energy expectation is therefore the considerably messier

\begin{equation}\label{eqn:variationHarmonicOscillator:290}
\begin{aligned}
\frac{\bra{\phi} H \ket{\phi}}{
\braket{\phi}{\phi}
}
&=
\frac{
\sum_n \Abs{c_n}^2 E_n
+ c_n^\conj c_\perp E_n
\braket{\psi_n}{\psi_\perp}
+ c_n c_\perp^\conj E_n \braket{\psi_\perp}{\psi_n}
+ \Abs{c_\perp}^2
\bra{\psi_\perp} H \ket{\psi_\perp}
}
{
\braket{\psi_\perp}{\psi_\perp} +
\sum_m \Abs{c_m}^2 + 2 \Real \left(c_m^\conj c_\perp \braket{\psi_m}{\psi_\perp} \right)
} \\
&\ge
\frac{
\sum_n \Abs{c_n}^2 E_0
+ c_n^\conj c_\perp E_n
\braket{\psi_n}{\psi_\perp}
+ c_n c_\perp^\conj E_n \braket{\psi_\perp}{\psi_n}
+ \Abs{c_\perp}^2
\bra{\psi_\perp} H \ket{\psi_\perp}
}
{
\braket{\psi_\perp}{\psi_\perp} +
\sum_m \Abs{c_m}^2 + 2 \Real \left(c_m^\conj c_\perp \braket{\psi_m}{\psi_\perp} \right)
}
\end{aligned}
\end{equation}

With a requirement to include the perpendicular cross terms the norm does not just cancel out, leaving us with a clean estimation of the ground state energy.  In order to utilize this variational method, we implicitly have an assumption that the \(\braket{\psi_\perp}{\psi_\perp}\) and \(\braket{\psi_m}{\psi_\perp}\) terms in the denominator are sufficiently small that they can be neglected.

\subsubsection{Calculating the Fourier terms}

In order to see how much a problem representing this trial function in the Harmonic oscillator wavefunction solution space, we can just calculate the Fourier fit.

Our first few basis functions, with \(\alpha = \sqrt{m \omega/\Hbar}\) are

\begin{equation}\label{eqn:variationHarmonicOscillator:611}
\begin{aligned}
u_0 &= \sqrt{\frac{\alpha}{\sqrt{\pi}}} e^{-\alpha^2 x^2/2} \\
u_1 &= \sqrt{\frac{\alpha}{2 \sqrt{\pi}}} (2 \alpha x) e^{-\alpha^2 x^2/2} \\
u_2 &= \sqrt{\frac{\alpha}{8 \sqrt{\pi}}} (4 \alpha^2 x^2 - 2) e^{-\alpha^2 x^2/2}
\end{aligned}
\end{equation}

In general our wavefunctions are

\begin{equation}\label{eqn:variationHarmonicOscillator:631}
\begin{aligned}
u_n &= N_n H_n(\alpha x) e^{-\alpha^2 x^2/2} \\
N_n &= \sqrt{
\frac{\alpha}{\sqrt{\pi} 2^n n!}
} \\
H_n(\eta) &= (-1)^n e^{\eta^2} \frac{d^n}{d\eta^n} e^{-\eta^2}
\end{aligned}
\end{equation}

From which we find

\begin{equation}\label{eqn:variationHarmonicOscillator:471}
\psi(x) = e^{-\alpha^2 x^2/2} (N_n)^2 H_n(\alpha x) \int_{-\infty}^\infty H_n(\alpha x) e^{-\alpha^2 x^2/2} \psi(x) dx
\end{equation}

Our wave function, with \(\beta=1\) is plotted in \cref{fig:variationHarmonicOscillator:expMinusBetsAbsX}

\imageFigure{../../figures/phy456/expMinusBetsAbsX}{Exponential trial function with absolute exponential die off}{fig:variationHarmonicOscillator:expMinusBetsAbsX}{0.2}

The zeroth order fitting using the Gaussian exponential is found to be

\begin{equation}\label{eqn:variationHarmonicOscillator:310}
\psi_0(x) = \sqrt{2 \beta}
\erfc\left(\frac{\beta }{\sqrt{2} \alpha }\right)
e^{- \alpha^2 x^2/2 +\beta^2/(2 \alpha^2)}
\end{equation}

With \(\alpha = \beta = 1\), this is plotted in \cref{fig:variationHarmonicOscillator:expMinusBetsAbsXfirstOrderFitting} and can be seen to match fairly well

\imageFigure{../../figures/phy456/expMinusBetsAbsXfirstOrderFitting}{First ten orders, fitting harmonic oscillator wavefunctions to this trial function}{fig:variationHarmonicOscillator:expMinusBetsAbsXfirstOrderFitting}{0.2}

The higher order terms get small fast, but we can see in \cref{fig:variationHarmonicOscillator:expMinusBetsAbsXtenthOrderFitting}, where a tenth order fitting is depicted that it would take a number of them to get anything close to the sharp peak that we have in our exponential trial function.

\imageFigure{../../figures/phy456/expMinusBetsAbsXtenthOrderFitting}{Tenth order harmonic oscillator wavefunction fitting}{fig:variationHarmonicOscillator:expMinusBetsAbsXtenthOrderFitting}{0.2}

Note that all the brakets of even orders in \(n\) with the trial function are zero, which is why the tenth order approximation is only a sum of six terms.

Details for this harmonic oscillator wavefunction fitting can be found in \nbref{gaussian_fitting_for_abs_function.nb} can be found separately, calculated using a Mathematica worksheet.

The question of interest is why we can approximate the trial function so nicely (except at the origin) even with just a first order approximation (polynomial times Gaussian functions where the polynomials are Hankel functions), and we can get an exact value for the lowest energy state using the first order approximation of our trial function, why do we get garbage from the variational method, where enough terms are implicitly included that the peak should be sharp.  It must therefore be important to consider the origin, but how do we give some meaning to the derivative of the absolute value function?  The key (supplied when asking Professor Sipe in office hours for the course) is to express the absolute value function in terms of Heavyside step functions, for which the derivative can be identified as the delta function.

\subsubsection{Correcting, treating the origin this way}

Here is how we can express the absolute value function using the Heavyside step

\begin{equation}\label{eqn:variationHarmonicOscillator:330}
\Abs{x} = x \theta(x) - x \theta(-x),
\end{equation}

where the step function is zero for \(x < 0\) and one for \(x > 0\) as plotted in \cref{fig:variationHarmonicOscillator:stepFunction}.

\imageFigure{../../figures/phy456/stepFunction}{stepFunction}{fig:variationHarmonicOscillator:stepFunction}{0.2}

Expressed this way, with the identification \(\theta'(x) = \delta(x)\), we have for the derivative of the absolute value function

\begin{equation}\label{eqn:variationHarmonicOscillator:651}
\begin{aligned}
\Abs{x}'
&= x' \theta(x) - x' \theta(-x) + x \theta'(x) - x \theta'(-x) \\
&= \theta(x) - \theta(-x) + x \delta(x) + x \delta(-x) \\
&= \theta(x) - \theta(-x) + x \delta(x) + x \delta(x) \\
\end{aligned}
\end{equation}

Observe that we have our expected unit derivative for \(x > 0\), and \(-1\) derivative for \(x < 0\).  At the origin our \(\theta\) contributions vanish, and we are left with

\begin{equation}\label{eqn:variationHarmonicOscillator:671}
\begin{aligned}
\evalbar{\Abs{x}' }{x=0}
= 2 \evalbar{x \delta(x)}{x=0} \\
\end{aligned}
\end{equation}

We have got zero times infinity here, so how do we give meaning to this?  As with any delta functional, we have got to apply it to a well behaved (square integrable) test function \(f(x)\) and integrate.  Doing so we have

\begin{equation}\label{eqn:variationHarmonicOscillator:691}
\begin{aligned}
\int_{-\infty}^\infty dx \Abs{x}' f(x)
&= 2 \int_{-\infty}^\infty dx x \delta(x) f(x) \\
&= 2 (0) f(0)
\end{aligned}
\end{equation}

This equals zero for any well behaved test function \(f(x)\).  Since the delta function only picks up the contribution at the origin, we can therefore identify \(\Abs{x}'\) as zero at the origin.

Using the same technique, we can express our trial function in terms of steps

\begin{equation}\label{eqn:variationHarmonicOscillator:350}
\psi = e^{-\beta \Abs{x}} = \theta(x) e^{-\beta x} + \theta(-x) e^{\beta x}.
\end{equation}

This we can now take derivatives of, even at the origin, and find

\begin{equation}\label{eqn:variationHarmonicOscillator:711}
\begin{aligned}
\psi'
&= \theta'(x) e^{-\beta x} + \theta'(-x) e^{\beta x} -\beta \theta(x) e^{-\beta x} + \beta \theta(-x) e^{\beta x} \\
&= \delta(x) e^{-\beta x} - \delta(-x) e^{\beta x} -\beta \theta(x) e^{-\beta x} + \beta \theta(-x) e^{\beta x} \\
&= \cancel{\delta(x) e^{-\beta x} - \delta(x) e^{\beta x}} -\beta \theta(x) e^{-\beta x} + \beta \theta(-x) e^{\beta x} \\
&= \beta \left(
-\theta(x) e^{-\beta x} + \theta(-x) e^{\beta x}
\right)
\end{aligned}
\end{equation}

Taking second derivatives we find

\begin{equation}\label{eqn:variationHarmonicOscillator:731}
\begin{aligned}
\psi''
&= \beta \left(
-\theta'(x) e^{-\beta x} + \theta'(-x) e^{\beta x}
+\beta \theta(x) e^{-\beta x} + \beta \theta(-x) e^{\beta x}
\right) \\
&=
\beta \left(
-\delta(x) e^{-\beta x} - \delta(-x) e^{\beta x}
+\beta \theta(x) e^{-\beta x} + \beta \theta(-x) e^{\beta x}
\right) \\
&= \beta^2 \psi - 2 \beta \delta(x)
\end{aligned}
\end{equation}

Now application of the Hamiltonian operator on our trial function gives us

\begin{equation}\label{eqn:variationHarmonicOscillator:370}
H \psi = -\frac{\Hbar^2}{2m} \left( \beta^2 \psi - 2 \delta(x) \right) + \inv{2} m \omega^2 x^2 \psi,
\end{equation}

so

\begin{equation}\label{eqn:variationHarmonicOscillator:751}
\begin{aligned}
\bra{\psi} H \ket{\psi} &=
\int_{-\infty}^\infty
\left(
-\frac{\Hbar^2 \beta^2}{2m} + \inv{2} m \omega^2 x^2
\right) e^{-2 \beta \Abs{x} }
+ \frac{\Hbar^2 \beta}{m}
\int_{-\infty}^\infty \delta(x)
e^{- \beta \Abs{x}} \\
&=
-\frac{\beta \Hbar^2}{2m} + \frac{m \omega^2}{4 \beta^3} + \frac{\Hbar^2 \beta}{m} \\
&=
\frac{\beta \Hbar^2}{2m} + \frac{m \omega^2}{4 \beta^3}.
\end{aligned}
\end{equation}

Normalized we have

\begin{equation}\label{eqn:variationHarmonicOscillator:390}
E[\beta] = \frac{\bra{\psi} H \ket{\psi}}{\braket{\psi}{\psi}} = \frac{\beta^2 \Hbar^2}{2m} + \frac{m \omega^2}{4 \beta^2}.
\end{equation}

This is looking much more promising.  We will have the sign alternation that we require to find a positive, non-complex, value for \(\beta\) when \(E[\beta]\) is minimized.  That is

\begin{equation}\label{eqn:variationHarmonicOscillator:410}
0 = \PD{\beta}{E} =
\frac{\beta \Hbar^2}{m} - \frac{m \omega^2}{2 \beta^3},
\end{equation}

so the extremum is found at

\begin{equation}\label{eqn:variationHarmonicOscillator:430}
\beta^4 = \frac{m^2 \omega^2}{2 \Hbar^2}.
\end{equation}

Plugging this back in we find that our trial function associated with the minimum energy (unnormalized still) is

\begin{equation}\label{eqn:variationHarmonicOscillator:450}
\psi = e^{-\sqrt{\frac{m \omega x^2}{\sqrt{2} \Hbar}}},
\end{equation}

and that energy, after substitution, is

\begin{equation}\label{eqn:variationHarmonicOscillator:470}
E[\beta_{\text{min}}] = \frac{\Hbar \omega}{2} \sqrt{2}
\end{equation}

We have something that is \(1.4 \times\) the true ground state energy, but is at least a ball park value.  However, to get this result, we have to be very careful to treat our point of singularity.  A derivative that we would call undefined in first year calculus, is not only defined, but required, for this treatment to work!

%
% Copyright � 2015 Peeter Joot.  All Rights Reserved.
% Licenced as described in the file LICENSE under the root directory of this GIT repository.
%
\documentclass[]{eliblog}

\usepackage{amsmath}
\usepackage{mathpazo}

%
% shorthand for bold symbols, convenient for vectors and matrices
%
\newcommand{\Ba}[0]{\mathbf{a}}
\newcommand{\Bb}[0]{\mathbf{b}}
\newcommand{\Bc}[0]{\mathbf{c}}
\newcommand{\Bd}[0]{\mathbf{d}}
\newcommand{\Be}[0]{\mathbf{e}}
\newcommand{\Bf}[0]{\mathbf{f}}
\newcommand{\Bg}[0]{\mathbf{g}}
\newcommand{\Bh}[0]{\mathbf{h}}
\newcommand{\Bi}[0]{\mathbf{i}}
\newcommand{\Bj}[0]{\mathbf{j}}
\newcommand{\Bk}[0]{\mathbf{k}}
\newcommand{\Bl}[0]{\mathbf{l}}
\newcommand{\Bm}[0]{\mathbf{m}}
\newcommand{\Bn}[0]{\mathbf{n}}
\newcommand{\Bo}[0]{\mathbf{o}}
\newcommand{\Bp}[0]{\mathbf{p}}
\newcommand{\Bq}[0]{\mathbf{q}}
\newcommand{\Br}[0]{\mathbf{r}}
\newcommand{\Bs}[0]{\mathbf{s}}
\newcommand{\Bt}[0]{\mathbf{t}}
\newcommand{\Bu}[0]{\mathbf{u}}
\newcommand{\Bv}[0]{\mathbf{v}}
\newcommand{\Bw}[0]{\mathbf{w}}
\newcommand{\Bx}[0]{\mathbf{x}}
\newcommand{\By}[0]{\mathbf{y}}
\newcommand{\Bz}[0]{\mathbf{z}}
\newcommand{\BA}[0]{\mathbf{A}}
\newcommand{\BB}[0]{\mathbf{B}}
\newcommand{\BC}[0]{\mathbf{C}}
\newcommand{\BD}[0]{\mathbf{D}}
\newcommand{\BE}[0]{\mathbf{E}}
\newcommand{\BF}[0]{\mathbf{F}}
\newcommand{\BG}[0]{\mathbf{G}}
\newcommand{\BH}[0]{\mathbf{H}}
\newcommand{\BI}[0]{\mathbf{I}}
\newcommand{\BJ}[0]{\mathbf{J}}
\newcommand{\BK}[0]{\mathbf{K}}
\newcommand{\BL}[0]{\mathbf{L}}
\newcommand{\BM}[0]{\mathbf{M}}
\newcommand{\BN}[0]{\mathbf{N}}
\newcommand{\BO}[0]{\mathbf{O}}
\newcommand{\BP}[0]{\mathbf{P}}
\newcommand{\BQ}[0]{\mathbf{Q}}
\newcommand{\BR}[0]{\mathbf{R}}
\newcommand{\BS}[0]{\mathbf{S}}
\newcommand{\BT}[0]{\mathbf{T}}
\newcommand{\BU}[0]{\mathbf{U}}
\newcommand{\BV}[0]{\mathbf{V}}
\newcommand{\BW}[0]{\mathbf{W}}
\newcommand{\BX}[0]{\mathbf{X}}
\newcommand{\BY}[0]{\mathbf{Y}}
\newcommand{\BZ}[0]{\mathbf{Z}}

\newcommand{\Bzero}[0]{\mathbf{0}}
\newcommand{\Btheta}[0]{\boldsymbol{\theta}}
\newcommand{\Btau}[0]{\boldsymbol{\tau}}
\newcommand{\Bomega}[0]{\boldsymbol{\omega}}

%
% shorthand for unit vectors
%
\newcommand{\acap}[0]{\hat{\Ba}}
\newcommand{\bcap}[0]{\hat{\Bb}}
\newcommand{\ccap}[0]{\hat{\Bc}}
\newcommand{\dcap}[0]{\hat{\Bd}}
\newcommand{\ecap}[0]{\hat{\Be}}
\newcommand{\fcap}[0]{\hat{\Bf}}
\newcommand{\gcap}[0]{\hat{\Bg}}
\newcommand{\hcap}[0]{\hat{\Bh}}
\newcommand{\icap}[0]{\hat{\Bi}}
\newcommand{\jcap}[0]{\hat{\Bj}}
\newcommand{\kcap}[0]{\hat{\Bk}}
\newcommand{\lcap}[0]{\hat{\Bl}}
\newcommand{\mcap}[0]{\hat{\Bm}}
\newcommand{\ncap}[0]{\hat{\Bn}}
\newcommand{\ocap}[0]{\hat{\Bo}}
\newcommand{\pcap}[0]{\hat{\Bp}}
\newcommand{\qcap}[0]{\hat{\Bq}}
\newcommand{\rcap}[0]{\hat{\Br}}
\newcommand{\scap}[0]{\hat{\Bs}}
\newcommand{\tcap}[0]{\hat{\Bt}}
\newcommand{\ucap}[0]{\hat{\Bu}}
\newcommand{\vcap}[0]{\hat{\Bv}}
\newcommand{\wcap}[0]{\hat{\Bw}}
\newcommand{\xcap}[0]{\hat{\Bx}}
\newcommand{\ycap}[0]{\hat{\By}}
\newcommand{\zcap}[0]{\hat{\Bz}}
\newcommand{\thetacap}[0]{\hat{\Btheta}}

%
% to write R^n and C^n in a distinguishable fashion.  Perhaps change this
% to the double lined characters upon figuring out how to do so.
%
\newcommand{\C}[1]{$\mathbb{C}^{#1}$}
\newcommand{\R}[1]{$\mathbb{R}^{#1}$}

%
% various generally useful helpers
%

% derivative of #1 wrt. #2:
\newcommand{\D}[2] {\frac {d#2} {d#1}}

\newcommand{\inv}[1]{\frac{1}{#1}}
\newcommand{\cross}[0]{\times}

\newcommand{\abs}[1]{\lvert{#1}\rvert}
\newcommand{\norm}[1]{\lVert{#1}\rVert}
\newcommand{\innerprod}[2]{\langle{#1}, {#2}\rangle}
\newcommand{\dotprod}[2]{{#1} \cdot {#2}}
\newcommand{\bdotprod}[2]{\left({#1} \cdot {#2}\right)}
\newcommand{\crossprod}[2]{{#1} \cross {#2}}
\newcommand{\tripleprod}[3]{\dotprod{\left(\crossprod{#1}{#2}\right)}{#3}}

\DeclareMathOperator{\Proj}{Proj}
\DeclareMathOperator{\Span}{span}
\DeclareMathOperator{\Sgn}{sgn}
\DeclareMathOperator{\Area}{Area}
\DeclareMathOperator{\Volume}{Volume}

%
% A few miscellaneous things specific to this document
%
\newcommand{\crossop}[1]{\crossprod{#1}{}}

% R2 vector.
\newcommand{\VectorTwo}[2]{
\begin{bmatrix}
 {#1} \\
 {#2}
\end{bmatrix}
}

\newcommand{\VectorN}[1]{
\begin{bmatrix}
{#1}_1 \\
{#1}_2 \\
\vdots \\
{#1}_N \\
\end{bmatrix}
}

\newcommand{\DETuvij}[4]{
\begin{vmatrix}
 {#1}_{#3} & {#1}_{#4} \\
 {#2}_{#3} & {#2}_{#4}
\end{vmatrix}
}

\newcommand{\DETuvwijk}[6]{
\begin{vmatrix}
 {#1}_{#4} & {#1}_{#5} & {#1}_{#6} \\
 {#2}_{#4} & {#2}_{#5} & {#2}_{#6} \\
 {#3}_{#4} & {#3}_{#5} & {#3}_{#6}
\end{vmatrix}
}

\newcommand{\DETuvwxijkl}[8]{
\begin{vmatrix}
 {#1}_{#5} & {#1}_{#6} & {#1}_{#7} & {#1}_{#8} \\
 {#2}_{#5} & {#2}_{#6} & {#2}_{#7} & {#2}_{#8} \\
 {#3}_{#5} & {#3}_{#6} & {#3}_{#7} & {#3}_{#8} \\
 {#4}_{#5} & {#4}_{#6} & {#4}_{#7} & {#4}_{#8} \\
\end{vmatrix}
}

%\newcommand{\DETuvwxyijklm}[10]{
%\begin{vmatrix}
% {#1}_{#6} & {#1}_{#7} & {#1}_{#8} & {#1}_{#9} & {#1}_{#10} \\
% {#2}_{#6} & {#2}_{#7} & {#2}_{#8} & {#2}_{#9} & {#2}_{#10} \\
% {#3}_{#6} & {#3}_{#7} & {#3}_{#8} & {#3}_{#9} & {#3}_{#10} \\
% {#4}_{#6} & {#4}_{#7} & {#4}_{#8} & {#4}_{#9} & {#4}_{#10} \\
% {#5}_{#6} & {#5}_{#7} & {#5}_{#8} & {#5}_{#9} & {#5}_{#10}
%\end{vmatrix}
%}

% R3 vector.
\newcommand{\VectorThree}[3]{
\begin{bmatrix}
 {#1} \\
 {#2} \\
 {#3}
\end{bmatrix}
}



\author{Peeter Joot}
\email{peeter.joot@gmail.com}

%\documentclass[]{eliblogwidescreen}

\usepackage{amsmath}
\usepackage{mathpazo}

%
% shorthand for bold symbols, convenient for vectors and matrices
%
\newcommand{\Ba}[0]{\mathbf{a}}
\newcommand{\Bb}[0]{\mathbf{b}}
\newcommand{\Bc}[0]{\mathbf{c}}
\newcommand{\Bd}[0]{\mathbf{d}}
\newcommand{\Be}[0]{\mathbf{e}}
\newcommand{\Bf}[0]{\mathbf{f}}
\newcommand{\Bg}[0]{\mathbf{g}}
\newcommand{\Bh}[0]{\mathbf{h}}
\newcommand{\Bi}[0]{\mathbf{i}}
\newcommand{\Bj}[0]{\mathbf{j}}
\newcommand{\Bk}[0]{\mathbf{k}}
\newcommand{\Bl}[0]{\mathbf{l}}
\newcommand{\Bm}[0]{\mathbf{m}}
\newcommand{\Bn}[0]{\mathbf{n}}
\newcommand{\Bo}[0]{\mathbf{o}}
\newcommand{\Bp}[0]{\mathbf{p}}
\newcommand{\Bq}[0]{\mathbf{q}}
\newcommand{\Br}[0]{\mathbf{r}}
\newcommand{\Bs}[0]{\mathbf{s}}
\newcommand{\Bt}[0]{\mathbf{t}}
\newcommand{\Bu}[0]{\mathbf{u}}
\newcommand{\Bv}[0]{\mathbf{v}}
\newcommand{\Bw}[0]{\mathbf{w}}
\newcommand{\Bx}[0]{\mathbf{x}}
\newcommand{\By}[0]{\mathbf{y}}
\newcommand{\Bz}[0]{\mathbf{z}}
\newcommand{\BA}[0]{\mathbf{A}}
\newcommand{\BB}[0]{\mathbf{B}}
\newcommand{\BC}[0]{\mathbf{C}}
\newcommand{\BD}[0]{\mathbf{D}}
\newcommand{\BE}[0]{\mathbf{E}}
\newcommand{\BF}[0]{\mathbf{F}}
\newcommand{\BG}[0]{\mathbf{G}}
\newcommand{\BH}[0]{\mathbf{H}}
\newcommand{\BI}[0]{\mathbf{I}}
\newcommand{\BJ}[0]{\mathbf{J}}
\newcommand{\BK}[0]{\mathbf{K}}
\newcommand{\BL}[0]{\mathbf{L}}
\newcommand{\BM}[0]{\mathbf{M}}
\newcommand{\BN}[0]{\mathbf{N}}
\newcommand{\BO}[0]{\mathbf{O}}
\newcommand{\BP}[0]{\mathbf{P}}
\newcommand{\BQ}[0]{\mathbf{Q}}
\newcommand{\BR}[0]{\mathbf{R}}
\newcommand{\BS}[0]{\mathbf{S}}
\newcommand{\BT}[0]{\mathbf{T}}
\newcommand{\BU}[0]{\mathbf{U}}
\newcommand{\BV}[0]{\mathbf{V}}
\newcommand{\BW}[0]{\mathbf{W}}
\newcommand{\BX}[0]{\mathbf{X}}
\newcommand{\BY}[0]{\mathbf{Y}}
\newcommand{\BZ}[0]{\mathbf{Z}}

\newcommand{\Bzero}[0]{\mathbf{0}}
\newcommand{\Btheta}[0]{\boldsymbol{\theta}}
\newcommand{\Btau}[0]{\boldsymbol{\tau}}
\newcommand{\Bomega}[0]{\boldsymbol{\omega}}

%
% shorthand for unit vectors
%
\newcommand{\acap}[0]{\hat{\Ba}}
\newcommand{\bcap}[0]{\hat{\Bb}}
\newcommand{\ccap}[0]{\hat{\Bc}}
\newcommand{\dcap}[0]{\hat{\Bd}}
\newcommand{\ecap}[0]{\hat{\Be}}
\newcommand{\fcap}[0]{\hat{\Bf}}
\newcommand{\gcap}[0]{\hat{\Bg}}
\newcommand{\hcap}[0]{\hat{\Bh}}
\newcommand{\icap}[0]{\hat{\Bi}}
\newcommand{\jcap}[0]{\hat{\Bj}}
\newcommand{\kcap}[0]{\hat{\Bk}}
\newcommand{\lcap}[0]{\hat{\Bl}}
\newcommand{\mcap}[0]{\hat{\Bm}}
\newcommand{\ncap}[0]{\hat{\Bn}}
\newcommand{\ocap}[0]{\hat{\Bo}}
\newcommand{\pcap}[0]{\hat{\Bp}}
\newcommand{\qcap}[0]{\hat{\Bq}}
\newcommand{\rcap}[0]{\hat{\Br}}
\newcommand{\scap}[0]{\hat{\Bs}}
\newcommand{\tcap}[0]{\hat{\Bt}}
\newcommand{\ucap}[0]{\hat{\Bu}}
\newcommand{\vcap}[0]{\hat{\Bv}}
\newcommand{\wcap}[0]{\hat{\Bw}}
\newcommand{\xcap}[0]{\hat{\Bx}}
\newcommand{\ycap}[0]{\hat{\By}}
\newcommand{\zcap}[0]{\hat{\Bz}}
\newcommand{\thetacap}[0]{\hat{\Btheta}}

%
% to write R^n and C^n in a distinguishable fashion.  Perhaps change this
% to the double lined characters upon figuring out how to do so.
%
\newcommand{\C}[1]{$\mathbb{C}^{#1}$}
\newcommand{\R}[1]{$\mathbb{R}^{#1}$}

%
% various generally useful helpers
%

% derivative of #1 wrt. #2:
\newcommand{\D}[2] {\frac {d#2} {d#1}}

\newcommand{\inv}[1]{\frac{1}{#1}}
\newcommand{\cross}[0]{\times}

\newcommand{\abs}[1]{\lvert{#1}\rvert}
\newcommand{\norm}[1]{\lVert{#1}\rVert}
\newcommand{\innerprod}[2]{\langle{#1}, {#2}\rangle}
\newcommand{\dotprod}[2]{{#1} \cdot {#2}}
\newcommand{\bdotprod}[2]{\left({#1} \cdot {#2}\right)}
\newcommand{\crossprod}[2]{{#1} \cross {#2}}
\newcommand{\tripleprod}[3]{\dotprod{\left(\crossprod{#1}{#2}\right)}{#3}}

\DeclareMathOperator{\Proj}{Proj}
\DeclareMathOperator{\Span}{span}
\DeclareMathOperator{\Sgn}{sgn}
\DeclareMathOperator{\Area}{Area}
\DeclareMathOperator{\Volume}{Volume}

%
% A few miscellaneous things specific to this document
%
\newcommand{\crossop}[1]{\crossprod{#1}{}}

% R2 vector.
\newcommand{\VectorTwo}[2]{
\begin{bmatrix}
 {#1} \\
 {#2}
\end{bmatrix}
}

\newcommand{\VectorN}[1]{
\begin{bmatrix}
{#1}_1 \\
{#1}_2 \\
\vdots \\
{#1}_N \\
\end{bmatrix}
}

\newcommand{\DETuvij}[4]{
\begin{vmatrix}
 {#1}_{#3} & {#1}_{#4} \\
 {#2}_{#3} & {#2}_{#4}
\end{vmatrix}
}

\newcommand{\DETuvwijk}[6]{
\begin{vmatrix}
 {#1}_{#4} & {#1}_{#5} & {#1}_{#6} \\
 {#2}_{#4} & {#2}_{#5} & {#2}_{#6} \\
 {#3}_{#4} & {#3}_{#5} & {#3}_{#6}
\end{vmatrix}
}

\newcommand{\DETuvwxijkl}[8]{
\begin{vmatrix}
 {#1}_{#5} & {#1}_{#6} & {#1}_{#7} & {#1}_{#8} \\
 {#2}_{#5} & {#2}_{#6} & {#2}_{#7} & {#2}_{#8} \\
 {#3}_{#5} & {#3}_{#6} & {#3}_{#7} & {#3}_{#8} \\
 {#4}_{#5} & {#4}_{#6} & {#4}_{#7} & {#4}_{#8} \\
\end{vmatrix}
}

%\newcommand{\DETuvwxyijklm}[10]{
%\begin{vmatrix}
% {#1}_{#6} & {#1}_{#7} & {#1}_{#8} & {#1}_{#9} & {#1}_{#10} \\
% {#2}_{#6} & {#2}_{#7} & {#2}_{#8} & {#2}_{#9} & {#2}_{#10} \\
% {#3}_{#6} & {#3}_{#7} & {#3}_{#8} & {#3}_{#9} & {#3}_{#10} \\
% {#4}_{#6} & {#4}_{#7} & {#4}_{#8} & {#4}_{#9} & {#4}_{#10} \\
% {#5}_{#6} & {#5}_{#7} & {#5}_{#8} & {#5}_{#9} & {#5}_{#10}
%\end{vmatrix}
%}

% R3 vector.
\newcommand{\VectorThree}[3]{
\begin{bmatrix}
 {#1} \\
 {#2} \\
 {#3}
\end{bmatrix}
}



\author{Peeter Joot}
\email{peeter.joot@gmail.com}


\chapter{Simple entanglement example.}
\label{chap:entangledSimpleExample}
%\useCCL
\blogpage{http://sites.google.com/site/peeterjoot/math2011/entangledSimpleExample.pdf}
\date{Sept 1, 2011}
\revisionInfo{entangledSimpleExample.tex}

\beginArtWithToc
%\beginArtNoToc

\section{Motivation.}

On the quiz we were given a three state system $\ket{1}, \ket{2}$ and $\ket{3}$, and a two state system $\ket{a}, \ket{b}$, and were asked to show that the composite system can be entangled.  It really would have been nice to have had at least one entanglement example or problem in the class or recitation before seeing it for real on the quiz.  Oh well!

\section{Guts}

What's the simplest composite state that we can create?  Suppose we have a pair of two state systems, say,

\begin{align}\label{eqn:entangledSimpleExample:n}
\ket{1} 
&= 
\begin{bmatrix}
1 \\
0
\end{bmatrix} \in H_1 \\
\ket{2} 
&= 
\begin{bmatrix}
0 \\
1 
\end{bmatrix} \in H_1,
\end{align}

and

\begin{align}\label{eqn:entangledSimpleExample:n}
\braket{x}{+} 
&= 
\frac{e^{i k x}}{\sqrt{2\pi}},
\text{where}\, \ket{+} \in H_2 \\
\braket{x}{+} 
&= 
&= 
\frac{e^{-i k x}}{\sqrt{2\pi}}
\text{where}\, \ket{-} \in H_2.
\end{align}

We can now enumerate the space of possible operators

\begin{align*}
A =
a_{11, ++} \ket{1}\bra{1} \otimes \ket{+}\bra{+}
+a_{22, --} \ket{2}\bra{2} \otimes \ket{-}\bra{-}
+a_{11, --} \ket{1}\bra{1} \otimes \ket{-}\bra{-}
+a_{22, ++} \ket{2}\bra{2} \otimes \ket{+}\bra{+}
+a_{12, ++} \ket{1}\bra{2} \otimes \ket{+}\bra{+}
+a_{12, --} \ket{1}\bra{2} \otimes \ket{-}\bra{-}
+a_{21, --} \ket{2}\bra{1} \otimes \ket{-}\bra{-}
+a_{21, ++} \ket{2}\bra{1} \otimes \ket{+}\bra{+}
+a_{11, +-} \ket{1}\bra{1} \otimes \ket{+}\bra{-}
+a_{22, +-} \ket{2}\bra{2} \otimes \ket{+}\bra{-}
+a_{11, -+} \ket{1}\bra{1} \otimes \ket{-}\bra{+}
+a_{22, -+} \ket{2}\bra{2} \otimes \ket{-}\bra{+}
\end{align*}

\EndArticle
%\EndNoBibArticle

%
% Copyright � 2012 Peeter Joot.  All Rights Reserved.
% Licenced as described in the file LICENSE under the root directory of this GIT repository.
%

\label{chap:qmTwoPs4}

%\blogpage{http://sites.google.com/site/peeterjoot/math2011/qmTwoPs4.pdf}
%\date{Oct 12, 2011}

%\section{Problem 2}

I was deceived by an incorrect result in Mathematica, which led me to believe that the second order energy perturbation was zero (whereas part (c) of the problem asked if it was greater or lesser than zero).  I started starting writing this up to show my reasoning, but our Professor quickly provided an example after class showing how this zero must be wrong, and I did not have to show him any of this.

\paragraph{Setup}

Recall first the one dimensional particle in a box.  Within the box we have to solve

\begin{equation}\label{eqn:qmTwoPs4:10}
\frac{P^2}{2m} \psi = E\psi
\end{equation}

and find 

\begin{equation}\label{eqn:qmTwoPs4:30}
\psi \sim
e^{\frac{i}{\Hbar} \sqrt{2 m E} x} 
\end{equation}

With
\begin{equation}\label{eqn:qmTwoPs4:50}
k = \frac{\sqrt{2 m E}}{\Hbar}
\end{equation}

our general state, involving terms of each sign, takes the form
\begin{equation}\label{eqn:qmTwoPs4:70}
\psi = 
A e^{ i k x } +B e^{ -i k x }
\end{equation}

Inserting boundary conditions gives us

\begin{equation}\label{eqn:qmTwoPs4:90}
\begin{bmatrix}
\psi(-L/2) \\
\psi(L/2)
\end{bmatrix}
\begin{bmatrix}
e^{ -i k \frac{L}{2} } +e^{ i k \frac{L}{2} } \\
e^{ i k \frac{L}{2} } +e^{ -i k \frac{L}{2} }
\end{bmatrix}
\begin{bmatrix}
A \\
B
\end{bmatrix}
\end{equation}

The determinant is zero

\begin{equation}\label{eqn:qmTwoPs4:110}
e^{-i k L} - e^{i k L} = 0,
\end{equation}

which provides our constraint on \(k\) 

\begin{equation}\label{eqn:qmTwoPs4:130}
e^{2 i k L} = 1.
\end{equation}

We require \(2 k L = 2 \pi n\) for any integer \(n\), or

\begin{equation}\label{eqn:qmTwoPs4:150}
k = \frac{\pi n}{L}.
\end{equation}

This quantizes the energy, and inverting \eqnref{eqn:qmTwoPs4:50} gives us

\begin{equation}\label{eqn:qmTwoPs4:170}
E = \inv{2m} \left( \frac{\Hbar \pi n }{L} \right)^2.
\end{equation}

To complete the task of matching boundary value conditions we cheat and recall that the particular linear combinations that we need to match the boundary constraint of zero at \(\pm L/2\) were sums and differences yielding cosines and sines respectively.  Since

\begin{equation}\label{eqn:qmTwoPs4:190}
\evalbar{\sin\left( \frac{\pi n x }{L} \right) }{x = \pm L/2} = 
\pm \sin\left(\frac{\pi n}{2}\right)
\end{equation}

So sines are the wave functions for \(n = 2, 4, ...\) since \(\sin(n \pi) = 0\) for integer \(n\).  Similarly

\begin{equation}\label{eqn:qmTwoPs4:210}
\evalbar{\cos\left( \frac{\pi n x }{L} \right) }{x = \pm L/2} = 
\cos\left(\frac{\pi n}{2}\right).
\end{equation}

Cosine becomes zero at \(\pi/2, 3\pi/2, \cdots\), so our wave function is the cosine for \(n = 1, 3, 5, \cdots\).

Normalizing gives us

\begin{equation}\label{eqn:qmTwoPs4:230}
\psi_n(x) = \sqrt{\frac{2}{L}}
\left\{
\begin{array}{l l}
\cos\left(\frac{\pi n x}{L}\right) & \quad n = 1, 3, 5, \cdots \\
\sin\left(\frac{\pi n x}{L}\right) & \quad n = 2, 4, 6, \cdots 
\end{array}
\right.
\end{equation}

\paragraph{Two non-interacting particles.  Three lowest energy levels and degeneracies}

Forming the Hamiltonian for two particles in the box without interaction, we have within the box

\begin{equation}\label{eqn:qmTwoPs4:250}
H = 
\frac{P_1^2}{2m} 
+\frac{P_2^2}{2m} 
\end{equation}

we can apply separation of variables, and it becomes clear that our wave functions have the form

\begin{equation}\label{eqn:qmTwoPs4:270}
\psi_{nm}(x_1, x_2) = \psi_n(x_1) \psi_m(x_2)
\end{equation}

Plugging in
\begin{equation}\label{eqn:qmTwoPs4:290}
H \psi = E \psi,
\end{equation}

supplies the energy levels for the two particle wavefunction, giving

\begin{equation}\label{eqn:qmTwoPs4:310}
\begin{aligned}
H \psi_{nm} 
&= 
\frac{\Hbar^2}{2m}
\left(
\left(\frac{\pi n}{L}\right)^2
+\left(\frac{\pi m}{L}\right)^2
\right)
\psi_{nm} \\
&= 
\frac{1}{2m} \left(\frac{\Hbar \pi}{L}\right)^2 ( n^2 + m^2 ) \psi_{nm}
\end{aligned}
\end{equation}

Letting \(n, m\) each range over \([1,3]\) for example we find
\begin{equation}\label{eqn:qmTwoPs4:330}
\begin{array}{l l l}
n & m & n^2 + m^2 \\
1 & 1 & 2 \\
1 & 2 & 5 \\
1 & 3 & 10 \\
2 & 1 & 5 \\
2 & 2 & 8 \\
2 & 3 & 13 \\
3 & 1 & 10 \\
3 & 2 & 13 \\
3 & 3 & 18
\end{array}
\end{equation}

It is clear that our lowest energy levels are 

\begin{equation}\label{eqn:qmTwoPs4:470}
\begin{aligned}
\frac{1}{m} \left(\frac{\Hbar \pi}{L}\right)^2  \\
\frac{5}{2m} \left(\frac{\Hbar \pi}{L}\right)^2  \\
\frac{4}{m} \left(\frac{\Hbar \pi}{L}\right)^2 
\end{aligned}
\end{equation}

with degeneracies \(1, 2, 1\) respectively.

\paragraph{Ground state energy with interaction perturbation to first order}

With \(c_0\) positive and an interaction potential of the form

\begin{equation}\label{eqn:qmTwoPs4:350}
U(X_1, X_2) = - c_0 \delta(X_1 - X_2)
\end{equation}

The second order perturbation of the ground state energy is

\begin{equation}\label{eqn:qmTwoPs4:370}
E = E_{11}^{(0)} + 
H_{11;11}' + 
\sum_{nm \ne 11} \frac{\Abs{H_{11;11}' }^2}{E_{11} - E_{nm}}
\end{equation}

where
\begin{equation}\label{eqn:qmTwoPs4:390}
E_{11}^{(0)} = \frac{1}{m} \left(\frac{\Hbar \pi}{L}\right)^2,
\end{equation}

and
\begin{equation}\label{eqn:qmTwoPs4:410}
H_{nm;ab}' = -c_0 \bra{\psi_{nm}} \delta(X_1 - X_2) \ket{\psi_{ab}}
\end{equation}

to proceed, we need to expand the matrix element

\begin{equation}\label{eqn:qmTwoPs4:490}
\begin{aligned}
\bra{\psi_{nm}} \delta(X_1 - X_2) \ket{\psi_{ab}}
&=
\int dx_1 dx_2 dy_1 dy_2
\braket{\psi_{nm}}{x_1 x_2} \bra{x_1 x_2} \delta(X_1 - X_2) \ket{y_1 y_2 } \braket{y_1 y_2}{\psi_{ab}} \\
&=
\int dx_1 dx_2 dy_1 dy_2
\braket{\psi_{nm}}{x_1 x_2} \delta(x_1 - x_2) \delta^2(\Bx - \By) \braket{y_1 y_2}{\psi_{ab}} \\
&=
\int dx_1 dx_2 
\braket{\psi_{nm}}{x_1 x_2} \delta(x_1 - x_2) \braket{x_1 x_2}{\psi_{ab}} \\
&=
\int_{-L/2}^{L/2} dx
\psi_{nm}(x, x)
\psi_{ab}(x, x)
\end{aligned}
\end{equation}

So, for our first order calculation we need 

\begin{equation}\label{eqn:qmTwoPs4:510}
\begin{aligned}
H_{11; 11}' 
&= 
- c_0
\int_{-L/2}^{L/2} dx
\psi_{11}(x, x)
\psi_{11}(x, x) \\
&=
\frac{4}{L^2}
\int_{-L/2}^{L/2} dx
\cos^4( \pi x /L ) \\
&=
- \frac{3 c_0}{2 L}
\end{aligned}
\end{equation}

For the second order perturbation of the energy, it is clear that this will reduce the first order approximation for each matrix element that is non-zero.

Attempting that calculation with \href{https://github.com/peeterjoot/physicsplay/blob/796c8e3739ae1a9ca26270a0e91384afba45661d/notes/phy456/problem\%20set\%204,\%20problem\%202.nb}{Mathematica} however, is deceiving, since Mathematica reports these all as zero after FullSimplify.  It appears, that as used, it does not allow for \(m = n\) and \(m = n \pm 1\) constraints properly where the denominators of the unsimplified integrals go zero.

This worksheet can be seen to be giving misleading results, by evaluating

\begin{equation}\label{eqn:qmTwoPs4:430}
\int_{-\frac{L}{2}}^{\frac{L}{2}} \left(\frac{2}{L}\right)^2 \cos ^2\left(\frac{\pi  x}{L}\right) \cos ^2\left(\frac{3 \pi  x}{L}\right) \, dx = \frac{1}{L}
\end{equation}

Yet, the FullSimplify gives

\begin{equation}\label{eqn:qmTwoPs4:450}
\text{FullSimplify}\left[\int_{-\frac{L}{2}}^{\frac{L}{2}} \text{Cos}\left[\frac{\pi  x}{L}\right]^2 \left(\frac{2}{L}\right)^2 \text{Cos}\left[\frac{(2 n+1) \pi  x}{L}\right] \text{Cos}\left[\frac{(2 m+1) \pi  x}{L}\right] \, dx,\{m,n\}\in \text{Integers}\right] = 0
\end{equation}

I am hoping that asking about this on \href{http://stackoverflow.com/questions/7743774/proper-way-to-simplify-integral-result-in-mathematica-given-integer-constraints}{stackoverflow} will clarify how to use Mathematica correctly for this calculation.




%\part{More worked problems.}

%\part{Cronology}
%\chapter{Cronological Index}
\begin{itemize}

\item October 13, 2007 \ref{chap:gaWiki} Comparison of many traditional vector and GA identities

\item October 13, 2007 \ref{chap:gaWikiTorque} Torque

\item October 16, 2007 \ref{chap:PJUnitDer} Derivatives of a unit vector

\item October 16, 2007 \ref{chap:gaWikiCramers} Cramer's rule

\item October 22, 2007 \ref{chap:PJRadialDer} Radial components of vector derivatives

\item January 1, 2008 \ref{chap:plane} More details on NFCM plane formulation

\item January 29, 2008 \ref{chap:PJAngVel} Rotational dynamics

\item January 29, 2008 \ref{chap:maxwellsGa} Maxwell's equations expressed with Geometric Algebra

\item February 2, 2008 \ref{chap:quaternion} Quaternions

\item February 4, 2008 \ref{chap:legendre} Legendre Polynomials

\item February 15, 2008 \ref{chap:inertialTensor} Inertia Tensor

\item February 19, 2008 \ref{chap:rotor} Rotor Notes

\item February 28, 2008 \ref{chap:laplace} Exponential Solutions to Laplace Equation in \R{N}

\item March 9, 2008 \ref{chap:bivector} Bivector Geometry

\item March 9, 2008 \ref{chap:trivector} Trivector geometry

\item March 12, 2008 \ref{chap:kvectorExponential} Exponential of a blade

\item March 16, 2008 \ref{chap:scalarCommutes} Multivector product grade zero terms

\item March 17, 2008 \ref{chap:angleBetweenLineAndPlane} Angle between geometric elements

\item March 17, 2008 \ref{chap:gaGradeDotWedge} An earlier attempt to intuitively introduce the dot, wedge, cross, and geometric products

\item March 25, 2008 \ref{chap:bladegradereduction} Blade grade reduction

\item March 29, 2008 \ref{chap:reciprocalFrame} Reciprocal Frame Vectors

\item March 31, 2008 \ref{chap:gradientAndForms} Exterior derivative and chain rule components of the gradient

\item April 1, 2008 \ref{chap:orthodecomp} Orthogonal decomposition take II

\item April 11, 2008 \ref{chap:matrixReview} Matrix review

\item April 13, 2008 \ref{chap:locateSatellite} Satellite triangulation over sphere

\item April 30, 2008 \ref{chap:PJKeRot} Kinetic Energy in rotational frame

\item May 7, 2008 \ref{chap:lorentzRotation} Lorentz Force Trajectory

\item May 16, 2008 \ref{chap:obliqueProj} Oblique projection and reciprocal frame vectors

\item May 16, 2008 \ref{chap:matrixOfLinearTx} Matrix of grade k multivector linear transformations

\item May 16, 2008 \ref{chap:projectionAndMoorePenroseVectorInverse} Projection and Moore-Penrose vector inverse

\item May 17, 2008 \ref{chap:PJprojGen} Projection with generalized dot product

\item June 6, 2008 \ref{chap:tensor} Gradient and tensor notes

\item June 10, 2008 \ref{chap:PJAngAcc} Angular Velocity and Acceleration.  Again

\item June 25, 2008 \ref{chap:lorentz} Wave equation based Lorentz transformation derivation

\item July 8, 2008 \ref{chap:PJAngAccCross} Cross product Radial decomposition

\item July 12, 2008 \ref{chap:PJMaxwell2} Back to Maxwell's equations

\item July 16, 2008 \ref{chap:spacetimegrad} Lorentz transformation of spacetime gradient

\item July 20, 2008 \ref{chap:sgMx41} Magnetic field between two parallel wires

\item August 1, 2008 \ref{chap:fourvecDotinvariance} Four vector dot product invariance and Lorentz rotors

\item August 9, 2008 \ref{chap:newtonianLagrangianAndGradient} Newton's Law from Lagrangian

\item August 13, 2008 \ref{chap:cauchyGradient} Cauchy Equations expressed as a gradient

\item August 13, 2008 \ref{chap:velocityTx} Understanding four velocity transform from rest frame

\item August 15, 2008 \ref{chap:emPotential} Four vector potential

\item August 16, 2008 \ref{chap:PJSrGAFPLorentzForce} Lorentz force Law

\item August 21, 2008 \ref{chap:PJSrLagrangian} Covariant Lagrangian, and electrodynamic potential

\item August 25, 2008 \ref{chap:PJTongMf1} Solutions to David Tong's mf1 Lagrangian problems

\item August 28, 2008 \ref{chap:massVaryLagrangian} Equations of motion given mass variation with spacetime position

\item September 1, 2008 \ref{chap:PJCanMomentum} Vector canonical momentum

\item September 2, 2008 \ref{chap:outermorphismDet} OuterMorphism Question 

\item September 5, 2008 \ref{chap:emBivectorMetricDependencies} Metric signature dependencies

\item September 7, 2008 \ref{chap:PJMaxwellTensor} Tensor relations from bivector field equation

\item September 8, 2008 \ref{chap:PJMaxwellLagrangian} Direct variation of Maxwell equations

\item September 9, 2008 \ref{chap:PJMaxwellProj} Vector forms of Maxwell's equations as projection and rejection operations

\item September 18, 2008 \ref{chap:PJStokes1} Stokes law in wedge product form

\item September 26, 2008 \ref{chap:stokesMaxwellApplication} Application of Stokes Integrals to Maxwell's Equation

\item September 27, 2008 \ref{chap:PJStokes2} Stokes Law revisited with algebraic enumeration of boundary

\item October 8, 2008 \ref{chap:PJSrLorentzForce} Revisit Lorentz force from Lagrangian

\item October 10, 2008 \ref{chap:PJFieldLagrangian} Derivation of Euler-Lagrange field equations

\item October 12, 2008 \ref{chap:maxwellTensorLagrangian} Tensor Derivation of Covariant Lorentz Force from Lagrangian

\item October 13, 2008 \ref{chap:PJEulerLagrange} Euler Lagrange Equations

\item October 19, 2008 \ref{chap:PJBoostMaxwell} Lorentz Invariance of Maxwell Lagrangian

\item October 22, 2008 \ref{chap:PJLorentzTxInteraction} Lorentz transform Noether current for interaction Lagrangian

\item October 26, 2008 \ref{chap:gem} GravitoElectroMagnetism

\item October 29, 2008 \ref{chap:PJNoethersField} Field form of Noether's Law

\item November 1, 2008 \ref{chap:eulerangle} Euler Angle Notes

\item November 8, 2008 \ref{chap:complex} Hyper complex numbers and symplectic structure

\item November 13, 2008 \ref{chap:sphericalPolar} Spherical polar coordinates

\item November 22, 2008 \ref{chap:gaussianSurface} Gaussian Surface invariance for radial field

\item November 23, 2008 \ref{chap:chargeArcElement} Field due to line charge in arc

\item November 23, 2008 \ref{chap:chargeLineElement} Charge line element

\item November 27, 2008 \ref{chap:nfcmCh2} Some NFCM exercise solutions and notes

\item November 30, 2008 \ref{chap:PJwaveFourVector} Expressing wave equation exponential solutions using four vectors

\item November 30, 2008 \ref{chap:slerp} Rotor interpolation calculation

\item December 6, 2008 \ref{chap:pauliMatrix} Pauli Matrixes in Clifford Algebra

\item December 11, 2008 \ref{chap:bohr} Bohr Model

\item December 13, 2008 \ref{chap:PJDiracGamma} Gamma Matrices

\item December 21, 2008 \ref{chap:diracLagrangian} Dirac Lagrangian

\item December 27, 2008 \ref{chap:PJrayleighJeans} Rayleigh-Jeans Law Notes

\item December 29, 2008 \ref{chap:PJpoynting} Poynting vector and Electromagnetic Energy conservation

\item January 1, 2009 \ref{chap:PJemstresstensor} Energy momentum tensor

\item January 3, 2009 \ref{chap:PJelectricFieldEnergy} Field and wave energy and momentum

\item January 5, 2009 \ref{chap:vectorDifferentialIdentities} Vector Differential Identities

\item January 6, 2009 \ref{chap:dcPower} DC Power consumption formula for resistive load

\item January 9, 2009 \ref{chap:PJqmFourier} Some Fourier transform notes

\item January 11, 2009 \ref{chap:schCurrent} Schr\"{o}dinger equation probability conservation

\item January 13, 2009 \ref{chap:radial} Polar velocity and acceleration

\item January 18, 2009 \ref{chap:PJpoyntingRate} Time rate of change of the Poynting vector, and its conservation law

\item January 19, 2009 \ref{chap:PJheatFourier} Fourier Solutions to Heat and Wave equations

\item January 21, 2009 \ref{chap:fourierNotation} A cheatsheet for Fourier transform conventions

\item January 25, 2009 \ref{chap:PJemWave} Electrodynamic wave equation solutions

\item January 26, 2009 \ref{chap:PJwaveFourier} Fourier transform solutions to the wave equation

\item January 29, 2009 \ref{chap:PJfourierMaxwellSecondOrder} Fourier transform solutions to Maxwell's equation

\item January 31, 2009 \ref{chap:PJfirstOrderMaxwell} First order Fourier transform solution of Maxwell's equation

\item February 1, 2009 \ref{chap:PJ4dFourier} 4D Fourier transforms applied to Maxwell's equation

\item February 3, 2009 \ref{chap:PJFourierVacuum} Fourier series Vacuum Maxwell's equations

\item February 7, 2009 \ref{chap:potentialFourier} Lorentz Gauge Fourier Vacuum potential solutions

\item February 8, 2009 \ref{chap:PJplaneWave} Plane wave Fourier series solutions to the Maxwell vacuum equation

\item February 13, 2009 \ref{chap:PJstressEnergyLorentz} Lorentz force relation to the energy momentum tensor

\item February 17, 2009 \ref{chap:en_m_tensor} Energy momentum tensor relation to Lorentz force

\item February 18, 2009 \ref{chap:PJpoisson} Poisson and retarded Potential Green's functions from Fourier kernels

\item February 26, 2009 \ref{chap:nvolume} Spherical and hyperspherical parametrization

\item March 13, 2009 \ref{chap:levi} Levi-Civitica summation identity

\item March 18, 2009 \ref{chap:electronRotor} Lorentz force rotor formulation

\item April 15, 2009 \ref{chap:lorentzForcePQA} Lorentz force Lagrangian with conjugate momentum

\item April 18, 2009 \ref{chap:biotSavart} Biot Savart Derivation

\item April 20, 2009 \ref{chap:maxwellTensorFromLagrangian} Tensor derivation of non-dual Maxwell equation from Lagrangian

\item April 28, 2009 \ref{chap:PJmultiTaylors} Developing some intuition for Multivariable and Multivector Taylor Series

\item May 23, 2009 \ref{chap:lorentzForceTx} Lorentz boost of Lorentz force equations

\item May 28, 2009 \ref{chap:macroscopicMaxwell} Macroscopic Maxwell's equation

\item June 1, 2009 \ref{chap:poincareTx} Poincare transformations

\item June 5, 2009 \ref{chap:stressEnergyNoethers} Canonical energy momentum tensor and Lagrangian translation

\item June 17, 2009 \ref{chap:lForceLag2} Comparison of two covariant Lorentz force Lagrangians

\item June 21, 2009 \ref{chap:emVacWave} Wave equation form of Maxwell's equations

\item June 27, 2009 \ref{chap:frequencyTx} Relativistic Doppler formula

\end{itemize}

\part{Bibliography}

% END INCLUDES.
%-------------------------------------------------------

\bibliography{myrefs}
\bibliographystyle{unsrturl}
  \addcontentsline{toc}{chapter}{Bibliography}

\end{document}
%%
% Copyright � 2012 Peeter Joot.  All Rights Reserved.
% Licenced as described in the file LICENSE under the root directory of this GIT repository.
%

%
%
%%
% Copyright � 2015 Peeter Joot.  All Rights Reserved.
% Licenced as described in the file LICENSE under the root directory of this GIT repository.
%
\documentclass[]{eliblog}

\usepackage{amsmath}
\usepackage{mathpazo}

%
% shorthand for bold symbols, convenient for vectors and matrices
%
\newcommand{\Ba}[0]{\mathbf{a}}
\newcommand{\Bb}[0]{\mathbf{b}}
\newcommand{\Bc}[0]{\mathbf{c}}
\newcommand{\Bd}[0]{\mathbf{d}}
\newcommand{\Be}[0]{\mathbf{e}}
\newcommand{\Bf}[0]{\mathbf{f}}
\newcommand{\Bg}[0]{\mathbf{g}}
\newcommand{\Bh}[0]{\mathbf{h}}
\newcommand{\Bi}[0]{\mathbf{i}}
\newcommand{\Bj}[0]{\mathbf{j}}
\newcommand{\Bk}[0]{\mathbf{k}}
\newcommand{\Bl}[0]{\mathbf{l}}
\newcommand{\Bm}[0]{\mathbf{m}}
\newcommand{\Bn}[0]{\mathbf{n}}
\newcommand{\Bo}[0]{\mathbf{o}}
\newcommand{\Bp}[0]{\mathbf{p}}
\newcommand{\Bq}[0]{\mathbf{q}}
\newcommand{\Br}[0]{\mathbf{r}}
\newcommand{\Bs}[0]{\mathbf{s}}
\newcommand{\Bt}[0]{\mathbf{t}}
\newcommand{\Bu}[0]{\mathbf{u}}
\newcommand{\Bv}[0]{\mathbf{v}}
\newcommand{\Bw}[0]{\mathbf{w}}
\newcommand{\Bx}[0]{\mathbf{x}}
\newcommand{\By}[0]{\mathbf{y}}
\newcommand{\Bz}[0]{\mathbf{z}}
\newcommand{\BA}[0]{\mathbf{A}}
\newcommand{\BB}[0]{\mathbf{B}}
\newcommand{\BC}[0]{\mathbf{C}}
\newcommand{\BD}[0]{\mathbf{D}}
\newcommand{\BE}[0]{\mathbf{E}}
\newcommand{\BF}[0]{\mathbf{F}}
\newcommand{\BG}[0]{\mathbf{G}}
\newcommand{\BH}[0]{\mathbf{H}}
\newcommand{\BI}[0]{\mathbf{I}}
\newcommand{\BJ}[0]{\mathbf{J}}
\newcommand{\BK}[0]{\mathbf{K}}
\newcommand{\BL}[0]{\mathbf{L}}
\newcommand{\BM}[0]{\mathbf{M}}
\newcommand{\BN}[0]{\mathbf{N}}
\newcommand{\BO}[0]{\mathbf{O}}
\newcommand{\BP}[0]{\mathbf{P}}
\newcommand{\BQ}[0]{\mathbf{Q}}
\newcommand{\BR}[0]{\mathbf{R}}
\newcommand{\BS}[0]{\mathbf{S}}
\newcommand{\BT}[0]{\mathbf{T}}
\newcommand{\BU}[0]{\mathbf{U}}
\newcommand{\BV}[0]{\mathbf{V}}
\newcommand{\BW}[0]{\mathbf{W}}
\newcommand{\BX}[0]{\mathbf{X}}
\newcommand{\BY}[0]{\mathbf{Y}}
\newcommand{\BZ}[0]{\mathbf{Z}}

\newcommand{\Bzero}[0]{\mathbf{0}}
\newcommand{\Btheta}[0]{\boldsymbol{\theta}}
\newcommand{\Btau}[0]{\boldsymbol{\tau}}
\newcommand{\Bomega}[0]{\boldsymbol{\omega}}

%
% shorthand for unit vectors
%
\newcommand{\acap}[0]{\hat{\Ba}}
\newcommand{\bcap}[0]{\hat{\Bb}}
\newcommand{\ccap}[0]{\hat{\Bc}}
\newcommand{\dcap}[0]{\hat{\Bd}}
\newcommand{\ecap}[0]{\hat{\Be}}
\newcommand{\fcap}[0]{\hat{\Bf}}
\newcommand{\gcap}[0]{\hat{\Bg}}
\newcommand{\hcap}[0]{\hat{\Bh}}
\newcommand{\icap}[0]{\hat{\Bi}}
\newcommand{\jcap}[0]{\hat{\Bj}}
\newcommand{\kcap}[0]{\hat{\Bk}}
\newcommand{\lcap}[0]{\hat{\Bl}}
\newcommand{\mcap}[0]{\hat{\Bm}}
\newcommand{\ncap}[0]{\hat{\Bn}}
\newcommand{\ocap}[0]{\hat{\Bo}}
\newcommand{\pcap}[0]{\hat{\Bp}}
\newcommand{\qcap}[0]{\hat{\Bq}}
\newcommand{\rcap}[0]{\hat{\Br}}
\newcommand{\scap}[0]{\hat{\Bs}}
\newcommand{\tcap}[0]{\hat{\Bt}}
\newcommand{\ucap}[0]{\hat{\Bu}}
\newcommand{\vcap}[0]{\hat{\Bv}}
\newcommand{\wcap}[0]{\hat{\Bw}}
\newcommand{\xcap}[0]{\hat{\Bx}}
\newcommand{\ycap}[0]{\hat{\By}}
\newcommand{\zcap}[0]{\hat{\Bz}}
\newcommand{\thetacap}[0]{\hat{\Btheta}}

%
% to write R^n and C^n in a distinguishable fashion.  Perhaps change this
% to the double lined characters upon figuring out how to do so.
%
\newcommand{\C}[1]{$\mathbb{C}^{#1}$}
\newcommand{\R}[1]{$\mathbb{R}^{#1}$}

%
% various generally useful helpers
%

% derivative of #1 wrt. #2:
\newcommand{\D}[2] {\frac {d#2} {d#1}}

\newcommand{\inv}[1]{\frac{1}{#1}}
\newcommand{\cross}[0]{\times}

\newcommand{\abs}[1]{\lvert{#1}\rvert}
\newcommand{\norm}[1]{\lVert{#1}\rVert}
\newcommand{\innerprod}[2]{\langle{#1}, {#2}\rangle}
\newcommand{\dotprod}[2]{{#1} \cdot {#2}}
\newcommand{\bdotprod}[2]{\left({#1} \cdot {#2}\right)}
\newcommand{\crossprod}[2]{{#1} \cross {#2}}
\newcommand{\tripleprod}[3]{\dotprod{\left(\crossprod{#1}{#2}\right)}{#3}}

\DeclareMathOperator{\Proj}{Proj}
\DeclareMathOperator{\Span}{span}
\DeclareMathOperator{\Sgn}{sgn}
\DeclareMathOperator{\Area}{Area}
\DeclareMathOperator{\Volume}{Volume}

%
% A few miscellaneous things specific to this document
%
\newcommand{\crossop}[1]{\crossprod{#1}{}}

% R2 vector.
\newcommand{\VectorTwo}[2]{
\begin{bmatrix}
 {#1} \\
 {#2}
\end{bmatrix}
}

\newcommand{\VectorN}[1]{
\begin{bmatrix}
{#1}_1 \\
{#1}_2 \\
\vdots \\
{#1}_N \\
\end{bmatrix}
}

\newcommand{\DETuvij}[4]{
\begin{vmatrix}
 {#1}_{#3} & {#1}_{#4} \\
 {#2}_{#3} & {#2}_{#4}
\end{vmatrix}
}

\newcommand{\DETuvwijk}[6]{
\begin{vmatrix}
 {#1}_{#4} & {#1}_{#5} & {#1}_{#6} \\
 {#2}_{#4} & {#2}_{#5} & {#2}_{#6} \\
 {#3}_{#4} & {#3}_{#5} & {#3}_{#6}
\end{vmatrix}
}

\newcommand{\DETuvwxijkl}[8]{
\begin{vmatrix}
 {#1}_{#5} & {#1}_{#6} & {#1}_{#7} & {#1}_{#8} \\
 {#2}_{#5} & {#2}_{#6} & {#2}_{#7} & {#2}_{#8} \\
 {#3}_{#5} & {#3}_{#6} & {#3}_{#7} & {#3}_{#8} \\
 {#4}_{#5} & {#4}_{#6} & {#4}_{#7} & {#4}_{#8} \\
\end{vmatrix}
}

%\newcommand{\DETuvwxyijklm}[10]{
%\begin{vmatrix}
% {#1}_{#6} & {#1}_{#7} & {#1}_{#8} & {#1}_{#9} & {#1}_{#10} \\
% {#2}_{#6} & {#2}_{#7} & {#2}_{#8} & {#2}_{#9} & {#2}_{#10} \\
% {#3}_{#6} & {#3}_{#7} & {#3}_{#8} & {#3}_{#9} & {#3}_{#10} \\
% {#4}_{#6} & {#4}_{#7} & {#4}_{#8} & {#4}_{#9} & {#4}_{#10} \\
% {#5}_{#6} & {#5}_{#7} & {#5}_{#8} & {#5}_{#9} & {#5}_{#10}
%\end{vmatrix}
%}

% R3 vector.
\newcommand{\VectorThree}[3]{
\begin{bmatrix}
 {#1} \\
 {#2} \\
 {#3}
\end{bmatrix}
}



\author{Peeter Joot}
\email{peeter.joot@gmail.com}

%\documentclass[]{eliblogwidescreen}

\usepackage{amsmath}
\usepackage{mathpazo}

%
% shorthand for bold symbols, convenient for vectors and matrices
%
\newcommand{\Ba}[0]{\mathbf{a}}
\newcommand{\Bb}[0]{\mathbf{b}}
\newcommand{\Bc}[0]{\mathbf{c}}
\newcommand{\Bd}[0]{\mathbf{d}}
\newcommand{\Be}[0]{\mathbf{e}}
\newcommand{\Bf}[0]{\mathbf{f}}
\newcommand{\Bg}[0]{\mathbf{g}}
\newcommand{\Bh}[0]{\mathbf{h}}
\newcommand{\Bi}[0]{\mathbf{i}}
\newcommand{\Bj}[0]{\mathbf{j}}
\newcommand{\Bk}[0]{\mathbf{k}}
\newcommand{\Bl}[0]{\mathbf{l}}
\newcommand{\Bm}[0]{\mathbf{m}}
\newcommand{\Bn}[0]{\mathbf{n}}
\newcommand{\Bo}[0]{\mathbf{o}}
\newcommand{\Bp}[0]{\mathbf{p}}
\newcommand{\Bq}[0]{\mathbf{q}}
\newcommand{\Br}[0]{\mathbf{r}}
\newcommand{\Bs}[0]{\mathbf{s}}
\newcommand{\Bt}[0]{\mathbf{t}}
\newcommand{\Bu}[0]{\mathbf{u}}
\newcommand{\Bv}[0]{\mathbf{v}}
\newcommand{\Bw}[0]{\mathbf{w}}
\newcommand{\Bx}[0]{\mathbf{x}}
\newcommand{\By}[0]{\mathbf{y}}
\newcommand{\Bz}[0]{\mathbf{z}}
\newcommand{\BA}[0]{\mathbf{A}}
\newcommand{\BB}[0]{\mathbf{B}}
\newcommand{\BC}[0]{\mathbf{C}}
\newcommand{\BD}[0]{\mathbf{D}}
\newcommand{\BE}[0]{\mathbf{E}}
\newcommand{\BF}[0]{\mathbf{F}}
\newcommand{\BG}[0]{\mathbf{G}}
\newcommand{\BH}[0]{\mathbf{H}}
\newcommand{\BI}[0]{\mathbf{I}}
\newcommand{\BJ}[0]{\mathbf{J}}
\newcommand{\BK}[0]{\mathbf{K}}
\newcommand{\BL}[0]{\mathbf{L}}
\newcommand{\BM}[0]{\mathbf{M}}
\newcommand{\BN}[0]{\mathbf{N}}
\newcommand{\BO}[0]{\mathbf{O}}
\newcommand{\BP}[0]{\mathbf{P}}
\newcommand{\BQ}[0]{\mathbf{Q}}
\newcommand{\BR}[0]{\mathbf{R}}
\newcommand{\BS}[0]{\mathbf{S}}
\newcommand{\BT}[0]{\mathbf{T}}
\newcommand{\BU}[0]{\mathbf{U}}
\newcommand{\BV}[0]{\mathbf{V}}
\newcommand{\BW}[0]{\mathbf{W}}
\newcommand{\BX}[0]{\mathbf{X}}
\newcommand{\BY}[0]{\mathbf{Y}}
\newcommand{\BZ}[0]{\mathbf{Z}}

\newcommand{\Bzero}[0]{\mathbf{0}}
\newcommand{\Btheta}[0]{\boldsymbol{\theta}}
\newcommand{\Btau}[0]{\boldsymbol{\tau}}
\newcommand{\Bomega}[0]{\boldsymbol{\omega}}

%
% shorthand for unit vectors
%
\newcommand{\acap}[0]{\hat{\Ba}}
\newcommand{\bcap}[0]{\hat{\Bb}}
\newcommand{\ccap}[0]{\hat{\Bc}}
\newcommand{\dcap}[0]{\hat{\Bd}}
\newcommand{\ecap}[0]{\hat{\Be}}
\newcommand{\fcap}[0]{\hat{\Bf}}
\newcommand{\gcap}[0]{\hat{\Bg}}
\newcommand{\hcap}[0]{\hat{\Bh}}
\newcommand{\icap}[0]{\hat{\Bi}}
\newcommand{\jcap}[0]{\hat{\Bj}}
\newcommand{\kcap}[0]{\hat{\Bk}}
\newcommand{\lcap}[0]{\hat{\Bl}}
\newcommand{\mcap}[0]{\hat{\Bm}}
\newcommand{\ncap}[0]{\hat{\Bn}}
\newcommand{\ocap}[0]{\hat{\Bo}}
\newcommand{\pcap}[0]{\hat{\Bp}}
\newcommand{\qcap}[0]{\hat{\Bq}}
\newcommand{\rcap}[0]{\hat{\Br}}
\newcommand{\scap}[0]{\hat{\Bs}}
\newcommand{\tcap}[0]{\hat{\Bt}}
\newcommand{\ucap}[0]{\hat{\Bu}}
\newcommand{\vcap}[0]{\hat{\Bv}}
\newcommand{\wcap}[0]{\hat{\Bw}}
\newcommand{\xcap}[0]{\hat{\Bx}}
\newcommand{\ycap}[0]{\hat{\By}}
\newcommand{\zcap}[0]{\hat{\Bz}}
\newcommand{\thetacap}[0]{\hat{\Btheta}}

%
% to write R^n and C^n in a distinguishable fashion.  Perhaps change this
% to the double lined characters upon figuring out how to do so.
%
\newcommand{\C}[1]{$\mathbb{C}^{#1}$}
\newcommand{\R}[1]{$\mathbb{R}^{#1}$}

%
% various generally useful helpers
%

% derivative of #1 wrt. #2:
\newcommand{\D}[2] {\frac {d#2} {d#1}}

\newcommand{\inv}[1]{\frac{1}{#1}}
\newcommand{\cross}[0]{\times}

\newcommand{\abs}[1]{\lvert{#1}\rvert}
\newcommand{\norm}[1]{\lVert{#1}\rVert}
\newcommand{\innerprod}[2]{\langle{#1}, {#2}\rangle}
\newcommand{\dotprod}[2]{{#1} \cdot {#2}}
\newcommand{\bdotprod}[2]{\left({#1} \cdot {#2}\right)}
\newcommand{\crossprod}[2]{{#1} \cross {#2}}
\newcommand{\tripleprod}[3]{\dotprod{\left(\crossprod{#1}{#2}\right)}{#3}}

\DeclareMathOperator{\Proj}{Proj}
\DeclareMathOperator{\Span}{span}
\DeclareMathOperator{\Sgn}{sgn}
\DeclareMathOperator{\Area}{Area}
\DeclareMathOperator{\Volume}{Volume}

%
% A few miscellaneous things specific to this document
%
\newcommand{\crossop}[1]{\crossprod{#1}{}}

% R2 vector.
\newcommand{\VectorTwo}[2]{
\begin{bmatrix}
 {#1} \\
 {#2}
\end{bmatrix}
}

\newcommand{\VectorN}[1]{
\begin{bmatrix}
{#1}_1 \\
{#1}_2 \\
\vdots \\
{#1}_N \\
\end{bmatrix}
}

\newcommand{\DETuvij}[4]{
\begin{vmatrix}
 {#1}_{#3} & {#1}_{#4} \\
 {#2}_{#3} & {#2}_{#4}
\end{vmatrix}
}

\newcommand{\DETuvwijk}[6]{
\begin{vmatrix}
 {#1}_{#4} & {#1}_{#5} & {#1}_{#6} \\
 {#2}_{#4} & {#2}_{#5} & {#2}_{#6} \\
 {#3}_{#4} & {#3}_{#5} & {#3}_{#6}
\end{vmatrix}
}

\newcommand{\DETuvwxijkl}[8]{
\begin{vmatrix}
 {#1}_{#5} & {#1}_{#6} & {#1}_{#7} & {#1}_{#8} \\
 {#2}_{#5} & {#2}_{#6} & {#2}_{#7} & {#2}_{#8} \\
 {#3}_{#5} & {#3}_{#6} & {#3}_{#7} & {#3}_{#8} \\
 {#4}_{#5} & {#4}_{#6} & {#4}_{#7} & {#4}_{#8} \\
\end{vmatrix}
}

%\newcommand{\DETuvwxyijklm}[10]{
%\begin{vmatrix}
% {#1}_{#6} & {#1}_{#7} & {#1}_{#8} & {#1}_{#9} & {#1}_{#10} \\
% {#2}_{#6} & {#2}_{#7} & {#2}_{#8} & {#2}_{#9} & {#2}_{#10} \\
% {#3}_{#6} & {#3}_{#7} & {#3}_{#8} & {#3}_{#9} & {#3}_{#10} \\
% {#4}_{#6} & {#4}_{#7} & {#4}_{#8} & {#4}_{#9} & {#4}_{#10} \\
% {#5}_{#6} & {#5}_{#7} & {#5}_{#8} & {#5}_{#9} & {#5}_{#10}
%\end{vmatrix}
%}

% R3 vector.
\newcommand{\VectorThree}[3]{
\begin{bmatrix}
 {#1} \\
 {#2} \\
 {#3}
\end{bmatrix}
}



\author{Peeter Joot}
\email{peeter.joot@gmail.com}


%\chapter{PHY456H1F: Quantum Mechanics II.  Lecture 11 (Taught by Prof J.E. Sipe).  Spin and Spinors}
%\chapter{Spin and Spinors}
\index{spin}
\index{spinors}
\label{chap:qmTwoL11}

\blogpage{http://sites.google.com/site/peeterjoot/math2011/qmTwoL11.pdf}
%\date{Oct 17, 2011}





\section{Generators}
\index{generator}

Covered in \S 26 of the text \citep{desai2009quantum}.

\makeexample{Time translation}{example:qmTwoL11:1}{

\begin{equation}\label{eqn:qmTwoL11:10}
\ket{\psi(t)} = e^{-i H t/\Hbar} \ket{\psi(0)} .
\end{equation}

The Hamiltonian ``generates'' evolution (or translation) in time.
}

\makeexample{Spatial translation}{example:qmTwoL11:2}{

\begin{equation}\label{eqn:qmTwoL11:30}
\ket{\Br + \Ba} =
e^{-i \Ba \cdot \BP/\Hbar}
\ket{\Br}.
\end{equation}

\imageFigure{../../figures/phy456/qmTwoL11fig1}{Vector translation}{fig:qmTwoL11:qmTwoL11fig1}{0.2}
%\cref{fig:qmTwoL11:qmTwoL11fig1}

\(\BP\) is the operator that generates translations.  Written out, we have

\begin{equation}\label{eqn:qmTwoL11:50}
\begin{aligned}
e^{-i \Ba \cdot \BP/\Hbar}
&= e^{- i (a_x P_x + a_y P_y + a_z P_z)/\Hbar} \\
&= e^{- i a_x P_x/\Hbar}
e^{- i a_y P_y/\Hbar}
e^{- i a_z P_z/\Hbar},
\end{aligned}
\end{equation}

where the factorization was possible because \(P_x\), \(P_y\), and \(P_z\) commute

\begin{equation}\label{eqn:qmTwoL11:70}
\antisymmetric{P_i}{P_j} = 0,
\end{equation}

for any \(i, j\) (including \(i = i\) as I dumbly questioned in class ... this is a  commutator, so \(\antisymmetric{P_i}{P_j} = P_i P_i - P_i P_i = 0\)).

The fact that the \(P_i\) commute means that successive translations can be done in any order and have the same result.

In class we were rewarded with a graphic demo of translation component commutation as Professor Sipe pulled a giant wood carving of a cat (or tiger?) out from beside the desk and proceeded to translate it around on the desk in two different orders, with the cat ending up in the same place each time.

\paragraph{Exponential commutation}

Note that in general

\begin{equation}\label{eqn:qmTwoL11:90}
e^{A + B} \ne e^A e^B,
\end{equation}

unless \(\antisymmetric{A}{B} = 0\).  To show this one can compare

\begin{equation}\label{eqn:qmTwoL11:110}
\begin{aligned}
e^{A + B}
&= 1 + A + B + \inv{2}(A + B)^2 + \cdots \\
&= 1 + A + B + \inv{2}(A^2 + A B + BA + B^2) + \cdots \\
\end{aligned}
\end{equation}

and
\begin{equation}\label{eqn:qmTwoL11:130}
\begin{aligned}
e^A e^B
&=
\left(1 + A + \inv{2}A^2 + \cdots\right)
\left(1 + B + \inv{2}B^2 + \cdots\right) \\
&= 1 + A + B + \inv{2}( A^2 + 2 A B + B^2 ) + \cdots
\end{aligned}
\end{equation}

Comparing the second order (for example) we see that we must have for equality

\begin{equation}\label{eqn:qmTwoL11:150}
A B + B A = 2 A B,
\end{equation}

or

\begin{equation}\label{eqn:qmTwoL11:170}
B A = A B,
\end{equation}

or
\begin{equation}\label{eqn:qmTwoL11:190}
\antisymmetric{A}{B} = 0
\end{equation}

\paragraph{Translating a ket}

If we consider the quantity

\begin{equation}\label{eqn:qmTwoL11:210}
e^{-i \Ba \cdot \BP/\Hbar}
\ket{\psi} = \ket{\psi'} ,
\end{equation}

does this ket ``translated'' by \(\Ba\) make any sense?  The vector \(\Ba\) lives in a 3D space and our ket \(\ket{\psi}\) lives in Hilbert space.  A quantity like this deserves some careful thought and is the subject of some such thought in the Interpretations of Quantum mechanics course.  For now, we can think of the operator and ket as a ``gadget'' that prepares a state.

A student in class pointed out that \(\ket{\psi}\) can be dependent on many degrees of freedom, for example, the positions of eight different particles.  This translation gadget in such a case acts on the whole kit and caboodle.

Now consider the matrix element

\begin{equation}\label{eqn:qmTwoL11:230}
\braket{\Br}{\psi'}
= \bra{\Br} e^{-i \Ba \cdot \BP/\Hbar} \ket{\psi}.
\end{equation}

Note that

\begin{equation}\label{eqn:qmTwoL11:670}
\begin{aligned}
\bra{\Br} e^{-i \Ba \cdot \BP/\Hbar}
&=
\left( e^{i \Ba \cdot \BP/\Hbar}
\ket{\Br} \right)^\dagger \\
&=
\left( \ket{\Br - \Ba} \right)^\dagger,
\end{aligned}
\end{equation}

so

\begin{equation}\label{eqn:qmTwoL11:250}
\braket{\Br}{\psi'}
= \braket{\Br -\Ba}{\psi},
\end{equation}

or

\begin{equation}\label{eqn:qmTwoL11:270}
\psi'(\Br) = \psi(\Br - \Ba)
\end{equation}

This is what we expect of a translated function, as illustrated in \cref{fig:qmTwoL11:qmTwoL11fig2}
\imageFigure{../../figures/phy456/qmTwoL11fig2}{Active spatial translation}{fig:qmTwoL11:qmTwoL11fig2}{0.2}
}

\makeexample{Spatial rotation}{example:qmTwoL11:3}{

We have been introduced to the angular momentum operator

\begin{equation}\label{eqn:qmTwoL11:290}
\BL = \BR \cross \BP,
\end{equation}

where
\begin{equation}\label{eqn:qmTwoL11:310}
\begin{aligned}
L_x &= Y P_z - Z P_y \\
L_y &= Z P_x - X P_z \\
L_z &= X P_y - Y P_x.
\end{aligned}
\end{equation}

We also found that

\begin{equation}\label{eqn:qmTwoL11:330}
\antisymmetric{L_i}{L_j} = i \Hbar \sum_k \epsilon_{ijk} L_k.
\end{equation}

These non-zero commutators show that the components of angular momentum \textunderline{do not} commute.

Define

\begin{equation}\label{eqn:qmTwoL11:350}
\ket{\calR(\Br)} =
e^{-i \theta \ncap \cdot \BL/\Hbar}
\ket{\Br} .
\end{equation}

This is the vector that we get by actively rotating the vector \(\Br\) by an angle \(\theta\) counterclockwise about \(\ncap\), as in \cref{fig:qmTwoL11:qmTwoL11fig3}

\imageFigure{../../figures/phy456/qmTwoL11fig3}{Active vector rotations}{fig:qmTwoL11:qmTwoL11fig3}{0.2}

An active rotation rotates the vector, leaving the coordinate system fixed, whereas a passive rotation is one for which the coordinate system is rotated, and the vector is left fixed.

Note that rotations do not commute.  Suppose that we have a pair of rotations as in \cref{fig:qmTwoL11:qmTwoL11fig4}
\imageFigure{../../figures/phy456/qmTwoL11fig4}{A example pair of non-commuting rotations}{fig:qmTwoL11:qmTwoL11fig4}{0.2}

Again, we get the graphic demo, with Professor Sipe rotating the big wooden cat sculpture.  Did he bring that in to class just to make this point (too bad I missed the first couple minutes of the lecture).

Rather amusingly, he points out that most things in life do not commute.  We get much different results if we apply the operations of putting water into the teapot and turning on the stove in different orders.

\paragraph{Rotating a ket}

With a rotation gadget

\begin{equation}\label{eqn:qmTwoL11:370}
\ket{\psi'} =
e^{-i \theta \ncap \cdot \BL/\Hbar }
\ket{\psi},
\end{equation}

we can form the matrix element
\begin{equation}\label{eqn:qmTwoL11:390}
\braket{\Br}{\psi'} =
\bra{\Br} e^{-i \theta \ncap \cdot \BL/\Hbar }
\ket{\psi}.
\end{equation}

In this we have
\begin{equation}\label{eqn:qmTwoL11:690}
\begin{aligned}
\bra{\Br} e^{-i \theta \ncap \cdot \BL/\Hbar }
&=
\left( e^{i \theta \ncap \cdot \BL/\Hbar } \ket{\Br} \right)^\dagger \\
&=
\left( \ket{\calR^{-1}(\Br) } \right)^\dagger,
\end{aligned}
\end{equation}

so
\begin{equation}\label{eqn:qmTwoL11:410}
\braket{\Br}{\psi'} =
\braket{\calR^{-1}(\Br) }{\psi'},
\end{equation}

or
\begin{equation}\label{eqn:qmTwoL11:430}
\psi'(\Br) = \psi( \calR^{-1}(\Br) )
\end{equation}
}

\section{Generalizations}

Recall what you did last year, where \(H\), \(\BP\), and \(\BL\) were defined mechanically.  We found

\begin{itemize}
\item \(H\) generates time evolution (or translation in time).
\item \(\BP\) generates spatial translation.
\item \(\BL\) generates spatial rotation.
\end{itemize}

For our mechanical definitions we have

\begin{equation}\label{eqn:qmTwoL11:450}
\antisymmetric{P_i}{P_j} = 0,
\end{equation}

and

\begin{equation}\label{eqn:qmTwoL11:470}
\antisymmetric{L_i}{L_j} = i \Hbar \sum_k \epsilon_{ijk} L_k.
\end{equation}

These are the relations that show us the way translations and rotations combine.  We want to move up to a higher plane, a new level of abstraction.  To do so we \textunderline{define} \(H\) as the operator that generates time evolution.  If we have a theory that covers the behavior of how anything evolves in time, \(H\) encodes the rules for this time evolution.

\textunderline{Define} \(\BP\) as the operator that generates translations in space.

\textunderline{Define} \(\BJ\) as the operator that generates rotations in space.

In order that these match expectations, we require

\begin{equation}\label{eqn:qmTwoL11:490}
\antisymmetric{P_i}{P_j} = 0,
\end{equation}

and

\begin{equation}\label{eqn:qmTwoL11:510}
\antisymmetric{J_i}{J_j} = i \Hbar \sum_k \epsilon_{ijk} J_k.
\end{equation}

In the simple theory of a spin less particle we have

\begin{equation}\label{eqn:qmTwoL11:530}
\BJ \equiv \BL = \BR \cross \BP.
\end{equation}

We actually need a generalization of this since this is, in fact, not good enough, even for low energy physics.

\paragraph{Many component wave functions}

We are free to construct tuples of spatial vector functions like

\begin{equation}\label{eqn:qmTwoL11:550}
\begin{bmatrix}
\Psi_I(\Br, t) \\
\Psi_{II}(\Br, t)
\end{bmatrix},
\end{equation}

or
\begin{equation}\label{eqn:qmTwoL11:570}
\begin{bmatrix}
\Psi_I(\Br, t) \\
\Psi_{II}(\Br, t) \\
\Psi_{III}(\Br, t)
\end{bmatrix},
\end{equation}

etc.

We will see that these behave qualitatively different than one component wave functions.  We also do not have to be considering multiple particle wave functions, but just \textunderline{one} particle that requires three functions in \R{3} to describe it (ie: we are moving in on spin).

\paragraph{Question:} Do these live in the same vector space?
\paragraph{Answer:} We will get to this.

\paragraph{A classical analogy}

``There is only bad analogies, since if the are good they would be describing the same thing.  We can however, produce some useful bad analogies''

\begin{enumerate}
\item A temperature field

\begin{equation}\label{eqn:qmTwoL11:590}
T(\Br)
\end{equation}

\item Electric field

\begin{equation}\label{eqn:qmTwoL11:610}
\begin{bmatrix}
E_x(\Br) \\
E_y(\Br) \\
E_z(\Br)
\end{bmatrix}
\end{equation}

\end{enumerate}

These behave in a much different way.  If we rotate a scalar field like \(T(\Br)\) as in \cref{fig:qmTwoL11:qmTwoL11fig5}
\imageFigure{../../figures/phy456/qmTwoL11fig5}{Rotated temperature (scalar) field}{fig:qmTwoL11:qmTwoL11fig5}{0.2}

Suppose we have a temperature field generated by, say, a match.  Rotating the match above, we have

\begin{equation}\label{eqn:qmTwoL11:630}
T'(\Br) = T(\calR^{-1}(\Br)).
\end{equation}

Compare this to the rotation of an electric field, perhaps one produced by a capacitor, as in \cref{fig:qmTwoL11:qmTwoL11fig6}

\imageFigure{../../figures/phy456/qmTwoL11fig6}{Rotating a capacitance electric field}{fig:qmTwoL11:qmTwoL11fig6}{0.2}

Is it true that we have
\begin{equation}\label{eqn:qmTwoL11:650}
\begin{bmatrix}
E_x(\Br) \\
E_y(\Br) \\
E_z(\Br)
\end{bmatrix}
\questionEquals
\begin{bmatrix}
E_x(\calR^{-1}(\Br)) \\
E_y(\calR^{-1}(\Br)) \\
E_z(\calR^{-1}(\Br))
\end{bmatrix}
\end{equation}

\paragraph{No.}  Because the components get mixed as well as the positions at which those components are evaluated.

We will work with many component wave functions, some of which will behave like vectors, and will have to develop the methods and language to tackle this.



