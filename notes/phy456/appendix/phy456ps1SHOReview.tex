%
% Copyright � 2013 Peeter Joot.  All Rights Reserved.
% Licenced as described in the file LICENSE under the root directory of this GIT repository.
%
\label{chap:phy456ps1SHO}

Consider
\begin{equation}\label{eqn:phy456ps1:10}
H_0 = \frac{P^2}{2m} + \inv{2} m \omega^2 X^2
\end{equation}

Since it has been a while let us compute the raising and lowering factorization that was used so extensively for this problem.

It was of the form

\begin{equation}\label{eqn:phy456ps1:30}
H_0 = (a X - i b P)(a X + i b P) + \cdots
\end{equation}

Why this factorization has an imaginary in it is a good question.  It is not one that is given any sort of rationale in the text \citep{desai2009quantum}.

It is clear that we want \(a = \sqrt{m/2} \omega\) and \(b = 1/\sqrt{2m}\).  The difference is then

\begin{equation}\label{eqn:phy456ps1:50}
H_0 - (a X - i b P)(a X + i b P)
=
- i a b \antisymmetric{X}{P}  = - i \frac{\omega}{2} \antisymmetric{X}{P}
\end{equation}

That commutator is an \(i\Hbar\) value, but what was the sign?  Let us compute so we do not get it wrong

\begin{equation}\label{eqn:phy456ps1SHOReview:320}
\begin{aligned}
\antisymmetric{x}{ p} \psi
&= -i \Hbar \antisymmetric{x}{\partial_x} \psi \\
&= -i \Hbar ( x \partial_x \psi - \partial_x (x \psi) ) \\
&= -i \Hbar ( - \psi ) \\
&= i \Hbar \psi
\end{aligned}
\end{equation}

So we have

\begin{equation}\label{eqn:phy456ps1:70}
H_0 =
\left(\omega \sqrt{\frac{m}{2}} X - i \sqrt{\frac{1}{2m}} P\right)\left(\omega \sqrt{\frac{m}{2}} X + i \sqrt{\frac{1}{2m}} P\right)
+ \frac{\Hbar \omega}{2}
\end{equation}

Factoring out an \(\Hbar \omega\) produces the form of the Hamiltonian that we used before

\begin{equation}\label{eqn:phy456ps1:90}
H_0 =
\Hbar \omega \left(
\left(\sqrt{\frac{m \omega}{2 \Hbar}} X - i \sqrt{\frac{1}{2m \Hbar \omega}} P\right)\left(\sqrt{\frac{m \omega}{2 \Hbar}} X + i \sqrt{\frac{1}{2m \Hbar \omega}} P\right)
+ \inv{2}
\right)
.
\end{equation}

The factors were labeled the uppering (\(a^\dagger\)) and lowering (\(a\)) operators respectively, and written

\begin{equation}\label{eqn:phy456ps1:110}
\begin{aligned}
H_0 &= \Hbar \omega \left( a^\dagger a + \inv{2} \right) \\
a &= \sqrt{\frac{m \omega}{2 \Hbar}} X + i \sqrt{\frac{1}{2m \Hbar \omega}} P \\
a^\dagger &= \sqrt{\frac{m \omega}{2 \Hbar}} X - i \sqrt{\frac{1}{2m \Hbar \omega}} P.
\end{aligned}
\end{equation}

Observe that we can find the inverse relations

\begin{equation}\label{eqn:phy456ps1:115}
\begin{aligned}
X &= \sqrt{ \frac{\Hbar}{2 m \omega} } \left( a + a^\dagger \right) \\
P &= i \sqrt{ \frac{m \Hbar \omega}{2} } \left( a^\dagger  - a \right)
\end{aligned}
\end{equation}

\paragraph{Question}
What is a good reason that we chose this particular factorization?  For example, a quick computation shows that we could have also picked

\begin{equation}\label{eqn:phy456ps1:130}
H_0 = \Hbar \omega \left( a a^\dagger - \inv{2} \right).
\end{equation}

I do not know that answer.  That said, this second factorization is useful in that it provides the commutator relation between the raising and lowering operators, since subtracting \eqnref{eqn:phy456ps1:130} and \eqnref{eqn:phy456ps1:110} yields

\begin{equation}\label{eqn:phy456ps1:150}
\antisymmetric{a}{a^\dagger} = 1.
\end{equation}

If we suppose that we have eigenstates for the operator \(a^\dagger a\) of the form

\begin{equation}\label{eqn:phy456ps1:200}
a^\dagger a \ket{n} = \lambda_n \ket{n},
\end{equation}

then the problem of finding the eigensolution of \(H_0\) reduces to solving this problem.  Because \(a^\dagger a\) commutes with \(1/2\), an eigenstate of \(a^\dagger a\) is also an eigenstate of \(H_0\).  Utilizing \eqnref{eqn:phy456ps1:150} we then have

\begin{equation}\label{eqn:phy456ps1SHOReview:340}
\begin{aligned}
a^\dagger a ( a \ket{n} )
&= (a a^\dagger - 1 ) a \ket{n} \\
&= a (a^\dagger a - 1 ) \ket{n} \\
&= a (\lambda_n - 1 ) \ket{n} \\
&= (\lambda_n - 1 ) a \ket{n},
\end{aligned}
\end{equation}

so we see that \(a \ket{n}\) is an eigenstate of \(a^\dagger a\) with eigenvalue \(\lambda_n - 1\).

Similarly for the raising operator

\begin{equation}\label{eqn:phy456ps1SHOReview:360}
\begin{aligned}
a^\dagger a ( a^\dagger \ket{n} )
&=
a^\dagger (a  a^\dagger) \ket{n} ) \\
&=
a^\dagger (a^\dagger a + 1) \ket{n} ) \\
&=
a^\dagger (\lambda_n + 1) \ket{n} ),
\end{aligned}
\end{equation}

and find that \(a^\dagger \ket{n}\) is also an eigenstate of \(a^\dagger a\) with eigenvalue \(\lambda_n + 1\).

Supposing that there is a lowest energy level (because the potential \(V(x) = m \omega x^2 /2\) has a lower bound of zero) then the state \(\ket{0}\) for which the energy is the lowest when operated on by \(a\) we have

\begin{equation}\label{eqn:phy456ps1:220}
a \ket{0} = 0
\end{equation}

Thus

\begin{equation}\label{eqn:phy456ps1:240}
a^\dagger a \ket{0} = 0,
\end{equation}

and

\begin{equation}\label{eqn:phy456ps1:260}
\lambda_0 = 0.
\end{equation}

This seems like a small bit of slight of hand, since it sneakily supplies an integer value to \(\lambda_0\) where up to this point \(0\) was just a label.

If the eigenvalue equation we are trying to solve for the Hamiltonian is

\begin{equation}\label{eqn:phy456ps1:280}
H_0 \ket{n} = E_n \ket{n}.
\end{equation}

Then we must then have

\begin{equation}\label{eqn:phy456ps1:300}
E_n = \Hbar \omega \left(\lambda_n + \inv{2} \right) = \Hbar \omega \left(n + \inv{2} \right)
\end{equation}
