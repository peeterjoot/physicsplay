%%
% Copyright � 2015 Peeter Joot.  All Rights Reserved.
% Licenced as described in the file LICENSE under the root directory of this GIT repository.
%
\documentclass[]{eliblog}

\usepackage{amsmath}
\usepackage{mathpazo}

%
% shorthand for bold symbols, convenient for vectors and matrices
%
\newcommand{\Ba}[0]{\mathbf{a}}
\newcommand{\Bb}[0]{\mathbf{b}}
\newcommand{\Bc}[0]{\mathbf{c}}
\newcommand{\Bd}[0]{\mathbf{d}}
\newcommand{\Be}[0]{\mathbf{e}}
\newcommand{\Bf}[0]{\mathbf{f}}
\newcommand{\Bg}[0]{\mathbf{g}}
\newcommand{\Bh}[0]{\mathbf{h}}
\newcommand{\Bi}[0]{\mathbf{i}}
\newcommand{\Bj}[0]{\mathbf{j}}
\newcommand{\Bk}[0]{\mathbf{k}}
\newcommand{\Bl}[0]{\mathbf{l}}
\newcommand{\Bm}[0]{\mathbf{m}}
\newcommand{\Bn}[0]{\mathbf{n}}
\newcommand{\Bo}[0]{\mathbf{o}}
\newcommand{\Bp}[0]{\mathbf{p}}
\newcommand{\Bq}[0]{\mathbf{q}}
\newcommand{\Br}[0]{\mathbf{r}}
\newcommand{\Bs}[0]{\mathbf{s}}
\newcommand{\Bt}[0]{\mathbf{t}}
\newcommand{\Bu}[0]{\mathbf{u}}
\newcommand{\Bv}[0]{\mathbf{v}}
\newcommand{\Bw}[0]{\mathbf{w}}
\newcommand{\Bx}[0]{\mathbf{x}}
\newcommand{\By}[0]{\mathbf{y}}
\newcommand{\Bz}[0]{\mathbf{z}}
\newcommand{\BA}[0]{\mathbf{A}}
\newcommand{\BB}[0]{\mathbf{B}}
\newcommand{\BC}[0]{\mathbf{C}}
\newcommand{\BD}[0]{\mathbf{D}}
\newcommand{\BE}[0]{\mathbf{E}}
\newcommand{\BF}[0]{\mathbf{F}}
\newcommand{\BG}[0]{\mathbf{G}}
\newcommand{\BH}[0]{\mathbf{H}}
\newcommand{\BI}[0]{\mathbf{I}}
\newcommand{\BJ}[0]{\mathbf{J}}
\newcommand{\BK}[0]{\mathbf{K}}
\newcommand{\BL}[0]{\mathbf{L}}
\newcommand{\BM}[0]{\mathbf{M}}
\newcommand{\BN}[0]{\mathbf{N}}
\newcommand{\BO}[0]{\mathbf{O}}
\newcommand{\BP}[0]{\mathbf{P}}
\newcommand{\BQ}[0]{\mathbf{Q}}
\newcommand{\BR}[0]{\mathbf{R}}
\newcommand{\BS}[0]{\mathbf{S}}
\newcommand{\BT}[0]{\mathbf{T}}
\newcommand{\BU}[0]{\mathbf{U}}
\newcommand{\BV}[0]{\mathbf{V}}
\newcommand{\BW}[0]{\mathbf{W}}
\newcommand{\BX}[0]{\mathbf{X}}
\newcommand{\BY}[0]{\mathbf{Y}}
\newcommand{\BZ}[0]{\mathbf{Z}}

\newcommand{\Bzero}[0]{\mathbf{0}}
\newcommand{\Btheta}[0]{\boldsymbol{\theta}}
\newcommand{\Btau}[0]{\boldsymbol{\tau}}
\newcommand{\Bomega}[0]{\boldsymbol{\omega}}

%
% shorthand for unit vectors
%
\newcommand{\acap}[0]{\hat{\Ba}}
\newcommand{\bcap}[0]{\hat{\Bb}}
\newcommand{\ccap}[0]{\hat{\Bc}}
\newcommand{\dcap}[0]{\hat{\Bd}}
\newcommand{\ecap}[0]{\hat{\Be}}
\newcommand{\fcap}[0]{\hat{\Bf}}
\newcommand{\gcap}[0]{\hat{\Bg}}
\newcommand{\hcap}[0]{\hat{\Bh}}
\newcommand{\icap}[0]{\hat{\Bi}}
\newcommand{\jcap}[0]{\hat{\Bj}}
\newcommand{\kcap}[0]{\hat{\Bk}}
\newcommand{\lcap}[0]{\hat{\Bl}}
\newcommand{\mcap}[0]{\hat{\Bm}}
\newcommand{\ncap}[0]{\hat{\Bn}}
\newcommand{\ocap}[0]{\hat{\Bo}}
\newcommand{\pcap}[0]{\hat{\Bp}}
\newcommand{\qcap}[0]{\hat{\Bq}}
\newcommand{\rcap}[0]{\hat{\Br}}
\newcommand{\scap}[0]{\hat{\Bs}}
\newcommand{\tcap}[0]{\hat{\Bt}}
\newcommand{\ucap}[0]{\hat{\Bu}}
\newcommand{\vcap}[0]{\hat{\Bv}}
\newcommand{\wcap}[0]{\hat{\Bw}}
\newcommand{\xcap}[0]{\hat{\Bx}}
\newcommand{\ycap}[0]{\hat{\By}}
\newcommand{\zcap}[0]{\hat{\Bz}}
\newcommand{\thetacap}[0]{\hat{\Btheta}}

%
% to write R^n and C^n in a distinguishable fashion.  Perhaps change this
% to the double lined characters upon figuring out how to do so.
%
\newcommand{\C}[1]{$\mathbb{C}^{#1}$}
\newcommand{\R}[1]{$\mathbb{R}^{#1}$}

%
% various generally useful helpers
%

% derivative of #1 wrt. #2:
\newcommand{\D}[2] {\frac {d#2} {d#1}}

\newcommand{\inv}[1]{\frac{1}{#1}}
\newcommand{\cross}[0]{\times}

\newcommand{\abs}[1]{\lvert{#1}\rvert}
\newcommand{\norm}[1]{\lVert{#1}\rVert}
\newcommand{\innerprod}[2]{\langle{#1}, {#2}\rangle}
\newcommand{\dotprod}[2]{{#1} \cdot {#2}}
\newcommand{\bdotprod}[2]{\left({#1} \cdot {#2}\right)}
\newcommand{\crossprod}[2]{{#1} \cross {#2}}
\newcommand{\tripleprod}[3]{\dotprod{\left(\crossprod{#1}{#2}\right)}{#3}}

\DeclareMathOperator{\Proj}{Proj}
\DeclareMathOperator{\Span}{span}
\DeclareMathOperator{\Sgn}{sgn}
\DeclareMathOperator{\Area}{Area}
\DeclareMathOperator{\Volume}{Volume}

%
% A few miscellaneous things specific to this document
%
\newcommand{\crossop}[1]{\crossprod{#1}{}}

% R2 vector.
\newcommand{\VectorTwo}[2]{
\begin{bmatrix}
 {#1} \\
 {#2}
\end{bmatrix}
}

\newcommand{\VectorN}[1]{
\begin{bmatrix}
{#1}_1 \\
{#1}_2 \\
\vdots \\
{#1}_N \\
\end{bmatrix}
}

\newcommand{\DETuvij}[4]{
\begin{vmatrix}
 {#1}_{#3} & {#1}_{#4} \\
 {#2}_{#3} & {#2}_{#4}
\end{vmatrix}
}

\newcommand{\DETuvwijk}[6]{
\begin{vmatrix}
 {#1}_{#4} & {#1}_{#5} & {#1}_{#6} \\
 {#2}_{#4} & {#2}_{#5} & {#2}_{#6} \\
 {#3}_{#4} & {#3}_{#5} & {#3}_{#6}
\end{vmatrix}
}

\newcommand{\DETuvwxijkl}[8]{
\begin{vmatrix}
 {#1}_{#5} & {#1}_{#6} & {#1}_{#7} & {#1}_{#8} \\
 {#2}_{#5} & {#2}_{#6} & {#2}_{#7} & {#2}_{#8} \\
 {#3}_{#5} & {#3}_{#6} & {#3}_{#7} & {#3}_{#8} \\
 {#4}_{#5} & {#4}_{#6} & {#4}_{#7} & {#4}_{#8} \\
\end{vmatrix}
}

%\newcommand{\DETuvwxyijklm}[10]{
%\begin{vmatrix}
% {#1}_{#6} & {#1}_{#7} & {#1}_{#8} & {#1}_{#9} & {#1}_{#10} \\
% {#2}_{#6} & {#2}_{#7} & {#2}_{#8} & {#2}_{#9} & {#2}_{#10} \\
% {#3}_{#6} & {#3}_{#7} & {#3}_{#8} & {#3}_{#9} & {#3}_{#10} \\
% {#4}_{#6} & {#4}_{#7} & {#4}_{#8} & {#4}_{#9} & {#4}_{#10} \\
% {#5}_{#6} & {#5}_{#7} & {#5}_{#8} & {#5}_{#9} & {#5}_{#10}
%\end{vmatrix}
%}

% R3 vector.
\newcommand{\VectorThree}[3]{
\begin{bmatrix}
 {#1} \\
 {#2} \\
 {#3}
\end{bmatrix}
}



\author{Peeter Joot}
\email{peeter.joot@gmail.com}

%\documentclass[]{eliblogwidescreen}

\usepackage{amsmath}
\usepackage{mathpazo}

%
% shorthand for bold symbols, convenient for vectors and matrices
%
\newcommand{\Ba}[0]{\mathbf{a}}
\newcommand{\Bb}[0]{\mathbf{b}}
\newcommand{\Bc}[0]{\mathbf{c}}
\newcommand{\Bd}[0]{\mathbf{d}}
\newcommand{\Be}[0]{\mathbf{e}}
\newcommand{\Bf}[0]{\mathbf{f}}
\newcommand{\Bg}[0]{\mathbf{g}}
\newcommand{\Bh}[0]{\mathbf{h}}
\newcommand{\Bi}[0]{\mathbf{i}}
\newcommand{\Bj}[0]{\mathbf{j}}
\newcommand{\Bk}[0]{\mathbf{k}}
\newcommand{\Bl}[0]{\mathbf{l}}
\newcommand{\Bm}[0]{\mathbf{m}}
\newcommand{\Bn}[0]{\mathbf{n}}
\newcommand{\Bo}[0]{\mathbf{o}}
\newcommand{\Bp}[0]{\mathbf{p}}
\newcommand{\Bq}[0]{\mathbf{q}}
\newcommand{\Br}[0]{\mathbf{r}}
\newcommand{\Bs}[0]{\mathbf{s}}
\newcommand{\Bt}[0]{\mathbf{t}}
\newcommand{\Bu}[0]{\mathbf{u}}
\newcommand{\Bv}[0]{\mathbf{v}}
\newcommand{\Bw}[0]{\mathbf{w}}
\newcommand{\Bx}[0]{\mathbf{x}}
\newcommand{\By}[0]{\mathbf{y}}
\newcommand{\Bz}[0]{\mathbf{z}}
\newcommand{\BA}[0]{\mathbf{A}}
\newcommand{\BB}[0]{\mathbf{B}}
\newcommand{\BC}[0]{\mathbf{C}}
\newcommand{\BD}[0]{\mathbf{D}}
\newcommand{\BE}[0]{\mathbf{E}}
\newcommand{\BF}[0]{\mathbf{F}}
\newcommand{\BG}[0]{\mathbf{G}}
\newcommand{\BH}[0]{\mathbf{H}}
\newcommand{\BI}[0]{\mathbf{I}}
\newcommand{\BJ}[0]{\mathbf{J}}
\newcommand{\BK}[0]{\mathbf{K}}
\newcommand{\BL}[0]{\mathbf{L}}
\newcommand{\BM}[0]{\mathbf{M}}
\newcommand{\BN}[0]{\mathbf{N}}
\newcommand{\BO}[0]{\mathbf{O}}
\newcommand{\BP}[0]{\mathbf{P}}
\newcommand{\BQ}[0]{\mathbf{Q}}
\newcommand{\BR}[0]{\mathbf{R}}
\newcommand{\BS}[0]{\mathbf{S}}
\newcommand{\BT}[0]{\mathbf{T}}
\newcommand{\BU}[0]{\mathbf{U}}
\newcommand{\BV}[0]{\mathbf{V}}
\newcommand{\BW}[0]{\mathbf{W}}
\newcommand{\BX}[0]{\mathbf{X}}
\newcommand{\BY}[0]{\mathbf{Y}}
\newcommand{\BZ}[0]{\mathbf{Z}}

\newcommand{\Bzero}[0]{\mathbf{0}}
\newcommand{\Btheta}[0]{\boldsymbol{\theta}}
\newcommand{\Btau}[0]{\boldsymbol{\tau}}
\newcommand{\Bomega}[0]{\boldsymbol{\omega}}

%
% shorthand for unit vectors
%
\newcommand{\acap}[0]{\hat{\Ba}}
\newcommand{\bcap}[0]{\hat{\Bb}}
\newcommand{\ccap}[0]{\hat{\Bc}}
\newcommand{\dcap}[0]{\hat{\Bd}}
\newcommand{\ecap}[0]{\hat{\Be}}
\newcommand{\fcap}[0]{\hat{\Bf}}
\newcommand{\gcap}[0]{\hat{\Bg}}
\newcommand{\hcap}[0]{\hat{\Bh}}
\newcommand{\icap}[0]{\hat{\Bi}}
\newcommand{\jcap}[0]{\hat{\Bj}}
\newcommand{\kcap}[0]{\hat{\Bk}}
\newcommand{\lcap}[0]{\hat{\Bl}}
\newcommand{\mcap}[0]{\hat{\Bm}}
\newcommand{\ncap}[0]{\hat{\Bn}}
\newcommand{\ocap}[0]{\hat{\Bo}}
\newcommand{\pcap}[0]{\hat{\Bp}}
\newcommand{\qcap}[0]{\hat{\Bq}}
\newcommand{\rcap}[0]{\hat{\Br}}
\newcommand{\scap}[0]{\hat{\Bs}}
\newcommand{\tcap}[0]{\hat{\Bt}}
\newcommand{\ucap}[0]{\hat{\Bu}}
\newcommand{\vcap}[0]{\hat{\Bv}}
\newcommand{\wcap}[0]{\hat{\Bw}}
\newcommand{\xcap}[0]{\hat{\Bx}}
\newcommand{\ycap}[0]{\hat{\By}}
\newcommand{\zcap}[0]{\hat{\Bz}}
\newcommand{\thetacap}[0]{\hat{\Btheta}}

%
% to write R^n and C^n in a distinguishable fashion.  Perhaps change this
% to the double lined characters upon figuring out how to do so.
%
\newcommand{\C}[1]{$\mathbb{C}^{#1}$}
\newcommand{\R}[1]{$\mathbb{R}^{#1}$}

%
% various generally useful helpers
%

% derivative of #1 wrt. #2:
\newcommand{\D}[2] {\frac {d#2} {d#1}}

\newcommand{\inv}[1]{\frac{1}{#1}}
\newcommand{\cross}[0]{\times}

\newcommand{\abs}[1]{\lvert{#1}\rvert}
\newcommand{\norm}[1]{\lVert{#1}\rVert}
\newcommand{\innerprod}[2]{\langle{#1}, {#2}\rangle}
\newcommand{\dotprod}[2]{{#1} \cdot {#2}}
\newcommand{\bdotprod}[2]{\left({#1} \cdot {#2}\right)}
\newcommand{\crossprod}[2]{{#1} \cross {#2}}
\newcommand{\tripleprod}[3]{\dotprod{\left(\crossprod{#1}{#2}\right)}{#3}}

\DeclareMathOperator{\Proj}{Proj}
\DeclareMathOperator{\Span}{span}
\DeclareMathOperator{\Sgn}{sgn}
\DeclareMathOperator{\Area}{Area}
\DeclareMathOperator{\Volume}{Volume}

%
% A few miscellaneous things specific to this document
%
\newcommand{\crossop}[1]{\crossprod{#1}{}}

% R2 vector.
\newcommand{\VectorTwo}[2]{
\begin{bmatrix}
 {#1} \\
 {#2}
\end{bmatrix}
}

\newcommand{\VectorN}[1]{
\begin{bmatrix}
{#1}_1 \\
{#1}_2 \\
\vdots \\
{#1}_N \\
\end{bmatrix}
}

\newcommand{\DETuvij}[4]{
\begin{vmatrix}
 {#1}_{#3} & {#1}_{#4} \\
 {#2}_{#3} & {#2}_{#4}
\end{vmatrix}
}

\newcommand{\DETuvwijk}[6]{
\begin{vmatrix}
 {#1}_{#4} & {#1}_{#5} & {#1}_{#6} \\
 {#2}_{#4} & {#2}_{#5} & {#2}_{#6} \\
 {#3}_{#4} & {#3}_{#5} & {#3}_{#6}
\end{vmatrix}
}

\newcommand{\DETuvwxijkl}[8]{
\begin{vmatrix}
 {#1}_{#5} & {#1}_{#6} & {#1}_{#7} & {#1}_{#8} \\
 {#2}_{#5} & {#2}_{#6} & {#2}_{#7} & {#2}_{#8} \\
 {#3}_{#5} & {#3}_{#6} & {#3}_{#7} & {#3}_{#8} \\
 {#4}_{#5} & {#4}_{#6} & {#4}_{#7} & {#4}_{#8} \\
\end{vmatrix}
}

%\newcommand{\DETuvwxyijklm}[10]{
%\begin{vmatrix}
% {#1}_{#6} & {#1}_{#7} & {#1}_{#8} & {#1}_{#9} & {#1}_{#10} \\
% {#2}_{#6} & {#2}_{#7} & {#2}_{#8} & {#2}_{#9} & {#2}_{#10} \\
% {#3}_{#6} & {#3}_{#7} & {#3}_{#8} & {#3}_{#9} & {#3}_{#10} \\
% {#4}_{#6} & {#4}_{#7} & {#4}_{#8} & {#4}_{#9} & {#4}_{#10} \\
% {#5}_{#6} & {#5}_{#7} & {#5}_{#8} & {#5}_{#9} & {#5}_{#10}
%\end{vmatrix}
%}

% R3 vector.
\newcommand{\VectorThree}[3]{
\begin{bmatrix}
 {#1} \\
 {#2} \\
 {#3}
\end{bmatrix}
}



\author{Peeter Joot}
\email{peeter.joot@gmail.com}


\chapter{Review: Composite systems}
\label{chap:qmTwoL1}
%\useCCL
\blogpage{http://sites.google.com/site/peeterjoot/math2011/qmTwoL1.pdf}
\date{Sept 12, 2011}
\revisionInfo{qmTwoL1.tex}

\beginArtWithToc
%\beginArtNoToc

\section{Composite systems.}

This is apparently covered as a side effect in the text \cite{desai2009quantum} in one of the advanced material sections.  FIXME: what section?

Example, one spin one half particle and one spin one particle.  We can describe either quantum mechanically, described by a pair of Hilbert spaces

\begin{equation}\label{eqn:qmTwoL1:10}
H_1,
\end{equation}

of dimension $D_1$

\begin{equation}\label{eqn:qmTwoL1:30}
H_2,
\end{equation}

of dimension $D_2$

Recall that a Hilbert space (finite or infinite dimensional) is the set of states that describe the system.  There were some additional details (completeness, normalizable, $L2$ integrable, ...) not really covered in the physics curriculum, but available in mathematical descriptions.

We form the composite (Hilbert) space

\begin{equation}\label{eqn:qmTwoL1:50}
H = H_1 \otimes H_2
\end{equation}

\begin{equation}\label{eqn:qmTwoL1:70}
H_1 : { \ket{\phi_1^{(i)}} }
\end{equation}

for any ket in $H_1$

\begin{equation}\label{eqn:qmTwoL1:90}
\ket{I} = \sum_{i=1}^{D_1} c_i \ket{\phi_1^{(i)}} 
\end{equation}

where

\begin{equation}\label{eqn:qmTwoL1:110}
\braket{ \phi_1^{(i)}}{ \phi_1^{(j)}} = \delta^{i j}
\end{equation}

Similarly
\begin{equation}\label{eqn:qmTwoL1:130}
H_2 : { \ket{\phi_2^{(i)}} }
\end{equation}

for any ket in $H_2$

\begin{equation}\label{eqn:qmTwoL1:150}
\ket{II} = \sum_{i=1}^{D_2} d_i \ket{\phi_2^{(i)}} 
\end{equation}

where

\begin{equation}\label{eqn:qmTwoL1:170}
\braket{ \phi_2^{(i)}}{ \phi_2^{(j)}} = \delta^{i j}
\end{equation}

The composite Hilbert space has dimension $D_1 D_2$

basis kets:

\begin{equation}\label{eqn:qmTwoL1:190}
\ket{ \phi_1^{(i)}} \otimes \ket{ \phi_2^{(j)}}  = \ket{ \phi^{(ij)}},
\end{equation}

where
\begin{equation}\label{eqn:qmTwoL1:210}
\braket{ \phi^{(ij)}}{ \phi^{(kl)}} = \delta^{ik} \delta^{jl}.
\end{equation}

Any ket in $H$ can be written

\begin{align*}
\ket{\psi} 
&= 
\sum_{i = 1}^{D_1}
\sum_{j = 1}^{D_2}
f_{ij}
\ket{ \phi_1^{(i)}} \otimes \ket{ \phi_2^{(j)}}  \\
&= 
\sum_{i = 1}^{D_1}
\sum_{j = 1}^{D_2}
f_{ij}
\ket{ \phi^{(ij)}}.
\end{align*}

\paragraph{Direct product of kets:}

\begin{align*}
\ket{I} \otimes \ket{II} 
&\equiv
\sum_{i = 1}^{D_1}
\sum_{j = 1}^{D_2}
c_i d_j
\ket{ \phi_1^{(i)}} \otimes \ket{ \phi_2^{(j)}} \\
&=
\sum_{i = 1}^{D_1}
\sum_{j = 1}^{D_2}
c_i d_j
\ket{ \phi^{(ij)}} 
\end{align*}

If $\ket{\psi}$ in $H$ cannot be written as $\ket{I} \otimes \ket{II}$, then $\ket{\psi}$ is said to be ``entangled''.

FIXME: insert a concrete example of this, with some low dimension.

\subsection{Operators.}

With operators $\mathcal{O}_1$ and $\mathcal{O}_2$ on the respective Hilbert spaces.  We'd now like to build 

\begin{equation}\label{eqn:qmTwoL1:230}
\mathcal{O}_1 \otimes \mathcal{O}_2
\end{equation}

If one defines
\begin{equation}\label{eqn:qmTwoL1:250}
\mathcal{O}_1 \otimes \mathcal{O}_2
\equiv
\sum_{i = 1}^{D_1}
\sum_{j = 1}^{D_2}
f_{ij}
\ket{ \mathcal{O}_1 \phi_1^{(i)}} \otimes \ket{ \mathcal{O}_2 \phi_2^{(j)}} 
\end{equation}

\paragraph{Q:Can every operator that can be defined on the composite space have a representation of this form?}

No.

Special cases.  The identity operators.  Suppose that 

\begin{equation}\label{eqn:qmTwoL1:270}
\ket{\psi}
=
\sum_{i = 1}^{D_1}
\sum_{j = 1}^{D_2}
f_{ij}
\ket{ \phi_1^{(i)}} \otimes \ket{ \phi_2^{(j)}} 
\end{equation}

then

\begin{equation}\label{eqn:qmTwoL1:290}
(\mathcal{O}_1 \otimes \mathcal{I}_2) \ket{\psi}
=
\sum_{i = 1}^{D_1}
\sum_{j = 1}^{D_2}
f_{ij}
\ket{ \mathcal{O}_1 \phi_1^{(i)}} \otimes \ket{ \phi_2^{(j)}} 
\end{equation}

\subsubsection{Example commutator.}

Can do other operations.  Example:

\begin{equation}\label{eqn:qmTwoL1:310}
\antisymmetric{ \mathcal{O}_1 \otimes \mathcal{I}_2 }{ \mathcal{I}_1 \otimes \mathcal{O}_2 } = 0
\end{equation}

Let's verify this one.  Suppose that our state has the representation

\begin{equation}\label{eqn:qmTwoL1:330}
\ket{\psi} 
= 
\sum_{i = 1}^{D_1}
\sum_{j = 1}^{D_2}
f_{ij}
\ket{ \phi_1^{(i)}} \otimes \ket{ \phi_2^{(j)}}
\end{equation}

so that the action on this ket from the composite operations are
\begin{align}\label{eqn:qmTwoL1:350}
(\mathcal{O}_1 \otimes \mathcal{I}_2)
\ket{\psi} 
&= 
\sum_{i = 1}^{D_1}
\sum_{j = 1}^{D_2}
f_{ij}
\ket{ \mathcal{O}_1 \phi_1^{(i)}} \otimes \ket{ \phi_2^{(j)}} \\
(\mathcal{I}_1 \otimes \mathcal{O}_2)
\ket{\psi} 
&= 
\sum_{i = 1}^{D_1}
\sum_{j = 1}^{D_2}
f_{ij}
\ket{ \phi_1^{(i)}} \otimes \ket{ \mathcal{O}_2 \phi_2^{(j)}}
\end{align}

Our commutator is
\begin{align*}
&\antisymmetric{(\mathcal{O}_1 \otimes \mathcal{I}_2)}{(\mathcal{I}_1 \otimes \mathcal{O}_2)}
\ket{\psi} \\
&=
(\mathcal{O}_1 \otimes \mathcal{I}_2)(\mathcal{I}_1 \otimes \mathcal{O}_2) 
\ket{\psi} 
-(\mathcal{I}_1 \otimes \mathcal{O}_2)(\mathcal{O}_1 \otimes \mathcal{I}_2)
\ket{\psi}  \\
&=
(\mathcal{O}_1 \otimes \mathcal{I}_2)
\sum_{i = 1}^{D_1}
\sum_{j = 1}^{D_2}
f_{ij}
\ket{ \phi_1^{(i)}} \otimes \ket{ \mathcal{O}_2 \phi_2^{(j)}}
-(\mathcal{I}_1 \otimes \mathcal{O}_2)
\sum_{i = 1}^{D_1}
\sum_{j = 1}^{D_2}
f_{ij}
\ket{ \mathcal{O}_1 \phi_1^{(i)}} \otimes \ket{ \phi_2^{(j)}} \\
&=
\sum_{i = 1}^{D_1}
\sum_{j = 1}^{D_2}
f_{ij}
\ket{ \mathcal{O}_1 \phi_1^{(i)}} \otimes \ket{ \mathcal{O}_2 \phi_2^{(j)}}
-
\sum_{i = 1}^{D_1}
\sum_{j = 1}^{D_2}
f_{ij}
\ket{ \mathcal{O}_1 \phi_1^{(i)}} \otimes \ket{ \mathcal{O}_2 \phi_2^{(j)}} \\
&=
0 \qquad \square
\end{align*}

\subsubsection{Generalizations.}

Can generalize to 

\begin{equation}\label{eqn:qmTwoL1:370}
H_1 \otimes H_2 \otimes H_3 \otimes \cdots
\end{equation}

Can also start with $H$ and seek factor spaces.  If $H$ is not prime there are, in general, many ways to find factor spaces

\begin{equation}\label{eqn:qmTwoL1:390}
H = 
H_1 \otimes H_2 =
H_1' \otimes H_2'
\end{equation}

A ket $\ket{\psi}$, if unentangled in the first factor space, then it will be in general entangled in a second space.  Thus ket entanglement is not a property of the ket itself, but instead is intrinsically related to the space in which it is represented.

\section{An example}

We had one example of a composite system in phy356 that I recall.
It was related to states of the silver atoms in a Stern Gerlach
apparatus, where we had one state from the Hamiltonian that governs
position and momentum and another from the Hamiltonian for the spin,
where each of these states was considered separately.

This makes me wonder what would the Hamiltonian for a system (say a single
electron) that includes both spin and position/momentum would look
like, and how is it that one can solve this taking spin and non-spin
states separately?

Professor Sipe, when asked said of this

``It is complicated because not only would the spin of the electron interact with the magnetic field, but its translational motion would respond to the magnetic field too.  A simpler case is a neutral atom with an electron with an unpaired spin.  Then there is no Lorentz force on the atom itself.  The Hamiltonian is just the sum of a free particle Hamiltonian and a Zeeman term due to the spin interacting with the magnetic field.  This is precisely the Stern-Gerlach problem''

I didn't remember what the Zeeman term looked like, but wikipedia does \cite{wiki:zeeman}, and it is the magnetic field interaction

\begin{equation}\label{eqn:qmTwoL1:n}
-\Bmu \cdot \BB
\end{equation}

that we get when we gauge transform the Dirac equation for the electron as covered in \S 36.4 of the text (also introduced in chapter 6, which wasn't covered in class).  That doesn't look too much like how we studied the Stern-Gerlach problem?  I thought that for that problem we had a Hamiltonian of the form

\begin{equation}\label{eqn:qmTwoL1:n}
H = a_{i j} \ket{i} \bra{j}
\end{equation}

It's not clear to me how this ket-bra Hamiltonian and the Zeeman Hamiltonian are related (ie: the spin Hamiltonians that we used in 356 and were on old 356 exams were all pulled out of magic hats and it wasn't obvious where these came from).

\EndArticle
