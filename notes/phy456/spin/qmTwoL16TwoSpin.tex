\section{Two spins}

READING: \S 28 of \cite{desai2009quantum}.

\paragraph{Example}: Consider two electrons, 1 in each of 2 quantum dots.

\begin{equation}\label{eqn:qmTwoL16:190}
H = H_{1} \otimes H_{2}
\end{equation}

where $H_1$ and $H_2$ are both spin Hamiltonian's for respective 2D Hilbert spaces.  Our complete Hilbert space is thus a 4D space.

We will write

\begin{equation}\label{eqn:qmTwoL16:210}
\begin{aligned}
\ket{+}_1 \otimes \ket{+}_2 &= \ket{++} \\
\ket{+}_1 \otimes \ket{-}_2 &= \ket{+-} \\
\ket{-}_1 \otimes \ket{+}_2 &= \ket{-+} \\
\ket{-}_1 \otimes \ket{-}_2 &= \ket{--} 
\end{aligned}
\end{equation}

Can introduce

\begin{align}\label{eqn:qmTwoL16:230}
\BS_1 &= \BS_1^{(1)} \otimes I^{(2)} \\
\BS_2 &= I^{(1)} \otimes \BS_2^{(2)}
\end{align}

Here we ``promote'' each of the individual spin operators to spin operators in the complete Hilbert space.

We write

\begin{align}\label{eqn:qmTwoL16:250}
S_{1z}\ket{++} &= \frac{\hbar}{2} \ket{++} \\
S_{1z}\ket{+-} &= \frac{\hbar}{2} \ket{+-}
\end{align}

Write
\begin{equation}\label{eqn:qmTwoL16:270}
\BS = \BS_1 + \BS_2,
\end{equation}

for the full spin angular momentum operator.  The $z$ component of this operator is 

\begin{equation}\label{eqn:qmTwoL16:290}
S_z = S_{1z} + S_{2z}
\end{equation}

\begin{align}\label{eqn:qmTwoL16:310}
S_z\ket{++} &= (S_{1z} + S_{2z}) \ket{++} = \left( \frac{\hbar}{2} +\frac{\hbar}{2} \right) \ket{++} = \hbar \ket{++} \\ 
S_z\ket{+-} &= (S_{1z} + S_{2z}) \ket{+-} = \left( \frac{\hbar}{2} -\frac{\hbar}{2} \right) \ket{+-} = 0 \\
S_z\ket{-+} &= (S_{1z} + S_{2z}) \ket{-+} = \left( -\frac{\hbar}{2} +\frac{\hbar}{2} \right) \ket{-+} = 0 \\
S_z\ket{--} &= (S_{1z} + S_{2z}) \ket{--} = \left( -\frac{\hbar}{2} -\frac{\hbar}{2} \right) \ket{--} = -\hbar \ket{--} 
\end{align}

So, we find that $\ket{x x}$ are all eigenkets of $S_z$.  These will also all be eigenkets of $\BS_1^2 = S_{1x}^2 +S_{1y}^2 +S_{1z}^2$ since we have

\begin{align}\label{eqn:qmTwoL16:330}
S_1^2 \ket{x x} &= \hbar^2 \left(\inv{2}\right) \left(1 + \inv{2}\right) \ket{x x} = \frac{3}{4} \hbar^2 \ket{x x} \\
S_2^2 \ket{x x} &= \hbar^2 \left(\inv{2}\right) \left(1 + \inv{2}\right) \ket{x x} = \frac{3}{4} \hbar^2 \ket{x x} 
\end{align}

\begin{equation}\label{eqn:qmTwoL16:350}
\begin{aligned}
S^2 &= 
(\BS_1
+\BS_2) 
\cdot
(\BS_1
+\BS_2)  \\
&= 
S_1^2 + S_2^2 + 2 \BS_1 \cdot \BS_2
\end{aligned}
\end{equation}

Note that we have a commutation assumption here $\antisymmetric{S_{1i}}{S_{2i}} = 0$, since we have written $2 \BS_1 \cdot \BS_2$ instead of $\sum_i S_{1i}S_{2i} + S_{2i}S_{1i}$.  The justification for this appears to be the promotion of the spin operators in \ref{eqn:qmTwoL16:230} to operators in the complete Hilbert space, since each of these spin operators acts only on the kets associated with their index.

Are all the product kets also eigenkets of $S^2$?  Calculate

\begin{align*}
S^2 \ket{+-} 
&= 
(S_1^2 + S_2^2 + 2 \BS_1 \cdot \BS_2) \ket{+-} \\
&=
\left(\frac{3}{4}\hbar^2
+\frac{3}{4}\hbar^2\right)
+ 2 S_{1x} S_{2x} \ket{+-} 
+ 2 S_{1y} S_{2y} \ket{+-} 
+ 2 S_{1z} S_{2z} \ket{+-} 
\end{align*}

For the $z$ mixed terms, we have

\begin{equation}\label{eqn:qmTwoL16:370}
2 S_{1z} S_{2z} \ket{+-}  = 
2 
\left(\frac{\hbar}{2}\right)
\left(-\frac{\hbar}{2}\right)
\ket{+-}
\end{equation}

So

\begin{equation}\label{eqn:qmTwoL16:390}
S^2\ket{+-} = 
\hbar^2 \ket{+-} 
+ 2 S_{1x} S_{2x} \ket{+-} 
+ 2 S_{1y} S_{2y} \ket{+-} 
\end{equation}

Since we have set our spin direction in the z direction with

\begin{align}\label{eqn:qmTwoL16:410}
\ket{+} &\rightarrow 
\begin{bmatrix}
1 \\
0
\end{bmatrix} \\
\ket{-} &\rightarrow 
\begin{bmatrix}
0 \\
1 
\end{bmatrix}
\end{align}

We have
\begin{align*}
S_x\ket{+} 
&\rightarrow 
\frac{\hbar}{2} \PauliX
\begin{bmatrix}
1 \\
0
\end{bmatrix} 
=
\frac{\hbar}{2}
\begin{bmatrix}
0 \\
1 
\end{bmatrix}
=
\frac{\hbar}{2} \ket{-} \\
S_x\ket{-} &\rightarrow 
\frac{\hbar}{2} \PauliX
\begin{bmatrix}
0 \\
1 
\end{bmatrix} 
=
\frac{\hbar}{2}
\begin{bmatrix}
1  \\
0 
\end{bmatrix}
=
\frac{\hbar}{2} \ket{+} \\
S_y\ket{+} &\rightarrow 
\frac{\hbar}{2} \PauliY
\begin{bmatrix}
1  \\
0 
\end{bmatrix} 
=
\frac{i\hbar}{2}
\begin{bmatrix}
0  \\
1 
\end{bmatrix}
=
\frac{i\hbar}{2} \ket{-} \\
S_y\ket{-} &\rightarrow 
\frac{\hbar}{2} \PauliY
\begin{bmatrix}
0  \\
1 
\end{bmatrix} 
=
\frac{-i\hbar}{2}
\begin{bmatrix}
1  \\
0 
\end{bmatrix}
=
-\frac{i\hbar}{2} \ket{+} \\
\end{align*}

And are able to arrive at the action of $S^2$ on our mixed composite state

\begin{equation}\label{eqn:qmTwoL16:430}
S^2\ket{+-} = \hbar^2 (\ket{+-} + \ket{-+} ).
\end{equation}

For the action on the $\ket{++}$ state we have

\begin{align*}
S^2 \ket{++} 
&=
\left(\frac{3}{4}\hbar^2 +\frac{3}{4}\hbar^2\right)
\ket{++} 
+ 2 \frac{\hbar^2}{4} 
\ket{--} 
+ 2 i^2 \frac{\hbar^2}{4} \ket{--} 
+2 
\left(\frac{\hbar}{2}\right)
\left(\frac{\hbar}{2}\right)
\ket{++} \\
&=
2 \hbar^2 \ket{++} \\
\end{align*}

and on the $\ket{--}$ state we have

\begin{align*}
S^2 \ket{--} 
&=
\left(\frac{3}{4}\hbar^2 +\frac{3}{4}\hbar^2\right)
\ket{--} 
+ 2 \frac{(-\hbar)^2}{4} 
\ket{++} 
+ 2 i^2 \frac{\hbar^2}{4} \ket{++} 
+2 
\left(-\frac{\hbar}{2}\right)
\left(-\frac{\hbar}{2}\right)
\ket{--} \\
&=
2 \hbar^2 \ket{--} 
\end{align*}

All of this can be assembled into a tidier matrix form

\begin{equation}\label{eqn:qmTwoL16:450}
S^2
\rightarrow 
\hbar^2
\begin{bmatrix}
2 & 0 & 0 & 0 \\
0 & 1 & 1 & 0 \\
0 & 1 & 1 & 0 \\
0 & 0 & 0 & 2 \\
\end{bmatrix},
\end{equation}

where the matrix is taken with respect to the (ordered) basis

\begin{equation}\label{eqn:qmTwoL16:470}
\{
\ket{++},
\ket{+-},
\ket{-+},
\ket{--}
\}.
\end{equation}

However, 

\begin{align}\label{eqn:qmTwoL16:490}
\antisymmetric{S^2}{S_z} &= 0 \\
\antisymmetric{S_i}{S_j} &= i \hbar \sum_k \epsilon_{ijk} S_k
\end{align}

(Also, $\antisymmetric{S^2}{S_i} = 0$.)

It should be possible to find eigenkets of $S^2$ and $S_z$

\begin{align}\label{eqn:qmTwoL16:510}
S^2 \ket{s m_s} &= s(s+1)\hbar^2 \ket{s m_s} \\
S_z \ket{s m_s} &= \hbar m_s \ket{s m_s} 
\end{align}

An orthonormal set of eigenkets of $S^2$ and $S_z$ is found to be

\begin{equation}\label{eqn:qmTwoL16:530}
\begin{array}{l l}
\ket{++} 
& \mbox{$s = 1$ and $m_s = 1$} \\
\inv{\sqrt{2}} \left( \ket{+-} + \ket{-+} \right) 
& \mbox{$s = 1$ and $m_s = 0$} \\
\ket{--} 
& \mbox{$s = 1$ and $m_s = -1$} \\
\inv{\sqrt{2}} \left( \ket{+-} - \ket{-+} \right) 
& \mbox{$s = 0$ and $m_s = 0$}
\end{array}
\end{equation}

The first three kets here can be grouped into a triplet in a 3D Hilbert space, whereas the last treated as a singlet in a 1D Hilbert space.

Form a grouping

\begin{equation}\label{eqn:qmTwoL16:550}
H = H_1 \otimes H_2
\end{equation}

Can write

\begin{equation}\label{eqn:qmTwoL16:570}
\inv{2} \otimes \inv{2} = 1 \oplus 0
\end{equation}

where the $1$ and $0$ here refer to the spin index $s$.

\subsection{Other examples}

Consider, perhaps, the $l=5$ state of the hydrogen atom

\begin{align}\label{eqn:qmTwoL16:590}
J_1^2 \ket{j_1 m_1} &= j_1(j_1+1)\hbar^2 \ket{j_1 m_1} \\
J_{1z} \ket{j_1 m_1} &= \hbar m_1 \ket{j_1 m_1} 
\end{align}

\begin{align}\label{eqn:qmTwoL16:610}
J_2^2 \ket{j_2 m_2} &= j_2(j_2+1)\hbar^2 \ket{j_2 m_2} \\
J_{2z} \ket{j_2 m_2} &= \hbar m_2 \ket{j_2 m_2} 
\end{align}

Consider the Hilbert space spanned by $\ket{j_1 m_1} \otimes \ket{j_2 m_2}$, a $(2 j_1 + 1)(2 j_2 + 1)$ dimensional space.  How to find the eigenkets of $J^2$ and $J_z$?


