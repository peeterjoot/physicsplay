%
% Copyright � 2012 Peeter Joot.  All Rights Reserved.
% Licenced as described in the file LICENSE under the root directory of this GIT repository.
%

% 
% 
%%
% Copyright � 2015 Peeter Joot.  All Rights Reserved.
% Licenced as described in the file LICENSE under the root directory of this GIT repository.
%
\documentclass[]{eliblog}

\usepackage{amsmath}
\usepackage{mathpazo}

%
% shorthand for bold symbols, convenient for vectors and matrices
%
\newcommand{\Ba}[0]{\mathbf{a}}
\newcommand{\Bb}[0]{\mathbf{b}}
\newcommand{\Bc}[0]{\mathbf{c}}
\newcommand{\Bd}[0]{\mathbf{d}}
\newcommand{\Be}[0]{\mathbf{e}}
\newcommand{\Bf}[0]{\mathbf{f}}
\newcommand{\Bg}[0]{\mathbf{g}}
\newcommand{\Bh}[0]{\mathbf{h}}
\newcommand{\Bi}[0]{\mathbf{i}}
\newcommand{\Bj}[0]{\mathbf{j}}
\newcommand{\Bk}[0]{\mathbf{k}}
\newcommand{\Bl}[0]{\mathbf{l}}
\newcommand{\Bm}[0]{\mathbf{m}}
\newcommand{\Bn}[0]{\mathbf{n}}
\newcommand{\Bo}[0]{\mathbf{o}}
\newcommand{\Bp}[0]{\mathbf{p}}
\newcommand{\Bq}[0]{\mathbf{q}}
\newcommand{\Br}[0]{\mathbf{r}}
\newcommand{\Bs}[0]{\mathbf{s}}
\newcommand{\Bt}[0]{\mathbf{t}}
\newcommand{\Bu}[0]{\mathbf{u}}
\newcommand{\Bv}[0]{\mathbf{v}}
\newcommand{\Bw}[0]{\mathbf{w}}
\newcommand{\Bx}[0]{\mathbf{x}}
\newcommand{\By}[0]{\mathbf{y}}
\newcommand{\Bz}[0]{\mathbf{z}}
\newcommand{\BA}[0]{\mathbf{A}}
\newcommand{\BB}[0]{\mathbf{B}}
\newcommand{\BC}[0]{\mathbf{C}}
\newcommand{\BD}[0]{\mathbf{D}}
\newcommand{\BE}[0]{\mathbf{E}}
\newcommand{\BF}[0]{\mathbf{F}}
\newcommand{\BG}[0]{\mathbf{G}}
\newcommand{\BH}[0]{\mathbf{H}}
\newcommand{\BI}[0]{\mathbf{I}}
\newcommand{\BJ}[0]{\mathbf{J}}
\newcommand{\BK}[0]{\mathbf{K}}
\newcommand{\BL}[0]{\mathbf{L}}
\newcommand{\BM}[0]{\mathbf{M}}
\newcommand{\BN}[0]{\mathbf{N}}
\newcommand{\BO}[0]{\mathbf{O}}
\newcommand{\BP}[0]{\mathbf{P}}
\newcommand{\BQ}[0]{\mathbf{Q}}
\newcommand{\BR}[0]{\mathbf{R}}
\newcommand{\BS}[0]{\mathbf{S}}
\newcommand{\BT}[0]{\mathbf{T}}
\newcommand{\BU}[0]{\mathbf{U}}
\newcommand{\BV}[0]{\mathbf{V}}
\newcommand{\BW}[0]{\mathbf{W}}
\newcommand{\BX}[0]{\mathbf{X}}
\newcommand{\BY}[0]{\mathbf{Y}}
\newcommand{\BZ}[0]{\mathbf{Z}}

\newcommand{\Bzero}[0]{\mathbf{0}}
\newcommand{\Btheta}[0]{\boldsymbol{\theta}}
\newcommand{\Btau}[0]{\boldsymbol{\tau}}
\newcommand{\Bomega}[0]{\boldsymbol{\omega}}

%
% shorthand for unit vectors
%
\newcommand{\acap}[0]{\hat{\Ba}}
\newcommand{\bcap}[0]{\hat{\Bb}}
\newcommand{\ccap}[0]{\hat{\Bc}}
\newcommand{\dcap}[0]{\hat{\Bd}}
\newcommand{\ecap}[0]{\hat{\Be}}
\newcommand{\fcap}[0]{\hat{\Bf}}
\newcommand{\gcap}[0]{\hat{\Bg}}
\newcommand{\hcap}[0]{\hat{\Bh}}
\newcommand{\icap}[0]{\hat{\Bi}}
\newcommand{\jcap}[0]{\hat{\Bj}}
\newcommand{\kcap}[0]{\hat{\Bk}}
\newcommand{\lcap}[0]{\hat{\Bl}}
\newcommand{\mcap}[0]{\hat{\Bm}}
\newcommand{\ncap}[0]{\hat{\Bn}}
\newcommand{\ocap}[0]{\hat{\Bo}}
\newcommand{\pcap}[0]{\hat{\Bp}}
\newcommand{\qcap}[0]{\hat{\Bq}}
\newcommand{\rcap}[0]{\hat{\Br}}
\newcommand{\scap}[0]{\hat{\Bs}}
\newcommand{\tcap}[0]{\hat{\Bt}}
\newcommand{\ucap}[0]{\hat{\Bu}}
\newcommand{\vcap}[0]{\hat{\Bv}}
\newcommand{\wcap}[0]{\hat{\Bw}}
\newcommand{\xcap}[0]{\hat{\Bx}}
\newcommand{\ycap}[0]{\hat{\By}}
\newcommand{\zcap}[0]{\hat{\Bz}}
\newcommand{\thetacap}[0]{\hat{\Btheta}}

%
% to write R^n and C^n in a distinguishable fashion.  Perhaps change this
% to the double lined characters upon figuring out how to do so.
%
\newcommand{\C}[1]{$\mathbb{C}^{#1}$}
\newcommand{\R}[1]{$\mathbb{R}^{#1}$}

%
% various generally useful helpers
%

% derivative of #1 wrt. #2:
\newcommand{\D}[2] {\frac {d#2} {d#1}}

\newcommand{\inv}[1]{\frac{1}{#1}}
\newcommand{\cross}[0]{\times}

\newcommand{\abs}[1]{\lvert{#1}\rvert}
\newcommand{\norm}[1]{\lVert{#1}\rVert}
\newcommand{\innerprod}[2]{\langle{#1}, {#2}\rangle}
\newcommand{\dotprod}[2]{{#1} \cdot {#2}}
\newcommand{\bdotprod}[2]{\left({#1} \cdot {#2}\right)}
\newcommand{\crossprod}[2]{{#1} \cross {#2}}
\newcommand{\tripleprod}[3]{\dotprod{\left(\crossprod{#1}{#2}\right)}{#3}}

\DeclareMathOperator{\Proj}{Proj}
\DeclareMathOperator{\Span}{span}
\DeclareMathOperator{\Sgn}{sgn}
\DeclareMathOperator{\Area}{Area}
\DeclareMathOperator{\Volume}{Volume}

%
% A few miscellaneous things specific to this document
%
\newcommand{\crossop}[1]{\crossprod{#1}{}}

% R2 vector.
\newcommand{\VectorTwo}[2]{
\begin{bmatrix}
 {#1} \\
 {#2}
\end{bmatrix}
}

\newcommand{\VectorN}[1]{
\begin{bmatrix}
{#1}_1 \\
{#1}_2 \\
\vdots \\
{#1}_N \\
\end{bmatrix}
}

\newcommand{\DETuvij}[4]{
\begin{vmatrix}
 {#1}_{#3} & {#1}_{#4} \\
 {#2}_{#3} & {#2}_{#4}
\end{vmatrix}
}

\newcommand{\DETuvwijk}[6]{
\begin{vmatrix}
 {#1}_{#4} & {#1}_{#5} & {#1}_{#6} \\
 {#2}_{#4} & {#2}_{#5} & {#2}_{#6} \\
 {#3}_{#4} & {#3}_{#5} & {#3}_{#6}
\end{vmatrix}
}

\newcommand{\DETuvwxijkl}[8]{
\begin{vmatrix}
 {#1}_{#5} & {#1}_{#6} & {#1}_{#7} & {#1}_{#8} \\
 {#2}_{#5} & {#2}_{#6} & {#2}_{#7} & {#2}_{#8} \\
 {#3}_{#5} & {#3}_{#6} & {#3}_{#7} & {#3}_{#8} \\
 {#4}_{#5} & {#4}_{#6} & {#4}_{#7} & {#4}_{#8} \\
\end{vmatrix}
}

%\newcommand{\DETuvwxyijklm}[10]{
%\begin{vmatrix}
% {#1}_{#6} & {#1}_{#7} & {#1}_{#8} & {#1}_{#9} & {#1}_{#10} \\
% {#2}_{#6} & {#2}_{#7} & {#2}_{#8} & {#2}_{#9} & {#2}_{#10} \\
% {#3}_{#6} & {#3}_{#7} & {#3}_{#8} & {#3}_{#9} & {#3}_{#10} \\
% {#4}_{#6} & {#4}_{#7} & {#4}_{#8} & {#4}_{#9} & {#4}_{#10} \\
% {#5}_{#6} & {#5}_{#7} & {#5}_{#8} & {#5}_{#9} & {#5}_{#10}
%\end{vmatrix}
%}

% R3 vector.
\newcommand{\VectorThree}[3]{
\begin{bmatrix}
 {#1} \\
 {#2} \\
 {#3}
\end{bmatrix}
}



\author{Peeter Joot}
\email{peeter.joot@gmail.com}

%\documentclass[]{eliblogwidescreen}

\usepackage{amsmath}
\usepackage{mathpazo}

%
% shorthand for bold symbols, convenient for vectors and matrices
%
\newcommand{\Ba}[0]{\mathbf{a}}
\newcommand{\Bb}[0]{\mathbf{b}}
\newcommand{\Bc}[0]{\mathbf{c}}
\newcommand{\Bd}[0]{\mathbf{d}}
\newcommand{\Be}[0]{\mathbf{e}}
\newcommand{\Bf}[0]{\mathbf{f}}
\newcommand{\Bg}[0]{\mathbf{g}}
\newcommand{\Bh}[0]{\mathbf{h}}
\newcommand{\Bi}[0]{\mathbf{i}}
\newcommand{\Bj}[0]{\mathbf{j}}
\newcommand{\Bk}[0]{\mathbf{k}}
\newcommand{\Bl}[0]{\mathbf{l}}
\newcommand{\Bm}[0]{\mathbf{m}}
\newcommand{\Bn}[0]{\mathbf{n}}
\newcommand{\Bo}[0]{\mathbf{o}}
\newcommand{\Bp}[0]{\mathbf{p}}
\newcommand{\Bq}[0]{\mathbf{q}}
\newcommand{\Br}[0]{\mathbf{r}}
\newcommand{\Bs}[0]{\mathbf{s}}
\newcommand{\Bt}[0]{\mathbf{t}}
\newcommand{\Bu}[0]{\mathbf{u}}
\newcommand{\Bv}[0]{\mathbf{v}}
\newcommand{\Bw}[0]{\mathbf{w}}
\newcommand{\Bx}[0]{\mathbf{x}}
\newcommand{\By}[0]{\mathbf{y}}
\newcommand{\Bz}[0]{\mathbf{z}}
\newcommand{\BA}[0]{\mathbf{A}}
\newcommand{\BB}[0]{\mathbf{B}}
\newcommand{\BC}[0]{\mathbf{C}}
\newcommand{\BD}[0]{\mathbf{D}}
\newcommand{\BE}[0]{\mathbf{E}}
\newcommand{\BF}[0]{\mathbf{F}}
\newcommand{\BG}[0]{\mathbf{G}}
\newcommand{\BH}[0]{\mathbf{H}}
\newcommand{\BI}[0]{\mathbf{I}}
\newcommand{\BJ}[0]{\mathbf{J}}
\newcommand{\BK}[0]{\mathbf{K}}
\newcommand{\BL}[0]{\mathbf{L}}
\newcommand{\BM}[0]{\mathbf{M}}
\newcommand{\BN}[0]{\mathbf{N}}
\newcommand{\BO}[0]{\mathbf{O}}
\newcommand{\BP}[0]{\mathbf{P}}
\newcommand{\BQ}[0]{\mathbf{Q}}
\newcommand{\BR}[0]{\mathbf{R}}
\newcommand{\BS}[0]{\mathbf{S}}
\newcommand{\BT}[0]{\mathbf{T}}
\newcommand{\BU}[0]{\mathbf{U}}
\newcommand{\BV}[0]{\mathbf{V}}
\newcommand{\BW}[0]{\mathbf{W}}
\newcommand{\BX}[0]{\mathbf{X}}
\newcommand{\BY}[0]{\mathbf{Y}}
\newcommand{\BZ}[0]{\mathbf{Z}}

\newcommand{\Bzero}[0]{\mathbf{0}}
\newcommand{\Btheta}[0]{\boldsymbol{\theta}}
\newcommand{\Btau}[0]{\boldsymbol{\tau}}
\newcommand{\Bomega}[0]{\boldsymbol{\omega}}

%
% shorthand for unit vectors
%
\newcommand{\acap}[0]{\hat{\Ba}}
\newcommand{\bcap}[0]{\hat{\Bb}}
\newcommand{\ccap}[0]{\hat{\Bc}}
\newcommand{\dcap}[0]{\hat{\Bd}}
\newcommand{\ecap}[0]{\hat{\Be}}
\newcommand{\fcap}[0]{\hat{\Bf}}
\newcommand{\gcap}[0]{\hat{\Bg}}
\newcommand{\hcap}[0]{\hat{\Bh}}
\newcommand{\icap}[0]{\hat{\Bi}}
\newcommand{\jcap}[0]{\hat{\Bj}}
\newcommand{\kcap}[0]{\hat{\Bk}}
\newcommand{\lcap}[0]{\hat{\Bl}}
\newcommand{\mcap}[0]{\hat{\Bm}}
\newcommand{\ncap}[0]{\hat{\Bn}}
\newcommand{\ocap}[0]{\hat{\Bo}}
\newcommand{\pcap}[0]{\hat{\Bp}}
\newcommand{\qcap}[0]{\hat{\Bq}}
\newcommand{\rcap}[0]{\hat{\Br}}
\newcommand{\scap}[0]{\hat{\Bs}}
\newcommand{\tcap}[0]{\hat{\Bt}}
\newcommand{\ucap}[0]{\hat{\Bu}}
\newcommand{\vcap}[0]{\hat{\Bv}}
\newcommand{\wcap}[0]{\hat{\Bw}}
\newcommand{\xcap}[0]{\hat{\Bx}}
\newcommand{\ycap}[0]{\hat{\By}}
\newcommand{\zcap}[0]{\hat{\Bz}}
\newcommand{\thetacap}[0]{\hat{\Btheta}}

%
% to write R^n and C^n in a distinguishable fashion.  Perhaps change this
% to the double lined characters upon figuring out how to do so.
%
\newcommand{\C}[1]{$\mathbb{C}^{#1}$}
\newcommand{\R}[1]{$\mathbb{R}^{#1}$}

%
% various generally useful helpers
%

% derivative of #1 wrt. #2:
\newcommand{\D}[2] {\frac {d#2} {d#1}}

\newcommand{\inv}[1]{\frac{1}{#1}}
\newcommand{\cross}[0]{\times}

\newcommand{\abs}[1]{\lvert{#1}\rvert}
\newcommand{\norm}[1]{\lVert{#1}\rVert}
\newcommand{\innerprod}[2]{\langle{#1}, {#2}\rangle}
\newcommand{\dotprod}[2]{{#1} \cdot {#2}}
\newcommand{\bdotprod}[2]{\left({#1} \cdot {#2}\right)}
\newcommand{\crossprod}[2]{{#1} \cross {#2}}
\newcommand{\tripleprod}[3]{\dotprod{\left(\crossprod{#1}{#2}\right)}{#3}}

\DeclareMathOperator{\Proj}{Proj}
\DeclareMathOperator{\Span}{span}
\DeclareMathOperator{\Sgn}{sgn}
\DeclareMathOperator{\Area}{Area}
\DeclareMathOperator{\Volume}{Volume}

%
% A few miscellaneous things specific to this document
%
\newcommand{\crossop}[1]{\crossprod{#1}{}}

% R2 vector.
\newcommand{\VectorTwo}[2]{
\begin{bmatrix}
 {#1} \\
 {#2}
\end{bmatrix}
}

\newcommand{\VectorN}[1]{
\begin{bmatrix}
{#1}_1 \\
{#1}_2 \\
\vdots \\
{#1}_N \\
\end{bmatrix}
}

\newcommand{\DETuvij}[4]{
\begin{vmatrix}
 {#1}_{#3} & {#1}_{#4} \\
 {#2}_{#3} & {#2}_{#4}
\end{vmatrix}
}

\newcommand{\DETuvwijk}[6]{
\begin{vmatrix}
 {#1}_{#4} & {#1}_{#5} & {#1}_{#6} \\
 {#2}_{#4} & {#2}_{#5} & {#2}_{#6} \\
 {#3}_{#4} & {#3}_{#5} & {#3}_{#6}
\end{vmatrix}
}

\newcommand{\DETuvwxijkl}[8]{
\begin{vmatrix}
 {#1}_{#5} & {#1}_{#6} & {#1}_{#7} & {#1}_{#8} \\
 {#2}_{#5} & {#2}_{#6} & {#2}_{#7} & {#2}_{#8} \\
 {#3}_{#5} & {#3}_{#6} & {#3}_{#7} & {#3}_{#8} \\
 {#4}_{#5} & {#4}_{#6} & {#4}_{#7} & {#4}_{#8} \\
\end{vmatrix}
}

%\newcommand{\DETuvwxyijklm}[10]{
%\begin{vmatrix}
% {#1}_{#6} & {#1}_{#7} & {#1}_{#8} & {#1}_{#9} & {#1}_{#10} \\
% {#2}_{#6} & {#2}_{#7} & {#2}_{#8} & {#2}_{#9} & {#2}_{#10} \\
% {#3}_{#6} & {#3}_{#7} & {#3}_{#8} & {#3}_{#9} & {#3}_{#10} \\
% {#4}_{#6} & {#4}_{#7} & {#4}_{#8} & {#4}_{#9} & {#4}_{#10} \\
% {#5}_{#6} & {#5}_{#7} & {#5}_{#8} & {#5}_{#9} & {#5}_{#10}
%\end{vmatrix}
%}

% R3 vector.
\newcommand{\VectorThree}[3]{
\begin{bmatrix}
 {#1} \\
 {#2} \\
 {#3}
\end{bmatrix}
}



\author{Peeter Joot}
\email{peeter.joot@gmail.com}


%\chapter{PHY456H1F.  My solutions to problem set 1 (ungraded)}
\chapter{Problem set 1}
\label{chap:phy456ps1}

\blogpage{http://sites.google.com/site/peeterjoot/math2011/phy456ps1.pdf}
%\date{Sept 12, 2011}





\section{Harmonic oscillator}

Consider
\begin{equation}\label{eqn:phy456ps1:10}
H_0 = \frac{P^2}{2m} + \inv{2} m \omega^2 X^2
\end{equation}

Since it's been a while let's compute the raising and lowering factorization that was used so extensively for this problem.

It was of the form

\begin{equation}\label{eqn:phy456ps1:30}
H_0 = (a X - i b P)(a X + i b P) + \cdots
\end{equation}

Why this factorization has an imaginary in it is a good question.  It's not one that is given any sort of rationale in the text (\cite{desai2009quantum}).

It's clear that we want $a = \sqrt{m/2} \omega$ and $b = 1/\sqrt{2m}$.  The difference is then

\begin{equation}\label{eqn:phy456ps1:50}
H_0 - (a X - i b P)(a X + i b P)
=
- i a b \antisymmetric{X}{P}  = - i \frac{\omega}{2} \antisymmetric{X}{P}
\end{equation}

That commutator is an $i\hbar$ value, but what was the sign?  Let's compute so we don't get it wrong

\begin{align*}
\antisymmetric{x}{ p} \psi
&= -i \hbar \antisymmetric{x}{\partial_x} \psi \\
&= -i \hbar ( x \partial_x \psi - \partial_x (x \psi) ) \\
&= -i \hbar ( - \psi ) \\
&= i \hbar \psi
\end{align*}

So we have

\begin{equation}\label{eqn:phy456ps1:70}
H_0 =
\left(\omega \sqrt{\frac{m}{2}} X - i \sqrt{\frac{1}{2m}} P\right)\left(\omega \sqrt{\frac{m}{2}} X + i \sqrt{\frac{1}{2m}} P\right)
+ \frac{\hbar \omega}{2}
\end{equation}

Factoring out an $\hbar \omega$ produces the form of the Hamiltonian that we used before

\begin{equation}\label{eqn:phy456ps1:90}
H_0 =
\hbar \omega \left(
\left(\sqrt{\frac{m \omega}{2 \hbar}} X - i \sqrt{\frac{1}{2m \hbar \omega}} P\right)\left(\sqrt{\frac{m \omega}{2 \hbar}} X + i \sqrt{\frac{1}{2m \hbar \omega}} P\right)
+ \inv{2}
\right)
.
\end{equation}

The factors were labeled the uppering ($a^\dagger$) and lowering ($a$) operators respectively, and written

\begin{align}\label{eqn:phy456ps1:110}
H_0 &= \hbar \omega \left( a^\dagger a + \inv{2} \right) \\
a &= \sqrt{\frac{m \omega}{2 \hbar}} X + i \sqrt{\frac{1}{2m \hbar \omega}} P \\
a^\dagger &= \sqrt{\frac{m \omega}{2 \hbar}} X - i \sqrt{\frac{1}{2m \hbar \omega}} P.
\end{align}

Observe that we can find the inverse relations

\begin{align}\label{eqn:phy456ps1:115}
X &= \sqrt{ \frac{\hbar}{2 m \omega} } \left( a + a^\dagger \right) \\
P &= i \sqrt{ \frac{m \hbar \omega}{2} } \left( a^\dagger  - a \right)
\end{align}

\paragraph{Question}
What is a good reason that we chose this particular factorization?  For example, a quick computation shows that we could have also picked

\begin{equation}\label{eqn:phy456ps1:130}
H_0 = \hbar \omega \left( a a^\dagger - \inv{2} \right).
\end{equation}

I don't know that answer.  That said, this second factorization is useful in that it provides the commutator relation between the raising and lowering operators, since subtracting \ref{eqn:phy456ps1:130} and \ref{eqn:phy456ps1:110} yields

\begin{equation}\label{eqn:phy456ps1:150}
\antisymmetric{a}{a^\dagger} = 1.
\end{equation}

If we suppose that we have eigenstates for the operator $a^\dagger a$ of the form

\begin{equation}\label{eqn:phy456ps1:200}
a^\dagger a \ket{n} = \lambda_n \ket{n},
\end{equation}

then the problem of finding the eigensolution of $H_0$ reduces to solving this problem.  Because $a^\dagger a$ commutes with $1/2$, an eigenstate of $a^\dagger a$ is also an eigenstate of $H_0$.  Utilizing \ref{eqn:phy456ps1:150} we then have

\begin{align*}
a^\dagger a ( a \ket{n} )
&= (a a^\dagger - 1 ) a \ket{n} \\
&= a (a^\dagger a - 1 ) \ket{n} \\
&= a (\lambda_n - 1 ) \ket{n} \\
&= (\lambda_n - 1 ) a \ket{n},
\end{align*}

so we see that $a \ket{n}$ is an eigenstate of $a^\dagger a$ with eigenvalue $\lambda_n - 1$.

Similarly for the raising operator

\begin{align*}
a^\dagger a ( a^\dagger \ket{n} )
&=
a^\dagger (a  a^\dagger) \ket{n} ) \\
&=
a^\dagger (a^\dagger a + 1) \ket{n} ) \\
&=
a^\dagger (\lambda_n + 1) \ket{n} ),
\end{align*}

and find that $a^\dagger \ket{n}$ is also an eigenstate of $a^\dagger a$ with eigenvalue $\lambda_n + 1$.

Supposing that there is a lowest energy level (because the potential $V(x) = m \omega x^2 /2$ has a lower bound of zero) then the state $\ket{0}$ for which the energy is the lowest when operated on by $a$ we have

\begin{equation}\label{eqn:phy456ps1:220}
a \ket{0} = 0
\end{equation}

Thus

\begin{equation}\label{eqn:phy456ps1:240}
a^\dagger a \ket{0} = 0,
\end{equation}

and

\begin{equation}\label{eqn:phy456ps1:260}
\lambda_0 = 0.
\end{equation}

This seems like a small bit of slight of hand, since it sneakily supplies an integer value to $\lambda_0$ where up to this point $0$ was just a label.

If the eigenvalue equation we are trying to solve for the Hamiltonian is

\begin{equation}\label{eqn:phy456ps1:280}
H_0 \ket{n} = E_n \ket{n}.
\end{equation}

Then we must then have

\begin{equation}\label{eqn:phy456ps1:300}
E_n = \hbar \omega \left(\lambda_n + \inv{2} \right) = \hbar \omega \left(n + \inv{2} \right)
\end{equation}

\subsection{Part (a)}

We've now got enough context to attempt the first part of the question, calculation of

\begin{equation}\label{eqn:phy456ps1:320}
\bra{n} X^4 \ket{n}
\end{equation}

We've calculated things like this before, such as
\begin{align*}
\bra{n} X^2 \ket{n}
&=
\frac{\hbar}{2 m \omega} \bra{n} (a + a^\dagger)^2 \ket{n}
\end{align*}

To continue we need an exact relation between $\ket{n}$ and $\ket{n \pm 1}$.  Recall that $a \ket{n}$ was an eigenstate of $a^\dagger a$ with eigenvalue $n - 1$.  This implies that the eigenstates $a \ket{n}$ and $\ket{n-1}$ are proportional

\begin{equation}\label{eqn:phy456ps1:340}
a \ket{n} = c_n \ket{n - 1},
\end{equation}

or
\begin{align*}
\bra{n} a^\dagger a \ket{n} &= \Abs{c_n}^2 \braket{n - 1}{n-1} = \Abs{c_n}^2 \\
n \braket{n}{n} &= \\
n &=
\end{align*}

so that

\begin{equation}\label{eqn:phy456ps1:380}
a \ket{n} = \sqrt{n} \ket{n - 1}.
\end{equation}

Similarly let

\begin{equation}\label{eqn:phy456ps1:400}
a^\dagger \ket{n} = b_n \ket{n + 1},
\end{equation}

or
\begin{align*}
\bra{n} a a^\dagger \ket{n} &= \Abs{b_n}^2 \braket{n - 1}{n-1} = \Abs{b_n}^2 \\
\bra{n} (1 + a^\dagger a) \ket{n} &= \\
1 + n &=
\end{align*}

so that

\begin{equation}\label{eqn:phy456ps1:440}
a^\dagger \ket{n} = \sqrt{n+1} \ket{n + 1}.
\end{equation}

We can now return to \ref{eqn:phy456ps1:320}, and find

\begin{align*}
\bra{n} X^4 \ket{n}
&=
\frac{\hbar^2}{4 m^2 \omega^2} \bra{n} (a + a^\dagger)^4 \ket{n}
\end{align*}

Consider half of this braket

\begin{align*}
(a + a^\dagger)^2 \ket{n}
&=
\left( a^2 + (a^\dagger)^2 + a^\dagger a + a a^\dagger \right) \ket{n} \\
&=
\left( a^2 + (a^\dagger)^2 + a^\dagger a + (1 + a^\dagger a) \right) \ket{n} \\
&=
\left( a^2 + (a^\dagger)^2 + 1 + 2 a^\dagger a \right) \ket{n} \\
&=
\sqrt{n-1}\sqrt{n-2} \ket{n-2}
+
\sqrt{n+1}\sqrt{n+2} \ket{n + 2}
+
\ket{n}
+  2 n \ket{n}
\end{align*}

Squaring, utilizing the Hermitian nature of the $X$ operator %, we have for $n > 2$

\begin{equation}\label{eqn:phy456ps1:500}
\bra{n} X^4 \ket{n}
=
\frac{\hbar^2}{4 m^2 \omega^2}
\left(
(n-1)(n-2) + (n+1)(n+2) + (1 + 2n)^2
\right)
=
\frac{\hbar^2}{4 m^2 \omega^2}
\left( 6 n^2 + 4 n + 5 \right)
\end{equation}

%It also looks like we can drop the $n > 2$ restriction since the $\sqrt{n-1}$ and $\sqrt{n-2}$ factors kill off the
\subsection{Part (b)}

Find the ground state energy of the Hamiltonian $H = H_0 + \gamma X^2$ for $\gamma > 0$.

The new Hamiltonian has the form

\begin{equation}\label{eqn:phy456ps1:520}
H = \frac{P^2}{2m} + \inv{2} m \left(\omega^2 + \frac{2 \gamma}{m} \right) X^2 =
\frac{P^2}{2m} + \inv{2} m {\omega'}^2 X^2,
\end{equation}

where
\begin{equation}\label{eqn:phy456ps1:540}
\omega' = \sqrt{ \omega^2 + \frac{2 \gamma}{m} }
\end{equation}

The energy states of the Hamiltonian are thus

\begin{equation}\label{eqn:phy456ps1:560}
E_n = \hbar \sqrt{ \omega^2 + \frac{2 \gamma}{m} } \left( n + \inv{2} \right)
\end{equation}

and the ground state of the modified Hamiltonian $H$ is thus

\begin{equation}\label{eqn:phy456ps1:580}
E_0 = \frac{\hbar}{2} \sqrt{ \omega^2 + \frac{2 \gamma}{m} }
\end{equation}

\subsection{Part (c)}

Find the ground state energy of the Hamiltonian $H = H_0 - \alpha X$.

With a bit of play, this new Hamiltonian can be factored into

\begin{equation}\label{eqn:phy456ps1:590}
H
= \hbar \omega \left( b^\dagger b + \inv{2} \right) - \frac{\alpha^2}{2 m \omega^2}
= \hbar \omega \left( b b^\dagger - \inv{2} \right) - \frac{\alpha^2}{2 m \omega^2},
\end{equation}

where

\begin{align}\label{eqn:phy456ps1:600}
b &= \sqrt{\frac{m \omega}{2\hbar}} X + \frac{i P}{\sqrt{2 m \hbar \omega}} - \frac{\alpha}{\omega \sqrt{ 2 m \hbar \omega }} \\
b^\dagger &= \sqrt{\frac{m \omega}{2\hbar}} X - \frac{i P}{\sqrt{2 m \hbar \omega}} - \frac{\alpha}{\omega \sqrt{ 2 m \hbar \omega }}.
\end{align}

From \ref{eqn:phy456ps1:590} we see that we have the same sort of commutator relationship as in the original Hamiltonian

\begin{equation}\label{eqn:phy456ps1:610}
\antisymmetric{b}{b^\dagger} = 1,
\end{equation}

and because of this, all the preceding arguments follow unchanged with the exception that the energy eigenstates of this Hamiltonian are shifted by a constant

\begin{equation}\label{eqn:phy456ps1:620}
H \ket{n} = \left( \hbar \omega \left( n + \inv{2} \right) - \frac{\alpha^2}{2 m \omega^2} \right) \ket{n},
\end{equation}

where the $\ket{n}$ states are simultaneous eigenstates of the $b^\dagger b$ operator

\begin{equation}\label{eqn:phy456ps1:630}
b^\dagger b \ket{n} = n \ket{n}.
\end{equation}

The ground state energy is then
\begin{equation}\label{eqn:phy456ps1:640}
E_0 = \frac{\hbar \omega }{2} - \frac{\alpha^2}{2 m \omega^2}.
\end{equation}

This makes sense.  A translation of the entire position of the system should not effect the energy level distribution of the system, but we have set our reference potential differently, and have this constant energy adjustment to the entire system.

\section{Hydrogen atom and spherical harmonics}

We are asked to show that for any eigenkets of the hydrogen atom $\ket{\Phi_{nlm}}$ we have

\begin{equation}\label{eqn:phy456ps1:700}
\bra{\Phi_{nlm}} X \ket{\Phi_{nlm}} 
=
\bra{\Phi_{nlm}} Y \ket{\Phi_{nlm}} 
=
\bra{\Phi_{nlm}} Z \ket{\Phi_{nlm}}.
\end{equation}

The summary sheet provides us with the wavefunction
\begin{equation}\label{eqn:phy456ps1:720}
\braket{\Br}{\Phi_{nlm}} = 
\frac{2}{n^2 a_0^{3/2}} \sqrt{\frac{(n-l-1)!}{(n+l)!)^3}} F_{nl}\left( \frac{2r}{n a_0} \right) Y_l^m(\theta, \phi),
\end{equation}

where $F_{nl}$ is a real valued function defined in terms of Lagueere polynomials.  Working with the expectation of the $X$ operator to start with we have

\begin{align*}
\bra{\Phi_{nlm}} X \ket{\Phi_{nlm}} 
&=
\int 
\braket{\Phi_{nlm}}{\Br'} \bra{\Br'} X \ket{\Br} \braket{\Br}{\Phi_{nlm}} d^3 \Br d^3 \Br' \\
&=
\int 
\braket{\Phi_{nlm}}{\Br'} \delta(\Br - \Br') r \sin\theta \cos\phi \braket{\Br}{\Phi_{nlm}} d^3 \Br d^3 \Br' \\
&=
\int 
\Phi_{nlm}^\conj(\Br) r \sin\theta \cos\phi \Phi_{nlm}(\Br) d^3 \Br \\
&\sim
\int r^2 dr \Abs{ F_{nl}\left(\frac{2 r}{ n a_0} \right)}^2 r 
\int \sin\theta d\theta d\phi
{Y_l^m}^\conj(\theta, \phi) \sin\theta \cos\phi Y_l^m(\theta, \phi) \\
\end{align*}

Recalling that the only $\phi$ dependence in $Y_l^m$ is $e^{i m \phi}$ we can perform the $d\phi$ integration directly, which is

\begin{equation}\label{eqn:phy456ps1:740}
\int_{\phi=0}^{2\pi} \cos\phi d\phi e^{-i m \phi} e^{i m \phi} = 0.
\end{equation}

We have the same story for the $Y$ expectation which is

\begin{equation}\label{eqn:phy456ps1:760}
\bra{\Phi_{nlm}} X \ket{\Phi_{nlm}} 
\sim
\int r^2 dr \Abs{F_{nl}\left( \frac{2 r}{ n a_0} \right)}^2 r 
\int \sin\theta d\theta d\phi
{Y_l^m}^\conj(\theta, \phi) \sin\theta \sin\phi Y_l^m(\theta, \phi).
\end{equation}

Our $\phi$ integral is then just

\begin{equation}\label{eqn:phy456ps1:780}
\int_{\phi=0}^{2\pi} \sin\phi d\phi e^{-i m \phi} e^{i m \phi} = 0,
\end{equation}

also zero.  The $Z$ expectation is a slightly different story.  There we have

\begin{equation}\label{eqn:phy456ps1:800}
\begin{aligned}
\bra{\Phi_{nlm}} Z \ket{\Phi_{nlm}} 
&\sim
\int dr \Abs{F_{nl}\left( \frac{2 r}{ n a_0} \right)}^2 r^3  \\
&\quad \int_0^{2\pi} d\phi
\int_0^\pi \sin \theta d\theta
\left( \sin\theta \right)^{-2m}
\left( \frac{d^{l - m}}{d (\cos\theta)^{l-m}} \sin^{2l}\theta \right)^2
\cos\theta.
\end{aligned}
\end{equation}

Within this last integral we can make the substitution

\begin{align}\label{eqn:phy456ps1:820}
u &= \cos\theta \\
\sin\theta d\theta &= - d(\cos\theta) = -du \\
u &\in [1, -1],
\end{align}

and the integral takes the form
\begin{equation}\label{eqn:phy456ps1:840}
-\int_{-1}^1 
(-du) 
\inv{(1 - u^2)^m} 
\left( \frac{d^{l-m}}{d u^{l -m }} (1 - u^2)^l
\right)^2 u.
\end{equation}

Here we have the product of two even functions, times one odd function ($u$), over a symmetric interval, so the end result is zero, completing the problem.

I wasn't able to see how to exploit the parity result suggested in the problem, but it wasn't so bad to show these directly.

\section{Angular momentum operator}

Working with the appropriate expressions in Cartesian components, confirm that $L_i \ket{\psi} = 0$ for each component of angular momentum $L_i$, if $\braket{\Br}{\psi} = \psi(\Br)$ is in fact only a function of $r = \Abs{\Br}$.

In order to proceed, we will have to consider a matrix element, so that we can operate on $\ket{\psi}$ in position space.  For that matrix element, we can proceed to insert complete states, and reduce the problem to a question of wavefunctions.  That is

\begin{align*}
\bra{\Br} L_i \ket{\psi}
&=
\int d^3 \Br' \bra{\Br} L_i \ket{\Br'} \braket{\Br'}{\psi} \\
&=
\int d^3 \Br' \bra{\Br} \epsilon_{i a b} X_a P_b \ket{\Br'} \braket{\Br'}{\psi} \\
&=
-i \hbar \epsilon_{i a b} \int d^3 \Br' x_a \bra{\Br} \PD{X_b}{\psi(\Br')} \ket{\Br'}  \\
&=
-i \hbar \epsilon_{i a b} \int d^3 \Br' x_a \PD{x_b}{\psi(\Br')} \braket{\Br}{\Br'}  \\
&=
-i \hbar \epsilon_{i a b} \int d^3 \Br' x_a \PD{x_b}{\psi(\Br')} \delta^3(\Br - \Br') \\
&=
-i \hbar \epsilon_{i a b} x_a \PD{x_b}{\psi(\Br)} 
\end{align*}

With $\psi(\Br) = \psi(r)$ we have

\begin{align*}
\bra{\Br} L_i \ket{\psi}
&=
-i \hbar \epsilon_{i a b} x_a \PD{x_b}{\psi(r)}  \\
&=
-i \hbar \epsilon_{i a b} x_a \PD{x_b}{r} \frac{d\psi(r)}{dr}  \\
&=
-i \hbar \epsilon_{i a b} x_a \inv{2} 2 x_b \inv{r} \frac{d\psi(r)}{dr}  \\
\end{align*}

We are left with an sum of a symmetric product $x_a x_b$ with the antisymmetric tensor $\epsilon_{i a b}$ so this is zero for all $i \in [1,3]$.


