%
% Copyright � 2012 Peeter Joot.  All Rights Reserved.
% Licenced as described in the file LICENSE under the root directory of this GIT repository.
%
%
%\newcommand{\authorname}{Peeter Joot}
\newcommand{\email}{peeterjoot@protonmail.com}
\newcommand{\basename}{FIXMEbasenameUndefined}
\newcommand{\dirname}{notes/FIXMEdirnameUndefined/}

%\renewcommand{\basename}{sincIntegral}
%\renewcommand{\dirname}{notes/phy456/}
%\newcommand{\dateintitle}{}
%\newcommand{\keywords}{}
%\date{Dec 10, 2011}
%
%\newcommand{\authorname}{Peeter Joot}
\newcommand{\onlineurl}{http://sites.google.com/site/peeterjoot2/math2013/\basename.pdf}
\newcommand{\sourcepath}{\dirname\basename.tex}
\newcommand{\generatetitle}[1]{\chapter{#1}}

\newcommand{\vcsinfo}{%
\section*{}
\noindent{\color{DarkOliveGreen}{\rule{\linewidth}{0.1mm}}}
\paragraph{Document version}
%\paragraph{\color{Maroon}{Document version}}
{
\small
\begin{itemize}
\item Available online at:\\ 
\href{\onlineurl}{\onlineurl}
\item Git Repository: \input{./.revinfo/gitRepo.tex}
\item Source: \sourcepath
\item last commit: \input{./.revinfo/gitCommitString.tex}
\item commit date: \input{./.revinfo/gitCommitDate.tex}
\end{itemize}
}
}

%\PassOptionsToPackage{dvipsnames,svgnames}{xcolor}
\PassOptionsToPackage{square,numbers}{natbib}
\documentclass{scrreprt}

\usepackage[left=2cm,right=2cm]{geometry}
\usepackage[svgnames]{xcolor}
\usepackage{peeters_layout}

\usepackage{natbib}

\usepackage[
colorlinks=true,
bookmarks=false,
pdfauthor={\authorname, \email},
backref 
]{hyperref}

% http://tex.stackexchange.com/questions/75773/how-to-reference-problems-by-the-text-label-in-an-exercise-envioronment
\usepackage[english]{cleveref}
\crefname{Exercise}{exercise}{exercises}
\Crefname{Exercise}{Exercise}{Exercises}

\RequirePackage{titlesec}
\RequirePackage{ifthen}

% http://stackoverflow.com/questions/4932910/date-in-the-tabular-environment
\makeatletter
\let\insertdate\@date
\makeatother

\titleformat{\chapter}[display]
{\bfseries\Large}
{\color{DarkSlateGrey}\filleft \authorname
\ifthenelse{\isundefined{\studentnumber}}{}{\\ \studentnumber}
\ifthenelse{\isundefined{\email}}{}{\\ \email}
\ifthenelse{\isundefined{\dateintitle}}{}{\\ \insertdate}
%\ifthenelse{\isundefined{\coursename}}{}{\\ \coursename} % put in title instead.
}
{4ex}
{\color{DarkOliveGreen}{\titlerule}\color{Maroon}
\vspace{2ex}%
\filright}
[\vspace{2ex}%
\color{DarkOliveGreen}\titlerule
]

\newcommand{\beginArtWithToc}[0]{\begin{document}\tableofcontents}
\newcommand{\beginArtNoToc}[0]{\begin{document}}
\newcommand{\EndNoBibArticle}[0]{\end{document}}
\newcommand{\EndArticle}[0]{\bibliography{Bibliography}\bibliographystyle{plainnat}\end{document}}

% 
%\newcommand{\citep}[1]{\cite{#1}}

\colorSectionsForArticle


%
%\renewcommand{\onlineurl}{http://sites.google.com/site/peeterjoot2/math2011/sincIntegral.pdf}
%\beginArtNoToc
%
%\generatetitle{Evaluating the squared sinc integral}
\label{chap:sincIntegral}

In the Fermi's golden rule lecture we used the result for the integral of the squared \(\sinc\) function.  Here is a reminder of the contours required to perform this integral.

We want to evaluate

\begin{equation}\label{eqn:sincIntegral:10}
\int_{-\infty}^\infty \frac{\sin^2 (x\Abs{\mu})}{x^2} dx
\end{equation}

We make a few change of variables

\begin{equation}\label{eqn:sincIntegral:110}
\begin{aligned}
\int_{-\infty}^\infty \frac{\sin^2 (x\Abs{\mu})}{x^2} dx
&=
\Abs{\mu} \int_{-\infty}^\infty \frac{\sin^2 (y)}{y^2} dy \\
&=
-i \Abs{\mu} \int_{-\infty}^\infty \frac{(e^{iy} - e^{-iy})^2}{(2 i y)^2} i dy \\
&=
-\frac{i \Abs{\mu}}{4} \int_{-i\infty}^{i\infty} \frac{e^{2z} + e^{-2z} - 2}{z^2} dz
\end{aligned}
\end{equation}

Now we pick a contour that is distorted to one side of the origin as in \cref{fig:sincIntegral:sincSquaredContour}
\imageFigure{../../figures/phy456/sincSquaredContour}{Contour distorted to one side of the double pole at the origin}{fig:sincIntegral:sincSquaredContour}{0.3}

We employ Jordan's theorem (\S 8.12 \citep{lepage1980cva}) now to pick the contours for each of the integrals since we need to ensure the \(e^{\pm z}\) terms converges as \(R \rightarrow \infty\) for the \(z = R e^{i\theta}\) part of the contour.  We can write

\begin{equation}\label{eqn:sincIntegral:30}
\int_{-\infty}^\infty \frac{\sin^2 (x\Abs{\mu})}{x^2} dx
=
-\frac{i \Abs{\mu}}{4} \left(
\int_{C_0 + C_2} \frac{e^{2z}}{z^2} dz
+\int_{C_0 + C_1} \frac{e^{-2z}}{z^2} dz
-\int_{C_0 + C_1} \frac{2}{z^2} dz
\right)
\end{equation}

The second two integrals both surround no poles, so we have only the first to deal with

\begin{equation}\label{eqn:sincIntegral:130}
\begin{aligned}
\int_{C_0 + C_2} \frac{e^{2z}}{z^2} dz
&= 2 \pi i \inv{1!} \evalbar{ \frac{d}{dz} e^{2z}}{z=0} \\
&= 4 \pi i 
\end{aligned}
\end{equation}

Putting everything back together we have

\begin{equation}\label{eqn:sincIntegral:50}
\int_{-\infty}^\infty \frac{\sin^2 (x\Abs{\mu})}{x^2} dx
= 
-\frac{i \Abs{\mu}}{4} 4 \pi i 
= \pi \Abs{\mu}
\end{equation}

\paragraph{On the cavalier choice of contours}

The choice of which contours to pick above may seem pretty arbitrary, but they are for good reason.  Suppose you picked \(C_0 + C_1\) for the first integral.  On the big \(C_1\) arc, then with a \(z = R e^{i \theta}\) substitution we have

\begin{dmath}\label{eqn:sincIntegral:70}
\Abs{
\int_{C_1} \frac{e^{2 z}}{z^2} dz
}
= 
\Abs{
\int_{\theta = \pi/2}^{-\pi/2} 
\frac{
e^{ 2 R (\cos\theta + i \sin\theta) }
}{
R^2 e^{ 2 i \theta}
}
R i e^{i \theta} d\theta
}
=
\frac{1}{R}
\Abs{
\int_{\theta = \pi/2}^{-\pi/2} 
e^{ 2 R (\cos\theta + i \sin\theta) }
e^{-i \theta} d\theta
}
\le 
\frac{1}{R}
\int_{\theta = -\pi/2}^{\pi/2} 
\Abs{
e^{ 2 R \cos\theta }
}
d\theta
\le 
\frac{\pi e^{2 R}}{R}
\end{dmath}

This clearly doesn't have the zero convergence property that we desire.  We need to pick the \(C_2\) contour for the first (positive exponent) integral since in that \([\pi/2, 3\pi/2]\) range, \(\cos\theta\) is always negative.  We can however, use the \(C_1\) contour for the second (negative exponent) integral.  Explicitly, again by example, using \(C_2\) contour for the first integral, over that portion of the arc we have

\begin{dmath}\label{eqn:sincIntegral:90}
\Abs{
\int_{C_2} \frac{e^{2 z}}{z^2} dz
}
= 
\Abs{
\int_{\theta = \pi/2}^{3 \pi/2} 
\frac{
e^{ 2 R (\cos\theta + i \sin\theta) }
}{
R^2 e^{ 2 i \theta}
}
R i e^{i \theta} d\theta
}
=
\frac{1}{R}
\Abs{
\int_{\theta = \pi/2}^{3 \pi/2} 
e^{ 2 R (\cos\theta + i \sin\theta) }
e^{-i \theta} d\theta
}
\le 
\frac{1}{R}
\int_{\theta = \pi/2}^{3 \pi/2} 
\Abs{
e^{ 2 R \cos\theta }
d\theta
}
\approx
\frac{1}{R}
\int_{\theta = \pi/2}^{3 \pi/2} 
\Abs{
e^{ -2 R }
d\theta
}
=
\frac{\pi e^{-2 R} }{R}
\end{dmath}

%\EndArticle
