%
% Copyright � 2016 Peeter Joot.  All Rights Reserved.
% Licenced as described in the file LICENSE under the root directory of this GIT repository.
%
\makeproblem{Maxwell Lagrangian}{qft:problemSet2:5}{ 

Maxwell's Lagrangian for the electromagnetic field is

\begin{dmath}\label{eqn:qftproblemSet2Problem5:20}
\LL = - \inv{4} F_{\mu\nu} F^{\mu\nu},
\end{dmath}

where \( F_{\mu\nu} = \partial_\mu A_\nu - \partial_\nu A_\mu \), and \( A_\mu \) is the 4-vector potential. (This is just the Lagrangian for a vector field from the first problem set with \( m = 0 \) and some slick new notation). 
\makesubproblem{}{qft:problemSet2:5a}
Show that \( \LL \) is invariant under the \textit{gauge transformation}

\begin{dmath}\label{eqn:qftProblemSet2Problem5:n}
A_\mu \rightarrow A_\mu + \partial_\mu \xi
\end{dmath}

where \( \xi = \xi(x) \) is a scalar field with arbitrary (differentiable) dependence on \( x \).
We will have much more to say about this later on in the course. 
\makesubproblem{}{qft:problemSet2:5b}
Use Noether's theorem and the spacetime translational invariance of the action to construct the energy-momentum tensor \( T^{\mu\nu} \) for the electromagnetic field.
\makesubproblem{}{qft:problemSet2:5c}
Show that the resulting object is neither symmetric nor gauge invariant. 
\makesubproblem{}{qft:problemSet2:5d}
Consider a new tensor given by

\begin{dmath}\label{eqn:qftproblemSet2Problem5:40}
\Theta^{\mu\nu} = T^{\mu\nu} - F^{\rho \mu} \partial_\rho A^\nu.
\end{dmath}

Show that this object also defines four conserved currents. 
\makesubproblem{}{qft:problemSet2:5e}
Moreover, show that it is symmetric, gauge invariant, and traceless. 

\textbf{Comment}: \( T^{\mu\nu} \) and \( \Theta^{\mu\nu} \) are both equally good definitions of the energy-momentum tensor.  
However, \( \Theta^{\mu\nu} \) clearly has nicer properties. Moreover, if you couple Maxwell's
Lagrangian to general relativity, then it is \( \Theta^{\mu\nu} \) which appears in Einstein's equations.
} % makeproblem

\makeanswer{qft:problemSet2:5}{ 
\makeSubAnswer{}{qft:problemSet2:5a}
\makeSubAnswer{}{qft:problemSet2:5b}
\makeSubAnswer{}{qft:problemSet2:5c}
\makeSubAnswer{}{qft:problemSet2:5d}
\makeSubAnswer{}{qft:problemSet2:5e}

TODO.
}
