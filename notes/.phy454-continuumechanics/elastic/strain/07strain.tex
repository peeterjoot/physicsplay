%
% Copyright � 2012 Peeter Joot.  All Rights Reserved.
% Licenced as described in the file LICENSE under the root directory of this GIT repository.
%
\label{chap:continuumL3}

\section{On the factor of two in the tensor definition}

%Strain is the measure of stretching.  This is illustrated pictorially in \cref{fig:continuumL3:continuumL3fig1}
%\imageFigure{../../figures/phy454/lec3_Stretched_line_elementsFig1}{Stretched line elements}{fig:continuumL3:continuumL3fig1}{0.2}
%
%\begin{equation}\label{eqn:continuumL3:10}
%{dl'}^2 - dl^2 = 2 e_{ik} dx_i dx_k,
%\end{equation}
%
%where \(e_{ik}\) is the strain tensor.  We found
%
%\begin{equation}\label{eqn:continuumL3:30}
%e_{ik} = \inv{2} \left(
%\PD{x_k}{e_i}
%+\PD{x_i}{e_k}
%+
%\PD{x_i}{e_l}
%\PD{x_k}{e_l}
%\right)
%\end{equation}
%
Why do we have a factor two in the strain tensor definition?  Observe that if the deformation is small we can write

\begin{equation}\label{eqn:07strain:230}
\begin{aligned}
{dl'}^2 - dl^2
&= (dl' - dl)(dl' + dl) \\
&\approx
 (dl' - dl) 2 dl
\end{aligned}
\end{equation}

so that we find

\begin{equation}\label{eqn:continuumL3:50}
\frac{{dl'}^2 - dl^2 }{dl^2}
\approx
\frac{dl' - dl }{dl}
\end{equation}

Suppose for example, that we have a diagonalized strain tensor, then we find

\begin{equation}\label{eqn:continuumL3:70}
{dl'}^2 - dl^2
= 2 e_{ii} \left(\frac{dx_i}{dl}\right)^2
\end{equation}

so that

\begin{equation}\label{eqn:continuumL3:90}
\frac{
{dl'}^2 - dl^2
}{dl^2}
= 2 e_{ii} dx_i^2
\end{equation}

Observe that here again we see this factor of two.

If we have a diagonalized strain tensor, the tensor is of the form

\begin{equation}\label{eqn:continuumL3:110}
\begin{bmatrix}
e_{11} & 0 & 0 \\
0 & e_{22} & 0 \\
0 & 0 & e_{33}
\end{bmatrix}
\end{equation}

we have

\begin{equation}\label{eqn:continuumL3:130}
{dx_i'}^2 - dx_i^2 = 2 e_{ii} dx_i^2
\end{equation}

\begin{equation}\label{eqn:continuumL3:150}
{dl'}^2 =
(1 + 2 e_{11}) dx_1^2
+(1 + 2 e_{22}) dx_2^2
+(1 + 2 e_{33}) dx_3^2
\end{equation}

\begin{equation}\label{eqn:continuumL3:170}
dl^2 =
dx_1^2
+dx_2^2
+dx_3^2
\end{equation}

so

\begin{equation}\label{eqn:continuumL3:190}
\begin{aligned}
dx_1' &= \sqrt{1 + 2 e_{11}} dx_1 \sim ( 1 + e_{11}) dx_1 \\
dx_2' &= \sqrt{1 + 2 e_{22}} dx_2 \sim ( 1 + e_{22}) dx_2 \\
dx_3' &= \sqrt{1 + 2 e_{33}} dx_3 \sim ( 1 + e_{33}) dx_3
\end{aligned}
\end{equation}

Observe that the change in the volume element becomes the trace

\begin{equation}\label{eqn:continuumL3:210}
dV' =
dx_1'
dx_2'
dx_3'
= dV(1 + e_{ii})
\end{equation}

