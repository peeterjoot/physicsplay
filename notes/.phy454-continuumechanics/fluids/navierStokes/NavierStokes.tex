%
% Copyright � 2012 Peeter Joot.  All Rights Reserved.
% Licenced as described in the file LICENSE under the root directory of this GIT repository.
%

%
%

%\chapter{PHY454H1S\\Continuum Mechanics.  Lecture 10: Navier-Stokes equation.  Taught by Prof. K. Das}
\label{chap:continuumL10}

%\section{Review.  Newtonian fluid}
%
%We stated the model for a Newtonian fluid
%
%\begin{equation}\label{eqn:continuumL10:10}
%\sigma_{ij} = -p \delta_{ij} + 2 \mu e_{ij}
%\end{equation}
%
%and started considering conservation of mass with a volume \(dV\) through an area element \(d\Bs\).  For the rate of change of mass \textunderline{flowing out of the volume \(V\)} is
%
%\begin{equation}\label{eqn:continuumL10:30}
%\oint \rho \Bu \cdot d\Bs = - \PD{t}{} \int_V \rho dV.
%\end{equation}
%
%Application of Green's theorem, for a fixed (in time) volume \(V\) produces
%
%\begin{equation}\label{eqn:continuumL10:50}
%0 = \int_V \left( \spacegrad \cdot (\rho \Bu) + \PD{t}{\rho} \right) dV,
%\end{equation}
%
%or in differential form for an infinitesimal volume
%
%\begin{equation}\label{eqn:continuumL10:70}
%0 = \PD{t}{\rho} + \spacegrad \cdot (\rho \Bu).
%\end{equation}
%
%Expanding out the divergence term using
%
%\begin{align*}
%\spacegrad \cdot (a \Bb)
%&=
%\partial_i (a b_i) \\
%&=
%b_i \partial_i a
%+
%a \partial_i b_i \\
%&=
%\Bb \cdot \spacegrad a
%+ a \spacegrad \cdot \Bb
%\end{align*}
%
%\begin{equation}\label{eqn:continuumL10:90}
%0 = \PD{t}{\rho}
%+ \rho \spacegrad \cdot \Bu
%+ \Bu \cdot \spacegrad \rho.
%\end{equation}
%
%For an incompressible fluid
%
%\begin{equation}\label{eqn:continuumL10:110}
%\spacegrad \cdot \Bu = 0
%\end{equation}
%
%so the conservation of mass equality relation takes the form
%\begin{equation}\label{eqn:continuumL10:90b}
%0 = \PD{t}{\rho} + \Bu \cdot \spacegrad \rho.
%\end{equation}
%
\section{Conservation of momentum (Navier-Stokes equation)}

Reading: \S 6.* from \citep{acheson1990elementary}.

In classical mechanics we have

\begin{equation}\label{eqn:continuumL10:130}
\Bf = m \Ba,
\end{equation}

our analogue here is found in terms of the stress tensor

\begin{equation}\label{eqn:continuumL10:150}
\int_V F_i dV = \int_V \PD{x_j}{\sigma_{ij}} dV
\end{equation}

Here \(F_i\) is the force per unit volume.  With body forces we have

\begin{equation}\label{eqn:continuumL10:170}
F_i = \rho \frac{du_i}{dt} = \PD{x_j}{\sigma_{ij}} + \rho f_i
\end{equation}

where \(f_i\) is an external force per unit volume.  Observe that \(\sigma_{ij}\), through the constitutive relation, includes both contributions of linear displacement and the vorticity component.

From the constitutive relation \eqnref{eqn:continuumL9:250}, we have

\begin{equation}\label{eqn:NavierStokes:350}
\begin{aligned}
\PD{x_j}{\sigma_{ij}}
&= - \PD{x_j}{p} \delta_{ij} + 2 \mu \PD{x_j}{e_{ij}} \\
&= - \PD{x_i}{p} + 2 \mu \PD{x_j}{} \left(
\inv{2} \left(
 \PD{x_j}{u_i}
+ \PD{x_i}{u_j}
\right)
\right) \\
&= - \PD{x_i}{p} + \mu \left(
\frac{\partial^2 u_i}{\partial x_j \partial x_j}
+\frac{\partial^2 u_j}{\partial x_i \partial x_j}
\right)
\end{aligned}
\end{equation}

Observe that the term

\begin{equation}\label{eqn:continuumL10:190}
\frac{\partial^2 u_i}{\partial x_j \partial x_j}
\end{equation}

is the \(i^{\text{th}}\) component of \(\spacegrad^2 \Bu\), whereas

\begin{equation}\label{eqn:NavierStokes:370}
\begin{aligned}
\frac{\partial^2 u_j}{\partial x_i \partial x_j}
&= \PD{x_i}{} \left( \PD{x_j}{u_j} \right) \\
&= \PD{x_i}{} (\spacegrad \cdot \Bu)
\end{aligned}
\end{equation}

is the \(i^{\text{th}}\) component of \(\spacegrad (\spacegrad \cdot \Bu)\).

We have therefore that

\begin{equation}\label{eqn:continuumL10:210}
\rho \frac{du_i}{dt} = \left( -\spacegrad p + \mu \spacegrad^2 \Bu
+ \mu \spacegrad (\spacegrad \cdot \Bu) + \rho \Bf
\right)_i,
\end{equation}

or in vector notation

\begin{equation}\label{eqn:continuumL10:230}
\rho \frac{d\Bu}{dt} = -\spacegrad p + \mu \spacegrad^2 \Bu
+ \mu \spacegrad (\spacegrad \cdot \Bu) + \rho \Bf.
\end{equation}

We can expand this a bit more writing our velocity \(\Bu = \Bu(x, y, z, t)\) differential

\begin{equation}\label{eqn:continuumL10:250}
du_i = \PD{x_j}{u_i} \delta x_j + \PD{t}{u_i} \delta t.
\end{equation}

Considering rates

\begin{equation}\label{eqn:continuumL10:270}
\frac{du_i}{dt} = \PD{x_j}{u_i} \frac{dx_j}{dt} + \PD{t}{u_i} .
\end{equation}

In vector notation we have

\begin{equation}\label{eqn:continuumL10:290}
\frac{d\Bu}{dt} = (\Bu \cdot \spacegrad) \Bu + \PD{t}{\Bu}.
\end{equation}

Newton's second law \eqnref{eqn:continuumL10:230} now becomes

\boxedEquation{eqn:continuumL10:230b}{
\rho
 (\Bu \cdot \spacegrad) \Bu + \rho \PD{t}{\Bu}
= -\spacegrad p + \mu \spacegrad^2 \Bu
+ \mu \spacegrad (\spacegrad \cdot \Bu) + \rho \Bf.
}

This is the Navier-Stokes equation.  Observe that we have an explicitly non-linear term

\begin{equation}\label{eqn:continuumL10:310}
(\Bu \cdot \spacegrad) \Bu ,
\end{equation}

something we do not encounter in most classical mechanics.  The impacts of this non-linear term are very significant and produce some interesting effects.

\section{Incompressible fluids}

We have seen that incompressibility was equivalent to

\begin{equation}\label{eqn:continuumL10:330}
\spacegrad \cdot \Bu = 0.
\end{equation}

With such a restriction the Navier-Stokes equation takes the much simpler form


\boxedEquation{eqn:continuumL10:230c}{
\rho
 (\Bu \cdot \spacegrad) \Bu + \rho \PD{t}{\Bu}
= -\spacegrad p + \mu \spacegrad^2 \Bu
+ \rho \Bf
\spacegrad \cdot \Bu = 0.
}

We will not treat compressible fluids in this course.

\section{Boundary value conditions}

In order to solve any sort of PDE we need to consider the boundary value conditions.  Consider the interface between two layers of liquids as in \cref{fig:continuumL9:continuumL10fig1}

\imageFigure{../../figures/phy454/lec10_Rocker_tank_with_two_viscosity_fluidsFig1}{Rocker tank with two viscosity fluids}{fig:continuumL9:continuumL10fig1}{0.2}

Also found an illustration of this in \href{watinst.ut.ac.ir/downloads/pdf/ebooks/white.pdf}{fig 1.13 of White's text online}

We see the fluids sticking together at the boundary.  This is due to matching of the tangential velocity components at the interface.
