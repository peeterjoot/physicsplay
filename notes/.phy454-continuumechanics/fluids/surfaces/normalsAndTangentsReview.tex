%
% Copyright � 2012 Peeter Joot.  All Rights Reserved.
% Licenced as described in the file LICENSE under the root directory of this GIT repository.
%
% L14
\section{Review.  Surfaces}

We are considering a surface as depicted in \cref{fig:continuumL14:continuumL14Fig13}

\imageFigure{../../figures/phy454/lec13_variable_surface_geometriesFig13}{Variable surface geometries}{fig:continuumL14:continuumL14Fig13}{0.2}

With the surface height given by

\begin{equation}\label{eqn:continuumL14:10}
z = h(x, t),
\end{equation}

where this describes the interface.  Taking the difference

\begin{equation}\label{eqn:continuumL14:30}
\phi = z - h(x, t) = 0,
\end{equation}

we define a surface.  We considered a small displacement as in \cref{fig:continuumL14:continuumL14Fig14}.

\imageFigure{../../figures/phy454/lec13_a_vector_differential_elementFig14}{A vector differential element}{fig:continuumL14:continuumL14Fig14}{0.2}

Recall that if \(\phi\) is a constant, then \(\spacegrad \phi\) is a normal to the surface.  We showed this by considering the differential

\begin{equation}\label{eqn:normalsAndTangentsReview:90}
\begin{aligned}
0
&= d\phi \\
&=
\PD{x}{\phi} dx
+\PD{y}{\phi} dy
+\PD{z}{\phi} dz \\
&=
(\spacegrad \phi) \cdot d\Br.
\end{aligned}
\end{equation}

We can construct the unit normal by scaling.  For our 1D example we have

\begin{equation}\label{eqn:normalsAndTangentsReview:110}
\begin{aligned}
\ncap
&= \frac{\spacegrad \phi}{\Abs{\spacegrad \phi}} \\
&= \inv{\Abs{\spacegrad \phi}}
\left(
\PD{x}{\phi},
\PD{y}{\phi}
\right)
\end{aligned}
\end{equation}

so that our unit normal is
\begin{equation}\label{eqn:continuumL14:50}
\ncap
= \inv{ \sqrt{1 + (h')^2}}
\left( -\PD{x}{h}, 1 \right)
\end{equation}

A unit tangent can also be constructed by inspection

\begin{equation}\label{eqn:continuumL14:70}
\taucap
= \inv{ \sqrt{1 + (h')^2}}
\left( 1, \PD{x}{h} \right).
\end{equation}
