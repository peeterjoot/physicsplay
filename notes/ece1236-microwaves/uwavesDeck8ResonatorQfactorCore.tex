%
% Copyright © 2016 Peeter Joot.  All Rights Reserved.
% Licenced as described in the file LICENSE under the root directory of this GIT repository.
%
F1

\begin{dmath}\label{eqn:uwavesDeck8ResonatorQfactorCore:20}
P_{\textrm{loss} = \inv{2} \Abs{I}^2 R
\end{dmath}

Average stored magnetic energy 
\index{energy!stored magnetic}

\begin{dmath}\label{eqn:uwavesDeck8ResonatorQfactorCore:40}
W_m = \inv{4} \Abs{I}^2 L
\end{dmath}

Average stored electric energy 
\index{energy!stored electric}

\begin{dmath}\label{eqn:uwavesDeck8ResonatorQfactorCore:60}
W_e 
= \inv{4} \Abs{V_c}^2 C
= \inv{4} \Abs{I}^2 \inv{\omega^2 C}
\end{dmath}

\index{resonance}
At resonance \( W_m = W_e \) or

\begin{dmath}\label{eqn:uwavesDeck8ResonatorQfactorCore:80}
L = \inv{\omega_0^2 C}
\end{dmath}

and 

\begin{dmath}\label{eqn:uwavesDeck8ResonatorQfactorCore:100}
\omega_0 = \inv{\sqrt{L C}}
\end{dmath}

\begin{dmath}\label{eqn:uwavesDeck8ResonatorQfactorCore:120}
Q = \omega_0 \frac{\textrm{average stored energy}}{\textrm{energy loss/second}}
= \omega_0 \frac{W_e + W_m}{P_\loss}
\end{dmath}

At resonance,  \( W_m = W_e = \inv{4} \Abs{I}^2 L\), so

\begin{dmath}\label{eqn:uwavesDeck8ResonatorQfactorCore:140}
Q 
= \omega_0 \frac{\inv{2} \Abs{I}^2 L}{\inv{2} \Abs{I}^2 R} 
= \omega_0 \frac{L}{R}
= \frac{X}{R}
\end{dmath}

High Q, means that we have low losses, but is also a measure of the bandwidth.

\paragraph{Bandwith of resonators and Q-factor}
\index{resonator bandwith}
\index{Q-factor}

To see why this measures the bandwidth, lets find the impedance

\begin{dmath}\label{eqn:uwavesDeck8ResonatorQfactorCore:160}
Z_{\textrm{in}} 
= R + j \omega L - \frac{j}{\omega C}
= R + j \omega L \lr{ 1 - \inv{\omega^2 L C} }
= R + j \omega L \frac{\omega^2 - \omega_0^2}{\omega^2}, 
\end{dmath}

where the resonant frequency is

\begin{dmath}\label{eqn:uwavesDeck8ResonatorQfactorCore:180}
\omega_0 = \inv{\sqrt{LC}}
\end{dmath}

Also, 

\begin{dmath}\label{eqn:uwavesDeck8ResonatorQfactorCore:200}
\omega^2 - \omega_0^2 = \lr{ \omega - \omega_0 }\lr{ \omega + \omega_0 }
= 
...
\end{dmath}

F2

The half-power bandwidth is defined when \( \Abs{Z_{\textrm{in}}} = \sqrt{2} R \).  In this case the fractional bandwidth 
\begin{dmath}\label{eqn:uwavesDeck8ResonatorQfactorCore:220}
BW = \frac{2 \Delta \omega}{\omega_0}
\end{dmath}

\begin{dmath}\label{eqn:uwavesDeck8ResonatorQfactorCore:240}
Z_{\textrm{in}} = R + j R Q BW, 
\end{dmath}

gives

\begin{dmath}\label{eqn:uwavesDeck8ResonatorQfactorCore:260}
\Abs{Z_{\textrm{in}}}^2 = R^2\lr{ 1 + Q^2 BW^2 } = 2 R^2,
\end{dmath}

or

\begin{dmath}\label{eqn:uwavesDeck8ResonatorQfactorCore:280}
Q^2 BW^2 = 1,
\end{dmath}

or

\begin{dmath}\label{eqn:uwavesDeck8ResonatorQfactorCore:300}
BW = \inv{Q}
\end{dmath}

...

F3

If you want a resonator (such as an antenna) you want a high-Q, but for a filter, low-Q.  It is the opposite for antennas

F8

We want low \( Q \) in this case

\begin{dmath}\label{eqn:uwavesDeck8ResonatorQfactorCore:n}
Q = \frac{\omega L}{R_\ant},
\end{dmath}

a small antenna conspires to produce a high Q, both from having a small radiation resistance, and due to large inductance.

F9

For microstrip circuits the Q's are small, perhaps in the 10-500 range.  For example a ring on a substrate

F4

you may have a high-Q ( \( Q \sim 500 \) ).

whereas printing on a CMOS grade silicon substrate, you may get a \(Q \sim 10\).

F5

In optics you may find really high Q's (say 10000).  The reason is that \( X \) value is huge because the structure can be made many wavelengths in size.

\paragraph{Q-Factor for a capacitor}
\index{capacitor!Q-Factor}

series case:

F6

\begin{dmath}\label{eqn:uwavesDeck8ResonatorQfactorCore:320}
V_c = \frac{I}{j \omega C}
\end{dmath}

\begin{dmath}\label{eqn:uwavesDeck8ResonatorQfactorCore:n}
Q = \frac{X_c}{R} = \inv{\omega R C}
\end{dmath}

Parallel (shunt) case:

\end{dmath}
Q 
= \omega \frac{ W_e + W_m }{P_L} 
= \omega \frac{\textrm{...}}{\textrm{...}}
\begin{dmath}\label{eqn:uwavesDeck8ResonatorQfactorCore:340}

F7

\begin{dmath}\label{eqn:uwavesDeck8ResonatorQfactorCore:360}
Q = \frac{B_c}{G} = \omega R C
\end{dmath}

Large resistance (low loss), means high Q.

\paragraph{Q-factor for an inductor}
\index{inductor!Q-factor}

F10

\begin{dmath}\label{eqn:uwavesDeck8ResonatorQfactorCore:n}
Q_\txts = \frac{X_L}{R} = \frac{\omega L}{R}
\end{dmath}

F11

\begin{dmath}\label{eqn:uwavesDeck8ResonatorQfactorCore:n}
Q_\txtp = \frac{R_L}{G} = \frac{R}{\omega L}
\end{dmath}

\paragraph{Plot of constant \( Q = X/R \) contours (circular arcs) on the Smith Chart}
\index{Smith chart!constant contours}

Referrring to 

F12

we can plot the ratio

\begin{dmath}\label{eqn:uwavesDeck8ResonatorQfactorCore:n}
Q = \frac{X}{R}
\end{dmath}


F13 (generated with Matlab)

The \( Q \) is then dictated by the load, for example, for a series inductive load

\begin{dmath}\label{eqn:uwavesDeck8ResonatorQfactorCore:n}
Q_\L = \frac{\omega L }{R}
\end{dmath}

so for a given load the bandwidth is dictated.

\section{Bode-Fano relations}
\index{Bode-Fano relations}

For a shunt RC load

F14

\begin{dmath}\label{eqn:uwavesDeck8ResonatorQfactorCore:n}
\int_0^\infty \ln \frac{1}{\Abs{\Gamma(\omega}} d\omega < \frac{\pi}{R C}
\end{dmath}

Suppose we have reflection relation like

F15

representing a finite bandwidth.  Where the reflection coefficient is unity, the log is zero, then the Bode

\begin{dmath}\label{eqn:uwavesDeck8ResonatorQfactorCore:n}
\ln \frac{1}{\Abs{\Gamma(\omega}} \Delta\omega < \frac{\pi}{R C}
\end{dmath}

So, to get more bandwidth, we have to sacrifice some reflectivity.  This also implies that if the reflection coefficient is zero, we have zero bandwidth.

In terms of

\begin{dmath}\label{eqn:uwavesDeck8ResonatorQfactorCore:n}
Q = \omega_0 R C ,
\end{dmath}

\begin{dmath}\label{eqn:uwavesDeck8ResonatorQfactorCore:n}
... \frac{ \Delta \omega}{\omega_0} < \frac{\pi}{Q}
\end{dmath}
...

\paragraph{Example with low \( Q = 10 \) }


\begin{itemize}
\item
If \( \Gamma_m \) is small, say \( \Gamma_m = 0.1 \), then

\begin{dmath}\label{eqn:uwavesDeck8ResonatorQfactorCore:n}
2.33 \frac{ \Delta \omega}{\omega_0} < ...
\end{dmath}

\item
If \( \Gamma_m \) is larger, say \( \Gamma_m = 0.5 \), then

\begin{dmath}\label{eqn:uwavesDeck8ResonatorQfactorCore:n}
0.7 \frac{ \Delta \omega}{\omega_0}  < 0.1 \pi
\end{dmath}

so 

\begin{dmath}\label{eqn:uwavesDeck8ResonatorQfactorCore:n}
\frac{ \Delta \omega}{\omega_0 } < 
\end{dmath}

\item If \( Q \) is very high.
\end{itemize}

The conclusions are that for a given load where \( R C \) is constant, a broader bandwidth can be achieved only at the expense of a higher reflection coefficient.

\paragraph{Bode-Fano relations for other loads}

F16

\paragraph{Derivation of Bode-Fano limit for a parallel RC load}

F17

where

\begin{dmath}\label{eqn:uwavesDeck8ResonatorQfactorCore:n}
\Gamma_\in = \frac{Z_\in - Z_0}{Z_\in + Z_0},
\end{dmath}

and

\begin{dmath}\label{eqn:uwavesDeck8ResonatorQfactorCore:n}
\Gamma_\out = \frac{Z_\out - R }{Z_\out + R}.
\end{dmath}

We need to show that 

\begin{dmath}\label{eqn:uwavesDeck8ResonatorQfactorCore:n}
\int_{-\infty}^\infty = \ln \inv{\Abs{\Gamma{\omega}} d\omega \le \frac{\pi}{R C }
\end{dmath}

Notice how we include \( C \) in the definition of \( Z_\out \).  Because the matching network (M.N.) is lossless 

\begin{dmath}\label{eqn:uwavesDeck8ResonatorQfactorCore:n}
\Abs{\Gamma_\in} = \Abs{\Gamma_\out},
\end{dmath}

It is more convenient to use \( \Abs{\Gamma_\out} \).  The key physical observation is that when the capacitor is shorted, we cannot pass power to the load (\(R\)).  That power has to go somewhere, and has to be reflected.

...

This can be understood intuitively but also if we have two cascaded 2-ports:

F18

Now consider the return loss function defined as

\begin{dmath}\label{eqn:uwavesDeck8ResonatorQfactorCore:n}
\ln \inv{\Abs{\Gamma_\out}} > 0
\end{dmath}

we would like to determine the maximum value of

\begin{dmath}\label{eqn:uwavesDeck8ResonatorQfactorCore:n}
\int 
\ln \inv{\Abs{\Gamma_\out}} d\omega
\end{dmath}

\Gamma(s) = \frac{Z_\in - Z_0}{Z_\in + Z_0}

F19

We can only get poles when

\begin{dmath}\label{eqn:uwavesDeck8ResonatorQfactorCore:n}
Z_\in + Z_0 = R_\in + j X_\in + Z_0 = 0,
\end{dmath}

so we have no poles.  On the other hand, we can have zeros where \( Z_\in - Z_0 = 0 \).

That implies that integration around a closed contour gives zero, for example

\begin{dmath}\label{eqn:uwavesDeck8ResonatorQfactorCore:n}
\oint \Abs{ \Gamma(s) } ds = 0 
\end{dmath}

Some basic relations can be obtained from this, but that isn't what Bode did.  Instead he picked the log of the reflection coefficient

\begin{dmath}\label{eqn:uwavesDeck8ResonatorQfactorCore:n}
\oint 
\ln \inv{\Abs{\Gamma}} d\omega
=
-\oint 
\ln \Abs{\Gamma} d\omega
\ne 0.
\end{dmath}

Because \( \Gamma \) can have zeros, the log makes those right hand plane poles

F20

Assume that \( a_1, a_2, \cdots \) are poles on the RHP due to the load and te matching network.  These happen when \( Z_\out = R \)
...

Now on the RHP there are no singularities, so on a contour considering of the imaginary axes (frequency axes) and a large semicircle of radius \( R_0 \rightarrow \infty \).

\begin{dmath}\label{eqn:uwavesDeck8ResonatorQfactorCore:n}
\oint \ln \Abs{ \inv{\Gamma} 
\frac{s - a_1}{s+ a_1}
\frac{s - a_2}{s+ a_2}
\frac{s - a_3}{s+ a_3}
\cdots
} d\omega
\end{dmath}

Now on the frequency asxes ( \( j \omega \) )

\Abs{\frac{s - a_1}{s+ a_1}} = 
\Abs{\frac{j\omega - a_1}{j \omega+ a_1}} = 1

%?
if \( a_1 \) is real.

...

Hense on the \( s = j \omega \) axis the integrant becomes equal to 

\begin{dmath}\label{eqn:uwavesDeck8ResonatorQfactorCore:n}
j \int \ln \inv{\Abs{\Gamma}} d\omega
\end{dmath}

On the large semicircle, each term 

\ln \Abs{\frac{s - a_i}{s + a_i} } contributes for \( s \rightarrow \infty \), as follows.  Let

\begin{dmath}\label{eqn:uwavesDeck8ResonatorQfactorCore:n}
s = R_0 e^{j \theta}
\end{dmath}

so

\begin{dmath}\label{eqn:uwavesDeck8ResonatorQfactorCore:n}
\ln \Abs{\frac{s - a_i}{s + a_i} } 
\approx
\ln \Abs{\frac{s - a_i}{s + a_i} } 
\end{dmath}
