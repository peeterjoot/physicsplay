%
% Copyright © 2016 Peeter Joot.  All Rights Reserved.
% Licenced as described in the file LICENSE under the root directory of this GIT repository.
%

\section{Matching networks}
\index{matching networks}

Available power from the source (maximum power that can be delivered) is

\begin{dmath}\label{eqn:uwavesDeck6LCmatchingCore:20}
P_A 
= \frac{\Abs{V_\txtG/2}^2}{2 Z_\txtG}
= \frac{\Abs{V_\txtG}^2}{8 R_\txtG}.
\end{dmath}

For the general matching network sketched in

F3

it is left as a problem to show that

\begin{dmath}\label{eqn:uwavesDeck6LCmatchingCore:40}
P_\txtL = P_A \frac{ 
\lr{ 1 - \Abs{\Gamma_\txtG}^2 }
\lr{ 1 - \Abs{\Gamma_\txtL}^2 } }
{
\Abs{ 1 - \Gamma_\txtG \Gamma_\txtL e^{-2 j \beta l } }^2
}
\end{dmath}

for a matched line 

\begin{dmath}\label{eqn:uwavesDeck6LCmatchingCore:60}
P_\txtL = P_A
\end{dmath}

Matching cases

\begin{enumerate}
\item Line matched to load \( Z_\txtL' = Z_0 \), so that

\begin{dmath}\label{eqn:uwavesDeck6LCmatchingCore:80}
\Gamma_\txtL 
= \frac{Z_\txtL' - Z_0}{Z_\txtL' + Z_0}
= 0.
\end{dmath}

In this case, 

\begin{dmath}\label{eqn:uwavesDeck6LCmatchingCore:100}
P_\txtL = P_A \lr{ 1 - \Abs{\Gamma_\txtG}^2 }
\end{dmath}

\item With a line matched to generator, \( Z_\txtG' = Z_0 \), so

\begin{dmath}\label{eqn:uwavesDeck6LCmatchingCore:120}
\Gamma_\txtG 
= \frac{Z_\txtG' - Z_0}{Z_\txtL' + Z_0}
= 0.
\end{dmath}

Now we have

\begin{dmath}\label{eqn:uwavesDeck6LCmatchingCore:140}
P_\txtL = P_A \lr{ 1 - \Abs{\Gamma_\txtL}^2 }
\end{dmath}

\item Line matched to both the load and the generator, with \( Z_\txtG' = Z_0, Z_\txtL' = Z_0 \), 
so that \( \Gamma_\txtG = \Gamma_\txtL = 0 \).
In this case \( P_\txtL = P_A \) and there are reflections on the line.

\item With conjugate matching, where \( Z_{\textrm{in}} = Z_\txtG^\conj \), we have

\begin{dmath}\label{eqn:uwavesDeck6LCmatchingCore:160}
\Gamma_\txtG 
= \frac{Z_\txtG' - Z_0}{Z_\txtG' + Z_0}
= \frac{Z_{\textrm{in}}^\conj - Z_0}{Z_\txtG' + Z_0}
\end{dmath}

FIXME: this part of his notes doesn't make sense.

\end{enumerate}

\section{Matching with lumped elements}
\index{lumped elements}

A lumped element is something very small compared to the line.  For L-section matching theory we are left to read \S 5.1 \citep{pozar2009microwave}.  The rough idea is that

F3:A

For (A) want 

\begin{equation}\label{eqn:uwavesDeck6LCmatchingCore:180}
Z_{\textrm{in}} = Z_0 = j X + \inv{ j B + \inv{Z_\txtL} }
\end{equation}

F3:B

For (B) 

\begin{equation}\label{eqn:uwavesDeck6LCmatchingCore:200}
Z_{\textrm{in}} = Z_0 = j X + \inv{ j B + \inv{R_\txtL + j X_\txtL} }
\end{equation}

See: Smith chart slide S1.

Example: For a \( 100 \Omega \) line

\begin{equation}\label{eqn:uwavesDeck6LCmatchingCore:220}
Z_\txtL = 200 -j 100 \Omega 
\end{equation}

Normalized is
\begin{equation}\label{eqn:uwavesDeck6LCmatchingCore:240}
Z_\txtL = 2 -j \Omega 
\end{equation}

This is outside the \( 1 + j X \) circle.
