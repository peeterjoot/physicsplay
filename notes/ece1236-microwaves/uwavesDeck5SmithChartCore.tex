%
% Copyright © 2016 Peeter Joot.  All Rights Reserved.
% Licenced as described in the file LICENSE under the root directory of this GIT repository.
%

\section{Short circuited line}

A short circuited line, also called a shorted stub, is sketched in \cref{fig:deck5smithChartsAndImpedenceTx:deck5smithChartsAndImpedenceTxFig1}.

\imageFigure{../../figures/ece1236/deck5smithChartsAndImpedenceTxFig1}{Short circuited line.}{fig:deck5smithChartsAndImpedenceTx:deck5smithChartsAndImpedenceTxFig1}{0.2}

With 
\begin{equation}\label{eqn:uwavesDeck5SmithChartCore:20}
Z_\txtL = 0,
\end{equation}

the input impedance is

\begin{equation}\label{eqn:uwavesDeck5SmithChartCore:40}
Z_{\textrm{in}} 
= Z_0 \frac{ Z_\txtL + j Z_0 \tan(\beta l) }{ Z_0 + j Z_\txtL \tan(\beta l)}
= j Z_0 \tan(\beta l)
\end{equation}

%At the load
%
%\begin{dmath}\label{eqn:uwavesDeck5SmithChartCore:60}
%\Gamma_\txtL
%= \frac{Z_\txtL - Z_0}{Z_\txtL + Z_0}
%= -1,
%\end{dmath}
%
%so

For short line sections \( \beta l \ll \pi/2 \), or \( l \ll \lambda/4 \), the input impedance is approximately

\begin{dmath}\label{eqn:uwavesDeck5SmithChartCore:80}
Z_{\textrm{in}} 
= j Z_0 \tan(\beta l)
\approx j Z_0 \sin(\beta l)
\approx j Z_0 \beta l
\end{dmath}

Introducing an equivalent inductance defined by \( Z_{\textrm{in}} = j \omega L_{\mathrm{eq}} \), we have

\begin{dmath}\label{eqn:uwavesDeck5SmithChartCore:100}
L_{\mathrm{eq}}
=
\frac{Z_0}{\omega} \beta l 
=
\frac{Z_0}{\omega} \frac{\omega}{v_\phi} l 
=
\frac{Z_0 l}{v_\phi}.
\end{dmath}

The inductance per unit length of the line is \( C = Z_0/v_\phi \).  An application for this result is that instead of using inductors, shorted stubs can be used in high frequency applications.

This is also the case for short sections of high impedance line.

\section{Open circuited line}

An open circuited line is sketched in \cref{fig:deck5smithChartsAndImpedenceTx:deck5smithChartsAndImpedenceTxFig2}.

\imageFigure{../../figures/ece1236/deck5smithChartsAndImpedenceTxFig2}{Open circuited line.}{fig:deck5smithChartsAndImpedenceTx:deck5smithChartsAndImpedenceTxFig2}{0.2}

This time with \( Z_\txtL \rightarrow \infty \) we have

\begin{dmath}\label{eqn:uwavesDeck5SmithChartCore:120}
Z_{\textrm{in}} 
= Z_0 \frac{ Z_\txtL + j Z_0 \tan(\beta l) }{ Z_0 + j Z_\txtL \tan(\beta l)}
= -j Z_0 \cot(\beta l).
\end{dmath}

This time we have an equivalent capacitance.  For short sections with \( \beta l \ll \pi/2 \)

\begin{dmath}\label{eqn:uwavesDeck5SmithChartCore:140}
Z_{\textrm{in}}
\approx
-j \frac{Z_0}{\beta l}
\end{dmath}

Introducing an equivalent capacitance defined by \( Z_{\textrm{in}} = 1/(j \omega C_{\mathrm{eq}}) \), we have

\begin{dmath}\label{eqn:uwavesDeck5SmithChartCore:160}
C_{\mathrm{eq}}
=
\frac{ \beta l}{\omega Z_0}
=
\frac{ \omega/v_\phi l}{\omega Z_0}
=
\frac{ l}{v_\phi Z_0}
\end{dmath}

The capacitance per unit length of the line is \( C = 1/(Z_0 v_\phi) \).

This is also the case for short sections of low impedance line.

\section{Half wavelength transformer.}

A half wavelength transmission line equivalent circuit is sketched in \cref{fig:deck5smithChartsAndImpedenceTx:deck5smithChartsAndImpedenceTxFig3}.

\imageFigure{../../figures/ece1236/deck5smithChartsAndImpedenceTxFig3}{Half wavelength transmission line.}{fig:deck5smithChartsAndImpedenceTx:deck5smithChartsAndImpedenceTxFig3}{0.2}

With \( l = \lambda/2 \)

\begin{dmath}\label{eqn:uwavesDeck5SmithChartCore:180}
\beta l 
= \frac{2 \pi}{\lambda} \frac{\lambda}{2}
= \pi.
\end{dmath}

Since \( \tan \pi = 0 \), the input impedance is

\begin{dmath}\label{eqn:uwavesDeck5SmithChartCore:200}
Z_{\textrm{in}} 
= Z_0 \frac{ Z_\txtL + j Z_0 \tan(\beta l) }{ Z_0 + j Z_\txtL \tan(\beta l)}
= Z_\txtL.
\end{dmath}

\section{Quarter wavelength transformer.}

A quarter wavelength transmission line equivalent circuit is sketched in \cref{fig:deck5smithChartsAndImpedenceTx:deck5smithChartsAndImpedenceTxFig4}.

\imageFigure{../../figures/ece1236/deck5smithChartsAndImpedenceTxFig4}{Quarter wavelength transmission line.}{fig:deck5smithChartsAndImpedenceTx:deck5smithChartsAndImpedenceTxFig4}{0.2}

With \( l = \lambda/4 \)

\begin{dmath}\label{eqn:uwavesDeck5SmithChartCore:220}
\beta l 
= \frac{2 \pi}{\lambda} \frac{\lambda}{4}
= \frac{\pi}{2}.
\end{dmath}

We have \( \tan \beta l \rightarrow \infty \), so the input impedance is

\begin{dmath}\label{eqn:uwavesDeck5SmithChartCore:240}
Z_{\textrm{in}} 
= Z_0 \frac{ Z_\txtL + j Z_0 \tan(\beta l) }{ Z_0 + j Z_\txtL \tan(\beta l)}
= \frac{Z_0^2}{Z_\txtL}.
\end{dmath}

This relation

%\begin{equation}\label{eqn:uwavesDeck5SmithChartCore:280}
\boxedEquation{eqn:uwavesDeck5SmithChartCore:280}
{
Z_{\textrm{in}} 
= \frac{Z_0^2}{Z_\txtL},
}
%\end{equation}

is called the \textAndIndex{impedance inverter}.

\begin{itemize}
\item A large impedance is transformed into a small one and vice-versa.
\item A short becomes an open and vice-versa.
\item A capacitive load becomes inductive and vice-versa.
\item If \( Z_\txtL \) is a series resonant circuit then \( Z_{\textrm{in}} \) becomes parallel resonant.
\end{itemize}

See \citep{seriesResonance} for an explanation of the term series resonant.

\paragraph{Matching with a \( \lambda/4 \) transformer.}

Matching for a quarter wavelength transmission line equivalent circuit is sketched in \cref{fig:deck5smithChartsAndImpedenceTx:deck5smithChartsAndImpedenceTxFig5}.

\imageFigure{../../figures/ece1236/deck5smithChartsAndImpedenceTxFig5}{Quarter wavelength transmission line matching.}{fig:deck5smithChartsAndImpedenceTx:deck5smithChartsAndImpedenceTxFig5}{0.2}
%\cref{fig:deck5smithChartsAndImpedenceTx:deck5smithChartsAndImpedenceTxFig6}.
%\imageFigure{../../figures/ece1236/deck5smithChartsAndImpedenceTxFig6}{CAPTION: deck5smithChartsAndImpedenceTxFig6}{fig:deck5smithChartsAndImpedenceTx:deck5smithChartsAndImpedenceTxFig6}{0.2}

For maximum power transfer

\begin{equation}\label{eqn:uwavesDeck5SmithChartCore:300}
Z_{\textrm{in}} = \frac{Z_0^2}{R_\txtL} = R_\txtG, 
\end{equation}

so

\begin{equation}\label{eqn:uwavesDeck5SmithChartCore:320}
Z_0 = \sqrt{ R_\txtG R_\txtL }.
\end{equation}

We have

\begin{equation}\label{eqn:uwavesDeck5SmithChartCore:340}
\Abs{\Gamma_\txtL} = \frac{ R_\txtL - Z_0 }{R_\txtL + Z_0} \ne 0,
\end{equation}

and still maximum power is transferred.

\section{Smith chart}

A Smith chart is a graphical tool for making the transformation \( \Gamma \leftrightarrow Z_{\textrm{in}} \).  Given

\begin{equation}\label{eqn:uwavesDeck5SmithChartCore:360}
Z_{\textrm{in}} = Z_0 \frac{ 1 + \Gamma }{ 1 - \Gamma },
\end{equation}

where \( \Gamma = \Gamma_\txtL e^{- 2 j \beta l } \), we begin by normalizing the input impedance, using an overbar to denote that normalization

\begin{equation}\label{eqn:uwavesDeck5SmithChartCore:380}
Z_{\textrm{in}} \rightarrow \overbar{Z}_{\textrm{in}} = \frac{Z_{\textrm{in}}}{Z_0}, 
\end{equation}

so

\begin{dmath}\label{eqn:uwavesDeck5SmithChartCore:400}
\overbar{Z}_{\textrm{in}} 
= \frac{ 1 + \Gamma }{ 1 - \Gamma }
= \frac{ (1 + \Gamma_r) + j \Gamma_i }{ (1 - \Gamma_r) - j \Gamma_i }
= \frac{ \lr{ (1 + \Gamma_r) + j \Gamma_i}\lr{(1 - \Gamma_r) + j \Gamma_i} }{ (1 - \Gamma_r)^2 + \Gamma_i^2 }
= \frac{ (1 - \Gamma_r^2 - \Gamma_i^2) + j \Gamma_i (1 - \Gamma_r + 1 + \Gamma_r ) }{ (1 - \Gamma_r)^2 + \Gamma_i^2 }
= \frac{ (1 - \Abs{\Gamma}^2) + 2 j \Gamma_i }{ (1 - \Gamma_r)^2 + \Gamma_i^2 }.
\end{dmath}

If we let \( \overbar{Z}_{\textrm{in}} = \overbar{\Gamma}_\txtL + j \overbar{X}_\txtL \), and equate real and imaginary parts we have

\begin{equation}\label{eqn:uwavesDeck5SmithChartCore:420}
\begin{aligned}
\overbar{\Gamma}_\txtL &= \frac{ 1 - \Abs{\Gamma}^2 }{ (1 - \Gamma_r)^2 + \Gamma_i^2 } \\
\overbar{X}_\txtL &= \frac{2 \Gamma_i }{ (1 - \Gamma_r)^2 + \Gamma_i^2 }
\end{aligned}
\end{equation}

It is left as an exercise to demonstrate that these can be rearranged into

\begin{equation}\label{eqn:uwavesDeck5SmithChartCore:440}
\begin{aligned}
\lr{ \Gamma_r - \frac{\overbar{\Gamma}_\txtL}{1 + \overbar{\Gamma}_\txtL } }^2 + \Gamma_i^2 &= \lr{ \inv{1 + \overbar{\Gamma}_\txtL }}^2 \\
\lr{ \Gamma_r - 1 }^2 + \lr{ \Gamma_i - \inv{\overbar{X}_\txtL } }^2 &= \inv{\overbar{X}_\txtL^2},
\end{aligned}
\end{equation}

which trace out circles in the \( \Gamma_r, \Gamma_i \) plane, one for the real part of \( \Gamma \) and one for the imaginary part.  This provides a graphical way for implementing the impedance transformation.  

\paragraph{Real impedance circle}

The circle for the real part is centered at 

\begin{dmath}\label{eqn:uwavesDeck5SmithChartCore:580}
\lr{ \frac{\overbar{\Gamma}_\txtL}{1 + \overbar{\Gamma}_\txtL }, 0 },
\end{dmath}

with radius
\begin{dmath}\label{eqn:uwavesDeck5SmithChartCore:600}
\inv{1 + \overbar{\Gamma}_\txtL }.
\end{dmath}

All these circles pass through the point \( (1,0) \), since
\begin{dmath}\label{eqn:uwavesDeck5SmithChartCore:620}
\frac{\overbar{\Gamma}_\txtL}{1 + \overbar{\Gamma}_\txtL } + \inv{1 + \overbar{\Gamma}_\txtL }
=
\frac{1 + \overbar{\Gamma}_\txtL}{1 + \overbar{\Gamma}_\txtL }
= 1.
\end{dmath}

For reactive loads where \( \overbar{\Gamma}_\txtL = 0 \), we have \( \Gamma_r^2 + \Gamma_i^2 = 1 \), a circle through the origin with unit radius.

For matched loads where \( \overbar{\Gamma}_\txtL = 1 \) the circle is centered at \( (1/2, 0) \), with radius \( 1/2 \).

\paragraph{Imaginary impedance circle}

The circle obtained by equating imaginary parts are constant reactance circles with center

\begin{dmath}\label{eqn:uwavesDeck5SmithChartCore:640}
\lr{ 1, \inv{\overbar{X}_\txtL } }, 
\end{dmath}

with radius 

\begin{dmath}\label{eqn:uwavesDeck5SmithChartCore:660}
\inv{\overbar{X}_\txtL}.
\end{dmath}

These circles also pass through the point \( (1,0) \).  These circles are orthogonal to the constant resistance circles.  Some of the features of a Smith chart are sketched in \cref{fig:deck5smithChartsAndImpedenceTx:deck5smithChartsAndImpedenceTxFig7}.

\imageFigure{../../figures/ece1236/deck5smithChartsAndImpedenceTxFig7}{Hand sketched Smith chart.}{fig:deck5smithChartsAndImpedenceTx:deck5smithChartsAndImpedenceTxFig7}{0.4}

A matlab produced blank Smith chart can be found in \cref{fig:smithchart:smithchartFig1}.

\mathImageFigure{../../figures/ece1236/smithchartFig1}{Blank Smith chart.}{fig:smithchart:smithchartFig1}{0.5}{smith:run.m}

\makeexample{Perform a transformation along a lossless line.}{example:uwavesDeck5SmithChartCore:680}{

%\cref{fig:deck5smithChartsAndImpedenceTx:deck5smithChartsAndImpedenceTxFig8}.
\imageFigure{../../figures/ece1236/deck5smithChartsAndImpedenceTxFig8}{Impedance transformation along lossless line.}{fig:deck5smithChartsAndImpedenceTx:deck5smithChartsAndImpedenceTxFig8}{0.2}

Given 

\begin{dmath}\label{eqn:uwavesDeck5SmithChartCore:700}
\overbar{Z} = \frac{1 + \Gamma}{1 - \Gamma},
\end{dmath}

\begin{dmath}\label{eqn:uwavesDeck5SmithChartCore:720}
\Gamma = \Gamma_\txtL e^{-2 j \beta l},
\end{dmath}

and

\begin{dmath}\label{eqn:uwavesDeck5SmithChartCore:740}
\Gamma_\txtL = \Abs{\Gamma_\txtL} e^{j \Theta_\txtL }
\end{dmath}

The total reflection coefficient is

\begin{dmath}\label{eqn:uwavesDeck5SmithChartCore:760}
\Gamma = \Abs{\Gamma_\txtL} e^{j (\Theta_\txtL - 2 \beta l) }
\end{dmath}

If \( \Gamma_\txtL = \Abs{\Gamma_\txtL} e^{j \Theta_\txtL } \) is plotted on the Smith chart, then in order to move towards the generator, a subtraction from \( \Theta_\txtL \) of \( 2 \beta l \) is required.

Some worked examples that demonstrate this can be found in \cref{fig:smithChartSlides:smithChartSlidesFig1}, \cref{fig:smithChartSlides:smithChartSlidesFig2}, and \cref{fig:smithChartSlides:smithChartSlidesFig3}.

\imageFigure{../../figures/ece1236/smithChartSlidesFig1}{Mapping an impedance value onto a Smith chart.}{fig:smithChartSlides:smithChartSlidesFig1}{0.5}
\imageFigure{../../figures/ece1236/smithChartSlidesFig2}{Moving on the Smith chart towards the generator.}{fig:smithChartSlides:smithChartSlidesFig2}{0.5}
\imageFigure{../../figures/ece1236/smithChartSlidesFig3}{Moving on the Smith chart.}{fig:smithChartSlides:smithChartSlidesFig3}{0.5}

%The class slides do this for a few loads.
} % example

\paragraph{Single stub tuning.}

Referring to \cref{fig:smithChartSlides:smithChartSlidesFig4}, the procedure for single stub tuning is

\imageFigure{../../figures/ece1236/smithChartSlidesFig4}{Single stub tuning example}{fig:smithChartSlides:smithChartSlidesFig4}{0.5}

\begin{enumerate}
\item Plot the load on the Smith Chart.
\item Trace the constant VSWR circle. (blue).
\item Move toward the generator until the constant resistance=1 circle is reached (red).  This determines the distance \(d\).
\item Now the input impedance is of the form \(Z_\txtA = 1 + j X\).
\item We now have to use the stub to cancel out the \( j X \) and make \( Z_{\textrm{in}} = 1 \) (matched).
\item This can be done on the Smith Chart. If \( X>0 \) then we need a capacitive stub (open). If \( X<0 \) then we need an inductive stub (shorted).
\item Say we need a capacitive stub (open): Start from the position of the open. Now the constant VSWR circle is the exterior unit
circle. Move toward the generator until you hit negative \( X \). This determines the length of the stub \( l \).
\end{enumerate}

Notes:
\begin{enumerate}[(a)]
\item In step (3) there are two points where the R=1 circle is intersected . Usually we chose the shortest one 
\item By adding multiples of half-wavelength lengths to either \(d\) or \(l\) an infinite number of solutions can be constructed.
\end{enumerate}


%%%\section{Feb 3}
%%%
%%%F1
%%%
%%%%Z_\txtL = 20 + 25 j
%%%%
%%%%Z_\txtA = Z_\txtL/Z_0 = 0.2 + 0.5 j
%%%
%%%F2
