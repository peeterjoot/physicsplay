%
% Copyright © 2016 Peeter Joot.  All Rights Reserved.
% Licenced as described in the file LICENSE under the root directory of this GIT repository.
%

\makeproblem{Find the Smith chart circle equations}{problem:uwavesDeck5SmithChartCore:1}{
Prove \cref{eqn:uwavesDeck5SmithChartCore:440}.
} % problem

\makeanswer{problem:uwavesDeck5SmithChartCore:1}{

We can write

\begin{dmath}\label{eqn:uwavesDeck5SmithChartCore:460}
(1 - \Gamma_r)^2 + \Gamma_i^2 = \frac{2 \Gamma_i }{\overbar{X}_\txtL},
\end{dmath}

or

\begin{dmath}\label{eqn:uwavesDeck5SmithChartCore:480}
(1 - \Gamma_r)^2 + \lr{ \Gamma_i - \inv{\overbar{X}_\txtL} }^2 = \inv{\lr{\overbar{X}_\txtL}^2},
\end{dmath}

which is one of the circular equations.  For the other, putting the \( \Gamma_r, \Gamma_i \) terms in the numerator, we have

\begin{dmath}\label{eqn:uwavesDeck5SmithChartCore:500}
\frac{1 - \Gamma_r^2  - \Gamma_i^2 }{\overbar{\Gamma}_\txtL}
=
(1 - \Gamma_r)^2 + \Gamma_i^2 
=
1 - 2 \Gamma_r + \Gamma_r^2 + \Gamma_i^2,
\end{dmath}

or
\begin{dmath}\label{eqn:uwavesDeck5SmithChartCore:520}
\Gamma_r^2 \lr{ 1 + \inv{\overbar{\Gamma}_\txtL} } - 2 \Gamma_r + \Gamma_i^2 \lr{ 1 + \inv{\overbar{\Gamma}_\txtL} } 
=
\inv{\overbar{\Gamma}_\txtL} - 1.
\end{dmath}

Dividing through by \( 1 + \ifrac{1}{\overbar{\Gamma}_\txtL} = (\overbar{\Gamma}_\txtL + 1)/\overbar{\Gamma}_\txtL \), we have

\begin{dmath}\label{eqn:uwavesDeck5SmithChartCore:540}
\Gamma_r^2 - 2 \Gamma_r \frac{ \overbar{\Gamma}_\txtL }{\overbar{\Gamma}_\txtL + 1} + \Gamma_i^2 
=
\frac{1 - \overbar{\Gamma}_\txtL}{\overbar{\Gamma}_\txtL} \frac{ \overbar{\Gamma}_\txtL }{\overbar{\Gamma}_\txtL + 1} 
=
\frac{1 - \overbar{\Gamma}_\txtL}{ \overbar{\Gamma}_\txtL + 1},
\end{dmath}

or
\begin{dmath}\label{eqn:uwavesDeck5SmithChartCore:560}
\lr{ \Gamma_r - \frac{ \overbar{\Gamma}_\txtL }{\overbar{\Gamma}_\txtL + 1} }^2 + \Gamma_i^2
=
\frac{1 - \overbar{\Gamma}_\txtL}{ \overbar{\Gamma}_\txtL + 1} + \lr{ \frac{ \overbar{\Gamma}_\txtL }{\overbar{\Gamma}_\txtL + 1} }^2
=
\frac{ 1 - \overbar{\Gamma}_\txtL^2 + \overbar{\Gamma}_\txtL^2 }{\lr{\overbar{\Gamma}_\txtL + 1}^2}
=
\frac{ 1 }{\lr{\overbar{\Gamma}_\txtL + 1}^2}.
\end{dmath}

} % answer
