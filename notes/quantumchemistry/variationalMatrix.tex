%
% Copyright � 2016 Peeter Joot.  All Rights Reserved.
% Licenced as described in the file LICENSE under the root directory of this GIT repository.
%
%{
\newcommand{\authorname}{Peeter Joot}
\newcommand{\email}{peeterjoot@protonmail.com}
\newcommand{\basename}{FIXMEbasenameUndefined}
\newcommand{\dirname}{notes/FIXMEdirnameUndefined/}

\renewcommand{\basename}{variationalMatrix}
\renewcommand{\dirname}{notes/phy1520/}
%\newcommand{\dateintitle}{}
%\newcommand{\keywords}{}

\newcommand{\authorname}{Peeter Joot}
\newcommand{\onlineurl}{http://sites.google.com/site/peeterjoot2/math2013/\basename.pdf}
\newcommand{\sourcepath}{\dirname\basename.tex}
\newcommand{\generatetitle}[1]{\chapter{#1}}

\newcommand{\vcsinfo}{%
\section*{}
\noindent{\color{DarkOliveGreen}{\rule{\linewidth}{0.1mm}}}
\paragraph{Document version}
%\paragraph{\color{Maroon}{Document version}}
{
\small
\begin{itemize}
\item Available online at:\\ 
\href{\onlineurl}{\onlineurl}
\item Git Repository: \input{./.revinfo/gitRepo.tex}
\item Source: \sourcepath
\item last commit: \input{./.revinfo/gitCommitString.tex}
\item commit date: \input{./.revinfo/gitCommitDate.tex}
\end{itemize}
}
}

%\PassOptionsToPackage{dvipsnames,svgnames}{xcolor}
\PassOptionsToPackage{square,numbers}{natbib}
\documentclass{scrreprt}

\usepackage[left=2cm,right=2cm]{geometry}
\usepackage[svgnames]{xcolor}
\usepackage{peeters_layout}

\usepackage{natbib}

\usepackage[
colorlinks=true,
bookmarks=false,
pdfauthor={\authorname, \email},
backref 
]{hyperref}

% http://tex.stackexchange.com/questions/75773/how-to-reference-problems-by-the-text-label-in-an-exercise-envioronment
\usepackage[english]{cleveref}
\crefname{Exercise}{exercise}{exercises}
\Crefname{Exercise}{Exercise}{Exercises}

\RequirePackage{titlesec}
\RequirePackage{ifthen}

% http://stackoverflow.com/questions/4932910/date-in-the-tabular-environment
\makeatletter
\let\insertdate\@date
\makeatother

\titleformat{\chapter}[display]
{\bfseries\Large}
{\color{DarkSlateGrey}\filleft \authorname
\ifthenelse{\isundefined{\studentnumber}}{}{\\ \studentnumber}
\ifthenelse{\isundefined{\email}}{}{\\ \email}
\ifthenelse{\isundefined{\dateintitle}}{}{\\ \insertdate}
%\ifthenelse{\isundefined{\coursename}}{}{\\ \coursename} % put in title instead.
}
{4ex}
{\color{DarkOliveGreen}{\titlerule}\color{Maroon}
\vspace{2ex}%
\filright}
[\vspace{2ex}%
\color{DarkOliveGreen}\titlerule
]

\newcommand{\beginArtWithToc}[0]{\begin{document}\tableofcontents}
\newcommand{\beginArtNoToc}[0]{\begin{document}}
\newcommand{\EndNoBibArticle}[0]{\end{document}}
\newcommand{\EndArticle}[0]{\bibliography{Bibliography}\bibliographystyle{plainnat}\end{document}}

% 
%\newcommand{\citep}[1]{\cite{#1}}

\colorSectionsForArticle



\usepackage{peeters_layout_exercise}
\usepackage{peeters_braket}
\usepackage{peeters_figures}
\usepackage{siunitx}

\beginArtNoToc

\generatetitle{Variational principle with two by two symmetric matrix}
%\chapter{Variational principle with two by two symmetric matrix}
%\label{chap:variationalMatrix}
% \citep{sakurai2014modern} pr X.Y
% \citep{pozar2009microwave}
% \citep{qftLectureNotes}

I pulled \citep{szabo1989modern}, one of too many lonely Dover books, off my shelf and started reading the review chapter.  It posed the following question, which I thought had an interesting subquestion.

\makeproblem{Variational principle with two by two symmetric matrix.}{problem:variationalMatrix:1}{

Consider a \( 2 \times 2 \) real symmetric matrix operator \(\BO \), with an arbitrary normalized trial vector

\begin{dmath}\label{eqn:variationalMatrix:20}
\Bc =
\begin{bmatrix}
\cos\theta \\
\sin\theta
\end{bmatrix}.
\end{dmath}

The variational principle requires that minimum value of \( \omega(\theta) = \Bc^\dagger \BO \Bc \) is greater than or equal to the lowest eigenvalue.
\makesubproblem{}{problem:variationalMatrix:1:a}
If that minimum value occurs at \( \omega(\theta_0) \), show that this is exactly equal to the lowest eigenvalue.
\makesubproblem{}{problem:variationalMatrix:1:b}
Explain why this is should have been anticipated.
} % problem

\makeanswer{problem:variationalMatrix:1}{

\makeSubAnswer{}{problem:variationalMatrix:1:a}
%\paragraph{Finding the minimum.}

If the operator representation is

\begin{dmath}\label{eqn:variationalMatrix:40}
\BO =
\begin{bmatrix}
a & b \\
b & d
\end{bmatrix},
\end{dmath}

then the variational product is

\begin{dmath}\label{eqn:variationalMatrix:80}
\omega(\theta)
=
\begin{bmatrix}
\cos\theta & \sin\theta
\end{bmatrix}
\begin{bmatrix}
a & b \\
b & d
\end{bmatrix}
\begin{bmatrix}
\cos\theta \\
\sin\theta
\end{bmatrix}
=
\begin{bmatrix}
\cos\theta & \sin\theta
\end{bmatrix}
\begin{bmatrix}
a \cos\theta + b \sin\theta \\
b \cos\theta + d \sin\theta
\end{bmatrix}
=
a \cos^2\theta + 2 b \sin\theta \cos\theta
+ d \sin^2\theta
=
a \cos^2\theta + b \sin( 2 \theta )
+ d \sin^2\theta.
\end{dmath}

The minimum is given by

\begin{dmath}\label{eqn:variationalMatrix:60}
0
=
\frac{d\omega}{d\theta}
=
-2 a \sin\theta \cos\theta + 2 b \cos( 2 \theta )
+ 2 d \sin\theta \cos\theta
=
2 b \cos( 2 \theta )
+ (d -a)\sin( 2 \theta )
%\begin{bmatrix}
%-\sin\theta & \cos\theta
%\end{bmatrix}
%\begin{bmatrix}
%a & b \\
%b & d
%\end{bmatrix}
%\begin{bmatrix}
%\cos\theta \\
%\sin\theta
%\end{bmatrix}
%+
%\begin{bmatrix}
%\cos\theta & \sin\theta
%\end{bmatrix}
%\begin{bmatrix}
%a & b \\
%b & d
%\end{bmatrix}
%\begin{bmatrix}
%-\sin\theta \\
%\cos\theta
%\end{bmatrix}
,
\end{dmath}

so the extreme values will be found at

\begin{dmath}\label{eqn:variationalMatrix:100}
\tan(2\theta_0) = \frac{2 b}{a - d}.
\end{dmath}

Solving for \( \cos(2\theta_0) \), with \( \alpha = 2b/(a-d) \), we have

\begin{dmath}\label{eqn:variationalMatrix:120}
1 - \cos^2(2\theta) = \alpha^2 \cos^2(2 \theta),
\end{dmath}

or

\begin{dmath}\label{eqn:variationalMatrix:140}
\cos^2(2\theta_0)
= \frac{1}{1 + \alpha^2}
= \frac{1}{1 + 4 b^2/(a-d)^2 }
= \frac{(a-d)^2}{(a-d)^2 + 4 b^2 }.
\end{dmath}

So,

\begin{equation}\label{eqn:variationalMatrix:200}
\begin{aligned}
\cos(2 \theta_0) &= \frac{ \pm (a-d) }{\sqrt{ (a-d)^2 + 4 b^2 }} \\
\sin(2 \theta_0) &= \frac{ \pm 2 b }{\sqrt{ (a-d)^2 + 4 b^2 }},
\end{aligned}
\end{equation}

Substituting this back into \( \omega(\theta_0) \) is a bit tedious.
I did it once on paper, then confirmed with Mathematica (quantumchemistry/twoByTwoSymmetricVariation.nb).  The end result is

\begin{dmath}\label{eqn:variationalMatrix:160}
\omega(\theta_0)
=
\inv{2} \lr{ a + d \pm \sqrt{ (a-d)^2 + 4 b^2 } }.
\end{dmath}

The eigenvalues of the operator are given by

\begin{dmath}\label{eqn:variationalMatrix:220}
0
= (a-\lambda)(d-\lambda) - b^2
= \lambda^2 - (a+d) \lambda + a d - b^2
= \lr{\lambda - \frac{a+d}{2}}^2 -\lr{ \frac{a+d}{2}}^2 + a d - b^2
= \lr{\lambda - \frac{a+d}{2}}^2 - \inv{4} \lr{ (a-d)^2 + 4 b^2 },
\end{dmath}

so the eigenvalues are exactly the values \cref{eqn:variationalMatrix:160} as stated by the problem statement.

\makeSubAnswer{}{problem:variationalMatrix:1:b}

%\paragraph{Why should this have been anticipated?}

If the eigenvectors are \( \Be_1, \Be_2 \), the operator can be diagonalized as

\begin{dmath}\label{eqn:variationalMatrix:240}
\BO = U D U^\T,
\end{dmath}

where \( U = \begin{bmatrix} \Be_1 & \Be_2 \end{bmatrix} \), and \( D \) has the eigenvalues along the diagonal.  The energy function \( \omega \) can now be written

\begin{dmath}\label{eqn:variationalMatrix:260}
\omega
= \Bc^\T U  D U^\T \Bc
= (U^\T \Bc)^\T  D U^\T \Bc.
\end{dmath}

We can show that the transformed vector \( U^\T \Bc \) is still a unit vector

\begin{dmath}\label{eqn:variationalMatrix:280}
U^\T \Bc
=
\begin{bmatrix}
\Be_1^\T \\
\Be_2^\T \\
\end{bmatrix}
\Bc
=
\begin{bmatrix}
\Be_1^\T \Bc \\
\Be_2^\T \Bc \\
\end{bmatrix},
\end{dmath}

so
\begin{dmath}\label{eqn:variationalMatrix:300}
\Abs{
U^\T \Bc
}^2
=
\Bc^\T \Be_1
\Be_1^\T \Bc
+
\Bc^\T \Be_2
\Be_2^\T \Bc
=
\Bc^\T \lr{ \Be_1 \Be_1^\T
+
\Be_2
\Be_2^\T } \Bc
=
\Bc^\T \Bc
= 1,
\end{dmath}

so the transformed vector can be written as

\begin{dmath}\label{eqn:variationalMatrix:320}
U^\T \Bc =
\begin{bmatrix}
\cos\phi \\
\sin\phi
\end{bmatrix},
\end{dmath}

for some \( \phi \).  With such a representation we have
\begin{dmath}\label{eqn:variationalMatrix:340}
\omega
=
\begin{bmatrix}
\cos\phi & \sin\phi
\end{bmatrix}
\begin{bmatrix}
\lambda_1 & 0 \\
0 & \lambda_2
\end{bmatrix}
\begin{bmatrix}
\cos\phi \\
\sin\phi
\end{bmatrix}
=
\begin{bmatrix}
\cos\phi & \sin\phi
\end{bmatrix}
\begin{bmatrix}
\lambda_1 \cos\phi \\
\lambda_2 \sin\phi
\end{bmatrix}
=
\lambda_1 \cos^2\phi + \lambda_2 \sin^2\phi.
\end{dmath}

This has it's minimums where \( 0 = \sin(2 \phi)( \lambda_2 - \lambda_1 ) \).  For the non-degenerate case, two zeros at \( \phi = n \pi/2 \) for integral \( n \).  For \( \phi = 0, \pi/2 \), we have

\begin{dmath}\label{eqn:variationalMatrix:360}
\Bc =
\begin{bmatrix}
1 \\
0
\end{bmatrix},
\begin{bmatrix}
0 \\
1
\end{bmatrix}.
\end{dmath}

We see that the extreme values of \( \omega \) occur when the trial vectors \( \Bc \) are eigenvectors of the operator.
} % answer

%
%}
\EndArticle
%\EndNoBibArticle
