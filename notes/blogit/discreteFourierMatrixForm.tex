%
% Copyright � 2014 Peeter Joot.  All Rights Reserved.
% Licenced as described in the file LICENSE under the root directory of this GIT repository.
%
\newcommand{\authorname}{Peeter Joot}
\newcommand{\email}{peeterjoot@protonmail.com}
\newcommand{\basename}{FIXMEbasenameUndefined}
\newcommand{\dirname}{notes/FIXMEdirnameUndefined/}

\renewcommand{\basename}{discreteFourierMatrixForm}
\renewcommand{\dirname}{notes/ece1254/}
%\newcommand{\dateintitle}{}
%\newcommand{\keywords}{}

\newcommand{\authorname}{Peeter Joot}
\newcommand{\onlineurl}{http://sites.google.com/site/peeterjoot2/math2013/\basename.pdf}
\newcommand{\sourcepath}{\dirname\basename.tex}
\newcommand{\generatetitle}[1]{\chapter{#1}}

\newcommand{\vcsinfo}{%
\section*{}
\noindent{\color{DarkOliveGreen}{\rule{\linewidth}{0.1mm}}}
\paragraph{Document version}
%\paragraph{\color{Maroon}{Document version}}
{
\small
\begin{itemize}
\item Available online at:\\ 
\href{\onlineurl}{\onlineurl}
\item Git Repository: \input{./.revinfo/gitRepo.tex}
\item Source: \sourcepath
\item last commit: \input{./.revinfo/gitCommitString.tex}
\item commit date: \input{./.revinfo/gitCommitDate.tex}
\end{itemize}
}
}

%\PassOptionsToPackage{dvipsnames,svgnames}{xcolor}
\PassOptionsToPackage{square,numbers}{natbib}
\documentclass{scrreprt}

\usepackage[left=2cm,right=2cm]{geometry}
\usepackage[svgnames]{xcolor}
\usepackage{peeters_layout}

\usepackage{natbib}

\usepackage[
colorlinks=true,
bookmarks=false,
pdfauthor={\authorname, \email},
backref 
]{hyperref}

% http://tex.stackexchange.com/questions/75773/how-to-reference-problems-by-the-text-label-in-an-exercise-envioronment
\usepackage[english]{cleveref}
\crefname{Exercise}{exercise}{exercises}
\Crefname{Exercise}{Exercise}{Exercises}

\RequirePackage{titlesec}
\RequirePackage{ifthen}

% http://stackoverflow.com/questions/4932910/date-in-the-tabular-environment
\makeatletter
\let\insertdate\@date
\makeatother

\titleformat{\chapter}[display]
{\bfseries\Large}
{\color{DarkSlateGrey}\filleft \authorname
\ifthenelse{\isundefined{\studentnumber}}{}{\\ \studentnumber}
\ifthenelse{\isundefined{\email}}{}{\\ \email}
\ifthenelse{\isundefined{\dateintitle}}{}{\\ \insertdate}
%\ifthenelse{\isundefined{\coursename}}{}{\\ \coursename} % put in title instead.
}
{4ex}
{\color{DarkOliveGreen}{\titlerule}\color{Maroon}
\vspace{2ex}%
\filright}
[\vspace{2ex}%
\color{DarkOliveGreen}\titlerule
]

\newcommand{\beginArtWithToc}[0]{\begin{document}\tableofcontents}
\newcommand{\beginArtNoToc}[0]{\begin{document}}
\newcommand{\EndNoBibArticle}[0]{\end{document}}
\newcommand{\EndArticle}[0]{\bibliography{Bibliography}\bibliographystyle{plainnat}\end{document}}

% 
%\newcommand{\citep}[1]{\cite{#1}}

\colorSectionsForArticle



\beginArtNoToc

\generatetitle{Matrix form for discrete time Fourier transform}
%\chapter{Matrix form for discrete time Fourier transform}
%\label{chap:discreteFourierMatrixForm}

The discrete time Fourier transform has been seen to have the form

\begin{subequations}
\begin{equation}\label{eqn:discreteFourierMatrixForm:20}
x_k = \sum_{n = -N}^N X_n e^{ 2 \pi j n k /(2 N + 1)} 
\end{equation}
\begin{equation}\label{eqn:discreteFourierMatrixForm:40}
X_n = \inv{2 N + 1} \sum_{k = -N}^N x_k e^{- 2 \pi j n k /(2 N + 1)}.
\end{equation}
\end{subequations}

A matrix representation of this form is desired.  Let

\begin{subequations}
\begin{equation}\label{eqn:discreteFourier:60}
\Bx = 
\begin{bmatrix}
x_N \\
\vdots \\
x_0 \\
\vdots \\
x_{-N}
\end{bmatrix}
\end{equation}
\begin{equation}\label{eqn:discreteFourier:80}
\BX = 
\begin{bmatrix}
X_N \\
\vdots \\
X_0 \\
\vdots \\
X_{-N}
\end{bmatrix}
\end{equation}
\end{subequations}

\Cref{eqn:discreteFourier:60} written out in full is
\begin{equation}\label{eqn:discreteFourierMatrixForm:100}
\begin{aligned}
x_k 
&= X_N e^{ 2 \pi j N k /(2 N + 1)}  \\
&+ X_{N-1} e^{ 2 \pi j \lr{N-1} k /(2 N + 1)}  \\
&+ \cdots \\
&+ X_0  \\
&+ \cdots \\
&+ X_{1-N} e^{ -2 \pi j \lr{N-1} k /(2 N + 1)}  \\
&+ X_{-N} e^{ -2 \pi j N k /(2 N + 1)}.
\end{aligned}
\end{equation}

With \( \alpha = e^{ 2 \pi j /(2 N + 1) } \) the matrix form is

\begin{equation}\label{eqn:discreteFourierMatrixForm:120}
\Bx = 
\begin{bmatrix}
 \alpha^{ N N } &  \alpha^{ \lr{N-1} N }  & \cdots & 1 & \cdots &  \alpha^{ -\lr{N-1} N }  &  \alpha^{ -N N } \\
 \alpha^{ N \lr{N-1} } &  \alpha^{ \lr{N-1} \lr{N-1} }  & \cdots & 1 & \cdots &  \alpha^{ -\lr{N-1} \lr{N-1} }  &  \alpha^{ -N \lr{N-1} } \\
 \vdots              &  \vdots                      & \vdots      & \vdots & \vdots      &  \vdots                    &  \vdots               \\
 1              &  1                      & 1      & 1 & 1      &  1                    &  1               \\
 \vdots              &  \vdots                      & \vdots      & \vdots & \vdots      &  \vdots                    &  \vdots               \\
 \alpha^{ -N \lr{N-1} } &  \alpha^{ -\lr{N-1} \lr{N-1} }  & \cdots & 1 & \cdots &  \alpha^{ {N-1} \lr{N-1} }  &  \alpha^{ N \lr{N-1} } \\
 \alpha^{ -N N } &  \alpha^{ -N N }  & \cdots & 1 & \cdots &  \alpha^{ \lr{N-1} N }  &  \alpha^{ N N } \\
\end{bmatrix}
\BX
\end{equation}

Similarily, from \cref{eqn:discreteFourierMatrixForm:40}, the inverse relation expands out to 

\begin{equation}\label{eqn:discreteFourierMatrixForm:140}
\begin{aligned}
( 2 N + 1 ) X_n 
&= x_N e^{- 2 \pi j n N /(2 N + 1)}  \\
&+ x_{N-1} e^{- 2 \pi j n \lr{N-1}/(2 N + 1)}  \\
&\cdots  \\
&+ x_0 \\
&\cdots  \\
&+ x_{1-N} e^{ 2 \pi j n \lr{ N-1 }/(2 N + 1)}  \\
&+ x_{-N} e^{ 2 \pi j n N/(2 N + 1)},
\end{aligned}
\end{equation}

with a matrix form of

\begin{equation}\label{eqn:discreteFourierMatrixForm:160}
( 2 N + 1 ) \BX =
\begin{bmatrix}
\alpha^{- N N } & \alpha^{- N \lr{N-1}} &\cdots & 1 &\cdots & \alpha^{ N \lr{ N-1 }} & \alpha^{ N N} \\
\alpha^{- \lr{N-1} N } & \alpha^{- \lr{N-1} \lr{N-1}} &\cdots & 1 &\cdots & \alpha^{ \lr{N-1} \lr{ N-1 }} & \alpha^{ \lr{N-1} N} \\
\vdots               & \vdots                     &  \vdots    & \vdots & \vdots     & \vdots                      &  \vdots            \\
1               & 1                     &  1    & 1 & 1     & 1                      &  1            \\
\vdots               & \vdots                     &  \vdots    & \vdots & \vdots     & \vdots                      &  \vdots            \\
\alpha^{\lr{N-1} N } & \alpha^{ \lr{N-1} \lr{N-1}} &\cdots & 1 &\cdots & \alpha^{ -\lr{N-1} \lr{ N-1 }} & \alpha^{ -\lr{N-1} N} \\
\alpha^{ N N } & \alpha^{ N \lr{N-1}} &\cdots & 1 &\cdots & \alpha^{ -N \lr{ N-1 }} & \alpha^{ -N N} \\
\end{bmatrix}
\end{equation}

Lettting

\begin{equation}\label{eqn:discreteFourierMatrixForm:220}
\BF = 
\begin{bmatrix}
 \alpha^{ N N } &  \alpha^{ \lr{N-1} N }  & \cdots & 1 & \cdots &  \alpha^{ -\lr{N-1} N }  &  \alpha^{ -N N } \\
 \alpha^{ N \lr{N-1} } &  \alpha^{ \lr{N-1} \lr{N-1} }  & \cdots & 1 & \cdots &  \alpha^{ -\lr{N-1} \lr{N-1} }  &  \alpha^{ -N \lr{N-1} } \\
 \vdots              &  \vdots                      & \vdots      & \vdots & \vdots      &  \vdots                    &  \vdots               \\
 1              &  1                      & 1      & 1 & 1      &  1                    &  1               \\
 \vdots              &  \vdots                      & \vdots      & \vdots & \vdots      &  \vdots                    &  \vdots               \\
 \alpha^{ -N \lr{N-1} } &  \alpha^{ -\lr{N-1} \lr{N-1} }  & \cdots & 1 & \cdots &  \alpha^{ {N-1} \lr{N-1} }  &  \alpha^{ N \lr{N-1} } \\
 \alpha^{ -N N } &  \alpha^{ -N N }  & \cdots & 1 & \cdots &  \alpha^{ \lr{N-1} N }  &  \alpha^{ N N } \\
\end{bmatrix},
\end{equation}

the discrete transform pair has the following compactly matrix representation

\begin{subequations}
\begin{equation}\label{eqn:discreteFourierMatrixForm:180}
\Bx = \BF \BX
\end{equation}
\begin{equation}\label{eqn:discreteFourierMatrixForm:200}
\BX = \inv{2 N + 1} \overbar{\BF} \Bx,
\end{equation}
\end{subequations}

where \( \overbar{\BF} \) is the complex conjugate of \( \BF \).

%\EndArticle
\EndNoBibArticle
