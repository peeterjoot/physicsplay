%
% Copyright � 2015 Peeter Joot.  All Rights Reserved.
% Licenced as described in the file LICENSE under the root directory of this GIT repository.
%
\documentclass[]{eliblog}

\usepackage{amsmath}
\usepackage{mathpazo}

%
% shorthand for bold symbols, convenient for vectors and matrices
%
\newcommand{\Ba}[0]{\mathbf{a}}
\newcommand{\Bb}[0]{\mathbf{b}}
\newcommand{\Bc}[0]{\mathbf{c}}
\newcommand{\Bd}[0]{\mathbf{d}}
\newcommand{\Be}[0]{\mathbf{e}}
\newcommand{\Bf}[0]{\mathbf{f}}
\newcommand{\Bg}[0]{\mathbf{g}}
\newcommand{\Bh}[0]{\mathbf{h}}
\newcommand{\Bi}[0]{\mathbf{i}}
\newcommand{\Bj}[0]{\mathbf{j}}
\newcommand{\Bk}[0]{\mathbf{k}}
\newcommand{\Bl}[0]{\mathbf{l}}
\newcommand{\Bm}[0]{\mathbf{m}}
\newcommand{\Bn}[0]{\mathbf{n}}
\newcommand{\Bo}[0]{\mathbf{o}}
\newcommand{\Bp}[0]{\mathbf{p}}
\newcommand{\Bq}[0]{\mathbf{q}}
\newcommand{\Br}[0]{\mathbf{r}}
\newcommand{\Bs}[0]{\mathbf{s}}
\newcommand{\Bt}[0]{\mathbf{t}}
\newcommand{\Bu}[0]{\mathbf{u}}
\newcommand{\Bv}[0]{\mathbf{v}}
\newcommand{\Bw}[0]{\mathbf{w}}
\newcommand{\Bx}[0]{\mathbf{x}}
\newcommand{\By}[0]{\mathbf{y}}
\newcommand{\Bz}[0]{\mathbf{z}}
\newcommand{\BA}[0]{\mathbf{A}}
\newcommand{\BB}[0]{\mathbf{B}}
\newcommand{\BC}[0]{\mathbf{C}}
\newcommand{\BD}[0]{\mathbf{D}}
\newcommand{\BE}[0]{\mathbf{E}}
\newcommand{\BF}[0]{\mathbf{F}}
\newcommand{\BG}[0]{\mathbf{G}}
\newcommand{\BH}[0]{\mathbf{H}}
\newcommand{\BI}[0]{\mathbf{I}}
\newcommand{\BJ}[0]{\mathbf{J}}
\newcommand{\BK}[0]{\mathbf{K}}
\newcommand{\BL}[0]{\mathbf{L}}
\newcommand{\BM}[0]{\mathbf{M}}
\newcommand{\BN}[0]{\mathbf{N}}
\newcommand{\BO}[0]{\mathbf{O}}
\newcommand{\BP}[0]{\mathbf{P}}
\newcommand{\BQ}[0]{\mathbf{Q}}
\newcommand{\BR}[0]{\mathbf{R}}
\newcommand{\BS}[0]{\mathbf{S}}
\newcommand{\BT}[0]{\mathbf{T}}
\newcommand{\BU}[0]{\mathbf{U}}
\newcommand{\BV}[0]{\mathbf{V}}
\newcommand{\BW}[0]{\mathbf{W}}
\newcommand{\BX}[0]{\mathbf{X}}
\newcommand{\BY}[0]{\mathbf{Y}}
\newcommand{\BZ}[0]{\mathbf{Z}}

\newcommand{\Bzero}[0]{\mathbf{0}}
\newcommand{\Btheta}[0]{\boldsymbol{\theta}}
\newcommand{\Btau}[0]{\boldsymbol{\tau}}
\newcommand{\Bomega}[0]{\boldsymbol{\omega}}

%
% shorthand for unit vectors
%
\newcommand{\acap}[0]{\hat{\Ba}}
\newcommand{\bcap}[0]{\hat{\Bb}}
\newcommand{\ccap}[0]{\hat{\Bc}}
\newcommand{\dcap}[0]{\hat{\Bd}}
\newcommand{\ecap}[0]{\hat{\Be}}
\newcommand{\fcap}[0]{\hat{\Bf}}
\newcommand{\gcap}[0]{\hat{\Bg}}
\newcommand{\hcap}[0]{\hat{\Bh}}
\newcommand{\icap}[0]{\hat{\Bi}}
\newcommand{\jcap}[0]{\hat{\Bj}}
\newcommand{\kcap}[0]{\hat{\Bk}}
\newcommand{\lcap}[0]{\hat{\Bl}}
\newcommand{\mcap}[0]{\hat{\Bm}}
\newcommand{\ncap}[0]{\hat{\Bn}}
\newcommand{\ocap}[0]{\hat{\Bo}}
\newcommand{\pcap}[0]{\hat{\Bp}}
\newcommand{\qcap}[0]{\hat{\Bq}}
\newcommand{\rcap}[0]{\hat{\Br}}
\newcommand{\scap}[0]{\hat{\Bs}}
\newcommand{\tcap}[0]{\hat{\Bt}}
\newcommand{\ucap}[0]{\hat{\Bu}}
\newcommand{\vcap}[0]{\hat{\Bv}}
\newcommand{\wcap}[0]{\hat{\Bw}}
\newcommand{\xcap}[0]{\hat{\Bx}}
\newcommand{\ycap}[0]{\hat{\By}}
\newcommand{\zcap}[0]{\hat{\Bz}}
\newcommand{\thetacap}[0]{\hat{\Btheta}}

%
% to write R^n and C^n in a distinguishable fashion.  Perhaps change this
% to the double lined characters upon figuring out how to do so.
%
\newcommand{\C}[1]{$\mathbb{C}^{#1}$}
\newcommand{\R}[1]{$\mathbb{R}^{#1}$}

%
% various generally useful helpers
%

% derivative of #1 wrt. #2:
\newcommand{\D}[2] {\frac {d#2} {d#1}}

\newcommand{\inv}[1]{\frac{1}{#1}}
\newcommand{\cross}[0]{\times}

\newcommand{\abs}[1]{\lvert{#1}\rvert}
\newcommand{\norm}[1]{\lVert{#1}\rVert}
\newcommand{\innerprod}[2]{\langle{#1}, {#2}\rangle}
\newcommand{\dotprod}[2]{{#1} \cdot {#2}}
\newcommand{\bdotprod}[2]{\left({#1} \cdot {#2}\right)}
\newcommand{\crossprod}[2]{{#1} \cross {#2}}
\newcommand{\tripleprod}[3]{\dotprod{\left(\crossprod{#1}{#2}\right)}{#3}}

\DeclareMathOperator{\Proj}{Proj}
\DeclareMathOperator{\Span}{span}
\DeclareMathOperator{\Sgn}{sgn}
\DeclareMathOperator{\Area}{Area}
\DeclareMathOperator{\Volume}{Volume}

%
% A few miscellaneous things specific to this document
%
\newcommand{\crossop}[1]{\crossprod{#1}{}}

% R2 vector.
\newcommand{\VectorTwo}[2]{
\begin{bmatrix}
 {#1} \\
 {#2}
\end{bmatrix}
}

\newcommand{\VectorN}[1]{
\begin{bmatrix}
{#1}_1 \\
{#1}_2 \\
\vdots \\
{#1}_N \\
\end{bmatrix}
}

\newcommand{\DETuvij}[4]{
\begin{vmatrix}
 {#1}_{#3} & {#1}_{#4} \\
 {#2}_{#3} & {#2}_{#4}
\end{vmatrix}
}

\newcommand{\DETuvwijk}[6]{
\begin{vmatrix}
 {#1}_{#4} & {#1}_{#5} & {#1}_{#6} \\
 {#2}_{#4} & {#2}_{#5} & {#2}_{#6} \\
 {#3}_{#4} & {#3}_{#5} & {#3}_{#6}
\end{vmatrix}
}

\newcommand{\DETuvwxijkl}[8]{
\begin{vmatrix}
 {#1}_{#5} & {#1}_{#6} & {#1}_{#7} & {#1}_{#8} \\
 {#2}_{#5} & {#2}_{#6} & {#2}_{#7} & {#2}_{#8} \\
 {#3}_{#5} & {#3}_{#6} & {#3}_{#7} & {#3}_{#8} \\
 {#4}_{#5} & {#4}_{#6} & {#4}_{#7} & {#4}_{#8} \\
\end{vmatrix}
}

%\newcommand{\DETuvwxyijklm}[10]{
%\begin{vmatrix}
% {#1}_{#6} & {#1}_{#7} & {#1}_{#8} & {#1}_{#9} & {#1}_{#10} \\
% {#2}_{#6} & {#2}_{#7} & {#2}_{#8} & {#2}_{#9} & {#2}_{#10} \\
% {#3}_{#6} & {#3}_{#7} & {#3}_{#8} & {#3}_{#9} & {#3}_{#10} \\
% {#4}_{#6} & {#4}_{#7} & {#4}_{#8} & {#4}_{#9} & {#4}_{#10} \\
% {#5}_{#6} & {#5}_{#7} & {#5}_{#8} & {#5}_{#9} & {#5}_{#10}
%\end{vmatrix}
%}

% R3 vector.
\newcommand{\VectorThree}[3]{
\begin{bmatrix}
 {#1} \\
 {#2} \\
 {#3}
\end{bmatrix}
}



\author{Peeter Joot}
\email{peeter.joot@gmail.com}

%\documentclass[]{eliblogwidescreen}

\usepackage{amsmath}
\usepackage{mathpazo}

%
% shorthand for bold symbols, convenient for vectors and matrices
%
\newcommand{\Ba}[0]{\mathbf{a}}
\newcommand{\Bb}[0]{\mathbf{b}}
\newcommand{\Bc}[0]{\mathbf{c}}
\newcommand{\Bd}[0]{\mathbf{d}}
\newcommand{\Be}[0]{\mathbf{e}}
\newcommand{\Bf}[0]{\mathbf{f}}
\newcommand{\Bg}[0]{\mathbf{g}}
\newcommand{\Bh}[0]{\mathbf{h}}
\newcommand{\Bi}[0]{\mathbf{i}}
\newcommand{\Bj}[0]{\mathbf{j}}
\newcommand{\Bk}[0]{\mathbf{k}}
\newcommand{\Bl}[0]{\mathbf{l}}
\newcommand{\Bm}[0]{\mathbf{m}}
\newcommand{\Bn}[0]{\mathbf{n}}
\newcommand{\Bo}[0]{\mathbf{o}}
\newcommand{\Bp}[0]{\mathbf{p}}
\newcommand{\Bq}[0]{\mathbf{q}}
\newcommand{\Br}[0]{\mathbf{r}}
\newcommand{\Bs}[0]{\mathbf{s}}
\newcommand{\Bt}[0]{\mathbf{t}}
\newcommand{\Bu}[0]{\mathbf{u}}
\newcommand{\Bv}[0]{\mathbf{v}}
\newcommand{\Bw}[0]{\mathbf{w}}
\newcommand{\Bx}[0]{\mathbf{x}}
\newcommand{\By}[0]{\mathbf{y}}
\newcommand{\Bz}[0]{\mathbf{z}}
\newcommand{\BA}[0]{\mathbf{A}}
\newcommand{\BB}[0]{\mathbf{B}}
\newcommand{\BC}[0]{\mathbf{C}}
\newcommand{\BD}[0]{\mathbf{D}}
\newcommand{\BE}[0]{\mathbf{E}}
\newcommand{\BF}[0]{\mathbf{F}}
\newcommand{\BG}[0]{\mathbf{G}}
\newcommand{\BH}[0]{\mathbf{H}}
\newcommand{\BI}[0]{\mathbf{I}}
\newcommand{\BJ}[0]{\mathbf{J}}
\newcommand{\BK}[0]{\mathbf{K}}
\newcommand{\BL}[0]{\mathbf{L}}
\newcommand{\BM}[0]{\mathbf{M}}
\newcommand{\BN}[0]{\mathbf{N}}
\newcommand{\BO}[0]{\mathbf{O}}
\newcommand{\BP}[0]{\mathbf{P}}
\newcommand{\BQ}[0]{\mathbf{Q}}
\newcommand{\BR}[0]{\mathbf{R}}
\newcommand{\BS}[0]{\mathbf{S}}
\newcommand{\BT}[0]{\mathbf{T}}
\newcommand{\BU}[0]{\mathbf{U}}
\newcommand{\BV}[0]{\mathbf{V}}
\newcommand{\BW}[0]{\mathbf{W}}
\newcommand{\BX}[0]{\mathbf{X}}
\newcommand{\BY}[0]{\mathbf{Y}}
\newcommand{\BZ}[0]{\mathbf{Z}}

\newcommand{\Bzero}[0]{\mathbf{0}}
\newcommand{\Btheta}[0]{\boldsymbol{\theta}}
\newcommand{\Btau}[0]{\boldsymbol{\tau}}
\newcommand{\Bomega}[0]{\boldsymbol{\omega}}

%
% shorthand for unit vectors
%
\newcommand{\acap}[0]{\hat{\Ba}}
\newcommand{\bcap}[0]{\hat{\Bb}}
\newcommand{\ccap}[0]{\hat{\Bc}}
\newcommand{\dcap}[0]{\hat{\Bd}}
\newcommand{\ecap}[0]{\hat{\Be}}
\newcommand{\fcap}[0]{\hat{\Bf}}
\newcommand{\gcap}[0]{\hat{\Bg}}
\newcommand{\hcap}[0]{\hat{\Bh}}
\newcommand{\icap}[0]{\hat{\Bi}}
\newcommand{\jcap}[0]{\hat{\Bj}}
\newcommand{\kcap}[0]{\hat{\Bk}}
\newcommand{\lcap}[0]{\hat{\Bl}}
\newcommand{\mcap}[0]{\hat{\Bm}}
\newcommand{\ncap}[0]{\hat{\Bn}}
\newcommand{\ocap}[0]{\hat{\Bo}}
\newcommand{\pcap}[0]{\hat{\Bp}}
\newcommand{\qcap}[0]{\hat{\Bq}}
\newcommand{\rcap}[0]{\hat{\Br}}
\newcommand{\scap}[0]{\hat{\Bs}}
\newcommand{\tcap}[0]{\hat{\Bt}}
\newcommand{\ucap}[0]{\hat{\Bu}}
\newcommand{\vcap}[0]{\hat{\Bv}}
\newcommand{\wcap}[0]{\hat{\Bw}}
\newcommand{\xcap}[0]{\hat{\Bx}}
\newcommand{\ycap}[0]{\hat{\By}}
\newcommand{\zcap}[0]{\hat{\Bz}}
\newcommand{\thetacap}[0]{\hat{\Btheta}}

%
% to write R^n and C^n in a distinguishable fashion.  Perhaps change this
% to the double lined characters upon figuring out how to do so.
%
\newcommand{\C}[1]{$\mathbb{C}^{#1}$}
\newcommand{\R}[1]{$\mathbb{R}^{#1}$}

%
% various generally useful helpers
%

% derivative of #1 wrt. #2:
\newcommand{\D}[2] {\frac {d#2} {d#1}}

\newcommand{\inv}[1]{\frac{1}{#1}}
\newcommand{\cross}[0]{\times}

\newcommand{\abs}[1]{\lvert{#1}\rvert}
\newcommand{\norm}[1]{\lVert{#1}\rVert}
\newcommand{\innerprod}[2]{\langle{#1}, {#2}\rangle}
\newcommand{\dotprod}[2]{{#1} \cdot {#2}}
\newcommand{\bdotprod}[2]{\left({#1} \cdot {#2}\right)}
\newcommand{\crossprod}[2]{{#1} \cross {#2}}
\newcommand{\tripleprod}[3]{\dotprod{\left(\crossprod{#1}{#2}\right)}{#3}}

\DeclareMathOperator{\Proj}{Proj}
\DeclareMathOperator{\Span}{span}
\DeclareMathOperator{\Sgn}{sgn}
\DeclareMathOperator{\Area}{Area}
\DeclareMathOperator{\Volume}{Volume}

%
% A few miscellaneous things specific to this document
%
\newcommand{\crossop}[1]{\crossprod{#1}{}}

% R2 vector.
\newcommand{\VectorTwo}[2]{
\begin{bmatrix}
 {#1} \\
 {#2}
\end{bmatrix}
}

\newcommand{\VectorN}[1]{
\begin{bmatrix}
{#1}_1 \\
{#1}_2 \\
\vdots \\
{#1}_N \\
\end{bmatrix}
}

\newcommand{\DETuvij}[4]{
\begin{vmatrix}
 {#1}_{#3} & {#1}_{#4} \\
 {#2}_{#3} & {#2}_{#4}
\end{vmatrix}
}

\newcommand{\DETuvwijk}[6]{
\begin{vmatrix}
 {#1}_{#4} & {#1}_{#5} & {#1}_{#6} \\
 {#2}_{#4} & {#2}_{#5} & {#2}_{#6} \\
 {#3}_{#4} & {#3}_{#5} & {#3}_{#6}
\end{vmatrix}
}

\newcommand{\DETuvwxijkl}[8]{
\begin{vmatrix}
 {#1}_{#5} & {#1}_{#6} & {#1}_{#7} & {#1}_{#8} \\
 {#2}_{#5} & {#2}_{#6} & {#2}_{#7} & {#2}_{#8} \\
 {#3}_{#5} & {#3}_{#6} & {#3}_{#7} & {#3}_{#8} \\
 {#4}_{#5} & {#4}_{#6} & {#4}_{#7} & {#4}_{#8} \\
\end{vmatrix}
}

%\newcommand{\DETuvwxyijklm}[10]{
%\begin{vmatrix}
% {#1}_{#6} & {#1}_{#7} & {#1}_{#8} & {#1}_{#9} & {#1}_{#10} \\
% {#2}_{#6} & {#2}_{#7} & {#2}_{#8} & {#2}_{#9} & {#2}_{#10} \\
% {#3}_{#6} & {#3}_{#7} & {#3}_{#8} & {#3}_{#9} & {#3}_{#10} \\
% {#4}_{#6} & {#4}_{#7} & {#4}_{#8} & {#4}_{#9} & {#4}_{#10} \\
% {#5}_{#6} & {#5}_{#7} & {#5}_{#8} & {#5}_{#9} & {#5}_{#10}
%\end{vmatrix}
%}

% R3 vector.
\newcommand{\VectorThree}[3]{
\begin{bmatrix}
 {#1} \\
 {#2} \\
 {#3}
\end{bmatrix}
}



\author{Peeter Joot}
\email{peeter.joot@gmail.com}


\chapter{PHY450H1S.  Relativistic Electrodynamics Lecture 10 (Taught by Prof. Erich Poppitz).  FIXME.}
\label{chap:relativisticElectrodynamicsL10}
%\useCCL
\blogpage{http://sites.google.com/site/peeterjoot/math2011/relativisticElectrodynamicsL10.pdf}
\date{Feb 8, 2011}
\revisionInfo{relativisticElectrodynamicsL10.tex}

%\beginArtWithToc
\beginArtNoToc

\section{Reading.}

Covering chapter N material from the text \cite{landau1980classical}.

Covering \href{}{}

http://www.physics.utoronto.ca/~poppitz/e-poppitz/PHY450_files/RelEMpp74-83.pdf

pp. 74-83: gauge transformations in 3-vector language (74); energy of a relativistic particle in EM field (75); variational
principle and equation of motion in 4-vector form (76-77); the field strength tensor (78-80); the fourth equation of motion (81);  Lorentz transformation of the strength tensor (82) [Tuesday, Feb. 8] [extra reading for the mathematically minded: gauge field, strength tensor, and gauge transformations in differential form language, not to be covered in class (83)] 

http://www.physics.utoronto.ca/~poppitz/e-poppitz/PHY450_files/RelEMpp84-102.pdf

pp. 84-102: Lorentz invariants of the electromagnetic field (84-86); Bianchi identity and the first half of Maxwell�s equations (87-90); relativity, gauge invariance, and superposition principles and the action for the electromagnetic field coupled to charged particles (91-95); the 4-current and its physical interpretation (96-102), including a needed mathematical interlude on delta-functions of functions (98-100) [Wednesday, Feb. 8; Thursday, Feb. 10]

http://www.physics.utoronto.ca/~poppitz/e-poppitz/PHY450_files/RelEMpp103-113.pdf

pp.103-113: variational principle for the electromagnetic field and the relevant boundary conditions (103-105); the second set of Maxwell�s equations from the variational principle (106-108); Maxwell�s equations in vacuum and the wave equation in the nonrelativistic Coulomb gauge (109-111); the wave equation in the relativistic Lorentz gauge (112-113) [Tuesday, Feb. 15; Wednesday, Feb.16]...

\section{.}

\section{What is the significance to this claim?}

Having argued that under a gauge transformation

A_i \rightarrow A_i + \partial_i \chi

the action for a particle changes by a boundary term 

\begin{equation}\label{eqn:lorentzForceHamiltonian:n}
- \frac{e}{c} ( \chi(x_b) - \chi(x_a) )
\end{equation}

Because $S$ changes by a boundary term only, variation problem is not affected.  The extremal trajectories are then the same, hence the EOM are the same.

\paragraph{A less high brow demonstration}

A_i = (\phi, -\BA)
A^i = (\phi, \BA)

A_0 -> A_0 + \PD{x^0}{\chi}
-\BA -> -\BA + \PD{\Bx}{\chi}

\phi -> \phi + \inv{c}\PD{t}{\chi}
\BA -> \BA - \spacegrad \chi

E = - \grad \phi - \inv{c} \pdt \BA -> 
- \grad \phi - \inv{c} \grad \pdt \chi 
- \grad \phi - \inv{c} \pdt \BA 
...
= E

B = \spacegrad \cross BA 
-> 
\spacegrad \cross BA  - \spacegrad (\cross \spacegrad \chi)

EM fields (as opposed to potentials) do not change under grauge transformations

{A_i} description is hugely redundant

Despite that, local \LL and H can only be written into A_i

\LL = - mc^2 \InvGamma + \fracec \BA \cdot \Bv - e \phi

we define the energy as 

\E = \Bv \cdot \pdV \LL - \LL

This is not neccessarily a conserved quantity, but we define it as the energy anyways (we don't really have a Hamiltonian when the fields are time dependent).  In general we have

\ddt \E = -pdt \LL

\E = \underbrace{\frac{m c^2}{\InvGamma}}_{\conj} + \Bv \cdot \pdV \left( \frac{e}{c} \BA \cdot \Bv - e \phi
\right)

The contribution of $\conj$ to $\E$ of $\LL = -m c^2 \InvGamma$

\E = \frac{m c^2}{\InvGamma} + \underbrac{e \phi}_{\text{"potential"}}

\frac{d}{dt} \E_\{\text{kinetic}} = e \BE \cdot \Bv

One way that this follows is from 

\ddt \Bp = e \left( \BE + ... ) lorentz force

\section{}

Using $ds = \sqrt{ dx^i dx_i } $ our action can be rewritten

S 
&= \int ( -m c ds - \fracec u^i A_i ds ) \\
&= \int ( -m c ds - \fracec dx^i A_i ) \\
&= \int ( -m c \sqrt{ dx^i dx_i} - \fracec dx^i A_i ) \\

x^i(\tau) is a worldline $x^i(0) = a^i$, $x^i(1) = b^i$, 

We want $\delta S = S[ x + \delta x ] - S[ x ] = 0$ (to linear order in $\delta x$)

\delta \sqrt{ dx^i dx_i } 
&= \inv{ 2 \sqrt{ dx^i dx_i }} \delta (dx^j dx_j)

\delta (dx^j dx_j) 
&= \delta ( dx^j g_{jk} dx^k ) \\
&= \delta ( dx^j ) g_{jk} dx^k ) + dx^j g_{jk} \delta ( dx^k ) \\
&= \delta ( dx^j ) g_{jk} dx^k ) + dx^k g_{kj} \delta ( dx^j ) \\
&= 2 \delta ( dx^j ) g_{jk} dx^k ) \\
&= 2 \delta ( dx^j ) dx_j 

\paragraph{TIP:} If this goes to quick, or there is any disbelief, write these all out explicitly as $dx^j dx_j = dx^0 dx_0 + dx^1 dx_1 + dx^2 dx_2 + dx^3 dx_3$ and compute it that way.

\delta A_i = A_i(x + \delta x) - A_i = \pd{x^j}{A_i} \delta x^j

(chain rule)

Completing the above we have
\delta \sqrt{ dx^i dx_i } 
= \inv{ \sqrt{ dx^i dx_i }} \delta (dx^j) dx_j

and our action variance is now

\delta S 
&= \int ( -m c \delta \sqrt{ dx^i dx_i} - \fracec d \delta x^i A_i -fracec dx^i \delta A_i ) \\
&= \int ( -m c \frac{ dx^j g_{jk} d \delta x^k }{\sqrt{ dx^i dx_i} - \fracec d \delta x^i A_i - \fracec dx^i \pd{x^j}{A_i} \delta x^j ) \\
&= \int ( -m c u^j g_{jk} d \delta x^k - \fracec d \delta x^i A_i - \fracec dx^i \pd{x^j}{A_i} \delta x^j ) \\
&= \int ( -m c u^j g_{jk} \delta x^k + d u^j g_{jk} \delta x^k m c + d ( -\fracec \delta x^i A_i + \fracec \delta x^i d A_i ) - \fracec dx^i \pd{x^j}{A_i} \delta x^j ) \\
&= \int ( -m c u^j g_{jk} d \delta x^k + d u^j g_{jk} + d ( -\fracec \delta x^i A_i + \fracec \delta x^i d A_i ) - \fracec dx^i \pd{x^j}{A_i} \delta x^j ) \\

\delta S = 
\int d ( - m c u^j g_{jk} \delta x^k - \fracec A_i \delta x^i
+ \int m c d u^i g_{ij} \delta x^j + \fracec \delta x^l d A_l - \fracec \delta x^i \pd x^i A_j d x^j

The first term is

 - mc u^i \delta x_i - \fracec A_i \delta x^i \vert^b_a = 0

since \delta x^i |vert_a = \delta x^i \vert_b = 0 (the variation at the endpoints is zero)

We are left with

\int ds m c \dds u^j g_{jk} \delta x^k + \fracec \delta x^l \pd x^k A_l \dds x^k ds - \fracec \pd x^l A_i \dds x^i ds
\int ds \delta x^l  ...

\delta S = \int ds \delta x^l ( m c \dds u_l + \fracec u^k ( \partial_k A_l - \partial_l A_k )

Since this is true for all variations $\delta x^l$, which is arbitrary, the interior part is zero everywhere in the tragectory

m c \dds u_l 
= - \fracec ( \pd x^l A_k - \pd x_k A_l ) u^k
= \fracec ( \pd x^k A_l - \pd x_l A_k ) u^k

m c \dds u_l = \fracec ( \underbrace{ \pd x^k A_l - \pd x_l A_k }_{F_{lk} }) u^k

The quantity $F_{lk}$ is called the electromagnetic field strength tensor ( 4-tensor, rank-2 antisymmetric ).

We write

\dds m c u_l = \fracec F_{lk} u^k 

\Abs{ F_{lk} } = 
\begin{bmatrix}
0 & E_x & E_y & E_z \\
-E_x & 0 & -B_z & B_y \\
-E_y & B_z & 0 & -B_x \\
-E_z & -B_y & B_x & 0
\end{bmatrix}

\EndArticle
