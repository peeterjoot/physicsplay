%
% Copyright � 2015 Peeter Joot.  All Rights Reserved.
% Licenced as described in the file LICENSE under the root directory of this GIT repository.
%
\documentclass[]{eliblog}

\usepackage{amsmath}
\usepackage{mathpazo}

%
% shorthand for bold symbols, convenient for vectors and matrices
%
\newcommand{\Ba}[0]{\mathbf{a}}
\newcommand{\Bb}[0]{\mathbf{b}}
\newcommand{\Bc}[0]{\mathbf{c}}
\newcommand{\Bd}[0]{\mathbf{d}}
\newcommand{\Be}[0]{\mathbf{e}}
\newcommand{\Bf}[0]{\mathbf{f}}
\newcommand{\Bg}[0]{\mathbf{g}}
\newcommand{\Bh}[0]{\mathbf{h}}
\newcommand{\Bi}[0]{\mathbf{i}}
\newcommand{\Bj}[0]{\mathbf{j}}
\newcommand{\Bk}[0]{\mathbf{k}}
\newcommand{\Bl}[0]{\mathbf{l}}
\newcommand{\Bm}[0]{\mathbf{m}}
\newcommand{\Bn}[0]{\mathbf{n}}
\newcommand{\Bo}[0]{\mathbf{o}}
\newcommand{\Bp}[0]{\mathbf{p}}
\newcommand{\Bq}[0]{\mathbf{q}}
\newcommand{\Br}[0]{\mathbf{r}}
\newcommand{\Bs}[0]{\mathbf{s}}
\newcommand{\Bt}[0]{\mathbf{t}}
\newcommand{\Bu}[0]{\mathbf{u}}
\newcommand{\Bv}[0]{\mathbf{v}}
\newcommand{\Bw}[0]{\mathbf{w}}
\newcommand{\Bx}[0]{\mathbf{x}}
\newcommand{\By}[0]{\mathbf{y}}
\newcommand{\Bz}[0]{\mathbf{z}}
\newcommand{\BA}[0]{\mathbf{A}}
\newcommand{\BB}[0]{\mathbf{B}}
\newcommand{\BC}[0]{\mathbf{C}}
\newcommand{\BD}[0]{\mathbf{D}}
\newcommand{\BE}[0]{\mathbf{E}}
\newcommand{\BF}[0]{\mathbf{F}}
\newcommand{\BG}[0]{\mathbf{G}}
\newcommand{\BH}[0]{\mathbf{H}}
\newcommand{\BI}[0]{\mathbf{I}}
\newcommand{\BJ}[0]{\mathbf{J}}
\newcommand{\BK}[0]{\mathbf{K}}
\newcommand{\BL}[0]{\mathbf{L}}
\newcommand{\BM}[0]{\mathbf{M}}
\newcommand{\BN}[0]{\mathbf{N}}
\newcommand{\BO}[0]{\mathbf{O}}
\newcommand{\BP}[0]{\mathbf{P}}
\newcommand{\BQ}[0]{\mathbf{Q}}
\newcommand{\BR}[0]{\mathbf{R}}
\newcommand{\BS}[0]{\mathbf{S}}
\newcommand{\BT}[0]{\mathbf{T}}
\newcommand{\BU}[0]{\mathbf{U}}
\newcommand{\BV}[0]{\mathbf{V}}
\newcommand{\BW}[0]{\mathbf{W}}
\newcommand{\BX}[0]{\mathbf{X}}
\newcommand{\BY}[0]{\mathbf{Y}}
\newcommand{\BZ}[0]{\mathbf{Z}}

\newcommand{\Bzero}[0]{\mathbf{0}}
\newcommand{\Btheta}[0]{\boldsymbol{\theta}}
\newcommand{\Btau}[0]{\boldsymbol{\tau}}
\newcommand{\Bomega}[0]{\boldsymbol{\omega}}

%
% shorthand for unit vectors
%
\newcommand{\acap}[0]{\hat{\Ba}}
\newcommand{\bcap}[0]{\hat{\Bb}}
\newcommand{\ccap}[0]{\hat{\Bc}}
\newcommand{\dcap}[0]{\hat{\Bd}}
\newcommand{\ecap}[0]{\hat{\Be}}
\newcommand{\fcap}[0]{\hat{\Bf}}
\newcommand{\gcap}[0]{\hat{\Bg}}
\newcommand{\hcap}[0]{\hat{\Bh}}
\newcommand{\icap}[0]{\hat{\Bi}}
\newcommand{\jcap}[0]{\hat{\Bj}}
\newcommand{\kcap}[0]{\hat{\Bk}}
\newcommand{\lcap}[0]{\hat{\Bl}}
\newcommand{\mcap}[0]{\hat{\Bm}}
\newcommand{\ncap}[0]{\hat{\Bn}}
\newcommand{\ocap}[0]{\hat{\Bo}}
\newcommand{\pcap}[0]{\hat{\Bp}}
\newcommand{\qcap}[0]{\hat{\Bq}}
\newcommand{\rcap}[0]{\hat{\Br}}
\newcommand{\scap}[0]{\hat{\Bs}}
\newcommand{\tcap}[0]{\hat{\Bt}}
\newcommand{\ucap}[0]{\hat{\Bu}}
\newcommand{\vcap}[0]{\hat{\Bv}}
\newcommand{\wcap}[0]{\hat{\Bw}}
\newcommand{\xcap}[0]{\hat{\Bx}}
\newcommand{\ycap}[0]{\hat{\By}}
\newcommand{\zcap}[0]{\hat{\Bz}}
\newcommand{\thetacap}[0]{\hat{\Btheta}}

%
% to write R^n and C^n in a distinguishable fashion.  Perhaps change this
% to the double lined characters upon figuring out how to do so.
%
\newcommand{\C}[1]{$\mathbb{C}^{#1}$}
\newcommand{\R}[1]{$\mathbb{R}^{#1}$}

%
% various generally useful helpers
%

% derivative of #1 wrt. #2:
\newcommand{\D}[2] {\frac {d#2} {d#1}}

\newcommand{\inv}[1]{\frac{1}{#1}}
\newcommand{\cross}[0]{\times}

\newcommand{\abs}[1]{\lvert{#1}\rvert}
\newcommand{\norm}[1]{\lVert{#1}\rVert}
\newcommand{\innerprod}[2]{\langle{#1}, {#2}\rangle}
\newcommand{\dotprod}[2]{{#1} \cdot {#2}}
\newcommand{\bdotprod}[2]{\left({#1} \cdot {#2}\right)}
\newcommand{\crossprod}[2]{{#1} \cross {#2}}
\newcommand{\tripleprod}[3]{\dotprod{\left(\crossprod{#1}{#2}\right)}{#3}}

\DeclareMathOperator{\Proj}{Proj}
\DeclareMathOperator{\Span}{span}
\DeclareMathOperator{\Sgn}{sgn}
\DeclareMathOperator{\Area}{Area}
\DeclareMathOperator{\Volume}{Volume}

%
% A few miscellaneous things specific to this document
%
\newcommand{\crossop}[1]{\crossprod{#1}{}}

% R2 vector.
\newcommand{\VectorTwo}[2]{
\begin{bmatrix}
 {#1} \\
 {#2}
\end{bmatrix}
}

\newcommand{\VectorN}[1]{
\begin{bmatrix}
{#1}_1 \\
{#1}_2 \\
\vdots \\
{#1}_N \\
\end{bmatrix}
}

\newcommand{\DETuvij}[4]{
\begin{vmatrix}
 {#1}_{#3} & {#1}_{#4} \\
 {#2}_{#3} & {#2}_{#4}
\end{vmatrix}
}

\newcommand{\DETuvwijk}[6]{
\begin{vmatrix}
 {#1}_{#4} & {#1}_{#5} & {#1}_{#6} \\
 {#2}_{#4} & {#2}_{#5} & {#2}_{#6} \\
 {#3}_{#4} & {#3}_{#5} & {#3}_{#6}
\end{vmatrix}
}

\newcommand{\DETuvwxijkl}[8]{
\begin{vmatrix}
 {#1}_{#5} & {#1}_{#6} & {#1}_{#7} & {#1}_{#8} \\
 {#2}_{#5} & {#2}_{#6} & {#2}_{#7} & {#2}_{#8} \\
 {#3}_{#5} & {#3}_{#6} & {#3}_{#7} & {#3}_{#8} \\
 {#4}_{#5} & {#4}_{#6} & {#4}_{#7} & {#4}_{#8} \\
\end{vmatrix}
}

%\newcommand{\DETuvwxyijklm}[10]{
%\begin{vmatrix}
% {#1}_{#6} & {#1}_{#7} & {#1}_{#8} & {#1}_{#9} & {#1}_{#10} \\
% {#2}_{#6} & {#2}_{#7} & {#2}_{#8} & {#2}_{#9} & {#2}_{#10} \\
% {#3}_{#6} & {#3}_{#7} & {#3}_{#8} & {#3}_{#9} & {#3}_{#10} \\
% {#4}_{#6} & {#4}_{#7} & {#4}_{#8} & {#4}_{#9} & {#4}_{#10} \\
% {#5}_{#6} & {#5}_{#7} & {#5}_{#8} & {#5}_{#9} & {#5}_{#10}
%\end{vmatrix}
%}

% R3 vector.
\newcommand{\VectorThree}[3]{
\begin{bmatrix}
 {#1} \\
 {#2} \\
 {#3}
\end{bmatrix}
}



\author{Peeter Joot}
\email{peeter.joot@gmail.com}

%\documentclass[]{eliblogwidescreen}

\usepackage{amsmath}
\usepackage{mathpazo}

%
% shorthand for bold symbols, convenient for vectors and matrices
%
\newcommand{\Ba}[0]{\mathbf{a}}
\newcommand{\Bb}[0]{\mathbf{b}}
\newcommand{\Bc}[0]{\mathbf{c}}
\newcommand{\Bd}[0]{\mathbf{d}}
\newcommand{\Be}[0]{\mathbf{e}}
\newcommand{\Bf}[0]{\mathbf{f}}
\newcommand{\Bg}[0]{\mathbf{g}}
\newcommand{\Bh}[0]{\mathbf{h}}
\newcommand{\Bi}[0]{\mathbf{i}}
\newcommand{\Bj}[0]{\mathbf{j}}
\newcommand{\Bk}[0]{\mathbf{k}}
\newcommand{\Bl}[0]{\mathbf{l}}
\newcommand{\Bm}[0]{\mathbf{m}}
\newcommand{\Bn}[0]{\mathbf{n}}
\newcommand{\Bo}[0]{\mathbf{o}}
\newcommand{\Bp}[0]{\mathbf{p}}
\newcommand{\Bq}[0]{\mathbf{q}}
\newcommand{\Br}[0]{\mathbf{r}}
\newcommand{\Bs}[0]{\mathbf{s}}
\newcommand{\Bt}[0]{\mathbf{t}}
\newcommand{\Bu}[0]{\mathbf{u}}
\newcommand{\Bv}[0]{\mathbf{v}}
\newcommand{\Bw}[0]{\mathbf{w}}
\newcommand{\Bx}[0]{\mathbf{x}}
\newcommand{\By}[0]{\mathbf{y}}
\newcommand{\Bz}[0]{\mathbf{z}}
\newcommand{\BA}[0]{\mathbf{A}}
\newcommand{\BB}[0]{\mathbf{B}}
\newcommand{\BC}[0]{\mathbf{C}}
\newcommand{\BD}[0]{\mathbf{D}}
\newcommand{\BE}[0]{\mathbf{E}}
\newcommand{\BF}[0]{\mathbf{F}}
\newcommand{\BG}[0]{\mathbf{G}}
\newcommand{\BH}[0]{\mathbf{H}}
\newcommand{\BI}[0]{\mathbf{I}}
\newcommand{\BJ}[0]{\mathbf{J}}
\newcommand{\BK}[0]{\mathbf{K}}
\newcommand{\BL}[0]{\mathbf{L}}
\newcommand{\BM}[0]{\mathbf{M}}
\newcommand{\BN}[0]{\mathbf{N}}
\newcommand{\BO}[0]{\mathbf{O}}
\newcommand{\BP}[0]{\mathbf{P}}
\newcommand{\BQ}[0]{\mathbf{Q}}
\newcommand{\BR}[0]{\mathbf{R}}
\newcommand{\BS}[0]{\mathbf{S}}
\newcommand{\BT}[0]{\mathbf{T}}
\newcommand{\BU}[0]{\mathbf{U}}
\newcommand{\BV}[0]{\mathbf{V}}
\newcommand{\BW}[0]{\mathbf{W}}
\newcommand{\BX}[0]{\mathbf{X}}
\newcommand{\BY}[0]{\mathbf{Y}}
\newcommand{\BZ}[0]{\mathbf{Z}}

\newcommand{\Bzero}[0]{\mathbf{0}}
\newcommand{\Btheta}[0]{\boldsymbol{\theta}}
\newcommand{\Btau}[0]{\boldsymbol{\tau}}
\newcommand{\Bomega}[0]{\boldsymbol{\omega}}

%
% shorthand for unit vectors
%
\newcommand{\acap}[0]{\hat{\Ba}}
\newcommand{\bcap}[0]{\hat{\Bb}}
\newcommand{\ccap}[0]{\hat{\Bc}}
\newcommand{\dcap}[0]{\hat{\Bd}}
\newcommand{\ecap}[0]{\hat{\Be}}
\newcommand{\fcap}[0]{\hat{\Bf}}
\newcommand{\gcap}[0]{\hat{\Bg}}
\newcommand{\hcap}[0]{\hat{\Bh}}
\newcommand{\icap}[0]{\hat{\Bi}}
\newcommand{\jcap}[0]{\hat{\Bj}}
\newcommand{\kcap}[0]{\hat{\Bk}}
\newcommand{\lcap}[0]{\hat{\Bl}}
\newcommand{\mcap}[0]{\hat{\Bm}}
\newcommand{\ncap}[0]{\hat{\Bn}}
\newcommand{\ocap}[0]{\hat{\Bo}}
\newcommand{\pcap}[0]{\hat{\Bp}}
\newcommand{\qcap}[0]{\hat{\Bq}}
\newcommand{\rcap}[0]{\hat{\Br}}
\newcommand{\scap}[0]{\hat{\Bs}}
\newcommand{\tcap}[0]{\hat{\Bt}}
\newcommand{\ucap}[0]{\hat{\Bu}}
\newcommand{\vcap}[0]{\hat{\Bv}}
\newcommand{\wcap}[0]{\hat{\Bw}}
\newcommand{\xcap}[0]{\hat{\Bx}}
\newcommand{\ycap}[0]{\hat{\By}}
\newcommand{\zcap}[0]{\hat{\Bz}}
\newcommand{\thetacap}[0]{\hat{\Btheta}}

%
% to write R^n and C^n in a distinguishable fashion.  Perhaps change this
% to the double lined characters upon figuring out how to do so.
%
\newcommand{\C}[1]{$\mathbb{C}^{#1}$}
\newcommand{\R}[1]{$\mathbb{R}^{#1}$}

%
% various generally useful helpers
%

% derivative of #1 wrt. #2:
\newcommand{\D}[2] {\frac {d#2} {d#1}}

\newcommand{\inv}[1]{\frac{1}{#1}}
\newcommand{\cross}[0]{\times}

\newcommand{\abs}[1]{\lvert{#1}\rvert}
\newcommand{\norm}[1]{\lVert{#1}\rVert}
\newcommand{\innerprod}[2]{\langle{#1}, {#2}\rangle}
\newcommand{\dotprod}[2]{{#1} \cdot {#2}}
\newcommand{\bdotprod}[2]{\left({#1} \cdot {#2}\right)}
\newcommand{\crossprod}[2]{{#1} \cross {#2}}
\newcommand{\tripleprod}[3]{\dotprod{\left(\crossprod{#1}{#2}\right)}{#3}}

\DeclareMathOperator{\Proj}{Proj}
\DeclareMathOperator{\Span}{span}
\DeclareMathOperator{\Sgn}{sgn}
\DeclareMathOperator{\Area}{Area}
\DeclareMathOperator{\Volume}{Volume}

%
% A few miscellaneous things specific to this document
%
\newcommand{\crossop}[1]{\crossprod{#1}{}}

% R2 vector.
\newcommand{\VectorTwo}[2]{
\begin{bmatrix}
 {#1} \\
 {#2}
\end{bmatrix}
}

\newcommand{\VectorN}[1]{
\begin{bmatrix}
{#1}_1 \\
{#1}_2 \\
\vdots \\
{#1}_N \\
\end{bmatrix}
}

\newcommand{\DETuvij}[4]{
\begin{vmatrix}
 {#1}_{#3} & {#1}_{#4} \\
 {#2}_{#3} & {#2}_{#4}
\end{vmatrix}
}

\newcommand{\DETuvwijk}[6]{
\begin{vmatrix}
 {#1}_{#4} & {#1}_{#5} & {#1}_{#6} \\
 {#2}_{#4} & {#2}_{#5} & {#2}_{#6} \\
 {#3}_{#4} & {#3}_{#5} & {#3}_{#6}
\end{vmatrix}
}

\newcommand{\DETuvwxijkl}[8]{
\begin{vmatrix}
 {#1}_{#5} & {#1}_{#6} & {#1}_{#7} & {#1}_{#8} \\
 {#2}_{#5} & {#2}_{#6} & {#2}_{#7} & {#2}_{#8} \\
 {#3}_{#5} & {#3}_{#6} & {#3}_{#7} & {#3}_{#8} \\
 {#4}_{#5} & {#4}_{#6} & {#4}_{#7} & {#4}_{#8} \\
\end{vmatrix}
}

%\newcommand{\DETuvwxyijklm}[10]{
%\begin{vmatrix}
% {#1}_{#6} & {#1}_{#7} & {#1}_{#8} & {#1}_{#9} & {#1}_{#10} \\
% {#2}_{#6} & {#2}_{#7} & {#2}_{#8} & {#2}_{#9} & {#2}_{#10} \\
% {#3}_{#6} & {#3}_{#7} & {#3}_{#8} & {#3}_{#9} & {#3}_{#10} \\
% {#4}_{#6} & {#4}_{#7} & {#4}_{#8} & {#4}_{#9} & {#4}_{#10} \\
% {#5}_{#6} & {#5}_{#7} & {#5}_{#8} & {#5}_{#9} & {#5}_{#10}
%\end{vmatrix}
%}

% R3 vector.
\newcommand{\VectorThree}[3]{
\begin{bmatrix}
 {#1} \\
 {#2} \\
 {#3}
\end{bmatrix}
}



\author{Peeter Joot}
\email{peeter.joot@gmail.com}


\chapter{PHY454H1S Continuum mechanics midterm reflection.}
\label{chap:continuumMidtermReflection}
%\useCCL
\blogpage{http://sites.google.com/site/peeterjoot2/math2012/continuumMidtermReflection.pdf}
\date{Mar 14, 2012}
\gitRevisionInfo{continuumMidtermReflection}

\beginArtWithToc
%\beginArtNoToc

\section{Motivation.}

I didn't manage my time well enough on the midterm to complete it (and also missed one easy part of the second question).  For later review purposes, here is either what I answered, or what I think I should have answered for these questions.

\section{Problem 1.}

\subsection{$\BP$-waves, $\BS$-waves, and Love-waves.}

\begin{itemize}
\item Show that in $\BP$-waves the divergence of the displacement vector represents a measure of the relative change in the volume of the body.

\paragraph{Answer.}
The $\BP$-wave equation was a result of operating on the displacement equation with the divergence operator

\begin{equation}\label{eqn:continuumMidTermReflection:10}
\spacegrad \cdot \left( 
\rho \PDSq{t}{\Be} = (\lambda + \mu) \spacegrad (\spacegrad \cdot \Be) + \mu \spacegrad^2 \Be
\right)
\end{equation}

we obtain

\begin{equation}\label{eqn:continuumMidTermReflection:30}
\PDSq{t}{} \left( \spacegrad \cdot \Be \right) = \frac{\lambda + 2 \mu}{\rho} \spacegrad^2 (\spacegrad \cdot \Be).
\end{equation}

We have a wave equation where the ``waving'' quanity is $\Theta = \spacegrad \cdot \Be$.  Explicitly

\begin{align*}
\Theta 
&= \spacegrad \cdot \Be \\
&= 
\PD{x}{e_1}
+\PD{y}{e_2}
+\PD{z}{e_3}
\end{align*}

Recall that, in a coordinate basis for which the strain $e_{ij}$ is diagonal we have

\begin{align}\label{eqn:continuumMidTermReflection:50}
dx' &= \sqrt{1 + 2 e_{11}} dx \\
dy' &= \sqrt{1 + 2 e_{22}} dy \\
dz' &= \sqrt{1 + 2 e_{33}} dz.
\end{align}

Expanding in Taylor series to $O(1)$ we have for $i = 1, 2, 3$ (no sum)

\begin{equation}\label{eqn:continuumMidTermReflection:70}
dx_i' \approx (1 + e_{ii}) dx_i.
\end{equation}

so the displaced volume is

\begin{align*}
dV' &= 
dx_1
dx_2
dx_3
(1 + e_{11})
(1 + e_{22})
(1 + e_{33}) \\
&=
dx_1
dx_2
dx_3
( 1  + e_{11} + e_{22} + e_{33} + O(e_{kk}^2) )
\end{align*}

Since 

\begin{align}\label{eqn:continuumMidTermReflection:90}
e_{11} &= \inv{2} \left( \PD{x}{e_1} +\PD{x}{e_1} \right) = \PD{x}{e_1} \\
e_{22} &= \inv{2} \left( \PD{y}{e_2} +\PD{y}{e_2} \right) = \PD{y}{e_2} \\
e_{33} &= \inv{2} \left( \PD{z}{e_3} +\PD{z}{e_3} \right) = \PD{z}{e_3}
\end{align}

We have

\begin{equation}\label{eqn:continuumMidTermReflection:110}
dV' = (1 + \spacegrad \cdot \Be) dV,
\end{equation}

or

\begin{equation}\label{eqn:continuumMidTermReflection:130}
\frac{dV' - dV}{dV} = \spacegrad \cdot \Be
\end{equation}

The relative change in volume can therefore be expressed as the divergence of $\Be$, the displacement vector, and it is this relative volume change that is ``waving'' in the $\BP$-wave equation as illustrated in the following (\ref{fig:continuumMidtermReflection:continuumMidtermReflectionFig1}) sample 1D compression wave

\begin{figure}[htp]
   \centering
   \includegraphics[totalheight=0.2\textheight]{continuumMidtermReflectionFig1}
   \caption{A 1D compression wave.}\label{fig:continuumMidtermReflection:continuumMidtermReflectionFig1}
\end{figure}

\item Between a $\BP$-wave and an $\BS$-wave which one is longitudinal and which one is transverse?

\paragraph{Answer.}
$\BP$-waves are longitudinal.

$\BS$-waves are transverse.

\item Whose speed is higher?

\paragraph{Answer.}
From the formula sheet we have

\begin{align*}
\left( \frac{c_L}{c_T} \right)^2 
&= \frac{ \lambda + 2 \mu}{\rho} \frac{\rho}{\mu}  \\
&= \frac{\lambda}{\mu} + 2  \\
&> 1
\end{align*}

so $\BP$-waves travel faster than $\BS$-waves.

\item Is Love wave a body wave or a surface wave?

\paragraph{Answer.}

Love waves are surface waves, travelling in a medium that can slide on top of another surface.  These are characterized by vorticity rotating backwards compared to the direction of propagation as shown in figure (\ref{fig:continuumMidtermReflection:continuumMidtermReflectionFig2})

\begin{figure}[htp]
   \centering
   \def\svgwidth{0.6\columnwidth}
   \input{continuumMidtermReflectionFig2.pdf_tex}
   \caption{Love wave illustrated.}\label{fig:continuumMidtermReflection:continuumMidtermReflectionFig2}
\end{figure}
\end{itemize}

\subsection{(b) constitutive relation, Newtonian fluids, and no-slip conditions.}

\begin{itemize}
\item In continuum mechanics what do you mean by \textit{consitutive relation}?
\paragraph{Answer.}  The constituitive relation is the stress-strain relation, generally

\begin{equation}\label{eqn:continuumMidTermReflection:150}
\sigma_{ij} = c_{abij} e_{ab}
\end{equation}

for isotropic solid's we model this as 

\begin{equation}\label{eqn:continuumMidTermReflection:170}
\sigma_{ij} = \lambda e_{kk} \delta_{ij} + 2 \mu e_{ij}
\end{equation}

and for Newtonian fluids

\begin{equation}\label{eqn:continuumMidTermReflection:190}
\sigma_{ij} = -p \delta_{ij} + 2 \mu e_{ij}
\end{equation}

\item What is the definition of a Non-Newtonian fluid?
\paragraph{Answer.}

A non-Newtonian fluid would be one with a more general constitutive relationship.

\paragraph{Grading note.}  I lost a mark here.  I think the answer that was being looked for (as in \cite{wiki:newtonianFluids}) was that a newtonian fluid is one with a linear stress strain relationship, and a non-Newtonian fluid would be one with a non-linear relationship.  According to \cite{wiki:nonNewtonianFluid} an example of a non-newtonian material that we are all familiar with is Silly Putty.  This linearity is also how a Newtonian fluid was defined in the notes, but I didn't remember that (this isn't really something we use since we assume all fluids and materials are Newtonian in any calculations that we do).

\item What do you mean by \emph{no-slip} boundary condition at a fluid-fluid interface?

\paragraph{Crap.} Somehow in my misguided attempt to be complete, I missed this question amongst the rest of my verbosity).

\paragraph{Answer.}
The no slip boundary condition is just one of velocity matching.  At a non-moving boundary, the no-slip condition means that we'll require the fluid to also have no velocity (ie. at that interface the fluid isn't slipping over the surface).  Between two fluids, this is a requirement that the velocities of both fluids match at that point (and all the rest of the points along the region of the interaction.)

\item Write down the continuity equation for an incompressible fluid.
\paragraph{Answer.}

An incompressible fluid has

\begin{equation}\label{eqn:continuumMidTermReflection:210}
\frac{d\rho}{dt} = 0,
\end{equation}

On the midterm, I'd written

\begin{align*}
0 
&=
\frac{d\rho}{dt} \\
&= - \rho (\spacegrad \cdot \Bu)  \\
&= 
\PD{t}{\rho} + (\Bu \cdot \spacegrad) \rho \\
&= 0,
\end{align*}

which makes no sense whatsoever.  After that I'd said that a consequence is $\spacegrad \cdot \Bu = 0$ for an incompressible fluid, and gotten my marks for that despite having spurted garbage before that.  What I must have been trying to refer to here was that this is a consequence of mass conservation.  The rate that mass leaves a volume can be expressed as

\begin{align*}
\frac{dm}{dt}
&= \int \frac{d\rho}{dt} dV \\
&= -\int_{\partial V} \rho \Bu \cdot d\BA \\
&= -\int_V \spacegrad \cdot (\rho \Bu) dV
\end{align*}

(the minus sign here signifying that the mass is leaving the volume through the surface, and that we are using an outwards facing normal on the volume.)

If the surface bounding the volume doesn't change with time (ie. $\PDi{t}{V} = 0$) we can write

\begin{equation}\label{eqn:continuumMidTermReflection:230}
\PD{t}{} \int \rho dV = -\int \spacegrad \cdot (\rho \Bu) dV,
\end{equation}

or

\begin{equation}\label{eqn:continuumMidTermReflection:250}
0 = \int \left( \PD{t}{\rho} + \spacegrad \cdot (\rho \Bu) \right) dV,
\end{equation}

so that in differential form we have

\begin{equation}\label{eqn:continuumMidTermReflection:270}
0 = \PD{t}{\rho} + \spacegrad \cdot (\rho \Bu).
\end{equation}

Expanding the divergence by chain rule we have

\begin{equation}\label{eqn:continuumMidTermReflection:290}
\PD{t}{\rho} +\Bu \cdot \spacegrad \rho = -\rho \spacegrad \cdot \Bu,
\end{equation}

but this is just

\begin{equation}\label{eqn:continuumMidTermReflection:310}
\frac{d\rho}{dt} = -\rho \spacegrad \cdot \Bu.
\end{equation}

So, for an incompressible fluid (one for which $d\rho/dt =0$), we must also have $\spacegrad \cdot \Bu = 0$.

\end{itemize}

\section{Problem 2.}
\subsection{Statement.}

Consider steady simple shearing flow $\Bu = \xcap u(y)$ as shown in figure (\ref{fig:continuumMidtermReflection:continuumMidtermReflectionFigQ1}) with imposed constant pressure gradient ($G = -dp/dx$), $G$ being a positive number, of a single layer fluid with viscosity $\mu$.

\begin{figure}[htp]
   \centering
   \includegraphics[totalheight=0.2\textheight]{continuumMidtermReflectionFigQ1}
   \caption{Shearing flow with pressure gradient and one moving boundary.}\label{fig:continuumMidtermReflection:continuumMidtermReflectionFigQ1}
\end{figure}

The boundary conditions are no-slip at the lower plate ($y = h$).  The top plate is moving with a velocity $-U$ at $y = h$ and fluid is sticking to it, so $u(h) = -U$, $U$ being a positive number.  Using the Navier-Stokes equation.

\begin{itemize}
\item derive the velocity profile of the fluid.
\item draw the velocity profile with the direction of the flow of the fluid when $U = 0$, $G \ne 0$.
\item draw the velocity profile with the direction of the flow of the fluid when $G = 0$, $U \ne 0$.
\item using linear superposition draw the velocity profile of the fluid with the dirrection of flow qualitatively when $U \ne 0$, $G \ne 0$. (i) low $U$, (ii) large $U$.
\item calculate the maximum speed when $U \ne 0$, $G \ne 0$.
\item calculate the flux (the volume flow rate) when $U \ne 0$, $G \ne 0$.
\item calculate the mean speed when $U \ne 0$, $G \ne 0$.
\item calcualte the tangential force (per unit width) $F_x$ on the strip $0 \le x \le L$ of the wall $y = -h$ when $U \ne 0$, $G \ne 0$.
\end{itemize}

\section{figures.}

figure (\ref{fig:continuumMidtermReflection:continuumMidtermReflectionFig3})
\begin{figure}[htp]
   \centering
   \includegraphics[totalheight=0.2\textheight]{continuumMidtermReflectionFig3}
   \caption{FIXME}\label{fig:continuumMidtermReflection:continuumMidtermReflectionFig3}
\end{figure}
figure (\ref{fig:continuumMidtermReflection:continuumMidtermReflectionFig4})
\begin{figure}[htp]
   \centering
   \includegraphics[totalheight=0.2\textheight]{continuumMidtermReflectionFig4}
   \caption{FIXME}\label{fig:continuumMidtermReflection:continuumMidtermReflectionFig4}
\end{figure}

\EndArticle
