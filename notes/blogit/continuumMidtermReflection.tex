%
% Copyright � 2015 Peeter Joot.  All Rights Reserved.
% Licenced as described in the file LICENSE under the root directory of this GIT repository.
%
\documentclass[]{eliblog}

\usepackage{amsmath}
\usepackage{mathpazo}

%
% shorthand for bold symbols, convenient for vectors and matrices
%
\newcommand{\Ba}[0]{\mathbf{a}}
\newcommand{\Bb}[0]{\mathbf{b}}
\newcommand{\Bc}[0]{\mathbf{c}}
\newcommand{\Bd}[0]{\mathbf{d}}
\newcommand{\Be}[0]{\mathbf{e}}
\newcommand{\Bf}[0]{\mathbf{f}}
\newcommand{\Bg}[0]{\mathbf{g}}
\newcommand{\Bh}[0]{\mathbf{h}}
\newcommand{\Bi}[0]{\mathbf{i}}
\newcommand{\Bj}[0]{\mathbf{j}}
\newcommand{\Bk}[0]{\mathbf{k}}
\newcommand{\Bl}[0]{\mathbf{l}}
\newcommand{\Bm}[0]{\mathbf{m}}
\newcommand{\Bn}[0]{\mathbf{n}}
\newcommand{\Bo}[0]{\mathbf{o}}
\newcommand{\Bp}[0]{\mathbf{p}}
\newcommand{\Bq}[0]{\mathbf{q}}
\newcommand{\Br}[0]{\mathbf{r}}
\newcommand{\Bs}[0]{\mathbf{s}}
\newcommand{\Bt}[0]{\mathbf{t}}
\newcommand{\Bu}[0]{\mathbf{u}}
\newcommand{\Bv}[0]{\mathbf{v}}
\newcommand{\Bw}[0]{\mathbf{w}}
\newcommand{\Bx}[0]{\mathbf{x}}
\newcommand{\By}[0]{\mathbf{y}}
\newcommand{\Bz}[0]{\mathbf{z}}
\newcommand{\BA}[0]{\mathbf{A}}
\newcommand{\BB}[0]{\mathbf{B}}
\newcommand{\BC}[0]{\mathbf{C}}
\newcommand{\BD}[0]{\mathbf{D}}
\newcommand{\BE}[0]{\mathbf{E}}
\newcommand{\BF}[0]{\mathbf{F}}
\newcommand{\BG}[0]{\mathbf{G}}
\newcommand{\BH}[0]{\mathbf{H}}
\newcommand{\BI}[0]{\mathbf{I}}
\newcommand{\BJ}[0]{\mathbf{J}}
\newcommand{\BK}[0]{\mathbf{K}}
\newcommand{\BL}[0]{\mathbf{L}}
\newcommand{\BM}[0]{\mathbf{M}}
\newcommand{\BN}[0]{\mathbf{N}}
\newcommand{\BO}[0]{\mathbf{O}}
\newcommand{\BP}[0]{\mathbf{P}}
\newcommand{\BQ}[0]{\mathbf{Q}}
\newcommand{\BR}[0]{\mathbf{R}}
\newcommand{\BS}[0]{\mathbf{S}}
\newcommand{\BT}[0]{\mathbf{T}}
\newcommand{\BU}[0]{\mathbf{U}}
\newcommand{\BV}[0]{\mathbf{V}}
\newcommand{\BW}[0]{\mathbf{W}}
\newcommand{\BX}[0]{\mathbf{X}}
\newcommand{\BY}[0]{\mathbf{Y}}
\newcommand{\BZ}[0]{\mathbf{Z}}

\newcommand{\Bzero}[0]{\mathbf{0}}
\newcommand{\Btheta}[0]{\boldsymbol{\theta}}
\newcommand{\Btau}[0]{\boldsymbol{\tau}}
\newcommand{\Bomega}[0]{\boldsymbol{\omega}}

%
% shorthand for unit vectors
%
\newcommand{\acap}[0]{\hat{\Ba}}
\newcommand{\bcap}[0]{\hat{\Bb}}
\newcommand{\ccap}[0]{\hat{\Bc}}
\newcommand{\dcap}[0]{\hat{\Bd}}
\newcommand{\ecap}[0]{\hat{\Be}}
\newcommand{\fcap}[0]{\hat{\Bf}}
\newcommand{\gcap}[0]{\hat{\Bg}}
\newcommand{\hcap}[0]{\hat{\Bh}}
\newcommand{\icap}[0]{\hat{\Bi}}
\newcommand{\jcap}[0]{\hat{\Bj}}
\newcommand{\kcap}[0]{\hat{\Bk}}
\newcommand{\lcap}[0]{\hat{\Bl}}
\newcommand{\mcap}[0]{\hat{\Bm}}
\newcommand{\ncap}[0]{\hat{\Bn}}
\newcommand{\ocap}[0]{\hat{\Bo}}
\newcommand{\pcap}[0]{\hat{\Bp}}
\newcommand{\qcap}[0]{\hat{\Bq}}
\newcommand{\rcap}[0]{\hat{\Br}}
\newcommand{\scap}[0]{\hat{\Bs}}
\newcommand{\tcap}[0]{\hat{\Bt}}
\newcommand{\ucap}[0]{\hat{\Bu}}
\newcommand{\vcap}[0]{\hat{\Bv}}
\newcommand{\wcap}[0]{\hat{\Bw}}
\newcommand{\xcap}[0]{\hat{\Bx}}
\newcommand{\ycap}[0]{\hat{\By}}
\newcommand{\zcap}[0]{\hat{\Bz}}
\newcommand{\thetacap}[0]{\hat{\Btheta}}

%
% to write R^n and C^n in a distinguishable fashion.  Perhaps change this
% to the double lined characters upon figuring out how to do so.
%
\newcommand{\C}[1]{$\mathbb{C}^{#1}$}
\newcommand{\R}[1]{$\mathbb{R}^{#1}$}

%
% various generally useful helpers
%

% derivative of #1 wrt. #2:
\newcommand{\D}[2] {\frac {d#2} {d#1}}

\newcommand{\inv}[1]{\frac{1}{#1}}
\newcommand{\cross}[0]{\times}

\newcommand{\abs}[1]{\lvert{#1}\rvert}
\newcommand{\norm}[1]{\lVert{#1}\rVert}
\newcommand{\innerprod}[2]{\langle{#1}, {#2}\rangle}
\newcommand{\dotprod}[2]{{#1} \cdot {#2}}
\newcommand{\bdotprod}[2]{\left({#1} \cdot {#2}\right)}
\newcommand{\crossprod}[2]{{#1} \cross {#2}}
\newcommand{\tripleprod}[3]{\dotprod{\left(\crossprod{#1}{#2}\right)}{#3}}

\DeclareMathOperator{\Proj}{Proj}
\DeclareMathOperator{\Span}{span}
\DeclareMathOperator{\Sgn}{sgn}
\DeclareMathOperator{\Area}{Area}
\DeclareMathOperator{\Volume}{Volume}

%
% A few miscellaneous things specific to this document
%
\newcommand{\crossop}[1]{\crossprod{#1}{}}

% R2 vector.
\newcommand{\VectorTwo}[2]{
\begin{bmatrix}
 {#1} \\
 {#2}
\end{bmatrix}
}

\newcommand{\VectorN}[1]{
\begin{bmatrix}
{#1}_1 \\
{#1}_2 \\
\vdots \\
{#1}_N \\
\end{bmatrix}
}

\newcommand{\DETuvij}[4]{
\begin{vmatrix}
 {#1}_{#3} & {#1}_{#4} \\
 {#2}_{#3} & {#2}_{#4}
\end{vmatrix}
}

\newcommand{\DETuvwijk}[6]{
\begin{vmatrix}
 {#1}_{#4} & {#1}_{#5} & {#1}_{#6} \\
 {#2}_{#4} & {#2}_{#5} & {#2}_{#6} \\
 {#3}_{#4} & {#3}_{#5} & {#3}_{#6}
\end{vmatrix}
}

\newcommand{\DETuvwxijkl}[8]{
\begin{vmatrix}
 {#1}_{#5} & {#1}_{#6} & {#1}_{#7} & {#1}_{#8} \\
 {#2}_{#5} & {#2}_{#6} & {#2}_{#7} & {#2}_{#8} \\
 {#3}_{#5} & {#3}_{#6} & {#3}_{#7} & {#3}_{#8} \\
 {#4}_{#5} & {#4}_{#6} & {#4}_{#7} & {#4}_{#8} \\
\end{vmatrix}
}

%\newcommand{\DETuvwxyijklm}[10]{
%\begin{vmatrix}
% {#1}_{#6} & {#1}_{#7} & {#1}_{#8} & {#1}_{#9} & {#1}_{#10} \\
% {#2}_{#6} & {#2}_{#7} & {#2}_{#8} & {#2}_{#9} & {#2}_{#10} \\
% {#3}_{#6} & {#3}_{#7} & {#3}_{#8} & {#3}_{#9} & {#3}_{#10} \\
% {#4}_{#6} & {#4}_{#7} & {#4}_{#8} & {#4}_{#9} & {#4}_{#10} \\
% {#5}_{#6} & {#5}_{#7} & {#5}_{#8} & {#5}_{#9} & {#5}_{#10}
%\end{vmatrix}
%}

% R3 vector.
\newcommand{\VectorThree}[3]{
\begin{bmatrix}
 {#1} \\
 {#2} \\
 {#3}
\end{bmatrix}
}



\author{Peeter Joot}
\email{peeter.joot@gmail.com}

%\documentclass[]{eliblogwidescreen}

\usepackage{amsmath}
\usepackage{mathpazo}

%
% shorthand for bold symbols, convenient for vectors and matrices
%
\newcommand{\Ba}[0]{\mathbf{a}}
\newcommand{\Bb}[0]{\mathbf{b}}
\newcommand{\Bc}[0]{\mathbf{c}}
\newcommand{\Bd}[0]{\mathbf{d}}
\newcommand{\Be}[0]{\mathbf{e}}
\newcommand{\Bf}[0]{\mathbf{f}}
\newcommand{\Bg}[0]{\mathbf{g}}
\newcommand{\Bh}[0]{\mathbf{h}}
\newcommand{\Bi}[0]{\mathbf{i}}
\newcommand{\Bj}[0]{\mathbf{j}}
\newcommand{\Bk}[0]{\mathbf{k}}
\newcommand{\Bl}[0]{\mathbf{l}}
\newcommand{\Bm}[0]{\mathbf{m}}
\newcommand{\Bn}[0]{\mathbf{n}}
\newcommand{\Bo}[0]{\mathbf{o}}
\newcommand{\Bp}[0]{\mathbf{p}}
\newcommand{\Bq}[0]{\mathbf{q}}
\newcommand{\Br}[0]{\mathbf{r}}
\newcommand{\Bs}[0]{\mathbf{s}}
\newcommand{\Bt}[0]{\mathbf{t}}
\newcommand{\Bu}[0]{\mathbf{u}}
\newcommand{\Bv}[0]{\mathbf{v}}
\newcommand{\Bw}[0]{\mathbf{w}}
\newcommand{\Bx}[0]{\mathbf{x}}
\newcommand{\By}[0]{\mathbf{y}}
\newcommand{\Bz}[0]{\mathbf{z}}
\newcommand{\BA}[0]{\mathbf{A}}
\newcommand{\BB}[0]{\mathbf{B}}
\newcommand{\BC}[0]{\mathbf{C}}
\newcommand{\BD}[0]{\mathbf{D}}
\newcommand{\BE}[0]{\mathbf{E}}
\newcommand{\BF}[0]{\mathbf{F}}
\newcommand{\BG}[0]{\mathbf{G}}
\newcommand{\BH}[0]{\mathbf{H}}
\newcommand{\BI}[0]{\mathbf{I}}
\newcommand{\BJ}[0]{\mathbf{J}}
\newcommand{\BK}[0]{\mathbf{K}}
\newcommand{\BL}[0]{\mathbf{L}}
\newcommand{\BM}[0]{\mathbf{M}}
\newcommand{\BN}[0]{\mathbf{N}}
\newcommand{\BO}[0]{\mathbf{O}}
\newcommand{\BP}[0]{\mathbf{P}}
\newcommand{\BQ}[0]{\mathbf{Q}}
\newcommand{\BR}[0]{\mathbf{R}}
\newcommand{\BS}[0]{\mathbf{S}}
\newcommand{\BT}[0]{\mathbf{T}}
\newcommand{\BU}[0]{\mathbf{U}}
\newcommand{\BV}[0]{\mathbf{V}}
\newcommand{\BW}[0]{\mathbf{W}}
\newcommand{\BX}[0]{\mathbf{X}}
\newcommand{\BY}[0]{\mathbf{Y}}
\newcommand{\BZ}[0]{\mathbf{Z}}

\newcommand{\Bzero}[0]{\mathbf{0}}
\newcommand{\Btheta}[0]{\boldsymbol{\theta}}
\newcommand{\Btau}[0]{\boldsymbol{\tau}}
\newcommand{\Bomega}[0]{\boldsymbol{\omega}}

%
% shorthand for unit vectors
%
\newcommand{\acap}[0]{\hat{\Ba}}
\newcommand{\bcap}[0]{\hat{\Bb}}
\newcommand{\ccap}[0]{\hat{\Bc}}
\newcommand{\dcap}[0]{\hat{\Bd}}
\newcommand{\ecap}[0]{\hat{\Be}}
\newcommand{\fcap}[0]{\hat{\Bf}}
\newcommand{\gcap}[0]{\hat{\Bg}}
\newcommand{\hcap}[0]{\hat{\Bh}}
\newcommand{\icap}[0]{\hat{\Bi}}
\newcommand{\jcap}[0]{\hat{\Bj}}
\newcommand{\kcap}[0]{\hat{\Bk}}
\newcommand{\lcap}[0]{\hat{\Bl}}
\newcommand{\mcap}[0]{\hat{\Bm}}
\newcommand{\ncap}[0]{\hat{\Bn}}
\newcommand{\ocap}[0]{\hat{\Bo}}
\newcommand{\pcap}[0]{\hat{\Bp}}
\newcommand{\qcap}[0]{\hat{\Bq}}
\newcommand{\rcap}[0]{\hat{\Br}}
\newcommand{\scap}[0]{\hat{\Bs}}
\newcommand{\tcap}[0]{\hat{\Bt}}
\newcommand{\ucap}[0]{\hat{\Bu}}
\newcommand{\vcap}[0]{\hat{\Bv}}
\newcommand{\wcap}[0]{\hat{\Bw}}
\newcommand{\xcap}[0]{\hat{\Bx}}
\newcommand{\ycap}[0]{\hat{\By}}
\newcommand{\zcap}[0]{\hat{\Bz}}
\newcommand{\thetacap}[0]{\hat{\Btheta}}

%
% to write R^n and C^n in a distinguishable fashion.  Perhaps change this
% to the double lined characters upon figuring out how to do so.
%
\newcommand{\C}[1]{$\mathbb{C}^{#1}$}
\newcommand{\R}[1]{$\mathbb{R}^{#1}$}

%
% various generally useful helpers
%

% derivative of #1 wrt. #2:
\newcommand{\D}[2] {\frac {d#2} {d#1}}

\newcommand{\inv}[1]{\frac{1}{#1}}
\newcommand{\cross}[0]{\times}

\newcommand{\abs}[1]{\lvert{#1}\rvert}
\newcommand{\norm}[1]{\lVert{#1}\rVert}
\newcommand{\innerprod}[2]{\langle{#1}, {#2}\rangle}
\newcommand{\dotprod}[2]{{#1} \cdot {#2}}
\newcommand{\bdotprod}[2]{\left({#1} \cdot {#2}\right)}
\newcommand{\crossprod}[2]{{#1} \cross {#2}}
\newcommand{\tripleprod}[3]{\dotprod{\left(\crossprod{#1}{#2}\right)}{#3}}

\DeclareMathOperator{\Proj}{Proj}
\DeclareMathOperator{\Span}{span}
\DeclareMathOperator{\Sgn}{sgn}
\DeclareMathOperator{\Area}{Area}
\DeclareMathOperator{\Volume}{Volume}

%
% A few miscellaneous things specific to this document
%
\newcommand{\crossop}[1]{\crossprod{#1}{}}

% R2 vector.
\newcommand{\VectorTwo}[2]{
\begin{bmatrix}
 {#1} \\
 {#2}
\end{bmatrix}
}

\newcommand{\VectorN}[1]{
\begin{bmatrix}
{#1}_1 \\
{#1}_2 \\
\vdots \\
{#1}_N \\
\end{bmatrix}
}

\newcommand{\DETuvij}[4]{
\begin{vmatrix}
 {#1}_{#3} & {#1}_{#4} \\
 {#2}_{#3} & {#2}_{#4}
\end{vmatrix}
}

\newcommand{\DETuvwijk}[6]{
\begin{vmatrix}
 {#1}_{#4} & {#1}_{#5} & {#1}_{#6} \\
 {#2}_{#4} & {#2}_{#5} & {#2}_{#6} \\
 {#3}_{#4} & {#3}_{#5} & {#3}_{#6}
\end{vmatrix}
}

\newcommand{\DETuvwxijkl}[8]{
\begin{vmatrix}
 {#1}_{#5} & {#1}_{#6} & {#1}_{#7} & {#1}_{#8} \\
 {#2}_{#5} & {#2}_{#6} & {#2}_{#7} & {#2}_{#8} \\
 {#3}_{#5} & {#3}_{#6} & {#3}_{#7} & {#3}_{#8} \\
 {#4}_{#5} & {#4}_{#6} & {#4}_{#7} & {#4}_{#8} \\
\end{vmatrix}
}

%\newcommand{\DETuvwxyijklm}[10]{
%\begin{vmatrix}
% {#1}_{#6} & {#1}_{#7} & {#1}_{#8} & {#1}_{#9} & {#1}_{#10} \\
% {#2}_{#6} & {#2}_{#7} & {#2}_{#8} & {#2}_{#9} & {#2}_{#10} \\
% {#3}_{#6} & {#3}_{#7} & {#3}_{#8} & {#3}_{#9} & {#3}_{#10} \\
% {#4}_{#6} & {#4}_{#7} & {#4}_{#8} & {#4}_{#9} & {#4}_{#10} \\
% {#5}_{#6} & {#5}_{#7} & {#5}_{#8} & {#5}_{#9} & {#5}_{#10}
%\end{vmatrix}
%}

% R3 vector.
\newcommand{\VectorThree}[3]{
\begin{bmatrix}
 {#1} \\
 {#2} \\
 {#3}
\end{bmatrix}
}



\author{Peeter Joot}
\email{peeter.joot@gmail.com}


\chapter{PHY454H1S Continuum mechanics midterm reflection.}
\label{chap:continuumMidtermReflection}
%\useCCL
\blogpage{http://sites.google.com/site/peeterjoot2/math2012/continuumMidtermReflection.pdf}
\date{Mar 14, 2012}
\gitRevisionInfo{continuumMidtermReflection}

\beginArtWithToc
%\beginArtNoToc

\section{Motivation.}

I didn't manage my time well enough on the midterm to complete it (and also missed one easy part of the second question).  For later review purposes, here is either what I answered, or what I think I should have answered for these questions.

\section{Problem 1.}

\subsection{$\BP$-waves, $\BS$-waves, and Love-waves.}

\begin{itemize}
\item Show that in $\BP$-waves the divergence of the displacement vector represents a measure of the relative change in the volume of the body.

\paragraph{Answer.}
The $\BP$-wave equation was a result of operating on the displacement equation with the divergence operator

\begin{equation}\label{eqn:continuumMidTermReflection:10}
\spacegrad \cdot \left( 
\rho \PDSq{t}{\Be} = (\lambda + \mu) \spacegrad (\spacegrad \cdot \Be) + \mu \spacegrad^2 \Be
\right)
\end{equation}

we obtain

\begin{equation}\label{eqn:continuumMidTermReflection:30}
\PDSq{t}{} \left( \spacegrad \cdot \Be \right) = \frac{\lambda + 2 \mu}{\rho} \spacegrad^2 (\spacegrad \cdot \Be).
\end{equation}

We have a wave equation where the ``waving'' quantity is $\Theta = \spacegrad \cdot \Be$.  Explicitly

\begin{align*}
\Theta 
&= \spacegrad \cdot \Be \\
&= 
\PD{x}{e_1}
+\PD{y}{e_2}
+\PD{z}{e_3}
\end{align*}

Recall that, in a coordinate basis for which the strain $e_{ij}$ is diagonal we have

\begin{align}\label{eqn:continuumMidTermReflection:50}
dx' &= \sqrt{1 + 2 e_{11}} dx \\
dy' &= \sqrt{1 + 2 e_{22}} dy \\
dz' &= \sqrt{1 + 2 e_{33}} dz.
\end{align}

Expanding in Taylor series to $O(1)$ we have for $i = 1, 2, 3$ (no sum)

\begin{equation}\label{eqn:continuumMidTermReflection:70}
dx_i' \approx (1 + e_{ii}) dx_i.
\end{equation}

so the displaced volume is

\begin{align*}
dV' &= 
dx_1
dx_2
dx_3
(1 + e_{11})
(1 + e_{22})
(1 + e_{33}) \\
&=
dx_1
dx_2
dx_3
( 1  + e_{11} + e_{22} + e_{33} + O(e_{kk}^2) )
\end{align*}

Since 

\begin{align}\label{eqn:continuumMidTermReflection:90}
e_{11} &= \inv{2} \left( \PD{x}{e_1} +\PD{x}{e_1} \right) = \PD{x}{e_1} \\
e_{22} &= \inv{2} \left( \PD{y}{e_2} +\PD{y}{e_2} \right) = \PD{y}{e_2} \\
e_{33} &= \inv{2} \left( \PD{z}{e_3} +\PD{z}{e_3} \right) = \PD{z}{e_3}
\end{align}

We have

\begin{equation}\label{eqn:continuumMidTermReflection:110}
dV' = (1 + \spacegrad \cdot \Be) dV,
\end{equation}

or

\begin{equation}\label{eqn:continuumMidTermReflection:130}
\frac{dV' - dV}{dV} = \spacegrad \cdot \Be
\end{equation}

The relative change in volume can therefore be expressed as the divergence of $\Be$, the displacement vector, and it is this relative volume change that is ``waving'' in the $\BP$-wave equation as illustrated in the following (\ref{fig:continuumMidtermReflection:continuumMidtermReflectionFig1}) sample 1D compression wave

\begin{figure}[htp]
   \centering
   \includegraphics[totalheight=0.2\textheight]{continuumMidtermReflectionFig1}
   \caption{A 1D compression wave.}\label{fig:continuumMidtermReflection:continuumMidtermReflectionFig1}
\end{figure}

\item Between a $\BP$-wave and an $\BS$-wave which one is longitudinal and which one is transverse?

\paragraph{Answer.}
$\BP$-waves are longitudinal.

$\BS$-waves are transverse.

\item Whose speed is higher?

\paragraph{Answer.}
From the formula sheet we have

\begin{align*}
\left( \frac{c_L}{c_T} \right)^2 
&= \frac{ \lambda + 2 \mu}{\rho} \frac{\rho}{\mu}  \\
&= \frac{\lambda}{\mu} + 2  \\
&> 1
\end{align*}

so $\BP$-waves travel faster than $\BS$-waves.

\item Is Love wave a body wave or a surface wave?

\paragraph{Answer.}

Love waves are surface waves, traveling in a medium that can slide on top of another surface.  These are characterized by vorticity rotating backwards compared to the direction of propagation as shown in figure (\ref{fig:continuumMidtermReflection:continuumMidtermReflectionFig2})

\begin{figure}[htp]
   \centering
   \def\svgwidth{0.6\columnwidth}
   \input{continuumMidtermReflectionFig2.pdf_tex}
   \caption{Love wave illustrated.}\label{fig:continuumMidtermReflection:continuumMidtermReflectionFig2}
\end{figure}
\end{itemize}

\subsection{(b) constitutive relation, Newtonian fluids, and no-slip conditions.}

\begin{itemize}
\item In continuum mechanics what do you mean by \textit{constitutive relation}?
\paragraph{Answer.}  The constitutive relation is the stress-strain relation, generally

\begin{equation}\label{eqn:continuumMidTermReflection:150}
\sigma_{ij} = c_{abij} e_{ab}
\end{equation}

for isotropic solids we model this as 

\begin{equation}\label{eqn:continuumMidTermReflection:170}
\sigma_{ij} = \lambda e_{kk} \delta_{ij} + 2 \mu e_{ij}
\end{equation}

and for Newtonian fluids

\begin{equation}\label{eqn:continuumMidTermReflection:190}
\sigma_{ij} = -p \delta_{ij} + 2 \mu e_{ij}
\end{equation}

\item What is the definition of a Non-Newtonian fluid?
\paragraph{Answer.}

A non-Newtonian fluid would be one with a more general constitutive relationship.

\paragraph{Grading note.}  I lost a mark here.  I think the answer that was being looked for (as in \cite{wiki:newtonianFluids}) was that a Newtonian fluid is one with a linear stress strain relationship, and a non-Newtonian fluid would be one with a non-linear relationship.  According to \cite{wiki:nonNewtonianFluid} an example of a non-Newtonian material that we are all familiar with is Silly Putty.  This linearity is also how a Newtonian fluid was defined in the notes, but I didn't remember that (this isn't really something we use since we assume all fluids and materials are Newtonian in any calculations that we do).

\item What do you mean by \emph{no-slip} boundary condition at a fluid-fluid interface?

\paragraph{Exam time management note.} Somehow in my misguided attempt to be complete, I missed this question amongst the rest of my verbosity).

\paragraph{Answer.}
The no slip boundary condition is just one of velocity matching.  At a non-moving boundary, the no-slip condition means that we'll require the fluid to also have no velocity (ie. at that interface the fluid isn't slipping over the surface).  Between two fluids, this is a requirement that the velocities of both fluids match at that point (and all the rest of the points along the region of the interaction.)

\item Write down the continuity equation for an incompressible fluid.
\paragraph{Answer.}

An incompressible fluid has

\begin{equation}\label{eqn:continuumMidTermReflection:210}
\frac{d\rho}{dt} = 0,
\end{equation}

but since we also have

\begin{align*}
0 
&=
\frac{d\rho}{dt} \\
&= - \rho (\spacegrad \cdot \Bu)  \\
&= 
\PD{t}{\rho} + (\Bu \cdot \spacegrad) \rho \\
&= 0.
\end{align*}

A consequence is that $\spacegrad \cdot \Bu = 0$ for an incompressible fluid.  Let's recall where this statement comes from.  Looking at mass conservation, the rate that mass leaves a volume can be expressed as

\begin{align*}
\frac{dm}{dt}
&= \int \frac{d\rho}{dt} dV \\
&= -\int_{\partial V} \rho \Bu \cdot d\BA \\
&= -\int_V \spacegrad \cdot (\rho \Bu) dV
\end{align*}

(the minus sign here signifying that the mass is leaving the volume through the surface, and that we are using an outwards facing normal on the volume.)

If the surface bounding the volume doesn't change with time (ie. $\PDi{t}{V} = 0$) we can write

\begin{equation}\label{eqn:continuumMidTermReflection:230}
\PD{t}{} \int \rho dV = -\int \spacegrad \cdot (\rho \Bu) dV,
\end{equation}

or

\begin{equation}\label{eqn:continuumMidTermReflection:250}
0 = \int \left( \PD{t}{\rho} + \spacegrad \cdot (\rho \Bu) \right) dV,
\end{equation}

so that in differential form we have

\begin{equation}\label{eqn:continuumMidTermReflection:270}
0 = \PD{t}{\rho} + \spacegrad \cdot (\rho \Bu).
\end{equation}

Expanding the divergence by chain rule we have

\begin{equation}\label{eqn:continuumMidTermReflection:290}
\PD{t}{\rho} +\Bu \cdot \spacegrad \rho = -\rho \spacegrad \cdot \Bu,
\end{equation}

but this is just

\begin{equation}\label{eqn:continuumMidTermReflection:310}
\frac{d\rho}{dt} = -\rho \spacegrad \cdot \Bu.
\end{equation}

So, for an incompressible fluid (one for which $d\rho/dt =0$), we must also have $\spacegrad \cdot \Bu = 0$.

\end{itemize}

\section{Problem 2.}
\subsection{Statement.}

Consider steady simple shearing flow $\Bu = \xcap u(y)$ as shown in figure (\ref{fig:continuumMidtermReflection:continuumMidtermReflectionFigQ1}) with imposed constant pressure gradient ($G = -dp/dx$), $G$ being a positive number, of a single layer fluid with viscosity $\mu$.

\begin{figure}[htp]
   \centering
   \includegraphics[totalheight=0.2\textheight]{continuumMidtermReflectionFigQ1}
   \caption{Shearing flow with pressure gradient and one moving boundary.}\label{fig:continuumMidtermReflection:continuumMidtermReflectionFigQ1}
\end{figure}

The boundary conditions are no-slip at the lower plate ($y = h$).  The top plate is moving with a velocity $-U$ at $y = h$ and fluid is sticking to it, so $u(h) = -U$, $U$ being a positive number.  Using the Navier-Stokes equation.

\begin{itemize}
\item Derive the velocity profile of the fluid.

\paragraph{Answer}

Our equations of motion are

\begin{subequations}
\begin{equation}\label{eqn:continuumMidTermReflection:330}
0 = \spacegrad \cdot \Bu
\end{equation}
\begin{equation}\label{eqn:continuumMidTermReflection:350}
\cancel{\rho \PD{t}{\Bu}} + (\Bu \cdot \spacegrad) \Bu = - \spacegrad p + \mu \spacegrad (\cancel{\spacegrad \cdot \Bu}) + \mu \spacegrad^2 \Bu + \cancel{\rho \Bg}
\end{equation}
\end{subequations}

Here, we've used the steady state condition and are neglecting gravity, and kill off our mass compression term with the incompressibility assumption.  In component form, what we have left is

\begin{align}\label{eqn:continuumMidTermReflection:370}
0 &= \partial_x u \\
u \cancel{\partial_x u} &= -\partial_x p + \mu \spacegrad^2 u \\
0 &= -\partial_y p \\
0 &= -\partial_z p
\end{align}

with $\partial_y p = \partial_z p = 0$, we must have

\begin{equation}\label{eqn:continuumMidTermReflection:390}
\PD{x}{p} = \frac{dp}{dx} = -G,
\end{equation}

which leaves us with just

\begin{align*}
0 
&= G + \mu \spacegrad^2 u(y)  \\
&= G + \mu \PDSq{y}{u} \\
&= G + \mu \frac{d^2 u}{dy^2}.
\end{align*}

Having dropped the partials we really just want to integrate our very simple ODE a couple times

\begin{equation}\label{eqn:continuumMidTermReflection:430}
u'' = -\frac{G}{\mu}.
\end{equation}

Integrate once

\begin{equation}\label{eqn:continuumMidTermReflection:450}
u' = -\frac{G}{\mu} y + \frac{A}{h},
\end{equation}

and once more to find the velocity

\begin{equation}\label{eqn:continuumMidTermReflection:470}
u = -\frac{G}{2 \mu} y^2 + \frac{A}{h} y + B'.
\end{equation}

Let's incorporate an additional constant into $B'$

\begin{equation}\label{eqn:continuumMidTermReflection:490}
B' = \frac{G}{2 \mu} h^2 + B
\end{equation}

so that we have

\begin{equation}\label{eqn:continuumMidTermReflection:510}
u = \frac{G}{2 \mu} (h^2 - y^2) + \frac{A}{h} y + B.
\end{equation}

(I didn't do use $B'$ this way on the exam, nor did I include the factor of $1/h$ in the first integration constant, but both of these should simplify the algebra since we'll be evaluating the boundary value conditions at $y = \pm h$.)

\begin{equation}\label{eqn:continuumMidTermReflection:530}
u = \frac{G}{2 \mu} (h^2 - y^2) + \frac{A}{h} y + B
\end{equation}

Applying the velocity matching conditions we have for the lower and upper plates respectively

\begin{align}\label{eqn:continuumMidTermReflection:550}
0 &= \frac{A}{h} (-h) + B \\
-U &= \frac{A}{h} (h) + B
\end{align}

Adding these we find

\begin{equation}\label{eqn:continuumMidTermReflection:570}
B = -\frac{U}{2}
\end{equation}

and subtracting find

\begin{equation}\label{eqn:continuumMidTermReflection:590}
A = -\frac{U}{2}.
\end{equation}

Our velocity is

\begin{equation}\label{eqn:continuumMidTermReflection:610}
u = \frac{G}{2 \mu} (h^2 - y^2) - \frac{U}{2 h} y -\frac{U}{2}
\end{equation}

or rearranged a bit

\begin{equation}\label{eqn:continuumMidTermReflection:630}
\boxed{
u(y) = \frac{G}{2 \mu} (h^2 - y^2) - \frac{U}{2} \left( 1 + \frac{y}{h} \right)
}
\end{equation}

\item Draw the velocity profile with the direction of the flow of the fluid when $U = 0$, $G \ne 0$.

\paragraph{Answer}

With $U = 0$ our velocity has a simple parabolic profile with a max of $\frac{G}{2 \mu} (h^2 - y^2)$ at $y = 0$

\begin{equation}\label{eqn:continuumMidTermReflection:650}
u(y) = \frac{G}{2 \mu} (h^2 - y^2).
\end{equation}

This is plotted in figure (\ref{fig:continuumMidtermReflection:continuumMidtermReflectionFig3})
\begin{figure}[htp]
   \centering
   \includegraphics[totalheight=0.2\textheight]{continuumMidtermReflectionFig3}
   \caption{Parabolic velocity profile.}\label{fig:continuumMidtermReflection:continuumMidtermReflectionFig3}
\end{figure}

\item Draw the velocity profile with the direction of the flow of the fluid when $G = 0$, $U \ne 0$.

\paragraph{Answer}

With $G = 0$, we have a plain old shear flow

\begin{equation}\label{eqn:continuumMidTermReflection:670}
u(y) = - \frac{U}{2} \left( 1 + \frac{y}{h} \right).
\end{equation}

This is linear with minimum velocity $u = 0$ at $y = -h$, and a maximum of $-U$ at $y = h$.  This is plotted in figure (\ref{fig:continuumMidtermReflection:continuumMidtermReflectionFig4})
\begin{figure}[htp]
   \centering
   \includegraphics[totalheight=0.2\textheight]{continuumMidtermReflectionFig4}
   \caption{Shear flow.}\label{fig:continuumMidtermReflection:continuumMidtermReflectionFig4}
\end{figure}

\item Using linear superposition draw the velocity profile of the fluid with the direction of flow qualitatively when $U \ne 0$, $G \ne 0$. (i) low $U$, (ii) large $U$.
\paragraph{Exam time management note.} Somehow I missed this question when I wrote the exam ... I figured this out right at the end when I'd run out of time by being too verbose elsewhere.  I'm really not very good at writing exams in tight time constraints anymore.

\paragraph{Answer}

For low $U$ we'll let the parabolic dominate, and can graphically add these two as in figure (\ref{fig:continuumMidtermReflection:continuumMidtermReflectionFig5})
\begin{figure}[htp]
   \centering
   \includegraphics[totalheight=0.2\textheight]{continuumMidtermReflectionFig5}
   \caption{Superposition of shear and parabolic flow (low $U$)}\label{fig:continuumMidtermReflection:continuumMidtermReflectionFig5}
\end{figure}
For high $U$, we'll let the shear flow dominate, and have plotted this in figure (\ref{fig:continuumMidtermReflection:continuumMidtermReflectionFig6})
\begin{figure}[htp]
   \centering
   \includegraphics[totalheight=0.2\textheight]{continuumMidtermReflectionFig6}
   \caption{Superposition of shear and parabolic flow (high $U$)}\label{fig:continuumMidtermReflection:continuumMidtermReflectionFig6}
\end{figure}

\item Calculate the maximum speed when $U \ne 0$, $G \ne 0$.
\paragraph{Answer}

Since our acceleration is

\begin{equation}\label{eqn:continuumMidTermReflection:690}
\frac{du}{dy} = -\frac{G}{\mu} y - \frac{U}{2 h}
\end{equation}

our extreme values occur at 

\begin{equation}\label{eqn:continuumMidTermReflection:710}
y_m = -\frac{U \mu}{2 h G}.
\end{equation}

At this point, our velocity is

\begin{align*}
u(y_m) 
&= 
\frac{G}{2 \mu} \left(h^2 - 
\left( \frac{U \mu}{2 h G} \right)^2
\right) - \frac{U}{2} \left( 1 
-\frac{U \mu}{2 h^2 G}
\right) \\
&=
\frac{G h^2}{2 \mu} -\frac{U}{2}
+ \frac{U^2 \mu}{4 h^2 G} \left(
1 -\inv{2}
\right)
\end{align*}

or just

\begin{equation}\label{eqn:continuumMidTermReflection:730}
u_{\text{max}} = \frac{G h^2}{2 \mu} -\frac{U}{2} + \frac{U^2 \mu}{8 h^2 G}.
\end{equation}

\item Calculate the flux (the volume flow rate) when $U \ne 0$, $G \ne 0$.
\paragraph{Answer}

An element of our volume flux is

\begin{equation}\label{eqn:continuumMidTermReflection:750}
\frac{dV}{dt} = dy dz \Bu \cdot \xcap
\end{equation}

Looking at the volume flux through the width $\Delta z$ is then

\begin{align*}
\text{Flux} 
&= \int_0^{\Delta z} dz \int_{-h}^h dy u(y) \\
&= \Delta z 
\int_{-h}^h dy 
\frac{G}{2 \mu} (h^2 - y^2) - \frac{U}{2} \left( 1 + \frac{y}{h} \right) \\
&= \Delta z 
\int_{-h}^h dy 
\frac{G}{2 \mu} \left(h^2 y - \inv{3} y^3 \right) - \frac{U}{2} \left( y + \frac{y^2}{2 h} \right) \\
&= \Delta z 
\left( \frac{2 G h^3}{3 \mu} - U h \right)
\end{align*}

\item Calculate the mean speed when $U \ne 0$, $G \ne 0$.
\paragraph{Answer}

\paragraph{Exam time management note.}  I squandered too much time on other stuff and didn't get to this part of the problem (which was unfortunately worth a lot).  This is how I think it should have been answered.

We've done most of the work above, and just have to divide the flux by $2 h \Delta z$.  That is

\begin{equation}\label{eqn:continuumMidTermReflection:770}
\expectation{u} = \frac{G h^2}{3 \mu} - \frac{U}{2}.
\end{equation}

\item Calculate the tangential force (per unit width) $F_x$ on the strip $0 \le x \le L$ of the wall $y = -h$ when $U \ne 0$, $G \ne 0$.
\paragraph{Answer}

Our traction vector is

\begin{align*}
T_1 
&= \sigma_{1j} n_j \\
&= \left( -p \delta_{1j} + 2 \mu e_{1j} \right) \delta_{2j} \\
&= 2 \mu e_{12} \\
&= \mu \left( 
\PD{y}{u}
+
\cancel{\PD{x}{v}}
\right)
\end{align*}

So the $\xcap$ directed component of the traction vector is just

\begin{equation}\label{eqn:continuumMidTermReflection:790}
T_1 = \mu \PD{y}{u}.
\end{equation}

We've calculated that derivative above in \ref{eqn:continuumMidTermReflection:690}, so we have

\begin{align*}
T_1 
&= \mu \left( -\frac{G}{\mu} y - \frac{U}{2 h} \right) \\
&= - G y - \frac{U \mu}{2 h} 
\end{align*}

so at $y = -h$ we have

\begin{equation}\label{eqn:continuumMidTermReflection:810}
T_1(-h) = G h - \frac{U \mu}{2 h}.
\end{equation}

To see the contribution of this force on the lower wall over an interval of length $L$ we integrate, but this amounts to just multiplying by the length of the segment of the wall

\begin{equation}\label{eqn:continuumMidTermReflection:830}
\int_0^L T_1(-h) dx = \left( G h - \frac{U \mu}{2 h} \right) L.
\end{equation}

\end{itemize}
\EndArticle
