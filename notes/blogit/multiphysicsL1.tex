%
% Copyright � 2014 Peeter Joot.  All Rights Reserved.
% Licenced as described in the file LICENSE under the root directory of this GIT repository.
%
\newcommand{\authorname}{Peeter Joot}
\newcommand{\email}{peeterjoot@protonmail.com}
\newcommand{\basename}{FIXMEbasenameUndefined}
\newcommand{\dirname}{notes/FIXMEdirnameUndefined/}

\renewcommand{\basename}{multiphysicsLecture1}
\renewcommand{\dirname}{notes/ece1254/}
\newcommand{\keywords}{Condensed matter physics, ECE1254H}
\newcommand{\authorname}{Peeter Joot}
\newcommand{\onlineurl}{http://sites.google.com/site/peeterjoot2/math2013/\basename.pdf}
\newcommand{\sourcepath}{\dirname\basename.tex}
\newcommand{\generatetitle}[1]{\chapter{#1}}

\newcommand{\vcsinfo}{%
\section*{}
\noindent{\color{DarkOliveGreen}{\rule{\linewidth}{0.1mm}}}
\paragraph{Document version}
%\paragraph{\color{Maroon}{Document version}}
{
\small
\begin{itemize}
\item Available online at:\\ 
\href{\onlineurl}{\onlineurl}
\item Git Repository: \input{./.revinfo/gitRepo.tex}
\item Source: \sourcepath
\item last commit: \input{./.revinfo/gitCommitString.tex}
\item commit date: \input{./.revinfo/gitCommitDate.tex}
\end{itemize}
}
}

%\PassOptionsToPackage{dvipsnames,svgnames}{xcolor}
\PassOptionsToPackage{square,numbers}{natbib}
\documentclass{scrreprt}

\usepackage[left=2cm,right=2cm]{geometry}
\usepackage[svgnames]{xcolor}
\usepackage{peeters_layout}

\usepackage{natbib}

\usepackage[
colorlinks=true,
bookmarks=false,
pdfauthor={\authorname, \email},
backref 
]{hyperref}

% http://tex.stackexchange.com/questions/75773/how-to-reference-problems-by-the-text-label-in-an-exercise-envioronment
\usepackage[english]{cleveref}
\crefname{Exercise}{exercise}{exercises}
\Crefname{Exercise}{Exercise}{Exercises}

\RequirePackage{titlesec}
\RequirePackage{ifthen}

% http://stackoverflow.com/questions/4932910/date-in-the-tabular-environment
\makeatletter
\let\insertdate\@date
\makeatother

\titleformat{\chapter}[display]
{\bfseries\Large}
{\color{DarkSlateGrey}\filleft \authorname
\ifthenelse{\isundefined{\studentnumber}}{}{\\ \studentnumber}
\ifthenelse{\isundefined{\email}}{}{\\ \email}
\ifthenelse{\isundefined{\dateintitle}}{}{\\ \insertdate}
%\ifthenelse{\isundefined{\coursename}}{}{\\ \coursename} % put in title instead.
}
{4ex}
{\color{DarkOliveGreen}{\titlerule}\color{Maroon}
\vspace{2ex}%
\filright}
[\vspace{2ex}%
\color{DarkOliveGreen}\titlerule
]

\newcommand{\beginArtWithToc}[0]{\begin{document}\tableofcontents}
\newcommand{\beginArtNoToc}[0]{\begin{document}}
\newcommand{\EndNoBibArticle}[0]{\end{document}}
\newcommand{\EndArticle}[0]{\bibliography{Bibliography}\bibliographystyle{plainnat}\end{document}}

% 
%\newcommand{\citep}[1]{\cite{#1}}

\colorSectionsForArticle



%\citep{harald2003solid} \S x.y
%\citep{ibach2009solid} \S x.y
%\reading \citep{ashcroft1976solid} \ch N.

%\usepackage{mhchem}
%\usepackage[version=3]{mhchem}
%\usepackage{units}
%\usepackage{bm} % \EE

\beginArtNoToc
\generatetitle{ECE1254H Modeling of Multiphysics Systems.  Lecture 1: Analogies to circuit systems.  Taught by Prof.\ Piero Triverio}
%\chapter{Analogies to circuit systems}
\label{chap:multiphysicsLecture1}

%\section{Disclaimer}
%
%Peeter's lecture notes from class.  May not be entirely coherent.

\section{In slides}

A review of systematic nodal analysis for a basic resistive circuit was outlined in slides.  We then proceeded to show how many similar linear systems can be modeled as circuits so that the same toolbox can be applied.  This included blood flow through a body (and blood flow to the brain), a model of antenna interference in a portable phone, heat conduction in a one dimensional conductor under a heat lamp, and a few other systems.

This discussion reminded me of the joke where the farmer, the butcher and the physicist are all invited to talk at a beef convention.  After meaningful and appropriate talks by the farmer and the butcher, the physicist gets his chance, and proceeds with ``We begin by modeling the cow as a sphere, ...''.  The ECE equivalent of that appears to be a Kirchhoff circuit problem.

\section{Mechanical structures example}

Continuing the application of circuits like linear systems to other systems, let's consider a truss system as illustrated in

F1, F2

Our unknowns are

\begin{itemize}
	\item positions of the joints after deformation $(x_i, y_i)$.
	\item force acting on each strut $\BF_j = (F_{j,x}, F_{j,y})$.
\end{itemize}

The constitutive equations, assuming static conditions (steady state, no transients)

\begin{itemize}
	\item Load force.  $\BF_L = (F_{L, x}, F_{L, y}) = (0, -m g)$.
	\item Strut forces.  Under static conditions the total resulting force on the strut is zero, so $\BF'_j = -\BF_j$.  For this problem it is redundant to label forces on both ends, so we mark the labeled end of the object with an asterisk.
\end{itemize}

\paragraph{Consider a simple case}

One strut as in F4

\begin{equation}\label{eqn:multiphysicsL1:n}
\BF^\conj = - \Ba_x 
\mathLabelBox
[
   labelstyle={yshift=1.2em},
   linestyle={}
]
{
\epsilon 
}
{constant}
\biglr{
\mathLabelBox
[
   labelstyle={below of=m\themathLableNode, below of=m\themathLableNode}
]
{
L 
}
{actual length $L = \Abs{x^\conj - 0}$}
- L_0}
\end{equation}

The constitutive law for a strut is

F5

direction:

\begin{equation}\label{eqn:multiphysicsL1:n}
\Be = \frac{\Br^\conj - \Br}{\Abs{\Br^\conj - \Br}}
\end{equation}

force:
\begin{equation}\label{eqn:multiphysicsL1:n}
	\BF^\conj = - \Be \epsilon \lr{ L - L_0 }
\end{equation}

values $\epsilon, L_0$ given by 

\begin{equation}\label{eqn:multiphysicsL1:n}
	L = \Abs{\Br^\conj - \Br} = \sqrt{(x^\conj - x)^2 + (y^\conj - y)^2}.
\end{equation}

Observe that the relation between $\BF^\conj$ and position is {\em nonlinear}!

Treatment of this system will be used as the prototype for our handling of other nonlinear systems.

%\EndArticle
\EndNoBibArticle
