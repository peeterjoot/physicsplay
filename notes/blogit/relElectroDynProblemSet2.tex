%\documentclass[]{eliblog}
%\usepackage{color}
%\usepackage{txfonts} % for xi
%\usepackage{amsmath}
\usepackage{mathpazo}

%
% shorthand for bold symbols, convenient for vectors and matrices
%
\newcommand{\Ba}[0]{\mathbf{a}}
\newcommand{\Bb}[0]{\mathbf{b}}
\newcommand{\Bc}[0]{\mathbf{c}}
\newcommand{\Bd}[0]{\mathbf{d}}
\newcommand{\Be}[0]{\mathbf{e}}
\newcommand{\Bf}[0]{\mathbf{f}}
\newcommand{\Bg}[0]{\mathbf{g}}
\newcommand{\Bh}[0]{\mathbf{h}}
\newcommand{\Bi}[0]{\mathbf{i}}
\newcommand{\Bj}[0]{\mathbf{j}}
\newcommand{\Bk}[0]{\mathbf{k}}
\newcommand{\Bl}[0]{\mathbf{l}}
\newcommand{\Bm}[0]{\mathbf{m}}
\newcommand{\Bn}[0]{\mathbf{n}}
\newcommand{\Bo}[0]{\mathbf{o}}
\newcommand{\Bp}[0]{\mathbf{p}}
\newcommand{\Bq}[0]{\mathbf{q}}
\newcommand{\Br}[0]{\mathbf{r}}
\newcommand{\Bs}[0]{\mathbf{s}}
\newcommand{\Bt}[0]{\mathbf{t}}
\newcommand{\Bu}[0]{\mathbf{u}}
\newcommand{\Bv}[0]{\mathbf{v}}
\newcommand{\Bw}[0]{\mathbf{w}}
\newcommand{\Bx}[0]{\mathbf{x}}
\newcommand{\By}[0]{\mathbf{y}}
\newcommand{\Bz}[0]{\mathbf{z}}
\newcommand{\BA}[0]{\mathbf{A}}
\newcommand{\BB}[0]{\mathbf{B}}
\newcommand{\BC}[0]{\mathbf{C}}
\newcommand{\BD}[0]{\mathbf{D}}
\newcommand{\BE}[0]{\mathbf{E}}
\newcommand{\BF}[0]{\mathbf{F}}
\newcommand{\BG}[0]{\mathbf{G}}
\newcommand{\BH}[0]{\mathbf{H}}
\newcommand{\BI}[0]{\mathbf{I}}
\newcommand{\BJ}[0]{\mathbf{J}}
\newcommand{\BK}[0]{\mathbf{K}}
\newcommand{\BL}[0]{\mathbf{L}}
\newcommand{\BM}[0]{\mathbf{M}}
\newcommand{\BN}[0]{\mathbf{N}}
\newcommand{\BO}[0]{\mathbf{O}}
\newcommand{\BP}[0]{\mathbf{P}}
\newcommand{\BQ}[0]{\mathbf{Q}}
\newcommand{\BR}[0]{\mathbf{R}}
\newcommand{\BS}[0]{\mathbf{S}}
\newcommand{\BT}[0]{\mathbf{T}}
\newcommand{\BU}[0]{\mathbf{U}}
\newcommand{\BV}[0]{\mathbf{V}}
\newcommand{\BW}[0]{\mathbf{W}}
\newcommand{\BX}[0]{\mathbf{X}}
\newcommand{\BY}[0]{\mathbf{Y}}
\newcommand{\BZ}[0]{\mathbf{Z}}

\newcommand{\Bzero}[0]{\mathbf{0}}
\newcommand{\Btheta}[0]{\boldsymbol{\theta}}
\newcommand{\Btau}[0]{\boldsymbol{\tau}}
\newcommand{\Bomega}[0]{\boldsymbol{\omega}}

%
% shorthand for unit vectors
%
\newcommand{\acap}[0]{\hat{\Ba}}
\newcommand{\bcap}[0]{\hat{\Bb}}
\newcommand{\ccap}[0]{\hat{\Bc}}
\newcommand{\dcap}[0]{\hat{\Bd}}
\newcommand{\ecap}[0]{\hat{\Be}}
\newcommand{\fcap}[0]{\hat{\Bf}}
\newcommand{\gcap}[0]{\hat{\Bg}}
\newcommand{\hcap}[0]{\hat{\Bh}}
\newcommand{\icap}[0]{\hat{\Bi}}
\newcommand{\jcap}[0]{\hat{\Bj}}
\newcommand{\kcap}[0]{\hat{\Bk}}
\newcommand{\lcap}[0]{\hat{\Bl}}
\newcommand{\mcap}[0]{\hat{\Bm}}
\newcommand{\ncap}[0]{\hat{\Bn}}
\newcommand{\ocap}[0]{\hat{\Bo}}
\newcommand{\pcap}[0]{\hat{\Bp}}
\newcommand{\qcap}[0]{\hat{\Bq}}
\newcommand{\rcap}[0]{\hat{\Br}}
\newcommand{\scap}[0]{\hat{\Bs}}
\newcommand{\tcap}[0]{\hat{\Bt}}
\newcommand{\ucap}[0]{\hat{\Bu}}
\newcommand{\vcap}[0]{\hat{\Bv}}
\newcommand{\wcap}[0]{\hat{\Bw}}
\newcommand{\xcap}[0]{\hat{\Bx}}
\newcommand{\ycap}[0]{\hat{\By}}
\newcommand{\zcap}[0]{\hat{\Bz}}
\newcommand{\thetacap}[0]{\hat{\Btheta}}

%
% to write R^n and C^n in a distinguishable fashion.  Perhaps change this
% to the double lined characters upon figuring out how to do so.
%
\newcommand{\C}[1]{$\mathbb{C}^{#1}$}
\newcommand{\R}[1]{$\mathbb{R}^{#1}$}

%
% various generally useful helpers
%

% derivative of #1 wrt. #2:
\newcommand{\D}[2] {\frac {d#2} {d#1}}

\newcommand{\inv}[1]{\frac{1}{#1}}
\newcommand{\cross}[0]{\times}

\newcommand{\abs}[1]{\lvert{#1}\rvert}
\newcommand{\norm}[1]{\lVert{#1}\rVert}
\newcommand{\innerprod}[2]{\langle{#1}, {#2}\rangle}
\newcommand{\dotprod}[2]{{#1} \cdot {#2}}
\newcommand{\bdotprod}[2]{\left({#1} \cdot {#2}\right)}
\newcommand{\crossprod}[2]{{#1} \cross {#2}}
\newcommand{\tripleprod}[3]{\dotprod{\left(\crossprod{#1}{#2}\right)}{#3}}

\DeclareMathOperator{\Proj}{Proj}
\DeclareMathOperator{\Span}{span}
\DeclareMathOperator{\Sgn}{sgn}
\DeclareMathOperator{\Area}{Area}
\DeclareMathOperator{\Volume}{Volume}

%
% A few miscellaneous things specific to this document
%
\newcommand{\crossop}[1]{\crossprod{#1}{}}

% R2 vector.
\newcommand{\VectorTwo}[2]{
\begin{bmatrix}
 {#1} \\
 {#2}
\end{bmatrix}
}

\newcommand{\VectorN}[1]{
\begin{bmatrix}
{#1}_1 \\
{#1}_2 \\
\vdots \\
{#1}_N \\
\end{bmatrix}
}

\newcommand{\DETuvij}[4]{
\begin{vmatrix}
 {#1}_{#3} & {#1}_{#4} \\
 {#2}_{#3} & {#2}_{#4}
\end{vmatrix}
}

\newcommand{\DETuvwijk}[6]{
\begin{vmatrix}
 {#1}_{#4} & {#1}_{#5} & {#1}_{#6} \\
 {#2}_{#4} & {#2}_{#5} & {#2}_{#6} \\
 {#3}_{#4} & {#3}_{#5} & {#3}_{#6}
\end{vmatrix}
}

\newcommand{\DETuvwxijkl}[8]{
\begin{vmatrix}
 {#1}_{#5} & {#1}_{#6} & {#1}_{#7} & {#1}_{#8} \\
 {#2}_{#5} & {#2}_{#6} & {#2}_{#7} & {#2}_{#8} \\
 {#3}_{#5} & {#3}_{#6} & {#3}_{#7} & {#3}_{#8} \\
 {#4}_{#5} & {#4}_{#6} & {#4}_{#7} & {#4}_{#8} \\
\end{vmatrix}
}

%\newcommand{\DETuvwxyijklm}[10]{
%\begin{vmatrix}
% {#1}_{#6} & {#1}_{#7} & {#1}_{#8} & {#1}_{#9} & {#1}_{#10} \\
% {#2}_{#6} & {#2}_{#7} & {#2}_{#8} & {#2}_{#9} & {#2}_{#10} \\
% {#3}_{#6} & {#3}_{#7} & {#3}_{#8} & {#3}_{#9} & {#3}_{#10} \\
% {#4}_{#6} & {#4}_{#7} & {#4}_{#8} & {#4}_{#9} & {#4}_{#10} \\
% {#5}_{#6} & {#5}_{#7} & {#5}_{#8} & {#5}_{#9} & {#5}_{#10}
%\end{vmatrix}
%}

% R3 vector.
\newcommand{\VectorThree}[3]{
\begin{bmatrix}
 {#1} \\
 {#2} \\
 {#3}
\end{bmatrix}
}



%
% Copyright � 2015 Peeter Joot.  All Rights Reserved.
% Licenced as described in the file LICENSE under the root directory of this GIT repository.
%
\documentclass[]{eliblog}

\usepackage{amsmath}
\usepackage{mathpazo}

%
% shorthand for bold symbols, convenient for vectors and matrices
%
\newcommand{\Ba}[0]{\mathbf{a}}
\newcommand{\Bb}[0]{\mathbf{b}}
\newcommand{\Bc}[0]{\mathbf{c}}
\newcommand{\Bd}[0]{\mathbf{d}}
\newcommand{\Be}[0]{\mathbf{e}}
\newcommand{\Bf}[0]{\mathbf{f}}
\newcommand{\Bg}[0]{\mathbf{g}}
\newcommand{\Bh}[0]{\mathbf{h}}
\newcommand{\Bi}[0]{\mathbf{i}}
\newcommand{\Bj}[0]{\mathbf{j}}
\newcommand{\Bk}[0]{\mathbf{k}}
\newcommand{\Bl}[0]{\mathbf{l}}
\newcommand{\Bm}[0]{\mathbf{m}}
\newcommand{\Bn}[0]{\mathbf{n}}
\newcommand{\Bo}[0]{\mathbf{o}}
\newcommand{\Bp}[0]{\mathbf{p}}
\newcommand{\Bq}[0]{\mathbf{q}}
\newcommand{\Br}[0]{\mathbf{r}}
\newcommand{\Bs}[0]{\mathbf{s}}
\newcommand{\Bt}[0]{\mathbf{t}}
\newcommand{\Bu}[0]{\mathbf{u}}
\newcommand{\Bv}[0]{\mathbf{v}}
\newcommand{\Bw}[0]{\mathbf{w}}
\newcommand{\Bx}[0]{\mathbf{x}}
\newcommand{\By}[0]{\mathbf{y}}
\newcommand{\Bz}[0]{\mathbf{z}}
\newcommand{\BA}[0]{\mathbf{A}}
\newcommand{\BB}[0]{\mathbf{B}}
\newcommand{\BC}[0]{\mathbf{C}}
\newcommand{\BD}[0]{\mathbf{D}}
\newcommand{\BE}[0]{\mathbf{E}}
\newcommand{\BF}[0]{\mathbf{F}}
\newcommand{\BG}[0]{\mathbf{G}}
\newcommand{\BH}[0]{\mathbf{H}}
\newcommand{\BI}[0]{\mathbf{I}}
\newcommand{\BJ}[0]{\mathbf{J}}
\newcommand{\BK}[0]{\mathbf{K}}
\newcommand{\BL}[0]{\mathbf{L}}
\newcommand{\BM}[0]{\mathbf{M}}
\newcommand{\BN}[0]{\mathbf{N}}
\newcommand{\BO}[0]{\mathbf{O}}
\newcommand{\BP}[0]{\mathbf{P}}
\newcommand{\BQ}[0]{\mathbf{Q}}
\newcommand{\BR}[0]{\mathbf{R}}
\newcommand{\BS}[0]{\mathbf{S}}
\newcommand{\BT}[0]{\mathbf{T}}
\newcommand{\BU}[0]{\mathbf{U}}
\newcommand{\BV}[0]{\mathbf{V}}
\newcommand{\BW}[0]{\mathbf{W}}
\newcommand{\BX}[0]{\mathbf{X}}
\newcommand{\BY}[0]{\mathbf{Y}}
\newcommand{\BZ}[0]{\mathbf{Z}}

\newcommand{\Bzero}[0]{\mathbf{0}}
\newcommand{\Btheta}[0]{\boldsymbol{\theta}}
\newcommand{\Btau}[0]{\boldsymbol{\tau}}
\newcommand{\Bomega}[0]{\boldsymbol{\omega}}

%
% shorthand for unit vectors
%
\newcommand{\acap}[0]{\hat{\Ba}}
\newcommand{\bcap}[0]{\hat{\Bb}}
\newcommand{\ccap}[0]{\hat{\Bc}}
\newcommand{\dcap}[0]{\hat{\Bd}}
\newcommand{\ecap}[0]{\hat{\Be}}
\newcommand{\fcap}[0]{\hat{\Bf}}
\newcommand{\gcap}[0]{\hat{\Bg}}
\newcommand{\hcap}[0]{\hat{\Bh}}
\newcommand{\icap}[0]{\hat{\Bi}}
\newcommand{\jcap}[0]{\hat{\Bj}}
\newcommand{\kcap}[0]{\hat{\Bk}}
\newcommand{\lcap}[0]{\hat{\Bl}}
\newcommand{\mcap}[0]{\hat{\Bm}}
\newcommand{\ncap}[0]{\hat{\Bn}}
\newcommand{\ocap}[0]{\hat{\Bo}}
\newcommand{\pcap}[0]{\hat{\Bp}}
\newcommand{\qcap}[0]{\hat{\Bq}}
\newcommand{\rcap}[0]{\hat{\Br}}
\newcommand{\scap}[0]{\hat{\Bs}}
\newcommand{\tcap}[0]{\hat{\Bt}}
\newcommand{\ucap}[0]{\hat{\Bu}}
\newcommand{\vcap}[0]{\hat{\Bv}}
\newcommand{\wcap}[0]{\hat{\Bw}}
\newcommand{\xcap}[0]{\hat{\Bx}}
\newcommand{\ycap}[0]{\hat{\By}}
\newcommand{\zcap}[0]{\hat{\Bz}}
\newcommand{\thetacap}[0]{\hat{\Btheta}}

%
% to write R^n and C^n in a distinguishable fashion.  Perhaps change this
% to the double lined characters upon figuring out how to do so.
%
\newcommand{\C}[1]{$\mathbb{C}^{#1}$}
\newcommand{\R}[1]{$\mathbb{R}^{#1}$}

%
% various generally useful helpers
%

% derivative of #1 wrt. #2:
\newcommand{\D}[2] {\frac {d#2} {d#1}}

\newcommand{\inv}[1]{\frac{1}{#1}}
\newcommand{\cross}[0]{\times}

\newcommand{\abs}[1]{\lvert{#1}\rvert}
\newcommand{\norm}[1]{\lVert{#1}\rVert}
\newcommand{\innerprod}[2]{\langle{#1}, {#2}\rangle}
\newcommand{\dotprod}[2]{{#1} \cdot {#2}}
\newcommand{\bdotprod}[2]{\left({#1} \cdot {#2}\right)}
\newcommand{\crossprod}[2]{{#1} \cross {#2}}
\newcommand{\tripleprod}[3]{\dotprod{\left(\crossprod{#1}{#2}\right)}{#3}}

\DeclareMathOperator{\Proj}{Proj}
\DeclareMathOperator{\Span}{span}
\DeclareMathOperator{\Sgn}{sgn}
\DeclareMathOperator{\Area}{Area}
\DeclareMathOperator{\Volume}{Volume}

%
% A few miscellaneous things specific to this document
%
\newcommand{\crossop}[1]{\crossprod{#1}{}}

% R2 vector.
\newcommand{\VectorTwo}[2]{
\begin{bmatrix}
 {#1} \\
 {#2}
\end{bmatrix}
}

\newcommand{\VectorN}[1]{
\begin{bmatrix}
{#1}_1 \\
{#1}_2 \\
\vdots \\
{#1}_N \\
\end{bmatrix}
}

\newcommand{\DETuvij}[4]{
\begin{vmatrix}
 {#1}_{#3} & {#1}_{#4} \\
 {#2}_{#3} & {#2}_{#4}
\end{vmatrix}
}

\newcommand{\DETuvwijk}[6]{
\begin{vmatrix}
 {#1}_{#4} & {#1}_{#5} & {#1}_{#6} \\
 {#2}_{#4} & {#2}_{#5} & {#2}_{#6} \\
 {#3}_{#4} & {#3}_{#5} & {#3}_{#6}
\end{vmatrix}
}

\newcommand{\DETuvwxijkl}[8]{
\begin{vmatrix}
 {#1}_{#5} & {#1}_{#6} & {#1}_{#7} & {#1}_{#8} \\
 {#2}_{#5} & {#2}_{#6} & {#2}_{#7} & {#2}_{#8} \\
 {#3}_{#5} & {#3}_{#6} & {#3}_{#7} & {#3}_{#8} \\
 {#4}_{#5} & {#4}_{#6} & {#4}_{#7} & {#4}_{#8} \\
\end{vmatrix}
}

%\newcommand{\DETuvwxyijklm}[10]{
%\begin{vmatrix}
% {#1}_{#6} & {#1}_{#7} & {#1}_{#8} & {#1}_{#9} & {#1}_{#10} \\
% {#2}_{#6} & {#2}_{#7} & {#2}_{#8} & {#2}_{#9} & {#2}_{#10} \\
% {#3}_{#6} & {#3}_{#7} & {#3}_{#8} & {#3}_{#9} & {#3}_{#10} \\
% {#4}_{#6} & {#4}_{#7} & {#4}_{#8} & {#4}_{#9} & {#4}_{#10} \\
% {#5}_{#6} & {#5}_{#7} & {#5}_{#8} & {#5}_{#9} & {#5}_{#10}
%\end{vmatrix}
%}

% R3 vector.
\newcommand{\VectorThree}[3]{
\begin{bmatrix}
 {#1} \\
 {#2} \\
 {#3}
\end{bmatrix}
}



\author{Peeter Joot}
\email{peeter.joot@gmail.com}

\author{Peeter Joot}
\email{peeter.joot@utoronto.ca, 920798560}

\chapter{PHY450H1S Problem Set 2.}
\label{chap:relElectroDynProblemSet2}
%\blogpage{http://sites.google.com/site/peeterjoot/math2011/relElectroDynProblemSet2.pdf}
\date{Feb 1, 2011}
\revisionInfo{relElectroDynProblemSet2.tex}

%\beginArtNoToc
\beginArtWithToc
%\section{Disclaimer.}
%
%This problem set is as yet ungraded.

\section{Problem 1.}
\subsection{Statement}

A particle of rest mass $m$ whose energy is three times its rest energy collides with an identical particle at rest.  Suppose they stick together.  Use conservation laws to find the mass of the resulting particle and its velocity.  Is its mass greater or smaller than $2m$?  Comment.

\subsection{Solution}

The energy of the initially moving particle before collision is

\begin{equation}\label{eqn:relElectroDynProblemSet2:n}
\mathcal{E} = \frac{m c^2 }{\InvGamma} = 3 m c^2.
\end{equation}

Solving for the velocity we have

\begin{equation}\label{eqn:relElectroDynProblemSet2:n}
\Abs{\frac{\Bv}{c}} = \frac{2 \sqrt{2}}{3}.
\end{equation}

Our four velocity is

\begin{equation}\label{eqn:relElectroDynProblemSet2:n}
u^i
= \gamma \left( 1, \frac{\Bv}{c} \right) = ( 3, 2 \sqrt{2} ).
\end{equation}

Designate the four momentum for this particle as

\begin{equation}\label{eqn:relElectroDynProblemSet2:n}
p_{(1)}^i = m c ( 3, 2 \sqrt{2} ).
\end{equation}

For the second particle we have

\begin{equation}\label{eqn:relElectroDynProblemSet2:n}
p_{(2)}^i = m c ( 1, 0 ).
\end{equation}

Our initial and final four momentum will be equal, and our resulting velocity can only be in the direction of the initial particle.  This leaves us with

\begin{align*}
p_{(f)}^i
&= M c \inv{\sqrt{1 - \frac{\Bv_f^2}{c^2}}} \left( 1, \frac{\Bv_f}{c} \right) \\
&= m c ( 1, 0 ) + m c ( 3, 2 \sqrt{2} )  \\
&= m c ( 4, 2 \sqrt{2} ) \\
&= 4 m c \left( 1, \inv{\sqrt{2}} \right)
\end{align*}

Our final velocity is $v_f = c/\sqrt{2}$.

We have $M \gamma = 4$ for the final particle, but we have

\begin{equation}\label{eqn:relElectroDynProblemSet2:n}
\gamma = \frac{1}{\sqrt{1 - 1/2}} = \sqrt{2},
\end{equation}

so our final mass is

\begin{equation}\label{eqn:relElectroDynProblemSet2:n}
M = \frac{4}{\sqrt{2}} = 2 \sqrt{2} > 2
\end{equation}

\section{Problem 2.}
\subsection{Statement}

This problem has three parts
\begin{enumerate}
\item Express the ``normal'' (i.e. not 4-, but 3-) acceleration, equal to $\dot{\Bv}$, or a particle in terms of its velocity, $\BE$, and $\BB$, using the equation of motion of a relativistic particle in an external electromagnetic field.
\item Consider now a beam of electrons, moving along the $x$ direction wwith a known energy $\mathcal{E}$, entering a region with constant homogeneous $\BE$ and $\BB$ fields.  The fields are perpendicular, $\BE$ is along the $y$ direction while $\BB$ is along the $z$ direction.
\begin{enumerate}
\item
Show that by tuning the values of $\BE$ and $BB$ it is possible to balance electric and magnetic forces so that the beam does not devaiate from its original direction (and, say, hits a screen directly ahead).
\item Find a relation determining the mass of the electron using $\mathcal{E}$ and the measured values of the fields for which no deviation occurs.  Do not assume a nonrelativistic limit and elucidate which part of this prlblem (a way to measure the mass of the electron) is affected by relativity.
\end{enumerate}
\item Solve for the motion (i.e. find the tragectories) of a relativistic charged particle in perpendicularconstant and homogeneous electric and magnetic fields; do not assume $\BE = \BB$.
\end{enumerate}
\subsection{Solution}
\subsubsection{1. Finding $\dot{\Bv}$}

With the particle's energy given by

\begin{equation}\label{eqn:relElectroDynProblemSet2:n}
\mathcal{E} = \gamma m c^2,
\end{equation}

we note that

\begin{equation}\label{eqn:relElectroDynProblemSet2:n}
\mathcal{E}\Bv = (\gamma m \Bv) c^2 = \Bp c^2.
\end{equation}

Taking derivatives we have

\begin{align*}
c^2 \ddt{\Bp} 
&= \Bv \ddt{\mathcal{E}} + \ddt{\Bv} \mathcal{E} \\
&= \Bv (e \BE \cdot \Bv) + \ddt{\Bv} \mathcal{E} \\
\end{align*}

Rearranging we have

\begin{equation}\label{eqn:relElectroDynProblemSet2:n}
\ddt{\Bv}
=
\frac{c^2 e \left( \BE + \frac{\Bv}{c} \cross \BB \right) - \Bv (e \BE \cdot \Bv) }{ \mathcal{E} } 
\end{equation}

which leaves us with the desired result
\begin{equation}\label{eqn:relElectroDynProblemSet2:n}
\boxed{
\dot{\Bv} =
\frac{e}{m} \InvGamma \left( \BE + \frac{\Bv}{c} \cross \BB - \frac{\Bv}{c} \left(\BE \cdot \frac{\Bv}{c} \right) \right)
}
\end{equation}

\subsubsection{1b.  On the energy change rate.}

Note that when the problem set was assigned, the relation

\begin{equation}\label{eqn:relElectroDynProblemSet2:n}
\ddt{\mathcal{E}} = e \BE \cdot \Bv
\end{equation}

had not been demonstrated.  To show this observe that we have

\begin{align*}
\frac{d}{dt} \mathcal{E}
&= m c^2 \frac{d\gamma}{dt} \\
&= m c^2 \frac{d}{dt} \inv{\InvGamma} \\
&= m c^2 \frac{\frac{\Bv}{c^2} \cdot \frac{d\Bv}{dt}}{\left(1 - \frac{\Bv^2}{c^2}\right)^{3/2}} \\
&= \frac{m \gamma \Bv \cdot \frac{d\Bv}{dt}}{1 - \frac{\Bv^2}{c^2}}
\end{align*}

We also have

\begin{align*}
\Bv \cdot \ddt{\Bp} 
&= \Bv \cdot \ddt{} \frac{m \Bv}{\InvGamma} \\
&= m\Bv^2 \ddt{\gamma} + m \gamma \Bv \cdot \ddt{\Bv} \\
&= m\Bv^2 \ddt{\gamma} + m c^2 \ddt{\gamma} \left( 1 - \frac{\Bv^2}{c^2} \right) \\
&= m c^2 \ddt{\gamma}.
\end{align*}

Utilizing the Lorentz force equation, we have

\begin{equation}\label{eqn:relElectroDynProblemSet2:n}
\Bv \cdot \ddt{\Bp} = e \left( \BE + \frac{\Bv}{c} \cross \BB \right) \cdot \Bv = e \BE \cdot \Bv
\end{equation}

and are able to assemble the above, and find that we have
\begin{equation}\label{eqn:relElectroDynProblemSet2:n}
\ddt{(m c^2 \gamma)} = e \BE \cdot \Bv 
\end{equation}

\subsubsection{2. (a). Tuning $\BE$ and $\BB$}

Using our previous result with $\BE = E \ycap$ and $\BB = B \zcap$, our system of equations takes the form

\begin{equation}\label{eqn:relElectroDynProblemSet2:n}
\dot{\Bv} = \frac{e}{m} \InvGamma \left( E \ycap + \xcap \frac{v_y}{c} B - \ycap \frac{y_x} B - \frac{\Bv}{c} E \frac{v_y}{c} \right)
\end{equation}

This is really three equations, but they are coupled with the nasty $\InvGamma$ term.  However, since it is specified that the particles have a known energy $\mathcal{E}$, and that energy is

\begin{equation}\label{eqn:relElectroDynProblemSet2:n}
\mathcal{E} = \frac{ m c^2 }{\InvGamma},
\end{equation}

we can write

\begin{equation}\label{eqn:relElectroDynProblemSet2:n}
\InvGamma = \frac{ m c^2 }{\mathcal{E}}
\end{equation}

This eliminates the worst of the coupling, leaving three less hairy equations to solve

\begin{align}\label{eqn:relElectroDynProblemSet2:n}
\dot{v_x} &= \frac{e c^2}{\mathcal{E}} \left( \frac{v_y}{c} B - \frac{v_x v_y}{c^2} E \right) \\
\dot{v_y} &= \frac{e c^2}{\mathcal{E}} \left( E - \frac{v_x}{c} B - \frac{v_y^2}{c^2} E \right) \\
\dot{v_z} &= \frac{e c^2}{\mathcal{E}} \left( - \frac{v_y v_z}{c^2} E \right)
\end{align}

We don't actually want to solve these.  Instead we just wish to examine the constraints on $E$ and $B$ that will keep $v_y = v_z = 0$.

First off we see from the $\dot{v_z}$ equation above that if $v_y = 0$ or $v_z = 0$ initially, then $\dot{v_z} = 0$, and $v_z(t) = \text{constant} = v_z(0) = 0$.  So, if the beam is initially aligned with the x direction, it will not deviate towards the $z$ axis (in the direction of the magnetic field) at all.

Next, if we initially have $v_y = 0$, then at that point of time, our equation for $\dot{v_x}$ and $\dot{v_y}$ are respectively

\begin{align}\label{eqn:relElectroDynProblemSet2:n}
\dot{v_x} &= 0 \\
\dot{v_y} &= \frac{e c^2}{\mathcal{E}} \left( E - \frac{v_x}{c} B \right) 
\end{align}

We are able to solve for the time evolution of the velocities directly

\begin{align}\label{eqn:relElectroDynProblemSet2:n}
v_x(t) &= \text{constant} = v_x(0) \\
v_y(t) &= \frac{e c^2}{\mathcal{E}} \left( E - \frac{v_x(0)}{c} B \right) t
\end{align}

We can maintain zero deviation in the $y$ direction ($v_y(t) = 0$) provided we pick

\begin{equation}\label{eqn:relElectroDynProblemSet2:n}
E = \frac{v_x(0)}{c} B
\end{equation}

\subsubsection{2. (b). Finding the mass of the electron.}

After measuring the fields that once adjusted produce no deviation in the $y$ and $z$ directions, our particles velocity must then be

\begin{equation}\label{eqn:relElectroDynProblemSet2:n}
\frac{v_x}{c} = \frac{E}{B}
\end{equation}

If the energy has also been measured, we have a relation between the mass from

\begin{equation}\label{eqn:relElectroDynProblemSet2:n}
\mathcal{E} = \frac{m c^2}{\sqrt{1 - v_x^2/c^2}} = \frac{ m c^2 }{ \sqrt{ 1 - E^2/B^2 }}
\end{equation}

With a slight rearrangement, our mass can then be calculated from the energy $\mathcal{E}$, and field measurements

\begin{equation}\label{eqn:relElectroDynProblemSet2:n}
m = \frac{ \mathcal{E} }{c^2} \sqrt{ 1 - E^2/B^2 }.
\end{equation}

\subsubsection{3.  Solve for the relativistic trajectory of a particle in perpendicular fields.}

Our equation to solve is

\begin{equation}\label{eqn:relElectroDynProblemSet2:n}
\dds{u^i} = \frac{e}{m c^2} F^{ij} g_{jk} u^k,
\end{equation}

where

\begin{equation}\label{eqn:relElectroDynProblemSet2:n}
\Norm{ F^{ij} g_{jk} } = 
\begin{bmatrix}
0 & -E_x & -E_y & -E_z \\
E_x & 0 & -B_z & B_y \\
E_y & B_z & 0 & -B_x \\
E_z & -B_y & B_x & 0
\end{bmatrix}
\begin{bmatrix}
1 & 0 & 0 & 0 \\
0 & -1 & 0 & 0 \\
0 & 0 & -1 & 0 \\
0 & 0 & 0 & -1 \\
\end{bmatrix}
=
\begin{bmatrix}
0 & E_x & E_y & E_z \\
E_x & 0 & B_z & -B_y \\
E_y & -B_z & 0 & B_x \\
E_z & B_y & -B_x & 0
\end{bmatrix}.
\end{equation}

However, with the fields being perpendicular, we are free to align them with our choice of axis.  As above, let's use $\BE = E \ycap$, and $\BB = B \zcap$.  Writing $u$ for the column vector with components $u^i$ we have a matrix equation to solve

\begin{equation}\label{eqn:relElectroDynProblemSet2:n}
\dds{u} = 
\frac{ e }{m c^2}
\begin{bmatrix}
0 & 0 & E & 0 \\
0 & 0 & B & 0 \\
E & -B & 0 & 0 \\
0 & 0 & 0 & 0
\end{bmatrix} u = F u.
\end{equation}

It is simple to verify that our characteristic equation is

\begin{align*}
0 
&= \Abs{ F - \lambda I } \\
&= \begin{vmatrix}
-\lambda & 0 & E & 0 \\
0 & -\lambda & B & 0 \\
E & -B & -\lambda & 0 \\
0 & 0 & 0 & -\lambda
\end{vmatrix} \\
&= -\lambda^2 ( -\lambda^2 - B^2 + E^2 )
\end{align*}

so that our eigenvalues are

\begin{equation}\label{eqn:relElectroDynProblemSet2:n}
\lambda = 0, 0, \pm \sqrt{E^2 - B^2}.
\end{equation}

Since the fields are constant, we can diagonalize this, and solve by exponentiation.

Let 

\begin{equation}\label{eqn:relElectroDynProblemSet2:n}
D = \sqrt{E^2 - B^2}.
\end{equation}

To solve for the eigenvector $e_D$ for $\lambda = D$ we need solutions to

\begin{equation}\label{eqn:relElectroDynProblemSet2:n}
\begin{bmatrix}
-D & 0 & E & 0 \\
0 & -D & B & 0 \\
E & -B & -D & 0 \\
0 & 0 & 0 & -D
\end{bmatrix} 
\begin{bmatrix} 
a \\
b \\
c \\
d
\end{bmatrix} 
 = 0,
\end{equation}

and it is straightforward to compute

\begin{equation}\label{eqn:relElectroDynProblemSet2:n}
e_D = 
\inv{\sqrt{2}E}
\begin{bmatrix} 
E \\
B \\
D \\
0
\end{bmatrix}.
\end{equation}

Similarily for the $\lambda = -D$ eigenvector $e_{-D}$ we wish to solve

\begin{equation}\label{eqn:relElectroDynProblemSet2:n}
\begin{bmatrix}
D & 0 & E & 0 \\
0 & D & B & 0 \\
E & -B & D & 0 \\
0 & 0 & 0 & D
\end{bmatrix} 
\begin{bmatrix} 
a \\
b \\
c \\
d
\end{bmatrix} 
 = 0,
\end{equation}

and find that

\begin{equation}\label{eqn:relElectroDynProblemSet2:n}
e_{-D} = 
\inv{\sqrt{2}E}
\begin{bmatrix} 
E \\
B \\
-D \\
0
\end{bmatrix}.
\end{equation}

We can also pick orthonormal eigenvectors for the degenerate zero eigenvalues from the null space of the matrix

\begin{equation}\label{eqn:relElectroDynProblemSet2:n}
\begin{bmatrix}
0 & 0 & E & 0 \\
0 & 0 & B & 0 \\
E & -B & 0 & 0 \\
0 & 0 & 0 & 0
\end{bmatrix} 
\end{equation}

By inspection, two such eigenvectors are 
\begin{equation}\label{eqn:relElectroDynProblemSet2:n}
\inv{\sqrt{E^2 + B^2}}
\begin{bmatrix} 
B \\
E \\
0 \\
0 
\end{bmatrix},
\begin{bmatrix} 
0 \\
0 \\
0 \\
1 
\end{bmatrix}.
\end{equation}

Unfortunately, the first is not generally orthonormal to either of $e_{\pm D}$, so our similarity transformation matrix is not invertable by Hermitian transposition.  Regardless, our matrix equation to solve can now be put into a much similar form.  With

\begin{equation}\label{eqn:relElectroDynProblemSet2:n}
S =
\begin{bmatrix}
\inv{\sqrt{2}E}
\begin{bmatrix} 
E \\
B \\
D \\
0
\end{bmatrix} 
&
\inv{\sqrt{2}E}
\begin{bmatrix} 
E \\
B \\
-D \\
0
\end{bmatrix} &
\inv{\sqrt{E^2 + B^2}}
\begin{bmatrix} 
B \\
E \\
0 \\
0 
\end{bmatrix} &
\begin{bmatrix} 
0 \\
0 \\
0 \\
1 
\end{bmatrix}
\end{bmatrix},
\end{equation}

and 

\begin{equation}\label{eqn:relElectroDynProblemSet2:n}
\Sigma = 
\begin{bmatrix}
D & 0 & 0 & 0 \\
0 & -D & 0 & 0 \\
0 & 0 & 0 & 0 \\
0 & 0 & 0 & 0 \\
\end{bmatrix}
\end{equation}

observe that our Lorentz force equation can now be written

\begin{equation}\label{eqn:relElectroDynProblemSet2:n}
\dds{u} = \frac{e}{m c^2} S \Sigma S^{-1} u.
\end{equation}

This we can rearrange, leaving us with a diagonal system that has a trivial solution

\begin{equation}\label{eqn:relElectroDynProblemSet2:n}
\dds{} (S^{-1} u) = \frac{e}{m c^2} \Sigma (S^{-1} u).
\end{equation}

Let's write

\begin{equation}\label{eqn:relElectroDynProblemSet2:n}
v = S^{-1} u,
\end{equation}

and introduce a sort of proper distance wave number

\begin{equation}\label{eqn:relElectroDynProblemSet2:n}
k = \frac{e \sqrt{E^2 - B^2}}{m c^2}.
\end{equation}

With this the Lorentz force equation is left in the form

\begin{equation}\label{eqn:relElectroDynProblemSet2:n}
\dds{v} = 
\begin{bmatrix}
k & 0 & 0 & 0 \\
0 & -k & 0 & 0 \\
0 & 0 & 0 & 0 \\
0 & 0 & 0 & 0 \\
\end{bmatrix} v.
\end{equation}

Integrating once, our solution is

\begin{equation}\label{eqn:relElectroDynProblemSet2:n}
v(s) = 
\begin{bmatrix}
e^{ks} & 0 & 0 & 0 \\
0 & e^{-ks} & 0 & 0 \\
0 & 0 & 1 & 0 \\
0 & 0 & 0 & 1 \\
\end{bmatrix} v(s=0)
\end{equation}

Our proper velocity is thus given by

\begin{equation}\label{eqn:relElectroDynProblemSet2:n}
u = \dds{X} = S 
\begin{bmatrix}
e^{ks} & 0 & 0 & 0 \\
0 & e^{-ks} & 0 & 0 \\
0 & 0 & 1 & 0 \\
0 & 0 & 0 & 1 \\
\end{bmatrix} S^{-1} u(s=0).
\end{equation}

We can integrate once more for our tragectory, parameterized by proper distance on the worldline of the particle.  That is

\begin{equation}\label{eqn:relElectroDynProblemSet2:n}
X(s) - X(0) 
= S \left( \int_{s'=0}^s 
ds'
\begin{bmatrix}
e^{ks'} & 0 & 0 & 0 \\
0 & e^{-ks'} & 0 & 0 \\
0 & 0 & 1 & 0 \\
0 & 0 & 0 & 1 \\
\end{bmatrix} \right) S^{-1} u(s=0).
\end{equation}

With $u(0) = \gamma_0 (1, \Bv_0/c)$, and $X = (c t_0, \Bx_0)$, our parametric equation for the tragectory is fully specified
%FIXME: reference to S = definition, and k = ... definition eqnos.

\begin{equation}\label{eqn:relElectroDynProblemSet2:n}
\begin{bmatrix}
c t(s) \\
\Bx^\T(s)
\end{bmatrix}
- 
\begin{bmatrix}
c t_0 \\
\Bx_0^\T
\end{bmatrix}
= S 
\begin{bmatrix}
\inv{k}(e^{ks} -1) & 0 & 0 & 0 \\
0 & -\inv{k}(e^{-ks} -1) & 0 & 0 \\
0 & 0 & s & 0 \\
0 & 0 & 0 & s \\
\end{bmatrix} S^{-1} \inv{\sqrt{1 - (\Bv_0)^2/c^2}}
\begin{bmatrix}
1 \\
\Bv_0^\T/c
\end{bmatrix}.
\end{equation}

Observe that for the case $E^2 > B^2$, our value $k$ is real, so the solution is entirely composed of linear combinations of the hyperbolic functions $\cosh(k s)$ and $\sinh(ks)$.  However, for the $E^2 < B^2$ case where our eigenvalues are purely complex, the constant $k$ is purely imaginary (and our eigenvectors $e_{\pm D}$ are complex).  In that case, we can take the real part of this equation, and will be left with a solution that is formed of linear combinations of $\sin(ks)$ and $\cos(ks)$ terms.  The $E = B$ case would have to be handled separately, and this is done in depth in the text, so there is little value repeating it here.

\section{Problem 3.}
\subsection{Statement}

In class, we introduced the 4-vector potential $A^i$ and its transformation law under Lorentz transformations.  While we have not yet discussed how $\BE$ and $\BB$ transform, knowing how $A^i$ transforms is enough to solve some concrete problems.  Suppose in one (unprimed) frame there is a charge at rest, which creates an electrostatic field: $A^0 = \phi = \frac{q}{r}, \BA = 0$.

\begin{enumerate}
\item Find the values of $\BE$ and $\BB$ in this frame.
\item Consider now the same field in a (primed) frame moving in the $x$-direction with velocity $v$.  Using the transformation law of the vector potential, find ${A^i}'$ in the primed frame.
\item Use the relations between electric and magnetic field strengths and vector potential (valid in every frame) to find the electric and magnetic fields in the primed frame (i.e. find the electromagnetic field of a moving charge).  Sketch the lines of constant electric and magnetic field and comment on the result.
\end{enumerate}

\subsection{Solution}
\subsubsection{1.}

In the unprimed frame we have

\begin{align*}
\BE 
&= - \spacegrad \phi - \inv{c} \PD{t}{\BA} \\
&= -\spacegrad \phi \\
&= - \rcap q \partial_r (1/r) \\
&= \rcap \frac{q}{r^2},
\end{align*}

and
\begin{align*}
\BB = \spacegrad \cross \BA = 0
\end{align*}

\subsubsection{2.}

The coordinates in the moving frame, assuming the frames are overlapping at $t=0$, are related to the unprimed coordinates by

\begin{equation}\label{eqn:relElectroDynProblemSet2:n}
\begin{bmatrix}
ct' \\
x' \\
y' \\
z'
\end{bmatrix}
=
\begin{bmatrix}
\gamma & -\gamma \beta & 0 & 0 \\
-\gamma \beta & \gamma & 0 & 0 \\
0 & 0 & 1 & 0 \\
0 & 0 & 0 & 1 
\end{bmatrix}
\begin{bmatrix}
ct \\
x \\
y \\
z
\end{bmatrix}
\end{equation}

Our four vector potential also transforms in the same fashion, and we have

\begin{equation}\label{eqn:relElectroDynProblemSet2:n}
\begin{bmatrix}
\phi' \\
A_x' \\
A_y' \\
A_z' \\
\end{bmatrix}
=
\begin{bmatrix}
\gamma & -\gamma \beta & 0 & 0 \\
-\gamma \beta & \gamma & 0 & 0 \\
0 & 0 & 1 & 0 \\
0 & 0 & 0 & 1 
\end{bmatrix}
\begin{bmatrix}
\phi \\
0 \\
0 \\
0 \\
\end{bmatrix}
= \gamma \phi ( 1, -\beta, 0, 0 )
\end{equation}

So in the primed frame we have
\begin{align}\label{eqn:relElectroDynProblemSet2:n}
\phi' &= \gamma \frac{q}{r} \\
A_x' &= -\gamma \beta \frac{q}{r} \\
A_y' &= 0 \\
A_z' &= 0 
\end{align}

\subsubsection{3.}

In the primed frame our electric and magnetic fields are

\begin{align}\label{eqn:relElectroDynProblemSet2:n}
\BE' &= - \spacegrad' \phi' - \inv{c} \PD{t'}{\BA'} \\
\BB' &= \spacegrad' \cross \BA'
\end{align}

We have $\phi'$ and $\BA'$ expressed in terms of the unprimed coordinates, so need to calculate the transformation of the gradient and time partial too.  These partials transform as

\begin{align}\label{eqn:relElectroDynProblemSet2:n}
\PD{c t'}{} &= \PD{ct'}{ct} \PD{ct}{} + \PD{ct'}{x} \PD{x}{} \\
\PD{x'}{} &= \PD{x'}{ct} \PD{ct}{} + \PD{x'}{x} \PD{x}{} \\
\PD{y'}{} &= \PD{y}{} \\
\PD{z'}{} &= \PD{z}{}
\end{align}

Utilizing the inverse transformation 

\begin{equation}\label{eqn:relElectroDynProblemSet2:n}
\begin{bmatrix}
ct \\
x \\
y \\
z
\end{bmatrix}
=
\begin{bmatrix}
\gamma & \gamma \beta & 0 & 0 \\
\gamma \beta & \gamma & 0 & 0 \\
0 & 0 & 1 & 0 \\
0 & 0 & 0 & 1 
\end{bmatrix}
\begin{bmatrix}
ct' \\
x' \\
y' \\
z'
\end{bmatrix}
\end{equation}

we have
\begin{align}\label{eqn:relElectroDynProblemSet2:n}
\PD{c t'}{} &= \gamma \PD{ct}{} + \gamma \beta \PD{x}{} \\
\PD{x'}{} &= \gamma \beta \PD{ct}{} + \gamma \PD{x}{} \\
\PD{y'}{} &= \PD{y}{} \\
\PD{z'}{} &= \PD{z}{}
\end{align}

Since neither $\phi'$ nor $\BA'$ have time dependence, we have for electric field in the primed frame

\begin{align*}
\BE' 
&= -\spacegrad' \phi' - \inv{c} \PD{t'}{\BA'} \\
&= 
-\left( \gamma \PD{x}{}, \PD{y}{}, \PD{z}{} \right) \phi'
- \gamma \beta \PD{x}{\BA'} \\
&= 
-\left( \gamma \PD{x}{}, \PD{y}{}, \PD{z}{} \right) \gamma \frac{q}{r}
- \gamma \beta \PD{x}{} \left( -\gamma \beta \frac{q}{r}, 0, 0 \right) \\
&= -q \left( \gamma^2 ( 1 - \beta^2 ) \PD{x}{}, \gamma \PD{y}{}, \gamma \PD{z}{} \right) \inv{r} \\
&= -q \left( \PD{x}{}, \gamma \PD{y}{}, \gamma \PD{z}{} \right) \inv{r}
\end{align*}

Our electric field in the primed frame is thus
\begin{equation}\label{eqn:relElectroDynProblemSet2:n}
\BE' = -\frac{q}{r^3} \left( x, \gamma y, \gamma z \right) 
\end{equation}

Now for the magnetic field.  We want

\begin{align*}
\BB' 
&= 
\begin{vmatrix}
\xcap & \ycap & \zcap \\
\partial_{x'} & \partial_{y'} & \partial_{z'} \\
-\gamma \beta q/r & 0 & 0
\end{vmatrix} \\
&=
\left( 0, \partial_{z'}, -\partial_{y'} \right) \frac{-\gamma \beta q}{r} \\
\end{align*}

\begin{equation}\label{eqn:relElectroDynProblemSet2:n}
\BB'
=
\frac{q \gamma \beta}{r^3} \left( 0, -z, y \right)
\end{equation}

FIXME: sketch and comment.

%\EndArticle
\EndNoBibArticle
