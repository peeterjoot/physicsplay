%\documentclass[]{eliblog}
%\usepackage{color}
%\usepackage{txfonts} % for xi
%\usepackage{amsmath}
\usepackage{mathpazo}

%
% shorthand for bold symbols, convenient for vectors and matrices
%
\newcommand{\Ba}[0]{\mathbf{a}}
\newcommand{\Bb}[0]{\mathbf{b}}
\newcommand{\Bc}[0]{\mathbf{c}}
\newcommand{\Bd}[0]{\mathbf{d}}
\newcommand{\Be}[0]{\mathbf{e}}
\newcommand{\Bf}[0]{\mathbf{f}}
\newcommand{\Bg}[0]{\mathbf{g}}
\newcommand{\Bh}[0]{\mathbf{h}}
\newcommand{\Bi}[0]{\mathbf{i}}
\newcommand{\Bj}[0]{\mathbf{j}}
\newcommand{\Bk}[0]{\mathbf{k}}
\newcommand{\Bl}[0]{\mathbf{l}}
\newcommand{\Bm}[0]{\mathbf{m}}
\newcommand{\Bn}[0]{\mathbf{n}}
\newcommand{\Bo}[0]{\mathbf{o}}
\newcommand{\Bp}[0]{\mathbf{p}}
\newcommand{\Bq}[0]{\mathbf{q}}
\newcommand{\Br}[0]{\mathbf{r}}
\newcommand{\Bs}[0]{\mathbf{s}}
\newcommand{\Bt}[0]{\mathbf{t}}
\newcommand{\Bu}[0]{\mathbf{u}}
\newcommand{\Bv}[0]{\mathbf{v}}
\newcommand{\Bw}[0]{\mathbf{w}}
\newcommand{\Bx}[0]{\mathbf{x}}
\newcommand{\By}[0]{\mathbf{y}}
\newcommand{\Bz}[0]{\mathbf{z}}
\newcommand{\BA}[0]{\mathbf{A}}
\newcommand{\BB}[0]{\mathbf{B}}
\newcommand{\BC}[0]{\mathbf{C}}
\newcommand{\BD}[0]{\mathbf{D}}
\newcommand{\BE}[0]{\mathbf{E}}
\newcommand{\BF}[0]{\mathbf{F}}
\newcommand{\BG}[0]{\mathbf{G}}
\newcommand{\BH}[0]{\mathbf{H}}
\newcommand{\BI}[0]{\mathbf{I}}
\newcommand{\BJ}[0]{\mathbf{J}}
\newcommand{\BK}[0]{\mathbf{K}}
\newcommand{\BL}[0]{\mathbf{L}}
\newcommand{\BM}[0]{\mathbf{M}}
\newcommand{\BN}[0]{\mathbf{N}}
\newcommand{\BO}[0]{\mathbf{O}}
\newcommand{\BP}[0]{\mathbf{P}}
\newcommand{\BQ}[0]{\mathbf{Q}}
\newcommand{\BR}[0]{\mathbf{R}}
\newcommand{\BS}[0]{\mathbf{S}}
\newcommand{\BT}[0]{\mathbf{T}}
\newcommand{\BU}[0]{\mathbf{U}}
\newcommand{\BV}[0]{\mathbf{V}}
\newcommand{\BW}[0]{\mathbf{W}}
\newcommand{\BX}[0]{\mathbf{X}}
\newcommand{\BY}[0]{\mathbf{Y}}
\newcommand{\BZ}[0]{\mathbf{Z}}

\newcommand{\Bzero}[0]{\mathbf{0}}
\newcommand{\Btheta}[0]{\boldsymbol{\theta}}
\newcommand{\Btau}[0]{\boldsymbol{\tau}}
\newcommand{\Bomega}[0]{\boldsymbol{\omega}}

%
% shorthand for unit vectors
%
\newcommand{\acap}[0]{\hat{\Ba}}
\newcommand{\bcap}[0]{\hat{\Bb}}
\newcommand{\ccap}[0]{\hat{\Bc}}
\newcommand{\dcap}[0]{\hat{\Bd}}
\newcommand{\ecap}[0]{\hat{\Be}}
\newcommand{\fcap}[0]{\hat{\Bf}}
\newcommand{\gcap}[0]{\hat{\Bg}}
\newcommand{\hcap}[0]{\hat{\Bh}}
\newcommand{\icap}[0]{\hat{\Bi}}
\newcommand{\jcap}[0]{\hat{\Bj}}
\newcommand{\kcap}[0]{\hat{\Bk}}
\newcommand{\lcap}[0]{\hat{\Bl}}
\newcommand{\mcap}[0]{\hat{\Bm}}
\newcommand{\ncap}[0]{\hat{\Bn}}
\newcommand{\ocap}[0]{\hat{\Bo}}
\newcommand{\pcap}[0]{\hat{\Bp}}
\newcommand{\qcap}[0]{\hat{\Bq}}
\newcommand{\rcap}[0]{\hat{\Br}}
\newcommand{\scap}[0]{\hat{\Bs}}
\newcommand{\tcap}[0]{\hat{\Bt}}
\newcommand{\ucap}[0]{\hat{\Bu}}
\newcommand{\vcap}[0]{\hat{\Bv}}
\newcommand{\wcap}[0]{\hat{\Bw}}
\newcommand{\xcap}[0]{\hat{\Bx}}
\newcommand{\ycap}[0]{\hat{\By}}
\newcommand{\zcap}[0]{\hat{\Bz}}
\newcommand{\thetacap}[0]{\hat{\Btheta}}

%
% to write R^n and C^n in a distinguishable fashion.  Perhaps change this
% to the double lined characters upon figuring out how to do so.
%
\newcommand{\C}[1]{$\mathbb{C}^{#1}$}
\newcommand{\R}[1]{$\mathbb{R}^{#1}$}

%
% various generally useful helpers
%

% derivative of #1 wrt. #2:
\newcommand{\D}[2] {\frac {d#2} {d#1}}

\newcommand{\inv}[1]{\frac{1}{#1}}
\newcommand{\cross}[0]{\times}

\newcommand{\abs}[1]{\lvert{#1}\rvert}
\newcommand{\norm}[1]{\lVert{#1}\rVert}
\newcommand{\innerprod}[2]{\langle{#1}, {#2}\rangle}
\newcommand{\dotprod}[2]{{#1} \cdot {#2}}
\newcommand{\bdotprod}[2]{\left({#1} \cdot {#2}\right)}
\newcommand{\crossprod}[2]{{#1} \cross {#2}}
\newcommand{\tripleprod}[3]{\dotprod{\left(\crossprod{#1}{#2}\right)}{#3}}

\DeclareMathOperator{\Proj}{Proj}
\DeclareMathOperator{\Span}{span}
\DeclareMathOperator{\Sgn}{sgn}
\DeclareMathOperator{\Area}{Area}
\DeclareMathOperator{\Volume}{Volume}

%
% A few miscellaneous things specific to this document
%
\newcommand{\crossop}[1]{\crossprod{#1}{}}

% R2 vector.
\newcommand{\VectorTwo}[2]{
\begin{bmatrix}
 {#1} \\
 {#2}
\end{bmatrix}
}

\newcommand{\VectorN}[1]{
\begin{bmatrix}
{#1}_1 \\
{#1}_2 \\
\vdots \\
{#1}_N \\
\end{bmatrix}
}

\newcommand{\DETuvij}[4]{
\begin{vmatrix}
 {#1}_{#3} & {#1}_{#4} \\
 {#2}_{#3} & {#2}_{#4}
\end{vmatrix}
}

\newcommand{\DETuvwijk}[6]{
\begin{vmatrix}
 {#1}_{#4} & {#1}_{#5} & {#1}_{#6} \\
 {#2}_{#4} & {#2}_{#5} & {#2}_{#6} \\
 {#3}_{#4} & {#3}_{#5} & {#3}_{#6}
\end{vmatrix}
}

\newcommand{\DETuvwxijkl}[8]{
\begin{vmatrix}
 {#1}_{#5} & {#1}_{#6} & {#1}_{#7} & {#1}_{#8} \\
 {#2}_{#5} & {#2}_{#6} & {#2}_{#7} & {#2}_{#8} \\
 {#3}_{#5} & {#3}_{#6} & {#3}_{#7} & {#3}_{#8} \\
 {#4}_{#5} & {#4}_{#6} & {#4}_{#7} & {#4}_{#8} \\
\end{vmatrix}
}

%\newcommand{\DETuvwxyijklm}[10]{
%\begin{vmatrix}
% {#1}_{#6} & {#1}_{#7} & {#1}_{#8} & {#1}_{#9} & {#1}_{#10} \\
% {#2}_{#6} & {#2}_{#7} & {#2}_{#8} & {#2}_{#9} & {#2}_{#10} \\
% {#3}_{#6} & {#3}_{#7} & {#3}_{#8} & {#3}_{#9} & {#3}_{#10} \\
% {#4}_{#6} & {#4}_{#7} & {#4}_{#8} & {#4}_{#9} & {#4}_{#10} \\
% {#5}_{#6} & {#5}_{#7} & {#5}_{#8} & {#5}_{#9} & {#5}_{#10}
%\end{vmatrix}
%}

% R3 vector.
\newcommand{\VectorThree}[3]{
\begin{bmatrix}
 {#1} \\
 {#2} \\
 {#3}
\end{bmatrix}
}



%
% Copyright � 2015 Peeter Joot.  All Rights Reserved.
% Licenced as described in the file LICENSE under the root directory of this GIT repository.
%
\documentclass[]{eliblog}

\usepackage{amsmath}
\usepackage{mathpazo}

%
% shorthand for bold symbols, convenient for vectors and matrices
%
\newcommand{\Ba}[0]{\mathbf{a}}
\newcommand{\Bb}[0]{\mathbf{b}}
\newcommand{\Bc}[0]{\mathbf{c}}
\newcommand{\Bd}[0]{\mathbf{d}}
\newcommand{\Be}[0]{\mathbf{e}}
\newcommand{\Bf}[0]{\mathbf{f}}
\newcommand{\Bg}[0]{\mathbf{g}}
\newcommand{\Bh}[0]{\mathbf{h}}
\newcommand{\Bi}[0]{\mathbf{i}}
\newcommand{\Bj}[0]{\mathbf{j}}
\newcommand{\Bk}[0]{\mathbf{k}}
\newcommand{\Bl}[0]{\mathbf{l}}
\newcommand{\Bm}[0]{\mathbf{m}}
\newcommand{\Bn}[0]{\mathbf{n}}
\newcommand{\Bo}[0]{\mathbf{o}}
\newcommand{\Bp}[0]{\mathbf{p}}
\newcommand{\Bq}[0]{\mathbf{q}}
\newcommand{\Br}[0]{\mathbf{r}}
\newcommand{\Bs}[0]{\mathbf{s}}
\newcommand{\Bt}[0]{\mathbf{t}}
\newcommand{\Bu}[0]{\mathbf{u}}
\newcommand{\Bv}[0]{\mathbf{v}}
\newcommand{\Bw}[0]{\mathbf{w}}
\newcommand{\Bx}[0]{\mathbf{x}}
\newcommand{\By}[0]{\mathbf{y}}
\newcommand{\Bz}[0]{\mathbf{z}}
\newcommand{\BA}[0]{\mathbf{A}}
\newcommand{\BB}[0]{\mathbf{B}}
\newcommand{\BC}[0]{\mathbf{C}}
\newcommand{\BD}[0]{\mathbf{D}}
\newcommand{\BE}[0]{\mathbf{E}}
\newcommand{\BF}[0]{\mathbf{F}}
\newcommand{\BG}[0]{\mathbf{G}}
\newcommand{\BH}[0]{\mathbf{H}}
\newcommand{\BI}[0]{\mathbf{I}}
\newcommand{\BJ}[0]{\mathbf{J}}
\newcommand{\BK}[0]{\mathbf{K}}
\newcommand{\BL}[0]{\mathbf{L}}
\newcommand{\BM}[0]{\mathbf{M}}
\newcommand{\BN}[0]{\mathbf{N}}
\newcommand{\BO}[0]{\mathbf{O}}
\newcommand{\BP}[0]{\mathbf{P}}
\newcommand{\BQ}[0]{\mathbf{Q}}
\newcommand{\BR}[0]{\mathbf{R}}
\newcommand{\BS}[0]{\mathbf{S}}
\newcommand{\BT}[0]{\mathbf{T}}
\newcommand{\BU}[0]{\mathbf{U}}
\newcommand{\BV}[0]{\mathbf{V}}
\newcommand{\BW}[0]{\mathbf{W}}
\newcommand{\BX}[0]{\mathbf{X}}
\newcommand{\BY}[0]{\mathbf{Y}}
\newcommand{\BZ}[0]{\mathbf{Z}}

\newcommand{\Bzero}[0]{\mathbf{0}}
\newcommand{\Btheta}[0]{\boldsymbol{\theta}}
\newcommand{\Btau}[0]{\boldsymbol{\tau}}
\newcommand{\Bomega}[0]{\boldsymbol{\omega}}

%
% shorthand for unit vectors
%
\newcommand{\acap}[0]{\hat{\Ba}}
\newcommand{\bcap}[0]{\hat{\Bb}}
\newcommand{\ccap}[0]{\hat{\Bc}}
\newcommand{\dcap}[0]{\hat{\Bd}}
\newcommand{\ecap}[0]{\hat{\Be}}
\newcommand{\fcap}[0]{\hat{\Bf}}
\newcommand{\gcap}[0]{\hat{\Bg}}
\newcommand{\hcap}[0]{\hat{\Bh}}
\newcommand{\icap}[0]{\hat{\Bi}}
\newcommand{\jcap}[0]{\hat{\Bj}}
\newcommand{\kcap}[0]{\hat{\Bk}}
\newcommand{\lcap}[0]{\hat{\Bl}}
\newcommand{\mcap}[0]{\hat{\Bm}}
\newcommand{\ncap}[0]{\hat{\Bn}}
\newcommand{\ocap}[0]{\hat{\Bo}}
\newcommand{\pcap}[0]{\hat{\Bp}}
\newcommand{\qcap}[0]{\hat{\Bq}}
\newcommand{\rcap}[0]{\hat{\Br}}
\newcommand{\scap}[0]{\hat{\Bs}}
\newcommand{\tcap}[0]{\hat{\Bt}}
\newcommand{\ucap}[0]{\hat{\Bu}}
\newcommand{\vcap}[0]{\hat{\Bv}}
\newcommand{\wcap}[0]{\hat{\Bw}}
\newcommand{\xcap}[0]{\hat{\Bx}}
\newcommand{\ycap}[0]{\hat{\By}}
\newcommand{\zcap}[0]{\hat{\Bz}}
\newcommand{\thetacap}[0]{\hat{\Btheta}}

%
% to write R^n and C^n in a distinguishable fashion.  Perhaps change this
% to the double lined characters upon figuring out how to do so.
%
\newcommand{\C}[1]{$\mathbb{C}^{#1}$}
\newcommand{\R}[1]{$\mathbb{R}^{#1}$}

%
% various generally useful helpers
%

% derivative of #1 wrt. #2:
\newcommand{\D}[2] {\frac {d#2} {d#1}}

\newcommand{\inv}[1]{\frac{1}{#1}}
\newcommand{\cross}[0]{\times}

\newcommand{\abs}[1]{\lvert{#1}\rvert}
\newcommand{\norm}[1]{\lVert{#1}\rVert}
\newcommand{\innerprod}[2]{\langle{#1}, {#2}\rangle}
\newcommand{\dotprod}[2]{{#1} \cdot {#2}}
\newcommand{\bdotprod}[2]{\left({#1} \cdot {#2}\right)}
\newcommand{\crossprod}[2]{{#1} \cross {#2}}
\newcommand{\tripleprod}[3]{\dotprod{\left(\crossprod{#1}{#2}\right)}{#3}}

\DeclareMathOperator{\Proj}{Proj}
\DeclareMathOperator{\Span}{span}
\DeclareMathOperator{\Sgn}{sgn}
\DeclareMathOperator{\Area}{Area}
\DeclareMathOperator{\Volume}{Volume}

%
% A few miscellaneous things specific to this document
%
\newcommand{\crossop}[1]{\crossprod{#1}{}}

% R2 vector.
\newcommand{\VectorTwo}[2]{
\begin{bmatrix}
 {#1} \\
 {#2}
\end{bmatrix}
}

\newcommand{\VectorN}[1]{
\begin{bmatrix}
{#1}_1 \\
{#1}_2 \\
\vdots \\
{#1}_N \\
\end{bmatrix}
}

\newcommand{\DETuvij}[4]{
\begin{vmatrix}
 {#1}_{#3} & {#1}_{#4} \\
 {#2}_{#3} & {#2}_{#4}
\end{vmatrix}
}

\newcommand{\DETuvwijk}[6]{
\begin{vmatrix}
 {#1}_{#4} & {#1}_{#5} & {#1}_{#6} \\
 {#2}_{#4} & {#2}_{#5} & {#2}_{#6} \\
 {#3}_{#4} & {#3}_{#5} & {#3}_{#6}
\end{vmatrix}
}

\newcommand{\DETuvwxijkl}[8]{
\begin{vmatrix}
 {#1}_{#5} & {#1}_{#6} & {#1}_{#7} & {#1}_{#8} \\
 {#2}_{#5} & {#2}_{#6} & {#2}_{#7} & {#2}_{#8} \\
 {#3}_{#5} & {#3}_{#6} & {#3}_{#7} & {#3}_{#8} \\
 {#4}_{#5} & {#4}_{#6} & {#4}_{#7} & {#4}_{#8} \\
\end{vmatrix}
}

%\newcommand{\DETuvwxyijklm}[10]{
%\begin{vmatrix}
% {#1}_{#6} & {#1}_{#7} & {#1}_{#8} & {#1}_{#9} & {#1}_{#10} \\
% {#2}_{#6} & {#2}_{#7} & {#2}_{#8} & {#2}_{#9} & {#2}_{#10} \\
% {#3}_{#6} & {#3}_{#7} & {#3}_{#8} & {#3}_{#9} & {#3}_{#10} \\
% {#4}_{#6} & {#4}_{#7} & {#4}_{#8} & {#4}_{#9} & {#4}_{#10} \\
% {#5}_{#6} & {#5}_{#7} & {#5}_{#8} & {#5}_{#9} & {#5}_{#10}
%\end{vmatrix}
%}

% R3 vector.
\newcommand{\VectorThree}[3]{
\begin{bmatrix}
 {#1} \\
 {#2} \\
 {#3}
\end{bmatrix}
}



\author{Peeter Joot}
\email{peeter.joot@gmail.com}

\author{Peeter Joot}
\email{peeter.joot@utoronto.ca, 920798560}

\chapter{PHY450H1S Problem Set 2.}
\label{chap:relElectroDynProblemSet2}
%\blogpage{http://sites.google.com/site/peeterjoot/math2011/relElectroDynProblemSet2.pdf}
\date{Feb 1, 2011}
\revisionInfo{relElectroDynProblemSet2.tex}

\beginArtNoToc
%\section{Disclaimer.}
%
%This problem set is as yet ungraded.

\section{Problem 1.}
\subsection{Statement}

A particle of rest mass $m$ whose energy is three times its rest energy collides with an identical particle at rest.  Suppose they stick together.  Use conservation laws to find the mass of the resulting particle and its velocity.  Is its mass greater or smaller than $2m$?  Comment.

\subsection{Solution}

The energy of the initially moving particle before collision is

\begin{equation}\label{eqn:relElectroDynProblemSet2:n}
\mathcal{E} = \frac{m c^2 }{\InvGamma} = 3 m c^2.
\end{equation}

Solving for the velocity we have

\begin{equation}\label{eqn:relElectroDynProblemSet2:n}
\Abs{\frac{\Bv}{c}} = \frac{2 \sqrt{2}}{3}.
\end{equation}

Our four velocity is

\begin{equation}\label{eqn:relElectroDynProblemSet2:n}
u^i
= \gamma \left( 1, \frac{\Bv}{c} \right) = ( 3, 2 \sqrt{2} ).
\end{equation}

Designate the four momentum for this particle as

\begin{equation}\label{eqn:relElectroDynProblemSet2:n}
p_{(1)}^i = m c ( 3, 2 \sqrt{2} ).
\end{equation}

For the second particle we have

\begin{equation}\label{eqn:relElectroDynProblemSet2:n}
p_{(2)}^i = m c ( 1, 0 ).
\end{equation}

Our initial and final four momentum will be equal, and our resulting velocity can only be in the direction of the initial particle.  This leaves us with

\begin{align*}
p_{(f)}^i
&= M c \inv{\sqrt{1 - \frac{\Bv_f^2}{c^2}}} \left( 1, \frac{\Bv_f}{c} \right) \\
&= m c ( 1, 0 ) + m c ( 3, 2 \sqrt{2} )  \\
&= m c ( 4, 2 \sqrt{2} ) \\
&= 4 m c \left( 1, \inv{\sqrt{2}} \right)
\end{align*}

Our final velocity is $v_f = c/\sqrt{2}$.

We have $M \gamma = 4$ for the final particle, but we have

\begin{equation}\label{eqn:relElectroDynProblemSet2:n}
\gamma = \frac{1}{\sqrt{1 - 1/2}} = \sqrt{2},
\end{equation}

so our final mass is

\begin{equation}\label{eqn:relElectroDynProblemSet2:n}
M = \frac{4}{\sqrt{2}} = 2 \sqrt{2} > 2
\end{equation}

\section{Problem 2.}
\subsection{Statement}

This problem has three parts
\begin{enumerate}
\item Express the ``normal'' (i.e. not 4-, but 3-) acceleration, equal to $\dot{\Bv}$, or a particle in terms of its velocity, $\BE$, and $\BB$, using the equation of motion of a relativistic particle in an external electromagnetic field.
\item
\item
\end{enumerate}
\subsection{Solution}
\subsubsection{1.}

With the particle's energy given by

\begin{equation}\label{eqn:relElectroDynProblemSet2:n}
\mathcal{E} = \gamma m c^2,
\end{equation}

we note that

\begin{equation}\label{eqn:relElectroDynProblemSet2:n}
\mathcal{E}\Bv = (\gamma m \Bv) c^2 = \Bp c^2.
\end{equation}

Taking derivatives we have

\begin{align*}
c^2 \ddt{\Bp} 
&= \Bv \ddt{\mathcal{E}} + \ddt{\Bv} \mathcal{E} \\
&= \Bv (e \BE \cdot \Bv) + \ddt{\Bv} \mathcal{E} \\
\end{align*}

Rearranging we have

\begin{equation}\label{eqn:relElectroDynProblemSet2:n}
\ddt{\Bv}
=
\frac{c^2 e \left( \BE + \frac{\Bv}{c} \cross \BB \right) - \Bv (e \BE \cdot \Bv) }{ \mathcal{E} } 
\end{equation}

which leaves us with the desired result
\begin{equation}\label{eqn:relElectroDynProblemSet2:n}
\boxed{
\dot{\Bv} =
\frac{e}{m} \InvGamma \left( \BE + \frac{\Bv}{c} \cross \BB - \frac{\Bv}{c} \left(\BE \cdot \frac{\Bv}{c} \right) \right)
}
\end{equation}

\subsubsection{1b.  On the energy change rate.}

Note that when the problem set was assigned, the relation

\begin{equation}\label{eqn:relElectroDynProblemSet2:n}
\ddt{\mathcal{E}} = e \BE \cdot \Bv
\end{equation}

had not been demonstrated.  To show this observe that we have

\begin{align*}
\frac{d}{dt} \mathcal{E}
&= m c^2 \frac{d\gamma}{dt} \\
&= m c^2 \frac{d}{dt} \inv{\InvGamma} \\
&= m c^2 \frac{\frac{\Bv}{c^2} \cdot \frac{d\Bv}{dt}}{\left(1 - \frac{\Bv^2}{c^2}\right)^{3/2}} \\
&= \frac{m \gamma \Bv \cdot \frac{d\Bv}{dt}}{1 - \frac{\Bv^2}{c^2}}
\end{align*}

We also have

\begin{align*}
\Bv \cdot \ddt{\Bp} 
&= \Bv \cdot \ddt{} \frac{m \Bv}{\InvGamma} \\
&= m\Bv^2 \ddt{\gamma} + m \gamma \Bv \cdot \ddt{\Bv} \\
&= m\Bv^2 \ddt{\gamma} + m c^2 \ddt{\gamma} \left( 1 - \frac{\Bv^2}{c^2} \right) \\
&= m c^2 \ddt{\gamma}.
\end{align*}

Utilizing the Lorentz force equation, we have

\begin{equation}\label{eqn:relElectroDynProblemSet2:n}
\Bv \cdot \ddt{\Bp} = e \left( \BE + \frac{\Bv}{c} \cross \BB \right) \cdot \Bv = e \BE \cdot \Bv
\end{equation}

and are able to assemble the above, and find that we have
\begin{equation}\label{eqn:relElectroDynProblemSet2:n}
\ddt{(m c^2 \gamma)} = e \BE \cdot \Bv 
\end{equation}

\subsubsection{2.}
\subsubsection{3.}

\section{Problem 3.}
\subsection{Statement}

In class, we introduced the 4-vector potential $A^i$ and its transformation law under Lorentz transformations.  While we have not yet discussed how $\BE$ and $\BB$ transform, knowing how $A^i$ transforms is enough to solve some concrete problems.  Suppose in one (unprimed) frame there is a charge at rest, which creates an electrostatic field: $A^0 = \phi = \frac{q}{r}, \BA = 0$.

\begin{enumerate}
\item Find the values of $\BE$ and $\BB$ in this frame.
\item Consider now the same field in a (primed) frame moving in the $x$-direction with velocity $v$.  Using the transformation law of the vector potential, find ${A^i}'$ in the primed frame.
\item Use the relations between electric and magnetic field strengths and vector potential (valid in every frame) to find the electric and magnetic fields in the primed frame (i.e. find the electromagnetic field of a moving charge).  Sketch the lines of constant electric and magnetic field and comment on the result.
\end{enumerate}

\subsection{Solution}
\subsubsection{1.}

In the unprimed frame we have

\begin{align*}
\BE 
&= - \spacegrad \phi - \inv{c} \PD{t}{\BA} \\
&= -\spacegrad \phi \\
&= - \rcap q \partial_r (1/r) \\
&= \rcap \frac{q}{r^2},
\end{align*}

and
\begin{align*}
\BB = \spacegrad \cross \BA = 0
\end{align*}

\subsubsection{2.}

The coordinates in the moving frame, assuming the frames are overlapping at $t=0$, are related to the unprimed coordinates by

\begin{equation}\label{eqn:relElectroDynProblemSet2:n}
\begin{bmatrix}
ct' \\
x' \\
y' \\
z'
\end{bmatrix}
=
\begin{bmatrix}
\gamma & -\gamma \beta & 0 & 0 \\
-\gamma \beta & \gamma & 0 & 0 \\
0 & 0 & 1 & 0 \\
0 & 0 & 0 & 1 
\end{bmatrix}
\begin{bmatrix}
ct \\
x \\
y \\
z
\end{bmatrix}
\end{equation}

Our four vector potential also transforms in the same fashion, and we have

\begin{equation}\label{eqn:relElectroDynProblemSet2:n}
\begin{bmatrix}
\phi' \\
A_x' \\
A_y' \\
A_z' \\
\end{bmatrix}
=
\begin{bmatrix}
\gamma & -\gamma \beta & 0 & 0 \\
-\gamma \beta & \gamma & 0 & 0 \\
0 & 0 & 1 & 0 \\
0 & 0 & 0 & 1 
\end{bmatrix}
\begin{bmatrix}
\phi \\
0 \\
0 \\
0 \\
\end{bmatrix}
= \gamma \phi ( 1, -\beta, 0, 0 )
\end{equation}

So in the primed frame we have
\begin{align}\label{eqn:relElectroDynProblemSet2:n}
\phi' &= \gamma \frac{q}{r} \\
A_x' &= -\gamma \beta \frac{q}{r} \\
A_y' &= 0 \\
A_z' &= 0 
\end{align}

\subsubsection{3.}

In the primed frame our electric and magnetic fields are

\begin{align}\label{eqn:relElectroDynProblemSet2:n}
\BE' &= - \spacegrad' \phi' - \inv{c} \PD{t'}{\BA'} \\
\BB' &= \spacegrad' \cross \BA'
\end{align}

We have $\phi'$ and $\BA'$ expressed in terms of the unprimed coordinates, so need to calculate the transformation of the gradient and time partial too.  These partials transform as

\begin{align}\label{eqn:relElectroDynProblemSet2:n}
\PD{c t'}{} &= \PD{ct'}{ct} \PD{ct}{} + \PD{ct'}{x} \PD{x}{} \\
\PD{x'}{} &= \PD{x'}{ct} \PD{ct}{} + \PD{x'}{x} \PD{x}{} \\
\PD{y'}{} &= \PD{y}{} \\
\PD{z'}{} &= \PD{z}{}
\end{align}

Utilizing the inverse transformation 

\begin{equation}\label{eqn:relElectroDynProblemSet2:n}
\begin{bmatrix}
ct \\
x \\
y \\
z
\end{bmatrix}
=
\begin{bmatrix}
\gamma & \gamma \beta & 0 & 0 \\
\gamma \beta & \gamma & 0 & 0 \\
0 & 0 & 1 & 0 \\
0 & 0 & 0 & 1 
\end{bmatrix}
\begin{bmatrix}
ct' \\
x' \\
y' \\
z'
\end{bmatrix}
\end{equation}

we have
\begin{align}\label{eqn:relElectroDynProblemSet2:n}
\PD{c t'}{} &= \gamma \PD{ct}{} + \gamma \beta \PD{x}{} \\
\PD{x'}{} &= \gamma \beta \PD{ct}{} + \gamma \PD{x}{} \\
\PD{y'}{} &= \PD{y}{} \\
\PD{z'}{} &= \PD{z}{}
\end{align}

Since neither $\phi'$ nor $\BA'$ have time dependence, we have for electric field in the primed frame

\begin{align*}
\BE' 
&= -\spacegrad' \phi' - \inv{c} \PD{t'}{\BA'} \\
&= 
-\left( \gamma \PD{x}{}, \PD{y}{}, \PD{z}{} \right) \phi'
- \gamma \beta \PD{x}{\BA'} \\
&= 
-\left( \gamma \PD{x}{}, \PD{y}{}, \PD{z}{} \right) \gamma \frac{q}{r}
- \gamma \beta \PD{x}{} \left( -\gamma \beta \frac{q}{r}, 0, 0 \right) \\
&= -q \left( \gamma^2 ( 1 - \beta^2 ) \PD{x}{}, \gamma \PD{y}{}, \gamma \PD{z}{} \right) \inv{r} \\
&= -q \left( \PD{x}{}, \gamma \PD{y}{}, \gamma \PD{z}{} \right) \inv{r}
\end{align*}

Our electric field in the primed frame is thus
\begin{equation}\label{eqn:relElectroDynProblemSet2:n}
\BE' = -\frac{q}{r^3} \left( x, \gamma y, \gamma z \right) 
\end{equation}

Now for the magnetic field.  We want

\begin{align*}
\BB' 
&= 
\begin{vmatrix}
\xcap & \ycap & \zcap \\
\partial_{x'} & \partial_{y'} & \partial_{z'} \\
-\gamma \beta q/r & 0 & 0
\end{vmatrix} \\
&=
\left( 0, \partial_{z'}, -\partial_{y'} \right) \frac{-\gamma \beta q}{r} \\
\end{align*}

\begin{equation}\label{eqn:relElectroDynProblemSet2:n}
\BB'
=
\frac{q \gamma \beta}{r^3} \left( 0, -z, y \right)
\end{equation}

FIXME: sketch and comment.

%\EndArticle
\EndNoBibArticle
