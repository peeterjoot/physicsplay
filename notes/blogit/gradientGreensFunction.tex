%
% Copyright � 2016 Peeter Joot.  All Rights Reserved.
% Licenced as described in the file LICENSE under the root directory of this GIT repository.
%
%{
\newcommand{\authorname}{Peeter Joot}
\newcommand{\email}{peeterjoot@protonmail.com}
\newcommand{\basename}{FIXMEbasenameUndefined}
\newcommand{\dirname}{notes/FIXMEdirnameUndefined/}

\renewcommand{\basename}{gradientGreensFunction}
\renewcommand{\dirname}{notes/phy1520/}
%\newcommand{\dateintitle}{}
%\newcommand{\keywords}{}

\newcommand{\authorname}{Peeter Joot}
\newcommand{\onlineurl}{http://sites.google.com/site/peeterjoot2/math2013/\basename.pdf}
\newcommand{\sourcepath}{\dirname\basename.tex}
\newcommand{\generatetitle}[1]{\chapter{#1}}

\newcommand{\vcsinfo}{%
\section*{}
\noindent{\color{DarkOliveGreen}{\rule{\linewidth}{0.1mm}}}
\paragraph{Document version}
%\paragraph{\color{Maroon}{Document version}}
{
\small
\begin{itemize}
\item Available online at:\\ 
\href{\onlineurl}{\onlineurl}
\item Git Repository: \input{./.revinfo/gitRepo.tex}
\item Source: \sourcepath
\item last commit: \input{./.revinfo/gitCommitString.tex}
\item commit date: \input{./.revinfo/gitCommitDate.tex}
\end{itemize}
}
}

%\PassOptionsToPackage{dvipsnames,svgnames}{xcolor}
\PassOptionsToPackage{square,numbers}{natbib}
\documentclass{scrreprt}

\usepackage[left=2cm,right=2cm]{geometry}
\usepackage[svgnames]{xcolor}
\usepackage{peeters_layout}

\usepackage{natbib}

\usepackage[
colorlinks=true,
bookmarks=false,
pdfauthor={\authorname, \email},
backref 
]{hyperref}

% http://tex.stackexchange.com/questions/75773/how-to-reference-problems-by-the-text-label-in-an-exercise-envioronment
\usepackage[english]{cleveref}
\crefname{Exercise}{exercise}{exercises}
\Crefname{Exercise}{Exercise}{Exercises}

\RequirePackage{titlesec}
\RequirePackage{ifthen}

% http://stackoverflow.com/questions/4932910/date-in-the-tabular-environment
\makeatletter
\let\insertdate\@date
\makeatother

\titleformat{\chapter}[display]
{\bfseries\Large}
{\color{DarkSlateGrey}\filleft \authorname
\ifthenelse{\isundefined{\studentnumber}}{}{\\ \studentnumber}
\ifthenelse{\isundefined{\email}}{}{\\ \email}
\ifthenelse{\isundefined{\dateintitle}}{}{\\ \insertdate}
%\ifthenelse{\isundefined{\coursename}}{}{\\ \coursename} % put in title instead.
}
{4ex}
{\color{DarkOliveGreen}{\titlerule}\color{Maroon}
\vspace{2ex}%
\filright}
[\vspace{2ex}%
\color{DarkOliveGreen}\titlerule
]

\newcommand{\beginArtWithToc}[0]{\begin{document}\tableofcontents}
\newcommand{\beginArtNoToc}[0]{\begin{document}}
\newcommand{\EndNoBibArticle}[0]{\end{document}}
\newcommand{\EndArticle}[0]{\bibliography{Bibliography}\bibliographystyle{plainnat}\end{document}}

% 
%\newcommand{\citep}[1]{\cite{#1}}

\colorSectionsForArticle



\usepackage{peeters_layout_exercise}
\usepackage{peeters_braket}
\usepackage{peeters_figures}
\usepackage{siunitx}

\beginArtNoToc

\generatetitle{Green's function for gradient in Euclidean spaces}
%\chapter{Green's function for gradient in Euclidean spaces}
%\label{chap:gradientGreensFunction}

In \citep{doran2003gap} it is stated that the Green's function for the gradient is

\begin{dmath}\label{eqn:gradientGreensFunction:20}
   G(x, x') = \inv{S_n} \frac{x - x'}{\Abs{x-x'}^n},
\end{dmath}

where \( n \) is the dimension of the space, \( S_n \) is the area of the unit sphere, and 
\begin{equation}\label{eqn:gradientGreensFunction:40}
   \grad G = \grad \cdot G = \delta(x - x').
\end{equation}

What I'd like to do here is verify that this Green's function operates as asserted.  The first part, verification that the gradient of this function is zero everywhere that \( x \ne x' \).
Note that I am following the convention of the text, using non-boldface for vectors that aren't explicitly from \R{3}.  Attention will be restricted to Euclidean spaces, since I don't know what it means to calculate the surface area of a unit sphere in a mixed signature space.

The starting point of the demonstration is the Fundamental Theorem, with the gradient acting bidirectionally on both the gradient and the test function.  Working in primed coordinates so that the final result is in terms of the unprimed, we have

\begin{dmath}\label{eqn:gradientGreensFunction:60}
   \int_V G(x,x') d^n x' \lrgrad' F(x') 
   = \int_{\partial V} G(x,x') d^{n-1} x' F(x').
\end{dmath}

Let \( d^n x' = dV' I \), \( d^{n-1} x' n = dA' I \), where \( n \) is the outward normal to the area element \( d^{n-1} x' \).  In a Euclidean space where \( n^2 = 1 \), this gives

\begin{dmath}\label{eqn:gradientGreensFunction:80}
\int_V dV' G(x,x') \lrgrad' F(x') 
=
\int_V dV' \lr{G(x,x') \lgrad'} F(x') 
+
\int_V dV' G(x,x') \lr{ \rgrad' F(x') }
= \int_{\partial V} dA' G(x,x') n F(x').
\end{dmath}

Here, the pseudoscalar \( I \) has been factored out by commuting it with \( G \), using \( G I = (-1)^{n-1} I G \), and then pre-multiplication with \( 1/((-1)^{n-1} I ) \).

%}
\EndArticle
