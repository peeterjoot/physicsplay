%
% Copyright � 2015 Peeter Joot.  All Rights Reserved.
% Licenced as described in the file LICENSE under the root directory of this GIT repository.
%
\documentclass[]{eliblog}

\usepackage{amsmath}
\usepackage{mathpazo}

%
% shorthand for bold symbols, convenient for vectors and matrices
%
\newcommand{\Ba}[0]{\mathbf{a}}
\newcommand{\Bb}[0]{\mathbf{b}}
\newcommand{\Bc}[0]{\mathbf{c}}
\newcommand{\Bd}[0]{\mathbf{d}}
\newcommand{\Be}[0]{\mathbf{e}}
\newcommand{\Bf}[0]{\mathbf{f}}
\newcommand{\Bg}[0]{\mathbf{g}}
\newcommand{\Bh}[0]{\mathbf{h}}
\newcommand{\Bi}[0]{\mathbf{i}}
\newcommand{\Bj}[0]{\mathbf{j}}
\newcommand{\Bk}[0]{\mathbf{k}}
\newcommand{\Bl}[0]{\mathbf{l}}
\newcommand{\Bm}[0]{\mathbf{m}}
\newcommand{\Bn}[0]{\mathbf{n}}
\newcommand{\Bo}[0]{\mathbf{o}}
\newcommand{\Bp}[0]{\mathbf{p}}
\newcommand{\Bq}[0]{\mathbf{q}}
\newcommand{\Br}[0]{\mathbf{r}}
\newcommand{\Bs}[0]{\mathbf{s}}
\newcommand{\Bt}[0]{\mathbf{t}}
\newcommand{\Bu}[0]{\mathbf{u}}
\newcommand{\Bv}[0]{\mathbf{v}}
\newcommand{\Bw}[0]{\mathbf{w}}
\newcommand{\Bx}[0]{\mathbf{x}}
\newcommand{\By}[0]{\mathbf{y}}
\newcommand{\Bz}[0]{\mathbf{z}}
\newcommand{\BA}[0]{\mathbf{A}}
\newcommand{\BB}[0]{\mathbf{B}}
\newcommand{\BC}[0]{\mathbf{C}}
\newcommand{\BD}[0]{\mathbf{D}}
\newcommand{\BE}[0]{\mathbf{E}}
\newcommand{\BF}[0]{\mathbf{F}}
\newcommand{\BG}[0]{\mathbf{G}}
\newcommand{\BH}[0]{\mathbf{H}}
\newcommand{\BI}[0]{\mathbf{I}}
\newcommand{\BJ}[0]{\mathbf{J}}
\newcommand{\BK}[0]{\mathbf{K}}
\newcommand{\BL}[0]{\mathbf{L}}
\newcommand{\BM}[0]{\mathbf{M}}
\newcommand{\BN}[0]{\mathbf{N}}
\newcommand{\BO}[0]{\mathbf{O}}
\newcommand{\BP}[0]{\mathbf{P}}
\newcommand{\BQ}[0]{\mathbf{Q}}
\newcommand{\BR}[0]{\mathbf{R}}
\newcommand{\BS}[0]{\mathbf{S}}
\newcommand{\BT}[0]{\mathbf{T}}
\newcommand{\BU}[0]{\mathbf{U}}
\newcommand{\BV}[0]{\mathbf{V}}
\newcommand{\BW}[0]{\mathbf{W}}
\newcommand{\BX}[0]{\mathbf{X}}
\newcommand{\BY}[0]{\mathbf{Y}}
\newcommand{\BZ}[0]{\mathbf{Z}}

\newcommand{\Bzero}[0]{\mathbf{0}}
\newcommand{\Btheta}[0]{\boldsymbol{\theta}}
\newcommand{\Btau}[0]{\boldsymbol{\tau}}
\newcommand{\Bomega}[0]{\boldsymbol{\omega}}

%
% shorthand for unit vectors
%
\newcommand{\acap}[0]{\hat{\Ba}}
\newcommand{\bcap}[0]{\hat{\Bb}}
\newcommand{\ccap}[0]{\hat{\Bc}}
\newcommand{\dcap}[0]{\hat{\Bd}}
\newcommand{\ecap}[0]{\hat{\Be}}
\newcommand{\fcap}[0]{\hat{\Bf}}
\newcommand{\gcap}[0]{\hat{\Bg}}
\newcommand{\hcap}[0]{\hat{\Bh}}
\newcommand{\icap}[0]{\hat{\Bi}}
\newcommand{\jcap}[0]{\hat{\Bj}}
\newcommand{\kcap}[0]{\hat{\Bk}}
\newcommand{\lcap}[0]{\hat{\Bl}}
\newcommand{\mcap}[0]{\hat{\Bm}}
\newcommand{\ncap}[0]{\hat{\Bn}}
\newcommand{\ocap}[0]{\hat{\Bo}}
\newcommand{\pcap}[0]{\hat{\Bp}}
\newcommand{\qcap}[0]{\hat{\Bq}}
\newcommand{\rcap}[0]{\hat{\Br}}
\newcommand{\scap}[0]{\hat{\Bs}}
\newcommand{\tcap}[0]{\hat{\Bt}}
\newcommand{\ucap}[0]{\hat{\Bu}}
\newcommand{\vcap}[0]{\hat{\Bv}}
\newcommand{\wcap}[0]{\hat{\Bw}}
\newcommand{\xcap}[0]{\hat{\Bx}}
\newcommand{\ycap}[0]{\hat{\By}}
\newcommand{\zcap}[0]{\hat{\Bz}}
\newcommand{\thetacap}[0]{\hat{\Btheta}}

%
% to write R^n and C^n in a distinguishable fashion.  Perhaps change this
% to the double lined characters upon figuring out how to do so.
%
\newcommand{\C}[1]{$\mathbb{C}^{#1}$}
\newcommand{\R}[1]{$\mathbb{R}^{#1}$}

%
% various generally useful helpers
%

% derivative of #1 wrt. #2:
\newcommand{\D}[2] {\frac {d#2} {d#1}}

\newcommand{\inv}[1]{\frac{1}{#1}}
\newcommand{\cross}[0]{\times}

\newcommand{\abs}[1]{\lvert{#1}\rvert}
\newcommand{\norm}[1]{\lVert{#1}\rVert}
\newcommand{\innerprod}[2]{\langle{#1}, {#2}\rangle}
\newcommand{\dotprod}[2]{{#1} \cdot {#2}}
\newcommand{\bdotprod}[2]{\left({#1} \cdot {#2}\right)}
\newcommand{\crossprod}[2]{{#1} \cross {#2}}
\newcommand{\tripleprod}[3]{\dotprod{\left(\crossprod{#1}{#2}\right)}{#3}}

\DeclareMathOperator{\Proj}{Proj}
\DeclareMathOperator{\Span}{span}
\DeclareMathOperator{\Sgn}{sgn}
\DeclareMathOperator{\Area}{Area}
\DeclareMathOperator{\Volume}{Volume}

%
% A few miscellaneous things specific to this document
%
\newcommand{\crossop}[1]{\crossprod{#1}{}}

% R2 vector.
\newcommand{\VectorTwo}[2]{
\begin{bmatrix}
 {#1} \\
 {#2}
\end{bmatrix}
}

\newcommand{\VectorN}[1]{
\begin{bmatrix}
{#1}_1 \\
{#1}_2 \\
\vdots \\
{#1}_N \\
\end{bmatrix}
}

\newcommand{\DETuvij}[4]{
\begin{vmatrix}
 {#1}_{#3} & {#1}_{#4} \\
 {#2}_{#3} & {#2}_{#4}
\end{vmatrix}
}

\newcommand{\DETuvwijk}[6]{
\begin{vmatrix}
 {#1}_{#4} & {#1}_{#5} & {#1}_{#6} \\
 {#2}_{#4} & {#2}_{#5} & {#2}_{#6} \\
 {#3}_{#4} & {#3}_{#5} & {#3}_{#6}
\end{vmatrix}
}

\newcommand{\DETuvwxijkl}[8]{
\begin{vmatrix}
 {#1}_{#5} & {#1}_{#6} & {#1}_{#7} & {#1}_{#8} \\
 {#2}_{#5} & {#2}_{#6} & {#2}_{#7} & {#2}_{#8} \\
 {#3}_{#5} & {#3}_{#6} & {#3}_{#7} & {#3}_{#8} \\
 {#4}_{#5} & {#4}_{#6} & {#4}_{#7} & {#4}_{#8} \\
\end{vmatrix}
}

%\newcommand{\DETuvwxyijklm}[10]{
%\begin{vmatrix}
% {#1}_{#6} & {#1}_{#7} & {#1}_{#8} & {#1}_{#9} & {#1}_{#10} \\
% {#2}_{#6} & {#2}_{#7} & {#2}_{#8} & {#2}_{#9} & {#2}_{#10} \\
% {#3}_{#6} & {#3}_{#7} & {#3}_{#8} & {#3}_{#9} & {#3}_{#10} \\
% {#4}_{#6} & {#4}_{#7} & {#4}_{#8} & {#4}_{#9} & {#4}_{#10} \\
% {#5}_{#6} & {#5}_{#7} & {#5}_{#8} & {#5}_{#9} & {#5}_{#10}
%\end{vmatrix}
%}

% R3 vector.
\newcommand{\VectorThree}[3]{
\begin{bmatrix}
 {#1} \\
 {#2} \\
 {#3}
\end{bmatrix}
}



\author{Peeter Joot}
\email{peeter.joot@gmail.com}

%\documentclass[]{eliblogwidescreen}

\usepackage{amsmath}
\usepackage{mathpazo}

%
% shorthand for bold symbols, convenient for vectors and matrices
%
\newcommand{\Ba}[0]{\mathbf{a}}
\newcommand{\Bb}[0]{\mathbf{b}}
\newcommand{\Bc}[0]{\mathbf{c}}
\newcommand{\Bd}[0]{\mathbf{d}}
\newcommand{\Be}[0]{\mathbf{e}}
\newcommand{\Bf}[0]{\mathbf{f}}
\newcommand{\Bg}[0]{\mathbf{g}}
\newcommand{\Bh}[0]{\mathbf{h}}
\newcommand{\Bi}[0]{\mathbf{i}}
\newcommand{\Bj}[0]{\mathbf{j}}
\newcommand{\Bk}[0]{\mathbf{k}}
\newcommand{\Bl}[0]{\mathbf{l}}
\newcommand{\Bm}[0]{\mathbf{m}}
\newcommand{\Bn}[0]{\mathbf{n}}
\newcommand{\Bo}[0]{\mathbf{o}}
\newcommand{\Bp}[0]{\mathbf{p}}
\newcommand{\Bq}[0]{\mathbf{q}}
\newcommand{\Br}[0]{\mathbf{r}}
\newcommand{\Bs}[0]{\mathbf{s}}
\newcommand{\Bt}[0]{\mathbf{t}}
\newcommand{\Bu}[0]{\mathbf{u}}
\newcommand{\Bv}[0]{\mathbf{v}}
\newcommand{\Bw}[0]{\mathbf{w}}
\newcommand{\Bx}[0]{\mathbf{x}}
\newcommand{\By}[0]{\mathbf{y}}
\newcommand{\Bz}[0]{\mathbf{z}}
\newcommand{\BA}[0]{\mathbf{A}}
\newcommand{\BB}[0]{\mathbf{B}}
\newcommand{\BC}[0]{\mathbf{C}}
\newcommand{\BD}[0]{\mathbf{D}}
\newcommand{\BE}[0]{\mathbf{E}}
\newcommand{\BF}[0]{\mathbf{F}}
\newcommand{\BG}[0]{\mathbf{G}}
\newcommand{\BH}[0]{\mathbf{H}}
\newcommand{\BI}[0]{\mathbf{I}}
\newcommand{\BJ}[0]{\mathbf{J}}
\newcommand{\BK}[0]{\mathbf{K}}
\newcommand{\BL}[0]{\mathbf{L}}
\newcommand{\BM}[0]{\mathbf{M}}
\newcommand{\BN}[0]{\mathbf{N}}
\newcommand{\BO}[0]{\mathbf{O}}
\newcommand{\BP}[0]{\mathbf{P}}
\newcommand{\BQ}[0]{\mathbf{Q}}
\newcommand{\BR}[0]{\mathbf{R}}
\newcommand{\BS}[0]{\mathbf{S}}
\newcommand{\BT}[0]{\mathbf{T}}
\newcommand{\BU}[0]{\mathbf{U}}
\newcommand{\BV}[0]{\mathbf{V}}
\newcommand{\BW}[0]{\mathbf{W}}
\newcommand{\BX}[0]{\mathbf{X}}
\newcommand{\BY}[0]{\mathbf{Y}}
\newcommand{\BZ}[0]{\mathbf{Z}}

\newcommand{\Bzero}[0]{\mathbf{0}}
\newcommand{\Btheta}[0]{\boldsymbol{\theta}}
\newcommand{\Btau}[0]{\boldsymbol{\tau}}
\newcommand{\Bomega}[0]{\boldsymbol{\omega}}

%
% shorthand for unit vectors
%
\newcommand{\acap}[0]{\hat{\Ba}}
\newcommand{\bcap}[0]{\hat{\Bb}}
\newcommand{\ccap}[0]{\hat{\Bc}}
\newcommand{\dcap}[0]{\hat{\Bd}}
\newcommand{\ecap}[0]{\hat{\Be}}
\newcommand{\fcap}[0]{\hat{\Bf}}
\newcommand{\gcap}[0]{\hat{\Bg}}
\newcommand{\hcap}[0]{\hat{\Bh}}
\newcommand{\icap}[0]{\hat{\Bi}}
\newcommand{\jcap}[0]{\hat{\Bj}}
\newcommand{\kcap}[0]{\hat{\Bk}}
\newcommand{\lcap}[0]{\hat{\Bl}}
\newcommand{\mcap}[0]{\hat{\Bm}}
\newcommand{\ncap}[0]{\hat{\Bn}}
\newcommand{\ocap}[0]{\hat{\Bo}}
\newcommand{\pcap}[0]{\hat{\Bp}}
\newcommand{\qcap}[0]{\hat{\Bq}}
\newcommand{\rcap}[0]{\hat{\Br}}
\newcommand{\scap}[0]{\hat{\Bs}}
\newcommand{\tcap}[0]{\hat{\Bt}}
\newcommand{\ucap}[0]{\hat{\Bu}}
\newcommand{\vcap}[0]{\hat{\Bv}}
\newcommand{\wcap}[0]{\hat{\Bw}}
\newcommand{\xcap}[0]{\hat{\Bx}}
\newcommand{\ycap}[0]{\hat{\By}}
\newcommand{\zcap}[0]{\hat{\Bz}}
\newcommand{\thetacap}[0]{\hat{\Btheta}}

%
% to write R^n and C^n in a distinguishable fashion.  Perhaps change this
% to the double lined characters upon figuring out how to do so.
%
\newcommand{\C}[1]{$\mathbb{C}^{#1}$}
\newcommand{\R}[1]{$\mathbb{R}^{#1}$}

%
% various generally useful helpers
%

% derivative of #1 wrt. #2:
\newcommand{\D}[2] {\frac {d#2} {d#1}}

\newcommand{\inv}[1]{\frac{1}{#1}}
\newcommand{\cross}[0]{\times}

\newcommand{\abs}[1]{\lvert{#1}\rvert}
\newcommand{\norm}[1]{\lVert{#1}\rVert}
\newcommand{\innerprod}[2]{\langle{#1}, {#2}\rangle}
\newcommand{\dotprod}[2]{{#1} \cdot {#2}}
\newcommand{\bdotprod}[2]{\left({#1} \cdot {#2}\right)}
\newcommand{\crossprod}[2]{{#1} \cross {#2}}
\newcommand{\tripleprod}[3]{\dotprod{\left(\crossprod{#1}{#2}\right)}{#3}}

\DeclareMathOperator{\Proj}{Proj}
\DeclareMathOperator{\Span}{span}
\DeclareMathOperator{\Sgn}{sgn}
\DeclareMathOperator{\Area}{Area}
\DeclareMathOperator{\Volume}{Volume}

%
% A few miscellaneous things specific to this document
%
\newcommand{\crossop}[1]{\crossprod{#1}{}}

% R2 vector.
\newcommand{\VectorTwo}[2]{
\begin{bmatrix}
 {#1} \\
 {#2}
\end{bmatrix}
}

\newcommand{\VectorN}[1]{
\begin{bmatrix}
{#1}_1 \\
{#1}_2 \\
\vdots \\
{#1}_N \\
\end{bmatrix}
}

\newcommand{\DETuvij}[4]{
\begin{vmatrix}
 {#1}_{#3} & {#1}_{#4} \\
 {#2}_{#3} & {#2}_{#4}
\end{vmatrix}
}

\newcommand{\DETuvwijk}[6]{
\begin{vmatrix}
 {#1}_{#4} & {#1}_{#5} & {#1}_{#6} \\
 {#2}_{#4} & {#2}_{#5} & {#2}_{#6} \\
 {#3}_{#4} & {#3}_{#5} & {#3}_{#6}
\end{vmatrix}
}

\newcommand{\DETuvwxijkl}[8]{
\begin{vmatrix}
 {#1}_{#5} & {#1}_{#6} & {#1}_{#7} & {#1}_{#8} \\
 {#2}_{#5} & {#2}_{#6} & {#2}_{#7} & {#2}_{#8} \\
 {#3}_{#5} & {#3}_{#6} & {#3}_{#7} & {#3}_{#8} \\
 {#4}_{#5} & {#4}_{#6} & {#4}_{#7} & {#4}_{#8} \\
\end{vmatrix}
}

%\newcommand{\DETuvwxyijklm}[10]{
%\begin{vmatrix}
% {#1}_{#6} & {#1}_{#7} & {#1}_{#8} & {#1}_{#9} & {#1}_{#10} \\
% {#2}_{#6} & {#2}_{#7} & {#2}_{#8} & {#2}_{#9} & {#2}_{#10} \\
% {#3}_{#6} & {#3}_{#7} & {#3}_{#8} & {#3}_{#9} & {#3}_{#10} \\
% {#4}_{#6} & {#4}_{#7} & {#4}_{#8} & {#4}_{#9} & {#4}_{#10} \\
% {#5}_{#6} & {#5}_{#7} & {#5}_{#8} & {#5}_{#9} & {#5}_{#10}
%\end{vmatrix}
%}

% R3 vector.
\newcommand{\VectorThree}[3]{
\begin{bmatrix}
 {#1} \\
 {#2} \\
 {#3}
\end{bmatrix}
}



\author{Peeter Joot}
\email{peeter.joot@gmail.com}


\chapter{Notes and problems for Desai Chapter V.}
\label{chap:desaiCh5}
%\useCCL
\blogpage{http://sites.google.com/site/peeterjoot/math2010/desaiCh5.pdf}
\date{Oct 18, 2010}
\revisionInfo{desaiCh5.tex}

\beginArtWithToc
%\beginArtNoToc

\section{Motivation.}

Chapter V notes for \cite{desai2009quantum}.

\section{Notes}
\section{Problems}

\subsection{Problem 1.}

\subsubsection{Statement.}
Obtain $S_x, S_y, S_z$ for spin 1 in the representation in which $S_z$ and $S^2$ are diagonal.

\subsubsection{Solution.}

For spin 1, we have

\begin{align}\label{eqn:desaiCh5:100}
S^2 = 1 (1+1) \hbar^2 \BOne
\end{align}

and are interested in the states $\ket{1,-1}, \ket{1, 0}, and \ket{1,1}$.  If, like angular momentum, we assume that we have for $m_s = -1,0,1$

\begin{align}\label{eqn:desaiCh5:101}
S_z \ket{1,m_s} = m_s \hbar \ket{1, m_s}
\end{align}

and introduce a column matrix representations for the kets as follows

\begin{align}\label{eqn:desaiCh5:102}
\ket{1,1} &=
\begin{bmatrix}
1 \\
0 \\
0
\end{bmatrix} \\
\ket{1,0} &=
\begin{bmatrix}
0 \\
1 \\
0
\end{bmatrix} \\
\ket{1,-1} &=
\begin{bmatrix}
0 \\
0 \\
-1
\end{bmatrix},
\end{align}

then we have, by inspection
\begin{align}\label{eqn:desaiCh5:103}
S_z &= \hbar
\begin{bmatrix}
1 & 0 & 0 \\
0 & 0 & 0 \\
0 & 0 & -1
\end{bmatrix}.
\end{align}

Note that, like the Pauli matrices, and unlike angular momentum, the spin states $\ket{-1, m_s}, \ket{0, m_s}$ have not been considered.  Do those have any physical interpretation?

That question aside, we can procede as in the text, utilizing the ladder operator commutators

\begin{align}\label{eqn:desaiCh5:104}
S_{\pm} &= S_x \pm i S_y,
\end{align}

to determine the values of $S_x$ and $S_y$ indirectly.  We find

\begin{align}\label{eqn:desaiCh5:105}
\antisymmetric{S_{+}}{S_{-}} &= 2 \hbar S_z \\
\antisymmetric{S_{+}}{S_{z}} &= -\hbar S_{+} \\
\antisymmetric{S_{-}}{S_{z}} &= \hbar S_{-}.
\end{align}

Let
\begin{align}\label{eqn:desaiCh5:106}
S_{+} &=
\begin{bmatrix}
a & b & c \\
d & e & f \\
g & h & i
\end{bmatrix}.
\end{align}

Looking for equality between $\antisymmetric{S_{z}}{S_{+}}/\hbar = S_{+}$, we find
\begin{align}\label{eqn:desaiCh5:107}
\begin{bmatrix}
0 & b & 2 c \\
-d & 0 & f \\
-2g & -h & 0
\end{bmatrix}
&=
\begin{bmatrix}
a & b & c \\
d & e & f \\
g & h & i
\end{bmatrix},
\end{align}

so we must have
\begin{align}\label{eqn:desaiCh5:108}
S_{+} &=
\begin{bmatrix}
0 & b & 0 \\
0 & 0 & f \\
0 & 0 & 0
\end{bmatrix}.
\end{align}

Furthermore, from $\antisymmetric{S_{+}}{S_{-}} = 2 \hbar S_z$, we find
\begin{align}\label{eqn:desaiCh5:109}
\begin{bmatrix}
\Abs{b}^2 & 0 & 0 \\
0 & \Abs{f}^2 - \Abs{b}^2 & 0 \\
0 & 0 & -\Abs{f}^2
\end{bmatrix}
&=
2 \hbar^2
\begin{bmatrix}
1 & 0 & 0 \\
0 & 0 & 0 \\
0 & 0 & -1
\end{bmatrix}.
\end{align}

We must have $\Abs{b}^2 = \Abs{f}^2 = 2 \hbar^2$.  We could probably pick any
$b = \sqrt{2} \hbar e^{i\phi}$, and $f = \sqrt{2} \hbar e^{i\theta}$, but assuming we have no reason for a non-zero phase we try

\begin{align}\label{eqn:desaiCh5:110}
S_{+}
&=
\sqrt{2} \hbar
\begin{bmatrix}
0 & 1 & 0 \\
0 & 0 & 1 \\
0 & 0 & 0
\end{bmatrix}.
\end{align}

Putting all the pieces back together, with $S_x = (S_{+} + S_{-})/2$, and $S_y = (S_{+} - S_{-})/2i$, we finally have
\begin{align}\label{eqn:desaiCh5:111}
S_x &=
\frac{\hbar}{\sqrt{2}}
\begin{bmatrix}
0 & 1 & 0 \\
1 & 0 & 1 \\
0 & 1 & 0
\end{bmatrix} \\
S_y &=
\frac{\hbar}{\sqrt{2} i}
\begin{bmatrix}
0 & 1 & 0 \\
-1 & 0 & 1 \\
0 & -1 & 0
\end{bmatrix} \\
S_z &=
\hbar
\begin{bmatrix}
1 & 0 & 0 \\
0 & 0 & 0 \\
0 & 0 & -1
\end{bmatrix}.
\end{align}

A quick calculation verifies that we have $S_x^2 + S_y^2 + S_z^2 = 2 \hbar \BOne$, as expected.

\subsection{Problem 2.}
\subsubsection{Statement.}

Obtain eigensolution for operator $A = a \sigma_y + b \sigma_z$.  Call the eigenstates $\ket{1}$ and $\ket{2}$, and determine the probabilities that they will correspond to $\sigma_x = +1$.

\subsubsection{Solution.}

The first part is straight forward, and we have
\begin{align*}
A &= a \PauliY + b \PauliZ \\
&=
\begin{bmatrix}
b & -i a \\
ia & -b
\end{bmatrix}.
\end{align*}

Taking $\Abs{A - \lambda I} = 0$ we get

\begin{align}\label{eqn:desaiCh5:202}
\lambda &= \pm \sqrt{a^2 + b^2},
\end{align}

with eigenvectors proportional to
\begin{align}\label{eqn:desaiCh5:203}
\ket{\pm} &=
\begin{bmatrix}
i a \\
b \mp \sqrt{a^2 + b^2}
\end{bmatrix}
\end{align}

The normalization constant is $1/\sqrt{2 (a^2 + b^2) \mp 2 b \sqrt{a^2 + b^2}}$.  Now we can call these $\ket{1}$, and $\ket{2}$ but what does the last part of the question mean?  What's meant by $\sigma_x = +1$?

Asking the prof about this, he says:

``I think it means that the result of a measurement of the x component of spin is $+1$. This corresponds to the eigenvalue of $\sigma_x$ being $+1$. The spin operator $S_x$ has eigenvalue $+\hbar/2$''.

Aside: Question to consider later.  Is is significant that $\bra{1} \sigma_x \ket{1} = \bra{2} \sigma_x \ket{2} = 0$?

So, how do we translate this into a mathematical statement?

First let's recall a couple of details.  Recall that the x spin operator has the matrix representation

\begin{align}\label{eqn:desaiCh5:204}
\sigma_x = \PauliX.
\end{align}

This has eigenvalues $\pm 1$, with eigenstates $(1,\pm 1)/\sqrt{2}$.  At the point when the x component spin is observed to be $+1$, the state of the system was then

\begin{align}\label{eqn:desaiCh5:205}
\ket{x+} =
\inv{\sqrt{2}}
\begin{bmatrix}
1 \\
1
\end{bmatrix}
\end{align}

Let's look at the ways that this state can be formed as linear combinations of our states $\ket{1}$, and $\ket{2}$.  That is

\begin{align}\label{eqn:desaiCh5:206}
\inv{\sqrt{2}}
\begin{bmatrix}
1 \\
1
\end{bmatrix}
&=
\alpha \ket{1}
+ \beta \ket{2},
\end{align}

or

\begin{align}\label{eqn:desaiCh5:207}
\begin{bmatrix}
1 \\
1
\end{bmatrix}
&=
\frac{\alpha}{\sqrt{(a^2 + b^2) - b \sqrt{a^2 + b^2}}}
\begin{bmatrix}
i a \\
b - \sqrt{a^2 + b^2}
\end{bmatrix}
+\frac{\beta}{\sqrt{(a^2 + b^2) + b \sqrt{a^2 + b^2}}}
\begin{bmatrix}
i a \\
b + \sqrt{a^2 + b^2}
\end{bmatrix}
\end{align}

Letting $c = \sqrt{a^2 + b^2}$, this is

\begin{align}\label{eqn:desaiCh5:208}
\begin{bmatrix}
1 \\
1
\end{bmatrix}
&=
\frac{\alpha}{\sqrt{c^2 - b c}}
\begin{bmatrix}
i a \\
b - c
\end{bmatrix}
+\frac{\beta}{\sqrt{c^2 + b c}}
\begin{bmatrix}
i a \\
b + c
\end{bmatrix}.
\end{align}

We can solve the $\alpha$ and $\beta$ with Cramer's rule, yielding
\begin{align*}
\begin{vmatrix}
1 & i a \\
1 & b - c
\end{vmatrix}
&=
\frac{\beta}{\sqrt{c^2 + b c}}
\begin{vmatrix}
i a  & i a \\
b + c & b - c
\end{vmatrix} \\
\begin{vmatrix}
1 & i a \\
1 & b + c
\end{vmatrix}
&=
\frac{\alpha}{\sqrt{c^2 - b c}}
\begin{vmatrix}
i a  & i a \\
b - c & b + c
\end{vmatrix},
\end{align*}

or
\begin{align}\label{eqn:desaiCh5:209}
\alpha &= \frac{(b + c - ia)\sqrt{c^2 - b c}}{2 i a c} \\ %= \frac{(a -i(b + c))\sqrt{1 - b/c}}{2 a} \\
\beta &= \frac{(b - c - ia)\sqrt{c^2 + b c}}{-2 i a c} %= \frac{(a + i(b - c))\sqrt{1 + b/c}}{2 a}.
\end{align}

It is $\Abs{\alpha}^2$ and $\Abs{\beta}^2$ that are probabilities, and after a bit of algebra we find that those are

\begin{align}\label{eqn:desaiCh5:210}
\Abs{\alpha}^2 = \Abs{\beta}^2 = \inv{2},
\end{align}

so if the x spin of the system is measured as $+1$, we have a $50\%$ chance that the measured eigenvalue for the operator $A$ would be $\sqrt{a^2 + b^2}$ (ie: with state $\ket{1}$.

Is that what the question was asking?  I think that I've actually got it backwards.  I think that the question was asking for the probability of finding state $\ket{x+}$ (measuring a spin 1 value for $\sigma_x$) given the state $\ket{1}$ or $\ket{2}$.

So, suppose that we have

\begin{align}\label{eqn:desaiCh5:211}
\mu_{+} \ket{x+} + \nu_{+} \ket{x-} &= \ket{1} \\
\mu_{-} \ket{x+} + \nu_{-} \ket{x-} &= \ket{2},
\end{align}

or (considering both cases simulaneously), 
\begin{align*}
\mu_{\pm}
\begin{bmatrix}
1 \\
1
\end{bmatrix}
+ \nu_{\pm}
\begin{bmatrix}
1 \\
-1
\end{bmatrix}
&= 
\inv{\sqrt{ c^2 \mp b c }} 
\begin{bmatrix}
i a \\
b \mp c
\end{bmatrix} \\
\implies \\
\mu_{\pm}
\begin{vmatrix}
1 & 1 \\
1 & -1
\end{vmatrix}
&= 
\inv{\sqrt{ c^2 \mp b c }} 
\begin{vmatrix}
i a & 1 \\
b \mp c & -1
\end{vmatrix},
\end{align*}

or
\begin{align}\label{eqn:desaiCh5:212}
\mu_{\pm} &= 
\frac{ia + b \mp c}{2 \sqrt{c^2 \mp bc}} .
\end{align}

Unsuprisingly, this mirrors the previous scenerio and we find that we have a probability $\Abs{\mu}^2 = 1/2$ of measuring a spin 1 value for $\sigma_x$ when the state of the operator $A$ has been measured as $\pm \sqrt{a^2 + b^2}$ (ie: in the states $\ket{1}$, or $\ket{2}$ respectively).

No measurement of the operator $A = a \sigma_y + b\sigma_z$ gives a biased prediction of the state of the state $\sigma_x$.  Loosely, this seems to justify calling these operators orthogonal.  This is consistent with the geometrical antisymetric nature of the spin components where we have $\sigma_y \sigma_x = -\sigma_x \sigma_y$, just like two orthogonal vectors under the Clifford product.

\subsection{Problem 3.}
\subsubsection{Statement.}

Obtain the expectation values of $S_x, S_y, S_z$ for the case of a spin $1/2$ particle with the spin pointed in the direction of a vector with azimuthal angle $\beta$ and polar angle $\alpha$.

\subsubsection{Solution.}

TODO.

\subsection{Problem 4.}
\subsubsection{Statement.}
\subsubsection{Solution.}

TODO.

\subsection{Problem 5.}
\subsubsection{Statement.}
\subsubsection{Solution.}

TODO.

\subsection{Problem 6.}
\subsubsection{Statement.}

If a Hamiltonian is given by $\Bsigma \cdot \Bn$ where $\Bn = (\sin\alpha\cos\beta, \sin\alpha\sin\beta, \cos\alpha)$, determine the time evolution operator as a 2 x 2 matrix.  If a state at $t = 0$ is given by 

\begin{align}\label{eqn:desaiCh5:600}
\ket{\phi(0)} = 
\begin{bmatrix}
a \\
b
\end{bmatrix},
\end{align}

then obtain $\ket{\phi(t)}$.

\subsubsection{Solution.}

Before diving into the meat of the problem, observe that a tidy factorization of the Hamiltonian is possible as a composition of rotations.  That is

\begin{align*}
H 
&= \Bsigma \cdot \Bn \\
&= \sin\alpha \sigma_1 ( \cos\beta + \sigma_1 \sigma_2 \sin\beta ) + \cos\alpha \sigma_3 \\
&= \sigma_3 \left(
\cos\alpha 
+ \sin\alpha \sigma_3 \sigma_1 e^{ i \sigma_3 \beta }
\right) \\
&= 
\sigma_3 \exp\left( \alpha i \sigma_2 
\exp\left( \beta i \sigma_3 
\right)
\right)
\end{align*}

So we have for the time evolution operator

\begin{align}\label{eqn:desaiCh5:610}
U(\Delta t) 
&=
\exp( -i \Delta t H /\hbar )
= 
\exp \left(
- \frac{\Delta t}{\hbar} i \sigma_3 \exp\Bigl( \alpha i \sigma_2 
\exp\left( \beta i \sigma_3 
\right)
\Bigr)
\right).
\end{align}

Does this really help?  I guess not, but it is nice and tidy.

Returning to the specifics of the problem, we note that squaring the Hamiltonian produces the identity matrix

\begin{align}\label{eqn:desaiCh5:615}
(\Bsigma \cdot \Bn)^2 &= I \Bn^2 = I.
\end{align}

This allows us to exponentiate $H$ by inspection utilizing

\begin{align}\label{eqn:desaiCh5:620}
e^{i \mu (\Bsigma \cdot \Bn) } = I \cos\mu + i (\Bsigma \cdot \Bn) \sin\mu
\end{align}

Writing $\sin\mu = S_\mu$, and $\cos\mu = C_\mu$, we have
\begin{align}\label{eqn:desaiCh5:625}
\Bsigma \cdot \Bn &=
\begin{bmatrix}
C_\alpha & S_\alpha e^{-i\beta} \\
S_\alpha e^{i\beta} & -C_\alpha
\end{bmatrix},
\end{align}

and thus
\begin{align}\label{eqn:desaiCh5:630}
U(\Delta t) = \exp( -i \Delta t H /\hbar )
=
\begin{bmatrix}
C_{\Delta t/\hbar} -i S_{\Delta t/\hbar} C_\alpha & -i S_{\Delta t/\hbar} S_\alpha e^{-i\beta} \\
-i S_{\Delta t/\hbar} S_\alpha e^{i\beta} & C_{\Delta t/\hbar} + i S_{\Delta t/\hbar} C_\alpha
\end{bmatrix}.
\end{align}

Note that as a sanity check we can calculate that $ U(\Delta t) U(\Delta t)^\dagger = 1$ as expected.

Now for $\Delta t = t$, we have 
\begin{align}\label{eqn:desaiCh5:640}
U(t,0) 
\begin{bmatrix}
a \\
b
\end{bmatrix}
&=
\begin{bmatrix}
a C_{t/\hbar} -a i S_{t/\hbar} C_\alpha  - b i S_{t/\hbar} S_\alpha e^{-i\beta} \\
-a i S_{t/\hbar} S_\alpha e^{i\beta} + b C_{t/\hbar} + b i S_{t/\hbar} C_\alpha
\end{bmatrix}.
\end{align}

It doesn't seem terribly illuminating to multiply this all out, but we can factor the results slightly to tidy it up.  That gives us

\begin{align}\label{eqn:desaiCh5:650}
U(t,0) 
\begin{bmatrix}
a \\
b
\end{bmatrix}
&=
\cos(t/\hbar)
\begin{bmatrix}
a \\
b
\end{bmatrix}
+ 
\sin(t/\hbar) \cos\alpha
\begin{bmatrix}
-a \\
b
\end{bmatrix}
+ i
\sin(t/\hbar) \sin\alpha
\begin{bmatrix}
b e^{-i\beta} \\
-a e^{i \beta}
\end{bmatrix}
\end{align}

\subsection{Problem 7.}
\subsubsection{Statement.}
\subsubsection{Solution.}

TODO.

\subsection{Problem 8.}
\subsubsection{Statement.}
\subsubsection{Solution.}

TODO.

\subsection{Problem 9.}
\subsubsection{Statement.}
\subsubsection{Solution.}

TODO.

\EndArticle
