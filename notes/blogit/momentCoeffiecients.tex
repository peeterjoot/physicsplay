%
% Copyright � 2016 Peeter Joot.  All Rights Reserved.
% Licenced as described in the file LICENSE under the root directory of this GIT repository.
%
%{
\newcommand{\authorname}{Peeter Joot}
\newcommand{\email}{peeterjoot@protonmail.com}
\newcommand{\basename}{FIXMEbasenameUndefined}
\newcommand{\dirname}{notes/FIXMEdirnameUndefined/}

\renewcommand{\basename}{momentCoeffiecients}
%\renewcommand{\dirname}{notes/phy1520/}
\renewcommand{\dirname}{notes/ece1228-electromagnetic-theory/}
%\newcommand{\dateintitle}{}
%\newcommand{\keywords}{}

\newcommand{\authorname}{Peeter Joot}
\newcommand{\onlineurl}{http://sites.google.com/site/peeterjoot2/math2013/\basename.pdf}
\newcommand{\sourcepath}{\dirname\basename.tex}
\newcommand{\generatetitle}[1]{\chapter{#1}}

\newcommand{\vcsinfo}{%
\section*{}
\noindent{\color{DarkOliveGreen}{\rule{\linewidth}{0.1mm}}}
\paragraph{Document version}
%\paragraph{\color{Maroon}{Document version}}
{
\small
\begin{itemize}
\item Available online at:\\ 
\href{\onlineurl}{\onlineurl}
\item Git Repository: \input{./.revinfo/gitRepo.tex}
\item Source: \sourcepath
\item last commit: \input{./.revinfo/gitCommitString.tex}
\item commit date: \input{./.revinfo/gitCommitDate.tex}
\end{itemize}
}
}

%\PassOptionsToPackage{dvipsnames,svgnames}{xcolor}
\PassOptionsToPackage{square,numbers}{natbib}
\documentclass{scrreprt}

\usepackage[left=2cm,right=2cm]{geometry}
\usepackage[svgnames]{xcolor}
\usepackage{peeters_layout}

\usepackage{natbib}

\usepackage[
colorlinks=true,
bookmarks=false,
pdfauthor={\authorname, \email},
backref 
]{hyperref}

% http://tex.stackexchange.com/questions/75773/how-to-reference-problems-by-the-text-label-in-an-exercise-envioronment
\usepackage[english]{cleveref}
\crefname{Exercise}{exercise}{exercises}
\Crefname{Exercise}{Exercise}{Exercises}

\RequirePackage{titlesec}
\RequirePackage{ifthen}

% http://stackoverflow.com/questions/4932910/date-in-the-tabular-environment
\makeatletter
\let\insertdate\@date
\makeatother

\titleformat{\chapter}[display]
{\bfseries\Large}
{\color{DarkSlateGrey}\filleft \authorname
\ifthenelse{\isundefined{\studentnumber}}{}{\\ \studentnumber}
\ifthenelse{\isundefined{\email}}{}{\\ \email}
\ifthenelse{\isundefined{\dateintitle}}{}{\\ \insertdate}
%\ifthenelse{\isundefined{\coursename}}{}{\\ \coursename} % put in title instead.
}
{4ex}
{\color{DarkOliveGreen}{\titlerule}\color{Maroon}
\vspace{2ex}%
\filright}
[\vspace{2ex}%
\color{DarkOliveGreen}\titlerule
]

\newcommand{\beginArtWithToc}[0]{\begin{document}\tableofcontents}
\newcommand{\beginArtNoToc}[0]{\begin{document}}
\newcommand{\EndNoBibArticle}[0]{\end{document}}
\newcommand{\EndArticle}[0]{\bibliography{Bibliography}\bibliographystyle{plainnat}\end{document}}

% 
%\newcommand{\citep}[1]{\cite{#1}}

\colorSectionsForArticle



\usepackage{peeters_layout_exercise}
\usepackage{peeters_braket}
\usepackage{peeters_figures}
\usepackage{siunitx}
%\usepackage{txfonts} % \ointclockwise

\beginArtNoToc

\generatetitle{Computing some moment coefficients}
%\chapter{Computing some moment coefficients}
%\label{chap:momentCoeffiecients}
% \citep{sakurai2014modern} pr X.Y
% \citep{pozar2009microwave}
% \citep{qftLectureNotes}
% \citep{doran2003gap}
% \citep{jackson1975cew}
% \citep{griffiths1999introduction}

In class today we calculated the \( q_{1,1} \) coefficient of the electrostatic moment

\begin{dmath}\label{eqn:momentCoeffiecients:20}
q_{l,m} = 
\int (r')^l \rho(\Bx') 
Y^\conj_{l,m}(\theta', \phi')
d^3 x',
\end{dmath}

The class notes also give the results for \( q_{0,0}, q_{1,0}, q_{2,2}, q_{2,1}, q_{2,0} \).  Let's verify those

\paragraph{\(q_{0,0}\)}

\begin{dmath}\label{eqn:momentCoeffiecients:40}
q_{0,0}
=
\int (r')^0 \rho(\Bx') 
Y^\conj_{0,0}(\theta', \phi')
d^3 x'
=
\inv{4\pi}
\int \rho(\Bx') d^3 x'
=
\frac{q}{4\pi}
\end{dmath}

\paragraph{\(q_{1,0}\)}

\begin{dmath}\label{eqn:momentCoeffiecients:60}
q_{1,0} 
= 
\int r' \rho(\Bx') 
Y^\conj_{1,0}(\theta', \phi')
d^3 x'
=
\sqrt{\frac{3}{4\pi}}
\int r' \rho(\Bx') 
\cos\theta'
d^3 x'
=
\sqrt{\frac{3}{4\pi}}
\int r' \rho(\Bx') \cos\theta' d^3 x'
=
\sqrt{\frac{3}{4\pi}}
\int z' \rho(\Bx') d^3 x'
=
\sqrt{\frac{3}{4\pi}} p_z
\end{dmath}

\paragraph{\(q_{2,2}\)}

\begin{dmath}\label{eqn:momentCoeffiecients:80}
q_{2,2} 
= 
\int (r')^2 \rho(\Bx') 
Y^\conj_{2,2}(\theta', \phi')
d^3 x'
= 
\sqrt{\frac{15}{32 \pi}}
\int (r')^2 \rho(\Bx') 
\sin^2 \theta e^{-2 i\phi}
d^3 x'
=
\sqrt{\frac{15}{32 \pi}}
\int (r')^2 \rho(\Bx') 
\sin^2 \theta \lr{ \cos \phi - i \sin\phi }^2
d^3 x'
=
\sqrt{\frac{15}{32 \pi}}
\int (r')^2 \rho(\Bx') 
\sin^2 \theta \lr{ \cos^2\phi - \sin^2\phi - 2 i \cos\phi \sin\phi }
d^3 x'
=
\sqrt{\frac{15}{32 \pi}}
\int \rho(\Bx') \lr{ 
(x')^2
- (y')^2
- 2 i x' y' } d^3 x'
=
\sqrt{\frac{15}{32 \pi}}
\int \rho(\Bx') \lr{ 
x' - i y'
}^2 d^3 x'
%=
%\sqrt{\frac{15}{32 \pi}}
%\lr{ p_x - i p_y }^2
\end{dmath}

\paragraph{\(q_{2,1}\)}

\begin{dmath}\label{eqn:momentCoeffiecients:100}
q_{2,1} 
= 
\int (r')^2 \rho(\Bx') 
Y^\conj_{2,1}(\theta', \phi')
d^3 x'
= 
-\sqrt{\frac{15}{8 \pi}}
\int (r')^2 \rho(\Bx') 
\sin\theta' \cos\theta' e^{-i \phi}
d^3 x'
= 
-\sqrt{\frac{15}{8 \pi}}
\int (r')^2 \rho(\Bx') 
\sin\theta' \cos\theta' \lr{ \cos\phi - i \sin\phi }
d^3 x'
= 
-\sqrt{\frac{15}{8 \pi}}
\int \rho(\Bx') 
\lr{ x' z' - i y' z' }
d^3 x'
%=
%-\sqrt{\frac{15}{8 \pi}} p_z \lr{ p_x - i p_y }.
\end{dmath}

\paragraph{\(q_{2,0}\)}

\begin{dmath}\label{eqn:momentCoeffiecients:260}
q_{2,0} 
= 
\int (r')^2 \rho(\Bx') 
Y^\conj_{2,0}(\theta', \phi')
d^3 x'
= 
\int (r')^2 \rho(\Bx') \sqrt{\frac{5}{4\pi}} \lr{ \frac{3}{2} \cos^2\theta - \inv{2} }
d^3 x'
= 
\inv{2} \sqrt{\frac{5}{4\pi}} 
\int \rho(\Bx') 
\lr{ 3 (z')^2 - (r')^2 }
d^3 x'.
\end{dmath}

\paragraph{\(Q_{ij}\)}

It was claimed in class that the quadropole term of the potential was

\begin{dmath}\label{eqn:momentCoeffiecients:120}
\inv{2} \sum_{ij} Q_{ij} \frac{x_i x_j}{r^5},
\end{dmath}

where

\begin{dmath}\label{eqn:momentCoeffiecients:140}
Q_{i,j} = \int \lr{ 3 x_i' x_j' - \delta_{ij} (r')^2 } \rho(\Bx') d^3 x'.
\end{dmath}

Let's verify this.  First note that

\begin{dmath}\label{eqn:momentCoeffiecients:160}
Y_{l,m} = \sqrt{\frac{2 l + 1}{4 \pi} \frac{(l-m)!}{(l+m)!}} P_l^m(\cos\theta) e^{i m \phi},
\end{dmath}

and
\begin{dmath}\label{eqn:momentCoeffiecients:180}
P_l^{-m}(x) = 
(-1)^m \frac{(l-m)!}{(l+m)!} P_l^m(x),
\end{dmath}

so
\begin{dmath}\label{eqn:momentCoeffiecients:200}
Y_{l,-m} 
= \sqrt{\frac{2 l + 1}{4 \pi} \frac{(l+m)!}{(l-m)!} }
P_l^{-m}(\cos\theta) 
e^{-i m \phi}
= 
(-1)^m 
\sqrt{\frac{2 l + 1}{4 \pi} \frac{(l-m)!}{(l+m)!} }
P_l^m(x)
e^{-i m \phi}
=
(-1)^m Y_{l,m}^\conj.
\end{dmath}

That means

\begin{dmath}\label{eqn:momentCoeffiecients:220}
q_{l,-m} 
= 
\int (r')^l \rho(\Bx') 
Y^\conj_{l,-m}(\theta', \phi')
d^3 x'
= 
(-1)^m
\int (r')^l \rho(\Bx') 
Y_{l,m}(\theta', \phi')
d^3 x'
=
(-1)^m q_{lm}^\conj.
\end{dmath}

In particular, for \( m \ne 0 \)

\begin{dmath}\label{eqn:momentCoeffiecients:n}
(r')^l Y_{l, m}^\conj (\theta', \phi') r^l Y_{l, m}(\theta, \phi) 
+ (r')^l Y_{l, -m}^\conj (\theta', \phi') r^l Y_{l, -m}(\theta, \phi) 
=
(r')^l Y_{l, m}^\conj (\theta', \phi') r^l Y_{l, m}(\theta, \phi) 
+ (r')^l Y_{l, m} (\theta', \phi') r^l Y_{l, m}^\conj(\theta, \phi) ,
\end{dmath}

or
\begin{dmath}\label{eqn:momentCoeffiecients:n}
(r')^l Y_{l, m}^\conj (\theta', \phi') r^l Y_{l, m}(\theta, \phi) 
+ (r')^l Y_{l, -m}^\conj (\theta', \phi') r^l Y_{l, -m}(\theta, \phi) 
=
2 \Real \lr{ (r')^l Y_{l, m}^\conj (\theta', \phi') r^l Y_{l, m}(\theta, \phi) }.
\end{dmath}

To verify the quadropole expansion formula in a compact way it is helpful to compute some intermediate results.

\begin{dmath}\label{eqn:momentCoeffiecients:n}
r Y_{1, 1} 
= -r \sqrt{\frac{3}{8 \pi}} \sin\theta e^{i\phi} 
= -\sqrt{\frac{3}{8 \pi}} (x + i y)
\end{dmath}

\begin{dmath}\label{eqn:momentCoeffiecients:n}
r Y_{1, 0} 
= r \sqrt{\frac{3}{4 \pi}} \cos\theta 
= \sqrt{\frac{3}{4 \pi}} z
\end{dmath}

\begin{dmath}\label{eqn:momentCoeffiecients:n}
r^2 Y_{2, 2} 
= r^2 ... 
\end{dmath}

So, for the dipole moment term we have

%%%We can use this last fact to group terms in the quadropole sum
%%%
%%%\begin{dmath}\label{eqn:momentCoeffiecients:240}
%%%Q =
%%%\inv{4 \pi \epsilon_0} 
%%%\sum_{m = -2}^2 \frac{4\pi} {5} 
%%%q_{2,m}
%%%\frac{
%%%Y_{2,m}(\theta, \phi)
%%%}
%%%{
%%%r^{3}
%%%}
%%%=
%%%\inv{5 \epsilon_0 r^3} 
%%%\lr{
%%%   q_{20} Y_{2,0}
%%%+
%%%   q_{21} Y_{2,1} + q_{2,-1} Y_{2,-1}
%%%+
%%%   q_{22} Y_{2,2} + q_{2,-2} Y_{2,-2}
%%%}
%%%=
%%%\inv{5 \epsilon_0 r^5} 
%%%\lr{
%%%   \lr{\sqrt{\frac{5}{16 \pi}}}^2 \int ( 3 (z')^2 - (r')^2 ) \rho(\Bx') d^3 \Bx' ( 3 z^2 - r^2 ) 
%%%   +
%%%   2 \Real\lr{ 
%%%   \lr{-\sqrt{\frac{15}{8\pi}}}^2 \int z' (x' - i y') \rho(\Bx') d^3 x' ( z (x + i y) )
%%%   }
%%%   +
%%%   2 \Real\lr{ 
%%%   \lr{\sqrt{\frac{15}{32\pi}}}^2 \int (x' - i y')^2 \rho(\Bx') d^3 x' (x + i y)^2
%%%   }
%%%}
%%%=
%%%\inv{4 \pi \epsilon_0 r^5} 
%%%\lr{
%%%   \frac{1}{4} \int ( 3 (z')^2 - (r')^2 ) \rho(\Bx') d^3 \Bx' ( 3 z^2 - r^2 ) 
%%%   +
%%%   \Real\lr{ 
%%%   3 \int z' (x' - i y') \rho(\Bx') d^3 x' ( z (x + i y) )
%%%   }
%%%   +
%%%   \Real\lr{ 
%%%   \frac{3}{4} \int (x' - i y')^2 \rho(\Bx') d^3 x' (x + i y)^2
%%%   }
%%%}.
%%%\end{dmath}
%%%
%%%Observe that
%%%\begin{dmath}\label{eqn:momentCoeffiecients:280}
%%%\Real \lr{ (x' - i y')^2 (x + i y)^2 }
%%%=
%%%\Real \lr{  ((x')^2 - (y')^2 - 2 i x' y' ) (x^2 + y^2 + 2 i y) }
%%%=
%%%((x')^2 - (y')^2) (x^2 + y^2) + 4 x' y' x y,
%%%\end{dmath}
%%%
%%%so
%%%
%%%\begin{dmath}\label{eqn:momentCoeffiecients:300}
%%%Q =
%%%\inv{4 \pi \epsilon_0 r^5} 
%%%\lr{
%%%   \frac{1}{4} \int ( 3 (z')^2 - (r')^2 ) \rho(\Bx') d^3 \Bx' ( 3 z^2 - r^2 ) 
%%%   +
%%%   3 \int z' z (x'x +  y' y) \rho(\Bx') d^3 
%%%   +
%%%   \frac{3}{4} \int 
%%%      \lr{ ((x')^2 - (y')^2) (x^2 + y^2) + 4 x' y' x y }
%%%      \rho(\Bx') d^3 x'
%%%}
%%%\end{dmath}

%}
\EndArticle
%\EndNoBibArticle
