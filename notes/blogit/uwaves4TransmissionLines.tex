%
% Copyright � 2016 Peeter Joot.  All Rights Reserved.
% Licenced as described in the file LICENSE under the root directory of this GIT repository.
%
%{
\newcommand{\authorname}{Peeter Joot}
\newcommand{\email}{peeterjoot@protonmail.com}
\newcommand{\basename}{FIXMEbasenameUndefined}
\newcommand{\dirname}{notes/FIXMEdirnameUndefined/}

\renewcommand{\basename}{uwaves4TransmissionLines}
\renewcommand{\dirname}{notes/ece1236/}
\newcommand{\keywords}{ECE1236H}
\newcommand{\authorname}{Peeter Joot}
\newcommand{\onlineurl}{http://sites.google.com/site/peeterjoot2/math2013/\basename.pdf}
\newcommand{\sourcepath}{\dirname\basename.tex}
\newcommand{\generatetitle}[1]{\chapter{#1}}

\newcommand{\vcsinfo}{%
\section*{}
\noindent{\color{DarkOliveGreen}{\rule{\linewidth}{0.1mm}}}
\paragraph{Document version}
%\paragraph{\color{Maroon}{Document version}}
{
\small
\begin{itemize}
\item Available online at:\\ 
\href{\onlineurl}{\onlineurl}
\item Git Repository: \input{./.revinfo/gitRepo.tex}
\item Source: \sourcepath
\item last commit: \input{./.revinfo/gitCommitString.tex}
\item commit date: \input{./.revinfo/gitCommitDate.tex}
\end{itemize}
}
}

%\PassOptionsToPackage{dvipsnames,svgnames}{xcolor}
\PassOptionsToPackage{square,numbers}{natbib}
\documentclass{scrreprt}

\usepackage[left=2cm,right=2cm]{geometry}
\usepackage[svgnames]{xcolor}
\usepackage{peeters_layout}

\usepackage{natbib}

\usepackage[
colorlinks=true,
bookmarks=false,
pdfauthor={\authorname, \email},
backref 
]{hyperref}

% http://tex.stackexchange.com/questions/75773/how-to-reference-problems-by-the-text-label-in-an-exercise-envioronment
\usepackage[english]{cleveref}
\crefname{Exercise}{exercise}{exercises}
\Crefname{Exercise}{Exercise}{Exercises}

\RequirePackage{titlesec}
\RequirePackage{ifthen}

% http://stackoverflow.com/questions/4932910/date-in-the-tabular-environment
\makeatletter
\let\insertdate\@date
\makeatother

\titleformat{\chapter}[display]
{\bfseries\Large}
{\color{DarkSlateGrey}\filleft \authorname
\ifthenelse{\isundefined{\studentnumber}}{}{\\ \studentnumber}
\ifthenelse{\isundefined{\email}}{}{\\ \email}
\ifthenelse{\isundefined{\dateintitle}}{}{\\ \insertdate}
%\ifthenelse{\isundefined{\coursename}}{}{\\ \coursename} % put in title instead.
}
{4ex}
{\color{DarkOliveGreen}{\titlerule}\color{Maroon}
\vspace{2ex}%
\filright}
[\vspace{2ex}%
\color{DarkOliveGreen}\titlerule
]

\newcommand{\beginArtWithToc}[0]{\begin{document}\tableofcontents}
\newcommand{\beginArtNoToc}[0]{\begin{document}}
\newcommand{\EndNoBibArticle}[0]{\end{document}}
\newcommand{\EndArticle}[0]{\bibliography{Bibliography}\bibliographystyle{plainnat}\end{document}}

% 
%\newcommand{\citep}[1]{\cite{#1}}

\colorSectionsForArticle



%\usepackage{ece1236}
\usepackage{peeters_braket}
\usepackage{peeters_layout_exercise}
\usepackage{peeters_figures}
\usepackage{mathtools}

\beginArtNoToc
\generatetitle{ECE1236H Microwave and Millimeter-Wave Techniques: Transmission lines.  Taught by Prof.\ G.V. Eleftheriades}
%\chapter{Transmission lines}
\label{chap:uwaves4TransmissionLines}

\paragraph{Disclaimer}

Peeter's lecture notes from class.  These may be incoherent and rough.

These are notes for the UofT course ECE1236H, Microwave and Millimeter-Wave Techniques, taught by Prof. G.V. Eleftheriades, covering \textchapref{{2}} \citep{pozar2009microwave} content.

\section{Requirements}

A transmission line requires two conductors as sketched in \cref{fig:deck4Txline:deck4TxlineFig1}, which shows a 2-wire line such a telephone line, a coaxial cable as found in cable TV distribution, and a microstrip line as found in cell phone RF interconnects.

\imageFigure{../../figures/ece1236/deck4TxlineFig1}{Transmission line examples.}{fig:deck4Txline:deck4TxlineFig1}{0.3}

A two-wire line becomes a transmission line when the wavelength of operation becomes comparable to the size of the line (or higher spectral component for pulses).  In general a transmission line much support (TEM) transverse electromagnetic modes.

\section{Time harmonic solutions on transmission lines}

In \cref{fig:deck4Txline:deck4TxlineFig2}, an electronic representation of a transmission line circuit is sketched.

\imageFigure{../../figures/ece1236/deck4TxlineFig2}{Transmission line equivalent circuit.}{fig:deck4Txline:deck4TxlineFig2}{0.2}

In this circuit all the elements have per-unit length units.  With \( I = C dV/dt \sim j \omega C V \), \( v = I R \), and \( V = L dI/dt \sim j \omega L I \), the KVL equation is

\begin{dmath}\label{eqn:uwaves4TransmissionLines:20}
V(z) - V(z + \Delta z) = I(z) \Delta z \lr{ R + j \omega L },
\end{dmath}

or in the \( \Delta z \rightarrow 0 \) limit

\begin{dmath}\label{eqn:uwaves4TransmissionLines:40}
\PD{z}{V} = -I(z) \lr{ R + j \omega L }.
\end{dmath}

The KCL equation at the interior node is

\begin{dmath}\label{eqn:uwaves4TransmissionLines:60}
-I(z) + I(z + \Delta z) + \lr{ j \omega C + G} V(z + \Delta z) = 0,
\end{dmath}

or
\begin{dmath}\label{eqn:uwaves4TransmissionLines:80}
\PD{z}{I} = -V(z) \lr{ j \omega C + G}.
\end{dmath}

This pair of equations is known as the \textAndIndex{telegrapher's equations}

\boxedEquation{eqn:uwaves4TransmissionLines:100}{
\begin{aligned}
\PD{z}{V} &= -I(z) \lr{ R + j \omega L } \\
\PD{z}{I} &= -V(z) \lr{ j \omega C + G}.
\end{aligned}
}

The second derivatives are

\begin{equation}\label{eqn:uwaves4TransmissionLines:120}
\begin{aligned}
\PDSq{z}{V} &= -\PD{z}{I} \lr{ R + j \omega L } \\
\PDSq{z}{I} &= -\PD{z}{V} \lr{ j \omega C + G},
\end{aligned}
\end{equation}

which allow the \( V, I \) to be decoupled
\boxedEquation{eqn:uwaves4TransmissionLines:140}{
\begin{aligned}
\PDSq{z}{V} &= V(z) \lr{ j \omega C + G} \lr{ R + j \omega L } \\
\PDSq{z}{I} &= I(z) \lr{ R + j \omega L } \lr{ j \omega C + G},
\end{aligned}
}

With a complex propagation constant

\begin{dmath}\label{eqn:uwaves4TransmissionLines:160}
\gamma
= \alpha + j \beta 
= \sqrt{ \lr{ j \omega C + G} \lr{ R + j \omega L } } 
= 
\sqrt{ R G - \omega^2 L C + j \omega ( L G + R C ) },
\end{dmath}

the decouple equations have the structure of a wave equation for a lossy line in the frequency domain


\boxedEquation{eqn:uwaves4TransmissionLines:180}{
\begin{aligned}
\PDSq{z}{V} - \gamma^2 V &= 0  \\
\PDSq{z}{I} - \gamma^2 I &= 0.
\end{aligned}
}

We write the solutions to these equations as

\begin{equation}\label{eqn:uwaves4TransmissionLines:200}
\begin{aligned}
V(z) &= V_0^{+} e^{-\gamma z} + V_0^{-} e^{+\gamma z} \\
I(z) &= I_0^{+} e^{-\gamma z} - I_0^{-} e^{+\gamma z} \\
\end{aligned}
\end{equation}

Only one of \( V \) or \( I \) is required since they are dependent through \cref{eqn:uwaves4TransmissionLines:100}, as can be seen by taking derivatives

\begin{dmath}\label{eqn:uwaves4TransmissionLines:220}
\PD{z}{V} 
= \gamma \lr{ -V_0^{+} e^{-\gamma z} + V_0^{-} e^{+\gamma z} }
= 
-I(z) \lr{ R + j \omega L },
\end{dmath}

so
\begin{equation}\label{eqn:uwaves4TransmissionLines:240}
I(z) 
=
\frac{\gamma}{ R + j \omega L } \lr{ V_0^{+} e^{-\gamma z} - V_0^{-} e^{+\gamma z} }.
\end{equation}

Introducing the characteristic impedance \( Z_0 \) of the line

\begin{dmath}\label{eqn:uwaves4TransmissionLines:260}
Z_0 = \frac{R + j \omega L}{\gamma} = \sqrt{ \frac{R + j \omega L}{G + j \omega C} },
\end{dmath}

we have

\begin{dmath}\label{eqn:uwaves4TransmissionLines:280}
I(z) 
=
\inv{Z_0} \lr{ V_0^{+} e^{-\gamma z} - V_0^{-} e^{+\gamma z} }.
=
I_0^{+} e^{-\gamma z} - I_0^{-} e^{+\gamma z},
\end{dmath}

where

\begin{equation}\label{eqn:uwaves4TransmissionLines:300}
\begin{aligned}
I_0^{+} &= \frac{V_0^{+}}{Z_0} \\
I_0^{-} &= \frac{V_0^{-}}{Z_0}.
\end{aligned}
\end{equation}

\section{Mapping TL geometry to per unit length \( C \) and \( L \) elements}

\makeexample{Coaxial cable.}{example:uwaves4TransmissionLines:1}{

From electrostatics and magnetostatics the per unit length induction and capacitance constants for a co-axial cable can be calculated.    For the cylindrical configuration sketched in \cref{fig:deck4Txline:deck4TxlineFig3}

\imageFigure{../../figures/ece1236/deck4TxlineFig3}{Coaxial cable.}{fig:deck4Txline:deck4TxlineFig3}{0.2}

From Gauss' law the total charge can be calculated assuming that the ends of the cable can be neglected

\begin{dmath}\label{eqn:uwaves4TransmissionLines:n}
Q 
= \int \spacegrad \cdot \BD dV 
= \oint \BD \cdot d\BA 
= \epsilon_0 \epsilon_r E ( 2 \pi r ) l,
\end{dmath}

This provides the radial electric field magnitude, in terms of the total charge

\begin{equation}\label{eqn:uwaves4TransmissionLines:320}
E = 
\frac{Q/l}{\epsilon_0 \epsilon_r ( 2 \pi r ) },
\end{equation}

which must be a radial field as sketched in \cref{fig:deck4Txline:deck4TxlineFig4}.

\imageFigure{../../figures/ece1236/deck4TxlineFig4}{Radial electric field for coaxial cable.}{fig:deck4Txline:deck4TxlineFig4}{0.2}

The potential difference from the inner transmission surface to the outer is

\begin{dmath}\label{eqn:uwaves4TransmissionLines:340}
V 
= \int_a^b E dr 
= 
\frac{Q/l}{2 \pi \epsilon_0 \epsilon_r }
\int_a^b \frac{dr}{r}
=
\frac{Q/l}{2 \pi \epsilon_0 \epsilon_r } \ln \frac{b}{a}.
\end{dmath}

Therefore the capacitance per unit length is

% i = C dV/dt => \int i dt = q = C V
\begin{equation}\label{eqn:uwaves4TransmissionLines:360}
C = \frac{Q/l}{V} = \frac{2 \pi \epsilon_0 \epsilon_r }{ \ln \frac{b}{a} } .
\end{equation}

% V = L di/dt = d\phi/dt
The inductance per unit length can be calculated form Ampere's law

\begin{dmath}\label{eqn:uwaves4TransmissionLines:380}
\int \lr{ \spacegrad \cross \BH } \cdot d\BS
=
\int \BJ \cdot d\BS + \PD{t}{} \int \cancel{\BD \cdot d\Bl}
I 
=
\oint \BH \cdot d\Bl
=
H ( 2 \pi r )
=
\frac{B}{\mu_0} ( 2 \pi r )
\end{dmath}

The flux is

\begin{dmath}\label{eqn:uwaves4TransmissionLines:400}
\Phi 
= \int \BB \cdot d\BA
= \frac{\mu_0 I}{ 2 \pi } \int_A \inv{r} d dr
= \frac{\mu_0 I}{ 2 \pi } \int_a^b \inv{r} l d dr
= \frac{\mu_0 I l}{ 2 \pi } \ln \frac{b}{a}.
\end{dmath}

The inductance per unit length is

\begin{equation}\label{eqn:uwaves4TransmissionLines:420}
L = \frac{\Phi/l}{I} = \frac{\mu_0}{ 2 \pi } \ln \frac{b}{a}.
\end{equation}

%\cref{fig:deck4Txline:deck4TxlineFig5}.
%\imageFigure{../../figures/ece1236/deck4TxlineFig5}{CAPTION: deck4TxlineFig5}{fig:deck4Txline:deck4TxlineFig5}{0.3}

For a lossless line where \( R = G = 0 \), we have \( \gamma = \sqrt{ (j \omega L)(j \omega C)} = j \omega \sqrt{L C} \),
so the phase velocity for a (lossless) coaxial cable is

\begin{dmath}\label{eqn:uwaves4TransmissionLines:440}
v_\phi 
= \frac{\omega}{\beta} 
= \frac{\omega}{\Imag(\gamma)}
= \frac{\omega}{\omega \sqrt{LC})}
= \frac{1}{\sqrt{LC})}.
\end{dmath}

This gives

\begin{dmath}\label{eqn:uwaves4TransmissionLines:460}
v_\phi^2 
= \inv{ L }  \inv{C}
= 
\frac{ 2 \pi }{ \mu_0 \ln \frac{b}{a} }
\frac
{\ln \frac{b}{a}}
{2 \pi \epsilon_0 \epsilon_r }
=
\frac{1 }{ \mu_0 \epsilon_0 \epsilon_r }
=
\frac{1 }{ \mu_0 \epsilon }.
\end{dmath}

So

\begin{dmath}\label{eqn:uwaves4TransmissionLines:480}
v_\phi = \inv{\sqrt{\epsilon \mu_0}},
\end{dmath}

which is the speed of light in the medium (\(\epsilon_r\)) that fills the co-axial cable.

This is \underline{not} a coincidence.  In any two-wire homogeneously filled transmission line, the phase velocity is equal to the speed of light in the unbounded medium that fills the line.

The characteristic impedance (again assuming the lossless \( R = G = 0 \) case) is

\begin{dmath}\label{eqn:uwaves4TransmissionLines:500}
Z_0 
= \sqrt{ \frac{\cancel{R} + j \omega L}{\cancel{G} + j \omega C} } 
= \sqrt{ \frac{L}{C} }
= \sqrt{
\frac{\Phi/l}{I} = \frac{\mu_0}{ 2 \pi } \ln \frac{b}{a}
\frac{ \ln \frac{b}{a} } 
{2 \pi \epsilon_0 \epsilon_r }
}
\end{dmath}

} % example

\EndArticle
%\EndNoBibArticle
%}
