%
% Copyright � 2015 Peeter Joot.  All Rights Reserved.
% Licenced as described in the file LICENSE under the root directory of this GIT repository.
%
\documentclass[]{eliblog}

\usepackage{amsmath}
\usepackage{mathpazo}

%
% shorthand for bold symbols, convenient for vectors and matrices
%
\newcommand{\Ba}[0]{\mathbf{a}}
\newcommand{\Bb}[0]{\mathbf{b}}
\newcommand{\Bc}[0]{\mathbf{c}}
\newcommand{\Bd}[0]{\mathbf{d}}
\newcommand{\Be}[0]{\mathbf{e}}
\newcommand{\Bf}[0]{\mathbf{f}}
\newcommand{\Bg}[0]{\mathbf{g}}
\newcommand{\Bh}[0]{\mathbf{h}}
\newcommand{\Bi}[0]{\mathbf{i}}
\newcommand{\Bj}[0]{\mathbf{j}}
\newcommand{\Bk}[0]{\mathbf{k}}
\newcommand{\Bl}[0]{\mathbf{l}}
\newcommand{\Bm}[0]{\mathbf{m}}
\newcommand{\Bn}[0]{\mathbf{n}}
\newcommand{\Bo}[0]{\mathbf{o}}
\newcommand{\Bp}[0]{\mathbf{p}}
\newcommand{\Bq}[0]{\mathbf{q}}
\newcommand{\Br}[0]{\mathbf{r}}
\newcommand{\Bs}[0]{\mathbf{s}}
\newcommand{\Bt}[0]{\mathbf{t}}
\newcommand{\Bu}[0]{\mathbf{u}}
\newcommand{\Bv}[0]{\mathbf{v}}
\newcommand{\Bw}[0]{\mathbf{w}}
\newcommand{\Bx}[0]{\mathbf{x}}
\newcommand{\By}[0]{\mathbf{y}}
\newcommand{\Bz}[0]{\mathbf{z}}
\newcommand{\BA}[0]{\mathbf{A}}
\newcommand{\BB}[0]{\mathbf{B}}
\newcommand{\BC}[0]{\mathbf{C}}
\newcommand{\BD}[0]{\mathbf{D}}
\newcommand{\BE}[0]{\mathbf{E}}
\newcommand{\BF}[0]{\mathbf{F}}
\newcommand{\BG}[0]{\mathbf{G}}
\newcommand{\BH}[0]{\mathbf{H}}
\newcommand{\BI}[0]{\mathbf{I}}
\newcommand{\BJ}[0]{\mathbf{J}}
\newcommand{\BK}[0]{\mathbf{K}}
\newcommand{\BL}[0]{\mathbf{L}}
\newcommand{\BM}[0]{\mathbf{M}}
\newcommand{\BN}[0]{\mathbf{N}}
\newcommand{\BO}[0]{\mathbf{O}}
\newcommand{\BP}[0]{\mathbf{P}}
\newcommand{\BQ}[0]{\mathbf{Q}}
\newcommand{\BR}[0]{\mathbf{R}}
\newcommand{\BS}[0]{\mathbf{S}}
\newcommand{\BT}[0]{\mathbf{T}}
\newcommand{\BU}[0]{\mathbf{U}}
\newcommand{\BV}[0]{\mathbf{V}}
\newcommand{\BW}[0]{\mathbf{W}}
\newcommand{\BX}[0]{\mathbf{X}}
\newcommand{\BY}[0]{\mathbf{Y}}
\newcommand{\BZ}[0]{\mathbf{Z}}

\newcommand{\Bzero}[0]{\mathbf{0}}
\newcommand{\Btheta}[0]{\boldsymbol{\theta}}
\newcommand{\Btau}[0]{\boldsymbol{\tau}}
\newcommand{\Bomega}[0]{\boldsymbol{\omega}}

%
% shorthand for unit vectors
%
\newcommand{\acap}[0]{\hat{\Ba}}
\newcommand{\bcap}[0]{\hat{\Bb}}
\newcommand{\ccap}[0]{\hat{\Bc}}
\newcommand{\dcap}[0]{\hat{\Bd}}
\newcommand{\ecap}[0]{\hat{\Be}}
\newcommand{\fcap}[0]{\hat{\Bf}}
\newcommand{\gcap}[0]{\hat{\Bg}}
\newcommand{\hcap}[0]{\hat{\Bh}}
\newcommand{\icap}[0]{\hat{\Bi}}
\newcommand{\jcap}[0]{\hat{\Bj}}
\newcommand{\kcap}[0]{\hat{\Bk}}
\newcommand{\lcap}[0]{\hat{\Bl}}
\newcommand{\mcap}[0]{\hat{\Bm}}
\newcommand{\ncap}[0]{\hat{\Bn}}
\newcommand{\ocap}[0]{\hat{\Bo}}
\newcommand{\pcap}[0]{\hat{\Bp}}
\newcommand{\qcap}[0]{\hat{\Bq}}
\newcommand{\rcap}[0]{\hat{\Br}}
\newcommand{\scap}[0]{\hat{\Bs}}
\newcommand{\tcap}[0]{\hat{\Bt}}
\newcommand{\ucap}[0]{\hat{\Bu}}
\newcommand{\vcap}[0]{\hat{\Bv}}
\newcommand{\wcap}[0]{\hat{\Bw}}
\newcommand{\xcap}[0]{\hat{\Bx}}
\newcommand{\ycap}[0]{\hat{\By}}
\newcommand{\zcap}[0]{\hat{\Bz}}
\newcommand{\thetacap}[0]{\hat{\Btheta}}

%
% to write R^n and C^n in a distinguishable fashion.  Perhaps change this
% to the double lined characters upon figuring out how to do so.
%
\newcommand{\C}[1]{$\mathbb{C}^{#1}$}
\newcommand{\R}[1]{$\mathbb{R}^{#1}$}

%
% various generally useful helpers
%

% derivative of #1 wrt. #2:
\newcommand{\D}[2] {\frac {d#2} {d#1}}

\newcommand{\inv}[1]{\frac{1}{#1}}
\newcommand{\cross}[0]{\times}

\newcommand{\abs}[1]{\lvert{#1}\rvert}
\newcommand{\norm}[1]{\lVert{#1}\rVert}
\newcommand{\innerprod}[2]{\langle{#1}, {#2}\rangle}
\newcommand{\dotprod}[2]{{#1} \cdot {#2}}
\newcommand{\bdotprod}[2]{\left({#1} \cdot {#2}\right)}
\newcommand{\crossprod}[2]{{#1} \cross {#2}}
\newcommand{\tripleprod}[3]{\dotprod{\left(\crossprod{#1}{#2}\right)}{#3}}

\DeclareMathOperator{\Proj}{Proj}
\DeclareMathOperator{\Span}{span}
\DeclareMathOperator{\Sgn}{sgn}
\DeclareMathOperator{\Area}{Area}
\DeclareMathOperator{\Volume}{Volume}

%
% A few miscellaneous things specific to this document
%
\newcommand{\crossop}[1]{\crossprod{#1}{}}

% R2 vector.
\newcommand{\VectorTwo}[2]{
\begin{bmatrix}
 {#1} \\
 {#2}
\end{bmatrix}
}

\newcommand{\VectorN}[1]{
\begin{bmatrix}
{#1}_1 \\
{#1}_2 \\
\vdots \\
{#1}_N \\
\end{bmatrix}
}

\newcommand{\DETuvij}[4]{
\begin{vmatrix}
 {#1}_{#3} & {#1}_{#4} \\
 {#2}_{#3} & {#2}_{#4}
\end{vmatrix}
}

\newcommand{\DETuvwijk}[6]{
\begin{vmatrix}
 {#1}_{#4} & {#1}_{#5} & {#1}_{#6} \\
 {#2}_{#4} & {#2}_{#5} & {#2}_{#6} \\
 {#3}_{#4} & {#3}_{#5} & {#3}_{#6}
\end{vmatrix}
}

\newcommand{\DETuvwxijkl}[8]{
\begin{vmatrix}
 {#1}_{#5} & {#1}_{#6} & {#1}_{#7} & {#1}_{#8} \\
 {#2}_{#5} & {#2}_{#6} & {#2}_{#7} & {#2}_{#8} \\
 {#3}_{#5} & {#3}_{#6} & {#3}_{#7} & {#3}_{#8} \\
 {#4}_{#5} & {#4}_{#6} & {#4}_{#7} & {#4}_{#8} \\
\end{vmatrix}
}

%\newcommand{\DETuvwxyijklm}[10]{
%\begin{vmatrix}
% {#1}_{#6} & {#1}_{#7} & {#1}_{#8} & {#1}_{#9} & {#1}_{#10} \\
% {#2}_{#6} & {#2}_{#7} & {#2}_{#8} & {#2}_{#9} & {#2}_{#10} \\
% {#3}_{#6} & {#3}_{#7} & {#3}_{#8} & {#3}_{#9} & {#3}_{#10} \\
% {#4}_{#6} & {#4}_{#7} & {#4}_{#8} & {#4}_{#9} & {#4}_{#10} \\
% {#5}_{#6} & {#5}_{#7} & {#5}_{#8} & {#5}_{#9} & {#5}_{#10}
%\end{vmatrix}
%}

% R3 vector.
\newcommand{\VectorThree}[3]{
\begin{bmatrix}
 {#1} \\
 {#2} \\
 {#3}
\end{bmatrix}
}



\author{Peeter Joot}
\email{peeter.joot@gmail.com}

%\documentclass[]{eliblogwidescreen}

\usepackage{amsmath}
\usepackage{mathpazo}

%
% shorthand for bold symbols, convenient for vectors and matrices
%
\newcommand{\Ba}[0]{\mathbf{a}}
\newcommand{\Bb}[0]{\mathbf{b}}
\newcommand{\Bc}[0]{\mathbf{c}}
\newcommand{\Bd}[0]{\mathbf{d}}
\newcommand{\Be}[0]{\mathbf{e}}
\newcommand{\Bf}[0]{\mathbf{f}}
\newcommand{\Bg}[0]{\mathbf{g}}
\newcommand{\Bh}[0]{\mathbf{h}}
\newcommand{\Bi}[0]{\mathbf{i}}
\newcommand{\Bj}[0]{\mathbf{j}}
\newcommand{\Bk}[0]{\mathbf{k}}
\newcommand{\Bl}[0]{\mathbf{l}}
\newcommand{\Bm}[0]{\mathbf{m}}
\newcommand{\Bn}[0]{\mathbf{n}}
\newcommand{\Bo}[0]{\mathbf{o}}
\newcommand{\Bp}[0]{\mathbf{p}}
\newcommand{\Bq}[0]{\mathbf{q}}
\newcommand{\Br}[0]{\mathbf{r}}
\newcommand{\Bs}[0]{\mathbf{s}}
\newcommand{\Bt}[0]{\mathbf{t}}
\newcommand{\Bu}[0]{\mathbf{u}}
\newcommand{\Bv}[0]{\mathbf{v}}
\newcommand{\Bw}[0]{\mathbf{w}}
\newcommand{\Bx}[0]{\mathbf{x}}
\newcommand{\By}[0]{\mathbf{y}}
\newcommand{\Bz}[0]{\mathbf{z}}
\newcommand{\BA}[0]{\mathbf{A}}
\newcommand{\BB}[0]{\mathbf{B}}
\newcommand{\BC}[0]{\mathbf{C}}
\newcommand{\BD}[0]{\mathbf{D}}
\newcommand{\BE}[0]{\mathbf{E}}
\newcommand{\BF}[0]{\mathbf{F}}
\newcommand{\BG}[0]{\mathbf{G}}
\newcommand{\BH}[0]{\mathbf{H}}
\newcommand{\BI}[0]{\mathbf{I}}
\newcommand{\BJ}[0]{\mathbf{J}}
\newcommand{\BK}[0]{\mathbf{K}}
\newcommand{\BL}[0]{\mathbf{L}}
\newcommand{\BM}[0]{\mathbf{M}}
\newcommand{\BN}[0]{\mathbf{N}}
\newcommand{\BO}[0]{\mathbf{O}}
\newcommand{\BP}[0]{\mathbf{P}}
\newcommand{\BQ}[0]{\mathbf{Q}}
\newcommand{\BR}[0]{\mathbf{R}}
\newcommand{\BS}[0]{\mathbf{S}}
\newcommand{\BT}[0]{\mathbf{T}}
\newcommand{\BU}[0]{\mathbf{U}}
\newcommand{\BV}[0]{\mathbf{V}}
\newcommand{\BW}[0]{\mathbf{W}}
\newcommand{\BX}[0]{\mathbf{X}}
\newcommand{\BY}[0]{\mathbf{Y}}
\newcommand{\BZ}[0]{\mathbf{Z}}

\newcommand{\Bzero}[0]{\mathbf{0}}
\newcommand{\Btheta}[0]{\boldsymbol{\theta}}
\newcommand{\Btau}[0]{\boldsymbol{\tau}}
\newcommand{\Bomega}[0]{\boldsymbol{\omega}}

%
% shorthand for unit vectors
%
\newcommand{\acap}[0]{\hat{\Ba}}
\newcommand{\bcap}[0]{\hat{\Bb}}
\newcommand{\ccap}[0]{\hat{\Bc}}
\newcommand{\dcap}[0]{\hat{\Bd}}
\newcommand{\ecap}[0]{\hat{\Be}}
\newcommand{\fcap}[0]{\hat{\Bf}}
\newcommand{\gcap}[0]{\hat{\Bg}}
\newcommand{\hcap}[0]{\hat{\Bh}}
\newcommand{\icap}[0]{\hat{\Bi}}
\newcommand{\jcap}[0]{\hat{\Bj}}
\newcommand{\kcap}[0]{\hat{\Bk}}
\newcommand{\lcap}[0]{\hat{\Bl}}
\newcommand{\mcap}[0]{\hat{\Bm}}
\newcommand{\ncap}[0]{\hat{\Bn}}
\newcommand{\ocap}[0]{\hat{\Bo}}
\newcommand{\pcap}[0]{\hat{\Bp}}
\newcommand{\qcap}[0]{\hat{\Bq}}
\newcommand{\rcap}[0]{\hat{\Br}}
\newcommand{\scap}[0]{\hat{\Bs}}
\newcommand{\tcap}[0]{\hat{\Bt}}
\newcommand{\ucap}[0]{\hat{\Bu}}
\newcommand{\vcap}[0]{\hat{\Bv}}
\newcommand{\wcap}[0]{\hat{\Bw}}
\newcommand{\xcap}[0]{\hat{\Bx}}
\newcommand{\ycap}[0]{\hat{\By}}
\newcommand{\zcap}[0]{\hat{\Bz}}
\newcommand{\thetacap}[0]{\hat{\Btheta}}

%
% to write R^n and C^n in a distinguishable fashion.  Perhaps change this
% to the double lined characters upon figuring out how to do so.
%
\newcommand{\C}[1]{$\mathbb{C}^{#1}$}
\newcommand{\R}[1]{$\mathbb{R}^{#1}$}

%
% various generally useful helpers
%

% derivative of #1 wrt. #2:
\newcommand{\D}[2] {\frac {d#2} {d#1}}

\newcommand{\inv}[1]{\frac{1}{#1}}
\newcommand{\cross}[0]{\times}

\newcommand{\abs}[1]{\lvert{#1}\rvert}
\newcommand{\norm}[1]{\lVert{#1}\rVert}
\newcommand{\innerprod}[2]{\langle{#1}, {#2}\rangle}
\newcommand{\dotprod}[2]{{#1} \cdot {#2}}
\newcommand{\bdotprod}[2]{\left({#1} \cdot {#2}\right)}
\newcommand{\crossprod}[2]{{#1} \cross {#2}}
\newcommand{\tripleprod}[3]{\dotprod{\left(\crossprod{#1}{#2}\right)}{#3}}

\DeclareMathOperator{\Proj}{Proj}
\DeclareMathOperator{\Span}{span}
\DeclareMathOperator{\Sgn}{sgn}
\DeclareMathOperator{\Area}{Area}
\DeclareMathOperator{\Volume}{Volume}

%
% A few miscellaneous things specific to this document
%
\newcommand{\crossop}[1]{\crossprod{#1}{}}

% R2 vector.
\newcommand{\VectorTwo}[2]{
\begin{bmatrix}
 {#1} \\
 {#2}
\end{bmatrix}
}

\newcommand{\VectorN}[1]{
\begin{bmatrix}
{#1}_1 \\
{#1}_2 \\
\vdots \\
{#1}_N \\
\end{bmatrix}
}

\newcommand{\DETuvij}[4]{
\begin{vmatrix}
 {#1}_{#3} & {#1}_{#4} \\
 {#2}_{#3} & {#2}_{#4}
\end{vmatrix}
}

\newcommand{\DETuvwijk}[6]{
\begin{vmatrix}
 {#1}_{#4} & {#1}_{#5} & {#1}_{#6} \\
 {#2}_{#4} & {#2}_{#5} & {#2}_{#6} \\
 {#3}_{#4} & {#3}_{#5} & {#3}_{#6}
\end{vmatrix}
}

\newcommand{\DETuvwxijkl}[8]{
\begin{vmatrix}
 {#1}_{#5} & {#1}_{#6} & {#1}_{#7} & {#1}_{#8} \\
 {#2}_{#5} & {#2}_{#6} & {#2}_{#7} & {#2}_{#8} \\
 {#3}_{#5} & {#3}_{#6} & {#3}_{#7} & {#3}_{#8} \\
 {#4}_{#5} & {#4}_{#6} & {#4}_{#7} & {#4}_{#8} \\
\end{vmatrix}
}

%\newcommand{\DETuvwxyijklm}[10]{
%\begin{vmatrix}
% {#1}_{#6} & {#1}_{#7} & {#1}_{#8} & {#1}_{#9} & {#1}_{#10} \\
% {#2}_{#6} & {#2}_{#7} & {#2}_{#8} & {#2}_{#9} & {#2}_{#10} \\
% {#3}_{#6} & {#3}_{#7} & {#3}_{#8} & {#3}_{#9} & {#3}_{#10} \\
% {#4}_{#6} & {#4}_{#7} & {#4}_{#8} & {#4}_{#9} & {#4}_{#10} \\
% {#5}_{#6} & {#5}_{#7} & {#5}_{#8} & {#5}_{#9} & {#5}_{#10}
%\end{vmatrix}
%}

% R3 vector.
\newcommand{\VectorThree}[3]{
\begin{bmatrix}
 {#1} \\
 {#2} \\
 {#3}
\end{bmatrix}
}



\author{Peeter Joot}
\email{peeter.joot@gmail.com}


\chapter{mytitle}
\label{chap:matrixVectorPotentials}
%\useCCL
\blogpage{http://sites.google.com/site/peeterjoot/math2011/matrixVectorPotentials.pdf}
\date{April 30, 2011}
\revisionInfo{matrixVectorPotentials.tex}

\beginArtWithToc
%\beginArtNoToc

\section{Motivation.}

A while ago I worked the problem of determining the equations of motion for a chain like object.  This was idealized as a set of $N$ interconnected spherical pendulums.  One of the aspects of that problem that I found fun was that it allowed me to use a new construct, factoring vectors into multivector matrix products, multiplied using the Geometric (Clifford) product.  It seemed at the time that this made the problem tractable, whereas a traditional formulation was much less so.  Later I realized that a very similar factorization was possible with matrices directly.

I encountered a new use for this idea of factoring a vector into a product of multivector matrices.  Namely, a calculation of the four vector Lienard-Wiechert potentials, given a general motion described in cylindrical coordinates.  This I thought I'd try since we had a similar problem on our exam (with the motion of the charged particle additionally constrained to a circle).

\section{The goal of the calculation.}

Our problem is to calculate

\begin{align}\label{eqn:matrixVectorPotentials:n}
A^0 &= \frac{q}{R^\conj} \\
\BA &= \frac{q \Bv_c}{c R^\conj}
\end{align}

where $\Bx_c(t)$ is the location of the charged particle, $\Br$ is the point that the field is measured, and 

\begin{align}\label{eqn:matrixVectorPotentials:n}
R^\conj &= R - \frac{\Bv_c}{c} \cdot \BR \\
R^2 &= \BR^2 = c^2( t - t_r)^2 \\
\BR &= \Br - \Bx_c(t_r) \\
\Bv_c &= \PD{t_r}{\Bx_c}.
\end{align}

\section{Calculating the potentials for an arbitrary cylindrical motion.}

Suppose that our charged particle has the trajectory

\begin{equation}\label{eqn:matrixVectorPotentials:n}
\Bx_c(t) = h(t) \Be_3 + a(t) \Be_1 e^{i \theta(t)}
\end{equation}

where $i = \Be_1 \Be_2$, and we measure the field at the point

\begin{equation}\label{eqn:matrixVectorPotentials:n}
\Br = z \Be_3 + \rho \Be_1 e^{i \phi}
\end{equation}

The vector separation between the two is

\begin{align*}
\BR 
&= \Br - \Bx_c \\
&= (z - h) \Be_3 + \Be_1 ( \rho e^{i\phi} - a e^{i\theta} ) \\
&=
\begin{bmatrix}
\Be_1 e^{i\phi} & - \Be_1 e^{i\theta} & \Be_3
\end{bmatrix}
\begin{bmatrix}
\rho \\
a \\
z - h
\end{bmatrix}
\end{align*}

Transposition does not change this at all, so the (squared) length of this vector difference is

\begin{align*}
\BR^2 &=
\begin{bmatrix}
\rho &
a & 
(z - h)
\end{bmatrix}
\begin{bmatrix}
\Be_1 e^{i\phi} \\
- \Be_1 e^{i\theta} \\
 \Be_3
\end{bmatrix}
\begin{bmatrix}
\Be_1 e^{i\phi} & - \Be_1 e^{i\theta} & \Be_3
\end{bmatrix}
\begin{bmatrix}
\rho \\
a \\
z - h
\end{bmatrix} \\
&=
\begin{bmatrix}
\rho &
a & 
(z - h)
\end{bmatrix}
\begin{bmatrix}
\Be_1 e^{i\phi} \Be_1 e^{i\phi} & - \Be_1 e^{i\phi} \Be_1 e^{i\theta} & \Be_1 e^{i\phi} \Be_3 \\
- \Be_1 e^{i\theta} \Be_1 e^{i\phi} & \Be_1 e^{i\theta} \Be_1 e^{i\theta} & - \Be_1 e^{i\theta} \Be_3 \\
 \Be_3 \Be_1 e^{i\phi} & -\Be_3 \Be_1 e^{i\theta} & \Be_3 \Be_3 \\
\end{bmatrix}
\begin{bmatrix}
\rho \\
a \\
z - h
\end{bmatrix} \\
&=
\begin{bmatrix}
\rho &
a & 
(z - h)
\end{bmatrix}
\begin{bmatrix}
1 & - e^{i(\theta-\phi)} & \Be_1 e^{i\phi} \Be_3 \\
- e^{i(\phi -\theta)} & 1 & - \Be_1 e^{i\theta} \Be_3 \\
 \Be_3 \Be_1 e^{i\phi} & -\Be_3 \Be_1 e^{i\theta} & 1 \\
\end{bmatrix}
\begin{bmatrix}
\rho \\
a \\
z - h
\end{bmatrix} \\
\end{align*}

\subsection{A motivation for a Hermitian like transposition operation.}

There are a few things of note about this matrix.  One of which is that it is \underline{not} symmetric.  This is a consequence of the non-commutative nature of the vector products.  What we do have is a Hermitian transpose like symmetry.  Observe that terms like the $(1,2)$ and the $(2,1)$ elements of the matrix are equal after all the vector products are reversed.

Using tilde to denote this reversion, we have

\begin{align*}
(e^{i (\theta - \phi)})^{\tilde{}}
&=
\cos(\theta - \phi)
+ (\Be_1 \Be_2)^{\tilde{}}
\sin(\theta - \phi) \\
&=
\cos(\theta - \phi)
+ \Be_2 \Be_1
\sin(\theta - \phi) \\
&=
\cos(\theta - \phi)
- \Be_1 \Be_2 
\sin(\theta - \phi) \\
&=
e^{-i (\theta -\phi)}.
\end{align*}

The fact that all the elements of this matrix, if non-scalar, have their reversed value in the transposed position, is sufficient to show that the end result is a scalar as expected.  Consider a general quadratic form where the matrix has scalar and bivector grades as above, where there is reversion in all the transposed positions.  That is

\begin{equation}\label{eqn:matrixVectorPotentials:n}
b^\T A b
\end{equation}

where $A = \Norm{A_{ij}}$, a $m \times m$ matrix where $A_{ij} = \tilde{A_{ji}}$ and contains scalar and bivector grades, and $b = \Norm{b_i}$, a $m\times 1$ column matrix of scalars.  Then the product is

\begin{align*}
\sum_{ij} b_i A_{ij} b_j
&=
\sum_{i<j} b_i A_{ij} b_j
+\sum_{j<i} b_i A_{ij} b_j
+\sum_{k} b_k A_{kk} b_k \\
&=
\sum_{i<j} b_i A_{ij} b_j
+\sum_{i<j} b_{j} A_{ji} b_i
+\sum_{k} b_k A_{kk} b_k \\
&=
\sum_{k} b_k A_{kk} b_k + 2 \sum_{i<j} b_i (A_{ij} + A_{ji}) b_j \\
&=
\sum_{k} b_k A_{kk} b_k + 2 \sum_{i<j} b_i (A_{ij} + \tilde{A_{ij}}) b_j
\end{align*}

The quantity in braces $A_{ij} + \tilde{A_{ij}}$ is a scalar since any of the bivector grades in $A_{ij}$ cancel out.  Consider a similar general product of a vector after the vector has been factored into a product of matrices of multivector elements

\begin{equation}\label{eqn:matrixVectorPotentials:n}
\Bx = 
\begin{bmatrix}
a_1 & a_2 & \hdots & a_m
\end{bmatrix}
\begin{bmatrix}
b_1 \\ b_2 \\ \vdots \\ b_m
\end{bmatrix}
%=
%\left(
%\begin{bmatrix}
%\tilde{b_1} & \tilde{b_2} & \hdots & \tilde{b_m}
%\end{bmatrix}
%\begin{bmatrix}
%\tilde{a_1} \\ \tilde{a_2} \\ \vdots \\ \tilde{a_m}
%\end{bmatrix}
%\right)^{\tilde{}}.
\end{equation}

The (squared) length of the vector is

\begin{align*}\label{eqn:matrixVectorPotentials:n}
\Bx^2 
&= (a_i b_i) (a_j b_j)  \\
&= (a_i b_i)^{\tilde{}} a_j b_j  \\
&= \tilde{b_i} \tilde{a_i} a_j b_j  \\
&= \tilde{b_i} (\tilde{a_i} a_j) b_j.
\end{align*}

It is clear that we want a transposition operation that includes reversal of its elements, so with a general factorization of a vector into matrices of multivectors $\Bx = A b$, it's square will be $\Bx = {\tilde{b}}^\T {\tilde{A}}^\T A b$.

As with purely complex valued matrices, it is convenient to use the dagger notation, and define

\begin{equation}\label{eqn:matrixVectorPotentials:n}
A^\dagger = \tilde{A}^\T
\end{equation}

where $\tilde{A}$ contains the reversed elements of $A$.  By extension, we can define dot and wedge products of vectors expressed as products of multivector matrices.  Given $\Bx = A b$, a row vector and column vector product, and $\By = C d$, where each of the rows or columns has $m$ elements, the dot and wedge products are

\begin{align}\label{eqn:matrixVectorPotentials:n}
\Bx \cdot \By &= \gpgradezero{ d^\dagger C^\dagger A b } \\
\Bx \wedge \By &= \gpgradetwo{ d^\dagger C^\dagger A b }.
\end{align}

In particular, if $b$ and $d$ are matrices of scalars we have

\begin{align}\label{eqn:matrixVectorPotentials:n}
\Bx \cdot \By &= d^\T \gpgradezero{C^\dagger A} b = d^\T \frac{C^\dagger A + A^\dagger C}{2} b \\
\Bx \wedge \By &= d^\T \gpgradetwo{C^\dagger A} b = d^\T \frac{C^\dagger A - A^\dagger C}{2} b.
\end{align}

The dot product is seen as a generator of symmetric matrices, and the wedge product a generator of purely antisymmetric matrices.

\subsection{Back to the problem}

Now, returning to the example above, where we want $\BR^2$.  We've seen that we can drop any bivector terms from the matrix, so that the squared length can be reduced as

\begin{align*}
\BR^2 
&=
\begin{bmatrix}
\rho &
a & 
(z - h)
\end{bmatrix}
\begin{bmatrix}
1 & - e^{i(\theta-\phi)} & 0 \\
- e^{i(\phi -\theta)} & 1 & 0 \\
0 & 0 & 1 \\
\end{bmatrix}
\begin{bmatrix}
\rho \\
a \\
z - h
\end{bmatrix} \\
&=
\begin{bmatrix}
\rho &
a & 
(z - h)
\end{bmatrix}
\begin{bmatrix}
1 & - \cos(\theta-\phi) & 0 \\
- \cos(\theta -\phi) & 1 & 0 \\
0 & 0 & 1 \\
\end{bmatrix}
\begin{bmatrix}
\rho \\
a \\
z - h
\end{bmatrix} \\
&=
\begin{bmatrix}
\rho &
a & 
(z - h)
\end{bmatrix}
\begin{bmatrix}
\rho - a \cos(\theta - \phi) \\
- \rho \cos(\theta - \phi) + a \\
z - h
\end{bmatrix}
\end{align*}

So we have
\begin{equation}\label{eqn:matrixVectorPotentials:n}
\BR^2 = \rho^2 + a^2 + (z -h)^2 - 2 a \rho \cos(\theta - \phi)
\end{equation}

\EndArticle
%\EndNoBibArticle
