%
% Copyright � 2016 Peeter Joot.  All Rights Reserved.
% Licenced as described in the file LICENSE under the root directory of this GIT repository.
%
%{
\newcommand{\authorname}{Peeter Joot}
\newcommand{\email}{peeterjoot@protonmail.com}
\newcommand{\basename}{FIXMEbasenameUndefined}
\newcommand{\dirname}{notes/FIXMEdirnameUndefined/}

\renewcommand{\basename}{noetherCurrentScalarField}
\renewcommand{\dirname}{notes/phy1520/}
%\newcommand{\dateintitle}{}
%\newcommand{\keywords}{}

\newcommand{\authorname}{Peeter Joot}
\newcommand{\onlineurl}{http://sites.google.com/site/peeterjoot2/math2013/\basename.pdf}
\newcommand{\sourcepath}{\dirname\basename.tex}
\newcommand{\generatetitle}[1]{\chapter{#1}}

\newcommand{\vcsinfo}{%
\section*{}
\noindent{\color{DarkOliveGreen}{\rule{\linewidth}{0.1mm}}}
\paragraph{Document version}
%\paragraph{\color{Maroon}{Document version}}
{
\small
\begin{itemize}
\item Available online at:\\ 
\href{\onlineurl}{\onlineurl}
\item Git Repository: \input{./.revinfo/gitRepo.tex}
\item Source: \sourcepath
\item last commit: \input{./.revinfo/gitCommitString.tex}
\item commit date: \input{./.revinfo/gitCommitDate.tex}
\end{itemize}
}
}

%\PassOptionsToPackage{dvipsnames,svgnames}{xcolor}
\PassOptionsToPackage{square,numbers}{natbib}
\documentclass{scrreprt}

\usepackage[left=2cm,right=2cm]{geometry}
\usepackage[svgnames]{xcolor}
\usepackage{peeters_layout}

\usepackage{natbib}

\usepackage[
colorlinks=true,
bookmarks=false,
pdfauthor={\authorname, \email},
backref 
]{hyperref}

% http://tex.stackexchange.com/questions/75773/how-to-reference-problems-by-the-text-label-in-an-exercise-envioronment
\usepackage[english]{cleveref}
\crefname{Exercise}{exercise}{exercises}
\Crefname{Exercise}{Exercise}{Exercises}

\RequirePackage{titlesec}
\RequirePackage{ifthen}

% http://stackoverflow.com/questions/4932910/date-in-the-tabular-environment
\makeatletter
\let\insertdate\@date
\makeatother

\titleformat{\chapter}[display]
{\bfseries\Large}
{\color{DarkSlateGrey}\filleft \authorname
\ifthenelse{\isundefined{\studentnumber}}{}{\\ \studentnumber}
\ifthenelse{\isundefined{\email}}{}{\\ \email}
\ifthenelse{\isundefined{\dateintitle}}{}{\\ \insertdate}
%\ifthenelse{\isundefined{\coursename}}{}{\\ \coursename} % put in title instead.
}
{4ex}
{\color{DarkOliveGreen}{\titlerule}\color{Maroon}
\vspace{2ex}%
\filright}
[\vspace{2ex}%
\color{DarkOliveGreen}\titlerule
]

\newcommand{\beginArtWithToc}[0]{\begin{document}\tableofcontents}
\newcommand{\beginArtNoToc}[0]{\begin{document}}
\newcommand{\EndNoBibArticle}[0]{\end{document}}
\newcommand{\EndArticle}[0]{\bibliography{Bibliography}\bibliographystyle{plainnat}\end{document}}

% 
%\newcommand{\citep}[1]{\cite{#1}}

\colorSectionsForArticle



\usepackage{peeters_layout_exercise}
\usepackage{peeters_braket}
\usepackage{peeters_figures}
\usepackage{macros_cal}

\beginArtNoToc

\generatetitle{Energy-momentum tensor for a scalar field}
%\chapter{Energy-momentum tensor for a scalar field}
%\label{chap:noetherCurrentScalarField}
% \citep{sakurai2014modern} pr X.Y
It is claimed in \citep{qftLectureNotes} (3.2.1) that the momentum components of the energy-momentum tensor simplifies to

\begin{dmath}\label{eqn:noetherCurrentScalarField:20}
\Be_n \int d^3 x T^{0 n} = \int d^3 k \Bk a_k^\dagger a_k.
\end{dmath}

Let's calculate this.

First, from the Noether current for the scalar field Lagrangian in question, what is the energy-momentum tensor explicitly?

\begin{dmath}\label{eqn:noetherCurrentScalarField:40}
T^{\mu \nu}
= \Pi^\mu \partial^\nu \phi - g^{\mu \nu} \LL
= \Pi^\mu \partial^\nu \phi - g^{\mu \nu} \inv{2} \lr{ \partial_\alpha \phi \partial^\alpha \phi - \mu^2 \phi^2 }
= \Pi^\mu \Pi^\nu - g^{\mu \nu} \inv{2} \lr{ \Pi_\alpha \Pi^\alpha - \mu^2 \phi^2 }
= \Pi^\mu \Pi^\nu - \inv{2} g^{\mu \nu} g_{\alpha\beta} \Pi^\beta \Pi^\alpha + \inv{2} g^{\mu \nu} \mu^2 \phi^2.
\end{dmath}

Consider some special cases for the indexes.  For \( \mu = \nu = 0 \), we get the Hamiltonian density

\begin{dmath}\label{eqn:noetherCurrentScalarField:200}
T^{00} 
= \Pi^0 \Pi^0 - \inv{2} g^{0 0} \Pi_\alpha \Pi^\alpha + \inv{2} g^{0 0} \mu^2 \phi^2
= \Pi^0 \Pi^0 - \inv{2} \Pi_\alpha \Pi^\alpha + \inv{2} \mu^2 \phi^2
= \inv{2} \Pi^0 \Pi^0 - \inv{2} \Pi_n \Pi^n + \inv{2} \mu^2 \phi^2
= \inv{2} \Pi^2 + \inv{2} (\spacegrad \phi)^2 + \inv{2} \mu^2 \phi^2,
\end{dmath}

where \( \Pi^2 = (\partial_0 \phi)^2 \ne \partial^2 \phi \).  For any \( \mu \ne \nu \) the off diagonal metric elements are zero, so we have just
\begin{dmath}\label{eqn:noetherCurrentScalarField:220}
T^{\mu\nu} = \Pi^\mu \Pi^\nu.
\end{dmath}

Finally, when \( n \ne 0 \), we have
\begin{dmath}\label{eqn:noetherCurrentScalarField:240}
T^{nn}
= \Pi^n \Pi^n - \inv{2} g^{n n} \Pi_\alpha \Pi^\alpha + \inv{2} g^{n n} n^2 \phi^2
= \Pi^n \Pi^n + \inv{2} \Pi_\alpha \Pi^\alpha - \inv{2} \mu^2 \phi^2
= \inv{2} \Pi^2 + \Pi^n \Pi^n - \inv{2} \Pi^m \Pi^m - \inv{2} \mu^2 \phi^2
= \inv{2} \Pi^2 + \inv{2} \Pi^n \Pi^n - \inv{2} \sum_{m\ne n,0} \Pi^m \Pi^m - \inv{2} \mu^2 \phi^2
= \inv{2} \sum_{m = n,0} \Pi^m \Pi^m - \inv{2} \sum_{m\ne n,0} \Pi^m \Pi^m - \inv{2} \mu^2 \phi^2.
\end{dmath}

The canonical momenta are

\begin{dmath}\label{eqn:noetherCurrentScalarField:60}
\Pi^\mu 
=
\partial^\mu
\int \frac{d^3 k}{(2\pi)^{3/2} \sqrt{ 2 \omega_k }} \lr{ a_k e^{-i k \cdot x} + a_k^\dagger e^{i k \cdot x} },
\end{dmath}

but
\begin{dmath}\label{eqn:noetherCurrentScalarField:80}
\partial^\mu e^{i k \cdot x}
=
\partial^\mu \exp\lr{ i k^\alpha x_\alpha }
=
i k^\mu \exp\lr{ i k \cdot x },
\end{dmath}

so
\begin{dmath}\label{eqn:noetherCurrentScalarField:100}
\Pi^\mu 
=
i
\int \frac{d^3 k k^\mu}{(2\pi)^{3/2} \sqrt{ 2 \omega_k }} \lr{ - a_k e^{-i k \cdot x} + a_k^\dagger e^{i k \cdot x} }
=
i
\int \frac{d^3 k k^\mu}{(2\pi)^{3/2} \sqrt{ 2 \omega_k }} \lr{ - a_k e^{-i \omega_k t + \Bk \cdot \Bx} + a_k^\dagger e^{i \omega_k t - i \Bk \cdot \Bx} }.
\end{dmath}

This gives
\begin{dmath}\label{eqn:noetherCurrentScalarField:120}
\int d^3 x \Pi^\mu \Pi^\nu 
=
-\inv{2} \int d^3 x \inv{(2\pi)^3}
\int d^3 k d^3 j \frac{k^\mu j^\nu}{\sqrt{\omega_k \omega_j}}
\lr{ - a_k e^{-i \omega_k t + \Bk \cdot \Bx} + a_k^\dagger e^{i \omega_k t - i \Bk \cdot \Bx} }
\lr{ - a_j e^{-i \omega_j t + \Bj \cdot \Bx} + a_j^\dagger e^{i \omega_j t - i \Bj \cdot \Bx} }
=
-\inv{2} \int d^3 x \inv{(2\pi)^3}
\int d^3 k d^3 j \frac{k^\mu j^\nu}{\sqrt{\omega_k \omega_j}}
\lr{
  a_k a_j e^{-i (\omega_j + \omega_k) t + (\Bj + \Bk) \cdot \Bx} 
- a_k a_j^\dagger e^{i (\omega_j - \omega_k) t - i (\Bj -\Bk) \cdot \Bx} 
- a_k^\dagger a_j e^{-i (\omega_j -\omega_k) t - (\Bk - \Bj) \cdot \Bx} 
+ a_k^\dagger a_j^\dagger e^{i (\omega_j + \omega_k) t - i (\Bj + \Bk) \cdot \Bx} 
}
=
-\inv{2} 
\int d^3 k d^3 j \frac{k^\mu j^\nu}{\sqrt{\omega_k \omega_j}}
\lr{
  a_k a_j e^{-i (\omega_j + \omega_k) t } \delta^3(\Bj + \Bk)
- a_k a_j^\dagger e^{i (\omega_j - \omega_k) t } \delta^3(\Bj -\Bk)
- a_k^\dagger a_j e^{-i (\omega_j -\omega_k) t } \delta^3 (\Bk - \Bj)
+ a_k^\dagger a_j^\dagger e^{i (\omega_j + \omega_k) t } \delta^3 (\Bj + \Bk)
}.
\end{dmath}

There are two cases here to consider.  The first is \( \nu = 0 \), for which we have

\begin{dmath}\label{eqn:noetherCurrentScalarField:140}
\int d^3 x \Pi^\mu \Pi^0
=
-\inv{2} 
\int d^3 k k^\mu 
\lr{
  a_k a_{-k} e^{-2 i \omega_k t }
- a_k a_k^\dagger 
- a_k^\dagger a_k 
+ a_k^\dagger a_{-k}^\dagger e^{2 i \omega_k t }
}.
\end{dmath}

For \( \nu \ne 0 \), we have

\begin{dmath}\label{eqn:noetherCurrentScalarField:160}
\int d^3 x \Pi^\mu \Pi^\nu
=
-\inv{2} 
\int d^3 k \frac{k^\mu k^\nu}{\omega_k}
\lr{
- a_k a_{-k} e^{- 2 i \omega_k t } 
- a_k a_k^\dagger 
- a_k^\dagger a_k 
- a_k^\dagger a_{-k}^\dagger e^{ 2 i \omega_k t } 
}
=
 \inv{2} 
\int d^3 k \frac{k^\mu k^\nu}{\omega_k}
\lr{
  a_k a_{-k} e^{- 2 i \omega_k t } 
+ a_k a_k^\dagger 
+ a_k^\dagger a_k 
+ a_k^\dagger a_{-k}^\dagger e^{ 2 i \omega_k t } 
}
\end{dmath}

Summarizing, we have 

\begin{subequations}
\label{eqn:noetherCurrentScalarField:260}
\begin{dmath}\label{eqn:noetherCurrentScalarField:300}
\int d^3 x \Pi^0 \Pi^0 
= 
-\inv{2} 
\int d^3 k \omega_k
\lr{
  a_k a_{-k} e^{-2 i \omega_k t }
- a_k a_k^\dagger 
- a_k^\dagger a_k 
+ a_k^\dagger a_{-k}^\dagger e^{2 i \omega_k t }
},
\end{dmath}
\begin{dmath}\label{eqn:noetherCurrentScalarField:280}
\int d^3 x \Pi^n \Pi^0 
= \int d^3 x \Pi^0 \Pi^n
= 
-\inv{2} 
\int d^3 k k^\mu 
\lr{
  a_k a_{-k} e^{-2 i \omega_k t }
- a_k a_k^\dagger 
- a_k^\dagger a_k 
+ a_k^\dagger a_{-k}^\dagger e^{2 i \omega_k t }
},
\end{dmath}
%\begin{dmath}\label{eqn:noetherCurrentScalarField:320}
%\int d^3 x \Pi^n \Pi^n
%=
% \inv{2} 
%\int d^3 k \frac{k^n k^n}{\omega_k}
%\lr{
%  a_k a_{-k} e^{- 2 i \omega_k t } 
%+ a_k a_k^\dagger 
%+ a_k^\dagger a_k 
%+ a_k^\dagger a_{-k}^\dagger e^{ 2 i \omega_k t } 
%},
%\end{dmath}
\begin{dmath}\label{eqn:noetherCurrentScalarField:160}
\int d^3 x \Pi^m \Pi^n
=
\inv{2} 
\int d^3 k \frac{k^m k^n}{\omega_k}
\lr{
  a_k a_{-k} e^{- 2 i \omega_k t } 
+ a_k a_k^\dagger 
+ a_k^\dagger a_k 
+ a_k^\dagger a_{-k}^\dagger e^{ 2 i \omega_k t } 
}.
\end{dmath}
\end{subequations}

For the mass term it was previously found that

\begin{dmath}\label{eqn:noetherCurrentScalarField:180}
\inv{2} \int d^3 x \mu^2 \phi^2
=
\frac{\mu^2}{4}
\int
d^3 k
\inv{ \omega_k }
\lr{
 a_{-k} a_k e^{- 2 i \omega_k t }
+a_{-k}^\dagger a_k^\dagger e^{2 i \omega_k t }
+a_k a_k^\dagger
+a_k^\dagger a_k
}.
\end{dmath}

%Those are all the pieces needed to assemble the (integrated) energy-momentum tensor.

%}
\EndArticle
%\EndNoBibArticle
