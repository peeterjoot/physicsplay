%
% Copyright � 2015 Peeter Joot.  All Rights Reserved.
% Licenced as described in the file LICENSE under the root directory of this GIT repository.
%
\documentclass[]{eliblog}

\usepackage{amsmath}
\usepackage{mathpazo}

%
% shorthand for bold symbols, convenient for vectors and matrices
%
\newcommand{\Ba}[0]{\mathbf{a}}
\newcommand{\Bb}[0]{\mathbf{b}}
\newcommand{\Bc}[0]{\mathbf{c}}
\newcommand{\Bd}[0]{\mathbf{d}}
\newcommand{\Be}[0]{\mathbf{e}}
\newcommand{\Bf}[0]{\mathbf{f}}
\newcommand{\Bg}[0]{\mathbf{g}}
\newcommand{\Bh}[0]{\mathbf{h}}
\newcommand{\Bi}[0]{\mathbf{i}}
\newcommand{\Bj}[0]{\mathbf{j}}
\newcommand{\Bk}[0]{\mathbf{k}}
\newcommand{\Bl}[0]{\mathbf{l}}
\newcommand{\Bm}[0]{\mathbf{m}}
\newcommand{\Bn}[0]{\mathbf{n}}
\newcommand{\Bo}[0]{\mathbf{o}}
\newcommand{\Bp}[0]{\mathbf{p}}
\newcommand{\Bq}[0]{\mathbf{q}}
\newcommand{\Br}[0]{\mathbf{r}}
\newcommand{\Bs}[0]{\mathbf{s}}
\newcommand{\Bt}[0]{\mathbf{t}}
\newcommand{\Bu}[0]{\mathbf{u}}
\newcommand{\Bv}[0]{\mathbf{v}}
\newcommand{\Bw}[0]{\mathbf{w}}
\newcommand{\Bx}[0]{\mathbf{x}}
\newcommand{\By}[0]{\mathbf{y}}
\newcommand{\Bz}[0]{\mathbf{z}}
\newcommand{\BA}[0]{\mathbf{A}}
\newcommand{\BB}[0]{\mathbf{B}}
\newcommand{\BC}[0]{\mathbf{C}}
\newcommand{\BD}[0]{\mathbf{D}}
\newcommand{\BE}[0]{\mathbf{E}}
\newcommand{\BF}[0]{\mathbf{F}}
\newcommand{\BG}[0]{\mathbf{G}}
\newcommand{\BH}[0]{\mathbf{H}}
\newcommand{\BI}[0]{\mathbf{I}}
\newcommand{\BJ}[0]{\mathbf{J}}
\newcommand{\BK}[0]{\mathbf{K}}
\newcommand{\BL}[0]{\mathbf{L}}
\newcommand{\BM}[0]{\mathbf{M}}
\newcommand{\BN}[0]{\mathbf{N}}
\newcommand{\BO}[0]{\mathbf{O}}
\newcommand{\BP}[0]{\mathbf{P}}
\newcommand{\BQ}[0]{\mathbf{Q}}
\newcommand{\BR}[0]{\mathbf{R}}
\newcommand{\BS}[0]{\mathbf{S}}
\newcommand{\BT}[0]{\mathbf{T}}
\newcommand{\BU}[0]{\mathbf{U}}
\newcommand{\BV}[0]{\mathbf{V}}
\newcommand{\BW}[0]{\mathbf{W}}
\newcommand{\BX}[0]{\mathbf{X}}
\newcommand{\BY}[0]{\mathbf{Y}}
\newcommand{\BZ}[0]{\mathbf{Z}}

\newcommand{\Bzero}[0]{\mathbf{0}}
\newcommand{\Btheta}[0]{\boldsymbol{\theta}}
\newcommand{\Btau}[0]{\boldsymbol{\tau}}
\newcommand{\Bomega}[0]{\boldsymbol{\omega}}

%
% shorthand for unit vectors
%
\newcommand{\acap}[0]{\hat{\Ba}}
\newcommand{\bcap}[0]{\hat{\Bb}}
\newcommand{\ccap}[0]{\hat{\Bc}}
\newcommand{\dcap}[0]{\hat{\Bd}}
\newcommand{\ecap}[0]{\hat{\Be}}
\newcommand{\fcap}[0]{\hat{\Bf}}
\newcommand{\gcap}[0]{\hat{\Bg}}
\newcommand{\hcap}[0]{\hat{\Bh}}
\newcommand{\icap}[0]{\hat{\Bi}}
\newcommand{\jcap}[0]{\hat{\Bj}}
\newcommand{\kcap}[0]{\hat{\Bk}}
\newcommand{\lcap}[0]{\hat{\Bl}}
\newcommand{\mcap}[0]{\hat{\Bm}}
\newcommand{\ncap}[0]{\hat{\Bn}}
\newcommand{\ocap}[0]{\hat{\Bo}}
\newcommand{\pcap}[0]{\hat{\Bp}}
\newcommand{\qcap}[0]{\hat{\Bq}}
\newcommand{\rcap}[0]{\hat{\Br}}
\newcommand{\scap}[0]{\hat{\Bs}}
\newcommand{\tcap}[0]{\hat{\Bt}}
\newcommand{\ucap}[0]{\hat{\Bu}}
\newcommand{\vcap}[0]{\hat{\Bv}}
\newcommand{\wcap}[0]{\hat{\Bw}}
\newcommand{\xcap}[0]{\hat{\Bx}}
\newcommand{\ycap}[0]{\hat{\By}}
\newcommand{\zcap}[0]{\hat{\Bz}}
\newcommand{\thetacap}[0]{\hat{\Btheta}}

%
% to write R^n and C^n in a distinguishable fashion.  Perhaps change this
% to the double lined characters upon figuring out how to do so.
%
\newcommand{\C}[1]{$\mathbb{C}^{#1}$}
\newcommand{\R}[1]{$\mathbb{R}^{#1}$}

%
% various generally useful helpers
%

% derivative of #1 wrt. #2:
\newcommand{\D}[2] {\frac {d#2} {d#1}}

\newcommand{\inv}[1]{\frac{1}{#1}}
\newcommand{\cross}[0]{\times}

\newcommand{\abs}[1]{\lvert{#1}\rvert}
\newcommand{\norm}[1]{\lVert{#1}\rVert}
\newcommand{\innerprod}[2]{\langle{#1}, {#2}\rangle}
\newcommand{\dotprod}[2]{{#1} \cdot {#2}}
\newcommand{\bdotprod}[2]{\left({#1} \cdot {#2}\right)}
\newcommand{\crossprod}[2]{{#1} \cross {#2}}
\newcommand{\tripleprod}[3]{\dotprod{\left(\crossprod{#1}{#2}\right)}{#3}}

\DeclareMathOperator{\Proj}{Proj}
\DeclareMathOperator{\Span}{span}
\DeclareMathOperator{\Sgn}{sgn}
\DeclareMathOperator{\Area}{Area}
\DeclareMathOperator{\Volume}{Volume}

%
% A few miscellaneous things specific to this document
%
\newcommand{\crossop}[1]{\crossprod{#1}{}}

% R2 vector.
\newcommand{\VectorTwo}[2]{
\begin{bmatrix}
 {#1} \\
 {#2}
\end{bmatrix}
}

\newcommand{\VectorN}[1]{
\begin{bmatrix}
{#1}_1 \\
{#1}_2 \\
\vdots \\
{#1}_N \\
\end{bmatrix}
}

\newcommand{\DETuvij}[4]{
\begin{vmatrix}
 {#1}_{#3} & {#1}_{#4} \\
 {#2}_{#3} & {#2}_{#4}
\end{vmatrix}
}

\newcommand{\DETuvwijk}[6]{
\begin{vmatrix}
 {#1}_{#4} & {#1}_{#5} & {#1}_{#6} \\
 {#2}_{#4} & {#2}_{#5} & {#2}_{#6} \\
 {#3}_{#4} & {#3}_{#5} & {#3}_{#6}
\end{vmatrix}
}

\newcommand{\DETuvwxijkl}[8]{
\begin{vmatrix}
 {#1}_{#5} & {#1}_{#6} & {#1}_{#7} & {#1}_{#8} \\
 {#2}_{#5} & {#2}_{#6} & {#2}_{#7} & {#2}_{#8} \\
 {#3}_{#5} & {#3}_{#6} & {#3}_{#7} & {#3}_{#8} \\
 {#4}_{#5} & {#4}_{#6} & {#4}_{#7} & {#4}_{#8} \\
\end{vmatrix}
}

%\newcommand{\DETuvwxyijklm}[10]{
%\begin{vmatrix}
% {#1}_{#6} & {#1}_{#7} & {#1}_{#8} & {#1}_{#9} & {#1}_{#10} \\
% {#2}_{#6} & {#2}_{#7} & {#2}_{#8} & {#2}_{#9} & {#2}_{#10} \\
% {#3}_{#6} & {#3}_{#7} & {#3}_{#8} & {#3}_{#9} & {#3}_{#10} \\
% {#4}_{#6} & {#4}_{#7} & {#4}_{#8} & {#4}_{#9} & {#4}_{#10} \\
% {#5}_{#6} & {#5}_{#7} & {#5}_{#8} & {#5}_{#9} & {#5}_{#10}
%\end{vmatrix}
%}

% R3 vector.
\newcommand{\VectorThree}[3]{
\begin{bmatrix}
 {#1} \\
 {#2} \\
 {#3}
\end{bmatrix}
}



\author{Peeter Joot}
\email{peeter.joot@gmail.com}

%\documentclass[]{eliblogwidescreen}

\usepackage{amsmath}
\usepackage{mathpazo}

%
% shorthand for bold symbols, convenient for vectors and matrices
%
\newcommand{\Ba}[0]{\mathbf{a}}
\newcommand{\Bb}[0]{\mathbf{b}}
\newcommand{\Bc}[0]{\mathbf{c}}
\newcommand{\Bd}[0]{\mathbf{d}}
\newcommand{\Be}[0]{\mathbf{e}}
\newcommand{\Bf}[0]{\mathbf{f}}
\newcommand{\Bg}[0]{\mathbf{g}}
\newcommand{\Bh}[0]{\mathbf{h}}
\newcommand{\Bi}[0]{\mathbf{i}}
\newcommand{\Bj}[0]{\mathbf{j}}
\newcommand{\Bk}[0]{\mathbf{k}}
\newcommand{\Bl}[0]{\mathbf{l}}
\newcommand{\Bm}[0]{\mathbf{m}}
\newcommand{\Bn}[0]{\mathbf{n}}
\newcommand{\Bo}[0]{\mathbf{o}}
\newcommand{\Bp}[0]{\mathbf{p}}
\newcommand{\Bq}[0]{\mathbf{q}}
\newcommand{\Br}[0]{\mathbf{r}}
\newcommand{\Bs}[0]{\mathbf{s}}
\newcommand{\Bt}[0]{\mathbf{t}}
\newcommand{\Bu}[0]{\mathbf{u}}
\newcommand{\Bv}[0]{\mathbf{v}}
\newcommand{\Bw}[0]{\mathbf{w}}
\newcommand{\Bx}[0]{\mathbf{x}}
\newcommand{\By}[0]{\mathbf{y}}
\newcommand{\Bz}[0]{\mathbf{z}}
\newcommand{\BA}[0]{\mathbf{A}}
\newcommand{\BB}[0]{\mathbf{B}}
\newcommand{\BC}[0]{\mathbf{C}}
\newcommand{\BD}[0]{\mathbf{D}}
\newcommand{\BE}[0]{\mathbf{E}}
\newcommand{\BF}[0]{\mathbf{F}}
\newcommand{\BG}[0]{\mathbf{G}}
\newcommand{\BH}[0]{\mathbf{H}}
\newcommand{\BI}[0]{\mathbf{I}}
\newcommand{\BJ}[0]{\mathbf{J}}
\newcommand{\BK}[0]{\mathbf{K}}
\newcommand{\BL}[0]{\mathbf{L}}
\newcommand{\BM}[0]{\mathbf{M}}
\newcommand{\BN}[0]{\mathbf{N}}
\newcommand{\BO}[0]{\mathbf{O}}
\newcommand{\BP}[0]{\mathbf{P}}
\newcommand{\BQ}[0]{\mathbf{Q}}
\newcommand{\BR}[0]{\mathbf{R}}
\newcommand{\BS}[0]{\mathbf{S}}
\newcommand{\BT}[0]{\mathbf{T}}
\newcommand{\BU}[0]{\mathbf{U}}
\newcommand{\BV}[0]{\mathbf{V}}
\newcommand{\BW}[0]{\mathbf{W}}
\newcommand{\BX}[0]{\mathbf{X}}
\newcommand{\BY}[0]{\mathbf{Y}}
\newcommand{\BZ}[0]{\mathbf{Z}}

\newcommand{\Bzero}[0]{\mathbf{0}}
\newcommand{\Btheta}[0]{\boldsymbol{\theta}}
\newcommand{\Btau}[0]{\boldsymbol{\tau}}
\newcommand{\Bomega}[0]{\boldsymbol{\omega}}

%
% shorthand for unit vectors
%
\newcommand{\acap}[0]{\hat{\Ba}}
\newcommand{\bcap}[0]{\hat{\Bb}}
\newcommand{\ccap}[0]{\hat{\Bc}}
\newcommand{\dcap}[0]{\hat{\Bd}}
\newcommand{\ecap}[0]{\hat{\Be}}
\newcommand{\fcap}[0]{\hat{\Bf}}
\newcommand{\gcap}[0]{\hat{\Bg}}
\newcommand{\hcap}[0]{\hat{\Bh}}
\newcommand{\icap}[0]{\hat{\Bi}}
\newcommand{\jcap}[0]{\hat{\Bj}}
\newcommand{\kcap}[0]{\hat{\Bk}}
\newcommand{\lcap}[0]{\hat{\Bl}}
\newcommand{\mcap}[0]{\hat{\Bm}}
\newcommand{\ncap}[0]{\hat{\Bn}}
\newcommand{\ocap}[0]{\hat{\Bo}}
\newcommand{\pcap}[0]{\hat{\Bp}}
\newcommand{\qcap}[0]{\hat{\Bq}}
\newcommand{\rcap}[0]{\hat{\Br}}
\newcommand{\scap}[0]{\hat{\Bs}}
\newcommand{\tcap}[0]{\hat{\Bt}}
\newcommand{\ucap}[0]{\hat{\Bu}}
\newcommand{\vcap}[0]{\hat{\Bv}}
\newcommand{\wcap}[0]{\hat{\Bw}}
\newcommand{\xcap}[0]{\hat{\Bx}}
\newcommand{\ycap}[0]{\hat{\By}}
\newcommand{\zcap}[0]{\hat{\Bz}}
\newcommand{\thetacap}[0]{\hat{\Btheta}}

%
% to write R^n and C^n in a distinguishable fashion.  Perhaps change this
% to the double lined characters upon figuring out how to do so.
%
\newcommand{\C}[1]{$\mathbb{C}^{#1}$}
\newcommand{\R}[1]{$\mathbb{R}^{#1}$}

%
% various generally useful helpers
%

% derivative of #1 wrt. #2:
\newcommand{\D}[2] {\frac {d#2} {d#1}}

\newcommand{\inv}[1]{\frac{1}{#1}}
\newcommand{\cross}[0]{\times}

\newcommand{\abs}[1]{\lvert{#1}\rvert}
\newcommand{\norm}[1]{\lVert{#1}\rVert}
\newcommand{\innerprod}[2]{\langle{#1}, {#2}\rangle}
\newcommand{\dotprod}[2]{{#1} \cdot {#2}}
\newcommand{\bdotprod}[2]{\left({#1} \cdot {#2}\right)}
\newcommand{\crossprod}[2]{{#1} \cross {#2}}
\newcommand{\tripleprod}[3]{\dotprod{\left(\crossprod{#1}{#2}\right)}{#3}}

\DeclareMathOperator{\Proj}{Proj}
\DeclareMathOperator{\Span}{span}
\DeclareMathOperator{\Sgn}{sgn}
\DeclareMathOperator{\Area}{Area}
\DeclareMathOperator{\Volume}{Volume}

%
% A few miscellaneous things specific to this document
%
\newcommand{\crossop}[1]{\crossprod{#1}{}}

% R2 vector.
\newcommand{\VectorTwo}[2]{
\begin{bmatrix}
 {#1} \\
 {#2}
\end{bmatrix}
}

\newcommand{\VectorN}[1]{
\begin{bmatrix}
{#1}_1 \\
{#1}_2 \\
\vdots \\
{#1}_N \\
\end{bmatrix}
}

\newcommand{\DETuvij}[4]{
\begin{vmatrix}
 {#1}_{#3} & {#1}_{#4} \\
 {#2}_{#3} & {#2}_{#4}
\end{vmatrix}
}

\newcommand{\DETuvwijk}[6]{
\begin{vmatrix}
 {#1}_{#4} & {#1}_{#5} & {#1}_{#6} \\
 {#2}_{#4} & {#2}_{#5} & {#2}_{#6} \\
 {#3}_{#4} & {#3}_{#5} & {#3}_{#6}
\end{vmatrix}
}

\newcommand{\DETuvwxijkl}[8]{
\begin{vmatrix}
 {#1}_{#5} & {#1}_{#6} & {#1}_{#7} & {#1}_{#8} \\
 {#2}_{#5} & {#2}_{#6} & {#2}_{#7} & {#2}_{#8} \\
 {#3}_{#5} & {#3}_{#6} & {#3}_{#7} & {#3}_{#8} \\
 {#4}_{#5} & {#4}_{#6} & {#4}_{#7} & {#4}_{#8} \\
\end{vmatrix}
}

%\newcommand{\DETuvwxyijklm}[10]{
%\begin{vmatrix}
% {#1}_{#6} & {#1}_{#7} & {#1}_{#8} & {#1}_{#9} & {#1}_{#10} \\
% {#2}_{#6} & {#2}_{#7} & {#2}_{#8} & {#2}_{#9} & {#2}_{#10} \\
% {#3}_{#6} & {#3}_{#7} & {#3}_{#8} & {#3}_{#9} & {#3}_{#10} \\
% {#4}_{#6} & {#4}_{#7} & {#4}_{#8} & {#4}_{#9} & {#4}_{#10} \\
% {#5}_{#6} & {#5}_{#7} & {#5}_{#8} & {#5}_{#9} & {#5}_{#10}
%\end{vmatrix}
%}

% R3 vector.
\newcommand{\VectorThree}[3]{
\begin{bmatrix}
 {#1} \\
 {#2} \\
 {#3}
\end{bmatrix}
}



\author{Peeter Joot}
\email{peeter.joot@gmail.com}


\chapter{PHY454H1S\\Continuum Mechanics.  Lecture 9: Newtonian fluids.  Mass conservation.  Constituative relation.  Incompressible fluids.  Taught by Prof. K. Das.}
\label{chap:continuumL9}
%\useCCL
\blogpage{http://sites.google.com/site/peeterjoot2/math2012/continuumL9.pdf}
\date{Feb 8, 2012}
\revisionInfo{continuumL9.tex}

\beginArtWithToc
%\beginArtNoToc

\section{FIXME:}

FIXME: Reading: \S XX from \cite{acheson1990elementary}?

\section{Disclaimer.}

Peeter's lecture notes from class.  May not be entirely coherent.

\section{Review: Relative motion near a point in a fluid}

Referring to figure (\ref{fig:continuumL9:continuumL9fig1})
\begin{figure}[htp]
   \centering
   \includegraphics[totalheight=0.2\textheight]{continuumL9fig1}
   \caption{velocity displacements at a fluid point.}\label{fig:continuumL9:continuumL9fig1}
\end{figure}

we write

\begin{equation}\label{eqn:continuumL9:10}
d\Bx' = d\Bx + d\Bu \delta t
\end{equation}

or in coordinate form

\begin{equation}\label{eqn:continuumL9:30}
\begin{aligned}
dx_i 
&= dx_i + du_i \delta t \\
&= dx_i + \PD{x_j}{u_i} dx_j \delta t 
\end{aligned}
\end{equation}

We can now split the components of the gradient of $u_i$ into symmetric and antisymmetric parts in the normal way

\begin{equation}\label{eqn:continuumL9:50}
\begin{aligned}
\PD{x_j}{u_i}
&= 
\inv{2} \left( 
\PD{x_j}{u_i}
+\PD{x_i}{u_j}
\right)
+
\inv{2} \left( 
\PD{x_j}{u_i}
-\PD{x_i}{u_j}
\right) \\
&\equiv e_{ij} + \omega_{ij}.
\end{aligned}
\end{equation}

\subsection{The antisymmetric term (name?)}

With

\begin{equation}\label{eqn:continuumL9:70}
\Bomega = \spacegrad \cross \Bu,
\end{equation}

we introduce the dual vector 

\begin{equation}\label{eqn:continuumL9:90}
\BOmega = \Omega_k \Be_k = \inv{2} \Bomega
\end{equation}

defined according to

\begin{align}\label{eqn:continuumL9:110}
\Omega_1 &= \inv{2} \omega_{32} = \inv{2} \omega_1 \\
\Omega_2 &= \inv{2} \omega_{13} = \inv{2} \omega_2 \\
\Omega_3 &= \inv{2} \omega_{21} = \inv{2} \omega_3 
\end{align}

With 
\begin{equation}\label{eqn:continuumL9:n}
\omega_{ij}
= \epsilon_{ijk} \partial_i u_j
\end{equation}

we can write
\begin{equation}\label{eqn:continuumL9:590}
\Omega_k = -\inv{2} \epsilon_{ijk} \partial_i u_j.
\end{equation}

In matrix form this becomes

\begin{equation}\label{eqn:continuumL9:130}
\omega_{ij} = 
\begin{bmatrix}
0 & -\Omega_3 & \Omega_2 \\
\Omega_3 & 0 & -\Omega_1  \\
-\Omega_2 & \Omega_1 & 0
\end{bmatrix}.
\end{equation}

For the special case $e_{ij} = 0$, our displacement equation in vector form becomes

\begin{equation}\label{eqn:continuumL9:150}
d\Bx' = d\Bx + \BOmega \cross d\Bx \delta t.
\end{equation}

Let's do a quick verification that this is all kosher.

\begin{align*}
(\BOmega \cross d\Bx)_i
&=
\Omega_r dx_s \epsilon_{rsi} \\
&=
\left(-\inv{2} \epsilon_{abr} \partial_a u_b \right) dx_s \epsilon_{rsi} \\
&=
-\inv{2} \partial_a u_b dx_s \delta^{[ab]}_{si} \\
&=
-\inv{2} (
\partial_s u_i
-\partial_i u_s
) dx_s  \\
&=
\inv{2} \left(
\PD{x_i}{u_s}
-\PD{x_s}{u_i}
\right) dx_s  \\
&=
\inv{2} \left(
\PD{x_i}{u_j}
-\PD{x_j}{u_i}
\right) dx_j  \\
&=
\omega_{ij} dx_j.
\end{align*}

All's good in the world of signs and indexes.

\subsection{The symmetric term (strain tensor).}

Now let's look at the symmetric term.  With the initial volume

\begin{equation}\label{eqn:continuumL9:170}
dV = dx_1 dx_2 dx_3,
\end{equation}

and the final volume written assuming that we are working in our principle strain basis, we have (very much like the solids case)

\begin{align*}
dV' 
&= dx_1' dx_2' dx_3' \\
&= 
(1 + e_{11} \delta t) dx_1
+(1 + e_{22} \delta t) dx_2
+(1 + e_{33} \delta t) dx_3
\\
&=
(1 + (e_{11} + e_{22} + e_{33}) \delta t) dx_1 dx_2 dx_3 + O((\delta t)^2) \\
&=
\left(1 + 
\left(
\PD{x_1}{u_1}
+\PD{x_2}{u_2}
+\PD{x_3}{u_3}
\right)
\delta t \right) dV \\
&=
\left(
1 + (\spacegrad \cdot \Bu) 
\delta t
\right) dV \\
\end{align*}

So much like we expressed the relative change of volume in solids, we now can express the relative change of volume per unit time as

\begin{equation}\label{eqn:continuumL9:190}
\frac{dV' - dV}{dV \delta t} = \spacegrad \cdot \Bu,
\end{equation}

or

\begin{equation}\label{eqn:continuumL9:210}
\frac{\delta(dV)}{dV \delta t} = \spacegrad \cdot \Bu,
\end{equation}

We identify the divergence of the displacement as the relative change in volume per unit time.

\section{Newtonian Fluids.}

\begin{definition}
\emph{(Newtonian Fluids)}
\label{dfn:continuumL9:230}
A fluid for which the rate of strain tensor is linearly related to stress tensor.
\end{definition}

For such a fluid, the constituitive relation takes the form

\begin{equation}\label{eqn:continuumL9:250}
\boxed{
\sigma_{ij} = - p \delta_{ij} + 2 \mu e_{ij},
}
\end{equation}

where $p$ is called the isotropic pressure, and $\mu$ is the viscosity of the fluid.

For comparison, in solids we had

\begin{equation}\label{eqn:continuumL9:270}
\sigma_{ij} = \lambda e_{kk} \delta_{ij} + 2 \mu e_{ij}
\end{equation}

While we are allowing for rotation in the fluids ($\omega_{ij}$) that we did not consider for solids, we now impose a requirement that the strain tensor trace is constant.  What is the physical justification for this?

\subsection{Dimensions}

\begin{equation}\label{eqn:continuumL9:290}
[\mu] = \frac{\text{M}}{\text{L}\text{T}}.
\end{equation}

Some examples

\begin{itemize}
\item $\mu_{\text{air}} = 1.8 \times 10^{-5} \frac{\text{kg}}{\text{m s}}$
\item $\mu_{\text{water}} = 1.1 \times 10^{-3} \frac{\text{kg}}{\text{m s}}$
\item $\mu_{\text{glycerine}} = 2.3 \frac{\text{kg}}{\text{m s}}$
\end{itemize}

\section{Conservation of mass in fluid.}

Referring to figure (\ref{fig:continuumL9:continuumL9fig2})
\begin{figure}[htp]
   \centering
   \includegraphics[totalheight=0.2\textheight]{continuumL9fig2}
   \caption{FIXME: continuumL9fig2}\label{fig:continuumL9:continuumL9fig2}
\end{figure}

we have a flow rate

\begin{equation}\label{eqn:continuumL9:310}
\rho \Bu \delta t ds
\end{equation}

or 
\begin{equation}\label{eqn:continuumL9:330}
\rho \Bu ds,
\end{equation}

per unit time.

Velocity of fluid particle is $\Bu$

\begin{equation}\label{eqn:continuumL9:350}
\oint \rho \Bu \cdot d\Bs,
\end{equation}

we must have

\begin{equation}\label{eqn:continuumL9:n}
\PD{t}{} \int \rho dV 
=
-\oint \rho \Bu \cdot d\Bs.
\end{equation}

\begin{equation}\label{eqn:continuumL9:370}
dm = \rho dV
\end{equation}

\begin{equation}\label{eqn:continuumL9:390}
\frac{dm}{dt} = \frac{d}{dt} (\rho dV)
\end{equation}

\begin{itemize}
\item 
positive if fluid is coming in.
\item 
negative if fluid is going out.
\end{itemize}

By Green's theorem

\begin{equation}\label{eqn:continuumL9:410}
\oint \BA \cdot d\Bs = \int_V (\spacegrad \cdot \Bu) dV,
\end{equation}

so we have

\begin{equation}\label{eqn:continuumL9:430}
-\oint \rho \Bu \cdot d\Bs = -\int \spacegrad \cdot (\rho \Bu ) dV,
\end{equation}

and must have

\begin{equation}\label{eqn:continuumL9:450}
\int \left( \PD{t}{\rho} + \spacegrad \cdot (\rho \Bu) \right) dV = 0.
\end{equation}

The total mass has to be conserved.  The mass that is leaving the volume per unit time must move throught the surface of the volume in that time.  In differential form this is

\begin{equation}\label{eqn:continuumL9:470}
\boxed{
\PD{t}{\rho} + \spacegrad \cdot (\rho \Bu) = 0.
}
\end{equation}

Operting by chain rule we can write this as

\begin{equation}\label{eqn:continuumL9:490}
\PD{t}{\rho} + \Bu \cdot \spacegrad \rho = - \rho \spacegrad \cdot \Bu.
\end{equation}

To make sense of this, observe that we have for $f = f(x, y, z, t)$

\begin{align*}
\delta f 
&= \lim_{\delta t \rightarrow 0} \frac{\delta f}{\delta t} dt \\
&=
\PD{x}{f} \frac{\delta x}{\delta t}
+\PD{y}{f} \frac{\delta y}{\delta t}
+\PD{z}{f} \frac{\delta z}{\delta t}
+ \PD{t}{f} \\
&=
(\spacegrad f) \cdot \Bu + \PD{t}{f}
\end{align*}

so we have

\begin{equation}\label{eqn:continuumL9:510}
\PD{t}{\rho} + \Bu \cdot \spacegrad \rho = \frac{d\rho}{dt}
\end{equation}

or
\begin{equation}\label{eqn:continuumL9:530}
\frac{d\rho}{dt} = - \rho \spacegrad \cdot \Bu.
\end{equation}

\subsection{Incompressible fluid}

When the density doesn't change note that we have

\begin{equation}\label{eqn:continuumL9:550}
\frac{d\rho}{dt} = 0
\end{equation}

which then implies

\begin{equation}\label{eqn:continuumL9:570}
\boxed{
\spacegrad \cdot \Bu = 0,
}
\end{equation}

at all points in the fluid.


\EndArticle
