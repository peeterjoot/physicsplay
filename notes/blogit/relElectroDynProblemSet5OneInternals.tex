\section{Problem 1.  Sinusoidal current density on an infinite flat conducting sheet.}
\subsection{Statement}

An infinitely thin flat conducting surface lying in the $x-z$ plane carries a surface current density:

\begin{equation}\label{eqn:relElectroDynProblemSet5:n}
\Bkappa - \Be_z \theta(t) \kappa_0 \sin\omega t
\end{equation}

Here $\Be_z$ is a unit vector in the $z$ direction, $\kappa_0$ is the peak value of the current density, and $\theta(t)$ is the theta function: $\theta(t < 0) - 0, \theta(t > 0) = 1$.

\begin{enumerate}
\item Write down the equations determining the electromagnetic potentials.  Specify which gauge you choose to work in.
\item Find the electromagnetic potentials outside the plane.
\item Find the electric and magnetic fields outside the plane.
\item Give a physical interpetation of the results of the previosu section.  Do they agree with your qualitative expectations?
\item Find the direction and magnitude of the energy flux outside the plane.
\item Consier a point at some distance from the plane.  Sketch the intensity of the electromagnetic field near this point as a function of time.  Explain physically.
\item Consider now a point near the plane.  Are the electric and magnetic fields you found continuous across the conducting plane?  Explain.
\end{enumerate}

\subsection{1. Solution}

TODO. 

\subsection{2. Solution}

TODO. 

\subsection{3. Solution}

TODO. 

\subsection{4. Solution}

TODO. 

\subsection{5. Solution}

TODO. 

\subsection{6. Solution}

TODO. 

\subsection{7. Solution}

TODO. 

