%
% Copyright � 2015 Peeter Joot.  All Rights Reserved.
% Licenced as described in the file LICENSE under the root directory of this GIT repository.
%
\newcommand{\authorname}{Peeter Joot}
\newcommand{\email}{peeterjoot@protonmail.com}
\newcommand{\basename}{FIXMEbasenameUndefined}
\newcommand{\dirname}{notes/FIXMEdirnameUndefined/}

\renewcommand{\basename}{rodbalancing}
\renewcommand{\dirname}{notes/phy1520/}
%\newcommand{\dateintitle}{}
%\newcommand{\keywords}{}

\newcommand{\authorname}{Peeter Joot}
\newcommand{\onlineurl}{http://sites.google.com/site/peeterjoot2/math2013/\basename.pdf}
\newcommand{\sourcepath}{\dirname\basename.tex}
\newcommand{\generatetitle}[1]{\chapter{#1}}

\newcommand{\vcsinfo}{%
\section*{}
\noindent{\color{DarkOliveGreen}{\rule{\linewidth}{0.1mm}}}
\paragraph{Document version}
%\paragraph{\color{Maroon}{Document version}}
{
\small
\begin{itemize}
\item Available online at:\\ 
\href{\onlineurl}{\onlineurl}
\item Git Repository: \input{./.revinfo/gitRepo.tex}
\item Source: \sourcepath
\item last commit: \input{./.revinfo/gitCommitString.tex}
\item commit date: \input{./.revinfo/gitCommitDate.tex}
\end{itemize}
}
}

%\PassOptionsToPackage{dvipsnames,svgnames}{xcolor}
\PassOptionsToPackage{square,numbers}{natbib}
\documentclass{scrreprt}

\usepackage[left=2cm,right=2cm]{geometry}
\usepackage[svgnames]{xcolor}
\usepackage{peeters_layout}

\usepackage{natbib}

\usepackage[
colorlinks=true,
bookmarks=false,
pdfauthor={\authorname, \email},
backref 
]{hyperref}

% http://tex.stackexchange.com/questions/75773/how-to-reference-problems-by-the-text-label-in-an-exercise-envioronment
\usepackage[english]{cleveref}
\crefname{Exercise}{exercise}{exercises}
\Crefname{Exercise}{Exercise}{Exercises}

\RequirePackage{titlesec}
\RequirePackage{ifthen}

% http://stackoverflow.com/questions/4932910/date-in-the-tabular-environment
\makeatletter
\let\insertdate\@date
\makeatother

\titleformat{\chapter}[display]
{\bfseries\Large}
{\color{DarkSlateGrey}\filleft \authorname
\ifthenelse{\isundefined{\studentnumber}}{}{\\ \studentnumber}
\ifthenelse{\isundefined{\email}}{}{\\ \email}
\ifthenelse{\isundefined{\dateintitle}}{}{\\ \insertdate}
%\ifthenelse{\isundefined{\coursename}}{}{\\ \coursename} % put in title instead.
}
{4ex}
{\color{DarkOliveGreen}{\titlerule}\color{Maroon}
\vspace{2ex}%
\filright}
[\vspace{2ex}%
\color{DarkOliveGreen}\titlerule
]

\newcommand{\beginArtWithToc}[0]{\begin{document}\tableofcontents}
\newcommand{\beginArtNoToc}[0]{\begin{document}}
\newcommand{\EndNoBibArticle}[0]{\end{document}}
\newcommand{\EndArticle}[0]{\bibliography{Bibliography}\bibliographystyle{plainnat}\end{document}}

% 
%\newcommand{\citep}[1]{\cite{#1}}

\colorSectionsForArticle



\usepackage{peeters_layout_exercise}
\usepackage{peeters_braket}
\usepackage{peeters_figures}
\usepackage{siunitx}

\beginArtNoToc

\generatetitle{Rod balancing stability vs uncertainty principle}
%\chapter{Rod balancing stability vs uncertainty principle}
%\label{chap:rodbalancing}
%\section{Motivation}
%\section{Guts}

In pr 1.x of \citep{sakurai2014modern} we are asked for the time before a very thin rod balancing on its tip will fall if the rod stability is constrained by the uncertainty principle.

Suppose the rod has mass \( M \) and length \( L \).  The mass of a small segment of the rod at height \( y \) is \( dM = \rho A dy \), and has potential energy

\begin{dmath}\label{eqn:rodbalancing:20}
dE
= dM g y 
= \rho A y dy.
\end{dmath}

The total potential energy of the rod is

\begin{dmath}\label{eqn:rodbalancing:40}
E 
= g \int_0^L \rho A y dy
= g \rho A \frac{L^2}{2}
= g \frac{M}{L} \frac{L^2}{2}
= \frac{g M L}{2}.
\end{dmath}

Using \( \Delta E \Delta t \ge \frac{\Hbar}{2} \), we can say that the time before the rod moves would be approximately

\begin{dmath}\label{eqn:rodbalancing:60}
\Delta t 
= \frac{\Hbar}{2} \frac{2}{g M L}
= \frac{\Hbar}{g M L}.
\end{dmath}

Alternately, given a circular area profile, diameter \( D \), density \( \rho \) and length \( L \), this time is

\begin{dmath}\label{eqn:rodbalancing:n}
\Delta 
= 
\frac{4 \Hbar}{g \rho \pi D^2 L^2 }.
\end{dmath}

For a rod (chopstick) of mass \( 10 \si{g} \), length \( 9.5" = 0.24 m \), \( g = 9.8 \si{m/s^2} \) and \( \Hbar = 10^{-34} \si{m^2 kg/s} \), this is an infinitesimal time

\begin{dmath}\label{eqn:rodbalancing:n}
\Delta t = 10^{-32} \si{s}
\end{dmath}

For a fragment of human hair with density \( \rho = 1300 \si{kg/m^3} \), diameter \( D = 6 \times 10^{-5} m \), and length \( 0.2 \si{mm} = 2 \times 10^{-4} \si{m} \), this time is increased to

\begin{dmath}\label{eqn:rodbalancing:n}
\Delta t = 10^{-21} \si{s},
\end{dmath}

which is still tiny.  There's a few options, one of which is that the uncertainty principle alone cannot be used in isolation to determine the stability of such a system, and another is that this crude energy-time uncertainty is not specific enough.  With the estimate above, clearly the scales much be reduced drastically to gain larger stability times.

\EndArticle
%\EndNoBibArticle
