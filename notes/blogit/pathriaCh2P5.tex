%
% Copyright � 2013 Peeter Joot.  All Rights Reserved.
% Licenced as described in the file LICENSE under the root directory of this GIT repository.
%
% pick one:
%\newcommand{\authorname}{Peeter Joot}
\newcommand{\email}{peeter.joot@utoronto.ca}
\newcommand{\studentnumber}{920798560}
\newcommand{\basename}{FIXMEbasenameUndefined}
\newcommand{\dirname}{notes/FIXMEdirnameUndefined/}

\newcommand{\authorname}{Peeter Joot}
\newcommand{\email}{peeterjoot@protonmail.com}
\newcommand{\basename}{FIXMEbasenameUndefined}
\newcommand{\dirname}{notes/FIXMEdirnameUndefined/}

\renewcommand{\basename}{pathriaCh2P5.tex}
\renewcommand{\dirname}{notes/phy452/}
%\newcommand{\dateintitle}{}
\newcommand{\keywords}{Statistical mechanics, PHY452H1S, phase space, one dimensional well, WKB}

\newcommand{\authorname}{Peeter Joot}
\newcommand{\onlineurl}{http://sites.google.com/site/peeterjoot2/math2013/\basename.pdf}
\newcommand{\sourcepath}{\dirname\basename.tex}
\newcommand{\generatetitle}[1]{\chapter{#1}}

\newcommand{\vcsinfo}{%
\section*{}
\noindent{\color{DarkOliveGreen}{\rule{\linewidth}{0.1mm}}}
\paragraph{Document version}
%\paragraph{\color{Maroon}{Document version}}
{
\small
\begin{itemize}
\item Available online at:\\ 
\href{\onlineurl}{\onlineurl}
\item Git Repository: \input{./.revinfo/gitRepo.tex}
\item Source: \sourcepath
\item last commit: \input{./.revinfo/gitCommitString.tex}
\item commit date: \input{./.revinfo/gitCommitDate.tex}
\end{itemize}
}
}

%\PassOptionsToPackage{dvipsnames,svgnames}{xcolor}
\PassOptionsToPackage{square,numbers}{natbib}
\documentclass{scrreprt}

\usepackage[left=2cm,right=2cm]{geometry}
\usepackage[svgnames]{xcolor}
\usepackage{peeters_layout}

\usepackage{natbib}

\usepackage[
colorlinks=true,
bookmarks=false,
pdfauthor={\authorname, \email},
backref 
]{hyperref}

% http://tex.stackexchange.com/questions/75773/how-to-reference-problems-by-the-text-label-in-an-exercise-envioronment
\usepackage[english]{cleveref}
\crefname{Exercise}{exercise}{exercises}
\Crefname{Exercise}{Exercise}{Exercises}

\RequirePackage{titlesec}
\RequirePackage{ifthen}

% http://stackoverflow.com/questions/4932910/date-in-the-tabular-environment
\makeatletter
\let\insertdate\@date
\makeatother

\titleformat{\chapter}[display]
{\bfseries\Large}
{\color{DarkSlateGrey}\filleft \authorname
\ifthenelse{\isundefined{\studentnumber}}{}{\\ \studentnumber}
\ifthenelse{\isundefined{\email}}{}{\\ \email}
\ifthenelse{\isundefined{\dateintitle}}{}{\\ \insertdate}
%\ifthenelse{\isundefined{\coursename}}{}{\\ \coursename} % put in title instead.
}
{4ex}
{\color{DarkOliveGreen}{\titlerule}\color{Maroon}
\vspace{2ex}%
\filright}
[\vspace{2ex}%
\color{DarkOliveGreen}\titlerule
]

\newcommand{\beginArtWithToc}[0]{\begin{document}\tableofcontents}
\newcommand{\beginArtNoToc}[0]{\begin{document}}
\newcommand{\EndNoBibArticle}[0]{\end{document}}
\newcommand{\EndArticle}[0]{\bibliography{Bibliography}\bibliographystyle{plainnat}\end{document}}

% 
%\newcommand{\citep}[1]{\cite{#1}}

\colorSectionsForArticle



\beginArtNoToc

\generatetitle{One dimensional well problem from Pathria chapter II}
%\chapter{One dimensional well problem from Pathria chapter II}
%\label{chap:pathriaCh2P5.tex}

I was attempting \citep{pathriastatistical} problem 2.5 on the commute today, and meant to ask you about it after class, but missed you.

The problem asks to show that 

\begin{equation}\label{eqn:pathriaCh2P5:20}
\oint p dq = \lr{ n + \inv{2} } h,
\end{equation}

provided the particle's potential is such that 

\begin{equation}\label{eqn:pathriaCh2P5:40}
m \hbar \Abs{ \frac{dV}{dq} } \ll \lr{ m ( E - V ) }^{3/2}.
\end{equation}

I was able to show that \eqref{eqn:pathriaCh2P5:40} is equivalent to the WKB condition

\begin{equation}\label{eqn:pathriaCh2P5:80}
\frac{k'}{k^2} \ll 1,
\end{equation}

for a solution of the form

\begin{subequations}
\begin{equation}\label{eqn:pathriaCh2P5:60}
k^2(q) = 2 m (E - V(q))/\hbar^2
\end{equation}
\begin{equation}\label{eqn:pathriaCh2P5:100}
\psi(q) = \inv{\sqrt{k}} e^{\pm i \int k(q) dq}.
\end{equation}
\end{subequations}

However, I'm not sure what to make of the condition on the allowed values of the momentum \eqref{eqn:pathriaCh2P5:20}.  In a classical context, this $\int p dq$ is the area contained by the phase space trajectory.  However, for this quantum problem $p$ and $q$ are operators.  What does this integral even mean?

Can you point me in the right direction.

\EndArticle
%\EndNoBibArticle
