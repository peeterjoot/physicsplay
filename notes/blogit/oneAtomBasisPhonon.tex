%
% Copyright � 2013 Peeter Joot.  All Rights Reserved.
% Licenced as described in the file LICENSE under the root directory of this GIT repository.
%
\newcommand{\authorname}{Peeter Joot}
\newcommand{\email}{peeterjoot@protonmail.com}
\newcommand{\basename}{FIXMEbasenameUndefined}
\newcommand{\dirname}{notes/FIXMEdirnameUndefined/}

\renewcommand{\basename}{oneAtomBasisPhonon}
\renewcommand{\dirname}{notes/phy487/}
\newcommand{\keywords}{Condensed matter physics, PHY487H1F, phonon, one atom basis, angular frequency, harmonic oscillator}
\newcommand{\authorname}{Peeter Joot}
\newcommand{\onlineurl}{http://sites.google.com/site/peeterjoot2/math2013/\basename.pdf}
\newcommand{\sourcepath}{\dirname\basename.tex}
\newcommand{\generatetitle}[1]{\chapter{#1}}

\newcommand{\vcsinfo}{%
\section*{}
\noindent{\color{DarkOliveGreen}{\rule{\linewidth}{0.1mm}}}
\paragraph{Document version}
%\paragraph{\color{Maroon}{Document version}}
{
\small
\begin{itemize}
\item Available online at:\\ 
\href{\onlineurl}{\onlineurl}
\item Git Repository: \input{./.revinfo/gitRepo.tex}
\item Source: \sourcepath
\item last commit: \input{./.revinfo/gitCommitString.tex}
\item commit date: \input{./.revinfo/gitCommitDate.tex}
\end{itemize}
}
}

%\PassOptionsToPackage{dvipsnames,svgnames}{xcolor}
\PassOptionsToPackage{square,numbers}{natbib}
\documentclass{scrreprt}

\usepackage[left=2cm,right=2cm]{geometry}
\usepackage[svgnames]{xcolor}
\usepackage{peeters_layout}

\usepackage{natbib}

\usepackage[
colorlinks=true,
bookmarks=false,
pdfauthor={\authorname, \email},
backref 
]{hyperref}

% http://tex.stackexchange.com/questions/75773/how-to-reference-problems-by-the-text-label-in-an-exercise-envioronment
\usepackage[english]{cleveref}
\crefname{Exercise}{exercise}{exercises}
\Crefname{Exercise}{Exercise}{Exercises}

\RequirePackage{titlesec}
\RequirePackage{ifthen}

% http://stackoverflow.com/questions/4932910/date-in-the-tabular-environment
\makeatletter
\let\insertdate\@date
\makeatother

\titleformat{\chapter}[display]
{\bfseries\Large}
{\color{DarkSlateGrey}\filleft \authorname
\ifthenelse{\isundefined{\studentnumber}}{}{\\ \studentnumber}
\ifthenelse{\isundefined{\email}}{}{\\ \email}
\ifthenelse{\isundefined{\dateintitle}}{}{\\ \insertdate}
%\ifthenelse{\isundefined{\coursename}}{}{\\ \coursename} % put in title instead.
}
{4ex}
{\color{DarkOliveGreen}{\titlerule}\color{Maroon}
\vspace{2ex}%
\filright}
[\vspace{2ex}%
\color{DarkOliveGreen}\titlerule
]

\newcommand{\beginArtWithToc}[0]{\begin{document}\tableofcontents}
\newcommand{\beginArtNoToc}[0]{\begin{document}}
\newcommand{\EndNoBibArticle}[0]{\end{document}}
\newcommand{\EndArticle}[0]{\bibliography{Bibliography}\bibliographystyle{plainnat}\end{document}}

% 
%\newcommand{\citep}[1]{\cite{#1}}

\colorSectionsForArticle



%\citep{harald2003solid} \S x.y
%\citep{ibach2009solid} \S x.y
%\reading \citep{ashcroft1976solid} \ch N.

%\usepackage{mhchem}
\usepackage[version=3]{mhchem}
\usepackage{units}
\usepackage{bm} % \EE
\newcommand{\nought}[0]{\circ}
%\newcommand{\EF}[0]{\epsilon_{\mathrm{F}}}
\newcommand{\EF}[0]{E_{\mathrm{F}}}
\newcommand{\kF}[0]{k_{\mathrm{F}}}

\beginArtNoToc
\generatetitle{One atom basis phonons in 2D}
%\chapter{One atom basis phonons in 2D}
\label{chap:oneAtomBasisPhonon}

%\section{One atom basis phonons in 2D}

Let's tackle a problem like the 2D problem of the final exam, but first more generally.  Instead of a square lattice consider the lattice with the geometry illustrated in \cref{fig:oneAtomBasisNearestNeighbours:oneAtomBasisNearestNeighboursFig1}.

\imageFigure{../../figures/phy487/oneAtomBasisNearestNeighboursFig1}{Oblique one atom basis}{fig:oneAtomBasisNearestNeighbours:oneAtomBasisNearestNeighboursFig1}{0.3}

Here, $\Ba$ and $\Bb$ are the vector differences between the equilibrium positions separating the masses along the $K_1$ and $K_2$ interaction directions respectively.  The equilibrium spacing for the cross coupling harmonic forces are

\begin{dmath}\label{eqn:oneAtomBasisPhonon:40}
\begin{aligned}
\Br &= (\Bb + \Ba)/2 \\
\Bs &= (\Bb - \Ba)/2.
\end{aligned}
\end{dmath}

Based on previous calculations, we can write the equations of motion by inspection

\begin{dmath}\label{eqn:oneAtomBasisPhonon:20}
\begin{aligned}
m \ddot{\Bu}_\Bn 
= 
&-K_1 \Proj_{\acap} \sum_\pm \lr{ \Bu_\Bn - \Bu_{\Bn \pm(1, 0)}}^2 \\
&-K_2 \Proj_{\bcap} \sum_\pm \lr{ \Bu_\Bn - \Bu_{\Bn \pm(0, 1)}}^2 \\
&-K_3 \Proj_{\rcap} \sum_\pm \lr{ \Bu_\Bn - \Bu_{\Bn \pm(1, 1)}}^2 \\
&-K_4 \Proj_{\scap} \sum_\pm \lr{ \Bu_\Bn - \Bu_{\Bn \pm(1, -1)}}^2.
\end{aligned}
\end{dmath}

Inserting the trial solution

\begin{dmath}\label{eqn:oneAtomBasisPhonon:60}
\Bu_\Bn = \inv{\sqrt{m}} \Bepsilon(\Bq) e^{i( \Br_\Bn \cdot \Bq - \omega t) },
\end{dmath}

and using the matrix form for the projection operators, we have

\begin{dmath}\label{eqn:oneAtomBasisPhonon:80}
\begin{aligned}
\omega^2 \Bepsilon 
&=
\frac{K_1}{m} \acap \acap^\T \Bepsilon
\sum_\pm
\lr{ 
1 - e^{\pm i \Ba \cdot \Bq}
} \\
& +
\frac{K_2}{m} \bcap \bcap^\T \Bepsilon
\sum_\pm
\lr{ 
1 - e^{\pm i \Bb \cdot \Bq}
} \\
& +
\frac{K_3}{m} \bcap \bcap^\T \Bepsilon
\sum_\pm
\lr{ 
1 - e^{\pm i (\Bb + \Ba) \cdot \Bq}
} \\
& +
\frac{K_3}{m} \bcap \bcap^\T \Bepsilon
\sum_\pm
\lr{ 
1 - e^{\pm i (\Bb - \Ba) \cdot \Bq}
} \\
&=
\frac{4 K_1}{m} \acap \acap^\T \Bepsilon \sin^2\lr{ \Ba \cdot \Bq/2 }
+
\frac{4 K_2}{m} \bcap \bcap^\T \Bepsilon \sin^2\lr{ \Bb \cdot \Bq/2 } \\
&+
\frac{4 K_3}{m} \rcap \rcap^\T \Bepsilon \sin^2\lr{ (\Bb + \Ba) \cdot \Bq/2 }
+
\frac{4 K_4}{m} \scap \scap^\T \Bepsilon \sin^2\lr{ (\Bb - \Ba) \cdot \Bq/2 }.
\end{aligned}
\end{dmath}

This fully specifies our eigenvalue problem.  Writing

\begin{subequations}
\begin{dmath}\label{eqn:oneAtomBasisPhonon:340}
\begin{aligned}
S_1 &= \sin^2\lr{ \Ba \cdot \Bq/2 } \\
S_2 &= \sin^2\lr{ \Bb \cdot \Bq/2 } \\
S_3 &= \sin^2\lr{ (\Bb + \Ba) \cdot \Bq/2 } \\
S_4 &= \sin^2\lr{ (\Bb - \Ba) \cdot \Bq/2 }
\end{aligned}
\end{dmath}
\begin{dmath}\label{eqn:oneAtomBasisPhonon:100}
A = 
\frac{4 K_1 S_1}{m} \acap \acap^\T 
+
\frac{4 K_2 S_2}{m} \bcap \bcap^\T 
+
\frac{4 K_3 S_3}{m} \rcap \rcap^\T 
+
\frac{4 K_4 S_4}{m} \scap \scap^\T,
\end{dmath}
\end{subequations}

we wish to solve

\begin{equation}\label{eqn:oneAtomBasisPhonon:120}
A \Bepsilon = \omega^2 \Bepsilon = \lambda \Bepsilon.
\end{equation}

Neglecting the specifics of the matrix at hand, consider a generic two by two matrix

\begin{dmath}\label{eqn:oneAtomBasisPhonon:140}
A = 
\begin{bmatrix}
a & b \\
c & d
\end{bmatrix},
\end{dmath}

for which the characteristic equation is

\begin{dmath}\label{eqn:oneAtomBasisPhonon:160}
0 = 
\begin{vmatrix}
\lambda - a & - b \\
-c & \lambda -d 
\end{vmatrix}
=
(\lambda - a)(\lambda - d) - b c
=
\lambda^2 - (a + d) \lambda + a d - b c
= 
\lambda^2 - (Tr A) \lambda + \Abs{A}
= 
\lr{\lambda - \frac{Tr A}{2}}^2
- \lr{\frac{Tr A}{2}}^2 + \Abs{A}.
\end{dmath}

So our angular frequencies are given by

\begin{dmath}\label{eqn:oneAtomBasisPhonon:180}
\omega^2 = 
\inv{2} \lr{ Tr A \pm \sqrt{ \lr{Tr A}^2 - 4 \Abs{A} }}.
\end{dmath}

The square root can be simplified slightly

\begin{dmath}\label{eqn:oneAtomBasisPhonon:480}
\lr{Tr A}^2 - 4 \Abs{A}
=
(a + d)^2 -4 (a d - b c)
=
a^2 + d^2 + 2 a d - 4 a d + 4 b c
=
(a - d)^2 + 4 b c,
\end{dmath}

so that, finally, the dispersion relation is

\begin{dmath}\label{eqn:oneAtomBasisPhonon:180}
\myBoxed{
\omega^2 = 
\inv{2} \lr{ d + a   \pm \sqrt{ (d - a)^2 + 4 b c } },
}
\end{dmath}

Our eigenvectors will be given by

\begin{dmath}\label{eqn:oneAtomBasisPhonon:200}
0 = (\lambda - a) \Bepsilon_1 - b\Bepsilon_2,
\end{dmath}

or

\begin{dmath}\label{eqn:oneAtomBasisPhonon:220}
\Bepsilon_1 \propto \frac{b}{\lambda - a}\Bepsilon_2.
\end{dmath}

So, our eigenvectors, the vectoral components of our atomic displacements, are

\begin{dmath}\label{eqn:oneAtomBasisPhonon:240}
\Bepsilon \propto
\begin{bmatrix}
b \\
\omega^2 - a
\end{bmatrix},
\end{dmath}

or

\begin{dmath}\label{eqn:oneAtomBasisPhonon:500}
\myBoxed{
\Bepsilon \propto
\begin{bmatrix}
2 b \\
d - a \pm \sqrt{ (d - a)^2 + 4 b c }
\end{bmatrix}.
}
\end{dmath}

\paragraph{Square lattice}

There is not too much to gain by expanding out the projection operators explicitly in general.  However, let's do this for the specific case of a square lattice (as on the exam problem).  In that case, our projection operators are

\begin{subequations}
\begin{equation}\label{eqn:oneAtomBasisPhonon:260}
\acap \acap^\T 
= 
\begin{bmatrix}
1 \\ 
0
\end{bmatrix}
\begin{bmatrix}
1 &  
0
\end{bmatrix}
=
\begin{bmatrix}
1 & 0 \\
0 & 0
\end{bmatrix}
\end{equation}
\begin{equation}\label{eqn:oneAtomBasisPhonon:280}
\bcap \bcap^\T 
= 
\begin{bmatrix}
0\\
1  
\end{bmatrix}
\begin{bmatrix}
0 &
1   
\end{bmatrix}
=
\begin{bmatrix}
0 & 0 \\
0 & 1
\end{bmatrix}
\end{equation}
\begin{equation}\label{eqn:oneAtomBasisPhonon:300}
\rcap \rcap^\T 
= 
\inv{2}
\begin{bmatrix}
1 \\
1  
\end{bmatrix}
\begin{bmatrix}
1 &
1   
\end{bmatrix}
=
\inv{2}
\begin{bmatrix}
1 & 1 \\
1 & 1
\end{bmatrix}
\end{equation}
\begin{equation}\label{eqn:oneAtomBasisPhonon:320}
\scap \scap^\T 
= 
\inv{2}
\begin{bmatrix}
-1 \\
1  
\end{bmatrix}
\begin{bmatrix}
-1 &
1   
\end{bmatrix}
=
\inv{2}
\begin{bmatrix}
1 & -1 \\
-1 & 1
\end{bmatrix}
\end{equation}
\end{subequations}


\begin{dmath}\label{eqn:oneAtomBasisPhonon:360}
\begin{aligned}
S_1 &= \sin^2\lr{ \Ba \cdot \Bq } \\
S_2 &= \sin^2\lr{ \Bb \cdot \Bq } \\
S_3 &= \sin^2\lr{ (\Bb + \Ba) \cdot \Bq } \\
S_4 &= \sin^2\lr{ (\Bb - \Ba) \cdot \Bq },
\end{aligned}
\end{dmath}


Our matrix is

\begin{dmath}\label{eqn:oneAtomBasisPhonon:460}
A = 
\frac{2}{m}
\begin{bmatrix}
2 K_1 S_1 + K_3 S_3 + K_4 S_4 & 2( K_3 S_3 - K_4 S_4 ) \\
2( K_3 S_3 - K_4 S_4 ) & 2 K_2 S_2 + K_3 S_3 + K_4 S_4
\end{bmatrix},
\end{dmath}

where, specifically, the squared sines for this geometry are

\begin{subequations}
\begin{equation}\label{eqn:oneAtomBasisPhonon:380}
S_1 = \sin^2 \lr{ \Ba \cdot \Bq/2 } = \sin^2 \lr{ a q_x/2}
\end{equation}
\begin{equation}\label{eqn:oneAtomBasisPhonon:400}
S_2 = \sin^2 \lr{ \Bb \cdot \Bq/2 } = \sin^2 \lr{ a q_y/2}
\end{equation}
\begin{equation}\label{eqn:oneAtomBasisPhonon:420}
S_3 = \sin^2 \lr{ (\Bb + \Ba) \cdot \Bq/2 } = \sin^2 \lr{ a (q_x + q_y)/2}
\end{equation}
\begin{equation}\label{eqn:oneAtomBasisPhonon:440}
S_4 = \sin^2 \lr{ (\Bb - \Ba) \cdot \Bq/2 } = \sin^2 \lr{ a (q_y - q_x)/2}.
\end{equation}
\end{subequations}


our squared sines are

%\begin{subequations}
%\begin{equation}\label{eqn:oneAtomBasisPhonon:380}
%S_1 = \sin^2 \lr{ \Ba \cdot \Bq/2 } = \sin^2 \lr{ a q_x/2}
%\end{equation}
%\begin{equation}\label{eqn:oneAtomBasisPhonon:400}
%S_2 = \sin^2 \lr{ \Bb \cdot \Bq/2 } = \sin^2 \lr{ a q_y/2}
%\end{equation}
%\begin{equation}\label{eqn:oneAtomBasisPhonon:420}
%S_3 = \sin^2 \lr{ (\Bb + \Ba) \cdot \Bq/2 } = \sin^2 \lr{ a (q_x + q_y)/2}
%\end{equation}
%\begin{equation}\label{eqn:oneAtomBasisPhonon:440}
%S_4 = \sin^2 \lr{ (\Bb - \Ba) \cdot \Bq/2 } = \sin^2 \lr{ a (q_y - q_x)/2}.
%\end{equation}
%\end{subequations}

The dispersion relation, and eigenvectors are

\begin{subequations}
\begin{dmath}\label{eqn:oneAtomBasisPhonon:520}
\omega^2 = 
\frac{2}{m} \lr{ \sum_i K_i S_i \pm \sqrt{ (K_2 S_2 - K_1 S_1)^2 + 4 (K_3 S_3 - K_4 S_4)^2 } }
\end{dmath}
\begin{dmath}\label{eqn:oneAtomBasisPhonon:540}
\Bepsilon \propto
\begin{bmatrix}
2 \lr{ K_3 S_3 - K_4 S_4 } \\
K_1 S_1 - K_2 S_2 
\pm \sqrt{ (K_2 S_2 - K_1 S_1)^2 + 4 (K_3 S_3 - K_4 S_4)^2 }
\end{bmatrix}.
\end{dmath}
\end{subequations}

This calculation, one that I messed up the simplification of, was done correctly in oneAtomBasisPhononSquareLatticeEigensystem.nb.
% FIXME: % \nbref{oneAtomBasisPhononSquareLatticeEigensystem.nb}

In the specific case that we had on the exam where $K_1 = K_2$ and $K_3 = K_4$, these are

\begin{subequations}
\begin{dmath}\label{eqn:oneAtomBasisPhonon:560}
\omega^2 = 
\frac{2}{m} \lr{ K_1 (S_1 + S_2) + K_3(S_3 + S_4) \pm \sqrt{ K_1^2 (S_2 - S_1)^2 + 4 K_3^2 (S_3 - S_4)^2 } }
\end{dmath}
\begin{dmath}\label{eqn:oneAtomBasisPhonon:580}
\Bepsilon \propto
\begin{bmatrix}
2 K_3 \lr{ S_3 - S_4 } \\
K_1 
\lr{
(S_1 - S_2)
\pm \sqrt{ (S_2 - S_1)^2 + \lr{\frac{2 K_3}{K_1}}^2 (S_3 - S_4)^2 }
}
\end{bmatrix}.
\end{dmath}
\end{subequations}

We see that the cross coupling is required to have a non-zero $x$ component.  This is likely what our Prof was after, when asking why this cross coupling is required for stability (i.e. without those spring constants we have a diagonal interaction matrix, and completely independent eigenvectors)

\paragraph{System in rotated coordinates}

On the exam, where we were asked to solve for motion along the cross directions explicitly, there was a strong hint to consider a rotated (by $\pi/4$) coordinate system.  We can make a change of variables for the reciprocal coordinates

\begin{subequations}
\begin{equation}\label{eqn:oneAtomBasisPhonon:600}
\begin{bmatrix}
k_u \\
k_v
\end{bmatrix}
=
\inv{\sqrt{2}}
\begin{bmatrix}
1 & 1 \\
-1 & 1
\end{bmatrix}
\begin{bmatrix}
q_x \\
q_y
\end{bmatrix}
=
\inv{\sqrt{2}}
\begin{bmatrix}
q_y + q_x \\
q_y - q_x
\end{bmatrix}
\end{equation}
\begin{equation}\label{eqn:oneAtomBasisPhonon:620}
\begin{bmatrix}
q_x \\
q_y
\end{bmatrix}
=
\inv{\sqrt{2}}
\begin{bmatrix}
1 & -1 \\
1 & 1
\end{bmatrix}
\begin{bmatrix}
k_u \\
k_v
\end{bmatrix}
=
\inv{\sqrt{2}}
\begin{bmatrix}
k_u - k_v \\
k_u + k_v
\end{bmatrix},
\end{equation}
\end{subequations}

and also rotate the lattice basis vectors $\Ba = a \Be_1, \Bb = a \Be_2$, and the projection matrices.  Writing $\rcap = \Bf_1$ and $\scap = \Bf_2$, where $\Bf_1 = (\Be_1 + \Be_2)/\sqrt{2}$, and $\Bf_2 = (\Be_2 - \Be_1)/\sqrt{2}$.

%\EndArticle
\EndNoBibArticle
