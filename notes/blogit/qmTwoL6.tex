%
% Copyright � 2015 Peeter Joot.  All Rights Reserved.
% Licenced as described in the file LICENSE under the root directory of this GIT repository.
%
\documentclass[]{eliblog}

\usepackage{amsmath}
\usepackage{mathpazo}

%
% shorthand for bold symbols, convenient for vectors and matrices
%
\newcommand{\Ba}[0]{\mathbf{a}}
\newcommand{\Bb}[0]{\mathbf{b}}
\newcommand{\Bc}[0]{\mathbf{c}}
\newcommand{\Bd}[0]{\mathbf{d}}
\newcommand{\Be}[0]{\mathbf{e}}
\newcommand{\Bf}[0]{\mathbf{f}}
\newcommand{\Bg}[0]{\mathbf{g}}
\newcommand{\Bh}[0]{\mathbf{h}}
\newcommand{\Bi}[0]{\mathbf{i}}
\newcommand{\Bj}[0]{\mathbf{j}}
\newcommand{\Bk}[0]{\mathbf{k}}
\newcommand{\Bl}[0]{\mathbf{l}}
\newcommand{\Bm}[0]{\mathbf{m}}
\newcommand{\Bn}[0]{\mathbf{n}}
\newcommand{\Bo}[0]{\mathbf{o}}
\newcommand{\Bp}[0]{\mathbf{p}}
\newcommand{\Bq}[0]{\mathbf{q}}
\newcommand{\Br}[0]{\mathbf{r}}
\newcommand{\Bs}[0]{\mathbf{s}}
\newcommand{\Bt}[0]{\mathbf{t}}
\newcommand{\Bu}[0]{\mathbf{u}}
\newcommand{\Bv}[0]{\mathbf{v}}
\newcommand{\Bw}[0]{\mathbf{w}}
\newcommand{\Bx}[0]{\mathbf{x}}
\newcommand{\By}[0]{\mathbf{y}}
\newcommand{\Bz}[0]{\mathbf{z}}
\newcommand{\BA}[0]{\mathbf{A}}
\newcommand{\BB}[0]{\mathbf{B}}
\newcommand{\BC}[0]{\mathbf{C}}
\newcommand{\BD}[0]{\mathbf{D}}
\newcommand{\BE}[0]{\mathbf{E}}
\newcommand{\BF}[0]{\mathbf{F}}
\newcommand{\BG}[0]{\mathbf{G}}
\newcommand{\BH}[0]{\mathbf{H}}
\newcommand{\BI}[0]{\mathbf{I}}
\newcommand{\BJ}[0]{\mathbf{J}}
\newcommand{\BK}[0]{\mathbf{K}}
\newcommand{\BL}[0]{\mathbf{L}}
\newcommand{\BM}[0]{\mathbf{M}}
\newcommand{\BN}[0]{\mathbf{N}}
\newcommand{\BO}[0]{\mathbf{O}}
\newcommand{\BP}[0]{\mathbf{P}}
\newcommand{\BQ}[0]{\mathbf{Q}}
\newcommand{\BR}[0]{\mathbf{R}}
\newcommand{\BS}[0]{\mathbf{S}}
\newcommand{\BT}[0]{\mathbf{T}}
\newcommand{\BU}[0]{\mathbf{U}}
\newcommand{\BV}[0]{\mathbf{V}}
\newcommand{\BW}[0]{\mathbf{W}}
\newcommand{\BX}[0]{\mathbf{X}}
\newcommand{\BY}[0]{\mathbf{Y}}
\newcommand{\BZ}[0]{\mathbf{Z}}

\newcommand{\Bzero}[0]{\mathbf{0}}
\newcommand{\Btheta}[0]{\boldsymbol{\theta}}
\newcommand{\Btau}[0]{\boldsymbol{\tau}}
\newcommand{\Bomega}[0]{\boldsymbol{\omega}}

%
% shorthand for unit vectors
%
\newcommand{\acap}[0]{\hat{\Ba}}
\newcommand{\bcap}[0]{\hat{\Bb}}
\newcommand{\ccap}[0]{\hat{\Bc}}
\newcommand{\dcap}[0]{\hat{\Bd}}
\newcommand{\ecap}[0]{\hat{\Be}}
\newcommand{\fcap}[0]{\hat{\Bf}}
\newcommand{\gcap}[0]{\hat{\Bg}}
\newcommand{\hcap}[0]{\hat{\Bh}}
\newcommand{\icap}[0]{\hat{\Bi}}
\newcommand{\jcap}[0]{\hat{\Bj}}
\newcommand{\kcap}[0]{\hat{\Bk}}
\newcommand{\lcap}[0]{\hat{\Bl}}
\newcommand{\mcap}[0]{\hat{\Bm}}
\newcommand{\ncap}[0]{\hat{\Bn}}
\newcommand{\ocap}[0]{\hat{\Bo}}
\newcommand{\pcap}[0]{\hat{\Bp}}
\newcommand{\qcap}[0]{\hat{\Bq}}
\newcommand{\rcap}[0]{\hat{\Br}}
\newcommand{\scap}[0]{\hat{\Bs}}
\newcommand{\tcap}[0]{\hat{\Bt}}
\newcommand{\ucap}[0]{\hat{\Bu}}
\newcommand{\vcap}[0]{\hat{\Bv}}
\newcommand{\wcap}[0]{\hat{\Bw}}
\newcommand{\xcap}[0]{\hat{\Bx}}
\newcommand{\ycap}[0]{\hat{\By}}
\newcommand{\zcap}[0]{\hat{\Bz}}
\newcommand{\thetacap}[0]{\hat{\Btheta}}

%
% to write R^n and C^n in a distinguishable fashion.  Perhaps change this
% to the double lined characters upon figuring out how to do so.
%
\newcommand{\C}[1]{$\mathbb{C}^{#1}$}
\newcommand{\R}[1]{$\mathbb{R}^{#1}$}

%
% various generally useful helpers
%

% derivative of #1 wrt. #2:
\newcommand{\D}[2] {\frac {d#2} {d#1}}

\newcommand{\inv}[1]{\frac{1}{#1}}
\newcommand{\cross}[0]{\times}

\newcommand{\abs}[1]{\lvert{#1}\rvert}
\newcommand{\norm}[1]{\lVert{#1}\rVert}
\newcommand{\innerprod}[2]{\langle{#1}, {#2}\rangle}
\newcommand{\dotprod}[2]{{#1} \cdot {#2}}
\newcommand{\bdotprod}[2]{\left({#1} \cdot {#2}\right)}
\newcommand{\crossprod}[2]{{#1} \cross {#2}}
\newcommand{\tripleprod}[3]{\dotprod{\left(\crossprod{#1}{#2}\right)}{#3}}

\DeclareMathOperator{\Proj}{Proj}
\DeclareMathOperator{\Span}{span}
\DeclareMathOperator{\Sgn}{sgn}
\DeclareMathOperator{\Area}{Area}
\DeclareMathOperator{\Volume}{Volume}

%
% A few miscellaneous things specific to this document
%
\newcommand{\crossop}[1]{\crossprod{#1}{}}

% R2 vector.
\newcommand{\VectorTwo}[2]{
\begin{bmatrix}
 {#1} \\
 {#2}
\end{bmatrix}
}

\newcommand{\VectorN}[1]{
\begin{bmatrix}
{#1}_1 \\
{#1}_2 \\
\vdots \\
{#1}_N \\
\end{bmatrix}
}

\newcommand{\DETuvij}[4]{
\begin{vmatrix}
 {#1}_{#3} & {#1}_{#4} \\
 {#2}_{#3} & {#2}_{#4}
\end{vmatrix}
}

\newcommand{\DETuvwijk}[6]{
\begin{vmatrix}
 {#1}_{#4} & {#1}_{#5} & {#1}_{#6} \\
 {#2}_{#4} & {#2}_{#5} & {#2}_{#6} \\
 {#3}_{#4} & {#3}_{#5} & {#3}_{#6}
\end{vmatrix}
}

\newcommand{\DETuvwxijkl}[8]{
\begin{vmatrix}
 {#1}_{#5} & {#1}_{#6} & {#1}_{#7} & {#1}_{#8} \\
 {#2}_{#5} & {#2}_{#6} & {#2}_{#7} & {#2}_{#8} \\
 {#3}_{#5} & {#3}_{#6} & {#3}_{#7} & {#3}_{#8} \\
 {#4}_{#5} & {#4}_{#6} & {#4}_{#7} & {#4}_{#8} \\
\end{vmatrix}
}

%\newcommand{\DETuvwxyijklm}[10]{
%\begin{vmatrix}
% {#1}_{#6} & {#1}_{#7} & {#1}_{#8} & {#1}_{#9} & {#1}_{#10} \\
% {#2}_{#6} & {#2}_{#7} & {#2}_{#8} & {#2}_{#9} & {#2}_{#10} \\
% {#3}_{#6} & {#3}_{#7} & {#3}_{#8} & {#3}_{#9} & {#3}_{#10} \\
% {#4}_{#6} & {#4}_{#7} & {#4}_{#8} & {#4}_{#9} & {#4}_{#10} \\
% {#5}_{#6} & {#5}_{#7} & {#5}_{#8} & {#5}_{#9} & {#5}_{#10}
%\end{vmatrix}
%}

% R3 vector.
\newcommand{\VectorThree}[3]{
\begin{bmatrix}
 {#1} \\
 {#2} \\
 {#3}
\end{bmatrix}
}



\author{Peeter Joot}
\email{peeter.joot@gmail.com}

%\documentclass[]{eliblogwidescreen}

\usepackage{amsmath}
\usepackage{mathpazo}

%
% shorthand for bold symbols, convenient for vectors and matrices
%
\newcommand{\Ba}[0]{\mathbf{a}}
\newcommand{\Bb}[0]{\mathbf{b}}
\newcommand{\Bc}[0]{\mathbf{c}}
\newcommand{\Bd}[0]{\mathbf{d}}
\newcommand{\Be}[0]{\mathbf{e}}
\newcommand{\Bf}[0]{\mathbf{f}}
\newcommand{\Bg}[0]{\mathbf{g}}
\newcommand{\Bh}[0]{\mathbf{h}}
\newcommand{\Bi}[0]{\mathbf{i}}
\newcommand{\Bj}[0]{\mathbf{j}}
\newcommand{\Bk}[0]{\mathbf{k}}
\newcommand{\Bl}[0]{\mathbf{l}}
\newcommand{\Bm}[0]{\mathbf{m}}
\newcommand{\Bn}[0]{\mathbf{n}}
\newcommand{\Bo}[0]{\mathbf{o}}
\newcommand{\Bp}[0]{\mathbf{p}}
\newcommand{\Bq}[0]{\mathbf{q}}
\newcommand{\Br}[0]{\mathbf{r}}
\newcommand{\Bs}[0]{\mathbf{s}}
\newcommand{\Bt}[0]{\mathbf{t}}
\newcommand{\Bu}[0]{\mathbf{u}}
\newcommand{\Bv}[0]{\mathbf{v}}
\newcommand{\Bw}[0]{\mathbf{w}}
\newcommand{\Bx}[0]{\mathbf{x}}
\newcommand{\By}[0]{\mathbf{y}}
\newcommand{\Bz}[0]{\mathbf{z}}
\newcommand{\BA}[0]{\mathbf{A}}
\newcommand{\BB}[0]{\mathbf{B}}
\newcommand{\BC}[0]{\mathbf{C}}
\newcommand{\BD}[0]{\mathbf{D}}
\newcommand{\BE}[0]{\mathbf{E}}
\newcommand{\BF}[0]{\mathbf{F}}
\newcommand{\BG}[0]{\mathbf{G}}
\newcommand{\BH}[0]{\mathbf{H}}
\newcommand{\BI}[0]{\mathbf{I}}
\newcommand{\BJ}[0]{\mathbf{J}}
\newcommand{\BK}[0]{\mathbf{K}}
\newcommand{\BL}[0]{\mathbf{L}}
\newcommand{\BM}[0]{\mathbf{M}}
\newcommand{\BN}[0]{\mathbf{N}}
\newcommand{\BO}[0]{\mathbf{O}}
\newcommand{\BP}[0]{\mathbf{P}}
\newcommand{\BQ}[0]{\mathbf{Q}}
\newcommand{\BR}[0]{\mathbf{R}}
\newcommand{\BS}[0]{\mathbf{S}}
\newcommand{\BT}[0]{\mathbf{T}}
\newcommand{\BU}[0]{\mathbf{U}}
\newcommand{\BV}[0]{\mathbf{V}}
\newcommand{\BW}[0]{\mathbf{W}}
\newcommand{\BX}[0]{\mathbf{X}}
\newcommand{\BY}[0]{\mathbf{Y}}
\newcommand{\BZ}[0]{\mathbf{Z}}

\newcommand{\Bzero}[0]{\mathbf{0}}
\newcommand{\Btheta}[0]{\boldsymbol{\theta}}
\newcommand{\Btau}[0]{\boldsymbol{\tau}}
\newcommand{\Bomega}[0]{\boldsymbol{\omega}}

%
% shorthand for unit vectors
%
\newcommand{\acap}[0]{\hat{\Ba}}
\newcommand{\bcap}[0]{\hat{\Bb}}
\newcommand{\ccap}[0]{\hat{\Bc}}
\newcommand{\dcap}[0]{\hat{\Bd}}
\newcommand{\ecap}[0]{\hat{\Be}}
\newcommand{\fcap}[0]{\hat{\Bf}}
\newcommand{\gcap}[0]{\hat{\Bg}}
\newcommand{\hcap}[0]{\hat{\Bh}}
\newcommand{\icap}[0]{\hat{\Bi}}
\newcommand{\jcap}[0]{\hat{\Bj}}
\newcommand{\kcap}[0]{\hat{\Bk}}
\newcommand{\lcap}[0]{\hat{\Bl}}
\newcommand{\mcap}[0]{\hat{\Bm}}
\newcommand{\ncap}[0]{\hat{\Bn}}
\newcommand{\ocap}[0]{\hat{\Bo}}
\newcommand{\pcap}[0]{\hat{\Bp}}
\newcommand{\qcap}[0]{\hat{\Bq}}
\newcommand{\rcap}[0]{\hat{\Br}}
\newcommand{\scap}[0]{\hat{\Bs}}
\newcommand{\tcap}[0]{\hat{\Bt}}
\newcommand{\ucap}[0]{\hat{\Bu}}
\newcommand{\vcap}[0]{\hat{\Bv}}
\newcommand{\wcap}[0]{\hat{\Bw}}
\newcommand{\xcap}[0]{\hat{\Bx}}
\newcommand{\ycap}[0]{\hat{\By}}
\newcommand{\zcap}[0]{\hat{\Bz}}
\newcommand{\thetacap}[0]{\hat{\Btheta}}

%
% to write R^n and C^n in a distinguishable fashion.  Perhaps change this
% to the double lined characters upon figuring out how to do so.
%
\newcommand{\C}[1]{$\mathbb{C}^{#1}$}
\newcommand{\R}[1]{$\mathbb{R}^{#1}$}

%
% various generally useful helpers
%

% derivative of #1 wrt. #2:
\newcommand{\D}[2] {\frac {d#2} {d#1}}

\newcommand{\inv}[1]{\frac{1}{#1}}
\newcommand{\cross}[0]{\times}

\newcommand{\abs}[1]{\lvert{#1}\rvert}
\newcommand{\norm}[1]{\lVert{#1}\rVert}
\newcommand{\innerprod}[2]{\langle{#1}, {#2}\rangle}
\newcommand{\dotprod}[2]{{#1} \cdot {#2}}
\newcommand{\bdotprod}[2]{\left({#1} \cdot {#2}\right)}
\newcommand{\crossprod}[2]{{#1} \cross {#2}}
\newcommand{\tripleprod}[3]{\dotprod{\left(\crossprod{#1}{#2}\right)}{#3}}

\DeclareMathOperator{\Proj}{Proj}
\DeclareMathOperator{\Span}{span}
\DeclareMathOperator{\Sgn}{sgn}
\DeclareMathOperator{\Area}{Area}
\DeclareMathOperator{\Volume}{Volume}

%
% A few miscellaneous things specific to this document
%
\newcommand{\crossop}[1]{\crossprod{#1}{}}

% R2 vector.
\newcommand{\VectorTwo}[2]{
\begin{bmatrix}
 {#1} \\
 {#2}
\end{bmatrix}
}

\newcommand{\VectorN}[1]{
\begin{bmatrix}
{#1}_1 \\
{#1}_2 \\
\vdots \\
{#1}_N \\
\end{bmatrix}
}

\newcommand{\DETuvij}[4]{
\begin{vmatrix}
 {#1}_{#3} & {#1}_{#4} \\
 {#2}_{#3} & {#2}_{#4}
\end{vmatrix}
}

\newcommand{\DETuvwijk}[6]{
\begin{vmatrix}
 {#1}_{#4} & {#1}_{#5} & {#1}_{#6} \\
 {#2}_{#4} & {#2}_{#5} & {#2}_{#6} \\
 {#3}_{#4} & {#3}_{#5} & {#3}_{#6}
\end{vmatrix}
}

\newcommand{\DETuvwxijkl}[8]{
\begin{vmatrix}
 {#1}_{#5} & {#1}_{#6} & {#1}_{#7} & {#1}_{#8} \\
 {#2}_{#5} & {#2}_{#6} & {#2}_{#7} & {#2}_{#8} \\
 {#3}_{#5} & {#3}_{#6} & {#3}_{#7} & {#3}_{#8} \\
 {#4}_{#5} & {#4}_{#6} & {#4}_{#7} & {#4}_{#8} \\
\end{vmatrix}
}

%\newcommand{\DETuvwxyijklm}[10]{
%\begin{vmatrix}
% {#1}_{#6} & {#1}_{#7} & {#1}_{#8} & {#1}_{#9} & {#1}_{#10} \\
% {#2}_{#6} & {#2}_{#7} & {#2}_{#8} & {#2}_{#9} & {#2}_{#10} \\
% {#3}_{#6} & {#3}_{#7} & {#3}_{#8} & {#3}_{#9} & {#3}_{#10} \\
% {#4}_{#6} & {#4}_{#7} & {#4}_{#8} & {#4}_{#9} & {#4}_{#10} \\
% {#5}_{#6} & {#5}_{#7} & {#5}_{#8} & {#5}_{#9} & {#5}_{#10}
%\end{vmatrix}
%}

% R3 vector.
\newcommand{\VectorThree}[3]{
\begin{bmatrix}
 {#1} \\
 {#2} \\
 {#3}
\end{bmatrix}
}



\author{Peeter Joot}
\email{peeter.joot@gmail.com}


\chapter{PHY456H1F: Quantum Mechanics II.  Lecture 6 (Taught by Prof J.E. Sipe).  Interaction picture.}
\label{chap:qmTwoL6}
%\useCCL
\blogpage{http://sites.google.com/site/peeterjoot/math2011/qmTwoL6.pdf}
\date{Sept 26, 2011}
\revisionInfo{qmTwoL6.tex}

\beginArtWithToc
%\beginArtNoToc

\section{Disclaimer.}

Peeter's lecture notes from class.  May not be entirely coherent.

\section{Interaction picture.}
\subsection{Recap.}

Recall our table comparing our two interaction pictures

\begin{align*}
\text{Schr\"{o}dinger picture} &\qquad \text{Heisenberg picture} \\
i \hbar \frac{d}{dt} \ket{\psi_s(t)} = H \ket{\psi_s(t)} &\qquad i \hbar \frac{d}{dt} O_H(t) = \antisymmetric{O_H}{H} \\
\bra{\psi_s(t)} O_S \ket{\psi_s(t)} &= \bra{\psi_H} O_H \ket{\psi_H} \\
\ket{\psi_s(0)} &= \ket{\psi_H} \\
O_S &= O_H(0)
\end{align*}

\subsection{A motivating example.}

While fundamental Hamiltonians are independent of time, in a number of common cases, we can form approximate Hamiltonians that are time dependent.  One such example is that of Coulomb excitations of an atom, as covered in \S 18.3 of the text \cite{desai2009quantum}, and shown in figure (\ref{fig:qmTwoL6fig1}).

\begin{figure}[htp]
\centering
\includegraphics[totalheight=0.4\textheight]{qmTwoL6fig1}
\caption{Coulomb interaction of a nucleus and heavy atom.}\label{fig:qmTwoL6fig1}
\end{figure}

We consider the interaction of a nucleus with a neutral atom, heavy enough that it can be considered classically.  From the atoms point of view, the effects of the heavy nucleus barreling by can be described using a time dependent Hamiltonian.  For the atom, that interaction Hamiltonian is

\begin{equation}\label{eqn:qmTwoL6:10}
H' = \sum_i \frac{ Z e q_i }{\Abs{\Br_N(t) - \BR_i}}.
\end{equation}

Here and $\Br_N$ is the position vector for the heavy nucleus, and $\BR_i$ is the position to each charge within the atom, where $i$ ranges over all the internal charges, positive and negative, within the atom.

Placing the origin close to the atom, we can write this interaction Hamiltonian as

\begin{equation}\label{eqn:qmTwoL6:30}
H'(t) = \cancel{\sum_i \frac{Z e q_i}{\Abs{\Br_N(t)}}}
+ \sum_i Z e q_i \BR_i \cdot 
\evalbar{\left(
\PD{\Br}{} \inv{\Abs{ \Br_N(t) - \Br}}
\right)}{\Br = 0}
\end{equation}

The first term vanishes because the total charge in our neutral atom is zero.  This leaves us with

\begin{equation}\label{eqn:qmTwoL6:50}
\begin{aligned}
H'(t) 
&= 
-\sum_i q_i \BR_i \cdot \evalbar{\left(
-\PD{\Br}{} \frac{ Z e}{\Abs{ \Br_N(t) - \Br}}
\right)}{\Br = 0} \\
&= - \sum_i q_i \BR_i \cdot \BE(t),
\end{aligned}
\end{equation}

where $\BE(t)$ is the electric field at the origin due to the nucleus.

Introducing a dipole moment \underline{operator} for the atom

\begin{equation}\label{eqn:qmTwoL6:70}
\Bmu = \sum_i q_i \BR_i,
\end{equation}

the interaction takes the form

\begin{equation}\label{eqn:qmTwoL6:90}
H'(t) = -\Bmu \cdot \BE(t).
\end{equation}

Here we have a quantum mechanical operator, and a classical field taken together.  This sort of dipole interaction also occurs when we treat a atom placed into an electromagnetic field, treated classically as depicted in figure (\ref{fig:qmTwoL6fig2})

\begin{figure}[htp]
\centering
\includegraphics[totalheight=0.4\textheight]{qmTwoL6fig2}
\caption{atom in a field}\label{fig:qmTwoL6fig2}
\end{figure}

In the figure, we can use the dipole interaction, provided $\lambda \gg a$, where $a$ is the ``width'' of the atom.

Because it is great for examples, we will see this dipole interaction a lot.

\subsection{The interaction picture.}

Having talked about both the Schr\"{o}dinger and Heisenberg pictures, we can now move on to describe a hybrid, one where our Hamiltonian has been split into static and time dependent parts

\begin{equation}\label{eqn:qmTwoL6:110}
H(t) = H_0 + H'(t)
\end{equation}

We will formulate an approach for dealing with problems of this sort called the interaction picture.

This is also covered in \S 3.3 of the text, albeit in a much harder to understand fashion (the text appears to try to not pull the result from a magic hat, but the steps to get to the end result are messy).  It would probably have been nicer to see it this way instead.

In the Schr\"{o}dinger picture our dynamics have the form

\begin{equation}\label{eqn:qmTwoL6:130}
i \hbar \frac{d}{dt} \ket{\psi_s(t)} = H \ket{\psi_s(t)}
\end{equation}

\section{Appendix for this lecture (not class notes).}

\subsection{Justifying the Taylor expansion.}

One way to justify the expansion of \ref{eqn:qmTwoL6:30} is to consider a Clifford algebra factorization (following notation from \cite{doran2003gap}) of the absolute vector difference, where $\BR$ is considered small.

\begin{align*}
\Abs{\Br - \BR}
&= \sqrt{ \left(\Br - \BR\right) \left(\Br - \BR\right) } \\
&= \sqrt{ \gpgradezero{\Br \left(1 - \inv{\Br} \BR\right) \left(1 - \BR \inv{\Br}\right) \Br} } \\
&= \sqrt{ \gpgradezero{\Br^2 \left(1 - \inv{\Br} \BR\right) \left(1 - \BR \inv{\Br}\right) } } \\
&= \Abs{\Br} \sqrt{ 1 - 2 \inv{\Br} \cdot \BR + \gpgradezero{\inv{\Br} \BR \BR \inv{\Br}}} \\
&= \Abs{\Br} \sqrt{ 1 - 2 \inv{\Br} \cdot \BR + \frac{\BR^2}{\Br^2}}
\end{align*}

Neglecting the $\BR^2$ term, we can then Taylor series expand this scalar expression 

\begin{equation}\label{eqn:qmTwoL6:150}
\inv{\Abs{\Br - \BR}} 
\approx
\inv{\Abs{\Br}} \left( 
1 + \inv{\Br} \cdot \BR
\right) 
=
\inv{\Abs{\Br}} 
+ \frac{\rcap}{\Br^2} \cdot \BR
=
\inv{\Abs{\Br}} 
+ \frac{\Br}{\Abs{\Br}^3} \cdot \BR
\end{equation}

If all went well, this should match the expression used in class

\begin{equation}\label{eqn:qmTwoL6:170}
\inv{\Abs{\Br - \BR}} \approx \inv{\Abs{\Br}} 
+ 
\BR \cdot \evalbar{\left(
\PD{\BR}{} \inv{\Abs{ \Br - \BR}}
\right)}{\BR = 0}.
\end{equation}

Let's check this one.  Evaluating the gradient we have

\begin{align*}
\PD{\BR}{} \inv{\Abs{ \Br - \BR}}
&=
\Be_i \PD{R^i}{} ((x^j - R^j)^2)^{-1/2} \\
&=
\Be_i \left(-\inv{2}\right) 2 (x^j - R^j) \PD{R^i}{(x^j - R^j)} \inv{\Abs{\Br - \BR}^3} \\
&= \frac{\Br - \BR}{
\Abs{\Br - \BR}^3} 
\end{align*}

Evaluated at $\BR = 0$ we have

\begin{equation}\label{eqn:qmTwoL6:190}
\inv{\Abs{\Br - \BR}} \approx \inv{\Abs{\Br}} 
+ 
\BR \cdot \frac{\Br}{\Abs{\Br}^3},
\end{equation}

which matches what we got with the Clifford algebra approach.

A better justification for what was done in the lecture is to simply recall the derivation of the multi-variable Taylor series.  Given a scalar valued function of some number of variables

\begin{equation}\label{eqn:qmTwoL6:210}
f(\Bu) = f(u^1, u^2, \cdots),
\end{equation}

consider the displacement operation applied to the vector argument

\begin{equation}\label{eqn:qmTwoL6:230}
f(\Ba + \Bx) = \evalbar{f(\Ba + t \Bx)}{t=1}.
\end{equation}

We can Taylor expand a single variable function without any trouble, so introduce

\begin{equation}\label{eqn:qmTwoL6:250}
g(t) = f(\Ba + t \Bx),
\end{equation}

where

\begin{equation}\label{eqn:qmTwoL6:270}
g(1) = f(\Ba + \Bx).
\end{equation}

We have 

\begin{equation}\label{eqn:qmTwoL6:290}
g(t) = g(0) 
+ t \evalbar{ \PD{t}{g} }{t = 0}
+ \frac{t^2}{2!} \evalbar{ \PD{t}{g} }{t = 0}
+ \cdots,
\end{equation}

so that

\begin{equation}\label{eqn:qmTwoL6:310}
g(1) = g(0) + 
+ \evalbar{ \PD{t}{g} }{t = 0}
+ \frac{1}{2!} \evalbar{ \PD{t}{g} }{t = 0}
+ \cdots.
\end{equation}

The multivariable Taylor series now becomes a plain old application of the chain rule, where we have to evaluate

\begin{align*}
\frac{dg}{dt} 
&= \ddt{} f(a^1 + t x^1, a^2 + t x^2, \cdots) \\
&= \sum_i \PD{(a^i + t x^i)}{} f(\Ba + t \Bx) \PD{t}{a^i + t x^i},
\end{align*}

so that

\begin{equation}\label{eqn:qmTwoL6:330}
\evalbar{\frac{dg}{dt} }{t=0}
= \sum_i x^i \left( 
\evalbar{ \PD{x^i}{f}}{x^i = a^i}
\right).
\end{equation}

Assuming an Euclidean space we can write this in the notationally more pleasant fashion using a gradient operator for the space

\begin{equation}\label{eqn:qmTwoL6:350}
\evalbar{\frac{dg}{dt} }{t=0} = \evalbar{\Bx \cdot \spacegrad_{\Bu} f(\Bu)}{\Bu = \Ba}.
\end{equation}

To handle the higher order terms, we repeat the chain rule application, yielding for example

\begin{align*}
\evalbar{\frac{d^2 f(\Ba + t \Bx)}{dt^2} }{t=0} 
&=
\evalbar{\ddt{} 
\sum_i x^i 
\PD{(a^i + t x^i)}{f(\Ba + t \Bx)} }{t=0}\\
&=
\evalbar{\sum_i x^i 
\PD{(a^i + t x^i)}{} \ddt{f(\Ba + t \Bx)}}{t=0} \\
&=
\evalbar{(\Bx \cdot \spacegrad_{\Bu})^2 f(\Bu)}{\Bu = \Ba}.
\end{align*}

Thus the Taylor series associated with a vector displacement takes the tidy form

\begin{equation}\label{eqn:qmTwoL6:370}
f(\Ba + \Bx) = \sum_{k=0}^\infty \inv{k!} \evalbar{(\Bx \cdot \spacegrad_{\Bu})^k f(\Bu)}{\Bu = \Ba}.
\end{equation}

Even more fancy, we can form the operator equation

\begin{equation}\label{eqn:qmTwoL6:390}
f(\Ba + \Bx) = \evalbar{e^{ \Bx \cdot \spacegrad_{\Bu} } f(\Bu)}{\Bu = \Ba}
\end{equation}

Here a dummy variable $\Bu$ has been retained as an instruction not to differentiate the $\Bx$ part of the directional derivative in any repeated applications of the $\Bx \cdot \spacegrad$ operator.

Once we derive this (or for those with lesser degrees of amnesia, recall it), we can see that \ref{eqn:qmTwoL6:30} was a direct application of this, retaining no second order or higher terms.

\EndArticle
