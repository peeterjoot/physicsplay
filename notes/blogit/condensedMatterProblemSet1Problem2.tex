%
% Copyright � 2013 Peeter Joot.  All Rights Reserved.
% Licenced as described in the file LICENSE under the root directory of this GIT repository.
%
\makeproblem{$sp^2$ hybrid orbitals}{condensedMatter:problemSet1:2}{ 

The $2s$ and $2p$ orbitals of a hydrogenic atom (i.e.\ one electron, nuclear 
  charge $Ze$) are: 

\begin{eqnarray*}
\phi_{2s}(\rho)   &=& {\cal N}\,{\rm e}^{-\rho}(1  - \rho) \\
\phi_{2p_z}(\rho) &=& {\cal N}\,{\rm e}^{-\rho}\rho\cos\theta \\
\phi_{2p_x}(\rho) &=& {\cal N}\,{\rm e}^{-\rho}\rho\sin\theta \, \cos\phi \\
\phi_{2p_y}(\rho) &=& {\cal N}\,{\rm e}^{-\rho}\rho\sin\theta \, \sin\phi
\end{eqnarray*}

where $\rho = Zr/2a_\circ$,
$a_\circ$ is the Bohr radius, $r$ is the radial distance from 
the nucleus, and $\theta$ and $\phi$ are the polar and azimuthal 
angles. ${\cal N} = (Z/2a_\circ)^{3/2}/\sqrt{\pi}$ is the normalization constant.

Four $sp^2$ hybrid orbitals are constructed from these orbitals as follows:
\begin{eqnarray*}
\psi_1  &=& \frac{1}{\sqrt{3}} \phi_{2s} + \sqrt{\frac{2}{3}} \phi_{2p_x} \\
\psi_2  &=& \frac{1}{\sqrt{3}} \phi_{2s} - \frac{1}{\sqrt{6}} \phi_{2p_x} + \frac{1}{\sqrt{2}} \phi_{2p_y} \\
\psi_3  &=& \frac{1}{\sqrt{3}} \phi_{2s} - \frac{1}{\sqrt{6}} \phi_{2p_x} - \frac{1}{\sqrt{2}} \phi_{2p_y} \\
\psi_4  &=& \phi_{2p_z} 
\end{eqnarray*}

\makesubproblem{Orthonormality}{condensedMatter:problemSet1:2a}
Assuming 
  that the $\phi_{2s}$ and $\phi_{2p}$ orbitals are orthogonal and normalized (i.e.\ 
  you don't need to show this), show that the $sp^2$ hybrid orbitals are also orthonormal.

\makesubproblem{Coordinates for maximum probability density}{condensedMatter:problemSet1:2b}
Find the $\phi$ and $\theta$ values for which the 
   probability density of the $\psi_1$ hybrid 
   orbital is maximized (i.e.\ find the direction in which this orbital is pointing).  

\makesubproblem{Contour plots}{condensedMatter:problemSet1:2c}
(c) Using whatever plotting package you wish (e.g. gnuplot, Matlab, or using the 
  `contour' or `contourf' functions in SciPy; and please see me 
  if you don't know how to approach this question), make two-dimensional contour 
  plots for the $\psi_1$ and $\psi_2$ hybrid orbital wave-functions, and their 
  moduli, in the $x-y$ plane 
  (that is for $\theta=\pi/2$).
Hand in the code you used to generate the contour plots, as well as a printout of 
  the plots (plots can be submitted by email if you want to submit a colour contour plot 
  and you don't have a colour printer).  

(I have put some hints for how do this using python with the Problem Set 
1 questions on the blackboard site.)

\makesubproblem{Sigma bonding}{condensedMatter:problemSet1:2d}

Bonus (for fun, will not be marked): 
    Modify your program so that two adjacent atoms have $sp^2$ hybrid 
    orbitals directed towards each other to form a $\sigma$ bond, and plot contours of the wave-function 
    in the $x-y$ plane. 
    Once you know how to do this you can do quite a lot.  It is easy for example to put atoms on 
    a honeycomb lattice, with the in-plane $sp^2$ hybrid orbitals forming covalent bonds, and the 
    resulting contour plot is a map of the wave function 
    in the plane of the carbon atoms in graphene.  Or you can put two $2p_z$ orbitals on adjacent atoms 
    on the $x$-axis and map out the wave-function of a $\pi$ orbital.  
    (In addition to the in-plane $sp^2$ hybrid orbitals graphene has $\pi$ bonds between the 
    $2p_z$ orbitals on adjacent atoms.)  Or, you can look at the electron density in anti-bonding 
    orbitals.  etc.  If you don't feel like doing this yourself, a bit of hacking around on the 
    internet will turn up many nice examples, but it's also nice to see how easy it is to do 
    it yourself. 

} % makeproblem

\makeanswer{condensedMatter:problemSet1:2}{ 
\makeSubAnswer{XXX}{condensedMatter:problemSet1:2a}

TODO.

\makeSubAnswer{XXX}{condensedMatter:problemSet1:2b}

TODO.

\makeSubAnswer{XXX}{condensedMatter:problemSet1:2c}

TODO.

\makeSubAnswer{XXX}{condensedMatter:problemSet1:2d}

TODO.
}
