%
% Copyright � 2015 Peeter Joot.  All Rights Reserved.
% Licenced as described in the file LICENSE under the root directory of this GIT repository.
%
\documentclass[]{eliblog}

\usepackage{amsmath}
\usepackage{mathpazo}

%
% shorthand for bold symbols, convenient for vectors and matrices
%
\newcommand{\Ba}[0]{\mathbf{a}}
\newcommand{\Bb}[0]{\mathbf{b}}
\newcommand{\Bc}[0]{\mathbf{c}}
\newcommand{\Bd}[0]{\mathbf{d}}
\newcommand{\Be}[0]{\mathbf{e}}
\newcommand{\Bf}[0]{\mathbf{f}}
\newcommand{\Bg}[0]{\mathbf{g}}
\newcommand{\Bh}[0]{\mathbf{h}}
\newcommand{\Bi}[0]{\mathbf{i}}
\newcommand{\Bj}[0]{\mathbf{j}}
\newcommand{\Bk}[0]{\mathbf{k}}
\newcommand{\Bl}[0]{\mathbf{l}}
\newcommand{\Bm}[0]{\mathbf{m}}
\newcommand{\Bn}[0]{\mathbf{n}}
\newcommand{\Bo}[0]{\mathbf{o}}
\newcommand{\Bp}[0]{\mathbf{p}}
\newcommand{\Bq}[0]{\mathbf{q}}
\newcommand{\Br}[0]{\mathbf{r}}
\newcommand{\Bs}[0]{\mathbf{s}}
\newcommand{\Bt}[0]{\mathbf{t}}
\newcommand{\Bu}[0]{\mathbf{u}}
\newcommand{\Bv}[0]{\mathbf{v}}
\newcommand{\Bw}[0]{\mathbf{w}}
\newcommand{\Bx}[0]{\mathbf{x}}
\newcommand{\By}[0]{\mathbf{y}}
\newcommand{\Bz}[0]{\mathbf{z}}
\newcommand{\BA}[0]{\mathbf{A}}
\newcommand{\BB}[0]{\mathbf{B}}
\newcommand{\BC}[0]{\mathbf{C}}
\newcommand{\BD}[0]{\mathbf{D}}
\newcommand{\BE}[0]{\mathbf{E}}
\newcommand{\BF}[0]{\mathbf{F}}
\newcommand{\BG}[0]{\mathbf{G}}
\newcommand{\BH}[0]{\mathbf{H}}
\newcommand{\BI}[0]{\mathbf{I}}
\newcommand{\BJ}[0]{\mathbf{J}}
\newcommand{\BK}[0]{\mathbf{K}}
\newcommand{\BL}[0]{\mathbf{L}}
\newcommand{\BM}[0]{\mathbf{M}}
\newcommand{\BN}[0]{\mathbf{N}}
\newcommand{\BO}[0]{\mathbf{O}}
\newcommand{\BP}[0]{\mathbf{P}}
\newcommand{\BQ}[0]{\mathbf{Q}}
\newcommand{\BR}[0]{\mathbf{R}}
\newcommand{\BS}[0]{\mathbf{S}}
\newcommand{\BT}[0]{\mathbf{T}}
\newcommand{\BU}[0]{\mathbf{U}}
\newcommand{\BV}[0]{\mathbf{V}}
\newcommand{\BW}[0]{\mathbf{W}}
\newcommand{\BX}[0]{\mathbf{X}}
\newcommand{\BY}[0]{\mathbf{Y}}
\newcommand{\BZ}[0]{\mathbf{Z}}

\newcommand{\Bzero}[0]{\mathbf{0}}
\newcommand{\Btheta}[0]{\boldsymbol{\theta}}
\newcommand{\Btau}[0]{\boldsymbol{\tau}}
\newcommand{\Bomega}[0]{\boldsymbol{\omega}}

%
% shorthand for unit vectors
%
\newcommand{\acap}[0]{\hat{\Ba}}
\newcommand{\bcap}[0]{\hat{\Bb}}
\newcommand{\ccap}[0]{\hat{\Bc}}
\newcommand{\dcap}[0]{\hat{\Bd}}
\newcommand{\ecap}[0]{\hat{\Be}}
\newcommand{\fcap}[0]{\hat{\Bf}}
\newcommand{\gcap}[0]{\hat{\Bg}}
\newcommand{\hcap}[0]{\hat{\Bh}}
\newcommand{\icap}[0]{\hat{\Bi}}
\newcommand{\jcap}[0]{\hat{\Bj}}
\newcommand{\kcap}[0]{\hat{\Bk}}
\newcommand{\lcap}[0]{\hat{\Bl}}
\newcommand{\mcap}[0]{\hat{\Bm}}
\newcommand{\ncap}[0]{\hat{\Bn}}
\newcommand{\ocap}[0]{\hat{\Bo}}
\newcommand{\pcap}[0]{\hat{\Bp}}
\newcommand{\qcap}[0]{\hat{\Bq}}
\newcommand{\rcap}[0]{\hat{\Br}}
\newcommand{\scap}[0]{\hat{\Bs}}
\newcommand{\tcap}[0]{\hat{\Bt}}
\newcommand{\ucap}[0]{\hat{\Bu}}
\newcommand{\vcap}[0]{\hat{\Bv}}
\newcommand{\wcap}[0]{\hat{\Bw}}
\newcommand{\xcap}[0]{\hat{\Bx}}
\newcommand{\ycap}[0]{\hat{\By}}
\newcommand{\zcap}[0]{\hat{\Bz}}
\newcommand{\thetacap}[0]{\hat{\Btheta}}

%
% to write R^n and C^n in a distinguishable fashion.  Perhaps change this
% to the double lined characters upon figuring out how to do so.
%
\newcommand{\C}[1]{$\mathbb{C}^{#1}$}
\newcommand{\R}[1]{$\mathbb{R}^{#1}$}

%
% various generally useful helpers
%

% derivative of #1 wrt. #2:
\newcommand{\D}[2] {\frac {d#2} {d#1}}

\newcommand{\inv}[1]{\frac{1}{#1}}
\newcommand{\cross}[0]{\times}

\newcommand{\abs}[1]{\lvert{#1}\rvert}
\newcommand{\norm}[1]{\lVert{#1}\rVert}
\newcommand{\innerprod}[2]{\langle{#1}, {#2}\rangle}
\newcommand{\dotprod}[2]{{#1} \cdot {#2}}
\newcommand{\bdotprod}[2]{\left({#1} \cdot {#2}\right)}
\newcommand{\crossprod}[2]{{#1} \cross {#2}}
\newcommand{\tripleprod}[3]{\dotprod{\left(\crossprod{#1}{#2}\right)}{#3}}

\DeclareMathOperator{\Proj}{Proj}
\DeclareMathOperator{\Span}{span}
\DeclareMathOperator{\Sgn}{sgn}
\DeclareMathOperator{\Area}{Area}
\DeclareMathOperator{\Volume}{Volume}

%
% A few miscellaneous things specific to this document
%
\newcommand{\crossop}[1]{\crossprod{#1}{}}

% R2 vector.
\newcommand{\VectorTwo}[2]{
\begin{bmatrix}
 {#1} \\
 {#2}
\end{bmatrix}
}

\newcommand{\VectorN}[1]{
\begin{bmatrix}
{#1}_1 \\
{#1}_2 \\
\vdots \\
{#1}_N \\
\end{bmatrix}
}

\newcommand{\DETuvij}[4]{
\begin{vmatrix}
 {#1}_{#3} & {#1}_{#4} \\
 {#2}_{#3} & {#2}_{#4}
\end{vmatrix}
}

\newcommand{\DETuvwijk}[6]{
\begin{vmatrix}
 {#1}_{#4} & {#1}_{#5} & {#1}_{#6} \\
 {#2}_{#4} & {#2}_{#5} & {#2}_{#6} \\
 {#3}_{#4} & {#3}_{#5} & {#3}_{#6}
\end{vmatrix}
}

\newcommand{\DETuvwxijkl}[8]{
\begin{vmatrix}
 {#1}_{#5} & {#1}_{#6} & {#1}_{#7} & {#1}_{#8} \\
 {#2}_{#5} & {#2}_{#6} & {#2}_{#7} & {#2}_{#8} \\
 {#3}_{#5} & {#3}_{#6} & {#3}_{#7} & {#3}_{#8} \\
 {#4}_{#5} & {#4}_{#6} & {#4}_{#7} & {#4}_{#8} \\
\end{vmatrix}
}

%\newcommand{\DETuvwxyijklm}[10]{
%\begin{vmatrix}
% {#1}_{#6} & {#1}_{#7} & {#1}_{#8} & {#1}_{#9} & {#1}_{#10} \\
% {#2}_{#6} & {#2}_{#7} & {#2}_{#8} & {#2}_{#9} & {#2}_{#10} \\
% {#3}_{#6} & {#3}_{#7} & {#3}_{#8} & {#3}_{#9} & {#3}_{#10} \\
% {#4}_{#6} & {#4}_{#7} & {#4}_{#8} & {#4}_{#9} & {#4}_{#10} \\
% {#5}_{#6} & {#5}_{#7} & {#5}_{#8} & {#5}_{#9} & {#5}_{#10}
%\end{vmatrix}
%}

% R3 vector.
\newcommand{\VectorThree}[3]{
\begin{bmatrix}
 {#1} \\
 {#2} \\
 {#3}
\end{bmatrix}
}



\author{Peeter Joot}
\email{peeter.joot@gmail.com}

%\documentclass[]{eliblogwidescreen}

\usepackage{amsmath}
\usepackage{mathpazo}

%
% shorthand for bold symbols, convenient for vectors and matrices
%
\newcommand{\Ba}[0]{\mathbf{a}}
\newcommand{\Bb}[0]{\mathbf{b}}
\newcommand{\Bc}[0]{\mathbf{c}}
\newcommand{\Bd}[0]{\mathbf{d}}
\newcommand{\Be}[0]{\mathbf{e}}
\newcommand{\Bf}[0]{\mathbf{f}}
\newcommand{\Bg}[0]{\mathbf{g}}
\newcommand{\Bh}[0]{\mathbf{h}}
\newcommand{\Bi}[0]{\mathbf{i}}
\newcommand{\Bj}[0]{\mathbf{j}}
\newcommand{\Bk}[0]{\mathbf{k}}
\newcommand{\Bl}[0]{\mathbf{l}}
\newcommand{\Bm}[0]{\mathbf{m}}
\newcommand{\Bn}[0]{\mathbf{n}}
\newcommand{\Bo}[0]{\mathbf{o}}
\newcommand{\Bp}[0]{\mathbf{p}}
\newcommand{\Bq}[0]{\mathbf{q}}
\newcommand{\Br}[0]{\mathbf{r}}
\newcommand{\Bs}[0]{\mathbf{s}}
\newcommand{\Bt}[0]{\mathbf{t}}
\newcommand{\Bu}[0]{\mathbf{u}}
\newcommand{\Bv}[0]{\mathbf{v}}
\newcommand{\Bw}[0]{\mathbf{w}}
\newcommand{\Bx}[0]{\mathbf{x}}
\newcommand{\By}[0]{\mathbf{y}}
\newcommand{\Bz}[0]{\mathbf{z}}
\newcommand{\BA}[0]{\mathbf{A}}
\newcommand{\BB}[0]{\mathbf{B}}
\newcommand{\BC}[0]{\mathbf{C}}
\newcommand{\BD}[0]{\mathbf{D}}
\newcommand{\BE}[0]{\mathbf{E}}
\newcommand{\BF}[0]{\mathbf{F}}
\newcommand{\BG}[0]{\mathbf{G}}
\newcommand{\BH}[0]{\mathbf{H}}
\newcommand{\BI}[0]{\mathbf{I}}
\newcommand{\BJ}[0]{\mathbf{J}}
\newcommand{\BK}[0]{\mathbf{K}}
\newcommand{\BL}[0]{\mathbf{L}}
\newcommand{\BM}[0]{\mathbf{M}}
\newcommand{\BN}[0]{\mathbf{N}}
\newcommand{\BO}[0]{\mathbf{O}}
\newcommand{\BP}[0]{\mathbf{P}}
\newcommand{\BQ}[0]{\mathbf{Q}}
\newcommand{\BR}[0]{\mathbf{R}}
\newcommand{\BS}[0]{\mathbf{S}}
\newcommand{\BT}[0]{\mathbf{T}}
\newcommand{\BU}[0]{\mathbf{U}}
\newcommand{\BV}[0]{\mathbf{V}}
\newcommand{\BW}[0]{\mathbf{W}}
\newcommand{\BX}[0]{\mathbf{X}}
\newcommand{\BY}[0]{\mathbf{Y}}
\newcommand{\BZ}[0]{\mathbf{Z}}

\newcommand{\Bzero}[0]{\mathbf{0}}
\newcommand{\Btheta}[0]{\boldsymbol{\theta}}
\newcommand{\Btau}[0]{\boldsymbol{\tau}}
\newcommand{\Bomega}[0]{\boldsymbol{\omega}}

%
% shorthand for unit vectors
%
\newcommand{\acap}[0]{\hat{\Ba}}
\newcommand{\bcap}[0]{\hat{\Bb}}
\newcommand{\ccap}[0]{\hat{\Bc}}
\newcommand{\dcap}[0]{\hat{\Bd}}
\newcommand{\ecap}[0]{\hat{\Be}}
\newcommand{\fcap}[0]{\hat{\Bf}}
\newcommand{\gcap}[0]{\hat{\Bg}}
\newcommand{\hcap}[0]{\hat{\Bh}}
\newcommand{\icap}[0]{\hat{\Bi}}
\newcommand{\jcap}[0]{\hat{\Bj}}
\newcommand{\kcap}[0]{\hat{\Bk}}
\newcommand{\lcap}[0]{\hat{\Bl}}
\newcommand{\mcap}[0]{\hat{\Bm}}
\newcommand{\ncap}[0]{\hat{\Bn}}
\newcommand{\ocap}[0]{\hat{\Bo}}
\newcommand{\pcap}[0]{\hat{\Bp}}
\newcommand{\qcap}[0]{\hat{\Bq}}
\newcommand{\rcap}[0]{\hat{\Br}}
\newcommand{\scap}[0]{\hat{\Bs}}
\newcommand{\tcap}[0]{\hat{\Bt}}
\newcommand{\ucap}[0]{\hat{\Bu}}
\newcommand{\vcap}[0]{\hat{\Bv}}
\newcommand{\wcap}[0]{\hat{\Bw}}
\newcommand{\xcap}[0]{\hat{\Bx}}
\newcommand{\ycap}[0]{\hat{\By}}
\newcommand{\zcap}[0]{\hat{\Bz}}
\newcommand{\thetacap}[0]{\hat{\Btheta}}

%
% to write R^n and C^n in a distinguishable fashion.  Perhaps change this
% to the double lined characters upon figuring out how to do so.
%
\newcommand{\C}[1]{$\mathbb{C}^{#1}$}
\newcommand{\R}[1]{$\mathbb{R}^{#1}$}

%
% various generally useful helpers
%

% derivative of #1 wrt. #2:
\newcommand{\D}[2] {\frac {d#2} {d#1}}

\newcommand{\inv}[1]{\frac{1}{#1}}
\newcommand{\cross}[0]{\times}

\newcommand{\abs}[1]{\lvert{#1}\rvert}
\newcommand{\norm}[1]{\lVert{#1}\rVert}
\newcommand{\innerprod}[2]{\langle{#1}, {#2}\rangle}
\newcommand{\dotprod}[2]{{#1} \cdot {#2}}
\newcommand{\bdotprod}[2]{\left({#1} \cdot {#2}\right)}
\newcommand{\crossprod}[2]{{#1} \cross {#2}}
\newcommand{\tripleprod}[3]{\dotprod{\left(\crossprod{#1}{#2}\right)}{#3}}

\DeclareMathOperator{\Proj}{Proj}
\DeclareMathOperator{\Span}{span}
\DeclareMathOperator{\Sgn}{sgn}
\DeclareMathOperator{\Area}{Area}
\DeclareMathOperator{\Volume}{Volume}

%
% A few miscellaneous things specific to this document
%
\newcommand{\crossop}[1]{\crossprod{#1}{}}

% R2 vector.
\newcommand{\VectorTwo}[2]{
\begin{bmatrix}
 {#1} \\
 {#2}
\end{bmatrix}
}

\newcommand{\VectorN}[1]{
\begin{bmatrix}
{#1}_1 \\
{#1}_2 \\
\vdots \\
{#1}_N \\
\end{bmatrix}
}

\newcommand{\DETuvij}[4]{
\begin{vmatrix}
 {#1}_{#3} & {#1}_{#4} \\
 {#2}_{#3} & {#2}_{#4}
\end{vmatrix}
}

\newcommand{\DETuvwijk}[6]{
\begin{vmatrix}
 {#1}_{#4} & {#1}_{#5} & {#1}_{#6} \\
 {#2}_{#4} & {#2}_{#5} & {#2}_{#6} \\
 {#3}_{#4} & {#3}_{#5} & {#3}_{#6}
\end{vmatrix}
}

\newcommand{\DETuvwxijkl}[8]{
\begin{vmatrix}
 {#1}_{#5} & {#1}_{#6} & {#1}_{#7} & {#1}_{#8} \\
 {#2}_{#5} & {#2}_{#6} & {#2}_{#7} & {#2}_{#8} \\
 {#3}_{#5} & {#3}_{#6} & {#3}_{#7} & {#3}_{#8} \\
 {#4}_{#5} & {#4}_{#6} & {#4}_{#7} & {#4}_{#8} \\
\end{vmatrix}
}

%\newcommand{\DETuvwxyijklm}[10]{
%\begin{vmatrix}
% {#1}_{#6} & {#1}_{#7} & {#1}_{#8} & {#1}_{#9} & {#1}_{#10} \\
% {#2}_{#6} & {#2}_{#7} & {#2}_{#8} & {#2}_{#9} & {#2}_{#10} \\
% {#3}_{#6} & {#3}_{#7} & {#3}_{#8} & {#3}_{#9} & {#3}_{#10} \\
% {#4}_{#6} & {#4}_{#7} & {#4}_{#8} & {#4}_{#9} & {#4}_{#10} \\
% {#5}_{#6} & {#5}_{#7} & {#5}_{#8} & {#5}_{#9} & {#5}_{#10}
%\end{vmatrix}
%}

% R3 vector.
\newcommand{\VectorThree}[3]{
\begin{bmatrix}
 {#1} \\
 {#2} \\
 {#3}
\end{bmatrix}
}



\author{Peeter Joot}
\email{peeter.joot@gmail.com}


\chapter{PHY450H1S.  Relativistic Electrodynamics Lecture 2 (Taught by Prof. Erich Poppitz).  Spacetime, events, worldlines, proper time, invariance.}
\label{chap:relativisticElectrodynamicsL2}
%\useCCL
\blogpage{http://sites.google.com/site/peeterjoot/math2011/relativisticElectrodynamicsL2.pdf}
\date{Jan 12, 2011}
\revisionInfo{relativisticElectrodynamicsL2.tex}

%\beginArtWithToc
\beginArtNoToc

\section{Reading.}

No reading from \cite{landau1980classical} appears to have been assigned, but relevant stuff can be found in chapter 1.

From \href{http://www.physics.utoronto.ca/~poppitz/e-poppitz/PHY450_files/RelEM12-26.pdf}{Professor Poppitz's lecture notes}, we have reading: pp.12-26: spacetime, spacetime points, worldlines, interval (12-14); invariance of infinitesimal intervals (15-17);  geometry of spacetime, lightlike, spacelike, timelike intervals, and worldlines (18-22); proper time (23-24); invariance of finite intervals (25-26).

\section{Followup for questions from last lecture.}

Yes we have speed of light different in media.  Example, speed of light in water is $3/4$ vacuum speed due to high index of refraction.  Also note that we can have effects like an electron moving in water can constantly emit light.  This is called Cerenkov radiation.

\section{Einstein's relativity principle}

\begin{enumerate}

\item Replace Galilean transformations between coordinates in differential inertial frames with Lorentz transforms between $(\Bx, t)$.  Postulate that these constitute the symmetries of physics.  Recall that Galilean transformations are symmetries of the laws of non-relativistic physics.  

Comment made that the symmetries impose the dynamics, and the symmetries provided the form of the Lagrangian in classical physics.  Go back and revisit this.

\item Speed of light $c$ is the same in all inertial frames.  Phrased in this form, relativity leads to ``relativity of simultaneity''.

PICTURE: Three people on a platform, at positions $1,3,2$, all with equidistant separation.  This stationary frame is labeled $O$.  1 and 2 flash light signals at the same time and in frame $O$ the reception of the light signal by 3 is observed as arriving at 3 simultaneously.

Now introduce a moving frame with origin $O'$ moving along the positive x axis.  To a stationary observer in $O'$ the three guys are seen to be moving in the $-x$ direction.  The middle guy (3) is eventually going to be seen to receive the light signal by this $O'$ observer, but less time is required for the light to get from 1 to 3, and more time is required for the light to get from 2 to 1 (3 is moving away from the light according to the $O'$ observer).  Because the speed of light is perceived as constant for all observers, the perception is then that the light must arrive at 3 at different times.

This is very non-intuitive since we are implicitly trained by our surroundings that Galilean transformations govern mechanical behavior.

In $O$, 1 and 2 send light signals simultaneously while in $O'$ 1 sends light later than 2.  The conclusion, rather surprisingly compared to intuition, is that simultaneity is relative.

\end{enumerate}

\section{Spacetime}

We will need to develop some tools to work with these concepts in a concrete fashion.  It is convenient to combine space \R{3} and time \R{1} into a 4d ``spacetime''.  In \cite{landau1980classical} this is called fictitious spacetime for reasons that are not clear.  Points in this space are also called ``events'', or ``spacetime points'', or ``world point''.  The ``world line'' is the trajectory for a particle in spacetime.

PICTURE: \R{3} represented as a plane, and $t$ up.  For every point we can plot an $\Bx(t)$ in this combined space.

\section{Spacetime intervals for light like behaviour.}

Consider two frames, one moving along the x-axis at a (constant) rate not yet specified.

``events'' have coordinates $(t, \Bx)$ in $O$ and $(t', \Bx')$ in $O'$.  Because we now have to model the mathematics without a notion of simultaneity, we must now also introduce different time coordinates $t$, and $t'$ in the two frames.  

Let's imagine that at at time $t_1$ light is emitted at $\Bx_1$, and at time $t_2$ this light is absorbed.  Our space time events are then $(t_1, \Bx_1)$ and $(t_2, \Bx_2)$.  In the $O$ frame, the light will go a distance $c(t_2 - t_1)$.  This same distance can also be expressed as

\begin{equation}\label{eqn:relativisticElectrodynamicsL1:10}
\sqrt{ (\Bx_1 - \Bx_2)^2}.
\end{equation}

These are equal.  It is convenient to work without the square roots, so we write

\begin{equation}\label{eqn:relativisticElectrodynamicsL1:20}
(\Bx_1 - \Bx_2)^2 = c^2 (t_2 - t_1)^2
\end{equation}

Or

\begin{equation}\label{eqn:relativisticElectrodynamicsL1:30}
c^2 (t_2 - t_1)^2 - (\Bx_1 - \Bx_2)^2 =
c^2 (t_2 - t_1)^2 
- (x_1 - x_2)^2
- (y_1 - y_2)^2
- (z_1 - z_2)^2 = 0.
\end{equation}

We can repeat the same argument for the primed frame.  In this frame, at time $t_1'$ light is emitted at $\Bx_1'$, and at time $t_2'$ this light is absorbed.  Our space time events in this frame are then $(t_1', \Bx_1')$ and $(t_2', \Bx_2')$.  As above, in this $O'$ frame, the light will go a distance $c(t_2' - t_1')$, with a similar Euclidean distance involving $\Bx_1'$ and $\Bx_2'$.  That is

\begin{equation}\label{eqn:relativisticElectrodynamicsL1:40}
c^2 (t_2' - t_1')^2 - (\Bx_1' - \Bx_2')^2 =
c^2 (t_2' - t_1')^2 
- (x_1' - x_2')^2
- (y_1' - y_2')^2
- (z_1' - z_2')^2 = 0.
\end{equation}

We get zero for this quantity in any inertial frame 1.  This quantity is found to be very important, and want to give this a label.  We call this the ``interval'', or the ``spacetime interval'', and write this as follows:

\begin{equation}\label{eqn:relativisticElectrodynamicsL1:50}
s_{12}^2 = c^2 (t_2 - t_1)^2 - (\Br_2 - \Br_1)^2
\end{equation}

This is a quantity calculated between any two spacetime points with coordinates $(t_2, \Br_2)$ and $(t_1, \Br_1)$ in some frame.

So far we have argued that $c$ being the same in any two frames implies that spacetime events ``separated by a zero interval'' in one frame are ``separated by a zero interval'' in any other frame.

\section{Invariance of infinitesimal intervals.}

For events that are infinitesimally close to each other.  i.e. $t_2 - t_1$ and $\Br_2 -\Br_1$ are small (infinitesimal), it is convient to denote $t_2 - t_1$ and $\Br_2 - \Br_1$ by $dt$ and $d\Br$ respectively.  We can then define

\begin{equation}\label{eqn:relativisticElectrodynamicsL1:60}
ds_{12}^2 = c^2 dt^2 - d\Br^2,
\end{equation}

or
\begin{equation}\label{eqn:relativisticElectrodynamicsL1:60b}
ds= \sqrt{c^2 dt^2 - d\Br^2}.
\end{equation}

We will use this a lot.

We have learned that if $s_{12} = 0$ in one frame, then $s_{12}' = 0$ in any other frame.  We generally expect that there is a relation $s_{12}' = F(s_12)$ between the intervals in two frames.  So far we have learned that $F(0) = 0$. 

Let's now consider the case where both of these intervals are infinitesimal.  Then we can write

\begin{equation}\label{eqn:relativisticElectrodynamicsL1:70}
ds_{12}' = F(ds_{12}) = F(0) + F'(0) ds_{12} + \cdots = F'(0) ds_{12} + \cdots.
\end{equation}

We will neglect terms $O(ds_{12})^2$ and higher.  Thus equality of zero intervals between two frames implies that 

\begin{equation}\label{eqn:relativisticElectrodynamicsL1:80}
ds_{12}' \propto ds_{12}.
\end{equation}

Now we must invoke an assumption (principle) of homogeneity of time and space and isotropy of space.  This interval should not depend on where these events take place, or on the time that the measurements were performed.  If this is the case then we conclude that the proportionality constant relating the two intervals is not a function of position or space.  We argue that this proportionality can then only be a function of the (absolute) relative speed between the frames.

We write this as
\begin{equation}\label{eqn:relativisticElectrodynamicsL1:90}
ds_{12}' = F(v_{12}) ds_{12}
\end{equation}

This argument can be turned around and we say that $ds_{12} = \tilde{F}(v_{12}) ds_{12}'$.  Thus $\tilde{F} = F$, because there is no distinction between $O$ and $O'$.  We want to conclude that 

\begin{equation}\label{eqn:relativisticElectrodynamicsL1:100}
ds_{12} = F(v_{12}) ds_{12}' = F(v_{12}) \tilde{F}(v_{12}) ds_{12}
\end{equation}

and then conclude that $F = \tilde{F} = 1$.  This argument is to be continued.  To complete this conclusion we will need to perform some additional math, once we cover finite intervals.

%\section{Geometry of spacetime: lightlike, spacelike, timelike intervals}
%
%\section{proper time}
%
%\section{invariance of finite intervals}

\EndArticle
