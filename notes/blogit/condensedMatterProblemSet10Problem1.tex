%
% Copyright � 2013 Peeter Joot.  All Rights Reserved.
% Licenced as described in the file LICENSE under the root directory of this GIT repository.
%
\makeproblem{Drude fromula for the conductivity}{condensedMatter:problemSet10:1}{ 
The density of point-like impurities in a metal can be characterized by a �mean free path,� $l_\nought$, of the conduction electrons, this being the distance between scattering centres that randomize the velocity.

\makesubproblem{}{condensedMatter:problemSet10:1a}

Assuming that electrons travel at the Fermi velocity between scattering centres in a
free electron metal, show that the Drude formula for the conductivity can be rewritten
as:

\begin{equation}\label{eqn:condensedMatterProblemSet10Problem1:20}
\sigma = \frac{\kF^2 e^2 l_\nought}{3 \Hbar \pi^2}
\end{equation}

\makesubproblem{}{condensedMatter:problemSet10:1b}
A sample of copper has a residual resistivity below $1 \Unit{K}$ of $10^{-8} \Omega \Unit{m}$. (Note: resistivity $\rho = 1/\sigma$.) Treating copper as a free electron metal with a spherical Fermi surface accomodating one charge carrier per copper atom, estimate $l_\nought$ below $1 \Unit{K}$ for this sample of copper. (Free-electron parameters of copper are given on page 141 of Ibach and Luth.)

\makesubproblem{}{condensedMatter:problemSet10:1c}
Calculate the mean scattering time $\tau$ for this sample of copper below $1 \Unit{K}$.

} % makeproblem

\makeanswer{condensedMatter:problemSet10:1}{ 
\makeSubAnswer{}{condensedMatter:problemSet10:1a}

The Fermi velocity is 

\begin{equation}\label{eqn:condensedMatterProblemSet10Problem1:40}
\vF = \frac{\pF}{m} = \frac{ \Hbar \kF}{m},
\end{equation}

so that the `mean free path' is

\begin{equation}\label{eqn:condensedMatterProblemSet10Problem1:60}
l_\nought = \vF \tau = \frac{ \Hbar \kF \tau}{m}.
\end{equation}

Putting these all together, the conductivity as given by \eqnref{eqn:condensedMatterProblemSet10Problem1:20} is

\begin{dmath}\label{eqn:condensedMatterProblemSet10Problem1:n}
\sigma 
= \frac{\kF^2 e^2}{3 \cancel{\Hbar} \pi^2} \frac{ \cancel{\Hbar} \kF \tau}{m}
= \frac{\kF^3}{3 \pi^2} \frac{ e^2 \tau}{m}
= n \frac{ e^2 \tau}{m},
\end{dmath}

which recovers the form we derived in class.

\makeSubAnswer{}{condensedMatter:problemSet10:1b}

TODO.
\makeSubAnswer{}{condensedMatter:problemSet10:1c}

TODO.
}
