%
% Copyright � 2013 Peeter Joot.  All Rights Reserved.
% Licenced as described in the file LICENSE under the root directory of this GIT repository.
%
\makeproblem{Drude fromula for the conductivity}{condensedMatter:problemSet10:1}{ 

The density of point-like impurities in a
metal can be characterized by a �mean free path,� l ? , of the conduction electrons, this
being the distance between scattering centres that randomize the velocity.

\makesubproblem{}{condensedMatter:problemSet10:1a}

Assuming that electrons travel at the Fermi velocity between scattering centres in a
free electron metal, show that the Drude formula for the conductivity can be rewritten
as:

%s =
%k 2
%F e
%2 l ?
%3� hp 2

\makesubproblem{}{condensedMatter:problemSet10:1b}
A sample of copper has a residual resistivity below 1 K of 10 -8 ?m. (Note: resistivity
? = 1/s.) Treating copper as a free electron metal with a spherical Fermi surface
accomodating one charge carrier per copper atom, estimate l ? below 1 K for this sample
of copper. (Free-electron parameters of copper are given on page 141 of Ibach and Luth.)

\makesubproblem{}{condensedMatter:problemSet10:1c}
Calculate the mean scattering time t for this sample of copper below 1 K.

} % makeproblem

\makeanswer{condensedMatter:problemSet10:1}{ 
\makeSubAnswer{}{condensedMatter:problemSet10:1a}

TODO.
\makeSubAnswer{}{condensedMatter:problemSet10:1b}

TODO.
\makeSubAnswer{}{condensedMatter:problemSet10:1c}

TODO.
}
