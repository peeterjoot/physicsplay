%
% Copyright � 2015 Peeter Joot.  All Rights Reserved.
% Licenced as described in the file LICENSE under the root directory of this GIT repository.
%
\makeproblem{Long thin wire dipoles}{advancedantenna:problemSet3:2}{ 

\makesubproblem{}{advancedantenna:problemSet3:2a}
On a single diagram, plot the polar patterns for 
\( l = 0.5 \lambda, l = 1.0 \lambda, l = 1.25 \lambda \) and \( l = 2.0 \lambda \) long thin wire dipole antennas.

\makesubproblem{}{advancedantenna:problemSet3:2b}
Use numerical integration to calculate the maximum directivity for each dipole.
Make a table with your results. Which length corresponds to the highest
directivity?

\makesubproblem{}{advancedantenna:problemSet3:2c}
Use numerical integration to calculate the radiation resisistance of the  \( l = 1.25 \lambda \) dipole. Do you expect this dipole to be capacitive or inductive?
} % makeproblem

\makeanswer{advancedantenna:problemSet3:2}{ 
\makeSubAnswer{}{advancedantenna:problemSet3:2a}

Assuming a \( \zcap \) oriented dipole, in the far field, the electric field is

\begin{dmath}\label{eqn:advancedantennaProblemSet3Problem2:20}
E_\theta \approx j \eta \frac{I_0 e^{-j k r}}{ 2 \pi r } 
\lr{ 
\frac{\cos\lr{ \frac{k l}{2} \cos\theta} - \cos\lr{\frac{kl}{2}}}{\sin\theta}
}.
\end{dmath}

%an average Poynting vector of
%
%\begin{dmath}\label{eqn:advancedantennaProblemSet3Problem2:40}
%\BW_{\textrm{av}} = 
%\rcap 
%\eta \frac{\Abs{I_0}^2}{8 \pi^2 r^2}
%\lr{ 
%\frac{\cos\lr{ \frac{k l}{2} \cos\theta} - \cos\lr{\frac{kl}{2}}}{\sin\theta}
%}^2.
%\end{dmath}
%
Writing 
\( l = \alpha \lambda \), and noting that 
the magnetic field is \( H_\phi \approx E_\theta/\eta \), 
the radiation intensity \( U = r^2 W_{\textrm{av}} \) is

\begin{dmath}\label{eqn:advancedantennaProblemSet3Problem2:60}
U = 
\eta \frac{\Abs{I_0}^2}{8 \pi^2}
\lr{ 
\frac{\cos\lr{ \pi \alpha \cos\theta} - \cos\lr{ \pi \alpha }}{\sin\theta}
}^2.
\end{dmath}

In \cref{fig:longDipole:longDipoleFig1} \( F(\theta) = 8 \pi^2 U /\eta \Abs{I_0}^2 \) is plotted for \( \alpha \in \setlr{0.5, 1, 1.25, 2.0 } \).  For \( \alpha = 1.25 \) some very small side lobes are just barely visible.  For \( \alpha = 2 \) the single lobe directivity is lost, and a significant split of the radiation field along two different directions can be observed.  These individual features can be explored more easily in 
\href{http://goo.gl/OjK4oc}{http://goo.gl/OjK4oc} which provides an interactive control for varying the \( l/\lambda \) ratio.

\imageFigure{../../figures/ece1229/longDipoleFig1}{Polar plot of radiation intensities for some electric z-axis oriented dipoles.}{fig:longDipole:longDipoleFig1}{0.3}

It is much more satisfactory to view these in a three dimensional plot as in \nbref{ps3:longDipoleInteractiveLength.nb}, and \cref{fig:longDipoleLequals2Lambda:longDipoleLequals2LambdaFig1}, but such a visualization does not work well for overlaid intensity patterns.

\imageFigure{../../figures/ece1229/longDipoleLequals2LambdaFig1}{Double wavelength radiation intensity.}{fig:longDipoleLequals2Lambda:longDipoleLequals2LambdaFig1}{0.3}

\makeSubAnswer{}{advancedantenna:problemSet3:2b}

The directivity is given by 

\begin{dmath}\label{eqn:advancedantennaProblemSet3Problem2:n}
D_0 = \frac{4 \pi \evalbar{F(\theta)}{\textrm{max}}}{2 \pi \int_0^\pi F(\theta) \sin \theta d\theta}.
\end{dmath}

These values, calculated in \nbref{ps3:directivityLongDipole.nb} using the Mathematica functions NMaximize and NIntegrate, are

\captionedTable{Directivities.}{tab:advancedantennaProblemSet3Problem2:10}{
\begin{tabular}{|l||l|l|l|l|}
\hline 
\(\alpha\) & 0.5 & 1 & 1.25 & 2 \\
\hline 
\(D_0\) & 1.64092 & 2.411 & 3.28248 & 2.52856 \\
\hline
\end{tabular}
}

The largest directivity for these specific values of \( l \) is found at \( l = 1.25 \lambda \).

\makeSubAnswer{}{advancedantenna:problemSet3:2c}

TODO.
}
