%
% Copyright � 2015 Peeter Joot.  All Rights Reserved.
% Licenced as described in the file LICENSE under the root directory of this GIT repository.
%
\documentclass[]{eliblog}

\usepackage{amsmath}
\usepackage{mathpazo}

%
% shorthand for bold symbols, convenient for vectors and matrices
%
\newcommand{\Ba}[0]{\mathbf{a}}
\newcommand{\Bb}[0]{\mathbf{b}}
\newcommand{\Bc}[0]{\mathbf{c}}
\newcommand{\Bd}[0]{\mathbf{d}}
\newcommand{\Be}[0]{\mathbf{e}}
\newcommand{\Bf}[0]{\mathbf{f}}
\newcommand{\Bg}[0]{\mathbf{g}}
\newcommand{\Bh}[0]{\mathbf{h}}
\newcommand{\Bi}[0]{\mathbf{i}}
\newcommand{\Bj}[0]{\mathbf{j}}
\newcommand{\Bk}[0]{\mathbf{k}}
\newcommand{\Bl}[0]{\mathbf{l}}
\newcommand{\Bm}[0]{\mathbf{m}}
\newcommand{\Bn}[0]{\mathbf{n}}
\newcommand{\Bo}[0]{\mathbf{o}}
\newcommand{\Bp}[0]{\mathbf{p}}
\newcommand{\Bq}[0]{\mathbf{q}}
\newcommand{\Br}[0]{\mathbf{r}}
\newcommand{\Bs}[0]{\mathbf{s}}
\newcommand{\Bt}[0]{\mathbf{t}}
\newcommand{\Bu}[0]{\mathbf{u}}
\newcommand{\Bv}[0]{\mathbf{v}}
\newcommand{\Bw}[0]{\mathbf{w}}
\newcommand{\Bx}[0]{\mathbf{x}}
\newcommand{\By}[0]{\mathbf{y}}
\newcommand{\Bz}[0]{\mathbf{z}}
\newcommand{\BA}[0]{\mathbf{A}}
\newcommand{\BB}[0]{\mathbf{B}}
\newcommand{\BC}[0]{\mathbf{C}}
\newcommand{\BD}[0]{\mathbf{D}}
\newcommand{\BE}[0]{\mathbf{E}}
\newcommand{\BF}[0]{\mathbf{F}}
\newcommand{\BG}[0]{\mathbf{G}}
\newcommand{\BH}[0]{\mathbf{H}}
\newcommand{\BI}[0]{\mathbf{I}}
\newcommand{\BJ}[0]{\mathbf{J}}
\newcommand{\BK}[0]{\mathbf{K}}
\newcommand{\BL}[0]{\mathbf{L}}
\newcommand{\BM}[0]{\mathbf{M}}
\newcommand{\BN}[0]{\mathbf{N}}
\newcommand{\BO}[0]{\mathbf{O}}
\newcommand{\BP}[0]{\mathbf{P}}
\newcommand{\BQ}[0]{\mathbf{Q}}
\newcommand{\BR}[0]{\mathbf{R}}
\newcommand{\BS}[0]{\mathbf{S}}
\newcommand{\BT}[0]{\mathbf{T}}
\newcommand{\BU}[0]{\mathbf{U}}
\newcommand{\BV}[0]{\mathbf{V}}
\newcommand{\BW}[0]{\mathbf{W}}
\newcommand{\BX}[0]{\mathbf{X}}
\newcommand{\BY}[0]{\mathbf{Y}}
\newcommand{\BZ}[0]{\mathbf{Z}}

\newcommand{\Bzero}[0]{\mathbf{0}}
\newcommand{\Btheta}[0]{\boldsymbol{\theta}}
\newcommand{\Btau}[0]{\boldsymbol{\tau}}
\newcommand{\Bomega}[0]{\boldsymbol{\omega}}

%
% shorthand for unit vectors
%
\newcommand{\acap}[0]{\hat{\Ba}}
\newcommand{\bcap}[0]{\hat{\Bb}}
\newcommand{\ccap}[0]{\hat{\Bc}}
\newcommand{\dcap}[0]{\hat{\Bd}}
\newcommand{\ecap}[0]{\hat{\Be}}
\newcommand{\fcap}[0]{\hat{\Bf}}
\newcommand{\gcap}[0]{\hat{\Bg}}
\newcommand{\hcap}[0]{\hat{\Bh}}
\newcommand{\icap}[0]{\hat{\Bi}}
\newcommand{\jcap}[0]{\hat{\Bj}}
\newcommand{\kcap}[0]{\hat{\Bk}}
\newcommand{\lcap}[0]{\hat{\Bl}}
\newcommand{\mcap}[0]{\hat{\Bm}}
\newcommand{\ncap}[0]{\hat{\Bn}}
\newcommand{\ocap}[0]{\hat{\Bo}}
\newcommand{\pcap}[0]{\hat{\Bp}}
\newcommand{\qcap}[0]{\hat{\Bq}}
\newcommand{\rcap}[0]{\hat{\Br}}
\newcommand{\scap}[0]{\hat{\Bs}}
\newcommand{\tcap}[0]{\hat{\Bt}}
\newcommand{\ucap}[0]{\hat{\Bu}}
\newcommand{\vcap}[0]{\hat{\Bv}}
\newcommand{\wcap}[0]{\hat{\Bw}}
\newcommand{\xcap}[0]{\hat{\Bx}}
\newcommand{\ycap}[0]{\hat{\By}}
\newcommand{\zcap}[0]{\hat{\Bz}}
\newcommand{\thetacap}[0]{\hat{\Btheta}}

%
% to write R^n and C^n in a distinguishable fashion.  Perhaps change this
% to the double lined characters upon figuring out how to do so.
%
\newcommand{\C}[1]{$\mathbb{C}^{#1}$}
\newcommand{\R}[1]{$\mathbb{R}^{#1}$}

%
% various generally useful helpers
%

% derivative of #1 wrt. #2:
\newcommand{\D}[2] {\frac {d#2} {d#1}}

\newcommand{\inv}[1]{\frac{1}{#1}}
\newcommand{\cross}[0]{\times}

\newcommand{\abs}[1]{\lvert{#1}\rvert}
\newcommand{\norm}[1]{\lVert{#1}\rVert}
\newcommand{\innerprod}[2]{\langle{#1}, {#2}\rangle}
\newcommand{\dotprod}[2]{{#1} \cdot {#2}}
\newcommand{\bdotprod}[2]{\left({#1} \cdot {#2}\right)}
\newcommand{\crossprod}[2]{{#1} \cross {#2}}
\newcommand{\tripleprod}[3]{\dotprod{\left(\crossprod{#1}{#2}\right)}{#3}}

\DeclareMathOperator{\Proj}{Proj}
\DeclareMathOperator{\Span}{span}
\DeclareMathOperator{\Sgn}{sgn}
\DeclareMathOperator{\Area}{Area}
\DeclareMathOperator{\Volume}{Volume}

%
% A few miscellaneous things specific to this document
%
\newcommand{\crossop}[1]{\crossprod{#1}{}}

% R2 vector.
\newcommand{\VectorTwo}[2]{
\begin{bmatrix}
 {#1} \\
 {#2}
\end{bmatrix}
}

\newcommand{\VectorN}[1]{
\begin{bmatrix}
{#1}_1 \\
{#1}_2 \\
\vdots \\
{#1}_N \\
\end{bmatrix}
}

\newcommand{\DETuvij}[4]{
\begin{vmatrix}
 {#1}_{#3} & {#1}_{#4} \\
 {#2}_{#3} & {#2}_{#4}
\end{vmatrix}
}

\newcommand{\DETuvwijk}[6]{
\begin{vmatrix}
 {#1}_{#4} & {#1}_{#5} & {#1}_{#6} \\
 {#2}_{#4} & {#2}_{#5} & {#2}_{#6} \\
 {#3}_{#4} & {#3}_{#5} & {#3}_{#6}
\end{vmatrix}
}

\newcommand{\DETuvwxijkl}[8]{
\begin{vmatrix}
 {#1}_{#5} & {#1}_{#6} & {#1}_{#7} & {#1}_{#8} \\
 {#2}_{#5} & {#2}_{#6} & {#2}_{#7} & {#2}_{#8} \\
 {#3}_{#5} & {#3}_{#6} & {#3}_{#7} & {#3}_{#8} \\
 {#4}_{#5} & {#4}_{#6} & {#4}_{#7} & {#4}_{#8} \\
\end{vmatrix}
}

%\newcommand{\DETuvwxyijklm}[10]{
%\begin{vmatrix}
% {#1}_{#6} & {#1}_{#7} & {#1}_{#8} & {#1}_{#9} & {#1}_{#10} \\
% {#2}_{#6} & {#2}_{#7} & {#2}_{#8} & {#2}_{#9} & {#2}_{#10} \\
% {#3}_{#6} & {#3}_{#7} & {#3}_{#8} & {#3}_{#9} & {#3}_{#10} \\
% {#4}_{#6} & {#4}_{#7} & {#4}_{#8} & {#4}_{#9} & {#4}_{#10} \\
% {#5}_{#6} & {#5}_{#7} & {#5}_{#8} & {#5}_{#9} & {#5}_{#10}
%\end{vmatrix}
%}

% R3 vector.
\newcommand{\VectorThree}[3]{
\begin{bmatrix}
 {#1} \\
 {#2} \\
 {#3}
\end{bmatrix}
}



\author{Peeter Joot}
\email{peeter.joot@gmail.com}

%\documentclass[]{eliblogwidescreen}

\usepackage{amsmath}
\usepackage{mathpazo}

%
% shorthand for bold symbols, convenient for vectors and matrices
%
\newcommand{\Ba}[0]{\mathbf{a}}
\newcommand{\Bb}[0]{\mathbf{b}}
\newcommand{\Bc}[0]{\mathbf{c}}
\newcommand{\Bd}[0]{\mathbf{d}}
\newcommand{\Be}[0]{\mathbf{e}}
\newcommand{\Bf}[0]{\mathbf{f}}
\newcommand{\Bg}[0]{\mathbf{g}}
\newcommand{\Bh}[0]{\mathbf{h}}
\newcommand{\Bi}[0]{\mathbf{i}}
\newcommand{\Bj}[0]{\mathbf{j}}
\newcommand{\Bk}[0]{\mathbf{k}}
\newcommand{\Bl}[0]{\mathbf{l}}
\newcommand{\Bm}[0]{\mathbf{m}}
\newcommand{\Bn}[0]{\mathbf{n}}
\newcommand{\Bo}[0]{\mathbf{o}}
\newcommand{\Bp}[0]{\mathbf{p}}
\newcommand{\Bq}[0]{\mathbf{q}}
\newcommand{\Br}[0]{\mathbf{r}}
\newcommand{\Bs}[0]{\mathbf{s}}
\newcommand{\Bt}[0]{\mathbf{t}}
\newcommand{\Bu}[0]{\mathbf{u}}
\newcommand{\Bv}[0]{\mathbf{v}}
\newcommand{\Bw}[0]{\mathbf{w}}
\newcommand{\Bx}[0]{\mathbf{x}}
\newcommand{\By}[0]{\mathbf{y}}
\newcommand{\Bz}[0]{\mathbf{z}}
\newcommand{\BA}[0]{\mathbf{A}}
\newcommand{\BB}[0]{\mathbf{B}}
\newcommand{\BC}[0]{\mathbf{C}}
\newcommand{\BD}[0]{\mathbf{D}}
\newcommand{\BE}[0]{\mathbf{E}}
\newcommand{\BF}[0]{\mathbf{F}}
\newcommand{\BG}[0]{\mathbf{G}}
\newcommand{\BH}[0]{\mathbf{H}}
\newcommand{\BI}[0]{\mathbf{I}}
\newcommand{\BJ}[0]{\mathbf{J}}
\newcommand{\BK}[0]{\mathbf{K}}
\newcommand{\BL}[0]{\mathbf{L}}
\newcommand{\BM}[0]{\mathbf{M}}
\newcommand{\BN}[0]{\mathbf{N}}
\newcommand{\BO}[0]{\mathbf{O}}
\newcommand{\BP}[0]{\mathbf{P}}
\newcommand{\BQ}[0]{\mathbf{Q}}
\newcommand{\BR}[0]{\mathbf{R}}
\newcommand{\BS}[0]{\mathbf{S}}
\newcommand{\BT}[0]{\mathbf{T}}
\newcommand{\BU}[0]{\mathbf{U}}
\newcommand{\BV}[0]{\mathbf{V}}
\newcommand{\BW}[0]{\mathbf{W}}
\newcommand{\BX}[0]{\mathbf{X}}
\newcommand{\BY}[0]{\mathbf{Y}}
\newcommand{\BZ}[0]{\mathbf{Z}}

\newcommand{\Bzero}[0]{\mathbf{0}}
\newcommand{\Btheta}[0]{\boldsymbol{\theta}}
\newcommand{\Btau}[0]{\boldsymbol{\tau}}
\newcommand{\Bomega}[0]{\boldsymbol{\omega}}

%
% shorthand for unit vectors
%
\newcommand{\acap}[0]{\hat{\Ba}}
\newcommand{\bcap}[0]{\hat{\Bb}}
\newcommand{\ccap}[0]{\hat{\Bc}}
\newcommand{\dcap}[0]{\hat{\Bd}}
\newcommand{\ecap}[0]{\hat{\Be}}
\newcommand{\fcap}[0]{\hat{\Bf}}
\newcommand{\gcap}[0]{\hat{\Bg}}
\newcommand{\hcap}[0]{\hat{\Bh}}
\newcommand{\icap}[0]{\hat{\Bi}}
\newcommand{\jcap}[0]{\hat{\Bj}}
\newcommand{\kcap}[0]{\hat{\Bk}}
\newcommand{\lcap}[0]{\hat{\Bl}}
\newcommand{\mcap}[0]{\hat{\Bm}}
\newcommand{\ncap}[0]{\hat{\Bn}}
\newcommand{\ocap}[0]{\hat{\Bo}}
\newcommand{\pcap}[0]{\hat{\Bp}}
\newcommand{\qcap}[0]{\hat{\Bq}}
\newcommand{\rcap}[0]{\hat{\Br}}
\newcommand{\scap}[0]{\hat{\Bs}}
\newcommand{\tcap}[0]{\hat{\Bt}}
\newcommand{\ucap}[0]{\hat{\Bu}}
\newcommand{\vcap}[0]{\hat{\Bv}}
\newcommand{\wcap}[0]{\hat{\Bw}}
\newcommand{\xcap}[0]{\hat{\Bx}}
\newcommand{\ycap}[0]{\hat{\By}}
\newcommand{\zcap}[0]{\hat{\Bz}}
\newcommand{\thetacap}[0]{\hat{\Btheta}}

%
% to write R^n and C^n in a distinguishable fashion.  Perhaps change this
% to the double lined characters upon figuring out how to do so.
%
\newcommand{\C}[1]{$\mathbb{C}^{#1}$}
\newcommand{\R}[1]{$\mathbb{R}^{#1}$}

%
% various generally useful helpers
%

% derivative of #1 wrt. #2:
\newcommand{\D}[2] {\frac {d#2} {d#1}}

\newcommand{\inv}[1]{\frac{1}{#1}}
\newcommand{\cross}[0]{\times}

\newcommand{\abs}[1]{\lvert{#1}\rvert}
\newcommand{\norm}[1]{\lVert{#1}\rVert}
\newcommand{\innerprod}[2]{\langle{#1}, {#2}\rangle}
\newcommand{\dotprod}[2]{{#1} \cdot {#2}}
\newcommand{\bdotprod}[2]{\left({#1} \cdot {#2}\right)}
\newcommand{\crossprod}[2]{{#1} \cross {#2}}
\newcommand{\tripleprod}[3]{\dotprod{\left(\crossprod{#1}{#2}\right)}{#3}}

\DeclareMathOperator{\Proj}{Proj}
\DeclareMathOperator{\Span}{span}
\DeclareMathOperator{\Sgn}{sgn}
\DeclareMathOperator{\Area}{Area}
\DeclareMathOperator{\Volume}{Volume}

%
% A few miscellaneous things specific to this document
%
\newcommand{\crossop}[1]{\crossprod{#1}{}}

% R2 vector.
\newcommand{\VectorTwo}[2]{
\begin{bmatrix}
 {#1} \\
 {#2}
\end{bmatrix}
}

\newcommand{\VectorN}[1]{
\begin{bmatrix}
{#1}_1 \\
{#1}_2 \\
\vdots \\
{#1}_N \\
\end{bmatrix}
}

\newcommand{\DETuvij}[4]{
\begin{vmatrix}
 {#1}_{#3} & {#1}_{#4} \\
 {#2}_{#3} & {#2}_{#4}
\end{vmatrix}
}

\newcommand{\DETuvwijk}[6]{
\begin{vmatrix}
 {#1}_{#4} & {#1}_{#5} & {#1}_{#6} \\
 {#2}_{#4} & {#2}_{#5} & {#2}_{#6} \\
 {#3}_{#4} & {#3}_{#5} & {#3}_{#6}
\end{vmatrix}
}

\newcommand{\DETuvwxijkl}[8]{
\begin{vmatrix}
 {#1}_{#5} & {#1}_{#6} & {#1}_{#7} & {#1}_{#8} \\
 {#2}_{#5} & {#2}_{#6} & {#2}_{#7} & {#2}_{#8} \\
 {#3}_{#5} & {#3}_{#6} & {#3}_{#7} & {#3}_{#8} \\
 {#4}_{#5} & {#4}_{#6} & {#4}_{#7} & {#4}_{#8} \\
\end{vmatrix}
}

%\newcommand{\DETuvwxyijklm}[10]{
%\begin{vmatrix}
% {#1}_{#6} & {#1}_{#7} & {#1}_{#8} & {#1}_{#9} & {#1}_{#10} \\
% {#2}_{#6} & {#2}_{#7} & {#2}_{#8} & {#2}_{#9} & {#2}_{#10} \\
% {#3}_{#6} & {#3}_{#7} & {#3}_{#8} & {#3}_{#9} & {#3}_{#10} \\
% {#4}_{#6} & {#4}_{#7} & {#4}_{#8} & {#4}_{#9} & {#4}_{#10} \\
% {#5}_{#6} & {#5}_{#7} & {#5}_{#8} & {#5}_{#9} & {#5}_{#10}
%\end{vmatrix}
%}

% R3 vector.
\newcommand{\VectorThree}[3]{
\begin{bmatrix}
 {#1} \\
 {#2} \\
 {#3}
\end{bmatrix}
}



\author{Peeter Joot}
\email{peeter.joot@gmail.com}


\chapter{Helmholtz Green's function.}
\label{chap:helmholtzGreens}
%\useCCL
\blogpage{http://sites.google.com/site/peeterjoot2/math2011/helmholtzGreens.pdf}
\date{Dec 12, 2011}
\revisionInfo{helmholtzGreens.tex}

\beginArtWithToc
%\beginArtNoToc

\section{Motivation.}

In class this week, looking at an instance of the Helmholtz equation

\begin{equation}\label{eqn:helmholtzGreens:10}
\left( \spacegrad^2 + \Bk^2\right) \psi_\Bk(\Br) = s(\Br).
\end{equation}

We were told that the Green's function

\begin{equation}\label{eqn:helmholtzGreens:30}
\left( \spacegrad^2 + \Bk^2\right) G^0(\Br, \Br') = \delta(\Br- \Br')
\end{equation}

that can be used to solve for a particular solution this differential equation vvia convolution

\begin{equation}\label{eqn:helmholtzGreens:50}
\psi_\Bk(\Br) = \int G^0(\Br, \Br') s(\Br') d^3 \Br',
\end{equation}

had the value

\begin{equation}\label{eqn:helmholtzGreens:70}
G^0(\Br, \Br') = - \inv{4 \pi} \frac{e^{i k \Abs{\Br - \Br'}} }{\Abs{\Br - \Br'}}.
\end{equation}

Let's try to verify this.

\section{Guts}

Application of the Helmoltz differential operator $\spacegrad^2 + \Bk^2$ on the presumed solution gives

\begin{equation}\label{eqn:helmholtzGreens:90}
(\spacegrad^2 + \Bk^2) \psi_\Bk(\Br) = 
- \inv{4 \pi} 
\int (\spacegrad^2 + \Bk^2) 
\frac{e^{i k \Abs{\Br - \Br'}} }{\Abs{\Br - \Br'}}
s(\Br') d^3 \Br'.
\end{equation}

To proceed we'll need to evaluate 

\begin{equation}\label{eqn:helmholtzGreens:110}
\spacegrad^2 \frac{e^{i k \Abs{\Br - \Br'}} }{\Abs{\Br - \Br'}}.
\end{equation}

Writing $\mu = \Abs{\Br - \Br'}$ we start with the computation of

\begin{align*}
\PD{x}{} \frac{e^{i k \mu} }{\mu}
&=
\PD{x}{\mu} \left( \frac{i k}{\mu} - \inv{\mu^2} \right) e^{i k \mu} \\
&=
\PD{x}{\mu} \left( i k - \inv{\mu} \right) \frac{e^{i k \mu}}{\mu}
\end{align*}

We see that we'll have

\begin{equation}\label{eqn:helmholtzGreens:130}
\spacegrad \frac{e^{i k \mu} }{\mu} = \left( i k - \inv{\mu} \right) \frac{e^{i k \mu}}{\mu} \spacegrad \mu.
\end{equation}

Taking second derivatives with respect to $x$ we find

\begin{align*}
\PDSq{x}{} \frac{e^{i k \mu} }{\mu}
&=
\PDSq{x}{\mu} \left( i k - \inv{\mu} \right) \frac{e^{i k \mu}}{\mu}
+\PD{x}{\mu} \PD{x}{\mu} \inv{\mu^2} \frac{e^{i k \mu}}{\mu}
+\left( \PD{x}{\mu} \right)^2 \left( i k - \inv{\mu} \right)^2 \frac{e^{i k \mu}}{\mu} \\
&=
\PDSq{x}{\mu} \left( i k - \inv{\mu} \right) \frac{e^{i k \mu}}{\mu}
+\left( \PD{x}{\mu} \right)^2 
\left( -k^2 - \frac{ 2 i k }{\mu} + \frac{2}{\mu^2} \right)
\frac{e^{i k \mu}}{\mu}.
\end{align*}

Our Laplacian is then

\begin{equation}\label{eqn:helmholtzGreens:150}
\spacegrad^2
\frac{e^{i k \mu} }{\mu} =
\left( i k - \inv{\mu} \right) \frac{e^{i k \mu}}{\mu} \spacegrad^2 \mu
+
\left( -k^2 - \frac{ 2 i k }{\mu} + \frac{2}{\mu^2} \right)
\frac{e^{i k \mu}}{\mu} (\spacegrad \mu)^2.
\end{equation}

Now lets calculate the derivatives of $\mu$.  Working on $x$ again, we have

\begin{align*}
\PD{x}{} \mu
&=
\PD{x}{} \sqrt{ 
(x - x')^2 
+(y - y')^2 
+(z - z')^2 
} \\
&=
\inv{2} 2 (x - x')
\inv{\sqrt{ 
(x - x')^2 
+(y - y')^2 
+(z - z')^2 
}} \\
&=
\frac{x - x'}{\mu}.
\end{align*}

So we have

\begin{align}\label{eqn:helmholtzGreens:170}
\spacegrad \mu &= \frac{\Br - \Br'}{\mu} \\
(\spacegrad \mu)^2 &= 1 
\end{align}

Taking second derivatives with respect to $x$ we find

\begin{align*}
\PDSq{x}{} \mu
&= \PD{x}{}
\frac{x - x'}{\mu} \\
&= 
\frac{1}{\mu} 
- (x - x') \PD{x}{\mu} \inv{\mu^2}
\\
&=
\frac{1}{\mu} 
- (x - x') \frac{x - x'}{\mu} \inv{\mu^2}
\\
&=
\frac{1}{\mu} 
- (x - x')^2 \inv{\mu^3}.
\end{align*}

So we find

\begin{equation}\label{eqn:helmholtzGreens:190}
\spacegrad^2 \mu = 
\frac{3}{\mu} 
- \inv{\mu},
\end{equation}

or

\begin{equation}\label{eqn:helmholtzGreens:210}
\spacegrad^2 \mu = \frac{2}{\mu}.
\end{equation}

Inserting this and $(\spacegrad \mu)^2$ into \ref{eqn:helmholtzGreens:150} we find

\begin{equation}\label{eqn:helmholtzGreens:n}
\begin{aligned}
\spacegrad^2
\frac{e^{i k \mu} }{\mu} 
&=
\left( i k - \inv{\mu} \right) \frac{e^{i k \mu}}{\mu} \frac{2}{\mu}
+
\left( -k^2 - \frac{ 2 i k }{\mu} + \frac{2}{\mu^2} \right)
\frac{e^{i k \mu}}{\mu}
&=
-k^2 \frac{e^{i k \mu}}{\mu}
\end{aligned}
\end{equation}

This shows us that provided $\Br \ne \Br'$ we have

\begin{equation}\label{eqn:helmholtzGreens:n}
(\spacegrad^2 + \Bk^2) G^0(\Br, \Br') = 0.
\end{equation}

We find therefore that we have the desired delta function behaviour away from $\Br = \Br'$.  Let's consider a volume element in the small spherical space centered on $\Br$ defined by $\Abs{\Br - \Br'} < \epsilon$.  Without loss of generality we can center this volume at $\Br = 0$.  Tossing on our physics hat, we assume that $s(\Br)$ is sufficiently continuous and well behaved that we can pull it out of the integral, yielding

\begin{align*}
(\spacegrad^2 + \Bk^2)
\left(
- \inv{4 \pi} 
\right)
\int 
\frac{e^{i k \Abs{\Br - \Br'}} }{\Abs{\Br - \Br'}}
s(\Br') d^3 \Br'
&=
(\spacegrad^2 + \Bk^2)
\left(
- \inv{4 \pi} 
\right)
\int_{r' = 0}^\epsilon
\int_{\theta' = 0}^\pi
\int_{\phi' = 0}^{2 \pi}
\frac{e^{i k r'} }{r'}
s(\Br') 
(r')^2 \sin\theta' dr' d\theta' d\phi' \\
&=
(\spacegrad^2 + \Bk^2)
s(0) 
\left(
- \inv{4 \pi} 
\right)
\int_{r' = 0}^\epsilon
\int_{\theta' = 0}^\pi
\int_{\phi' = 0}^{2 \pi}
\frac{e^{i k r'} }{r'}
(r')^2 \sin\theta' dr' d\theta' d\phi' \\
&=
-k^2
s(0) 
\int_{r' = 0}^\epsilon
r' e^{i k r'} 
dr' \\
&=
s(0) 
\int_{x = 0}^{k \epsilon}
x e^{i x} 
d x \\
\end{align*}

%\int_0^\epsilon x e^{i k x} dx
%=
%\frac{1+e^{i k \epsilon } (-1+i k \epsilon )}{i^2 k^2}

\EndArticle
%\EndNoBibArticle
