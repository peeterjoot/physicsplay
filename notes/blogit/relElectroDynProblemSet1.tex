%\documentclass[]{eliblog}
%\usepackage{color}
%\usepackage{txfonts} % for xi
%\usepackage{amsmath}
\usepackage{mathpazo}

%
% shorthand for bold symbols, convenient for vectors and matrices
%
\newcommand{\Ba}[0]{\mathbf{a}}
\newcommand{\Bb}[0]{\mathbf{b}}
\newcommand{\Bc}[0]{\mathbf{c}}
\newcommand{\Bd}[0]{\mathbf{d}}
\newcommand{\Be}[0]{\mathbf{e}}
\newcommand{\Bf}[0]{\mathbf{f}}
\newcommand{\Bg}[0]{\mathbf{g}}
\newcommand{\Bh}[0]{\mathbf{h}}
\newcommand{\Bi}[0]{\mathbf{i}}
\newcommand{\Bj}[0]{\mathbf{j}}
\newcommand{\Bk}[0]{\mathbf{k}}
\newcommand{\Bl}[0]{\mathbf{l}}
\newcommand{\Bm}[0]{\mathbf{m}}
\newcommand{\Bn}[0]{\mathbf{n}}
\newcommand{\Bo}[0]{\mathbf{o}}
\newcommand{\Bp}[0]{\mathbf{p}}
\newcommand{\Bq}[0]{\mathbf{q}}
\newcommand{\Br}[0]{\mathbf{r}}
\newcommand{\Bs}[0]{\mathbf{s}}
\newcommand{\Bt}[0]{\mathbf{t}}
\newcommand{\Bu}[0]{\mathbf{u}}
\newcommand{\Bv}[0]{\mathbf{v}}
\newcommand{\Bw}[0]{\mathbf{w}}
\newcommand{\Bx}[0]{\mathbf{x}}
\newcommand{\By}[0]{\mathbf{y}}
\newcommand{\Bz}[0]{\mathbf{z}}
\newcommand{\BA}[0]{\mathbf{A}}
\newcommand{\BB}[0]{\mathbf{B}}
\newcommand{\BC}[0]{\mathbf{C}}
\newcommand{\BD}[0]{\mathbf{D}}
\newcommand{\BE}[0]{\mathbf{E}}
\newcommand{\BF}[0]{\mathbf{F}}
\newcommand{\BG}[0]{\mathbf{G}}
\newcommand{\BH}[0]{\mathbf{H}}
\newcommand{\BI}[0]{\mathbf{I}}
\newcommand{\BJ}[0]{\mathbf{J}}
\newcommand{\BK}[0]{\mathbf{K}}
\newcommand{\BL}[0]{\mathbf{L}}
\newcommand{\BM}[0]{\mathbf{M}}
\newcommand{\BN}[0]{\mathbf{N}}
\newcommand{\BO}[0]{\mathbf{O}}
\newcommand{\BP}[0]{\mathbf{P}}
\newcommand{\BQ}[0]{\mathbf{Q}}
\newcommand{\BR}[0]{\mathbf{R}}
\newcommand{\BS}[0]{\mathbf{S}}
\newcommand{\BT}[0]{\mathbf{T}}
\newcommand{\BU}[0]{\mathbf{U}}
\newcommand{\BV}[0]{\mathbf{V}}
\newcommand{\BW}[0]{\mathbf{W}}
\newcommand{\BX}[0]{\mathbf{X}}
\newcommand{\BY}[0]{\mathbf{Y}}
\newcommand{\BZ}[0]{\mathbf{Z}}

\newcommand{\Bzero}[0]{\mathbf{0}}
\newcommand{\Btheta}[0]{\boldsymbol{\theta}}
\newcommand{\Btau}[0]{\boldsymbol{\tau}}
\newcommand{\Bomega}[0]{\boldsymbol{\omega}}

%
% shorthand for unit vectors
%
\newcommand{\acap}[0]{\hat{\Ba}}
\newcommand{\bcap}[0]{\hat{\Bb}}
\newcommand{\ccap}[0]{\hat{\Bc}}
\newcommand{\dcap}[0]{\hat{\Bd}}
\newcommand{\ecap}[0]{\hat{\Be}}
\newcommand{\fcap}[0]{\hat{\Bf}}
\newcommand{\gcap}[0]{\hat{\Bg}}
\newcommand{\hcap}[0]{\hat{\Bh}}
\newcommand{\icap}[0]{\hat{\Bi}}
\newcommand{\jcap}[0]{\hat{\Bj}}
\newcommand{\kcap}[0]{\hat{\Bk}}
\newcommand{\lcap}[0]{\hat{\Bl}}
\newcommand{\mcap}[0]{\hat{\Bm}}
\newcommand{\ncap}[0]{\hat{\Bn}}
\newcommand{\ocap}[0]{\hat{\Bo}}
\newcommand{\pcap}[0]{\hat{\Bp}}
\newcommand{\qcap}[0]{\hat{\Bq}}
\newcommand{\rcap}[0]{\hat{\Br}}
\newcommand{\scap}[0]{\hat{\Bs}}
\newcommand{\tcap}[0]{\hat{\Bt}}
\newcommand{\ucap}[0]{\hat{\Bu}}
\newcommand{\vcap}[0]{\hat{\Bv}}
\newcommand{\wcap}[0]{\hat{\Bw}}
\newcommand{\xcap}[0]{\hat{\Bx}}
\newcommand{\ycap}[0]{\hat{\By}}
\newcommand{\zcap}[0]{\hat{\Bz}}
\newcommand{\thetacap}[0]{\hat{\Btheta}}

%
% to write R^n and C^n in a distinguishable fashion.  Perhaps change this
% to the double lined characters upon figuring out how to do so.
%
\newcommand{\C}[1]{$\mathbb{C}^{#1}$}
\newcommand{\R}[1]{$\mathbb{R}^{#1}$}

%
% various generally useful helpers
%

% derivative of #1 wrt. #2:
\newcommand{\D}[2] {\frac {d#2} {d#1}}

\newcommand{\inv}[1]{\frac{1}{#1}}
\newcommand{\cross}[0]{\times}

\newcommand{\abs}[1]{\lvert{#1}\rvert}
\newcommand{\norm}[1]{\lVert{#1}\rVert}
\newcommand{\innerprod}[2]{\langle{#1}, {#2}\rangle}
\newcommand{\dotprod}[2]{{#1} \cdot {#2}}
\newcommand{\bdotprod}[2]{\left({#1} \cdot {#2}\right)}
\newcommand{\crossprod}[2]{{#1} \cross {#2}}
\newcommand{\tripleprod}[3]{\dotprod{\left(\crossprod{#1}{#2}\right)}{#3}}

\DeclareMathOperator{\Proj}{Proj}
\DeclareMathOperator{\Span}{span}
\DeclareMathOperator{\Sgn}{sgn}
\DeclareMathOperator{\Area}{Area}
\DeclareMathOperator{\Volume}{Volume}

%
% A few miscellaneous things specific to this document
%
\newcommand{\crossop}[1]{\crossprod{#1}{}}

% R2 vector.
\newcommand{\VectorTwo}[2]{
\begin{bmatrix}
 {#1} \\
 {#2}
\end{bmatrix}
}

\newcommand{\VectorN}[1]{
\begin{bmatrix}
{#1}_1 \\
{#1}_2 \\
\vdots \\
{#1}_N \\
\end{bmatrix}
}

\newcommand{\DETuvij}[4]{
\begin{vmatrix}
 {#1}_{#3} & {#1}_{#4} \\
 {#2}_{#3} & {#2}_{#4}
\end{vmatrix}
}

\newcommand{\DETuvwijk}[6]{
\begin{vmatrix}
 {#1}_{#4} & {#1}_{#5} & {#1}_{#6} \\
 {#2}_{#4} & {#2}_{#5} & {#2}_{#6} \\
 {#3}_{#4} & {#3}_{#5} & {#3}_{#6}
\end{vmatrix}
}

\newcommand{\DETuvwxijkl}[8]{
\begin{vmatrix}
 {#1}_{#5} & {#1}_{#6} & {#1}_{#7} & {#1}_{#8} \\
 {#2}_{#5} & {#2}_{#6} & {#2}_{#7} & {#2}_{#8} \\
 {#3}_{#5} & {#3}_{#6} & {#3}_{#7} & {#3}_{#8} \\
 {#4}_{#5} & {#4}_{#6} & {#4}_{#7} & {#4}_{#8} \\
\end{vmatrix}
}

%\newcommand{\DETuvwxyijklm}[10]{
%\begin{vmatrix}
% {#1}_{#6} & {#1}_{#7} & {#1}_{#8} & {#1}_{#9} & {#1}_{#10} \\
% {#2}_{#6} & {#2}_{#7} & {#2}_{#8} & {#2}_{#9} & {#2}_{#10} \\
% {#3}_{#6} & {#3}_{#7} & {#3}_{#8} & {#3}_{#9} & {#3}_{#10} \\
% {#4}_{#6} & {#4}_{#7} & {#4}_{#8} & {#4}_{#9} & {#4}_{#10} \\
% {#5}_{#6} & {#5}_{#7} & {#5}_{#8} & {#5}_{#9} & {#5}_{#10}
%\end{vmatrix}
%}

% R3 vector.
\newcommand{\VectorThree}[3]{
\begin{bmatrix}
 {#1} \\
 {#2} \\
 {#3}
\end{bmatrix}
}



%
% Copyright � 2015 Peeter Joot.  All Rights Reserved.
% Licenced as described in the file LICENSE under the root directory of this GIT repository.
%
\documentclass[]{eliblog}

\usepackage{amsmath}
\usepackage{mathpazo}

%
% shorthand for bold symbols, convenient for vectors and matrices
%
\newcommand{\Ba}[0]{\mathbf{a}}
\newcommand{\Bb}[0]{\mathbf{b}}
\newcommand{\Bc}[0]{\mathbf{c}}
\newcommand{\Bd}[0]{\mathbf{d}}
\newcommand{\Be}[0]{\mathbf{e}}
\newcommand{\Bf}[0]{\mathbf{f}}
\newcommand{\Bg}[0]{\mathbf{g}}
\newcommand{\Bh}[0]{\mathbf{h}}
\newcommand{\Bi}[0]{\mathbf{i}}
\newcommand{\Bj}[0]{\mathbf{j}}
\newcommand{\Bk}[0]{\mathbf{k}}
\newcommand{\Bl}[0]{\mathbf{l}}
\newcommand{\Bm}[0]{\mathbf{m}}
\newcommand{\Bn}[0]{\mathbf{n}}
\newcommand{\Bo}[0]{\mathbf{o}}
\newcommand{\Bp}[0]{\mathbf{p}}
\newcommand{\Bq}[0]{\mathbf{q}}
\newcommand{\Br}[0]{\mathbf{r}}
\newcommand{\Bs}[0]{\mathbf{s}}
\newcommand{\Bt}[0]{\mathbf{t}}
\newcommand{\Bu}[0]{\mathbf{u}}
\newcommand{\Bv}[0]{\mathbf{v}}
\newcommand{\Bw}[0]{\mathbf{w}}
\newcommand{\Bx}[0]{\mathbf{x}}
\newcommand{\By}[0]{\mathbf{y}}
\newcommand{\Bz}[0]{\mathbf{z}}
\newcommand{\BA}[0]{\mathbf{A}}
\newcommand{\BB}[0]{\mathbf{B}}
\newcommand{\BC}[0]{\mathbf{C}}
\newcommand{\BD}[0]{\mathbf{D}}
\newcommand{\BE}[0]{\mathbf{E}}
\newcommand{\BF}[0]{\mathbf{F}}
\newcommand{\BG}[0]{\mathbf{G}}
\newcommand{\BH}[0]{\mathbf{H}}
\newcommand{\BI}[0]{\mathbf{I}}
\newcommand{\BJ}[0]{\mathbf{J}}
\newcommand{\BK}[0]{\mathbf{K}}
\newcommand{\BL}[0]{\mathbf{L}}
\newcommand{\BM}[0]{\mathbf{M}}
\newcommand{\BN}[0]{\mathbf{N}}
\newcommand{\BO}[0]{\mathbf{O}}
\newcommand{\BP}[0]{\mathbf{P}}
\newcommand{\BQ}[0]{\mathbf{Q}}
\newcommand{\BR}[0]{\mathbf{R}}
\newcommand{\BS}[0]{\mathbf{S}}
\newcommand{\BT}[0]{\mathbf{T}}
\newcommand{\BU}[0]{\mathbf{U}}
\newcommand{\BV}[0]{\mathbf{V}}
\newcommand{\BW}[0]{\mathbf{W}}
\newcommand{\BX}[0]{\mathbf{X}}
\newcommand{\BY}[0]{\mathbf{Y}}
\newcommand{\BZ}[0]{\mathbf{Z}}

\newcommand{\Bzero}[0]{\mathbf{0}}
\newcommand{\Btheta}[0]{\boldsymbol{\theta}}
\newcommand{\Btau}[0]{\boldsymbol{\tau}}
\newcommand{\Bomega}[0]{\boldsymbol{\omega}}

%
% shorthand for unit vectors
%
\newcommand{\acap}[0]{\hat{\Ba}}
\newcommand{\bcap}[0]{\hat{\Bb}}
\newcommand{\ccap}[0]{\hat{\Bc}}
\newcommand{\dcap}[0]{\hat{\Bd}}
\newcommand{\ecap}[0]{\hat{\Be}}
\newcommand{\fcap}[0]{\hat{\Bf}}
\newcommand{\gcap}[0]{\hat{\Bg}}
\newcommand{\hcap}[0]{\hat{\Bh}}
\newcommand{\icap}[0]{\hat{\Bi}}
\newcommand{\jcap}[0]{\hat{\Bj}}
\newcommand{\kcap}[0]{\hat{\Bk}}
\newcommand{\lcap}[0]{\hat{\Bl}}
\newcommand{\mcap}[0]{\hat{\Bm}}
\newcommand{\ncap}[0]{\hat{\Bn}}
\newcommand{\ocap}[0]{\hat{\Bo}}
\newcommand{\pcap}[0]{\hat{\Bp}}
\newcommand{\qcap}[0]{\hat{\Bq}}
\newcommand{\rcap}[0]{\hat{\Br}}
\newcommand{\scap}[0]{\hat{\Bs}}
\newcommand{\tcap}[0]{\hat{\Bt}}
\newcommand{\ucap}[0]{\hat{\Bu}}
\newcommand{\vcap}[0]{\hat{\Bv}}
\newcommand{\wcap}[0]{\hat{\Bw}}
\newcommand{\xcap}[0]{\hat{\Bx}}
\newcommand{\ycap}[0]{\hat{\By}}
\newcommand{\zcap}[0]{\hat{\Bz}}
\newcommand{\thetacap}[0]{\hat{\Btheta}}

%
% to write R^n and C^n in a distinguishable fashion.  Perhaps change this
% to the double lined characters upon figuring out how to do so.
%
\newcommand{\C}[1]{$\mathbb{C}^{#1}$}
\newcommand{\R}[1]{$\mathbb{R}^{#1}$}

%
% various generally useful helpers
%

% derivative of #1 wrt. #2:
\newcommand{\D}[2] {\frac {d#2} {d#1}}

\newcommand{\inv}[1]{\frac{1}{#1}}
\newcommand{\cross}[0]{\times}

\newcommand{\abs}[1]{\lvert{#1}\rvert}
\newcommand{\norm}[1]{\lVert{#1}\rVert}
\newcommand{\innerprod}[2]{\langle{#1}, {#2}\rangle}
\newcommand{\dotprod}[2]{{#1} \cdot {#2}}
\newcommand{\bdotprod}[2]{\left({#1} \cdot {#2}\right)}
\newcommand{\crossprod}[2]{{#1} \cross {#2}}
\newcommand{\tripleprod}[3]{\dotprod{\left(\crossprod{#1}{#2}\right)}{#3}}

\DeclareMathOperator{\Proj}{Proj}
\DeclareMathOperator{\Span}{span}
\DeclareMathOperator{\Sgn}{sgn}
\DeclareMathOperator{\Area}{Area}
\DeclareMathOperator{\Volume}{Volume}

%
% A few miscellaneous things specific to this document
%
\newcommand{\crossop}[1]{\crossprod{#1}{}}

% R2 vector.
\newcommand{\VectorTwo}[2]{
\begin{bmatrix}
 {#1} \\
 {#2}
\end{bmatrix}
}

\newcommand{\VectorN}[1]{
\begin{bmatrix}
{#1}_1 \\
{#1}_2 \\
\vdots \\
{#1}_N \\
\end{bmatrix}
}

\newcommand{\DETuvij}[4]{
\begin{vmatrix}
 {#1}_{#3} & {#1}_{#4} \\
 {#2}_{#3} & {#2}_{#4}
\end{vmatrix}
}

\newcommand{\DETuvwijk}[6]{
\begin{vmatrix}
 {#1}_{#4} & {#1}_{#5} & {#1}_{#6} \\
 {#2}_{#4} & {#2}_{#5} & {#2}_{#6} \\
 {#3}_{#4} & {#3}_{#5} & {#3}_{#6}
\end{vmatrix}
}

\newcommand{\DETuvwxijkl}[8]{
\begin{vmatrix}
 {#1}_{#5} & {#1}_{#6} & {#1}_{#7} & {#1}_{#8} \\
 {#2}_{#5} & {#2}_{#6} & {#2}_{#7} & {#2}_{#8} \\
 {#3}_{#5} & {#3}_{#6} & {#3}_{#7} & {#3}_{#8} \\
 {#4}_{#5} & {#4}_{#6} & {#4}_{#7} & {#4}_{#8} \\
\end{vmatrix}
}

%\newcommand{\DETuvwxyijklm}[10]{
%\begin{vmatrix}
% {#1}_{#6} & {#1}_{#7} & {#1}_{#8} & {#1}_{#9} & {#1}_{#10} \\
% {#2}_{#6} & {#2}_{#7} & {#2}_{#8} & {#2}_{#9} & {#2}_{#10} \\
% {#3}_{#6} & {#3}_{#7} & {#3}_{#8} & {#3}_{#9} & {#3}_{#10} \\
% {#4}_{#6} & {#4}_{#7} & {#4}_{#8} & {#4}_{#9} & {#4}_{#10} \\
% {#5}_{#6} & {#5}_{#7} & {#5}_{#8} & {#5}_{#9} & {#5}_{#10}
%\end{vmatrix}
%}

% R3 vector.
\newcommand{\VectorThree}[3]{
\begin{bmatrix}
 {#1} \\
 {#2} \\
 {#3}
\end{bmatrix}
}



\author{Peeter Joot}
\email{peeter.joot@gmail.com}

\author{Peeter Joot}
\email{peeter.joot@utoronto.ca, 920798560}

\chapter{PHY450H1S Problem Set 1.}
\label{chap:relElectroDynProblemSet1}
%\blogpage{http://sites.google.com/site/peeterjoot/math2011/relElectroDynProblemSet1.pdf}
\date{Jan 22, 2011}
\revisionInfo{relElectroDynProblemSet1.tex}

\beginArtNoToc
%\section{Disclaimer.}
%
%This problem set is as yet ungraded.

\section{Problem 1.}
\subsection{Statement}
\subsection{Solution}

\section{Problem 2.}
\subsection{Statement}

From the Lorentz transformations of space and time coordinates.

\begin{enumerate}
\item Derive the transformation of velocities.  With a particle moving with $\Bv$ in the unprimed (stationary) frame, find its velocity $\Bv'$ in the primed frame.  The primed frame is moving with some $\BV$ with respect to the unprimed one.  Make sure to finally derive the general ``addition of velocities'' equation in terms of vectors and dot products, as given in \cite{landau1971classical}.
\item Then, use the addition of velocities rule to shoiw that: a) if $v < c$ in one frame, then $v' < c$ in any other frame.  b.) If $v = c$ in one frame, then $v' = c$ in any other frame, and c.) if $v> c$ in one frame, than $v' > c$ in any other frame.
\end{enumerate}

\subsection{Solution}
\subsubsection{Part 1.}

We need a vector form of the Lorentz transform to start with.  Let's write $\Bsigma$ for a unit vector colinear with the primed frame velocity $\BV$, so that $\BV = (\BV \cdot \Bsigma) \Bsigma$.  When our boost was in the $x$ direction, our Lorentz transformation was in terms of $x = \Bx \cdot \xcap$.  The component in the direction of the boost is now $\Bx \cdot \Bsigma$, and we have

\begin{subequations}
\begin{align}\label{eqn:relativisticElectrodynamicsA1:200}
c t' &= \gamma \left( ct - (\Bx \cdot \Bsigma) \frac{\BV \cdot \Bsigma}{c} \right) \\
\Bx' \cdot \Bsigma &= \gamma \left( \Bx \cdot \Bsigma - \frac{\BV \cdot \Bsigma}{c} c t \right) \\
\Bx' \wedge \Bsigma &= \Bx \wedge \Bsigma .
\end{align}
\end{subequations}

We can add the vector components using $\Bx = (\Bx \cdot \Bsigma) \Bsigma + (\Bx \wedge \Bsigma) \Bsigma$, leaving

\begin{subequations}
\begin{align}\label{eqn:relativisticElectrodynamicsA1:210}
c t' &= \gamma \left( ct - (\Bx \cdot \Bsigma) \frac{\BV \cdot \Bsigma}{c} \right) \\
\Bx' &= (\Bx \wedge \Bsigma) \Bsigma + \gamma \left( (\Bx \cdot \Bsigma) \Bsigma - \frac{\BV}{c} c t \right) .
\end{align}
\end{subequations}

Writing $(\Bx \wedge \Bsigma) \Bsigma = \Bx - (\Bx \cdot \Bsigma)\Bsigma$ we have for the spatial component transformation

\begin{equation}\label{eqn:relativisticElectrodynamicsA1:220}
\Bx' = \Bx + (\Bx \cdot \Bsigma) \Bsigma (\gamma - 1) - \gamma \frac{\BV}{c} c t.
\end{equation}

Now we are set to take derivatives to calculate the velocities.  This gives us
\begin{subequations}
\begin{align}\label{eqn:relativisticElectrodynamicsA1:230}
\frac{dt'}{dt} &= \gamma \left( 1 - \left( \frac{d\Bx}{dt} \cdot \Bsigma \right) \frac{\BV \cdot \Bsigma}{c^2} \right) \\
\frac{d\Bx'}{dt'} \frac{d t'}{dt} &= \frac{d\Bx}{dt} + \left(\frac{d\Bx}{dt} \cdot \Bsigma\right) \Bsigma (\gamma - 1) - \gamma \frac{\BV}{c} c .
\end{align}
\end{subequations}

Dividing this pair of equations, and using $\Bv = d\Bx/dt$, and $\Bv' = d\Bx'/dt'$, this is

\begin{equation}\label{eqn:relativisticElectrodynamicsA1:240}
\Bv' = \frac{\gamma^{-1} \Bv + (\Bv \cdot \Bsigma) \Bsigma (1 - \gamma^{-1}) - \BV}{ 1 - \left( \Bv \cdot \Bsigma \right) (\BV \cdot \Bsigma)/c^2 }.
\end{equation}

Since $\BV$ and our direction vector $\Bsigma$ are colinear, we have $(\Bv \cdot \Bsigma) (\BV \cdot \Bsigma) = \Bv \cdot \Bsigma$, and can simpify this last expression slightly

\begin{equation}\label{eqn:relativisticElectrodynamicsA1:250}
\Bv' = \frac{\gamma^{-1} \Bv + (\Bv \cdot \Bsigma) \Bsigma (1 - \gamma^{-1}) - \BV}{ 1 - \Bv \cdot \BV/c^2 }.
\end{equation}

Finally, if we are to compare to the text, we note that the inverse expression requires replacement of $\BV$ with $-\BV$ and switching $\Bv$ with $\Bv'$.  That gives us

\begin{equation}\label{eqn:relativisticElectrodynamicsA1:250i}
\Bv = \frac{\gamma^{-1} \Bv' + (\Bv' \cdot \Bsigma) \Bsigma (1 - \gamma^{-1}) + \BV}{ 1 + \Bv' \cdot \BV/c^2 }.
\end{equation}

The expression in the text is also a small velocity approximation.  For $\Abs{\BV} \ll c$, we have $\gamma^{-1} \approx 1$, and $(1 + \Bv' \cdot \BV/c^2)^{-1} \approx 1 - \Bv' \cdot \BV/c^2$.  This gives us

\begin{equation}\label{eqn:relativisticElectrodynamicsA1:250a}
\Bv \approx (\Bv' + \BV)( 1 - \Bv' \cdot \BV/c^2 ) \approx \BV + \Bv' - \Bv' (\Bv' \cdot \BV)/c^2
\end{equation}

One additional approximation was made dropping the $\BV (\Bv' \cdot \BV)/c^2$ term which is quadratic in $\BV/c$, which leave us with equation $5.3$ in the text as desired.

\subsubsection{Part 2.}

In \ref{eqn:relativisticElectrodynamicsA1:250i}, let's write $\Bv' = u \Bu$, where $\Bu$ is a unit vector, $V = \BV \cdot \Bsigma$, and $\alpha = \Bu \cdot \Bsigma$ for the direction cosine between the primed frame's direction of motion and the particle's velocity direction (also in the unprimed frame).  The stationary frame's particle velocity is then

\begin{equation}\label{eqn:relativisticElectrodynamicsA1:260}
\Bv = \frac{\gamma^{-1} u \Bu + u \alpha \Bsigma (1 - \gamma^{-1}) + V \Bsigma}{ 1 + \alpha u V/c^2 }.
\end{equation}

As a check, note that for $1 = \alpha = \Bu \cdot \Bsigma = \cos(0)$, we recover the familiar addition of velocities formula 

\begin{equation}\label{eqn:relativisticElectrodynamicsA1:260b}
\Bv = \Bu \frac{u + V}{ 1 + u V/c^2 }.
\end{equation}

We want to put \ref{eqn:relativisticElectrodynamicsA1:260} into a form that renders it more tractable for general angles too.  Factoring out the $\gamma^{-1}$ term appears to do the job, yielding

\begin{equation}\label{eqn:relativisticElectrodynamicsA1:260c}
\Bv = \frac{u \gamma^{-1} (\Bu -\alpha \Bsigma) + (u \alpha + V) \Bsigma}{ 1 + \alpha u V/c^2 }.
\end{equation}

After a bit of reduction and rearranging we can dot this with itself to calculate

\begin{equation}\label{eqn:relativisticElectrodynamicsA1:270}
\Bv^2 = \frac{V^2(1 - \alpha^2)(1 - u^2/c^2) + (u + \alpha V)^2}{ (1 + \alpha u V/c^2)^2 }
\end{equation}

Note that for $u = c$, we have $\Bv^2 = c^2$, regardless of the direction of $\BV$ with respect to the motion of the particle in the unprimed frame.  This shouldn't be surprising since this light like invariance is exactly what the Lorentz transformation is designed to maintain.  It is however slightly comforting to know that the algebra appears to be still be kosher after all this.  This also answers part (b) of this question, since we have tackled the $v = c$ case in the primed frame, and seen that the speed remains $v = c$ in the unprimed frame (and thus any frame moving at constant speed relative to another).

Observe that since $1 - \alpha^2 = \sin^2\theta$, and $u \le c$, this is positive definite as expected.  If one allowed $u > c$ in some frame, our speed could go imaginary!
%  Since we've been asked to consider both the physical $u < c$ and the non-physical $u > c$ cases, let's examine the behaviour of the numerator of \ref{eqn:relativisticElectrodynamicsA1:270} in the neighbourhood of $u = c$.

Let's examine \ref{eqn:relativisticElectrodynamicsA1:270} for the extreme values of the direction cosine $\alpha$.  The cases $\alpha = \pm 1$ (velocities colinear), and $\alpha = 0$ (completely perpendicular) are of particular interest.

For $\alpha = \pm 1$ we have

\begin{equation}\label{eqn:relativisticElectrodynamicsA1:270d}
\Bv^2 = c^2 \frac{(u \pm V)^2}{ (c^2 \pm u V)^2 }.
\end{equation}

While not any sort of rigourous proof, one can plot these for various values and see that this function is a surface bounded in magnitude by one.  It's also notable that the function $(u + V)/(c^2 + u V)$ is extremized at $V^2 = c^2$ for all $u$.

For $\alpha = 0$ we have

\begin{equation}\label{eqn:relativisticElectrodynamicsA1:270e}
\Bv^2 = u^2(1 - V^2/c^2) + V^2
\end{equation}

\section{Problem 3.}
\subsection{Statement}
\subsection{Solution}

\EndArticle
