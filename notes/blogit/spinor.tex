%
% Copyright � 2017 Peeter Joot.  All Rights Reserved.
% Licenced as described in the file LICENSE under the root directory of this GIT repository.
%
%{
\newcommand{\authorname}{Peeter Joot}
\newcommand{\email}{peeterjoot@protonmail.com}
\newcommand{\basename}{FIXMEbasenameUndefined}
\newcommand{\dirname}{notes/FIXMEdirnameUndefined/}

\renewcommand{\basename}{spinor}
%\renewcommand{\dirname}{notes/phy1520/}
\renewcommand{\dirname}{notes/ece1228-electromagnetic-theory/}
%\newcommand{\dateintitle}{}
%\newcommand{\keywords}{}

\newcommand{\authorname}{Peeter Joot}
\newcommand{\onlineurl}{http://sites.google.com/site/peeterjoot2/math2013/\basename.pdf}
\newcommand{\sourcepath}{\dirname\basename.tex}
\newcommand{\generatetitle}[1]{\chapter{#1}}

\newcommand{\vcsinfo}{%
\section*{}
\noindent{\color{DarkOliveGreen}{\rule{\linewidth}{0.1mm}}}
\paragraph{Document version}
%\paragraph{\color{Maroon}{Document version}}
{
\small
\begin{itemize}
\item Available online at:\\ 
\href{\onlineurl}{\onlineurl}
\item Git Repository: \input{./.revinfo/gitRepo.tex}
\item Source: \sourcepath
\item last commit: \input{./.revinfo/gitCommitString.tex}
\item commit date: \input{./.revinfo/gitCommitDate.tex}
\end{itemize}
}
}

%\PassOptionsToPackage{dvipsnames,svgnames}{xcolor}
\PassOptionsToPackage{square,numbers}{natbib}
\documentclass{scrreprt}

\usepackage[left=2cm,right=2cm]{geometry}
\usepackage[svgnames]{xcolor}
\usepackage{peeters_layout}

\usepackage{natbib}

\usepackage[
colorlinks=true,
bookmarks=false,
pdfauthor={\authorname, \email},
backref 
]{hyperref}

% http://tex.stackexchange.com/questions/75773/how-to-reference-problems-by-the-text-label-in-an-exercise-envioronment
\usepackage[english]{cleveref}
\crefname{Exercise}{exercise}{exercises}
\Crefname{Exercise}{Exercise}{Exercises}

\RequirePackage{titlesec}
\RequirePackage{ifthen}

% http://stackoverflow.com/questions/4932910/date-in-the-tabular-environment
\makeatletter
\let\insertdate\@date
\makeatother

\titleformat{\chapter}[display]
{\bfseries\Large}
{\color{DarkSlateGrey}\filleft \authorname
\ifthenelse{\isundefined{\studentnumber}}{}{\\ \studentnumber}
\ifthenelse{\isundefined{\email}}{}{\\ \email}
\ifthenelse{\isundefined{\dateintitle}}{}{\\ \insertdate}
%\ifthenelse{\isundefined{\coursename}}{}{\\ \coursename} % put in title instead.
}
{4ex}
{\color{DarkOliveGreen}{\titlerule}\color{Maroon}
\vspace{2ex}%
\filright}
[\vspace{2ex}%
\color{DarkOliveGreen}\titlerule
]

\newcommand{\beginArtWithToc}[0]{\begin{document}\tableofcontents}
\newcommand{\beginArtNoToc}[0]{\begin{document}}
\newcommand{\EndNoBibArticle}[0]{\end{document}}
\newcommand{\EndArticle}[0]{\bibliography{Bibliography}\bibliographystyle{plainnat}\end{document}}

% 
%\newcommand{\citep}[1]{\cite{#1}}

\colorSectionsForArticle



\usepackage{peeters_layout_exercise}
\usepackage{peeters_braket}
\usepackage{peeters_figures}
\usepackage{siunitx}
%\usepackage{mhchem} % \ce{}
%\usepackage{macros_bm} % \bcM
%\usepackage{macros_qed} % \qedmarker
%\usepackage{txfonts} % \ointclockwise

\beginArtNoToc

\generatetitle{XXX}
%\chapter{XXX}
%\label{chap:spinor}
% \citep{sakurai2014modern} pr X.Y
% \citep{pozar2009microwave}
% \citep{qftLectureNotes}
% \citep{doran2003gap}
% \citep{jackson1975cew}
% \citep{griffiths1999introduction}

This isn't a full answer, but perhaps enough to get you started.  Because \( \hat{\Omega} \) commutes with \( \chi \), it also commutes with \( \tilde{\chi} \), as does the pseudoscalar \( I \).  This means you can write

\begin{dmath}\label{eqn:spinor:20}
\begin{aligned}
\frac{1}{2} I \omega
\lr{ \hat{\Omega} \chi E \tilde{\chi} -
\chi E \tilde{\chi}\hat{\Omega} }
&=
\frac{1}{2} I \omega
\lr{
\chi
\hat{\Omega}
E \tilde{\chi} -
\chi E
\hat{\Omega}
\tilde{\chi}
} \\
&=
\frac{1}{2} \omega
\chi
\lr{
I \hat{\Omega}
E
-
E
I \hat{\Omega}
}
\tilde{\chi} \\
&=
\chi
\omega
\inv{2}
\lr{
I \hat{\Omega}
E
-
E
I \hat{\Omega}
}
\tilde{\chi}
\end{aligned}
\end{dmath}

Note that \( I \hat{\Omega} \) is a vector, so

\begin{dmath}\label{eqn:spinor:40}
\inv{2}
\lr{
I \hat{\Omega}
E
-
E
I \hat{\Omega}
}
=
\lr{ I \hat{\Omega} } \wedge E_1
+
\lr{ I \hat{\Omega} } \cdot E_2,
\end{dmath}

where \( E_1 = \gpgradeone{E} \), and \( E_2 = \gpgradetwo{E} \), the grade one and two components of \( E \) respectively.  This sum, like \( E \), therefore also has only vector and bivector grades, so your problem is now reduced to the form

\begin{dmath}\label{eqn:spinor:60}
\chi
A
\tilde{\chi}
=
B,
\end{dmath}

where
\begin{dmath}\label{eqn:spinor:80}
\begin{aligned}
A
&= -\omega^2 E +
\frac{\omega}{2}
\lr{
E
I \hat{\Omega}
-
I \hat{\Omega}
E
} \\
B &=
\omega_p^2 E,
\end{aligned}
\end{dmath}

are both multivectors with only vector and bivector grades.

%}
\EndArticle
%\EndNoBibArticle
