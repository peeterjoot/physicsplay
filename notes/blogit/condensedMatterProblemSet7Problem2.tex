%
% Copyright � 2013 Peeter Joot.  All Rights Reserved.
% Licenced as described in the file LICENSE under the root directory of this GIT repository.
%
\makeproblem{nearly free electron model}{condensedMatter:problemSet7:2}{ 

\Cref{fig:condensedMatterProblemSet7:condensedMatterProblemSet7Fig1} shows
the free electron dispersion relation for a one-dimensional metal, in the reduced zone scheme.

\imageFigure{condensedMatterProblemSet7Fig1}{dispersion relation for a one-dimensional metal}{fig:condensedMatterProblemSet7:condensedMatterProblemSet7Fig1}{0.3}

\makesubproblem{}{condensedMatter:problemSet7:2a}

For each of the branches (1), (2), (3) and (4),
state which of the coefficients $C_{k -G}$ are non-zero, and give the corresponding $u_k(x)$ in the electron wave-function

\begin{dmath}\label{eqn:condensedMatterProblemSet7Problem2:20}
\psi_k(x) = \sum_G C_{k-G} e^{-iG x} e^{i k x} = u_k(x) e^{i k x}.
\end{dmath}

\makesubproblem{}{condensedMatter:problemSet7:2b}
For the circled regions there are two nearly-degenerate energy solutions. If a nearly
free electron potential

\begin{dmath}\label{eqn:condensedMatterProblemSet7Problem2:40}
V(x) = V_1 cos(2\pi x/a) + V_2 cos(4 \pi x/a)
\end{dmath}
is introduced, what is the magnitude of the energy gap that opens up at the band
crossings in the circled regions `a', `b' and `c'?

\makesubproblem{}{condensedMatter:problemSet7:2c}
What are the gaps at `a', `b' and `c' if
instead the lattice potential is described by
a periodic array of delta functions:

\begin{dmath}\label{eqn:condensedMatterProblemSet7Problem2:60}
V(x) =
\sum_{-\infty}^\infty
V_\nought \delta(x - na).
\end{dmath}

\makesubproblem{}{condensedMatter:problemSet7:2d}
Write the form of the wave-function for the two solutions at the level crossing at `b'
on the diagram (i.e. where $k = 0$). Sketch in real space the charge density associated
with these solutions relative to the atom positions on the one-dimensional lattice. Also,
write these wave functions in the Bloch form, $\psi_k(x) = e^{i k x} u_k(x)$, and thus identify $u_k(x)$
for the two solutions at `b'.
\makesubproblem{}{condensedMatter:problemSet7:2e}
Near point `a', investigate how the wave-function of the lower-energy branch (branch
$G = 0$) evolves as you move away from the Brillouin zone boundary.  To do this, calculate
$\Abs{\psi_k(x)}^2$, and from this sketch the charge density in real space, explaining how it changes
as $k$ moves away from $\pi/a$. If it helps, you may start far enough away from the Brillouin
zone boundary that $E_{k - 2\pi/a}^\nought -E_{k}^\nought \gg \Abs{V_{2 \pi/a}}$.
} % makeproblem

\makeanswer{condensedMatter:problemSet7:2}{ 
\makeSubAnswer{}{condensedMatter:problemSet7:2a}

TODO.
\makeSubAnswer{}{condensedMatter:problemSet7:2b}

TODO.
\makeSubAnswer{}{condensedMatter:problemSet7:2c}

TODO.
\makeSubAnswer{}{condensedMatter:problemSet7:2d}

TODO.
\makeSubAnswer{}{condensedMatter:problemSet7:2e}

TODO.
}
