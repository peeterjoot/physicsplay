%
% Copyright � 2013 Peeter Joot.  All Rights Reserved.
% Licenced as described in the file LICENSE under the root directory of this GIT repository.
%
\makeproblem{nearly free electron model}{condensedMatter:problemSet7:2}{ 

\Cref{fig:condensedMatterProblemSet7:condensedMatterProblemSet7Fig1} shows
the free electron dispersion relation for a one-dimensional metal, in the reduced zone scheme.

\imageFigure{condensedMatterProblemSet7Fig1}{dispersion relation for a one-dimensional metal}{fig:condensedMatterProblemSet7:condensedMatterProblemSet7Fig1}{0.3}

\makesubproblem{}{condensedMatter:problemSet7:2a}

For each of the branches (1), (2), (3) and (4),
state which of the coefficients $C_{k -G}$ are non-zero, and give the corresponding $u_k(x)$ in the electron wave-function

\begin{dmath}\label{eqn:condensedMatterProblemSet7Problem2:20}
\psi_k(x) = \sum_G C_{k-G} e^{-iG x} e^{i k x} = u_k(x) e^{i k x}.
\end{dmath}

\makesubproblem{}{condensedMatter:problemSet7:2b}
For the circled regions there are two nearly-degenerate energy solutions. If a nearly
free electron potential

\begin{dmath}\label{eqn:condensedMatterProblemSet7Problem2:40}
V(x) = V_1 cos(2\pi x/a) + V_2 cos(4 \pi x/a)
\end{dmath}
is introduced, what is the magnitude of the energy gap that opens up at the band
crossings in the circled regions `a', `b' and `c'?

\makesubproblem{}{condensedMatter:problemSet7:2c}
What are the gaps at `a', `b' and `c' if
instead the lattice potential is described by
a periodic array of delta functions:

\begin{dmath}\label{eqn:condensedMatterProblemSet7Problem2:60}
V(x) =
\sum_{-\infty}^\infty
V_\nought \delta(x - na).
\end{dmath}

\makesubproblem{}{condensedMatter:problemSet7:2d}
Write the form of the wave-function for the two solutions at the level crossing at `b'
on the diagram (i.e. where $k = 0$). Sketch in real space the charge density associated
with these solutions relative to the atom positions on the one-dimensional lattice. Also,
write these wave functions in the Bloch form, $\psi_k(x) = e^{i k x} u_k(x)$, and thus identify $u_k(x)$
for the two solutions at `b'.
\makesubproblem{}{condensedMatter:problemSet7:2e}
Near point `a', investigate how the wave-function of the lower-energy branch (branch
$G = 0$) evolves as you move away from the Brillouin zone boundary.  To do this, calculate
$\Abs{\psi_k(x)}^2$, and from this sketch the charge density in real space, explaining how it changes
as $k$ moves away from $\pi/a$. If it helps, you may start far enough away from the Brillouin
zone boundary that $E_{k - 2\pi/a}^\nought -E_{k}^\nought \gg \Abs{V_{2 \pi/a}}$.
} % makeproblem

\makeanswer{condensedMatter:problemSet7:2}{ 
\makeSubAnswer{}{condensedMatter:problemSet7:2a}

Considering each branch in turn:

\begin{enumerate}
\item On this branch, the wave function is
\begin{dmath}\label{eqn:condensedMatterProblemSet7Problem2:80}
\psi_k(x) = 
\inv{\sqrt{a}} 
e^{i k x},
\end{dmath}

so

\begin{dmath}\label{eqn:condensedMatterProblemSet7Problem2:160}
\begin{aligned}
u_k(x) &= 
\inv{\sqrt{a}} \\
C_k &= 
\inv{\sqrt{a}} \\
C_{k - 2 \pi m/a} &= 0, \qquad m \in \{\pm 1, \pm 2, \cdots \}.
\end{aligned}
\end{dmath}

\item On this branch, the wave function is
\begin{dmath}\label{eqn:condensedMatterProblemSet7Problem2:100}
\psi_k(x) 
= \inv{\sqrt{a}} 
e^{i \lr{k - 2 \pi/a} x}
=
\lr{
\inv{\sqrt{a}} e^{-i \frac{2 \pi}{a} x}
}
e^{i k x},
\end{dmath}

so

\begin{dmath}\label{eqn:condensedMatterProblemSet7Problem2:180}
\begin{aligned}
u_k(x) &= 
\inv{\sqrt{a}} e^{-i \frac{2 \pi}{a} x} \\
C_{k - 2\pi/a} &= 
\inv{\sqrt{a}} \\
C_{k - 2 \pi m/a} &= 0, \qquad m \in \{0, -1, \pm 2, \pm 3, \cdots \}.
\end{aligned}
\end{dmath}

\item On this branch, the wave function is
\begin{dmath}\label{eqn:condensedMatterProblemSet7Problem2:400}
\psi_k(x) 
= \inv{\sqrt{a}} 
e^{i \lr{k + 2 \pi/a} x}
=
\lr{
\inv{\sqrt{a}} e^{-i \frac{-2 \pi}{a} x}
}
e^{i k x},
\end{dmath}

so

\begin{dmath}\label{eqn:condensedMatterProblemSet7Problem2:420}
\begin{aligned}
u_k(x) &= 
\inv{\sqrt{a}} e^{i \frac{2 \pi}{a} x} \\
C_{k + 2\pi/a} &= 
\inv{\sqrt{a}} \\
C_{k - 2 \pi m/a} &= 0, \qquad m \in \{0, 1, \pm 2, \pm 3, \cdots \}.
\end{aligned}
\end{dmath}

\item On this branch, the wave function is
\begin{dmath}\label{eqn:condensedMatterProblemSet7Problem2:200}
\psi_k(x) 
= \inv{\sqrt{a}} 
e^{i \lr{k - 4 \pi/a} x}
=
\lr{
\inv{\sqrt{a}} e^{-i \frac{4 \pi}{a} x}
}
e^{i k x},
\end{dmath}

so

\begin{dmath}\label{eqn:condensedMatterProblemSet7Problem2:220}
\begin{aligned}
u_k(x) &= 
\inv{\sqrt{a}} e^{-i \frac{4 \pi}{a} x} \\
C_{k - 4 \pi/a} &=
\inv{\sqrt{a}} \\
C_{k - 2 \pi m/a} &= 0, \qquad m \in \{0, \pm 1, -2, \pm 3, \pm 4, \cdots \}.
\end{aligned}
\end{dmath}

\end{enumerate}

\makeSubAnswer{}{condensedMatter:problemSet7:2b}

With $G = 2 \pi/a$, we can write the potential of \eqnref{eqn:condensedMatterProblemSet7Problem2:40} as

\begin{dmath}\label{eqn:condensedMatterProblemSet7Problem2:240}
V(x) = 
\frac{V_1}{2} \lr{ 
e^{ i G x } 
+ e^{ -i G x } 
}
+
\frac{V_2}{2} \lr{ 
e^{ 2 i G x } 
+ e^{ -2 i G x } 
}.
\end{dmath}

That is
\begin{dmath}\label{eqn:condensedMatterProblemSet7Problem2:260}
\begin{aligned}
V_{1 G} &= V_{-1 G} = \frac{V_1}{2} \\
V_{2 G} &= V_{-2 G} = \frac{V_2}{2}.
\end{aligned}
\end{dmath}

The coefficients $C_k$ would follow from a solution of

\begin{dmath}\label{eqn:condensedMatterProblemSet7Problem2:280}
0 =
\lr{ \frac{\Hbar^2}{2m} \lr{k - n G}^2 - E } C_{k - n G}
+ V_G \lr{ C_{ k - (1 + n) G } + C_{ k + (1 - n) G } }
+ V_{2G} \lr{ C_{ k - (2 + n) G } + C_{ k + (2 - n)G } }
\end{dmath}

Written out in full this includes 

\begin{dmath}\label{eqn:condensedMatterProblemSet7Problem2:300}
\begin{aligned}
0 &=
\lr{ \frac{\Hbar^2}{2m} (k + 2 G)^2 - E } C_{k + 2 G}
+ V_G \lr{ C_{ k + G } + C_{ k + 3 G } }
+ V_{2G} \lr{ C_{ k } + C_{ k + 4 G } } \\
0 &=
\lr{ \frac{\Hbar^2}{2m} (k + G)^2 - E } C_{k + G}
+ V_G \lr{ C_{ k } + C_{ k + 2 G } }
+ V_{2G} \lr{ C_{ k - G } + C_{ k + 3 G } } \\
0 &=
\lr{ \frac{\Hbar^2}{2m} k^2 - E } C_{k}
+ V_G \lr{ C_{ k - G } + C_{ k + G } }
+ V_{2G} \lr{ C_{ k - 2 G } + C_{ k + 2G } } \\
0 &=
\lr{ \frac{\Hbar^2}{2m} (k - G)^2 - E } C_{k - G}
+ V_G \lr{ C_{ k - 2 G } + C_{ k } }
+ V_{2G} \lr{ C_{ k - 3 G } + C_{ k + G } } \\
0 &=
\lr{ \frac{\Hbar^2}{2m} (k - 2 G)^2 - E } C_{k - 2 G}
+ V_G \lr{ C_{ k - 3 G } + C_{ k - G } }
+ V_{2G} \lr{ C_{ k - 4 G } + C_{ k } } \\
\end{aligned}
\end{dmath}

Dropping $C_{k - 4 G}, C_{k - 3 G}, C_{k + 3 G}, C_{k + 4G}$, this is

%\begin{dmath}\label{eqn:condensedMatterProblemSet7Problem2:320}
%0 =
%\begin{vmatrix}
%\vdots \\
%V_{2G} & V_{G}  & {E_{k -2 G}^\nought} - E & V_G         & V_{2G}   & 0                & 0      & 0           & 0      \\
%0      & V_{2G} & V_{G}        & {E_{k - G}^\nought} - E & V_G      & V_{2G}           & 0      & 0           & 0      \\
%0      & 0      & V_{2G}       & V_{G}       & {E_{k }^\nought} - E & V_G              & V_{2G} & 0           & 0      \\
%0      & 0      & 0            & V_{2G}      & V_{G}       & {E_{k + G }^\nought} - E & V_G    & V_{2G}      & 0      \\
%0      & 0      & 0            & 0           & V_{2G}      & V_{G}       & {E_{k + 2 G }^\nought} - E & V_G    & V_{2G} \\
%\vdots \\
%\end{vmatrix}
%\end{dmath}

\begin{dmath}\label{eqn:condensedMatterProblemSet7Problem2:320}
0 =
\begin{vmatrix}
                  {E_{k -2 G}^\nought} - E & V_G         & V_{2G}   & 0                & 0            \\
                  V_{G}        & {E_{k - G}^\nought} - E & V_G      & V_{2G}           & 0            \\
                  V_{2G}       & V_{G}       & {E_{k }^\nought} - E & V_G              & V_{2G}       \\
                  0            & V_{2G}      & V_{G}       & {E_{k + G }^\nought} - E & V_G           \\
                  0            & 0           & V_{2G}      & V_{G}       & {E_{k + 2 G }^\nought} - E \\
\end{vmatrix}
\end{dmath}

\paragraph{At point (a)} we have ${E_{k }^\nought} = {E_{k -G}^\nought}$, and can get a rough idea of the separation by further dropping $C_{k - G}, C_{k + G}, C_{k + 2G}$, so that the system to solve is

\begin{dmath}\label{eqn:condensedMatterProblemSet7Problem2:340}
0 =
\begin{vmatrix}
                                 {E_{k}^\nought} - E & V_G      \\
                                 V_{G}       & {E_{k }^\nought} - E
\end{vmatrix},
\end{dmath}

or

\begin{dmath}\label{eqn:condensedMatterProblemSet7Problem2:360}
E_{\pm} = {E_{k }^\nought} \pm V_G = E \pm \frac{V_1}{2}.
\end{dmath}

The (approximate) separation between the energy curves at that point is

\begin{dmath}\label{eqn:condensedMatterProblemSet7Problem2:380}
\Delta E = E_{+} - E_{-} = \frac{V_1}{2} - - \frac{V_1}{2} = V_1.
\end{dmath}

\makeSubAnswer{}{condensedMatter:problemSet7:2c}

TODO.
\makeSubAnswer{}{condensedMatter:problemSet7:2d}

TODO.
\makeSubAnswer{}{condensedMatter:problemSet7:2e}

TODO.
}
