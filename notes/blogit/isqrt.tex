%
% Copyright � 2016 Peeter Joot.  All Rights Reserved.
% Licenced as described in the file LICENSE under the root directory of this GIT repository.
%
%{
\newcommand{\authorname}{Peeter Joot}
\newcommand{\email}{peeterjoot@protonmail.com}
\newcommand{\basename}{FIXMEbasenameUndefined}
\newcommand{\dirname}{notes/FIXMEdirnameUndefined/}

\renewcommand{\basename}{isqrt}
\renewcommand{\dirname}{notes/misc/}
%\newcommand{\dateintitle}{}
%\newcommand{\keywords}{}

\newcommand{\authorname}{Peeter Joot}
\newcommand{\onlineurl}{http://sites.google.com/site/peeterjoot2/math2013/\basename.pdf}
\newcommand{\sourcepath}{\dirname\basename.tex}
\newcommand{\generatetitle}[1]{\chapter{#1}}

\newcommand{\vcsinfo}{%
\section*{}
\noindent{\color{DarkOliveGreen}{\rule{\linewidth}{0.1mm}}}
\paragraph{Document version}
%\paragraph{\color{Maroon}{Document version}}
{
\small
\begin{itemize}
\item Available online at:\\ 
\href{\onlineurl}{\onlineurl}
\item Git Repository: \input{./.revinfo/gitRepo.tex}
\item Source: \sourcepath
\item last commit: \input{./.revinfo/gitCommitString.tex}
\item commit date: \input{./.revinfo/gitCommitDate.tex}
\end{itemize}
}
}

%\PassOptionsToPackage{dvipsnames,svgnames}{xcolor}
\PassOptionsToPackage{square,numbers}{natbib}
\documentclass{scrreprt}

\usepackage[left=2cm,right=2cm]{geometry}
\usepackage[svgnames]{xcolor}
\usepackage{peeters_layout}

\usepackage{natbib}

\usepackage[
colorlinks=true,
bookmarks=false,
pdfauthor={\authorname, \email},
backref 
]{hyperref}

% http://tex.stackexchange.com/questions/75773/how-to-reference-problems-by-the-text-label-in-an-exercise-envioronment
\usepackage[english]{cleveref}
\crefname{Exercise}{exercise}{exercises}
\Crefname{Exercise}{Exercise}{Exercises}

\RequirePackage{titlesec}
\RequirePackage{ifthen}

% http://stackoverflow.com/questions/4932910/date-in-the-tabular-environment
\makeatletter
\let\insertdate\@date
\makeatother

\titleformat{\chapter}[display]
{\bfseries\Large}
{\color{DarkSlateGrey}\filleft \authorname
\ifthenelse{\isundefined{\studentnumber}}{}{\\ \studentnumber}
\ifthenelse{\isundefined{\email}}{}{\\ \email}
\ifthenelse{\isundefined{\dateintitle}}{}{\\ \insertdate}
%\ifthenelse{\isundefined{\coursename}}{}{\\ \coursename} % put in title instead.
}
{4ex}
{\color{DarkOliveGreen}{\titlerule}\color{Maroon}
\vspace{2ex}%
\filright}
[\vspace{2ex}%
\color{DarkOliveGreen}\titlerule
]

\newcommand{\beginArtWithToc}[0]{\begin{document}\tableofcontents}
\newcommand{\beginArtNoToc}[0]{\begin{document}}
\newcommand{\EndNoBibArticle}[0]{\end{document}}
\newcommand{\EndArticle}[0]{\bibliography{Bibliography}\bibliographystyle{plainnat}\end{document}}

% 
%\newcommand{\citep}[1]{\cite{#1}}

\colorSectionsForArticle



%\usepackage{peeters_layout_exercise}
%\usepackage{peeters_braket}
\usepackage{peeters_figures}
%\usepackage{siunitx}
\usepackage{listings}
\lstset{ %
language=C++,                % choose the language of the code
basicstyle=\footnotesize,       % the size of the fonts that are used for the code
numbers=left,                   % where to put the line-numbers
numberstyle=\footnotesize,      % the size of the fonts that are used for the line-numbers
stepnumber=1,                   % the step between two line-numbers. If it is 1 each line will be numbered
numbersep=5pt,                  % how far the line-numbers are from the code
backgroundcolor=\color{white},  % choose the background color. You must add \usepackage{color}
showspaces=false,               % show spaces adding particular underscores
showstringspaces=false,         % underline spaces within strings
showtabs=false,                 % show tabs within strings adding particular underscores
frame=single,                   % adds a frame around the code
tabsize=2,              % sets default tabsize to 2 spaces
captionpos=b,                   % sets the caption-position to bottom
breaklines=true,        % sets automatic line breaking
breakatwhitespace=false,    % sets if automatic breaks should only happen at whitespace
%escapeinside={\%}{)}          % if you want to add a comment within your code
}

\beginArtNoToc

\generatetitle{Integer square root}
%\chapter{Integer square root}
%\label{chap:isqrt}

In \citep{stroustrup2014c++} is a rather mysterious looking constant expression formula for an integer square root.  This is a function that returns the smallest integer for which the square is less than the value to take the root of.  Check out the black magic he used

%\input{../../programming/isqrt/is.h}
\begin{lstlisting}
// Stroustrup 10.4:  constexpr capable integer square root function
constexpr int isqrt_helper( int sq, int d, int n )
{
	return sq <= n ? isqrt_helper( sq + d, d + 2, n ) : d ;
}

constexpr int isqrt( int n )
{
	return isqrt_helper( 1, 3, n )/2 - 1 ;
}
\end{lstlisting}

\section{Let's take this apart a bit.}

Consider the first few values of \( n > 0 \).

\begin{itemize}
\item \( n = 0 \).  Here we have a call to \( \textrm{isqrt_helper}( 1, 3, 0 ) \) so the \( 1 \le 0 \) predicate is false, and the return value is just \( 3 \).

For that value we have (using integer arithmetic):

\begin{dmath}\label{eqn:isqrt:20}
\frac{3}{2} - 1 = 0,
\end{dmath}

as desired.
\item \( n = 1 \).  Here we have a call to \( \textrm{isqrt_helper}( 1, 3, 1 ) \) so the \( 1 \le 1 \) predicate is true, resulting in a second call \( \textrm{isqrt_helper}( 4, 5, 1 ) \).  For that call the \( 4 \le 1 \) predicate is false, resulting in a return value of \( 5 \).

This time we have a final result of

\begin{dmath}\label{eqn:isqrt:40}
\frac{5}{2} - 1 = 1,
\end{dmath}

as desired again.  The result will be the same for any value \( n \in [1,3] \).
\item \( n = 4 \).  We will end up with a call to \( \textrm{isqrt_helper}( 4, 5, 4 ) \) for which the \( 4 \le 4 \) predicate is true, resulting in a followup call of \( \textrm{isqrt_helper}( 9, 7, 4 ) \).  For that call the \( 9 \le 4 \) predicate is false, resulting in a return value of \( 7 \).

This time we have a final result of

\begin{dmath}\label{eqn:isqrt:45}
\frac{7}{2} - 1 = 2,
\end{dmath}

as expected.  We get the same result for any value \( n \in [4,8] \).
\end{itemize}

\section{Recurrence relations}

The rough pattern of the magic involved can be seen.  We have a sequence of calls

\begin{itemize}
\item \( \textrm{isqrt_helper}( 1, 3, n ) \),
\item \( \textrm{isqrt_helper}( 4, 5, n ) \),
\item \( \textrm{isqrt_helper}( 9, 7, n ) \),
\item \( \textrm{isqrt_helper}( 16, 9, n ) \),
\end{itemize}

which terminates at the point where the first (square) parameter exceeds that value that we are taking the root of.  Let the parameters of the sequence of calls be \( s_k \), and \( d_k \), so that with \( s_0 = 1, d_0 = 3 \) the \( k \in [0,...] \) call to the helper function is \( q_k = \textrm{isqrt_helper}( s_k, d_k, n ) \).  

The sequence for the second parameter, the eventual return value, can be summarized compactly as \( d_k = 3 + 2 k \).  It is not entirely obvious how we end up with a square for the values \( s_k = s_{k-1} + d_{k-1} \), but this follows by summation.  For \( k > 1 \) that is

\begin{dmath}\label{eqn:isqrt:60}
s_k 
= s_{k-1} + d_{k-1}
= s_0 + d_0 + d_1 + d_{k-1}
= s_0 + \sum_{m=0}^{k-1} d_m
= s_0 + \sum_{m=0}^{k-1} (3 + 2 m )
= s_0 + \sum_{m=1}^{k} (3 + 2 (m-1) )
= s_0 + \sum_{m=1}^{k} (1 + 2 m )
= 1 + k + 2 \sum_{m=1}^{k} m
= 1 + k + 2 \frac{k(k+1)}{2}
= k^2 + 2 k + 1
= (k+1)^2.
\end{dmath}

This clearly holds for the boundary cases \( k = 0,1 \) as well.  This allows the helper function action to be summarized more compactly

\begin{dmath}\label{eqn:isqrt:80}
\textrm{isqrt_helper}(1, 3, n) = 3 + 2 k,
\end{dmath}

where \( k \) is the smallest integer such that \( (k+1)^2 > n \).  After integer scaling the final result is

\begin{dmath}\label{eqn:isqrt:100}
(3 + 2 k)/2 -1 = k.
\end{dmath}

This little beastie makes sense after deconstruction, but it was very Jackson like to toss this into the book without comment or explanation.  

As pointed out by Pramod Gupta, there's a \href
{https://youtu.be/CPPX4kwqh80?t=123}
{spooky appearance of collaboration}
between Stroustrup and Jackson's publishers, not entirely limited to the book covers.

%https://www.youtube.com/watch?v=CPPX4kwqh80

%}
\EndArticle
%\EndNoBibArticle
