%
% Copyright � 2012 Peeter Joot.  All Rights Reserved.
% Licenced as described in the file LICENSE under the root directory of this GIT repository.
%
% pick one:
%\newcommand{\authorname}{Peeter Joot}
\newcommand{\email}{peeter.joot@utoronto.ca}
\newcommand{\studentnumber}{920798560}
\newcommand{\basename}{FIXMEbasenameUndefined}
\newcommand{\dirname}{notes/FIXMEdirnameUndefined/}

\newcommand{\authorname}{Peeter Joot}
\newcommand{\email}{peeterjoot@protonmail.com}
\newcommand{\basename}{FIXMEbasenameUndefined}
\newcommand{\dirname}{notes/FIXMEdirnameUndefined/}

\renewcommand{\basename}{optics2010ExamGaussian}
\renewcommand{\dirname}{notes/phy485/}
%\newcommand{\dateintitle}{}
%\newcommand{\keywords}{}

\newcommand{\authorname}{Peeter Joot}
\newcommand{\onlineurl}{http://sites.google.com/site/peeterjoot2/math2013/\basename.pdf}
\newcommand{\sourcepath}{\dirname\basename.tex}
\newcommand{\generatetitle}[1]{\chapter{#1}}

\newcommand{\vcsinfo}{%
\section*{}
\noindent{\color{DarkOliveGreen}{\rule{\linewidth}{0.1mm}}}
\paragraph{Document version}
%\paragraph{\color{Maroon}{Document version}}
{
\small
\begin{itemize}
\item Available online at:\\ 
\href{\onlineurl}{\onlineurl}
\item Git Repository: \input{./.revinfo/gitRepo.tex}
\item Source: \sourcepath
\item last commit: \input{./.revinfo/gitCommitString.tex}
\item commit date: \input{./.revinfo/gitCommitDate.tex}
\end{itemize}
}
}

%\PassOptionsToPackage{dvipsnames,svgnames}{xcolor}
\PassOptionsToPackage{square,numbers}{natbib}
\documentclass{scrreprt}

\usepackage[left=2cm,right=2cm]{geometry}
\usepackage[svgnames]{xcolor}
\usepackage{peeters_layout}

\usepackage{natbib}

\usepackage[
colorlinks=true,
bookmarks=false,
pdfauthor={\authorname, \email},
backref 
]{hyperref}

% http://tex.stackexchange.com/questions/75773/how-to-reference-problems-by-the-text-label-in-an-exercise-envioronment
\usepackage[english]{cleveref}
\crefname{Exercise}{exercise}{exercises}
\Crefname{Exercise}{Exercise}{Exercises}

\RequirePackage{titlesec}
\RequirePackage{ifthen}

% http://stackoverflow.com/questions/4932910/date-in-the-tabular-environment
\makeatletter
\let\insertdate\@date
\makeatother

\titleformat{\chapter}[display]
{\bfseries\Large}
{\color{DarkSlateGrey}\filleft \authorname
\ifthenelse{\isundefined{\studentnumber}}{}{\\ \studentnumber}
\ifthenelse{\isundefined{\email}}{}{\\ \email}
\ifthenelse{\isundefined{\dateintitle}}{}{\\ \insertdate}
%\ifthenelse{\isundefined{\coursename}}{}{\\ \coursename} % put in title instead.
}
{4ex}
{\color{DarkOliveGreen}{\titlerule}\color{Maroon}
\vspace{2ex}%
\filright}
[\vspace{2ex}%
\color{DarkOliveGreen}\titlerule
]

\newcommand{\beginArtWithToc}[0]{\begin{document}\tableofcontents}
\newcommand{\beginArtNoToc}[0]{\begin{document}}
\newcommand{\EndNoBibArticle}[0]{\end{document}}
\newcommand{\EndArticle}[0]{\bibliography{Bibliography}\bibliographystyle{plainnat}\end{document}}

% 
%\newcommand{\citep}[1]{\cite{#1}}

\colorSectionsForArticle



\beginArtNoToc

%\generatetitle{FIXME put title here}
%\chapter{FIXME put title here}
\label{chap:\basename}
%\section{Motivation}
%\section{Guts}

\makeproblem{Gaussian beam (2010 final exam q5)}{pr:optics2010ExamGaussian:1}{

\makesubproblem{Spot size}{modernOptics:optics2010ExamGaussian:a}

A Gaussian beam with wavelength $0.8 \mu \text{m}$ has its minimum waist of $0.5 \text{m m}$ located in the middle of a parallel glass plate of thickness $5 \text{c m}$ and refractive index $1.5$.  The axis of the beam is perpendicular to the surfaces of the glass plate.  The beaam emerges from the glass plate and strikes a mirror at normal incidence $10 \text{c m}$ away.  When the beam passes back through its original location what is its spot size?

\makesubproblem{Angular divergence}{modernOptics:optics2010ExamGaussian:b}

When the beam emerges from the plate again after re-passing its beam waist, what is its angular divergence?

} % makeproblem

\makeanswer{pr:optics2010ExamGaussian:1}{

\makeSubAnswer{Spot size}{modernOptics:optics2010ExamGaussian:a}

TODO.

\makeSubAnswer{Angular divergence}{modernOptics:optics2010ExamGaussian:b}

TODO.

} % makeanswer

% this is to produce the sites.google url and version info and so forth (for blog posts)
%\vcsinfo
\EndArticle
%\EndNoBibArticle
