%
% Copyright � 2015 Peeter Joot.  All Rights Reserved.
% Licenced as described in the file LICENSE under the root directory of this GIT repository.
%
\makeproblem{Polarization power loss.}{advancedantenna:problemSet1:4}{ 
Transmitting and receiving antennas operating at 1 \si{GHz} have gains of 20 and 15 \si{dB}
respectively and are separated by a distance of 1 \si{km}. Find the power delivered to a
matched load when the input power is 150 \si{W} and when
\makesubproblem{}{advancedantenna:problemSet1:4a}
both antennas are polarization matched.
\makesubproblem{}{advancedantenna:problemSet1:4b}
One antenna is lineraly polarized and the other is circularly polarized.
} % makeproblem

\makeanswer{advancedantenna:problemSet1:4}{ 
\makeSubAnswer{}{advancedantenna:problemSet1:4a}

Answering this requires an application of the Friis transmittion equation.  First note that the gains in non-dB units are

\begin{subequations}
\begin{dmath}\label{eqn:advancedantennaProblemSet1Problem4:20}
G_1 = 10^{20/10},
\end{dmath}
\begin{dmath}\label{eqn:advancedantennaProblemSet1Problem4:40}
G_2 = 10^{15/10}
\end{dmath}
\end{subequations}

The wavelength is

\begin{dmath}\label{eqn:advancedantennaProblemSet1Problem4:60}
\lambda = \frac{c}{\nu} = \frac{3 \times 10^8 \,\si{m/s}}{10^9 \,\si{s^{-1}}} = 0.3 \, \si{m}
\end{dmath}

From the Friss equation, the receiving antenna has power

\begin{dmath}\label{eqn:advancedantennaProblemSet1Problem4:80}
P_\txtr 
= P_\txtt \lr{ \frac{\lambda}{4 \pi R}}^2 G_1 G_2
= 150 \, \si{W} \lr{ \frac{0.3 \,\si{m}}{4 \pi (10^6 \,\si{m})} }^2 10^{3.5}
= 0.27 \, \si{n W}.
\end{dmath}

\makeSubAnswer{}{advancedantenna:problemSet1:4b}

TODO.
}

