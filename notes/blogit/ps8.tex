%
% Copyright � 2016 Peeter Joot.  All Rights Reserved.
% Licenced as described in the file LICENSE under the root directory of this GIT repository.
%
%{
\newcommand{\authorname}{Peeter Joot}
\newcommand{\email}{peeterjoot@protonmail.com}
\newcommand{\basename}{FIXMEbasenameUndefined}
\newcommand{\dirname}{notes/FIXMEdirnameUndefined/}

\renewcommand{\basename}{ps8}
\renewcommand{\dirname}{phy1610-scientific-computing/ps8}
%\newcommand{\dateintitle}{}
%\newcommand{\keywords}{}

\newcommand{\authorname}{Peeter Joot}
\newcommand{\onlineurl}{http://sites.google.com/site/peeterjoot2/math2013/\basename.pdf}
\newcommand{\sourcepath}{\dirname\basename.tex}
\newcommand{\generatetitle}[1]{\chapter{#1}}

\newcommand{\vcsinfo}{%
\section*{}
\noindent{\color{DarkOliveGreen}{\rule{\linewidth}{0.1mm}}}
\paragraph{Document version}
%\paragraph{\color{Maroon}{Document version}}
{
\small
\begin{itemize}
\item Available online at:\\ 
\href{\onlineurl}{\onlineurl}
\item Git Repository: \input{./.revinfo/gitRepo.tex}
\item Source: \sourcepath
\item last commit: \input{./.revinfo/gitCommitString.tex}
\item commit date: \input{./.revinfo/gitCommitDate.tex}
\end{itemize}
}
}

%\PassOptionsToPackage{dvipsnames,svgnames}{xcolor}
\PassOptionsToPackage{square,numbers}{natbib}
\documentclass{scrreprt}

\usepackage[left=2cm,right=2cm]{geometry}
\usepackage[svgnames]{xcolor}
\usepackage{peeters_layout}

\usepackage{natbib}

\usepackage[
colorlinks=true,
bookmarks=false,
pdfauthor={\authorname, \email},
backref 
]{hyperref}

% http://tex.stackexchange.com/questions/75773/how-to-reference-problems-by-the-text-label-in-an-exercise-envioronment
\usepackage[english]{cleveref}
\crefname{Exercise}{exercise}{exercises}
\Crefname{Exercise}{Exercise}{Exercises}

\RequirePackage{titlesec}
\RequirePackage{ifthen}

% http://stackoverflow.com/questions/4932910/date-in-the-tabular-environment
\makeatletter
\let\insertdate\@date
\makeatother

\titleformat{\chapter}[display]
{\bfseries\Large}
{\color{DarkSlateGrey}\filleft \authorname
\ifthenelse{\isundefined{\studentnumber}}{}{\\ \studentnumber}
\ifthenelse{\isundefined{\email}}{}{\\ \email}
\ifthenelse{\isundefined{\dateintitle}}{}{\\ \insertdate}
%\ifthenelse{\isundefined{\coursename}}{}{\\ \coursename} % put in title instead.
}
{4ex}
{\color{DarkOliveGreen}{\titlerule}\color{Maroon}
\vspace{2ex}%
\filright}
[\vspace{2ex}%
\color{DarkOliveGreen}\titlerule
]

\newcommand{\beginArtWithToc}[0]{\begin{document}\tableofcontents}
\newcommand{\beginArtNoToc}[0]{\begin{document}}
\newcommand{\EndNoBibArticle}[0]{\end{document}}
\newcommand{\EndArticle}[0]{\bibliography{Bibliography}\bibliographystyle{plainnat}\end{document}}

% 
%\newcommand{\citep}[1]{\cite{#1}}

\colorSectionsForArticle



\usepackage{peeters_layout_exercise}
\usepackage{peeters_braket}
\usepackage{peeters_figures}
%\usepackage{siunitx}

\beginArtNoToc

\generatetitle{phy1620 ps8 notes and report}
%\chapter{phy1620 ps8 notes and report}
%\label{chap:ps8}

\section{Time evolution matrix}

For forward time differences, the discretized diffusion equation has the form

\begin{dmath}\label{eqn:ps8:20}
T_{i+1} - T_i = \frac{D \Delta t}{\Delta x^2} \lr{ T_i(j+1) - 2 T_i(j) + T_i(j-1) },
\end{dmath}

where the spatial indexes \( j-1, j, j+1 \) are to be treated as cyclic variables over \( [1, 2, \cdots, N-1, N] \), and \( i, i+1, \cdots\) are the timestep indexes.  With

\begin{dmath}\label{eqn:ps8:40}
\alpha = \frac{D \Delta t}{\Delta x^2},
\end{dmath}

This takes the form

\begin{dmath}\label{eqn:ps8:60}
T_{i+1} = 
\alpha T_i(j+1) + \lr{ 1 - 2 \alpha }  T_i(j) + \alpha T_i(j-1)
.
\end{dmath}

Writing this out explicitly, but ignoring the time dependence index, we have
\begin{equation}\label{eqn:ps8:80}
\begin{aligned}
T(1) &\rightarrow \alpha T(2) + \lr{ 1 - 2 \alpha } T(1) + \alpha T(N) \\
T(2) &\rightarrow \alpha T(3) + \lr{ 1 - 2 \alpha } T(2) + \alpha T(1) \\
       & \vdots \\
T(N-1) &\rightarrow \alpha T(N) + \lr{ 1 - 2 \alpha }  T(N-1) + \alpha T(N-2) \\
T(N) &\rightarrow \alpha T(1) + \lr{ 1 - 2 \alpha }  T(N) + \alpha T(N-1),
\end{aligned}
\end{equation}

or in matrix form

\begin{dmath}\label{eqn:ps8:100}
\begin{bmatrix}
T_{i+1}(1) \\
T_{i+1}(2) \\
\vdots \\
T_{i+1}(N-1) \\
T_{i+1}(N) \\
\end{bmatrix}
=
\begin{bmatrix}
1 - 2 \alpha & \alpha       &           &              & \alpha       \\
\alpha       & 1 - 2 \alpha & \alpha    &              &              \\
             &              & \ddots    &              &              \\ 
             &              & \alpha    & 1 - 2 \alpha & \alpha       \\ 
\alpha       &              &           &  \alpha      & 1 - 2 \alpha \\ 
\end{bmatrix}
\begin{bmatrix}
T_{i}(1) \\
T_{i}(2) \\
\vdots \\
T_{i}(N-1) \\
T_{i}(N) \\
\end{bmatrix}
\end{dmath}

\section{PDE graphical results}

The graphical sparkline display for PDE method looked odd, and is shown in \cref{fig:diffringPuttyOutput:diffringPuttyOutputFig1}.  However, looking at the raw data from diffout.dat, I see that the density vector is almost entirely uniform

\imageFigure{../../figures/phy1610/diffringPuttyOutputFig1}{sparkline Histogram display for PDE method.}{fig:diffringPuttyOutput:diffringPuttyOutputFig1}{0.3}

\begin{dmath}\label{eqn:ps8:120}
\Brho = \lr{ 0.00417546,0.00417545,0.00417544,0.00417543, \cdots }
\end{dmath}

This is what we expect given no sources.  The oddness of the display turns out to be an issue with putty, which doesn't display the boxes in a nicely visible fashion.

\section{Walkring results}

Looking at the graphical results for the walkring method, even for very high walker counts, my results looked like garbage.  For example, two such graphs can be seen in 
\cref{fig:walkringPuttyOutput:walkringPuttyOutputFig3}, and
\cref{fig:walkringPuttyOutput:walkringPuttyOutputFig2}.

\imageFigure{../../figures/phy1610/walkringPuttyOutputFig3}{(putty) Walkring histogram data for \( Z = 1000 \).}{fig:walkringPuttyOutput:walkringPuttyOutputFig3}{0.3}
\imageFigure{../../figures/phy1610/walkringPuttyOutputFig2}{(putty) Walkring histogram data for \( Z = 12000 \).}{fig:walkringPuttyOutput:walkringPuttyOutputFig2}{0.3}

It wasn't possible to easily compare the walkring walker position vector dumps in walkout.dat to the 
density vector dumps from diffout.dat.

To rectify this I reworked the output code, so that walkout.dat displays the density vector instead of
the position vector.  Doing so I see that the \( Z = 1000 \) density vector is not actually terribly unreasonable
looking

\begin{dmath}\label{eqn:ps8:140}
\Brho = \lr{ 0.006,0.002,0.003,0.003,0.001,0.006,0.004,0.008,\cdots }.
\end{dmath}

For \( Z = 12000 \) things are still more uniform
\begin{dmath}\label{eqn:ps8:160}
\Brho = \lr{ 0.005,0.00341667,0.00475,0.00441667,0.00458333,0.00416667,0.0045,0.00391667,0.00358333,0.00333333, \cdots }.
\end{dmath}

Because the final results are expected to be uniformly flat, using mean and stddev was a good metric
for comparison.  I have added two additional columns for those values (for both diffring and walkring, which now share code).

Using those I was able to see the variation of the stddev with \( Z \).  
The question of what value of \( Z \) is required for equivalence of the two methods also requires a required precison.  The chart of \cref{fig:stddevCompareSheet:stddevCompareSheetFig4}, shows that a stddev of \( 0.0004 \) can be obtained with \( Z = 24000 \).  Due to time constraints and the slow evaluation of the random iteration for large \( Z \) I did not attempt to find an equivalence better than that.

\imageFigure{../../figures/phy1610/stddevCompareSheetFig4}{stddev variation with \( Z\).}{fig:stddevCompareSheet:stddevCompareSheetFig4}{0.3}

A visual inspection of the \( Z = 24000 \) density vector from the final timestep iteration shows the uniformity has still not matched that of the PDE method in the same number of steps.

\begin{dmath}\label{eqn:ps8:n}
\Brho = \lr{ 0.00408333,0.00416667,0.00408333,0.00420833,0.00525,0.00358333,0.00375,0.00445833,0.00395833,\cdots}.
\end{dmath}

Only after obtaining these results did I find that the odd graphical displays observed above were an artifact of the putty terminal emumlator.  After logging into my system directly using the X11 interface, I see much more reasonable results from the sparkline code as illustrated in \cref{fig:diffringOutput:diffringOutputFig6} and \cref{fig:walkring1000Output:walkring1000OutputFig5}.

\imageFigure{../../figures/phy1610/diffringOutputFig6}{(X11)diffring profile.}{fig:diffringOutput:diffringOutputFig6}{0.3}
\imageFigure{../../figures/phy1610/walkring1000OutputFig5}{(X11) walkring profile \( Z = 1000\).}{fig:walkring1000Output:walkring1000OutputFig5}{0.3}

\section{Preferred method.}

For this particular simulation, the PDE method is the clear winner, as it converges very quickly, even when run in a serial mode.  There are clear parallelization opportunties for the walkring code, but the effort to code those for this problem do not seem worthwhile.

%}
%\EndArticle
\EndNoBibArticle
