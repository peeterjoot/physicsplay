%
% Copyright � 2014 Peeter Joot.  All Rights Reserved.
% Licenced as described in the file LICENSE under the root directory of this GIT repository.
%
\makeproblem{Finite element methods}{multiphysics:problemSet2b:2}{ 

When modeling distributions of charged particles governed by drift-diffusion equations, equilibrium analysis typically leads to the following nonlinear Poisson equation

\begin{equation}\label{eqn:multiphysicsProblemSet2bProblem2:20}
- \PDSq{x}{\psi} = - \lr{ e^{\psi(x)} - e^{-\psi(x)} },
\end{equation}

where we will consider the interval \( x \in [0,1] \) with the boundary conditions \( \psi(0) = -V \) and \( \psi(1) = V\) .
If a simple finite-difference scheme is used to solve the nonlinear Poisson equation on an N-node grid, the discrete equations are


\begin{equation}\label{eqn:multiphysicsProblemSet2bProblem2:40}
2 \psi_i - \psi_{i+1} - \psi_{i-1} + \Delta x^2 \lr{ e^{\psi_i} - e^{-\psi_i} } = 0,
\end{equation}

for \( i \in [2, \cdots, N - 1] \),

\begin{equation}\label{eqn:multiphysicsProblemSet2bProblem2:60}
2 \psi_1 - \psi_{2} - \lr{ -V } + \Delta x^2 \lr{ e^{\psi_1} - e^{-\psi_1} } = 0,
\end{equation}

and
\begin{equation}\label{eqn:multiphysicsProblemSet2bProblem2:80}
2 \psi_N - \psi_{N-1} - V + \Delta x^2 \lr{ e^{\psi_N} - e^{-\psi_N} } = 0.
\end{equation}

Note, \( \Delta x = 1/(N +1) \) and not \( 1/(N -1) \). The nodes at \( x = 0 \) and \( x = 1 \) are not included in the discretization, but rather enter through the boundary conditions.

\makesubproblem{}{multiphysics:problemSet2b:2a}
Prove that the Jacobian associated with the above discretized equations is nonsingular regardless of the values for the \( \psi_i \)'s.

\makesubproblem{}{multiphysics:problemSet2b:2b}
Based on this observation, how do you expect damped Newton methods will perform, when applied to solving this problem?

\makesubproblem{}{multiphysics:problemSet2b:2c}
Solve the above equations with a multidimensional Newton's method using a zero initial guess for the \( \psi_i\)'s, and \( N = 100 \).  Demonstrate that your program achieves quadratic convergence.

\makesubproblem{}{multiphysics:problemSet2b:2d}
and determine the number of Newton iterations required to insure one part in \( 10^6 \) accuracy in the solution for two cases: when \( V = 1 \) and when \( V = 20 \).

\makesubproblem{}{multiphysics:problemSet2b:2e}
What happens when \( V = 100 \)?

} % makeproblem

\makeanswer{multiphysics:problemSet2b:2}{ 
\makeSubAnswer{}{multiphysics:problemSet2b:2a}

Before starting consider \cref{fig:ps2bFiniteElement:ps2bFiniteElementFig3}.  that illustrates the partitioning of a possible solution.  This shows why \( \Delta x = 1/(N+1) \).

\imageFigure{../../figures/ece1254/ps2bFiniteElementFig3}{Partitioning intervals}{fig:ps2bFiniteElement:ps2bFiniteElementFig3}{0.3}

In matrix form the nodal equations are

%2 \psi_i - \psi_{i+1} - \psi_{i-1} + \Delta x^2 \lr{ e^{\psi_i} - e^{-\psi_i} } = 0,
%2 \psi_2 - \psi_{3} - \psi_{1} + \Delta x^2 \lr{ e^{\psi_2} - e^{-\psi_2} } = 0,
%2 \psi_N-1 - \psi_{N} - \psi_{N-2} + \Delta x^2 \lr{ e^{\psi_N-1} - e^{-\psi_N-1} } = 0,
\begin{equation}\label{eqn:multiphysicsProblemSet2bProblem2:100}
\begin{bmatrix}
 2 & -1 &         &    & \\
-1 &  2 & -1      &    & \\
   &    & \ddots  &    & \\
   &    &    -1   & 2  &  -1  \\
   &    &         & -1 &  2  \\
\end{bmatrix}
\begin{bmatrix}
\psi_1 \\
\psi_2 \\
\vdots \\
\psi_{N-1} \\
\psi_N \\
\end{bmatrix}
+
\lr{\Delta x}^2
\begin{bmatrix}
e^{\psi_1} - e^{-\psi_1} \\
e^{\psi_2} - e^{-\psi_2} \\
\vdots \\
e^{\psi_{N-1}} - e^{-\psi_{N-1}} \\
e^{\psi_N} - e^{-\psi_N} \\
\end{bmatrix}
=
\begin{bmatrix}
-V \\
0 \\
\vdots \\
0 \\
V
\end{bmatrix}.
\end{equation}

The difference of exponentials can be expanded in Taylor series

\begin{dmath}\label{eqn:multiphysicsProblemSet2bProblem2:120}
e^{\psi_i} - e^{-\psi_i}
=
\lr{ 1 + \psi_i + \psi_i^3/2 + \psi_i^3/6 + \cdots }
- 
\lr{ 1 - \psi_i + \psi_i^3/2 - \psi_i^3/6 + \cdots }
=
2 \psi_i + \inv{3} \psi_i^3 + \cdots,
\end{dmath}

so the Jacobian is just \( 2 I \), which is never singular.  The linear approximation of the nodal equations is

\begin{equation}\label{eqn:multiphysicsProblemSet2bProblem2:140}
\begin{bmatrix}
 2 \lr{1 + \lr{\Delta x}^2} & -1 &         &    & \\
-1 &  2 \lr{1 + \lr{\Delta x}^2} & -1      &    & \\
   &    & \ddots  &    & \\
   &    &    -1   & 2 \lr{ 1 + \lr{\Delta x}^2 } &  -1  \\
   &    &         & -1 &  2 \lr{ 1 + \lr{\Delta x}^2 } \\
\end{bmatrix}
\begin{bmatrix}
\psi_1 \\
\psi_2 \\
\vdots \\
\psi_{N-1} \\
\psi_N \\
\end{bmatrix}
=
\begin{bmatrix}
-V \\
0 \\
\vdots \\
0 \\
V
\end{bmatrix}.
\end{equation}

FIXME: does he want us to show that this complete Jacobian is non-singular (which is what matters since that's what has to be inverted to apply newton's method).

\makeSubAnswer{}{multiphysics:problemSet2b:2b}

TODO.
\makeSubAnswer{}{multiphysics:problemSet2b:2c}

TODO.
\makeSubAnswer{}{multiphysics:problemSet2b:2d}

TODO.
\makeSubAnswer{}{multiphysics:problemSet2b:2e}

TODO.
}
