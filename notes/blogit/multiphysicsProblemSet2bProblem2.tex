%
% Copyright � 2014 Peeter Joot.  All Rights Reserved.
% Licenced as described in the file LICENSE under the root directory of this GIT repository.
%
\makeproblem{description}{multiphysics:problemSet2b:2}{ 

When modeling distributions of charged particles governed by drift-diffusion equations, equilibrium analysis typically leads to the following nonlinear Poisson equation

\begin{equation}\label{eqn:multiphysicsProblemSet2bProblem2:20}
- \PDSq{x}{\psi} = - \lr{ e^{\psi(x)} - e^{-\psi(x)} },
\end{equation}

where we will consider the interval \( x \in [0,1] \) with the boundary conditions \( \psi(0) = -V \) and \( \psi(1) = V\) .
If a simple finite-difference scheme is used to solve the nonlinear Poisson equation on an N-node grid, the discrete equations are


\begin{equation}\label{eqn:multiphysicsProblemSet2bProblem2:40}
2 \psi_i - \psi_{i+1} - \psi_{i-1} + \Delta x^2 \lr{ e^{\psi_i} - e^{-\psi_i} } = 0,
\end{equation}

for \( i \in [2, \cdots, N - 1] \),

\begin{equation}\label{eqn:multiphysicsProblemSet2bProblem2:60}
2 \psi_1 - \psi_{2} - \lr{ -V } + \Delta x^2 \lr{ e^{\psi_1} - e^{-\psi_1} } = 0,
\end{equation}

and
\begin{equation}\label{eqn:multiphysicsProblemSet2bProblem2:80}
2 \psi_N - \psi_{N-1} - V + \Delta x^2 \lr{ e^{\psi_N} - e^{-\psi_N} } = 0.
\end{equation}

Note, \( \Delta x = 1/(N +1) \) and not \( 1/(N -1) \). The nodes at \( x = 0 \) and \( x = 1 \) are not included in the discretization, but rather enter through the boundary conditions.

\makesubproblem{}{multiphysics:problemSet2b:2a}
Prove that the Jacobian associated with the above discretized equations is nonsingular regardless of the values for the \( \psi_i \)�s.

\makesubproblem{}{multiphysics:problemSet2b:2b}
Based on this observation, how do you expect damped Newton methods will perform, when applied to solving this problem?

\makesubproblem{}{multiphysics:problemSet2b:2c}
Solve the above equations with a multidimensional Newton�s method using a zero initial guess for the \( \psi_i\)�s, and \( N = 100 \).  Demonstrate that your program achieves quadratic convergence.

\makesubproblem{}{multiphysics:problemSet2b:2d}
and determine the number of Newton iterations required to insure one part in \( 10^6 \) accuracy in the solution for two cases: when \( V = 1 \) and when \( V = 20 \).

\makesubproblem{}{multiphysics:problemSet2b:2e}
What happens when \( V = 100 \)?

} % makeproblem

\makeanswer{multiphysics:problemSet2b:2}{ 
\makeSubAnswer{}{multiphysics:problemSet2b:2a}

TODO.
\makeSubAnswer{}{multiphysics:problemSet2b:2b}

TODO.
\makeSubAnswer{}{multiphysics:problemSet2b:2c}

TODO.
\makeSubAnswer{}{multiphysics:problemSet2b:2d}

TODO.
\makeSubAnswer{}{multiphysics:problemSet2b:2e}

TODO.
}
