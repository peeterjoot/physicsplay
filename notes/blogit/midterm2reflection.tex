%
% Copyright � 2013 Peeter Joot.  All Rights Reserved.
% Licenced as described in the file LICENSE under the root directory of this GIT repository.
%
\newcommand{\authorname}{Peeter Joot}
\newcommand{\email}{peeterjoot@protonmail.com}
\newcommand{\basename}{FIXMEbasenameUndefined}
\newcommand{\dirname}{notes/FIXMEdirnameUndefined/}

\renewcommand{\basename}{midterm2reflection}
\renewcommand{\dirname}{notes/phy452/}
%\newcommand{\dateintitle}{}
\newcommand{\keywords}{Statistical mechanics, PHY452H1S}

\newcommand{\nbref}[1]{notes/phy452/mathematica/#1}

\newcommand{\authorname}{Peeter Joot}
\newcommand{\onlineurl}{http://sites.google.com/site/peeterjoot2/math2013/\basename.pdf}
\newcommand{\sourcepath}{\dirname\basename.tex}
\newcommand{\generatetitle}[1]{\chapter{#1}}

\newcommand{\vcsinfo}{%
\section*{}
\noindent{\color{DarkOliveGreen}{\rule{\linewidth}{0.1mm}}}
\paragraph{Document version}
%\paragraph{\color{Maroon}{Document version}}
{
\small
\begin{itemize}
\item Available online at:\\ 
\href{\onlineurl}{\onlineurl}
\item Git Repository: \input{./.revinfo/gitRepo.tex}
\item Source: \sourcepath
\item last commit: \input{./.revinfo/gitCommitString.tex}
\item commit date: \input{./.revinfo/gitCommitDate.tex}
\end{itemize}
}
}

%\PassOptionsToPackage{dvipsnames,svgnames}{xcolor}
\PassOptionsToPackage{square,numbers}{natbib}
\documentclass{scrreprt}

\usepackage[left=2cm,right=2cm]{geometry}
\usepackage[svgnames]{xcolor}
\usepackage{peeters_layout}

\usepackage{natbib}

\usepackage[
colorlinks=true,
bookmarks=false,
pdfauthor={\authorname, \email},
backref 
]{hyperref}

% http://tex.stackexchange.com/questions/75773/how-to-reference-problems-by-the-text-label-in-an-exercise-envioronment
\usepackage[english]{cleveref}
\crefname{Exercise}{exercise}{exercises}
\Crefname{Exercise}{Exercise}{Exercises}

\RequirePackage{titlesec}
\RequirePackage{ifthen}

% http://stackoverflow.com/questions/4932910/date-in-the-tabular-environment
\makeatletter
\let\insertdate\@date
\makeatother

\titleformat{\chapter}[display]
{\bfseries\Large}
{\color{DarkSlateGrey}\filleft \authorname
\ifthenelse{\isundefined{\studentnumber}}{}{\\ \studentnumber}
\ifthenelse{\isundefined{\email}}{}{\\ \email}
\ifthenelse{\isundefined{\dateintitle}}{}{\\ \insertdate}
%\ifthenelse{\isundefined{\coursename}}{}{\\ \coursename} % put in title instead.
}
{4ex}
{\color{DarkOliveGreen}{\titlerule}\color{Maroon}
\vspace{2ex}%
\filright}
[\vspace{2ex}%
\color{DarkOliveGreen}\titlerule
]

\newcommand{\beginArtWithToc}[0]{\begin{document}\tableofcontents}
\newcommand{\beginArtNoToc}[0]{\begin{document}}
\newcommand{\EndNoBibArticle}[0]{\end{document}}
\newcommand{\EndArticle}[0]{\bibliography{Bibliography}\bibliographystyle{plainnat}\end{document}}

% 
%\newcommand{\citep}[1]{\cite{#1}}

\colorSectionsForArticle



\beginArtNoToc

\generatetitle{Midterm II reflection}
%\chapter{Midterm II reflection}
\label{chap:midterm2reflection}

Here's some reflection about this Thursday's midterm, redoing the problems without the mad scramble.  I don't think my results are too different from what I did in the midterm, even doing them casually now, but I'll have to see after grading if these solutions are good.

\makeoproblem{Magnetic field spin level splitting}{pr:midterm2reflection:1}{2013 midterm II p1}{
%(5 points)
A particle with spin $S$ has $2 S + 1$ states $-S, -S + 1, \cdots S-1, S$.  When exposed to a magnetic field, state splitting results in energy $E_m = \hbar m B$.  Calculate the partition function, and use this to find the temperature specific magnetization.  A ``sum the geometric series'' hint was given.
} % makeoproblem

\makeanswer{pr:midterm2reflection:1}{ 

Our partition function is

\begin{dmath}\label{eqn:midterm2reflection:20}
Z 
= \sum_{m = -S}^S e^{-\hbar \beta m B}
= 
e^{-\hbar \beta S B}
\sum_{m = -S}^S e^{-\hbar \beta (m + S) B}
= 
e^{\hbar \beta S B}
\sum_{n = 0}^{2 S} e^{-\hbar \beta n B}.
\end{dmath}

Writing

\begin{equation}\label{eqn:midterm2reflection:40}
a = e^{-\hbar \beta B},
\end{equation}

that is

\begin{dmath}\label{eqn:basicStatMechLecture13:60}
Z = 
a^{-S}
\sum_{n = 0}^{2 S} a^n 
=
a^{-S} \frac{ a^{2 S + 1} - 1 }{a - 1}
=
\frac{ a^{S + 1} - a^{-S} }{a - 1}
=
\frac{ a^{S + 1/2} - a^{-S - 1/2} }{a^{1/2} - a^{-1/2}}.
\end{dmath}

Substitution of $a$ gives us
\begin{equation}\label{eqn:midterm2reflection:80}
\myBoxed{
Z = 
\frac
{ \sinh( \hbar \beta B (S + 1/2) ) }
{ \sinh( \hbar \beta B /2 )  }.
}
\end{equation}

To calculate the magnetization $M$, I used

\begin{equation}\label{eqn:midterm2reflection:100}
M = -\expectation{H}/B.
\end{equation}

As \citep{kittel1980thermal} defines magnetization for a spin system.  It was pointed out to me after the test that magnetization was defined differently in class as

\begin{equation}\label{eqn:midterm2reflection:140}
\mu = \PD{F}{B}.
\end{equation}

These are, up to a sign, identical, at least in this case, since we have $\beta$ and $B$ travelling together in the partition function.  In terms of the average energy

\begin{dmath}\label{eqn:basicStatMechLecture13:160}
M = -\frac{\expectation{H}}{B}
= \inv{B} \PD{\beta}{} \ln Z(\beta B)
= \inv{Z B} \PD{\beta}{}Z(\beta B)
= \inv{Z} \PD{(\beta B)}{} Z(\beta B)
\end{dmath}

Compare this to the in-class definition of magnetization

\begin{dmath}\label{eqn:basicStatMechLecture13:180}
\mu 
= \PD{B}{F}
= \PD{B}{} \lr{ - k_{\mathrm{B}} T \ln Z(\beta B) }
= -\PD{B}{} \frac{\ln Z (\beta B)}{\beta}
= -\inv{\beta Z} \PD{B}{} Z(\beta B)
= -\inv{Z} \PD{(\beta B)}{} Z(\beta B).
\end{dmath}

For this derivative we have

\begin{dmath}\label{eqn:basicStatMechLecture13:200}
\PD{(\beta B)}{} \ln Z
=
\PD{(\beta B)}{} \ln 
\frac
{ \sinh( \hbar \beta B (S + 1/2) ) }
{ \sinh( \hbar \beta B /2 )  }
=
\PD{(\beta B)}{} \lr{
   \ln \sinh( \hbar \beta B (S + 1/2) ) 
   - \ln \sinh( \hbar \beta B /2 ) 
}
=
\frac{\hbar }{2}
\lr{
   (2 S + 1) \coth( \hbar \beta B (S + 1/2) ) 
   - \coth( \hbar \beta B /2 ) 
}.
\end{dmath}

This gives us
\begin{dmath}\label{eqn:basicStatMechLecture13:220}
\mu 
= 
-\inv{Z} 
\frac{\hbar }{2}
\lr{
   (2 S + 1) \coth( \hbar \beta B (S + 1/2) ) 
   - \coth( \hbar \beta B /2 ) 
}
=
-
\frac
{ \sinh( \hbar \beta B /2 )  }
{ \sinh( \hbar \beta B (S + 1/2) ) }
\frac{\hbar }{2}
\lr{
   (2 S + 1) \coth( \hbar \beta B (S + 1/2) ) 
   - \coth( \hbar \beta B /2 ) 
}
\end{dmath}

After some simplification (done offline in \nbref{midtermTwoQ1FinalSimplificationMu.nb}) we get

\begin{equation}\label{eqn:midterm2reflection:240}
\myBoxed{
\mu
=
\hbar 
\frac{
(s+1) \sinh(\hbar \beta B s)
-s \sinh(\hbar \beta B (s+1))
}{
\cosh(\hbar \beta B(2 s+1)) - 1
}.
}
\end{equation}

I got something like this on the midterm, but recall doing it somehow much differently.
%  To see how I'll have to wait and see when I get my graded results back.

} % makeanswer

\makeoproblem{Pertubation of classical harmonic oscillator}{pr:midterm2reflection:2}{2013 midterm II p2}{ 
%(10 points)

Consider a single particle perturbation of a classical simple harmonic oscillator Hamiltonian 

\begin{equation}\label{eqn:midterm2reflection:120}
H = \inv{2} m \omega^2 \lr{x^2 + y^2} + \inv{2 m} \lr{p_x^2 + p_y^2} + a x^4 + by^6
\end{equation}

Calculate the canonical partition function, mean energy and specific heat of this system.

There were some instructions about the form to put the integrals in.  
%I may not have followed those correctly since people were talking about Taylor expansion and I differentiated under the integral sign, leaving the $x$ and $y$ integrals unevaluated.
} % makeoproblem

\makeanswer{pr:midterm2reflection:2}{ 

The canonical partition function is

\begin{dmath}\label{eqn:midterm2reflection:260}
Z 
= 
\int dx dy dp_x dp_y e^{-\beta H}
=
\int dx e^{-\beta 
\lr{
\inv{2} m \omega^2 x^2 + a x^4
}
}
\int dy e^{-\beta 
\lr{
\inv{2} m \omega^2 y^2 + b y^6
}
}
\int dp_x dp_y e^{-\beta p_x^2/2 m} e^{-\beta p_y^2/2 m}.
\end{dmath}

With 

\begin{subequations}
\begin{dmath}\label{eqn:midterm2reflection:280}
u = \sqrt{\frac{\beta}{2m}} p_x
\end{dmath}
\begin{dmath}\label{eqn:midterm2reflection:300}
v = \sqrt{\frac{\beta}{2m}} p_y,
\end{dmath}
\end{subequations}

the momentum integrals are

\begin{dmath}\label{eqn:midterm2reflection:320}
\int dp_x dp_y e^{-\beta p_x^2/2 m} e^{-\beta p_y^2/2 m}
=
\frac{2m}{\beta}
\int du du e^{- u^2 - v^2}
=
\frac{m}{\beta}
2 \pi
\int 2 r dr e^{- r^2}
=
\frac{2 \pi m}{\beta}.
\end{dmath}

Writing 

\begin{subequations}
\begin{dmath}\label{eqn:midterm2reflection:340}
f(x) = \inv{2} m \omega^2 x^2 + a x^4
\end{dmath}
\begin{dmath}\label{eqn:midterm2reflection:360}
g(x) = \inv{2} m \omega^2 y^2 + b y^4,
\end{dmath}
\end{subequations}

we have

\begin{dmath}\label{eqn:midterm2reflection:380}
\myBoxed{
Z = 
\frac{2 \pi m}{\beta}
\int dx e^{- \beta f(x)}
\int dy e^{- \beta g(y)}.
}
\end{dmath}

The mean energy is

\begin{dmath}\label{eqn:midterm2reflection:400}
\expectation{H} 
= \frac{\int H e^{-\beta H}}{\int e^{-\beta H}}
= -\PD{\beta}{} \ln \int e^{-\beta H}
= \PD{\beta}{} 
\lr{
\ln \beta
-\ln \int dx e^{- \beta f(x)}
-\ln \int dy e^{- \beta g(y)}
}
=
\inv{\beta} 
+ \frac{
\int dx f(x) e^{- \beta f(x)}
}
{
\int dx e^{- \beta f(x)}
}
+ \frac{
\int dy g(y) e^{- \beta g(y)}
}
{
\int dy e^{- \beta g(y)}
}.
\end{dmath}

The specific heat follows by differentiating once more
\begin{dmath}\label{eqn:midterm2reflection:420}
C_{\mathrm{V}}
=
\PD{T}{\expectation{H}}
=
\PD{T}{\beta}
\PD{\beta}{\expectation{H}}
= -\inv{k_{\mathrm{B}} T^2}
\PD{\beta}{\expectation{H}}
= -k_{\mathrm{B}} \beta^2
\PD{\beta}{\expectation{H}}
=
- k_{\mathrm{B}} \beta^2
\lr{
-\inv{\beta^2}
+ \PD{\beta}{}
\lr{
\frac{
\int dx f(x) e^{- \beta f(x)}
}
{
\int dx e^{- \beta f(x)}
}
+ \frac{
\int dy g(y) e^{- \beta g(y)}
}
{
\int dy e^{- \beta g(y)}
}
}
}.
\end{dmath}

Differentiating the integral terms we have, for example,

\begin{dmath}\label{eqn:midterm2reflection:440}
\PD{\beta}{} 
\frac{
\int dx f(x) e^{- \beta f(x)}
}
{
\int dx e^{- \beta f(x)}
}
=
-\frac{
\int dx f^2(x) e^{- \beta f(x)}
}
{
\int dx e^{- \beta f(x)}
}
-
\lr{
\frac{
\int dx f(x) e^{- \beta f(x)}
}
{
\int dx e^{- \beta f(x)}
}
}^2,
\end{dmath}

so that the specific heat is
\begin{dmath}\label{eqn:midterm2reflection:460}
\myBoxed{
C_{\mathrm{V}} =
k_{\mathrm{B}} 
\lr{
1 
+ 
\frac{
\int dx f^2(x) e^{- \beta f(x)}
}
{
\int dx e^{- \beta f(x)}
}
+
\lr{
\frac{
\int dx f(x) e^{- \beta f(x)}
}
{
\int dx e^{- \beta f(x)}
}
}^2
+ 
\frac{
\int dy g^2(y) e^{- \beta g(y)}
}
{
\int dy e^{- \beta g(y)}
}
+
\lr{
\frac{
\int dy g(y) e^{- \beta g(y)}
}
{
\int dy e^{- \beta g(y)}
}
}^2
}.
}
\end{dmath}

That's as far as I took this problem.  There was a discussion after the midterm with Eric about Taylor expansion of these integrals.  That's not something that I tried.
} % makeanswer

\EndArticle
