%
% Copyright � 2013 Peeter Joot.  All Rights Reserved.
% Licenced as described in the file LICENSE under the root directory of this GIT repository.
%
\newcommand{\authorname}{Peeter Joot}
\newcommand{\email}{peeterjoot@protonmail.com}
\newcommand{\basename}{FIXMEbasenameUndefined}
\newcommand{\dirname}{notes/FIXMEdirnameUndefined/}

\renewcommand{\basename}{midterm2reflection}
\renewcommand{\dirname}{notes/phy452/}
%\newcommand{\dateintitle}{}
\newcommand{\keywords}{Statistical mechanics, PHY452H1S}

\newcommand{\authorname}{Peeter Joot}
\newcommand{\onlineurl}{http://sites.google.com/site/peeterjoot2/math2013/\basename.pdf}
\newcommand{\sourcepath}{\dirname\basename.tex}
\newcommand{\generatetitle}[1]{\chapter{#1}}

\newcommand{\vcsinfo}{%
\section*{}
\noindent{\color{DarkOliveGreen}{\rule{\linewidth}{0.1mm}}}
\paragraph{Document version}
%\paragraph{\color{Maroon}{Document version}}
{
\small
\begin{itemize}
\item Available online at:\\ 
\href{\onlineurl}{\onlineurl}
\item Git Repository: \input{./.revinfo/gitRepo.tex}
\item Source: \sourcepath
\item last commit: \input{./.revinfo/gitCommitString.tex}
\item commit date: \input{./.revinfo/gitCommitDate.tex}
\end{itemize}
}
}

%\PassOptionsToPackage{dvipsnames,svgnames}{xcolor}
\PassOptionsToPackage{square,numbers}{natbib}
\documentclass{scrreprt}

\usepackage[left=2cm,right=2cm]{geometry}
\usepackage[svgnames]{xcolor}
\usepackage{peeters_layout}

\usepackage{natbib}

\usepackage[
colorlinks=true,
bookmarks=false,
pdfauthor={\authorname, \email},
backref 
]{hyperref}

% http://tex.stackexchange.com/questions/75773/how-to-reference-problems-by-the-text-label-in-an-exercise-envioronment
\usepackage[english]{cleveref}
\crefname{Exercise}{exercise}{exercises}
\Crefname{Exercise}{Exercise}{Exercises}

\RequirePackage{titlesec}
\RequirePackage{ifthen}

% http://stackoverflow.com/questions/4932910/date-in-the-tabular-environment
\makeatletter
\let\insertdate\@date
\makeatother

\titleformat{\chapter}[display]
{\bfseries\Large}
{\color{DarkSlateGrey}\filleft \authorname
\ifthenelse{\isundefined{\studentnumber}}{}{\\ \studentnumber}
\ifthenelse{\isundefined{\email}}{}{\\ \email}
\ifthenelse{\isundefined{\dateintitle}}{}{\\ \insertdate}
%\ifthenelse{\isundefined{\coursename}}{}{\\ \coursename} % put in title instead.
}
{4ex}
{\color{DarkOliveGreen}{\titlerule}\color{Maroon}
\vspace{2ex}%
\filright}
[\vspace{2ex}%
\color{DarkOliveGreen}\titlerule
]

\newcommand{\beginArtWithToc}[0]{\begin{document}\tableofcontents}
\newcommand{\beginArtNoToc}[0]{\begin{document}}
\newcommand{\EndNoBibArticle}[0]{\end{document}}
\newcommand{\EndArticle}[0]{\bibliography{Bibliography}\bibliographystyle{plainnat}\end{document}}

% 
%\newcommand{\citep}[1]{\cite{#1}}

\colorSectionsForArticle



\beginArtNoToc

\generatetitle{Midterm II reflection}
%\chapter{Midterm II reflection}
\label{chap:midterm2reflection}

\makeoproblem{description}{pr:midterm2reflection:1}{2013 midterm II p1}{
%(5 points)
A particle with spin $S$ has $2 S + 1$ states $-S, -S + 1, \cdots S-1, S$.  When exposed to a magnetic field, state splitting results in energy $E_m = \hbar m B$.  Calculate the partition function, and use this to find the temperature specific magnetization.  A ``sum the geometric series'' hint was given.

Notes: I calculated:

\begin{equation}\label{eqn:midterm2reflection:n}
M = -\expectation{E}/B
\end{equation}

%This may have been the wrong definition of magnetization.  <NAME?: Eng Sci guy (with Israli heritage?) used $\partial F/\partial B$.

} % makeoproblem
\makeanswer{pr:midterm2reflection:1}{ TODO. } % makeanswer

\makeoproblem{description}{pr:midterm2reflection:2}{2013 midterm II p2}{ 
%(10 points)

Consider a single particle perturbation of a classical simple harmonic oscillator Hamiltonian 

\begin{equation}\label{eqn:midterm2reflection:n}
H = \inv{2} m \omega^2 \lr{x^2 + y^2} + \inv{2 m} \lr{p_x^2 + p_y^2} + a x^4 + by^6
\end{equation}

Calculate the canonical partition function, mean energy and specific heat of this system.

There were some instructions about the form to put the integrals in.  I may not have followed those correctly since people were talking about Taylor expansion and I differentiated under the integral sign, leaving the $x$ and $y$ integrals unevaluated.
} % makeoproblem

\makeanswer{pr:midterm2reflection:2}{ TODO. } % makeanswer

\EndArticle
%\EndNoBibArticle
