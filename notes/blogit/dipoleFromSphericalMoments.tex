%
% Copyright � 2016 Peeter Joot.  All Rights Reserved.
% Licenced as described in the file LICENSE under the root directory of this GIT repository.
%
%{
\newcommand{\authorname}{Peeter Joot}
\newcommand{\email}{peeterjoot@protonmail.com}
\newcommand{\basename}{FIXMEbasenameUndefined}
\newcommand{\dirname}{notes/FIXMEdirnameUndefined/}

\renewcommand{\basename}{dipoleFromSphericalMoments}
%\renewcommand{\dirname}{notes/phy1520/}
\renewcommand{\dirname}{notes/ece1228-electromagnetic-theory/}
%\newcommand{\dateintitle}{}
%\newcommand{\keywords}{}

\newcommand{\authorname}{Peeter Joot}
\newcommand{\onlineurl}{http://sites.google.com/site/peeterjoot2/math2013/\basename.pdf}
\newcommand{\sourcepath}{\dirname\basename.tex}
\newcommand{\generatetitle}[1]{\chapter{#1}}

\newcommand{\vcsinfo}{%
\section*{}
\noindent{\color{DarkOliveGreen}{\rule{\linewidth}{0.1mm}}}
\paragraph{Document version}
%\paragraph{\color{Maroon}{Document version}}
{
\small
\begin{itemize}
\item Available online at:\\ 
\href{\onlineurl}{\onlineurl}
\item Git Repository: \input{./.revinfo/gitRepo.tex}
\item Source: \sourcepath
\item last commit: \input{./.revinfo/gitCommitString.tex}
\item commit date: \input{./.revinfo/gitCommitDate.tex}
\end{itemize}
}
}

%\PassOptionsToPackage{dvipsnames,svgnames}{xcolor}
\PassOptionsToPackage{square,numbers}{natbib}
\documentclass{scrreprt}

\usepackage[left=2cm,right=2cm]{geometry}
\usepackage[svgnames]{xcolor}
\usepackage{peeters_layout}

\usepackage{natbib}

\usepackage[
colorlinks=true,
bookmarks=false,
pdfauthor={\authorname, \email},
backref 
]{hyperref}

% http://tex.stackexchange.com/questions/75773/how-to-reference-problems-by-the-text-label-in-an-exercise-envioronment
\usepackage[english]{cleveref}
\crefname{Exercise}{exercise}{exercises}
\Crefname{Exercise}{Exercise}{Exercises}

\RequirePackage{titlesec}
\RequirePackage{ifthen}

% http://stackoverflow.com/questions/4932910/date-in-the-tabular-environment
\makeatletter
\let\insertdate\@date
\makeatother

\titleformat{\chapter}[display]
{\bfseries\Large}
{\color{DarkSlateGrey}\filleft \authorname
\ifthenelse{\isundefined{\studentnumber}}{}{\\ \studentnumber}
\ifthenelse{\isundefined{\email}}{}{\\ \email}
\ifthenelse{\isundefined{\dateintitle}}{}{\\ \insertdate}
%\ifthenelse{\isundefined{\coursename}}{}{\\ \coursename} % put in title instead.
}
{4ex}
{\color{DarkOliveGreen}{\titlerule}\color{Maroon}
\vspace{2ex}%
\filright}
[\vspace{2ex}%
\color{DarkOliveGreen}\titlerule
]

\newcommand{\beginArtWithToc}[0]{\begin{document}\tableofcontents}
\newcommand{\beginArtNoToc}[0]{\begin{document}}
\newcommand{\EndNoBibArticle}[0]{\end{document}}
\newcommand{\EndArticle}[0]{\bibliography{Bibliography}\bibliographystyle{plainnat}\end{document}}

% 
%\newcommand{\citep}[1]{\cite{#1}}

\colorSectionsForArticle



\usepackage{peeters_layout_exercise}
\usepackage{peeters_braket}
\usepackage{peeters_figures}
\usepackage{siunitx}
%\usepackage{mhchem} % \ce{}
%\usepackage{macros_bm} % \bcM
%\usepackage{txfonts} % \ointclockwise

\beginArtNoToc

\generatetitle{Dipole field from spherical harmonics}
%\chapter{Dipole field from spherical harmonics}
%\label{chap:dipoleFromSphericalMoments}

As indicated in Jackson \citep{jackson1975cew}, the components of the electric field can be obtained directly from the multipole moments

\begin{dmath}\label{eqn:dipoleFromSphericalMoments:20}
\Phi(\Bx) 
= \inv{4 \pi \epsilon_0} \sum \frac{4 \pi}{ (2 l + 1) r^{l + 1} } q_{l m} Y_{l m},
\end{dmath}

so for the \( l,m \) contribution to this sum the components of the electric field are

\begin{dmath}\label{eqn:dipoleFromSphericalMoments:40}
E_r 
=
\inv{\epsilon_0} \sum \frac{l+1}{ (2 l + 1) r^{l + 2} } q_{l m} Y_{l m},
\end{dmath}

\begin{dmath}\label{eqn:dipoleFromSphericalMoments:60}
E_\theta 
= -\inv{\epsilon_0} \sum \frac{1}{ (2 l + 1) r^{l + 2} } q_{l m} \partial_\theta Y_{l m}
\end{dmath}

\begin{dmath}\label{eqn:dipoleFromSphericalMoments:80}
E_\phi 
= -\inv{\epsilon_0} \sum \frac{1}{ (2 l + 1) r^{l + 2} \sin\theta } q_{l m} \partial_\phi Y_{l m}
= -\inv{\epsilon_0} \sum \frac{j m}{ (2 l + 1) r^{l + 2} \sin\theta } q_{l m} Y_{l m}.
\end{dmath}

Here I've translated from CGS to SI.  Let's calculate the \( l = 1 \) electric field components directly from these expressions and check against the previously calculated results.

\begin{dmath}\label{eqn:dipoleFromSphericalMoments:100}
E_r 
=
\inv{\epsilon_0} \frac{2}{ 3 r^{3} } 
\lr{
   2 \lr{ -\sqrt{\frac{3}{8\pi}} }^2 \Real \lr{ 
      (p_x - j p_y) \sin\theta e^{j\phi}
   }
   +
   \lr{ \sqrt{\frac{3}{4\pi}} }^2 p_z \cos\theta
}
=
\frac{2}{4 \pi \epsilon_0 r^3} 
\lr{
   p_x \sin\theta \cos\phi + p_y \sin\theta \sin\phi + p_z \cos\theta
}
= 
\frac{1}{4 \pi \epsilon_0 r^3} 2 \Bp \cdot \rcap.
\end{dmath}

Note that 

\begin{dmath}\label{eqn:dipoleFromSphericalMoments:120}
\partial_\theta Y_{11} = -\sqrt{\frac{3}{8\pi}} \cos\theta e^{j \phi},
\end{dmath}

and

\begin{dmath}\label{eqn:dipoleFromSphericalMoments:140}
\partial_\theta Y_{1,-1} = \sqrt{\frac{3}{8\pi}} \cos\theta e^{-j \phi},
\end{dmath}

so

\begin{dmath}\label{eqn:dipoleFromSphericalMoments:160}
E_\theta 
=
-\inv{\epsilon_0} \frac{1}{ 3 r^{3} } 
\lr{
   2 \lr{ -\sqrt{\frac{3}{8\pi}} }^2 \Real \lr{ 
      (p_x - j p_y) \cos\theta e^{j\phi}
   }
   -
   \lr{ \sqrt{\frac{3}{4\pi}} }^2 p_z \sin\theta
}
=
-\frac{1}{4 \pi \epsilon_0 r^3} 
\lr{
   p_x \cos\theta \cos\phi + p_y \cos\theta \sin\phi - p_z \sin\theta
}
=
-\frac{1}{4 \pi \epsilon_0 r^3} \Bp \cdot \thetacap.
\end{dmath}

For the \(\phicap\) component, the \( m = 0 \) term is killed.  This leaves

\begin{dmath}\label{eqn:dipoleFromSphericalMoments:180}
E_\phi
=
-\frac{1}{\epsilon_0} \frac{1}{ 3 r^{3} \sin\theta } 
\lr{
j q_{11} Y_{11} - j q_{1,-1} Y_{1,-1}
}
=
-\frac{1}{3 \epsilon_0 r^{3} \sin\theta } 
\lr{
j q_{11} Y_{11} - j (-1)^{2m} q_{11}^\conj Y_{11}^\conj
}
=
\frac{2}{\epsilon_0} \frac{1}{ 3 r^{3} \sin\theta } 
\Imag q_{11} Y_{11}
=
\frac{2}{3 \epsilon_0 r^{3} \sin\theta } 
\Imag \lr{
   \lr{ -\sqrt{\frac{3}{8\pi}} }^2 (p_x - j p_y) \sin\theta e^{j \phi}
}
=
\frac{1}{ 4 \pi \epsilon_0 r^{3} } 
\Imag \lr{
   (p_x - j p_y) e^{j \phi}
}
=
\frac{1}{ 4 \pi \epsilon_0 r^{3} } 
\lr{
   p_x \sin\phi - p_y \cos\phi
}
=
-\frac{\Bp \cdot \phicap}{ 4 \pi \epsilon_0 r^3}.
\end{dmath}

That is
%\begin{dmath}\label{eqn:dipoleFromSphericalMoments:200}
\boxedEquation{eqn:dipoleFromSphericalMoments:200}{
\begin{aligned}
E_r &= 
\frac{2}{4 \pi \epsilon_0 r^3} 
\Bp \cdot \rcap \\
E_\theta &= -
\frac{1}{4 \pi \epsilon_0 r^3} 
\Bp \cdot \phicap \\
E_\phi &= -
\frac{1}{4 \pi \epsilon_0 r^3} 
\Bp \cdot \phicap.
\end{aligned}
}
%\end{dmath}

These are consistent with equations (4.12) from the text for when \( \Bp \) is aligned with the z-axis.

Observe that we can sum each of the projections of \( \BE \) to construct the total electric field due to this \( l = 1 \) term of the multipole moment sum

\begin{dmath}\label{eqn:dipoleFromSphericalMoments:n}
\BE 
=
\frac{1}{4 \pi \epsilon_0 r^3} 
\lr{
2 \rcap (\Bp \cdot \rcap) 
-
\phicap ( \Bp \cdot \phicap) 
-
\thetacap ( \Bp \cdot \thetacap) 
}
=
\frac{1}{4 \pi \epsilon_0 r^3} 
\lr{
3 \rcap (\Bp \cdot \rcap) 
-
\Bp
},
\end{dmath}

which recovers the expected dipole moment approximation.

%}
\EndArticle
