%
% Copyright � 2012 Peeter Joot.  All Rights Reserved.
% Licenced as described in the file LICENSE under the root directory of this GIT repository.
%
\newcommand{\authorname}{Peeter Joot}
\newcommand{\email}{peeterjoot@protonmail.com}
\newcommand{\basename}{FIXMEbasenameUndefined}
\newcommand{\dirname}{notes/FIXMEdirnameUndefined/}

\renewcommand{\basename}{qmTwoR1}
\renewcommand{\dirname}{notes/phy456/}
%\newcommand{\dateintitle}{}
%\newcommand{\keywords}{}
%\blogpage{http://sites.google.com/site/peeterjoot/math2011/qmTwoR1.pdf}
%\date{Sept 16, 2011}

\newcommand{\authorname}{Peeter Joot}
\newcommand{\onlineurl}{http://sites.google.com/site/peeterjoot2/math2013/\basename.pdf}
\newcommand{\sourcepath}{\dirname\basename.tex}
\newcommand{\generatetitle}[1]{\chapter{#1}}

\newcommand{\vcsinfo}{%
\section*{}
\noindent{\color{DarkOliveGreen}{\rule{\linewidth}{0.1mm}}}
\paragraph{Document version}
%\paragraph{\color{Maroon}{Document version}}
{
\small
\begin{itemize}
\item Available online at:\\ 
\href{\onlineurl}{\onlineurl}
\item Git Repository: \input{./.revinfo/gitRepo.tex}
\item Source: \sourcepath
\item last commit: \input{./.revinfo/gitCommitString.tex}
\item commit date: \input{./.revinfo/gitCommitDate.tex}
\end{itemize}
}
}

%\PassOptionsToPackage{dvipsnames,svgnames}{xcolor}
\PassOptionsToPackage{square,numbers}{natbib}
\documentclass{scrreprt}

\usepackage[left=2cm,right=2cm]{geometry}
\usepackage[svgnames]{xcolor}
\usepackage{peeters_layout}

\usepackage{natbib}

\usepackage[
colorlinks=true,
bookmarks=false,
pdfauthor={\authorname, \email},
backref 
]{hyperref}

% http://tex.stackexchange.com/questions/75773/how-to-reference-problems-by-the-text-label-in-an-exercise-envioronment
\usepackage[english]{cleveref}
\crefname{Exercise}{exercise}{exercises}
\Crefname{Exercise}{Exercise}{Exercises}

\RequirePackage{titlesec}
\RequirePackage{ifthen}

% http://stackoverflow.com/questions/4932910/date-in-the-tabular-environment
\makeatletter
\let\insertdate\@date
\makeatother

\titleformat{\chapter}[display]
{\bfseries\Large}
{\color{DarkSlateGrey}\filleft \authorname
\ifthenelse{\isundefined{\studentnumber}}{}{\\ \studentnumber}
\ifthenelse{\isundefined{\email}}{}{\\ \email}
\ifthenelse{\isundefined{\dateintitle}}{}{\\ \insertdate}
%\ifthenelse{\isundefined{\coursename}}{}{\\ \coursename} % put in title instead.
}
{4ex}
{\color{DarkOliveGreen}{\titlerule}\color{Maroon}
\vspace{2ex}%
\filright}
[\vspace{2ex}%
\color{DarkOliveGreen}\titlerule
]

\newcommand{\beginArtWithToc}[0]{\begin{document}\tableofcontents}
\newcommand{\beginArtNoToc}[0]{\begin{document}}
\newcommand{\EndNoBibArticle}[0]{\end{document}}
\newcommand{\EndArticle}[0]{\bibliography{Bibliography}\bibliographystyle{plainnat}\end{document}}

% 
%\newcommand{\citep}[1]{\cite{#1}}

\colorSectionsForArticle



\beginArtNoToc

\generatetitle{PHY450H1F: Quantum Mechanics II.  Recitation 1}
%\chapter{PHY450H1F: Quantum Mechanics II.  Recitation 1 (Taught by Mr. Federico Duque Gomez).  XXX}
\label{chap:qmTwoR1}

\section{Review}

Suggestions
\begin{itemize}
\item Moving back and forth between Dirac notation and wavefunctions.
\item Examples of composite systems.
\item Connection between parity and symmetric potentials.
\end{itemize}

\subsection{Dirac notation}

<did not take notes for this>.

\subsection{Composite systems}

An example that we are familiar with is the SG apparatus

FIXME: FIG1

where due to the Zeeman term \(\sim \sigma \cdot \BB\) in the Hamiltonian is described by wavefunctions of the form

\begin{equation}\label{eqn:qmTwoR1:10}
\psi(\Br) \ket{\phi_s}
\end{equation}

where there is both position and spin dependence.

In a more general context with a Hilbert space

\begin{equation}\label{eqn:qmTwoR1:30}
H = H_1 \otimes H_2
\end{equation}

We can sometimes split the Hamiltonian for the system into independent portions

\begin{equation}\label{eqn:qmTwoR1:50}
\mathcal{H} = \mathcal{H}_1 + \mathcal{H}_2
\end{equation}

Really this is

\begin{equation}\label{eqn:qmTwoR1:70}
\mathcal{H} =
\mathcal{H}_1 \otimes \mathcal{I}_2
+\mathcal{I}_2 \otimes \mathcal{H}_1
\end{equation}

Example.  Square well in two dimensions

FIXME: FIG2

<lots of confused discussion about validity of the example, that the Hamiltonian can not be split because it depends on a product>

Another example of this is the separation of radial from angular dependence in the hydrogen atom.

Also, the split of the Hamiltonian for the hydrogen atom into a portion for the center of mass, and another Hamiltonian for the states that move with the center of mass (oblivious to that motion).

\subsection{Parity and symmetric potentials}

That is a curious question.  I wonder what the context for that question was?  The parity operator, as disussed in \citep{wiki:parity}, inverts the sign of sign, so a potential that depends only on the absolute square would conserve parity.  Formally, this operator has the property

\begin{equation}\label{eqn:qmTwoR1:90}
\Pi \ket{\Br} = \ket{-\Br}
\end{equation}

So that
\begin{equation}\label{eqn:qmTwoR1:110}
\bra{\Br} \Pi \ket{\Psi} = \Psi(-\Br)
\end{equation}

Denoting \(+\) as even and \(-\) as odd, we can split a state into even and odd parts

\begin{equation}\label{eqn:qmTwoR1:130}
\ket{\psi} =
\ket{\psi_{+}}
+\ket{\psi_{-}}
\end{equation}

An even operator

\begin{equation}\label{eqn:qmTwoR1:150}
B_{+} : \antisymmetric{B_{+}}{\Pi} = 0
\end{equation}

And an odd operator

\begin{equation}\label{eqn:qmTwoR1:170}
B_{-} : \antisymmetric{B_{-}}{\Pi} = 0
\end{equation}

Also have the selection rule

\begin{equation}\label{eqn:qmTwoR1:190}
\bra{\phi} B_{+} \ket{\psi} = 0
\end{equation}

when \(\ket{\phi}\) and \(\ket{\psi} = 0\) have opposite parity.

\paragraph{Stopped taking notes}.  not going anywhere useful.

%\citep{desai2009quantum}

\EndArticle
