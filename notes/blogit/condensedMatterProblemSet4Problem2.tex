%
% Copyright � 2013 Peeter Joot.  All Rights Reserved.
% Licenced as described in the file LICENSE under the root directory of this GIT repository.
%
\makeproblem{one-dimensional chain with two spring constants}{condensedMatter:problemSet4:2}{ 

Consider the one-dimensional chain of \cref{fig:two_springs:two_springsFig1} in which all masses have the same value, $m$, but the spring constant alternates between $k$ and $k'$, as shown below.  The equilibrium length of the springs is the same, and equal to $a/2$, where $a$ is the lattice spacing.

\imageFigure{two_springsFig1}{Two springs chain}{fig:two_springs:two_springsFig1}{0.1}
%
%\includegraphics[width=12cm]{./figures/two_springs.pdf}

\makesubproblem{}{condensedMatter:problemSet4:2a}
Find the coupled equations of motion for
$u_{n,1}$ and $u_{n,2}$, and then make the ansatz
\begin{eqnarray*}
%u_{n,\alpha} = u_{\alpha}(q) {\rm e}^{i(qx_n - \omega t)},
u_{n,\alpha} = u_{\alpha}(q) e^{i(qx_n - \omega t)},
\end{eqnarray*}
to obtain coupled equations for $u_1(q)$ and $u_2(q)$.

\makesubproblem{}{condensedMatter:problemSet4:2b}
Solve these coupled equations, putting $k = k_\nought + \delta$
and $k' = k_\nought - \delta$, to show that the dispersion relation is:
\begin{eqnarray*}
\omega^2 = \frac{2k_\nought}{m}\left( 1 \pm \sqrt{ \cos^2\frac{qa}{2}
               + \frac{\delta^2}{k_\nought^2} \sin^2 \frac{qa}{2}} \,\, \right).
\end{eqnarray*}

\makesubproblem{}{condensedMatter:problemSet4:2c}
At $q = 0$, plug the two solutions for $\omega$ back into the
coupled equations for $u_1(q)$ and  $u_2(q)$ to find the relative
motion of masses 1 and 2 in each primitive unit cell.  Explain how 
these relative motions of the two masses give rise to the two 
frequencies at $q=0$. 

\makesubproblem{}{condensedMatter:problemSet4:2d}
Plot $\omega(q)$ vs.\ $q$ in the range $q=0$ to $q=2\pi/a$,
for $\delta/k_\nought = 0.9,0.1$ and 0.  % fixed this
Indicate the boundary of the first
Brillouin zone.  (I did this by programming my equation for
$\omega(q)$ into a plotting package; you can also do it by hand.)

\makesubproblem{}{condensedMatter:problemSet4:2e}
For $\delta/k_\nought = 0.9$ explain why the two branches of the
phonon spectrum are so far apart.  What happens in the limit where
$\delta = k_\nought$?  Explain the dispersion relation (or rather, lack
thereof) of the acoustic and optical branches in this limit.

\makesubproblem{}{condensedMatter:problemSet4:2f}
For $\delta = 0$, explain the form of your plot (hint: when
  $\delta=0$ the two springs are identical, so think about the
  primitive unit cell and first Brillouin zone size in this case; 
  also, you may recall the calculation of the structure factor 
  in which we treated a bcc lattice as simple cubic with a two-atom 
  basis).
} % makeproblem

\makeanswer{condensedMatter:problemSet4:2}{ 
\makeSubAnswer{}{condensedMatter:problemSet4:2a}

In order to avoid the mental trauma of trying to figure out all the signs for the spring constant potential coefficients, we can describe the system by the Lagrangian

\begin{dmath}\label{eqn:condensedMatterProblemSet4Problem2:20}
\LL = \sum_{n, \alpha} \frac{m}{2} \dot{u}_{n, \alpha}^2 
- \frac{k'}{2}\lr{ u_{n, 1} - u_{n - 1, 2}}^2 
- \frac{k}{2}\lr{ u_{n, 2} - u_{n, 1}}^2 
- \frac{k'}{2}\lr{ u_{n + 1, 1} - u_{n, 2}}^2 
- \cdots
\end{dmath}

The force equations then follow directly from the Euler-Lagrange equations

\begin{dmath}\label{eqn:condensedMatterProblemSet4Problem2:40}
0 = \ddt{} \PD{\dot{u}_{n, \alpha}}{\LL} 
- \PD{u_{n, \alpha}}{\LL}.
\end{dmath}

That is

\begin{subequations}
\begin{dmath}\label{eqn:condensedMatterProblemSet4Problem2:60}
0 = m \ddot{u}_{n, 1} 
+ 
( k + k')
u_{n, 1}
- k
u_{n, 2}
- k'
u_{n - 1, 2}
\end{dmath}
\begin{dmath}\label{eqn:condensedMatterProblemSet4Problem2:80}
0 = m \ddot{u}_{n, 2} 
+ 
( k + k')
u_{n, 2}
- k
u_{n, 1}
- k'
u_{n + 1, 1}
\end{dmath}
\end{subequations}

Using the trial solution

\begin{dmath}\label{eqn:condensedMatterProblemSet4Problem2:100}
u_{n,\alpha} = u_{\alpha}(q) e^{i(qx_{n, \alpha} - \omega t)}, 
\end{dmath}

\begin{subequations}
\begin{dmath}\label{eqn:condensedMatterProblemSet4Problem2:120}
0 = -m \omega^2 u_1 e^{i( q a n - \omega t)} 
+ 
( k + k')
u_1 e^{i( q a n - \omega t)}
- k
u_2 e^{i( q a (n + 1/2) - \omega t)}
- k'
u_2 e^{i( q a (n - 1/2) - \omega t)}
\end{dmath}
\begin{dmath}\label{eqn:condensedMatterProblemSet4Problem2:140}
0 = -m \omega^2 u_2 e^{i( q a (n + 1/2) - \omega t)} 
+ 
( k + k')
u_2 e^{i( q a (n + 1/2) - \omega t)}
- k
u_1 e^{i( q a n - \omega t)}
- k'
u_1 e^{i( q a (n + 1) - \omega t)}
\end{dmath}
\end{subequations}

Cancelling terms, substitution for $k$ and $k'$, and a bit of rearranging yields

\begin{subequations}
\begin{dmath}\label{eqn:condensedMatterProblemSet4Problem2:160}
0 = -m \omega^2 u_1 
+ 
2 k_\nought
u_1 
- (k_\nought + \delta)
u_2 e^{i q a /2 }
- (k_\nought - \delta)
u_2 e^{-i q a /2 }
\end{dmath}
\begin{dmath}\label{eqn:condensedMatterProblemSet4Problem2:180}
0 = -m \omega^2 u_2 
+ 
2 k_\nought
u_2 
- (k_\nought + \delta)
u_1 
e^{-i q a /2 }
- (k_\nought - \delta)
u_1 e^{i q a/2 }.
\end{dmath}
\end{subequations}

This has a solution when the determinant is zero
\begin{dmath}\label{eqn:condensedMatterProblemSet4Problem2:200}
0 = 
\begin{vmatrix}
-m \omega^2 +
2 k_\nought &
- 2 k_\nought \cos \lr{ q a /2 }
- 2 i \delta \sin \lr{ q a /2 }
%- (k_\nought + \delta) e^{i q a /2 }
%- (k_\nought - \delta) e^{-i q a /2 } 
\\
- 2 k_\nought \cos \lr{ q a /2 }
+ 2 i \delta \sin \lr{ q a /2 }
%- (k_\nought + \delta) e^{-i q a /2 }
%- (k_\nought - \delta) e^{i q a/2 }
& 
-m \omega^2 + 2 k_\nought
\end{vmatrix}
=
\lr{ -m \omega^2 + 2 k_\nought }^2
+ 4 \sqrt{
k_\nought^2 \cos^2 \lr{ q a /2 }
+
\delta^2 \sin^2 \lr{ q a /2 }
},
\end{dmath}

or
\begin{dmath}\label{eqn:condensedMatterProblemSet4Problem2:n}
\omega^2 = 
\frac{2 k_\nought }{m}
\pm \frac{2 k_\nought}{m} \sqrt{
\cos^2 \lr{ q a /2 }
+
\frac{\delta^2}{k_\nought} \sin^2 \lr{ q a /2 }
},
\end{dmath}

as desired.

TODO.
\makeSubAnswer{}{condensedMatter:problemSet4:2b}

TODO.
\makeSubAnswer{}{condensedMatter:problemSet4:2c}

TODO.
\makeSubAnswer{}{condensedMatter:problemSet4:2d}

%\cref{fig:qmSolidsPs4P2d:qmSolidsPs4P2dFig1}.
\imageFigure{qmSolidsPs4P2dFig1}{CAPTION}{fig:qmSolidsPs4P2d:qmSolidsPs4P2dFig1}{0.3}

TODO.
\makeSubAnswer{}{condensedMatter:problemSet4:2e}

TODO.
\makeSubAnswer{}{condensedMatter:problemSet4:2f}

TODO.
}

