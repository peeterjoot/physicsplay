%
% Copyright � 2013 Peeter Joot.  All Rights Reserved.
% Licenced as described in the file LICENSE under the root directory of this GIT repository.
%
\makeproblem{one-dimensional chain with two spring constants}{condensedMatter:problemSet4:2}{ 

Consider the one-dimensional chain of \cref{fig:two_springs:two_springsFig1} in which all masses have the same value, $m$, but the spring constant alternates between $k$ and $k'$, as shown below.  The equilibrium length of the springs is the same, and equal to $a/2$, where $a$ is the lattice spacing.

\imageFigure{two_springsFig1}{Two springs chain}{fig:two_springs:two_springsFig1}{0.1}
%
%\includegraphics[width=12cm]{./figures/two_springs.pdf}

\makesubproblem{}{condensedMatter:problemSet4:2a}
Find the coupled equations of motion for
$u_{n,1}$ and $u_{n,2}$, and then make the ansatz
\begin{eqnarray*}
u_{n,\alpha} = u_{\alpha}(q) {\rm e}^{i(qx_n - \omega t)},
\end{eqnarray*}
to obtain coupled equations for $u_1(q)$ and $u_2(q)$.

\makesubproblem{}{condensedMatter:problemSet4:2b}
Solve these coupled equations, putting $k = k_\circ + \delta$
and $k' = k_\circ - \delta$, to show that the dispersion relation is:
\begin{eqnarray*}
\omega^2 = \frac{2k_\circ}{m}\left( 1 \pm \sqrt{ \cos^2\frac{qa}{2}
               + \frac{\delta^2}{k_\circ^2} \sin^2 \frac{qa}{2}} \,\, \right).
\end{eqnarray*}

\makesubproblem{}{condensedMatter:problemSet4:2c}
At $q = 0$, plug the two solutions for $\omega$ back into the
coupled equations for $u_1(q)$ and  $u_2(q)$ to find the relative
motion of masses 1 and 2 in each primitive unit cell.  Explain how 
these relative motions of the two masses give rise to the two 
frequencies at $q=0$. 

\makesubproblem{}{condensedMatter:problemSet4:2d}
Plot $\omega(q)$ vs.\ $q$ in the range $q=0$ to $q=2\pi/a$,
for $\delta/k_0 = 0.9,0.1$ and 0.  % fixed this
Indicate the boundary of the first
Brillouin zone.  (I did this by programming my equation for
$\omega(q)$ into a plotting package; you can also do it by hand.)

\makesubproblem{}{condensedMatter:problemSet4:2e}
For $\delta/k_\circ = 0.9$ explain why the two branches of the
phonon spectrum are so far apart.  What happens in the limit where
$\delta = k_\circ$?  Explain the dispersion relation (or rather, lack
thereof) of the acoustic and optical branches in this limit.

\makesubproblem{}{condensedMatter:problemSet4:2f}
For $\delta = 0$, explain the form of your plot (hint: when
  $\delta=0$ the two springs are identical, so think about the
  primitive unit cell and first Brillouin zone size in this case; 
  also, you may recall the calculation of the structure factor 
  in which we treated a bcc lattice as simple cubic with a two-atom 
  basis).
} % makeproblem

\makeanswer{condensedMatter:problemSet4:2}{ 
\makeSubAnswer{}{condensedMatter:problemSet4:2a}
\makeSubAnswer{}{condensedMatter:problemSet4:2b}
\makeSubAnswer{}{condensedMatter:problemSet4:2c}
\makeSubAnswer{}{condensedMatter:problemSet4:2d}
\makeSubAnswer{}{condensedMatter:problemSet4:2e}
\makeSubAnswer{}{condensedMatter:problemSet4:2f}

TODO.
}

