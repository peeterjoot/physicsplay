%
% Copyright � 2014 Peeter Joot.  All Rights Reserved.
% Licenced as described in the file LICENSE under the root directory of this GIT repository.
%
\makeproblem{Model order reduction methods}{multiphysics:problemSet3b:1}{ 

In this problem you will apply different model order reduction methods to
the heat conducting bar. Use Backward Euler for time integration.

\makesubproblem{}{multiphysics:problemSet3b:1a}
In this problem we shall introduce a heat source \( u(t) \) and its
spatial distribution in the form of \( h(x) \).

\begin{equation}\label{eqn:multiphysicsProblemSet3b:20}
\PD{t}{T(x,t)}
=
\PDSq{x}{T(x,t)}
- \alpha T(x,t) + h(x)u(t) \qquad x \in [0,1]
\end{equation}

where \( T(x,t) \) is the temperature at location \( x \) at time \( t \). The boundary condition is that no heat flows away from the two ends of the bar, i.e.

\begin{equation}\label{eqn:multiphysicsProblemSet3b:40}
\evalbar{\PD{x}{T}}{x = 0}
=
\evalbar{\PD{x}{T}}{x = 1}
= 0.
\end{equation}

The term with \( \alpha \) models the heat dissipation from the bar to the surrounding
environment. Let \( \alpha = 0.01 \). You can assume zero initial conditions. Explain
how this heat equation can be formulated into an equivalent dynamical system in the form

\begin{equation}\label{eqn:multiphysicsProblemSet3b:60}
\BG \Bx(t) + \BC \dot{\Bx}(t) = \BB \Bu(t)
\end{equation}

Explain the choice of your matrices \( \BG,\BC,\BB \). What do the states \( \Bx(t) \) of
your dynamical system correspond to, physically? We define the output \( \By(t) \)
as \( \By(t) = \BL^\T \Bx(t) \), where \( {}^\T \) denotes the matrix transpose.

\makesubproblem{}{multiphysics:problemSet3b:1b}
We are interested in the following scenarios:

\begin{itemize}
   \item [\textbf{Case 1}]
\par
Input: heat flow at the left end of the bar
\par
Output: temperature at the right end
   \item [\textbf{Case 2}]
\par
Input: heat flow at the left end of the bar
\par
Output: average temperature along the bar
   \item [\textbf{Case 3}]
\par
Input: uniform heating
\par
Output: temperature at the right end
   \item [\textbf{Case 4}]
\par
Input: uniform heating
\par
Output: average temperature along the bar
\end{itemize}

Explain how you will pick the matrices \( \BB, \BL \) in each case.
\textbf{In all remaining questions, consider only case 1.}

\makesubproblem{}{multiphysics:problemSet3b:1c}

Write a Matlab routine \matlabFunc{PlotFreqResp}(\(\omega\),G,C,B,L) which takes
in \( \omega,G,C,B,L \) as input and plots the system frequency response. Here \( \omega \) 
is a vector of frequencies in rad/s. For \( N = 500 \) (500 nodes) plot the real
and imaginary part of the frequency response for case 1.
\paragraph{Hints:} To plot the frequency response, use the command \matlabFunc{semilogx}(x,y).
You can generate the frequency vector with the command: \( \omega \) = \matlabFunc{logspace}(-8,4,500).

\makesubproblem{}{multiphysics:problemSet3b:1d}

Reduce the dynamical system for \( N = 500 \) using the modal
truncation method. Retain the modes that are associated with the ``slowest'' eigenvalues. Note that you will have to convert your system from the
modified nodal analysis representation to the state-space representation. Use
\( q = 1, 2, 4, 10, 50 \) (\( q = \text{number of states in the reduced system} \)). Plot the
frequency response of the original system and frequency response of the reduced system in the same plot. Select a ``reasonably small'' order \( q \) for the
reduced model, that ensures a reasonable approximation of the original system. Clearly explain how did you pick that \( q \), which factors did you look at.


\makesubproblem{}{multiphysics:problemSet3b:1e}

Using your time domain solver, compute the output of the
original system (\( N=500 \)) and of the reduced model (with the order \( q \) you
selected) for each of the following two inputs:

\begin{enumerate}
\item Take the first input \( u(t) = u_1(t) \) as

\begin{equation}\label{eqn:multiphysicsProblemSet3b:80}
u_1(t) =
\left\{
\begin{array}{l l}
1 & \quad \mbox{if \( t \ge 0 \)} \\
0 & \quad \mbox{if \( t < 0 \)} 
\end{array}
\right.
\end{equation}

For \( u_1(t) \) pick \( t_{\textrm{stop}} \) such that the system reaches steady state. 

\item Take the second input \( u(t) = u_2(t) \) as

\begin{equation}\label{eqn:multiphysicsProblemSet3b:100}
u_2(t) = \sin(0.01 t) \quad \mbox{for \( t_{\textrm{stop}} = 10000 \)}
\end{equation}
\end{enumerate}

Be sure to pick an appropriate time step and explain your choice. How much
speed-up did you get for your reduced models for time domain simulations?
Explain your results.

\makesubproblem{}{multiphysics:problemSet3b:1f}
Repeat 
part \ref{multiphysics:problemSet3b:1d}
using the PRIMA model order reduction algorithm (see provided routine \textbf{prima.m}). In order to maximize the efficiency of
the reduced model, after you generate it with prima, bring it to state-space
form (by multiplying by \(\BC^{-1}\) ). Then, make the \( \BA \) matrix of the state space
model diagonal. In this way, the reduced model becomes sparse rather than
full, and can be solved more quickly.

\makesubproblem{}{multiphysics:problemSet3b:1g}

Repeat 
part \ref{multiphysics:problemSet3b:1e}
using the PRIMA model order reduction algorithm.

\makesubproblem{}{multiphysics:problemSet3b:1h}

Compare the results obtained with the two methods. A table
with a few columns will suffice.

} % makeproblem

\makeanswer{multiphysics:problemSet3b:1}{ 
\makeSubAnswer{}{multiphysics:problemSet3b:1a}

A spatial discretization with width \( \Delta x = 1 / N \)
is illustrated in \cref{fig:ps3bDiscretization:ps3bDiscretizationFig1}

\imageFigure{../../figures/ece1254/ps3bDiscretizationFig1}{Discretization interval}{fig:ps3bDiscretization:ps3bDiscretizationFig1}{0.2}

Let

\begin{equation}\label{eqn:multiphysicsProblemSet3b:120}
\begin{aligned}
x^k &= (k -1)\Delta x , \quad k \in \{ 1, \cdots, N + 1 \} \\
\tau^{k}(t) &= T( x^k, t ) \\
h^k &= h( x^k ) \\
i^k(t) &= \frac{ \tau^k - \tau^{k-1}}{\Delta x}
\end{aligned}
\end{equation}

The temperature has an electric circuit equivalent to voltage, and the heat flow \( i^k \) has the structure of an electric circuit current \( I = \Delta V/R \), through a resistance \( \Delta x \).

In the interior the discretization of the heat equation takes the form of a KCL equation

\begin{equation}\label{eqn:multiphysicsProblemSet3b:640}
0 =
\lr{\Delta x}^2 \ddt{\tau^k}
-\lr{ i^k - i^{k-1} }
+ \lr{\Delta x}^2 \alpha \tau^k - \lr{\Delta x}^2 h^k u(t).
\end{equation}

This is illustrated in \cref{fig:ps3bInteriorNode:ps3bInteriorNodeFig1}.

\imageFigure{../../figures/ece1254/ps3bInteriorNodeFig1}{Equivalent circuit for heat equation}{fig:ps3bInteriorNode:ps3bInteriorNodeFig1}{0.3}

The initial node where \( i^1 = 0 \), or \( \tau^2 = \tau^1 \), can be modelled with a zero voltage source, replacing the resistance.  This is illustrated in \cref{fig:ps3bInitialNode:ps3bInitialNodeFig2}.

\imageFigure{../../figures/ece1254/ps3bInitialNodeFig2}{Initial node}{fig:ps3bInitialNode:ps3bInitialNodeFig2}{0.3}

Similarly, the terminal node boundary value constraint \( i^{N+1} = 0 \), or \( \tau^N = \tau^{N+1} \) can also be modelled by a zero voltage source, as illustrated in \cref{fig:ps3bTerminalNode:ps3bTerminalNodeFig3}.

\imageFigure{../../figures/ece1254/ps3bTerminalNodeFig3}{Terminal node}{fig:ps3bTerminalNode:ps3bTerminalNodeFig3}{0.3}

The pattern of \( \BG \) is illustrated well by the \( N = 6 \) case

\begin{subequations}
\begin{equation}\label{eqn:multiphysicsProblemSet3b:660}
\BG = 
\begin{bmatrix}
    \alpha \lr{ \Delta x}^2 &       0 &       0 &       0 &       0 &       0 &       0 & -1 &       0 \\
         0 &  N + \alpha \lr{ \Delta x}^2 & -N &       0 &       0 &       0 &       0 &  1 &       0 \\
         0 & -N & 2 N + \alpha \lr{ \Delta x}^2 & -N &       0 &       0 &       0 &       0 &       0 \\
         0 &       0 & -N & 2 N + \alpha \lr{ \Delta x}^2 & -N &       0 &       0 &       0 &       0 \\
         0 &       0 &       0 & -N & 2 N + \alpha \lr{ \Delta x}^2 & -N &       0 &       0 &       0 \\
         0 &       0 &       0 &       0 & -N &  N + \alpha \lr{ \Delta x}^2 &       0 &       0 & -1 \\
         0 &       0 &       0 &       0 &       0 &       0 &  \alpha \lr{ \Delta x}^2 &       0 &  1 \\
    1 & -1 &       0 &       0 &       0 &       0 &       0 &       0 &       0 \\
         0 &       0 &       0 &       0 &       0 &  1 & -1 &       0 &       0 \\
\end{bmatrix}
\end{equation}
\begin{equation}\label{eqn:multiphysicsProblemSet3b:680}
\Bx = 
\begin{bmatrix}
\tau^1 \\
\tau^2 \\
\vdots \\
\tau^{N+1} \\
i^1 \\
i^N
\end{bmatrix}
\end{equation}
\begin{equation}\label{eqn:multiphysicsProblemSet3b:700}
\BC = \lr{\Delta x}^2 
\begin{bmatrix}
\BI_{N+1} & \Bzero_2 \\
\Bzero_2 & \Bzero_2 
\end{bmatrix}
\end{equation}
\begin{equation}\label{eqn:multiphysicsProblemSet3b:720}
\BB \Bu
=
\lr{\Delta x}^2 u(t)
\begin{bmatrix}
h^1 \\
h^2 \\
\vdots \\
h^{N+1} \\
0 \\
0 \\
\end{bmatrix}
\end{equation}
\end{subequations}

\makeSubAnswer{}{multiphysics:problemSet3b:1b}

\begin{itemize}
   \item [\textbf{Case 1}]
\begin{equation}\label{eqn:multiphysicsProblemSet3b:380}
\begin{aligned}
\BB &= h(0)
%\begin{bmatrix}
%1 & 0  & \hdots & 0 \\
%0 & 0  & \hdots & 0 \\
%\vdots  & 0 & \ddots & 0 \\
%0 & 0  & \hdots & 0 \\
%\end{bmatrix} \\
\begin{bmatrix}
1 & 0 \\
0        & \Bzero_{N+2}
\end{bmatrix} \\
\BL^T &= 
\kbordermatrix{
 & 1 & 2 & \cdots & N & N+1 & & \\
 & 0 & 0 & \cdots & 0 & 1 & 0 & 0
}
\end{aligned}
\end{equation}
   \item [\textbf{Case 2}]
\begin{equation}\label{eqn:multiphysicsProblemSet3b:420}
\begin{aligned}
\BB &= h(0)
%\begin{bmatrix}
%1 & 0  & \hdots & 0 \\
%0 & 0  & \hdots & 0 \\
%\vdots  & 0 & \ddots & 0 \\
%0 & 0  & \hdots & 0 \\
%\end{bmatrix} \\
\begin{bmatrix}
1 & 0 \\
0        & \Bzero_{N+2}
\end{bmatrix} \\
\BL^T &= 
\inv{N+1}
%\begin{bmatrix}
%1 & 1 & \cdots & 1 & 1 & 0 & 0
%\end{bmatrix}
\kbordermatrix{
 & 1 & 2 & \cdots & N & N+1 & & \\
 & 1 & 1 & \cdots & 1 & 1 & 0 & 0
}
\end{aligned}
\end{equation}
   \item [\textbf{Case 3}]
\begin{equation}\label{eqn:multiphysicsProblemSet3b:440}
\begin{aligned}
\BB &= h(1) 
\begin{bmatrix}
\BI_{N+1} & \Bzero_2 \\
\Bzero_2 & \Bzero_2
\end{bmatrix} \\
\BL^T &= 
\kbordermatrix{
  & 1 & 2 & \cdots & N & N+1 &   & \\
  & 0 & 0 & \cdots & 0 & 1   & 0 & 0
}
\end{aligned}
\end{equation}
   \item [\textbf{Case 4}]
\begin{equation}\label{eqn:multiphysicsProblemSet3b:460}
\begin{aligned}
\BB &= h(1) 
\begin{bmatrix}
\BI_{N+1} & \Bzero_2 \\
\Bzero_2 & \Bzero_2
\end{bmatrix} 
\\
\BL^T &= 
\inv{N+1}
%\begin{bmatrix}
%1 & 1 & \cdots & 1 & 1 & 0 & 0
%\end{bmatrix}
\kbordermatrix{
 & 1 & 2 & \cdots & N & N+1 & & \\
 & 1 & 1 & \cdots & 1 & 1 & 0 & 0
}
\end{aligned}
\end{equation}
\end{itemize}


\makeSubAnswer{}{multiphysics:problemSet3b:1c}

The MLN equations for the \( N = 500 \) case were generated indirectly using
\matlabFuncPath{generateNetlist}{ps3b:generateNetlist.m}(), to create a netlist file, with the previous implementation of
\matlabFuncPath{NodalAnalysis}{ps2a:NodalAnalysis.m}() doing the grunt work.
A driver \matlabFuncPath{plotFreqPartC}{ps3b:plotFreqPartC.m}() calls these and \matlabFuncPath{PlotFreqResp}{ps3b:PlotFreqResp.m}() to produce the plot of \cref{fig:ps3bFreqResponsePartC:ps3bFreqResponsePartCFig1}.

\imageFigure{../../figures/ece1254/ps3bFreqResponsePartCFig1}{\( N = 500 \) frequency response for case 1.}{fig:ps3bFreqResponsePartC:ps3bFreqResponsePartCFig1}{0.3}

REST: TODO.
\makeSubAnswer{}{multiphysics:problemSet3b:1d}

TODO.
\makeSubAnswer{}{multiphysics:problemSet3b:1e}

TODO.
\makeSubAnswer{}{multiphysics:problemSet3b:1f}

TODO.
\makeSubAnswer{}{multiphysics:problemSet3b:1g}

TODO.
\makeSubAnswer{}{multiphysics:problemSet3b:1h}

TODO.
}
