%
% Copyright � 2013 Peeter Joot.  All Rights Reserved.
% Licenced as described in the file LICENSE under the root directory of this GIT repository.
%
\newcommand{\authorname}{Peeter Joot}
\newcommand{\email}{peeterjoot@protonmail.com}
\newcommand{\basename}{FIXMEbasenameUndefined}
\newcommand{\dirname}{notes/FIXMEdirnameUndefined/}

\renewcommand{\basename}{tmp}
\renewcommand{\dirname}{notes/FIXMEwheretodirname/}
%\newcommand{\dateintitle}{}
%\newcommand{\keywords}{}

\newcommand{\authorname}{Peeter Joot}
\newcommand{\onlineurl}{http://sites.google.com/site/peeterjoot2/math2013/\basename.pdf}
\newcommand{\sourcepath}{\dirname\basename.tex}
\newcommand{\generatetitle}[1]{\chapter{#1}}

\newcommand{\vcsinfo}{%
\section*{}
\noindent{\color{DarkOliveGreen}{\rule{\linewidth}{0.1mm}}}
\paragraph{Document version}
%\paragraph{\color{Maroon}{Document version}}
{
\small
\begin{itemize}
\item Available online at:\\ 
\href{\onlineurl}{\onlineurl}
\item Git Repository: \input{./.revinfo/gitRepo.tex}
\item Source: \sourcepath
\item last commit: \input{./.revinfo/gitCommitString.tex}
\item commit date: \input{./.revinfo/gitCommitDate.tex}
\end{itemize}
}
}

%\PassOptionsToPackage{dvipsnames,svgnames}{xcolor}
\PassOptionsToPackage{square,numbers}{natbib}
\documentclass{scrreprt}

\usepackage[left=2cm,right=2cm]{geometry}
\usepackage[svgnames]{xcolor}
\usepackage{peeters_layout}

\usepackage{natbib}

\usepackage[
colorlinks=true,
bookmarks=false,
pdfauthor={\authorname, \email},
backref 
]{hyperref}

% http://tex.stackexchange.com/questions/75773/how-to-reference-problems-by-the-text-label-in-an-exercise-envioronment
\usepackage[english]{cleveref}
\crefname{Exercise}{exercise}{exercises}
\Crefname{Exercise}{Exercise}{Exercises}

\RequirePackage{titlesec}
\RequirePackage{ifthen}

% http://stackoverflow.com/questions/4932910/date-in-the-tabular-environment
\makeatletter
\let\insertdate\@date
\makeatother

\titleformat{\chapter}[display]
{\bfseries\Large}
{\color{DarkSlateGrey}\filleft \authorname
\ifthenelse{\isundefined{\studentnumber}}{}{\\ \studentnumber}
\ifthenelse{\isundefined{\email}}{}{\\ \email}
\ifthenelse{\isundefined{\dateintitle}}{}{\\ \insertdate}
%\ifthenelse{\isundefined{\coursename}}{}{\\ \coursename} % put in title instead.
}
{4ex}
{\color{DarkOliveGreen}{\titlerule}\color{Maroon}
\vspace{2ex}%
\filright}
[\vspace{2ex}%
\color{DarkOliveGreen}\titlerule
]

\newcommand{\beginArtWithToc}[0]{\begin{document}\tableofcontents}
\newcommand{\beginArtNoToc}[0]{\begin{document}}
\newcommand{\EndNoBibArticle}[0]{\end{document}}
\newcommand{\EndArticle}[0]{\bibliography{Bibliography}\bibliographystyle{plainnat}\end{document}}

% 
%\newcommand{\citep}[1]{\cite{#1}}

\colorSectionsForArticle



\beginArtNoToc

\generatetitle{XXX}
%\chapter{XXX}
%\label{chap:tmp}

\section{Stokes theorem for integration over a four volume}

For integration of the curl of a trivector function, an integration over a four volume is required.  I commend you if you have the geometrical intuition required to understand, a-priori, how to express the oriented boundary of such a subspace.

Since I have no such intuition, stepping back and 

We are now prepared to go on to the meat of the issue.  The first order of business is the expansion of the curl and volume element product

\begin{equation}\label{eqn:tmp:n}
\begin{aligned}
( \grad \wedge f ) \cdot d^k x
&=
( \gamma^i \wedge \partial_i f ) \cdot d^k x \\
&=
\gpgradezero{ ( \gamma^i \wedge \partial_i f ) d^k x } \\
\end{aligned}
\end{equation}

The wedge product within the scalar grade selection operator can be expanded in symmetric or antisymmetric sums, but this is a grade dependent operation.  For odd grade blades $A$ (vector, trivector, ...), and vector $a$ we have for the dot and wedge product respectively

\begin{equation}\label{eqn:tmp:n}
\begin{aligned}
a \wedge A = \inv{2} (a A - A a) \\
a \cdot A = \inv{2} (a A + A a)
\end{aligned}
\end{equation}

Similarly for even grade blades we have

\begin{equation}\label{eqn:tmp:n}
\begin{aligned}
a \wedge A = \inv{2} (a A + A a) \\
a \cdot A = \inv{2} (a A - A a)
\end{aligned}
\end{equation}

First treating the odd grade case for $f$ we have

\begin{equation}\label{eqn:tmp:n}
\begin{aligned}
( \grad \wedge f ) \cdot d^k x
&=
\inv{2} \gpgradezero{ \gamma^i \partial_i f d^k x } - \inv{2} \gpgradezero{ \partial_i f \gamma^i d^k x } \\
\end{aligned}
\end{equation}

Employing cyclic scalar reordering within the scalar product for the first term

\begin{equation}\label{eqn:tmp:n}
\begin{aligned}
\gpgradezero{a b c} = \gpgradezero{b c a}
\end{aligned}
\end{equation}

we have

\begin{equation}\label{eqn:tmp:n}
\begin{aligned}
( \grad \wedge f ) \cdot d^k x
&=
\inv{2} \gpgradezero{ \partial_i f (d^k x \gamma^i - \gamma^i d^k x)} \\
&=
\inv{2} \gpgradezero{ \partial_i f (d^k x \cdot \gamma^i - \gamma^i d^k x)} \\
&=
\gpgradezero{ \partial_i f (d^k x \cdot \gamma^i)} \\
\end{aligned}
\end{equation}

The end result is 

\begin{equation}\label{eqn:tmp:n}
\begin{aligned}
( \grad \wedge f ) \cdot d^k x &= \partial_i f \cdot (d^k x \cdot \gamma^i) 
\end{aligned}
\end{equation}

For even grade $f$ (and thus odd grade $d^k x$) it is straightforward to show that (\eqnref{eqn:stokesNoTensor:startingPoint}) also holds.

\subsection{Expanding the volume dot product}

We want to expand the volume integral dot product

\begin{equation}\label{eqn:tmp:n}
\begin{aligned}
d^k x \cdot \gamma^i
\end{aligned}
\end{equation}

Picking $k = 4$ will serve to illustrate the pattern, and the generalization (or degeneralization to lower grades) will be clear.  We have

\begin{equation}\label{eqn:tmp:n}
\begin{aligned}
d^4 x \cdot \gamma^i
&=
( dx_1 \wedge dx_2 \wedge dx_3 \wedge dx_4 ) \cdot \gamma^i \\
&= ( dx_1 \wedge dx_2 \wedge dx_3 ) dx_4 \cdot \gamma^i \\
&-( dx_1 \wedge dx_2 \wedge dx_4 ) dx_3 \cdot \gamma^i \\
&+( dx_1 \wedge dx_3 \wedge dx_4 ) dx_2 \cdot \gamma^i \\
&-( dx_2 \wedge dx_3 \wedge dx_4 ) dx_1 \cdot \gamma^i  \\
\end{aligned}
\end{equation}

This avoids the requirement to do the entire Jacobian expansion of (\eqnref{eqn:stokesNoTensor:jacobian}).  The dot product of the differential displacement $dx_m$ with $\gamma^i$ can now be made explicit without as much mess.

\begin{equation}\label{eqn:tmp:n}
\begin{aligned}
dx_m \cdot \gamma^i 
&=
da_m \PD{a_m}{x^j} \gamma_j \cdot \gamma^i \\
&=
da_m \PD{a_m}{x^i} \\
\end{aligned}
\end{equation}

We now have products of the form

\begin{equation}\label{eqn:tmp:n}
\begin{aligned}
\partial_i f da_m \PD{a_m}{x^i} 
&=
da_m \PD{a_m}{x^i} \PD{x^i}{f} \\
&=
da_m \PD{a_m}{f} \\
\end{aligned}
\end{equation}

Now we see that the differential form of (\eqnref{eqn:stokesNoTensor:startingPoint}) for this $k=4$ example is reduced to

\begin{equation}\label{eqn:tmp:n}
\begin{aligned}
( \grad \wedge f ) \cdot d^4 x 
&= da_4 \PD{a_4}{f} \cdot ( dx_1 \wedge dx_2 \wedge dx_3 ) \\
&- da_3 \PD{a_3}{f} \cdot ( dx_1 \wedge dx_2 \wedge dx_4 ) \\
&+ da_2 \PD{a_2}{f} \cdot ( dx_1 \wedge dx_3 \wedge dx_4 ) \\
&- da_1 \PD{a_1}{f} \cdot ( dx_2 \wedge dx_3 \wedge dx_4 ) \\
\end{aligned}
\end{equation}

While \eqnref{eqn:stokesNoTensor:startingPoint} was a statement of Stokes theorem in this Geometric Algebra formulation, it was really incomplete without this explicit expansion of $(\partial_i f) \cdot (d^k x \cdot \gamma^i)$.  This expansion for the $k=4$ case serves to illustrate that we would write Stokes theorem as

\begin{equation}\label{eqn:tmp:n}
\myBoxed{
\int
( \grad \wedge f ) \cdot d^k x 
=
\inv{(k-1)!} \epsilon^{ r s \cdots t u } \int da_u \PD{a_{u}}{f} \cdot 
(dx_r \wedge dx_s \wedge \cdots \wedge dx_t)
}
\end{equation}
%\left( 
%\PD{a_r}{x} \wedge \PD{a_s}{x} \wedge \cdots \wedge \PD{a_t}{x} \right)
%da_1 da_2 \cdots da_k

Here the indices have the range $\{r, s, \cdots, t, u\} \in \{1, 2, \cdots k\}$.  This with the definitions \eqnref{eqn:stokesNoTensor:oneForm}, and \eqnref{eqn:stokesNoTensor:volumeElement} is really Stokes theorem in its full glory.

Observe that in this Geometric algebra form, the one forms $dx_i = da_i \PDi{a_i}{x}, i \in [1,k]$ are nothing more abstract that plain old vector differential elements.  In the formalism of differential forms, this would be vectors, and $(\grad \wedge f) \cdot d^k x$ would be a $k$ form.  In a context where we are working with vectors, or blades already, the Geometric Algebra statement of the theorem avoids a requirement to translate to the language of forms.

With a statement of the general theorem complete, let us return to our $k=4$ case where we can now integrate over each of the $a_1, a_2, \cdots, a_k$ parameters.  That is

\begin{equation}\label{eqn:tmp:n}
\begin{aligned}
\int ( \grad \wedge f ) \cdot d^4 x 
&= \int (f(a_4(1)) - f(a_4(0))) \cdot ( dx_1 \wedge dx_2 \wedge dx_3 ) \\
&- \int (f(a_3(1)) - f(a_3(0))) \cdot ( dx_1 \wedge dx_2 \wedge dx_4 ) \\
&+ \int (f(a_2(1)) - f(a_2(0))) \cdot ( dx_1 \wedge dx_3 \wedge dx_4 ) \\
&- \int (f(a_1(1)) - f(a_1(0))) \cdot ( dx_2 \wedge dx_3 \wedge dx_4 ) \\
\end{aligned}
\end{equation}

This is precisely Stokes theorem for the trivector case and makes the enumeration of the boundary surfaces explicit.  As derived there was no requirement for an orthonormal basis, nor a Euclidean metric, nor a parametrization along the basis directions.  The only requirement of the parametrization is that the associated volume element is non-trivial (i.e. none of $dx_q \wedge dx_r = 0$).  %The issue of how to extend this from the hyper-parallelepiped volume element to a general has been skipped, and should perhaps be thought through more carefully.

For completeness, note that our boundary surface and associated Stokes statement for the bivector and vector cases is, by inspection respectively

\begin{equation}\label{eqn:tmp:n}
\begin{aligned}
\int ( \grad \wedge f ) \cdot d^3 x 
&= \int (f(a_3(1)) - f(a_3(0))) \cdot ( dx_1 \wedge dx_2 ) \\
&- \int (f(a_2(1)) - f(a_2(0))) \cdot ( dx_1 \wedge dx_3 ) \\
&+ \int (f(a_1(1)) - f(a_1(0))) \cdot ( dx_2 \wedge dx_3 ) \\
\end{aligned}
\end{equation}

and
\begin{equation}\label{eqn:tmp:n}
\begin{aligned}
\int ( \grad \wedge f ) \cdot d^2 x 
&= \int (f(a_2(1)) - f(a_2(0))) \cdot dx_1 \\
&- \int (f(a_1(1)) - f(a_1(0))) \cdot dx_2 \\
\end{aligned}
\end{equation}

These three expansions can be summarized by the original single statement of (\eqnref{eqn:stokesNoTensor:stokes}), which repeating for reference, is

\begin{equation}\label{eqn:tmp:n}
\begin{aligned}
\int ( \grad \wedge f ) \cdot d^k x = \int f \cdot d^{k-1} x 
\end{aligned}
\end{equation}

Where it is implied that the blade $f$ is evaluated on the boundaries and dotted with the associated hypersurface boundary element.  However, having expanded this we now have an explicit statement of exactly what that surface element is now for any desired parametrization.


%\EndArticle
\EndNoBibArticle
