%
% Copyright � 2015 Peeter Joot.  All Rights Reserved.
% Licenced as described in the file LICENSE under the root directory of this GIT repository.
%
\makeproblem{description}{gradQuantum:problemSet3:1}{ 

1. Aharonov Bohm effect (5 points)
(i) Consider Young's double slit experiment with electrons, having a monoenergetic source of electrons hitting a double
slit with slit spacing d, with the electrons then landing on a screen at a distance D away from the double slit. For
electrons with energy E, find the de Broglie wavelength \( \lambda \), and hence the spacing between the fringes on the screen.
You can ignore the drop in intensity as the electron beam `spreads' when it travels from the slits to the screen (recall
that the slits act as effective point sources), so just take phase changes into account along the travel path.
(ii) Next, imagine a thin solenoidal flux \( \Phi \) being placed between the two slits, so that electron paths which encircle
the flux once will pick up an Aharonov Bohm phase \( e \Phi/\Hbar \). Compute the resulting shift in the interference pattern
on the screen. Show that when the flux \( \Phi \) is increased from \( 0 \rightarrow h/e \), the interference pattern shifts by exactly one
fringe, so the new pattern appears the same as the old. This is the same flux periodicity we saw in class for the energy
levels versus flux for a particle on a ring.

\makesubproblem{}{gradQuantum:problemSet3:1a}
} % makeproblem

\makeanswer{gradQuantum:problemSet3:1}{ 
\makeSubAnswer{}{gradQuantum:problemSet3:1a}

TODO.
}

