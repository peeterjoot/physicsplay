\makeproblem{Sackur-Tetrode entropy of an Ideal Gas}{basicStatMech:problemSet4:1}{ 
% - Schroeder 2.35
The entropy of an ideal gas is given by

\begin{equation}\label{eqn:basicStatMechProblemSet4Problem1:20}
S = 
N k_{\mathrm{B}}
\lr{
\ln
\lr{
\frac{V}{N} 
\lr{
\frac{4 \pi m E}{3 N h^2}
}^{3/2}
}
+ \frac{5}{2}
}
\end{equation}

Find the temperature of this gas via $(\partial S/ \partial E)_{V,N} = 1/T$. Find the energy per particle at which the entropy becomes negative. Is there any meaning to this temperature?
} % makeproblem

\makeanswer{basicStatMech:problemSet4:1}{ 

Taking derivatives we find

\begin{dmath}\label{eqn:basicStatMechProblemSet4Problem1:40}
\inv{T} = 
\PD{E}{}
\lr{
\cancel{
N k_{\mathrm{B}}
\ln
\frac{V}{N} 
}
+
N k_{\mathrm{B}}
\frac{3}{2} \ln
\lr{ \frac{4 \pi m E}{3 N h^2} }
+ 
\cancel{N k_{\mathrm{B}}
\frac{5}{2}
}
}
=
\frac{3}{2} N k_{\mathrm{B}} \inv{E}
\end{dmath}

or
\begin{equation}\label{eqn:basicStatMechProblemSet4Problem1:60}
\myBoxed{
%\inv{T} = \frac{3}{2 E} N k_{\mathrm{B}} 
T = \frac{2}{3} \frac{E}{N k_{\mathrm{B}} }
}
\end{equation}

The energies for which the entropy is negative are given by

\begin{equation}\label{eqn:basicStatMechProblemSet4Problem1:80}
\lr{
\frac{4 \pi m E}{3 N h^2}
}^{3/2}
\le \frac{N}{V} e^{-5/2},
\end{equation}

or
\begin{equation}\label{eqn:basicStatMechProblemSet4Problem1:100}
E \le 
\frac{3 N h^2}{4 \pi m} \lr{\frac{N}{V e^{5/2}} }^{2/3}
=
\frac{3 h^2 N^{5/3}}{4 \pi m V^{2/3} e^{5/2}}.
\end{equation}

In terms of the temperature $T$ this negative entropy condition is given by

\begin{equation}\label{eqn:basicStatMechProblemSet4Problem1:120}
\cancel{\frac{3 N}{2}} k_{\mathrm{B}} T \le \cancel{\frac{3 N}{2}} \lr{\frac{ N}{V}}^{2/3} \frac{h^2}{e^{5/2}},
\end{equation}

or

\begin{equation}\label{eqn:basicStatMechProblemSet4Problem1:140}
\myBoxed{
\frac{\sqrt{2 \pi m k_{\mathrm{B}} T}}{h} \le \lr{\frac{N}{V}}^{1/3} \inv{e^{5/4}}.
}
\end{equation}

There will be a particle density $V/N$ for which this distance $h/\sqrt{2 \pi m k_{\mathrm{B}} T}$ will start approaching the distance between atoms.  This distance constrains the validity of the ideal gas law entropy equation.  Putting this quantity back into the entropy \eqnref{eqn:basicStatMechProblemSet4Problem1:20} we have

\begin{equation}\label{eqn:basicStatMechProblemSet4Problem1:160}
\frac{S}{N k_{\mathrm{B}}} = \ln \frac{V}{N} \lr{\frac{\sqrt{2 \pi m k_{\mathrm{B}} T}}{h}}^3 + \frac{5}{2}
\end{equation}

We see that a positive entropy requirement puts a bound on this distance (as a function of temperature) since we must also have

\begin{equation}\label{eqn:basicStatMechProblemSet4Problem1:180}
\frac{h}{\sqrt{2 \pi m k_{\mathrm{B}} T}} \ll \lr{\frac{V}{N}}^{1/3},
\end{equation}

for the gas to be in the classical domain.  I'd actually expect a gas to liquify before this transition point, making such a low temperature unphysical.

FIXME: check this for hydrogen (say).
}
