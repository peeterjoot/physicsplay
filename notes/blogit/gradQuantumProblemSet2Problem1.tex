%
% Copyright � 2015 Peeter Joot.  All Rights Reserved.
% Licenced as described in the file LICENSE under the root directory of this GIT repository.
%

\makeproblem{Coherent States}{gradQuantum:problemSet2:1}{

Consider the harmonic oscillator Hamiltonian \( H = p^2/2m + m \omega^2 x^2/2\). Define the coherent state \( \ket{z} \) as the eigenfunction of the annihilation operator, via \( a \ket{z} = z \ket{z} \), where \( a \) is the oscillator annihilation operator and \( z \) is some complex number which characterizes the coherent state.

\makesubproblem{}{gradQuantum:problemSet2:1a}
Expanding \( \ket{ z } \) in terms of oscillator energy eigenstates \( \ket{n} \), show that \( \ket{z} = C e^{z a^\dagger} \ket{0} \). Find the normalization constant C.

\makesubproblem{}{gradQuantum:problemSet2:1b}
Calculate the overlap \( \braket{z}{z'} \) for normalized coherent states \( \ket{z} \).

\makesubproblem{}{gradQuantum:problemSet2:1c}
Using the wavefunction \( \ket{z} \), compute \( \expectation{x}, \expectation{p}, \expectation{x^2}\), and \( \expectation{p^2} \) by defining \( x,p \) in terms of \( a, a^\dagger \).

\makesubproblem{}{gradQuantum:problemSet2:1d}

The time evolution of any observed quantity in quantum mechanics can be described in two ways:

\begin{enumerate}[(a)]
\item Schrodinger: the wavefunction evolves as \( \ket{\psi(t)} = e^{-i H t/\Hbar} \ket{\psi(0)} \) and the operator \( A \) is time-independent, or
\item Heisenberg: the wavefunction is fixed to its value at \( t = 0 \), say \(  \ket{\psi} \) , and operators evolve as \( A(t) = e^{i H t/\Hbar} A e^{-i H t/\Hbar}\) .
\end{enumerate}

Show that both prescriptions yield the same result for any matrix elements or measured quantities.

\makesubproblem{}{gradQuantum:problemSet2:1e}
Using the Heisenberg picture, compute the time evolution of \( \expectation{x}(t), \expectation{p}(t), \expectation{x^2}(t)\), and \( \expectation{p^2}(t) \) in the coherent state \( \ket{z} \).  Comment on connections to classical dynamics of the oscillator in phase space.
} % makeproblem

\makeanswer{gradQuantum:problemSet2:1}{
\makeSubAnswer{}{gradQuantum:problemSet2:1a}

TODO.
}

