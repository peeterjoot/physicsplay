%
% Copyright � 2013 Peeter Joot.  All Rights Reserved.
% Licenced as described in the file LICENSE under the root directory of this GIT repository.
%
\makeproblem{Maximum entropy principle}{basicStatMech:problemSet6:1}{ 
%(3 points)
%\makesubproblem{}{basicStatMech:problemSet6:1a}
Consider the �Gibbs entropy� S = -kB
P
ipilnpiwhere piis the equilibrium probability of occurrence of a microstate
i in the ensemble.
(i) For a microcanonical ensemble with ? configurations (each having the same energy), assigning an equal probability
pi= 1/? to each microstate leads to S = kBln?. Show that this result follows from maximizing the Gibbs entropy
with respect to the parameters pisubject to the constraint ofP
ipi= 1 (for pito be meaningful as probabilities). In
order to do the minimization with this constraint, use the method of Lagrange multipliers - first, do an unconstrained
minimization of the function S - aP
ipi, then fix a by demanding that the constraint be satisfied.
(ii) For a canonical ensemble (no constraint on total energy, but all microstates having the same number of particles
N), maximize the Gibbs entropy with respect to the parameters pi subject to the constraint ofP
ipi = 1 (for pi
to be meaningful as probabilities) and with a given fixed average energy hEi =P
iEipi, where Ei is the energy
of microstate i.
Use the method of Lagrange multipliers, doing an unconstrained minimization of the function
S - aP
ipi- �P
iEipi, then fix a,� by demanding that the constraint be satisfied. What is the resulting pi?
(iii) For a grand canonical ensemble (no constraint on total energy, or the number of particles), maximize the Gibbs
entropy with respect to the parameters pisubject to the constraint ofP
ipi= 1 (for pito be meaningful as probabili-
ties) and with a given fixed average energy hEi =P
iEipi, and a given fixed average particle number hNi =P
iNipi.
Here Ei,Nirepresent the energy and number of particles in microstate i. Use the method of Lagrange multipliers,
doing an unconstrained minimization of the function S-aP

} % makeproblem

\makeanswer{basicStatMech:problemSet6:1}{ 
%\makeSubAnswer{XXX}{basicStatMech:problemSet6:1a}

TODO.
}

