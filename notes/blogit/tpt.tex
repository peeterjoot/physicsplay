%
% Copyright � 2016 Peeter Joot.  All Rights Reserved.
% Licenced as described in the file LICENSE under the root directory of this GIT repository.
%
%{
\newcommand{\authorname}{Peeter Joot}
\newcommand{\email}{peeterjoot@protonmail.com}
\newcommand{\basename}{FIXMEbasenameUndefined}
\newcommand{\dirname}{notes/FIXMEdirnameUndefined/}

\renewcommand{\basename}{poyntingTimeHarmonic}
%\renewcommand{\dirname}{notes/phy1520/}
\renewcommand{\dirname}{notes/ece1228-electromagnetic-theory/}
%\newcommand{\dateintitle}{}
%\newcommand{\keywords}{}

\newcommand{\authorname}{Peeter Joot}
\newcommand{\onlineurl}{http://sites.google.com/site/peeterjoot2/math2013/\basename.pdf}
\newcommand{\sourcepath}{\dirname\basename.tex}
\newcommand{\generatetitle}[1]{\chapter{#1}}

\newcommand{\vcsinfo}{%
\section*{}
\noindent{\color{DarkOliveGreen}{\rule{\linewidth}{0.1mm}}}
\paragraph{Document version}
%\paragraph{\color{Maroon}{Document version}}
{
\small
\begin{itemize}
\item Available online at:\\ 
\href{\onlineurl}{\onlineurl}
\item Git Repository: \input{./.revinfo/gitRepo.tex}
\item Source: \sourcepath
\item last commit: \input{./.revinfo/gitCommitString.tex}
\item commit date: \input{./.revinfo/gitCommitDate.tex}
\end{itemize}
}
}

%\PassOptionsToPackage{dvipsnames,svgnames}{xcolor}
\PassOptionsToPackage{square,numbers}{natbib}
\documentclass{scrreprt}

\usepackage[left=2cm,right=2cm]{geometry}
\usepackage[svgnames]{xcolor}
\usepackage{peeters_layout}

\usepackage{natbib}

\usepackage[
colorlinks=true,
bookmarks=false,
pdfauthor={\authorname, \email},
backref 
]{hyperref}

% http://tex.stackexchange.com/questions/75773/how-to-reference-problems-by-the-text-label-in-an-exercise-envioronment
\usepackage[english]{cleveref}
\crefname{Exercise}{exercise}{exercises}
\Crefname{Exercise}{Exercise}{Exercises}

\RequirePackage{titlesec}
\RequirePackage{ifthen}

% http://stackoverflow.com/questions/4932910/date-in-the-tabular-environment
\makeatletter
\let\insertdate\@date
\makeatother

\titleformat{\chapter}[display]
{\bfseries\Large}
{\color{DarkSlateGrey}\filleft \authorname
\ifthenelse{\isundefined{\studentnumber}}{}{\\ \studentnumber}
\ifthenelse{\isundefined{\email}}{}{\\ \email}
\ifthenelse{\isundefined{\dateintitle}}{}{\\ \insertdate}
%\ifthenelse{\isundefined{\coursename}}{}{\\ \coursename} % put in title instead.
}
{4ex}
{\color{DarkOliveGreen}{\titlerule}\color{Maroon}
\vspace{2ex}%
\filright}
[\vspace{2ex}%
\color{DarkOliveGreen}\titlerule
]

\newcommand{\beginArtWithToc}[0]{\begin{document}\tableofcontents}
\newcommand{\beginArtNoToc}[0]{\begin{document}}
\newcommand{\EndNoBibArticle}[0]{\end{document}}
\newcommand{\EndArticle}[0]{\bibliography{Bibliography}\bibliographystyle{plainnat}\end{document}}

% 
%\newcommand{\citep}[1]{\cite{#1}}

\colorSectionsForArticle



\usepackage{peeters_layout_exercise}
\usepackage{peeters_braket}
\usepackage{peeters_figures}
\usepackage{siunitx}
\usepackage{macros_bm}
%\usepackage{txfonts} % \ointclockwise

\beginArtNoToc

\generatetitle{Frequency domain time averaged Poynting theorem}
%\chapter{Frequency domain time averaged Poynting theorem}
%\label{chap:poyntingTimeHarmonic}
\index{Poynting theorem}
\index{time harmonic!Poynting theorem}

The time domain Poynting relationship was found to be

\begin{dmath}\label{eqn:poyntingTimeHarmonic:20}
0
=
\spacegrad \cdot \lr{ \BE \cross \BH } 
+ \frac{\epsilon}{2} \BE \cdot \PD{t}{\BE}
+ \frac{\mu}{2} \BH \cdot \PD{t}{\BH}
+ \BH \cdot \BM_i 
+ \BE \cdot \BJ_i 
+ \sigma \BE \cdot \BE.
\end{dmath}

Let's derive the equivalent relationship for the time averaged portion of the time-harmonic Poynting vector.  The time domain representation of the Poynting vector in terms of the time-harmonic (phasor) vectors is

\begin{dmath}\label{eqn:poyntingTimeHarmonic:40}
\begin{aligned}
\bcE \cross \bcH
&= \inv{4} 
\lr{ 
\BE e^{j\omega t}
+ \BE^\conj e^{-j\omega t}
}
\cross
\lr{ 
\BH e^{j\omega t}
+ \BH^\conj e^{-j\omega t}
} \\
&=
\inv{2} \Real \lr{ \BE \cross \BH^\conj + \BE \cross \BH e^{j \omega t} },
\end{aligned}
\end{dmath}

so if we are looking for the relationships that effect only the time averaged Poynting vector, over integral multiples of the period, we are interested in evaluating the divergence of

\begin{dmath}\label{eqn:poyntingTimeHarmonic:60}
\inv{2} \BE \cross \BH^\conj.
\end{dmath}

The time-harmonic Maxwell's equations are
\begin{dmath}\label{eqn:poyntingTimeHarmonic:80}
\begin{aligned}
\spacegrad \cross \BE &= - j \omega \mu \BH - \BM_i \\
\spacegrad \cross \BH &= j \omega \epsilon \BE + \BJ_i + \sigma \BE \\
\end{aligned}
\end{dmath}

The latter after conjugation is

\begin{dmath}\label{eqn:poyntingTimeHarmonic:100}
\spacegrad \cross \BH^\conj = -j \omega \epsilon^\conj \BE^\conj + \BJ_i^\conj + \sigma^\conj \BE^\conj.
\end{dmath}

For the divergence we have

\begin{dmath}\label{eqn:poyntingTimeHarmonic:120}
\begin{aligned}
\spacegrad \cdot \lr{ \BE \cross \BH^\conj }
&=
\BH^\conj \cdot \lr{ \spacegrad \cdot \BE }
-\BE \cdot \lr{ \spacegrad \cdot \BH^\conj } \\
&=
\BH^\conj \cdot \lr{ - j \omega \mu \BH - \BM_i }
- \BE \cdot \lr{ -j \omega \epsilon^\conj \BE^\conj + \BJ_i^\conj + \sigma^\conj \BE^\conj },
\end{aligned}
\end{dmath}

or

\begin{dmath}\label{eqn:poyntingTimeHarmonic:140}
0 
=
\spacegrad \cdot \lr{ \BE \cross \BH^\conj }
+
\BH^\conj \cdot \lr{ j \omega \mu \BH + \BM_i }
+ \BE \cdot \lr{ -j \omega \epsilon^\conj \BE^\conj + \BJ_i^\conj + \sigma^\conj \BE^\conj },
\end{dmath}

so
\begin{dmath}\label{eqn:poyntingTimeHarmonic:160}
\boxed{
0 
=
\spacegrad \cdot \inv{2} \lr{ \BE \cross \BH^\conj }
+ \inv{2} \lr{ \BH^\conj \cdot \BM_i 
+ \BE \cdot \BJ_i^\conj }
+ \inv{2} j \omega \lr{ \mu \Abs{\BH}^2 - \epsilon^\conj \Abs{\BE}^2 }
+ \inv{2} \sigma^\conj \Abs{\BE}^2.
}
\end{dmath}

%}
\EndNoBibArticle
