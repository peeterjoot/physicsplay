%
% Copyright � 2013 Peeter Joot.  All Rights Reserved.
% Licenced as described in the file LICENSE under the root directory of this GIT repository.
%
\makeproblem{Fcc reciprocal lattice}{condensedMatter:problemSet3:2}{ 
Now for a face-centred cubic lattice with conventional unit cell 
  of side length $a$: 
\makesubproblem{}{condensedMatter:problemSet3:2a}
  Draw the conventional unit cell and number 
  all of the corner and face-centred atoms, and demonstrate that the vectors 
  $\Ba_1 = \frac{a}{2}(1,1,0)$, 
  $\Ba_2 = \frac{a}{2}(1,0,1)$ and 
  $\Ba_3 = \frac{a}{2}(0,1,1)$, are primitive lattice vectors 
  in the sense that you can get to every lattice point in the unit 
  cell using these vectors. 

\makesubproblem{}{condensedMatter:problemSet3:2b}
Using the formula from the lectures show that the volume of the 
  primitive unit cell is 1/4 of the volume of the conventional unit cell.

\makesubproblem{}{condensedMatter:problemSet3:2c}
Using the formula from the lectures,  find the basis vectors of 
  the corresponding reciprocal lattice, 
  and show that these basis vectors generate a body-centred-cubic 
  lattice in reciprocal space.  

} % makeproblem

\makeanswer{condensedMatter:problemSet3:2}{ 
\makeSubAnswer{}{condensedMatter:problemSet3:2a}

TODO.
\makeSubAnswer{}{condensedMatter:problemSet3:2b}

TODO.
\makeSubAnswer{}{condensedMatter:problemSet3:2c}

Computing the three sets of cross products we have

\begin{subequations}
\begin{equation}\label{eqn:condensedMatterProblemSet3Problem2:20}
\Ba_2 \cross \Ba_3 = 
\lr{\frac{a}{2}}^2
\begin{vmatrix}
\xcap & \ycap & \zcap \\
1 & 0 & 1 \\
0 & 1 & 1
\end{vmatrix}
= 
\lr{\frac{a}{2}}^2
\begin{bmatrix}
-1 \\
-1 \\ 
1
\end{bmatrix}
\end{equation}
\begin{equation}\label{eqn:condensedMatterProblemSet3Problem2:40}
\Ba_3 \cross \Ba_1 = 
\lr{\frac{a}{2}}^2
\begin{vmatrix}
\xcap & \ycap & \zcap \\
0 & 1 & 1 \\
1 & 1 & 0
\end{vmatrix}
= 
\lr{\frac{a}{2}}^2
\begin{bmatrix}
-1 \\
1 \\ 
-1
\end{bmatrix}
\end{equation}
\begin{equation}\label{eqn:condensedMatterProblemSet3Problem2:60}
\Ba_1 \cross \Ba_2 = 
\lr{\frac{a}{2}}^2
\begin{vmatrix}
\xcap & \ycap & \zcap \\
1 & 1 & 0 \\
1 & 0 & 1
\end{vmatrix}
= 
\lr{\frac{a}{2}}^2
\begin{bmatrix}
1 \\
-1 \\ 
-1
\end{bmatrix}.
\end{equation}
\end{subequations}

Our triplet product is
\begin{dmath}\label{eqn:condensedMatterProblemSet3Problem2:80}
\Ba_1 \cdot \lr{ \Ba_2 \cross \Ba_3 } = 
\lr{\frac{a}{2}}^3 \lr{ -2 }
\end{dmath}

Putting these together we have
\begin{subequations}
\label{eqn:condensedMatterProblemSet3Problem2:100a}
\begin{equation}\label{eqn:condensedMatterProblemSet3Problem2:100}
\Bg_1 
= 2 \pi \frac{ \Ba_2 \cross \Ba_3 }{ 
\Ba_1 \cdot \lr{ \Ba_2 \cross \Ba_3 } }
=  \frac{2 \pi}{a} 
\begin{bmatrix}
1 \\ 1 \\ -1
\end{bmatrix}
\end{equation}
\begin{equation}\label{eqn:condensedMatterProblemSet3Problem2:120}
\Bg_2 
= 2 \pi \frac{ \Ba_3 \cross \Ba_1 }{ 
\Ba_1 \cdot \lr{ \Ba_2 \cross \Ba_3 } }
=  \frac{2 \pi}{a} 
\begin{bmatrix}
1 \\ -1 \\ 1
\end{bmatrix}
\end{equation}
\begin{equation}\label{eqn:condensedMatterProblemSet3Problem2:140}
\Bg_3 
= 2 \pi \frac{ \Ba_3 \cross \Ba_1 }{ 
\Ba_1 \cdot \lr{ \Ba_2 \cross \Ba_3 } }
=  \frac{2 \pi}{a} 
\begin{bmatrix}
-1 \\ 1 \\ 1
\end{bmatrix}
\end{equation}
\end{subequations}

Note that we can also compute all the reciprocal basis vectors more directly by inversion

\begin{dmath}\label{eqn:condensedMatterProblemSet3Problem2:160}
\begin{bmatrix}
\Bg_1 & \Bg_2 & \Bg_3 
\end{bmatrix}
=
2 \pi
\begin{bmatrix}
\Ba_1^\T \\
\Ba_2^\T \\
\Ba_3^\T
\end{bmatrix}
=
2 \pi
\frac{2}{a}
{
\begin{bmatrix}
1&1&0 \\
1&0&1 \\
0&1&1
\end{bmatrix}
}
^{-1}
=
\frac{2 \pi}{a}
\begin{bmatrix}
1 & 1 & -1 \\
1 & -1 & 1 \\
-1 & 1 & 1
\end{bmatrix},
\end{dmath}

consistent with the cross product calculation of \cref{eqn:condensedMatterProblemSet3Problem2:100a}.
}

