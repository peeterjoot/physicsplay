%%
% Copyright � 2015 Peeter Joot.  All Rights Reserved.
% Licenced as described in the file LICENSE under the root directory of this GIT repository.
%
\documentclass[]{eliblog}

\usepackage{amsmath}
\usepackage{mathpazo}

%
% shorthand for bold symbols, convenient for vectors and matrices
%
\newcommand{\Ba}[0]{\mathbf{a}}
\newcommand{\Bb}[0]{\mathbf{b}}
\newcommand{\Bc}[0]{\mathbf{c}}
\newcommand{\Bd}[0]{\mathbf{d}}
\newcommand{\Be}[0]{\mathbf{e}}
\newcommand{\Bf}[0]{\mathbf{f}}
\newcommand{\Bg}[0]{\mathbf{g}}
\newcommand{\Bh}[0]{\mathbf{h}}
\newcommand{\Bi}[0]{\mathbf{i}}
\newcommand{\Bj}[0]{\mathbf{j}}
\newcommand{\Bk}[0]{\mathbf{k}}
\newcommand{\Bl}[0]{\mathbf{l}}
\newcommand{\Bm}[0]{\mathbf{m}}
\newcommand{\Bn}[0]{\mathbf{n}}
\newcommand{\Bo}[0]{\mathbf{o}}
\newcommand{\Bp}[0]{\mathbf{p}}
\newcommand{\Bq}[0]{\mathbf{q}}
\newcommand{\Br}[0]{\mathbf{r}}
\newcommand{\Bs}[0]{\mathbf{s}}
\newcommand{\Bt}[0]{\mathbf{t}}
\newcommand{\Bu}[0]{\mathbf{u}}
\newcommand{\Bv}[0]{\mathbf{v}}
\newcommand{\Bw}[0]{\mathbf{w}}
\newcommand{\Bx}[0]{\mathbf{x}}
\newcommand{\By}[0]{\mathbf{y}}
\newcommand{\Bz}[0]{\mathbf{z}}
\newcommand{\BA}[0]{\mathbf{A}}
\newcommand{\BB}[0]{\mathbf{B}}
\newcommand{\BC}[0]{\mathbf{C}}
\newcommand{\BD}[0]{\mathbf{D}}
\newcommand{\BE}[0]{\mathbf{E}}
\newcommand{\BF}[0]{\mathbf{F}}
\newcommand{\BG}[0]{\mathbf{G}}
\newcommand{\BH}[0]{\mathbf{H}}
\newcommand{\BI}[0]{\mathbf{I}}
\newcommand{\BJ}[0]{\mathbf{J}}
\newcommand{\BK}[0]{\mathbf{K}}
\newcommand{\BL}[0]{\mathbf{L}}
\newcommand{\BM}[0]{\mathbf{M}}
\newcommand{\BN}[0]{\mathbf{N}}
\newcommand{\BO}[0]{\mathbf{O}}
\newcommand{\BP}[0]{\mathbf{P}}
\newcommand{\BQ}[0]{\mathbf{Q}}
\newcommand{\BR}[0]{\mathbf{R}}
\newcommand{\BS}[0]{\mathbf{S}}
\newcommand{\BT}[0]{\mathbf{T}}
\newcommand{\BU}[0]{\mathbf{U}}
\newcommand{\BV}[0]{\mathbf{V}}
\newcommand{\BW}[0]{\mathbf{W}}
\newcommand{\BX}[0]{\mathbf{X}}
\newcommand{\BY}[0]{\mathbf{Y}}
\newcommand{\BZ}[0]{\mathbf{Z}}

\newcommand{\Bzero}[0]{\mathbf{0}}
\newcommand{\Btheta}[0]{\boldsymbol{\theta}}
\newcommand{\Btau}[0]{\boldsymbol{\tau}}
\newcommand{\Bomega}[0]{\boldsymbol{\omega}}

%
% shorthand for unit vectors
%
\newcommand{\acap}[0]{\hat{\Ba}}
\newcommand{\bcap}[0]{\hat{\Bb}}
\newcommand{\ccap}[0]{\hat{\Bc}}
\newcommand{\dcap}[0]{\hat{\Bd}}
\newcommand{\ecap}[0]{\hat{\Be}}
\newcommand{\fcap}[0]{\hat{\Bf}}
\newcommand{\gcap}[0]{\hat{\Bg}}
\newcommand{\hcap}[0]{\hat{\Bh}}
\newcommand{\icap}[0]{\hat{\Bi}}
\newcommand{\jcap}[0]{\hat{\Bj}}
\newcommand{\kcap}[0]{\hat{\Bk}}
\newcommand{\lcap}[0]{\hat{\Bl}}
\newcommand{\mcap}[0]{\hat{\Bm}}
\newcommand{\ncap}[0]{\hat{\Bn}}
\newcommand{\ocap}[0]{\hat{\Bo}}
\newcommand{\pcap}[0]{\hat{\Bp}}
\newcommand{\qcap}[0]{\hat{\Bq}}
\newcommand{\rcap}[0]{\hat{\Br}}
\newcommand{\scap}[0]{\hat{\Bs}}
\newcommand{\tcap}[0]{\hat{\Bt}}
\newcommand{\ucap}[0]{\hat{\Bu}}
\newcommand{\vcap}[0]{\hat{\Bv}}
\newcommand{\wcap}[0]{\hat{\Bw}}
\newcommand{\xcap}[0]{\hat{\Bx}}
\newcommand{\ycap}[0]{\hat{\By}}
\newcommand{\zcap}[0]{\hat{\Bz}}
\newcommand{\thetacap}[0]{\hat{\Btheta}}

%
% to write R^n and C^n in a distinguishable fashion.  Perhaps change this
% to the double lined characters upon figuring out how to do so.
%
\newcommand{\C}[1]{$\mathbb{C}^{#1}$}
\newcommand{\R}[1]{$\mathbb{R}^{#1}$}

%
% various generally useful helpers
%

% derivative of #1 wrt. #2:
\newcommand{\D}[2] {\frac {d#2} {d#1}}

\newcommand{\inv}[1]{\frac{1}{#1}}
\newcommand{\cross}[0]{\times}

\newcommand{\abs}[1]{\lvert{#1}\rvert}
\newcommand{\norm}[1]{\lVert{#1}\rVert}
\newcommand{\innerprod}[2]{\langle{#1}, {#2}\rangle}
\newcommand{\dotprod}[2]{{#1} \cdot {#2}}
\newcommand{\bdotprod}[2]{\left({#1} \cdot {#2}\right)}
\newcommand{\crossprod}[2]{{#1} \cross {#2}}
\newcommand{\tripleprod}[3]{\dotprod{\left(\crossprod{#1}{#2}\right)}{#3}}

\DeclareMathOperator{\Proj}{Proj}
\DeclareMathOperator{\Span}{span}
\DeclareMathOperator{\Sgn}{sgn}
\DeclareMathOperator{\Area}{Area}
\DeclareMathOperator{\Volume}{Volume}

%
% A few miscellaneous things specific to this document
%
\newcommand{\crossop}[1]{\crossprod{#1}{}}

% R2 vector.
\newcommand{\VectorTwo}[2]{
\begin{bmatrix}
 {#1} \\
 {#2}
\end{bmatrix}
}

\newcommand{\VectorN}[1]{
\begin{bmatrix}
{#1}_1 \\
{#1}_2 \\
\vdots \\
{#1}_N \\
\end{bmatrix}
}

\newcommand{\DETuvij}[4]{
\begin{vmatrix}
 {#1}_{#3} & {#1}_{#4} \\
 {#2}_{#3} & {#2}_{#4}
\end{vmatrix}
}

\newcommand{\DETuvwijk}[6]{
\begin{vmatrix}
 {#1}_{#4} & {#1}_{#5} & {#1}_{#6} \\
 {#2}_{#4} & {#2}_{#5} & {#2}_{#6} \\
 {#3}_{#4} & {#3}_{#5} & {#3}_{#6}
\end{vmatrix}
}

\newcommand{\DETuvwxijkl}[8]{
\begin{vmatrix}
 {#1}_{#5} & {#1}_{#6} & {#1}_{#7} & {#1}_{#8} \\
 {#2}_{#5} & {#2}_{#6} & {#2}_{#7} & {#2}_{#8} \\
 {#3}_{#5} & {#3}_{#6} & {#3}_{#7} & {#3}_{#8} \\
 {#4}_{#5} & {#4}_{#6} & {#4}_{#7} & {#4}_{#8} \\
\end{vmatrix}
}

%\newcommand{\DETuvwxyijklm}[10]{
%\begin{vmatrix}
% {#1}_{#6} & {#1}_{#7} & {#1}_{#8} & {#1}_{#9} & {#1}_{#10} \\
% {#2}_{#6} & {#2}_{#7} & {#2}_{#8} & {#2}_{#9} & {#2}_{#10} \\
% {#3}_{#6} & {#3}_{#7} & {#3}_{#8} & {#3}_{#9} & {#3}_{#10} \\
% {#4}_{#6} & {#4}_{#7} & {#4}_{#8} & {#4}_{#9} & {#4}_{#10} \\
% {#5}_{#6} & {#5}_{#7} & {#5}_{#8} & {#5}_{#9} & {#5}_{#10}
%\end{vmatrix}
%}

% R3 vector.
\newcommand{\VectorThree}[3]{
\begin{bmatrix}
 {#1} \\
 {#2} \\
 {#3}
\end{bmatrix}
}



\author{Peeter Joot}
\email{peeter.joot@gmail.com}

%
%
%
% Copyright � 2012 Peeter Joot
% All Rights Reserved
%
% This file may be reproduced and distributed in whole or in part, without fee, subject to the following conditions:
%
% o The copyright notice above and this permission notice must be preserved complete on all complete or partial copies.
%
% o Any translation or derived work must be approved by the author in writing before distribution.
%
% o If you distribute this work in part, instructions for obtaining the complete version of this file must be included, and a means for obtaining a complete version provided.
%
%
% Exceptions to these rules may be granted for academic purposes: Write to the author and ask.
%
%
%
% journal style swiped from:
%
% http://arxiv.org/abs/1006.3552

% this one is a double column compact form.
%\documentclass[iop]{emulateapj}
% slightly less space used with the tighten option.
\documentclass[iop,tighten]{emulateapj}
%\documentclass[iop,onecolumn]{emulateapj}
%
% http://www.tug.org/texlive/Contents/live/texmf-dist/doc/latex/aastex/aasguide.pdf
% full width variation:
%\documentclass[12pt,preprint]{aastex}
% double column variation, not as compact as the emulateapj above
%\documentclass[12pt,preprint2]{aastex}

%\usepackage{apjfonts}
%\usepackage[numbers]{natbib}

\usepackage{amsmath}
\usepackage{mathpazo}

%
% shorthand for bold symbols, convenient for vectors and matrices
%
\newcommand{\Ba}[0]{\mathbf{a}}
\newcommand{\Bb}[0]{\mathbf{b}}
\newcommand{\Bc}[0]{\mathbf{c}}
\newcommand{\Bd}[0]{\mathbf{d}}
\newcommand{\Be}[0]{\mathbf{e}}
\newcommand{\Bf}[0]{\mathbf{f}}
\newcommand{\Bg}[0]{\mathbf{g}}
\newcommand{\Bh}[0]{\mathbf{h}}
\newcommand{\Bi}[0]{\mathbf{i}}
\newcommand{\Bj}[0]{\mathbf{j}}
\newcommand{\Bk}[0]{\mathbf{k}}
\newcommand{\Bl}[0]{\mathbf{l}}
\newcommand{\Bm}[0]{\mathbf{m}}
\newcommand{\Bn}[0]{\mathbf{n}}
\newcommand{\Bo}[0]{\mathbf{o}}
\newcommand{\Bp}[0]{\mathbf{p}}
\newcommand{\Bq}[0]{\mathbf{q}}
\newcommand{\Br}[0]{\mathbf{r}}
\newcommand{\Bs}[0]{\mathbf{s}}
\newcommand{\Bt}[0]{\mathbf{t}}
\newcommand{\Bu}[0]{\mathbf{u}}
\newcommand{\Bv}[0]{\mathbf{v}}
\newcommand{\Bw}[0]{\mathbf{w}}
\newcommand{\Bx}[0]{\mathbf{x}}
\newcommand{\By}[0]{\mathbf{y}}
\newcommand{\Bz}[0]{\mathbf{z}}
\newcommand{\BA}[0]{\mathbf{A}}
\newcommand{\BB}[0]{\mathbf{B}}
\newcommand{\BC}[0]{\mathbf{C}}
\newcommand{\BD}[0]{\mathbf{D}}
\newcommand{\BE}[0]{\mathbf{E}}
\newcommand{\BF}[0]{\mathbf{F}}
\newcommand{\BG}[0]{\mathbf{G}}
\newcommand{\BH}[0]{\mathbf{H}}
\newcommand{\BI}[0]{\mathbf{I}}
\newcommand{\BJ}[0]{\mathbf{J}}
\newcommand{\BK}[0]{\mathbf{K}}
\newcommand{\BL}[0]{\mathbf{L}}
\newcommand{\BM}[0]{\mathbf{M}}
\newcommand{\BN}[0]{\mathbf{N}}
\newcommand{\BO}[0]{\mathbf{O}}
\newcommand{\BP}[0]{\mathbf{P}}
\newcommand{\BQ}[0]{\mathbf{Q}}
\newcommand{\BR}[0]{\mathbf{R}}
\newcommand{\BS}[0]{\mathbf{S}}
\newcommand{\BT}[0]{\mathbf{T}}
\newcommand{\BU}[0]{\mathbf{U}}
\newcommand{\BV}[0]{\mathbf{V}}
\newcommand{\BW}[0]{\mathbf{W}}
\newcommand{\BX}[0]{\mathbf{X}}
\newcommand{\BY}[0]{\mathbf{Y}}
\newcommand{\BZ}[0]{\mathbf{Z}}

\newcommand{\Bzero}[0]{\mathbf{0}}
\newcommand{\Btheta}[0]{\boldsymbol{\theta}}
\newcommand{\Btau}[0]{\boldsymbol{\tau}}
\newcommand{\Bomega}[0]{\boldsymbol{\omega}}

%
% shorthand for unit vectors
%
\newcommand{\acap}[0]{\hat{\Ba}}
\newcommand{\bcap}[0]{\hat{\Bb}}
\newcommand{\ccap}[0]{\hat{\Bc}}
\newcommand{\dcap}[0]{\hat{\Bd}}
\newcommand{\ecap}[0]{\hat{\Be}}
\newcommand{\fcap}[0]{\hat{\Bf}}
\newcommand{\gcap}[0]{\hat{\Bg}}
\newcommand{\hcap}[0]{\hat{\Bh}}
\newcommand{\icap}[0]{\hat{\Bi}}
\newcommand{\jcap}[0]{\hat{\Bj}}
\newcommand{\kcap}[0]{\hat{\Bk}}
\newcommand{\lcap}[0]{\hat{\Bl}}
\newcommand{\mcap}[0]{\hat{\Bm}}
\newcommand{\ncap}[0]{\hat{\Bn}}
\newcommand{\ocap}[0]{\hat{\Bo}}
\newcommand{\pcap}[0]{\hat{\Bp}}
\newcommand{\qcap}[0]{\hat{\Bq}}
\newcommand{\rcap}[0]{\hat{\Br}}
\newcommand{\scap}[0]{\hat{\Bs}}
\newcommand{\tcap}[0]{\hat{\Bt}}
\newcommand{\ucap}[0]{\hat{\Bu}}
\newcommand{\vcap}[0]{\hat{\Bv}}
\newcommand{\wcap}[0]{\hat{\Bw}}
\newcommand{\xcap}[0]{\hat{\Bx}}
\newcommand{\ycap}[0]{\hat{\By}}
\newcommand{\zcap}[0]{\hat{\Bz}}
\newcommand{\thetacap}[0]{\hat{\Btheta}}

%
% to write R^n and C^n in a distinguishable fashion.  Perhaps change this
% to the double lined characters upon figuring out how to do so.
%
\newcommand{\C}[1]{$\mathbb{C}^{#1}$}
\newcommand{\R}[1]{$\mathbb{R}^{#1}$}

%
% various generally useful helpers
%

% derivative of #1 wrt. #2:
\newcommand{\D}[2] {\frac {d#2} {d#1}}

\newcommand{\inv}[1]{\frac{1}{#1}}
\newcommand{\cross}[0]{\times}

\newcommand{\abs}[1]{\lvert{#1}\rvert}
\newcommand{\norm}[1]{\lVert{#1}\rVert}
\newcommand{\innerprod}[2]{\langle{#1}, {#2}\rangle}
\newcommand{\dotprod}[2]{{#1} \cdot {#2}}
\newcommand{\bdotprod}[2]{\left({#1} \cdot {#2}\right)}
\newcommand{\crossprod}[2]{{#1} \cross {#2}}
\newcommand{\tripleprod}[3]{\dotprod{\left(\crossprod{#1}{#2}\right)}{#3}}

\DeclareMathOperator{\Proj}{Proj}
\DeclareMathOperator{\Span}{span}
\DeclareMathOperator{\Sgn}{sgn}
\DeclareMathOperator{\Area}{Area}
\DeclareMathOperator{\Volume}{Volume}

%
% A few miscellaneous things specific to this document
%
\newcommand{\crossop}[1]{\crossprod{#1}{}}

% R2 vector.
\newcommand{\VectorTwo}[2]{
\begin{bmatrix}
 {#1} \\
 {#2}
\end{bmatrix}
}

\newcommand{\VectorN}[1]{
\begin{bmatrix}
{#1}_1 \\
{#1}_2 \\
\vdots \\
{#1}_N \\
\end{bmatrix}
}

\newcommand{\DETuvij}[4]{
\begin{vmatrix}
 {#1}_{#3} & {#1}_{#4} \\
 {#2}_{#3} & {#2}_{#4}
\end{vmatrix}
}

\newcommand{\DETuvwijk}[6]{
\begin{vmatrix}
 {#1}_{#4} & {#1}_{#5} & {#1}_{#6} \\
 {#2}_{#4} & {#2}_{#5} & {#2}_{#6} \\
 {#3}_{#4} & {#3}_{#5} & {#3}_{#6}
\end{vmatrix}
}

\newcommand{\DETuvwxijkl}[8]{
\begin{vmatrix}
 {#1}_{#5} & {#1}_{#6} & {#1}_{#7} & {#1}_{#8} \\
 {#2}_{#5} & {#2}_{#6} & {#2}_{#7} & {#2}_{#8} \\
 {#3}_{#5} & {#3}_{#6} & {#3}_{#7} & {#3}_{#8} \\
 {#4}_{#5} & {#4}_{#6} & {#4}_{#7} & {#4}_{#8} \\
\end{vmatrix}
}

%\newcommand{\DETuvwxyijklm}[10]{
%\begin{vmatrix}
% {#1}_{#6} & {#1}_{#7} & {#1}_{#8} & {#1}_{#9} & {#1}_{#10} \\
% {#2}_{#6} & {#2}_{#7} & {#2}_{#8} & {#2}_{#9} & {#2}_{#10} \\
% {#3}_{#6} & {#3}_{#7} & {#3}_{#8} & {#3}_{#9} & {#3}_{#10} \\
% {#4}_{#6} & {#4}_{#7} & {#4}_{#8} & {#4}_{#9} & {#4}_{#10} \\
% {#5}_{#6} & {#5}_{#7} & {#5}_{#8} & {#5}_{#9} & {#5}_{#10}
%\end{vmatrix}
%}

% R3 vector.
\newcommand{\VectorThree}[3]{
\begin{bmatrix}
 {#1} \\
 {#2} \\
 {#3}
\end{bmatrix}
}



% No idea what this is:
%%%\catcode`\@=11
%%%\newcommand{\@versim}[2]
%%%  {\lower3.1truept\vbox{\baselineskip0pt\lineskip0.5truept
%%%\ialign{$\m@th#1\hfil##\hfil$\crcr#2\crcr\sim\crcr}}}
%%%\catcode`\@=12
%%%% This is an extra line so that table is not too long.
%%%\setlength{\tabcolsep}{2.8pt}

\newcommand{\chapter}[1]{\title{#1}}

\newcommand{\blogpage}[1]{}
\newcommand{\revisionInfo}[1]{}

\newcommand{\beginArtWithToc}[0]{\begin{document}}
\newcommand{\beginArtNoToc}[0]{\begin{document}}

\newcommand{\EndNoBibArticle}[0]{\end{document}}
\newcommand{\EndArticle}[0]{\bibliography{myrefs}\bibliographystyle{unsrtnat}\end{document}}


\label{chap:grahamSchmidtLorentz}
\blogpage{http://sites.google.com/site/peeterjoot/math2011/grahamSchmidtLorentz.pdf}
%\date{April 14, 2011}
\revisionInfo{grahamSchmidtLorentz.tex}

%\journalinfo{ApJ Letters, in press}
%\submitted{}
%\received{June 16, 2010} \accepted{October 20, 2010}
%\shorttitle{short title goes here.}

% these two need generalization.  have to switch order with journal format.
\beginArtNoToc
\chapter{Change of basis and Gram-Schmidt orthonormalization as pedagogical tools for special relativity}

\author{Peeter Joot \altaffilmark{1}}
\altaffiltext{1}{peeter.joot@utoronto.ca}

\begin{abstract}

The standard language for the teaching and study of special relativity is that of tensor algebra.
While an explicit basis is common in the study of Euclidean spaces, it is usually implied in the study of inertial relativistic systems.
For the study of inertial relativistic systems it will be shown that introducing an explicit basis opens up some of the linear algebra toolbox, and has some distinct conceptual advantages.
The concept of reciprocal basis and the vector dual are introduced, allowing concepts from non-orthonormal Euclidean geometry to be applied to relativisitic study.
As with matrix representations of four vectors, the use of an explicit basis allows the details of the coordinate representation to be suppressed, allowing four vectors to be manipulated as complete entities in an inner products space.
It will be shown that Lorentz transformations can be viewed as change of basis operations, and example calculations of Lorentz transform matrices using the Gram-Schmidt orthonormalization algorithm will be provided.

\end{abstract}

\keywords{Lorentz boost, change of basis, special relativity, inertial frame, reciprocal basis, dual vector, Gram-Schmidt orthonormalization.}

\section{Abstract}

%\section{Introduction}

\section{Preliminaries, notation, and definitions}

\subsection{Four vectors, and the standard basis}

Four vectors will be written as a tuples of time and space coordinates.  These will be represented herein as non-bold letters of the form

\begin{equation}\label{eqn:grahamSchmidtLorentz:10}
x = (c t, \Bx) = ( c t, x, y, z ) = ( x^0, x^1, x^2, x^3 ),
\end{equation}

with bold letters will reserved for Euclidean vectors.  As usual, the factor of $c$ in the time coordinate is included so that the units of any of the coordinates in the tuple have dimensions of distance.  With

\begin{equation}\label{eqn:grahamSchmidtLorentz:70}
\begin{aligned}
e_0 &= (1, 0, 0, 0) \\
e_1 &= (0, 1, 0, 0) \\
e_2 &= (0, 0, 1, 0) \\
e_3 &= (0, 0, 0, 1)
\end{aligned}
\end{equation}

the ordered set $\{e_0, e_1, e_2, e_3\}$ will be referred to as the standard basis.  Upper indexes will be used for the coordinates of the four vector in the standard basis (so $x^2$ is the 2-indexed coordinate of the four vector and not the square of $x$).  Lower indexed four vector coordinates will be introduced later once the reciprocal basis is introduced.

Repeated mixed upper and lower indexes will imply summation, with Greek indexes used for temporal and spatial indexes $\{0, 1, 2, 3\}$, and Latin indexes used in a Euclidean context $\{1, 2, \cdots N\}$.

\subsection{Relativistic inner product}

At the heart of special relativity is the definition of the invariant length or interval, which defines a distance like measure between any a pair of vectors, relating both time and space coordinates.  This invariance may be codified by defining an inner product for the spacetime vector space of the form

\begin{equation}\label{eqn:grahamSchmidtLorentz:50}
x \cdot y = \pm( x^0 y^0 - \Bx \cdot \By ).
\end{equation}

Here $y = (y^0, \By)$.  This inner product is non-positive definite, with opposite signs required for the spatial and temporal portions of the product.
No attempt to motivate why such a mixed sign for the time and space coordinates will be made here.  That more difficult job is deferred to any number of books covering special relativity
(e.g. \citep{landau1980classical}.)
%\citep[e.g.][]{landau1980classical}.
The overall sign is arbitrary and conventions vary by author.  A positive sign will be used herein.

The use of a non-orthonormal basis, even in Euclidean spaces, makes life a bit more difficult.  There is, however, no choice in the matter for special relativity, since the standard basis elements \ref{eqn:grahamSchmidtLorentz:70} are unity only up to a sign.  With the overall sign of the inner product \ref{eqn:grahamSchmidtLorentz:50} chosen to be positive

\begin{equation}\label{eqn:grahamSchmidtLorentz:860}
e_1 \cdot e_1 = e_2 \cdot e_2 = e_3 \cdot e_3 = -( e_0 \cdot e_0 ) = -1.
\end{equation}

A relativistic basis cannot be constructed for which all the basis vectors have strictly unit norm.  Unit vector will be used here loosely to refer to any vector $u$ such that $u \cdot u = \pm 1$.

\subsection{Reciprocal basis, duality, and coordinate representation with a non-orthonormal basis}

It is convenient to introduce the concept of a reciprocal basis when dealing with non-orthonormal spaces.  The utility of a reciprocal basis is not limited to the non-Euclidean vector space of special relativity.  The reciprocal basis elements are defined implicitly such that

\begin{equation}\label{eqn:grahamSchmidtLorentz:130}
e_\alpha \cdot e^\beta = {\delta_\alpha}^\beta.
\end{equation}

The vector $e^\alpha$ is referred to as the dual of $e_\alpha$, and the ordered set of vectors $\{e^\alpha\}$ is called the reciprocal basis $\{e_\alpha\}$.

Given a coordinate representation

\begin{equation}\label{eqn:grahamSchmidtLorentz:880}
x = x^\alpha e_\alpha,
\end{equation}

the coordinates may be extracted by taking dot products with the reciprocal basis elements

\begin{equation}\label{eqn:grahamSchmidtLorentz:340}
x \cdot e^\alpha = (x^\beta e_\beta) \cdot e^\alpha = x^\beta {\delta_\beta}^\alpha = x^\alpha.
\end{equation}

Similarily, for the same vector $x$ represented in the reciprocal basis

\begin{equation}\label{eqn:grahamSchmidtLorentz:900}
x = x_\alpha e^\alpha,
\end{equation}

the coordinates may be extracted by taking dot products with $e_\alpha$

\begin{equation}\label{eqn:grahamSchmidtLorentz:340b}
x \cdot e_\alpha = (x_\beta e^\beta) \cdot e_\alpha = x_\beta {\delta^\beta}_\alpha = x_\alpha.
\end{equation}

In this context there is nothing special about either upper or lower indexes.  They are just coordinates with respect to a basis and its reciprocal basis, respectively.  When the original basis happens to be orthonormal, there is equality between the basis vectors and their reciprocal duals $e_\alpha = e^\alpha$, as well as between the coordinates calculated from those bases respectively $x_\alpha = x^\alpha$.  In tensor algebra, upper indexes are ``special'' since the invariant transformations are defined in terms of those coordinates, but that is really just a choice of basis.  It is in fact possible \citep{doran2003gap} to express Lorentz transformations in a completely coordinate free fashion, where there is freedom to employ upper or lower index representation of the coordinates, or coordinates with respect to any basis, even one that is not normal.

\subsubsection{Reciprocal basis example in 2D Euclidean space}

The calculation of the reciprocal basis elements may be dependent on the complete set of elements in the non-dual basis.  This can be illustrated nicely by considering an example of an oblique basis in a Euclidean space.

Given a basis $A = \{e_1, e_2\}$ in a 2 dimensional Euclidean space where

\begin{equation}\label{eqn:grahamSchmidtLorentz:90}
e_1 =
\begin{bmatrix}
1 \\
1
\end{bmatrix},\qquad
e_2 =
\begin{bmatrix}
1 \\
2
\end{bmatrix},
\end{equation}

In this column vector representation the duality relation \ref{eqn:grahamSchmidtLorentz:130} takes the form

\begin{equation}\label{eqn:grahamSchmidtLorentz:150}
\begin{bmatrix}
{e_1}^\T \\
{e_2}^T
\end{bmatrix}
\begin{bmatrix}
e^1 & e^2
\end{bmatrix} = I.
\end{equation}

Inversion provides the dual vectors
\begin{equation}\label{eqn:grahamSchmidtLorentz:170}
\begin{bmatrix}
e^1 & e^2
\end{bmatrix}
=
{
\begin{bmatrix}
{e_1}^\T \\
{e_2}^T
\end{bmatrix}
}^{-1}
=
{\begin{bmatrix}
1 & 1 \\
1 & 2
\end{bmatrix}}^{-1}
=
\begin{bmatrix}
2 & -1 \\
-1 & 1
\end{bmatrix},
\end{equation}

or

\begin{equation}\label{eqn:grahamSchmidtLorentz:190}
e^1 =
\begin{bmatrix}
2 \\
-1
\end{bmatrix}, \qquad
e^2 =
\begin{bmatrix}
-1 \\
1
\end{bmatrix}.
\end{equation}

The problem of solving for the coordinates $a,b$ of a vector $x = a e_1 + b e_2$ in this oblique basis now reduces to taking dot products

\begin{align}\label{eqn:grahamSchmidtLorentz:920}
x \cdot e^1 &= a e_1 \cdot e^1 + b \cancel{e_2 \cdot e^1} = a \\
x \cdot e^2 &= a \cancel{e_1 \cdot e^2} + b e_2 \cdot e^2 = b.
\end{align}

As a concrete example consider

\begin{equation}\label{eqn:grahamSchmidtLorentz:230}
\begin{aligned}
x &=
\begin{bmatrix}
4 \\
2
\end{bmatrix}  \\
&=
\left(
\begin{bmatrix}
4 \\
2
\end{bmatrix}
\cdot e^1
\right)
e_1
+
\left(
\begin{bmatrix}
4 \\
2
\end{bmatrix}
\cdot e^2
\right)
e_2 \\
&= 6
\begin{bmatrix}
1 \\
1
\end{bmatrix}
+
-2
\begin{bmatrix}
1 \\
2
\end{bmatrix}
\end{aligned}
\end{equation}

Coordinates may also be computed with respect to the reciprocal basis.  With $x = c e^1 + d e^2$, dotting with $e_1$ and $e_2$ respectively provides these

\begin{equation}\label{eqn:grahamSchmidtLorentz:360}
\begin{aligned}
x \cdot e_1 &= c e^1 \cdot e_1 + d \cancel{e^2 \cdot e_1} = c \\
x \cdot e_2 &= c \cancel{e^1 \cdot e_2} + d e^2 \cdot e_2 = d
\end{aligned}
\end{equation}

Again considering the concrete example above

\begin{equation}\label{eqn:grahamSchmidtLorentz:290}
\begin{aligned}
x
&=
\begin{bmatrix}
4 \\
2
\end{bmatrix} \\
&=
\left(
\begin{bmatrix}
4 \\
2
\end{bmatrix}
\cdot e_1
\right)
e^1
+
\left(
\begin{bmatrix}
4 \\
2
\end{bmatrix}
\cdot e_2
\right)
e^2 \\
&= 6
\begin{bmatrix}
2 \\
-1
\end{bmatrix}
+
8
\begin{bmatrix}
-1 \\
1
\end{bmatrix}
\end{aligned}
\end{equation}

This pair of coordinate calculations is depicted in figure (\ref{fig:obliqueReciprocal}).

\begin{figure}[htp]
\centering
\includegraphics[totalheight=0.3\textheight]{obliqueReciprocal}
\caption{Vector projections in oblique and reciprocal frames.}\label{fig:obliqueReciprocal}
\end{figure}

Observe that the projections onto the basis elements are

\begin{align}\label{eqn:grahamSchmidtLorentz:n}
\Proj_{e_i}(x) &= (x \cdot e^i) e_i \\
\Proj_{e^i}(x) &= (x \cdot e_i) e^i
\end{align}

(no sum,) and not 

\begin{equation}\label{eqn:grahamSchmidtLorentz:n}
\Proj_{e_i}(x) = \Proj_{e^i}(x) = \frac{x \cdot e_i}{e_i \cdot e_i} e_i,
\end{equation}

as is the case for a orthonormal basis.

\subsubsection{Projections}

Examples of projections onto the Euclidean non-orthonormal basis above have been seen.  In general the relations \ref{eqn:grahamSchmidtLorentz:340}, and \ref{eqn:grahamSchmidtLorentz:340b} allow for a such Fourier decomposition of a vector into components in each of the respective basis directions

\begin{equation}\label{eqn:grahamSchmidtLorentz:330}
\begin{aligned}
x &= x^\alpha e_\alpha = (x \cdot e^\alpha) e_\alpha \\
  &= x_\alpha e^\alpha = (x \cdot e_\alpha) e^\alpha.
\end{aligned}
\end{equation}

With \ref{eqn:grahamSchmidtLorentz:330} containing $x$ on both the LHS and in the RHS as $(x \cdot e^\alpha) e_\alpha$, this relation has an appearance of being somewhat recursive.  This is however, an important property, since each of the RHS terms represents a projection.  The projection of a vector onto the basis element $e_\alpha$ is

\begin{equation}\label{eqn:grahamSchmidtLorentz:560}
\Proj_{e_\alpha}(x) = (x \cdot e^\alpha) e_\alpha,
\end{equation}

(no sum implied.)  
This will be important since the Gram-Schmidt procedure is essentially just the repeated subtraction of projections, and knowledge of how to express projections for a non-orthonormal basis is required.

\subsubsection{Gram-Schmidt procedure generalized to non-orthonormal bases}

Aside for some additional care required to express projections, the Gram-Schmidt procedure is the same as in Euclidean space.
Given a set of mutually normal unit vectors $\{f_0, \cdots, f_\alpha\}$, the set may be extended by an additional normal vector.  Provided a vector $a$ lying outside of the span of this set can be found, subtraction of the projections of $a$ from the all the elements of this set leaves only the component of $a$ normal to all vectors in this set.  That is

\begin{equation}\label{eqn:grahamSchmidtLorentz:660}
\begin{aligned}
b
&= a - \sum_{\sigma \le \alpha} \Proj_{f_\sigma}(a) \\
&= a - \sum_{\sigma \le \alpha} (a \cdot f_\sigma) f^\sigma
\end{aligned}
\end{equation}

Normalization $f_{\alpha+1} = b/\sqrt{\Abs{b \cdot b}}$ allows the set to be extended by an additional unit vector.  This process can be repeated until a complete basis is formed.

\subsubsection{Reciprocal basis for relativity}

Because it is not possible to have a strictly orthonormal basis in a relativistic context, the reciprocal basis necessarily has a place in the geometry of relativity.
It is easily verified that the vectors

\begin{equation}\label{eqn:grahamSchmidtLorentz:70b}
\begin{aligned}
e^0 &= (1, 0, 0, 0) \\
e^1 &= (0, -1, 0, 0) \\
e^2 &= (0, 0, -1, 0) \\
e^3 &= (0, 0, 0, -1)
\end{aligned}
\end{equation}

are dual to the standard basis elements \ref{eqn:grahamSchmidtLorentz:70} according to the definition \ref{eqn:grahamSchmidtLorentz:130}.

\subsubsection{Metric tensors}

Upper and lower index coordinates with respect to any basis and its reciprocal, orthonormal or not, are related by dot products of the basis elements, and are not independent.  Given a vector with both upper and lower index representation

\begin{equation}\label{eqn:grahamSchmidtLorentz:n}
x = x^\alpha e_\alpha = x_\beta e^\beta,
\end{equation}

taking dot products with $e_\mu$, and $e^\mu$ provides the relationships

\begin{align}\label{eqn:grahamSchmidtLorentz:n}
x_\mu &= x \cdot e_\mu = (e_\mu \cdot e_\alpha) x_\alpha \\
x^\mu &= x \cdot e^\mu = (e^\mu \cdot e^\beta) x_\beta.
\end{align}

These pairs of dot products define the metric tensors for the pair of bases

\begin{align}\label{eqn:grahamSchmidtLorentz:n}
g_{\mu \nu} &\equiv e_\mu \cdot e_\nu \\
g^{\mu \nu} &\equiv e^\mu \cdot e^\nu,
\end{align}

with which the index lowering and raising operations take their standard tensor form

\begin{align}\label{eqn:grahamSchmidtLorentz:n}
x_\mu &= g_{\mu \nu} x^\nu \\
x^\mu &= g^{\mu \nu} x_\nu.
\end{align}

The metric tensors are basis dependent and not generally diagonal or identical.

Observe that the invariant (squared) length of a vector has the expected form and follows from the definition of the coordinates in terms of the respective bases

\begin{equation}\label{eqn:grahamSchmidtLorentz:n}
x \cdot x = (x^\alpha e_\alpha) \cdot (x_\beta e^\beta) = x^\alpha x_\alpha.
\end{equation}

\subsubsection{Change of basis}

Consider a vector with coordinate representations in a pair of bases, not neccessarily orthonormal

\begin{equation}\label{eqn:grahamSchmidtLorentz:n}
x = y^\alpha f_\alpha = x^\beta e_\beta.
\end{equation}

Taking dot products with reciprocal frame elements relates the coordinates

\begin{align}\label{eqn:grahamSchmidtLorentz:n}
y^\mu &= (f^\mu \cdot e_\nu ) x^\nu \\
x^\mu &= (e^\mu \cdot f_\nu) y^\nu.
\end{align}

This can be expressed more conventionally in tensor form

\begin{align}\label{eqn:grahamSchmidtLorentz:n}
{\wedge^\mu}_\nu &= f^\mu \cdot e_\nu \\
({\wedge^\mu}_\nu)^{-1} &= e^\mu \cdot f_\nu = g^{\alpha \mu} g_{\beta \nu} {\wedge^\beta}_\alpha
\end{align}

so that the coordinates are related by

\begin{align}\label{eqn:grahamSchmidtLorentz:n}
y^\mu &= {\wedge^\mu}_\nu x^\nu \\
x^\mu &= ({\wedge^\mu}_\nu)^{-1} y^\nu.
\end{align}

It is natural to consider the coordinates $y^\alpha$ after transformation as the same vector that had the coordinates $x^\alpha$.  They are just representations under different bases, and unsurprisingly must then have the same invariant length.

%\subsubsection{Relation to tensor form}
%
%FIXME: now out of sequence.
%While only one and two spatial direction examples were considered, the generalization to three spatial directions, and expression in tensor form follows trivially
%
%
%or
%\begin{align}\label{eqn:grahamSchmidtLorentz:780}
%y^\alpha &=
%{\wedge^\alpha}_\beta
%x^\beta \\
%{\wedge^\alpha}_\beta
%&=
%f^\alpha \cdot e_\beta.
%\end{align}
%
%
%FIXME:we
%We can see this from the tensor expressions as well, once \ref{eqn:grahamSchmidtLorentz:780} is expressed in terms of the basis vectors
%
%\begin{align*}
%y^\alpha
%&=
%{\wedge^\alpha}_\beta
%x^\beta \\
%&= x^\beta
%(f^\alpha \cdot e_\beta)
%\\
%y \cdot f^\alpha &= \\
%&=(x \cdot e^\beta)
%(f^\alpha \cdot e_\beta)
%\\
%\implies
%(y \cdot f^\alpha) f_\alpha &= (x \cdot e^\beta)
%(f^\alpha \cdot e_\beta)
%f_\alpha \\
%y
%&= (x \cdot e^\beta) e_\beta \\
%&= x
%\end{align*}
%

\section{Relativity}
\subsection{Proper separation}

FIXME: one
Given a particle parametrization along a trajectory $x(\lambda)$ in spacetime, one can compute the average spacetime length between a pair of points on this path

\begin{equation}\label{eqn:grahamSchmidtLorentz:400}
s_b - s_a = \int_{\lambda = a}^b \sqrt{ \frac{d x(\lambda)}{d\lambda} \cdot \frac{d x(\lambda)}{d\lambda} } d\lambda.
\end{equation}

%For infinitesimally separated points on the tragectory, we have
%
%\begin{equation}\label{eqn:grahamSchmidtLorentz:420}
%ds^2 = d\lambda^2 \left( \frac{d x(\lambda)}{d\lambda} \cdot \frac{d x(\lambda)} \right)^2.
%\end{equation}
%
%Being a bit loose with our differentials this is
%
%\begin{align}\label{eqn:grahamSchmidtLorentz:420b}
%\left( \frac{ds}{d\lambda} )^2 &= \left( \frac{d x(\lambda)}{d\lambda} \cdot \frac{d x(\lambda) \right)^2 \\
%\left( \frac{d\lambda}{ds} )^2 &= 1/\left( \frac{d x(\lambda)}{d\lambda} \cdot \frac{d x(\lambda) \right)^2
%\end{align}

FIXME: one
A trajectory $x(\lambda)$ may be reparametrized in terms of the instantaneous proper separation $s$.  Doing so, one finds that the first derivative of $x(s)$ is a timelike unit vector (squares to $1$, not $-1$ like $e_0 = (1, 0, 0, 0)$) along any point of the curve.

This is nicely demonstrated by example.
%  That is
%
%\begin{align*}
%\left( \frac{dx}{ds} \right)^2
%&=
%\left( \frac{dx}{d\lambda} \frac{d\lambda}{ds} \right)^2  \\
%&=
%\left( \frac{dx}{d\lambda} \right)^2 \left( \frac{d\lambda}{ds} \right)^2  \\
%&= 1
%\end{align*}
%
Consider an inertial system, with a particle moving along a constant velocity trajectory, parametrized by an external observers time $t$

\begin{equation}\label{eqn:grahamSchmidtLorentz:440}
x(t) = (c t, \Bv t)
\end{equation}

FIXME: our
Our proper separation anywhere along this spacetime curve is

\begin{equation}\label{eqn:grahamSchmidtLorentz:460}
\begin{aligned}
s
&= \int \sqrt{(c t, \Bv t)^2} dt  \\
&= \int \sqrt{ c^2 - \Bv^2 } dt  \\
&= \sqrt{ c^2 - \Bv^2 } t
\end{aligned}
\end{equation}

FIXME: our
Our proper length reparametrization of this path is thus

\begin{equation}\label{eqn:grahamSchmidtLorentz:480}
x(s) = \inv{\sqrt{c^2 - \Bv^2}} (c, \Bv) s.
\end{equation}

FIXME: our
Our first derivative

\begin{equation}\label{eqn:grahamSchmidtLorentz:500}
\frac{dx}{ds} = \inv{\sqrt{c^2 - \Bv^2}} (c, \Bv)
\end{equation}

is now easily observed to be of unit length

\begin{equation}\label{eqn:grahamSchmidtLorentz:520}
\left( \frac{dx}{ds} \right)^2 = \inv{c^2 - \Bv^2} (c^2 - \Bv^2) = 1.
\end{equation}

FIXME:we
For an arbitrary reparametrization $x(\mu(\lambda))$ we can utilize that parametrization to decompose a trajectory into tangential and perpendicular components

\begin{equation}\label{eqn:grahamSchmidtLorentz:540}
x(\mu) = \frac{dx}{d\mu} \mu + \left( x - \frac{dx}{d\mu} \mu \right).
\end{equation}

FIXME:we
Observe that for an inertial system, this implies that we have only a timelike component for the trajectory, when parametrized by proper length.  That is

\begin{equation}\label{eqn:grahamSchmidtLorentz:580}
x(s) = \frac{dx}{ds} s.
\end{equation}

FIXME:we
This means that a proper length parametrization corresponds to the time in the frame for which the particle is instantaneously at rest.  A student of special relativity is familiar with being able to switch to a frame in which the particle is instantaneously at rest by performing a Lorentz boost, so we have indirectly obtained a first hint that we can interpret such a transformation as nothing more than a change of basis.

\subsection{Lorentz boost as a change of basis}

FIXME:we
For the trajectory \ref{eqn:grahamSchmidtLorentz:440} we found that the proper length derivative was a timelike unit vector for the frame in which the particle was at rest.  That unit vector can be used as part of a basis for that rest frame.  Once this basis is completed with its spatial unit vectors, we will see how the particle's rest basis and an observer basis are related by Lorentz transformation.

\subsubsection{Illustration by example.  One spatial dimension}

For simplicity, consider a two dimensional spacetime vector space, with a particle trajectory in an inertial frame parametrized by its proper length

\begin{equation}\label{eqn:grahamSchmidtLorentz:600}
x(s) = \gamma (1, \beta) s.
\end{equation}

FIXME:we
Let's label the time like unit vector in the particle's rest frame $f_0$, so that we have

\begin{equation}\label{eqn:grahamSchmidtLorentz:620}
f_0 = \frac{dx}{ds} = \gamma (1, \beta) = f^0.
\end{equation}

The particle's trajectory in the rest frame, in terms of the basis to be determined is thus

\begin{equation}\label{eqn:grahamSchmidtLorentz:640}
x(s) = s f_0 + 0 f_1.
\end{equation}

The task is to compute this basis $\{f_0, f_1\}$ for the particle's rest frame, and this can be done using the Gram-Schmidt procedure.

FIXME:we
For this one dimensional spatial example, we can pick any vector out of the span of $f_0$ and find a vector normal to that.  Picking rather arbitrarily, let's use $e_1 = (0, 1)$,

\begin{align*}
b
&= e_1 - (e_1 \cdot f_0) f^0 \\
&= (0, 1) - (0, 1) \cdot (1, \beta) \gamma^2 (1, \beta) \\
&= (0, 1) + \beta \gamma^2 (1, \beta) \\
&= (\gamma^2 \beta, 1 + \beta^2 \gamma^2) \\
&= \gamma^2 (\beta, 1 ).
\end{align*}

FIXME: our
This can be normalized as $f_1 = \pm \gamma(\beta, 1)$.  To determine the sign, let's require that the particle's rest frame basis has the same orientation as the standard basis.  This is a requirement that the determinants of the coordinates of the unit vectors match.  Our standard basis in the sequence $e_0, e_1$ can be defined as positively oriented

\begin{equation}\label{eqn:grahamSchmidtLorentz:680}
\begin{vmatrix}
e_0 \\
e_1
\end{vmatrix}
=
\begin{vmatrix}
1 & 0 \\
0 & 1
\end{vmatrix}
= 1.
\end{equation}

FIXME:we
For this new basis, a basis for which $f_0$ is the timelike unit vector for the particle in its rest frame, we have

\begin{equation}\label{eqn:grahamSchmidtLorentz:685}
\begin{vmatrix}
f_0 \\
f_1
\end{vmatrix}
=
\begin{vmatrix}
\gamma & \gamma \beta \\
\pm \gamma \beta & \pm \gamma
\end{vmatrix}
=
\pm \gamma^2 ( 1 - \beta^2)
= \pm 1.
\end{equation}

FIXME:we
Therefore, we want the positive sign, $f_0 = \gamma (\beta, 1)$.

Using the standard basis as the observer frame, let's figure out how to relate these to the coordinates of a vector in the particles rest frame.  With

\begin{equation}\label{eqn:grahamSchmidtLorentz:700}
x = x^0 e_0 + x^1 e_1 = y^0 f_0 + y^1 f_1,
\end{equation}

taking dot products with the dual vectors relates the coordinates

\begin{equation}\label{eqn:grahamSchmidtLorentz:720}
\begin{aligned}
\begin{bmatrix}
y^0 \\
y^1
\end{bmatrix}
&=
\begin{bmatrix}
x \cdot f^0 \\
x \cdot f^1 \\
\end{bmatrix} \\
&=
\begin{bmatrix}
f^0 \cdot e_0
x^0
+
f^0 \cdot e_1
x^1
\\
f^1 \cdot e_0
x^0
+
f^1 \cdot e_1
x^1
\end{bmatrix} \\
&=
\begin{bmatrix}
f^0 \cdot e_0
&
f^0 \cdot e_1
\\
f^1 \cdot e_0
 &
f^1 \cdot e_1
\end{bmatrix}
\begin{bmatrix}
x^0 \\
x^1
\end{bmatrix}
\end{aligned}
\end{equation}

FIXME:we
FIXME: our
For our example, we have

\begin{equation}\label{eqn:grahamSchmidtLorentz:740}
\begin{aligned}
&\begin{bmatrix}
f^0 \cdot e_0 & f^0 \cdot e_1 \\
f^1 \cdot e_0 & f^1 \cdot e_1
\end{bmatrix} \\
&=
\begin{bmatrix}
( 1, \beta )
\cdot
(1, 0)
\gamma &
( 1, \beta )
\cdot
(0, 1)
\gamma \\
( -\beta, -1)
\cdot
(1, 0)
\gamma &
( -\beta, -1)
\cdot
(0, 1)
\gamma
\end{bmatrix} \\
&=
\gamma
\begin{bmatrix}
1 & - \beta \\
- \beta & 1
\end{bmatrix}
\end{aligned}
\end{equation}

FIXME:we
We ``derive'' the Lorentz boost matrix, considering nothing more than the geometry of the vector space, and how to represent a vector in two different bases.

\subsubsection{On uniqueness}

A Lorentz transformation in any number of spatial dimensions can be calculated in this fashion, however, some care is required since this it is not unique.  In particular the matrix of the change of basis transformation is not necessarily a Lorentz boost.  As an example, in two spatial dimensions for a trajectory

\begin{equation}\label{eqn:grahamSchmidtLorentz:800}
x(s) = \gamma s ( 1, \beta \cos\theta, \beta \sin\theta)
\end{equation}

FIXME:we
FIXME: our
Our timelike unit vector is $f_0 = ( 1, \beta \cos\theta, \beta \sin\theta )$, and we can easily compute two additional normal spatial vectors using these methods

\begin{align}\label{eqn:grahamSchmidtLorentz:820}
f_1 &= \gamma ( \beta, \cos\theta, \sin\theta ) \\
f_2 &= \gamma ( 0, -\sin\theta, \cos\theta )
\end{align}

The matrix of this linear transformation is

\begin{equation}\label{eqn:grahamSchmidtLorentz:840}
%{\wedge^\alpha}_\beta =
\begin{bmatrix}
\gamma & - \gamma \beta \cos\theta & - \gamma \beta \sin\theta \\
-\gamma \beta & \gamma \cos\theta & \gamma \sin\theta \\
0 & -\sin\theta & \cos\theta
\end{bmatrix}.
\end{equation}

While this has unit determinant, and necessarily preserves the invariant length of a vector, it does not have the symmetric form of the boost associated with the spatial velocity $c \beta (\cos\theta, \sin\theta)$, so this must be interpreted as a composition of boost and rotation.

%\section{Discussion}
%\acknowledgments

\section{Conclusion}

Some of the concepts used herein, especially that of the reciprocal basis, have been borrowed from the context of Geometric Algebra
\citep{doran2003gap}
, where coordinate free methods are developed in considerably more depth and generality.

Without the learning curve of attempting a study of Geometric Algebra, an attempt has been made to illustrate some of the conceptual advantages of including the basis in treatments of special relativity.  It is hoped that this also shows how some of the ideas and tools of Euclidean vector algebra can be applied to the study of special relativity.

\EndArticle
