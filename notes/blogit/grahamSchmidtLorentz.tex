%%
% Copyright � 2015 Peeter Joot.  All Rights Reserved.
% Licenced as described in the file LICENSE under the root directory of this GIT repository.
%
\documentclass[]{eliblog}

\usepackage{amsmath}
\usepackage{mathpazo}

%
% shorthand for bold symbols, convenient for vectors and matrices
%
\newcommand{\Ba}[0]{\mathbf{a}}
\newcommand{\Bb}[0]{\mathbf{b}}
\newcommand{\Bc}[0]{\mathbf{c}}
\newcommand{\Bd}[0]{\mathbf{d}}
\newcommand{\Be}[0]{\mathbf{e}}
\newcommand{\Bf}[0]{\mathbf{f}}
\newcommand{\Bg}[0]{\mathbf{g}}
\newcommand{\Bh}[0]{\mathbf{h}}
\newcommand{\Bi}[0]{\mathbf{i}}
\newcommand{\Bj}[0]{\mathbf{j}}
\newcommand{\Bk}[0]{\mathbf{k}}
\newcommand{\Bl}[0]{\mathbf{l}}
\newcommand{\Bm}[0]{\mathbf{m}}
\newcommand{\Bn}[0]{\mathbf{n}}
\newcommand{\Bo}[0]{\mathbf{o}}
\newcommand{\Bp}[0]{\mathbf{p}}
\newcommand{\Bq}[0]{\mathbf{q}}
\newcommand{\Br}[0]{\mathbf{r}}
\newcommand{\Bs}[0]{\mathbf{s}}
\newcommand{\Bt}[0]{\mathbf{t}}
\newcommand{\Bu}[0]{\mathbf{u}}
\newcommand{\Bv}[0]{\mathbf{v}}
\newcommand{\Bw}[0]{\mathbf{w}}
\newcommand{\Bx}[0]{\mathbf{x}}
\newcommand{\By}[0]{\mathbf{y}}
\newcommand{\Bz}[0]{\mathbf{z}}
\newcommand{\BA}[0]{\mathbf{A}}
\newcommand{\BB}[0]{\mathbf{B}}
\newcommand{\BC}[0]{\mathbf{C}}
\newcommand{\BD}[0]{\mathbf{D}}
\newcommand{\BE}[0]{\mathbf{E}}
\newcommand{\BF}[0]{\mathbf{F}}
\newcommand{\BG}[0]{\mathbf{G}}
\newcommand{\BH}[0]{\mathbf{H}}
\newcommand{\BI}[0]{\mathbf{I}}
\newcommand{\BJ}[0]{\mathbf{J}}
\newcommand{\BK}[0]{\mathbf{K}}
\newcommand{\BL}[0]{\mathbf{L}}
\newcommand{\BM}[0]{\mathbf{M}}
\newcommand{\BN}[0]{\mathbf{N}}
\newcommand{\BO}[0]{\mathbf{O}}
\newcommand{\BP}[0]{\mathbf{P}}
\newcommand{\BQ}[0]{\mathbf{Q}}
\newcommand{\BR}[0]{\mathbf{R}}
\newcommand{\BS}[0]{\mathbf{S}}
\newcommand{\BT}[0]{\mathbf{T}}
\newcommand{\BU}[0]{\mathbf{U}}
\newcommand{\BV}[0]{\mathbf{V}}
\newcommand{\BW}[0]{\mathbf{W}}
\newcommand{\BX}[0]{\mathbf{X}}
\newcommand{\BY}[0]{\mathbf{Y}}
\newcommand{\BZ}[0]{\mathbf{Z}}

\newcommand{\Bzero}[0]{\mathbf{0}}
\newcommand{\Btheta}[0]{\boldsymbol{\theta}}
\newcommand{\Btau}[0]{\boldsymbol{\tau}}
\newcommand{\Bomega}[0]{\boldsymbol{\omega}}

%
% shorthand for unit vectors
%
\newcommand{\acap}[0]{\hat{\Ba}}
\newcommand{\bcap}[0]{\hat{\Bb}}
\newcommand{\ccap}[0]{\hat{\Bc}}
\newcommand{\dcap}[0]{\hat{\Bd}}
\newcommand{\ecap}[0]{\hat{\Be}}
\newcommand{\fcap}[0]{\hat{\Bf}}
\newcommand{\gcap}[0]{\hat{\Bg}}
\newcommand{\hcap}[0]{\hat{\Bh}}
\newcommand{\icap}[0]{\hat{\Bi}}
\newcommand{\jcap}[0]{\hat{\Bj}}
\newcommand{\kcap}[0]{\hat{\Bk}}
\newcommand{\lcap}[0]{\hat{\Bl}}
\newcommand{\mcap}[0]{\hat{\Bm}}
\newcommand{\ncap}[0]{\hat{\Bn}}
\newcommand{\ocap}[0]{\hat{\Bo}}
\newcommand{\pcap}[0]{\hat{\Bp}}
\newcommand{\qcap}[0]{\hat{\Bq}}
\newcommand{\rcap}[0]{\hat{\Br}}
\newcommand{\scap}[0]{\hat{\Bs}}
\newcommand{\tcap}[0]{\hat{\Bt}}
\newcommand{\ucap}[0]{\hat{\Bu}}
\newcommand{\vcap}[0]{\hat{\Bv}}
\newcommand{\wcap}[0]{\hat{\Bw}}
\newcommand{\xcap}[0]{\hat{\Bx}}
\newcommand{\ycap}[0]{\hat{\By}}
\newcommand{\zcap}[0]{\hat{\Bz}}
\newcommand{\thetacap}[0]{\hat{\Btheta}}

%
% to write R^n and C^n in a distinguishable fashion.  Perhaps change this
% to the double lined characters upon figuring out how to do so.
%
\newcommand{\C}[1]{$\mathbb{C}^{#1}$}
\newcommand{\R}[1]{$\mathbb{R}^{#1}$}

%
% various generally useful helpers
%

% derivative of #1 wrt. #2:
\newcommand{\D}[2] {\frac {d#2} {d#1}}

\newcommand{\inv}[1]{\frac{1}{#1}}
\newcommand{\cross}[0]{\times}

\newcommand{\abs}[1]{\lvert{#1}\rvert}
\newcommand{\norm}[1]{\lVert{#1}\rVert}
\newcommand{\innerprod}[2]{\langle{#1}, {#2}\rangle}
\newcommand{\dotprod}[2]{{#1} \cdot {#2}}
\newcommand{\bdotprod}[2]{\left({#1} \cdot {#2}\right)}
\newcommand{\crossprod}[2]{{#1} \cross {#2}}
\newcommand{\tripleprod}[3]{\dotprod{\left(\crossprod{#1}{#2}\right)}{#3}}

\DeclareMathOperator{\Proj}{Proj}
\DeclareMathOperator{\Span}{span}
\DeclareMathOperator{\Sgn}{sgn}
\DeclareMathOperator{\Area}{Area}
\DeclareMathOperator{\Volume}{Volume}

%
% A few miscellaneous things specific to this document
%
\newcommand{\crossop}[1]{\crossprod{#1}{}}

% R2 vector.
\newcommand{\VectorTwo}[2]{
\begin{bmatrix}
 {#1} \\
 {#2}
\end{bmatrix}
}

\newcommand{\VectorN}[1]{
\begin{bmatrix}
{#1}_1 \\
{#1}_2 \\
\vdots \\
{#1}_N \\
\end{bmatrix}
}

\newcommand{\DETuvij}[4]{
\begin{vmatrix}
 {#1}_{#3} & {#1}_{#4} \\
 {#2}_{#3} & {#2}_{#4}
\end{vmatrix}
}

\newcommand{\DETuvwijk}[6]{
\begin{vmatrix}
 {#1}_{#4} & {#1}_{#5} & {#1}_{#6} \\
 {#2}_{#4} & {#2}_{#5} & {#2}_{#6} \\
 {#3}_{#4} & {#3}_{#5} & {#3}_{#6}
\end{vmatrix}
}

\newcommand{\DETuvwxijkl}[8]{
\begin{vmatrix}
 {#1}_{#5} & {#1}_{#6} & {#1}_{#7} & {#1}_{#8} \\
 {#2}_{#5} & {#2}_{#6} & {#2}_{#7} & {#2}_{#8} \\
 {#3}_{#5} & {#3}_{#6} & {#3}_{#7} & {#3}_{#8} \\
 {#4}_{#5} & {#4}_{#6} & {#4}_{#7} & {#4}_{#8} \\
\end{vmatrix}
}

%\newcommand{\DETuvwxyijklm}[10]{
%\begin{vmatrix}
% {#1}_{#6} & {#1}_{#7} & {#1}_{#8} & {#1}_{#9} & {#1}_{#10} \\
% {#2}_{#6} & {#2}_{#7} & {#2}_{#8} & {#2}_{#9} & {#2}_{#10} \\
% {#3}_{#6} & {#3}_{#7} & {#3}_{#8} & {#3}_{#9} & {#3}_{#10} \\
% {#4}_{#6} & {#4}_{#7} & {#4}_{#8} & {#4}_{#9} & {#4}_{#10} \\
% {#5}_{#6} & {#5}_{#7} & {#5}_{#8} & {#5}_{#9} & {#5}_{#10}
%\end{vmatrix}
%}

% R3 vector.
\newcommand{\VectorThree}[3]{
\begin{bmatrix}
 {#1} \\
 {#2} \\
 {#3}
\end{bmatrix}
}



\author{Peeter Joot}
\email{peeter.joot@gmail.com}

%\documentclass[]{eliblogwidescreen}

\usepackage{amsmath}
\usepackage{mathpazo}

%
% shorthand for bold symbols, convenient for vectors and matrices
%
\newcommand{\Ba}[0]{\mathbf{a}}
\newcommand{\Bb}[0]{\mathbf{b}}
\newcommand{\Bc}[0]{\mathbf{c}}
\newcommand{\Bd}[0]{\mathbf{d}}
\newcommand{\Be}[0]{\mathbf{e}}
\newcommand{\Bf}[0]{\mathbf{f}}
\newcommand{\Bg}[0]{\mathbf{g}}
\newcommand{\Bh}[0]{\mathbf{h}}
\newcommand{\Bi}[0]{\mathbf{i}}
\newcommand{\Bj}[0]{\mathbf{j}}
\newcommand{\Bk}[0]{\mathbf{k}}
\newcommand{\Bl}[0]{\mathbf{l}}
\newcommand{\Bm}[0]{\mathbf{m}}
\newcommand{\Bn}[0]{\mathbf{n}}
\newcommand{\Bo}[0]{\mathbf{o}}
\newcommand{\Bp}[0]{\mathbf{p}}
\newcommand{\Bq}[0]{\mathbf{q}}
\newcommand{\Br}[0]{\mathbf{r}}
\newcommand{\Bs}[0]{\mathbf{s}}
\newcommand{\Bt}[0]{\mathbf{t}}
\newcommand{\Bu}[0]{\mathbf{u}}
\newcommand{\Bv}[0]{\mathbf{v}}
\newcommand{\Bw}[0]{\mathbf{w}}
\newcommand{\Bx}[0]{\mathbf{x}}
\newcommand{\By}[0]{\mathbf{y}}
\newcommand{\Bz}[0]{\mathbf{z}}
\newcommand{\BA}[0]{\mathbf{A}}
\newcommand{\BB}[0]{\mathbf{B}}
\newcommand{\BC}[0]{\mathbf{C}}
\newcommand{\BD}[0]{\mathbf{D}}
\newcommand{\BE}[0]{\mathbf{E}}
\newcommand{\BF}[0]{\mathbf{F}}
\newcommand{\BG}[0]{\mathbf{G}}
\newcommand{\BH}[0]{\mathbf{H}}
\newcommand{\BI}[0]{\mathbf{I}}
\newcommand{\BJ}[0]{\mathbf{J}}
\newcommand{\BK}[0]{\mathbf{K}}
\newcommand{\BL}[0]{\mathbf{L}}
\newcommand{\BM}[0]{\mathbf{M}}
\newcommand{\BN}[0]{\mathbf{N}}
\newcommand{\BO}[0]{\mathbf{O}}
\newcommand{\BP}[0]{\mathbf{P}}
\newcommand{\BQ}[0]{\mathbf{Q}}
\newcommand{\BR}[0]{\mathbf{R}}
\newcommand{\BS}[0]{\mathbf{S}}
\newcommand{\BT}[0]{\mathbf{T}}
\newcommand{\BU}[0]{\mathbf{U}}
\newcommand{\BV}[0]{\mathbf{V}}
\newcommand{\BW}[0]{\mathbf{W}}
\newcommand{\BX}[0]{\mathbf{X}}
\newcommand{\BY}[0]{\mathbf{Y}}
\newcommand{\BZ}[0]{\mathbf{Z}}

\newcommand{\Bzero}[0]{\mathbf{0}}
\newcommand{\Btheta}[0]{\boldsymbol{\theta}}
\newcommand{\Btau}[0]{\boldsymbol{\tau}}
\newcommand{\Bomega}[0]{\boldsymbol{\omega}}

%
% shorthand for unit vectors
%
\newcommand{\acap}[0]{\hat{\Ba}}
\newcommand{\bcap}[0]{\hat{\Bb}}
\newcommand{\ccap}[0]{\hat{\Bc}}
\newcommand{\dcap}[0]{\hat{\Bd}}
\newcommand{\ecap}[0]{\hat{\Be}}
\newcommand{\fcap}[0]{\hat{\Bf}}
\newcommand{\gcap}[0]{\hat{\Bg}}
\newcommand{\hcap}[0]{\hat{\Bh}}
\newcommand{\icap}[0]{\hat{\Bi}}
\newcommand{\jcap}[0]{\hat{\Bj}}
\newcommand{\kcap}[0]{\hat{\Bk}}
\newcommand{\lcap}[0]{\hat{\Bl}}
\newcommand{\mcap}[0]{\hat{\Bm}}
\newcommand{\ncap}[0]{\hat{\Bn}}
\newcommand{\ocap}[0]{\hat{\Bo}}
\newcommand{\pcap}[0]{\hat{\Bp}}
\newcommand{\qcap}[0]{\hat{\Bq}}
\newcommand{\rcap}[0]{\hat{\Br}}
\newcommand{\scap}[0]{\hat{\Bs}}
\newcommand{\tcap}[0]{\hat{\Bt}}
\newcommand{\ucap}[0]{\hat{\Bu}}
\newcommand{\vcap}[0]{\hat{\Bv}}
\newcommand{\wcap}[0]{\hat{\Bw}}
\newcommand{\xcap}[0]{\hat{\Bx}}
\newcommand{\ycap}[0]{\hat{\By}}
\newcommand{\zcap}[0]{\hat{\Bz}}
\newcommand{\thetacap}[0]{\hat{\Btheta}}

%
% to write R^n and C^n in a distinguishable fashion.  Perhaps change this
% to the double lined characters upon figuring out how to do so.
%
\newcommand{\C}[1]{$\mathbb{C}^{#1}$}
\newcommand{\R}[1]{$\mathbb{R}^{#1}$}

%
% various generally useful helpers
%

% derivative of #1 wrt. #2:
\newcommand{\D}[2] {\frac {d#2} {d#1}}

\newcommand{\inv}[1]{\frac{1}{#1}}
\newcommand{\cross}[0]{\times}

\newcommand{\abs}[1]{\lvert{#1}\rvert}
\newcommand{\norm}[1]{\lVert{#1}\rVert}
\newcommand{\innerprod}[2]{\langle{#1}, {#2}\rangle}
\newcommand{\dotprod}[2]{{#1} \cdot {#2}}
\newcommand{\bdotprod}[2]{\left({#1} \cdot {#2}\right)}
\newcommand{\crossprod}[2]{{#1} \cross {#2}}
\newcommand{\tripleprod}[3]{\dotprod{\left(\crossprod{#1}{#2}\right)}{#3}}

\DeclareMathOperator{\Proj}{Proj}
\DeclareMathOperator{\Span}{span}
\DeclareMathOperator{\Sgn}{sgn}
\DeclareMathOperator{\Area}{Area}
\DeclareMathOperator{\Volume}{Volume}

%
% A few miscellaneous things specific to this document
%
\newcommand{\crossop}[1]{\crossprod{#1}{}}

% R2 vector.
\newcommand{\VectorTwo}[2]{
\begin{bmatrix}
 {#1} \\
 {#2}
\end{bmatrix}
}

\newcommand{\VectorN}[1]{
\begin{bmatrix}
{#1}_1 \\
{#1}_2 \\
\vdots \\
{#1}_N \\
\end{bmatrix}
}

\newcommand{\DETuvij}[4]{
\begin{vmatrix}
 {#1}_{#3} & {#1}_{#4} \\
 {#2}_{#3} & {#2}_{#4}
\end{vmatrix}
}

\newcommand{\DETuvwijk}[6]{
\begin{vmatrix}
 {#1}_{#4} & {#1}_{#5} & {#1}_{#6} \\
 {#2}_{#4} & {#2}_{#5} & {#2}_{#6} \\
 {#3}_{#4} & {#3}_{#5} & {#3}_{#6}
\end{vmatrix}
}

\newcommand{\DETuvwxijkl}[8]{
\begin{vmatrix}
 {#1}_{#5} & {#1}_{#6} & {#1}_{#7} & {#1}_{#8} \\
 {#2}_{#5} & {#2}_{#6} & {#2}_{#7} & {#2}_{#8} \\
 {#3}_{#5} & {#3}_{#6} & {#3}_{#7} & {#3}_{#8} \\
 {#4}_{#5} & {#4}_{#6} & {#4}_{#7} & {#4}_{#8} \\
\end{vmatrix}
}

%\newcommand{\DETuvwxyijklm}[10]{
%\begin{vmatrix}
% {#1}_{#6} & {#1}_{#7} & {#1}_{#8} & {#1}_{#9} & {#1}_{#10} \\
% {#2}_{#6} & {#2}_{#7} & {#2}_{#8} & {#2}_{#9} & {#2}_{#10} \\
% {#3}_{#6} & {#3}_{#7} & {#3}_{#8} & {#3}_{#9} & {#3}_{#10} \\
% {#4}_{#6} & {#4}_{#7} & {#4}_{#8} & {#4}_{#9} & {#4}_{#10} \\
% {#5}_{#6} & {#5}_{#7} & {#5}_{#8} & {#5}_{#9} & {#5}_{#10}
%\end{vmatrix}
%}

% R3 vector.
\newcommand{\VectorThree}[3]{
\begin{bmatrix}
 {#1} \\
 {#2} \\
 {#3}
\end{bmatrix}
}



\author{Peeter Joot}
\email{peeter.joot@gmail.com}


%\chapter{Change of basis and Graham Schmidt orthonormalization as pedagogical tools for special relativity.}
%\label{chap:grahamSchmidtLorentz}
%%\useCCL
%\blogpage{http://sites.google.com/site/peeterjoot/math2011/grahamSchmidtLorentz.pdf}
%\date{April 14, 2011}
%\revisionInfo{grahamSchmidtLorentz.tex}
%
%\beginArtWithToc
%\beginArtNoToc

% journal style swiped from:
%
% http://arxiv.org/abs/1006.3552

\documentclass[iop,tighten]{emulateapj}
%\documentclass[12pt,preprint]{aastex}

%\usepackage{apjfonts}
%\usepackage[numbers]{natbib}

\usepackage{amsmath}
\usepackage{mathpazo}

%
% shorthand for bold symbols, convenient for vectors and matrices
%
\newcommand{\Ba}[0]{\mathbf{a}}
\newcommand{\Bb}[0]{\mathbf{b}}
\newcommand{\Bc}[0]{\mathbf{c}}
\newcommand{\Bd}[0]{\mathbf{d}}
\newcommand{\Be}[0]{\mathbf{e}}
\newcommand{\Bf}[0]{\mathbf{f}}
\newcommand{\Bg}[0]{\mathbf{g}}
\newcommand{\Bh}[0]{\mathbf{h}}
\newcommand{\Bi}[0]{\mathbf{i}}
\newcommand{\Bj}[0]{\mathbf{j}}
\newcommand{\Bk}[0]{\mathbf{k}}
\newcommand{\Bl}[0]{\mathbf{l}}
\newcommand{\Bm}[0]{\mathbf{m}}
\newcommand{\Bn}[0]{\mathbf{n}}
\newcommand{\Bo}[0]{\mathbf{o}}
\newcommand{\Bp}[0]{\mathbf{p}}
\newcommand{\Bq}[0]{\mathbf{q}}
\newcommand{\Br}[0]{\mathbf{r}}
\newcommand{\Bs}[0]{\mathbf{s}}
\newcommand{\Bt}[0]{\mathbf{t}}
\newcommand{\Bu}[0]{\mathbf{u}}
\newcommand{\Bv}[0]{\mathbf{v}}
\newcommand{\Bw}[0]{\mathbf{w}}
\newcommand{\Bx}[0]{\mathbf{x}}
\newcommand{\By}[0]{\mathbf{y}}
\newcommand{\Bz}[0]{\mathbf{z}}
\newcommand{\BA}[0]{\mathbf{A}}
\newcommand{\BB}[0]{\mathbf{B}}
\newcommand{\BC}[0]{\mathbf{C}}
\newcommand{\BD}[0]{\mathbf{D}}
\newcommand{\BE}[0]{\mathbf{E}}
\newcommand{\BF}[0]{\mathbf{F}}
\newcommand{\BG}[0]{\mathbf{G}}
\newcommand{\BH}[0]{\mathbf{H}}
\newcommand{\BI}[0]{\mathbf{I}}
\newcommand{\BJ}[0]{\mathbf{J}}
\newcommand{\BK}[0]{\mathbf{K}}
\newcommand{\BL}[0]{\mathbf{L}}
\newcommand{\BM}[0]{\mathbf{M}}
\newcommand{\BN}[0]{\mathbf{N}}
\newcommand{\BO}[0]{\mathbf{O}}
\newcommand{\BP}[0]{\mathbf{P}}
\newcommand{\BQ}[0]{\mathbf{Q}}
\newcommand{\BR}[0]{\mathbf{R}}
\newcommand{\BS}[0]{\mathbf{S}}
\newcommand{\BT}[0]{\mathbf{T}}
\newcommand{\BU}[0]{\mathbf{U}}
\newcommand{\BV}[0]{\mathbf{V}}
\newcommand{\BW}[0]{\mathbf{W}}
\newcommand{\BX}[0]{\mathbf{X}}
\newcommand{\BY}[0]{\mathbf{Y}}
\newcommand{\BZ}[0]{\mathbf{Z}}

\newcommand{\Bzero}[0]{\mathbf{0}}
\newcommand{\Btheta}[0]{\boldsymbol{\theta}}
\newcommand{\Btau}[0]{\boldsymbol{\tau}}
\newcommand{\Bomega}[0]{\boldsymbol{\omega}}

%
% shorthand for unit vectors
%
\newcommand{\acap}[0]{\hat{\Ba}}
\newcommand{\bcap}[0]{\hat{\Bb}}
\newcommand{\ccap}[0]{\hat{\Bc}}
\newcommand{\dcap}[0]{\hat{\Bd}}
\newcommand{\ecap}[0]{\hat{\Be}}
\newcommand{\fcap}[0]{\hat{\Bf}}
\newcommand{\gcap}[0]{\hat{\Bg}}
\newcommand{\hcap}[0]{\hat{\Bh}}
\newcommand{\icap}[0]{\hat{\Bi}}
\newcommand{\jcap}[0]{\hat{\Bj}}
\newcommand{\kcap}[0]{\hat{\Bk}}
\newcommand{\lcap}[0]{\hat{\Bl}}
\newcommand{\mcap}[0]{\hat{\Bm}}
\newcommand{\ncap}[0]{\hat{\Bn}}
\newcommand{\ocap}[0]{\hat{\Bo}}
\newcommand{\pcap}[0]{\hat{\Bp}}
\newcommand{\qcap}[0]{\hat{\Bq}}
\newcommand{\rcap}[0]{\hat{\Br}}
\newcommand{\scap}[0]{\hat{\Bs}}
\newcommand{\tcap}[0]{\hat{\Bt}}
\newcommand{\ucap}[0]{\hat{\Bu}}
\newcommand{\vcap}[0]{\hat{\Bv}}
\newcommand{\wcap}[0]{\hat{\Bw}}
\newcommand{\xcap}[0]{\hat{\Bx}}
\newcommand{\ycap}[0]{\hat{\By}}
\newcommand{\zcap}[0]{\hat{\Bz}}
\newcommand{\thetacap}[0]{\hat{\Btheta}}

%
% to write R^n and C^n in a distinguishable fashion.  Perhaps change this
% to the double lined characters upon figuring out how to do so.
%
\newcommand{\C}[1]{$\mathbb{C}^{#1}$}
\newcommand{\R}[1]{$\mathbb{R}^{#1}$}

%
% various generally useful helpers
%

% derivative of #1 wrt. #2:
\newcommand{\D}[2] {\frac {d#2} {d#1}}

\newcommand{\inv}[1]{\frac{1}{#1}}
\newcommand{\cross}[0]{\times}

\newcommand{\abs}[1]{\lvert{#1}\rvert}
\newcommand{\norm}[1]{\lVert{#1}\rVert}
\newcommand{\innerprod}[2]{\langle{#1}, {#2}\rangle}
\newcommand{\dotprod}[2]{{#1} \cdot {#2}}
\newcommand{\bdotprod}[2]{\left({#1} \cdot {#2}\right)}
\newcommand{\crossprod}[2]{{#1} \cross {#2}}
\newcommand{\tripleprod}[3]{\dotprod{\left(\crossprod{#1}{#2}\right)}{#3}}

\DeclareMathOperator{\Proj}{Proj}
\DeclareMathOperator{\Span}{span}
\DeclareMathOperator{\Sgn}{sgn}
\DeclareMathOperator{\Area}{Area}
\DeclareMathOperator{\Volume}{Volume}

%
% A few miscellaneous things specific to this document
%
\newcommand{\crossop}[1]{\crossprod{#1}{}}

% R2 vector.
\newcommand{\VectorTwo}[2]{
\begin{bmatrix}
 {#1} \\
 {#2}
\end{bmatrix}
}

\newcommand{\VectorN}[1]{
\begin{bmatrix}
{#1}_1 \\
{#1}_2 \\
\vdots \\
{#1}_N \\
\end{bmatrix}
}

\newcommand{\DETuvij}[4]{
\begin{vmatrix}
 {#1}_{#3} & {#1}_{#4} \\
 {#2}_{#3} & {#2}_{#4}
\end{vmatrix}
}

\newcommand{\DETuvwijk}[6]{
\begin{vmatrix}
 {#1}_{#4} & {#1}_{#5} & {#1}_{#6} \\
 {#2}_{#4} & {#2}_{#5} & {#2}_{#6} \\
 {#3}_{#4} & {#3}_{#5} & {#3}_{#6}
\end{vmatrix}
}

\newcommand{\DETuvwxijkl}[8]{
\begin{vmatrix}
 {#1}_{#5} & {#1}_{#6} & {#1}_{#7} & {#1}_{#8} \\
 {#2}_{#5} & {#2}_{#6} & {#2}_{#7} & {#2}_{#8} \\
 {#3}_{#5} & {#3}_{#6} & {#3}_{#7} & {#3}_{#8} \\
 {#4}_{#5} & {#4}_{#6} & {#4}_{#7} & {#4}_{#8} \\
\end{vmatrix}
}

%\newcommand{\DETuvwxyijklm}[10]{
%\begin{vmatrix}
% {#1}_{#6} & {#1}_{#7} & {#1}_{#8} & {#1}_{#9} & {#1}_{#10} \\
% {#2}_{#6} & {#2}_{#7} & {#2}_{#8} & {#2}_{#9} & {#2}_{#10} \\
% {#3}_{#6} & {#3}_{#7} & {#3}_{#8} & {#3}_{#9} & {#3}_{#10} \\
% {#4}_{#6} & {#4}_{#7} & {#4}_{#8} & {#4}_{#9} & {#4}_{#10} \\
% {#5}_{#6} & {#5}_{#7} & {#5}_{#8} & {#5}_{#9} & {#5}_{#10}
%\end{vmatrix}
%}

% R3 vector.
\newcommand{\VectorThree}[3]{
\begin{bmatrix}
 {#1} \\
 {#2} \\
 {#3}
\end{bmatrix}
}


%\newcommand{\EndArticle}[0]{\bibliography{myrefs}\bibliographystyle{unsrturl}\end{document}}

\def\etal{{et~al.}}

\catcode`\@=11
\newcommand{\@versim}[2]
  {\lower3.1truept\vbox{\baselineskip0pt\lineskip0.5truept
\ialign{$\m@th#1\hfil##\hfil$\crcr#2\crcr\sim\crcr}}}
\catcode`\@=12
% This is an extra line so that table is not too long.
\setlength{\tabcolsep}{2.8pt}

%\journalinfo{ApJ Letters, in press}
%\submitted{}
%\received{June 16, 2010} \accepted{October 20, 2010}
%\shorttitle{short title goes here.}

\begin{document}

\title{Change of basis and Graham Schmidt orthonormalization as pedagogical tools for special relativity.}

\author{
Peeter Joot \altaffilmark{1}
}

\altaffiltext{1}{Department of Physics, University of Toronto, peeter.joot@utoronto.ca} 
%\altaffiltext{2} {Cavendish Laboratory; 19 J.J. Thomson Ave., 
%Cambridge CB3 0HE, UK}
%\altaffiltext{3}{NASA/GSFC, Code 660, Greenbelt, MD 20771}
%\altaffiltext{4}{Department of Electrical and Computer Engineering, 
%Duke University, Durham, NC 27708}

\begin{abstract}

The standard language for the teaching and study of special relativity is that of tensor algebra.  
While an explicit basis is common in the study of Euclidean spaces, it is usually implied in the study of inertial relativistic systems.  
For the study of inertial relativistic systems it will be shown that introducing an explicit basis opens up some of the standard linear algebra toolbox, and has some distinct conceptual advantages.
As with matrix representations of four vectors, the use of an explicit basis allows the details of the coordinate representation to be suppressed, allowing four vectors to be manipulated as complete entities in an inner products space.
It will be shown that Lorentz transformations can be viewed as change of basis operations.  It will also be seen that the Lorentz boost matrix can be calculated using the Graham-Schmidt orthonormalization algorithm, a procedure that most students of relativity are likely already comfortable using in its Euclidean form.
%This treatment is not intended for a new student of relativity, but it is the author's opinion that many of the ideas presented could be utilized in the instruction of the subject.

\end{abstract}

\keywords{
Lorentz boost, change of basis, special relativity, inertial frame
}

%\section{Introduction}
%\label{intro}
%
%Intro text.
%
%\section{Observations}
%\label{obssec}
%
%Observations.
%
%\section{Spectral Analysis}
%
%\section{Discussion}
%
%We summarize our results as follows:
%\begin{enumerate}
%
%\item blah
%
%\item blah
%
%\end{enumerate}
%
%\acknowledgments
%
%This work was supported by NASA through Chandra General Observer
%Program grant SAO GO6-7059X.




\section{Abstract.}


\section{Preliminaries, notation, and definitions.}

\subsection{Four vectors and invariant length}

We define a four vector, or event, as a tuple of time and space coordinates.  These will be represented herein as non-bold letters of the form

\begin{equation}\label{eqn:grahamSchmidtLorentz:10}
x = (c t, \Bx) = ( c t, x, y, z ) = ( x^0, x^1, x^2, x^3 ).
\end{equation}

Bold letters are reserved for Euclidean vectors.  As usual, the factor of $c$ in the time coordinate is included so that the units of any of the coordinates in the tuple have dimensions of distance.  As conventional, upper indexes are used for the coordinates of the four vector in the standard basis (so $x^2$ is the 2-indexed coordinate of the four vector and not the square of $x$).  Lower indexed four vector coordinates will be introduced later once the reciprocal basis is introduced.

At the heart of special relativity is the definition of the invariant length or interval, which defines a ``distance'' like measure between any a pair of events.  The (squared) invariant length of the vector $x$ above from the origin of the space time vector space $(0, 0, 0, 0)$ can be defined in terms a non-positive definite inner product

\begin{equation}\label{eqn:grahamSchmidtLorentz:30}
x^2 = x \cdot x = \pm ((c t)^2 - \Bx^2),
\end{equation}

where $\Bx^2 = \Bx \cdot \Bx$ is used as shorthand for the Euclidean dot product.  The choice of a positive sign will be used herein.  Equivalent to the definition above, one may define a dot product between two vectors.  For $y = (y^0, \By)$ we have

\begin{equation}\label{eqn:grahamSchmidtLorentz:50}
x \cdot y = x^0 y^0 - \Bx \cdot \By.
\end{equation}

Observer that no attempt at motivating why a mixed sign length of this form is required for special relativity will be made here.  That more difficult job is deferred to any number of books that cover special relativity \citep{landau1980classical}).

\subsection{Coordinate representation with a non-orthonormal basis}

\subsubsection{Standard basis.}

The use of a non-orthonormal basis, even in Euclidean spaces, makes life a bit more difficult.  There is, however, no choice in the matter for special relativity.  The closest that we can get to an orthonormal basis is the standard basis $\{e_0, e_1, e_2, e_3\}$, where

\begin{equation}\label{eqn:grahamSchmidtLorentz:70}
\begin{aligned}
e_0 &= (1, 0, 0, 0) \\
e_1 &= (0, 1, 0, 0) \\
e_2 &= (0, 0, 1, 0) \\
e_3 &= (0, 0, 0, 1) 
\end{aligned}
\end{equation}

From our definition of invariant length we have $(e_1)^2 = (e_2)^2 = (e_3)^2 = -1$, and $(e_0)^2 = 1$.  A relativistic basis cannot be constructed for which all the basis vectors are unity.

\subsubsection{Reciprocal basis by example in Euclidean space.}

It is convenient to introduce the concept of a reciprocal basis when dealing with non-orthonormal spaces \citep{doran2003gap}.

The utility of a reciprocal basis is not limited to the non-Euclidean vector spaces of special relativity.  We can use a simple oblique Euclidean basis in a plane to illustrate the ideas.  For example, suppose that we have a basis $A = \{e_1, e_2\}$ in a 2 dimensional Euclidean space

\begin{equation}\label{eqn:grahamSchmidtLorentz:90}
\begin{aligned}
e_1 &= 
\begin{bmatrix}
1 \\
0
\end{bmatrix} \\
e_2 &= 
\begin{bmatrix}
1 \\
2
\end{bmatrix} 
\end{aligned}
\end{equation}

Consider the problem of solving for the coordinates $a,b$ of a vector with such a basis

\begin{equation}\label{eqn:grahamSchmidtLorentz:110}
x = a e_1 + b e_2.
\end{equation}

Provided the basis is complete, it is always possible to compute a second basis ${e^1, e^2}$ for which one has

\begin{equation}\label{eqn:grahamSchmidtLorentz:130}
e_i \cdot e^j = {\delta_i}^j.
\end{equation}

This set of vectors is called the reciprocal basis.  We can calculate these vectors easily, based on the requirement that

\begin{equation}\label{eqn:grahamSchmidtLorentz:150}
\begin{bmatrix}
{e_1}^\T \\
{e_2}^T
\end{bmatrix}
\begin{bmatrix}
e^1 & e^2
\end{bmatrix} = I.
\end{equation}

Working that out gives us

\begin{equation}\label{eqn:grahamSchmidtLorentz:170}
\begin{bmatrix}
e^1 & e^2
\end{bmatrix}
=
{
\begin{bmatrix}
{e_1}^\T \\
{e_2}^T
\end{bmatrix}
}^{-1}
= 
{\begin{bmatrix}
1 & 0 \\
1 & 2
\end{bmatrix}}^{-1}
= 
\inv{2} 
\begin{bmatrix}
2 & 0 \\
-1 & 1
\end{bmatrix},
\end{equation}

or

\begin{equation}\label{eqn:grahamSchmidtLorentz:190}
\begin{aligned}
e_1 &= 
\begin{bmatrix}
1 \\
-1/2
\end{bmatrix} \\
e_2 &= 
\begin{bmatrix}
0 \\
1/2
\end{bmatrix} 
\end{aligned}
\end{equation}

The task of computing the coordinates of our vector is now simple.  We have

\begin{equation}\label{eqn:grahamSchmidtLorentz:340}
\begin{aligned}
x \cdot e^1 &= a e_1 \cdot e^1 + b \cancel{e_2 \cdot e^1} = a \\
x \cdot e^2 &= a \cancel{e_1 \cdot e^2} + b e_2 \cdot e^2 = b
\end{aligned}
\end{equation}

Here's a concrete example

\begin{equation}\label{eqn:grahamSchmidtLorentz:230}
x = 
\begin{bmatrix}
4 \\
2
\end{bmatrix} 
=
\left(
\begin{bmatrix}
4 \\
2
\end{bmatrix} 
\cdot e^1 
\right)
e_1
+
\left(
\begin{bmatrix}
4 \\
2
\end{bmatrix} 
\cdot e^2
\right)
e_2
= 3 
\begin{bmatrix}
1 \\
0
\end{bmatrix} 
+ 
1
\begin{bmatrix}
1 \\
2
\end{bmatrix}.
\end{equation}

There is also nothing stopping us from using the reciprocal basis, and calculating the components with respect to that.  For the same vector, if we were to write

\begin{equation}\label{eqn:grahamSchmidtLorentz:250}
x = c e^1 + d e^2,
\end{equation}

we can dot with $e_1$ and $e_2$ respectively to compute $c$ and $d$ and find

\begin{equation}\label{eqn:grahamSchmidtLorentz:360}
\begin{aligned}
x \cdot e_1 &= c e^1 \cdot e_1 + d \cancel{e^2 \cdot e_1} = c \\
x \cdot e_2 &= c \cancel{e^1 \cdot e_2} + d e^2 \cdot e_2 = d
\end{aligned}
\end{equation}

For the concrete example we have

\begin{equation}\label{eqn:grahamSchmidtLorentz:290}
\begin{aligned}
x 
&= 
\begin{bmatrix}
4 \\
2
\end{bmatrix} \\
&=
\left(
\begin{bmatrix}
4 \\
2
\end{bmatrix} 
\cdot e_1
\right)
e^1
+
\left(
\begin{bmatrix}
4 \\
2
\end{bmatrix} 
\cdot e_2
\right)
e^2 \\
&= 4 
\begin{bmatrix}
1 \\
-1/2
\end{bmatrix} 
+ 
8
\begin{bmatrix}
0 \\
1/2
\end{bmatrix}
\end{aligned}
\end{equation}

It is natural to employ the tensor notation writing $x^1 = a, x^2 = b, x_1 = c, x_2 = d$, so that we have

\begin{equation}\label{eqn:grahamSchmidtLorentz:310}
x = x^1 e_1 + x^2 e_2 = x_1 e^1 + x_2 e^2,
\end{equation}

or with summation implied over repeated indexes just

\begin{equation}\label{eqn:grahamSchmidtLorentz:330}
\begin{aligned}
x &= x^i e_i = (x \cdot e^i) e_i \\
  &= x_i e^i = (x \cdot e_i) e^i
\end{aligned}
\end{equation}

In this context there is nothing special about either upper or lower indexes.  They are just coordinates with respect to a basis and its reciprocal basis, respectively.  Observe that when the original basis happens to be orthonormal, one has equality between the basis vectors and their reciprocal duals $e_i = e^i$, and between the coordinates calculated from those bases respectively $x_i = x^i$.

While the dot product expression in \ref{eqn:grahamSchmidtLorentz:330} may seem like a recursive definition, this describes an important property.  Each of the far RHS terms represents a projection.  The projection of a vector onto the basis direction $e_i$

\begin{equation}\label{eqn:grahamSchmidtLorentz:560}
\Proj_{e_i} x = (x \cdot e^i) e_i,
\end{equation}

(no sum implied.)  This will be important since the Graham-Schmidt procedure is essentially just the repeated subtraction of projections, and we will need to know how to express this for a non-orthonormal basis.

\subsubsection{Reciprocal basis for relativity.}

Because it is not possible to have an orthonormal basis in a relativistic context, the reciprocal basis necessarily has a place in the geometry of relativity.  The vectors in the reciprocal basis are still implicitly defined by \ref{eqn:grahamSchmidtLorentz:130}, and while the calculation of these is slightly messier in general, the reciprocal basis $\{e^i\}$ associated with the standard basis $\{e_i\}$ of \ref{eqn:grahamSchmidtLorentz:70} can be obtained by inspection

\begin{equation}\label{eqn:grahamSchmidtLorentz:70b}
\begin{aligned}
e^0 &= (1, 0, 0, 0) \\
e^1 &= (0, -1, 0, 0) \\
e^2 &= (0, 0, -1, 0) \\
e^3 &= (0, 0, 0, -1) 
\end{aligned}
\end{equation}

Utilizing this pair of bases we have for a spacetime vector

\begin{equation}\label{eqn:grahamSchmidtLorentz:331}
\begin{aligned}
x &= x^\alpha e_\alpha = (x \cdot e^\alpha) e_\alpha \\
  &= x_\alpha e^\alpha = (x \cdot e_\alpha) e^\alpha,
\end{aligned}
\end{equation}

where repeated Greek indexes imply summation over both temporal and spatial indexes $\{0, 1, 2, 3\}$.  Again in this context, there is nothing special about either upper or lower indexes.  In tensor algebra, upper indexes are ``special'' since the invariant transformations are defined in terms of those coordinates, but this is really just a choice of basis.  Ideally we would seek a coordinate free representation of our Lorentz transformations consistent with this statement.

\subsection{Proper separation}

Given a particle parametrization along a trajectory $x(\lambda)$ in spacetime, one can compute the average spacetime length between a pair of points on this path

\begin{equation}\label{eqn:grahamSchmidtLorentz:400}
s_b - s_a = \int_{\lambda = a}^b \sqrt{ \frac{d x(\lambda)}{d\lambda} \cdot \frac{d x(\lambda)}{d\lambda} } d\lambda.
\end{equation}

%For infinitesimally separated points on the tragectory, we have
%
%\begin{equation}\label{eqn:grahamSchmidtLorentz:420}
%ds^2 = d\lambda^2 \left( \frac{d x(\lambda)}{d\lambda} \cdot \frac{d x(\lambda)} \right)^2.
%\end{equation}
%
%Being a bit loose with our differentials this is
%
%\begin{align}\label{eqn:grahamSchmidtLorentz:420b}
%\left( \frac{ds}{d\lambda} )^2 &= \left( \frac{d x(\lambda)}{d\lambda} \cdot \frac{d x(\lambda) \right)^2 \\
%\left( \frac{d\lambda}{ds} )^2 &= 1/\left( \frac{d x(\lambda)}{d\lambda} \cdot \frac{d x(\lambda) \right)^2
%\end{align}

A trajectory $x(\lambda)$ may be reparametrized in terms of the instantaneous proper separation $s$.  Doing so, one finds that the first derivative of $x(s)$ is a timelike unit vector (squares to $1$, not $-1$ like $e_0 = (1, 0, 0, 0)$) along any point of the curve.

This is nicely demonstrated by example.
%  That is
%
%\begin{align*}
%\left( \frac{dx}{ds} \right)^2 
%&= 
%\left( \frac{dx}{d\lambda} \frac{d\lambda}{ds} \right)^2  \\
%&= 
%\left( \frac{dx}{d\lambda} \right)^2 \left( \frac{d\lambda}{ds} \right)^2  \\
%&= 1
%\end{align*}
%
Consider an inertial system, with a particle moving along a constant velocity trajectory, parametrized by an external observers time $t$

\begin{equation}\label{eqn:grahamSchmidtLorentz:440}
x(t) = (c t, \Bv t)
\end{equation}

Our proper separation anywhere along this spacetime curve is

\begin{equation}\label{eqn:grahamSchmidtLorentz:460}
s = \int \sqrt{(c t, \Bv t)^2} dt = \int \sqrt{ c^2 - \Bv^2 } dt = \sqrt{ c^2 - \Bv^2 } t,
\end{equation}

so our proper length reparametrization of this path is

\begin{equation}\label{eqn:grahamSchmidtLorentz:480}
x(s) = \inv{\sqrt{c^2 - \Bv^2}} (c, \Bv) s.
\end{equation}

Our first derivative

\begin{equation}\label{eqn:grahamSchmidtLorentz:500}
\frac{dx}{ds} = \inv{\sqrt{c^2 - \Bv^2}} (c, \Bv) 
\end{equation}

is now easily observed to be of unit length 

\begin{equation}\label{eqn:grahamSchmidtLorentz:520}
\left( \frac{dx}{ds} \right)^2 = \inv{c^2 - \Bv^2} (c^2 - \Bv^2) = 1.
\end{equation}

For an arbitrary reparametrization $x(\mu(\lambda))$ we can utilize that parametrization to decompose a trajectory into tangential and perpendicular components

\begin{equation}\label{eqn:grahamSchmidtLorentz:540}
x(\mu) = \frac{dx}{d\mu} \mu + \left( x - \frac{dx}{d\mu} \mu \right).
\end{equation}

Observe that for an inertial system, this implies that we have only a timelike component for the trajectory, when parametrized by proper length.  That is

\begin{equation}\label{eqn:grahamSchmidtLorentz:580}
x(s) = \frac{dx}{ds} s.
\end{equation}

This means that a proper length parametrization corresponds to the time in the frame for which the particle is instantaneously at rest.  A student of special relativity is familiar with being able to switch to a frame in which the particle is instantaneously at rest by performing a Lorentz boost, so we have indirectly obtained a first hint that we can interpret such a transformation as nothing more than a change of basis.

\section{Lorentz boost as a change of basis.}

For the trajectory \ref{eqn:grahamSchmidtLorentz:440} we found that the proper length derivative was a timelike unit vector for the frame in which the particle was at rest.  That unit vector can be used as part of a basis for that rest frame.  Once this basis is completed with its spatial unit vectors, we will see how the particle's rest basis and an observer basis are related by Lorentz transformation.

\subsection{Illustration by example.  One spatial dimension.}

For simplicity, consider a two dimensional spacetime vector space, with a particle trajectory in an inertial frame parameterized by its proper length

\begin{equation}\label{eqn:grahamSchmidtLorentz:600}
x(s) = \gamma (1, \beta) s.
\end{equation}

Let's lable the time like unit vector in the particle's rest frame $f_0$, so that we have

\begin{equation}\label{eqn:grahamSchmidtLorentz:620}
f_0 = \frac{dx}{ds} = \gamma (1, \beta) = f^0.
\end{equation}

The particle's trajectory in the rest frame, in terms of the basis to be determined is thus

\begin{equation}\label{eqn:grahamSchmidtLorentz:640}
x(s) = s f_0 + 0 f_1.
\end{equation}

The task is to compute this basis $\{f_0, f_1\}$ for the particle's rest frame, and this can be done using the Graham-Schmidt procedure.  The procedure is the same as in Euclidean space.  We wish to determine a set of normal vectors $\{f_\beta\}$, and have some subset of that, say $\{f_0, \cdots, f_\alpha\}, \alpha < \beta$, where $\beta$ is the dimension of the space.  This can be extended by an additional normal vector, by picking any one vector $a$ that lies outside of the span of this set, and subtract the projections of $a$ vector from the set of normal vectors computed so far.  Such a vector, outside of the span of the inital set of normal vectors is

\begin{equation}\label{eqn:grahamSchmidtLorentz:660}
b = a - \sum_{\sigma \le \alpha} (a \cdot f_\sigma) f^\sigma
\end{equation}

Once normalized $f_{\alpha+1} = b/\sqrt{\Abs{b^2}}$, we can repeat until the basis is complete.

For this one dimensional spatial example, we can pick any vector out of the span of $f_0$ and find a vector normal to that.  Picking rather arbitrarily, let's use $e_1 = (0, 1)$, 

\begin{align*}
b 
&= e_1 - (e_1 \cdot f_0) f^0 \\
&= (0, 1) - (0, 1) \cdot (1, \beta) \gamma^2 (1, \beta) \\
&= (0, 1) + \beta \gamma^2 (1, \beta) \\
&= (\gamma^2 \beta, 1 + \beta^2 \gamma^2) \\
&= \gamma^2 (\beta, 1 ).
\end{align*}

This can be normalized as $f_1 = \pm \gamma(\beta, 1)$.  To determine the sign, let's require that the particle's rest frame basis has the same orientation as the standard basis.  This is a requirement that the determinants of the coordinates of the unit vectors match.  Our standard basis in the sequence $e_0, e_1$ can be defined as positively oriented

\begin{equation}\label{eqn:grahamSchmidtLorentz:680}
\begin{vmatrix}
e_0 \\
e_1
\end{vmatrix} 
=
\begin{vmatrix}
1 & 0 \\
0 & 1
\end{vmatrix} 
= 1.
\end{equation}

For this new basis, a basis for which $f_0$ is the timelike unit vector for the particle in its rest frame, we have

\begin{equation}\label{eqn:grahamSchmidtLorentz:685}
\begin{vmatrix}
f_0 \\
f_1
\end{vmatrix} 
=
\begin{vmatrix}
\gamma & \gamma \beta \\
\pm \gamma \beta & \pm \gamma
\end{vmatrix} 
=
\pm \gamma^2 ( 1 - \beta^2)
= \pm 1.
\end{equation}

Therefore, we want the positive sign, $f_0 = \gamma (\beta, 1)$.

Using the standard basis as the observer frame, let's figure out how to relate these to the coordinates of a vector in the particles rest frame

\begin{equation}\label{eqn:grahamSchmidtLorentz:700}
x = x^0 e_0 + x^1 e_1 = y^0 f_0 + y^1 f_1.
\end{equation}

We find

\begin{equation}\label{eqn:grahamSchmidtLorentz:720}
\begin{aligned}
\begin{bmatrix}
y^0 \\
y^1
\end{bmatrix}
&=
\begin{bmatrix}
x \cdot f^0 \\
x \cdot f^1 \\
\end{bmatrix} \\
&=
\begin{bmatrix}
x^0 e_0 \cdot f^0 + x^1 e_1 \cdot f^0 \\
x^0 e_0 \cdot f^1 + x^1 e_1 \cdot f^1
\end{bmatrix} \\
&=
\begin{bmatrix}
e_0 \cdot f^0 & e_1 \cdot f^0 \\
e_0 \cdot f^1 & e_1 \cdot f^1
\end{bmatrix}
\begin{bmatrix}
x^0 \\
x^1
\end{bmatrix}
\end{aligned}
\end{equation}

For our example, we have

\begin{equation}\label{eqn:grahamSchmidtLorentz:740}
\begin{aligned}
\begin{bmatrix}
e_0 \cdot f^0 & e_1 \cdot f^0 \\
e_0 \cdot f^1 & e_1 \cdot f^1
\end{bmatrix}
&=
\begin{bmatrix}
(1, 0) \cdot ( 1, \beta ) \gamma & (0, 1) \cdot ( 1, \beta ) \gamma \\
(1, 0) \cdot ( -\beta, -1) \gamma & (0, 1) \cdot ( -\beta, -1) \gamma
\end{bmatrix} \\
&=
\gamma 
\begin{bmatrix}
1 & - \beta \\
- \beta & 1
\end{bmatrix}
\end{aligned}
\end{equation}

We ``derive'' the Lorentz boost matrix, considering nothing more than the geometry of the vector space, and how to represent a vector in two different bases.

\subsection{Translation to tensor form.}

Observe that our transformation can be written in tensor form easily.  While we only considered a two dimension example \ref{eqn:grahamSchmidtLorentz:700} generalizes to

\begin{equation}\label{eqn:grahamSchmidtLorentz:760}
y^\alpha = x^\beta (e_\beta \cdot f^\alpha),
\end{equation}

or
\begin{align}\label{eqn:grahamSchmidtLorentz:780}
y^\alpha &= {\wedge_\beta}^\alpha x^\beta \\
{\wedge_\beta}^\alpha &= e_\beta \cdot f^\alpha.
\end{align}

It is natural to consider the coordinates $y^\alpha$ after transformation as the same vector that had the coordinates $x^\alpha$.  They are just representations under different bases, and unsuprisingly must then have the same invariant length.

We can see this from the tensor expressions as well, once \ref{eqn:grahamSchmidtLorentz:780} is expressed in terms of the basis vectors

\begin{align*}
y^\alpha 
&= {\wedge_\beta}^\alpha x^\beta \\
&= x^\beta (e_\beta \cdot f^\alpha) \\
y \cdot f^\alpha &= \\
&=(x \cdot e^\beta) (e_\beta \cdot f^\alpha) \\
\implies
(y \cdot f^\alpha) f_\alpha &= (x \cdot e^\beta) (e_\beta \cdot f^\alpha) f_\alpha \\
y 
&= (x \cdot e^\beta) e_\beta \\
&= x 
\end{align*}

\subsection{One more example.  Arbitrary boost in two spatial dimensions.}

%\bibliography{myrefs}\bibliographystyle{unsrturl}
%\bibliography{myrefs}\bibliographystyle{unsrtnat}
%\bibliography{myrefs}\bibliographystyle{abbrvnat}
\bibliography{myrefs}\bibliographystyle{plainnat}

\end{document}

%\EndArticle
%\EndNoBibArticle
