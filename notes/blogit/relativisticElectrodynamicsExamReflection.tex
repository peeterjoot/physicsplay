%
% Copyright � 2015 Peeter Joot.  All Rights Reserved.
% Licenced as described in the file LICENSE under the root directory of this GIT repository.
%
\documentclass[]{eliblog}

\usepackage{amsmath}
\usepackage{mathpazo}

%
% shorthand for bold symbols, convenient for vectors and matrices
%
\newcommand{\Ba}[0]{\mathbf{a}}
\newcommand{\Bb}[0]{\mathbf{b}}
\newcommand{\Bc}[0]{\mathbf{c}}
\newcommand{\Bd}[0]{\mathbf{d}}
\newcommand{\Be}[0]{\mathbf{e}}
\newcommand{\Bf}[0]{\mathbf{f}}
\newcommand{\Bg}[0]{\mathbf{g}}
\newcommand{\Bh}[0]{\mathbf{h}}
\newcommand{\Bi}[0]{\mathbf{i}}
\newcommand{\Bj}[0]{\mathbf{j}}
\newcommand{\Bk}[0]{\mathbf{k}}
\newcommand{\Bl}[0]{\mathbf{l}}
\newcommand{\Bm}[0]{\mathbf{m}}
\newcommand{\Bn}[0]{\mathbf{n}}
\newcommand{\Bo}[0]{\mathbf{o}}
\newcommand{\Bp}[0]{\mathbf{p}}
\newcommand{\Bq}[0]{\mathbf{q}}
\newcommand{\Br}[0]{\mathbf{r}}
\newcommand{\Bs}[0]{\mathbf{s}}
\newcommand{\Bt}[0]{\mathbf{t}}
\newcommand{\Bu}[0]{\mathbf{u}}
\newcommand{\Bv}[0]{\mathbf{v}}
\newcommand{\Bw}[0]{\mathbf{w}}
\newcommand{\Bx}[0]{\mathbf{x}}
\newcommand{\By}[0]{\mathbf{y}}
\newcommand{\Bz}[0]{\mathbf{z}}
\newcommand{\BA}[0]{\mathbf{A}}
\newcommand{\BB}[0]{\mathbf{B}}
\newcommand{\BC}[0]{\mathbf{C}}
\newcommand{\BD}[0]{\mathbf{D}}
\newcommand{\BE}[0]{\mathbf{E}}
\newcommand{\BF}[0]{\mathbf{F}}
\newcommand{\BG}[0]{\mathbf{G}}
\newcommand{\BH}[0]{\mathbf{H}}
\newcommand{\BI}[0]{\mathbf{I}}
\newcommand{\BJ}[0]{\mathbf{J}}
\newcommand{\BK}[0]{\mathbf{K}}
\newcommand{\BL}[0]{\mathbf{L}}
\newcommand{\BM}[0]{\mathbf{M}}
\newcommand{\BN}[0]{\mathbf{N}}
\newcommand{\BO}[0]{\mathbf{O}}
\newcommand{\BP}[0]{\mathbf{P}}
\newcommand{\BQ}[0]{\mathbf{Q}}
\newcommand{\BR}[0]{\mathbf{R}}
\newcommand{\BS}[0]{\mathbf{S}}
\newcommand{\BT}[0]{\mathbf{T}}
\newcommand{\BU}[0]{\mathbf{U}}
\newcommand{\BV}[0]{\mathbf{V}}
\newcommand{\BW}[0]{\mathbf{W}}
\newcommand{\BX}[0]{\mathbf{X}}
\newcommand{\BY}[0]{\mathbf{Y}}
\newcommand{\BZ}[0]{\mathbf{Z}}

\newcommand{\Bzero}[0]{\mathbf{0}}
\newcommand{\Btheta}[0]{\boldsymbol{\theta}}
\newcommand{\Btau}[0]{\boldsymbol{\tau}}
\newcommand{\Bomega}[0]{\boldsymbol{\omega}}

%
% shorthand for unit vectors
%
\newcommand{\acap}[0]{\hat{\Ba}}
\newcommand{\bcap}[0]{\hat{\Bb}}
\newcommand{\ccap}[0]{\hat{\Bc}}
\newcommand{\dcap}[0]{\hat{\Bd}}
\newcommand{\ecap}[0]{\hat{\Be}}
\newcommand{\fcap}[0]{\hat{\Bf}}
\newcommand{\gcap}[0]{\hat{\Bg}}
\newcommand{\hcap}[0]{\hat{\Bh}}
\newcommand{\icap}[0]{\hat{\Bi}}
\newcommand{\jcap}[0]{\hat{\Bj}}
\newcommand{\kcap}[0]{\hat{\Bk}}
\newcommand{\lcap}[0]{\hat{\Bl}}
\newcommand{\mcap}[0]{\hat{\Bm}}
\newcommand{\ncap}[0]{\hat{\Bn}}
\newcommand{\ocap}[0]{\hat{\Bo}}
\newcommand{\pcap}[0]{\hat{\Bp}}
\newcommand{\qcap}[0]{\hat{\Bq}}
\newcommand{\rcap}[0]{\hat{\Br}}
\newcommand{\scap}[0]{\hat{\Bs}}
\newcommand{\tcap}[0]{\hat{\Bt}}
\newcommand{\ucap}[0]{\hat{\Bu}}
\newcommand{\vcap}[0]{\hat{\Bv}}
\newcommand{\wcap}[0]{\hat{\Bw}}
\newcommand{\xcap}[0]{\hat{\Bx}}
\newcommand{\ycap}[0]{\hat{\By}}
\newcommand{\zcap}[0]{\hat{\Bz}}
\newcommand{\thetacap}[0]{\hat{\Btheta}}

%
% to write R^n and C^n in a distinguishable fashion.  Perhaps change this
% to the double lined characters upon figuring out how to do so.
%
\newcommand{\C}[1]{$\mathbb{C}^{#1}$}
\newcommand{\R}[1]{$\mathbb{R}^{#1}$}

%
% various generally useful helpers
%

% derivative of #1 wrt. #2:
\newcommand{\D}[2] {\frac {d#2} {d#1}}

\newcommand{\inv}[1]{\frac{1}{#1}}
\newcommand{\cross}[0]{\times}

\newcommand{\abs}[1]{\lvert{#1}\rvert}
\newcommand{\norm}[1]{\lVert{#1}\rVert}
\newcommand{\innerprod}[2]{\langle{#1}, {#2}\rangle}
\newcommand{\dotprod}[2]{{#1} \cdot {#2}}
\newcommand{\bdotprod}[2]{\left({#1} \cdot {#2}\right)}
\newcommand{\crossprod}[2]{{#1} \cross {#2}}
\newcommand{\tripleprod}[3]{\dotprod{\left(\crossprod{#1}{#2}\right)}{#3}}

\DeclareMathOperator{\Proj}{Proj}
\DeclareMathOperator{\Span}{span}
\DeclareMathOperator{\Sgn}{sgn}
\DeclareMathOperator{\Area}{Area}
\DeclareMathOperator{\Volume}{Volume}

%
% A few miscellaneous things specific to this document
%
\newcommand{\crossop}[1]{\crossprod{#1}{}}

% R2 vector.
\newcommand{\VectorTwo}[2]{
\begin{bmatrix}
 {#1} \\
 {#2}
\end{bmatrix}
}

\newcommand{\VectorN}[1]{
\begin{bmatrix}
{#1}_1 \\
{#1}_2 \\
\vdots \\
{#1}_N \\
\end{bmatrix}
}

\newcommand{\DETuvij}[4]{
\begin{vmatrix}
 {#1}_{#3} & {#1}_{#4} \\
 {#2}_{#3} & {#2}_{#4}
\end{vmatrix}
}

\newcommand{\DETuvwijk}[6]{
\begin{vmatrix}
 {#1}_{#4} & {#1}_{#5} & {#1}_{#6} \\
 {#2}_{#4} & {#2}_{#5} & {#2}_{#6} \\
 {#3}_{#4} & {#3}_{#5} & {#3}_{#6}
\end{vmatrix}
}

\newcommand{\DETuvwxijkl}[8]{
\begin{vmatrix}
 {#1}_{#5} & {#1}_{#6} & {#1}_{#7} & {#1}_{#8} \\
 {#2}_{#5} & {#2}_{#6} & {#2}_{#7} & {#2}_{#8} \\
 {#3}_{#5} & {#3}_{#6} & {#3}_{#7} & {#3}_{#8} \\
 {#4}_{#5} & {#4}_{#6} & {#4}_{#7} & {#4}_{#8} \\
\end{vmatrix}
}

%\newcommand{\DETuvwxyijklm}[10]{
%\begin{vmatrix}
% {#1}_{#6} & {#1}_{#7} & {#1}_{#8} & {#1}_{#9} & {#1}_{#10} \\
% {#2}_{#6} & {#2}_{#7} & {#2}_{#8} & {#2}_{#9} & {#2}_{#10} \\
% {#3}_{#6} & {#3}_{#7} & {#3}_{#8} & {#3}_{#9} & {#3}_{#10} \\
% {#4}_{#6} & {#4}_{#7} & {#4}_{#8} & {#4}_{#9} & {#4}_{#10} \\
% {#5}_{#6} & {#5}_{#7} & {#5}_{#8} & {#5}_{#9} & {#5}_{#10}
%\end{vmatrix}
%}

% R3 vector.
\newcommand{\VectorThree}[3]{
\begin{bmatrix}
 {#1} \\
 {#2} \\
 {#3}
\end{bmatrix}
}



\author{Peeter Joot}
\email{peeter.joot@gmail.com}

%\documentclass[]{eliblogwidescreen}

\usepackage{amsmath}
\usepackage{mathpazo}

%
% shorthand for bold symbols, convenient for vectors and matrices
%
\newcommand{\Ba}[0]{\mathbf{a}}
\newcommand{\Bb}[0]{\mathbf{b}}
\newcommand{\Bc}[0]{\mathbf{c}}
\newcommand{\Bd}[0]{\mathbf{d}}
\newcommand{\Be}[0]{\mathbf{e}}
\newcommand{\Bf}[0]{\mathbf{f}}
\newcommand{\Bg}[0]{\mathbf{g}}
\newcommand{\Bh}[0]{\mathbf{h}}
\newcommand{\Bi}[0]{\mathbf{i}}
\newcommand{\Bj}[0]{\mathbf{j}}
\newcommand{\Bk}[0]{\mathbf{k}}
\newcommand{\Bl}[0]{\mathbf{l}}
\newcommand{\Bm}[0]{\mathbf{m}}
\newcommand{\Bn}[0]{\mathbf{n}}
\newcommand{\Bo}[0]{\mathbf{o}}
\newcommand{\Bp}[0]{\mathbf{p}}
\newcommand{\Bq}[0]{\mathbf{q}}
\newcommand{\Br}[0]{\mathbf{r}}
\newcommand{\Bs}[0]{\mathbf{s}}
\newcommand{\Bt}[0]{\mathbf{t}}
\newcommand{\Bu}[0]{\mathbf{u}}
\newcommand{\Bv}[0]{\mathbf{v}}
\newcommand{\Bw}[0]{\mathbf{w}}
\newcommand{\Bx}[0]{\mathbf{x}}
\newcommand{\By}[0]{\mathbf{y}}
\newcommand{\Bz}[0]{\mathbf{z}}
\newcommand{\BA}[0]{\mathbf{A}}
\newcommand{\BB}[0]{\mathbf{B}}
\newcommand{\BC}[0]{\mathbf{C}}
\newcommand{\BD}[0]{\mathbf{D}}
\newcommand{\BE}[0]{\mathbf{E}}
\newcommand{\BF}[0]{\mathbf{F}}
\newcommand{\BG}[0]{\mathbf{G}}
\newcommand{\BH}[0]{\mathbf{H}}
\newcommand{\BI}[0]{\mathbf{I}}
\newcommand{\BJ}[0]{\mathbf{J}}
\newcommand{\BK}[0]{\mathbf{K}}
\newcommand{\BL}[0]{\mathbf{L}}
\newcommand{\BM}[0]{\mathbf{M}}
\newcommand{\BN}[0]{\mathbf{N}}
\newcommand{\BO}[0]{\mathbf{O}}
\newcommand{\BP}[0]{\mathbf{P}}
\newcommand{\BQ}[0]{\mathbf{Q}}
\newcommand{\BR}[0]{\mathbf{R}}
\newcommand{\BS}[0]{\mathbf{S}}
\newcommand{\BT}[0]{\mathbf{T}}
\newcommand{\BU}[0]{\mathbf{U}}
\newcommand{\BV}[0]{\mathbf{V}}
\newcommand{\BW}[0]{\mathbf{W}}
\newcommand{\BX}[0]{\mathbf{X}}
\newcommand{\BY}[0]{\mathbf{Y}}
\newcommand{\BZ}[0]{\mathbf{Z}}

\newcommand{\Bzero}[0]{\mathbf{0}}
\newcommand{\Btheta}[0]{\boldsymbol{\theta}}
\newcommand{\Btau}[0]{\boldsymbol{\tau}}
\newcommand{\Bomega}[0]{\boldsymbol{\omega}}

%
% shorthand for unit vectors
%
\newcommand{\acap}[0]{\hat{\Ba}}
\newcommand{\bcap}[0]{\hat{\Bb}}
\newcommand{\ccap}[0]{\hat{\Bc}}
\newcommand{\dcap}[0]{\hat{\Bd}}
\newcommand{\ecap}[0]{\hat{\Be}}
\newcommand{\fcap}[0]{\hat{\Bf}}
\newcommand{\gcap}[0]{\hat{\Bg}}
\newcommand{\hcap}[0]{\hat{\Bh}}
\newcommand{\icap}[0]{\hat{\Bi}}
\newcommand{\jcap}[0]{\hat{\Bj}}
\newcommand{\kcap}[0]{\hat{\Bk}}
\newcommand{\lcap}[0]{\hat{\Bl}}
\newcommand{\mcap}[0]{\hat{\Bm}}
\newcommand{\ncap}[0]{\hat{\Bn}}
\newcommand{\ocap}[0]{\hat{\Bo}}
\newcommand{\pcap}[0]{\hat{\Bp}}
\newcommand{\qcap}[0]{\hat{\Bq}}
\newcommand{\rcap}[0]{\hat{\Br}}
\newcommand{\scap}[0]{\hat{\Bs}}
\newcommand{\tcap}[0]{\hat{\Bt}}
\newcommand{\ucap}[0]{\hat{\Bu}}
\newcommand{\vcap}[0]{\hat{\Bv}}
\newcommand{\wcap}[0]{\hat{\Bw}}
\newcommand{\xcap}[0]{\hat{\Bx}}
\newcommand{\ycap}[0]{\hat{\By}}
\newcommand{\zcap}[0]{\hat{\Bz}}
\newcommand{\thetacap}[0]{\hat{\Btheta}}

%
% to write R^n and C^n in a distinguishable fashion.  Perhaps change this
% to the double lined characters upon figuring out how to do so.
%
\newcommand{\C}[1]{$\mathbb{C}^{#1}$}
\newcommand{\R}[1]{$\mathbb{R}^{#1}$}

%
% various generally useful helpers
%

% derivative of #1 wrt. #2:
\newcommand{\D}[2] {\frac {d#2} {d#1}}

\newcommand{\inv}[1]{\frac{1}{#1}}
\newcommand{\cross}[0]{\times}

\newcommand{\abs}[1]{\lvert{#1}\rvert}
\newcommand{\norm}[1]{\lVert{#1}\rVert}
\newcommand{\innerprod}[2]{\langle{#1}, {#2}\rangle}
\newcommand{\dotprod}[2]{{#1} \cdot {#2}}
\newcommand{\bdotprod}[2]{\left({#1} \cdot {#2}\right)}
\newcommand{\crossprod}[2]{{#1} \cross {#2}}
\newcommand{\tripleprod}[3]{\dotprod{\left(\crossprod{#1}{#2}\right)}{#3}}

\DeclareMathOperator{\Proj}{Proj}
\DeclareMathOperator{\Span}{span}
\DeclareMathOperator{\Sgn}{sgn}
\DeclareMathOperator{\Area}{Area}
\DeclareMathOperator{\Volume}{Volume}

%
% A few miscellaneous things specific to this document
%
\newcommand{\crossop}[1]{\crossprod{#1}{}}

% R2 vector.
\newcommand{\VectorTwo}[2]{
\begin{bmatrix}
 {#1} \\
 {#2}
\end{bmatrix}
}

\newcommand{\VectorN}[1]{
\begin{bmatrix}
{#1}_1 \\
{#1}_2 \\
\vdots \\
{#1}_N \\
\end{bmatrix}
}

\newcommand{\DETuvij}[4]{
\begin{vmatrix}
 {#1}_{#3} & {#1}_{#4} \\
 {#2}_{#3} & {#2}_{#4}
\end{vmatrix}
}

\newcommand{\DETuvwijk}[6]{
\begin{vmatrix}
 {#1}_{#4} & {#1}_{#5} & {#1}_{#6} \\
 {#2}_{#4} & {#2}_{#5} & {#2}_{#6} \\
 {#3}_{#4} & {#3}_{#5} & {#3}_{#6}
\end{vmatrix}
}

\newcommand{\DETuvwxijkl}[8]{
\begin{vmatrix}
 {#1}_{#5} & {#1}_{#6} & {#1}_{#7} & {#1}_{#8} \\
 {#2}_{#5} & {#2}_{#6} & {#2}_{#7} & {#2}_{#8} \\
 {#3}_{#5} & {#3}_{#6} & {#3}_{#7} & {#3}_{#8} \\
 {#4}_{#5} & {#4}_{#6} & {#4}_{#7} & {#4}_{#8} \\
\end{vmatrix}
}

%\newcommand{\DETuvwxyijklm}[10]{
%\begin{vmatrix}
% {#1}_{#6} & {#1}_{#7} & {#1}_{#8} & {#1}_{#9} & {#1}_{#10} \\
% {#2}_{#6} & {#2}_{#7} & {#2}_{#8} & {#2}_{#9} & {#2}_{#10} \\
% {#3}_{#6} & {#3}_{#7} & {#3}_{#8} & {#3}_{#9} & {#3}_{#10} \\
% {#4}_{#6} & {#4}_{#7} & {#4}_{#8} & {#4}_{#9} & {#4}_{#10} \\
% {#5}_{#6} & {#5}_{#7} & {#5}_{#8} & {#5}_{#9} & {#5}_{#10}
%\end{vmatrix}
%}

% R3 vector.
\newcommand{\VectorThree}[3]{
\begin{bmatrix}
 {#1} \\
 {#2} \\
 {#3}
\end{bmatrix}
}



\author{Peeter Joot}
\email{peeter.joot@gmail.com}


\chapter{Some exam reflection.}
\label{chap:relativisticElectrodynamicsExamReflection}
%\useCCL
\blogpage{http://sites.google.com/site/peeterjoot/math2011/relativisticElectrodynamicsExamReflection.pdf}
\date{April 13, 2011}
\revisionInfo{relativisticElectrodynamicsExamReflection.tex}

\beginArtWithToc
%\beginArtNoToc

\section{Charged particle in a circle.}

From the 2008 PHY353 exam, given a particle of charge $q$ moving in a circle of radius $a$ at constant angular frequency $\omega$.

\begin{itemize}
\item Find the Lienard-Wiechert potentials for points on the z-axis.
\item Find the electric and magnetic fields at the center.
\end{itemize}

When I tried this I did it for points not just on the z-axis.  It turns out that we also got this question on the exam (but stated slightly differently).  Since I'll not get to see my exam solution again, let's work through this at a leisurely rate, and see if things look right.  The problem as stated in this old practise exam is easier since it doesn't say to calculate the fields from the four potentials, so there was nothing preventing one from just grinding away and plugging stuff into the Lienard-Wiechert equations for the fields (as I did when I tried it for practise).

Let's set up our coordinate system in cylindrical coordinates.  For the charged particle and the point that we measure the field, with $i = \Be_1 \Be_2$

\begin{align}\label{eqn:relativisticElectrodynamicsExamReflection:n}
\Bx(t) &= a \Be_1 e^{i \omega t} \\
\Br &= z \Be_3 + \rho \Be_1 e^{i \phi}
\end{align}

Here I'm using the geometric product of vectors (if that's unfamiliar then just substitute

\begin{equation}\label{eqn:relativisticElectrodynamicsExamReflection:n}
\{\Be_1, \Be_2, \Be_3\} \rightarrow \{\sigma_1, \sigma_2, \sigma_3\}
\end{equation}

We can do that since the Pauli matrices also have the same semantics (with a small difference since the geometric square of a unit vector is defined as the unit scalar, whereas the Pauli matrix square is the identity matrix).  The semantics we require of this vector product are just $\Be_\alpha^2 = 1$ and $\Be_\alpha \Be_\beta = - \Be_\beta \Be_\alpha$ for any $\alpha \ne \beta$.

I'll also be loose with notation and use $\Real(X) = \gpgradezero{X}$ to select the scalar part of a multivector (or with the Pauli matrixes, the portion proportional to the identity matrix).

Our task is to compute the Lienard-Wiechert potentials.  Those are

\begin{align}\label{eqn:relativisticElectrodynamicsExamReflection:n}
\phi &= \frac{q}{R^\conj} \\
\BA &= \phi \frac{\Bv}{c},
\end{align}

where

\begin{align}\label{eqn:relativisticElectrodynamicsExamReflection:n}
\BR &= \Br - \Bx(t_r) \\
R = \Abs{\BR} &= c (t - t_r) \\
R^\conj &= R - \frac{\Bv}{c} \cdot \BR \\
\Bv &= \frac{d\Bx}{dt_r}.
\end{align}

We'll need (eventually)

\begin{align}\label{eqn:relativisticElectrodynamicsExamReflection:n}
\Bv &= a \omega \Be_2 e^{i \omega t_r} \\
\dot{\Bv} &= -a \omega^2 \Be_1 e^{i \omega t_r},
\end{align}

and also need our retarded distance vector

\begin{equation}\label{eqn:relativisticElectrodynamicsExamReflection:n}
\BR = z \Be_3 + \Be_1 (\rho e^{i \phi} - a e^{i \omega t_r} ),
\end{equation}

From this we have

\begin{align*}
R^2 
&= z^2 + \Abs{\Be_1 (\rho e^{i \phi} - a e^{i \omega t_r} )}^2 \\
&= z^2 + \rho^2 + a^2 - 2 \rho a (\Be_1 \rho e^{i \phi}) \cdot (\Be_1 e^{i \omega t_r}) \\
&= z^2 + \rho^2 + a^2 - 2 \rho a \Real( e^{ i(\phi - \omega t_r) } ) \\
&= z^2 + \rho^2 + a^2 - 2 \rho a \cos(\phi - \omega t_r)
\end{align*}

So
\begin{equation}\label{eqn:relativisticElectrodynamicsExamReflection:n}
R = \sqrt{z^2 + \rho^2 + a^2 - 2 \rho a \cos( \phi - \omega t_r ) }.
\end{equation}

Next we need

\begin{align*}
\BR \cdot \Bv/c
&= 
(z \Be_3 + \Be_1 (\rho e^{i \phi} - a e^{i \omega t_r} )) \cdot  
\left(a \frac{\omega}{c} \Be_2 e^{i \omega t_r} \right) \\
&=
a \frac{\omega }{c}
\Real(
i (\rho e^{-i \phi} - a e^{-i \omega t_r} ) e^{i \omega t_r} ) \\
&=
a \frac{\omega }{c}
\rho \Real( i e^{-i \phi + i \omega t_r} ) \\
&=
a \frac{\omega }{c}
\rho \sin(\phi - \omega t_r)
\end{align*}

So we have

\begin{equation}\label{eqn:relativisticElectrodynamicsExamReflection:n}
R^\conj = \sqrt{z^2 + \rho^2 + a^2 - 2 \rho a \cos( \phi - \omega t_r ) }
-a \frac{\omega }{c} \rho \sin(\phi - \omega t_r)
\end{equation}

\subsection{General fields for this system.}

\subsection{An approximation near the center.}

Unlike the old exam I did, where it didn't specify that the potentials had to be used to calculate the fields, and the problem was reduced to one of algebraic manipulation, our exam explicitly asked for the potentials to be used to calculate the fields.

There was also the restriction to compute them near the center.  Setting $\rho = 0$ so that we are looking only near the z-axis, we have

\begin{align}\label{eqn:relativisticElectrodynamicsExamReflection:n}
\phi &= \frac{q}{\sqrt{z^2 + a^2}} \\
\BA 
&= 
\frac{q a \omega \Be_2 e^{i \omega t_r} }{\sqrt{z^2 + a^2}} 
= 
\frac{q a \omega (-\sin \omega t_r, \cos\omega t_r, 0)}{\sqrt{z^2 + a^2}} \\
t_r &= t - R/c = t - \sqrt{z^2 + a^2}/c
\end{align}

Now we are set to calculate the electric and magnetic fields directly from these.  Observe that we have a spatial dependence in due to the $t_r$ quantities and that will have an effect when we operate with the gradient.  

In the exam I'd asked Simon (our TA) if this question was asking for the fields at the origin (ie: in the plane of the charge's motion in the center) or along the z-axis.  He said in the plane.  That would simplify things, but perhaps too much since $\phi$ becomes constant (in my exam attempt I somehow fudged this to get what I wanted for the $v = 0$ case, but that must have been wrong, and was the result of rushed work).

Let's now proceed with the field calculation from these potentials

\begin{align}\label{eqn:relativisticElectrodynamicsExamReflection:n}
\BE &= - \spacegrad \phi - \inv{c} \PD{t}{\BA} \\
\BB &= \spacegrad \cross \BA.
\end{align}

For the electric field we need

\begin{align*}
\spacegrad \phi 
&= q \Be_3 \partial_z (z^2 + a^2)^{-1/2} \\
&= -q \Be_3 \frac{z}{(\sqrt{z^2 + a^2})^3},
\end{align*}

and

\begin{equation}\label{eqn:relativisticElectrodynamicsExamReflection:n}
\PD{t}{\BA} =
\frac{q a \omega^2 \Be_2 \Be_1 \Be_2 e^{i \omega t_r} }{\sqrt{z^2 + a^2}}.
\end{equation}

Putting these together, our electric field near the z-axis is

\begin{equation}\label{eqn:relativisticElectrodynamicsExamReflection:n}
\BE = 
-q \Be_3 \frac{z}{(\sqrt{z^2 + a^2})^3}
+
\frac{q a \omega^2 \Be_1 e^{i \omega t_r} }{c \sqrt{z^2 + a^2}}.
\end{equation}

(another mistake I made on the exam, since I somehow fooled myself into forcing what I knew had to be in the gradient term, despite having essentially a constant scalar potential (having taken $z = 0$)).

What do we get for the magnetic field.  In that case we have

\begin{align*}
\spacegrad \cross \BA(z)
&=
\Be_\alpha \cross \partial_\alpha \BA \\
&=
\Be_3 \cross \partial_z \BA \\
&=
\Be_3 \cross (\Be_2 e^{i \omega t_r} ) q a \omega \PD{z}{} \inv{\sqrt{z^2 + a^2}} 
+
q a \omega \inv{\sqrt{z^2 + a^2}} \Be_3 \cross (\Be_2 \partial_z e^{i \omega t_r} ) \\
&=
-\Be_3 \cross (\Be_2 e^{i \omega t_r} ) q a \omega \frac{z}{(\sqrt{z^2 + a^2})^3} 
+
q a \omega \inv{\sqrt{z^2 + a^2}} \Be_3 \cross \left( \Be_1 \frac{\omega}{c} e^{i \omega t_r} \frac{z}{\sqrt{z^2 + a^2} } \right)
\end{align*}

%\begin{align*}
%\Be_2 \partial_z e^{i \omega t_r} 
%&=
%\Be_2 \Be_1 \Be_2 \omega e^{i \omega t_r} \partial_z \left( -\inv{c} \sqrt{z^2 + a^2} \right) \\
%&=
%\end{align*}

For the direction vectors in the cross products above we have

\begin{align*}
\Be_3 \cross (\Be_2 e^{i \mu})
&=
\Be_3 \cross (\Be_2 \cos\mu - \Be_1 \sin\mu) \\
&=
-\Be_1 \cos\mu - \Be_2 \sin\mu \\
&=
-\Be_1 e^{i \mu}
\end{align*}

and

\begin{align*}
\Be_3 \cross (\Be_1 e^{i \mu})
&=
\Be_3 \cross (\Be_1 \cos\mu + \Be_2 \sin\mu) \\
&=
\Be_2 \cos\mu - \Be_1 \sin\mu \\
&=
\Be_2 e^{i \mu}
\end{align*}

Putting everything, and summarizing results for the fields, we have

\begin{align}\label{eqn:relativisticElectrodynamicsExamReflection:n}
\BE &= 
-q \Be_3 \frac{z}{(\sqrt{z^2 + a^2})^3}
+
\frac{q a \omega^2 \Be_1 e^{i \omega t_r} }{c \sqrt{z^2 + a^2}} \\
\BB 
&= \frac{q a \omega z}{ z^2 + a^2} \left( \frac{\Be_1}{\sqrt{z^2 + a^2}} + \frac{\omega}{c} \Be_2 \right) e^{i \omega t_r}
\end{align}

If all this worked we should also have

\begin{equation}\label{eqn:relativisticElectrodynamicsExamReflection:n}
\BB = \frac{z \Be_3 - \Be_1 a e^{i \omega t_r}}{\sqrt{z^2 + a^2}} \cross \BE
\end{equation}

However, if I do this check I get

\begin{equation}\label{eqn:relativisticElectrodynamicsExamReflection:n}
\BB = \frac{q a z}{z^2 + a^2} \left( \inv{z^2 + a^2} - \frac{\omega^2 }{c} \right) (\Be_1 e^{i \omega t_r}) \cross \Be_3
\end{equation}

Notice that the dimensions in the middle are all messed up.  Did I get my electric field calculation wrong?  Tricky to get this all right.  With so much grunt work calculation involved, perhaps its time to give up and resort to something like Mathematica.

\section{Collison of photon and electron.}

I made a dumb error on the exam on this one.  I setup the four momentum conservation statement, but then didn't multiply out the cross terms properly.  This led me to incorrectly assume that I had to try doing this the hard way (something akin to what I did on the midterm).  Simon later told us in the tutorial the simple way, and that's all we needed here too.  Here's the setup.

An electron at rest initially has four momentum

\begin{equation}\label{eqn:relativisticElectrodynamicsExamReflection:n}
(m c, 0)
\end{equation}

where the incoming photon has four momentum

\begin{equation}\label{eqn:relativisticElectrodynamicsExamReflection:n}
\left(\hbar \frac{\omega}{c}, \hbar \Bk \right)
\end{equation}

After the collison our electron has some velocity so its four momentum becomes (say)

\begin{equation}\label{eqn:relativisticElectrodynamicsExamReflection:n}
\gamma (m c, m \Bv),
\end{equation}

and our new photon, going off on an angle $\theta$ relative to $\Bk$ has four momentum

\begin{equation}\label{eqn:relativisticElectrodynamicsExamReflection:n}
\left(\hbar \frac{\omega'}{c}, \hbar \Bk' \right)
\end{equation}

Our conservation relationship is thus
\begin{equation}\label{eqn:relativisticElectrodynamicsExamReflection:n}
(m c, 0) + \left(\hbar \frac{\omega}{c}, \hbar \Bk \right)
=
\gamma (m c, m \Bv)
+
\left(\hbar \frac{\omega'}{c}, \hbar \Bk' \right)
\end{equation}

I squared both sides, but dropped my cross terms, which was just plain wrong, and costly for both time and effort on the exam.  What I should have done was just

\begin{equation}\label{eqn:relativisticElectrodynamicsExamReflection:n}
\gamma (m c, m \Bv) =
(m c, 0) + \left(\hbar \frac{\omega}{c}, \hbar \Bk \right)
-\left(\hbar \frac{\omega'}{c}, \hbar \Bk' \right),
\end{equation}

and then square this (really making contractions of the form $p_i p^i$).  That gives (and this time keeping my cross terms)

\begin{align*}
(\gamma (m c, m \Bv) )^2 
&= \gamma^2 m^2 (c^2 - \Bv^2) \\
&= m^2 c^2 \\
&=
m^2 c^2 + 0 + 0
+ 2 (m c, 0) 
\cdot \left(\hbar \frac{\omega}{c}, \hbar \Bk \right)
- 2 (m c, 0) \cdot \left(\hbar \frac{\omega'}{c}, \hbar \Bk' \right)
- 2 
\cdot \left(\hbar \frac{\omega}{c}, \hbar \Bk \right)
\cdot \left(\hbar \frac{\omega'}{c}, \hbar \Bk' \right) \\
&=
m^2 c^2 + 2 m c \hbar \frac{\omega}{c} - 2 m c \hbar \frac{\omega'}{c}
- 2\hbar^2 \left(
\frac{\omega}{c} \frac{\omega'}{c}
- 
\Bk \cdot \Bk'
\right) \\
&=
m^2 c^2 + 2 m c \hbar \frac{\omega}{c} - 2 m c \hbar \frac{\omega'}{c}
- 2\hbar^2 
\frac{\omega}{c} \frac{\omega'}{c} (1 - \cos\theta)
\end{align*}

Rearranging a bit we have

\begin{equation}\label{eqn:relativisticElectrodynamicsExamReflection:n}
\omega' \left( m + \frac{\hbar \omega}{c^2} ( 1 - \cos\theta ) \right) = m \omega,
\end{equation}

or
\begin{equation}\label{eqn:relativisticElectrodynamicsExamReflection:n}
\omega' = \frac{\omega}{
1 + \frac{\hbar \omega}{m c^2} ( 1 - \cos\theta ) 
}
\end{equation}

%\EndArticle
\EndNoBibArticle
