%
% Copyright � 2013 Peeter Joot.  All Rights Reserved.
% Licenced as described in the file LICENSE under the root directory of this GIT repository.
%
\makeproblem{Follow up on reading assignments}{condensedMatter:problemSet1:1}{ 
({\em The reading assignments covered sections 1, 1.1, 1.5 and 1.6 of Ibach and Luth.}) 

\makesubproblem{Explain $5s$ $4d$ orbital filling ordering.}{condensedMatter:problemSet1:1a}

In a hydrogenic atom (nuclear charge $Ze$, only one electron) 
the $4d$ levels have a lower energy than the 
$5s$ levels.  According to the periodic table, however, the $5s$ levels become 
occupied before the $4d$ levels do. Explain why.

\makesubproblem{Explain power of Van der Waals potential.}{condensedMatter:problemSet1:1b}

Briefly explain why the Van der Waals potential has a $1/r^6$ dependence. 

} % makeproblem

\makeanswer{condensedMatter:problemSet1:1}{ 
\makeSubAnswer{}{condensedMatter:problemSet1:1a}

The $4d$ levels has a lower energy than the $5s$ level in a single electron system.  In a multiple electron system completely filled (and perhaps partially filled) orbitals have the effect of shielding additional electrons from a subset of the nuclear charge, reducing the effective total charge of the system with respect to that additional electron.  This shifts the $d$, $p$, $f$ orbitals up in the staircase like fashion illustrated in class, and puts $5s$ below $4d$ in such multiple electron systems.

\makeSubAnswer{}{condensedMatter:problemSet1:1b}

TODO.
}
