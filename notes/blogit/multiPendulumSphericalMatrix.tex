%
% Copyright � 2012 Peeter Joot.  All Rights Reserved.
% Licenced as described in the file LICENSE under the root directory of this GIT repository.
%
%{
\newcommand{\authorname}{Peeter Joot}
\newcommand{\email}{peeterjoot@protonmail.com}
\newcommand{\basename}{FIXMEbasenameUndefined}
\newcommand{\dirname}{notes/FIXMEdirnameUndefined/}

\renewcommand{\basename}{multiPendulumSphericalMatrix}
\renewcommand{\dirname}{notes/classicalmechanics/}
%\newcommand{\dateintitle}{}
%\newcommand{\keywords}{}

\newcommand{\authorname}{Peeter Joot}
\newcommand{\onlineurl}{http://sites.google.com/site/peeterjoot2/math2013/\basename.pdf}
\newcommand{\sourcepath}{\dirname\basename.tex}
\newcommand{\generatetitle}[1]{\chapter{#1}}

\newcommand{\vcsinfo}{%
\section*{}
\noindent{\color{DarkOliveGreen}{\rule{\linewidth}{0.1mm}}}
\paragraph{Document version}
%\paragraph{\color{Maroon}{Document version}}
{
\small
\begin{itemize}
\item Available online at:\\ 
\href{\onlineurl}{\onlineurl}
\item Git Repository: \input{./.revinfo/gitRepo.tex}
\item Source: \sourcepath
\item last commit: \input{./.revinfo/gitCommitString.tex}
\item commit date: \input{./.revinfo/gitCommitDate.tex}
\end{itemize}
}
}

%\PassOptionsToPackage{dvipsnames,svgnames}{xcolor}
\PassOptionsToPackage{square,numbers}{natbib}
\documentclass{scrreprt}

\usepackage[left=2cm,right=2cm]{geometry}
\usepackage[svgnames]{xcolor}
\usepackage{peeters_layout}

\usepackage{natbib}

\usepackage[
colorlinks=true,
bookmarks=false,
pdfauthor={\authorname, \email},
backref 
]{hyperref}

% http://tex.stackexchange.com/questions/75773/how-to-reference-problems-by-the-text-label-in-an-exercise-envioronment
\usepackage[english]{cleveref}
\crefname{Exercise}{exercise}{exercises}
\Crefname{Exercise}{Exercise}{Exercises}

\RequirePackage{titlesec}
\RequirePackage{ifthen}

% http://stackoverflow.com/questions/4932910/date-in-the-tabular-environment
\makeatletter
\let\insertdate\@date
\makeatother

\titleformat{\chapter}[display]
{\bfseries\Large}
{\color{DarkSlateGrey}\filleft \authorname
\ifthenelse{\isundefined{\studentnumber}}{}{\\ \studentnumber}
\ifthenelse{\isundefined{\email}}{}{\\ \email}
\ifthenelse{\isundefined{\dateintitle}}{}{\\ \insertdate}
%\ifthenelse{\isundefined{\coursename}}{}{\\ \coursename} % put in title instead.
}
{4ex}
{\color{DarkOliveGreen}{\titlerule}\color{Maroon}
\vspace{2ex}%
\filright}
[\vspace{2ex}%
\color{DarkOliveGreen}\titlerule
]

\newcommand{\beginArtWithToc}[0]{\begin{document}\tableofcontents}
\newcommand{\beginArtNoToc}[0]{\begin{document}}
\newcommand{\EndNoBibArticle}[0]{\end{document}}
\newcommand{\EndArticle}[0]{\bibliography{Bibliography}\bibliographystyle{plainnat}\end{document}}

% 
%\newcommand{\citep}[1]{\cite{#1}}

\colorSectionsForArticle



%\usepackage{peeters_layout_exercise}
%\usepackage{peeters_braket}
\usepackage{peeters_figures}
\usepackage{macros_cal}

\newcommand{\chapcite}[1]{\ref{chap:#1}}
\newcommand{\gpgradezeroNoOp}[1]{{#1}}

%\newcommand{\small}[1]{#1}

%%% %\documentclass[11pt,twocolumn]{article}
%%%
%%% % small font:
%%% %\documentclass[11pt]{article}
%%% %\setlength\topmargin{-0.5in}
%%% %\setlength\columnsep{0.2in}
%%% %\setlength\headsep{0.0in}
%%% %\setlength\textheight{9.5in}
%%% %\setlength\textwidth{7in}
%%% %\setlength\oddsidemargin{-0.25in}
%%% %\setlength\evensidemargin{-0.25in}
%%%
%%% % readable?  Font still seems small?
%%% \documentclass[11pt]{article}
%%% \setlength{\textwidth}{\paperwidth}
%%% \addtolength{\textwidth}{-2in}
%%% \setlength{\oddsidemargin}{0pt}
%%% \setlength{\evensidemargin}{0pt}
%%% \setlength{\headheight}{15pt}
%%% \setlength{\headsep}{15pt}
%%% \setlength{\topmargin}{0in}
%%% \addtolength{\topmargin}{-\headheight}
%%% \addtolength{\topmargin}{-\headsep}
%%% \setlength{\footskip}{29pt}
%%% \setlength{\textheight}{\paperheight}
%%% \addtolength{\textheight}{-2.2in}
%%% \setlength{\marginparwidth}{.5in}
%%% \setlength{\marginparsep}{5pt}
%%%
%%%
%%% %\usepackage{mathpazo}
%%% \usepackage{color}
%%% \usepackage{amsmath}
%%% \usepackage{amsfonts}
%%% \usepackage{graphicx}
%%% %\usepackage[bookmarks=false]{hyperref}
%%% \usepackage{hyperref}
%%% \usepackage{subfigure}
%%% \usepackage{titlesec}
%%% \usepackage{indentfirst}
%%%
%%% \newtheorem{theorem}{Theorem}[section]
%%% \newtheorem{definition}[theorem]{Definition}
%%% \newtheorem{axiom}[theorem]{Axiom}
%%%
%%% \newcommand{\Abs}[1]{{\left\lvert{#1}\right\rvert}}
%%% \newcommand{\evalbar}[2]{{\left.{#1}\right\vert}_{#2}}
%%% \newcommand{\evalnobar}[2]{{#1}_{#2}}
%%% \newcommand{\Bu}[0]{\mathbf{u}}
%%% \newcommand{\Bv}[0]{\mathbf{v}}
%%% \newcommand{\Be}[0]{\mathbf{e}}
%%% \newcommand{\Ba}[0]{\mathbf{a}}
%%% \newcommand{\Bb}[0]{\mathbf{b}}
%%% \newcommand{\Bc}[0]{\mathbf{c}}
%%% \newcommand{\Br}[0]{\mathbf{r}}
%%% \newcommand{\Bx}[0]{\mathbf{x}}
%%% \newcommand{\Bk}[0]{\mathbf{k}}
%%% \newcommand{\Bq}[0]{\mathbf{q}}
%%% \newcommand{\Bs}[0]{\mathbf{s}}
%%% \newcommand{\Bt}[0]{\mathbf{t}}
%%% \newcommand{\rcap}[0]{\hat{\Br}}
%%% \newcommand{\inv}[1]{\frac{1}{#1}}
%%% \newcommand{\grad}[0]{\nabla}
%%% \newcommand{\LL}[0]{\calL}
%%% \newcommand{\PD}[2]{\frac{\partial {#2}}{\partial {#1}}}
%%% \newcommand{\PDi}[2]{{\partial {#2}}/{\partial {#1}}}
%%% \newcommand{\gpgrade}[2] {{\left\langle{{#1}}\right\rangle}_{#2}}
%%% \newcommand{\gpgradezero}[1] {\gpgrade{#1}{}}
%%% \newcommand{\gpgradeone}[1] {\gpgrade{#1}{1}}
%%% \newcommand{\BTheta}[0]{\boldsymbol{\Theta}}
%%% \newcommand{\Brho}[0]{\boldsymbol{\rho}}
%%% \newcommand{\T}[0]{\text{T}}
%%%
%%% \begin{document}

%\chapter{Lagrangian and Euler-Lagrange equation evaluation for the spherical N-pendulum problem}
%\author{Peeter Joot
%\smallskip\\
%\small{e-mail: peeterjoot\(@\)protonmail.com}}
%\date{\small{\today}}
%\maketitle

\newcommand{\nbref}[1]{%
%\itemRef{classicalmechanics}{#1}%
%\index{Mathematica}%
fixme:nbref:#1
}

\beginArtNoToc

\generatetitle{Lagrangian and Euler-Lagrange equation evaluation for the spherical N-pendulum problem}
%
% Copyright � 2012 Peeter Joot.  All Rights Reserved.
% Licenced as described in the file LICENSE under the root directory of this GIT repository.
%
%{
\newcommand{\authorname}{Peeter Joot}
\newcommand{\email}{peeterjoot@protonmail.com}
\newcommand{\basename}{FIXMEbasenameUndefined}
\newcommand{\dirname}{notes/FIXMEdirnameUndefined/}

\renewcommand{\basename}{multiPendulumSphericalMatrix}
\renewcommand{\dirname}{notes/classicalmechanics/}
%\newcommand{\dateintitle}{}
%\newcommand{\keywords}{}

\newcommand{\authorname}{Peeter Joot}
\newcommand{\onlineurl}{http://sites.google.com/site/peeterjoot2/math2013/\basename.pdf}
\newcommand{\sourcepath}{\dirname\basename.tex}
\newcommand{\generatetitle}[1]{\chapter{#1}}

\newcommand{\vcsinfo}{%
\section*{}
\noindent{\color{DarkOliveGreen}{\rule{\linewidth}{0.1mm}}}
\paragraph{Document version}
%\paragraph{\color{Maroon}{Document version}}
{
\small
\begin{itemize}
\item Available online at:\\ 
\href{\onlineurl}{\onlineurl}
\item Git Repository: \input{./.revinfo/gitRepo.tex}
\item Source: \sourcepath
\item last commit: \input{./.revinfo/gitCommitString.tex}
\item commit date: \input{./.revinfo/gitCommitDate.tex}
\end{itemize}
}
}

%\PassOptionsToPackage{dvipsnames,svgnames}{xcolor}
\PassOptionsToPackage{square,numbers}{natbib}
\documentclass{scrreprt}

\usepackage[left=2cm,right=2cm]{geometry}
\usepackage[svgnames]{xcolor}
\usepackage{peeters_layout}

\usepackage{natbib}

\usepackage[
colorlinks=true,
bookmarks=false,
pdfauthor={\authorname, \email},
backref 
]{hyperref}

% http://tex.stackexchange.com/questions/75773/how-to-reference-problems-by-the-text-label-in-an-exercise-envioronment
\usepackage[english]{cleveref}
\crefname{Exercise}{exercise}{exercises}
\Crefname{Exercise}{Exercise}{Exercises}

\RequirePackage{titlesec}
\RequirePackage{ifthen}

% http://stackoverflow.com/questions/4932910/date-in-the-tabular-environment
\makeatletter
\let\insertdate\@date
\makeatother

\titleformat{\chapter}[display]
{\bfseries\Large}
{\color{DarkSlateGrey}\filleft \authorname
\ifthenelse{\isundefined{\studentnumber}}{}{\\ \studentnumber}
\ifthenelse{\isundefined{\email}}{}{\\ \email}
\ifthenelse{\isundefined{\dateintitle}}{}{\\ \insertdate}
%\ifthenelse{\isundefined{\coursename}}{}{\\ \coursename} % put in title instead.
}
{4ex}
{\color{DarkOliveGreen}{\titlerule}\color{Maroon}
\vspace{2ex}%
\filright}
[\vspace{2ex}%
\color{DarkOliveGreen}\titlerule
]

\newcommand{\beginArtWithToc}[0]{\begin{document}\tableofcontents}
\newcommand{\beginArtNoToc}[0]{\begin{document}}
\newcommand{\EndNoBibArticle}[0]{\end{document}}
\newcommand{\EndArticle}[0]{\bibliography{Bibliography}\bibliographystyle{plainnat}\end{document}}

% 
%\newcommand{\citep}[1]{\cite{#1}}

\colorSectionsForArticle



%\usepackage{peeters_layout_exercise}
%\usepackage{peeters_braket}
\usepackage{peeters_figures}
\usepackage{macros_cal}

\newcommand{\chapcite}[1]{\ref{chap:#1}}
\newcommand{\gpgradezeroNoOp}[1]{{#1}}

%\newcommand{\small}[1]{#1}

%%% %\documentclass[11pt,twocolumn]{article}
%%%
%%% % small font:
%%% %\documentclass[11pt]{article}
%%% %\setlength\topmargin{-0.5in}
%%% %\setlength\columnsep{0.2in}
%%% %\setlength\headsep{0.0in}
%%% %\setlength\textheight{9.5in}
%%% %\setlength\textwidth{7in}
%%% %\setlength\oddsidemargin{-0.25in}
%%% %\setlength\evensidemargin{-0.25in}
%%%
%%% % readable?  Font still seems small?
%%% \documentclass[11pt]{article}
%%% \setlength{\textwidth}{\paperwidth}
%%% \addtolength{\textwidth}{-2in}
%%% \setlength{\oddsidemargin}{0pt}
%%% \setlength{\evensidemargin}{0pt}
%%% \setlength{\headheight}{15pt}
%%% \setlength{\headsep}{15pt}
%%% \setlength{\topmargin}{0in}
%%% \addtolength{\topmargin}{-\headheight}
%%% \addtolength{\topmargin}{-\headsep}
%%% \setlength{\footskip}{29pt}
%%% \setlength{\textheight}{\paperheight}
%%% \addtolength{\textheight}{-2.2in}
%%% \setlength{\marginparwidth}{.5in}
%%% \setlength{\marginparsep}{5pt}
%%%
%%%
%%% %\usepackage{mathpazo}
%%% \usepackage{color}
%%% \usepackage{amsmath}
%%% \usepackage{amsfonts}
%%% \usepackage{graphicx}
%%% %\usepackage[bookmarks=false]{hyperref}
%%% \usepackage{hyperref}
%%% \usepackage{subfigure}
%%% \usepackage{titlesec}
%%% \usepackage{indentfirst}
%%%
%%% \newtheorem{theorem}{Theorem}[section]
%%% \newtheorem{definition}[theorem]{Definition}
%%% \newtheorem{axiom}[theorem]{Axiom}
%%%
%%% \newcommand{\Abs}[1]{{\left\lvert{#1}\right\rvert}}
%%% \newcommand{\evalbar}[2]{{\left.{#1}\right\vert}_{#2}}
%%% \newcommand{\evalnobar}[2]{{#1}_{#2}}
%%% \newcommand{\Bu}[0]{\mathbf{u}}
%%% \newcommand{\Bv}[0]{\mathbf{v}}
%%% \newcommand{\Be}[0]{\mathbf{e}}
%%% \newcommand{\Ba}[0]{\mathbf{a}}
%%% \newcommand{\Bb}[0]{\mathbf{b}}
%%% \newcommand{\Bc}[0]{\mathbf{c}}
%%% \newcommand{\Br}[0]{\mathbf{r}}
%%% \newcommand{\Bx}[0]{\mathbf{x}}
%%% \newcommand{\Bk}[0]{\mathbf{k}}
%%% \newcommand{\Bq}[0]{\mathbf{q}}
%%% \newcommand{\Bs}[0]{\mathbf{s}}
%%% \newcommand{\Bt}[0]{\mathbf{t}}
%%% \newcommand{\rcap}[0]{\hat{\Br}}
%%% \newcommand{\inv}[1]{\frac{1}{#1}}
%%% \newcommand{\grad}[0]{\nabla}
%%% \newcommand{\LL}[0]{\calL}
%%% \newcommand{\PD}[2]{\frac{\partial {#2}}{\partial {#1}}}
%%% \newcommand{\PDi}[2]{{\partial {#2}}/{\partial {#1}}}
%%% \newcommand{\gpgrade}[2] {{\left\langle{{#1}}\right\rangle}_{#2}}
%%% \newcommand{\gpgradezero}[1] {\gpgrade{#1}{}}
%%% \newcommand{\gpgradeone}[1] {\gpgrade{#1}{1}}
%%% \newcommand{\BTheta}[0]{\boldsymbol{\Theta}}
%%% \newcommand{\Brho}[0]{\boldsymbol{\rho}}
%%% \newcommand{\T}[0]{\text{T}}
%%%
%%% \begin{document}

%\chapter{Lagrangian and Euler-Lagrange equation evaluation for the spherical N-pendulum problem}
%\author{Peeter Joot
%\smallskip\\
%\small{e-mail: peeterjoot\(@\)protonmail.com}}
%\date{\small{\today}}
%\maketitle

\newcommand{\nbref}[1]{%
%\itemRef{classicalmechanics}{#1}%
%\index{Mathematica}%
fixme:nbref:#1
}

\beginArtNoToc

\generatetitle{Lagrangian and Euler-Lagrange equation evaluation for the spherical N-pendulum problem}
%
% Copyright � 2012 Peeter Joot.  All Rights Reserved.
% Licenced as described in the file LICENSE under the root directory of this GIT repository.
%
%{
\newcommand{\authorname}{Peeter Joot}
\newcommand{\email}{peeterjoot@protonmail.com}
\newcommand{\basename}{FIXMEbasenameUndefined}
\newcommand{\dirname}{notes/FIXMEdirnameUndefined/}

\renewcommand{\basename}{multiPendulumSphericalMatrix}
\renewcommand{\dirname}{notes/classicalmechanics/}
%\newcommand{\dateintitle}{}
%\newcommand{\keywords}{}

\newcommand{\authorname}{Peeter Joot}
\newcommand{\onlineurl}{http://sites.google.com/site/peeterjoot2/math2013/\basename.pdf}
\newcommand{\sourcepath}{\dirname\basename.tex}
\newcommand{\generatetitle}[1]{\chapter{#1}}

\newcommand{\vcsinfo}{%
\section*{}
\noindent{\color{DarkOliveGreen}{\rule{\linewidth}{0.1mm}}}
\paragraph{Document version}
%\paragraph{\color{Maroon}{Document version}}
{
\small
\begin{itemize}
\item Available online at:\\ 
\href{\onlineurl}{\onlineurl}
\item Git Repository: \input{./.revinfo/gitRepo.tex}
\item Source: \sourcepath
\item last commit: \input{./.revinfo/gitCommitString.tex}
\item commit date: \input{./.revinfo/gitCommitDate.tex}
\end{itemize}
}
}

%\PassOptionsToPackage{dvipsnames,svgnames}{xcolor}
\PassOptionsToPackage{square,numbers}{natbib}
\documentclass{scrreprt}

\usepackage[left=2cm,right=2cm]{geometry}
\usepackage[svgnames]{xcolor}
\usepackage{peeters_layout}

\usepackage{natbib}

\usepackage[
colorlinks=true,
bookmarks=false,
pdfauthor={\authorname, \email},
backref 
]{hyperref}

% http://tex.stackexchange.com/questions/75773/how-to-reference-problems-by-the-text-label-in-an-exercise-envioronment
\usepackage[english]{cleveref}
\crefname{Exercise}{exercise}{exercises}
\Crefname{Exercise}{Exercise}{Exercises}

\RequirePackage{titlesec}
\RequirePackage{ifthen}

% http://stackoverflow.com/questions/4932910/date-in-the-tabular-environment
\makeatletter
\let\insertdate\@date
\makeatother

\titleformat{\chapter}[display]
{\bfseries\Large}
{\color{DarkSlateGrey}\filleft \authorname
\ifthenelse{\isundefined{\studentnumber}}{}{\\ \studentnumber}
\ifthenelse{\isundefined{\email}}{}{\\ \email}
\ifthenelse{\isundefined{\dateintitle}}{}{\\ \insertdate}
%\ifthenelse{\isundefined{\coursename}}{}{\\ \coursename} % put in title instead.
}
{4ex}
{\color{DarkOliveGreen}{\titlerule}\color{Maroon}
\vspace{2ex}%
\filright}
[\vspace{2ex}%
\color{DarkOliveGreen}\titlerule
]

\newcommand{\beginArtWithToc}[0]{\begin{document}\tableofcontents}
\newcommand{\beginArtNoToc}[0]{\begin{document}}
\newcommand{\EndNoBibArticle}[0]{\end{document}}
\newcommand{\EndArticle}[0]{\bibliography{Bibliography}\bibliographystyle{plainnat}\end{document}}

% 
%\newcommand{\citep}[1]{\cite{#1}}

\colorSectionsForArticle



%\usepackage{peeters_layout_exercise}
%\usepackage{peeters_braket}
\usepackage{peeters_figures}
\usepackage{macros_cal}

\newcommand{\chapcite}[1]{\ref{chap:#1}}
\newcommand{\gpgradezeroNoOp}[1]{{#1}}

%\newcommand{\small}[1]{#1}

%%% %\documentclass[11pt,twocolumn]{article}
%%%
%%% % small font:
%%% %\documentclass[11pt]{article}
%%% %\setlength\topmargin{-0.5in}
%%% %\setlength\columnsep{0.2in}
%%% %\setlength\headsep{0.0in}
%%% %\setlength\textheight{9.5in}
%%% %\setlength\textwidth{7in}
%%% %\setlength\oddsidemargin{-0.25in}
%%% %\setlength\evensidemargin{-0.25in}
%%%
%%% % readable?  Font still seems small?
%%% \documentclass[11pt]{article}
%%% \setlength{\textwidth}{\paperwidth}
%%% \addtolength{\textwidth}{-2in}
%%% \setlength{\oddsidemargin}{0pt}
%%% \setlength{\evensidemargin}{0pt}
%%% \setlength{\headheight}{15pt}
%%% \setlength{\headsep}{15pt}
%%% \setlength{\topmargin}{0in}
%%% \addtolength{\topmargin}{-\headheight}
%%% \addtolength{\topmargin}{-\headsep}
%%% \setlength{\footskip}{29pt}
%%% \setlength{\textheight}{\paperheight}
%%% \addtolength{\textheight}{-2.2in}
%%% \setlength{\marginparwidth}{.5in}
%%% \setlength{\marginparsep}{5pt}
%%%
%%%
%%% %\usepackage{mathpazo}
%%% \usepackage{color}
%%% \usepackage{amsmath}
%%% \usepackage{amsfonts}
%%% \usepackage{graphicx}
%%% %\usepackage[bookmarks=false]{hyperref}
%%% \usepackage{hyperref}
%%% \usepackage{subfigure}
%%% \usepackage{titlesec}
%%% \usepackage{indentfirst}
%%%
%%% \newtheorem{theorem}{Theorem}[section]
%%% \newtheorem{definition}[theorem]{Definition}
%%% \newtheorem{axiom}[theorem]{Axiom}
%%%
%%% \newcommand{\Abs}[1]{{\left\lvert{#1}\right\rvert}}
%%% \newcommand{\evalbar}[2]{{\left.{#1}\right\vert}_{#2}}
%%% \newcommand{\evalnobar}[2]{{#1}_{#2}}
%%% \newcommand{\Bu}[0]{\mathbf{u}}
%%% \newcommand{\Bv}[0]{\mathbf{v}}
%%% \newcommand{\Be}[0]{\mathbf{e}}
%%% \newcommand{\Ba}[0]{\mathbf{a}}
%%% \newcommand{\Bb}[0]{\mathbf{b}}
%%% \newcommand{\Bc}[0]{\mathbf{c}}
%%% \newcommand{\Br}[0]{\mathbf{r}}
%%% \newcommand{\Bx}[0]{\mathbf{x}}
%%% \newcommand{\Bk}[0]{\mathbf{k}}
%%% \newcommand{\Bq}[0]{\mathbf{q}}
%%% \newcommand{\Bs}[0]{\mathbf{s}}
%%% \newcommand{\Bt}[0]{\mathbf{t}}
%%% \newcommand{\rcap}[0]{\hat{\Br}}
%%% \newcommand{\inv}[1]{\frac{1}{#1}}
%%% \newcommand{\grad}[0]{\nabla}
%%% \newcommand{\LL}[0]{\calL}
%%% \newcommand{\PD}[2]{\frac{\partial {#2}}{\partial {#1}}}
%%% \newcommand{\PDi}[2]{{\partial {#2}}/{\partial {#1}}}
%%% \newcommand{\gpgrade}[2] {{\left\langle{{#1}}\right\rangle}_{#2}}
%%% \newcommand{\gpgradezero}[1] {\gpgrade{#1}{}}
%%% \newcommand{\gpgradeone}[1] {\gpgrade{#1}{1}}
%%% \newcommand{\BTheta}[0]{\boldsymbol{\Theta}}
%%% \newcommand{\Brho}[0]{\boldsymbol{\rho}}
%%% \newcommand{\T}[0]{\text{T}}
%%%
%%% \begin{document}

%\chapter{Lagrangian and Euler-Lagrange equation evaluation for the spherical N-pendulum problem}
%\author{Peeter Joot
%\smallskip\\
%\small{e-mail: peeterjoot\(@\)protonmail.com}}
%\date{\small{\today}}
%\maketitle

\newcommand{\nbref}[1]{%
%\itemRef{classicalmechanics}{#1}%
%\index{Mathematica}%
fixme:nbref:#1
}

\beginArtNoToc

\generatetitle{Lagrangian and Euler-Lagrange equation evaluation for the spherical N-pendulum problem}
%
% Copyright � 2012 Peeter Joot.  All Rights Reserved.
% Licenced as described in the file LICENSE under the root directory of this GIT repository.
%
%{
\input{../latex/blogpost.tex}
\renewcommand{\basename}{multiPendulumSphericalMatrix}
\renewcommand{\dirname}{notes/classicalmechanics/}
%\newcommand{\dateintitle}{}
%\newcommand{\keywords}{}

\input{../latex/peeter_prologue_print2.tex}

%\usepackage{peeters_layout_exercise}
%\usepackage{peeters_braket}
\usepackage{peeters_figures}
\usepackage{macros_cal}

\newcommand{\chapcite}[1]{\ref{chap:#1}}
\newcommand{\gpgradezeroNoOp}[1]{{#1}}

%\newcommand{\small}[1]{#1}

%%% %\documentclass[11pt,twocolumn]{article}
%%%
%%% % small font:
%%% %\documentclass[11pt]{article}
%%% %\setlength\topmargin{-0.5in}
%%% %\setlength\columnsep{0.2in}
%%% %\setlength\headsep{0.0in}
%%% %\setlength\textheight{9.5in}
%%% %\setlength\textwidth{7in}
%%% %\setlength\oddsidemargin{-0.25in}
%%% %\setlength\evensidemargin{-0.25in}
%%%
%%% % readable?  Font still seems small?
%%% \documentclass[11pt]{article}
%%% \setlength{\textwidth}{\paperwidth}
%%% \addtolength{\textwidth}{-2in}
%%% \setlength{\oddsidemargin}{0pt}
%%% \setlength{\evensidemargin}{0pt}
%%% \setlength{\headheight}{15pt}
%%% \setlength{\headsep}{15pt}
%%% \setlength{\topmargin}{0in}
%%% \addtolength{\topmargin}{-\headheight}
%%% \addtolength{\topmargin}{-\headsep}
%%% \setlength{\footskip}{29pt}
%%% \setlength{\textheight}{\paperheight}
%%% \addtolength{\textheight}{-2.2in}
%%% \setlength{\marginparwidth}{.5in}
%%% \setlength{\marginparsep}{5pt}
%%%
%%%
%%% %\usepackage{mathpazo}
%%% \usepackage{color}
%%% \usepackage{amsmath}
%%% \usepackage{amsfonts}
%%% \usepackage{graphicx}
%%% %\usepackage[bookmarks=false]{hyperref}
%%% \usepackage{hyperref}
%%% \usepackage{subfigure}
%%% \usepackage{titlesec}
%%% \usepackage{indentfirst}
%%%
%%% \newtheorem{theorem}{Theorem}[section]
%%% \newtheorem{definition}[theorem]{Definition}
%%% \newtheorem{axiom}[theorem]{Axiom}
%%%
%%% \newcommand{\Abs}[1]{{\left\lvert{#1}\right\rvert}}
%%% \newcommand{\evalbar}[2]{{\left.{#1}\right\vert}_{#2}}
%%% \newcommand{\evalnobar}[2]{{#1}_{#2}}
%%% \newcommand{\Bu}[0]{\mathbf{u}}
%%% \newcommand{\Bv}[0]{\mathbf{v}}
%%% \newcommand{\Be}[0]{\mathbf{e}}
%%% \newcommand{\Ba}[0]{\mathbf{a}}
%%% \newcommand{\Bb}[0]{\mathbf{b}}
%%% \newcommand{\Bc}[0]{\mathbf{c}}
%%% \newcommand{\Br}[0]{\mathbf{r}}
%%% \newcommand{\Bx}[0]{\mathbf{x}}
%%% \newcommand{\Bk}[0]{\mathbf{k}}
%%% \newcommand{\Bq}[0]{\mathbf{q}}
%%% \newcommand{\Bs}[0]{\mathbf{s}}
%%% \newcommand{\Bt}[0]{\mathbf{t}}
%%% \newcommand{\rcap}[0]{\hat{\Br}}
%%% \newcommand{\inv}[1]{\frac{1}{#1}}
%%% \newcommand{\grad}[0]{\nabla}
%%% \newcommand{\LL}[0]{\calL}
%%% \newcommand{\PD}[2]{\frac{\partial {#2}}{\partial {#1}}}
%%% \newcommand{\PDi}[2]{{\partial {#2}}/{\partial {#1}}}
%%% \newcommand{\gpgrade}[2] {{\left\langle{{#1}}\right\rangle}_{#2}}
%%% \newcommand{\gpgradezero}[1] {\gpgrade{#1}{}}
%%% \newcommand{\gpgradeone}[1] {\gpgrade{#1}{1}}
%%% \newcommand{\BTheta}[0]{\boldsymbol{\Theta}}
%%% \newcommand{\Brho}[0]{\boldsymbol{\rho}}
%%% \newcommand{\T}[0]{\text{T}}
%%%
%%% \begin{document}

%\chapter{Lagrangian and Euler-Lagrange equation evaluation for the spherical N-pendulum problem}
%\author{Peeter Joot
%\smallskip\\
%\small{e-mail: peeterjoot\(@\)protonmail.com}}
%\date{\small{\today}}
%\maketitle

\newcommand{\nbref}[1]{%
%\itemRef{classicalmechanics}{#1}%
%\index{Mathematica}%
fixme:nbref:#1
}

\beginArtNoToc

\generatetitle{Lagrangian and Euler-Lagrange equation evaluation for the spherical N-pendulum problem}
\input{../classicalmechanics/mine/multiPendulumSphericalMatrix.tex}

%\bibliography{myrefs}
%\bibliographystyle{unsrturl}

%\section{graphics experiment}
%
%% TIP: draw the picture on graph paper and copy the numbers.
%% http://www.ursoswald.ch/LaTeXGraphics/picture/picture.html
%
%Forces on the Catenary
%
%% features: arrows, axis's, gravity vector.
%
%\setlength{\unitlength}{3cm}
%\begin{picture}(1.75, 2.75)(0, -0.1)
%  \put(0,0){\vector(1,0){1.75}}
%  \put(1.85, -0.03){\(x\)}
%  \put(0,0){\vector(0,1){2.75}}
%  \put(0.07, 2.70){\(y\)}
%  \thicklines
%  \qbezier(0.399, 0.467)(0.797, 0.998)
%          (1.118, 1.962)
%  \thinlines
%  \multiput(0.399, 0)(0, 0.1){3}
%           {\line(0,1){0.05}}
%  \multiput(1.118, 0)(0, 0.1){20}
%           {\line(0,1){0.05}}
%  \put(0.399, -0.08){\makebox(0,0){\(x_1\)}}
%  \put(1.118, -0.08){\makebox(0,0){\(x_2\)}}
%  \multiput(0.399, 0.467)(0, -0.267){2}
%           {\line(-1, 0){0.2}}
%  \multiput(0.399, 0.467)(-0.2, 0){2}
%           {\line(0, -1){0.267}}
%  \put(0.399, 0.467){\vector(-1, 0){0.2}}
%  \put(0.399, 0.467){\vector(0, -1){0.267}}
%  \put(0.399, 0.467){\vector(-3, -4){0.2}}
%  \put(0.32,0.55){\makebox(0,0){\(H\)}}
%  \put(0.48, 0.33){\makebox(0,0){\(V_1\)}}
%  \multiput(1.118, 1.962)(0, 0.6){2}
%           {\line(1,0){0.2}}
%  \multiput(1.118, 1.962)(0.2, 0){2}
%           {\line(0,1){0.6}}
%  \put(1.118, 1.962){\vector(1,0){0.2}}
%  \put(1.118, 1.962){\vector(0,1){0.6}}
%  \put(1.118, 1.962){\vector(1,3){0.2}}
%  \put(1.22, 1.87){\makebox(0,0){\(H\)}}
%  \put(1.02, 2.22){\makebox(0,0){\(V_2\)}}
%  \put(0.797, 1.195){\vector(0,-1){0.333}}
%  \put(0.777, 1.195){\line(1,0){0.04}}
%  \put(0.83,1.0){\(G\)}
%\end{picture}
%
%% feature: big dots.
%Simultaneousness
%
%\setlength{\unitlength}{1mm}
%\begin{picture}(60,50)
%  \put(0 ,15){\vector(1,0){53}}
%  \put(54,14){\(x_A\)}
%  \put( 8,10){\vector(0,1){37}}
%  \put( 0,46){\(ct_A\)}
%  \multiput(13,9)(15,0){3}{\line(0,1){35}}
%  \put(11,5){\(A''\)}
%  \put(26,5){\(A'\)}
%  \put(41,5){\(A'''\)}
%  \multiput(13,37)(30,0){2}{\circle*{2}}
%  \put(28,22){\circle*{2}}
%  \put(28,22){\vector(1,1){14}}
%  \put(15,39){\(E_1\)}
%  \put(28,22){\vector(-1,1){14}}
%  \put(45,39){\(E_2\)}
%  \multiput(0,37)(4,0){13}{\line(1,0){2}}
%\end{picture}
%
%
%\begin{picture}(0,0)%
%\includegraphics{3axisPl.pdf}%
%\end{picture}%
%\setlength{\unitlength}{3947sp}%
%%
%\begingroup\makeatletter\ifx\SetFigFont\undefined%
%\gdef\SetFigFont#1#2#3#4#5{%
%  \reset@font\fontsize{#1}{#2pt}%
%  \fontfamily{#3}\fontseries{#4}\fontshape{#5}%
%  \selectfont}%
%\fi\endgroup%
%\begin{picture}(6624,7149)(1189,-7498)
%\end{picture}%

%}
\EndArticle
%\EndNoBibArticle


%\bibliography{myrefs}
%\bibliographystyle{unsrturl}

%\section{graphics experiment}
%
%% TIP: draw the picture on graph paper and copy the numbers.
%% http://www.ursoswald.ch/LaTeXGraphics/picture/picture.html
%
%Forces on the Catenary
%
%% features: arrows, axis's, gravity vector.
%
%\setlength{\unitlength}{3cm}
%\begin{picture}(1.75, 2.75)(0, -0.1)
%  \put(0,0){\vector(1,0){1.75}}
%  \put(1.85, -0.03){\(x\)}
%  \put(0,0){\vector(0,1){2.75}}
%  \put(0.07, 2.70){\(y\)}
%  \thicklines
%  \qbezier(0.399, 0.467)(0.797, 0.998)
%          (1.118, 1.962)
%  \thinlines
%  \multiput(0.399, 0)(0, 0.1){3}
%           {\line(0,1){0.05}}
%  \multiput(1.118, 0)(0, 0.1){20}
%           {\line(0,1){0.05}}
%  \put(0.399, -0.08){\makebox(0,0){\(x_1\)}}
%  \put(1.118, -0.08){\makebox(0,0){\(x_2\)}}
%  \multiput(0.399, 0.467)(0, -0.267){2}
%           {\line(-1, 0){0.2}}
%  \multiput(0.399, 0.467)(-0.2, 0){2}
%           {\line(0, -1){0.267}}
%  \put(0.399, 0.467){\vector(-1, 0){0.2}}
%  \put(0.399, 0.467){\vector(0, -1){0.267}}
%  \put(0.399, 0.467){\vector(-3, -4){0.2}}
%  \put(0.32,0.55){\makebox(0,0){\(H\)}}
%  \put(0.48, 0.33){\makebox(0,0){\(V_1\)}}
%  \multiput(1.118, 1.962)(0, 0.6){2}
%           {\line(1,0){0.2}}
%  \multiput(1.118, 1.962)(0.2, 0){2}
%           {\line(0,1){0.6}}
%  \put(1.118, 1.962){\vector(1,0){0.2}}
%  \put(1.118, 1.962){\vector(0,1){0.6}}
%  \put(1.118, 1.962){\vector(1,3){0.2}}
%  \put(1.22, 1.87){\makebox(0,0){\(H\)}}
%  \put(1.02, 2.22){\makebox(0,0){\(V_2\)}}
%  \put(0.797, 1.195){\vector(0,-1){0.333}}
%  \put(0.777, 1.195){\line(1,0){0.04}}
%  \put(0.83,1.0){\(G\)}
%\end{picture}
%
%% feature: big dots.
%Simultaneousness
%
%\setlength{\unitlength}{1mm}
%\begin{picture}(60,50)
%  \put(0 ,15){\vector(1,0){53}}
%  \put(54,14){\(x_A\)}
%  \put( 8,10){\vector(0,1){37}}
%  \put( 0,46){\(ct_A\)}
%  \multiput(13,9)(15,0){3}{\line(0,1){35}}
%  \put(11,5){\(A''\)}
%  \put(26,5){\(A'\)}
%  \put(41,5){\(A'''\)}
%  \multiput(13,37)(30,0){2}{\circle*{2}}
%  \put(28,22){\circle*{2}}
%  \put(28,22){\vector(1,1){14}}
%  \put(15,39){\(E_1\)}
%  \put(28,22){\vector(-1,1){14}}
%  \put(45,39){\(E_2\)}
%  \multiput(0,37)(4,0){13}{\line(1,0){2}}
%\end{picture}
%
%
%\begin{picture}(0,0)%
%\includegraphics{3axisPl.pdf}%
%\end{picture}%
%\setlength{\unitlength}{3947sp}%
%%
%\begingroup\makeatletter\ifx\SetFigFont\undefined%
%\gdef\SetFigFont#1#2#3#4#5{%
%  \reset@font\fontsize{#1}{#2pt}%
%  \fontfamily{#3}\fontseries{#4}\fontshape{#5}%
%  \selectfont}%
%\fi\endgroup%
%\begin{picture}(6624,7149)(1189,-7498)
%\end{picture}%

%}
\EndArticle
%\EndNoBibArticle


%\bibliography{myrefs}
%\bibliographystyle{unsrturl}

%\section{graphics experiment}
%
%% TIP: draw the picture on graph paper and copy the numbers.
%% http://www.ursoswald.ch/LaTeXGraphics/picture/picture.html
%
%Forces on the Catenary
%
%% features: arrows, axis's, gravity vector.
%
%\setlength{\unitlength}{3cm}
%\begin{picture}(1.75, 2.75)(0, -0.1)
%  \put(0,0){\vector(1,0){1.75}}
%  \put(1.85, -0.03){\(x\)}
%  \put(0,0){\vector(0,1){2.75}}
%  \put(0.07, 2.70){\(y\)}
%  \thicklines
%  \qbezier(0.399, 0.467)(0.797, 0.998)
%          (1.118, 1.962)
%  \thinlines
%  \multiput(0.399, 0)(0, 0.1){3}
%           {\line(0,1){0.05}}
%  \multiput(1.118, 0)(0, 0.1){20}
%           {\line(0,1){0.05}}
%  \put(0.399, -0.08){\makebox(0,0){\(x_1\)}}
%  \put(1.118, -0.08){\makebox(0,0){\(x_2\)}}
%  \multiput(0.399, 0.467)(0, -0.267){2}
%           {\line(-1, 0){0.2}}
%  \multiput(0.399, 0.467)(-0.2, 0){2}
%           {\line(0, -1){0.267}}
%  \put(0.399, 0.467){\vector(-1, 0){0.2}}
%  \put(0.399, 0.467){\vector(0, -1){0.267}}
%  \put(0.399, 0.467){\vector(-3, -4){0.2}}
%  \put(0.32,0.55){\makebox(0,0){\(H\)}}
%  \put(0.48, 0.33){\makebox(0,0){\(V_1\)}}
%  \multiput(1.118, 1.962)(0, 0.6){2}
%           {\line(1,0){0.2}}
%  \multiput(1.118, 1.962)(0.2, 0){2}
%           {\line(0,1){0.6}}
%  \put(1.118, 1.962){\vector(1,0){0.2}}
%  \put(1.118, 1.962){\vector(0,1){0.6}}
%  \put(1.118, 1.962){\vector(1,3){0.2}}
%  \put(1.22, 1.87){\makebox(0,0){\(H\)}}
%  \put(1.02, 2.22){\makebox(0,0){\(V_2\)}}
%  \put(0.797, 1.195){\vector(0,-1){0.333}}
%  \put(0.777, 1.195){\line(1,0){0.04}}
%  \put(0.83,1.0){\(G\)}
%\end{picture}
%
%% feature: big dots.
%Simultaneousness
%
%\setlength{\unitlength}{1mm}
%\begin{picture}(60,50)
%  \put(0 ,15){\vector(1,0){53}}
%  \put(54,14){\(x_A\)}
%  \put( 8,10){\vector(0,1){37}}
%  \put( 0,46){\(ct_A\)}
%  \multiput(13,9)(15,0){3}{\line(0,1){35}}
%  \put(11,5){\(A''\)}
%  \put(26,5){\(A'\)}
%  \put(41,5){\(A'''\)}
%  \multiput(13,37)(30,0){2}{\circle*{2}}
%  \put(28,22){\circle*{2}}
%  \put(28,22){\vector(1,1){14}}
%  \put(15,39){\(E_1\)}
%  \put(28,22){\vector(-1,1){14}}
%  \put(45,39){\(E_2\)}
%  \multiput(0,37)(4,0){13}{\line(1,0){2}}
%\end{picture}
%
%
%\begin{picture}(0,0)%
%\includegraphics{3axisPl.pdf}%
%\end{picture}%
%\setlength{\unitlength}{3947sp}%
%%
%\begingroup\makeatletter\ifx\SetFigFont\undefined%
%\gdef\SetFigFont#1#2#3#4#5{%
%  \reset@font\fontsize{#1}{#2pt}%
%  \fontfamily{#3}\fontseries{#4}\fontshape{#5}%
%  \selectfont}%
%\fi\endgroup%
%\begin{picture}(6624,7149)(1189,-7498)
%\end{picture}%

%}
\EndArticle
%\EndNoBibArticle


%\bibliography{myrefs}
%\bibliographystyle{unsrturl}

%\section{graphics experiment}
%
%% TIP: draw the picture on graph paper and copy the numbers.
%% http://www.ursoswald.ch/LaTeXGraphics/picture/picture.html
%
%Forces on the Catenary
%
%% features: arrows, axis's, gravity vector.
%
%\setlength{\unitlength}{3cm}
%\begin{picture}(1.75, 2.75)(0, -0.1)
%  \put(0,0){\vector(1,0){1.75}}
%  \put(1.85, -0.03){\(x\)}
%  \put(0,0){\vector(0,1){2.75}}
%  \put(0.07, 2.70){\(y\)}
%  \thicklines
%  \qbezier(0.399, 0.467)(0.797, 0.998)
%          (1.118, 1.962)
%  \thinlines
%  \multiput(0.399, 0)(0, 0.1){3}
%           {\line(0,1){0.05}}
%  \multiput(1.118, 0)(0, 0.1){20}
%           {\line(0,1){0.05}}
%  \put(0.399, -0.08){\makebox(0,0){\(x_1\)}}
%  \put(1.118, -0.08){\makebox(0,0){\(x_2\)}}
%  \multiput(0.399, 0.467)(0, -0.267){2}
%           {\line(-1, 0){0.2}}
%  \multiput(0.399, 0.467)(-0.2, 0){2}
%           {\line(0, -1){0.267}}
%  \put(0.399, 0.467){\vector(-1, 0){0.2}}
%  \put(0.399, 0.467){\vector(0, -1){0.267}}
%  \put(0.399, 0.467){\vector(-3, -4){0.2}}
%  \put(0.32,0.55){\makebox(0,0){\(H\)}}
%  \put(0.48, 0.33){\makebox(0,0){\(V_1\)}}
%  \multiput(1.118, 1.962)(0, 0.6){2}
%           {\line(1,0){0.2}}
%  \multiput(1.118, 1.962)(0.2, 0){2}
%           {\line(0,1){0.6}}
%  \put(1.118, 1.962){\vector(1,0){0.2}}
%  \put(1.118, 1.962){\vector(0,1){0.6}}
%  \put(1.118, 1.962){\vector(1,3){0.2}}
%  \put(1.22, 1.87){\makebox(0,0){\(H\)}}
%  \put(1.02, 2.22){\makebox(0,0){\(V_2\)}}
%  \put(0.797, 1.195){\vector(0,-1){0.333}}
%  \put(0.777, 1.195){\line(1,0){0.04}}
%  \put(0.83,1.0){\(G\)}
%\end{picture}
%
%% feature: big dots.
%Simultaneousness
%
%\setlength{\unitlength}{1mm}
%\begin{picture}(60,50)
%  \put(0 ,15){\vector(1,0){53}}
%  \put(54,14){\(x_A\)}
%  \put( 8,10){\vector(0,1){37}}
%  \put( 0,46){\(ct_A\)}
%  \multiput(13,9)(15,0){3}{\line(0,1){35}}
%  \put(11,5){\(A''\)}
%  \put(26,5){\(A'\)}
%  \put(41,5){\(A'''\)}
%  \multiput(13,37)(30,0){2}{\circle*{2}}
%  \put(28,22){\circle*{2}}
%  \put(28,22){\vector(1,1){14}}
%  \put(15,39){\(E_1\)}
%  \put(28,22){\vector(-1,1){14}}
%  \put(45,39){\(E_2\)}
%  \multiput(0,37)(4,0){13}{\line(1,0){2}}
%\end{picture}
%
%
%\begin{picture}(0,0)%
%\includegraphics{3axisPl.pdf}%
%\end{picture}%
%\setlength{\unitlength}{3947sp}%
%%
%\begingroup\makeatletter\ifx\SetFigFont\undefined%
%\gdef\SetFigFont#1#2#3#4#5{%
%  \reset@font\fontsize{#1}{#2pt}%
%  \fontfamily{#3}\fontseries{#4}\fontshape{#5}%
%  \selectfont}%
%\fi\endgroup%
%\begin{picture}(6624,7149)(1189,-7498)
%\end{picture}%

%}
\EndArticle
%\EndNoBibArticle
