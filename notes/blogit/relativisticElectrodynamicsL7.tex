%
% Copyright � 2015 Peeter Joot.  All Rights Reserved.
% Licenced as described in the file LICENSE under the root directory of this GIT repository.
%
\documentclass[]{eliblog}

\usepackage{amsmath}
\usepackage{mathpazo}

%
% shorthand for bold symbols, convenient for vectors and matrices
%
\newcommand{\Ba}[0]{\mathbf{a}}
\newcommand{\Bb}[0]{\mathbf{b}}
\newcommand{\Bc}[0]{\mathbf{c}}
\newcommand{\Bd}[0]{\mathbf{d}}
\newcommand{\Be}[0]{\mathbf{e}}
\newcommand{\Bf}[0]{\mathbf{f}}
\newcommand{\Bg}[0]{\mathbf{g}}
\newcommand{\Bh}[0]{\mathbf{h}}
\newcommand{\Bi}[0]{\mathbf{i}}
\newcommand{\Bj}[0]{\mathbf{j}}
\newcommand{\Bk}[0]{\mathbf{k}}
\newcommand{\Bl}[0]{\mathbf{l}}
\newcommand{\Bm}[0]{\mathbf{m}}
\newcommand{\Bn}[0]{\mathbf{n}}
\newcommand{\Bo}[0]{\mathbf{o}}
\newcommand{\Bp}[0]{\mathbf{p}}
\newcommand{\Bq}[0]{\mathbf{q}}
\newcommand{\Br}[0]{\mathbf{r}}
\newcommand{\Bs}[0]{\mathbf{s}}
\newcommand{\Bt}[0]{\mathbf{t}}
\newcommand{\Bu}[0]{\mathbf{u}}
\newcommand{\Bv}[0]{\mathbf{v}}
\newcommand{\Bw}[0]{\mathbf{w}}
\newcommand{\Bx}[0]{\mathbf{x}}
\newcommand{\By}[0]{\mathbf{y}}
\newcommand{\Bz}[0]{\mathbf{z}}
\newcommand{\BA}[0]{\mathbf{A}}
\newcommand{\BB}[0]{\mathbf{B}}
\newcommand{\BC}[0]{\mathbf{C}}
\newcommand{\BD}[0]{\mathbf{D}}
\newcommand{\BE}[0]{\mathbf{E}}
\newcommand{\BF}[0]{\mathbf{F}}
\newcommand{\BG}[0]{\mathbf{G}}
\newcommand{\BH}[0]{\mathbf{H}}
\newcommand{\BI}[0]{\mathbf{I}}
\newcommand{\BJ}[0]{\mathbf{J}}
\newcommand{\BK}[0]{\mathbf{K}}
\newcommand{\BL}[0]{\mathbf{L}}
\newcommand{\BM}[0]{\mathbf{M}}
\newcommand{\BN}[0]{\mathbf{N}}
\newcommand{\BO}[0]{\mathbf{O}}
\newcommand{\BP}[0]{\mathbf{P}}
\newcommand{\BQ}[0]{\mathbf{Q}}
\newcommand{\BR}[0]{\mathbf{R}}
\newcommand{\BS}[0]{\mathbf{S}}
\newcommand{\BT}[0]{\mathbf{T}}
\newcommand{\BU}[0]{\mathbf{U}}
\newcommand{\BV}[0]{\mathbf{V}}
\newcommand{\BW}[0]{\mathbf{W}}
\newcommand{\BX}[0]{\mathbf{X}}
\newcommand{\BY}[0]{\mathbf{Y}}
\newcommand{\BZ}[0]{\mathbf{Z}}

\newcommand{\Bzero}[0]{\mathbf{0}}
\newcommand{\Btheta}[0]{\boldsymbol{\theta}}
\newcommand{\Btau}[0]{\boldsymbol{\tau}}
\newcommand{\Bomega}[0]{\boldsymbol{\omega}}

%
% shorthand for unit vectors
%
\newcommand{\acap}[0]{\hat{\Ba}}
\newcommand{\bcap}[0]{\hat{\Bb}}
\newcommand{\ccap}[0]{\hat{\Bc}}
\newcommand{\dcap}[0]{\hat{\Bd}}
\newcommand{\ecap}[0]{\hat{\Be}}
\newcommand{\fcap}[0]{\hat{\Bf}}
\newcommand{\gcap}[0]{\hat{\Bg}}
\newcommand{\hcap}[0]{\hat{\Bh}}
\newcommand{\icap}[0]{\hat{\Bi}}
\newcommand{\jcap}[0]{\hat{\Bj}}
\newcommand{\kcap}[0]{\hat{\Bk}}
\newcommand{\lcap}[0]{\hat{\Bl}}
\newcommand{\mcap}[0]{\hat{\Bm}}
\newcommand{\ncap}[0]{\hat{\Bn}}
\newcommand{\ocap}[0]{\hat{\Bo}}
\newcommand{\pcap}[0]{\hat{\Bp}}
\newcommand{\qcap}[0]{\hat{\Bq}}
\newcommand{\rcap}[0]{\hat{\Br}}
\newcommand{\scap}[0]{\hat{\Bs}}
\newcommand{\tcap}[0]{\hat{\Bt}}
\newcommand{\ucap}[0]{\hat{\Bu}}
\newcommand{\vcap}[0]{\hat{\Bv}}
\newcommand{\wcap}[0]{\hat{\Bw}}
\newcommand{\xcap}[0]{\hat{\Bx}}
\newcommand{\ycap}[0]{\hat{\By}}
\newcommand{\zcap}[0]{\hat{\Bz}}
\newcommand{\thetacap}[0]{\hat{\Btheta}}

%
% to write R^n and C^n in a distinguishable fashion.  Perhaps change this
% to the double lined characters upon figuring out how to do so.
%
\newcommand{\C}[1]{$\mathbb{C}^{#1}$}
\newcommand{\R}[1]{$\mathbb{R}^{#1}$}

%
% various generally useful helpers
%

% derivative of #1 wrt. #2:
\newcommand{\D}[2] {\frac {d#2} {d#1}}

\newcommand{\inv}[1]{\frac{1}{#1}}
\newcommand{\cross}[0]{\times}

\newcommand{\abs}[1]{\lvert{#1}\rvert}
\newcommand{\norm}[1]{\lVert{#1}\rVert}
\newcommand{\innerprod}[2]{\langle{#1}, {#2}\rangle}
\newcommand{\dotprod}[2]{{#1} \cdot {#2}}
\newcommand{\bdotprod}[2]{\left({#1} \cdot {#2}\right)}
\newcommand{\crossprod}[2]{{#1} \cross {#2}}
\newcommand{\tripleprod}[3]{\dotprod{\left(\crossprod{#1}{#2}\right)}{#3}}

\DeclareMathOperator{\Proj}{Proj}
\DeclareMathOperator{\Span}{span}
\DeclareMathOperator{\Sgn}{sgn}
\DeclareMathOperator{\Area}{Area}
\DeclareMathOperator{\Volume}{Volume}

%
% A few miscellaneous things specific to this document
%
\newcommand{\crossop}[1]{\crossprod{#1}{}}

% R2 vector.
\newcommand{\VectorTwo}[2]{
\begin{bmatrix}
 {#1} \\
 {#2}
\end{bmatrix}
}

\newcommand{\VectorN}[1]{
\begin{bmatrix}
{#1}_1 \\
{#1}_2 \\
\vdots \\
{#1}_N \\
\end{bmatrix}
}

\newcommand{\DETuvij}[4]{
\begin{vmatrix}
 {#1}_{#3} & {#1}_{#4} \\
 {#2}_{#3} & {#2}_{#4}
\end{vmatrix}
}

\newcommand{\DETuvwijk}[6]{
\begin{vmatrix}
 {#1}_{#4} & {#1}_{#5} & {#1}_{#6} \\
 {#2}_{#4} & {#2}_{#5} & {#2}_{#6} \\
 {#3}_{#4} & {#3}_{#5} & {#3}_{#6}
\end{vmatrix}
}

\newcommand{\DETuvwxijkl}[8]{
\begin{vmatrix}
 {#1}_{#5} & {#1}_{#6} & {#1}_{#7} & {#1}_{#8} \\
 {#2}_{#5} & {#2}_{#6} & {#2}_{#7} & {#2}_{#8} \\
 {#3}_{#5} & {#3}_{#6} & {#3}_{#7} & {#3}_{#8} \\
 {#4}_{#5} & {#4}_{#6} & {#4}_{#7} & {#4}_{#8} \\
\end{vmatrix}
}

%\newcommand{\DETuvwxyijklm}[10]{
%\begin{vmatrix}
% {#1}_{#6} & {#1}_{#7} & {#1}_{#8} & {#1}_{#9} & {#1}_{#10} \\
% {#2}_{#6} & {#2}_{#7} & {#2}_{#8} & {#2}_{#9} & {#2}_{#10} \\
% {#3}_{#6} & {#3}_{#7} & {#3}_{#8} & {#3}_{#9} & {#3}_{#10} \\
% {#4}_{#6} & {#4}_{#7} & {#4}_{#8} & {#4}_{#9} & {#4}_{#10} \\
% {#5}_{#6} & {#5}_{#7} & {#5}_{#8} & {#5}_{#9} & {#5}_{#10}
%\end{vmatrix}
%}

% R3 vector.
\newcommand{\VectorThree}[3]{
\begin{bmatrix}
 {#1} \\
 {#2} \\
 {#3}
\end{bmatrix}
}



\author{Peeter Joot}
\email{peeter.joot@gmail.com}

%\documentclass[]{eliblogwidescreen}

\usepackage{amsmath}
\usepackage{mathpazo}

%
% shorthand for bold symbols, convenient for vectors and matrices
%
\newcommand{\Ba}[0]{\mathbf{a}}
\newcommand{\Bb}[0]{\mathbf{b}}
\newcommand{\Bc}[0]{\mathbf{c}}
\newcommand{\Bd}[0]{\mathbf{d}}
\newcommand{\Be}[0]{\mathbf{e}}
\newcommand{\Bf}[0]{\mathbf{f}}
\newcommand{\Bg}[0]{\mathbf{g}}
\newcommand{\Bh}[0]{\mathbf{h}}
\newcommand{\Bi}[0]{\mathbf{i}}
\newcommand{\Bj}[0]{\mathbf{j}}
\newcommand{\Bk}[0]{\mathbf{k}}
\newcommand{\Bl}[0]{\mathbf{l}}
\newcommand{\Bm}[0]{\mathbf{m}}
\newcommand{\Bn}[0]{\mathbf{n}}
\newcommand{\Bo}[0]{\mathbf{o}}
\newcommand{\Bp}[0]{\mathbf{p}}
\newcommand{\Bq}[0]{\mathbf{q}}
\newcommand{\Br}[0]{\mathbf{r}}
\newcommand{\Bs}[0]{\mathbf{s}}
\newcommand{\Bt}[0]{\mathbf{t}}
\newcommand{\Bu}[0]{\mathbf{u}}
\newcommand{\Bv}[0]{\mathbf{v}}
\newcommand{\Bw}[0]{\mathbf{w}}
\newcommand{\Bx}[0]{\mathbf{x}}
\newcommand{\By}[0]{\mathbf{y}}
\newcommand{\Bz}[0]{\mathbf{z}}
\newcommand{\BA}[0]{\mathbf{A}}
\newcommand{\BB}[0]{\mathbf{B}}
\newcommand{\BC}[0]{\mathbf{C}}
\newcommand{\BD}[0]{\mathbf{D}}
\newcommand{\BE}[0]{\mathbf{E}}
\newcommand{\BF}[0]{\mathbf{F}}
\newcommand{\BG}[0]{\mathbf{G}}
\newcommand{\BH}[0]{\mathbf{H}}
\newcommand{\BI}[0]{\mathbf{I}}
\newcommand{\BJ}[0]{\mathbf{J}}
\newcommand{\BK}[0]{\mathbf{K}}
\newcommand{\BL}[0]{\mathbf{L}}
\newcommand{\BM}[0]{\mathbf{M}}
\newcommand{\BN}[0]{\mathbf{N}}
\newcommand{\BO}[0]{\mathbf{O}}
\newcommand{\BP}[0]{\mathbf{P}}
\newcommand{\BQ}[0]{\mathbf{Q}}
\newcommand{\BR}[0]{\mathbf{R}}
\newcommand{\BS}[0]{\mathbf{S}}
\newcommand{\BT}[0]{\mathbf{T}}
\newcommand{\BU}[0]{\mathbf{U}}
\newcommand{\BV}[0]{\mathbf{V}}
\newcommand{\BW}[0]{\mathbf{W}}
\newcommand{\BX}[0]{\mathbf{X}}
\newcommand{\BY}[0]{\mathbf{Y}}
\newcommand{\BZ}[0]{\mathbf{Z}}

\newcommand{\Bzero}[0]{\mathbf{0}}
\newcommand{\Btheta}[0]{\boldsymbol{\theta}}
\newcommand{\Btau}[0]{\boldsymbol{\tau}}
\newcommand{\Bomega}[0]{\boldsymbol{\omega}}

%
% shorthand for unit vectors
%
\newcommand{\acap}[0]{\hat{\Ba}}
\newcommand{\bcap}[0]{\hat{\Bb}}
\newcommand{\ccap}[0]{\hat{\Bc}}
\newcommand{\dcap}[0]{\hat{\Bd}}
\newcommand{\ecap}[0]{\hat{\Be}}
\newcommand{\fcap}[0]{\hat{\Bf}}
\newcommand{\gcap}[0]{\hat{\Bg}}
\newcommand{\hcap}[0]{\hat{\Bh}}
\newcommand{\icap}[0]{\hat{\Bi}}
\newcommand{\jcap}[0]{\hat{\Bj}}
\newcommand{\kcap}[0]{\hat{\Bk}}
\newcommand{\lcap}[0]{\hat{\Bl}}
\newcommand{\mcap}[0]{\hat{\Bm}}
\newcommand{\ncap}[0]{\hat{\Bn}}
\newcommand{\ocap}[0]{\hat{\Bo}}
\newcommand{\pcap}[0]{\hat{\Bp}}
\newcommand{\qcap}[0]{\hat{\Bq}}
\newcommand{\rcap}[0]{\hat{\Br}}
\newcommand{\scap}[0]{\hat{\Bs}}
\newcommand{\tcap}[0]{\hat{\Bt}}
\newcommand{\ucap}[0]{\hat{\Bu}}
\newcommand{\vcap}[0]{\hat{\Bv}}
\newcommand{\wcap}[0]{\hat{\Bw}}
\newcommand{\xcap}[0]{\hat{\Bx}}
\newcommand{\ycap}[0]{\hat{\By}}
\newcommand{\zcap}[0]{\hat{\Bz}}
\newcommand{\thetacap}[0]{\hat{\Btheta}}

%
% to write R^n and C^n in a distinguishable fashion.  Perhaps change this
% to the double lined characters upon figuring out how to do so.
%
\newcommand{\C}[1]{$\mathbb{C}^{#1}$}
\newcommand{\R}[1]{$\mathbb{R}^{#1}$}

%
% various generally useful helpers
%

% derivative of #1 wrt. #2:
\newcommand{\D}[2] {\frac {d#2} {d#1}}

\newcommand{\inv}[1]{\frac{1}{#1}}
\newcommand{\cross}[0]{\times}

\newcommand{\abs}[1]{\lvert{#1}\rvert}
\newcommand{\norm}[1]{\lVert{#1}\rVert}
\newcommand{\innerprod}[2]{\langle{#1}, {#2}\rangle}
\newcommand{\dotprod}[2]{{#1} \cdot {#2}}
\newcommand{\bdotprod}[2]{\left({#1} \cdot {#2}\right)}
\newcommand{\crossprod}[2]{{#1} \cross {#2}}
\newcommand{\tripleprod}[3]{\dotprod{\left(\crossprod{#1}{#2}\right)}{#3}}

\DeclareMathOperator{\Proj}{Proj}
\DeclareMathOperator{\Span}{span}
\DeclareMathOperator{\Sgn}{sgn}
\DeclareMathOperator{\Area}{Area}
\DeclareMathOperator{\Volume}{Volume}

%
% A few miscellaneous things specific to this document
%
\newcommand{\crossop}[1]{\crossprod{#1}{}}

% R2 vector.
\newcommand{\VectorTwo}[2]{
\begin{bmatrix}
 {#1} \\
 {#2}
\end{bmatrix}
}

\newcommand{\VectorN}[1]{
\begin{bmatrix}
{#1}_1 \\
{#1}_2 \\
\vdots \\
{#1}_N \\
\end{bmatrix}
}

\newcommand{\DETuvij}[4]{
\begin{vmatrix}
 {#1}_{#3} & {#1}_{#4} \\
 {#2}_{#3} & {#2}_{#4}
\end{vmatrix}
}

\newcommand{\DETuvwijk}[6]{
\begin{vmatrix}
 {#1}_{#4} & {#1}_{#5} & {#1}_{#6} \\
 {#2}_{#4} & {#2}_{#5} & {#2}_{#6} \\
 {#3}_{#4} & {#3}_{#5} & {#3}_{#6}
\end{vmatrix}
}

\newcommand{\DETuvwxijkl}[8]{
\begin{vmatrix}
 {#1}_{#5} & {#1}_{#6} & {#1}_{#7} & {#1}_{#8} \\
 {#2}_{#5} & {#2}_{#6} & {#2}_{#7} & {#2}_{#8} \\
 {#3}_{#5} & {#3}_{#6} & {#3}_{#7} & {#3}_{#8} \\
 {#4}_{#5} & {#4}_{#6} & {#4}_{#7} & {#4}_{#8} \\
\end{vmatrix}
}

%\newcommand{\DETuvwxyijklm}[10]{
%\begin{vmatrix}
% {#1}_{#6} & {#1}_{#7} & {#1}_{#8} & {#1}_{#9} & {#1}_{#10} \\
% {#2}_{#6} & {#2}_{#7} & {#2}_{#8} & {#2}_{#9} & {#2}_{#10} \\
% {#3}_{#6} & {#3}_{#7} & {#3}_{#8} & {#3}_{#9} & {#3}_{#10} \\
% {#4}_{#6} & {#4}_{#7} & {#4}_{#8} & {#4}_{#9} & {#4}_{#10} \\
% {#5}_{#6} & {#5}_{#7} & {#5}_{#8} & {#5}_{#9} & {#5}_{#10}
%\end{vmatrix}
%}

% R3 vector.
\newcommand{\VectorThree}[3]{
\begin{bmatrix}
 {#1} \\
 {#2} \\
 {#3}
\end{bmatrix}
}



\author{Peeter Joot}
\email{peeter.joot@gmail.com}


\chapter{PHY450H1S.  Relativistic Electrodynamics Lecture 7 (Taught by Prof. Erich Poppitz).  Action and relativistic dynamics.}
\label{chap:relativisticElectrodynamicsL7}
%\useCCL
\blogpage{http://sites.google.com/site/peeterjoot/math2011/relativisticElectrodynamicsL7.pdf}
\date{Jan 26, 2011}
\revisionInfo{relativisticElectrodynamicsL7.tex}

%\beginArtWithToc
\beginArtNoToc

\section{Reading.}

Will now be covering chapter 2 material from the text \cite{landau1980classical}.

Covering \href{http://www.physics.utoronto.ca/~poppitz/e-poppitz/PHY450_files/RelEMpp52-56.pdf}{Professor Poppitz's lecture notes}: equation of motion, symmetries, and conserved quantities (energy-momentum 4 vector) from relativistic particle action [Wednesday, Jan. 26, Tuesday, Feb. 1]

These notes are also augmented by \href{http://www.physics.utoronto.ca/~poppitz/e-poppitz/PHY450_files/RelEMp53.1.pdf}{some additional notes completing an argument on page 53}.

\section{The relativity principle}

The relativity principle implies that the EOM should be expressed in 4-vector form, just like Newton's EOM are expressed in 3-vector form

\begin{equation}\label{eqn:relativisticElectrodynamicsL7:10}
m \ddot{\Br} = \Bf
\end{equation}

Observe that in coordinate form this is
\begin{equation}\label{eqn:relativisticElectrodynamicsL7:20}
m \ddot{r}^i = f^i, \qquad i = 1,2,3
\end{equation}

or for a rotated frame $O'$
\begin{equation}\label{eqn:relativisticElectrodynamicsL7:30}
m \ddot{r'}^i = {f'}^i, \qquad i = 1,2,3
\end{equation}

Need to generalize to 4 vectors, so we need 4-velocity and 4-acceleration.

Later we will study action and Lagrangian, and then relativity will require that the action be a Lorentz scalar.  The analogy for a Newtonian point particle is a scalar under rotations.

\subsection{Four vector velocity}

\paragraph{Definition:} Velocity s the rate of change of position in $(ct, \Bx)$-space.  Position means specifying both $ct$ and $\Bx$ for a point in spacetime.

PICTURE: $x^0 = ct$ axis up, and $x^1, x^2, x^3$ axis over, with worldline $x = x(\tau)$.  Here $\tau$ is a parameter for the worldline, and provides a mapping for the curve in spacetime.

PICTURE: 3D vectors, $\Br(t)$, $\Br(t + \Delta t)$, and the difference vector $\Br(t + \Delta t) - \Br(t)$.

We write

\begin{equation}\label{eqn:relativisticElectrodynamicsL7:40}
\Bv(t) \equiv \lim_{\Delta t \rightarrow zero} \frac{\Br(t + \Delta t) - \Br(t)}{ \Delta t}
\end{equation}

For four vectors we will parameterize the worldline by its ``length'', with $O$ taken from some arbitrary point on it.  We can also take $\tau$ to be the proper time, and the only difference will be the factor of $c$ (which becomes especially easy with the choice $c=1$ that is avoided in this class).

\begin{equation}\label{eqn:relativisticElectrodynamicsL7:50}
\frac{x^i(\tau + \Delta \tau) - x^i(\tau)}{\Delta \tau}
\end{equation}

We'll take

\begin{equation}\label{eqn:relativisticElectrodynamicsL7:60}
\Delta \tau = ds = \frac{dx^i}{ds}
\end{equation}

This is a nice quantity, we are dividing a vector by a scalar, and thus get a four vector as a result (i.e. the result transforms as a four vector).

PICTURE: small fragment of a worldline with constant slope over the infinitesimal interval.  $dx^0$ up and $dx^1$ over.

\begin{equation}\label{eqn:relativisticElectrodynamicsL7:70}
u^i \equiv \frac{dx^i}{ds}
\end{equation}

\begin{align*}
ds^2 
&= (dx^0)^2 - (dx^1)^2 \\
&= c^2 \left( (dt)^2 - \inv{c^2} (dx^1)^2 \right) \\
&= c^2 (dt)^2 \left( 1 - \inv{c^2} \frac{dx^1}{dt^2} \right) 
\end{align*}

Or 

\begin{align}\label{eqn:relativisticElectrodynamicsL7:90}
ds = c dt \sqrt{1 - \inv{c^2} \frac{dx^1}{dt^2} }
\end{align}

NOTE: Prof admits pulling a fast one, since he has aligned the worldline along the $x^1$ axis, however this is always possible by rotating the coordinate system.

\begin{align*}
u^0 
&= \frac{dx^0}{ds} \\
&= \frac{c dt}{ c dt \sqrt{ 1 - \Bv^2/c^2} } \\
&= \frac{1}{ \sqrt{ 1 - \Bv^2/c^2} } \\
&= \gamma
\end{align*}

\begin{align*}
u^1 
&= \frac{dx^1}{ds} \\
&= \frac{dx^1 }{ c dt \sqrt{ 1 - \Bv^2/c^2} } \\
&= \frac{v^1/c}{ \sqrt{ 1 - \Bv^2/c^2} } \\
&= \gamma \frac{v^1}{c}
\end{align*}

Similarily
\begin{align*}
u^2 &= \gamma \frac{v^2}{c} \\
u^3 &= \gamma \frac{v^2}{c}
\end{align*}

We've now unpacked the four velocity, and have

\begin{equation}\label{eqn:relativisticElectrodynamicsL7:100}
u^i = ( \gamma, \frac{\Bv}{c} \gamma )
\end{equation}

\subsection{Length of the four velocity vector}

Recall that this length is

\begin{align*}
u^i g_{ij} u^j 
&= u^i u_i  \\
&= u_i u^i  \\
&= (u^0)^2 - (u_i)^2 \\
&= \gamma^2 - \gamma^2 \frac{\Bv}{c} \cdot \frac{\Bv}{c} \\
&= \gamma^2 \left(1 - \frac{\Bv^2}{c^2} \right)
\end{align*}

The four velocity in physics is
\begin{equation}\label{eqn:relativisticElectrodynamicsL7:110}
u^i = ( \gamma, \frac{\Bv}{c} \gamma )
\end{equation}

but in mathematics the meaning of $u^i u_i = 1$ means that this quantity is the unit tangent vector to the worldline.

\subsection{Four acceleration}

In Newtonian physics we have 

\begin{equation}\label{eqn:relativisticElectrodynamicsL7:111}
\Ba = \frac{\Bv}{dt}
\end{equation}

Our relativistic mapping of this, with $v \rightarrow u^i$ and $t \rightarrow s$, gives

\begin{equation}\label{eqn:relativisticElectrodynamicsL7:120}
w^i = \frac{d u^i}{ds}
\end{equation}

Geometrically $w^i$ is the normal to the worldline.  This follows from $u^i g_{ij} u^j = 1$, so

\begin{align*}
\frac{d}{ds} \left( u^i g_{ij} u^j \right) 
&=
\frac{d u^i}{ds} g_{ij} u^j 
+u^i g_{ij} \frac{d u^j}{ds} \\
&=
\frac{d u^i}{ds} g_{ij} u^j 
+u^j \underbrace{g_{ji}}_{= g_{ij}} \frac{d u^i}{ds} \\
&=
\frac{d u^i}{ds} g_{ij} u^j 
+u^j g_{ji} \frac{d u^i}{ds} \\
&=
2 \frac{d u^i}{ds} g_{ij} u^j 
\end{align*}

Note that we've utilized the fact above that the dummy summation indexes can be changed, or swapped.

The conclusion is that the dot product of the acceleration and the velocity is zero

\begin{equation}\label{eqn:relativisticElectrodynamicsL7:130}
w_i u^i = 0.
\end{equation}

\section{Relativistic action.}

\begin{equation}\label{eqn:relativisticElectrodynamicsL7:140}
S_{ab} = ?
\end{equation}

What is the action for a worldline from $a \rightarrow b$.

We want something that has velocity dependence ($u^i$ not $\Bv$), but that is Lorentz invariant and has only first derivatives.

The relativisitic length is the simplest so we could form

\begin{equation}\label{eqn:relativisticElectrodynamicsL7:150}
\int ds u^i u_i
\end{equation}

but that's not interesting since $u^i u_i = 1$.  We could form

\begin{equation}\label{eqn:relativisticElectrodynamicsL7:160}
\int ds u^i \frac{u_i}{ds} = \int ds w^i u_i
\end{equation}

but then this is just zero.

We could form something like

\begin{equation}\label{eqn:relativisticElectrodynamicsL7:170}
\int ds \frac{w^i}{ds} u_i
\end{equation}

This is non zero and non-constant, but evaluating the EOM for such an action would produce a result that has higher than second order derivatives.

We are left with
\begin{equation}\label{eqn:relativisticElectrodynamicsL7:180}
S_{ab} = \text{constant} \int_a^b ds 
\end{equation}

To fix this constant we note that if we want to minimize the action over the infinitesimal interval, then we need a minus sign.  Dimensional analysis, noting that we require the action to have dimensions of momentum.  

FIXME: think on that statement.

The value $mc$ has these dimensions, so we construct an action with the following form

\begin{equation}\label{eqn:relativisticElectrodynamicsL7:190}
S_{ab} = - m c\int_a^b ds.
\end{equation}

Here ``m'' is a characteristic of the particle, which \underline{is a Lorentz scalar}.  It also happens to have dimensions of mass.  With $ds = c dt \sqrt{1 - \Bv^2/c^2}$, we have

\begin{equation}\label{eqn:relativisticElectrodynamicsL7:200}
S_{ab} = - m c^2 \int_{t_a}^{t_b} dt \sqrt{ 1 - \inv{c^2} \left( \frac{d \Bx(t) }{dt} \right)^2 }
\end{equation}

Now everything looks like it was in classical mechanics.

\begin{equation}\label{eqn:relativisticElectrodynamicsL7:210}
S_{ab} = \int_{t_a}^{t_b} \LL(\dot{\Bx}(t)) dt
\end{equation}
\begin{equation}\label{eqn:relativisticElectrodynamicsL7:220}
\LL(\dot{\Bx}(t)) = -m c^2 
\end{equation}

Now find the extremum of $S$.

\subsection{Some notation to followup on}
He wrote (without really explaining the notation) that we want

\begin{align}\label{eqn:relativisticElectrodynamicsL7:230}
S[\Bx(t) + \delta \Bx(t)] - S[ \Bx(t) ] \\
0 &= \delta \Bx(t_a) = \delta \Bx(t_b) \\
\Bx(t_a) &= \Bx_a \\
\Bx(t_b) &= \Bx_b
\end{align}

I'll have to review his classical mechanics notes to see exactly what he means.

\subsection{Back to the OEM calculation.}

\begin{equation}\label{eqn:relativisticElectrodynamicsL7:240}
\frac{d}{dt} \PD{\dot{\Bx}}{\LL} = \PD{\Bx}{\LL} = 0
\end{equation}

This last is zero because it's a free particle with no position dependence.

\begin{align*}
0 
&= -m c^2 \frac{d}{dt} \PD{\dot{\Bx}}{} \sqrt{ 1 - \dot{\Bx}^2 } \\
&= -m c^2 \frac{d}{dt} \frac{- \dot{\Bx}}{\sqrt{ 1 - \dot{\Bx}^2 } } \\
&= m c^2 \frac{d}{dt} \gamma \dot\Bx
\end{align*}

So we have

\begin{equation}\label{eqn:relativisticElectrodynamicsL7:n}
\frac{d}{dt} (\gamma \dot{\Bx}) = 0
\end{equation}

By evaluating this, we can eventually show that we can construct a four vector equation.  Doing this we have

\begin{align*}
\frac{d}{dt} (\gamma \Bv) 
&=
\frac{d}{dt} \left( \left(1 - \Bv^2/c^2\right)^{-1/2} \Bv \right) \\
&=
-2 (-1/2) \Bv (\Bv \cdot \dot{\Bv})/c^2 \left(1 - \Bv^2/c^2\right)^{-3/2} + \left(1 - \Bv^2/c^2\right)^{-1/2} \dot{\Bv} \\
&=
\gamma \left( \frac{\Bv (\Bv \cdot \dot{\Bv}) }{ c^2 - \Bv^2 } + \dot{\Bv} \right)
\end{align*}

Clearly $\dot{\Bv} = 0$ is a solution, but is it the only solution?

\section{Next time}

We want to finish up and show how this results in a four velocity equation.  We have

\begin{equation}\label{eqn:relativisticElectrodynamicsL7:250}
\frac{d}{dt} ( \gamma \Bv) = 0
\end{equation}

which is

\begin{equation}\label{eqn:relativisticElectrodynamicsL7:260}
\frac{d}{dt} ( u^\alpha ) = 0, \qquad \text{for} u^\alpha = u^1, u^2, u^3
\end{equation}

eventually, we will show that we also have

\begin{equation}\label{eqn:relativisticElectrodynamicsL7:270}
\frac{d}{dt} ( u^i ) = 0
\end{equation}

\EndArticle
