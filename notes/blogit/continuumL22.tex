%
% Copyright � 2015 Peeter Joot.  All Rights Reserved.
% Licenced as described in the file LICENSE under the root directory of this GIT repository.
%
\documentclass[]{eliblog}

\usepackage{amsmath}
\usepackage{mathpazo}

%
% shorthand for bold symbols, convenient for vectors and matrices
%
\newcommand{\Ba}[0]{\mathbf{a}}
\newcommand{\Bb}[0]{\mathbf{b}}
\newcommand{\Bc}[0]{\mathbf{c}}
\newcommand{\Bd}[0]{\mathbf{d}}
\newcommand{\Be}[0]{\mathbf{e}}
\newcommand{\Bf}[0]{\mathbf{f}}
\newcommand{\Bg}[0]{\mathbf{g}}
\newcommand{\Bh}[0]{\mathbf{h}}
\newcommand{\Bi}[0]{\mathbf{i}}
\newcommand{\Bj}[0]{\mathbf{j}}
\newcommand{\Bk}[0]{\mathbf{k}}
\newcommand{\Bl}[0]{\mathbf{l}}
\newcommand{\Bm}[0]{\mathbf{m}}
\newcommand{\Bn}[0]{\mathbf{n}}
\newcommand{\Bo}[0]{\mathbf{o}}
\newcommand{\Bp}[0]{\mathbf{p}}
\newcommand{\Bq}[0]{\mathbf{q}}
\newcommand{\Br}[0]{\mathbf{r}}
\newcommand{\Bs}[0]{\mathbf{s}}
\newcommand{\Bt}[0]{\mathbf{t}}
\newcommand{\Bu}[0]{\mathbf{u}}
\newcommand{\Bv}[0]{\mathbf{v}}
\newcommand{\Bw}[0]{\mathbf{w}}
\newcommand{\Bx}[0]{\mathbf{x}}
\newcommand{\By}[0]{\mathbf{y}}
\newcommand{\Bz}[0]{\mathbf{z}}
\newcommand{\BA}[0]{\mathbf{A}}
\newcommand{\BB}[0]{\mathbf{B}}
\newcommand{\BC}[0]{\mathbf{C}}
\newcommand{\BD}[0]{\mathbf{D}}
\newcommand{\BE}[0]{\mathbf{E}}
\newcommand{\BF}[0]{\mathbf{F}}
\newcommand{\BG}[0]{\mathbf{G}}
\newcommand{\BH}[0]{\mathbf{H}}
\newcommand{\BI}[0]{\mathbf{I}}
\newcommand{\BJ}[0]{\mathbf{J}}
\newcommand{\BK}[0]{\mathbf{K}}
\newcommand{\BL}[0]{\mathbf{L}}
\newcommand{\BM}[0]{\mathbf{M}}
\newcommand{\BN}[0]{\mathbf{N}}
\newcommand{\BO}[0]{\mathbf{O}}
\newcommand{\BP}[0]{\mathbf{P}}
\newcommand{\BQ}[0]{\mathbf{Q}}
\newcommand{\BR}[0]{\mathbf{R}}
\newcommand{\BS}[0]{\mathbf{S}}
\newcommand{\BT}[0]{\mathbf{T}}
\newcommand{\BU}[0]{\mathbf{U}}
\newcommand{\BV}[0]{\mathbf{V}}
\newcommand{\BW}[0]{\mathbf{W}}
\newcommand{\BX}[0]{\mathbf{X}}
\newcommand{\BY}[0]{\mathbf{Y}}
\newcommand{\BZ}[0]{\mathbf{Z}}

\newcommand{\Bzero}[0]{\mathbf{0}}
\newcommand{\Btheta}[0]{\boldsymbol{\theta}}
\newcommand{\Btau}[0]{\boldsymbol{\tau}}
\newcommand{\Bomega}[0]{\boldsymbol{\omega}}

%
% shorthand for unit vectors
%
\newcommand{\acap}[0]{\hat{\Ba}}
\newcommand{\bcap}[0]{\hat{\Bb}}
\newcommand{\ccap}[0]{\hat{\Bc}}
\newcommand{\dcap}[0]{\hat{\Bd}}
\newcommand{\ecap}[0]{\hat{\Be}}
\newcommand{\fcap}[0]{\hat{\Bf}}
\newcommand{\gcap}[0]{\hat{\Bg}}
\newcommand{\hcap}[0]{\hat{\Bh}}
\newcommand{\icap}[0]{\hat{\Bi}}
\newcommand{\jcap}[0]{\hat{\Bj}}
\newcommand{\kcap}[0]{\hat{\Bk}}
\newcommand{\lcap}[0]{\hat{\Bl}}
\newcommand{\mcap}[0]{\hat{\Bm}}
\newcommand{\ncap}[0]{\hat{\Bn}}
\newcommand{\ocap}[0]{\hat{\Bo}}
\newcommand{\pcap}[0]{\hat{\Bp}}
\newcommand{\qcap}[0]{\hat{\Bq}}
\newcommand{\rcap}[0]{\hat{\Br}}
\newcommand{\scap}[0]{\hat{\Bs}}
\newcommand{\tcap}[0]{\hat{\Bt}}
\newcommand{\ucap}[0]{\hat{\Bu}}
\newcommand{\vcap}[0]{\hat{\Bv}}
\newcommand{\wcap}[0]{\hat{\Bw}}
\newcommand{\xcap}[0]{\hat{\Bx}}
\newcommand{\ycap}[0]{\hat{\By}}
\newcommand{\zcap}[0]{\hat{\Bz}}
\newcommand{\thetacap}[0]{\hat{\Btheta}}

%
% to write R^n and C^n in a distinguishable fashion.  Perhaps change this
% to the double lined characters upon figuring out how to do so.
%
\newcommand{\C}[1]{$\mathbb{C}^{#1}$}
\newcommand{\R}[1]{$\mathbb{R}^{#1}$}

%
% various generally useful helpers
%

% derivative of #1 wrt. #2:
\newcommand{\D}[2] {\frac {d#2} {d#1}}

\newcommand{\inv}[1]{\frac{1}{#1}}
\newcommand{\cross}[0]{\times}

\newcommand{\abs}[1]{\lvert{#1}\rvert}
\newcommand{\norm}[1]{\lVert{#1}\rVert}
\newcommand{\innerprod}[2]{\langle{#1}, {#2}\rangle}
\newcommand{\dotprod}[2]{{#1} \cdot {#2}}
\newcommand{\bdotprod}[2]{\left({#1} \cdot {#2}\right)}
\newcommand{\crossprod}[2]{{#1} \cross {#2}}
\newcommand{\tripleprod}[3]{\dotprod{\left(\crossprod{#1}{#2}\right)}{#3}}

\DeclareMathOperator{\Proj}{Proj}
\DeclareMathOperator{\Span}{span}
\DeclareMathOperator{\Sgn}{sgn}
\DeclareMathOperator{\Area}{Area}
\DeclareMathOperator{\Volume}{Volume}

%
% A few miscellaneous things specific to this document
%
\newcommand{\crossop}[1]{\crossprod{#1}{}}

% R2 vector.
\newcommand{\VectorTwo}[2]{
\begin{bmatrix}
 {#1} \\
 {#2}
\end{bmatrix}
}

\newcommand{\VectorN}[1]{
\begin{bmatrix}
{#1}_1 \\
{#1}_2 \\
\vdots \\
{#1}_N \\
\end{bmatrix}
}

\newcommand{\DETuvij}[4]{
\begin{vmatrix}
 {#1}_{#3} & {#1}_{#4} \\
 {#2}_{#3} & {#2}_{#4}
\end{vmatrix}
}

\newcommand{\DETuvwijk}[6]{
\begin{vmatrix}
 {#1}_{#4} & {#1}_{#5} & {#1}_{#6} \\
 {#2}_{#4} & {#2}_{#5} & {#2}_{#6} \\
 {#3}_{#4} & {#3}_{#5} & {#3}_{#6}
\end{vmatrix}
}

\newcommand{\DETuvwxijkl}[8]{
\begin{vmatrix}
 {#1}_{#5} & {#1}_{#6} & {#1}_{#7} & {#1}_{#8} \\
 {#2}_{#5} & {#2}_{#6} & {#2}_{#7} & {#2}_{#8} \\
 {#3}_{#5} & {#3}_{#6} & {#3}_{#7} & {#3}_{#8} \\
 {#4}_{#5} & {#4}_{#6} & {#4}_{#7} & {#4}_{#8} \\
\end{vmatrix}
}

%\newcommand{\DETuvwxyijklm}[10]{
%\begin{vmatrix}
% {#1}_{#6} & {#1}_{#7} & {#1}_{#8} & {#1}_{#9} & {#1}_{#10} \\
% {#2}_{#6} & {#2}_{#7} & {#2}_{#8} & {#2}_{#9} & {#2}_{#10} \\
% {#3}_{#6} & {#3}_{#7} & {#3}_{#8} & {#3}_{#9} & {#3}_{#10} \\
% {#4}_{#6} & {#4}_{#7} & {#4}_{#8} & {#4}_{#9} & {#4}_{#10} \\
% {#5}_{#6} & {#5}_{#7} & {#5}_{#8} & {#5}_{#9} & {#5}_{#10}
%\end{vmatrix}
%}

% R3 vector.
\newcommand{\VectorThree}[3]{
\begin{bmatrix}
 {#1} \\
 {#2} \\
 {#3}
\end{bmatrix}
}



\author{Peeter Joot}
\email{peeter.joot@gmail.com}

%\documentclass[]{eliblogwidescreen}

\usepackage{amsmath}
\usepackage{mathpazo}

%
% shorthand for bold symbols, convenient for vectors and matrices
%
\newcommand{\Ba}[0]{\mathbf{a}}
\newcommand{\Bb}[0]{\mathbf{b}}
\newcommand{\Bc}[0]{\mathbf{c}}
\newcommand{\Bd}[0]{\mathbf{d}}
\newcommand{\Be}[0]{\mathbf{e}}
\newcommand{\Bf}[0]{\mathbf{f}}
\newcommand{\Bg}[0]{\mathbf{g}}
\newcommand{\Bh}[0]{\mathbf{h}}
\newcommand{\Bi}[0]{\mathbf{i}}
\newcommand{\Bj}[0]{\mathbf{j}}
\newcommand{\Bk}[0]{\mathbf{k}}
\newcommand{\Bl}[0]{\mathbf{l}}
\newcommand{\Bm}[0]{\mathbf{m}}
\newcommand{\Bn}[0]{\mathbf{n}}
\newcommand{\Bo}[0]{\mathbf{o}}
\newcommand{\Bp}[0]{\mathbf{p}}
\newcommand{\Bq}[0]{\mathbf{q}}
\newcommand{\Br}[0]{\mathbf{r}}
\newcommand{\Bs}[0]{\mathbf{s}}
\newcommand{\Bt}[0]{\mathbf{t}}
\newcommand{\Bu}[0]{\mathbf{u}}
\newcommand{\Bv}[0]{\mathbf{v}}
\newcommand{\Bw}[0]{\mathbf{w}}
\newcommand{\Bx}[0]{\mathbf{x}}
\newcommand{\By}[0]{\mathbf{y}}
\newcommand{\Bz}[0]{\mathbf{z}}
\newcommand{\BA}[0]{\mathbf{A}}
\newcommand{\BB}[0]{\mathbf{B}}
\newcommand{\BC}[0]{\mathbf{C}}
\newcommand{\BD}[0]{\mathbf{D}}
\newcommand{\BE}[0]{\mathbf{E}}
\newcommand{\BF}[0]{\mathbf{F}}
\newcommand{\BG}[0]{\mathbf{G}}
\newcommand{\BH}[0]{\mathbf{H}}
\newcommand{\BI}[0]{\mathbf{I}}
\newcommand{\BJ}[0]{\mathbf{J}}
\newcommand{\BK}[0]{\mathbf{K}}
\newcommand{\BL}[0]{\mathbf{L}}
\newcommand{\BM}[0]{\mathbf{M}}
\newcommand{\BN}[0]{\mathbf{N}}
\newcommand{\BO}[0]{\mathbf{O}}
\newcommand{\BP}[0]{\mathbf{P}}
\newcommand{\BQ}[0]{\mathbf{Q}}
\newcommand{\BR}[0]{\mathbf{R}}
\newcommand{\BS}[0]{\mathbf{S}}
\newcommand{\BT}[0]{\mathbf{T}}
\newcommand{\BU}[0]{\mathbf{U}}
\newcommand{\BV}[0]{\mathbf{V}}
\newcommand{\BW}[0]{\mathbf{W}}
\newcommand{\BX}[0]{\mathbf{X}}
\newcommand{\BY}[0]{\mathbf{Y}}
\newcommand{\BZ}[0]{\mathbf{Z}}

\newcommand{\Bzero}[0]{\mathbf{0}}
\newcommand{\Btheta}[0]{\boldsymbol{\theta}}
\newcommand{\Btau}[0]{\boldsymbol{\tau}}
\newcommand{\Bomega}[0]{\boldsymbol{\omega}}

%
% shorthand for unit vectors
%
\newcommand{\acap}[0]{\hat{\Ba}}
\newcommand{\bcap}[0]{\hat{\Bb}}
\newcommand{\ccap}[0]{\hat{\Bc}}
\newcommand{\dcap}[0]{\hat{\Bd}}
\newcommand{\ecap}[0]{\hat{\Be}}
\newcommand{\fcap}[0]{\hat{\Bf}}
\newcommand{\gcap}[0]{\hat{\Bg}}
\newcommand{\hcap}[0]{\hat{\Bh}}
\newcommand{\icap}[0]{\hat{\Bi}}
\newcommand{\jcap}[0]{\hat{\Bj}}
\newcommand{\kcap}[0]{\hat{\Bk}}
\newcommand{\lcap}[0]{\hat{\Bl}}
\newcommand{\mcap}[0]{\hat{\Bm}}
\newcommand{\ncap}[0]{\hat{\Bn}}
\newcommand{\ocap}[0]{\hat{\Bo}}
\newcommand{\pcap}[0]{\hat{\Bp}}
\newcommand{\qcap}[0]{\hat{\Bq}}
\newcommand{\rcap}[0]{\hat{\Br}}
\newcommand{\scap}[0]{\hat{\Bs}}
\newcommand{\tcap}[0]{\hat{\Bt}}
\newcommand{\ucap}[0]{\hat{\Bu}}
\newcommand{\vcap}[0]{\hat{\Bv}}
\newcommand{\wcap}[0]{\hat{\Bw}}
\newcommand{\xcap}[0]{\hat{\Bx}}
\newcommand{\ycap}[0]{\hat{\By}}
\newcommand{\zcap}[0]{\hat{\Bz}}
\newcommand{\thetacap}[0]{\hat{\Btheta}}

%
% to write R^n and C^n in a distinguishable fashion.  Perhaps change this
% to the double lined characters upon figuring out how to do so.
%
\newcommand{\C}[1]{$\mathbb{C}^{#1}$}
\newcommand{\R}[1]{$\mathbb{R}^{#1}$}

%
% various generally useful helpers
%

% derivative of #1 wrt. #2:
\newcommand{\D}[2] {\frac {d#2} {d#1}}

\newcommand{\inv}[1]{\frac{1}{#1}}
\newcommand{\cross}[0]{\times}

\newcommand{\abs}[1]{\lvert{#1}\rvert}
\newcommand{\norm}[1]{\lVert{#1}\rVert}
\newcommand{\innerprod}[2]{\langle{#1}, {#2}\rangle}
\newcommand{\dotprod}[2]{{#1} \cdot {#2}}
\newcommand{\bdotprod}[2]{\left({#1} \cdot {#2}\right)}
\newcommand{\crossprod}[2]{{#1} \cross {#2}}
\newcommand{\tripleprod}[3]{\dotprod{\left(\crossprod{#1}{#2}\right)}{#3}}

\DeclareMathOperator{\Proj}{Proj}
\DeclareMathOperator{\Span}{span}
\DeclareMathOperator{\Sgn}{sgn}
\DeclareMathOperator{\Area}{Area}
\DeclareMathOperator{\Volume}{Volume}

%
% A few miscellaneous things specific to this document
%
\newcommand{\crossop}[1]{\crossprod{#1}{}}

% R2 vector.
\newcommand{\VectorTwo}[2]{
\begin{bmatrix}
 {#1} \\
 {#2}
\end{bmatrix}
}

\newcommand{\VectorN}[1]{
\begin{bmatrix}
{#1}_1 \\
{#1}_2 \\
\vdots \\
{#1}_N \\
\end{bmatrix}
}

\newcommand{\DETuvij}[4]{
\begin{vmatrix}
 {#1}_{#3} & {#1}_{#4} \\
 {#2}_{#3} & {#2}_{#4}
\end{vmatrix}
}

\newcommand{\DETuvwijk}[6]{
\begin{vmatrix}
 {#1}_{#4} & {#1}_{#5} & {#1}_{#6} \\
 {#2}_{#4} & {#2}_{#5} & {#2}_{#6} \\
 {#3}_{#4} & {#3}_{#5} & {#3}_{#6}
\end{vmatrix}
}

\newcommand{\DETuvwxijkl}[8]{
\begin{vmatrix}
 {#1}_{#5} & {#1}_{#6} & {#1}_{#7} & {#1}_{#8} \\
 {#2}_{#5} & {#2}_{#6} & {#2}_{#7} & {#2}_{#8} \\
 {#3}_{#5} & {#3}_{#6} & {#3}_{#7} & {#3}_{#8} \\
 {#4}_{#5} & {#4}_{#6} & {#4}_{#7} & {#4}_{#8} \\
\end{vmatrix}
}

%\newcommand{\DETuvwxyijklm}[10]{
%\begin{vmatrix}
% {#1}_{#6} & {#1}_{#7} & {#1}_{#8} & {#1}_{#9} & {#1}_{#10} \\
% {#2}_{#6} & {#2}_{#7} & {#2}_{#8} & {#2}_{#9} & {#2}_{#10} \\
% {#3}_{#6} & {#3}_{#7} & {#3}_{#8} & {#3}_{#9} & {#3}_{#10} \\
% {#4}_{#6} & {#4}_{#7} & {#4}_{#8} & {#4}_{#9} & {#4}_{#10} \\
% {#5}_{#6} & {#5}_{#7} & {#5}_{#8} & {#5}_{#9} & {#5}_{#10}
%\end{vmatrix}
%}

% R3 vector.
\newcommand{\VectorThree}[3]{
\begin{bmatrix}
 {#1} \\
 {#2} \\
 {#3}
\end{bmatrix}
}



\author{Peeter Joot}
\email{peeter.joot@gmail.com}


\chapter{PHY454H1S Continuum Mechanics.  Lecture 22: Stability (cont).  Taught by Prof. K. Das.}
\label{chap:continuumL22}
%\useCCL
\blogpage{http://sites.google.com/site/peeterjoot2/math2012/continuumL22.pdf}
\date{Apr 4, 2012}
\gitRevisionInfo{continuumL22}
\keywords{Navier-Stokes, PHY454H1S, Prandtl number, Reynold number, stability}

\beginArtWithToc
%\beginArtNoToc

\section{Disclaimer.}

Peeter's lecture notes from class.  May not be entirely coherent.

\section{Review.  Rayleigh Benard Problem}

FIXME: F1

We'll take initial conditions

\begin{align}\label{eqn:continuumL22:10}
\PD{t}{\Bu} &= 0 \\
\PD{t}{T} &= 0
\end{align}

\begin{equation}\label{eqn:continuumL22:30}
\rho \PD{t}{\Bu} + \rho (\Bu \cdot \spacegrad) \Bu = - \spacegrad p + \mu \spacegrad^2 \Bu + \rho \Bg.
\end{equation}

Our energy equation is

\begin{equation}\label{eqn:continuumL22:50}
\PD{t}{T} + (\Bu \cdot \spacegrad) T = \kappa \spacegrad^2 T.
\end{equation}

We have this $\Bu \cdot \spacegrad$ term because our heat can be carried from one place to the other, due to the fluid motion.  We'd not have this convective term for heat dissipation in solids because elements of a solid are not moving around in the bulk.

We'll also use 

\begin{equation}\label{eqn:continuumL22:70}
\spacegrad p_s = \rho_s \Bg
\end{equation}

In the steady (base) state we have

\begin{equation}\label{eqn:continuumL22:90}
0 = \kappa \spacegrad^2 T = 
\kappa \left( 
\PDSq{x}{}
+\PDSq{y}{}
+\PDSq{z}{} \right) T,
\end{equation}

but since we are only considering spatial variation with $z$ we have

\begin{equation}\label{eqn:continuumL22:110}
\kappa \PDSq{z}{} T_s = 0,
\end{equation}

with solution

\begin{equation}\label{eqn:continuumL22:130}
T_s = T_0 - \frac{\Delta T}{d} z.
\end{equation}

We found that after application of the pertubation

\begin{align}\label{eqn:continuumL22:150}
\Bu &\rightarrow 0 + \delta \Bu \\
p &\rightarrow p_s + \delta p \\
\rho &\rightarrow p_s + \delta \rho \\
T &\rightarrow p_s + \delta T
\end{align}

to the base state equations, our perturbed Navier-Stokes equation was

\begin{equation}\label{eqn:continuumL22:170}
\spacegrad^2 \left( \PD{t}{} - \nu \spacegrad^2 \right) \delta w = g \alpha 
\left(
\PDSq{x}{}
+\PDSq{y}{}
\right)
\delta T.
\end{equation}

\section{Application of the pertubation to the energy equation.}

\begin{equation}\label{eqn:continuumL22:190}
\PD{t}{T_s + \delta T} + (\delta \Bu \cdot \spacegrad) (T_s + \delta T) = \kappa \spacegrad^2 (T_s + \delta T)
\end{equation}

We've got

\begin{equation}\label{eqn:continuumL22:210}
\PD{t}{T_s} = 0.
\end{equation}

Using this, and \ref{eqn:continuumL22:110}, and neglecting any terms of second order smallness we have

\begin{equation}\label{eqn:continuumL22:230}
\boxed{
\PD{t}{\delta T} + \delta \Bu \cdot \spacegrad T_s = \kappa \spacegrad^2 \delta T.
}
\end{equation}

We'd like to solve this and \ref{eqn:continuumL22:170} simultaneously.

\subsection{Non-dimensionalisation}

We'd like to scale

\begin{equation}\label{eqn:continuumL22:250}
\begin{array}{l l}
x,y,z & \quad \mbox{with $d$} \\
t & \quad \mbox{with $d^2/\nu$} \\
\delta w & \quad \mbox{with $\kappa/d$} \\
\delta T & \quad \mbox{with $\delta T$}
\end{array}
\end{equation}

Sanity check of dimensions: we note that

\begin{equation}\label{eqn:continuumL22:270}
[\nu] \sim [ \frac{1}{\text{T}} ]/ \inv{\text{L}^2} ] \sim \frac{\text{L}^2}{\text{T}},
\end{equation}

so the dimensions of are appropriate

\begin{equation}\label{eqn:continuumL22:290}
\left[\frac{d^2}{\nu}\right] \sim \frac{\text{L}^2}{\text{L}^2 \text{T}^{-1}} \sim \text{T}.
\end{equation}

In the primed coordinates (dropping primes) we have

\begin{equation}\label{eqn:continuumL22:310}
\boxed{
\spacegrad^2 \left( 
\PD{t} - \spacegrad^2 
\right)
\delta w
=
R \left( 
\PDSq{x}{}
+\PDSq{y}{}
\right) \delta T.
}
\end{equation}

FIXME: do this in more detail.

where

\begin{equation}\label{eqn:continuumL22:330}
R = \frac{g \Delta T d^2}{\nu \alpha}.
\end{equation}

For our energy equation, our non-dimensionalization (also implicitly using primed coordinates) gives us

\begin{equation}\label{eqn:continuumL22:350}
\boxed{
\left( 
\text{P}_r
\PD{t}{} - \spacegrad^2 \right) \delta T = \delta w
}
\end{equation}

where 
\begin{equation}\label{eqn:continuumL22:370}
\text{P}_r = \frac{\nu}{\kappa},
\end{equation}

is the \textit{Prandtl number}.

\subsection{Normal mode analysis.}

Let's assume plane wave solutions

\begin{align}\label{eqn:continuumL22:390}
\delta w &= w(z) e^{ i ( k_1 x + k_2 y) + \sigma t} \\
\delta T &= \Theta(z) e^{ i ( k_1 x + k_2 y) + \sigma t}.
\end{align}

Here $\sigma$ is the growth rate.  We also note that $k_1$ and $k_2$ are the same in both equations, where we are solving for a coupled state between the two equations \ref{eqn:continuumL22:310}, \ref{eqn:continuumL22:350}.

%  We can illustrate these as in
%
%FIXME: F2

Observe that 

\begin{equation}\label{eqn:continuumL22:410}
\spacegrad^2 \delta w 
= 
\left( 
\PDSq{x}{}
+\PDSq{y}{}
+\PDSq{z}{}
\right) \delta w
=
\left( \frac{d^2}{dz^2} - \Bk^2 \right) \delta w,
\end{equation}

where 

\begin{equation}\label{eqn:continuumL22:430}
\Bk^2 = k_1^2 + k_2^2.
\end{equation}

NOTE: $a^2$ was used for $\Bk^2$ in class.  I don't know if that's a convention, but $\Bk^2$ seemed more intuitive.

Writing

\begin{equation}\label{eqn:continuumL22:570}
D = \frac{d}{dz}
\end{equation}

we find

\begin{align}\label{eqn:continuumL22:450}
(D^2 - \Bk^2) ( \sigma - (D^2 - \Bk^2) ) w &= -R \Bk^2 w \\
(D^2 - \Bk^2 -\text{P}_r ) \theta &= - w.
\end{align}

We find

\begin{equation}\label{eqn:continuumL22:590}
w = D^2 w = 0
\end{equation}

on $z = 0,1$, and solutions

\begin{align}\label{eqn:continuumL22:470}
w(z) &= A_n \sin( n \pi z ) \\
\Theta(z) &= B_n \sin( n \pi z ).
\end{align}

Observe that we have, for any $A_n$, $B_n$ a zero determinant for $\sigma = 0$

\begin{equation}\label{eqn:continuumL22:490}
0 = 
\begin{vmatrix}
(n^2 \pi^2 - \Bk^2)^2 - \sigma (n^2 \pi^2 - a^2) & -R \Bk^2 \\
1 & n^2 \pi^2 - \Bk^2 - \text{P}_r \sigma
\end{vmatrix}.
\end{equation}

On the other hand for
\begin{itemize}
\item $\Real(\sigma) > 0$ ($\Delta T > \Delta T_e$), we have an instable system.
\item $\Real(\sigma) < 0$ ($\Delta T > \Delta T_e$), we have a stable system.
\end{itemize}

The critical value of Reynold's number

\begin{equation}\label{eqn:continuumL22:510}
R = \frac{n^2 \pi^2 - \Bk^2}{\Bk^2}.
\end{equation}

This is illustrated in 

FIXME: F3

The instability means that we'll have instable flows as illustrated in 

FIXME: F4.

Solving for these critical points we find

\begin{subequations}
\begin{equation}\label{eqn:continuumL22:530}
a_e^2 = \frac{\pi^2}{2}
\end{equation}
\begin{equation}\label{eqn:continuumL22:550}
R_e = \frac{27 \pi^4}{4}
\end{equation}
\end{subequations}

%FIXME: Reading: \S XX from \cite{acheson1990elementary}

%\EndArticle
\EndNoBibArticle
