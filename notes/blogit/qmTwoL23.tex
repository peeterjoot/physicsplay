%
% Copyright � 2015 Peeter Joot.  All Rights Reserved.
% Licenced as described in the file LICENSE under the root directory of this GIT repository.
%
\documentclass[]{eliblog}

\usepackage{amsmath}
\usepackage{mathpazo}

%
% shorthand for bold symbols, convenient for vectors and matrices
%
\newcommand{\Ba}[0]{\mathbf{a}}
\newcommand{\Bb}[0]{\mathbf{b}}
\newcommand{\Bc}[0]{\mathbf{c}}
\newcommand{\Bd}[0]{\mathbf{d}}
\newcommand{\Be}[0]{\mathbf{e}}
\newcommand{\Bf}[0]{\mathbf{f}}
\newcommand{\Bg}[0]{\mathbf{g}}
\newcommand{\Bh}[0]{\mathbf{h}}
\newcommand{\Bi}[0]{\mathbf{i}}
\newcommand{\Bj}[0]{\mathbf{j}}
\newcommand{\Bk}[0]{\mathbf{k}}
\newcommand{\Bl}[0]{\mathbf{l}}
\newcommand{\Bm}[0]{\mathbf{m}}
\newcommand{\Bn}[0]{\mathbf{n}}
\newcommand{\Bo}[0]{\mathbf{o}}
\newcommand{\Bp}[0]{\mathbf{p}}
\newcommand{\Bq}[0]{\mathbf{q}}
\newcommand{\Br}[0]{\mathbf{r}}
\newcommand{\Bs}[0]{\mathbf{s}}
\newcommand{\Bt}[0]{\mathbf{t}}
\newcommand{\Bu}[0]{\mathbf{u}}
\newcommand{\Bv}[0]{\mathbf{v}}
\newcommand{\Bw}[0]{\mathbf{w}}
\newcommand{\Bx}[0]{\mathbf{x}}
\newcommand{\By}[0]{\mathbf{y}}
\newcommand{\Bz}[0]{\mathbf{z}}
\newcommand{\BA}[0]{\mathbf{A}}
\newcommand{\BB}[0]{\mathbf{B}}
\newcommand{\BC}[0]{\mathbf{C}}
\newcommand{\BD}[0]{\mathbf{D}}
\newcommand{\BE}[0]{\mathbf{E}}
\newcommand{\BF}[0]{\mathbf{F}}
\newcommand{\BG}[0]{\mathbf{G}}
\newcommand{\BH}[0]{\mathbf{H}}
\newcommand{\BI}[0]{\mathbf{I}}
\newcommand{\BJ}[0]{\mathbf{J}}
\newcommand{\BK}[0]{\mathbf{K}}
\newcommand{\BL}[0]{\mathbf{L}}
\newcommand{\BM}[0]{\mathbf{M}}
\newcommand{\BN}[0]{\mathbf{N}}
\newcommand{\BO}[0]{\mathbf{O}}
\newcommand{\BP}[0]{\mathbf{P}}
\newcommand{\BQ}[0]{\mathbf{Q}}
\newcommand{\BR}[0]{\mathbf{R}}
\newcommand{\BS}[0]{\mathbf{S}}
\newcommand{\BT}[0]{\mathbf{T}}
\newcommand{\BU}[0]{\mathbf{U}}
\newcommand{\BV}[0]{\mathbf{V}}
\newcommand{\BW}[0]{\mathbf{W}}
\newcommand{\BX}[0]{\mathbf{X}}
\newcommand{\BY}[0]{\mathbf{Y}}
\newcommand{\BZ}[0]{\mathbf{Z}}

\newcommand{\Bzero}[0]{\mathbf{0}}
\newcommand{\Btheta}[0]{\boldsymbol{\theta}}
\newcommand{\Btau}[0]{\boldsymbol{\tau}}
\newcommand{\Bomega}[0]{\boldsymbol{\omega}}

%
% shorthand for unit vectors
%
\newcommand{\acap}[0]{\hat{\Ba}}
\newcommand{\bcap}[0]{\hat{\Bb}}
\newcommand{\ccap}[0]{\hat{\Bc}}
\newcommand{\dcap}[0]{\hat{\Bd}}
\newcommand{\ecap}[0]{\hat{\Be}}
\newcommand{\fcap}[0]{\hat{\Bf}}
\newcommand{\gcap}[0]{\hat{\Bg}}
\newcommand{\hcap}[0]{\hat{\Bh}}
\newcommand{\icap}[0]{\hat{\Bi}}
\newcommand{\jcap}[0]{\hat{\Bj}}
\newcommand{\kcap}[0]{\hat{\Bk}}
\newcommand{\lcap}[0]{\hat{\Bl}}
\newcommand{\mcap}[0]{\hat{\Bm}}
\newcommand{\ncap}[0]{\hat{\Bn}}
\newcommand{\ocap}[0]{\hat{\Bo}}
\newcommand{\pcap}[0]{\hat{\Bp}}
\newcommand{\qcap}[0]{\hat{\Bq}}
\newcommand{\rcap}[0]{\hat{\Br}}
\newcommand{\scap}[0]{\hat{\Bs}}
\newcommand{\tcap}[0]{\hat{\Bt}}
\newcommand{\ucap}[0]{\hat{\Bu}}
\newcommand{\vcap}[0]{\hat{\Bv}}
\newcommand{\wcap}[0]{\hat{\Bw}}
\newcommand{\xcap}[0]{\hat{\Bx}}
\newcommand{\ycap}[0]{\hat{\By}}
\newcommand{\zcap}[0]{\hat{\Bz}}
\newcommand{\thetacap}[0]{\hat{\Btheta}}

%
% to write R^n and C^n in a distinguishable fashion.  Perhaps change this
% to the double lined characters upon figuring out how to do so.
%
\newcommand{\C}[1]{$\mathbb{C}^{#1}$}
\newcommand{\R}[1]{$\mathbb{R}^{#1}$}

%
% various generally useful helpers
%

% derivative of #1 wrt. #2:
\newcommand{\D}[2] {\frac {d#2} {d#1}}

\newcommand{\inv}[1]{\frac{1}{#1}}
\newcommand{\cross}[0]{\times}

\newcommand{\abs}[1]{\lvert{#1}\rvert}
\newcommand{\norm}[1]{\lVert{#1}\rVert}
\newcommand{\innerprod}[2]{\langle{#1}, {#2}\rangle}
\newcommand{\dotprod}[2]{{#1} \cdot {#2}}
\newcommand{\bdotprod}[2]{\left({#1} \cdot {#2}\right)}
\newcommand{\crossprod}[2]{{#1} \cross {#2}}
\newcommand{\tripleprod}[3]{\dotprod{\left(\crossprod{#1}{#2}\right)}{#3}}

\DeclareMathOperator{\Proj}{Proj}
\DeclareMathOperator{\Span}{span}
\DeclareMathOperator{\Sgn}{sgn}
\DeclareMathOperator{\Area}{Area}
\DeclareMathOperator{\Volume}{Volume}

%
% A few miscellaneous things specific to this document
%
\newcommand{\crossop}[1]{\crossprod{#1}{}}

% R2 vector.
\newcommand{\VectorTwo}[2]{
\begin{bmatrix}
 {#1} \\
 {#2}
\end{bmatrix}
}

\newcommand{\VectorN}[1]{
\begin{bmatrix}
{#1}_1 \\
{#1}_2 \\
\vdots \\
{#1}_N \\
\end{bmatrix}
}

\newcommand{\DETuvij}[4]{
\begin{vmatrix}
 {#1}_{#3} & {#1}_{#4} \\
 {#2}_{#3} & {#2}_{#4}
\end{vmatrix}
}

\newcommand{\DETuvwijk}[6]{
\begin{vmatrix}
 {#1}_{#4} & {#1}_{#5} & {#1}_{#6} \\
 {#2}_{#4} & {#2}_{#5} & {#2}_{#6} \\
 {#3}_{#4} & {#3}_{#5} & {#3}_{#6}
\end{vmatrix}
}

\newcommand{\DETuvwxijkl}[8]{
\begin{vmatrix}
 {#1}_{#5} & {#1}_{#6} & {#1}_{#7} & {#1}_{#8} \\
 {#2}_{#5} & {#2}_{#6} & {#2}_{#7} & {#2}_{#8} \\
 {#3}_{#5} & {#3}_{#6} & {#3}_{#7} & {#3}_{#8} \\
 {#4}_{#5} & {#4}_{#6} & {#4}_{#7} & {#4}_{#8} \\
\end{vmatrix}
}

%\newcommand{\DETuvwxyijklm}[10]{
%\begin{vmatrix}
% {#1}_{#6} & {#1}_{#7} & {#1}_{#8} & {#1}_{#9} & {#1}_{#10} \\
% {#2}_{#6} & {#2}_{#7} & {#2}_{#8} & {#2}_{#9} & {#2}_{#10} \\
% {#3}_{#6} & {#3}_{#7} & {#3}_{#8} & {#3}_{#9} & {#3}_{#10} \\
% {#4}_{#6} & {#4}_{#7} & {#4}_{#8} & {#4}_{#9} & {#4}_{#10} \\
% {#5}_{#6} & {#5}_{#7} & {#5}_{#8} & {#5}_{#9} & {#5}_{#10}
%\end{vmatrix}
%}

% R3 vector.
\newcommand{\VectorThree}[3]{
\begin{bmatrix}
 {#1} \\
 {#2} \\
 {#3}
\end{bmatrix}
}



\author{Peeter Joot}
\email{peeter.joot@gmail.com}

%\documentclass[]{eliblogwidescreen}

\usepackage{amsmath}
\usepackage{mathpazo}

%
% shorthand for bold symbols, convenient for vectors and matrices
%
\newcommand{\Ba}[0]{\mathbf{a}}
\newcommand{\Bb}[0]{\mathbf{b}}
\newcommand{\Bc}[0]{\mathbf{c}}
\newcommand{\Bd}[0]{\mathbf{d}}
\newcommand{\Be}[0]{\mathbf{e}}
\newcommand{\Bf}[0]{\mathbf{f}}
\newcommand{\Bg}[0]{\mathbf{g}}
\newcommand{\Bh}[0]{\mathbf{h}}
\newcommand{\Bi}[0]{\mathbf{i}}
\newcommand{\Bj}[0]{\mathbf{j}}
\newcommand{\Bk}[0]{\mathbf{k}}
\newcommand{\Bl}[0]{\mathbf{l}}
\newcommand{\Bm}[0]{\mathbf{m}}
\newcommand{\Bn}[0]{\mathbf{n}}
\newcommand{\Bo}[0]{\mathbf{o}}
\newcommand{\Bp}[0]{\mathbf{p}}
\newcommand{\Bq}[0]{\mathbf{q}}
\newcommand{\Br}[0]{\mathbf{r}}
\newcommand{\Bs}[0]{\mathbf{s}}
\newcommand{\Bt}[0]{\mathbf{t}}
\newcommand{\Bu}[0]{\mathbf{u}}
\newcommand{\Bv}[0]{\mathbf{v}}
\newcommand{\Bw}[0]{\mathbf{w}}
\newcommand{\Bx}[0]{\mathbf{x}}
\newcommand{\By}[0]{\mathbf{y}}
\newcommand{\Bz}[0]{\mathbf{z}}
\newcommand{\BA}[0]{\mathbf{A}}
\newcommand{\BB}[0]{\mathbf{B}}
\newcommand{\BC}[0]{\mathbf{C}}
\newcommand{\BD}[0]{\mathbf{D}}
\newcommand{\BE}[0]{\mathbf{E}}
\newcommand{\BF}[0]{\mathbf{F}}
\newcommand{\BG}[0]{\mathbf{G}}
\newcommand{\BH}[0]{\mathbf{H}}
\newcommand{\BI}[0]{\mathbf{I}}
\newcommand{\BJ}[0]{\mathbf{J}}
\newcommand{\BK}[0]{\mathbf{K}}
\newcommand{\BL}[0]{\mathbf{L}}
\newcommand{\BM}[0]{\mathbf{M}}
\newcommand{\BN}[0]{\mathbf{N}}
\newcommand{\BO}[0]{\mathbf{O}}
\newcommand{\BP}[0]{\mathbf{P}}
\newcommand{\BQ}[0]{\mathbf{Q}}
\newcommand{\BR}[0]{\mathbf{R}}
\newcommand{\BS}[0]{\mathbf{S}}
\newcommand{\BT}[0]{\mathbf{T}}
\newcommand{\BU}[0]{\mathbf{U}}
\newcommand{\BV}[0]{\mathbf{V}}
\newcommand{\BW}[0]{\mathbf{W}}
\newcommand{\BX}[0]{\mathbf{X}}
\newcommand{\BY}[0]{\mathbf{Y}}
\newcommand{\BZ}[0]{\mathbf{Z}}

\newcommand{\Bzero}[0]{\mathbf{0}}
\newcommand{\Btheta}[0]{\boldsymbol{\theta}}
\newcommand{\Btau}[0]{\boldsymbol{\tau}}
\newcommand{\Bomega}[0]{\boldsymbol{\omega}}

%
% shorthand for unit vectors
%
\newcommand{\acap}[0]{\hat{\Ba}}
\newcommand{\bcap}[0]{\hat{\Bb}}
\newcommand{\ccap}[0]{\hat{\Bc}}
\newcommand{\dcap}[0]{\hat{\Bd}}
\newcommand{\ecap}[0]{\hat{\Be}}
\newcommand{\fcap}[0]{\hat{\Bf}}
\newcommand{\gcap}[0]{\hat{\Bg}}
\newcommand{\hcap}[0]{\hat{\Bh}}
\newcommand{\icap}[0]{\hat{\Bi}}
\newcommand{\jcap}[0]{\hat{\Bj}}
\newcommand{\kcap}[0]{\hat{\Bk}}
\newcommand{\lcap}[0]{\hat{\Bl}}
\newcommand{\mcap}[0]{\hat{\Bm}}
\newcommand{\ncap}[0]{\hat{\Bn}}
\newcommand{\ocap}[0]{\hat{\Bo}}
\newcommand{\pcap}[0]{\hat{\Bp}}
\newcommand{\qcap}[0]{\hat{\Bq}}
\newcommand{\rcap}[0]{\hat{\Br}}
\newcommand{\scap}[0]{\hat{\Bs}}
\newcommand{\tcap}[0]{\hat{\Bt}}
\newcommand{\ucap}[0]{\hat{\Bu}}
\newcommand{\vcap}[0]{\hat{\Bv}}
\newcommand{\wcap}[0]{\hat{\Bw}}
\newcommand{\xcap}[0]{\hat{\Bx}}
\newcommand{\ycap}[0]{\hat{\By}}
\newcommand{\zcap}[0]{\hat{\Bz}}
\newcommand{\thetacap}[0]{\hat{\Btheta}}

%
% to write R^n and C^n in a distinguishable fashion.  Perhaps change this
% to the double lined characters upon figuring out how to do so.
%
\newcommand{\C}[1]{$\mathbb{C}^{#1}$}
\newcommand{\R}[1]{$\mathbb{R}^{#1}$}

%
% various generally useful helpers
%

% derivative of #1 wrt. #2:
\newcommand{\D}[2] {\frac {d#2} {d#1}}

\newcommand{\inv}[1]{\frac{1}{#1}}
\newcommand{\cross}[0]{\times}

\newcommand{\abs}[1]{\lvert{#1}\rvert}
\newcommand{\norm}[1]{\lVert{#1}\rVert}
\newcommand{\innerprod}[2]{\langle{#1}, {#2}\rangle}
\newcommand{\dotprod}[2]{{#1} \cdot {#2}}
\newcommand{\bdotprod}[2]{\left({#1} \cdot {#2}\right)}
\newcommand{\crossprod}[2]{{#1} \cross {#2}}
\newcommand{\tripleprod}[3]{\dotprod{\left(\crossprod{#1}{#2}\right)}{#3}}

\DeclareMathOperator{\Proj}{Proj}
\DeclareMathOperator{\Span}{span}
\DeclareMathOperator{\Sgn}{sgn}
\DeclareMathOperator{\Area}{Area}
\DeclareMathOperator{\Volume}{Volume}

%
% A few miscellaneous things specific to this document
%
\newcommand{\crossop}[1]{\crossprod{#1}{}}

% R2 vector.
\newcommand{\VectorTwo}[2]{
\begin{bmatrix}
 {#1} \\
 {#2}
\end{bmatrix}
}

\newcommand{\VectorN}[1]{
\begin{bmatrix}
{#1}_1 \\
{#1}_2 \\
\vdots \\
{#1}_N \\
\end{bmatrix}
}

\newcommand{\DETuvij}[4]{
\begin{vmatrix}
 {#1}_{#3} & {#1}_{#4} \\
 {#2}_{#3} & {#2}_{#4}
\end{vmatrix}
}

\newcommand{\DETuvwijk}[6]{
\begin{vmatrix}
 {#1}_{#4} & {#1}_{#5} & {#1}_{#6} \\
 {#2}_{#4} & {#2}_{#5} & {#2}_{#6} \\
 {#3}_{#4} & {#3}_{#5} & {#3}_{#6}
\end{vmatrix}
}

\newcommand{\DETuvwxijkl}[8]{
\begin{vmatrix}
 {#1}_{#5} & {#1}_{#6} & {#1}_{#7} & {#1}_{#8} \\
 {#2}_{#5} & {#2}_{#6} & {#2}_{#7} & {#2}_{#8} \\
 {#3}_{#5} & {#3}_{#6} & {#3}_{#7} & {#3}_{#8} \\
 {#4}_{#5} & {#4}_{#6} & {#4}_{#7} & {#4}_{#8} \\
\end{vmatrix}
}

%\newcommand{\DETuvwxyijklm}[10]{
%\begin{vmatrix}
% {#1}_{#6} & {#1}_{#7} & {#1}_{#8} & {#1}_{#9} & {#1}_{#10} \\
% {#2}_{#6} & {#2}_{#7} & {#2}_{#8} & {#2}_{#9} & {#2}_{#10} \\
% {#3}_{#6} & {#3}_{#7} & {#3}_{#8} & {#3}_{#9} & {#3}_{#10} \\
% {#4}_{#6} & {#4}_{#7} & {#4}_{#8} & {#4}_{#9} & {#4}_{#10} \\
% {#5}_{#6} & {#5}_{#7} & {#5}_{#8} & {#5}_{#9} & {#5}_{#10}
%\end{vmatrix}
%}

% R3 vector.
\newcommand{\VectorThree}[3]{
\begin{bmatrix}
 {#1} \\
 {#2} \\
 {#3}
\end{bmatrix}
}



\author{Peeter Joot}
\email{peeter.joot@gmail.com}


\chapter{PHY456H1F: Quantum Mechanics II.  Lecture L23 (Taught by Prof J.E. Sipe).  3D Scattering.}
\label{chap:qmTwoL23}
%\useCCL
\blogpage{http://sites.google.com/site/peeterjoot2/math2011/qmTwoL23.pdf}
\date{Nov 30, 2011}
\revisionInfo{qmTwoL23.tex}

\beginArtWithToc
%\beginArtNoToc

\section{Disclaimer.}

Peeter's lecture notes from class.  May not be entirely coherent.

\section{3D Scattering.}

READING: \S 20, and \S 4.8 of our text \cite{desai2009quantum}.

We continue to consider scattering off of a positive potential as depicted in figure (\ref{fig:qmTwoL23:qmTwoL23fig1})
\begin{figure}[htp]
   \centering
   \includegraphics[totalheight=0.4\textheight]{qmTwoL23fig1}
   \caption{Radially bounded potential.}\label{fig:qmTwoL23:qmTwoL23fig1}
\end{figure}

Here we have $V(r) = 0$ for $r > r_0$.  The wave function

\begin{equation}\label{eqn:qmTwoL23:20}
e^{i k \ncap \cdot \Br}
\end{equation}

is found to be a solution of the free particle Schr\"{o}dinger equation.

\begin{equation}\label{eqn:qmTwoL23:40}
- \frac{\hbar^2}{2\mu} \spacegrad^2
e^{i k \ncap \cdot \Br}
 = \frac{\hbar^2 \Bk^2}{2 \mu}
e^{i k \ncap \cdot \Br}
\end{equation}

What other solutions can be found for $r > r_0$, where our potential $V(r) = 0$?  We are looking for $\Phi(\Br)$ such that

\begin{equation}\label{eqn:qmTwoL23:60}
- \frac{\hbar^2}{2\mu} \spacegrad^2
\Phi(r)
 = \frac{\hbar^2 \Bk^2}{2 \mu}
\Phi(r)
\end{equation}

What can we find?

We split our Laplacian into radial and angular components as we did for the hydrogen atom

\begin{equation}\label{eqn:qmTwoL23:80}
- \frac{\hbar^2}{2\mu} \PDSq{r}{} (r \Phi(\Br)) +
\frac{\mathcal{L}^2}{2 \mu r^2}
\Phi(\Br)
=
E \Phi(\Br),
\end{equation}

where

\begin{equation}\label{eqn:qmTwoL23:100}
\mathcal{L}^2 = -\hbar^2 \left(
\PDSq{\theta}{}
+ \inv{\tan\theta} \PD{\theta}
+ \inv{\sin^2\theta} \PDSq{\phi}{}
\right)
\end{equation}

Assuming a solution of

\begin{equation}\label{eqn:qmTwoL23:120}
\Phi(\Br) = R(r) Y_l^m(\theta, \phi),
\end{equation}

and noting that

\begin{equation}\label{eqn:qmTwoL23:140}
\mathcal{L}^2 Y_l^m(\theta, \phi) = \hbar^2 l (l+1) Y_l^m(\theta, \phi),
\end{equation}

we find that our radial equation becomes

\begin{equation}\label{eqn:qmTwoL23:160}
- \frac{\hbar^2}{2 \mu r} \PDSq{r}{} (r R(r))
+\frac{\hbar^2 l (l+1)
}{2 \mu r^2}
R(r)
=
E R(r)
=
\frac{\hbar^2 k^2}{2\mu} R(r).
\end{equation}

Writing

\begin{equation}\label{eqn:qmTwoL23:180}
R(r) = \frac{u(r)}{r},
\end{equation}

we have

\begin{equation}\label{eqn:qmTwoL23:200}
- \frac{\hbar^2}{2 \mu r} \PDSq{r}{u(r)}
+\frac{\hbar^2 l (l+1)
}{2 \mu r}
u(r)
=
\frac{\hbar^2 k^2}{2\mu}
 \frac{u(r)}{r},
\end{equation}

or

\begin{equation}\label{eqn:qmTwoL23:220}
\left( \frac{d^2}{dr^2} + k^2 -\frac{l (l+1) }{r^2} \right) u(r) = 0
\end{equation}

Writing $\rho = k r$, we have

\begin{equation}\label{eqn:qmTwoL23:240}
\left( \frac{d^2}{d\rho^2} + 1 -\frac{l (l+1) }{\rho^2} \right) u(r) = 0
\end{equation}

With a last substitution of $u(r) = U( k r ) = U(\rho)$, and introducing an explicit $l$ suffix on our eigenfunction $U(\rho)$ we have

\begin{equation}\label{eqn:qmTwoL23:260}
\left( -\frac{d^2}{d\rho^2} +\frac{l (l+1) }{\rho^2} \right) U_l(\rho) = U_l(\rho).
\end{equation}

We'd not have done this before with the hydrogen atomc since we had only finite $E = \hbar^2 k^2/2 \mu$.  Now this can be anything.  Solutions to this are the Spherical Bessel functions of order $l$

\begin{equation}\label{eqn:qmTwoL23:280}
\frac{U_l(\rho)}{\rho} = j_l(\rho) = (-\rho)^l \left( \inv{\rho} \frac{d}{d\rho} \right)^l \left( \frac{\sin\rho}{\rho} \right)
\end{equation}

We can easily calculate

\begin{align}\label{eqn:qmTwoL23:300}
U_0(\rho) &= \rho j_0(\rho) = \sin\rho \\
U_1(\rho) &= \rho j_1(\rho) = -\cos\rho + \frac{\sin\rho}{\rho}
\end{align}

and can plug these into \ref{eqn:qmTwoL23:260} to verify that they are a solution.  A more general proof looks a bit tricker.

There's another set of functions, called the Neumann functions (of order $l$) that are also solutions, and those are

\begin{equation}\label{eqn:qmTwoL23:320}
n_l(\rho) = (-\rho)^l \left( \inv{\rho} \frac{d}{d\rho} \right)^l \left( -\frac{\cos\rho}{\rho} \right).
\end{equation}

Observe that these ones are less well behaved at the origin.  To calculate the first few Bessel and Neumann functions we first compute

\begin{align*}
\inv{\rho} \ddrho{} \frac{\sin\rho}{\rho}
&= \inv{\rho} \left(
\frac{\cos\rho}{\rho}
-\frac{\sin\rho}{\rho^2}
\right) \\
&=
\frac{\cos\rho}{\rho^2}
-\frac{\sin\rho}{\rho^3}
\end{align*}

\begin{align*}
\left( \inv{\rho} \ddrho{} \right)^2 \frac{\sin\rho}{\rho}
&= \inv{\rho} \left(
-\frac{\sin\rho}{\rho^2}
-2\frac{\cos\rho}{\rho^3}
-\frac{\cos\rho}{\rho^3}
+3\frac{\sin\rho}{\rho^4}
\right) \\
&=
\sin\rho\left(
-\frac{1}{\rho^3}
+\frac{3}{\rho^5}
\right)
-3\frac{\cos\rho}{\rho^4}
\end{align*}

and
\begin{align*}
\inv{\rho} \ddrho{} -\frac{\cos\rho}{\rho}
&= \inv{\rho} \left(
\frac{\sin\rho}{\rho}
+\frac{\cos\rho}{\rho^2}
\right) \\
&=
\frac{\sin\rho}{\rho^2}
+\frac{\cos\rho}{\rho^3}
\end{align*}

\begin{align*}
\left( \inv{\rho} \ddrho{} \right)^2 -\frac{\cos\rho}{\rho}
&= \inv{\rho} \left(
\frac{\cos\rho}{\rho^2}
-2\frac{\sin\rho}{\rho^3}
-\frac{\sin\rho}{\rho^3}
-3\frac{\cos\rho}{\rho^4}
\right) \\
&=
\cos\rho\left(
\frac{1}{\rho^3}
-\frac{3}{\rho^5}
\right)
-3\frac{\sin\rho}{\rho^4}
\end{align*}

so we find

\begin{equation}\label{eqn:qmTwoL23:340}
\begin{array}{l l l l}
j_0(\rho) &= \frac{\sin\rho}{\rho} 					& n_0(\rho) &= -\frac{\cos\rho}{\rho} 	\\
j_1(\rho) &= \frac{\sin\rho}{\rho^2} -\frac{\cos\rho}{\rho} 		& n_1(\rho) &= -\frac{\cos\rho}{\rho^2} -\frac{\sin\rho}{\rho} \\
j_2(\rho) &= \sin\rho \left(-\frac{1}{\rho} + \frac{3}{\rho^3} \right) +\cos\rho \left(-\frac{3}{\rho^2} \right)
& n_2(\rho) &= \cos\rho \left(\frac{1}{\rho} - \frac{3}{\rho^3} \right) +\sin\rho \left(-\frac{3}{\rho^2} \right)
\end{array}
\end{equation}

Observe that our radial functions $R(r)$ are proportional to these Bessel and Neumann functions

\begin{align*}
R(r)
&= \frac{u(r)}{r}  \\
&= \frac{U(kr)}{r}  \\
&=
\left\{
\begin{array}{l}
\frac{j_l(\rho) \rho}{r} \\
\frac{n_l(\rho) \rho}{r}
\end{array}
\right. \\
&=
\left\{
\begin{array}{l}
\frac{j_l(\rho) k \cancel{r}}{\cancel{r}} \\
\frac{n_l(\rho) k \cancel{r}}{\cancel{r}}
\end{array}
\right.
\end{align*}

Or

\begin{equation}\label{eqn:qmTwoL23:360}
R(r) \sim j_l(\rho), n_l(\rho).
\end{equation}

\subsection{Limits of spherical Bessel and Neumann functions}

With $n!!$ denoting the double factorial, like factorial but skipping every other term

\begin{equation}\label{eqn:qmTwoL23:400}
n!! = n(n-2)(n-4) \cdots,
\end{equation}

we can show that in the limit as $\rho \rightarrow 0$ we have

\begin{align}\label{eqn:qmTwoL23:380}
j_l(\rho) &\rightarrow \frac{\rho^l}{(2 l + 1)!!} \\
n_l(\rho) &\rightarrow -\frac{(2 l - 1)!!}{\rho^{(l+1)}},
\end{align}

Comparing this to our explicit expansion for $j_1(\rho)$ in \ref{eqn:qmTwoL23:340} where we appear to have a $1/\rho$ dependence for small $\rho$ it is not obvious that this would be the case.  To compute this we need to start with a power series expansion for $\sin\rho/\rho$, which is well behaved at $\rho =0$ and then the result follows (done later).

It is apparently also possible to show that as $\rho \rightarrow \infty$ we have

\begin{align}\label{eqn:qmTwoL23:420}
j_l(\rho) &\rightarrow \inv{\rho} \sin\left( \rho - \frac{l \pi}{2} \right) \\
n_l(\rho) &\rightarrow -\inv{\rho} \cos\left( \rho - \frac{l \pi}{2} \right) .
\end{align}


For $r > r_0$ we can construct (for fixed $k$) a superposition of the spherical functions

\begin{equation}\label{eqn:qmTwoL23:480}
\sum_l \sum_m \left( A_l j_l( k r ) + B_l n_l(k r) \right) Y_l^m(\theta, \phi)
\end{equation}

we want outgoing waves, and as $r \rightarrow \infty$, we have

\begin{align}\label{eqn:qmTwoL23:500}
j_l(k r) &\rightarrow \frac{\sin\left(kr - \frac{l \pi}{2}\right)}{k r} \\
n_l(k r) &\rightarrow -\frac{\cos\left(kr - \frac{l \pi}{2}\right)}{k r}
\end{align}

Put $A_l/B_l = -i$ for a given $l$ we have

\begin{equation}\label{eqn:qmTwoL23:520}
\inv{k r} \left( -i
\frac{\sin\left(kr - \frac{l \pi}{2}\right)}{k r}
-\frac{\cos\left(kr - \frac{l \pi}{2}\right)}{k r} \right)
\sim \inv{k r} e^{i (k r - \pi l/2)}
\end{equation}

For

\begin{equation}\label{eqn:qmTwoL23:540}
\sum_l
\sum_m B_l
\inv{k r} e^{i (k r - \pi l/2)} Y_l^m(\theta, \phi).
\end{equation}

Making this choice to achieve \underline{outgoing} waves (and factoring a $(-i)^l$ out of $B_l$ for some reason, we have another wave function that satisfies our Hamiltonian equation

\begin{equation}\label{eqn:qmTwoL23:560}
\frac{e^{i k r}}{k r}
\sum_l
\sum_m
(-1)^l
B_l
Y_l^m(\theta, \phi).
\end{equation}

The $B_l$ coefficients will depend on $V(r)$ for the incident wave $e^{i \Bk \cdot \Br}$.  Suppose we encapsulate that dependence in a helper function $f_\Bk(\theta, \phi)$ and write

\begin{equation}\label{eqn:qmTwoL23:580}
\frac{e^{i k r}}{r} f_\Bk(\theta, \phi)
\end{equation}

We seek a solution $\psi_\Bk(\Br)$

\begin{equation}\label{eqn:qmTwoL23:60b}
\left( - \frac{\hbar^2}{2\mu} \spacegrad^2
+ V(\Br)
\right)
\psi_\Bk(\Br)
 = \frac{\hbar^2 \Bk^2}{2 \mu}
\psi_\Bk(\Br),
\end{equation}

where as $r \rightarrow \infty$

\begin{equation}\label{eqn:qmTwoL23:600}
\psi_\Bk(\Br) \rightarrow e^{i \Bk \cdot \Br} + \frac{e^{i k r}}{r} f_\Bk(\theta, \phi).
\end{equation}

Note that for $r < r_0$ in general for finite $r$, $\psi_k(\Br)$, is much more complicated.  This is the analogue of the plane wave result

\begin{equation}\label{eqn:qmTwoL23:620}
\psi(x) = e^{i k x} + \beta_k e^{-i k x}
\end{equation}


\section{Scattering geometry and nomenclature.}

We can think classically first, and imagine a scattering of a stream of particles barraging a target as in
figure (\ref{fig:qmTwoL23:qmTwoL23fig2})

\begin{figure}[htp]
   \centering
   \includegraphics[totalheight=0.3\textheight]{qmTwoL23fig2}
   \caption{Scattering cross section.}\label{fig:qmTwoL23:qmTwoL23fig2}
\end{figure}

Here we assume that $d\Omega$ is far enought away that it includes no non-scattering particles.

Write $P$ for the number density

\begin{equation}\label{eqn:qmTwoL23:680}
P = \frac{\text{number of particles}}{\text{unit volume}},
\end{equation}

and

\begin{equation}\label{eqn:qmTwoL23:700}
J = P v_0 =
\frac{
\text{Number of particles flowing through}
}{
\text{a unit area in unit time}
}
\end{equation}

We want to count the rate of particles per unit time $dN$ through this solid angle $d\Omega$ and write

\begin{equation}\label{eqn:qmTwoL23:720}
dN = J \left( \frac{d \sigma(\Omega)}{d\Omega} \right) d\Omega.
\end{equation}

The factor

\begin{equation}\label{eqn:qmTwoL23:740}
\frac{d \sigma(\Omega)}{d\Omega},
\end{equation}

is called the differential cross section, and has ``units'' of 

\begin{equation}\label{eqn:qmTwoL23:760}
\frac{\text{area}}{\text{steradians}}
\end{equation}

(recalling that steradians are radian like measures of solid angle \cite{wiki:steradian}).

The total number of particles through the volume per unit time is then

\begin{equation}\label{eqn:qmTwoL23:780}
\int J \frac{d \sigma(\Omega)}{d\Omega} d\Omega
= J \int \frac{d \sigma(\Omega)}{d\Omega} d\Omega
= J \sigma
\end{equation}

where $\sigma$ is the total cross section and has units of area.  The cross section $\sigma$ his the effective size of the area required to collect all particles, and characterizes the scattering, but isn't neccessarily entirely geometrical.  For example, in photon scattering we may have frequency matching with atomic resonance, finding $\sigma \sim \lambda^2$, something that can be much bigger than the actual total area involved.


\section{Appendix}

\subsection{Q: Are Bessel and Neumann functions orthogonal?}

\paragraph{Answer:} There is an orthogonality relation, but it is not one of plain old multiplication.

Curious about this, I find an orthogonality condition in \cite{wiki:bessel}

\begin{equation}\label{eqn:qmTwoL23:640}
\int_0^\infty J_\alpha(z) J_\beta(z) \frac{dz}{z} = \frac{2}{\pi} \frac{\sin\left(\frac{\pi}{2}\left( \alpha - \beta\right) \right) }{\alpha^2 - \beta^2},
\end{equation}

from which we find for the spherical Bessel functions

\begin{equation}\label{eqn:qmTwoL23:660}
\int_0^\infty j_l(\rho) j_m(\rho) d\rho =
\frac{\sin\left(\frac{\pi}{2}\left( l - m \right) \right) }{(l+ 1/2)^2 - (m + 1/2)^2}.
\end{equation}

Is this a satisfactory orthogonality integral?  At a glance it doesn't appear to be well behaved for $l = m$, but perhaps we the limit can be taken.

\subsection{Deriving the small limit Bessel and Neumann function approximations.}

Writing the $\sinc$ function in series form

\begin{equation}\label{eqn:qmTwoL23:800}
\frac{\sin x}{x} = \sum_k=0^\infty (-1)^k \frac{x^{2k}}{(2k + 1)!},
\end{equation}

we can differentiate easily

\begin{equation}\label{eqn:qmTwoL23:820}
\begin{aligned}
\inv{x} \ddx{} \frac{\sin x}{x} 
&= \sum_k=1^\infty (-1)^k (2k) \frac{x^{2k-2}}{(2k + 1)!} \\
% j = k - 1
% k = j + 1
% 2k = 2j + 2
% j -> k
&= (-1) \sum_k=0^\infty (-1)^k (2k + 2) \frac{x^{2k}}{(2k + 3)!} \\
&= (-1) \sum_k=0^\infty (-1)^k \inv{2k + 3} \frac{x^{2k}}{(2k + 1)!} \\
\end{aligned}
\end{equation}

\subsection{Deriving the large limit Bessel and Neumann function approximations.}

For \ref{eqn:qmTwoL23:420} we are referred to any ``good book on electromagnetism'' for details.  Perhaps \cite{jackson1975cew} is thought to be such a book.  In \S 16.1 the spherical Bessel and Neumann functions are related to the plain old Bessel functions with

\begin{align}\label{eqn:qmTwoL23:440}
j_l(x) &= \sqrt{\frac{\pi}{2x} } J_{l+1/2}(x) \\
n_l(x) &= \sqrt{\frac{\pi}{2x} } N_{l+1/2}(x)
\end{align}

Referring back to \S 3.7 of that text where the limiting forms of the Bessel functions are given

\begin{align}\label{eqn:qmTwoL23:460}
J_\nu(x) &\rightarrow \sqrt{\frac{2}{\pi x}} \cos\left(x - \frac{\nu\pi}{2} - \frac{\pi}{4} \right) \\
N_\nu(x) &\rightarrow \sqrt{\frac{2}{\pi x}} \sin\left(x - \frac{\nu\pi}{2} - \frac{\pi}{4} \right)
\end{align}

This does give us our desired identities, but there's no hint in the text how one would derive \ref{eqn:qmTwoL23:460} from the power series that was computed by solving the Bessel equation.

\EndArticle
