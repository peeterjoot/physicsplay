%
% Copyright � 2015 Peeter Joot.  All Rights Reserved.
% Licenced as described in the file LICENSE under the root directory of this GIT repository.
%
\documentclass[]{eliblog}

\usepackage{amsmath}
\usepackage{mathpazo}

%
% shorthand for bold symbols, convenient for vectors and matrices
%
\newcommand{\Ba}[0]{\mathbf{a}}
\newcommand{\Bb}[0]{\mathbf{b}}
\newcommand{\Bc}[0]{\mathbf{c}}
\newcommand{\Bd}[0]{\mathbf{d}}
\newcommand{\Be}[0]{\mathbf{e}}
\newcommand{\Bf}[0]{\mathbf{f}}
\newcommand{\Bg}[0]{\mathbf{g}}
\newcommand{\Bh}[0]{\mathbf{h}}
\newcommand{\Bi}[0]{\mathbf{i}}
\newcommand{\Bj}[0]{\mathbf{j}}
\newcommand{\Bk}[0]{\mathbf{k}}
\newcommand{\Bl}[0]{\mathbf{l}}
\newcommand{\Bm}[0]{\mathbf{m}}
\newcommand{\Bn}[0]{\mathbf{n}}
\newcommand{\Bo}[0]{\mathbf{o}}
\newcommand{\Bp}[0]{\mathbf{p}}
\newcommand{\Bq}[0]{\mathbf{q}}
\newcommand{\Br}[0]{\mathbf{r}}
\newcommand{\Bs}[0]{\mathbf{s}}
\newcommand{\Bt}[0]{\mathbf{t}}
\newcommand{\Bu}[0]{\mathbf{u}}
\newcommand{\Bv}[0]{\mathbf{v}}
\newcommand{\Bw}[0]{\mathbf{w}}
\newcommand{\Bx}[0]{\mathbf{x}}
\newcommand{\By}[0]{\mathbf{y}}
\newcommand{\Bz}[0]{\mathbf{z}}
\newcommand{\BA}[0]{\mathbf{A}}
\newcommand{\BB}[0]{\mathbf{B}}
\newcommand{\BC}[0]{\mathbf{C}}
\newcommand{\BD}[0]{\mathbf{D}}
\newcommand{\BE}[0]{\mathbf{E}}
\newcommand{\BF}[0]{\mathbf{F}}
\newcommand{\BG}[0]{\mathbf{G}}
\newcommand{\BH}[0]{\mathbf{H}}
\newcommand{\BI}[0]{\mathbf{I}}
\newcommand{\BJ}[0]{\mathbf{J}}
\newcommand{\BK}[0]{\mathbf{K}}
\newcommand{\BL}[0]{\mathbf{L}}
\newcommand{\BM}[0]{\mathbf{M}}
\newcommand{\BN}[0]{\mathbf{N}}
\newcommand{\BO}[0]{\mathbf{O}}
\newcommand{\BP}[0]{\mathbf{P}}
\newcommand{\BQ}[0]{\mathbf{Q}}
\newcommand{\BR}[0]{\mathbf{R}}
\newcommand{\BS}[0]{\mathbf{S}}
\newcommand{\BT}[0]{\mathbf{T}}
\newcommand{\BU}[0]{\mathbf{U}}
\newcommand{\BV}[0]{\mathbf{V}}
\newcommand{\BW}[0]{\mathbf{W}}
\newcommand{\BX}[0]{\mathbf{X}}
\newcommand{\BY}[0]{\mathbf{Y}}
\newcommand{\BZ}[0]{\mathbf{Z}}

\newcommand{\Bzero}[0]{\mathbf{0}}
\newcommand{\Btheta}[0]{\boldsymbol{\theta}}
\newcommand{\Btau}[0]{\boldsymbol{\tau}}
\newcommand{\Bomega}[0]{\boldsymbol{\omega}}

%
% shorthand for unit vectors
%
\newcommand{\acap}[0]{\hat{\Ba}}
\newcommand{\bcap}[0]{\hat{\Bb}}
\newcommand{\ccap}[0]{\hat{\Bc}}
\newcommand{\dcap}[0]{\hat{\Bd}}
\newcommand{\ecap}[0]{\hat{\Be}}
\newcommand{\fcap}[0]{\hat{\Bf}}
\newcommand{\gcap}[0]{\hat{\Bg}}
\newcommand{\hcap}[0]{\hat{\Bh}}
\newcommand{\icap}[0]{\hat{\Bi}}
\newcommand{\jcap}[0]{\hat{\Bj}}
\newcommand{\kcap}[0]{\hat{\Bk}}
\newcommand{\lcap}[0]{\hat{\Bl}}
\newcommand{\mcap}[0]{\hat{\Bm}}
\newcommand{\ncap}[0]{\hat{\Bn}}
\newcommand{\ocap}[0]{\hat{\Bo}}
\newcommand{\pcap}[0]{\hat{\Bp}}
\newcommand{\qcap}[0]{\hat{\Bq}}
\newcommand{\rcap}[0]{\hat{\Br}}
\newcommand{\scap}[0]{\hat{\Bs}}
\newcommand{\tcap}[0]{\hat{\Bt}}
\newcommand{\ucap}[0]{\hat{\Bu}}
\newcommand{\vcap}[0]{\hat{\Bv}}
\newcommand{\wcap}[0]{\hat{\Bw}}
\newcommand{\xcap}[0]{\hat{\Bx}}
\newcommand{\ycap}[0]{\hat{\By}}
\newcommand{\zcap}[0]{\hat{\Bz}}
\newcommand{\thetacap}[0]{\hat{\Btheta}}

%
% to write R^n and C^n in a distinguishable fashion.  Perhaps change this
% to the double lined characters upon figuring out how to do so.
%
\newcommand{\C}[1]{$\mathbb{C}^{#1}$}
\newcommand{\R}[1]{$\mathbb{R}^{#1}$}

%
% various generally useful helpers
%

% derivative of #1 wrt. #2:
\newcommand{\D}[2] {\frac {d#2} {d#1}}

\newcommand{\inv}[1]{\frac{1}{#1}}
\newcommand{\cross}[0]{\times}

\newcommand{\abs}[1]{\lvert{#1}\rvert}
\newcommand{\norm}[1]{\lVert{#1}\rVert}
\newcommand{\innerprod}[2]{\langle{#1}, {#2}\rangle}
\newcommand{\dotprod}[2]{{#1} \cdot {#2}}
\newcommand{\bdotprod}[2]{\left({#1} \cdot {#2}\right)}
\newcommand{\crossprod}[2]{{#1} \cross {#2}}
\newcommand{\tripleprod}[3]{\dotprod{\left(\crossprod{#1}{#2}\right)}{#3}}

\DeclareMathOperator{\Proj}{Proj}
\DeclareMathOperator{\Span}{span}
\DeclareMathOperator{\Sgn}{sgn}
\DeclareMathOperator{\Area}{Area}
\DeclareMathOperator{\Volume}{Volume}

%
% A few miscellaneous things specific to this document
%
\newcommand{\crossop}[1]{\crossprod{#1}{}}

% R2 vector.
\newcommand{\VectorTwo}[2]{
\begin{bmatrix}
 {#1} \\
 {#2}
\end{bmatrix}
}

\newcommand{\VectorN}[1]{
\begin{bmatrix}
{#1}_1 \\
{#1}_2 \\
\vdots \\
{#1}_N \\
\end{bmatrix}
}

\newcommand{\DETuvij}[4]{
\begin{vmatrix}
 {#1}_{#3} & {#1}_{#4} \\
 {#2}_{#3} & {#2}_{#4}
\end{vmatrix}
}

\newcommand{\DETuvwijk}[6]{
\begin{vmatrix}
 {#1}_{#4} & {#1}_{#5} & {#1}_{#6} \\
 {#2}_{#4} & {#2}_{#5} & {#2}_{#6} \\
 {#3}_{#4} & {#3}_{#5} & {#3}_{#6}
\end{vmatrix}
}

\newcommand{\DETuvwxijkl}[8]{
\begin{vmatrix}
 {#1}_{#5} & {#1}_{#6} & {#1}_{#7} & {#1}_{#8} \\
 {#2}_{#5} & {#2}_{#6} & {#2}_{#7} & {#2}_{#8} \\
 {#3}_{#5} & {#3}_{#6} & {#3}_{#7} & {#3}_{#8} \\
 {#4}_{#5} & {#4}_{#6} & {#4}_{#7} & {#4}_{#8} \\
\end{vmatrix}
}

%\newcommand{\DETuvwxyijklm}[10]{
%\begin{vmatrix}
% {#1}_{#6} & {#1}_{#7} & {#1}_{#8} & {#1}_{#9} & {#1}_{#10} \\
% {#2}_{#6} & {#2}_{#7} & {#2}_{#8} & {#2}_{#9} & {#2}_{#10} \\
% {#3}_{#6} & {#3}_{#7} & {#3}_{#8} & {#3}_{#9} & {#3}_{#10} \\
% {#4}_{#6} & {#4}_{#7} & {#4}_{#8} & {#4}_{#9} & {#4}_{#10} \\
% {#5}_{#6} & {#5}_{#7} & {#5}_{#8} & {#5}_{#9} & {#5}_{#10}
%\end{vmatrix}
%}

% R3 vector.
\newcommand{\VectorThree}[3]{
\begin{bmatrix}
 {#1} \\
 {#2} \\
 {#3}
\end{bmatrix}
}



\author{Peeter Joot}
\email{peeter.joot@gmail.com}

%\documentclass[]{eliblogwidescreen}

\usepackage{amsmath}
\usepackage{mathpazo}

%
% shorthand for bold symbols, convenient for vectors and matrices
%
\newcommand{\Ba}[0]{\mathbf{a}}
\newcommand{\Bb}[0]{\mathbf{b}}
\newcommand{\Bc}[0]{\mathbf{c}}
\newcommand{\Bd}[0]{\mathbf{d}}
\newcommand{\Be}[0]{\mathbf{e}}
\newcommand{\Bf}[0]{\mathbf{f}}
\newcommand{\Bg}[0]{\mathbf{g}}
\newcommand{\Bh}[0]{\mathbf{h}}
\newcommand{\Bi}[0]{\mathbf{i}}
\newcommand{\Bj}[0]{\mathbf{j}}
\newcommand{\Bk}[0]{\mathbf{k}}
\newcommand{\Bl}[0]{\mathbf{l}}
\newcommand{\Bm}[0]{\mathbf{m}}
\newcommand{\Bn}[0]{\mathbf{n}}
\newcommand{\Bo}[0]{\mathbf{o}}
\newcommand{\Bp}[0]{\mathbf{p}}
\newcommand{\Bq}[0]{\mathbf{q}}
\newcommand{\Br}[0]{\mathbf{r}}
\newcommand{\Bs}[0]{\mathbf{s}}
\newcommand{\Bt}[0]{\mathbf{t}}
\newcommand{\Bu}[0]{\mathbf{u}}
\newcommand{\Bv}[0]{\mathbf{v}}
\newcommand{\Bw}[0]{\mathbf{w}}
\newcommand{\Bx}[0]{\mathbf{x}}
\newcommand{\By}[0]{\mathbf{y}}
\newcommand{\Bz}[0]{\mathbf{z}}
\newcommand{\BA}[0]{\mathbf{A}}
\newcommand{\BB}[0]{\mathbf{B}}
\newcommand{\BC}[0]{\mathbf{C}}
\newcommand{\BD}[0]{\mathbf{D}}
\newcommand{\BE}[0]{\mathbf{E}}
\newcommand{\BF}[0]{\mathbf{F}}
\newcommand{\BG}[0]{\mathbf{G}}
\newcommand{\BH}[0]{\mathbf{H}}
\newcommand{\BI}[0]{\mathbf{I}}
\newcommand{\BJ}[0]{\mathbf{J}}
\newcommand{\BK}[0]{\mathbf{K}}
\newcommand{\BL}[0]{\mathbf{L}}
\newcommand{\BM}[0]{\mathbf{M}}
\newcommand{\BN}[0]{\mathbf{N}}
\newcommand{\BO}[0]{\mathbf{O}}
\newcommand{\BP}[0]{\mathbf{P}}
\newcommand{\BQ}[0]{\mathbf{Q}}
\newcommand{\BR}[0]{\mathbf{R}}
\newcommand{\BS}[0]{\mathbf{S}}
\newcommand{\BT}[0]{\mathbf{T}}
\newcommand{\BU}[0]{\mathbf{U}}
\newcommand{\BV}[0]{\mathbf{V}}
\newcommand{\BW}[0]{\mathbf{W}}
\newcommand{\BX}[0]{\mathbf{X}}
\newcommand{\BY}[0]{\mathbf{Y}}
\newcommand{\BZ}[0]{\mathbf{Z}}

\newcommand{\Bzero}[0]{\mathbf{0}}
\newcommand{\Btheta}[0]{\boldsymbol{\theta}}
\newcommand{\Btau}[0]{\boldsymbol{\tau}}
\newcommand{\Bomega}[0]{\boldsymbol{\omega}}

%
% shorthand for unit vectors
%
\newcommand{\acap}[0]{\hat{\Ba}}
\newcommand{\bcap}[0]{\hat{\Bb}}
\newcommand{\ccap}[0]{\hat{\Bc}}
\newcommand{\dcap}[0]{\hat{\Bd}}
\newcommand{\ecap}[0]{\hat{\Be}}
\newcommand{\fcap}[0]{\hat{\Bf}}
\newcommand{\gcap}[0]{\hat{\Bg}}
\newcommand{\hcap}[0]{\hat{\Bh}}
\newcommand{\icap}[0]{\hat{\Bi}}
\newcommand{\jcap}[0]{\hat{\Bj}}
\newcommand{\kcap}[0]{\hat{\Bk}}
\newcommand{\lcap}[0]{\hat{\Bl}}
\newcommand{\mcap}[0]{\hat{\Bm}}
\newcommand{\ncap}[0]{\hat{\Bn}}
\newcommand{\ocap}[0]{\hat{\Bo}}
\newcommand{\pcap}[0]{\hat{\Bp}}
\newcommand{\qcap}[0]{\hat{\Bq}}
\newcommand{\rcap}[0]{\hat{\Br}}
\newcommand{\scap}[0]{\hat{\Bs}}
\newcommand{\tcap}[0]{\hat{\Bt}}
\newcommand{\ucap}[0]{\hat{\Bu}}
\newcommand{\vcap}[0]{\hat{\Bv}}
\newcommand{\wcap}[0]{\hat{\Bw}}
\newcommand{\xcap}[0]{\hat{\Bx}}
\newcommand{\ycap}[0]{\hat{\By}}
\newcommand{\zcap}[0]{\hat{\Bz}}
\newcommand{\thetacap}[0]{\hat{\Btheta}}

%
% to write R^n and C^n in a distinguishable fashion.  Perhaps change this
% to the double lined characters upon figuring out how to do so.
%
\newcommand{\C}[1]{$\mathbb{C}^{#1}$}
\newcommand{\R}[1]{$\mathbb{R}^{#1}$}

%
% various generally useful helpers
%

% derivative of #1 wrt. #2:
\newcommand{\D}[2] {\frac {d#2} {d#1}}

\newcommand{\inv}[1]{\frac{1}{#1}}
\newcommand{\cross}[0]{\times}

\newcommand{\abs}[1]{\lvert{#1}\rvert}
\newcommand{\norm}[1]{\lVert{#1}\rVert}
\newcommand{\innerprod}[2]{\langle{#1}, {#2}\rangle}
\newcommand{\dotprod}[2]{{#1} \cdot {#2}}
\newcommand{\bdotprod}[2]{\left({#1} \cdot {#2}\right)}
\newcommand{\crossprod}[2]{{#1} \cross {#2}}
\newcommand{\tripleprod}[3]{\dotprod{\left(\crossprod{#1}{#2}\right)}{#3}}

\DeclareMathOperator{\Proj}{Proj}
\DeclareMathOperator{\Span}{span}
\DeclareMathOperator{\Sgn}{sgn}
\DeclareMathOperator{\Area}{Area}
\DeclareMathOperator{\Volume}{Volume}

%
% A few miscellaneous things specific to this document
%
\newcommand{\crossop}[1]{\crossprod{#1}{}}

% R2 vector.
\newcommand{\VectorTwo}[2]{
\begin{bmatrix}
 {#1} \\
 {#2}
\end{bmatrix}
}

\newcommand{\VectorN}[1]{
\begin{bmatrix}
{#1}_1 \\
{#1}_2 \\
\vdots \\
{#1}_N \\
\end{bmatrix}
}

\newcommand{\DETuvij}[4]{
\begin{vmatrix}
 {#1}_{#3} & {#1}_{#4} \\
 {#2}_{#3} & {#2}_{#4}
\end{vmatrix}
}

\newcommand{\DETuvwijk}[6]{
\begin{vmatrix}
 {#1}_{#4} & {#1}_{#5} & {#1}_{#6} \\
 {#2}_{#4} & {#2}_{#5} & {#2}_{#6} \\
 {#3}_{#4} & {#3}_{#5} & {#3}_{#6}
\end{vmatrix}
}

\newcommand{\DETuvwxijkl}[8]{
\begin{vmatrix}
 {#1}_{#5} & {#1}_{#6} & {#1}_{#7} & {#1}_{#8} \\
 {#2}_{#5} & {#2}_{#6} & {#2}_{#7} & {#2}_{#8} \\
 {#3}_{#5} & {#3}_{#6} & {#3}_{#7} & {#3}_{#8} \\
 {#4}_{#5} & {#4}_{#6} & {#4}_{#7} & {#4}_{#8} \\
\end{vmatrix}
}

%\newcommand{\DETuvwxyijklm}[10]{
%\begin{vmatrix}
% {#1}_{#6} & {#1}_{#7} & {#1}_{#8} & {#1}_{#9} & {#1}_{#10} \\
% {#2}_{#6} & {#2}_{#7} & {#2}_{#8} & {#2}_{#9} & {#2}_{#10} \\
% {#3}_{#6} & {#3}_{#7} & {#3}_{#8} & {#3}_{#9} & {#3}_{#10} \\
% {#4}_{#6} & {#4}_{#7} & {#4}_{#8} & {#4}_{#9} & {#4}_{#10} \\
% {#5}_{#6} & {#5}_{#7} & {#5}_{#8} & {#5}_{#9} & {#5}_{#10}
%\end{vmatrix}
%}

% R3 vector.
\newcommand{\VectorThree}[3]{
\begin{bmatrix}
 {#1} \\
 {#2} \\
 {#3}
\end{bmatrix}
}



\author{Peeter Joot}
\email{peeter.joot@gmail.com}


\chapter{Notes and problems for Desai Chapter 26.}
\label{chap:desaiCh26}
%\useCCL
\blogpage{http://sites.google.com/site/peeterjoot/math2010/desaiCh26.pdf}
\date{Dec 9, 2010}
\revisionInfo{desaiCh26.tex}

\beginArtWithToc
%\beginArtNoToc

\section{Motivation.}

Chapter 26 notes for \cite{desai2009quantum}.

\section{Guts}

\subsection{Trig relations.}

To verify equations 26.3-5 in the text it's worth noting that 

\begin{align*}
\cos(a + b) 
&= \Re( e^{ia} e^{ib} ) \\
&= \Re( (\cos a + i \sin a)( \cos b + i \sin b) ) \\
&= \cos a \cos b - \sin a \sin b
\end{align*}

and
\begin{align*}
\sin(a + b) 
&= \Im( e^{ia} e^{ib} ) \\
&= \Im( (\cos a + i \sin a)( \cos b + i \sin b) ) \\
&= \cos a \sin b + \sin a \cos b
\end{align*}

So, for 
\begin{align}\label{eqn:desaiCh26:10}
x &= \rho \cos\alpha \\
y &= \rho \sin\alpha 
\end{align}

the transformed coordinates are
\begin{align*}
x' 
&= \rho \cos(\alpha + \phi) \\
&= \rho (\cos \alpha \cos \phi - \sin \alpha \sin \phi) \\
&= x \cos \phi - y \sin \phi
\end{align*}

and
\begin{align*}
y' 
&= \rho \sin(\alpha + \phi) \\
&= \rho (\cos \alpha \sin \phi + \sin \alpha \cos \phi) \\
&= x \sin \phi + y \cos \phi \\
\end{align*}

This allows us to read off the rotation matrix.  Without all the messy trig, we can also derive this matrix with geometric algebra.

\begin{align*}
\Bv' 
&= e^{- \Be_1 \Be_2 \phi/2 } \Bv e^{ \Be_1 \Be_2 \phi/2 } \\
&= v_3 \Be_3 + (v_1 \Be_1 + v_2 \Be_2) e^{ \Be_1 \Be_2 \phi } \\
&= v_3 \Be_3 + (v_1 \Be_1 + v_2 \Be_2) (\cos \phi + \Be_1 \Be_2 \sin\phi) \\
&= v_3 \Be_3 
+ \Be_1 (v_1 \cos\phi - v_2 \sin\phi)
+ \Be_2 (v_2 \cos\phi + v_1 \sin\phi)
\end{align*}

Here we use the Pauli-matrix like identities

\begin{align}\label{eqn:desaiCh26:20}
\Be_k^2 &= 1 \\
\Be_i \Be_j &= -\Be_j \Be_i,\quad i\ne j
\end{align}

and also note that $\Be_3$ commutes with the bivector for the $x,y$ plane $\Be_1 \Be_2$.  We can also read off the rotation matrix from this.

\subsection{Infinitesimal transformations.}

Recall that in the problems of Chapter 5, one representation of spin one matrices were calculated \chapcite{desaiCh5}.  Since the choice of the basis vectors was arbitrary in that exersize, we ended up with a different representation.  For $S_x, S_y, S_z$ as found in (26.20) and (26.23) we can also verify easily that we have eigenvalues $0, \pm \hbar$.  We can also show that our spin kets in this non-diagonal representation have the following column matrix representations:

\begin{align}\label{eqn:desaiCh26:30}
\ket{1,\pm 1}_x 
&=
\inv{\sqrt{2}} \begin{bmatrix}
0 \\
1 \\
\pm i
\end{bmatrix} \\
\ket{1,0}_x 
&=
\begin{bmatrix}
1 \\
0 \\
0 
\end{bmatrix} \\
\ket{1,\pm 1}_y 
&=
\inv{\sqrt{2}} \begin{bmatrix}
\pm i \\
0 \\
1 
\end{bmatrix} \\
\ket{1,0}_y 
&=
\begin{bmatrix}
0 \\
1 \\
0 
\end{bmatrix} \\
\ket{1,\pm 1}_z 
&=
\inv{\sqrt{2}} \begin{bmatrix}
1 \\
\pm i \\
0
\end{bmatrix} \\
\ket{1,0}_z 
&=
\begin{bmatrix}
0 \\
0 \\
1
\end{bmatrix} 
\end{align}

\subsection{Verifying the commutator relations.}

Given the (summation convention) matrix representation for the spin one operators

\begin{equation}\label{eqn:desaiCh26:40}
(S_i)_{jk} = - i \hbar \epsilon_{ijk},
\end{equation}

let's demonstrate the commutator relation of (26.25).

\begin{align*}
{\antisymmetric{S_i}{S_j}}_{rs} 
&=
(S_i S_j - S_j S_i)_{rs} \\
&=
\sum_t (S_i)_{rt} (S_j)_{ts} - (S_j)_{rt} (S_i)_{ts} \\
&=
(-i\hbar)^2 \sum_t \epsilon_{irt} \epsilon_{jts} - \epsilon_{jrt} \epsilon_{its} \\
&=
-(-i\hbar)^2 \sum_t \epsilon_{tir} \epsilon_{tjs} - \epsilon_{tjr} \epsilon_{tis} \\
\end{align*}

Now we can employ the summation rule for sums products of antisymmetic tensors over one free index (4.179) 

\begin{equation}\label{eqn:desaiCh26:50}
\sum_i 
\epsilon_{ijk} \epsilon_{iab}
= 
\delta_{ja}
\delta_{kb}
-\delta_{jb}
\delta_{ka}.
\end{equation}

Continuing we get
\begin{align*}
{\antisymmetric{S_i}{S_j}}_{rs} 
&=
-(-i\hbar)^2 \left(
\delta_{ij}
\delta_{rs}
-\delta_{is}
\delta_{rj}
-\delta_{ji}
\delta_{rs}
+\delta_{js}
\delta_{ri} \right) \\
&=
(-i\hbar)^2 \left( 
\delta_{is}
\delta_{jr}
-
\delta_{ir} 
\delta_{js}
\right)
\\
&=
(-i\hbar)^2 \sum_t \epsilon_{tij} \epsilon_{tsr}
\\
&=
i\hbar \sum_t \epsilon_{tij} (S_t)_{rs}
\qquad\square
\end{align*}

\subsection{General infinitesimal rotation.}

Equation (26.26) has for an infinitesimal rotation counterclockwise around the unit axis of rotation vector $\Bn$

\begin{equation}\label{eqn:desaiCh26:60}
\BV' = \BV + \epsilon \Bn \cross \BV.
\end{equation}

Let's derive this using the geometric algebra rotation expression for the same

\begin{align*}
\BV' 
&=
e^{-I\Bn \alpha/2}
\BV 
e^{I\Bn \alpha/2} \\
&=
e^{-I\Bn \alpha/2}
\left(
(\BV \cdot \Bn)\Bn
+(\BV \wedge \Bn)\Bn
\right)
e^{I\Bn \alpha/2} \\
&=
(\BV \cdot \Bn)\Bn
+(\BV \wedge \Bn)\Bn
e^{I\Bn \alpha}
\end{align*}

We note that $I\Bn$ and thus the exponential commutes with $\Bn$, and the projection component in the normal direction.  Similarily $I\Bn$ anticommutes with $(\BV \wedge \Bn) \Bn$.  This leaves us with

\begin{align*}
\BV' 
&=
(\BV \cdot \Bn)\Bn
\left(
+(\BV \wedge \Bn)\Bn
\right)
( \cos \alpha + I \Bn \sin\alpha)
\end{align*}

For $\alpha = \epsilon \rightarrow 0$, this is

\begin{align*}
\BV' 
&=
(\BV \cdot \Bn)\Bn
+(\BV \wedge \Bn)\Bn
( 1 + I \Bn \epsilon) \\
&=
(\BV \cdot \Bn)\Bn 
+(\BV \wedge \Bn)\Bn
+\epsilon I^2(\BV \cross \Bn)\Bn^2 \\
&=
\BV
+ \epsilon (\Bn \cross \BV) \qquad\square
\end{align*}

\EndArticle
