%
% Copyright � 2012 Peeter Joot.  All Rights Reserved.
% Licenced as described in the file LICENSE under the root directory of this GIT repository.
%
\newcommand{\authorname}{Peeter Joot}
\newcommand{\email}{peeterjoot@protonmail.com}
\newcommand{\basename}{FIXMEbasenameUndefined}
\newcommand{\dirname}{notes/FIXMEdirnameUndefined/}

\renewcommand{\basename}{pipeFlowConstPressureGradient}
\renewcommand{\dirname}{notes/phy454/}
%\newcommand{\dateintitle}{}
%\newcommand{\keywords}{}

\newcommand{\authorname}{Peeter Joot}
\newcommand{\onlineurl}{http://sites.google.com/site/peeterjoot2/math2013/\basename.pdf}
\newcommand{\sourcepath}{\dirname\basename.tex}
\newcommand{\generatetitle}[1]{\chapter{#1}}

\newcommand{\vcsinfo}{%
\section*{}
\noindent{\color{DarkOliveGreen}{\rule{\linewidth}{0.1mm}}}
\paragraph{Document version}
%\paragraph{\color{Maroon}{Document version}}
{
\small
\begin{itemize}
\item Available online at:\\ 
\href{\onlineurl}{\onlineurl}
\item Git Repository: \input{./.revinfo/gitRepo.tex}
\item Source: \sourcepath
\item last commit: \input{./.revinfo/gitCommitString.tex}
\item commit date: \input{./.revinfo/gitCommitDate.tex}
\end{itemize}
}
}

%\PassOptionsToPackage{dvipsnames,svgnames}{xcolor}
\PassOptionsToPackage{square,numbers}{natbib}
\documentclass{scrreprt}

\usepackage[left=2cm,right=2cm]{geometry}
\usepackage[svgnames]{xcolor}
\usepackage{peeters_layout}

\usepackage{natbib}

\usepackage[
colorlinks=true,
bookmarks=false,
pdfauthor={\authorname, \email},
backref 
]{hyperref}

% http://tex.stackexchange.com/questions/75773/how-to-reference-problems-by-the-text-label-in-an-exercise-envioronment
\usepackage[english]{cleveref}
\crefname{Exercise}{exercise}{exercises}
\Crefname{Exercise}{Exercise}{Exercises}

\RequirePackage{titlesec}
\RequirePackage{ifthen}

% http://stackoverflow.com/questions/4932910/date-in-the-tabular-environment
\makeatletter
\let\insertdate\@date
\makeatother

\titleformat{\chapter}[display]
{\bfseries\Large}
{\color{DarkSlateGrey}\filleft \authorname
\ifthenelse{\isundefined{\studentnumber}}{}{\\ \studentnumber}
\ifthenelse{\isundefined{\email}}{}{\\ \email}
\ifthenelse{\isundefined{\dateintitle}}{}{\\ \insertdate}
%\ifthenelse{\isundefined{\coursename}}{}{\\ \coursename} % put in title instead.
}
{4ex}
{\color{DarkOliveGreen}{\titlerule}\color{Maroon}
\vspace{2ex}%
\filright}
[\vspace{2ex}%
\color{DarkOliveGreen}\titlerule
]

\newcommand{\beginArtWithToc}[0]{\begin{document}\tableofcontents}
\newcommand{\beginArtNoToc}[0]{\begin{document}}
\newcommand{\EndNoBibArticle}[0]{\end{document}}
\newcommand{\EndArticle}[0]{\bibliography{Bibliography}\bibliographystyle{plainnat}\end{document}}

% 
%\newcommand{\citep}[1]{\cite{#1}}

\colorSectionsForArticle



\beginArtNoToc

\generatetitle{Pipe flow with constant pressure gradient}
%\chapter{Pipe flow with constant pressure gradient}
\label{chap:pipeFlowConstPressureGradient}
%\date{Feb 28, 2012}

\section{Motivation}

In \citep{acheson1990elementary}, problem 2.3 (ii), we are asked to show that for a pipe with circular cross section \(r = a\) and constant pressure gradient \(P = -dp/dz\) one has

\begin{equation}\label{eqn:pipeFlowConstPressureGradient:170}
\begin{aligned}%\label{eqn:pipeFlowConstPressureGradient:10}
u_z &= \frac{P}{4 \mu}\left( a^2 - r^2 \right) \\
u_r &= 0 \\
u_\theta &= 0.
\end{aligned}
\end{equation}

It seems reasonable to assume that the radial and rotational velocity components \(u_r = 0\), and \(u_\theta = 0\) by virtue of symmetry.  Is there also a physical principle that also leads to that conclusion?

If one does assume that \(u_r = u_\theta = 0\), then the Navier-stokes equations for an incompressible steady state flow takes the form

\begin{equation*}%\label{eqn:pipeFlowConstPressureGradient:110}
u_z \PD{z}{u_z} = \frac{P}{\rho}
+ \nu \left(
\inv{r}
\PD{r}{}\left(r \PD{r}{u_z} \right) + \PDSq{\theta}{u_z} + \PDSq{z}{u_z}
\right)
\end{equation*}
\begin{equation*}%\label{eqn:pipeFlowConstPressureGradient:130}
0 = \PD{r}{p}
\end{equation*}
\begin{equation*}%\label{eqn:pipeFlowConstPressureGradient:150}
0 = \PD{\theta}{p}
\end{equation*}

\begin{equation*}
w \PD{z}{w} = \frac{P}{\rho}
+ \nu \left(
\inv{r}
\PD{r}{}\left(r \PD{r}{u_z} \right) + \PDSq{z}{u_z}
\right).
\end{equation*}

Attempting separation of variables with \(w = Z(z) R(r)\), this appears to be inseparable due to the non-linearity of the \((\Bu \cdot \spacegrad) \Bu\) term on the LHS.  The fact that this is inseparable does not seem like it is a good justificaion for requiring that \(w = w(r)\) only and not \(w = w(r,z)\).  I could imagine that it is still possible to find solutions of the form \(w = w(r,z)\).  Is there some other argument that can be made to show that there is no \(z\) dependence in the remaining component of the fluid's velocity?

\subsection{stress tensor in cylindrical coordinates?}

In class, for rectangular systems, we considered matching of the traction vectors \(\sigma_{ij} n_j = T_i\) at the interface, and thought that might be relevant.

It is not clear to me how one would even express the traction vector in cylindrical coordinates.  For the stress tensor we have (in rectangular coordinates)

\begin{equation}\label{eqn:pipeFlowConstPressureGradient:30}
\sigma_{ij} = -p \delta_{ij} + 2 \mu u_{ij}
\end{equation}

In rectangular coordinates the strain portion of that tensor is

\begin{equation}\label{eqn:pipeFlowConstPressureGradient:50}
u_{ij} = \inv{2} \left(
\PD{x_j}{u_i}
+
\PD{x_i}{u_j}
\right).
\end{equation}

I did not find it too hard to compute the strain tensor in cylindrical coordinates, and it can also be looked up (1.8 in \citep{landau1960theory} for example).  But how would the \(\delta_{ij}\)

\subsection{fluid stress tensor in cylindrical coordinates?}

%http://www.physicsforums.com/showthread.php?t=582395

For fluid with viscosity \(\mu\) our stress strain relationship takes the form

\begin{equation}\label{eqn:pipeFlowConstPressureGradient:70}
\sigma_{ij} = -p \delta_{ij} + 2 \mu u_{ij}.
\end{equation}

I was wondering how to express this in cylindrical coordinates.  The strain tensor I can calculate in cylindrical coordinates.  What I get matches what I find in (1.8 in \citep{landau1960theory}).  But how would the \(\delta_{ij}\) portion of the stress strain relationship be expressed in cylindrical coordinates?

For example, if we considered a non-viscous fluid, we have in rectangular coordinates

\begin{equation}\label{eqn:pipeFlowConstPressureGradient:90}
\sigma_{ij} = -p \delta_{ij}.
\end{equation}

It is not obvious to me how this would be expressed in cylindrical form (my end goal was thinking about the traction vector \(T_i = \sigma_{ij} n_j\) in this coordinate system, but I am getting stuck even expressing the stress.

\EndArticle
