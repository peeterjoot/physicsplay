%
% Copyright � 2015 Peeter Joot.  All Rights Reserved.
% Licenced as described in the file LICENSE under the root directory of this GIT repository.
%
\documentclass[]{eliblog}

\usepackage{amsmath}
\usepackage{mathpazo}

%
% shorthand for bold symbols, convenient for vectors and matrices
%
\newcommand{\Ba}[0]{\mathbf{a}}
\newcommand{\Bb}[0]{\mathbf{b}}
\newcommand{\Bc}[0]{\mathbf{c}}
\newcommand{\Bd}[0]{\mathbf{d}}
\newcommand{\Be}[0]{\mathbf{e}}
\newcommand{\Bf}[0]{\mathbf{f}}
\newcommand{\Bg}[0]{\mathbf{g}}
\newcommand{\Bh}[0]{\mathbf{h}}
\newcommand{\Bi}[0]{\mathbf{i}}
\newcommand{\Bj}[0]{\mathbf{j}}
\newcommand{\Bk}[0]{\mathbf{k}}
\newcommand{\Bl}[0]{\mathbf{l}}
\newcommand{\Bm}[0]{\mathbf{m}}
\newcommand{\Bn}[0]{\mathbf{n}}
\newcommand{\Bo}[0]{\mathbf{o}}
\newcommand{\Bp}[0]{\mathbf{p}}
\newcommand{\Bq}[0]{\mathbf{q}}
\newcommand{\Br}[0]{\mathbf{r}}
\newcommand{\Bs}[0]{\mathbf{s}}
\newcommand{\Bt}[0]{\mathbf{t}}
\newcommand{\Bu}[0]{\mathbf{u}}
\newcommand{\Bv}[0]{\mathbf{v}}
\newcommand{\Bw}[0]{\mathbf{w}}
\newcommand{\Bx}[0]{\mathbf{x}}
\newcommand{\By}[0]{\mathbf{y}}
\newcommand{\Bz}[0]{\mathbf{z}}
\newcommand{\BA}[0]{\mathbf{A}}
\newcommand{\BB}[0]{\mathbf{B}}
\newcommand{\BC}[0]{\mathbf{C}}
\newcommand{\BD}[0]{\mathbf{D}}
\newcommand{\BE}[0]{\mathbf{E}}
\newcommand{\BF}[0]{\mathbf{F}}
\newcommand{\BG}[0]{\mathbf{G}}
\newcommand{\BH}[0]{\mathbf{H}}
\newcommand{\BI}[0]{\mathbf{I}}
\newcommand{\BJ}[0]{\mathbf{J}}
\newcommand{\BK}[0]{\mathbf{K}}
\newcommand{\BL}[0]{\mathbf{L}}
\newcommand{\BM}[0]{\mathbf{M}}
\newcommand{\BN}[0]{\mathbf{N}}
\newcommand{\BO}[0]{\mathbf{O}}
\newcommand{\BP}[0]{\mathbf{P}}
\newcommand{\BQ}[0]{\mathbf{Q}}
\newcommand{\BR}[0]{\mathbf{R}}
\newcommand{\BS}[0]{\mathbf{S}}
\newcommand{\BT}[0]{\mathbf{T}}
\newcommand{\BU}[0]{\mathbf{U}}
\newcommand{\BV}[0]{\mathbf{V}}
\newcommand{\BW}[0]{\mathbf{W}}
\newcommand{\BX}[0]{\mathbf{X}}
\newcommand{\BY}[0]{\mathbf{Y}}
\newcommand{\BZ}[0]{\mathbf{Z}}

\newcommand{\Bzero}[0]{\mathbf{0}}
\newcommand{\Btheta}[0]{\boldsymbol{\theta}}
\newcommand{\Btau}[0]{\boldsymbol{\tau}}
\newcommand{\Bomega}[0]{\boldsymbol{\omega}}

%
% shorthand for unit vectors
%
\newcommand{\acap}[0]{\hat{\Ba}}
\newcommand{\bcap}[0]{\hat{\Bb}}
\newcommand{\ccap}[0]{\hat{\Bc}}
\newcommand{\dcap}[0]{\hat{\Bd}}
\newcommand{\ecap}[0]{\hat{\Be}}
\newcommand{\fcap}[0]{\hat{\Bf}}
\newcommand{\gcap}[0]{\hat{\Bg}}
\newcommand{\hcap}[0]{\hat{\Bh}}
\newcommand{\icap}[0]{\hat{\Bi}}
\newcommand{\jcap}[0]{\hat{\Bj}}
\newcommand{\kcap}[0]{\hat{\Bk}}
\newcommand{\lcap}[0]{\hat{\Bl}}
\newcommand{\mcap}[0]{\hat{\Bm}}
\newcommand{\ncap}[0]{\hat{\Bn}}
\newcommand{\ocap}[0]{\hat{\Bo}}
\newcommand{\pcap}[0]{\hat{\Bp}}
\newcommand{\qcap}[0]{\hat{\Bq}}
\newcommand{\rcap}[0]{\hat{\Br}}
\newcommand{\scap}[0]{\hat{\Bs}}
\newcommand{\tcap}[0]{\hat{\Bt}}
\newcommand{\ucap}[0]{\hat{\Bu}}
\newcommand{\vcap}[0]{\hat{\Bv}}
\newcommand{\wcap}[0]{\hat{\Bw}}
\newcommand{\xcap}[0]{\hat{\Bx}}
\newcommand{\ycap}[0]{\hat{\By}}
\newcommand{\zcap}[0]{\hat{\Bz}}
\newcommand{\thetacap}[0]{\hat{\Btheta}}

%
% to write R^n and C^n in a distinguishable fashion.  Perhaps change this
% to the double lined characters upon figuring out how to do so.
%
\newcommand{\C}[1]{$\mathbb{C}^{#1}$}
\newcommand{\R}[1]{$\mathbb{R}^{#1}$}

%
% various generally useful helpers
%

% derivative of #1 wrt. #2:
\newcommand{\D}[2] {\frac {d#2} {d#1}}

\newcommand{\inv}[1]{\frac{1}{#1}}
\newcommand{\cross}[0]{\times}

\newcommand{\abs}[1]{\lvert{#1}\rvert}
\newcommand{\norm}[1]{\lVert{#1}\rVert}
\newcommand{\innerprod}[2]{\langle{#1}, {#2}\rangle}
\newcommand{\dotprod}[2]{{#1} \cdot {#2}}
\newcommand{\bdotprod}[2]{\left({#1} \cdot {#2}\right)}
\newcommand{\crossprod}[2]{{#1} \cross {#2}}
\newcommand{\tripleprod}[3]{\dotprod{\left(\crossprod{#1}{#2}\right)}{#3}}

\DeclareMathOperator{\Proj}{Proj}
\DeclareMathOperator{\Span}{span}
\DeclareMathOperator{\Sgn}{sgn}
\DeclareMathOperator{\Area}{Area}
\DeclareMathOperator{\Volume}{Volume}

%
% A few miscellaneous things specific to this document
%
\newcommand{\crossop}[1]{\crossprod{#1}{}}

% R2 vector.
\newcommand{\VectorTwo}[2]{
\begin{bmatrix}
 {#1} \\
 {#2}
\end{bmatrix}
}

\newcommand{\VectorN}[1]{
\begin{bmatrix}
{#1}_1 \\
{#1}_2 \\
\vdots \\
{#1}_N \\
\end{bmatrix}
}

\newcommand{\DETuvij}[4]{
\begin{vmatrix}
 {#1}_{#3} & {#1}_{#4} \\
 {#2}_{#3} & {#2}_{#4}
\end{vmatrix}
}

\newcommand{\DETuvwijk}[6]{
\begin{vmatrix}
 {#1}_{#4} & {#1}_{#5} & {#1}_{#6} \\
 {#2}_{#4} & {#2}_{#5} & {#2}_{#6} \\
 {#3}_{#4} & {#3}_{#5} & {#3}_{#6}
\end{vmatrix}
}

\newcommand{\DETuvwxijkl}[8]{
\begin{vmatrix}
 {#1}_{#5} & {#1}_{#6} & {#1}_{#7} & {#1}_{#8} \\
 {#2}_{#5} & {#2}_{#6} & {#2}_{#7} & {#2}_{#8} \\
 {#3}_{#5} & {#3}_{#6} & {#3}_{#7} & {#3}_{#8} \\
 {#4}_{#5} & {#4}_{#6} & {#4}_{#7} & {#4}_{#8} \\
\end{vmatrix}
}

%\newcommand{\DETuvwxyijklm}[10]{
%\begin{vmatrix}
% {#1}_{#6} & {#1}_{#7} & {#1}_{#8} & {#1}_{#9} & {#1}_{#10} \\
% {#2}_{#6} & {#2}_{#7} & {#2}_{#8} & {#2}_{#9} & {#2}_{#10} \\
% {#3}_{#6} & {#3}_{#7} & {#3}_{#8} & {#3}_{#9} & {#3}_{#10} \\
% {#4}_{#6} & {#4}_{#7} & {#4}_{#8} & {#4}_{#9} & {#4}_{#10} \\
% {#5}_{#6} & {#5}_{#7} & {#5}_{#8} & {#5}_{#9} & {#5}_{#10}
%\end{vmatrix}
%}

% R3 vector.
\newcommand{\VectorThree}[3]{
\begin{bmatrix}
 {#1} \\
 {#2} \\
 {#3}
\end{bmatrix}
}



\author{Peeter Joot}
\email{peeter.joot@gmail.com}

%\documentclass[]{eliblogwidescreen}

\usepackage{amsmath}
\usepackage{mathpazo}

%
% shorthand for bold symbols, convenient for vectors and matrices
%
\newcommand{\Ba}[0]{\mathbf{a}}
\newcommand{\Bb}[0]{\mathbf{b}}
\newcommand{\Bc}[0]{\mathbf{c}}
\newcommand{\Bd}[0]{\mathbf{d}}
\newcommand{\Be}[0]{\mathbf{e}}
\newcommand{\Bf}[0]{\mathbf{f}}
\newcommand{\Bg}[0]{\mathbf{g}}
\newcommand{\Bh}[0]{\mathbf{h}}
\newcommand{\Bi}[0]{\mathbf{i}}
\newcommand{\Bj}[0]{\mathbf{j}}
\newcommand{\Bk}[0]{\mathbf{k}}
\newcommand{\Bl}[0]{\mathbf{l}}
\newcommand{\Bm}[0]{\mathbf{m}}
\newcommand{\Bn}[0]{\mathbf{n}}
\newcommand{\Bo}[0]{\mathbf{o}}
\newcommand{\Bp}[0]{\mathbf{p}}
\newcommand{\Bq}[0]{\mathbf{q}}
\newcommand{\Br}[0]{\mathbf{r}}
\newcommand{\Bs}[0]{\mathbf{s}}
\newcommand{\Bt}[0]{\mathbf{t}}
\newcommand{\Bu}[0]{\mathbf{u}}
\newcommand{\Bv}[0]{\mathbf{v}}
\newcommand{\Bw}[0]{\mathbf{w}}
\newcommand{\Bx}[0]{\mathbf{x}}
\newcommand{\By}[0]{\mathbf{y}}
\newcommand{\Bz}[0]{\mathbf{z}}
\newcommand{\BA}[0]{\mathbf{A}}
\newcommand{\BB}[0]{\mathbf{B}}
\newcommand{\BC}[0]{\mathbf{C}}
\newcommand{\BD}[0]{\mathbf{D}}
\newcommand{\BE}[0]{\mathbf{E}}
\newcommand{\BF}[0]{\mathbf{F}}
\newcommand{\BG}[0]{\mathbf{G}}
\newcommand{\BH}[0]{\mathbf{H}}
\newcommand{\BI}[0]{\mathbf{I}}
\newcommand{\BJ}[0]{\mathbf{J}}
\newcommand{\BK}[0]{\mathbf{K}}
\newcommand{\BL}[0]{\mathbf{L}}
\newcommand{\BM}[0]{\mathbf{M}}
\newcommand{\BN}[0]{\mathbf{N}}
\newcommand{\BO}[0]{\mathbf{O}}
\newcommand{\BP}[0]{\mathbf{P}}
\newcommand{\BQ}[0]{\mathbf{Q}}
\newcommand{\BR}[0]{\mathbf{R}}
\newcommand{\BS}[0]{\mathbf{S}}
\newcommand{\BT}[0]{\mathbf{T}}
\newcommand{\BU}[0]{\mathbf{U}}
\newcommand{\BV}[0]{\mathbf{V}}
\newcommand{\BW}[0]{\mathbf{W}}
\newcommand{\BX}[0]{\mathbf{X}}
\newcommand{\BY}[0]{\mathbf{Y}}
\newcommand{\BZ}[0]{\mathbf{Z}}

\newcommand{\Bzero}[0]{\mathbf{0}}
\newcommand{\Btheta}[0]{\boldsymbol{\theta}}
\newcommand{\Btau}[0]{\boldsymbol{\tau}}
\newcommand{\Bomega}[0]{\boldsymbol{\omega}}

%
% shorthand for unit vectors
%
\newcommand{\acap}[0]{\hat{\Ba}}
\newcommand{\bcap}[0]{\hat{\Bb}}
\newcommand{\ccap}[0]{\hat{\Bc}}
\newcommand{\dcap}[0]{\hat{\Bd}}
\newcommand{\ecap}[0]{\hat{\Be}}
\newcommand{\fcap}[0]{\hat{\Bf}}
\newcommand{\gcap}[0]{\hat{\Bg}}
\newcommand{\hcap}[0]{\hat{\Bh}}
\newcommand{\icap}[0]{\hat{\Bi}}
\newcommand{\jcap}[0]{\hat{\Bj}}
\newcommand{\kcap}[0]{\hat{\Bk}}
\newcommand{\lcap}[0]{\hat{\Bl}}
\newcommand{\mcap}[0]{\hat{\Bm}}
\newcommand{\ncap}[0]{\hat{\Bn}}
\newcommand{\ocap}[0]{\hat{\Bo}}
\newcommand{\pcap}[0]{\hat{\Bp}}
\newcommand{\qcap}[0]{\hat{\Bq}}
\newcommand{\rcap}[0]{\hat{\Br}}
\newcommand{\scap}[0]{\hat{\Bs}}
\newcommand{\tcap}[0]{\hat{\Bt}}
\newcommand{\ucap}[0]{\hat{\Bu}}
\newcommand{\vcap}[0]{\hat{\Bv}}
\newcommand{\wcap}[0]{\hat{\Bw}}
\newcommand{\xcap}[0]{\hat{\Bx}}
\newcommand{\ycap}[0]{\hat{\By}}
\newcommand{\zcap}[0]{\hat{\Bz}}
\newcommand{\thetacap}[0]{\hat{\Btheta}}

%
% to write R^n and C^n in a distinguishable fashion.  Perhaps change this
% to the double lined characters upon figuring out how to do so.
%
\newcommand{\C}[1]{$\mathbb{C}^{#1}$}
\newcommand{\R}[1]{$\mathbb{R}^{#1}$}

%
% various generally useful helpers
%

% derivative of #1 wrt. #2:
\newcommand{\D}[2] {\frac {d#2} {d#1}}

\newcommand{\inv}[1]{\frac{1}{#1}}
\newcommand{\cross}[0]{\times}

\newcommand{\abs}[1]{\lvert{#1}\rvert}
\newcommand{\norm}[1]{\lVert{#1}\rVert}
\newcommand{\innerprod}[2]{\langle{#1}, {#2}\rangle}
\newcommand{\dotprod}[2]{{#1} \cdot {#2}}
\newcommand{\bdotprod}[2]{\left({#1} \cdot {#2}\right)}
\newcommand{\crossprod}[2]{{#1} \cross {#2}}
\newcommand{\tripleprod}[3]{\dotprod{\left(\crossprod{#1}{#2}\right)}{#3}}

\DeclareMathOperator{\Proj}{Proj}
\DeclareMathOperator{\Span}{span}
\DeclareMathOperator{\Sgn}{sgn}
\DeclareMathOperator{\Area}{Area}
\DeclareMathOperator{\Volume}{Volume}

%
% A few miscellaneous things specific to this document
%
\newcommand{\crossop}[1]{\crossprod{#1}{}}

% R2 vector.
\newcommand{\VectorTwo}[2]{
\begin{bmatrix}
 {#1} \\
 {#2}
\end{bmatrix}
}

\newcommand{\VectorN}[1]{
\begin{bmatrix}
{#1}_1 \\
{#1}_2 \\
\vdots \\
{#1}_N \\
\end{bmatrix}
}

\newcommand{\DETuvij}[4]{
\begin{vmatrix}
 {#1}_{#3} & {#1}_{#4} \\
 {#2}_{#3} & {#2}_{#4}
\end{vmatrix}
}

\newcommand{\DETuvwijk}[6]{
\begin{vmatrix}
 {#1}_{#4} & {#1}_{#5} & {#1}_{#6} \\
 {#2}_{#4} & {#2}_{#5} & {#2}_{#6} \\
 {#3}_{#4} & {#3}_{#5} & {#3}_{#6}
\end{vmatrix}
}

\newcommand{\DETuvwxijkl}[8]{
\begin{vmatrix}
 {#1}_{#5} & {#1}_{#6} & {#1}_{#7} & {#1}_{#8} \\
 {#2}_{#5} & {#2}_{#6} & {#2}_{#7} & {#2}_{#8} \\
 {#3}_{#5} & {#3}_{#6} & {#3}_{#7} & {#3}_{#8} \\
 {#4}_{#5} & {#4}_{#6} & {#4}_{#7} & {#4}_{#8} \\
\end{vmatrix}
}

%\newcommand{\DETuvwxyijklm}[10]{
%\begin{vmatrix}
% {#1}_{#6} & {#1}_{#7} & {#1}_{#8} & {#1}_{#9} & {#1}_{#10} \\
% {#2}_{#6} & {#2}_{#7} & {#2}_{#8} & {#2}_{#9} & {#2}_{#10} \\
% {#3}_{#6} & {#3}_{#7} & {#3}_{#8} & {#3}_{#9} & {#3}_{#10} \\
% {#4}_{#6} & {#4}_{#7} & {#4}_{#8} & {#4}_{#9} & {#4}_{#10} \\
% {#5}_{#6} & {#5}_{#7} & {#5}_{#8} & {#5}_{#9} & {#5}_{#10}
%\end{vmatrix}
%}

% R3 vector.
\newcommand{\VectorThree}[3]{
\begin{bmatrix}
 {#1} \\
 {#2} \\
 {#3}
\end{bmatrix}
}



\author{Peeter Joot}
\email{peeter.joot@gmail.com}


\chapter{PHY450H1S.  Relativistic Electrodynamics Lecture 23 (Taught by Prof. Erich Poppitz).  Energy Momentum Tensor.}
\label{chap:relativisticElectrodynamicsL23}
%\useCCL
\blogpage{http://sites.google.com/site/peeterjoot/math2011/relativisticElectrodynamicsL23.pdf}
\date{Mar 29, 2011}
\revisionInfo{relativisticElectrodynamicsL23.tex}

%\beginArtWithToc
\beginArtNoToc

\section{Reading.}

FIXME:
Covering chapter X material from the text \cite{landau1980classical}.

Covering \href{http://www.physics.utoronto.ca/~poppitz/epoppitz/PHY450_files/RelEMpp166-180.pdf}{lecture notes pp. 173-178:} the force on a surface element of a body (177-178); a plane wave example (179-180).

%http://www.physics.utoronto.ca/~poppitz/epoppitz/PHY450_files/RelEMpp181-195.pdf
pp. 181-195: [read on your own: on diagonalizability of energy-momentum tensor (181)]; the Lagrangian for a system of nonrelativistic charged particles to zeroth order in $(v/c)$: electrostatic energy of a system of charges and .mass renormalization. (182-189) [Tuesday, Mar. 29]; the EM potentials to order $(v/c)^2$ (190-193); the ``Darwin Lagrangian.  and Hamiltonian for a system of nonrelativistic charged particles to order $(v/c)^2$ and its many uses in physics (194-195) [Wednesday, Mar. 30]

Next week (last topic): attempt to go to the next order $(v/c)^3$ - radiation damping, the limitations of classical electrodynamics, and the relevant time/length/energy scales.

\section{Recap.}

Last time we found that spacetime translation invariance led to the four conservation relations

\begin{equation}\label{eqn:relativisticElectrodynamicsL23:n}
\partial_k T^{k m} = 0
\end{equation}

where

\begin{equation}\label{eqn:relativisticElectrodynamicsL23:n}
T^{k m} = \inv{4 \pi} \left( F^{k j} F^{m l} g_{j l} + \inv{4} g^{k m} F_{i j} F^{i j} \right)
\end{equation}

last time we found for $m = 0$

\begin{equation}\label{eqn:relativisticElectrodynamicsL23:n}
\inv{c} \PD{t}{} T^{0 0} + \PD{x^\alpha}{} T^{\alpha 0} = 0
\end{equation}

Here

\begin{align}\label{eqn:relativisticElectrodynamicsL23:n}
T^{0 0} &= \inv{8 \pi} (\BE^2 + \BB^2) & \mbox{energy density} \\
c T^{\alpha 0} &= \BS^\alpha &= \mbox{energy flux}
\end{align}

\section{Spatial components of $T^{k m}$}

Now for $m = 1,2,3$ we write

\begin{equation}\label{eqn:relativisticElectrodynamicsL23:n}
\partial_k T^{k \alpha} = 0
\end{equation}

so we write

\begin{equation}\label{eqn:relativisticElectrodynamicsL23:n}
\PD{t}{} \BS^\alpha/c^2 + \partial_\beta T^{\beta \alpha} = 0
\end{equation}

Recall that we argued that

\begin{equation}\label{eqn:relativisticElectrodynamicsL23:n}
\frac{\BS}{c^2} = \text{momentum density}
\end{equation}

(it also comes from Noether's theorem).

\begin{equation}\label{eqn:relativisticElectrodynamicsL23:n}
\PD{t}{} \left( \frac{T^{0\alpha}}{c} \right) + \PD{x^\beta}{} T^{\beta \alpha} = 0
\end{equation}

or

\begin{equation}\label{eqn:relativisticElectrodynamicsL23:n}
\PD{t}{} \left( \frac{\BS^{\alpha}}{c^2} \right) + \PD{x^\beta}{} T^{\beta \alpha} = 0.
\end{equation}

Integrating over $V$ we have

\begin{align*}
\PD{t}{} \int_V d^3\Bx \left( \frac{\BS^{\alpha}}{c^2} \right) 
&= -\int_V d^3\Bx \PD{x^\beta}{} T^{\beta \alpha} \\
&= -\int_V d^3\Bx \spacegrad \cdot (\Be_\beta T^{\beta \alpha}) \\
&= -\int_{\partial V} d \sigma \Bn \cdot \Be_\beta) T^{\beta \alpha} \\
&\equiv -\int_{\partial V} d \sigma^\beta T^{\beta \alpha} \\
\end{align*}

We write this as

\begin{equation}\label{eqn:relativisticElectrodynamicsL23:n}
\PD{t}{} \left( \text{momentum of EM fields in V} \right)^\alpha = - \int_{\partial V} d\sigma^\beta T^{\beta \alpha}
\end{equation}

and describe our spatial tensor components as

\begin{equation}\label{eqn:relativisticElectrodynamicsL23:n}
T^{\beta\alpha} = \text{flux of $\alpha$-th momentum through a unit area $\perp \beta$},
\end{equation}

where

\begin{align*}
T^{\alpha\beta} 
&= \inv{4\pi} \left( -F^{\alpha j} F^{\beta m} g_{m j} + \inv{4} g^{\alpha\beta} F^{ij}F_{ij} \right) \\
&= \inv{4\pi} \left( 
-F^{\alpha 0} F^{\beta 0} 
+F^{\alpha \sigma} F^{\beta \sigma} 
- \inv{4} \delta^{\alpha\beta} 2 (\BB^2 - \BE^2) \right) \\
&= \inv{4\pi} \left( 
-E^\alpha E^\beta
+\sum_\sigma 
(\epsilon^{\mu\alpha\sigma} B^\mu)
(\epsilon^{\nu\beta\sigma} B^\nu)
- \inv{2} \delta^{\alpha\beta} (\BB^2 - \BE^2) \right) \\
&= \inv{4\pi} \left( 
-E^\alpha E^\beta
+\sum_{\mu,\nu} \underbrace{\delta^{\mu\alpha}_{[\nu\beta]}
B^\mu B^\nu
}_{
(\delta^{\alpha \beta} \delta^{\mu \nu} - \delta^{\alpha \nu} \delta^{\beta \mu} ) 
B^\mu B^\nu
=
\delta^{\alpha\beta}\BB^2 - B^\alpha B^\beta
} 
+ \inv{2} \delta^{\alpha\beta} (\BE^2 - \BB^2) \right) \\
&=
- \inv{4\pi} \left(
E^\alpha E^\beta
+B^\alpha B^\beta
+ \delta^{\alpha\beta} 
\left(
-\frac{\BE^2}{2} + \frac{\BB^2}{2} - \BB^2 \right)
\right) \\
&=
- \inv{4\pi} \left(
E^\alpha E^\beta
+B^\alpha B^\beta
- \frac{\delta^{\alpha\beta} }{2}
\left(
\BE^2 + \BB^2 
\right)
\right) \\
\end{align*}

We define

\begin{equation}\label{eqn:relativisticElectrodynamicsL23:n}
\sigma^{\alpha\beta}
=
-T^{\alpha\beta}
=
 \inv{4\pi} \left(
E^\alpha E^\beta
+B^\alpha B^\beta
- \frac{\delta^{\alpha\beta} }{2}
\left(
\BE^2 + \BB^2 
\right)
\right) 
\end{equation}

This is the Maxwell stress tensor.  Maxwell apparently derived this without any use of four vectors or symmetry arguments.  I'd be curious what his arguments were and how he related this to the Lorentz force?

In giagantic matrix form, where symmetric opposites are omitted, we have for $\Norm{T^{ij}}$

\begin{equation}\label{eqn:relativisticElectrodynamicsL23:n}
\begin{bmatrix}
\inv{8 \pi}(\BE^2 + \BB^2) & \inv{4\pi} (\BE \cross \BB)^x & \inv{4\pi} (\BE \cross \BB)^y & \inv{4\pi} (\BE \cross \BB)^z \\
. 
& -\inv{4 \pi}\left(E_x^2 + B_x^2 - \inv{2}(\BE^2 + \BB^2)\right) 
& -\inv{4 \pi}\left(E_x E_y + B_x B_y \right)
& -\inv{4 \pi}\left(E_x E_z + B_x B_z \right) \\
. & . & -\inv{4 \pi}\left(E_y^2 + B_y^2 - \inv{2}(\BE^2 + \BB^2)\right) & -\inv{4 \pi}\left(E_y E_z + B_y B_z \right) \\
. & . & . & -\inv{4 \pi}\left(E_z^2 + B_z^2 - \inv{2}(\BE^2 + \BB^2)\right) 
\end{bmatrix}
\end{equation}

In words this matrix is
\begin{equation}\label{eqn:relativisticElectrodynamicsL23:n}
\begin{bmatrix}
\text{energy density} & \inv{c} (\text{energy flux in $\xcap$}) & \inv{c} (\text{energy flux in $\ycap$}) & \inv{c} (\text{energy flux in $\zcap$}) \\
c \times (\text{momentum density})^x
& (\text{momentum})^x \text{flux in $\xcap$}
& (\text{momentum})^x \text{flux in $\ycap$}
& (\text{momentum})^x \text{flux in $\zcap$} \\
c \times (\text{momentum density})^y
& (\text{momentum})^y \text{flux in $\xcap$}
& (\text{momentum})^y \text{flux in $\ycap$}
& (\text{momentum})^y \text{flux in $\zcap$} \\
c \times (\text{momentum density})^z
& (\text{momentum})^z \text{flux in $\xcap$}
& (\text{momentum})^z \text{flux in $\ycap$}
& (\text{momentum})^z \text{flux in $\zcap$} \\
\end{bmatrix}
\end{equation}

\EndArticle
