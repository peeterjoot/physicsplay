%
% Copyright � 2015 Peeter Joot.  All Rights Reserved.
% Licenced as described in the file LICENSE under the root directory of this GIT repository.
%
\newcommand{\authorname}{Peeter Joot}
\newcommand{\email}{peeterjoot@protonmail.com}
\newcommand{\basename}{FIXMEbasenameUndefined}
\newcommand{\dirname}{notes/FIXMEdirnameUndefined/}

\renewcommand{\basename}{angularMomentumAndCentralForceCommutators}
\renewcommand{\dirname}{notes/phy1520/}
%\newcommand{\dateintitle}{}
%\newcommand{\keywords}{}

\newcommand{\authorname}{Peeter Joot}
\newcommand{\onlineurl}{http://sites.google.com/site/peeterjoot2/math2013/\basename.pdf}
\newcommand{\sourcepath}{\dirname\basename.tex}
\newcommand{\generatetitle}[1]{\chapter{#1}}

\newcommand{\vcsinfo}{%
\section*{}
\noindent{\color{DarkOliveGreen}{\rule{\linewidth}{0.1mm}}}
\paragraph{Document version}
%\paragraph{\color{Maroon}{Document version}}
{
\small
\begin{itemize}
\item Available online at:\\ 
\href{\onlineurl}{\onlineurl}
\item Git Repository: \input{./.revinfo/gitRepo.tex}
\item Source: \sourcepath
\item last commit: \input{./.revinfo/gitCommitString.tex}
\item commit date: \input{./.revinfo/gitCommitDate.tex}
\end{itemize}
}
}

%\PassOptionsToPackage{dvipsnames,svgnames}{xcolor}
\PassOptionsToPackage{square,numbers}{natbib}
\documentclass{scrreprt}

\usepackage[left=2cm,right=2cm]{geometry}
\usepackage[svgnames]{xcolor}
\usepackage{peeters_layout}

\usepackage{natbib}

\usepackage[
colorlinks=true,
bookmarks=false,
pdfauthor={\authorname, \email},
backref 
]{hyperref}

% http://tex.stackexchange.com/questions/75773/how-to-reference-problems-by-the-text-label-in-an-exercise-envioronment
\usepackage[english]{cleveref}
\crefname{Exercise}{exercise}{exercises}
\Crefname{Exercise}{Exercise}{Exercises}

\RequirePackage{titlesec}
\RequirePackage{ifthen}

% http://stackoverflow.com/questions/4932910/date-in-the-tabular-environment
\makeatletter
\let\insertdate\@date
\makeatother

\titleformat{\chapter}[display]
{\bfseries\Large}
{\color{DarkSlateGrey}\filleft \authorname
\ifthenelse{\isundefined{\studentnumber}}{}{\\ \studentnumber}
\ifthenelse{\isundefined{\email}}{}{\\ \email}
\ifthenelse{\isundefined{\dateintitle}}{}{\\ \insertdate}
%\ifthenelse{\isundefined{\coursename}}{}{\\ \coursename} % put in title instead.
}
{4ex}
{\color{DarkOliveGreen}{\titlerule}\color{Maroon}
\vspace{2ex}%
\filright}
[\vspace{2ex}%
\color{DarkOliveGreen}\titlerule
]

\newcommand{\beginArtWithToc}[0]{\begin{document}\tableofcontents}
\newcommand{\beginArtNoToc}[0]{\begin{document}}
\newcommand{\EndNoBibArticle}[0]{\end{document}}
\newcommand{\EndArticle}[0]{\bibliography{Bibliography}\bibliographystyle{plainnat}\end{document}}

% 
%\newcommand{\citep}[1]{\cite{#1}}

\colorSectionsForArticle



\usepackage{peeters_layout_exercise}
\usepackage{peeters_braket}
\usepackage{peeters_figures}

\beginArtNoToc

\generatetitle{Commutators of angular momentum and a central force Hamiltonian}
%\chapter{Commutators of angular momentum and a central force Hamiltonian}
%\label{chap:angularMomentumAndCentralForceCommutators}

\paragraph{Commutators for angular momentum}

In problem 1.17 of \citep{sakurai2014modern} we are to show that non-commuting operators that both commute with the Hamiltonian, have, in general, degenerate energy eigenvalues.  It suggests considering \( L_x, L_z \) and a central force Hamiltonian \( H = \Bp^2/2m + V(r) \) as examples.

Let's just demonstrate these commutators act as expected in these cases.

With \( \BL = \Bx \cross \Bp \), we have

\begin{equation}\label{eqn:angularMomentumAndCentralForceCommutators:20}
\begin{aligned}
L_x &= y p_z - z p_y \\
L_y &= z p_x - x p_z \\
L_z &= x p_y - y p_x.
\end{aligned}
\end{equation}

The \( L_x, L_z \) commutator is

\begin{equation}\label{eqn:angularMomentumAndCentralForceCommutators:40}
\begin{aligned}
\antisymmetric{L_x}{L_z}
&=
\antisymmetric{y p_z - z p_y }{x p_y - y p_x} \\
&=
\antisymmetric{y p_z}{x p_y}
-\antisymmetric{y p_z}{y p_x}
-\antisymmetric{z p_y }{x p_y}
+\antisymmetric{z p_y }{y p_x} \\
&=
x p_z \antisymmetric{y}{p_y}
+ z p_x \antisymmetric{p_y }{y} \\
&=
i \Hbar \lr{ x p_z - z p_x } \\
&=
- i \Hbar L_y
\end{aligned}
\end{equation}

cyclicly permuting the indexes shows that no pairs of different \( \BL \) components commute.  For \( L_y, L_x \) that is

\begin{equation}\label{eqn:angularMomentumAndCentralForceCommutators:60}
\begin{aligned}
\antisymmetric{L_y}{L_x}
&=
\antisymmetric{z p_x - x p_z }{y p_z - z p_y} \\
&=
\antisymmetric{z p_x}{y p_z}
-\antisymmetric{z p_x}{z p_y}
-\antisymmetric{x p_z }{y p_z}
+\antisymmetric{x p_z }{z p_y} \\
&=
y p_x \antisymmetric{z}{p_z}
+ x p_y \antisymmetric{p_z }{z} \\
&=
i \Hbar \lr{ y p_x - x p_y } \\
&=
- i \Hbar L_z,
\end{aligned}
\end{equation}

and for \( L_z, L_y \)

\begin{equation}\label{eqn:angularMomentumAndCentralForceCommutators:80}
\begin{aligned}
\antisymmetric{L_z}{L_y}
&=
\antisymmetric{x p_y - y p_x }{z p_x - x p_z} \\
&=
\antisymmetric{x p_y}{z p_x}
-\antisymmetric{x p_y}{x p_z}
-\antisymmetric{y p_x }{z p_x}
+\antisymmetric{y p_x }{x p_z} \\
&=
z p_y \antisymmetric{x}{p_x}
+ y p_z \antisymmetric{p_x }{x} \\
&=
i \Hbar \lr{ z p_y - y p_z } \\
&=
- i \Hbar L_x.
\end{aligned}
\end{equation}

If these angular momentum components are also shown to commute with themselves (which they do), the commutator relations above can be summarized as

\begin{equation}\label{eqn:angularMomentumAndCentralForceCommutators:100}
\antisymmetric{L_a}{L_b} = i \Hbar \epsilon_{a b c} L_c.
\end{equation}

In the example to consider, we'll have to consider the commutators with \( \Bp^2 \) and \( V(r) \).  Picking any one component of \( \BL \) is sufficent due to the symmetries of the problem.  For example

\begin{equation}\label{eqn:angularMomentumAndCentralForceCommutators:120}
\begin{aligned}
\antisymmetric{L_x}{\Bp^2}
&=
\antisymmetric{y p_z - z p_y}{p_x^2 + p_y^2 + p_z^2} \\
&=
\antisymmetric{y p_z}{\cancel{p_x^2} + p_y^2 + \cancel{p_z^2}}
-\antisymmetric{z p_y}{\cancel{p_x^2} + \cancel{p_y^2} + p_z^2} \\
&=
p_z \antisymmetric{y}{p_y^2}
-p_y \antisymmetric{z}{p_z^2} \\
&=
p_z 2 i \Hbar p_y 
2 i \Hbar p_y 
-p_y 2 i \Hbar p_z  \\
&= 
0.
\end{aligned}
\end{equation}

How about the commutator of \( \BL \) with the potential?  It is sufficient to consider one component again, for example

\begin{equation}\label{eqn:angularMomentumAndCentralForceCommutators:140}
\begin{aligned}
\antisymmetric{L_x}{V}
&=
\antisymmetric{y p_z - z p_y}{V} \\
&=
y \antisymmetric{p_z}{V} - z \antisymmetric{p_y}{V} \\
&=
-i \Hbar y \PD{z}{V(r)} + i \Hbar z \PD{y}{V(r)} \\
&=
-i \Hbar y \PD{r}{V}\PD{z}{r} + i \Hbar z \PD{r}{V}\PD{y}{r}  \\
&=
-i \Hbar y \PD{r}{V} \frac{z}{r} + i \Hbar z \PD{r}{V}\frac{y}{r}  \\
&=
0.
\end{aligned}
\end{equation}

We've shown that all the components of \( \BL \) commute with a central force Hamiltonian, and each different component of \( \BL \) do not commute.

%The next step will be figuring out how to use this to show that there are energy degeneracies.

\paragraph{Matrix example of non-commuting commutators}

\EndArticle
