%
% Copyright � 2013 Peeter Joot.  All Rights Reserved.
% Licenced as described in the file LICENSE under the root directory of this GIT repository.
%
\newcommand{\authorname}{Peeter Joot}
\newcommand{\email}{peeterjoot@protonmail.com}
\newcommand{\basename}{FIXMEbasenameUndefined}
\newcommand{\dirname}{notes/FIXMEdirnameUndefined/}

\renewcommand{\basename}{basicStatMechLecture15}
\renewcommand{\dirname}{notes/phy452/}
\newcommand{\keywords}{Statistical mechanics, PHY452H1S}
\newcommand{\authorname}{Peeter Joot}
\newcommand{\onlineurl}{http://sites.google.com/site/peeterjoot2/math2013/\basename.pdf}
\newcommand{\sourcepath}{\dirname\basename.tex}
\newcommand{\generatetitle}[1]{\chapter{#1}}

\newcommand{\vcsinfo}{%
\section*{}
\noindent{\color{DarkOliveGreen}{\rule{\linewidth}{0.1mm}}}
\paragraph{Document version}
%\paragraph{\color{Maroon}{Document version}}
{
\small
\begin{itemize}
\item Available online at:\\ 
\href{\onlineurl}{\onlineurl}
\item Git Repository: \input{./.revinfo/gitRepo.tex}
\item Source: \sourcepath
\item last commit: \input{./.revinfo/gitCommitString.tex}
\item commit date: \input{./.revinfo/gitCommitDate.tex}
\end{itemize}
}
}

%\PassOptionsToPackage{dvipsnames,svgnames}{xcolor}
\PassOptionsToPackage{square,numbers}{natbib}
\documentclass{scrreprt}

\usepackage[left=2cm,right=2cm]{geometry}
\usepackage[svgnames]{xcolor}
\usepackage{peeters_layout}

\usepackage{natbib}

\usepackage[
colorlinks=true,
bookmarks=false,
pdfauthor={\authorname, \email},
backref 
]{hyperref}

% http://tex.stackexchange.com/questions/75773/how-to-reference-problems-by-the-text-label-in-an-exercise-envioronment
\usepackage[english]{cleveref}
\crefname{Exercise}{exercise}{exercises}
\Crefname{Exercise}{Exercise}{Exercises}

\RequirePackage{titlesec}
\RequirePackage{ifthen}

% http://stackoverflow.com/questions/4932910/date-in-the-tabular-environment
\makeatletter
\let\insertdate\@date
\makeatother

\titleformat{\chapter}[display]
{\bfseries\Large}
{\color{DarkSlateGrey}\filleft \authorname
\ifthenelse{\isundefined{\studentnumber}}{}{\\ \studentnumber}
\ifthenelse{\isundefined{\email}}{}{\\ \email}
\ifthenelse{\isundefined{\dateintitle}}{}{\\ \insertdate}
%\ifthenelse{\isundefined{\coursename}}{}{\\ \coursename} % put in title instead.
}
{4ex}
{\color{DarkOliveGreen}{\titlerule}\color{Maroon}
\vspace{2ex}%
\filright}
[\vspace{2ex}%
\color{DarkOliveGreen}\titlerule
]

\newcommand{\beginArtWithToc}[0]{\begin{document}\tableofcontents}
\newcommand{\beginArtNoToc}[0]{\begin{document}}
\newcommand{\EndNoBibArticle}[0]{\end{document}}
\newcommand{\EndArticle}[0]{\bibliography{Bibliography}\bibliographystyle{plainnat}\end{document}}

% 
%\newcommand{\citep}[1]{\cite{#1}}

\colorSectionsForArticle



\beginArtNoToc
\generatetitle{PHY452H1S Basic Statistical Mechanics.  Lecture 15: Grand Canonical/Fermion-Bosons.  Taught by Prof.\ Arun Paramekanti}
%\chapter{Grand Canonical/Fermion-Bosons}
\label{chap:basicStatMechLecture15}

\section{Disclaimer}

Peeter's lecture notes from class.  May not be entirely coherent.

\section{Grand Canonical/Fermion-Bosons}

Was mentioned that three dimensions confines us to looking at either Fermions or Bosons, and that two dimensions is a rich subject (interchange of two particles isn't the same as one particle cycling around the other ending up in the same place -- how is that different than a particle cyling around another in a two dimensional space?)

Definitions

\begin{itemize}
\item Fermion \index{Fermion}.  Antisymmetric under exchange.  $n_k = 0, 1$
\item Boson \index{Boson}.  Symmetric under exchange.  $n_k = 0, 1, 2, \cdots$
\end{itemize}

In either case we have, as in 

F1

\begin{dmath}\label{eqn:basicStatMechLecture15:20}
\epsilon_k = \frac{\hbar^2 k^2}{2m}
\end{dmath}

Our Hamiltonian is

\begin{dmath}\label{eqn:basicStatMechLecture15:40}
H = \sum_k \hat{n}_k \epsilon_k,
\end{dmath}

where we have a number operator

\begin{dmath}\label{eqn:basicStatMechLecture15:60}
N = \sum \hat{n}_k,
\end{dmath}

such that 

\begin{dmath}\label{eqn:basicStatMechLecture15:80}
\antisymmetric{N}{H} = 0.
\end{dmath}

\begin{dmath}\label{eqn:basicStatMechLecture15:100}
\ZG 
= \sum_{N=0}^\infty e^{\beta \mu N}
\sum_{n_k, \sum n_k = N} e^{-\beta \sum_k n_k \epsilon_k}.
\end{dmath}

While the second sum is constrained, because we are summing over all $n_k$, this is essentially an unconstrained sum, so we can write

\begin{dmath}\label{eqn:basicStatMechLecture15:120}
\ZG
= \sum_{n_k}
e^{\beta \mu \sum_k n_k}
e^{-\beta \sum_k n_k \epsilon_k}
=
\sum_{n_k} \lr{ \prod_k e^{-\beta(\epsilon_k - \mu) n_k}}
=
\prod_{n} \lr{ \sum_{n_k} e^{-\beta(\epsilon_k - \mu) n_k}}.
\end{dmath}

\paragraph{Fermions}

\begin{dmath}\label{eqn:basicStatMechLecture15:140}
\sum_{n_k = 0}^1 e^{-\beta(\epsilon_k - \mu) n_k} 
= 
1 + e^{-\beta(\epsilon_k - \mu)}
\end{dmath}

\paragraph{Bosons}

\begin{dmath}\label{eqn:basicStatMechLecture15:160}
\sum_{n_k = 0}^\infty e^{-\beta(\epsilon_k - \mu) n_k} = 
\inv{
1 - e^{-\beta(\epsilon_k - \mu)}
}
\end{dmath}

($\epsilon_k - \mu \ge 0$).

Our grand partition functions are then

\begin{subequations}
\begin{dmath}\label{eqn:basicStatMechLecture15:180}
\ZG^f = \prod_k 
\lr{ 1 + e^{-\beta(\epsilon_k - \mu)} }
\end{dmath}
\begin{dmath}\label{eqn:basicStatMechLecture15:200}
\ZG^b = \prod_k 
\inv{ 1 - e^{-\beta(\epsilon_k - \mu)} }
\end{dmath}
\end{subequations}

We can use these to compute the average number of particles

\begin{dmath}\label{eqn:basicStatMechLecture15:220}
\expectation{n_k^f} 
= \frac{1 \times 0 + 
e^{-\beta(\epsilon_k - \mu)} \times 1}
{ 1 + e^{-\beta(\epsilon_k - \mu)} }
=
\inv{ 1 + e^{-\beta(\epsilon_k - \mu)} }
\end{dmath}

\begin{dmath}\label{eqn:basicStatMechLecture15:240}
\expectation{n_k^b} 
= \frac{
1 \times 0 + 
e^{-\beta(\epsilon_k - \mu)} \times 1
+
e^{-2 \beta(\epsilon_k - \mu)} \times 2 + \cdots
}
{ 
1
+
e^{-\beta(\epsilon_k - \mu)} 
+
e^{-2 \beta(\epsilon_k - \mu)} 
}
%=
%\inv{ e^{\beta(\epsilon_k - \mu)} - 1}.
\end{dmath}

This chemical potential over temperature exponential 

\begin{dmath}\label{eqn:basicStatMechLecture15:260}
e^{\beta \mu} \equiv z,
\end{dmath}

is called the \underlineAndIndex{fugacity}.  The denominator has the form

\begin{dmath}\label{eqn:basicStatMechLecture15:280}
D = 
1 
+ z e^{-\beta \epsilon_k}
+ z^2 e^{-2 \beta \epsilon_k},
\end{dmath}

so we see that

\begin{dmath}\label{eqn:basicStatMechLecture15:300}
z \PD{z}{D} = 
  z e^{-\beta \epsilon_k}
+ 2 z^2 e^{-2 \beta \epsilon_k}
+ 3 z^3 e^{-3 \beta \epsilon_k}
+ \cdots
\end{dmath}

Thus the numerator is

\begin{dmath}\label{eqn:basicStatMechLecture15:320}
N = z \PD{z}{D},
\end{dmath}

and

\begin{dmath}\label{eqn:basicStatMechLecture15:340}
\expectation{n_k^b} 
= \frac{z \PD{z}{D_k} }{D_k}
= z \PD{z}{} \ln D_k
= \cdots
=
\inv{ e^{\beta(\epsilon_k - \mu)} - 1}
\end{dmath}

\paragraph{What is the density $\rho$?}

For Fermions

\begin{dmath}\label{eqn:basicStatMechLecture15:360}
\rho = \frac{N}{V} =
\inv{V} \sum_{\Bk}
\inv{ e^{\beta(\epsilon_\Bk - \mu)} + 1}
\end{dmath}

Using a ``particle in a box'' quantizatation where $k_\alpha = 2 \pi m_\alpha/L$, in a $d$-dimensional space, we can approximate this as

\begin{dmath}\label{eqn:basicStatMechLecture15:380}
\myBoxed{
\rho = 
\int \frac{d^d k}{(2 \pi)^d}
\inv{ e^{\beta(\epsilon_k - \mu)} - 1}.
}
\end{dmath}

This integral is actually difficult to evaluate.  For $T \rightarrow 0$ ($\beta \rightarrow \infty$, where

\begin{dmath}\label{eqn:basicStatMechLecture15:400}
n_k = \Theta(\mu - \epsilon_k).
\end{dmath}

This is illustrated in, where we also show the smearing that occurs as temperature increases

F2

With 

\begin{equation}\label{eqn:basicStatMechLecture15:420}
E_{\mathrm{F}} = \mu(T = 0),
\end{equation}

we want to ask what is the radius of the ball for which

\begin{dmath}\label{eqn:basicStatMechLecture15:560}
\epsilon_k = E_{\mathrm{F}}
\end{dmath}

or 

\begin{dmath}\label{eqn:basicStatMechLecture15:440}
E_{\mathrm{F}} = \frac{\hbar^2 k_{\mathrm{F}}^2}{2m},
\end{dmath}

so that 

\begin{dmath}\label{eqn:basicStatMechLecture15:460}
k_{\mathrm{F}} = \sqrt{\frac{2 m E_{\mathrm{F}}}{\hbar^2}},
\end{dmath}

so that our density where $\epsilon_k = \mu$ is

\begin{dmath}\label{eqn:basicStatMechLecture15:480}
\rho
=
\int_{k \le k_{\mathrm{F}}} \frac{d^3 k}{(2 \pi)^3} \times 1 
= 
\inv{(2\pi)^3} 4 \pi \int^{k_{\mathrm{F}}} k^2 dk
= \frac{4 \pi}{3} k_{\mathrm{F}}^3 \inv{(2 \pi)^3},
\end{dmath}

so that 

\begin{dmath}\label{eqn:basicStatMechLecture15:500}
k_{\mathrm{F}} = (6 \pi^2 \rho)^{1/3},
\end{dmath}

Our chemical potential at zero temperature is then

\begin{equation}\label{eqn:basicStatMechLecture15:520}
\mu(T = 0) = \frac{\hbar^2}{2m} (6 \pi^2 \rho)^{2/3}.
\end{equation}

\begin{dmath}\label{eqn:basicStatMechLecture15:540}
\rho^{-1/3} = \mbox{interparticle spacing}.
\end{dmath}

We can convince ourself that the chemical potential must have the form

F3

Given large negative chemical potential at high temperatures our number distribution will have the form

\begin{dmath}\label{eqn:basicStatMechLecture15:580}
\expectation{n_k} = e^{-\beta (\epsilon_k - \mu)} \propto e^{-\beta \epsilon_k}
\end{dmath}

We see that the classical Boltzman distribution is recovered for high temperatures.

We can also calculate the chemical potential at high temperatures.  We'll find that this has the form

\begin{dmath}\label{eqn:basicStatMechLecture15:n}
e^{\beta \mu} = \frac{4}{3} \rho \lambda_T^3,
\end{dmath}

where this quantity $\lambda_T$ is called the \underlineAndIndex{Thermal de Broglie wavelength}.

\begin{dmath}\label{eqn:basicStatMechLecture15:600}
\lambda_T = \sqrt{\frac{ 2 \pi \hbar^2}{m \kB T}}.
\end{dmath}

%\EndArticle
\EndNoBibArticle
