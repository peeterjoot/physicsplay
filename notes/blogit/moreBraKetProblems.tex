%
% Copyright � 2015 Peeter Joot.  All Rights Reserved.
% Licenced as described in the file LICENSE under the root directory of this GIT repository.
%
\newcommand{\authorname}{Peeter Joot}
\newcommand{\email}{peeterjoot@protonmail.com}
\newcommand{\basename}{FIXMEbasenameUndefined}
\newcommand{\dirname}{notes/FIXMEdirnameUndefined/}

\renewcommand{\basename}{moreBraKetProblems}
\renewcommand{\dirname}{notes/FIXMEwheretodirname/}
%\newcommand{\dateintitle}{}
%\newcommand{\keywords}{}

\newcommand{\authorname}{Peeter Joot}
\newcommand{\onlineurl}{http://sites.google.com/site/peeterjoot2/math2013/\basename.pdf}
\newcommand{\sourcepath}{\dirname\basename.tex}
\newcommand{\generatetitle}[1]{\chapter{#1}}

\newcommand{\vcsinfo}{%
\section*{}
\noindent{\color{DarkOliveGreen}{\rule{\linewidth}{0.1mm}}}
\paragraph{Document version}
%\paragraph{\color{Maroon}{Document version}}
{
\small
\begin{itemize}
\item Available online at:\\ 
\href{\onlineurl}{\onlineurl}
\item Git Repository: \input{./.revinfo/gitRepo.tex}
\item Source: \sourcepath
\item last commit: \input{./.revinfo/gitCommitString.tex}
\item commit date: \input{./.revinfo/gitCommitDate.tex}
\end{itemize}
}
}

%\PassOptionsToPackage{dvipsnames,svgnames}{xcolor}
\PassOptionsToPackage{square,numbers}{natbib}
\documentclass{scrreprt}

\usepackage[left=2cm,right=2cm]{geometry}
\usepackage[svgnames]{xcolor}
\usepackage{peeters_layout}

\usepackage{natbib}

\usepackage[
colorlinks=true,
bookmarks=false,
pdfauthor={\authorname, \email},
backref 
]{hyperref}

% http://tex.stackexchange.com/questions/75773/how-to-reference-problems-by-the-text-label-in-an-exercise-envioronment
\usepackage[english]{cleveref}
\crefname{Exercise}{exercise}{exercises}
\Crefname{Exercise}{Exercise}{Exercises}

\RequirePackage{titlesec}
\RequirePackage{ifthen}

% http://stackoverflow.com/questions/4932910/date-in-the-tabular-environment
\makeatletter
\let\insertdate\@date
\makeatother

\titleformat{\chapter}[display]
{\bfseries\Large}
{\color{DarkSlateGrey}\filleft \authorname
\ifthenelse{\isundefined{\studentnumber}}{}{\\ \studentnumber}
\ifthenelse{\isundefined{\email}}{}{\\ \email}
\ifthenelse{\isundefined{\dateintitle}}{}{\\ \insertdate}
%\ifthenelse{\isundefined{\coursename}}{}{\\ \coursename} % put in title instead.
}
{4ex}
{\color{DarkOliveGreen}{\titlerule}\color{Maroon}
\vspace{2ex}%
\filright}
[\vspace{2ex}%
\color{DarkOliveGreen}\titlerule
]

\newcommand{\beginArtWithToc}[0]{\begin{document}\tableofcontents}
\newcommand{\beginArtNoToc}[0]{\begin{document}}
\newcommand{\EndNoBibArticle}[0]{\end{document}}
\newcommand{\EndArticle}[0]{\bibliography{Bibliography}\bibliographystyle{plainnat}\end{document}}

% 
%\newcommand{\citep}[1]{\cite{#1}}

\colorSectionsForArticle



\usepackage{peeters_layout_exercise}
\usepackage{peeters_braket}

\beginArtNoToc

\generatetitle{A couple more bra-ket problems}
%\chapter{A couple more bra-ket problems}
%\label{chap:moreBraKetProblems}


\makeoproblem{Operator matrix representation}{problem:moreBraKetProblems:1.5}{\citep{sakurai2014modern} pr. 1.5}{

\makesubproblem{}{problem:moreBraKetProblems:1.5:a}

Determine the matrix representation of \( \ket{\alpha}\bra{\beta} \) given a complete set of eigenvectors \( \ket{a^r} \).

\makesubproblem{}{problem:moreBraKetProblems:1.5:b}

Verify with \( \ket{\alpha} = \ket{s_z = \hbar/2}, \ket{s_x = \hbar/2} \).

} % problem

\makeanswer{problem:moreBraKetProblems:1.5}{

\makeSubAnswer{}{problem:moreBraKetProblems:1.5:a}

Forming the matrix element

\begin{dmath}\label{eqn:moreBraKetProblems:20}
\bra{a^r} \lr{ \ket{\alpha}\bra{\beta} } \ket{a^s}
=
\braket{a^r}{\alpha}\braket{\beta}{a^s}
=
\braket{a^r}{\alpha}
\braket{a^s}{\beta}^\conj,
\end{dmath}

the matrix representation is seen to be

\begin{dmath}\label{eqn:moreBraKetProblems:40}
\ket{\alpha}\bra{\beta}
\sim
\begin{bmatrix}
\bra{a^1} \lr{ \ket{\alpha}\bra{\beta} } \ket{a^1} & \bra{a^1} \lr{ \ket{\alpha}\bra{\beta} } \ket{a^2} & \cdots \\
\bra{a^2} \lr{ \ket{\alpha}\bra{\beta} } \ket{a^1} & \bra{a^2} \lr{ \ket{\alpha}\bra{\beta} } \ket{a^2} & \cdots \\
\vdots & \vdots & \ddots \\
\end{bmatrix}
=
\begin{bmatrix}
\braket{a^1}{\alpha} \braket{a^1}{\beta}^\conj & \braket{a^1}{\alpha} \braket{a^2}{\beta}^\conj & \cdots \\
\braket{a^2}{\alpha} \braket{a^1}{\beta}^\conj & \braket{a^2}{\alpha} \braket{a^2}{\beta}^\conj & \cdots \\
\vdots & \vdots & \ddots \\
\end{bmatrix}.
\end{dmath}

\makeSubAnswer{}{problem:moreBraKetProblems:1.5:b}

First compute the spin-z representation of \( \ket{s_x = \hbar/2 } \).  

\begin{dmath}\label{eqn:moreBraKetProblems:60}
\lr{ S_x - \hbar/2 I }
\begin{bmatrix}
a \\
b
\end{bmatrix}
=
\lr{
\begin{bmatrix}
0 & \hbar/2 \\
\hbar/2 & 0 \\
\end{bmatrix}
-
\begin{bmatrix}
\hbar/2 & 0 \\
0 & \hbar/2 \\
\end{bmatrix}
}
=
\begin{bmatrix}
a \\
b
\end{bmatrix}
=
\frac{\hbar}{2}
\begin{bmatrix}
-1 & 1 \\
1 & -1 \\
\end{bmatrix}
\begin{bmatrix}
a \\
b
\end{bmatrix},
\end{dmath}

so \( \ket{s_x = \hbar/2 } \propto (1,1) \).

Normalized we have

\begin{equation}\label{eqn:moreBraKetProblems:80}
\begin{aligned}
\ket{\alpha} &= \ket{s_z = \hbar/2 } = 
\begin{bmatrix}
1 \\
0
\end{bmatrix} \\
\ket{\beta} &= \ket{s_z = \hbar/2 }
\inv{\sqrt{2}}
\begin{bmatrix}
1 \\
1
\end{bmatrix}.
\end{aligned}
\end{equation}

Using \cref{eqn:moreBraKetProblems:40} the matrix representation is

\begin{dmath}\label{eqn:moreBraKetProblems:100}
\ket{\alpha}\bra{\beta}
\sim
\begin{bmatrix}
(1) (1/\sqrt{2})^\conj & (1) (1/\sqrt{2})^\conj \\
(0) (1/\sqrt{2})^\conj & (0) (1/\sqrt{2})^\conj \\
\end{bmatrix}
=
\inv{\sqrt{2}}
\begin{bmatrix}
1 & 1 \\
0 & 0
\end{bmatrix}.
\end{dmath}

This can be confirmed with direct computation
\begin{dmath}\label{eqn:moreBraKetProblems:120}
\ket{\alpha}\bra{\beta}
=
\begin{bmatrix}
1 \\
0
\end{bmatrix}
\inv{\sqrt{2}}
\begin{bmatrix}
1 & 1
\end{bmatrix}
=
\inv{\sqrt{2}}
\begin{bmatrix}
1 & 1 \\
0 & 0
\end{bmatrix}.
\end{dmath}

} % answer

\EndArticle
