%
% Copyright � 2015 Peeter Joot.  All Rights Reserved.
% Licenced as described in the file LICENSE under the root directory of this GIT repository.
%
\newcommand{\authorname}{Peeter Joot}
\newcommand{\email}{peeterjoot@protonmail.com}
\newcommand{\basename}{FIXMEbasenameUndefined}
\newcommand{\dirname}{notes/FIXMEdirnameUndefined/}

\renewcommand{\basename}{gaMaxwellMagneticSources}
\renewcommand{\dirname}{notes/ece1229/}
%\newcommand{\dateintitle}{}
%\newcommand{\keywords}{}

\newcommand{\authorname}{Peeter Joot}
\newcommand{\onlineurl}{http://sites.google.com/site/peeterjoot2/math2013/\basename.pdf}
\newcommand{\sourcepath}{\dirname\basename.tex}
\newcommand{\generatetitle}[1]{\chapter{#1}}

\newcommand{\vcsinfo}{%
\section*{}
\noindent{\color{DarkOliveGreen}{\rule{\linewidth}{0.1mm}}}
\paragraph{Document version}
%\paragraph{\color{Maroon}{Document version}}
{
\small
\begin{itemize}
\item Available online at:\\ 
\href{\onlineurl}{\onlineurl}
\item Git Repository: \input{./.revinfo/gitRepo.tex}
\item Source: \sourcepath
\item last commit: \input{./.revinfo/gitCommitString.tex}
\item commit date: \input{./.revinfo/gitCommitDate.tex}
\end{itemize}
}
}

%\PassOptionsToPackage{dvipsnames,svgnames}{xcolor}
\PassOptionsToPackage{square,numbers}{natbib}
\documentclass{scrreprt}

\usepackage[left=2cm,right=2cm]{geometry}
\usepackage[svgnames]{xcolor}
\usepackage{peeters_layout}

\usepackage{natbib}

\usepackage[
colorlinks=true,
bookmarks=false,
pdfauthor={\authorname, \email},
backref 
]{hyperref}

% http://tex.stackexchange.com/questions/75773/how-to-reference-problems-by-the-text-label-in-an-exercise-envioronment
\usepackage[english]{cleveref}
\crefname{Exercise}{exercise}{exercises}
\Crefname{Exercise}{Exercise}{Exercises}

\RequirePackage{titlesec}
\RequirePackage{ifthen}

% http://stackoverflow.com/questions/4932910/date-in-the-tabular-environment
\makeatletter
\let\insertdate\@date
\makeatother

\titleformat{\chapter}[display]
{\bfseries\Large}
{\color{DarkSlateGrey}\filleft \authorname
\ifthenelse{\isundefined{\studentnumber}}{}{\\ \studentnumber}
\ifthenelse{\isundefined{\email}}{}{\\ \email}
\ifthenelse{\isundefined{\dateintitle}}{}{\\ \insertdate}
%\ifthenelse{\isundefined{\coursename}}{}{\\ \coursename} % put in title instead.
}
{4ex}
{\color{DarkOliveGreen}{\titlerule}\color{Maroon}
\vspace{2ex}%
\filright}
[\vspace{2ex}%
\color{DarkOliveGreen}\titlerule
]

\newcommand{\beginArtWithToc}[0]{\begin{document}\tableofcontents}
\newcommand{\beginArtNoToc}[0]{\begin{document}}
\newcommand{\EndNoBibArticle}[0]{\end{document}}
\newcommand{\EndArticle}[0]{\bibliography{Bibliography}\bibliographystyle{plainnat}\end{document}}

% 
%\newcommand{\citep}[1]{\cite{#1}}

\colorSectionsForArticle



\usepackage{ece1229}

\beginArtNoToc

\generatetitle{Maxwell's (phasor) equations in Geometric Algebra}
\index{Geometric Algebra}
\index{Maxwell equation}
\index{Maxwell's equation}
%\chapter{Maxwell's (phasor) equations in Geometric Algebra}
%\label{chap:gaMaxwellMagneticSources}

Maxwell's equations including both electric and magnetic charges and currents have the form
\index{electric charge density}
\index{electric current density}
\index{magnetic charge density}
\index{magnetic current density}

\begin{subequations}
\begin{dmath}\label{eqn:gaMaxwellMagneticSources:20}
\spacegrad \cross \BE = -\PD{t}{\BB} -\BM
\end{dmath}
\begin{dmath}\label{eqn:gaMaxwellMagneticSources:40}
\spacegrad \cross \BH = \BJ + \PD{t}{\BD}
\end{dmath}
\begin{dmath}\label{eqn:gaMaxwellMagneticSources:60}
\spacegrad \cdot \BD = \rho
\end{dmath}
\begin{dmath}\label{eqn:gaMaxwellMagneticSources:80}
\spacegrad \cdot \BB = \rho_\txtm.
\end{dmath}
\end{subequations}

Assuming linear media \( \BB = \mu_0 \BH \), \( \BD = \epsilon_0 \BE \) these reduce to
\index{linear media}

\begin{subequations}
\label{eqn:gaMaxwellMagneticSources:99}
\begin{dmath}\label{eqn:gaMaxwellMagneticSources:100}
\spacegrad \cross \BE = - \PD{t}{\BB} -\BM
\end{dmath}
\begin{dmath}\label{eqn:gaMaxwellMagneticSources:120}
\spacegrad \cross \BB = \mu_0 \BJ + \epsilon_0 \mu_0 \PD{t}{\BE}
\end{dmath}
\begin{dmath}\label{eqn:gaMaxwellMagneticSources:140}
\spacegrad \cdot \BE = \rho/\epsilon_0
\end{dmath}
\begin{dmath}\label{eqn:gaMaxwellMagneticSources:160}
\spacegrad \cdot \BB = \rho_\txtm.
\end{dmath}
\end{subequations}

The cross products can be eliminated using the duality relation \( \Ba \cross \Bb = -I \Ba \wedge \Bb \), which gives

\begin{subequations}
\label{eqn:gaMaxwellMagneticSources:180}
\begin{dmath}\label{eqn:gaMaxwellMagneticSources:200}
\spacegrad \wedge \BE = -I \lr{ \PD{t}{\BB} +\BM }
\end{dmath}
\begin{dmath}\label{eqn:gaMaxwellMagneticSources:220}
\spacegrad \wedge \BB = I \lr{\mu_0 \BJ + \epsilon_0 \mu_0 \PD{t}{\BE}}.
\end{dmath}
\end{subequations}

Now the divergence equations can be added to the curls

\begin{subequations}
\label{eqn:gaMaxwellMagneticSources:240}
\begin{dmath}\label{eqn:gaMaxwellMagneticSources:260}
\spacegrad \BE = \rho/\epsilon_0 -I \lr{ \PD{t}{\BB} +\BM }
\end{dmath}
\begin{dmath}\label{eqn:gaMaxwellMagneticSources:280}
\spacegrad \BB = \rho_\txtm + I \lr{\mu_0 \BJ + \epsilon_0 \mu_0 \PD{t}{\BE}}.
\end{dmath}
\end{subequations}

Noting that \( \epsilon_0 \mu_0 = 1/c^2 \) this can be summed as

\begin{dmath}\label{eqn:gaMaxwellMagneticSources:300}
\spacegrad \lr{ \BE + I c \BB } 
= \rho/\epsilon_0 -\lr{ \inv{c} \PD{t}{I c \BB} + I\BM }
+ I c \rho_\txtm - \lr{c \mu_0 \BJ + \inv{c} \PD{t}{\BE}},
\end{dmath}

or

\begin{dmath}\label{eqn:gaMaxwellMagneticSources:320}
\lr{ \spacegrad + \inv{c} \PD{t}{} } \lr{ \BE + I c \BB } 
= \rho/\epsilon_0 - c \mu_0 \BJ + I c \rho_\txtm - I \BM 
\end{dmath}

\section{Relativistic form}

Premultiplying by \( \gamma_0 \) and using the spatial basis representation \( \sigma_k = \gamma_k \gamma_0 \), the four vector form of Maxwell's equation is

\begin{dmath}\label{eqn:gaMaxwellMagneticSources:n}
\begin{aligned}
\grad F &= J + I M \\
F &= \BE + I c \BB \\
J &= \inv{\epsilon_0 c} \lr{ c \rho, \BJ } \\
M &= \lr{ c \rho_\txtm, \BM }.
\end{aligned}
\end{dmath}

This can be separated into a pair of equations
\begin{dmath}\label{eqn:gaMaxwellMagneticSources:n}
\begin{aligned}
\grad \wedge F &= I M \\
\grad \cdot F &= J.
\end{aligned}
\end{dmath}

When there is no magnetic four-current, we can write \( F = \grad \wedge A \), and the equations reduce to

\begin{dmath}\label{eqn:gaMaxwellMagneticSources:n}
\grad^2 A - \grad (\grad \cdot A) = J.
\end{dmath}

This can be further simplified using the Lorentz gauge, but this approach doesn't work when \( M \ne 0 \).  Suppose instead that \( J = 0 \).  A duality transformation of the form \( F = I G \) can be used in that case.  This gives

\begin{dmath}\label{eqn:gaMaxwellMagneticSources:n}
I M 
= \gpgradethree{ \grad I G  } 
= -\gpgradethree{ I \grad G  } 
= -I \grad \cdot G,
\end{dmath}

and
\begin{dmath}\label{eqn:gaMaxwellMagneticSources:n}
0 
= \gpgradeone{ \grad I G  } 
= -\gpgradeone{ I \grad G  } 
= -I \grad \wedge G,
\end{dmath}

or
\begin{dmath}\label{eqn:gaMaxwellMagneticSources:n}
\begin{aligned}
\grad \wedge G &= 0 \\
\grad \cdot G &= - M
\end{aligned}
\end{dmath}

It is clear that this can be reduced to one equation with a trial solution of \( G = \grad \wedge A_\txtm \), or

\begin{dmath}\label{eqn:gaMaxwellMagneticSources:n}
\grad^2 A_\txtm - \grad \lr{ \grad \cdot A_\txtm } = -M.
\end{dmath}

This can also be simplified with a Lorentz gauge selection for the four-potential.

\EndArticle
