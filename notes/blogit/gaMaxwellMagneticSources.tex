%
% Copyright � 2015 Peeter Joot.  All Rights Reserved.
% Licenced as described in the file LICENSE under the root directory of this GIT repository.
%
\newcommand{\authorname}{Peeter Joot}
\newcommand{\email}{peeterjoot@protonmail.com}
\newcommand{\basename}{FIXMEbasenameUndefined}
\newcommand{\dirname}{notes/FIXMEdirnameUndefined/}

\renewcommand{\basename}{gaMaxwellMagneticSources}
\renewcommand{\dirname}{notes/ece1229/}
%\newcommand{\dateintitle}{}
%\newcommand{\keywords}{}

\newcommand{\authorname}{Peeter Joot}
\newcommand{\onlineurl}{http://sites.google.com/site/peeterjoot2/math2013/\basename.pdf}
\newcommand{\sourcepath}{\dirname\basename.tex}
\newcommand{\generatetitle}[1]{\chapter{#1}}

\newcommand{\vcsinfo}{%
\section*{}
\noindent{\color{DarkOliveGreen}{\rule{\linewidth}{0.1mm}}}
\paragraph{Document version}
%\paragraph{\color{Maroon}{Document version}}
{
\small
\begin{itemize}
\item Available online at:\\ 
\href{\onlineurl}{\onlineurl}
\item Git Repository: \input{./.revinfo/gitRepo.tex}
\item Source: \sourcepath
\item last commit: \input{./.revinfo/gitCommitString.tex}
\item commit date: \input{./.revinfo/gitCommitDate.tex}
\end{itemize}
}
}

%\PassOptionsToPackage{dvipsnames,svgnames}{xcolor}
\PassOptionsToPackage{square,numbers}{natbib}
\documentclass{scrreprt}

\usepackage[left=2cm,right=2cm]{geometry}
\usepackage[svgnames]{xcolor}
\usepackage{peeters_layout}

\usepackage{natbib}

\usepackage[
colorlinks=true,
bookmarks=false,
pdfauthor={\authorname, \email},
backref 
]{hyperref}

% http://tex.stackexchange.com/questions/75773/how-to-reference-problems-by-the-text-label-in-an-exercise-envioronment
\usepackage[english]{cleveref}
\crefname{Exercise}{exercise}{exercises}
\Crefname{Exercise}{Exercise}{Exercises}

\RequirePackage{titlesec}
\RequirePackage{ifthen}

% http://stackoverflow.com/questions/4932910/date-in-the-tabular-environment
\makeatletter
\let\insertdate\@date
\makeatother

\titleformat{\chapter}[display]
{\bfseries\Large}
{\color{DarkSlateGrey}\filleft \authorname
\ifthenelse{\isundefined{\studentnumber}}{}{\\ \studentnumber}
\ifthenelse{\isundefined{\email}}{}{\\ \email}
\ifthenelse{\isundefined{\dateintitle}}{}{\\ \insertdate}
%\ifthenelse{\isundefined{\coursename}}{}{\\ \coursename} % put in title instead.
}
{4ex}
{\color{DarkOliveGreen}{\titlerule}\color{Maroon}
\vspace{2ex}%
\filright}
[\vspace{2ex}%
\color{DarkOliveGreen}\titlerule
]

\newcommand{\beginArtWithToc}[0]{\begin{document}\tableofcontents}
\newcommand{\beginArtNoToc}[0]{\begin{document}}
\newcommand{\EndNoBibArticle}[0]{\end{document}}
\newcommand{\EndArticle}[0]{\bibliography{Bibliography}\bibliographystyle{plainnat}\end{document}}

% 
%\newcommand{\citep}[1]{\cite{#1}}

\colorSectionsForArticle



\usepackage{siunitx}
\usepackage{esint} % \oiint
%\usepackage{ece1229}

\beginArtNoToc

\generatetitle{Boundary conditions for Maxwell's equations in linear media}
\index{Geometric Algebra}
\index{Maxwell equation}
\index{Maxwell's equation}
%\chapter{Boundary conditions for Maxwell's equations in linear media}
%\label{chap:gaMaxwellMagneticSources}

A previous analysis enumerated the boundary conditions implied by Maxwell's equations in free space.  Let's now generalize those to linear media, also allowing for magnetic charges and currents, a convient formalism in antenna theory \citep{balanis2005antenna}.  The equations in their traditional vector form are
\index{electric charge density}
\index{electric current density}
\index{magnetic charge density}
\index{magnetic current density}

\begin{subequations}
\begin{dmath}\label{eqn:gaMaxwellMagneticSources:20}
\spacegrad \cross \BE = -\PD{t}{\BB} -\BM
\end{dmath}
\begin{dmath}\label{eqn:gaMaxwellMagneticSources:40}
\spacegrad \cross \BH = \BJ + \PD{t}{\BD}
\end{dmath}
\begin{dmath}\label{eqn:gaMaxwellMagneticSources:60}
\spacegrad \cdot \BD = \rho
\end{dmath}
\begin{dmath}\label{eqn:gaMaxwellMagneticSources:80}
\spacegrad \cdot \BB = \rho_\txtm,
\end{dmath}
\end{subequations}

where, for linear media \( \BB = \mu \BH \) and \( \BD = \epsilon \BE \).  We will see that this describes waves that propagate with velocity \( v = 1/\sqrt{\mu\epsilon} \).
\index{linear media}

After scaling, application of duality transformations \( \Ba \cross \Bb = -I \Ba \wedge \Bb \), and the use of \( \spacegrad \cdot \Bx + \spacegrad \wedge \Bx = \spacegrad \Bx \), we have

%\begin{subequations}
%\label{eqn:gaMaxwellMagneticSources:180}
%\begin{dmath}\label{eqn:gaMaxwellMagneticSources:200}
%\spacegrad \wedge \BD = -I \epsilon \lr{ \PD{t}{\BB} +\BM }
%\end{dmath}
%\begin{dmath}\label{eqn:gaMaxwellMagneticSources:220}
%\spacegrad \wedge \BH = I \lr{\BJ + \PD{t}{\BD}}.
%\end{dmath}
%\begin{dmath}\label{eqn:gaMaxwellMagneticSources:480}
%\spacegrad \cdot \BD = \rho
%\end{dmath}
%\begin{dmath}\label{eqn:gaMaxwellMagneticSources:500}
%\spacegrad \cdot \BH = \inv{\mu} \rho_\txtm,
%\end{dmath}
%\end{subequations}
%
%Because a geometric product of the gradient with a vector decomposes into a dot and curl ( \( \spacegrad \Bx = \spacegrad \cdot \Bx + \spacegrad \wedge \Bx \) ), the reverse transformation can be applied
%
\begin{subequations}
\label{eqn:gaMaxwellMagneticSources:240}
\begin{dmath}\label{eqn:gaMaxwellMagneticSources:520}
\spacegrad \BD = \rho + -I \epsilon \lr{ \mu \PD{t}{\BH} +\BM }
\end{dmath}
\begin{dmath}\label{eqn:gaMaxwellMagneticSources:540}
\spacegrad \BH = \inv{\mu} \rho_\txtm + I \lr{\BJ + \PD{t}{\BD}}.
\end{dmath}
\end{subequations}

Since the dimensions \( [\BH] = \si{A/m} \), and \( [\BD] = \si{A/s/m^2} \), these can be put in a dimensionally consistent form as follows

\begin{subequations}
\label{eqn:gaMaxwellMagneticSources:580}
\begin{dmath}\label{eqn:gaMaxwellMagneticSources:600}
\spacegrad \BD + \inv{v} \PD{t}{}(I \BH/v) = \rho -I \epsilon \BM
\end{dmath}
\begin{dmath}\label{eqn:gaMaxwellMagneticSources:620}
\spacegrad (I \BH/v) + \inv{v} \PD{t}{\BD} = \inv{\mu v} I \rho_\txtm - \inv{v} \BJ,
\end{dmath}
\end{subequations}

or, in multivector form

\begin{dmath}\label{eqn:gaMaxwellMagneticSources:560}
\lr{ \inv{v}\PD{t}{} + \spacegrad } \lr{ \BD + \inv{v} I \BH } 
= \inv{v} \lr{ v \rho - \BJ } 
  + \frac{\epsilon}{v} I \lr{ \rho_\txtm - v \BM }.
\end{dmath}

Observe that pre-multiplication with \( \inv{v} \partial_t - \spacegrad \) produces a wave equation operator \( \inv{v^2}\partial_{tt} - \spacegrad^2 \), validating the wave velocity assertion.

This multivector formula can be summarized nicely in the traditional GA Maxwell's form 

%\begin{dmath}\label{eqn:gaMaxwellMagneticSources:640}
\boxedEquation{eqn:gaMaxwellMagneticSources:640}{
\grad F &= J \\
\begin{aligned}
\grad &= \gamma_0 \lr{ \inv{v}\PD{t}{} + \spacegrad } \\
F &= \BD + \inv{v} I \BH \\
J &= \frac{\gamma_0}{v} \lr{ v \rho - \BJ } + \frac{\epsilon \gamma_0}{v} I \lr{ \rho_\txtm - v \BM }.
\end{aligned}
}
%\end{dmath}

We have a different than usual scaling of the STA bivector \( F \) than with the free space formulation.  Inclusion of magnetic charges and currents means that \( J \) is now a multivector that also has a trivector component as well as the vector component.

\EndArticle
