%
% Copyright � 2015 Peeter Joot.  All Rights Reserved.
% Licenced as described in the file LICENSE under the root directory of this GIT repository.
%
\documentclass[]{eliblog}

\usepackage{amsmath}
\usepackage{mathpazo}

%
% shorthand for bold symbols, convenient for vectors and matrices
%
\newcommand{\Ba}[0]{\mathbf{a}}
\newcommand{\Bb}[0]{\mathbf{b}}
\newcommand{\Bc}[0]{\mathbf{c}}
\newcommand{\Bd}[0]{\mathbf{d}}
\newcommand{\Be}[0]{\mathbf{e}}
\newcommand{\Bf}[0]{\mathbf{f}}
\newcommand{\Bg}[0]{\mathbf{g}}
\newcommand{\Bh}[0]{\mathbf{h}}
\newcommand{\Bi}[0]{\mathbf{i}}
\newcommand{\Bj}[0]{\mathbf{j}}
\newcommand{\Bk}[0]{\mathbf{k}}
\newcommand{\Bl}[0]{\mathbf{l}}
\newcommand{\Bm}[0]{\mathbf{m}}
\newcommand{\Bn}[0]{\mathbf{n}}
\newcommand{\Bo}[0]{\mathbf{o}}
\newcommand{\Bp}[0]{\mathbf{p}}
\newcommand{\Bq}[0]{\mathbf{q}}
\newcommand{\Br}[0]{\mathbf{r}}
\newcommand{\Bs}[0]{\mathbf{s}}
\newcommand{\Bt}[0]{\mathbf{t}}
\newcommand{\Bu}[0]{\mathbf{u}}
\newcommand{\Bv}[0]{\mathbf{v}}
\newcommand{\Bw}[0]{\mathbf{w}}
\newcommand{\Bx}[0]{\mathbf{x}}
\newcommand{\By}[0]{\mathbf{y}}
\newcommand{\Bz}[0]{\mathbf{z}}
\newcommand{\BA}[0]{\mathbf{A}}
\newcommand{\BB}[0]{\mathbf{B}}
\newcommand{\BC}[0]{\mathbf{C}}
\newcommand{\BD}[0]{\mathbf{D}}
\newcommand{\BE}[0]{\mathbf{E}}
\newcommand{\BF}[0]{\mathbf{F}}
\newcommand{\BG}[0]{\mathbf{G}}
\newcommand{\BH}[0]{\mathbf{H}}
\newcommand{\BI}[0]{\mathbf{I}}
\newcommand{\BJ}[0]{\mathbf{J}}
\newcommand{\BK}[0]{\mathbf{K}}
\newcommand{\BL}[0]{\mathbf{L}}
\newcommand{\BM}[0]{\mathbf{M}}
\newcommand{\BN}[0]{\mathbf{N}}
\newcommand{\BO}[0]{\mathbf{O}}
\newcommand{\BP}[0]{\mathbf{P}}
\newcommand{\BQ}[0]{\mathbf{Q}}
\newcommand{\BR}[0]{\mathbf{R}}
\newcommand{\BS}[0]{\mathbf{S}}
\newcommand{\BT}[0]{\mathbf{T}}
\newcommand{\BU}[0]{\mathbf{U}}
\newcommand{\BV}[0]{\mathbf{V}}
\newcommand{\BW}[0]{\mathbf{W}}
\newcommand{\BX}[0]{\mathbf{X}}
\newcommand{\BY}[0]{\mathbf{Y}}
\newcommand{\BZ}[0]{\mathbf{Z}}

\newcommand{\Bzero}[0]{\mathbf{0}}
\newcommand{\Btheta}[0]{\boldsymbol{\theta}}
\newcommand{\Btau}[0]{\boldsymbol{\tau}}
\newcommand{\Bomega}[0]{\boldsymbol{\omega}}

%
% shorthand for unit vectors
%
\newcommand{\acap}[0]{\hat{\Ba}}
\newcommand{\bcap}[0]{\hat{\Bb}}
\newcommand{\ccap}[0]{\hat{\Bc}}
\newcommand{\dcap}[0]{\hat{\Bd}}
\newcommand{\ecap}[0]{\hat{\Be}}
\newcommand{\fcap}[0]{\hat{\Bf}}
\newcommand{\gcap}[0]{\hat{\Bg}}
\newcommand{\hcap}[0]{\hat{\Bh}}
\newcommand{\icap}[0]{\hat{\Bi}}
\newcommand{\jcap}[0]{\hat{\Bj}}
\newcommand{\kcap}[0]{\hat{\Bk}}
\newcommand{\lcap}[0]{\hat{\Bl}}
\newcommand{\mcap}[0]{\hat{\Bm}}
\newcommand{\ncap}[0]{\hat{\Bn}}
\newcommand{\ocap}[0]{\hat{\Bo}}
\newcommand{\pcap}[0]{\hat{\Bp}}
\newcommand{\qcap}[0]{\hat{\Bq}}
\newcommand{\rcap}[0]{\hat{\Br}}
\newcommand{\scap}[0]{\hat{\Bs}}
\newcommand{\tcap}[0]{\hat{\Bt}}
\newcommand{\ucap}[0]{\hat{\Bu}}
\newcommand{\vcap}[0]{\hat{\Bv}}
\newcommand{\wcap}[0]{\hat{\Bw}}
\newcommand{\xcap}[0]{\hat{\Bx}}
\newcommand{\ycap}[0]{\hat{\By}}
\newcommand{\zcap}[0]{\hat{\Bz}}
\newcommand{\thetacap}[0]{\hat{\Btheta}}

%
% to write R^n and C^n in a distinguishable fashion.  Perhaps change this
% to the double lined characters upon figuring out how to do so.
%
\newcommand{\C}[1]{$\mathbb{C}^{#1}$}
\newcommand{\R}[1]{$\mathbb{R}^{#1}$}

%
% various generally useful helpers
%

% derivative of #1 wrt. #2:
\newcommand{\D}[2] {\frac {d#2} {d#1}}

\newcommand{\inv}[1]{\frac{1}{#1}}
\newcommand{\cross}[0]{\times}

\newcommand{\abs}[1]{\lvert{#1}\rvert}
\newcommand{\norm}[1]{\lVert{#1}\rVert}
\newcommand{\innerprod}[2]{\langle{#1}, {#2}\rangle}
\newcommand{\dotprod}[2]{{#1} \cdot {#2}}
\newcommand{\bdotprod}[2]{\left({#1} \cdot {#2}\right)}
\newcommand{\crossprod}[2]{{#1} \cross {#2}}
\newcommand{\tripleprod}[3]{\dotprod{\left(\crossprod{#1}{#2}\right)}{#3}}

\DeclareMathOperator{\Proj}{Proj}
\DeclareMathOperator{\Span}{span}
\DeclareMathOperator{\Sgn}{sgn}
\DeclareMathOperator{\Area}{Area}
\DeclareMathOperator{\Volume}{Volume}

%
% A few miscellaneous things specific to this document
%
\newcommand{\crossop}[1]{\crossprod{#1}{}}

% R2 vector.
\newcommand{\VectorTwo}[2]{
\begin{bmatrix}
 {#1} \\
 {#2}
\end{bmatrix}
}

\newcommand{\VectorN}[1]{
\begin{bmatrix}
{#1}_1 \\
{#1}_2 \\
\vdots \\
{#1}_N \\
\end{bmatrix}
}

\newcommand{\DETuvij}[4]{
\begin{vmatrix}
 {#1}_{#3} & {#1}_{#4} \\
 {#2}_{#3} & {#2}_{#4}
\end{vmatrix}
}

\newcommand{\DETuvwijk}[6]{
\begin{vmatrix}
 {#1}_{#4} & {#1}_{#5} & {#1}_{#6} \\
 {#2}_{#4} & {#2}_{#5} & {#2}_{#6} \\
 {#3}_{#4} & {#3}_{#5} & {#3}_{#6}
\end{vmatrix}
}

\newcommand{\DETuvwxijkl}[8]{
\begin{vmatrix}
 {#1}_{#5} & {#1}_{#6} & {#1}_{#7} & {#1}_{#8} \\
 {#2}_{#5} & {#2}_{#6} & {#2}_{#7} & {#2}_{#8} \\
 {#3}_{#5} & {#3}_{#6} & {#3}_{#7} & {#3}_{#8} \\
 {#4}_{#5} & {#4}_{#6} & {#4}_{#7} & {#4}_{#8} \\
\end{vmatrix}
}

%\newcommand{\DETuvwxyijklm}[10]{
%\begin{vmatrix}
% {#1}_{#6} & {#1}_{#7} & {#1}_{#8} & {#1}_{#9} & {#1}_{#10} \\
% {#2}_{#6} & {#2}_{#7} & {#2}_{#8} & {#2}_{#9} & {#2}_{#10} \\
% {#3}_{#6} & {#3}_{#7} & {#3}_{#8} & {#3}_{#9} & {#3}_{#10} \\
% {#4}_{#6} & {#4}_{#7} & {#4}_{#8} & {#4}_{#9} & {#4}_{#10} \\
% {#5}_{#6} & {#5}_{#7} & {#5}_{#8} & {#5}_{#9} & {#5}_{#10}
%\end{vmatrix}
%}

% R3 vector.
\newcommand{\VectorThree}[3]{
\begin{bmatrix}
 {#1} \\
 {#2} \\
 {#3}
\end{bmatrix}
}



\author{Peeter Joot}
\email{peeter.joot@gmail.com}

%\documentclass[]{eliblogwidescreen}

\usepackage{amsmath}
\usepackage{mathpazo}

%
% shorthand for bold symbols, convenient for vectors and matrices
%
\newcommand{\Ba}[0]{\mathbf{a}}
\newcommand{\Bb}[0]{\mathbf{b}}
\newcommand{\Bc}[0]{\mathbf{c}}
\newcommand{\Bd}[0]{\mathbf{d}}
\newcommand{\Be}[0]{\mathbf{e}}
\newcommand{\Bf}[0]{\mathbf{f}}
\newcommand{\Bg}[0]{\mathbf{g}}
\newcommand{\Bh}[0]{\mathbf{h}}
\newcommand{\Bi}[0]{\mathbf{i}}
\newcommand{\Bj}[0]{\mathbf{j}}
\newcommand{\Bk}[0]{\mathbf{k}}
\newcommand{\Bl}[0]{\mathbf{l}}
\newcommand{\Bm}[0]{\mathbf{m}}
\newcommand{\Bn}[0]{\mathbf{n}}
\newcommand{\Bo}[0]{\mathbf{o}}
\newcommand{\Bp}[0]{\mathbf{p}}
\newcommand{\Bq}[0]{\mathbf{q}}
\newcommand{\Br}[0]{\mathbf{r}}
\newcommand{\Bs}[0]{\mathbf{s}}
\newcommand{\Bt}[0]{\mathbf{t}}
\newcommand{\Bu}[0]{\mathbf{u}}
\newcommand{\Bv}[0]{\mathbf{v}}
\newcommand{\Bw}[0]{\mathbf{w}}
\newcommand{\Bx}[0]{\mathbf{x}}
\newcommand{\By}[0]{\mathbf{y}}
\newcommand{\Bz}[0]{\mathbf{z}}
\newcommand{\BA}[0]{\mathbf{A}}
\newcommand{\BB}[0]{\mathbf{B}}
\newcommand{\BC}[0]{\mathbf{C}}
\newcommand{\BD}[0]{\mathbf{D}}
\newcommand{\BE}[0]{\mathbf{E}}
\newcommand{\BF}[0]{\mathbf{F}}
\newcommand{\BG}[0]{\mathbf{G}}
\newcommand{\BH}[0]{\mathbf{H}}
\newcommand{\BI}[0]{\mathbf{I}}
\newcommand{\BJ}[0]{\mathbf{J}}
\newcommand{\BK}[0]{\mathbf{K}}
\newcommand{\BL}[0]{\mathbf{L}}
\newcommand{\BM}[0]{\mathbf{M}}
\newcommand{\BN}[0]{\mathbf{N}}
\newcommand{\BO}[0]{\mathbf{O}}
\newcommand{\BP}[0]{\mathbf{P}}
\newcommand{\BQ}[0]{\mathbf{Q}}
\newcommand{\BR}[0]{\mathbf{R}}
\newcommand{\BS}[0]{\mathbf{S}}
\newcommand{\BT}[0]{\mathbf{T}}
\newcommand{\BU}[0]{\mathbf{U}}
\newcommand{\BV}[0]{\mathbf{V}}
\newcommand{\BW}[0]{\mathbf{W}}
\newcommand{\BX}[0]{\mathbf{X}}
\newcommand{\BY}[0]{\mathbf{Y}}
\newcommand{\BZ}[0]{\mathbf{Z}}

\newcommand{\Bzero}[0]{\mathbf{0}}
\newcommand{\Btheta}[0]{\boldsymbol{\theta}}
\newcommand{\Btau}[0]{\boldsymbol{\tau}}
\newcommand{\Bomega}[0]{\boldsymbol{\omega}}

%
% shorthand for unit vectors
%
\newcommand{\acap}[0]{\hat{\Ba}}
\newcommand{\bcap}[0]{\hat{\Bb}}
\newcommand{\ccap}[0]{\hat{\Bc}}
\newcommand{\dcap}[0]{\hat{\Bd}}
\newcommand{\ecap}[0]{\hat{\Be}}
\newcommand{\fcap}[0]{\hat{\Bf}}
\newcommand{\gcap}[0]{\hat{\Bg}}
\newcommand{\hcap}[0]{\hat{\Bh}}
\newcommand{\icap}[0]{\hat{\Bi}}
\newcommand{\jcap}[0]{\hat{\Bj}}
\newcommand{\kcap}[0]{\hat{\Bk}}
\newcommand{\lcap}[0]{\hat{\Bl}}
\newcommand{\mcap}[0]{\hat{\Bm}}
\newcommand{\ncap}[0]{\hat{\Bn}}
\newcommand{\ocap}[0]{\hat{\Bo}}
\newcommand{\pcap}[0]{\hat{\Bp}}
\newcommand{\qcap}[0]{\hat{\Bq}}
\newcommand{\rcap}[0]{\hat{\Br}}
\newcommand{\scap}[0]{\hat{\Bs}}
\newcommand{\tcap}[0]{\hat{\Bt}}
\newcommand{\ucap}[0]{\hat{\Bu}}
\newcommand{\vcap}[0]{\hat{\Bv}}
\newcommand{\wcap}[0]{\hat{\Bw}}
\newcommand{\xcap}[0]{\hat{\Bx}}
\newcommand{\ycap}[0]{\hat{\By}}
\newcommand{\zcap}[0]{\hat{\Bz}}
\newcommand{\thetacap}[0]{\hat{\Btheta}}

%
% to write R^n and C^n in a distinguishable fashion.  Perhaps change this
% to the double lined characters upon figuring out how to do so.
%
\newcommand{\C}[1]{$\mathbb{C}^{#1}$}
\newcommand{\R}[1]{$\mathbb{R}^{#1}$}

%
% various generally useful helpers
%

% derivative of #1 wrt. #2:
\newcommand{\D}[2] {\frac {d#2} {d#1}}

\newcommand{\inv}[1]{\frac{1}{#1}}
\newcommand{\cross}[0]{\times}

\newcommand{\abs}[1]{\lvert{#1}\rvert}
\newcommand{\norm}[1]{\lVert{#1}\rVert}
\newcommand{\innerprod}[2]{\langle{#1}, {#2}\rangle}
\newcommand{\dotprod}[2]{{#1} \cdot {#2}}
\newcommand{\bdotprod}[2]{\left({#1} \cdot {#2}\right)}
\newcommand{\crossprod}[2]{{#1} \cross {#2}}
\newcommand{\tripleprod}[3]{\dotprod{\left(\crossprod{#1}{#2}\right)}{#3}}

\DeclareMathOperator{\Proj}{Proj}
\DeclareMathOperator{\Span}{span}
\DeclareMathOperator{\Sgn}{sgn}
\DeclareMathOperator{\Area}{Area}
\DeclareMathOperator{\Volume}{Volume}

%
% A few miscellaneous things specific to this document
%
\newcommand{\crossop}[1]{\crossprod{#1}{}}

% R2 vector.
\newcommand{\VectorTwo}[2]{
\begin{bmatrix}
 {#1} \\
 {#2}
\end{bmatrix}
}

\newcommand{\VectorN}[1]{
\begin{bmatrix}
{#1}_1 \\
{#1}_2 \\
\vdots \\
{#1}_N \\
\end{bmatrix}
}

\newcommand{\DETuvij}[4]{
\begin{vmatrix}
 {#1}_{#3} & {#1}_{#4} \\
 {#2}_{#3} & {#2}_{#4}
\end{vmatrix}
}

\newcommand{\DETuvwijk}[6]{
\begin{vmatrix}
 {#1}_{#4} & {#1}_{#5} & {#1}_{#6} \\
 {#2}_{#4} & {#2}_{#5} & {#2}_{#6} \\
 {#3}_{#4} & {#3}_{#5} & {#3}_{#6}
\end{vmatrix}
}

\newcommand{\DETuvwxijkl}[8]{
\begin{vmatrix}
 {#1}_{#5} & {#1}_{#6} & {#1}_{#7} & {#1}_{#8} \\
 {#2}_{#5} & {#2}_{#6} & {#2}_{#7} & {#2}_{#8} \\
 {#3}_{#5} & {#3}_{#6} & {#3}_{#7} & {#3}_{#8} \\
 {#4}_{#5} & {#4}_{#6} & {#4}_{#7} & {#4}_{#8} \\
\end{vmatrix}
}

%\newcommand{\DETuvwxyijklm}[10]{
%\begin{vmatrix}
% {#1}_{#6} & {#1}_{#7} & {#1}_{#8} & {#1}_{#9} & {#1}_{#10} \\
% {#2}_{#6} & {#2}_{#7} & {#2}_{#8} & {#2}_{#9} & {#2}_{#10} \\
% {#3}_{#6} & {#3}_{#7} & {#3}_{#8} & {#3}_{#9} & {#3}_{#10} \\
% {#4}_{#6} & {#4}_{#7} & {#4}_{#8} & {#4}_{#9} & {#4}_{#10} \\
% {#5}_{#6} & {#5}_{#7} & {#5}_{#8} & {#5}_{#9} & {#5}_{#10}
%\end{vmatrix}
%}

% R3 vector.
\newcommand{\VectorThree}[3]{
\begin{bmatrix}
 {#1} \\
 {#2} \\
 {#3}
\end{bmatrix}
}



\author{Peeter Joot}
\email{peeter.joot@gmail.com}


\chapter{Desai Chapter 9 notes and problems.}
\label{chap:desaiCh9}
%\useCCL
\blogpage{http://sites.google.com/site/peeterjoot/math2010/desaiCh9.pdf}
\date{Nov 19, 2010}
\revisionInfo{desaiCh9.tex}

\beginArtWithToc
%\beginArtNoToc

\section{Motivation.}

Chapter 9 notes for \cite{desai2009quantum}.

\section{Notes}
\section{Problems}

\subsection{Problem 4.}
\subsubsection{Statement.}

Consider the following two-dimensional harmonic oscilator problem:

\begin{align}\label{eqn:desaiCh9:400}
-\frac{\hbar^2}{2m} \frac{\partial^2 u}{\partial x^2}
-\frac{\hbar^2}{2m} \frac{\partial^2 u}{\partial y^2}
+ \inv{2} K_1 x^2 u
+ \inv{2} K_2 y^2 u
= E u
\end{align}

where $(x,y)$ are the coordinates of the particle.  Use the separation of variables technique to obtain the energy eigenvalues.  Discuss the degeneracy in the eigenvalues if $K_1 = K_2$.

\subsubsection{Solution.}

Write $u = A(x) B(y)$.  Substitute and dividing throughout by $u$ we have

\begin{align}\label{eqn:desaiCh9:401}
\left( -\frac{\hbar^2}{2m} \frac{A''}{A} + \inv{2} K_1 x^2 \right)
+\left( -\frac{\hbar^2}{2m} \frac{B''}{B} + \inv{2} K_2 y^2 \right)
= E
\end{align}

Introduction of a pair of constants $E_1, E_2$ for each of the independent terms we have

\begin{align}\label{eqn:desaiCh9:403}
H_1 A &= -\frac{\hbar^2}{2m} A'' + \inv{2} K_1 x^2 A = E_1 A \\
H_2 B &= -\frac{\hbar^2}{2m} B'' + \inv{2} K_1 y^2 B = E_2 B \\
H &= H_1 + H_2 \\
E  &= E_1 + E_2
\end{align}

For each of these equations we have a set of quantized eigenvalues and can write

\begin{align}\label{eqn:desaiCh9:404}
E_{1m} &= \left(m + \inv{2}\right) \hbar \sqrt{\frac{K_1}{m}} \\
E_{2n} &= \left(n + \inv{2}\right) \hbar \sqrt{\frac{K_2}{m}} \\
H_1 A_m(x) &= E_{1m} A_m(x) \\
H_2 A_n(y) &= E_{2n} B_n(y)
\end{align}

The complete eigenstates are then

\begin{align}\label{eqn:desaiCh9:405}
u_{mn}(x,y) &= A_m(x) B_n(y)
\end{align}

with total energy satisfying
\begin{align}\label{eqn:desaiCh9:406}
H u_{mn}(x,y) &=
\frac{\hbar}{\sqrt{m}} \left( \left(m + \inv{2}\right) \sqrt{K_1} + \left(n + \inv{2}\right) \sqrt{K_2} \right) u_{mn}(x,y)
\end{align}

A general state requires a double sum over the possible combinations of states $\Psi = \sum_{mn} c_{mn} u_{mn}$, however if $K_1 = K_2 = K$, we cannot distinguish between $u_{mn}$ and $u_{nm}$ based on the energy eigenvalues

\begin{align}\label{eqn:desaiCh9:407}
H u_{mn}(x,y) &= \hbar\sqrt{\frac{K}{m}} \left( m + n + 1 \right) u_{mn}(x,y) = H u_{nm}(x,y)
\end{align}

In this case, we can write the wave function corresponding to a general state for the system as just $\Psi = \sum_{m+ n = \text{constant}} c_{mn} u_{mn}$.  This reduction in the cardinality of this set of basis eigenstates is the degeneracy to be discussed.

\subsection{Problem 5,6.}
\subsubsection{Statement.}

Consider now a variation on Problem 4 in which we have a coupled oscillator with the potential given by

\begin{align}\label{eqn:desaiCh9:500}
V(x,y) = \inv{2} K \Bigl( x^2 + y^2 + 2 \lambda x y \Bigr)
\end{align}

Obtain the energy eigenvalues by changing variables $(x,y)$ to $(x', y')$ such that the new potential is quadratic in $(x', y')$, without the coupling term.

\subsubsection{Solution.}

This has the look of a diagonalization problem so we write the potential in matrix form

\begin{align}\label{eqn:desaiCh9:501}
V(x,y)
= \inv{2} K
\begin{bmatrix}
x & y
\end{bmatrix}
\begin{bmatrix}
1 & \lambda \\
\lambda & 1
\end{bmatrix}
\begin{bmatrix}
x \\ y
\end{bmatrix} = \inv{2} K \tilde{X} M X
\end{align}

The similarity transformation required is
\begin{align}\label{eqn:desaiCh9:502}
M = \inv{\sqrt{2}}
\begin{bmatrix}
1 & 1 \\
1 & -1
\end{bmatrix}
\begin{bmatrix}
1+ \lambda & 0 \\
0 & 1 - \lambda
\end{bmatrix}
\inv{\sqrt{2}}
\begin{bmatrix}
1 & 1 \\
1 & -1
\end{bmatrix}
\end{align}

Our change of variables is therefore
\begin{align}\label{eqn:desaiCh9:503}
X' =
\inv{\sqrt{2}}
\begin{bmatrix}
1 & 1 \\
1 & -1
\end{bmatrix}
X
=
\inv{\sqrt{2}}
\begin{bmatrix}
x + y \\
x - y
\end{bmatrix}
\end{align}

Our Laplacian should also remain diagonal under this orthonormal transformation, but we can verify this by expanding out the partials explicitly
\begin{align}\label{eqn:desaiCh9:504}
\PD{x}{} &=
\PD{x}{x'}\PD{x'}{}
+\PD{x}{y'}\PD{y'}{} = \inv{\sqrt{2}} \left( \PD{x'}{} + \PD{y'}{} \right) \\
\PD{y}{} &=
\PD{y}{x'}\PD{x'}{} +\PD{y}{y'}\PD{y'}{}
= \inv{\sqrt{2}}
\left( \PD{x'}{} - \PD{y'}{} \right)
\end{align}

Squaring and summing we have
\begin{align}\label{eqn:desaiCh9:505}
\frac{\partial^2}{\partial x^2} +
\frac{\partial^2}{\partial y^2}
&=
\inv{2} \left( \PD{x'}{} + \PD{y'}{} \right)^2
+\inv{2} \left( \PD{x'}{} - \PD{y'}{} \right)^2
=
\frac{\partial^2}{\partial {x'}^2} +
\frac{\partial^2}{\partial {y'}^2}
\end{align}

Our transformed Hamiltonian operator is thus

\begin{align}\label{eqn:desaiCh9:506}
-\frac{\hbar^2}{2m} \frac{\partial^2 u}{\partial {x'}^2}
-\frac{\hbar^2}{2m} \frac{\partial^2 u}{\partial {y'}^2}
+ \inv{2} K(1+\lambda) {x'}^2 u
+ \inv{2} K(1-\lambda) {y'}^2 u
= E u
\end{align}

So, provided $\Abs{\lambda} < 1$, the energy eigenvalue equation is given by \ref{eqn:desaiCh9:406} with $K_1 = K(1+ \lambda)$, and $K_2 = K(1 -\lambda)$.

\subsection{Problem 7.}
\subsubsection{Statement.}

Consider two coupled harmonic oscillators in one dimension of natural length $a$ and spring constant $K$ connecting three particles located at $x_1, x_2$, and $x_3$.  The corresponding Schr\"{o}dinger equation is given as

\begin{align}\label{eqn:desaiCh9:700}
-\frac{\hbar^2}{2m} \frac{\partial^2 u}{\partial {x_1}^2}
-\frac{\hbar^2}{2m} \frac{\partial^2 u}{\partial {x_2}^2}
-\frac{\hbar^2}{2m} \frac{\partial^2 u}{\partial {x_3}^2}
+ \frac{K}{2}
\left(
(x_2 - x_1 - a)^2
+(x_3 - x_2 - a)^2
\right) u
= E u
\end{align}

Obtain the energy eigenvalues using the matrix method.

\subsubsection{Solution.}

Let's start with an initial simplifying substutition to get rid of the factors of $a$.  Write

\begin{align}\label{eqn:desaiCh9:701}
r_1 &= x_1 + a \\
r_2 &= x_2 \\
r_3 &= x_3 - a
\end{align}

These were picked so that the differences in our quadratic terms involve only factors of $r_k$

\begin{align}\label{eqn:desaiCh9:702}
x_2 - x_1 - a &= r_2 - r_1 \\
x_3 - x_2 - a &= r_3 - r_2
\end{align}

Schr\"{o}dinger's equation is now

\begin{align}\label{eqn:desaiCh9:700a}
-\frac{\hbar^2}{2m} \frac{\partial^2 u}{\partial {r_1}^2}
-\frac{\hbar^2}{2m} \frac{\partial^2 u}{\partial {r_2}^2}
-\frac{\hbar^2}{2m} \frac{\partial^2 u}{\partial {r_3}^2}
+ \frac{K}{2}
\left(
(r_2 - r_1)^2
+(r_3 - r_2)^2
\right) u
= E u
\end{align}

Putting our potential into matrix form, we have

\begin{align}\label{eqn:desaiCh9:703}
V(r_1, r_2, r_3) &=
\frac{K}{2}
\left(
(r_2 - r_1)^2
+(r_3 - r_2)^2
\right)
=
\frac{K}{2}
\begin{bmatrix}
r_1 & r_2 & r_3
\end{bmatrix}
\begin{bmatrix}
1 & -1 & 0 \\
-1 & 2 & -1 \\
0 & -1 & 1
\end{bmatrix}
\begin{bmatrix}
r_1 \\ r_2 \\ r_3
\end{bmatrix}
\end{align}

This symmetric matrix, let's call it M
\begin{align}\label{eqn:desaiCh9:704}
M=
\begin{bmatrix}
1 & -1 & 0 \\
-1 & 2 & -1 \\
0 & -1 & 1
\end{bmatrix}
\end{align}

has eigenvalues $0,1,3$, with orthonormal eigenvectors
\begin{align}\label{eqn:desaiCh9:705}
e_0 &=
\inv{\sqrt{3}}
\begin{bmatrix}
1 \\
1 \\
1
\end{bmatrix} \\
e_1 &=
\inv{\sqrt{2}}
\begin{bmatrix}
1 \\
0 \\
-1
\end{bmatrix} \\
e_3 &=
\inv{\sqrt{6}}
\begin{bmatrix}
1 \\
-2 \\
1
\end{bmatrix}
\end{align}

Writing

\begin{align}\label{eqn:desaiCh9:706}
U = [e_0 e_1 e_3]
=
\begin{bmatrix}
\inv{\sqrt{3}} & \inv{\sqrt{2}}  & \inv{\sqrt{6}}  \\
\inv{\sqrt{3}} & 0  & -\frac{2}{\sqrt{6}}  \\
\inv{\sqrt{3}} & -\inv{\sqrt{2}}  & \inv{\sqrt{6}}
\end{bmatrix}
\end{align}

\begin{align}\label{eqn:desaiCh9:707}
M = U
\begin{bmatrix}
0 & 0 & 0 \\
0 & 1 & 0 \\
0 & 0 & 3
\end{bmatrix}
\tilde{U}
=
U D \tilde{U}
\end{align}

Writing $R' = \tilde{U} R$, and $\spacegrad' = \tilde{U} \spacegrad$, we see that the Laplacian has no mixed partial terms after transformation

\begin{align*}
\spacegrad' \cdot \spacegrad' 
&= 
(\tilde{U} \spacegrad)^{\tilde{}} \tilde{U} \spacegrad \\
&= 
\tilde{\spacegrad } \spacegrad \\
&= 
\spacegrad \cdot \spacegrad
\end{align*}

Schr\"{o}dinger's equation is then just
\begin{align}\label{eqn:desaiCh9:708}
\left( -\frac{\hbar^2}{2m} {\spacegrad'}^2 + \frac{K}{2} \tilde{R'} D R' \right) u = E u
\end{align}

Or
\begin{align}\label{eqn:desaiCh9:708b}
-\frac{\hbar^2}{2m} \frac{\partial^2 u}{\partial {r_1'}^2}
-\frac{\hbar^2}{2m} \frac{\partial^2 u}{\partial {r_2'}^2}
-\frac{\hbar^2}{2m} \frac{\partial^2 u}{\partial {r_3'}^2}
+ \frac{K}{2}
\left(
{r_2'}^2
+3 {r_3'}^2
\right) u
= E u
\end{align}

Separation of variables provides us with one free particle wave equation, and two harmonic oscillator equations

\begin{align}\label{eqn:desaiCh9:708c}
-\frac{\hbar^2}{2m} \frac{\partial^2 u_1}{\partial {r_1'}^2} &= E_1 u_1 \\
-\frac{\hbar^2}{2m} \frac{\partial^2 u}{\partial {r_2'}^2} + \frac{K}{2} {r_2'}^2 u_2 &= E_2 u_2 \\
-\frac{\hbar^2}{2m} \frac{\partial^2 u}{\partial {r_3'}^2} + \frac{3 K}{2} {r_3'}^2 u_3 &= E_3 u_3
\end{align}

We can borrow the Harmonic oscillator energy eigenvalues from problem 4 again with $K_1 = K$, and $K_2 = 3 K$.

\subsection{Problem 8.}
\subsubsection{Statement.}

As a variation of Problem 7 assume that the middle particle at $x_2$ has a different mass $M$.  Reduce this problem to the form of Problem 7 by a scale change in $x_2$ and then use the matrix method to obtain the energy eigenvalues.

\subsubsection{Solution.}

We write $\sqrt{M} x_2 = \sqrt{m} x_2', x_1 + a = x_1', x_3 - a = x_3'$, and then  Schr\"{o}dinger's equation takes the form

\begin{align}\label{eqn:desaiCh9:800}
\left( -\frac{\hbar^2}{2m} {\spacegrad'}^2 + V(X') \right) u &= E u 
\end{align}
\begin{align}\label{eqn:desaiCh9:801}
V(X') = \frac{K}{2} 
\left( 
\left( \sqrt{\frac{m}{M}} x_2' - x_1' 
\right)^2
+\left( -\sqrt{\frac{m}{M}} x_2' + x_3' 
\right)^2
\right)
\end{align}

With $\mu = \sqrt{m/M}$, we have
\begin{align}\label{eqn:desaiCh9:802}
V(X') = \frac{K}{2} 
\tilde{X'}
\begin{bmatrix}
1 & -\mu & 0 \\
-\mu & 2 \mu^2 & -\mu \\
0 & -\mu & 1
\end{bmatrix}
X'
\end{align}

We find that this symmetric matrix has eigenvalues $0, 1, 1 + 2 \mu^2$, and eigenvectors

\begin{align}\label{eqn:desaiCh9:803}
e_0 &=
\inv{\sqrt{1 + 2 \mu^2}}
\begin{bmatrix}
\mu \\ 1 \\ \mu
\end{bmatrix} \\
e_1 &=
\inv{\sqrt{2}}
\begin{bmatrix}
1 \\ 0 \\ -1
\end{bmatrix} \\
e_{1+ 2 \mu^2} &=
\inv{\sqrt{2 + 4 \mu^2}}
\begin{bmatrix}
1 \\ -2 \mu \\ 1
\end{bmatrix} 
\end{align}

The rest of the problem is now no different than the tail end of Problem 7, and we end up with $K_1 = K$, $K_2 = (1 + 2 \mu^2) K$.

\EndArticle
