\makeproblem{Quantum electric dipole}{basicStatMech:problemSet5:3}{ 
%(3 points)
A quantum electric dipole at a fixed space point has its energy determined by two parts - a part which comes from its angular motion and a part coming from its interaction with an applied electric field $\EE$. This leads to a quantum Hamiltonian

\begin{equation}\label{eqn:basicStatMechProblemSet5Problem3:20}
H = \frac{\BL \cdot \BL}{2 I} - \mu \mathcal{E} L_z
\end{equation}

where $I$ is the moment of inertia, and we have assumed an electric field $\EE = \mathcal{E} \zcap$.  This Hamiltonian has eigenstates described by spherical harmonics $Y_{l, m}(\theta, \phi)$, with $m$ taking on $2l+1$ possible integral values, $m = -l, -l + 1, \cdots, l -1, l$.  The corresponding eigenvalues are

\begin{equation}\label{eqn:basicStatMechProblemSet5Problem3:40}
\lambda_{l, m} = \frac{l(l+1) \hbar^2}{2I} - \mu \mathcal{E} m
\end{equation}

(Recall that $l$ is the total angular momentum eigenvalue, while $m$ is the eigenvalue corresponding to $L_z$.) 

\makesubproblem{}{pr:basicStatMechProblemSet5Problem3:a}
Schematically sketch these eigenvalues as a function of $\mathcal{E}$ for $l = 0,1,2$.  

\makesubproblem{}{pr:basicStatMechProblemSet5Problem3:b}
Find the quantum partition function, assuming only $l = 0$ and $l = 1$ contribute to the sum.  

\makesubproblem{}{pr:basicStatMechProblemSet5Problem3:c}
Using this partition function, find the average dipole moment $\mu \expectation{L_z}$ as a function of the electric field and temperature for small electric fields, commenting on its behavior at very high temperature and very low temperature.  

\makesubproblem{}{pr:basicStatMechProblemSet5Problem3:d}
Estimate the temperature above which discarding higher angular momentum states, with $l \ge 2$, is not a good approximation.
} % makeproblem

\makeanswer{basicStatMech:problemSet5:3}{ 
\makeSubAnswer{}{pr:basicStatMechProblemSet5Problem3:a}
TODO.
\makeSubAnswer{}{pr:basicStatMechProblemSet5Problem3:b}
TODO.
\makeSubAnswer{}{pr:basicStatMechProblemSet5Problem3:c}
TODO.
\makeSubAnswer{}{pr:basicStatMechProblemSet5Problem3:d}
TODO.
}
