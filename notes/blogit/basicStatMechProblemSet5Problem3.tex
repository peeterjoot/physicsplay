\makeproblem{Quantum electric dipole}{basicStatMech:problemSet5:3}{ 
%(3 points)
A quantum electric dipole at a fixed space point has its energy determined by two parts - a part which comes from its angular motion and a part coming from its interaction with an applied electric field $\mathcal{E}$. This leads to a quantum Hamiltonian

\begin{equation}\label{eqn:basicStatMechProblemSet5Problem3:20}
H = \frac{\BL \cdot \BL}{2 I} - \mu \mathcal{E} L_z
\end{equation}

where $I$ is the moment of inertia, and we have assumed an electric field $\mathcal{E} = \mathcal{E} \zcap$.  This Hamiltonian has eigenstates described by spherical harmonics $Y_{l, m}(\theta, \phi)$, with $m$ taking on $2l+1$ possible integral values, $m = -l, -l + 1, \cdots, l -1, l$.  The corresponding eigenvalues are

\begin{equation}\label{eqn:basicStatMechProblemSet5Problem3:40}
\lambda_{l, m} = \frac{l(l+1) \hbar^2}{2I} - \mu \mathcal{E} m \hbar
\end{equation}

(Recall that $l$ is the total angular momentum eigenvalue, while $m$ is the eigenvalue corresponding to $L_z$.) 

\makesubproblem{}{pr:basicStatMechProblemSet5Problem3:a}
Schematically sketch these eigenvalues as a function of $\mathcal{E}$ for $l = 0,1,2$.  

\makesubproblem{}{pr:basicStatMechProblemSet5Problem3:b}
Find the quantum partition function, assuming only $l = 0$ and $l = 1$ contribute to the sum.  

\makesubproblem{}{pr:basicStatMechProblemSet5Problem3:c}
Using this partition function, find the average dipole moment $\mu \expectation{L_z}$ as a function of the electric field and temperature for small electric fields, commenting on its behavior at very high temperature and very low temperature.  

\makesubproblem{}{pr:basicStatMechProblemSet5Problem3:d}
Estimate the temperature above which discarding higher angular momentum states, with $l \ge 2$, is not a good approximation.
} % makeproblem

\makeanswer{basicStatMech:problemSet5:3}{ 
\makeSubAnswer{Sketch the energy eigenvalues}{pr:basicStatMechProblemSet5Problem3:a}

Let's summarize the values of the energy eigenvalues $\lambda_{l,m}$ for $l = 0, 1, 2$ before attempting to plot them.

\paragraph{$l = 0$}

For $l = 0$, the azimuthal quantum number can only take the value $m = 0$, so we have

\begin{equation}\label{eqn:basicStatMechProblemSet5Problem3:60}
\lambda_{0,0} = 0.
\end{equation}

\paragraph{$l = 1$}

For $l = 1$ we have

\begin{equation}\label{eqn:basicStatMechProblemSet5Problem3:80}
\frac{l(l+1)}{2} = 1(2)/2 = 1,
\end{equation}

so we have

\begin{subequations}
\begin{equation}\label{eqn:basicStatMechProblemSet5Problem3:100}
\lambda_{1,0} = \frac{\hbar^2}{I} 
\end{equation}
\begin{equation}\label{eqn:basicStatMechProblemSet5Problem3:120}
\lambda_{1,\pm 1} = \frac{\hbar^2}{I} \mp \mu \mathcal{E} \hbar.
\end{equation}
\end{subequations}

\paragraph{$l = 2$}

For $l = 2$ we have

\begin{equation}\label{eqn:basicStatMechProblemSet5Problem3:140}
\frac{l(l+1)}{2} = 2(3)/2 = 3,
\end{equation}

so we have

\begin{subequations}
\begin{equation}\label{eqn:basicStatMechProblemSet5Problem3:160}
\lambda_{2,0} = \frac{3 \hbar^2}{I} 
\end{equation}
\begin{equation}\label{eqn:basicStatMechProblemSet5Problem3:180}
\lambda_{2,\pm 1} = \frac{3 \hbar^2}{I} \mp \mu \mathcal{E} \hbar.
\end{equation}
\begin{equation}\label{eqn:basicStatMechProblemSet5Problem3:200}
\lambda_{2,\pm 2} = \frac{3 \hbar^2}{I} \mp 2 \mu \mathcal{E} \hbar.
\end{equation}
\end{subequations}

These are sketched as a function of $\mathcal{E}$ in \cref{fig:basicStatMechProblemSet5Problem3:basicStatMechProblemSet5Problem3Fig1}.

\imageFigure{basicStatMechProblemSet5Problem3Fig1}{Energy eigenvalues for $l = 0,1, 2$}{fig:basicStatMechProblemSet5Problem3:basicStatMechProblemSet5Problem3Fig1}{0.4}

\makeSubAnswer{Partition function}{pr:basicStatMechProblemSet5Problem3:b}

Our partition function, in general, is

\begin{dmath}\label{eqn:basicStatMechProblemSet5Problem3:220}
Z 
= \sum_{l = 0}^\infty \sum_{m = -l}^l e^{-\lambda_{l,m} \beta}
= \sum_{l = 0}^\infty e^{-l(l+1) \hbar^2 \beta/I} \sum_{m = -l}^l e^{-m \hbar \mu \mathcal{E} \beta}.
\end{dmath}

Dropping all but $l = 0, 1$ this is

\begin{equation}\label{eqn:basicStatMechProblemSet5Problem3:320}
Z 
\approx 
1 + e^{-\hbar^2 \beta/I} 
\lr{ 1 + e^{- \mu \hbar \mathcal{E} \beta } + e^{ \mu \hbar \mathcal{E} \beta}}
\end{equation}

or
\begin{equation}\label{eqn:basicStatMechProblemSet5Problem3:340}
\myBoxed{
Z 
\approx 
1 + e^{-\hbar^2 \beta/I} 
\lr{1 + 2 \cosh( \mu \hbar \mathcal{E} \beta)}.
}
\end{equation}

\makeSubAnswer{Average dipole moment}{pr:basicStatMechProblemSet5Problem3:c}

For the average dipole moment we have

\begin{dmath}\label{eqn:basicStatMechProblemSet5Problem3:260}
Z \expectation{ \mu L_z } 
= \sum_{l = 0}^\infty \sum_{m = -l}^l 
\bra{l m} \mu L_z \ket{l m} e^{-\beta \lambda_{l, m}}
= \sum_{l = 0}^\infty \sum_{m = -l}^l 
\mu \bra{l m} m \hbar \ket{l m} e^{-\beta \lambda_{l, m}}
= \mu \hbar \sum_{l = 0}^\infty e^{-\hbar^2 l (l+1) \beta/2 I}
\sum_{m = -l}^l 
m e^{- \mu m \hbar \mathcal{E} \beta}
= \mu \hbar \sum_{l = 0}^\infty e^{-\hbar^2 l (l+1) \beta/2 I}
\sum_{m = 1}^l 
m 
\lr{ e^{- \mu m \hbar \mathcal{E} \beta} -e^{\mu m \hbar \mathcal{E} \beta} }
= -2 \mu \hbar \sum_{l = 0}^\infty e^{-\hbar^2 l (l+1) \beta/2 I}
\sum_{m = 1}^l m \sinh (\mu m \hbar \mathcal{E} \beta)
\end{dmath}

For the cap of $l = 1$ we have

\begin{equation}\label{eqn:basicStatMechProblemSet5Problem3:280}
\expectation{ \mu L_z } \approx
- \frac{2 \mu \hbar }{Z}
\lr{ 1 (0) + e^{-\hbar^2 \beta/ I} \sinh (\mu \hbar \mathcal{E} \beta) }
\approx
-2 \mu \hbar 
\frac{
e^{-\hbar^2 \beta/ I} \sinh (\mu \hbar \mathcal{E} \beta) 
}
{
1 + e^{-\hbar^2 \beta/I} 
\lr{1 + 2 \cosh( \mu \hbar \mathcal{E} \beta)}
}
\end{equation}

or
\begin{equation}\label{eqn:basicStatMechProblemSet5Problem3:300}
\myBoxed{
\expectation{ \mu L_z } \approx
-2 \mu \hbar 
\frac{
\sinh (\mu \hbar \mathcal{E} \beta) 
}
{
e^{\hbar^2 \beta/I} +
1 + 2 \cosh( \mu \hbar \mathcal{E} \beta)
}.
}
\end{equation}

For high temperatures $\mu \hbar \mathcal{E} \beta \ll 1$ or $k_{\mathrm{B}} T \gg \mu \hbar \mathcal{E}$, we have 

\begin{equation}\label{eqn:basicStatMechProblemSet5Problem3:360}
\expectation{ \mu L_z } \approx - 2 \mu \hbar \frac{0}{1 + 1 + 2} = 0.
\end{equation}

For low temperatures $k_{\mathrm{B}} T \ll \mu \hbar \mathcal{E}$, we have

TODO.

\makeSubAnswer{Approximation validation}{pr:basicStatMechProblemSet5Problem3:d}
TODO.
}
