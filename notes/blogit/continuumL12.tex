%
% Copyright � 2015 Peeter Joot.  All Rights Reserved.
% Licenced as described in the file LICENSE under the root directory of this GIT repository.
%
\documentclass[]{eliblog}

\usepackage{amsmath}
\usepackage{mathpazo}

%
% shorthand for bold symbols, convenient for vectors and matrices
%
\newcommand{\Ba}[0]{\mathbf{a}}
\newcommand{\Bb}[0]{\mathbf{b}}
\newcommand{\Bc}[0]{\mathbf{c}}
\newcommand{\Bd}[0]{\mathbf{d}}
\newcommand{\Be}[0]{\mathbf{e}}
\newcommand{\Bf}[0]{\mathbf{f}}
\newcommand{\Bg}[0]{\mathbf{g}}
\newcommand{\Bh}[0]{\mathbf{h}}
\newcommand{\Bi}[0]{\mathbf{i}}
\newcommand{\Bj}[0]{\mathbf{j}}
\newcommand{\Bk}[0]{\mathbf{k}}
\newcommand{\Bl}[0]{\mathbf{l}}
\newcommand{\Bm}[0]{\mathbf{m}}
\newcommand{\Bn}[0]{\mathbf{n}}
\newcommand{\Bo}[0]{\mathbf{o}}
\newcommand{\Bp}[0]{\mathbf{p}}
\newcommand{\Bq}[0]{\mathbf{q}}
\newcommand{\Br}[0]{\mathbf{r}}
\newcommand{\Bs}[0]{\mathbf{s}}
\newcommand{\Bt}[0]{\mathbf{t}}
\newcommand{\Bu}[0]{\mathbf{u}}
\newcommand{\Bv}[0]{\mathbf{v}}
\newcommand{\Bw}[0]{\mathbf{w}}
\newcommand{\Bx}[0]{\mathbf{x}}
\newcommand{\By}[0]{\mathbf{y}}
\newcommand{\Bz}[0]{\mathbf{z}}
\newcommand{\BA}[0]{\mathbf{A}}
\newcommand{\BB}[0]{\mathbf{B}}
\newcommand{\BC}[0]{\mathbf{C}}
\newcommand{\BD}[0]{\mathbf{D}}
\newcommand{\BE}[0]{\mathbf{E}}
\newcommand{\BF}[0]{\mathbf{F}}
\newcommand{\BG}[0]{\mathbf{G}}
\newcommand{\BH}[0]{\mathbf{H}}
\newcommand{\BI}[0]{\mathbf{I}}
\newcommand{\BJ}[0]{\mathbf{J}}
\newcommand{\BK}[0]{\mathbf{K}}
\newcommand{\BL}[0]{\mathbf{L}}
\newcommand{\BM}[0]{\mathbf{M}}
\newcommand{\BN}[0]{\mathbf{N}}
\newcommand{\BO}[0]{\mathbf{O}}
\newcommand{\BP}[0]{\mathbf{P}}
\newcommand{\BQ}[0]{\mathbf{Q}}
\newcommand{\BR}[0]{\mathbf{R}}
\newcommand{\BS}[0]{\mathbf{S}}
\newcommand{\BT}[0]{\mathbf{T}}
\newcommand{\BU}[0]{\mathbf{U}}
\newcommand{\BV}[0]{\mathbf{V}}
\newcommand{\BW}[0]{\mathbf{W}}
\newcommand{\BX}[0]{\mathbf{X}}
\newcommand{\BY}[0]{\mathbf{Y}}
\newcommand{\BZ}[0]{\mathbf{Z}}

\newcommand{\Bzero}[0]{\mathbf{0}}
\newcommand{\Btheta}[0]{\boldsymbol{\theta}}
\newcommand{\Btau}[0]{\boldsymbol{\tau}}
\newcommand{\Bomega}[0]{\boldsymbol{\omega}}

%
% shorthand for unit vectors
%
\newcommand{\acap}[0]{\hat{\Ba}}
\newcommand{\bcap}[0]{\hat{\Bb}}
\newcommand{\ccap}[0]{\hat{\Bc}}
\newcommand{\dcap}[0]{\hat{\Bd}}
\newcommand{\ecap}[0]{\hat{\Be}}
\newcommand{\fcap}[0]{\hat{\Bf}}
\newcommand{\gcap}[0]{\hat{\Bg}}
\newcommand{\hcap}[0]{\hat{\Bh}}
\newcommand{\icap}[0]{\hat{\Bi}}
\newcommand{\jcap}[0]{\hat{\Bj}}
\newcommand{\kcap}[0]{\hat{\Bk}}
\newcommand{\lcap}[0]{\hat{\Bl}}
\newcommand{\mcap}[0]{\hat{\Bm}}
\newcommand{\ncap}[0]{\hat{\Bn}}
\newcommand{\ocap}[0]{\hat{\Bo}}
\newcommand{\pcap}[0]{\hat{\Bp}}
\newcommand{\qcap}[0]{\hat{\Bq}}
\newcommand{\rcap}[0]{\hat{\Br}}
\newcommand{\scap}[0]{\hat{\Bs}}
\newcommand{\tcap}[0]{\hat{\Bt}}
\newcommand{\ucap}[0]{\hat{\Bu}}
\newcommand{\vcap}[0]{\hat{\Bv}}
\newcommand{\wcap}[0]{\hat{\Bw}}
\newcommand{\xcap}[0]{\hat{\Bx}}
\newcommand{\ycap}[0]{\hat{\By}}
\newcommand{\zcap}[0]{\hat{\Bz}}
\newcommand{\thetacap}[0]{\hat{\Btheta}}

%
% to write R^n and C^n in a distinguishable fashion.  Perhaps change this
% to the double lined characters upon figuring out how to do so.
%
\newcommand{\C}[1]{$\mathbb{C}^{#1}$}
\newcommand{\R}[1]{$\mathbb{R}^{#1}$}

%
% various generally useful helpers
%

% derivative of #1 wrt. #2:
\newcommand{\D}[2] {\frac {d#2} {d#1}}

\newcommand{\inv}[1]{\frac{1}{#1}}
\newcommand{\cross}[0]{\times}

\newcommand{\abs}[1]{\lvert{#1}\rvert}
\newcommand{\norm}[1]{\lVert{#1}\rVert}
\newcommand{\innerprod}[2]{\langle{#1}, {#2}\rangle}
\newcommand{\dotprod}[2]{{#1} \cdot {#2}}
\newcommand{\bdotprod}[2]{\left({#1} \cdot {#2}\right)}
\newcommand{\crossprod}[2]{{#1} \cross {#2}}
\newcommand{\tripleprod}[3]{\dotprod{\left(\crossprod{#1}{#2}\right)}{#3}}

\DeclareMathOperator{\Proj}{Proj}
\DeclareMathOperator{\Span}{span}
\DeclareMathOperator{\Sgn}{sgn}
\DeclareMathOperator{\Area}{Area}
\DeclareMathOperator{\Volume}{Volume}

%
% A few miscellaneous things specific to this document
%
\newcommand{\crossop}[1]{\crossprod{#1}{}}

% R2 vector.
\newcommand{\VectorTwo}[2]{
\begin{bmatrix}
 {#1} \\
 {#2}
\end{bmatrix}
}

\newcommand{\VectorN}[1]{
\begin{bmatrix}
{#1}_1 \\
{#1}_2 \\
\vdots \\
{#1}_N \\
\end{bmatrix}
}

\newcommand{\DETuvij}[4]{
\begin{vmatrix}
 {#1}_{#3} & {#1}_{#4} \\
 {#2}_{#3} & {#2}_{#4}
\end{vmatrix}
}

\newcommand{\DETuvwijk}[6]{
\begin{vmatrix}
 {#1}_{#4} & {#1}_{#5} & {#1}_{#6} \\
 {#2}_{#4} & {#2}_{#5} & {#2}_{#6} \\
 {#3}_{#4} & {#3}_{#5} & {#3}_{#6}
\end{vmatrix}
}

\newcommand{\DETuvwxijkl}[8]{
\begin{vmatrix}
 {#1}_{#5} & {#1}_{#6} & {#1}_{#7} & {#1}_{#8} \\
 {#2}_{#5} & {#2}_{#6} & {#2}_{#7} & {#2}_{#8} \\
 {#3}_{#5} & {#3}_{#6} & {#3}_{#7} & {#3}_{#8} \\
 {#4}_{#5} & {#4}_{#6} & {#4}_{#7} & {#4}_{#8} \\
\end{vmatrix}
}

%\newcommand{\DETuvwxyijklm}[10]{
%\begin{vmatrix}
% {#1}_{#6} & {#1}_{#7} & {#1}_{#8} & {#1}_{#9} & {#1}_{#10} \\
% {#2}_{#6} & {#2}_{#7} & {#2}_{#8} & {#2}_{#9} & {#2}_{#10} \\
% {#3}_{#6} & {#3}_{#7} & {#3}_{#8} & {#3}_{#9} & {#3}_{#10} \\
% {#4}_{#6} & {#4}_{#7} & {#4}_{#8} & {#4}_{#9} & {#4}_{#10} \\
% {#5}_{#6} & {#5}_{#7} & {#5}_{#8} & {#5}_{#9} & {#5}_{#10}
%\end{vmatrix}
%}

% R3 vector.
\newcommand{\VectorThree}[3]{
\begin{bmatrix}
 {#1} \\
 {#2} \\
 {#3}
\end{bmatrix}
}



\author{Peeter Joot}
\email{peeter.joot@gmail.com}

%\documentclass[]{eliblogwidescreen}

\usepackage{amsmath}
\usepackage{mathpazo}

%
% shorthand for bold symbols, convenient for vectors and matrices
%
\newcommand{\Ba}[0]{\mathbf{a}}
\newcommand{\Bb}[0]{\mathbf{b}}
\newcommand{\Bc}[0]{\mathbf{c}}
\newcommand{\Bd}[0]{\mathbf{d}}
\newcommand{\Be}[0]{\mathbf{e}}
\newcommand{\Bf}[0]{\mathbf{f}}
\newcommand{\Bg}[0]{\mathbf{g}}
\newcommand{\Bh}[0]{\mathbf{h}}
\newcommand{\Bi}[0]{\mathbf{i}}
\newcommand{\Bj}[0]{\mathbf{j}}
\newcommand{\Bk}[0]{\mathbf{k}}
\newcommand{\Bl}[0]{\mathbf{l}}
\newcommand{\Bm}[0]{\mathbf{m}}
\newcommand{\Bn}[0]{\mathbf{n}}
\newcommand{\Bo}[0]{\mathbf{o}}
\newcommand{\Bp}[0]{\mathbf{p}}
\newcommand{\Bq}[0]{\mathbf{q}}
\newcommand{\Br}[0]{\mathbf{r}}
\newcommand{\Bs}[0]{\mathbf{s}}
\newcommand{\Bt}[0]{\mathbf{t}}
\newcommand{\Bu}[0]{\mathbf{u}}
\newcommand{\Bv}[0]{\mathbf{v}}
\newcommand{\Bw}[0]{\mathbf{w}}
\newcommand{\Bx}[0]{\mathbf{x}}
\newcommand{\By}[0]{\mathbf{y}}
\newcommand{\Bz}[0]{\mathbf{z}}
\newcommand{\BA}[0]{\mathbf{A}}
\newcommand{\BB}[0]{\mathbf{B}}
\newcommand{\BC}[0]{\mathbf{C}}
\newcommand{\BD}[0]{\mathbf{D}}
\newcommand{\BE}[0]{\mathbf{E}}
\newcommand{\BF}[0]{\mathbf{F}}
\newcommand{\BG}[0]{\mathbf{G}}
\newcommand{\BH}[0]{\mathbf{H}}
\newcommand{\BI}[0]{\mathbf{I}}
\newcommand{\BJ}[0]{\mathbf{J}}
\newcommand{\BK}[0]{\mathbf{K}}
\newcommand{\BL}[0]{\mathbf{L}}
\newcommand{\BM}[0]{\mathbf{M}}
\newcommand{\BN}[0]{\mathbf{N}}
\newcommand{\BO}[0]{\mathbf{O}}
\newcommand{\BP}[0]{\mathbf{P}}
\newcommand{\BQ}[0]{\mathbf{Q}}
\newcommand{\BR}[0]{\mathbf{R}}
\newcommand{\BS}[0]{\mathbf{S}}
\newcommand{\BT}[0]{\mathbf{T}}
\newcommand{\BU}[0]{\mathbf{U}}
\newcommand{\BV}[0]{\mathbf{V}}
\newcommand{\BW}[0]{\mathbf{W}}
\newcommand{\BX}[0]{\mathbf{X}}
\newcommand{\BY}[0]{\mathbf{Y}}
\newcommand{\BZ}[0]{\mathbf{Z}}

\newcommand{\Bzero}[0]{\mathbf{0}}
\newcommand{\Btheta}[0]{\boldsymbol{\theta}}
\newcommand{\Btau}[0]{\boldsymbol{\tau}}
\newcommand{\Bomega}[0]{\boldsymbol{\omega}}

%
% shorthand for unit vectors
%
\newcommand{\acap}[0]{\hat{\Ba}}
\newcommand{\bcap}[0]{\hat{\Bb}}
\newcommand{\ccap}[0]{\hat{\Bc}}
\newcommand{\dcap}[0]{\hat{\Bd}}
\newcommand{\ecap}[0]{\hat{\Be}}
\newcommand{\fcap}[0]{\hat{\Bf}}
\newcommand{\gcap}[0]{\hat{\Bg}}
\newcommand{\hcap}[0]{\hat{\Bh}}
\newcommand{\icap}[0]{\hat{\Bi}}
\newcommand{\jcap}[0]{\hat{\Bj}}
\newcommand{\kcap}[0]{\hat{\Bk}}
\newcommand{\lcap}[0]{\hat{\Bl}}
\newcommand{\mcap}[0]{\hat{\Bm}}
\newcommand{\ncap}[0]{\hat{\Bn}}
\newcommand{\ocap}[0]{\hat{\Bo}}
\newcommand{\pcap}[0]{\hat{\Bp}}
\newcommand{\qcap}[0]{\hat{\Bq}}
\newcommand{\rcap}[0]{\hat{\Br}}
\newcommand{\scap}[0]{\hat{\Bs}}
\newcommand{\tcap}[0]{\hat{\Bt}}
\newcommand{\ucap}[0]{\hat{\Bu}}
\newcommand{\vcap}[0]{\hat{\Bv}}
\newcommand{\wcap}[0]{\hat{\Bw}}
\newcommand{\xcap}[0]{\hat{\Bx}}
\newcommand{\ycap}[0]{\hat{\By}}
\newcommand{\zcap}[0]{\hat{\Bz}}
\newcommand{\thetacap}[0]{\hat{\Btheta}}

%
% to write R^n and C^n in a distinguishable fashion.  Perhaps change this
% to the double lined characters upon figuring out how to do so.
%
\newcommand{\C}[1]{$\mathbb{C}^{#1}$}
\newcommand{\R}[1]{$\mathbb{R}^{#1}$}

%
% various generally useful helpers
%

% derivative of #1 wrt. #2:
\newcommand{\D}[2] {\frac {d#2} {d#1}}

\newcommand{\inv}[1]{\frac{1}{#1}}
\newcommand{\cross}[0]{\times}

\newcommand{\abs}[1]{\lvert{#1}\rvert}
\newcommand{\norm}[1]{\lVert{#1}\rVert}
\newcommand{\innerprod}[2]{\langle{#1}, {#2}\rangle}
\newcommand{\dotprod}[2]{{#1} \cdot {#2}}
\newcommand{\bdotprod}[2]{\left({#1} \cdot {#2}\right)}
\newcommand{\crossprod}[2]{{#1} \cross {#2}}
\newcommand{\tripleprod}[3]{\dotprod{\left(\crossprod{#1}{#2}\right)}{#3}}

\DeclareMathOperator{\Proj}{Proj}
\DeclareMathOperator{\Span}{span}
\DeclareMathOperator{\Sgn}{sgn}
\DeclareMathOperator{\Area}{Area}
\DeclareMathOperator{\Volume}{Volume}

%
% A few miscellaneous things specific to this document
%
\newcommand{\crossop}[1]{\crossprod{#1}{}}

% R2 vector.
\newcommand{\VectorTwo}[2]{
\begin{bmatrix}
 {#1} \\
 {#2}
\end{bmatrix}
}

\newcommand{\VectorN}[1]{
\begin{bmatrix}
{#1}_1 \\
{#1}_2 \\
\vdots \\
{#1}_N \\
\end{bmatrix}
}

\newcommand{\DETuvij}[4]{
\begin{vmatrix}
 {#1}_{#3} & {#1}_{#4} \\
 {#2}_{#3} & {#2}_{#4}
\end{vmatrix}
}

\newcommand{\DETuvwijk}[6]{
\begin{vmatrix}
 {#1}_{#4} & {#1}_{#5} & {#1}_{#6} \\
 {#2}_{#4} & {#2}_{#5} & {#2}_{#6} \\
 {#3}_{#4} & {#3}_{#5} & {#3}_{#6}
\end{vmatrix}
}

\newcommand{\DETuvwxijkl}[8]{
\begin{vmatrix}
 {#1}_{#5} & {#1}_{#6} & {#1}_{#7} & {#1}_{#8} \\
 {#2}_{#5} & {#2}_{#6} & {#2}_{#7} & {#2}_{#8} \\
 {#3}_{#5} & {#3}_{#6} & {#3}_{#7} & {#3}_{#8} \\
 {#4}_{#5} & {#4}_{#6} & {#4}_{#7} & {#4}_{#8} \\
\end{vmatrix}
}

%\newcommand{\DETuvwxyijklm}[10]{
%\begin{vmatrix}
% {#1}_{#6} & {#1}_{#7} & {#1}_{#8} & {#1}_{#9} & {#1}_{#10} \\
% {#2}_{#6} & {#2}_{#7} & {#2}_{#8} & {#2}_{#9} & {#2}_{#10} \\
% {#3}_{#6} & {#3}_{#7} & {#3}_{#8} & {#3}_{#9} & {#3}_{#10} \\
% {#4}_{#6} & {#4}_{#7} & {#4}_{#8} & {#4}_{#9} & {#4}_{#10} \\
% {#5}_{#6} & {#5}_{#7} & {#5}_{#8} & {#5}_{#9} & {#5}_{#10}
%\end{vmatrix}
%}

% R3 vector.
\newcommand{\VectorThree}[3]{
\begin{bmatrix}
 {#1} \\
 {#2} \\
 {#3}
\end{bmatrix}
}



\author{Peeter Joot}
\email{peeter.joot@gmail.com}


\chapter{PHY454H1S\\Continuum Mechanics.  Lecture 12: Flow in a pipe.  Taught by Prof. K. Das.}
\label{chap:continuumL12}
%\useCCL
\blogpage{http://sites.google.com/site/peeterjoot2/math2012/continuumL12.pdf}
\date{Feb 12, 2012}
\gitRevisionInfo{continuumL12}

\beginArtWithToc
%\beginArtNoToc

\section{Disclaimer.}

Peeter's lecture notes from class.  May not be entirely coherent.

\section{Review.  Steady rectilinear flow.}

Steady:

\begin{equation}\label{eqn:continuumL12:10}
\PD{t}{} = 0
\end{equation}

Rectilinear is a unidirectional flow such as

\begin{equation}\label{eqn:continuumL12:30}
\Bu = \xcap u( x, y, z ),
\end{equation}

\begin{enumerate}
\item 
From continuity we have $\spacegrad \cdot \Bu = 0$, so for this case we have

\begin{equation*}
\PD{x}{u} = 0
\end{equation*}

FIXME: recall that we related $d\rho/dt = 0$ and $\spacegrad \cdot \Bu = 0$

or 

\begin{equation*}
u = u(y, z)
\end{equation*}

\item Nonlinear term is zero.  $(\Bu \cdot \spacegrad) \Bu = 0$
\item $p = p(x)$.  $\frac{d^2 p}{dx^2} = 0 \implies \frac{dp}{dx} = -G$

\item $\mu \left( \PDSq{y}{u} + \PDSq{z}{u} \right) = G$

\end{enumerate}

\section{Some handwaving examples.}

FIXME: F1, F2, F3

(simple shear flow, channel flow, combination of the two: flow over plate).

Example 2.  Fluid in a container.  If the surface tension is altered on one side, we induce a flow on the surface, leading to a circulation flow.  This can be done for example, by introducing a heat source or addition of surfactant.

This is illustrated in figure

FIXME: F4.

This sort of flow (hard to analyize, first done by Steve Davis (1980's).

The point here is that we can use some level of intuition to guide our attempts at solution.

\section{Flow down a pipe.}

We use a cylindrical coordinate $(r, \theta, z)$ convention for this problem as in

FIXME: F5

\begin{equation}\label{eqn:continuumL12:50}
\Bu = \zcap w(r)
\end{equation}

\begin{equation}\label{eqn:continuumL12:70}
\inv{r} \frac{d}{dr} \left( r \frac{dw}{dr} \right) = - \frac{G}{\mu}
\end{equation}

\begin{equation}\label{eqn:continuumL12:90}
r \frac{dw}{dr} = - \frac{G r^2}{2\mu} + A
\end{equation}

\begin{equation}\label{eqn:continuumL12:110}
w = -\frac{G r^2}{4 \mu} + A \ln(r) + B
\end{equation}

Requiring finite solutions for $r = 0$ means that we must have $A = 0$.  Also $w(a) = 0$, we have $B = G a^2/4 \mu$ so we must have

\begin{equation}\label{eqn:continuumL12:130}
w(r) = \frac{G}{4 \mu}( a^2 - r^2 )
\end{equation}

\section{Example: Gravity driven flow of a liquid film}

Coordinates as in

FIXME: F6

\begin{equation}\label{eqn:continuumL12:150}
\Bu = \xcap u(y)
\end{equation}

Boundary conditions

\begin{enumerate}
\item $u(y = 0) = 0$
\item Tangential stress at the air-liquid interface $y = h$ is equal.

\begin{equation}\label{eqn:continuumL12:170}
\Btau \cdot (\Bsigma_l \cdot \ncap) = \Btau \cdot (\Bsigma_a \cdot \ncap),
\end{equation}
\end{enumerate}

We write 

\begin{align}\label{eqn:continuumL12:190}
\Btau &= 
\begin{bmatrix}
1 \\
0 \\
0
\end{bmatrix} \\
\ncap &= 
\begin{bmatrix}
0 \\
1 \\
0
\end{bmatrix} \\
\end{align}

\begin{align}\label{eqn:continuumL12:210}
\sigma_{ij}^l 
&= - p \delta_{ij} + \mu^l \left( 
\PD{x_j}{u_i} +
\PD{x_i}{u_j}
\right) \\
\sigma_{ij}^a 
&= - p \delta_{ij} + \mu^a \left( 
\PD{x_j}{u_i} +
\PD{x_i}{u_j}
\right) \\
\end{align}

We expect that the flow of liquid will induce a flow of air at the interface, but may be able to make a one-sided approximation.  Let's see how far we get before we have to introduce any approximations and compute the traction vector for the liquid

\begin{align*}
\Bsigma^l \cdot \ncap = 
\begin{bmatrix}
-p & \mu^l \PDi{y}{u} & 0 \\
\mu^l \PDi{y}{u} & -p & 0 \\
0 & 0 & 0
\end{bmatrix}
\begin{bmatrix}
0 \\
1 \\
0
\end{bmatrix} \\
&=
\begin{bmatrix}
\mu^l \PDi{y}{u} \\
-p \\
0
\end{bmatrix}
\end{align*}

So

\begin{equation}\label{eqn:continuumL12:230}
\Btau \cdot (\Bsigma^l \ncap)
=
\begin{bmatrix}
1 & 0 & 0
\end{bmatrix}
\begin{bmatrix}
\mu^l \PDi{y}{u} \\
-p \\
0
\end{bmatrix}
=
\mu^l \PD{y}{u}
\end{equation}

Our boundary value condition is therefore

\begin{equation}\label{eqn:continuumL12:250}
\evalbar{\mu^l \PD{y}{u^l}}{y = h} =
\evalbar{\mu^a \PD{y}{u^a}}{y = h}
\end{equation}

When can we decouple this, treating only the liquid?  Observe that we have

\begin{equation}\label{eqn:continuumL12:270}
\evalbar{\PD{y}{u^l}}{y = h} =
\evalbar{\frac{\mu^a}{\mu^l} \PD{y}{u^a}}{y = h}
\end{equation}

so if

\begin{equation}\label{eqn:continuumL12:290}
\frac{\mu_a}{\mu_l} \ll 1
\end{equation}

we can treat only the liquid portion of the problem, with a boundary value condition

\begin{equation}\label{eqn:continuumL12:310}
\evalbar{\PD{y}{u^l}}{y = h} = 0.
\end{equation}

Let's look at the component of the traction vector in the direction of the normal (liquid pressure acting on the air)

\begin{equation}\label{eqn:continuumL12:330}
\ncap (\Bsigma^l \ncap) = \ncap (\Bsigma^a \ncap) 
\end{equation}

or

\begin{equation}\label{eqn:continuumL12:350}
\begin{bmatrix}
0 & 1 & 0
\end{bmatrix}
\begin{bmatrix}
\mu^l \PD{y}{u} \\
-p^l \\
0
\end{bmatrix}
= -\evalbar{p^l}{y = h} = -\evalbar{p^a}{y = h}
\end{equation}

i.e. We have pressure matching at the interface.

Our body force is

\begin{equation}\label{eqn:continuumL12:370}
\Bf = 
\begin{bmatrix}
g \sin\alpha \\
-g \cos\alpha \\
0
\end{bmatrix}
\end{equation}

Going back to the NS equation, our only surviving parts are

\begin{align}\label{eqn:continuumL12:390}
0 &= -\PD{x}{p} + \mu \PDSq{y}{u} + g \sin\alpha \\
0 &= -\PD{y}{p} - g \cos\alpha \\
0 &= -\PD{z}{p} 
\end{align}

The last gives us $p \ne p(z)$.  Integrating the second we have

\begin{equation}\label{eqn:continuumL12:410}
p = \rho g y \cos\alpha + p_1
\end{equation}

Since $p = p_{\text{atm}}$ at $y = h$, we have

\begin{equation}\label{eqn:continuumL12:430}
p_{\text{atm}} = \rho g h \cos\alpha + p_1
\end{equation}

Our first NS equation becomes

\begin{equation}\label{eqn:continuumL12:450}
0 = \mu \PDSq{y}{u} + g \sin\alpha,
\end{equation}

or
\begin{equation}\label{eqn:continuumL12:470}
\PDSq{y}{u} = \frac{g}{\mu} \sin\alpha
\end{equation}

Solving we have

\begin{equation}\label{eqn:continuumL12:490}
u = - \rho g \frac{\sin\alpha}{2 \mu} y^2 + A y + B
\end{equation}

With

\begin{align}\label{eqn:continuumL12:510}
u(0) &= 0 \\
\evalbar{\PD{y}{u}}{y = h} &= 0
\end{align}

\begin{equation}\label{eqn:classicalMechanicsPs2:530}
u = \rho g \frac{\sin\alpha}{2 \mu} \left( 2 h y - y^2 \right) 
\end{equation}

FIXME: F7.

It's important to note that in these problems we have to derive our boundary value conditions!  They are not given.

In this discussion, the height $h$ was assumed to be constant, with the tangential direction constant and parallel to the surface that the liquid is flowing on.  It's claimed in class that this is actually a consequence of surface tension only!  That's not at all intuitive, but will be covered when we learn about ``stability conditions''.

\section{Study note.}

Memorizing the NS equation required for midterm, but more complex stuff (like cylindrical forms of the strain tensor if required) will be given.

%FIXME: Reading: \S XX from \cite{acheson1990elementary}

%\EndArticle
\EndNoBibArticle
