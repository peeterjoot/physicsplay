%
% Copyright � 2015 Peeter Joot.  All Rights Reserved.
% Licenced as described in the file LICENSE under the root directory of this GIT repository.
%
\documentclass[]{eliblog}

\usepackage{amsmath}
\usepackage{mathpazo}

%
% shorthand for bold symbols, convenient for vectors and matrices
%
\newcommand{\Ba}[0]{\mathbf{a}}
\newcommand{\Bb}[0]{\mathbf{b}}
\newcommand{\Bc}[0]{\mathbf{c}}
\newcommand{\Bd}[0]{\mathbf{d}}
\newcommand{\Be}[0]{\mathbf{e}}
\newcommand{\Bf}[0]{\mathbf{f}}
\newcommand{\Bg}[0]{\mathbf{g}}
\newcommand{\Bh}[0]{\mathbf{h}}
\newcommand{\Bi}[0]{\mathbf{i}}
\newcommand{\Bj}[0]{\mathbf{j}}
\newcommand{\Bk}[0]{\mathbf{k}}
\newcommand{\Bl}[0]{\mathbf{l}}
\newcommand{\Bm}[0]{\mathbf{m}}
\newcommand{\Bn}[0]{\mathbf{n}}
\newcommand{\Bo}[0]{\mathbf{o}}
\newcommand{\Bp}[0]{\mathbf{p}}
\newcommand{\Bq}[0]{\mathbf{q}}
\newcommand{\Br}[0]{\mathbf{r}}
\newcommand{\Bs}[0]{\mathbf{s}}
\newcommand{\Bt}[0]{\mathbf{t}}
\newcommand{\Bu}[0]{\mathbf{u}}
\newcommand{\Bv}[0]{\mathbf{v}}
\newcommand{\Bw}[0]{\mathbf{w}}
\newcommand{\Bx}[0]{\mathbf{x}}
\newcommand{\By}[0]{\mathbf{y}}
\newcommand{\Bz}[0]{\mathbf{z}}
\newcommand{\BA}[0]{\mathbf{A}}
\newcommand{\BB}[0]{\mathbf{B}}
\newcommand{\BC}[0]{\mathbf{C}}
\newcommand{\BD}[0]{\mathbf{D}}
\newcommand{\BE}[0]{\mathbf{E}}
\newcommand{\BF}[0]{\mathbf{F}}
\newcommand{\BG}[0]{\mathbf{G}}
\newcommand{\BH}[0]{\mathbf{H}}
\newcommand{\BI}[0]{\mathbf{I}}
\newcommand{\BJ}[0]{\mathbf{J}}
\newcommand{\BK}[0]{\mathbf{K}}
\newcommand{\BL}[0]{\mathbf{L}}
\newcommand{\BM}[0]{\mathbf{M}}
\newcommand{\BN}[0]{\mathbf{N}}
\newcommand{\BO}[0]{\mathbf{O}}
\newcommand{\BP}[0]{\mathbf{P}}
\newcommand{\BQ}[0]{\mathbf{Q}}
\newcommand{\BR}[0]{\mathbf{R}}
\newcommand{\BS}[0]{\mathbf{S}}
\newcommand{\BT}[0]{\mathbf{T}}
\newcommand{\BU}[0]{\mathbf{U}}
\newcommand{\BV}[0]{\mathbf{V}}
\newcommand{\BW}[0]{\mathbf{W}}
\newcommand{\BX}[0]{\mathbf{X}}
\newcommand{\BY}[0]{\mathbf{Y}}
\newcommand{\BZ}[0]{\mathbf{Z}}

\newcommand{\Bzero}[0]{\mathbf{0}}
\newcommand{\Btheta}[0]{\boldsymbol{\theta}}
\newcommand{\Btau}[0]{\boldsymbol{\tau}}
\newcommand{\Bomega}[0]{\boldsymbol{\omega}}

%
% shorthand for unit vectors
%
\newcommand{\acap}[0]{\hat{\Ba}}
\newcommand{\bcap}[0]{\hat{\Bb}}
\newcommand{\ccap}[0]{\hat{\Bc}}
\newcommand{\dcap}[0]{\hat{\Bd}}
\newcommand{\ecap}[0]{\hat{\Be}}
\newcommand{\fcap}[0]{\hat{\Bf}}
\newcommand{\gcap}[0]{\hat{\Bg}}
\newcommand{\hcap}[0]{\hat{\Bh}}
\newcommand{\icap}[0]{\hat{\Bi}}
\newcommand{\jcap}[0]{\hat{\Bj}}
\newcommand{\kcap}[0]{\hat{\Bk}}
\newcommand{\lcap}[0]{\hat{\Bl}}
\newcommand{\mcap}[0]{\hat{\Bm}}
\newcommand{\ncap}[0]{\hat{\Bn}}
\newcommand{\ocap}[0]{\hat{\Bo}}
\newcommand{\pcap}[0]{\hat{\Bp}}
\newcommand{\qcap}[0]{\hat{\Bq}}
\newcommand{\rcap}[0]{\hat{\Br}}
\newcommand{\scap}[0]{\hat{\Bs}}
\newcommand{\tcap}[0]{\hat{\Bt}}
\newcommand{\ucap}[0]{\hat{\Bu}}
\newcommand{\vcap}[0]{\hat{\Bv}}
\newcommand{\wcap}[0]{\hat{\Bw}}
\newcommand{\xcap}[0]{\hat{\Bx}}
\newcommand{\ycap}[0]{\hat{\By}}
\newcommand{\zcap}[0]{\hat{\Bz}}
\newcommand{\thetacap}[0]{\hat{\Btheta}}

%
% to write R^n and C^n in a distinguishable fashion.  Perhaps change this
% to the double lined characters upon figuring out how to do so.
%
\newcommand{\C}[1]{$\mathbb{C}^{#1}$}
\newcommand{\R}[1]{$\mathbb{R}^{#1}$}

%
% various generally useful helpers
%

% derivative of #1 wrt. #2:
\newcommand{\D}[2] {\frac {d#2} {d#1}}

\newcommand{\inv}[1]{\frac{1}{#1}}
\newcommand{\cross}[0]{\times}

\newcommand{\abs}[1]{\lvert{#1}\rvert}
\newcommand{\norm}[1]{\lVert{#1}\rVert}
\newcommand{\innerprod}[2]{\langle{#1}, {#2}\rangle}
\newcommand{\dotprod}[2]{{#1} \cdot {#2}}
\newcommand{\bdotprod}[2]{\left({#1} \cdot {#2}\right)}
\newcommand{\crossprod}[2]{{#1} \cross {#2}}
\newcommand{\tripleprod}[3]{\dotprod{\left(\crossprod{#1}{#2}\right)}{#3}}

\DeclareMathOperator{\Proj}{Proj}
\DeclareMathOperator{\Span}{span}
\DeclareMathOperator{\Sgn}{sgn}
\DeclareMathOperator{\Area}{Area}
\DeclareMathOperator{\Volume}{Volume}

%
% A few miscellaneous things specific to this document
%
\newcommand{\crossop}[1]{\crossprod{#1}{}}

% R2 vector.
\newcommand{\VectorTwo}[2]{
\begin{bmatrix}
 {#1} \\
 {#2}
\end{bmatrix}
}

\newcommand{\VectorN}[1]{
\begin{bmatrix}
{#1}_1 \\
{#1}_2 \\
\vdots \\
{#1}_N \\
\end{bmatrix}
}

\newcommand{\DETuvij}[4]{
\begin{vmatrix}
 {#1}_{#3} & {#1}_{#4} \\
 {#2}_{#3} & {#2}_{#4}
\end{vmatrix}
}

\newcommand{\DETuvwijk}[6]{
\begin{vmatrix}
 {#1}_{#4} & {#1}_{#5} & {#1}_{#6} \\
 {#2}_{#4} & {#2}_{#5} & {#2}_{#6} \\
 {#3}_{#4} & {#3}_{#5} & {#3}_{#6}
\end{vmatrix}
}

\newcommand{\DETuvwxijkl}[8]{
\begin{vmatrix}
 {#1}_{#5} & {#1}_{#6} & {#1}_{#7} & {#1}_{#8} \\
 {#2}_{#5} & {#2}_{#6} & {#2}_{#7} & {#2}_{#8} \\
 {#3}_{#5} & {#3}_{#6} & {#3}_{#7} & {#3}_{#8} \\
 {#4}_{#5} & {#4}_{#6} & {#4}_{#7} & {#4}_{#8} \\
\end{vmatrix}
}

%\newcommand{\DETuvwxyijklm}[10]{
%\begin{vmatrix}
% {#1}_{#6} & {#1}_{#7} & {#1}_{#8} & {#1}_{#9} & {#1}_{#10} \\
% {#2}_{#6} & {#2}_{#7} & {#2}_{#8} & {#2}_{#9} & {#2}_{#10} \\
% {#3}_{#6} & {#3}_{#7} & {#3}_{#8} & {#3}_{#9} & {#3}_{#10} \\
% {#4}_{#6} & {#4}_{#7} & {#4}_{#8} & {#4}_{#9} & {#4}_{#10} \\
% {#5}_{#6} & {#5}_{#7} & {#5}_{#8} & {#5}_{#9} & {#5}_{#10}
%\end{vmatrix}
%}

% R3 vector.
\newcommand{\VectorThree}[3]{
\begin{bmatrix}
 {#1} \\
 {#2} \\
 {#3}
\end{bmatrix}
}



\author{Peeter Joot}
\email{peeter.joot@gmail.com}

%\documentclass[]{eliblogwidescreen}

\usepackage{amsmath}
\usepackage{mathpazo}

%
% shorthand for bold symbols, convenient for vectors and matrices
%
\newcommand{\Ba}[0]{\mathbf{a}}
\newcommand{\Bb}[0]{\mathbf{b}}
\newcommand{\Bc}[0]{\mathbf{c}}
\newcommand{\Bd}[0]{\mathbf{d}}
\newcommand{\Be}[0]{\mathbf{e}}
\newcommand{\Bf}[0]{\mathbf{f}}
\newcommand{\Bg}[0]{\mathbf{g}}
\newcommand{\Bh}[0]{\mathbf{h}}
\newcommand{\Bi}[0]{\mathbf{i}}
\newcommand{\Bj}[0]{\mathbf{j}}
\newcommand{\Bk}[0]{\mathbf{k}}
\newcommand{\Bl}[0]{\mathbf{l}}
\newcommand{\Bm}[0]{\mathbf{m}}
\newcommand{\Bn}[0]{\mathbf{n}}
\newcommand{\Bo}[0]{\mathbf{o}}
\newcommand{\Bp}[0]{\mathbf{p}}
\newcommand{\Bq}[0]{\mathbf{q}}
\newcommand{\Br}[0]{\mathbf{r}}
\newcommand{\Bs}[0]{\mathbf{s}}
\newcommand{\Bt}[0]{\mathbf{t}}
\newcommand{\Bu}[0]{\mathbf{u}}
\newcommand{\Bv}[0]{\mathbf{v}}
\newcommand{\Bw}[0]{\mathbf{w}}
\newcommand{\Bx}[0]{\mathbf{x}}
\newcommand{\By}[0]{\mathbf{y}}
\newcommand{\Bz}[0]{\mathbf{z}}
\newcommand{\BA}[0]{\mathbf{A}}
\newcommand{\BB}[0]{\mathbf{B}}
\newcommand{\BC}[0]{\mathbf{C}}
\newcommand{\BD}[0]{\mathbf{D}}
\newcommand{\BE}[0]{\mathbf{E}}
\newcommand{\BF}[0]{\mathbf{F}}
\newcommand{\BG}[0]{\mathbf{G}}
\newcommand{\BH}[0]{\mathbf{H}}
\newcommand{\BI}[0]{\mathbf{I}}
\newcommand{\BJ}[0]{\mathbf{J}}
\newcommand{\BK}[0]{\mathbf{K}}
\newcommand{\BL}[0]{\mathbf{L}}
\newcommand{\BM}[0]{\mathbf{M}}
\newcommand{\BN}[0]{\mathbf{N}}
\newcommand{\BO}[0]{\mathbf{O}}
\newcommand{\BP}[0]{\mathbf{P}}
\newcommand{\BQ}[0]{\mathbf{Q}}
\newcommand{\BR}[0]{\mathbf{R}}
\newcommand{\BS}[0]{\mathbf{S}}
\newcommand{\BT}[0]{\mathbf{T}}
\newcommand{\BU}[0]{\mathbf{U}}
\newcommand{\BV}[0]{\mathbf{V}}
\newcommand{\BW}[0]{\mathbf{W}}
\newcommand{\BX}[0]{\mathbf{X}}
\newcommand{\BY}[0]{\mathbf{Y}}
\newcommand{\BZ}[0]{\mathbf{Z}}

\newcommand{\Bzero}[0]{\mathbf{0}}
\newcommand{\Btheta}[0]{\boldsymbol{\theta}}
\newcommand{\Btau}[0]{\boldsymbol{\tau}}
\newcommand{\Bomega}[0]{\boldsymbol{\omega}}

%
% shorthand for unit vectors
%
\newcommand{\acap}[0]{\hat{\Ba}}
\newcommand{\bcap}[0]{\hat{\Bb}}
\newcommand{\ccap}[0]{\hat{\Bc}}
\newcommand{\dcap}[0]{\hat{\Bd}}
\newcommand{\ecap}[0]{\hat{\Be}}
\newcommand{\fcap}[0]{\hat{\Bf}}
\newcommand{\gcap}[0]{\hat{\Bg}}
\newcommand{\hcap}[0]{\hat{\Bh}}
\newcommand{\icap}[0]{\hat{\Bi}}
\newcommand{\jcap}[0]{\hat{\Bj}}
\newcommand{\kcap}[0]{\hat{\Bk}}
\newcommand{\lcap}[0]{\hat{\Bl}}
\newcommand{\mcap}[0]{\hat{\Bm}}
\newcommand{\ncap}[0]{\hat{\Bn}}
\newcommand{\ocap}[0]{\hat{\Bo}}
\newcommand{\pcap}[0]{\hat{\Bp}}
\newcommand{\qcap}[0]{\hat{\Bq}}
\newcommand{\rcap}[0]{\hat{\Br}}
\newcommand{\scap}[0]{\hat{\Bs}}
\newcommand{\tcap}[0]{\hat{\Bt}}
\newcommand{\ucap}[0]{\hat{\Bu}}
\newcommand{\vcap}[0]{\hat{\Bv}}
\newcommand{\wcap}[0]{\hat{\Bw}}
\newcommand{\xcap}[0]{\hat{\Bx}}
\newcommand{\ycap}[0]{\hat{\By}}
\newcommand{\zcap}[0]{\hat{\Bz}}
\newcommand{\thetacap}[0]{\hat{\Btheta}}

%
% to write R^n and C^n in a distinguishable fashion.  Perhaps change this
% to the double lined characters upon figuring out how to do so.
%
\newcommand{\C}[1]{$\mathbb{C}^{#1}$}
\newcommand{\R}[1]{$\mathbb{R}^{#1}$}

%
% various generally useful helpers
%

% derivative of #1 wrt. #2:
\newcommand{\D}[2] {\frac {d#2} {d#1}}

\newcommand{\inv}[1]{\frac{1}{#1}}
\newcommand{\cross}[0]{\times}

\newcommand{\abs}[1]{\lvert{#1}\rvert}
\newcommand{\norm}[1]{\lVert{#1}\rVert}
\newcommand{\innerprod}[2]{\langle{#1}, {#2}\rangle}
\newcommand{\dotprod}[2]{{#1} \cdot {#2}}
\newcommand{\bdotprod}[2]{\left({#1} \cdot {#2}\right)}
\newcommand{\crossprod}[2]{{#1} \cross {#2}}
\newcommand{\tripleprod}[3]{\dotprod{\left(\crossprod{#1}{#2}\right)}{#3}}

\DeclareMathOperator{\Proj}{Proj}
\DeclareMathOperator{\Span}{span}
\DeclareMathOperator{\Sgn}{sgn}
\DeclareMathOperator{\Area}{Area}
\DeclareMathOperator{\Volume}{Volume}

%
% A few miscellaneous things specific to this document
%
\newcommand{\crossop}[1]{\crossprod{#1}{}}

% R2 vector.
\newcommand{\VectorTwo}[2]{
\begin{bmatrix}
 {#1} \\
 {#2}
\end{bmatrix}
}

\newcommand{\VectorN}[1]{
\begin{bmatrix}
{#1}_1 \\
{#1}_2 \\
\vdots \\
{#1}_N \\
\end{bmatrix}
}

\newcommand{\DETuvij}[4]{
\begin{vmatrix}
 {#1}_{#3} & {#1}_{#4} \\
 {#2}_{#3} & {#2}_{#4}
\end{vmatrix}
}

\newcommand{\DETuvwijk}[6]{
\begin{vmatrix}
 {#1}_{#4} & {#1}_{#5} & {#1}_{#6} \\
 {#2}_{#4} & {#2}_{#5} & {#2}_{#6} \\
 {#3}_{#4} & {#3}_{#5} & {#3}_{#6}
\end{vmatrix}
}

\newcommand{\DETuvwxijkl}[8]{
\begin{vmatrix}
 {#1}_{#5} & {#1}_{#6} & {#1}_{#7} & {#1}_{#8} \\
 {#2}_{#5} & {#2}_{#6} & {#2}_{#7} & {#2}_{#8} \\
 {#3}_{#5} & {#3}_{#6} & {#3}_{#7} & {#3}_{#8} \\
 {#4}_{#5} & {#4}_{#6} & {#4}_{#7} & {#4}_{#8} \\
\end{vmatrix}
}

%\newcommand{\DETuvwxyijklm}[10]{
%\begin{vmatrix}
% {#1}_{#6} & {#1}_{#7} & {#1}_{#8} & {#1}_{#9} & {#1}_{#10} \\
% {#2}_{#6} & {#2}_{#7} & {#2}_{#8} & {#2}_{#9} & {#2}_{#10} \\
% {#3}_{#6} & {#3}_{#7} & {#3}_{#8} & {#3}_{#9} & {#3}_{#10} \\
% {#4}_{#6} & {#4}_{#7} & {#4}_{#8} & {#4}_{#9} & {#4}_{#10} \\
% {#5}_{#6} & {#5}_{#7} & {#5}_{#8} & {#5}_{#9} & {#5}_{#10}
%\end{vmatrix}
%}

% R3 vector.
\newcommand{\VectorThree}[3]{
\begin{bmatrix}
 {#1} \\
 {#2} \\
 {#3}
\end{bmatrix}
}



\author{Peeter Joot}
\email{peeter.joot@gmail.com}


\chapter{Some worked problems from old PHY356 exams.}
\label{chap:qmIexamPractice}
%\useCCL
\blogpage{http://sites.google.com/site/peeterjoot/math2010/qmIexamPractice.pdf}
\date{Dec X, 2010}
\revisionInfo{qmIexamPractice.tex}

\beginArtWithToc
%\beginArtNoToc

\section{Motivation.}

Some of the old exam questions that I did for preparation for the exam I liked, and thought I'd write up some of them for potential future reference.

\section{Questions from the Dec 2007 PHY355H1F exam.}
\subsection{1b.}

\paragraph{Q:} If $\Pi$ is the parity operator, defined by $\Pi \ket{x} = \ket{-x}$, where $\ket{x}$ is the eigenket of the position operator $X$ with eigenvalue $x$), and $P$ is the momentum operator conjugate to $X$, show (carefully) that $\Pi P \Pi = -P$.

\paragraph{A:}

Consider the matrix element $\bra{-x'} \antisymmetric{\Pi}{P} \ket{x}$.  This is

\begin{align*}
\bra{-x'} \antisymmetric{\Pi}{P} \ket{x}
&=
\bra{-x'} \Pi P - P \Pi \ket{x} \\
&=
\bra{-x'} \Pi P \ket{x} - \bra{-x} P \Pi \ket{x} \\
&=
\bra{x'} P \ket{x} - \bra{-x} P \ket{-x} \\
&=
- i \hbar \left(
\delta(x'-x) \PD{}{x}
-\underbrace{\delta(-x -(-x'))}_{= \delta(x'-x) = \delta(x-x')} \PD{}{-x}
\right) \\
&=
- 2 i \hbar 
\delta(x'-x) \PD{}{x} \\
&=
2 \bra{x'} P \ket{x} \\
&=
2 \bra{-x'} \Pi P \ket{x} \\
\end{align*}

We've taken advantage of the Hermitian property of $P$ and $\Pi$ here, and can rearrange for

\begin{equation}\label{eqn:qmIexamPractice2007Dec:1b:10}
\bra{-x'} \Pi P - P \Pi - 2 \Pi P \ket{x} = 0
\end{equation}

Since this is true for all $\bra{-x}$ and $\ket{x}$ we have

\begin{equation}\label{eqn:qmIexamPractice2007Dec:1b:20}
\Pi P + P \Pi = 0.
\end{equation}

Right multiplication by $\Pi$ and rearranging we have
\begin{equation}\label{eqn:qmIexamPractice2007Dec:1b:30}
\Pi P \Pi = - P \Pi \Pi = - P.
\end{equation}

\subsection{1f.}

\paragraph{Q:} For a free particle moving in one-dimention, the propagator (i.e. the coordinate representation of the evolution opeator),

\begin{equation}\label{eqn:qmIexamPractice2007Dec:1f:10}
G(x,x';t) = \bra{x} U(t) \ket{x'}
\end{equation}

is given by

\begin{equation}\label{eqn:qmIexamPractice2007Dec:1f:20}
G(x,x';t) = \sqrt{\frac{m}{2 \pi i \hbar t}} e^{i m (x-x')^2/ (2 \hbar t)}.
\end{equation}

\paragraph{A:}

This problem is actually fairly straightforward, but it is nice to work it having had a similar problem set question where we were asked about this time evolution operator matrix element (ie: what it's physical meaning is).  Here we have a concrete example of the form of this matrix operator.

Proceeding directly, we have

\begin{align*}
\bra{x} U \ket{x'}
&=
\int \braket{x}{p'} \bra{p'} U \ket{p} \braket{p}{x'} dp dp' \\
&=
\int u_{p'}(x) \bra{p'} e^{-i P^2 t/(2 m \hbar)} \ket{p} u_p^\conj(x') dp dp' \\
&=
\int u_{p'}(x) e^{-i p^2 t/(2 m \hbar)} \delta(p-p') u_p^\conj(x') dp dp' \\
&=
\int u_{p}(x) e^{-i p^2 t/(2 m \hbar)} u_p^\conj(x') dp \\
&=
\inv{(\sqrt{2 \pi \hbar})^2} \int e^{i p (x-x')/\hbar} e^{-i p^2 t/(2 m \hbar)} dp \\
&=
\inv{2 \pi \hbar} \int e^{i p (x-x')/\hbar} e^{-i p^2 t/(2 m \hbar)} dp \\
&=
\inv{2 \pi} 
\int e^{i k (x-x')} e^{-i \hbar k^2 t/(2 m)} dk \\
&=
\inv{2 \pi} 
\int dk e^{- \left(k^2 \frac{ i \hbar t}{2m} - i k (x-x')
\right)
} \\
&=
\inv{2 \pi} 
\int dk e^{- \frac{ i \hbar t}{2m}\left(k - i \frac{2m}{i \hbar t}\frac{(x-x')}{2} \right)^2
- \frac{i^2 2 m (x-x')^2}{4 i \hbar t} 
} \\
&=
\inv{2 \pi}  \sqrt{\pi} \sqrt{\frac{2m}{i \hbar t}}
e^{\frac{ i m (x-x')^2}{2 \hbar t}},
\end{align*}

which is the desired result.  Now, let's look at how this would be used.  We can express our time evolved state using this matrix element by introducing an identity

\begin{align*}
\braket{x}{\psi(t)} 
&=
\bra{x} U \ket{\psi(0)} \\
&=
\int dx' \bra{x} U \ket{x'} \braket{x'}{\psi(0)} \\
&=
\sqrt{\frac{m}{2 \pi i \hbar t}} 
\int dx' 
e^{i m (x-x')^2/ (2 \hbar t)}
\braket{x'}{\psi(0)} \\
\end{align*}

This gives us
\begin{equation}\label{eqn:qmIexamPractice2007Dec:1f:40}
\psi(x, t)
=
\sqrt{\frac{m}{2 \pi i \hbar t}} 
\int dx' 
e^{i m (x-x')^2/ (2 \hbar t)} \psi(x', 0)
\end{equation}

However, note that our free particle wave function at time zero is

\begin{equation}\label{eqn:qmIexamPractice2007Dec:1f:50}
\psi(x, 0) = \frac{e^{i p x/\hbar}}{\sqrt{2 \pi \hbar}}
\end{equation}

So the convolution integral \ref{eqn:qmIexamPractice2007Dec:1f:40} does not exist.  We likely have to require that the solution be not a pure state, but instead a superposition of a set of continuous states (a wave packet in position or momentum space related by Fourier transforms).  That is

\begin{align}\label{eqn:qmIexamPractice2007Dec:1f:60}
\psi(x, 0) &= 
\inv{\sqrt{2 \pi \hbar}} \int \hat{\psi}(p, 0) e^{i p x/\hbar} dp \\
\hat{\psi}(p, 0) &= 
\inv{\sqrt{2 \pi \hbar}} \int \psi(x'', 0) e^{-i p x''/\hbar} dx''
\end{align}

The time evolution of this wave packet is then determined by the propagator, and is

\begin{equation}\label{eqn:qmIexamPractice2007Dec:1f:70}
\psi(x,t) =
\sqrt{\frac{m}{2 \pi i \hbar t}} 
\inv{\sqrt{2 \pi \hbar}} 
\int dx' dp
e^{i m (x-x')^2/ (2 \hbar t)}
\hat{\psi}(p, 0) e^{i p x'/\hbar} ,
\end{equation}

or in terms of the position space wave packet evaluated at time zero

\begin{equation}\label{eqn:qmIexamPractice2007Dec:1f:80}
\psi(x,t) =
\sqrt{\frac{m}{2 \pi i \hbar t}}
\inv{2 \pi}
\int dx' dx'' dk
e^{i m (x-x')^2/ (2 \hbar t)}
e^{i k (x' - x'')} \psi(x'', 0)
\end{equation}

We see that the propagator also ends up with a Fourier transform structure, and we have

\begin{align}\label{eqn:qmIexamPractice2007Dec:1f:90}
\psi(x,t) &= \int dx' U(x, x' ; t) \psi(x', 0) \\
U(x, x' ; t) &=
\sqrt{\frac{m}{2 \pi i \hbar t}}
\inv{2 \pi}
\int du dk
e^{i m (x - x' - u)^2/ (2 \hbar t)}
e^{i k u }
\end{align}

Does that Fourier transform exist?  I'd not be suprised if it ended up with a delta function representation.  I'll hold off attempting to evaluate and reduce it until another day.

\subsection{2.}

%\EndArticle
\EndNoBibArticle
