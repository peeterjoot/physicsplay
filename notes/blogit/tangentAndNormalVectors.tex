%
% Copyright � 2012 Peeter Joot.  All Rights Reserved.
% Licenced as described in the file LICENSE under the root directory of this GIT repository.
%
% pick one:
%\newcommand{\authorname}{Peeter Joot}
\newcommand{\email}{peeter.joot@utoronto.ca}
\newcommand{\studentnumber}{920798560}
\newcommand{\basename}{FIXMEbasenameUndefined}
\newcommand{\dirname}{notes/FIXMEdirnameUndefined/}

\newcommand{\authorname}{Peeter Joot}
\newcommand{\email}{peeterjoot@protonmail.com}
\newcommand{\basename}{FIXMEbasenameUndefined}
\newcommand{\dirname}{notes/FIXMEdirnameUndefined/}

\renewcommand{\basename}{tangentAndNormalVectors}
\renewcommand{\dirname}{notes/gabook/}
%\newcommand{\dateintitle}{}
\newcommand{\keywords}{tangent plane, surface normal, gradient, 3D, 4D, reciprocal basis, duality, pseudoscalar, geometric algebra, bivector, trivector, Minkowski space}

\newcommand{\authorname}{Peeter Joot}
\newcommand{\onlineurl}{http://sites.google.com/site/peeterjoot2/math2013/\basename.pdf}
\newcommand{\sourcepath}{\dirname\basename.tex}
\newcommand{\generatetitle}[1]{\chapter{#1}}

\newcommand{\vcsinfo}{%
\section*{}
\noindent{\color{DarkOliveGreen}{\rule{\linewidth}{0.1mm}}}
\paragraph{Document version}
%\paragraph{\color{Maroon}{Document version}}
{
\small
\begin{itemize}
\item Available online at:\\ 
\href{\onlineurl}{\onlineurl}
\item Git Repository: \input{./.revinfo/gitRepo.tex}
\item Source: \sourcepath
\item last commit: \input{./.revinfo/gitCommitString.tex}
\item commit date: \input{./.revinfo/gitCommitDate.tex}
\end{itemize}
}
}

%\PassOptionsToPackage{dvipsnames,svgnames}{xcolor}
\PassOptionsToPackage{square,numbers}{natbib}
\documentclass{scrreprt}

\usepackage[left=2cm,right=2cm]{geometry}
\usepackage[svgnames]{xcolor}
\usepackage{peeters_layout}

\usepackage{natbib}

\usepackage[
colorlinks=true,
bookmarks=false,
pdfauthor={\authorname, \email},
backref 
]{hyperref}

% http://tex.stackexchange.com/questions/75773/how-to-reference-problems-by-the-text-label-in-an-exercise-envioronment
\usepackage[english]{cleveref}
\crefname{Exercise}{exercise}{exercises}
\Crefname{Exercise}{Exercise}{Exercises}

\RequirePackage{titlesec}
\RequirePackage{ifthen}

% http://stackoverflow.com/questions/4932910/date-in-the-tabular-environment
\makeatletter
\let\insertdate\@date
\makeatother

\titleformat{\chapter}[display]
{\bfseries\Large}
{\color{DarkSlateGrey}\filleft \authorname
\ifthenelse{\isundefined{\studentnumber}}{}{\\ \studentnumber}
\ifthenelse{\isundefined{\email}}{}{\\ \email}
\ifthenelse{\isundefined{\dateintitle}}{}{\\ \insertdate}
%\ifthenelse{\isundefined{\coursename}}{}{\\ \coursename} % put in title instead.
}
{4ex}
{\color{DarkOliveGreen}{\titlerule}\color{Maroon}
\vspace{2ex}%
\filright}
[\vspace{2ex}%
\color{DarkOliveGreen}\titlerule
]

\newcommand{\beginArtWithToc}[0]{\begin{document}\tableofcontents}
\newcommand{\beginArtNoToc}[0]{\begin{document}}
\newcommand{\EndNoBibArticle}[0]{\end{document}}
\newcommand{\EndArticle}[0]{\bibliography{Bibliography}\bibliographystyle{plainnat}\end{document}}

% 
%\newcommand{\citep}[1]{\cite{#1}}

\colorSectionsForArticle



\beginArtNoToc

\generatetitle{Tangent planes and normals in three and four dimensions}
%\chapter{Tangent planes and normals in three and four dimensions}
%\label{chap:\basename}
\section{Motivation}

I was reviewing the method of Lagranage in my old first year calculus book \citep{salas1990coa} and found that I needed a review of some of the geometry ideas associated with the gradient (that it is normal to the surface).  The approach in the text used 3D level surfaces $f(x, y, z) = c$, which is general but not the most intuitive.  

If we define a surface in the simpler explicit form $z = f(x, y)$, then how would you show this normal property?  Here we explore this in 3D and 4D, using geometric and wedge products to express the tangent planes and tangent volumes respectively.

In the 4D approach, with a vector $x$ defined by coordinates $x^\mu$ and basis $\{\gamma_\mu\}$ so that

\begin{dmath}\label{eqn:tangentAndNormalVectors:20}
x = \gamma_\mu x^\mu,
\end{dmath}

the reciprocal basis ${\gamma^\mu}$ is defined implicitly by the dot product relations

\begin{dmath}\label{eqn:tangentAndNormalVectors:40}
\gamma^\mu \cdot \gamma_\nu = {\delta^\mu}_\nu.
\end{dmath}

Assuming such a basis makes the result general enough that the 4D (or a trivial generalization to N dimensions) holds for both Euclidean spaces as well as mixed metric (i.e. Minkowski) spaces, and avoids having to detail the specific metric in question.

\section{3D surface}

We start by considering \cref{fig:tangentAndNormalVectors:tangentAndNormalVectorsFig1}.

\imageFigure{tangentAndNormalVectorsFig1}{A portion of a surface in 3D}{fig:tangentAndNormalVectors:tangentAndNormalVectorsFig1}{0.3}

We wish to determine the bivector for the tangent plane in the neighbourhood of the point $\Bp$

\begin{dmath}\label{eqn:tangentAndNormalVectors:60}
\Bp = ( x, y, f(x, y) ),
\end{dmath}

then using duality determine the normal vector to that plane at this point.  Holding either of the two free parameters constant, we find the tangent vectors on that surface to be

\begin{subequations}
\begin{dmath}\label{eqn:tangentAndNormalVectors:80}
\Bp_1 
= \left( dx, 0, \PD{x}{f} dx \right) 
\propto \left( 1, 0, \PD{x}{f} \right) 
\end{dmath}
\begin{dmath}\label{eqn:tangentAndNormalVectors:100}
\Bp_2 
= \left( 0, dy, \PD{y}{f} dy \right) 
\propto \left( 0, 1, \PD{y}{f} \right) 
\end{dmath}
\end{subequations}

The tangent plane is then

\begin{dmath}\label{eqn:tangentAndNormalVectors:120}
\Bp_1 \wedge \Bp_2 = 
\left( 1, 0, \PD{x}{f} \right) \wedge
\left( 0, 1, \PD{y}{f} \right) 
=
\left( \Be_1 + \Be_3 \PD{x}{f} \right) 
\wedge
\left( \Be_2 + \Be_3 \PD{y}{f} \right) 
=
\Be_1 \Be_2 
+ \Be_1 \Be_3 \PD{y}{f} 
+ \Be_3 \Be_2 \PD{x}{f}.
\end{dmath}

We can factor out the pseudoscalar 3D volume element $I = \Be_1 \Be_2 \Be_3$, assuming a Euclidean space for which $\Be_k^2 = 1$.  That is

\begin{dmath}\label{eqn:tangentAndNormalVectors:140}
\Bp_1 \wedge \Bp_2 = 
\Be_1 \Be_2 \Be_3 \left(
\Be_3
- \Be_2 \PD{y}{f} 
- \Be_1 \PD{x}{f}
\right)
\end{dmath}

Multiplying through by the volume element $I$ we find that the normal to the surface at this point is 

\begin{dmath}\label{eqn:tangentAndNormalVectors:160}
\Bn 
\propto -I(\Bp_1 \wedge \Bp_2) 
= 
\Be_3
- \Be_1 \PD{x}{f}
- \Be_2 \PD{y}{f}.
\end{dmath}

Observe that we can write this as

\begin{dmath}\label{eqn:tangentAndNormalVectors:180}
\boxed{
\Bn = \spacegrad ( z - f(x, y) ).
}
\end{dmath}

Let's see how this works in 4D, so that we know how to handle the Minkowski spaces we find in special relativity.

\section{4D surface}

% this is to produce the sites.google url and version info and so forth (for blog posts)
%\vcsinfo
\EndArticle
