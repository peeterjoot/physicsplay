%
% Copyright � 2015 Peeter Joot.  All Rights Reserved.
% Licenced as described in the file LICENSE under the root directory of this GIT repository.
%
\documentclass[]{eliblog}

\usepackage{amsmath}
\usepackage{mathpazo}

%
% shorthand for bold symbols, convenient for vectors and matrices
%
\newcommand{\Ba}[0]{\mathbf{a}}
\newcommand{\Bb}[0]{\mathbf{b}}
\newcommand{\Bc}[0]{\mathbf{c}}
\newcommand{\Bd}[0]{\mathbf{d}}
\newcommand{\Be}[0]{\mathbf{e}}
\newcommand{\Bf}[0]{\mathbf{f}}
\newcommand{\Bg}[0]{\mathbf{g}}
\newcommand{\Bh}[0]{\mathbf{h}}
\newcommand{\Bi}[0]{\mathbf{i}}
\newcommand{\Bj}[0]{\mathbf{j}}
\newcommand{\Bk}[0]{\mathbf{k}}
\newcommand{\Bl}[0]{\mathbf{l}}
\newcommand{\Bm}[0]{\mathbf{m}}
\newcommand{\Bn}[0]{\mathbf{n}}
\newcommand{\Bo}[0]{\mathbf{o}}
\newcommand{\Bp}[0]{\mathbf{p}}
\newcommand{\Bq}[0]{\mathbf{q}}
\newcommand{\Br}[0]{\mathbf{r}}
\newcommand{\Bs}[0]{\mathbf{s}}
\newcommand{\Bt}[0]{\mathbf{t}}
\newcommand{\Bu}[0]{\mathbf{u}}
\newcommand{\Bv}[0]{\mathbf{v}}
\newcommand{\Bw}[0]{\mathbf{w}}
\newcommand{\Bx}[0]{\mathbf{x}}
\newcommand{\By}[0]{\mathbf{y}}
\newcommand{\Bz}[0]{\mathbf{z}}
\newcommand{\BA}[0]{\mathbf{A}}
\newcommand{\BB}[0]{\mathbf{B}}
\newcommand{\BC}[0]{\mathbf{C}}
\newcommand{\BD}[0]{\mathbf{D}}
\newcommand{\BE}[0]{\mathbf{E}}
\newcommand{\BF}[0]{\mathbf{F}}
\newcommand{\BG}[0]{\mathbf{G}}
\newcommand{\BH}[0]{\mathbf{H}}
\newcommand{\BI}[0]{\mathbf{I}}
\newcommand{\BJ}[0]{\mathbf{J}}
\newcommand{\BK}[0]{\mathbf{K}}
\newcommand{\BL}[0]{\mathbf{L}}
\newcommand{\BM}[0]{\mathbf{M}}
\newcommand{\BN}[0]{\mathbf{N}}
\newcommand{\BO}[0]{\mathbf{O}}
\newcommand{\BP}[0]{\mathbf{P}}
\newcommand{\BQ}[0]{\mathbf{Q}}
\newcommand{\BR}[0]{\mathbf{R}}
\newcommand{\BS}[0]{\mathbf{S}}
\newcommand{\BT}[0]{\mathbf{T}}
\newcommand{\BU}[0]{\mathbf{U}}
\newcommand{\BV}[0]{\mathbf{V}}
\newcommand{\BW}[0]{\mathbf{W}}
\newcommand{\BX}[0]{\mathbf{X}}
\newcommand{\BY}[0]{\mathbf{Y}}
\newcommand{\BZ}[0]{\mathbf{Z}}

\newcommand{\Bzero}[0]{\mathbf{0}}
\newcommand{\Btheta}[0]{\boldsymbol{\theta}}
\newcommand{\Btau}[0]{\boldsymbol{\tau}}
\newcommand{\Bomega}[0]{\boldsymbol{\omega}}

%
% shorthand for unit vectors
%
\newcommand{\acap}[0]{\hat{\Ba}}
\newcommand{\bcap}[0]{\hat{\Bb}}
\newcommand{\ccap}[0]{\hat{\Bc}}
\newcommand{\dcap}[0]{\hat{\Bd}}
\newcommand{\ecap}[0]{\hat{\Be}}
\newcommand{\fcap}[0]{\hat{\Bf}}
\newcommand{\gcap}[0]{\hat{\Bg}}
\newcommand{\hcap}[0]{\hat{\Bh}}
\newcommand{\icap}[0]{\hat{\Bi}}
\newcommand{\jcap}[0]{\hat{\Bj}}
\newcommand{\kcap}[0]{\hat{\Bk}}
\newcommand{\lcap}[0]{\hat{\Bl}}
\newcommand{\mcap}[0]{\hat{\Bm}}
\newcommand{\ncap}[0]{\hat{\Bn}}
\newcommand{\ocap}[0]{\hat{\Bo}}
\newcommand{\pcap}[0]{\hat{\Bp}}
\newcommand{\qcap}[0]{\hat{\Bq}}
\newcommand{\rcap}[0]{\hat{\Br}}
\newcommand{\scap}[0]{\hat{\Bs}}
\newcommand{\tcap}[0]{\hat{\Bt}}
\newcommand{\ucap}[0]{\hat{\Bu}}
\newcommand{\vcap}[0]{\hat{\Bv}}
\newcommand{\wcap}[0]{\hat{\Bw}}
\newcommand{\xcap}[0]{\hat{\Bx}}
\newcommand{\ycap}[0]{\hat{\By}}
\newcommand{\zcap}[0]{\hat{\Bz}}
\newcommand{\thetacap}[0]{\hat{\Btheta}}

%
% to write R^n and C^n in a distinguishable fashion.  Perhaps change this
% to the double lined characters upon figuring out how to do so.
%
\newcommand{\C}[1]{$\mathbb{C}^{#1}$}
\newcommand{\R}[1]{$\mathbb{R}^{#1}$}

%
% various generally useful helpers
%

% derivative of #1 wrt. #2:
\newcommand{\D}[2] {\frac {d#2} {d#1}}

\newcommand{\inv}[1]{\frac{1}{#1}}
\newcommand{\cross}[0]{\times}

\newcommand{\abs}[1]{\lvert{#1}\rvert}
\newcommand{\norm}[1]{\lVert{#1}\rVert}
\newcommand{\innerprod}[2]{\langle{#1}, {#2}\rangle}
\newcommand{\dotprod}[2]{{#1} \cdot {#2}}
\newcommand{\bdotprod}[2]{\left({#1} \cdot {#2}\right)}
\newcommand{\crossprod}[2]{{#1} \cross {#2}}
\newcommand{\tripleprod}[3]{\dotprod{\left(\crossprod{#1}{#2}\right)}{#3}}

\DeclareMathOperator{\Proj}{Proj}
\DeclareMathOperator{\Span}{span}
\DeclareMathOperator{\Sgn}{sgn}
\DeclareMathOperator{\Area}{Area}
\DeclareMathOperator{\Volume}{Volume}

%
% A few miscellaneous things specific to this document
%
\newcommand{\crossop}[1]{\crossprod{#1}{}}

% R2 vector.
\newcommand{\VectorTwo}[2]{
\begin{bmatrix}
 {#1} \\
 {#2}
\end{bmatrix}
}

\newcommand{\VectorN}[1]{
\begin{bmatrix}
{#1}_1 \\
{#1}_2 \\
\vdots \\
{#1}_N \\
\end{bmatrix}
}

\newcommand{\DETuvij}[4]{
\begin{vmatrix}
 {#1}_{#3} & {#1}_{#4} \\
 {#2}_{#3} & {#2}_{#4}
\end{vmatrix}
}

\newcommand{\DETuvwijk}[6]{
\begin{vmatrix}
 {#1}_{#4} & {#1}_{#5} & {#1}_{#6} \\
 {#2}_{#4} & {#2}_{#5} & {#2}_{#6} \\
 {#3}_{#4} & {#3}_{#5} & {#3}_{#6}
\end{vmatrix}
}

\newcommand{\DETuvwxijkl}[8]{
\begin{vmatrix}
 {#1}_{#5} & {#1}_{#6} & {#1}_{#7} & {#1}_{#8} \\
 {#2}_{#5} & {#2}_{#6} & {#2}_{#7} & {#2}_{#8} \\
 {#3}_{#5} & {#3}_{#6} & {#3}_{#7} & {#3}_{#8} \\
 {#4}_{#5} & {#4}_{#6} & {#4}_{#7} & {#4}_{#8} \\
\end{vmatrix}
}

%\newcommand{\DETuvwxyijklm}[10]{
%\begin{vmatrix}
% {#1}_{#6} & {#1}_{#7} & {#1}_{#8} & {#1}_{#9} & {#1}_{#10} \\
% {#2}_{#6} & {#2}_{#7} & {#2}_{#8} & {#2}_{#9} & {#2}_{#10} \\
% {#3}_{#6} & {#3}_{#7} & {#3}_{#8} & {#3}_{#9} & {#3}_{#10} \\
% {#4}_{#6} & {#4}_{#7} & {#4}_{#8} & {#4}_{#9} & {#4}_{#10} \\
% {#5}_{#6} & {#5}_{#7} & {#5}_{#8} & {#5}_{#9} & {#5}_{#10}
%\end{vmatrix}
%}

% R3 vector.
\newcommand{\VectorThree}[3]{
\begin{bmatrix}
 {#1} \\
 {#2} \\
 {#3}
\end{bmatrix}
}



\author{Peeter Joot}
\email{peeter.joot@gmail.com}

%\documentclass[]{eliblogwidescreen}

\usepackage{amsmath}
\usepackage{mathpazo}

%
% shorthand for bold symbols, convenient for vectors and matrices
%
\newcommand{\Ba}[0]{\mathbf{a}}
\newcommand{\Bb}[0]{\mathbf{b}}
\newcommand{\Bc}[0]{\mathbf{c}}
\newcommand{\Bd}[0]{\mathbf{d}}
\newcommand{\Be}[0]{\mathbf{e}}
\newcommand{\Bf}[0]{\mathbf{f}}
\newcommand{\Bg}[0]{\mathbf{g}}
\newcommand{\Bh}[0]{\mathbf{h}}
\newcommand{\Bi}[0]{\mathbf{i}}
\newcommand{\Bj}[0]{\mathbf{j}}
\newcommand{\Bk}[0]{\mathbf{k}}
\newcommand{\Bl}[0]{\mathbf{l}}
\newcommand{\Bm}[0]{\mathbf{m}}
\newcommand{\Bn}[0]{\mathbf{n}}
\newcommand{\Bo}[0]{\mathbf{o}}
\newcommand{\Bp}[0]{\mathbf{p}}
\newcommand{\Bq}[0]{\mathbf{q}}
\newcommand{\Br}[0]{\mathbf{r}}
\newcommand{\Bs}[0]{\mathbf{s}}
\newcommand{\Bt}[0]{\mathbf{t}}
\newcommand{\Bu}[0]{\mathbf{u}}
\newcommand{\Bv}[0]{\mathbf{v}}
\newcommand{\Bw}[0]{\mathbf{w}}
\newcommand{\Bx}[0]{\mathbf{x}}
\newcommand{\By}[0]{\mathbf{y}}
\newcommand{\Bz}[0]{\mathbf{z}}
\newcommand{\BA}[0]{\mathbf{A}}
\newcommand{\BB}[0]{\mathbf{B}}
\newcommand{\BC}[0]{\mathbf{C}}
\newcommand{\BD}[0]{\mathbf{D}}
\newcommand{\BE}[0]{\mathbf{E}}
\newcommand{\BF}[0]{\mathbf{F}}
\newcommand{\BG}[0]{\mathbf{G}}
\newcommand{\BH}[0]{\mathbf{H}}
\newcommand{\BI}[0]{\mathbf{I}}
\newcommand{\BJ}[0]{\mathbf{J}}
\newcommand{\BK}[0]{\mathbf{K}}
\newcommand{\BL}[0]{\mathbf{L}}
\newcommand{\BM}[0]{\mathbf{M}}
\newcommand{\BN}[0]{\mathbf{N}}
\newcommand{\BO}[0]{\mathbf{O}}
\newcommand{\BP}[0]{\mathbf{P}}
\newcommand{\BQ}[0]{\mathbf{Q}}
\newcommand{\BR}[0]{\mathbf{R}}
\newcommand{\BS}[0]{\mathbf{S}}
\newcommand{\BT}[0]{\mathbf{T}}
\newcommand{\BU}[0]{\mathbf{U}}
\newcommand{\BV}[0]{\mathbf{V}}
\newcommand{\BW}[0]{\mathbf{W}}
\newcommand{\BX}[0]{\mathbf{X}}
\newcommand{\BY}[0]{\mathbf{Y}}
\newcommand{\BZ}[0]{\mathbf{Z}}

\newcommand{\Bzero}[0]{\mathbf{0}}
\newcommand{\Btheta}[0]{\boldsymbol{\theta}}
\newcommand{\Btau}[0]{\boldsymbol{\tau}}
\newcommand{\Bomega}[0]{\boldsymbol{\omega}}

%
% shorthand for unit vectors
%
\newcommand{\acap}[0]{\hat{\Ba}}
\newcommand{\bcap}[0]{\hat{\Bb}}
\newcommand{\ccap}[0]{\hat{\Bc}}
\newcommand{\dcap}[0]{\hat{\Bd}}
\newcommand{\ecap}[0]{\hat{\Be}}
\newcommand{\fcap}[0]{\hat{\Bf}}
\newcommand{\gcap}[0]{\hat{\Bg}}
\newcommand{\hcap}[0]{\hat{\Bh}}
\newcommand{\icap}[0]{\hat{\Bi}}
\newcommand{\jcap}[0]{\hat{\Bj}}
\newcommand{\kcap}[0]{\hat{\Bk}}
\newcommand{\lcap}[0]{\hat{\Bl}}
\newcommand{\mcap}[0]{\hat{\Bm}}
\newcommand{\ncap}[0]{\hat{\Bn}}
\newcommand{\ocap}[0]{\hat{\Bo}}
\newcommand{\pcap}[0]{\hat{\Bp}}
\newcommand{\qcap}[0]{\hat{\Bq}}
\newcommand{\rcap}[0]{\hat{\Br}}
\newcommand{\scap}[0]{\hat{\Bs}}
\newcommand{\tcap}[0]{\hat{\Bt}}
\newcommand{\ucap}[0]{\hat{\Bu}}
\newcommand{\vcap}[0]{\hat{\Bv}}
\newcommand{\wcap}[0]{\hat{\Bw}}
\newcommand{\xcap}[0]{\hat{\Bx}}
\newcommand{\ycap}[0]{\hat{\By}}
\newcommand{\zcap}[0]{\hat{\Bz}}
\newcommand{\thetacap}[0]{\hat{\Btheta}}

%
% to write R^n and C^n in a distinguishable fashion.  Perhaps change this
% to the double lined characters upon figuring out how to do so.
%
\newcommand{\C}[1]{$\mathbb{C}^{#1}$}
\newcommand{\R}[1]{$\mathbb{R}^{#1}$}

%
% various generally useful helpers
%

% derivative of #1 wrt. #2:
\newcommand{\D}[2] {\frac {d#2} {d#1}}

\newcommand{\inv}[1]{\frac{1}{#1}}
\newcommand{\cross}[0]{\times}

\newcommand{\abs}[1]{\lvert{#1}\rvert}
\newcommand{\norm}[1]{\lVert{#1}\rVert}
\newcommand{\innerprod}[2]{\langle{#1}, {#2}\rangle}
\newcommand{\dotprod}[2]{{#1} \cdot {#2}}
\newcommand{\bdotprod}[2]{\left({#1} \cdot {#2}\right)}
\newcommand{\crossprod}[2]{{#1} \cross {#2}}
\newcommand{\tripleprod}[3]{\dotprod{\left(\crossprod{#1}{#2}\right)}{#3}}

\DeclareMathOperator{\Proj}{Proj}
\DeclareMathOperator{\Span}{span}
\DeclareMathOperator{\Sgn}{sgn}
\DeclareMathOperator{\Area}{Area}
\DeclareMathOperator{\Volume}{Volume}

%
% A few miscellaneous things specific to this document
%
\newcommand{\crossop}[1]{\crossprod{#1}{}}

% R2 vector.
\newcommand{\VectorTwo}[2]{
\begin{bmatrix}
 {#1} \\
 {#2}
\end{bmatrix}
}

\newcommand{\VectorN}[1]{
\begin{bmatrix}
{#1}_1 \\
{#1}_2 \\
\vdots \\
{#1}_N \\
\end{bmatrix}
}

\newcommand{\DETuvij}[4]{
\begin{vmatrix}
 {#1}_{#3} & {#1}_{#4} \\
 {#2}_{#3} & {#2}_{#4}
\end{vmatrix}
}

\newcommand{\DETuvwijk}[6]{
\begin{vmatrix}
 {#1}_{#4} & {#1}_{#5} & {#1}_{#6} \\
 {#2}_{#4} & {#2}_{#5} & {#2}_{#6} \\
 {#3}_{#4} & {#3}_{#5} & {#3}_{#6}
\end{vmatrix}
}

\newcommand{\DETuvwxijkl}[8]{
\begin{vmatrix}
 {#1}_{#5} & {#1}_{#6} & {#1}_{#7} & {#1}_{#8} \\
 {#2}_{#5} & {#2}_{#6} & {#2}_{#7} & {#2}_{#8} \\
 {#3}_{#5} & {#3}_{#6} & {#3}_{#7} & {#3}_{#8} \\
 {#4}_{#5} & {#4}_{#6} & {#4}_{#7} & {#4}_{#8} \\
\end{vmatrix}
}

%\newcommand{\DETuvwxyijklm}[10]{
%\begin{vmatrix}
% {#1}_{#6} & {#1}_{#7} & {#1}_{#8} & {#1}_{#9} & {#1}_{#10} \\
% {#2}_{#6} & {#2}_{#7} & {#2}_{#8} & {#2}_{#9} & {#2}_{#10} \\
% {#3}_{#6} & {#3}_{#7} & {#3}_{#8} & {#3}_{#9} & {#3}_{#10} \\
% {#4}_{#6} & {#4}_{#7} & {#4}_{#8} & {#4}_{#9} & {#4}_{#10} \\
% {#5}_{#6} & {#5}_{#7} & {#5}_{#8} & {#5}_{#9} & {#5}_{#10}
%\end{vmatrix}
%}

% R3 vector.
\newcommand{\VectorThree}[3]{
\begin{bmatrix}
 {#1} \\
 {#2} \\
 {#3}
\end{bmatrix}
}



\author{Peeter Joot}
\email{peeter.joot@gmail.com}


\chapter{Review of approximation results.}
\label{chap:qmTwoReview}
%\useCCL
\blogpage{http://sites.google.com/site/peeterjoot/math2011/qmTwoReview.pdf}
\date{Nov 6, 2011}
\revisionInfo{qmTwoReview.tex}

\beginArtWithToc
%\beginArtNoToc

\section{Motivation.}

Here I'll summarize what I'd put on a cheat sheet for the tests or exam, if one would be allowed.  While I can derive these results, memorization unfortunately appears required for good test performance in this class, and this will give me a good reference of what to memorize.

\section{Variational method}

We can find an estimate of our ground state energy using

\begin{equation}\label{eqn:qmTwoReview:n}
\boxed{
\frac{
\bra{\Psi} H \ket{\Psi}
}{
\braket{\Psi}{\Psi}
}
\ge E_0
}
\end{equation}

\section{Time independent pertubation}

Given a perturbed Hamiltonian and an associated solution for the unperturbed state

\begin{equation}\label{eqn:qmTwoReview:n}\
\boxed{
\begin{aligned}
H &= H_0 + \lambda H', \qquad \lambda \in [0,1] \\
H_0 \ket{{\psi_m}^{(0)}} &= {E_m}^{(0)} \ket{{\psi_m}^{(0)}},
\end{aligned}
}
\end{equation}

we assume a power series solution for the energy and kets

\begin{equation}\label{eqn:qmTwoReview:n}
E_m = {E_m}^{(0)} + \lambda {E_m}^{(1)} + \lambda^2 {E_m}^{(2)} + \cdots
\end{equation}

\begin{equation}\label{eqn:qmTwoL4:130}
\begin{aligned}
\ket{\psi_m} &= 
\sum_n {c_{nm}}^{(0)} \ket{{\psi_n}^{(0)}} 
+
\lambda
\sum_n {c_{nm}}^{(1)} \ket{{\psi_n}^{(0)}} 
+ 
\lambda^2
\sum_n {c_{nm}}^{(2)} \ket{{\psi_n}^{(0)}} 
+ \cdots \\
&\propto
\ket{{\psi_m}^{(0)}} 
+ 
\lambda
\sum_{n \ne m} {\bar{c}_{nm}}^{(1)} \ket{{\psi_n}^{(0)}} 
+
\lambda^2
\sum_{n \ne m} {\bar{c}_{nm}}^{(2)} \ket{{\psi_n}^{(0)}} 
+ \cdots
\end{aligned}
\end{equation}

We found to second order in energy and first order in the kets

\begin{equation}\label{eqn:qmTwoReview:n}
\boxed{
\begin{aligned}
E_m &= E_m^{(0)} + \lambda {H_{mm}}' + \lambda^2 
\sum_{n \ne m} 
\frac{\Abs{{H_{nm}}'}^2 }
{ E_m^{(0)} - E_n^{(0)} } 
+ \cdots
\\
\ket{\psi_m} &\propto \ket{{\psi_m}^{(0)}} + \lambda
\sum_{n \ne m} 
\frac{{H_{nm}}'}
{ E_m^{(0)} - E_n^{(0)} } \ket{{\psi_n}^{(0)}}
+ \cdots \\
H_{nm}' &=
\bra{{\psi_n}^{(0)}}
H'
\ket{{\psi_s}^{(0)}}.
\end{aligned}
}
\end{equation}

If there are degenerate states, then we must sum over those too, but the initial state cannot itself be degenerate for this approximation to remain valid.

\section{Degeneracy.}

FIXME: TODO.

\section{Interaction picture.}

We split of the Hamiltonian into time independent and time dependent parts, and also factorize the time evolution operator

\begin{equation}\label{eqn:qmTwoReview:n}
\boxed{
\begin{aligned}
H &= H_0 + H_I(t) \\
\ket{\alpha_S} &= e^{-i H_0 t/\hbar } \ket{\alpha_I(t)} = e^{-i H_0 t/\hbar } U_I(t) \ket{\alpha_I(0)} .
\end{aligned}
}
\end{equation}

Plugging into Schr\"{o}dinger's equation we find

\begin{equation}\label{eqn:qmTwoReview:n}
\boxed{
\begin{aligned}
i \hbar \ddt{} \ket{\alpha_I(t)} &= H_I(t) \ket{\alpha_I(t)} \\
i \hbar \ddt{U_I} &= H_I' U_I \\
H_I'(t) &= e^{i H_0 t/\hbar } H_I(t) e^{-i H_0 t/\hbar } 
\end{aligned}
}
\end{equation}

\section{Time dependent pertubation.}

We moved on to time dependent pertubations of the form

\begin{equation}\label{eqn:qmTwoReview:n}
\boxed{
\begin{aligned}
H(t) &= H_0 + H'(t) \\
H_0 \ket{\psi_n^{(0)} } &= \hbar \omega_n \ket{\psi_n^{(0)} }.
\end{aligned}
}
\end{equation}

where $\hbar \omega_n$ are the energy eigenvalues, and $\ket{\psi_n^{(0)} }$ the energy eigenstates of the unperturbed Hamiltonian.

Use of the interaction picture led quickly to the problem of seeking the coefficients describing the perturbed state

\begin{equation}\label{eqn:qmTwoReview:n}
\ket{\psi(t)} = \sum_n c_n(t) e^{-i \omega_n t} \ket{\psi_n^{(0)} },
\end{equation}

and plugging in we found

\begin{equation}\label{eqn:qmTwoReview:n}
\boxed{
\begin{aligned}
i \hbar \dot{c}_s &= \sum_n H_{sn}'(t) e^{i \omega_{sn} t} c_n(t) \\
\omega_{sn} &= \omega_s - \omega_n \\
H_{sn}'(t) &= \bra{\psi_s^{(0)}} H'(t) \ket{\psi_n^{(0)} },
\end{aligned}
}
\end{equation}

\subsection{Pertubation expansion in series.}

Introducing a $\lambda$ parameterized dependence in the perturbation above, and assuming a power series expansion of our coefficients

\begin{equation}\label{eqn:qmTwoReview:n}
\boxed{
\begin{aligned}
H'(t) &\rightarrow \lambda H'(t) \\
c_s(t) &= c_s^{(0)}(t) + \lambda c_s^{(1)}(t) + \lambda^2 c_s^{(2)}(t) + \cdots
\end{aligned}
}
\end{equation}

we found, after equating powers of $\lambda$ a set of coupled differential equations

\begin{equation}\label{eqn:qmTwoReview:n}
\begin{aligned}
i \hbar \dot{c}_s^{(0)}(t) &= 0  \\
i \hbar \dot{c}_s^{(1)}(t) &= \sum_{n} H_{sn}'(t) e^{i \omega_{sn} t} c_n^{(0)}(t) \\
i \hbar \dot{c}_s^{(2)}(t) &= \sum_{n} H_{sn}'(t) e^{i \omega_{sn} t} c_n^{(1)}(t) \\
&\vdots
\end{aligned}
\end{equation}

Of particular value was the expansion, assuming that we started with an initial state in energy level $m$ before the pertubation was ``turned on'' (ie: $\lambda = 0$).

\begin{equation}\label{eqn:qmTwoReview:n}
\ket{\psi(t)} = e^{-i \omega_m t} \ket{\psi_m^{(0)} }
\end{equation}

So that $c_n^{(0)}(t) = \delta_{nm}$.  We then found a first order approximation for the transition probability coeffcient of

\begin{equation}\label{eqn:qmTwoReview:n}
\boxed{
i \hbar \dot{c}_m^{(1)} = H_{ms}'(t) e^{i \omega_{ms} t}
}
\end{equation}

\section{Sudden pertubations.}

The idea here is that we integrate Schr\"{o}dinger's equation over the small interval containing the changing Hamiltonian

\begin{equation}\label{eqn:qmTwoReview:n}
\ket{\psi(t)} = \ket{\psi(t_0)} + \inv{i\hbar} \int_{t_0}^t H(t') \ket{\psi(t')} dt'
\end{equation}

and find
\begin{equation}\label{eqn:qmTwoReview:n}
\boxed{
\ket{\psi_\text{after}} = \ket{\psi_\text{before}}.
}
\end{equation}

An implication is that, say, we start with a system measured in a given energy, that same system after the change to the Hamiltonian will then be in a state that is now a superposition of eigenkets from the new Hamiltonian.

\section{Adiabatic pertubations.}

Given a Hamiltonian that turns on slowly at $t=0$, a set of instantaneous eigenkets for the duration of the time dependent interval, and a representation in terms of the instantaneous eigenkets

\begin{equation}\label{eqn:qmTwoReview:n}
\boxed{
\begin{aligned}
H(t) &= H_0, \qquad t \le 0 \\
H(t) \ket{\psihat_n(t)} &= E_n(t) \ket{\psihat_n(t)} \\
\ket{\psi} &= \sum_n \bar{b}_n(t) e^{-i\alpha_n + i \beta_n} \ket{\psihat_n} \\
\alpha_n(t) &= \inv{\hbar} \int_0^t dt' E_n(t'),
\end{aligned}
}
\end{equation}

plugging into Schr\"{o}dinger's equation we find
\begin{equation}\label{eqn:qmTwoReview:n}
\boxed{
\begin{aligned}
\ddt{\bar{b}_m} &= - \sum_{n \ne m} \bar{b}_n e^{-i \gamma_{nm} } \bra{\psihat_m(t)} \ddt{} \ket{\psihat_n(t)}  \\
\gamma_{nm}(t) &= \alpha_n(t) - \alpha_m(t) - (\beta_n(t) - \beta_m(t)) \\
\beta_n(t) &= \int_0^t dt' \Gamma_n(t') \\
\Gamma_n(t) &= i \bra{\psihat_n(t)} \ddt{} \ket{\psihat_n(t)} \\
\end{aligned}
}
\end{equation}

\subsection{Evolution of a given state.}

Given a system initially measured with energy $E_m(0)$ before the time dependence is ``turned on''

\begin{equation}\label{eqn:qmTwoReview:n}
\boxed{
\ket{\psi(0)} = \ket{\psihat_m(0)},
}
\end{equation}

we find that the first order Taylor series expansion for the transition probability coefficents are
\begin{equation}\label{eqn:adiabaticApprox:290}
\boxed{
\barb_s(t) = \delta_{sm} - t (1 - \delta_{sm}) \bra{\psihat_s(0)} \evalbar{\ddt{} \ket{\psihat_m(t)}}{t=0}.
}
\end{equation}

\section{Fermi's golden rule.}

FIXME: TODO.

\section{WKB.}

We write Schr\"{o}dinger's equation as

\begin{equation}\label{eqn:qmTwoReview:n}
\boxed{
\begin{aligned}
0 &= \frac{d^2 U}{dx^2} + k^2 U \\
k^2 &= -\kappa^2 = \frac{2m (E - V)}{\hbar}.
\end{aligned}
}
\end{equation}

and seek solutions of the form $U \propto e^{i\phi}$.  Schr\"{o}dinger's equation takes the form

\begin{equation}\label{eqn:qmTwoReview:n}
- (\phi'(x))^2 + i \phi''(x) + k^2(x) = 0.
\end{equation}

Initially setting $\phi'' = 0$ we refine our approximation to find
\begin{equation}\label{eqn:qmTwoReview:n}
\phi'(x) 
= k(x) \sqrt{ 1 + i \frac{k'(x)}{k^2(x)} } .
\end{equation}

To first order, this gives us

\begin{equation}\label{eqn:qmTwoReview:n}
\boxed{
U(x) \propto \inv{\sqrt{k(x)}} e^{\pm i \int dx k(x)} 
}
\end{equation}

%\EndArticle
\EndNoBibArticle
