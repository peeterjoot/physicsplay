%
% Copyright � 2015 Peeter Joot.  All Rights Reserved.
% Licenced as described in the file LICENSE under the root directory of this GIT repository.
%
\newcommand{\authorname}{Peeter Joot}
\newcommand{\email}{peeterjoot@protonmail.com}
\newcommand{\basename}{FIXMEbasenameUndefined}
\newcommand{\dirname}{notes/FIXMEdirnameUndefined/}

\renewcommand{\basename}{unimodularAndRotation}
\renewcommand{\dirname}{notes/phy1520/}
%\newcommand{\dateintitle}{}
%\newcommand{\keywords}{}

\newcommand{\authorname}{Peeter Joot}
\newcommand{\onlineurl}{http://sites.google.com/site/peeterjoot2/math2013/\basename.pdf}
\newcommand{\sourcepath}{\dirname\basename.tex}
\newcommand{\generatetitle}[1]{\chapter{#1}}

\newcommand{\vcsinfo}{%
\section*{}
\noindent{\color{DarkOliveGreen}{\rule{\linewidth}{0.1mm}}}
\paragraph{Document version}
%\paragraph{\color{Maroon}{Document version}}
{
\small
\begin{itemize}
\item Available online at:\\ 
\href{\onlineurl}{\onlineurl}
\item Git Repository: \input{./.revinfo/gitRepo.tex}
\item Source: \sourcepath
\item last commit: \input{./.revinfo/gitCommitString.tex}
\item commit date: \input{./.revinfo/gitCommitDate.tex}
\end{itemize}
}
}

%\PassOptionsToPackage{dvipsnames,svgnames}{xcolor}
\PassOptionsToPackage{square,numbers}{natbib}
\documentclass{scrreprt}

\usepackage[left=2cm,right=2cm]{geometry}
\usepackage[svgnames]{xcolor}
\usepackage{peeters_layout}

\usepackage{natbib}

\usepackage[
colorlinks=true,
bookmarks=false,
pdfauthor={\authorname, \email},
backref 
]{hyperref}

% http://tex.stackexchange.com/questions/75773/how-to-reference-problems-by-the-text-label-in-an-exercise-envioronment
\usepackage[english]{cleveref}
\crefname{Exercise}{exercise}{exercises}
\Crefname{Exercise}{Exercise}{Exercises}

\RequirePackage{titlesec}
\RequirePackage{ifthen}

% http://stackoverflow.com/questions/4932910/date-in-the-tabular-environment
\makeatletter
\let\insertdate\@date
\makeatother

\titleformat{\chapter}[display]
{\bfseries\Large}
{\color{DarkSlateGrey}\filleft \authorname
\ifthenelse{\isundefined{\studentnumber}}{}{\\ \studentnumber}
\ifthenelse{\isundefined{\email}}{}{\\ \email}
\ifthenelse{\isundefined{\dateintitle}}{}{\\ \insertdate}
%\ifthenelse{\isundefined{\coursename}}{}{\\ \coursename} % put in title instead.
}
{4ex}
{\color{DarkOliveGreen}{\titlerule}\color{Maroon}
\vspace{2ex}%
\filright}
[\vspace{2ex}%
\color{DarkOliveGreen}\titlerule
]

\newcommand{\beginArtWithToc}[0]{\begin{document}\tableofcontents}
\newcommand{\beginArtNoToc}[0]{\begin{document}}
\newcommand{\EndNoBibArticle}[0]{\end{document}}
\newcommand{\EndArticle}[0]{\bibliography{Bibliography}\bibliographystyle{plainnat}\end{document}}

% 
%\newcommand{\citep}[1]{\cite{#1}}

\colorSectionsForArticle



\usepackage{peeters_layout_exercise}
\usepackage{peeters_braket}
\usepackage{peeters_figures}

\beginArtNoToc

\generatetitle{Unimodular transformation}
%\chapter{Unimodular transformation}
%\label{chap:unimodularAndRotation}

\paragraph{Q: Show that }

Given the matrix

\begin{dmath}\label{eqn:unimodularAndRotation:20}
U = 
\frac
{a_0 + i \sigma \cdot \Ba}
{a_0 - i \sigma \cdot \Ba},
\end{dmath}

where \( a_0, \Ba \) are real valued constant and vector respectively.

\begin{itemize}
\item Show that this is a unimodular and unitary transformation.
\item A unitary transformation can represent an arbitary rotation.  Determine the rotation angle and direction in terms of \( a_0, \Ba \).
\end{itemize}

\paragraph{A:}

Let's call these factors \( A_{\pm} \), which expand to

\begin{dmath}\label{eqn:unimodularAndRotation:40}
A_{\pm} 
=
a_0 \pm i \sigma \cdot \Ba
=
\begin{bmatrix}
a_0 \pm i a_z  & \pm \lr{ a_y + i a_x} \\
\mp (a_y - i a_x) & a_0 \mp i a_z \\
\end{bmatrix},
\end{dmath}

or with \( z = a_0 + i a_z \), and \( w = a_y + i a_x \), these are

\begin{dmath}\label{eqn:unimodularAndRotation:120}
A_{+}
=
\begin{bmatrix}
z & w \\
-w^\conj & z^\conj
\end{bmatrix}
\end{dmath}
\begin{dmath}\label{eqn:unimodularAndRotation:180}
A_{-}
=
\begin{bmatrix}
z^\conj & -w \\
w^\conj & z
\end{bmatrix}.
\end{dmath}

These both have a determinant of
\begin{dmath}\label{eqn:unimodularAndRotation:60}
\Abs{z}^2 + \Abs{w}^2
=
\Abs{a_0 + i a_z}^2 + \Abs{a_y + i a_x}^2 
= a_0^2 + \Ba^2.
\end{dmath}

The inverse of the latter is
\begin{dmath}\label{eqn:unimodularAndRotation:200}
A_{-}^{-1}
=
\inv{ a_0^2 + \Ba^2 }
\begin{bmatrix}
z & w \\
-w^\conj & z^\conj
\end{bmatrix}
\end{dmath}

Noting that the numerator and denominator commute the inverse can be applied in either order.  Picking one, the transformation of interest, after writing \( A = a_0^2 + \Ba^2 \), is

\begin{dmath}\label{eqn:unimodularAndRotation:100}
U = 
\inv{A}
\begin{bmatrix}
z & w \\
-w^\conj & z^\conj
\end{bmatrix} 
\begin{bmatrix}
z & w \\
-w^\conj & z^\conj
\end{bmatrix}
=
\inv{A}
\begin{bmatrix}
z^2 - \Abs{w}^2          & w( z + z^\conj) \\
-w^\conj (z^\conj + z )  & (z^\conj)^2 - \Abs{w}^2
\end{bmatrix}.
\end{dmath}

Recall that a unimodular transformation is one that has the form 

\begin{dmath}\label{eqn:unimodularAndRotation:140}
\begin{bmatrix}
z & w \\
-w^\conj & z^\conj
\end{bmatrix},
\end{dmath}

provided \( \Abs{z}^2 + \Abs{w}^2 = 1 \), so \cref{eqn:unimodularAndRotation:100} is unimodular if the following sum is unity, which is the case

\begin{dmath}\label{eqn:unimodularAndRotation:160}
\frac{\Abs{z^2 - \Abs{w}^2}^2}{\lr{ \Abs{z}^2 + \Abs{w}^2}^2 } + \Abs{w}^2 \frac{\Abs{z + z^\conj}^2 }{\lr{ \Abs{z}^2 + \Abs{w}^2}^2 }
=
\frac{
\lr{ z^2 - \Abs{w}^2 } \lr{ (z^\conj)^2 - \Abs{w}^2 }
+ \Abs{w}^2 \lr{ z + z^\conj }^2
}{
\lr{ \Abs{z}^2 + \Abs{w}^2}^2
}
=
\frac{
\Abs{z}^4 + \Abs{w}^4 - \Abs{w}^2 \lr{ \cancel{z^2 + (z^\conj)^2} }
+ \Abs{w}^2 \lr{ \cancel{z^2 + (z^\conj)^2} + 2 \Abs{z}^2 }
}{
\lr{ \Abs{z}^2 + \Abs{w}^2}^2
}
= 1.
\end{dmath}

\EndArticle
