%
% Copyright � 2015 Peeter Joot.  All Rights Reserved.
% Licenced as described in the file LICENSE under the root directory of this GIT repository.
%
\newcommand{\authorname}{Peeter Joot}
\newcommand{\email}{peeterjoot@protonmail.com}
\newcommand{\basename}{FIXMEbasenameUndefined}
\newcommand{\dirname}{notes/FIXMEdirnameUndefined/}

\renewcommand{\basename}{unimodularAndRotation}
\renewcommand{\dirname}{notes/phy1520/}
%\newcommand{\dateintitle}{}
%\newcommand{\keywords}{}

\newcommand{\authorname}{Peeter Joot}
\newcommand{\onlineurl}{http://sites.google.com/site/peeterjoot2/math2013/\basename.pdf}
\newcommand{\sourcepath}{\dirname\basename.tex}
\newcommand{\generatetitle}[1]{\chapter{#1}}

\newcommand{\vcsinfo}{%
\section*{}
\noindent{\color{DarkOliveGreen}{\rule{\linewidth}{0.1mm}}}
\paragraph{Document version}
%\paragraph{\color{Maroon}{Document version}}
{
\small
\begin{itemize}
\item Available online at:\\ 
\href{\onlineurl}{\onlineurl}
\item Git Repository: \input{./.revinfo/gitRepo.tex}
\item Source: \sourcepath
\item last commit: \input{./.revinfo/gitCommitString.tex}
\item commit date: \input{./.revinfo/gitCommitDate.tex}
\end{itemize}
}
}

%\PassOptionsToPackage{dvipsnames,svgnames}{xcolor}
\PassOptionsToPackage{square,numbers}{natbib}
\documentclass{scrreprt}

\usepackage[left=2cm,right=2cm]{geometry}
\usepackage[svgnames]{xcolor}
\usepackage{peeters_layout}

\usepackage{natbib}

\usepackage[
colorlinks=true,
bookmarks=false,
pdfauthor={\authorname, \email},
backref 
]{hyperref}

% http://tex.stackexchange.com/questions/75773/how-to-reference-problems-by-the-text-label-in-an-exercise-envioronment
\usepackage[english]{cleveref}
\crefname{Exercise}{exercise}{exercises}
\Crefname{Exercise}{Exercise}{Exercises}

\RequirePackage{titlesec}
\RequirePackage{ifthen}

% http://stackoverflow.com/questions/4932910/date-in-the-tabular-environment
\makeatletter
\let\insertdate\@date
\makeatother

\titleformat{\chapter}[display]
{\bfseries\Large}
{\color{DarkSlateGrey}\filleft \authorname
\ifthenelse{\isundefined{\studentnumber}}{}{\\ \studentnumber}
\ifthenelse{\isundefined{\email}}{}{\\ \email}
\ifthenelse{\isundefined{\dateintitle}}{}{\\ \insertdate}
%\ifthenelse{\isundefined{\coursename}}{}{\\ \coursename} % put in title instead.
}
{4ex}
{\color{DarkOliveGreen}{\titlerule}\color{Maroon}
\vspace{2ex}%
\filright}
[\vspace{2ex}%
\color{DarkOliveGreen}\titlerule
]

\newcommand{\beginArtWithToc}[0]{\begin{document}\tableofcontents}
\newcommand{\beginArtNoToc}[0]{\begin{document}}
\newcommand{\EndNoBibArticle}[0]{\end{document}}
\newcommand{\EndArticle}[0]{\bibliography{Bibliography}\bibliographystyle{plainnat}\end{document}}

% 
%\newcommand{\citep}[1]{\cite{#1}}

\colorSectionsForArticle



\usepackage{peeters_layout_exercise}
\usepackage{peeters_braket}
\usepackage{peeters_figures}

\beginArtNoToc

\generatetitle{Unimodular transformation}
%\chapter{Unimodular transformation}
%\label{chap:unimodularAndRotation}

\paragraph{Q: Show that }

Given the matrix

\begin{dmath}\label{eqn:unimodularAndRotation:n}
U = 
\frac
{a_0 + i \sigma \cdot \Ba}
{a_0 - i \sigma \cdot \Ba},
\end{dmath}

where \( a_0 \) is a real constant, and \( \Ba \) a real vector,

\begin{itemize}
\item show that this is a unimodular and unitary transformation.
\item A unitary transformation can represent an arbitary rotation.  Determine the rotation angle and direction in terms of \( a_0, \Ba \).
\end{itemize}

\paragraph{A:}

The factors expand to

\begin{dmath}\label{eqn:unimodularAndRotation:n}
a_0 \pm i \sigma \cdot \Ba
=
\begin{bmatrix}
a_0 \pm i a_z  & a_y + i a_x \\
-(a_y - i a_x) & -(a_0 \pm i a_z) \\
\end{bmatrix}.
\end{dmath}

This has determinant
\begin{dmath}\label{eqn:unimodularAndRotation:n}
-(a_0 \pm i a_z)^2 + \Abs{a_y + i a_x}^2
\end{dmath}

%The inverse of the denominator is
%\begin{dmath}\label{eqn:unimodularAndRotation:n}
%\lr{a_0 - i \sigma \cdot \Ba}^{-1}
%=
%\inv{ -(a_0 - i a_z)^2
%\begin{bmatrix}
%\end{bmatrix}
%\end{dmath}

\EndArticle
%\EndNoBibArticle
