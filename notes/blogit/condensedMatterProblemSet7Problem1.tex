%
% Copyright � 2013 Peeter Joot.  All Rights Reserved.
% Licenced as described in the file LICENSE under the root directory of this GIT repository.
.%
\makeproblem{Thomas-Fermi screening}{condensedMatter:problemSet7:1}{ 

\makesubproblem{}{condensedMatter:problemSet7:1a}
For $f(E) = 1/(e^{(E-\mu)/{\kB T}} + 1)$ (i.e. the Fermi-Dirac distribution function), show
that in the limit as $T \rightarrow 0$ K, $-\PDi{E}{f}$ has the following properties expected of the
Dirac delta-function:

\begin{itemize}
\item It is zero everywhere, except at $E = \mu$ where it is infinite;
\item $- \int_{-\infty}^\infty dE \PDi{E}{f} = 1$
\end{itemize}

\paragraph{Note}: you may know that the Dirac delta function is the derivative of the so-called
Heaviside function, so the correspondence between $\PDi{E}{f}$ and $\delta(E -\mu)$ is not a
surprise.

\makesubproblem{}{condensedMatter:problemSet7:1b}
Consider an externally applied periodic charge density wave of the form $\delta \rho_\nought(x) = \delta \rho_\nought \cos( q x)$, inside a metal. In practice this could be a modulation of the ionic charge density due to a static or dynamic charge density wave.

Show, using a result we derived in class, $\delta \rho_{\mathrm{el}}(\Br) = - e^2 D(\EF) U(\Br)$, that the induced electron density due to this applied periodic charge density wave is

\begin{dmath}\label{eqn:condensedMatterProblemSet2Problem1:20}
\delta \rho_{\mathrm{el}}(x) = 
-\frac{e^2 D(\EF)}{\epsilon_\nought}
\frac{\delta \rho_\nought}{q^2 + \kappa^2} 
\cos(q x),
\end{dmath}

where $\kappa^2 = e^2 D(\EF)/\epsilon_\nought$.
} % makeproblem

\makeanswer{condensedMatter:problemSet7:1}{ 
\makeSubAnswer{}{condensedMatter:problemSet7:1a}

TODO.
\makeSubAnswer{}{condensedMatter:problemSet7:1b}

TODO.
}
