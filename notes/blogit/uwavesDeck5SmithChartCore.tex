%
% Copyright © 2016 Peeter Joot.  All Rights Reserved.
% Licenced as described in the file LICENSE under the root directory of this GIT repository.
%

\section{Short circuited line}

F1

With 
\begin{equation}\label{eqn:uwavesDeck5SmithChartCore:20}
Z_\txtL = 0,
\end{equation}

the input impedance is

\begin{equation}\label{eqn:uwavesDeck5SmithChartCore:40}
Z_{\textrm{in}} 
= Z_0 \frac{ Z_\txtL + j Z_0 \tan(\beta l) }{ Z_0 + j Z_\txtL \tan(\beta l)}
= j Z_0 \tan(\beta l)
\end{equation}

%At the load
%
%\begin{dmath}\label{eqn:uwavesDeck5SmithChartCore:60}
%\Gamma_\txtL
%= \frac{Z_\txtL - Z_0}{Z_\txtL + Z_0}
%= -1,
%\end{dmath}
%
%so

For short line sections \( \beta l \ll \pi/2 \), or \( l \ll \lambda/4 \), the input impedance is approximately

\begin{dmath}\label{eqn:uwavesDeck5SmithChartCore:80}
Z_{\textrm{in}} 
= j Z_0 \tan(\beta l)
\approx j Z_0 \sin(\beta l)
\approx j Z_0 \beta l
\end{dmath}

Introducing an equivalent inductance defined by \( Z_{\textrm{in}} = j \omega L_{\mathrm{eq}} \), we have

\begin{dmath}\label{eqn:uwavesDeck5SmithChartCore:100}
L_{\mathrm{eq}}
=
\frac{Z_0}{\omega} \beta l 
=
\frac{Z_0}{\omega} \frac{\omega}{v_\phi} l 
=
\frac{Z_0 l}{v_\phi}.
\end{dmath}

The inductance per unit length of the line is \( C = Z_0/v_\phi \).  

This is also the case for short sections of high impedance line.

\section{Open circuited line}

F2

This time with \( Z_\txtL \rightarrow \infty \) we have

\begin{dmath}\label{eqn:uwavesDeck5SmithChartCore:120}
Z_{\textrm{in}} 
= Z_0 \frac{ Z_\txtL + j Z_0 \tan(\beta l) }{ Z_0 + j Z_\txtL \tan(\beta l)}
= -j Z_0 \cot(\beta l).
\end{dmath}

This time we have an equivalent capacitance.  For short sections with \( \beta l \ll \pi/2 \)

\begin{dmath}\label{eqn:uwavesDeck5SmithChartCore:140}
Z_{\textrm{in}}
\approx
-j \frac{Z_0}{\beta l}
\end{dmath}

Introducing an equivalent capacitance defined by \( Z_{\textrm{in}} = 1/(j \omega C_{\mathrm{eq}}) \), we have

\begin{dmath}\label{eqn:uwavesDeck5SmithChartCore:160}
C_{\mathrm{eq}}
=
\frac{ \beta l}{\omega Z_0}
=
\frac{ \omega/v_\phi l}{\omega Z_0}
=
\frac{ l}{v_\phi Z_0}
\end{dmath}

The capacitance per unit length of the line is \( C = 1/(Z_0 v_\phi) \).

This is also the case for short sections of low impedance line.

\section{Half wavelength transformer.}

F3

With \( l = \lambda/2 \)

\begin{dmath}\label{eqn:uwavesDeck5SmithChartCore:180}
\beta l 
= \frac{2 \pi}{\lambda} \frac{\lambda}{2}
= \pi.
\end{dmath}

Since \( \tan \pi = 0 \), the input impededence is

\begin{dmath}\label{eqn:uwavesDeck5SmithChartCore:200}
Z_{\textrm{in}} 
= Z_0 \frac{ Z_\txtL + j Z_0 \tan(\beta l) }{ Z_0 + j Z_\txtL \tan(\beta l)}
= Z_\txtL.
\end{dmath}

\section{Quarter wavelength transformer.}

F4

With \( l = \lambda/4 \)

\begin{dmath}\label{eqn:uwavesDeck5SmithChartCore:220}
\beta l 
= \frac{2 \pi}{\lambda} \frac{\lambda}{4}
= \frac{\pi}{2}.
\end{dmath}

We have \( \tan \beta l \rightarrow \infty \), so the input impededence is

\begin{dmath}\label{eqn:uwavesDeck5SmithChartCore:240}
Z_{\textrm{in}} 
= Z_0 \frac{ Z_\txtL + j Z_0 \tan(\beta l) }{ Z_0 + j Z_\txtL \tan(\beta l)}
= \frac{Z_0^2}{Z_\txtL}.
\end{dmath}

This relation

%\begin{equation}\label{eqn:uwavesDeck5SmithChartCore:280}
\boxedEquation{eqn:uwavesDeck5SmithChartCore:280}
{
Z_{\textrm{in}} 
= \frac{Z_0^2}{Z_\txtL},
}
%\end{equation}

is called the \textAndIndex{impedence inverter}.

\begin{itemize}
\item A large impedance is transformed into a small one and vice-versa.
\item A short becomes an open and vice-versa.
\item A capacitive load becomes inductive and vice-versa.
\item If \( Z_\txtL \) is a series resonant circuit then \( Z_{\textrm{in}} \) becomes parallel resonant.
\end{itemize}

FIXME: what does series resonant and parallel resonant mean?

\paragraph{Matching with a \( \lambda/4 \) transformer.}

F5

For maximum power transfer

\begin{equation}\label{eqn:uwavesDeck5SmithChartCore:300}
Z_{\textrm{in}} = \frac{Z_0^2}{R_\txtL} = R_\txtG, 
\end{equation}

so

\begin{equation}\label{eqn:uwavesDeck5SmithChartCore:320}
Z_0 = \sqrt{ R_\txtG R_\txtL }.
\end{equation}

We have

\begin{equation}\label{eqn:uwavesDeck5SmithChartCore:340}
\Abs{\Gamma_\txtL} = \frac{ R_\txtL - Z_0 }{R_\txtL + Z_0} \ne 0,
\end{equation}

and still maximum power is transferred.

\section{Smith chart}

A Smith chart is a graphical tool for making the transformation \( \Gamma \leftrightarrow Z_{\textrm{in}} \).  Given

\begin{equation}\label{eqn:uwavesDeck5SmithChartCore:360}
Z_{\textrm{in}} = Z_0 \frac{ 1 + \Gamma }{ 1 - \Gamma },
\end{equation}

where \( \Gamma = \Gamma_\txtL e^{- 2 j \beta l } \), we begin by normalizing the input impedance

\begin{equation}\label{eqn:uwavesDeck5SmithChartCore:380}
Z_{\textrm{in}} \rightarrow \overbar{Z}_{\textrm{in}} = \frac{Z_{\textrm{in}}}{Z_0}, 
\end{equation}

so

\begin{dmath}\label{eqn:uwavesDeck5SmithChartCore:400}
\overbar{Z}_{\textrm{in}} 
= \frac{ 1 + \Gamma }{ 1 - \Gamma }
= \frac{ (1 + \Gamma_r) + j \Gamma_i }{ (1 - \Gamma_r) - j \Gamma_i }
= \frac{ \lr{ (1 + \Gamma_r) + j \Gamma_i}\lr{(1 - \Gamma_r) + j \Gamma_i} }{ (1 - \Gamma_r)^2 + \Gamma_i^2 }
= \frac{ (1 - \Gamma_r^2 - \Gamma_i^2) + j \Gamma_i (1 - \Gamma_r + 1 + \Gamma_r ) }{ (1 - \Gamma_r)^2 + \Gamma_i^2 }
= \frac{ (1 - \Abs{\Gamma}^2) + 2 j \Gamma_i }{ (1 - \Gamma_r)^2 + \Gamma_i^2 }.
\end{dmath}

If we let \( \overbar{Z}_{\textrm{in}} = \overbar{\Gamma}_\txtL + j \overbar{X}_\txtL \), and equate real and imaginary parts we have

\begin{equation}\label{eqn:uwavesDeck5SmithChartCore:420}
\begin{aligned}
\overbar{\Gamma}_\txtL &= \frac{ 1 - \Abs{\Gamma}^2 }{ (1 - \Gamma_r)^2 + \Gamma_i^2 }
\overbar{X}_\txtL &= \frac{2 \Gamma_i }{ (1 - \Gamma_r)^2 + \Gamma_i^2 }
\end{aligned}
\end{equation}

\section{Feb 3}

F1

%Z_L = 20 + 25 j
%
%Z_A = Z_L/Z_0 = 0.2 + 0.5 j

F2
