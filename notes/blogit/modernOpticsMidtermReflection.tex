%
% Copyright � 2012 Peeter Joot.  All Rights Reserved.
% Licenced as described in the file LICENSE under the root directory of this GIT repository.
%
% pick one:
%\newcommand{\authorname}{Peeter Joot}
\newcommand{\email}{peeter.joot@utoronto.ca}
\newcommand{\studentnumber}{920798560}
\newcommand{\basename}{FIXMEbasenameUndefined}
\newcommand{\dirname}{notes/FIXMEdirnameUndefined/}

\newcommand{\authorname}{Peeter Joot}
\newcommand{\email}{peeterjoot@protonmail.com}
\newcommand{\basename}{FIXMEbasenameUndefined}
\newcommand{\dirname}{notes/FIXMEdirnameUndefined/}

\renewcommand{\basename}{modernOpticsMidtermReflection}
\renewcommand{\dirname}{notes/phy485/}
%\newcommand{\dateintitle}{}
%\newcommand{\keywords}{}

\newcommand{\authorname}{Peeter Joot}
\newcommand{\onlineurl}{http://sites.google.com/site/peeterjoot2/math2013/\basename.pdf}
\newcommand{\sourcepath}{\dirname\basename.tex}
\newcommand{\generatetitle}[1]{\chapter{#1}}

\newcommand{\vcsinfo}{%
\section*{}
\noindent{\color{DarkOliveGreen}{\rule{\linewidth}{0.1mm}}}
\paragraph{Document version}
%\paragraph{\color{Maroon}{Document version}}
{
\small
\begin{itemize}
\item Available online at:\\ 
\href{\onlineurl}{\onlineurl}
\item Git Repository: \input{./.revinfo/gitRepo.tex}
\item Source: \sourcepath
\item last commit: \input{./.revinfo/gitCommitString.tex}
\item commit date: \input{./.revinfo/gitCommitDate.tex}
\end{itemize}
}
}

%\PassOptionsToPackage{dvipsnames,svgnames}{xcolor}
\PassOptionsToPackage{square,numbers}{natbib}
\documentclass{scrreprt}

\usepackage[left=2cm,right=2cm]{geometry}
\usepackage[svgnames]{xcolor}
\usepackage{peeters_layout}

\usepackage{natbib}

\usepackage[
colorlinks=true,
bookmarks=false,
pdfauthor={\authorname, \email},
backref 
]{hyperref}

% http://tex.stackexchange.com/questions/75773/how-to-reference-problems-by-the-text-label-in-an-exercise-envioronment
\usepackage[english]{cleveref}
\crefname{Exercise}{exercise}{exercises}
\Crefname{Exercise}{Exercise}{Exercises}

\RequirePackage{titlesec}
\RequirePackage{ifthen}

% http://stackoverflow.com/questions/4932910/date-in-the-tabular-environment
\makeatletter
\let\insertdate\@date
\makeatother

\titleformat{\chapter}[display]
{\bfseries\Large}
{\color{DarkSlateGrey}\filleft \authorname
\ifthenelse{\isundefined{\studentnumber}}{}{\\ \studentnumber}
\ifthenelse{\isundefined{\email}}{}{\\ \email}
\ifthenelse{\isundefined{\dateintitle}}{}{\\ \insertdate}
%\ifthenelse{\isundefined{\coursename}}{}{\\ \coursename} % put in title instead.
}
{4ex}
{\color{DarkOliveGreen}{\titlerule}\color{Maroon}
\vspace{2ex}%
\filright}
[\vspace{2ex}%
\color{DarkOliveGreen}\titlerule
]

\newcommand{\beginArtWithToc}[0]{\begin{document}\tableofcontents}
\newcommand{\beginArtNoToc}[0]{\begin{document}}
\newcommand{\EndNoBibArticle}[0]{\end{document}}
\newcommand{\EndArticle}[0]{\bibliography{Bibliography}\bibliographystyle{plainnat}\end{document}}

% 
%\newcommand{\citep}[1]{\cite{#1}}

\colorSectionsForArticle



\beginArtNoToc

\generatetitle{Midterm Reflection: Lloyd's mirror}
\chapter{Lloyd's mirror}
\label{chap:modernOpticsMidtermReflection}
\section{Motivation}

I got confused about the geometry in a Lloyd's mirror configration on the midterm.  After a more careful figure, it becomes more clear how to (correctly) tackle the problem.  Here's a new try after the fact.

\makeproblem{Lloyd's mirror.  Temporal coherence}{modernOpticsMidtermReflection:pr:1}{

Calculate the coherence length for the Lloyd's mirror configuration of \cref{fig:modernOpticsMidtermReflection:modernOpticsMidtermReflectionFig1}.

\imageFigure{modernOpticsMidtermReflectionFig1}{Lloyd's mirror}{fig:modernOpticsMidtermReflection:modernOpticsMidtermReflectionFig1}{0.3}
}
\makeanswer{modernOpticsMidtermReflection:pr:1}{ 

We want to consider the pathlength differences along the direct path $c$, to that of $b = 2 a + b_2$.  Noting that I've made the unfortunate choice of $c$ for ohe of these paths, not to be confused here with the speed of light, the lengths of these paths are

\begin{subequations}
\begin{equation}\label{eqn:modernOpticsMidtermReflection:20}
c = \sqrt{ L^2 + x^2 }
\end{equation}
\begin{equation}\label{eqn:modernOpticsMidtermReflection:40}
b = \sqrt{L^2 + (2 h + x)^2}
\end{equation}
\end{subequations}

The pathlength difference, for $x, h \ll L$ is then
\begin{dmath}\label{eqn:modernOpticsMidtermReflection:60}
b - c 
=
L 
\left(
\sqrt{1 + \left( \frac{2 h + x}{L} \right)^2}
-
\sqrt{ 1 + \frac{x^2}{L^2} }
\right)
\sim
L \inv{2 L^2}( 4 h^2 + 4 h x )
= \frac{2 h}{L}( h + x )
\end{dmath}

So, supposing that we have a spherical wave emitted from the point source, allowing for a different (random) in each direction, the waves that arrive at the observation point, having travelled along paths $c$ and $b$ respectively are

\begin{subequations}
\begin{dmath}\label{eqn:modernOpticsMidtermReflection:80}
\Psi_c = \frac{\Psi_0}{c} e^{i k c - i \omega t + \phi_c(t)}
\end{dmath}
\begin{dmath}\label{eqn:modernOpticsMidtermReflection:100}
\Psi_b = \frac{\Psi_0}{b} e^{i k b - i \omega t + \phi_b(t)}
\end{dmath}
\end{subequations}

Our average intensity, should there be a time delay of $\tau$ in the $b$ path, is

\begin{dmath}\label{eqn:modernOpticsMidtermReflection:120}
I(\tau) 
= \expectation{ 
\left( \Psi_c(t) + \Psi_b(t + \tau) \right)
\left( \Psi_c^\conj(t) + \Psi_b^\conj(t + \tau) \right)
}
= 
\expectation{ \Abs{\Psi_c(t)}^2 }
+
\expectation{ \Abs{\Psi_b(t + \tau)}^2 }
+
2 \Real
\expectation{
\frac{\Psi_0}{c} e^{i k c - i \omega t + \phi_c(t)}
\frac{\Psi_0^\conj}{b} e^{-i k b + i \omega (t + \tau) - \phi_b(t + \tau)}
}
=
I_c + I_b 
+ 2 \Real \left(
\frac{\Psi_0}{c} 
\frac{\Psi_0^\conj}{b} 
e^{ i \omega \tau}
e^{ i k (c - b) }
\expectation{
e^{ i \phi_c(t) - i \phi_b(t + \tau) }
}
\right)
\sim
I_c + I_b 
+ 2 \Real \left(
\frac{\Psi_0}{c} 
\frac{\Psi_0^\conj}{b} 
e^{ i \omega \tau}
e^{ i k (c - b) }
\left( 1 - \frac{\tau}{\tau_0} \right) 
\right)
\Theta(\tau_0 - \tau)
\end{dmath}

The interference term, scaling by requiring a unit value at $\tau = 0$, and writing $x' = x + h$ (effectively repositioning our origin) is therefore

\begin{dmath}\label{eqn:modernOpticsMidtermReflection:140}
\gamma(x', \tau) \sim
\cos \left( 
\frac{2 k h}{L} x'
- \omega \tau
\right)
\left( 1 - \frac{\tau}{\tau_0} \right) 
\Theta(\tau_0 - \tau)
\end{dmath}

This has a maximum at $(x', \tau) = (0, 0)$.  We've also got a set of level curves, marking the amplitudes of equal magnitude

\begin{dmath}\label{eqn:modernOpticsMidtermReflection:160}
\tau = \frac{2 k h x'}{L \omega} = \frac{2 h x' c}{L} = \text{constant}.
\end{dmath}

This isn't at all what I recall calculating on the midterm.  There must have been something else in the statement of the Lloyd's mirror problem that I've forgotten.  Could it have been a Gaussian wave packet emitted from the source?  Yes, that was it.  We were told to assume that our source had a Gaussian frequency distribution.  Let's play around with evolving things and suppose that we slightly generalize the spherical waves we'd interfered by allowing a superposition of the form

\begin{dmath}\label{eqn:modernOpticsMidtermReflection:180}
\Psi_i = \frac{1}{\sqrt{2 \pi} r_i} \int d\omega \tilde{\Psi}_0(\omega) e^{i \omega \left( r_i/c - t \right) + \phi_i(t)}.
\end{dmath}

Now, if we delay one such wave function in time at the source by time $\tau$, our resultant field is

\begin{dmath}\label{eqn:modernOpticsMidtermReflection:200}
\Psi = 
\frac{1}{\sqrt{2 \pi} r_1} \int d\omega \tilde{\Psi}_0(\omega) e^{i \omega \left( r_1/c - t \right) + i \phi_1(t)}
+\frac{1}{\sqrt{2 \pi} r_2} \int d\omega \tilde{\Psi}_0(\omega) e^{i \omega \left( r_2/c - t - \tau \right) + i\phi_2(t + \tau)},
\end{dmath}

with average intensity proportional to

\begin{dmath}\label{eqn:modernOpticsMidtermReflection:220}
I = I_1 + I_2 + 
\frac{\Abs{\Psi_0}^2}{\pi r_1 r_2} 
\Real
\expectation{
\int d\omega \tilde{\Psi}_0(\omega) e^{i \omega \left( r_1/c - t \right) + i \phi_1(t)}
\int d\omega' {\tilde{\Psi}}^\conj_0(\omega') e^{-i \omega' \left( r_2/c - t - \tau \right) - i\phi_2(t + \tau)}
}
\end{dmath}

Letting
\begin{dmath}\label{eqn:modernOpticsMidtermReflection:240}
\Psi_0(t) = 
\frac{1}{\sqrt{2 \pi}} \int d\omega \tilde{\Psi}_0(\omega) e^{-i \omega  t},
\end{dmath}

we have for the interference term

\begin{dmath}\label{eqn:modernOpticsMidtermReflection:260}
\Gamma(\tau) = \frac{2}{r_1 r_2 T} \int_{-T/2}^{T/2} dt
\Psi_0(t - r_1/c) 
\Psi_0^\conj(t + \tau - r_2/c ) 
e^{i \phi_1(t) - i \phi_2(t + \tau)}.
\end{dmath}

Clearly this will be more tractable if we fold the random phase functions $\phi_i(t)$ into the Fourier integrals, as in

\begin{dmath}\label{eqn:modernOpticsMidtermReflection:280}
\Psi_i = \frac{1}{\sqrt{2 \pi} r_i} \int d\omega \tilde{\Psi}_0(\omega) e^{i \omega \left( r_i/c - t \right)}.
\end{dmath}

so that our interference term is reduced to
\begin{dmath}\label{eqn:modernOpticsMidtermReflection:300}
\Gamma(\tau) 
%= \frac{2}{r_1 r_2 T} \int_{-T/2}^{T/2} dt
%\Psi_0(r_1/c - t) 
%\Psi_0^\conj(r_2/c - t - \tau) 
= \frac{2}{r_1 r_2 T} \int_{-T/2}^{T/2} dt
\Psi_0(t - r_1/c) 
\Psi_0^\conj(t + \tau - r_2/c) 
= \frac{2}{r_1 r_2 T} \int_{-T/2 - r_1/c}^{T/2 - r_1/c} du
\Psi_0(u) 
\Psi_0^\conj(u + \tau + (r_1 - r_2)/c) 
\end{dmath}

Now let's impose an additional constraint requiring for some finite $T$ that we have for $\Abs{u} > T$

\begin{dmath}\label{eqn:modernOpticsMidtermReflection:320}
\Psi_0(u) = 0
\end{dmath}

Ignoring the details about the range restriction we require for $\tau$ for now, our interference term will then be proportional to the plain old convolution integral, and we can reexpress this beastie in terms of assumed transform pairs

\begin{subequations}
\begin{dmath}\label{eqn:modernOpticsLecture13:360}
\tilde{\Psi_0}(\omega) = 
\frac{1}{\sqrt{2 \pi}} \int dt \Psi_0(t) e^{i \omega  t}
\end{dmath}
\begin{dmath}\label{eqn:modernOpticsLecture13:380}
\Psi_0(t) = 
\frac{1}{\sqrt{2 \pi}} \int d\omega \tilde{\Psi}_0(\omega) e^{-i \omega  t}
\end{dmath}
\end{subequations}

\begin{dmath}\label{eqn:modernOpticsMidtermReflection:340}
\Gamma(\tau) \sim \int_{-\infty}^\infty
\Psi_0(u) 
\Psi_0^\conj(u + \tau + (r_1 - r_2)/c) 
=
\inv{2\pi} \int du
d\omega \tilde{\Psi}_0(\omega) e^{i \omega  u}
d\omega' {\tilde{\Psi}}^\conj_0(\omega') e^{-i \omega' (u + \tau + (r_1 - r_2)/c)}
=
\int \delta(\omega - \omega')
d\omega \tilde{\Psi}_0(\omega) 
d\omega' {\tilde{\Psi}}^\conj_0(\omega') e^{-i \omega' (\tau + (r_1 - r_2)/c)}
=
\int 
d\omega \tilde{\Psi}_0(\omega) 
{\tilde{\Psi}}^\conj_0(\omega) e^{-i \omega (\tau + (r_1 - r_2)/c)}
=
\int 
d\omega \Abs{\tilde{\Psi}_0(\omega) }^2
e^{-i \omega (\tau + (r_1 - r_2)/c)}
\sim 
\evalbar{ \FF \left( \Abs{\tilde{\Psi}_0(\omega) }^2 \right) }{ t = \tau + (r_1 - r_2)/c}
\end{dmath}

} % makeanswer

%\section{Spatial coherence}

\vcsinfo
%\EndArticle
\EndNoBibArticle
