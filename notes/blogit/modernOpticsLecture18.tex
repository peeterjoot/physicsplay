%
% Copyright � 2012 Peeter Joot.  All Rights Reserved.
% Licenced as described in the file LICENSE under the root directory of this GIT repository.
%
\newcommand{\authorname}{Peeter Joot}
\newcommand{\email}{peeterjoot@protonmail.com}
\newcommand{\basename}{FIXMEbasenameUndefined}
\newcommand{\dirname}{notes/FIXMEdirnameUndefined/}

\renewcommand{\basename}{modernOpticsLecture18}
\renewcommand{\dirname}{notes/phy485/}
\newcommand{\keywords}{Optics, PHY485H1F}
\newcommand{\authorname}{Peeter Joot}
\newcommand{\onlineurl}{http://sites.google.com/site/peeterjoot2/math2013/\basename.pdf}
\newcommand{\sourcepath}{\dirname\basename.tex}
\newcommand{\generatetitle}[1]{\chapter{#1}}

\newcommand{\vcsinfo}{%
\section*{}
\noindent{\color{DarkOliveGreen}{\rule{\linewidth}{0.1mm}}}
\paragraph{Document version}
%\paragraph{\color{Maroon}{Document version}}
{
\small
\begin{itemize}
\item Available online at:\\ 
\href{\onlineurl}{\onlineurl}
\item Git Repository: \input{./.revinfo/gitRepo.tex}
\item Source: \sourcepath
\item last commit: \input{./.revinfo/gitCommitString.tex}
\item commit date: \input{./.revinfo/gitCommitDate.tex}
\end{itemize}
}
}

%\PassOptionsToPackage{dvipsnames,svgnames}{xcolor}
\PassOptionsToPackage{square,numbers}{natbib}
\documentclass{scrreprt}

\usepackage[left=2cm,right=2cm]{geometry}
\usepackage[svgnames]{xcolor}
\usepackage{peeters_layout}

\usepackage{natbib}

\usepackage[
colorlinks=true,
bookmarks=false,
pdfauthor={\authorname, \email},
backref 
]{hyperref}

% http://tex.stackexchange.com/questions/75773/how-to-reference-problems-by-the-text-label-in-an-exercise-envioronment
\usepackage[english]{cleveref}
\crefname{Exercise}{exercise}{exercises}
\Crefname{Exercise}{Exercise}{Exercises}

\RequirePackage{titlesec}
\RequirePackage{ifthen}

% http://stackoverflow.com/questions/4932910/date-in-the-tabular-environment
\makeatletter
\let\insertdate\@date
\makeatother

\titleformat{\chapter}[display]
{\bfseries\Large}
{\color{DarkSlateGrey}\filleft \authorname
\ifthenelse{\isundefined{\studentnumber}}{}{\\ \studentnumber}
\ifthenelse{\isundefined{\email}}{}{\\ \email}
\ifthenelse{\isundefined{\dateintitle}}{}{\\ \insertdate}
%\ifthenelse{\isundefined{\coursename}}{}{\\ \coursename} % put in title instead.
}
{4ex}
{\color{DarkOliveGreen}{\titlerule}\color{Maroon}
\vspace{2ex}%
\filright}
[\vspace{2ex}%
\color{DarkOliveGreen}\titlerule
]

\newcommand{\beginArtWithToc}[0]{\begin{document}\tableofcontents}
\newcommand{\beginArtNoToc}[0]{\begin{document}}
\newcommand{\EndNoBibArticle}[0]{\end{document}}
\newcommand{\EndArticle}[0]{\bibliography{Bibliography}\bibliographystyle{plainnat}\end{document}}

% 
%\newcommand{\citep}[1]{\cite{#1}}

\colorSectionsForArticle



\usepackage{tikz}

\usepackage[draft]{fixme}
\fxusetheme{color}

\beginArtNoToc
\generatetitle{PHY485H1F Modern Optics.  Lecture 18: Gaussian modes.  Taught by Prof.\ Joseph Thywissen}
%\chapter{Gaussian modes}
\label{chap:modernOpticsLecture18}

%\section{Disclaimer}
%
%Peeter's lecture notes from class.  May not be entirely coherent.

\section{Gaussian modes}

\fxwarning{review lecture 18}{work through this lecture in detail.}

READING: Yariv van Driel

We'll start thinking about transverse modes in a cavity.  Our starting point is Maxwell's equations

\begin{subequations}
\begin{dmath}\label{eqn:modernOpticsLecture18:20}
\spacegrad \cross \BH = \epsilon \PD{t}{\BE}
\end{dmath}
\begin{dmath}\label{eqn:modernOpticsLecture18:40}
\spacegrad \cross \BE = -\mu \PD{t}{\BH}
\end{dmath}
\begin{dmath}\label{eqn:modernOpticsLecture18:60}
\spacegrad \cdot \BB = 0
\end{dmath}
\begin{dmath}\label{eqn:modernOpticsLecture18:80}
\spacegrad \cdot \BE = 0.
\end{dmath}
\end{subequations}

We won't actually need the $\BE$ divergence equation, and will be looking for a wave equation where $\epsilon(\Br)$ varies ``slowly''.

Reminder

\begin{subequations}
\begin{equation}\label{eqn:modernOpticsLecture18:100}
n^2 = \frac{\epsilon}{\epsilon_0}
\end{equation}
\begin{equation}\label{eqn:modernOpticsLecture18:120}
v = \frac{c}{n} = \inv{\sqrt{\epsilon\mu}}
\end{equation}
\begin{equation}\label{eqn:modernOpticsLecture18:140}
k = \omega \sqrt{\epsilon \mu} = \frac{\omega n}{c}
\end{equation}
\end{subequations}

Recall that 

\begin{equation}\label{eqn:modernOpticsLecture18:160}
\spacegrad (\cross \spacegrad \BA) = \spacegrad (\spacegrad \cdot \BA) - \spacegrad^2 \BA
\end{equation}

Taking curls of both sides ...

\begin{equation}\label{eqn:modernOpticsLecture18:180}
\spacegrad^2 \BE - \mu \epsilon(\Br) \PDSq{t}{\BE} = 
\mathLabelBox{
-\spacegrad \left( \inv{\epsilon} \BE \cdot \spacegrad \epsilon \right)
}{Neglect this if $\epsilon$ varies slowly compared to $\lambda$}
\end{equation}

We suppose that the time dependence of the electric field is monochromatic, so that

\begin{equation}\label{eqn:modernOpticsLecture18:200}
\BE(\Br, t) = \BE(\Br) e^{-i \omega t}
\end{equation}

We then find

\begin{equation}\label{eqn:modernOpticsLecture18:220}
\Abs{
\spacegrad \left( \inv{\epsilon} \BE \cdot \spacegrad \epsilon \right)
} \ll k^2 \BE,
\end{equation}

so that

\begin{equation}\label{eqn:modernOpticsLecture18:240}
\boxed{
\spacegrad^2 \BE(\Br) + k^2(\Br) \BE(\Br) = 0.
}
\end{equation}

Choose $\epsilon(\Br)$ such that with $k_0 = \omega/c$, we have

\begin{equation}\label{eqn:modernOpticsLecture18:260}
\boxed{
k^2(\Br) = k_0^2 - k k_x r^2.
}
\end{equation}

\fxwarning{what was this connected to?}

\begin{equation}\label{eqn:modernOpticsLecture18:280}
\cdots + \omega^2 \mu \epsilon(\Br) \BE(\Br)
\equiv \left( \frac{n \omega}{c} \right)^2 - k k_2 r^2
\end{equation}

Perhaps this medium looks like

F1.

Let's look for solutions of the form

\begin{equation}\label{eqn:modernOpticsLecture18:n}
\BE = 
\mathLabelBox{
\BE_0
}{A vector, with a chosen polarity}
\mathLabelBox{
u(r, \theta, z) 
}{
Slowly varying (complex) envelope
}
e^{i k z}
\end{equation}

We can now work with a scalar amplitude

\begin{equation}\label{eqn:modernOpticsLecture18:n}
\Psi(r, \theta, z) = u e^{i k z}.
\end{equation}

Recall that our Laplacian in cylindrical coordinates is

\begin{equation}\label{eqn:modernOpticsLecture18:n}
\spacegrad^2 = 
\mathLabelBox{
\PDSq{r}{} + \inv{r} \PD{r}{}
}{$\spacegrad_{\mathrm{T}}^2$}
+ \inv{r^2} \PDSq{\theta}{}
+ \PDSq{z}{}
\end{equation}

We'll look for cylindrically symmetric solutions so that we can ignore the $\theta$ dependence in the Laplacian.

\begin{dmath}\label{eqn:modernOpticsLecture18:n}
\PDSq{z}{} u e^{i k z}
= 
\PD{z}{}
\left( 
\PD{z}{u}
e^{i k z}
+i k u 
e^{i k z}
\right)
=
\PDSq{z}{u} e^{i k z} + 2 i k \PD{Z}{U} e^{i k z} - k^2 u e^{i k z}.
\end{dmath}

Noting that 

\begin{dmath}\label{eqn:modernOpticsLecture18:n}
\spacegrad_T^2 u e^{i k z} = (\spacegrad_T^2 u \right) e^{i k z},
\end{dmath}

we can assemble and cancel the expoentials

\begin{dmath}\label{eqn:modernOpticsLecture18:n}
\boxed{
\PDSq{z}{u} + 2 i k \PD{z}{u} + \spacegrad_T^2 u - k k_2 r^2 u = 0
}
\end{dmath}

This is the \underline{paraxial wave equation}.  If 

\begin{dmath}\label{eqn:modernOpticsLecture18:n}
\Abs{ \PDSq{z}{u} } \ll k \Abs{ \PD{z}{u} },
\end{dmath}

so that $u$ is slowing varying on the wavelength scale, then we can neglect the first term

\begin{dmath}\label{eqn:modernOpticsLecture18:n}
\boxed{
2 i k \PD{z}{u} + \spacegrad_T^2 u - k k_2 r^2 u = 0
}
\end{dmath}

Also note that we've dropped our suffix for $k$ along the way somewhere $k_0 \rightarrow k$.

Also note that we didn't need to use cylindrical coordinates here, and could have grouped the transverse Laplacian as just

\begin{dmath}\label{eqn:modernOpticsLecture18:n}
\spacegrad_T^2 = 
\PDSq{x}{}
+\PDSq{y}{}
\end{dmath}

Let's rewrite this in a slightly different order

\begin{dmath}\label{eqn:modernOpticsLecture18:n}
\boxed{
\spacegrad_T^2 u 
- k k_2 r^2 u = 
-2 i k \PD{z}{u} 
}
\end{dmath}

Observe that this has the same form as the 2D Schr\"{o}dinger equation

\begin{dmath}\label{eqn:modernOpticsLecture18:n}
\Hcap = \inv{2 m} \pcap^2 + \inv{2} m \omega^2 (\xcap^2 + \ycap^2),
\end{dmath}

or in a position basis

\begin{dmath}\label{eqn:modernOpticsLecture18:n}
H \rightarrow -\frac{\hbar^2}{2m} 
\mathLabelBox{
\left( 
\PDSq{x}{}
+\PDSq{y}{}
\right)
}{$\spacegrad_T^2$}
+ \inv{2} m \omega^2 
\mathLabelBox{
(x^2 + y^2)
}{$r^2$}
\end{dmath}

\begin{dmath}\label{eqn:modernOpticsLecture18:n}
\spacegrad_T^2 \Psi
= -\frac{m^2 \omega^2}{\hbar^2} r^2 \Psi 
= 
- \frac{2 m i}{\hbar} \PD{t}{\Psi}
\end{dmath}



%\EndArticle
\EndNoBibArticle
