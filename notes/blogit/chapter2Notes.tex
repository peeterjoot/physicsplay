%
% Copyright � 2015 Peeter Joot.  All Rights Reserved.
% Licenced as described in the file LICENSE under the root directory of this GIT repository.
%
\newcommand{\authorname}{Peeter Joot}
\newcommand{\email}{peeterjoot@protonmail.com}
\newcommand{\basename}{FIXMEbasenameUndefined}
\newcommand{\dirname}{notes/FIXMEdirnameUndefined/}

\renewcommand{\basename}{chapter2Notes}
\renewcommand{\dirname}{notes/ece1229/}
%\newcommand{\dateintitle}{}
%\newcommand{\keywords}{}

\newcommand{\tiso}[0]{{\textrm{iso}}}
\newcommand{\tin}[0]{{\textrm{in}}}
\newcommand{\trad}[0]{{\textrm{rad}}}
\newcommand{\trms}[0]{{\textrm{rms}}}
\newcommand{\tmax}[0]{{\textrm{max}}}
\newcommand{\tA}[0]{{\textrm{A}}}
\newcommand{\tav}[0]{{\textrm{av}}}
%\newcommand{\tdB}[0]{{\textrm{dB}}}
%\newcommand{\tdBi}[0]{{\textrm{dBi}}}

%\newcommand{\calE}[0]{{\mathcal{E}}}
%\newcommand{\calH}[0]{{\mathcal{H}}}
%\newcommand{\calB}[0]{{\mathcal{B}}}
%\newcommand{\calW}[0]{{\mathcal{W}}}

% http://tex.stackexchange.com/a/2784/15
%\usepackage{bm}
%\newcommand{\bcE}[0]{{\bm{\mathcal{E}}}}
%\newcommand{\bcH}[0]{{\bm{\mathcal{H}}}}
%\newcommand{\bcB}[0]{{\bm{\mathcal{B}}}}
%\newcommand{\bcW}[0]{{\bm{\mathcal{W}}}}

\newcommand{\timeaverage}[1]{\left[#1\right]}

\newcommand{\chaptext}[0]{ch.}

\newcommand{\nbref}[1]{%
\itemRef{ece1229}{#1}%
}

% with an alternate label for the link.
% {nb}{text}
% nb of the form: ps2b:countItersAndPlot.m
\newcommand{\nbcite}[2]{%
\itemCite{ece1229}{#1}{#2}%
}

% \mathImageFigure{path}{caption}{label}{width}{nbpath}
% nbpath like: ps2b:countItersAndPlot.m
\newcommand{\mathImageFigure}[5]{%
\imageFigure{#1}{\nbcite{#5}{#2}}{#3}{#4}
}

\newcommand{\matlabFunc}[1]{%
\textbf{#1}%
}

% {func}{path}
\newcommand{\matlabFuncPath}[2]{%
\nbcite{#2}{\textbf{#1}}%
}

\newcommand{\authorname}{Peeter Joot}
\newcommand{\onlineurl}{http://sites.google.com/site/peeterjoot2/math2013/\basename.pdf}
\newcommand{\sourcepath}{\dirname\basename.tex}
\newcommand{\generatetitle}[1]{\chapter{#1}}

\newcommand{\vcsinfo}{%
\section*{}
\noindent{\color{DarkOliveGreen}{\rule{\linewidth}{0.1mm}}}
\paragraph{Document version}
%\paragraph{\color{Maroon}{Document version}}
{
\small
\begin{itemize}
\item Available online at:\\ 
\href{\onlineurl}{\onlineurl}
\item Git Repository: \input{./.revinfo/gitRepo.tex}
\item Source: \sourcepath
\item last commit: \input{./.revinfo/gitCommitString.tex}
\item commit date: \input{./.revinfo/gitCommitDate.tex}
\end{itemize}
}
}

%\PassOptionsToPackage{dvipsnames,svgnames}{xcolor}
\PassOptionsToPackage{square,numbers}{natbib}
\documentclass{scrreprt}

\usepackage[left=2cm,right=2cm]{geometry}
\usepackage[svgnames]{xcolor}
\usepackage{peeters_layout}

\usepackage{natbib}

\usepackage[
colorlinks=true,
bookmarks=false,
pdfauthor={\authorname, \email},
backref 
]{hyperref}

% http://tex.stackexchange.com/questions/75773/how-to-reference-problems-by-the-text-label-in-an-exercise-envioronment
\usepackage[english]{cleveref}
\crefname{Exercise}{exercise}{exercises}
\Crefname{Exercise}{Exercise}{Exercises}

\RequirePackage{titlesec}
\RequirePackage{ifthen}

% http://stackoverflow.com/questions/4932910/date-in-the-tabular-environment
\makeatletter
\let\insertdate\@date
\makeatother

\titleformat{\chapter}[display]
{\bfseries\Large}
{\color{DarkSlateGrey}\filleft \authorname
\ifthenelse{\isundefined{\studentnumber}}{}{\\ \studentnumber}
\ifthenelse{\isundefined{\email}}{}{\\ \email}
\ifthenelse{\isundefined{\dateintitle}}{}{\\ \insertdate}
%\ifthenelse{\isundefined{\coursename}}{}{\\ \coursename} % put in title instead.
}
{4ex}
{\color{DarkOliveGreen}{\titlerule}\color{Maroon}
\vspace{2ex}%
\filright}
[\vspace{2ex}%
\color{DarkOliveGreen}\titlerule
]

\newcommand{\beginArtWithToc}[0]{\begin{document}\tableofcontents}
\newcommand{\beginArtNoToc}[0]{\begin{document}}
\newcommand{\EndNoBibArticle}[0]{\end{document}}
\newcommand{\EndArticle}[0]{\bibliography{Bibliography}\bibliographystyle{plainnat}\end{document}}

% 
%\newcommand{\citep}[1]{\cite{#1}}

\colorSectionsForArticle



\usepackage{siunitx}
\usepackage{macros_bm}
\usepackage{macros_mathematica}

%\mathematicaListing{
%\begin{shaded}
%\begin{mat}
%\end{mat}
%\end{shaded}
%
%\usepackage{listings}
%\usepackage{framed}
%\usepackage{xcolor}
%\usepackage{amsmath}
%\colorlet{shadecolor}{gray!20}
%
%\lstnewenvironment{mat}
%{\lstset{language=mathematica,mathescape,columns=flexible}}
%{}
%

\beginArtNoToc

\generatetitle{Fundamental parameters of antennas}
%\chapter{Fundamental parameters of Antennas}
%\label{chap:chapter2Notes}

Reading notes for \chaptext 2 \citep{balanis2005antenna}.

\section{Typical far-field radiation intensity}

It was mentioned that

\begin{dmath}\label{eqn:advancedantennaL1:20}
U(\theta, \phi) 
= 
\frac{r^2}{2 \eta_0} \Abs{ \BE( r, \theta, \phi) }^2 
=
\frac{1}{2 \eta_0} \lr{ \Abs{ E_\theta(\theta, \phi) }^2 + \Abs{ E_\phi(\theta, \phi) }^2},
\end{dmath}

where the intrinsic impedance of free space is

\begin{dmath}\label{eqn:advancedantennaL1:480}
\eta_0 = \sqrt{\frac{\mu_0}{\epsilon_0}} = 377 \Omega.
\end{dmath}

(this is also eq. 2-19 in the text.)

To get an understanding where this comes from, consider the far field radial solutions to the electric and magnetic dipole problems, which have the respective forms (from \citep{griffith1981introduction}) of

\begin{subequations}
\begin{dmath}\label{eqn:chapter2Notes:740}
\begin{aligned}
\BE &= \frac{\mu_0 p_0 \omega^2 }{4 \pi } \frac{\sin\theta}{r} \cos\lr{w t - k r} \thetacap \\
\BB &= \frac{\mu_0 p_0 \omega^2 }{4 \pi c} \frac{\sin\theta}{r} \cos\lr{w t - k r} \phicap \\
\end{aligned}
\end{dmath}
\begin{dmath}\label{eqn:chapter2Notes:760}
\begin{aligned}
\BE &= \frac{\mu_0 m_0 \omega^2 }{4 \pi c} \frac{\sin\theta}{r} \cos\lr{w t - k r} \phicap \\
\BB &= \frac{\mu_0 m_0 \omega^2 }{4 \pi c^2} \frac{\sin\theta}{r} \cos\lr{w t - k r} \thetacap \\
\end{aligned}
\end{dmath}
\end{subequations}

In neither case is there a component in the direction of propagation, and in both cases

\begin{dmath}\label{eqn:chapter2Notes:780}
\Abs{\BH} 
= \frac{\Abs{\BE}}{\mu_0 c}
= \Abs{\BE} \sqrt{\frac{\epsilon_0}{\mu_0}}
= \Abs{\BE} /\eta_0.
\end{dmath}

FIXME: construct a superposition \( \alpha \BE_e + \beta \BE_m \) for these pairs of dipole solutions, and then work out the Poynting vector for the superposition.  It should supply the relation \cref{eqn:advancedantennaL1:20}.

\section{Poynting vector}

The Poynting vector was written in an unfamiliar form
 
\begin{dmath}\label{eqn:chapter2Notes:560}
\bcW = \bcE \cross \bcH.
\end{dmath}

I'm used to seeing the \( \ifrac{c}{4\pi} \) factor, and thought there was something like that in SI units too.  
Per \citep{griffith1981introduction} that something is a \( \mu_0 \), as in

\begin{dmath}\label{eqn:chapter2Notes:580}
\bcW = \inv{\mu_0} \bcE \cross \bcB.
\end{dmath}

Note that the use of \( \bcH \) instead of \( \bcB \) is what wipes out the requirement for the \( \ifrac{1}{\mu_0} \) term since \( \bcH = \bcB/\mu_0 \), at least in the abscence of magnetization.

\section{Field plots}

We can plot the fields, or intensity (or log plots in \si{dB} of these).
It is pointed out in 
\citep{griffith1981introduction}
that when there is \( r \) dependence these plots are done by considering the values of at fixed \( r \).

The field plots are conceptually the simplest, since that vector parametrizes a surface.  Any such radial field with magnitude \( f(r, \theta, \phi) \) can be plotted in Mathematica in the \( \phi = 0 \) plane at \( r = r_0 \), or in 3D (respectively, but also at \( r = r_0\)) with code like \cref{eqn:chapter2Notes:600}

%\mathematicaListing{
% can't use the newcommand with pure-function parameters (#1 is also a latex beastie).
\begin{figure}
\caption{Plot methods for fields and intensities}\label{eqn:chapter2Notes:600}
\begin{shaded}
\begin{mat}
rcap = {Cos[#], Sin[#]} & ;
scap = {Sin[#1] Cos[#2], Sin[#1] Sin[#2], Cos[#1]} & ;
ParametricPlot[ f[$r_0$, $\theta$, 0] rcap, {$\theta$, 0, Pi}]
ParametricPlot3D[ f[$r_0$, $\theta$, $\phi$] scap, {$\theta$, 0, Pi}, {$\phi$, 0, 2 Pi}]
\end{mat}
\end{shaded}
\end{figure}

Intensity plots can use the same code, with the only difference being the interpretation.  The surface doesn't represent the value of a vector valued radial function, but is the magnitude of a scalar valued function evaluated at \( f( r_0, \theta, \phi) \).

%For example, to plot an average power density with the form \( \BP_\trad = \rcap \sin^2\theta /r^2 \), at the point \( (r,\theta, \phi) \), 
%as sketched in the \( \phi = 0 \) plane in \cref{fig:antennaIntensityPlot:antennaIntensityPlotFig1}, and in spherical coordinates in \cref{fig:antennaIntensityPlot:antennaIntensityPlotFig2}.
%
%\imageFigure{../../figures/ece1229/antennaIntensityPlotFig1}{power density components in a plane}{fig:antennaIntensityPlot:antennaIntensityPlotFig1}{0.3}
%\imageFigure{../../figures/ece1229/antennaIntensityPlotFig2}{power density components in spherical coordinates}{fig:antennaIntensityPlot:antennaIntensityPlotFig2}{0.3}
%
%Noting that in the plane
%
%\begin{dmath}\label{eqn:chapter2Notes:520}
%\rcap = \lr{\cos\theta, \sin\theta},
%\end{dmath}
%
%and in 3D 
%
%\begin{dmath}\label{eqn:chapter2Notes:540}
%\rcap = \lr{\sin\theta \cos\phi, \sin\theta\sin\phi, \cos\theta}.
%\end{dmath}

%A plot of nt to 
%For the choice of \( \BP_\trad \) above \( U \) happens to not depend on \( r \), so the surface parameterized by \( \rcap U = r^2 \BP_\trad \) visualizes the intensity at all points on the surface.
%For more general \( r \) dependence, it is pointed out in 
%\citep{griffith1981introduction}
%that such antenna radiation intensity plots
%can be made by considering the values of \( U \) at fixed \( r \).

The surfaces for \( U = \sin\theta, \sin^2\theta \) in the plane are parametrically plotted in \cref{fig:SineAndSinSq:SineAndSinSqFig3}, and for cosines in \cref{fig:CoSineAndCoSineSq:CoSineAndCoSineSqFig1} to compare with textbook figures.

\mathImageFigure{../../figures/ece1229/CoSineAndCoSineSqFig1}{Cosinusoidal radiation intensities}{fig:CoSineAndCoSineSq:CoSineAndCoSineSqFig1}{0.3}{sphericalPlot3d.nb}
\mathImageFigure{../../figures/ece1229/SineAndSinSqFig3}{Sinusoidal radiation intenities}{fig:SineAndSinSq:SineAndSinSqFig3}{0.3}{sphericalPlot3d.nb}

Visualizations of \( U = \sin^2 \theta\) and 
\( U = \cos^2 \theta\) can be found in \cref{fig:SineSq3D:SineSq3DFig4} and
\cref{fig:CoSineSq3D:CoSineSq3DFig2} respectively.  Even for such simple functions these look pretty cool.

\mathImageFigure{../../figures/ece1229/SineSq3DFig4}{Square sinuoidal radiation intensity}{fig:SineSq3D:SineSq3DFig4}{0.3}{sphericalPlot3d.nb}
\mathImageFigure{../../figures/ece1229/CoSineSq3DFig2}{Square cosinuoidal radiation intensity}{fig:CoSineSq3D:CoSineSq3DFig2}{0.3}{sphericalPlot3d.nb}

\section{\si{dB} vs \si{dBi}}

Note that \si{dBi} is used to indicate that the gain is with respect to an ``isotropic'' radiator.
This is detailed more in \citep{digidbivsdbd}.

\section{Trig integrals}

Tables \cref{tab:chapter2Notes:1} and \cref{tab:chapter2Notes:2} produced with \nbref{tableOfTrigIntegrals.nb}
have some of the sine and cosine integrals that are pervasive in this chapter.

\captionedTable{\(\int_0^{\pi/2} \sin^r\theta \cos^s\theta d\theta\)}{tab:chapter2Notes:1}{
\begin{tabular}{|l|l|l|l|l|l|l|}
\hline
   & \multicolumn{6}{|l|}{r} \\ \hline
 s & 0 & 1 & 2 & 3 & 4 & 5 \\ \hline \hline
 0 & \(\ifrac{\pi}{2}\) & \(1\) & \(\ifrac{\pi}{4}\) & \(\ifrac{2}{3}\) & \(\ifrac{3 \pi}{16}\) & \(\ifrac{8}{15} \) \\ \hline
 1 & \(1\) & \(\ifrac{1}{2}\) & \(\ifrac{1}{3}\) & \(\ifrac{1}{4}\) & \(\ifrac{1}{5}\) & \(\ifrac{1}{6} \) \\ \hline
 2 & \(\ifrac{\pi}{4}\) & \(\ifrac{1}{3}\) & \(\ifrac{\pi}{16}\) & \(\ifrac{2}{15}\) & \(\ifrac{\pi}{32}\) & \(\ifrac{8}{105} \) \\ \hline
 3 & \(\ifrac{2}{3}\) & \(\ifrac{1}{4}\) & \(\ifrac{2}{15}\) & \(\ifrac{1}{12}\) & \(\ifrac{2}{35}\) & \(\ifrac{1}{24} \) \\ \hline
 4 & \(\ifrac{3 \pi}{16}\) & \(\ifrac{1}{5}\) & \(\ifrac{\pi}{32}\) & \(\ifrac{2}{35}\) & \(\ifrac{3 \pi}{256}\) & \(\ifrac{8}{315} \) \\ \hline
 5 & \(\ifrac{8}{15}\) & \(\ifrac{1}{6}\) & \(\ifrac{8}{105}\) & \(\ifrac{1}{24}\) & \(\ifrac{8}{315}\) & \(\ifrac{1}{60} \) \\ \hline
\end{tabular}
}

\captionedTable{\(\int_0^{\pi} \sin^r\theta \cos^s\theta d\theta\)}{tab:chapter2Notes:2}{
\begin{tabular}{|l|l|l|l|l|l|l|}
\hline
   & \multicolumn{6}{|l|}{r} \\ \hline
 s & 0 & 1 & 2 & 3 & 4 & 5 \\ \hline \hline
 0 & \(\pi \) & \(0\) & \(\ifrac{\pi }{2}\) & \(0\) & \(\ifrac{3 \pi }{8}\) & \(0 \) \\ \hline
 1 & \(2\) & \(0\) & \(\ifrac{2}{3}\) & \(0\) & \(\ifrac{2}{5}\) & \(0 \) \\ \hline
 2 & \(\ifrac{\pi }{2}\) & \(0\) & \(\ifrac{\pi }{8}\) & \(0\) & \(\ifrac{\pi }{16}\) & \(0 \) \\ \hline
 3 & \(\ifrac{4}{3}\) & \(0\) & \(\ifrac{4}{15}\) & \(0\) & \(\ifrac{4}{35}\) & \(0 \) \\ \hline
 4 & \(\ifrac{3 \pi }{8}\) & \(0\) & \(\ifrac{\pi }{16}\) & \(0\) & \(\ifrac{3 \pi }{128}\) & \(0 \) \\ \hline
 5 & \(\ifrac{16}{15}\) & \(0\) & \(\ifrac{16}{105}\) & \(0\) & \(\ifrac{16}{315}\) & \(0 \) \\ \hline
\end{tabular}
}

\section{Polarization vectors}

FIXME: todo.  Here the class notes include something not in the text.

\section{Phasor power}

In \S 2.13 the phasor power is written as

\begin{dmath}\label{eqn:chapter2Notes:620}
I^2 R/2,
\end{dmath}

where \( I, R \) are the magnitudes of phasors in the circuit.

I vaguely recall this relation, but had to refer back to 
\citep{irwin2007bec}
for the details.
This relation expresses average power over a period associated with the frequency of the phasor

\begin{dmath}\label{eqn:chapter2Notes:640}
P 
= \inv{T} \int_{t_0}^{t_0 + T} p(t) dt
= \inv{T} \int_{t_0}^{t_0 + T} \Abs{\BV} \cos\lr{ \omega t + \phi_V } \Abs{\BI} \cos\lr{ \omega t + \phi_I} dt
= \inv{T} \int_{t_0}^{t_0 + T} \Abs{\BV} \Abs{\BI}
\lr{
   \cos\lr{ \phi_V - \phi_I } + \cos\lr{ 2 \omega t + \phi_V + \phi_I} 
}
dt
= \inv{2} \Abs{\BV} \Abs{\BI} \cos\lr{ \phi_V - \phi_I }.
\end{dmath}

Introducing the impedence for this circuit element

\begin{dmath}\label{eqn:chapter2Notes:660}
\BZ = \frac{ \Abs{\BV} e^{j\phi_V} }{ \Abs{\BI} e^{j\phi_I} } = \frac{\Abs{\BV}}{\Abs{\BI}} e^{j\lr{\phi_V - \phi_I}},
\end{dmath}

this average power can be written in phasor form

\begin{dmath}\label{eqn:chapter2Notes:680}
\BP = \inv{2} \Abs{\BI}^2 \BZ,
\end{dmath}

with
\begin{dmath}\label{eqn:chapter2Notes:700}
P = \Real \BP.
\end{dmath}

Observe that we have to be careful to use the absolute value of the current phasor \( \BI \), since \( \BI^2 \) differs in phase from \( \Abs{\BI}^2 \).  This explains the conjugation in the \citep{irwin2007bec} definition of complex power, which had the form

\begin{dmath}\label{eqn:chapter2Notes:720}
\BS = \BV_\trms \BI^\conj_\trms.
\end{dmath}

\section{Notation}

\begin{itemize}
\item Time average.
Both Prof. Eleftheriades 
and the text \citep{balanis2005antenna} use square brackets \( \timeaverage{\cdots} \) for time averages. 
\item Prof. Eleftheriades 
writes omega as a circle floating above a face up square bracket, as in \cref{fig:greekStyleOmega:greekStyleOmegaFig1}.
\item Bold vectors are usually phasors, with caligraphic script used for the time domain fields.  Example: \( \BE(x,y,z,t) = E(x,y) e^{j \omega t - k z}, \bcE(x, y, z, t) = \Real \BE \).
\end{itemize}

\imageFigure{../../figures/ece1229/greekStyleOmegaFig1}{``I'm Greek'' style Omega \( \Omega \)}{fig:greekStyleOmega:greekStyleOmegaFig1}{0.05}

\section{Mathematica notebooks}

\input{../ece1229/mathematica.tex}

\EndArticle
%\EndNoBibArticle
