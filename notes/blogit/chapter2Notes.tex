%
% Copyright � 2015 Peeter Joot.  All Rights Reserved.
% Licenced as described in the file LICENSE under the root directory of this GIT repository.
%
\newcommand{\authorname}{Peeter Joot}
\newcommand{\email}{peeterjoot@protonmail.com}
\newcommand{\basename}{FIXMEbasenameUndefined}
\newcommand{\dirname}{notes/FIXMEdirnameUndefined/}

\renewcommand{\basename}{chapter2Notes}
\renewcommand{\dirname}{notes/FIXMEwheretodirname/}
%\newcommand{\dateintitle}{}
%\newcommand{\keywords}{}

\newcommand{\tiso}[0]{{\textrm{iso}}}
\newcommand{\tin}[0]{{\textrm{in}}}
\newcommand{\trad}[0]{{\textrm{rad}}}
\newcommand{\tmax}[0]{{\textrm{max}}}
\newcommand{\tA}[0]{{\textrm{A}}}
\newcommand{\tav}[0]{{\textrm{av}}}
\newcommand{\tdB}[0]{{\textrm{dB}}}
\newcommand{\tdBi}[0]{{\textrm{dBi}}}

\newcommand{\timeaverage}[1]{\left[#1\right]}

\newcommand{\ch}[0]{ch.}

\newcommand{\authorname}{Peeter Joot}
\newcommand{\onlineurl}{http://sites.google.com/site/peeterjoot2/math2013/\basename.pdf}
\newcommand{\sourcepath}{\dirname\basename.tex}
\newcommand{\generatetitle}[1]{\chapter{#1}}

\newcommand{\vcsinfo}{%
\section*{}
\noindent{\color{DarkOliveGreen}{\rule{\linewidth}{0.1mm}}}
\paragraph{Document version}
%\paragraph{\color{Maroon}{Document version}}
{
\small
\begin{itemize}
\item Available online at:\\ 
\href{\onlineurl}{\onlineurl}
\item Git Repository: \input{./.revinfo/gitRepo.tex}
\item Source: \sourcepath
\item last commit: \input{./.revinfo/gitCommitString.tex}
\item commit date: \input{./.revinfo/gitCommitDate.tex}
\end{itemize}
}
}

%\PassOptionsToPackage{dvipsnames,svgnames}{xcolor}
\PassOptionsToPackage{square,numbers}{natbib}
\documentclass{scrreprt}

\usepackage[left=2cm,right=2cm]{geometry}
\usepackage[svgnames]{xcolor}
\usepackage{peeters_layout}

\usepackage{natbib}

\usepackage[
colorlinks=true,
bookmarks=false,
pdfauthor={\authorname, \email},
backref 
]{hyperref}

% http://tex.stackexchange.com/questions/75773/how-to-reference-problems-by-the-text-label-in-an-exercise-envioronment
\usepackage[english]{cleveref}
\crefname{Exercise}{exercise}{exercises}
\Crefname{Exercise}{Exercise}{Exercises}

\RequirePackage{titlesec}
\RequirePackage{ifthen}

% http://stackoverflow.com/questions/4932910/date-in-the-tabular-environment
\makeatletter
\let\insertdate\@date
\makeatother

\titleformat{\chapter}[display]
{\bfseries\Large}
{\color{DarkSlateGrey}\filleft \authorname
\ifthenelse{\isundefined{\studentnumber}}{}{\\ \studentnumber}
\ifthenelse{\isundefined{\email}}{}{\\ \email}
\ifthenelse{\isundefined{\dateintitle}}{}{\\ \insertdate}
%\ifthenelse{\isundefined{\coursename}}{}{\\ \coursename} % put in title instead.
}
{4ex}
{\color{DarkOliveGreen}{\titlerule}\color{Maroon}
\vspace{2ex}%
\filright}
[\vspace{2ex}%
\color{DarkOliveGreen}\titlerule
]

\newcommand{\beginArtWithToc}[0]{\begin{document}\tableofcontents}
\newcommand{\beginArtNoToc}[0]{\begin{document}}
\newcommand{\EndNoBibArticle}[0]{\end{document}}
\newcommand{\EndArticle}[0]{\bibliography{Bibliography}\bibliographystyle{plainnat}\end{document}}

% 
%\newcommand{\citep}[1]{\cite{#1}}

\colorSectionsForArticle



\beginArtNoToc

\generatetitle{Fundamental parameters of antennas}
%\chapter{Fundamental parameters of Antennas}
%\label{chap:chapter2Notes}

Reading notes for \ch 2 \citep{balanis2012antenna}.

\section{Typical far-field radiation intensity}

It was mentioned that

\begin{dmath}\label{eqn:advancedantennaL1:20}
U(\theta, \phi) 
= 
\frac{r^2}{2 \eta_0} \Abs{ \BE(r, \theta, \phi) }^2,
\end{dmath}

where the intrinsic impedance of free space is

\begin{dmath}\label{eqn:advancedantennaL1:480}
\eta_0 = \sqrt{\frac{\mu_0}{\epsilon_0}} = 377 \Omega.
\end{dmath}

TODO: Derive \cref{eqn:advancedantennaL1:20}, from the Poynting relation.  Want to calculate \( \BB \) for a spherical wave

\begin{dmath}\label{eqn:advancedantennaL1:500}
\BE(r, \theta, \phi) = \BE^0\lr{ \theta, \phi } \frac{e^{-j k r}}{r}.
\end{dmath}

\section{Intensity plots}

FIXME: rewrite this descriptive text when I'm not so tired.

Should we wish to plot the average power density \( \BP_\trad = \rcap \sin^2\theta /r^2 \), at the point \( (r,\theta, \phi) \), as sketched in the \( \phi = 0 \) plane in \cref{fig:antennaIntensityPlot:antennaIntensityPlotFig1}, and in spherical coordinates in \cref{fig:antennaIntensityPlot:antennaIntensityPlotFig2}.

\imageFigure{../../figures/ece1229/antennaIntensityPlotFig1}{power density components in a plane}{fig:antennaIntensityPlot:antennaIntensityPlotFig1}{0.3}
\imageFigure{../../figures/ece1229/antennaIntensityPlotFig2}{power density components in spherical coordinates}{fig:antennaIntensityPlot:antennaIntensityPlotFig2}{0.3}

Noting that in the plane

\begin{dmath}\label{eqn:chapter2Notes:520}
\rcap = \lr{\cos\theta, \sin\theta},
\end{dmath}

and in 3D 

\begin{dmath}\label{eqn:chapter2Notes:540}
\rcap = \lr{\sin\theta \cos\phi, \sin\theta\sin\phi, \cos\theta}.
\end{dmath}

For the choice of \( \BP_\trad \) above \( U \) happens to not depend on \( r \), so the surface parameterized by \( \rcap U = r^2 \BP_\trad \) visualizes the intensity at all points on the surface.  For more general \( r \) dependence, it is pointed out in 
\citep{griffith1981introduction} that such antenna radiation intensity plots
can be made by considering the values of \( U \) at fixed \( r \).  The surfaces for \( U = \sin\theta, \sin^2\theta \) in the plane are parametrically plotted in \cref{fig:SineAndSinSq:SineAndSinSqFig3}, and for cosines in \cref{fig:CoSineAndCoSineSq:CoSineAndCoSineSqFig1}.

\imageFigure{../../figures/ece1229/CoSineAndCoSineSqFig1}{Cosinusoidal radiation intensities}{fig:CoSineAndCoSineSq:CoSineAndCoSineSqFig1}{0.3}
\imageFigure{../../figures/ece1229/SineAndSinSqFig3}{Sinusoidal radiation intenities}{fig:SineAndSinSq:SineAndSinSqFig3}{0.3}

ParametricPlot3D visualizations of \( U = \sin^2 \theta\) and 
\( U = \cos^2 \theta\) can be found in \cref{fig:SineSq3D:SineSq3DFig4} and
\cref{fig:CoSineSq3D:CoSineSq3DFig2} respectively.

\imageFigure{../../figures/ece1229/SineSq3DFig4}{Square sinuoidal radiation intensity}{fig:SineSq3D:SineSq3DFig4}{0.3}
\imageFigure{../../figures/ece1229/CoSineSq3DFig2}{Square cosinuoidal radiation intensity}{fig:CoSineSq3D:CoSineSq3DFig2}{0.3}

\section{dB vs dBi}

Note that \( \tdBi \) is used to indicate that the gain is with respect to an ``isotropic'' radiator.  This is detailed more in \citep{digidbivsdbd}.

\section{Trig integrals}

Tables \cref{tab:chapter2Notes:1} and \cref{tab:chapter2Notes:2} have some of the sine and cosine integrals that are pervasive in this chapter.

FIXME: nbref.

\captionedTable{\(\int_0^{\pi/2} \sin^r\theta \cos^s\theta d\theta\)}{tab:chapter2Notes:1}{
\begin{tabular}{|l|l|l|l|l|l|l|}
\hline
   & \multicolumn{6}{|l|}{r} \\ \hline
 s & 0 & 1 & 2 & 3 & 4 & 5 \\ \hline \hline
 0 & \(\ifrac{\pi}{2}\) & \(1\) & \(\ifrac{\pi}{4}\) & \(\ifrac{2}{3}\) & \(\ifrac{3 \pi}{16}\) & \(\ifrac{8}{15} \) \\ \hline
 1 & \(1\) & \(\ifrac{1}{2}\) & \(\ifrac{1}{3}\) & \(\ifrac{1}{4}\) & \(\ifrac{1}{5}\) & \(\ifrac{1}{6} \) \\ \hline
 2 & \(\ifrac{\pi}{4}\) & \(\ifrac{1}{3}\) & \(\ifrac{\pi}{16}\) & \(\ifrac{2}{15}\) & \(\ifrac{\pi}{32}\) & \(\ifrac{8}{105} \) \\ \hline
 3 & \(\ifrac{2}{3}\) & \(\ifrac{1}{4}\) & \(\ifrac{2}{15}\) & \(\ifrac{1}{12}\) & \(\ifrac{2}{35}\) & \(\ifrac{1}{24} \) \\ \hline
 4 & \(\ifrac{3 \pi}{16}\) & \(\ifrac{1}{5}\) & \(\ifrac{\pi}{32}\) & \(\ifrac{2}{35}\) & \(\ifrac{3 \pi}{256}\) & \(\ifrac{8}{315} \) \\ \hline
 5 & \(\ifrac{8}{15}\) & \(\ifrac{1}{6}\) & \(\ifrac{8}{105}\) & \(\ifrac{1}{24}\) & \(\ifrac{8}{315}\) & \(\ifrac{1}{60} \) \\ \hline
\end{tabular}
}

\captionedTable{\(\int_0^{\pi} \sin^r\theta \cos^s\theta d\theta\)}{tab:chapter2Notes:2}{
\begin{tabular}{|l|l|l|l|l|l|l|}
\hline
   & \multicolumn{6}{|l|}{r} \\ \hline
 s & 0 & 1 & 2 & 3 & 4 & 5 \\ \hline \hline
 0 & \(\pi \) & \(0\) & \(\ifrac{\pi }{2}\) & \(0\) & \(\ifrac{3 \pi }{8}\) & \(0 \) \\ \hline
 1 & \(2\) & \(0\) & \(\ifrac{2}{3}\) & \(0\) & \(\ifrac{2}{5}\) & \(0 \) \\ \hline
 2 & \(\ifrac{\pi }{2}\) & \(0\) & \(\ifrac{\pi }{8}\) & \(0\) & \(\ifrac{\pi }{16}\) & \(0 \) \\ \hline
 3 & \(\ifrac{4}{3}\) & \(0\) & \(\ifrac{4}{15}\) & \(0\) & \(\ifrac{4}{35}\) & \(0 \) \\ \hline
 4 & \(\ifrac{3 \pi }{8}\) & \(0\) & \(\ifrac{\pi }{16}\) & \(0\) & \(\ifrac{3 \pi }{128}\) & \(0 \) \\ \hline
 5 & \(\ifrac{16}{15}\) & \(0\) & \(\ifrac{16}{105}\) & \(0\) & \(\ifrac{16}{315}\) & \(0 \) \\ \hline
\end{tabular}
}

\section{Notation}

\begin{itemize}
\item Time average.
Both our Prof and the text \citep{balanis2012antenna} use square brackets \( \timeaverage{\cdots} \) for time averages. 
\item Our Prof writes omega as a circle floating above a face up square bracket.  I've used \( \Omega \) in these notes.
FIXME: figure.
\end{itemize}

\EndArticle
%\EndNoBibArticle
