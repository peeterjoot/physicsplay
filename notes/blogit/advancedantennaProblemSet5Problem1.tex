%
% Copyright � 2015 Peeter Joot.  All Rights Reserved.
% Licenced as described in the file LICENSE under the root directory of this GIT repository.
%
\makeproblem{Aperature antenna}{advancedantenna:problemSet5:1}{ 

A rectangular aperture lies along the x-y plane and has dimensions \( a \times b \). Let the electric
field aperture distribution be given by,

\begin{dmath}\label{eqn:advancedantennaProblemSet5Problem1:n}
\BE_{\textrm{ap}} = \ycap \cos\lr{ \frac{\pi}{a} x }.
\end{dmath}

The aperture is cut out of an infinite perfectly electric conductor. 
The origin of the coordinate system is at the center of the aperture.

Using the theory of radiation from apertures based on the equivalnce principle, calculate:

\begin{enumerate}
\item
An expression for \( E_\theta(\theta, \phi) \)
\item
An expression for \( E_\phi(\theta, \phi) \)
\item
Consider an aperture of dimensions \( a = b = 10 \si{cm} \) at \( f = 9.8 \si{GHz} \).
\end{enumerate}

\makesubproblem{}{advancedantenna:problemSet5:1a}
Plot the E-plane and H-plane patterns (power).
\makesubproblem{}{advancedantenna:problemSet5:1b}
Calculate the positions of the first nulls in the E and H planes.
\makesubproblem{}{advancedantenna:problemSet5:1c}
From the plot determine the levels of the first sidelobe in the E and H planes.
\makesubproblem{}{advancedantenna:problemSet5:1d}
From the plot determine the 3dB beamwidth of the main lobe in the E and H planes.
} % makeproblem

\makeanswer{advancedantenna:problemSet5:1}{ 
\makeSubAnswer{}{advancedantenna:problemSet5:1a}

TODO.
\makeSubAnswer{}{advancedantenna:problemSet5:1b}

TODO.
\makeSubAnswer{}{advancedantenna:problemSet5:1c}

TODO.
\makeSubAnswer{}{advancedantenna:problemSet5:1d}

TODO.
}
