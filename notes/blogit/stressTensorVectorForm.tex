%
% Copyright � 2015 Peeter Joot.  All Rights Reserved.
% Licenced as described in the file LICENSE under the root directory of this GIT repository.
%
\documentclass[]{eliblog}

\usepackage{amsmath}
\usepackage{mathpazo}

%
% shorthand for bold symbols, convenient for vectors and matrices
%
\newcommand{\Ba}[0]{\mathbf{a}}
\newcommand{\Bb}[0]{\mathbf{b}}
\newcommand{\Bc}[0]{\mathbf{c}}
\newcommand{\Bd}[0]{\mathbf{d}}
\newcommand{\Be}[0]{\mathbf{e}}
\newcommand{\Bf}[0]{\mathbf{f}}
\newcommand{\Bg}[0]{\mathbf{g}}
\newcommand{\Bh}[0]{\mathbf{h}}
\newcommand{\Bi}[0]{\mathbf{i}}
\newcommand{\Bj}[0]{\mathbf{j}}
\newcommand{\Bk}[0]{\mathbf{k}}
\newcommand{\Bl}[0]{\mathbf{l}}
\newcommand{\Bm}[0]{\mathbf{m}}
\newcommand{\Bn}[0]{\mathbf{n}}
\newcommand{\Bo}[0]{\mathbf{o}}
\newcommand{\Bp}[0]{\mathbf{p}}
\newcommand{\Bq}[0]{\mathbf{q}}
\newcommand{\Br}[0]{\mathbf{r}}
\newcommand{\Bs}[0]{\mathbf{s}}
\newcommand{\Bt}[0]{\mathbf{t}}
\newcommand{\Bu}[0]{\mathbf{u}}
\newcommand{\Bv}[0]{\mathbf{v}}
\newcommand{\Bw}[0]{\mathbf{w}}
\newcommand{\Bx}[0]{\mathbf{x}}
\newcommand{\By}[0]{\mathbf{y}}
\newcommand{\Bz}[0]{\mathbf{z}}
\newcommand{\BA}[0]{\mathbf{A}}
\newcommand{\BB}[0]{\mathbf{B}}
\newcommand{\BC}[0]{\mathbf{C}}
\newcommand{\BD}[0]{\mathbf{D}}
\newcommand{\BE}[0]{\mathbf{E}}
\newcommand{\BF}[0]{\mathbf{F}}
\newcommand{\BG}[0]{\mathbf{G}}
\newcommand{\BH}[0]{\mathbf{H}}
\newcommand{\BI}[0]{\mathbf{I}}
\newcommand{\BJ}[0]{\mathbf{J}}
\newcommand{\BK}[0]{\mathbf{K}}
\newcommand{\BL}[0]{\mathbf{L}}
\newcommand{\BM}[0]{\mathbf{M}}
\newcommand{\BN}[0]{\mathbf{N}}
\newcommand{\BO}[0]{\mathbf{O}}
\newcommand{\BP}[0]{\mathbf{P}}
\newcommand{\BQ}[0]{\mathbf{Q}}
\newcommand{\BR}[0]{\mathbf{R}}
\newcommand{\BS}[0]{\mathbf{S}}
\newcommand{\BT}[0]{\mathbf{T}}
\newcommand{\BU}[0]{\mathbf{U}}
\newcommand{\BV}[0]{\mathbf{V}}
\newcommand{\BW}[0]{\mathbf{W}}
\newcommand{\BX}[0]{\mathbf{X}}
\newcommand{\BY}[0]{\mathbf{Y}}
\newcommand{\BZ}[0]{\mathbf{Z}}

\newcommand{\Bzero}[0]{\mathbf{0}}
\newcommand{\Btheta}[0]{\boldsymbol{\theta}}
\newcommand{\Btau}[0]{\boldsymbol{\tau}}
\newcommand{\Bomega}[0]{\boldsymbol{\omega}}

%
% shorthand for unit vectors
%
\newcommand{\acap}[0]{\hat{\Ba}}
\newcommand{\bcap}[0]{\hat{\Bb}}
\newcommand{\ccap}[0]{\hat{\Bc}}
\newcommand{\dcap}[0]{\hat{\Bd}}
\newcommand{\ecap}[0]{\hat{\Be}}
\newcommand{\fcap}[0]{\hat{\Bf}}
\newcommand{\gcap}[0]{\hat{\Bg}}
\newcommand{\hcap}[0]{\hat{\Bh}}
\newcommand{\icap}[0]{\hat{\Bi}}
\newcommand{\jcap}[0]{\hat{\Bj}}
\newcommand{\kcap}[0]{\hat{\Bk}}
\newcommand{\lcap}[0]{\hat{\Bl}}
\newcommand{\mcap}[0]{\hat{\Bm}}
\newcommand{\ncap}[0]{\hat{\Bn}}
\newcommand{\ocap}[0]{\hat{\Bo}}
\newcommand{\pcap}[0]{\hat{\Bp}}
\newcommand{\qcap}[0]{\hat{\Bq}}
\newcommand{\rcap}[0]{\hat{\Br}}
\newcommand{\scap}[0]{\hat{\Bs}}
\newcommand{\tcap}[0]{\hat{\Bt}}
\newcommand{\ucap}[0]{\hat{\Bu}}
\newcommand{\vcap}[0]{\hat{\Bv}}
\newcommand{\wcap}[0]{\hat{\Bw}}
\newcommand{\xcap}[0]{\hat{\Bx}}
\newcommand{\ycap}[0]{\hat{\By}}
\newcommand{\zcap}[0]{\hat{\Bz}}
\newcommand{\thetacap}[0]{\hat{\Btheta}}

%
% to write R^n and C^n in a distinguishable fashion.  Perhaps change this
% to the double lined characters upon figuring out how to do so.
%
\newcommand{\C}[1]{$\mathbb{C}^{#1}$}
\newcommand{\R}[1]{$\mathbb{R}^{#1}$}

%
% various generally useful helpers
%

% derivative of #1 wrt. #2:
\newcommand{\D}[2] {\frac {d#2} {d#1}}

\newcommand{\inv}[1]{\frac{1}{#1}}
\newcommand{\cross}[0]{\times}

\newcommand{\abs}[1]{\lvert{#1}\rvert}
\newcommand{\norm}[1]{\lVert{#1}\rVert}
\newcommand{\innerprod}[2]{\langle{#1}, {#2}\rangle}
\newcommand{\dotprod}[2]{{#1} \cdot {#2}}
\newcommand{\bdotprod}[2]{\left({#1} \cdot {#2}\right)}
\newcommand{\crossprod}[2]{{#1} \cross {#2}}
\newcommand{\tripleprod}[3]{\dotprod{\left(\crossprod{#1}{#2}\right)}{#3}}

\DeclareMathOperator{\Proj}{Proj}
\DeclareMathOperator{\Span}{span}
\DeclareMathOperator{\Sgn}{sgn}
\DeclareMathOperator{\Area}{Area}
\DeclareMathOperator{\Volume}{Volume}

%
% A few miscellaneous things specific to this document
%
\newcommand{\crossop}[1]{\crossprod{#1}{}}

% R2 vector.
\newcommand{\VectorTwo}[2]{
\begin{bmatrix}
 {#1} \\
 {#2}
\end{bmatrix}
}

\newcommand{\VectorN}[1]{
\begin{bmatrix}
{#1}_1 \\
{#1}_2 \\
\vdots \\
{#1}_N \\
\end{bmatrix}
}

\newcommand{\DETuvij}[4]{
\begin{vmatrix}
 {#1}_{#3} & {#1}_{#4} \\
 {#2}_{#3} & {#2}_{#4}
\end{vmatrix}
}

\newcommand{\DETuvwijk}[6]{
\begin{vmatrix}
 {#1}_{#4} & {#1}_{#5} & {#1}_{#6} \\
 {#2}_{#4} & {#2}_{#5} & {#2}_{#6} \\
 {#3}_{#4} & {#3}_{#5} & {#3}_{#6}
\end{vmatrix}
}

\newcommand{\DETuvwxijkl}[8]{
\begin{vmatrix}
 {#1}_{#5} & {#1}_{#6} & {#1}_{#7} & {#1}_{#8} \\
 {#2}_{#5} & {#2}_{#6} & {#2}_{#7} & {#2}_{#8} \\
 {#3}_{#5} & {#3}_{#6} & {#3}_{#7} & {#3}_{#8} \\
 {#4}_{#5} & {#4}_{#6} & {#4}_{#7} & {#4}_{#8} \\
\end{vmatrix}
}

%\newcommand{\DETuvwxyijklm}[10]{
%\begin{vmatrix}
% {#1}_{#6} & {#1}_{#7} & {#1}_{#8} & {#1}_{#9} & {#1}_{#10} \\
% {#2}_{#6} & {#2}_{#7} & {#2}_{#8} & {#2}_{#9} & {#2}_{#10} \\
% {#3}_{#6} & {#3}_{#7} & {#3}_{#8} & {#3}_{#9} & {#3}_{#10} \\
% {#4}_{#6} & {#4}_{#7} & {#4}_{#8} & {#4}_{#9} & {#4}_{#10} \\
% {#5}_{#6} & {#5}_{#7} & {#5}_{#8} & {#5}_{#9} & {#5}_{#10}
%\end{vmatrix}
%}

% R3 vector.
\newcommand{\VectorThree}[3]{
\begin{bmatrix}
 {#1} \\
 {#2} \\
 {#3}
\end{bmatrix}
}



\author{Peeter Joot}
\email{peeter.joot@gmail.com}

%\documentclass[]{eliblogwidescreen}

\usepackage{amsmath}
\usepackage{mathpazo}

%
% shorthand for bold symbols, convenient for vectors and matrices
%
\newcommand{\Ba}[0]{\mathbf{a}}
\newcommand{\Bb}[0]{\mathbf{b}}
\newcommand{\Bc}[0]{\mathbf{c}}
\newcommand{\Bd}[0]{\mathbf{d}}
\newcommand{\Be}[0]{\mathbf{e}}
\newcommand{\Bf}[0]{\mathbf{f}}
\newcommand{\Bg}[0]{\mathbf{g}}
\newcommand{\Bh}[0]{\mathbf{h}}
\newcommand{\Bi}[0]{\mathbf{i}}
\newcommand{\Bj}[0]{\mathbf{j}}
\newcommand{\Bk}[0]{\mathbf{k}}
\newcommand{\Bl}[0]{\mathbf{l}}
\newcommand{\Bm}[0]{\mathbf{m}}
\newcommand{\Bn}[0]{\mathbf{n}}
\newcommand{\Bo}[0]{\mathbf{o}}
\newcommand{\Bp}[0]{\mathbf{p}}
\newcommand{\Bq}[0]{\mathbf{q}}
\newcommand{\Br}[0]{\mathbf{r}}
\newcommand{\Bs}[0]{\mathbf{s}}
\newcommand{\Bt}[0]{\mathbf{t}}
\newcommand{\Bu}[0]{\mathbf{u}}
\newcommand{\Bv}[0]{\mathbf{v}}
\newcommand{\Bw}[0]{\mathbf{w}}
\newcommand{\Bx}[0]{\mathbf{x}}
\newcommand{\By}[0]{\mathbf{y}}
\newcommand{\Bz}[0]{\mathbf{z}}
\newcommand{\BA}[0]{\mathbf{A}}
\newcommand{\BB}[0]{\mathbf{B}}
\newcommand{\BC}[0]{\mathbf{C}}
\newcommand{\BD}[0]{\mathbf{D}}
\newcommand{\BE}[0]{\mathbf{E}}
\newcommand{\BF}[0]{\mathbf{F}}
\newcommand{\BG}[0]{\mathbf{G}}
\newcommand{\BH}[0]{\mathbf{H}}
\newcommand{\BI}[0]{\mathbf{I}}
\newcommand{\BJ}[0]{\mathbf{J}}
\newcommand{\BK}[0]{\mathbf{K}}
\newcommand{\BL}[0]{\mathbf{L}}
\newcommand{\BM}[0]{\mathbf{M}}
\newcommand{\BN}[0]{\mathbf{N}}
\newcommand{\BO}[0]{\mathbf{O}}
\newcommand{\BP}[0]{\mathbf{P}}
\newcommand{\BQ}[0]{\mathbf{Q}}
\newcommand{\BR}[0]{\mathbf{R}}
\newcommand{\BS}[0]{\mathbf{S}}
\newcommand{\BT}[0]{\mathbf{T}}
\newcommand{\BU}[0]{\mathbf{U}}
\newcommand{\BV}[0]{\mathbf{V}}
\newcommand{\BW}[0]{\mathbf{W}}
\newcommand{\BX}[0]{\mathbf{X}}
\newcommand{\BY}[0]{\mathbf{Y}}
\newcommand{\BZ}[0]{\mathbf{Z}}

\newcommand{\Bzero}[0]{\mathbf{0}}
\newcommand{\Btheta}[0]{\boldsymbol{\theta}}
\newcommand{\Btau}[0]{\boldsymbol{\tau}}
\newcommand{\Bomega}[0]{\boldsymbol{\omega}}

%
% shorthand for unit vectors
%
\newcommand{\acap}[0]{\hat{\Ba}}
\newcommand{\bcap}[0]{\hat{\Bb}}
\newcommand{\ccap}[0]{\hat{\Bc}}
\newcommand{\dcap}[0]{\hat{\Bd}}
\newcommand{\ecap}[0]{\hat{\Be}}
\newcommand{\fcap}[0]{\hat{\Bf}}
\newcommand{\gcap}[0]{\hat{\Bg}}
\newcommand{\hcap}[0]{\hat{\Bh}}
\newcommand{\icap}[0]{\hat{\Bi}}
\newcommand{\jcap}[0]{\hat{\Bj}}
\newcommand{\kcap}[0]{\hat{\Bk}}
\newcommand{\lcap}[0]{\hat{\Bl}}
\newcommand{\mcap}[0]{\hat{\Bm}}
\newcommand{\ncap}[0]{\hat{\Bn}}
\newcommand{\ocap}[0]{\hat{\Bo}}
\newcommand{\pcap}[0]{\hat{\Bp}}
\newcommand{\qcap}[0]{\hat{\Bq}}
\newcommand{\rcap}[0]{\hat{\Br}}
\newcommand{\scap}[0]{\hat{\Bs}}
\newcommand{\tcap}[0]{\hat{\Bt}}
\newcommand{\ucap}[0]{\hat{\Bu}}
\newcommand{\vcap}[0]{\hat{\Bv}}
\newcommand{\wcap}[0]{\hat{\Bw}}
\newcommand{\xcap}[0]{\hat{\Bx}}
\newcommand{\ycap}[0]{\hat{\By}}
\newcommand{\zcap}[0]{\hat{\Bz}}
\newcommand{\thetacap}[0]{\hat{\Btheta}}

%
% to write R^n and C^n in a distinguishable fashion.  Perhaps change this
% to the double lined characters upon figuring out how to do so.
%
\newcommand{\C}[1]{$\mathbb{C}^{#1}$}
\newcommand{\R}[1]{$\mathbb{R}^{#1}$}

%
% various generally useful helpers
%

% derivative of #1 wrt. #2:
\newcommand{\D}[2] {\frac {d#2} {d#1}}

\newcommand{\inv}[1]{\frac{1}{#1}}
\newcommand{\cross}[0]{\times}

\newcommand{\abs}[1]{\lvert{#1}\rvert}
\newcommand{\norm}[1]{\lVert{#1}\rVert}
\newcommand{\innerprod}[2]{\langle{#1}, {#2}\rangle}
\newcommand{\dotprod}[2]{{#1} \cdot {#2}}
\newcommand{\bdotprod}[2]{\left({#1} \cdot {#2}\right)}
\newcommand{\crossprod}[2]{{#1} \cross {#2}}
\newcommand{\tripleprod}[3]{\dotprod{\left(\crossprod{#1}{#2}\right)}{#3}}

\DeclareMathOperator{\Proj}{Proj}
\DeclareMathOperator{\Span}{span}
\DeclareMathOperator{\Sgn}{sgn}
\DeclareMathOperator{\Area}{Area}
\DeclareMathOperator{\Volume}{Volume}

%
% A few miscellaneous things specific to this document
%
\newcommand{\crossop}[1]{\crossprod{#1}{}}

% R2 vector.
\newcommand{\VectorTwo}[2]{
\begin{bmatrix}
 {#1} \\
 {#2}
\end{bmatrix}
}

\newcommand{\VectorN}[1]{
\begin{bmatrix}
{#1}_1 \\
{#1}_2 \\
\vdots \\
{#1}_N \\
\end{bmatrix}
}

\newcommand{\DETuvij}[4]{
\begin{vmatrix}
 {#1}_{#3} & {#1}_{#4} \\
 {#2}_{#3} & {#2}_{#4}
\end{vmatrix}
}

\newcommand{\DETuvwijk}[6]{
\begin{vmatrix}
 {#1}_{#4} & {#1}_{#5} & {#1}_{#6} \\
 {#2}_{#4} & {#2}_{#5} & {#2}_{#6} \\
 {#3}_{#4} & {#3}_{#5} & {#3}_{#6}
\end{vmatrix}
}

\newcommand{\DETuvwxijkl}[8]{
\begin{vmatrix}
 {#1}_{#5} & {#1}_{#6} & {#1}_{#7} & {#1}_{#8} \\
 {#2}_{#5} & {#2}_{#6} & {#2}_{#7} & {#2}_{#8} \\
 {#3}_{#5} & {#3}_{#6} & {#3}_{#7} & {#3}_{#8} \\
 {#4}_{#5} & {#4}_{#6} & {#4}_{#7} & {#4}_{#8} \\
\end{vmatrix}
}

%\newcommand{\DETuvwxyijklm}[10]{
%\begin{vmatrix}
% {#1}_{#6} & {#1}_{#7} & {#1}_{#8} & {#1}_{#9} & {#1}_{#10} \\
% {#2}_{#6} & {#2}_{#7} & {#2}_{#8} & {#2}_{#9} & {#2}_{#10} \\
% {#3}_{#6} & {#3}_{#7} & {#3}_{#8} & {#3}_{#9} & {#3}_{#10} \\
% {#4}_{#6} & {#4}_{#7} & {#4}_{#8} & {#4}_{#9} & {#4}_{#10} \\
% {#5}_{#6} & {#5}_{#7} & {#5}_{#8} & {#5}_{#9} & {#5}_{#10}
%\end{vmatrix}
%}

% R3 vector.
\newcommand{\VectorThree}[3]{
\begin{bmatrix}
 {#1} \\
 {#2} \\
 {#3}
\end{bmatrix}
}



\author{Peeter Joot}
\email{peeter.joot@gmail.com}


\chapter{Putting the stress tensor (and traction vector) into explicit vector form.}
\label{chap:continuumstressTensorVectorForm}
%\useCCL
\blogpage{http://sites.google.com/site/peeterjoot2/math2012/continuumstressTensorVectorForm.pdf}
\date{Apr 4, 2012}
\gitRevisionInfo{continuumstressTensorVectorForm}
\keywords{Navier-Stokes, PHY454H1S} 

\beginArtWithToc
%\beginArtNoToc

\section{Motivation.}

Exersize 6.1 from \cite{acheson1990elementary} is to show that the traction vector can be written in vector form (a rather curious thing to have to say) as

\begin{equation}\label{eqn:stressTensorVectorForm:10}
\Bt = -p \ncap + \mu ( 2 (\ncap \cdot \spacegrad)\Bu + \ncap \cross (\spacegrad \cross \Bu)).
\end{equation}

Note that the text uses a wedge symbol for the cross product, and I've switched to standard notation.  I've done so because the use of a Geometric-Algebra wedge product also can be used to express this relationship, in which case we would write

\begin{equation}\label{eqn:stressTensorVectorForm:30}
\Bt = -p \ncap + \mu ( 2 (\ncap \cdot \spacegrad) \Bu + (\spacegrad \wedge \Bu) \cdot \ncap).
\end{equation}

In either case we have

\begin{equation}\label{eqn:stressTensorVectorForm:50}
(\spacegrad \wedge \Bu) \cdot \ncap
= 
\ncap \cross (\spacegrad \cross \Bu)
=
\spacegrad' (\ncap \cdot \Bu') - (\ncap \cdot \spacegrad) \Bu 
\end{equation}

(where the primes indicate the scope of the gradient, showing here that we are operating only on $\Bu$, and not $\ncap$).

After computing this, lets also compute the stress tensor in cylindrical and spherical coordinates, something that this allows us to do fairly easily without having to deal with the second order terms that we encountered doing this by computing the difference of squared displacements.

We'll work primarily with just the strain tensor portion of the traction vector expressions above, calculating

\begin{equation}\label{eqn:stressTensorVectorForm:250}
2 {\Be}_{\ncap}
=
2 (\ncap \cdot \spacegrad)\Bu + \ncap \cross (\spacegrad \cross \Bu)
=
2 (\ncap \cdot \spacegrad)\Bu + (\spacegrad \wedge \Bu) \cdot \ncap.
\end{equation}

\section{Verifying the relationship.}

Let's start with the the plain old cross product version

\begin{align*}
(\ncap \cross (\spacegrad \cross \Bu) + 2 (\ncap \cdot \spacegrad) \Bu)_i
&=
n_a (\spacegrad \cross \Bu)_b \epsilon_{a b i}  + 2 n_a \partial_a u_i \\
&=
n_a \partial_r u_s \epsilon_{r s b} \epsilon_{a b i}  + 2 n_a \partial_a u_i \\
&=
n_a \partial_r u_s \delta_{ia}^{[rs]} + 2 n_a \partial_a u_i \\
&=
n_a ( \partial_i u_a -\partial_a u_i ) + 2 n_a \partial_a u_i \\
&=
n_a \partial_i u_a + n_a \partial_a u_i \\
&=
n_a (\partial_i u_a + \partial_a u_i) \\
&=
\sigma_{i a } n_a 
\end{align*}

We can also put the double cross product in wedge product form

\begin{align*}
\ncap \cross (\spacegrad \cross \Bu)
&=
-I \ncap \wedge (\spacegrad \cross \Bu) \\
&=
-\frac{I}{2} 
\left(
\ncap (\spacegrad \cross \Bu) 
- (\spacegrad \cross \Bu) \ncap
\right) \\
&=
-\frac{I}{2} 
\left(
-I \ncap (\spacegrad \wedge \Bu) 
+ I (\spacegrad \wedge \Bu) \ncap
\right) \\
&=
-\frac{I^2}{2} 
\left(
- \ncap (\spacegrad \wedge \Bu) 
+ (\spacegrad \wedge \Bu) \ncap
\right) \\
&=
(\spacegrad \wedge \Bu) \cdot \ncap
\end{align*}

Equivalently (and easier) we can just expand the dot product of the wedge and the vector using the relationship

\begin{equation}\label{eqn:stressTensorVectorForm:70}
\Ba \cdot (\Bc \wedge \Bd \wedge \Be \wedge \cdots )
=
(\Ba \cdot \Bc) (\Bd \wedge \Be \wedge \cdots ) - (\Ba \cdot \Bd) (\Bc \wedge \Be \wedge \cdots ) +
\end{equation}

so we find

\begin{align*}
((\spacegrad \wedge \Bu) \cdot \ncap + 2 (\ncap \cdot \spacegrad) \Bu
)_i
&=
(
\spacegrad' (\Bu' \cdot \ncap)
-
(\ncap \cdot \spacegrad) \Bu
+ 2 (\ncap \cdot \spacegrad) \Bu
)_i \\
&=
\partial_i u_a n_a
+
n_a \partial_a u_i \\
&=
\sigma_{ia} n_a.
\end{align*}

\section{Cylindrical strain tensor.}

Let's now compute the strain tensor (and implicitly the traction vector) in cylindrical coordinates.

\section{Spherical strain tensor.}

Having done a first order cylindrical derivation of the strain tensor, let's also do the spherical case for completeness.  Would this have much utility in fluids?  Perhaps for flow over a spherical barrier?

We need the gradient in spherical coordinates.  Recall that our spherical coordinate velocity was

\begin{equation}\label{eqn:stressTensorVectorForm:90}
\frac{d\Br}{dt} = \rcap \rdot + \thetacap (r \thetadot) + \phicap ( r \sin\theta \phidot ),
\end{equation}

and our gradient mirrors this structure

\begin{equation}\label{eqn:stressTensorVectorForm:110}
\spacegrad = \rcap \PD{r}{} + \thetacap \inv{r }\PD{\theta}{} + \phicap \inv{r \sin\theta} \PD{\phi}{}
\end{equation}

We also previously calculated \inbookref{phy454:continuumL2}{eqn:continuumL2:1010} the unit vector differentials

\begin{subequations}
\begin{equation}\label{eqn:stressTensorVectorForm:130}
d\rcap = \phicap \sin\theta d\phi + \thetacap d\theta 
\end{equation}
\begin{equation}\label{eqn:stressTensorVectorForm:150}
d\thetacap = \phicap \cos\theta d\phi - \rcap d\theta 
\end{equation}
\begin{equation}\label{eqn:stressTensorVectorForm:170}
d\phicap = -(\rcap \sin\theta + \thetacap \cos\theta) d\phi
\end{equation}
\end{subequations}

and can use those to read off the partials of all the unit vectors

\begin{align}\label{eqn:stressTensorVectorForm:190}
\frac{\partial \rcap}{\partial \{r,\theta, \phi\}} &= \{0, \thetacap, \phicap \sin\theta \} \\
\frac{\partial \thetacap}{\partial \{r,\theta, \phi\}} &= \{0, -\rcap, \phicap \cos\theta \} \\
\frac{\partial \phicap}{\partial \{r,\theta, \phi\}} &= \{0, 0, -\rcap \sin\theta -\thetacap \cos\theta \}
\end{align}

Finally, our velocity in spherical coordinates is just

\begin{equation}\label{eqn:stressTensorVectorForm:210}
\Bu = \rcap u_r + \thetacap u_\theta + \phicap u_\phi,
\end{equation}

from which we can now compute the curl, and the directional derivative.  Starting with the curl we have

\begin{align*}
\spacegrad \wedge \Bu 
&=
\left( \rcap \PD{r}{} + \thetacap \inv{r }\PD{\theta}{} + \phicap \inv{r \sin\theta} \PD{\phi}{} \right) \wedge
\left( \rcap u_r + \thetacap u_\theta + \phicap u_\phi \right) \\
&=
\rcap \wedge \thetacap
\left( \partial_r u_\theta - \inv{r} \partial_\theta u_r
\right) 
\\
& +
\thetacap \wedge \phicap
\left(
\inv{r} \partial_\theta u_\phi - \inv{r \sin\theta} \partial_\phi u_\theta
\right)
\\
& +
\phicap \wedge \rcap
\left(
\inv{r \sin\theta} \partial_\phi u_r - \partial_r u_\phi
\right)
\\
& +
\inv{r} \thetacap \wedge \left(
u_\theta \underbrace{\partial_\theta \thetacap}_{-\rcap}
+
u_\phi \underbrace{\partial_\theta \phicap}_{0}
\right)
\\
& +
\inv{r \sin\theta} \phicap \wedge \left(
u_\theta \underbrace{\partial_\phi \thetacap}_{\phicap \cos\theta}
+
u_\phi \underbrace{\partial_\phi \phicap}_{-\rcap \sin\theta - \thetacap \cos\theta}
\right).
\end{align*}

So we have

\begin{equation}\label{eqn:stressTensorVectorForm:230}
\begin{aligned}
\spacegrad \wedge \Bu
&=
\rcap \wedge \thetacap
\left( \partial_r u_\theta - \inv{r} \partial_\theta u_r + \frac{u_\theta}{r}
\right) 
\\
& +
\thetacap \wedge \phicap
\left(
\inv{r} \partial_\theta u_\phi - \inv{r \sin\theta} \partial_\phi u_\theta
+ \frac{u_\phi \cot\theta}{r}
\right)
\\
& +
\phicap \wedge \rcap
\left(
\inv{r \sin\theta} \partial_\phi u_r - \partial_r u_\phi
- \frac{u_\phi}{r}
\right).
\end{aligned}
\end{equation}

\subsection{With $\ncap = \rcap$.}

The directional derivative portion of our strain is

\begin{align*}
2 (\rcap \cdot \spacegrad) \Bu
&=
2 \partial_r (
\rcap u_r + \thetacap u_\theta + \phicap u_\phi ) \\
&=
2 (
\rcap \partial_r u_r + \thetacap \partial_r u_\theta + \phicap \partial_r u_\phi )
\end{align*}

The other portion of our strain tensor is

\begin{align*}
(\spacegrad \wedge \Bu) \cdot \rcap
&=
(\rcap \wedge \thetacap) \cdot \rcap
\left( \partial_r u_\theta - \inv{r} \partial_\theta u_r + \frac{u_\theta}{r}
\right) 
\\
& +
(\thetacap \wedge \phicap) \cdot \rcap
\left(
\inv{r} \partial_\theta u_\phi - \inv{r \sin\theta} \partial_\phi u_\theta
+ \frac{u_\phi \cot\theta}{r}
\right)
\\
& +
(\phicap \wedge \rcap) \cdot \rcap
\left(
\inv{r \sin\theta} \partial_\phi u_r - \partial_r u_\phi
- \frac{u_\phi}{r}
\right) \\
&=
-\thetacap
\left( \partial_r u_\theta - \inv{r} \partial_\theta u_r + \frac{u_\theta}{r}
\right) 
\\
& +
\phicap 
\left(
\inv{r \sin\theta} \partial_\phi u_r - \partial_r u_\phi
- \frac{u_\phi}{r}
\right) 
\end{align*}

Putting these together we find

\begin{align*}
2 {\Be}_{\rcap}
&=
2 (\rcap \cdot \spacegrad)\Bu + (\spacegrad \wedge \Bu) \cdot \rcap \\
&=
2 (
\rcap \partial_r u_r + \thetacap \partial_r u_\theta + \phicap \partial_r u_\phi )
-\thetacap
\left( 
\partial_r u_\theta - \inv{r} \partial_\theta u_r + \frac{u_\theta}{r}
\right) 
+
\phicap 
\left(
\inv{r \sin\theta} \partial_\phi u_r - \partial_r u_\phi
- \frac{u_\phi}{r}
\right) \\
&=
\rcap
\left(
2 \partial_r u_r 
\right)
+
\thetacap
\left(
2 \partial_r u_\theta 
-\partial_r u_\theta + \inv{r} \partial_\theta u_r - \frac{u_\theta}{r}
\right)
+
\phicap
\left(
2 \partial_r u_\phi 
+ \inv{r \sin\theta} \partial_\phi u_r - \partial_r u_\phi
- \frac{u_\phi}{r}
\right)
\end{align*}

Which gives

\begin{equation}\label{eqn:stressTensorVectorForm:270}
2 {\Be}_{\rcap}
=
\rcap
\left(
2 \partial_r u_r 
\right)
+
\thetacap
\left(
\partial_r u_\theta 
+ \inv{r} \partial_\theta u_r - \frac{u_\theta}{r}
\right)
+
\phicap
\left(
\partial_r u_\phi 
+ \inv{r \sin\theta} \partial_\phi u_r 
- \frac{u_\phi}{r}
\right)
\end{equation}

For our stress tensor

\begin{equation}\label{eqn:stressTensorVectorForm:290}
\Bsigma_{\rcap} = - p \rcap + 2 \mu e_{\rcap},
\end{equation}

we can now read off our components by taking dot products

\begin{subequations}
\begin{equation}\label{eqn:stressTensorVectorForm:310}
\sigma_{rr}
=
-p + 2 \mu \PD{r}{u_r}
\end{equation}
\begin{equation}\label{eqn:stressTensorVectorForm:330}
\sigma_{r \theta}
=
\mu \left(
\PD{r}{u_\theta}
+ \inv{r} \PD{\theta}{u_r} - \frac{u_\theta}{r}
\right)
\end{equation}
\begin{equation}\label{eqn:stressTensorVectorForm:350}
\sigma_{r \phi}
=
\mu \left(
\PD{r}{u_\phi}
+ \inv{r \sin\theta} \PD{\phi}{u_r}
- \frac{u_\phi}{r}
\right).
\end{equation}
\end{subequations}

This is consistent with (15.20) from \cite{landau1987course} (after adjusting for minor notational differences).

\subsection{With $\ncap = \thetacap$.}
\subsection{With $\ncap = \phicap$.}

\EndArticle
