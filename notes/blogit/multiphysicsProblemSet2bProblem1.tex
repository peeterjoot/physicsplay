%
% Copyright � 2014 Peeter Joot.  All Rights Reserved.
% Licenced as described in the file LICENSE under the root directory of this GIT repository.
%
\makeproblem{Newton's method}{multiphysics:problemSet2b:1}{ 

Consider the circuit in \cref{fig:ps2b:ps2bFig1}.  By setting \( R = 1, I{{s1} = 5, I_{s2} = 10^{-6}\), and using diode parameters \( I_0 = 10^{-6} \) and \( 1/v_T = 80 \), we can obtain the following nodal analysis equation.

\begin{equation}\label{eqn:multiphysicsProblemSet2bProblem1:20}
x + 10^-6 e^{80x} = 5 
\end{equation}

\imageFigure{../../figures/ece1254/ps2bFig1}{Circuit}{fig:ps2b:ps2bFig1}{0.3}

Write a program to use Newton�s method to solve this equation for x, and use the program to answer the questions below:

\makesubproblem{}{multiphysics:problemSet2b:1a}
Starting with an initial guess of \( x = 0 \), how many Newton iterations are required to compute an \( x \) that is within \( 10^{-6} \) of the exact solution?

\makesubproblem{}{multiphysics:problemSet2b:1b}
How did you determine the accuracy of your solution?

\makesubproblem{}{multiphysics:problemSet2b:1c}
Try using a simple source-stepping style continuation scheme to solve the system of equations, as in

\begin{equation}\label{eqn:multiphysicsProblemSet2bProblem1:40}
x + 10^{-6} e^{80x} = \lambda * 5,
\end{equation}

where \( \lambda \) is a continuation parameter that varies from zero to one. How many Newton iterations do you need to solve the original problem?

\makesubproblem{}{multiphysics:problemSet2b:1d}
Compare the two approaches discussed in the slides for generating an initial guess for each step of your continuation scheme. That is, compare using just the previous step�s converged solution

\begin{equation}\label{eqn:multiphysicsProblemSet2bProblem1:60}
x^0 (\lambda) = x(\lambda_{\textrm{prev}} ),
\end{equation}

to updating the converged solution using the derivative with respect to \( \lambda \), that is:

\begin{equation}\label{eqn:multiphysicsProblemSet2bProblem1:80}
x^0 (\lambda) = x(\lambda_{\textrm{prev}} ) + \frac{dx}{d\lambda} \delta \lambda.
\end{equation}

Does using the derivative help?

} % makeproblem

\makeanswer{multiphysics:problemSet2b:1}{ 
\makeSubAnswer{}{multiphysics:problemSet2b:1a}

TODO.
\makeSubAnswer{}{multiphysics:problemSet2b:1b}

TODO.
\makeSubAnswer{}{multiphysics:problemSet2b:1c}

TODO.
\makeSubAnswer{}{multiphysics:problemSet2b:1d}

TODO.
}

