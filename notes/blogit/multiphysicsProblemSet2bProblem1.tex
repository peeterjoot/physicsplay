%
% Copyright � 2014 Peeter Joot.  All Rights Reserved.
% Licenced as described in the file LICENSE under the root directory of this GIT repository.
%
\makeproblem{Newton's method}{multiphysics:problemSet2b:1}{ 

Consider the circuit in \cref{fig:ps2b:ps2bFig1}.  By setting \( R = 1, I_{s1} = 5, I_{s2} = 10^{-6}\), and using diode parameters \( I_0 = 10^{-6} \) and \( 1/v_T = 80 \), we can obtain the following nodal analysis equation.

\begin{equation}\label{eqn:multiphysicsProblemSet2bProblem1:20}
x + 10^{-6} e^{80x} = 5 
\end{equation}

\imageFigure{../../figures/ece1254/ps2bFig1}{Circuit}{fig:ps2b:ps2bFig1}{0.3}

Write a program to use Newton's method to solve this equation for x, and use the program to answer the questions below:

\makesubproblem{}{multiphysics:problemSet2b:1a}
Starting with an initial guess of \( x = 0 \), how many Newton iterations are required to compute an \( x \) that is within \( 10^{-6} \) of the exact solution?

\makesubproblem{}{multiphysics:problemSet2b:1b}
How did you determine the accuracy of your solution?

\makesubproblem{}{multiphysics:problemSet2b:1c}
Try using a simple source-stepping style continuation scheme to solve the system of equations, as in

\begin{equation}\label{eqn:multiphysicsProblemSet2bProblem1:40}
x + 10^{-6} e^{80x} = \lambda * 5,
\end{equation}

where \( \lambda \) is a continuation parameter that varies from zero to one. How many Newton iterations do you need to solve the original problem?

\makesubproblem{}{multiphysics:problemSet2b:1d}
Compare the two approaches discussed in the slides for generating an initial guess for each step of your continuation scheme. That is, compare using just the previous step's converged solution

\begin{equation}\label{eqn:multiphysicsProblemSet2bProblem1:60}
x^0 (\lambda) = x(\lambda_{\textrm{prev}} ),
\end{equation}

to updating the converged solution using the derivative with respect to \( \lambda \), that is:

\begin{equation}\label{eqn:multiphysicsProblemSet2bProblem1:80}
	x^0 (\lambda) = x(\lambda_{\textrm{prev}} ) + \frac{dx}{d\lambda} \lr{\lambda_{\textrm{prev}}} \delta \lambda.
\end{equation}

Does using the derivative help?

} % makeproblem

\makeanswer{multiphysics:problemSet2b:1}{ 
\makeSubAnswer{}{multiphysics:problemSet2b:1a}

The zero of \( f(x) = x - 5 + 10^{-6} \exp(80 x) \) is sought.  This function is plotted in \cref{fig:ps2bPlotDiode:ps2bPlotDiodeFig2}.  It is clear looking at the low slope at \( x < 0.1 \) that any guess in the \( x < 0.1 \) region, or an iteration that lands in that region, is going to blast \( x \) into a really large range, for which the value of \( f(x) \) will be correspondingly large.

\imageFigure{../../figures/ece1254/ps2bPlotDiodeFig2}{Diode plot}{fig:ps2bPlotDiode:ps2bPlotDiodeFig2}{0.3}

A call to

[x, f, iter, absF, diffX] = \textbf{newtonsDiode}( 0, 1e-6, 1e-6, 1e-6, 1000 ) ;

was used to find the zero \( f(0.192322) = 0 \).  This terminated with \( \text{iter} = 390 \).

\makeSubAnswer{}{multiphysics:problemSet2b:1b}

This iteration was stopped when all of \( f(x) < 10^{-6}, \Abs{\Delta x} < 10^{-6} \) and the relative error \( \Abs{\Delta x}/\Abs{x} < 10^{-6} \).  The last was the final convergence criteria to have been met in this application of Newton's method.

A selection of the convergence threshold values for this Newton's implementation are tabulated in \ref{tab:multiphysicsProblemSet2bProblem1:100}.

\captionedTable{Diode Newton's method convergence}{tab:multiphysicsProblemSet2bProblem1:100}{
\begin{tabular}{|l|l|l|l|l|}
\hline
i &  x &  \( x - 5 + 10^{-6} e^{80 x} \) &  \(\Abs{\Delta x} \) &  \( \Abs{\Delta x}/\Abs{x} \) \\ \hline
0   &  0.000000 &  -5.0 &  0.0 &  0.0 \\ \hline
1   &  4.999599 &  \( 5.1 \times 10^{167} \) &  5.0 &  \( \infty \) \\ \hline
2   &  4.987099 &  \( 1.9 \times 10^{167} \) &  \( 1.3 \times 10^{-2} \) &  \( 2.5 \times 10^{-3} \) \\ \hline
3   &  4.974599 &  \( 6.8 \times 10^{166} \) &  \( 1.3 \times 10^{-2} \) &  \( 2.5 \times 10^{-3} \) \\ \hline
4   &  4.962099 &  \( 2.5 \times 10^{166} \) &  \( 1.3 \times 10^{-2} \) &  \( 2.5 \times 10^{-3} \) \\ \hline
5   &  4.949599 &  \( 9.3 \times 10^{165} \) &  \( 1.3 \times 10^{-2} \) &  \( 2.5 \times 10^{-3} \) \\ \hline
%6   &  4.937099 &  \( 3.4 \times 10^{165} \) &  \( 1.3 \times 10^{-2} \) &  \( 2.5 \times 10^{-3} \) \\ \hline
%7   &  4.924599 &  \( 1.3 \times 10^{165} \) &  \( 1.3 \times 10^{-2} \) &  \( 2.5 \times 10^{-3} \) \\ \hline
%8   &  4.912099 &  \( 4.6 \times 10^{164} \) &  \( 1.3 \times 10^{-2} \) &  \( 2.5 \times 10^{-3} \) \\ \hline
%9   &  4.899599 &  \( 1.7 \times 10^{164} \) &  \( 1.3 \times 10^{-2} \) &  \( 2.5 \times 10^{-3} \) \\ \hline
50  &  4.387099 &  \( 2.7 \times 10^{146} \) &  \( 1.3 \times 10^{-2} \) &  \( 2.8 \times 10^{-3} \) \\ \hline
100 &  3.762099 &  \( 5.1 \times 10^{124} \) &  \( 1.3 \times 10^{-2} \) &  \( 3.3 \times 10^{-3} \) \\ \hline
150 &  3.137099 &  \( 9.9 \times 10^{102} \) &  \( 1.3 \times 10^{-2} \) &  \( 4.0 \times 10^{-3} \) \\ \hline
200 &  2.512099 &  \( 1.9 \times 10^{81} \) &  \( 1.3 \times 10^{-2} \) &  \( 5.0 \times 10^{-3} \) \\ \hline
250 &  1.887099 &  \( 3.7 \times 10^{59} \) &  \( 1.2 \times 10^{-2} \) &  \( 6.6 \times 10^{-3} \) \\ \hline
300 &  1.262099 &  \( 7.1 \times 10^{37} \) &  \( 1.2 \times 10^{-2} \) &  \( 9.8 \times 10^{-3} \) \\ \hline
350 &  0.637099 &  \( 1.4 \times 10^{16} \) &  \( 1.2 \times 10^{-2} \) &  \( 1.9 \times 10^{-2} \) \\ \hline
381 &  0.249672 &  \( 4.7 \times 10^{2} \) &  \( 1.2 \times 10^{-2} \) &  \( 4.8 \times 10^{-2} \) \\ \hline
382 &  0.237298 &  \( 1.7 \times 10^{2} \) &  \( 1.2 \times 10^{-2} \) &  \( 5.0 \times 10^{-2} \) \\ \hline
383 &  0.225138 &  \( 6.2 \times 10^{1} \) &  \( 1.2 \times 10^{-2} \) &  \( 5.1 \times 10^{-2} \) \\ \hline
384 &  0.213539 &  \( 2.1 \times 10^{1} \) &  \( 1.2 \times 10^{-2} \) &  \( 5.2 \times 10^{-2} \) \\ \hline
385 &  0.203324 &  6.8  &  \( 1.0 \times 10^{-2} \) &  \( 4.8 \times 10^{-2} \) \\ \hline
386 &  0.196004 &  1.7  &  \( 7.3 \times 10^{-3} \) &  \( 3.6 \times 10^{-2} \) \\ \hline
387 &  0.192813 &  \( 1.9 \times 10^{-1} \) &  \( 3.2 \times 10^{-3} \) &  \( 1.6 \times 10^{-2} \) \\ \hline
388 &  0.192331 &  \( 3.7 \times 10^{-3} \) &  \( 4.8 \times 10^{-4} \) &  \( 2.5 \times 10^{-3} \) \\ \hline
389 &  0.192322 &  \( 1.4 \times 10^{-6} \) &  \( 9.5 \times 10^{-6} \) &  \( 5.0 \times 10^{-5} \) \\ \hline
390 &  0.192322 &  \( 2.0 \times 10^{-13} \) &  \( 3.6 \times 10^{-9} \) &  \( 1.9 \times 10^{-8} \) \\ \hline
\end{tabular}
}



\makeSubAnswer{}{multiphysics:problemSet2b:1c}

TODO.
\makeSubAnswer{}{multiphysics:problemSet2b:1d}

TODO.
}

