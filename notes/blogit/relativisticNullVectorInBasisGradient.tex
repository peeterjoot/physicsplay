%
% Copyright � 2016 Peeter Joot.  All Rights Reserved.
% Licenced as described in the file LICENSE under the root directory of this GIT repository.
%
%{
\newcommand{\authorname}{Peeter Joot}
\newcommand{\email}{peeterjoot@protonmail.com}
\newcommand{\basename}{FIXMEbasenameUndefined}
\newcommand{\dirname}{notes/FIXMEdirnameUndefined/}

\renewcommand{\basename}{relativisticNullVectorInBasisGradient}
\renewcommand{\dirname}{notes/phy1520/}
%\newcommand{\dateintitle}{}
%\newcommand{\keywords}{}

\newcommand{\authorname}{Peeter Joot}
\newcommand{\onlineurl}{http://sites.google.com/site/peeterjoot2/math2013/\basename.pdf}
\newcommand{\sourcepath}{\dirname\basename.tex}
\newcommand{\generatetitle}[1]{\chapter{#1}}

\newcommand{\vcsinfo}{%
\section*{}
\noindent{\color{DarkOliveGreen}{\rule{\linewidth}{0.1mm}}}
\paragraph{Document version}
%\paragraph{\color{Maroon}{Document version}}
{
\small
\begin{itemize}
\item Available online at:\\ 
\href{\onlineurl}{\onlineurl}
\item Git Repository: \input{./.revinfo/gitRepo.tex}
\item Source: \sourcepath
\item last commit: \input{./.revinfo/gitCommitString.tex}
\item commit date: \input{./.revinfo/gitCommitDate.tex}
\end{itemize}
}
}

%\PassOptionsToPackage{dvipsnames,svgnames}{xcolor}
\PassOptionsToPackage{square,numbers}{natbib}
\documentclass{scrreprt}

\usepackage[left=2cm,right=2cm]{geometry}
\usepackage[svgnames]{xcolor}
\usepackage{peeters_layout}

\usepackage{natbib}

\usepackage[
colorlinks=true,
bookmarks=false,
pdfauthor={\authorname, \email},
backref 
]{hyperref}

% http://tex.stackexchange.com/questions/75773/how-to-reference-problems-by-the-text-label-in-an-exercise-envioronment
\usepackage[english]{cleveref}
\crefname{Exercise}{exercise}{exercises}
\Crefname{Exercise}{Exercise}{Exercises}

\RequirePackage{titlesec}
\RequirePackage{ifthen}

% http://stackoverflow.com/questions/4932910/date-in-the-tabular-environment
\makeatletter
\let\insertdate\@date
\makeatother

\titleformat{\chapter}[display]
{\bfseries\Large}
{\color{DarkSlateGrey}\filleft \authorname
\ifthenelse{\isundefined{\studentnumber}}{}{\\ \studentnumber}
\ifthenelse{\isundefined{\email}}{}{\\ \email}
\ifthenelse{\isundefined{\dateintitle}}{}{\\ \insertdate}
%\ifthenelse{\isundefined{\coursename}}{}{\\ \coursename} % put in title instead.
}
{4ex}
{\color{DarkOliveGreen}{\titlerule}\color{Maroon}
\vspace{2ex}%
\filright}
[\vspace{2ex}%
\color{DarkOliveGreen}\titlerule
]

\newcommand{\beginArtWithToc}[0]{\begin{document}\tableofcontents}
\newcommand{\beginArtNoToc}[0]{\begin{document}}
\newcommand{\EndNoBibArticle}[0]{\end{document}}
\newcommand{\EndArticle}[0]{\bibliography{Bibliography}\bibliographystyle{plainnat}\end{document}}

% 
%\newcommand{\citep}[1]{\cite{#1}}

\colorSectionsForArticle



\usepackage{peeters_layout_exercise}
\usepackage{peeters_braket}
\usepackage{peeters_figures}
\usepackage{siunitx}

\beginArtNoToc

\generatetitle{XXX}
%\chapter{XXX}
%\label{chap:relativisticNullVectorInBasisGradient}
% \citep{sakurai2014modern} pr X.Y
% \citep{pozar2009microwave}
% \citep{qftLectureNotes}

One of \( e_1 \) and \( e_2 \) being null does not imply that \( I \) has no inverse.  For example, consider a Dirac (relativistic) basis \( \setlr{\gamma_0, \gamma_1} \), where

\begin{dmath}\label{eqn:relativisticNullVectorInBasisGradient:20}
\gamma_0^2 = 1 = -\gamma_1^2,
\end{dmath}

and
\begin{dmath}\label{eqn:relativisticNullVectorInBasisGradient:40}
\begin{aligned}
\gamma^0 &= \gamma_0 \\
\gamma^1 &= -\gamma_1 \\
\end{aligned}
\end{dmath}

Suppose that
\begin{dmath}\label{eqn:relativisticNullVectorInBasisGradient:60}
\begin{aligned}
e_x &= \gamma_0 \\
e_y &= \gamma_0 + \gamma_1.
\end{aligned}
\end{dmath}

These vectors can be used as a basis for the space, and have the properties specified \( e_x^2 = 1, e_y^2 = 0 \).  For this basis the pseudoscalar is

\begin{dmath}\label{eqn:relativisticNullVectorInBasisGradient:80}
\begin{aligned}
I 
&= e_x \wedge e_y \\
&= \gamma_0 \wedge \lr{ \gamma_0 + \gamma_1 } \\
&= \gamma_0 \wedge \gamma_1 \\
&= \gamma_0 \gamma_1.
\end{aligned}
\end{dmath}

This is actually its own inverse

\begin{dmath}\label{eqn:relativisticNullVectorInBasisGradient:100}
\begin{aligned}
I^{-1} 
&=  \gamma^1 \gamma^0 \\
&= -\gamma_1 \gamma_0 \\
&=  \gamma_0 \gamma_1 \\
&= I.
\end{aligned}
\end{dmath}

The reciprocal basis vectors can now be calculated
\begin{dmath}\label{eqn:relativisticNullVectorInBasisGradient:120}
\begin{aligned}
e^x
&= e_y I \\
&= \lr{ \gamma_0 + \gamma_1 } \gamma^1 \gamma^0 \\
&= \gamma^0 - \gamma^1,
\end{aligned}
\end{dmath}

and
\begin{dmath}\label{eqn:relativisticNullVectorInBasisGradient:140}
\begin{aligned}
e^y
&= -e_x I \\
&= -\gamma_0 \gamma^1 \gamma^0 \\
&= \gamma^1.
\end{aligned}
\end{dmath}

In this case the representation of the gradient corresponding to the vector parameterization \( \Bx(x,y) = x e_x + y e_y = (x + y) \gamma_0 + y \gamma_1 \) is

\begin{dmath}\label{eqn:relativisticNullVectorInBasisGradient:160}
\spacegrad = \lr{ \gamma^0 - \gamma^1 } \PD{x}{} + \gamma^1 \PD{y}{}.
\end{dmath}

Aside: An alternative way of looking at this problem is presented in the excellent little text "Vector and Geometric Calculus" by Alan Macdonald.  See 5.4 Curvilinear Coordinates, and 5.5 The Vector Derivative.  For a complete parameterization of the space, as is the case here, the vector derivative (defined for curvilinear coordinates on a surface) is identical to the gradient, and the text presents the differential operators that can be used to define both the curvilinear basis and its reciprocal basis.  In this case neither basis varies with the parameterization, but the toolbox of curvilinear coordinates applies nicely.

%}
\EndArticle
%\EndNoBibArticle
