%
% Copyright � 2013 Peeter Joot.  All Rights Reserved.
% Licenced as described in the file LICENSE under the root directory of this GIT repository.
%
\newcommand{\authorname}{Peeter Joot}
\newcommand{\email}{peeterjoot@protonmail.com}
\newcommand{\basename}{FIXMEbasenameUndefined}
\newcommand{\dirname}{notes/FIXMEdirnameUndefined/}

\renewcommand{\basename}{condensedMatterLecture23}
\renewcommand{\dirname}{notes/phy487/}
\newcommand{\keywords}{Condensed matter physics, PHY487H1F}
\newcommand{\authorname}{Peeter Joot}
\newcommand{\onlineurl}{http://sites.google.com/site/peeterjoot2/math2013/\basename.pdf}
\newcommand{\sourcepath}{\dirname\basename.tex}
\newcommand{\generatetitle}[1]{\chapter{#1}}

\newcommand{\vcsinfo}{%
\section*{}
\noindent{\color{DarkOliveGreen}{\rule{\linewidth}{0.1mm}}}
\paragraph{Document version}
%\paragraph{\color{Maroon}{Document version}}
{
\small
\begin{itemize}
\item Available online at:\\ 
\href{\onlineurl}{\onlineurl}
\item Git Repository: \input{./.revinfo/gitRepo.tex}
\item Source: \sourcepath
\item last commit: \input{./.revinfo/gitCommitString.tex}
\item commit date: \input{./.revinfo/gitCommitDate.tex}
\end{itemize}
}
}

%\PassOptionsToPackage{dvipsnames,svgnames}{xcolor}
\PassOptionsToPackage{square,numbers}{natbib}
\documentclass{scrreprt}

\usepackage[left=2cm,right=2cm]{geometry}
\usepackage[svgnames]{xcolor}
\usepackage{peeters_layout}

\usepackage{natbib}

\usepackage[
colorlinks=true,
bookmarks=false,
pdfauthor={\authorname, \email},
backref 
]{hyperref}

% http://tex.stackexchange.com/questions/75773/how-to-reference-problems-by-the-text-label-in-an-exercise-envioronment
\usepackage[english]{cleveref}
\crefname{Exercise}{exercise}{exercises}
\Crefname{Exercise}{Exercise}{Exercises}

\RequirePackage{titlesec}
\RequirePackage{ifthen}

% http://stackoverflow.com/questions/4932910/date-in-the-tabular-environment
\makeatletter
\let\insertdate\@date
\makeatother

\titleformat{\chapter}[display]
{\bfseries\Large}
{\color{DarkSlateGrey}\filleft \authorname
\ifthenelse{\isundefined{\studentnumber}}{}{\\ \studentnumber}
\ifthenelse{\isundefined{\email}}{}{\\ \email}
\ifthenelse{\isundefined{\dateintitle}}{}{\\ \insertdate}
%\ifthenelse{\isundefined{\coursename}}{}{\\ \coursename} % put in title instead.
}
{4ex}
{\color{DarkOliveGreen}{\titlerule}\color{Maroon}
\vspace{2ex}%
\filright}
[\vspace{2ex}%
\color{DarkOliveGreen}\titlerule
]

\newcommand{\beginArtWithToc}[0]{\begin{document}\tableofcontents}
\newcommand{\beginArtNoToc}[0]{\begin{document}}
\newcommand{\EndNoBibArticle}[0]{\end{document}}
\newcommand{\EndArticle}[0]{\bibliography{Bibliography}\bibliographystyle{plainnat}\end{document}}

% 
%\newcommand{\citep}[1]{\cite{#1}}

\colorSectionsForArticle



%\citep{harald2003solid} \S x.y
%\citep{ibach2009solid} \S x.y

%\usepackage{mhchem}
\usepackage[version=3]{mhchem}
\usepackage{units}
\usepackage{bm} % \EE
\newcommand{\nought}[0]{\circ}
%\newcommand{\EF}[0]{\epsilon_{\mathrm{F}}}
\newcommand{\EF}[0]{E_{\mathrm{F}}}
\newcommand{\kF}[0]{k_{\mathrm{F}}}

\beginArtNoToc
\generatetitle{PHY487H1F Condensed Matter Physics.  Lecture 23: Superconductivity.  Taught by Prof.\ Stephen Julian}
%\chapter{Superconductivity}
\label{chap:condensedMatterLecture23}

%\section{Disclaimer}
%
%Peeter's lecture notes from class.  May not be entirely coherent.

\section{Superconductivity}

\index{superconductivity}

Reading: \citep{ibach2009solid} \S 10.1

The effect is actually not as rare as one would imagine (see \citep{ibach2009solid} fig. 10.2 for a chart shown of many superconductive elements).  Even \ce{O} and \ce{S} are superconductive, with fairly high transition temperatures, if pressurized enough to metalize it.

Some of the non-superconductive elements are those with very spherical Fermi surfaces.

The basic phenomina is sketched in

F1: electrical resistivity

F2: specific heat of superconductor

The exponential indicates that there's a gap to the first excited state (cf exponentials from the semiconductor theory?)

The essential feature of superconductivity is not that they are ``perfect conductors'', but that they are perfect dimagnets as illustrated in \citep{ibach2009solid} fig. 10.4.

F3: diamagnetism illustrated

We have a circulating current that screens the field.  This costs energy.  There are two types of superconductors

\begin{itemize}
\item Type I, where the superconductivity collapses above $H_{\mathrm{c}}$.
\item Type II, particle flux expulsion above $H_{\mathrm{c}}$.
\end{itemize}

F4: Type I
F5: Type II

\section{London equations, and perfect conductors}

Reading: \citep{ibach2009solid} \S 10.2

With resistivity $\rho = 0$, we have

\begin{dmath}\label{eqn:qmSL23:20}
m \dot{\Bv} = - e \EE,
\end{dmath}

and current density

\begin{dmath}\label{eqn:qmSL23:40}
\Bj = - n_{\mathrm{s}} e \Bv,
\end{dmath}

for

\begin{dmath}\label{eqn:qmSL23:60}
\PD{t}{\Bj} = \frac{n_{\mathrm{s}} e^2}{m} \EE,
\end{dmath}

With Maxwell's third

\begin{dmath}\label{eqn:qmSL23:80}
\spacegrad \cross \EE = - \PD{t}{\BE},
\end{dmath}

we have

\begin{dmath}\label{eqn:qmSL23:100}
\PD{t}{} \lr{ 
\frac{m}{n_{\mathrm{s}} e^2} \spacegrad \cross \Bj + \BB
}
= 0,
\end{dmath}

so that the equation for a \underlineAndIndex{perfect diamagnet} is

\begin{dmath}\label{eqn:qmSL23:120}
\myBoxed{
\frac{m}{n_{\mathrm{s}} e^2} \spacegrad \cross \Bj = - \BB
}
\end{dmath}

\begin{equation}\label{eqn:qmSL23:140}
\spacegrad \cross \Bj = - \frac{n_{\mathrm{s}} e^2}{m} \BB = -\inv{\lambda_{\mathrm{L}}} \BB.
\end{equation}

\paragraph{Proof:}

With 

\begin{equation}\label{eqn:qmSL23:260}
\BB = (B_x, 0, 0).
\end{equation}

\begin{equation}\label{eqn:qmSL23:160}
\spacegrad \cross \BB = \mu_\nought \Bj,
\end{equation}

so that 

\begin{equation}\label{eqn:qmSL23:180}
\spacegrad \cross \lr{ \spacegrad \cross \Bj } = - \frac{\mu_\nought}{\lambda_{\mathrm{L}}} \Bj,
\end{equation}

or
\begin{equation}\label{eqn:qmSL23:200}
-\spacegrad^2 \Bj = - \frac{\mu_\nought}{\lambda_{\mathrm{L}}} \Bj,
\end{equation}

This gives

\begin{equation}\label{eqn:qmSL23:220}
\Bj = \Bj_\nought e^{\sqrt{\mu_\nought/\lambda_{\mathrm{L}}} z},
\end{equation}

and 
\begin{equation}\label{eqn:qmSL23:240}
\BB = \xcap B_x e^{-\sqrt{\mu_\nought/\lambda_{\mathrm{L}}} z}.
\end{equation}

\section{Cooper pairing}

Reading: \citep{ibach2009solid} \S 10.3

There's an effective attraction between the electrons in a metal that is mediated by phonons.

F7: lattice distorts after the electron passes.

The second electron can lower it's energy by travelling in the ``wake'' of the first.  Note that this is actually an incorrect picture, since what really goes on is that the second electron passes in the opposite direction from the first to take advantage of the wake.

It is important that the lattice reaction is retarded in time.

\paragraph{electron-phonon interaction}

The Cooper calculation considered a filled Fermi surface (such as a sphere) at $T = 0$.  He considered what happens if you add two electrons above $\EF$.  

F8: Filled fermi sphere at $T = 0$

The electron phonon interaction looks like 

\begin{equation}\label{eqn:qmSL23:280}
\rho_{\mathrm{el}}( x, t ) \rho_{\mathrm{ph}}(x, t )
\end{equation}

\begin{equation}\label{eqn:qmSL23:n}
\ket{ \Bk, \up } \rightarrow \ket{ \Bk, \uparrow, \text{ 0 phonons }}
+ \sum_\Bq \alpha_q | \Bk - \Bq_i \ket{ \Bq }
\end{equation}

Here the state has a Bloch form

\begin{equation}\label{eqn:qmSL23:n}
\ket{ \Bk, \up } = \inv{\sqrt{L^3}} e^{i \Bk \cdot \Br } \ket{\sigma}
\end{equation}

electron density

\begin{dmath}\label{eqn:qmSL23:n}
\braket{\psi}{\psi} = 
\inv{L^3} 
\lr{ 
e^{-i \Bk \cdot \Br} \bra{0_q} + \alpha_q e^{-i(\Bk - \Bq) \cdot \Br} \bra{1_q}
}
\lr{ 
e^{i \Bk \cdot \Br} \ket{0_q} + e^{i(\Bk - \Bq) \cdot \Br} \ket{1_q}
}
=
\inv{L^3} 
{
1 + \alpha_q^2 + \alpha_q \lr{ 
e^{i \Bq \cdot \Br }
+ e^{-i \Bq \cdot \Br }
}
}
=
\inv{L^3} 
{
1 + \alpha_q^2 + \alpha_q \cos\lr{ \Bq \cdot \Br }
}
\end{dmath}

Here the cosine is the modulated charge density.

F9:

\EndArticle
