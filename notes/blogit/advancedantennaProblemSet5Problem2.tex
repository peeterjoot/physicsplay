%
% Copyright � 2015 Peeter Joot.  All Rights Reserved.
% Licenced as described in the file LICENSE under the root directory of this GIT repository.
%
\makeproblem{Patch antenna.}{advancedantenna:problemSet5:2}{ 

A microstrip patch antenna is printed on a substrate with \( h = 0.1588 \si{cm} \), \( \epsilon_r = 2.2 \) at \( f_0 = 10 \si{GHz} \).
Give your length answers in \si{cm}.

Using the transmission-line model :

\makesubproblem{}{advancedantenna:problemSet5:2a}
Calculate the width \( W \).

\makesubproblem{}{advancedantenna:problemSet5:2b}
Calculate the effective relative permittivity \( \epsilon_{\textrm{eff}} \).

\makesubproblem{}{advancedantenna:problemSet5:2c}
Calculate the length of the patch \( L_0 \) 
if no fringing-field effects are accounted for.

\makesubproblem{}{advancedantenna:problemSet5:2d}
Calculate the corrected length \( L = L_0 - m \Delta L \)
where \( m = 2 \) and \( \Delta L \) is the correction due
to the fringing fields.

\makesubproblem{}{advancedantenna:problemSet5:2e}
Estimate the admittance of each radiating slot \( Y_\txts = G + j B \).

\makesubproblem{}{advancedantenna:problemSet5:2f}

Now transform \( Y_\txts \) of the second slot (right) to the plane of the first slot (left) using the
impedance transformation,

\begin{dmath}\label{eqn:advancedantennaProblemSet5Problem2:20}
Z_{\textrm{in} 2} = Z_0 \frac
{Z_\txts + j Z_0 \tan(\beta L) }
{Z_0 + j Z_\txts \tan(\beta L) }
\end{dmath}

where \( \beta = k_0 \sqrt{\epsilon_{\textrm{ff}}} \) 
is the effective propagation constant, and use \( Z_0 = 26 \Omega \) as the
characteristic impedance of the microstrip line. 
What is the value of \( Z_{\textrm{in} 2} \) and \( Y_{\textrm{in} 2} = 1/Z_{\textrm{in} 2} \).

\makesubproblem{}{advancedantenna:problemSet5:2g}
Based on the above, calculate the total input impedance of the patch antenna \( Z_{\textrm{in}} \) 
at the terminals of the first slot.

\makesubproblem{}{advancedantenna:problemSet5:2h}
If the imaginary part of \( Z_{\textrm{in}} \)
is not zero, adjust the length parameter m (in \partref{advancedantenna:problemSet5:2d})
between \( 0 < 3 < m \) to make the patch resonant (i.e. make the imaginary part of \( Z_{\textrm{in}} \) vanish).
What is the new input impedance in this case?  } % makeproblem

\makeanswer{advancedantenna:problemSet5:2}{ 
\makeSubAnswer{}{advancedantenna:problemSet5:2a}
TODO.
\makeSubAnswer{}{advancedantenna:problemSet5:2b}
TODO.
\makeSubAnswer{}{advancedantenna:problemSet5:2c}
TODO.
\makeSubAnswer{}{advancedantenna:problemSet5:2d}
TODO.
\makeSubAnswer{}{advancedantenna:problemSet5:2e}
TODO.
\makeSubAnswer{}{advancedantenna:problemSet5:2f}
TODO.
\makeSubAnswer{}{advancedantenna:problemSet5:2g}
TODO.
\makeSubAnswer{}{advancedantenna:problemSet5:2h}
TODO.
}
