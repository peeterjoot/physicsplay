%
% Copyright � 2015 Peeter Joot.  All Rights Reserved.
% Licenced as described in the file LICENSE under the root directory of this GIT repository.
%
\documentclass[]{eliblog}

\usepackage{amsmath}
\usepackage{mathpazo}

%
% shorthand for bold symbols, convenient for vectors and matrices
%
\newcommand{\Ba}[0]{\mathbf{a}}
\newcommand{\Bb}[0]{\mathbf{b}}
\newcommand{\Bc}[0]{\mathbf{c}}
\newcommand{\Bd}[0]{\mathbf{d}}
\newcommand{\Be}[0]{\mathbf{e}}
\newcommand{\Bf}[0]{\mathbf{f}}
\newcommand{\Bg}[0]{\mathbf{g}}
\newcommand{\Bh}[0]{\mathbf{h}}
\newcommand{\Bi}[0]{\mathbf{i}}
\newcommand{\Bj}[0]{\mathbf{j}}
\newcommand{\Bk}[0]{\mathbf{k}}
\newcommand{\Bl}[0]{\mathbf{l}}
\newcommand{\Bm}[0]{\mathbf{m}}
\newcommand{\Bn}[0]{\mathbf{n}}
\newcommand{\Bo}[0]{\mathbf{o}}
\newcommand{\Bp}[0]{\mathbf{p}}
\newcommand{\Bq}[0]{\mathbf{q}}
\newcommand{\Br}[0]{\mathbf{r}}
\newcommand{\Bs}[0]{\mathbf{s}}
\newcommand{\Bt}[0]{\mathbf{t}}
\newcommand{\Bu}[0]{\mathbf{u}}
\newcommand{\Bv}[0]{\mathbf{v}}
\newcommand{\Bw}[0]{\mathbf{w}}
\newcommand{\Bx}[0]{\mathbf{x}}
\newcommand{\By}[0]{\mathbf{y}}
\newcommand{\Bz}[0]{\mathbf{z}}
\newcommand{\BA}[0]{\mathbf{A}}
\newcommand{\BB}[0]{\mathbf{B}}
\newcommand{\BC}[0]{\mathbf{C}}
\newcommand{\BD}[0]{\mathbf{D}}
\newcommand{\BE}[0]{\mathbf{E}}
\newcommand{\BF}[0]{\mathbf{F}}
\newcommand{\BG}[0]{\mathbf{G}}
\newcommand{\BH}[0]{\mathbf{H}}
\newcommand{\BI}[0]{\mathbf{I}}
\newcommand{\BJ}[0]{\mathbf{J}}
\newcommand{\BK}[0]{\mathbf{K}}
\newcommand{\BL}[0]{\mathbf{L}}
\newcommand{\BM}[0]{\mathbf{M}}
\newcommand{\BN}[0]{\mathbf{N}}
\newcommand{\BO}[0]{\mathbf{O}}
\newcommand{\BP}[0]{\mathbf{P}}
\newcommand{\BQ}[0]{\mathbf{Q}}
\newcommand{\BR}[0]{\mathbf{R}}
\newcommand{\BS}[0]{\mathbf{S}}
\newcommand{\BT}[0]{\mathbf{T}}
\newcommand{\BU}[0]{\mathbf{U}}
\newcommand{\BV}[0]{\mathbf{V}}
\newcommand{\BW}[0]{\mathbf{W}}
\newcommand{\BX}[0]{\mathbf{X}}
\newcommand{\BY}[0]{\mathbf{Y}}
\newcommand{\BZ}[0]{\mathbf{Z}}

\newcommand{\Bzero}[0]{\mathbf{0}}
\newcommand{\Btheta}[0]{\boldsymbol{\theta}}
\newcommand{\Btau}[0]{\boldsymbol{\tau}}
\newcommand{\Bomega}[0]{\boldsymbol{\omega}}

%
% shorthand for unit vectors
%
\newcommand{\acap}[0]{\hat{\Ba}}
\newcommand{\bcap}[0]{\hat{\Bb}}
\newcommand{\ccap}[0]{\hat{\Bc}}
\newcommand{\dcap}[0]{\hat{\Bd}}
\newcommand{\ecap}[0]{\hat{\Be}}
\newcommand{\fcap}[0]{\hat{\Bf}}
\newcommand{\gcap}[0]{\hat{\Bg}}
\newcommand{\hcap}[0]{\hat{\Bh}}
\newcommand{\icap}[0]{\hat{\Bi}}
\newcommand{\jcap}[0]{\hat{\Bj}}
\newcommand{\kcap}[0]{\hat{\Bk}}
\newcommand{\lcap}[0]{\hat{\Bl}}
\newcommand{\mcap}[0]{\hat{\Bm}}
\newcommand{\ncap}[0]{\hat{\Bn}}
\newcommand{\ocap}[0]{\hat{\Bo}}
\newcommand{\pcap}[0]{\hat{\Bp}}
\newcommand{\qcap}[0]{\hat{\Bq}}
\newcommand{\rcap}[0]{\hat{\Br}}
\newcommand{\scap}[0]{\hat{\Bs}}
\newcommand{\tcap}[0]{\hat{\Bt}}
\newcommand{\ucap}[0]{\hat{\Bu}}
\newcommand{\vcap}[0]{\hat{\Bv}}
\newcommand{\wcap}[0]{\hat{\Bw}}
\newcommand{\xcap}[0]{\hat{\Bx}}
\newcommand{\ycap}[0]{\hat{\By}}
\newcommand{\zcap}[0]{\hat{\Bz}}
\newcommand{\thetacap}[0]{\hat{\Btheta}}

%
% to write R^n and C^n in a distinguishable fashion.  Perhaps change this
% to the double lined characters upon figuring out how to do so.
%
\newcommand{\C}[1]{$\mathbb{C}^{#1}$}
\newcommand{\R}[1]{$\mathbb{R}^{#1}$}

%
% various generally useful helpers
%

% derivative of #1 wrt. #2:
\newcommand{\D}[2] {\frac {d#2} {d#1}}

\newcommand{\inv}[1]{\frac{1}{#1}}
\newcommand{\cross}[0]{\times}

\newcommand{\abs}[1]{\lvert{#1}\rvert}
\newcommand{\norm}[1]{\lVert{#1}\rVert}
\newcommand{\innerprod}[2]{\langle{#1}, {#2}\rangle}
\newcommand{\dotprod}[2]{{#1} \cdot {#2}}
\newcommand{\bdotprod}[2]{\left({#1} \cdot {#2}\right)}
\newcommand{\crossprod}[2]{{#1} \cross {#2}}
\newcommand{\tripleprod}[3]{\dotprod{\left(\crossprod{#1}{#2}\right)}{#3}}

\DeclareMathOperator{\Proj}{Proj}
\DeclareMathOperator{\Span}{span}
\DeclareMathOperator{\Sgn}{sgn}
\DeclareMathOperator{\Area}{Area}
\DeclareMathOperator{\Volume}{Volume}

%
% A few miscellaneous things specific to this document
%
\newcommand{\crossop}[1]{\crossprod{#1}{}}

% R2 vector.
\newcommand{\VectorTwo}[2]{
\begin{bmatrix}
 {#1} \\
 {#2}
\end{bmatrix}
}

\newcommand{\VectorN}[1]{
\begin{bmatrix}
{#1}_1 \\
{#1}_2 \\
\vdots \\
{#1}_N \\
\end{bmatrix}
}

\newcommand{\DETuvij}[4]{
\begin{vmatrix}
 {#1}_{#3} & {#1}_{#4} \\
 {#2}_{#3} & {#2}_{#4}
\end{vmatrix}
}

\newcommand{\DETuvwijk}[6]{
\begin{vmatrix}
 {#1}_{#4} & {#1}_{#5} & {#1}_{#6} \\
 {#2}_{#4} & {#2}_{#5} & {#2}_{#6} \\
 {#3}_{#4} & {#3}_{#5} & {#3}_{#6}
\end{vmatrix}
}

\newcommand{\DETuvwxijkl}[8]{
\begin{vmatrix}
 {#1}_{#5} & {#1}_{#6} & {#1}_{#7} & {#1}_{#8} \\
 {#2}_{#5} & {#2}_{#6} & {#2}_{#7} & {#2}_{#8} \\
 {#3}_{#5} & {#3}_{#6} & {#3}_{#7} & {#3}_{#8} \\
 {#4}_{#5} & {#4}_{#6} & {#4}_{#7} & {#4}_{#8} \\
\end{vmatrix}
}

%\newcommand{\DETuvwxyijklm}[10]{
%\begin{vmatrix}
% {#1}_{#6} & {#1}_{#7} & {#1}_{#8} & {#1}_{#9} & {#1}_{#10} \\
% {#2}_{#6} & {#2}_{#7} & {#2}_{#8} & {#2}_{#9} & {#2}_{#10} \\
% {#3}_{#6} & {#3}_{#7} & {#3}_{#8} & {#3}_{#9} & {#3}_{#10} \\
% {#4}_{#6} & {#4}_{#7} & {#4}_{#8} & {#4}_{#9} & {#4}_{#10} \\
% {#5}_{#6} & {#5}_{#7} & {#5}_{#8} & {#5}_{#9} & {#5}_{#10}
%\end{vmatrix}
%}

% R3 vector.
\newcommand{\VectorThree}[3]{
\begin{bmatrix}
 {#1} \\
 {#2} \\
 {#3}
\end{bmatrix}
}



\author{Peeter Joot}
\email{peeter.joot@gmail.com}

%\usepackage{ulem}
%\usepackage{cancel}
%\documentclass[]{eliblogwidescreen}

\usepackage{amsmath}
\usepackage{mathpazo}

%
% shorthand for bold symbols, convenient for vectors and matrices
%
\newcommand{\Ba}[0]{\mathbf{a}}
\newcommand{\Bb}[0]{\mathbf{b}}
\newcommand{\Bc}[0]{\mathbf{c}}
\newcommand{\Bd}[0]{\mathbf{d}}
\newcommand{\Be}[0]{\mathbf{e}}
\newcommand{\Bf}[0]{\mathbf{f}}
\newcommand{\Bg}[0]{\mathbf{g}}
\newcommand{\Bh}[0]{\mathbf{h}}
\newcommand{\Bi}[0]{\mathbf{i}}
\newcommand{\Bj}[0]{\mathbf{j}}
\newcommand{\Bk}[0]{\mathbf{k}}
\newcommand{\Bl}[0]{\mathbf{l}}
\newcommand{\Bm}[0]{\mathbf{m}}
\newcommand{\Bn}[0]{\mathbf{n}}
\newcommand{\Bo}[0]{\mathbf{o}}
\newcommand{\Bp}[0]{\mathbf{p}}
\newcommand{\Bq}[0]{\mathbf{q}}
\newcommand{\Br}[0]{\mathbf{r}}
\newcommand{\Bs}[0]{\mathbf{s}}
\newcommand{\Bt}[0]{\mathbf{t}}
\newcommand{\Bu}[0]{\mathbf{u}}
\newcommand{\Bv}[0]{\mathbf{v}}
\newcommand{\Bw}[0]{\mathbf{w}}
\newcommand{\Bx}[0]{\mathbf{x}}
\newcommand{\By}[0]{\mathbf{y}}
\newcommand{\Bz}[0]{\mathbf{z}}
\newcommand{\BA}[0]{\mathbf{A}}
\newcommand{\BB}[0]{\mathbf{B}}
\newcommand{\BC}[0]{\mathbf{C}}
\newcommand{\BD}[0]{\mathbf{D}}
\newcommand{\BE}[0]{\mathbf{E}}
\newcommand{\BF}[0]{\mathbf{F}}
\newcommand{\BG}[0]{\mathbf{G}}
\newcommand{\BH}[0]{\mathbf{H}}
\newcommand{\BI}[0]{\mathbf{I}}
\newcommand{\BJ}[0]{\mathbf{J}}
\newcommand{\BK}[0]{\mathbf{K}}
\newcommand{\BL}[0]{\mathbf{L}}
\newcommand{\BM}[0]{\mathbf{M}}
\newcommand{\BN}[0]{\mathbf{N}}
\newcommand{\BO}[0]{\mathbf{O}}
\newcommand{\BP}[0]{\mathbf{P}}
\newcommand{\BQ}[0]{\mathbf{Q}}
\newcommand{\BR}[0]{\mathbf{R}}
\newcommand{\BS}[0]{\mathbf{S}}
\newcommand{\BT}[0]{\mathbf{T}}
\newcommand{\BU}[0]{\mathbf{U}}
\newcommand{\BV}[0]{\mathbf{V}}
\newcommand{\BW}[0]{\mathbf{W}}
\newcommand{\BX}[0]{\mathbf{X}}
\newcommand{\BY}[0]{\mathbf{Y}}
\newcommand{\BZ}[0]{\mathbf{Z}}

\newcommand{\Bzero}[0]{\mathbf{0}}
\newcommand{\Btheta}[0]{\boldsymbol{\theta}}
\newcommand{\Btau}[0]{\boldsymbol{\tau}}
\newcommand{\Bomega}[0]{\boldsymbol{\omega}}

%
% shorthand for unit vectors
%
\newcommand{\acap}[0]{\hat{\Ba}}
\newcommand{\bcap}[0]{\hat{\Bb}}
\newcommand{\ccap}[0]{\hat{\Bc}}
\newcommand{\dcap}[0]{\hat{\Bd}}
\newcommand{\ecap}[0]{\hat{\Be}}
\newcommand{\fcap}[0]{\hat{\Bf}}
\newcommand{\gcap}[0]{\hat{\Bg}}
\newcommand{\hcap}[0]{\hat{\Bh}}
\newcommand{\icap}[0]{\hat{\Bi}}
\newcommand{\jcap}[0]{\hat{\Bj}}
\newcommand{\kcap}[0]{\hat{\Bk}}
\newcommand{\lcap}[0]{\hat{\Bl}}
\newcommand{\mcap}[0]{\hat{\Bm}}
\newcommand{\ncap}[0]{\hat{\Bn}}
\newcommand{\ocap}[0]{\hat{\Bo}}
\newcommand{\pcap}[0]{\hat{\Bp}}
\newcommand{\qcap}[0]{\hat{\Bq}}
\newcommand{\rcap}[0]{\hat{\Br}}
\newcommand{\scap}[0]{\hat{\Bs}}
\newcommand{\tcap}[0]{\hat{\Bt}}
\newcommand{\ucap}[0]{\hat{\Bu}}
\newcommand{\vcap}[0]{\hat{\Bv}}
\newcommand{\wcap}[0]{\hat{\Bw}}
\newcommand{\xcap}[0]{\hat{\Bx}}
\newcommand{\ycap}[0]{\hat{\By}}
\newcommand{\zcap}[0]{\hat{\Bz}}
\newcommand{\thetacap}[0]{\hat{\Btheta}}

%
% to write R^n and C^n in a distinguishable fashion.  Perhaps change this
% to the double lined characters upon figuring out how to do so.
%
\newcommand{\C}[1]{$\mathbb{C}^{#1}$}
\newcommand{\R}[1]{$\mathbb{R}^{#1}$}

%
% various generally useful helpers
%

% derivative of #1 wrt. #2:
\newcommand{\D}[2] {\frac {d#2} {d#1}}

\newcommand{\inv}[1]{\frac{1}{#1}}
\newcommand{\cross}[0]{\times}

\newcommand{\abs}[1]{\lvert{#1}\rvert}
\newcommand{\norm}[1]{\lVert{#1}\rVert}
\newcommand{\innerprod}[2]{\langle{#1}, {#2}\rangle}
\newcommand{\dotprod}[2]{{#1} \cdot {#2}}
\newcommand{\bdotprod}[2]{\left({#1} \cdot {#2}\right)}
\newcommand{\crossprod}[2]{{#1} \cross {#2}}
\newcommand{\tripleprod}[3]{\dotprod{\left(\crossprod{#1}{#2}\right)}{#3}}

\DeclareMathOperator{\Proj}{Proj}
\DeclareMathOperator{\Span}{span}
\DeclareMathOperator{\Sgn}{sgn}
\DeclareMathOperator{\Area}{Area}
\DeclareMathOperator{\Volume}{Volume}

%
% A few miscellaneous things specific to this document
%
\newcommand{\crossop}[1]{\crossprod{#1}{}}

% R2 vector.
\newcommand{\VectorTwo}[2]{
\begin{bmatrix}
 {#1} \\
 {#2}
\end{bmatrix}
}

\newcommand{\VectorN}[1]{
\begin{bmatrix}
{#1}_1 \\
{#1}_2 \\
\vdots \\
{#1}_N \\
\end{bmatrix}
}

\newcommand{\DETuvij}[4]{
\begin{vmatrix}
 {#1}_{#3} & {#1}_{#4} \\
 {#2}_{#3} & {#2}_{#4}
\end{vmatrix}
}

\newcommand{\DETuvwijk}[6]{
\begin{vmatrix}
 {#1}_{#4} & {#1}_{#5} & {#1}_{#6} \\
 {#2}_{#4} & {#2}_{#5} & {#2}_{#6} \\
 {#3}_{#4} & {#3}_{#5} & {#3}_{#6}
\end{vmatrix}
}

\newcommand{\DETuvwxijkl}[8]{
\begin{vmatrix}
 {#1}_{#5} & {#1}_{#6} & {#1}_{#7} & {#1}_{#8} \\
 {#2}_{#5} & {#2}_{#6} & {#2}_{#7} & {#2}_{#8} \\
 {#3}_{#5} & {#3}_{#6} & {#3}_{#7} & {#3}_{#8} \\
 {#4}_{#5} & {#4}_{#6} & {#4}_{#7} & {#4}_{#8} \\
\end{vmatrix}
}

%\newcommand{\DETuvwxyijklm}[10]{
%\begin{vmatrix}
% {#1}_{#6} & {#1}_{#7} & {#1}_{#8} & {#1}_{#9} & {#1}_{#10} \\
% {#2}_{#6} & {#2}_{#7} & {#2}_{#8} & {#2}_{#9} & {#2}_{#10} \\
% {#3}_{#6} & {#3}_{#7} & {#3}_{#8} & {#3}_{#9} & {#3}_{#10} \\
% {#4}_{#6} & {#4}_{#7} & {#4}_{#8} & {#4}_{#9} & {#4}_{#10} \\
% {#5}_{#6} & {#5}_{#7} & {#5}_{#8} & {#5}_{#9} & {#5}_{#10}
%\end{vmatrix}
%}

% R3 vector.
\newcommand{\VectorThree}[3]{
\begin{bmatrix}
 {#1} \\
 {#2} \\
 {#3}
\end{bmatrix}
}



\author{Peeter Joot}
\email{peeter.joot@gmail.com}


\chapter{Problem Set 6.}
\label{chap:relElectroDynProblemSet6}
\blogpage{http://sites.google.com/site/peeterjoot/math2011/relElectroDynProblemSet6.pdf}
\date{Mar 25, 2011}
\revisionInfo{relElectroDynProblemSet6.tex}

\beginArtWithToc
%\beginArtNoToc

\section{Disclaimer.}

This problem set is as yet ungraded (although only the first question will be graded).

\section{Problem 1.  Energy-momentum tensor and electromagnetic forces.}
\subsection{Statement}

In class, we argued that in the absence of charges and currents, the energy-momentum tensor (or the ``stress-energy'' tensor) of the electromagnetic field

\begin{equation}\label{eqn:relativisticElectrodynamicsPS6P1:10}
T^{k m} = -\inv{4\pi} F^{k j} {F^{m}}_j + \inv{16 \pi} g^{k m} F^{i j} F_{i j},
\end{equation}

is conserved:

\begin{equation}\label{eqn:relativisticElectrodynamicsPS6P1:20}
\partial_k T^{k m} = 0.
\end{equation}

In this problem, you will study the fate of \ref{eqn:relativisticElectrodynamicsPS6P1:10}, the law of energy and momentum conservation in the presence of charged particles and currents given by a 4-vector current $j^l$.

\subsection{1. Conservation relation in the presence of sources.}

\subsubsection{Statement.}

Use the equations of motion in the presence of sources, $\partial_l F^{l k} = \frac{4 \pi}{c} j^m$, the fact that $F^{l k} = \partial^l A^m - \partial^m A^l$, and appropriate index gymnastics to show that \ref{eqn:relativisticElectrodynamicsPS6P1:20} is now replaced by 

\begin{equation}\label{eqn:relativisticElectrodynamicsPS6P1:30}
\partial_k T^{k m} = -\inv{c} F^{m l} j_l.
\end{equation}

\subsubsection{Solution.}

\subsection{2. Timelike component of the conservation relation.}

\subsubsection{Statement.}

Consider the $m = 0$ components of \ref{eqn:relativisticElectrodynamicsPS6P1:30}.  Show that it implies the energy conservation equation already discussed in class (see notes pp. 125-127):

\begin{equation}\label{eqn:relativisticElectrodynamicsPS6P1:40}
\PD{t}{\mathcal{E}} + \spacegrad \cdot \BS = - \BE \cdot \Bj.
\end{equation}

Recall the physical interpretation of the various terms in this equation.

\subsubsection{Solution.}

\subsection{3. Spacelike component of the conservation relation.}

\subsubsection{Statement.}

Consider the $m = \alpha$ components of \ref{eqn:relativisticElectrodynamicsPS6P1:30}.  Show that it implies that:

\begin{equation}\label{eqn:relativisticElectrodynamicsPS6P1:40}
\PD{t}{}\left( \frac{S^\alpha}{c^2} \right) + \PD{x^\beta}{} T^{\beta \alpha}
= - \left( \rho E^\alpha + \inv{c} \left( \Bj \cross \BB \right)^\alpha \right) \equiv - f^\alpha
\end{equation}

Give a physical interpretation of $f^\alpha$.

\subsubsection{Solution.}

\subsection{4. Integrated over a volume.}

\subsubsection{Statement.}

Integrate \ref{eqn:relativisticElectrodynamicsPS6P1:40} over a closed volume $V$ and use integration by parts to obtain

\begin{equation}\label{eqn:relativisticElectrodynamicsPS6P1:50}
\PD{t}{} \int_V d^3 \Bx \frac{S^\alpha}{c^2} 
= 
- \int_{\partial V = S} d \sigma^\beta T^{\beta \alpha} - \int_V d^3 \Bx f^\alpha 
\end{equation}

Give a physical interpretation of \ref{eqn:relativisticElectrodynamicsPS6P1:50} as expressing momentum conservation.  In particular, explain how, if the volume $V$ is that of a body (made of charged particles -- bound or otherwise), this implies that:

\begin{equation}\label{eqn:relativisticElectrodynamicsPS6P1:60}
\begin{aligned}
&\ddt{} \left( 
\Bp_{\text{EM field in $V$}} + 
\Bp_{\text{charged particles in $V$}} + 
\right)^\alpha \\
&\qquad = \int_{\text{surface of body}} \left( 
(\text{surface force})^\alpha \text{on body due to shears and pressures}
\right)
\end{aligned}
\end{equation}

(Note that here $d \sigma^\beta$ is an outward normal vector to the surface of the body, so the surface has a relative minus signs w.r.t the one from class, where an inward normal was used.)

\subsubsection{Solution.}

\subsection{5. Pressure and shear of linearly polarized EM wave.}

\subsubsection{Statement.}

Imagine that a place linearly polarized electromagnetic wave is falling on a flat surface at an angle of incidence $\alpha$, and is completely absorbeed by the body.  Find the pressure and shear on a unit area of the surface using the Maxwell stress tensor.

\subsubsection{Solution.}



\section{Problem 2.  Monochromatic stress energy tensor.}
\subsection{Statement}

Show that the energy momentum tensor of a plane monochromatic wave with 4-vector

\begin{equation}\label{eqn:relativisticElectrodynamicsPS6:10}
k^i = \left( \frac{\omega}{c}, \Bk \right),
\end{equation}

and energy density $\mathcal{E}$ can be written as

\begin{equation}\label{eqn:relativisticElectrodynamicsPS6:20}
T^{i j} = \frac{\mathcal{E} c^2}{\omega^2} k^i k^j.
\end{equation}

Can one conclude now that $\frac{\mathcal{E} c^2}{\omega^2}$ for a plane wave is a Lorentz scalar?

\subsection{Solution.  Determining the stress energy tensor.}

In the Coulomb gauge we used Fourier methods to find that the potential had the form

\begin{align}\label{eqn:relElectroDynProblemSet6:30}
\phi &= 0 \\
\BA &= \Bbeta \cos(\omega t - \Bk \cdot \Bx) \\
c^2 \Bk^2 &= \omega^2 \\
\Bbeta \cdot \Bk &= 0.
\end{align}

For this problem it appears that working in the Lorentz gauge is required, and we want solutions of the form

\begin{equation}\label{eqn:relElectroDynProblemSet6:50}
A^m = D^m \cos( k_a x^a ).
\end{equation}

First, observe that the Lorentz gauge condition $\partial_m A^m = 0$ requires

\begin{equation}\label{eqn:relElectroDynProblemSet6:70}
-D^m k_m \sin( k_a x^a ) = 0.
\end{equation}

Application of the wave equation operator

\begin{equation}\label{eqn:relElectroDynProblemSet6:90}
\partial_b \partial^b A^m = 0,
\end{equation}

gives us
\begin{equation}\label{eqn:relElectroDynProblemSet6:110}
-D^m k_b k^b \cos( k_a x^a ) = 0,
\end{equation}

providing the lightlike constraint on $k$.  All told our four potential with constraints is

\begin{align}\label{eqn:relElectroDynProblemSet6:130}
A^m &= D^m \cos( k_a x^a ) \\
k^a k_a &= 0 \\
D^m k_m &= 0.
\end{align}

We could also arrive at this point using 4D Fourier methods, which would be fun, but a bit more time consuming, and a little overkill given that the problem only requires us to tackle the linear monochromatic case.

On to the problem.  We now need our electromagnetic tensor components.

\begin{align*}
F^{ i j} 
&= \partial^i A^j - \partial^j A^i \\
&= 
D^j \partial^i \cos( k^a x_a ) 
-D^i \partial^j \cos( k^a x_a ) \\
&= 
\sin( k^a x_a ) ( D^i k^j - D^j k^i ) 
\end{align*}

Our stress energy tensor is 

\begin{align*}
T^{i j} 
&= \inv{4 \pi}
\left(
- F^{i a} F_{b a} g^{b j} + \inv{4} g^{i j} F_{ab} F^{ ab}
\right) \\
&= \inv{4 \pi}
\left(
- F^{a i} F_{a b} g^{b j} + \inv{4} g^{i j} F_{ab} F^{ ab}
\right)
\end{align*}

Let's now expand the product of tensors

\begin{align*}
F_{a b} F^{a i} 
&=
\sin^2( k^a x_a)
( D_a k_b - D_b k_a ) 
( D^a k^i - D^i k^a ) \\
&=
\sin^2( k^a x_a) (
D_a k_b D^a k^i 
-D_a k_b D^i k^a 
- D_b k_a D^a k^i 
+ D_b k_a D^i k^a ) \\
&=
\sin^2( k^a x_a) (
D_a D^a k_b k^i 
- \cancel{D_a k^a} k_b D^i 
- D_b \cancel{k_a D^a} k^i 
+ D_b D^i \cancel{k_a k^a} ) \\
&=
\sin^2( k^a x_a) D_a D^a k_b k^i 
\end{align*}

We see from this that our action term is zero

\begin{equation}\label{eqn:relElectroDynProblemSet6:150}
F_{a b} F^{a b} = \sin^2( k^a x_a) D_a D^a \cancel{k_b k^b},
\end{equation}

so the stress energy tensor is reduced to

\begin{align*}
T^{i j} 
&= -\inv{4 \pi} \sin^2( k^a x_a) D_a D^a k_b k^i g^{j b} \\
&= -\inv{4 \pi} \sin^2( k^a x_a) D_a D^a k^j k^i \\
\end{align*}

The energy density term of the stress energy tensor encapsulates most of these terms

\begin{equation}\label{eqn:relElectroDynProblemSet6:170}
T^{0 0} 
= -\inv{4 \pi} \sin^2( k^a x_a) D_a D^a \frac{\omega^2}{c^2} = \mathcal{E},
\end{equation}

so we can write

\begin{equation}\label{eqn:relElectroDynProblemSet6:190}
T^{i j} 
= \mathcal{E} \frac{c^2}{\omega^2} k^i k^j,
\end{equation}

which completes the first part of this problem.

\subsection{On the question of the Lorentz scalar.}

\paragraph{Q:} Can one conclude now that $\frac{\mathcal{E} c^2}{\omega^2}$ for a plane wave is a Lorentz scalar?
\paragraph{A:} Yes.

Observe that the $k^i k^j$ transforms as a rank 2 tensor, as does $T^{i j}$.  Because the product $\mathcal{E} c^2/\omega^2$ and $k^i k^j$ must transform as a rank 2 tensor, this can only mean that the $\mathcal{E} c^2/\omega^2$ portion transforms as a Lorentz scalar.

\section{Problem 3.  Force from an incoming wave.}
\subsection{Statement}

(Problem from the book.)  Find the force acting on a wall which reflects, with reflection coefficient $R$, and incoming electromagnetic wave; a general incidence angle is assumed, and is, of course, equal to the angle of reflection.

In this problem, if you decide use the stress tensor and look at the solution given in the text \cite{landau1980classical}, the stress tensor is argued to be  $T^{\alpha \beta} = T^{\alpha \beta}(\text{incoming wave}) + T^{\alpha \beta}(\text{reflected wave})$.  This actually holds only for the components of $T$ where one index, i.e., $\alpha$ is the direction perpendicular to the wall (i.e. for the components of the stress tensor relevant for calculating the pressure and shear). To be completely happy with the use of the stress tensor, you may want to derive this fact, starting from the expressions for the electric and magnetic field (3-vectors) $\BE = \BE_1 + \BE_2$, $\BB = \BB_1 + \BB_2$, where $(\BE_1, \BE_2)$ and $(\BB_1, \BB_2)$ correspond to the (incoming, reflected) wave, noticing that $\Abs{\BE_2} = \sqrt{R}\Abs{\BE_1}$, just like you did for Problem 1.5 (HW6).  Also, see Problem 3 of HW5.

\subsection{Solution}

\section{Appendix.}

\section{Three dimensional divergence theorem with generally parametrized volume element.}

With the divergence of the energy momentum tensor converted from a volume to a surface integral given by

\begin{equation}\label{eqn:relativisticElectrodynamicsPS6:340}
\int_V d^3 \Bx \partial_\beta T^{\beta \alpha} = \oint_{\partial V} d \sigma^\beta T^{\beta \alpha},
\end{equation}

I got to wondering what a closed form algebraic expression for this curious (and foreign seeming) quantity $d \sigma^\beta$ was.  It obviously must be related to the normal to the surface.  It seemed to me that a natural way to answer this question was to consider this divergence integral over an arbitrarily parametrized volume.  This turns out to be overkill, but a useful seeming digression.

\subsection{A generally parametrized parallelepiped volume element.}

Suppose we parametrize a volume by specifying that all the points in that volume are covered by the position vector from the origin, given by

\begin{equation}\label{eqn:relativisticElectrodynamicsPS6:360}
\Bx = \Bx(a_1, a_2, a_3).
\end{equation}

At any point in the volume of interest, we can create a level curve, holding two of the parameters $a_\alpha$ constant, and varying the remaining one.  In particular, we can construct three direction vectors along these level curves, one for each parameter not held constant

\begin{align}\label{eqn:relativisticElectrodynamicsPS6:380}
d\Bx_1 &= da_1 \PD{a_1}{\Bx} \\
d\Bx_2 &= da_2 \PD{a_2}{\Bx} \\
d\Bx_3 &= da_3 \PD{a_3}{\Bx}
\end{align}

The span of these vectors, provided they are non-degenerate, forms a parallelepiped, the volume of which is

\begin{equation}\label{eqn:relativisticElectrodynamicsPS6:400}
d^3\Bx = d\Bx_3 \cdot (d\Bx_1 \cross d\Bx_2).
\end{equation}

This volume element can be expanded in a number of ways

\begin{align*}
d^3\Bx 
&= \PD{a_1}{\Bx} \cdot \left( \PD{a_2}{\Bx} \cross \PD{a_3}{\Bx} \right) \\
&= 
\PD{a_1}{x^\alpha} 
\PD{a_2}{x^\beta} 
\PD{a_3}{x^\gamma} 
\epsilon_{\alpha \beta \gamma} 
da_1 da_2 da_3 \\
&= 
\PD{a_\alpha}{x^1} 
\PD{a_\beta}{x^2} 
\PD{a_\gamma}{x^3} 
\epsilon_{\alpha \beta \gamma} 
da_1 da_2 da_3 \\
&= 
\PD{a_{[1}}{x^1} 
\PD{a_{2}}{x^2} 
\PD{a_{3]}}{x^3} 
da_1 da_2 da_2 \\
&= 
\Abs{ \frac{\partial(x^1, x^2, x^3)}{\partial (a_1, a_2, a_3)}}
da_1 da_2 da_3 \\
\end{align*}

where the Jacobian determinant is given by

\begin{equation}\label{eqn:relativisticElectrodynamicsPS6:420}
\Abs{ \frac{\partial(x^1, x^2, x^3)}{\partial (a_1, a_2, a_3)}}
= 
\begin{vmatrix}
 \PD{a_1}{x^1} & \PD{a_1}{x^2} & \PD{a_1}{x^3} \\
 \PD{a_2}{x^1} & \PD{a_2}{x^2} & \PD{a_2}{x^3} \\
 \PD{a_3}{x^1} & \PD{a_3}{x^2} & \PD{a_3}{x^3}
\end{vmatrix}.
\end{equation}

Provided we are interested in a volume for which the sign of this Jacobian determinant does not change sign, our task is to evaluate and reduce the integral 

\begin{equation}\label{eqn:relativisticElectrodynamicsPS6:620}
\int 
\Abs{ \frac{\partial(x^1, x^2, x^3)}{\partial (a_1, a_2, a_3)}}
da_1 da_2 da_3 
\PD{x^\beta}{T^{\beta \alpha}}
\end{equation}

to a set (and sum of) two dimensional integrals.

\subsection{On the geometry of the surfaces.}

Suppose that we integrate over the ranges $[a_{1-}, a_{1+}]$, $[a_{2-}, a_{2+}]$, $[a_{3-}, a_{3+}]$.  Observe that the outwards normals along the $a_1 = a_1+$ face is $d\Bn_{1+} = da_2 da_3 \PDi{a_2}{\Bx} \cross \PDi{a_3}{\Bx}$.  This is

\begin{equation}\label{eqn:relativisticElectrodynamicsPS6:440}
d\Bn_{1+} 
= da_2 da_3 \PD{a_2}{\Bx} \cross \PD{a_3}{\Bx}
= da_2 da_3 \PD{a_2}{x^\mu} \PD{a_3}{x^\nu} \epsilon_{\mu \nu \gamma} \Be_\gamma
\end{equation}

Similarly our normal on the $a_2 = a_{2+}$ face is

\begin{equation}\label{eqn:relativisticElectrodynamicsPS6:460}
d\Bn_{2+} 
= da_3 da_1 \PD{a_3}{\Bx} \cross \PD{a_1}{\Bx}
= da_3 da_1 \PD{a_3}{x^\mu} \PD{a_1}{x^\nu} \epsilon_{\mu \nu \gamma} \Be_\gamma,
\end{equation}

and on the $a_3 = a_{3+}$ face the outward normal is

\begin{equation}\label{eqn:relativisticElectrodynamicsPS6:480}
d\Bn_{3+} 
= da_1 da_2 \PD{a_1}{\Bx} \cross \PD{a_2}{\Bx}
= da_1 da_2 \PD{a_1}{x^\mu} \PD{a_2}{x^\nu} \epsilon_{\mu \nu \gamma} \Be_\gamma.
\end{equation}

Along the $a_{\alpha-}$ faces these are just negated.  We can summarize these as

\begin{equation}\label{eqn:relativisticElectrodynamicsPS6:500}
d\Bn_{\sigma\pm} 
= \pm \inv{2!} da_\alpha da_\beta \PD{a_\alpha}{\Bx} \cross \PD{a_\beta}{\Bx} \epsilon_{\alpha \beta \sigma}
= \pm \inv{2!} da_\alpha da_\beta \PD{a_\alpha}{x^\mu} \PD{a_\beta}{x^\nu} \epsilon_{\alpha \beta \sigma} \epsilon_{\mu \nu \gamma} \Be_\gamma 
\end{equation}

\subsection{Expansion of the Jacobian determinant}

Suppose, to start with, our divergence volume integral \ref{eqn:relativisticElectrodynamicsPS6:620} has just the following term

\begin{equation}\label{eqn:relativisticElectrodynamicsPS6:520}
\int d^3 \Bx \partial_3 M.
\end{equation}

The specifics of how the scalar $M = T^{3 \alpha}$ is indexed will not matter yet, so let's suppress it.  The Jacobian determinant can be expanded along the $\PD{a_\alpha}{x^3}$ column for

\begin{align*}
\int d^3 \Bx \partial_3 M
&=
\int da_1 da_2 da_3
\Abs{ \frac{\partial(x^1, x^2, x^3)}{\partial (a_1, a_2, a_3)}} 
\PD{x^3}{M} \\
&=
\int da_1 da_2 da_3
\left(
\PD{a_{[1}}{x^1} 
\PD{a_{2}}{x^2} 
\PD{a_{3]}}{x^3} 
\right)
\PD{x^3}{M} \\
&=
\int da_1 da_2 da_3
\left(
\PD{a_{[1}}{x^1} 
\PD{a_{2]}}{x^2} 
\PD{a_{3}}{x^3} 
+
\PD{a_{[2}}{x^1} 
\PD{a_{3]}}{x^2} 
\PD{a_{1}}{x^3} 
+
\PD{a_{[3}}{x^1} 
\PD{a_{1]}}{x^2} 
\PD{a_{2}}{x^3} 
\right)
\PD{x^3}{M} \\
&=
\int da_1 da_2 da_3
\left(
\Abs{ \frac{\partial(x^1, x^2)}{\partial (a_1, a_2)}} 
\PD{a_{3}}{x^3} 
+
\Abs{ \frac{\partial(x^1, x^2)}{\partial (a_2, a_3)}} 
\PD{a_{1}}{x^3} 
+
\Abs{ \frac{\partial(x^1, x^2)}{\partial (a_3, a_1)}} 
\PD{a_{2}}{x^3} 
\right)
\PD{x^3}{M} \\
&=
\int da_1 da_2 \Abs{ \frac{\partial(x^1, x^2)}{\partial (a_1, a_2)}} 
\int da_3 \PD{a_{3}}{x^3} \PD{x^3}{M}  \\
&\qquad +
\int da_2 da_3 \Abs{ \frac{\partial(x^1, x^2)}{\partial (a_2, a_3)}} 
\int da_1 \PD{a_{1}}{x^3} \PD{x^3}{M}  \\
&\qquad +
\int da_3 da_1 \Abs{ \frac{\partial(x^1, x^2)}{\partial (a_3, a_1)}} 
\int da_2 \PD{a_{2}}{x^3} \PD{x^3}{M}  \\
&=
\int da_1 da_2 \Abs{ \frac{\partial(x^1, x^2)}{\partial (a_1, a_2)}} 
\int da_3 \PD{a_{3}}{M}  \\
&\qquad +
\int da_2 da_3 \Abs{ \frac{\partial(x^1, x^2)}{\partial (a_2, a_3)}} 
\int da_1 \PD{a_{1}}{M}  \\
&\qquad +
\int da_3 da_1 \Abs{ \frac{\partial(x^1, x^2)}{\partial (a_3, a_1)}} 
\int da_2 \PD{a_{2}}{M}  \\
&=
\int da_1 da_2 \Abs{ \frac{\partial(x^1, x^2)}{\partial (a_1, a_2)}} 
\Bigl( M(a_{3+}) - M(a_{3+}) \Bigr) \\
&\qquad +
\int da_2 da_3 \Abs{ \frac{\partial(x^1, x^2)}{\partial (a_2, a_3)}} 
\Bigl( M(a_{1+}) - M(a_{1+}) \Bigr) \\
&\qquad +
\int da_3 da_1 \Abs{ \frac{\partial(x^1, x^2)}{\partial (a_3, a_1)}} 
\Bigl( M(a_{2+}) - M(a_{2+}) \Bigr)
\end{align*}

Performing the same task (really just performing cyclic permutation of indexes) we can now construct the whole divergence integral

\begin{align*}
\int d^3 \Bx \partial_\beta T^{\beta \alpha}
&=
\int da_1 da_2 \Abs{ \frac{\partial(x^1, x^2)}{\partial (a_1, a_2)}} 
\Bigl( T^{3 \alpha}(a_{3+}) - T^{3 \alpha}(a_{3+}) \Bigr) \\
&\qquad +
\int da_2 da_3 \Abs{ \frac{\partial(x^1, x^2)}{\partial (a_2, a_3)}} 
\Bigl( T^{3 \alpha}(a_{1+}) - T^{3 \alpha}(a_{1+}) \Bigr) \\
&\qquad +
\int da_3 da_1 \Abs{ \frac{\partial(x^1, x^2)}{\partial (a_3, a_1)}} 
\Bigl( T^{3 \alpha}(a_{2+}) - T^{3 \alpha}(a_{2+}) \Bigr) \\
&+\int da_1 da_2 \Abs{ \frac{\partial(x^2, x^3)}{\partial (a_1, a_2)}} 
\Bigl( T^{1 \alpha}(a_{3+}) - T^{1 \alpha}(a_{3+}) \Bigr) \\
&\qquad +
\int da_2 da_3 \Abs{ \frac{\partial(x^2, x^3)}{\partial (a_2, a_3)}} 
\Bigl( T^{1 \alpha}(a_{1+}) - T^{1 \alpha}(a_{1+}) \Bigr) \\
&\qquad +
\int da_3 da_1 \Abs{ \frac{\partial(x^2, x^3)}{\partial (a_3, a_1)}} 
\Bigl( T^{1 \alpha}(a_{2+}) - T^{1 \alpha}(a_{2+}) \Bigr) \\
&+\int da_1 da_2 \Abs{ \frac{\partial(x^3, x^1)}{\partial (a_1, a_2)}} 
\Bigl( T^{2 \alpha}(a_{3+}) - T^{2 \alpha}(a_{3+}) \Bigr) \\
&\qquad +
\int da_2 da_3 \Abs{ \frac{\partial(x^3, x^1)}{\partial (a_2, a_3)}} 
\Bigl( T^{2 \alpha}(a_{1+}) - T^{2 \alpha}(a_{1+}) \Bigr) \\
&\qquad +
\int da_3 da_1 \Abs{ \frac{\partial(x^3, x^1)}{\partial (a_3, a_1)}} 
\Bigl( T^{2 \alpha}(a_{2+}) - T^{2 \alpha}(a_{2+}) \Bigr).
\end{align*}

Regrouping we have

\begin{align*}
\int d^3 \Bx \partial_\beta T^{\beta \alpha}
&=
\int da_1 da_2 \left( 
\Abs{ \frac{\partial(x^1, x^2)}{\partial (a_1, a_2)}} 
\evalbar{T^{3 \alpha}}{\Delta a_3}
+\Abs{ \frac{\partial(x^2, x^3)}{\partial (a_1, a_2)}} 
\evalbar{T^{1 \alpha}}{\Delta a_3}
+\Abs{ \frac{\partial(x^3, x^1)}{\partial (a_1, a_2)}} 
\evalbar{T^{2 \alpha}}{\Delta a_3}
\right) \\
&+
\int da_2 da_3 \left( 
\Abs{ \frac{\partial(x^1, x^2)}{\partial (a_2, a_3)}} 
\evalbar{T^{3 \alpha}}{\Delta a_3}
+\Abs{ \frac{\partial(x^2, x^3)}{\partial (a_2, a_3)}} 
\evalbar{T^{1 \alpha}}{\Delta a_3}
+\Abs{ \frac{\partial(x^3, x^1)}{\partial (a_2, a_3)}} 
\evalbar{T^{2 \alpha}}{\Delta a_3}
\right) \\
&+
\int da_3 da_1 \left( 
\Abs{ \frac{\partial(x^1, x^2)}{\partial (a_3, a_1)}} 
\evalbar{T^{3 \alpha}}{\Delta a_3}
+\Abs{ \frac{\partial(x^2, x^3)}{\partial (a_3, a_1)}} 
\evalbar{T^{1 \alpha}}{\Delta a_3}
+\Abs{ \frac{\partial(x^3, x^1)}{\partial (a_3, a_1)}} 
\evalbar{T^{2 \alpha}}{\Delta a_3}
\right).
\end{align*}

Observe that we can factor these sums utilizing the normals for the parallelepiped volume element

\begin{align*}
\int d^3 \Bx \partial_\beta T^{\beta \alpha}
&=
\int da_1 da_2 
\Abs{ \frac{\partial(x^\mu, x^\nu)}{\partial (a_1, a_2)}} \epsilon_{\mu \nu \gamma} \Be_\gamma \cdot \Be_\beta
\evalbar{T^{\beta \alpha}}{\Delta a_3} \\
&+
\int da_2 da_3 
\Abs{ \frac{\partial(x^\mu, x^\nu)}{\partial (a_2, a_3)}} \epsilon_{\mu \nu \gamma} \Be_\gamma \cdot \Be_\beta
\evalbar{T^{\beta \alpha}}{\Delta a_1} \\
&+
\int da_3 da_1 
\Abs{ \frac{\partial(x^\mu, x^\nu)}{\partial (a_3, a_1)}} \epsilon_{\mu \nu \gamma} \Be_\gamma \cdot \Be_\beta
\evalbar{T^{\beta \alpha}}{\Delta a_2}
\end{align*}

Let's look at the first of these integrals in more detail.  We integrate the values of the $\Be_\beta T^{\beta \alpha}$ evaluated on the points of the surface for which $a_3 = a_{3+}$.  To perform this integral we dot against the outward normal area element $da_1 da_2 \PDi{a_1}{x^\mu} \PDi{a_2}{x^\nu} \epsilon_{\mu\nu\gamma} \Be_\gamma$.  We do the same, but subtract the integral where $\Be_\beta T^{\beta\alpha}$ is evaluated on the surface $a_3 = a_{3-}$, where we dot with the area element that has the inwards normal direction on that surface.  This is then done for each of the surfaces of the parallelepiped that we are integrating over.

In terms of the outwards (area scaled) normals $d\Bn_3, d\Bn_1, d\Bn_2$ on the $a_{3+}, a_{1+}$ and $a_{2+}$ surfaces respectively we can write

\begin{equation}\label{eqn:relativisticElectrodynamicsPS6:540}
\int d^3 \Bx \partial_\beta T^{\beta \alpha} = 
\int d\Bn_3 \cdot \Be_\beta \evalbar{T^{\beta}{\alpha}}{\Delta a_3}
+\int d\Bn_1 \cdot \Be_\beta \evalbar{T^{\beta}{\alpha}}{\Delta a_1}
+\int d\Bn_2 \cdot \Be_\beta \evalbar{T^{\beta}{\alpha}}{\Delta a_2}.
\end{equation}

This can be written more concisely in index form with

\begin{equation}\label{eqn:relativisticElectrodynamicsPS6:640}
d \sigma^\beta = 
\epsilon_{\mu\nu\beta} \left(
\PD{a_2}{x^\mu}
\PD{a_3}{x^\nu} da_2 da_3
+\PD{a_3}{x^\mu}
\PD{a_1}{x^\nu} da_3 da_1
+\PD{a_1}{x^\mu}
\PD{a_2}{x^\nu} da_1 da_2
\right),
\end{equation}

so that the divergence integral is just

\begin{equation}\label{eqn:relativisticElectrodynamicsPS6:560}
\int d^3 \Bx = 
\int_{\text{over level surfaces $a_{1+}$, $a_{2+}$, $a_{3+}$}} d \sigma^\beta T^{\beta \alpha}
-\int_{\text{over level surfaces $a_{1-}$, $a_{2-}$, $a_{3-}$}} d \sigma^\beta T^{\beta \alpha}
\end{equation}

In each case, for the $a_{\alpha-}$ surfaces, our negated inwards normal form can be redefined so that we integrate over only the outwards normal directions, and we can use the oriented integral notation

\begin{equation}\label{eqn:relativisticElectrodynamicsPS6:580}
\int d^3 \Bx = \oint d \sigma^\beta T^{\beta \alpha},
\end{equation}

To encode (or imply) whether we require a positive or negative sign on the area element tensor of \ref{eqn:relativisticElectrodynamicsPS6:640} for the surface in question.

\subsection{A look back, and looking forward.}

Now, having performed this long winded calculation, the meaning of $d \sigma^\beta$ becomes clear.  What's also clear is how this could have been arrived at directly utilizing the divergence theorem in its normal vector form.  We had only to re-write our equation as a vector equation in terms of the gradient

\begin{equation}\label{eqn:relativisticElectrodynamicsPS6:600}
\int_V d^3 \Bx \PD{x^\alpha}{T^{\beta \alpha}} = \int_V d^3 \Bx \spacegrad \cdot (\Be_\beta T^{\beta \alpha}) = \int_{\partial_V} dA \Bn \cdot \Be_\beta T^{\beta \alpha}
\end{equation}

From this we see directly that $d\sigma^\beta = dA \Bn \cdot \Be_\beta$.

Despite there being an easier way to find the form of $d\sigma^\beta$, I still consider this a worthwhile exercise.  It hints how one could generalize the arguments to the higher dimensional cases.  The main task would be to construct the normals to the hypersurfaces bounding the hypervolume, and how to do this algebraically utilizing determinants may not be too hard (since we want a Jacobian determinant as the hypervolume element in the ``volume'' integral).  We also got more than the normal physics text book proof of the divergence theorem for Cartesian coordinates, and did it here for a general parametrization.  This wasn't a complete argument since we didn't consider a general surface, broken down into a triangular mesh.  We really want volume elements with triangular sides instead of parallelograms.

\EndArticle
%\EndNoBibArticle
