%
% Copyright � 2013 Peeter Joot.  All Rights Reserved.
% Licenced as described in the file LICENSE under the root directory of this GIT repository.
%
\newcommand{\authorname}{Peeter Joot}
\newcommand{\email}{peeterjoot@protonmail.com}
\newcommand{\basename}{FIXMEbasenameUndefined}
\newcommand{\dirname}{notes/FIXMEdirnameUndefined/}

\renewcommand{\basename}{basicStatMechLecture16}
\renewcommand{\dirname}{notes/phy452/}
\newcommand{\keywords}{Statistical mechanics, PHY452H1S}
\newcommand{\authorname}{Peeter Joot}
\newcommand{\onlineurl}{http://sites.google.com/site/peeterjoot2/math2013/\basename.pdf}
\newcommand{\sourcepath}{\dirname\basename.tex}
\newcommand{\generatetitle}[1]{\chapter{#1}}

\newcommand{\vcsinfo}{%
\section*{}
\noindent{\color{DarkOliveGreen}{\rule{\linewidth}{0.1mm}}}
\paragraph{Document version}
%\paragraph{\color{Maroon}{Document version}}
{
\small
\begin{itemize}
\item Available online at:\\ 
\href{\onlineurl}{\onlineurl}
\item Git Repository: \input{./.revinfo/gitRepo.tex}
\item Source: \sourcepath
\item last commit: \input{./.revinfo/gitCommitString.tex}
\item commit date: \input{./.revinfo/gitCommitDate.tex}
\end{itemize}
}
}

%\PassOptionsToPackage{dvipsnames,svgnames}{xcolor}
\PassOptionsToPackage{square,numbers}{natbib}
\documentclass{scrreprt}

\usepackage[left=2cm,right=2cm]{geometry}
\usepackage[svgnames]{xcolor}
\usepackage{peeters_layout}

\usepackage{natbib}

\usepackage[
colorlinks=true,
bookmarks=false,
pdfauthor={\authorname, \email},
backref 
]{hyperref}

% http://tex.stackexchange.com/questions/75773/how-to-reference-problems-by-the-text-label-in-an-exercise-envioronment
\usepackage[english]{cleveref}
\crefname{Exercise}{exercise}{exercises}
\Crefname{Exercise}{Exercise}{Exercises}

\RequirePackage{titlesec}
\RequirePackage{ifthen}

% http://stackoverflow.com/questions/4932910/date-in-the-tabular-environment
\makeatletter
\let\insertdate\@date
\makeatother

\titleformat{\chapter}[display]
{\bfseries\Large}
{\color{DarkSlateGrey}\filleft \authorname
\ifthenelse{\isundefined{\studentnumber}}{}{\\ \studentnumber}
\ifthenelse{\isundefined{\email}}{}{\\ \email}
\ifthenelse{\isundefined{\dateintitle}}{}{\\ \insertdate}
%\ifthenelse{\isundefined{\coursename}}{}{\\ \coursename} % put in title instead.
}
{4ex}
{\color{DarkOliveGreen}{\titlerule}\color{Maroon}
\vspace{2ex}%
\filright}
[\vspace{2ex}%
\color{DarkOliveGreen}\titlerule
]

\newcommand{\beginArtWithToc}[0]{\begin{document}\tableofcontents}
\newcommand{\beginArtNoToc}[0]{\begin{document}}
\newcommand{\EndNoBibArticle}[0]{\end{document}}
\newcommand{\EndArticle}[0]{\bibliography{Bibliography}\bibliographystyle{plainnat}\end{document}}

% 
%\newcommand{\citep}[1]{\cite{#1}}

\colorSectionsForArticle



\beginArtNoToc
\generatetitle{PHY452H1S Basic Statistical Mechanics.  Lecture 16: Fermi gas.  Taught by Prof.\ Arun Paramekanti}
%\chapter{Fermi gas}
\label{chap:basicStatMechLecture16}

\section{Disclaimer}

Peeter's lecture notes from class.  May not be entirely coherent.

\section{Fermi gas}

\paragraph{Review}

Continuing a discussion of \citep{pathriastatistical} \S 8.1 content.

We found

\begin{dmath}\label{eqn:basicStatMechLecture16:20}
n_{\Bk} = \inv{e^{\beta(\epsilon_k - \mu)} + 1}
\end{dmath}

With no spin

\begin{dmath}\label{eqn:basicStatMechLecture16:40}
\int n_\Bk \times \frac{d^3 k}{(2\pi)^3} = \rho
\end{dmath}

F1

\begin{dmath}\label{eqn:basicStatMechLecture16:480}
\epsilon_{\mathrm{F}} = \frac{\hbar^2 k_{\mathrm{F}}^2}{2m}
\end{dmath}

gives

\begin{dmath}\label{eqn:basicStatMechLecture16:60}
k_{\mathrm{F}} = (6 \pi \rho)^{1/3}
\end{dmath}

\begin{dmath}\label{eqn:basicStatMechLecture16:80}
\sum_\Bk n_\Bk = N
\end{dmath}

\begin{dmath}\label{eqn:basicStatMechLecture16:100}
\Bk = \frac{2\pi}{L}(n_x, n_y, n_z)
\end{dmath}

This is for periodic boundary conditions (not particle in a box), where

\begin{dmath}\label{eqn:basicStatMechLecture16:120}
\Psi(x + L) = \Psi(x)
\end{dmath}

\paragraph{Moving on}

\begin{dmath}\label{eqn:basicStatMechLecture16:140}
\sum_{k_x} n(\Bk) = \sum_{p_x} \Delta p_x n(\Bk)
\end{dmath}

with
\begin{dmath}\label{eqn:basicStatMechLecture16:160}
\Delta k_x = \frac{2 \pi}{L} \Delta p_x
\end{dmath}

this gives

\begin{dmath}\label{eqn:basicStatMechLecture16:180}
\sum_{k_x} n(\Bk) 
= \sum_{n_x} \frac{L}{2\pi} \Delta k_x 
\rightarrow 
\frac{L}{2\pi} \int d k_x
\end{dmath}

Over all dimensions

\begin{dmath}\label{eqn:basicStatMechLecture16:200}
\sum_{\Bk} n_\Bk = 
\lr{\frac{L}{2\pi}}^3 
\lr{ \int d^3 \Bk}
n(\Bk)
=
N
\end{dmath}

so that

\begin{dmath}\label{eqn:basicStatMechLecture16:220}
\rho = \int \frac{d^3 \Bk}{(2 \pi)^3}
\end{dmath}

Again 

\begin{dmath}\label{eqn:basicStatMechLecture16:240}
k_{\mathrm{F}} = (6 \pi \rho)^{1/3}
\end{dmath}

\paragraph{Spin considerations}

%FIXME:
%\begin{itemize}
%\item 
%S = \inv{2} \rightarrow \mbox{$2$ states at each $\Bk$}
%\item 
%Spin $S \rightarrow \mbox{($2 S + 1$) states at each $\Bk$, where $m_s = -S, -S + 1, \cdots, S$.}
%\end{itemize}

\begin{dmath}\label{eqn:basicStatMechLecture16:260}
\sum_{\Bk, m_s} = N
\end{dmath}
\begin{dmath}\label{eqn:basicStatMechLecture16:280}
\sum_{\Bk, m_s} \inv{e^{\beta(\epsilon_k - \mu)} + 1} 
= (2 S + 1)
\lr{ \int \frac{d^3 \Bk}{(2 \pi)^3} n(\Bk) } L^3
\end{dmath}

This gives us

\begin{dmath}\label{eqn:basicStatMechLecture16:300}
k_{\mathrm{F}} = 
\lr{\frac{ 6 \pi \rho }{2 S + 1}}^{1/3}
\end{dmath}

and again

\begin{dmath}\label{eqn:basicStatMechLecture16:320}
\epsilon_{\mathrm{F}} = \frac{\hbar^2 \kF^2}{2m}
\end{dmath}

\paragraph{High Temperatures}

F2

\begin{dmath}\label{eqn:basicStatMechLecture16:340}
\mu(T = 0) = \epsilon_{\mathrm{F}}
\end{dmath}
\begin{dmath}\label{eqn:basicStatMechLecture16:360}
\mu(T \rightarrow \infty) \rightarrow - \infty
\end{dmath}

so that for large $T$ we haave

\begin{dmath}\label{eqn:basicStatMechLecture16:380}
\inv{e^{\beta(\epsilon_k - \mu)} + 1} \rightarrow e^{-\beta(\epsilon_k - \mu)}
\end{dmath}

\begin{dmath}\label{eqn:basicStatMechLecture16:500}
\rho 
= \int \frac{d^3 \Bk}{(2 \pi)^3} e^{\beta \mu} e^{-\beta \epsilon_k} 
= 
e^{\beta \mu} 
\int \frac{d^3 \Bk}{(2 \pi)^3} 
e^{-\beta \epsilon_k} 
\equiv
e^{\beta \mu}  \inv{\lambda^3}
\end{dmath}

where

\begin{dmath}\label{eqn:basicStatMechLecture16:400}
\lambda = \frac{h}{\sqrt{2 \pi m \kB T}}
\end{dmath}

where $\lambda$ is the \underlineAndIndex{thermal de Broglie wavelength}, $\lambda^3 \sim T^{-3/2}$.

We can write

\begin{dmath}\label{eqn:basicStatMechLecture16:420}
\myBoxed{
e^{\beta \mu} = \lr{\rho \lambda^3}
}
\end{dmath}

\begin{dmath}\label{eqn:basicStatMechLecture16:440}
\frac{\mu}{\kB T} = \ln \lr{ \rho \lambda^3 } \sim -\frac{3}{2} \ln T
\end{dmath}

or

\begin{dmath}\label{eqn:basicStatMechLecture16:460}
\mu \propto -T \ln T
\end{dmath}

Plotting this

F3

\paragraph{Pressure}

\begin{dmath}\label{eqn:basicStatMechLecture16:520}
P = - \PD{V}{E}
\end{dmath}

For a classical ideal gas as in 

F4

we have

\begin{dmath}\label{eqn:basicStatMechLecture16:540}
P = \rho \kB T
\end{dmath}

For a Fermi gas at $T = 0$ we have

\begin{dmath}\label{eqn:basicStatMechLecture16:560}
E 
= \sum_\Bk \epsilon_k n_k
= \frac{V}{(2\pi)^3} \int_0^{\kF} \frac{\hbar^2 \Bk^2}{2 m} d^3 \Bk
= \frac{V}{(2\pi)^3} 
\frac{\hbar^2}{2 m} 
\int_0^{\kF} k^2 4 \pi k^2 d k
\propto \kF^5
\end{dmath}

Specifically, 

\begin{dmath}\label{eqn:basicStatMechLecture16:580}
E(T = 0) = V \times 
%\lr{ 
\frac{3}{5} 
\mathLabelBox{
\epsilon_{\mathrm{F}}
}{$\sim \kF^2$}
\mathLabelBox
[
   labelstyle={below of=m\themathLableNode, below of=m\themathLableNode}
]
{
\rho
}{$\sim \kF^3$}
%}
\end{dmath}

or

\begin{dmath}\label{eqn:basicStatMechLecture16:600}
\frac{E}{N} = \frac{3}{5} \epsilon_{\mathrm{F}}
\end{dmath}

\begin{dmath}\label{eqn:basicStatMechLecture16:620}
E 
= \frac{3}{5} N \frac{\hbar^2}{2 m} \lr{ 6 \pi^2 \frac{N}{V} }^{2/3}.
= a V^{-2/3},
\end{dmath}

so that

\begin{dmath}\label{eqn:basicStatMechLecture16:640}
\PD{V}{E} = -\frac{2}{3} a V^{-5/3}.
\end{dmath}

\begin{dmath}\label{eqn:basicStatMechLecture16:660}
P 
= -\PD{V}{E} 
= \frac{2}{3} \lr{a V^{-2/3} } V^{-1}
= \frac{2}{3} \frac{E}{V}
= 
\frac{2}{3} \lr{
\frac{3}{5} \epsilon_{\mathrm{F}} \rho 
}
=
\frac{2}{5} 
\epsilon_{\mathrm{F}} \rho.
\end{dmath}

We see that the pressure ends up deviating from the classical result at low temperatures.  This low temperature limit for the pressure $2 \epsilon_{\mathrm{F}} \rho/5$ is called the \underlineAndIndex{degeneracy pressure}.

\EndArticle
