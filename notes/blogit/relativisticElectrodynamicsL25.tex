%
% Copyright � 2015 Peeter Joot.  All Rights Reserved.
% Licenced as described in the file LICENSE under the root directory of this GIT repository.
%
\documentclass[]{eliblog}

\usepackage{amsmath}
\usepackage{mathpazo}

%
% shorthand for bold symbols, convenient for vectors and matrices
%
\newcommand{\Ba}[0]{\mathbf{a}}
\newcommand{\Bb}[0]{\mathbf{b}}
\newcommand{\Bc}[0]{\mathbf{c}}
\newcommand{\Bd}[0]{\mathbf{d}}
\newcommand{\Be}[0]{\mathbf{e}}
\newcommand{\Bf}[0]{\mathbf{f}}
\newcommand{\Bg}[0]{\mathbf{g}}
\newcommand{\Bh}[0]{\mathbf{h}}
\newcommand{\Bi}[0]{\mathbf{i}}
\newcommand{\Bj}[0]{\mathbf{j}}
\newcommand{\Bk}[0]{\mathbf{k}}
\newcommand{\Bl}[0]{\mathbf{l}}
\newcommand{\Bm}[0]{\mathbf{m}}
\newcommand{\Bn}[0]{\mathbf{n}}
\newcommand{\Bo}[0]{\mathbf{o}}
\newcommand{\Bp}[0]{\mathbf{p}}
\newcommand{\Bq}[0]{\mathbf{q}}
\newcommand{\Br}[0]{\mathbf{r}}
\newcommand{\Bs}[0]{\mathbf{s}}
\newcommand{\Bt}[0]{\mathbf{t}}
\newcommand{\Bu}[0]{\mathbf{u}}
\newcommand{\Bv}[0]{\mathbf{v}}
\newcommand{\Bw}[0]{\mathbf{w}}
\newcommand{\Bx}[0]{\mathbf{x}}
\newcommand{\By}[0]{\mathbf{y}}
\newcommand{\Bz}[0]{\mathbf{z}}
\newcommand{\BA}[0]{\mathbf{A}}
\newcommand{\BB}[0]{\mathbf{B}}
\newcommand{\BC}[0]{\mathbf{C}}
\newcommand{\BD}[0]{\mathbf{D}}
\newcommand{\BE}[0]{\mathbf{E}}
\newcommand{\BF}[0]{\mathbf{F}}
\newcommand{\BG}[0]{\mathbf{G}}
\newcommand{\BH}[0]{\mathbf{H}}
\newcommand{\BI}[0]{\mathbf{I}}
\newcommand{\BJ}[0]{\mathbf{J}}
\newcommand{\BK}[0]{\mathbf{K}}
\newcommand{\BL}[0]{\mathbf{L}}
\newcommand{\BM}[0]{\mathbf{M}}
\newcommand{\BN}[0]{\mathbf{N}}
\newcommand{\BO}[0]{\mathbf{O}}
\newcommand{\BP}[0]{\mathbf{P}}
\newcommand{\BQ}[0]{\mathbf{Q}}
\newcommand{\BR}[0]{\mathbf{R}}
\newcommand{\BS}[0]{\mathbf{S}}
\newcommand{\BT}[0]{\mathbf{T}}
\newcommand{\BU}[0]{\mathbf{U}}
\newcommand{\BV}[0]{\mathbf{V}}
\newcommand{\BW}[0]{\mathbf{W}}
\newcommand{\BX}[0]{\mathbf{X}}
\newcommand{\BY}[0]{\mathbf{Y}}
\newcommand{\BZ}[0]{\mathbf{Z}}

\newcommand{\Bzero}[0]{\mathbf{0}}
\newcommand{\Btheta}[0]{\boldsymbol{\theta}}
\newcommand{\Btau}[0]{\boldsymbol{\tau}}
\newcommand{\Bomega}[0]{\boldsymbol{\omega}}

%
% shorthand for unit vectors
%
\newcommand{\acap}[0]{\hat{\Ba}}
\newcommand{\bcap}[0]{\hat{\Bb}}
\newcommand{\ccap}[0]{\hat{\Bc}}
\newcommand{\dcap}[0]{\hat{\Bd}}
\newcommand{\ecap}[0]{\hat{\Be}}
\newcommand{\fcap}[0]{\hat{\Bf}}
\newcommand{\gcap}[0]{\hat{\Bg}}
\newcommand{\hcap}[0]{\hat{\Bh}}
\newcommand{\icap}[0]{\hat{\Bi}}
\newcommand{\jcap}[0]{\hat{\Bj}}
\newcommand{\kcap}[0]{\hat{\Bk}}
\newcommand{\lcap}[0]{\hat{\Bl}}
\newcommand{\mcap}[0]{\hat{\Bm}}
\newcommand{\ncap}[0]{\hat{\Bn}}
\newcommand{\ocap}[0]{\hat{\Bo}}
\newcommand{\pcap}[0]{\hat{\Bp}}
\newcommand{\qcap}[0]{\hat{\Bq}}
\newcommand{\rcap}[0]{\hat{\Br}}
\newcommand{\scap}[0]{\hat{\Bs}}
\newcommand{\tcap}[0]{\hat{\Bt}}
\newcommand{\ucap}[0]{\hat{\Bu}}
\newcommand{\vcap}[0]{\hat{\Bv}}
\newcommand{\wcap}[0]{\hat{\Bw}}
\newcommand{\xcap}[0]{\hat{\Bx}}
\newcommand{\ycap}[0]{\hat{\By}}
\newcommand{\zcap}[0]{\hat{\Bz}}
\newcommand{\thetacap}[0]{\hat{\Btheta}}

%
% to write R^n and C^n in a distinguishable fashion.  Perhaps change this
% to the double lined characters upon figuring out how to do so.
%
\newcommand{\C}[1]{$\mathbb{C}^{#1}$}
\newcommand{\R}[1]{$\mathbb{R}^{#1}$}

%
% various generally useful helpers
%

% derivative of #1 wrt. #2:
\newcommand{\D}[2] {\frac {d#2} {d#1}}

\newcommand{\inv}[1]{\frac{1}{#1}}
\newcommand{\cross}[0]{\times}

\newcommand{\abs}[1]{\lvert{#1}\rvert}
\newcommand{\norm}[1]{\lVert{#1}\rVert}
\newcommand{\innerprod}[2]{\langle{#1}, {#2}\rangle}
\newcommand{\dotprod}[2]{{#1} \cdot {#2}}
\newcommand{\bdotprod}[2]{\left({#1} \cdot {#2}\right)}
\newcommand{\crossprod}[2]{{#1} \cross {#2}}
\newcommand{\tripleprod}[3]{\dotprod{\left(\crossprod{#1}{#2}\right)}{#3}}

\DeclareMathOperator{\Proj}{Proj}
\DeclareMathOperator{\Span}{span}
\DeclareMathOperator{\Sgn}{sgn}
\DeclareMathOperator{\Area}{Area}
\DeclareMathOperator{\Volume}{Volume}

%
% A few miscellaneous things specific to this document
%
\newcommand{\crossop}[1]{\crossprod{#1}{}}

% R2 vector.
\newcommand{\VectorTwo}[2]{
\begin{bmatrix}
 {#1} \\
 {#2}
\end{bmatrix}
}

\newcommand{\VectorN}[1]{
\begin{bmatrix}
{#1}_1 \\
{#1}_2 \\
\vdots \\
{#1}_N \\
\end{bmatrix}
}

\newcommand{\DETuvij}[4]{
\begin{vmatrix}
 {#1}_{#3} & {#1}_{#4} \\
 {#2}_{#3} & {#2}_{#4}
\end{vmatrix}
}

\newcommand{\DETuvwijk}[6]{
\begin{vmatrix}
 {#1}_{#4} & {#1}_{#5} & {#1}_{#6} \\
 {#2}_{#4} & {#2}_{#5} & {#2}_{#6} \\
 {#3}_{#4} & {#3}_{#5} & {#3}_{#6}
\end{vmatrix}
}

\newcommand{\DETuvwxijkl}[8]{
\begin{vmatrix}
 {#1}_{#5} & {#1}_{#6} & {#1}_{#7} & {#1}_{#8} \\
 {#2}_{#5} & {#2}_{#6} & {#2}_{#7} & {#2}_{#8} \\
 {#3}_{#5} & {#3}_{#6} & {#3}_{#7} & {#3}_{#8} \\
 {#4}_{#5} & {#4}_{#6} & {#4}_{#7} & {#4}_{#8} \\
\end{vmatrix}
}

%\newcommand{\DETuvwxyijklm}[10]{
%\begin{vmatrix}
% {#1}_{#6} & {#1}_{#7} & {#1}_{#8} & {#1}_{#9} & {#1}_{#10} \\
% {#2}_{#6} & {#2}_{#7} & {#2}_{#8} & {#2}_{#9} & {#2}_{#10} \\
% {#3}_{#6} & {#3}_{#7} & {#3}_{#8} & {#3}_{#9} & {#3}_{#10} \\
% {#4}_{#6} & {#4}_{#7} & {#4}_{#8} & {#4}_{#9} & {#4}_{#10} \\
% {#5}_{#6} & {#5}_{#7} & {#5}_{#8} & {#5}_{#9} & {#5}_{#10}
%\end{vmatrix}
%}

% R3 vector.
\newcommand{\VectorThree}[3]{
\begin{bmatrix}
 {#1} \\
 {#2} \\
 {#3}
\end{bmatrix}
}



\author{Peeter Joot}
\email{peeter.joot@gmail.com}

%\documentclass[]{eliblogwidescreen}

\usepackage{amsmath}
\usepackage{mathpazo}

%
% shorthand for bold symbols, convenient for vectors and matrices
%
\newcommand{\Ba}[0]{\mathbf{a}}
\newcommand{\Bb}[0]{\mathbf{b}}
\newcommand{\Bc}[0]{\mathbf{c}}
\newcommand{\Bd}[0]{\mathbf{d}}
\newcommand{\Be}[0]{\mathbf{e}}
\newcommand{\Bf}[0]{\mathbf{f}}
\newcommand{\Bg}[0]{\mathbf{g}}
\newcommand{\Bh}[0]{\mathbf{h}}
\newcommand{\Bi}[0]{\mathbf{i}}
\newcommand{\Bj}[0]{\mathbf{j}}
\newcommand{\Bk}[0]{\mathbf{k}}
\newcommand{\Bl}[0]{\mathbf{l}}
\newcommand{\Bm}[0]{\mathbf{m}}
\newcommand{\Bn}[0]{\mathbf{n}}
\newcommand{\Bo}[0]{\mathbf{o}}
\newcommand{\Bp}[0]{\mathbf{p}}
\newcommand{\Bq}[0]{\mathbf{q}}
\newcommand{\Br}[0]{\mathbf{r}}
\newcommand{\Bs}[0]{\mathbf{s}}
\newcommand{\Bt}[0]{\mathbf{t}}
\newcommand{\Bu}[0]{\mathbf{u}}
\newcommand{\Bv}[0]{\mathbf{v}}
\newcommand{\Bw}[0]{\mathbf{w}}
\newcommand{\Bx}[0]{\mathbf{x}}
\newcommand{\By}[0]{\mathbf{y}}
\newcommand{\Bz}[0]{\mathbf{z}}
\newcommand{\BA}[0]{\mathbf{A}}
\newcommand{\BB}[0]{\mathbf{B}}
\newcommand{\BC}[0]{\mathbf{C}}
\newcommand{\BD}[0]{\mathbf{D}}
\newcommand{\BE}[0]{\mathbf{E}}
\newcommand{\BF}[0]{\mathbf{F}}
\newcommand{\BG}[0]{\mathbf{G}}
\newcommand{\BH}[0]{\mathbf{H}}
\newcommand{\BI}[0]{\mathbf{I}}
\newcommand{\BJ}[0]{\mathbf{J}}
\newcommand{\BK}[0]{\mathbf{K}}
\newcommand{\BL}[0]{\mathbf{L}}
\newcommand{\BM}[0]{\mathbf{M}}
\newcommand{\BN}[0]{\mathbf{N}}
\newcommand{\BO}[0]{\mathbf{O}}
\newcommand{\BP}[0]{\mathbf{P}}
\newcommand{\BQ}[0]{\mathbf{Q}}
\newcommand{\BR}[0]{\mathbf{R}}
\newcommand{\BS}[0]{\mathbf{S}}
\newcommand{\BT}[0]{\mathbf{T}}
\newcommand{\BU}[0]{\mathbf{U}}
\newcommand{\BV}[0]{\mathbf{V}}
\newcommand{\BW}[0]{\mathbf{W}}
\newcommand{\BX}[0]{\mathbf{X}}
\newcommand{\BY}[0]{\mathbf{Y}}
\newcommand{\BZ}[0]{\mathbf{Z}}

\newcommand{\Bzero}[0]{\mathbf{0}}
\newcommand{\Btheta}[0]{\boldsymbol{\theta}}
\newcommand{\Btau}[0]{\boldsymbol{\tau}}
\newcommand{\Bomega}[0]{\boldsymbol{\omega}}

%
% shorthand for unit vectors
%
\newcommand{\acap}[0]{\hat{\Ba}}
\newcommand{\bcap}[0]{\hat{\Bb}}
\newcommand{\ccap}[0]{\hat{\Bc}}
\newcommand{\dcap}[0]{\hat{\Bd}}
\newcommand{\ecap}[0]{\hat{\Be}}
\newcommand{\fcap}[0]{\hat{\Bf}}
\newcommand{\gcap}[0]{\hat{\Bg}}
\newcommand{\hcap}[0]{\hat{\Bh}}
\newcommand{\icap}[0]{\hat{\Bi}}
\newcommand{\jcap}[0]{\hat{\Bj}}
\newcommand{\kcap}[0]{\hat{\Bk}}
\newcommand{\lcap}[0]{\hat{\Bl}}
\newcommand{\mcap}[0]{\hat{\Bm}}
\newcommand{\ncap}[0]{\hat{\Bn}}
\newcommand{\ocap}[0]{\hat{\Bo}}
\newcommand{\pcap}[0]{\hat{\Bp}}
\newcommand{\qcap}[0]{\hat{\Bq}}
\newcommand{\rcap}[0]{\hat{\Br}}
\newcommand{\scap}[0]{\hat{\Bs}}
\newcommand{\tcap}[0]{\hat{\Bt}}
\newcommand{\ucap}[0]{\hat{\Bu}}
\newcommand{\vcap}[0]{\hat{\Bv}}
\newcommand{\wcap}[0]{\hat{\Bw}}
\newcommand{\xcap}[0]{\hat{\Bx}}
\newcommand{\ycap}[0]{\hat{\By}}
\newcommand{\zcap}[0]{\hat{\Bz}}
\newcommand{\thetacap}[0]{\hat{\Btheta}}

%
% to write R^n and C^n in a distinguishable fashion.  Perhaps change this
% to the double lined characters upon figuring out how to do so.
%
\newcommand{\C}[1]{$\mathbb{C}^{#1}$}
\newcommand{\R}[1]{$\mathbb{R}^{#1}$}

%
% various generally useful helpers
%

% derivative of #1 wrt. #2:
\newcommand{\D}[2] {\frac {d#2} {d#1}}

\newcommand{\inv}[1]{\frac{1}{#1}}
\newcommand{\cross}[0]{\times}

\newcommand{\abs}[1]{\lvert{#1}\rvert}
\newcommand{\norm}[1]{\lVert{#1}\rVert}
\newcommand{\innerprod}[2]{\langle{#1}, {#2}\rangle}
\newcommand{\dotprod}[2]{{#1} \cdot {#2}}
\newcommand{\bdotprod}[2]{\left({#1} \cdot {#2}\right)}
\newcommand{\crossprod}[2]{{#1} \cross {#2}}
\newcommand{\tripleprod}[3]{\dotprod{\left(\crossprod{#1}{#2}\right)}{#3}}

\DeclareMathOperator{\Proj}{Proj}
\DeclareMathOperator{\Span}{span}
\DeclareMathOperator{\Sgn}{sgn}
\DeclareMathOperator{\Area}{Area}
\DeclareMathOperator{\Volume}{Volume}

%
% A few miscellaneous things specific to this document
%
\newcommand{\crossop}[1]{\crossprod{#1}{}}

% R2 vector.
\newcommand{\VectorTwo}[2]{
\begin{bmatrix}
 {#1} \\
 {#2}
\end{bmatrix}
}

\newcommand{\VectorN}[1]{
\begin{bmatrix}
{#1}_1 \\
{#1}_2 \\
\vdots \\
{#1}_N \\
\end{bmatrix}
}

\newcommand{\DETuvij}[4]{
\begin{vmatrix}
 {#1}_{#3} & {#1}_{#4} \\
 {#2}_{#3} & {#2}_{#4}
\end{vmatrix}
}

\newcommand{\DETuvwijk}[6]{
\begin{vmatrix}
 {#1}_{#4} & {#1}_{#5} & {#1}_{#6} \\
 {#2}_{#4} & {#2}_{#5} & {#2}_{#6} \\
 {#3}_{#4} & {#3}_{#5} & {#3}_{#6}
\end{vmatrix}
}

\newcommand{\DETuvwxijkl}[8]{
\begin{vmatrix}
 {#1}_{#5} & {#1}_{#6} & {#1}_{#7} & {#1}_{#8} \\
 {#2}_{#5} & {#2}_{#6} & {#2}_{#7} & {#2}_{#8} \\
 {#3}_{#5} & {#3}_{#6} & {#3}_{#7} & {#3}_{#8} \\
 {#4}_{#5} & {#4}_{#6} & {#4}_{#7} & {#4}_{#8} \\
\end{vmatrix}
}

%\newcommand{\DETuvwxyijklm}[10]{
%\begin{vmatrix}
% {#1}_{#6} & {#1}_{#7} & {#1}_{#8} & {#1}_{#9} & {#1}_{#10} \\
% {#2}_{#6} & {#2}_{#7} & {#2}_{#8} & {#2}_{#9} & {#2}_{#10} \\
% {#3}_{#6} & {#3}_{#7} & {#3}_{#8} & {#3}_{#9} & {#3}_{#10} \\
% {#4}_{#6} & {#4}_{#7} & {#4}_{#8} & {#4}_{#9} & {#4}_{#10} \\
% {#5}_{#6} & {#5}_{#7} & {#5}_{#8} & {#5}_{#9} & {#5}_{#10}
%\end{vmatrix}
%}

% R3 vector.
\newcommand{\VectorThree}[3]{
\begin{bmatrix}
 {#1} \\
 {#2} \\
 {#3}
\end{bmatrix}
}



\author{Peeter Joot}
\email{peeter.joot@gmail.com}

%\usepackage{cancel}

\chapter{PHY450H1S.  Relativistic Electrodynamics Lecture 25 (Taught by Prof. Erich Poppitz).  Second order interaction, the Darwin Lagrangian.}
\label{chap:relativisticElectrodynamicsL25}
%\useCCL
\blogpage{http://sites.google.com/site/peeterjoot/math2011/relativisticElectrodynamicsL25.pdf}
\date{Mar 31, 2011}
\revisionInfo{relativisticElectrodynamicsL25.tex}

%\beginArtWithToc
\beginArtNoToc

\section{Reading.}

Covering chapter 8 \S 65 material from the text \cite{landau1980classical}.

Covering \href{http://www.physics.utoronto.ca/~poppitz/epoppitz/PHY450_files/RelEMpp181-195.pdf}{pp. 181-195}: (182-189) [Tuesday, Mar. 29]; the EM potentials to order $(v/c)^2$ (190-193); the ``Darwin Lagrangian.  and Hamiltonian for a system of nonrelativistic charged particles to order $(v/c)^2$ and its many uses in physics (194-195) [Wednesday, Mar. 30]

Next week (last topic): attempt to go to the next order $(v/c)^3$ - radiation damping, the limitations of classical electrodynamics, and the relevant time/length/energy scales.

\section{Recap.}

Last time we found that the 

\begin{equation}\label{eqn:relativisticElectrodynamicsL25:10}
\LL_a = \inv{2} m_a \Bv_a^2 - q_a \phi(\Bx_a, t)
\end{equation}

where

\begin{equation}\label{eqn:relativisticElectrodynamicsL25:30}
\phi(\Bx_a, t) = \inv{2} \sum_{a \ne b} \frac{q_b }{\Abs{\Bx_a - \Bx_b}}
+ \frac{q_a}{``\Bx_a - \Bx_a''}
\end{equation}

where the second term is something that no sane person would write, and represents the infinite electrostatic self energy of a charge.  This is infinite because we've assumed that the charge is pointlike, and our solution is to omit it, essentially treating the charge of the electron as distributed, and avoid looking specifically where it is located.

The logic here is that this does not affect the motion (i.e. The Euler Lagrange equations) for the particle, provided it is viewed from afar, with distances $\gg$ \underline{size of particle}.

We made an estimate of the scale for which our Lagrangian does not apply.  Namely

\begin{equation}\label{eqn:relativisticElectrodynamicsL25:50}
\frac{e^2}{r_e} \sim m_e c^2,
\end{equation}

so we were able to conclude that the ``classical radius of the electron'', something that does not really exist, was of the scale

\begin{equation}\label{eqn:relativisticElectrodynamicsL25:70}
r_e \sim \frac{e^2 }{m_e c^2} \sim 10^{-13} \text{cm}
\end{equation}

(We do see this quantity arise in physics, but it is not a radius in the classical sense).

If this estimate was right, we'd calculate that classical EM is value at $r \gg r_e \sim 10^{-13} \text{cm}$.  In reality, classical electrodynamics breaks down at much larger distances.

NOTE: LHC is probing $\sim 10^{-16} \text{cm}$.

Our strategy here is to focus on the structure that can be observed.  We don't have a way to probe to the small scale distances where the structure of the electron is relevant, so our description avoids that small range.

\section{Moving on to the next order in $(v/c)$}

Recall that we dropped terms from the original Lagrangian, which was

\begin{equation}\label{eqn:relativisticElectrodynamicsL25:90}
\LL_a = - m c^2 \sqrt{ 1 - \frac{\Bv_a^2}{c^2}} - q_a \phi(\Bx_a, t) + q_a \frac{\Bv_a}{c} \cdot \BA(\Bx_a, t).
\end{equation}

To the next order, for this particle we have

\begin{equation}\label{eqn:relativisticElectrodynamicsL25:110}
\LL_a = \inv{2} m_a \Bv_a^2 - \frac{m_a}{8} \frac{\Bv_a^4}{c^2} - q_a \phi(\Bx_a, t) + q_a \frac{\Bv_a}{c} \cdot \BA(\Bx_a, t)
\end{equation}

\paragraph{Goal:} Calculate $\phi(\Bx_a)$, $\BA(\Bx_a)$ due to all other particles in a $\Bv/c$ expansion.

We write

\begin{equation}\label{eqn:relativisticElectrodynamicsL25:130}
\phi(\Bx_a, t) = 
\phi^{(0)}(\Bx_a, t)
+\phi^{(1)}(\Bx_a, t)
+\phi^{(2)}(\Bx_a, t).
\end{equation}

Last time we found that the zeroth order term in this approximation was

\begin{equation}\label{eqn:relativisticElectrodynamicsL25:150}
\phi^{(0)}(\Bx_a, t) = \sum_{b \ne a} \frac{q_b}{\Abs{\Bx_a(t) - \Bx_b(t)}},
\end{equation}

and we wish to calculate the next term in the expansion.

We also want to a first order approximation of the vector potential

\begin{equation}\label{eqn:relativisticElectrodynamicsL25:170}
\BA(\Bx_a, t) = 
\cancel{\BA^{(0)}(\Bx_a, t)}
+\BA^{(1)}(\Bx_a, t)
+\cancel{\BA^{(2)}(\Bx_a, t)}
\end{equation}

There is no zero order term and we don't need the second order term (today).

Because

\begin{equation}\label{eqn:relativisticElectrodynamicsL25:190}
\square \BA \sim \frac{\rho \Bv}{c}
\end{equation}

We know the charge and current distributions

\begin{equation}\label{eqn:relativisticElectrodynamicsL25:210}
\phi(\Bx, t) = \int d^3 \Bx \frac{\rho\left(\Bx', t - \Abs{\Bx - \Bx'}/c\right)}{\Abs{\Bx - \Bx}}
\end{equation}

\begin{align}\label{eqn:relativisticElectrodynamicsL25:230}
\rho(\Bx, t) &= \sum_b q_b \delta^3 (\Bx - \Bx_b(t)) \\
\Bj(\Bx, t) &= \sum_b q_b \Bv_b(t) \delta^3 (\Bx - \Bx_b(t))
\end{align}

We'll use the fact that particles have $v \ll c$.  The typical time where the charge distribution will change significantly is of order $\frac{r_{ab}}{v} \gg \frac{r_{ab}}{c}$.  (Here $r_ab/c$ is the time that it takes light to cross the interval, whereas $r_ab/v$ is the time that it takes the particle to do the same).

In order words, in time $\Abs{\Bx - \Bx'}/c \sim r_{ab}/c$, $\rho$ will not change much.

\begin{equation}\label{eqn:relativisticElectrodynamicsL25:250}
\rho\left(\Bx', t - \Abs{\Bx - \Bx'}/c\right) \approx \rho(\Bx', t) 
- \frac{\Abs{\Bx - \Bx'}}{c} \PD{t}{} \rho(\Bx', t) + \inv{2} \left(\frac{\Abs{\Bx - \Bx'}}{c}\right)^2 \PDSq{t}{} \rho(\Bx', t) 
\end{equation}

\begin{equation}\label{eqn:relativisticElectrodynamicsL25:270}
\phi(\Bx, t) 
= \int d^3 \Bx' \frac{\rho(\Bx', t)}{\Abs{\Bx - \Bx'}} - \PD{t}{} \int d^3 \Bx' \inv{c} \rho(\Bx', t) 
+
\inv{2 c^2} \int d^3 \Bx \Abs{\Bx - \Bx'} \PDSq{t}{} \rho(\Bx', t) 
\end{equation}

The second integral is the total charge $\times 1/c$, and doesn't change in time.  So to first order our charge density is

\begin{equation}\label{eqn:relativisticElectrodynamicsL25:290}
\rho\left(\Bx', t - \Abs{\Bx - \Bx'}/c\right) \approx \rho(\Bx', t) = \sum_b \frac{q_b}{\Abs{\Bx - \Bx_b(t)}}
\end{equation}

%FIXME: do we want to second order here?

How about $\BA$?

\begin{equation}\label{eqn:relativisticElectrodynamicsL25:310}
\BA(\Bx_a, t) = 
\cancel{\BA^{(0)}(\Bx_a, t)}
+\BA^{(1)}(\Bx_a, t)
+\cancel{\BA^{(2)}(\Bx_a, t)}
\end{equation}

\begin{align*}
A^{(1)} 
&= \inv{c} \int d^3 \Bx' \inv{\Abs{\Bx - \Bx'}} \Bj\left(\Bx', t - \Abs{\Bx - \Bx'}/c\right)  \\
&\approx \inv{c} \int d^3 \Bx' \inv{\Abs{\Bx - \Bx'}} \Bj(\Bx', t)
\end{align*}

Ah, this shows why it was written that there's no second order term.  Because $\Bj \sim \Bv_a$, we neccessarily have $\Bv_a/c$ dependence even in the zeroth order expansion about $t=0$ in our retarded time expansion of $\BA(\Bx', t_r)$.

Assembling all the results, we have

\begin{equation}\label{eqn:relativisticElectrodynamicsL25:330}
\LL_a = \inv{2} m_a \Bv_a^2 - \frac{m_a}{8} \frac{\Bv_a^4}{c^2} - q_a \phi^{(0)}(\Bx_a, t) - q_a \phi^{(2)}(\Bx_a, t) + q_a \frac{\Bv_a}{c} \cdot \BA^{(1)}(\Bx_a, t)
\end{equation}

\begin{align*}
\phi^{(2)}(\Bx, t) 
&= \PD{t}{} \left( \inv{2 c^2} \PD{t}{} \int d^3 \Bx' \Abs{\Bx - \Bx'} \rho(\Bx', t) \right) \\
&= \PD{t}{} \left( \inv{2 c^2} \PD{t}{} \int d^3 \Bx' \Abs{\Bx - \Bx'} \sum_b q_b \delta^3(\Bx - \Bx_b(t)) \right) \\
&= \PD{t}{} \left( \inv{2 c^2} \PD{t}{} \sum_b q_b \Abs{\Bx - \Bx_b(t)} \right)
\end{align*}

And 

\begin{align*}
\BA^{(1)}(\Bx, t) 
&= \inv{c} \int d^3 \Bx' \inv{\Abs{\Bx - \Bx'}} \Bj(\Bx, t) \\
&= \inv{c} \int d^3 \Bx' \inv{\Abs{\Bx - \Bx'}} \sum_b q_b \Bv_b \delta^3(\Bx -\Bx_b) \\
&= \inv{c} \sum_b q_b \Bv_b \inv{\Abs{\Bx -\Bx_b}} \\
\end{align*}

Recall that $\phi^{(0)}$ was given by \ref{eqn:relativisticElectrodynamicsL25:150}.

\section{A gauge transformation to simplify things.}

\paragraph{Remember:} Gauge transformation

\begin{align}\label{eqn:relativisticElectrodynamicsL25:350}
\phi'(\Bx, t) &= \phi(\Bx, t) - \inv{c} \PD{t}{f(\Bx, t)} \\
\BA'(\Bx, t) &= \BA(\Bx, t) + \spacegrad f(\Bx, t) \\
\end{align}

This will not change the physics.  Take

\begin{equation}\label{eqn:relativisticElectrodynamicsL25:370}
f(\Bx, t) = \sum_b \frac{q_b}{2 c} \PD{t}{} \Abs{\Bx - \Bx_b(t)}
\end{equation}

Then 

\begin{equation}\label{eqn:relativisticElectrodynamicsL25:390}
{\phi'}^{(2)} = 0
\end{equation}

\begin{equation}\label{eqn:relativisticElectrodynamicsL25:410}
{\BA'}^{(1)}(\Bx, t) = \inv{c} \sum_b \frac{q_b \Bv_b}{\Abs{\Bx - \Bx_b}} + \spacegrad \sum_b \frac{q_b}{2 c} \PD{t}{} \Abs{\Bx - \Bx_b}
\end{equation}

With 

\begin{equation}\label{eqn:relativisticElectrodynamicsL25:430}
\Bn = \frac{\Bx - \Bx_b(t)}{\Abs{\Bx - \Bx_b}}
\end{equation}

we have

\begin{equation}\label{eqn:relativisticElectrodynamicsL25:450}
{\BA'}^{(1)}(\Bx, t) = \sum_b \frac{\Bv_b + \Bn (\Bn \cdot \Bv_b)}{2 c \Abs{\Bx - \Bx_b} }
\end{equation}

\EndArticle
