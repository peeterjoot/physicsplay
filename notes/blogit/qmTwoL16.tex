%
% Copyright � 2015 Peeter Joot.  All Rights Reserved.
% Licenced as described in the file LICENSE under the root directory of this GIT repository.
%
\documentclass[]{eliblog}

\usepackage{amsmath}
\usepackage{mathpazo}

%
% shorthand for bold symbols, convenient for vectors and matrices
%
\newcommand{\Ba}[0]{\mathbf{a}}
\newcommand{\Bb}[0]{\mathbf{b}}
\newcommand{\Bc}[0]{\mathbf{c}}
\newcommand{\Bd}[0]{\mathbf{d}}
\newcommand{\Be}[0]{\mathbf{e}}
\newcommand{\Bf}[0]{\mathbf{f}}
\newcommand{\Bg}[0]{\mathbf{g}}
\newcommand{\Bh}[0]{\mathbf{h}}
\newcommand{\Bi}[0]{\mathbf{i}}
\newcommand{\Bj}[0]{\mathbf{j}}
\newcommand{\Bk}[0]{\mathbf{k}}
\newcommand{\Bl}[0]{\mathbf{l}}
\newcommand{\Bm}[0]{\mathbf{m}}
\newcommand{\Bn}[0]{\mathbf{n}}
\newcommand{\Bo}[0]{\mathbf{o}}
\newcommand{\Bp}[0]{\mathbf{p}}
\newcommand{\Bq}[0]{\mathbf{q}}
\newcommand{\Br}[0]{\mathbf{r}}
\newcommand{\Bs}[0]{\mathbf{s}}
\newcommand{\Bt}[0]{\mathbf{t}}
\newcommand{\Bu}[0]{\mathbf{u}}
\newcommand{\Bv}[0]{\mathbf{v}}
\newcommand{\Bw}[0]{\mathbf{w}}
\newcommand{\Bx}[0]{\mathbf{x}}
\newcommand{\By}[0]{\mathbf{y}}
\newcommand{\Bz}[0]{\mathbf{z}}
\newcommand{\BA}[0]{\mathbf{A}}
\newcommand{\BB}[0]{\mathbf{B}}
\newcommand{\BC}[0]{\mathbf{C}}
\newcommand{\BD}[0]{\mathbf{D}}
\newcommand{\BE}[0]{\mathbf{E}}
\newcommand{\BF}[0]{\mathbf{F}}
\newcommand{\BG}[0]{\mathbf{G}}
\newcommand{\BH}[0]{\mathbf{H}}
\newcommand{\BI}[0]{\mathbf{I}}
\newcommand{\BJ}[0]{\mathbf{J}}
\newcommand{\BK}[0]{\mathbf{K}}
\newcommand{\BL}[0]{\mathbf{L}}
\newcommand{\BM}[0]{\mathbf{M}}
\newcommand{\BN}[0]{\mathbf{N}}
\newcommand{\BO}[0]{\mathbf{O}}
\newcommand{\BP}[0]{\mathbf{P}}
\newcommand{\BQ}[0]{\mathbf{Q}}
\newcommand{\BR}[0]{\mathbf{R}}
\newcommand{\BS}[0]{\mathbf{S}}
\newcommand{\BT}[0]{\mathbf{T}}
\newcommand{\BU}[0]{\mathbf{U}}
\newcommand{\BV}[0]{\mathbf{V}}
\newcommand{\BW}[0]{\mathbf{W}}
\newcommand{\BX}[0]{\mathbf{X}}
\newcommand{\BY}[0]{\mathbf{Y}}
\newcommand{\BZ}[0]{\mathbf{Z}}

\newcommand{\Bzero}[0]{\mathbf{0}}
\newcommand{\Btheta}[0]{\boldsymbol{\theta}}
\newcommand{\Btau}[0]{\boldsymbol{\tau}}
\newcommand{\Bomega}[0]{\boldsymbol{\omega}}

%
% shorthand for unit vectors
%
\newcommand{\acap}[0]{\hat{\Ba}}
\newcommand{\bcap}[0]{\hat{\Bb}}
\newcommand{\ccap}[0]{\hat{\Bc}}
\newcommand{\dcap}[0]{\hat{\Bd}}
\newcommand{\ecap}[0]{\hat{\Be}}
\newcommand{\fcap}[0]{\hat{\Bf}}
\newcommand{\gcap}[0]{\hat{\Bg}}
\newcommand{\hcap}[0]{\hat{\Bh}}
\newcommand{\icap}[0]{\hat{\Bi}}
\newcommand{\jcap}[0]{\hat{\Bj}}
\newcommand{\kcap}[0]{\hat{\Bk}}
\newcommand{\lcap}[0]{\hat{\Bl}}
\newcommand{\mcap}[0]{\hat{\Bm}}
\newcommand{\ncap}[0]{\hat{\Bn}}
\newcommand{\ocap}[0]{\hat{\Bo}}
\newcommand{\pcap}[0]{\hat{\Bp}}
\newcommand{\qcap}[0]{\hat{\Bq}}
\newcommand{\rcap}[0]{\hat{\Br}}
\newcommand{\scap}[0]{\hat{\Bs}}
\newcommand{\tcap}[0]{\hat{\Bt}}
\newcommand{\ucap}[0]{\hat{\Bu}}
\newcommand{\vcap}[0]{\hat{\Bv}}
\newcommand{\wcap}[0]{\hat{\Bw}}
\newcommand{\xcap}[0]{\hat{\Bx}}
\newcommand{\ycap}[0]{\hat{\By}}
\newcommand{\zcap}[0]{\hat{\Bz}}
\newcommand{\thetacap}[0]{\hat{\Btheta}}

%
% to write R^n and C^n in a distinguishable fashion.  Perhaps change this
% to the double lined characters upon figuring out how to do so.
%
\newcommand{\C}[1]{$\mathbb{C}^{#1}$}
\newcommand{\R}[1]{$\mathbb{R}^{#1}$}

%
% various generally useful helpers
%

% derivative of #1 wrt. #2:
\newcommand{\D}[2] {\frac {d#2} {d#1}}

\newcommand{\inv}[1]{\frac{1}{#1}}
\newcommand{\cross}[0]{\times}

\newcommand{\abs}[1]{\lvert{#1}\rvert}
\newcommand{\norm}[1]{\lVert{#1}\rVert}
\newcommand{\innerprod}[2]{\langle{#1}, {#2}\rangle}
\newcommand{\dotprod}[2]{{#1} \cdot {#2}}
\newcommand{\bdotprod}[2]{\left({#1} \cdot {#2}\right)}
\newcommand{\crossprod}[2]{{#1} \cross {#2}}
\newcommand{\tripleprod}[3]{\dotprod{\left(\crossprod{#1}{#2}\right)}{#3}}

\DeclareMathOperator{\Proj}{Proj}
\DeclareMathOperator{\Span}{span}
\DeclareMathOperator{\Sgn}{sgn}
\DeclareMathOperator{\Area}{Area}
\DeclareMathOperator{\Volume}{Volume}

%
% A few miscellaneous things specific to this document
%
\newcommand{\crossop}[1]{\crossprod{#1}{}}

% R2 vector.
\newcommand{\VectorTwo}[2]{
\begin{bmatrix}
 {#1} \\
 {#2}
\end{bmatrix}
}

\newcommand{\VectorN}[1]{
\begin{bmatrix}
{#1}_1 \\
{#1}_2 \\
\vdots \\
{#1}_N \\
\end{bmatrix}
}

\newcommand{\DETuvij}[4]{
\begin{vmatrix}
 {#1}_{#3} & {#1}_{#4} \\
 {#2}_{#3} & {#2}_{#4}
\end{vmatrix}
}

\newcommand{\DETuvwijk}[6]{
\begin{vmatrix}
 {#1}_{#4} & {#1}_{#5} & {#1}_{#6} \\
 {#2}_{#4} & {#2}_{#5} & {#2}_{#6} \\
 {#3}_{#4} & {#3}_{#5} & {#3}_{#6}
\end{vmatrix}
}

\newcommand{\DETuvwxijkl}[8]{
\begin{vmatrix}
 {#1}_{#5} & {#1}_{#6} & {#1}_{#7} & {#1}_{#8} \\
 {#2}_{#5} & {#2}_{#6} & {#2}_{#7} & {#2}_{#8} \\
 {#3}_{#5} & {#3}_{#6} & {#3}_{#7} & {#3}_{#8} \\
 {#4}_{#5} & {#4}_{#6} & {#4}_{#7} & {#4}_{#8} \\
\end{vmatrix}
}

%\newcommand{\DETuvwxyijklm}[10]{
%\begin{vmatrix}
% {#1}_{#6} & {#1}_{#7} & {#1}_{#8} & {#1}_{#9} & {#1}_{#10} \\
% {#2}_{#6} & {#2}_{#7} & {#2}_{#8} & {#2}_{#9} & {#2}_{#10} \\
% {#3}_{#6} & {#3}_{#7} & {#3}_{#8} & {#3}_{#9} & {#3}_{#10} \\
% {#4}_{#6} & {#4}_{#7} & {#4}_{#8} & {#4}_{#9} & {#4}_{#10} \\
% {#5}_{#6} & {#5}_{#7} & {#5}_{#8} & {#5}_{#9} & {#5}_{#10}
%\end{vmatrix}
%}

% R3 vector.
\newcommand{\VectorThree}[3]{
\begin{bmatrix}
 {#1} \\
 {#2} \\
 {#3}
\end{bmatrix}
}



\author{Peeter Joot}
\email{peeter.joot@gmail.com}

%\documentclass[]{eliblogwidescreen}

\usepackage{amsmath}
\usepackage{mathpazo}

%
% shorthand for bold symbols, convenient for vectors and matrices
%
\newcommand{\Ba}[0]{\mathbf{a}}
\newcommand{\Bb}[0]{\mathbf{b}}
\newcommand{\Bc}[0]{\mathbf{c}}
\newcommand{\Bd}[0]{\mathbf{d}}
\newcommand{\Be}[0]{\mathbf{e}}
\newcommand{\Bf}[0]{\mathbf{f}}
\newcommand{\Bg}[0]{\mathbf{g}}
\newcommand{\Bh}[0]{\mathbf{h}}
\newcommand{\Bi}[0]{\mathbf{i}}
\newcommand{\Bj}[0]{\mathbf{j}}
\newcommand{\Bk}[0]{\mathbf{k}}
\newcommand{\Bl}[0]{\mathbf{l}}
\newcommand{\Bm}[0]{\mathbf{m}}
\newcommand{\Bn}[0]{\mathbf{n}}
\newcommand{\Bo}[0]{\mathbf{o}}
\newcommand{\Bp}[0]{\mathbf{p}}
\newcommand{\Bq}[0]{\mathbf{q}}
\newcommand{\Br}[0]{\mathbf{r}}
\newcommand{\Bs}[0]{\mathbf{s}}
\newcommand{\Bt}[0]{\mathbf{t}}
\newcommand{\Bu}[0]{\mathbf{u}}
\newcommand{\Bv}[0]{\mathbf{v}}
\newcommand{\Bw}[0]{\mathbf{w}}
\newcommand{\Bx}[0]{\mathbf{x}}
\newcommand{\By}[0]{\mathbf{y}}
\newcommand{\Bz}[0]{\mathbf{z}}
\newcommand{\BA}[0]{\mathbf{A}}
\newcommand{\BB}[0]{\mathbf{B}}
\newcommand{\BC}[0]{\mathbf{C}}
\newcommand{\BD}[0]{\mathbf{D}}
\newcommand{\BE}[0]{\mathbf{E}}
\newcommand{\BF}[0]{\mathbf{F}}
\newcommand{\BG}[0]{\mathbf{G}}
\newcommand{\BH}[0]{\mathbf{H}}
\newcommand{\BI}[0]{\mathbf{I}}
\newcommand{\BJ}[0]{\mathbf{J}}
\newcommand{\BK}[0]{\mathbf{K}}
\newcommand{\BL}[0]{\mathbf{L}}
\newcommand{\BM}[0]{\mathbf{M}}
\newcommand{\BN}[0]{\mathbf{N}}
\newcommand{\BO}[0]{\mathbf{O}}
\newcommand{\BP}[0]{\mathbf{P}}
\newcommand{\BQ}[0]{\mathbf{Q}}
\newcommand{\BR}[0]{\mathbf{R}}
\newcommand{\BS}[0]{\mathbf{S}}
\newcommand{\BT}[0]{\mathbf{T}}
\newcommand{\BU}[0]{\mathbf{U}}
\newcommand{\BV}[0]{\mathbf{V}}
\newcommand{\BW}[0]{\mathbf{W}}
\newcommand{\BX}[0]{\mathbf{X}}
\newcommand{\BY}[0]{\mathbf{Y}}
\newcommand{\BZ}[0]{\mathbf{Z}}

\newcommand{\Bzero}[0]{\mathbf{0}}
\newcommand{\Btheta}[0]{\boldsymbol{\theta}}
\newcommand{\Btau}[0]{\boldsymbol{\tau}}
\newcommand{\Bomega}[0]{\boldsymbol{\omega}}

%
% shorthand for unit vectors
%
\newcommand{\acap}[0]{\hat{\Ba}}
\newcommand{\bcap}[0]{\hat{\Bb}}
\newcommand{\ccap}[0]{\hat{\Bc}}
\newcommand{\dcap}[0]{\hat{\Bd}}
\newcommand{\ecap}[0]{\hat{\Be}}
\newcommand{\fcap}[0]{\hat{\Bf}}
\newcommand{\gcap}[0]{\hat{\Bg}}
\newcommand{\hcap}[0]{\hat{\Bh}}
\newcommand{\icap}[0]{\hat{\Bi}}
\newcommand{\jcap}[0]{\hat{\Bj}}
\newcommand{\kcap}[0]{\hat{\Bk}}
\newcommand{\lcap}[0]{\hat{\Bl}}
\newcommand{\mcap}[0]{\hat{\Bm}}
\newcommand{\ncap}[0]{\hat{\Bn}}
\newcommand{\ocap}[0]{\hat{\Bo}}
\newcommand{\pcap}[0]{\hat{\Bp}}
\newcommand{\qcap}[0]{\hat{\Bq}}
\newcommand{\rcap}[0]{\hat{\Br}}
\newcommand{\scap}[0]{\hat{\Bs}}
\newcommand{\tcap}[0]{\hat{\Bt}}
\newcommand{\ucap}[0]{\hat{\Bu}}
\newcommand{\vcap}[0]{\hat{\Bv}}
\newcommand{\wcap}[0]{\hat{\Bw}}
\newcommand{\xcap}[0]{\hat{\Bx}}
\newcommand{\ycap}[0]{\hat{\By}}
\newcommand{\zcap}[0]{\hat{\Bz}}
\newcommand{\thetacap}[0]{\hat{\Btheta}}

%
% to write R^n and C^n in a distinguishable fashion.  Perhaps change this
% to the double lined characters upon figuring out how to do so.
%
\newcommand{\C}[1]{$\mathbb{C}^{#1}$}
\newcommand{\R}[1]{$\mathbb{R}^{#1}$}

%
% various generally useful helpers
%

% derivative of #1 wrt. #2:
\newcommand{\D}[2] {\frac {d#2} {d#1}}

\newcommand{\inv}[1]{\frac{1}{#1}}
\newcommand{\cross}[0]{\times}

\newcommand{\abs}[1]{\lvert{#1}\rvert}
\newcommand{\norm}[1]{\lVert{#1}\rVert}
\newcommand{\innerprod}[2]{\langle{#1}, {#2}\rangle}
\newcommand{\dotprod}[2]{{#1} \cdot {#2}}
\newcommand{\bdotprod}[2]{\left({#1} \cdot {#2}\right)}
\newcommand{\crossprod}[2]{{#1} \cross {#2}}
\newcommand{\tripleprod}[3]{\dotprod{\left(\crossprod{#1}{#2}\right)}{#3}}

\DeclareMathOperator{\Proj}{Proj}
\DeclareMathOperator{\Span}{span}
\DeclareMathOperator{\Sgn}{sgn}
\DeclareMathOperator{\Area}{Area}
\DeclareMathOperator{\Volume}{Volume}

%
% A few miscellaneous things specific to this document
%
\newcommand{\crossop}[1]{\crossprod{#1}{}}

% R2 vector.
\newcommand{\VectorTwo}[2]{
\begin{bmatrix}
 {#1} \\
 {#2}
\end{bmatrix}
}

\newcommand{\VectorN}[1]{
\begin{bmatrix}
{#1}_1 \\
{#1}_2 \\
\vdots \\
{#1}_N \\
\end{bmatrix}
}

\newcommand{\DETuvij}[4]{
\begin{vmatrix}
 {#1}_{#3} & {#1}_{#4} \\
 {#2}_{#3} & {#2}_{#4}
\end{vmatrix}
}

\newcommand{\DETuvwijk}[6]{
\begin{vmatrix}
 {#1}_{#4} & {#1}_{#5} & {#1}_{#6} \\
 {#2}_{#4} & {#2}_{#5} & {#2}_{#6} \\
 {#3}_{#4} & {#3}_{#5} & {#3}_{#6}
\end{vmatrix}
}

\newcommand{\DETuvwxijkl}[8]{
\begin{vmatrix}
 {#1}_{#5} & {#1}_{#6} & {#1}_{#7} & {#1}_{#8} \\
 {#2}_{#5} & {#2}_{#6} & {#2}_{#7} & {#2}_{#8} \\
 {#3}_{#5} & {#3}_{#6} & {#3}_{#7} & {#3}_{#8} \\
 {#4}_{#5} & {#4}_{#6} & {#4}_{#7} & {#4}_{#8} \\
\end{vmatrix}
}

%\newcommand{\DETuvwxyijklm}[10]{
%\begin{vmatrix}
% {#1}_{#6} & {#1}_{#7} & {#1}_{#8} & {#1}_{#9} & {#1}_{#10} \\
% {#2}_{#6} & {#2}_{#7} & {#2}_{#8} & {#2}_{#9} & {#2}_{#10} \\
% {#3}_{#6} & {#3}_{#7} & {#3}_{#8} & {#3}_{#9} & {#3}_{#10} \\
% {#4}_{#6} & {#4}_{#7} & {#4}_{#8} & {#4}_{#9} & {#4}_{#10} \\
% {#5}_{#6} & {#5}_{#7} & {#5}_{#8} & {#5}_{#9} & {#5}_{#10}
%\end{vmatrix}
%}

% R3 vector.
\newcommand{\VectorThree}[3]{
\begin{bmatrix}
 {#1} \\
 {#2} \\
 {#3}
\end{bmatrix}
}



\author{Peeter Joot}
\email{peeter.joot@gmail.com}


\chapter{PHY456H1F: Quantum Mechanics II.  Lecture 16 (Taught by Prof J.E. Sipe).  Hydrogen atom with spin.}
\label{chap:qmTwoL16}
%\useCCL
\blogpage{http://sites.google.com/site/peeterjoot/math2011/qmTwoL16.pdf}
\date{Nov 2, 2011}
\revisionInfo{qmTwoL16.tex}

\beginArtWithToc
%\beginArtNoToc

%\text{rel} = \text{rel}
%\text{electron} = \text{electron}
%\text{proton} = \text{proton}
%\text{CM} = \text{CM}

\section{Disclaimer.}

Peeter's lecture notes from class.  May not be entirely coherent.

\section{The hydrogen atom with spin.}

READING: what chapter of \cite{desai2009quantum} ?

For a spinless hydrogen atom, the Hamiltonian was

\begin{equation}\label{eqn:qmTwoL16:n}
H = H_{\text{CM}} \otimes H_{\text{rel}}
\end{equation}

where we have independent Hamiltonians for the motion of the center of mass and the relative motion of the electron to the proton.

The basis kets for these could be designated $\ket{\Bp_\text{CM}}$ and $\ket{\Bp_\text{rel}}$ respectively.

Now we want to augment this, treating

\begin{equation}\label{eqn:qmTwoL16:n}
H = H_{\text{CM}} \otimes H_{\text{rel}} \otimes H_{\text{s}}
\end{equation}

where $H_{\text{s}}$ is the Hamiltonian for the spin of the electron.  We are neglecting the spin of the proton, but that could also be included (this turns out to be a lesser effect).

We'll introduce a Hamiltonian including the dynamics of the relative motion and the electron spin

\begin{equation}\label{eqn:qmTwoL16:n}
H_{\text{rel}} \otimes H_{\text{s}}
\end{equation}

Covering the Hilbert space for this system we'll use basis kets 

\begin{equation}\label{eqn:qmTwoL16:n}
\ket{nlm\pm}
\end{equation}

\begin{equation}\label{eqn:qmTwoL16:n}
\begin{aligned}
\ket{nlm+} 
&\rightarrow 
\begin{bmatrix}
\braket{\Br+}{nlm+} \\
\braket{\Br-}{nlm+} \\
\end{bmatrix}
=
\begin{bmatrix}
\Phi_{nlm}(\Br) \\
0
\end{bmatrix} \\
\ket{nlm-} 
&\rightarrow 
\begin{bmatrix}
\braket{\Br+}{nlm-} \\
\braket{\Br-}{nlm-} \\
\end{bmatrix}
=
\begin{bmatrix}
0 \\
\Phi_{nlm}(\Br) 
\end{bmatrix}.
\end{aligned}
\end{equation}

Here $\Br$ should be understood to really mean $\Br_\text{rel}$.  Our full Hamiltonian was

\begin{equation}\label{eqn:qmTwoL16:n}
H = 
\frac{P_\text{CM}^2}{2M} 
+ 
\left(
\frac{P_\text{rel}^2}{2\mu}
-
\frac{e^2}{R_\text{rel}}
\right)
- \Bmu_0 \cdot \BB
- \Bmu_s \cdot \BB
\end{equation}

where 

\begin{equation}\label{eqn:qmTwoL16:n}
M = m_\text{proton} + m_\text{electron},
\end{equation}

and
\begin{equation}\label{eqn:qmTwoL16:n}
\inv{\mu} = \inv{m_\text{proton}} + \inv{m_\text{electron}}.
\end{equation}

For a uniform magnetic field

\begin{align}\label{eqn:qmTwoL16:n}
\Bmu_0 &= \left( -\frac{e}{2 m c} \right) \BL \\
\Bmu_s &= g \left( -\frac{e}{2 m c} \right) \BS
\end{align}

We also have higher order terms (higher order multipoles) and relativistic corrections (like spin orbit coupling).

\section{Two spins.}

READING: \S 28 of \cite{desai2009quantum}.

\paragraph{Example}: Consider two electrons, 1 in each of 2 quantum dots.

\begin{equation}\label{eqn:qmTwoL16:n}
H = H_{1} \otimes H_{2}
\end{equation}

where $H_1$ and $H_2$ are both spin Hamiltonians for respective 2D Hilbert spaces.  Our complete Hilbert space is thus a 4D space.

We'll write

\begin{equation}\label{eqn:qmTwoL16:n}
\begin{aligned}
\ket{+}_1 \otimes \ket{+}_2 &= \ket{++} \\
\ket{+}_1 \otimes \ket{-}_2 &= \ket{+-} \\
\ket{-}_1 \otimes \ket{+}_2 &= \ket{-+} \\
\ket{-}_1 \otimes \ket{-}_2 &= \ket{--} 
\end{aligned}
\end{equation}

Can introduce

\begin{align}\label{eqn:qmTwoL16:n}
\BS_1 &= \BS_1^{(1)} \otimes I^{(2)} \\
\BS_2 &= I^{(1)} \otimes \BS_2^{(2)}
\end{align}

Here we ``promote'' each of the individual spin operators to spin operators in the complete Hilbert space.

We write

\begin{align}\label{eqn:qmTwoL16:n}
S_{1z}\ket{++} &= \frac{\hbar}{2} \ket{++} \\
S_{1z}\ket{+-} &= \frac{\hbar}{2} \ket{+-}
\end{align}

Where
\begin{equation}\label{eqn:qmTwoL16:n}
\BS = BS_1 + \BS_2
\end{equation}

is the full spin angular momentum operator.

\begin{equation}\label{eqn:qmTwoL16:n}
S_z = S_{1z} + S_{2z}
\end{equation}

\begin{align}\label{eqn:qmTwoL16:n}
S_z\ket{++} &= (S_{1z} + S_{2z}) \ket{++} = \left( \frac{\hbar}{2} +\frac{\hbar}{2} \right) \ket{++} = \hbar \ket{++} \\ 
S_z\ket{+-} &= 0 \\
S_z\ket{-+} &= 0 \\
S_z\ket{--} &= (S_{1z} + S_{2z}) \ket{--} = -\left( \frac{\hbar}{2} +\frac{\hbar}{2} \right) \ket{--} = -\hbar \ket{--} 
\end{align}

All eigenkets of $S_z$.  All eigenkets of $\BS_1^2 = S_{1x}^2 +S_{1y}^2 +S_{1z}^2$, where

\begin{align}\label{eqn:qmTwoL16:n}
S_1^2 \ket{x x} &= \frac{3}{4} \hbar^2 \ket{x x} \\
S_2^2 \ket{x x} &= \frac{3}{4} \hbar^2 \ket{x x}
\end{align}

\begin{equation}\label{eqn:qmTwoL16:n}
\begin{aligned}
S^2 &= 
(\BS_1^2
+\BS_2^2) 
\cdot
(\BS_1^2
+\BS_2^2)  \\
&= 
S_1^2 + S_2^2 + 2 \BS_1 \cdot \BS_2
\end{aligned}
\end{equation}

Are all the product kets eigenkets of $S^2$?  Calculate

\begin{align*}
S^2 \ket{+-} 
&= 
(S_1^2 + S_2^2 + 2 \BS_1 \cdot \BS_2) \ket{+-} \\
&=
\left(\frac{3}{4}\hbar^2
+\frac{3}{4}\hbar^2\right)
+ 2 S_{1x} S_{2x} \ket{+-} 
+ 2 S_{1y} S_{2y} \ket{+-} 
+ 2 S_{1z} S_{2z} \ket{+-} 
\end{align*}

For the $z$ mixed terms, we have

\begin{equation}\label{eqn:qmTwoL16:n}
2 S_{1x} S_{2y} \ket{+-}  = 2 
\left(\frac{\hbar}{2}\right)
\left(-\frac{\hbar}{2}\right)
\ket{+-}
\end{equation}

So

\begin{equation}\label{eqn:qmTwoL16:n}
S^2\ket{+-} = 
\hbar^2 \ket{+-} 
+ 2 S_{1x} S_{2x} \ket{+-} 
+ 2 S_{1y} S_{2y} \ket{+-} 
\end{equation}

Recall

\begin{align*}
S_x\ket{+} 
&\rightarrow 
\frac{\hbar}{2} \PauliX
\begin{bmatrix}
1 \\
0
\end{bmatrix} \\
=
\frac{\hbar}{2}
\begin{bmatrix}
0 \\
1 
\end{bmatrix}
=
\frac{\hbar}{2} \ket{-} \\
S_x\ket{-} 
&\rightarrow 
\frac{\hbar}{2} \ket{+} \\
S_y\ket{+} 
&\rightarrow 
\frac{i\hbar}{2} \ket{-} \\
S_y\ket{-} 
&\rightarrow 
\frac{-i\hbar}{2} \ket{+} 
\end{align*}

So

\begin{equation}\label{eqn:qmTwoL16:n}
S^2
\rightarrow 
\hbar^2
\begin{bmatrix}
2 & 0 & 0 & 0 \\
0 & 1 & 1 & 0 \\
0 & 1 & 1 & 0 \\
0 & 0 & 0 & 2 \\
\end{bmatrix}
\end{equation}

where the matrix is taken with respect to the (ordered) basis

\begin{equation}\label{eqn:qmTwoL16:n}
\{
\ket{++},
\ket{+-},
\ket{-+},
\ket{--}
\}
\end{equation}

However, 

\begin{align}\label{eqn:qmTwoL16:n}
\antisymmetric{S^2}{S_z} &= 0 \\
\antisymmetric{S_i}{S_j} &= i \hbar \sum_k \epsilon_{ijk} S_k
\end{align}

It should be possible to find eigenkets of $S^2$ and $S_z$

\begin{align}\label{eqn:qmTwoL16:n}
S^2 \ket{s m_s} &= s(s+1)\hbar^2 \ket{s m_s} \\
S_z \ket{s m_s} &= \hbar m_s \ket{s m_s} 
\end{align}

Eigenkets of $S^2$ and $S_z$

\begin{equation}\label{eqn:qmTwoL16:n}
\begin{array}{l l}
\ket{++} 
& \mbox{$s = 1$ and $m_s = 1$} \\
\inv{\sqrt{2}} \left( \ket{+-} + \ket{-+} \right) 
& \mbox{$s = 1$ and $m_s = 0$} \\
\ket{--} 
& \mbox{$s = 1$ and $m_s = -1$} \\
\inv{\sqrt{2}} \left( \ket{+-} - \ket{-+} \right) 
& \mbox{$s = 0$ and $m_s = 0$}
\end{array}
\end{equation}

The first three kets here can be grouped into a triplet in a 3D Hilbert space, whereas the last treated as a singlet in a 1D Hilbert space.

Form a grouping

\begin{equation}\label{eqn:qmTwoL16:n}
H = H_1 \otimes H_2
\end{equation}

Can write

\begin{equation}\label{eqn:qmTwoL16:n}
\inv{2} \otimes \inv{2} = 1 \oplus 0
\end{equation}

where the $1$ and $0$ here refer to the spin indice $s$.

\subsection{Other examples}

Consider, perhaps, the $l=5$ state of the hydrogen atom

\begin{align}\label{eqn:qmTwoL16:n}
J_1^2 \ket{j_1 m_1} &= j_1(j_1+1)\hbar^2 \ket{j_1 m_1} \\
J_{1z} \ket{j_1 m_1} &= \hbar m_1 \ket{j_1 m_1} 
\end{align}

\begin{align}\label{eqn:qmTwoL16:n}
J_2^2 \ket{j_2 m_2} &= j_2(j_2+1)\hbar^2 \ket{j_2 m_2} \\
J_{2z} \ket{j_2 m_2} &= \hbar m_2 \ket{j_2 m_2} 
\end{align}

Consider the Hilbert space spanned by $\ket{j_1 m_1} \otimes \ket{j_2 m_2}$, a $(2 j_1 + 1)(2 j_2 + 1)$ dimensional space.  How to find the eigenkets of $J^2$ and $J_z$?

\EndArticle
