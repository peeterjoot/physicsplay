%
% Copyright � 2015 Peeter Joot.  All Rights Reserved.
% Licenced as described in the file LICENSE under the root directory of this GIT repository.
%
\newcommand{\authorname}{Peeter Joot}
\newcommand{\email}{peeterjoot@protonmail.com}
\newcommand{\basename}{FIXMEbasenameUndefined}
\newcommand{\dirname}{notes/FIXMEdirnameUndefined/}

\renewcommand{\basename}{gaMagneticSourcesToTensorToVector}
\renewcommand{\dirname}{notes/FIXMEwheretodirname/}
%\newcommand{\dateintitle}{}
%\newcommand{\keywords}{}

\newcommand{\authorname}{Peeter Joot}
\newcommand{\onlineurl}{http://sites.google.com/site/peeterjoot2/math2013/\basename.pdf}
\newcommand{\sourcepath}{\dirname\basename.tex}
\newcommand{\generatetitle}[1]{\chapter{#1}}

\newcommand{\vcsinfo}{%
\section*{}
\noindent{\color{DarkOliveGreen}{\rule{\linewidth}{0.1mm}}}
\paragraph{Document version}
%\paragraph{\color{Maroon}{Document version}}
{
\small
\begin{itemize}
\item Available online at:\\ 
\href{\onlineurl}{\onlineurl}
\item Git Repository: \input{./.revinfo/gitRepo.tex}
\item Source: \sourcepath
\item last commit: \input{./.revinfo/gitCommitString.tex}
\item commit date: \input{./.revinfo/gitCommitDate.tex}
\end{itemize}
}
}

%\PassOptionsToPackage{dvipsnames,svgnames}{xcolor}
\PassOptionsToPackage{square,numbers}{natbib}
\documentclass{scrreprt}

\usepackage[left=2cm,right=2cm]{geometry}
\usepackage[svgnames]{xcolor}
\usepackage{peeters_layout}

\usepackage{natbib}

\usepackage[
colorlinks=true,
bookmarks=false,
pdfauthor={\authorname, \email},
backref 
]{hyperref}

% http://tex.stackexchange.com/questions/75773/how-to-reference-problems-by-the-text-label-in-an-exercise-envioronment
\usepackage[english]{cleveref}
\crefname{Exercise}{exercise}{exercises}
\Crefname{Exercise}{Exercise}{Exercises}

\RequirePackage{titlesec}
\RequirePackage{ifthen}

% http://stackoverflow.com/questions/4932910/date-in-the-tabular-environment
\makeatletter
\let\insertdate\@date
\makeatother

\titleformat{\chapter}[display]
{\bfseries\Large}
{\color{DarkSlateGrey}\filleft \authorname
\ifthenelse{\isundefined{\studentnumber}}{}{\\ \studentnumber}
\ifthenelse{\isundefined{\email}}{}{\\ \email}
\ifthenelse{\isundefined{\dateintitle}}{}{\\ \insertdate}
%\ifthenelse{\isundefined{\coursename}}{}{\\ \coursename} % put in title instead.
}
{4ex}
{\color{DarkOliveGreen}{\titlerule}\color{Maroon}
\vspace{2ex}%
\filright}
[\vspace{2ex}%
\color{DarkOliveGreen}\titlerule
]

\newcommand{\beginArtWithToc}[0]{\begin{document}\tableofcontents}
\newcommand{\beginArtNoToc}[0]{\begin{document}}
\newcommand{\EndNoBibArticle}[0]{\end{document}}
\newcommand{\EndArticle}[0]{\bibliography{Bibliography}\bibliographystyle{plainnat}\end{document}}

% 
%\newcommand{\citep}[1]{\cite{#1}}

\colorSectionsForArticle



\usepackage{macros_bm}

\beginArtNoToc

\generatetitle{Maxwell's equations in tensor form with magnetic sources}
%\chapter{Maxwell's equations in tensor form with magnetic sources}
%\label{chap:gaMagneticSourcesToTensorToVector}
%\section{Motivation}
%\section{Guts}

Following the principle of relating new formalisms to things previously learned, I'd like to know what Maxwell's equations look like in tensor form when magnetic sources are included.  As a verification that the previous Geometric Algebra form of Maxwell's equation that includes magnetic sources is correct, I'll start with the GA form of Maxwell's equation, find the tensor form, and then verify that the vector form of Maxwell's equations can be recovered from the tensor form.

\paragraph{Tensor form}

With four-vector potential \( A \), and bivector electromagnetic field \( F = \grad \wedge A \), the GA form of Maxwell's equation is

\begin{dmath}\label{eqn:gaMagneticSourcesToTensorToVector:20}
\grad F = \frac{J}{\epsilon_0 c} + M I.
\end{dmath}

The left hand side can be unpacked into vector and trivector terms \( \grad F = \grad \cdot F + \grad \wedge F \), which happens to also separate the sources nicely as a side effect

\begin{subequations}
\label{eqn:gaMagneticSourcesToTensorToVector:40}
\begin{dmath}\label{eqn:gaMagneticSourcesToTensorToVector:60}
\grad \cdot F = \frac{J}{\epsilon_0 c}
\end{dmath}
\begin{dmath}\label{eqn:gaMagneticSourcesToTensorToVector:80}
\grad \wedge F = M I.
\end{dmath}
\end{subequations}

The electric source equation can be unpacked into tensor form by dotting with the four vector basic vectors.  With the usual definition \( F^{\alpha \beta} = \partial^\alpha A^\beta - \partial^\beta A^\alpha \), that is

\begin{dmath}\label{eqn:gaMagneticSourcesToTensorToVector:100}
\gamma^\mu \cdot \lr{ \grad \cdot F } 
=
\gamma^\mu \cdot \lr{ \grad \cdot \lr{ \grad \wedge A } } 
=
\gamma^\mu \cdot \lr{ \gamma^\nu \partial_\nu \cdot 
\lr{ \gamma_\alpha \partial^\alpha \wedge \gamma_\beta A^\beta } }
=
\gamma^\mu \cdot \lr{ \gamma^\nu \cdot \lr{ \gamma_\alpha \wedge \gamma_\beta } } \partial_\nu \partial^\alpha A^\beta
=
\inv{2}
\gamma^\mu \cdot \lr{ \gamma^\nu \cdot \lr{ \gamma_\alpha \wedge \gamma_\beta } } 
\partial_\nu F^{\alpha \beta}
=
\inv{2} \delta^{\nu \mu}_{[\alpha \beta]} \partial_\nu F^{\alpha \beta}
=
\inv{2} \partial_\nu F^{\nu \mu}
-
\inv{2} \partial_\nu F^{\mu \nu}
=
\partial_\nu F^{\nu \mu}.
\end{dmath}

So the first tensor equation is

\boxedEquation{eqn:gaMagneticSourcesToTensorToVector:120}{
%\begin{dmath}\label{eqn:gaMagneticSourcesToTensorToVector:120}
%\boxed{
\partial_\nu F^{\nu \mu} = \inv{c \epsilon_0} J^\mu.
%\end{dmath}
}

To unpack the magnetic source portion of Maxwell's equation, put it first into dual form, so that it has four vectors on each side

\begin{dmath}\label{eqn:gaMagneticSourcesToTensorToVector:140}
M 
= -I \grad \wedge F
= -\frac{I}{2} \lr{ \grad F + F \grad }
= -\frac{1}{2} \lr{ -\grad I F + I F \grad }
= - \grad \cdot \lr{ I F }.
\end{dmath}

Dotting with \( \gamma^\mu \) gives

\begin{dmath}\label{eqn:gaMagneticSourcesToTensorToVector:160}
M^\mu 
= \gamma^\mu \cdot \lr{ \grad \cdot \lr{ -I F } }
= \gamma^\mu \cdot \lr{ \gamma^\nu \partial_\nu \cdot \lr{ -\frac{I}{2} \gamma^\alpha \wedge \gamma^\beta F_{\alpha \beta} } }
= -\inv{2} 
\gpgradezero{ 
\gamma^\mu \cdot \lr{ \gamma^\nu \cdot \lr{ I \gamma^\alpha \wedge \gamma^\beta } } 
}
\partial_\nu F_{\alpha \beta}.
\end{dmath}

That scalar grade selection is a complete antisymmetrization of the indexes

\begin{dmath}\label{eqn:gaMagneticSourcesToTensorToVector:180}
\gpgradezero{ 
\gamma^\mu \cdot \lr{ \gamma^\nu \cdot \lr{ I \gamma^\alpha \wedge \gamma^\beta } } 
}
=
\gpgradezero{ 
\gamma^\mu \cdot \lr{ \gamma^\nu \cdot \lr{ \gamma_0 \gamma_1 \gamma_2 \gamma_3 \gamma^\alpha \gamma^\beta } } 
}
=
\gpgradezero{ 
\gamma_0 \gamma_1 \gamma_2 \gamma_3 
\gamma^\mu \gamma^\nu \gamma^\alpha \gamma^\beta 
}
=
\delta^{\mu \nu \alpha \beta}_{3 2 1 0}
=
\epsilon^{\mu \nu \alpha \beta },
\end{dmath}

so the magnetic source portion of Maxwell's equation, in tensor form, is

\boxedEquation{eqn:gaMagneticSourcesToTensorToVector:200}{
%\begin{dmath}\label{eqn:gaMagneticSourcesToTensorToVector:200}
%\boxed{
\inv{2} \epsilon^{\nu \alpha \beta \mu}
\partial_\nu F_{\alpha \beta}
=
M^\mu.
}
%\end{dmath}

\paragraph{Relating the tensor to the fields}

The electromagnetic field has been identified with the electric and magnetic fields by

\begin{dmath}\label{eqn:gaMagneticSourcesToTensorToVector:220}
F = \bcE + I c \mu_0 \bcH,
\end{dmath}

or in coordinates

\begin{dmath}\label{eqn:gaMagneticSourcesToTensorToVector:240}
\inv{2} \gamma_\mu \wedge \gamma_\nu F^{\mu \nu} 
=  E^a \gamma_a \gamma_0 + I c \mu_0 H^a \gamma_a \gamma_0.
\end{dmath}

By forming the dot product sequence \( F^{\alpha \beta} = \gamma^\beta \cdot \lr{ \gamma^\alpha \cdot F } \), the electric and magnetic field components can be related to the tensor components.  The electric field components follow by inspection and are

\begin{dmath}\label{eqn:gaMagneticSourcesToTensorToVector:260}
E^b = \gamma^0 \cdot \lr{ \gamma^b \cdot F } = F^{b 0}.
\end{dmath}

The magnetic field relation to the tensor components follow from

\begin{dmath}\label{eqn:gaMagneticSourcesToTensorToVector:280}
F^{r s} 
= F_{r s} 
= \gamma_s \cdot \lr{ \gamma_r \cdot \lr{ I c \mu_0 H^a \gamma_a \gamma_0 } }
= 
c \mu_0 H^a \gpgradezero{ I \gamma_s \gamma_r \gamma_a \gamma_0 }
= 
c \mu_0 H^a \gpgradezero{ -\cancel{\gamma^0} \gamma^1 \gamma^2 \gamma^3 \gamma_s \gamma_r \gamma_a \cancel{\gamma_0} }
=
- c \mu_0 H^a \delta^{[3 2 1]}_{s r a}
= 
c \mu_0 H^a \epsilon_{ s r a }.
\end{dmath}

Expanding this for each pair of spacelike coordinates gives

\begin{subequations}
\label{eqn:gaMagneticSourcesToTensorToVector:300}
\begin{equation}\label{eqn:gaMagneticSourcesToTensorToVector:320}
F^{1 2} = c \mu_0 H^3 \epsilon_{ 2 1 3 } = - c \mu_0 H^3
\end{equation}
\begin{equation}\label{eqn:gaMagneticSourcesToTensorToVector:340}
F^{2 3} = c \mu_0 H^1 \epsilon_{ 3 2 1 } = - c \mu_0 H^1
\end{equation}
\begin{equation}\label{eqn:gaMagneticSourcesToTensorToVector:360}
F^{3 1} = c \mu_0 H^2 \epsilon_{ 1 3 2 } = - c \mu_0 H^2,
\end{equation}
\end{subequations}

or

%\begin{equation}\label{eqn:gaMagneticSourcesToTensorToVector:380}
%\boxed{
\boxedEquation{eqn:gaMagneticSourcesToTensorToVector:380}{
\begin{aligned}
E^1 &= F^{1 0} \\
E^2 &= F^{2 0} \\
E^3 &= F^{3 0} \\
H^1 &= -\inv{c \mu_0} F^{2 3} \\
H^2 &= -\inv{c \mu_0} F^{3 1} \\
H^3 &= -\inv{c \mu_0} F^{1 2}.
\end{aligned}
}
%\end{equation}

\paragraph{Recover the vector equations from the tensor equations}

... when done.

This takes things full circle, going from the vector differential Maxwell's equations, to the Geometric Algebra form of Maxwell's equation, to Maxwell's equations in tensor form, and back to the vector form.

%\EndArticle
\EndNoBibArticle
