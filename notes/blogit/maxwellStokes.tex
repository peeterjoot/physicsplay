%
% Copyright � 2016 Peeter Joot.  All Rights Reserved.
% Licenced as described in the file LICENSE under the root directory of this GIT repository.
%
%{
\newcommand{\authorname}{Peeter Joot}
\newcommand{\email}{peeterjoot@protonmail.com}
\newcommand{\basename}{FIXMEbasenameUndefined}
\newcommand{\dirname}{notes/FIXMEdirnameUndefined/}

\renewcommand{\basename}{maxwellStokes}
\renewcommand{\dirname}{notes/phy1520/}
%\newcommand{\dateintitle}{}
%\newcommand{\keywords}{}

\newcommand{\authorname}{Peeter Joot}
\newcommand{\onlineurl}{http://sites.google.com/site/peeterjoot2/math2013/\basename.pdf}
\newcommand{\sourcepath}{\dirname\basename.tex}
\newcommand{\generatetitle}[1]{\chapter{#1}}

\newcommand{\vcsinfo}{%
\section*{}
\noindent{\color{DarkOliveGreen}{\rule{\linewidth}{0.1mm}}}
\paragraph{Document version}
%\paragraph{\color{Maroon}{Document version}}
{
\small
\begin{itemize}
\item Available online at:\\ 
\href{\onlineurl}{\onlineurl}
\item Git Repository: \input{./.revinfo/gitRepo.tex}
\item Source: \sourcepath
\item last commit: \input{./.revinfo/gitCommitString.tex}
\item commit date: \input{./.revinfo/gitCommitDate.tex}
\end{itemize}
}
}

%\PassOptionsToPackage{dvipsnames,svgnames}{xcolor}
\PassOptionsToPackage{square,numbers}{natbib}
\documentclass{scrreprt}

\usepackage[left=2cm,right=2cm]{geometry}
\usepackage[svgnames]{xcolor}
\usepackage{peeters_layout}

\usepackage{natbib}

\usepackage[
colorlinks=true,
bookmarks=false,
pdfauthor={\authorname, \email},
backref 
]{hyperref}

% http://tex.stackexchange.com/questions/75773/how-to-reference-problems-by-the-text-label-in-an-exercise-envioronment
\usepackage[english]{cleveref}
\crefname{Exercise}{exercise}{exercises}
\Crefname{Exercise}{Exercise}{Exercises}

\RequirePackage{titlesec}
\RequirePackage{ifthen}

% http://stackoverflow.com/questions/4932910/date-in-the-tabular-environment
\makeatletter
\let\insertdate\@date
\makeatother

\titleformat{\chapter}[display]
{\bfseries\Large}
{\color{DarkSlateGrey}\filleft \authorname
\ifthenelse{\isundefined{\studentnumber}}{}{\\ \studentnumber}
\ifthenelse{\isundefined{\email}}{}{\\ \email}
\ifthenelse{\isundefined{\dateintitle}}{}{\\ \insertdate}
%\ifthenelse{\isundefined{\coursename}}{}{\\ \coursename} % put in title instead.
}
{4ex}
{\color{DarkOliveGreen}{\titlerule}\color{Maroon}
\vspace{2ex}%
\filright}
[\vspace{2ex}%
\color{DarkOliveGreen}\titlerule
]

\newcommand{\beginArtWithToc}[0]{\begin{document}\tableofcontents}
\newcommand{\beginArtNoToc}[0]{\begin{document}}
\newcommand{\EndNoBibArticle}[0]{\end{document}}
\newcommand{\EndArticle}[0]{\bibliography{Bibliography}\bibliographystyle{plainnat}\end{document}}

% 
%\newcommand{\citep}[1]{\cite{#1}}

\colorSectionsForArticle



\usepackage{peeters_layout_exercise}
\usepackage{peeters_braket}
\usepackage{peeters_figures}
\usepackage{siunitx}

\beginArtNoToc

\generatetitle{Application of Stokes Theorem to Maxwell equation}
%\chapter{Application of Stokes Theorem to Maxwell equation}
%\label{chap:maxwellStokes}
% \citep{sakurai2014modern} pr X.Y
% \citep{pozar2009microwave}
% \citep{qftLectureNotes}
% \citep{griffiths1999introduction}

The relativistic form of Maxwell's equation in Geometric Algebra is

\begin{dmath}\label{eqn:maxwellStokes:20}
\grad F = \inv{c \epsilon_0} J,
\end{dmath}

where \( \grad = \gamma^\mu \partial_\mu \) is the spacetime gradient, \( F = \BE + I c \BB \) is the electromagnetic field bivector, and \( J = (c\rho, \BJ) \) is the four (vector) current density.

A dual representation, with \( F = I G \) is also possible

\begin{dmath}\label{eqn:maxwellStokes:60}
\grad G = \frac{I}{c \epsilon_0} J.
\end{dmath}

Either form of Maxwell's equation can be split into grade one and three components.  The standard (non-dual) form is

\begin{dmath}\label{eqn:maxwellStokes:40}
\begin{aligned}
\grad \cdot F &= \inv{c \epsilon_0} J \\
\grad \wedge F &= 0,
\end{aligned}
\end{dmath}

and the dual form is 

\begin{dmath}\label{eqn:maxwellStokes:41}
\begin{aligned}
\grad \cdot G &= 0 \\
\grad \wedge G &= \frac{I}{c \epsilon_0} J.
\end{aligned}
\end{dmath}

In both cases a potential representation \( F = \grad \wedge A \), where \( A \) is a four vector potential can be used to kill off the non-current equation.  Such a potential representation reduces Maxwell's equation to

\begin{dmath}\label{eqn:maxwellStokes:80}
\grad \cdot F = \inv{c \epsilon_0} J,
\end{dmath}

or 
\begin{dmath}\label{eqn:maxwellStokes:100}
\grad \wedge G = \frac{I}{c \epsilon_0} J.
\end{dmath}

In both cases, these reduce to 
\begin{dmath}\label{eqn:maxwellStokes:120}
\grad^2 A - \grad \lr{ \grad \cdot A } = \inv{c \epsilon_0} J.
\end{dmath}

This can clearly be further simplified by using the Lorentz gauge, where \( \grad \cdot A = 0 \).  However, the aim for now is to try applying Stokes theorem to Maxwell's equation.  The dual form \eqnref{eqn:maxwellStokes:100} has the curl structure required for the application of Stokes.  Suppose that we evaluate this curl over the three parameter volume element \( d^3 x = i\, dx^0 dx^1 dx^2 \), where \( i = \gamma_0 \gamma_1 \gamma_2 \) is the unit pseudoscalar for the spacetime volume element.

\begin{dmath}\label{eqn:maxwellStokes:101}
\int_V d^3 x \cdot \lr{ \grad \wedge G }
=
\int_V d^3 x \cdot \lr{ \gamma^\mu \wedge \partial_\mu G }
=
\int_V \lr{ d^3 x \cdot \gamma^\mu } \cdot \partial_\mu G 
=
\sum_{\mu \ne 3} \int_V \lr{ d^3 x \cdot \gamma^\mu } \cdot \partial_\mu G.
\end{dmath}

This uses the distibution identity \( A_s \cdot (a \wedge A_r) = (A_s \cdot a) \cdot A_r \) which holds for blades \( A_s, A_r \) provided \( s > r > 0 \).  Observe that only the component of the gradient that lies in the tangent space of the three volume manifold contributes to the integral, allowing the gradient to be used in the Stokes integral instead of the vector derivative (see: \citep{aMacdonaldVAGC}).
Defining the the surface area element

\begin{dmath}\label{eqn:maxwellStokes:140}
d^2 x 
= \sum_{\mu \ne 3} i \cdot \gamma^\mu \inv{dx^\mu} d^3 x
= \gamma_1 \gamma_2 dx dy
+ c \gamma_2 \gamma_0 dt dy
+ c \gamma_0 \gamma_1 dt dx,
\end{dmath}

Stokes theorem for this volume element is now completely specified

\begin{dmath}\label{eqn:maxwellStokes:200}
\int_V d^3 x \cdot \lr{ \grad \wedge G }
=
\int_{\partial V} d^2 \cdot G.
\end{dmath}

Application to the dual Maxwell equation gives

\begin{dmath}\label{eqn:maxwellStokes:160}
\int_{\partial V} d^2 x \cdot G 
= \inv{c \epsilon_0} \int_V d^3 x \cdot (I J).
\end{dmath}

After some manipulation, this can be restated in the non-dual form

%\begin{dmath}\label{eqn:maxwellStokes:180}
\boxedEquation{eqn:maxwellStokes:180}{
\int_{\partial V} \inv{I} d^2 x \wedge F 
= \frac{1}{c \epsilon_0 I} \int_V d^3 x \wedge J.
}
%\end{dmath}

It can be demonstrated that using this with each of the standard basis spacetime 3-volume elements recovers Gauss's law and the Ampere-Maxwell equation.  So, what happened to Faraday's law and Gauss's law for magnetism?  With application of Stokes to the curl equation from \eqnref{eqn:maxwellStokes:40}, those equations take the form

%\begin{dmath}\label{eqn:maxwellStokes:240}
\boxedEquation{eqn:maxwellStokes:240}{
\int_{\partial V} d^2 x \cdot F = 0.
}
%\end{dmath}

FIXME: problem/solution to show: \eqnref{eqn:maxwellStokes:180}.

\makeproblem{}{problem:maxwellStokes:1}{
Using each of the four possible spacetime volume elements, take the volume element to its infinesimal limit, and recover the traditional differential forms of Maxwell's equations.
} % problem

\makeanswer{problem:maxwellStokes:1}{
The four volume and associated area elements are
\begin{dmath}\label{eqn:maxwellStokes:220}
\begin{aligned}
d^3 x = c \gamma_0 \gamma_1 \gamma_2 dt dx dy & \qquad d^2 x = \gamma_1 \gamma_2 dx dy + c \gamma_2 \gamma_0 dy dt + c \gamma_0 \gamma_1 dt dx \\
d^3 x = c \gamma_0 \gamma_1 \gamma_3 dt dx dy & \qquad d^2 x = \gamma_1 \gamma_3 dx dy + c \gamma_3 \gamma_0 dy dt + c \gamma_0 \gamma_1 dt dx \\
d^3 x = c \gamma_0 \gamma_2 \gamma_3 dt dx dy & \qquad d^2 x = \gamma_2 \gamma_3 dx dy + c \gamma_3 \gamma_0 dy dt + c \gamma_0 \gamma_2 dt dx \\
d^3 x = \gamma_1 \gamma_2 \gamma_3 dx dy dz & \qquad d^2 x = \gamma_1 \gamma_2 dx dy + \gamma_2 \gamma_3 dy dz + c \gamma_3 \gamma_1 dz dx \\
\end{aligned}
\end{dmath}

Wedging the area element with \( F \) will produce pseudoscalar multiples of the various \( \BE \) and \( \BB \) components, but a recipe for these components is required.
} % answer

%}
\EndArticle
%\EndNoBibArticle
