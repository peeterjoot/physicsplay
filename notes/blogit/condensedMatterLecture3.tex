%
% Copyright � 2013 Peeter Joot.  All Rights Reserved.
% Licenced as described in the file LICENSE under the root directory of this GIT repository.
%
\newcommand{\authorname}{Peeter Joot}
\newcommand{\email}{peeterjoot@protonmail.com}
\newcommand{\basename}{FIXMEbasenameUndefined}
\newcommand{\dirname}{notes/FIXMEdirnameUndefined/}

\renewcommand{\basename}{condensedMatterLecture3}
\renewcommand{\dirname}{notes/phy487/}
\newcommand{\keywords}{Condensed matter physics, PHY487H1F}
\newcommand{\authorname}{Peeter Joot}
\newcommand{\onlineurl}{http://sites.google.com/site/peeterjoot2/math2013/\basename.pdf}
\newcommand{\sourcepath}{\dirname\basename.tex}
\newcommand{\generatetitle}[1]{\chapter{#1}}

\newcommand{\vcsinfo}{%
\section*{}
\noindent{\color{DarkOliveGreen}{\rule{\linewidth}{0.1mm}}}
\paragraph{Document version}
%\paragraph{\color{Maroon}{Document version}}
{
\small
\begin{itemize}
\item Available online at:\\ 
\href{\onlineurl}{\onlineurl}
\item Git Repository: \input{./.revinfo/gitRepo.tex}
\item Source: \sourcepath
\item last commit: \input{./.revinfo/gitCommitString.tex}
\item commit date: \input{./.revinfo/gitCommitDate.tex}
\end{itemize}
}
}

%\PassOptionsToPackage{dvipsnames,svgnames}{xcolor}
\PassOptionsToPackage{square,numbers}{natbib}
\documentclass{scrreprt}

\usepackage[left=2cm,right=2cm]{geometry}
\usepackage[svgnames]{xcolor}
\usepackage{peeters_layout}

\usepackage{natbib}

\usepackage[
colorlinks=true,
bookmarks=false,
pdfauthor={\authorname, \email},
backref 
]{hyperref}

% http://tex.stackexchange.com/questions/75773/how-to-reference-problems-by-the-text-label-in-an-exercise-envioronment
\usepackage[english]{cleveref}
\crefname{Exercise}{exercise}{exercises}
\Crefname{Exercise}{Exercise}{Exercises}

\RequirePackage{titlesec}
\RequirePackage{ifthen}

% http://stackoverflow.com/questions/4932910/date-in-the-tabular-environment
\makeatletter
\let\insertdate\@date
\makeatother

\titleformat{\chapter}[display]
{\bfseries\Large}
{\color{DarkSlateGrey}\filleft \authorname
\ifthenelse{\isundefined{\studentnumber}}{}{\\ \studentnumber}
\ifthenelse{\isundefined{\email}}{}{\\ \email}
\ifthenelse{\isundefined{\dateintitle}}{}{\\ \insertdate}
%\ifthenelse{\isundefined{\coursename}}{}{\\ \coursename} % put in title instead.
}
{4ex}
{\color{DarkOliveGreen}{\titlerule}\color{Maroon}
\vspace{2ex}%
\filright}
[\vspace{2ex}%
\color{DarkOliveGreen}\titlerule
]

\newcommand{\beginArtWithToc}[0]{\begin{document}\tableofcontents}
\newcommand{\beginArtNoToc}[0]{\begin{document}}
\newcommand{\EndNoBibArticle}[0]{\end{document}}
\newcommand{\EndArticle}[0]{\bibliography{Bibliography}\bibliographystyle{plainnat}\end{document}}

% 
%\newcommand{\citep}[1]{\cite{#1}}

\colorSectionsForArticle



%\citep{harald2003solid} \S x.y
%\citep{ibach2009solid} \S x.y

\usepackage{mhchem}

\beginArtNoToc
\generatetitle{PHY487H1F Condensed Matter Physics.  Lecture 3: XXX.  Taught by Prof.\ Stephen Julian}
%\chapter{XXX}
\label{chap:condensedMatterLecture3}

\section{Disclaimer}

Peeter's lecture notes from class.  May not be entirely coherent.

\section{Bonding (cont.)}

\subsection{Ionic bonding}

We introduce the \underlineAndIndex{Madelung constant} for the potential energy of the solid configuration

\begin{dmath}\label{eqn:condensedMatterLecture3:20}
\Phi_{tot} = \sum_i \phi_i = \inv{2} \sum_{i \ne j} \phi_{ij}
= \inv{2} 
\mathLabelBox
[
   labelstyle={xshift=-2cm},
   linestyle={out=270,in=90, latex-}
]
{N}{number of ions in solid}
\lr{
-\frac{e^2}{4 \pi \epsilon_0 r} 
\mathLabelBox
[
   labelstyle={xshift=5cm},
   linestyle={out=270,in=90, latex-}
]
{
\sum_{i \ne j} \frac{\lr{\pm 1}}{ p_{ij} }
}{Madelung constant}
+ \frac{B}{r^n}
\sum_{i \ne j} \inv{p_{ij}^n}
}
\end{dmath}

Here $r$ is the nn separation (center to center), and $r_{ij} = p_{ij} r$

F1
\cref{fig:qmSolidsL3:qmSolidsL3Fig1}.
\imageFigure{qmSolidsL3Fig1}{CAPTION}{fig:qmSolidsL3:qmSolidsL3Fig1}{0.3}

F2
\cref{fig:qmSolidsL3:qmSolidsL3Fig2}.
\imageFigure{qmSolidsL3Fig2}{CAPTION}{fig:qmSolidsL3:qmSolidsL3Fig2}{0.3}

%STRONG HINT: LIKELY ON EXAM (since it can be calculated)

Examples
\begin{itemize}
\item \ce{NaCl} structure $A = 1.748$
\item \ce{CsCl} structure $A = 1.763$
\end{itemize}

Ionic bonds are weaker than covalent, non-directional.  One indicator of this is the melting points

\begin{subequations}
\begin{dmath}\label{eqn:condensedMatterLecture3:40}
T_m = 1074 K \ce{NaCl}
\end{dmath}
\begin{dmath}\label{eqn:condensedMatterLecture3:60}
T_m = 918 K \ce{CsCl}
\end{dmath}
\end{subequations}

\subsection{Metallic bonding}

Metallic bonding regions in the periodic table

F3
\cref{fig:qmSolidsL3:qmSolidsL3Fig3}.
\imageFigure{qmSolidsL3Fig3}{CAPTION}{fig:qmSolidsL3:qmSolidsL3Fig3}{0.3}

Note: just because something is a metal doesn't mean it is metallically bonded.

\begin{itemize}
\item s-orbitals from $2s to 5s$
\item p-orbitals from $n = 4, 5, 6, 7$
\end{itemize}

These are big orbitals that extend beyond the nn, as illustrated in \cref{fig:qmSolidsL3:qmSolidsL3Fig4}.

\imageFigure{qmSolidsL3Fig4}{Extensive wave function}{fig:qmSolidsL3:qmSolidsL3Fig4}{0.3}

Weakly bound electrons overlap many nearby potential wells.  This lowers the Coloumb energy.  This is like a non-directional covalent bond.  This non-directionality results in malleability.

Pure metallic bonds are weak.  Melting points are corresondingly low, where for column 1 elements we have $T_m = \mbox{room temperature to $200$ C}$.
FIXME: degree symbol

\subsection{Transition metals}

Here we have both metallically bonded s-orbitals and covalent d-orbitals.

\cref{fig:qmSolidsL3:qmSolidsL3Fig5}.
\imageFigure{qmSolidsL3Fig5}{CAPTION}{fig:qmSolidsL3:qmSolidsL3Fig5}{0.3}

\begin{dmath}\label{eqn:condensedMatterLecture3:80}
T_m \approx \ge 2000 C.
\end{dmath}

F6
\cref{fig:qmSolidsL3:qmSolidsL3Fig6}.
\imageFigure{qmSolidsL3Fig6}{CAPTION}{fig:qmSolidsL3:qmSolidsL3Fig6}{0.3}

A clarification of the sign conventions.  The $+, -$'s assume real representation of wave functions.  Here bonding is matching signs (constructive interference), and antibonding is when the signs are in opposition (destructive interference).

Reading: \S 1.5, 1.6 \citep{ibach2009solid}.

\section{Crystal structures}

Reading: \S 2.0 \citep{ibach2009solid}.

Most solids prefer a periodic arrangement of their atoms.  This is due to directional bonding, and is easy to see in some cases (such as our diamond tetrahedral pattern)

F7
\cref{fig:qmSolidsL3:qmSolidsL3Fig7}.
\imageFigure{qmSolidsL3Fig7}{CAPTION}{fig:qmSolidsL3:qmSolidsL3Fig7}{0.3}

Unproven theorem of no name: lowest energy configuration of atoms in a solid is periodic.

Minimum Coloumb energy for integer ratios of atoms
\ce{Na1Cl1},
\ce{Fe2O3},
\ce{Fe3O4},
\ce{PrOs4Sb12},
\ce{YbCO2Zn20}.

There's a lot of info on ``amorphous/glassy'' materials are \underline{not} periodic.  We won't consider these.

\subsection{Mathematical description of periodicity}

Starting with a 2D lattice.  Two vectors can generate a lattice (or 2 lengths and 1 angle)

F8
\cref{fig:qmSolidsL3:qmSolidsL3Fig8}.
\imageFigure{qmSolidsL3Fig8}{CAPTION}{fig:qmSolidsL3:qmSolidsL3Fig8}{0.3}

We'll assign each atomic center a vector

\begin{dmath}\label{eqn:condensedMatterLecture3:100}
\Br_\Bn = n_1 \Ba + n_2 \Bb
\end{dmath}

Only five cases (in 2D) that are symmetrically distinct that leave no spaces

F9  square lattice.  $a = b$, $\gamma = \pi/2$.
\cref{fig:qmSolidsL3:qmSolidsL3Fig9}.
\imageFigure{qmSolidsL3Fig9}{CAPTION}{fig:qmSolidsL3:qmSolidsL3Fig9}{0.3}

F10 rectangular.  $a \ne b$, $\gamma = \pi/2$.
\cref{fig:qmSolidsL3:qmSolidsL3Fig10}.
\imageFigure{qmSolidsL3Fig10}{CAPTION}{fig:qmSolidsL3:qmSolidsL3Fig10}{0.3}

F11 hexagonal close packed
\cref{fig:qmSolidsL3:qmSolidsL3Fig11}.
\imageFigure{qmSolidsL3Fig11}{CAPTION}{fig:qmSolidsL3:qmSolidsL3Fig11}{0.3}

F12 rhombic, or centered rectangular.  $a = b$, $\gamma \ne \pi/2, \pi/3$.
\cref{fig:qmSolidsL3:qmSolidsL3Fig12}.
\imageFigure{qmSolidsL3Fig12}{CAPTION}{fig:qmSolidsL3:qmSolidsL3Fig12}{0.3}

F13: No figure.  $a \ne b$, $\gamma = \pi/2$.

\EndArticle
