%
% Copyright � 2015 Peeter Joot.  All Rights Reserved.
% Licenced as described in the file LICENSE under the root directory of this GIT repository.
%
\newcommand{\authorname}{Peeter Joot}
\newcommand{\email}{peeterjoot@protonmail.com}
\newcommand{\basename}{FIXMEbasenameUndefined}
\newcommand{\dirname}{notes/FIXMEdirnameUndefined/}

\renewcommand{\basename}{chapter3Notes}
\renewcommand{\dirname}{notes/ece1520/}
\newcommand{\keywords}{PHY1520H}
\newcommand{\authorname}{Peeter Joot}
\newcommand{\onlineurl}{http://sites.google.com/site/peeterjoot2/math2013/\basename.pdf}
\newcommand{\sourcepath}{\dirname\basename.tex}
\newcommand{\generatetitle}[1]{\chapter{#1}}

\newcommand{\vcsinfo}{%
\section*{}
\noindent{\color{DarkOliveGreen}{\rule{\linewidth}{0.1mm}}}
\paragraph{Document version}
%\paragraph{\color{Maroon}{Document version}}
{
\small
\begin{itemize}
\item Available online at:\\ 
\href{\onlineurl}{\onlineurl}
\item Git Repository: \input{./.revinfo/gitRepo.tex}
\item Source: \sourcepath
\item last commit: \input{./.revinfo/gitCommitString.tex}
\item commit date: \input{./.revinfo/gitCommitDate.tex}
\end{itemize}
}
}

%\PassOptionsToPackage{dvipsnames,svgnames}{xcolor}
\PassOptionsToPackage{square,numbers}{natbib}
\documentclass{scrreprt}

\usepackage[left=2cm,right=2cm]{geometry}
\usepackage[svgnames]{xcolor}
\usepackage{peeters_layout}

\usepackage{natbib}

\usepackage[
colorlinks=true,
bookmarks=false,
pdfauthor={\authorname, \email},
backref 
]{hyperref}

% http://tex.stackexchange.com/questions/75773/how-to-reference-problems-by-the-text-label-in-an-exercise-envioronment
\usepackage[english]{cleveref}
\crefname{Exercise}{exercise}{exercises}
\Crefname{Exercise}{Exercise}{Exercises}

\RequirePackage{titlesec}
\RequirePackage{ifthen}

% http://stackoverflow.com/questions/4932910/date-in-the-tabular-environment
\makeatletter
\let\insertdate\@date
\makeatother

\titleformat{\chapter}[display]
{\bfseries\Large}
{\color{DarkSlateGrey}\filleft \authorname
\ifthenelse{\isundefined{\studentnumber}}{}{\\ \studentnumber}
\ifthenelse{\isundefined{\email}}{}{\\ \email}
\ifthenelse{\isundefined{\dateintitle}}{}{\\ \insertdate}
%\ifthenelse{\isundefined{\coursename}}{}{\\ \coursename} % put in title instead.
}
{4ex}
{\color{DarkOliveGreen}{\titlerule}\color{Maroon}
\vspace{2ex}%
\filright}
[\vspace{2ex}%
\color{DarkOliveGreen}\titlerule
]

\newcommand{\beginArtWithToc}[0]{\begin{document}\tableofcontents}
\newcommand{\beginArtNoToc}[0]{\begin{document}}
\newcommand{\EndNoBibArticle}[0]{\end{document}}
\newcommand{\EndArticle}[0]{\bibliography{Bibliography}\bibliographystyle{plainnat}\end{document}}

% 
%\newcommand{\citep}[1]{\cite{#1}}

\colorSectionsForArticle



%\usepackage{phy1520}
\usepackage{peeters_braket}
\usepackage{peeters_layout_exercise}

\beginArtNoToc
%\generatetitle{PHY1520H Graduate Quantum Mechanics.  Lecture 3: Density matrix (cont.).  Taught by Prof.\ Arun Paramekanti}
%\generatetitle{Density matrix (cont.)}
%\chapter{Density matrix (cont.)}
\label{chap:chapter3Notes}

\paragraph{Disclaimer}

Peeter's lecture notes from class.  These may be incoherent and rough.

These are notes for the UofT course PHY1520, Graduate Quantum Mechanics, taught by Prof. Paramekanti, covering \textchapref{{1}} \citep{sakurai2014modern} content.

\paragraph{Density matrix (cont.)}

Example of partitioned system with four total states (two spin 1/2 particles)

F1

Example of partitioned system with eight total states (three spin 1/2 particles)

F2

The density matrix 

\begin{dmath}\label{eqn:qmLecture3:20}
\hat{\rho} = \ket{\Psi}\bra{\Psi}
\end{dmath}

is clearly an operator as can be seen by applying it to a state

\begin{dmath}\label{eqn:qmLecture3:40}
\hat{\rho} \ket{\phi} = \ket{\Psi} \lr{ \braket{ \Psi }{\phi} }.
\end{dmath}

The quanitity in braces is just a complex number.

After expanding the pure state \( \ket{\Psi} \) in terms of basis states for each of the two partitions

\begin{dmath}\label{eqn:qmLecture3:60}
\ket{\Psi} = \sum_{m,n} C_{m, n} \ket{m}_\txtL \ket{n}_\txtR,
\end{dmath}

we have

\begin{dmath}\label{eqn:qmLecture3:80}
\hat{\rho} = 
\sum_{m,n} 
C_{m, n} 
C_{m', n'}^\conj
\ket{m}_\txtL \ket{n}_\txtR
\sum_{m',n'}
\bra{m'}_\txtL \bra{n'}_\txtR.
\end{dmath}

Suppose we trace over the right partition of the state space

\begin{dmath}\label{eqn:qmLecture3:100}
\Tr(\hat{\rho})
= \sum_{\tilde{n}} \bra{\tilde{n}}_\txtR \hat{\rho} \ket{ \tilde{n} }_\txtR
= \sum_{\tilde{n}} 
\sum_{m,n}
\sum_{m',n'}
C_{m, n} 
C_{m', n'}^\conj
\bra{\tilde{n}}_\txtR 
\ket{m}_\txtL \ket{n}_\txtR
\bra{m'}_\txtL \bra{n'}_\txtR
\ket{ \tilde{n} }_\txtR
= \sum_{\tilde{n}} 
\sum_{m,n}
\sum_{m',n'}
C_{m, n} 
C_{m', n'}^\conj
\ket{m}_\txtL \delta_{n n'}
\bra{m'}_\txtL 
\delta_{ \tilde{n} n' }
=
...
\end{dmath}
FIXME: complete.
FIXME: somewhere in here the reduced density matrix was introduced.

\begin{dmath}\label{eqn:qmLecture3:120}
\bra{\tilde{m}} \hat{\rho} \ket{\tilde{m}}
= 
\sum_{m, m', \tilde{n}} C_{m, \tilde{n}} C_{m', \tilde{n}}^\conj \braket{ \tilde{m}}{m}_\txtL \braket{m'}{\tilde{m}}_\txtL
= 
\sum_{\tilde{n}} \Abs{C_{\tilde{m}, \tilde{n}} }^2.
\end{dmath}

This is the probability that the left partition is in state \( \tilde{m} \).

\paragraph{Average of an observable}

Suppose we have two spin half particles.  For such a system the total magnetization is

\begin{dmath}\label{eqn:qmLecture3:140}
S_{\textrm{Total}} = 
S_1^z
+
S_1^z,
\end{dmath}

as sketched in

F3

The average of some observable is 

\begin{dmath}\label{eqn:qmLecture3:160}
\expectation{\hatA}
= \sum_{m, n, m', n'} C_{m, n}^\conj C_{m', n'}
\bra{m}\bra{n} \hatA \ket{n'} \ket{m'}.
\end{dmath}

Consider the trace of the density operator observable product

\begin{dmath}\label{eqn:qmLecture3:180}
\Tr( \hat{\rho} \hatA )
= \sum_{m, n} \braket{m n}{\Psi} \bra{\Psi} \hatA \ket{m, n}.
\end{dmath}

Let

\begin{dmath}\label{eqn:qmLecture3:200}
\ket{\Psi} = \sum_{m, n} C_{m n} \ket{m, n},
\end{dmath}

so that

\begin{dmath}\label{eqn:qmLecture3:220}
\Tr( \hat{\rho} \hatA ) 
= \sum_{m, n, m', n', m'', n''} C_{m', n'} C_{m'', n''}^\conj
\braket{m n}{m', n'} \bra{m'', n''} \hatA \ket{m, n}.
= \sum_{m, n, m'', n''} C_{m, n} C_{m'', n''}^\conj
\bra{m'', n''} \hatA \ket{m, n}.
\end{dmath}

This is just

%\begin{dmath}\label{eqn:qmLecture3:240}
\boxedEquation{eqn:qmLecture3:240}{
\bra{\Psi} \hatA \ket{\Psi} = \Tr( \hat{\rho} \hatA ).
}
%\end{dmath}

\paragraph{Left observables}

Consider 

\begin{dmath}\label{eqn:qmLecture3:260}
\bra{\Psi} \hatA_\txtL \ket{\Psi}
= \Tr(\hat{\rho} \hatA_\txtL)
= 
\Tr_\txtL
\Tr_\txtR
(\hat{\rho} \hatA_\txtL)
= 
\Tr_\txtL
\lr{
\lr{
\Tr_\txtR \hat{\rho} 
}
\hatA_\txtL)
}
= 
\Tr_\txtL
\lr{
\hat{\rho}_{\textrm{red}}
\hatA_\txtL)
}.
\end{dmath}

We see

\begin{dmath}\label{eqn:qmLecture3:280}
\bra{\Psi} \hatA_\txtL \ket{\Psi}
= 
\Tr_\txtL \lr{ \hat{\rho}_{\textrm{red}, \txtL} \hatA_\txtL }.
\end{dmath}

We find that we don't need to know the state of the complete system to answer questions about portions of the system, but instead just need \( \hat{\rho} \), a ``probability operator'' that provides all the required information about the partitioning of the system.

\paragraph{Pure states vs. mixed states}

For pure states we can assign a state vector and talk about reduced scenerios.  For mixed states we must work with reduced density matrices.

\makeexample{Two particle spin half pure states}{example:qmLecture3:1}{

Consider 

\begin{dmath}\label{eqn:qmLecture3:300}
\ket{\psi_1} = \inv{\sqrt{2}} \lr{ \ket{ \uparrow \downarrow } - \ket{ \downarrow \uparrow } }
\end{dmath}

\begin{dmath}\label{eqn:qmLecture3:320}
\ket{\psi_2} = \inv{\sqrt{2}} \lr{ \ket{ \uparrow \downarrow } + \ket{ \uparrow \uparrow } }.
\end{dmath}

For the first pure state the density operator is
\begin{dmath}\label{eqn:qmLecture3:360}
\hat{\rho} = \inv{2} 
\lr{ \ket{ \uparrow \downarrow } - \ket{ \downarrow \uparrow } }
\lr{ \bra{ \uparrow \downarrow } - \bra{ \downarrow \uparrow } }
\end{dmath}

What are the reduced density matrices?

\begin{dmath}\label{eqn:qmLecture3:340}
\hat{\rho}_\txtL
= \Tr_\txtR \lr{ \hat{\rho} }
= 
\inv{2} (-1)(-1) \ket{\downarrow}\bra{\downarrow} 
+\inv{2} (+1)(+1) \ket{\uparrow}\bra{\uparrow},
\end{dmath}

so the matrix representation of this reduced density operator is

\begin{dmath}\label{eqn:qmLecture3:380}
\hat{\rho}_\txtL 
=
\inv{2}
\begin{bmatrix}
1 & 0 \\
0 & 1
\end{bmatrix}.
\end{dmath}

For the second pure state the density operator is
\begin{dmath}\label{eqn:qmLecture3:400}
\hat{\rho} = \inv{2} 
\lr{ \ket{ \uparrow \downarrow } + \ket{ \uparrow \uparrow } }
\lr{ \bra{ \uparrow \downarrow } + \bra{ \uparrow \uparrow } }.
\end{dmath}

This has a reduced density matrice

\begin{dmath}\label{eqn:qmLecture3:420}
\hat{\rho}_\txtL
= \Tr_\txtR \lr{ \hat{\rho} }
= 
\inv{2} \ket{\uparrow}\bra{\uparrow} 
+\inv{2} \ket{\uparrow}\bra{\uparrow},
= 
\ket{\uparrow}\bra{\uparrow} .
\end{dmath}

This has a matrix representation

\begin{dmath}\label{eqn:qmLecture3:440}
\hat{\rho}_\txtL 
=
\begin{bmatrix}
1 & 0 \\
0 & 0
\end{bmatrix}.
\end{dmath}

In this second example, we have more information about the left partition.  That will be seen as a zero enganglement entropy in the problem set.  In contrast we have less information about the first state, and will find a non-zero positive entanglement entropy in that case.

} % example

\EndArticle
%\EndNoBibArticle
