%
% Copyright � 2012 Peeter Joot.  All Rights Reserved.
% Licenced as described in the file LICENSE under the root directory of this GIT repository.
%
\newcommand{\authorname}{Peeter Joot}
\newcommand{\email}{peeterjoot@protonmail.com}
\newcommand{\basename}{FIXMEbasenameUndefined}
\newcommand{\dirname}{notes/FIXMEdirnameUndefined/}

\renewcommand{\basename}{modernOpticsLecture8}
\renewcommand{\dirname}{notes/phy485/}
\newcommand{\keywords}{Optics, PHY485H1F}
\newcommand{\authorname}{Peeter Joot}
\newcommand{\onlineurl}{http://sites.google.com/site/peeterjoot2/math2013/\basename.pdf}
\newcommand{\sourcepath}{\dirname\basename.tex}
\newcommand{\generatetitle}[1]{\chapter{#1}}

\newcommand{\vcsinfo}{%
\section*{}
\noindent{\color{DarkOliveGreen}{\rule{\linewidth}{0.1mm}}}
\paragraph{Document version}
%\paragraph{\color{Maroon}{Document version}}
{
\small
\begin{itemize}
\item Available online at:\\ 
\href{\onlineurl}{\onlineurl}
\item Git Repository: \input{./.revinfo/gitRepo.tex}
\item Source: \sourcepath
\item last commit: \input{./.revinfo/gitCommitString.tex}
\item commit date: \input{./.revinfo/gitCommitDate.tex}
\end{itemize}
}
}

%\PassOptionsToPackage{dvipsnames,svgnames}{xcolor}
\PassOptionsToPackage{square,numbers}{natbib}
\documentclass{scrreprt}

\usepackage[left=2cm,right=2cm]{geometry}
\usepackage[svgnames]{xcolor}
\usepackage{peeters_layout}

\usepackage{natbib}

\usepackage[
colorlinks=true,
bookmarks=false,
pdfauthor={\authorname, \email},
backref 
]{hyperref}

% http://tex.stackexchange.com/questions/75773/how-to-reference-problems-by-the-text-label-in-an-exercise-envioronment
\usepackage[english]{cleveref}
\crefname{Exercise}{exercise}{exercises}
\Crefname{Exercise}{Exercise}{Exercises}

\RequirePackage{titlesec}
\RequirePackage{ifthen}

% http://stackoverflow.com/questions/4932910/date-in-the-tabular-environment
\makeatletter
\let\insertdate\@date
\makeatother

\titleformat{\chapter}[display]
{\bfseries\Large}
{\color{DarkSlateGrey}\filleft \authorname
\ifthenelse{\isundefined{\studentnumber}}{}{\\ \studentnumber}
\ifthenelse{\isundefined{\email}}{}{\\ \email}
\ifthenelse{\isundefined{\dateintitle}}{}{\\ \insertdate}
%\ifthenelse{\isundefined{\coursename}}{}{\\ \coursename} % put in title instead.
}
{4ex}
{\color{DarkOliveGreen}{\titlerule}\color{Maroon}
\vspace{2ex}%
\filright}
[\vspace{2ex}%
\color{DarkOliveGreen}\titlerule
]

\newcommand{\beginArtWithToc}[0]{\begin{document}\tableofcontents}
\newcommand{\beginArtNoToc}[0]{\begin{document}}
\newcommand{\EndNoBibArticle}[0]{\end{document}}
\newcommand{\EndArticle}[0]{\bibliography{Bibliography}\bibliographystyle{plainnat}\end{document}}

% 
%\newcommand{\citep}[1]{\cite{#1}}

\colorSectionsForArticle


\beginArtNoToc
\generatetitle{PHY485H1F Modern Optics.  Lecture 8: Coherence (cont.).  Taught by Prof.\ Joseph Thywissen}
%\chapter{Coherence (cont.)}
\label{chap:modernOpticsLecture8}

\section{Disclaimer}

Peeter's lecture notes from class.  May not be entirely coherent.

\section{Zoology of interferometers}

\makedefinition{Coherence}{dfn:modernOpticsLecture8:1}{(Operational definition)  Something measured by an interferometer}

\paragraph{Types of dual path interferometers}

FIXME: F1-F5

\paragraph{Types of multi-path interferometers}

FIXME: F6-F8

\section{Lloyd's interferometer}

Using a virtual ray we can think of the Lloyd's interferometer setup as equivalent to a Young's double slit setup as illustrated in

FIXME: F9-F10

Consider two sources as in

FIXME: F11

Looking at this mathematically we have

\begin{dmath}\label{eqn:modernOpticsLecture8:n}
I 
= \expectation{ \Abs{\Psi}^2 }
= \expectation{ \Abs{\Psi(\Br_1, t) + \Psi(\Br_2, t) }^2 }
= 
I(\Br_1) + 
I(\Br_2) + 
2 \Real \expectation{ \Psi(\Br_1, t) \Psi^\conj(\Br_2, t) }
\end{dmath}

All the action is in the cross term.  The portion of this that is hard to calculate, we call the \underline{Mutual coherence}

\begin{dmath}\label{eqn:modernOpticsLecture8:n}
\Gamma_{12} \equiv
\expectation{ \Psi(\Br_1, t) \Psi^\conj(\Br_2, t) }
\end{dmath}

\section{Types of coherence}

\subsection{Longitudinal coherence}

Consider the measurement of the relative interference at two points as in 

FIXME: F12

where we have a device that measures the relative interference at these points as in 

FIXME: F13,F14

where we suppose that there's something that has introduced a small amount of delay or path length.  The extra pathlength like a time delay

\begin{dmath}\label{eqn:modernOpticsLecture8:n}
\tau = \frac{s_2 - s_1}{c}
\end{dmath}

With a \underline{coherence time} defined as

\begin{dmath}\label{eqn:modernOpticsLecture8:n}
\tau_{\text{coh}} = \inv{\Delta w}
\end{dmath}

where $\Delta w$ is the spectral width of the source.

We will show that if

\begin{dmath}\label{eqn:modernOpticsLecture8:n}
s_2 - s_1 \ll c \tau_{\text{coh}}
\end{dmath}

we have good visibility.

We want to think about what happens when the source gets broad as in 

FIXME: F16

\subsection{Transverse coherence}

As illustrated in

FIXME: F15

with 

\begin{dmath}\label{eqn:modernOpticsLecture8:n}
\text{length} = \frac{\lambda}{\Delta \theta_s}
\end{dmath}

we will show that we get a good fringe if

\begin{dmath}\label{eqn:modernOpticsLecture8:n}
x \ll 
\frac{\lambda}{\Delta \theta_s}
\end{dmath}

A \underline{point source} is one for which $\Delta \theta_s \rightarrow 0$, so that $\lambda/\Delta \theta_s \rightarrow \infty$.

We want to think about what happens when the source gets big.

\section{Questions}

\question{Doesn't the intensity loss in the $P_1,P_2$ linear interference setup matter?}

FIXME: F19

If $R$ is small, then the resulting intensities are similar.

\subsection{Mathematical treatment of the mutual coherence}

TO BE CONTINUED.

%\vcsinfo
%\EndArticle
\EndNoBibArticle
