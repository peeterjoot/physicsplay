%
% Copyright � 2015 Peeter Joot.  All Rights Reserved.
% Licenced as described in the file LICENSE under the root directory of this GIT repository.
%
\documentclass[]{eliblog}

\usepackage{amsmath}
\usepackage{mathpazo}

%
% shorthand for bold symbols, convenient for vectors and matrices
%
\newcommand{\Ba}[0]{\mathbf{a}}
\newcommand{\Bb}[0]{\mathbf{b}}
\newcommand{\Bc}[0]{\mathbf{c}}
\newcommand{\Bd}[0]{\mathbf{d}}
\newcommand{\Be}[0]{\mathbf{e}}
\newcommand{\Bf}[0]{\mathbf{f}}
\newcommand{\Bg}[0]{\mathbf{g}}
\newcommand{\Bh}[0]{\mathbf{h}}
\newcommand{\Bi}[0]{\mathbf{i}}
\newcommand{\Bj}[0]{\mathbf{j}}
\newcommand{\Bk}[0]{\mathbf{k}}
\newcommand{\Bl}[0]{\mathbf{l}}
\newcommand{\Bm}[0]{\mathbf{m}}
\newcommand{\Bn}[0]{\mathbf{n}}
\newcommand{\Bo}[0]{\mathbf{o}}
\newcommand{\Bp}[0]{\mathbf{p}}
\newcommand{\Bq}[0]{\mathbf{q}}
\newcommand{\Br}[0]{\mathbf{r}}
\newcommand{\Bs}[0]{\mathbf{s}}
\newcommand{\Bt}[0]{\mathbf{t}}
\newcommand{\Bu}[0]{\mathbf{u}}
\newcommand{\Bv}[0]{\mathbf{v}}
\newcommand{\Bw}[0]{\mathbf{w}}
\newcommand{\Bx}[0]{\mathbf{x}}
\newcommand{\By}[0]{\mathbf{y}}
\newcommand{\Bz}[0]{\mathbf{z}}
\newcommand{\BA}[0]{\mathbf{A}}
\newcommand{\BB}[0]{\mathbf{B}}
\newcommand{\BC}[0]{\mathbf{C}}
\newcommand{\BD}[0]{\mathbf{D}}
\newcommand{\BE}[0]{\mathbf{E}}
\newcommand{\BF}[0]{\mathbf{F}}
\newcommand{\BG}[0]{\mathbf{G}}
\newcommand{\BH}[0]{\mathbf{H}}
\newcommand{\BI}[0]{\mathbf{I}}
\newcommand{\BJ}[0]{\mathbf{J}}
\newcommand{\BK}[0]{\mathbf{K}}
\newcommand{\BL}[0]{\mathbf{L}}
\newcommand{\BM}[0]{\mathbf{M}}
\newcommand{\BN}[0]{\mathbf{N}}
\newcommand{\BO}[0]{\mathbf{O}}
\newcommand{\BP}[0]{\mathbf{P}}
\newcommand{\BQ}[0]{\mathbf{Q}}
\newcommand{\BR}[0]{\mathbf{R}}
\newcommand{\BS}[0]{\mathbf{S}}
\newcommand{\BT}[0]{\mathbf{T}}
\newcommand{\BU}[0]{\mathbf{U}}
\newcommand{\BV}[0]{\mathbf{V}}
\newcommand{\BW}[0]{\mathbf{W}}
\newcommand{\BX}[0]{\mathbf{X}}
\newcommand{\BY}[0]{\mathbf{Y}}
\newcommand{\BZ}[0]{\mathbf{Z}}

\newcommand{\Bzero}[0]{\mathbf{0}}
\newcommand{\Btheta}[0]{\boldsymbol{\theta}}
\newcommand{\Btau}[0]{\boldsymbol{\tau}}
\newcommand{\Bomega}[0]{\boldsymbol{\omega}}

%
% shorthand for unit vectors
%
\newcommand{\acap}[0]{\hat{\Ba}}
\newcommand{\bcap}[0]{\hat{\Bb}}
\newcommand{\ccap}[0]{\hat{\Bc}}
\newcommand{\dcap}[0]{\hat{\Bd}}
\newcommand{\ecap}[0]{\hat{\Be}}
\newcommand{\fcap}[0]{\hat{\Bf}}
\newcommand{\gcap}[0]{\hat{\Bg}}
\newcommand{\hcap}[0]{\hat{\Bh}}
\newcommand{\icap}[0]{\hat{\Bi}}
\newcommand{\jcap}[0]{\hat{\Bj}}
\newcommand{\kcap}[0]{\hat{\Bk}}
\newcommand{\lcap}[0]{\hat{\Bl}}
\newcommand{\mcap}[0]{\hat{\Bm}}
\newcommand{\ncap}[0]{\hat{\Bn}}
\newcommand{\ocap}[0]{\hat{\Bo}}
\newcommand{\pcap}[0]{\hat{\Bp}}
\newcommand{\qcap}[0]{\hat{\Bq}}
\newcommand{\rcap}[0]{\hat{\Br}}
\newcommand{\scap}[0]{\hat{\Bs}}
\newcommand{\tcap}[0]{\hat{\Bt}}
\newcommand{\ucap}[0]{\hat{\Bu}}
\newcommand{\vcap}[0]{\hat{\Bv}}
\newcommand{\wcap}[0]{\hat{\Bw}}
\newcommand{\xcap}[0]{\hat{\Bx}}
\newcommand{\ycap}[0]{\hat{\By}}
\newcommand{\zcap}[0]{\hat{\Bz}}
\newcommand{\thetacap}[0]{\hat{\Btheta}}

%
% to write R^n and C^n in a distinguishable fashion.  Perhaps change this
% to the double lined characters upon figuring out how to do so.
%
\newcommand{\C}[1]{$\mathbb{C}^{#1}$}
\newcommand{\R}[1]{$\mathbb{R}^{#1}$}

%
% various generally useful helpers
%

% derivative of #1 wrt. #2:
\newcommand{\D}[2] {\frac {d#2} {d#1}}

\newcommand{\inv}[1]{\frac{1}{#1}}
\newcommand{\cross}[0]{\times}

\newcommand{\abs}[1]{\lvert{#1}\rvert}
\newcommand{\norm}[1]{\lVert{#1}\rVert}
\newcommand{\innerprod}[2]{\langle{#1}, {#2}\rangle}
\newcommand{\dotprod}[2]{{#1} \cdot {#2}}
\newcommand{\bdotprod}[2]{\left({#1} \cdot {#2}\right)}
\newcommand{\crossprod}[2]{{#1} \cross {#2}}
\newcommand{\tripleprod}[3]{\dotprod{\left(\crossprod{#1}{#2}\right)}{#3}}

\DeclareMathOperator{\Proj}{Proj}
\DeclareMathOperator{\Span}{span}
\DeclareMathOperator{\Sgn}{sgn}
\DeclareMathOperator{\Area}{Area}
\DeclareMathOperator{\Volume}{Volume}

%
% A few miscellaneous things specific to this document
%
\newcommand{\crossop}[1]{\crossprod{#1}{}}

% R2 vector.
\newcommand{\VectorTwo}[2]{
\begin{bmatrix}
 {#1} \\
 {#2}
\end{bmatrix}
}

\newcommand{\VectorN}[1]{
\begin{bmatrix}
{#1}_1 \\
{#1}_2 \\
\vdots \\
{#1}_N \\
\end{bmatrix}
}

\newcommand{\DETuvij}[4]{
\begin{vmatrix}
 {#1}_{#3} & {#1}_{#4} \\
 {#2}_{#3} & {#2}_{#4}
\end{vmatrix}
}

\newcommand{\DETuvwijk}[6]{
\begin{vmatrix}
 {#1}_{#4} & {#1}_{#5} & {#1}_{#6} \\
 {#2}_{#4} & {#2}_{#5} & {#2}_{#6} \\
 {#3}_{#4} & {#3}_{#5} & {#3}_{#6}
\end{vmatrix}
}

\newcommand{\DETuvwxijkl}[8]{
\begin{vmatrix}
 {#1}_{#5} & {#1}_{#6} & {#1}_{#7} & {#1}_{#8} \\
 {#2}_{#5} & {#2}_{#6} & {#2}_{#7} & {#2}_{#8} \\
 {#3}_{#5} & {#3}_{#6} & {#3}_{#7} & {#3}_{#8} \\
 {#4}_{#5} & {#4}_{#6} & {#4}_{#7} & {#4}_{#8} \\
\end{vmatrix}
}

%\newcommand{\DETuvwxyijklm}[10]{
%\begin{vmatrix}
% {#1}_{#6} & {#1}_{#7} & {#1}_{#8} & {#1}_{#9} & {#1}_{#10} \\
% {#2}_{#6} & {#2}_{#7} & {#2}_{#8} & {#2}_{#9} & {#2}_{#10} \\
% {#3}_{#6} & {#3}_{#7} & {#3}_{#8} & {#3}_{#9} & {#3}_{#10} \\
% {#4}_{#6} & {#4}_{#7} & {#4}_{#8} & {#4}_{#9} & {#4}_{#10} \\
% {#5}_{#6} & {#5}_{#7} & {#5}_{#8} & {#5}_{#9} & {#5}_{#10}
%\end{vmatrix}
%}

% R3 vector.
\newcommand{\VectorThree}[3]{
\begin{bmatrix}
 {#1} \\
 {#2} \\
 {#3}
\end{bmatrix}
}



\author{Peeter Joot}
\email{peeter.joot@gmail.com}

%\documentclass[]{eliblogwidescreen}

\usepackage{amsmath}
\usepackage{mathpazo}

%
% shorthand for bold symbols, convenient for vectors and matrices
%
\newcommand{\Ba}[0]{\mathbf{a}}
\newcommand{\Bb}[0]{\mathbf{b}}
\newcommand{\Bc}[0]{\mathbf{c}}
\newcommand{\Bd}[0]{\mathbf{d}}
\newcommand{\Be}[0]{\mathbf{e}}
\newcommand{\Bf}[0]{\mathbf{f}}
\newcommand{\Bg}[0]{\mathbf{g}}
\newcommand{\Bh}[0]{\mathbf{h}}
\newcommand{\Bi}[0]{\mathbf{i}}
\newcommand{\Bj}[0]{\mathbf{j}}
\newcommand{\Bk}[0]{\mathbf{k}}
\newcommand{\Bl}[0]{\mathbf{l}}
\newcommand{\Bm}[0]{\mathbf{m}}
\newcommand{\Bn}[0]{\mathbf{n}}
\newcommand{\Bo}[0]{\mathbf{o}}
\newcommand{\Bp}[0]{\mathbf{p}}
\newcommand{\Bq}[0]{\mathbf{q}}
\newcommand{\Br}[0]{\mathbf{r}}
\newcommand{\Bs}[0]{\mathbf{s}}
\newcommand{\Bt}[0]{\mathbf{t}}
\newcommand{\Bu}[0]{\mathbf{u}}
\newcommand{\Bv}[0]{\mathbf{v}}
\newcommand{\Bw}[0]{\mathbf{w}}
\newcommand{\Bx}[0]{\mathbf{x}}
\newcommand{\By}[0]{\mathbf{y}}
\newcommand{\Bz}[0]{\mathbf{z}}
\newcommand{\BA}[0]{\mathbf{A}}
\newcommand{\BB}[0]{\mathbf{B}}
\newcommand{\BC}[0]{\mathbf{C}}
\newcommand{\BD}[0]{\mathbf{D}}
\newcommand{\BE}[0]{\mathbf{E}}
\newcommand{\BF}[0]{\mathbf{F}}
\newcommand{\BG}[0]{\mathbf{G}}
\newcommand{\BH}[0]{\mathbf{H}}
\newcommand{\BI}[0]{\mathbf{I}}
\newcommand{\BJ}[0]{\mathbf{J}}
\newcommand{\BK}[0]{\mathbf{K}}
\newcommand{\BL}[0]{\mathbf{L}}
\newcommand{\BM}[0]{\mathbf{M}}
\newcommand{\BN}[0]{\mathbf{N}}
\newcommand{\BO}[0]{\mathbf{O}}
\newcommand{\BP}[0]{\mathbf{P}}
\newcommand{\BQ}[0]{\mathbf{Q}}
\newcommand{\BR}[0]{\mathbf{R}}
\newcommand{\BS}[0]{\mathbf{S}}
\newcommand{\BT}[0]{\mathbf{T}}
\newcommand{\BU}[0]{\mathbf{U}}
\newcommand{\BV}[0]{\mathbf{V}}
\newcommand{\BW}[0]{\mathbf{W}}
\newcommand{\BX}[0]{\mathbf{X}}
\newcommand{\BY}[0]{\mathbf{Y}}
\newcommand{\BZ}[0]{\mathbf{Z}}

\newcommand{\Bzero}[0]{\mathbf{0}}
\newcommand{\Btheta}[0]{\boldsymbol{\theta}}
\newcommand{\Btau}[0]{\boldsymbol{\tau}}
\newcommand{\Bomega}[0]{\boldsymbol{\omega}}

%
% shorthand for unit vectors
%
\newcommand{\acap}[0]{\hat{\Ba}}
\newcommand{\bcap}[0]{\hat{\Bb}}
\newcommand{\ccap}[0]{\hat{\Bc}}
\newcommand{\dcap}[0]{\hat{\Bd}}
\newcommand{\ecap}[0]{\hat{\Be}}
\newcommand{\fcap}[0]{\hat{\Bf}}
\newcommand{\gcap}[0]{\hat{\Bg}}
\newcommand{\hcap}[0]{\hat{\Bh}}
\newcommand{\icap}[0]{\hat{\Bi}}
\newcommand{\jcap}[0]{\hat{\Bj}}
\newcommand{\kcap}[0]{\hat{\Bk}}
\newcommand{\lcap}[0]{\hat{\Bl}}
\newcommand{\mcap}[0]{\hat{\Bm}}
\newcommand{\ncap}[0]{\hat{\Bn}}
\newcommand{\ocap}[0]{\hat{\Bo}}
\newcommand{\pcap}[0]{\hat{\Bp}}
\newcommand{\qcap}[0]{\hat{\Bq}}
\newcommand{\rcap}[0]{\hat{\Br}}
\newcommand{\scap}[0]{\hat{\Bs}}
\newcommand{\tcap}[0]{\hat{\Bt}}
\newcommand{\ucap}[0]{\hat{\Bu}}
\newcommand{\vcap}[0]{\hat{\Bv}}
\newcommand{\wcap}[0]{\hat{\Bw}}
\newcommand{\xcap}[0]{\hat{\Bx}}
\newcommand{\ycap}[0]{\hat{\By}}
\newcommand{\zcap}[0]{\hat{\Bz}}
\newcommand{\thetacap}[0]{\hat{\Btheta}}

%
% to write R^n and C^n in a distinguishable fashion.  Perhaps change this
% to the double lined characters upon figuring out how to do so.
%
\newcommand{\C}[1]{$\mathbb{C}^{#1}$}
\newcommand{\R}[1]{$\mathbb{R}^{#1}$}

%
% various generally useful helpers
%

% derivative of #1 wrt. #2:
\newcommand{\D}[2] {\frac {d#2} {d#1}}

\newcommand{\inv}[1]{\frac{1}{#1}}
\newcommand{\cross}[0]{\times}

\newcommand{\abs}[1]{\lvert{#1}\rvert}
\newcommand{\norm}[1]{\lVert{#1}\rVert}
\newcommand{\innerprod}[2]{\langle{#1}, {#2}\rangle}
\newcommand{\dotprod}[2]{{#1} \cdot {#2}}
\newcommand{\bdotprod}[2]{\left({#1} \cdot {#2}\right)}
\newcommand{\crossprod}[2]{{#1} \cross {#2}}
\newcommand{\tripleprod}[3]{\dotprod{\left(\crossprod{#1}{#2}\right)}{#3}}

\DeclareMathOperator{\Proj}{Proj}
\DeclareMathOperator{\Span}{span}
\DeclareMathOperator{\Sgn}{sgn}
\DeclareMathOperator{\Area}{Area}
\DeclareMathOperator{\Volume}{Volume}

%
% A few miscellaneous things specific to this document
%
\newcommand{\crossop}[1]{\crossprod{#1}{}}

% R2 vector.
\newcommand{\VectorTwo}[2]{
\begin{bmatrix}
 {#1} \\
 {#2}
\end{bmatrix}
}

\newcommand{\VectorN}[1]{
\begin{bmatrix}
{#1}_1 \\
{#1}_2 \\
\vdots \\
{#1}_N \\
\end{bmatrix}
}

\newcommand{\DETuvij}[4]{
\begin{vmatrix}
 {#1}_{#3} & {#1}_{#4} \\
 {#2}_{#3} & {#2}_{#4}
\end{vmatrix}
}

\newcommand{\DETuvwijk}[6]{
\begin{vmatrix}
 {#1}_{#4} & {#1}_{#5} & {#1}_{#6} \\
 {#2}_{#4} & {#2}_{#5} & {#2}_{#6} \\
 {#3}_{#4} & {#3}_{#5} & {#3}_{#6}
\end{vmatrix}
}

\newcommand{\DETuvwxijkl}[8]{
\begin{vmatrix}
 {#1}_{#5} & {#1}_{#6} & {#1}_{#7} & {#1}_{#8} \\
 {#2}_{#5} & {#2}_{#6} & {#2}_{#7} & {#2}_{#8} \\
 {#3}_{#5} & {#3}_{#6} & {#3}_{#7} & {#3}_{#8} \\
 {#4}_{#5} & {#4}_{#6} & {#4}_{#7} & {#4}_{#8} \\
\end{vmatrix}
}

%\newcommand{\DETuvwxyijklm}[10]{
%\begin{vmatrix}
% {#1}_{#6} & {#1}_{#7} & {#1}_{#8} & {#1}_{#9} & {#1}_{#10} \\
% {#2}_{#6} & {#2}_{#7} & {#2}_{#8} & {#2}_{#9} & {#2}_{#10} \\
% {#3}_{#6} & {#3}_{#7} & {#3}_{#8} & {#3}_{#9} & {#3}_{#10} \\
% {#4}_{#6} & {#4}_{#7} & {#4}_{#8} & {#4}_{#9} & {#4}_{#10} \\
% {#5}_{#6} & {#5}_{#7} & {#5}_{#8} & {#5}_{#9} & {#5}_{#10}
%\end{vmatrix}
%}

% R3 vector.
\newcommand{\VectorThree}[3]{
\begin{bmatrix}
 {#1} \\
 {#2} \\
 {#3}
\end{bmatrix}
}



\author{Peeter Joot}
\email{peeter.joot@gmail.com}


\chapter{PHY450H1S.  Relativistic Electrodynamics Lecture 6 (Taught by Prof. Erich Poppitz).  Four vectors and tensors.}
\label{chap:relativisticElectrodynamicsL6}
%\useCCL
\blogpage{http://sites.google.com/site/peeterjoot/math2011/relativisticElectrodynamicsL6.pdf}
\date{Jan 25, 2011}
\revisionInfo{relativisticElectrodynamicsL6.tex}

%\beginArtWithToc
\beginArtNoToc

\section{Reading.}

Still covering chapter 1 material from the text \cite{landau1980classical}?

Covering \href{http://www.physics.utoronto.ca/~poppitz/e-poppitz/PHY450_files/RelEM27-44.pdf}{Professor Poppitz's lecture notes}: nonrelativistic limit of boosts (33); number of parameters of Lorentz transformations (34-35); introducing four-vectors, the metric tensor, the invariant ``dot-product and SO(1,3) (36-40); the Poincare group (41); the convenience of ``upper'' and ``lower''indices (42-43); tensors (44) 

\section{}

This generalizes to Lorentz boosts!  There are two differences

\begin{enumerate}
\item Lorentz trasnforms should be $4 \times 4$ not $3 \times 3$ and act in $(ct, x, y, z)$, and NOT $(x,y,z)$.
\item They shoudl leave invariant NOT $\Br_1 \cdot \Br_2$, but $c2 t_2 t_1 - \B r_2 \cdot \Br_1$.
\end{enumerate}

Don't get confused that I demanded $c^2 t_2 t_1 - \Br_2 \cdot \Br_1 = \text{invariant}$ rather than $c^2 (t_2 - t_1)^2 - (\Br_2 - \Br_1)^2 = \text{invariant}$.  Expansion of this (squared) interval, provides just this four vector dot product and its invariance condition

FIXME: DO THIS.

\subsection{Introduce the four vector}

\begin{align*}
x^0 &= ct \\
x^1 &= x \\
x^2 &= y \\
x^3 &= z 
\end{align*}

Or $(x^0, x^1, x^2, x^3) = \{ x^i, i = 0,1,2,3 \}$.

We will also write

\begin{align*}
x^i &= (ct, \Br) \\
\tilde{x}^i &= (c\tilde{t}, \tilde{\Br})
\end{align*}

Our inner product is

c^2 t \tilde{t} - \Br \cdot \tilde{\Br}

Introduce the $4 \times 4$ matrix 

% used double bar abs (norm) here
\Norm{g_{ij}} = 
\begin{bmatrix}
1 & 0 & 0 & 0
0 & -1 & 0 & 0
.. % FIXME
\end{bmatrix}

This is called the Minkowski spacetime metric.

Then 

c^2 t \tilde{t} - \Br \cdot \tilde{\Br}
&\equiv \sum_{i, j = 0}^3 \tilde{x}^i g_{ij} x^j \\
&= \sum_{i, j = 0}^3 \tilde{x}^i g_{ij} x^j
& 
\tilde{x}^0 x^0 
+\tilde{x}^1 x^1 
+\tilde{x}^2 x^2 
+\tilde{x}^3 x^3 

\paragraph{Einstein summation convention}.  Whenever indexes are repeated that are assumed to be summed over.

We also write

X = 
\begin{bmatrix}
x^0 \\
x^1 \\
x^2 \\
x^3 \\
\end{bmatrix}
\tilde{X} = 
\begin{bmatrix}
\tilde{x}^0 \\
\tilde{x}^1 \\
\tilde{x}^2 \\
\tilde{x}^3 \\
\end{bmatrix}

G = 
\begin{bmatrix}
1 & 
0 & -1 ...
\end{bmatrix}

Our inner product 

c^2 t \tilde{t} ... = \tilde{X}^\T G X 
&=
\begin{bmatrix}
\tilde{x}^0 & \tilde{x}^1 & \tilde{x}^2 & \tilde{x}^3 
\end{bmatrix}
\begin{bmatrix}
1 & 
0 & -1 ...
\end{bmatrix}
\begin{bmatrix}
\tilde{x}^0 \\
\tilde{x}^1 \\
\tilde{x}^2 \\
\tilde{x}^3 \\
\end{bmatrix}

Under Lorentz boosts, we have

X = \hat{O} X', where 

O =
\begin{bmatrix}
\gamma & - \gamma v_x/c & 0 & 0
...
\end{bmatrix}

(for x-direction boosts)

\tilde{X} = \hat{O} \tilde{X}'
\tilde{X}^\T = \tilde{X}'^\T \hat{O}^\T

But $\hat{O}$ must be such that $\tilde{X}^\T G X$ is invariant.  i.e.

%FIXME: Y = \tilde{X}
%FIXME: O = \hat{O}
Y G X = Y^\T' (O^\T G O) X' = X'^\T (G) X'

\forall X' and Y'

\implies

O^\T G O = G % BOXED.

Such $O$'s are called ``pseudoorthogonal''.

Lorentz transformations are represented by the set of all $4 \times 4$ pseudoorthogonal matrices.

In symbols

O^T G O = G

Just as before we can take the determinant of both sides.  Doing so we have

\det(O^T G O) = \det(O^T) \det(G) \det(O) = \det(G)

The \det(G) terms cancel, and since $\det(O^T) = \det(O)$, this leaves us with (\det(O))^2 = 1, or

\det(O) = \pm 1

We take the $det 0 = + 1$ case only, so that the transformations do not change orientation (no reflection in space or time).  This set of transformation forms the group

SO(1,3)

special orthogonal, one time, 3 space dimensions.

Einstein relativity can be defined as the ``laws of physics that leave four vectors invariant in the

SO(1,3) \times T^4

symmetry group.

Here $T^4$ is the group of translations in spacetime with 4 continuous parameters.   The complete group of transformations that form the group of relativistic physics has $10 = 3 + 3 + 4$ continuous parameters.

This group is called the Poincare group of symmetry transforms.

\section{More notation}

% y = \tilde{x}

Our inner product is written

y^i g_{ij} x^j

but this is very cumbersome.  The convient way to write this is instead

y^i g_{ij} x^j = y_j x^j = y^i x_i

where 

x_i = g_{ij} x^j = g_{ji} x^j

Note: A check that we should always be able to make.  Indexes that are not summed over should be conserved.  So in the above we have a free $i$ on the LHS, and should have a non-summed $i$ index on the RHS too (also lower matching lower, or upper matching upper).

Non-matched indexes are bad in the same sort of sense that an expression like

\Br = 1

isn't well defined (assuming a vector space $\Br$ and not a multivector clifford algebra that is;)

Example explicitly:

x_0 &= g_{0 0} x^0 = ct  \\
x_1 &= g_{1 j} x^j = g_{11} x^1 = -x^1 \\
x_2 &= g_{2 j} x^j = g_{22} x^2 = -x^2 \\
x_3 &= g_{3 j} x^j = g_{33} x^3 = -x^3

We would not have objects of the form 

x^i x^i = (ct)^2 + \Br^2

for example.  This is not a Lorentz invariant quantity.

Lorentz scalar example: y^i x_i
Lorentz vector example: x^i

This last is also called a rank-1 tensor.

Lorentz rank-2 tensors: ex: g_{ij}

or other 2-index objects.

Why in the world would we ever want to consider two index objects.  We aren't just trying to be hard on ourselves.  Recall from classical mechanics that we have a two index object, the inertial tensor.

In mechanics, for a rigid body we had the energy

T = \sum_{ij = 1}^3 \Omega_i I_{ij} \Omega_j

The inertial tensor was this object 

I_{ij} = \sum_{a = 1}^N m_a \left(\delta_{ij} \Br_a^2 - r_{a_i} r_{a_j} \right)

or for a continuous body

I_{ij} = \int \rho(\Br) \left(\delta_{ij} \Br^2 - r_{i} r_{j} \right)

In electrostatics we have the quadropole tensor, ... and we have other such objects all over physics.

Note that the energy $T$ of the body above cannot depend on the coordinate system in use.  This is a general property of tensors.  These are object that transform as products of vectors, as $I_{ij}$ does.  

We call $I_{ij}$ a rank-2 3-tensor.  rank-2 because there are two indexes, and 3 because the indexes range from $1$ to $3$.

The point is that tensors have the property that the transformed tensors transform as

I_{ij}' = \sum_{l, m = 1,2,3} O_{il} O_{jm} I_{lm}

FIXME: show this based on the definition above of $I_{ij}$.

Another example: the completely antisymmetric rank 3, 3-tensor

\epsilon_{ijk}

\section{Dynamics}

In dynamics we have 

m \ddot{\Br} = \Bf

An equation of motion should be expressed in terms of vectors.  We are going to look to the symmetries and the invariance of the action as a motivator

This equation is written in a way that shows that the law of physics is independent of the choice of coordinates.  The next task is to find a way that we can 

...
We want to express relativisitic dynamics in a similar way, and will have to express the action as a Lorentz scalar.

We are going to impose the symmetries of the Poincare group to determine the relativisitic laws of dynamics.

\EndArticle
