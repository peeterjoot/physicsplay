%
% Copyright � 2013 Peeter Joot.  All Rights Reserved.
% Licenced as described in the file LICENSE under the root directory of this GIT repository.
%
\newcommand{\authorname}{Peeter Joot}
\newcommand{\email}{peeterjoot@protonmail.com}
\newcommand{\basename}{FIXMEbasenameUndefined}
\newcommand{\dirname}{notes/FIXMEdirnameUndefined/}

\renewcommand{\basename}{condensedMatterLecture14}
\renewcommand{\dirname}{notes/phy487/}
\newcommand{\keywords}{Condensed matter physics, PHY487H1F}
\newcommand{\authorname}{Peeter Joot}
\newcommand{\onlineurl}{http://sites.google.com/site/peeterjoot2/math2013/\basename.pdf}
\newcommand{\sourcepath}{\dirname\basename.tex}
\newcommand{\generatetitle}[1]{\chapter{#1}}

\newcommand{\vcsinfo}{%
\section*{}
\noindent{\color{DarkOliveGreen}{\rule{\linewidth}{0.1mm}}}
\paragraph{Document version}
%\paragraph{\color{Maroon}{Document version}}
{
\small
\begin{itemize}
\item Available online at:\\ 
\href{\onlineurl}{\onlineurl}
\item Git Repository: \input{./.revinfo/gitRepo.tex}
\item Source: \sourcepath
\item last commit: \input{./.revinfo/gitCommitString.tex}
\item commit date: \input{./.revinfo/gitCommitDate.tex}
\end{itemize}
}
}

%\PassOptionsToPackage{dvipsnames,svgnames}{xcolor}
\PassOptionsToPackage{square,numbers}{natbib}
\documentclass{scrreprt}

\usepackage[left=2cm,right=2cm]{geometry}
\usepackage[svgnames]{xcolor}
\usepackage{peeters_layout}

\usepackage{natbib}

\usepackage[
colorlinks=true,
bookmarks=false,
pdfauthor={\authorname, \email},
backref 
]{hyperref}

% http://tex.stackexchange.com/questions/75773/how-to-reference-problems-by-the-text-label-in-an-exercise-envioronment
\usepackage[english]{cleveref}
\crefname{Exercise}{exercise}{exercises}
\Crefname{Exercise}{Exercise}{Exercises}

\RequirePackage{titlesec}
\RequirePackage{ifthen}

% http://stackoverflow.com/questions/4932910/date-in-the-tabular-environment
\makeatletter
\let\insertdate\@date
\makeatother

\titleformat{\chapter}[display]
{\bfseries\Large}
{\color{DarkSlateGrey}\filleft \authorname
\ifthenelse{\isundefined{\studentnumber}}{}{\\ \studentnumber}
\ifthenelse{\isundefined{\email}}{}{\\ \email}
\ifthenelse{\isundefined{\dateintitle}}{}{\\ \insertdate}
%\ifthenelse{\isundefined{\coursename}}{}{\\ \coursename} % put in title instead.
}
{4ex}
{\color{DarkOliveGreen}{\titlerule}\color{Maroon}
\vspace{2ex}%
\filright}
[\vspace{2ex}%
\color{DarkOliveGreen}\titlerule
]

\newcommand{\beginArtWithToc}[0]{\begin{document}\tableofcontents}
\newcommand{\beginArtNoToc}[0]{\begin{document}}
\newcommand{\EndNoBibArticle}[0]{\end{document}}
\newcommand{\EndArticle}[0]{\bibliography{Bibliography}\bibliographystyle{plainnat}\end{document}}

% 
%\newcommand{\citep}[1]{\cite{#1}}

\colorSectionsForArticle



%\citep{harald2003solid} \S x.y
%\citep{ibach2009solid} \S x.y

%\usepackage{mhchem}
\usepackage[version=3]{mhchem}
\newcommand{\nought}[0]{\circ}
\newcommand{\EF}[0]{\epsilon_{\mathrm{F}}}
\newcommand{\kF}[0]{k_{\mathrm{F}}}

\beginArtNoToc
\generatetitle{PHY487H1F Condensed Matter Physics.  Lecture 14: Electrons in a periodic lattice.  Taught by Prof.\ Stephen Julian}
%\chapter{Electrons in a periodic lattice}
\label{chap:condensedMatterLecture14}

%\section{Disclaimer}
%
%Peeter's lecture notes from class.  May not be entirely coherent.

\section{Electrons in a periodic lattice}

We want to look at the general properties of 1 electon in a periodic potential.  We want to solve the Schr\"{o}dinger equation

\begin{dmath}\label{eqn:condensedMatterLecture14:20}
\lr{
-\frac{\Hbar^2}{2m} \spacegrad^2 + V(\Br)
}
\Psi(\Br) = E \Psi(\Br)
\end{dmath}

Here $V(\Br)$ is periodic $V(\Br + \Br_n) = V(\Br)$, with

\begin{dmath}\label{eqn:condensedMatterLecture14:40}
\Br_n = 
n_1 \Ba_1
+ n_2 \Ba_2
+ n_3 \Ba_3,
\end{dmath}

so that 

\begin{dmath}\label{eqn:condensedMatterLecture14:60}
V(\Br) = \sum_\BG V_\BG e^{i \BG \cdot \Br },
\end{dmath}

where

\begin{subequations}
\begin{dmath}\label{eqn:condensedMatterLecture14:80}
\BG = 
h \Bg_1
+ k \Bg_2
+ l \Bg_3,
\end{dmath}
\begin{dmath}\label{eqn:condensedMatterLecture14:100}
\Bg_i \cdot \Ba_j = \delta_{ij}.
\end{dmath}
\end{subequations}

Example potential sketched in

F1

Use a plane wave basis 

\begin{dmath}\label{eqn:condensedMatterLecture14:120}
\Psi(\Br) = \sum_\Bk C_\Bk e^{i \Bk \cdot \Br },
\end{dmath}

\begin{dmath}\label{eqn:condensedMatterLecture14:140}
\sum_\Bk' \frac{\Hbar^2 {\Bk'}^2 }{2m} C_\Bk' e^{i \Bk' \cdot \Br},
+
\sum_{\Bk'', \BG} V_\BG e^{i (\Bk'' + \BG) \cdot \Br}
=
E \sum_\Bk' 
e^{i \Bk' \cdot \Br},
\end{dmath}

with 

\begin{dmath}\label{eqn:condensedMatterLecture14:160}
\Bk' = \Bk'' + \BG
\end{dmath}

or
\begin{dmath}\label{eqn:condensedMatterLecture14:180}
\Bk'' = \Bk' - \BG,
\end{dmath}

this is

%%%\begin{dmath}\label{eqn:condensedMatterLecture14:200}
%%%0 = 
%%%\sum_\Bk' 
%%%\lr{ \frac{\Hbar^2 {\Bk'}^2 }{2m} - E) C_\Bk' 
%%%+
%%%\sum_\BG V_\BG C_{\Bk' - \BG}
%%%}
%%%\end{dmath}
%%%
%%%Using 
%%%
%%%\begin{dmath}\label{eqn:condensedMatterLecture14:220}
%%%\int d^\Br e^{ -k (\Bk' - \Bk) \cdot \Br } \propto \delta(\Bk' - \Bk),
%%%\end{dmath}
%%%
%%%and operating with
%%%\begin{dmath}\label{eqn:condensedMatterLecture14:240}
%%%\int d^\Br e^{ -k (\Bk' - \Bk) \cdot \Br } \times \cdots,
%%%\end{dmath}
%%%
%%%we decouple the system 
%%%
%%%\begin{dmath}\label{eqn:condensedMatterLecture14:260}
%%%\myBoxed{
%%%0 = 
%%%\lr{ \frac{\Hbar^2 {\Bk}^2 }{2m} - E) C_\Bk 
%%%+
%%%\sum_\BG V_\BG C_{\Bk - \BG}.
%%%}
%%%\end{dmath}
%%%
%%%Each eigenstate only involves $C_\Bk$'s that differ by r.l.v's (?)
%%%
%%%F2
%%%
%%%\begin{dmath}\label{eqn:condensedMatterLecture14:280}
%%%\Psi \sim 
%%%C_k e^{i k x} 
%%%+ C_{k + 2 \pi/a} e^{i (k + 2 \pi/a) x} 
%%%+ C_{k - 2 \pi/a} e^{i (k - 2 \pi/a) x} 
%%%+ \cdots
%%%\end{dmath}
%%%
%%%Label each eigenstate with $\Bk$

\begin{subequations}
\begin{equation}\label{eqn:condensedMatterLecture14:300}
\Psi_\Bk(\Br)
\end{equation}
\begin{equation}\label{eqn:condensedMatterLecture14:320}
E = E_\Bk = E(\Bk),
\end{equation}
\end{subequations}

%%%\begin{dmath}\label{eqn:condensedMatterLecture14:340}
%%%\Psi_\Bk(\Br) 
%%%= \sum_\BG C_{\Bk - \BG} e^{ i (\Bk - \BG) \cdot \Br}
%%%= 
%%%\mathLabelBox
%%%{
%%%\lr{
%%%\sum_\BG C_{\Bk - \BG} e^{ -i \BG \cdot \Br}
%%%}
%%%}
%%%{Periodic in $\Br$ with lattice periodicity}
%%%\mathLabelBox
%%%[
%%%   labelstyle={below of=m\themathLableNode, below of=m\themathLableNode}
%%%]
%%%{
%%%e^{ i \Bk \cdot \Br}
%%%}
%%%{plane wave}
%%%\end{dmath}
%%%
%%%\begin{dmath}\label{eqn:condensedMatterLecture14:360}
%%%\myBoxed{
%%%\Psi_\Bk(\Br) = U_\Bk(\Br) e^{i \Bk \cdot \Br},
%%%}
%%%\end{dmath}
%%%
%%%where $U_\Bk(\Br)$ is a period function.  This is \underlineAndIndex{Bloch's theorem}.
%%%
%%%With $\Psi_\Bk(\Br)$ periodic in $\Bk$ we have
%%%
%%%\begin{dmath}\label{eqn:condensedMatterLecture14:380}
%%%\Psi_{\Bk + \BG}(\Br)
%%%=
%%%\lr{
%%%\sum_{\BG'} C_{\Bk + \BG - \BG'} e^{ i (\Bk + \BG - \BG') \cdot \Br}
%%%\end{dmath}
%%%
%%%With 
%%%
%%%\begin{dmath}\label{eqn:condensedMatterLecture14:400}
%%%\BG'' = \BG' - \BG,
%%%\end{dmath}
%%%
%%%this is
%%%
%%%\begin{dmath}\label{eqn:condensedMatterLecture14:420}
%%%\Psi_{\Bk + \BG}(\Br)
%%%=
%%%\lr{
%%%\sum_{\BG''} C_{\Bk - \BG''} e^{ i (\Bk - \BG'') \cdot \Br}
%%%=
%%%\Psi_\Bk(\Br).
%%%\end{dmath}
%%%
%%%\begin{dmath}\label{eqn:condensedMatterLecture14:440}
%%%\myBoxed{
%%%\begin{aligned}
%%%\Psi_{\Bk + \BG}(\Br)
%%%&=
%%%\Psi_\Bk(\Br) \\
%%%E(\Bk + \BG) = E(\Bk).
%%%\end{aligned}
%%%}
%%%\end{dmath}
%%%
%%%We want to examine what this means.
%%%
%%%\section{Nearly free electron model}
%%%
%%%Consider $V(\Br)$ in the limit $V(\Br) \rightarrow 0$, but still keep periodicity.  This leads to 
%%%
%%%\begin{dmath}\label{eqn:condensedMatterLecture14:460}
%%%\begin{aligned}
%%%\Psi_{\Bk}(\Br) &= \inv{\sqrt{L}} e^{i \Bk \cdot \Br} \\
%%%E(\Bk) &= \frac{\Hbar^2 \Bk^2}{2m}
%%%\end{aligned}
%%%\end{dmath}
%%%
%%%F3
%%%
%%%Solutions outside the first Brillion zone are redundant.  This is called the \textAndIndex{reduced zone scheme}.  The periodicity folds the extended solution into the first Brillion zone.
%%%
%%%F4
%%%
%%%%The meaning of this periodicity in momen
%%%
%%%\paragraph{Bloch's theorem}
%%%
%%%F5
%%%%\frac{\Psi_{\Bk}(\Br) }{ \inv{\sqrt{L}} e^{i k x} }
%%%
%%%%\EndArticle
\EndNoBibArticle
