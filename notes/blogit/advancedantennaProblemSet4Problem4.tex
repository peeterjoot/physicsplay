%
% Copyright � 2015 Peeter Joot.  All Rights Reserved.
% Licenced as described in the file LICENSE under the root directory of this GIT repository.
%
\makeproblem{Dolph-Chebyshev}{advancedantenna:problemSet4:4}{ 

Design a five-element, \( -40 \si{dB} \) sidelobe level Dolph-Chebyshev array of isotropic elements. 
The elements are placed along the x-axis with an inter-element spacing \( d = \lambda /2 \).
Determine,

\makesubproblem{}{advancedantenna:problemSet4:4a}
the normalized amplitude coefficients
\makesubproblem{}{advancedantenna:problemSet4:4b}
the array factor
\makesubproblem{}{advancedantenna:problemSet4:4c}
Use numerical integration to calculate the directivity
\makesubproblem{}{advancedantenna:problemSet4:4d}
and the null-to-null beamwidth
\makesubproblem{}{advancedantenna:problemSet4:4e}
Repeat \partref{advancedantenna:problemSet4:4a}-\ref{advancedantenna:problemSet4:4c}
 for a uniform broadside array of the same spacing
\makesubproblem{}{advancedantenna:problemSet4:4f}
Plot the power array-factor patterns for the two arrays on the same plot.
} % makeproblem

\makeanswer{advancedantenna:problemSet4:4}{ 
\makeSubAnswer{}{advancedantenna:problemSet4:4a}

The \( 40 \si{dB} \) level is equivalent to 

\begin{dmath}\label{eqn:advancedantennaProblemSet4Problem4:20}
20 \log_{10} R = 40,
\end{dmath}

or

\begin{equation}\label{eqn:advancedantennaProblemSet4Problem4:40}
R = 10^{2} = 100.
\end{equation}

The Chebychev scaling factor for a five element array is

\begin{dmath}\label{eqn:advancedantennaProblemSet4Problem4:60}
x_0 = \cosh\lr{ \inv{4} \cosh^{-1} R } \approx 2.
\end{dmath}

With \( x = x_0 \cos(u/2) \), the unnormalized array factor is

\begin{dmath}\label{eqn:advancedantennaProblemSet4Problem4:160}
\textrm{AF}(u) 
= T_4( x ) 
= T_4( x_0 \cos(u/2) )
= 
8 x_0^4 \cos^4(u/2) - 8 x_0^2 \cos^2(u/2) + 1.
\end{dmath}

Since

\begin{equation}\label{eqn:advancedantennaProblemSet4Problem4:80}
\begin{aligned}
\cos^2(u/2) &= \inv{2} \lr{ \cos(u) + 1 } \\
\cos^4(u/2) &= \inv{8} \lr{ \cos(2 u) + 4 \cos(u) + 3 },
\end{aligned}
\end{equation}

the array factor can be expanded in \( \cos( m u ) \), as

\begin{dmath}\label{eqn:advancedantennaProblemSet4Problem4:100}
\textrm{AF}(u) 
= 
x_0^4 \lr{ \cos(2 u) + 4 \cos(u) + 3 }
- 4 x_0^2 \lr{ \cos(u) + 1 }
+ 1
=
x_0^4 \cos(2 u) 
+ \lr{ 4 x_0^4 - 4 x_0^2 } \cos(u)
+ 3 x_0^4 - 4 x_0^2 + 1.
\end{dmath}

After normalization this is

\begin{equation}\label{eqn:advancedantennaProblemSet4Problem4:120}
\begin{aligned}
\textrm{AF}(u) &= \alpha \cos( 2 u ) + \beta \cos(u ) + \gamma \\
\alpha &= \frac{x_0^4}{8 x_0^4 - 8 x_0^2 + 1} \\
\beta &= \frac{ 4 x_0^4 - 4 x_0^2 }{8 x_0^4 - 8 x_0^2 + 1} \\
\gamma &= \frac{ 3 x_0^4 - 4 x_0^2 + 1 }{8 x_0^4 - 8 x_0^2 + 1}
\end{aligned}
\end{equation}

The array coeffients are found to have the values

\begin{equation}\label{eqn:advancedantennaProblemSet4Problem4:140}
\begin{aligned}
I_{-2} &= \frac{\alpha}{2} = 0.08 \\
I_{-1} &= \frac{\beta}{2} = 0.25 \\
I_{0} &= \gamma  = 0.34 \\
I_{1} &= \frac{\beta}{2} = 0.25 \\
I_{2} &= \frac{\alpha}{2} = 0.08.
\end{aligned}
\end{equation}

\makeSubAnswer{}{advancedantenna:problemSet4:4b}

For the x-axis array we have \( u = \pi \sin\theta \cos\phi \), and for the broadside array (placed along the z-axis) we have \( u = \pi \cos\theta \).  Both are specified by \cref{eqn:advancedantennaProblemSet4Problem4:120}, \cref{eqn:advancedantennaProblemSet4Problem4:140}.

\makeSubAnswer{}{advancedantenna:problemSet4:4c}

TODO.
\makeSubAnswer{}{advancedantenna:problemSet4:4d}

TODO.
\makeSubAnswer{}{advancedantenna:problemSet4:4e}

TODO.
\makeSubAnswer{}{advancedantenna:problemSet4:4f}

TODO.
}
