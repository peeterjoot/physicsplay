%
% Copyright � 2016 Peeter Joot.  All Rights Reserved.
% Licenced as described in the file LICENSE under the root directory of this GIT repository.
%
%{
\newcommand{\authorname}{Peeter Joot}
\newcommand{\email}{peeterjoot@protonmail.com}
\newcommand{\basename}{FIXMEbasenameUndefined}
\newcommand{\dirname}{notes/FIXMEdirnameUndefined/}

\renewcommand{\basename}{scalarFieldCreationOpCommutator}
\renewcommand{\dirname}{notes/phy2403/}
%\newcommand{\dateintitle}{}
%\newcommand{\keywords}{}

\newcommand{\authorname}{Peeter Joot}
\newcommand{\onlineurl}{http://sites.google.com/site/peeterjoot2/math2013/\basename.pdf}
\newcommand{\sourcepath}{\dirname\basename.tex}
\newcommand{\generatetitle}[1]{\chapter{#1}}

\newcommand{\vcsinfo}{%
\section*{}
\noindent{\color{DarkOliveGreen}{\rule{\linewidth}{0.1mm}}}
\paragraph{Document version}
%\paragraph{\color{Maroon}{Document version}}
{
\small
\begin{itemize}
\item Available online at:\\ 
\href{\onlineurl}{\onlineurl}
\item Git Repository: \input{./.revinfo/gitRepo.tex}
\item Source: \sourcepath
\item last commit: \input{./.revinfo/gitCommitString.tex}
\item commit date: \input{./.revinfo/gitCommitDate.tex}
\end{itemize}
}
}

%\PassOptionsToPackage{dvipsnames,svgnames}{xcolor}
\PassOptionsToPackage{square,numbers}{natbib}
\documentclass{scrreprt}

\usepackage[left=2cm,right=2cm]{geometry}
\usepackage[svgnames]{xcolor}
\usepackage{peeters_layout}

\usepackage{natbib}

\usepackage[
colorlinks=true,
bookmarks=false,
pdfauthor={\authorname, \email},
backref 
]{hyperref}

% http://tex.stackexchange.com/questions/75773/how-to-reference-problems-by-the-text-label-in-an-exercise-envioronment
\usepackage[english]{cleveref}
\crefname{Exercise}{exercise}{exercises}
\Crefname{Exercise}{Exercise}{Exercises}

\RequirePackage{titlesec}
\RequirePackage{ifthen}

% http://stackoverflow.com/questions/4932910/date-in-the-tabular-environment
\makeatletter
\let\insertdate\@date
\makeatother

\titleformat{\chapter}[display]
{\bfseries\Large}
{\color{DarkSlateGrey}\filleft \authorname
\ifthenelse{\isundefined{\studentnumber}}{}{\\ \studentnumber}
\ifthenelse{\isundefined{\email}}{}{\\ \email}
\ifthenelse{\isundefined{\dateintitle}}{}{\\ \insertdate}
%\ifthenelse{\isundefined{\coursename}}{}{\\ \coursename} % put in title instead.
}
{4ex}
{\color{DarkOliveGreen}{\titlerule}\color{Maroon}
\vspace{2ex}%
\filright}
[\vspace{2ex}%
\color{DarkOliveGreen}\titlerule
]

\newcommand{\beginArtWithToc}[0]{\begin{document}\tableofcontents}
\newcommand{\beginArtNoToc}[0]{\begin{document}}
\newcommand{\EndNoBibArticle}[0]{\end{document}}
\newcommand{\EndArticle}[0]{\bibliography{Bibliography}\bibliographystyle{plainnat}\end{document}}

% 
%\newcommand{\citep}[1]{\cite{#1}}

\colorSectionsForArticle



\usepackage{peeters_layout_exercise}
\usepackage{peeters_braket}
\usepackage{peeters_figures}

\beginArtNoToc

\generatetitle{Scalar field creation operator commutator}
%\chapter{Scalar field creation operator commutator}
%\label{chap:scalarFieldCreationOpCommutator}

\makeproblem{}{problem:scalarFieldCreationOpCommutator:1}{
In \citep{qftLectureNotes} it is stated that the creation operators of eq. 2.78

\begin{dmath}\label{eqn:scalarFieldCreationOpCommutator:20}
\alpha_k = \inv{2} \int \frac{d^3k}{(2\pi)^3} \lr{
\phi(x,0) + \frac{i}{\omega_k} \partial_0 \phi(x,0) 
} 
e^{-i \Bk \cdot \Bx }
\end{dmath}

associated with field operator \( \phi \) commute.  Verify that.

} % problem

\makeanswer{problem:scalarFieldCreationOpCommutator:1}{

\begin{dmath}\label{eqn:scalarFieldCreationOpCommutator:40}
\antisymmetric{\alpha_k}{\alpha_m}
=
\inv{4} 
\frac{1}{(2\pi)^6} 
\int d^3 x d^3 y
e^{-i \Bk \cdot \Bx }
e^{-i \Bm \cdot \By }
\antisymmetric
{
\phi(x,0) + \frac{i}{\omega_k} \partial_0 \phi(x,0) 
} 
{
\phi(y,0) + \frac{i}{\omega_m} \partial_0 \phi(y,0) 
} 
=
\frac{i}{4} 
\frac{1}{(2\pi)^6} 
\int d^3 x d^3 y
e^{-i \Bk \cdot \Bx }
e^{-i \Bm \cdot \By }
\lr{
\antisymmetric{\phi(x,0)}{\inv{\omega_m} \partial_0 \phi(y,0)}
+
\antisymmetric{\inv{\omega_k} \partial_0 \phi(x,0)}{\phi(y,0)}
} 
=
\frac{i}{4} 
\frac{1}{(2\pi)^6} 
\int d^3 x d^3 y
e^{-i \Bk \cdot \Bx }
e^{-i \Bm \cdot \By }
\lr{
\frac{i}{\omega_m} \delta^3(\Bx - \By)
-
\frac{i}{\omega_k} \delta^3(\Bx - \By)
} 
=
-\frac{1}{4} 
\frac{1}{(2\pi)^6} 
\int d^3 x 
e^{ -i (\Bk + \Bm) \cdot \Bx }
\lr{
\frac{1}{\omega_m} 
-
\frac{1}{\omega_k}
} 
=
-\frac{1}{4} 
\frac{1}{(2\pi)^3} 
\lr{
\frac{1}{\omega_m} 
-
\frac{1}{\omega_k}
} 
\delta^3(\Bk + \Bm)
=
-\frac{1}{4} 
\frac{1}{(2\pi)^3} 
\lr{
\frac{1}{\omega_{\Abs{-\Bk}}} 
-
\frac{1}{\omega_{\Abs{\Bk}}}
} 
\delta^3(\Bk + \Bm)
= 
0.
\end{dmath}
} % answer

%}
\EndArticle
