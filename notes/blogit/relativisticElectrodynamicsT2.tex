%
% Copyright � 2015 Peeter Joot.  All Rights Reserved.
% Licenced as described in the file LICENSE under the root directory of this GIT repository.
%
\documentclass[]{eliblog}

\usepackage{amsmath}
\usepackage{mathpazo}

%
% shorthand for bold symbols, convenient for vectors and matrices
%
\newcommand{\Ba}[0]{\mathbf{a}}
\newcommand{\Bb}[0]{\mathbf{b}}
\newcommand{\Bc}[0]{\mathbf{c}}
\newcommand{\Bd}[0]{\mathbf{d}}
\newcommand{\Be}[0]{\mathbf{e}}
\newcommand{\Bf}[0]{\mathbf{f}}
\newcommand{\Bg}[0]{\mathbf{g}}
\newcommand{\Bh}[0]{\mathbf{h}}
\newcommand{\Bi}[0]{\mathbf{i}}
\newcommand{\Bj}[0]{\mathbf{j}}
\newcommand{\Bk}[0]{\mathbf{k}}
\newcommand{\Bl}[0]{\mathbf{l}}
\newcommand{\Bm}[0]{\mathbf{m}}
\newcommand{\Bn}[0]{\mathbf{n}}
\newcommand{\Bo}[0]{\mathbf{o}}
\newcommand{\Bp}[0]{\mathbf{p}}
\newcommand{\Bq}[0]{\mathbf{q}}
\newcommand{\Br}[0]{\mathbf{r}}
\newcommand{\Bs}[0]{\mathbf{s}}
\newcommand{\Bt}[0]{\mathbf{t}}
\newcommand{\Bu}[0]{\mathbf{u}}
\newcommand{\Bv}[0]{\mathbf{v}}
\newcommand{\Bw}[0]{\mathbf{w}}
\newcommand{\Bx}[0]{\mathbf{x}}
\newcommand{\By}[0]{\mathbf{y}}
\newcommand{\Bz}[0]{\mathbf{z}}
\newcommand{\BA}[0]{\mathbf{A}}
\newcommand{\BB}[0]{\mathbf{B}}
\newcommand{\BC}[0]{\mathbf{C}}
\newcommand{\BD}[0]{\mathbf{D}}
\newcommand{\BE}[0]{\mathbf{E}}
\newcommand{\BF}[0]{\mathbf{F}}
\newcommand{\BG}[0]{\mathbf{G}}
\newcommand{\BH}[0]{\mathbf{H}}
\newcommand{\BI}[0]{\mathbf{I}}
\newcommand{\BJ}[0]{\mathbf{J}}
\newcommand{\BK}[0]{\mathbf{K}}
\newcommand{\BL}[0]{\mathbf{L}}
\newcommand{\BM}[0]{\mathbf{M}}
\newcommand{\BN}[0]{\mathbf{N}}
\newcommand{\BO}[0]{\mathbf{O}}
\newcommand{\BP}[0]{\mathbf{P}}
\newcommand{\BQ}[0]{\mathbf{Q}}
\newcommand{\BR}[0]{\mathbf{R}}
\newcommand{\BS}[0]{\mathbf{S}}
\newcommand{\BT}[0]{\mathbf{T}}
\newcommand{\BU}[0]{\mathbf{U}}
\newcommand{\BV}[0]{\mathbf{V}}
\newcommand{\BW}[0]{\mathbf{W}}
\newcommand{\BX}[0]{\mathbf{X}}
\newcommand{\BY}[0]{\mathbf{Y}}
\newcommand{\BZ}[0]{\mathbf{Z}}

\newcommand{\Bzero}[0]{\mathbf{0}}
\newcommand{\Btheta}[0]{\boldsymbol{\theta}}
\newcommand{\Btau}[0]{\boldsymbol{\tau}}
\newcommand{\Bomega}[0]{\boldsymbol{\omega}}

%
% shorthand for unit vectors
%
\newcommand{\acap}[0]{\hat{\Ba}}
\newcommand{\bcap}[0]{\hat{\Bb}}
\newcommand{\ccap}[0]{\hat{\Bc}}
\newcommand{\dcap}[0]{\hat{\Bd}}
\newcommand{\ecap}[0]{\hat{\Be}}
\newcommand{\fcap}[0]{\hat{\Bf}}
\newcommand{\gcap}[0]{\hat{\Bg}}
\newcommand{\hcap}[0]{\hat{\Bh}}
\newcommand{\icap}[0]{\hat{\Bi}}
\newcommand{\jcap}[0]{\hat{\Bj}}
\newcommand{\kcap}[0]{\hat{\Bk}}
\newcommand{\lcap}[0]{\hat{\Bl}}
\newcommand{\mcap}[0]{\hat{\Bm}}
\newcommand{\ncap}[0]{\hat{\Bn}}
\newcommand{\ocap}[0]{\hat{\Bo}}
\newcommand{\pcap}[0]{\hat{\Bp}}
\newcommand{\qcap}[0]{\hat{\Bq}}
\newcommand{\rcap}[0]{\hat{\Br}}
\newcommand{\scap}[0]{\hat{\Bs}}
\newcommand{\tcap}[0]{\hat{\Bt}}
\newcommand{\ucap}[0]{\hat{\Bu}}
\newcommand{\vcap}[0]{\hat{\Bv}}
\newcommand{\wcap}[0]{\hat{\Bw}}
\newcommand{\xcap}[0]{\hat{\Bx}}
\newcommand{\ycap}[0]{\hat{\By}}
\newcommand{\zcap}[0]{\hat{\Bz}}
\newcommand{\thetacap}[0]{\hat{\Btheta}}

%
% to write R^n and C^n in a distinguishable fashion.  Perhaps change this
% to the double lined characters upon figuring out how to do so.
%
\newcommand{\C}[1]{$\mathbb{C}^{#1}$}
\newcommand{\R}[1]{$\mathbb{R}^{#1}$}

%
% various generally useful helpers
%

% derivative of #1 wrt. #2:
\newcommand{\D}[2] {\frac {d#2} {d#1}}

\newcommand{\inv}[1]{\frac{1}{#1}}
\newcommand{\cross}[0]{\times}

\newcommand{\abs}[1]{\lvert{#1}\rvert}
\newcommand{\norm}[1]{\lVert{#1}\rVert}
\newcommand{\innerprod}[2]{\langle{#1}, {#2}\rangle}
\newcommand{\dotprod}[2]{{#1} \cdot {#2}}
\newcommand{\bdotprod}[2]{\left({#1} \cdot {#2}\right)}
\newcommand{\crossprod}[2]{{#1} \cross {#2}}
\newcommand{\tripleprod}[3]{\dotprod{\left(\crossprod{#1}{#2}\right)}{#3}}

\DeclareMathOperator{\Proj}{Proj}
\DeclareMathOperator{\Span}{span}
\DeclareMathOperator{\Sgn}{sgn}
\DeclareMathOperator{\Area}{Area}
\DeclareMathOperator{\Volume}{Volume}

%
% A few miscellaneous things specific to this document
%
\newcommand{\crossop}[1]{\crossprod{#1}{}}

% R2 vector.
\newcommand{\VectorTwo}[2]{
\begin{bmatrix}
 {#1} \\
 {#2}
\end{bmatrix}
}

\newcommand{\VectorN}[1]{
\begin{bmatrix}
{#1}_1 \\
{#1}_2 \\
\vdots \\
{#1}_N \\
\end{bmatrix}
}

\newcommand{\DETuvij}[4]{
\begin{vmatrix}
 {#1}_{#3} & {#1}_{#4} \\
 {#2}_{#3} & {#2}_{#4}
\end{vmatrix}
}

\newcommand{\DETuvwijk}[6]{
\begin{vmatrix}
 {#1}_{#4} & {#1}_{#5} & {#1}_{#6} \\
 {#2}_{#4} & {#2}_{#5} & {#2}_{#6} \\
 {#3}_{#4} & {#3}_{#5} & {#3}_{#6}
\end{vmatrix}
}

\newcommand{\DETuvwxijkl}[8]{
\begin{vmatrix}
 {#1}_{#5} & {#1}_{#6} & {#1}_{#7} & {#1}_{#8} \\
 {#2}_{#5} & {#2}_{#6} & {#2}_{#7} & {#2}_{#8} \\
 {#3}_{#5} & {#3}_{#6} & {#3}_{#7} & {#3}_{#8} \\
 {#4}_{#5} & {#4}_{#6} & {#4}_{#7} & {#4}_{#8} \\
\end{vmatrix}
}

%\newcommand{\DETuvwxyijklm}[10]{
%\begin{vmatrix}
% {#1}_{#6} & {#1}_{#7} & {#1}_{#8} & {#1}_{#9} & {#1}_{#10} \\
% {#2}_{#6} & {#2}_{#7} & {#2}_{#8} & {#2}_{#9} & {#2}_{#10} \\
% {#3}_{#6} & {#3}_{#7} & {#3}_{#8} & {#3}_{#9} & {#3}_{#10} \\
% {#4}_{#6} & {#4}_{#7} & {#4}_{#8} & {#4}_{#9} & {#4}_{#10} \\
% {#5}_{#6} & {#5}_{#7} & {#5}_{#8} & {#5}_{#9} & {#5}_{#10}
%\end{vmatrix}
%}

% R3 vector.
\newcommand{\VectorThree}[3]{
\begin{bmatrix}
 {#1} \\
 {#2} \\
 {#3}
\end{bmatrix}
}



\author{Peeter Joot}
\email{peeter.joot@gmail.com}

%\documentclass[]{eliblogwidescreen}

\usepackage{amsmath}
\usepackage{mathpazo}

%
% shorthand for bold symbols, convenient for vectors and matrices
%
\newcommand{\Ba}[0]{\mathbf{a}}
\newcommand{\Bb}[0]{\mathbf{b}}
\newcommand{\Bc}[0]{\mathbf{c}}
\newcommand{\Bd}[0]{\mathbf{d}}
\newcommand{\Be}[0]{\mathbf{e}}
\newcommand{\Bf}[0]{\mathbf{f}}
\newcommand{\Bg}[0]{\mathbf{g}}
\newcommand{\Bh}[0]{\mathbf{h}}
\newcommand{\Bi}[0]{\mathbf{i}}
\newcommand{\Bj}[0]{\mathbf{j}}
\newcommand{\Bk}[0]{\mathbf{k}}
\newcommand{\Bl}[0]{\mathbf{l}}
\newcommand{\Bm}[0]{\mathbf{m}}
\newcommand{\Bn}[0]{\mathbf{n}}
\newcommand{\Bo}[0]{\mathbf{o}}
\newcommand{\Bp}[0]{\mathbf{p}}
\newcommand{\Bq}[0]{\mathbf{q}}
\newcommand{\Br}[0]{\mathbf{r}}
\newcommand{\Bs}[0]{\mathbf{s}}
\newcommand{\Bt}[0]{\mathbf{t}}
\newcommand{\Bu}[0]{\mathbf{u}}
\newcommand{\Bv}[0]{\mathbf{v}}
\newcommand{\Bw}[0]{\mathbf{w}}
\newcommand{\Bx}[0]{\mathbf{x}}
\newcommand{\By}[0]{\mathbf{y}}
\newcommand{\Bz}[0]{\mathbf{z}}
\newcommand{\BA}[0]{\mathbf{A}}
\newcommand{\BB}[0]{\mathbf{B}}
\newcommand{\BC}[0]{\mathbf{C}}
\newcommand{\BD}[0]{\mathbf{D}}
\newcommand{\BE}[0]{\mathbf{E}}
\newcommand{\BF}[0]{\mathbf{F}}
\newcommand{\BG}[0]{\mathbf{G}}
\newcommand{\BH}[0]{\mathbf{H}}
\newcommand{\BI}[0]{\mathbf{I}}
\newcommand{\BJ}[0]{\mathbf{J}}
\newcommand{\BK}[0]{\mathbf{K}}
\newcommand{\BL}[0]{\mathbf{L}}
\newcommand{\BM}[0]{\mathbf{M}}
\newcommand{\BN}[0]{\mathbf{N}}
\newcommand{\BO}[0]{\mathbf{O}}
\newcommand{\BP}[0]{\mathbf{P}}
\newcommand{\BQ}[0]{\mathbf{Q}}
\newcommand{\BR}[0]{\mathbf{R}}
\newcommand{\BS}[0]{\mathbf{S}}
\newcommand{\BT}[0]{\mathbf{T}}
\newcommand{\BU}[0]{\mathbf{U}}
\newcommand{\BV}[0]{\mathbf{V}}
\newcommand{\BW}[0]{\mathbf{W}}
\newcommand{\BX}[0]{\mathbf{X}}
\newcommand{\BY}[0]{\mathbf{Y}}
\newcommand{\BZ}[0]{\mathbf{Z}}

\newcommand{\Bzero}[0]{\mathbf{0}}
\newcommand{\Btheta}[0]{\boldsymbol{\theta}}
\newcommand{\Btau}[0]{\boldsymbol{\tau}}
\newcommand{\Bomega}[0]{\boldsymbol{\omega}}

%
% shorthand for unit vectors
%
\newcommand{\acap}[0]{\hat{\Ba}}
\newcommand{\bcap}[0]{\hat{\Bb}}
\newcommand{\ccap}[0]{\hat{\Bc}}
\newcommand{\dcap}[0]{\hat{\Bd}}
\newcommand{\ecap}[0]{\hat{\Be}}
\newcommand{\fcap}[0]{\hat{\Bf}}
\newcommand{\gcap}[0]{\hat{\Bg}}
\newcommand{\hcap}[0]{\hat{\Bh}}
\newcommand{\icap}[0]{\hat{\Bi}}
\newcommand{\jcap}[0]{\hat{\Bj}}
\newcommand{\kcap}[0]{\hat{\Bk}}
\newcommand{\lcap}[0]{\hat{\Bl}}
\newcommand{\mcap}[0]{\hat{\Bm}}
\newcommand{\ncap}[0]{\hat{\Bn}}
\newcommand{\ocap}[0]{\hat{\Bo}}
\newcommand{\pcap}[0]{\hat{\Bp}}
\newcommand{\qcap}[0]{\hat{\Bq}}
\newcommand{\rcap}[0]{\hat{\Br}}
\newcommand{\scap}[0]{\hat{\Bs}}
\newcommand{\tcap}[0]{\hat{\Bt}}
\newcommand{\ucap}[0]{\hat{\Bu}}
\newcommand{\vcap}[0]{\hat{\Bv}}
\newcommand{\wcap}[0]{\hat{\Bw}}
\newcommand{\xcap}[0]{\hat{\Bx}}
\newcommand{\ycap}[0]{\hat{\By}}
\newcommand{\zcap}[0]{\hat{\Bz}}
\newcommand{\thetacap}[0]{\hat{\Btheta}}

%
% to write R^n and C^n in a distinguishable fashion.  Perhaps change this
% to the double lined characters upon figuring out how to do so.
%
\newcommand{\C}[1]{$\mathbb{C}^{#1}$}
\newcommand{\R}[1]{$\mathbb{R}^{#1}$}

%
% various generally useful helpers
%

% derivative of #1 wrt. #2:
\newcommand{\D}[2] {\frac {d#2} {d#1}}

\newcommand{\inv}[1]{\frac{1}{#1}}
\newcommand{\cross}[0]{\times}

\newcommand{\abs}[1]{\lvert{#1}\rvert}
\newcommand{\norm}[1]{\lVert{#1}\rVert}
\newcommand{\innerprod}[2]{\langle{#1}, {#2}\rangle}
\newcommand{\dotprod}[2]{{#1} \cdot {#2}}
\newcommand{\bdotprod}[2]{\left({#1} \cdot {#2}\right)}
\newcommand{\crossprod}[2]{{#1} \cross {#2}}
\newcommand{\tripleprod}[3]{\dotprod{\left(\crossprod{#1}{#2}\right)}{#3}}

\DeclareMathOperator{\Proj}{Proj}
\DeclareMathOperator{\Span}{span}
\DeclareMathOperator{\Sgn}{sgn}
\DeclareMathOperator{\Area}{Area}
\DeclareMathOperator{\Volume}{Volume}

%
% A few miscellaneous things specific to this document
%
\newcommand{\crossop}[1]{\crossprod{#1}{}}

% R2 vector.
\newcommand{\VectorTwo}[2]{
\begin{bmatrix}
 {#1} \\
 {#2}
\end{bmatrix}
}

\newcommand{\VectorN}[1]{
\begin{bmatrix}
{#1}_1 \\
{#1}_2 \\
\vdots \\
{#1}_N \\
\end{bmatrix}
}

\newcommand{\DETuvij}[4]{
\begin{vmatrix}
 {#1}_{#3} & {#1}_{#4} \\
 {#2}_{#3} & {#2}_{#4}
\end{vmatrix}
}

\newcommand{\DETuvwijk}[6]{
\begin{vmatrix}
 {#1}_{#4} & {#1}_{#5} & {#1}_{#6} \\
 {#2}_{#4} & {#2}_{#5} & {#2}_{#6} \\
 {#3}_{#4} & {#3}_{#5} & {#3}_{#6}
\end{vmatrix}
}

\newcommand{\DETuvwxijkl}[8]{
\begin{vmatrix}
 {#1}_{#5} & {#1}_{#6} & {#1}_{#7} & {#1}_{#8} \\
 {#2}_{#5} & {#2}_{#6} & {#2}_{#7} & {#2}_{#8} \\
 {#3}_{#5} & {#3}_{#6} & {#3}_{#7} & {#3}_{#8} \\
 {#4}_{#5} & {#4}_{#6} & {#4}_{#7} & {#4}_{#8} \\
\end{vmatrix}
}

%\newcommand{\DETuvwxyijklm}[10]{
%\begin{vmatrix}
% {#1}_{#6} & {#1}_{#7} & {#1}_{#8} & {#1}_{#9} & {#1}_{#10} \\
% {#2}_{#6} & {#2}_{#7} & {#2}_{#8} & {#2}_{#9} & {#2}_{#10} \\
% {#3}_{#6} & {#3}_{#7} & {#3}_{#8} & {#3}_{#9} & {#3}_{#10} \\
% {#4}_{#6} & {#4}_{#7} & {#4}_{#8} & {#4}_{#9} & {#4}_{#10} \\
% {#5}_{#6} & {#5}_{#7} & {#5}_{#8} & {#5}_{#9} & {#5}_{#10}
%\end{vmatrix}
%}

% R3 vector.
\newcommand{\VectorThree}[3]{
\begin{bmatrix}
 {#1} \\
 {#2} \\
 {#3}
\end{bmatrix}
}



\author{Peeter Joot}
\email{peeter.joot@gmail.com}


\chapter{PHY450H1S.  Relativistic Electrodynamics Tutorial 2 (TA: Simon Freedman).  Two worked problems.}
\label{chap:relativisticElectrodynamicsT2}
%\useCCL
\blogpage{http://sites.google.com/site/peeterjoot/math2011/relativisticElectrodynamicsT2.pdf}
\date{Jan 27, 2011}
\revisionInfo{relativisticElectrodynamicsT2.tex}

\beginArtWithToc
%\beginArtNoToc

\section{What we will discuss.}

\begin{itemize}
\item 4-vectors: position, velocity, acceleration
\item non-inertial observers
\end{itemize}

\section{Problem 1.}

\subsection{Statement}
A particle moves on the x-axis along a world line described by

\begin{align}\label{eqn:relativisticElectrodynamicsT2:10}
ct(\sigma) &= \inv{a} \sinh(\sigma) \\
x(\sigma) &= \inv{a} \cosh(\sigma)
\end{align}

where the dimension of the constant $[a] = \inv{L}$, is inverse length, and our parameter takes any values $-\infty < \sigma < \infty$.

Find $x^i(\tau)$, $u^i(\tau)$, $a^i(\tau)$.

\subsection{Solution}

First note that we can re-parametrize $x = x^1$ in terms of $t$.  That is

\begin{align*}
\cosh(\sigma)
&= \sqrt{1 + \sinh^2(\sigma)}  \\
&= \sqrt{ 1 + (act)^2 }  \\
&= a \sqrt{ a^{-2} + (ct)^2 }
\end{align*}

So

\begin{equation}\label{eqn:relativisticElectrodynamicsT2:20}
x(t) = \sqrt{ a^{-2} + (ct)^2 }
\end{equation}

Squaring and rearranging, shows that our particle is moving through half of a hyperbolic arc in spacetime (two such paths are possible, one for strictly positive $x$ and one for strictly negative).

\begin{equation}\label{eqn:relativisticElectrodynamicsT2:30}
x^2 - (ct)^2 = a^{-2}
\end{equation}

Observe that the asymptotes of this curve are the lightcone boundaries.  Taking derivatives we have

\begin{equation}\label{eqn:relativisticElectrodynamicsT2:40}
2 x \frac{dx}{d(ct)} -2 (ct) = 0,
\end{equation}

and rearranging we have

\begin{align*}
\frac{dx}{d(ct)}
&= \frac{c t}{x} \\
&= \frac{ct}{\sqrt{(ct)^2 + a^{-2}}} \\
&\rightarrow \pm 1
\end{align*}

\paragraph{Is this timelike?}

Let's compute the interval between two worldpoints.  That is

\begin{align*}
s_{12}^2
&= (ct(\sigma_1) - ct(\sigma_2))^2 - (x(\sigma_1) - x(\sigma_2))^2  \\
&= a^{-2} (\sinh \sigma_1 - \sinh \sigma_2)^2 - a^{-2} (\cosh\sigma_1 - \cosh\sigma_2)^2 \\
&= 2 a^{-2} \left( -1 - \sinh\sigma_1 \sinh \sigma_2 + \cosh\sigma_1 \cosh\sigma_2 \right) \\
&= 2 a^{-2} \left( \cosh( \sigma_2 - \sigma_1) -1 \right) \ge 0
\end{align*}

Yes, this is timelike.  That's what we want for a particle that is realistic moving along a worldline in spacetime.  If the spacetime interval between any two points were to be negative, we would be talking about something of tachyon like hypothetical nature.

Our first task is to compute $x^i(\tau)$.  We have $x^i(\sigma)$ so we need the relation between our proper length $\tau$ and the worldline parameter $\sigma$.  Such a relation is implicitly provided by the differential spacetime interval

\begin{align*}
\left(\frac{d\tau}{d\sigma}\right)^2
&= \inv{c^2} \left(\frac{ds}{d\sigma}\right)^2 \\
&= \inv{c^2} \left(
\left( \frac{d(x^0)}{d\sigma}\right)^2
-\left( \frac{d(x^1)}{d\sigma}\right)^2
\right) \\
&= \inv{c^2} \left( a^{-2} \cosh^2 \sigma - a^{-2} \sinh^2 \sigma \right) \\
&= \inv{a^2 c^2}.
\end{align*}

Taking roots we have
\begin{equation}\label{eqn:relativisticElectrodynamicsT2:50}
\frac{d\tau}{d\sigma} = \pm \inv{a c},
\end{equation}

We take the positive root, so that the worldline is traversed in a strictly increasing fashion, and then integrate once

\begin{equation}\label{eqn:relativisticElectrodynamicsT2:60}
\tau = \inv{ac} \sigma + \tau_s.
\end{equation}

We are free to let $\tau_s = 0$, effectively starting our proper time at $t=0$.

\begin{equation}\label{eqn:relativisticElectrodynamicsT2:70}
x^i(\tau) = ( a^{-1} \sinh( a c \tau), a^{-1} \cosh( a c \tau ), 0, 0 )
\end{equation}

As noted already this is a hyperbola (or degenerate hyperboloid) in spacetime, with asymptote 1 (ie: approaching the speed of light).

The next computational task is now simple.
\begin{equation}\label{eqn:relativisticElectrodynamicsT2:77}
u^i
= \frac{dx^i}{d\tau} 
= c ( \cosh( a c \tau ), \sinh( a c \tau ), 0, 0) \\
\end{equation}

Is this light like or time like?  We can answer this by considering the four vector square

\begin{equation}\label{eqn:relativisticElectrodynamicsT2:80}
u \cdot u 
\end{equation}

What is a light like or a time like vector?

Recall that we have defined lightlike, spacelike, and timelike intervals.  A lightlike interval between two world points had $(ct - c\tilde{t})^2 - (\Bx -\tilde{\Bx})^2 = 0$, whereas a timelike interval had $(ct - c\tilde{t})^2 - (\Bx -\tilde{\Bx})^2 > 0$.  Taking the vector $(c \tilde{t}, \tilde{\Bx})$ as the origin, the distance to any single four vector from the origin is then just that vector's square, so it logically makes sense to call a vector light like if it has a zero square, and time like if it has a positive square.

Consider the very simplest example of a time like trajectory, that of a particle at rest at a fixed position $\Bx_0$.  Such a particle's worldline is

\begin{equation}\label{eqn:relativisticElectrodynamicsT2:71}
X = ( c t, \Bx_0 )
\end{equation}

While we interpret $t$ here as time, it functions as a parametrization of the curve, just as $\sigma$ does in this question.  If we want to compute the proper time interval between two points on this worldline we have

\begin{align*}
\tau_b - \tau_0 
&=
\inv{c} \int_{\lambda = t_a}^{t_b} \sqrt{ \frac{dX(\lambda)}{d\lambda} \cdot \frac{dX(\lambda)}{d\lambda} } d\lambda \\
&=
\inv{c} \int_{\lambda = t_a}^{t_b} \sqrt{ (c, 0)^2 } d\lambda \\
&=
\inv{c} \int_{\lambda = t_a}^{t_b} c d\lambda \\
&= t_b - t_a
\end{align*}

The conclusion (arrived at the hard way, but methodologically) is that proper time on this worldline is just the parameter $t$ itself.

Now examine the proper velocity for this trajectory.  This is

\begin{equation}\label{eqn:relativisticElectrodynamicsT2:72}
u = \frac{dX(\tau)}{d\tau} = (c, 0, 0, 0)
\end{equation}

We can compute the dot product $u \cdot u = c^2 > 0$ easily enough, and in this case for the particle at rest (but moving in time) we see that this four-vector velocity does have a time like separation from the origin, and it therefore makes sense to label the four-velocity vector itself as time like.

Now, let's return to our non-inertial system.  Is our four velocity vector time like?  Let's compute it's square to check

\begin{equation}\label{eqn:relativisticElectrodynamicsT2:90}
u \cdot u = c^2 ( \cosh^2 - \sinh^2 ) = c^2 > 0
\end{equation}

Yes, it is timelike.

Now, let's calculate our spatial velocity

\begin{equation}\label{eqn:relativisticElectrodynamicsT2:100}
v^\alpha
= \frac{dx^\alpha}{dt}
=
\frac{dx^\alpha}{d\tau} \frac{d\tau}{dt}
\end{equation}

Since $ct = \sinh( a c \tau )/a$ we have

\begin{equation}\label{eqn:relativisticElectrodynamicsT2:110}
c = \inv{a} a c \cosh( a c \tau ) \frac{d\tau}{dt},
\end{equation}

or
\begin{equation}\label{eqn:relativisticElectrodynamicsT2:110b}
\frac{d\tau}{dt} = \inv{\cosh( a c \tau) }
\end{equation}

Similarily from \ref{eqn:relativisticElectrodynamicsT2:70}, we have

\begin{equation}\label{eqn:relativisticElectrodynamicsT2:120}
\frac{dx^1}{d\tau} = c \sinh( a c \tau )
\end{equation}

So our spatial velocity is $\sinh/\cosh = \tanh$, and we have

\begin{equation}\label{eqn:relativisticElectrodynamicsT2:130}
v^\alpha = (c \tanh( a c \tau), 0, 0)
\end{equation}

Note how tricky this index notation is.  Four our four vector velocity we use $u^i = dx^i/d\tau$, whereas our spatial velocity is distinguished by a change of letter as well as the indexes, so when we write $v^\alpha$ we are taking our derivatives with respect to time and not proper time (i.e. $v^\alpha = dx^\alpha/dt$).

\subsection{ What is our four-acceleration? }

From \ref{eqn:relativisticElectrodynamicsT2:77}, we have

\begin{align*}
w^i (\tau) = \frac{ du^i }{d\tau} = a c^2 x^i(\tau)
\end{align*}

Observe that our four-velocity square is

\begin{equation}\label{eqn:relativisticElectrodynamicsT2:78}
w \cdot w = a^2 c^2 a^{-1} (-1)
\end{equation}

What does this really signify?  Think on this.  A check to verify that things are okay is to see if this four-acceleration is orthogonal to our four-velocity as expected

\begin{align*}
w \cdot u 
&= 
a c^2 ( a^{-1} \sinh( a c \tau), a^{-1} \cosh( a c \tau ), 0, 0 ) \cdot c ( \cosh( a c \tau ), \sinh( a c \tau ), 0, 0) \\
&=
c^3 ( \sinh(a c \tau)\cosh(a c \tau) - \cosh(a c \tau) \sinh(a c \tau) ) \\
&=
0
\end{align*}

A last beastie that we can compute is the spatial acceleration.

\begin{align*}
a^\alpha 
&= \frac{du^\alpha}{dt} \\
&= \frac{dv^\alpha}{d\tau} \frac{d \tau}{dt} \\
&= \frac{a c^2}{\cosh^2(a c \tau)} \inv{\cosh(a c \tau) } \\
&= \frac{a c^2}{\cosh^3(a c \tau)} \\
%&= \frac{a c^2}{(a x)^3} \\
%&= \frac{c^2}{a^2 x^3}
\end{align*}

Summarizing

\begin{align}\label{eqn:relativisticElectrodynamicsT2:150}
x^i(\tau) &= \left( a^{-1} \sinh( a c \tau), a^{-1} \cosh( a c \tau ), 0, 0 \right) \\
u^i(\tau) &= c \left( \cosh( a c \tau ), \sinh( a c \tau ), 0, 0\right) \\
v^\alpha(\tau) &= \left( c \tanh(a c \tau), 0, 0 \right) \\
w^i(\tau) &= a c^2 x^i(\tau) \\
a^\alpha(\tau) &= \left( \frac{a c^2}{\cosh^3 (a c \tau)}, 0, 0 \right) \\
\end{align}

\section{Problem 2.  Local observers.}

Observations are made of either the three-vector, or the time like components of four-vectors, since these are the quantities that we can measure from our local observer frame.  This is something that can be viewed in an approximate sense as being inertial, provided that we ignore the earth's rotation, the rotation around the solar system, the rotation of the solar system in the galaxy, the rotation of the galaxy in the local cluster, and so forth.  Provided none of these are changing too fast relative to our measurements, we can make the inertial approximation.

Example.  If we want to measure energy, it is the timelike component of the momentum.

\begin{equation}\label{eqn:relativisticElectrodynamicsT2:500}
E = c p^0
\end{equation}

PICTURE:  Let's imagine a moving worldline in three dimensions.  We can setup a frame and associated basis along the worldline of the particle, as well as a frame and basis for the stationary observer.

In class Simon used notation like $\{ e_{\hat{o}}^i \}$, and $\{ e_{\hat{a}}^i \}$, but also used $e_{\hat{0}}^i$, $e_{\hat{1}}^i$, $e_{\hat{2}}^i$, $e_{\hat{3}}^i$.  It was fairly clear by the context what was meant, but lets avoid any more than one index at a time, and write $\{ e^i \}$ for the frame moving along the worldline, and $\{ \gamma^i \}$ for the stationary frame.  The use of $\gamma^i$ as a spacetime basis is borrowed from \cite{doran2003gap}.

For any timelike four-vector worldline we have a four-vector velocity of magnitude $c$, so we are free to define a timelike basis vector for our moving frame as

\begin{equation}\label{eqn:relativisticElectrodynamicsT2:510}
e^0 = u / c
\end{equation}

going back to the first problem for $u^i$ we have

\begin{equation*}
e^0 = ( \cosh( a c t ), \sinh( a c t), 0, 0 ) 
\end{equation*}

We are free to pick spatial unit vectors perpendicular to this, so for the $y$ and $z$ components it is natural to use
\begin{align*}
e^2 &= ( 0, 0, 1, 0 ) \\
e^3 &= ( 0, 0, 0, 1 )
\end{align*}

We need one more, that's perpendicular to each of the above.  By inspection one can pick

\begin{equation*}
e^1 = ( \sinh( a c t ), \cosh( a c t), 0, 0) 
\end{equation*}

Did Simon use any other principle to define this last beastie?  I missed it if he did.  I see that this happens to be the unit vector proportional to $x^i$.

\subsection{Consider the stationary observer.}

For a stationary observer, our worldline and four velocity respectively, for some constant $\Bx_0$ is

\begin{align}\label{eqn:relativisticElectrodynamicsT2:600}
X &= ( ct, \Bx_0 ) \\
\frac{dX}{d\tau} &= c ( 1, \Bzero) 
\end{align}

Our time like unit vector is very simple

\begin{equation}\label{eqn:relativisticElectrodynamicsT2:610}
\gamma^0 = \inv{c} \frac{dX}{d\tau} = ( 1, \Bzero ) 
\end{equation}

For the spatial unit vectors we have many choices.  One would be aligned from the origin to the position vector

\begin{equation}\label{eqn:relativisticElectrodynamicsT2:620}
\gamma^1 = \left( 0, \frac{\Bx}{\Abs{\Bx}} \right),
\end{equation}

with $\gamma^2$ and $\gamma^3$ oriented in any pair of mutually perpendicular spatial directions.  Another option would be simply pick a $\gamma$ for each of the normal Euclidean basis directions

\begin{align}\label{eqn:relativisticElectrodynamicsT2:630}
\gamma^1 &= ( 0, 1, 0, 0 ) \\
\gamma^2 &= ( 0, 0, 1, 0 ) \\
\gamma^3 &= ( 0, 0, 0, 1 )
\end{align}

Observe, that just as in Geometric Algebra, we have $\gamma^\alpha \cdot \gamma^\alpha = -1$ (and $\gamma^0 \cdot \gamma^0 = 1$).

\subsection{Consider an inertial observer.}

Now lets consider a slightly more complex case, where an observer is moving with some constant velocity $\BV = c \Bbeta$.  Our worldline is

\begin{equation}\label{eqn:relativisticElectrodynamicsT2:700}
X = ( ct, \Bx_0 + \Bbeta c t) .
\end{equation}

Let's calculate the four velocity.  We have

\begin{equation}\label{eqn:relativisticElectrodynamicsT2:710}
\frac{dX}{dt} = c ( 1, \Bbeta ).
\end{equation}

From this our proper time is

\begin{equation}\label{eqn:relativisticElectrodynamicsT2:720}
\tau = \inv{c} \int_0^t c \sqrt{ (1, \Bbeta)^2 } dt = \sqrt{1 - \Beta^2} t.
\end{equation}

Our worldline and four-velocity, parameterized in terms of proper time, with $\gamma = (1 - \Bbeta^2)^{-1/2}$, are then

\begin{align}\label{eqn:relativisticElectrodynamicsT2:730}
X &= ( \gamma c\tau, \Bx_0 + \gamma \Bbeta c \tau) \\
u &= \gamma c ( 1, \Bbeta )
\end{align}

From this our time like unit vector is

\begin{equation}\label{eqn:relativisticElectrodynamicsT2:740}
e^0 = \gamma ( 1, \Bbeta )
\end{equation}

We observe that this has the desired time like property, $(e^0)^2 = 1 > 0$.

%%%
%%%
%%%
%%%
%%%\begin{equation}\label{eqn:relativisticElectrodynamicsT2:n}
%%%p^0 = p \cdot x^0
%%%\end{equation}
%%%
%%%\begin{equation}\label{eqn:relativisticElectrodynamicsT2:n}
%%%E_obs = c p \cdot e_ohat \equiv c p^ohat
%%%\end{equation}
%%%
%%%In the observers reference frame
%%%
%%%\begin{equation}\label{eqn:relativisticElectrodynamicsT2:n}
%%%{p'}^i = (mc,0, 0, 0)
%%%\end{equation}
%%%
%%%\begin{equation}\label{eqn:relativisticElectrodynamicsT2:n}
%%%{u'}^i_obs = c \gamma (( 1, v/c , 0, 0)
%%%\end{equation}
%%%
%%%\begin{equation}\label{eqn:relativisticElectrodynamicsT2:n}
%%%u^i_obj c (1, 0, 0, 0)
%%%\end{equation}
%%%
%%%\begin{equation}\label{eqn:relativisticElectrodynamicsT2:n}
%%%{u'}_obs^i =
%%%\begin{bmatrix}
%%%\gamma & \gamma v/c  & 0 & 0 \\
%%%\gamma v/c  & \gamma  & 0 & 0 \\
%%%0 & 0 & 0 & 0 \\
%%%0 & 0 & 0 & 0
%%%\end{bmatrix}
%%%\end{equation}
%%%
%%%\begin{equation}\label{eqn:relativisticElectrodynamicsT2:n}
%%%p^ohat = \gamma m c
%%%\end{equation}
%%%
%%%Suppose we have a star far away.  What is the frequency of the light emitted
%%%
%%%\begin{equation}\label{eqn:relativisticElectrodynamicsT2:n}
%%%\hat{\omega} = \omega e^{- a c \tau }
%%%\end{equation}
%%%
%%%FIXME: derive.
%%%
%%%where $\omega$ is the emitted frequency.
%%%
%%%FIXME: This implied an elapsed time before the star would no longer be visible?

\EndArticle
