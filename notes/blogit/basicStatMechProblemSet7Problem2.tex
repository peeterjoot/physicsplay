%
% Copyright � 2013 Peeter Joot.  All Rights Reserved.
% Licenced as described in the file LICENSE under the root directory of this GIT repository.
%
\makeproblem{Estimating the BEC transition temperature}{basicStatMech:problemSet7:2}{ 
\makesubproblem{}{basicStatMech:problemSet7:2a}

Find data for the atomic mass of liquid ${}^4$ He and its density at ambient atmospheric pressure and hence estimate its BEC temperature assuming interactions are unimportant (even though this assumption is a very bad one!). 

\makesubproblem{}{basicStatMech:problemSet7:2b}

For dilute atomic gases of the sort used in Professor 
Thywissen's lab
, one typically has a cloud of $10^6$ atoms confined to an approximate cubic region with linear dimension 1 $\mu\,m$.  Find the density - it is pretty low, so interactions can be assumed to be extremely weak.

\makesubproblem{}{basicStatMech:problemSet7:2c}

Assuming these are ${}^{87}$ Rb atoms, estimate the BEC transition temperature.

} % makeproblem

\makeanswer{basicStatMech:problemSet7:2}{ 
\makeSubAnswer{}{basicStatMech:problemSet7:2a}

% http://en.wikipedia.org/wiki/Liquid_helium
With an atomic weight of 4.0026, the mass in grams for one atom of Helium is

\begin{dmath}\label{eqn:basicStatMechProblemSet7Problem2:20}
4.0026 \,\text{amu} \times \frac{\text{g}}{6.022 \times 10^{23} \text{amu}}
=
6.64 \times 10^{-24} \text{g}
=
6.64 \times 10^{-27} \text{kg}.
\end{dmath}

With the density of liquid He-4, at 5.2K (boiling point): 125 grams per liter, the number density is

\begin{dmath}\label{eqn:basicStatMechProblemSet7Problem2:40}
\rho 
= \frac{\text{mass}}{\text{volume}} \times \inv{\text{mass of one He atom}}
= \frac{125 \text{g}}{10^{-3} m^3} \times \inv{6.64 \times 10^{-24} g}
= \frac{125 \text{g}}{10^{-3} m^3} \times \inv{6.64 \times 10^{-24} g}
= 1.88 \times 10^{28} m^{-3}
\end{dmath}

In class the $T_{\mathrm{BEC}}$ was found to be
\begin{dmath}\label{eqn:basicStatMechProblemSet7Problem2:60}
T_{\mathrm{BEC}} 
= 
\inv{\kB}
\lr{ \frac{\rho}{\zeta(3/2)} }
^{2/3} \frac{ 2 \pi \Hbar^2}{M}
= 
\inv{1.3806488 \times 10^{-23} m^2 kg/s^2 }
\lr{ \frac{\rho}{ 2.61238 } }
^{2/3} \frac{ 2 \pi (1.05457173 � 10^{-34} m^2 kg / s)^2}{M}
=
\end{dmath}

\makeSubAnswer{}{basicStatMech:problemSet7:2b}

The density of the gasses in Thywissen's lab, we have

\begin{dmath}\label{eqn:basicStatMechProblemSet7Problem2:80}
\rho 
= \frac{10^6 \times 6.64 \times 10^{-24} \text{g}}{(10^{-6} \text{m})^3}
= 6.64 g/m^3
= 6.64 \times 10^{-3} g/L
\end{dmath}

TODO.

\makeSubAnswer{}{basicStatMech:problemSet7:2c}

TODO.
}
