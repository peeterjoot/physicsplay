%
% Copyright � 2016 Peeter Joot.  All Rights Reserved.
% Licenced as described in the file LICENSE under the root directory of this GIT repository.
%
\newcommand{\authorname}{Peeter Joot}
\newcommand{\email}{peeterjoot@protonmail.com}
\newcommand{\basename}{FIXMEbasenameUndefined}
\newcommand{\dirname}{notes/FIXMEdirnameUndefined/}

\renewcommand{\basename}{uwaves3}
\renewcommand{\dirname}{notes/ece1236/}
\newcommand{\keywords}{ECE1236H}
\newcommand{\authorname}{Peeter Joot}
\newcommand{\onlineurl}{http://sites.google.com/site/peeterjoot2/math2013/\basename.pdf}
\newcommand{\sourcepath}{\dirname\basename.tex}
\newcommand{\generatetitle}[1]{\chapter{#1}}

\newcommand{\vcsinfo}{%
\section*{}
\noindent{\color{DarkOliveGreen}{\rule{\linewidth}{0.1mm}}}
\paragraph{Document version}
%\paragraph{\color{Maroon}{Document version}}
{
\small
\begin{itemize}
\item Available online at:\\ 
\href{\onlineurl}{\onlineurl}
\item Git Repository: \input{./.revinfo/gitRepo.tex}
\item Source: \sourcepath
\item last commit: \input{./.revinfo/gitCommitString.tex}
\item commit date: \input{./.revinfo/gitCommitDate.tex}
\end{itemize}
}
}

%\PassOptionsToPackage{dvipsnames,svgnames}{xcolor}
\PassOptionsToPackage{square,numbers}{natbib}
\documentclass{scrreprt}

\usepackage[left=2cm,right=2cm]{geometry}
\usepackage[svgnames]{xcolor}
\usepackage{peeters_layout}

\usepackage{natbib}

\usepackage[
colorlinks=true,
bookmarks=false,
pdfauthor={\authorname, \email},
backref 
]{hyperref}

% http://tex.stackexchange.com/questions/75773/how-to-reference-problems-by-the-text-label-in-an-exercise-envioronment
\usepackage[english]{cleveref}
\crefname{Exercise}{exercise}{exercises}
\Crefname{Exercise}{Exercise}{Exercises}

\RequirePackage{titlesec}
\RequirePackage{ifthen}

% http://stackoverflow.com/questions/4932910/date-in-the-tabular-environment
\makeatletter
\let\insertdate\@date
\makeatother

\titleformat{\chapter}[display]
{\bfseries\Large}
{\color{DarkSlateGrey}\filleft \authorname
\ifthenelse{\isundefined{\studentnumber}}{}{\\ \studentnumber}
\ifthenelse{\isundefined{\email}}{}{\\ \email}
\ifthenelse{\isundefined{\dateintitle}}{}{\\ \insertdate}
%\ifthenelse{\isundefined{\coursename}}{}{\\ \coursename} % put in title instead.
}
{4ex}
{\color{DarkOliveGreen}{\titlerule}\color{Maroon}
\vspace{2ex}%
\filright}
[\vspace{2ex}%
\color{DarkOliveGreen}\titlerule
]

\newcommand{\beginArtWithToc}[0]{\begin{document}\tableofcontents}
\newcommand{\beginArtNoToc}[0]{\begin{document}}
\newcommand{\EndNoBibArticle}[0]{\end{document}}
\newcommand{\EndArticle}[0]{\bibliography{Bibliography}\bibliographystyle{plainnat}\end{document}}

% 
%\newcommand{\citep}[1]{\cite{#1}}

\colorSectionsForArticle



%\usepackage{ece1236}
\usepackage{peeters_braket}
\usepackage{siunitx}
%\usepackage{peeters_layout_exercise}
\usepackage{peeters_figures}
\usepackage{mathtools}

\beginArtNoToc
\generatetitle{ECE1236H Microwave and Millimeter-Wave Techniques.  Lecture 3: Plane wave propagation in lossy media.  Taught by Prof.\ G.V. Eleftheriades}
%\chapter{Plane wave propagation in lossy media}
\label{chap:uwaves3}

\paragraph{Disclaimer}

Peeter's lecture notes from class.  These may be incoherent and rough.

These are notes for the UofT course ECE1236H, Microwave and Millimeter-Wave Techniques, taught by Prof. G.V. Eleftheriades, covering \textchapref{{1}} \citep{pozar2009microwave} content.

\section{Lossy media}

In a lossy medium characterized by \( \gamma = \alpha + j \beta \) such that the wave behaves like \( e^{-\gamma z } \) for propagation along \( +z \), or 

\begin{equation}\label{eqn:uwavesLecture3:20}
e^{-\gamma z} = e^{-\alpha z} e^{-j \beta z}
\end{equation}

The \( e^{-\alpha z} \) term introduces an exponential attenuation with \( z \).  The wave equation in the phasor domain is then

\begin{equation}\label{eqn:uwavesLecture3:40}
\spacegrad^2 \BE - \gamma^2 \BE = 0,
\end{equation}

where 

\begin{equation}\label{eqn:uwavesLecture3:60}
\gamma^2 = - \omega^2 \mu \epsilon_{\mathrm{eff}}.
\end{equation}

The effective complex permittivity \( \epsilon_{\mathrm{eff}} = \epsilon' -j \epsilon'' \) is obtained from Ampere's law

\begin{dmath}\label{eqn:uwavesLecture3:80}
\spacegrad \cross \BH 
= \BJ + j \omega \epsilon \BE
= \sigma \BE + j \omega \epsilon \BE
= \lr{ \sigma + j \omega \epsilon } \BE 
= j \omega \lr{ \epsilon -j \frac{\sigma}{\omega} } \BE ,
\end{dmath}

so
\begin{equation}\label{eqn:uwavesLecture3:100}
\begin{aligned}
\epsilon_{\mathrm{eff}} &= \epsilon' -j \epsilon'' \\
\epsilon' &= \epsilon \\
\epsilon'' &= \frac{\sigma}{\omega}
\end{aligned}
\end{equation}

Since \( \gamma = j \omega \sqrt{ \mu \epsilon_{\mathrm{eff}} } \) and \( \gamma^2 = -\omega^2 \mu \epsilon_{\mathrm{eff}} \) we see that

\begin{dmath}\label{eqn:uwavesLecture3:120}
(\alpha + j \beta)^2 
= (\alpha^2 - \beta^2) + j 2 \alpha \beta 
= - \omega^2 \mu \epsilon_{\mathrm{eff}}
= - \omega^2 \mu \epsilon'
  + j \omega^2 \mu \epsilon'',
\end{dmath}

therefore

\begin{equation}\label{eqn:uwavesLecture3:140}
\begin{aligned}
\alpha^2 - \beta^2 &= - \omega^2 \mu \epsilon' \\
2 \alpha \beta &= \omega^2 \mu \epsilon''.
\end{aligned}
\end{equation}

Solving for \( \alpha \) and \( \beta \) yields

\begin{equation}\label{eqn:uwavesLecture3:160}
\begin{aligned}
\alpha &= \omega \sqrt{ \frac{\mu \epsilon'}{2} \lr{ \sqrt{ 1 + \lr{\frac{\epsilon''}{\epsilon'}}^2 } - 1 } }, \qquad \si{Np/m}  \\
\beta &= \omega \sqrt{ \frac{\mu \epsilon'}{2} \lr{ \sqrt{ 1 + \lr{\frac{\epsilon''}{\epsilon'}}^2 } + 1 } } , \qquad \si{rad/m} \\
\end{aligned}
\end{equation}

Assuming propagation along \( + \zcap \) 

\begin{dmath}\label{eqn:uwavesLecture3:180}
\BE(z) 
= \xcap E_x(z) 
= \xcap E_{x0} e^{-\gamma z} 
= \xcap E_{x0} e^{-\alpha z} e^{-j \beta z}
\end{dmath}

The magnetic field can be determined from 

\begin{dmath}\label{eqn:uwavesLecture3:200}
\spacegrad \cross \BE = - j \omega \mu \BH,
\end{dmath}

or

\begin{dmath}\label{eqn:uwavesLecture3:220}
\BH = \ycap \frac{E_x}{\eta_\txtc} = 
= \ycap \frac{E_{x0}}{\eta_\txtc} e^{-\alpha z} e^{-j \beta z},
\end{dmath}

where
\begin{dmath}\label{eqn:uwavesLecture3:240}
\eta_\txtc = \sqrt{\frac{\mu}{\epsilon_{\mathrm{eff}}}} = \sqrt{\frac{\mu}{\epsilon'}} \lr{ 1 - j \frac{\epsilon''}{\epsilon'} }^{-1/2} \qquad \Omega,
\end{dmath}

is the complex intrinsic impedance of the medium.  Note that since \( \eta_\txtc \) is complex the electric znd magnetic fields are no longer in phase.

\section{Skin depth}

Note that 

\begin{equation}\label{eqn:uwavesLecture3:260}
\begin{aligned}
\Abs{ E_x } &= \Abs{ E_{x0} } e^{-\alpha z} \\
\Abs{ H_y } &= \Abs{ H_{x0} } e^{-\alpha z}.
\end{aligned}
\end{equation}

The skin depth \( \delta_\txts \) is defined as the distance that the wave needs to travel to reduce its magnitude by \( 1/e \).  Hence

\begin{equation}\label{eqn:uwavesLecture3:280}
e^{ -\alpha \delta_\txts } = e^{-1},
\end{equation}

or

\boxedEquation{eqn:uwavesLecture3:300}{
\delta_\txts  = \frac{1}{\alpha}.
}

F1: p4.

\paragraph{Low loss dielectrics}

\begin{dmath}\label{eqn:uwavesLecture3:320}
\gamma 
= 
j \omega \sqrt{\mu \epsilon' } \lr{ 1 -j \frac{\epsilon''}{\epsilon'}}^{1/2} 
\approx 
j \omega \sqrt{\mu \epsilon' } \lr{ 1 - j \frac{\epsilon''}{2 \epsilon'}}.
\end{dmath}

Hence 

\begin{dmath}\label{eqn:uwavesLecture3:340}
\alpha 
= 
\omega \sqrt{\mu \epsilon'} \frac{\epsilon''}{2 \epsilon'}
\end{dmath}

\EndArticle
%\EndNoBibArticle
