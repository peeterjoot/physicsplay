%
% Copyright � 2015 Peeter Joot.  All Rights Reserved.
% Licenced as described in the file LICENSE under the root directory of this GIT repository.
%
\documentclass[]{eliblog}

\usepackage{amsmath}
\usepackage{mathpazo}

%
% shorthand for bold symbols, convenient for vectors and matrices
%
\newcommand{\Ba}[0]{\mathbf{a}}
\newcommand{\Bb}[0]{\mathbf{b}}
\newcommand{\Bc}[0]{\mathbf{c}}
\newcommand{\Bd}[0]{\mathbf{d}}
\newcommand{\Be}[0]{\mathbf{e}}
\newcommand{\Bf}[0]{\mathbf{f}}
\newcommand{\Bg}[0]{\mathbf{g}}
\newcommand{\Bh}[0]{\mathbf{h}}
\newcommand{\Bi}[0]{\mathbf{i}}
\newcommand{\Bj}[0]{\mathbf{j}}
\newcommand{\Bk}[0]{\mathbf{k}}
\newcommand{\Bl}[0]{\mathbf{l}}
\newcommand{\Bm}[0]{\mathbf{m}}
\newcommand{\Bn}[0]{\mathbf{n}}
\newcommand{\Bo}[0]{\mathbf{o}}
\newcommand{\Bp}[0]{\mathbf{p}}
\newcommand{\Bq}[0]{\mathbf{q}}
\newcommand{\Br}[0]{\mathbf{r}}
\newcommand{\Bs}[0]{\mathbf{s}}
\newcommand{\Bt}[0]{\mathbf{t}}
\newcommand{\Bu}[0]{\mathbf{u}}
\newcommand{\Bv}[0]{\mathbf{v}}
\newcommand{\Bw}[0]{\mathbf{w}}
\newcommand{\Bx}[0]{\mathbf{x}}
\newcommand{\By}[0]{\mathbf{y}}
\newcommand{\Bz}[0]{\mathbf{z}}
\newcommand{\BA}[0]{\mathbf{A}}
\newcommand{\BB}[0]{\mathbf{B}}
\newcommand{\BC}[0]{\mathbf{C}}
\newcommand{\BD}[0]{\mathbf{D}}
\newcommand{\BE}[0]{\mathbf{E}}
\newcommand{\BF}[0]{\mathbf{F}}
\newcommand{\BG}[0]{\mathbf{G}}
\newcommand{\BH}[0]{\mathbf{H}}
\newcommand{\BI}[0]{\mathbf{I}}
\newcommand{\BJ}[0]{\mathbf{J}}
\newcommand{\BK}[0]{\mathbf{K}}
\newcommand{\BL}[0]{\mathbf{L}}
\newcommand{\BM}[0]{\mathbf{M}}
\newcommand{\BN}[0]{\mathbf{N}}
\newcommand{\BO}[0]{\mathbf{O}}
\newcommand{\BP}[0]{\mathbf{P}}
\newcommand{\BQ}[0]{\mathbf{Q}}
\newcommand{\BR}[0]{\mathbf{R}}
\newcommand{\BS}[0]{\mathbf{S}}
\newcommand{\BT}[0]{\mathbf{T}}
\newcommand{\BU}[0]{\mathbf{U}}
\newcommand{\BV}[0]{\mathbf{V}}
\newcommand{\BW}[0]{\mathbf{W}}
\newcommand{\BX}[0]{\mathbf{X}}
\newcommand{\BY}[0]{\mathbf{Y}}
\newcommand{\BZ}[0]{\mathbf{Z}}

\newcommand{\Bzero}[0]{\mathbf{0}}
\newcommand{\Btheta}[0]{\boldsymbol{\theta}}
\newcommand{\Btau}[0]{\boldsymbol{\tau}}
\newcommand{\Bomega}[0]{\boldsymbol{\omega}}

%
% shorthand for unit vectors
%
\newcommand{\acap}[0]{\hat{\Ba}}
\newcommand{\bcap}[0]{\hat{\Bb}}
\newcommand{\ccap}[0]{\hat{\Bc}}
\newcommand{\dcap}[0]{\hat{\Bd}}
\newcommand{\ecap}[0]{\hat{\Be}}
\newcommand{\fcap}[0]{\hat{\Bf}}
\newcommand{\gcap}[0]{\hat{\Bg}}
\newcommand{\hcap}[0]{\hat{\Bh}}
\newcommand{\icap}[0]{\hat{\Bi}}
\newcommand{\jcap}[0]{\hat{\Bj}}
\newcommand{\kcap}[0]{\hat{\Bk}}
\newcommand{\lcap}[0]{\hat{\Bl}}
\newcommand{\mcap}[0]{\hat{\Bm}}
\newcommand{\ncap}[0]{\hat{\Bn}}
\newcommand{\ocap}[0]{\hat{\Bo}}
\newcommand{\pcap}[0]{\hat{\Bp}}
\newcommand{\qcap}[0]{\hat{\Bq}}
\newcommand{\rcap}[0]{\hat{\Br}}
\newcommand{\scap}[0]{\hat{\Bs}}
\newcommand{\tcap}[0]{\hat{\Bt}}
\newcommand{\ucap}[0]{\hat{\Bu}}
\newcommand{\vcap}[0]{\hat{\Bv}}
\newcommand{\wcap}[0]{\hat{\Bw}}
\newcommand{\xcap}[0]{\hat{\Bx}}
\newcommand{\ycap}[0]{\hat{\By}}
\newcommand{\zcap}[0]{\hat{\Bz}}
\newcommand{\thetacap}[0]{\hat{\Btheta}}

%
% to write R^n and C^n in a distinguishable fashion.  Perhaps change this
% to the double lined characters upon figuring out how to do so.
%
\newcommand{\C}[1]{$\mathbb{C}^{#1}$}
\newcommand{\R}[1]{$\mathbb{R}^{#1}$}

%
% various generally useful helpers
%

% derivative of #1 wrt. #2:
\newcommand{\D}[2] {\frac {d#2} {d#1}}

\newcommand{\inv}[1]{\frac{1}{#1}}
\newcommand{\cross}[0]{\times}

\newcommand{\abs}[1]{\lvert{#1}\rvert}
\newcommand{\norm}[1]{\lVert{#1}\rVert}
\newcommand{\innerprod}[2]{\langle{#1}, {#2}\rangle}
\newcommand{\dotprod}[2]{{#1} \cdot {#2}}
\newcommand{\bdotprod}[2]{\left({#1} \cdot {#2}\right)}
\newcommand{\crossprod}[2]{{#1} \cross {#2}}
\newcommand{\tripleprod}[3]{\dotprod{\left(\crossprod{#1}{#2}\right)}{#3}}

\DeclareMathOperator{\Proj}{Proj}
\DeclareMathOperator{\Span}{span}
\DeclareMathOperator{\Sgn}{sgn}
\DeclareMathOperator{\Area}{Area}
\DeclareMathOperator{\Volume}{Volume}

%
% A few miscellaneous things specific to this document
%
\newcommand{\crossop}[1]{\crossprod{#1}{}}

% R2 vector.
\newcommand{\VectorTwo}[2]{
\begin{bmatrix}
 {#1} \\
 {#2}
\end{bmatrix}
}

\newcommand{\VectorN}[1]{
\begin{bmatrix}
{#1}_1 \\
{#1}_2 \\
\vdots \\
{#1}_N \\
\end{bmatrix}
}

\newcommand{\DETuvij}[4]{
\begin{vmatrix}
 {#1}_{#3} & {#1}_{#4} \\
 {#2}_{#3} & {#2}_{#4}
\end{vmatrix}
}

\newcommand{\DETuvwijk}[6]{
\begin{vmatrix}
 {#1}_{#4} & {#1}_{#5} & {#1}_{#6} \\
 {#2}_{#4} & {#2}_{#5} & {#2}_{#6} \\
 {#3}_{#4} & {#3}_{#5} & {#3}_{#6}
\end{vmatrix}
}

\newcommand{\DETuvwxijkl}[8]{
\begin{vmatrix}
 {#1}_{#5} & {#1}_{#6} & {#1}_{#7} & {#1}_{#8} \\
 {#2}_{#5} & {#2}_{#6} & {#2}_{#7} & {#2}_{#8} \\
 {#3}_{#5} & {#3}_{#6} & {#3}_{#7} & {#3}_{#8} \\
 {#4}_{#5} & {#4}_{#6} & {#4}_{#7} & {#4}_{#8} \\
\end{vmatrix}
}

%\newcommand{\DETuvwxyijklm}[10]{
%\begin{vmatrix}
% {#1}_{#6} & {#1}_{#7} & {#1}_{#8} & {#1}_{#9} & {#1}_{#10} \\
% {#2}_{#6} & {#2}_{#7} & {#2}_{#8} & {#2}_{#9} & {#2}_{#10} \\
% {#3}_{#6} & {#3}_{#7} & {#3}_{#8} & {#3}_{#9} & {#3}_{#10} \\
% {#4}_{#6} & {#4}_{#7} & {#4}_{#8} & {#4}_{#9} & {#4}_{#10} \\
% {#5}_{#6} & {#5}_{#7} & {#5}_{#8} & {#5}_{#9} & {#5}_{#10}
%\end{vmatrix}
%}

% R3 vector.
\newcommand{\VectorThree}[3]{
\begin{bmatrix}
 {#1} \\
 {#2} \\
 {#3}
\end{bmatrix}
}



\author{Peeter Joot}
\email{peeter.joot@gmail.com}

%\documentclass[]{eliblogwidescreen}

\usepackage{amsmath}
\usepackage{mathpazo}

%
% shorthand for bold symbols, convenient for vectors and matrices
%
\newcommand{\Ba}[0]{\mathbf{a}}
\newcommand{\Bb}[0]{\mathbf{b}}
\newcommand{\Bc}[0]{\mathbf{c}}
\newcommand{\Bd}[0]{\mathbf{d}}
\newcommand{\Be}[0]{\mathbf{e}}
\newcommand{\Bf}[0]{\mathbf{f}}
\newcommand{\Bg}[0]{\mathbf{g}}
\newcommand{\Bh}[0]{\mathbf{h}}
\newcommand{\Bi}[0]{\mathbf{i}}
\newcommand{\Bj}[0]{\mathbf{j}}
\newcommand{\Bk}[0]{\mathbf{k}}
\newcommand{\Bl}[0]{\mathbf{l}}
\newcommand{\Bm}[0]{\mathbf{m}}
\newcommand{\Bn}[0]{\mathbf{n}}
\newcommand{\Bo}[0]{\mathbf{o}}
\newcommand{\Bp}[0]{\mathbf{p}}
\newcommand{\Bq}[0]{\mathbf{q}}
\newcommand{\Br}[0]{\mathbf{r}}
\newcommand{\Bs}[0]{\mathbf{s}}
\newcommand{\Bt}[0]{\mathbf{t}}
\newcommand{\Bu}[0]{\mathbf{u}}
\newcommand{\Bv}[0]{\mathbf{v}}
\newcommand{\Bw}[0]{\mathbf{w}}
\newcommand{\Bx}[0]{\mathbf{x}}
\newcommand{\By}[0]{\mathbf{y}}
\newcommand{\Bz}[0]{\mathbf{z}}
\newcommand{\BA}[0]{\mathbf{A}}
\newcommand{\BB}[0]{\mathbf{B}}
\newcommand{\BC}[0]{\mathbf{C}}
\newcommand{\BD}[0]{\mathbf{D}}
\newcommand{\BE}[0]{\mathbf{E}}
\newcommand{\BF}[0]{\mathbf{F}}
\newcommand{\BG}[0]{\mathbf{G}}
\newcommand{\BH}[0]{\mathbf{H}}
\newcommand{\BI}[0]{\mathbf{I}}
\newcommand{\BJ}[0]{\mathbf{J}}
\newcommand{\BK}[0]{\mathbf{K}}
\newcommand{\BL}[0]{\mathbf{L}}
\newcommand{\BM}[0]{\mathbf{M}}
\newcommand{\BN}[0]{\mathbf{N}}
\newcommand{\BO}[0]{\mathbf{O}}
\newcommand{\BP}[0]{\mathbf{P}}
\newcommand{\BQ}[0]{\mathbf{Q}}
\newcommand{\BR}[0]{\mathbf{R}}
\newcommand{\BS}[0]{\mathbf{S}}
\newcommand{\BT}[0]{\mathbf{T}}
\newcommand{\BU}[0]{\mathbf{U}}
\newcommand{\BV}[0]{\mathbf{V}}
\newcommand{\BW}[0]{\mathbf{W}}
\newcommand{\BX}[0]{\mathbf{X}}
\newcommand{\BY}[0]{\mathbf{Y}}
\newcommand{\BZ}[0]{\mathbf{Z}}

\newcommand{\Bzero}[0]{\mathbf{0}}
\newcommand{\Btheta}[0]{\boldsymbol{\theta}}
\newcommand{\Btau}[0]{\boldsymbol{\tau}}
\newcommand{\Bomega}[0]{\boldsymbol{\omega}}

%
% shorthand for unit vectors
%
\newcommand{\acap}[0]{\hat{\Ba}}
\newcommand{\bcap}[0]{\hat{\Bb}}
\newcommand{\ccap}[0]{\hat{\Bc}}
\newcommand{\dcap}[0]{\hat{\Bd}}
\newcommand{\ecap}[0]{\hat{\Be}}
\newcommand{\fcap}[0]{\hat{\Bf}}
\newcommand{\gcap}[0]{\hat{\Bg}}
\newcommand{\hcap}[0]{\hat{\Bh}}
\newcommand{\icap}[0]{\hat{\Bi}}
\newcommand{\jcap}[0]{\hat{\Bj}}
\newcommand{\kcap}[0]{\hat{\Bk}}
\newcommand{\lcap}[0]{\hat{\Bl}}
\newcommand{\mcap}[0]{\hat{\Bm}}
\newcommand{\ncap}[0]{\hat{\Bn}}
\newcommand{\ocap}[0]{\hat{\Bo}}
\newcommand{\pcap}[0]{\hat{\Bp}}
\newcommand{\qcap}[0]{\hat{\Bq}}
\newcommand{\rcap}[0]{\hat{\Br}}
\newcommand{\scap}[0]{\hat{\Bs}}
\newcommand{\tcap}[0]{\hat{\Bt}}
\newcommand{\ucap}[0]{\hat{\Bu}}
\newcommand{\vcap}[0]{\hat{\Bv}}
\newcommand{\wcap}[0]{\hat{\Bw}}
\newcommand{\xcap}[0]{\hat{\Bx}}
\newcommand{\ycap}[0]{\hat{\By}}
\newcommand{\zcap}[0]{\hat{\Bz}}
\newcommand{\thetacap}[0]{\hat{\Btheta}}

%
% to write R^n and C^n in a distinguishable fashion.  Perhaps change this
% to the double lined characters upon figuring out how to do so.
%
\newcommand{\C}[1]{$\mathbb{C}^{#1}$}
\newcommand{\R}[1]{$\mathbb{R}^{#1}$}

%
% various generally useful helpers
%

% derivative of #1 wrt. #2:
\newcommand{\D}[2] {\frac {d#2} {d#1}}

\newcommand{\inv}[1]{\frac{1}{#1}}
\newcommand{\cross}[0]{\times}

\newcommand{\abs}[1]{\lvert{#1}\rvert}
\newcommand{\norm}[1]{\lVert{#1}\rVert}
\newcommand{\innerprod}[2]{\langle{#1}, {#2}\rangle}
\newcommand{\dotprod}[2]{{#1} \cdot {#2}}
\newcommand{\bdotprod}[2]{\left({#1} \cdot {#2}\right)}
\newcommand{\crossprod}[2]{{#1} \cross {#2}}
\newcommand{\tripleprod}[3]{\dotprod{\left(\crossprod{#1}{#2}\right)}{#3}}

\DeclareMathOperator{\Proj}{Proj}
\DeclareMathOperator{\Span}{span}
\DeclareMathOperator{\Sgn}{sgn}
\DeclareMathOperator{\Area}{Area}
\DeclareMathOperator{\Volume}{Volume}

%
% A few miscellaneous things specific to this document
%
\newcommand{\crossop}[1]{\crossprod{#1}{}}

% R2 vector.
\newcommand{\VectorTwo}[2]{
\begin{bmatrix}
 {#1} \\
 {#2}
\end{bmatrix}
}

\newcommand{\VectorN}[1]{
\begin{bmatrix}
{#1}_1 \\
{#1}_2 \\
\vdots \\
{#1}_N \\
\end{bmatrix}
}

\newcommand{\DETuvij}[4]{
\begin{vmatrix}
 {#1}_{#3} & {#1}_{#4} \\
 {#2}_{#3} & {#2}_{#4}
\end{vmatrix}
}

\newcommand{\DETuvwijk}[6]{
\begin{vmatrix}
 {#1}_{#4} & {#1}_{#5} & {#1}_{#6} \\
 {#2}_{#4} & {#2}_{#5} & {#2}_{#6} \\
 {#3}_{#4} & {#3}_{#5} & {#3}_{#6}
\end{vmatrix}
}

\newcommand{\DETuvwxijkl}[8]{
\begin{vmatrix}
 {#1}_{#5} & {#1}_{#6} & {#1}_{#7} & {#1}_{#8} \\
 {#2}_{#5} & {#2}_{#6} & {#2}_{#7} & {#2}_{#8} \\
 {#3}_{#5} & {#3}_{#6} & {#3}_{#7} & {#3}_{#8} \\
 {#4}_{#5} & {#4}_{#6} & {#4}_{#7} & {#4}_{#8} \\
\end{vmatrix}
}

%\newcommand{\DETuvwxyijklm}[10]{
%\begin{vmatrix}
% {#1}_{#6} & {#1}_{#7} & {#1}_{#8} & {#1}_{#9} & {#1}_{#10} \\
% {#2}_{#6} & {#2}_{#7} & {#2}_{#8} & {#2}_{#9} & {#2}_{#10} \\
% {#3}_{#6} & {#3}_{#7} & {#3}_{#8} & {#3}_{#9} & {#3}_{#10} \\
% {#4}_{#6} & {#4}_{#7} & {#4}_{#8} & {#4}_{#9} & {#4}_{#10} \\
% {#5}_{#6} & {#5}_{#7} & {#5}_{#8} & {#5}_{#9} & {#5}_{#10}
%\end{vmatrix}
%}

% R3 vector.
\newcommand{\VectorThree}[3]{
\begin{bmatrix}
 {#1} \\
 {#2} \\
 {#3}
\end{bmatrix}
}



\author{Peeter Joot}
\email{peeter.joot@gmail.com}


%(Taught by Mr. Federico Duque Gomez).  XXX}
\chapter{PHY456H1F: Quantum Mechanics II.  Lecture 17 (Taught by Prof J.E. Sipe).  XXX}
\label{chap:qmTwoL17}
%\useCCL
\blogpage{http://sites.google.com/site/peeterjoot/math2011/qmTwoL17.pdf}
\date{Nov 9, 2011}
\revisionInfo{qmTwoL17.tex}

\beginArtWithToc
%\beginArtNoToc

\section{Disclaimer.}

Peeter's lecture notes from class.  May not be entirely coherent.

\section{More on two spin systems.}

READING: Covering \S 26.5 of the text \cite{desai2009quantum}.

\begin{equation}\label{eqn:qmTwoL17:n}
\inv{2} \otimes \inv{2} = 1 \oplus 0
\end{equation}

where $1$ is a triplet state for $s=1$ and $0$ the ``singlet'' state with $s=0$.  We want to consider the angular momentum of the entire system

\begin{equation}\label{eqn:qmTwoL17:n}
j_1 \otimes j_2 = ?
\end{equation}

Why bother?  Often it is true that 

\begin{equation}\label{eqn:qmTwoL17:n}
\antisymmetric{H}{\BJ} = 0,
\end{equation}

so, in that case, the eigenstates of the total angular momentum are also energy eigenstates, so considering the angular momentum problem can help in finding these energy eigenstates.

\paragraph{Rotation operator}

\begin{equation}\label{eqn:qmTwoL17:n}
e^{-i \theta \ncap \cdot \BJ/\hbar}
\end{equation}

\begin{equation}\label{eqn:qmTwoL17:n}
\ncap \cdot \BJ = n_x J_x + n_y J_y + n_z J_z
\end{equation}

Recall the definitions of the raising or lowering operators

\begin{equation}\label{eqn:qmTwoL17:n}
J_\pm = J_x \pm i J_y,
\end{equation}

or
\begin{align}\label{eqn:qmTwoL17:n}
J_x &= \inv{2} (J_{+} + J_{-})
J_y &= \inv{2i} (J_{+} - J_{-})
\end{align}

We have
\begin{equation}\label{eqn:qmTwoL17:n}
\ncap \cdot \BJ = 
n_x 
\inv{2} (J_{+} + J_{-})
+ 
n_y 
\inv{2i} (J_{+} - J_{-})
+ 
n_z J_z,
\end{equation}

and 

\begin{equation}\label{eqn:qmTwoL17:n}
J_\pm \ket{j m} = \hbar \Bigl(
(j \mp m)(j \pm m_1)
\Bigr)^{1/2}
\ket{j, m \pm 1}
\end{equation}

So

\begin{equation}\label{eqn:qmTwoL17:n}
\bra{j' m'} e^{-i \theta \ncap \cdot \BJ/\hbar} \ket{j m} = 0
\end{equation}

unless $j = j'$.

\begin{equation}\label{eqn:qmTwoL17:n}
\bra{j m'} e^{-i \theta \ncap \cdot \BJ/\hbar} \ket{j m}
\end{equation}

is a $(2j + 1) \times (2 j+ 1)$ matrix.

Combining rotations

\begin{equation}\label{eqn:qmTwoL17:n}
\bra{j m'} 
e^{-i \theta_b \ncap_a \cdot \BJ/\hbar}
e^{-i \theta_a \ncap_b \cdot \BJ/\hbar}
 \ket{j m}
=
\sum_{m''}
\bra{j m'} 
e^{-i \theta_b \ncap_a \cdot \BJ/\hbar}
\ket{j m''} \bra{j m''}
e^{-i \theta_a \ncap_b \cdot \BJ/\hbar}
 \ket{j m}
\end{equation}

If
\begin{equation}\label{eqn:qmTwoL17:n}
e^{-i \theta \ncap \cdot \BJ/\hbar}
=
e^{-i \theta_b \ncap_a \cdot \BJ/\hbar}
e^{-i \theta_a \ncap_b \cdot \BJ/\hbar}
\end{equation}

(something that may be hard to compute but possible), then

\begin{equation}\label{eqn:qmTwoL17:n}
\bra{j m'} e^{-i \theta \ncap \cdot \BJ/\hbar} \ket{j m}
=
\sum_{m''}
\bra{j m'} 
e^{-i \theta_b \ncap_a \cdot \BJ/\hbar}
\ket{j m''} \bra{j m''}
e^{-i \theta_a \ncap_b \cdot \BJ/\hbar}
 \ket{j m}
\end{equation}

For fixed $j$, the matrices $\bra{j m'} e^{-i \theta \ncap \cdot \BJ/\hbar} \ket{j m}$ form a representation of the rotation group.  The $(2 j + 1)$ representations are irreducible.  (This won't be proven).

It may be that there may be big blocks of zeros in some of the matrices, but they cannot be simplified any further?

Back to the two particle system

\begin{equation}\label{eqn:qmTwoL17:n}
j_1 \otimes j_2 = ?
\end{equation}

If we use
\begin{equation}\label{eqn:qmTwoL17:n}
\ket{j_1 m_1} \otimes \ket{j_2 m_2}
\end{equation}

If a $j_1$ and a $j_2$ are picked then

\begin{equation}\label{eqn:qmTwoL17:n}
\bra{j_1 m_1' ; j_2 m_2'} e^{-i \theta \ncap \cdot \BJ/\hbar} \ket{j_1 m_1 ; j_2 m_2}
\end{equation}

is also a representation of the rotation group, but these sort of matrices can be simplified a \underline{lot}.  This basis of dimensionality $(2 j_1 + 1)(2 j_2 + 1)$ is \underline{reducible}.

A lot of this is motivation, and we still want a representation of $j_1 \otimes j_2$.

Recall that 

\begin{equation}\label{eqn:qmTwoL17:n}
\inv{2} \otimes \inv{2} = 1 \oplus 0
= 
\left(\inv{2} + \inv{2} \right)
 \oplus 
\left(\inv{2} - \inv{2} \right)
\end{equation}

Might guess that, for $j_1 \ge j_2$

\begin{equation}\label{eqn:qmTwoL17:n}
j_1 \otimes j_2 = 
\left( j_1 + j_2 \right)
 \oplus 
\left( j_1 + j_2 - 1 \right)
 \oplus 
\cdots
\left( j_1 - j_2 \right)
\end{equation}

Suppose that this is right.  Then

\begin{equation}\label{eqn:qmTwoL17:n}
5 \otimes \inv{2} = \frac{11}{2} \oplus \frac{9}{2}
\end{equation}

Check for dimensions.
FIXME: aligned:
\begin{equation}\label{eqn:qmTwoL17:n}
1 \otimes 1 = 2 \oplus 1 \oplus 0
3  \times 3 = 5  +     3  +     1
\end{equation}

\paragraph{Q}: What was this $\oplus$?

It was just made up.  We are creating a shorthand to say that we have a number of different basis states for each of the groupings.  \underline{I Need an example!}

Check for dimensions in general

(2 j_1 + 1)(2 j_2 + 1) 
\stackrel{?}{=}

\sum_{j_1 - j_2}^{j_1 + j_2} (2 j+ 1) 
&= 
\sum_{j=0}^{j_1 + j_2} (2 j + 1) 
-
\sum_{j=0}^{j_1 - j_2 - 1} (2 j + 1) 
&=
(2 j_1 + 1)(2 j_2 + 1) 

Using 

\sum_{n=0}^N n = \frac{N(N+1)}{2}

\begin{equation}\label{eqn:qmTwoL17:n}
j_1 \otimes j_2
= 
(j_1 + j_2) \oplus
(j_1 + j_2 - 1) \oplus
\cdots
(j_1 - j_2) 
\end{equation}

In fact, this is correct.  Proof ``by construction'' to follow.

\begin{equation}\label{eqn:qmTwoL17:n}
\ket{j_1 m_1} 
\otimes
\ket{j_2 m_2} 
\end{equation}

\begin{align}\label{eqn:qmTwoL17:n}
J^2 \ket{j m} &= j (j+1) \hbar^2 \ket{j m} \\
J_z \ket{j m} &= m \hbar \ket{j m}
\end{align}

denote also by
\begin{equation}\label{eqn:qmTwoL17:n}
\ket{j m ; j_1 j_2},
\end{equation}

but will often omit the $,j_1 j_2$ portion.

With 

j &= j_1 + j_2				& j_1 + j_2 -1 				& \cdots & j_1 - j_2
  &  \ket{j_1 + j_2, j_1 + j_2}	 	&					&
  &  \ket{j_1 + j_2, j_1 + j_2 - 1}	&  \ket{j_1 + j_2 - 1, j_1 + j_2 - 1}	&
  &                                     & \ket{j_1 + j_2 - 1, j_1 + j_2 - 2}	&
  & \vdots 	 			&					& \ket{j_1 - j_2, j_1 - j_2}
  & \vdots 	 			&					& \vdots
  & \vdots 	 			&					& \ket{j_1 - j_2, -(j_1 - j_2)}
  & \vdots 	 			&					&
  &  \ket{j_1 + j_2, -(j_1 + j_2 - 1)}	& \ket{j_1 + j_2 -1, -(j_1 + j_2 - 1)}	&
  &  \ket{j_1 + j_2, -(j_1 + j_2)}	&					&

Look at 

\begin{equation}\label{eqn:qmTwoL17:n}
\ket{j_1 + j+2, j_1 + j_2}
\end{equation}

\begin{equation}\label{eqn:qmTwoL17:n}
J_z
\ket{j_1 + j_2, j_1 + j_2}
= 
(j_1 + j_2) \hbar
\ket{j_1 + j_2, j_1 + j_2}
\end{equation}

\begin{equation}\label{eqn:qmTwoL17:n}
J_z 
\Bigl(\ket{j_1 m_1} \otimes \ket{j_2 m_2} \Bigr)
= 
(m_1 + m_2) \hbar
\Bigl(\ket{j_1 m_1} \otimes \ket{j_2 m_2} \Bigr)
\end{equation}

we must have

\begin{equation}\label{eqn:qmTwoL17:n}
\ket{j_1 + j_2, j_1 + j_2}
= e^{i\phi}
\Bigl(\ket{j_1 j_1} \otimes \ket{j_2 j_2} \Bigr)
\end{equation}

So $\ket{j_1 + j_2, j_1 + j_2}$ must be a superposition of states $\ket{j_1 m_1} \otimes \ket{j_2 m_2} $ with $m_1 + m_2 = j_1 + j_2$.  Choosing $e^{i\phi} = 1$ is called the Conbon Shotley convention.

\begin{equation}\label{eqn:qmTwoL17:n}
\ket{j_1 + j_2, j_1 + j_2}
= 
\ket{j_1 j_1} \otimes \ket{j_2 j_2}
\end{equation}

We now move down column.

\begin{equation}\label{eqn:qmTwoL17:n}
J_{-} \ket{j_1 + j_2, j_1 + j_2}
=
\hbar
\Bigl(
2 (j_1 + j_2)
\Bigr)^{1/2}
\ket{j_1 + j_2, j_1 + j_2 - 1}
\end{equation}

So 
\begin{equation}\label{eqn:qmTwoL17:n}
\ket{j_1 + j_2, j_1 + j_2 - 1}
&=
\frac{
J_{-} \ket{j_1 + j_2, j_1 + j_2}
}{
\hbar
\Bigl(
2 (j_1 + j_2)
\Bigr)^{1/2}
}
&=
\frac{
(J_{1-} + J_{2-}) \ket{j_1 j_1} \otimes \ket{j_2 j_2}
}{
\hbar
\Bigl(
2 (j_1 + j_2)
\Bigr)^{1/2}
}
\end{equation}

\EndArticle
