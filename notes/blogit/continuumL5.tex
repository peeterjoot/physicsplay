%
% Copyright � 2015 Peeter Joot.  All Rights Reserved.
% Licenced as described in the file LICENSE under the root directory of this GIT repository.
%
\documentclass[]{eliblog}

\usepackage{amsmath}
\usepackage{mathpazo}

%
% shorthand for bold symbols, convenient for vectors and matrices
%
\newcommand{\Ba}[0]{\mathbf{a}}
\newcommand{\Bb}[0]{\mathbf{b}}
\newcommand{\Bc}[0]{\mathbf{c}}
\newcommand{\Bd}[0]{\mathbf{d}}
\newcommand{\Be}[0]{\mathbf{e}}
\newcommand{\Bf}[0]{\mathbf{f}}
\newcommand{\Bg}[0]{\mathbf{g}}
\newcommand{\Bh}[0]{\mathbf{h}}
\newcommand{\Bi}[0]{\mathbf{i}}
\newcommand{\Bj}[0]{\mathbf{j}}
\newcommand{\Bk}[0]{\mathbf{k}}
\newcommand{\Bl}[0]{\mathbf{l}}
\newcommand{\Bm}[0]{\mathbf{m}}
\newcommand{\Bn}[0]{\mathbf{n}}
\newcommand{\Bo}[0]{\mathbf{o}}
\newcommand{\Bp}[0]{\mathbf{p}}
\newcommand{\Bq}[0]{\mathbf{q}}
\newcommand{\Br}[0]{\mathbf{r}}
\newcommand{\Bs}[0]{\mathbf{s}}
\newcommand{\Bt}[0]{\mathbf{t}}
\newcommand{\Bu}[0]{\mathbf{u}}
\newcommand{\Bv}[0]{\mathbf{v}}
\newcommand{\Bw}[0]{\mathbf{w}}
\newcommand{\Bx}[0]{\mathbf{x}}
\newcommand{\By}[0]{\mathbf{y}}
\newcommand{\Bz}[0]{\mathbf{z}}
\newcommand{\BA}[0]{\mathbf{A}}
\newcommand{\BB}[0]{\mathbf{B}}
\newcommand{\BC}[0]{\mathbf{C}}
\newcommand{\BD}[0]{\mathbf{D}}
\newcommand{\BE}[0]{\mathbf{E}}
\newcommand{\BF}[0]{\mathbf{F}}
\newcommand{\BG}[0]{\mathbf{G}}
\newcommand{\BH}[0]{\mathbf{H}}
\newcommand{\BI}[0]{\mathbf{I}}
\newcommand{\BJ}[0]{\mathbf{J}}
\newcommand{\BK}[0]{\mathbf{K}}
\newcommand{\BL}[0]{\mathbf{L}}
\newcommand{\BM}[0]{\mathbf{M}}
\newcommand{\BN}[0]{\mathbf{N}}
\newcommand{\BO}[0]{\mathbf{O}}
\newcommand{\BP}[0]{\mathbf{P}}
\newcommand{\BQ}[0]{\mathbf{Q}}
\newcommand{\BR}[0]{\mathbf{R}}
\newcommand{\BS}[0]{\mathbf{S}}
\newcommand{\BT}[0]{\mathbf{T}}
\newcommand{\BU}[0]{\mathbf{U}}
\newcommand{\BV}[0]{\mathbf{V}}
\newcommand{\BW}[0]{\mathbf{W}}
\newcommand{\BX}[0]{\mathbf{X}}
\newcommand{\BY}[0]{\mathbf{Y}}
\newcommand{\BZ}[0]{\mathbf{Z}}

\newcommand{\Bzero}[0]{\mathbf{0}}
\newcommand{\Btheta}[0]{\boldsymbol{\theta}}
\newcommand{\Btau}[0]{\boldsymbol{\tau}}
\newcommand{\Bomega}[0]{\boldsymbol{\omega}}

%
% shorthand for unit vectors
%
\newcommand{\acap}[0]{\hat{\Ba}}
\newcommand{\bcap}[0]{\hat{\Bb}}
\newcommand{\ccap}[0]{\hat{\Bc}}
\newcommand{\dcap}[0]{\hat{\Bd}}
\newcommand{\ecap}[0]{\hat{\Be}}
\newcommand{\fcap}[0]{\hat{\Bf}}
\newcommand{\gcap}[0]{\hat{\Bg}}
\newcommand{\hcap}[0]{\hat{\Bh}}
\newcommand{\icap}[0]{\hat{\Bi}}
\newcommand{\jcap}[0]{\hat{\Bj}}
\newcommand{\kcap}[0]{\hat{\Bk}}
\newcommand{\lcap}[0]{\hat{\Bl}}
\newcommand{\mcap}[0]{\hat{\Bm}}
\newcommand{\ncap}[0]{\hat{\Bn}}
\newcommand{\ocap}[0]{\hat{\Bo}}
\newcommand{\pcap}[0]{\hat{\Bp}}
\newcommand{\qcap}[0]{\hat{\Bq}}
\newcommand{\rcap}[0]{\hat{\Br}}
\newcommand{\scap}[0]{\hat{\Bs}}
\newcommand{\tcap}[0]{\hat{\Bt}}
\newcommand{\ucap}[0]{\hat{\Bu}}
\newcommand{\vcap}[0]{\hat{\Bv}}
\newcommand{\wcap}[0]{\hat{\Bw}}
\newcommand{\xcap}[0]{\hat{\Bx}}
\newcommand{\ycap}[0]{\hat{\By}}
\newcommand{\zcap}[0]{\hat{\Bz}}
\newcommand{\thetacap}[0]{\hat{\Btheta}}

%
% to write R^n and C^n in a distinguishable fashion.  Perhaps change this
% to the double lined characters upon figuring out how to do so.
%
\newcommand{\C}[1]{$\mathbb{C}^{#1}$}
\newcommand{\R}[1]{$\mathbb{R}^{#1}$}

%
% various generally useful helpers
%

% derivative of #1 wrt. #2:
\newcommand{\D}[2] {\frac {d#2} {d#1}}

\newcommand{\inv}[1]{\frac{1}{#1}}
\newcommand{\cross}[0]{\times}

\newcommand{\abs}[1]{\lvert{#1}\rvert}
\newcommand{\norm}[1]{\lVert{#1}\rVert}
\newcommand{\innerprod}[2]{\langle{#1}, {#2}\rangle}
\newcommand{\dotprod}[2]{{#1} \cdot {#2}}
\newcommand{\bdotprod}[2]{\left({#1} \cdot {#2}\right)}
\newcommand{\crossprod}[2]{{#1} \cross {#2}}
\newcommand{\tripleprod}[3]{\dotprod{\left(\crossprod{#1}{#2}\right)}{#3}}

\DeclareMathOperator{\Proj}{Proj}
\DeclareMathOperator{\Span}{span}
\DeclareMathOperator{\Sgn}{sgn}
\DeclareMathOperator{\Area}{Area}
\DeclareMathOperator{\Volume}{Volume}

%
% A few miscellaneous things specific to this document
%
\newcommand{\crossop}[1]{\crossprod{#1}{}}

% R2 vector.
\newcommand{\VectorTwo}[2]{
\begin{bmatrix}
 {#1} \\
 {#2}
\end{bmatrix}
}

\newcommand{\VectorN}[1]{
\begin{bmatrix}
{#1}_1 \\
{#1}_2 \\
\vdots \\
{#1}_N \\
\end{bmatrix}
}

\newcommand{\DETuvij}[4]{
\begin{vmatrix}
 {#1}_{#3} & {#1}_{#4} \\
 {#2}_{#3} & {#2}_{#4}
\end{vmatrix}
}

\newcommand{\DETuvwijk}[6]{
\begin{vmatrix}
 {#1}_{#4} & {#1}_{#5} & {#1}_{#6} \\
 {#2}_{#4} & {#2}_{#5} & {#2}_{#6} \\
 {#3}_{#4} & {#3}_{#5} & {#3}_{#6}
\end{vmatrix}
}

\newcommand{\DETuvwxijkl}[8]{
\begin{vmatrix}
 {#1}_{#5} & {#1}_{#6} & {#1}_{#7} & {#1}_{#8} \\
 {#2}_{#5} & {#2}_{#6} & {#2}_{#7} & {#2}_{#8} \\
 {#3}_{#5} & {#3}_{#6} & {#3}_{#7} & {#3}_{#8} \\
 {#4}_{#5} & {#4}_{#6} & {#4}_{#7} & {#4}_{#8} \\
\end{vmatrix}
}

%\newcommand{\DETuvwxyijklm}[10]{
%\begin{vmatrix}
% {#1}_{#6} & {#1}_{#7} & {#1}_{#8} & {#1}_{#9} & {#1}_{#10} \\
% {#2}_{#6} & {#2}_{#7} & {#2}_{#8} & {#2}_{#9} & {#2}_{#10} \\
% {#3}_{#6} & {#3}_{#7} & {#3}_{#8} & {#3}_{#9} & {#3}_{#10} \\
% {#4}_{#6} & {#4}_{#7} & {#4}_{#8} & {#4}_{#9} & {#4}_{#10} \\
% {#5}_{#6} & {#5}_{#7} & {#5}_{#8} & {#5}_{#9} & {#5}_{#10}
%\end{vmatrix}
%}

% R3 vector.
\newcommand{\VectorThree}[3]{
\begin{bmatrix}
 {#1} \\
 {#2} \\
 {#3}
\end{bmatrix}
}



\author{Peeter Joot}
\email{peeter.joot@gmail.com}

%\documentclass[]{eliblogwidescreen}

\usepackage{amsmath}
\usepackage{mathpazo}

%
% shorthand for bold symbols, convenient for vectors and matrices
%
\newcommand{\Ba}[0]{\mathbf{a}}
\newcommand{\Bb}[0]{\mathbf{b}}
\newcommand{\Bc}[0]{\mathbf{c}}
\newcommand{\Bd}[0]{\mathbf{d}}
\newcommand{\Be}[0]{\mathbf{e}}
\newcommand{\Bf}[0]{\mathbf{f}}
\newcommand{\Bg}[0]{\mathbf{g}}
\newcommand{\Bh}[0]{\mathbf{h}}
\newcommand{\Bi}[0]{\mathbf{i}}
\newcommand{\Bj}[0]{\mathbf{j}}
\newcommand{\Bk}[0]{\mathbf{k}}
\newcommand{\Bl}[0]{\mathbf{l}}
\newcommand{\Bm}[0]{\mathbf{m}}
\newcommand{\Bn}[0]{\mathbf{n}}
\newcommand{\Bo}[0]{\mathbf{o}}
\newcommand{\Bp}[0]{\mathbf{p}}
\newcommand{\Bq}[0]{\mathbf{q}}
\newcommand{\Br}[0]{\mathbf{r}}
\newcommand{\Bs}[0]{\mathbf{s}}
\newcommand{\Bt}[0]{\mathbf{t}}
\newcommand{\Bu}[0]{\mathbf{u}}
\newcommand{\Bv}[0]{\mathbf{v}}
\newcommand{\Bw}[0]{\mathbf{w}}
\newcommand{\Bx}[0]{\mathbf{x}}
\newcommand{\By}[0]{\mathbf{y}}
\newcommand{\Bz}[0]{\mathbf{z}}
\newcommand{\BA}[0]{\mathbf{A}}
\newcommand{\BB}[0]{\mathbf{B}}
\newcommand{\BC}[0]{\mathbf{C}}
\newcommand{\BD}[0]{\mathbf{D}}
\newcommand{\BE}[0]{\mathbf{E}}
\newcommand{\BF}[0]{\mathbf{F}}
\newcommand{\BG}[0]{\mathbf{G}}
\newcommand{\BH}[0]{\mathbf{H}}
\newcommand{\BI}[0]{\mathbf{I}}
\newcommand{\BJ}[0]{\mathbf{J}}
\newcommand{\BK}[0]{\mathbf{K}}
\newcommand{\BL}[0]{\mathbf{L}}
\newcommand{\BM}[0]{\mathbf{M}}
\newcommand{\BN}[0]{\mathbf{N}}
\newcommand{\BO}[0]{\mathbf{O}}
\newcommand{\BP}[0]{\mathbf{P}}
\newcommand{\BQ}[0]{\mathbf{Q}}
\newcommand{\BR}[0]{\mathbf{R}}
\newcommand{\BS}[0]{\mathbf{S}}
\newcommand{\BT}[0]{\mathbf{T}}
\newcommand{\BU}[0]{\mathbf{U}}
\newcommand{\BV}[0]{\mathbf{V}}
\newcommand{\BW}[0]{\mathbf{W}}
\newcommand{\BX}[0]{\mathbf{X}}
\newcommand{\BY}[0]{\mathbf{Y}}
\newcommand{\BZ}[0]{\mathbf{Z}}

\newcommand{\Bzero}[0]{\mathbf{0}}
\newcommand{\Btheta}[0]{\boldsymbol{\theta}}
\newcommand{\Btau}[0]{\boldsymbol{\tau}}
\newcommand{\Bomega}[0]{\boldsymbol{\omega}}

%
% shorthand for unit vectors
%
\newcommand{\acap}[0]{\hat{\Ba}}
\newcommand{\bcap}[0]{\hat{\Bb}}
\newcommand{\ccap}[0]{\hat{\Bc}}
\newcommand{\dcap}[0]{\hat{\Bd}}
\newcommand{\ecap}[0]{\hat{\Be}}
\newcommand{\fcap}[0]{\hat{\Bf}}
\newcommand{\gcap}[0]{\hat{\Bg}}
\newcommand{\hcap}[0]{\hat{\Bh}}
\newcommand{\icap}[0]{\hat{\Bi}}
\newcommand{\jcap}[0]{\hat{\Bj}}
\newcommand{\kcap}[0]{\hat{\Bk}}
\newcommand{\lcap}[0]{\hat{\Bl}}
\newcommand{\mcap}[0]{\hat{\Bm}}
\newcommand{\ncap}[0]{\hat{\Bn}}
\newcommand{\ocap}[0]{\hat{\Bo}}
\newcommand{\pcap}[0]{\hat{\Bp}}
\newcommand{\qcap}[0]{\hat{\Bq}}
\newcommand{\rcap}[0]{\hat{\Br}}
\newcommand{\scap}[0]{\hat{\Bs}}
\newcommand{\tcap}[0]{\hat{\Bt}}
\newcommand{\ucap}[0]{\hat{\Bu}}
\newcommand{\vcap}[0]{\hat{\Bv}}
\newcommand{\wcap}[0]{\hat{\Bw}}
\newcommand{\xcap}[0]{\hat{\Bx}}
\newcommand{\ycap}[0]{\hat{\By}}
\newcommand{\zcap}[0]{\hat{\Bz}}
\newcommand{\thetacap}[0]{\hat{\Btheta}}

%
% to write R^n and C^n in a distinguishable fashion.  Perhaps change this
% to the double lined characters upon figuring out how to do so.
%
\newcommand{\C}[1]{$\mathbb{C}^{#1}$}
\newcommand{\R}[1]{$\mathbb{R}^{#1}$}

%
% various generally useful helpers
%

% derivative of #1 wrt. #2:
\newcommand{\D}[2] {\frac {d#2} {d#1}}

\newcommand{\inv}[1]{\frac{1}{#1}}
\newcommand{\cross}[0]{\times}

\newcommand{\abs}[1]{\lvert{#1}\rvert}
\newcommand{\norm}[1]{\lVert{#1}\rVert}
\newcommand{\innerprod}[2]{\langle{#1}, {#2}\rangle}
\newcommand{\dotprod}[2]{{#1} \cdot {#2}}
\newcommand{\bdotprod}[2]{\left({#1} \cdot {#2}\right)}
\newcommand{\crossprod}[2]{{#1} \cross {#2}}
\newcommand{\tripleprod}[3]{\dotprod{\left(\crossprod{#1}{#2}\right)}{#3}}

\DeclareMathOperator{\Proj}{Proj}
\DeclareMathOperator{\Span}{span}
\DeclareMathOperator{\Sgn}{sgn}
\DeclareMathOperator{\Area}{Area}
\DeclareMathOperator{\Volume}{Volume}

%
% A few miscellaneous things specific to this document
%
\newcommand{\crossop}[1]{\crossprod{#1}{}}

% R2 vector.
\newcommand{\VectorTwo}[2]{
\begin{bmatrix}
 {#1} \\
 {#2}
\end{bmatrix}
}

\newcommand{\VectorN}[1]{
\begin{bmatrix}
{#1}_1 \\
{#1}_2 \\
\vdots \\
{#1}_N \\
\end{bmatrix}
}

\newcommand{\DETuvij}[4]{
\begin{vmatrix}
 {#1}_{#3} & {#1}_{#4} \\
 {#2}_{#3} & {#2}_{#4}
\end{vmatrix}
}

\newcommand{\DETuvwijk}[6]{
\begin{vmatrix}
 {#1}_{#4} & {#1}_{#5} & {#1}_{#6} \\
 {#2}_{#4} & {#2}_{#5} & {#2}_{#6} \\
 {#3}_{#4} & {#3}_{#5} & {#3}_{#6}
\end{vmatrix}
}

\newcommand{\DETuvwxijkl}[8]{
\begin{vmatrix}
 {#1}_{#5} & {#1}_{#6} & {#1}_{#7} & {#1}_{#8} \\
 {#2}_{#5} & {#2}_{#6} & {#2}_{#7} & {#2}_{#8} \\
 {#3}_{#5} & {#3}_{#6} & {#3}_{#7} & {#3}_{#8} \\
 {#4}_{#5} & {#4}_{#6} & {#4}_{#7} & {#4}_{#8} \\
\end{vmatrix}
}

%\newcommand{\DETuvwxyijklm}[10]{
%\begin{vmatrix}
% {#1}_{#6} & {#1}_{#7} & {#1}_{#8} & {#1}_{#9} & {#1}_{#10} \\
% {#2}_{#6} & {#2}_{#7} & {#2}_{#8} & {#2}_{#9} & {#2}_{#10} \\
% {#3}_{#6} & {#3}_{#7} & {#3}_{#8} & {#3}_{#9} & {#3}_{#10} \\
% {#4}_{#6} & {#4}_{#7} & {#4}_{#8} & {#4}_{#9} & {#4}_{#10} \\
% {#5}_{#6} & {#5}_{#7} & {#5}_{#8} & {#5}_{#9} & {#5}_{#10}
%\end{vmatrix}
%}

% R3 vector.
\newcommand{\VectorThree}[3]{
\begin{bmatrix}
 {#1} \\
 {#2} \\
 {#3}
\end{bmatrix}
}



\author{Peeter Joot}
\email{peeter.joot@gmail.com}


\chapter{PHY454H1S\\Continuum Mechanics.  Lecture 5: Constituative relationship.  Taught by Prof. K. Das.}
\label{chap:continuumL5}
%\useCCL
\blogpage{http://sites.google.com/site/peeterjoot2/math2012/continuumL5.pdf}
\date{Jan 25, 2012}
\revisionInfo{continuumL5.tex}

\beginArtWithToc
%\beginArtNoToc

\section{Disclaimer.}

Peeter's lecture notes from class.  May not be entirely coherent.

\section{Review: Cauchy Tetrahedron.}

Referring to figure (\ref{fig:continuumL5:continuumL5fig1})
\begin{figure}[htp]
   \centering
   \includegraphics[totalheight=0.2\textheight]{continuumL5fig1}
   \caption{Cauchy tetrahedron direction cosines.}\label{fig:continuumL5:continuumL5fig1}
\end{figure}

recall that we can decompose our force into components that refer to our direction cosines $n_i = \cos\phi_i$

\begin{align}\label{eqn:continuumL5:10}
f_1 &= \sigma_{11} n_1 + \sigma_{12} n_2 + \sigma_{13} n_3 \\
f_2 &= \sigma_{21} n_1 + \sigma_{22} n_2 + \sigma_{23} n_3 \\
f_3 &= \sigma_{31} n_1 + \sigma_{32} n_2 + \sigma_{33} n_3
\end{align}

Or in tensor form

\begin{equation}\label{eqn:continuumL5:30}
f_i = \sigma_{ij} n_j.
\end{equation}

We call this the traction vector and denote it in vector form as

\begin{equation}\label{eqn:continuumL5:50}
\BT = \Bsigma \cdot \ncap
=
\begin{bmatrix}
\sigma_{11} & \sigma_{12} & \sigma_{13} \\
\sigma_{21} & \sigma_{22} & \sigma_{23} \\
\sigma_{31} & \sigma_{32} & \sigma_{33}
\end{bmatrix}
\begin{bmatrix}
n_1 \\
n_2 \\
n_3
\end{bmatrix}
\end{equation}

\section{Constituative relation.}

Reading: \S 4 and \S 5 from the text \cite{landau1960theory}.

We can find the relationship between stress and strain, both anilitically and experimentally, and call this the Constituative relation.  We prefer to deal with ranges of distortion that are small enough that we can make a linear approximation for this relation.  In general such a linear relationship takes the form

\begin{equation}\label{eqn:continuumL5:70}
\sigma_{ij} = c_{ijkl} e_{kl}.
\end{equation}

Consider the number of components that we are talking about for various rank tensors

\begin{equation}\label{eqn:continuumL5:90}
\begin{array}{l l}
\mbox{$0^\text{th}$ rank tensor} & \mbox{$3^0 = 1$ components} \\
\mbox{$1^\text{st}$ rank tensor} & \mbox{$3^1 = 3$ components} \\
\mbox{$2^\text{nd}$ rank tensor} & \mbox{$3^2 = 9$ components} \\
\mbox{$3^\text{rd}$ rank tensor} & \mbox{$3^3 = 81$ components}
\end{array}
\end{equation}

We have a lot of components, even for a linear relation between stress and strain.  For isotropic materials we model the constituative relation instead as

\begin{equation}\label{eqn:continuumL5:110}
\boxed{
\sigma_{ij} = \lambda e_{kk} \delta_{ij} + 2 \mu e_{ij}.
}
\end{equation}

For such a modelling of the material the (measured) values $\lambda$ and $\mu$ are called the Lam\'e parameters.

It will be useful to compute the trace of the stress tensor in the form of the constituitive relation for the isotropic model.  We find

\begin{align*}
\sigma_{ii}
&= \lambda e_{kk} \delta_{ii} + 2 \mu e_{ii} \\
&= 3 \lambda e_{kk} + 2 \mu e_{ii},
\end{align*}

or

\begin{equation}\label{eqn:continuumL5:n}
\sigma_{ii} = (3 \lambda + 2 \mu) e_{ii}.
\end{equation}

We can now also invert this, to find the trace of the strain tensor in terms of the stress tensor

\begin{equation}\label{eqn:continuumL5:130}
e_{ii} = \frac{\sigma_{ii}}{3 \lambda + 2 \mu}
\end{equation}

Substituting back into our original relationship \ref{eqn:continuumL5:110}, and find

\begin{equation}\label{eqn:continuumL5:110b}
\sigma_{ij} = \lambda \frac{\sigma_{kk}}{3 \lambda + 2 \mu} \delta_{ij} + 2 \mu e_{ij},
\end{equation}

which finally provides an inverted expression with the strain tensor expressed in terms of the stress tensor

\begin{equation}\label{eqn:continuumL5:110b}
2 \mu e_{ij} =
\sigma_{ij} - \lambda \frac{\sigma_{kk}}{3 \lambda + 2 \mu} \delta_{ij}.
\end{equation}

\subsection{Special cases.}
\subsubsection{Hydrostatic compression}

Hydrostatic compression is when we have no shear sttress, only normal components of the stress matrix $\sigma_{ij}$ is nonzero.  Strictly speaking we define Hyrdostatic compression as

\begin{equation}\label{eqn:continuumL5:n}
\sigma_{ij} = -p \delta_{ij},
\end{equation}

i.e. not only diagonal, but with all the components of the stress tensor equal.

We can write the trace of the stress tensor as

\begin{equation}\label{eqn:continuumL5:n}
\sigma_{ii} = - 3 p = (3 \lambda + 2 \mu) e_{kk}.
\end{equation}

Now, from our discussion of the strain tensor $e_{ij}$ recall that we found in the limit

\begin{equation}\label{eqn:continuumL5:n}
dV' = (1 + e_{ii}) dV,
\end{equation}

allowing us to express the change in volume relative to the original volume in terms of the strain trace

\begin{equation}\label{eqn:continuumL5:n}
e_{ii} = \frac{dV' - dV}{dV}.
\end{equation}

Writing that relative volume difference as $\Delta V/V$ we find

\begin{equation}\label{eqn:continuumL5:n}
- 3 p = (3 \lambda + 2 \mu) \frac{\Delta V}{V},
\end{equation}

or

\begin{equation}\label{eqn:continuumL5:n}
- \frac{ p V}{\Delta V} = \left( \lambda + \frac{2}{3} \mu \right) = K,
\end{equation}

where $K$ is called the Bulk modulus.

\subsubsection{Uniaxial stress}

Again illustrated in the plane as in figure (\ref{fig:continuumL5:continuumL5fig2})
\begin{figure}[htp]
   \centering
   \includegraphics[totalheight=0.2\textheight]{continuumL5fig2}
   \caption{Uniaxial stress.}\label{fig:continuumL5:continuumL5fig2}
\end{figure}

\begin{equation}\label{eqn:continuumL5:n}
\Bsigma =
\begin{bmatrix}
\sigma_{11} & 0 & 0\\
0 & 0 & 0 \\
0 & 0 & 0
\end{bmatrix}
\end{equation}

\begin{align*}
2 \mu e_{11}
&= \sigma_{11} - \frac{\lambda ( \sigma_{11} + \cancel{\sigma_{22}} ) }{3 \lambda + 2 \mu} \\
&= \sigma_{11} \frac{3 \lambda + 2 \mu - \lambda }{3 \lambda + 2 \mu} \\
&= 2 \sigma_{11} \frac{\lambda + \mu }{3 \lambda + 2 \mu}
\end{align*}

or

\begin{equation}\label{eqn:continuumL5:n}
%\frac{e_{11}}{\sigma_{11}} = \frac{\lambda + \mu }{\mu(3 \lambda + 2 \mu)}
\frac{\sigma_{11}}{e_{11}} = \frac{\mu(3 \lambda + 2 \mu)}{\lambda + \mu } = E
\end{equation}

where $E$ is Young's modulus.

FIXME: figure (\ref{fig:continuumL5:continuumL5fig3}) reference?

\begin{figure}[htp]
   \centering
   \includegraphics[totalheight=0.2\textheight]{continuumL5fig3}
   \caption{stress associated with Young's modulus}\label{fig:continuumL5:continuumL5fig3}
\end{figure}

We define Poisson's ratio $\nu$ as the quantity

\begin{equation}\label{eqn:continuumL5:n}
\frac{e_{22}}{e_{11}} = \frac{e_{33}}{e_{11}} = - \nu.
\end{equation}

Note that we are still talking about uniaxial stress here.  Referring back to \ref{eqn:continuumL5:110b} we have

\begin{align*}
%2 \mu e_{i j} = \sigma_{i j} - \lambda \frac{\sigma_{k k}}{3 \lambda + 2 \mu} \delta_{i j}
2 \mu e_{2 2}
&= \sigma_{2 2} - \lambda \frac{\sigma_{k k}}{3 \lambda + 2 \mu} \delta_{2 2} \\
&= \sigma_{2 2} - \lambda \frac{\sigma_{k k}}{3 \lambda + 2 \mu}
\end{align*}

FIXME: review from here and make sense of things.

\begin{equation}\label{eqn:continuumL5:n}
2 \mu e_{22} = - \frac{\lambda \sigma_{11}}{3 \lambda + 2 \mu}
\end{equation}

\begin{equation}\label{eqn:continuumL5:n}
\sigma_{11} = \frac{\mu (2 \lambda + 2 \mu}{\lambda + \mu} e_{11}
\end{equation}

\begin{align*}
2 \mu e_{22} 
&= - \frac{\lambda}{3 \lambda + 2 \mu} \frac{ \sigma_{11}}{e_{11}} e_{11} \\
&= - \frac{\lambda}{\cancel{3 \lambda + 2 \mu}} \frac{ \mu (\cancel{3 \lambda + 2\mu})}{\lambda + \mu} e_{11}
\end{align*}

\begin{equation}\label{eqn:continuumL5:n}
\frac{e_{22}}{e_{11}} = -\frac{ \lambda \mu }{2 \mu (\lambda + \mu)} 
\end{equation}

\begin{equation}\label{eqn:continuumL5:n}
\nu = \frac{e_{22}}{e_{11}} = - \frac{\lambda}{2 (\lambda + \mu)}
\end{equation}

\begin{equation}\label{eqn:continuumL5:n}
\mu = \frac{E}{1 + 4 \mu}
\end{equation}

\begin{equation}\label{eqn:continuumL5:n}
\lambda = \frac{2 E \nu}{(1 - 2 \nu)(1 + 4 \mu)}
\end{equation}
 
\begin{align}\label{eqn:continuumL5:n}
e_{11} &= \inv{E}\left( \sigma_{11} - \nu(\sigma_{22} + \sigma_{33}) \right) \\
e_{22} &= \inv{E}\left( \sigma_{22} - \nu(\sigma_{11} + \sigma_{33}) \right) \\
e_{33} &= \inv{E}\left( \sigma_{33} - \nu(\sigma_{11} + \sigma_{22}) \right)
\end{align}

\EndArticle
