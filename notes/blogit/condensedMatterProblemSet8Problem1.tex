%
% Copyright � 2013 Peeter Joot.  All Rights Reserved.
% Licenced as described in the file LICENSE under the root directory of this GIT repository.
%
\makeproblem{Tight Binding Band Structure of a square lattice}{condensedMatter:problemSet8:1}{ 

Consider a two-dimensional square lattice with lattice parameter $a$, 
and thus basis vectors $(a,0)$ and $(0,a)$.  We will construct a 
tight binding band from an $s$-orbital $\phi_s$ that is a solution 
of the Schrodinger equation for the isolated atom, with eigenvalue 
$E_s$: $\hat{\mathcal{H}_A}(\Br - \Br_n) \phi_s(\Br - \Br_n) = 
E_s\phi_s(\Br - \Br_n)$, where $\Br_n$ is a lattice vector.

\makesubproblem{}{condensedMatter:problemSet8:1a}
If the tight binding integrals (defined in class) are 
\begin{eqnarray*}
A &\equiv& - \int\, d\Br\, \phi_s^*(\Br - \Br_n) v(\Br - \Br_n) 
                          \phi_s(\Br - \Br_n)    \ \ \ \mathrm{and} \\ 
B &\equiv& - \int\, d\Br\, \phi_s^*(\Br - \Br_n \pm (a,0) ) v(\Br - \Br_n) 
                          \phi_s(\Br - \Br_n) = B_x  \\
  &=& - \int\, d\Br\, \phi_s^*(\Br - \Br_n \pm (0,a) ) v(\Br - \Br_n) 
                          \phi_s(\Br - \Br_n) = B_y,  
\end{eqnarray*}
show that 
\begin{eqnarray*}
E(\Bk) \simeq E_s - A - 2B(\cos(k_xa) + \cos(k_ya)).
\end{eqnarray*}

\makesubproblem{}{condensedMatter:problemSet8:1b}
Plot $E(\Bk)$ along the following lines in $k$-space: 
  (i) from $\Bk = (0,0)$ to $(2\pi/a, 0)$; 
  (ii) from $\Bk = (0,0)$ to $(0,2\pi/a)$; 
  (iii) from $\Bk = (0,0)$ to $(2\pi/a, 2\pi/a)$. 

\makesubproblem{}{condensedMatter:problemSet8:1c}
What is the bandwidth of this tight-binding band, as a multiple of $B$?

\makesubproblem{}{condensedMatter:problemSet8:1d}
Plot contours of constant energy in the first Brillouin zone (i.e.\ the 
$(k_x,k_y)$ plane, using only the first Brillouin zone), for the following 
energies: (i) $E = E_s - A - 2B$; (ii) $E = E_s - A - B$; (iii) $E = E_s - A$; 
(iv) $E = E_s - A + 2B$. You may use a plotting package, or plot by hand by 
calculating $k_x$ and $k_y$ along a few directions in $k$-space and then 
interpolating.  

Use these plots to identify which constant energy contour represents the 
``half-filled state" (the state where, if all of the levels up to $E = E_F$ are filled 
then there is one electron per site, or $N$ electrons in total, 
where $N$ is the number of atoms in the lattice). 

\makesubproblem{}{condensedMatter:problemSet8:1e}
By considering how the $A$ and $B$ integrals would be affected, discuss in qualitative 
terms how the $E(k)$ relation changes if, instead of atomic $s$-orbitals, the 
basis functions for this band are $p_x$-orbitals.   
Sketch contours of constant 
energy as the `filling' of the band changes from $E_F$ near the bottom of the 
band, to $E_F$ near the top of the band. [9 marks] 

\paragraph{Notes and Hints:}  Note that the $p_x$ orbitals break 
the 90 degree rotational symmetry, so now $B_y \ne B_x$. 
If the lattice parameter is large, 
so that there is weak overlap as is assumed in tight-binding calculations, then 
you will have $\Abs{B_y} \ll \Abs{B_x}$.  Explain why.   
Note too that the $p_x$-orbitals have odd-parity, compared 
with the even-parity of the $s$-orbitals.  
You may find it helpful, in discussing the $B$ orbitals, to sketch the $p_x$ orbitals 
on neighbouring atoms, to visualize how they overlap.
} % makeproblem

\makeanswer{condensedMatter:problemSet8:1}{ 
\makeSubAnswer{}{condensedMatter:problemSet8:1a}

In class (or \citep{ibach2009solid} eq. (7.37)) we found for the tight binding energy at the lattice point at $\Br_n$

\begin{dmath}\label{eqn:condensedMatterProblemSet8Problem1:160}
E(\Bk) \approx E_i - A - B \sum_{m = \text{nn of n}} e^{i \Bk \cdot (\Br_n - \Br_m)}.
\end{dmath}

The nearest neighbor differences are illustrated in \cref{fig:qmSolidsPs8a:qmSolidsPs8aFig1}, which we see are

\imageFigure{qmSolidsPs8aFig1}{Cubic nearest neighbor differences}{fig:qmSolidsPs8a:qmSolidsPs8aFig1}{0.3}

\begin{dmath}\label{eqn:condensedMatterProblemSet8Problem1:180}
\Br_m - \Br \in \{(0, a), (0, -a), (a, 0), (-a, 0)\}.
\end{dmath}

The sum of exponentials is just

\begin{dmath}\label{eqn:condensedMatterProblemSet8Problem1:200}
\sum_m e^{i \Bk \cdot (\Br_n - \Br_m)}
=
e^{ i (k_x, k_y) \cdot (0, -a) }
+
e^{ i (k_x, k_y) \cdot (0, a) }
+
e^{ i (k_x, k_y) \cdot (a, 0) }
+
e^{ i (k_x, k_y) \cdot (-a, 0) }
=
e^{ -i a k_y }
+e^{ i a k_y }
+e^{ -i a k_x }
+e^{ i a k_x }
= 2 \cos a k_x + 2 \cos a k_y
\end{dmath}

\Eqnref{eqn:condensedMatterProblemSet8Problem1:160} takes the form

\begin{dmath}\label{eqn:condensedMatterProblemSet8Problem1:220}
E(\Bk) \approx E_i - A - 2 B \lr{
\cos a k_x + \cos a k_y
},
\end{dmath}

as desired.

\makeSubAnswer{}{condensedMatter:problemSet8:1b}

\paragraph{(i)}  Parameterize this trajectory with

\begin{dmath}\label{eqn:condensedMatterProblemSet8Problem1:20}
\Bk(u) = \frac{2 \pi}{a} u (1, 0),
\end{dmath}

so the energy on this trajectory is

\begin{dmath}\label{eqn:condensedMatterProblemSet8Problem1:40}
E(\Bk(u)) 
= 
E_s - A - 2 B \lr{
\cos 2 \pi u + 1
}
= 
E_s - A - 2 B - 2B \cos 2 \pi u.
\end{dmath}

This is plotted in \cref{fig:qmSolidsPs8bi:qmSolidsPs8biFig1}.

\imageFigure{qmSolidsPs8biFig1}{$E(k)$ on $k \in [(0,0), 2 \pi (1,0)/a]$}{fig:qmSolidsPs8bi:qmSolidsPs8biFig1}{0.3}

\paragraph{(ii)}  Parameterize this trajectory with

\begin{dmath}\label{eqn:condensedMatterProblemSet8Problem1:60}
\Bk(v) = \frac{2 \pi}{a} v (0, 1),
\end{dmath}

so the energy on this trajectory is

\begin{dmath}\label{eqn:condensedMatterProblemSet8Problem1:80}
E(\Bk(v)) 
= 
E_s - A - 2 B \lr{
1 + \cos 2 \pi v 
}
= 
E_s - A - 2 B - 2B \cos 2 \pi v.
\end{dmath}

This, identical to (i) in form, is plotted in \cref{fig:qmSolidsPs8bii:qmSolidsPs8biiFig2}.

\imageFigure{qmSolidsPs8biiFig2}{$E(k)$ on $k \in [(0,0), 2 \pi (0,1)/a]$}{fig:qmSolidsPs8bii:qmSolidsPs8biiFig2}{0.3}

\paragraph{(iii)}  Parameterize this trajectory with

\begin{dmath}\label{eqn:condensedMatterProblemSet8Problem1:100}
\Bk(w) = \frac{2 \pi}{a} w (1, 1),
\end{dmath}

so the energy on this trajectory is

\begin{dmath}\label{eqn:condensedMatterProblemSet8Problem1:120}
E(\Bk(w)) 
= 
E_s - A - 2 B \lr{
2 \cos 2 \pi w 
}
= 
E_s - A - 4 B \cos 2 \pi w.
\end{dmath}

This, identical to (i) and (ii) in form, but with different extremums, is plotted in \cref{fig:qmSolidsPs8biii:qmSolidsPs8biiiFig3}.

\imageFigure{qmSolidsPs8biiiFig3}{$E(k)$ on $k \in [(0,0), 2 \pi (1,1)/a]$}{fig:qmSolidsPs8biii:qmSolidsPs8biiiFig3}{0.3}

\makeSubAnswer{}{condensedMatter:problemSet8:1c}

$E(\Bk)$ ranges from $E_s - A - 2 B(1 + 1)$ to $E_s - A - 2 B ( - 1 - 1)$.  That maximum difference is

\begin{dmath}\label{eqn:condensedMatterProblemSet8Problem1:140}
4 B - (-4B) = 8B.
\end{dmath}

\makeSubAnswer{}{condensedMatter:problemSet8:1d}

\paragraph{(i)} The first contour is that defined by

\begin{dmath}\label{eqn:condensedMatterProblemSet8Problem1:240}
E = E_s - A - 2 B \lr{ \cos k_x a + \cos k_y a } = 
E_s - A - 2 B,
\end{dmath}

or
\begin{dmath}\label{eqn:condensedMatterProblemSet8Problem1:260}
\cos k_x a + \cos k_y a = 1
\end{dmath}

\paragraph{(ii)} Next we have the contour defined by

\begin{dmath}\label{eqn:condensedMatterProblemSet8Problem1:280}
\cos k_x a + \cos k_y a = \inv{2}
\end{dmath}

\paragraph{(iii)} This contour is defined by

\begin{dmath}\label{eqn:condensedMatterProblemSet8Problem1:300}
\cos k_x a + \cos k_y a = 0
\end{dmath}

\paragraph{(iv)} And the last contour defined by

\begin{dmath}\label{eqn:condensedMatterProblemSet8Problem1:320}
\cos k_x a + \cos k_y a = -1
\end{dmath}

These are plotted as functions of $u = k_x a$ and $v = k_y a$ in \cref{fig:qmSolidsPs8d:qmSolidsPs8dFig1}.

\imageFigure{qmSolidsPs8dFig1}{2D Contour plots of selected tight binding energy levels}{fig:qmSolidsPs8d:qmSolidsPs8dFig1}{0.3}

\makeSubAnswer{}{condensedMatter:problemSet8:1e}

TODO.
}
