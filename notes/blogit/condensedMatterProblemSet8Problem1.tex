%
% Copyright � 2013 Peeter Joot.  All Rights Reserved.
% Licenced as described in the file LICENSE under the root directory of this GIT repository.
%
\makeproblem{Tight Binding Band Structure of a square lattice}{condensedMatter:problemSet8:1}{ 

Consider a two-dimensional square lattice with lattice parameter $a$, 
and thus basis vectors $(a,0)$ and $(0,a)$.  We will construct a 
tight binding band from an $s$-orbital $\phi_s$ that is a solution 
of the Schrodinger equation for the isolated atom, with eigenvalue 
$E_s$: $\hat{{\cal H}_A}(\vec{r} - \vec{r}_n) \phi_s(\vec{r} - \vec{r}_n) = 
E_s\phi_s(\vec{r} - \vec{r}_n)$, where $\vec{r}_n$ is a lattice vector.

\makesubproblem{}{condensedMatter:problemSet8:1a}
If the tight binding integrals (defined in class) are 
\begin{eqnarray*}
A &\equiv& - \int\, d\vec{r}\, \phi_s^*(\vec{r} - \vec{r}_n) v(\vec{r} - \vec{r}_n) 
                          \phi_s(\vec{r} - \vec{r}_n)    {\rm\ \ \ and} \\ 
B &\equiv& - \int\, d\vec{r}\, \phi_s^*(\vec{r} - \vec{r}_n \pm (a,0) ) v(\vec{r} - \vec{r}_n) 
                          \phi_s(\vec{r} - \vec{r}_n) = B_x  \\
  &=& - \int\, d\vec{r}\, \phi_s^*(\vec{r} - \vec{r}_n \pm (0,a) ) v(\vec{r} - \vec{r}_n) 
                          \phi_s(\vec{r} - \vec{r}_n) = B_y,  
\end{eqnarray*}
show that 
\begin{eqnarray*}
E(\vec{k}) \simeq E_s - A - 2B(\cos(k_xa) + \cos(k_ya)).
\end{eqnarray*}

\makesubproblem{}{condensedMatter:problemSet8:1b}
Plot $E(\vec{k})$ along the following lines in $k$-space: 
  (i) from $\vec{k} = (0,0)$ to $(2\pi/a, 0)$; 
  (ii) from $\vec{k} = (0,0)$ to $(0,2\pi/a)$; 
  (iii) from $\vec{k} = (0,0)$ to $(2\pi/a, 2\pi/a)$. 

\makesubproblem{}{condensedMatter:problemSet8:1c}
What is the bandwidth of this tight-binding band, as a multiple of $B$?

\makesubproblem{}{condensedMatter:problemSet8:1d}
Plot contours of constant energy in the first Brillouin zone (i.e.\ the 
$(k_x,k_y)$ plane, using only the first Brillouin zone), for the following 
energies: (i) $E = E_s - A - 2B$; (ii) $E = E_s - A - B$; (iii) $E = E_s - A$; 
(iv) $E = E_s - A + 2B$. You may use a plotting package, or plot by hand by 
calculating $k_x$ and $k_y$ along a few directions in $k$-space and then 
interpolating.  

Use these plots to identify which constant energy contour represents the 
``half-filled state" (the state where, if all of the levels up to $E = E_F$ are filled 
then there is one electron per site, or $N$ electrons in total, 
where $N$ is the number of atoms in the lattice). 

\makesubproblem{}{condensedMatter:problemSet8:1e}
By considering how the $A$ and $B$ integrals would be affected, discuss in qualitative 
terms how the $E(k)$ relation changes if, instead of atomic $s$-orbitals, the 
basis functions for this band are $p_x$-orbitals.   
Sketch contours of constant 
energy as the `filling' of the band changes from $E_F$ near the bottom of the 
band, to $E_F$ near the top of the band. [9 marks] 

\paragraph{Notes and Hints:}  Note that the $p_x$ orbitals break 
the 90 degree rotational symmetry, so now $B_y \ne B_x$. 
If the lattice parameter is large, 
so that there is weak overlap as is assumed in tight-binding calculations, then 
you will have $|B_y| \ll |B_x|$.  Explain why.   
Note too that the $p_x$-orbitals have odd-parity, compared 
with the even-parity of the $s$-orbitals.  
You may find it helpful, in discussing the $B$ orbitals, to sketch the $p_x$ orbitals 
on neighbouring atoms, to visualize how they overlap.)  }

} % makeproblem

\makeanswer{condensedMatter:problemSet8:1}{ 
\makeSubAnswer{}{condensedMatter:problemSet8:1a}

TODO.
\makeSubAnswer{}{condensedMatter:problemSet8:1b}

TODO.
\makeSubAnswer{}{condensedMatter:problemSet8:1c}

TODO.
\makeSubAnswer{}{condensedMatter:problemSet8:1d}

TODO.
\makeSubAnswer{}{condensedMatter:problemSet8:1e}

TODO.
}

