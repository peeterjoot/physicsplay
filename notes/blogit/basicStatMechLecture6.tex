%
% Copyright � 2013 Peeter Joot.  All Rights Reserved.
% Licenced as described in the file LICENSE under the root directory of this GIT repository.
%
\newcommand{\authorname}{Peeter Joot}
\newcommand{\email}{peeterjoot@protonmail.com}
\newcommand{\basename}{FIXMEbasenameUndefined}
\newcommand{\dirname}{notes/FIXMEdirnameUndefined/}

\renewcommand{\basename}{basicStatMechLecture6}
\renewcommand{\dirname}{notes/phy452/}
\newcommand{\keywords}{Statistical mechanics, PHY452H1S}
\newcommand{\authorname}{Peeter Joot}
\newcommand{\onlineurl}{http://sites.google.com/site/peeterjoot2/math2013/\basename.pdf}
\newcommand{\sourcepath}{\dirname\basename.tex}
\newcommand{\generatetitle}[1]{\chapter{#1}}

\newcommand{\vcsinfo}{%
\section*{}
\noindent{\color{DarkOliveGreen}{\rule{\linewidth}{0.1mm}}}
\paragraph{Document version}
%\paragraph{\color{Maroon}{Document version}}
{
\small
\begin{itemize}
\item Available online at:\\ 
\href{\onlineurl}{\onlineurl}
\item Git Repository: \input{./.revinfo/gitRepo.tex}
\item Source: \sourcepath
\item last commit: \input{./.revinfo/gitCommitString.tex}
\item commit date: \input{./.revinfo/gitCommitDate.tex}
\end{itemize}
}
}

%\PassOptionsToPackage{dvipsnames,svgnames}{xcolor}
\PassOptionsToPackage{square,numbers}{natbib}
\documentclass{scrreprt}

\usepackage[left=2cm,right=2cm]{geometry}
\usepackage[svgnames]{xcolor}
\usepackage{peeters_layout}

\usepackage{natbib}

\usepackage[
colorlinks=true,
bookmarks=false,
pdfauthor={\authorname, \email},
backref 
]{hyperref}

% http://tex.stackexchange.com/questions/75773/how-to-reference-problems-by-the-text-label-in-an-exercise-envioronment
\usepackage[english]{cleveref}
\crefname{Exercise}{exercise}{exercises}
\Crefname{Exercise}{Exercise}{Exercises}

\RequirePackage{titlesec}
\RequirePackage{ifthen}

% http://stackoverflow.com/questions/4932910/date-in-the-tabular-environment
\makeatletter
\let\insertdate\@date
\makeatother

\titleformat{\chapter}[display]
{\bfseries\Large}
{\color{DarkSlateGrey}\filleft \authorname
\ifthenelse{\isundefined{\studentnumber}}{}{\\ \studentnumber}
\ifthenelse{\isundefined{\email}}{}{\\ \email}
\ifthenelse{\isundefined{\dateintitle}}{}{\\ \insertdate}
%\ifthenelse{\isundefined{\coursename}}{}{\\ \coursename} % put in title instead.
}
{4ex}
{\color{DarkOliveGreen}{\titlerule}\color{Maroon}
\vspace{2ex}%
\filright}
[\vspace{2ex}%
\color{DarkOliveGreen}\titlerule
]

\newcommand{\beginArtWithToc}[0]{\begin{document}\tableofcontents}
\newcommand{\beginArtNoToc}[0]{\begin{document}}
\newcommand{\EndNoBibArticle}[0]{\end{document}}
\newcommand{\EndArticle}[0]{\bibliography{Bibliography}\bibliographystyle{plainnat}\end{document}}

% 
%\newcommand{\citep}[1]{\cite{#1}}

\colorSectionsForArticle



%\usepackage[draft]{fixme}
%\fxusetheme{color}
%\fxwarning{review lecture 6}{work through this lecture in detail.}

\beginArtNoToc
\generatetitle{PHY452H1S Basic Statistical Mechanics.  Lecture 6: XXX.  Taught by Prof.\ Arun Paramekanti}
%\chapter{XXX}
\label{chap:basicStatMechLecture6}

\section{Disclaimer}

Peeter's lecture notes from class.  May not be entirely coherent.

\section{Liouville's theorem}

We've looked at the continuity equation of phase space density

\begin{equation}\label{eqn:basicStatMechLecture6:n}
0 = 
\PD{t}{\rho} + \sum_{i_\alpha} \left(
\PD{pialpha} \left( \rho \dot{p}_ialpha \right) + \PD{xialpha}{\rho \dot{x}_ialpha
\right)
\end{equation}

which with

\begin{equation}\label{eqn:basicStatMechLecture6:n}
\PD{pialpha}{pdotialpha} + \PD{xialpha}{xdotialpha} = 0
\end{equation}

led us to \underline{Liouville's theorem}

\begin{equation}\label{eqn:basicStatMechLecture6:n}\
\myBoxed{
\ddt{\rho}(x, p, t) = 0
}.
\end{equation}

We define \underline{Ergodic}, meaning that with time, as you wait for $t \rightarrow \infty$, all \underline{available} phase space will be covered.  Not all systems are neccessarily ergodic, but the hope is that all sufficiently complicated systems will be so.

%\EndArticle
\EndNoBibArticle
