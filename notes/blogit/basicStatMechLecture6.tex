%
% Copyright � 2013 Peeter Joot.  All Rights Reserved.
% Licenced as described in the file LICENSE under the root directory of this GIT repository.
%
\newcommand{\authorname}{Peeter Joot}
\newcommand{\email}{peeterjoot@protonmail.com}
\newcommand{\basename}{FIXMEbasenameUndefined}
\newcommand{\dirname}{notes/FIXMEdirnameUndefined/}

\renewcommand{\basename}{basicStatMechLecture6}
\renewcommand{\dirname}{notes/phy452/}
\newcommand{\keywords}{Statistical mechanics, PHY452H1S}
\newcommand{\authorname}{Peeter Joot}
\newcommand{\onlineurl}{http://sites.google.com/site/peeterjoot2/math2013/\basename.pdf}
\newcommand{\sourcepath}{\dirname\basename.tex}
\newcommand{\generatetitle}[1]{\chapter{#1}}

\newcommand{\vcsinfo}{%
\section*{}
\noindent{\color{DarkOliveGreen}{\rule{\linewidth}{0.1mm}}}
\paragraph{Document version}
%\paragraph{\color{Maroon}{Document version}}
{
\small
\begin{itemize}
\item Available online at:\\ 
\href{\onlineurl}{\onlineurl}
\item Git Repository: \input{./.revinfo/gitRepo.tex}
\item Source: \sourcepath
\item last commit: \input{./.revinfo/gitCommitString.tex}
\item commit date: \input{./.revinfo/gitCommitDate.tex}
\end{itemize}
}
}

%\PassOptionsToPackage{dvipsnames,svgnames}{xcolor}
\PassOptionsToPackage{square,numbers}{natbib}
\documentclass{scrreprt}

\usepackage[left=2cm,right=2cm]{geometry}
\usepackage[svgnames]{xcolor}
\usepackage{peeters_layout}

\usepackage{natbib}

\usepackage[
colorlinks=true,
bookmarks=false,
pdfauthor={\authorname, \email},
backref 
]{hyperref}

% http://tex.stackexchange.com/questions/75773/how-to-reference-problems-by-the-text-label-in-an-exercise-envioronment
\usepackage[english]{cleveref}
\crefname{Exercise}{exercise}{exercises}
\Crefname{Exercise}{Exercise}{Exercises}

\RequirePackage{titlesec}
\RequirePackage{ifthen}

% http://stackoverflow.com/questions/4932910/date-in-the-tabular-environment
\makeatletter
\let\insertdate\@date
\makeatother

\titleformat{\chapter}[display]
{\bfseries\Large}
{\color{DarkSlateGrey}\filleft \authorname
\ifthenelse{\isundefined{\studentnumber}}{}{\\ \studentnumber}
\ifthenelse{\isundefined{\email}}{}{\\ \email}
\ifthenelse{\isundefined{\dateintitle}}{}{\\ \insertdate}
%\ifthenelse{\isundefined{\coursename}}{}{\\ \coursename} % put in title instead.
}
{4ex}
{\color{DarkOliveGreen}{\titlerule}\color{Maroon}
\vspace{2ex}%
\filright}
[\vspace{2ex}%
\color{DarkOliveGreen}\titlerule
]

\newcommand{\beginArtWithToc}[0]{\begin{document}\tableofcontents}
\newcommand{\beginArtNoToc}[0]{\begin{document}}
\newcommand{\EndNoBibArticle}[0]{\end{document}}
\newcommand{\EndArticle}[0]{\bibliography{Bibliography}\bibliographystyle{plainnat}\end{document}}

% 
%\newcommand{\citep}[1]{\cite{#1}}

\colorSectionsForArticle



%\usepackage[draft]{fixme}
%\fxusetheme{color}
%\fxwarning{review lecture 6}{work through this lecture in detail.}

\beginArtNoToc
\generatetitle{PHY452H1S Basic Statistical Mechanics.  Lecture 6: Volumes in phase space.  Taught by Prof.\ Arun Paramekanti}
%\chapter{Volumes in phase space}
\label{chap:basicStatMechLecture6}

\section{Disclaimer}

Peeter's lecture notes from class.  May not be entirely coherent.

\section{Liouville's theorem}

We've looked at the continuity equation of phase space density

\begin{equation}\label{eqn:basicStatMechLecture6:20}
0 = 
\PD{t}{\rho} + \sum_{i_\alpha} \left(
\PD{p_{i_\alpha}}{} \left( \rho \dot{p}_{i_\alpha} \right) + \PD{x_{i_\alpha}}{\left( \rho \dot{x}_{i_\alpha} \right) }
\right)
\end{equation}

which with

\begin{equation}\label{eqn:basicStatMechLecture6:40}
\PD{p_{i_\alpha}}{\dot{p}_{i_\alpha}} + \PD{x_{i_\alpha}}{\dot{x}_{i_\alpha}} = 0
\end{equation}

led us to \underline{Liouville's theorem}

\begin{equation}\label{eqn:basicStatMechLecture6:n}\
\myBoxed{
\ddt{\rho}(x, p, t) = 0
}.
\end{equation}

We define \underline{Ergodic}, meaning that with time, as you wait for $t \rightarrow \infty$, all \underline{available} phase space will be covered.  Not all systems are neccessarily ergodic, but the hope is that all sufficiently complicated systems will be so.

We hope that

\begin{equation}\label{eqn:basicStatMechLecture6:60}
\rho(x, p, t \rightarrow \infty) \implies \PD{t}{\rho} = 0 \qquad \mbox{in steady state}
\end{equation}

In particular for $\rho = \text{constant}$, we see that our continuity equation \ref{eqn:basicStatMechLecture6:20} results in \ref{eqn:basicStatMechLecture6:40}. % (essentially a statement that the magnitude of the mixed partials of the Hamiltonian are equal).

For example in a SHO system with a cyclic phase space, as in \cref{fig:basicStatMechLecture6:basicStatMechLecture6Fig1}.

\imageFigure{basicStatMechLecture6Fig1}{Phase space volume trajectory}{fig:basicStatMechLecture6:basicStatMechLecture6Fig1}{0.3}

\begin{equation}\label{eqn:basicStatMechLecture6:80}
\expectation{A} = \inv{\tau} \int_0^\tau dt A( x_0(t), p_0(t) ),
\end{equation}

or equivalently with an \underline{ensemble average}, imagining that we are averaging over a number of different systems

\begin{equation}\label{eqn:basicStatMechLecture6:100}
\expectation{A} = \inv{\tau} \int dx dp A( x, p ) 
\mathLabelBox{
\rho(x, p)
}{constant}
\end{equation}

If we say that 

\begin{equation}\label{eqn:basicStatMechLecture6:120}
\rho(x, p) = \text{constant} = \inv{\Omega},
\end{equation}

so that 

\begin{equation}\label{eqn:basicStatMechLecture6:140}
\expectation{A} = \inv{\Omega} \int dx dp A( x, p ) 
\end{equation}

then what is this constant.  We fix this by the constraint

\begin{equation}\label{eqn:basicStatMechLecture6:160}
\int dx dp \rho(x, p) = 1
\end{equation}

So, $\Omega$ is the allowed ``volume'' of phase space, the number of states that the system can take that is consistent with conservation of energy.

What's the probability for a given configuration.  We'll have to enumerate all the possible configurations.  For a coin toss example, we can also ask how many configurations exist where the sum of ``coin tosses'' are fixed.

\section{A worked example: Ideal gas calculation of $\Omega$}

\begin{enumerate}
\item $N$ gas atoms at phase space points $\Bx_i, \Bp_i$
\item constrained to volume $V$
\item Energy fixed at $E$.
\end{enumerate}

\begin{dmath}\label{eqn:basicStatMechLecture6:180}
\Omega(N, V, E) = \int_V 
d\Bx_1 
d\Bx_2
\cdots
d\Bx_N 
\int
d\Bp_1 
d\Bp_2
\cdots
d\Bp_N 
\delta \left(
E 
- \frac{\Bp_1^2}{2m}
- \frac{\Bp_2^2}{2m}
\cdots
- \frac{\Bp_N^2}{2m}
\right)
=
\mathLabelBox
[
   labelstyle={xshift=2cm},
   linestyle={out=270,in=90, latex-}
]
{
V^N
}{Real space volume, not $N$ dimensional ``volume''}
\int
d\Bp_1 
d\Bp_2
\cdots
d\Bp_N 
\delta \left(
E 
- \frac{\Bp_1^2}{2m}
- \frac{\Bp_2^2}{2m}
\cdots
- \frac{\Bp_N^2}{2m}
\right)
\end{dmath}

With $\gamma$ defined implicitly by

\begin{equation}\label{eqn:basicStatMechLecture6:200}
\frac{d\gamma}{dE} = \Omega
\end{equation}

so that with Heavyside theta as in \cref{fig:basicStatMechLecture6:basicStatMechLecture6Fig2}.

\begin{subequations}
\begin{equation}\label{eqn:basicStatMechLecture6:220}
\Theta(x) = 
\left\{
\begin{array}{l l}
1 & \quad x \ge 0 \\
0 & \quad x < 0
\end{array}
\right.
\end{equation}
\begin{equation}\label{eqn:basicStatMechLecture6:480}
\frac{d\Theta}{dx} = \delta(x),
\end{equation}
\end{subequations}

\imageFigure{basicStatMechLecture6Fig2}{Heavyside theta, $\Theta(x)$}{fig:basicStatMechLecture6:basicStatMechLecture6Fig2}{0.3}

we have

\begin{equation}\label{eqn:basicStatMechLecture6:240}
\gamma(N, V, E) = V^N 
\int
d\Bp_1 
d\Bp_2
\cdots
d\Bp_N 
\Theta \left(
E 
- \sum_i \frac{\Bp_i^2}{2m}
\right)
\end{equation}

In three dimensions $(p_x, p_y, p_z)$, the dimension of momentum part of the phase space is 3.  In general the dimension of the space is $3N$.  Here

\begin{equation}\label{eqn:basicStatMechLecture6:440}
\int
d\Bp_1 
d\Bp_2
\cdots
d\Bp_N 
\Theta \left(
E 
- \sum_i \frac{\Bp_i^2}{2m}
\right),
\end{equation}

is the volume of a ``sphere'' in $3N$- dimensions, which we found in the problem set to be

\begin{subequations}
\begin{equation}\label{eqn:basicStatMechLecture6:460}
V_{m} 
= 
\frac{ \pi^{m/2} R^{m} }
{
   \Gamma\left( m/2 + 1 \right)
}.
\end{equation}
\begin{equation}\label{eqn:basicStatMechLecture6:320}
\Gamma(x) = \int_0^\infty dy e^{-y} y^{x-1}
\end{equation}
\begin{equation}\label{eqn:basicStatMechLecture6:340}
\Gamma(x + 1) = x \Gamma(x) = x!
\end{equation}
\end{subequations}

Since we have

\begin{equation}\label{eqn:basicStatMechLecture6:260}
\Bp_1^2 + \cdots \Bp_N^2 \le 2 m E
\end{equation}

the radius is

\begin{equation}\label{eqn:basicStatMechLecture6:280}
\text{radius} = \sqrt{ 2 m E}.
\end{equation}

This gives

\begin{dmath}\label{eqn:basicStatMechLecture6:300}
\gamma(N, V, E) 
= V^N \frac{ \pi^{3 N/2} ( 2 m E)^{3 N/2}}{\Gamma( 3N/2 + 1) }
= V^N \frac{2}{3N} \frac{ \pi^{3 N/2} ( 2 m E)^{3 N/2}}{\Gamma( 3N/2 ) },
\end{dmath}

and
\begin{equation}\label{eqn:basicStatMechLecture6:360}
\Omega(N, V, E) = V^N \pi^{3 N/2} ( 2 m E)^{3 N/2 - 1} \frac{2 m}{\Gamma( 3N/2 ) }
\end{equation}

This result is almost correct, and we have to correct in 2 ways.  We have to fix the counting since we need an assumption that all the particles are indistinguishable.

\begin{enumerate}
\item Indistinguishability.  We must divide by $N!$.
\item $\Omega$ is not dimensionless.  We need to divide by $h^{3N}$, where $h$ is Planck's constant.
\end{enumerate}

In the real world we have to consider this as a quantum mechanical system.  Imagine a two dimensional phase space.  The allowed points are illustrated in \cref{fig:basicStatMechLecture6:basicStatMechLecture6Fig3}.

\imageFigure{basicStatMechLecture6Fig3}{Phase space volume adjustment for the uncertainty principle}{fig:basicStatMechLecture6:basicStatMechLecture6Fig3}{0.3}

Since $\Delta x \Delta p \sim \hbar$, the question of how many boxes there are, we calculate the total volume, and then divide by the volume of each box.  This sort of handwaving wouldn't be required if we did a proper quantum mechanical treatment.

The corrected result is

\begin{equation}\label{eqn:basicStatMechLecture6:380}
\boxed{
\Omega_{\mathrm{correct}} = \frac{V^N}{N!} \inv{h^{3N}} \frac{( 2 \pi m E)^{3 N/2 }}{E} \frac{1}{\Gamma( 3N/2 ) }
}
\end{equation}

\section{To come}

We'll look at entropy

\begin{equation}\label{eqn:basicStatMechLecture6:420}
\mathLabelBox
[
   labelstyle={xshift=-2cm},
   linestyle={out=270,in=90, latex-}
]
{S}{Entropy}
 = 
\mathLabelBox
[
   labelstyle={below of=m\themathLableNode, below of=m\themathLableNode}
]
{k_{\mathrm{B}}}{Boltzmann's constant}
\ln 
\mathLabelBox
[
   labelstyle={xshift=2cm},
   linestyle={out=270,in=90, latex-}
]
{
\Omega_{\mathrm{correct}}
}{phase space volume (number of configurations)}
\end{equation}

%\EndArticle
\EndNoBibArticle
