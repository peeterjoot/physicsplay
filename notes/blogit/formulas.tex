%
% Copyright � 2018 Peeter Joot.  All Rights Reserved.
% Licenced as described in the file LICENSE under the root directory of this GIT repository.
%
%{
\input{../latex/blogpost.tex}
\renewcommand{\basename}{formulas}
%\renewcommand{\dirname}{notes/phy1520/}
\renewcommand{\dirname}{notes/ece1228-electromagnetic-theory/}
%\newcommand{\dateintitle}{}
%\newcommand{\keywords}{}

\input{../latex/peeter_prologue_print2.tex}

\usepackage{peeters_layout_exercise}
\usepackage{peeters_braket}
\usepackage{peeters_figures}
\usepackage{siunitx}
\usepackage{macros_cal}
%\usepackage{mhchem} % \ce{}
%\usepackage{macros_bm} % \bcM
%\usepackage{macros_qed} % \qedmarker
%\usepackage{txfonts} % \ointclockwise

\beginArtNoToc

\generatetitle{Some fun formulas}
%\chapter{Some fun formulas}
%\label{chap:formulas}
% \citep{sakurai2014modern} pr X.Y
% \citep{pozar2009microwave}
% \citep{qftLectureNotes}
% \citep{doran2003gap}
% \citep{jackson1975cew}
% \citep{griffiths1999introduction}

Lance got a custom glow in the dark ``math, physics, and chemistry'' shirt for his birthday a year or two ago.

I included some of the what I thought were the most interesting relations:

\begin{itemize}
   \item Gamma function, which generalizes factorial to non-integer values.
\begin{dmath}\label{eqn:formulas:20}
   \Gamma(z + 1) = \int_0^\infty t^z e^{-t} dt.
\end{dmath}
This formula satisfies \( n! = \Gamma(n + 1) \).
\item Euler's:
\begin{dmath}\label{eqn:formulas:40}
   e^{n i \theta} = \lr{ \cos\theta + i \sin\theta }^n.
\end{dmath}
\item Schr\"{o}dinger's equation
\begin{dmath}\label{eqn:formulas:60}
   i \, \hbar \PD{t}{} \ket{\psi} = H \ket{\psi}.
\end{dmath}
\item Taylor series
\begin{dmath}\label{eqn:formulas:80}
   f(x) = \sum_{k = 0}^\infty \frac{f^{(k)}(0)}{k!} x^k.
\end{dmath}
\item Euler-Lagrange equations:
\begin{dmath}\label{eqn:formulas:100}
   \PD{x_i}{\LL} = \frac{d}{dt} \PD{\dot{x}_i}{\LL}.
\end{dmath}
These formulas can be used to express most dynamics relations.  You can think of them as basically being the consequence that physical laws are either inherently greedy or lazy.
\item Stokes' theorem, in its geoemetric algebra form
\begin{dmath}\label{eqn:formulas:120}
   \oint_{\partial V} d^n \Bx \cdot T = \int_V d^n \Bx \cdot \lr{ \spacegrad \cdot T }.
\end{dmath}
\item Quantum commutator relationships between position and momentum
\begin{dmath}\label{eqn:formulas:140}
   \antisymmetric{X}{P} = i \, \hbar.
\end{dmath}
\item Fourier transform
\begin{dmath}\label{eqn:formulas:160}
   \tilde{f}(\omega) = \int_{-\infty}^\infty f(t) e^{-i \omega t} dt.
\end{dmath}
\item Vector product
\begin{dmath}\label{eqn:formulas:180}
   \Ba \Bb = \Ba \cdot \Bb + \Ba \wedge \Bb.
\end{dmath}
In geometric algebra the vector product has a dot and wedge product split.  In \R{3} you can write this as
\( \Ba \Bb = \Ba \cdot \Bb + I (\Ba \cross \Bb) \), where \( I = \Be_1 \Be_2 \Be_3 \).
\item Relativisitic energy (Einstein's)
\begin{dmath}\label{eqn:formulas:200}
   E = \frac{m c^2}{\sqrt{1 - (\Bv/c)^2}}.
\end{dmath}
\item Cauchy contour integral relationships for the nth derivative
\begin{dmath}\label{eqn:formulas:220}
   f^{(n)}(s) = \frac{n!}{2 \pi i} \int_C dz \frac{f(z)}{(z-s)^{n+1}}
\end{dmath}
\item Maxwell's equations
\begin{dmath}\label{eqn:formulas:240}
\begin{aligned}
   \partial_\mu F^{\mu\nu} &= J^\nu \\
   \epsilon^{\mu \nu \rho \sigma} \partial_\nu F_{\rho \sigma} &= 0.
\end{aligned}
\end{dmath}
I'm partial to the geometric algebra form of Maxwell's equations \( \grad F = J \), but that wouldn't have looked as good on the shirt.
\item Dirac equation
\begin{dmath}\label{eqn:formulas:260}
   0 = \lr{ i \gamma^\mu \partial_\mu - m } \psi.
\end{dmath}
I didn't use the slash notation, and set \( c = 1 \), as is conventional in field theory.
\end{itemize}

Lance's physics teacher didn't know all the relations on his shirt, but I'm not suprised, since there's some fairly advanced material here.

%}
\EndArticle
%\EndNoBibArticle
