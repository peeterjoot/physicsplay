%
% Copyright � 2012 Peeter Joot.  All Rights Reserved.
% Licenced as described in the file LICENSE under the root directory of this GIT repository.
%
\newcommand{\authorname}{Peeter Joot}
\newcommand{\email}{peeter.joot@utoronto.ca}
\newcommand{\studentnumber}{920798560}
\newcommand{\basename}{FIXMEbasenameUndefined}
\newcommand{\dirname}{notes/FIXMEdirnameUndefined/}

\renewcommand{\basename}{modernOpticsProblem Set1}
\renewcommand{\dirname}{notes/phy485/}
\newcommand{\keywords}{Optics, PHY485H1F}
\newcommand{\dateintitle}{}
\newcommand{\authorname}{Peeter Joot}
\newcommand{\onlineurl}{http://sites.google.com/site/peeterjoot2/math2013/\basename.pdf}
\newcommand{\sourcepath}{\dirname\basename.tex}
\newcommand{\generatetitle}[1]{\chapter{#1}}

\newcommand{\vcsinfo}{%
\section*{}
\noindent{\color{DarkOliveGreen}{\rule{\linewidth}{0.1mm}}}
\paragraph{Document version}
%\paragraph{\color{Maroon}{Document version}}
{
\small
\begin{itemize}
\item Available online at:\\ 
\href{\onlineurl}{\onlineurl}
\item Git Repository: \input{./.revinfo/gitRepo.tex}
\item Source: \sourcepath
\item last commit: \input{./.revinfo/gitCommitString.tex}
\item commit date: \input{./.revinfo/gitCommitDate.tex}
\end{itemize}
}
}

%\PassOptionsToPackage{dvipsnames,svgnames}{xcolor}
\PassOptionsToPackage{square,numbers}{natbib}
\documentclass{scrreprt}

\usepackage[left=2cm,right=2cm]{geometry}
\usepackage[svgnames]{xcolor}
\usepackage{peeters_layout}

\usepackage{natbib}

\usepackage[
colorlinks=true,
bookmarks=false,
pdfauthor={\authorname, \email},
backref 
]{hyperref}

% http://tex.stackexchange.com/questions/75773/how-to-reference-problems-by-the-text-label-in-an-exercise-envioronment
\usepackage[english]{cleveref}
\crefname{Exercise}{exercise}{exercises}
\Crefname{Exercise}{Exercise}{Exercises}

\RequirePackage{titlesec}
\RequirePackage{ifthen}

% http://stackoverflow.com/questions/4932910/date-in-the-tabular-environment
\makeatletter
\let\insertdate\@date
\makeatother

\titleformat{\chapter}[display]
{\bfseries\Large}
{\color{DarkSlateGrey}\filleft \authorname
\ifthenelse{\isundefined{\studentnumber}}{}{\\ \studentnumber}
\ifthenelse{\isundefined{\email}}{}{\\ \email}
\ifthenelse{\isundefined{\dateintitle}}{}{\\ \insertdate}
%\ifthenelse{\isundefined{\coursename}}{}{\\ \coursename} % put in title instead.
}
{4ex}
{\color{DarkOliveGreen}{\titlerule}\color{Maroon}
\vspace{2ex}%
\filright}
[\vspace{2ex}%
\color{DarkOliveGreen}\titlerule
]

\newcommand{\beginArtWithToc}[0]{\begin{document}\tableofcontents}
\newcommand{\beginArtNoToc}[0]{\begin{document}}
\newcommand{\EndNoBibArticle}[0]{\end{document}}
\newcommand{\EndArticle}[0]{\bibliography{Bibliography}\bibliographystyle{plainnat}\end{document}}

% 
%\newcommand{\citep}[1]{\cite{#1}}

\colorSectionsForArticle


\beginArtNoToc
\generatetitle{PHY485H1F Modern Optics.  Problem Set 1: Geometric Optics}
%\chapter{Geometric Optics}
\label{chap:modernOpticsProblem Set1}

\makeproblem{ABCD Matrices}{modernOptics:problemSet1:1}{
Using the ABCD matrices from lecture, prove these well-known rules of geometric optics. In each case, {\bf make an illustration}, tracing some important rays that illustrate the rule. 
\begin{enumerate}
\item[(a)] {\em An image is formed when $1/f = 1/s_o + 1/s_i$.} Solve this problem using the result we found in class: when B=0 for a system matrix, the input and output are conjugate planes.
\item[(b)] {\em An image with magnification $-x'/f$ is formed when $x x' = f^2$.} Repeat part (a), but in ``Newton's form'': replace $s_o$ with $f + x$, and replace $s_i$ with $f + x'$. 
\item[(c)] {\em The position distribution at the focus of a lens is the angular position of the incident beam.} (In other words, a lens does a kind of Fourier transform, as you may know already.) Find where the input plane has to be located for $y_o = f \alpha_i$.
\item[(d)] {\em Two identical lenses spaced by $2f$ image an object at $f$ with unity magnification.}
\item[(e)] Two identical lenses spaced by $2f$ are {\em telecentric}, meaning that an object at $f+x$ from the first lens has a magnification independent of $x$, in contrast to a simple lens.
\item[(f)] A lens and a flat mirror spaced by distance $f$ create a {\em cat's eye}. What are its properties? Consider, in particular, and emitter located $f$ in front of the Cat's eye and located at $y_i = 0$. 
\end{enumerate}
}

\makeanswer{modernOptics:problemSet1:1}{ }

\makeproblem{Ray in a linear index gradient}{modernOptics:problemSet1:2}{
What is the shape of a ray moving into a linear index gradient (figure \ref{fig:modernOpticsProblemSet1:FigureRayLinGrad})? You'd expect something like a parabola from the intuition that the Ray Equation is `Newton-like'. Find out what you actually get! To establish some conventions: take $n(y) = n_0 - \beta y$; choose parameterization of the ray so that $s=0$ at the top of the trajectory: $\Br(0) = 0$, and $d \Br /ds = \hat{x}$ at $s=0$. In this case the ray will remain in the $xy$ plane, so your task is to find $x(s)$ and $y(s)$. 
\imageFigure{FigureRayLinGrad}{}{fig:modernOpticsProblemSet1:FigureRayLinGrad}{0.2}
\begin{enumerate}
\item[(a)] Start with the Ray Equation $\frac{d}{ds} \{ n \frac{d}{ds} \Br \} = \spacegrad n$. Integrate both sides with respect to $s$, and use initial conditions to determine contants of integration. You should be left with two first-order differential equations.
\item[(b)] Solve the $dy/ds$ equation first, by integrating again with respect to $s$. Give an exact expression for $z(s)$. Also give approximate expressions for $y(s)$  in two limits: small $s$, and large $s$. 
\item[(c)] Now solve the $dx/ds$ equation. Again, give an exact expression for $x(s)$, and approximate expressions for $x(s)$  in two limits: small $s$, and large $s$.
\item[(d)] Combine your results to give $x(y)$. (This may seem strange, but an exact result for $y(x)$ is hard to write down. You'll have to restrict yourself to $x>0$ for this curve to be functional.) You can again find an exact result, an small-s approximation, and a large-s approximation.
\item[(e)] Is the trajectory of an optical ray a parabola in any limit? If so, what is gravitational acceleration?
\item[(f)] Use your favorite software (mathematica, ...) to make a plot of $x(s)$, $y(s)$, and $x(y)$. In each plot, compare the exact expression (as a solid line) to the two limiting expressions (as dashed lines). Nondimensionalize in terms of $L=n_0/\beta$: in other words, use the variables $x/L$, $y/L$, and $s/L$.
\end{enumerate}
}

\makeanswer{modernOptics:problemSet1:2}{ 
\begin{enumerate}
\item[(a)] Our ray equation, after computation of the gradient of the index of refraction for the material becomes

\begin{dmath}\label{eqn:modernOpticsProblemSet1:10}
\dds{} \left( n(\Br) \dds{\Br} \right) 
= \spacegrad n(\Br) 
= \spacegrad \left( n_0 - \beta y \right)
= -\beta \ycap.
\end{dmath}

In components this is

\begin{equation}\label{eqn:modernOpticsProblemSet1:30}
\begin{aligned}
\dds{} \left( \left( n_0 - \beta y \right) \dds{x} \right) &= 0 \\
\dds{} \left( \left( n_0 - \beta y \right) \dds{y} \right) &= -\beta \\
\dds{} \left( \left( n_0 - \beta y \right) \dds{z} \right) &= 0
\end{aligned}
\end{equation}

Integrating once we have
\begin{equation}\label{eqn:modernOpticsProblemSet1:50}
\begin{aligned}
\left( n_0 - \beta y \right) \dds{x} &= A n_0 \\
\left( n_0 - \beta y \right) \dds{y} &= -\beta s + B n_0 \\
\left( n_0 - \beta y \right) \dds{z} &= C n_0
\end{aligned}
\end{equation}

In particular, at $s = 0$, where $x(0) = y(0) = z(0) = 0$, $x'(0) = 1$ and $y'(0) = z'(0) = 0$, we have

\begin{equation}\label{eqn:modernOpticsProblemSet1:70}
\begin{aligned}
n_0 (1) &= A n_0 \\
n_0 (0) &= B n_0 \\
n_0 (0) &= C n_0
\end{aligned}
\end{equation}

Our equations of motion become
\begin{equation}\label{eqn:modernOpticsProblemSet1:90}
\begin{aligned}
\left( n_0 - \beta y \right) \dds{x} &= n_0 \\
\left( n_0 - \beta y \right) \dds{y} &= -\beta s \\
\left( n_0 - \beta y \right) \dds{z} &= 0
\end{aligned}
\end{equation}

We have two non-trivial differential equations to solve.

\item[(b)]

First observe that unless $n_0 = \beta y(s)$ for all $s$, then $z(s)$ must be constant.  However, our boundary condition $\Br(0) = 0$ means that this constant is zero

\begin{equation}\label{eqn:modernOpticsProblemSet1:110}
z(s) = \text{constant} = z(0) = 0.
\end{equation}

Solving for $y(s)$ next we have after rearranging

\begin{dmath}\label{eqn:modernOpticsProblemSet1:130}
\int \left( n_0 - \beta y \right) dy = -\beta \int s ds
\end{dmath}

This yields

\begin{dmath}\label{eqn:modernOpticsProblemSet1:150}
n_0 y - \frac{\beta}{2} y^2 = -\frac{\beta}{2} s^2 + C.
\end{dmath}

Noting that $y(0) = 0$ we have $C = 0$

\begin{dmath}\label{eqn:modernOpticsProblemSet1:170}
y^2 - s^2 - 2 \frac{n_0}{\beta} y = 0.
\end{dmath}

Completing the square

\begin{dmath}\label{eqn:modernOpticsProblemSet1:230}
\left( y - \frac{n_0}{\beta} \right)^2 = s^2 + \left( \frac{n_0}{\beta} \right)^2 
\end{dmath}

or

\begin{dmath}\label{eqn:modernOpticsProblemSet1:190}
y = \frac{n_0}{\beta} \pm \sqrt{ s^2 + \left( \frac{n_0}{\beta} \right)^2 }.
\end{dmath}

Given the $y(0) = 0$ boundary constraint, we can only pick the negative root.  Borrowing the $L = n_0/\beta$ notation from later in the problem, we have

\begin{dmath}\label{eqn:modernOpticsProblemSet1:210}
\boxed{
%y(s) = \frac{n_0}{\beta} - \sqrt{ s^2 + \left( \frac{n_0}{\beta} \right)^2 }.
y(s) = L \left( 1 - \sqrt{ \left( \frac{s}{L}\right)^2 + 1 } \right) .
}
\end{dmath}

Let's look at the small limit where $s \ll L$  

\begin{dmath}\label{eqn:modernOpticsProblemSet1:250}
y(s) 
%= L \left( 1 - \sqrt{ 1 + \left(\frac{s}{L}\right)^2 } \right)
\sim L \left( 1 - \left( 1 + \inv{2} \left(\frac{s}{L}\right)^2 \right) \right)
= -\frac{s^2}{2 L}.
\end{dmath}

In the large limit for $s \gg L$ the $s^2$ term dominates, leaving

\begin{dmath}\label{eqn:modernOpticsProblemSet1:270}
y(s) \sim - s.
\end{dmath}

A plot of $y/L$, $-s/L$, and $-s^2/2 L^2$ can be found in figure (\ref{fig:modernOpticsProblemSet1:modernOpticsProblemSet1Fig2b}).

\imageFigure{modernOpticsProblemSet1Fig2b}{Plots of $y(s)$ and corresponding big and small limiting forms}{fig:modernOpticsProblemSet1:modernOpticsProblemSet1Fig2b}{0.2}

\item[(c)]
We are now set to solve our x component ray equation

\begin{dmath}\label{eqn:modernOpticsProblemSet1:290}
(L - y) \dds{x} = L,
\end{dmath}

or
\begin{dmath}\label{eqn:modernOpticsProblemSet1:310}
\sqrt{s^2 + L^2} \dds{x} = L.
\end{dmath}

Integrating we have

\begin{dmath}\label{eqn:modernOpticsProblemSet1:330}
x 
= L \int_0^s \frac{ds'}{\sqrt{{s'}^2 + L^2}}
= L \int_0^s \frac{ds'}{\sqrt{{s'}^2 + L^2}}
= L \int_0^{s/L} \frac{dt}{\sqrt{t^2 + 1}}
= L \evalrange{ \ln\left( t + \sqrt{ t^2 + 1} \right) }{0}{s/L}
\end{dmath}

This is

\begin{dmath}\label{eqn:modernOpticsProblemSet1:350}
\boxed{
x(s) = L \ln\left( \frac{s}{L} + \sqrt{ \left( \frac{s}{L} \right)^2 + 1} \right).
}
\end{dmath}

In the large limit for $s \gg L$ the $s^2$ term in the square root dominates, leaving

\begin{dmath}\label{eqn:modernOpticsProblemSet1:370}
x(s) 
\sim L \ln\left( \frac{2 s}{L} \right)
%= L\ln 2 + L \ln \frac{s}{L}
%\sim L \ln \frac{s}{L}
\end{dmath}

In the small limit $s \ll L$

\begin{dmath}\label{eqn:modernOpticsProblemSet1:390}
x(s) \sim L \ln\left( \frac{s}{L} + 1 \right)
= L \left( 
\frac{s}{L}
 -\inv{2} 
\left(
\frac{s}{L}
\right)^2
 +\inv{3} 
\left(
\frac{s}{L}
\right)^3
- \cdots
\right)
\sim 
s
\end{dmath}

With $t = s/L$, we have a plot of $u(t) = x(Lt)/L$, and the small and large limit approximations above in figure (\ref{fig:modernOpticsProblemSet1:modernOpticsProblemSet1Fig2d}).

\imageFigure{modernOpticsProblemSet1Fig2d}{Plots of $x(s)$ and corresponding big and small limiting forms}{fig:modernOpticsProblemSet1:modernOpticsProblemSet1Fig2d}{0.2}

\item[(d)]
With $t = s/L$, $u = x/L$, and $v = y/L$ we have

\begin{subequations}
\begin{dmath}\label{eqn:modernOpticsProblemSet1:410}
u = \ln\left( t + \sqrt{t^2 + 1} \right)
\end{dmath}
\begin{dmath}\label{eqn:modernOpticsProblemSet1:430}
v = 1 - \sqrt{t^2 + 1}
\end{dmath}
\end{subequations}

Rearranging for $t$ and $\sqrt{1 + t^2}$, we have

\begin{subequations}
\begin{dmath}\label{eqn:modernOpticsProblemSet1:450}
\sqrt{t^2 + 1} = 1 - v
\end{dmath}
\begin{dmath}\label{eqn:modernOpticsProblemSet1:470}
t = \sqrt{(1 - v)^2 - 1},
\end{dmath}
\end{subequations}

so

\begin{dmath}\label{eqn:modernOpticsProblemSet1:490}
u(v) = \ln\left( \sqrt{v^2 - 2 v} + 1 - v \right)
\end{dmath}

or
\begin{dmath}\label{eqn:modernOpticsProblemSet1:510}
\boxed{
x(y) = L \ln\left( \sqrt{\left(\frac{y}{L}\right)^2 - 2 \frac{y}{L}} + 1 - \frac{y}{L} \right).
}
\end{dmath}

FIXME: approximations.

\item[(e)]

In the small limit we found

\begin{subequations}
\begin{dmath}\label{eqn:modernOpticsProblemSet1:530}
x(s) \sim s
\end{dmath}
\begin{dmath}\label{eqn:modernOpticsProblemSet1:550}
y(s) \sim -\frac{s^2}{2 L},
\end{dmath}
\end{subequations}

so we have

\begin{dmath}\label{eqn:modernOpticsProblemSet1:570}
y \sim -\frac{x^2}{2 L},
\end{dmath}

a parabolic trajectory.  Comparing to $y'' = g$, where $y = g t^2/2 + y_0' t + y_0$, the quantity that's analoguous to the gravitational acceleration in \ref{eqn:modernOpticsProblemSet1:570} is

\begin{equation}\label{eqn:modernOpticsProblemSet1:590}
- \inv{L} = -\frac{\beta}{n_0} \rightarrow g.
\end{equation}

\item[(f)]
These plots were included above with the results they were associated with.
\end{enumerate}
}

\makeproblem{Ray equation at a surface}{modernOptics:problemSet1:3}{
Show that Snell's law can be derived from the {\em transverse} component of the ray equation applied at an index step. Set up the problem with an index step from $n_1$ in the half-plane $x<0$; and $n_2$ in the half-plane $x>0$ (figure \ref{fig:modernOpticsProblemSet1:FigureSnell}). Define your rays according to two straight-line trajectories: a ray in the $xy$ plane defined by $x=s \cos{\theta_1}$ and $y=s \sin{\theta_1}$ for $x<0$; and $x=s \cos{\theta_2}$ and $y=s \sin{\theta_2}$ for $x>0$. 
\begin{enumerate}
\item[(a)] Solve the { transverse} (or y-) component of the Ray Equation. Show that it gives Snell's law. 
\item[(b)] Show that the {\em normal} (or x-) component of the Ray Equation is contradictory, unless the limit of a small index step is taken. Why is this? What is missing?
\end{enumerate}

\imageFigure{FigureSnell}{}{fig:modernOpticsProblemSet1:FigureSnell}{0.2}
}
\makeanswer{modernOptics:problemSet1:3}{ }

%\vcsinfo
%\EndArticle
\EndNoBibArticle

