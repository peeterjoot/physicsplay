%
% Copyright � 2013 Peeter Joot.  All Rights Reserved.
% Licenced as described in the file LICENSE under the root directory of this GIT repository.
%
\newcommand{\authorname}{Peeter Joot}
\newcommand{\email}{peeterjoot@protonmail.com}
\newcommand{\basename}{FIXMEbasenameUndefined}
\newcommand{\dirname}{notes/FIXMEdirnameUndefined/}

\renewcommand{\basename}{twoMassHarmonic}
\renewcommand{\dirname}{notes/classicalmechanics/}
%\newcommand{\dateintitle}{}
\newcommand{\keywords}{Lagrangian, Harmonic oscillator, center of mass, two body problem}

\newcommand{\nought}[0]{\circ}

\newcommand{\authorname}{Peeter Joot}
\newcommand{\onlineurl}{http://sites.google.com/site/peeterjoot2/math2013/\basename.pdf}
\newcommand{\sourcepath}{\dirname\basename.tex}
\newcommand{\generatetitle}[1]{\chapter{#1}}

\newcommand{\vcsinfo}{%
\section*{}
\noindent{\color{DarkOliveGreen}{\rule{\linewidth}{0.1mm}}}
\paragraph{Document version}
%\paragraph{\color{Maroon}{Document version}}
{
\small
\begin{itemize}
\item Available online at:\\ 
\href{\onlineurl}{\onlineurl}
\item Git Repository: \input{./.revinfo/gitRepo.tex}
\item Source: \sourcepath
\item last commit: \input{./.revinfo/gitCommitString.tex}
\item commit date: \input{./.revinfo/gitCommitDate.tex}
\end{itemize}
}
}

%\PassOptionsToPackage{dvipsnames,svgnames}{xcolor}
\PassOptionsToPackage{square,numbers}{natbib}
\documentclass{scrreprt}

\usepackage[left=2cm,right=2cm]{geometry}
\usepackage[svgnames]{xcolor}
\usepackage{peeters_layout}

\usepackage{natbib}

\usepackage[
colorlinks=true,
bookmarks=false,
pdfauthor={\authorname, \email},
backref 
]{hyperref}

% http://tex.stackexchange.com/questions/75773/how-to-reference-problems-by-the-text-label-in-an-exercise-envioronment
\usepackage[english]{cleveref}
\crefname{Exercise}{exercise}{exercises}
\Crefname{Exercise}{Exercise}{Exercises}

\RequirePackage{titlesec}
\RequirePackage{ifthen}

% http://stackoverflow.com/questions/4932910/date-in-the-tabular-environment
\makeatletter
\let\insertdate\@date
\makeatother

\titleformat{\chapter}[display]
{\bfseries\Large}
{\color{DarkSlateGrey}\filleft \authorname
\ifthenelse{\isundefined{\studentnumber}}{}{\\ \studentnumber}
\ifthenelse{\isundefined{\email}}{}{\\ \email}
\ifthenelse{\isundefined{\dateintitle}}{}{\\ \insertdate}
%\ifthenelse{\isundefined{\coursename}}{}{\\ \coursename} % put in title instead.
}
{4ex}
{\color{DarkOliveGreen}{\titlerule}\color{Maroon}
\vspace{2ex}%
\filright}
[\vspace{2ex}%
\color{DarkOliveGreen}\titlerule
]

\newcommand{\beginArtWithToc}[0]{\begin{document}\tableofcontents}
\newcommand{\beginArtNoToc}[0]{\begin{document}}
\newcommand{\EndNoBibArticle}[0]{\end{document}}
\newcommand{\EndArticle}[0]{\bibliography{Bibliography}\bibliographystyle{plainnat}\end{document}}

% 
%\newcommand{\citep}[1]{\cite{#1}}

\colorSectionsForArticle



\beginArtNoToc

\generatetitle{Two body harmonic oscillator in 3D}
%\chapter{Two body harmonic oscillator in 3D}
%\label{chap:twoMassHarmonic}

\makeproblem{Two body harmonic oscillator in 3D}{pr:twoMassHarmonic:1}{

In lattice problems, we consider normal modes of harmonic coupled systems.  As review, before switching to displacement coordinates, solve the two body problem exactly using absolute coordinates.
} % makeproblem

\makeanswer{pr:twoMassHarmonic:1}{ 

For the system illustrated in our Lagrangian is

FIXME: figure

\begin{dmath}\label{eqn:twoMassHarmonic:20}
\LL = 
\inv{2} m_1 \lr{ \dot{\Br}_1 }^2
+\inv{2} m_2 \lr{ \dot{\Br}_2 }^2
- \frac{K}{2} \lr{ \Br_1 - \Br_2 }^2.
\end{dmath}

We wish to solve the equations of motion

\begin{dmath}\label{eqn:twoMassHarmonic:40}
\ddt{} \spacegrad_{\dot{\Br}_i} \LL = 
\spacegrad_{\Br_i} \LL.
\end{dmath}

Noting that $\spacegrad_\Bx \Ba \cdot \Bx = \Ba$, the coupled system to solve is found to be

\begin{dmath}\label{eqn:twoMassHarmonic:60}
\begin{aligned}
m_1 \ddot{\Br}_1 &= - K \lr{ \Br_1 - \Br_2 } \\
m_2 \ddot{\Br}_2 &= - K \lr{ \Br_2 - \Br_1 }.
\end{aligned}
\end{dmath}

These can be decoupled using differences and sums

\begin{dmath}\label{eqn:twoMassHarmonic:80}
\begin{aligned}
m_2 \lr{ m_1 \ddot{\Br}_1 } - m_1 \lr{ m_2 \ddot{\Br}_2 } &= - (m_1 + m_2) K \lr{ \Br_1 - \Br_2 } \\
m_1 \ddot{\Br}_1 + m_2 \ddot{\Br}_2 &= 0
\end{aligned}
\end{dmath}

The second is the equation for the acceleration of the center of mass $\BR_{\mathrm{CM}}(t)$, and it directly integrable.  With $M = m_1 + m_2$, that is

\begin{dmath}\label{eqn:twoMassHarmonic:100}
M \BR_{\mathrm{CM}}(t) 
= 
m_1 \Br_1 + m_2 \Br_2 = 
 (t - t_\nought) 
M \BV_{\mathrm{CM}}
+ M \BR_{\mathrm{CM}}(t_\nought).
\end{dmath}

The first is the harmonic oscillation about the center of mass position.  Introducing the reduced mass

\begin{dmath}\label{eqn:twoMassHarmonic:120}
\mu = \frac{m_1 m_2}{m_1 + m_2},
\end{dmath}

that oscillation equation is

\begin{dmath}\label{eqn:twoMassHarmonic:140}
\frac{d^2}{dt^2}
\lr{ \Br_1 - \Br_2 }
= -\frac{K}{\mu} 
\lr{ \Br_1 - \Br_2 }.
\end{dmath}

With 

\begin{dmath}\label{eqn:twoMassHarmonic:160}
\omega^2 = \frac{K}{\mu},
\end{dmath}

this difference has the solution

\begin{dmath}\label{eqn:twoMassHarmonic:180}
\Br_1(t) - \Br_2(t) = \lr{ \Br_1(t_\nought) - \Br_2(t_\nought) } \cos\lr{ \omega(t - t_\nought) }.
\end{dmath}

This form of solution allows for arbitrary phase shift, including both sine and cosine solutions.  For example, for cosine solutions, pick $t_\nought$ as the time for which the amplitude difference is maximized.  

Writing $\Delta \Br_\nought = \lr{ \Br_1(t_\nought) - \Br_2(t_\nought) }$, the matrix inversion problem for 
individual $\Br_i$ vectors is

\begin{dmath}\label{eqn:twoMassHarmonic:200}
\begin{bmatrix}
1 & -1 \\
m_1 & m_2 
\end{bmatrix}
\begin{bmatrix}
\Br_1 \\
\Br_2
\end{bmatrix}
=
\begin{bmatrix}
\Delta \Br_\nought
\cos\lr{ \omega(t - t_\nought) } \\
M \BR_{\mathrm{CM}}(t) 
\end{bmatrix},
\end{dmath}

or

\begin{dmath}\label{eqn:twoMassHarmonic:220}
\begin{bmatrix}
\Br_1 \\
\Br_2
\end{bmatrix}
=
\inv{m_2 + m_1}
\begin{bmatrix}
m_2 & 1 \\
-m_1 & 1 
\end{bmatrix}
\begin{bmatrix}
\Delta \Br_\nought
\cos\lr{ \omega(t - t_\nought) } 
\\
M \BR_{\mathrm{CM}}(t) 
\end{bmatrix}
\end{dmath}

The final solution is

\begin{dmath}\label{eqn:twoMassHarmonic:240}
\myBoxed{
\begin{aligned}
\Br_1(t) &=
\frac{\mu}{m_1}
\Delta \Br_\nought
\cos\lr{ \omega(t - t_\nought) } 
+ \BR_{\mathrm{CM}}(t) \\
\Br_2(t) &=
-
\frac{\mu}{m_2}
\Delta \Br_\nought
\cos\lr{ \omega(t - t_\nought) } 
+ \BR_{\mathrm{CM}}(t) 
\end{aligned}
}
\end{dmath}

FIXME: something not right here!  Seems to allow for masses to pass through each other, through the center of mass position.  Think that I need a different general solution to the difference equation, one that allows for both initial position and velocities.
} % makeanswer

%\EndArticle
\EndNoBibArticle
