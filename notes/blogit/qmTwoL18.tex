%
% Copyright � 2015 Peeter Joot.  All Rights Reserved.
% Licenced as described in the file LICENSE under the root directory of this GIT repository.
%
\documentclass[]{eliblog}

\usepackage{amsmath}
\usepackage{mathpazo}

%
% shorthand for bold symbols, convenient for vectors and matrices
%
\newcommand{\Ba}[0]{\mathbf{a}}
\newcommand{\Bb}[0]{\mathbf{b}}
\newcommand{\Bc}[0]{\mathbf{c}}
\newcommand{\Bd}[0]{\mathbf{d}}
\newcommand{\Be}[0]{\mathbf{e}}
\newcommand{\Bf}[0]{\mathbf{f}}
\newcommand{\Bg}[0]{\mathbf{g}}
\newcommand{\Bh}[0]{\mathbf{h}}
\newcommand{\Bi}[0]{\mathbf{i}}
\newcommand{\Bj}[0]{\mathbf{j}}
\newcommand{\Bk}[0]{\mathbf{k}}
\newcommand{\Bl}[0]{\mathbf{l}}
\newcommand{\Bm}[0]{\mathbf{m}}
\newcommand{\Bn}[0]{\mathbf{n}}
\newcommand{\Bo}[0]{\mathbf{o}}
\newcommand{\Bp}[0]{\mathbf{p}}
\newcommand{\Bq}[0]{\mathbf{q}}
\newcommand{\Br}[0]{\mathbf{r}}
\newcommand{\Bs}[0]{\mathbf{s}}
\newcommand{\Bt}[0]{\mathbf{t}}
\newcommand{\Bu}[0]{\mathbf{u}}
\newcommand{\Bv}[0]{\mathbf{v}}
\newcommand{\Bw}[0]{\mathbf{w}}
\newcommand{\Bx}[0]{\mathbf{x}}
\newcommand{\By}[0]{\mathbf{y}}
\newcommand{\Bz}[0]{\mathbf{z}}
\newcommand{\BA}[0]{\mathbf{A}}
\newcommand{\BB}[0]{\mathbf{B}}
\newcommand{\BC}[0]{\mathbf{C}}
\newcommand{\BD}[0]{\mathbf{D}}
\newcommand{\BE}[0]{\mathbf{E}}
\newcommand{\BF}[0]{\mathbf{F}}
\newcommand{\BG}[0]{\mathbf{G}}
\newcommand{\BH}[0]{\mathbf{H}}
\newcommand{\BI}[0]{\mathbf{I}}
\newcommand{\BJ}[0]{\mathbf{J}}
\newcommand{\BK}[0]{\mathbf{K}}
\newcommand{\BL}[0]{\mathbf{L}}
\newcommand{\BM}[0]{\mathbf{M}}
\newcommand{\BN}[0]{\mathbf{N}}
\newcommand{\BO}[0]{\mathbf{O}}
\newcommand{\BP}[0]{\mathbf{P}}
\newcommand{\BQ}[0]{\mathbf{Q}}
\newcommand{\BR}[0]{\mathbf{R}}
\newcommand{\BS}[0]{\mathbf{S}}
\newcommand{\BT}[0]{\mathbf{T}}
\newcommand{\BU}[0]{\mathbf{U}}
\newcommand{\BV}[0]{\mathbf{V}}
\newcommand{\BW}[0]{\mathbf{W}}
\newcommand{\BX}[0]{\mathbf{X}}
\newcommand{\BY}[0]{\mathbf{Y}}
\newcommand{\BZ}[0]{\mathbf{Z}}

\newcommand{\Bzero}[0]{\mathbf{0}}
\newcommand{\Btheta}[0]{\boldsymbol{\theta}}
\newcommand{\Btau}[0]{\boldsymbol{\tau}}
\newcommand{\Bomega}[0]{\boldsymbol{\omega}}

%
% shorthand for unit vectors
%
\newcommand{\acap}[0]{\hat{\Ba}}
\newcommand{\bcap}[0]{\hat{\Bb}}
\newcommand{\ccap}[0]{\hat{\Bc}}
\newcommand{\dcap}[0]{\hat{\Bd}}
\newcommand{\ecap}[0]{\hat{\Be}}
\newcommand{\fcap}[0]{\hat{\Bf}}
\newcommand{\gcap}[0]{\hat{\Bg}}
\newcommand{\hcap}[0]{\hat{\Bh}}
\newcommand{\icap}[0]{\hat{\Bi}}
\newcommand{\jcap}[0]{\hat{\Bj}}
\newcommand{\kcap}[0]{\hat{\Bk}}
\newcommand{\lcap}[0]{\hat{\Bl}}
\newcommand{\mcap}[0]{\hat{\Bm}}
\newcommand{\ncap}[0]{\hat{\Bn}}
\newcommand{\ocap}[0]{\hat{\Bo}}
\newcommand{\pcap}[0]{\hat{\Bp}}
\newcommand{\qcap}[0]{\hat{\Bq}}
\newcommand{\rcap}[0]{\hat{\Br}}
\newcommand{\scap}[0]{\hat{\Bs}}
\newcommand{\tcap}[0]{\hat{\Bt}}
\newcommand{\ucap}[0]{\hat{\Bu}}
\newcommand{\vcap}[0]{\hat{\Bv}}
\newcommand{\wcap}[0]{\hat{\Bw}}
\newcommand{\xcap}[0]{\hat{\Bx}}
\newcommand{\ycap}[0]{\hat{\By}}
\newcommand{\zcap}[0]{\hat{\Bz}}
\newcommand{\thetacap}[0]{\hat{\Btheta}}

%
% to write R^n and C^n in a distinguishable fashion.  Perhaps change this
% to the double lined characters upon figuring out how to do so.
%
\newcommand{\C}[1]{$\mathbb{C}^{#1}$}
\newcommand{\R}[1]{$\mathbb{R}^{#1}$}

%
% various generally useful helpers
%

% derivative of #1 wrt. #2:
\newcommand{\D}[2] {\frac {d#2} {d#1}}

\newcommand{\inv}[1]{\frac{1}{#1}}
\newcommand{\cross}[0]{\times}

\newcommand{\abs}[1]{\lvert{#1}\rvert}
\newcommand{\norm}[1]{\lVert{#1}\rVert}
\newcommand{\innerprod}[2]{\langle{#1}, {#2}\rangle}
\newcommand{\dotprod}[2]{{#1} \cdot {#2}}
\newcommand{\bdotprod}[2]{\left({#1} \cdot {#2}\right)}
\newcommand{\crossprod}[2]{{#1} \cross {#2}}
\newcommand{\tripleprod}[3]{\dotprod{\left(\crossprod{#1}{#2}\right)}{#3}}

\DeclareMathOperator{\Proj}{Proj}
\DeclareMathOperator{\Span}{span}
\DeclareMathOperator{\Sgn}{sgn}
\DeclareMathOperator{\Area}{Area}
\DeclareMathOperator{\Volume}{Volume}

%
% A few miscellaneous things specific to this document
%
\newcommand{\crossop}[1]{\crossprod{#1}{}}

% R2 vector.
\newcommand{\VectorTwo}[2]{
\begin{bmatrix}
 {#1} \\
 {#2}
\end{bmatrix}
}

\newcommand{\VectorN}[1]{
\begin{bmatrix}
{#1}_1 \\
{#1}_2 \\
\vdots \\
{#1}_N \\
\end{bmatrix}
}

\newcommand{\DETuvij}[4]{
\begin{vmatrix}
 {#1}_{#3} & {#1}_{#4} \\
 {#2}_{#3} & {#2}_{#4}
\end{vmatrix}
}

\newcommand{\DETuvwijk}[6]{
\begin{vmatrix}
 {#1}_{#4} & {#1}_{#5} & {#1}_{#6} \\
 {#2}_{#4} & {#2}_{#5} & {#2}_{#6} \\
 {#3}_{#4} & {#3}_{#5} & {#3}_{#6}
\end{vmatrix}
}

\newcommand{\DETuvwxijkl}[8]{
\begin{vmatrix}
 {#1}_{#5} & {#1}_{#6} & {#1}_{#7} & {#1}_{#8} \\
 {#2}_{#5} & {#2}_{#6} & {#2}_{#7} & {#2}_{#8} \\
 {#3}_{#5} & {#3}_{#6} & {#3}_{#7} & {#3}_{#8} \\
 {#4}_{#5} & {#4}_{#6} & {#4}_{#7} & {#4}_{#8} \\
\end{vmatrix}
}

%\newcommand{\DETuvwxyijklm}[10]{
%\begin{vmatrix}
% {#1}_{#6} & {#1}_{#7} & {#1}_{#8} & {#1}_{#9} & {#1}_{#10} \\
% {#2}_{#6} & {#2}_{#7} & {#2}_{#8} & {#2}_{#9} & {#2}_{#10} \\
% {#3}_{#6} & {#3}_{#7} & {#3}_{#8} & {#3}_{#9} & {#3}_{#10} \\
% {#4}_{#6} & {#4}_{#7} & {#4}_{#8} & {#4}_{#9} & {#4}_{#10} \\
% {#5}_{#6} & {#5}_{#7} & {#5}_{#8} & {#5}_{#9} & {#5}_{#10}
%\end{vmatrix}
%}

% R3 vector.
\newcommand{\VectorThree}[3]{
\begin{bmatrix}
 {#1} \\
 {#2} \\
 {#3}
\end{bmatrix}
}



\author{Peeter Joot}
\email{peeter.joot@gmail.com}

%\documentclass[]{eliblogwidescreen}

\usepackage{amsmath}
\usepackage{mathpazo}

%
% shorthand for bold symbols, convenient for vectors and matrices
%
\newcommand{\Ba}[0]{\mathbf{a}}
\newcommand{\Bb}[0]{\mathbf{b}}
\newcommand{\Bc}[0]{\mathbf{c}}
\newcommand{\Bd}[0]{\mathbf{d}}
\newcommand{\Be}[0]{\mathbf{e}}
\newcommand{\Bf}[0]{\mathbf{f}}
\newcommand{\Bg}[0]{\mathbf{g}}
\newcommand{\Bh}[0]{\mathbf{h}}
\newcommand{\Bi}[0]{\mathbf{i}}
\newcommand{\Bj}[0]{\mathbf{j}}
\newcommand{\Bk}[0]{\mathbf{k}}
\newcommand{\Bl}[0]{\mathbf{l}}
\newcommand{\Bm}[0]{\mathbf{m}}
\newcommand{\Bn}[0]{\mathbf{n}}
\newcommand{\Bo}[0]{\mathbf{o}}
\newcommand{\Bp}[0]{\mathbf{p}}
\newcommand{\Bq}[0]{\mathbf{q}}
\newcommand{\Br}[0]{\mathbf{r}}
\newcommand{\Bs}[0]{\mathbf{s}}
\newcommand{\Bt}[0]{\mathbf{t}}
\newcommand{\Bu}[0]{\mathbf{u}}
\newcommand{\Bv}[0]{\mathbf{v}}
\newcommand{\Bw}[0]{\mathbf{w}}
\newcommand{\Bx}[0]{\mathbf{x}}
\newcommand{\By}[0]{\mathbf{y}}
\newcommand{\Bz}[0]{\mathbf{z}}
\newcommand{\BA}[0]{\mathbf{A}}
\newcommand{\BB}[0]{\mathbf{B}}
\newcommand{\BC}[0]{\mathbf{C}}
\newcommand{\BD}[0]{\mathbf{D}}
\newcommand{\BE}[0]{\mathbf{E}}
\newcommand{\BF}[0]{\mathbf{F}}
\newcommand{\BG}[0]{\mathbf{G}}
\newcommand{\BH}[0]{\mathbf{H}}
\newcommand{\BI}[0]{\mathbf{I}}
\newcommand{\BJ}[0]{\mathbf{J}}
\newcommand{\BK}[0]{\mathbf{K}}
\newcommand{\BL}[0]{\mathbf{L}}
\newcommand{\BM}[0]{\mathbf{M}}
\newcommand{\BN}[0]{\mathbf{N}}
\newcommand{\BO}[0]{\mathbf{O}}
\newcommand{\BP}[0]{\mathbf{P}}
\newcommand{\BQ}[0]{\mathbf{Q}}
\newcommand{\BR}[0]{\mathbf{R}}
\newcommand{\BS}[0]{\mathbf{S}}
\newcommand{\BT}[0]{\mathbf{T}}
\newcommand{\BU}[0]{\mathbf{U}}
\newcommand{\BV}[0]{\mathbf{V}}
\newcommand{\BW}[0]{\mathbf{W}}
\newcommand{\BX}[0]{\mathbf{X}}
\newcommand{\BY}[0]{\mathbf{Y}}
\newcommand{\BZ}[0]{\mathbf{Z}}

\newcommand{\Bzero}[0]{\mathbf{0}}
\newcommand{\Btheta}[0]{\boldsymbol{\theta}}
\newcommand{\Btau}[0]{\boldsymbol{\tau}}
\newcommand{\Bomega}[0]{\boldsymbol{\omega}}

%
% shorthand for unit vectors
%
\newcommand{\acap}[0]{\hat{\Ba}}
\newcommand{\bcap}[0]{\hat{\Bb}}
\newcommand{\ccap}[0]{\hat{\Bc}}
\newcommand{\dcap}[0]{\hat{\Bd}}
\newcommand{\ecap}[0]{\hat{\Be}}
\newcommand{\fcap}[0]{\hat{\Bf}}
\newcommand{\gcap}[0]{\hat{\Bg}}
\newcommand{\hcap}[0]{\hat{\Bh}}
\newcommand{\icap}[0]{\hat{\Bi}}
\newcommand{\jcap}[0]{\hat{\Bj}}
\newcommand{\kcap}[0]{\hat{\Bk}}
\newcommand{\lcap}[0]{\hat{\Bl}}
\newcommand{\mcap}[0]{\hat{\Bm}}
\newcommand{\ncap}[0]{\hat{\Bn}}
\newcommand{\ocap}[0]{\hat{\Bo}}
\newcommand{\pcap}[0]{\hat{\Bp}}
\newcommand{\qcap}[0]{\hat{\Bq}}
\newcommand{\rcap}[0]{\hat{\Br}}
\newcommand{\scap}[0]{\hat{\Bs}}
\newcommand{\tcap}[0]{\hat{\Bt}}
\newcommand{\ucap}[0]{\hat{\Bu}}
\newcommand{\vcap}[0]{\hat{\Bv}}
\newcommand{\wcap}[0]{\hat{\Bw}}
\newcommand{\xcap}[0]{\hat{\Bx}}
\newcommand{\ycap}[0]{\hat{\By}}
\newcommand{\zcap}[0]{\hat{\Bz}}
\newcommand{\thetacap}[0]{\hat{\Btheta}}

%
% to write R^n and C^n in a distinguishable fashion.  Perhaps change this
% to the double lined characters upon figuring out how to do so.
%
\newcommand{\C}[1]{$\mathbb{C}^{#1}$}
\newcommand{\R}[1]{$\mathbb{R}^{#1}$}

%
% various generally useful helpers
%

% derivative of #1 wrt. #2:
\newcommand{\D}[2] {\frac {d#2} {d#1}}

\newcommand{\inv}[1]{\frac{1}{#1}}
\newcommand{\cross}[0]{\times}

\newcommand{\abs}[1]{\lvert{#1}\rvert}
\newcommand{\norm}[1]{\lVert{#1}\rVert}
\newcommand{\innerprod}[2]{\langle{#1}, {#2}\rangle}
\newcommand{\dotprod}[2]{{#1} \cdot {#2}}
\newcommand{\bdotprod}[2]{\left({#1} \cdot {#2}\right)}
\newcommand{\crossprod}[2]{{#1} \cross {#2}}
\newcommand{\tripleprod}[3]{\dotprod{\left(\crossprod{#1}{#2}\right)}{#3}}

\DeclareMathOperator{\Proj}{Proj}
\DeclareMathOperator{\Span}{span}
\DeclareMathOperator{\Sgn}{sgn}
\DeclareMathOperator{\Area}{Area}
\DeclareMathOperator{\Volume}{Volume}

%
% A few miscellaneous things specific to this document
%
\newcommand{\crossop}[1]{\crossprod{#1}{}}

% R2 vector.
\newcommand{\VectorTwo}[2]{
\begin{bmatrix}
 {#1} \\
 {#2}
\end{bmatrix}
}

\newcommand{\VectorN}[1]{
\begin{bmatrix}
{#1}_1 \\
{#1}_2 \\
\vdots \\
{#1}_N \\
\end{bmatrix}
}

\newcommand{\DETuvij}[4]{
\begin{vmatrix}
 {#1}_{#3} & {#1}_{#4} \\
 {#2}_{#3} & {#2}_{#4}
\end{vmatrix}
}

\newcommand{\DETuvwijk}[6]{
\begin{vmatrix}
 {#1}_{#4} & {#1}_{#5} & {#1}_{#6} \\
 {#2}_{#4} & {#2}_{#5} & {#2}_{#6} \\
 {#3}_{#4} & {#3}_{#5} & {#3}_{#6}
\end{vmatrix}
}

\newcommand{\DETuvwxijkl}[8]{
\begin{vmatrix}
 {#1}_{#5} & {#1}_{#6} & {#1}_{#7} & {#1}_{#8} \\
 {#2}_{#5} & {#2}_{#6} & {#2}_{#7} & {#2}_{#8} \\
 {#3}_{#5} & {#3}_{#6} & {#3}_{#7} & {#3}_{#8} \\
 {#4}_{#5} & {#4}_{#6} & {#4}_{#7} & {#4}_{#8} \\
\end{vmatrix}
}

%\newcommand{\DETuvwxyijklm}[10]{
%\begin{vmatrix}
% {#1}_{#6} & {#1}_{#7} & {#1}_{#8} & {#1}_{#9} & {#1}_{#10} \\
% {#2}_{#6} & {#2}_{#7} & {#2}_{#8} & {#2}_{#9} & {#2}_{#10} \\
% {#3}_{#6} & {#3}_{#7} & {#3}_{#8} & {#3}_{#9} & {#3}_{#10} \\
% {#4}_{#6} & {#4}_{#7} & {#4}_{#8} & {#4}_{#9} & {#4}_{#10} \\
% {#5}_{#6} & {#5}_{#7} & {#5}_{#8} & {#5}_{#9} & {#5}_{#10}
%\end{vmatrix}
%}

% R3 vector.
\newcommand{\VectorThree}[3]{
\begin{bmatrix}
 {#1} \\
 {#2} \\
 {#3}
\end{bmatrix}
}



\author{Peeter Joot}
\email{peeter.joot@gmail.com}


%(Taught by Mr. Federico Duque Gomez).  XXX}
\chapter{PHY456H1F: Quantum Mechanics II.  Lecture 18 (Taught by Prof J.E. Sipe).  XXX}
\label{chap:qmTwoL18}
%\useCCL
\blogpage{http://sites.google.com/site/peeterjoot/math2011/qmTwoL18.pdf}
\date{Nov 12, 2011}
\revisionInfo{qmTwoL18.tex}

\beginArtWithToc
%\beginArtNoToc

\section{Disclaimer.}

Peeter's lecture notes from class.  May not be entirely coherent.

\section{Recap.}

Recall our table

\begin{equation}\label{eqn:qmTwoL18:n}
\begin{array}{| l | l | l | l | l |}
\hline
j = & j_1 + j_2				& j_1 + j_2 -1 				& \cdots 	& j_1 - j_2 \\
\hline
\hline
  &  \ket{j_1 + j_2, j_1 + j_2}	 	&					& 		& \\
\hline
  &  \ket{j_1 + j_2, j_1 + j_2 - 1}	&  \ket{j_1 + j_2 - 1, j_1 + j_2 - 1}	& 		& \\
\hline
  &                                     & \ket{j_1 + j_2 - 1, j_1 + j_2 - 2}	& 		& \\
\hline
  & \vdots 	 			&					& 		& \ket{j_1 - j_2, j_1 - j_2} \\
\hline
  & \vdots 	 			&					& 		& \vdots \\
\hline
  & \vdots 	 			&					& 		& \ket{j_1 - j_2, -(j_1 - j_2)} \\
\hline
  & \vdots 	 			&					& 		& \\
\hline
  &  \ket{j_1 + j_2, -(j_1 + j_2 - 1)}	& \ket{j_1 + j_2 -1, -(j_1 + j_2 - 1)}	& 		& \\
\hline
  &  \ket{j_1 + j_2, -(j_1 + j_2)}	&					& 		&  \\
\hline
\end{array}
\end{equation}

\begin{equation}\label{eqn:qmTwoL18:n}
\ket{j_1 + j_2, j_1 + j_2} = \ket{j_1 j_1} \otimes \ket{j_2 j_2}
\end{equation}

Applying the lowering operator get

\begin{align*}
\ket{j_1 + j_2, j_1 + j_2 -1} 
&= 
\frac{J_{-} \ket{j_1 j_1} \otimes \ket{j_2 j_2}}{
\left(2 (j_1+ j_2)\right)^{1/2} \hbar
} \\
&=
\frac{(J_{1-} + J_{2-}) \ket{j_1 j_1} \otimes \ket{j_2 j_2}}{
\left(2 (j_1+ j_2)\right)^{1/2} \hbar
} \\
&=
\left(\frac{j_1}{j_1 + j_2}\right)^{1/2}
\ket{j_1 (j_1-1)} \otimes \ket{j_2 j_2}
+
\left(\frac{j_2}{j_1 + j_2}\right)^{1/2}
\ket{j_1 j_1} \otimes \ket{j_2 (j_2-1)}
 \\
\end{align*}

and finish off the column.

Move on to the second column

\begin{equation}\label{eqn:qmTwoL18:n}
\ket{j_1 + j_2 - 1, j_1 + j_2 -1} 
\end{equation}

can only be made up of $\ket{j_1 m_1} \otimes \ket{j_2 m_2}$ with $m_1 + m_2 = \
j1 + j_2 -1$.  There are two possibilities

FIXME: Table
%m_1 = j_1	m_2 = j_2 - 1
%m_1 = j_1 - 1   m_2 = j_2

So

\begin{equation}\label{eqn:qmTwoL18:n}
\ket{j_1 + j_2 - 1, j_1 + j_2 -1} 
=
A
\ket{j_1 j_1} \otimes \ket{j_2 (j_2-1)}
+
B
\ket{j_1 (j_1-1)} \otimes \ket{j_2 j_2}
\end{equation}

Since $\ket{j_1 + j_2, j_1 + j_2 -1}$ and $\ket{j_1 + j_2 - 1, j_1 + j_2 -1}$ are orthogonal.  So we can, for example, take

\begin{equation}\label{eqn:qmTwoL18:n}
\ket{j_1 + j_2 - 1, j_1 + j_2 -1} 
=
\left(\frac{j_1}{j_1 + j_2}\right)^{1/2}
\ket{j_1 j_1} \otimes \ket{j_2 (j_2-1)}
-
\left(\frac{j_2}{j_1 + j_2}\right)^{1/2}
\ket{j_1 (j_1-1)} \otimes \ket{j_2 j_2}
\end{equation}

This will work, but we can multiply this by any phase factor.  This is the convention we will use, where we

\begin{itemize}
\item choose the coefficients to be real.
\item require the coefficient of the $m_1 = j_1$ term to be $\ge 0$
\end{itemize}

This gives us the first state in the second column, and we can proceed to iterate using the lowering operators to get all those values.

Moving on to the third column

\begin{equation}\label{eqn:qmTwoL18:n}
\ket{j_1 + j_2 - 2, j_1 + j_2 -2} 
\end{equation}

can only be made up of $\ket{j_1 m_1} \otimes \ket{j_2 m_2}$ with $m_1 + m_2 = \
j1 + j_2 -2$.  There are now three possibilities

FIXME: Table
%m_1 = j_1	m_2 = j_2 - 2
%m_1 = j_1 - 2   m_2 = j_2
%m_1 = j_1 - 1   m_2 = j_2 - 1

and 2 orthogonality conditions, plus conventions.  This is enough to determine the ket in the third column.

We can formally write

\begin{equation}\label{eqn:qmTwoL18:n}
\ket{jm ; j_1 j_2} = 
\sum_{m_1, m_2}
\ket{ j_1 m_1, j_2 m_2}
\braket{ j_1 m_1, j_2 m_2}{jm ; j_1 j_2} 
\end{equation}

where
\begin{equation}\label{eqn:qmTwoL18:n}
\ket{ j_1 m_1, j_2 m_2} = \ket{j_1 m_1} \otimes \ket{j_2 m_2},
\end{equation}
and

\begin{equation}\label{eqn:qmTwoL18:n}
\braket{ j_1 m_1, j_2 m_2}{jm ; j_1 j_2} 
\end{equation}
are the Clebsch-Gordon coefficients, sometimes written as 

\begin{equation}\label{eqn:qmTwoL18:n}
\braket{ j_1 m_1, j_2 m_2 }{ jm }
\end{equation}

Properties
\begin{enumerate}
\item $\braket{ j_1 m_1, j_2 m_2 }{ jm } \ne 0$ only if $j_1 - j_2 \le j \le j_1 + j+2$

This is sometimes called the triangle inequality.

FIXME: F1.

\item $\braket{ j_1 m_1, j_2 m_2 }{ jm } \ne 0$ only if $m = m_1 + m_2$.

\item Real (convention).

\item $\braket{ j_1 j_1, j_2 (j - j_1) }{ j j }$ positive (convention again).

\item Proved in the text.  If follows that 

\begin{equation}\label{eqn:qmTwoL18:n}
\braket{ j_1 m_1, j_2 m_2 }{ j m }
=
(-1)^{j_1 + j_2 - j}
\braket{ j_1 (-m_1), j_2 (-m_2) }{ j (-m) }
\end{equation}
\end{enumerate}

Note that the $\braket{ j_1 m_1, j_2 m_2 }{ j m }$ are all real.  So, they can be assembled into an orthogonal matrix.  Example

\begin{equation}\label{eqn:qmTwoL18:n}
\begin{bmatrix}
\ket{11}
\ket{10}
\ket{\overline{11}}
\ket{00}
\end{bmatrix}
=
\begin{bmatrix}
1 & 0 & 0 & 0 \\
0 & \inv{\sqrt{2}} & \inv{\sqrt{2}} & 0 \\
0 & 0 & 0 & 1 \\
0 & \inv{\sqrt{2}} & \frac{-1}{\sqrt{2}} & 0 \\
\end{bmatrix}
\begin{bmatrix}
\ket{++}
\ket{++}
\ket{-+}
\ket{--}
\end{bmatrix}
\end{equation}

A special case for electrons, a spin $1/2$ particle with $s = 1/2$ and $m_s = \pm 1/2$ where we have 

\begin{equation}\label{eqn:qmTwoL18:n}
\BJ = \BL + \BS
\end{equation}

\begin{equation}\label{eqn:qmTwoL18:n}
\ket{lm} \otimes \ket{\inv{2} m_s}
\end{equation}

possible values of $j$ are $l \pm 1/2$

\begin{equation}\label{eqn:qmTwoL18:n}
l \otimes \inv{2} = 
\left(
l + \inv{2}
\right)
\oplus
\left(
l - \inv{2}
\right)
\end{equation}

Our table representation is then

\begin{equation}\label{eqn:qmTwoL18:n}
\begin{array}{| l | l | l |}
\hline
j = & l + \inv{2} 			& l - \inv{2} \\
\hline
\hline
  &  \ket{l + \inv{2}, l + \inv{2}}	 	&					 \\
\hline
  &  \ket{l + \inv{2}, l + \inv{2} - 1}	&  \ket{l - \inv{2}, l - \inv{2}}	 \\
\hline
  &                                     & \ket{l - \inv{2}, -(l - \inv{2}}	 \\
\hline
  &  \ket{l + \inv{2}, -(l + \inv{2})}	&					 \\
\hline
\end{array}
\end{equation}

Here $\ket{l + \inv{2}, m}$

can \underline{only} have contributions from 

\begin{align}\label{eqn:qmTwoL18:n}
\ket{l, m-\inv{2}} &\otimes \ket{\inv{2}\inv{2}} \\
\ket{l, m+\inv{2}} &\otimes \ket{\inv{2}\overline{\inv{2}}}
\end{align}

$\ket{l - \inv{2}, m}$ from the same two.  So using this and conventions we can work out (in \S 28 page 524, of our text \cite{desai2009quantum}).

\begin{equation}\label{eqn:qmTwoL18:n}
\ket{l\pm \inv{2}, m} =
\inv{\sqrt{2 l + 1}}
\left(
\pm (l + \inv{2} \pm m)^{1/2}
\ket{l, m - \inv{2}} \times \ket{\inv{2}\inv{2}}
+
\pm (l + \inv{2} \mp m)^{1/2}
\ket{l, m + \inv{2}} \times \ket{\inv{2} \overline{\inv{2}}}
\right)
\end{equation}

\section{Tensor operators}

\S 29 of the text.

Recall how we characterized a rotation

\begin{equation}\label{eqn:qmTwoL18:n}
\Br \rightarrow \mathcal{R}(\Br)
\end{equation}

FIXME: F2: active rotation.

Suppose that 

\begin{equation}\label{eqn:qmTwoL18:n}
{\begin{bmatrix}
\mathcal{R}(\Br)
\end{bmatrix}
}_i
= 
\sum_j M_{ij} r_j
\end{equation}

so that
\begin{equation}\label{eqn:qmTwoL18:n}
U = e^{-i \theta \ncap \cdot \BJ/\hbar}
\end{equation}

rotates in the same way.

FIXME: F3.

Rotating a ket

\begin{equation}\label{eqn:qmTwoL18:n}
\ket{\psi}
\end{equation}

using the prescription

\begin{equation}\label{eqn:qmTwoL18:n}
\ket{\psi'} = e^{-i \theta \ncap \cdot \BJ/\hbar} \ket{\psi}
\end{equation}

and write

\begin{equation}\label{eqn:qmTwoL18:n}
\ket{\psi'} = U[M] \ket{\psi}
\end{equation}

Now look at 

\begin{equation}\label{eqn:qmTwoL18:n}
\bra{\psi} \mathcal{O} \ket{\psi}
\end{equation}

and compare with

\begin{equation}\label{eqn:qmTwoL18:n}
\bra{\psi'} \mathcal{O} \ket{\psi'}
=
\bra{\psi} \underbrace{U^\dagger[M] \mathcal{O} U[M]}{\conj} \ket{\psi}
\end{equation}

We'll be looking in more detail at $\conj$.

\EndArticle
%\EndNoBibArticle
