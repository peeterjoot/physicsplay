%
% Copyright � 2013 Peeter Joot.  All Rights Reserved.
% Licenced as described in the file LICENSE under the root directory of this GIT repository.
%
\newcommand{\authorname}{Peeter Joot}
\newcommand{\email}{peeterjoot@protonmail.com}
\newcommand{\basename}{FIXMEbasenameUndefined}
\newcommand{\dirname}{notes/FIXMEdirnameUndefined/}

\renewcommand{\basename}{mortgageInterest}
\renewcommand{\dirname}{notes/money/}
%\newcommand{\dateintitle}{}
\newcommand{\keywords}{interest, principle, future value, present value}

\newcommand{\authorname}{Peeter Joot}
\newcommand{\onlineurl}{http://sites.google.com/site/peeterjoot2/math2013/\basename.pdf}
\newcommand{\sourcepath}{\dirname\basename.tex}
\newcommand{\generatetitle}[1]{\chapter{#1}}

\newcommand{\vcsinfo}{%
\section*{}
\noindent{\color{DarkOliveGreen}{\rule{\linewidth}{0.1mm}}}
\paragraph{Document version}
%\paragraph{\color{Maroon}{Document version}}
{
\small
\begin{itemize}
\item Available online at:\\ 
\href{\onlineurl}{\onlineurl}
\item Git Repository: \input{./.revinfo/gitRepo.tex}
\item Source: \sourcepath
\item last commit: \input{./.revinfo/gitCommitString.tex}
\item commit date: \input{./.revinfo/gitCommitDate.tex}
\end{itemize}
}
}

%\PassOptionsToPackage{dvipsnames,svgnames}{xcolor}
\PassOptionsToPackage{square,numbers}{natbib}
\documentclass{scrreprt}

\usepackage[left=2cm,right=2cm]{geometry}
\usepackage[svgnames]{xcolor}
\usepackage{peeters_layout}

\usepackage{natbib}

\usepackage[
colorlinks=true,
bookmarks=false,
pdfauthor={\authorname, \email},
backref 
]{hyperref}

% http://tex.stackexchange.com/questions/75773/how-to-reference-problems-by-the-text-label-in-an-exercise-envioronment
\usepackage[english]{cleveref}
\crefname{Exercise}{exercise}{exercises}
\Crefname{Exercise}{Exercise}{Exercises}

\RequirePackage{titlesec}
\RequirePackage{ifthen}

% http://stackoverflow.com/questions/4932910/date-in-the-tabular-environment
\makeatletter
\let\insertdate\@date
\makeatother

\titleformat{\chapter}[display]
{\bfseries\Large}
{\color{DarkSlateGrey}\filleft \authorname
\ifthenelse{\isundefined{\studentnumber}}{}{\\ \studentnumber}
\ifthenelse{\isundefined{\email}}{}{\\ \email}
\ifthenelse{\isundefined{\dateintitle}}{}{\\ \insertdate}
%\ifthenelse{\isundefined{\coursename}}{}{\\ \coursename} % put in title instead.
}
{4ex}
{\color{DarkOliveGreen}{\titlerule}\color{Maroon}
\vspace{2ex}%
\filright}
[\vspace{2ex}%
\color{DarkOliveGreen}\titlerule
]

\newcommand{\beginArtWithToc}[0]{\begin{document}\tableofcontents}
\newcommand{\beginArtNoToc}[0]{\begin{document}}
\newcommand{\EndNoBibArticle}[0]{\end{document}}
\newcommand{\EndArticle}[0]{\bibliography{Bibliography}\bibliographystyle{plainnat}\end{document}}

% 
%\newcommand{\citep}[1]{\cite{#1}}

\colorSectionsForArticle



\beginArtNoToc

\generatetitle{How much difference will shopping around for mortgage rates make?}
\chapter{How much difference will shopping around for mortgage rates make?}
%\label{chap:mortgageInterest}
\section{Motivation}

The banks are all offering variable rate mortgages at slight differences from prime.  I have asked for a few competing rate quotes to see which is best.  Let's compare those here and see how much difference these quotes result in.

\section{Guts}

Consider first a principle amount $-P$, and set of payments $A, B, C, D, ...$, equally spaced in time, corresponding with some effective interest rate per period.  This is sketched in \cref{fig:payments:paymentsFig1}.

\imageFigure{../../figures/money/paymentsFig1}{Payments at fixed intervals}{fig:payments:paymentsFig1}{0.3}

We want to refresh our memory about future value calculations for such a set of payments.  Suppose the interest rate per period is $i$, for example $i = 0.03$ for a 3\% rate, then at the first, second, third, and fourth intervals, we have respectively

\begin{equation}\label{eqn:mortgageInterest:20}
\begin{aligned}
-P(1 + i) + A & \\
\lr{ -P(1 + i) + A }(1 + i) + B &= -P(1+i)^2 + A(1 + i) + B \\
\lr{ \lr{ -P(1 + i) + A }(1 + i) + B}(1 + i) + C &= -P(1+i)^3 + A(1 + i)^2 + B(1 + i) + C \\
\lr{ \lr{ \lr{ -P(1 + i) + A }(1 + i) + B}(1 + i) + C}(1 + i) + D
&= -P(1+i)^4 + A(1 + i)^3 + B(1 + i)^2 + C(1+i) + D.
\end{aligned}
\end{equation}

We can treat the payments independently, each with a separate $(1+i)^k$ factor adjusting the future value of that payment.  The case where the payments are of equal value is of particular interest.  For that, after $k$ payments, the future value of the initial principle offset by any of the payments is

\begin{dmath}\label{eqn:mortgageInterest:40}
F_k 
= -P(1+i)^k + A(1 + i)^{k-1} + A(1 + i)^{k-2} + \cdots A.
\end{dmath}

Recall that a geometric sum 

\begin{dmath}\label{eqn:mortgageInterest:60}
S_k = 1 + a + a^2 + \cdots + a^{k-1},
\end{dmath}

can be solved by writing

\begin{dmath}\label{eqn:mortgageInterest:80}
a S_k - S_k = a^k - 1,
\end{dmath}

so that 

\begin{dmath}\label{eqn:mortgageInterest:100}
S_k = \frac{a^k - 1}{a - 1}.
\end{dmath}

The future value thus sums to

\begin{dmath}\label{eqn:mortgageInterest:120}
F_k 
= -P(1+i)^k + A \frac{ (1 + i)^k - 1}{ 1 + i - 1 }
=
\lr{ -P + \frac{A}{i} } (1 + i)^k - \frac{A}{i}.
\end{dmath}

It's clear that this will be always negative unless $-P + A/i > 0$, or 

\begin{dmath}\label{eqn:mortgageInterest:140}
A > i P.
\end{dmath}

For example, at $i = 3\%$ interest per year, compounded monthly, the monthly break even payment rate for various mortgage amounts $P$ is

\begin{table}[h]
\begin{tabular}{|l|llll|}
\hline %\\
 P & i - 0.6\% & i - 0.55\% & i - 0.53\% & i - 0.5\% \\
\hline %\\
 100000 & 200 & 204 & 206 & 208 \\
 125000 & 250 & 255 & 257 & 260 \\
 150000 & 300 & 306 & 309 & 312 \\
 175000 & 350 & 357 & 360 & 365 \\
 200000 & 400 & 408 & 412 & 417 \\
 225000 & 450 & 459 & 463 & 469 \\
\hline 
\end{tabular}
\end{table}

This provides a first hint that the 0.5-0.6 less than prime rates that the various banks offer will make a difference.  For a principle of $200 K$, we require $17$ dollars more per month to break even (not paying down the principle at all) when comparing prime less $0.5\%$ and $0.6\%$.

Suppose we make $1000$ per month payments at prime minus various amounts.  At the end of a 5 year (60 month) term, we have the following future values

\begin{table}[h]
\begin{tabular}{|l|lllll|}
\hline %\\
 P & i & i - 0.5\% & i - 0.53\% & i - 0.55\% & i - 0.6\% \\
\hline %\\
100000 & -51515 & -49460 & -49338 & -49257 & -49055 \\
125000 & -80555 & -77785 & -77621 & -77512 & -77239 \\
150000 & -109596 & -106110 & -105903 & -105766 & -105423 \\
175000 & -138636 & -134435 & -134186 & -134021 & -133607 \\
200000 & -167677 & -162760 & -162469 & -162275 & -161791 \\
225000 & -196717 & -191085 & -190751 & -190529 & -189976 \\
250000 & -225757 & -219410 & -219034 & -218784 & -218160 \\
\hline 
\end{tabular}
\end{table}

The banks are all offering a default rate of prime minus $0.5\%$ at the moment.  One of them can be negotiated down to prime minus $0.6\%$, and the others down to the various rates posted above.  This negotiation appears to be worth about one thousand dollars over 5 years (for a $200 K$ mortgage).

% this is to produce the sites.google url and version info and so forth (for blog posts)
%\vcsinfo
\EndArticle
%\EndNoBibArticle
