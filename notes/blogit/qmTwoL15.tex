%
% Copyright � 2015 Peeter Joot.  All Rights Reserved.
% Licenced as described in the file LICENSE under the root directory of this GIT repository.
%
\documentclass[]{eliblog}

\usepackage{amsmath}
\usepackage{mathpazo}

%
% shorthand for bold symbols, convenient for vectors and matrices
%
\newcommand{\Ba}[0]{\mathbf{a}}
\newcommand{\Bb}[0]{\mathbf{b}}
\newcommand{\Bc}[0]{\mathbf{c}}
\newcommand{\Bd}[0]{\mathbf{d}}
\newcommand{\Be}[0]{\mathbf{e}}
\newcommand{\Bf}[0]{\mathbf{f}}
\newcommand{\Bg}[0]{\mathbf{g}}
\newcommand{\Bh}[0]{\mathbf{h}}
\newcommand{\Bi}[0]{\mathbf{i}}
\newcommand{\Bj}[0]{\mathbf{j}}
\newcommand{\Bk}[0]{\mathbf{k}}
\newcommand{\Bl}[0]{\mathbf{l}}
\newcommand{\Bm}[0]{\mathbf{m}}
\newcommand{\Bn}[0]{\mathbf{n}}
\newcommand{\Bo}[0]{\mathbf{o}}
\newcommand{\Bp}[0]{\mathbf{p}}
\newcommand{\Bq}[0]{\mathbf{q}}
\newcommand{\Br}[0]{\mathbf{r}}
\newcommand{\Bs}[0]{\mathbf{s}}
\newcommand{\Bt}[0]{\mathbf{t}}
\newcommand{\Bu}[0]{\mathbf{u}}
\newcommand{\Bv}[0]{\mathbf{v}}
\newcommand{\Bw}[0]{\mathbf{w}}
\newcommand{\Bx}[0]{\mathbf{x}}
\newcommand{\By}[0]{\mathbf{y}}
\newcommand{\Bz}[0]{\mathbf{z}}
\newcommand{\BA}[0]{\mathbf{A}}
\newcommand{\BB}[0]{\mathbf{B}}
\newcommand{\BC}[0]{\mathbf{C}}
\newcommand{\BD}[0]{\mathbf{D}}
\newcommand{\BE}[0]{\mathbf{E}}
\newcommand{\BF}[0]{\mathbf{F}}
\newcommand{\BG}[0]{\mathbf{G}}
\newcommand{\BH}[0]{\mathbf{H}}
\newcommand{\BI}[0]{\mathbf{I}}
\newcommand{\BJ}[0]{\mathbf{J}}
\newcommand{\BK}[0]{\mathbf{K}}
\newcommand{\BL}[0]{\mathbf{L}}
\newcommand{\BM}[0]{\mathbf{M}}
\newcommand{\BN}[0]{\mathbf{N}}
\newcommand{\BO}[0]{\mathbf{O}}
\newcommand{\BP}[0]{\mathbf{P}}
\newcommand{\BQ}[0]{\mathbf{Q}}
\newcommand{\BR}[0]{\mathbf{R}}
\newcommand{\BS}[0]{\mathbf{S}}
\newcommand{\BT}[0]{\mathbf{T}}
\newcommand{\BU}[0]{\mathbf{U}}
\newcommand{\BV}[0]{\mathbf{V}}
\newcommand{\BW}[0]{\mathbf{W}}
\newcommand{\BX}[0]{\mathbf{X}}
\newcommand{\BY}[0]{\mathbf{Y}}
\newcommand{\BZ}[0]{\mathbf{Z}}

\newcommand{\Bzero}[0]{\mathbf{0}}
\newcommand{\Btheta}[0]{\boldsymbol{\theta}}
\newcommand{\Btau}[0]{\boldsymbol{\tau}}
\newcommand{\Bomega}[0]{\boldsymbol{\omega}}

%
% shorthand for unit vectors
%
\newcommand{\acap}[0]{\hat{\Ba}}
\newcommand{\bcap}[0]{\hat{\Bb}}
\newcommand{\ccap}[0]{\hat{\Bc}}
\newcommand{\dcap}[0]{\hat{\Bd}}
\newcommand{\ecap}[0]{\hat{\Be}}
\newcommand{\fcap}[0]{\hat{\Bf}}
\newcommand{\gcap}[0]{\hat{\Bg}}
\newcommand{\hcap}[0]{\hat{\Bh}}
\newcommand{\icap}[0]{\hat{\Bi}}
\newcommand{\jcap}[0]{\hat{\Bj}}
\newcommand{\kcap}[0]{\hat{\Bk}}
\newcommand{\lcap}[0]{\hat{\Bl}}
\newcommand{\mcap}[0]{\hat{\Bm}}
\newcommand{\ncap}[0]{\hat{\Bn}}
\newcommand{\ocap}[0]{\hat{\Bo}}
\newcommand{\pcap}[0]{\hat{\Bp}}
\newcommand{\qcap}[0]{\hat{\Bq}}
\newcommand{\rcap}[0]{\hat{\Br}}
\newcommand{\scap}[0]{\hat{\Bs}}
\newcommand{\tcap}[0]{\hat{\Bt}}
\newcommand{\ucap}[0]{\hat{\Bu}}
\newcommand{\vcap}[0]{\hat{\Bv}}
\newcommand{\wcap}[0]{\hat{\Bw}}
\newcommand{\xcap}[0]{\hat{\Bx}}
\newcommand{\ycap}[0]{\hat{\By}}
\newcommand{\zcap}[0]{\hat{\Bz}}
\newcommand{\thetacap}[0]{\hat{\Btheta}}

%
% to write R^n and C^n in a distinguishable fashion.  Perhaps change this
% to the double lined characters upon figuring out how to do so.
%
\newcommand{\C}[1]{$\mathbb{C}^{#1}$}
\newcommand{\R}[1]{$\mathbb{R}^{#1}$}

%
% various generally useful helpers
%

% derivative of #1 wrt. #2:
\newcommand{\D}[2] {\frac {d#2} {d#1}}

\newcommand{\inv}[1]{\frac{1}{#1}}
\newcommand{\cross}[0]{\times}

\newcommand{\abs}[1]{\lvert{#1}\rvert}
\newcommand{\norm}[1]{\lVert{#1}\rVert}
\newcommand{\innerprod}[2]{\langle{#1}, {#2}\rangle}
\newcommand{\dotprod}[2]{{#1} \cdot {#2}}
\newcommand{\bdotprod}[2]{\left({#1} \cdot {#2}\right)}
\newcommand{\crossprod}[2]{{#1} \cross {#2}}
\newcommand{\tripleprod}[3]{\dotprod{\left(\crossprod{#1}{#2}\right)}{#3}}

\DeclareMathOperator{\Proj}{Proj}
\DeclareMathOperator{\Span}{span}
\DeclareMathOperator{\Sgn}{sgn}
\DeclareMathOperator{\Area}{Area}
\DeclareMathOperator{\Volume}{Volume}

%
% A few miscellaneous things specific to this document
%
\newcommand{\crossop}[1]{\crossprod{#1}{}}

% R2 vector.
\newcommand{\VectorTwo}[2]{
\begin{bmatrix}
 {#1} \\
 {#2}
\end{bmatrix}
}

\newcommand{\VectorN}[1]{
\begin{bmatrix}
{#1}_1 \\
{#1}_2 \\
\vdots \\
{#1}_N \\
\end{bmatrix}
}

\newcommand{\DETuvij}[4]{
\begin{vmatrix}
 {#1}_{#3} & {#1}_{#4} \\
 {#2}_{#3} & {#2}_{#4}
\end{vmatrix}
}

\newcommand{\DETuvwijk}[6]{
\begin{vmatrix}
 {#1}_{#4} & {#1}_{#5} & {#1}_{#6} \\
 {#2}_{#4} & {#2}_{#5} & {#2}_{#6} \\
 {#3}_{#4} & {#3}_{#5} & {#3}_{#6}
\end{vmatrix}
}

\newcommand{\DETuvwxijkl}[8]{
\begin{vmatrix}
 {#1}_{#5} & {#1}_{#6} & {#1}_{#7} & {#1}_{#8} \\
 {#2}_{#5} & {#2}_{#6} & {#2}_{#7} & {#2}_{#8} \\
 {#3}_{#5} & {#3}_{#6} & {#3}_{#7} & {#3}_{#8} \\
 {#4}_{#5} & {#4}_{#6} & {#4}_{#7} & {#4}_{#8} \\
\end{vmatrix}
}

%\newcommand{\DETuvwxyijklm}[10]{
%\begin{vmatrix}
% {#1}_{#6} & {#1}_{#7} & {#1}_{#8} & {#1}_{#9} & {#1}_{#10} \\
% {#2}_{#6} & {#2}_{#7} & {#2}_{#8} & {#2}_{#9} & {#2}_{#10} \\
% {#3}_{#6} & {#3}_{#7} & {#3}_{#8} & {#3}_{#9} & {#3}_{#10} \\
% {#4}_{#6} & {#4}_{#7} & {#4}_{#8} & {#4}_{#9} & {#4}_{#10} \\
% {#5}_{#6} & {#5}_{#7} & {#5}_{#8} & {#5}_{#9} & {#5}_{#10}
%\end{vmatrix}
%}

% R3 vector.
\newcommand{\VectorThree}[3]{
\begin{bmatrix}
 {#1} \\
 {#2} \\
 {#3}
\end{bmatrix}
}



\author{Peeter Joot}
\email{peeter.joot@gmail.com}

%\documentclass[]{eliblogwidescreen}

\usepackage{amsmath}
\usepackage{mathpazo}

%
% shorthand for bold symbols, convenient for vectors and matrices
%
\newcommand{\Ba}[0]{\mathbf{a}}
\newcommand{\Bb}[0]{\mathbf{b}}
\newcommand{\Bc}[0]{\mathbf{c}}
\newcommand{\Bd}[0]{\mathbf{d}}
\newcommand{\Be}[0]{\mathbf{e}}
\newcommand{\Bf}[0]{\mathbf{f}}
\newcommand{\Bg}[0]{\mathbf{g}}
\newcommand{\Bh}[0]{\mathbf{h}}
\newcommand{\Bi}[0]{\mathbf{i}}
\newcommand{\Bj}[0]{\mathbf{j}}
\newcommand{\Bk}[0]{\mathbf{k}}
\newcommand{\Bl}[0]{\mathbf{l}}
\newcommand{\Bm}[0]{\mathbf{m}}
\newcommand{\Bn}[0]{\mathbf{n}}
\newcommand{\Bo}[0]{\mathbf{o}}
\newcommand{\Bp}[0]{\mathbf{p}}
\newcommand{\Bq}[0]{\mathbf{q}}
\newcommand{\Br}[0]{\mathbf{r}}
\newcommand{\Bs}[0]{\mathbf{s}}
\newcommand{\Bt}[0]{\mathbf{t}}
\newcommand{\Bu}[0]{\mathbf{u}}
\newcommand{\Bv}[0]{\mathbf{v}}
\newcommand{\Bw}[0]{\mathbf{w}}
\newcommand{\Bx}[0]{\mathbf{x}}
\newcommand{\By}[0]{\mathbf{y}}
\newcommand{\Bz}[0]{\mathbf{z}}
\newcommand{\BA}[0]{\mathbf{A}}
\newcommand{\BB}[0]{\mathbf{B}}
\newcommand{\BC}[0]{\mathbf{C}}
\newcommand{\BD}[0]{\mathbf{D}}
\newcommand{\BE}[0]{\mathbf{E}}
\newcommand{\BF}[0]{\mathbf{F}}
\newcommand{\BG}[0]{\mathbf{G}}
\newcommand{\BH}[0]{\mathbf{H}}
\newcommand{\BI}[0]{\mathbf{I}}
\newcommand{\BJ}[0]{\mathbf{J}}
\newcommand{\BK}[0]{\mathbf{K}}
\newcommand{\BL}[0]{\mathbf{L}}
\newcommand{\BM}[0]{\mathbf{M}}
\newcommand{\BN}[0]{\mathbf{N}}
\newcommand{\BO}[0]{\mathbf{O}}
\newcommand{\BP}[0]{\mathbf{P}}
\newcommand{\BQ}[0]{\mathbf{Q}}
\newcommand{\BR}[0]{\mathbf{R}}
\newcommand{\BS}[0]{\mathbf{S}}
\newcommand{\BT}[0]{\mathbf{T}}
\newcommand{\BU}[0]{\mathbf{U}}
\newcommand{\BV}[0]{\mathbf{V}}
\newcommand{\BW}[0]{\mathbf{W}}
\newcommand{\BX}[0]{\mathbf{X}}
\newcommand{\BY}[0]{\mathbf{Y}}
\newcommand{\BZ}[0]{\mathbf{Z}}

\newcommand{\Bzero}[0]{\mathbf{0}}
\newcommand{\Btheta}[0]{\boldsymbol{\theta}}
\newcommand{\Btau}[0]{\boldsymbol{\tau}}
\newcommand{\Bomega}[0]{\boldsymbol{\omega}}

%
% shorthand for unit vectors
%
\newcommand{\acap}[0]{\hat{\Ba}}
\newcommand{\bcap}[0]{\hat{\Bb}}
\newcommand{\ccap}[0]{\hat{\Bc}}
\newcommand{\dcap}[0]{\hat{\Bd}}
\newcommand{\ecap}[0]{\hat{\Be}}
\newcommand{\fcap}[0]{\hat{\Bf}}
\newcommand{\gcap}[0]{\hat{\Bg}}
\newcommand{\hcap}[0]{\hat{\Bh}}
\newcommand{\icap}[0]{\hat{\Bi}}
\newcommand{\jcap}[0]{\hat{\Bj}}
\newcommand{\kcap}[0]{\hat{\Bk}}
\newcommand{\lcap}[0]{\hat{\Bl}}
\newcommand{\mcap}[0]{\hat{\Bm}}
\newcommand{\ncap}[0]{\hat{\Bn}}
\newcommand{\ocap}[0]{\hat{\Bo}}
\newcommand{\pcap}[0]{\hat{\Bp}}
\newcommand{\qcap}[0]{\hat{\Bq}}
\newcommand{\rcap}[0]{\hat{\Br}}
\newcommand{\scap}[0]{\hat{\Bs}}
\newcommand{\tcap}[0]{\hat{\Bt}}
\newcommand{\ucap}[0]{\hat{\Bu}}
\newcommand{\vcap}[0]{\hat{\Bv}}
\newcommand{\wcap}[0]{\hat{\Bw}}
\newcommand{\xcap}[0]{\hat{\Bx}}
\newcommand{\ycap}[0]{\hat{\By}}
\newcommand{\zcap}[0]{\hat{\Bz}}
\newcommand{\thetacap}[0]{\hat{\Btheta}}

%
% to write R^n and C^n in a distinguishable fashion.  Perhaps change this
% to the double lined characters upon figuring out how to do so.
%
\newcommand{\C}[1]{$\mathbb{C}^{#1}$}
\newcommand{\R}[1]{$\mathbb{R}^{#1}$}

%
% various generally useful helpers
%

% derivative of #1 wrt. #2:
\newcommand{\D}[2] {\frac {d#2} {d#1}}

\newcommand{\inv}[1]{\frac{1}{#1}}
\newcommand{\cross}[0]{\times}

\newcommand{\abs}[1]{\lvert{#1}\rvert}
\newcommand{\norm}[1]{\lVert{#1}\rVert}
\newcommand{\innerprod}[2]{\langle{#1}, {#2}\rangle}
\newcommand{\dotprod}[2]{{#1} \cdot {#2}}
\newcommand{\bdotprod}[2]{\left({#1} \cdot {#2}\right)}
\newcommand{\crossprod}[2]{{#1} \cross {#2}}
\newcommand{\tripleprod}[3]{\dotprod{\left(\crossprod{#1}{#2}\right)}{#3}}

\DeclareMathOperator{\Proj}{Proj}
\DeclareMathOperator{\Span}{span}
\DeclareMathOperator{\Sgn}{sgn}
\DeclareMathOperator{\Area}{Area}
\DeclareMathOperator{\Volume}{Volume}

%
% A few miscellaneous things specific to this document
%
\newcommand{\crossop}[1]{\crossprod{#1}{}}

% R2 vector.
\newcommand{\VectorTwo}[2]{
\begin{bmatrix}
 {#1} \\
 {#2}
\end{bmatrix}
}

\newcommand{\VectorN}[1]{
\begin{bmatrix}
{#1}_1 \\
{#1}_2 \\
\vdots \\
{#1}_N \\
\end{bmatrix}
}

\newcommand{\DETuvij}[4]{
\begin{vmatrix}
 {#1}_{#3} & {#1}_{#4} \\
 {#2}_{#3} & {#2}_{#4}
\end{vmatrix}
}

\newcommand{\DETuvwijk}[6]{
\begin{vmatrix}
 {#1}_{#4} & {#1}_{#5} & {#1}_{#6} \\
 {#2}_{#4} & {#2}_{#5} & {#2}_{#6} \\
 {#3}_{#4} & {#3}_{#5} & {#3}_{#6}
\end{vmatrix}
}

\newcommand{\DETuvwxijkl}[8]{
\begin{vmatrix}
 {#1}_{#5} & {#1}_{#6} & {#1}_{#7} & {#1}_{#8} \\
 {#2}_{#5} & {#2}_{#6} & {#2}_{#7} & {#2}_{#8} \\
 {#3}_{#5} & {#3}_{#6} & {#3}_{#7} & {#3}_{#8} \\
 {#4}_{#5} & {#4}_{#6} & {#4}_{#7} & {#4}_{#8} \\
\end{vmatrix}
}

%\newcommand{\DETuvwxyijklm}[10]{
%\begin{vmatrix}
% {#1}_{#6} & {#1}_{#7} & {#1}_{#8} & {#1}_{#9} & {#1}_{#10} \\
% {#2}_{#6} & {#2}_{#7} & {#2}_{#8} & {#2}_{#9} & {#2}_{#10} \\
% {#3}_{#6} & {#3}_{#7} & {#3}_{#8} & {#3}_{#9} & {#3}_{#10} \\
% {#4}_{#6} & {#4}_{#7} & {#4}_{#8} & {#4}_{#9} & {#4}_{#10} \\
% {#5}_{#6} & {#5}_{#7} & {#5}_{#8} & {#5}_{#9} & {#5}_{#10}
%\end{vmatrix}
%}

% R3 vector.
\newcommand{\VectorThree}[3]{
\begin{bmatrix}
 {#1} \\
 {#2} \\
 {#3}
\end{bmatrix}
}



\author{Peeter Joot}
\email{peeter.joot@gmail.com}


\chapter{PHY456H1F: Quantum Mechanics II.  Lecture 15 (Taught by Prof J.E. Sipe).  Rotation operator in spin space}
\label{chap:qmTwoL15}
%\useCCL
\blogpage{http://sites.google.com/site/peeterjoot/math2011/qmTwoL15.pdf}
\date{Oct 31, 2011}
\revisionInfo{qmTwoL15.tex}

\beginArtWithToc
%\beginArtNoToc

\section{Disclaimer.}

Peeter's lecture notes from class.  May not be entirely coherent.

\section{Rotation operator in spin space.}

We can formally expand our rotation operator in Taylor series

\begin{equation}\label{eqn:qmTwoL15:10}
e^{-i \theta \ncap \cdot \BS/\hbar}
= 
I 
+
\left(-i \theta \ncap \cdot \BS/\hbar\right)
+
\inv{2!}
\left(-i \theta \ncap \cdot \BS/\hbar\right)^2
+
\inv{3!}
\left(-i \theta \ncap \cdot \BS/\hbar\right)^3
+ \cdots
\end{equation}

or
\begin{align*}
e^{-i \theta \ncap \cdot \Bsigma/2}
&= 
I 
+
\left(-i \theta \ncap \cdot \Bsigma/2\right)
+
\inv{2!}
\left(-i \theta \ncap \cdot \Bsigma/2\right)^2
+
\inv{3!}
\left(-i \theta \ncap \cdot \Bsigma/2\right)^3
+ \cdots \\
&=
\sigma_0 
+
\left(\frac{-i \theta}{2}\right) (\ncap \cdot \Bsigma)
+
\inv{2!} \left(\frac{-i \theta}{2}\right) (\ncap \cdot \Bsigma)^2
+
\inv{3!} \left(\frac{-i \theta}{2}\right) (\ncap \cdot \Bsigma)^3
+ \cdots \\
&=
\sigma_0 
+
\left(\frac{-i \theta}{2}\right) (\ncap \cdot \Bsigma)
+
\inv{2!} \left(\frac{-i \theta}{2}\right) \sigma_0
+
\inv{3!} \left(\frac{-i \theta}{2}\right) (\ncap \cdot \Bsigma) 
+ \cdots \\
&=
\sigma_0 \left( 1 - \inv{2!}\left(\frac{\theta}{2}\right)^2 + \cdots \right) 
+
(\ncap \cdot \Bsigma) \left( \frac{\theta}{2} - \inv{3!}\left(\frac{\theta}{2}\right)^3 + \cdots \right) \\
&=
\cos(\theta/2) \sigma_0 + \sin(\theta/2) (\ncap \cdot \Bsigma)
\end{align*}

where we've used the fact that $(\ncap \cdot \Bsigma)^2 = \sigma_0$.

So our representation of the spin operator is

\begin{equation}\label{eqn:qmTwoL15:30}
\begin{aligned}
e^{-i \theta \ncap \cdot \BS/\hbar} 
&\rightarrow
\cos(\theta/2) \sigma_0 + \sin(\theta/2) (\ncap \cdot \Bsigma) \\
&=
\cos(\theta/2) \sigma_0 + \sin(\theta/2) 
\left(n_x \PauliX + n_y \PauliY + n_z \PauliZ \right) \\
&=
\begin{bmatrix}
\cos(\theta/2) -i n_z \sin(\theta/2) & -i (n_x -i n_y) \sin(\theta/2) \\
-i (n_x + i n_y) \sin(\theta/2) & \cos(\theta/2) +i n_z \sin(\theta/2) 
\end{bmatrix}
\end{aligned}
\end{equation}

Note that, in particular, 

\begin{equation}\label{eqn:qmTwoL15:50}
e^{-2 \pi i \ncap \cdot \BS/\hbar} \rightarrow \cos\pi \sigma_0 = -\sigma_0
\end{equation}

This ``rotates'' the ket, but introduces a phase factor.

Can do this in general for other degrees of spin, for $s = 1/2, 3/2, 5/2, \cdots$.

\paragraph{Unfortunate interjection by me:} Mentioned the half angle rotation operator that requires a half angle operator sandwich.  Sipe thought I might be talking about a Heisenburg picture representation, where we have something like this in expectation values

\begin{equation}\label{eqn:qmTwoL15:70}
\ket{\psi'} = e^{-i \theta \ncap \cdot \BJ/\hbar} \ket{\psi}
\end{equation}

so that 
\begin{equation}\label{eqn:qmTwoL15:90}
\bra{\psi'}
\mathcal{O}
\ket{\psi'} = \bra{\psi} 
e^{i \theta \ncap \cdot \BJ/\hbar} 
\mathcal{O}
e^{-i \theta \ncap \cdot \BJ/\hbar} 
\ket{\psi}
\end{equation}

\section{Spin dynamics}

At least classically, the angular momentum of charged objects is associated with a magnetic moment

FIXME: F1.

\begin{equation}\label{eqn:qmTwoL15:110}
\Bmu = I A \Be_\perp
\end{equation}

In our scheme, following the (cgs?) text conventions of \cite{desai2009quantum}, where the $\BE$ and $\BB$ have the same units, we write

\begin{equation}\label{eqn:qmTwoL15:130}
\Bmu = \frac{I A}{c} \Be_\perp
\end{equation}

For a charge moving in a circle

FIXME: F2

\begin{equation}\label{eqn:qmTwoL15:150}
\begin{aligned}
I 
&= \frac{\text{charge}}{\text{time}} \\
&= 
\frac{\text{distance}}{\text{time}} \frac{\text{charge}}{\text{distance}} \\
&= 
\frac{q v}{ 2 \pi r}
\end{aligned}
\end{equation}

so the magnetic moment is 

\begin{equation}\label{eqn:qmTwoL15:170}
\begin{aligned}
\mu 
&= \frac{q v}{ 2 \pi r} \frac{\pi r^2}{c}  \\
&= \frac{q }{ 2 m c } (m v r) \\
&= \gamma L
\end{aligned}
\end{equation}

Here $\gamma$ is the gyromagnetic ratio

Recall that we have a torque 

FIXME: F3

\begin{equation}\label{eqn:qmTwoL15:190}
\BT = \Bmu \cross \BB
\end{equation}

tending to line up $\Bmu$ with $\BB$.  The energy is then

\begin{equation}\label{eqn:qmTwoL15:210}
-\Bmu \cdot \BB
\end{equation}

Also recall that this torque leads to precession 

\begin{equation}\label{eqn:qmTwoL15:230}
\ddt{\BL} = \BT = \gamma \BL \cross \BB
\end{equation}

FIXME: F4

with precession frequency 

\begin{equation}\label{eqn:qmTwoL15:250}
\Bomega = - \gamma \BB.
\end{equation}

For a current due to a moving electron

\begin{equation}\label{eqn:qmTwoL15:270}
\gamma = -\frac{e}{2 m c} < 0
\end{equation}

where we are, here, writing for charge on the electron $-e$.  

\paragraph{Question: steady state currents only?}.  Yes, this is only true for steady state currents.

For the translational motion of an electron, even if it is not moving in a steady way, regardless of it's dynamics

\begin{equation}\label{eqn:qmTwoL15:290}
\Bmu_0 = - \frac{e}{2 m c} \BL
\end{equation}

Now, back to quantum mechanics, we turn $\Bmu_0$ into a dipole moment operator and $\BL$ is ``promoted'' to an angular momentum operator.

\begin{equation}\label{eqn:qmTwoL15:310}
H_{\text{int}} = - \Bmu_0 \cdot \BB
\end{equation}

What about the ``spin''?

Perhaps 

\begin{equation}\label{eqn:qmTwoL15:330}
\Bmu_s = \gamma_s \BS
\end{equation}

we write this as 
\begin{equation}\label{eqn:qmTwoL15:350}
\Bmu_s = g \left( -\frac{e}{ 2 m c} \right)\BS
\end{equation}

so that

\begin{equation}\label{eqn:qmTwoL15:370}
\gamma_s = - \frac{g e}{ 2 m c} 
\end{equation}

Experimentally, one finds to very good approximation

\begin{equation}\label{eqn:qmTwoL15:390}
g = 2
\end{equation}

There was a lot of trouble with this in early quantum mechanics where people got things wrong, and cancelled the wrong factors of $2$.

In fact, Dirac's relativistic theory for the electron predicts $g=2$.

When this is measured experimentally, one does not get exactly $g=2$, and a theory that also incorporates photon creation and destruction and the interaction with the electron with such (virtual) photons.  We get

\begin{equation}\label{eqn:qmTwoL15:410}
\begin{aligned}
g_{\text{theory}} 
&= 2 \left(
1.001159652140 (\pm 28)
\right) \\
g_{\text{experimental}} 
&= 2 \left(
1.0011596521884 (\pm 43)
\right)
\end{aligned}
\end{equation}

Feynman compared this to measuring the distance between New York and Las Angeles to within the accuracy of a human hair.

\EndArticle
