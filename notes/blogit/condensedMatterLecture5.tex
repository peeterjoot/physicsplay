%
% Copyright � 2013 Peeter Joot.  All Rights Reserved.
% Licenced as described in the file LICENSE under the root directory of this GIT repository.
%
\newcommand{\authorname}{Peeter Joot}
\newcommand{\email}{peeterjoot@protonmail.com}
\newcommand{\basename}{FIXMEbasenameUndefined}
\newcommand{\dirname}{notes/FIXMEdirnameUndefined/}

\renewcommand{\basename}{condensedMatterLecture5}
\renewcommand{\dirname}{notes/phy487/}
\newcommand{\keywords}{Condensed matter physics, PHY487H1F}
\newcommand{\authorname}{Peeter Joot}
\newcommand{\onlineurl}{http://sites.google.com/site/peeterjoot2/math2013/\basename.pdf}
\newcommand{\sourcepath}{\dirname\basename.tex}
\newcommand{\generatetitle}[1]{\chapter{#1}}

\newcommand{\vcsinfo}{%
\section*{}
\noindent{\color{DarkOliveGreen}{\rule{\linewidth}{0.1mm}}}
\paragraph{Document version}
%\paragraph{\color{Maroon}{Document version}}
{
\small
\begin{itemize}
\item Available online at:\\ 
\href{\onlineurl}{\onlineurl}
\item Git Repository: \input{./.revinfo/gitRepo.tex}
\item Source: \sourcepath
\item last commit: \input{./.revinfo/gitCommitString.tex}
\item commit date: \input{./.revinfo/gitCommitDate.tex}
\end{itemize}
}
}

%\PassOptionsToPackage{dvipsnames,svgnames}{xcolor}
\PassOptionsToPackage{square,numbers}{natbib}
\documentclass{scrreprt}

\usepackage[left=2cm,right=2cm]{geometry}
\usepackage[svgnames]{xcolor}
\usepackage{peeters_layout}

\usepackage{natbib}

\usepackage[
colorlinks=true,
bookmarks=false,
pdfauthor={\authorname, \email},
backref 
]{hyperref}

% http://tex.stackexchange.com/questions/75773/how-to-reference-problems-by-the-text-label-in-an-exercise-envioronment
\usepackage[english]{cleveref}
\crefname{Exercise}{exercise}{exercises}
\Crefname{Exercise}{Exercise}{Exercises}

\RequirePackage{titlesec}
\RequirePackage{ifthen}

% http://stackoverflow.com/questions/4932910/date-in-the-tabular-environment
\makeatletter
\let\insertdate\@date
\makeatother

\titleformat{\chapter}[display]
{\bfseries\Large}
{\color{DarkSlateGrey}\filleft \authorname
\ifthenelse{\isundefined{\studentnumber}}{}{\\ \studentnumber}
\ifthenelse{\isundefined{\email}}{}{\\ \email}
\ifthenelse{\isundefined{\dateintitle}}{}{\\ \insertdate}
%\ifthenelse{\isundefined{\coursename}}{}{\\ \coursename} % put in title instead.
}
{4ex}
{\color{DarkOliveGreen}{\titlerule}\color{Maroon}
\vspace{2ex}%
\filright}
[\vspace{2ex}%
\color{DarkOliveGreen}\titlerule
]

\newcommand{\beginArtWithToc}[0]{\begin{document}\tableofcontents}
\newcommand{\beginArtNoToc}[0]{\begin{document}}
\newcommand{\EndNoBibArticle}[0]{\end{document}}
\newcommand{\EndArticle}[0]{\bibliography{Bibliography}\bibliographystyle{plainnat}\end{document}}

% 
%\newcommand{\citep}[1]{\cite{#1}}

\colorSectionsForArticle



%\citep{harald2003solid} \S x.y
%\citep{ibach2009solid} \S x.y

\usepackage[version=3]{mhchem}

\beginArtNoToc
\generatetitle{PHY487H1F Condensed Matter Physics.  Lecture 5: General theory of diffraction.  Taught by Prof.\ Stephen Julian}
%\chapter{General theory of diffraction}
\label{chap:condensedMatterLecture5}

\section{Disclaimer}

Peeter's lecture notes from class.  May not be entirely coherent.

\section{General theory of diffraction}

F1

F2

Incident beam makes electrons vibrate at frequency $\omega_0$, and reradiate at $\omega_0$.

The diffraction pattern is the constructive interference of the reradiated x-rays (or neutrons) 

At $P$, primary beam has amplitude

\begin{dmath}\label{eqn:condensedMatterLecture5:20}
A_p(\Br, t) = 
%\mathLabelBox{
A_0 
%}{Constant everywhere}
\exp\lr{ i \lr{ 
%\mathLabelBox
%{
k_0 
%}{
%$\frac{2 \pi}{\lambda_0}$
%}
\cdot 
\lr{ \BR + \Br} - \omega_0 t}  }
\end{dmath}

Here $A_0$ is constant everywhere and $k_0 = 2 \pi/\lambda_0$.

Scattered wave at $P$ has amplitude 

\begin{dmath}\label{eqn:condensedMatterLecture5:120}
\rho(\Br) A_p(\Br, t)
\end{dmath}

where

\begin{dmath}\label{eqn:condensedMatterLecture5:40}
\rho(\Br) = \text{``scattering density''}
\end{dmath}

For x-rays

\begin{dmath}\label{eqn:condensedMatterLecture5:60}
\rho(\Br) \sim electron density
\end{dmath}

This density is largest near the necleus, at least for large Z (heavy) atoms like \ce{La}, and \ce{Ac}.  Large Z atoms are easier to see.

Amplitude at $B$, due to $\Br$

\begin{dmath}\label{eqn:condensedMatterLecture5:80}
A_B(\Br) = A_p(\Br) \rho(\Br) 
\frac{
e^{ 
i \Bk \cdot 
\lr{ \BR' - \Br} 
}
}
{
\Abs{
\BR' - \Br 
}
}
\approx
A_p(\Br, t) \rho(\Br) 
\frac{
e^{ 
	i \Bk \cdot 
	\lr{ \BR' - \Br} 
  }
}
{
  \Abs{\BR'}
}
\end{dmath}

time independent part of $\rho(\Br)$ => secondary beam has frequency $\omega_0$, or

\begin{dmath}\label{eqn:condensedMatterLecture5:100}
\Abs{\Bk} = \Abs{\Bk_0} = \frac{\omega_0}{c}
\end{dmath}

This is the \underlineAndIndex{elastic scattering}.  We ignore inelastic scattering.

The vector $\BR'$ determines the direction of $\Bk$.

If $R' \gg r$, $\Bk$ is close to the same for all $\Br$.  Then \cref{eqn:condensedMatterLecture5:80} is approximately

\begin{dmath}\label{eqn:condensedMatterLecture5:140}
A_B(\Br) \approx 
\mathLabelBox
{
\frac{A_0}{R'} e^{ i \lr{ \Bk_0 \cdot \BR + \Bk \cdot \BR'} }
}{independent of $\Br$}
\mathLabelBox
{
\rho(\Br) e^{ i \lr{ \Bk_0 - \Bk} \cdot \Br } e^{-i \omega_0 t}
}
{dependent of $\Br$}
\end{dmath}

Total amplitude at $B$

\begin{dmath}\label{eqn:condensedMatterLecture5:160}
A_B \propto e^{-i \omega_0 t } \int \rho(\Br) e^{ i 
\mathLabelBox{
\lr{ \Bk_0 -\Bk} 
}
{ $\equiv - \BK$}
\cdot \Br }
\end{dmath}

So that the intensity is

\begin{dmath}\label{eqn:condensedMatterLecture5:180}
I_\BB \propto \Abs{A_B}^2 \propto 
\Abs{
\int \rho(\Br) e^{-i \BK \cdot \Br}
d\Br
}
\end{dmath}

This is the Fourier transform of the scattering density.

\section{3D periodic structures}

- 1D crystal

F3

\begin{dmath}\label{eqn:condensedMatterLecture5:n}
\rho(x) = \sum_n p_n e^{ i n 2 \pi x/a }
\end{dmath}

Show that 

\begin{dmath}\label{eqn:condensedMatterLecture5:n}
\rho(x + m a) = \rho(x)
\end{dmath}

- 3D lattice

\begin{dmath}\label{eqn:condensedMatterLecture5:n}
\rho(\Br) = \sum_\BG \rho_\BG e^{ i \BG \cdot \Br }
\end{dmath}

With

\begin{dmath}\label{eqn:condensedMatterLecture5:n}
\Br_n = 
n_1 \Ba_1 +
n_2 \Ba_2 +
n_3 \Ba_3,
\end{dmath}

find the $\BG$'s, such that 

\begin{dmath}\label{eqn:condensedMatterLecture5:n}
\rho(\Br + \Br_n) = \rho(\Br).
\end{dmath}

\begin{dmath}\label{eqn:condensedMatterLecture5:n}
e^{ i \BG \cdot \Br_n } = 1,
\end{dmath}

or
\begin{dmath}\label{eqn:condensedMatterLecture5:n}
\BG \cdot \Br_n = 2 \pi m,
\end{dmath}

where $m$ is an integer.

Try 

\begin{dmath}\label{eqn:condensedMatterLecture5:n}
\BG = 
h \Bg_1 +
k \Bg_2 +
l \Bg_3,
\end{dmath}

where $h, k, l$ are integers.

The $\BG$'s are wave vectors of waves with the periodicity of the lattice.

\makeexample{2d periodic lattice}{example:condensedMatterLecture5:n}{

F4: brutalized.  see 05 lecture.pdf

Wave 1 :

\begin{subequations}
\begin{dmath}\label{eqn:condensedMatterLecture5:n}
\Bg_1 \cdot \Ba_1 = g_1 a_1 \cos\theta = 2 \pi
\end{dmath}
\begin{dmath}\label{eqn:condensedMatterLecture5:n}
\Bg_1 \cdot \Ba_2 = 0
\end{dmath}
\end{subequations}

Wave 2 : $\Bg_2 \cdot \Ba_1 = 0, $

\begin{subequations}
\begin{dmath}\label{eqn:condensedMatterLecture5:n}
\Bg_2 \cdot \Ba_1 = 0
\end{dmath}
\begin{dmath}\label{eqn:condensedMatterLecture5:n}
\Bg_2 \cdot \Ba_2 
= g_1 a_1 \cos\theta = 2 \pi
\end{dmath}
\end{subequations}
}

We introduce \underlineAndIndex{reciprocal vectors} defined by

\begin{dmath}\label{eqn:condensedMatterLecture5:n}
\Bg_i \cdot \Ba_j = 2 \pi \delta_{ij}
\end{dmath}

The general formula in 3D is

\begin{dmath}\label{eqn:condensedMatterLecture5:n}
\Bg_1 = 2 \pi 
\frac
{ \Ba_2 \cross \Ba_3 }
{ \Ba_1 \cdot \lr{ \Ba_2 \cross \Ba_3} }
\end{dmath}

The numerator cross product ensures that $\Bg_1$ is perpendicular to $\Ba_2$ and $\Ba_3$.  The one in the denominator ``cancels'' the $\Ba_2 \cross \Ba_3$ in the numerator.    This gives 

\begin{dmath}\label{eqn:condensedMatterLecture5:n}
\lambda_1 = a_0 \cos \theta_1
\end{dmath}

%\EndArticle
\EndNoBibArticle
