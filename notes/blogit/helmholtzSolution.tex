%
% Copyright � 2016 Peeter Joot.  All Rights Reserved.
% Licenced as described in the file LICENSE under the root directory of this GIT repository.
%
%{
\newcommand{\authorname}{Peeter Joot}
\newcommand{\email}{peeterjoot@protonmail.com}
\newcommand{\basename}{FIXMEbasenameUndefined}
\newcommand{\dirname}{notes/FIXMEdirnameUndefined/}

\renewcommand{\basename}{helmholtzSolution}
%\renewcommand{\dirname}{notes/phy1520/}
\renewcommand{\dirname}{notes/ece1228-electromagnetic-theory/}
%\newcommand{\dateintitle}{}
%\newcommand{\keywords}{}

\newcommand{\authorname}{Peeter Joot}
\newcommand{\onlineurl}{http://sites.google.com/site/peeterjoot2/math2013/\basename.pdf}
\newcommand{\sourcepath}{\dirname\basename.tex}
\newcommand{\generatetitle}[1]{\chapter{#1}}

\newcommand{\vcsinfo}{%
\section*{}
\noindent{\color{DarkOliveGreen}{\rule{\linewidth}{0.1mm}}}
\paragraph{Document version}
%\paragraph{\color{Maroon}{Document version}}
{
\small
\begin{itemize}
\item Available online at:\\ 
\href{\onlineurl}{\onlineurl}
\item Git Repository: \input{./.revinfo/gitRepo.tex}
\item Source: \sourcepath
\item last commit: \input{./.revinfo/gitCommitString.tex}
\item commit date: \input{./.revinfo/gitCommitDate.tex}
\end{itemize}
}
}

%\PassOptionsToPackage{dvipsnames,svgnames}{xcolor}
\PassOptionsToPackage{square,numbers}{natbib}
\documentclass{scrreprt}

\usepackage[left=2cm,right=2cm]{geometry}
\usepackage[svgnames]{xcolor}
\usepackage{peeters_layout}

\usepackage{natbib}

\usepackage[
colorlinks=true,
bookmarks=false,
pdfauthor={\authorname, \email},
backref 
]{hyperref}

% http://tex.stackexchange.com/questions/75773/how-to-reference-problems-by-the-text-label-in-an-exercise-envioronment
\usepackage[english]{cleveref}
\crefname{Exercise}{exercise}{exercises}
\Crefname{Exercise}{Exercise}{Exercises}

\RequirePackage{titlesec}
\RequirePackage{ifthen}

% http://stackoverflow.com/questions/4932910/date-in-the-tabular-environment
\makeatletter
\let\insertdate\@date
\makeatother

\titleformat{\chapter}[display]
{\bfseries\Large}
{\color{DarkSlateGrey}\filleft \authorname
\ifthenelse{\isundefined{\studentnumber}}{}{\\ \studentnumber}
\ifthenelse{\isundefined{\email}}{}{\\ \email}
\ifthenelse{\isundefined{\dateintitle}}{}{\\ \insertdate}
%\ifthenelse{\isundefined{\coursename}}{}{\\ \coursename} % put in title instead.
}
{4ex}
{\color{DarkOliveGreen}{\titlerule}\color{Maroon}
\vspace{2ex}%
\filright}
[\vspace{2ex}%
\color{DarkOliveGreen}\titlerule
]

\newcommand{\beginArtWithToc}[0]{\begin{document}\tableofcontents}
\newcommand{\beginArtNoToc}[0]{\begin{document}}
\newcommand{\EndNoBibArticle}[0]{\end{document}}
\newcommand{\EndArticle}[0]{\bibliography{Bibliography}\bibliographystyle{plainnat}\end{document}}

% 
%\newcommand{\citep}[1]{\cite{#1}}

\colorSectionsForArticle



\usepackage{peeters_layout_exercise}
\usepackage{peeters_braket}
\usepackage{peeters_figures}
\usepackage{siunitx}
%\usepackage{mhchem} % \ce{}
%\usepackage{macros_bm} % \bcM
%\usepackage{txfonts} % \ointclockwise

\beginArtNoToc

\generatetitle{Helmholtz solution}
%\chapter{Helmholtz solution}
%\label{chap:helmholtzSolution}

We know that the Laplace equation

\begin{dmath}\label{eqn:helmholtzSolution:40}
\spacegrad^2 \phi = \rho/\epsilon,
\end{dmath}

has solution

\begin{dmath}\label{eqn:helmholtzSolution:60}
\phi(\Bx) = \inv{4 \pi \epsilon} \int \frac{\rho(\Bx')}{\Abs{\Br - \Br'}} dV'.
\end{dmath}

We also know that the wave equation
\begin{dmath}\label{eqn:helmholtzSolution:20}
\spacegrad^2 \phi = \inv{v^2} \PDSq{t}{\phi}
\end{dmath}

has solutions of the form

%f(x,t) = f(x - t/v)
\begin{dmath}\label{eqn:helmholtzSolution:240}
\phi(\Bx, t) = f( t \pm \frac{\Bbeta}{\omega} \cdot \Bx ),
\end{dmath}

or
\begin{dmath}\label{eqn:helmholtzSolution:260}
\phi(\Bx, t) = f( t \pm \frac{\Bbeta}{\omega} \cdot \Bx ),
\end{dmath}

\begin{dmath}\label{eqn:helmholtzSolution:300}
\beta^2 \phi = \rho/\epsilon
\end{dmath}

In class it was argued that the solution of the non-homogeneous Helmholtz equation

\begin{dmath}\label{eqn:helmholtzSolution:20}
\spacegrad^2 \phi + \beta^2 \phi = \rho/\epsilon
\end{dmath}

should have the same form, but retarded in time

\begin{dmath}\label{eqn:helmholtzSolution:80}
\phi(\Bx, t)
= \inv{4 \pi \epsilon} \int \frac{\rho(\Bx', t - R/v)}{\Abs{\Br - \Br'}} dV'
= \Real \lr{
\inv{4 \pi \epsilon} \int \frac{\rho(\Bx') e^{j \omega( t - R/v)}}{\Abs{\Br - \Br'}} dV'
}.
\end{dmath}

so if the time domain solution is \( \phi(\Bx, t) = \Real \lr{ \phi(\Bx) e^{j \omega t} } \), the time harmonic potential is

\begin{dmath}\label{eqn:helmholtzSolution:100}
\phi(\Bx)
=
\inv{4 \pi \epsilon} \int \frac{\rho(\Bx') e^{-j \omega R/v}}{\Abs{\Br - \Br'}} dV'.
\end{dmath}

This was written as

\begin{dmath}\label{eqn:helmholtzSolution:120}
\phi(\Bx)
=
\inv{4 \pi \epsilon} \int \frac{\rho(\Bx') e^{-j \beta R}}{\Abs{\Br - \Br'}} dV'.
\end{dmath}

where we must have
\begin{dmath}\label{eqn:helmholtzSolution:140}
\beta = \omega/v.
\end{dmath}

The justification for the time retardation was physical, arguing that there is a delay required for any change to manifest at a distance.

\paragraph{TEM, TM, AND TE modes}

The TEM mode is that for which

\begin{equation}\label{eqn:helmholtzSolution:180}
\BH \cdot \zcap = \BE \cdot \zcap = 0,
\end{equation}

whereas the TM mode is one for which

\begin{equation}\label{eqn:helmholtzSolution:200}
\BH \cdot \zcap = 0,
\end{equation}

and the TE mode is one for which
\begin{equation}\label{eqn:helmholtzSolution:220}
\BE \cdot \zcap = 0.
\end{equation}

Strictly speaking these are the \( \textrm{TEM}^z, \textrm{TM}^z \) and \( \textrm{TE}^z \) modes respectively.

See \citep{cheng1989field} for a justification that \( \BJ \) must be axial (like a coaxial cable) for a TEM mode satisfying

\begin{dmath}\label{eqn:helmholtzSolution:160}
\spacegrad \cross \BH = \BJ + j \omega \epsilon \BE.
\end{dmath}

From the expansion of the ``Master solution equation'', one way of constructing a TEM mode was pointed out in the notes.  A similar TEM solution, can be constructed in a similar swapping \( \BA \) and \( \BF \).  HW: Do this as an exersize as well as walking through the details.

\paragraph{TE}

The class notes give the  \( \textrm{TE}^z \) solution as
\begin{dmath}\label{eqn:helmholtzSolution:280}
\begin{aligned}
E_x &= -j \inv{\omega \mu \epsilon} \frac{\partial^2 A_z}{\partial_x \partial_y} ...
\end{aligned}
\end{dmath}

HW: In class it was claimed that \( \BE \cdot \BH = 0 \), but that wasn't obvious by inspection.

Note: be mindful that some books will express this with one of the field components equal to zero.  This is due to the fact that a rotation of coordinates can kill one of the field components.  HW: show this.

\paragraph{more on tunnelling.}
%superluminal tunnelling through wave guides: big -> small -> big.  similar to the tunnelling in prisms coupled by dielelectric even past critical angle.

%}
\EndArticle
%\EndNoBibArticle
