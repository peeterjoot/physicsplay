%
% Copyright � 2013 Peeter Joot.  All Rights Reserved.
% Licenced as described in the file LICENSE under the root directory of this GIT repository.
%
\newcommand{\authorname}{Peeter Joot}
\newcommand{\email}{peeterjoot@protonmail.com}
\newcommand{\basename}{FIXMEbasenameUndefined}
\newcommand{\dirname}{notes/FIXMEdirnameUndefined/}

\renewcommand{\basename}{threeSpringLoop}
\renewcommand{\dirname}{notes/FIXMEwheretodirname/}
%\newcommand{\dateintitle}{}
%\newcommand{\keywords}{}

\newcommand{\authorname}{Peeter Joot}
\newcommand{\onlineurl}{http://sites.google.com/site/peeterjoot2/math2013/\basename.pdf}
\newcommand{\sourcepath}{\dirname\basename.tex}
\newcommand{\generatetitle}[1]{\chapter{#1}}

\newcommand{\vcsinfo}{%
\section*{}
\noindent{\color{DarkOliveGreen}{\rule{\linewidth}{0.1mm}}}
\paragraph{Document version}
%\paragraph{\color{Maroon}{Document version}}
{
\small
\begin{itemize}
\item Available online at:\\ 
\href{\onlineurl}{\onlineurl}
\item Git Repository: \input{./.revinfo/gitRepo.tex}
\item Source: \sourcepath
\item last commit: \input{./.revinfo/gitCommitString.tex}
\item commit date: \input{./.revinfo/gitCommitDate.tex}
\end{itemize}
}
}

%\PassOptionsToPackage{dvipsnames,svgnames}{xcolor}
\PassOptionsToPackage{square,numbers}{natbib}
\documentclass{scrreprt}

\usepackage[left=2cm,right=2cm]{geometry}
\usepackage[svgnames]{xcolor}
\usepackage{peeters_layout}

\usepackage{natbib}

\usepackage[
colorlinks=true,
bookmarks=false,
pdfauthor={\authorname, \email},
backref 
]{hyperref}

% http://tex.stackexchange.com/questions/75773/how-to-reference-problems-by-the-text-label-in-an-exercise-envioronment
\usepackage[english]{cleveref}
\crefname{Exercise}{exercise}{exercises}
\Crefname{Exercise}{Exercise}{Exercises}

\RequirePackage{titlesec}
\RequirePackage{ifthen}

% http://stackoverflow.com/questions/4932910/date-in-the-tabular-environment
\makeatletter
\let\insertdate\@date
\makeatother

\titleformat{\chapter}[display]
{\bfseries\Large}
{\color{DarkSlateGrey}\filleft \authorname
\ifthenelse{\isundefined{\studentnumber}}{}{\\ \studentnumber}
\ifthenelse{\isundefined{\email}}{}{\\ \email}
\ifthenelse{\isundefined{\dateintitle}}{}{\\ \insertdate}
%\ifthenelse{\isundefined{\coursename}}{}{\\ \coursename} % put in title instead.
}
{4ex}
{\color{DarkOliveGreen}{\titlerule}\color{Maroon}
\vspace{2ex}%
\filright}
[\vspace{2ex}%
\color{DarkOliveGreen}\titlerule
]

\newcommand{\beginArtWithToc}[0]{\begin{document}\tableofcontents}
\newcommand{\beginArtNoToc}[0]{\begin{document}}
\newcommand{\EndNoBibArticle}[0]{\end{document}}
\newcommand{\EndArticle}[0]{\bibliography{Bibliography}\bibliographystyle{plainnat}\end{document}}

% 
%\newcommand{\citep}[1]{\cite{#1}}

\colorSectionsForArticle



\beginArtNoToc

\generatetitle{exponential solutions to second order linear system}
%\chapter{exponential solutions to second order linear system}
%\label{chap:threeSpringLoop}
\section{Motivation}

We're discussing specific forms to systems of coupled linear differential equations, such as a loop of ``spring'' connected masses (i.e. atoms interacting with harmonic oscillator potentials) as sketched in \cref{fig:threeSpringLoop:threeSpringLoopFig1}.

\imageFigure{threeSpringLoopFig1}{Three springs loop}{fig:threeSpringLoop:threeSpringLoopFig1}{0.3}

Instead of assuming a solution, let's see how far we can get attacking this problem systematically.

\section{Guts}

Suppose that we have a set of $N$ masses constrained to a circle interacting with harmonic potentials.  The Lagrangian for such a system (using modulo $N$ indexing) is

\begin{dmath}\label{eqn:threeSpringLoop:60}
\LL = \inv{2} \sum_{k = 0}^{2} m_k \dot{u}_k^2 
- \inv{2} \sum_{k = 0}^2 \kappa_k \lr{ u_{k+1} - u_k }^2.
\end{dmath}

The force equations follow directly from the Euler-Lagrange equations

\begin{dmath}\label{eqn:condensedMatterProblemSet4Problem2:40}
0 = \ddt{} \PD{\dot{u}_{n, \alpha}}{\LL}
- \PD{u_{n, \alpha}}{\LL}.
\end{dmath}

For the simple three particle system depicted above, this is

\begin{dmath}\label{eqn:threeSpringLoop:80}
\LL = 
\inv{2} m_0 \dot{u}_0^2 
+\inv{2} m_1 \dot{u}_1^2 
+\inv{2} m_2 \dot{u}_2^2 
- \inv{2} \kappa_0 \lr{ u_1 - u_0 }^2
- \inv{2} \kappa_1 \lr{ u_2 - u_1 }^2
- \inv{2} \kappa_2 \lr{ u_0 - u_2 }^2,
\end{dmath}

with equations of motion
\begin{dmath}\label{eqn:threeSpringLoop:100}
\begin{aligned}
m_0 \ddot{u}_0 &= \kappa_0 \lr{ u_0 - u_1 } + \kappa_2 \lr{ u_0 - u_2 } \\
m_1 \ddot{u}_1 &= \kappa_1 \lr{ u_1 - u_2 } + \kappa_0 \lr{ u_1 - u_0 } \\
m_2 \ddot{u}_2 &= \kappa_2 \lr{ u_2 - u_0 } + \kappa_1 \lr{ u_2 - u_1 },
\end{aligned}
\end{dmath}

Let's partially non-dimensionalize this.  First introduce average mass $\overbar{m}$ and spring constants $\overbar{\kappa}$, and rearrange slightly

\begin{dmath}\label{eqn:threeSpringLoop:120}
\begin{aligned}
\frac{\overbar{m}}{\overbar{k}} \ddot{u}_0 &= \frac{\kappa_0 \overbar{m}}{\overbar{k} m_0} \lr{ u_0 - u_1 } + \frac{\kappa_2 \overbar{m}}{\overbar{k} m_0} \lr{ u_0 - u_2 } \\
\frac{\overbar{m}}{\overbar{k}} \ddot{u}_1 &= \frac{\kappa_1 \overbar{m}}{\overbar{k} m_1} \lr{ u_1 - u_2 } + \frac{\kappa_0 \overbar{m}}{\overbar{k} m_1} \lr{ u_1 - u_0 } \\
\frac{\overbar{m}}{\overbar{k}} \ddot{u}_2 &= \frac{\kappa_2 \overbar{m}}{\overbar{k} m_2} \lr{ u_2 - u_0 } + \frac{\kappa_1 \overbar{m}}{\overbar{k} m_2} \lr{ u_2 - u_1 }.
\end{aligned}
\end{dmath}

With 

\begin{subequations}
\begin{dmath}\label{eqn:threeSpringLoop:140}
\tau = \sqrt{\frac{\overbar{k}}{\overbar{m}}} t
\end{dmath}
\begin{dmath}\label{eqn:threeSpringLoop:160}
\Bu = 
\begin{bmatrix}
u_0 \\
u_1 \\
u_2
\end{bmatrix}
\end{dmath}
\begin{dmath}\label{eqn:threeSpringLoop:180}
B = 
\begin{bmatrix}
\frac{\kappa_0 \overbar{m}}{\overbar{k} m_0} + \frac{\kappa_2 \overbar{m}}{\overbar{k} m_0} &
-\frac{\kappa_0 \overbar{m}}{\overbar{k} m_0} &
-\frac{\kappa_2 \overbar{m}}{\overbar{k} m_0} \\
- \frac{\kappa_0 \overbar{m}}{\overbar{k} m_1} &
\frac{\kappa_1 \overbar{m}}{\overbar{k} m_1} + \frac{\kappa_0 \overbar{m}}{\overbar{k} m_1} &
-\frac{\kappa_1 \overbar{m}}{\overbar{k} m_1}  \\
-\frac{\kappa_2 \overbar{m}}{\overbar{k} m_2} &
- \frac{\kappa_1 \overbar{m}}{\overbar{k} m_2} &
\frac{\kappa_2 \overbar{m}}{\overbar{k} m_2} + \frac{\kappa_1 \overbar{m}}{\overbar{k} m_2} 
\end{bmatrix}.
\end{dmath}
\end{subequations}

Our system takes the form

\begin{dmath}\label{eqn:threeSpringLoop:200}
\frac{d^2 \Bu}{d\tau^2} = B \Bu.
\end{dmath}

We can at least theoretically solve this in a simple fashion if we first convert it to a first order system.  We can do that by augmenting our vector of displacements with their first derivatives

\begin{dmath}\label{eqn:threeSpringLoop:220}
\Bw =
\begin{bmatrix}
\Bu \\
\frac{d \Bu}{d\tau} 
\end{bmatrix},
\end{dmath}

So that 
\begin{equation}\label{eqn:threeSpringLoop:240}
\frac{d \Bw}{d\tau} =
\begin{bmatrix}
0 & I \\
B & 0
\end{bmatrix} 
\Bw
= A \Bw.
\end{equation}

Now the solution is conceptually trivial

\begin{dmath}\label{eqn:threeSpringLoop:260}
\Bw = e^{A \tau} \Bw_0,
\end{dmath}

however, we are faced with the task of exponentiating the matrix $A$.  We need the powers of this matrix $A$

\begin{subequations}
\begin{equation}\label{eqn:threeSpringLoop:280}
{\begin{bmatrix}
0 & I \\
B & 0
\end{bmatrix} }^2
=
\begin{bmatrix}
0 & I \\
B & 0
\end{bmatrix} 
\begin{bmatrix}
0 & I \\
B & 0
\end{bmatrix} 
=
\begin{bmatrix}
B & 0 \\
0 & B
\end{bmatrix} 
\end{equation}
\begin{equation}\label{eqn:threeSpringLoop:300}
{\begin{bmatrix}
0 & I \\
B & 0
\end{bmatrix} }^3
=
\begin{bmatrix}
B & 0 \\
0 & B
\end{bmatrix} 
\begin{bmatrix}
0 & I \\
B & 0
\end{bmatrix} 
=
\begin{bmatrix}
0 & B \\
B^2 & 0
\end{bmatrix} 
\end{equation}
\begin{equation}\label{eqn:threeSpringLoop:320}
{\begin{bmatrix}
0 & I \\
B & 0
\end{bmatrix} }^4
=
\begin{bmatrix}
0 & B \\
B^2 & 0
\end{bmatrix} 
\begin{bmatrix}
0 & I \\
B & 0
\end{bmatrix} 
=
\begin{bmatrix}
B^2 & 0 \\
0 & B^2
\end{bmatrix},
\end{equation}
\end{subequations}

allowing us to write out the matrix exponential
\begin{dmath}\label{eqn:threeSpringLoop:340}
e^{A \tau} = 
\sum_{k = 0}^\infty \frac{\tau^{2k}}{(2k)!} 
\begin{bmatrix}
B^k & 0 \\
0 & B^k
\end{bmatrix}
+
\sum_{k = 0}^\infty \frac{\tau^{2k + 1}}{(2k + 1)!} 
\begin{bmatrix}
0 & B^k \\
B^{k+1} & 0
\end{bmatrix}.
\end{dmath}

We can factor the odd powers of $A$ as
\begin{dmath}\label{eqn:threeSpringLoop:360}
\begin{bmatrix}
0 & B^k \\
B^{k+1} & 0
\end{bmatrix}
=
\begin{bmatrix}
0 & B^{-1/2} \\
B^{1/2} & 0
\end{bmatrix}
\begin{bmatrix}
B^{k + 1/2} & 0 \\
0 & B^{k + 1/2}
\end{bmatrix},
\end{dmath}

so that

\begin{dmath}\label{eqn:threeSpringLoop:380}
e^{A \tau} = 
\sum_{k = 0}^\infty \frac{\tau^{2k}}{(2k)!} 
\begin{bmatrix}
\sqrt{B}^{2k} & 0 \\
0 & \sqrt{B}^{2k}
\end{bmatrix}
+
\begin{bmatrix}
0 & B^{-1/2} \\
B^{1/2} & 0
\end{bmatrix}
\sum_{k = 0}^\infty \frac{\tau^{2k + 1}}{(2k + 1)!} 
\begin{bmatrix}
\sqrt{B}^{2 k + 1} & 0 \\
0 & \sqrt{B}^{2 k+1} 
\end{bmatrix}
=
\cosh
\begin{bmatrix}
\sqrt{B} \tau & 0 \\
0 & \sqrt{B} \tau
\end{bmatrix}
+ 
\begin{bmatrix}
0 & 1/\sqrt{B} \tau \\
\sqrt{B} \tau & 0
\end{bmatrix}
\sinh
\begin{bmatrix}
\sqrt{B} \tau & 0 \\
0 & \sqrt{B} \tau
\end{bmatrix}.
\end{dmath}

This is
\begin{dmath}\label{eqn:threeSpringLoop:400}
e^{A \tau}
=
\begin{bmatrix}
\cosh \sqrt{B} \tau & (1/\sqrt{B}) \sinh \sqrt{B} \tau \\
\sqrt{B} \sinh \sqrt{B} \tau & \cosh \sqrt{B} \tau 
\end{bmatrix}.
\end{dmath}

The solution, written out is

\begin{dmath}\label{eqn:threeSpringLoop:420}
\begin{bmatrix}
\Bu \\
\Bu'
\end{bmatrix}
=
\begin{bmatrix}
\cosh \sqrt{B} \tau & (1/\sqrt{B}) \sinh \sqrt{B} \tau \\
\sqrt{B} \sinh \sqrt{B} \tau & \cosh \sqrt{B} \tau 
\end{bmatrix}
\begin{bmatrix}
\Bu_0 \\
\Bu_0'
\end{bmatrix},
\end{dmath}

so that 

\begin{dmath}\label{eqn:threeSpringLoop:440}
\Bu = \cosh \sqrt{B} \tau \Bu_0 + \inv{\sqrt{B}} \sinh \sqrt{B} \tau \Bu_0'.
\end{dmath}

As a check differentiation twice shows that this is in fact the general solution, since we have
\begin{dmath}\label{eqn:threeSpringLoop:460}
\Bu' = \sqrt{B} \sinh \sqrt{B} \tau \Bu_0 + \sinh \sqrt{B} \tau \Bu_0',
\end{dmath}

and
\begin{dmath}\label{eqn:threeSpringLoop:480}
\Bu'' = B \cosh \sqrt{B} \tau \Bu_0 + \sqrt{B} \cosh \sqrt{B} \tau \Bu_0'
= B \lr{ \cosh \sqrt{B} \tau \Bu_0 + \inv{\sqrt{B}} \cosh \sqrt{B} \tau \Bu_0' }
= B \Bu.
\end{dmath}

%\EndArticle
\EndNoBibArticle
