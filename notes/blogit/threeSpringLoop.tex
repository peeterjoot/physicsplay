%
% Copyright � 2013 Peeter Joot.  All Rights Reserved.
% Licenced as described in the file LICENSE under the root directory of this GIT repository.
%
\newcommand{\authorname}{Peeter Joot}
\newcommand{\email}{peeterjoot@protonmail.com}
\newcommand{\basename}{FIXMEbasenameUndefined}
\newcommand{\dirname}{notes/FIXMEdirnameUndefined/}

\renewcommand{\basename}{threeSpringLoop}
\renewcommand{\dirname}{notes/phy487/}
%\newcommand{\dateintitle}{}
%\newcommand{\keywords}{}

\newcommand{\authorname}{Peeter Joot}
\newcommand{\onlineurl}{http://sites.google.com/site/peeterjoot2/math2013/\basename.pdf}
\newcommand{\sourcepath}{\dirname\basename.tex}
\newcommand{\generatetitle}[1]{\chapter{#1}}

\newcommand{\vcsinfo}{%
\section*{}
\noindent{\color{DarkOliveGreen}{\rule{\linewidth}{0.1mm}}}
\paragraph{Document version}
%\paragraph{\color{Maroon}{Document version}}
{
\small
\begin{itemize}
\item Available online at:\\ 
\href{\onlineurl}{\onlineurl}
\item Git Repository: \input{./.revinfo/gitRepo.tex}
\item Source: \sourcepath
\item last commit: \input{./.revinfo/gitCommitString.tex}
\item commit date: \input{./.revinfo/gitCommitDate.tex}
\end{itemize}
}
}

%\PassOptionsToPackage{dvipsnames,svgnames}{xcolor}
\PassOptionsToPackage{square,numbers}{natbib}
\documentclass{scrreprt}

\usepackage[left=2cm,right=2cm]{geometry}
\usepackage[svgnames]{xcolor}
\usepackage{peeters_layout}

\usepackage{natbib}

\usepackage[
colorlinks=true,
bookmarks=false,
pdfauthor={\authorname, \email},
backref 
]{hyperref}

% http://tex.stackexchange.com/questions/75773/how-to-reference-problems-by-the-text-label-in-an-exercise-envioronment
\usepackage[english]{cleveref}
\crefname{Exercise}{exercise}{exercises}
\Crefname{Exercise}{Exercise}{Exercises}

\RequirePackage{titlesec}
\RequirePackage{ifthen}

% http://stackoverflow.com/questions/4932910/date-in-the-tabular-environment
\makeatletter
\let\insertdate\@date
\makeatother

\titleformat{\chapter}[display]
{\bfseries\Large}
{\color{DarkSlateGrey}\filleft \authorname
\ifthenelse{\isundefined{\studentnumber}}{}{\\ \studentnumber}
\ifthenelse{\isundefined{\email}}{}{\\ \email}
\ifthenelse{\isundefined{\dateintitle}}{}{\\ \insertdate}
%\ifthenelse{\isundefined{\coursename}}{}{\\ \coursename} % put in title instead.
}
{4ex}
{\color{DarkOliveGreen}{\titlerule}\color{Maroon}
\vspace{2ex}%
\filright}
[\vspace{2ex}%
\color{DarkOliveGreen}\titlerule
]

\newcommand{\beginArtWithToc}[0]{\begin{document}\tableofcontents}
\newcommand{\beginArtNoToc}[0]{\begin{document}}
\newcommand{\EndNoBibArticle}[0]{\end{document}}
\newcommand{\EndArticle}[0]{\bibliography{Bibliography}\bibliographystyle{plainnat}\end{document}}

% 
%\newcommand{\citep}[1]{\cite{#1}}

\colorSectionsForArticle



\newcommand{\nought}[0]{\circ}

\beginArtNoToc

\generatetitle{Exponential solutions to second order linear system}
%\chapter{Exponential solutions to second order linear system}
%\label{chap:threeSpringLoop}
\section{Motivation}

We're discussing specific forms to systems of coupled linear differential equations, such as a loop of ``spring'' connected masses (i.e. atoms interacting with harmonic oscillator potentials) as sketched in \cref{fig:threeSpringLoop:threeSpringLoopFig1}.

\imageFigure{threeSpringLoopFig1}{Three springs loop}{fig:threeSpringLoop:threeSpringLoopFig1}{0.3}

Instead of assuming a solution, let's see how far we can get attacking this problem systematically.

\section{Matrix methods}

Suppose that we have a set of $N$ masses constrained to a circle interacting with harmonic potentials.  The Lagrangian for such a system (using modulo $N$ indexing) is

\begin{dmath}\label{eqn:threeSpringLoop:60}
\LL = \inv{2} \sum_{k = 0}^{2} m_k \dot{u}_k^2 
- \inv{2} \sum_{k = 0}^2 \kappa_k \lr{ u_{k+1} - u_k }^2.
\end{dmath}

The force equations follow directly from the Euler-Lagrange equations

\begin{dmath}\label{eqn:condensedMatterProblemSet4Problem2:40}
0 = \ddt{} \PD{\dot{u}_{n, \alpha}}{\LL}
- \PD{u_{n, \alpha}}{\LL}.
\end{dmath}

For the simple three particle system depicted above, this is

\begin{dmath}\label{eqn:threeSpringLoop:80}
\LL = 
\inv{2} m_0 \dot{u}_0^2 
+\inv{2} m_1 \dot{u}_1^2 
+\inv{2} m_2 \dot{u}_2^2 
- \inv{2} \kappa_0 \lr{ u_1 - u_0 }^2
- \inv{2} \kappa_1 \lr{ u_2 - u_1 }^2
- \inv{2} \kappa_2 \lr{ u_0 - u_2 }^2,
\end{dmath}

with equations of motion
\begin{dmath}\label{eqn:threeSpringLoop:100}
\begin{aligned}
0 &= m_0 \ddot{u}_0 + \kappa_0 \lr{ u_0 - u_1 } + \kappa_2 \lr{ u_0 - u_2 } \\
0 &= m_1 \ddot{u}_1 + \kappa_1 \lr{ u_1 - u_2 } + \kappa_0 \lr{ u_1 - u_0 } \\
0 &= m_2 \ddot{u}_2 + \kappa_2 \lr{ u_2 - u_0 } + \kappa_1 \lr{ u_2 - u_1 },
\end{aligned}
\end{dmath}

Let's partially non-dimensionalize this.  First introduce average mass $\overbar{m}$ and spring constants $\overbar{\kappa}$, and rearrange slightly

\begin{dmath}\label{eqn:threeSpringLoop:120}
\begin{aligned}
\frac{\overbar{m}}{\overbar{k}} \ddot{u}_0 &= -\frac{\kappa_0 \overbar{m}}{\overbar{k} m_0} \lr{ u_0 - u_1 } - \frac{\kappa_2 \overbar{m}}{\overbar{k} m_0} \lr{ u_0 - u_2 } \\
\frac{\overbar{m}}{\overbar{k}} \ddot{u}_1 &= -\frac{\kappa_1 \overbar{m}}{\overbar{k} m_1} \lr{ u_1 - u_2 } - \frac{\kappa_0 \overbar{m}}{\overbar{k} m_1} \lr{ u_1 - u_0 } \\
\frac{\overbar{m}}{\overbar{k}} \ddot{u}_2 &= -\frac{\kappa_2 \overbar{m}}{\overbar{k} m_2} \lr{ u_2 - u_0 } - \frac{\kappa_1 \overbar{m}}{\overbar{k} m_2} \lr{ u_2 - u_1 }.
\end{aligned}
\end{dmath}

With 

\begin{subequations}
\begin{dmath}\label{eqn:threeSpringLoop:140}
\tau = \sqrt{\frac{\overbar{k}}{\overbar{m}}} t = \Omega t
\end{dmath}
\begin{dmath}\label{eqn:threeSpringLoop:160}
\Bu = 
\begin{bmatrix}
u_0 \\
u_1 \\
u_2
\end{bmatrix}
\end{dmath}
\begin{dmath}\label{eqn:threeSpringLoop:180}
B = 
\begin{bmatrix}
-\frac{\kappa_0 \overbar{m}}{\overbar{k} m_0} - \frac{\kappa_2 \overbar{m}}{\overbar{k} m_0} &
\frac{\kappa_0 \overbar{m}}{\overbar{k} m_0} &
\frac{\kappa_2 \overbar{m}}{\overbar{k} m_0} \\
\frac{\kappa_0 \overbar{m}}{\overbar{k} m_1} &
-\frac{\kappa_1 \overbar{m}}{\overbar{k} m_1} - \frac{\kappa_0 \overbar{m}}{\overbar{k} m_1} &
\frac{\kappa_1 \overbar{m}}{\overbar{k} m_1}  \\
\frac{\kappa_2 \overbar{m}}{\overbar{k} m_2} &
 \frac{\kappa_1 \overbar{m}}{\overbar{k} m_2} &
-\frac{\kappa_2 \overbar{m}}{\overbar{k} m_2} - \frac{\kappa_1 \overbar{m}}{\overbar{k} m_2} 
\end{bmatrix}.
\end{dmath}
\end{subequations}

Our system takes the form

\begin{dmath}\label{eqn:threeSpringLoop:200}
\frac{d^2 \Bu}{d\tau^2} = B \Bu.
\end{dmath}

We can at least theoretically solve this in a simple fashion if we first convert it to a first order system.  We can do that by augmenting our vector of displacements with their first derivatives

\begin{dmath}\label{eqn:threeSpringLoop:220}
\Bw =
\begin{bmatrix}
\Bu \\
\frac{d \Bu}{d\tau} 
\end{bmatrix},
\end{dmath}

So that 
\begin{equation}\label{eqn:threeSpringLoop:240}
\frac{d \Bw}{d\tau} =
\begin{bmatrix}
0 & I \\
B & 0
\end{bmatrix} 
\Bw
= A \Bw.
\end{equation}

Now the solution is conceptually trivial

\begin{dmath}\label{eqn:threeSpringLoop:260}
\Bw = e^{A \tau} \Bw_\nought.
\end{dmath}

We are however, faced with the task of exponentiating the matrix $A$.  All the powers of this matrix $A$ will be required, but they turn out to be easy to calculate

\begin{subequations}
\begin{equation}\label{eqn:threeSpringLoop:280}
{\begin{bmatrix}
0 & I \\
B & 0
\end{bmatrix} }^2
=
\begin{bmatrix}
0 & I \\
B & 0
\end{bmatrix} 
\begin{bmatrix}
0 & I \\
B & 0
\end{bmatrix} 
=
\begin{bmatrix}
B & 0 \\
0 & B
\end{bmatrix} 
\end{equation}
\begin{equation}\label{eqn:threeSpringLoop:300}
{\begin{bmatrix}
0 & I \\
B & 0
\end{bmatrix} }^3
=
\begin{bmatrix}
B & 0 \\
0 & B
\end{bmatrix} 
\begin{bmatrix}
0 & I \\
B & 0
\end{bmatrix} 
=
\begin{bmatrix}
0 & B \\
B^2 & 0
\end{bmatrix} 
\end{equation}
\begin{equation}\label{eqn:threeSpringLoop:320}
{\begin{bmatrix}
0 & I \\
B & 0
\end{bmatrix} }^4
=
\begin{bmatrix}
0 & B \\
B^2 & 0
\end{bmatrix} 
\begin{bmatrix}
0 & I \\
B & 0
\end{bmatrix} 
=
\begin{bmatrix}
B^2 & 0 \\
0 & B^2
\end{bmatrix},
\end{equation}
\end{subequations}

allowing us to write out the matrix exponential
\begin{dmath}\label{eqn:threeSpringLoop:340}
e^{A \tau} = 
\sum_{k = 0}^\infty \frac{\tau^{2k}}{(2k)!} 
\begin{bmatrix}
B^k & 0 \\
0 & B^k
\end{bmatrix}
+
\sum_{k = 0}^\infty \frac{\tau^{2k + 1}}{(2k + 1)!} 
\begin{bmatrix}
0 & B^k \\
B^{k+1} & 0
\end{bmatrix}.
\end{dmath}

\paragraph{Case I: No zero eigenvalues}

Provided that $B$ has no zero eigenvalues, we could factor this as

\begin{dmath}\label{eqn:threeSpringLoop:360}
\begin{bmatrix}
0 & B^k \\
B^{k+1} & 0
\end{bmatrix}
=
\begin{bmatrix}
0 & B^{-1/2} \\
B^{1/2} & 0
\end{bmatrix}
\begin{bmatrix}
B^{k + 1/2} & 0 \\
0 & B^{k + 1/2}
\end{bmatrix},
\end{dmath}

This initially leads us to believe the following, but we'll find out that the three springs interaction matrix $B$ does have a zero eigenvalue, and we'll have to be more careful.  If there were any such interaction matrices that did not have such a zero we could simply write

\begin{dmath}\label{eqn:threeSpringLoop:380}
e^{A \tau} = 
\sum_{k = 0}^\infty \frac{\tau^{2k}}{(2k)!} 
\begin{bmatrix}
\sqrt{B}^{2k} & 0 \\
0 & \sqrt{B}^{2k}
\end{bmatrix}
+
\begin{bmatrix}
0 & B^{-1/2} \\
B^{1/2} & 0
\end{bmatrix}
\sum_{k = 0}^\infty \frac{\tau^{2k + 1}}{(2k + 1)!} 
\begin{bmatrix}
\sqrt{B}^{2 k + 1} & 0 \\
0 & \sqrt{B}^{2 k+1} 
\end{bmatrix}
=
\cosh
\begin{bmatrix}
\sqrt{B} \tau & 0 \\
0 & \sqrt{B} \tau
\end{bmatrix}
+ 
\begin{bmatrix}
0 & 1/\sqrt{B} \tau \\
\sqrt{B} \tau & 0
\end{bmatrix}
\sinh
\begin{bmatrix}
\sqrt{B} \tau & 0 \\
0 & \sqrt{B} \tau
\end{bmatrix}.
\end{dmath}

This is
\begin{dmath}\label{eqn:threeSpringLoop:400}
e^{A \tau}
=
\begin{bmatrix}
\cosh \sqrt{B} \tau & (1/\sqrt{B}) \sinh \sqrt{B} \tau \\
\sqrt{B} \sinh \sqrt{B} \tau & \cosh \sqrt{B} \tau 
\end{bmatrix}.
\end{dmath}

The solution, written out is

\begin{dmath}\label{eqn:threeSpringLoop:420}
\begin{bmatrix}
\Bu \\
\Bu'
\end{bmatrix}
=
\begin{bmatrix}
\cosh \sqrt{B} \tau & (1/\sqrt{B}) \sinh \sqrt{B} \tau \\
\sqrt{B} \sinh \sqrt{B} \tau & \cosh \sqrt{B} \tau 
\end{bmatrix}
\begin{bmatrix}
\Bu_\nought \\
\Bu_\nought'
\end{bmatrix},
\end{dmath}

so that 

\begin{dmath}\label{eqn:threeSpringLoop:440}
\myBoxed{
\Bu = \cosh \sqrt{B} \tau \Bu_\nought + \inv{\sqrt{B}} \sinh \sqrt{B} \tau \Bu_\nought'.
}
\end{dmath}

As a check differentiation twice shows that this is in fact the general solution, since we have
\begin{dmath}\label{eqn:threeSpringLoop:460}
\Bu' = \sqrt{B} \sinh \sqrt{B} \tau \Bu_\nought + \cosh \sqrt{B} \tau \Bu_\nought',
\end{dmath}

and
\begin{dmath}\label{eqn:threeSpringLoop:480}
\Bu'' = B \cosh \sqrt{B} \tau \Bu_\nought + \sqrt{B} \sinh \sqrt{B} \tau \Bu_\nought'
= B \lr{ \cosh \sqrt{B} \tau \Bu_\nought + \inv{\sqrt{B}} \sinh \sqrt{B} \tau \Bu_\nought' }
= B \Bu.
\end{dmath}

Observe that this solution is a general solution to second order constant coefficient linear systems of the form we have in \eqnref{eqn:threeSpringLoop:120}.  However, to make it meaningful we do have the additional computational task of performing an eigensystem decomposition of the matrix $B$.  We expect negative eigenvalues that will give us oscillatory solutions (ie: the matrix square roots will have imaginary eigenvalues).

\makeexample{An example diagonalization to try things out}{example:threeSpringLoop:1}{

Let's do that diagonalization for the simplest of the three springs system as an example, with $\kappa_j = \overbar{k}$ and $m_j = \overbar{m}$, so that we have

\begin{dmath}\label{eqn:threeSpringLoop:500}
B = 
\begin{bmatrix}
-2 & 1 & 1 \\
1 & -2 & 1 \\
1 & 1 & -2 
\end{bmatrix}.
\end{dmath}

% From \nbref{periodicComplexExponentialPV.nb} (using Eigensystem and manual Gram-Schmidt on the degenerate eigenvectors) we find that a
A orthonormal eigensystem for $B$ is 

\begin{dmath}\label{eqn:threeSpringLoop:520}
\left\{
\Be_{-3, 1}, 
\Be_{-3, 2}, 
\Be_{0, 1} 
\right\}
=
\left\{
\inv{\sqrt{6}}
\begin{bmatrix}
 -1  \\
 -1  \\
 2 
\end{bmatrix},
\inv{\sqrt{2}}
\begin{bmatrix}
 -1  \\
 1  \\
 0 
\end{bmatrix},
\inv{\sqrt{3}}
\begin{bmatrix}
 1 \\
 1 \\
 1 
\end{bmatrix}
\right\}.
\end{dmath}

With

\begin{dmath}\label{eqn:threeSpringLoop:540}
\begin{aligned}
U &= 
\inv{\sqrt{6}}
\begin{bmatrix}
 -\sqrt{3} & -1 & \sqrt{2} \\
 0 & 2 & \sqrt{2} \\
 \sqrt{3} & -1 & \sqrt{2} 
\end{bmatrix} \\
D &= 
\begin{bmatrix}
-3 & 0 & 0 \\
0 & -3 & 0 \\
0 & 0  & 0 \\
\end{bmatrix}
\end{aligned}
\end{dmath}

We have

\begin{dmath}\label{eqn:threeSpringLoop:560}
B = U D U^\T,
\end{dmath}

We also find that $B$ and its root are intimately related in a surprising way

\begin{dmath}\label{eqn:threeSpringLoop:580}
\sqrt{B} = 
\sqrt{3} i
U 
\begin{bmatrix}
1 & 0 & 0 \\
0 & 1 & 0 \\
0 & 0  & 0 \\
\end{bmatrix}
U^\T
=
\inv{\sqrt{3} i} B.
\end{dmath}

We also see, unfortunately that $B$ has a zero eigenvalue, so we can't compute $1/\sqrt{B}$.  We'll have to back and up and start again differently.
}

\paragraph{Case II: allowing for zero eigenvalues}

Now that we realize we have to deal with zero eigenvalues, a different approach suggests itself.  Instead of reducing our system using a Hamiltonian transformation to a first order system, let's utilize that diagonalization directly.  Our system is 

\begin{dmath}\label{eqn:threeSpringLoop:600}
\Bu'' = B \Bu = U D U^{-1} \Bu,
\end{dmath}

where $D = [ \lambda_i \delta_{ij} ]$ and

\begin{dmath}\label{eqn:threeSpringLoop:620}
\lr{ U^{-1} \Bu}'' = D \lr{ U^{-1} \Bu}.
\end{dmath}

Let 

\begin{dmath}\label{eqn:threeSpringLoop:720}
\Bz = U^{-1} \Bu,
\end{dmath}

so that our system is just

\begin{dmath}\label{eqn:threeSpringLoop:700}
\Bz'' = D \Bz,
\end{dmath}

or

\begin{dmath}\label{eqn:threeSpringLoop:640}
z_i'' = \lambda_i z_i.
\end{dmath}

This is $N$ equations, each decoupled and solvable by inspection.  Suppose we group the eigenvalues into sets $\{ \lambda_n < 0, \lambda_p > 0, \lambda_z = 0 \}$.  Our solution is then

\begin{dmath}\label{eqn:threeSpringLoop:660}
\Bz 
= 
\sum_{ \lambda_n < 0, \lambda_p > 0, \lambda_z = 0}
\lr{
a_n \cos \sqrt{-\lambda_n} \tau 
+ 
b_n \sin \sqrt{-\lambda_n} \tau 
}
\Be_n
+
\lr{
a_p \cosh \sqrt{\lambda_p} \tau 
+ 
b_p \sinh \sqrt{\lambda_p} \tau 
}
\Be_p
+
\lr{
a_z + b_z \tau
} \Be_z.
\end{dmath}

Transforming back to lattice coordinates using $\Bu = U \Bz$, we have

\begin{dmath}\label{eqn:threeSpringLoop:740}
\Bu 
= 
\sum_{ \lambda_n < 0, \lambda_p > 0, \lambda_z = 0}
\lr{
a_n \cos \sqrt{-\lambda_n} \tau 
+ 
b_n \sin \sqrt{-\lambda_n} \tau 
}
U \Be_n
+
\lr{
a_p \cosh \sqrt{\lambda_p} \tau 
+ 
b_p \sinh \sqrt{\lambda_p} \tau 
}
U \Be_p
+
\lr{
a_z + b_z \tau
} U \Be_z.
\end{dmath}

We see that the zero eigenvalues integration terms have no contribution to the lattice coordinates, since $U \Be_z = \lambda_z \Be_z = 0$, for all $\lambda_z = 0$.

If $U = [ \Be_i ]$ are a set of not neccessarily orthonormal eigenvectors for $B$, then the vectors $\Bf_i$, where $\Be_i \cdot \Bf_j = \delta_{ij}$ are the reciprocal frame vectors.  These can be extracted from $U^{-1} = [ \Bf_i ]^\T$ (i.e., the rows of $U^{-1}$).  Taking dot products between $\Bf_i$ with $\Bu(0) = \Bu_\nought$ and $\Bu'(0) = \Bu'_\nought$, provides us with the unknown coefficients $a_n, b_n$

\begin{dmath}\label{eqn:threeSpringLoop:680}
\Bu(\tau)
= 
\sum_{ \lambda_n < 0, \lambda_p > 0 }
\lr{
(\Bu_\nought \cdot \Bf_n) 
\cos \sqrt{-\lambda_n} \tau 
+ 
\frac{\Bu_\nought' \cdot \Bf_n}{\sqrt{-\lambda_n} } 
\sin \sqrt{-\lambda_n} \tau 
}
\Be_n
+
\lr{
(\Bu_\nought \cdot \Bf_p) 
\cosh \sqrt{\lambda_p} \tau 
+ 
\frac{\Bu_\nought' \cdot \Bf_p}{\sqrt{-\lambda_p} } 
\sinh \sqrt{\lambda_p} \tau 
}
\Be_p.
\end{dmath}

Supposing that we constrain ourself to looking at just the oscillatory solutions (i.e. the lattice does not shake itself to pieces), then we have

\begin{dmath}\label{eqn:threeSpringLoop:680b}
\myBoxed{
\Bu(\tau)
= 
\sum_{ \lambda_n < 0 }
\lr{
\sum_j
\Be_{n,j} \Bf_{n,j}^\T
}
\lr{
\Bu_\nought
\cos \sqrt{-\lambda_n} \tau 
+ 
\frac{\Bu_\nought' }{\sqrt{-\lambda_n} } 
\sin \sqrt{-\lambda_n} \tau 
}.
}
\end{dmath}

Eigenvectors for eigenvalues that were degenerate have been explicitly enumerated here, something previously implied.  Observe that the dot products of the form $(\Ba \cdot \Bf_i) \Be_i$ have been put into projector operator form to group terms more nicely.  The solution can be thought of as a weighted projector operator working as a time evolution operator from the initial state.

\makeexample{Our example interaction revisited}{example:threeSpringLoop:2}{

Recall that we had an orthonormal basis for the $\lambda = -3$ eigensubspace for the interaction example of \eqnref{eqn:threeSpringLoop:500} again, so $\Be_{-3,i} = \Bf_{-3, i}$.  We can sum $\Be_{-3,1} \Be_{-3, 1}^\T + \Be_{-3,2} \Be_{-3, 2}^\T$ to find

\begin{dmath}\label{eqn:threeSpringLoop:760}
\Bu(\tau)
= 
\inv{3}
\begin{bmatrix}
2 & -1 & -1 \\
-1 & 2 & -1 \\
-1 & -1 & 2
\end{bmatrix}
\lr{
\Bu_\nought 
\cos \sqrt{3} \tau 
+ 
\frac{ \Bu_\nought' }{\sqrt{3} } 
\sin \sqrt{3} \tau 
}.
\end{dmath}

The leading matrix is an orthonormal projector of the initial conditions onto the eigen subspace for $\lambda_n = -3$.  Observe that this is proportional to $B$ itself, scaled by the square of the non-zero eigenvalue of $\sqrt{B}$.  From this we can confirm by inspection that this is a solution to $\Bu'' = B \Bu$, as desired.
}

\section{Fourier transform methods}

%\EndArticle
\EndNoBibArticle
