%
% Copyright � 2013 Peeter Joot.  All Rights Reserved.
% Licenced as described in the file LICENSE under the root directory of this GIT repository.
%
\makeproblem{The Madelung constant (Ibach and Luth, Q1, Chapter 1)}{condensedMatter:problemSet1:3}{ 

\makesubproblem{Calculate the Madelung constant $A$ for a linear ionic chain.}{condensedMatter:problemSet1:3a}

\makesubproblem{Approximate numerical calculations for the \ce{NaCl} lattice.}{condensedMatter:problemSet1:3b}

Make approximate numerical calculations (on a computer) 
for the \ce{NaCl} lattice.  First use a cubic geometry 
in which $2ma$ is the cube side-length, with $a$ the 
separation of nearest neighbours, and second a spherical 
geometry where $ma$ is the radius of the sphere. Carry out 
the calculation for $m$ values of 97, 98 and 99, and compare 
the results.  Please submit your code for 
this question, as well as a discussion of why calculations in 
the two geometries behave so differently. 


} % makeproblem

\makeanswer{condensedMatter:problemSet1:3}{ 

\makeSubAnswer{XXX}{condensedMatter:problemSet1:3a}

TODO.

\makeSubAnswer{XXX}{condensedMatter:problemSet1:3b}

TODO.
}
