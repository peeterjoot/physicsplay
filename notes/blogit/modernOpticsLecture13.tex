%
% Copyright � 2012 Peeter Joot.  All Rights Reserved.
% Licenced as described in the file LICENSE under the root directory of this GIT repository.
%
\newcommand{\authorname}{Peeter Joot}
\newcommand{\email}{peeterjoot@protonmail.com}
\newcommand{\basename}{FIXMEbasenameUndefined}
\newcommand{\dirname}{notes/FIXMEdirnameUndefined/}

\renewcommand{\basename}{modernOpticsLecture13}
\renewcommand{\dirname}{notes/phy485/}
\newcommand{\keywords}{Optics, PHY485H1F}
\newcommand{\authorname}{Peeter Joot}
\newcommand{\onlineurl}{http://sites.google.com/site/peeterjoot2/math2013/\basename.pdf}
\newcommand{\sourcepath}{\dirname\basename.tex}
\newcommand{\generatetitle}[1]{\chapter{#1}}

\newcommand{\vcsinfo}{%
\section*{}
\noindent{\color{DarkOliveGreen}{\rule{\linewidth}{0.1mm}}}
\paragraph{Document version}
%\paragraph{\color{Maroon}{Document version}}
{
\small
\begin{itemize}
\item Available online at:\\ 
\href{\onlineurl}{\onlineurl}
\item Git Repository: \input{./.revinfo/gitRepo.tex}
\item Source: \sourcepath
\item last commit: \input{./.revinfo/gitCommitString.tex}
\item commit date: \input{./.revinfo/gitCommitDate.tex}
\end{itemize}
}
}

%\PassOptionsToPackage{dvipsnames,svgnames}{xcolor}
\PassOptionsToPackage{square,numbers}{natbib}
\documentclass{scrreprt}

\usepackage[left=2cm,right=2cm]{geometry}
\usepackage[svgnames]{xcolor}
\usepackage{peeters_layout}

\usepackage{natbib}

\usepackage[
colorlinks=true,
bookmarks=false,
pdfauthor={\authorname, \email},
backref 
]{hyperref}

% http://tex.stackexchange.com/questions/75773/how-to-reference-problems-by-the-text-label-in-an-exercise-envioronment
\usepackage[english]{cleveref}
\crefname{Exercise}{exercise}{exercises}
\Crefname{Exercise}{Exercise}{Exercises}

\RequirePackage{titlesec}
\RequirePackage{ifthen}

% http://stackoverflow.com/questions/4932910/date-in-the-tabular-environment
\makeatletter
\let\insertdate\@date
\makeatother

\titleformat{\chapter}[display]
{\bfseries\Large}
{\color{DarkSlateGrey}\filleft \authorname
\ifthenelse{\isundefined{\studentnumber}}{}{\\ \studentnumber}
\ifthenelse{\isundefined{\email}}{}{\\ \email}
\ifthenelse{\isundefined{\dateintitle}}{}{\\ \insertdate}
%\ifthenelse{\isundefined{\coursename}}{}{\\ \coursename} % put in title instead.
}
{4ex}
{\color{DarkOliveGreen}{\titlerule}\color{Maroon}
\vspace{2ex}%
\filright}
[\vspace{2ex}%
\color{DarkOliveGreen}\titlerule
]

\newcommand{\beginArtWithToc}[0]{\begin{document}\tableofcontents}
\newcommand{\beginArtNoToc}[0]{\begin{document}}
\newcommand{\EndNoBibArticle}[0]{\end{document}}
\newcommand{\EndArticle}[0]{\bibliography{Bibliography}\bibliographystyle{plainnat}\end{document}}

% 
%\newcommand{\citep}[1]{\cite{#1}}

\colorSectionsForArticle


\beginArtNoToc
\generatetitle{PHY485H1F Modern Optics.  Lecture 13: Multiple interference.  Taught by Prof.\ Joseph Thywissen}
%\chapter{Multiple interference}
\label{chap:modernOpticsLecture13}

%\section{Disclaimer}
%
%Peeter's lecture notes from class.  May not be entirely coherent.
%
%\section{Multiple interference}

\section{Last time}

Found 

\begin{subequations}
\begin{dmath}\label{eqn:modernOpticsLecture13:20}
I_t = \frac{I_{\text{max}}}{1 + F \sin^2 \Delta/2}
\end{dmath}
\begin{dmath}\label{eqn:modernOpticsLecture13:40}
F = \frac{4 R}{(1 - R)^2}
\end{dmath}
\begin{dmath}\label{eqn:modernOpticsLecture13:60}
\Delta = 2 \delta_r + \delta
\end{dmath}
\begin{dmath}\label{eqn:modernOpticsLecture13:80}
\delta = 2 L k \cos\theta
\end{dmath}
\end{subequations}

We've got sharp peaks at $\Delta = 2 \pi m$

How good is an etalon at resolving frequency?

Suppose we've shined in two beams of the same frequency, and then slowly start changing the frequency of the other beam, until we get to the point where we've got both peaks centered at $2 \pi m \omega_k$ as in

F3

re-label with

\begin{dmath}\label{eqn:modernOpticsLecture13:100}
2 k_1 L = \frac{ 2 \omega_1 L }{c}
\end{dmath}

F4

We'll consider this ``resolved'' when the second peak is centered at the point when our first peak has lost half of its intensity as in

F5

In mathese, this resolution is

\begin{dmath}\label{eqn:modernOpticsLecture13:120}
\frac{I}{I_{\text{max}}}
= 
\inv{2}
\end{dmath}

at the peak for $\omega_2$.  That is

\begin{dmath}\label{eqn:modernOpticsLecture13:140}
1 + F\sin^2\frac{\Delta_1}{2} = 2
\end{dmath}

\begin{dmath}\label{eqn:modernOpticsLecture13:280}
\sin\frac{\Delta_1}{2} = \inv{\sqrt{F}}
\end{dmath}

\begin{dmath}\label{eqn:modernOpticsLecture13:160}
x = \frac{2}{\sqrt{F}}
\end{dmath}

\begin{dmath}\label{eqn:modernOpticsLecture13:180}
\Delta_1 = 2 \pi m + x 
\end{dmath}

Define, the \underline{Finesse}, as 

\begin{dmath}\label{eqn:modernOpticsLecture13:200}
\FF = \pi \frac{\sqrt{R}}{1 - R} = \frac{\pi}{2} \sqrt{F} \sim \frac{\pi}{T}
\end{dmath}

\begin{dmath}\label{eqn:modernOpticsLecture13:220}
\omega_1 - \omega_2 = \frac{2 c}{L \sqrt{F}}
=\frac{\pi c}{L \FF}
\end{dmath}

\begin{dmath}\label{eqn:modernOpticsLecture13:240}
\frac{\omega_1 - \omega_2}{\overbar{\omega}} = \inv{\FF m}
\end{dmath}

Roughly speaking $\FF$ is an instruction to ``buy good mirrors'', whereas $m$ means ``use a long cavity''

How many reflections?  

\begin{dmath}\label{eqn:modernOpticsLecture13:260}
N \sim \inv{T} \sim \FF
\end{dmath}

$N$-wave interference.

Cavity length is important.  Suppose we had

F6

which gives

\begin{dmath}\label{eqn:modernOpticsLecture13:300}
\Delta = \text{offset} + 2 \frac{\overbar{\omega}}{c} L = 2 \pi (m + j)
\end{dmath}

Neglecting the offset so that

\begin{dmath}\label{eqn:modernOpticsLecture13:320}
2 \frac{\overbar{\omega}}{c} L = 2 \pi (m + j)
\end{dmath}

or

\begin{dmath}\label{eqn:modernOpticsLecture13:340}
\overbar{\omega} = \frac{\pi c}{L} (m + j)
= \omega_0 + j \text{F S R}
\end{dmath}

where $\text{F S R}$ is the \underline{Free Spectral Range}.  Replotting in

F7

This is called a \underline{Fabry-Perot Spectrometer}.  These guys, who were first able to achieve a good spectrometer of this sort, achieved $\FF \sim 30-100$, using $\lambda/100$ flatness, where a typical mirror has $\lambda/10$ flatness!

Reading: \S 4 of \citep{fowles1989introduction}.

\EndArticle
