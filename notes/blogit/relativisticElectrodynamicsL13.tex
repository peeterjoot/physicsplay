%
% Copyright � 2015 Peeter Joot.  All Rights Reserved.
% Licenced as described in the file LICENSE under the root directory of this GIT repository.
%
\documentclass[]{eliblog}

\usepackage{amsmath}
\usepackage{mathpazo}

%
% shorthand for bold symbols, convenient for vectors and matrices
%
\newcommand{\Ba}[0]{\mathbf{a}}
\newcommand{\Bb}[0]{\mathbf{b}}
\newcommand{\Bc}[0]{\mathbf{c}}
\newcommand{\Bd}[0]{\mathbf{d}}
\newcommand{\Be}[0]{\mathbf{e}}
\newcommand{\Bf}[0]{\mathbf{f}}
\newcommand{\Bg}[0]{\mathbf{g}}
\newcommand{\Bh}[0]{\mathbf{h}}
\newcommand{\Bi}[0]{\mathbf{i}}
\newcommand{\Bj}[0]{\mathbf{j}}
\newcommand{\Bk}[0]{\mathbf{k}}
\newcommand{\Bl}[0]{\mathbf{l}}
\newcommand{\Bm}[0]{\mathbf{m}}
\newcommand{\Bn}[0]{\mathbf{n}}
\newcommand{\Bo}[0]{\mathbf{o}}
\newcommand{\Bp}[0]{\mathbf{p}}
\newcommand{\Bq}[0]{\mathbf{q}}
\newcommand{\Br}[0]{\mathbf{r}}
\newcommand{\Bs}[0]{\mathbf{s}}
\newcommand{\Bt}[0]{\mathbf{t}}
\newcommand{\Bu}[0]{\mathbf{u}}
\newcommand{\Bv}[0]{\mathbf{v}}
\newcommand{\Bw}[0]{\mathbf{w}}
\newcommand{\Bx}[0]{\mathbf{x}}
\newcommand{\By}[0]{\mathbf{y}}
\newcommand{\Bz}[0]{\mathbf{z}}
\newcommand{\BA}[0]{\mathbf{A}}
\newcommand{\BB}[0]{\mathbf{B}}
\newcommand{\BC}[0]{\mathbf{C}}
\newcommand{\BD}[0]{\mathbf{D}}
\newcommand{\BE}[0]{\mathbf{E}}
\newcommand{\BF}[0]{\mathbf{F}}
\newcommand{\BG}[0]{\mathbf{G}}
\newcommand{\BH}[0]{\mathbf{H}}
\newcommand{\BI}[0]{\mathbf{I}}
\newcommand{\BJ}[0]{\mathbf{J}}
\newcommand{\BK}[0]{\mathbf{K}}
\newcommand{\BL}[0]{\mathbf{L}}
\newcommand{\BM}[0]{\mathbf{M}}
\newcommand{\BN}[0]{\mathbf{N}}
\newcommand{\BO}[0]{\mathbf{O}}
\newcommand{\BP}[0]{\mathbf{P}}
\newcommand{\BQ}[0]{\mathbf{Q}}
\newcommand{\BR}[0]{\mathbf{R}}
\newcommand{\BS}[0]{\mathbf{S}}
\newcommand{\BT}[0]{\mathbf{T}}
\newcommand{\BU}[0]{\mathbf{U}}
\newcommand{\BV}[0]{\mathbf{V}}
\newcommand{\BW}[0]{\mathbf{W}}
\newcommand{\BX}[0]{\mathbf{X}}
\newcommand{\BY}[0]{\mathbf{Y}}
\newcommand{\BZ}[0]{\mathbf{Z}}

\newcommand{\Bzero}[0]{\mathbf{0}}
\newcommand{\Btheta}[0]{\boldsymbol{\theta}}
\newcommand{\Btau}[0]{\boldsymbol{\tau}}
\newcommand{\Bomega}[0]{\boldsymbol{\omega}}

%
% shorthand for unit vectors
%
\newcommand{\acap}[0]{\hat{\Ba}}
\newcommand{\bcap}[0]{\hat{\Bb}}
\newcommand{\ccap}[0]{\hat{\Bc}}
\newcommand{\dcap}[0]{\hat{\Bd}}
\newcommand{\ecap}[0]{\hat{\Be}}
\newcommand{\fcap}[0]{\hat{\Bf}}
\newcommand{\gcap}[0]{\hat{\Bg}}
\newcommand{\hcap}[0]{\hat{\Bh}}
\newcommand{\icap}[0]{\hat{\Bi}}
\newcommand{\jcap}[0]{\hat{\Bj}}
\newcommand{\kcap}[0]{\hat{\Bk}}
\newcommand{\lcap}[0]{\hat{\Bl}}
\newcommand{\mcap}[0]{\hat{\Bm}}
\newcommand{\ncap}[0]{\hat{\Bn}}
\newcommand{\ocap}[0]{\hat{\Bo}}
\newcommand{\pcap}[0]{\hat{\Bp}}
\newcommand{\qcap}[0]{\hat{\Bq}}
\newcommand{\rcap}[0]{\hat{\Br}}
\newcommand{\scap}[0]{\hat{\Bs}}
\newcommand{\tcap}[0]{\hat{\Bt}}
\newcommand{\ucap}[0]{\hat{\Bu}}
\newcommand{\vcap}[0]{\hat{\Bv}}
\newcommand{\wcap}[0]{\hat{\Bw}}
\newcommand{\xcap}[0]{\hat{\Bx}}
\newcommand{\ycap}[0]{\hat{\By}}
\newcommand{\zcap}[0]{\hat{\Bz}}
\newcommand{\thetacap}[0]{\hat{\Btheta}}

%
% to write R^n and C^n in a distinguishable fashion.  Perhaps change this
% to the double lined characters upon figuring out how to do so.
%
\newcommand{\C}[1]{$\mathbb{C}^{#1}$}
\newcommand{\R}[1]{$\mathbb{R}^{#1}$}

%
% various generally useful helpers
%

% derivative of #1 wrt. #2:
\newcommand{\D}[2] {\frac {d#2} {d#1}}

\newcommand{\inv}[1]{\frac{1}{#1}}
\newcommand{\cross}[0]{\times}

\newcommand{\abs}[1]{\lvert{#1}\rvert}
\newcommand{\norm}[1]{\lVert{#1}\rVert}
\newcommand{\innerprod}[2]{\langle{#1}, {#2}\rangle}
\newcommand{\dotprod}[2]{{#1} \cdot {#2}}
\newcommand{\bdotprod}[2]{\left({#1} \cdot {#2}\right)}
\newcommand{\crossprod}[2]{{#1} \cross {#2}}
\newcommand{\tripleprod}[3]{\dotprod{\left(\crossprod{#1}{#2}\right)}{#3}}

\DeclareMathOperator{\Proj}{Proj}
\DeclareMathOperator{\Span}{span}
\DeclareMathOperator{\Sgn}{sgn}
\DeclareMathOperator{\Area}{Area}
\DeclareMathOperator{\Volume}{Volume}

%
% A few miscellaneous things specific to this document
%
\newcommand{\crossop}[1]{\crossprod{#1}{}}

% R2 vector.
\newcommand{\VectorTwo}[2]{
\begin{bmatrix}
 {#1} \\
 {#2}
\end{bmatrix}
}

\newcommand{\VectorN}[1]{
\begin{bmatrix}
{#1}_1 \\
{#1}_2 \\
\vdots \\
{#1}_N \\
\end{bmatrix}
}

\newcommand{\DETuvij}[4]{
\begin{vmatrix}
 {#1}_{#3} & {#1}_{#4} \\
 {#2}_{#3} & {#2}_{#4}
\end{vmatrix}
}

\newcommand{\DETuvwijk}[6]{
\begin{vmatrix}
 {#1}_{#4} & {#1}_{#5} & {#1}_{#6} \\
 {#2}_{#4} & {#2}_{#5} & {#2}_{#6} \\
 {#3}_{#4} & {#3}_{#5} & {#3}_{#6}
\end{vmatrix}
}

\newcommand{\DETuvwxijkl}[8]{
\begin{vmatrix}
 {#1}_{#5} & {#1}_{#6} & {#1}_{#7} & {#1}_{#8} \\
 {#2}_{#5} & {#2}_{#6} & {#2}_{#7} & {#2}_{#8} \\
 {#3}_{#5} & {#3}_{#6} & {#3}_{#7} & {#3}_{#8} \\
 {#4}_{#5} & {#4}_{#6} & {#4}_{#7} & {#4}_{#8} \\
\end{vmatrix}
}

%\newcommand{\DETuvwxyijklm}[10]{
%\begin{vmatrix}
% {#1}_{#6} & {#1}_{#7} & {#1}_{#8} & {#1}_{#9} & {#1}_{#10} \\
% {#2}_{#6} & {#2}_{#7} & {#2}_{#8} & {#2}_{#9} & {#2}_{#10} \\
% {#3}_{#6} & {#3}_{#7} & {#3}_{#8} & {#3}_{#9} & {#3}_{#10} \\
% {#4}_{#6} & {#4}_{#7} & {#4}_{#8} & {#4}_{#9} & {#4}_{#10} \\
% {#5}_{#6} & {#5}_{#7} & {#5}_{#8} & {#5}_{#9} & {#5}_{#10}
%\end{vmatrix}
%}

% R3 vector.
\newcommand{\VectorThree}[3]{
\begin{bmatrix}
 {#1} \\
 {#2} \\
 {#3}
\end{bmatrix}
}



\author{Peeter Joot}
\email{peeter.joot@gmail.com}

%\documentclass[]{eliblogwidescreen}

\usepackage{amsmath}
\usepackage{mathpazo}

%
% shorthand for bold symbols, convenient for vectors and matrices
%
\newcommand{\Ba}[0]{\mathbf{a}}
\newcommand{\Bb}[0]{\mathbf{b}}
\newcommand{\Bc}[0]{\mathbf{c}}
\newcommand{\Bd}[0]{\mathbf{d}}
\newcommand{\Be}[0]{\mathbf{e}}
\newcommand{\Bf}[0]{\mathbf{f}}
\newcommand{\Bg}[0]{\mathbf{g}}
\newcommand{\Bh}[0]{\mathbf{h}}
\newcommand{\Bi}[0]{\mathbf{i}}
\newcommand{\Bj}[0]{\mathbf{j}}
\newcommand{\Bk}[0]{\mathbf{k}}
\newcommand{\Bl}[0]{\mathbf{l}}
\newcommand{\Bm}[0]{\mathbf{m}}
\newcommand{\Bn}[0]{\mathbf{n}}
\newcommand{\Bo}[0]{\mathbf{o}}
\newcommand{\Bp}[0]{\mathbf{p}}
\newcommand{\Bq}[0]{\mathbf{q}}
\newcommand{\Br}[0]{\mathbf{r}}
\newcommand{\Bs}[0]{\mathbf{s}}
\newcommand{\Bt}[0]{\mathbf{t}}
\newcommand{\Bu}[0]{\mathbf{u}}
\newcommand{\Bv}[0]{\mathbf{v}}
\newcommand{\Bw}[0]{\mathbf{w}}
\newcommand{\Bx}[0]{\mathbf{x}}
\newcommand{\By}[0]{\mathbf{y}}
\newcommand{\Bz}[0]{\mathbf{z}}
\newcommand{\BA}[0]{\mathbf{A}}
\newcommand{\BB}[0]{\mathbf{B}}
\newcommand{\BC}[0]{\mathbf{C}}
\newcommand{\BD}[0]{\mathbf{D}}
\newcommand{\BE}[0]{\mathbf{E}}
\newcommand{\BF}[0]{\mathbf{F}}
\newcommand{\BG}[0]{\mathbf{G}}
\newcommand{\BH}[0]{\mathbf{H}}
\newcommand{\BI}[0]{\mathbf{I}}
\newcommand{\BJ}[0]{\mathbf{J}}
\newcommand{\BK}[0]{\mathbf{K}}
\newcommand{\BL}[0]{\mathbf{L}}
\newcommand{\BM}[0]{\mathbf{M}}
\newcommand{\BN}[0]{\mathbf{N}}
\newcommand{\BO}[0]{\mathbf{O}}
\newcommand{\BP}[0]{\mathbf{P}}
\newcommand{\BQ}[0]{\mathbf{Q}}
\newcommand{\BR}[0]{\mathbf{R}}
\newcommand{\BS}[0]{\mathbf{S}}
\newcommand{\BT}[0]{\mathbf{T}}
\newcommand{\BU}[0]{\mathbf{U}}
\newcommand{\BV}[0]{\mathbf{V}}
\newcommand{\BW}[0]{\mathbf{W}}
\newcommand{\BX}[0]{\mathbf{X}}
\newcommand{\BY}[0]{\mathbf{Y}}
\newcommand{\BZ}[0]{\mathbf{Z}}

\newcommand{\Bzero}[0]{\mathbf{0}}
\newcommand{\Btheta}[0]{\boldsymbol{\theta}}
\newcommand{\Btau}[0]{\boldsymbol{\tau}}
\newcommand{\Bomega}[0]{\boldsymbol{\omega}}

%
% shorthand for unit vectors
%
\newcommand{\acap}[0]{\hat{\Ba}}
\newcommand{\bcap}[0]{\hat{\Bb}}
\newcommand{\ccap}[0]{\hat{\Bc}}
\newcommand{\dcap}[0]{\hat{\Bd}}
\newcommand{\ecap}[0]{\hat{\Be}}
\newcommand{\fcap}[0]{\hat{\Bf}}
\newcommand{\gcap}[0]{\hat{\Bg}}
\newcommand{\hcap}[0]{\hat{\Bh}}
\newcommand{\icap}[0]{\hat{\Bi}}
\newcommand{\jcap}[0]{\hat{\Bj}}
\newcommand{\kcap}[0]{\hat{\Bk}}
\newcommand{\lcap}[0]{\hat{\Bl}}
\newcommand{\mcap}[0]{\hat{\Bm}}
\newcommand{\ncap}[0]{\hat{\Bn}}
\newcommand{\ocap}[0]{\hat{\Bo}}
\newcommand{\pcap}[0]{\hat{\Bp}}
\newcommand{\qcap}[0]{\hat{\Bq}}
\newcommand{\rcap}[0]{\hat{\Br}}
\newcommand{\scap}[0]{\hat{\Bs}}
\newcommand{\tcap}[0]{\hat{\Bt}}
\newcommand{\ucap}[0]{\hat{\Bu}}
\newcommand{\vcap}[0]{\hat{\Bv}}
\newcommand{\wcap}[0]{\hat{\Bw}}
\newcommand{\xcap}[0]{\hat{\Bx}}
\newcommand{\ycap}[0]{\hat{\By}}
\newcommand{\zcap}[0]{\hat{\Bz}}
\newcommand{\thetacap}[0]{\hat{\Btheta}}

%
% to write R^n and C^n in a distinguishable fashion.  Perhaps change this
% to the double lined characters upon figuring out how to do so.
%
\newcommand{\C}[1]{$\mathbb{C}^{#1}$}
\newcommand{\R}[1]{$\mathbb{R}^{#1}$}

%
% various generally useful helpers
%

% derivative of #1 wrt. #2:
\newcommand{\D}[2] {\frac {d#2} {d#1}}

\newcommand{\inv}[1]{\frac{1}{#1}}
\newcommand{\cross}[0]{\times}

\newcommand{\abs}[1]{\lvert{#1}\rvert}
\newcommand{\norm}[1]{\lVert{#1}\rVert}
\newcommand{\innerprod}[2]{\langle{#1}, {#2}\rangle}
\newcommand{\dotprod}[2]{{#1} \cdot {#2}}
\newcommand{\bdotprod}[2]{\left({#1} \cdot {#2}\right)}
\newcommand{\crossprod}[2]{{#1} \cross {#2}}
\newcommand{\tripleprod}[3]{\dotprod{\left(\crossprod{#1}{#2}\right)}{#3}}

\DeclareMathOperator{\Proj}{Proj}
\DeclareMathOperator{\Span}{span}
\DeclareMathOperator{\Sgn}{sgn}
\DeclareMathOperator{\Area}{Area}
\DeclareMathOperator{\Volume}{Volume}

%
% A few miscellaneous things specific to this document
%
\newcommand{\crossop}[1]{\crossprod{#1}{}}

% R2 vector.
\newcommand{\VectorTwo}[2]{
\begin{bmatrix}
 {#1} \\
 {#2}
\end{bmatrix}
}

\newcommand{\VectorN}[1]{
\begin{bmatrix}
{#1}_1 \\
{#1}_2 \\
\vdots \\
{#1}_N \\
\end{bmatrix}
}

\newcommand{\DETuvij}[4]{
\begin{vmatrix}
 {#1}_{#3} & {#1}_{#4} \\
 {#2}_{#3} & {#2}_{#4}
\end{vmatrix}
}

\newcommand{\DETuvwijk}[6]{
\begin{vmatrix}
 {#1}_{#4} & {#1}_{#5} & {#1}_{#6} \\
 {#2}_{#4} & {#2}_{#5} & {#2}_{#6} \\
 {#3}_{#4} & {#3}_{#5} & {#3}_{#6}
\end{vmatrix}
}

\newcommand{\DETuvwxijkl}[8]{
\begin{vmatrix}
 {#1}_{#5} & {#1}_{#6} & {#1}_{#7} & {#1}_{#8} \\
 {#2}_{#5} & {#2}_{#6} & {#2}_{#7} & {#2}_{#8} \\
 {#3}_{#5} & {#3}_{#6} & {#3}_{#7} & {#3}_{#8} \\
 {#4}_{#5} & {#4}_{#6} & {#4}_{#7} & {#4}_{#8} \\
\end{vmatrix}
}

%\newcommand{\DETuvwxyijklm}[10]{
%\begin{vmatrix}
% {#1}_{#6} & {#1}_{#7} & {#1}_{#8} & {#1}_{#9} & {#1}_{#10} \\
% {#2}_{#6} & {#2}_{#7} & {#2}_{#8} & {#2}_{#9} & {#2}_{#10} \\
% {#3}_{#6} & {#3}_{#7} & {#3}_{#8} & {#3}_{#9} & {#3}_{#10} \\
% {#4}_{#6} & {#4}_{#7} & {#4}_{#8} & {#4}_{#9} & {#4}_{#10} \\
% {#5}_{#6} & {#5}_{#7} & {#5}_{#8} & {#5}_{#9} & {#5}_{#10}
%\end{vmatrix}
%}

% R3 vector.
\newcommand{\VectorThree}[3]{
\begin{bmatrix}
 {#1} \\
 {#2} \\
 {#3}
\end{bmatrix}
}



\author{Peeter Joot}
\email{peeter.joot@gmail.com}


\chapter{PHY450H1S.  Relativistic Electrodynamics Lecture 11 (Taught by Prof. Erich Poppitz).  FIXME.}
\label{chap:relativisticElectrodynamicsL13}
%\useCCL
\blogpage{http://sites.google.com/site/peeterjoot/math2011/relativisticElectrodynamicsL13.pdf}
\date{Feb 15, 2011}
\revisionInfo{relativisticElectrodynamicsL13.tex}

%\beginArtWithToc
\beginArtNoToc

\section{Reading.}

Covering chapter 4 material from the text \cite{landau1980classical}.

Covering \href{http://www.physics.utoronto.ca/~poppitz/e-poppitz/PHY450_files/RelEMpp103-113.pdf}{lecture notes pp.103-113}: variational principle for the electromagnetic field and the relevant boundary conditions (103-105); the second set of Maxwell�s equations from the variational principle (106-108); Maxwell�s equations in vacuum and the wave equation in the nonrelativistic Coulomb gauge (109-111)

\section{Review.  Our action.}

\begin{align*}\label{eqn:relativisticElectrodynamicsL13:10}
S 
&= S_{\text{particles}} + S_{\text{interaction}} + S_{\text{EM field}}
&= \sum_A \int_{x_A^i(\tau)} ds ( -m_A c )
- \sum_A 
\frac{e_A}{c}
\int dx_A^i A_i(x_A) 
- \inv{16 \pi c} \int d^4 x F^{ij } F_{ij}
\end{align*}

Our dynamics variables are 

\begin{equation}\label{eqn:relativisticElectrodynamicsL13:30}
\left\{
\begin{array}{l l}
x_A^i(\tau) & \quad \mbox{$A = 1, \cdots, N$} \\
A^i(x) & \quad \mbox{$A = 1, \cdots, N$}
\end{array}
\right.
\end{equation}

Also
\begin{equation}\label{eqn:relativisticElectrodynamicsL13:50}
S_{\text{interaction}} 
= -\inv{c^2} \int d^4x j^i(x) A_i(x)
\end{equation}

\begin{equation}\label{eqn:relativisticElectrodynamicsL13:70}
j^i(x) = \sum_A c e_A \int dx_A^i \delta^4( x - x_A(\tau))
\end{equation}

Variation with respect to $x_A^i(\tau)$ we have

\begin{equation}\label{eqn:relativisticElectrodynamicsL13:90}
m c \dds{u^i_A} = \frac{e}{c} u_A^j F_{ij}
\end{equation}

Note that it's easy to get the sign mixed up here.  With our $(+,-,-,-)$ metric tensor, if the second index is the summation index, we have a positive sign.

Only the $S_{\text{particles}}$ and $S_{\text{interaction}}$ depend on $x_A^i(\tau)$.

\section{The field action variation.}

\paragraph{Today}: We'll find the EOM for $A^i(x)$.  The dynamical degrees of freedom are $A^i(\Bx,t)$

\begin{equation}\label{eqn:relativisticElectrodynamicsL13:110}
S[A^i(\Bx, t)] = -\inv{16 \pi c} \int d^4x F_{ij}F^{ij} - \inv{c^2} \int d^4 x A^i j_i
\end{equation}

Here $j^i$ are treated as ``sources''.

We demand that 

\begin{equation}\label{eqn:relativisticElectrodynamicsL13:130}
\delta S = S[ A^i(\Bx, t) + \delta A^i(\Bx, t)] - S[ A^i(\Bx, t) ] = 0 + O(\delta A)^2
\end{equation}

We need to impose two conditions.
\begin{itemize}
\item At spatial $\infty$, i.e. at $\Abs{\Bx} \rightarrow \infty, \forall t$, we'll impose the condition 

\begin{equation}\label{eqn:relativisticElectrodynamicsL13:150}
\evalbar{A^i(\Bx, t)}{\Abs{\Bx} \rightarrow \infty} \rightarrow 0
\end{equation}

This is sensible, because fields are created by charges, and charges are assumed to be localized in a bounded region.  The field outside charges will $\rightarrow 0$ at $\Abs{\Bx} \rightarrow \infty$.  Later we will treat the integration range as finite, and bounded, then later allow the boundary to go to infitiy.

\item at $t = -T$ and $t = T$ we'll imagine that the values of $A^i(\Bx, \pm T)$ are fixed.

This is analogous to $x(t_i) = x_1$ and $x(t_f) = x_2$ in particle mechanics.

Since $A^i(\Bx, \pm T)$ is given, and equivalent to the initial and final field configurations, our extremes at the boundary is zero

\begin{equation}\label{eqn:relativisticElectrodynamicsL13:170}
\delta A^i(\Bx, \pm T) = 0
\end{equation}

\end{itemize}

PICTURE: a cylinder in spacetime, with an attempt to depict the boundary.

\section{Computing the variation.}

\begin{equation}\label{eqn:relativisticElectrodynamicsL13:190}
\delta S[A^i(\Bx, t)] 
= -\inv{16 \pi c} \int d^4 x \delta (F_{ij}F^{ij}) - \inv{c^2} \int d^4 x \delta(A^i) j_i 
\end{equation}

But 

\begin{align*}
\delta (F_{ij}F^{ij}) 
&= 
\delta(F_{ij}) F^{ij} + F_{ij} \delta(F^{ij}) \\
&= 
2 \delta(F^{ij}) F_{ij} \\
&= 
2 \delta(\partial^i A^j - \partial^j A^i) F_{ij} \\
&= 
2 \delta(\partial^i A^j) F_{ij} - 2 \delta(\partial^j A^i) F_{ij} \\
&= 
2 \delta(\partial^j A^i) F_{ji} - 2 \delta(\partial^j A^i) F_{ij} \\
&= 
4 \delta(\partial^i A^j) F_{ij} \\
&= 
4 F_{ij} \partial^i \delta(A^j) \\
\end{align*}

So
\begin{align*}
\delta S[A^i(\Bx, t)] 
&= -\inv{4 \pi c} \int d^4 x F_{ij} \partial^i \delta(A^j) - \inv{c^2} \int d^4 x j^i \delta(A_i) \\
\end{align*}

\begin{align*}
\int d^4 x F_{ij} \partial^i \delta(A^j) 
=
\int d^4 x F^{ij} \partial_i \delta(A_j) 
=
\int d^4 x F^{ij} \PD{x^i}{} \delta(A_j) 
=
\int d^4 x \PD{x^i}{} \delta A_j F^{ij}
\delta(A^j) 
\end{align*}

DIY.  Integrate by parts.

= ... 0 (zero over the boundary)

Noting that 

\begin{align*}
\PD{x^\alpha}{}\left( \delta A_j F^{\alpha j} \right) \\
\spacegrad \cdot \BC, \BC^\alpha = \delta A_j F^{\alpha j} \\
...
\end{align*}

We are left with

\begin{align*}
\delta S[A^i(\Bx, t)] 
&= -\inv{4 \pi c} \int d^4 x \delta (A_j) \partial_i F^{ij} - \inv{c^2} \int d^4 x j^i \delta(A_i) \\
&= 
\int d^4 x \delta A_j(x)
\left(
-\inv{4 \pi c} \partial_i F^{ij}(x) - \inv{c^2} j^i 
\right)  \\
&= 0
\end{align*}

This gives us

\begin{equation}\label{eqn:relativisticElectrodynamicsL13:210}
\boxed{
\partial_i F^{ij} = \frac{4 \pi}{c} j^j
}
\end{equation}

\section{Unpacking these.}

Recall that 

\begin{equation}\label{eqn:relativisticElectrodynamicsL13:230}
e^{ijkl} \partial_j F_{kl} = 0
\end{equation}

gave us

\begin{align}\label{eqn:relativisticElectrodynamicsL13:250}
\spacegrad \cdot \BB &= 0 \\
\spacegrad \cross \BE &= -\inv{c} \PD{t}{\BB}
\end{align}

\begin{equation}\label{eqn:relativisticElectrodynamicsL13:270}
\partial_\alpha F^{\alpha 0} = \frac{4 \pi}{c} j^0 = 4 \pi \rho
\end{equation}

(since $j^0 = c \rho$).

Because 

\begin{equation}\label{eqn:relativisticElectrodynamicsL13:290}
F^{\alpha 0} = (\BE)^\alpha
\end{equation}

or
\begin{equation}\label{eqn:relativisticElectrodynamicsL13:310}
\spacegrad \cdot \BE = 4 \pi \rho
\end{equation}

The messier one to deal with is

\begin{equation}\label{eqn:relativisticElectrodynamicsL13:330}
\partial_i F^{i\alpha} = \frac{4 \pi}{c} j^\alpha
\end{equation}

...

\begin{align*}
\partial_i F^{i\alpha} 
&= \partial_\beta F^{\beta \alpha} + \partial_0 F^{0 \alpha} \\
&= \partial_\beta F^{\beta \alpha} - \inv{c} \PD{t}{(\BE)^\alpha} \\
\end{align*}

Details: DIY.  We get

\begin{equation}\label{eqn:relativisticElectrodynamicsL13:350}
\frac{4 \pi}{c} j^\alpha
= (\spacegrad \cross \BB)^\alpha - \inv{c} \PD{t}{(\BE)^\alpha} \\
\end{equation}

or

\begin{equation}\label{eqn:relativisticElectrodynamicsL13:370}
\spacegrad \cross \BB - \inv{c} \PD{t}{\BE} = \frac{4 \pi}{c} \Bj
\end{equation}

Summarizing what we know so far, we have

\begin{equation}\label{eqn:relativisticElectrodynamicsL13:390}
\boxed{
\begin{aligned}
\partial_i F^{ij} &= \frac{4 \pi}{c} j^j \\
\epsilon^{ijkl} \partial_j F_{kl} &= 0
\end{aligned}
}
\end{equation}

or in vector form

\begin{equation}\label{eqn:relativisticElectrodynamicsL13:410}
\boxed{
\begin{aligned}
\spacegrad \cdot \BE &= 4 \pi \rho \\
\spacegrad \cross \BB -\inv{c} \PD{t}{\BE} &= \frac{4 \pi}{c} \Bj \\
\spacegrad \cdot \BB &= 0 \\
\spacegrad \cross \BE +\inv{c} \PD{t}{\BB} &= 0
\end{aligned}
}
\end{equation}

\section{Speed of light}

\paragraph{Claim}: ``$c$'' is the speed of EM waves in vacuum.

Study equations in vacuum (no sources, so $j^i = 0$) for $A^i = (\phi, \BA)$.

\begin{align}\label{eqn:relativisticElectrodynamicsL13:430}
\spacegrad \cdot \BE &= 0 \\
\spacegrad \cross \BB &= \inv{c} \PD{t}{\BE} 
\end{align}

where

\begin{align}\label{eqn:relativisticElectrodynamicsL13:450}
\BE &= - \spacegrad \phi - \inv{c} \PD{t}{\BA} \\
\BB &= \spacegrad \cross \BA
\end{align}

In terms of potentials

\begin{align}\label{eqn:relativisticElectrodynamicsL13:470}
\spacegrad \cross (\spacegrad \spacegrad \BA) &= - \inv{c} \spacegrad \PD{t}{\phi} - \inv{c} \frac{\partial^2}{\partial t^2} \BA
-\spacegrad^2 \phi - \inv{c} \spacegrad \cdot \BA &= 0
\end{align}

Can make a gauge transformation

\begin{align}\label{eqn:relativisticElectrodynamicsL13:490}
(\phi, \BA) \rightarrow (\phi', \BA')
\end{align}

with 

\begin{align}\label{eqn:relativisticElectrodynamicsL13:510}
\phi &= \phi' + \inv{c} \PD{t}{\chi} \\
\BA &= \BA' - \spacegrad \chi
\end{align}

Can choose $\chi(\Bx, t)$ to make $\phi' = 0$ ($\forall \phi \exists \chi, \phi' = 0$)

\begin{align}\label{eqn:relativisticElectrodynamicsL13:530}
\inv{c} \PD{t}{} \chi(\Bx, t) = \phi(\Bx, t)
\end{align}

\begin{align}\label{eqn:relativisticElectrodynamicsL13:550}
\chi(\Bx, t) = c \int_{-\infty}^t dt' \phi(\Bx, t')
\end{align}

Can also find a transformation that also allows $\spacegrad \cdot \BA = 0$

This is the Coulomb gauge 

\begin{align}\label{eqn:relativisticElectrodynamicsL13:570}
\phi &= 0 \\
\spacegrad \cdot \BA &= 0
\end{align}

DIY.  Can use this above to kill a bunch of the terms, leaving 

\begin{equation}\label{eqn:relativisticElectrodynamicsL13:590}
\spacegrad \cross (\spacegrad \cross \BA') = -\inv{c^2} \frac{\partial^2 \BA'}{\partial t^2}
\end{equation}

Using 

\begin{equation}\label{eqn:relativisticElectrodynamicsL13:600}
\spacegrad \cross (\spacegrad \cross \BA') = \spacegrad (\spacegrad \cdot \BA') - \spacegrad^2 \BA',
\end{equation}

we have

\begin{equation}\label{eqn:relativisticElectrodynamicsL13:610}
\inv{c^2} \frac{\partial^2 \BA'}{\partial t^2} -\spacegrad^2 \BA' = 0
\end{equation}

which is the wave equation for the propagation of the vector potential $\BA'(\Bx, t)$ through space at velocity $c$.

\EndArticle
