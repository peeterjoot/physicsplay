%
% Copyright � 2013 Peeter Joot.  All Rights Reserved.
% Licenced as described in the file LICENSE under the root directory of this GIT repository.
%
\newcommand{\authorname}{Peeter Joot}
\newcommand{\email}{peeterjoot@protonmail.com}
\newcommand{\basename}{FIXMEbasenameUndefined}
\newcommand{\dirname}{notes/FIXMEdirnameUndefined/}

\renewcommand{\basename}{fermiNumberDensityLargeVolume}
\renewcommand{\dirname}{notes/phy452/}
\newcommand{\keywords}{Statistical mechanics, PHY452H1S, Fermi gas, number density, large volume, surface with binding sites}

\newcommand{\authorname}{Peeter Joot}
\newcommand{\onlineurl}{http://sites.google.com/site/peeterjoot2/math2013/\basename.pdf}
\newcommand{\sourcepath}{\dirname\basename.tex}
\newcommand{\generatetitle}[1]{\chapter{#1}}

\newcommand{\vcsinfo}{%
\section*{}
\noindent{\color{DarkOliveGreen}{\rule{\linewidth}{0.1mm}}}
\paragraph{Document version}
%\paragraph{\color{Maroon}{Document version}}
{
\small
\begin{itemize}
\item Available online at:\\ 
\href{\onlineurl}{\onlineurl}
\item Git Repository: \input{./.revinfo/gitRepo.tex}
\item Source: \sourcepath
\item last commit: \input{./.revinfo/gitCommitString.tex}
\item commit date: \input{./.revinfo/gitCommitDate.tex}
\end{itemize}
}
}

%\PassOptionsToPackage{dvipsnames,svgnames}{xcolor}
\PassOptionsToPackage{square,numbers}{natbib}
\documentclass{scrreprt}

\usepackage[left=2cm,right=2cm]{geometry}
\usepackage[svgnames]{xcolor}
\usepackage{peeters_layout}

\usepackage{natbib}

\usepackage[
colorlinks=true,
bookmarks=false,
pdfauthor={\authorname, \email},
backref 
]{hyperref}

% http://tex.stackexchange.com/questions/75773/how-to-reference-problems-by-the-text-label-in-an-exercise-envioronment
\usepackage[english]{cleveref}
\crefname{Exercise}{exercise}{exercises}
\Crefname{Exercise}{Exercise}{Exercises}

\RequirePackage{titlesec}
\RequirePackage{ifthen}

% http://stackoverflow.com/questions/4932910/date-in-the-tabular-environment
\makeatletter
\let\insertdate\@date
\makeatother

\titleformat{\chapter}[display]
{\bfseries\Large}
{\color{DarkSlateGrey}\filleft \authorname
\ifthenelse{\isundefined{\studentnumber}}{}{\\ \studentnumber}
\ifthenelse{\isundefined{\email}}{}{\\ \email}
\ifthenelse{\isundefined{\dateintitle}}{}{\\ \insertdate}
%\ifthenelse{\isundefined{\coursename}}{}{\\ \coursename} % put in title instead.
}
{4ex}
{\color{DarkOliveGreen}{\titlerule}\color{Maroon}
\vspace{2ex}%
\filright}
[\vspace{2ex}%
\color{DarkOliveGreen}\titlerule
]

\newcommand{\beginArtWithToc}[0]{\begin{document}\tableofcontents}
\newcommand{\beginArtNoToc}[0]{\begin{document}}
\newcommand{\EndNoBibArticle}[0]{\end{document}}
\newcommand{\EndArticle}[0]{\bibliography{Bibliography}\bibliographystyle{plainnat}\end{document}}

% 
%\newcommand{\citep}[1]{\cite{#1}}

\colorSectionsForArticle



\beginArtNoToc

\generatetitle{Large volume Fermi gas density}
%\chapter{Large volume fermi gas density}
\label{chap:fermiNumberDensityLargeVolume}

\makeoproblem{Large volume Fermi gas density}{pr:fermiNumberDensityLargeVolume:1}{2013 final exam pr 1}{

\makesubproblem{}{pr:fermiNumberDensityLargeVolume:1:a}

Write down the expression for the grand canonical partition function $\ZG$ of an ideal three-dimensional Fermi gas with atoms having mass $m$ at a temperature $T$ and a chemical potential $\mu$ (or equivalently a fugacity $z = e^{\beta \mu}$).  Consider the high temperature ``classical limit'' of this ideal gas, where $z \ll 1$ and one gets an effective Boltzmann distribution, and obtain the equation for the density of the particles

\begin{dmath}\label{eqn:fermiNumberDensityLargeVolume:20}
n = \inv{V \beta} \PD{\mu}{\ln \ZG}
\end{dmath}

by converting momentum sums into integrals.  Invert this relationship to find the chemical potential $\mu$ as a function of the density $n$.

Hint: In the limit of a large volume $V$:

\begin{dmath}\label{eqn:fermiNumberDensityLargeVolume:40}
\sum_\Bk \rightarrow V \int \frac{d^3 \Bk}{(2 \pi)^3}.
\end{dmath}

\makesubproblem{}{pr:fermiNumberDensityLargeVolume:1:b}
Next consider a two-dimensional surface having binding sites where these Fermi atoms could be absorbed from the surrounding gas environment, such that zero atoms or one atom could sit at each site.  The binding sites are completely independent of one another.  The energy when an atom sits at such a binding site is $-B$ (with $B > 0$).  This surface is exposed to the above gas, allowing atoms to absorb or leave the surface.  Find the fraction of binding sites which will be occupied by atoms assuming that the surface and the surrounding gas are in equilibrium - with a common temperature and a common chemical potential.  Hint: Assume the surrounding gas is ``infinite'', so that the common chemical potential and temperature are determined solely by the gas as in part \ref{pr:fermiNumberDensityLargeVolume:1:a}.

} % makeoproblem

\makeanswer{pr:fermiNumberDensityLargeVolume:1}{

\makeSubAnswer{}{pr:fermiNumberDensityLargeVolume:1:a}
TODO.

\makeSubAnswer{}{pr:fermiNumberDensityLargeVolume:1:b}
TODO.

} % makeanswer

%\EndArticle
\EndNoBibArticle
