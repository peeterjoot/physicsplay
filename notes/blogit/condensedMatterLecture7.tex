%
% Copyright � 2013 Peeter Joot.  All Rights Reserved.
% Licenced as described in the file LICENSE under the root directory of this GIT repository.
%
\newcommand{\authorname}{Peeter Joot}
\newcommand{\email}{peeterjoot@protonmail.com}
\newcommand{\basename}{FIXMEbasenameUndefined}
\newcommand{\dirname}{notes/FIXMEdirnameUndefined/}

\renewcommand{\basename}{condensedMatterLecture7}
\renewcommand{\dirname}{notes/phy487/}
\newcommand{\keywords}{Condensed matter physics, PHY487H1F}
\newcommand{\authorname}{Peeter Joot}
\newcommand{\onlineurl}{http://sites.google.com/site/peeterjoot2/math2013/\basename.pdf}
\newcommand{\sourcepath}{\dirname\basename.tex}
\newcommand{\generatetitle}[1]{\chapter{#1}}

\newcommand{\vcsinfo}{%
\section*{}
\noindent{\color{DarkOliveGreen}{\rule{\linewidth}{0.1mm}}}
\paragraph{Document version}
%\paragraph{\color{Maroon}{Document version}}
{
\small
\begin{itemize}
\item Available online at:\\ 
\href{\onlineurl}{\onlineurl}
\item Git Repository: \input{./.revinfo/gitRepo.tex}
\item Source: \sourcepath
\item last commit: \input{./.revinfo/gitCommitString.tex}
\item commit date: \input{./.revinfo/gitCommitDate.tex}
\end{itemize}
}
}

%\PassOptionsToPackage{dvipsnames,svgnames}{xcolor}
\PassOptionsToPackage{square,numbers}{natbib}
\documentclass{scrreprt}

\usepackage[left=2cm,right=2cm]{geometry}
\usepackage[svgnames]{xcolor}
\usepackage{peeters_layout}

\usepackage{natbib}

\usepackage[
colorlinks=true,
bookmarks=false,
pdfauthor={\authorname, \email},
backref 
]{hyperref}

% http://tex.stackexchange.com/questions/75773/how-to-reference-problems-by-the-text-label-in-an-exercise-envioronment
\usepackage[english]{cleveref}
\crefname{Exercise}{exercise}{exercises}
\Crefname{Exercise}{Exercise}{Exercises}

\RequirePackage{titlesec}
\RequirePackage{ifthen}

% http://stackoverflow.com/questions/4932910/date-in-the-tabular-environment
\makeatletter
\let\insertdate\@date
\makeatother

\titleformat{\chapter}[display]
{\bfseries\Large}
{\color{DarkSlateGrey}\filleft \authorname
\ifthenelse{\isundefined{\studentnumber}}{}{\\ \studentnumber}
\ifthenelse{\isundefined{\email}}{}{\\ \email}
\ifthenelse{\isundefined{\dateintitle}}{}{\\ \insertdate}
%\ifthenelse{\isundefined{\coursename}}{}{\\ \coursename} % put in title instead.
}
{4ex}
{\color{DarkOliveGreen}{\titlerule}\color{Maroon}
\vspace{2ex}%
\filright}
[\vspace{2ex}%
\color{DarkOliveGreen}\titlerule
]

\newcommand{\beginArtWithToc}[0]{\begin{document}\tableofcontents}
\newcommand{\beginArtNoToc}[0]{\begin{document}}
\newcommand{\EndNoBibArticle}[0]{\end{document}}
\newcommand{\EndArticle}[0]{\bibliography{Bibliography}\bibliographystyle{plainnat}\end{document}}

% 
%\newcommand{\citep}[1]{\cite{#1}}

\colorSectionsForArticle



%\citep{harald2003solid} \S x.y
%\citep{ibach2009solid} \S x.y

%\usepackage{mhchem}
\usepackage[version=3]{mhchem}

\beginArtNoToc
\generatetitle{PHY487H1F Condensed Matter Physics.  Lecture 7: Structure factor.  Taught by Prof.\ Stephen Julian}
%\chapter{Structure factor}
\label{chap:condensedMatterLecture7}

\section{Disclaimer}

Peeter's lecture notes from class.  May not be entirely coherent.

\section{Structure factor}

%\cref{fig:qmSolidsL7:qmSolidsL7Fig1}.
%\imageFigure{qmSolidsL7Fig1}{CAPTION}{fig:qmSolidsL7:qmSolidsL7Fig1}{0.3}
Starting with

\begin{dmath}\label{eqn:condensedMatterLecture7:20}
\rho(\Br) = \sum_{h k l} \rho_{h k l} e^{i \BG_{h k l} \cdot \Br}
\end{dmath}

we consider a unit cell defined by

\begin{dmath}\label{eqn:condensedMatterLecture7:40}
\rho_{h k l} = \inv{V_{\text{unit cell}}} e^{-i \BG \cdot \Br},
\end{dmath}

so that $\rho(\Br)$ is large close to each nucleus.  This gives

\begin{dmath}\label{eqn:condensedMatterLecture7:60}
\rho_{h k l} \sim \inv{V_{\text{cell}}} \sum_\alpha 
e^{i \BG_{h k l} \cdot \Br_\alpha}
\mathLabelBox
[
   labelstyle={xshift=2cm},
   linestyle={out=270,in=90, latex-}
]
{
\int \rho_\alpha(\Br') 
e^{-i \BG_{h k l} \cdot \Br} d\Br'
}
{$f_\alpha$, the \indexAndText{atomic scattering factor} (tabulated)},
\end{dmath}

where the index $\alpha$ is used to sum over all atoms in a primative unit cell.

In terms of the scattering factor, this is

\begin{dmath}\label{eqn:condensedMatterLecture7:80}
\rho_{h k l} \sim \inv{V_{\text{cell}}} 
\mathLabelBox
[
   labelstyle={xshift=2cm},
   linestyle={out=270,in=90, latex-}
]
{
\sum_\alpha f_\alpha
e^{-i \BG_{h k l} \cdot \Br_\alpha} 
}
{$S_{h k l}$, the \indexAndText{structure factor}}.
\end{dmath}

\makeexample{Treating bcc lattice as simple cubic and 2 atom basis}{example:condensedMatterLecture7:1}{

%F2
%\cref{fig:qmSolidsL7:qmSolidsL7Fig2}.
\imageFigure{qmSolidsL7Fig2}{Bcc}{fig:qmSolidsL7:qmSolidsL7Fig2}{0.3}

\begin{subequations}
\begin{dmath}\label{eqn:condensedMatterLecture7:100}
\Br_1 = 
\begin{bmatrix}
0 \\ 
0 \\
0
\end{bmatrix}
\end{dmath}
\begin{dmath}\label{eqn:condensedMatterLecture7:120}
\Br_2 = 
\frac{a}{2}
\begin{bmatrix}
1 \\ 
1 \\
1
\end{bmatrix}
\end{dmath}
\end{subequations}

\begin{dmath}\label{eqn:condensedMatterLecture7:140}
\BG_{h k l} \cdot \Br_n = 2 \pi m
\end{dmath}

\begin{dmath}\label{eqn:condensedMatterLecture7:160}
S_{h k l} = f \lr{ 
\mathLabelBox
%[
%   labelstyle={xshift=1cm},
%   linestyle={out=270,in=90, latex-}
%]
{
1 
}{corner}
+ 
e^{-i 
   \mathLabelBox
%[
%   labelstyle={xshift=-1cm},
%   linestyle={out=270,in=90, latex-}
%]
[
   labelstyle={below of=m\themathLableNode, below of=m\themathLableNode}
]
   {
   \pi ( h + k + l )
   }
   {
   body center
   }
  }
}
\end{dmath}

A bcc lattice has same diffraction pattern as simple cubic, except all $h + k + l = \text{odd}$ spots are missing.
}

Reading: \citep{ibach2009solid} \S 3.7

\section{Brillouin zones}

We can define a special primative unit cell, by bisecting the (reciprocal) lattice vectors with a plane.  In 2D consider \cref{fig:qmSolidsL7:qmSolidsL7Fig3}

\imageFigure{qmSolidsL7Fig3}{First Brillouin zone}{fig:qmSolidsL7:qmSolidsL7Fig3}{0.3}

All points inside the first Brillouin zone are closer to $(0, 0)$ than to any other lattice point.

Example: 07 lecture.pdf

Fcc lattic has a bcc reciprocal lattice.

Some 3D figures from \citep{wiki:BrillouinZone} were shown in slides.

Typical exam question: draw a primative unit cell for a lattice: follow this procedure.

\section{Phonons}

Here will do an introductory calculation not in the text.  Consider a 1D chain of $N$ atoms, coupled by harmonic springs with periodic boundary conditions.  We suppose that we have $N \sim 10^{23}$.  This is illustrated in \cref{fig:qmSolidsL7:qmSolidsL7Fig4}.

\imageFigure{qmSolidsL7Fig4}{Coupled period oscillators}{fig:qmSolidsL7:qmSolidsL7Fig4}{0.3}

With equilbrium positions $x_j$, and displacement distances from equilibrium of $u_j$, as in \cref{fig:qmSolidsL7:qmSolidsL7Fig5}.

\imageFigure{qmSolidsL7Fig5}{Equilibrium and displacement positions}{fig:qmSolidsL7:qmSolidsL7Fig5}{0.3}

Our force balance is

\begin{dmath}\label{eqn:condensedMatterLecture7:180}
m \uddot_j = K \lr{ u_{j + 1} - u_j } + K \lr{ u_{j - 1} - u_j} 
\end{dmath}

We have $10^{23}$ coupled equations.

Look for solutions of the form

\begin{dmath}\label{eqn:condensedMatterLecture7:200}
u_j = \inv{m} \sum_q u_q e^{i \lr{ q x_j - \omega_q t} }.
\end{dmath}

The periodicity requirement imposes a constraint on

\begin{dmath}\label{eqn:condensedMatterLecture7:220}
e^{i q( x_j + N a) },
\end{dmath}

so that 

\begin{dmath}\label{eqn:condensedMatterLecture7:240}
q \mathLabelBox{N a}{$L$} = 2 \pi n,
\end{dmath}

or
\begin{dmath}\label{eqn:condensedMatterLecture7:260}
q = \frac{2 \pi n}{L}.
\end{dmath}

This gives:

rest on paper.

\EndArticle
