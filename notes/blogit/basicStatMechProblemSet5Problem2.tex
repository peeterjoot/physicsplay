\makeproblem{Independent one-dimensional harmonic oscillators}{basicStatMech:problemSet5:2}{ 
%(4 points)
Consider a set of $N$ independent classical harmonic oscillators, each having a frequency $\omega$. 

\makesubproblem{}{pr:basicStatMechProblemSet5Problem2:a}
Find the canonical partition at a temperature $T$ for this system of oscillators keeping track of correction factors of Planck constant. (Note that the oscillators are distinguishable, and we do not need $1/N!$ correction factor.)  

\makesubproblem{}{pr:basicStatMechProblemSet5Problem2:b}
Using this, derive the mean energy and the specific heat at temperature $T$. 

\makesubproblem{}{pr:basicStatMechProblemSet5Problem2:c}
For quantum oscillators, the partition function of each oscillator is simply $\sum_n e^{-\beta E_n}$ where $E_n$ are the (discrete) energy levels given by $(n + 1/2)\hbar \omega$, with $n = 0,1,2,\cdots$.  Hence, find the canonical partition function for $N$ independent distinguishable quantum oscillators, and find the mean energy and specific heat at temperature $T$. 

\makesubproblem{}{pr:basicStatMechProblemSet5Problem2:d}
Show that the quantum results go over into the classical results at high temperature $k_{\mathrm{B}} T \gg \hbar \omega$, and comment on why this makes sense.  

\makesubproblem{}{pr:basicStatMechProblemSet5Problem2:e}
Also find the low temperature behavior of the specific heat in both classical and quantum cases when $k_{\mathrm{B}} T \ll \hbar \omega$.
} % makeproblem

\makeanswer{basicStatMech:problemSet5:2}{ 
\makeSubAnswer{Classical partition function}{pr:basicStatMechProblemSet5Problem2:a}
TODO.
\makeSubAnswer{Classical mean energy and heat capacity}{pr:basicStatMechProblemSet5Problem2:b}
TODO.
\makeSubAnswer{Quantum partition function, mean energy and heat capacity}{pr:basicStatMechProblemSet5Problem2:c}
TODO.
\makeSubAnswer{Classical limits}{pr:basicStatMechProblemSet5Problem2:d}
TODO.
\makeSubAnswer{Low temperature limits}{pr:basicStatMechProblemSet5Problem2:e}
TODO.
}
