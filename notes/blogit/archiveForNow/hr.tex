%
% Copyright � 2012 Peeter Joot.  All Rights Reserved.
% Licenced as described in the file LICENSE under the root directory of this GIT repository.
%

%
%
\documentclass[]{eliblog}

\usepackage{amsmath}
\usepackage{mathpazo}

%
% shorthand for bold symbols, convenient for vectors and matrices
%
\newcommand{\Ba}[0]{\mathbf{a}}
\newcommand{\Bb}[0]{\mathbf{b}}
\newcommand{\Bc}[0]{\mathbf{c}}
\newcommand{\Bd}[0]{\mathbf{d}}
\newcommand{\Be}[0]{\mathbf{e}}
\newcommand{\Bf}[0]{\mathbf{f}}
\newcommand{\Bg}[0]{\mathbf{g}}
\newcommand{\Bh}[0]{\mathbf{h}}
\newcommand{\Bi}[0]{\mathbf{i}}
\newcommand{\Bj}[0]{\mathbf{j}}
\newcommand{\Bk}[0]{\mathbf{k}}
\newcommand{\Bl}[0]{\mathbf{l}}
\newcommand{\Bm}[0]{\mathbf{m}}
\newcommand{\Bn}[0]{\mathbf{n}}
\newcommand{\Bo}[0]{\mathbf{o}}
\newcommand{\Bp}[0]{\mathbf{p}}
\newcommand{\Bq}[0]{\mathbf{q}}
\newcommand{\Br}[0]{\mathbf{r}}
\newcommand{\Bs}[0]{\mathbf{s}}
\newcommand{\Bt}[0]{\mathbf{t}}
\newcommand{\Bu}[0]{\mathbf{u}}
\newcommand{\Bv}[0]{\mathbf{v}}
\newcommand{\Bw}[0]{\mathbf{w}}
\newcommand{\Bx}[0]{\mathbf{x}}
\newcommand{\By}[0]{\mathbf{y}}
\newcommand{\Bz}[0]{\mathbf{z}}
\newcommand{\BA}[0]{\mathbf{A}}
\newcommand{\BB}[0]{\mathbf{B}}
\newcommand{\BC}[0]{\mathbf{C}}
\newcommand{\BD}[0]{\mathbf{D}}
\newcommand{\BE}[0]{\mathbf{E}}
\newcommand{\BF}[0]{\mathbf{F}}
\newcommand{\BG}[0]{\mathbf{G}}
\newcommand{\BH}[0]{\mathbf{H}}
\newcommand{\BI}[0]{\mathbf{I}}
\newcommand{\BJ}[0]{\mathbf{J}}
\newcommand{\BK}[0]{\mathbf{K}}
\newcommand{\BL}[0]{\mathbf{L}}
\newcommand{\BM}[0]{\mathbf{M}}
\newcommand{\BN}[0]{\mathbf{N}}
\newcommand{\BO}[0]{\mathbf{O}}
\newcommand{\BP}[0]{\mathbf{P}}
\newcommand{\BQ}[0]{\mathbf{Q}}
\newcommand{\BR}[0]{\mathbf{R}}
\newcommand{\BS}[0]{\mathbf{S}}
\newcommand{\BT}[0]{\mathbf{T}}
\newcommand{\BU}[0]{\mathbf{U}}
\newcommand{\BV}[0]{\mathbf{V}}
\newcommand{\BW}[0]{\mathbf{W}}
\newcommand{\BX}[0]{\mathbf{X}}
\newcommand{\BY}[0]{\mathbf{Y}}
\newcommand{\BZ}[0]{\mathbf{Z}}

\newcommand{\Bzero}[0]{\mathbf{0}}
\newcommand{\Btheta}[0]{\boldsymbol{\theta}}
\newcommand{\Btau}[0]{\boldsymbol{\tau}}
\newcommand{\Bomega}[0]{\boldsymbol{\omega}}

%
% shorthand for unit vectors
%
\newcommand{\acap}[0]{\hat{\Ba}}
\newcommand{\bcap}[0]{\hat{\Bb}}
\newcommand{\ccap}[0]{\hat{\Bc}}
\newcommand{\dcap}[0]{\hat{\Bd}}
\newcommand{\ecap}[0]{\hat{\Be}}
\newcommand{\fcap}[0]{\hat{\Bf}}
\newcommand{\gcap}[0]{\hat{\Bg}}
\newcommand{\hcap}[0]{\hat{\Bh}}
\newcommand{\icap}[0]{\hat{\Bi}}
\newcommand{\jcap}[0]{\hat{\Bj}}
\newcommand{\kcap}[0]{\hat{\Bk}}
\newcommand{\lcap}[0]{\hat{\Bl}}
\newcommand{\mcap}[0]{\hat{\Bm}}
\newcommand{\ncap}[0]{\hat{\Bn}}
\newcommand{\ocap}[0]{\hat{\Bo}}
\newcommand{\pcap}[0]{\hat{\Bp}}
\newcommand{\qcap}[0]{\hat{\Bq}}
\newcommand{\rcap}[0]{\hat{\Br}}
\newcommand{\scap}[0]{\hat{\Bs}}
\newcommand{\tcap}[0]{\hat{\Bt}}
\newcommand{\ucap}[0]{\hat{\Bu}}
\newcommand{\vcap}[0]{\hat{\Bv}}
\newcommand{\wcap}[0]{\hat{\Bw}}
\newcommand{\xcap}[0]{\hat{\Bx}}
\newcommand{\ycap}[0]{\hat{\By}}
\newcommand{\zcap}[0]{\hat{\Bz}}
\newcommand{\thetacap}[0]{\hat{\Btheta}}

%
% to write R^n and C^n in a distinguishable fashion.  Perhaps change this
% to the double lined characters upon figuring out how to do so.
%
\newcommand{\C}[1]{$\mathbb{C}^{#1}$}
\newcommand{\R}[1]{$\mathbb{R}^{#1}$}

%
% various generally useful helpers
%

% derivative of #1 wrt. #2:
\newcommand{\D}[2] {\frac {d#2} {d#1}}

\newcommand{\inv}[1]{\frac{1}{#1}}
\newcommand{\cross}[0]{\times}

\newcommand{\abs}[1]{\lvert{#1}\rvert}
\newcommand{\norm}[1]{\lVert{#1}\rVert}
\newcommand{\innerprod}[2]{\langle{#1}, {#2}\rangle}
\newcommand{\dotprod}[2]{{#1} \cdot {#2}}
\newcommand{\bdotprod}[2]{\left({#1} \cdot {#2}\right)}
\newcommand{\crossprod}[2]{{#1} \cross {#2}}
\newcommand{\tripleprod}[3]{\dotprod{\left(\crossprod{#1}{#2}\right)}{#3}}

\DeclareMathOperator{\Proj}{Proj}
\DeclareMathOperator{\Span}{span}
\DeclareMathOperator{\Sgn}{sgn}
\DeclareMathOperator{\Area}{Area}
\DeclareMathOperator{\Volume}{Volume}

%
% A few miscellaneous things specific to this document
%
\newcommand{\crossop}[1]{\crossprod{#1}{}}

% R2 vector.
\newcommand{\VectorTwo}[2]{
\begin{bmatrix}
 {#1} \\
 {#2}
\end{bmatrix}
}

\newcommand{\VectorN}[1]{
\begin{bmatrix}
{#1}_1 \\
{#1}_2 \\
\vdots \\
{#1}_N \\
\end{bmatrix}
}

\newcommand{\DETuvij}[4]{
\begin{vmatrix}
 {#1}_{#3} & {#1}_{#4} \\
 {#2}_{#3} & {#2}_{#4}
\end{vmatrix}
}

\newcommand{\DETuvwijk}[6]{
\begin{vmatrix}
 {#1}_{#4} & {#1}_{#5} & {#1}_{#6} \\
 {#2}_{#4} & {#2}_{#5} & {#2}_{#6} \\
 {#3}_{#4} & {#3}_{#5} & {#3}_{#6}
\end{vmatrix}
}

\newcommand{\DETuvwxijkl}[8]{
\begin{vmatrix}
 {#1}_{#5} & {#1}_{#6} & {#1}_{#7} & {#1}_{#8} \\
 {#2}_{#5} & {#2}_{#6} & {#2}_{#7} & {#2}_{#8} \\
 {#3}_{#5} & {#3}_{#6} & {#3}_{#7} & {#3}_{#8} \\
 {#4}_{#5} & {#4}_{#6} & {#4}_{#7} & {#4}_{#8} \\
\end{vmatrix}
}

%\newcommand{\DETuvwxyijklm}[10]{
%\begin{vmatrix}
% {#1}_{#6} & {#1}_{#7} & {#1}_{#8} & {#1}_{#9} & {#1}_{#10} \\
% {#2}_{#6} & {#2}_{#7} & {#2}_{#8} & {#2}_{#9} & {#2}_{#10} \\
% {#3}_{#6} & {#3}_{#7} & {#3}_{#8} & {#3}_{#9} & {#3}_{#10} \\
% {#4}_{#6} & {#4}_{#7} & {#4}_{#8} & {#4}_{#9} & {#4}_{#10} \\
% {#5}_{#6} & {#5}_{#7} & {#5}_{#8} & {#5}_{#9} & {#5}_{#10}
%\end{vmatrix}
%}

% R3 vector.
\newcommand{\VectorThree}[3]{
\begin{bmatrix}
 {#1} \\
 {#2} \\
 {#3}
\end{bmatrix}
}



\author{Peeter Joot}
\email{peeter.joot@gmail.com}


\chapter{hamiltonian}
\label{chap:hamiltonian}
%\useCCL
\blogpage{http://sites.google.com/site/peeterjoot/math2009/hamiltonian.pdf}
\date{Nov 15, 2009}
\revisionInfo{$RCSfile: hr.tex,v $ Last $Revision: 1.6 $ $Date: 2009/12/03 03:24:40 $}

\beginArtWithToc
%\beginArtNoToc

\section{Relativistic force free Hamiltonian.}

Having considered a number of Newtonian mechanical systems, lets move on to the force free relativistic Lagrangian in preparation for considering electromagnetic forces in a Hamiltonian context.

\subsection{Covariant Lagrangian.  Force free equations of motion.}

We have two options for the relativistic Lagrangian (and perhaps more).  One of which is the covariant form with exactly the same structure as a purely kinetic Newtonian Lagrangian $m \Bv^2/2$, except for replacement of the velocity dot product with a four vector dot product of proper time velocities.  That is

\begin{align}\label{eqn:hamiltonian:qqq1}
\LL &= \inv{2} m \xdot^\mu \xdot_\mu \\
\xdot^\mu &= \frac{d x^\mu}{d \tau},
\end{align}

where the corresponding covariant action is an integral over proper time

\begin{align}\label{eqn:hamiltonian:qqq2}
S = \int d\tau \LL.
\end{align}

Varying the proper time action, and picking the upper index coordinates as our generalized coordinates, we get for the Euler-Lagrange equations 

\begin{align}\label{eqn:hamiltonian:qqq3}
\frac{d}{d \tau} \frac{d \LL}{d \xdot^\alpha} = \frac{d \LL}{d x^\alpha}.
\end{align}

This is the setup required to evaluate the Euler-Lagrange equations.  First for the left hand side we want the conjugate momenta

\begin{align*}
P_\mu 
&= \frac{d \LL}{d \xdot^\alpha} \\
&= m \xdot_\mu.
\end{align*}

Our equations of motion, without yet using the Hamiltonian equations are just

\begin{align}\label{eqn:hamiltonian:qqq2x}
\frac{d}{d\tau} \left( \xdot_\mu \right) = 0.
\end{align}

Provided the covariant Lagrangian has no explicit proper time dependence, following the same procedure as in the non-relativistic case, we find

\begin{align}\label{eqn:hamiltonian:foo2r}
\frac{d}{d\tau} \left( \dot{x}^\mu \PD{\dot{x}^\mu}{\LL} -\LL \right) = 0.
\end{align}

We therefore have a constant for the system associated with proper time translation

\begin{align}\label{eqn:hamiltonian:foo2q}
H = \dot{x}^\mu \PD{\dot{x}^\mu}{\LL} -\LL.
\end{align}

Taking partials we find easily the covariant form of the Hamiltonian equations

\begin{align}\label{eqn:hamiltonian:qqq30}
\PD{P_\mu}{H} &= \dot{x}^\mu \\
\PD{x^\mu}{H} &= -\dot{P}_\mu 
\end{align}

Returning to the specifics of our covariant Lagrangian, the Hamiltonian is unsurprisingly just the Lagrangian

\begin{align}\label{eqn:hamiltonian:foo2s}
H = \inv{2} m \dot{x}^\mu \dot{x}_\mu = \LL = \inv{2m} P^\mu P_\mu.
\end{align}

Writing this out explicitly in space time coordinates we have

\begin{align}\label{eqn:hamiltonian:qqq10}
H = \inv{2} m
\left( \frac{dt}{d\tau} \right)^2
(c^2 - \Bv^2)
\end{align}

In particular in the rest frame where $t = \tau$ and $\Bv = 0$, we have

\begin{align}\label{eqn:hamiltonian:qqq11}
H = \inv{2} m c^2.
\end{align}

An immediate consequence, without even evaluating the Hamiltonian equations, is that our $\gamma$ factor falls out

\begin{align}\label{eqn:hamiltonian:qqq12}
\frac{dt}{d\tau} = \inv{\sqrt{1 - \Bv^2/c^2}}.
\end{align}

This also allows the equations of motion \eqnref{eqn:hamiltonian:qqq2x} to be transformed from covariant to a spacetime form

%FIXME.
%m \frac{d}{d\tau}

\subsection{An aside.}

The fact that the Hamiltonian is a constant of motion also follows directly just by taking derivatives and applying the Euler-Lagrange equations.  That is

\begin{align*}
\frac{d}{dt} \left( \inv{2} m \dot{x}^\mu \dot{x}_\mu \right) 
=
\dot{x}^\mu \underbrace{\frac{d}{dt} \left( m \dot{x}_\mu \right)}_{=0} \\
\end{align*}

\subsection{Applying the Hamiltonian equations.}

Okay.  Let us apply the Hamiltonian equations \eqnref{eqn:hamiltonian:qqq30}.  For our force free Lagrangian where we have no $x^\mu$ dependence, we recover \eqnref{eqn:hamiltonian:qqq2x} immediately by evaluating

\begin{align}\label{eqn:hamiltonian:qqq30b}
\PD{x^\mu}{H} &= -\dot{P}_\mu,
\end{align}

since the left hand side is zero.

%\EndArticle
\EndNoBibArticle
