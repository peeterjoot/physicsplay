%
% Copyright � 2012 Peeter Joot.  All Rights Reserved.
% Licenced as described in the file LICENSE under the root directory of this GIT repository.
%
\newcommand{\authorname}{Peeter Joot}
\newcommand{\email}{peeterjoot@protonmail.com}
\newcommand{\basename}{FIXMEbasenameUndefined}
\newcommand{\dirname}{notes/FIXMEdirnameUndefined/}

\renewcommand{\basename}{modernOpticsLecture19}
\renewcommand{\dirname}{notes/phy485/}
\newcommand{\keywords}{Optics, PHY485H1F}
\newcommand{\authorname}{Peeter Joot}
\newcommand{\onlineurl}{http://sites.google.com/site/peeterjoot2/math2013/\basename.pdf}
\newcommand{\sourcepath}{\dirname\basename.tex}
\newcommand{\generatetitle}[1]{\chapter{#1}}

\newcommand{\vcsinfo}{%
\section*{}
\noindent{\color{DarkOliveGreen}{\rule{\linewidth}{0.1mm}}}
\paragraph{Document version}
%\paragraph{\color{Maroon}{Document version}}
{
\small
\begin{itemize}
\item Available online at:\\ 
\href{\onlineurl}{\onlineurl}
\item Git Repository: \input{./.revinfo/gitRepo.tex}
\item Source: \sourcepath
\item last commit: \input{./.revinfo/gitCommitString.tex}
\item commit date: \input{./.revinfo/gitCommitDate.tex}
\end{itemize}
}
}

%\PassOptionsToPackage{dvipsnames,svgnames}{xcolor}
\PassOptionsToPackage{square,numbers}{natbib}
\documentclass{scrreprt}

\usepackage[left=2cm,right=2cm]{geometry}
\usepackage[svgnames]{xcolor}
\usepackage{peeters_layout}

\usepackage{natbib}

\usepackage[
colorlinks=true,
bookmarks=false,
pdfauthor={\authorname, \email},
backref 
]{hyperref}

% http://tex.stackexchange.com/questions/75773/how-to-reference-problems-by-the-text-label-in-an-exercise-envioronment
\usepackage[english]{cleveref}
\crefname{Exercise}{exercise}{exercises}
\Crefname{Exercise}{Exercise}{Exercises}

\RequirePackage{titlesec}
\RequirePackage{ifthen}

% http://stackoverflow.com/questions/4932910/date-in-the-tabular-environment
\makeatletter
\let\insertdate\@date
\makeatother

\titleformat{\chapter}[display]
{\bfseries\Large}
{\color{DarkSlateGrey}\filleft \authorname
\ifthenelse{\isundefined{\studentnumber}}{}{\\ \studentnumber}
\ifthenelse{\isundefined{\email}}{}{\\ \email}
\ifthenelse{\isundefined{\dateintitle}}{}{\\ \insertdate}
%\ifthenelse{\isundefined{\coursename}}{}{\\ \coursename} % put in title instead.
}
{4ex}
{\color{DarkOliveGreen}{\titlerule}\color{Maroon}
\vspace{2ex}%
\filright}
[\vspace{2ex}%
\color{DarkOliveGreen}\titlerule
]

\newcommand{\beginArtWithToc}[0]{\begin{document}\tableofcontents}
\newcommand{\beginArtNoToc}[0]{\begin{document}}
\newcommand{\EndNoBibArticle}[0]{\end{document}}
\newcommand{\EndArticle}[0]{\bibliography{Bibliography}\bibliographystyle{plainnat}\end{document}}

% 
%\newcommand{\citep}[1]{\cite{#1}}

\colorSectionsForArticle



\usepackage[draft]{fixme}
\fxusetheme{color}

\beginArtNoToc
\generatetitle{PHY485H1F Modern Optics.  Lecture 19: XXX.  Taught by Prof.\ Joseph Thywissen}
%\chapter{XXX}
\label{chap:modernOpticsLecture19}

%\section{Disclaimer}
%
%Peeter's lecture notes from class.  May not be entirely coherent.

\section{Gaussian beams (cont.)}

\fxwarning{review lecture 19}{work through this lecture in detail.}

We are going to start with the paraxial wave equation for a quadratic index profile

\begin{dmath}\label{eqn:modernOpticsLecture19:20}
\spacegrad_{\mathrm{T}}^2 u + 2 i k \PD{u}{z} = 0
\end{dmath}

where our wave function was

\Psi = \Psi_0 u(x, y, z) e^{i k z}

F0

Try a Gaussian solution

\begin{dmath}\label{eqn:modernOpticsLecture19:40}
u = \exp\left( i p(z) + i \frac{ k r^2}{2 q(z)} \right)
\end{dmath}

Find for $q = q(z)$, $p = p(z)$ 

\begin{dmath}\label{eqn:modernOpticsLecture19:60}
-\frac{k^2}{q^2} r^2 + 2 i \frac{k}{q} - k^2 r^2 \left( \inv{q} \right)' - 2 k r' - k k_2 r^2 = 0.
\end{dmath}

\begin{dmath}\label{eqn:modernOpticsLecture19:80}
\inv{q^2} + \frac{d}{dz} \left( \inv{q} \right) + \frac{k_2}{k} = 0
\end{dmath}

\begin{dmath}\label{eqn:modernOpticsLecture19:100}
\frac{dp'}{dz} = \frac{i}{q}
\end{dmath}

\fxwarning{where did that come from?}

Now kill off the $k_2$ term

\begin{dmath}\label{eqn:modernOpticsLecture19:120}
0 
=
\inv{q^2} + \frac{d}{dz} \left( \inv{q} \right) + \cancel{\frac{k_2}{k}} 
=
\inv{q^2} - \inv{q^2} q'
=
\inv{q^2}( 1 - q')
\end{dmath}

so that 

\begin{dmath}\label{eqn:modernOpticsLecture19:140}
q = z + \text{constant} \equiv z + q_0
\end{dmath}

Choosing $p(0) = 0$, then

\begin{dmath}\label{eqn:modernOpticsLecture19:160}
u_{00}(x, y, z = 0) = \exp \left( -\frac{r^2}{w_0^2} \right)
\end{dmath}

We call $w_0$ the ``waste'' or the \underline{beam waste}.

Define

\begin{dmath}\label{eqn:modernOpticsLecture19:180}
q_0 = -i \frac{w_0^2 k}{2} = -i \pi \frac{w_0^2}{\lambda} = -i z_0
\end{dmath}

\begin{dmath}\label{eqn:modernOpticsLecture19:200}
z_0 = \pi w_0^2\lambda
\end{dmath}

This is the \underline{Raleigh range}.

Observe that $p(z)$ can be written

\begin{dmath}\label{eqn:modernOpticsLecture19:220}
p(z) 
= i \ln \left( 1 + i \frac{z}{z_0} \right)
= i \ln \left( 1 + i \frac{q + i z_0}{z_0} \right)
= i \ln \left( i \frac{q}{z_0} \right)
\end{dmath}

Our trial solution is found to be

\begin{dmath}\label{eqn:modernOpticsLecture19:240}
u 
= \exp\left( i p(z) + i \frac{ k r^2}{2 q(z)} \right)
= \exp\left( i i \ln \left( i \frac{q}{z_0} \right) + i \frac{ k r^2}{2 q(z)} \right)
\end{dmath}

or
\begin{dmath}\label{eqn:modernOpticsLecture19:260}
\boxed{
u 
= \frac{z_0}{iq} \exp\left( i \frac{ k r^2}{2 q(z)} \right)
}
\end{dmath}

Can show \fxwarning{do it}

\begin{dmath}\label{eqn:modernOpticsLecture19:280}
\boxed{
u 
= \frac{w_0}{w(z)} \exp\left( 
-\frac{r^2}{w(z)^2} 
i \frac{ k R(z)^2}{2 q(z)} 
- i \phi(z)
\right)
}
\end{dmath}

where

\begin{dmath}\label{eqn:modernOpticsLecture19:300}
\phi(z) = \Atan \left( \frac{z}{z_0} \right).
\end{dmath}

\begin{dmath}\label{eqn:modernOpticsLecture19:320}
w(z)^2 = w_0^2 \left( 1  + \frac{z^2}{z_0^2} \right)
\end{dmath}

This is the \underline{beam radius}.

\begin{dmath}\label{eqn:modernOpticsLecture19:340}
\inv{R(z)} = \frac{z}{z^2 + z_0^2}.
\end{dmath}

This is the \underline{phase curvature}.

Note that in the process of writing this out in terms of non-complex stuff we used

\begin{dmath}\label{eqn:modernOpticsLecture19:360}
\inv{q(z)} = \inv{R(z)} + \frac{i \lambda}{\pi w(z)}
\end{dmath}

\paragraph{Some observations}

For $z \gg z_0$, $R \rightarrow z$, $w(z) \rightarrow \frac{w_0 z}{z_0}$.   We see that

\begin{dmath}\label{eqn:modernOpticsLecture19:380}
u \rightarrow \frac{z}{z_0} \exp \left( 
-\frac{r^2}{w^2} + i \frac{k }{2 z} r^2
\right)
\end{dmath}

Compare to wave emitted by a point source.

F1

\begin{dmath}\label{eqn:modernOpticsLecture19:400}
\Psi \sim \inv{R} e^{i k R}
\end{dmath}

but 

\begin{dmath}\label{eqn:modernOpticsLecture19:420}
R^2 = z^2 + r^2
\end{dmath}

Taylor expanding to first order

\begin{dmath}\label{eqn:modernOpticsLecture19:440}
R \approx z + \inv{2 z} r^2
\end{dmath}

we've got

\begin{dmath}\label{eqn:modernOpticsLecture19:460}
\Psi \sim \inv{z} \exp\left( i k z + i k \frac{r^2}{2z} \right)
\end{dmath}

So if we are looking at a point source slightly off axis, we have what looks like a Gaussian beam.

\paragraph{Waste anglular dependence}

F3

With 

\begin{dmath}\label{eqn:modernOpticsLecture19:480}
w(z) 
= w_0 \sqrt{ 1 + \frac{z^2}{z_0^2} }
\approx \frac{w_0}{z_0} z
= \Theta_{\mathrm{div}} z
\end{dmath}

where 

\begin{dmath}\label{eqn:modernOpticsLecture19:500}
\Theta_{\mathrm{div}} =
\frac{w_0}{\pi w_0^2} \lambda
\end{dmath}

%\EndArticle
\EndNoBibArticle
