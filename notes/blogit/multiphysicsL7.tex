%
% Copyright � 2014 Peeter Joot.  All Rights Reserved.
% Licenced as described in the file LICENSE under the root directory of this GIT repository.
%
% for template copy, run:
%
% ~/bin/ct multiphysicsL1  multiphysicsLectureN tl1
%
\newcommand{\authorname}{Peeter Joot}
\newcommand{\email}{peeterjoot@protonmail.com}
\newcommand{\basename}{FIXMEbasenameUndefined}
\newcommand{\dirname}{notes/FIXMEdirnameUndefined/}

\renewcommand{\basename}{multiphysicsL7}
\renewcommand{\dirname}{notes/ece1254/}
\newcommand{\keywords}{Condensed matter physics, ECE1254H}
\newcommand{\authorname}{Peeter Joot}
\newcommand{\onlineurl}{http://sites.google.com/site/peeterjoot2/math2013/\basename.pdf}
\newcommand{\sourcepath}{\dirname\basename.tex}
\newcommand{\generatetitle}[1]{\chapter{#1}}

\newcommand{\vcsinfo}{%
\section*{}
\noindent{\color{DarkOliveGreen}{\rule{\linewidth}{0.1mm}}}
\paragraph{Document version}
%\paragraph{\color{Maroon}{Document version}}
{
\small
\begin{itemize}
\item Available online at:\\ 
\href{\onlineurl}{\onlineurl}
\item Git Repository: \input{./.revinfo/gitRepo.tex}
\item Source: \sourcepath
\item last commit: \input{./.revinfo/gitCommitString.tex}
\item commit date: \input{./.revinfo/gitCommitDate.tex}
\end{itemize}
}
}

%\PassOptionsToPackage{dvipsnames,svgnames}{xcolor}
\PassOptionsToPackage{square,numbers}{natbib}
\documentclass{scrreprt}

\usepackage[left=2cm,right=2cm]{geometry}
\usepackage[svgnames]{xcolor}
\usepackage{peeters_layout}

\usepackage{natbib}

\usepackage[
colorlinks=true,
bookmarks=false,
pdfauthor={\authorname, \email},
backref 
]{hyperref}

% http://tex.stackexchange.com/questions/75773/how-to-reference-problems-by-the-text-label-in-an-exercise-envioronment
\usepackage[english]{cleveref}
\crefname{Exercise}{exercise}{exercises}
\Crefname{Exercise}{Exercise}{Exercises}

\RequirePackage{titlesec}
\RequirePackage{ifthen}

% http://stackoverflow.com/questions/4932910/date-in-the-tabular-environment
\makeatletter
\let\insertdate\@date
\makeatother

\titleformat{\chapter}[display]
{\bfseries\Large}
{\color{DarkSlateGrey}\filleft \authorname
\ifthenelse{\isundefined{\studentnumber}}{}{\\ \studentnumber}
\ifthenelse{\isundefined{\email}}{}{\\ \email}
\ifthenelse{\isundefined{\dateintitle}}{}{\\ \insertdate}
%\ifthenelse{\isundefined{\coursename}}{}{\\ \coursename} % put in title instead.
}
{4ex}
{\color{DarkOliveGreen}{\titlerule}\color{Maroon}
\vspace{2ex}%
\filright}
[\vspace{2ex}%
\color{DarkOliveGreen}\titlerule
]

\newcommand{\beginArtWithToc}[0]{\begin{document}\tableofcontents}
\newcommand{\beginArtNoToc}[0]{\begin{document}}
\newcommand{\EndNoBibArticle}[0]{\end{document}}
\newcommand{\EndArticle}[0]{\bibliography{Bibliography}\bibliographystyle{plainnat}\end{document}}

% 
%\newcommand{\citep}[1]{\cite{#1}}

\colorSectionsForArticle



%\usepackage{kbordermatrix}
\usepackage{algorithmic}

\beginArtNoToc
\generatetitle{ECE1254H Modeling of Multiphysics Systems.  Lecture 7: Sparse factorization.  Taught by Prof.\ Piero Triverio}
%\chapter{Sparse factorization}
\label{chap:multiphysicsL7}

\section{Disclaimer}

Peeter's lecture notes from class.  These may be incoherent and rough.

\section{Fill ins}

The problem of fill ins in LU computations arise in locations where rows and columns cross over zero positions.

Rows and columns can be permuted to deal with these.  Here is an ad-hoc permutation of rows and columns that will result in less fill ins.

\begin{equation}\label{eqn:multiphysicsL7:n}
\begin{aligned}
&\begin{bmatrix}
a & b & c & 0 \\
d & e & 0 & 0 \\
0 & f & g & 0 \\
0 & h & 0 & i \\
\end{bmatrix}
\begin{bmatrix}
x_1 \\
x_2 \\
x_3 \\
x_4
\end{bmatrix}
=
\begin{bmatrix}
b_1 \\
b_2 \\
b_3 \\
b_4
\end{bmatrix} \\
\implies &
\begin{bmatrix}
a & c & 0 & b \\
d & 0 & 0 & e \\
0 & g & 0 & f \\
0 & 0 & i & h \\
\end{bmatrix}
\begin{bmatrix}
x_1 \\
x_4 \\
x_3 \\
x_2 \\
\end{bmatrix}
=
\begin{bmatrix}
b_1 \\
b_2 \\
b_3 \\
b_4 \\
\end{bmatrix} \\
\implies &
\begin{bmatrix}
0 & a & c & b \\
0 & d & 0 & e \\
0 & 0 & g & f \\
i & 0 & 0 & h \\
\end{bmatrix}
\begin{bmatrix}
x_3 \\
x_4 \\
x_1 \\
x_2 \\
\end{bmatrix}
=
\begin{bmatrix}
b_1 \\
b_2 \\
b_3 \\
b_4 \\
\end{bmatrix} \\
\implies &
\begin{bmatrix}
i & 0 & 0 & h \\
0 & a & c & b \\
0 & d & 0 & e \\
0 & 0 & g & f \\
\end{bmatrix}
\begin{bmatrix}
x_3 \\
x_4 \\
x_1 \\
x_2 \\
\end{bmatrix}
=
\begin{bmatrix}
b_4 \\
b_1 \\
b_2 \\
b_3 \\
\end{bmatrix} \\
\implies &
\begin{bmatrix}
i & 0 & 0 & h \\
0 & c & a & b \\
0 & 0 & d & e \\
0 & g & 0 & f \\
\end{bmatrix}
\begin{bmatrix}
x_3 \\
x_1 \\
x_4 \\
x_2 \\
\end{bmatrix}
=
\begin{bmatrix}
b_4 \\
b_1 \\
b_2 \\
b_3 \\
\end{bmatrix} \\
\end{aligned}
\end{equation}

\sectionAndIndex{Markowitz product}

To facilitate such permutations the Markowitz product that estimates the amount of fill in required.  This is

\makedefinition{Markowitz product}{dfn:multiphysicsL7:20}{
\begin{equation*}
\begin{aligned}
		  \text{Markowitz product} =
		  &\lr{\text{Non zeros in unfactored part of Row -1}} \times \\
&\lr{\text{Non zeros in unfactored part of Col -1}}
\end{aligned}
\end{equation*}
}

\section{On slides}

Covered:
\begin{itemize}
\item Markowitz product.  FIXME: Compute some examples.
\item Markowitz reordering.  FIXME: Work an example.
\item Graph approaches.  Delete the most connected nodes.  Claimed that the Markowitz products are the squares of the node degrees.  FIXME: work example to come to terms with this.
\end{itemize}

\section{Summary of factorization costs}

\paragraph{LU (dense)}

\begin{itemize}
   \item cost: \( O(n^3) \)
   \item cost depends only on size
\end{itemize}

Computation can be affordable up to a few thousand elements.

\paragraph{LU (sparse)}

\begin{itemize}
   \item cost: Diagonal and tridiagonal are \( O(n) \), but we can have up to \( O(n^3) \) depending on sparsity and the method of dealing with the sparsity.
   \item cost depends on size and sparsity
\end{itemize}

Computation can be affordable up to a few million elements.

\paragraph{Iterative methods}

Can be cheap if done right.  Convergence requires careful preconditioning.

\section{Iterative methods}

Suppose that we have an initial guess \( \overbar{x}_0 \).   Iterative methods are generally of the form

\begin{algorithmic}
\REPEAT
\STATE \(\overbar{r} = \overbar{b} - M \overbar{x}_i\)
\UNTIL{\(\Norm{\overbar{r}} < \epsilon \).}
\end{algorithmic}

The difference \( \overbar{r} \) is called the residual.  For as long as it is bigger than desired, continue improving the estimate \( \overbar{x}_i \).

The matrix vector product \( M \overbar{x}_i \), if dense, is of \( O(n^2) \).  Suppose, for example, that we can perform the iteration in ten iterations.  If the matrix is dense, we can have \( 10 O(n^2) \) performance.  If sparse, this can be worse than just direct computation.

\section{Gradient method}

This is a method for iterative solution of the equation \( M \overbar{x} = \overbar{b} \).

This requires symmetric positive definite matrix \( M = M^\T \), with \( M > 0 \).

We introduce an \textAndIndex{energy function}

\begin{equation}\label{eqn:multiphysicsL7:60}
\Psi(\overbar{y}) \equiv \inv{2} \overbar{y}^\T M \overbar{y} - \overbar{y}^\T \overbar{b}
\end{equation}

For a two variable system this is illustrated in \cref{fig:lecture7:lecture7Fig1}.

\imageFigure{../../figures/ece1254/lecture7Fig1}{Positive definite energy function}{fig:lecture7:lecture7Fig1}{0.3}

Claim: This energy function has a minimum for 

\begin{equation}\label{eqn:multiphysicsL7:80}
\overbar{y} = M^{-1} \overbar{b} = \overbar{x}.
\end{equation}

FIXME: justify.

This is called the gradient method because the gradient moves us along the path of steepest descent towards the minimum if it exists.

The method is

\begin{equation}\label{eqn:multiphysicsL7:100}
\overbar{x}^{(k+1)} = \overbar{x}^{(k)} + 
\mathLabelBox
{ \alpha_k }{step size}
\mathLabelBox
[
   labelstyle={below of=m\themathLableNode, below of=m\themathLableNode}
]
{
\overbar{d}^{(k)}
}{direction},
\end{equation}

where the direction is

\begin{equation}\label{eqn:multiphysicsL7:120}
\overbar{d}^{(k)} = - \spacegrad \Phi = r^{(k)}.
\end{equation}

\paragraph{Optimal step size}

Note that for the minimization of \( \Phi \lr{ \overbar{x}^{(k+1)} } \), we note

\begin{dmath}\label{eqn:multiphysicsL7:140}
\Phi \lr{ \overbar{x}^{(k+1)} }
= \Phi\lr{ \overbar{x}^{(k)} + \alpha_k \overbar{d}^{(k)} }
= 
\inv{2} 
\lr{ \overbar{x}^{(k)} + \alpha_k \overbar{d}^{(k)} }^\T
M
\lr{ \overbar{x}^{(k)} + \alpha_k \overbar{d}^{(k)} }
-
\lr{ \overbar{x}^{(k)} + \alpha_k \overbar{d}^{(k)} }^\T \overbar{b}
\end{dmath}

If we take the derivative of this with respect to \( \alpha_k \) to find the minimum, we find (after work left to the user)

\begin{equation}\label{eqn:multiphysicsL7:160}
\alpha_k 
= \frac{
\lr{\overbar{r}^{(k)}}^\T 
\overbar{r}^{(k)}
}{
\lr{\overbar{r}^{(k)}}^\T 
M
\overbar{r}^{(k)}
}
\end{equation}

We will see that this method is not optimal when we pick one direction and keep going down that path.

\section{Definitions and theorems}

\makedefinition{Positive (negative) definite}{dfn:multiphysicsL7:40}{
   A matrix \( M \) is positive (negative) definite, denoted \( M > 0 (<0) \) if \( \overbar{y}^\T M \overbar{y} > 0 (<0), \quad \forall \overbar{y} \).

   If a matrix is neither positive, nor negative definite, it is called indefinite.
}

\maketheorem{Positive (negative) definite}{thm:multiphysicsL7:60}{
   A symmetric matrix \( M > 0 (<0)\) iff \( \lambda_i > 0 (<0)\) for all eigenvalues \( \lambda_i \), or is indefinite iff its eigenvalues \( \lambda_i \) are of mixed sign.
}

%\EndArticle
\EndNoBibArticle
