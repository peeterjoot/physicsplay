\makeproblem{Solar interference}{modernOptics:problemSet2:3}{ 
Let's consider the prospects for interference fringes using direct sunlight. 

\makesubproblem{
Consider the sun to be a disc subtending a 0.5 degree diameter. Using the van Cittert-Zernike theorem, find the mutual coherence function on earth from sunlight. How close would two pinholes need to be to see a 50\% visibility interference pattern behind them? For this part, make the (wrong) assumption that the sun is a quasimonochromatic source centred at $\lambda =$500 nm.
}{modernOptics:problemSet2:3a}
\makesubproblem{
Another difficulty with the sun (when using it as a source for interferometry) is that it is spectrally broadband. As with any blackbody, a typical spectral width is $\Delta \omega = k_B T/\hbar$, where $T \approx 5000$\,K for the sun. {\em Estimate} the effect of finite coherence time on fringe visibility, and make a qualitative sketch of the fringe pattern you would expect to observe.
}{modernOptics:problemSet2:3b}
\makesubproblem{
In what situations is the spectral width a more severe problem for visibility than the spatial coherence? 
}{modernOptics:problemSet2:3c}
}

\makeanswer{modernOptics:problemSet2:3}{ 
\makesubanswer{TODO.}{modernOptics:problemSet2:3a}
\makesubanswer{TODO.}{modernOptics:problemSet2:3b}
\makesubanswer{TODO.}{modernOptics:problemSet2:3c}
}
