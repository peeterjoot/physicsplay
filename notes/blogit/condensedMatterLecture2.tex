%
% Copyright � 2013 Peeter Joot.  All Rights Reserved.
% Licenced as described in the file LICENSE under the root directory of this GIT repository.
%
\newcommand{\authorname}{Peeter Joot}
\newcommand{\email}{peeterjoot@protonmail.com}
\newcommand{\basename}{FIXMEbasenameUndefined}
\newcommand{\dirname}{notes/FIXMEdirnameUndefined/}

\renewcommand{\basename}{condensedMatterLecture2}
\renewcommand{\dirname}{notes/phy487/}
\newcommand{\keywords}{Condensed matter physics, PHY487H1S, Covalent bonding, Ionic bonding, orbital, promotion, hybridization, nearest neighbour, binding energy, lattice}
\newcommand{\authorname}{Peeter Joot}
\newcommand{\onlineurl}{http://sites.google.com/site/peeterjoot2/math2013/\basename.pdf}
\newcommand{\sourcepath}{\dirname\basename.tex}
\newcommand{\generatetitle}[1]{\chapter{#1}}

\newcommand{\vcsinfo}{%
\section*{}
\noindent{\color{DarkOliveGreen}{\rule{\linewidth}{0.1mm}}}
\paragraph{Document version}
%\paragraph{\color{Maroon}{Document version}}
{
\small
\begin{itemize}
\item Available online at:\\ 
\href{\onlineurl}{\onlineurl}
\item Git Repository: \input{./.revinfo/gitRepo.tex}
\item Source: \sourcepath
\item last commit: \input{./.revinfo/gitCommitString.tex}
\item commit date: \input{./.revinfo/gitCommitDate.tex}
\end{itemize}
}
}

%\PassOptionsToPackage{dvipsnames,svgnames}{xcolor}
\PassOptionsToPackage{square,numbers}{natbib}
\documentclass{scrreprt}

\usepackage[left=2cm,right=2cm]{geometry}
\usepackage[svgnames]{xcolor}
\usepackage{peeters_layout}

\usepackage{natbib}

\usepackage[
colorlinks=true,
bookmarks=false,
pdfauthor={\authorname, \email},
backref 
]{hyperref}

% http://tex.stackexchange.com/questions/75773/how-to-reference-problems-by-the-text-label-in-an-exercise-envioronment
\usepackage[english]{cleveref}
\crefname{Exercise}{exercise}{exercises}
\Crefname{Exercise}{Exercise}{Exercises}

\RequirePackage{titlesec}
\RequirePackage{ifthen}

% http://stackoverflow.com/questions/4932910/date-in-the-tabular-environment
\makeatletter
\let\insertdate\@date
\makeatother

\titleformat{\chapter}[display]
{\bfseries\Large}
{\color{DarkSlateGrey}\filleft \authorname
\ifthenelse{\isundefined{\studentnumber}}{}{\\ \studentnumber}
\ifthenelse{\isundefined{\email}}{}{\\ \email}
\ifthenelse{\isundefined{\dateintitle}}{}{\\ \insertdate}
%\ifthenelse{\isundefined{\coursename}}{}{\\ \coursename} % put in title instead.
}
{4ex}
{\color{DarkOliveGreen}{\titlerule}\color{Maroon}
\vspace{2ex}%
\filright}
[\vspace{2ex}%
\color{DarkOliveGreen}\titlerule
]

\newcommand{\beginArtWithToc}[0]{\begin{document}\tableofcontents}
\newcommand{\beginArtNoToc}[0]{\begin{document}}
\newcommand{\EndNoBibArticle}[0]{\end{document}}
\newcommand{\EndArticle}[0]{\bibliography{Bibliography}\bibliographystyle{plainnat}\end{document}}

% 
%\newcommand{\citep}[1]{\cite{#1}}

\colorSectionsForArticle



\usepackage{mhchem}

\beginArtNoToc
\generatetitle{PHY487H1S Condensed Matter Physics.  Lecture 2: Bonding and lattice structures.  Taught by Prof.\ Stephen Julian}
%\chapter{Bonding and lattice structures}
\label{chap:condensedMatterLecture2}

\section{Disclaimer}

Peeter's lecture notes from class.  May not be entirely coherent.

\section{Covalent bonding (cont.)}

Bonding where electrons are shared between materials, forming between partially filled orbitals on small atoms.  Example \ce{H_2}.

Only half filled orbitals (eg. $2 p_z^1$) form covalent bonds.  Two shared electrons in bonding orbitals.  We need small, directional orbitals.  We find this sort of bonding in the upper triangular segment of the periodic table

F1

An example of such a covalent bond is that of two $2p_z^1$ orbitals of Floride \ce{F}.

With Florine we cannot make a covalently bonded solid, since there are no orbitals left over for bonding with anything else.

\paragraph{Florine solid?}

Can we get a Florine solid with promotion of two $2 p$ states to $3s^2$, then have three orbitals left for bonding?

Probably, but the energy cost of doing so may be exorbitant, and could require high pressures.  This is more likely with (FIXME) since the difference between the $4s^2$ ? and ? states is less.

\paragraph{First important cases, \ce{C}, \ce{Si}, \ce{Ge}}

M3

Looks like two bonds form.  Normally carbon is 3 or 4 fold coordinated.  Two such mechanisms for carbon bonding are \underlineAndIndex{promotion} and \underlineAndIndex{hybridization}

See: \citep{ibach2009solid} \S 6-9

F4:

Turns out that 

\begin{equation}\label{eqn:condensedMatterLecture2:n}
E_{\text{4 bonds}} + E_{\text{promotion}} < E_{\text{2 bonds}}
\end{equation}

Then make linear combinations of $2s + 2p$ orbitals => highly directional compact orbitals, perfect for covalent bonding.

F5

Covalent bonds are some of the strongest.  For example, for diamond we have a melting point

\begin{equation}\label{eqn:condensedMatterLecture2:n}
T_m \sim 4000 K
\end{equation}

and has the highest hardness of any material.

\section{Ionic bonding}

Here we are combining different atoms, especially the left and right hand sides of the period table.

F6

Examples: \ce{NaCl}, \ce{KF}, \ce{CsCl}, \ce{Li_2 O}, \ce{CaO}

\paragraph{\ce{NaCl}}

\ce{Na} has 1 weakly bound 3s electron.  \ce{Cl} has one vacency in its $3p$ shell.

\paragraph{energetics}

F7

F8

F9

Solid \ce{NaCl}  (See: 02_lecture.pdf, and \citep{ibach2009solid} fig 1.6)

Have 2 interpenetrating free lattices.  Each \ce{Na+ Cl-} has 6 \underlineAndIndex{nn} (nearest neighbour) \ce{Cl- Na+}.  Binding energy 7.95 eV/pair.

\paragraph{CsCl}

The \ce{CsCl} strucure.  \ce{Cl} on corners.  \ce{Cs} in centre of cube.  8 nn.  Better than \ce{NaCl} structure.  But, \ce{Na+} is small, so in the \ce{CsCl} (where \ce{Cs} is big compared to \ce{Na}) structure, the next nn (nnn) \ce{Cl-} would touch.  There's a strong Coloumb repulsion.

\EndArticle
