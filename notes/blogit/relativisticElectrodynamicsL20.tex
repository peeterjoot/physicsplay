%
% Copyright � 2015 Peeter Joot.  All Rights Reserved.
% Licenced as described in the file LICENSE under the root directory of this GIT repository.
%
\documentclass[]{eliblog}

\usepackage{amsmath}
\usepackage{mathpazo}

%
% shorthand for bold symbols, convenient for vectors and matrices
%
\newcommand{\Ba}[0]{\mathbf{a}}
\newcommand{\Bb}[0]{\mathbf{b}}
\newcommand{\Bc}[0]{\mathbf{c}}
\newcommand{\Bd}[0]{\mathbf{d}}
\newcommand{\Be}[0]{\mathbf{e}}
\newcommand{\Bf}[0]{\mathbf{f}}
\newcommand{\Bg}[0]{\mathbf{g}}
\newcommand{\Bh}[0]{\mathbf{h}}
\newcommand{\Bi}[0]{\mathbf{i}}
\newcommand{\Bj}[0]{\mathbf{j}}
\newcommand{\Bk}[0]{\mathbf{k}}
\newcommand{\Bl}[0]{\mathbf{l}}
\newcommand{\Bm}[0]{\mathbf{m}}
\newcommand{\Bn}[0]{\mathbf{n}}
\newcommand{\Bo}[0]{\mathbf{o}}
\newcommand{\Bp}[0]{\mathbf{p}}
\newcommand{\Bq}[0]{\mathbf{q}}
\newcommand{\Br}[0]{\mathbf{r}}
\newcommand{\Bs}[0]{\mathbf{s}}
\newcommand{\Bt}[0]{\mathbf{t}}
\newcommand{\Bu}[0]{\mathbf{u}}
\newcommand{\Bv}[0]{\mathbf{v}}
\newcommand{\Bw}[0]{\mathbf{w}}
\newcommand{\Bx}[0]{\mathbf{x}}
\newcommand{\By}[0]{\mathbf{y}}
\newcommand{\Bz}[0]{\mathbf{z}}
\newcommand{\BA}[0]{\mathbf{A}}
\newcommand{\BB}[0]{\mathbf{B}}
\newcommand{\BC}[0]{\mathbf{C}}
\newcommand{\BD}[0]{\mathbf{D}}
\newcommand{\BE}[0]{\mathbf{E}}
\newcommand{\BF}[0]{\mathbf{F}}
\newcommand{\BG}[0]{\mathbf{G}}
\newcommand{\BH}[0]{\mathbf{H}}
\newcommand{\BI}[0]{\mathbf{I}}
\newcommand{\BJ}[0]{\mathbf{J}}
\newcommand{\BK}[0]{\mathbf{K}}
\newcommand{\BL}[0]{\mathbf{L}}
\newcommand{\BM}[0]{\mathbf{M}}
\newcommand{\BN}[0]{\mathbf{N}}
\newcommand{\BO}[0]{\mathbf{O}}
\newcommand{\BP}[0]{\mathbf{P}}
\newcommand{\BQ}[0]{\mathbf{Q}}
\newcommand{\BR}[0]{\mathbf{R}}
\newcommand{\BS}[0]{\mathbf{S}}
\newcommand{\BT}[0]{\mathbf{T}}
\newcommand{\BU}[0]{\mathbf{U}}
\newcommand{\BV}[0]{\mathbf{V}}
\newcommand{\BW}[0]{\mathbf{W}}
\newcommand{\BX}[0]{\mathbf{X}}
\newcommand{\BY}[0]{\mathbf{Y}}
\newcommand{\BZ}[0]{\mathbf{Z}}

\newcommand{\Bzero}[0]{\mathbf{0}}
\newcommand{\Btheta}[0]{\boldsymbol{\theta}}
\newcommand{\Btau}[0]{\boldsymbol{\tau}}
\newcommand{\Bomega}[0]{\boldsymbol{\omega}}

%
% shorthand for unit vectors
%
\newcommand{\acap}[0]{\hat{\Ba}}
\newcommand{\bcap}[0]{\hat{\Bb}}
\newcommand{\ccap}[0]{\hat{\Bc}}
\newcommand{\dcap}[0]{\hat{\Bd}}
\newcommand{\ecap}[0]{\hat{\Be}}
\newcommand{\fcap}[0]{\hat{\Bf}}
\newcommand{\gcap}[0]{\hat{\Bg}}
\newcommand{\hcap}[0]{\hat{\Bh}}
\newcommand{\icap}[0]{\hat{\Bi}}
\newcommand{\jcap}[0]{\hat{\Bj}}
\newcommand{\kcap}[0]{\hat{\Bk}}
\newcommand{\lcap}[0]{\hat{\Bl}}
\newcommand{\mcap}[0]{\hat{\Bm}}
\newcommand{\ncap}[0]{\hat{\Bn}}
\newcommand{\ocap}[0]{\hat{\Bo}}
\newcommand{\pcap}[0]{\hat{\Bp}}
\newcommand{\qcap}[0]{\hat{\Bq}}
\newcommand{\rcap}[0]{\hat{\Br}}
\newcommand{\scap}[0]{\hat{\Bs}}
\newcommand{\tcap}[0]{\hat{\Bt}}
\newcommand{\ucap}[0]{\hat{\Bu}}
\newcommand{\vcap}[0]{\hat{\Bv}}
\newcommand{\wcap}[0]{\hat{\Bw}}
\newcommand{\xcap}[0]{\hat{\Bx}}
\newcommand{\ycap}[0]{\hat{\By}}
\newcommand{\zcap}[0]{\hat{\Bz}}
\newcommand{\thetacap}[0]{\hat{\Btheta}}

%
% to write R^n and C^n in a distinguishable fashion.  Perhaps change this
% to the double lined characters upon figuring out how to do so.
%
\newcommand{\C}[1]{$\mathbb{C}^{#1}$}
\newcommand{\R}[1]{$\mathbb{R}^{#1}$}

%
% various generally useful helpers
%

% derivative of #1 wrt. #2:
\newcommand{\D}[2] {\frac {d#2} {d#1}}

\newcommand{\inv}[1]{\frac{1}{#1}}
\newcommand{\cross}[0]{\times}

\newcommand{\abs}[1]{\lvert{#1}\rvert}
\newcommand{\norm}[1]{\lVert{#1}\rVert}
\newcommand{\innerprod}[2]{\langle{#1}, {#2}\rangle}
\newcommand{\dotprod}[2]{{#1} \cdot {#2}}
\newcommand{\bdotprod}[2]{\left({#1} \cdot {#2}\right)}
\newcommand{\crossprod}[2]{{#1} \cross {#2}}
\newcommand{\tripleprod}[3]{\dotprod{\left(\crossprod{#1}{#2}\right)}{#3}}

\DeclareMathOperator{\Proj}{Proj}
\DeclareMathOperator{\Span}{span}
\DeclareMathOperator{\Sgn}{sgn}
\DeclareMathOperator{\Area}{Area}
\DeclareMathOperator{\Volume}{Volume}

%
% A few miscellaneous things specific to this document
%
\newcommand{\crossop}[1]{\crossprod{#1}{}}

% R2 vector.
\newcommand{\VectorTwo}[2]{
\begin{bmatrix}
 {#1} \\
 {#2}
\end{bmatrix}
}

\newcommand{\VectorN}[1]{
\begin{bmatrix}
{#1}_1 \\
{#1}_2 \\
\vdots \\
{#1}_N \\
\end{bmatrix}
}

\newcommand{\DETuvij}[4]{
\begin{vmatrix}
 {#1}_{#3} & {#1}_{#4} \\
 {#2}_{#3} & {#2}_{#4}
\end{vmatrix}
}

\newcommand{\DETuvwijk}[6]{
\begin{vmatrix}
 {#1}_{#4} & {#1}_{#5} & {#1}_{#6} \\
 {#2}_{#4} & {#2}_{#5} & {#2}_{#6} \\
 {#3}_{#4} & {#3}_{#5} & {#3}_{#6}
\end{vmatrix}
}

\newcommand{\DETuvwxijkl}[8]{
\begin{vmatrix}
 {#1}_{#5} & {#1}_{#6} & {#1}_{#7} & {#1}_{#8} \\
 {#2}_{#5} & {#2}_{#6} & {#2}_{#7} & {#2}_{#8} \\
 {#3}_{#5} & {#3}_{#6} & {#3}_{#7} & {#3}_{#8} \\
 {#4}_{#5} & {#4}_{#6} & {#4}_{#7} & {#4}_{#8} \\
\end{vmatrix}
}

%\newcommand{\DETuvwxyijklm}[10]{
%\begin{vmatrix}
% {#1}_{#6} & {#1}_{#7} & {#1}_{#8} & {#1}_{#9} & {#1}_{#10} \\
% {#2}_{#6} & {#2}_{#7} & {#2}_{#8} & {#2}_{#9} & {#2}_{#10} \\
% {#3}_{#6} & {#3}_{#7} & {#3}_{#8} & {#3}_{#9} & {#3}_{#10} \\
% {#4}_{#6} & {#4}_{#7} & {#4}_{#8} & {#4}_{#9} & {#4}_{#10} \\
% {#5}_{#6} & {#5}_{#7} & {#5}_{#8} & {#5}_{#9} & {#5}_{#10}
%\end{vmatrix}
%}

% R3 vector.
\newcommand{\VectorThree}[3]{
\begin{bmatrix}
 {#1} \\
 {#2} \\
 {#3}
\end{bmatrix}
}



\author{Peeter Joot}
\email{peeter.joot@gmail.com}

%\documentclass[]{eliblogwidescreen}

\usepackage{amsmath}
\usepackage{mathpazo}

%
% shorthand for bold symbols, convenient for vectors and matrices
%
\newcommand{\Ba}[0]{\mathbf{a}}
\newcommand{\Bb}[0]{\mathbf{b}}
\newcommand{\Bc}[0]{\mathbf{c}}
\newcommand{\Bd}[0]{\mathbf{d}}
\newcommand{\Be}[0]{\mathbf{e}}
\newcommand{\Bf}[0]{\mathbf{f}}
\newcommand{\Bg}[0]{\mathbf{g}}
\newcommand{\Bh}[0]{\mathbf{h}}
\newcommand{\Bi}[0]{\mathbf{i}}
\newcommand{\Bj}[0]{\mathbf{j}}
\newcommand{\Bk}[0]{\mathbf{k}}
\newcommand{\Bl}[0]{\mathbf{l}}
\newcommand{\Bm}[0]{\mathbf{m}}
\newcommand{\Bn}[0]{\mathbf{n}}
\newcommand{\Bo}[0]{\mathbf{o}}
\newcommand{\Bp}[0]{\mathbf{p}}
\newcommand{\Bq}[0]{\mathbf{q}}
\newcommand{\Br}[0]{\mathbf{r}}
\newcommand{\Bs}[0]{\mathbf{s}}
\newcommand{\Bt}[0]{\mathbf{t}}
\newcommand{\Bu}[0]{\mathbf{u}}
\newcommand{\Bv}[0]{\mathbf{v}}
\newcommand{\Bw}[0]{\mathbf{w}}
\newcommand{\Bx}[0]{\mathbf{x}}
\newcommand{\By}[0]{\mathbf{y}}
\newcommand{\Bz}[0]{\mathbf{z}}
\newcommand{\BA}[0]{\mathbf{A}}
\newcommand{\BB}[0]{\mathbf{B}}
\newcommand{\BC}[0]{\mathbf{C}}
\newcommand{\BD}[0]{\mathbf{D}}
\newcommand{\BE}[0]{\mathbf{E}}
\newcommand{\BF}[0]{\mathbf{F}}
\newcommand{\BG}[0]{\mathbf{G}}
\newcommand{\BH}[0]{\mathbf{H}}
\newcommand{\BI}[0]{\mathbf{I}}
\newcommand{\BJ}[0]{\mathbf{J}}
\newcommand{\BK}[0]{\mathbf{K}}
\newcommand{\BL}[0]{\mathbf{L}}
\newcommand{\BM}[0]{\mathbf{M}}
\newcommand{\BN}[0]{\mathbf{N}}
\newcommand{\BO}[0]{\mathbf{O}}
\newcommand{\BP}[0]{\mathbf{P}}
\newcommand{\BQ}[0]{\mathbf{Q}}
\newcommand{\BR}[0]{\mathbf{R}}
\newcommand{\BS}[0]{\mathbf{S}}
\newcommand{\BT}[0]{\mathbf{T}}
\newcommand{\BU}[0]{\mathbf{U}}
\newcommand{\BV}[0]{\mathbf{V}}
\newcommand{\BW}[0]{\mathbf{W}}
\newcommand{\BX}[0]{\mathbf{X}}
\newcommand{\BY}[0]{\mathbf{Y}}
\newcommand{\BZ}[0]{\mathbf{Z}}

\newcommand{\Bzero}[0]{\mathbf{0}}
\newcommand{\Btheta}[0]{\boldsymbol{\theta}}
\newcommand{\Btau}[0]{\boldsymbol{\tau}}
\newcommand{\Bomega}[0]{\boldsymbol{\omega}}

%
% shorthand for unit vectors
%
\newcommand{\acap}[0]{\hat{\Ba}}
\newcommand{\bcap}[0]{\hat{\Bb}}
\newcommand{\ccap}[0]{\hat{\Bc}}
\newcommand{\dcap}[0]{\hat{\Bd}}
\newcommand{\ecap}[0]{\hat{\Be}}
\newcommand{\fcap}[0]{\hat{\Bf}}
\newcommand{\gcap}[0]{\hat{\Bg}}
\newcommand{\hcap}[0]{\hat{\Bh}}
\newcommand{\icap}[0]{\hat{\Bi}}
\newcommand{\jcap}[0]{\hat{\Bj}}
\newcommand{\kcap}[0]{\hat{\Bk}}
\newcommand{\lcap}[0]{\hat{\Bl}}
\newcommand{\mcap}[0]{\hat{\Bm}}
\newcommand{\ncap}[0]{\hat{\Bn}}
\newcommand{\ocap}[0]{\hat{\Bo}}
\newcommand{\pcap}[0]{\hat{\Bp}}
\newcommand{\qcap}[0]{\hat{\Bq}}
\newcommand{\rcap}[0]{\hat{\Br}}
\newcommand{\scap}[0]{\hat{\Bs}}
\newcommand{\tcap}[0]{\hat{\Bt}}
\newcommand{\ucap}[0]{\hat{\Bu}}
\newcommand{\vcap}[0]{\hat{\Bv}}
\newcommand{\wcap}[0]{\hat{\Bw}}
\newcommand{\xcap}[0]{\hat{\Bx}}
\newcommand{\ycap}[0]{\hat{\By}}
\newcommand{\zcap}[0]{\hat{\Bz}}
\newcommand{\thetacap}[0]{\hat{\Btheta}}

%
% to write R^n and C^n in a distinguishable fashion.  Perhaps change this
% to the double lined characters upon figuring out how to do so.
%
\newcommand{\C}[1]{$\mathbb{C}^{#1}$}
\newcommand{\R}[1]{$\mathbb{R}^{#1}$}

%
% various generally useful helpers
%

% derivative of #1 wrt. #2:
\newcommand{\D}[2] {\frac {d#2} {d#1}}

\newcommand{\inv}[1]{\frac{1}{#1}}
\newcommand{\cross}[0]{\times}

\newcommand{\abs}[1]{\lvert{#1}\rvert}
\newcommand{\norm}[1]{\lVert{#1}\rVert}
\newcommand{\innerprod}[2]{\langle{#1}, {#2}\rangle}
\newcommand{\dotprod}[2]{{#1} \cdot {#2}}
\newcommand{\bdotprod}[2]{\left({#1} \cdot {#2}\right)}
\newcommand{\crossprod}[2]{{#1} \cross {#2}}
\newcommand{\tripleprod}[3]{\dotprod{\left(\crossprod{#1}{#2}\right)}{#3}}

\DeclareMathOperator{\Proj}{Proj}
\DeclareMathOperator{\Span}{span}
\DeclareMathOperator{\Sgn}{sgn}
\DeclareMathOperator{\Area}{Area}
\DeclareMathOperator{\Volume}{Volume}

%
% A few miscellaneous things specific to this document
%
\newcommand{\crossop}[1]{\crossprod{#1}{}}

% R2 vector.
\newcommand{\VectorTwo}[2]{
\begin{bmatrix}
 {#1} \\
 {#2}
\end{bmatrix}
}

\newcommand{\VectorN}[1]{
\begin{bmatrix}
{#1}_1 \\
{#1}_2 \\
\vdots \\
{#1}_N \\
\end{bmatrix}
}

\newcommand{\DETuvij}[4]{
\begin{vmatrix}
 {#1}_{#3} & {#1}_{#4} \\
 {#2}_{#3} & {#2}_{#4}
\end{vmatrix}
}

\newcommand{\DETuvwijk}[6]{
\begin{vmatrix}
 {#1}_{#4} & {#1}_{#5} & {#1}_{#6} \\
 {#2}_{#4} & {#2}_{#5} & {#2}_{#6} \\
 {#3}_{#4} & {#3}_{#5} & {#3}_{#6}
\end{vmatrix}
}

\newcommand{\DETuvwxijkl}[8]{
\begin{vmatrix}
 {#1}_{#5} & {#1}_{#6} & {#1}_{#7} & {#1}_{#8} \\
 {#2}_{#5} & {#2}_{#6} & {#2}_{#7} & {#2}_{#8} \\
 {#3}_{#5} & {#3}_{#6} & {#3}_{#7} & {#3}_{#8} \\
 {#4}_{#5} & {#4}_{#6} & {#4}_{#7} & {#4}_{#8} \\
\end{vmatrix}
}

%\newcommand{\DETuvwxyijklm}[10]{
%\begin{vmatrix}
% {#1}_{#6} & {#1}_{#7} & {#1}_{#8} & {#1}_{#9} & {#1}_{#10} \\
% {#2}_{#6} & {#2}_{#7} & {#2}_{#8} & {#2}_{#9} & {#2}_{#10} \\
% {#3}_{#6} & {#3}_{#7} & {#3}_{#8} & {#3}_{#9} & {#3}_{#10} \\
% {#4}_{#6} & {#4}_{#7} & {#4}_{#8} & {#4}_{#9} & {#4}_{#10} \\
% {#5}_{#6} & {#5}_{#7} & {#5}_{#8} & {#5}_{#9} & {#5}_{#10}
%\end{vmatrix}
%}

% R3 vector.
\newcommand{\VectorThree}[3]{
\begin{bmatrix}
 {#1} \\
 {#2} \\
 {#3}
\end{bmatrix}
}



\author{Peeter Joot}
\email{peeter.joot@gmail.com}


\chapter{PHY450H1S.  Relativistic Electrodynamics Lecture 20 (Taught by Prof. Erich Poppitz).  Lienard-Wiechert potentials.}
\label{chap:relativisticElectrodynamicsL20}
%\useCCL
\blogpage{http://sites.google.com/site/peeterjoot/math2011/relativisticElectrodynamicsL20.pdf}
\date{Mar 15, 2011}
\revisionInfo{relativisticElectrodynamicsL20.tex}

%\beginArtWithToc
\beginArtNoToc

\section{Reading.}

Covering chapter 8 material from the text \cite{landau1980classical}.

FIXME:
Covering \href{http://www.physics.utoronto.ca/~poppitz/epoppitz/PHY450_files/RelEMpp136-146.pdf}{lecture notes pp. 136-146}: the Lienard-Wiechert potentials (143-146) [Wednesday, Mar. 9...]

pp. 147-165: EM fields of a moving source (147-148+HW5); a particle at rest (148); a constant velocity particle (149-152); behavior of EM fields ``at infinity'' for a general-worldline source and radiation (152-153) [Tuesday, Mar. 15]; radiated power (154); fields in the ``wave zone'' and discussions of approximations made (155-159); EM fields due to electric dipole radiation (160-163); Poynting vector, angular distribution, and power of dipole radiation (164-165) [Wednesday, Mar. 16...]

\section{.}

\begin{equation}\label{eqn:relativisticElectrodynamicsL20:10}
A^i(\Bx, t) = \inv{c} \int d^3 \Bx' j^i\left(\Bx', t - \frac{\Abs{\Bx - \Bx'}}{c}\right) \inv{\Abs{\Bx - \Bx'}}
\end{equation}

This integral is over the region of space where the sources $j^i$ are non-vanishing, but this region is limited.  The value $\Abs{\Bx'} \le l$, so we can expand the denominator in multipole expansion

\begin{align*}
\inv{\Abs{\Bx - \Bx'}}
&=
\inv{\sqrt{(\Bx - \Bx')^2}} \\
&=
\inv{\sqrt{\Bx^2 + {\Bx'}^2 - 2 \Bx \cdot \Bx'}} \\
&=
\inv{\Abs{\Bx}} \inv{\sqrt{1 + \frac{{\Bx'}^2}{\Bx^2} - 2 \frac{\xcap}{\Abs{\Bx}} \cdot \Bx'}} \\
&\approx
\inv{\Abs{\Bx}} \inv{\sqrt{1 - 2 \frac{\xcap}{\Abs{\Bx}} \cdot \Bx'}} \\
&\approx
\inv{\Abs{\Bx}} \left(1 + \frac{\xcap}{\Abs{\Bx}} \cdot \Bx' \right).
\end{align*}

so
\begin{equation}\label{eqn:relativisticElectrodynamicsL20:30}
\inv{\Abs{\Bx - \Bx'}}
\approx 
\inv{\Abs{\Bx}} + \frac{\Bx}{\Abs{\Bx}^3} \cdot \Bx' 
\end{equation}

similarily 

\begin{align*}
t - \frac{\Abs{\Bx - \Bx'}}{c} 
&= t - \frac{\Abs{\Bx}}{c} \left( 1 - \frac{\Bx \cdot \Bx'}{\Abs{\Bx}^2} \right) \\
&= t - \frac{\Abs{\Bx}}{c} + \frac{\Bx \cdot \Bx'}{\Abs{\Bx}}
\end{align*}

The retardation effect is the time argument of the $f'$ the source $j^i$

FIXME: what's $f'$?

\begin{equation}\label{eqn:relativisticElectrodynamicsL20:50}
j^i\left(\Bx', t - \frac{\Abs{\Bx}{c}} + \frac{\Bx \cdot \Bx'}{\Abs{\Bx}} + \cdots \right)
\approx
j^i\left(\Bx', t - \frac{\Abs{\Bx}}{c}\right) + j^i \left(\Bx, t - \frac{\Abs{\Bx}}{c}\right) \frac{\Bx \cdot \Bx'}{\Abs{\Bx}}
\end{equation}

to elucidate the physics, imagine that time dependence of the source is periodic with angular frequency $\omega_0$.  Then

\begin{equation}\label{eqn:relativisticElectrodynamicsL20:70}
j^i \approx \omega_0 j^i
\end{equation}

magnitude fo the second term 

\begin{equation}\label{eqn:relativisticElectrodynamicsL20:90}
j^i \frac{\Bx \cdot \Bx'}{\Abs{\Bx}} = \omega_0 \frac{\Bx \cdot \Bx'}{\Abs{\Bx}} = \omega_0 
\end{equation}

while the first time is $\propto j^i$.

Requiring second term much less than the first term means 

\begin{equation}\label{eqn:relativisticElectrodynamicsL20:110}
\Abs{\omega_0 \frac{\Bx \cdot \Bx'}{\Abs{\Bx}}} \ll 1
\end{equation}

But recall 
\begin{equation}\label{eqn:relativisticElectrodynamicsL20:130}
\Abs{\frac{\Bx \cdot \Bx'}{\Abs{\Bx}}} \le l
\end{equation}

therefore

\begin{equation}\label{eqn:relativisticElectrodynamicsL20:150}
\Abs{\omega_0 \frac{\Bx \cdot \Bx'}{\Abs{\Bx}}} \le \frac{\omega_0 l}{c} \ll 1
\end{equation}

This is a physical requirement size of the wavelength of the emitter (if the wavvelength doesn't meet this requirement, this expansion does not work).

\begin{equation}\label{eqn:relativisticElectrodynamicsL20:170}
\frac{\omega_0}{c} = \frac{2 \pi}{c T} \propto \inv{\lambda}
\end{equation}

\paragraph{Moral:} We'll utilize two expandsions (we need two small parameters)

\begin{enumerate}
\item $\Abs{\Bx} \gg l$
\item $\lambda \gg l$
\end{enumerate}

Plugging into our current

\begin{equation}\label{eqn:relativisticElectrodynamicsL20:190}
A^i(\Bx, t) 
\approx \inv{c} \int d^3 \Bx' 
\left( j^i\left(\Bx', t - \frac{\Abs{\Bx}}{c}\right) + j^i \left(\Bx, t - \frac{\Abs{\Bx}}{c}\right) \frac{\Bx \cdot \Bx'}{\Abs{\Bx}} \right)
\left( \inv{\Abs{\Bx}} + \frac{\Bx}{\Abs{\Bx}^3} \cdot \Bx' \right)
\end{equation}

\begin{equation}\label{eqn:relativisticElectrodynamicsL20:210}
A^0(\Bx, t) 
\approx 
\inv{\Abs{\Bx}} \int d^3 \Bx' \rho\left(\Bx', t - \frac{\Abs{\Bx}}{c}\right)
+\inv{\Abs{\Bx}^3} \int d^3 \Bx' \Bx' \rho \left(\Bx', t - \frac{\Abs{\Bx}}{c}\right)
+\inv{\Abs{\Bx}^2} \Bx \cdot \int d^3 \Bx' \Bx' \PD{t}{} \rho\left(\Bx', t - \frac{\Abs{\Bx}}{c}\right)
\end{equation}

Note that this term

\begin{equation}\label{eqn:relativisticElectrodynamicsL20:230}
\int d^3 \Bx' \Bx' \rho\left(\Bx', t - \frac{\Abs{\Bx}}{c}\right) = \Bd,
\end{equation}

is the dipole moment.  In the last term we can pull out the time derivative (because we are integrating over $\Bx'$)

\begin{align*}
\inv{\Abs{\Bx}^2} \Bx \cdot \int d^3 \Bx' \Bx' \PD{t}{} \rho\left(\Bx', t - \frac{\Abs{\Bx}}{c}\right)
&=
\inv{\Abs{\Bx}^2} \Bx \cdot \PD{t}{} \int d^3 \Bx' \Bx' \rho\left(\Bx', t - \frac{\Abs{\Bx}}{c}\right)
\rho\left(\Bx', t - \frac{\Abs{\Bx}}{c}\right)
&=
\inv{\Abs{\Bx}^2} \Bx \cdot \PD{t}{}\Bd \left(t - \frac{\Abs{\Bx}}{c}\right)
\end{align*}

For the spatial components of the current lets just keep the first term

\begin{align*}
A^\alpha(\Bx, t) 
&\approx
\inv{ c \Abs{\Bx}} \int d^3 \Bx' j^\alpha\left(\Bx', t - \frac{\Abs{\Bx}}{c}\right) \\
&=
\inv{ c \Abs{\Bx}} \int d^3 \Bx' (\spacegrad_{\Bx'} x^\alpha \cdot \Bj\left(\Bx', t - \frac{\Abs{\Bx}}{c}\right)  \\
&=
\inv{ c \Abs{\Bx}} \int d^3 \Bx' 
\left(
\spacegrad \left( {x'}^\alpha \Bj \left(\Bx', t - \frac{\Abs{\Bx}}{c}\right) \right)
- {x'}^\alpha \spacegrad_{\Bx'} \cdot \Bj\left(\Bx', t - \frac{\Abs{\Bx}}{c}\right) 
\right) \\
&=
\inv{ c \Abs{\Bx}} \oint_{S^2_\infty} d^2 \Bsigma \cdot {x'}^\alpha \Bj\left(\Bx', t - \frac{\Abs{\Bx}}{c}\right)
+\inv{ c \Abs{\Bx}} \int d^3 \Bx' {x'}^\alpha \PD{t}{}\rho\left(\Bx', t - \frac{\Abs{\Bx}}{c}\right)
\end{align*}

%boxed:
\begin{align}\label{eqn:relativisticElectrodynamicsL20:250}
A^0(\Bx, t) &= \frac{Q\left(t - \frac{\Abs{\Bx}}{c}\right)}{\Abs{\Bx}} + \frac{\Bx \cdot \Bd\left(t - \frac{\Abs{\Bx}}{c}\right)}{Abs{\Bx}^3} + \frac{\Bx \cdot \dot{\Bd}\left(t - \frac{\Abs{\Bx}}{c}\right)}{c \Abs{\Bx}^2} \\
\BA(\Bx, t) &= \inv{c \Abs{\Bx}} \dot{\Bd}\left(t - \frac{\Abs{\Bx}}{c}\right)
\end{align}

\subsection{Example}

Electric dipole radiation.

PICTURE: two closely separated oppositely charges, wiggling along the line connecting them.

$-q$ at rest, while $+q$ oscillates.

\begin{equation}\label{eqn:relativisticElectrodynamicsL20:270}
z_+(t) = z_0 + a \sin\omega t
\end{equation}

\begin{equation}\label{eqn:relativisticElectrodynamicsL20:290}
\Bd(t) = \Be_z q (z_0 + a \sin\omega t)
\end{equation}

Thus

\begin{equation}\label{eqn:relativisticElectrodynamicsL20:310}
A^0(\Bx, t) = 
\end{equation}

\EndArticle
