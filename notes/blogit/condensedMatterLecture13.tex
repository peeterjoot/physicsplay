%
% Copyright � 2013 Peeter Joot.  All Rights Reserved.
% Licenced as described in the file LICENSE under the root directory of this GIT repository.
%
\newcommand{\authorname}{Peeter Joot}
\newcommand{\email}{peeterjoot@protonmail.com}
\newcommand{\basename}{FIXMEbasenameUndefined}
\newcommand{\dirname}{notes/FIXMEdirnameUndefined/}

\renewcommand{\basename}{condensedMatterLecture13}
\renewcommand{\dirname}{notes/phy487/}
\newcommand{\keywords}{Condensed matter physics, PHY487H1F}
\newcommand{\authorname}{Peeter Joot}
\newcommand{\onlineurl}{http://sites.google.com/site/peeterjoot2/math2013/\basename.pdf}
\newcommand{\sourcepath}{\dirname\basename.tex}
\newcommand{\generatetitle}[1]{\chapter{#1}}

\newcommand{\vcsinfo}{%
\section*{}
\noindent{\color{DarkOliveGreen}{\rule{\linewidth}{0.1mm}}}
\paragraph{Document version}
%\paragraph{\color{Maroon}{Document version}}
{
\small
\begin{itemize}
\item Available online at:\\ 
\href{\onlineurl}{\onlineurl}
\item Git Repository: \input{./.revinfo/gitRepo.tex}
\item Source: \sourcepath
\item last commit: \input{./.revinfo/gitCommitString.tex}
\item commit date: \input{./.revinfo/gitCommitDate.tex}
\end{itemize}
}
}

%\PassOptionsToPackage{dvipsnames,svgnames}{xcolor}
\PassOptionsToPackage{square,numbers}{natbib}
\documentclass{scrreprt}

\usepackage[left=2cm,right=2cm]{geometry}
\usepackage[svgnames]{xcolor}
\usepackage{peeters_layout}

\usepackage{natbib}

\usepackage[
colorlinks=true,
bookmarks=false,
pdfauthor={\authorname, \email},
backref 
]{hyperref}

% http://tex.stackexchange.com/questions/75773/how-to-reference-problems-by-the-text-label-in-an-exercise-envioronment
\usepackage[english]{cleveref}
\crefname{Exercise}{exercise}{exercises}
\Crefname{Exercise}{Exercise}{Exercises}

\RequirePackage{titlesec}
\RequirePackage{ifthen}

% http://stackoverflow.com/questions/4932910/date-in-the-tabular-environment
\makeatletter
\let\insertdate\@date
\makeatother

\titleformat{\chapter}[display]
{\bfseries\Large}
{\color{DarkSlateGrey}\filleft \authorname
\ifthenelse{\isundefined{\studentnumber}}{}{\\ \studentnumber}
\ifthenelse{\isundefined{\email}}{}{\\ \email}
\ifthenelse{\isundefined{\dateintitle}}{}{\\ \insertdate}
%\ifthenelse{\isundefined{\coursename}}{}{\\ \coursename} % put in title instead.
}
{4ex}
{\color{DarkOliveGreen}{\titlerule}\color{Maroon}
\vspace{2ex}%
\filright}
[\vspace{2ex}%
\color{DarkOliveGreen}\titlerule
]

\newcommand{\beginArtWithToc}[0]{\begin{document}\tableofcontents}
\newcommand{\beginArtNoToc}[0]{\begin{document}}
\newcommand{\EndNoBibArticle}[0]{\end{document}}
\newcommand{\EndArticle}[0]{\bibliography{Bibliography}\bibliographystyle{plainnat}\end{document}}

% 
%\newcommand{\citep}[1]{\cite{#1}}

\colorSectionsForArticle



%\citep{harald2003solid} \S x.y
%\citep{ibach2009solid} \S x.y

%\usepackage{mhchem}
\usepackage[version=3]{mhchem}
\newcommand{\nought}[0]{\circ}
\newcommand{\EF}[0]{\epsilon_{\mathrm{F}}}
\newcommand{\kF}[0]{k_{\mathrm{F}}}

\beginArtNoToc
\generatetitle{PHY487H1F Condensed Matter Physics.  Lecture 13: Free electron model of metals.  Taught by Prof.\ Stephen Julian}
%\chapter{Free electron model of metals}
\label{chap:condensedMatterLecture13}

%\section{Disclaimer}
%
%Peeter's lecture notes from class.  May not be entirely coherent.

\section{Heat capacity of free electrons (cont.)}

Last time we found the \textAndIndex{density of states} for Fermions in a period potential

\begin{dmath}\label{eqn:condensedMatterLecture13:20}
D(E) = \inv{2 \pi^2} \lr{\frac{2m}{\Hbar^2}}{3/2} \sqrt{E}.
\end{dmath}

Using the \textAndIndex{Fermi-Dirac distribution} \cref{fig:fermiDiracLevelCurves:fermiDiracLevelCurvesFig1}

\imageFigure{fermiDiracLevelCurvesFig1}{Fermi-Dirac distribution}{fig:fermiDiracLevelCurves:fermiDiracLevelCurvesFig1}{0.3}

\begin{dmath}\label{eqn:condensedMatterLecture13:40}
f(E, T) = \inv{ e^{(E - \mu)/\kB T} + 1},
\end{dmath}

we calculated an approximate value for the specific heat

\begin{dmath}\label{eqn:condensedMatterLecture13:60}
C(T) \sim 2 \kB D(\EF) T.
\end{dmath}

We will now move on and calculate a more exact expression for the specific heat, defined by

\begin{subequations}
\begin{dmath}\label{eqn:condensedMatterLecture13:80}
U(T) = \int dE D(E) E f(E, T)
\end{dmath}
\begin{dmath}\label{eqn:condensedMatterLecture13:100}
C(T) = \PD{T}{U},
\end{dmath}
\end{subequations}

or, in terms of the density of states

\begin{dmath}\label{eqn:condensedMatterLecture13:120}
C(T) = \int dE D(E) E \PD{T}{f(E, T)}.
\end{dmath}

In \cref{fig:fermiDiracLevelCurves:fermiDiracLevelCurvesFig2}, are plots of the Fermi-Dirac distribution functions at $T_2 > T_1$ and their difference.  Observe that this difference is zero most everywhere

\imageFigure{fermiDiracLevelCurvesFig2}{Fermi-Dirac curves and their difference}{fig:fermiDiracLevelCurves:fermiDiracLevelCurvesFig2}{0.3}

Calculating that derivative explicitly, we have

\begin{dmath}\label{eqn:condensedMatterLecture13:140}
\PD{T}{f(E, T)}
=
-
\frac{ 
e^{(E - \mu)/\kB T}
}
{
(e^{(E - \mu)/\kB T} + 1)^2
}
\frac{E - \mu}{\kB} \frac{-1}{T^2}
=
\frac{ 
e^{(E - \mu)/\kB T}
}
{
(e^{(E - \mu)/\kB T} + 1)^2
}
\frac{E - \mu}{\kB T^2}.
\end{dmath}

This is plotted in \cref{fig:fermiDiracLevelCurvesAndDerivatives:fermiDiracLevelCurvesAndDerivativesFig3}.

\imageFigure{fermiDiracLevelCurvesAndDerivativesFig3}{Fermi-Dirac distribution and derivatives}{fig:fermiDiracLevelCurvesAndDerivatives:fermiDiracLevelCurvesAndDerivativesFig3}{0.3}

\paragraph{Review from here down.}

With constant n

\begin{dmath}\label{eqn:condensedMatterLecture13:160}
n = \text{constant} = \int dE D(E) f(E, T).
\end{dmath}

So, 

\begin{dmath}\label{eqn:condensedMatterLecture13:180}
\EF \PD{T}{n} = 0 = 
\int dE D(E) \EF \PD{T}{f}
\end{dmath}

This gives

\begin{dmath}\label{eqn:condensedMatterLecture13:200}
C(T) = \int_0^\infty dE D(E) 
( E - \EF)
\mathLabelBox
{
\frac{ E - \EF}{\kB T^2}
\frac{ 
e^{(E - \mu)/\kB T}
}
{
(e^{(E - \mu)/\kB T} + 1)^2
}
}
{
zero except within a few $\kB T$ of $\EF$
}
\end{dmath}

This allows us to extend the integration range

\begin{dmath}\label{eqn:condensedMatterLecture13:220}
\int_0^\infty \rightarrow \int_{-\infty}^\infty,
\end{dmath}

and $D(E) \rightarrow D(\EF)$.

%\cref{fig:qmSolidsL13:qmSolidsL13Fig3}.
\imageFigure{qmSolidsL13Fig3}{3: CAPTION}{fig:qmSolidsL13:qmSolidsL13Fig3}{0.3}

To proceed with the integration, let

\begin{dmath}\label{eqn:condensedMatterLecture13:240}
x = \frac{E - \EF}{\kB T}.
\end{dmath}

\begin{dmath}\label{eqn:condensedMatterLecture13:260}
C(T) \approx
D(\EF) 
\int_{-\infty}^\infty
\mathLabelBox
{
(dx \kB T)
}
{
$dE$
}
\frac{x^2 \cancel{\kB^2} \cancel{T^2}}{\cancel{\kB} \cancel{T^2}}
\frac{e^x}{e^x + 1)^2}
=
D(\EF) 
\kB^2 T
\mathLabelBox
{
\int_{-\infty}^\infty
dx \frac{x^2 e^x}{(e^x + 1)^2}
}
{
$\pi^2/3$
}.
\end{dmath}

We have finally

\begin{dmath}\label{eqn:condensedMatterLecture13:280}
\myBoxed{
C(T) \approx \frac{\pi^2}{3} \kB^2 D(\EF) T.
}
\end{dmath}

The linear T specific heat is a signature of the Fermi surface (sharp boundary in k-space between occupied and unoccupied states).  This doesn't depend on the details form of $D(E)$, but only on $D(\EF)$.  This therefore works for \underline{all} metals.

%\cref{fig:qmSolidsL13:qmSolidsL13Fig4a}.
\imageFigure{qmSolidsL13Fig4a}{4a: CAPTION}{fig:qmSolidsL13:qmSolidsL13Fig4a}{0.3}
%\cref{fig:qmSolidsL13:qmSolidsL13Fig4b}.
\imageFigure{qmSolidsL13Fig4b}{4b: CAPTION}{fig:qmSolidsL13:qmSolidsL13Fig4b}{0.3}

If you see a $C \sim T^3$ (cubic) relationship you can realize that we are dealing with a bosonic system where there is a linear frequency relationship.

\paragraph{electrons}

%\cref{fig:qmSolidsL13:qmSolidsL13Fig5a}.
\imageFigure{qmSolidsL13Fig5a}{5a: CAPTION}{fig:qmSolidsL13:qmSolidsL13Fig5a}{0.3}
%\cref{fig:qmSolidsL13:qmSolidsL13Fig5b}.
\imageFigure{qmSolidsL13Fig5b}{5b: CAPTION}{fig:qmSolidsL13:qmSolidsL13Fig5b}{0.3}

At $T > 0$, thin shell of width $\kB T$ thermally excited.  Volume is 

\begin{dmath}\label{eqn:condensedMatterLecture13:300}
4 \pi \kF^2 \delta k \propto 4 \pi \kF^2 \delta T
\end{dmath}

Each thermally excieted electron has thermal energy $\kB T$, so

\begin{dmath}\label{eqn:condensedMatterLecture13:320}
U(T) \sim T^2 \implies C(T) \propto T
\end{dmath}

If you see a $C \sim T$ (linear) relationship you can realize that we are dealing with a Fermionic system.

\sectionAndIndex{Thomas-Fermi screening}

Recall that $\BE = 0$ inside a metal in equilibrium.

%\cref{fig:qmSolidsL13:qmSolidsL13Fig6}.
\imageFigure{qmSolidsL13Fig6}{6: CAPTION}{fig:qmSolidsL13:qmSolidsL13Fig6}{0.3}

%\cref{fig:qmSolidsL13:qmSolidsL13Fig7}.
\imageFigure{qmSolidsL13Fig7}{7: CAPTION}{fig:qmSolidsL13:qmSolidsL13Fig7}{0.3}

Put a point charge $Q$ at $\Br = 0$, and an electric potential $\phi(\Br)$.  Our Maxwell equation is

\begin{dmath}\label{eqn:condensedMatterLecture13:340}
\spacegrad^2 \phi(\Br) = - \frac{\rho(\Br)}{\epsilon_\nought}.
\end{dmath}

Split the charge density as

\begin{dmath}\label{eqn:condensedMatterLecture13:360}
\rho(\Br) = 
%\cancel{
   \mathLabelBox
   [
      labelstyle={xshift=-2cm},
      linestyle={out=270,in=90, latex-}
   ]
   {
   \overbar{\rho}_{\mathrm{el}} 
   }
   {average electron density}
   + 
   \mathLabelBox
   [
      labelstyle={below of=m\themathLableNode, below of=m\themathLableNode}
   ]
   {
   \overbar{\rho}_{\mathrm{ion}}
   }
   {
   positive background
   }
%} 
+
\mathLabelBox
[
   labelstyle={xshift=2cm},
   linestyle={out=270,in=90, latex-}
]
{
\delta \rho_{\mathrm{el}} 
}
{
pertubation to Q
}
\end{dmath}

The first two terms cancel giving

\begin{dmath}\label{eqn:condensedMatterLecture13:380}
\rho_{\mathrm{el}} 
= -e
\int dE D(E) 
\inv{ e^{(E 
- \mu - e \phi(\Br)
)/\kB T} + 1}
\approx
-e
\int dE D(E) 
\lr{
\inv{ e^{(E - \mu)/\kB T} + 1}
- e \phi(\Br) \PD{E}{f} 
+ \cdots
}
\end{dmath}

Here $\mu + e \phi(\Br)$ is the chemical potential shifted at $\Br$ by $e \phi(\Br)$.

%\cref{fig:qmSolidsL13:qmSolidsL13Fig8}.
\imageFigure{qmSolidsL13Fig8}{8: CAPTION}{fig:qmSolidsL13:qmSolidsL13Fig8}{0.3}

As $T \rightarrow 0$

\begin{dmath}\label{eqn:condensedMatterLecture13:400}
\int_
{\EF - \Delta}
^{\EF + \Delta} 
dE \PD{E}{f}
= -1,
\end{dmath}

where the width of $\PDi{E}{f} \rightarrow 0$.  This is very much like a delta function, so we can write

\begin{dmath}\label{eqn:condensedMatterLecture13:420}
\rho_{\mathrm{el}} 
\approx
\overbar{\rho}_{\mathrm{el}} 
- e^2 \phi(\Br) \int dE D(E) \delta(E - \EF)
\approx
\overbar{\rho}_{\mathrm{el}} 
- e^2 \phi(\Br) D(\EF)
\end{dmath}

A handwaving discussion of this can be found in \citep{ibach2009solid} eq. 6.5.2.

For the potential we have

\begin{dmath}\label{eqn:condensedMatterLecture13:440}
\phi(\Br) = \phi_{\mathrm{avg}} + \delta \phi(\Br).
\end{dmath}

\begin{dmath}\label{eqn:condensedMatterLecture13:460}
\inv{r^2} \PD{r}{} \lr{ r^2 \delta \phi(\Br) } = -e^2 D(\EF) \delta(\Br).
\end{dmath}

This has solution

\begin{dmath}\label{eqn:condensedMatterLecture13:480}
\delta \phi(\Br) = \frac{\alpha e^{ -r/r_{\mathrm{TF} } }}{r}.
\end{dmath}

Here $r_{\mathrm{TF} }$ is the \textAndIndex{Thomas-Fermi screening} length

\begin{dmath}\label{eqn:condensedMatterLecture13:500}
r_{\mathrm{TF}}
= \sqrt{\frac{\epsilon_\nought}{e^2 D(\EF) }}.
\end{dmath}

%\cref{fig:qmSolidsL13:qmSolidsL13Fig9}.
\imageFigure{qmSolidsL13Fig9}{9: CAPTION}{fig:qmSolidsL13:qmSolidsL13Fig9}{0.3}

\makeexample{Copper}{example:condensedMatterLecture13:1}{
\begin{dmath}\label{eqn:condensedMatterLecture13:520}
r_{\mathrm{TF} }
\sim 
0.5
\angstrom
\end{dmath}
}

Reading: \S 6.5, especially Mott transition.

\EndArticle
