%
% Copyright � 2015 Peeter Joot.  All Rights Reserved.
% Licenced as described in the file LICENSE under the root directory of this GIT repository.
%
\documentclass[]{eliblog}

\usepackage{amsmath}
\usepackage{mathpazo}

%
% shorthand for bold symbols, convenient for vectors and matrices
%
\newcommand{\Ba}[0]{\mathbf{a}}
\newcommand{\Bb}[0]{\mathbf{b}}
\newcommand{\Bc}[0]{\mathbf{c}}
\newcommand{\Bd}[0]{\mathbf{d}}
\newcommand{\Be}[0]{\mathbf{e}}
\newcommand{\Bf}[0]{\mathbf{f}}
\newcommand{\Bg}[0]{\mathbf{g}}
\newcommand{\Bh}[0]{\mathbf{h}}
\newcommand{\Bi}[0]{\mathbf{i}}
\newcommand{\Bj}[0]{\mathbf{j}}
\newcommand{\Bk}[0]{\mathbf{k}}
\newcommand{\Bl}[0]{\mathbf{l}}
\newcommand{\Bm}[0]{\mathbf{m}}
\newcommand{\Bn}[0]{\mathbf{n}}
\newcommand{\Bo}[0]{\mathbf{o}}
\newcommand{\Bp}[0]{\mathbf{p}}
\newcommand{\Bq}[0]{\mathbf{q}}
\newcommand{\Br}[0]{\mathbf{r}}
\newcommand{\Bs}[0]{\mathbf{s}}
\newcommand{\Bt}[0]{\mathbf{t}}
\newcommand{\Bu}[0]{\mathbf{u}}
\newcommand{\Bv}[0]{\mathbf{v}}
\newcommand{\Bw}[0]{\mathbf{w}}
\newcommand{\Bx}[0]{\mathbf{x}}
\newcommand{\By}[0]{\mathbf{y}}
\newcommand{\Bz}[0]{\mathbf{z}}
\newcommand{\BA}[0]{\mathbf{A}}
\newcommand{\BB}[0]{\mathbf{B}}
\newcommand{\BC}[0]{\mathbf{C}}
\newcommand{\BD}[0]{\mathbf{D}}
\newcommand{\BE}[0]{\mathbf{E}}
\newcommand{\BF}[0]{\mathbf{F}}
\newcommand{\BG}[0]{\mathbf{G}}
\newcommand{\BH}[0]{\mathbf{H}}
\newcommand{\BI}[0]{\mathbf{I}}
\newcommand{\BJ}[0]{\mathbf{J}}
\newcommand{\BK}[0]{\mathbf{K}}
\newcommand{\BL}[0]{\mathbf{L}}
\newcommand{\BM}[0]{\mathbf{M}}
\newcommand{\BN}[0]{\mathbf{N}}
\newcommand{\BO}[0]{\mathbf{O}}
\newcommand{\BP}[0]{\mathbf{P}}
\newcommand{\BQ}[0]{\mathbf{Q}}
\newcommand{\BR}[0]{\mathbf{R}}
\newcommand{\BS}[0]{\mathbf{S}}
\newcommand{\BT}[0]{\mathbf{T}}
\newcommand{\BU}[0]{\mathbf{U}}
\newcommand{\BV}[0]{\mathbf{V}}
\newcommand{\BW}[0]{\mathbf{W}}
\newcommand{\BX}[0]{\mathbf{X}}
\newcommand{\BY}[0]{\mathbf{Y}}
\newcommand{\BZ}[0]{\mathbf{Z}}

\newcommand{\Bzero}[0]{\mathbf{0}}
\newcommand{\Btheta}[0]{\boldsymbol{\theta}}
\newcommand{\Btau}[0]{\boldsymbol{\tau}}
\newcommand{\Bomega}[0]{\boldsymbol{\omega}}

%
% shorthand for unit vectors
%
\newcommand{\acap}[0]{\hat{\Ba}}
\newcommand{\bcap}[0]{\hat{\Bb}}
\newcommand{\ccap}[0]{\hat{\Bc}}
\newcommand{\dcap}[0]{\hat{\Bd}}
\newcommand{\ecap}[0]{\hat{\Be}}
\newcommand{\fcap}[0]{\hat{\Bf}}
\newcommand{\gcap}[0]{\hat{\Bg}}
\newcommand{\hcap}[0]{\hat{\Bh}}
\newcommand{\icap}[0]{\hat{\Bi}}
\newcommand{\jcap}[0]{\hat{\Bj}}
\newcommand{\kcap}[0]{\hat{\Bk}}
\newcommand{\lcap}[0]{\hat{\Bl}}
\newcommand{\mcap}[0]{\hat{\Bm}}
\newcommand{\ncap}[0]{\hat{\Bn}}
\newcommand{\ocap}[0]{\hat{\Bo}}
\newcommand{\pcap}[0]{\hat{\Bp}}
\newcommand{\qcap}[0]{\hat{\Bq}}
\newcommand{\rcap}[0]{\hat{\Br}}
\newcommand{\scap}[0]{\hat{\Bs}}
\newcommand{\tcap}[0]{\hat{\Bt}}
\newcommand{\ucap}[0]{\hat{\Bu}}
\newcommand{\vcap}[0]{\hat{\Bv}}
\newcommand{\wcap}[0]{\hat{\Bw}}
\newcommand{\xcap}[0]{\hat{\Bx}}
\newcommand{\ycap}[0]{\hat{\By}}
\newcommand{\zcap}[0]{\hat{\Bz}}
\newcommand{\thetacap}[0]{\hat{\Btheta}}

%
% to write R^n and C^n in a distinguishable fashion.  Perhaps change this
% to the double lined characters upon figuring out how to do so.
%
\newcommand{\C}[1]{$\mathbb{C}^{#1}$}
\newcommand{\R}[1]{$\mathbb{R}^{#1}$}

%
% various generally useful helpers
%

% derivative of #1 wrt. #2:
\newcommand{\D}[2] {\frac {d#2} {d#1}}

\newcommand{\inv}[1]{\frac{1}{#1}}
\newcommand{\cross}[0]{\times}

\newcommand{\abs}[1]{\lvert{#1}\rvert}
\newcommand{\norm}[1]{\lVert{#1}\rVert}
\newcommand{\innerprod}[2]{\langle{#1}, {#2}\rangle}
\newcommand{\dotprod}[2]{{#1} \cdot {#2}}
\newcommand{\bdotprod}[2]{\left({#1} \cdot {#2}\right)}
\newcommand{\crossprod}[2]{{#1} \cross {#2}}
\newcommand{\tripleprod}[3]{\dotprod{\left(\crossprod{#1}{#2}\right)}{#3}}

\DeclareMathOperator{\Proj}{Proj}
\DeclareMathOperator{\Span}{span}
\DeclareMathOperator{\Sgn}{sgn}
\DeclareMathOperator{\Area}{Area}
\DeclareMathOperator{\Volume}{Volume}

%
% A few miscellaneous things specific to this document
%
\newcommand{\crossop}[1]{\crossprod{#1}{}}

% R2 vector.
\newcommand{\VectorTwo}[2]{
\begin{bmatrix}
 {#1} \\
 {#2}
\end{bmatrix}
}

\newcommand{\VectorN}[1]{
\begin{bmatrix}
{#1}_1 \\
{#1}_2 \\
\vdots \\
{#1}_N \\
\end{bmatrix}
}

\newcommand{\DETuvij}[4]{
\begin{vmatrix}
 {#1}_{#3} & {#1}_{#4} \\
 {#2}_{#3} & {#2}_{#4}
\end{vmatrix}
}

\newcommand{\DETuvwijk}[6]{
\begin{vmatrix}
 {#1}_{#4} & {#1}_{#5} & {#1}_{#6} \\
 {#2}_{#4} & {#2}_{#5} & {#2}_{#6} \\
 {#3}_{#4} & {#3}_{#5} & {#3}_{#6}
\end{vmatrix}
}

\newcommand{\DETuvwxijkl}[8]{
\begin{vmatrix}
 {#1}_{#5} & {#1}_{#6} & {#1}_{#7} & {#1}_{#8} \\
 {#2}_{#5} & {#2}_{#6} & {#2}_{#7} & {#2}_{#8} \\
 {#3}_{#5} & {#3}_{#6} & {#3}_{#7} & {#3}_{#8} \\
 {#4}_{#5} & {#4}_{#6} & {#4}_{#7} & {#4}_{#8} \\
\end{vmatrix}
}

%\newcommand{\DETuvwxyijklm}[10]{
%\begin{vmatrix}
% {#1}_{#6} & {#1}_{#7} & {#1}_{#8} & {#1}_{#9} & {#1}_{#10} \\
% {#2}_{#6} & {#2}_{#7} & {#2}_{#8} & {#2}_{#9} & {#2}_{#10} \\
% {#3}_{#6} & {#3}_{#7} & {#3}_{#8} & {#3}_{#9} & {#3}_{#10} \\
% {#4}_{#6} & {#4}_{#7} & {#4}_{#8} & {#4}_{#9} & {#4}_{#10} \\
% {#5}_{#6} & {#5}_{#7} & {#5}_{#8} & {#5}_{#9} & {#5}_{#10}
%\end{vmatrix}
%}

% R3 vector.
\newcommand{\VectorThree}[3]{
\begin{bmatrix}
 {#1} \\
 {#2} \\
 {#3}
\end{bmatrix}
}



\author{Peeter Joot}
\email{peeter.joot@gmail.com}

%\documentclass[]{eliblogwidescreen}

\usepackage{amsmath}
\usepackage{mathpazo}

%
% shorthand for bold symbols, convenient for vectors and matrices
%
\newcommand{\Ba}[0]{\mathbf{a}}
\newcommand{\Bb}[0]{\mathbf{b}}
\newcommand{\Bc}[0]{\mathbf{c}}
\newcommand{\Bd}[0]{\mathbf{d}}
\newcommand{\Be}[0]{\mathbf{e}}
\newcommand{\Bf}[0]{\mathbf{f}}
\newcommand{\Bg}[0]{\mathbf{g}}
\newcommand{\Bh}[0]{\mathbf{h}}
\newcommand{\Bi}[0]{\mathbf{i}}
\newcommand{\Bj}[0]{\mathbf{j}}
\newcommand{\Bk}[0]{\mathbf{k}}
\newcommand{\Bl}[0]{\mathbf{l}}
\newcommand{\Bm}[0]{\mathbf{m}}
\newcommand{\Bn}[0]{\mathbf{n}}
\newcommand{\Bo}[0]{\mathbf{o}}
\newcommand{\Bp}[0]{\mathbf{p}}
\newcommand{\Bq}[0]{\mathbf{q}}
\newcommand{\Br}[0]{\mathbf{r}}
\newcommand{\Bs}[0]{\mathbf{s}}
\newcommand{\Bt}[0]{\mathbf{t}}
\newcommand{\Bu}[0]{\mathbf{u}}
\newcommand{\Bv}[0]{\mathbf{v}}
\newcommand{\Bw}[0]{\mathbf{w}}
\newcommand{\Bx}[0]{\mathbf{x}}
\newcommand{\By}[0]{\mathbf{y}}
\newcommand{\Bz}[0]{\mathbf{z}}
\newcommand{\BA}[0]{\mathbf{A}}
\newcommand{\BB}[0]{\mathbf{B}}
\newcommand{\BC}[0]{\mathbf{C}}
\newcommand{\BD}[0]{\mathbf{D}}
\newcommand{\BE}[0]{\mathbf{E}}
\newcommand{\BF}[0]{\mathbf{F}}
\newcommand{\BG}[0]{\mathbf{G}}
\newcommand{\BH}[0]{\mathbf{H}}
\newcommand{\BI}[0]{\mathbf{I}}
\newcommand{\BJ}[0]{\mathbf{J}}
\newcommand{\BK}[0]{\mathbf{K}}
\newcommand{\BL}[0]{\mathbf{L}}
\newcommand{\BM}[0]{\mathbf{M}}
\newcommand{\BN}[0]{\mathbf{N}}
\newcommand{\BO}[0]{\mathbf{O}}
\newcommand{\BP}[0]{\mathbf{P}}
\newcommand{\BQ}[0]{\mathbf{Q}}
\newcommand{\BR}[0]{\mathbf{R}}
\newcommand{\BS}[0]{\mathbf{S}}
\newcommand{\BT}[0]{\mathbf{T}}
\newcommand{\BU}[0]{\mathbf{U}}
\newcommand{\BV}[0]{\mathbf{V}}
\newcommand{\BW}[0]{\mathbf{W}}
\newcommand{\BX}[0]{\mathbf{X}}
\newcommand{\BY}[0]{\mathbf{Y}}
\newcommand{\BZ}[0]{\mathbf{Z}}

\newcommand{\Bzero}[0]{\mathbf{0}}
\newcommand{\Btheta}[0]{\boldsymbol{\theta}}
\newcommand{\Btau}[0]{\boldsymbol{\tau}}
\newcommand{\Bomega}[0]{\boldsymbol{\omega}}

%
% shorthand for unit vectors
%
\newcommand{\acap}[0]{\hat{\Ba}}
\newcommand{\bcap}[0]{\hat{\Bb}}
\newcommand{\ccap}[0]{\hat{\Bc}}
\newcommand{\dcap}[0]{\hat{\Bd}}
\newcommand{\ecap}[0]{\hat{\Be}}
\newcommand{\fcap}[0]{\hat{\Bf}}
\newcommand{\gcap}[0]{\hat{\Bg}}
\newcommand{\hcap}[0]{\hat{\Bh}}
\newcommand{\icap}[0]{\hat{\Bi}}
\newcommand{\jcap}[0]{\hat{\Bj}}
\newcommand{\kcap}[0]{\hat{\Bk}}
\newcommand{\lcap}[0]{\hat{\Bl}}
\newcommand{\mcap}[0]{\hat{\Bm}}
\newcommand{\ncap}[0]{\hat{\Bn}}
\newcommand{\ocap}[0]{\hat{\Bo}}
\newcommand{\pcap}[0]{\hat{\Bp}}
\newcommand{\qcap}[0]{\hat{\Bq}}
\newcommand{\rcap}[0]{\hat{\Br}}
\newcommand{\scap}[0]{\hat{\Bs}}
\newcommand{\tcap}[0]{\hat{\Bt}}
\newcommand{\ucap}[0]{\hat{\Bu}}
\newcommand{\vcap}[0]{\hat{\Bv}}
\newcommand{\wcap}[0]{\hat{\Bw}}
\newcommand{\xcap}[0]{\hat{\Bx}}
\newcommand{\ycap}[0]{\hat{\By}}
\newcommand{\zcap}[0]{\hat{\Bz}}
\newcommand{\thetacap}[0]{\hat{\Btheta}}

%
% to write R^n and C^n in a distinguishable fashion.  Perhaps change this
% to the double lined characters upon figuring out how to do so.
%
\newcommand{\C}[1]{$\mathbb{C}^{#1}$}
\newcommand{\R}[1]{$\mathbb{R}^{#1}$}

%
% various generally useful helpers
%

% derivative of #1 wrt. #2:
\newcommand{\D}[2] {\frac {d#2} {d#1}}

\newcommand{\inv}[1]{\frac{1}{#1}}
\newcommand{\cross}[0]{\times}

\newcommand{\abs}[1]{\lvert{#1}\rvert}
\newcommand{\norm}[1]{\lVert{#1}\rVert}
\newcommand{\innerprod}[2]{\langle{#1}, {#2}\rangle}
\newcommand{\dotprod}[2]{{#1} \cdot {#2}}
\newcommand{\bdotprod}[2]{\left({#1} \cdot {#2}\right)}
\newcommand{\crossprod}[2]{{#1} \cross {#2}}
\newcommand{\tripleprod}[3]{\dotprod{\left(\crossprod{#1}{#2}\right)}{#3}}

\DeclareMathOperator{\Proj}{Proj}
\DeclareMathOperator{\Span}{span}
\DeclareMathOperator{\Sgn}{sgn}
\DeclareMathOperator{\Area}{Area}
\DeclareMathOperator{\Volume}{Volume}

%
% A few miscellaneous things specific to this document
%
\newcommand{\crossop}[1]{\crossprod{#1}{}}

% R2 vector.
\newcommand{\VectorTwo}[2]{
\begin{bmatrix}
 {#1} \\
 {#2}
\end{bmatrix}
}

\newcommand{\VectorN}[1]{
\begin{bmatrix}
{#1}_1 \\
{#1}_2 \\
\vdots \\
{#1}_N \\
\end{bmatrix}
}

\newcommand{\DETuvij}[4]{
\begin{vmatrix}
 {#1}_{#3} & {#1}_{#4} \\
 {#2}_{#3} & {#2}_{#4}
\end{vmatrix}
}

\newcommand{\DETuvwijk}[6]{
\begin{vmatrix}
 {#1}_{#4} & {#1}_{#5} & {#1}_{#6} \\
 {#2}_{#4} & {#2}_{#5} & {#2}_{#6} \\
 {#3}_{#4} & {#3}_{#5} & {#3}_{#6}
\end{vmatrix}
}

\newcommand{\DETuvwxijkl}[8]{
\begin{vmatrix}
 {#1}_{#5} & {#1}_{#6} & {#1}_{#7} & {#1}_{#8} \\
 {#2}_{#5} & {#2}_{#6} & {#2}_{#7} & {#2}_{#8} \\
 {#3}_{#5} & {#3}_{#6} & {#3}_{#7} & {#3}_{#8} \\
 {#4}_{#5} & {#4}_{#6} & {#4}_{#7} & {#4}_{#8} \\
\end{vmatrix}
}

%\newcommand{\DETuvwxyijklm}[10]{
%\begin{vmatrix}
% {#1}_{#6} & {#1}_{#7} & {#1}_{#8} & {#1}_{#9} & {#1}_{#10} \\
% {#2}_{#6} & {#2}_{#7} & {#2}_{#8} & {#2}_{#9} & {#2}_{#10} \\
% {#3}_{#6} & {#3}_{#7} & {#3}_{#8} & {#3}_{#9} & {#3}_{#10} \\
% {#4}_{#6} & {#4}_{#7} & {#4}_{#8} & {#4}_{#9} & {#4}_{#10} \\
% {#5}_{#6} & {#5}_{#7} & {#5}_{#8} & {#5}_{#9} & {#5}_{#10}
%\end{vmatrix}
%}

% R3 vector.
\newcommand{\VectorThree}[3]{
\begin{bmatrix}
 {#1} \\
 {#2} \\
 {#3}
\end{bmatrix}
}



\author{Peeter Joot}
\email{peeter.joot@gmail.com}


%There will be drop in hours this week.  I will hold the usual office hour
%Friday: 4-5.
%
%Exam: Dec 12.
%Extra office hours:
%Larkin Building: Rm. 212
%(Devonshire place)
%
%Sat 9-5
%Sun 9-5
%
%There will be a lecture on Wed.

\chapter{PHY456H1F: Quantum Mechanics II.  Lecture L24 (Taught by Prof J.E. Sipe).  3D Scattering cross sections (cont.)}
\label{chap:qmTwoL24}
%\useCCL
\blogpage{http://sites.google.com/site/peeterjoot2/math2011/qmTwoL24.pdf}
\date{Dec 5, 2011}
\revisionInfo{qmTwoL24.tex}

\beginArtWithToc
%\beginArtNoToc

\section{Disclaimer.}

Peeter's lecture notes from class.  May not be entirely coherent.

\section{Scattering cross sections.}

READING: \S 20 \cite{desai2009quantum}

Recall that we are studing the case of a potential that is zero outside of a fixed bound, $V(\Br) = 0$ for $r > r_0$, as in figure (\ref{fig:qmTwoL24:qmTwoL22fig5})

\begin{figure}[htp]
   \centering
   \includegraphics[totalheight=0.3\textheight]{qmTwoL22fig5}
   \caption{Bounded potential.}\label{fig:qmTwoL24:qmTwoL22fig5}
\end{figure}

and were looking for solutions to Schr\"{o}dinger's equation 

\begin{equation}\label{eqn:qmTwoL24:10}
-\frac{\hbar^2}{2\mu} \spacegrad^2
\psi_\Bk(\Br)
+ V(\Br)
\psi_\Bk(\Br)
=
\frac{\hbar^2 \Bk^2}{2 \mu}
\psi_\Bk(\Br),
\end{equation}

in regions of space, where $r > r_0$ is very large.  We found

\begin{equation}\label{eqn:qmTwoL24:30}
\psi_\Bk(\Br) \sim e^{i \Bk \cdot \Br} + \frac{e^{i k r}}{r} f_\Bk(\theta, \phi).
\end{equation}

For $r \le r_0$ this will be something much more complicated.

To study scattering we'll use the concept of probability flux as in electromagnetism

\begin{equation}\label{eqn:qmTwoL24:50}
\spacegrad \cdot \Bj + \dot{\rho} = 0
\end{equation}

Using

\begin{equation}\label{eqn:qmTwoL24:70}
\psi(\Br, t) =
\psi_\Bk(\Br)^\conj
\psi_\Bk(\Br)
\end{equation}

we find

\begin{equation}\label{eqn:qmTwoL24:90}
\Bj(\Br, t) = \frac{\hbar}{2 \mu i} \Bigl(
\psi_\Bk(\Br)^\conj \spacegrad \psi_\Bk(\Br)
- (\spacegrad \psi_\Bk^\conj(\Br)) \psi_\Bk(\Br)
\Bigr)
\end{equation}

when

\begin{equation}\label{eqn:qmTwoL24:110}
-\frac{\hbar^2}{2\mu} \spacegrad^2
\psi_\Bk(\Br)
+ V(\Br)
\psi_\Bk(\Br)
=
i \hbar \PD{t}{
\psi_\Bk(\Br)
}
\end{equation}

In a fashion similar to what we did in the 1D case, let's suppose that we can write our wave function

\begin{equation}\label{eqn:qmTwoL24:130}
\psi(\Br, t_{\text{initial}}) = \int d^3k \alpha(\Bk, t_{\text{initial}}) \psi_\Bk(\Br)
\end{equation}

and treat the scattering as the scattering of a plane wave front (idealizing a set of wave packets) off of the object of interest as depicted in figure (\ref{fig:qmTwoL24:qmTwoL24fig3})

\begin{figure}[htp]
   \centering
   \includegraphics[totalheight=0.3\textheight]{qmTwoL24fig3}
   \caption{plane wave front incident on particle}\label{fig:qmTwoL24:qmTwoL24fig3}
\end{figure}

We assume that our incoming particles are sufficiently localized in $k$ space as depicted in the idealized representation of figure (\ref{fig:qmTwoL24:qmTwoL24fig4})

\begin{figure}[htp]
   \centering
   \includegraphics[totalheight=0.3\textheight]{qmTwoL24fig4}
   \caption{k space localized wave packet}\label{fig:qmTwoL24:qmTwoL24fig4}
\end{figure}

we assume that $\alpha(\Bk, t_{\text{initial}})$ is localized.

\begin{equation}\label{eqn:qmTwoL24:150}
\psi(\Br, t_{\text{initial}}) =
\int d^3k
\left(
\alpha(\Bk, t_{\text{initial}})
e^{i k_z z}
+
\alpha(\Bk, t_{\text{initial}}) \frac{e^{i k r}}{r} f_\Bk(\theta, \phi)
\right)
\end{equation}

We suppose that

\begin{equation}\label{eqn:qmTwoL24:170}
\alpha(\Bk, t_{\text{initial}}) = \alpha(\Bk) e^{-i \hbar k^2 t_{\text{initial}}/ 2\mu}
\end{equation}

where this is chosen ($\alpha(\Bk, t_{\text{initial}})$ is built in this fashion) so that this is non-zero for $z$ large in magnitude and negative.

This last integral can be approximated
\begin{equation}\label{eqn:qmTwoL24:190}
\begin{aligned}
\int d^3k
\alpha(\Bk, t_{\text{initial}}) \frac{e^{i k r}}{r} f_\Bk(\theta, \phi)
&\approx
\frac{f_{\Bk_0}(\theta, \phi)}{r}
\int d^3k
\alpha(\Bk, t_{\text{initial}}) e^{i k r} \\
&\rightarrow 0
\end{aligned}
\end{equation}

This is very much like the 1D case where we found no reflected component for our initial time.

We'll normally look in a locality well away from the wave front as indicted in figure (\ref{fig:qmTwoL24:qmTwoL24fig5})

\begin{figure}[htp]
   \centering
   \includegraphics[totalheight=0.3\textheight]{qmTwoL24fig5}
   \caption{point of measurement of scattering cross section}\label{fig:qmTwoL24:qmTwoL24fig5}
\end{figure}

There are situations where we do look in the locality of the wave front that has been scattered.

\subsection{}

Our income wave is of the form

\begin{equation}\label{eqn:qmTwoL24:210}
\psi_i = A e^{i k z} e^{-i \hbar k^2 t/2 \mu}
\end{equation}

Here we've made the approximation that $k = \Abs{\Bk} \sim k_z$.  We can calculate the probability current

\begin{equation}\label{eqn:qmTwoL24:230}
\Bj = \zcap \frac{\hbar k}{\mu} A
\end{equation}

(notice the $v = p/m$ like term above, with $p = \hbar k$).

For the scattered wave (dropping $A$ factor)

\begin{align*}
\Bj &=
\frac{\hbar}{2 \mu i}
\left(
f_\Bk^\conj(\theta, \phi) \frac{e^{-i k r}}{r} \spacegrad \left(
f_\Bk(\theta, \phi) \frac{e^{i k r}}{r}
\right)
-
\spacegrad \left(
f_\Bk^\conj(\theta, \phi) \frac{e^{-i k r}}{r}
\right)
f_\Bk(\theta, \phi) \frac{e^{i k r}}{r}
\right)
\\
&\approx
\frac{\hbar}{2 \mu i}
\left(
f_\Bk^\conj(\theta, \phi) \frac{e^{-i k r}}{r} i k \rcap f_\Bk(\theta, \phi)
\frac{e^{i k r}}{r}
-
f_\Bk^\conj(\theta, \phi) \frac{e^{-i k r}}{r} (-i k \rcap) f_\Bk(\theta, \phi)
\frac{e^{i k r}}{r}
\right)
\end{align*}

We find that the radial portion of the current density is

\begin{align*}
\rcap \cdot \Bj
&= \frac{\hbar}{2 \mu i} \Abs{f}^2 \frac{ 2 i k }{r^2} \\
&= \frac{\hbar k}{\mu} \inv{r^2} \Abs{f}^2,
\end{align*}

and the flux through our element of solid angle is

\begin{align*}
\rcap dA \cdot \Bj
&=
\frac{\text{probability}}{\text{unit area per time}} \times \text{area}  \\
&= \frac{\text{probability}}{\text{unit time}} \\
&=
\frac{\hbar k}{\mu} \frac{\Abs{f_\Bk(\theta, \phi)}^2}{r^2} r^2 d\Omega \\
&=
\frac{\hbar k }{\mu}
\Abs{f_\Bk(\theta, \phi)}^2 d\Omega \\
&=
j_{\text{incoming}}
\underbrace{\Abs{f_\Bk(\theta, \phi)}^2}_{d\sigma/d\Omega} d\Omega.
\end{align*}

We identify the scattering cross section above

\begin{equation}\label{eqn:qmTwoL24:250}
\frac{d\sigma}{d\Omega}
=
\Abs{f_\Bk(\theta, \phi)}^2
\end{equation}

\begin{equation}\label{eqn:qmTwoL24:270}
\sigma = \int \Abs{f_\Bk(\theta, \phi)}^2 d\Omega
\end{equation}

We've been somewhat unrealistic here since we've used a plane wave approximation, and can as in figure (\ref{fig:qmTwoL24:qmTwoL24fig6})

\begin{figure}[htp]
   \centering
   \includegraphics[totalheight=0.3\textheight]{qmTwoL24fig6}
   \caption{Plane wave vs packet wave front}\label{fig:qmTwoL24:qmTwoL24fig6}
\end{figure}

will actually produce the same answer.  For details we are referred to \cite{messiah1999quantum} and \cite{taylor1972scattering}.

\subsection{Working towards a solution}

We've done a bunch of stuff here but are not much closer to a real solution because we don't actually know what $f_\Bk$ is.

Let's write Schr\"{o}dinger

\begin{equation}\label{eqn:qmTwoL24:290}
-\frac{\hbar^2}{2\mu} \spacegrad^2
\psi_\Bk(\Br)
+ V(\Br)
\psi_\Bk(\Br)
=
\frac{\hbar^2 \Bk^2}{2 \mu}
\psi_\Bk(\Br),
\end{equation}

instead as

\begin{equation}\label{eqn:qmTwoL24:310}
(\spacegrad^2 + \Bk^2)
\psi_\Bk(\Br)
= s(\Br)
\end{equation}

where

\begin{equation}\label{eqn:qmTwoL24:330}
s(\Br) = \frac{2\mu}{\hbar} V(\Br) \psi_\Bk(\Br)
\end{equation}

where $s(\Br)$ is really the particular solution to this differential problem.   We want

\begin{equation}\label{eqn:qmTwoL24:350}
\psi_\Bk(\Br) =
\psi_\Bk^{\text{homogeneous}}(\Br)
+ \psi_\Bk^{\text{particular}}(\Br)
\end{equation}

and

\begin{equation}\label{eqn:qmTwoL24:370}
\psi_\Bk^{\text{homogeneous}}(\Br) = e^{i \Bk \cdot \Br}
\end{equation}

\EndArticle
