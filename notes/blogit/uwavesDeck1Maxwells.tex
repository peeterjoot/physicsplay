%
% Copyright � 2016 Peeter Joot.  All Rights Reserved.
% Licenced as described in the file LICENSE under the root directory of this GIT repository.
%
\newcommand{\authorname}{Peeter Joot}
\newcommand{\email}{peeterjoot@protonmail.com}
\newcommand{\basename}{FIXMEbasenameUndefined}
\newcommand{\dirname}{notes/FIXMEdirnameUndefined/}

\renewcommand{\basename}{uwaves1}
\renewcommand{\dirname}{notes/ece1236/}
\newcommand{\keywords}{ECE1236H}
\newcommand{\authorname}{Peeter Joot}
\newcommand{\onlineurl}{http://sites.google.com/site/peeterjoot2/math2013/\basename.pdf}
\newcommand{\sourcepath}{\dirname\basename.tex}
\newcommand{\generatetitle}[1]{\chapter{#1}}

\newcommand{\vcsinfo}{%
\section*{}
\noindent{\color{DarkOliveGreen}{\rule{\linewidth}{0.1mm}}}
\paragraph{Document version}
%\paragraph{\color{Maroon}{Document version}}
{
\small
\begin{itemize}
\item Available online at:\\ 
\href{\onlineurl}{\onlineurl}
\item Git Repository: \input{./.revinfo/gitRepo.tex}
\item Source: \sourcepath
\item last commit: \input{./.revinfo/gitCommitString.tex}
\item commit date: \input{./.revinfo/gitCommitDate.tex}
\end{itemize}
}
}

%\PassOptionsToPackage{dvipsnames,svgnames}{xcolor}
\PassOptionsToPackage{square,numbers}{natbib}
\documentclass{scrreprt}

\usepackage[left=2cm,right=2cm]{geometry}
\usepackage[svgnames]{xcolor}
\usepackage{peeters_layout}

\usepackage{natbib}

\usepackage[
colorlinks=true,
bookmarks=false,
pdfauthor={\authorname, \email},
backref 
]{hyperref}

% http://tex.stackexchange.com/questions/75773/how-to-reference-problems-by-the-text-label-in-an-exercise-envioronment
\usepackage[english]{cleveref}
\crefname{Exercise}{exercise}{exercises}
\Crefname{Exercise}{Exercise}{Exercises}

\RequirePackage{titlesec}
\RequirePackage{ifthen}

% http://stackoverflow.com/questions/4932910/date-in-the-tabular-environment
\makeatletter
\let\insertdate\@date
\makeatother

\titleformat{\chapter}[display]
{\bfseries\Large}
{\color{DarkSlateGrey}\filleft \authorname
\ifthenelse{\isundefined{\studentnumber}}{}{\\ \studentnumber}
\ifthenelse{\isundefined{\email}}{}{\\ \email}
\ifthenelse{\isundefined{\dateintitle}}{}{\\ \insertdate}
%\ifthenelse{\isundefined{\coursename}}{}{\\ \coursename} % put in title instead.
}
{4ex}
{\color{DarkOliveGreen}{\titlerule}\color{Maroon}
\vspace{2ex}%
\filright}
[\vspace{2ex}%
\color{DarkOliveGreen}\titlerule
]

\newcommand{\beginArtWithToc}[0]{\begin{document}\tableofcontents}
\newcommand{\beginArtNoToc}[0]{\begin{document}}
\newcommand{\EndNoBibArticle}[0]{\end{document}}
\newcommand{\EndArticle}[0]{\bibliography{Bibliography}\bibliographystyle{plainnat}\end{document}}

% 
%\newcommand{\citep}[1]{\cite{#1}}

\colorSectionsForArticle



%\usepackage{ece1236}
\usepackage{peeters_braket}
\usepackage{siunitx}
\usepackage{enumerate}
\usepackage{peeters_layout_exercise}
\usepackage{peeters_figures}
\usepackage{mathtools}
\usepackage{esint} % oiint

\beginArtNoToc
\generatetitle{ECE1236H Microwave and Millimeter-Wave Techniques.  Lecture 1: Maxwell's equations review.  Taught by Prof.\ G.V. Eleftheriades}
%\chapter{Maxwell's equations}
\label{chap:uwaves1}

\paragraph{Disclaimer}

Peeter's lecture notes from class.  These may be incoherent and rough.

These are notes for the UofT course ECE1236H, Microwave and Millimeter-Wave Techniques, taught by Prof. G.V. Eleftheriades, covering \textchapref{{1}} \citep{pozar2009microwave} content.

\section{Maxwell's equations in integral form}
\paragraph{Faraday's law}

For the integration surface sketched in \cref{fig:deck1Maxwells:deck1MaxwellsFig1} Faraday's law is

\imageFigure{../../figures/ece1236/deck1MaxwellsFig1}{Stokes integration surface}{fig:deck1Maxwells:deck1MaxwellsFig1}{0.2}

\begin{dmath}\label{eqn:uwavesLecture1:20}
\int_\txtC \BE \cdot d\Bl = - \PD{t}{} \iint_\txtS \BB \cdot \ncap dS
\end{dmath}

\paragraph{Ampere's law}
\begin{dmath}\label{eqn:uwavesLecture1:40}
\int \BH \cdot d\Bl = \iint \BJ \cdot \ncap dS + \PD{t}{} \iint \BD \cdot \ncap dS
\end{dmath}

On the RHS we have the sum of the conduction and displacement currents respectively.

\paragraph{Units}

\begin{itemize}
\item \( \BE \) \si{V/m} : Electric field
\item \( \BH \) \si{A/m} : Magnetic field
\item \( \BB \) \si{Weber/m^2} : Magnetic flux density
\item \( \BD \) \si{Cb/m^2} : Electric flux density (Coloumbs/meter-squared)
\item \( \BJ \) \si{A/m^2} : Electric current density
\item \( \rho \) \si{Cb/m^3} : Electric charge density
\end{itemize}

\paragraph{Charge conservation law}

Consider a closed surface S with interior volume V, as sketched in \cref{fig:deck1Maxwells:deck1MaxwellsFig2}.  The charge conservation relation is

\imageFigure{../../figures/ece1236/deck1MaxwellsFig2}{Divergence theorem integration volume}{fig:deck1Maxwells:deck1MaxwellsFig2}{0.2}

\begin{dmath}\label{eqn:uwavesLecture1:60}
\oiint \BJ \cdot \ncap dS = -\PD{t}{} \int \rho dV
\end{dmath}

\paragraph{Maxwell's equations in integral form}

Faraday's
\begin{dmath}\label{eqn:uwavesLecture1:80}
\int_\txtC \BE \cdot d\Bl = - \PD{t}{} \iint_\txtS \BB \cdot \ncap dS
\end{dmath}

Ampere-Maxwell's law:

\begin{dmath}\label{eqn:uwavesLecture1:100}
\int \BH \cdot d\Bl = \iint \BJ \cdot \ncap dS + \PD{t}{} \iint \BD \cdot \ncap dS
\end{dmath}

Charge conservation:

\begin{dmath}\label{eqn:uwavesLecture1:120}
\oiint \BJ \cdot \ncap dS = -\PD{t}{} \int \rho dV
\end{dmath}

\section{Maxwell's equations in differential form}

Useful calculus theorems

\paragraph{Stokes' theorem}

For the integration surface sketched in \cref{fig:deck1Maxwells:deck1MaxwellsFig1} the 3D Stokes theorem statement is

\begin{dmath}\label{eqn:uwavesLecture1:140}
\oint_\txtC \BF \cdot d\Bl = \iint_\txtS \lr{ \spacegrad \cross \BF } \cdot \ncap dS
\end{dmath}

\paragraph{Gauss's theorem (divergence theorem)}

Integrating over a closed surface

\begin{dmath}\label{eqn:uwavesLecture1:160}
\oiint \BF \cdot \ncap dS = \iiint \spacegrad \cdot \BF dV
\end{dmath}

\begin{itemize}
\item \( \spacegrad f \) :  vector ( \( f \) is a scalar ) : Gradient of a scalar.
\item \( \spacegrad \cdot \BF \) :  vector ( \( \BF \) is a vector ) : Divergence of a vector.
\item \( \spacegrad \cross \BF \) :  vector ( \( \BF \) is a vector ) : Curl of a vector.
\end{itemize}

\paragraph{Faradays's law}

Referring again to \cref{fig:deck1Maxwells:deck1MaxwellsFig1}, and Stokes theorem, Faraday's law

\begin{dmath}\label{eqn:uwavesLecture1:180}
\oint_\txtC \BE \cdot d\Bl = - \PD{t}{} \iint_\txtS \BB \cdot \ncap dS
\end{dmath}

can be expressed using Stokes's theorem as

\begin{dmath}\label{eqn:uwavesLecture1:200}
\oint_\txtC \BE \cdot d\Bl = \iint_\txtS \lr{ \spacegrad \cross \BE} \cdot \ncap d\BS
=
- \PD{t}{} \iint_\txtS \BB \cdot \ncap dS.
\end{dmath}

Hence
\begin{dmath}\label{eqn:uwavesLecture1:220}
\oint_\txtC 
\lr{
\spacegrad \cross \BE
+ \PD{t}{} \BB 
}
\cdot \ncap dS
= 0,
\end{dmath}

or

\begin{dmath}\label{eqn:uwavesLecture1:240}
\spacegrad \cross \BE = -\PD{t}{\BB},
\end{dmath}

which is Faraday's law in differential form.

Similarly, Ampere's law in differential form is

\begin{dmath}\label{eqn:uwavesLecture1:260}
\spacegrad \cross \BH = \BJ + \PD{t}{\BD}.
\end{dmath}

\paragraph{Charge conservation law}

Starting with

\begin{dmath}\label{eqn:uwavesLecture1:280}
\oiint \BJ \cdot \ncap dS = -\PD{t}{} \int \rho dV,
\end{dmath}

and using Gauss's theorem
\begin{dmath}\label{eqn:uwavesLecture1:300}
\oiint  \BJ \cdot \ncap dS = \int_\txtV \spacegrad \cdot \BJ dV = 
-\PD{t}{} \int \rho dV,
\end{dmath}

or
\begin{dmath}\label{eqn:uwavesLecture1:320}
\int_\txtV \lr{ 
\spacegrad \cdot \BJ +\PD{t}{\rho }} dV = 0.
\end{dmath}

Hence
\begin{dmath}\label{eqn:uwavesLecture1:340}
\spacegrad \cdot \BJ = -\PD{t}{\rho}.
\end{dmath}

\paragraph{Summary: Maxwell's equations in differential form}

Faraday's law, Ampere-Maxwell law, and charge conservation law respectively:

\begin{subequations}
\label{eqn:uwavesLecture1:360}
\begin{dmath}\label{eqn:uwavesLecture1:380}
\spacegrad \cross \BE = -\PD{t}{\BB}
\end{dmath}
\begin{dmath}\label{eqn:uwavesLecture1:400}
\spacegrad \cross \BH = \BJ + \PD{t}{\BD}
\end{dmath}
\begin{dmath}\label{eqn:uwavesLecture1:420}
\spacegrad \cdot \BJ = - \PD{t}{\rho}
\end{dmath}
\end{subequations}

\paragraph{Derived laws}

Provides a vector \( \BF \) has second derivatives that commute, we must have

\begin{dmath}\label{eqn:uwavesLecture1:440}
\spacegrad \cdot \lr{ \spacegrad \cross \BF } = 0,
\end{dmath}

Faraday's law

\begin{dmath}\label{eqn:uwavesLecture1:460}
\spacegrad \cross \BE = -\PD{t}{\BB}
\end{dmath}

for a continuous electric field \( \BE \) gives

\begin{dmath}\label{eqn:uwavesLecture1:480}
\spacegrad \cdot \lr{ \spacegrad \cross \BE } = -\PD{t}{} \spacegrad \cdot \BB = 0.
\end{dmath}

The divergence must be independent of time or constant.  If constant, we can probably assume that a non-zero constant isn't physically relavant, in which case we would have

\begin{dmath}\label{eqn:uwavesLecture1:500}
\spacegrad \cdot \BB = 0.
\end{dmath}

Similarly, Ampere's law gives

\begin{dmath}\label{eqn:uwavesLecture1:520}
0 = \spacegrad \cdot \lr{ \spacegrad \BH } 
= \spacegrad \cdot \lr{ \BJ + \PD{t}{\BD} }
= - \PD{t}{\rho} + \PD{t}{} \spacegrad \cdot \BD,
\end{dmath}

where the last line follows from charge conservation.  This means that the quantity

\begin{dmath}\label{eqn:uwavesLecture1:540}
-\rho + \spacegrad \cdot \BD,
\end{dmath}

is constant or independent of time.  Again, assuming a non-zero constant value or time independant value isn't physically relavant, we have

\begin{dmath}\label{eqn:uwavesLecture1:560}
\spacegrad \cdot \BD = \rho.
\end{dmath}

Comment: Peeter: It is interesting to see \( \spacegrad \cdot \BB = 0\), and \( \spacegrad \cdot \BD = \rho \) presented as derived values, but it seems to me that some handwaving through the time independent and non-zero constant cases is required to get there.

\paragraph{Summary}

Independent equations, Faraday's, Ampere-Maxwell, and charge conservation:

\begin{subequations}
\label{eqn:uwavesLecture1:580}
\begin{dmath}\label{eqn:uwavesLecture1:600}
\spacegrad \cross \BE = -\PD{t}{\BB}
\end{dmath}
\begin{dmath}\label{eqn:uwavesLecture1:620}
\spacegrad \cross \BH = \BJ + \PD{t}{\BD}
\end{dmath}
\begin{dmath}\label{eqn:uwavesLecture1:640}
\spacegrad \cdot \BJ = - \PD{t}{\rho}
\end{dmath}
\end{subequations}

Derived equations (no magnetic charges, and Gauss's law respectively),

\begin{subequations}
\label{eqn:uwavesLecture1:660}
\begin{dmath}\label{eqn:uwavesLecture1:680}
\spacegrad \cdot \BB = 0
\end{dmath}
\begin{dmath}\label{eqn:uwavesLecture1:700}
\spacegrad \cdot \BD = \rho
\end{dmath}
\end{subequations}

Note can consider Gauss's law as the third independent equation, in which case the charge conservation law becomes a derived law.

\section{Constitutive relations}

Consider the number of unknowns in this mix of equations: \( \BE, \BH, \BD, \BB, \BJ \), and \( \rho \).  Five vectors and one scalar: 16 unknowns.  We have only \( 3 + 3 + 1 = 7 \) equations, so can consider Maxwell's equations to have an indeterminant form.

Further information is provided by the nature of the medium in which the fields exist.

\begin{equation}\label{eqn:uwavesLecture1:720}
\begin{aligned}
\BB &= \BF_1\lr{ \BE, \PD{t}{\BE}, \Abs{\PD{t}{\BE}}^2, \cdots \BH, \PD{t}{\BH}, \Abs{\PD{t}{\BH}}^2, \cdots } \\
\BD &= \BF_2\lr{ \BE, \PD{t}{\BE}, \Abs{\PD{t}{\BE}}^2, \cdots \BH, \PD{t}{\BH}, \Abs{\PD{t}{\BH}}^2, \cdots } \\
\BJ &= \BF_3\lr{ \BE, \PD{t}{\BE}, \Abs{\PD{t}{\BE}}^2, \cdots \BH, \PD{t}{\BH}, \Abs{\PD{t}{\BH}}^2, \cdots },
\end{aligned}
\end{equation}

where \( \BF_1, \BF_2, \BF_3 \) are designated vector functions which characterize the medium where the fields exist.

Assume 

\begin{enumerate}[(i)]
\item The medium is stationary.  i.e. \( \BF_1, \BF_2 \) and \( \BF_3 \) do not depend on time.
\item The medium is linear.  i.e. there is no dependence upon \( \BE^2, \BH^2, \Abs{\PD{t}{\BE}}^2, \Abs{\PD{t}{\BH}}^2 \) and higher order power terms.
\item The medium is homogeneous.  i.e. there is no \( x, y, z\) variation of \( \BF_1, \BF_2 \) and \( \BF_3 \).
\item The medium repsonds independently to \( \BE \) and \( \BH \).  Then one can write

\begin{equation}\label{eqn:uwavesLecture1:740}
\begin{aligned}
\BB &= \overbar{\Bmu} \cdot \BH \\
\BD &= \overbar{\Bepsilon} \cdot \BE \\
\BJ &= \overbar{\Bsigma} \cdot \BE
\end{aligned}
\end{equation}

where \( \overbar{\Bmu}, \overbar{\Bepsilon} \) and \( \overbar{\Bsigma} \) are the magnetic permeabilit, electric permittivity, and electric conductivity tensors respectively.  The last equation above is Ohm's law.
\item If the medium is also isotropic, meaning that \( \overbar{\Bmu}, \overbar{\Bepsilon} \) and \( \overbar{\Bsigma} \) are scalars, then we have simple media described by
\begin{equation}\label{eqn:uwavesLecture1:760}
\begin{aligned}
\BB &= \overbar{\mu}  \BH \\
\BD &= \overbar{\epsilon}  \BE \\
\BJ &= \overbar{\sigma}  \BE.
\end{aligned}
\end{equation}

In simple media we now have another \( 3 + 3 + 3 = 9 \) equations.  With 7 eauations from Maxwell's laws we have \( 9 + 7 = 16 \) equations and 16 unknowns.
\end{enumerate}

An example of a simple medium is free space where we have

\begin{equation}\label{eqn:uwavesLecture1:780}
\begin{aligned}
\mu &= \mu_0 = 4 \pi \times 10^{-7} \si{H/m} \\
\epsilon &= \epsilon_0 = 8.854 \times 10^{-12} \si{F/m} \\
\sigma &= 0.
\end{aligned}
\end{equation}

Maxwell's equations in simple media are

\begin{equation}\label{eqn:uwavesLecture1:800}
\begin{aligned}
\spacegrad \cross \BE &= -\mu \PD{t}{\BH} \\
\spacegrad \cross \BH &= \lr{ \sigma + \epsilon \PD{t}{} } \BE \\
\sigma \spacegrad \cdot \BE &= - \PD{t}{\rho}.
\end{aligned}
\end{equation}

\makeexample{Anisotropic media}{example:uwavesLecture1:1}{
Here \( \BD \) is not parallel with \( \BE \), and 
\( \BB \) is not parallel with \( \BH \).

\begin{itemize}
\item Crystals in the principle axis coordinate system can be described by diagonal tensors of the form

\begin{dmath}\label{eqn:uwavesLecture1:820}
\overbar{\Bepsilon} = 
\begin{bmatrix}
\epsilon_x & 0 & 0 \\
0 & \epsilon_y & 0 \\
0 & 0 & \epsilon_z \\
\end{bmatrix}.
\end{dmath}

This is electric anisotropy.

\item Uniaxial crystals where \( \epsilon_x = \epsilon_y \), so that

\begin{dmath}\label{eqn:uwavesLecture1:840}
\overbar{\Bepsilon} = 
\begin{bmatrix}
\epsilon & 0 & 0 \\
0 & \epsilon & 0 \\
0 & 0 & \epsilon_z \\
\end{bmatrix}.
\end{dmath}

The z axis is the optical axis.  
Positive uniaxila crystal if \( \epsilon_z > \epsilon \) and 
negative uniaxila crystal if \( \epsilon_z < \epsilon \).

\item Biaxial crystals when \(  \epsilon_x \ne \epsilon_y  \ne \epsilon_z \).
\end{itemize}
} % example

\makeexample{Bi-anisotropic media}{example:uwavesLecture1:2}{

\begin{equation}\label{eqn:uwavesLecture1:860}
\begin{aligned}
\BD &= \overbar{\Bepsilon} \cdot \BE + \overbar{\Bzeta} \cdot \BH \\
\BB &= \overbar{\BJ} \cdot \BE + \overbar{\Bmu} \cdot \BH
\end{aligned}
\end{equation}
} % example

\makeexample{}{example:uwavesLecture1:880}{
This is the case when \( 
\overbar{\Bepsilon}, \overbar{\Bzeta}, \overbar{\BJ} \) and \( \overbar{\Bmu} \) are all scalar quantities.  i.e. 

\begin{equation}\label{eqn:uwavesLecture1:900}
\begin{aligned}
\BD &= \epsilon \BE + \zeta \BH \\
\BB &= J \BE + \mu \BH
\end{aligned}
\end{equation}

An example is Chiral media like sugars, and DNA, where we have

\begin{equation}\label{eqn:uwavesLecture1:920}
\begin{aligned}
\BD &= \epsilon \BE - \chi \PD{t}{\BH} \\
\BB &= \mu \BH  + \chi \PD{t}{\BE},
\end{aligned}
\end{equation}

where \( \chi \) is the chiral parameter.
} % example

\paragraph{Simple medium}

When we have a stationary, linear, homogenous, and isotropic medium where 

\begin{equation}\label{eqn:uwavesLecture1:940}
\begin{aligned}
\BB &= \mu \BH \\
\BD &= \epsilon \BE \\
\BJ &= \sigma \BE
\end{aligned}
\end{equation}

then

\begin{enumerate}[(i)]
\item If \( \sigma = 0 \) : dielectric.
\item If \( \sigma \rightarrow \infty \) : conductor.
\item \( \epsilon \) (\si{F/m}): Permittivity.  \( \epsilon_r = \epsilon/\epsilon_0 \) is the relative permittivity (dielectric constant).
\item \( \mu \) (\si{H/m}) : Permeability.  \( \mu_r = \mu/\mu_0 \) is the relative permeability.
\end{enumerate}

Some material types:

\begin{itemize}
\item Non-magnetic materials, \( \mu_r = 1 \).
\item Diamagnetic materials, \( \mu_r < 1 \).  This is not common.
\item Paramagnetic materials, \( \mu_r > 1 \).
\item Ferromagnetic materials, \( \mu_r \gg 1 \).
\end{itemize}

\EndArticle
%\EndNoBibArticle
