%
% Copyright � 2016 Peeter Joot.  All Rights Reserved.
% Licenced as described in the file LICENSE under the root directory of this GIT repository.
%
%{
\newcommand{\authorname}{Peeter Joot}
\newcommand{\email}{peeterjoot@protonmail.com}
\newcommand{\basename}{FIXMEbasenameUndefined}
\newcommand{\dirname}{notes/FIXMEdirnameUndefined/}

\renewcommand{\basename}{magnetostaticsJacksonNotesForceAndTorque}
%\renewcommand{\dirname}{notes/phy1520/}
\renewcommand{\dirname}{notes/ece1228-electromagnetic-theory/}
%\newcommand{\dateintitle}{}
%\newcommand{\keywords}{}

\newcommand{\authorname}{Peeter Joot}
\newcommand{\onlineurl}{http://sites.google.com/site/peeterjoot2/math2013/\basename.pdf}
\newcommand{\sourcepath}{\dirname\basename.tex}
\newcommand{\generatetitle}[1]{\chapter{#1}}

\newcommand{\vcsinfo}{%
\section*{}
\noindent{\color{DarkOliveGreen}{\rule{\linewidth}{0.1mm}}}
\paragraph{Document version}
%\paragraph{\color{Maroon}{Document version}}
{
\small
\begin{itemize}
\item Available online at:\\ 
\href{\onlineurl}{\onlineurl}
\item Git Repository: \input{./.revinfo/gitRepo.tex}
\item Source: \sourcepath
\item last commit: \input{./.revinfo/gitCommitString.tex}
\item commit date: \input{./.revinfo/gitCommitDate.tex}
\end{itemize}
}
}

%\PassOptionsToPackage{dvipsnames,svgnames}{xcolor}
\PassOptionsToPackage{square,numbers}{natbib}
\documentclass{scrreprt}

\usepackage[left=2cm,right=2cm]{geometry}
\usepackage[svgnames]{xcolor}
\usepackage{peeters_layout}

\usepackage{natbib}

\usepackage[
colorlinks=true,
bookmarks=false,
pdfauthor={\authorname, \email},
backref 
]{hyperref}

% http://tex.stackexchange.com/questions/75773/how-to-reference-problems-by-the-text-label-in-an-exercise-envioronment
\usepackage[english]{cleveref}
\crefname{Exercise}{exercise}{exercises}
\Crefname{Exercise}{Exercise}{Exercises}

\RequirePackage{titlesec}
\RequirePackage{ifthen}

% http://stackoverflow.com/questions/4932910/date-in-the-tabular-environment
\makeatletter
\let\insertdate\@date
\makeatother

\titleformat{\chapter}[display]
{\bfseries\Large}
{\color{DarkSlateGrey}\filleft \authorname
\ifthenelse{\isundefined{\studentnumber}}{}{\\ \studentnumber}
\ifthenelse{\isundefined{\email}}{}{\\ \email}
\ifthenelse{\isundefined{\dateintitle}}{}{\\ \insertdate}
%\ifthenelse{\isundefined{\coursename}}{}{\\ \coursename} % put in title instead.
}
{4ex}
{\color{DarkOliveGreen}{\titlerule}\color{Maroon}
\vspace{2ex}%
\filright}
[\vspace{2ex}%
\color{DarkOliveGreen}\titlerule
]

\newcommand{\beginArtWithToc}[0]{\begin{document}\tableofcontents}
\newcommand{\beginArtNoToc}[0]{\begin{document}}
\newcommand{\EndNoBibArticle}[0]{\end{document}}
\newcommand{\EndArticle}[0]{\bibliography{Bibliography}\bibliographystyle{plainnat}\end{document}}

% 
%\newcommand{\citep}[1]{\cite{#1}}

\colorSectionsForArticle



\usepackage{peeters_layout_exercise}
\usepackage{peeters_braket}
\usepackage{peeters_figures}
\usepackage{siunitx}
%\usepackage{txfonts} % \ointclockwise

\beginArtNoToc

\generatetitle{Magnetostatic force and torque}
%\chapter{Magnetostatic force and torque}

%\label{chap:magnetostaticsJacksonNotesForceAndTorque}
% \citep{sakurai2014modern} pr X.Y
% \citep{pozar2009microwave}
% \citep{qftLectureNotes}
% \citep{doran2003gap}
% \citep{jackson1975cew}
% \citep{griffiths1999introduction}

In Jackson, the following equations for the vector potential, magnetostatic force and torque are derived

\begin{dmath}\label{eqn:magnetostaticsJacksonNotesForceAndTorque:n}
\Bm = \inv{2} \int \Bx' \cross \BJ(\Bx') d^3 x'
\end{dmath}
\begin{dmath}\label{eqn:magnetostaticsJacksonNotesForceAndTorque:n}
\BA = \frac{\mu_0}{4\pi} \frac{\Bm \cross \Bx}{\Abs{\Bx}^3}
\end{dmath}
\begin{dmath}\label{eqn:magnetostaticsJacksonNotesForceAndTorque:n}
\BF = \spacegrad( \Bm \cdot \BB ),
\end{dmath}
\begin{dmath}\label{eqn:magnetostaticsJacksonNotesForceAndTorque:n}
\BN = \Bm \cross \BB,
\end{dmath}

where \( \BB \) is an applied external magnetic field and \( \Bm \) is the magnetic dipole for the current in question.

These all follow from an analysis of localized current densities \( \BJ \), evaluated far enough away from the current sources.  I worked through the vector potential results, and made sense of his derivation (lots of sneaky tricks are required).  I've also done the same for the force and torque derivations.  While I now understand the mathematical steps he uses, there's a detail about the starting point of his derivation, where he writes

\begin{dmath}\label{eqn:magnetostaticsJacksonNotesForceAndTorque:n}
\BF = \int \BJ(\Bx) \cross \BB(\Bx) d^3 x
\end{dmath}

This is clearly the continuum generalization of the point particle Lorentz force equation, which for \( BE = 0 \) is:

\begin{dmath}\label{eqn:magnetostaticsJacksonNotesForceAndTorque:n}
\BF = q \Bv \cross \BB
\end{dmath}

For the point particle, this is the force on the particle when it is in the external field \( BB \).  i.e. this is the force at the position of the particle.

However, for the continuum Force equation, it integrates over all space.  How can we have a force that is applied to all space, as opposed to a force applied at a single point?


%}
\EndArticle
%\EndNoBibArticle
