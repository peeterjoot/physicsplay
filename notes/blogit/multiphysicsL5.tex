%
% Copyright � 2014 Peeter Joot.  All Rights Reserved.
% Licenced as described in the file LICENSE under the root directory of this GIT repository.
%
\newcommand{\authorname}{Peeter Joot}
\newcommand{\email}{peeterjoot@protonmail.com}
\newcommand{\basename}{FIXMEbasenameUndefined}
\newcommand{\dirname}{notes/FIXMEdirnameUndefined/}

\renewcommand{\basename}{multiphysicsL5}
\renewcommand{\dirname}{notes/ece1254/}
\newcommand{\keywords}{Condensed matter physics, ECE1254H}
\newcommand{\authorname}{Peeter Joot}
\newcommand{\onlineurl}{http://sites.google.com/site/peeterjoot2/math2013/\basename.pdf}
\newcommand{\sourcepath}{\dirname\basename.tex}
\newcommand{\generatetitle}[1]{\chapter{#1}}

\newcommand{\vcsinfo}{%
\section*{}
\noindent{\color{DarkOliveGreen}{\rule{\linewidth}{0.1mm}}}
\paragraph{Document version}
%\paragraph{\color{Maroon}{Document version}}
{
\small
\begin{itemize}
\item Available online at:\\ 
\href{\onlineurl}{\onlineurl}
\item Git Repository: \input{./.revinfo/gitRepo.tex}
\item Source: \sourcepath
\item last commit: \input{./.revinfo/gitCommitString.tex}
\item commit date: \input{./.revinfo/gitCommitDate.tex}
\end{itemize}
}
}

%\PassOptionsToPackage{dvipsnames,svgnames}{xcolor}
\PassOptionsToPackage{square,numbers}{natbib}
\documentclass{scrreprt}

\usepackage[left=2cm,right=2cm]{geometry}
\usepackage[svgnames]{xcolor}
\usepackage{peeters_layout}

\usepackage{natbib}

\usepackage[
colorlinks=true,
bookmarks=false,
pdfauthor={\authorname, \email},
backref 
]{hyperref}

% http://tex.stackexchange.com/questions/75773/how-to-reference-problems-by-the-text-label-in-an-exercise-envioronment
\usepackage[english]{cleveref}
\crefname{Exercise}{exercise}{exercises}
\Crefname{Exercise}{Exercise}{Exercises}

\RequirePackage{titlesec}
\RequirePackage{ifthen}

% http://stackoverflow.com/questions/4932910/date-in-the-tabular-environment
\makeatletter
\let\insertdate\@date
\makeatother

\titleformat{\chapter}[display]
{\bfseries\Large}
{\color{DarkSlateGrey}\filleft \authorname
\ifthenelse{\isundefined{\studentnumber}}{}{\\ \studentnumber}
\ifthenelse{\isundefined{\email}}{}{\\ \email}
\ifthenelse{\isundefined{\dateintitle}}{}{\\ \insertdate}
%\ifthenelse{\isundefined{\coursename}}{}{\\ \coursename} % put in title instead.
}
{4ex}
{\color{DarkOliveGreen}{\titlerule}\color{Maroon}
\vspace{2ex}%
\filright}
[\vspace{2ex}%
\color{DarkOliveGreen}\titlerule
]

\newcommand{\beginArtWithToc}[0]{\begin{document}\tableofcontents}
\newcommand{\beginArtNoToc}[0]{\begin{document}}
\newcommand{\EndNoBibArticle}[0]{\end{document}}
\newcommand{\EndArticle}[0]{\bibliography{Bibliography}\bibliographystyle{plainnat}\end{document}}

% 
%\newcommand{\citep}[1]{\cite{#1}}

\colorSectionsForArticle



\usepackage{kbordermatrix}
\usepackage{listings}

% FIXME: -'s aren't showing up:
\lstset{ %
language=C++,                % choose the language of the code
basicstyle=\footnotesize,       % the size of the fonts that are used for the code
numbers=left,                   % where to put the line-numbers
numberstyle=\footnotesize,      % the size of the fonts that are used for the line-numbers
stepnumber=1,                   % the step between two line-numbers. If it is 1 each line will be numbered
numbersep=5pt,                  % how far the line-numbers are from the code
backgroundcolor=\color{white},  % choose the background color. You must add \usepackage{color}
showspaces=false,               % show spaces adding particular underscores
showstringspaces=false,         % underline spaces within strings
showtabs=false,                 % show tabs within strings adding particular underscores
frame=single,                   % adds a frame around the code
tabsize=2,              % sets default tabsize to 2 spaces
captionpos=b,                   % sets the caption-position to bottom
breaklines=true,        % sets automatic line breaking
breakatwhitespace=false,    % sets if automatic breaks should only happen at whitespace
%escapeinside={\%}{)}          % if you want to add a comment within your code
}

\beginArtNoToc
\generatetitle{ECE1254H Modeling of Multiphysics Systems.  Lecture 5: Numerical error and conditioning.  Taught by Prof.\ Piero Triverio}
%\chapter{Numerical error and conditioning}
\label{chap:multiphysicsL5}

\section{Disclaimer}

Peeter's lecture notes from class.  These may be incoherent and rough.

\section{Numerical error and conditioning}

Related to a theorem on one of the slides:

\makedefinition{Strictly diagonally dominant}{dfn:multiphysicsL5:20}{

A matrix \( [ M_{ij} ] \) is strictly diagonally dominant if

\begin{dmath}\label{eqn:multiphysicsL5:40}
	\Abs{M_{ii}} > \sum_{j \ne i} \Abs{ M_{ij} } \forall i
\end{dmath}
}

For example, the stamp matrix

\begin{equation}\label{eqn:multiphysicsL5:60}
\kbordermatrix{
    & i      & j \\
i & \inv{R}  & -\inv{R} \\
j & -\inv{R} & \inv{R}
}
\end{equation}

is not strictly diagonally dominant.  For row $i$ this strict dominance can be achieved by adding a reference resistor

\begin{equation}\label{eqn:multiphysicsL5:80}
\kbordermatrix{
    & i      & j \\
i & \inv{R_0} + \inv{R}  & -\inv{R} \\
j & -\inv{R} & \inv{R}
}
\end{equation}

However, even with strict dominance, we can have trouble with ill posed (perturbative) systems.

Round off error examples with double precision 

\begin{dmath}\label{eqn:multiphysicsL5:100}
\lr{ 1 - 1 } + \pi 10^{-17} = \pi 10^{-17},
\end{dmath}

vs.

\begin{dmath}\label{eqn:multiphysicsL5:120}
\lr{ 1 + \pi 10^{-17} } -1 = 0.
\end{dmath}

This is demonstrated by

\begin{lstlisting}
#include <stdio.h>
#include <math.h>

// produces:
// 0 3.14159e-17
int main()
{
   double d1 = (1 + M_PI * 1e-17) - 1 ;
   double d2 = M_PI * 1e-17 ;

   printf( "%g %g\n", d1, d2 ) ;

   return 0 ;
}
\end{lstlisting}

Note that a union can be useful for exploring double precision representation

%\begin{lstlisting}
\begin{verbatim}
#define __STDC_FORMAT_MACROS
#include <stdio.h>
#include <math.h>
#include <inttypes.h>

struct doubleRepresentation
{
#if defined __LITTLE_ENDIAN // The ENDIAN macros here assume that the compiler is GCC
   uint64_t m : 52 ;
   uint64_t e : 11 ;
   uint64_t s : 1 ;
#elif defined __BIG_ENDIAN
   uint64_t m : 52 ;
   uint64_t e : 11 ;
   uint64_t s : 1 ;
#else
   #error unknown endian order.
#endif
} ;

union doubleU
{
   double               d ;
   uint64_t             u ;
   doubleRepresentation r ;
} ;

void printDoubleRep( const double d )
{
   doubleU un ;
   un.d = d ;

   if ( un.u )
   {
      printf( "%g: %s1.0x%" PRIx64" x 2^%" PRId64 "\n", un.d, un.r.s ? "-" : "", un.r.m, (int64_t)un.r.e - 0x3ff ) ;
   }
   else
   {
      printf( "%g: 0\n", un.d ) ;
   }
}

int main()
{
   double d1 = (1 + M_PI * 1e-17) - 1 ;
   double d2 = M_PI * 1e-17 ;

   printDoubleRep( d1 ) ;
   printDoubleRep( d2 ) ;
   printDoubleRep( 1 ) ;
   printDoubleRep( -0.5 ) ;
   printDoubleRep( 3 ) ;

   return 0 ;
}
\end{verbatim}
%\end{lstlisting}

\section{Exploring uniqueness and existence}

For a matrix system \( \overbar{M} x = \overbar{b} \) in column format, with

\begin{dmath}\label{eqn:multiphysicsL5:140}
\begin{bmatrix}
\overbar{M}_1 & \overbar{M}_2 & \cdots & \overbar{M}_N 
\end{bmatrix}
\begin{bmatrix}
	x_1 \\
	x_2 \\
	\vdots \\
	x_N
\end{bmatrix}
= \overbar{b}.
\end{dmath}

This can be written as

\begin{dmath}\label{eqn:multiphysicsL5:160}
\mathLabelBox
{
	x_1 
}
{weight}
	\overbar{M}_1 + x_2 \overbar{M}_2 + \cdots x_N \overbar{M}_N = \overbar{b}.
\end{dmath}

Linear dependence means

\begin{dmath}\label{eqn:multiphysicsL5:180}
y_1 \overbar{M}_1 + y_2 \overbar{M}_2 + \cdots y_N \overbar{M}_N = 0,
\end{dmath}

or \( M \overbar{y} = 0 \).

With a linear dependency an additional solution, given solution \( \overbar{x} \) is \( \overbar{x}^1 = \overbar{x} + \alpha y \).  This becomes relevant for numerical processing since for a system 

\begin{dmath}\label{eqn:multiphysicsL5:220}
M \overbar{x}^1 = \overbar{b}
\end{dmath}

we can often find 

\begin{dmath}\label{eqn:multiphysicsL5:200}
M \overbar{x} + \alpha M \overbar{y} = \overbar{b},
\end{dmath}

where \( \alpha M \overbar{y} \) is of order \( 10^{-20} \).

\section{Perturbation and norms}

Consider a perturbation to the system \( M \overbar{x} = \overbar{b} \)

\begin{dmath}\label{eqn:multiphysicsL5:240}
	\lr{ M + \delta M } \lr{ \overbar{x} + \delta \overbar{x} } = \overbar{b}.
\end{dmath}

Some vector norms

\begin{itemize}
\item \(L_1\) norm

\begin{dmath}\label{eqn:multiphysicsL5:260}
\Norm{ \overbar{x} }_1 = \sum_i \Abs{ x_i }
\end{dmath}

\item \(L_2\) norm

\begin{dmath}\label{eqn:multiphysicsL5:280}
\Norm{ \overbar{x} }_2 = \sqrt{ \sum_i \Abs{ x_i }^2 }
\end{dmath}

\item \(L_\infty\) norm

\begin{dmath}\label{eqn:multiphysicsL5:300}
\Norm{ \overbar{x} }_\infty = \max_i \Abs{ x_i }.
\end{dmath}
\end{itemize}

These are illustrated  for \( \overbar{x} = (x_1, x_2) \) in

F2

\section{matrix norm}

For

\begin{dmath}\label{eqn:multiphysicsL5:320}
\overbar{y} = M \overbar{x}
\end{dmath}

F3

The 1-norm is

\begin{dmath}\label{eqn:multiphysicsL5:340}
\Norm{ M } = \max_{ \Norm{\overbar{x}} = 1 } \Norm{ M \overbar{x} },
\end{dmath}

and the 2-norm is

\begin{dmath}\label{eqn:multiphysicsL5:360}
\Norm{ M }_2 = \max_{ \Norm{\overbar{x}}_2 = 1 } \Norm{ M \overbar{x} }_2.
\end{dmath}

\EndNoBibArticle
