%
% Copyright � 2015 Peeter Joot.  All Rights Reserved.
% Licenced as described in the file LICENSE under the root directory of this GIT repository.
%
\documentclass[]{eliblog}

\usepackage{amsmath}
\usepackage{mathpazo}

%
% shorthand for bold symbols, convenient for vectors and matrices
%
\newcommand{\Ba}[0]{\mathbf{a}}
\newcommand{\Bb}[0]{\mathbf{b}}
\newcommand{\Bc}[0]{\mathbf{c}}
\newcommand{\Bd}[0]{\mathbf{d}}
\newcommand{\Be}[0]{\mathbf{e}}
\newcommand{\Bf}[0]{\mathbf{f}}
\newcommand{\Bg}[0]{\mathbf{g}}
\newcommand{\Bh}[0]{\mathbf{h}}
\newcommand{\Bi}[0]{\mathbf{i}}
\newcommand{\Bj}[0]{\mathbf{j}}
\newcommand{\Bk}[0]{\mathbf{k}}
\newcommand{\Bl}[0]{\mathbf{l}}
\newcommand{\Bm}[0]{\mathbf{m}}
\newcommand{\Bn}[0]{\mathbf{n}}
\newcommand{\Bo}[0]{\mathbf{o}}
\newcommand{\Bp}[0]{\mathbf{p}}
\newcommand{\Bq}[0]{\mathbf{q}}
\newcommand{\Br}[0]{\mathbf{r}}
\newcommand{\Bs}[0]{\mathbf{s}}
\newcommand{\Bt}[0]{\mathbf{t}}
\newcommand{\Bu}[0]{\mathbf{u}}
\newcommand{\Bv}[0]{\mathbf{v}}
\newcommand{\Bw}[0]{\mathbf{w}}
\newcommand{\Bx}[0]{\mathbf{x}}
\newcommand{\By}[0]{\mathbf{y}}
\newcommand{\Bz}[0]{\mathbf{z}}
\newcommand{\BA}[0]{\mathbf{A}}
\newcommand{\BB}[0]{\mathbf{B}}
\newcommand{\BC}[0]{\mathbf{C}}
\newcommand{\BD}[0]{\mathbf{D}}
\newcommand{\BE}[0]{\mathbf{E}}
\newcommand{\BF}[0]{\mathbf{F}}
\newcommand{\BG}[0]{\mathbf{G}}
\newcommand{\BH}[0]{\mathbf{H}}
\newcommand{\BI}[0]{\mathbf{I}}
\newcommand{\BJ}[0]{\mathbf{J}}
\newcommand{\BK}[0]{\mathbf{K}}
\newcommand{\BL}[0]{\mathbf{L}}
\newcommand{\BM}[0]{\mathbf{M}}
\newcommand{\BN}[0]{\mathbf{N}}
\newcommand{\BO}[0]{\mathbf{O}}
\newcommand{\BP}[0]{\mathbf{P}}
\newcommand{\BQ}[0]{\mathbf{Q}}
\newcommand{\BR}[0]{\mathbf{R}}
\newcommand{\BS}[0]{\mathbf{S}}
\newcommand{\BT}[0]{\mathbf{T}}
\newcommand{\BU}[0]{\mathbf{U}}
\newcommand{\BV}[0]{\mathbf{V}}
\newcommand{\BW}[0]{\mathbf{W}}
\newcommand{\BX}[0]{\mathbf{X}}
\newcommand{\BY}[0]{\mathbf{Y}}
\newcommand{\BZ}[0]{\mathbf{Z}}

\newcommand{\Bzero}[0]{\mathbf{0}}
\newcommand{\Btheta}[0]{\boldsymbol{\theta}}
\newcommand{\Btau}[0]{\boldsymbol{\tau}}
\newcommand{\Bomega}[0]{\boldsymbol{\omega}}

%
% shorthand for unit vectors
%
\newcommand{\acap}[0]{\hat{\Ba}}
\newcommand{\bcap}[0]{\hat{\Bb}}
\newcommand{\ccap}[0]{\hat{\Bc}}
\newcommand{\dcap}[0]{\hat{\Bd}}
\newcommand{\ecap}[0]{\hat{\Be}}
\newcommand{\fcap}[0]{\hat{\Bf}}
\newcommand{\gcap}[0]{\hat{\Bg}}
\newcommand{\hcap}[0]{\hat{\Bh}}
\newcommand{\icap}[0]{\hat{\Bi}}
\newcommand{\jcap}[0]{\hat{\Bj}}
\newcommand{\kcap}[0]{\hat{\Bk}}
\newcommand{\lcap}[0]{\hat{\Bl}}
\newcommand{\mcap}[0]{\hat{\Bm}}
\newcommand{\ncap}[0]{\hat{\Bn}}
\newcommand{\ocap}[0]{\hat{\Bo}}
\newcommand{\pcap}[0]{\hat{\Bp}}
\newcommand{\qcap}[0]{\hat{\Bq}}
\newcommand{\rcap}[0]{\hat{\Br}}
\newcommand{\scap}[0]{\hat{\Bs}}
\newcommand{\tcap}[0]{\hat{\Bt}}
\newcommand{\ucap}[0]{\hat{\Bu}}
\newcommand{\vcap}[0]{\hat{\Bv}}
\newcommand{\wcap}[0]{\hat{\Bw}}
\newcommand{\xcap}[0]{\hat{\Bx}}
\newcommand{\ycap}[0]{\hat{\By}}
\newcommand{\zcap}[0]{\hat{\Bz}}
\newcommand{\thetacap}[0]{\hat{\Btheta}}

%
% to write R^n and C^n in a distinguishable fashion.  Perhaps change this
% to the double lined characters upon figuring out how to do so.
%
\newcommand{\C}[1]{$\mathbb{C}^{#1}$}
\newcommand{\R}[1]{$\mathbb{R}^{#1}$}

%
% various generally useful helpers
%

% derivative of #1 wrt. #2:
\newcommand{\D}[2] {\frac {d#2} {d#1}}

\newcommand{\inv}[1]{\frac{1}{#1}}
\newcommand{\cross}[0]{\times}

\newcommand{\abs}[1]{\lvert{#1}\rvert}
\newcommand{\norm}[1]{\lVert{#1}\rVert}
\newcommand{\innerprod}[2]{\langle{#1}, {#2}\rangle}
\newcommand{\dotprod}[2]{{#1} \cdot {#2}}
\newcommand{\bdotprod}[2]{\left({#1} \cdot {#2}\right)}
\newcommand{\crossprod}[2]{{#1} \cross {#2}}
\newcommand{\tripleprod}[3]{\dotprod{\left(\crossprod{#1}{#2}\right)}{#3}}

\DeclareMathOperator{\Proj}{Proj}
\DeclareMathOperator{\Span}{span}
\DeclareMathOperator{\Sgn}{sgn}
\DeclareMathOperator{\Area}{Area}
\DeclareMathOperator{\Volume}{Volume}

%
% A few miscellaneous things specific to this document
%
\newcommand{\crossop}[1]{\crossprod{#1}{}}

% R2 vector.
\newcommand{\VectorTwo}[2]{
\begin{bmatrix}
 {#1} \\
 {#2}
\end{bmatrix}
}

\newcommand{\VectorN}[1]{
\begin{bmatrix}
{#1}_1 \\
{#1}_2 \\
\vdots \\
{#1}_N \\
\end{bmatrix}
}

\newcommand{\DETuvij}[4]{
\begin{vmatrix}
 {#1}_{#3} & {#1}_{#4} \\
 {#2}_{#3} & {#2}_{#4}
\end{vmatrix}
}

\newcommand{\DETuvwijk}[6]{
\begin{vmatrix}
 {#1}_{#4} & {#1}_{#5} & {#1}_{#6} \\
 {#2}_{#4} & {#2}_{#5} & {#2}_{#6} \\
 {#3}_{#4} & {#3}_{#5} & {#3}_{#6}
\end{vmatrix}
}

\newcommand{\DETuvwxijkl}[8]{
\begin{vmatrix}
 {#1}_{#5} & {#1}_{#6} & {#1}_{#7} & {#1}_{#8} \\
 {#2}_{#5} & {#2}_{#6} & {#2}_{#7} & {#2}_{#8} \\
 {#3}_{#5} & {#3}_{#6} & {#3}_{#7} & {#3}_{#8} \\
 {#4}_{#5} & {#4}_{#6} & {#4}_{#7} & {#4}_{#8} \\
\end{vmatrix}
}

%\newcommand{\DETuvwxyijklm}[10]{
%\begin{vmatrix}
% {#1}_{#6} & {#1}_{#7} & {#1}_{#8} & {#1}_{#9} & {#1}_{#10} \\
% {#2}_{#6} & {#2}_{#7} & {#2}_{#8} & {#2}_{#9} & {#2}_{#10} \\
% {#3}_{#6} & {#3}_{#7} & {#3}_{#8} & {#3}_{#9} & {#3}_{#10} \\
% {#4}_{#6} & {#4}_{#7} & {#4}_{#8} & {#4}_{#9} & {#4}_{#10} \\
% {#5}_{#6} & {#5}_{#7} & {#5}_{#8} & {#5}_{#9} & {#5}_{#10}
%\end{vmatrix}
%}

% R3 vector.
\newcommand{\VectorThree}[3]{
\begin{bmatrix}
 {#1} \\
 {#2} \\
 {#3}
\end{bmatrix}
}



\author{Peeter Joot}
\email{peeter.joot@gmail.com}

%\documentclass[]{eliblogwidescreen}

\usepackage{amsmath}
\usepackage{mathpazo}

%
% shorthand for bold symbols, convenient for vectors and matrices
%
\newcommand{\Ba}[0]{\mathbf{a}}
\newcommand{\Bb}[0]{\mathbf{b}}
\newcommand{\Bc}[0]{\mathbf{c}}
\newcommand{\Bd}[0]{\mathbf{d}}
\newcommand{\Be}[0]{\mathbf{e}}
\newcommand{\Bf}[0]{\mathbf{f}}
\newcommand{\Bg}[0]{\mathbf{g}}
\newcommand{\Bh}[0]{\mathbf{h}}
\newcommand{\Bi}[0]{\mathbf{i}}
\newcommand{\Bj}[0]{\mathbf{j}}
\newcommand{\Bk}[0]{\mathbf{k}}
\newcommand{\Bl}[0]{\mathbf{l}}
\newcommand{\Bm}[0]{\mathbf{m}}
\newcommand{\Bn}[0]{\mathbf{n}}
\newcommand{\Bo}[0]{\mathbf{o}}
\newcommand{\Bp}[0]{\mathbf{p}}
\newcommand{\Bq}[0]{\mathbf{q}}
\newcommand{\Br}[0]{\mathbf{r}}
\newcommand{\Bs}[0]{\mathbf{s}}
\newcommand{\Bt}[0]{\mathbf{t}}
\newcommand{\Bu}[0]{\mathbf{u}}
\newcommand{\Bv}[0]{\mathbf{v}}
\newcommand{\Bw}[0]{\mathbf{w}}
\newcommand{\Bx}[0]{\mathbf{x}}
\newcommand{\By}[0]{\mathbf{y}}
\newcommand{\Bz}[0]{\mathbf{z}}
\newcommand{\BA}[0]{\mathbf{A}}
\newcommand{\BB}[0]{\mathbf{B}}
\newcommand{\BC}[0]{\mathbf{C}}
\newcommand{\BD}[0]{\mathbf{D}}
\newcommand{\BE}[0]{\mathbf{E}}
\newcommand{\BF}[0]{\mathbf{F}}
\newcommand{\BG}[0]{\mathbf{G}}
\newcommand{\BH}[0]{\mathbf{H}}
\newcommand{\BI}[0]{\mathbf{I}}
\newcommand{\BJ}[0]{\mathbf{J}}
\newcommand{\BK}[0]{\mathbf{K}}
\newcommand{\BL}[0]{\mathbf{L}}
\newcommand{\BM}[0]{\mathbf{M}}
\newcommand{\BN}[0]{\mathbf{N}}
\newcommand{\BO}[0]{\mathbf{O}}
\newcommand{\BP}[0]{\mathbf{P}}
\newcommand{\BQ}[0]{\mathbf{Q}}
\newcommand{\BR}[0]{\mathbf{R}}
\newcommand{\BS}[0]{\mathbf{S}}
\newcommand{\BT}[0]{\mathbf{T}}
\newcommand{\BU}[0]{\mathbf{U}}
\newcommand{\BV}[0]{\mathbf{V}}
\newcommand{\BW}[0]{\mathbf{W}}
\newcommand{\BX}[0]{\mathbf{X}}
\newcommand{\BY}[0]{\mathbf{Y}}
\newcommand{\BZ}[0]{\mathbf{Z}}

\newcommand{\Bzero}[0]{\mathbf{0}}
\newcommand{\Btheta}[0]{\boldsymbol{\theta}}
\newcommand{\Btau}[0]{\boldsymbol{\tau}}
\newcommand{\Bomega}[0]{\boldsymbol{\omega}}

%
% shorthand for unit vectors
%
\newcommand{\acap}[0]{\hat{\Ba}}
\newcommand{\bcap}[0]{\hat{\Bb}}
\newcommand{\ccap}[0]{\hat{\Bc}}
\newcommand{\dcap}[0]{\hat{\Bd}}
\newcommand{\ecap}[0]{\hat{\Be}}
\newcommand{\fcap}[0]{\hat{\Bf}}
\newcommand{\gcap}[0]{\hat{\Bg}}
\newcommand{\hcap}[0]{\hat{\Bh}}
\newcommand{\icap}[0]{\hat{\Bi}}
\newcommand{\jcap}[0]{\hat{\Bj}}
\newcommand{\kcap}[0]{\hat{\Bk}}
\newcommand{\lcap}[0]{\hat{\Bl}}
\newcommand{\mcap}[0]{\hat{\Bm}}
\newcommand{\ncap}[0]{\hat{\Bn}}
\newcommand{\ocap}[0]{\hat{\Bo}}
\newcommand{\pcap}[0]{\hat{\Bp}}
\newcommand{\qcap}[0]{\hat{\Bq}}
\newcommand{\rcap}[0]{\hat{\Br}}
\newcommand{\scap}[0]{\hat{\Bs}}
\newcommand{\tcap}[0]{\hat{\Bt}}
\newcommand{\ucap}[0]{\hat{\Bu}}
\newcommand{\vcap}[0]{\hat{\Bv}}
\newcommand{\wcap}[0]{\hat{\Bw}}
\newcommand{\xcap}[0]{\hat{\Bx}}
\newcommand{\ycap}[0]{\hat{\By}}
\newcommand{\zcap}[0]{\hat{\Bz}}
\newcommand{\thetacap}[0]{\hat{\Btheta}}

%
% to write R^n and C^n in a distinguishable fashion.  Perhaps change this
% to the double lined characters upon figuring out how to do so.
%
\newcommand{\C}[1]{$\mathbb{C}^{#1}$}
\newcommand{\R}[1]{$\mathbb{R}^{#1}$}

%
% various generally useful helpers
%

% derivative of #1 wrt. #2:
\newcommand{\D}[2] {\frac {d#2} {d#1}}

\newcommand{\inv}[1]{\frac{1}{#1}}
\newcommand{\cross}[0]{\times}

\newcommand{\abs}[1]{\lvert{#1}\rvert}
\newcommand{\norm}[1]{\lVert{#1}\rVert}
\newcommand{\innerprod}[2]{\langle{#1}, {#2}\rangle}
\newcommand{\dotprod}[2]{{#1} \cdot {#2}}
\newcommand{\bdotprod}[2]{\left({#1} \cdot {#2}\right)}
\newcommand{\crossprod}[2]{{#1} \cross {#2}}
\newcommand{\tripleprod}[3]{\dotprod{\left(\crossprod{#1}{#2}\right)}{#3}}

\DeclareMathOperator{\Proj}{Proj}
\DeclareMathOperator{\Span}{span}
\DeclareMathOperator{\Sgn}{sgn}
\DeclareMathOperator{\Area}{Area}
\DeclareMathOperator{\Volume}{Volume}

%
% A few miscellaneous things specific to this document
%
\newcommand{\crossop}[1]{\crossprod{#1}{}}

% R2 vector.
\newcommand{\VectorTwo}[2]{
\begin{bmatrix}
 {#1} \\
 {#2}
\end{bmatrix}
}

\newcommand{\VectorN}[1]{
\begin{bmatrix}
{#1}_1 \\
{#1}_2 \\
\vdots \\
{#1}_N \\
\end{bmatrix}
}

\newcommand{\DETuvij}[4]{
\begin{vmatrix}
 {#1}_{#3} & {#1}_{#4} \\
 {#2}_{#3} & {#2}_{#4}
\end{vmatrix}
}

\newcommand{\DETuvwijk}[6]{
\begin{vmatrix}
 {#1}_{#4} & {#1}_{#5} & {#1}_{#6} \\
 {#2}_{#4} & {#2}_{#5} & {#2}_{#6} \\
 {#3}_{#4} & {#3}_{#5} & {#3}_{#6}
\end{vmatrix}
}

\newcommand{\DETuvwxijkl}[8]{
\begin{vmatrix}
 {#1}_{#5} & {#1}_{#6} & {#1}_{#7} & {#1}_{#8} \\
 {#2}_{#5} & {#2}_{#6} & {#2}_{#7} & {#2}_{#8} \\
 {#3}_{#5} & {#3}_{#6} & {#3}_{#7} & {#3}_{#8} \\
 {#4}_{#5} & {#4}_{#6} & {#4}_{#7} & {#4}_{#8} \\
\end{vmatrix}
}

%\newcommand{\DETuvwxyijklm}[10]{
%\begin{vmatrix}
% {#1}_{#6} & {#1}_{#7} & {#1}_{#8} & {#1}_{#9} & {#1}_{#10} \\
% {#2}_{#6} & {#2}_{#7} & {#2}_{#8} & {#2}_{#9} & {#2}_{#10} \\
% {#3}_{#6} & {#3}_{#7} & {#3}_{#8} & {#3}_{#9} & {#3}_{#10} \\
% {#4}_{#6} & {#4}_{#7} & {#4}_{#8} & {#4}_{#9} & {#4}_{#10} \\
% {#5}_{#6} & {#5}_{#7} & {#5}_{#8} & {#5}_{#9} & {#5}_{#10}
%\end{vmatrix}
%}

% R3 vector.
\newcommand{\VectorThree}[3]{
\begin{bmatrix}
 {#1} \\
 {#2} \\
 {#3}
\end{bmatrix}
}



\author{Peeter Joot}
\email{peeter.joot@gmail.com}


\chapter{PHY454H1S\\Continuum Mechanics.  Lecture 11: Worked examples of Navier-Stokes solutions.  Taught by Prof. K. Das.}
\label{chap:continuumL11}
%\useCCL
\blogpage{http://sites.google.com/site/peeterjoot2/math2012/continuumL11.pdf}
\date{Feb 15, 2012}
\gitRevisionInfo{continuumL11}

\beginArtWithToc
%\beginArtNoToc

\section{Disclaimer.}

Peeter's lecture notes from class.  May not be entirely coherent.

\section{Navier-Stokes equation.}

The Navier-Stokes equation (our fluids equivalent to Newton's second law) was found to be

\begin{equation}\label{eqn:continuumL11:10}
\rho \PD{t}{\Bu} + \rho (\Bu \cdot \spacegrad) \Bu = - \spacegrad p + \mu \spacegrad^2 \Bu + \rho \Bf.
\end{equation}

In this course we'll focus on the incompressible case where we have

\begin{equation}\label{eqn:continuumL11:30}
\spacegrad \cdot \Bu = 0
\end{equation}

We watched a video of the rocking tank as in figure (\ref{fig:continuumL11:continuumL11fig1})
\begin{figure}[htp]
   \centering
   \includegraphics[totalheight=0.2\textheight]{continuumL11fig1}
   \caption{Rocking tank velocity matching.}\label{fig:continuumL11:continuumL11fig1}
\end{figure}

The boundary condition that accounted for the matching of the die marker is that we have \underline{no slipping} at the interface.  Writing for the tangent to the interface

\begin{equation}\label{eqn:continuumL11:50}
\Btau
\end{equation}

then this condition at the interface is described mathematically by the conditions

\begin{align}\label{eqn:continuumL11:70}
\Bu_A \cdot \Btau &= \Bu_B \cdot \Btau \\
\Bu_A \cdot \ncap &= \Bu_B \cdot \ncap.
\end{align}

Referring to figure (\ref{fig:continuumL11:continuumL11fig2}) where the tangents and normals are depicted

\begin{figure}[htp]
   \centering
   \includegraphics[totalheight=0.2\textheight]{continuumL11fig2}
   \caption{Normals and tangents at interface for 2D system}\label{fig:continuumL11:continuumL11fig2}
\end{figure}

an example representation of the normal and tangent vectors for the fluids are

\begin{align}\label{eqn:continuumL11:90}
\Btau &= 
\begin{bmatrix}
1 \\
0
\end{bmatrix} \\
\ncap &= 
\begin{bmatrix}
0 \\
1
\end{bmatrix} 
\end{align}

For the traction vector

\begin{equation}\label{eqn:continuumL11:110}
T_i = \sigma_{ij} n_j,
\end{equation}

we also have at the interface we must have matching of

\begin{equation}\label{eqn:continuumL11:130}
\Btau \cdot \BT.
\end{equation}

More explicitly, in coordinates this is

\begin{equation}\label{eqn:continuumL11:150}
\evalbar{\tau_i (\sigma_{ij} n_j)}{A} =
\evalbar{\tau_i (\sigma_{ij} n_j)}{B}
\end{equation}

\section{Steady rectilinear (unidirectional) flow.}

In this case we can fix our axis so that

\begin{equation}\label{eqn:continuumL11:170}
\Bu = \xcap u(x, y, z, t) 
\end{equation}

and have zero velocity components in the other directions

\begin{align}\label{eqn:continuumL11:190}
v &= 0 \\
w &= 0
\end{align}

Steady state is 

\begin{equation}\label{eqn:continuumL11:210}
\PD{t}{\Bu} = 0
\end{equation}

% prof called this the continuity condition?
The incompressibility condition, written explicitly, is

\begin{equation}\label{eqn:continuumL11:230}
\spacegrad \cdot \Bu = 0
\end{equation}

\begin{equation}\label{eqn:continuumL11:250}
\PD{x}{u} + \PD{y}{v} + \PD{z}{w} = 0
\end{equation}

This implies 

\begin{equation}\label{eqn:continuumL11:270}
\PD{x}{u} = 0
\end{equation}

so our velocity can only be function of the $y$ and $z$ coordinates only

\begin{equation}\label{eqn:continuumL11:290}
u = u(y, z).
\end{equation}

How about the non-linear term of N-S?

\begin{equation}\label{eqn:continuumL11:510}
(\spacegrad \cdot \Bu) \Bu
=
\left(u \PD{x}{}
+v \PD{y}{}
+w \PD{z}{} \right) (\xcap u( y, z) + 0 + 0 )
= 0
\end{equation}

N-S for incompressible fluids in the absence of body forces is reduced to

\begin{equation}\label{eqn:continuumL11:310}
0 = - \spacegrad p + \mu \spacegrad^2 \Bu 
\end{equation}

or

\begin{align}\label{eqn:continuumL11:330}
\PD{x}{p} &= \mu \left( \PDSq{y}{u} + \PDSq{z}{u} \right) \\
\PD{y}{p} &= 0 \\
\PD{z}{p} &= 0 
\end{align}

Operating on the first with an x partial we find

\begin{equation}\label{eqn:continuumL11:350}
\PDSq{x}{p} = \mu \left( \PDSq{y}{} \PD{x}{u} + \PDSq{z}{}\PD{x}{u} \right) = 0
\end{equation}

Since we have

\begin{equation}\label{eqn:continuumL11:370}
\PDSq{x}{p} = 0
\end{equation}

we also have

\begin{equation}\label{eqn:continuumL11:390}
\frac{d^2 p}{dx^2} = 0,
\end{equation}

so our pressure must be linear with position

\begin{equation}\label{eqn:continuumL11:410}
p = A x + B,
\end{equation}

as illustrated in figure (\ref{fig:continuumL11:continuumL11fig3})
\begin{figure}[htp]
   \centering
   \includegraphics[totalheight=0.2\textheight]{continuumL11fig3}
   \caption{Pressure gradient in 1D system.}\label{fig:continuumL11:continuumL11fig3}
\end{figure}

FIXME: latex:
\begin{equation}\label{eqn:continuumL11:430}
p =
\left\{
\begin{array}{l l}
p_0 & \quad \mbox{$x = 0$} \\
p_L & \quad \mbox{$x = L$} 
\end{array}
\right.
\end{equation}

we have

\begin{equation}\label{eqn:continuumL11:450}
p = \frac{p_L - p_0}{L} x + p_0
\end{equation}

and

\begin{equation}\label{eqn:continuumL11:470}
\frac{dp}{dx} = \frac{p_L - p_0}{L} = \text{constant} \equiv -G
\end{equation}

\subsection{Example: Shearing flow.}

The flows of this sort don't have to be trivial.  For example, even with constant pressure ($p_0 = p_L$) as in figure (\ref{fig:continuumL11:continuumL11fig4})

\begin{figure}[htp]
   \centering
   \includegraphics[totalheight=0.2\textheight]{continuumL11fig4}
   \caption{Velocity variation with height in shearing flow.}\label{fig:continuumL11:continuumL11fig4}
\end{figure}

we can have a ``shearing flow'' where the fluids at the top surface are not necessarily moving at the same rates as the fluid below that surface.  We have fluid flow in the $x$ direction only, and our velocity is a function only of the $y$ coordinate.

\begin{align}\label{eqn:continuumL11:490}
\Bu &= \xcap u(y) \\
G &= 0 \\
u(0) &= 0 \\
u(h) &= U.
\end{align}

For such a flow (FIXME: first of) \ref{eqn:continuumL11:330} simplifies to

\begin{equation}\label{eqn:continuumL11:530}
\frac{d^2 u}{dy^2} = 0
\end{equation}

with solution

\begin{equation}\label{eqn:continuumL11:550}
u = \frac{U}{h} y + u_0.
\end{equation}

%FIXME: Reading: \S XX from \cite{acheson1990elementary}

\subsection{Example: Channel flow}

\begin{align}\label{eqn:continuumL11:570}
\Bu &= \xcap u(y) \\
G &= - \frac{dp}{dx} \ne 0
\end{align}

N-S becomes

\begin{equation}\label{eqn:continuumL11:590}
\mu \frac{d^2 u}{dy^2} = -G
\end{equation}

or

\begin{equation}\label{eqn:continuumL11:610}
u = -\frac{G}{2 \mu} y^2 + A y + B
\end{equation}

The boundary value conditions are illustrated in figure (\ref{fig:continuumL11:continuumL11fig5})

\begin{figure}[htp]
   \centering
   \includegraphics[totalheight=0.2\textheight]{continuumL11fig5}
   \caption{1D Channel flow coordinate system setup.}\label{fig:continuumL11:continuumL11fig5}
\end{figure}

we find

\begin{equation}\label{eqn:continuumL11:630}
u(\pm h) = 
-\frac{G}{2 \mu} h^2 \pm A h + B = 0
\end{equation}

We find

\begin{equation}\label{eqn:continuumL11:650}
A = 0
\end{equation}

and 

\begin{equation}\label{eqn:continuumL11:670}
B = \frac{G}{2 \mu} h
\end{equation}

so our solution becomes

\begin{equation}\label{eqn:continuumL11:690}
u = \frac{G}{2 \mu} \Bigl( h^2 - y^2 \Bigr),
\end{equation}

a parabolic velocity flow.  Illustrated graphically in figure (\ref{fig:continuumL11:continuumL11fig6})

\begin{figure}[htp]
   \centering
   \includegraphics[totalheight=0.2\textheight]{continuumL11fig6}
   \caption{Parabolic velocity distribution.}\label{fig:continuumL11:continuumL11fig6}
\end{figure}

From the Navier-Stokes equations we have

\begin{equation}\label{eqn:continuumL11:710}
\evalbar{\frac{du}{dy}}{y = y_{\text{max}}} = 0.
\end{equation}

This implies

\begin{equation}\label{eqn:continuumL11:830}
\frac{G}{\mu} y = 0
\end{equation}

\begin{equation}\label{eqn:continuumL11:850}
u_max = \frac{G}{2\mu} h^2
\end{equation}

The flux, or flow rate is

\begin{align*}
Q 
&= \iint_S \Bu \cdot \xcap ds \\
&= \int_0^1 dz \int_{-h}^h dy u(y) \\
&=
\frac{2 G h^3}{3}
\end{align*}

Let's now compute the strain ($e_{ij}$) and the stress ($\sigma_{ij}$)

\begin{equation}\label{eqn:continuumL11:730}
e_{12} = e_{21} = \inv{2} \left( \PD{y}{u} \right) = - \frac{G y}{2 \mu}
\end{equation}

stress

\begin{equation}\label{eqn:continuumL11:750}
\sigma_{12} = 2 \mu e_{12} = -G y
\end{equation}

This can be used to compute the forces on the inner surfaces of the tube.  As illustrated in figure (\ref{fig:continuumL11:continuumL11fig7})
\begin{figure}[htp]
   \centering
   \includegraphics[totalheight=0.2\textheight]{continuumL11fig7}
   \caption{Normals in 1D channel flow system}\label{fig:continuumL11:continuumL11fig7}
\end{figure}

Our normals at $\pm h$ are $\mp \ycap$ respectively.  The traction vector in the $y$ direction is at $y = h$ is

FIXME: don't like this hodge-podge mix of vectors and coordinates.
\begin{equation}\label{eqn:continuumL11:770}
\BT = \sigma_{i 2} \evalbar{\ncap}{y = h} = G h \xcap
\end{equation}

(here the $x$ directionality comes from the $i$ index of the stress tensor)

\begin{equation}\label{eqn:continuumL11:790}
F_{x_L} = \xcap \cdot \BT = G h
\end{equation}

The total force is then

\begin{equation}\label{eqn:continuumL11:810}
\int_0^L F_x dx = + G h L
\end{equation}

%\EndArticle
\EndNoBibArticle
