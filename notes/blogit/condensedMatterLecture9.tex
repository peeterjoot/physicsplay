%
% Copyright � 2013 Peeter Joot.  All Rights Reserved.
% Licenced as described in the file LICENSE under the root directory of this GIT repository.
%
\newcommand{\authorname}{Peeter Joot}
\newcommand{\email}{peeterjoot@protonmail.com}
\newcommand{\basename}{FIXMEbasenameUndefined}
\newcommand{\dirname}{notes/FIXMEdirnameUndefined/}

\renewcommand{\basename}{condensedMatterLecture9}
\renewcommand{\dirname}{notes/phy487/}
\newcommand{\keywords}{Condensed matter physics, PHY487H1F}
\newcommand{\authorname}{Peeter Joot}
\newcommand{\onlineurl}{http://sites.google.com/site/peeterjoot2/math2013/\basename.pdf}
\newcommand{\sourcepath}{\dirname\basename.tex}
\newcommand{\generatetitle}[1]{\chapter{#1}}

\newcommand{\vcsinfo}{%
\section*{}
\noindent{\color{DarkOliveGreen}{\rule{\linewidth}{0.1mm}}}
\paragraph{Document version}
%\paragraph{\color{Maroon}{Document version}}
{
\small
\begin{itemize}
\item Available online at:\\ 
\href{\onlineurl}{\onlineurl}
\item Git Repository: \input{./.revinfo/gitRepo.tex}
\item Source: \sourcepath
\item last commit: \input{./.revinfo/gitCommitString.tex}
\item commit date: \input{./.revinfo/gitCommitDate.tex}
\end{itemize}
}
}

%\PassOptionsToPackage{dvipsnames,svgnames}{xcolor}
\PassOptionsToPackage{square,numbers}{natbib}
\documentclass{scrreprt}

\usepackage[left=2cm,right=2cm]{geometry}
\usepackage[svgnames]{xcolor}
\usepackage{peeters_layout}

\usepackage{natbib}

\usepackage[
colorlinks=true,
bookmarks=false,
pdfauthor={\authorname, \email},
backref 
]{hyperref}

% http://tex.stackexchange.com/questions/75773/how-to-reference-problems-by-the-text-label-in-an-exercise-envioronment
\usepackage[english]{cleveref}
\crefname{Exercise}{exercise}{exercises}
\Crefname{Exercise}{Exercise}{Exercises}

\RequirePackage{titlesec}
\RequirePackage{ifthen}

% http://stackoverflow.com/questions/4932910/date-in-the-tabular-environment
\makeatletter
\let\insertdate\@date
\makeatother

\titleformat{\chapter}[display]
{\bfseries\Large}
{\color{DarkSlateGrey}\filleft \authorname
\ifthenelse{\isundefined{\studentnumber}}{}{\\ \studentnumber}
\ifthenelse{\isundefined{\email}}{}{\\ \email}
\ifthenelse{\isundefined{\dateintitle}}{}{\\ \insertdate}
%\ifthenelse{\isundefined{\coursename}}{}{\\ \coursename} % put in title instead.
}
{4ex}
{\color{DarkOliveGreen}{\titlerule}\color{Maroon}
\vspace{2ex}%
\filright}
[\vspace{2ex}%
\color{DarkOliveGreen}\titlerule
]

\newcommand{\beginArtWithToc}[0]{\begin{document}\tableofcontents}
\newcommand{\beginArtNoToc}[0]{\begin{document}}
\newcommand{\EndNoBibArticle}[0]{\end{document}}
\newcommand{\EndArticle}[0]{\bibliography{Bibliography}\bibliographystyle{plainnat}\end{document}}

% 
%\newcommand{\citep}[1]{\cite{#1}}

\colorSectionsForArticle



%\citep{harald2003solid} \S x.y

%\usepackage{mhchem}
\usepackage[version=3]{mhchem}

\beginArtNoToc
\generatetitle{PHY487H1F Condensed Matter Physics.  Lecture 9: Thermal properties.  Taught by Prof.\ Stephen Julian}
%\chapter{Thermal properties}
\label{chap:condensedMatterLecture9}

%\section{Disclaimer}
%
%Peeter's lecture notes from class.  May not be entirely coherent.

\section{Thermal properties}

We'll want to calculate the total energy stored in the lattice.  There are a number of steps to this problem

\begin{enumerate}
\item Quantization.

As an example, the classic SHO problem

\begin{dmath}\label{eqn:condensedMatterLecture9:20}
m \ddot{u} = -k u,
\end{dmath}

results in a single frequency

\begin{dmath}\label{eqn:condensedMatterLecture9:40}
\omega = \sqrt{\frac{k}{m}}.
\end{dmath}

With quantization we find only discrete frequencies, as in \cref{fig:qmSolidsL9:qmSolidsL9Fig1}.

\imageFigure{qmSolidsL9Fig1}{Quantized SHO energy levels}{fig:qmSolidsL9:qmSolidsL9Fig1}{0.1}

\begin{dmath}\label{eqn:condensedMatterLecture9:60}
\epsilon_n = \lr{ n + \inv{2} } \Hbar \omega.
\end{dmath}

For lattices we've been seeking solutions of the force equations

\begin{dmath}\label{eqn:condensedMatterLecture9:80}
M_\alpha \ddot{u}_{n \alpha i} = - \sum_{m \beta j} 
\Phi_{n \alpha i}^{m \beta j} u_{m \beta j}.
\end{dmath}

This provided us lattice frequencies $\omega_q$, which quantize as 

\begin{dmath}\label{eqn:condensedMatterLecture9:100}
\epsilon_{q, n} = \lr{ n_q + \inv{2} } \Hbar \omega_q.
\end{dmath}

Prof hoping that we are willing to this without proof since the proof is hard, and would take a couple days, and is not sure if it's in the text.

We've apparently seen such a derivation indirectly in blackbody calculations.

\item

If we accept \eqnref{eqn:condensedMatterLecture9:100}, then the thermal energy is

\begin{dmath}\label{eqn:condensedMatterLecture9:120}
U(T) = \mathLabelBox
[
   labelstyle={below of=m\themathLableNode, below of=m\themathLableNode}
]
{\sum_q}{sum over modes} 
( 
\mathLabelBox
{
n_q 
}
{average thermal occupancy}
+ \inv{2} 
) \Hbar \omega_q
\end{dmath}

\item figuring out how to evaluate such a sum.

\end{enumerate}

\section{Density of states}

Reading: \citep{ibach2009solid} \S 5.1

The density of states is the number of Phonon modes per unit energy.

\begin{itemize}
\item 
$n_q$ depends only on energy (see below), so that we can group states by energy.
\item 
$q$ is quasi-continuous 
\end{itemize}

%F2
%\cref{fig:qmSolidsL9:qmSolidsL9Fig2}.
\imageFigure{qmSolidsL9Fig2}{Period boundary conditions}{fig:qmSolidsL9:qmSolidsL9Fig2}{0.15}

due to periodic boundary

\begin{dmath}\label{eqn:condensedMatterLecture9:140}
e^{i \Bq \cdot \lr { \Br_n + \BL_x } } = 
e^{i \Bq \cdot \Br_n },
\end{dmath}

or

\begin{dmath}\label{eqn:condensedMatterLecture9:160}
q_x L_x = 2 \pi l,
%\Bq \cdot \Br_n = 2 \pi l,
\end{dmath}

for integer $l$.

Similarily for $q_y$, $q_z$ we have 

\begin{dmath}\label{eqn:condensedMatterLecture9:180}
\Bq = l \frac{ 2 \pi }{L_x} \xcap
+ m \frac{ 2 \pi }{L_y} \ycap
+ o \frac{ 2 \pi }{L_z} \zcap
\end{dmath}

The volume per $\Bq$ point is

\begin{dmath}\label{eqn:condensedMatterLecture9:200}
\frac{2 \pi}{L_x}
\frac{2 \pi}{L_y}
\frac{2 \pi}{L_z}
= \frac{\lr{ 2 \pi }^3 }{V}
\end{dmath}

%F3
%\cref{fig:qmSolidsL9:qmSolidsL9Fig3}.
\imageFigure{qmSolidsL9Fig3}{constant energy surface}{fig:qmSolidsL9:qmSolidsL9Fig3}{0.15}

\begin{dmath}\label{eqn:condensedMatterLecture9:220}
\sum_\Bq \rightarrow \frac{V}{(2 \pi)^3} \int d\Bq
\end{dmath}

We group states of same $\Hbar \omega_q$.  In $d\Bq$ group same energy states

%F4
%\cref{fig:qmSolidsL9:qmSolidsL9Fig4}.
\imageFigure{qmSolidsL9Fig4}{area element}{fig:qmSolidsL9:qmSolidsL9Fig4}{0.15}

$d f_\omega$ are constant energy area elements, so that the volume element is

\begin{dmath}\label{eqn:condensedMatterLecture9:240}
df_\omega dq_\perp
\end{dmath}

We can write

\begin{dmath}\label{eqn:condensedMatterLecture9:260}
dq_\perp = \frac{d\omega}{\Abs{\spacegrad_\Bq \omega(\Bq) }},
\end{dmath}

or

\begin{dmath}\label{eqn:condensedMatterLecture9:280}
\frac{V}{(2\pi)^3 } \int d\Bq 
\rightarrow
\frac{V}{(2\pi)^3 } \int d f_\omega d q_\perp
\rightarrow
\frac{V}{(2\pi)^3 } \int d f_\omega d q_\perp
\frac{df_\omega}{\Abs{\spacegrad_\Bq \omega(\Bq) }} d \omega
=
\int Z(\omega) d\omega
\end{dmath}

where $Z(\omega)$ is the \indexAndText{density of states}, the number of modes per unit energy.

\paragraph{EXAM EXPECTATION:} He'll expect us to look at a phonon diagram and see where the density of states is high or low.

\makeexample{1D diatomic chain}{example:condensedMatterLecture9:1}{

%F5
%\cref{fig:qmSolidsL9:qmSolidsL9Fig5}.
\imageFigure{qmSolidsL9Fig5}{frequency distribution and density of states for diatomic chain}{fig:qmSolidsL9:qmSolidsL9Fig5}{0.15}

%F6
%\cref{fig:qmSolidsL9:qmSolidsL9Fig6}.
\imageFigure{qmSolidsL9Fig6}{density of states}{fig:qmSolidsL9:qmSolidsL9Fig6}{0.15}

}

\makeexample{3D solid}{example:condensedMatterLecture9:2}{

%F7
%\cref{fig:qmSolidsL9:qmSolidsL9Fig7}.
\imageFigure{qmSolidsL9Fig7}{3D solid frequency distribution and density of states}{fig:qmSolidsL9:qmSolidsL9Fig7}{0.15}

%F8
%\cref{fig:qmSolidsL9:qmSolidsL9Fig8}.
\imageFigure{qmSolidsL9Fig8}{3D solid density of states}{fig:qmSolidsL9:qmSolidsL9Fig8}{0.15}

}

\section{Isotropic model (Debye)}

\index{isotropic model}
\index{Deybe}

With a 1 atom basis (for now), we have only acoustic modes

\begin{dmath}\label{eqn:condensedMatterLecture9:300}
\omega(\Bq) \rightarrow \omega(q)
\end{dmath}

same in all directions, so that the constant energy surfaces are spheres.

%F9
%\cref{fig:qmSolidsL9:qmSolidsL9Fig9}.
\imageFigure{qmSolidsL9Fig9}{Deybe surface}{fig:qmSolidsL9:qmSolidsL9Fig9}{0.15}

\begin{dmath}\label{eqn:condensedMatterLecture9:320}
\spacegrad_\Bq \omega(q) = \frac{d\omega}{dq} \qcap,
\end{dmath}

at small $q$, 

\begin{dmath}\label{eqn:condensedMatterLecture9:340}
\omega = 
\left\{
\begin{array}{l l}
C_L q & \quad \mbox{longitudinal accoustic} \\
C_T q & \quad \mbox{transverse accoustic} 
\end{array}
\right.
\end{dmath}

This gives

\begin{dmath}\label{eqn:condensedMatterLecture9:360}
Z(\omega) d\omega
=
\sum_{LA, TA} \frac{V}{(2\pi)^3} \int \frac{d f_\omega }{\Abs{ \spacegrad_\Bq \omega(\Bq)}}
=
\frac{V}{(2\pi)^3} 
\lr{
\inv{C_L} + \frac{2}{C_T}
}
4 \pi q^2 d\omega
=
\frac{V}{2\pi^2} 
\lr{
\frac{q^2}{C_L} + \frac{2 q^2}{C_T}
}
d \omega
=
\frac{V}{2\pi^2} 
\lr{
\frac{1}{C_L^3} + \frac{2}{C_T^3}
}
\omega^2 d \omega,
\end{dmath}

FIXME: where'd these cubic terms come from?

so that,
because of $4 \pi q^2$ phase space factor, we have

\begin{dmath}\label{eqn:condensedMatterLecture9:n}
Z(\omega) \propto \omega^2.
\end{dmath}

\paragraph{Deybe approximation}

Define 

\begin{dmath}\label{eqn:condensedMatterLecture9:380}
\omega_D = \int Z(\omega) d\omega = 3 r N
\end{dmath}

%F10
%\cref{fig:qmSolidsL9:qmSolidsL9Fig10}.
\imageFigure{qmSolidsL9Fig10}{Deybe approximation}{fig:qmSolidsL9:qmSolidsL9Fig10}{0.15}

%F11
%\cref{fig:qmSolidsL9:qmSolidsL9Fig11}.
\imageFigure{qmSolidsL9Fig11}{Linear Deybe approximation}{fig:qmSolidsL9:qmSolidsL9Fig11}{0.15}

and pretend we can cut off in a way that applies to all $q$.  We get 

\begin{dmath}\label{eqn:condensedMatterLecture9:400}
\omega_D = \frac{V}{ 2 \pi^2} \lr{ \inv{C_L^2} + \frac{ 2}{C_T^2} } \int_0^{\omega_D} \omega^2 d\omega 
= 9 r N
\end{dmath}

\EndArticle
