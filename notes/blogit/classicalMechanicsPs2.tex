%
% Copyright � 2012 Peeter Joot.  All Rights Reserved.
% Licenced as described in the file LICENSE under the root directory of this GIT repository.
%

% 
% 
%
% Copyright � 2015 Peeter Joot.  All Rights Reserved.
% Licenced as described in the file LICENSE under the root directory of this GIT repository.
%
\documentclass[]{eliblog}

\usepackage{amsmath}
\usepackage{mathpazo}

%
% shorthand for bold symbols, convenient for vectors and matrices
%
\newcommand{\Ba}[0]{\mathbf{a}}
\newcommand{\Bb}[0]{\mathbf{b}}
\newcommand{\Bc}[0]{\mathbf{c}}
\newcommand{\Bd}[0]{\mathbf{d}}
\newcommand{\Be}[0]{\mathbf{e}}
\newcommand{\Bf}[0]{\mathbf{f}}
\newcommand{\Bg}[0]{\mathbf{g}}
\newcommand{\Bh}[0]{\mathbf{h}}
\newcommand{\Bi}[0]{\mathbf{i}}
\newcommand{\Bj}[0]{\mathbf{j}}
\newcommand{\Bk}[0]{\mathbf{k}}
\newcommand{\Bl}[0]{\mathbf{l}}
\newcommand{\Bm}[0]{\mathbf{m}}
\newcommand{\Bn}[0]{\mathbf{n}}
\newcommand{\Bo}[0]{\mathbf{o}}
\newcommand{\Bp}[0]{\mathbf{p}}
\newcommand{\Bq}[0]{\mathbf{q}}
\newcommand{\Br}[0]{\mathbf{r}}
\newcommand{\Bs}[0]{\mathbf{s}}
\newcommand{\Bt}[0]{\mathbf{t}}
\newcommand{\Bu}[0]{\mathbf{u}}
\newcommand{\Bv}[0]{\mathbf{v}}
\newcommand{\Bw}[0]{\mathbf{w}}
\newcommand{\Bx}[0]{\mathbf{x}}
\newcommand{\By}[0]{\mathbf{y}}
\newcommand{\Bz}[0]{\mathbf{z}}
\newcommand{\BA}[0]{\mathbf{A}}
\newcommand{\BB}[0]{\mathbf{B}}
\newcommand{\BC}[0]{\mathbf{C}}
\newcommand{\BD}[0]{\mathbf{D}}
\newcommand{\BE}[0]{\mathbf{E}}
\newcommand{\BF}[0]{\mathbf{F}}
\newcommand{\BG}[0]{\mathbf{G}}
\newcommand{\BH}[0]{\mathbf{H}}
\newcommand{\BI}[0]{\mathbf{I}}
\newcommand{\BJ}[0]{\mathbf{J}}
\newcommand{\BK}[0]{\mathbf{K}}
\newcommand{\BL}[0]{\mathbf{L}}
\newcommand{\BM}[0]{\mathbf{M}}
\newcommand{\BN}[0]{\mathbf{N}}
\newcommand{\BO}[0]{\mathbf{O}}
\newcommand{\BP}[0]{\mathbf{P}}
\newcommand{\BQ}[0]{\mathbf{Q}}
\newcommand{\BR}[0]{\mathbf{R}}
\newcommand{\BS}[0]{\mathbf{S}}
\newcommand{\BT}[0]{\mathbf{T}}
\newcommand{\BU}[0]{\mathbf{U}}
\newcommand{\BV}[0]{\mathbf{V}}
\newcommand{\BW}[0]{\mathbf{W}}
\newcommand{\BX}[0]{\mathbf{X}}
\newcommand{\BY}[0]{\mathbf{Y}}
\newcommand{\BZ}[0]{\mathbf{Z}}

\newcommand{\Bzero}[0]{\mathbf{0}}
\newcommand{\Btheta}[0]{\boldsymbol{\theta}}
\newcommand{\Btau}[0]{\boldsymbol{\tau}}
\newcommand{\Bomega}[0]{\boldsymbol{\omega}}

%
% shorthand for unit vectors
%
\newcommand{\acap}[0]{\hat{\Ba}}
\newcommand{\bcap}[0]{\hat{\Bb}}
\newcommand{\ccap}[0]{\hat{\Bc}}
\newcommand{\dcap}[0]{\hat{\Bd}}
\newcommand{\ecap}[0]{\hat{\Be}}
\newcommand{\fcap}[0]{\hat{\Bf}}
\newcommand{\gcap}[0]{\hat{\Bg}}
\newcommand{\hcap}[0]{\hat{\Bh}}
\newcommand{\icap}[0]{\hat{\Bi}}
\newcommand{\jcap}[0]{\hat{\Bj}}
\newcommand{\kcap}[0]{\hat{\Bk}}
\newcommand{\lcap}[0]{\hat{\Bl}}
\newcommand{\mcap}[0]{\hat{\Bm}}
\newcommand{\ncap}[0]{\hat{\Bn}}
\newcommand{\ocap}[0]{\hat{\Bo}}
\newcommand{\pcap}[0]{\hat{\Bp}}
\newcommand{\qcap}[0]{\hat{\Bq}}
\newcommand{\rcap}[0]{\hat{\Br}}
\newcommand{\scap}[0]{\hat{\Bs}}
\newcommand{\tcap}[0]{\hat{\Bt}}
\newcommand{\ucap}[0]{\hat{\Bu}}
\newcommand{\vcap}[0]{\hat{\Bv}}
\newcommand{\wcap}[0]{\hat{\Bw}}
\newcommand{\xcap}[0]{\hat{\Bx}}
\newcommand{\ycap}[0]{\hat{\By}}
\newcommand{\zcap}[0]{\hat{\Bz}}
\newcommand{\thetacap}[0]{\hat{\Btheta}}

%
% to write R^n and C^n in a distinguishable fashion.  Perhaps change this
% to the double lined characters upon figuring out how to do so.
%
\newcommand{\C}[1]{$\mathbb{C}^{#1}$}
\newcommand{\R}[1]{$\mathbb{R}^{#1}$}

%
% various generally useful helpers
%

% derivative of #1 wrt. #2:
\newcommand{\D}[2] {\frac {d#2} {d#1}}

\newcommand{\inv}[1]{\frac{1}{#1}}
\newcommand{\cross}[0]{\times}

\newcommand{\abs}[1]{\lvert{#1}\rvert}
\newcommand{\norm}[1]{\lVert{#1}\rVert}
\newcommand{\innerprod}[2]{\langle{#1}, {#2}\rangle}
\newcommand{\dotprod}[2]{{#1} \cdot {#2}}
\newcommand{\bdotprod}[2]{\left({#1} \cdot {#2}\right)}
\newcommand{\crossprod}[2]{{#1} \cross {#2}}
\newcommand{\tripleprod}[3]{\dotprod{\left(\crossprod{#1}{#2}\right)}{#3}}

\DeclareMathOperator{\Proj}{Proj}
\DeclareMathOperator{\Span}{span}
\DeclareMathOperator{\Sgn}{sgn}
\DeclareMathOperator{\Area}{Area}
\DeclareMathOperator{\Volume}{Volume}

%
% A few miscellaneous things specific to this document
%
\newcommand{\crossop}[1]{\crossprod{#1}{}}

% R2 vector.
\newcommand{\VectorTwo}[2]{
\begin{bmatrix}
 {#1} \\
 {#2}
\end{bmatrix}
}

\newcommand{\VectorN}[1]{
\begin{bmatrix}
{#1}_1 \\
{#1}_2 \\
\vdots \\
{#1}_N \\
\end{bmatrix}
}

\newcommand{\DETuvij}[4]{
\begin{vmatrix}
 {#1}_{#3} & {#1}_{#4} \\
 {#2}_{#3} & {#2}_{#4}
\end{vmatrix}
}

\newcommand{\DETuvwijk}[6]{
\begin{vmatrix}
 {#1}_{#4} & {#1}_{#5} & {#1}_{#6} \\
 {#2}_{#4} & {#2}_{#5} & {#2}_{#6} \\
 {#3}_{#4} & {#3}_{#5} & {#3}_{#6}
\end{vmatrix}
}

\newcommand{\DETuvwxijkl}[8]{
\begin{vmatrix}
 {#1}_{#5} & {#1}_{#6} & {#1}_{#7} & {#1}_{#8} \\
 {#2}_{#5} & {#2}_{#6} & {#2}_{#7} & {#2}_{#8} \\
 {#3}_{#5} & {#3}_{#6} & {#3}_{#7} & {#3}_{#8} \\
 {#4}_{#5} & {#4}_{#6} & {#4}_{#7} & {#4}_{#8} \\
\end{vmatrix}
}

%\newcommand{\DETuvwxyijklm}[10]{
%\begin{vmatrix}
% {#1}_{#6} & {#1}_{#7} & {#1}_{#8} & {#1}_{#9} & {#1}_{#10} \\
% {#2}_{#6} & {#2}_{#7} & {#2}_{#8} & {#2}_{#9} & {#2}_{#10} \\
% {#3}_{#6} & {#3}_{#7} & {#3}_{#8} & {#3}_{#9} & {#3}_{#10} \\
% {#4}_{#6} & {#4}_{#7} & {#4}_{#8} & {#4}_{#9} & {#4}_{#10} \\
% {#5}_{#6} & {#5}_{#7} & {#5}_{#8} & {#5}_{#9} & {#5}_{#10}
%\end{vmatrix}
%}

% R3 vector.
\newcommand{\VectorThree}[3]{
\begin{bmatrix}
 {#1} \\
 {#2} \\
 {#3}
\end{bmatrix}
}



\author{Peeter Joot}
\email{peeter.joot@gmail.com}

%\documentclass[]{eliblogwidescreen}

\usepackage{amsmath}
\usepackage{mathpazo}

%
% shorthand for bold symbols, convenient for vectors and matrices
%
\newcommand{\Ba}[0]{\mathbf{a}}
\newcommand{\Bb}[0]{\mathbf{b}}
\newcommand{\Bc}[0]{\mathbf{c}}
\newcommand{\Bd}[0]{\mathbf{d}}
\newcommand{\Be}[0]{\mathbf{e}}
\newcommand{\Bf}[0]{\mathbf{f}}
\newcommand{\Bg}[0]{\mathbf{g}}
\newcommand{\Bh}[0]{\mathbf{h}}
\newcommand{\Bi}[0]{\mathbf{i}}
\newcommand{\Bj}[0]{\mathbf{j}}
\newcommand{\Bk}[0]{\mathbf{k}}
\newcommand{\Bl}[0]{\mathbf{l}}
\newcommand{\Bm}[0]{\mathbf{m}}
\newcommand{\Bn}[0]{\mathbf{n}}
\newcommand{\Bo}[0]{\mathbf{o}}
\newcommand{\Bp}[0]{\mathbf{p}}
\newcommand{\Bq}[0]{\mathbf{q}}
\newcommand{\Br}[0]{\mathbf{r}}
\newcommand{\Bs}[0]{\mathbf{s}}
\newcommand{\Bt}[0]{\mathbf{t}}
\newcommand{\Bu}[0]{\mathbf{u}}
\newcommand{\Bv}[0]{\mathbf{v}}
\newcommand{\Bw}[0]{\mathbf{w}}
\newcommand{\Bx}[0]{\mathbf{x}}
\newcommand{\By}[0]{\mathbf{y}}
\newcommand{\Bz}[0]{\mathbf{z}}
\newcommand{\BA}[0]{\mathbf{A}}
\newcommand{\BB}[0]{\mathbf{B}}
\newcommand{\BC}[0]{\mathbf{C}}
\newcommand{\BD}[0]{\mathbf{D}}
\newcommand{\BE}[0]{\mathbf{E}}
\newcommand{\BF}[0]{\mathbf{F}}
\newcommand{\BG}[0]{\mathbf{G}}
\newcommand{\BH}[0]{\mathbf{H}}
\newcommand{\BI}[0]{\mathbf{I}}
\newcommand{\BJ}[0]{\mathbf{J}}
\newcommand{\BK}[0]{\mathbf{K}}
\newcommand{\BL}[0]{\mathbf{L}}
\newcommand{\BM}[0]{\mathbf{M}}
\newcommand{\BN}[0]{\mathbf{N}}
\newcommand{\BO}[0]{\mathbf{O}}
\newcommand{\BP}[0]{\mathbf{P}}
\newcommand{\BQ}[0]{\mathbf{Q}}
\newcommand{\BR}[0]{\mathbf{R}}
\newcommand{\BS}[0]{\mathbf{S}}
\newcommand{\BT}[0]{\mathbf{T}}
\newcommand{\BU}[0]{\mathbf{U}}
\newcommand{\BV}[0]{\mathbf{V}}
\newcommand{\BW}[0]{\mathbf{W}}
\newcommand{\BX}[0]{\mathbf{X}}
\newcommand{\BY}[0]{\mathbf{Y}}
\newcommand{\BZ}[0]{\mathbf{Z}}

\newcommand{\Bzero}[0]{\mathbf{0}}
\newcommand{\Btheta}[0]{\boldsymbol{\theta}}
\newcommand{\Btau}[0]{\boldsymbol{\tau}}
\newcommand{\Bomega}[0]{\boldsymbol{\omega}}

%
% shorthand for unit vectors
%
\newcommand{\acap}[0]{\hat{\Ba}}
\newcommand{\bcap}[0]{\hat{\Bb}}
\newcommand{\ccap}[0]{\hat{\Bc}}
\newcommand{\dcap}[0]{\hat{\Bd}}
\newcommand{\ecap}[0]{\hat{\Be}}
\newcommand{\fcap}[0]{\hat{\Bf}}
\newcommand{\gcap}[0]{\hat{\Bg}}
\newcommand{\hcap}[0]{\hat{\Bh}}
\newcommand{\icap}[0]{\hat{\Bi}}
\newcommand{\jcap}[0]{\hat{\Bj}}
\newcommand{\kcap}[0]{\hat{\Bk}}
\newcommand{\lcap}[0]{\hat{\Bl}}
\newcommand{\mcap}[0]{\hat{\Bm}}
\newcommand{\ncap}[0]{\hat{\Bn}}
\newcommand{\ocap}[0]{\hat{\Bo}}
\newcommand{\pcap}[0]{\hat{\Bp}}
\newcommand{\qcap}[0]{\hat{\Bq}}
\newcommand{\rcap}[0]{\hat{\Br}}
\newcommand{\scap}[0]{\hat{\Bs}}
\newcommand{\tcap}[0]{\hat{\Bt}}
\newcommand{\ucap}[0]{\hat{\Bu}}
\newcommand{\vcap}[0]{\hat{\Bv}}
\newcommand{\wcap}[0]{\hat{\Bw}}
\newcommand{\xcap}[0]{\hat{\Bx}}
\newcommand{\ycap}[0]{\hat{\By}}
\newcommand{\zcap}[0]{\hat{\Bz}}
\newcommand{\thetacap}[0]{\hat{\Btheta}}

%
% to write R^n and C^n in a distinguishable fashion.  Perhaps change this
% to the double lined characters upon figuring out how to do so.
%
\newcommand{\C}[1]{$\mathbb{C}^{#1}$}
\newcommand{\R}[1]{$\mathbb{R}^{#1}$}

%
% various generally useful helpers
%

% derivative of #1 wrt. #2:
\newcommand{\D}[2] {\frac {d#2} {d#1}}

\newcommand{\inv}[1]{\frac{1}{#1}}
\newcommand{\cross}[0]{\times}

\newcommand{\abs}[1]{\lvert{#1}\rvert}
\newcommand{\norm}[1]{\lVert{#1}\rVert}
\newcommand{\innerprod}[2]{\langle{#1}, {#2}\rangle}
\newcommand{\dotprod}[2]{{#1} \cdot {#2}}
\newcommand{\bdotprod}[2]{\left({#1} \cdot {#2}\right)}
\newcommand{\crossprod}[2]{{#1} \cross {#2}}
\newcommand{\tripleprod}[3]{\dotprod{\left(\crossprod{#1}{#2}\right)}{#3}}

\DeclareMathOperator{\Proj}{Proj}
\DeclareMathOperator{\Span}{span}
\DeclareMathOperator{\Sgn}{sgn}
\DeclareMathOperator{\Area}{Area}
\DeclareMathOperator{\Volume}{Volume}

%
% A few miscellaneous things specific to this document
%
\newcommand{\crossop}[1]{\crossprod{#1}{}}

% R2 vector.
\newcommand{\VectorTwo}[2]{
\begin{bmatrix}
 {#1} \\
 {#2}
\end{bmatrix}
}

\newcommand{\VectorN}[1]{
\begin{bmatrix}
{#1}_1 \\
{#1}_2 \\
\vdots \\
{#1}_N \\
\end{bmatrix}
}

\newcommand{\DETuvij}[4]{
\begin{vmatrix}
 {#1}_{#3} & {#1}_{#4} \\
 {#2}_{#3} & {#2}_{#4}
\end{vmatrix}
}

\newcommand{\DETuvwijk}[6]{
\begin{vmatrix}
 {#1}_{#4} & {#1}_{#5} & {#1}_{#6} \\
 {#2}_{#4} & {#2}_{#5} & {#2}_{#6} \\
 {#3}_{#4} & {#3}_{#5} & {#3}_{#6}
\end{vmatrix}
}

\newcommand{\DETuvwxijkl}[8]{
\begin{vmatrix}
 {#1}_{#5} & {#1}_{#6} & {#1}_{#7} & {#1}_{#8} \\
 {#2}_{#5} & {#2}_{#6} & {#2}_{#7} & {#2}_{#8} \\
 {#3}_{#5} & {#3}_{#6} & {#3}_{#7} & {#3}_{#8} \\
 {#4}_{#5} & {#4}_{#6} & {#4}_{#7} & {#4}_{#8} \\
\end{vmatrix}
}

%\newcommand{\DETuvwxyijklm}[10]{
%\begin{vmatrix}
% {#1}_{#6} & {#1}_{#7} & {#1}_{#8} & {#1}_{#9} & {#1}_{#10} \\
% {#2}_{#6} & {#2}_{#7} & {#2}_{#8} & {#2}_{#9} & {#2}_{#10} \\
% {#3}_{#6} & {#3}_{#7} & {#3}_{#8} & {#3}_{#9} & {#3}_{#10} \\
% {#4}_{#6} & {#4}_{#7} & {#4}_{#8} & {#4}_{#9} & {#4}_{#10} \\
% {#5}_{#6} & {#5}_{#7} & {#5}_{#8} & {#5}_{#9} & {#5}_{#10}
%\end{vmatrix}
%}

% R3 vector.
\newcommand{\VectorThree}[3]{
\begin{bmatrix}
 {#1} \\
 {#2} \\
 {#3}
\end{bmatrix}
}



\author{Peeter Joot}
\email{peeter.joot@gmail.com}


\chapter{PHY354 Advanced Classical Mechanics.  Problem set 2}
\label{chap:classicalMechanicsPs2}
%\useCCL
\blogpage{http://sites.google.com/site/peeterjoot2/math2012/classicalMechanicsPs2.pdf}
\date{Feb 13, 2012}
\gitRevisionInfo{classicalMechanicsPs2}

\beginArtWithToc
%\beginArtNoToc

\section{Disclaimer}

Ungraded solutions to posted problem set 2 (I am auditing half the lectures for this course and will not be submitting any solutions for grading).

\section{Problem 1. Symmetries and conservation laws in external e.m. fields}
\subsection{Statement}

Let us continue studying the Lagrangian of Problem 1 of Homework 1, namely, its symmetries, and the relevant conserved quantities. To this end, we will consider various cases of external scalar and vector potentials.

\begin{enumerate}
\item Consider first the case of time-independent $\BA$ and $\phi$. Find the expression for the conserved energy, \mathcal{E}, of the particle.
\item For external $\BA$ and $\phi$ dependent on time, find $\frac{dE}{dt}$.
\item Let now $\BA$ and $\phi$ be spatially homogeneous, i.e. $\Br$-independent. Find the conserved momentum. Is it equal to the usual $m \Bv$?
\item Consider motion in the field of an electrostatic source (creating an external static $\phi(\Br)$). Find the angular momentum of the particle. Is it conserved for all $\phi(\Br)$?
\end{enumerate}

\subsection{Solution}
\section{Problem 2. Integrals of motion in a helix potential}
\subsection{Statement}
Consider a particle moving in the external potential field (it could be gravitational or Coulomb, it is inessential
for this problem) of an infinite homogeneous cylindrical helix. Find the conserved (linear combination of)
components of $\BP$ and $\BM$.

\subsection{Solution}
\section{Problem 3.  ``Hidden'' symmetries and integrals of motion}
\subsection{Statement}
Consider the equation of motion of a particle of charge q moving in the field of a magnetic monopole:

\begin{equation}\label{eqn:classicalMechanicsPs2:n}
\BB = g \frac{\Br}{r^3}, \BE = 0.
\end{equation}

Here $g$ is the magnetic charge, which can be found using the magnetic analogue of Gausses law: $4 \pi Q_{\text{magn}} = \int_{S^2} d\BSigma \cdot \BB = 4 \pi g$.  In this problem, do not start with a Lagrangian^\footnote{
It is a fun and challenging problem to find~A for the magnetic monopole, but it does not belong in this class.
},but simply use the Lorentz force
equation. Since we no longer deal with central forces, the angular momentum is no longer conserved, and
the motion is no longer necessarily planar. However, it can be thought that a certain amount of angular
momentum resides in the magnetic field, and, as first observed by Poincar\'e the total angular momentumm

\begin{equation}\label{eqn:classicalMechanicsPs2:n}
\BD = \BM + c\frac{\Br}{r}
\end{equation}

is conserved (here $\BM$ is the mechanical angular momentum $\Br \cross \Bv$, and $c$ is a constant).

\begin{enumerate}
\item Find the value of $c$ ensuring that $d \BD/dt = 0$ for solutions of the equations of motion.
\item Show that the radius vector of the trajectory for a particle moving in the field of a monopole obeys $\dot{\rcap} \cdot \BD = c$, where $\dot{\rcap} = \Br/r$.
\item For a given value of $\BD$, determined by the initial conditions, the relation found above restricts the
motion of the particle on a particular surface in space. Draw this surface. What happens to this
surface when $g \rightarrow 0$? Is it surprising?
\item Is energy conserved? Find the energy of the particle $\mathcal{E}$ as a function of $\Bv$, $\Br$, \cdots as appropriate to the
case in hand.
\end{enumerate}

Besides being a fun problem, the moral here is that finding symmetries and the related integrals of motion is not
always so obvious. Usually, when integrals of motion exist, they are due to a symmetry. In this example, there is a
``hidden'' symmetry responsible for the conservation of $\BD$.

\subsection{Solution}
\section{Problem 4.  Symmetries and conservation laws}
\subsection{Statement}

What components of the momentum and angular momentum are conserved when a particle moves in the
(say, gravitational) field of the following objects:

\begin{enumerate}
\item An infinite homogeneous plane
\item An infinite homogeneous cylinder
\item Two point particles
\item An infinite homogeneous prism
\item A homogeneous cone
\item Three point particles
\item A homogeneous torus
\end{enumerate}

This is a problem from last year's midterm. Problems 2. and 4. are rather similar in nature. Be sure to ``internalize'' them -- they will come handy during exams but also in your life as physicists.

\subsection{Solution}
\section{Problem 5}
\subsection{Statement}

\begin{enumerate}
\item Find $M_x, M_y, M_z, \BM^2$ in spherical coordinates $(r, \theta, \phi)$.
\item Find $M_x, M_y, M_z, \BM^2$ in cylindrical coordinates $(r, \phi, z)$.
\end{enumerate}

%Everybody should do this once in their lives (and should be able to do at any point thereafter).
%My SQL-guru, but non-physisist girlfriend, was amused by the additional comment in the statement of the problem ``\underline{Everybody} should do this once in their lives (and should be able to do at any point thereafter).

\subsection{Solution. Spherical coordinates}

In cartesian coordinates our angular momentum is

\begin{align*}
\BM
&= \Br \cross (m \Bv) \\
&=
m (y v_z - z v_y) \xcap
+m (z v_x - x v_z) \ycap
+m (x v_y - y v_x) \zcap
\end{align*}

Substituting $x,y,z$ is easy since we have

\begin{equation}\label{eqn:classicalMechanicsPs2:10}
\begin{bmatrix}
x \\
y \\
z
\end{bmatrix}
=
r
\begin{bmatrix}
\sin\theta \cos\phi \\
\sin\theta \sin\phi \\
\cos\theta
\end{bmatrix},
\end{equation}

but the $\Bv$ substitution requires more work.  We have

\begin{align*}
\Bv
&= \frac{d \Br}{dt} \\
&= \frac{d}{dt} (r \rcap) \\
&= \rdot \rcap + r \ddt{\rcap}
\end{align*}

\begin{align*}
\ddt{\rcap}
&=
\ddt{}
\begin{bmatrix}
\sin\theta \cos\phi \\
\sin\theta \sin\phi \\
\cos\theta
\end{bmatrix} \\
&=
\begin{bmatrix}
\cos\theta \cos\phi \thetadot - \sin\theta \sin\phi \phidot \\
\cos\theta \sin\phi \thetadot + \sin\theta \cos\phi \phidot \\
-\sin\theta \thetadot
\end{bmatrix}
\end{align*}

So we have

\begin{equation}\label{eqn:classicalMechanicsPs2:30}
\Bv =
\begin{bmatrix}
\rdot \sin\theta \cos\phi + r \cos\theta \cos\phi \thetadot - r \sin\theta \sin\phi \phidot \\
\rdot \sin\theta \sin\phi + r \cos\theta \sin\phi \thetadot + r \sin\theta \cos\phi \phidot \\
\rdot \cos\theta -r \sin\theta \thetadot
\end{bmatrix}
\end{equation}

\begin{align*}
\BM &=
m r
\begin{bmatrix}
\sin\theta \sin\phi v_z - \cos\theta v_y \\
\cos\theta v_x - \sin\theta \cos\phi v_z \\
\sin\theta \cos\phi v_y - \sin\theta \sin\phi v_x
\end{bmatrix} \\
&=
m r
\begin{bmatrix}
S_\theta S_\phi (\rdot C_\theta -r S_\theta \thetadot) - C_\theta (\rdot S_\theta S_\phi + r C_\theta S_\phi \thetadot + r S_\theta C_\phi \phidot) \\
C_\theta (\rdot S_\theta C_\phi + r C_\theta C_\phi \thetadot - r S_\theta S_\phi \phidot) - S_\theta C_\phi (\rdot C_\theta -r S_\theta \thetadot) \\
S_\theta C_\phi (\rdot S_\theta S_\phi + r C_\theta S_\phi \thetadot + r S_\theta C_\phi \phidot) - S_\theta S_\phi (\rdot S_\theta C_\phi + r C_\theta C_\phi \thetadot - r S_\theta S_\phi \phidot)
\end{bmatrix} \\
&=
m r
\begin{bmatrix}
\cancel{\rdot C_\theta S_\theta S_\phi }
- r \thetadot S_\theta^2 S_\phi
- \cancel{\rdot S_\theta C_\theta S_\phi }
- r \thetadot C_\theta^2 S_\phi
- r \phidot S_\theta C_\theta C_\phi
\\
\cancel{\rdot S_\theta C_\theta C_\phi }
+ r \thetadot C_\theta^2 C_\phi
- r \phidot S_\theta C_\theta S_\phi
-\cancel{\rdot C_\theta S_\theta C_\phi }
+r \thetadot S_\theta^2 C_\phi
\\
\cancel{\rdot S_\theta^2 S_\phi C_\phi }
+ \cancel{r \thetadot C_\theta S_\theta C_\phi S_\phi }
+ r \phidot S_\theta^2 C_\phi^2
- \cancel{\rdot S_\theta^2 S_\phi C_\phi }
- \cancel{r \thetadot C_\theta S_\theta S_\phi C_\phi }
+ r \phidot S_\theta^2 S_\phi^2
\end{bmatrix} \\
&=
m r
\begin{bmatrix}
- r \thetadot S_\phi
- r \phidot S_\theta C_\theta C_\phi
\\
+ r \thetadot C_\phi
- r \phidot S_\theta C_\theta S_\phi
\\
+ r \phidot S_\theta^2
\end{bmatrix} \\
\end{align*}

In matrix form, we have (and can read off $M_x, M_y, M_z$)

\begin{equation}\label{eqn:classicalMechanicsPs2:50}
\boxed{
\BM =
\inv{2} m r^2
\begin{bmatrix}
-  2 \sin\phi & - \sin(2\theta) \cos\phi \\
  2 \cos\phi & - \sin(2\theta) \sin\phi \\
0 & 1 - \cos(2\theta)
\end{bmatrix}
\begin{bmatrix}
\thetadot \\
\phidot
\end{bmatrix}.
}
\end{equation}

We have also been asked to find $\BM^2$ and can write this as a quadratic form

\begin{align*}
\BM^2
&=
\inv{4} m^2 r^4
\begin{bmatrix}
\thetadot & \phidot
\end{bmatrix}
\begin{bmatrix}
-  2 \sin\phi  & 2 \cos\phi  & 0 \\
- \sin(2\theta) \cos\phi & - \sin(2\theta) \sin\phi  & 1 - \cos(2\theta)
\end{bmatrix}
\begin{bmatrix}
-  2 \sin\phi & - \sin(2\theta) \cos\phi \\
  2 \cos\phi & - \sin(2\theta) \sin\phi \\
0 & 1 - \cos(2\theta)
\end{bmatrix}
\begin{bmatrix}
\thetadot \\
\phidot
\end{bmatrix} \\
&=
\inv{4} m^2 r^4
\begin{bmatrix}
\thetadot & \phidot
\end{bmatrix}
\begin{bmatrix}
4 & 0 \\
0 & 2(1 - \cos(2\theta))
\end{bmatrix}
\begin{bmatrix}
\thetadot \\
\phidot
\end{bmatrix}  \\
\end{align*}

This simplifies suprisingly, leaving only

\begin{equation}\label{eqn:classicalMechanicsPs2:70}
\boxed{
\BM^2
=
m^2 r^4 ( \thetadot^2 + \sin^2\theta \phidot^2 ).
}
\end{equation}

\subsubsection{A smarter way}

Observe that we have no $\rdot$ factors in the angular momentum.  This makes sense when we consider that the total angular momentum is

\begin{equation}\label{eqn:classicalMechanicsPs2:190}
\BM = m r \rcap \cross \Bv,
\end{equation}

so the $\rdot \rcap$ term of the velocity is neccessarily killed.  Let us do that simplification first.  We want our velocity completely specified in a $\{\rcap, \thetacap, \phicap\}$ basis, and note that our basis vectors are

\begin{align}\label{eqn:classicalMechanicsPs2:210}
\rcap &=
\begin{bmatrix}
\sin\theta \cos\phi \\
\sin\theta \sin\phi \\
\cos\theta
\end{bmatrix} \\
\thetacap &=
\begin{bmatrix}
\cos\theta \cos\phi \\
\cos\theta \sin\phi \\
-\sin\theta
\end{bmatrix} \\
\phicap &=
\begin{bmatrix}
-\sin\phi \\
\cos\phi \\
0
\end{bmatrix} .
\end{align}

We wish to rewrite

\begin{equation}\label{eqn:classicalMechanicsPs2:230}
\ddt{\rcap} =
\begin{bmatrix}
\cos\theta \cos\phi & - \sin\theta \sin\phi \\
\cos\theta \sin\phi & \sin\theta \cos\phi \\
-\sin\theta & 0
\end{bmatrix}
\begin{bmatrix}
\thetadot \\
\phidot
\end{bmatrix},
\end{equation}

in terms of these spherical unit vectors and find

\begin{align}\label{eqn:classicalMechanicsPs2:250}
\frac{d\rcap}{dt} \cdot \rcap &= \rcap^\T \ddt{\rcap} = 0  \\
\frac{d\rcap}{dt} \cdot \thetacap &= \thetacap^\T \ddt{\rcap} = \thetadot \\
\frac{d\rcap}{dt} \cdot \phicap &= \phicap^\T \ddt{\rcap} = \phidot \sin\theta.
\end{align}

So our velocity is

\begin{equation}\label{eqn:classicalMechanicsPs2:270}
\Bv = \rdot \rcap + r (\thetadot \thetacap + \phidot \sin\theta \phicap).
\end{equation}

As an aside, now that we know the Euler-Lagrange methods, we could also compute this velocity from the spherical free particle Lagrangian by writing out the canoncial momentum in vector form.  We have

\begin{equation}\label{eqn:classicalMechanicsPs2:290}
\LL = \inv{2} m ( \rdot^2 + r^2 \thetadot^2 + r^2 \phidot^2 \sin^2 \theta)
\end{equation}

We expect our canonical momentum in vector form to be

\begin{align*}
\BP &=
\PD{\rdot}{\LL} \rcap
+\PD{\thetadot}{\LL} \frac{\thetacap}{r}
+\PD{\phidot}{\LL} \frac{\phicap}{r \sin\theta} \\
&=
m \rdot \rcap 
+ m r^2 \thetadot \frac{\thetacap}{r}
+ m r^2 \sin^2\theta \phidot \frac{\phicap}{r \sin\theta} \\
&=
m
\left( 
\rdot \rcap + r \thetadot \thetacap + r \sin\theta \phidot \phicap
\right) \\
&= m \Bv
\end{align*}

This is consistent with \ref{eqn:classicalMechanicsPs2:270} calculated hard way, and is a nice verification that the canonical momentum matches the expectation of being nothing more than how to express the momentum in different coordinate systems.  Returning to the angular momentum calculation we have

\begin{align*}
\rcap \cross \Bv
&=
r \rcap \cross (\thetadot \thetacap + \phidot \sin\theta \phicap) \\
&=
r \left( \thetadot \phicap - \phidot \sin\theta \thetacap \right) ,
\end{align*}

So that our total angular momentum in vector form is

\begin{equation}\label{eqn:classicalMechanicsPs2:310}
\BM = m r^2 
\left( \thetadot \phicap - \phidot \sin\theta \thetacap \right),
\end{equation}

Now, should we wish to extract coordinates with respect to $x,y,z$ we just have to write our our vectors $\phicap$ and $\thetacap$ in the $\{\Be_1, \Be_2, \Be_3\}$ basis and have

\begin{align*}
\BM 
&= 
m r^2 
\begin{bmatrix}
\phicap & -\sin\theta \thetacap 
\end{bmatrix}
\begin{bmatrix}
\thetadot \\
\phidot
\end{bmatrix} \\
&=
m r^2 
\begin{bmatrix}
-\sin\phi & -\sin\theta(\cos\theta \cos\phi) \\
\cos\phi & -\sin\theta(\cos\theta \sin\phi) \\
0 & \sin^2\theta 
\end{bmatrix}
\begin{bmatrix}
\thetadot \\
\phidot
\end{bmatrix} \\
\end{align*}

This matches \ref{eqn:classicalMechanicsPs2:50}, but all the messy trig is isolated to the calculation of $\Bv$ in the spherical polar basis.

\subsection{Solution.  Cylindrical coordinates}

This one should be easier.  To start our position vector is

\begin{equation}\label{eqn:classicalMechanicsPs2:90}
\Br =
\begin{bmatrix}
\rho \cos\phi \\
\rho \sin\phi \\
z
\end{bmatrix}
= \rho \rhocap + z \zcap.
\end{equation}

Our velocity is

\begin{align*}
\Bv
&= \rhodot \rhocap + \rho \ddt{\rhocap} + \zdot \zcap \\
&= \rhodot \rhocap + \rho \ddt{}(\Be_1 e^{i\phi}) + \zdot \zcap \\
&= \rhodot \rhocap + \rho \phidot \Be_2 e^{i\phi} + \zdot \zcap \\
&= \rhodot \rhocap + \rho \phidot \phicap + \zdot \zcap \\
\end{align*}

Here, I have used the Clifford algebra representation of $\rhocap$ with the plane bivector $i = \Be_1 \Be_2$.  In coordinates we have

\begin{equation}\label{eqn:classicalMechanicsPs2:110}
\phicap = \Be_2 (\cos\phi + \Be_1 \Be_2 \sin\phi) = -\Be_1 \sin\phi + \Be_2 \cos\phi,
\end{equation}

so our velocity in matrix form is
\begin{equation}\label{eqn:classicalMechanicsPs2:130}
\Bv = \rhodot
\begin{bmatrix}
\cos\phi \\
\sin\phi \\
0
\end{bmatrix}
+
\rho \phidot
\begin{bmatrix}
-\sin\phi \\
\cos\phi \\
0
\end{bmatrix}
+
\zdot
\begin{bmatrix}
0 \\
0 \\
1
\end{bmatrix}
=
\begin{bmatrix}
\rhodot \cos\phi - \rho \phidot \sin\phi \\
\rhodot \sin\phi + \rho \phidot \cos\phi \\
\zdot
\end{bmatrix}
\end{equation}

For our angular momentum we get

\begin{align*}
\BM
&= \Br \cross (m \Bv) \\
&=
m
\begin{bmatrix}
\rho\sin\phi \zdot - z (\rhodot \sin\phi + \rho \phidot \cos\phi) \\
z (\rhodot \cos\phi - \rho \phidot \sin\phi) - \rho \cos\phi \zdot \\
\rho \cos\phi (\cancel{\rhodot \sin\phi} + \rho \phidot \cos\phi) - \rho\sin\phi (\cancel{\rhodot \cos\phi} - \rho \phidot \sin\phi)
\end{bmatrix} \\
\end{align*}

We can now read off $M_x, M_y, M_z$ by inspection

\begin{equation}\label{eqn:classicalMechanicsPs2:150}
\BM =
m
\begin{bmatrix}
(\rho \zdot 
- z \rhodot )\sin\phi
- z \rho \phidot \cos\phi
\\
( z \rhodot 
- \rho \zdot ) \cos\phi
- z \rho \phidot \sin\phi
\\
\rho^2 \phidot
\end{bmatrix}.
\end{equation}

We also want the (squared) magnitude, which is

\begin{equation}\label{eqn:classicalMechanicsPs2:170}
\BM^2
%=
%m^2 \left(
%(\rho \zdot - z \rhodot )^2 
%+ (z \rho \phidot)^2
%+ \rho^4 \phidot^2
%\right)
=
m^2 \left(
(\rho \zdot - z \rhodot )^2 
+ \rho^2 \phidot^2 ( z^2 + \rho^2 )
%+ (z \rho \phidot)^2
%+ \rho^4 \phidot^2
\right)
\end{equation}

%\EndArticle
\EndNoBibArticle
