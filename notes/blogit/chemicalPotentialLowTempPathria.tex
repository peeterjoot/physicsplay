%
% Copyright � 2013 Peeter Joot.  All Rights Reserved.
% Licenced as described in the file LICENSE under the root directory of this GIT repository.
%
\newcommand{\authorname}{Peeter Joot}
\newcommand{\email}{peeterjoot@protonmail.com}
\newcommand{\basename}{FIXMEbasenameUndefined}
\newcommand{\dirname}{notes/FIXMEdirnameUndefined/}

\renewcommand{\basename}{chemicalPotentialLowTempPathria}
\renewcommand{\dirname}{notes/phy452/}
\newcommand{\keywords}{Statistical mechanics, PHY452H1S, Fermi gas, chemical potential}

\newcommand{\authorname}{Peeter Joot}
\newcommand{\onlineurl}{http://sites.google.com/site/peeterjoot2/math2013/\basename.pdf}
\newcommand{\sourcepath}{\dirname\basename.tex}
\newcommand{\generatetitle}[1]{\chapter{#1}}

\newcommand{\vcsinfo}{%
\section*{}
\noindent{\color{DarkOliveGreen}{\rule{\linewidth}{0.1mm}}}
\paragraph{Document version}
%\paragraph{\color{Maroon}{Document version}}
{
\small
\begin{itemize}
\item Available online at:\\ 
\href{\onlineurl}{\onlineurl}
\item Git Repository: \input{./.revinfo/gitRepo.tex}
\item Source: \sourcepath
\item last commit: \input{./.revinfo/gitCommitString.tex}
\item commit date: \input{./.revinfo/gitCommitDate.tex}
\end{itemize}
}
}

%\PassOptionsToPackage{dvipsnames,svgnames}{xcolor}
\PassOptionsToPackage{square,numbers}{natbib}
\documentclass{scrreprt}

\usepackage[left=2cm,right=2cm]{geometry}
\usepackage[svgnames]{xcolor}
\usepackage{peeters_layout}

\usepackage{natbib}

\usepackage[
colorlinks=true,
bookmarks=false,
pdfauthor={\authorname, \email},
backref 
]{hyperref}

% http://tex.stackexchange.com/questions/75773/how-to-reference-problems-by-the-text-label-in-an-exercise-envioronment
\usepackage[english]{cleveref}
\crefname{Exercise}{exercise}{exercises}
\Crefname{Exercise}{Exercise}{Exercises}

\RequirePackage{titlesec}
\RequirePackage{ifthen}

% http://stackoverflow.com/questions/4932910/date-in-the-tabular-environment
\makeatletter
\let\insertdate\@date
\makeatother

\titleformat{\chapter}[display]
{\bfseries\Large}
{\color{DarkSlateGrey}\filleft \authorname
\ifthenelse{\isundefined{\studentnumber}}{}{\\ \studentnumber}
\ifthenelse{\isundefined{\email}}{}{\\ \email}
\ifthenelse{\isundefined{\dateintitle}}{}{\\ \insertdate}
%\ifthenelse{\isundefined{\coursename}}{}{\\ \coursename} % put in title instead.
}
{4ex}
{\color{DarkOliveGreen}{\titlerule}\color{Maroon}
\vspace{2ex}%
\filright}
[\vspace{2ex}%
\color{DarkOliveGreen}\titlerule
]

\newcommand{\beginArtWithToc}[0]{\begin{document}\tableofcontents}
\newcommand{\beginArtNoToc}[0]{\begin{document}}
\newcommand{\EndNoBibArticle}[0]{\end{document}}
\newcommand{\EndArticle}[0]{\bibliography{Bibliography}\bibliographystyle{plainnat}\end{document}}

% 
%\newcommand{\citep}[1]{\cite{#1}}

\colorSectionsForArticle



\beginArtNoToc

\generatetitle{Next order low temperature Fermi gas chemical potential}
%\chapter{Next order low temperature Fermi gas chemical potential}
\label{chap:chemicalPotentialLowTempPathria}

\makeproblem{Next order low temperature Fermi gas chemical potential}{pr:chemicalPotentialLowTempPathria:1}{

\citep{pathriastatistical} \S 8.1 equation (33) provides an implicit function for $\mu \equiv \kB T \ln z$

\begin{equation}\label{eqn:chemicalPotentialLowTempPathria:20}
n = \frac{4 \pi g}{3} 
\lr{ \frac{2m}{h^2} }
^{3/2}
\mu
^{3/2}
\lr{
1 + \frac{\pi^2}{8} \frac{ (\kB T)^2 }{ \mu^2 }
},
\end{equation}

or

\begin{equation}\label{eqn:chemicalPotentialLowTempPathria:40}
\EF^{3/2} = \mu^{3/2} \lr{ 1 + \frac{\pi^2}{8} \frac{ (\kB T)^2 }{ \mu^2 } }.
\end{equation}

It's stated that a first approximation is

\begin{equation}\label{eqn:chemicalPotentialLowTempPathria:60}
\mu \approx \EF 
\lr{ 1 - \frac{\pi^2}{12} \lr{ \frac{\kB T}{\EF} }^2 }.
\end{equation}

It's not obvious looking at the text how this is follows.  In class, we assumed that $\mu$ was quadratic in $\kB T$, which is also not obvious.  Try to show this directly without making a quadratic assumption.

} % makeproblem

\makeanswer{pr:chemicalPotentialLowTempPathria:1}{
} % makeanswer

\EndArticle
