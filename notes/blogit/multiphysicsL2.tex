%
% Copyright � 2014 Peeter Joot.  All Rights Reserved.
% Licenced as described in the file LICENSE under the root directory of this GIT repository.
%
% for template copy, run:
%
% ~/bin/ct multiphysicsL1  multiphysicsLectureN tl1
%
\newcommand{\authorname}{Peeter Joot}
\newcommand{\email}{peeterjoot@protonmail.com}
\newcommand{\basename}{FIXMEbasenameUndefined}
\newcommand{\dirname}{notes/FIXMEdirnameUndefined/}

\renewcommand{\basename}{multiphysicsLecture{N}}
\renewcommand{\dirname}{notes/ece1254/}
\newcommand{\keywords}{Condensed matter physics, ECE1254H}
\newcommand{\authorname}{Peeter Joot}
\newcommand{\onlineurl}{http://sites.google.com/site/peeterjoot2/math2013/\basename.pdf}
\newcommand{\sourcepath}{\dirname\basename.tex}
\newcommand{\generatetitle}[1]{\chapter{#1}}

\newcommand{\vcsinfo}{%
\section*{}
\noindent{\color{DarkOliveGreen}{\rule{\linewidth}{0.1mm}}}
\paragraph{Document version}
%\paragraph{\color{Maroon}{Document version}}
{
\small
\begin{itemize}
\item Available online at:\\ 
\href{\onlineurl}{\onlineurl}
\item Git Repository: \input{./.revinfo/gitRepo.tex}
\item Source: \sourcepath
\item last commit: \input{./.revinfo/gitCommitString.tex}
\item commit date: \input{./.revinfo/gitCommitDate.tex}
\end{itemize}
}
}

%\PassOptionsToPackage{dvipsnames,svgnames}{xcolor}
\PassOptionsToPackage{square,numbers}{natbib}
\documentclass{scrreprt}

\usepackage[left=2cm,right=2cm]{geometry}
\usepackage[svgnames]{xcolor}
\usepackage{peeters_layout}

\usepackage{natbib}

\usepackage[
colorlinks=true,
bookmarks=false,
pdfauthor={\authorname, \email},
backref 
]{hyperref}

% http://tex.stackexchange.com/questions/75773/how-to-reference-problems-by-the-text-label-in-an-exercise-envioronment
\usepackage[english]{cleveref}
\crefname{Exercise}{exercise}{exercises}
\Crefname{Exercise}{Exercise}{Exercises}

\RequirePackage{titlesec}
\RequirePackage{ifthen}

% http://stackoverflow.com/questions/4932910/date-in-the-tabular-environment
\makeatletter
\let\insertdate\@date
\makeatother

\titleformat{\chapter}[display]
{\bfseries\Large}
{\color{DarkSlateGrey}\filleft \authorname
\ifthenelse{\isundefined{\studentnumber}}{}{\\ \studentnumber}
\ifthenelse{\isundefined{\email}}{}{\\ \email}
\ifthenelse{\isundefined{\dateintitle}}{}{\\ \insertdate}
%\ifthenelse{\isundefined{\coursename}}{}{\\ \coursename} % put in title instead.
}
{4ex}
{\color{DarkOliveGreen}{\titlerule}\color{Maroon}
\vspace{2ex}%
\filright}
[\vspace{2ex}%
\color{DarkOliveGreen}\titlerule
]

\newcommand{\beginArtWithToc}[0]{\begin{document}\tableofcontents}
\newcommand{\beginArtNoToc}[0]{\begin{document}}
\newcommand{\EndNoBibArticle}[0]{\end{document}}
\newcommand{\EndArticle}[0]{\bibliography{Bibliography}\bibliographystyle{plainnat}\end{document}}

% 
%\newcommand{\citep}[1]{\cite{#1}}

\colorSectionsForArticle



\beginArtNoToc
\generatetitle{ECE1254H Modeling of Multiphysics Systems.  Lecture {N}: Assembling system equations automatically.  Taught by Prof.\ Piero Triverio}
%\chapter{Assembling system equations automatically}
\label{chap:multiphysicsLecture{N}}

\section{Disclaimer}

Peeter's lecture notes from class.  These may be incoherent and rough.

\section{Assembling system equations automatically.  Node/branch method}

Consider the sample circuit of \cref{fig:lecture2:lecture2Fig1}.

\imageFigure{../../figures/ece1254/lecture2Fig1}{Sample resistive circuit}{fig:lecture2:lecture2Fig1}{0.3}


\paragraph{Step 1.  Choose unknowns:}  For this problem, let's take 

\begin{itemize}
\item 
node voltages: $V_1, V_2, V_3, V_4$
\item
branch currents: $i_A, i_B, i_C, i_D, i_E$
\end{itemize}

We do not need to introduce additional labels for the source current sources.
We always introduce a reference node and call that zero.

For a circuit with $N$ nodes, and $B$ resistors, there will be $N-1$ unknown node voltages and $B$ unknown branch currents, for a total number of $N - 1 + B$ unknowns.

\paragraph{Step 2.  Conservation equations:}  KCL

\begin{itemize}
	\item 0: $i_A + i_E - i_D = 0$
	\item 1: $-i_A + i_B + i_{S,A} = 0$
	\item 2: $-i_B + i_{S,B} - i_E + i_{S,C} = 0$
	\item 3: $i_C - i_{S,C} = 0$
	\item 4: $-i_{S,A} - i_{S,B} + i_D - i_C = 0$
\end{itemize}

Grouping unknown currents, this is

\begin{itemize}
	\item 0: $i_A + i_E - i_D = 0$
	\item 1: $-i_A + i_B = -i_{S,A}$
	\item 2: $-i_B -i_E = -i_{S,B} -i_{S,C}$
	\item 3: $i_C = i_{S,C}$
	\item 4: $i_D - i_C = i_{S,A} + i_{S,B}$
\end{itemize}

Note that one of these equations is redundant (sum 1-4).  In a circuit with $N$ nodes, we can write at most $N-1$ independent KCLs.

In matrix form

\begin{equation}\label{eqn:multiphysicsL1:20}
\begin{bmatrix}
	-1 & 1 & 0 & 0 & 0 \\
	0 & -1 & 0 & 0 & -1 \\
	0 & 0 & 1 & 0 & 0 \\
	0 & 0 & -1 & 1 & 0 
\end{bmatrix}
\begin{bmatrix}
	i_A \\
	i_B \\
	i_C \\
	i_D \\
	i_E \\
\end{bmatrix}
=
\begin{bmatrix}
	-i_{S,A} \\
	-i_{S,B} -i_{S,C} \\
	i_{S,C} \\
	i_{S,A} + i_{S,B} 
\end{bmatrix}
\end{equation}

We call this first matrix of ones and minus ones the \textAndIndex{incidence matfix} $A$.  This matrix has $B$ columns and $N-1$ rows.  We call the known current matrix $\overbar{I}_S$, and the branch currents $\overbar{I}_B$.  That is

\begin{equation}\label{eqn:multiphysicsL1:40}
	A \overbar{I}_B = \overbar{I}_S.
\end{equation}

Observe that we have both a plus and minus one in all columns except for those columns impacted by our neglect of the reference node current conservation equation.

\paragraph{Algorithm for filling $A$}

In the input file, to describe a resistor of \cref{fig:lecture2:lecture2Fig2}, you'll have a line

\imageFigure{../../figures/ece1254/lecture2Fig2}{Resistor node convention}{fig:lecture2:lecture2Fig2}{0.1}


%FIXME
%R & \text{name} & n_1 & n_2 & \text{value}

Algorthim

FIXME:
%% A = [] ; # zeros
%% ic = 0 ;
%% read a new resistor, adding one column to $A$, ic++
%% if n_1 != 0, A(n_1, ic) = +1 ;
%% if n_2 != 0, A(n_2, ic) = -1 ;

\paragraph{Algorithm for filling $\overbar{I}_S$}

Current sources, as in \cref{fig:lecture2:lecture2Fig3}, a line will have the specification

\imageFigure{../../figures/ece1254/lecture2Fig3}{Current source conventions}{fig:lecture2:lecture2Fig3}{0.1}

FIXME:
%% \overbar{I}_S = zeros(N-1, 1)
%% read a new current 
%% \overbar{I}_S(n_1) -= val ;
%% \overbar{I}_S(n_2) += val ;

\paragraph{Step 3.  Constitutive equations:}  

\begin{equation}\label{eqn:multiphysicsL1:60}
\begin{bmatrix}
i_A \\
i_B \\
i_C \\
i_D \\
i_E
\end{bmatrix}
=
\begin{bmatrix}
1/R_A & & & & \\
      & 1/R_B & & & & \\
      & & 1/R_C & & & \\
      & & & 1/R_D & & \\
      & & & & & 1/R_E
\end{bmatrix}
\begin{bmatrix}
v_A \\
v_B \\
v_C \\
v_D \\
v_E
\end{bmatrix}
\end{equation}

Or 

\begin{equation}\label{eqn:multiphysicsL1:80}
	\overbar{I}_B = \alpha \overbar{V}_B,
\end{equation}

where $\overbar{V}_B$ are the branch voltages, not unknowns of interest directly.  We can write

\begin{equation}\label{eqn:multiphysicsL1:n}
\begin{bmatrix}
v_A \\
v_B \\
v_C \\
v_D \\
v_E
\end{bmatrix}
=
\begin{bmatrix}
	-1 & & &  \\
	1 & -1 & & \\
	& & 1 & -1 \\
	& &  & 1 \\
	& -1 & & 
\end{bmatrix}
\begin{bmatrix}
v_1 \\
v_2 \\
v_3 \\
v_4 
\end{bmatrix}
\end{equation}

Observe that $\alpha$ is the transpose of $A$, allowing us to write

\begin{equation}\label{eqn:multiphysicsL1:n}
\overbar{V}_B = A^\T \overbar{V}_N.
\end{equation}

\paragraph{Solving}

KCLs: $A \overbar{I}_B = \overbar{I}_S$
constituative: $\overbar{I}_B = \alpha \overbar{V}_B \implies \overbar{I}_B = \alpha A^\T \overbar{V}_N$
branch and node voltages: $\overbar{V}_B = A^\T \overbar{V}_N$

In block matrix form, this is

\begin{equation}\label{eqn:multiphysicsL1:n}
\begin{bmatrix}
	A & 0 \\
	I & -\alpha A^\T
\end{bmatrix}
\begin{bmatrix}
	\overbar{I}_B \\
	\overbar{V}_N
\end{bmatrix}
=
\begin{bmatrix}
	\overbar{I}_S \\
	0
\end{bmatrix}.
\end{equation}

Is it square?
%N-1 rows in A.  B columns.
%B rows in I.
%N-1 columns

%\EndArticle
\EndNoBibArticle
