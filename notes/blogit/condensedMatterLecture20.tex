%
% Copyright � 2013 Peeter Joot.  All Rights Reserved.
% Licenced as described in the file LICENSE under the root directory of this GIT repository.
%
\newcommand{\authorname}{Peeter Joot}
\newcommand{\email}{peeterjoot@protonmail.com}
\newcommand{\basename}{FIXMEbasenameUndefined}
\newcommand{\dirname}{notes/FIXMEdirnameUndefined/}

\renewcommand{\basename}{condensedMatterLecture20}
\renewcommand{\dirname}{notes/phy487/}
\newcommand{\keywords}{Condensed matter physics, PHY487H1F}
\newcommand{\authorname}{Peeter Joot}
\newcommand{\onlineurl}{http://sites.google.com/site/peeterjoot2/math2013/\basename.pdf}
\newcommand{\sourcepath}{\dirname\basename.tex}
\newcommand{\generatetitle}[1]{\chapter{#1}}

\newcommand{\vcsinfo}{%
\section*{}
\noindent{\color{DarkOliveGreen}{\rule{\linewidth}{0.1mm}}}
\paragraph{Document version}
%\paragraph{\color{Maroon}{Document version}}
{
\small
\begin{itemize}
\item Available online at:\\ 
\href{\onlineurl}{\onlineurl}
\item Git Repository: \input{./.revinfo/gitRepo.tex}
\item Source: \sourcepath
\item last commit: \input{./.revinfo/gitCommitString.tex}
\item commit date: \input{./.revinfo/gitCommitDate.tex}
\end{itemize}
}
}

%\PassOptionsToPackage{dvipsnames,svgnames}{xcolor}
\PassOptionsToPackage{square,numbers}{natbib}
\documentclass{scrreprt}

\usepackage[left=2cm,right=2cm]{geometry}
\usepackage[svgnames]{xcolor}
\usepackage{peeters_layout}

\usepackage{natbib}

\usepackage[
colorlinks=true,
bookmarks=false,
pdfauthor={\authorname, \email},
backref 
]{hyperref}

% http://tex.stackexchange.com/questions/75773/how-to-reference-problems-by-the-text-label-in-an-exercise-envioronment
\usepackage[english]{cleveref}
\crefname{Exercise}{exercise}{exercises}
\Crefname{Exercise}{Exercise}{Exercises}

\RequirePackage{titlesec}
\RequirePackage{ifthen}

% http://stackoverflow.com/questions/4932910/date-in-the-tabular-environment
\makeatletter
\let\insertdate\@date
\makeatother

\titleformat{\chapter}[display]
{\bfseries\Large}
{\color{DarkSlateGrey}\filleft \authorname
\ifthenelse{\isundefined{\studentnumber}}{}{\\ \studentnumber}
\ifthenelse{\isundefined{\email}}{}{\\ \email}
\ifthenelse{\isundefined{\dateintitle}}{}{\\ \insertdate}
%\ifthenelse{\isundefined{\coursename}}{}{\\ \coursename} % put in title instead.
}
{4ex}
{\color{DarkOliveGreen}{\titlerule}\color{Maroon}
\vspace{2ex}%
\filright}
[\vspace{2ex}%
\color{DarkOliveGreen}\titlerule
]

\newcommand{\beginArtWithToc}[0]{\begin{document}\tableofcontents}
\newcommand{\beginArtNoToc}[0]{\begin{document}}
\newcommand{\EndNoBibArticle}[0]{\end{document}}
\newcommand{\EndArticle}[0]{\bibliography{Bibliography}\bibliographystyle{plainnat}\end{document}}

% 
%\newcommand{\citep}[1]{\cite{#1}}

\colorSectionsForArticle



%\citep{harald2003solid} \S x.y
%\citep{ibach2009solid} \S x.y

%\usepackage{mhchem}
\usepackage{bm} % \EE
\usepackage[version=3]{mhchem}
\usepackage{units}
\newcommand{\nought}[0]{\circ}
%\newcommand{\EF}[0]{\epsilon_{\mathrm{F}}}
\newcommand{\EF}[0]{E_{\mathrm{F}}}
\newcommand{\kF}[0]{k_{\mathrm{F}}}
\newcommand{\vF}[0]{v_{\mathrm{F}}}

\beginArtNoToc
\generatetitle{PHY487H1F Condensed Matter Physics.  Lecture 20: Electric current (cont.).  Taught by Prof.\ Stephen Julian}
%\chapter{Electric current (cont.)}
\label{chap:condensedMatterLecture20}

%\section{Disclaimer}
%
%Peeter's lecture notes from class.  May not be entirely coherent.

\section{Electric current (cont.)}

$\EE \ne 0$ shifts the Fermi distribution so that so that $+\Bv_1 - \Bv$ no longer cancel.

F1

Current flows, unless the band is completely full, or empty.

Full band has $f(E, \BE) = F(E, 0)$, so that there is no current 

F2

An \dquoteAndIndex{insulator} has all bands either completely full or completely empty.

A \dquoteAndIndex{metal} is a solid with a Fermi surface (partly filled band(s)).

The $\EE$ field displaces the Fermi surface, but scattering restores equilibrium, limiting $\Bj$.

Note that a periodic lattice does \underline{not} cause scattering, it causes \underline{band structure}.

Scattering is due to \underline{departures} from periodicity, and is due to impurities and/or vacancies and lattice vibrations (phonons).

F3

A metric for this is called the resistivity, a temperature dependent effect

F4

\paragraph{Non-equilibrium $f(E, \EE)$}

F5

Electrons in the range (1) cannot scatter due to the Pauli exclusion principle, whereas those in the energy range (2) \underline{can} scatter from $+k$ to $-k$.  The net scattering is from $+\Bv$ to $-\Bv$.

For 3D see \citep{ibach2009solid} fig. 9.5, roughly:

F6: a displaced Fermi sphere.

\paragraph{Steady state} rate of scattering from $+\Bv$ to $-\Bv$ is the rate at which new $+\Bv$ carriers appear, where

\begin{dmath}\label{eqn:condensedMatterLecture20:n}
\Hbar \dot{\Bk} = - e \EE.
\end{dmath}

Introduce $\tau$ as the mean scattering time, and let

\begin{dmath}\label{eqn:condensedMatterLecture20:n}
\Delta k = \dot{k} \tau = -\frac{e \mathcal{E}}{\Hbar} \tau,
\end{dmath}

the amount by which the Fermi surface is displaced.

F7

Here 
In the shaded region, the electrons are inert, because the $-\Bv$ and $\Bv$ contributions cancel.  It's only the non-shaded portions of the overlapping spheres that we have to consider.  

\begin{dmath}\label{eqn:condensedMatterLecture20:n}
j_x 
= -\frac{e}{V} \sum_{k, \sigma} v_x(k, \sigma)
= -\frac{e}{\cancel{V}}  \frac{2 \cancel{V} }{(2 \pi)^3}
\int 
\mathLabelBox
{
\kF^2 \sin\theta d\theta d\phi 
}
{
area element at $\theta, $\phi$
}
\mathLabelBox
[
   labelstyle={below of=m\themathLableNode, below of=m\themathLableNode}
]
{
\Bigr{ -\frac{e \mathcal{E}}{\Hbar} \tau \cos\theta \Bigl} 
}
{
$\delta k$ at $\theta, \phi$
}
\times
\mathLabelBox
{
\vF \cos\theta
}
{
v_x
}
\approx 
\frac{e^2}{4 \pi^3} \tau \kF^2 \frac{\vF}{\Hbar} 
\mathLabelBox
{
\int_0^\pi \sin\theta \cos^2\theta d\theta
}
{$
-\int_0^\pi \cos^2\theta d \cos\theta = - \int_1^{-1} u^2 du = 2/3
$}
\mathLabelBox
[
   labelstyle={below of=m\themathLableNode, below of=m\themathLableNode}
]
{
2 \pi
}
{
$\phi$ integral
}
\mathcal{E}_x
\end{dmath}

With 

\begin{dmath}\label{eqn:condensedMatterLecture20:n}
\vF = \frac{\Hbar \kF}{m^\conj},
\end{dmath}

this is

\begin{dmath}\label{eqn:condensedMatterLecture20:n}
j_x \approx
\frac{e^2\tau}{m^\conj} \frac{\kF^3}{3 \pi^2} \mathcal{E}_x
\end{dmath}

\begin{dmath}\label{eqn:condensedMatterLecture20:n}
j_x = \frac{n e^2 \tau}{m^\conj} \mathcal{E}_x
\end{dmath}

or

\begin{dmath}\label{eqn:condensedMatterLecture20:n}
\myBoxed{
\sigma = \frac{n e^2 \tau}{m^\conj}
}
\end{dmath}

This is the \textAndIndex{Drude formula for conductivity}.

Note that Drude's derivation predated quantum mechanics.  He treated the electrons classically introducing a drift velocity

F8

where

\begin{dmath}\label{eqn:condensedMatterLecture20:n}
v_{\text{drift}} = \text{acceleration} \tau = -\frac{e \mathcal{E}}{m} \tau,
\end{dmath}

so that 

\begin{dmath}\label{eqn:condensedMatterLecture20:n}
\Bj = - e n v_{\text{drift}} \EE = \frac{ n e^2 \tau }{m} \EE.
\end{dmath}

%\EndArticle
\EndNoBibArticle
