%
% Copyright � 2015 Peeter Joot.  All Rights Reserved.
% Licenced as described in the file LICENSE under the root directory of this GIT repository.
%
\documentclass[]{eliblog}

\usepackage{amsmath}
\usepackage{mathpazo}

%
% shorthand for bold symbols, convenient for vectors and matrices
%
\newcommand{\Ba}[0]{\mathbf{a}}
\newcommand{\Bb}[0]{\mathbf{b}}
\newcommand{\Bc}[0]{\mathbf{c}}
\newcommand{\Bd}[0]{\mathbf{d}}
\newcommand{\Be}[0]{\mathbf{e}}
\newcommand{\Bf}[0]{\mathbf{f}}
\newcommand{\Bg}[0]{\mathbf{g}}
\newcommand{\Bh}[0]{\mathbf{h}}
\newcommand{\Bi}[0]{\mathbf{i}}
\newcommand{\Bj}[0]{\mathbf{j}}
\newcommand{\Bk}[0]{\mathbf{k}}
\newcommand{\Bl}[0]{\mathbf{l}}
\newcommand{\Bm}[0]{\mathbf{m}}
\newcommand{\Bn}[0]{\mathbf{n}}
\newcommand{\Bo}[0]{\mathbf{o}}
\newcommand{\Bp}[0]{\mathbf{p}}
\newcommand{\Bq}[0]{\mathbf{q}}
\newcommand{\Br}[0]{\mathbf{r}}
\newcommand{\Bs}[0]{\mathbf{s}}
\newcommand{\Bt}[0]{\mathbf{t}}
\newcommand{\Bu}[0]{\mathbf{u}}
\newcommand{\Bv}[0]{\mathbf{v}}
\newcommand{\Bw}[0]{\mathbf{w}}
\newcommand{\Bx}[0]{\mathbf{x}}
\newcommand{\By}[0]{\mathbf{y}}
\newcommand{\Bz}[0]{\mathbf{z}}
\newcommand{\BA}[0]{\mathbf{A}}
\newcommand{\BB}[0]{\mathbf{B}}
\newcommand{\BC}[0]{\mathbf{C}}
\newcommand{\BD}[0]{\mathbf{D}}
\newcommand{\BE}[0]{\mathbf{E}}
\newcommand{\BF}[0]{\mathbf{F}}
\newcommand{\BG}[0]{\mathbf{G}}
\newcommand{\BH}[0]{\mathbf{H}}
\newcommand{\BI}[0]{\mathbf{I}}
\newcommand{\BJ}[0]{\mathbf{J}}
\newcommand{\BK}[0]{\mathbf{K}}
\newcommand{\BL}[0]{\mathbf{L}}
\newcommand{\BM}[0]{\mathbf{M}}
\newcommand{\BN}[0]{\mathbf{N}}
\newcommand{\BO}[0]{\mathbf{O}}
\newcommand{\BP}[0]{\mathbf{P}}
\newcommand{\BQ}[0]{\mathbf{Q}}
\newcommand{\BR}[0]{\mathbf{R}}
\newcommand{\BS}[0]{\mathbf{S}}
\newcommand{\BT}[0]{\mathbf{T}}
\newcommand{\BU}[0]{\mathbf{U}}
\newcommand{\BV}[0]{\mathbf{V}}
\newcommand{\BW}[0]{\mathbf{W}}
\newcommand{\BX}[0]{\mathbf{X}}
\newcommand{\BY}[0]{\mathbf{Y}}
\newcommand{\BZ}[0]{\mathbf{Z}}

\newcommand{\Bzero}[0]{\mathbf{0}}
\newcommand{\Btheta}[0]{\boldsymbol{\theta}}
\newcommand{\Btau}[0]{\boldsymbol{\tau}}
\newcommand{\Bomega}[0]{\boldsymbol{\omega}}

%
% shorthand for unit vectors
%
\newcommand{\acap}[0]{\hat{\Ba}}
\newcommand{\bcap}[0]{\hat{\Bb}}
\newcommand{\ccap}[0]{\hat{\Bc}}
\newcommand{\dcap}[0]{\hat{\Bd}}
\newcommand{\ecap}[0]{\hat{\Be}}
\newcommand{\fcap}[0]{\hat{\Bf}}
\newcommand{\gcap}[0]{\hat{\Bg}}
\newcommand{\hcap}[0]{\hat{\Bh}}
\newcommand{\icap}[0]{\hat{\Bi}}
\newcommand{\jcap}[0]{\hat{\Bj}}
\newcommand{\kcap}[0]{\hat{\Bk}}
\newcommand{\lcap}[0]{\hat{\Bl}}
\newcommand{\mcap}[0]{\hat{\Bm}}
\newcommand{\ncap}[0]{\hat{\Bn}}
\newcommand{\ocap}[0]{\hat{\Bo}}
\newcommand{\pcap}[0]{\hat{\Bp}}
\newcommand{\qcap}[0]{\hat{\Bq}}
\newcommand{\rcap}[0]{\hat{\Br}}
\newcommand{\scap}[0]{\hat{\Bs}}
\newcommand{\tcap}[0]{\hat{\Bt}}
\newcommand{\ucap}[0]{\hat{\Bu}}
\newcommand{\vcap}[0]{\hat{\Bv}}
\newcommand{\wcap}[0]{\hat{\Bw}}
\newcommand{\xcap}[0]{\hat{\Bx}}
\newcommand{\ycap}[0]{\hat{\By}}
\newcommand{\zcap}[0]{\hat{\Bz}}
\newcommand{\thetacap}[0]{\hat{\Btheta}}

%
% to write R^n and C^n in a distinguishable fashion.  Perhaps change this
% to the double lined characters upon figuring out how to do so.
%
\newcommand{\C}[1]{$\mathbb{C}^{#1}$}
\newcommand{\R}[1]{$\mathbb{R}^{#1}$}

%
% various generally useful helpers
%

% derivative of #1 wrt. #2:
\newcommand{\D}[2] {\frac {d#2} {d#1}}

\newcommand{\inv}[1]{\frac{1}{#1}}
\newcommand{\cross}[0]{\times}

\newcommand{\abs}[1]{\lvert{#1}\rvert}
\newcommand{\norm}[1]{\lVert{#1}\rVert}
\newcommand{\innerprod}[2]{\langle{#1}, {#2}\rangle}
\newcommand{\dotprod}[2]{{#1} \cdot {#2}}
\newcommand{\bdotprod}[2]{\left({#1} \cdot {#2}\right)}
\newcommand{\crossprod}[2]{{#1} \cross {#2}}
\newcommand{\tripleprod}[3]{\dotprod{\left(\crossprod{#1}{#2}\right)}{#3}}

\DeclareMathOperator{\Proj}{Proj}
\DeclareMathOperator{\Span}{span}
\DeclareMathOperator{\Sgn}{sgn}
\DeclareMathOperator{\Area}{Area}
\DeclareMathOperator{\Volume}{Volume}

%
% A few miscellaneous things specific to this document
%
\newcommand{\crossop}[1]{\crossprod{#1}{}}

% R2 vector.
\newcommand{\VectorTwo}[2]{
\begin{bmatrix}
 {#1} \\
 {#2}
\end{bmatrix}
}

\newcommand{\VectorN}[1]{
\begin{bmatrix}
{#1}_1 \\
{#1}_2 \\
\vdots \\
{#1}_N \\
\end{bmatrix}
}

\newcommand{\DETuvij}[4]{
\begin{vmatrix}
 {#1}_{#3} & {#1}_{#4} \\
 {#2}_{#3} & {#2}_{#4}
\end{vmatrix}
}

\newcommand{\DETuvwijk}[6]{
\begin{vmatrix}
 {#1}_{#4} & {#1}_{#5} & {#1}_{#6} \\
 {#2}_{#4} & {#2}_{#5} & {#2}_{#6} \\
 {#3}_{#4} & {#3}_{#5} & {#3}_{#6}
\end{vmatrix}
}

\newcommand{\DETuvwxijkl}[8]{
\begin{vmatrix}
 {#1}_{#5} & {#1}_{#6} & {#1}_{#7} & {#1}_{#8} \\
 {#2}_{#5} & {#2}_{#6} & {#2}_{#7} & {#2}_{#8} \\
 {#3}_{#5} & {#3}_{#6} & {#3}_{#7} & {#3}_{#8} \\
 {#4}_{#5} & {#4}_{#6} & {#4}_{#7} & {#4}_{#8} \\
\end{vmatrix}
}

%\newcommand{\DETuvwxyijklm}[10]{
%\begin{vmatrix}
% {#1}_{#6} & {#1}_{#7} & {#1}_{#8} & {#1}_{#9} & {#1}_{#10} \\
% {#2}_{#6} & {#2}_{#7} & {#2}_{#8} & {#2}_{#9} & {#2}_{#10} \\
% {#3}_{#6} & {#3}_{#7} & {#3}_{#8} & {#3}_{#9} & {#3}_{#10} \\
% {#4}_{#6} & {#4}_{#7} & {#4}_{#8} & {#4}_{#9} & {#4}_{#10} \\
% {#5}_{#6} & {#5}_{#7} & {#5}_{#8} & {#5}_{#9} & {#5}_{#10}
%\end{vmatrix}
%}

% R3 vector.
\newcommand{\VectorThree}[3]{
\begin{bmatrix}
 {#1} \\
 {#2} \\
 {#3}
\end{bmatrix}
}



\author{Peeter Joot}
\email{peeter.joot@gmail.com}

%\documentclass[]{eliblogwidescreen}

\usepackage{amsmath}
\usepackage{mathpazo}

%
% shorthand for bold symbols, convenient for vectors and matrices
%
\newcommand{\Ba}[0]{\mathbf{a}}
\newcommand{\Bb}[0]{\mathbf{b}}
\newcommand{\Bc}[0]{\mathbf{c}}
\newcommand{\Bd}[0]{\mathbf{d}}
\newcommand{\Be}[0]{\mathbf{e}}
\newcommand{\Bf}[0]{\mathbf{f}}
\newcommand{\Bg}[0]{\mathbf{g}}
\newcommand{\Bh}[0]{\mathbf{h}}
\newcommand{\Bi}[0]{\mathbf{i}}
\newcommand{\Bj}[0]{\mathbf{j}}
\newcommand{\Bk}[0]{\mathbf{k}}
\newcommand{\Bl}[0]{\mathbf{l}}
\newcommand{\Bm}[0]{\mathbf{m}}
\newcommand{\Bn}[0]{\mathbf{n}}
\newcommand{\Bo}[0]{\mathbf{o}}
\newcommand{\Bp}[0]{\mathbf{p}}
\newcommand{\Bq}[0]{\mathbf{q}}
\newcommand{\Br}[0]{\mathbf{r}}
\newcommand{\Bs}[0]{\mathbf{s}}
\newcommand{\Bt}[0]{\mathbf{t}}
\newcommand{\Bu}[0]{\mathbf{u}}
\newcommand{\Bv}[0]{\mathbf{v}}
\newcommand{\Bw}[0]{\mathbf{w}}
\newcommand{\Bx}[0]{\mathbf{x}}
\newcommand{\By}[0]{\mathbf{y}}
\newcommand{\Bz}[0]{\mathbf{z}}
\newcommand{\BA}[0]{\mathbf{A}}
\newcommand{\BB}[0]{\mathbf{B}}
\newcommand{\BC}[0]{\mathbf{C}}
\newcommand{\BD}[0]{\mathbf{D}}
\newcommand{\BE}[0]{\mathbf{E}}
\newcommand{\BF}[0]{\mathbf{F}}
\newcommand{\BG}[0]{\mathbf{G}}
\newcommand{\BH}[0]{\mathbf{H}}
\newcommand{\BI}[0]{\mathbf{I}}
\newcommand{\BJ}[0]{\mathbf{J}}
\newcommand{\BK}[0]{\mathbf{K}}
\newcommand{\BL}[0]{\mathbf{L}}
\newcommand{\BM}[0]{\mathbf{M}}
\newcommand{\BN}[0]{\mathbf{N}}
\newcommand{\BO}[0]{\mathbf{O}}
\newcommand{\BP}[0]{\mathbf{P}}
\newcommand{\BQ}[0]{\mathbf{Q}}
\newcommand{\BR}[0]{\mathbf{R}}
\newcommand{\BS}[0]{\mathbf{S}}
\newcommand{\BT}[0]{\mathbf{T}}
\newcommand{\BU}[0]{\mathbf{U}}
\newcommand{\BV}[0]{\mathbf{V}}
\newcommand{\BW}[0]{\mathbf{W}}
\newcommand{\BX}[0]{\mathbf{X}}
\newcommand{\BY}[0]{\mathbf{Y}}
\newcommand{\BZ}[0]{\mathbf{Z}}

\newcommand{\Bzero}[0]{\mathbf{0}}
\newcommand{\Btheta}[0]{\boldsymbol{\theta}}
\newcommand{\Btau}[0]{\boldsymbol{\tau}}
\newcommand{\Bomega}[0]{\boldsymbol{\omega}}

%
% shorthand for unit vectors
%
\newcommand{\acap}[0]{\hat{\Ba}}
\newcommand{\bcap}[0]{\hat{\Bb}}
\newcommand{\ccap}[0]{\hat{\Bc}}
\newcommand{\dcap}[0]{\hat{\Bd}}
\newcommand{\ecap}[0]{\hat{\Be}}
\newcommand{\fcap}[0]{\hat{\Bf}}
\newcommand{\gcap}[0]{\hat{\Bg}}
\newcommand{\hcap}[0]{\hat{\Bh}}
\newcommand{\icap}[0]{\hat{\Bi}}
\newcommand{\jcap}[0]{\hat{\Bj}}
\newcommand{\kcap}[0]{\hat{\Bk}}
\newcommand{\lcap}[0]{\hat{\Bl}}
\newcommand{\mcap}[0]{\hat{\Bm}}
\newcommand{\ncap}[0]{\hat{\Bn}}
\newcommand{\ocap}[0]{\hat{\Bo}}
\newcommand{\pcap}[0]{\hat{\Bp}}
\newcommand{\qcap}[0]{\hat{\Bq}}
\newcommand{\rcap}[0]{\hat{\Br}}
\newcommand{\scap}[0]{\hat{\Bs}}
\newcommand{\tcap}[0]{\hat{\Bt}}
\newcommand{\ucap}[0]{\hat{\Bu}}
\newcommand{\vcap}[0]{\hat{\Bv}}
\newcommand{\wcap}[0]{\hat{\Bw}}
\newcommand{\xcap}[0]{\hat{\Bx}}
\newcommand{\ycap}[0]{\hat{\By}}
\newcommand{\zcap}[0]{\hat{\Bz}}
\newcommand{\thetacap}[0]{\hat{\Btheta}}

%
% to write R^n and C^n in a distinguishable fashion.  Perhaps change this
% to the double lined characters upon figuring out how to do so.
%
\newcommand{\C}[1]{$\mathbb{C}^{#1}$}
\newcommand{\R}[1]{$\mathbb{R}^{#1}$}

%
% various generally useful helpers
%

% derivative of #1 wrt. #2:
\newcommand{\D}[2] {\frac {d#2} {d#1}}

\newcommand{\inv}[1]{\frac{1}{#1}}
\newcommand{\cross}[0]{\times}

\newcommand{\abs}[1]{\lvert{#1}\rvert}
\newcommand{\norm}[1]{\lVert{#1}\rVert}
\newcommand{\innerprod}[2]{\langle{#1}, {#2}\rangle}
\newcommand{\dotprod}[2]{{#1} \cdot {#2}}
\newcommand{\bdotprod}[2]{\left({#1} \cdot {#2}\right)}
\newcommand{\crossprod}[2]{{#1} \cross {#2}}
\newcommand{\tripleprod}[3]{\dotprod{\left(\crossprod{#1}{#2}\right)}{#3}}

\DeclareMathOperator{\Proj}{Proj}
\DeclareMathOperator{\Span}{span}
\DeclareMathOperator{\Sgn}{sgn}
\DeclareMathOperator{\Area}{Area}
\DeclareMathOperator{\Volume}{Volume}

%
% A few miscellaneous things specific to this document
%
\newcommand{\crossop}[1]{\crossprod{#1}{}}

% R2 vector.
\newcommand{\VectorTwo}[2]{
\begin{bmatrix}
 {#1} \\
 {#2}
\end{bmatrix}
}

\newcommand{\VectorN}[1]{
\begin{bmatrix}
{#1}_1 \\
{#1}_2 \\
\vdots \\
{#1}_N \\
\end{bmatrix}
}

\newcommand{\DETuvij}[4]{
\begin{vmatrix}
 {#1}_{#3} & {#1}_{#4} \\
 {#2}_{#3} & {#2}_{#4}
\end{vmatrix}
}

\newcommand{\DETuvwijk}[6]{
\begin{vmatrix}
 {#1}_{#4} & {#1}_{#5} & {#1}_{#6} \\
 {#2}_{#4} & {#2}_{#5} & {#2}_{#6} \\
 {#3}_{#4} & {#3}_{#5} & {#3}_{#6}
\end{vmatrix}
}

\newcommand{\DETuvwxijkl}[8]{
\begin{vmatrix}
 {#1}_{#5} & {#1}_{#6} & {#1}_{#7} & {#1}_{#8} \\
 {#2}_{#5} & {#2}_{#6} & {#2}_{#7} & {#2}_{#8} \\
 {#3}_{#5} & {#3}_{#6} & {#3}_{#7} & {#3}_{#8} \\
 {#4}_{#5} & {#4}_{#6} & {#4}_{#7} & {#4}_{#8} \\
\end{vmatrix}
}

%\newcommand{\DETuvwxyijklm}[10]{
%\begin{vmatrix}
% {#1}_{#6} & {#1}_{#7} & {#1}_{#8} & {#1}_{#9} & {#1}_{#10} \\
% {#2}_{#6} & {#2}_{#7} & {#2}_{#8} & {#2}_{#9} & {#2}_{#10} \\
% {#3}_{#6} & {#3}_{#7} & {#3}_{#8} & {#3}_{#9} & {#3}_{#10} \\
% {#4}_{#6} & {#4}_{#7} & {#4}_{#8} & {#4}_{#9} & {#4}_{#10} \\
% {#5}_{#6} & {#5}_{#7} & {#5}_{#8} & {#5}_{#9} & {#5}_{#10}
%\end{vmatrix}
%}

% R3 vector.
\newcommand{\VectorThree}[3]{
\begin{bmatrix}
 {#1} \\
 {#2} \\
 {#3}
\end{bmatrix}
}



\author{Peeter Joot}
\email{peeter.joot@gmail.com}


\chapter{PHY450H1S.  Relativistic Electrodynamics Lecture 17 (Taught by Prof. Erich Poppitz).  Energy and momentum density.  Starting a Green's function solution to Maxwell's equation.}
\label{chap:relativisticElectrodynamicsL17}
%\useCCL
\blogpage{http://sites.google.com/site/peeterjoot/math2011/relativisticElectrodynamicsL17.pdf}
\date{Mar 8, 2011}
\revisionInfo{relativisticElectrodynamicsL17.tex}

%\beginArtWithToc
\beginArtNoToc

\section{Reading.}

Covering chapter 6 material from the text \cite{landau1980classical}.

FIXME: COVERING.  Also, is the chapter reference correct?
%Covering \href{}{lecture notes pp. xx-yy}:

\section{Review.  Energy density and Poynting vector.}

Last time we showed that Maxwell's equations imply

\begin{equation}\label{eqn:relativisticElectrodynamicsL17:n}
\PD{t}{} \frac{\BE^2 + \BB^2 }{8 \pi} = -\Bj c\dot \BE - \spacegrad \cdot \BS
\end{equation}

In the lecture, Professor Poppitz said he was free here to use a full time derivative.  When asked why, it was because he was considering $\BE$ and $\BB$ here to be functions of time only, since they were measured at a fixed point in space.  This is really the same thing as using a time partial, so I'll just be explicit (and somewhat less confusing).

\begin{equation}\label{eqn:relativisticElectrodynamicsL17:n}
\BS = \frac{c}{4 \pi} \BE \cross \BB
\end{equation}

\begin{equation}\label{eqn:relativisticElectrodynamicsL17:n}
\PD{t}{} \int_V \frac{\BE^2 + \BB^2 }{8 \pi} = - \int_V \Bj c\dot \BE - \int_{\partial_V} d^2 \sigma \cdot \BS
\end{equation}

Any change in the energy must either due to currents, or energy escaping through the surface.

\begin{align}\label{eqn:relativisticElectrodynamicsL17:n}
\mathcal{E} = \frac{\BE^2 + \BB^2 }{8 \pi} &= \mbox{Energy density of the EM field} \\
\BS = \frac{c}{4 \pi} \BE \cross \BB &= \mbox{Energy flux of the EM fields}
\end{align}

The energy flux of the EM field: this is the energy flowing through $d^2 \BA$ in unit time ($\BS \cdot d^2 \BA$).

\section{How about electromagnetic waves?}

In a plane wave moving in direction $\Bk$. 

PICTURE: $\BE \parallel \zcap$, $\BB \parallel \xcap$, $\Bk \parallel \ycap$.

So, $\BS \parallel \Bk$ since $\BE \cross \BB \propto \Bk$.

$\Abs{\BS}$ for a plane wave is the amount of energy through unit area perpendicular to $\Bk$ in unit time.

Recall that we calculated

\begin{align}\label{eqn:relativisticElectrodynamicsL17:n}
\BB &= (\Bp \cross \Bbeta) \sin(\omega t - \Bp \cdot \Bx) \\
\BE &= \Bbeta \Abs{\Bp} \sin(\omega t - \Bp \cdot \Bx)
\end{align}

So see that $\Abs{\BE} = \Abs{\BB}$

\begin{align*}
\BS 
&= \frac{\Bk}{\Abs{\Bk} \frac{c}{4 \pi} \BE^2  \\
&= \frac{\Bk}{\Abs{\Bk} c \frac{\BE^2 + \BB^2}{8 \pi} \\
&= \frac{\Bk}{\Abs{\Bk} e \mathcal{E}
\end{align*}

\begin{equation}\label{eqn:relativisticElectrodynamicsL17:n}
[\BS] = \frac{\text{energy}}{\text{time $\times$ area}} = \frac{\text{momentum $\times$ speed}}{\text{time $\times$ area}}
\end{equation}

\begin{align*}
\left[\frac{\BS}{c^2} \right] 
&= \frac{\text{momentum}}{\text{time $\times$ area $\times$ speed}} \\
&= \frac{\text{momentum}}{\text{area $\times$ distance}} \\
&= \frac{\text{momentum}}{\text{volume}} \\
\end{align*}

So we wee that $\BS/c^2$ is indeed rightly called ``the momentum density'' of the EM field.

We will later find that $\mathcal{E}$ and $\BS$ are components of a rank-2 four tensor

\begin{equation}\label{eqn:relativisticElectrodynamicsL17:n}
T^{ij} = 
\begin{bmatrix}
\mathcal{E} & \frac{S^1}{c^2} & \frac{S^2}{c^2} & \frac{S^3}{c^2} \\
\frac{S^1}{c^2}  &
\frac{S^1}{c^2}  &
\frac{S^1}{c^2}  &
\begin{bmatrix}
\sigma^{\alpha\beta}
\end{bmatrix}
\end{bmatrix}
\end{equation}

where $\sigma^{\alpha\beta}$ is the stress tensor.  We will get to all this in more detail later.

For EM wave we have

\begin{equation}\label{eqn:relativisticElectrodynamicsL17:n}
\BS = \kcap c \mathcal{E}
\end{equation}

(this is the energy flux)

\begin{equation}\label{eqn:relativisticElectrodynamicsL17:n}
\frac{\BS}{c^2} = \kcap \frac{\mathcal{E}}{c}
\end{equation}

(the momentum density of the wave).

\begin{equation}\label{eqn:relativisticElectrodynamicsL17:n}
c \Abs{\frac{\BS}{c^2}} = \mathcal{E}
\end{equation}

(recall $\mathcal{E} = c\mathcal{\Bp}$ for massless particles.

EM waves carry energy and momentum so when absorbed or reflected these are transferred to bodies.

Kepler speculated that this was the fact because he had observed that the tails of the comets were being pushed by the sunlight, since the tails faced away from the sun.

Maxwell also suggested that light would extert a force (presumably he wrote down the ``Maxwell stress tensor'' $T^{ij}$ that is named after him).

This was actually measured later in 1901, by Peter Lebedev (Russia).

PICTURE: pole with flags in vacuum jar.  Black (absorber) on one side, and Silver (reflector) on the other.  Between the two of these, momentum conservation will introduce rotation (in the direction of the silver).

This is actually a tricky experiment and requires the vacuum, since the black surface warms up, and heats up the nearby gas molecules, which causes a rotation in the opposite direction due to just these thermal effects.

Another example (a factor) that prevents star collapse under gravition is the radiation pressure of the light.

\section{Moving on.  Solving Maxwell's equation}

Our equations are

\begin{align}\label{eqn:relativisticElectrodynamicsL17:n}
\epsilon^{i j k l} \partial_j F_{k l} &= 0 \\
\partial_i F^{i k} &= \frac{4 \pi}{c} j^k,
\end{align}

where we assume that $j^k(\Bx, t)$ is a given.  Our task is to find $F^{i k}$, the $(\BE, \BB)$ fields.

Proceed by finding $A^i$.  First, as usual when $F_{i j} = \partial_i A_j - \partial_j A_i$.  The Bianchi identity is satisfied so we focus on the current equation.

In terms of potentials

\begin{equation}\label{eqn:relativisticElectrodynamicsL17:n}
\partial_i (\partial^i A^k - \partial^k A^i) = \frac{ 4 \pi}{c} j^k
\end{equation}

or

\begin{equation}\label{eqn:relativisticElectrodynamicsL17:n}
\partial_i \partial^i A^k - \partial^k (\partial_i A^i) = \frac{ 4 \pi}{c} j^k
\end{equation}

We want to work in the Lorentz gauge $\partial_i A^i = 0$.  This is justified by the simplicity of the remaining problem

\begin{equation}\label{eqn:relativisticElectrodynamicsL17:n}
\partial_i \partial^i A^k = \frac{4 \pi}{c} j^k
\end{equation}

Write

\begin{align}\label{eqn:relativisticElectrodynamicsL17:n}
\partial_i \partial^i = \inv{c^2} \PDSq{t^2}{} - \Delta &= \square \\
\Delta &= \PDSq{x}{} + \PDSq{y}{} + \PDSq{z}{}
\end{align}

This $\square$ is the d'Alembert operator (``d'Alembertian'').

Our equation is 

\begin{equation}\label{eqn:relativisticElectrodynamicsL17:n}
\square A^k = \frac{4 \pi}{c} j^k
\end{equation}

(in the Lorentz gauge)

If we learn how to solve (**), then we've learned all.

Method: Green's function's

In electrostatics where $j^0 = 0$, $A^0 \ne 0$ only, we have

\begin{equation}\label{eqn:relativisticElectrodynamicsL17:n}
\Delta A^0 = 4 pi \rho
\end{equation}

Solution 

\begin{equation}\label{eqn:relativisticElectrodynamicsL17:n}
\Delta_{\Bx} G(\Bx - \Bx') = \delta^3( \Bx - \Bx')
\end{equation}

PICTURE: 

\begin{equation}\label{eqn:relativisticElectrodynamicsL17:n}
\rho(\Bx') d^3 \Bx'
\end{equation}

(a small box)

acting through distance $\Abs{\Bx - \Bx'}$, acting at point $\Bx$.

we (can) end up finding that 

\begin{equation}\label{eqn:relativisticElectrodynamicsL17:n}
\phi(\Bx) = \int \frac{\rho(\Bx')}{\Abs{\Bx - \Bx'}} d^3 \Bx'
\end{equation}

\EndArticle
