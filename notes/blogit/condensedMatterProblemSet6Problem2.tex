%
% Copyright � 2013 Peeter Joot.  All Rights Reserved.
% Licenced as described in the file LICENSE under the root directory of this GIT repository.
%
\makeoproblem{Electronic specific heat of one and two-dimensional free electron metals}{condensedMatter:problemSet6:2}{\citep{ibach2009solid} pr 6.1}{ 
\makesubproblem{}{condensedMatter:problemSet6:2a}
Calculate the density of states for a two-dimensional gas of free electrons in a so-called quantum well.  The boundary conditions for the electronic wavefunction are:

\begin{dmath}\label{eqn:condensedMatterProblemSet6Problem2:20}
\psi(x, y, z) = 0, \quad \mbox{for $\Abs{x} > a$},
\end{dmath}

where $a$ is of atomic dimensions.

\makesubproblem{}{condensedMatter:problemSet6:2b}
Calculate the density of states for a one-dimensional gas of free electrons in a so-called quantum wire, the the boundary conditions:

\begin{dmath}\label{eqn:condensedMatterProblemSet6Problem2:40}
\psi(x, y, z) = 0, \quad \mbox{for $\Abs{x} > a$, and $\Abs{y} > b$},
\end{dmath}

where $a$ and $b$ are of atomic dimensions.

\makesubproblem{}{condensedMatter:problemSet6:2c}
Can such electron gases be realized physically?

\paragraph{Clarification}

The question says something rather obscure about $\psi(x,y,z)=0$ 
for $\Abs{x}>a$, where $a$ is of atomic dimensions.  What they mean 
is that a two-dimensional electron gas can be thought of as 
contained in a three-dimensional box, with two of the dimensions 
being macroscopic, but the third dimension (perpendicular to the 
plane of the electron gas) being microscopic.  As a result, the energy 
levels are quasi-continuous in two dimensions, but discrete in the
third dimension, and we assume that only the ground state of the 
discrete spectrum is occupied, so there is no summation over 
that third dimension. Similarly, in the one-dimensional electron gas, only 
one direction in $q$ space is quasi-continuous.)
} % makeproblem

\makeanswer{condensedMatter:problemSet6:2}{ 
\makeSubAnswer{}{condensedMatter:problemSet6:2a}

TODO.
\makeSubAnswer{}{condensedMatter:problemSet6:2b}

TODO.
\makeSubAnswer{}{condensedMatter:problemSet6:2c}

TODO.
}

