%
% Copyright � 2014 Peeter Joot.  All Rights Reserved.
% Licenced as described in the file LICENSE under the root directory of this GIT repository.
%
\newcommand{\authorname}{Peeter Joot}
\newcommand{\email}{peeterjoot@protonmail.com}
\newcommand{\basename}{FIXMEbasenameUndefined}
\newcommand{\dirname}{notes/FIXMEdirnameUndefined/}

\renewcommand{\basename}{multiphysicsL6}
\renewcommand{\dirname}{notes/ece1254/}
\newcommand{\keywords}{Condensed matter physics, ECE1254H}
\newcommand{\authorname}{Peeter Joot}
\newcommand{\onlineurl}{http://sites.google.com/site/peeterjoot2/math2013/\basename.pdf}
\newcommand{\sourcepath}{\dirname\basename.tex}
\newcommand{\generatetitle}[1]{\chapter{#1}}

\newcommand{\vcsinfo}{%
\section*{}
\noindent{\color{DarkOliveGreen}{\rule{\linewidth}{0.1mm}}}
\paragraph{Document version}
%\paragraph{\color{Maroon}{Document version}}
{
\small
\begin{itemize}
\item Available online at:\\ 
\href{\onlineurl}{\onlineurl}
\item Git Repository: \input{./.revinfo/gitRepo.tex}
\item Source: \sourcepath
\item last commit: \input{./.revinfo/gitCommitString.tex}
\item commit date: \input{./.revinfo/gitCommitDate.tex}
\end{itemize}
}
}

%\PassOptionsToPackage{dvipsnames,svgnames}{xcolor}
\PassOptionsToPackage{square,numbers}{natbib}
\documentclass{scrreprt}

\usepackage[left=2cm,right=2cm]{geometry}
\usepackage[svgnames]{xcolor}
\usepackage{peeters_layout}

\usepackage{natbib}

\usepackage[
colorlinks=true,
bookmarks=false,
pdfauthor={\authorname, \email},
backref 
]{hyperref}

% http://tex.stackexchange.com/questions/75773/how-to-reference-problems-by-the-text-label-in-an-exercise-envioronment
\usepackage[english]{cleveref}
\crefname{Exercise}{exercise}{exercises}
\Crefname{Exercise}{Exercise}{Exercises}

\RequirePackage{titlesec}
\RequirePackage{ifthen}

% http://stackoverflow.com/questions/4932910/date-in-the-tabular-environment
\makeatletter
\let\insertdate\@date
\makeatother

\titleformat{\chapter}[display]
{\bfseries\Large}
{\color{DarkSlateGrey}\filleft \authorname
\ifthenelse{\isundefined{\studentnumber}}{}{\\ \studentnumber}
\ifthenelse{\isundefined{\email}}{}{\\ \email}
\ifthenelse{\isundefined{\dateintitle}}{}{\\ \insertdate}
%\ifthenelse{\isundefined{\coursename}}{}{\\ \coursename} % put in title instead.
}
{4ex}
{\color{DarkOliveGreen}{\titlerule}\color{Maroon}
\vspace{2ex}%
\filright}
[\vspace{2ex}%
\color{DarkOliveGreen}\titlerule
]

\newcommand{\beginArtWithToc}[0]{\begin{document}\tableofcontents}
\newcommand{\beginArtNoToc}[0]{\begin{document}}
\newcommand{\EndNoBibArticle}[0]{\end{document}}
\newcommand{\EndArticle}[0]{\bibliography{Bibliography}\bibliographystyle{plainnat}\end{document}}

% 
%\newcommand{\citep}[1]{\cite{#1}}

\colorSectionsForArticle



%\usepackage{kbordermatrix}

\beginArtNoToc
\generatetitle{ECE1254H Modeling of Multiphysics Systems.  Lecture 6: Matrix norm, singular decomposition, and conditioning number.  Taught by Prof.\ Piero Triverio}
%\chapter{Matrix norm, singular decomposition, and conditioning number}
\label{chap:multiphysicsL6}

\section{Disclaimer}

Peeter's lecture notes from class.  These may be incoherent and rough.

\section{Matrix norm}

We've defined the matrix norm of \( M \), for the system \( \overbar{y} = M \overbar{x} \) as

\begin{dmath}\label{eqn:multiphysicsL6:21}
\Norm{M} = \max_{\Norm{\overbar{x}} = 1} \Norm{ M \overbar{x} }.
\end{dmath}

We will typically use the \( L_2 \) norm, so that the matrix norm is
\begin{dmath}\label{eqn:multiphysicsL6:41}
\Norm{M}_2 = \max_{\Norm{\overbar{x}}_2 = 1} \Norm{ M \overbar{x} }_2.
\end{dmath}

It can be shown that 

\begin{dmath}\label{eqn:multiphysicsL6:61}
   \Norm{M}_2 = \max_i \sigma_i(M),
\end{dmath}

where \( \sigma_i(M) \) are the singular values.

\index{SVD}
\index{Singular value decomposition}
\makedefinition{Singular value decomposition}{dfn:multiphysicsL6:1}{
        Given \( M \in \text{\R{n \times m}} \), we can find a representation of \( M \)

\begin{dmath}\label{eqn:multiphysicsL6:81}
M = U \Sigma V^\T,
\end{dmath}

where \( U \) and \( V\) are orthogonal matrices such that \( U^\T U = 1 \), and \( V^\T V = 1 \), and

\begin{dmath}\label{eqn:multiphysicsL6:101}
   \Sigma = 
\begin{bmatrix}
  \sigma_1 &          &        &          &         &         &\\
           & \sigma_2 &        &          &         &         &\\
           &          & \ddots &          &         &         &\\
           &          &        & \sigma_r &         &         &\\
           &          &          &        &       0 &         &         \\
           &          &          &        &         & \ddots  &         \\
           &          &          &        &         &         & 0       \\
\end{bmatrix}
\end{dmath}

The values \( \sigma_i, \sigma_{i} > \sigma_{i+1} > 0 \) are called the singular values of \( M \), and \(r\) is the rank of \( M \).

%The Matlab command to compute this is \underline{svd}.
% O(4 n^3) div O(26 n^3) ???
}

example: 2 x 2 case (for column vectors \( \overbar{u}_i, \overbar{v}_j \)).

\begin{dmath}\label{eqn:multiphysicsL6:n}
M = 
\begin{bmatrix}
\overbar{u}_1 & \overbar{u}_2
\end{bmatrix}
\begin{bmatrix}
        \sigma_1 &  \\
        & \sigma_2
\end{bmatrix}
\begin{bmatrix}
\overbar{v}_1^\T \\
\overbar{v}_2^\T 
\end{bmatrix}
\end{dmath}

Consider \( \overbar{y} = M \overbar{x} \), and take an \( \overbar{x} \) with \( \Norm{\overbar{x}}_2 = 1 \)

F1

\section{Conditioning number}

Given a pertubation of \( M \overbar{x} = \overbar{b} \) to 

\begin{dmath}\label{eqn:multiphysicsL6:121}
        \lr{ M + \delta M }
        \lr{ \overbar{x} + \delta \overbar{x} } = \overbar{b},
\end{dmath}

or
\begin{dmath}\label{eqn:multiphysicsL6:141}
        \cancel{ M \overbar{x} - \overbar{b} } + \delta M \overbar{x} + M \delta \overbar{x} + \delta M \delta \overbar{x} = 0.
\end{dmath}

This gives

\begin{dmath}\label{eqn:multiphysicsL6:161}
        M \delta \overbar{x} = - \delta M \overbar{x} - \delta M \delta \overbar{x},
\end{dmath}

or
\begin{dmath}\label{eqn:multiphysicsL6:181}
        \delta \overbar{x} = - M^{-1} \delta M \lr{ \overbar{x} + \delta \overbar{x} }.
\end{dmath}

Taking norms

\begin{dmath}\label{eqn:multiphysicsL6:201}
\Norm{ \delta \overbar{x}}_2 = \Norm{
        M^{-1} \delta M \lr{ \overbar{x} + \delta \overbar{x} }
}_2
\le
\Norm{ M^{-1} }_2 \Norm{ \delta M }_2 \Norm{ \overbar{x} + \delta \overbar{x} }_2,
\end{dmath}

or 
\begin{dmath}\label{eqn:multiphysicsL6:221}
\mathLabelBox
[
   labelstyle={yshift=1.2em},
   linestyle={}
]
{
\frac{\Norm{ \delta \overbar{x}}_2 }{
\Norm{ \overbar{x} + \delta \overbar{x}}_2 }
}
{relative error of solution}
\le 
\mathLabelBox
[
   labelstyle={below of=m\themathLableNode, below of=m\themathLableNode}
]
{
\Norm{M}_2
\Norm{M^{-1}}_2
}
{conditioning number of \(M\)}
\mathLabelBox
[
   labelstyle={xshift=2cm},
   linestyle={out=270,in=90, latex-}
]
{
\frac{ \Norm{ \delta M}_2 }
{\Norm{M}_2}
}
{
relative pertubation of \( M \)
}.
\end{dmath}

The \textAndIndex{conditioning number} can be shown to be

\begin{dmath}\label{eqn:multiphysicsL6:241}
        \cond(M) = 
        \frac
        {\sigma_{\mathrm{max}}}
        {\sigma_{\mathrm{min}}}
        \ge 1
\end{dmath}

FIXME: justify.

%\EndArticle
\EndNoBibArticle
