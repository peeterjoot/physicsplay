%
% Copyright © 2015 Peeter Joot.  All Rights Reserved.
% Licenced as described in the file LICENSE under the root directory of this GIT repository.
%
\newcommand{\authorname}{Peeter Joot}
\newcommand{\email}{peeterjoot@protonmail.com}
\newcommand{\basename}{FIXMEbasenameUndefined}
\newcommand{\dirname}{notes/FIXMEdirnameUndefined/}

\renewcommand{\basename}{chapter3Notes}
\renewcommand{\dirname}{notes/ece1229/}
%\newcommand{\dateintitle}{}
%\newcommand{\keywords}{}

\newcommand{\authorname}{Peeter Joot}
\newcommand{\onlineurl}{http://sites.google.com/site/peeterjoot2/math2013/\basename.pdf}
\newcommand{\sourcepath}{\dirname\basename.tex}
\newcommand{\generatetitle}[1]{\chapter{#1}}

\newcommand{\vcsinfo}{%
\section*{}
\noindent{\color{DarkOliveGreen}{\rule{\linewidth}{0.1mm}}}
\paragraph{Document version}
%\paragraph{\color{Maroon}{Document version}}
{
\small
\begin{itemize}
\item Available online at:\\ 
\href{\onlineurl}{\onlineurl}
\item Git Repository: \input{./.revinfo/gitRepo.tex}
\item Source: \sourcepath
\item last commit: \input{./.revinfo/gitCommitString.tex}
\item commit date: \input{./.revinfo/gitCommitDate.tex}
\end{itemize}
}
}

%\PassOptionsToPackage{dvipsnames,svgnames}{xcolor}
\PassOptionsToPackage{square,numbers}{natbib}
\documentclass{scrreprt}

\usepackage[left=2cm,right=2cm]{geometry}
\usepackage[svgnames]{xcolor}
\usepackage{peeters_layout}

\usepackage{natbib}

\usepackage[
colorlinks=true,
bookmarks=false,
pdfauthor={\authorname, \email},
backref 
]{hyperref}

% http://tex.stackexchange.com/questions/75773/how-to-reference-problems-by-the-text-label-in-an-exercise-envioronment
\usepackage[english]{cleveref}
\crefname{Exercise}{exercise}{exercises}
\Crefname{Exercise}{Exercise}{Exercises}

\RequirePackage{titlesec}
\RequirePackage{ifthen}

% http://stackoverflow.com/questions/4932910/date-in-the-tabular-environment
\makeatletter
\let\insertdate\@date
\makeatother

\titleformat{\chapter}[display]
{\bfseries\Large}
{\color{DarkSlateGrey}\filleft \authorname
\ifthenelse{\isundefined{\studentnumber}}{}{\\ \studentnumber}
\ifthenelse{\isundefined{\email}}{}{\\ \email}
\ifthenelse{\isundefined{\dateintitle}}{}{\\ \insertdate}
%\ifthenelse{\isundefined{\coursename}}{}{\\ \coursename} % put in title instead.
}
{4ex}
{\color{DarkOliveGreen}{\titlerule}\color{Maroon}
\vspace{2ex}%
\filright}
[\vspace{2ex}%
\color{DarkOliveGreen}\titlerule
]

\newcommand{\beginArtWithToc}[0]{\begin{document}\tableofcontents}
\newcommand{\beginArtNoToc}[0]{\begin{document}}
\newcommand{\EndNoBibArticle}[0]{\end{document}}
\newcommand{\EndArticle}[0]{\bibliography{Bibliography}\bibliographystyle{plainnat}\end{document}}

% 
%\newcommand{\citep}[1]{\cite{#1}}

\colorSectionsForArticle



\usepackage{ece1229}

\beginArtNoToc

\generatetitle{Maxwell's equations}
%\chapter{Fundamental parameters of Antennas}
%\label{chap:chapter3Notes}

These are notes for the UofT course ECE1229, Advanced Antenna Theory, taught by Prof. Eleftheriades, covering \chaptext 3 \citep{balanis2005antenna} content.

Unlike most of the other classes I have taken, I am not attempting to take comprehensive notes for this class.  The class is taught on slides that match the textbook so closely, there is little value to me taking notes that just replicate the text.  Instead, I am annotating my copy of textbook with little details instead.  My usual notes collection for the class will contain musings of details that were unclear, or in some cases, details that were provided in class, but are not in the text (and too long to pencil into my book.)

\section{Maxwell's equation review}

For reasons that are yet to be seen (and justified), we work with a generalization of Maxwell's equations to include 
electric AND magnetic charge densities.

\begin{subequations}
\begin{dmath}\label{eqn:chapter3Notes:20}
\spacegrad \cross \bcE = - \bcM - \PD{t}{\bcB}
\end{dmath}
\begin{dmath}\label{eqn:chapter3Notes:40}
\spacegrad \cross \bcH = \bcJ + \PD{t}{\bcD}
\end{dmath}
\begin{dmath}\label{eqn:chapter3Notes:60}
\spacegrad \cdot \bcD = \rho
\end{dmath}
\begin{dmath}\label{eqn:chapter3Notes:80}
\spacegrad \cdot \bcB = \rho_m.
\end{dmath}
\end{subequations}

Assuming a phasor relationships of the form \( \bcE = \Real \lr{ \BE(\Br) e^{j \omega t}} \) for the fields and the currents, these reduce to

\begin{subequations}
\label{eqn:chapter3Notes:99}
\begin{dmath}\label{eqn:chapter3Notes:100}
\spacegrad \cross \BE = - \BM - j \omega \BB
\end{dmath}
\begin{dmath}\label{eqn:chapter3Notes:120}
\spacegrad \cross \BH = \BJ + j \omega \BD
\end{dmath}
\begin{dmath}\label{eqn:chapter3Notes:140}
\spacegrad \cdot \BD = \rho
\end{dmath}
\begin{dmath}\label{eqn:chapter3Notes:160}
\spacegrad \cdot \BB = \rho_m.
\end{dmath}
\end{subequations}

In engineering the fields

\begin{itemize}
	\item \( \BE \) : Electric field intensity (\si{V/m}, \si{Volts/meter}).
	\item \( \BH \) : Magnetic field intensity (\si{A/m}, \si{Amperes/meter}).
\end{itemize}

are designated primary fields, whereas

\begin{itemize}
	\item \( \BD \) : Electric flux density (or displacement vector) (\si{C/m}, \si{Coulombs/meter}).
	\item \( \BB \) : Magnetic flux density (\si{W/m}, \si{Webers/meter}).
\end{itemize}

are designated the induced fields.  The currents and charges are

\begin{itemize}
	\item \( \BJ \) : Electric current density (\si{A/m}).
	\item \( \BM \) : Magnetic current density (\si{V/m}).
	\item \( \rho \) : Electric charge density (\si{C/m^3}).
	\item \( \rho_m \) : Magnetic charge density (\si{W/m^3}).
\end{itemize}

Because \( \spacegrad \cdot \lr{ \spacegrad \cross \Bf } = 0 \) for any (sufficiently continuous) vector \( \Bf \), divergence relations between the currents and the charges follow from \cref{eqn:chapter3Notes:99}

\begin{dmath}\label{eqn:chapter3Notes:180}
0 
= -\spacegrad \cdot \BM - j \omega \spacegrad \cdot \BB 
= -\spacegrad \cdot \BM - j \omega \rho_m,
\end{dmath}

and

\begin{dmath}\label{eqn:chapter3Notes:200}
0 
= \spacegrad \cdot \BJ + j \omega \spacegrad \cdot \BD
= \spacegrad \cdot \BJ + j \omega \rho,
\end{dmath}

These are the phasor forms of the continuity equations

\begin{subequations}
\begin{dmath}\label{eqn:chapter3Notes:220}
\spacegrad \cdot \BM = - j \omega \rho_m
\end{dmath}
\begin{dmath}\label{eqn:chapter3Notes:240}
\spacegrad \cdot \BJ = -j \omega \rho.
\end{dmath}
\end{subequations}

\paragraph{Integral forms}

The integral forms of Maxwell's equations follow from Stokes' theorem and the divergence theorems.  Stokes' theorem is a relation between the integral of the curl and the outwards normal differential area element of a surface, to the boundary of that surface, and applies to any surface with that boundary

\begin{dmath}\label{eqn:chapter3Notes:260}
\iint
d\BA \cdot \lr{\spacegrad \cross \Bf} 
= \ointctrclockwise \Bf \cdot d\Bl.
\end{dmath}

The divergence theorem, a special case of the general Stokes' theorem is

\begin{dmath}\label{eqn:chapter3Notes:280}
\iiint_{V} \spacegrad \cdot \Bf dV
= \iint_{\partial V} \Bf \cdot d\BA,
\end{dmath}

where the integral is over the surface of the volume, and the area element of the bounding integral has an outwards normal orientation.

See \citep{gabook:stokesTheoremGeometricAlgebra} for a derivation of this and various generalizations.

Applying these to \cref{eqn:chapter3Notes:99} gives

\begin{subequations}
\begin{dmath}\label{eqn:chapter3Notes:320}
\ointctrclockwise d\Bl \cdot \BE = - 
\iint d\BA \cdot \lr{
\BM + j \omega \BB
}
\end{dmath}
\begin{dmath}\label{eqn:chapter3Notes:340}
\ointctrclockwise d\Bl \cdot \BH =  
\iint d\BA \cdot \lr{
\BJ + j \omega \BD
}
\end{dmath}
\begin{dmath}\label{eqn:chapter3Notes:360}
\iint_{\partial V} d\BA \cdot \BD = \iiint \rho dV
\end{dmath}
\begin{dmath}\label{eqn:chapter3Notes:380}
\iint_{\partial V} d\BA \cdot \BB = \iiint \rho_m dV
\end{dmath}
\end{subequations}

\section{Constituativity relations}

For linear isotropic homogeneous materials, the following constitutive relations apply

\begin{itemize}
\item \( \BD = \epsilon \BE \)
\item \( \BB = \mu \BH \)
\item \( \BJ = \sigma \BE \)
\end{itemize}

where
FIXME
% \begin{itemize}
% \item \( \epsilon = \epsilon \BE \)
% \item \( \BB = \mu \BH \)
% \item \( \BJ = \sigma \BE \)
% \end{itemize}

In AM radio, will see ferrite cores with the inductors, which introduces non-unit \( \mu_r \).  This is to increase the radiation resistance.

\section{Boundary conditions}

For good electric conductor \( \BE = 0 \).
For good magnetic conductor \( \BB = 0 \).

\section{Linear time invarient}

Linear time invarient meant that the impulse response \( h(t,t') \) was a function of just the difference in times \( h(t,t') = h(t-t') \).

\section{Green's functions}

For electromagnetic problems the impulse function sources \( \delta(\Br - \Br') \) also has a direction, and can yield any of \( E_x, E_y, E_z \).  A tensor impulse response is required.

Some overview of an approach that uses such tensor Green's functions is outlined on the slides.  It gets really messy since we require four tensor Green's functions to handle electric and magnetic current and charges.  Because of this complexity, we don't go down this path, and use potentials instead.

\section{Followup}

- duality slides... was there a typo there on the fields?

- problem.  Show

\begin{dmath}\label{eqn:chapter3Notes:300}
\spacegrad^2 \phi_e + k^2 \phi_e = \rho_e/\epsilon
\end{dmath}

- compare Prof's Helmholtz Green's derivation to my old 456 helmholtz Green's derivation.

How did he get away with doing it so quickly?

\section{Notation}

\begin{itemize}
\item Phasor frequency terms are written as \( e^{j \omega t} \), not \( e^{-j \omega t} \), as done in physics.  I didn't recall this, and wouldn't have assumed it.  This is the case in both \citep{jackson1975cew:simpleRadiating} and \citep{griffiths1999introduction:waves}.  The latter however, also uses \( \cos(\omega t - k r) \) for spherical waves possibly implying an alternate phasor sign convention in that content, so I'd be wary about trusting any absolute ``engineering'' vs. physics sign convention without checking carefully.
\item Prof's J looks like pi with a tail.  Mu's look almost like M's.
\item In Green's functions \( G(\Br, \Br') \), \( \Br \) is the point of observation, and \( \Br' \) is the point in the convolution integration space.
\end{itemize}
FIXME: Prof's \( \lambda \) should be here too...pretty whacky.

%\section{Mathematica notebooks}
%
%\input{../ece1229/mathematica.tex}

\EndArticle
