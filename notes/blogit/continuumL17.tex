%
% Copyright � 2015 Peeter Joot.  All Rights Reserved.
% Licenced as described in the file LICENSE under the root directory of this GIT repository.
%
\documentclass[]{eliblog}

\usepackage{amsmath}
\usepackage{mathpazo}

%
% shorthand for bold symbols, convenient for vectors and matrices
%
\newcommand{\Ba}[0]{\mathbf{a}}
\newcommand{\Bb}[0]{\mathbf{b}}
\newcommand{\Bc}[0]{\mathbf{c}}
\newcommand{\Bd}[0]{\mathbf{d}}
\newcommand{\Be}[0]{\mathbf{e}}
\newcommand{\Bf}[0]{\mathbf{f}}
\newcommand{\Bg}[0]{\mathbf{g}}
\newcommand{\Bh}[0]{\mathbf{h}}
\newcommand{\Bi}[0]{\mathbf{i}}
\newcommand{\Bj}[0]{\mathbf{j}}
\newcommand{\Bk}[0]{\mathbf{k}}
\newcommand{\Bl}[0]{\mathbf{l}}
\newcommand{\Bm}[0]{\mathbf{m}}
\newcommand{\Bn}[0]{\mathbf{n}}
\newcommand{\Bo}[0]{\mathbf{o}}
\newcommand{\Bp}[0]{\mathbf{p}}
\newcommand{\Bq}[0]{\mathbf{q}}
\newcommand{\Br}[0]{\mathbf{r}}
\newcommand{\Bs}[0]{\mathbf{s}}
\newcommand{\Bt}[0]{\mathbf{t}}
\newcommand{\Bu}[0]{\mathbf{u}}
\newcommand{\Bv}[0]{\mathbf{v}}
\newcommand{\Bw}[0]{\mathbf{w}}
\newcommand{\Bx}[0]{\mathbf{x}}
\newcommand{\By}[0]{\mathbf{y}}
\newcommand{\Bz}[0]{\mathbf{z}}
\newcommand{\BA}[0]{\mathbf{A}}
\newcommand{\BB}[0]{\mathbf{B}}
\newcommand{\BC}[0]{\mathbf{C}}
\newcommand{\BD}[0]{\mathbf{D}}
\newcommand{\BE}[0]{\mathbf{E}}
\newcommand{\BF}[0]{\mathbf{F}}
\newcommand{\BG}[0]{\mathbf{G}}
\newcommand{\BH}[0]{\mathbf{H}}
\newcommand{\BI}[0]{\mathbf{I}}
\newcommand{\BJ}[0]{\mathbf{J}}
\newcommand{\BK}[0]{\mathbf{K}}
\newcommand{\BL}[0]{\mathbf{L}}
\newcommand{\BM}[0]{\mathbf{M}}
\newcommand{\BN}[0]{\mathbf{N}}
\newcommand{\BO}[0]{\mathbf{O}}
\newcommand{\BP}[0]{\mathbf{P}}
\newcommand{\BQ}[0]{\mathbf{Q}}
\newcommand{\BR}[0]{\mathbf{R}}
\newcommand{\BS}[0]{\mathbf{S}}
\newcommand{\BT}[0]{\mathbf{T}}
\newcommand{\BU}[0]{\mathbf{U}}
\newcommand{\BV}[0]{\mathbf{V}}
\newcommand{\BW}[0]{\mathbf{W}}
\newcommand{\BX}[0]{\mathbf{X}}
\newcommand{\BY}[0]{\mathbf{Y}}
\newcommand{\BZ}[0]{\mathbf{Z}}

\newcommand{\Bzero}[0]{\mathbf{0}}
\newcommand{\Btheta}[0]{\boldsymbol{\theta}}
\newcommand{\Btau}[0]{\boldsymbol{\tau}}
\newcommand{\Bomega}[0]{\boldsymbol{\omega}}

%
% shorthand for unit vectors
%
\newcommand{\acap}[0]{\hat{\Ba}}
\newcommand{\bcap}[0]{\hat{\Bb}}
\newcommand{\ccap}[0]{\hat{\Bc}}
\newcommand{\dcap}[0]{\hat{\Bd}}
\newcommand{\ecap}[0]{\hat{\Be}}
\newcommand{\fcap}[0]{\hat{\Bf}}
\newcommand{\gcap}[0]{\hat{\Bg}}
\newcommand{\hcap}[0]{\hat{\Bh}}
\newcommand{\icap}[0]{\hat{\Bi}}
\newcommand{\jcap}[0]{\hat{\Bj}}
\newcommand{\kcap}[0]{\hat{\Bk}}
\newcommand{\lcap}[0]{\hat{\Bl}}
\newcommand{\mcap}[0]{\hat{\Bm}}
\newcommand{\ncap}[0]{\hat{\Bn}}
\newcommand{\ocap}[0]{\hat{\Bo}}
\newcommand{\pcap}[0]{\hat{\Bp}}
\newcommand{\qcap}[0]{\hat{\Bq}}
\newcommand{\rcap}[0]{\hat{\Br}}
\newcommand{\scap}[0]{\hat{\Bs}}
\newcommand{\tcap}[0]{\hat{\Bt}}
\newcommand{\ucap}[0]{\hat{\Bu}}
\newcommand{\vcap}[0]{\hat{\Bv}}
\newcommand{\wcap}[0]{\hat{\Bw}}
\newcommand{\xcap}[0]{\hat{\Bx}}
\newcommand{\ycap}[0]{\hat{\By}}
\newcommand{\zcap}[0]{\hat{\Bz}}
\newcommand{\thetacap}[0]{\hat{\Btheta}}

%
% to write R^n and C^n in a distinguishable fashion.  Perhaps change this
% to the double lined characters upon figuring out how to do so.
%
\newcommand{\C}[1]{$\mathbb{C}^{#1}$}
\newcommand{\R}[1]{$\mathbb{R}^{#1}$}

%
% various generally useful helpers
%

% derivative of #1 wrt. #2:
\newcommand{\D}[2] {\frac {d#2} {d#1}}

\newcommand{\inv}[1]{\frac{1}{#1}}
\newcommand{\cross}[0]{\times}

\newcommand{\abs}[1]{\lvert{#1}\rvert}
\newcommand{\norm}[1]{\lVert{#1}\rVert}
\newcommand{\innerprod}[2]{\langle{#1}, {#2}\rangle}
\newcommand{\dotprod}[2]{{#1} \cdot {#2}}
\newcommand{\bdotprod}[2]{\left({#1} \cdot {#2}\right)}
\newcommand{\crossprod}[2]{{#1} \cross {#2}}
\newcommand{\tripleprod}[3]{\dotprod{\left(\crossprod{#1}{#2}\right)}{#3}}

\DeclareMathOperator{\Proj}{Proj}
\DeclareMathOperator{\Span}{span}
\DeclareMathOperator{\Sgn}{sgn}
\DeclareMathOperator{\Area}{Area}
\DeclareMathOperator{\Volume}{Volume}

%
% A few miscellaneous things specific to this document
%
\newcommand{\crossop}[1]{\crossprod{#1}{}}

% R2 vector.
\newcommand{\VectorTwo}[2]{
\begin{bmatrix}
 {#1} \\
 {#2}
\end{bmatrix}
}

\newcommand{\VectorN}[1]{
\begin{bmatrix}
{#1}_1 \\
{#1}_2 \\
\vdots \\
{#1}_N \\
\end{bmatrix}
}

\newcommand{\DETuvij}[4]{
\begin{vmatrix}
 {#1}_{#3} & {#1}_{#4} \\
 {#2}_{#3} & {#2}_{#4}
\end{vmatrix}
}

\newcommand{\DETuvwijk}[6]{
\begin{vmatrix}
 {#1}_{#4} & {#1}_{#5} & {#1}_{#6} \\
 {#2}_{#4} & {#2}_{#5} & {#2}_{#6} \\
 {#3}_{#4} & {#3}_{#5} & {#3}_{#6}
\end{vmatrix}
}

\newcommand{\DETuvwxijkl}[8]{
\begin{vmatrix}
 {#1}_{#5} & {#1}_{#6} & {#1}_{#7} & {#1}_{#8} \\
 {#2}_{#5} & {#2}_{#6} & {#2}_{#7} & {#2}_{#8} \\
 {#3}_{#5} & {#3}_{#6} & {#3}_{#7} & {#3}_{#8} \\
 {#4}_{#5} & {#4}_{#6} & {#4}_{#7} & {#4}_{#8} \\
\end{vmatrix}
}

%\newcommand{\DETuvwxyijklm}[10]{
%\begin{vmatrix}
% {#1}_{#6} & {#1}_{#7} & {#1}_{#8} & {#1}_{#9} & {#1}_{#10} \\
% {#2}_{#6} & {#2}_{#7} & {#2}_{#8} & {#2}_{#9} & {#2}_{#10} \\
% {#3}_{#6} & {#3}_{#7} & {#3}_{#8} & {#3}_{#9} & {#3}_{#10} \\
% {#4}_{#6} & {#4}_{#7} & {#4}_{#8} & {#4}_{#9} & {#4}_{#10} \\
% {#5}_{#6} & {#5}_{#7} & {#5}_{#8} & {#5}_{#9} & {#5}_{#10}
%\end{vmatrix}
%}

% R3 vector.
\newcommand{\VectorThree}[3]{
\begin{bmatrix}
 {#1} \\
 {#2} \\
 {#3}
\end{bmatrix}
}



\author{Peeter Joot}
\email{peeter.joot@gmail.com}

%\documentclass[]{eliblogwidescreen}

\usepackage{amsmath}
\usepackage{mathpazo}

%
% shorthand for bold symbols, convenient for vectors and matrices
%
\newcommand{\Ba}[0]{\mathbf{a}}
\newcommand{\Bb}[0]{\mathbf{b}}
\newcommand{\Bc}[0]{\mathbf{c}}
\newcommand{\Bd}[0]{\mathbf{d}}
\newcommand{\Be}[0]{\mathbf{e}}
\newcommand{\Bf}[0]{\mathbf{f}}
\newcommand{\Bg}[0]{\mathbf{g}}
\newcommand{\Bh}[0]{\mathbf{h}}
\newcommand{\Bi}[0]{\mathbf{i}}
\newcommand{\Bj}[0]{\mathbf{j}}
\newcommand{\Bk}[0]{\mathbf{k}}
\newcommand{\Bl}[0]{\mathbf{l}}
\newcommand{\Bm}[0]{\mathbf{m}}
\newcommand{\Bn}[0]{\mathbf{n}}
\newcommand{\Bo}[0]{\mathbf{o}}
\newcommand{\Bp}[0]{\mathbf{p}}
\newcommand{\Bq}[0]{\mathbf{q}}
\newcommand{\Br}[0]{\mathbf{r}}
\newcommand{\Bs}[0]{\mathbf{s}}
\newcommand{\Bt}[0]{\mathbf{t}}
\newcommand{\Bu}[0]{\mathbf{u}}
\newcommand{\Bv}[0]{\mathbf{v}}
\newcommand{\Bw}[0]{\mathbf{w}}
\newcommand{\Bx}[0]{\mathbf{x}}
\newcommand{\By}[0]{\mathbf{y}}
\newcommand{\Bz}[0]{\mathbf{z}}
\newcommand{\BA}[0]{\mathbf{A}}
\newcommand{\BB}[0]{\mathbf{B}}
\newcommand{\BC}[0]{\mathbf{C}}
\newcommand{\BD}[0]{\mathbf{D}}
\newcommand{\BE}[0]{\mathbf{E}}
\newcommand{\BF}[0]{\mathbf{F}}
\newcommand{\BG}[0]{\mathbf{G}}
\newcommand{\BH}[0]{\mathbf{H}}
\newcommand{\BI}[0]{\mathbf{I}}
\newcommand{\BJ}[0]{\mathbf{J}}
\newcommand{\BK}[0]{\mathbf{K}}
\newcommand{\BL}[0]{\mathbf{L}}
\newcommand{\BM}[0]{\mathbf{M}}
\newcommand{\BN}[0]{\mathbf{N}}
\newcommand{\BO}[0]{\mathbf{O}}
\newcommand{\BP}[0]{\mathbf{P}}
\newcommand{\BQ}[0]{\mathbf{Q}}
\newcommand{\BR}[0]{\mathbf{R}}
\newcommand{\BS}[0]{\mathbf{S}}
\newcommand{\BT}[0]{\mathbf{T}}
\newcommand{\BU}[0]{\mathbf{U}}
\newcommand{\BV}[0]{\mathbf{V}}
\newcommand{\BW}[0]{\mathbf{W}}
\newcommand{\BX}[0]{\mathbf{X}}
\newcommand{\BY}[0]{\mathbf{Y}}
\newcommand{\BZ}[0]{\mathbf{Z}}

\newcommand{\Bzero}[0]{\mathbf{0}}
\newcommand{\Btheta}[0]{\boldsymbol{\theta}}
\newcommand{\Btau}[0]{\boldsymbol{\tau}}
\newcommand{\Bomega}[0]{\boldsymbol{\omega}}

%
% shorthand for unit vectors
%
\newcommand{\acap}[0]{\hat{\Ba}}
\newcommand{\bcap}[0]{\hat{\Bb}}
\newcommand{\ccap}[0]{\hat{\Bc}}
\newcommand{\dcap}[0]{\hat{\Bd}}
\newcommand{\ecap}[0]{\hat{\Be}}
\newcommand{\fcap}[0]{\hat{\Bf}}
\newcommand{\gcap}[0]{\hat{\Bg}}
\newcommand{\hcap}[0]{\hat{\Bh}}
\newcommand{\icap}[0]{\hat{\Bi}}
\newcommand{\jcap}[0]{\hat{\Bj}}
\newcommand{\kcap}[0]{\hat{\Bk}}
\newcommand{\lcap}[0]{\hat{\Bl}}
\newcommand{\mcap}[0]{\hat{\Bm}}
\newcommand{\ncap}[0]{\hat{\Bn}}
\newcommand{\ocap}[0]{\hat{\Bo}}
\newcommand{\pcap}[0]{\hat{\Bp}}
\newcommand{\qcap}[0]{\hat{\Bq}}
\newcommand{\rcap}[0]{\hat{\Br}}
\newcommand{\scap}[0]{\hat{\Bs}}
\newcommand{\tcap}[0]{\hat{\Bt}}
\newcommand{\ucap}[0]{\hat{\Bu}}
\newcommand{\vcap}[0]{\hat{\Bv}}
\newcommand{\wcap}[0]{\hat{\Bw}}
\newcommand{\xcap}[0]{\hat{\Bx}}
\newcommand{\ycap}[0]{\hat{\By}}
\newcommand{\zcap}[0]{\hat{\Bz}}
\newcommand{\thetacap}[0]{\hat{\Btheta}}

%
% to write R^n and C^n in a distinguishable fashion.  Perhaps change this
% to the double lined characters upon figuring out how to do so.
%
\newcommand{\C}[1]{$\mathbb{C}^{#1}$}
\newcommand{\R}[1]{$\mathbb{R}^{#1}$}

%
% various generally useful helpers
%

% derivative of #1 wrt. #2:
\newcommand{\D}[2] {\frac {d#2} {d#1}}

\newcommand{\inv}[1]{\frac{1}{#1}}
\newcommand{\cross}[0]{\times}

\newcommand{\abs}[1]{\lvert{#1}\rvert}
\newcommand{\norm}[1]{\lVert{#1}\rVert}
\newcommand{\innerprod}[2]{\langle{#1}, {#2}\rangle}
\newcommand{\dotprod}[2]{{#1} \cdot {#2}}
\newcommand{\bdotprod}[2]{\left({#1} \cdot {#2}\right)}
\newcommand{\crossprod}[2]{{#1} \cross {#2}}
\newcommand{\tripleprod}[3]{\dotprod{\left(\crossprod{#1}{#2}\right)}{#3}}

\DeclareMathOperator{\Proj}{Proj}
\DeclareMathOperator{\Span}{span}
\DeclareMathOperator{\Sgn}{sgn}
\DeclareMathOperator{\Area}{Area}
\DeclareMathOperator{\Volume}{Volume}

%
% A few miscellaneous things specific to this document
%
\newcommand{\crossop}[1]{\crossprod{#1}{}}

% R2 vector.
\newcommand{\VectorTwo}[2]{
\begin{bmatrix}
 {#1} \\
 {#2}
\end{bmatrix}
}

\newcommand{\VectorN}[1]{
\begin{bmatrix}
{#1}_1 \\
{#1}_2 \\
\vdots \\
{#1}_N \\
\end{bmatrix}
}

\newcommand{\DETuvij}[4]{
\begin{vmatrix}
 {#1}_{#3} & {#1}_{#4} \\
 {#2}_{#3} & {#2}_{#4}
\end{vmatrix}
}

\newcommand{\DETuvwijk}[6]{
\begin{vmatrix}
 {#1}_{#4} & {#1}_{#5} & {#1}_{#6} \\
 {#2}_{#4} & {#2}_{#5} & {#2}_{#6} \\
 {#3}_{#4} & {#3}_{#5} & {#3}_{#6}
\end{vmatrix}
}

\newcommand{\DETuvwxijkl}[8]{
\begin{vmatrix}
 {#1}_{#5} & {#1}_{#6} & {#1}_{#7} & {#1}_{#8} \\
 {#2}_{#5} & {#2}_{#6} & {#2}_{#7} & {#2}_{#8} \\
 {#3}_{#5} & {#3}_{#6} & {#3}_{#7} & {#3}_{#8} \\
 {#4}_{#5} & {#4}_{#6} & {#4}_{#7} & {#4}_{#8} \\
\end{vmatrix}
}

%\newcommand{\DETuvwxyijklm}[10]{
%\begin{vmatrix}
% {#1}_{#6} & {#1}_{#7} & {#1}_{#8} & {#1}_{#9} & {#1}_{#10} \\
% {#2}_{#6} & {#2}_{#7} & {#2}_{#8} & {#2}_{#9} & {#2}_{#10} \\
% {#3}_{#6} & {#3}_{#7} & {#3}_{#8} & {#3}_{#9} & {#3}_{#10} \\
% {#4}_{#6} & {#4}_{#7} & {#4}_{#8} & {#4}_{#9} & {#4}_{#10} \\
% {#5}_{#6} & {#5}_{#7} & {#5}_{#8} & {#5}_{#9} & {#5}_{#10}
%\end{vmatrix}
%}

% R3 vector.
\newcommand{\VectorThree}[3]{
\begin{bmatrix}
 {#1} \\
 {#2} \\
 {#3}
\end{bmatrix}
}



\author{Peeter Joot}
\email{peeter.joot@gmail.com}


\chapter{PHY454H1S Continuum Mechanics.  Lecture 17: Impulsive flow.  Boundary layers.  Oscillatory driven flow.  Taught by Prof. K. Das.}
\label{chap:continuumL17}
%\useCCL
\blogpage{http://sites.google.com/site/peeterjoot2/math2012/continuumL17.pdf}
\date{Mar 16, 2012}
\gitRevisionInfo{continuumL17}

\beginArtWithToc
%\beginArtNoToc

\section{Disclaimer.}

Peeter's lecture notes from class.  May not be entirely coherent.

\section{Review.  Impulsively started flow.}

Were looking at flow driven by an impulse, a sudden motion of the plate, as in figure (\ref{fig:continuumL17:continuumL17Fig1})
\begin{figure}[htp]
   \centering
   \includegraphics[totalheight=0.3\textheight]{continuumL17Fig1}
   \caption{Impulsively driven time dependent fluid flow.}\label{fig:continuumL17:continuumL17Fig1}
\end{figure}

where the fluid at the origin is pushed so that it is given the velocity

\begin{equation}\label{eqn:continuumL17:20}
u(0, t) = 
\left\{
\begin{array}{l l}
0 & \quad \mbox{for $t < 0$} \\
U(t) & \quad \mbox{for $t \ge 0$} \\
\end{array}
\right.
\end{equation}

where $U \rightarrow 0$ as $y \rightarrow \infty$.

Navier-Stokes takes the form

\begin{equation}\label{eqn:continuumL17:40}
\PD{t}{u} = \nu \PDSq{y}{u}.
\end{equation}

With a similarity variable

\begin{equation}\label{eqn:continuumL17:60}
\eta = \frac{y}{2 \sqrt{\nu t}},
\end{equation}

and 

\begin{equation}\label{eqn:continuumL17:80}
u = U f(\eta),
\end{equation}

we found that we needed to solve

\begin{equation}\label{eqn:continuumL17:100}
f'' + 2 \eta f' = 0
\end{equation}

where 

\begin{equation}\label{eqn:continuumL17:120}
f' = \frac{df}{d\eta}
\end{equation}

with solution

\begin{equation}\label{eqn:continuumL17:140}
u(y, t) = U(1 - \erf(\eta)).
\end{equation}

Here, we've used the error function

\begin{equation}\label{eqn:continuumL17:160}
\erf(\eta) = \frac{2}{\sqrt{\pi}} \int_0^\eta e^{-s^2} ds,
\end{equation}

as plotted in figure (\ref{fig:continuumL17:continuumL17Fig2})
\begin{figure}[htp]
   \centering
   \includegraphics[totalheight=0.3\textheight]{continuumL17Fig2}
   \caption{Error function.}\label{fig:continuumL17:continuumL17Fig2}
\end{figure}

\section{Boundary layers.}

Let's look at spacetime points which are constant in $\eta$

\begin{equation}\label{eqn:continuumL17:180}
\frac{y_1}{2 \sqrt{\nu t_1}} = \frac{y_2}{2 \sqrt{\nu t_2}},
\end{equation}

so that the speed at $(y_1, t_1)$ equals the speed at $(y_2, t_2)$.  This is illustrated in figure (\ref{fig:continuumL17:continuumL17Fig3})
\begin{figure}[htp]
   \centering
   \def\svgwidth{0.6\columnwidth}
   \input{continuumL17Fig3.pdf_tex}
   \caption{Velocity profiles at different times.}\label{fig:continuumL17:continuumL17Fig3}
\end{figure}

\subsection{Universal behavior}

Looking at a plot with different viscosities for position vs time scaled as $\sqrt{\nu t}$ as in figure (\ref{fig:continuumL17:continuumL17Fig4}) we see a sort of universal behavior

\begin{figure}[htp]
   \centering
   \def\svgwidth{0.6\columnwidth}
   \input{continuumL17Fig4.pdf_tex}
   \caption{Universal behaviour.}\label{fig:continuumL17:continuumL17Fig4}
\end{figure}

Characterizing this we introduce the concept of boundary layer thickness

\begin{definition}
\emph{(Boundary layer thickness)}
\label{dfn:continuumL17:200}
The length scale over which viscosity is dominant.  This is the viscous length scale.
\end{definition}

This is similar to what we have in the heat equation 

\begin{equation}\label{eqn:continuumL17:220}
\PD{t}{T} = \kappa \PDSq{y}{T},
\end{equation}

where the time scale for the diffusion can be expressed as

\begin{equation}\label{eqn:continuumL17:240}
[\kappa_t] = \frac{d^2}{\kappa}.
\end{equation}

We could consider a scenario such as a heated plate in a cavity of height $\delta$ as in figure (\ref{fig:continuumL17:continuumL17Fig5})
\begin{figure}[htp]
   \centering
   \def\svgwidth{0.5\columnwidth}
   \input{continuumL17Fig5.pdf_tex}
   \caption{Characteristic distances in heat flow problems.}\label{fig:continuumL17:continuumL17Fig5}
\end{figure}

with a temperature $T$ on the bottom plate.  We can ask how fast the heat propagates through the medium.

\section{Another worked problem.}

Consider an oscillating plate, driving the motion of the fluid, as in figure (\ref{fig:continuumL17:continuumL17Fig6})
\begin{figure}[htp]
   \centering
   \includegraphics[totalheight=0.3\textheight]{continuumL17Fig6}
   \caption{Time dependent fluid motion due to oscillating plate.}\label{fig:continuumL17:continuumL17Fig6}
\end{figure}

\begin{equation}\label{eqn:continuumL17:260}
U(t) = U_0 \cos \Omega t = \Real\left( U_0 e^{i \Omega t} \right).
\end{equation}

(we are thinking here about the always oscillating case, and not an impulsive plate motion).

We write

\begin{equation}\label{eqn:continuumL17:280}
u(y, t) = \Real\left( f(y) e^{i \Omega t} \right)
\end{equation}

with substitution into 

\begin{equation}\label{eqn:continuumL17:300}
\PD{t}{u} = \nu \PDSq{y}{u},
\end{equation}

we have

\begin{equation}\label{eqn:continuumL17:320}
i \Omega f(y) e^{i \Omega t} = \nu f'' e^{i \Omega t}
\end{equation}

or

\begin{equation}\label{eqn:continuumL17:340}
i \Omega f(y) = \nu f'' 
\end{equation}

This is an equation of the form

\begin{equation}\label{eqn:continuumL17:360}
f'' = m^2 f
\end{equation}

where

\begin{equation}\label{eqn:continuumL17:380}
m^2 = \frac{i \Omega}{\nu}.
\end{equation}

or

\begin{equation}\label{eqn:continuumL17:400}
m = \sqrt{\frac{i \Omega}{\nu}} = \lambda (1 + i),
\end{equation}

where

\begin{equation}\label{eqn:continuumL17:420}
\lambda = \sqrt{\frac{\Omega}{2 \nu}}.
\end{equation}

check:

\begin{align*}
m^2 
&= 
\frac{\Omega}{2 \nu} (i + 1)^2 \\
&= 
\frac{\Omega}{2 \nu} (i^2 + 1 + 2 i) \\
&= 
\frac{\Omega}{\nu} i
\end{align*}

Considering the boundary value constraints we have

\begin{equation}\label{eqn:continuumL17:440}
f(y) = 
A e^{\lambda (1 + i) y}
+ B e^{-\lambda (1 + i) y}
\end{equation}

Since $u(\infty, t) \rightarrow 0$ we must have

\begin{equation}\label{eqn:continuumL17:460}
f(\infty) = 0,
\end{equation}

so we must kill off the exponentially increasing (albeit also oscillating) term by setting $A = 0$.  Also, since

\begin{equation}\label{eqn:continuumL17:480}
u(0, t) = U(t)
\end{equation}

we must have

\begin{equation}\label{eqn:continuumL17:500}
f(0) = U_0
\end{equation}

or 

\begin{equation}\label{eqn:continuumL17:520}
B = U_0
\end{equation}

so 

\begin{equation}\label{eqn:continuumL17:540}
f(y) = U_0 e^{-\lambda (1 + i) y}
\end{equation}

and

\begin{equation}\label{eqn:continuumL17:560}
u(y, t) = 
\Real\left(
U_0 e^{-\lambda y} e^{ -i (\lambda y - \Omega t) }
\right)
\end{equation}

or

\begin{equation}\label{eqn:continuumL17:580}
u(y, t) = 
U_0 e^{-\lambda y} \cos\left( -i (\lambda y - \Omega t) \right).
\end{equation}

This is a damped transverse wave function

\begin{equation}\label{eqn:continuumL17:600}
u(y, t) = f(y - c t),
\end{equation}

where 

\begin{equation}\label{eqn:continuumL17:620}
c = \frac{\Omega}{\lambda},
\end{equation}

is the wave speed.

Since we have an exponential damping here, the flow of fluid will essentially be confined to a boundary layer, where after distance $y = n/\lambda$, the oscillation falls off as

\begin{equation}\label{eqn:continuumL17:640}
\inv{e^n}.
\end{equation}

We can find a nice illustration of such a flow in \cite{wiki:StokesBoundary}.

\section{Flow over static object.}

Next time we'll start considering fluid flow around a fixed object as in figure (\ref{fig:continuumL17:continuumL17Fig7})

\begin{figure}[htp]
   \centering
   \includegraphics[totalheight=0.3\textheight]{continuumL17Fig7}
   \caption{Fluid flow around fixed object.}\label{fig:continuumL17:continuumL17Fig7}
\end{figure}

%FIXME: Reading: \S XX from \cite{acheson1990elementary}

\EndArticle
