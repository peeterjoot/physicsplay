%
% Copyright � 2015 Peeter Joot.  All Rights Reserved.
% Licenced as described in the file LICENSE under the root directory of this GIT repository.
%
\newcommand{\authorname}{Peeter Joot}
\newcommand{\email}{peeterjoot@protonmail.com}
\newcommand{\basename}{FIXMEbasenameUndefined}
\newcommand{\dirname}{notes/FIXMEdirnameUndefined/}

\renewcommand{\basename}{diracPlaneWaveDiagonalization}
\renewcommand{\dirname}{notes/phy1520/}
%\newcommand{\dateintitle}{}
%\newcommand{\keywords}{}

\newcommand{\authorname}{Peeter Joot}
\newcommand{\onlineurl}{http://sites.google.com/site/peeterjoot2/math2013/\basename.pdf}
\newcommand{\sourcepath}{\dirname\basename.tex}
\newcommand{\generatetitle}[1]{\chapter{#1}}

\newcommand{\vcsinfo}{%
\section*{}
\noindent{\color{DarkOliveGreen}{\rule{\linewidth}{0.1mm}}}
\paragraph{Document version}
%\paragraph{\color{Maroon}{Document version}}
{
\small
\begin{itemize}
\item Available online at:\\ 
\href{\onlineurl}{\onlineurl}
\item Git Repository: \input{./.revinfo/gitRepo.tex}
\item Source: \sourcepath
\item last commit: \input{./.revinfo/gitCommitString.tex}
\item commit date: \input{./.revinfo/gitCommitDate.tex}
\end{itemize}
}
}

%\PassOptionsToPackage{dvipsnames,svgnames}{xcolor}
\PassOptionsToPackage{square,numbers}{natbib}
\documentclass{scrreprt}

\usepackage[left=2cm,right=2cm]{geometry}
\usepackage[svgnames]{xcolor}
\usepackage{peeters_layout}

\usepackage{natbib}

\usepackage[
colorlinks=true,
bookmarks=false,
pdfauthor={\authorname, \email},
backref 
]{hyperref}

% http://tex.stackexchange.com/questions/75773/how-to-reference-problems-by-the-text-label-in-an-exercise-envioronment
\usepackage[english]{cleveref}
\crefname{Exercise}{exercise}{exercises}
\Crefname{Exercise}{Exercise}{Exercises}

\RequirePackage{titlesec}
\RequirePackage{ifthen}

% http://stackoverflow.com/questions/4932910/date-in-the-tabular-environment
\makeatletter
\let\insertdate\@date
\makeatother

\titleformat{\chapter}[display]
{\bfseries\Large}
{\color{DarkSlateGrey}\filleft \authorname
\ifthenelse{\isundefined{\studentnumber}}{}{\\ \studentnumber}
\ifthenelse{\isundefined{\email}}{}{\\ \email}
\ifthenelse{\isundefined{\dateintitle}}{}{\\ \insertdate}
%\ifthenelse{\isundefined{\coursename}}{}{\\ \coursename} % put in title instead.
}
{4ex}
{\color{DarkOliveGreen}{\titlerule}\color{Maroon}
\vspace{2ex}%
\filright}
[\vspace{2ex}%
\color{DarkOliveGreen}\titlerule
]

\newcommand{\beginArtWithToc}[0]{\begin{document}\tableofcontents}
\newcommand{\beginArtNoToc}[0]{\begin{document}}
\newcommand{\EndNoBibArticle}[0]{\end{document}}
\newcommand{\EndArticle}[0]{\bibliography{Bibliography}\bibliographystyle{plainnat}\end{document}}

% 
%\newcommand{\citep}[1]{\cite{#1}}

\colorSectionsForArticle



\usepackage{peeters_layout_exercise}
\usepackage{peeters_braket}
\usepackage{peeters_figures}
\usepackage{enumerate}

\beginArtNoToc

\generatetitle{Dirac Hamiltonian diagonalization for plane wave solution}
%\chapter{Dirac Hamiltonian diagonalization for plane wave solution}
%\label{chap:diracPlaneWaveDiagonalization}

We talked about diagonalization of the Dirac Hamiltonian by introducing a rotation, and then figuring out the rotation angle required.

To understand the form form of the eigenkets for particles and antiparticles in both forward and backwards moving configurations, lets do this diagonalization explicitly for both forwards and backwards solutions.

For the forward solution, given \( \Psi = \Psi_0 e^{i(k x - E t/\Hbar) } \), the Dirac equation is

\begin{dmath}\label{eqn:diracPlaneWaveDiagonalization:20}
i \Hbar (-i E/\Hbar) \Psi = 
\begin{bmatrix}
-i \Hbar (i k) + V_0 & m c^2 \\
m c^2 & i \Hbar (i k) 
\end{bmatrix},
\Psi
\end{dmath}

or
\begin{dmath}\label{eqn:diracPlaneWaveDiagonalization:40}
\begin{bmatrix}
E - V_0 & 0 \\
0 & E - V_0
\end{bmatrix}
\Psi
=
\begin{bmatrix}
\Hbar k & m c^2 \\
m c^2 &  - \Hbar k
\end{bmatrix}
\Psi.
\end{dmath}

Similarily, for the backwards moving wave \( \Psi = e^{i(-k x - E t/\Hbar)} \), we have

\begin{dmath}\label{eqn:diracPlaneWaveDiagonalization:60}
\begin{bmatrix}
E - V_0 & 0 \\
0 & E - V_0
\end{bmatrix}
\Psi
=
\begin{bmatrix}
-\Hbar k & m c^2 \\
m c^2 & \Hbar k
\end{bmatrix}
\Psi.
\end{dmath}

Working with \( \Hbar = c = 1 \) temporarily, we want to compute the eigensolutions for the matrix

\begin{dmath}\label{eqn:diracPlaneWaveDiagonalization:80}
H_{\pm k}
=
\begin{bmatrix}
\pm k & m \\
m & \mp k
\end{bmatrix}.
\end{dmath}

The eigenvalues \( \epsilon \) of both are the same

\begin{dmath}\label{eqn:diracPlaneWaveDiagonalization:100}
0 
= 
\Abs{ H_{\pm k} - \epsilon }
=
(\pm k - \epsilon)(\mp k - \epsilon) - m^2
=
(\mp k + \epsilon)(\pm k + \epsilon) - m^2
= 
\epsilon^2 - k^2 - m^2,
\end{dmath}

or

\begin{dmath}\label{eqn:diracPlaneWaveDiagonalization:120}
\epsilon = \pm \sqrt{k^2 + m^2}.
\end{dmath}

\paragraph{Eigenkets for \( H_k \)}

For the positive(negative) energy eigenvalues, we have

\begin{dmath}\label{eqn:diracPlaneWaveDiagonalization:140}
0 
=
\begin{bmatrix}
k \mp \epsilon & m \\
m & -k \mp \epsilon
\end{bmatrix}
\begin{bmatrix}
a \\
b
\end{bmatrix},
\end{dmath}

for some \( a, b\).  That is

\begin{dmath}\label{eqn:diracPlaneWaveDiagonalization:160}
(k \mp \epsilon) a + m b = 0,
\end{dmath}

or 

\begin{dmath}\label{eqn:diracPlaneWaveDiagonalization:180}
\begin{bmatrix}
a \\
b
\end{bmatrix}
\propto
\begin{bmatrix}
- m \\
k \mp \epsilon
\end{bmatrix}.
\end{dmath}

For the normalization note that

\begin{dmath}\label{eqn:diracPlaneWaveDiagonalization:200}
m^2 + \lr{ k \mp \epsilon }^2
=
m^2 + k^2 + \epsilon^2 \mp 2 k \epsilon
=
2 \epsilon^2 \mp 2 k \epsilon
= 
2 \epsilon (\epsilon \mp k),
\end{dmath}

so the normalized kets are

\begin{dmath}\label{eqn:diracPlaneWaveDiagonalization:220}
\ket{k; \pm\epsilon} = 
\inv{\sqrt{2 \epsilon(\epsilon \mp k)}}
\begin{bmatrix}
\pm m \\
\epsilon \mp k
\end{bmatrix}.
\end{dmath}

\paragraph{Eigenkets for \( H_{-k} \)}

This time, for the positive(negative) energy eigenvalues, we have

\begin{dmath}\label{eqn:diracPlaneWaveDiagonalization:141}
0 
=
\begin{bmatrix}
-k \mp \epsilon & m \\
m & k \mp \epsilon
\end{bmatrix}
\begin{bmatrix}
a \\
b
\end{bmatrix},
\end{dmath}

for some \( a, b\).  That is

\begin{dmath}\label{eqn:diracPlaneWaveDiagonalization:161}
(-k \mp \epsilon) a + m b = 0,
\end{dmath}

or 

\begin{dmath}\label{eqn:diracPlaneWaveDiagonalization:181}
\begin{bmatrix}
a \\
b
\end{bmatrix}
\propto
\begin{bmatrix}
- m \\
-k \mp \epsilon
\end{bmatrix}
\propto
\begin{bmatrix}
m \\
k \pm \epsilon
\end{bmatrix}.
\end{dmath}

For the normalization note that

\begin{dmath}\label{eqn:diracPlaneWaveDiagonalization:201}
m^2 + \lr{ k \pm \epsilon }^2
=
m^2 + k^2 + \epsilon^2 \pm 2 k \epsilon
=
2 \epsilon^2 \pm 2 k \epsilon
= 
2 \epsilon (\epsilon \pm k),
\end{dmath}

so the normalized kets are

\begin{dmath}\label{eqn:diracPlaneWaveDiagonalization:221}
\ket{-k; \pm\epsilon} = 
\inv{\sqrt{2 \epsilon(\epsilon \mp k)}}
\begin{bmatrix}
\pm m \\
\epsilon \pm k
\end{bmatrix}.
\end{dmath}

\paragraph{Simplification of the rotation matrices}

The eigenvalue equations have the form

\begin{dmath}\label{eqn:diracPlaneWaveDiagonalization:240}
H
\begin{bmatrix}
\ket{+} & \ket{-}
\end{bmatrix}
=
\begin{bmatrix}
\ket{+} & \ket{-}
\end{bmatrix}
\begin{bmatrix}
\epsilon & 0 \\
0 & -\epsilon
\end{bmatrix}
\end{dmath}

With \( R = \begin{bmatrix} \ket{+} & \ket{-} \end{bmatrix} \), this has the form \( H R = R \Omega \), or \( H = R \Omega R^{-1} \).  The rotation matrixes have been found to be

\begin{dmath}\label{eqn:diracPlaneWaveDiagonalization:260}
R_{+k} 
= 
\inv{\sqrt{2\epsilon}}
\begin{bmatrix}
\frac{m}{\sqrt{\epsilon - k}} & \frac{-m}{\sqrt{\epsilon + k}} \\
\sqrt{\epsilon - k} & \sqrt{\epsilon + k} 
\end{bmatrix},
\end{dmath}

and
\begin{dmath}\label{eqn:diracPlaneWaveDiagonalization:280}
R_{-k} 
= 
\inv{\sqrt{2\epsilon}}
\begin{bmatrix}
\frac{m}{\sqrt{\epsilon + k}} & \frac{-m}{\sqrt{\epsilon - k}} \\
\sqrt{\epsilon + k} & \sqrt{\epsilon - k} 
\end{bmatrix}.
\end{dmath}

These don't look very much like rotation matrixes as is, but not all the terms are independent.  Writing

\begin{dmath}\label{eqn:diracPlaneWaveDiagonalization:300}
\begin{aligned}
a &= \frac{m}{\sqrt{2\epsilon(\epsilon - k)}}  \\
b &= \frac{m}{\sqrt{2\epsilon(\epsilon + k)}} \\
\alpha &= \sqrt{\frac{\epsilon + k}{2 \epsilon}}  \\
\beta &= \sqrt{\frac{\epsilon - k}{2 \epsilon}} 
\end{aligned},
\end{dmath}

the rotation matrixes are

\begin{dmath}\label{eqn:diracPlaneWaveDiagonalization:320}
R_{+k} 
= 
\begin{bmatrix}
a & - b \\
\beta & \alpha
\end{bmatrix},
\end{dmath}

and
\begin{dmath}\label{eqn:diracPlaneWaveDiagonalization:340}
R_{-k} 
= 
\begin{bmatrix}
b & -a \\
\alpha & \beta
\end{bmatrix}.
\end{dmath}

Note that 
\begin{equation}\label{eqn:diracPlaneWaveDiagonalization:360}
\begin{aligned}
\frac{a}{\alpha} &= \frac{b}{\beta} \\
&= \frac{m}{\sqrt{\epsilon^2 - k^2}}  \\
&= \frac{m}{\sqrt{m^2}}  \\
&= 1.
\end{aligned}
\end{equation}

So
\begin{dmath}\label{eqn:diracPlaneWaveDiagonalization:380}
R_{+k} 
= 
\begin{bmatrix}
a & - b \\
b & a
\end{bmatrix},
\end{dmath}

and
\begin{dmath}\label{eqn:diracPlaneWaveDiagonalization:400}
R_{-k} 
= 
\begin{bmatrix}
b & -a \\
a & b 
\end{bmatrix}.
\end{dmath}

These are expected to have a unit determinant, which is verified easily

\begin{dmath}\label{eqn:diracPlaneWaveDiagonalization:420}
a^2 + b^2 
= 
\frac{m^2}{2 \epsilon}
\lr{
\frac{1}{\epsilon + k}
+
\frac{1}{\epsilon - k}
}
=
\frac{m^2}{2 \epsilon}
\frac{2 \epsilon}{\epsilon_2 - k^2}
= 1.
\end{dmath}

We see that both sets of matrices invert by transposition, so we are free to make a trigonometric identification

\begin{equation}\label{eqn:diracPlaneWaveDiagonalization:440}
a = \cos\theta 
= 
\frac{m}{\sqrt{2\epsilon(\epsilon - k)}} 
\end{equation}

\begin{equation}\label{eqn:diracPlaneWaveDiagonalization:460}
b = \sin\theta 
= 
\frac{m}{\sqrt{2\epsilon(\epsilon + k)}}.
\end{equation}

Using the double angle formulation used in class these are

\begin{dmath}\label{eqn:diracPlaneWaveDiagonalization:480}
\cos(2 \theta) 
= 
\frac{m^2}{2\epsilon(\epsilon - k)} -
\frac{m^2}{2\epsilon(\epsilon + k)} 
=
\frac{m^2}{2 \epsilon} \lr{ \inv{\epsilon - k} - \inv{\epsilon + k} }
=
\frac{2 k m^2}{2 \epsilon (\epsilon^2 - k^2)}
=
\frac{k}{\epsilon},
\end{dmath}

and
\begin{dmath}\label{eqn:diracPlaneWaveDiagonalization:500}
\sin(2 \theta) 
= 
2 \frac{m^2}{2 \epsilon} \inv{\sqrt{\epsilon^2 - k^2}}
= 
\frac{m}{\epsilon}.
\end{dmath}

Putting back in the \( \Hbar \), and \( c\) factors, the diagonaling transformation is now fully specified

\begin{equation}\label{eqn:diracPlaneWaveDiagonalization:520}
\begin{aligned}
H_{\pm k} &= 
\begin{bmatrix}
\pm \Hbar k c & m c^2 \\
m c^2 & \mp \Hbar k c
\end{bmatrix}
=
R_{\pm k} \Omega R_{\pm k}^\T \\
R_{+ k} &= 
\begin{bmatrix}
\cos\theta & -\sin\theta  \\
\sin\theta & \cos\theta
\end{bmatrix} 
=
\begin{bmatrix}
\ket{+k;+\epsilon} &
\ket{+k;-\epsilon}
\end{bmatrix}
\\
R_{- k} &= 
\begin{bmatrix}
\sin\theta & -\cos\theta  \\
\cos\theta & \sin\theta
\end{bmatrix} 
=
\begin{bmatrix}
\ket{-k;+\epsilon} &
\ket{-k;-\epsilon}
\end{bmatrix}
\\
\tan(2 \theta) &= \frac{m c}{\Hbar k} \\
\Omega &= 
\begin{bmatrix}
\epsilon & 0 \\
0 & -\epsilon
\end{bmatrix} \\
\epsilon &= \sqrt{(\Hbar k c)^2 + (m c^2)^2}.
\end{aligned}
\end{equation}

The functions \( \Psi = \ket{\pm k; \pm \epsilon} e^{\pm i k x - i E t/\Hbar} \) are eigenfunctions of the Hamiltonian.  For example, for a non-antiparticle forward moving state

\begin{dmath}\label{eqn:diracPlaneWaveDiagonalization:540}
(E - V_0) \ket{k; +\epsilon}
=
H_k \ket{k; +\epsilon}
=
\begin{bmatrix}
\ket{k;+\epsilon} &
\ket{k;-\epsilon}
\end{bmatrix}
\begin{bmatrix}
\epsilon & 0 \\
0 & -\epsilon
\end{bmatrix}
\begin{bmatrix}
\bra{k;+\epsilon} \\
\bra{k;-\epsilon}
\end{bmatrix}
\ket{k; \epsilon}
=
\begin{bmatrix}
\ket{k;+\epsilon} &
\ket{k;-\epsilon}
\end{bmatrix}
\begin{bmatrix}
\epsilon & 0 \\
0 & -\epsilon
\end{bmatrix}
\begin{bmatrix}
\braket{k;+\epsilon}{k; \epsilon} \\
\braket{k;-\epsilon}{k; \epsilon}
\end{bmatrix}
=
\begin{bmatrix}
\ket{k;+\epsilon} &
\ket{k;-\epsilon}
\end{bmatrix}
\begin{bmatrix}
\epsilon & 0 \\
0 & -\epsilon
\end{bmatrix}
\begin{bmatrix}
1 \\
0
\end{bmatrix}
=
\begin{bmatrix}
\ket{k;+\epsilon} &
\ket{k;-\epsilon}
\end{bmatrix}
\begin{bmatrix}
\epsilon \\
0 
\end{bmatrix}
=
\epsilon
\ket{k;+\epsilon}.
\end{dmath}

The total energy for this particle is

\begin{dmath}\label{eqn:diracPlaneWaveDiagonalization:560}
E = V_0 + \sqrt{(\Hbar k c)^2 + (m c^2)^2}.
\end{dmath}

This can be plotted nicely, as \( \frac{E}{m c^2} \) vs. \( \frac{\Hbar k}{m c} \) if non-dimensionalized as

\begin{dmath}\label{eqn:diracPlaneWaveDiagonalization:580}
\frac{E}{m c^2} = \frac{V_0}{m c^2} + \sqrt{1 + \frac{(\Hbar k)^2}{(m c)^2} }.
\end{dmath}

For the step potential we have 

\begin{dmath}\label{eqn:diracPlaneWaveDiagonalization:600}
\frac{E}{m c^2} 
= \sqrt{ 1 + {\frac{ \Hbar k_1}{m c} }^2 } 
= \pm \sqrt{ 1  {\frac{ \Hbar k_2}{m c} }^2 } + \frac{V_0}{m c^2},
\end{dmath}

or
\begin{dmath}\label{eqn:diracPlaneWaveDiagonalization:620}
\frac{ \Hbar k_2}{m c}
=
\lr{ \lr{ \pm \sqrt{ 1 + {\frac{ \Hbar k_1}{m c} }^2 } - \frac{V_0}{m c^2} }^2 - 1 }^{1/2}.
\end{dmath}

When this is real valued, there is transmission in the barrier region.  There are a few cases of interest, plotted in 
\cref{fig:ps4DiracStepPotential:ps4DiracStepPotentialFig1},
\cref{fig:ps4DiracStepPotential:ps4DiracStepPotentialFig2},
\cref{fig:ps4DiracStepPotential:ps4DiracStepPotentialFig3},
\cref{fig:ps4DiracStepPotential:ps4DiracStepPotentialFig4}, and
\cref{fig:ps4DiracStepPotential:ps4DiracStepPotentialFig5}.


\imageFigure{../../figures/phy1520/ps4DiracStepPotentialFig1}{\( V_0 < 2 m c^2 \), with momentum small enough for only decaying transmission.}{fig:ps4DiracStepPotential:ps4DiracStepPotentialFig1}{0.3}

\imageFigure{../../figures/phy1520/ps4DiracStepPotentialFig2}{\( V_0 < 2 m c^2 \), with high enough momentum to have transmision.}{fig:ps4DiracStepPotential:ps4DiracStepPotentialFig2}{0.3}

\imageFigure{../../figures/phy1520/ps4DiracStepPotentialFig3}{\( V_0 > 2 m c^2 \), with momentum low enough that there is anti-particle transmission.}{fig:ps4DiracStepPotential:ps4DiracStepPotentialFig3}{0.3}

\imageFigure{../../figures/phy1520/ps4DiracStepPotentialFig4}{\( V_0 > 2 m c^2 \), with momentum in a decaying transmission domain.}{fig:ps4DiracStepPotential:ps4DiracStepPotentialFig4}{0.3}

\imageFigure{../../figures/phy1520/ps4DiracStepPotentialFig5}{\( V_0 > 2 m c^2 \), with momentum high enough for transmission.}{fig:ps4DiracStepPotential:ps4DiracStepPotentialFig5}{0.3}

We want to find the wave incident, reflected, and transmitted wave function.  From the diagonalization above, these are

\begin{dmath}\label{eqn:diracPlaneWaveDiagonalization:640}
\Psi_{\textrm{inc}} = 
A
\begin{bmatrix}
\cos\theta_{k_1} \\
\sin\theta_{k_1} \\
\end{bmatrix}
e^{i (k_1 x - E t/\Hbar)}
\end{dmath}
\begin{dmath}\label{eqn:diracPlaneWaveDiagonalization:660}
\Psi_{\textrm{ref}} = 
B
\begin{bmatrix}
\sin\theta_{k_1} \\
\cos\theta_{k_1} \\
\end{bmatrix}
e^{i (-k_1 x - E t/\Hbar)}
\end{dmath}

When there is a transmitted particle (anti-particle) in region II, the transmitted wave function are respectively

\begin{dmath}\label{eqn:diracPlaneWaveDiagonalization:680}
\Psi_{\textrm{ref}} = 
D
\begin{bmatrix}
\cos\theta_{k_2} \\
\sin\theta_{k_2} \\
\end{bmatrix}
e^{i (k_2 x - E t/\Hbar)}
\end{dmath}
\begin{dmath}\label{eqn:diracPlaneWaveDiagonalization:700}
\Psi_{\textrm{ref}} = 
C
\begin{bmatrix}
-\sin\theta_{k_2} \\
\cos\theta_{k_2} \\
\end{bmatrix}
e^{i (k_2 x - E t/\Hbar)}.
\end{dmath}

There are a few cases to consider

\begin{enumerate}[(i)]
\item Total reflection.  Are there any special values of momentum that allow for total reflection?  If so, at the boundary both components must be zero

\begin{dmath}\label{eqn:diracPlaneWaveDiagonalization:720}
\begin{bmatrix}
\cos\theta_{k_1} \\
\sin\theta_{k_1} \\
\end{bmatrix}
+
\frac{B}{A}
\begin{bmatrix}
\sin\theta_{k_1} \\
\cos\theta_{k_1} \\
\end{bmatrix}
=
\begin{bmatrix}
0 \\
0
\end{bmatrix}
\end{dmath}

This is possible if \( -B/A = \cot\theta_{k_1} = \tan\theta_{k_1} \), which requires

\begin{equation}\label{eqn:diracPlaneWaveDiagonalization:740}
\theta_{k_1} = \frac{\pi}{4} \lr{ 1 + 2 n }, \qquad n \in \bbZ.
\end{equation}

However, we must also have

\begin{dmath}\label{eqn:diracPlaneWaveDiagonalization:760}
\tan 2 \theta_1 = \frac{m c}{\Hbar k},
\end{dmath}

so

\begin{dmath}\label{eqn:diracPlaneWaveDiagonalization:780}
\theta_1 = \inv{2} \Atan\lr{ \frac{m c}{\Hbar k} }.
\end{dmath}

Simultaneous solutions of \cref{eqn:diracPlaneWaveDiagonalization:740}, \cref{eqn:diracPlaneWaveDiagonalization:780} only occur at \( k = \pm 0 \), where \( \tan( (1 + 2 n)\pi/2 ) = \pm \infty \).  Since a particle at rest is not at interest in a reflection scenerio, this shows that a decaying solution in region II must be introduced to match the boundary value constraints.

\item Ordinary matter transmission.
\item Anti-matter transmission.
\item Decaying transmission.
\end{enumerate}

%\EndArticle
\EndNoBibArticle
