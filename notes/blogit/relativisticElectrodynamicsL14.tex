%
% Copyright � 2015 Peeter Joot.  All Rights Reserved.
% Licenced as described in the file LICENSE under the root directory of this GIT repository.
%
\documentclass[]{eliblog}

\usepackage{amsmath}
\usepackage{mathpazo}

%
% shorthand for bold symbols, convenient for vectors and matrices
%
\newcommand{\Ba}[0]{\mathbf{a}}
\newcommand{\Bb}[0]{\mathbf{b}}
\newcommand{\Bc}[0]{\mathbf{c}}
\newcommand{\Bd}[0]{\mathbf{d}}
\newcommand{\Be}[0]{\mathbf{e}}
\newcommand{\Bf}[0]{\mathbf{f}}
\newcommand{\Bg}[0]{\mathbf{g}}
\newcommand{\Bh}[0]{\mathbf{h}}
\newcommand{\Bi}[0]{\mathbf{i}}
\newcommand{\Bj}[0]{\mathbf{j}}
\newcommand{\Bk}[0]{\mathbf{k}}
\newcommand{\Bl}[0]{\mathbf{l}}
\newcommand{\Bm}[0]{\mathbf{m}}
\newcommand{\Bn}[0]{\mathbf{n}}
\newcommand{\Bo}[0]{\mathbf{o}}
\newcommand{\Bp}[0]{\mathbf{p}}
\newcommand{\Bq}[0]{\mathbf{q}}
\newcommand{\Br}[0]{\mathbf{r}}
\newcommand{\Bs}[0]{\mathbf{s}}
\newcommand{\Bt}[0]{\mathbf{t}}
\newcommand{\Bu}[0]{\mathbf{u}}
\newcommand{\Bv}[0]{\mathbf{v}}
\newcommand{\Bw}[0]{\mathbf{w}}
\newcommand{\Bx}[0]{\mathbf{x}}
\newcommand{\By}[0]{\mathbf{y}}
\newcommand{\Bz}[0]{\mathbf{z}}
\newcommand{\BA}[0]{\mathbf{A}}
\newcommand{\BB}[0]{\mathbf{B}}
\newcommand{\BC}[0]{\mathbf{C}}
\newcommand{\BD}[0]{\mathbf{D}}
\newcommand{\BE}[0]{\mathbf{E}}
\newcommand{\BF}[0]{\mathbf{F}}
\newcommand{\BG}[0]{\mathbf{G}}
\newcommand{\BH}[0]{\mathbf{H}}
\newcommand{\BI}[0]{\mathbf{I}}
\newcommand{\BJ}[0]{\mathbf{J}}
\newcommand{\BK}[0]{\mathbf{K}}
\newcommand{\BL}[0]{\mathbf{L}}
\newcommand{\BM}[0]{\mathbf{M}}
\newcommand{\BN}[0]{\mathbf{N}}
\newcommand{\BO}[0]{\mathbf{O}}
\newcommand{\BP}[0]{\mathbf{P}}
\newcommand{\BQ}[0]{\mathbf{Q}}
\newcommand{\BR}[0]{\mathbf{R}}
\newcommand{\BS}[0]{\mathbf{S}}
\newcommand{\BT}[0]{\mathbf{T}}
\newcommand{\BU}[0]{\mathbf{U}}
\newcommand{\BV}[0]{\mathbf{V}}
\newcommand{\BW}[0]{\mathbf{W}}
\newcommand{\BX}[0]{\mathbf{X}}
\newcommand{\BY}[0]{\mathbf{Y}}
\newcommand{\BZ}[0]{\mathbf{Z}}

\newcommand{\Bzero}[0]{\mathbf{0}}
\newcommand{\Btheta}[0]{\boldsymbol{\theta}}
\newcommand{\Btau}[0]{\boldsymbol{\tau}}
\newcommand{\Bomega}[0]{\boldsymbol{\omega}}

%
% shorthand for unit vectors
%
\newcommand{\acap}[0]{\hat{\Ba}}
\newcommand{\bcap}[0]{\hat{\Bb}}
\newcommand{\ccap}[0]{\hat{\Bc}}
\newcommand{\dcap}[0]{\hat{\Bd}}
\newcommand{\ecap}[0]{\hat{\Be}}
\newcommand{\fcap}[0]{\hat{\Bf}}
\newcommand{\gcap}[0]{\hat{\Bg}}
\newcommand{\hcap}[0]{\hat{\Bh}}
\newcommand{\icap}[0]{\hat{\Bi}}
\newcommand{\jcap}[0]{\hat{\Bj}}
\newcommand{\kcap}[0]{\hat{\Bk}}
\newcommand{\lcap}[0]{\hat{\Bl}}
\newcommand{\mcap}[0]{\hat{\Bm}}
\newcommand{\ncap}[0]{\hat{\Bn}}
\newcommand{\ocap}[0]{\hat{\Bo}}
\newcommand{\pcap}[0]{\hat{\Bp}}
\newcommand{\qcap}[0]{\hat{\Bq}}
\newcommand{\rcap}[0]{\hat{\Br}}
\newcommand{\scap}[0]{\hat{\Bs}}
\newcommand{\tcap}[0]{\hat{\Bt}}
\newcommand{\ucap}[0]{\hat{\Bu}}
\newcommand{\vcap}[0]{\hat{\Bv}}
\newcommand{\wcap}[0]{\hat{\Bw}}
\newcommand{\xcap}[0]{\hat{\Bx}}
\newcommand{\ycap}[0]{\hat{\By}}
\newcommand{\zcap}[0]{\hat{\Bz}}
\newcommand{\thetacap}[0]{\hat{\Btheta}}

%
% to write R^n and C^n in a distinguishable fashion.  Perhaps change this
% to the double lined characters upon figuring out how to do so.
%
\newcommand{\C}[1]{$\mathbb{C}^{#1}$}
\newcommand{\R}[1]{$\mathbb{R}^{#1}$}

%
% various generally useful helpers
%

% derivative of #1 wrt. #2:
\newcommand{\D}[2] {\frac {d#2} {d#1}}

\newcommand{\inv}[1]{\frac{1}{#1}}
\newcommand{\cross}[0]{\times}

\newcommand{\abs}[1]{\lvert{#1}\rvert}
\newcommand{\norm}[1]{\lVert{#1}\rVert}
\newcommand{\innerprod}[2]{\langle{#1}, {#2}\rangle}
\newcommand{\dotprod}[2]{{#1} \cdot {#2}}
\newcommand{\bdotprod}[2]{\left({#1} \cdot {#2}\right)}
\newcommand{\crossprod}[2]{{#1} \cross {#2}}
\newcommand{\tripleprod}[3]{\dotprod{\left(\crossprod{#1}{#2}\right)}{#3}}

\DeclareMathOperator{\Proj}{Proj}
\DeclareMathOperator{\Span}{span}
\DeclareMathOperator{\Sgn}{sgn}
\DeclareMathOperator{\Area}{Area}
\DeclareMathOperator{\Volume}{Volume}

%
% A few miscellaneous things specific to this document
%
\newcommand{\crossop}[1]{\crossprod{#1}{}}

% R2 vector.
\newcommand{\VectorTwo}[2]{
\begin{bmatrix}
 {#1} \\
 {#2}
\end{bmatrix}
}

\newcommand{\VectorN}[1]{
\begin{bmatrix}
{#1}_1 \\
{#1}_2 \\
\vdots \\
{#1}_N \\
\end{bmatrix}
}

\newcommand{\DETuvij}[4]{
\begin{vmatrix}
 {#1}_{#3} & {#1}_{#4} \\
 {#2}_{#3} & {#2}_{#4}
\end{vmatrix}
}

\newcommand{\DETuvwijk}[6]{
\begin{vmatrix}
 {#1}_{#4} & {#1}_{#5} & {#1}_{#6} \\
 {#2}_{#4} & {#2}_{#5} & {#2}_{#6} \\
 {#3}_{#4} & {#3}_{#5} & {#3}_{#6}
\end{vmatrix}
}

\newcommand{\DETuvwxijkl}[8]{
\begin{vmatrix}
 {#1}_{#5} & {#1}_{#6} & {#1}_{#7} & {#1}_{#8} \\
 {#2}_{#5} & {#2}_{#6} & {#2}_{#7} & {#2}_{#8} \\
 {#3}_{#5} & {#3}_{#6} & {#3}_{#7} & {#3}_{#8} \\
 {#4}_{#5} & {#4}_{#6} & {#4}_{#7} & {#4}_{#8} \\
\end{vmatrix}
}

%\newcommand{\DETuvwxyijklm}[10]{
%\begin{vmatrix}
% {#1}_{#6} & {#1}_{#7} & {#1}_{#8} & {#1}_{#9} & {#1}_{#10} \\
% {#2}_{#6} & {#2}_{#7} & {#2}_{#8} & {#2}_{#9} & {#2}_{#10} \\
% {#3}_{#6} & {#3}_{#7} & {#3}_{#8} & {#3}_{#9} & {#3}_{#10} \\
% {#4}_{#6} & {#4}_{#7} & {#4}_{#8} & {#4}_{#9} & {#4}_{#10} \\
% {#5}_{#6} & {#5}_{#7} & {#5}_{#8} & {#5}_{#9} & {#5}_{#10}
%\end{vmatrix}
%}

% R3 vector.
\newcommand{\VectorThree}[3]{
\begin{bmatrix}
 {#1} \\
 {#2} \\
 {#3}
\end{bmatrix}
}



\author{Peeter Joot}
\email{peeter.joot@gmail.com}

%\documentclass[]{eliblogwidescreen}

\usepackage{amsmath}
\usepackage{mathpazo}

%
% shorthand for bold symbols, convenient for vectors and matrices
%
\newcommand{\Ba}[0]{\mathbf{a}}
\newcommand{\Bb}[0]{\mathbf{b}}
\newcommand{\Bc}[0]{\mathbf{c}}
\newcommand{\Bd}[0]{\mathbf{d}}
\newcommand{\Be}[0]{\mathbf{e}}
\newcommand{\Bf}[0]{\mathbf{f}}
\newcommand{\Bg}[0]{\mathbf{g}}
\newcommand{\Bh}[0]{\mathbf{h}}
\newcommand{\Bi}[0]{\mathbf{i}}
\newcommand{\Bj}[0]{\mathbf{j}}
\newcommand{\Bk}[0]{\mathbf{k}}
\newcommand{\Bl}[0]{\mathbf{l}}
\newcommand{\Bm}[0]{\mathbf{m}}
\newcommand{\Bn}[0]{\mathbf{n}}
\newcommand{\Bo}[0]{\mathbf{o}}
\newcommand{\Bp}[0]{\mathbf{p}}
\newcommand{\Bq}[0]{\mathbf{q}}
\newcommand{\Br}[0]{\mathbf{r}}
\newcommand{\Bs}[0]{\mathbf{s}}
\newcommand{\Bt}[0]{\mathbf{t}}
\newcommand{\Bu}[0]{\mathbf{u}}
\newcommand{\Bv}[0]{\mathbf{v}}
\newcommand{\Bw}[0]{\mathbf{w}}
\newcommand{\Bx}[0]{\mathbf{x}}
\newcommand{\By}[0]{\mathbf{y}}
\newcommand{\Bz}[0]{\mathbf{z}}
\newcommand{\BA}[0]{\mathbf{A}}
\newcommand{\BB}[0]{\mathbf{B}}
\newcommand{\BC}[0]{\mathbf{C}}
\newcommand{\BD}[0]{\mathbf{D}}
\newcommand{\BE}[0]{\mathbf{E}}
\newcommand{\BF}[0]{\mathbf{F}}
\newcommand{\BG}[0]{\mathbf{G}}
\newcommand{\BH}[0]{\mathbf{H}}
\newcommand{\BI}[0]{\mathbf{I}}
\newcommand{\BJ}[0]{\mathbf{J}}
\newcommand{\BK}[0]{\mathbf{K}}
\newcommand{\BL}[0]{\mathbf{L}}
\newcommand{\BM}[0]{\mathbf{M}}
\newcommand{\BN}[0]{\mathbf{N}}
\newcommand{\BO}[0]{\mathbf{O}}
\newcommand{\BP}[0]{\mathbf{P}}
\newcommand{\BQ}[0]{\mathbf{Q}}
\newcommand{\BR}[0]{\mathbf{R}}
\newcommand{\BS}[0]{\mathbf{S}}
\newcommand{\BT}[0]{\mathbf{T}}
\newcommand{\BU}[0]{\mathbf{U}}
\newcommand{\BV}[0]{\mathbf{V}}
\newcommand{\BW}[0]{\mathbf{W}}
\newcommand{\BX}[0]{\mathbf{X}}
\newcommand{\BY}[0]{\mathbf{Y}}
\newcommand{\BZ}[0]{\mathbf{Z}}

\newcommand{\Bzero}[0]{\mathbf{0}}
\newcommand{\Btheta}[0]{\boldsymbol{\theta}}
\newcommand{\Btau}[0]{\boldsymbol{\tau}}
\newcommand{\Bomega}[0]{\boldsymbol{\omega}}

%
% shorthand for unit vectors
%
\newcommand{\acap}[0]{\hat{\Ba}}
\newcommand{\bcap}[0]{\hat{\Bb}}
\newcommand{\ccap}[0]{\hat{\Bc}}
\newcommand{\dcap}[0]{\hat{\Bd}}
\newcommand{\ecap}[0]{\hat{\Be}}
\newcommand{\fcap}[0]{\hat{\Bf}}
\newcommand{\gcap}[0]{\hat{\Bg}}
\newcommand{\hcap}[0]{\hat{\Bh}}
\newcommand{\icap}[0]{\hat{\Bi}}
\newcommand{\jcap}[0]{\hat{\Bj}}
\newcommand{\kcap}[0]{\hat{\Bk}}
\newcommand{\lcap}[0]{\hat{\Bl}}
\newcommand{\mcap}[0]{\hat{\Bm}}
\newcommand{\ncap}[0]{\hat{\Bn}}
\newcommand{\ocap}[0]{\hat{\Bo}}
\newcommand{\pcap}[0]{\hat{\Bp}}
\newcommand{\qcap}[0]{\hat{\Bq}}
\newcommand{\rcap}[0]{\hat{\Br}}
\newcommand{\scap}[0]{\hat{\Bs}}
\newcommand{\tcap}[0]{\hat{\Bt}}
\newcommand{\ucap}[0]{\hat{\Bu}}
\newcommand{\vcap}[0]{\hat{\Bv}}
\newcommand{\wcap}[0]{\hat{\Bw}}
\newcommand{\xcap}[0]{\hat{\Bx}}
\newcommand{\ycap}[0]{\hat{\By}}
\newcommand{\zcap}[0]{\hat{\Bz}}
\newcommand{\thetacap}[0]{\hat{\Btheta}}

%
% to write R^n and C^n in a distinguishable fashion.  Perhaps change this
% to the double lined characters upon figuring out how to do so.
%
\newcommand{\C}[1]{$\mathbb{C}^{#1}$}
\newcommand{\R}[1]{$\mathbb{R}^{#1}$}

%
% various generally useful helpers
%

% derivative of #1 wrt. #2:
\newcommand{\D}[2] {\frac {d#2} {d#1}}

\newcommand{\inv}[1]{\frac{1}{#1}}
\newcommand{\cross}[0]{\times}

\newcommand{\abs}[1]{\lvert{#1}\rvert}
\newcommand{\norm}[1]{\lVert{#1}\rVert}
\newcommand{\innerprod}[2]{\langle{#1}, {#2}\rangle}
\newcommand{\dotprod}[2]{{#1} \cdot {#2}}
\newcommand{\bdotprod}[2]{\left({#1} \cdot {#2}\right)}
\newcommand{\crossprod}[2]{{#1} \cross {#2}}
\newcommand{\tripleprod}[3]{\dotprod{\left(\crossprod{#1}{#2}\right)}{#3}}

\DeclareMathOperator{\Proj}{Proj}
\DeclareMathOperator{\Span}{span}
\DeclareMathOperator{\Sgn}{sgn}
\DeclareMathOperator{\Area}{Area}
\DeclareMathOperator{\Volume}{Volume}

%
% A few miscellaneous things specific to this document
%
\newcommand{\crossop}[1]{\crossprod{#1}{}}

% R2 vector.
\newcommand{\VectorTwo}[2]{
\begin{bmatrix}
 {#1} \\
 {#2}
\end{bmatrix}
}

\newcommand{\VectorN}[1]{
\begin{bmatrix}
{#1}_1 \\
{#1}_2 \\
\vdots \\
{#1}_N \\
\end{bmatrix}
}

\newcommand{\DETuvij}[4]{
\begin{vmatrix}
 {#1}_{#3} & {#1}_{#4} \\
 {#2}_{#3} & {#2}_{#4}
\end{vmatrix}
}

\newcommand{\DETuvwijk}[6]{
\begin{vmatrix}
 {#1}_{#4} & {#1}_{#5} & {#1}_{#6} \\
 {#2}_{#4} & {#2}_{#5} & {#2}_{#6} \\
 {#3}_{#4} & {#3}_{#5} & {#3}_{#6}
\end{vmatrix}
}

\newcommand{\DETuvwxijkl}[8]{
\begin{vmatrix}
 {#1}_{#5} & {#1}_{#6} & {#1}_{#7} & {#1}_{#8} \\
 {#2}_{#5} & {#2}_{#6} & {#2}_{#7} & {#2}_{#8} \\
 {#3}_{#5} & {#3}_{#6} & {#3}_{#7} & {#3}_{#8} \\
 {#4}_{#5} & {#4}_{#6} & {#4}_{#7} & {#4}_{#8} \\
\end{vmatrix}
}

%\newcommand{\DETuvwxyijklm}[10]{
%\begin{vmatrix}
% {#1}_{#6} & {#1}_{#7} & {#1}_{#8} & {#1}_{#9} & {#1}_{#10} \\
% {#2}_{#6} & {#2}_{#7} & {#2}_{#8} & {#2}_{#9} & {#2}_{#10} \\
% {#3}_{#6} & {#3}_{#7} & {#3}_{#8} & {#3}_{#9} & {#3}_{#10} \\
% {#4}_{#6} & {#4}_{#7} & {#4}_{#8} & {#4}_{#9} & {#4}_{#10} \\
% {#5}_{#6} & {#5}_{#7} & {#5}_{#8} & {#5}_{#9} & {#5}_{#10}
%\end{vmatrix}
%}

% R3 vector.
\newcommand{\VectorThree}[3]{
\begin{bmatrix}
 {#1} \\
 {#2} \\
 {#3}
\end{bmatrix}
}



\author{Peeter Joot}
\email{peeter.joot@gmail.com}


\chapter{PHY450H1S.  Relativistic Electrodynamics Lecture 14 (Taught by Simon Freedman).  Wave equation in Coulomb and Lorentz gauges.}
\label{chap:relativisticElectrodynamicsL14}
%\useCCL
\blogpage{http://sites.google.com/site/peeterjoot/math2011/relativisticElectrodynamicsL14.pdf}
\date{Feb 16, 2011}
\revisionInfo{relativisticElectrodynamicsL14.tex}

%\beginArtWithToc
\beginArtNoToc

\section{Reading.}

Covering chapter 4 material from the text \cite{landau1980classical}.

Covering \href{http://www.physics.utoronto.ca/~poppitz/e-poppitz/PHY450_files/RelEMpp103-114.pdf}{lecture notes pp.103-114}: the wave equation in the relativistic Lorentz gauge (114-114) [Tuesday, Feb. 15; Wednesday, Feb.16]...

Covering \href{http://www.physics.utoronto.ca/~poppitz/e-poppitz/PHY450_files/RelEMpp114-127.pdf}{lecture notes pp. 114-127}: reminder on wave equations (114); reminder on Fourier series and integral (115-117); Fourier expansion of the EM potential in Coulomb gauge and equation of motion for the spatial Fourier components (118-119); the general solution of Maxwell's equations in vacuum (120-121) [Tuesday, Mar. 1]; properties of monochromatic plane EM waves (122-124); energy and energy flux of the EM field and energy conservation from the equations of motion (125-127)  [Wednesday, Mar. 2]

\section{Trying to understand ``c''}

\begin{align}\label{eqn:relativisticElectrodynamicsL14:10}
\spacegrad \cdot \BE &= 0 \\
\spacegrad \cross \BB &= \inv{c} \PD{t}{\BE}
\end{align}

Maxwell's equations in a vacuum were

\begin{align}\label{eqn:relativisticElectrodynamicsL14:30}
\spacegrad (\spacegrad \cdot \BA) &= \spacegrad^2 \BA  -\inv{c} \PD{t}{} \spacegrad \phi - \inv{c^2} \frac{\partial^2 \BA}{\partial t^2} \\
\spacegrad \cdot \BE &= - \spacegrad^2 \phi - \inv{c} \PD{t}{\spacegrad \cdot \BA} 
\end{align}

There's a redundancy here since we can change $\phi$ and $\BA$ without changing the EOM

\begin{equation}\label{eqn:relativisticElectrodynamicsL14:50}
(\phi, \BA) \rightarrow (\phi', \BA')
\end{equation}

with

\begin{align}\label{eqn:relativisticElectrodynamicsL14:70}
\phi &= \phi' + \inv{c} \PD{t}{\chi} \\
\BA &= \BA' - \spacegrad \chi
\end{align}

\begin{equation}\label{eqn:relativisticElectrodynamicsL14:90}
\chi(\Bx, t) = c \int dt \phi(\Bx, t)
\end{equation}

which gives 

\begin{equation}\label{eqn:relativisticElectrodynamicsL14:110}
\phi' = 0
\end{equation}

\begin{align}\label{eqn:relativisticElectrodynamicsL14:130}
(\phi, \BA) \sim (\phi = 0, \BA')
\end{align}

Maxwell's equations are now

\begin{align*}
\spacegrad (\spacegrad \cdot \BA') &= \spacegrad^2 \BA'  - \inv{c^2} \frac{\partial^2 \BA'}{\partial t^2} \\
\PD{t}{\spacegrad \cdot \BA'}  &= 0
\end{align*}

Can we make $\spacegrad \cdot \BA'' = 0$, while $\phi'' = 0$.

\begin{align}\label{eqn:relativisticElectrodynamicsL14:150}
\underbrace{\phi}_{=0} &= \underbrace{\phi'}_{=0} + \inv{c} \PD{t}{\chi'} \\
\end{align}

We need 
\begin{equation}\label{eqn:relativisticElectrodynamicsL14:170}
\PD{t}{\chi'} = 0
\end{equation}

How about $\BA'$

\begin{equation}\label{eqn:relativisticElectrodynamicsL14:190}
\BA' = \BA'' - \spacegrad \chi'
\end{equation}

We want the divergence of $\BA'$ to be zero, which means

\begin{equation}\label{eqn:relativisticElectrodynamicsL14:210}
\spacegrad \cdot \BA' = \underbrace{\spacegrad \cdot \BA''}_{=0} - \spacegrad^2 \chi'
\end{equation}

So we want

\begin{equation}\label{eqn:relativisticElectrodynamicsL14:230}
\spacegrad^2 \chi' = \spacegrad \cdot \BA'
\end{equation}

Can we solve this?

Recall that in electrostatics we have

\begin{equation}\label{eqn:relativisticElectrodynamicsL14:250}
\spacegrad \cdot \BE = 4 \pi \rho
\end{equation}

and 

\begin{equation}\label{eqn:relativisticElectrodynamicsL14:270}
\BE = -\spacegrad \phi
\end{equation}

which meant that we had 

\begin{equation}\label{eqn:relativisticElectrodynamicsL14:290}
\spacegrad^2 \phi = 4 \pi \rho
\end{equation}

This has the identical form (with $\phi \sim \chi$, and $4 \pi \rho \sim \spacegrad \cdot \BA'$).

While we aren't trying to actually solve this (just show that it can be solved).  One way to look at this problem is that it is just a Laplace equation, and we could utilize a Green's function solution if desired.

FIXME: What's the Green's function?

What are the Maxwell's vacuum equations now?

With the second gauge substitution we have

\begin{align*}
\spacegrad (\spacegrad \cdot \BA'') &= \spacegrad^2 \BA''  - \inv{c^2} \frac{\partial^2 \BA''}{\partial t^2} \\
\PD{t}{\spacegrad \cdot \BA''}  &= 0
\end{align*}

but we can utilize

\begin{equation}\label{eqn:relativisticElectrodynamicsL14:310}
\spacegrad \cross (\spacegrad \cross \BA) = \spacegrad (\spacegrad \cdot \BA) - \spacegrad^2 \BA,
\end{equation}

to reduce Maxwell's equations (after dropping primes) to just

\begin{equation}\label{eqn:relativisticElectrodynamicsL14:330}
\inv{c^2} \frac{\partial^2 \BA''}{\partial t^2} - \Delta \BA = 0
\end{equation}

where 
\begin{equation}\label{eqn:relativisticElectrodynamicsL14:350}
\Delta = \spacegrad^2 = \spacegrad \cdot \spacegrad = 
\frac{\partial^2}{\partial x^2}
+\frac{\partial^2}{\partial y^2}
+\frac{\partial^2}{\partial y^2}
\end{equation}

Note that for this to be correct we have to also explicitly include the gauge condition used.  This particular gauge is called the \underline{Coulomb gauge}.

\begin{align}\label{eqn:relativisticElectrodynamicsL14:370}
\phi &= 0 \\
\spacegrad \cdot \BA'' &= 0 
\end{align}

\section{Claim: EM waves propagate with speed $c$ and are transverse.}

\paragraph{Note:} Is the Coulomb gauge Lorentz invariant?
\paragraph{No.} We can boost which will introduce a non-zero $\phi$.

The gauge that is Lorentz Invariant is the ``Lorentz gauge''.  This one uses

\begin{equation}\label{eqn:relativisticElectrodynamicsL14:390}
\partial_i A^i = 0
\end{equation}

Recall that Maxwell's equations are

\begin{equation}\label{eqn:relativisticElectrodynamicsL14:410}
\partial_i F^{ij} = j^j = 0
\end{equation}

where 

\begin{align}\label{eqn:relativisticElectrodynamicsL14:430}
\partial_i &= \PD{x^i}{} \\
\partial^i &= \PD{x_i}{}
\end{align}

Writing out the equations in terms of potentials we have
\begin{align*}
0 &= \partial_i (\partial^i A^j - \partial^j A^i)  \\
&= \partial_i \partial^i A^j - \partial_i \partial^j A^i \\
&= \partial_i \partial^i A^j - \partial^j \partial_i A^i \\
\end{align*}

So, if we pick the gauge condition $\partial_i A^i = 0$, we are left with just 

\begin{equation}\label{eqn:relativisticElectrodynamicsL14:450}
0 = \partial_i \partial^i A^j
\end{equation}

Can we choose ${A'}^i$ such that $\partial_i A^i = 0$?

Our gauge condition is 

\begin{equation}\label{eqn:relativisticElectrodynamicsL14:470}
A^i = {A'}^i + \partial^i \chi
\end{equation}

Hit it with a derivative for

\begin{equation}\label{eqn:relativisticElectrodynamicsL14:490}
\partial_i A^i = \partial_i {A'}^i + \partial_i \partial^i \chi
\end{equation}

If we want $\partial_i A^i = 0$, then we have

\begin{equation}\label{eqn:relativisticElectrodynamicsL14:510}
-\partial_i {A'}^i = \partial_i \partial^i \chi = \left( \inv{c^2} \frac{\partial^2}{\partial t^2} - \Delta \right) \chi
\end{equation}

This is the physicist proof.  Yes, it can be solved.  To really solve this, we'd want to use Green's functions.

Returning to Maxwell's equations we have

\begin{align}\label{eqn:relativisticElectrodynamicsL14:530}
0 &= \partial_i \partial^i A^j \\
0 &= \partial_i A^i ,
\end{align}

where the first is Maxwell's equation, and the second is our gauge condition.

Observe that the gauge condition is now a Lorentz scalar.

\begin{equation}\label{eqn:relativisticElectrodynamicsL14:550}
\partial^i A_i \rightarrow \partial^j {O_j}^i {O_i}^k A_k
\end{equation}

But the Lorentz transform matrices multiply out to identity, in the same way that they do for the transformation of a plain old four vector dot product $x^i y_i$.

\section{What happens with a Massive vector field?}

FIXME: go through this section and fill in the details.

\begin{equation}\label{eqn:relativisticElectrodynamicsL14:570}
S = \int d^4 x \left( \inv{4} F^{ij} F_{ij} + \frac{m^2}{2} A^i A_i \right)
\end{equation}

We call this a mass term because the units require it.

\begin{align*}
\delta S 
&= 0  \\
&= \int d^4 x \left( \partial^i \partial_i A_j - \partial_j \partial_i A^i \delta A^i + m^2 A_i \delta A^i \right)
\end{align*}

We get 

\begin{equation}\label{eqn:relativisticElectrodynamicsL14:590}
\left( (\Delta - m^2) g_{ij} - \partial_i \partial_j \right) A^i = 0
\end{equation}

FIXME: The EOM (DIY)

\begin{equation}\label{eqn:relativisticElectrodynamicsL14:610}
m^2 \partial_i A^i = 0
\end{equation}

With the mass term, the Lorentz gauge is already chosen for us?

\EndArticle
