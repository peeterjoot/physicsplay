%
% Copyright � 2015 Peeter Joot.  All Rights Reserved.
% Licenced as described in the file LICENSE under the root directory of this GIT repository.
%
\documentclass[]{eliblog}

\usepackage{amsmath}
\usepackage{mathpazo}

%
% shorthand for bold symbols, convenient for vectors and matrices
%
\newcommand{\Ba}[0]{\mathbf{a}}
\newcommand{\Bb}[0]{\mathbf{b}}
\newcommand{\Bc}[0]{\mathbf{c}}
\newcommand{\Bd}[0]{\mathbf{d}}
\newcommand{\Be}[0]{\mathbf{e}}
\newcommand{\Bf}[0]{\mathbf{f}}
\newcommand{\Bg}[0]{\mathbf{g}}
\newcommand{\Bh}[0]{\mathbf{h}}
\newcommand{\Bi}[0]{\mathbf{i}}
\newcommand{\Bj}[0]{\mathbf{j}}
\newcommand{\Bk}[0]{\mathbf{k}}
\newcommand{\Bl}[0]{\mathbf{l}}
\newcommand{\Bm}[0]{\mathbf{m}}
\newcommand{\Bn}[0]{\mathbf{n}}
\newcommand{\Bo}[0]{\mathbf{o}}
\newcommand{\Bp}[0]{\mathbf{p}}
\newcommand{\Bq}[0]{\mathbf{q}}
\newcommand{\Br}[0]{\mathbf{r}}
\newcommand{\Bs}[0]{\mathbf{s}}
\newcommand{\Bt}[0]{\mathbf{t}}
\newcommand{\Bu}[0]{\mathbf{u}}
\newcommand{\Bv}[0]{\mathbf{v}}
\newcommand{\Bw}[0]{\mathbf{w}}
\newcommand{\Bx}[0]{\mathbf{x}}
\newcommand{\By}[0]{\mathbf{y}}
\newcommand{\Bz}[0]{\mathbf{z}}
\newcommand{\BA}[0]{\mathbf{A}}
\newcommand{\BB}[0]{\mathbf{B}}
\newcommand{\BC}[0]{\mathbf{C}}
\newcommand{\BD}[0]{\mathbf{D}}
\newcommand{\BE}[0]{\mathbf{E}}
\newcommand{\BF}[0]{\mathbf{F}}
\newcommand{\BG}[0]{\mathbf{G}}
\newcommand{\BH}[0]{\mathbf{H}}
\newcommand{\BI}[0]{\mathbf{I}}
\newcommand{\BJ}[0]{\mathbf{J}}
\newcommand{\BK}[0]{\mathbf{K}}
\newcommand{\BL}[0]{\mathbf{L}}
\newcommand{\BM}[0]{\mathbf{M}}
\newcommand{\BN}[0]{\mathbf{N}}
\newcommand{\BO}[0]{\mathbf{O}}
\newcommand{\BP}[0]{\mathbf{P}}
\newcommand{\BQ}[0]{\mathbf{Q}}
\newcommand{\BR}[0]{\mathbf{R}}
\newcommand{\BS}[0]{\mathbf{S}}
\newcommand{\BT}[0]{\mathbf{T}}
\newcommand{\BU}[0]{\mathbf{U}}
\newcommand{\BV}[0]{\mathbf{V}}
\newcommand{\BW}[0]{\mathbf{W}}
\newcommand{\BX}[0]{\mathbf{X}}
\newcommand{\BY}[0]{\mathbf{Y}}
\newcommand{\BZ}[0]{\mathbf{Z}}

\newcommand{\Bzero}[0]{\mathbf{0}}
\newcommand{\Btheta}[0]{\boldsymbol{\theta}}
\newcommand{\Btau}[0]{\boldsymbol{\tau}}
\newcommand{\Bomega}[0]{\boldsymbol{\omega}}

%
% shorthand for unit vectors
%
\newcommand{\acap}[0]{\hat{\Ba}}
\newcommand{\bcap}[0]{\hat{\Bb}}
\newcommand{\ccap}[0]{\hat{\Bc}}
\newcommand{\dcap}[0]{\hat{\Bd}}
\newcommand{\ecap}[0]{\hat{\Be}}
\newcommand{\fcap}[0]{\hat{\Bf}}
\newcommand{\gcap}[0]{\hat{\Bg}}
\newcommand{\hcap}[0]{\hat{\Bh}}
\newcommand{\icap}[0]{\hat{\Bi}}
\newcommand{\jcap}[0]{\hat{\Bj}}
\newcommand{\kcap}[0]{\hat{\Bk}}
\newcommand{\lcap}[0]{\hat{\Bl}}
\newcommand{\mcap}[0]{\hat{\Bm}}
\newcommand{\ncap}[0]{\hat{\Bn}}
\newcommand{\ocap}[0]{\hat{\Bo}}
\newcommand{\pcap}[0]{\hat{\Bp}}
\newcommand{\qcap}[0]{\hat{\Bq}}
\newcommand{\rcap}[0]{\hat{\Br}}
\newcommand{\scap}[0]{\hat{\Bs}}
\newcommand{\tcap}[0]{\hat{\Bt}}
\newcommand{\ucap}[0]{\hat{\Bu}}
\newcommand{\vcap}[0]{\hat{\Bv}}
\newcommand{\wcap}[0]{\hat{\Bw}}
\newcommand{\xcap}[0]{\hat{\Bx}}
\newcommand{\ycap}[0]{\hat{\By}}
\newcommand{\zcap}[0]{\hat{\Bz}}
\newcommand{\thetacap}[0]{\hat{\Btheta}}

%
% to write R^n and C^n in a distinguishable fashion.  Perhaps change this
% to the double lined characters upon figuring out how to do so.
%
\newcommand{\C}[1]{$\mathbb{C}^{#1}$}
\newcommand{\R}[1]{$\mathbb{R}^{#1}$}

%
% various generally useful helpers
%

% derivative of #1 wrt. #2:
\newcommand{\D}[2] {\frac {d#2} {d#1}}

\newcommand{\inv}[1]{\frac{1}{#1}}
\newcommand{\cross}[0]{\times}

\newcommand{\abs}[1]{\lvert{#1}\rvert}
\newcommand{\norm}[1]{\lVert{#1}\rVert}
\newcommand{\innerprod}[2]{\langle{#1}, {#2}\rangle}
\newcommand{\dotprod}[2]{{#1} \cdot {#2}}
\newcommand{\bdotprod}[2]{\left({#1} \cdot {#2}\right)}
\newcommand{\crossprod}[2]{{#1} \cross {#2}}
\newcommand{\tripleprod}[3]{\dotprod{\left(\crossprod{#1}{#2}\right)}{#3}}

\DeclareMathOperator{\Proj}{Proj}
\DeclareMathOperator{\Span}{span}
\DeclareMathOperator{\Sgn}{sgn}
\DeclareMathOperator{\Area}{Area}
\DeclareMathOperator{\Volume}{Volume}

%
% A few miscellaneous things specific to this document
%
\newcommand{\crossop}[1]{\crossprod{#1}{}}

% R2 vector.
\newcommand{\VectorTwo}[2]{
\begin{bmatrix}
 {#1} \\
 {#2}
\end{bmatrix}
}

\newcommand{\VectorN}[1]{
\begin{bmatrix}
{#1}_1 \\
{#1}_2 \\
\vdots \\
{#1}_N \\
\end{bmatrix}
}

\newcommand{\DETuvij}[4]{
\begin{vmatrix}
 {#1}_{#3} & {#1}_{#4} \\
 {#2}_{#3} & {#2}_{#4}
\end{vmatrix}
}

\newcommand{\DETuvwijk}[6]{
\begin{vmatrix}
 {#1}_{#4} & {#1}_{#5} & {#1}_{#6} \\
 {#2}_{#4} & {#2}_{#5} & {#2}_{#6} \\
 {#3}_{#4} & {#3}_{#5} & {#3}_{#6}
\end{vmatrix}
}

\newcommand{\DETuvwxijkl}[8]{
\begin{vmatrix}
 {#1}_{#5} & {#1}_{#6} & {#1}_{#7} & {#1}_{#8} \\
 {#2}_{#5} & {#2}_{#6} & {#2}_{#7} & {#2}_{#8} \\
 {#3}_{#5} & {#3}_{#6} & {#3}_{#7} & {#3}_{#8} \\
 {#4}_{#5} & {#4}_{#6} & {#4}_{#7} & {#4}_{#8} \\
\end{vmatrix}
}

%\newcommand{\DETuvwxyijklm}[10]{
%\begin{vmatrix}
% {#1}_{#6} & {#1}_{#7} & {#1}_{#8} & {#1}_{#9} & {#1}_{#10} \\
% {#2}_{#6} & {#2}_{#7} & {#2}_{#8} & {#2}_{#9} & {#2}_{#10} \\
% {#3}_{#6} & {#3}_{#7} & {#3}_{#8} & {#3}_{#9} & {#3}_{#10} \\
% {#4}_{#6} & {#4}_{#7} & {#4}_{#8} & {#4}_{#9} & {#4}_{#10} \\
% {#5}_{#6} & {#5}_{#7} & {#5}_{#8} & {#5}_{#9} & {#5}_{#10}
%\end{vmatrix}
%}

% R3 vector.
\newcommand{\VectorThree}[3]{
\begin{bmatrix}
 {#1} \\
 {#2} \\
 {#3}
\end{bmatrix}
}



\author{Peeter Joot}
\email{peeter.joot@gmail.com}

%\documentclass[]{eliblogwidescreen}

\usepackage{amsmath}
\usepackage{mathpazo}

%
% shorthand for bold symbols, convenient for vectors and matrices
%
\newcommand{\Ba}[0]{\mathbf{a}}
\newcommand{\Bb}[0]{\mathbf{b}}
\newcommand{\Bc}[0]{\mathbf{c}}
\newcommand{\Bd}[0]{\mathbf{d}}
\newcommand{\Be}[0]{\mathbf{e}}
\newcommand{\Bf}[0]{\mathbf{f}}
\newcommand{\Bg}[0]{\mathbf{g}}
\newcommand{\Bh}[0]{\mathbf{h}}
\newcommand{\Bi}[0]{\mathbf{i}}
\newcommand{\Bj}[0]{\mathbf{j}}
\newcommand{\Bk}[0]{\mathbf{k}}
\newcommand{\Bl}[0]{\mathbf{l}}
\newcommand{\Bm}[0]{\mathbf{m}}
\newcommand{\Bn}[0]{\mathbf{n}}
\newcommand{\Bo}[0]{\mathbf{o}}
\newcommand{\Bp}[0]{\mathbf{p}}
\newcommand{\Bq}[0]{\mathbf{q}}
\newcommand{\Br}[0]{\mathbf{r}}
\newcommand{\Bs}[0]{\mathbf{s}}
\newcommand{\Bt}[0]{\mathbf{t}}
\newcommand{\Bu}[0]{\mathbf{u}}
\newcommand{\Bv}[0]{\mathbf{v}}
\newcommand{\Bw}[0]{\mathbf{w}}
\newcommand{\Bx}[0]{\mathbf{x}}
\newcommand{\By}[0]{\mathbf{y}}
\newcommand{\Bz}[0]{\mathbf{z}}
\newcommand{\BA}[0]{\mathbf{A}}
\newcommand{\BB}[0]{\mathbf{B}}
\newcommand{\BC}[0]{\mathbf{C}}
\newcommand{\BD}[0]{\mathbf{D}}
\newcommand{\BE}[0]{\mathbf{E}}
\newcommand{\BF}[0]{\mathbf{F}}
\newcommand{\BG}[0]{\mathbf{G}}
\newcommand{\BH}[0]{\mathbf{H}}
\newcommand{\BI}[0]{\mathbf{I}}
\newcommand{\BJ}[0]{\mathbf{J}}
\newcommand{\BK}[0]{\mathbf{K}}
\newcommand{\BL}[0]{\mathbf{L}}
\newcommand{\BM}[0]{\mathbf{M}}
\newcommand{\BN}[0]{\mathbf{N}}
\newcommand{\BO}[0]{\mathbf{O}}
\newcommand{\BP}[0]{\mathbf{P}}
\newcommand{\BQ}[0]{\mathbf{Q}}
\newcommand{\BR}[0]{\mathbf{R}}
\newcommand{\BS}[0]{\mathbf{S}}
\newcommand{\BT}[0]{\mathbf{T}}
\newcommand{\BU}[0]{\mathbf{U}}
\newcommand{\BV}[0]{\mathbf{V}}
\newcommand{\BW}[0]{\mathbf{W}}
\newcommand{\BX}[0]{\mathbf{X}}
\newcommand{\BY}[0]{\mathbf{Y}}
\newcommand{\BZ}[0]{\mathbf{Z}}

\newcommand{\Bzero}[0]{\mathbf{0}}
\newcommand{\Btheta}[0]{\boldsymbol{\theta}}
\newcommand{\Btau}[0]{\boldsymbol{\tau}}
\newcommand{\Bomega}[0]{\boldsymbol{\omega}}

%
% shorthand for unit vectors
%
\newcommand{\acap}[0]{\hat{\Ba}}
\newcommand{\bcap}[0]{\hat{\Bb}}
\newcommand{\ccap}[0]{\hat{\Bc}}
\newcommand{\dcap}[0]{\hat{\Bd}}
\newcommand{\ecap}[0]{\hat{\Be}}
\newcommand{\fcap}[0]{\hat{\Bf}}
\newcommand{\gcap}[0]{\hat{\Bg}}
\newcommand{\hcap}[0]{\hat{\Bh}}
\newcommand{\icap}[0]{\hat{\Bi}}
\newcommand{\jcap}[0]{\hat{\Bj}}
\newcommand{\kcap}[0]{\hat{\Bk}}
\newcommand{\lcap}[0]{\hat{\Bl}}
\newcommand{\mcap}[0]{\hat{\Bm}}
\newcommand{\ncap}[0]{\hat{\Bn}}
\newcommand{\ocap}[0]{\hat{\Bo}}
\newcommand{\pcap}[0]{\hat{\Bp}}
\newcommand{\qcap}[0]{\hat{\Bq}}
\newcommand{\rcap}[0]{\hat{\Br}}
\newcommand{\scap}[0]{\hat{\Bs}}
\newcommand{\tcap}[0]{\hat{\Bt}}
\newcommand{\ucap}[0]{\hat{\Bu}}
\newcommand{\vcap}[0]{\hat{\Bv}}
\newcommand{\wcap}[0]{\hat{\Bw}}
\newcommand{\xcap}[0]{\hat{\Bx}}
\newcommand{\ycap}[0]{\hat{\By}}
\newcommand{\zcap}[0]{\hat{\Bz}}
\newcommand{\thetacap}[0]{\hat{\Btheta}}

%
% to write R^n and C^n in a distinguishable fashion.  Perhaps change this
% to the double lined characters upon figuring out how to do so.
%
\newcommand{\C}[1]{$\mathbb{C}^{#1}$}
\newcommand{\R}[1]{$\mathbb{R}^{#1}$}

%
% various generally useful helpers
%

% derivative of #1 wrt. #2:
\newcommand{\D}[2] {\frac {d#2} {d#1}}

\newcommand{\inv}[1]{\frac{1}{#1}}
\newcommand{\cross}[0]{\times}

\newcommand{\abs}[1]{\lvert{#1}\rvert}
\newcommand{\norm}[1]{\lVert{#1}\rVert}
\newcommand{\innerprod}[2]{\langle{#1}, {#2}\rangle}
\newcommand{\dotprod}[2]{{#1} \cdot {#2}}
\newcommand{\bdotprod}[2]{\left({#1} \cdot {#2}\right)}
\newcommand{\crossprod}[2]{{#1} \cross {#2}}
\newcommand{\tripleprod}[3]{\dotprod{\left(\crossprod{#1}{#2}\right)}{#3}}

\DeclareMathOperator{\Proj}{Proj}
\DeclareMathOperator{\Span}{span}
\DeclareMathOperator{\Sgn}{sgn}
\DeclareMathOperator{\Area}{Area}
\DeclareMathOperator{\Volume}{Volume}

%
% A few miscellaneous things specific to this document
%
\newcommand{\crossop}[1]{\crossprod{#1}{}}

% R2 vector.
\newcommand{\VectorTwo}[2]{
\begin{bmatrix}
 {#1} \\
 {#2}
\end{bmatrix}
}

\newcommand{\VectorN}[1]{
\begin{bmatrix}
{#1}_1 \\
{#1}_2 \\
\vdots \\
{#1}_N \\
\end{bmatrix}
}

\newcommand{\DETuvij}[4]{
\begin{vmatrix}
 {#1}_{#3} & {#1}_{#4} \\
 {#2}_{#3} & {#2}_{#4}
\end{vmatrix}
}

\newcommand{\DETuvwijk}[6]{
\begin{vmatrix}
 {#1}_{#4} & {#1}_{#5} & {#1}_{#6} \\
 {#2}_{#4} & {#2}_{#5} & {#2}_{#6} \\
 {#3}_{#4} & {#3}_{#5} & {#3}_{#6}
\end{vmatrix}
}

\newcommand{\DETuvwxijkl}[8]{
\begin{vmatrix}
 {#1}_{#5} & {#1}_{#6} & {#1}_{#7} & {#1}_{#8} \\
 {#2}_{#5} & {#2}_{#6} & {#2}_{#7} & {#2}_{#8} \\
 {#3}_{#5} & {#3}_{#6} & {#3}_{#7} & {#3}_{#8} \\
 {#4}_{#5} & {#4}_{#6} & {#4}_{#7} & {#4}_{#8} \\
\end{vmatrix}
}

%\newcommand{\DETuvwxyijklm}[10]{
%\begin{vmatrix}
% {#1}_{#6} & {#1}_{#7} & {#1}_{#8} & {#1}_{#9} & {#1}_{#10} \\
% {#2}_{#6} & {#2}_{#7} & {#2}_{#8} & {#2}_{#9} & {#2}_{#10} \\
% {#3}_{#6} & {#3}_{#7} & {#3}_{#8} & {#3}_{#9} & {#3}_{#10} \\
% {#4}_{#6} & {#4}_{#7} & {#4}_{#8} & {#4}_{#9} & {#4}_{#10} \\
% {#5}_{#6} & {#5}_{#7} & {#5}_{#8} & {#5}_{#9} & {#5}_{#10}
%\end{vmatrix}
%}

% R3 vector.
\newcommand{\VectorThree}[3]{
\begin{bmatrix}
 {#1} \\
 {#2} \\
 {#3}
\end{bmatrix}
}



\author{Peeter Joot}
\email{peeter.joot@gmail.com}


\chapter{PHY454H1S Continuum Mechanics.  Lecture 21: Stability.  Taught by Prof. K. Das.}
\label{chap:continuumL21}
%\useCCL
\blogpage{http://sites.google.com/site/peeterjoot2/math2012/continuumL21.pdf}
\date{Mar 29, 2012}
\gitRevisionInfo{continuumL21}
\keywords{Navier-Stokes, PHY454H1S} 

\beginArtWithToc
%\beginArtNoToc

\section{Disclaimer.}

Peeter's lecture notes from class.  May not be entirely coherent.

\section{Stability.  Some graphical illustrations.}

What do we mean by stability?  A configuration is stable if after a small disturbance it returns to it's original position.  A couple systems to consider are shown in

FIXME: F1

We can examine how a displacement $\delta x$ changes with time after making it.  In a stable configuration without friction we will induce an oscillation.  With friction we'll have a damping effect.  For the parabolic well, these are illustrated in

FIXME: F2

For the inverted parabola our displacement takes the form

FIXME: F3

For the ball on the table, assuming some friction that stops the ball, fairly quickly, we'll have a displacement as illustrated in

FIXME: F1c''

\section{Characterizing stability}

Let's suppose that our displacement can be described in exponential form

\begin{equation}\label{eqn:continuumL21:10}
\delta x \sim e^{\sigma t}
\end{equation}

where $\sigma$ is the \textit{growth rate of pertubation}, and is in general a complex number of the form

\begin{equation}\label{eqn:continuumL21:30}
\sigma = \sigma_\text{R} + i \sigma_\text{I}
\end{equation}

\subsection{Case I.  Oscillatory unstability}

A system of the form

\begin{align*}
\sigma_{\text{R}} &= 0 \\
\sigma_{\text{I}} &> 0
\end{align*}

\textit{oscillatory unstable}.  An example of this is the undamped parabolic system illustrated above.

\subsection{Case II.  Marginal unstability.}

FIXME:
%\begin{align*}
%\delta x 
%&\sim e^{\sigma_{\text{R}} t} e^{i \simgaI t} \\
%&\sim e^{\sigma_{\text{R}} t} \left( \cos \sigma_{\text{I}} t + i \sin \simgaI t \right)
%\end{align*}

We'll call systems of the form

\begin{align*}
\sigma_{\text{I}} &= 0 \\
\sigma_{\text{R}} &> 0
\end{align*}

\textit{marginally unstable}.  We can have unstable systems with $\sigma_{\text{I}} \ne 0$ but still $\sigma_{\text{R}} > 0$, but these are less common.

\subsection{Case III.  Neutral stability.}

\begin{align*}
\sigma &= 0 \\
\sigma_{\text{R}} &= \sigma_{\text{I}} &= 0
\end{align*}

An example of this was the billard table example where the ball moved to a new location on the table after being bumped slightly.

\section{A mathematical description.}

For fluids we have to bring in the heat equation (which isn't in the scope of this course to derive).  Our equations are the Navier-Stokes

\begin{equation}\label{eqn:continuumL21:50}
\rho \PD{t}{\Bu} + \rho (\Bu \cdot \spacegrad) \Bu = - \spacegrad p + \mu \spacegrad^2 \Bu - \rho \zcap g
\end{equation}

For steady state with $\Bu = 0$ initially (our base state), we'll call the following the equation of the base state

\begin{equation}\label{eqn:continuumL21:70}
\boxed{
\spacegrad p_s = -\rho \zcap g
}
\end{equation}

FIXME: should this be:

\begin{equation}\label{eqn:continuumL21:70bb}
\boxed{
\spacegrad p_s = -\rho_s \zcap g
}
\end{equation}

We'll allow pertubations of each of our variables

\begin{align*}
\Bu &= \Bu_\text{base} + \delta \Bu = 0 + \delta \Bu \\
p &= p_s + \delta p \\
\rho &= \rho_s + \delta \rho
\end{align*}

After pertubation NS takes the form

\begin{equation}\label{eqn:continuumL21:90}
(\rho_s + \delta \rho )\PD{t}{(0 + \delta \Bu)} + (\rho_s + \delta \rho) ((0 + \delta \Bu) \cdot \spacegrad) (0 + \delta \Bu) = - \spacegrad (p_s + \delta p) + \mu \spacegrad^2 (0 + \delta \Bu) - (\rho_s + \delta \rho) \zcap g
\end{equation}

Retaining only terms that are of first order of smallness.

\begin{equation}\label{eqn:continuumL21:110}
\rho_s \PD{t}{\delta \Bu} = - \spacegrad p_s - \spacegrad \delta p + \mu \spacegrad^2 \delta \Bu - \rho_s \zcap g - \delta \rho \zcap g
\end{equation}

applying our equation of base state \ref{eqn:continuumL21:70}, we have
%\spacegrad p_s = -\rho_s \zcap g
\begin{equation}\label{eqn:continuumL21:110b}
\rho_s \PD{t}{\delta \Bu} = \cancel{\rho_s \zcap g} - \spacegrad \delta p + \mu \spacegrad^2 \delta \Bu - \cancel{\rho_s \zcap g} - \delta \rho \zcap g,
\end{equation}

or

\begin{equation}\label{eqn:continuumL21:110c}
\rho_s \PD{t}{\delta \Bu} = - \spacegrad \delta p + \mu \spacegrad^2 \delta \Bu - \delta \rho \zcap g.
\end{equation}

we can write

\begin{equation}\label{eqn:continuumL21:130}
\left( \PD{t}{} - \nu \spacegrad^2 \right) \delta \Bu = -\inv{\rho_s} \spacegrad \delta p - \frac{\delta \rho}{\rho_s} \zcap g
\end{equation}

Applying the divergence operation on both sides, and using $\spacegrad \Bu = 0$ so that $\spacegrad \delta \Bu = 0$ we have

\begin{equation}\label{eqn:continuumL21:150}
\left( \PD{t}{} - \nu \spacegrad^2 \right) \cancel{\spacegrad \cdot \delta \Bu} = -\inv{\rho_s} \spacegrad^2 \delta p - (\zcap \cdot \spacegrad ) \frac{\delta \rho}{\rho_s} g,
\end{equation}

or

\begin{equation}\label{eqn:continuumL21:170}
\inv{\rho_s} \spacegrad^2 \delta p = - (\zcap \cdot \spacegrad ) \frac{\delta \rho}{\rho_s} g.
\end{equation}

Assuming that $\rho_s$ is constant (actually that's already been done above), we can cancel it, leaving

\begin{equation}\label{eqn:continuumL21:190}
\spacegrad^2 \delta p = - (\zcap \cdot \spacegrad ) g \delta \rho = -g \PD{z}{} \delta \rho.
\end{equation}

operating once more with $\PDi{z}{}$ we have

\begin{equation}\label{eqn:continuumL21:210}
\spacegrad^2 \PD{z}{\delta p} = -g \PDSq{z}{\delta \rho}.
\end{equation}

NEW SECTION?  Missed where this came from.  Taking curls?

\begin{equation}\label{eqn:continuumL21:230}
\left( \PD{t}{} - \nu \spacegrad^2 \right) \delta \omega = -\inv{\rho_s} \PD{z}{\delta p} - \frac{\delta \rho}{\rho_s} g
\end{equation}

\begin{align*}
\left( \PD{t}{} - \nu \spacegrad^2 \right) \spacegrad^2 \delta \omega 
&= -\inv{\rho_s} \PD{z}{ \spacegrad^2 \delta p} - \frac{g}{\rho_s} \spacegrad^2 \delta \rho \\
&= -\frac{g}{\rho_s} \PDSq{z}{\delta \rho} - \frac{g}{\rho_s} \spacegrad^2 \delta \rho \\
&= 
-\frac{g}{\rho_s} \left( 
\PDSq{x}{}
+\PDSq{y}{}
\right)
\delta \rho \\
&=
g \alpha \left( 
\PDSq{x}{}
+\PDSq{y}{}
\right)
\delta T
\end{align*}

in the last step we use the following assumed relation for temperature

\begin{equation}\label{eqn:continuumL21:250}
\delta \rho = - \rho_s \alpha \delta T
\end{equation}

(where does this come from?)

We have finally

\begin{equation}\label{eqn:continuumL21:290}
\left( \PD{t}{} - \nu \spacegrad^2 \right) \spacegrad^2 \delta \omega 
= 
g \alpha \left( 
\PDSq{x}{}
+\PDSq{y}{}
\right)
\delta T
\end{equation}

Solving this is the Rayleigh-Benard instabilit problem.

\begin{equation}\label{eqn:continuumL21:270}
\left( \PD{t}{} - \kappa \spacegrad^2 \right) \delta T = \Delta T \frac{\delta \omega}{d}
\end{equation}

%\section{Stuff.}

%FIXME: Reading: \S XX from \cite{acheson1990elementary}

%\EndArticle
\EndNoBibArticle
