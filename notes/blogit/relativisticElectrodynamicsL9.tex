%
% Copyright � 2015 Peeter Joot.  All Rights Reserved.
% Licenced as described in the file LICENSE under the root directory of this GIT repository.
%
\documentclass[]{eliblog}

\usepackage{amsmath}
\usepackage{mathpazo}

%
% shorthand for bold symbols, convenient for vectors and matrices
%
\newcommand{\Ba}[0]{\mathbf{a}}
\newcommand{\Bb}[0]{\mathbf{b}}
\newcommand{\Bc}[0]{\mathbf{c}}
\newcommand{\Bd}[0]{\mathbf{d}}
\newcommand{\Be}[0]{\mathbf{e}}
\newcommand{\Bf}[0]{\mathbf{f}}
\newcommand{\Bg}[0]{\mathbf{g}}
\newcommand{\Bh}[0]{\mathbf{h}}
\newcommand{\Bi}[0]{\mathbf{i}}
\newcommand{\Bj}[0]{\mathbf{j}}
\newcommand{\Bk}[0]{\mathbf{k}}
\newcommand{\Bl}[0]{\mathbf{l}}
\newcommand{\Bm}[0]{\mathbf{m}}
\newcommand{\Bn}[0]{\mathbf{n}}
\newcommand{\Bo}[0]{\mathbf{o}}
\newcommand{\Bp}[0]{\mathbf{p}}
\newcommand{\Bq}[0]{\mathbf{q}}
\newcommand{\Br}[0]{\mathbf{r}}
\newcommand{\Bs}[0]{\mathbf{s}}
\newcommand{\Bt}[0]{\mathbf{t}}
\newcommand{\Bu}[0]{\mathbf{u}}
\newcommand{\Bv}[0]{\mathbf{v}}
\newcommand{\Bw}[0]{\mathbf{w}}
\newcommand{\Bx}[0]{\mathbf{x}}
\newcommand{\By}[0]{\mathbf{y}}
\newcommand{\Bz}[0]{\mathbf{z}}
\newcommand{\BA}[0]{\mathbf{A}}
\newcommand{\BB}[0]{\mathbf{B}}
\newcommand{\BC}[0]{\mathbf{C}}
\newcommand{\BD}[0]{\mathbf{D}}
\newcommand{\BE}[0]{\mathbf{E}}
\newcommand{\BF}[0]{\mathbf{F}}
\newcommand{\BG}[0]{\mathbf{G}}
\newcommand{\BH}[0]{\mathbf{H}}
\newcommand{\BI}[0]{\mathbf{I}}
\newcommand{\BJ}[0]{\mathbf{J}}
\newcommand{\BK}[0]{\mathbf{K}}
\newcommand{\BL}[0]{\mathbf{L}}
\newcommand{\BM}[0]{\mathbf{M}}
\newcommand{\BN}[0]{\mathbf{N}}
\newcommand{\BO}[0]{\mathbf{O}}
\newcommand{\BP}[0]{\mathbf{P}}
\newcommand{\BQ}[0]{\mathbf{Q}}
\newcommand{\BR}[0]{\mathbf{R}}
\newcommand{\BS}[0]{\mathbf{S}}
\newcommand{\BT}[0]{\mathbf{T}}
\newcommand{\BU}[0]{\mathbf{U}}
\newcommand{\BV}[0]{\mathbf{V}}
\newcommand{\BW}[0]{\mathbf{W}}
\newcommand{\BX}[0]{\mathbf{X}}
\newcommand{\BY}[0]{\mathbf{Y}}
\newcommand{\BZ}[0]{\mathbf{Z}}

\newcommand{\Bzero}[0]{\mathbf{0}}
\newcommand{\Btheta}[0]{\boldsymbol{\theta}}
\newcommand{\Btau}[0]{\boldsymbol{\tau}}
\newcommand{\Bomega}[0]{\boldsymbol{\omega}}

%
% shorthand for unit vectors
%
\newcommand{\acap}[0]{\hat{\Ba}}
\newcommand{\bcap}[0]{\hat{\Bb}}
\newcommand{\ccap}[0]{\hat{\Bc}}
\newcommand{\dcap}[0]{\hat{\Bd}}
\newcommand{\ecap}[0]{\hat{\Be}}
\newcommand{\fcap}[0]{\hat{\Bf}}
\newcommand{\gcap}[0]{\hat{\Bg}}
\newcommand{\hcap}[0]{\hat{\Bh}}
\newcommand{\icap}[0]{\hat{\Bi}}
\newcommand{\jcap}[0]{\hat{\Bj}}
\newcommand{\kcap}[0]{\hat{\Bk}}
\newcommand{\lcap}[0]{\hat{\Bl}}
\newcommand{\mcap}[0]{\hat{\Bm}}
\newcommand{\ncap}[0]{\hat{\Bn}}
\newcommand{\ocap}[0]{\hat{\Bo}}
\newcommand{\pcap}[0]{\hat{\Bp}}
\newcommand{\qcap}[0]{\hat{\Bq}}
\newcommand{\rcap}[0]{\hat{\Br}}
\newcommand{\scap}[0]{\hat{\Bs}}
\newcommand{\tcap}[0]{\hat{\Bt}}
\newcommand{\ucap}[0]{\hat{\Bu}}
\newcommand{\vcap}[0]{\hat{\Bv}}
\newcommand{\wcap}[0]{\hat{\Bw}}
\newcommand{\xcap}[0]{\hat{\Bx}}
\newcommand{\ycap}[0]{\hat{\By}}
\newcommand{\zcap}[0]{\hat{\Bz}}
\newcommand{\thetacap}[0]{\hat{\Btheta}}

%
% to write R^n and C^n in a distinguishable fashion.  Perhaps change this
% to the double lined characters upon figuring out how to do so.
%
\newcommand{\C}[1]{$\mathbb{C}^{#1}$}
\newcommand{\R}[1]{$\mathbb{R}^{#1}$}

%
% various generally useful helpers
%

% derivative of #1 wrt. #2:
\newcommand{\D}[2] {\frac {d#2} {d#1}}

\newcommand{\inv}[1]{\frac{1}{#1}}
\newcommand{\cross}[0]{\times}

\newcommand{\abs}[1]{\lvert{#1}\rvert}
\newcommand{\norm}[1]{\lVert{#1}\rVert}
\newcommand{\innerprod}[2]{\langle{#1}, {#2}\rangle}
\newcommand{\dotprod}[2]{{#1} \cdot {#2}}
\newcommand{\bdotprod}[2]{\left({#1} \cdot {#2}\right)}
\newcommand{\crossprod}[2]{{#1} \cross {#2}}
\newcommand{\tripleprod}[3]{\dotprod{\left(\crossprod{#1}{#2}\right)}{#3}}

\DeclareMathOperator{\Proj}{Proj}
\DeclareMathOperator{\Span}{span}
\DeclareMathOperator{\Sgn}{sgn}
\DeclareMathOperator{\Area}{Area}
\DeclareMathOperator{\Volume}{Volume}

%
% A few miscellaneous things specific to this document
%
\newcommand{\crossop}[1]{\crossprod{#1}{}}

% R2 vector.
\newcommand{\VectorTwo}[2]{
\begin{bmatrix}
 {#1} \\
 {#2}
\end{bmatrix}
}

\newcommand{\VectorN}[1]{
\begin{bmatrix}
{#1}_1 \\
{#1}_2 \\
\vdots \\
{#1}_N \\
\end{bmatrix}
}

\newcommand{\DETuvij}[4]{
\begin{vmatrix}
 {#1}_{#3} & {#1}_{#4} \\
 {#2}_{#3} & {#2}_{#4}
\end{vmatrix}
}

\newcommand{\DETuvwijk}[6]{
\begin{vmatrix}
 {#1}_{#4} & {#1}_{#5} & {#1}_{#6} \\
 {#2}_{#4} & {#2}_{#5} & {#2}_{#6} \\
 {#3}_{#4} & {#3}_{#5} & {#3}_{#6}
\end{vmatrix}
}

\newcommand{\DETuvwxijkl}[8]{
\begin{vmatrix}
 {#1}_{#5} & {#1}_{#6} & {#1}_{#7} & {#1}_{#8} \\
 {#2}_{#5} & {#2}_{#6} & {#2}_{#7} & {#2}_{#8} \\
 {#3}_{#5} & {#3}_{#6} & {#3}_{#7} & {#3}_{#8} \\
 {#4}_{#5} & {#4}_{#6} & {#4}_{#7} & {#4}_{#8} \\
\end{vmatrix}
}

%\newcommand{\DETuvwxyijklm}[10]{
%\begin{vmatrix}
% {#1}_{#6} & {#1}_{#7} & {#1}_{#8} & {#1}_{#9} & {#1}_{#10} \\
% {#2}_{#6} & {#2}_{#7} & {#2}_{#8} & {#2}_{#9} & {#2}_{#10} \\
% {#3}_{#6} & {#3}_{#7} & {#3}_{#8} & {#3}_{#9} & {#3}_{#10} \\
% {#4}_{#6} & {#4}_{#7} & {#4}_{#8} & {#4}_{#9} & {#4}_{#10} \\
% {#5}_{#6} & {#5}_{#7} & {#5}_{#8} & {#5}_{#9} & {#5}_{#10}
%\end{vmatrix}
%}

% R3 vector.
\newcommand{\VectorThree}[3]{
\begin{bmatrix}
 {#1} \\
 {#2} \\
 {#3}
\end{bmatrix}
}



\author{Peeter Joot}
\email{peeter.joot@gmail.com}

%\documentclass[]{eliblogwidescreen}

\usepackage{amsmath}
\usepackage{mathpazo}

%
% shorthand for bold symbols, convenient for vectors and matrices
%
\newcommand{\Ba}[0]{\mathbf{a}}
\newcommand{\Bb}[0]{\mathbf{b}}
\newcommand{\Bc}[0]{\mathbf{c}}
\newcommand{\Bd}[0]{\mathbf{d}}
\newcommand{\Be}[0]{\mathbf{e}}
\newcommand{\Bf}[0]{\mathbf{f}}
\newcommand{\Bg}[0]{\mathbf{g}}
\newcommand{\Bh}[0]{\mathbf{h}}
\newcommand{\Bi}[0]{\mathbf{i}}
\newcommand{\Bj}[0]{\mathbf{j}}
\newcommand{\Bk}[0]{\mathbf{k}}
\newcommand{\Bl}[0]{\mathbf{l}}
\newcommand{\Bm}[0]{\mathbf{m}}
\newcommand{\Bn}[0]{\mathbf{n}}
\newcommand{\Bo}[0]{\mathbf{o}}
\newcommand{\Bp}[0]{\mathbf{p}}
\newcommand{\Bq}[0]{\mathbf{q}}
\newcommand{\Br}[0]{\mathbf{r}}
\newcommand{\Bs}[0]{\mathbf{s}}
\newcommand{\Bt}[0]{\mathbf{t}}
\newcommand{\Bu}[0]{\mathbf{u}}
\newcommand{\Bv}[0]{\mathbf{v}}
\newcommand{\Bw}[0]{\mathbf{w}}
\newcommand{\Bx}[0]{\mathbf{x}}
\newcommand{\By}[0]{\mathbf{y}}
\newcommand{\Bz}[0]{\mathbf{z}}
\newcommand{\BA}[0]{\mathbf{A}}
\newcommand{\BB}[0]{\mathbf{B}}
\newcommand{\BC}[0]{\mathbf{C}}
\newcommand{\BD}[0]{\mathbf{D}}
\newcommand{\BE}[0]{\mathbf{E}}
\newcommand{\BF}[0]{\mathbf{F}}
\newcommand{\BG}[0]{\mathbf{G}}
\newcommand{\BH}[0]{\mathbf{H}}
\newcommand{\BI}[0]{\mathbf{I}}
\newcommand{\BJ}[0]{\mathbf{J}}
\newcommand{\BK}[0]{\mathbf{K}}
\newcommand{\BL}[0]{\mathbf{L}}
\newcommand{\BM}[0]{\mathbf{M}}
\newcommand{\BN}[0]{\mathbf{N}}
\newcommand{\BO}[0]{\mathbf{O}}
\newcommand{\BP}[0]{\mathbf{P}}
\newcommand{\BQ}[0]{\mathbf{Q}}
\newcommand{\BR}[0]{\mathbf{R}}
\newcommand{\BS}[0]{\mathbf{S}}
\newcommand{\BT}[0]{\mathbf{T}}
\newcommand{\BU}[0]{\mathbf{U}}
\newcommand{\BV}[0]{\mathbf{V}}
\newcommand{\BW}[0]{\mathbf{W}}
\newcommand{\BX}[0]{\mathbf{X}}
\newcommand{\BY}[0]{\mathbf{Y}}
\newcommand{\BZ}[0]{\mathbf{Z}}

\newcommand{\Bzero}[0]{\mathbf{0}}
\newcommand{\Btheta}[0]{\boldsymbol{\theta}}
\newcommand{\Btau}[0]{\boldsymbol{\tau}}
\newcommand{\Bomega}[0]{\boldsymbol{\omega}}

%
% shorthand for unit vectors
%
\newcommand{\acap}[0]{\hat{\Ba}}
\newcommand{\bcap}[0]{\hat{\Bb}}
\newcommand{\ccap}[0]{\hat{\Bc}}
\newcommand{\dcap}[0]{\hat{\Bd}}
\newcommand{\ecap}[0]{\hat{\Be}}
\newcommand{\fcap}[0]{\hat{\Bf}}
\newcommand{\gcap}[0]{\hat{\Bg}}
\newcommand{\hcap}[0]{\hat{\Bh}}
\newcommand{\icap}[0]{\hat{\Bi}}
\newcommand{\jcap}[0]{\hat{\Bj}}
\newcommand{\kcap}[0]{\hat{\Bk}}
\newcommand{\lcap}[0]{\hat{\Bl}}
\newcommand{\mcap}[0]{\hat{\Bm}}
\newcommand{\ncap}[0]{\hat{\Bn}}
\newcommand{\ocap}[0]{\hat{\Bo}}
\newcommand{\pcap}[0]{\hat{\Bp}}
\newcommand{\qcap}[0]{\hat{\Bq}}
\newcommand{\rcap}[0]{\hat{\Br}}
\newcommand{\scap}[0]{\hat{\Bs}}
\newcommand{\tcap}[0]{\hat{\Bt}}
\newcommand{\ucap}[0]{\hat{\Bu}}
\newcommand{\vcap}[0]{\hat{\Bv}}
\newcommand{\wcap}[0]{\hat{\Bw}}
\newcommand{\xcap}[0]{\hat{\Bx}}
\newcommand{\ycap}[0]{\hat{\By}}
\newcommand{\zcap}[0]{\hat{\Bz}}
\newcommand{\thetacap}[0]{\hat{\Btheta}}

%
% to write R^n and C^n in a distinguishable fashion.  Perhaps change this
% to the double lined characters upon figuring out how to do so.
%
\newcommand{\C}[1]{$\mathbb{C}^{#1}$}
\newcommand{\R}[1]{$\mathbb{R}^{#1}$}

%
% various generally useful helpers
%

% derivative of #1 wrt. #2:
\newcommand{\D}[2] {\frac {d#2} {d#1}}

\newcommand{\inv}[1]{\frac{1}{#1}}
\newcommand{\cross}[0]{\times}

\newcommand{\abs}[1]{\lvert{#1}\rvert}
\newcommand{\norm}[1]{\lVert{#1}\rVert}
\newcommand{\innerprod}[2]{\langle{#1}, {#2}\rangle}
\newcommand{\dotprod}[2]{{#1} \cdot {#2}}
\newcommand{\bdotprod}[2]{\left({#1} \cdot {#2}\right)}
\newcommand{\crossprod}[2]{{#1} \cross {#2}}
\newcommand{\tripleprod}[3]{\dotprod{\left(\crossprod{#1}{#2}\right)}{#3}}

\DeclareMathOperator{\Proj}{Proj}
\DeclareMathOperator{\Span}{span}
\DeclareMathOperator{\Sgn}{sgn}
\DeclareMathOperator{\Area}{Area}
\DeclareMathOperator{\Volume}{Volume}

%
% A few miscellaneous things specific to this document
%
\newcommand{\crossop}[1]{\crossprod{#1}{}}

% R2 vector.
\newcommand{\VectorTwo}[2]{
\begin{bmatrix}
 {#1} \\
 {#2}
\end{bmatrix}
}

\newcommand{\VectorN}[1]{
\begin{bmatrix}
{#1}_1 \\
{#1}_2 \\
\vdots \\
{#1}_N \\
\end{bmatrix}
}

\newcommand{\DETuvij}[4]{
\begin{vmatrix}
 {#1}_{#3} & {#1}_{#4} \\
 {#2}_{#3} & {#2}_{#4}
\end{vmatrix}
}

\newcommand{\DETuvwijk}[6]{
\begin{vmatrix}
 {#1}_{#4} & {#1}_{#5} & {#1}_{#6} \\
 {#2}_{#4} & {#2}_{#5} & {#2}_{#6} \\
 {#3}_{#4} & {#3}_{#5} & {#3}_{#6}
\end{vmatrix}
}

\newcommand{\DETuvwxijkl}[8]{
\begin{vmatrix}
 {#1}_{#5} & {#1}_{#6} & {#1}_{#7} & {#1}_{#8} \\
 {#2}_{#5} & {#2}_{#6} & {#2}_{#7} & {#2}_{#8} \\
 {#3}_{#5} & {#3}_{#6} & {#3}_{#7} & {#3}_{#8} \\
 {#4}_{#5} & {#4}_{#6} & {#4}_{#7} & {#4}_{#8} \\
\end{vmatrix}
}

%\newcommand{\DETuvwxyijklm}[10]{
%\begin{vmatrix}
% {#1}_{#6} & {#1}_{#7} & {#1}_{#8} & {#1}_{#9} & {#1}_{#10} \\
% {#2}_{#6} & {#2}_{#7} & {#2}_{#8} & {#2}_{#9} & {#2}_{#10} \\
% {#3}_{#6} & {#3}_{#7} & {#3}_{#8} & {#3}_{#9} & {#3}_{#10} \\
% {#4}_{#6} & {#4}_{#7} & {#4}_{#8} & {#4}_{#9} & {#4}_{#10} \\
% {#5}_{#6} & {#5}_{#7} & {#5}_{#8} & {#5}_{#9} & {#5}_{#10}
%\end{vmatrix}
%}

% R3 vector.
\newcommand{\VectorThree}[3]{
\begin{bmatrix}
 {#1} \\
 {#2} \\
 {#3}
\end{bmatrix}
}



\author{Peeter Joot}
\email{peeter.joot@gmail.com}


\chapter{PHY450H1S.  Relativistic Electrodynamics Lecture 9 (Taught by Prof. Erich Poppitz).  Dynamics in a vector field.}
\label{chap:relativisticElectrodynamicsL9}
%\useCCL
\blogpage{http://sites.google.com/site/peeterjoot/math2011/relativisticElectrodynamicsL9.pdf}
\date{Feb 1, 2011}
\revisionInfo{relativisticElectrodynamicsL9.tex}

%\beginArtWithToc
\beginArtNoToc

\section{Reading.}

Covering chapter 2 material from the text \cite{landau1980classical}.

Covering \href{http://www.physics.utoronto.ca/~poppitz/e-poppitz/PHY450_files/RelEMpp56.1-73.pdf}{lecture notes pp. 56.1-72}: comments on mass, energy, momentum, and massless particles (56.1-58); particles in external fields: Lorentz scalar field (59-62); reminder of a vector field under spatial rotations (63) and a Lorentz vector field (64-65) [Tuesday, Feb. 1]; the action for a relativistic particle in an external 4-vector field (65-66); the equation of motion of a relativistic particle in an external electromagnetic (4-vector) field (67,68,73) [Wednesday, Feb. 2]; mathematical interlude: (69-72): on 3x3 antisymmetric matrices, 3-vectors, and totally antisymmetric 3-index tensor - please read by yourselves, preferably by Wed., Feb. 2 class! (this is important, we�ll also soon need the 4-dimensional generalization)

\section{More on the action.}

Action for a relativistic particle in an external 4-scalar field

S = -m c \int ds - g \int ds \phi(x)

Unfortunately we have no 4-vector scalar fields (at least for particles that are long lived and stable).

PICTURE: 3-vector field, some arrows in various directions.

PICTURE: A vector $\BA$ in an $x,y$ frame, and a rotated (counterclockwise by angle $\alpha$) $x', y'$ frame with the components in each shown pictorially.

We have

A_x'(x', y') &= \cos\alpha A_x(x,y) + \sin\alpha A_y(x,y) \\
A_y'(x', y') &= -\sin\alpha A_x(x,y) + \cos\alpha A_y(x,y) 

\begin{bmatrix}
A_x'(x', y') \\
A_y'(x', y')
\end{bmatrix}
=
\begin{bmatrix}
\cos\alpha A_x(x,y) & \sin\alpha A_y(x,y) \\
-\sin\alpha A_x(x,y) & \cos\alpha A_y(x,y) 
\end{bmatrix}
\begin{bmatrix}
A_x(x, y) \\
A_y(x, y)
\end{bmatrix}

More generally we have

\begin{bmatrix}
A_x'(x', y', z') \\
A_y'(x', y', z')
A_z'(x', y', z')
\end{bmatrix}
=
\hat{O}
\begin{bmatrix}
A_x(x, y, z) \\
A_y(x, y, z)
A_z(x, y, z)
\end{bmatrix}

Here $\hat{O}$ is an $SO(3)$ matrix rotating $x \rightarrow x'$

\BA(\Bx) \cdot \By = \BA'(\Bx') \cdot \By'

\BA \cdot \BB = \text{invariant}

A four vector field is $A^i(x)$, with $x = x^i, i = 0,1,2,3$ and we'd write

\begin{bmatrix}
(x^0)' \\
(x^1)' \\
(x^2)' \\
(x^3)'
\end{bmatrix}
=
\hat{O}
\begin{bmatrix}
x^0 \\
x^1 \\
x^2 \\
x^3
\end{bmatrix}

Now $\hat{O}$ is an $SO(1,3)$ matrix.    Our four vector field is then

\begin{bmatrix}
(A^0)' \\
(A^1)' \\
(A^2)' \\
(A^3)'
\end{bmatrix}
=
\hat{O}
\begin{bmatrix}
A^0 \\
A^1 \\
A^2 \\
A^3
\end{bmatrix}

We have 

A^i g_{ij} x^i = \text{invariant} = A^i' g_{ij} x^i' 

From electrodynamics we know that we have a scalar field, the electrostatic potential, and a vector field 

What's a plausible action?

How about

\int ds x^i g_{ij} A^j

This isn't translation invariant.

\int ds x^i g_{ij} A^j

Next simplest is

\int ds u^i g_{ij} A^j

Could also do

\int ds A^i g_{ij} A^j

but it turns out that this isn't gauge invariant (to be defined and discussed in detail).

Note that the convention for this course is to write

u^i = \left( \gamma, \gamma \frac{\Bv}{c} \right) = \frac{dx^i}{ds}

Where $u^i$ is dimensionless ($u^i u_i = 1$)
Some authors use 

u^i = \left( \gamma c, \gamma \Bv \right) = \frac{dx^i}{d\tau}

The simplest action for a four vector field $A^i$ is then

S = - m c \int ds - \frac{e}{c} \int ds u^i A_i

(Recall that $u^i A_i = u^i g_{ij} A^j$).

In this action $e$ is nothing but a Lorentz scalar, a property of the particle that describes how it ``couples'' (or ``feels'') the electrodynamics field.

Similarily $mc$ is a Lorentz scalar which is a property of the particle (inertia).

It turns out that all the electric charges in nature are quantized, and there are some deep reasons (in magnetic monopoles exist) for this.

Another reason for charge quantitization apparently has to do with gauge invariance and associated compact groups.  Poppitz is amusing himself a bit here, hinting at some stuff that we can eventually learn.

Returning to our discussion, we have

S = - m c \int ds - \frac{e}{c} \int ds u^i g_{ij} A^j

with the electrodynamics four vector potential

A^i &= (\phi, \BA)
u^i &= (\gamma, \gamma \Bv/c)
u^i g_{ij} A^j = \gamma \phi - \gamma \frac{\Bv \cdot \BA}{c}

SQR = \sqrt{1 - m\Bv^2/c^2}

S 
&= - m c^2 \int dt SQR - \frac{e}{c} \int c dt SQR \left( \gamma \phi - \gamma \frac{\Bv}{c} \cdot \BA \right) \\
&= \int dt \left(
- m c^2 SQR - e \phi(\Bx, t) + e/c \Bv \cdot \BA(\Bx, t)
\right) \\

\PD{\Bv}{\LL} = m c^2/SQR \Bv/c^2 + e/c \BA(\Bx, t)

ddt \PD{\Bv}{\LL} = m ddt (\gamma \Bv) + e/c PDt \BA + e/c \PD{x^\alpha}{\BA} v^\alpha

Here
\alpha,\beta = 1,2,3

and summed over.

Equating these we have 

\PD{x^\alpha}{\LL} = - e \partial_\alpha \phi + e/c v^\beta \partial_\alpha A^\beta

ddt m \gamma v^\alpha = e UNDERBRACE=E^\alpha \left( - \inv{c} PDt A^\alpha - \partial_\alpha \phi) + e/c UNDERBRACE=B^\lambda v^\beta \left( \partial_alpha - \partial_beta A^\alpha \right)

\section{antisymmetric matrixes}

M_{\mu\nu} 
&= \partial_\mu A^\nu - \partial_\nu A^\mu 
&= \epsilon_{\mu\nu\lambda} B_\lambda

where 

B_\lambda = \inv{2} \epsilon_{\lambda\mu\nu} M_{\mu\nu}

PROVE THIS.

\epsilon_{\mu\nu\lambda} = 0, if any two indexes coincide
\epsilon_{\mu\nu\lambda} = 1, for even permutations of \mu\nu\lambda
\epsilon_{\mu\nu\lambda} = -1, for odd permutations of \mu\nu\lambda

example:
\epsilon_{123} = 1
\epsilon_{213} = -1
\epsilon_{231} = 1

B_1 
&= \inv{2} ( \epsilon_{123} M_{23} + \epsilon_{132} M_{32}) 
&=\inv{2} ( M_{23} - M_{32}) 
&= \partial_2 A_3 - \partial_3 A_2

So we have

ddt ( m \gamma v^\alpha ) = e E^\alpha + e/c \epsilon_{\alpha\beta\gamma} B_\gamma

\epsilon_{\alpha\beta\gamma} B_\gamma = (\Bv \cross \BB)_\alpha

So

ddt ( m \gamma \Bv ) = e \BE + e/c \Bv \cross \BB

or

ddt ( \Bp ) = e \left( \BE + \frac{\Bv}{c} \cross \BB \right)

\paragraph{What is the energy component of the Lorentz force equation}

Observe that this is almost a relativisitic equation, but we aren't going to get to the full equation yet.  The energy component can be obtained from

\frac{du^0}{ds} = e F^{0j} u_j

Since the full equation is

\frac{du^i}{ds} = e F^{ij} u_j

(take with a grain of salt, may be off by sign, or factors of $c$).

\section{}

Claim

S_{\text{interaction}} = - \frac{e}{c} \int ds u^i A_i

changes by boundary terms only under 

``gauge transformation'' :

A_i = A_i' + \partial_i \chi

where $\chi$ is a Lorentz scalar.  This $\partial_i$ is the four gradient.  Let's see this

Therefore the equations of motion are the same in an external $A^i$ and ${A'}^i$.

Recall that the $\BE$ and $\BB$ fields do not change under such transformations.  Let's see how the action transforms

S 
&= - \frac{e}{c} \int ds u^i A_i 
&= - \frac{e}{c} \int ds u^i \left( {A'}_i + \partial_i \chi \right) \\
&= 
- \frac{e}{c} \int ds u^i {A'}_i 
- \frac{e}{c} \int ds dx^i/ds \partial_i \chi \\
&= 
- \frac{e}{c} \int ds u^i {A'}_i 
- \frac{e}{c} \int ds dds (\chi )

dds \chi(x^{0,1,2,3}) 
= 
\partial_0 \chi dx^0/ds + 
\partial_1 \chi dx^1/ds + 
\partial_2 \chi dx^2/ds + 
\partial_3 \chi dx^3/ds
&= 
\partial_i \chi dds x^i

We can then evaluate the line integral above and find it only depends on the end points of the interval

S 
&= 
- \frac{e}{c} \int ds u^i {A'}_i 
- \frac{e}{c} ( \chi(x_b) - \chi(x_a) )

\EndArticle
