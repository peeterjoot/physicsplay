%
% Copyright � 2012 Peeter Joot.  All Rights Reserved.
% Licenced as described in the file LICENSE under the root directory of this GIT repository.
%
\newcommand{\authorname}{Peeter Joot}
\newcommand{\email}{peeterjoot@protonmail.com}
\newcommand{\basename}{FIXMEbasenameUndefined}
\newcommand{\dirname}{notes/FIXMEdirnameUndefined/}

\renewcommand{\basename}{maxwellsFromGA}
\renewcommand{\dirname}{notes/gabook/}
%\newcommand{\dateintitle}{}
%\newcommand{\keywords}{}
%\blogpage{http://sites.google.com/site/peeterjoot/math2010/maxwellsFromGA.pdf}
%\date{Nov 8, 2010}

\newcommand{\authorname}{Peeter Joot}
\newcommand{\onlineurl}{http://sites.google.com/site/peeterjoot2/math2013/\basename.pdf}
\newcommand{\sourcepath}{\dirname\basename.tex}
\newcommand{\generatetitle}[1]{\chapter{#1}}

\newcommand{\vcsinfo}{%
\section*{}
\noindent{\color{DarkOliveGreen}{\rule{\linewidth}{0.1mm}}}
\paragraph{Document version}
%\paragraph{\color{Maroon}{Document version}}
{
\small
\begin{itemize}
\item Available online at:\\ 
\href{\onlineurl}{\onlineurl}
\item Git Repository: \input{./.revinfo/gitRepo.tex}
\item Source: \sourcepath
\item last commit: \input{./.revinfo/gitCommitString.tex}
\item commit date: \input{./.revinfo/gitCommitDate.tex}
\end{itemize}
}
}

%\PassOptionsToPackage{dvipsnames,svgnames}{xcolor}
\PassOptionsToPackage{square,numbers}{natbib}
\documentclass{scrreprt}

\usepackage[left=2cm,right=2cm]{geometry}
\usepackage[svgnames]{xcolor}
\usepackage{peeters_layout}

\usepackage{natbib}

\usepackage[
colorlinks=true,
bookmarks=false,
pdfauthor={\authorname, \email},
backref 
]{hyperref}

% http://tex.stackexchange.com/questions/75773/how-to-reference-problems-by-the-text-label-in-an-exercise-envioronment
\usepackage[english]{cleveref}
\crefname{Exercise}{exercise}{exercises}
\Crefname{Exercise}{Exercise}{Exercises}

\RequirePackage{titlesec}
\RequirePackage{ifthen}

% http://stackoverflow.com/questions/4932910/date-in-the-tabular-environment
\makeatletter
\let\insertdate\@date
\makeatother

\titleformat{\chapter}[display]
{\bfseries\Large}
{\color{DarkSlateGrey}\filleft \authorname
\ifthenelse{\isundefined{\studentnumber}}{}{\\ \studentnumber}
\ifthenelse{\isundefined{\email}}{}{\\ \email}
\ifthenelse{\isundefined{\dateintitle}}{}{\\ \insertdate}
%\ifthenelse{\isundefined{\coursename}}{}{\\ \coursename} % put in title instead.
}
{4ex}
{\color{DarkOliveGreen}{\titlerule}\color{Maroon}
\vspace{2ex}%
\filright}
[\vspace{2ex}%
\color{DarkOliveGreen}\titlerule
]

\newcommand{\beginArtWithToc}[0]{\begin{document}\tableofcontents}
\newcommand{\beginArtNoToc}[0]{\begin{document}}
\newcommand{\EndNoBibArticle}[0]{\end{document}}
\newcommand{\EndArticle}[0]{\bibliography{Bibliography}\bibliographystyle{plainnat}\end{document}}

% 
%\newcommand{\citep}[1]{\cite{#1}}

\colorSectionsForArticle



\beginArtNoToc

\generatetitle{Maxwell's equation in STA}
\chapter{Maxwell's equation in STA}
\label{chap:maxwellsFromGA}

\section{Reciprocal frame}

Associated with the basis \(\{\gamma_\mu\}\) is the reciprocal basis \(\{\gamma^\mu = \inv{\gamma^\mu}\}\), satisfying the relation

\begin{equation}\label{eqn:maxwellsFromGA:20}
\begin{aligned}
\gamma_\mu \cdot \gamma^\nu &= {\delta_\mu}^\nu
\end{aligned}
\end{equation}

These reciprocal frame vectors differ only by a sign, with \(\gamma^0 = \gamma_0\), and \(\gamma^k = -\gamma_k\).

A vector may be represented in either upper or lower index coordinates \(a = a^\mu \gamma_\mu = a_\mu \gamma^\mu\), where the coordinates may be extracted by taking dot products with the basis vectors or their reciprocals

\begin{equation}\label{eqn:maxwellsFromGA:40}
\begin{aligned}
a \cdot \gamma^\nu &= a^\nu \\
a \cdot \gamma_\nu &= a_\nu
\end{aligned}
\end{equation}

\section{Space time gradient}

The spacetime gradient, like the gradient in a Euclidean space, is defined such that the directional derivative relationship is satisfied

\begin{equation}\label{eqn:maxwellsFromGA:60}
\begin{aligned}
a \cdot \grad F(x)= \lim_{\tau \rightarrow 0} \frac{F(x + a\tau) - F(x)}{\tau}
\end{aligned}
\end{equation}

One can show that this requires the definition of the gradient to be
\begin{equation}\label{eqn:maxwellsFromGA:80}
\begin{aligned}
\grad &= \inv{\gamma_\mu} \PD{x^\mu}{} \equiv \gamma^\mu \partial_\mu \\
      &= \inv{\gamma^\mu} \PD{x_\mu}{} \equiv \gamma_\mu \partial^\mu.
\end{aligned}
\end{equation}

Written out explicitly with \(x = ct \gamma_0 + x^k \gamma_k\), these partials are
\begin{equation}\label{eqn:maxwellsFromGA:100}
\begin{aligned}
\partial_0 &= \partial^0 = \inv{c} \PD{t}{} \\
\partial_k &= \PD{x^k}{} \\
\partial^k &= \PD{x_k}{} = -\partial_k \\
\end{aligned}
\end{equation}

\section{Space time split}

Pre or post multiplication by the time like basis vector \(\gamma_0\) serves to split a four vector into a scalar and spatial vector components.  With \(x = x^\mu \gamma_\mu\) we have

\begin{equation}\label{eqn:maxwellsFromGA:120}
\begin{aligned}
x \gamma_0 &= x^0 + x^k \gamma_k \gamma_0 \\
\gamma_0 x &= x^0 - x^k \gamma_k \gamma_0
\end{aligned}
\end{equation}

As these bivectors \(\gamma_k \gamma_0\) square to unity, they serve as a spatial basis.  Utilizing the Pauli matrix notation, these are written \(\sigma_k = \gamma_k \gamma_0\).  Spatial vectors in STA are denoted in boldface, so for the spacetime split

\begin{equation}\label{eqn:maxwellsFromGA:140}
\begin{aligned}
\gamma_0 x = x^0 + x^k \sigma_k,
\end{aligned}
\end{equation}

one writes \(\Bx = x^k \sigma_k\).

\section{blah}
\begin{equation}\label{eqn:maxwellsFromGA:160}
\begin{aligned}
F &= \grad \wedge A = \BE + I c \BB \\
A &= \gamma_\mu A^\mu \\
J &= \gamma_\mu J^\mu \\
I &= \gamma_0 \gamma_1 \gamma_2 \gamma_3 \\
\end{aligned}
\end{equation}

\section{Maxwell's equation in Geometric Algebra}

Like differential forms and tensor notations, Geometric Algebra allows Maxwell's equations to be united into a compact form

\begin{equation}\label{eqn:maxwellsFromGA:180}
\begin{aligned}
\grad F = J/\epsilon_0 c.
\end{aligned}
\end{equation}

\section{Basic notation}
Where
\begin{equation}\label{eqn:maxwellsFromGA:200}
\begin{aligned}
\grad &= \gamma^\mu \partial_\mu \\
F &= \grad \wedge A = \BE + I c \BB \\
A &= \gamma_\mu A^\mu \\
J &= \gamma_\mu J^\mu \\
I &= \gamma_0 \gamma_1 \gamma_2 \gamma_3 \\
\partial_0 &= \inv{c} \PD{t}{} \\
\partial_k &= \PD{x^k}{} \\
\gamma_\mu \cdot \gamma^\nu &= {\delta_\mu}^\nu
\end{aligned}
\end{equation}

With a metric choice of \(\gamma_0^2 = -\gamma_k^2 = 1\), the Geometric Algebra with basis \(\{\gamma_\mu\}\) and its associated reciprocal basis \(\{\gamma^\mu\}\) is known as the Space Time Algebra (STA).  Spatial vectors, designated with bold above, are bivectors in STA, as in the following example for the electric field

\begin{equation}\label{eqn:maxwellsFromGA:220}
\begin{aligned}
\BE &= E^k \sigma_k \\
\sigma^k = \sigma_k &= \gamma_k \gamma_0.
\end{aligned}
\end{equation}

Note that because of the mixed signature these bivectors \(\sigma_k\) all square to unity, like a unit vector in a Euclidian vector space.

\section{Recovering the traditional vector form of Maxwell's equations}

Recovery of the vector form of Maxwell's equations requires expansion of

\begin{equation}\label{eqn:maxwellsFromGA:240}
\begin{aligned}
\gamma_0 \left( \grad F - J/\epsilon_0 c \right) = 0
\end{aligned}
\end{equation}

To do so first note that
\begin{equation}\label{eqn:maxwellsFromGA:260}
\begin{aligned}
\gamma_0 \grad &= \partial_t + \spacegrad
\end{aligned}
\end{equation}

where a bold font is used for the spatial gradient.

\begin{equation}\label{eqn:maxwellsFromGA:280}
\begin{aligned}
\spacegrad &= \sigma^k \partial_k
\end{aligned}
\end{equation}

Similarily, multiplication of the four vector current density also has scalar and spatial vector components.  With \(\BJ = J^k \sigma_k\), and \(\rho = J^0/c\) this is

\begin{equation}\label{eqn:maxwellsFromGA:300}
\begin{aligned}
\gamma_0 J &= c \rho - \BJ.
\end{aligned}
\end{equation}

One obtains
\begin{equation}\label{eqn:maxwellsFromGA:320}
\begin{aligned}
\left( \inv{c} \partial_t + \spacegrad \right) (\BE + I c \BB) - \frac{\rho}{\epsilon_0} + \frac{\BJ}{\epsilon_0 c} = 0.
\end{aligned}
\end{equation}

Noting that the pseudoscalar \(I\) commutes with all spatial vectors, that \(I^2 = -1\), \(\epsilon_0 \mu_0 c^2 = 1\), and \(\spacegrad \BX = \spacegrad \cdot \BX + I \spacegrad \cross \BX\) for spatial vectors \(\BX\), one can expand and regroup this yielding

\begin{equation}\label{eqn:maxwellsFromGA:340}
\begin{aligned}
\left( \spacegrad \cdot \BE - \frac{\rho}{\epsilon_0} \right)
-
c \left( \spacegrad \cross \BB - \mu_0 \epsilon_0 \PD{t}{\BE} - \mu_0 \BJ \right)
+ \left( \spacegrad \cross \BE + \PD{t}{\BB} \right)
+ I c \left( \spacegrad \cdot \BB \right)
= 0
\end{aligned}
\end{equation}

In the spatial basis \(\{\sigma_k\}\) we have scalar, vector, bivector, and trivector grades.  Equating each to zero recovers all of Maxwell's equations in their traditional vector form.

%\EndArticle
\EndNoBibArticle
