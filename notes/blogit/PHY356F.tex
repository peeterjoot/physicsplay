%
% Copyright � 2015 Peeter Joot.  All Rights Reserved.
% Licenced as described in the file LICENSE under the root directory of this GIT repository.
%
\documentclass[]{eliblog}

\usepackage{amsmath}
\usepackage{mathpazo}

%
% shorthand for bold symbols, convenient for vectors and matrices
%
\newcommand{\Ba}[0]{\mathbf{a}}
\newcommand{\Bb}[0]{\mathbf{b}}
\newcommand{\Bc}[0]{\mathbf{c}}
\newcommand{\Bd}[0]{\mathbf{d}}
\newcommand{\Be}[0]{\mathbf{e}}
\newcommand{\Bf}[0]{\mathbf{f}}
\newcommand{\Bg}[0]{\mathbf{g}}
\newcommand{\Bh}[0]{\mathbf{h}}
\newcommand{\Bi}[0]{\mathbf{i}}
\newcommand{\Bj}[0]{\mathbf{j}}
\newcommand{\Bk}[0]{\mathbf{k}}
\newcommand{\Bl}[0]{\mathbf{l}}
\newcommand{\Bm}[0]{\mathbf{m}}
\newcommand{\Bn}[0]{\mathbf{n}}
\newcommand{\Bo}[0]{\mathbf{o}}
\newcommand{\Bp}[0]{\mathbf{p}}
\newcommand{\Bq}[0]{\mathbf{q}}
\newcommand{\Br}[0]{\mathbf{r}}
\newcommand{\Bs}[0]{\mathbf{s}}
\newcommand{\Bt}[0]{\mathbf{t}}
\newcommand{\Bu}[0]{\mathbf{u}}
\newcommand{\Bv}[0]{\mathbf{v}}
\newcommand{\Bw}[0]{\mathbf{w}}
\newcommand{\Bx}[0]{\mathbf{x}}
\newcommand{\By}[0]{\mathbf{y}}
\newcommand{\Bz}[0]{\mathbf{z}}
\newcommand{\BA}[0]{\mathbf{A}}
\newcommand{\BB}[0]{\mathbf{B}}
\newcommand{\BC}[0]{\mathbf{C}}
\newcommand{\BD}[0]{\mathbf{D}}
\newcommand{\BE}[0]{\mathbf{E}}
\newcommand{\BF}[0]{\mathbf{F}}
\newcommand{\BG}[0]{\mathbf{G}}
\newcommand{\BH}[0]{\mathbf{H}}
\newcommand{\BI}[0]{\mathbf{I}}
\newcommand{\BJ}[0]{\mathbf{J}}
\newcommand{\BK}[0]{\mathbf{K}}
\newcommand{\BL}[0]{\mathbf{L}}
\newcommand{\BM}[0]{\mathbf{M}}
\newcommand{\BN}[0]{\mathbf{N}}
\newcommand{\BO}[0]{\mathbf{O}}
\newcommand{\BP}[0]{\mathbf{P}}
\newcommand{\BQ}[0]{\mathbf{Q}}
\newcommand{\BR}[0]{\mathbf{R}}
\newcommand{\BS}[0]{\mathbf{S}}
\newcommand{\BT}[0]{\mathbf{T}}
\newcommand{\BU}[0]{\mathbf{U}}
\newcommand{\BV}[0]{\mathbf{V}}
\newcommand{\BW}[0]{\mathbf{W}}
\newcommand{\BX}[0]{\mathbf{X}}
\newcommand{\BY}[0]{\mathbf{Y}}
\newcommand{\BZ}[0]{\mathbf{Z}}

\newcommand{\Bzero}[0]{\mathbf{0}}
\newcommand{\Btheta}[0]{\boldsymbol{\theta}}
\newcommand{\Btau}[0]{\boldsymbol{\tau}}
\newcommand{\Bomega}[0]{\boldsymbol{\omega}}

%
% shorthand for unit vectors
%
\newcommand{\acap}[0]{\hat{\Ba}}
\newcommand{\bcap}[0]{\hat{\Bb}}
\newcommand{\ccap}[0]{\hat{\Bc}}
\newcommand{\dcap}[0]{\hat{\Bd}}
\newcommand{\ecap}[0]{\hat{\Be}}
\newcommand{\fcap}[0]{\hat{\Bf}}
\newcommand{\gcap}[0]{\hat{\Bg}}
\newcommand{\hcap}[0]{\hat{\Bh}}
\newcommand{\icap}[0]{\hat{\Bi}}
\newcommand{\jcap}[0]{\hat{\Bj}}
\newcommand{\kcap}[0]{\hat{\Bk}}
\newcommand{\lcap}[0]{\hat{\Bl}}
\newcommand{\mcap}[0]{\hat{\Bm}}
\newcommand{\ncap}[0]{\hat{\Bn}}
\newcommand{\ocap}[0]{\hat{\Bo}}
\newcommand{\pcap}[0]{\hat{\Bp}}
\newcommand{\qcap}[0]{\hat{\Bq}}
\newcommand{\rcap}[0]{\hat{\Br}}
\newcommand{\scap}[0]{\hat{\Bs}}
\newcommand{\tcap}[0]{\hat{\Bt}}
\newcommand{\ucap}[0]{\hat{\Bu}}
\newcommand{\vcap}[0]{\hat{\Bv}}
\newcommand{\wcap}[0]{\hat{\Bw}}
\newcommand{\xcap}[0]{\hat{\Bx}}
\newcommand{\ycap}[0]{\hat{\By}}
\newcommand{\zcap}[0]{\hat{\Bz}}
\newcommand{\thetacap}[0]{\hat{\Btheta}}

%
% to write R^n and C^n in a distinguishable fashion.  Perhaps change this
% to the double lined characters upon figuring out how to do so.
%
\newcommand{\C}[1]{$\mathbb{C}^{#1}$}
\newcommand{\R}[1]{$\mathbb{R}^{#1}$}

%
% various generally useful helpers
%

% derivative of #1 wrt. #2:
\newcommand{\D}[2] {\frac {d#2} {d#1}}

\newcommand{\inv}[1]{\frac{1}{#1}}
\newcommand{\cross}[0]{\times}

\newcommand{\abs}[1]{\lvert{#1}\rvert}
\newcommand{\norm}[1]{\lVert{#1}\rVert}
\newcommand{\innerprod}[2]{\langle{#1}, {#2}\rangle}
\newcommand{\dotprod}[2]{{#1} \cdot {#2}}
\newcommand{\bdotprod}[2]{\left({#1} \cdot {#2}\right)}
\newcommand{\crossprod}[2]{{#1} \cross {#2}}
\newcommand{\tripleprod}[3]{\dotprod{\left(\crossprod{#1}{#2}\right)}{#3}}

\DeclareMathOperator{\Proj}{Proj}
\DeclareMathOperator{\Span}{span}
\DeclareMathOperator{\Sgn}{sgn}
\DeclareMathOperator{\Area}{Area}
\DeclareMathOperator{\Volume}{Volume}

%
% A few miscellaneous things specific to this document
%
\newcommand{\crossop}[1]{\crossprod{#1}{}}

% R2 vector.
\newcommand{\VectorTwo}[2]{
\begin{bmatrix}
 {#1} \\
 {#2}
\end{bmatrix}
}

\newcommand{\VectorN}[1]{
\begin{bmatrix}
{#1}_1 \\
{#1}_2 \\
\vdots \\
{#1}_N \\
\end{bmatrix}
}

\newcommand{\DETuvij}[4]{
\begin{vmatrix}
 {#1}_{#3} & {#1}_{#4} \\
 {#2}_{#3} & {#2}_{#4}
\end{vmatrix}
}

\newcommand{\DETuvwijk}[6]{
\begin{vmatrix}
 {#1}_{#4} & {#1}_{#5} & {#1}_{#6} \\
 {#2}_{#4} & {#2}_{#5} & {#2}_{#6} \\
 {#3}_{#4} & {#3}_{#5} & {#3}_{#6}
\end{vmatrix}
}

\newcommand{\DETuvwxijkl}[8]{
\begin{vmatrix}
 {#1}_{#5} & {#1}_{#6} & {#1}_{#7} & {#1}_{#8} \\
 {#2}_{#5} & {#2}_{#6} & {#2}_{#7} & {#2}_{#8} \\
 {#3}_{#5} & {#3}_{#6} & {#3}_{#7} & {#3}_{#8} \\
 {#4}_{#5} & {#4}_{#6} & {#4}_{#7} & {#4}_{#8} \\
\end{vmatrix}
}

%\newcommand{\DETuvwxyijklm}[10]{
%\begin{vmatrix}
% {#1}_{#6} & {#1}_{#7} & {#1}_{#8} & {#1}_{#9} & {#1}_{#10} \\
% {#2}_{#6} & {#2}_{#7} & {#2}_{#8} & {#2}_{#9} & {#2}_{#10} \\
% {#3}_{#6} & {#3}_{#7} & {#3}_{#8} & {#3}_{#9} & {#3}_{#10} \\
% {#4}_{#6} & {#4}_{#7} & {#4}_{#8} & {#4}_{#9} & {#4}_{#10} \\
% {#5}_{#6} & {#5}_{#7} & {#5}_{#8} & {#5}_{#9} & {#5}_{#10}
%\end{vmatrix}
%}

% R3 vector.
\newcommand{\VectorThree}[3]{
\begin{bmatrix}
 {#1} \\
 {#2} \\
 {#3}
\end{bmatrix}
}



\author{Peeter Joot}
\email{peeter.joot@gmail.com}

%\documentclass[]{eliblogwidescreen}

\usepackage{amsmath}
\usepackage{mathpazo}

%
% shorthand for bold symbols, convenient for vectors and matrices
%
\newcommand{\Ba}[0]{\mathbf{a}}
\newcommand{\Bb}[0]{\mathbf{b}}
\newcommand{\Bc}[0]{\mathbf{c}}
\newcommand{\Bd}[0]{\mathbf{d}}
\newcommand{\Be}[0]{\mathbf{e}}
\newcommand{\Bf}[0]{\mathbf{f}}
\newcommand{\Bg}[0]{\mathbf{g}}
\newcommand{\Bh}[0]{\mathbf{h}}
\newcommand{\Bi}[0]{\mathbf{i}}
\newcommand{\Bj}[0]{\mathbf{j}}
\newcommand{\Bk}[0]{\mathbf{k}}
\newcommand{\Bl}[0]{\mathbf{l}}
\newcommand{\Bm}[0]{\mathbf{m}}
\newcommand{\Bn}[0]{\mathbf{n}}
\newcommand{\Bo}[0]{\mathbf{o}}
\newcommand{\Bp}[0]{\mathbf{p}}
\newcommand{\Bq}[0]{\mathbf{q}}
\newcommand{\Br}[0]{\mathbf{r}}
\newcommand{\Bs}[0]{\mathbf{s}}
\newcommand{\Bt}[0]{\mathbf{t}}
\newcommand{\Bu}[0]{\mathbf{u}}
\newcommand{\Bv}[0]{\mathbf{v}}
\newcommand{\Bw}[0]{\mathbf{w}}
\newcommand{\Bx}[0]{\mathbf{x}}
\newcommand{\By}[0]{\mathbf{y}}
\newcommand{\Bz}[0]{\mathbf{z}}
\newcommand{\BA}[0]{\mathbf{A}}
\newcommand{\BB}[0]{\mathbf{B}}
\newcommand{\BC}[0]{\mathbf{C}}
\newcommand{\BD}[0]{\mathbf{D}}
\newcommand{\BE}[0]{\mathbf{E}}
\newcommand{\BF}[0]{\mathbf{F}}
\newcommand{\BG}[0]{\mathbf{G}}
\newcommand{\BH}[0]{\mathbf{H}}
\newcommand{\BI}[0]{\mathbf{I}}
\newcommand{\BJ}[0]{\mathbf{J}}
\newcommand{\BK}[0]{\mathbf{K}}
\newcommand{\BL}[0]{\mathbf{L}}
\newcommand{\BM}[0]{\mathbf{M}}
\newcommand{\BN}[0]{\mathbf{N}}
\newcommand{\BO}[0]{\mathbf{O}}
\newcommand{\BP}[0]{\mathbf{P}}
\newcommand{\BQ}[0]{\mathbf{Q}}
\newcommand{\BR}[0]{\mathbf{R}}
\newcommand{\BS}[0]{\mathbf{S}}
\newcommand{\BT}[0]{\mathbf{T}}
\newcommand{\BU}[0]{\mathbf{U}}
\newcommand{\BV}[0]{\mathbf{V}}
\newcommand{\BW}[0]{\mathbf{W}}
\newcommand{\BX}[0]{\mathbf{X}}
\newcommand{\BY}[0]{\mathbf{Y}}
\newcommand{\BZ}[0]{\mathbf{Z}}

\newcommand{\Bzero}[0]{\mathbf{0}}
\newcommand{\Btheta}[0]{\boldsymbol{\theta}}
\newcommand{\Btau}[0]{\boldsymbol{\tau}}
\newcommand{\Bomega}[0]{\boldsymbol{\omega}}

%
% shorthand for unit vectors
%
\newcommand{\acap}[0]{\hat{\Ba}}
\newcommand{\bcap}[0]{\hat{\Bb}}
\newcommand{\ccap}[0]{\hat{\Bc}}
\newcommand{\dcap}[0]{\hat{\Bd}}
\newcommand{\ecap}[0]{\hat{\Be}}
\newcommand{\fcap}[0]{\hat{\Bf}}
\newcommand{\gcap}[0]{\hat{\Bg}}
\newcommand{\hcap}[0]{\hat{\Bh}}
\newcommand{\icap}[0]{\hat{\Bi}}
\newcommand{\jcap}[0]{\hat{\Bj}}
\newcommand{\kcap}[0]{\hat{\Bk}}
\newcommand{\lcap}[0]{\hat{\Bl}}
\newcommand{\mcap}[0]{\hat{\Bm}}
\newcommand{\ncap}[0]{\hat{\Bn}}
\newcommand{\ocap}[0]{\hat{\Bo}}
\newcommand{\pcap}[0]{\hat{\Bp}}
\newcommand{\qcap}[0]{\hat{\Bq}}
\newcommand{\rcap}[0]{\hat{\Br}}
\newcommand{\scap}[0]{\hat{\Bs}}
\newcommand{\tcap}[0]{\hat{\Bt}}
\newcommand{\ucap}[0]{\hat{\Bu}}
\newcommand{\vcap}[0]{\hat{\Bv}}
\newcommand{\wcap}[0]{\hat{\Bw}}
\newcommand{\xcap}[0]{\hat{\Bx}}
\newcommand{\ycap}[0]{\hat{\By}}
\newcommand{\zcap}[0]{\hat{\Bz}}
\newcommand{\thetacap}[0]{\hat{\Btheta}}

%
% to write R^n and C^n in a distinguishable fashion.  Perhaps change this
% to the double lined characters upon figuring out how to do so.
%
\newcommand{\C}[1]{$\mathbb{C}^{#1}$}
\newcommand{\R}[1]{$\mathbb{R}^{#1}$}

%
% various generally useful helpers
%

% derivative of #1 wrt. #2:
\newcommand{\D}[2] {\frac {d#2} {d#1}}

\newcommand{\inv}[1]{\frac{1}{#1}}
\newcommand{\cross}[0]{\times}

\newcommand{\abs}[1]{\lvert{#1}\rvert}
\newcommand{\norm}[1]{\lVert{#1}\rVert}
\newcommand{\innerprod}[2]{\langle{#1}, {#2}\rangle}
\newcommand{\dotprod}[2]{{#1} \cdot {#2}}
\newcommand{\bdotprod}[2]{\left({#1} \cdot {#2}\right)}
\newcommand{\crossprod}[2]{{#1} \cross {#2}}
\newcommand{\tripleprod}[3]{\dotprod{\left(\crossprod{#1}{#2}\right)}{#3}}

\DeclareMathOperator{\Proj}{Proj}
\DeclareMathOperator{\Span}{span}
\DeclareMathOperator{\Sgn}{sgn}
\DeclareMathOperator{\Area}{Area}
\DeclareMathOperator{\Volume}{Volume}

%
% A few miscellaneous things specific to this document
%
\newcommand{\crossop}[1]{\crossprod{#1}{}}

% R2 vector.
\newcommand{\VectorTwo}[2]{
\begin{bmatrix}
 {#1} \\
 {#2}
\end{bmatrix}
}

\newcommand{\VectorN}[1]{
\begin{bmatrix}
{#1}_1 \\
{#1}_2 \\
\vdots \\
{#1}_N \\
\end{bmatrix}
}

\newcommand{\DETuvij}[4]{
\begin{vmatrix}
 {#1}_{#3} & {#1}_{#4} \\
 {#2}_{#3} & {#2}_{#4}
\end{vmatrix}
}

\newcommand{\DETuvwijk}[6]{
\begin{vmatrix}
 {#1}_{#4} & {#1}_{#5} & {#1}_{#6} \\
 {#2}_{#4} & {#2}_{#5} & {#2}_{#6} \\
 {#3}_{#4} & {#3}_{#5} & {#3}_{#6}
\end{vmatrix}
}

\newcommand{\DETuvwxijkl}[8]{
\begin{vmatrix}
 {#1}_{#5} & {#1}_{#6} & {#1}_{#7} & {#1}_{#8} \\
 {#2}_{#5} & {#2}_{#6} & {#2}_{#7} & {#2}_{#8} \\
 {#3}_{#5} & {#3}_{#6} & {#3}_{#7} & {#3}_{#8} \\
 {#4}_{#5} & {#4}_{#6} & {#4}_{#7} & {#4}_{#8} \\
\end{vmatrix}
}

%\newcommand{\DETuvwxyijklm}[10]{
%\begin{vmatrix}
% {#1}_{#6} & {#1}_{#7} & {#1}_{#8} & {#1}_{#9} & {#1}_{#10} \\
% {#2}_{#6} & {#2}_{#7} & {#2}_{#8} & {#2}_{#9} & {#2}_{#10} \\
% {#3}_{#6} & {#3}_{#7} & {#3}_{#8} & {#3}_{#9} & {#3}_{#10} \\
% {#4}_{#6} & {#4}_{#7} & {#4}_{#8} & {#4}_{#9} & {#4}_{#10} \\
% {#5}_{#6} & {#5}_{#7} & {#5}_{#8} & {#5}_{#9} & {#5}_{#10}
%\end{vmatrix}
%}

% R3 vector.
\newcommand{\VectorThree}[3]{
\begin{bmatrix}
 {#1} \\
 {#2} \\
 {#3}
\end{bmatrix}
}



\author{Peeter Joot}
\email{peeter.joot@gmail.com}


\chapter{PHY356F lecture notes.}
\label{chap:PHY356F}
%\useCCL
\blogpage{http://sites.google.com/site/peeterjoot/math2010/PHY356F.pdf}
%\date{Oct X, 2010}
\revisionInfo{PHY356F.tex}

\beginArtWithToc
%\beginArtNoToc

\section{Oct 12.}

Review.  What have we learned?

\subsection{Chapter 1.}
Information about systems comes from vectors and operators.  Express the vector $\ket{\phi}$ describing the system in terms of eigenvectors $\ket{a_n}$.  $n \in 1,2,3,\cdots$.

of some operator $A$.  What are the coefficients $c_n$?  Act on both sides by $\bra{a_m}$ to find

\begin{align*}
\braket{a_m}{\phi}
&= \sum_n c_n \underbrace{\braket{a_m}{a_n}}_{\text{Kronicker delta}}  \\
&= \sum c_n \delta_{mn} \\
&= c_m
\end{align*}

\begin{align*}
c_m = \braket{a_m}{\phi}
\end{align*}

Analogy

\begin{align*}
\Bv = \sum_i v_i \Be_i
\end{align*}

\begin{align*}
\Be_1 \cdot \Bv = \sum_i v_i \Be_1 \cdot \Be_i = v_1
\end{align*}

Physical information comes from the probability for obtaining a measurement of the physical entity associated with operator $A$.  The probability of obtaining outcome $a_m$, an eigenvalue of $A$, is $\Abs{c_m}^2$

\subsection{Chapter 2.}

Deal with operators that have continuous eigenvalues and eigenvectors.

We now express

\begin{align*}
\ket{\phi} = \int dk \underbrace{f(k)}_{\text{coefficients analogous to $c_n$}} \ket{k}
\end{align*}

Now if we project onto $k'$

\begin{align*}
\braket{k'}{\phi}
&= \int dk f(k) \underbrace{\braket{k'}{k}}_{\text{Dirac delta}} \\
&= \int dk f(k) \delta(k' -k) \\
&= f(k')
\end{align*}

Unlike the discrete case, this is not a probability.  Probability density for obtaining outcome $k'$ is $\Abs{f(k')}^2$.

Example 2.
\begin{align*}
\ket{\phi} = \int dk f(k) \ket{k}
\end{align*}

Now if we project x onto both sides

\begin{align*}
\braket{x}{\phi}
&= \int dk f(k) \braket{x}{k} \\
\end{align*}

With $\braket{x}{k} = u_k(x)$

\begin{align*}
\phi(x)
&\equiv \braket{x}{\phi} \\
&= \int dk f(k) u_k(x)  \\
&= \int dk f(k) \inv{\sqrt{L}} e^{ikx}
\end{align*}

This is with periodic boundary value conditions for the normalization.  The infinite normalization is also possible.

\begin{align*}
\phi(x)
&= \inv{\sqrt{L}} \int dk f(k) e^{ikx}
\end{align*}

Multiply both sides by $e^{-ik'x}/\sqrt{L}$ and integrate.  This is analogous to multiplying $\ket{\phi} = \int f(k) \ket{k} dk$ by $\bra{k'}$.  We get

\begin{align*}
\int \phi(x) \inv{\sqrt{L}} e^{-ik'x} dx
&= \inv{L} \iint dk f(k) e^{i(k-k')x} dx \\
&= \int dk f(k) \Bigl( \inv{L} \int e^{i(k-k')x} \Bigr) \\
&= \int dk f(k) \delta(k-k') \\
&= f(k')
\end{align*}

\begin{align*}
f(k') &=
\int \phi(x) \inv{\sqrt{L}} e^{-ik'x} dx
\end{align*}

We can talk about the state vector in terms of its position basis $\phi(x)$ or in the momentum space via Fourier transformation.  This is the equivalent thing, but just expressed different.  The question of interpretation in terms of probabilities works out the same.  Either way we look at the probability density.

The quantity

\begin{align*}
\ket{\phi} = \int dk f(k) \ket{k}
\end{align*}

is also called a wave packet state since it involves a superposition of many stats $\ket{k}$.  Example: See Fig 4.1 (Gaussian wave packet, with $\Abs{\phi}^2$ as the height).  This wave packet is a snapshot of the wave function amplitude at one specific time instant.  The evolution of this wave packet is governed by the Hamiltonian, which brings us to chapter 3.

\subsection{Chapter 3.}

For
\begin{align*}
\ket{\phi} = \int dk f(k) \ket{k}
\end{align*}

How do we find $\ket{\phi(t)}$, the time evolved state?  Here we have the option of choosing which of the pictures (Schr\"{o}dinger, Heisenberg, interaction) we deal with.  Since the Heisenberg picture deals with time evolved operators, and the interaction picture with evolving Hamiltonian's, neither of these is required to answer this question.  Consider the Schr\"{o}dinger picture which gives

\begin{align*}
\ket{\phi(t)} = \int dk f(k) \ket{k} e^{-i E_k t/\hbar}
\end{align*}

where $E_k$ is the eigenvalue of the Hamiltonian operator $H$.

STRONG SEEMING HINT: If looking for additional problems and homework, consider in detail the time evolution of the Gaussian wave packet state.

\subsection{Chapter 4.}

For three dimensions with $V(x,y,z) = 0$

\begin{align*}
H &= \inv{2m} \Bp^2 \\
\Bp &= \sum_i p_i \Be_i \\
\end{align*}

In the position representation, where

\begin{align*}
p_i &= -i \hbar \frac{d}{dx_i}
\end{align*}

the Sch equation is
\begin{align*}
H u(x,y,z) &= E u(x,y,z) \\
H &= -\frac{\hbar^2}{2m} \spacegrad^2 \\
= -\frac{\hbar^2}{2m} \left(
\frac{\partial^2}{\partial {x}^2}
+\frac{\partial^2}{\partial {y}^2}
+\frac{\partial^2}{\partial {z}^2}
\right)
\end{align*}

Separation of variables assumes it is possible to let

\begin{align*}
u(x,y,z) = X(x) Y(y) Z(z)
\end{align*}

(these capital letters are functions, not operators).

\begin{align*}
-\frac{\hbar^2}{2m} \left(
YZ \frac{\partial^2 X}{\partial {x}^2}
+ XZ \frac{\partial^2 Y}{\partial {y}^2}
+ YZ \frac{\partial^2 Z}{\partial {z}^2}\right)
&= E X Y Z
\end{align*}

Dividing as usual by $XYZ$ we have

\begin{align*}
-\frac{\hbar^2}{2m} \left(
\inv{X} \frac{\partial^2 X}{\partial {x}^2}
+ \inv{Y} \frac{\partial^2 Y}{\partial {y}^2}
+ \inv{Z} \frac{\partial^2 Z}{\partial {z}^2} \right)
&= E
\end{align*}

The curious thing is that we have these three derivatives, which is supposed to be related to an Energy, which is independent of any $x,y,z$, so it must be that each of these is separately constant.  We can separate these into three individual equations

\begin{align*}
-\frac{\hbar^2}{2m} \inv{X} \frac{\partial^2 X}{\partial {x}^2} &= E_1 \\
-\frac{\hbar^2}{2m} \inv{Y} \frac{\partial^2 Y}{\partial {x}^2} &= E_2 \\
-\frac{\hbar^2}{2m} \inv{Z} \frac{\partial^2 Z}{\partial {x}^2} &= E_3
\end{align*}

or
\begin{align*}
\frac{\partial^2 X}{\partial {x}^2} &= \left( - \frac{2m E_1}{\hbar^2} \right) X  \\
\frac{\partial^2 Y}{\partial {x}^2} &= \left( - \frac{2m E_2}{\hbar^2} \right) Y  \\
\frac{\partial^2 Z}{\partial {x}^2} &= \left( - \frac{2m E_3}{\hbar^2} \right) Z
\end{align*}

We have then

\begin{align*}
X(x) = C_1 e^{i k x}
\end{align*}

with
\begin{align*}
E_1 &= \frac{\hbar^2 k_1^2 }{2m} = \frac{p_1^2}{2m} \\
E_2 &= \frac{\hbar^2 k_2^2 }{2m} = \frac{p_2^2}{2m} \\
E_3 &= \frac{\hbar^2 k_3^2 }{2m} = \frac{p_3^2}{2m}
\end{align*}

We are free to use any sort of normalization procedure we wish (periodic boundary conditions, infinite Dirac, ...)

\subsection{Angular momentum.}

HOMEWORK: go through the steps to understand how to formulate $\spacegrad^2$ in spherical polar coordinates.  This is a lot of work, but is good practice and background for dealing with the Hydrogen atom, something with spherical symmetry that is most naturally analyzed in the spherical polar coordinates.

In spherical coordinates (We won't go through this here, but it is good practice) with

\begin{align*}
x &= r \sin\theta \cos\phi \\
y &= r \sin\theta \sin\phi \\
z &= r \cos\theta
\end{align*}

we have with $u = u(r,\theta, \phi)$

\begin{align*}
-\frac{\hbar^2}{2m} \left(
\inv{r} \partial_{rr} (r u) +  \inv{r^2 \sin\theta} \partial_\theta (\sin\theta \partial_\theta u)
+ \inv{r^2 \sin^2\theta} \partial_{\phi\phi} u
 \right)
&= E u
\end{align*}

We see the start of a separation of variables attack with $u = R(r) Y(\theta, \phi)$.  We end up with

\begin{align*}
-\frac{\hbar^2}{2m} &\left(
\frac{r}{R} (r R')' +  \inv{Y \sin\theta} \partial_\theta (\sin\theta \partial_\theta Y)
+ \inv{Y \sin^2\theta} \partial_{\phi\phi} Y
 \right) \\
\end{align*}

\begin{align*}
r (r R')' + \left( \frac{2m E}{\hbar^2} r^2 - \lambda \right) R &= 0
\end{align*}
\begin{align*}
\inv{Y \sin\theta} \partial_\theta (\sin\theta \partial_\theta Y) + \inv{Y \sin^2\theta} \partial_{\phi\phi} Y &= -\lambda
\end{align*}

Application of separation of variables again, with $Y = P(\theta) Q(\phi)$ gives us

\begin{align*}
\inv{P \sin\theta} \partial_\theta (\sin\theta \partial_\theta P) + \inv{Q \sin^2\theta} \partial_{\phi\phi} Q &= -\lambda
\end{align*}

\begin{align*}
\frac{\sin\theta}{P } \partial_\theta (\sin\theta \partial_\theta P)
+\lambda  \sin^2\theta
+ \inv{Q } \partial_{\phi\phi} Q &= 0
\end{align*}

\begin{align*}
\frac{\sin\theta}{P } \partial_\theta (\sin\theta \partial_\theta P) + \lambda \sin^2\theta - \mu = 0
\inv{Q } \partial_{\phi\phi} Q &= -\mu
\end{align*}

or
\begin{align}\label{eqn:PHY356F:1000}
\frac{1}{P \sin\theta} \partial_\theta (\sin\theta \partial_\theta P) +\lambda -\frac{\mu}{\sin^2\theta} &= 0
\end{align}
\begin{align}\label{eqn:PHY356F:2000}
\partial_{\phi\phi} Q &= -\mu Q
\end{align}

The equation for $P$ can be solved using the Legendre function $P_l^m(\cos\theta)$ where $\lambda = l(l+1)$ and $l$ is an integer

Replacing $\mu$ with $m^2$, where $m$ is an integer

\begin{align*}
\frac{d^2 Q}{d\phi^2} &= -m^2 Q
\end{align*}

Imposing a periodic boundary condition $Q(\phi) = Q(\phi + 2\pi)$, where ($m = 0, \pm 1, \pm 2, \cdots$) we have

\begin{align*}
Q &= \inv{\sqrt{2\pi}} e^{im\phi}
\end{align*}

There is the overall solution $r(r,\theta,\phi) = R(r) Y(\theta, \phi)$ for a free particle.  The functions $Y(\theta, \phi)$ are

\begin{align*}
Y_{lm}(\theta, \phi)
&= N \left( \inv{\sqrt{2\pi}} e^{im\phi} \right) \underbrace{ P_l^m(\cos\theta) }_{ -l \le m \le l }
\end{align*}

where $N$ is a normalization constant, and $m = 0, \pm 1, \pm 2, \cdots$.  $Y_{lm}$ is an eigenstate of the $\BL^2$ operator and $L_z$ (two for the price of one).  There's no specific reason for the direction $z$, but it is the direction picked out of convention.

Angular momentum is given by

\begin{align*}
\BL = \Br \cross \Bp
\end{align*}

where

\begin{align*}
\BR = x \xcap + y\ycap + z\zcap
\end{align*}

and
\begin{align*}
\Bp = p_x \xcap + p_y\ycap + p_z\zcap
\end{align*}

The important thing to remember is that the aim of following all the math is to show that

\begin{align*}
\BL^2 Y_{lm} = \hbar^2 l (l+1) Y_{lm}
\end{align*}

and simultaneously

\begin{align*}
\BL_z Y_{lm} = \hbar m Y_{lm}
\end{align*}

Part of the solution involves working with $\antisymmetric{L_z}{L_{+}}$, and $\antisymmetric{L_z}{L_{-}}$, where

\begin{align*}
L_{+} &= L_x + i L_y \\
L_{-} &= L_x - i L_y
\end{align*}

An exercise (not in the book) is to evaluate
\begin{align}\label{eqn:PHY356F:4000}
\antisymmetric{L_z}{L_{+}}
&= L_z L_x + i L_z L_y - L_x L_z - i L_y L_z
\end{align}

where
\begin{align}\label{eqn:PHY356F:5000}
\antisymmetric{L_x}{L_y}  &= i \hbar L_z \\
\antisymmetric{L_y}{L_z}  &= i \hbar L_x \\
\antisymmetric{L_z}{L_x}  &= i \hbar L_y
\end{align}

Substitution back in \ref{eqn:PHY356F:4000} we have

\begin{align*}
\antisymmetric{L_z}{L_{+}}
&=
\antisymmetric{L_z}{L_x}
+ i \antisymmetric{L_z}{L_y}  \\
&=
i \hbar ( L_y - i L_x ) \\
&=
\hbar ( i L_y +  L_x ) \\
&=
\hbar L_{+}
\end{align*}

\section{Oct 19.}

Last time, we started thinking about angular momentum.  This time, we will examine orbital and intrinsic angular momentum.

Orbital angular momentum in classical physics and quantum physics is expressed as

\begin{align}\label{eqn:PHY356Foct19:1000}
\BL &= \Br \cross \Bp,
\end{align}

and
\begin{align}\label{eqn:PHY356Foct19:1001}
\BL &= \BR \cross \BP,
\end{align}

where $\BR$ and $\BP$ are quantum mechanical operators corresponding to position and momentum

\begin{align}\label{eqn:PHY356Foct19:1002}
\BR &= X \xcap + Y \ycap + Z \zcap \\
\BP &= P_x \xcap + P_y \ycap + P_z \zcap \\
\BL &= L_x \xcap + L_y \ycap + L_z \zcap
\end{align}

Practice problems:
\begin{itemize}
\item a) Determine the commutators $\antisymmetric{L_x}{L_y}, \antisymmetric{L_y}{L_z}, \antisymmetric{L_z}{L_x}$ and

\begin{align*}
\antisymmetric{L_x}{L_y}
&=
(r_y p_z -r_z p_y)
(r_z p_x -r_x p_z)
-
(r_z p_x -r_x p_z)
(r_y p_z -r_z p_y) \\
&=
r_y p_z (r_z p_x -r_x p_z)
-r_z p_y (r_z p_x -r_x p_z)
- r_z p_x (r_y p_z -r_z p_y)
+ r_x p_z (r_y p_z -r_z p_y) \\
&=
r_y p_z r_z p_x
-r_y p_z r_x p_z
-r_z p_y r_z p_x
+r_z p_y r_x p_z
- r_z p_x r_y p_z
+ r_z p_x r_z p_y
+ r_x p_z r_y p_z
- r_x p_z r_z p_y \\
\end{align*}

With $p_i r_j = r_j p_i - i \hbar \delta_{ij}$, we have

\begin{align*}
\antisymmetric{L_x}{L_y}
&=
r_y r_z p_z p_x
-r_y r_z p_x p_z
-r_z r_y p_z p_x
+r_z r_y p_x p_z
- r_z r_x p_y p_z
+ r_z r_x p_z p_y
+ r_x r_z p_y p_z
- r_x r_z p_z p_y \\
&+
-i \hbar 
\left(
%r_y p_z
%+r_z p_x
%-r_z p_z
%+ r_z p_z
%- r_z p_y
%- r_x p_z
r_y p_x
- r_x p_y 
\right)
\end{align*}

Since the $p_i p_j$ operators commute, all the first terms cancel, leaving just

\begin{align*}
\antisymmetric{L_x}{L_y}
&=i \hbar L_z
\end{align*}

%Oops.  Mistake above to fix.  Much easier to do this problem in abstract index notation.

\item b) $L_z$ in spherical coordinates.

The answer is
\begin{align}\label{eqn:PHY356Foct19:1003}
L_z \leftrightarrow -i \hbar \PD{\phi}{}
\end{align}

Work through this.
\end{itemize}

Part of the task in this intro QM treatment is to carefully determine the eigenfunctions for these operators.

The spherical harmonics are given by $Y_{lm}(\theta, \phi)$ such that

\begin{align}\label{eqn:PHY356Foct19:1004}
Y_{lm}(\theta, \phi) \propto e^{i m \phi}
\end{align}

\begin{align*}
L_z Y_{lm}(\theta, \phi)
&= -i \hbar \PD{\phi}{} Y_{lm}(\theta, \phi) \\
&= -i \hbar \PD{\phi}{} \text{constants} (e^{im \phi}) \\
&= \hbar m \text{constants} e^{i m \phi} \\
&= \hbar m Y_{lm}(\theta, \phi)
\end{align*}

The z-component is quantized since, $m$ is an integer $m = 0, \pm 1, \pm 2, ...$.  The total angular momentum

\begin{align}\label{eqn:PHY356Foct19:1005}
\BL^2 = \BL \cdot \BL = L_x^2 + L_y^2 + L_z^2
\end{align}

is also quantized (details in the book).

The eigenvalue properties here represent very important physical features.  This is also important in the hydrogen atom problem.  In the hydrogen atom problem, the particle is effectively free in the angular components, having only $r$ dependence.  This allows us to apply the work for the free particle to our subsequent potential bounded solution.

Note that for the above, we also have the alternate, abstract ket notation, method of writing the eigenvalue behavior.
\begin{align}\label{eqn:PHY356Foct19:1006}
L_z \ket{lm} = \hbar m \ket{lm}
\end{align}

Just like
\begin{align}\label{eqn:PHY356Foct19:1007}
X \ket{x} &= x \ket{x} \\
P \ket{p} &= p \ket{p}
\end{align}

For the total angular momentum our spherical harmonic eigenfunctions have the property

\begin{align}\label{eqn:PHY356Foct19:1008}
\BL^2 \ket{lm} &= \hbar^2 l (l + 1)\ket{l m}
\end{align}

with $l = 0, 1, 2, \cdots$.

Alternately in plain old non-abstract notation we can write this as
\begin{align}\label{eqn:PHY356Foct19:1009}
\BL^2 Y_{lm}(\theta, \phi) &= \hbar^2 l (l + 1) Y_{lm}(\theta, \phi)
\end{align}

Now we can introduce the Raising and Lowering Operators, which are

\begin{align}\label{eqn:PHY356Foct19:1010}
L_{+} &= L_x + i L_y \\
L_{-} &= L_x - i L_y,
\end{align}

respectively.  These are abstract quantities, but also physically important since they relate quantum levels of the angular momentum.  How do we show this?

Last time, we saw that
\begin{align}\label{eqn:PHY356Foct19:1011}
\antisymmetric{L_z}{L_{+}} &= +\hbar L_{+} \\
\antisymmetric{L_z}{L_{-}} &= -\hbar L_{-}
\end{align}

Note that it is implied that we are operating on ket vectors

\begin{align*}
L_z (L_{-} \ket{lm} )
\end{align*}

with
\begin{align}\label{eqn:PHY356Foct19:1012}
\ket{lm} \leftrightarrow Y_{lm}(\theta, \phi)
\end{align}

Question: What is $L_{-} \ket{lm}$?

Substitute
\begin{align*}
L_z L_{-} - L_{-} L_z &= - \hbar L_{-} \\
\implies \\
L_z L_{-} &= L_{-} L_z - \hbar L_{-}
\end{align*}

\begin{align*}
L_z \left( L_{-} \ket{lm} \right)
&=
L_{-} L_z \ket{lm} - \hbar L_{-} \ket{lm} \\
&=
L_{-} m \hbar \ket{lm} - L_{-} \ket{lm} \\
&=
\hbar \left( m L_{-} \ket{lm} - L_{-} \ket{lm} \right) \\
&=
\hbar (m-1) \left( L_{-} \ket{lm} \right)
\end{align*}

So $L_{-} \ket{lm} = \ket{\psi}$ is another spherical harmonic, and we have

\begin{align}\label{eqn:PHY356Foct19:1013}
L_z \ket{\psi} &= \hbar (m-1) \ket{\psi}
\end{align}

This lowering operator quantity causes a physical change in the state of the system, lowering the observable state (ie: the eigenvalue) by $\hbar$.

Now we want to normalize $\ket{\psi} = L_{-} \ket{lm}$, via $\braket{\psi}{\psi} = 1$.

\begin{align*}
1
&= \braket{\psi}{\psi} \\
&= \bra{lm} L_{-}^\dagger L_{-} \ket{\psi} \\
&= \bra{lm} L_{+} L_{-} \ket{\psi}
\end{align*}

We can use
\begin{align}\label{eqn:PHY356Foct19:1014}
L_{+} L_{-} = \BL^2 - L_z^2 + \hbar L_z,
\end{align}

So, knowing (how exactly?) that

\begin{align}\label{eqn:PHY356Foct19:1015}
L_{-} \ket{lm} = C \ket{l,m-1}
\end{align}

we have from \ref{eqn:PHY356Foct19:1014}

\begin{align*}
\Abs{C}^2
&= \bra{lm} (\BL^2 - L_z^2 + \hbar L_z ) \ket{\psi}  \\
&= \underbrace{\braket{lm}{lm}}_{=1} \left(\hbar^2 l(l+1) - (\hbar m)^2 + \hbar^2 m \right)  \\
&= \hbar^2 \left(l(l+1) - m^2 + m \right).
\end{align*}

we have
\begin{align}\label{eqn:PHY356Foct19:1016}
\Abs{C}^2 \underbrace{\braket{l,m-1}{l,m-1}}_{=1}
&= \hbar^2 \left(l(l+1) - m^2 + m \right).
\end{align}

and can normalize the functions $\ket{\psi}$ as

\begin{align}\label{eqn:PHY356Foct19:1017}
L_{-} \ket{lm} &= \hbar \left(l(l+1) - m^2 + m \right)^{1/2} \ket{l, m-1}
\end{align}

Abstract notation side note:

\begin{align}\label{eqn:PHY356Foct19:1018}
\braket{\theta,\phi}{lm} = Y_{lm}(\theta, \phi)
\end{align}

\subsection{Generalizing orbital angular momentum.}

To explain the results of the Stern-Gerlach experiment, assume that there is an intrinsic angular momentum $\BS$ that has most of the same properties as $\BL$.  But $\BS$ has no classical counterpart such as $\Br \cross \Bp$.

This experiment is the classic QM experiment because the silver atoms not only have the orbital angular momentum, but also have an additional observed intrinsic spin in the outermost electron.  In turns out that if you combine relativity and QM, you can get out something that looks like the the $\BS$ operator.  That marriage produces the Dirac electron theory.

We assume the commutation relations

\begin{align}\label{eqn:PHY356Foct19:2000}
\antisymmetric{S_x}{S_y} &= i \hbar S_z \\
\antisymmetric{S_y}{S_z} &= i \hbar S_x \\
\antisymmetric{S_z}{S_x} &= i \hbar S_y
\end{align}

Where we have the analogous eigenproperties

\begin{align}\label{eqn:PHY356Foct19:2001}
\BS^2 \ket{sm} &= \hbar^2 s(s+1) \ket{sm} \\
S_z \ket{sm} &= \hbar m \ket{sm}
\end{align}

with $s = 0, 1/2, 1, 3/2, ...$

Electrons and protons are examples of particles that have spin one half.

Note that there is no position representation of $\ket{sm}$.  We cannot project these states.

This basic quantum mechanics end up applying to quantum computing and cryptography as well, when we apply the mathematics we are learning here to explain the Stern-Gerlach experiment to photon spin states.

(DRAWS Stern-Gerlach picture with spin up and down labeled $\ket{z+}$, and $\ket{z-}$ with the magnetic field oriented in along the $z$ axis.)

Silver atoms have $s = 1/2$ and $m= \pm 1/2$, where $m$ is the quantum number associated with the z-direction intrinsic angular momentum.  The angular momentum that is being acted on in the Stern-Gerlach experiment is primarily due to the outermost electron.

\begin{align}\label{eqn:PHY356Foct19:2005}
S_z \ket{z+} &= \frac{\hbar}{2} \ket{z+} \\
S_z \ket{z-} &= -\frac{\hbar}{2} \ket{z-} \\
\BS^2 \ket{z\pm} &= \inv{2} \left( \inv{2} + 1 \right) \hbar^2 \ket{z\pm}
\end{align}

where
\begin{align}\label{eqn:PHY356Foct19:2006}
\ket{z+} &= \ket{ \inv{2} \inv{2} } \\
\ket{z-} &= \ket{ \inv{2} -\inv{2} }
\end{align}

%You can imagine p

What about $S_x$?  We can leave the detector in the $x,z$ plane, but rotate the magnet so that it lies in the $x$ direction.  

We have the correspondence

\begin{align}\label{eqn:PHY356Foct19:2007}
S_z \leftrightarrow \frac{\hbar}{2} \PauliX,
\end{align}

but this is perhaps more properly viewed as the matrix representation of the less specific form

\begin{align}\label{eqn:PHY356Foct19:2008}
S_z = \frac{\hbar}{2} \left(
\ket{z+} \bra{z+}
-\ket{z-} \bra{z-}
\right).
\end{align}

Where the translation to the form of \ref{eqn:PHY356Foct19:2007} is via the matrix elements

\begin{align}\label{eqn:PHY356Foct19:2020}
&\bra{z+} S_z \ket{z+} \\
&\bra{z+} S_z \ket{z-} \\
&\bra{z-} S_z \ket{z+} \\
&\bra{z-} S_z \ket{z-}.
\end{align}

We can work out the same for $S_x$ using $S_{+}$ and $S_{-}$, or equivalently for $\sigma_x$ using $\sigma_{+}$ and $\sigma_{-}$, where

\begin{align}\label{eqn:PHY356Foct19:2009}
S_x &= \frac{\hbar}{2} \sigma_x \\
S_y &= \frac{\hbar}{2} \sigma_y \\
S_z &= \frac{\hbar}{2} \sigma_z
\end{align}

The operators $\sigma_x, \sigma_y, \sigma_z$ are the Pauli operators, and avoid the pesky $\hbar/2$ factors.

We find

\begin{align}\label{eqn:PHY356Foct19:2010}
\sigma_x &= \PauliX \\
\sigma_y &= \PauliY \\
\sigma_z &= \PauliZ
\end{align}

And from $\Abs{\sigma_x - \lambda I} = (-\lambda)^2 -1$, we have eigenvalues $\lambda = \pm 1$ for the $\sigma_x$ operator.

The corresponding eigenkets in column matrix notation are found 

\begin{align*}
\begin{bmatrix}
\mp 1 & 1 \\
1 & \mp 1 
\end{bmatrix}
\begin{bmatrix}
a_1 \\
a_2 
\end{bmatrix}
&= 0 \\
\implies
\mp a_1 + a_2 &= 0 \\
\implies
a_2 &= \pm a_1
\end{align*}

Or
\begin{align*}
\ket{x\pm} \propto 
\begin{bmatrix}
a_1 \\
a_2 
\end{bmatrix}
=
a_1
\begin{bmatrix}
1 \\
\pm 1 
\end{bmatrix}
\end{align*}

which can be normalized as
\begin{align}\label{eqn:PHY356Foct19:2011}
\ket{x\pm} =
\inv{\sqrt{2}}
\begin{bmatrix}
1 \\
\pm 1 
\end{bmatrix}
\end{align}

We see that this is different from

\begin{align}\label{eqn:PHY356Foct19:2012}
\ket{z+} = 
\begin{bmatrix}
1 \\
0
\end{bmatrix}
\end{align}

We will still end up with two spots, but there has been a projection of spin in a different fashion?  Does this mean the measurement will be different.  There's still a lot more to learn before understanding exactly how to relate the spin operators to a real physical system.

%\EndArticle
\EndNoBibArticle
