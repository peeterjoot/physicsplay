%
% Copyright � 2015 Peeter Joot.  All Rights Reserved.
% Licenced as described in the file LICENSE under the root directory of this GIT repository.
%
\documentclass[]{eliblog}

\usepackage{amsmath}
\usepackage{mathpazo}

%
% shorthand for bold symbols, convenient for vectors and matrices
%
\newcommand{\Ba}[0]{\mathbf{a}}
\newcommand{\Bb}[0]{\mathbf{b}}
\newcommand{\Bc}[0]{\mathbf{c}}
\newcommand{\Bd}[0]{\mathbf{d}}
\newcommand{\Be}[0]{\mathbf{e}}
\newcommand{\Bf}[0]{\mathbf{f}}
\newcommand{\Bg}[0]{\mathbf{g}}
\newcommand{\Bh}[0]{\mathbf{h}}
\newcommand{\Bi}[0]{\mathbf{i}}
\newcommand{\Bj}[0]{\mathbf{j}}
\newcommand{\Bk}[0]{\mathbf{k}}
\newcommand{\Bl}[0]{\mathbf{l}}
\newcommand{\Bm}[0]{\mathbf{m}}
\newcommand{\Bn}[0]{\mathbf{n}}
\newcommand{\Bo}[0]{\mathbf{o}}
\newcommand{\Bp}[0]{\mathbf{p}}
\newcommand{\Bq}[0]{\mathbf{q}}
\newcommand{\Br}[0]{\mathbf{r}}
\newcommand{\Bs}[0]{\mathbf{s}}
\newcommand{\Bt}[0]{\mathbf{t}}
\newcommand{\Bu}[0]{\mathbf{u}}
\newcommand{\Bv}[0]{\mathbf{v}}
\newcommand{\Bw}[0]{\mathbf{w}}
\newcommand{\Bx}[0]{\mathbf{x}}
\newcommand{\By}[0]{\mathbf{y}}
\newcommand{\Bz}[0]{\mathbf{z}}
\newcommand{\BA}[0]{\mathbf{A}}
\newcommand{\BB}[0]{\mathbf{B}}
\newcommand{\BC}[0]{\mathbf{C}}
\newcommand{\BD}[0]{\mathbf{D}}
\newcommand{\BE}[0]{\mathbf{E}}
\newcommand{\BF}[0]{\mathbf{F}}
\newcommand{\BG}[0]{\mathbf{G}}
\newcommand{\BH}[0]{\mathbf{H}}
\newcommand{\BI}[0]{\mathbf{I}}
\newcommand{\BJ}[0]{\mathbf{J}}
\newcommand{\BK}[0]{\mathbf{K}}
\newcommand{\BL}[0]{\mathbf{L}}
\newcommand{\BM}[0]{\mathbf{M}}
\newcommand{\BN}[0]{\mathbf{N}}
\newcommand{\BO}[0]{\mathbf{O}}
\newcommand{\BP}[0]{\mathbf{P}}
\newcommand{\BQ}[0]{\mathbf{Q}}
\newcommand{\BR}[0]{\mathbf{R}}
\newcommand{\BS}[0]{\mathbf{S}}
\newcommand{\BT}[0]{\mathbf{T}}
\newcommand{\BU}[0]{\mathbf{U}}
\newcommand{\BV}[0]{\mathbf{V}}
\newcommand{\BW}[0]{\mathbf{W}}
\newcommand{\BX}[0]{\mathbf{X}}
\newcommand{\BY}[0]{\mathbf{Y}}
\newcommand{\BZ}[0]{\mathbf{Z}}

\newcommand{\Bzero}[0]{\mathbf{0}}
\newcommand{\Btheta}[0]{\boldsymbol{\theta}}
\newcommand{\Btau}[0]{\boldsymbol{\tau}}
\newcommand{\Bomega}[0]{\boldsymbol{\omega}}

%
% shorthand for unit vectors
%
\newcommand{\acap}[0]{\hat{\Ba}}
\newcommand{\bcap}[0]{\hat{\Bb}}
\newcommand{\ccap}[0]{\hat{\Bc}}
\newcommand{\dcap}[0]{\hat{\Bd}}
\newcommand{\ecap}[0]{\hat{\Be}}
\newcommand{\fcap}[0]{\hat{\Bf}}
\newcommand{\gcap}[0]{\hat{\Bg}}
\newcommand{\hcap}[0]{\hat{\Bh}}
\newcommand{\icap}[0]{\hat{\Bi}}
\newcommand{\jcap}[0]{\hat{\Bj}}
\newcommand{\kcap}[0]{\hat{\Bk}}
\newcommand{\lcap}[0]{\hat{\Bl}}
\newcommand{\mcap}[0]{\hat{\Bm}}
\newcommand{\ncap}[0]{\hat{\Bn}}
\newcommand{\ocap}[0]{\hat{\Bo}}
\newcommand{\pcap}[0]{\hat{\Bp}}
\newcommand{\qcap}[0]{\hat{\Bq}}
\newcommand{\rcap}[0]{\hat{\Br}}
\newcommand{\scap}[0]{\hat{\Bs}}
\newcommand{\tcap}[0]{\hat{\Bt}}
\newcommand{\ucap}[0]{\hat{\Bu}}
\newcommand{\vcap}[0]{\hat{\Bv}}
\newcommand{\wcap}[0]{\hat{\Bw}}
\newcommand{\xcap}[0]{\hat{\Bx}}
\newcommand{\ycap}[0]{\hat{\By}}
\newcommand{\zcap}[0]{\hat{\Bz}}
\newcommand{\thetacap}[0]{\hat{\Btheta}}

%
% to write R^n and C^n in a distinguishable fashion.  Perhaps change this
% to the double lined characters upon figuring out how to do so.
%
\newcommand{\C}[1]{$\mathbb{C}^{#1}$}
\newcommand{\R}[1]{$\mathbb{R}^{#1}$}

%
% various generally useful helpers
%

% derivative of #1 wrt. #2:
\newcommand{\D}[2] {\frac {d#2} {d#1}}

\newcommand{\inv}[1]{\frac{1}{#1}}
\newcommand{\cross}[0]{\times}

\newcommand{\abs}[1]{\lvert{#1}\rvert}
\newcommand{\norm}[1]{\lVert{#1}\rVert}
\newcommand{\innerprod}[2]{\langle{#1}, {#2}\rangle}
\newcommand{\dotprod}[2]{{#1} \cdot {#2}}
\newcommand{\bdotprod}[2]{\left({#1} \cdot {#2}\right)}
\newcommand{\crossprod}[2]{{#1} \cross {#2}}
\newcommand{\tripleprod}[3]{\dotprod{\left(\crossprod{#1}{#2}\right)}{#3}}

\DeclareMathOperator{\Proj}{Proj}
\DeclareMathOperator{\Span}{span}
\DeclareMathOperator{\Sgn}{sgn}
\DeclareMathOperator{\Area}{Area}
\DeclareMathOperator{\Volume}{Volume}

%
% A few miscellaneous things specific to this document
%
\newcommand{\crossop}[1]{\crossprod{#1}{}}

% R2 vector.
\newcommand{\VectorTwo}[2]{
\begin{bmatrix}
 {#1} \\
 {#2}
\end{bmatrix}
}

\newcommand{\VectorN}[1]{
\begin{bmatrix}
{#1}_1 \\
{#1}_2 \\
\vdots \\
{#1}_N \\
\end{bmatrix}
}

\newcommand{\DETuvij}[4]{
\begin{vmatrix}
 {#1}_{#3} & {#1}_{#4} \\
 {#2}_{#3} & {#2}_{#4}
\end{vmatrix}
}

\newcommand{\DETuvwijk}[6]{
\begin{vmatrix}
 {#1}_{#4} & {#1}_{#5} & {#1}_{#6} \\
 {#2}_{#4} & {#2}_{#5} & {#2}_{#6} \\
 {#3}_{#4} & {#3}_{#5} & {#3}_{#6}
\end{vmatrix}
}

\newcommand{\DETuvwxijkl}[8]{
\begin{vmatrix}
 {#1}_{#5} & {#1}_{#6} & {#1}_{#7} & {#1}_{#8} \\
 {#2}_{#5} & {#2}_{#6} & {#2}_{#7} & {#2}_{#8} \\
 {#3}_{#5} & {#3}_{#6} & {#3}_{#7} & {#3}_{#8} \\
 {#4}_{#5} & {#4}_{#6} & {#4}_{#7} & {#4}_{#8} \\
\end{vmatrix}
}

%\newcommand{\DETuvwxyijklm}[10]{
%\begin{vmatrix}
% {#1}_{#6} & {#1}_{#7} & {#1}_{#8} & {#1}_{#9} & {#1}_{#10} \\
% {#2}_{#6} & {#2}_{#7} & {#2}_{#8} & {#2}_{#9} & {#2}_{#10} \\
% {#3}_{#6} & {#3}_{#7} & {#3}_{#8} & {#3}_{#9} & {#3}_{#10} \\
% {#4}_{#6} & {#4}_{#7} & {#4}_{#8} & {#4}_{#9} & {#4}_{#10} \\
% {#5}_{#6} & {#5}_{#7} & {#5}_{#8} & {#5}_{#9} & {#5}_{#10}
%\end{vmatrix}
%}

% R3 vector.
\newcommand{\VectorThree}[3]{
\begin{bmatrix}
 {#1} \\
 {#2} \\
 {#3}
\end{bmatrix}
}



\author{Peeter Joot}
\email{peeter.joot@gmail.com}

%\documentclass[]{eliblogwidescreen}

\usepackage{amsmath}
\usepackage{mathpazo}

%
% shorthand for bold symbols, convenient for vectors and matrices
%
\newcommand{\Ba}[0]{\mathbf{a}}
\newcommand{\Bb}[0]{\mathbf{b}}
\newcommand{\Bc}[0]{\mathbf{c}}
\newcommand{\Bd}[0]{\mathbf{d}}
\newcommand{\Be}[0]{\mathbf{e}}
\newcommand{\Bf}[0]{\mathbf{f}}
\newcommand{\Bg}[0]{\mathbf{g}}
\newcommand{\Bh}[0]{\mathbf{h}}
\newcommand{\Bi}[0]{\mathbf{i}}
\newcommand{\Bj}[0]{\mathbf{j}}
\newcommand{\Bk}[0]{\mathbf{k}}
\newcommand{\Bl}[0]{\mathbf{l}}
\newcommand{\Bm}[0]{\mathbf{m}}
\newcommand{\Bn}[0]{\mathbf{n}}
\newcommand{\Bo}[0]{\mathbf{o}}
\newcommand{\Bp}[0]{\mathbf{p}}
\newcommand{\Bq}[0]{\mathbf{q}}
\newcommand{\Br}[0]{\mathbf{r}}
\newcommand{\Bs}[0]{\mathbf{s}}
\newcommand{\Bt}[0]{\mathbf{t}}
\newcommand{\Bu}[0]{\mathbf{u}}
\newcommand{\Bv}[0]{\mathbf{v}}
\newcommand{\Bw}[0]{\mathbf{w}}
\newcommand{\Bx}[0]{\mathbf{x}}
\newcommand{\By}[0]{\mathbf{y}}
\newcommand{\Bz}[0]{\mathbf{z}}
\newcommand{\BA}[0]{\mathbf{A}}
\newcommand{\BB}[0]{\mathbf{B}}
\newcommand{\BC}[0]{\mathbf{C}}
\newcommand{\BD}[0]{\mathbf{D}}
\newcommand{\BE}[0]{\mathbf{E}}
\newcommand{\BF}[0]{\mathbf{F}}
\newcommand{\BG}[0]{\mathbf{G}}
\newcommand{\BH}[0]{\mathbf{H}}
\newcommand{\BI}[0]{\mathbf{I}}
\newcommand{\BJ}[0]{\mathbf{J}}
\newcommand{\BK}[0]{\mathbf{K}}
\newcommand{\BL}[0]{\mathbf{L}}
\newcommand{\BM}[0]{\mathbf{M}}
\newcommand{\BN}[0]{\mathbf{N}}
\newcommand{\BO}[0]{\mathbf{O}}
\newcommand{\BP}[0]{\mathbf{P}}
\newcommand{\BQ}[0]{\mathbf{Q}}
\newcommand{\BR}[0]{\mathbf{R}}
\newcommand{\BS}[0]{\mathbf{S}}
\newcommand{\BT}[0]{\mathbf{T}}
\newcommand{\BU}[0]{\mathbf{U}}
\newcommand{\BV}[0]{\mathbf{V}}
\newcommand{\BW}[0]{\mathbf{W}}
\newcommand{\BX}[0]{\mathbf{X}}
\newcommand{\BY}[0]{\mathbf{Y}}
\newcommand{\BZ}[0]{\mathbf{Z}}

\newcommand{\Bzero}[0]{\mathbf{0}}
\newcommand{\Btheta}[0]{\boldsymbol{\theta}}
\newcommand{\Btau}[0]{\boldsymbol{\tau}}
\newcommand{\Bomega}[0]{\boldsymbol{\omega}}

%
% shorthand for unit vectors
%
\newcommand{\acap}[0]{\hat{\Ba}}
\newcommand{\bcap}[0]{\hat{\Bb}}
\newcommand{\ccap}[0]{\hat{\Bc}}
\newcommand{\dcap}[0]{\hat{\Bd}}
\newcommand{\ecap}[0]{\hat{\Be}}
\newcommand{\fcap}[0]{\hat{\Bf}}
\newcommand{\gcap}[0]{\hat{\Bg}}
\newcommand{\hcap}[0]{\hat{\Bh}}
\newcommand{\icap}[0]{\hat{\Bi}}
\newcommand{\jcap}[0]{\hat{\Bj}}
\newcommand{\kcap}[0]{\hat{\Bk}}
\newcommand{\lcap}[0]{\hat{\Bl}}
\newcommand{\mcap}[0]{\hat{\Bm}}
\newcommand{\ncap}[0]{\hat{\Bn}}
\newcommand{\ocap}[0]{\hat{\Bo}}
\newcommand{\pcap}[0]{\hat{\Bp}}
\newcommand{\qcap}[0]{\hat{\Bq}}
\newcommand{\rcap}[0]{\hat{\Br}}
\newcommand{\scap}[0]{\hat{\Bs}}
\newcommand{\tcap}[0]{\hat{\Bt}}
\newcommand{\ucap}[0]{\hat{\Bu}}
\newcommand{\vcap}[0]{\hat{\Bv}}
\newcommand{\wcap}[0]{\hat{\Bw}}
\newcommand{\xcap}[0]{\hat{\Bx}}
\newcommand{\ycap}[0]{\hat{\By}}
\newcommand{\zcap}[0]{\hat{\Bz}}
\newcommand{\thetacap}[0]{\hat{\Btheta}}

%
% to write R^n and C^n in a distinguishable fashion.  Perhaps change this
% to the double lined characters upon figuring out how to do so.
%
\newcommand{\C}[1]{$\mathbb{C}^{#1}$}
\newcommand{\R}[1]{$\mathbb{R}^{#1}$}

%
% various generally useful helpers
%

% derivative of #1 wrt. #2:
\newcommand{\D}[2] {\frac {d#2} {d#1}}

\newcommand{\inv}[1]{\frac{1}{#1}}
\newcommand{\cross}[0]{\times}

\newcommand{\abs}[1]{\lvert{#1}\rvert}
\newcommand{\norm}[1]{\lVert{#1}\rVert}
\newcommand{\innerprod}[2]{\langle{#1}, {#2}\rangle}
\newcommand{\dotprod}[2]{{#1} \cdot {#2}}
\newcommand{\bdotprod}[2]{\left({#1} \cdot {#2}\right)}
\newcommand{\crossprod}[2]{{#1} \cross {#2}}
\newcommand{\tripleprod}[3]{\dotprod{\left(\crossprod{#1}{#2}\right)}{#3}}

\DeclareMathOperator{\Proj}{Proj}
\DeclareMathOperator{\Span}{span}
\DeclareMathOperator{\Sgn}{sgn}
\DeclareMathOperator{\Area}{Area}
\DeclareMathOperator{\Volume}{Volume}

%
% A few miscellaneous things specific to this document
%
\newcommand{\crossop}[1]{\crossprod{#1}{}}

% R2 vector.
\newcommand{\VectorTwo}[2]{
\begin{bmatrix}
 {#1} \\
 {#2}
\end{bmatrix}
}

\newcommand{\VectorN}[1]{
\begin{bmatrix}
{#1}_1 \\
{#1}_2 \\
\vdots \\
{#1}_N \\
\end{bmatrix}
}

\newcommand{\DETuvij}[4]{
\begin{vmatrix}
 {#1}_{#3} & {#1}_{#4} \\
 {#2}_{#3} & {#2}_{#4}
\end{vmatrix}
}

\newcommand{\DETuvwijk}[6]{
\begin{vmatrix}
 {#1}_{#4} & {#1}_{#5} & {#1}_{#6} \\
 {#2}_{#4} & {#2}_{#5} & {#2}_{#6} \\
 {#3}_{#4} & {#3}_{#5} & {#3}_{#6}
\end{vmatrix}
}

\newcommand{\DETuvwxijkl}[8]{
\begin{vmatrix}
 {#1}_{#5} & {#1}_{#6} & {#1}_{#7} & {#1}_{#8} \\
 {#2}_{#5} & {#2}_{#6} & {#2}_{#7} & {#2}_{#8} \\
 {#3}_{#5} & {#3}_{#6} & {#3}_{#7} & {#3}_{#8} \\
 {#4}_{#5} & {#4}_{#6} & {#4}_{#7} & {#4}_{#8} \\
\end{vmatrix}
}

%\newcommand{\DETuvwxyijklm}[10]{
%\begin{vmatrix}
% {#1}_{#6} & {#1}_{#7} & {#1}_{#8} & {#1}_{#9} & {#1}_{#10} \\
% {#2}_{#6} & {#2}_{#7} & {#2}_{#8} & {#2}_{#9} & {#2}_{#10} \\
% {#3}_{#6} & {#3}_{#7} & {#3}_{#8} & {#3}_{#9} & {#3}_{#10} \\
% {#4}_{#6} & {#4}_{#7} & {#4}_{#8} & {#4}_{#9} & {#4}_{#10} \\
% {#5}_{#6} & {#5}_{#7} & {#5}_{#8} & {#5}_{#9} & {#5}_{#10}
%\end{vmatrix}
%}

% R3 vector.
\newcommand{\VectorThree}[3]{
\begin{bmatrix}
 {#1} \\
 {#2} \\
 {#3}
\end{bmatrix}
}



\author{Peeter Joot}
\email{peeter.joot@gmail.com}


\chapter{PHY356F lecture notes.}
\label{chap:PHY356F}
%\useCCL
\blogpage{http://sites.google.com/site/peeterjoot/math2010/PHY356F.pdf}
%\date{Oct X, 2010}
\revisionInfo{PHY356F.tex}

%\beginArtWithToc
\beginArtNoToc

\section{Oct 12.}

Review.  What have we learned?

\subsection{Chapter 1.}
Information about systems comes from vectors and operators.  Express the vector $\ket0$ describing the system in terms of eigenvectors $\ket{a_n}$.  $n \in 1,2,3,\cdots$.

of some operator $A$.  What are the coefficients $c_n$?  Act on both sides by $\bra{a_m}$ to find

\begin{align*}
\braket{a_m}{\phi} 
&= \sum_n c_n \underbrace{\braket{a_m}{a_n}}_{\text{Kronicker delta}}  \\
&= \sum c_n \delta_{mn} \\
&= c_m
\end{align*}

\begin{align*}
c_m = \braket{a_m}{\phi}
\end{align*}

Analogy

\begin{align*}
\Bv = \sum_i v_i \Be_i 
\end{align*}

\begin{align*}
\Be_1 \cdot \Bv = \sum_i v_i \Be_1 \cdot \Be_i = v_1
\end{align*}

Physical information comes from the probability for obtaining a measurement of the physical entity associated with operator $A$.  The probability of obtaining outcome $a_m$, an eigenvalue of $A$, is $\Abs{c_n}^2$

\subsection{Chapter 2.}

Deal with operators that have continuous eigenvalues and eigenvectors.

We now express 

\begin{align*}
\ket{\phi} = \int dk \underbrace{f(k)}_{\text{coefficients analogous to $c_n$}} \ket{k}
\end{align*}

Now if we project onto $k'$

\begin{align*}
\braket{k'}{\phi} 
&= \int dk f(k) \underbrace{\braket{k'}{k}}_{\text{Dirac delta}} \\
&= \int dk f(k) \delta(k' -k) \\
&= f(k') 
\end{align*}

Unlike the discrete case, this is not a probability.  Probability density for obtaining outcome $k'$ is $\Abs{f(k')}^2$.

Example 2.
\begin{align*}
\ket{\phi} = \int dk f(k) \ket{k}
\end{align*}

Now if we project x onto both sides

\begin{align*}
\braket{x}{\phi} 
&= \int dk f(k) \braket{x}{k} \\
\end{align*}

With $\braket{x}{k} = u_k(x)$

\begin{align*}
\phi(x) 
&\equiv \braket{x}{\phi} \\
&= \int dk f(k) u_k(x)  \\
&= \int dk f(k) \inv{\sqrt{L}} e^{ikx}
\end{align*}

This is with periodic boundary value conditions for the normalization.  The infinite normalization is also possible.

\begin{align*}
\phi(x) 
&= \inv{\sqrt{L}} \int dk f(k) e^{ikx}
\end{align*}

Multiply both sides by $e^{-ik'x}/\sqrt{L}$ and integrate.  This is analogous to multiplying $\ket{\phi} = \int f(k) \ket{k} dk$ by $\bra{k'}$.  We get

\begin{align*}
\int \phi(x) \inv{\sqrt{L}} e^{-ik'x} dx
&= \inv{L} \iint dk f(k) e^{i(k-k')x} dx \\
&= \int dk f(k) \Bigl( \inv{L} \int e^{i(k-k')x} \Bigr) \\
&= \int dk f(k) \delta(k-k') \\
&= f(k')
\end{align*}

\begin{align*}
f(k') &=
\int \phi(x) \inv{\sqrt{L}} e^{-ik'x} dx
\end{align*}

We can talk about the state vector in terms of its position basis $\phi(x)$ or in the momentum space via Fourier transformation.  This is the equivalent thing, but just expressed different.  The question of interpretation in terms of probabilities works out the same.  Either way we look at the probability density.

The quantity

\begin{align*}
\ket{\phi} = \int dk f(k) \ket{k}
\end{align*}

is also called a wave packet state since it involves a superposition of many stats $\ket{k}$.  Example: See Fig 4.1 (Gaussian wave packet, with $\Abs{\phi}^2$ as the height).  This wave packet is a snapshot of the wave function amplitude at one specific time instant.  The evolution of this wave packet is governed by the Hamiltonian, which brings us to chapter 3.

\subsection{Chapter 3.}

For 
\begin{align*}
\ket{\phi} = \int dk f(k) \ket{k}
\end{align*}

How do we find $\ket{\phi(t)}$, the time evolved state?  Here we have the option of choosing which of the pictures (Schr\"{o}dinger, Heisenberg, interaction) we deal with.  Since the Heisenberg picture deals with time evolved operators, and the interaction picture with evolving Hamiltonian's, neither of these is required to answer this question.  Consider the Schr\"{o}dinger picture which gives 

\begin{align*}
\ket{\phi(t)} = \int dk f(k) \ket{k} e^{-i E_k t/\hbar}
\end{align*}

where $E_k$ is the eigenvalue of the Hamiltonian operator $H$.

STRONG SEEMING HINT: If looking for additional problems and homework, consider in detail the time evolution of the Gaussian wave packet state.

\subsection{Chapter 4.}

For three dimensions with $V(x,y,z) = 0$

\begin{align*}
H &= \inv{2m} \Bp^2 \\
\Bp &= \sum_i p_i \Be_i \\
\end{align*}

In the position representation, where

\begin{align*}
p_i &= -i \hbar \frac{d}{dx_i}
\end{align*}

the Sch equation is
\begin{align*}
H u(x,y,z) &= E u(x,y,z) \\
H &= -\frac{\hbar^2}{2m} \spacegrad^2 \\
= -\frac{\hbar^2}{2m} \left( 
\frac{\partial^2}{\partial {x}^2}
+\frac{\partial^2}{\partial {y}^2}
+\frac{\partial^2}{\partial {z}^2}
\right) 
\end{align*}

Separation of variables assumes it is possible to let

\begin{align*}
u(x,y,z) = X(x) Y(y) Z(z)
\end{align*}

(these capital letters are functions, not operators).

\begin{align*}
-\frac{\hbar^2}{2m} \left( 
YZ \frac{\partial^2 X}{\partial {x}^2}
+ XZ \frac{\partial^2 Y}{\partial {y}^2}
+ YZ \frac{\partial^2 Z}{\partial {z}^2}\right)
&= E X Y Z
\end{align*}

Dividing as usual by $XYZ$ we have

\begin{align*}
-\frac{\hbar^2}{2m} \left( 
\inv{X} \frac{\partial^2 X}{\partial {x}^2}
+ \inv{Y} \frac{\partial^2 Y}{\partial {y}^2}
+ \inv{Z} \frac{\partial^2 Z}{\partial {z}^2} \right)
&= E 
\end{align*}

The curious thing is that we have these three derivatives, which is supposed to be related to an Energy, which is independent of any $x,y,z$, so it must be that each of these is separately constant.  We can separate these into three individual equations

\begin{align*}
-\frac{\hbar^2}{2m} \inv{X} \frac{\partial^2 X}{\partial {x}^2} &= E_1 \\
-\frac{\hbar^2}{2m} \inv{Y} \frac{\partial^2 Y}{\partial {x}^2} &= E_2 \\
-\frac{\hbar^2}{2m} \inv{Z} \frac{\partial^2 Z}{\partial {x}^2} &= E_3
\end{align*}

or
\begin{align*}
\frac{\partial^2 X}{\partial {x}^2} &= \left( - \frac{2m E_1}{\hbar^2} \right) X  \\
\frac{\partial^2 Y}{\partial {x}^2} &= \left( - \frac{2m E_2}{\hbar^2} \right) Y  \\
\frac{\partial^2 Z}{\partial {x}^2} &= \left( - \frac{2m E_3}{\hbar^2} \right) Z 
\end{align*}

We have then

\begin{align*}
X(x) = C_1 e^{i k x}
\end{align*}

with
\begin{align*}
E_1 &= \frac{\hbar^2 k_1^2 }{2m} = \frac{p_1^2}{2m} \\
E_2 &= \frac{\hbar^2 k_2^2 }{2m} = \frac{p_2^2}{2m} \\
E_3 &= \frac{\hbar^2 k_3^2 }{2m} = \frac{p_3^2}{2m} 
\end{align*}

We are free to use any sort of normalization procedure we wish (periodic boundary conditions, infinite Dirac, ...)

\subsection{Angular momentum.}

HOMEWORK: go through the steps to understand how to formulate $\spacegrad^2$ in spherical polar coordinates.  This is a lot of work, but is good practice and background for dealing with the Hydrogen atom, something with spherical symmetry that is most naturally analyzed in the spherical polar coordinates.

In spherical coordinates (We won't go through this here, but it is good practice) with

\begin{align*}
x &= r \sin\theta \cos\phi \\
y &= r \sin\theta \sin\phi \\
z &= r \cos\theta
\end{align*}

we have with $u = u(r,\theta, \phi)$

\begin{align*}
-\frac{\hbar^2}{2m} \left( 
\inv{r} \partial_{rr} (r u) +  \inv{r^2 \sin\theta} \partial_\theta (\sin\theta \partial_\theta u) 
+ \inv{r^2 \sin^2\theta} \partial_{\phi\phi} u
 \right)
&= E u
\end{align*}

We see the start of a separation of variables attack with $u = R(r) Y(\theta, \phi)$.  We end up with

\begin{align*}
-\frac{\hbar^2}{2m} &\left( 
\frac{r}{R} (r R')' +  \inv{Y \sin\theta} \partial_\theta (\sin\theta \partial_\theta Y) 
+ \inv{Y \sin^2\theta} \partial_{\phi\phi} Y
 \right) \\
\end{align*}

\begin{align*}
r (r R')' + \left( \frac{2m E}{\hbar^2} r^2 - \lambda \right) R &= 0
\end{align*}
\begin{align*}
\inv{Y \sin\theta} \partial_\theta (\sin\theta \partial_\theta Y) + \inv{Y \sin^2\theta} \partial_{\phi\phi} Y &= -\lambda
\end{align*}

Application of separation of variables again, with $Y = P(\theta) Q(\phi)$ gives us

\begin{align*}
\inv{P \sin\theta} \partial_\theta (\sin\theta \partial_\theta P) + \inv{Q \sin^2\theta} \partial_{\phi\phi} Q &= -\lambda 
\end{align*}

\begin{align*}
\frac{\sin\theta}{P } \partial_\theta (\sin\theta \partial_\theta P) 
+\lambda  \sin^2\theta
+ \inv{Q } \partial_{\phi\phi} Q &= 0
\end{align*}

\begin{align*}
\frac{\sin\theta}{P } \partial_\theta (\sin\theta \partial_\theta P) + \lambda \sin^2\theta - \mu = 0
\inv{Q } \partial_{\phi\phi} Q &= -\mu
\end{align*}

or
\begin{align}\label{eqn:PHY356F:1000}
\frac{1}{P \sin\theta} \partial_\theta (\sin\theta \partial_\theta P) +\lambda -\frac{\mu}{\sin^2\theta} &= 0
\end{align}
\begin{align}\label{eqn:PHY356F:2000}
\partial_{\phi\phi} Q &= -\mu Q
\end{align}

The equation for $P$ can be solved using the Legendre function $P_l^m(\cos\theta)$ where $\lambda = l(l+1)$ and $l$ is an integer

Replacing $\mu$ with $m^2$, where $m$ is an integer

\begin{align*}
\frac{d^2 Q}{d\phi^2} &= -m^2 Q
\end{align*}

Imposing a periodic boundary condition $Q(\phi) = Q(\phi + 2\pi)$, where ($m = 0, \pm 1, \pm 2, \cdots$) we have

\begin{align*}
Q &= \inv{\sqrt{2\pi}} e^{im\phi}
\end{align*}

There is the overall solution $r(r,\theta,\phi) = R(r) Y(\theta, \phi)$ for a free particle.  The functions $Y(\theta, \phi)$ are

\begin{align*}
Y_{lm}(\theta, \phi) 
&= N \left( \inv{\sqrt{2\pi}} e^{im\phi} \right) \underbrace{ P_l^m(\cos\theta) }_{ -l \le m \le l }
\end{align*}

where $N$ is a normalization constant, and $m = 0, \pm 1, \pm 2, \cdots$.  $Y_{lm}$ is an eigenstate of the $\BL^2$ operator and $L_z$ (two for the price of one).  There's no specific reason for the direction $z$, but it is the direction picked out of convention.

Angular momentum is given by 

\begin{align*}
\BL = \Br \cross \Bp
\end{align*}

where 

\begin{align*}
\BR = x \xcap + y\ycap + z\zcap
\end{align*}

and 
\begin{align*}
\Bp = p_x \xcap + p_y\ycap + p_z\zcap
\end{align*}

The important thing to remember is that the aim of following all the math is to show that

\begin{align*}
\BL^2 Y_{lm} = \hbar^2 l (l+1) Y_{lm}
\end{align*}

and simultaneously 

\begin{align*}
\BL_z Y_{lm} = \hbar m Y_{lm}
\end{align*}

Part of the solution involves working with $\antisymmetric{L_z}{L_{+}}$, and $\antisymmetric{L_z}{L_{-}}$, where

\begin{align*}
L_{+} &= L_x + i L_y \\
L_{-} &= L_x - i L_y
\end{align*}

An exercise (not in the book) is to evaluate
\begin{align}\label{eqn:PHY356F:4000}
\antisymmetric{L_z}{L_{+}} 
&= L_z L_x + i L_z L_y - L_x L_z - i L_y L_z 
\end{align}

where
\begin{align}\label{eqn:PHY356F:5000}
\antisymmetric{L_x}{L_y}  &= i \hbar L_z \\
\antisymmetric{L_y}{L_z}  &= i \hbar L_x \\
\antisymmetric{L_z}{L_x}  &= i \hbar L_y
\end{align}

Substitution back in \ref{eqn:PHY356F:4000} we have

\begin{align*}
\antisymmetric{L_z}{L_{+}} 
&=
\antisymmetric{L_z}{L_x} 
+ i \antisymmetric{L_z}{L_y}  \\
&=
i \hbar ( L_y - i L_x ) \\
&=
\hbar ( i L_y +  L_x ) \\
&=
\hbar L_{+}
\end{align*}

%\EndArticle
\EndNoBibArticle
