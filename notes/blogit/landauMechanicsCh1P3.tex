%
% Copyright � 2012 Peeter Joot.  All Rights Reserved.
% Licenced as described in the file LICENSE under the root directory of this GIT repository.
%
% pick one:
%\newcommand{\authorname}{Peeter Joot}
\newcommand{\email}{peeter.joot@utoronto.ca}
\newcommand{\studentnumber}{920798560}
\newcommand{\basename}{FIXMEbasenameUndefined}
\newcommand{\dirname}{notes/FIXMEdirnameUndefined/}

\newcommand{\authorname}{Peeter Joot}
\newcommand{\email}{peeterjoot@protonmail.com}
\newcommand{\basename}{FIXMEbasenameUndefined}
\newcommand{\dirname}{notes/FIXMEdirnameUndefined/}

\renewcommand{\basename}{landauMechanicsCh1P3}
\newcommand{\dirname}{notes/phy354/}
\newcommand{\keywords}{Landau,mechanics,Lagrangian}

\newcommand{\authorname}{Peeter Joot}
\newcommand{\onlineurl}{http://sites.google.com/site/peeterjoot2/math2013/\basename.pdf}
\newcommand{\sourcepath}{\dirname\basename.tex}
\newcommand{\generatetitle}[1]{\chapter{#1}}

\newcommand{\vcsinfo}{%
\section*{}
\noindent{\color{DarkOliveGreen}{\rule{\linewidth}{0.1mm}}}
\paragraph{Document version}
%\paragraph{\color{Maroon}{Document version}}
{
\small
\begin{itemize}
\item Available online at:\\ 
\href{\onlineurl}{\onlineurl}
\item Git Repository: \input{./.revinfo/gitRepo.tex}
\item Source: \sourcepath
\item last commit: \input{./.revinfo/gitCommitString.tex}
\item commit date: \input{./.revinfo/gitCommitDate.tex}
\end{itemize}
}
}

%\PassOptionsToPackage{dvipsnames,svgnames}{xcolor}
\PassOptionsToPackage{square,numbers}{natbib}
\documentclass{scrreprt}

\usepackage[left=2cm,right=2cm]{geometry}
\usepackage[svgnames]{xcolor}
\usepackage{peeters_layout}

\usepackage{natbib}

\usepackage[
colorlinks=true,
bookmarks=false,
pdfauthor={\authorname, \email},
backref 
]{hyperref}

% http://tex.stackexchange.com/questions/75773/how-to-reference-problems-by-the-text-label-in-an-exercise-envioronment
\usepackage[english]{cleveref}
\crefname{Exercise}{exercise}{exercises}
\Crefname{Exercise}{Exercise}{Exercises}

\RequirePackage{titlesec}
\RequirePackage{ifthen}

% http://stackoverflow.com/questions/4932910/date-in-the-tabular-environment
\makeatletter
\let\insertdate\@date
\makeatother

\titleformat{\chapter}[display]
{\bfseries\Large}
{\color{DarkSlateGrey}\filleft \authorname
\ifthenelse{\isundefined{\studentnumber}}{}{\\ \studentnumber}
\ifthenelse{\isundefined{\email}}{}{\\ \email}
\ifthenelse{\isundefined{\dateintitle}}{}{\\ \insertdate}
%\ifthenelse{\isundefined{\coursename}}{}{\\ \coursename} % put in title instead.
}
{4ex}
{\color{DarkOliveGreen}{\titlerule}\color{Maroon}
\vspace{2ex}%
\filright}
[\vspace{2ex}%
\color{DarkOliveGreen}\titlerule
]

\newcommand{\beginArtWithToc}[0]{\begin{document}\tableofcontents}
\newcommand{\beginArtNoToc}[0]{\begin{document}}
\newcommand{\EndNoBibArticle}[0]{\end{document}}
\newcommand{\EndArticle}[0]{\bibliography{Bibliography}\bibliographystyle{plainnat}\end{document}}

% 
%\newcommand{\citep}[1]{\cite{#1}}

\colorSectionsForArticle



\beginArtNoToc

\generatetitle{Typo in Landau Mechanics problem?}
\label{chap:\basename}
\section{Motivation}

Attempting a mechanics problem from Landau I get a different answer.  I wrote up my solution to see if I can spot either where I went wrong, or demonstrate the error, and then posted it to \href{http://www.physicsforums.com/showthread.php?t=620775}{physicsforums}.  I wasn't wrong, but the text wasn't either.  Here's the complete result.

\section{Guts}

\makeproblem{Pendulum with support moving in circle}{landauMechanics:ch1:pr1}{
\S 1 problem 3a of \citep{landau1976mechanics} is to calculate the Lagrangian of a 
\href{http://goo.gl/IjqeO}{pendulum where the point of support is moving in a circle (figure and full text for problem in this Google books reference)}
 }
\makeanswer{landauMechanics:ch1:pr1}{
The coordinates of the mass are 

\begin{dmath}\label{eqn:landauMechanicsCh1P3:10}
p = a e^{i \gamma t} + i l e^{i\phi},
\end{dmath}

or in coordinates

\begin{dmath}\label{eqn:landauMechanicsCh1P3:30}
p = (a \cos\gamma t + l \sin\phi, -a \sin\gamma t + l \cos\phi).
\end{dmath}

The velocity is

\begin{dmath}\label{eqn:landauMechanicsCh1P3:50}
\pdot = (-a \gamma \sin\gamma t + l \phidot \cos\phi, -a \gamma \cos\gamma t - l \phidot \sin\phi),
\end{dmath}

and in the square
\begin{dmath}\label{eqn:landauMechanicsCh1P3:70}
\pdot^2 = 
a^2 \gamma^2 + l^2 \phidot^2 - 2 a \gamma \phidot \sin\gamma t \cos\phi + 2 a \gamma l \phidot \cos \gamma t \sin\phi
=
a^2 \gamma^2 + l^2 \phidot^2 + 2 a \gamma l \phidot \sin (\gamma t - \phi).
\end{dmath}

For the potential our height above the minimum is

\begin{dmath}\label{eqn:landauMechanicsCh1P3:90}
h = 2a + l - a (1 -\cos\gamma t) - l \cos\phi = a ( 1 + \cos\gamma t) + l (1 - \cos\phi).
\end{dmath}

In the potential the total derivative $\cos\gamma t$ can be dropped, as can all the constant terms, leaving

\begin{dmath}\label{eqn:landauMechanicsCh1P3:110}
U = - m g l \cos\phi, 
\end{dmath}

so by the above the Lagrangian should be (after also dropping the constant term $m a^2 \gamma^2/2$
\begin{dmath}\label{eqn:landauMechanicsCh1P3:130}
\LL = 
\inv{2} m \left( l^2 \phidot^2 + 2 a \gamma l \phidot \sin (\gamma t - \phi) \right) + m g l \cos\phi.
\end{dmath}

This is almost the stated value in the text
\begin{dmath}\label{eqn:landauMechanicsCh1P3:150}
\LL = 
\inv{2} m \left( l^2 \phidot^2 + 2 a \gamma^2 l \sin (\gamma t - \phi) \right) + m g l \cos\phi.
\end{dmath}

We have what appears to be an innocent looking typo (text putting in a $\gamma$ instead of a $\phidot$), but the subsequent text also didn't make sense.  That referred to the omission of the total derivative $m l a \gamma \cos( \phi - \gamma t)$, which isn't even a term that I have in my result.

In the physicsforums response it was cleverly pointed out by Dickfore that \ref{eqn:landauMechanicsCh1P3:130} can be recast into a total derivative

\begin{dmath}\label{eqn:landauMechanicsCh1P3:170}
m a l \gamma \phidot \sin (\gamma t - \phi) 
=
m a l \gamma ( \phidot - \gamma ) \sin (\gamma t - \phi) 
+m a l \gamma^2 \sin (\gamma t - \phi) 
=
\ddt{}\left(
m a l \gamma \cos (\gamma t - \phi) 
\right)
+m a l \gamma^2 \sin (\gamma t - \phi),
\end{dmath}

which resolves the conundrum!
}
\shipoutAnswer

%\vcsinfo
\EndArticle
