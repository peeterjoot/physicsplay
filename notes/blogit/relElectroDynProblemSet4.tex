%
% Copyright � 2015 Peeter Joot.  All Rights Reserved.
% Licenced as described in the file LICENSE under the root directory of this GIT repository.
%
\documentclass[]{eliblog}

\usepackage{amsmath}
\usepackage{mathpazo}

%
% shorthand for bold symbols, convenient for vectors and matrices
%
\newcommand{\Ba}[0]{\mathbf{a}}
\newcommand{\Bb}[0]{\mathbf{b}}
\newcommand{\Bc}[0]{\mathbf{c}}
\newcommand{\Bd}[0]{\mathbf{d}}
\newcommand{\Be}[0]{\mathbf{e}}
\newcommand{\Bf}[0]{\mathbf{f}}
\newcommand{\Bg}[0]{\mathbf{g}}
\newcommand{\Bh}[0]{\mathbf{h}}
\newcommand{\Bi}[0]{\mathbf{i}}
\newcommand{\Bj}[0]{\mathbf{j}}
\newcommand{\Bk}[0]{\mathbf{k}}
\newcommand{\Bl}[0]{\mathbf{l}}
\newcommand{\Bm}[0]{\mathbf{m}}
\newcommand{\Bn}[0]{\mathbf{n}}
\newcommand{\Bo}[0]{\mathbf{o}}
\newcommand{\Bp}[0]{\mathbf{p}}
\newcommand{\Bq}[0]{\mathbf{q}}
\newcommand{\Br}[0]{\mathbf{r}}
\newcommand{\Bs}[0]{\mathbf{s}}
\newcommand{\Bt}[0]{\mathbf{t}}
\newcommand{\Bu}[0]{\mathbf{u}}
\newcommand{\Bv}[0]{\mathbf{v}}
\newcommand{\Bw}[0]{\mathbf{w}}
\newcommand{\Bx}[0]{\mathbf{x}}
\newcommand{\By}[0]{\mathbf{y}}
\newcommand{\Bz}[0]{\mathbf{z}}
\newcommand{\BA}[0]{\mathbf{A}}
\newcommand{\BB}[0]{\mathbf{B}}
\newcommand{\BC}[0]{\mathbf{C}}
\newcommand{\BD}[0]{\mathbf{D}}
\newcommand{\BE}[0]{\mathbf{E}}
\newcommand{\BF}[0]{\mathbf{F}}
\newcommand{\BG}[0]{\mathbf{G}}
\newcommand{\BH}[0]{\mathbf{H}}
\newcommand{\BI}[0]{\mathbf{I}}
\newcommand{\BJ}[0]{\mathbf{J}}
\newcommand{\BK}[0]{\mathbf{K}}
\newcommand{\BL}[0]{\mathbf{L}}
\newcommand{\BM}[0]{\mathbf{M}}
\newcommand{\BN}[0]{\mathbf{N}}
\newcommand{\BO}[0]{\mathbf{O}}
\newcommand{\BP}[0]{\mathbf{P}}
\newcommand{\BQ}[0]{\mathbf{Q}}
\newcommand{\BR}[0]{\mathbf{R}}
\newcommand{\BS}[0]{\mathbf{S}}
\newcommand{\BT}[0]{\mathbf{T}}
\newcommand{\BU}[0]{\mathbf{U}}
\newcommand{\BV}[0]{\mathbf{V}}
\newcommand{\BW}[0]{\mathbf{W}}
\newcommand{\BX}[0]{\mathbf{X}}
\newcommand{\BY}[0]{\mathbf{Y}}
\newcommand{\BZ}[0]{\mathbf{Z}}

\newcommand{\Bzero}[0]{\mathbf{0}}
\newcommand{\Btheta}[0]{\boldsymbol{\theta}}
\newcommand{\Btau}[0]{\boldsymbol{\tau}}
\newcommand{\Bomega}[0]{\boldsymbol{\omega}}

%
% shorthand for unit vectors
%
\newcommand{\acap}[0]{\hat{\Ba}}
\newcommand{\bcap}[0]{\hat{\Bb}}
\newcommand{\ccap}[0]{\hat{\Bc}}
\newcommand{\dcap}[0]{\hat{\Bd}}
\newcommand{\ecap}[0]{\hat{\Be}}
\newcommand{\fcap}[0]{\hat{\Bf}}
\newcommand{\gcap}[0]{\hat{\Bg}}
\newcommand{\hcap}[0]{\hat{\Bh}}
\newcommand{\icap}[0]{\hat{\Bi}}
\newcommand{\jcap}[0]{\hat{\Bj}}
\newcommand{\kcap}[0]{\hat{\Bk}}
\newcommand{\lcap}[0]{\hat{\Bl}}
\newcommand{\mcap}[0]{\hat{\Bm}}
\newcommand{\ncap}[0]{\hat{\Bn}}
\newcommand{\ocap}[0]{\hat{\Bo}}
\newcommand{\pcap}[0]{\hat{\Bp}}
\newcommand{\qcap}[0]{\hat{\Bq}}
\newcommand{\rcap}[0]{\hat{\Br}}
\newcommand{\scap}[0]{\hat{\Bs}}
\newcommand{\tcap}[0]{\hat{\Bt}}
\newcommand{\ucap}[0]{\hat{\Bu}}
\newcommand{\vcap}[0]{\hat{\Bv}}
\newcommand{\wcap}[0]{\hat{\Bw}}
\newcommand{\xcap}[0]{\hat{\Bx}}
\newcommand{\ycap}[0]{\hat{\By}}
\newcommand{\zcap}[0]{\hat{\Bz}}
\newcommand{\thetacap}[0]{\hat{\Btheta}}

%
% to write R^n and C^n in a distinguishable fashion.  Perhaps change this
% to the double lined characters upon figuring out how to do so.
%
\newcommand{\C}[1]{$\mathbb{C}^{#1}$}
\newcommand{\R}[1]{$\mathbb{R}^{#1}$}

%
% various generally useful helpers
%

% derivative of #1 wrt. #2:
\newcommand{\D}[2] {\frac {d#2} {d#1}}

\newcommand{\inv}[1]{\frac{1}{#1}}
\newcommand{\cross}[0]{\times}

\newcommand{\abs}[1]{\lvert{#1}\rvert}
\newcommand{\norm}[1]{\lVert{#1}\rVert}
\newcommand{\innerprod}[2]{\langle{#1}, {#2}\rangle}
\newcommand{\dotprod}[2]{{#1} \cdot {#2}}
\newcommand{\bdotprod}[2]{\left({#1} \cdot {#2}\right)}
\newcommand{\crossprod}[2]{{#1} \cross {#2}}
\newcommand{\tripleprod}[3]{\dotprod{\left(\crossprod{#1}{#2}\right)}{#3}}

\DeclareMathOperator{\Proj}{Proj}
\DeclareMathOperator{\Span}{span}
\DeclareMathOperator{\Sgn}{sgn}
\DeclareMathOperator{\Area}{Area}
\DeclareMathOperator{\Volume}{Volume}

%
% A few miscellaneous things specific to this document
%
\newcommand{\crossop}[1]{\crossprod{#1}{}}

% R2 vector.
\newcommand{\VectorTwo}[2]{
\begin{bmatrix}
 {#1} \\
 {#2}
\end{bmatrix}
}

\newcommand{\VectorN}[1]{
\begin{bmatrix}
{#1}_1 \\
{#1}_2 \\
\vdots \\
{#1}_N \\
\end{bmatrix}
}

\newcommand{\DETuvij}[4]{
\begin{vmatrix}
 {#1}_{#3} & {#1}_{#4} \\
 {#2}_{#3} & {#2}_{#4}
\end{vmatrix}
}

\newcommand{\DETuvwijk}[6]{
\begin{vmatrix}
 {#1}_{#4} & {#1}_{#5} & {#1}_{#6} \\
 {#2}_{#4} & {#2}_{#5} & {#2}_{#6} \\
 {#3}_{#4} & {#3}_{#5} & {#3}_{#6}
\end{vmatrix}
}

\newcommand{\DETuvwxijkl}[8]{
\begin{vmatrix}
 {#1}_{#5} & {#1}_{#6} & {#1}_{#7} & {#1}_{#8} \\
 {#2}_{#5} & {#2}_{#6} & {#2}_{#7} & {#2}_{#8} \\
 {#3}_{#5} & {#3}_{#6} & {#3}_{#7} & {#3}_{#8} \\
 {#4}_{#5} & {#4}_{#6} & {#4}_{#7} & {#4}_{#8} \\
\end{vmatrix}
}

%\newcommand{\DETuvwxyijklm}[10]{
%\begin{vmatrix}
% {#1}_{#6} & {#1}_{#7} & {#1}_{#8} & {#1}_{#9} & {#1}_{#10} \\
% {#2}_{#6} & {#2}_{#7} & {#2}_{#8} & {#2}_{#9} & {#2}_{#10} \\
% {#3}_{#6} & {#3}_{#7} & {#3}_{#8} & {#3}_{#9} & {#3}_{#10} \\
% {#4}_{#6} & {#4}_{#7} & {#4}_{#8} & {#4}_{#9} & {#4}_{#10} \\
% {#5}_{#6} & {#5}_{#7} & {#5}_{#8} & {#5}_{#9} & {#5}_{#10}
%\end{vmatrix}
%}

% R3 vector.
\newcommand{\VectorThree}[3]{
\begin{bmatrix}
 {#1} \\
 {#2} \\
 {#3}
\end{bmatrix}
}



\author{Peeter Joot}
\email{peeter.joot@gmail.com}

%\documentclass[]{eliblogwidescreen}

\usepackage{amsmath}
\usepackage{mathpazo}

%
% shorthand for bold symbols, convenient for vectors and matrices
%
\newcommand{\Ba}[0]{\mathbf{a}}
\newcommand{\Bb}[0]{\mathbf{b}}
\newcommand{\Bc}[0]{\mathbf{c}}
\newcommand{\Bd}[0]{\mathbf{d}}
\newcommand{\Be}[0]{\mathbf{e}}
\newcommand{\Bf}[0]{\mathbf{f}}
\newcommand{\Bg}[0]{\mathbf{g}}
\newcommand{\Bh}[0]{\mathbf{h}}
\newcommand{\Bi}[0]{\mathbf{i}}
\newcommand{\Bj}[0]{\mathbf{j}}
\newcommand{\Bk}[0]{\mathbf{k}}
\newcommand{\Bl}[0]{\mathbf{l}}
\newcommand{\Bm}[0]{\mathbf{m}}
\newcommand{\Bn}[0]{\mathbf{n}}
\newcommand{\Bo}[0]{\mathbf{o}}
\newcommand{\Bp}[0]{\mathbf{p}}
\newcommand{\Bq}[0]{\mathbf{q}}
\newcommand{\Br}[0]{\mathbf{r}}
\newcommand{\Bs}[0]{\mathbf{s}}
\newcommand{\Bt}[0]{\mathbf{t}}
\newcommand{\Bu}[0]{\mathbf{u}}
\newcommand{\Bv}[0]{\mathbf{v}}
\newcommand{\Bw}[0]{\mathbf{w}}
\newcommand{\Bx}[0]{\mathbf{x}}
\newcommand{\By}[0]{\mathbf{y}}
\newcommand{\Bz}[0]{\mathbf{z}}
\newcommand{\BA}[0]{\mathbf{A}}
\newcommand{\BB}[0]{\mathbf{B}}
\newcommand{\BC}[0]{\mathbf{C}}
\newcommand{\BD}[0]{\mathbf{D}}
\newcommand{\BE}[0]{\mathbf{E}}
\newcommand{\BF}[0]{\mathbf{F}}
\newcommand{\BG}[0]{\mathbf{G}}
\newcommand{\BH}[0]{\mathbf{H}}
\newcommand{\BI}[0]{\mathbf{I}}
\newcommand{\BJ}[0]{\mathbf{J}}
\newcommand{\BK}[0]{\mathbf{K}}
\newcommand{\BL}[0]{\mathbf{L}}
\newcommand{\BM}[0]{\mathbf{M}}
\newcommand{\BN}[0]{\mathbf{N}}
\newcommand{\BO}[0]{\mathbf{O}}
\newcommand{\BP}[0]{\mathbf{P}}
\newcommand{\BQ}[0]{\mathbf{Q}}
\newcommand{\BR}[0]{\mathbf{R}}
\newcommand{\BS}[0]{\mathbf{S}}
\newcommand{\BT}[0]{\mathbf{T}}
\newcommand{\BU}[0]{\mathbf{U}}
\newcommand{\BV}[0]{\mathbf{V}}
\newcommand{\BW}[0]{\mathbf{W}}
\newcommand{\BX}[0]{\mathbf{X}}
\newcommand{\BY}[0]{\mathbf{Y}}
\newcommand{\BZ}[0]{\mathbf{Z}}

\newcommand{\Bzero}[0]{\mathbf{0}}
\newcommand{\Btheta}[0]{\boldsymbol{\theta}}
\newcommand{\Btau}[0]{\boldsymbol{\tau}}
\newcommand{\Bomega}[0]{\boldsymbol{\omega}}

%
% shorthand for unit vectors
%
\newcommand{\acap}[0]{\hat{\Ba}}
\newcommand{\bcap}[0]{\hat{\Bb}}
\newcommand{\ccap}[0]{\hat{\Bc}}
\newcommand{\dcap}[0]{\hat{\Bd}}
\newcommand{\ecap}[0]{\hat{\Be}}
\newcommand{\fcap}[0]{\hat{\Bf}}
\newcommand{\gcap}[0]{\hat{\Bg}}
\newcommand{\hcap}[0]{\hat{\Bh}}
\newcommand{\icap}[0]{\hat{\Bi}}
\newcommand{\jcap}[0]{\hat{\Bj}}
\newcommand{\kcap}[0]{\hat{\Bk}}
\newcommand{\lcap}[0]{\hat{\Bl}}
\newcommand{\mcap}[0]{\hat{\Bm}}
\newcommand{\ncap}[0]{\hat{\Bn}}
\newcommand{\ocap}[0]{\hat{\Bo}}
\newcommand{\pcap}[0]{\hat{\Bp}}
\newcommand{\qcap}[0]{\hat{\Bq}}
\newcommand{\rcap}[0]{\hat{\Br}}
\newcommand{\scap}[0]{\hat{\Bs}}
\newcommand{\tcap}[0]{\hat{\Bt}}
\newcommand{\ucap}[0]{\hat{\Bu}}
\newcommand{\vcap}[0]{\hat{\Bv}}
\newcommand{\wcap}[0]{\hat{\Bw}}
\newcommand{\xcap}[0]{\hat{\Bx}}
\newcommand{\ycap}[0]{\hat{\By}}
\newcommand{\zcap}[0]{\hat{\Bz}}
\newcommand{\thetacap}[0]{\hat{\Btheta}}

%
% to write R^n and C^n in a distinguishable fashion.  Perhaps change this
% to the double lined characters upon figuring out how to do so.
%
\newcommand{\C}[1]{$\mathbb{C}^{#1}$}
\newcommand{\R}[1]{$\mathbb{R}^{#1}$}

%
% various generally useful helpers
%

% derivative of #1 wrt. #2:
\newcommand{\D}[2] {\frac {d#2} {d#1}}

\newcommand{\inv}[1]{\frac{1}{#1}}
\newcommand{\cross}[0]{\times}

\newcommand{\abs}[1]{\lvert{#1}\rvert}
\newcommand{\norm}[1]{\lVert{#1}\rVert}
\newcommand{\innerprod}[2]{\langle{#1}, {#2}\rangle}
\newcommand{\dotprod}[2]{{#1} \cdot {#2}}
\newcommand{\bdotprod}[2]{\left({#1} \cdot {#2}\right)}
\newcommand{\crossprod}[2]{{#1} \cross {#2}}
\newcommand{\tripleprod}[3]{\dotprod{\left(\crossprod{#1}{#2}\right)}{#3}}

\DeclareMathOperator{\Proj}{Proj}
\DeclareMathOperator{\Span}{span}
\DeclareMathOperator{\Sgn}{sgn}
\DeclareMathOperator{\Area}{Area}
\DeclareMathOperator{\Volume}{Volume}

%
% A few miscellaneous things specific to this document
%
\newcommand{\crossop}[1]{\crossprod{#1}{}}

% R2 vector.
\newcommand{\VectorTwo}[2]{
\begin{bmatrix}
 {#1} \\
 {#2}
\end{bmatrix}
}

\newcommand{\VectorN}[1]{
\begin{bmatrix}
{#1}_1 \\
{#1}_2 \\
\vdots \\
{#1}_N \\
\end{bmatrix}
}

\newcommand{\DETuvij}[4]{
\begin{vmatrix}
 {#1}_{#3} & {#1}_{#4} \\
 {#2}_{#3} & {#2}_{#4}
\end{vmatrix}
}

\newcommand{\DETuvwijk}[6]{
\begin{vmatrix}
 {#1}_{#4} & {#1}_{#5} & {#1}_{#6} \\
 {#2}_{#4} & {#2}_{#5} & {#2}_{#6} \\
 {#3}_{#4} & {#3}_{#5} & {#3}_{#6}
\end{vmatrix}
}

\newcommand{\DETuvwxijkl}[8]{
\begin{vmatrix}
 {#1}_{#5} & {#1}_{#6} & {#1}_{#7} & {#1}_{#8} \\
 {#2}_{#5} & {#2}_{#6} & {#2}_{#7} & {#2}_{#8} \\
 {#3}_{#5} & {#3}_{#6} & {#3}_{#7} & {#3}_{#8} \\
 {#4}_{#5} & {#4}_{#6} & {#4}_{#7} & {#4}_{#8} \\
\end{vmatrix}
}

%\newcommand{\DETuvwxyijklm}[10]{
%\begin{vmatrix}
% {#1}_{#6} & {#1}_{#7} & {#1}_{#8} & {#1}_{#9} & {#1}_{#10} \\
% {#2}_{#6} & {#2}_{#7} & {#2}_{#8} & {#2}_{#9} & {#2}_{#10} \\
% {#3}_{#6} & {#3}_{#7} & {#3}_{#8} & {#3}_{#9} & {#3}_{#10} \\
% {#4}_{#6} & {#4}_{#7} & {#4}_{#8} & {#4}_{#9} & {#4}_{#10} \\
% {#5}_{#6} & {#5}_{#7} & {#5}_{#8} & {#5}_{#9} & {#5}_{#10}
%\end{vmatrix}
%}

% R3 vector.
\newcommand{\VectorThree}[3]{
\begin{bmatrix}
 {#1} \\
 {#2} \\
 {#3}
\end{bmatrix}
}



\author{Peeter Joot}
\email{peeter.joot@gmail.com}


\chapter{PHY450H1S Problem Set 4.}
\label{chap:relElectroDynProblemSet4}
\blogpage{http://sites.google.com/site/peeterjoot/math2011/relElectroDynProblemSet4.pdf}
\date{Mar 3, 2011}
\revisionInfo{relElectroDynProblemSet4.tex}

\beginArtWithToc
%\beginArtNoToc

\section{Disclaimer.}

This problem set is as yet ungraded (although only the second question will be graded).

\section{Problem 1.  Energy, momentum, etc., of EM waves.}

\subsection{Statement}

\begin{enumerate}
\item Calculate the energy density, energy flux, and momentum density of a plane monochromatic linearly polarized electromagnetic wave.
\item Calculate the values of these quantities averaged over a period.
\item Imagine that a plane monochromatic linearly polarized wave incident on a surface (let the angle between the wave vector and the normal to the surface be $\theta$) is completely reflected.  Find the pressure that the EM wave exerts on the surface.
\item To plug in some numbrers, note that the intensity of sunlight hitting the Earth is about $1300 W/m^2$ ( the intensity is the average power per unit area transported by the wave).  If sunlight strikes a perfect absorber, what is the pressure exerted?  What if it strikes a perfect reflector?  What fraction of the atmoshperic pressure does this amount to?
\end{enumerate}
\subsection{Solution}
\subsubsection{Part 1.  Energy and momentum density.}

Because it doesn't add too much complexity, I'm going to calculate these using the more general eliptically polarized wave solutions.  Our vector potential (in the Coulomb gauge $\phi = 0$, $\spacegrad \cdot \BA = 0$) has the form

\begin{equation}\label{eqn:relElectroDynProblemSet4:10}
\BA = \Real \Bbeta e^{i (\omega t - \Bk \cdot \Bx) }.
\end{equation}

The eliptical polarization case only differs from the linear by allowing $\Bbeta$ to be complex, rather than purely real or purely imaginary.  Observe that the Coulomb gauge condition $\spacegrad \cdot \BA$ implies

\begin{equation}\label{eqn:relElectroDynProblemSet4:30}
\Bbeta \cdot \Bk = 0,
\end{equation}

a fact that will kill of terms in a number of places in the following manipulations.

Also observe that for this to be a solution to the wave equation operator

\begin{equation}\label{eqn:relElectroDynProblemSet4:50}
\inv{c^2} \PDSq{t}{} - \Delta,
\end{equation}

the frequency and wave vector must be related by the condition

\begin{equation}\label{eqn:relElectroDynProblemSet4:70}
\frac{\omega}{c} = \Abs{\Bk} = k.
\end{equation}

For the time and spatial phase let's write

\begin{equation}\label{eqn:relElectroDynProblemSet4:90}
\theta = \omega t - \Bk \cdot \Bx.
\end{equation}

In the Coulomb gauge, our electric and magnetic fields are

\begin{align}\label{eqn:relElectroDynProblemSet4:110}
\BE &= -\inv{c}\PD{t}{\BA} = \Real \frac{-i\omega}{c} \Bbeta e^{i\theta} \\
\BB &= \spacegrad \cross \BA = \Real i \Bbeta \cross \Bk e^{i\theta}
\end{align}

Similar to \S 48 of the text \cite{landau1980classical}, let's split $\Bbeta$ into a phase and perpendicular vector components so that

\begin{equation}\label{eqn:relElectroDynProblemSet4:130}
\Bbeta = \Bb e^{-i\alpha}
\end{equation}

where $\Bb$ has a real square

\begin{equation}\label{eqn:relElectroDynProblemSet4:150}
\Bb^2 = \Abs{\Bbeta}^2.
\end{equation}

This allows a split into two perpendicular real vectors

\begin{equation}\label{eqn:relElectroDynProblemSet4:170}
\Bb = \Bb_1 + i \Bb_2,
\end{equation}

where $\Bb_1 \cdot \Bb_2 = 0$ since $\Bb^2 = \Bb_1^2 - \Bb_2^2 + 2 \Bb_1 \cdot \Bb_2$ is real.

Our electric and magnetic fields are now reduced to
\begin{align}\label{eqn:relElectroDynProblemSet4:190}
\BE &= \Real \left( \frac{-i\omega}{c} \Bb e^{i(\theta - \alpha)} \right) \\
\BB &= \Real \left( i \Bb \cross \Bk e^{i(\theta - \alpha)} \right) 
\end{align}

or explicitly in terms of $\Bb_1$ and $\Bb_2$ 

\begin{align}\label{eqn:relElectroDynProblemSet4:210}
\BE &= \frac{\omega}{c} ( \Bb_1 \sin(\theta-\alpha) + \Bb_2 \cos(\theta-\alpha)) \\
\BB &= ( \Bk \cross \Bb_1 ) \sin(\theta-\alpha) + (\Bk \cross \Bb_2) \cos(\theta-\alpha) 
\end{align}

The special case of interest for this problem, since it only strictly asked for linear polarization, is where $\alpha = 0$ and one of $\Bb_1$ or $\Bb_2$ is zero (i.e. $\Bbeta$ is strictly real or strictly imaginary).  The case with $\Beta$ strictly real, as done in class, is

\begin{align}\label{eqn:relElectroDynProblemSet4:230}
\BE &= \frac{\omega}{c} \Bb_1 \sin(\theta-\alpha) \\
\BB &= ( \Bk \cross \Bb_1 ) \sin(\theta-\alpha) 
\end{align}

Now lets calculate the energy density and Poynting vectors.  We'll need a few intermediate results.

\begin{align*}
(\Real \Bd e^{i\phi})^2 
&= \inv{4} ( \Bd e^{i\phi} + \Bd^\conj e^{-i\phi})^2 \\
&= \inv{4} ( \Bd^2 e^{2 i \phi} + (\Bd^\conj)^2 e^{-2 i \phi} + 2 \Abs{\Bd}^2 ) \\
&= \inv{2} \left( \Abs{\Bd}^2 + \Real ( \Bd e^{i \phi} )^2 \right),
\end{align*}

and

\begin{align*}
(\Real \Bd e^{i\phi}) \cross (\Real \Be e^{i\phi}) 
&= \inv{4} 
( \Bd e^{i\phi} + \Bd^\conj e^{-i\phi}) \cross ( \Be e^{i\phi} + \Be^\conj e^{-i\phi}) \\
&= \inv{2} \Real \left( \Bd \cross \Be^\conj + (\Bd \cross \Be) e^{2 i \phi} \right).
\end{align*}

Let's use arrowed vectors for the phasor parts

\begin{align}\label{eqn:relElectroDynProblemSet4:250}
\vec{E} &= \frac{-i\omega}{c} \Bb e^{i(\theta - \alpha)} \\
\vec{B} &= i \Bb \cross \Bk e^{i(\theta - \alpha)},
\end{align}

where we can recover our vector quantities by taking real parts $\BE = \Real \vec{E}$, $\BB = \Real \vec{B}$.  Our energy density in terms of these phasors is then

\begin{equation}\label{eqn:relElectroDynProblemSet4:270}
\mathcal{E} 
= \inv{8\pi} (\BE^2 + \BB^2)
= \inv{16\pi} \left( \Abs{\vec{E}}^2 + \Abs{\vec{B}}^2 + \Real ({\vec{E}}^2 + {\vec{B}}^2) \right).
\end{equation}

This is
\begin{align*}
\mathcal{E} 
&=
\inv{16\pi}
\left(
\frac{\omega^2}{c^2} \Abs{\Bb}^2 + \Abs{\Bb \cross \Bk}^2
-\Real \left(
\frac{\omega^2}{c^2} \Bb^2 + (\Bb \cross \Bk)^2
\right)
e^{2 i(\theta - \alpha)} 
\right).
\end{align*}

Note that $\omega^2/c^2 = \Bk^2$, and $\Abs{\Bb \cross \Bk} = \Abs{\Bb}^2 \Bk^2$ (since $\Bb \cdot \Bk = 0$).  Also $(\Bb \cross \Bk)^2 = \Bb^2 \Bk^2$, so we have

\begin{equation}\label{eqn:relElectroDynProblemSet4:290}
\boxed{
\mathcal{E} 
=
\frac{ \Bk^2 }{8\pi}
\left(
\Abs{\Bb}^2 
-\Real \Bb^2 e^{2 i(\theta - \alpha)} 
\right).
}
\end{equation}

Now, for the Poynting vector.  We have

\begin{equation}\label{eqn:relElectroDynProblemSet4:310}
S = \frac{c}{4 \pi} \BE \cross \BB = \frac{c}{8 \pi} \Real \left( \vec{E} \cross \vec{B}^\conj + \vec{E} \cross \vec{B} \right).
\end{equation}

This is
\begin{align*}
S 
&= \frac{c}{8 \pi} \Real \left( -k \Bb \cross (\Bb^\conj \cross \Bk) + k \Bb \cross (\Bb \cross \Bk ) e^{2 i(\theta - \alpha)} \right) \\
\end{align*}

Reducing the terms we get $\Bb \cross (\Bb^\conj \cross \Bk) = -\Bk \Abs{\Bb}^2$, and $\Bb \cross (\Bb \cross \Bk) = -\Bk \Bb^2$, leaving

\begin{equation}\label{eqn:relElectroDynProblemSet4:330}
\boxed{
S 
= \frac{c \hat{\Bk} \Bk^2 }{8 \pi} \left( \Abs{\Bb}^2 - \Real \Bb^2 e^{2 i(\theta - \alpha)} \right) = c \hat{\Bk} \mathcal{E}
}
\end{equation}

Now, the text in \S 47 defines the energy flux as the Poynting vector, and the momentum density as $\BS/c^2$, so we just divide \ref{eqn:relElectroDynProblemSet4:330} by $c^2$ for the momentum density and we are done.  For the linearly polarized case (all that was actually asked for, but less cool to calculate), where $\Bb$ is real, we have

\begin{align}\label{eqn:relElectroDynProblemSet4:350}
\mbox{Energy density} &= \mathcal{E} = \frac{ \Bk^2 \Bb^2 }{8\pi} ( 1 - \cos( 2 (\omega t - \Bk \cdot \Bx)) ) \\
\mbox{Energy flux} &= \BS = c \hat{\Bk} \mathcal{E} \\
\mbox{Momentum density} &= \inv{c^2} \BS = \frac{\hat{\Bk}}{c} \mathcal{E}
\end{align}

\subsubsection{Part 2.  Averaged.}
\subsubsection{Part 3.  Pressure.}
\subsubsection{Part 4.  Sunlight.}

\section{Problem 2.}

\subsection{Statement}
\subsection{Solution}


\EndArticle
%\EndNoBibArticle
