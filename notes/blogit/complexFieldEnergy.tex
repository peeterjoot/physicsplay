\documentclass[]{eliblog}

\usepackage{amsmath}
\usepackage{mathpazo}

%
% shorthand for bold symbols, convenient for vectors and matrices
%
\newcommand{\Ba}[0]{\mathbf{a}}
\newcommand{\Bb}[0]{\mathbf{b}}
\newcommand{\Bc}[0]{\mathbf{c}}
\newcommand{\Bd}[0]{\mathbf{d}}
\newcommand{\Be}[0]{\mathbf{e}}
\newcommand{\Bf}[0]{\mathbf{f}}
\newcommand{\Bg}[0]{\mathbf{g}}
\newcommand{\Bh}[0]{\mathbf{h}}
\newcommand{\Bi}[0]{\mathbf{i}}
\newcommand{\Bj}[0]{\mathbf{j}}
\newcommand{\Bk}[0]{\mathbf{k}}
\newcommand{\Bl}[0]{\mathbf{l}}
\newcommand{\Bm}[0]{\mathbf{m}}
\newcommand{\Bn}[0]{\mathbf{n}}
\newcommand{\Bo}[0]{\mathbf{o}}
\newcommand{\Bp}[0]{\mathbf{p}}
\newcommand{\Bq}[0]{\mathbf{q}}
\newcommand{\Br}[0]{\mathbf{r}}
\newcommand{\Bs}[0]{\mathbf{s}}
\newcommand{\Bt}[0]{\mathbf{t}}
\newcommand{\Bu}[0]{\mathbf{u}}
\newcommand{\Bv}[0]{\mathbf{v}}
\newcommand{\Bw}[0]{\mathbf{w}}
\newcommand{\Bx}[0]{\mathbf{x}}
\newcommand{\By}[0]{\mathbf{y}}
\newcommand{\Bz}[0]{\mathbf{z}}
\newcommand{\BA}[0]{\mathbf{A}}
\newcommand{\BB}[0]{\mathbf{B}}
\newcommand{\BC}[0]{\mathbf{C}}
\newcommand{\BD}[0]{\mathbf{D}}
\newcommand{\BE}[0]{\mathbf{E}}
\newcommand{\BF}[0]{\mathbf{F}}
\newcommand{\BG}[0]{\mathbf{G}}
\newcommand{\BH}[0]{\mathbf{H}}
\newcommand{\BI}[0]{\mathbf{I}}
\newcommand{\BJ}[0]{\mathbf{J}}
\newcommand{\BK}[0]{\mathbf{K}}
\newcommand{\BL}[0]{\mathbf{L}}
\newcommand{\BM}[0]{\mathbf{M}}
\newcommand{\BN}[0]{\mathbf{N}}
\newcommand{\BO}[0]{\mathbf{O}}
\newcommand{\BP}[0]{\mathbf{P}}
\newcommand{\BQ}[0]{\mathbf{Q}}
\newcommand{\BR}[0]{\mathbf{R}}
\newcommand{\BS}[0]{\mathbf{S}}
\newcommand{\BT}[0]{\mathbf{T}}
\newcommand{\BU}[0]{\mathbf{U}}
\newcommand{\BV}[0]{\mathbf{V}}
\newcommand{\BW}[0]{\mathbf{W}}
\newcommand{\BX}[0]{\mathbf{X}}
\newcommand{\BY}[0]{\mathbf{Y}}
\newcommand{\BZ}[0]{\mathbf{Z}}

\newcommand{\Bzero}[0]{\mathbf{0}}
\newcommand{\Btheta}[0]{\boldsymbol{\theta}}
\newcommand{\Btau}[0]{\boldsymbol{\tau}}
\newcommand{\Bomega}[0]{\boldsymbol{\omega}}

%
% shorthand for unit vectors
%
\newcommand{\acap}[0]{\hat{\Ba}}
\newcommand{\bcap}[0]{\hat{\Bb}}
\newcommand{\ccap}[0]{\hat{\Bc}}
\newcommand{\dcap}[0]{\hat{\Bd}}
\newcommand{\ecap}[0]{\hat{\Be}}
\newcommand{\fcap}[0]{\hat{\Bf}}
\newcommand{\gcap}[0]{\hat{\Bg}}
\newcommand{\hcap}[0]{\hat{\Bh}}
\newcommand{\icap}[0]{\hat{\Bi}}
\newcommand{\jcap}[0]{\hat{\Bj}}
\newcommand{\kcap}[0]{\hat{\Bk}}
\newcommand{\lcap}[0]{\hat{\Bl}}
\newcommand{\mcap}[0]{\hat{\Bm}}
\newcommand{\ncap}[0]{\hat{\Bn}}
\newcommand{\ocap}[0]{\hat{\Bo}}
\newcommand{\pcap}[0]{\hat{\Bp}}
\newcommand{\qcap}[0]{\hat{\Bq}}
\newcommand{\rcap}[0]{\hat{\Br}}
\newcommand{\scap}[0]{\hat{\Bs}}
\newcommand{\tcap}[0]{\hat{\Bt}}
\newcommand{\ucap}[0]{\hat{\Bu}}
\newcommand{\vcap}[0]{\hat{\Bv}}
\newcommand{\wcap}[0]{\hat{\Bw}}
\newcommand{\xcap}[0]{\hat{\Bx}}
\newcommand{\ycap}[0]{\hat{\By}}
\newcommand{\zcap}[0]{\hat{\Bz}}
\newcommand{\thetacap}[0]{\hat{\Btheta}}

%
% to write R^n and C^n in a distinguishable fashion.  Perhaps change this
% to the double lined characters upon figuring out how to do so.
%
\newcommand{\C}[1]{$\mathbb{C}^{#1}$}
\newcommand{\R}[1]{$\mathbb{R}^{#1}$}

%
% various generally useful helpers
%

% derivative of #1 wrt. #2:
\newcommand{\D}[2] {\frac {d#2} {d#1}}

\newcommand{\inv}[1]{\frac{1}{#1}}
\newcommand{\cross}[0]{\times}

\newcommand{\abs}[1]{\lvert{#1}\rvert}
\newcommand{\norm}[1]{\lVert{#1}\rVert}
\newcommand{\innerprod}[2]{\langle{#1}, {#2}\rangle}
\newcommand{\dotprod}[2]{{#1} \cdot {#2}}
\newcommand{\bdotprod}[2]{\left({#1} \cdot {#2}\right)}
\newcommand{\crossprod}[2]{{#1} \cross {#2}}
\newcommand{\tripleprod}[3]{\dotprod{\left(\crossprod{#1}{#2}\right)}{#3}}

\DeclareMathOperator{\Proj}{Proj}
\DeclareMathOperator{\Span}{span}
\DeclareMathOperator{\Sgn}{sgn}
\DeclareMathOperator{\Area}{Area}
\DeclareMathOperator{\Volume}{Volume}

%
% A few miscellaneous things specific to this document
%
\newcommand{\crossop}[1]{\crossprod{#1}{}}

% R2 vector.
\newcommand{\VectorTwo}[2]{
\begin{bmatrix}
 {#1} \\
 {#2}
\end{bmatrix}
}

\newcommand{\VectorN}[1]{
\begin{bmatrix}
{#1}_1 \\
{#1}_2 \\
\vdots \\
{#1}_N \\
\end{bmatrix}
}

\newcommand{\DETuvij}[4]{
\begin{vmatrix}
 {#1}_{#3} & {#1}_{#4} \\
 {#2}_{#3} & {#2}_{#4}
\end{vmatrix}
}

\newcommand{\DETuvwijk}[6]{
\begin{vmatrix}
 {#1}_{#4} & {#1}_{#5} & {#1}_{#6} \\
 {#2}_{#4} & {#2}_{#5} & {#2}_{#6} \\
 {#3}_{#4} & {#3}_{#5} & {#3}_{#6}
\end{vmatrix}
}

\newcommand{\DETuvwxijkl}[8]{
\begin{vmatrix}
 {#1}_{#5} & {#1}_{#6} & {#1}_{#7} & {#1}_{#8} \\
 {#2}_{#5} & {#2}_{#6} & {#2}_{#7} & {#2}_{#8} \\
 {#3}_{#5} & {#3}_{#6} & {#3}_{#7} & {#3}_{#8} \\
 {#4}_{#5} & {#4}_{#6} & {#4}_{#7} & {#4}_{#8} \\
\end{vmatrix}
}

%\newcommand{\DETuvwxyijklm}[10]{
%\begin{vmatrix}
% {#1}_{#6} & {#1}_{#7} & {#1}_{#8} & {#1}_{#9} & {#1}_{#10} \\
% {#2}_{#6} & {#2}_{#7} & {#2}_{#8} & {#2}_{#9} & {#2}_{#10} \\
% {#3}_{#6} & {#3}_{#7} & {#3}_{#8} & {#3}_{#9} & {#3}_{#10} \\
% {#4}_{#6} & {#4}_{#7} & {#4}_{#8} & {#4}_{#9} & {#4}_{#10} \\
% {#5}_{#6} & {#5}_{#7} & {#5}_{#8} & {#5}_{#9} & {#5}_{#10}
%\end{vmatrix}
%}

% R3 vector.
\newcommand{\VectorThree}[3]{
\begin{bmatrix}
 {#1} \\
 {#2} \\
 {#3}
\end{bmatrix}
}



\author{Peeter Joot}
\email{peeter.joot@gmail.com}


\chapter{Energy and momentum for Complex electric and magnetic field phasors.}
\label{chap:complexFieldEnergy}
%\useCCL
\blogpage{http://sites.google.com/site/peeterjoot/math2009/complexFieldEnergy.pdf}
\date{Dec 13, 2009}
\revisionInfo{complexFieldEnergy.tex}

\beginArtWithToc
%\beginArtNoToc

\section{Motivation.}

In \cite{jackson1975cew} a complex phasor representations of the electric and magnetic fields is used

\begin{align}\label{eqn:complexFieldEnergy:1}
\BE &= \EE e^{-i\omega t} \\
\BB &= \BB e^{-i\omega t}.
\end{align}

Here the vectors $\EE$ and $\BB$ are allowed to take on complex values.  Jackson uses the real part of these complex vectors as the true fields, so we are really interested in just these quantities

\begin{subequations}
\begin{align}\label{eqn:complexFieldEnergy:2}
\Re \BE &= \EE_r \cos(\omega t) + \EE_i \sin(\omega t) \\
\Re \BB &= \BB_r \cos(\omega t) + \BB_i \sin(\omega t),
\end{align}
\end{subequations}

but carry the whole thing in manipulations to make things simpler.  It is stated that the energy for such complex vector fields takes the form (ignoring constant scaling factors and units)

\begin{align}\label{eqn:complexFieldEnergy:3}
\text{Energy} \propto \BE \cdot \conjugateStar{\BE} + \BB \cdot \conjugateStar{\BB}.
\end{align}

In some ways this is an obvious generalization.  Less obvious is how this and the Poynting vector are related in their corresponding conservation relationships.  

Here I explore this, employing a Geometric Algebra representation of the energy momentum tensor based on the real field representation found in \cite{doran2003gap}.  Given the complex valued fields and a requirement that both the real and imaginary parts of the field satisfy Maxwell's equation, it should be possible to derive the conservation relationship between the energy density and Poynting vector from first principles.

\section{Review of GA formalism for real fields.}

In SI units the Geometric algebra form of Maxwell's equation is

\label{eqn:complexFieldEnergy:4}
\begin{align}
\grad F &= J/\epsilon_0 c,
\end{align}

where we have for the symbols

\begin{subequations}
\begin{align}
\label{eqn:complexFieldEnergy:5}
F &= \BE + c I \BB \\
I &= \gamma_0 \gamma_1 \gamma_2 \gamma_3 \\
\BE &= E^k \gamma_k \gamma_0  \\
\BB &= B^k \gamma_k \gamma_0  \\
(\gamma^0)^2 &= -(\gamma^k)^2 = 1 \\
\gamma^\mu \cdot \gamma_\nu &= {\delta^\mu}_\nu \\
J &= c \rho \gamma_0 + J^k \gamma_k \\
\grad &= \gamma^\mu \partial_\mu = \gamma^\mu \PDi{x^\mu}{}.
\end{align}
\end{subequations}

The symmetric electrodynamic energy momentum tensor for real fields $\BE$ and $\BB$ is

\begin{align}\label{eqn:complexFieldEnergy:6}
T(a) &= \frac{-\epsilon_0}{2} F a F = \frac{\epsilon_0}{2} F a \tilde{F}.
\end{align}

It may not be obvious that this is in fact a four vector, but this can be seen since it can only have grade one and three components, and also equals its reverse implying that the grade three terms are all zero.  To illustrate this explicitly consider the components of $T^{\mu 0}$

\begin{align*}
2 \epsilon_0 T(\gamma^0) 
&= -(\BE + c I \BB) \gamma^0 (\BE + c I \BB) \\
&= (\BE + c I \BB) (\BE - c I \BB) \gamma^0 \\
&= (\BE^2 + c^2 \BB^2 + c I (\BB \BE - \BE \BB)) \gamma^0 \\
&= (\BE^2 + c^2 \BB^2) \gamma^0 + 2 c I ( \BB \wedge \BE ) \gamma^0 \\
&= (\BE^2 + c^2 \BB^2) \gamma^0 + 2 c ( \BE \cross \BB ) \gamma^0 \\
\end{align*}

Our result is a four vector in the Dirac basis as expected

\begin{subequations}
\begin{align}\label{eqn:complexFieldEnergy:7}
T(\gamma^0) &= T^{\mu 0} \gamma_\mu \\
T^{0 0} &= \inv{2 \epsilon_0} (\BE^2 + c^2 \BB^2) \\
T^{k 0} &= \frac{c}{\epsilon_0} (\BE \cross \BB)_k 
\end{align}
\end{subequations}

Similar expansions are possible for the general tensor components $T^{\mu\nu}$ but performing the more general expansion is not particularily illuminating.  The main point here is to remind ourself how to express the energy momentum tensor in a fashion that is natural in a GA context.  We also know that we have a conservation relationship associated with the divergence of this tensor $\grad \cdot T(a)$ (ie. $\partial_\mu T^{\mu\nu}$), and want to rederive this relationship after guessing what form the GA expression for the energy momentum tensor takes when we allow the field vectors to take complex values.

\section{Computing the conservation relationship for complex field vectors.}

As in \ref{eqn:complexFieldEnergy:3}, if we want 

\begin{align}\label{eqn:complexFieldEnergy:8}
T^{0 0} \propto \BE \cdot \conjugateStar{\BE} + c^2 \BB \cdot \conjugateStar{\BB},
\end{align}

it is reasonable to assume that our energy momentum tensor will take the form

\begin{align}\label{eqn:complexFieldEnergy:9}
T(a) &= \frac{\epsilon_0}{4} \left( \conjugateStar{F} a \tilde{F} + \tilde{F} a \conjugateStar{F} \right).
\end{align}

For real vector fields this reduces to the previous results and should produce the desired mix of real and imaginary dot products for the energy density term of the tensor.  This is also a real four vector even when the field is complex, so the energy density and power density terms will all be real valued, which seems desirable.

Working with this, let's calculate the divergence and see what we find for the corresponding conservation relationship.  It will be convienient as well to work temporarily with natural units $c = \epsilon_0 = 1$, restoring these at the end of the calculation.

\EndArticle
%\EndNoBibArticle
