%
% Copyright � 2015 Peeter Joot.  All Rights Reserved.
% Licenced as described in the file LICENSE under the root directory of this GIT repository.
%
\documentclass[]{eliblog}

\usepackage{amsmath}
\usepackage{mathpazo}

%
% shorthand for bold symbols, convenient for vectors and matrices
%
\newcommand{\Ba}[0]{\mathbf{a}}
\newcommand{\Bb}[0]{\mathbf{b}}
\newcommand{\Bc}[0]{\mathbf{c}}
\newcommand{\Bd}[0]{\mathbf{d}}
\newcommand{\Be}[0]{\mathbf{e}}
\newcommand{\Bf}[0]{\mathbf{f}}
\newcommand{\Bg}[0]{\mathbf{g}}
\newcommand{\Bh}[0]{\mathbf{h}}
\newcommand{\Bi}[0]{\mathbf{i}}
\newcommand{\Bj}[0]{\mathbf{j}}
\newcommand{\Bk}[0]{\mathbf{k}}
\newcommand{\Bl}[0]{\mathbf{l}}
\newcommand{\Bm}[0]{\mathbf{m}}
\newcommand{\Bn}[0]{\mathbf{n}}
\newcommand{\Bo}[0]{\mathbf{o}}
\newcommand{\Bp}[0]{\mathbf{p}}
\newcommand{\Bq}[0]{\mathbf{q}}
\newcommand{\Br}[0]{\mathbf{r}}
\newcommand{\Bs}[0]{\mathbf{s}}
\newcommand{\Bt}[0]{\mathbf{t}}
\newcommand{\Bu}[0]{\mathbf{u}}
\newcommand{\Bv}[0]{\mathbf{v}}
\newcommand{\Bw}[0]{\mathbf{w}}
\newcommand{\Bx}[0]{\mathbf{x}}
\newcommand{\By}[0]{\mathbf{y}}
\newcommand{\Bz}[0]{\mathbf{z}}
\newcommand{\BA}[0]{\mathbf{A}}
\newcommand{\BB}[0]{\mathbf{B}}
\newcommand{\BC}[0]{\mathbf{C}}
\newcommand{\BD}[0]{\mathbf{D}}
\newcommand{\BE}[0]{\mathbf{E}}
\newcommand{\BF}[0]{\mathbf{F}}
\newcommand{\BG}[0]{\mathbf{G}}
\newcommand{\BH}[0]{\mathbf{H}}
\newcommand{\BI}[0]{\mathbf{I}}
\newcommand{\BJ}[0]{\mathbf{J}}
\newcommand{\BK}[0]{\mathbf{K}}
\newcommand{\BL}[0]{\mathbf{L}}
\newcommand{\BM}[0]{\mathbf{M}}
\newcommand{\BN}[0]{\mathbf{N}}
\newcommand{\BO}[0]{\mathbf{O}}
\newcommand{\BP}[0]{\mathbf{P}}
\newcommand{\BQ}[0]{\mathbf{Q}}
\newcommand{\BR}[0]{\mathbf{R}}
\newcommand{\BS}[0]{\mathbf{S}}
\newcommand{\BT}[0]{\mathbf{T}}
\newcommand{\BU}[0]{\mathbf{U}}
\newcommand{\BV}[0]{\mathbf{V}}
\newcommand{\BW}[0]{\mathbf{W}}
\newcommand{\BX}[0]{\mathbf{X}}
\newcommand{\BY}[0]{\mathbf{Y}}
\newcommand{\BZ}[0]{\mathbf{Z}}

\newcommand{\Bzero}[0]{\mathbf{0}}
\newcommand{\Btheta}[0]{\boldsymbol{\theta}}
\newcommand{\Btau}[0]{\boldsymbol{\tau}}
\newcommand{\Bomega}[0]{\boldsymbol{\omega}}

%
% shorthand for unit vectors
%
\newcommand{\acap}[0]{\hat{\Ba}}
\newcommand{\bcap}[0]{\hat{\Bb}}
\newcommand{\ccap}[0]{\hat{\Bc}}
\newcommand{\dcap}[0]{\hat{\Bd}}
\newcommand{\ecap}[0]{\hat{\Be}}
\newcommand{\fcap}[0]{\hat{\Bf}}
\newcommand{\gcap}[0]{\hat{\Bg}}
\newcommand{\hcap}[0]{\hat{\Bh}}
\newcommand{\icap}[0]{\hat{\Bi}}
\newcommand{\jcap}[0]{\hat{\Bj}}
\newcommand{\kcap}[0]{\hat{\Bk}}
\newcommand{\lcap}[0]{\hat{\Bl}}
\newcommand{\mcap}[0]{\hat{\Bm}}
\newcommand{\ncap}[0]{\hat{\Bn}}
\newcommand{\ocap}[0]{\hat{\Bo}}
\newcommand{\pcap}[0]{\hat{\Bp}}
\newcommand{\qcap}[0]{\hat{\Bq}}
\newcommand{\rcap}[0]{\hat{\Br}}
\newcommand{\scap}[0]{\hat{\Bs}}
\newcommand{\tcap}[0]{\hat{\Bt}}
\newcommand{\ucap}[0]{\hat{\Bu}}
\newcommand{\vcap}[0]{\hat{\Bv}}
\newcommand{\wcap}[0]{\hat{\Bw}}
\newcommand{\xcap}[0]{\hat{\Bx}}
\newcommand{\ycap}[0]{\hat{\By}}
\newcommand{\zcap}[0]{\hat{\Bz}}
\newcommand{\thetacap}[0]{\hat{\Btheta}}

%
% to write R^n and C^n in a distinguishable fashion.  Perhaps change this
% to the double lined characters upon figuring out how to do so.
%
\newcommand{\C}[1]{$\mathbb{C}^{#1}$}
\newcommand{\R}[1]{$\mathbb{R}^{#1}$}

%
% various generally useful helpers
%

% derivative of #1 wrt. #2:
\newcommand{\D}[2] {\frac {d#2} {d#1}}

\newcommand{\inv}[1]{\frac{1}{#1}}
\newcommand{\cross}[0]{\times}

\newcommand{\abs}[1]{\lvert{#1}\rvert}
\newcommand{\norm}[1]{\lVert{#1}\rVert}
\newcommand{\innerprod}[2]{\langle{#1}, {#2}\rangle}
\newcommand{\dotprod}[2]{{#1} \cdot {#2}}
\newcommand{\bdotprod}[2]{\left({#1} \cdot {#2}\right)}
\newcommand{\crossprod}[2]{{#1} \cross {#2}}
\newcommand{\tripleprod}[3]{\dotprod{\left(\crossprod{#1}{#2}\right)}{#3}}

\DeclareMathOperator{\Proj}{Proj}
\DeclareMathOperator{\Span}{span}
\DeclareMathOperator{\Sgn}{sgn}
\DeclareMathOperator{\Area}{Area}
\DeclareMathOperator{\Volume}{Volume}

%
% A few miscellaneous things specific to this document
%
\newcommand{\crossop}[1]{\crossprod{#1}{}}

% R2 vector.
\newcommand{\VectorTwo}[2]{
\begin{bmatrix}
 {#1} \\
 {#2}
\end{bmatrix}
}

\newcommand{\VectorN}[1]{
\begin{bmatrix}
{#1}_1 \\
{#1}_2 \\
\vdots \\
{#1}_N \\
\end{bmatrix}
}

\newcommand{\DETuvij}[4]{
\begin{vmatrix}
 {#1}_{#3} & {#1}_{#4} \\
 {#2}_{#3} & {#2}_{#4}
\end{vmatrix}
}

\newcommand{\DETuvwijk}[6]{
\begin{vmatrix}
 {#1}_{#4} & {#1}_{#5} & {#1}_{#6} \\
 {#2}_{#4} & {#2}_{#5} & {#2}_{#6} \\
 {#3}_{#4} & {#3}_{#5} & {#3}_{#6}
\end{vmatrix}
}

\newcommand{\DETuvwxijkl}[8]{
\begin{vmatrix}
 {#1}_{#5} & {#1}_{#6} & {#1}_{#7} & {#1}_{#8} \\
 {#2}_{#5} & {#2}_{#6} & {#2}_{#7} & {#2}_{#8} \\
 {#3}_{#5} & {#3}_{#6} & {#3}_{#7} & {#3}_{#8} \\
 {#4}_{#5} & {#4}_{#6} & {#4}_{#7} & {#4}_{#8} \\
\end{vmatrix}
}

%\newcommand{\DETuvwxyijklm}[10]{
%\begin{vmatrix}
% {#1}_{#6} & {#1}_{#7} & {#1}_{#8} & {#1}_{#9} & {#1}_{#10} \\
% {#2}_{#6} & {#2}_{#7} & {#2}_{#8} & {#2}_{#9} & {#2}_{#10} \\
% {#3}_{#6} & {#3}_{#7} & {#3}_{#8} & {#3}_{#9} & {#3}_{#10} \\
% {#4}_{#6} & {#4}_{#7} & {#4}_{#8} & {#4}_{#9} & {#4}_{#10} \\
% {#5}_{#6} & {#5}_{#7} & {#5}_{#8} & {#5}_{#9} & {#5}_{#10}
%\end{vmatrix}
%}

% R3 vector.
\newcommand{\VectorThree}[3]{
\begin{bmatrix}
 {#1} \\
 {#2} \\
 {#3}
\end{bmatrix}
}



\author{Peeter Joot}
\email{peeter.joot@gmail.com}

%\documentclass[]{eliblogwidescreen}

\usepackage{amsmath}
\usepackage{mathpazo}

%
% shorthand for bold symbols, convenient for vectors and matrices
%
\newcommand{\Ba}[0]{\mathbf{a}}
\newcommand{\Bb}[0]{\mathbf{b}}
\newcommand{\Bc}[0]{\mathbf{c}}
\newcommand{\Bd}[0]{\mathbf{d}}
\newcommand{\Be}[0]{\mathbf{e}}
\newcommand{\Bf}[0]{\mathbf{f}}
\newcommand{\Bg}[0]{\mathbf{g}}
\newcommand{\Bh}[0]{\mathbf{h}}
\newcommand{\Bi}[0]{\mathbf{i}}
\newcommand{\Bj}[0]{\mathbf{j}}
\newcommand{\Bk}[0]{\mathbf{k}}
\newcommand{\Bl}[0]{\mathbf{l}}
\newcommand{\Bm}[0]{\mathbf{m}}
\newcommand{\Bn}[0]{\mathbf{n}}
\newcommand{\Bo}[0]{\mathbf{o}}
\newcommand{\Bp}[0]{\mathbf{p}}
\newcommand{\Bq}[0]{\mathbf{q}}
\newcommand{\Br}[0]{\mathbf{r}}
\newcommand{\Bs}[0]{\mathbf{s}}
\newcommand{\Bt}[0]{\mathbf{t}}
\newcommand{\Bu}[0]{\mathbf{u}}
\newcommand{\Bv}[0]{\mathbf{v}}
\newcommand{\Bw}[0]{\mathbf{w}}
\newcommand{\Bx}[0]{\mathbf{x}}
\newcommand{\By}[0]{\mathbf{y}}
\newcommand{\Bz}[0]{\mathbf{z}}
\newcommand{\BA}[0]{\mathbf{A}}
\newcommand{\BB}[0]{\mathbf{B}}
\newcommand{\BC}[0]{\mathbf{C}}
\newcommand{\BD}[0]{\mathbf{D}}
\newcommand{\BE}[0]{\mathbf{E}}
\newcommand{\BF}[0]{\mathbf{F}}
\newcommand{\BG}[0]{\mathbf{G}}
\newcommand{\BH}[0]{\mathbf{H}}
\newcommand{\BI}[0]{\mathbf{I}}
\newcommand{\BJ}[0]{\mathbf{J}}
\newcommand{\BK}[0]{\mathbf{K}}
\newcommand{\BL}[0]{\mathbf{L}}
\newcommand{\BM}[0]{\mathbf{M}}
\newcommand{\BN}[0]{\mathbf{N}}
\newcommand{\BO}[0]{\mathbf{O}}
\newcommand{\BP}[0]{\mathbf{P}}
\newcommand{\BQ}[0]{\mathbf{Q}}
\newcommand{\BR}[0]{\mathbf{R}}
\newcommand{\BS}[0]{\mathbf{S}}
\newcommand{\BT}[0]{\mathbf{T}}
\newcommand{\BU}[0]{\mathbf{U}}
\newcommand{\BV}[0]{\mathbf{V}}
\newcommand{\BW}[0]{\mathbf{W}}
\newcommand{\BX}[0]{\mathbf{X}}
\newcommand{\BY}[0]{\mathbf{Y}}
\newcommand{\BZ}[0]{\mathbf{Z}}

\newcommand{\Bzero}[0]{\mathbf{0}}
\newcommand{\Btheta}[0]{\boldsymbol{\theta}}
\newcommand{\Btau}[0]{\boldsymbol{\tau}}
\newcommand{\Bomega}[0]{\boldsymbol{\omega}}

%
% shorthand for unit vectors
%
\newcommand{\acap}[0]{\hat{\Ba}}
\newcommand{\bcap}[0]{\hat{\Bb}}
\newcommand{\ccap}[0]{\hat{\Bc}}
\newcommand{\dcap}[0]{\hat{\Bd}}
\newcommand{\ecap}[0]{\hat{\Be}}
\newcommand{\fcap}[0]{\hat{\Bf}}
\newcommand{\gcap}[0]{\hat{\Bg}}
\newcommand{\hcap}[0]{\hat{\Bh}}
\newcommand{\icap}[0]{\hat{\Bi}}
\newcommand{\jcap}[0]{\hat{\Bj}}
\newcommand{\kcap}[0]{\hat{\Bk}}
\newcommand{\lcap}[0]{\hat{\Bl}}
\newcommand{\mcap}[0]{\hat{\Bm}}
\newcommand{\ncap}[0]{\hat{\Bn}}
\newcommand{\ocap}[0]{\hat{\Bo}}
\newcommand{\pcap}[0]{\hat{\Bp}}
\newcommand{\qcap}[0]{\hat{\Bq}}
\newcommand{\rcap}[0]{\hat{\Br}}
\newcommand{\scap}[0]{\hat{\Bs}}
\newcommand{\tcap}[0]{\hat{\Bt}}
\newcommand{\ucap}[0]{\hat{\Bu}}
\newcommand{\vcap}[0]{\hat{\Bv}}
\newcommand{\wcap}[0]{\hat{\Bw}}
\newcommand{\xcap}[0]{\hat{\Bx}}
\newcommand{\ycap}[0]{\hat{\By}}
\newcommand{\zcap}[0]{\hat{\Bz}}
\newcommand{\thetacap}[0]{\hat{\Btheta}}

%
% to write R^n and C^n in a distinguishable fashion.  Perhaps change this
% to the double lined characters upon figuring out how to do so.
%
\newcommand{\C}[1]{$\mathbb{C}^{#1}$}
\newcommand{\R}[1]{$\mathbb{R}^{#1}$}

%
% various generally useful helpers
%

% derivative of #1 wrt. #2:
\newcommand{\D}[2] {\frac {d#2} {d#1}}

\newcommand{\inv}[1]{\frac{1}{#1}}
\newcommand{\cross}[0]{\times}

\newcommand{\abs}[1]{\lvert{#1}\rvert}
\newcommand{\norm}[1]{\lVert{#1}\rVert}
\newcommand{\innerprod}[2]{\langle{#1}, {#2}\rangle}
\newcommand{\dotprod}[2]{{#1} \cdot {#2}}
\newcommand{\bdotprod}[2]{\left({#1} \cdot {#2}\right)}
\newcommand{\crossprod}[2]{{#1} \cross {#2}}
\newcommand{\tripleprod}[3]{\dotprod{\left(\crossprod{#1}{#2}\right)}{#3}}

\DeclareMathOperator{\Proj}{Proj}
\DeclareMathOperator{\Span}{span}
\DeclareMathOperator{\Sgn}{sgn}
\DeclareMathOperator{\Area}{Area}
\DeclareMathOperator{\Volume}{Volume}

%
% A few miscellaneous things specific to this document
%
\newcommand{\crossop}[1]{\crossprod{#1}{}}

% R2 vector.
\newcommand{\VectorTwo}[2]{
\begin{bmatrix}
 {#1} \\
 {#2}
\end{bmatrix}
}

\newcommand{\VectorN}[1]{
\begin{bmatrix}
{#1}_1 \\
{#1}_2 \\
\vdots \\
{#1}_N \\
\end{bmatrix}
}

\newcommand{\DETuvij}[4]{
\begin{vmatrix}
 {#1}_{#3} & {#1}_{#4} \\
 {#2}_{#3} & {#2}_{#4}
\end{vmatrix}
}

\newcommand{\DETuvwijk}[6]{
\begin{vmatrix}
 {#1}_{#4} & {#1}_{#5} & {#1}_{#6} \\
 {#2}_{#4} & {#2}_{#5} & {#2}_{#6} \\
 {#3}_{#4} & {#3}_{#5} & {#3}_{#6}
\end{vmatrix}
}

\newcommand{\DETuvwxijkl}[8]{
\begin{vmatrix}
 {#1}_{#5} & {#1}_{#6} & {#1}_{#7} & {#1}_{#8} \\
 {#2}_{#5} & {#2}_{#6} & {#2}_{#7} & {#2}_{#8} \\
 {#3}_{#5} & {#3}_{#6} & {#3}_{#7} & {#3}_{#8} \\
 {#4}_{#5} & {#4}_{#6} & {#4}_{#7} & {#4}_{#8} \\
\end{vmatrix}
}

%\newcommand{\DETuvwxyijklm}[10]{
%\begin{vmatrix}
% {#1}_{#6} & {#1}_{#7} & {#1}_{#8} & {#1}_{#9} & {#1}_{#10} \\
% {#2}_{#6} & {#2}_{#7} & {#2}_{#8} & {#2}_{#9} & {#2}_{#10} \\
% {#3}_{#6} & {#3}_{#7} & {#3}_{#8} & {#3}_{#9} & {#3}_{#10} \\
% {#4}_{#6} & {#4}_{#7} & {#4}_{#8} & {#4}_{#9} & {#4}_{#10} \\
% {#5}_{#6} & {#5}_{#7} & {#5}_{#8} & {#5}_{#9} & {#5}_{#10}
%\end{vmatrix}
%}

% R3 vector.
\newcommand{\VectorThree}[3]{
\begin{bmatrix}
 {#1} \\
 {#2} \\
 {#3}
\end{bmatrix}
}



\author{Peeter Joot}
\email{peeter.joot@gmail.com}


\chapter{PHY454H1S Continuum Mechanics.  Lecture 19: Boundary layers and stream functions.  Taught by Prof. K. Das.}
\label{chap:continuumL19}
%\useCCL
\blogpage{http://sites.google.com/site/peeterjoot2/math2012/continuumL19.pdf}
\date{Mar 23, 2012}
\gitRevisionInfo{continuumL19}

\beginArtWithToc
%\beginArtNoToc

\section{Disclaimer.}

Peeter's lecture notes from class.  May not be entirely coherent.

\section{Review.  Laminar boundary layer theory.}

In the boundary layer we found
\begin{enumerate}
\item Continuity equation
\begin{equation}\label{eqn:continuumL19:n}
\PD{x}{u} + \PD{y}{v} = 0
\end{equation}
\item $x$ momentum equation
\begin{equation}\label{eqn:continuumL19:n}
u \PD{x}{u} + v \PD{y}{u} = - \inv{\rho} \PD{x}{p} + \nu \PDSq{y}{u}
\end{equation}
\item $y$ momentum equation
\begin{equation}\label{eqn:continuumL19:n}
\PD{y}{p} = 0
\end{equation}
\end{enumerate}

In the invisid region $p$ is a constant in $y$

\begin{equation}\label{eqn:continuumL19:n}
\PD{y}{p} = 0
\end{equation}

This will be approximately true in the boundary layer too as illustrated in

FIXME: F1k

Starting with

\begin{equation}\label{eqn:continuumL19:n}
\PD{t}{\Bu} + (\Bu \cdot \spacegrad) \Bu = -\spacegrad \left( \frac{p}{\rho} + \chi \right) + \nu \spacegrad^2 \Bu,
\end{equation}

we were able to show that inviscid irrotational incompressible flows are governed by Bernoulli's equation

\begin{equation}\label{eqn:continuumL19:n}
\frac{p}{\rho} + \chi + \inv{2} \Bu^2 = \text{constant}
\end{equation}

In the abscence of body forces (or constant potentials), we have

\begin{equation}\label{eqn:continuumL19:n}
-\PD{x}{}\frac{p}{\rho} = \PD{x}{} \left(\inv{2} \Bu^2 \right) \sim U \PD{x}{U}
\end{equation}

so that our boundary layer equations are 

\begin{subequations}
\begin{equation}\label{eqn:continuumL18:n}
u \PD{x}{u} + v \PD{y}{u} = U \frac{dU}{dx} + \nu \PDSq{y}{u}
\end{equation}
\begin{equation}\label{eqn:continuumL18:n}
\PD{y}{p} = 0
\end{equation}
\begin{equation}\label{eqn:continuumL18:n}
\PD{x}{u} + \PD{y}{v} = 0
\end{equation}
\end{subequations}

With boundary conditions

\begin{align}\label{eqn:continuumProblemSet2:n}
U(x, 0) &= 0 \\
U(x, \infty) &= U(x) \\
V(x, 0) &= 0
\end{align}

\section{Fluid flow over a flat plate (Blasius problem).}

Define a similarity variable $\eta$ 

\begin{equation}\label{eqn:continuumL19:n}
\eta \sim \frac{y}{\sqrt{\nu t}}
\end{equation}

Suppose we want

\begin{equation}\label{eqn:continuumL19:n}
\eta \sim f(y, x)
\end{equation}

Since we have

\begin{equation}\label{eqn:continuumL19:n}
x \sim U t,
\end{equation}

or

\begin{equation}\label{eqn:continuumL19:n}
t \sim \frac{x}{U}.
\end{equation}

We can make the transformation

\begin{equation}\label{eqn:continuumL19:n}
\eta = \frac{y}{\sqrt{2 \frac{\nu x}{U}}}
\end{equation}

We can introduce stream functions

\begin{align}\label{eqn:continuumProblemSet2:n}
u &= \PD{y}{\psi} \\
v &= -\PD{x}{\psi}
\end{align}

We can check that this satisfies the continuity equation since we have

\begin{align*}
\PD{x}{u} + \PD{y}{v} 
&=
\frac{\partial^2 \psi}{\partial x \partial y}
-\frac{\partial^2 \psi}{\partial y \partial x} \\
&= 0
\end{align*}

Now introduce a similarity variable 

\begin{equation}\label{eqn:continuumL19:n}
f(\eta) = \frac{\psi}{\delta U_0} = \frac{\psi}{\sqrt{2 \nu x U_0}}
\end{equation}

We find

\begin{equation}\label{eqn:continuumL19:n}
f''' + f f'' = 0.
\end{equation}

FIXME: show.

We can solve this numerically and find solutions that look like

FIXME: F2

Our boundary conditions are

\begin{equation}\label{eqn:continuumL19:n}
\begin{array}{l l}
f = f' = 0 & \quad \mbox{$\eta = 0$} \\
f' = 1 & \quad \mbox{$\eta = \infty$} \\
\end{array}
\end{equation}

FIXME: show this.

This is the Blasius solution to the problem of fluid flow over a flat plate.

\section{Singlular pertubation theory.}

In the boundary layer our inertial term

\begin{equation}\label{eqn:continuumL19:n}
u \PD{x}{u} \sim \nu \PDSq{y}{u}
\end{equation}

or

\begin{equation}\label{eqn:continuumL19:n}
\frac{U}{L} \sim \frac{\nu U}{\delta^2}
\end{equation}

or
FIXME: show in better detail:
\begin{equation}\label{eqn:continuumL19:n}
\delta  \sim \frac{\nu L}{U}
\end{equation}

finally
\begin{equation}\label{eqn:continuumL19:n}
\frac{\delta}{L} \sim \frac{\nu}{U L} \sim (\text{Re})^{-1/2}
\end{equation}

so if

\begin{equation}\label{eqn:continuumL19:n}
\frac{\delta}{L} \ll 1
\end{equation}

\begin{equation}\label{eqn:continuumL19:n}
\frac{\delta}{L} \ll \inv{\sqrt{\text{Re}}}
\end{equation}

so 

\begin{equation}\label{eqn:continuumL19:n}
\frac{\delta}{L} \ll 1
\end{equation}

when $\text{Re} >> 1$.

(this is the whole reason that we were able to do the previous analysis).

Our EOM is

\begin{equation}\label{eqn:continuumL19:n}
(\Bu \cdot \spacegrad) \Bu = -\frac{\spacegrad p}{\rho } + \nu \spacegrad^2 \Bu
\end{equation}

with 

\begin{align}\label{eqn:continuumL19:n}
\Bu &\rightarrow U \\
p &\rightarrow U^2 \rho
\end{align}

as $x, y \rightarrow L$

performing a non-dimensionalization we have

\begin{equation}\label{eqn:continuumL19:n}
(\Bu' \cdot \spacegrad') \Bu' = -\spacegrad' p' + \frac{\nu}{U L} {\spacegrad'}^2 \Bu'
\end{equation}

or

\begin{equation}\label{eqn:continuumL19:n}
(\Bu \cdot \spacegrad) \Bu = -\spacegrad p + \inv{\text{Re}} {\spacegrad}^2 \Bu
\end{equation}

to force $\text{Re} \rightarrow \infty$, we can write

\begin{equation}\label{eqn:continuumL19:n}
\inv{\text{Re}} = \epsilon.
\end{equation}

so that as $\epsilon \rightarrow 0$ we have $\text{Re} \rightarrow \infty$.

With a very small number modifying the highest degree partial term, we have a class of differential equations that doesn't end up converging should we attempt a standard pertubation treatment.  An example that is analogous is the differential equation

\begin{equation}\label{eqn:continuumL19:n}
\epsilon \frac{du}{dx} + u = x,
\end{equation}

where $\epsilon \ll 1$ and $u(0) = 1$.  The exact solution of this ill conditioned differential equation is

\begin{equation}\label{eqn:continuumL19:n}
u = (1 + \epsilon) e^{-x/\epsilon} + x - \epsilon
\end{equation}

This is illustrated in

FIXME: F3

Study of this class of problems is called \em{Singular pertubation theory}.

When $x \gg \epsilon$ we have approximately

\begin{equation}\label{eqn:continuumL19:n}
u \sim x,
\end{equation}

but when $x \sim O(\epsilon)$, $x/\epsilon \sim O(1)$ we have approximately

\begin{equation}\label{eqn:continuumL19:n}
u \sim e^{-x/\epsilon}.
\end{equation}

%FIXME: Reading: \S XX from \cite{acheson1990elementary} 
%\EndArticle
\EndNoBibArticle
