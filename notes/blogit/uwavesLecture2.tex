%
% Copyright � 2016 Peeter Joot.  All Rights Reserved.
% Licenced as described in the file LICENSE under the root directory of this GIT repository.
%
\newcommand{\authorname}{Peeter Joot}
\newcommand{\email}{peeterjoot@protonmail.com}
\newcommand{\basename}{FIXMEbasenameUndefined}
\newcommand{\dirname}{notes/FIXMEdirnameUndefined/}

\renewcommand{\basename}{uwaves2}
\renewcommand{\dirname}{notes/ece1236/}
\newcommand{\keywords}{ECE1236H}
\newcommand{\authorname}{Peeter Joot}
\newcommand{\onlineurl}{http://sites.google.com/site/peeterjoot2/math2013/\basename.pdf}
\newcommand{\sourcepath}{\dirname\basename.tex}
\newcommand{\generatetitle}[1]{\chapter{#1}}

\newcommand{\vcsinfo}{%
\section*{}
\noindent{\color{DarkOliveGreen}{\rule{\linewidth}{0.1mm}}}
\paragraph{Document version}
%\paragraph{\color{Maroon}{Document version}}
{
\small
\begin{itemize}
\item Available online at:\\ 
\href{\onlineurl}{\onlineurl}
\item Git Repository: \input{./.revinfo/gitRepo.tex}
\item Source: \sourcepath
\item last commit: \input{./.revinfo/gitCommitString.tex}
\item commit date: \input{./.revinfo/gitCommitDate.tex}
\end{itemize}
}
}

%\PassOptionsToPackage{dvipsnames,svgnames}{xcolor}
\PassOptionsToPackage{square,numbers}{natbib}
\documentclass{scrreprt}

\usepackage[left=2cm,right=2cm]{geometry}
\usepackage[svgnames]{xcolor}
\usepackage{peeters_layout}

\usepackage{natbib}

\usepackage[
colorlinks=true,
bookmarks=false,
pdfauthor={\authorname, \email},
backref 
]{hyperref}

% http://tex.stackexchange.com/questions/75773/how-to-reference-problems-by-the-text-label-in-an-exercise-envioronment
\usepackage[english]{cleveref}
\crefname{Exercise}{exercise}{exercises}
\Crefname{Exercise}{Exercise}{Exercises}

\RequirePackage{titlesec}
\RequirePackage{ifthen}

% http://stackoverflow.com/questions/4932910/date-in-the-tabular-environment
\makeatletter
\let\insertdate\@date
\makeatother

\titleformat{\chapter}[display]
{\bfseries\Large}
{\color{DarkSlateGrey}\filleft \authorname
\ifthenelse{\isundefined{\studentnumber}}{}{\\ \studentnumber}
\ifthenelse{\isundefined{\email}}{}{\\ \email}
\ifthenelse{\isundefined{\dateintitle}}{}{\\ \insertdate}
%\ifthenelse{\isundefined{\coursename}}{}{\\ \coursename} % put in title instead.
}
{4ex}
{\color{DarkOliveGreen}{\titlerule}\color{Maroon}
\vspace{2ex}%
\filright}
[\vspace{2ex}%
\color{DarkOliveGreen}\titlerule
]

\newcommand{\beginArtWithToc}[0]{\begin{document}\tableofcontents}
\newcommand{\beginArtNoToc}[0]{\begin{document}}
\newcommand{\EndNoBibArticle}[0]{\end{document}}
\newcommand{\EndArticle}[0]{\bibliography{Bibliography}\bibliographystyle{plainnat}\end{document}}

% 
%\newcommand{\citep}[1]{\cite{#1}}

\colorSectionsForArticle



%\usepackage{ece1236}
\usepackage{peeters_braket}
%\usepackage{peeters_layout_exercise}
\usepackage{peeters_figures}
\usepackage{mathtools}

\beginArtNoToc
\generatetitle{ECE1236H Microwave and Millimeter-Wave Techniques.  Lecture 2: Wave equation.  Taught by Prof.\ G.V. Eleftheriades}
%\chapter{Wave equation}
\label{chap:uwaves2}

\paragraph{Disclaimer}

Peeter's lecture notes from class.  These may be incoherent and rough.

These are notes for the UofT course ECE1236H, Microwave and Millimeter-Wave Techniques, taught by Prof. G.V. Eleftheriades, covering \textchapref{{1}} \citep{pozar2009microwave} content.

\section{Plane waves}

In simple media with no sources

\begin{subequations}
\label{eqn:uwavesLecture2:20}
\begin{equation}\label{eqn:uwavesLecture2:40}
\spacegrad \cross \BE = - \PD{t}{\BB} = - \mu \PD{t}{\BH},
\end{equation}
\begin{equation}\label{eqn:uwavesLecture2:60}
\spacegrad \cdot \BH = 0.
\end{equation}
\begin{equation}\label{eqn:uwavesLecture2:80}
\spacegrad \cross \BH = 
\BJ + \PD{t}{\BD} = 
\lr{ \sigma + \epsilon \PD{t}{} } \BE
\end{equation}
\begin{equation}\label{eqn:uwavesLecture2:100}
\spacegrad \cdot \BE = \frac{\rho}{\epsilon} = 0
\end{equation}
\end{subequations}

Here \( \BJ = \sigma \BE \) is considered to be an induced current, and the \( \rho/\epsilon \) term vanishes since we have no sources.

From Faraday's law \cref{eqn:uwavesLecture2:40}

\begin{dmath}\label{eqn:uwavesLecture2:120}
\spacegrad \cross \lr{ \spacegrad \cross \BE }
= - \mu \PD{t}{} \lr { \spacegrad \cross \BH }
= - \mu \PD{t}{} \lr { \sigma + \epsilon \PD{t}{} } \BE
= - \mu \sigma \PD{t}{\BE} - \epsilon \mu \PDSq{t}{\BE}.
\end{dmath}

We can also use the vector identity

\begin{dmath}\label{eqn:uwavesLecture2:140}
\spacegrad \cross \lr{ \spacegrad \cross \BF } = \spacegrad \lr{ \spacegrad \cdot \BF } - \spacegrad^2 \BF,
\end{dmath}

so 
\begin{dmath}\label{eqn:uwavesLecture2:160}
\spacegrad \cross \lr{ \spacegrad \cross \BE }
=
\spacegrad \lr{ \cancel{\spacegrad \cdot \BE} } - \spacegrad^2 \BE,
\end{dmath}

or
\begin{dmath}\label{eqn:uwavesLecture2:180}
0 = \spacegrad^2 \BE - \mu \sigma \PD{t}{\BE} - \epsilon \mu \PDSq{t}{\BE}.
\end{dmath}

The \( \mu \sigma \PDi{t}{\BE} \) term is a damping contribution.

Similarily for the magnetic field
\begin{dmath}\label{eqn:uwavesLecture2:200}
0 = \spacegrad^2 \BH - \mu \sigma \PD{t}{\BH} - \epsilon \mu \PDSq{t}{\BH}.
\end{dmath}

This is the wave or Helmholtz equation.

Note that for the no loss condition \( \sigma = 0 \), we have the undamped wave equations
\begin{subequations}
\label{eqn:uwavesLecture2:220}
\begin{dmath}\label{eqn:uwavesLecture2:240}
\spacegrad^2 \BE = \epsilon \mu \PDSq{t}{\BE}.
\end{dmath}
\begin{dmath}\label{eqn:uwavesLecture2:260}
\spacegrad^2 \BH = \epsilon \mu \PDSq{t}{\BH}.
\end{dmath}
\end{subequations}

This is one wave equation for each of \( E_x, E_y, E_z, H_x, H_y\) and \( H_z \).

\paragraph{Propagation in one direction}

Consider the undamped propagation along the z direction of \( E_x(z, t) \), which must satisfy
\begin{dmath}\label{eqn:uwavesLecture2:400}
\spacegrad^2 E_x = \epsilon \mu \PDSq{t}{E_x},
\end{dmath}

or
\begin{dmath}\label{eqn:uwavesLecture2:420}
\PDSq{z}{E_x} = \epsilon \mu \PDSq{t}{E_x}.
\end{dmath}

This is analogous to the voltage transmission line equation
\begin{dmath}\label{eqn:uwavesLecture2:440}
\PDSq{z}{V} = L C \PDSq{t}{V},
\end{dmath}

With solution 

\begin{equation}\label{eqn:uwavesLecture2:280}
V(z, t) = 
V_0^{+} f(z - v_\phi t)
+
V_0^{-} f(z + v_\phi t),
\end{equation}

where \( v_\phi = \ifrac{1}{\sqrt{L C}} \) is the phase velocity, \( f(z - v_\phi t)\) is the forward wave, and \( f(z + v_\phi t) \) is the reflected wave.  By analogy the solution to the 1D wave equation is

\begin{equation}\label{eqn:uwavesLecture2:300}
E_x(z, t) = 
E_0^{+} f(z - v_\phi t)
+
E_0^{-} f(z + v_\phi t),
\end{equation}

where \( v_\phi = \ifrac{1}{\sqrt{\epsilon \mu}} \) is the phase velocity.

For myself, the transmission line equation is something that I only encountered in this class, so this analogy isn't a great one.  That said, wave equation solutions are very familiar, so not much motivation for the structure of the solution is really required.

\paragraph{More rigorous derivation of the 1D solution}

Suppose that we assume the form of the solution is

\begin{equation}\label{eqn:uwavesLecture2:320}
\BE = \BE_0 f(z - v_\phi t ),
\end{equation}

where \( \BE_0 \) is a constant vector, so this describes a wave propagation in the z-direction, but without a-priori knowledge of the direction of this vector in space.

For source free media, we have

\begin{dmath}\label{eqn:uwavesLecture2:340}
\spacegrad \cdot ( \BE_0 f ) 
= \spacegrad f \cdot \BE_0 + f \cancel{ \spacegrad \cdot \BE }
= (\zcap f') \cdot \BE_0.
\end{dmath}

Because \( \BE_0 \cdot \zcap = 0 \), the vector \( \BE_0 \) is perpendicular to the direction of propagation ( z ).  The next task is to find the magnetic field that couples to this electric field solution.  Suppose the coordinate axis are picked so that 

\begin{equation}\label{eqn:uwavesLecture2:360}
\BE(z, t) = \BE_0 E_x( z \pm v_\phi t).
\end{equation}

Using Faraday's law \( \spacegrad \cross \BE = -\mu \PDi{t}{\BH} \) gives
\begin{dmath}\label{eqn:uwavesLecture2:380}
\spacegrad \cross (\BE_0 E_x)
=
(\spacegrad \cross \BE_0) E_x
-\BE_0 \cross \spacegrad E_x
=
-\BE_0 \cross \spacegrad E_x
=
-\BE_0 \cross \zcap \PD{z}{E_x},
\end{dmath}

so

\begin{dmath}\label{eqn:uwavesLecture2:460}
-\mu \PD{t}{\BH} = 
\lr{ \zcap \cross \BE_0 } \PD{z}{E_x}.
\end{dmath}

Because of the \( E_x( z \pm v_\phi t ) \) dependence on \( z, t \), we must have

\begin{dmath}\label{eqn:uwavesLecture2:480}
\PD{z}{E_x} = \pm \inv{v_\phi} \PD{t}{E_x},
\end{dmath}

or
\begin{dmath}\label{eqn:uwavesLecture2:500}
-\mu \PD{t}{\BH} = 
\lr{ \zcap \cross \BE_0 } 
= \pm \inv{v_\phi} \PD{t}{E_x}.
\end{dmath}

Integrating with respect to \( t \) and assuming a zero integration constant (physical justification for that?), we have

\begin{dmath}\label{eqn:uwavesLecture2:520}
\BH = \mp E_x(z, t) \frac{\zcap \cross \BE_0}{v_\phi \mu}.
\end{dmath}

The \( v_\phi \mu \) product is the intrinsic wave impedance of the medium

\begin{dmath}\label{eqn:uwavesLecture2:540}
v_\phi \mu = \frac{\mu}{\sqrt{\epsilon\mu}} = \sqrt{\frac{\mu}{\epsilon}} = \eta.
\end{dmath}

The magnetic field can be written in terms of the propagation direction \( \ncap = \mp \zcap \) has

\begin{equation}\label{eqn:uwavesLecture2:560}
\BH 
= \mp \frac{\zcap \cross \BE}{\eta}
= \frac{\ncap \cross \BE}{\eta}.
\end{equation}

Note that in free space we have \( \eta \approx 377 \Omega \).

\paragraph{summary}

For 1D wave propagation along the z axis we have

\begin{equation}\label{eqn:uwavesLecture2:580}
\begin{aligned}
\spacegrad^2 \BE &= \epsilon \mu \PDSq{t}{\BE} \\
\spacegrad^2 \BH &= \epsilon \mu \PDSq{t}{\BH} \\
\end{aligned}
\end{equation}

where 

\begin{equation}\label{eqn:uwavesLecture2:600}
\begin{aligned}
\BE &= \BE_0 f( z \mp v_\phi t ) \\
0 &= \BE_0 \cdot \zcap \\
\BH &= \pm \frac{\zcap \cross \BE_0}{\eta} f( z \mp v_\phi t )
\end{aligned}
\end{equation}

Note that \( \pm \zcap \) is the direction of propagation, with \( +z \) being the forward wave, and \( -z \) the backwards wave.

The phase velocity is

\begin{equation}\label{eqn:uwavesLecture2:620}
v_\phi = \frac{1}{\sqrt{\mu\epsilon}},
\end{equation}

and the intrinsic wave impedance is 

\begin{equation}\label{eqn:uwavesLecture2:640}
\eta = \sqrt{\frac{\mu}{\epsilon}}.
\end{equation}

Notes:

\begin{itemize}
\item \( \BE, \BH \) are perpendicular to the direction of propagation \( \zcap \).  These are transverse waves.
\item \( \Abs{\BE}/\Abs{\BH} = \eta \).  An analogy with transmission lines is
\begin{equation}\label{eqn:uwavesLecture2:660}
\begin{aligned}
V &\leftrightarrow E \\
I &\leftrightarrow H \\
Z_0 &\leftrightarrow \eta
\end{aligned}
\end{equation}
\item \( \BE, \BH \) are orthogonal to each other

F1: B.5
\end{itemize}

\section{Time harmonic fields}

\EndArticle
%\EndNoBibArticle
