%
% Copyright � 2013 Peeter Joot.  All Rights Reserved.
% Licenced as described in the file LICENSE under the root directory of this GIT repository.
%
\newcommand{\authorname}{Peeter Joot}
\newcommand{\email}{peeterjoot@protonmail.com}
\newcommand{\basename}{FIXMEbasenameUndefined}
\newcommand{\dirname}{notes/FIXMEdirnameUndefined/}

\renewcommand{\basename}{relativisiticDensityOfStates}
\renewcommand{\dirname}{notes/phy452/}
\newcommand{\keywords}{Statistics mechanics, PHY452H1S, density of states, relativistic}

\newcommand{\authorname}{Peeter Joot}
\newcommand{\onlineurl}{http://sites.google.com/site/peeterjoot2/math2013/\basename.pdf}
\newcommand{\sourcepath}{\dirname\basename.tex}
\newcommand{\generatetitle}[1]{\chapter{#1}}

\newcommand{\vcsinfo}{%
\section*{}
\noindent{\color{DarkOliveGreen}{\rule{\linewidth}{0.1mm}}}
\paragraph{Document version}
%\paragraph{\color{Maroon}{Document version}}
{
\small
\begin{itemize}
\item Available online at:\\ 
\href{\onlineurl}{\onlineurl}
\item Git Repository: \input{./.revinfo/gitRepo.tex}
\item Source: \sourcepath
\item last commit: \input{./.revinfo/gitCommitString.tex}
\item commit date: \input{./.revinfo/gitCommitDate.tex}
\end{itemize}
}
}

%\PassOptionsToPackage{dvipsnames,svgnames}{xcolor}
\PassOptionsToPackage{square,numbers}{natbib}
\documentclass{scrreprt}

\usepackage[left=2cm,right=2cm]{geometry}
\usepackage[svgnames]{xcolor}
\usepackage{peeters_layout}

\usepackage{natbib}

\usepackage[
colorlinks=true,
bookmarks=false,
pdfauthor={\authorname, \email},
backref 
]{hyperref}

% http://tex.stackexchange.com/questions/75773/how-to-reference-problems-by-the-text-label-in-an-exercise-envioronment
\usepackage[english]{cleveref}
\crefname{Exercise}{exercise}{exercises}
\Crefname{Exercise}{Exercise}{Exercises}

\RequirePackage{titlesec}
\RequirePackage{ifthen}

% http://stackoverflow.com/questions/4932910/date-in-the-tabular-environment
\makeatletter
\let\insertdate\@date
\makeatother

\titleformat{\chapter}[display]
{\bfseries\Large}
{\color{DarkSlateGrey}\filleft \authorname
\ifthenelse{\isundefined{\studentnumber}}{}{\\ \studentnumber}
\ifthenelse{\isundefined{\email}}{}{\\ \email}
\ifthenelse{\isundefined{\dateintitle}}{}{\\ \insertdate}
%\ifthenelse{\isundefined{\coursename}}{}{\\ \coursename} % put in title instead.
}
{4ex}
{\color{DarkOliveGreen}{\titlerule}\color{Maroon}
\vspace{2ex}%
\filright}
[\vspace{2ex}%
\color{DarkOliveGreen}\titlerule
]

\newcommand{\beginArtWithToc}[0]{\begin{document}\tableofcontents}
\newcommand{\beginArtNoToc}[0]{\begin{document}}
\newcommand{\EndNoBibArticle}[0]{\end{document}}
\newcommand{\EndArticle}[0]{\bibliography{Bibliography}\bibliographystyle{plainnat}\end{document}}

% 
%\newcommand{\citep}[1]{\cite{#1}}

\colorSectionsForArticle



\beginArtNoToc

\generatetitle{Relativisitic density of states}
%\chapter{Relativisitic density of states}
\label{chap:relativisiticDensityOfStates}

\paragraph{Setup}

For photons and high velocity particles our non-relativisitic density of states is insufficient.  Let's redo these calculations for particles for which the energy is given by

\begin{dmath}\label{eqn:relativisiticDensityOfStates:20}
\epsilon = \sqrt{ 
\lr{m c^2}^2 
+ (p c)^2 }.
\end{dmath}

We want to convert a sum over momentum values to an energy integral

\begin{dmath}\label{eqn:relativisiticDensityOfStates:40}
\mathcal{D}_3(\epsilon)
=
\sum_\Bp 
\delta( \epsilon - \epsilon_\Bp )
\rightarrow 
L^d
\int \frac{d^d \Bk}{(2 \pi)^d}
\delta( \epsilon - \epsilon_\Bp )
=
L^d
\int \frac{d^3 \Bp}{(2 \pi \, \hbar)^d}
\delta( \epsilon - \epsilon_\Bp )
=
L^d
\int \frac{d^d (c \Bp)}{(c h)^d}
\delta( \epsilon - \epsilon_\Bp ).
\end{dmath}

Now we want to use

\begin{dmath}\label{eqn:relativisiticDensityOfStates:60}
\delta(g(x)) 
= \sum_{x_0} \frac{ \delta(x - x_0)}{ 
\Abs{g'(x)}_{x = x_0}
},
\end{dmath}

where $x_0$ are the roots of $g(x)$.  With 

\begin{dmath}\label{eqn:relativisiticDensityOfStates:80}
g( cp ) = \epsilon - \sqrt{ 
\lr{ m c^2 }^2 
+ ( c p )^2 }.
\end{dmath}

Writing $p^\conj$ for the roots we have

\begin{dmath}\label{eqn:relativisiticDensityOfStates:100}
c p^\conj 
= 
\sqrt{ \epsilon^2 - 
\lr{ m c^2 }^2 
}.
\end{dmath}

Note that 

\begin{equation}\label{eqn:relativisiticDensityOfStates:120}
\sqrt{ 
\lr{ m c^2 }^2
 + ( c p^\conj )^2 }
=
\sqrt{ \epsilon^2 } = \epsilon.
\end{equation}

we have
\begin{dmath}\label{eqn:relativisiticDensityOfStates:140}
\Abs{g'( c p )}_{p = p^\conj}
= \inv{2} \frac{2 (c p^\conj)}
{
	\sqrt{ 
		\lr{ m c^2 }^2 
		+ ( c p^\conj )^2 
	}
}
=
\frac{ \sqrt{ \epsilon^2 - \lr{ m c^2 }^2 } }{\epsilon}.
\end{dmath}

\paragraph{3D case}

We can now evaluate the density of states, and do the 3D case first.  We have

\begin{dmath}\label{eqn:relativisiticDensityOfStates:160}
\mathcal{D}_3(\epsilon)
=
\frac{V}{ (c h)^3 } \int_0^\infty 4 \pi (c p)^2 d (c p)
\lr{
	\delta\lr{ c p - \sqrt{ \epsilon^2 - \lr{ m c^2 }^2 } }
	+
	\delta\lr{ c p + \sqrt{ \epsilon^2 - \lr{ m c^2 }^2 } }
}
\frac{ \sqrt{ \epsilon^2 - \lr{ m c^2 }^2 } }{\epsilon}.
\end{dmath}

Observe that in the switch to spherical coordinates in momentum space, our integration is now over a ``radius'' of momentum space, requiring just integration over the positive values.  This will kill off one of our delta functions, leaving just

\begin{dmath}\label{eqn:relativisiticDensityOfStates:340}
\mathcal{D}_3(\epsilon)
=
\frac{4 \pi V}{ (c h)^3 } 
\lr{
	\epsilon^2 - \lr{ m c^2 }^2 
}
\frac{ \sqrt{ \epsilon^2 - \lr{ m c^2 }^2 } }{\epsilon},
\end{dmath}

or

\begin{dmath}\label{eqn:relativisiticDensityOfStates:360}
\myBoxed{
\mathcal{D}_3(\epsilon)
=
\frac{4 \pi V}{ (c h)^3 } 
\frac{\lr{
	\epsilon^2 - \lr{ m c^2 }^2 
}^{3/2}
}{\epsilon}.
}
\end{dmath}

In particular, for very high energy particles where $\epsilon \gg \lr{m c^2}$, our 3D density of states is

\begin{dmath}\label{eqn:relativisiticDensityOfStates:380}
\myBoxed{
\mathcal{D}_3(\epsilon)
\approx
\frac{4 \pi V}{ (c h)^3 } \epsilon^2
}
\end{dmath}

This is also the desired result for photons or other massless particles.

\paragraph{2D case}

For 2D we have

\begin{dmath}\label{eqn:relativisiticDensityOfStates:200}
\mathcal{D}_2(\epsilon)
=
\frac{A}{ (c h)^2 } \int_0^\infty 2 \pi \Abs{c p} d (c p)
\lr{
	\delta\lr{ c p - \sqrt{ \epsilon^2 - \lr{ m c^2 }^2 } }
	+
	\delta\lr{ c p + \sqrt{ \epsilon^2 - \lr{ m c^2 }^2 } }
}
\frac{ \sqrt{ \epsilon^2 - \lr{ m c^2 }^2 } }{\epsilon}.
\end{dmath}

Note again that we are dealing with a ``radius'' over this shell of momentum space volume.  This is a strictly positive value.  That and the cooresponding integration range is important in this case since including the negative range of $c p$ would kill the entire density function because of the pair of delta functions.  That wasn't the case in 3D, where it would have resulted in an off by two error instead.  Continuing the evaluation we have

\begin{dmath}\label{eqn:relativisiticDensityOfStates:220}
\mathcal{D}_2(\epsilon)
=
\frac{2 \pi A}{ (c h)^2 } 
\sqrt{ \epsilon^2 - \lr{ m c^2 }^2 } 
\frac{ \sqrt{ \epsilon^2 - \lr{ m c^2 }^2 } }{\epsilon},
\end{dmath}

or

\begin{dmath}\label{eqn:relativisiticDensityOfStates:240}
\myBoxed{
\mathcal{D}_2(\epsilon)
=
\frac{2 \pi A}{ (c h)^2 } 
\frac{ \epsilon^2 - \lr{ m c^2 }^2 }{ \epsilon }.
}
\end{dmath}

For an extreme relativisitic gas where $\epsilon \gg m c^2$ (or photons where $m = 0$), we have

\begin{dmath}\label{eqn:relativisiticDensityOfStates:260}
\myBoxed{
\mathcal{D}_2(\epsilon)
\approx
\frac{2 \pi A}{ (c h)^2 } \epsilon.
}
\end{dmath}

\paragraph{1D case}

\begin{dmath}\label{eqn:relativisiticDensityOfStates:280}
\mathcal{D}_1(\epsilon)
=
\frac{L}{ c h } \int d (c p)
\lr{
	\delta\lr{ c p - \sqrt{ \epsilon^2 - \lr{ m c^2 }^2 } }
	+
	\delta\lr{ c p + \sqrt{ \epsilon^2 - \lr{ m c^2 }^2 } }
}
\frac{ \sqrt{ \epsilon^2 - \lr{ m c^2 }^2 } }{\epsilon}.
\end{dmath}

\paragraph{Question}: For the 1D case, we don't have to make a switch to spherical or cylindrical coordinates, so it looks like the second delta function has to be included, and the integration range over both positive and negative values of $c p$?

Assuming that's the case, we have

\begin{dmath}\label{eqn:relativisiticDensityOfStates:300}
\myBoxed{
\mathcal{D}_1(\epsilon)
=
\frac{2 L}{ c h } 
\frac{ \sqrt{ \epsilon^2 - \lr{ m c^2 }^2 } }{\epsilon},
}
\end{dmath}

and for $\epsilon \gg m c^2$ or $m = 0$

\begin{dmath}\label{eqn:relativisiticDensityOfStates:320}
\myBoxed{
\mathcal{D}_1(\epsilon)
=
\frac{2 L}{ c h }.
}
\end{dmath}

\EndNoBibArticle
