%
% Copyright � 2017 Peeter Joot.  All Rights Reserved.
% Licenced as described in the file LICENSE under the root directory of this GIT repository.
%
%{
\newcommand{\authorname}{Peeter Joot}
\newcommand{\email}{peeterjoot@protonmail.com}
\newcommand{\basename}{FIXMEbasenameUndefined}
\newcommand{\dirname}{notes/FIXMEdirnameUndefined/}

\renewcommand{\basename}{gendot}
%\renewcommand{\dirname}{notes/phy1520/}
\renewcommand{\dirname}{notes/ece1228-electromagnetic-theory/}
%\newcommand{\dateintitle}{}
%\newcommand{\keywords}{}

\newcommand{\authorname}{Peeter Joot}
\newcommand{\onlineurl}{http://sites.google.com/site/peeterjoot2/math2013/\basename.pdf}
\newcommand{\sourcepath}{\dirname\basename.tex}
\newcommand{\generatetitle}[1]{\chapter{#1}}

\newcommand{\vcsinfo}{%
\section*{}
\noindent{\color{DarkOliveGreen}{\rule{\linewidth}{0.1mm}}}
\paragraph{Document version}
%\paragraph{\color{Maroon}{Document version}}
{
\small
\begin{itemize}
\item Available online at:\\ 
\href{\onlineurl}{\onlineurl}
\item Git Repository: \input{./.revinfo/gitRepo.tex}
\item Source: \sourcepath
\item last commit: \input{./.revinfo/gitCommitString.tex}
\item commit date: \input{./.revinfo/gitCommitDate.tex}
\end{itemize}
}
}

%\PassOptionsToPackage{dvipsnames,svgnames}{xcolor}
\PassOptionsToPackage{square,numbers}{natbib}
\documentclass{scrreprt}

\usepackage[left=2cm,right=2cm]{geometry}
\usepackage[svgnames]{xcolor}
\usepackage{peeters_layout}

\usepackage{natbib}

\usepackage[
colorlinks=true,
bookmarks=false,
pdfauthor={\authorname, \email},
backref 
]{hyperref}

% http://tex.stackexchange.com/questions/75773/how-to-reference-problems-by-the-text-label-in-an-exercise-envioronment
\usepackage[english]{cleveref}
\crefname{Exercise}{exercise}{exercises}
\Crefname{Exercise}{Exercise}{Exercises}

\RequirePackage{titlesec}
\RequirePackage{ifthen}

% http://stackoverflow.com/questions/4932910/date-in-the-tabular-environment
\makeatletter
\let\insertdate\@date
\makeatother

\titleformat{\chapter}[display]
{\bfseries\Large}
{\color{DarkSlateGrey}\filleft \authorname
\ifthenelse{\isundefined{\studentnumber}}{}{\\ \studentnumber}
\ifthenelse{\isundefined{\email}}{}{\\ \email}
\ifthenelse{\isundefined{\dateintitle}}{}{\\ \insertdate}
%\ifthenelse{\isundefined{\coursename}}{}{\\ \coursename} % put in title instead.
}
{4ex}
{\color{DarkOliveGreen}{\titlerule}\color{Maroon}
\vspace{2ex}%
\filright}
[\vspace{2ex}%
\color{DarkOliveGreen}\titlerule
]

\newcommand{\beginArtWithToc}[0]{\begin{document}\tableofcontents}
\newcommand{\beginArtNoToc}[0]{\begin{document}}
\newcommand{\EndNoBibArticle}[0]{\end{document}}
\newcommand{\EndArticle}[0]{\bibliography{Bibliography}\bibliographystyle{plainnat}\end{document}}

% 
%\newcommand{\citep}[1]{\cite{#1}}

\colorSectionsForArticle



\usepackage{peeters_layout_exercise}
\usepackage{peeters_braket}
\usepackage{peeters_figures}
\usepackage{siunitx}
%\usepackage{mhchem} % \ce{}
%\usepackage{macros_bm} % \bcM
%\usepackage{macros_qed} % \qedmarker
%\usepackage{txfonts} % \ointclockwise

\beginArtNoToc

\generatetitle{XXX}
%\chapter{XXX}
%\label{chap:gendot}
% \citep{sakurai2014modern} pr X.Y
% \citep{pozar2009microwave}
% \citep{qftLectureNotes}
% \citep{doran2003gap}
% \citep{jackson1975cew}
% \citep{griffiths1999introduction}

I don't have that book by MacDonald so I'm not sure of the lingo that he uses.

Define a k-vector as a quantity having a single grade.

Examples of a 2-vector:

\begin{dmath}\label{eqn:gendot:20}
\begin{aligned}
   &\Be_1 \Be_2 + \Be_3 \Be_4 \\
   &\Be_1 \Be_3
\end{aligned}
\end{dmath}

The general dot product formula for two k-vectors \( a_r, b_s \), of grades r and s respectively, is typically defined as a grade selection of the following sort:

\begin{dmath}\label{eqn:gendot:40}
a_r \cdot b_s
=
\gpgrade{ a_r b_s }{\Abs{r - s}}.
\end{dmath}

For 5.3.4 we have
\begin{dmath}\label{eqn:gendot:60}
\begin{aligned}
   \Ba \cdot (\Ba \wedge \Bb)
   &=
   \gpgradeone{ \Ba (\Ba \wedge \Bb) } \\
   &=
   \gpgradeone{ \Ba (\Ba \Bb - \Ba \cdot \Bb) } \\
   &=
   \gpgradeone{ \Ba \Ba \Bb }
   -\gpgradeone{ \Ba } (\Ba \cdot \Bb)  \\
   &=
   \Ba^2 \gpgradeone{ \Bb }
   -\gpgradeone{ \Ba } (\Ba \cdot \Bb)  \\
   &=
   \Ba^2 \Bb
   -\Ba (\Ba \cdot \Bb) ,
\end{aligned}
\end{dmath}

but \( \Ba \cdot \Bb = 0 \) for this problem, proving the desired result.

The dot product of a multivector is typically defined of the dot products of all the grade components of the multivector, as in

\begin{dmath}\label{eqn:gendot:80}
\begin{aligned}
A &= \sum_k \gpgrade{A}{k} \\
B &= \sum_k \gpgrade{B}{k} \\
A \cdot B
&=
\sum_{r, s} \gpgrade{A}{r} \cdot \gpgrade{B}{s},
\end{aligned}
\end{dmath}

so

\begin{dmath}\label{eqn:gendot:100}
\begin{aligned}
   \Ba \cdot ( \Bb \Bc )
   &=
   \Ba \cdot ( \Bb \cdot \Bc + \Bb \wedge \Bc ) \\
   &=
   \gpgrade{\Ba ( \Bb \cdot \Bc)}{1-0}
   +
   \gpgrade{\Ba ( \Bb \wedge \Bc)}{2-1} \\
   &=
   (\Bb \cdot \Bc) \Ba
   +
   (\Ba \cdot \Bb) \Bc
   -
   (\Ba \cdot \Bc) \Bb.
\end{aligned}
\end{dmath}

This is not generally equal to \( (\Ba \cdot \Bb) \Bc \).

%}
\EndArticle
%\EndNoBibArticle
