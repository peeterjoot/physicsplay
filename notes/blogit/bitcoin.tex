%
% Copyright � 2013 Peeter Joot.   All Rights Reserved.
% Licenced as described in the file LICENSE under the root directory of this GIT repository.
%
\newcommand{\authorname}{Peeter Joot}
\newcommand{\email}{peeterjoot@protonmail.com}
\newcommand{\basename}{FIXMEbasenameUndefined}
\newcommand{\dirname}{notes/FIXMEdirnameUndefined/}

\renewcommand{\basename}{bitcoin}
\renewcommand{\dirname}{notes/incoherentramblings/}
%\newcommand{\dateintitle}{}
%\newcommand{\keywords}{}

\newcommand{\authorname}{Peeter Joot}
\newcommand{\onlineurl}{http://sites.google.com/site/peeterjoot2/math2013/\basename.pdf}
\newcommand{\sourcepath}{\dirname\basename.tex}
\newcommand{\generatetitle}[1]{\chapter{#1}}

\newcommand{\vcsinfo}{%
\section*{}
\noindent{\color{DarkOliveGreen}{\rule{\linewidth}{0.1mm}}}
\paragraph{Document version}
%\paragraph{\color{Maroon}{Document version}}
{
\small
\begin{itemize}
\item Available online at:\\ 
\href{\onlineurl}{\onlineurl}
\item Git Repository: \input{./.revinfo/gitRepo.tex}
\item Source: \sourcepath
\item last commit: \input{./.revinfo/gitCommitString.tex}
\item commit date: \input{./.revinfo/gitCommitDate.tex}
\end{itemize}
}
}

%\PassOptionsToPackage{dvipsnames,svgnames}{xcolor}
\PassOptionsToPackage{square,numbers}{natbib}
\documentclass{scrreprt}

\usepackage[left=2cm,right=2cm]{geometry}
\usepackage[svgnames]{xcolor}
\usepackage{peeters_layout}

\usepackage{natbib}

\usepackage[
colorlinks=true,
bookmarks=false,
pdfauthor={\authorname, \email},
backref 
]{hyperref}

% http://tex.stackexchange.com/questions/75773/how-to-reference-problems-by-the-text-label-in-an-exercise-envioronment
\usepackage[english]{cleveref}
\crefname{Exercise}{exercise}{exercises}
\Crefname{Exercise}{Exercise}{Exercises}

\RequirePackage{titlesec}
\RequirePackage{ifthen}

% http://stackoverflow.com/questions/4932910/date-in-the-tabular-environment
\makeatletter
\let\insertdate\@date
\makeatother

\titleformat{\chapter}[display]
{\bfseries\Large}
{\color{DarkSlateGrey}\filleft \authorname
\ifthenelse{\isundefined{\studentnumber}}{}{\\ \studentnumber}
\ifthenelse{\isundefined{\email}}{}{\\ \email}
\ifthenelse{\isundefined{\dateintitle}}{}{\\ \insertdate}
%\ifthenelse{\isundefined{\coursename}}{}{\\ \coursename} % put in title instead.
}
{4ex}
{\color{DarkOliveGreen}{\titlerule}\color{Maroon}
\vspace{2ex}%
\filright}
[\vspace{2ex}%
\color{DarkOliveGreen}\titlerule
]

\newcommand{\beginArtWithToc}[0]{\begin{document}\tableofcontents}
\newcommand{\beginArtNoToc}[0]{\begin{document}}
\newcommand{\EndNoBibArticle}[0]{\end{document}}
\newcommand{\EndArticle}[0]{\bibliography{Bibliography}\bibliographystyle{plainnat}\end{document}}

% 
%\newcommand{\citep}[1]{\cite{#1}}

\colorSectionsForArticle



\beginArtNoToc

\generatetitle{Some thoughts about bitcoin}
%\chapter{Some thoughts about bitcoin}
%\label{chap:bitcoin}
\section{Why care about the monetary system?}

Reading Ellen Brown's ``Web of Debt'' \citep{brown2008web} was an eye opener for me.   

A portion of world history (mainly North American) is told from the viewpoint of banking manipulations.  Unlike conventional history, most warfare is framed as motivated by banking interests.  I was surprised how this was alternate narrative was framed as self evident, instead of as a thesis with supporting arguments and documentation.  The history that is presented has a high degree of self consistency that lends it some credence, but it is unconventional enough that it would have merited a more thorough treatment in and of itself without being coupled to some of the other opinions of this book.

Many subtle details of the modern debt based monetary system are explained in this book.  In particular, the mechanism that allows money to be created from debt is covered extensively.  The tricky concept of short selling was explained along with the ways that it can be used to manipulate the market.  Some of the concepts explained were confusing, and I'll personally have to study much more to make sense of them.

A large portion of this book is about the United States ``Fed'', its history, its backers and the sorts of manipulations that it allows, both internally and externally.  As a Canadian, it really makes me wonder what the state of our own system is, since the troubles of our neighbor probably allow for significant manipulations to go under the radar.

You'll find lots of details about how the world financial system got to be in such a mess.  I found that learning the extent of this mess was almost physically painful, a response that makes it easy to understand why so many people put up with the status-quo.  It is easier and more comfortable to just not know.

A number of possible alternative monetary schemes are considered.  The most weight is put on nationalization of banks, with debt free creation of money used instead of government borrowing the money that they print from private banks.  It seems implausible that enough people will learn the mechanics of our monetary system to force a grass roots movement that could wrest control of money creation from the private banking industry.  Given that it will be interesting to see if some of the smaller scale local monetary ideas catch on.

With so much insanity in this world done in the name of money, it is well worth the time to read this book to understand how money itself and debt based banking it is associated with works.


%\section{Guts}

% this is to produce the sites.google url and version info and so forth (for blog posts)
%\vcsinfo

\EndArticle
%\EndNoBibArticle
