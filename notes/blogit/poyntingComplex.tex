%
% Copyright � 2012 Peeter Joot.  All Rights Reserved.
% Licenced as described in the file LICENSE under the root directory of this GIT repository.
%
% pick one:
%\newcommand{\authorname}{Peeter Joot}
\newcommand{\email}{peeter.joot@utoronto.ca}
\newcommand{\studentnumber}{920798560}
\newcommand{\basename}{FIXMEbasenameUndefined}
\newcommand{\dirname}{notes/FIXMEdirnameUndefined/}

\newcommand{\authorname}{Peeter Joot}
\newcommand{\email}{peeterjoot@protonmail.com}
\newcommand{\basename}{FIXMEbasenameUndefined}
\newcommand{\dirname}{notes/FIXMEdirnameUndefined/}

\renewcommand{\basename}{poyntingComplex}
\renewcommand{\dirname}{notes/blogit/}
\newcommand{\keywords}{Poynting vector, complex, phasor}

\newcommand{\authorname}{Peeter Joot}
\newcommand{\onlineurl}{http://sites.google.com/site/peeterjoot2/math2013/\basename.pdf}
\newcommand{\sourcepath}{\dirname\basename.tex}
\newcommand{\generatetitle}[1]{\chapter{#1}}

\newcommand{\vcsinfo}{%
\section*{}
\noindent{\color{DarkOliveGreen}{\rule{\linewidth}{0.1mm}}}
\paragraph{Document version}
%\paragraph{\color{Maroon}{Document version}}
{
\small
\begin{itemize}
\item Available online at:\\ 
\href{\onlineurl}{\onlineurl}
\item Git Repository: \input{./.revinfo/gitRepo.tex}
\item Source: \sourcepath
\item last commit: \input{./.revinfo/gitCommitString.tex}
\item commit date: \input{./.revinfo/gitCommitDate.tex}
\end{itemize}
}
}

%\PassOptionsToPackage{dvipsnames,svgnames}{xcolor}
\PassOptionsToPackage{square,numbers}{natbib}
\documentclass{scrreprt}

\usepackage[left=2cm,right=2cm]{geometry}
\usepackage[svgnames]{xcolor}
\usepackage{peeters_layout}

\usepackage{natbib}

\usepackage[
colorlinks=true,
bookmarks=false,
pdfauthor={\authorname, \email},
backref 
]{hyperref}

% http://tex.stackexchange.com/questions/75773/how-to-reference-problems-by-the-text-label-in-an-exercise-envioronment
\usepackage[english]{cleveref}
\crefname{Exercise}{exercise}{exercises}
\Crefname{Exercise}{Exercise}{Exercises}

\RequirePackage{titlesec}
\RequirePackage{ifthen}

% http://stackoverflow.com/questions/4932910/date-in-the-tabular-environment
\makeatletter
\let\insertdate\@date
\makeatother

\titleformat{\chapter}[display]
{\bfseries\Large}
{\color{DarkSlateGrey}\filleft \authorname
\ifthenelse{\isundefined{\studentnumber}}{}{\\ \studentnumber}
\ifthenelse{\isundefined{\email}}{}{\\ \email}
\ifthenelse{\isundefined{\dateintitle}}{}{\\ \insertdate}
%\ifthenelse{\isundefined{\coursename}}{}{\\ \coursename} % put in title instead.
}
{4ex}
{\color{DarkOliveGreen}{\titlerule}\color{Maroon}
\vspace{2ex}%
\filright}
[\vspace{2ex}%
\color{DarkOliveGreen}\titlerule
]

\newcommand{\beginArtWithToc}[0]{\begin{document}\tableofcontents}
\newcommand{\beginArtNoToc}[0]{\begin{document}}
\newcommand{\EndNoBibArticle}[0]{\end{document}}
\newcommand{\EndArticle}[0]{\bibliography{Bibliography}\bibliographystyle{plainnat}\end{document}}

% 
%\newcommand{\citep}[1]{\cite{#1}}

\colorSectionsForArticle



\beginArtNoToc

\generatetitle{Complex form of Poynting relationship}
\label{chap:\basename}
%\section{Motivation}

This is a problem from \cite{fowles1989introduction}, something that I'd tried back when reading \cite{jackson1975cew} but in a way that involved Geometric Algebra and the covariant representation of the energy momentum tensor.  Let's try this with plain old complex vector algebra instead.

%\section{Guts}

\makeproblem{Average Poynting flux for complex 2D fields (problem 2.4)}{ch2:pr4}{ 

Given a complex field phasor representation of the form

\begin{dmath}\label{eqn:poyntingComplex:10}
\tilde{\BE} = \BE_0 e^{i (\Bk \cdot \Bx - \omega t)}
\end{dmath}
\begin{dmath}\label{eqn:poyntingComplex:30}
\tilde{\BH} = \BH_0 e^{i (\Bk \cdot \Bx - \omega t)}.
\end{dmath}

Here we allow the components of $\BE_0$ and $\BH_0$ to be complex.  As usual our fields are defined as the real parts of the phasors

\begin{dmath}\label{eqn:poyntingComplex:50}
\BE = \Real( \tilde{\BE} )
\end{dmath}
\begin{dmath}\label{eqn:poyntingComplex:70}
\BH = \Real( \tilde{\BH} ).
\end{dmath}

Show that the average Poynting vector has the value

\begin{dmath}\label{eqn:poyntingComplex:90}
\expectation{ \BS } = \expectation{ \BE \cross \BH } = \inv{2} \Real( \BE_0 \cross \BH_0^\conj ).
\end{dmath}
}

\makeanswer{ch2:pr4}{ 
While the text works with two dimensional quantities in the $x,y$ plane, I found this problem easier when tackled in three dimensions.  Suppose we write the complex phasor components as

\begin{dmath}\label{eqn:poyntingComplex:110}
\BE_0 = \sum_k (\BE_{kr} + i \BE_{ki}) \Be_k = \sum_k \Abs{\BE_k} e^{i \phi_k} \Be_k
\end{dmath}
\begin{dmath}\label{eqn:poyntingComplex:130}
\BH_0 = \sum_k (\BH_{kr} + i \BH_{ki}) \Be_k = \sum_k \Abs{\BH_k} e^{i \psi_k} \Be_k,
\end{dmath}

and also write $\phi_k' = \phi_k + \Bk \cdot \Bx$, and $\psi_k' = \psi_k + \Bk \cdot \Bx$, then our (real) fields are

\begin{dmath}\label{eqn:poyntingComplex:150}
\BE = \sum_k \Abs{\BE_k} \cos(\phi_k' - \omega t) \Be_k
\end{dmath}
\begin{dmath}\label{eqn:poyntingComplex:170}
\BH = \sum_k \Abs{\BH_k} \cos(\psi_k' - \omega t) \Be_k,
\end{dmath}

and our Poynting vector before averaging (in these units) is

\begin{dmath}\label{eqn:poyntingComplex:190}
\BE \cross \BH = \sum_{klm} \Abs{\BE_k} 
\Abs{\BH_l} 
\cos(\phi_k' - \omega t) 
\cos(\psi_l' - \omega t) \epsilon_{klm} \Be_m.
\end{dmath}

We are tasked with computing the average of cosines

\begin{dmath}\label{eqn:poyntingComplex:210}
\expectation{ \cos(a - \omega t) \cos(b - \omega t) }
=
\inv{T} \int_0^T 
\cos(a - \omega t) \cos(b - \omega t) dt
=
\inv{\omega T} \int_0^T 
\cos(a - \omega t) \cos(b - \omega t) \omega dt
=
\inv{2 \pi} \int_0^{2 \pi}
\cos(a - u) \cos(b - u) du
=
\inv{4 \pi} \int_0^{2 \pi}
\cos(a + b - 2 u) + \cos(a - b) du
=
\inv{2} \cos(a - b).
\end{dmath}

So, our average Poynting vector is
\begin{dmath}\label{eqn:poyntingComplex:230}
\expectation{\BE \cross \BH} = \inv{2} \sum_{klm} \Abs{\BE_k} 
\Abs{\BH_l} 
\cos(\phi_k - \psi_l) 
\epsilon_{klm} \Be_m.
\end{dmath}

We have only to compare this to the desired expression

\begin{dmath}\label{eqn:poyntingComplex:250}
\inv{2} \Real( \BE_0 \cross \BH_0^\conj )
= \inv{2} 
\sum_{klm} \Real\left(
\Abs{\BE_k} e^{i\phi_k}
\Abs{\BH_l} e^{-i\psi_l}
\right)
\epsilon_{klm} \Be_m 
= \inv{2} 
\sum_{klm} 
\Abs{\BE_k} 
\Abs{\BH_l} 
\cos( \phi_k - \psi_l )
\epsilon_{klm} \Be_m.
\end{dmath}

This proves the desired result.
}

\vcsinfo
\EndArticle
