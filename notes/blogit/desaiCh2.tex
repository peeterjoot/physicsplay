%
% Copyright � 2015 Peeter Joot.  All Rights Reserved.
% Licenced as described in the file LICENSE under the root directory of this GIT repository.
%
\documentclass[]{eliblog}

\usepackage{amsmath}
\usepackage{mathpazo}

%
% shorthand for bold symbols, convenient for vectors and matrices
%
\newcommand{\Ba}[0]{\mathbf{a}}
\newcommand{\Bb}[0]{\mathbf{b}}
\newcommand{\Bc}[0]{\mathbf{c}}
\newcommand{\Bd}[0]{\mathbf{d}}
\newcommand{\Be}[0]{\mathbf{e}}
\newcommand{\Bf}[0]{\mathbf{f}}
\newcommand{\Bg}[0]{\mathbf{g}}
\newcommand{\Bh}[0]{\mathbf{h}}
\newcommand{\Bi}[0]{\mathbf{i}}
\newcommand{\Bj}[0]{\mathbf{j}}
\newcommand{\Bk}[0]{\mathbf{k}}
\newcommand{\Bl}[0]{\mathbf{l}}
\newcommand{\Bm}[0]{\mathbf{m}}
\newcommand{\Bn}[0]{\mathbf{n}}
\newcommand{\Bo}[0]{\mathbf{o}}
\newcommand{\Bp}[0]{\mathbf{p}}
\newcommand{\Bq}[0]{\mathbf{q}}
\newcommand{\Br}[0]{\mathbf{r}}
\newcommand{\Bs}[0]{\mathbf{s}}
\newcommand{\Bt}[0]{\mathbf{t}}
\newcommand{\Bu}[0]{\mathbf{u}}
\newcommand{\Bv}[0]{\mathbf{v}}
\newcommand{\Bw}[0]{\mathbf{w}}
\newcommand{\Bx}[0]{\mathbf{x}}
\newcommand{\By}[0]{\mathbf{y}}
\newcommand{\Bz}[0]{\mathbf{z}}
\newcommand{\BA}[0]{\mathbf{A}}
\newcommand{\BB}[0]{\mathbf{B}}
\newcommand{\BC}[0]{\mathbf{C}}
\newcommand{\BD}[0]{\mathbf{D}}
\newcommand{\BE}[0]{\mathbf{E}}
\newcommand{\BF}[0]{\mathbf{F}}
\newcommand{\BG}[0]{\mathbf{G}}
\newcommand{\BH}[0]{\mathbf{H}}
\newcommand{\BI}[0]{\mathbf{I}}
\newcommand{\BJ}[0]{\mathbf{J}}
\newcommand{\BK}[0]{\mathbf{K}}
\newcommand{\BL}[0]{\mathbf{L}}
\newcommand{\BM}[0]{\mathbf{M}}
\newcommand{\BN}[0]{\mathbf{N}}
\newcommand{\BO}[0]{\mathbf{O}}
\newcommand{\BP}[0]{\mathbf{P}}
\newcommand{\BQ}[0]{\mathbf{Q}}
\newcommand{\BR}[0]{\mathbf{R}}
\newcommand{\BS}[0]{\mathbf{S}}
\newcommand{\BT}[0]{\mathbf{T}}
\newcommand{\BU}[0]{\mathbf{U}}
\newcommand{\BV}[0]{\mathbf{V}}
\newcommand{\BW}[0]{\mathbf{W}}
\newcommand{\BX}[0]{\mathbf{X}}
\newcommand{\BY}[0]{\mathbf{Y}}
\newcommand{\BZ}[0]{\mathbf{Z}}

\newcommand{\Bzero}[0]{\mathbf{0}}
\newcommand{\Btheta}[0]{\boldsymbol{\theta}}
\newcommand{\Btau}[0]{\boldsymbol{\tau}}
\newcommand{\Bomega}[0]{\boldsymbol{\omega}}

%
% shorthand for unit vectors
%
\newcommand{\acap}[0]{\hat{\Ba}}
\newcommand{\bcap}[0]{\hat{\Bb}}
\newcommand{\ccap}[0]{\hat{\Bc}}
\newcommand{\dcap}[0]{\hat{\Bd}}
\newcommand{\ecap}[0]{\hat{\Be}}
\newcommand{\fcap}[0]{\hat{\Bf}}
\newcommand{\gcap}[0]{\hat{\Bg}}
\newcommand{\hcap}[0]{\hat{\Bh}}
\newcommand{\icap}[0]{\hat{\Bi}}
\newcommand{\jcap}[0]{\hat{\Bj}}
\newcommand{\kcap}[0]{\hat{\Bk}}
\newcommand{\lcap}[0]{\hat{\Bl}}
\newcommand{\mcap}[0]{\hat{\Bm}}
\newcommand{\ncap}[0]{\hat{\Bn}}
\newcommand{\ocap}[0]{\hat{\Bo}}
\newcommand{\pcap}[0]{\hat{\Bp}}
\newcommand{\qcap}[0]{\hat{\Bq}}
\newcommand{\rcap}[0]{\hat{\Br}}
\newcommand{\scap}[0]{\hat{\Bs}}
\newcommand{\tcap}[0]{\hat{\Bt}}
\newcommand{\ucap}[0]{\hat{\Bu}}
\newcommand{\vcap}[0]{\hat{\Bv}}
\newcommand{\wcap}[0]{\hat{\Bw}}
\newcommand{\xcap}[0]{\hat{\Bx}}
\newcommand{\ycap}[0]{\hat{\By}}
\newcommand{\zcap}[0]{\hat{\Bz}}
\newcommand{\thetacap}[0]{\hat{\Btheta}}

%
% to write R^n and C^n in a distinguishable fashion.  Perhaps change this
% to the double lined characters upon figuring out how to do so.
%
\newcommand{\C}[1]{$\mathbb{C}^{#1}$}
\newcommand{\R}[1]{$\mathbb{R}^{#1}$}

%
% various generally useful helpers
%

% derivative of #1 wrt. #2:
\newcommand{\D}[2] {\frac {d#2} {d#1}}

\newcommand{\inv}[1]{\frac{1}{#1}}
\newcommand{\cross}[0]{\times}

\newcommand{\abs}[1]{\lvert{#1}\rvert}
\newcommand{\norm}[1]{\lVert{#1}\rVert}
\newcommand{\innerprod}[2]{\langle{#1}, {#2}\rangle}
\newcommand{\dotprod}[2]{{#1} \cdot {#2}}
\newcommand{\bdotprod}[2]{\left({#1} \cdot {#2}\right)}
\newcommand{\crossprod}[2]{{#1} \cross {#2}}
\newcommand{\tripleprod}[3]{\dotprod{\left(\crossprod{#1}{#2}\right)}{#3}}

\DeclareMathOperator{\Proj}{Proj}
\DeclareMathOperator{\Span}{span}
\DeclareMathOperator{\Sgn}{sgn}
\DeclareMathOperator{\Area}{Area}
\DeclareMathOperator{\Volume}{Volume}

%
% A few miscellaneous things specific to this document
%
\newcommand{\crossop}[1]{\crossprod{#1}{}}

% R2 vector.
\newcommand{\VectorTwo}[2]{
\begin{bmatrix}
 {#1} \\
 {#2}
\end{bmatrix}
}

\newcommand{\VectorN}[1]{
\begin{bmatrix}
{#1}_1 \\
{#1}_2 \\
\vdots \\
{#1}_N \\
\end{bmatrix}
}

\newcommand{\DETuvij}[4]{
\begin{vmatrix}
 {#1}_{#3} & {#1}_{#4} \\
 {#2}_{#3} & {#2}_{#4}
\end{vmatrix}
}

\newcommand{\DETuvwijk}[6]{
\begin{vmatrix}
 {#1}_{#4} & {#1}_{#5} & {#1}_{#6} \\
 {#2}_{#4} & {#2}_{#5} & {#2}_{#6} \\
 {#3}_{#4} & {#3}_{#5} & {#3}_{#6}
\end{vmatrix}
}

\newcommand{\DETuvwxijkl}[8]{
\begin{vmatrix}
 {#1}_{#5} & {#1}_{#6} & {#1}_{#7} & {#1}_{#8} \\
 {#2}_{#5} & {#2}_{#6} & {#2}_{#7} & {#2}_{#8} \\
 {#3}_{#5} & {#3}_{#6} & {#3}_{#7} & {#3}_{#8} \\
 {#4}_{#5} & {#4}_{#6} & {#4}_{#7} & {#4}_{#8} \\
\end{vmatrix}
}

%\newcommand{\DETuvwxyijklm}[10]{
%\begin{vmatrix}
% {#1}_{#6} & {#1}_{#7} & {#1}_{#8} & {#1}_{#9} & {#1}_{#10} \\
% {#2}_{#6} & {#2}_{#7} & {#2}_{#8} & {#2}_{#9} & {#2}_{#10} \\
% {#3}_{#6} & {#3}_{#7} & {#3}_{#8} & {#3}_{#9} & {#3}_{#10} \\
% {#4}_{#6} & {#4}_{#7} & {#4}_{#8} & {#4}_{#9} & {#4}_{#10} \\
% {#5}_{#6} & {#5}_{#7} & {#5}_{#8} & {#5}_{#9} & {#5}_{#10}
%\end{vmatrix}
%}

% R3 vector.
\newcommand{\VectorThree}[3]{
\begin{bmatrix}
 {#1} \\
 {#2} \\
 {#3}
\end{bmatrix}
}



\author{Peeter Joot}
\email{peeter.joot@gmail.com}

%\documentclass[]{eliblogwidescreen}

\usepackage{amsmath}
\usepackage{mathpazo}

%
% shorthand for bold symbols, convenient for vectors and matrices
%
\newcommand{\Ba}[0]{\mathbf{a}}
\newcommand{\Bb}[0]{\mathbf{b}}
\newcommand{\Bc}[0]{\mathbf{c}}
\newcommand{\Bd}[0]{\mathbf{d}}
\newcommand{\Be}[0]{\mathbf{e}}
\newcommand{\Bf}[0]{\mathbf{f}}
\newcommand{\Bg}[0]{\mathbf{g}}
\newcommand{\Bh}[0]{\mathbf{h}}
\newcommand{\Bi}[0]{\mathbf{i}}
\newcommand{\Bj}[0]{\mathbf{j}}
\newcommand{\Bk}[0]{\mathbf{k}}
\newcommand{\Bl}[0]{\mathbf{l}}
\newcommand{\Bm}[0]{\mathbf{m}}
\newcommand{\Bn}[0]{\mathbf{n}}
\newcommand{\Bo}[0]{\mathbf{o}}
\newcommand{\Bp}[0]{\mathbf{p}}
\newcommand{\Bq}[0]{\mathbf{q}}
\newcommand{\Br}[0]{\mathbf{r}}
\newcommand{\Bs}[0]{\mathbf{s}}
\newcommand{\Bt}[0]{\mathbf{t}}
\newcommand{\Bu}[0]{\mathbf{u}}
\newcommand{\Bv}[0]{\mathbf{v}}
\newcommand{\Bw}[0]{\mathbf{w}}
\newcommand{\Bx}[0]{\mathbf{x}}
\newcommand{\By}[0]{\mathbf{y}}
\newcommand{\Bz}[0]{\mathbf{z}}
\newcommand{\BA}[0]{\mathbf{A}}
\newcommand{\BB}[0]{\mathbf{B}}
\newcommand{\BC}[0]{\mathbf{C}}
\newcommand{\BD}[0]{\mathbf{D}}
\newcommand{\BE}[0]{\mathbf{E}}
\newcommand{\BF}[0]{\mathbf{F}}
\newcommand{\BG}[0]{\mathbf{G}}
\newcommand{\BH}[0]{\mathbf{H}}
\newcommand{\BI}[0]{\mathbf{I}}
\newcommand{\BJ}[0]{\mathbf{J}}
\newcommand{\BK}[0]{\mathbf{K}}
\newcommand{\BL}[0]{\mathbf{L}}
\newcommand{\BM}[0]{\mathbf{M}}
\newcommand{\BN}[0]{\mathbf{N}}
\newcommand{\BO}[0]{\mathbf{O}}
\newcommand{\BP}[0]{\mathbf{P}}
\newcommand{\BQ}[0]{\mathbf{Q}}
\newcommand{\BR}[0]{\mathbf{R}}
\newcommand{\BS}[0]{\mathbf{S}}
\newcommand{\BT}[0]{\mathbf{T}}
\newcommand{\BU}[0]{\mathbf{U}}
\newcommand{\BV}[0]{\mathbf{V}}
\newcommand{\BW}[0]{\mathbf{W}}
\newcommand{\BX}[0]{\mathbf{X}}
\newcommand{\BY}[0]{\mathbf{Y}}
\newcommand{\BZ}[0]{\mathbf{Z}}

\newcommand{\Bzero}[0]{\mathbf{0}}
\newcommand{\Btheta}[0]{\boldsymbol{\theta}}
\newcommand{\Btau}[0]{\boldsymbol{\tau}}
\newcommand{\Bomega}[0]{\boldsymbol{\omega}}

%
% shorthand for unit vectors
%
\newcommand{\acap}[0]{\hat{\Ba}}
\newcommand{\bcap}[0]{\hat{\Bb}}
\newcommand{\ccap}[0]{\hat{\Bc}}
\newcommand{\dcap}[0]{\hat{\Bd}}
\newcommand{\ecap}[0]{\hat{\Be}}
\newcommand{\fcap}[0]{\hat{\Bf}}
\newcommand{\gcap}[0]{\hat{\Bg}}
\newcommand{\hcap}[0]{\hat{\Bh}}
\newcommand{\icap}[0]{\hat{\Bi}}
\newcommand{\jcap}[0]{\hat{\Bj}}
\newcommand{\kcap}[0]{\hat{\Bk}}
\newcommand{\lcap}[0]{\hat{\Bl}}
\newcommand{\mcap}[0]{\hat{\Bm}}
\newcommand{\ncap}[0]{\hat{\Bn}}
\newcommand{\ocap}[0]{\hat{\Bo}}
\newcommand{\pcap}[0]{\hat{\Bp}}
\newcommand{\qcap}[0]{\hat{\Bq}}
\newcommand{\rcap}[0]{\hat{\Br}}
\newcommand{\scap}[0]{\hat{\Bs}}
\newcommand{\tcap}[0]{\hat{\Bt}}
\newcommand{\ucap}[0]{\hat{\Bu}}
\newcommand{\vcap}[0]{\hat{\Bv}}
\newcommand{\wcap}[0]{\hat{\Bw}}
\newcommand{\xcap}[0]{\hat{\Bx}}
\newcommand{\ycap}[0]{\hat{\By}}
\newcommand{\zcap}[0]{\hat{\Bz}}
\newcommand{\thetacap}[0]{\hat{\Btheta}}

%
% to write R^n and C^n in a distinguishable fashion.  Perhaps change this
% to the double lined characters upon figuring out how to do so.
%
\newcommand{\C}[1]{$\mathbb{C}^{#1}$}
\newcommand{\R}[1]{$\mathbb{R}^{#1}$}

%
% various generally useful helpers
%

% derivative of #1 wrt. #2:
\newcommand{\D}[2] {\frac {d#2} {d#1}}

\newcommand{\inv}[1]{\frac{1}{#1}}
\newcommand{\cross}[0]{\times}

\newcommand{\abs}[1]{\lvert{#1}\rvert}
\newcommand{\norm}[1]{\lVert{#1}\rVert}
\newcommand{\innerprod}[2]{\langle{#1}, {#2}\rangle}
\newcommand{\dotprod}[2]{{#1} \cdot {#2}}
\newcommand{\bdotprod}[2]{\left({#1} \cdot {#2}\right)}
\newcommand{\crossprod}[2]{{#1} \cross {#2}}
\newcommand{\tripleprod}[3]{\dotprod{\left(\crossprod{#1}{#2}\right)}{#3}}

\DeclareMathOperator{\Proj}{Proj}
\DeclareMathOperator{\Span}{span}
\DeclareMathOperator{\Sgn}{sgn}
\DeclareMathOperator{\Area}{Area}
\DeclareMathOperator{\Volume}{Volume}

%
% A few miscellaneous things specific to this document
%
\newcommand{\crossop}[1]{\crossprod{#1}{}}

% R2 vector.
\newcommand{\VectorTwo}[2]{
\begin{bmatrix}
 {#1} \\
 {#2}
\end{bmatrix}
}

\newcommand{\VectorN}[1]{
\begin{bmatrix}
{#1}_1 \\
{#1}_2 \\
\vdots \\
{#1}_N \\
\end{bmatrix}
}

\newcommand{\DETuvij}[4]{
\begin{vmatrix}
 {#1}_{#3} & {#1}_{#4} \\
 {#2}_{#3} & {#2}_{#4}
\end{vmatrix}
}

\newcommand{\DETuvwijk}[6]{
\begin{vmatrix}
 {#1}_{#4} & {#1}_{#5} & {#1}_{#6} \\
 {#2}_{#4} & {#2}_{#5} & {#2}_{#6} \\
 {#3}_{#4} & {#3}_{#5} & {#3}_{#6}
\end{vmatrix}
}

\newcommand{\DETuvwxijkl}[8]{
\begin{vmatrix}
 {#1}_{#5} & {#1}_{#6} & {#1}_{#7} & {#1}_{#8} \\
 {#2}_{#5} & {#2}_{#6} & {#2}_{#7} & {#2}_{#8} \\
 {#3}_{#5} & {#3}_{#6} & {#3}_{#7} & {#3}_{#8} \\
 {#4}_{#5} & {#4}_{#6} & {#4}_{#7} & {#4}_{#8} \\
\end{vmatrix}
}

%\newcommand{\DETuvwxyijklm}[10]{
%\begin{vmatrix}
% {#1}_{#6} & {#1}_{#7} & {#1}_{#8} & {#1}_{#9} & {#1}_{#10} \\
% {#2}_{#6} & {#2}_{#7} & {#2}_{#8} & {#2}_{#9} & {#2}_{#10} \\
% {#3}_{#6} & {#3}_{#7} & {#3}_{#8} & {#3}_{#9} & {#3}_{#10} \\
% {#4}_{#6} & {#4}_{#7} & {#4}_{#8} & {#4}_{#9} & {#4}_{#10} \\
% {#5}_{#6} & {#5}_{#7} & {#5}_{#8} & {#5}_{#9} & {#5}_{#10}
%\end{vmatrix}
%}

% R3 vector.
\newcommand{\VectorThree}[3]{
\begin{bmatrix}
 {#1} \\
 {#2} \\
 {#3}
\end{bmatrix}
}



\author{Peeter Joot}
\email{peeter.joot@gmail.com}


\chapter{Desai Chapter II notes and problems.}
\label{chap:desaiCh2}
%\useCCL
\blogpage{http://sites.google.com/site/peeterjoot/math2010/desaiCh2.pdf}
\date{Sept 19, 2010}
\revisionInfo{desaiCh2.tex}

\beginArtWithToc
%\beginArtNoToc

\section{Motivation.}

Chapter II notes for \cite{desai2009quantum}.

\section{Notes}
\subsection{Canonical Commutator}

Based on the canonical relationship $[X,P] = i\hbar$, and $\braket{x'}{x} = \delta(x'-x)$, Desai determines the form of the $P$ operator in continuous space.  A consequence of this is that the matrix element of the momentum operator is found to have a delta function specification

\begin{align*}
\bra{x'} P \ket{x} = \delta(x - x') \left( -i \hbar \frac{d}{dx} \right).
\end{align*}

In particular the matrix element associated with the state $\ket{\phi}$ is found to be

\begin{align*}
\bra{x'} P \ket{\phi} = -i \hbar \frac{d}{dx'} \phi(x').
\end{align*}

Compare this to \cite{liboff2003iqm}, where this last is taken as the definition of the momentum operator, and the relationship to the delta function is not spelled out explicitly.  This canonical commutator approach, while more abstract, seems to have less black magic involved in the setup.  We do require the commutator relationship $[X,P] = i\hbar$ to be pulled out of a magic hat, but at least the magic show is a structured one based on a small set of core assumptions.

It will likely be good to come back to this later when trying to reconcile this new (for me) Dirac notation with the more basic notation I'm already comfortable with.  When trying to compare the two, it will be good to note that there is a matrix element that is implied in the more old fashioned treatment in a book such as \cite{bohm1989qt}.

There is one fundamental assumption that appears to be made in this section that isn't justified by anything except the end result.  That is the assumption that $P$ is a derivative like operator, acting with a product rule action.  That's used to obtain (2.28) and is a fairly black magic operation.  This same assumption, is also hiding, somewhat sneakily, in the manipulation for (2.44).

If one has to make that assumption that $P$ is a derivative like operator, I don't feel this method of introducing it is any less arbitrary seeming.  It is still pulled out of a magic hat, only because the answer is known ahead of time.  The approach of \cite{bohm1989qt}, where the derivative nature is presented as consequence of transforming (via Fourier transforms) from the position to the momentum representation, seems much more intuitive and less arbitrary.

\subsection{Generalized momentum commutator.}

It is stated that

\begin{align*}
[P,X^n] = - n i \hbar X^{n-1}.
\end{align*}

Let's prove this.  The $n=1$ case is the canonical commutator, which is assumed.  Is there any good way to justify that from first principles, as presented in the text?  We have to prove this for $n$, given the relationship for $n-1$.  Expanding the $n$th power commutator we have

\begin{align*}
[P,X^n] 
&= P X^n - X^n P \\
&= P X^{n-1} X - X^{n } P \\
\end{align*}

Rearranging the $n-1$ result we have

\begin{align*}
P X^{n-1} = X^{n-1} P - (n-1) i \hbar X^{n-2},
\end{align*}

and can insert that in our $[P,X^n]$ expansion for

\begin{align*}
[P,X^n] 
&= \left( X^{n-1} P - (n-1) i \hbar X^{n-2} \right)X - X^{n } P \\
&= X^{n-1} (PX) - (n-1) i \hbar X^{n-1} - X^{n } P \\
&= X^{n-1} ( X P - i\hbar) - (n-1) i \hbar X^{n-1} - X^{n } P \\
&= -X^{n-1} i\hbar - (n-1) i \hbar X^{n-1} \\
&= -n i \hbar X^{n-1} 
\qquad\square
\end{align*}

\subsection{Uncertainty principle.}

The origin of the statement $[\Delta A, \Delta B] = [A, B]$ is not something that seemed obvious.  Expanding this out however is straightforward, and clarifies things.  That is

\begin{align*}
[\Delta A, \Delta B] 
&= (A - \expectation{A}) (B - \expectation{B}) - (B - \expectation{B}) (A - \expectation{A}) \\
&= 
\left( A B - \expectation{A} B - \expectation{B} A +\expectation{A} \expectation{B} \right)
-\left( B A - \expectation{B} A - \expectation{A} B +\expectation{B} \expectation{A} \right) \\
&= 
A B - B A \\
&= 
[A, B]
\qquad\square
\end{align*}

\subsection{Size of a particle}

I found it curious that using $\Delta x \Delta p \approx \hbar$ instead of $\Delta x \Delta p \ge \hbar/2$, was sufficient to obtain the hydrogen ground state energy $E_{\text{min}} = -e^2/2 a_0$, without also having to do any factor of two fudging.

\subsection{Space displacement operator.}

I'd be curious to know if others find the loose use of equality for approximation after approximation slightly disturbing too?

I also find it curious that (2.140) is written

\begin{align*}
D(x) = \exp\left( -i \frac{P}{\hbar} x \right),
\end{align*}

and not
\begin{align*}
D(x) = \exp\left( -i x \frac{P}{\hbar} \right).
\end{align*}

Is this intentional?  It doesn't seem like $P$ ought to be acting on $x$ in this case, so why order the terms that way?

Expanding the application of this operator, or at least its first order Taylor series, is helpful to get an idea about this.  Doing so, with the original $\Delta x'$ value used in the derivation of the text we have to start

\begin{align*}
D(\Delta x') \ket{\phi} 
&\approx \left(1 - i \frac{P}{\hbar} \Delta x' \right) \ket{\phi} \\
&= \left(1 - i \left( -i \hbar \delta(x -x') \frac{\partial}{\partial x} \right) \inv{\hbar} \Delta x'\right) \ket{\phi} \\
\end{align*}

This shows that the $\Delta x$ factor can be commuted with the momentum operator, as it is not a function of $x'$, so the question of $P x$, vs $x P$ above appears to be a non-issue.

Regardless of that conclusion, it seems worthy to continue an attempt at expanding this shift operator action on the state vector.  Let's do so, but do so by computing the matrix element $\bra{x'} D(\Delta x') \ket{\phi}$.  That is

\begin{align*}
\bra{x'} D(\Delta x') \ket{\phi} 
&\approx
\braket{x'}{\phi} - \bra{x'} \delta(x -x') \frac{\partial}{\partial x} \Delta x' \ket{\phi} \\
&=
\phi(x') - \int \bra{x'} \delta(x -x') \frac{\partial}{\partial x} \Delta x' \ket{x'} \braket{x'}{\phi} dx' \\
&=
\phi(x') - \Delta x' \int \delta(x -x') \frac{\partial}{\partial x} \braket{x'}{\phi} dx' \\
&=
\phi(x') - \Delta x' \frac{\partial}{\partial x'} \braket{x'}{\phi} \\
&=
\phi(x') - \Delta x' \frac{\partial}{\partial x'} \phi(x') \\
\end{align*}

This is consistent with the text.  It is interesting, and initially surprising that the space displacement operator when applied to a state vector introduces a negative shift in the wave function associated with that state vector.  In the derivation of the text, this was associated with the use of integration by parts (ie: due to the sign change in that integration).  Here we see it sneak back in, due to the $i^2$ once the momentum operator is expanded completely.

As last note and question.  The first order Taylor approximation of the momentum operator was used.  If the higher order terms are retained, as in

\begin{align*}
\exp\left( -i \Delta x' \frac{P}{\hbar} \right) = 
1 - \Delta x' \delta(x -x') \frac{\partial}{\partial x} + 
\inv{2} \left( - \Delta x' \delta(x -x') \frac{\partial}{\partial x} \right)^2 + \cdots,
\end{align*}

then how does one evaluate a squared delta function (or Nth power)?

Talked to Vatche about this after class.  The key to this is sequential evaluation.  Considering the simple case for $P^2$, we evaluate one operator at a time, and never actually square the delta function

\begin{align*}
\bra{x'} P^2 \ket{\phi} 
%&= \bra{x'} P (P \ket{\phi}) \\
%&= -i \hbar \int dx' \bra{x'} P (\delta(x-x') \PD{x}{} \ket{x'} \braket{x'}{\phi}) \\
%&= -i \hbar \bra{x'} P \PD{x'}{} \ket{x'} \braket{x'}{\phi}) \\
\end{align*}

I was also questioned why I was including the delta function at this point.  Why would I do that.  Thinking further on this, I see that isn't a reasonable thing to do.  That delta function only comes into the mix when one takes the matrix element of the momentum operator as in

\begin{align*}
\bra{x'} P \ket{x} = -i \hbar \delta(x-x') \frac{d}{dx'}. 
\end{align*}

This is very much like the fact that the delta function only shows up in the continuous representation in other context where one has matrix elements.  The most simple example of which is just

\begin{align*}
\braket{x'}{x} = \delta(x-x').
\end{align*}

I also see now that the momentum operator is directly identified with the derivative (no delta function) in two other places in the text.  These are equations (2.32) and (2.46) respectively:

\begin{align*}
P(x) &= -i \hbar \frac{d}{dx} \\
P &= -i \hbar \frac{d}{dX}.
\end{align*}

In the first, (2.32), I thought the $P(x)$ was somehow different, just a helpful expression found along the way, but now it occurs to me that this was intended to be an unambiguous representation of the momentum operator itself.

FIXME: rework the expansion of $D(\Delta x') \phi(x)$ above in light of this.  Examine the connection between that and 
$D(\Delta x') \ket{\phi}$ to get a feel for all the notational magic.

\subsection{Time evolution operator}

The phrase ``we identify time evolution with the Hamiltonian''.  What a magic hat maneuver!  Is there a way that this would be logical without already knowing the answer?

\subsection{Dispersion delta function representation.}

The Principle part notation here I found a bit unclear.  He writes

\begin{align*}
\lim_{\epsilon \rightarrow 0} 
\frac{(x'-x)}{(x'-x)^2 + \epsilon^2}
= 
P\left( \inv{x' - x} \right).
\end{align*}

In complex variables the principle part is the negative power series terms.  For example for $f(z) = \sum a_k z^k$, the principle part is

\begin{align*}
\sum_{k = -\infty}^{-1} a_k z^k
\end{align*}

This doesn't vanish at $z = 0$ as the principle part in this section is stated to.  In (2.202) he pulls the $P$ out of the integral, but I think the intention is really to keep this associated with the $1/(x'-x)$, as in

\begin{align*}
\lim_{\epsilon \rightarrow 0} 
\inv{\pi} \int_0^\infty dx' \frac{f(x')}{x'-x - i \epsilon}
= 
\inv{\pi} \int_0^\infty dx' f(x') P\left( \inv{x' - x} \right) + i f(x)
\end{align*}

Will this even have any relevance in this text?

\section{Problems.}
\subsection{1. Cauchy-Shwartz identity.}
\subsubsection{1. Blundering on the standard trick.}

This proof is a standard one from a linear algebra book, one for which has a tricky trick that I never remember.  For $\ket{c} = \ket{a} + \lambda \ket{b}$, we wish to take the inner product and find the magic value of $\lambda$ that leaves only real terms in the resulting expression.

That is

\begin{align*}
\Abs{\ket{a} + \lambda \ket{b}}^2 
&=
\braket{a + \lambda b}{a + \lambda b} \\
&= \Abs{\ket{a}}^2 + \Abs{\lambda}^2 \Abs{\ket{b}}^2 + \lambda \braket{a}{b} + \lambda^\conj \braket{b}{a} \\
\end{align*}

Let's try $\lambda = \braket{b}{a}/\Abs{\ket{b}}^n$.  This gives

\begin{align*}
\Abs{\ket{a} + \lambda \ket{b}}^2 
&= \Abs{\ket{a}}^2 + \Abs{\braket{b}{a}}^2 \Abs{\ket{b}}^{2-2n} + \Abs{\braket{a}{b}}^2/\Abs{\ket{b}}^n + \Abs{\braket{b}{a}}^2/\Abs{\ket{b}}^n \\
\end{align*}

We want terms to drop out so we pick $2 -2n = -n$, or $n = 2$.  This gives

\begin{align*}
\Abs{\ket{a} + \lambda \ket{b}}^2 
&= \Abs{\ket{a}}^2 + 3 \Abs{\braket{b}{a}}^2/\Abs{\ket{b}}^2 \\
\end{align*}

It appears that this wasn't the magic value desired.  If we pick $\lambda = -\braket{b}{a}/\Abs{\ket{b}}^n$ instead, we have what we want:

\begin{align*}
\Abs{\ket{a} + \lambda \ket{b}}^2 
&= \Abs{\ket{a}}^2 - \Abs{\braket{b}{a}}^2/\Abs{\ket{b}}^2 \ge 0
\end{align*}

And rearranging, and dropping the $\Abs{}$ notation in favour of inner products, we have the desired result:

\begin{align*}
\braket{a}{a} \braket{b}{b} \ge \braket{b}{a}\braket{a}{b}.
\end{align*}

\subsubsection{1. As a min/max problem.}

The trial and error approach above, where we trying to blunder upon the well known but obscure trick, kind of sucks.  We can also do this as a min/max problem, although this is made slightly messier by the complex vector space.  Define

\begin{align*}
f(\lambda) =
\braket{a}{a} + \lambda \lambda^\conj \braket{b}{b} + \lambda \braket{a}{b} + \lambda^\conj \braket{b}{a} \\
\end{align*}

Now, set $df/d\lambda = 0$

\begin{align*}
\frac{df}{d\lambda} 
&=
\left(\lambda^\conj + \lambda \frac{d\lambda^\conj}{d\lambda}\right) \braket{b}{b} + \braket{a}{b} + \frac{d\lambda^\conj}{d\lambda} \braket{b}{a} \\
&=
\lambda^\conj \braket{b}{b} + \braket{a}{b} 
+
\frac{d\lambda^\conj}{d\lambda} \Bigl( 
\lambda \braket{b}{b} + \braket{b}{a} \Bigr)
\end{align*}

Now, we have a bit of a problem with $d\lambda^\conj/d\lambda$, since that doesn't actually exist.  However, if we insist that what multiplies it is zero, we have $df/d\lambda = 0$ as desired.  This yields 

\begin{align*}
\lambda = - \frac{\braket{b}{a} }{ \braket{b}{b}  }
\end{align*}

as desired, and the remainder of the proof follows as before.  This is nicer, as slightly less black magic is required.

\subsection{2.}
\subsection{3.}
\subsection{4.}
\subsection{5. Hermitian radial differential operator.}

Show that the operator 

\begin{align*}
R = -i \hbar \PD{r}{},
\end{align*}

is not Hermitian, and find the constant $a$ so that 

\begin{align*}
T = -i \hbar \left( \PD{r}{} + \frac{a}{r} \right),
\end{align*}

is Hermitian.

For the first part of the problem we can show that

\begin{align*}
\left( \bra{\psicap} R \ket{\phicap} \right)^\conj \ne \bra{\phicap} R \ket{\psicap}.
\end{align*}

For the RHS we have

\begin{align*}
\bra{\phicap} R \ket{\psicap} 
= -i \hbar \iiint dr d\theta d\phi r^2 \sin\theta \phicap^\conj \PD{r}{\psicap}
\end{align*}

and for the LHS we have

\begin{align*}
\left( \bra{\psicap} R \ket{\phicap} \right)^\conj
&= i \hbar \iiint dr d\theta d\phi r^2 \sin\theta \psicap \PD{r}{\phicap^\conj} \\
&= -i \hbar \iiint dr d\theta d\phi \sin\theta 
\left( 2 r \psicap 
+ r^2 \PD{\psicap}{r} 
\right)
\phicap^\conj 
\\
\end{align*}

So, unless $r\psicap = 0$, the operator $R$ is not Hermitian.

Moving on to finding the constant $a$ such that $T$ is Hermitian we calculate

\begin{align*}
\left( \bra{\psicap} T \ket{\phicap} \right)^\conj
&= i \hbar \iiint dr d\theta d\phi r^2 \sin\theta \psicap \left( \PD{r}{} + \frac{a}{r} \right) \phicap^\conj \\
&= i \hbar \iiint dr d\theta d\phi \sin\theta \psicap \left( r^2 \PD{r}{} + a r \right) \phicap^\conj \\
&= -i \hbar \iiint dr d\theta d\phi \sin\theta \left( r^2 \PD{r}{\psicap} + 2 r \psicap - a r \psicap \right) \phicap^\conj \\
\end{align*}

and

\begin{align*}
\bra{\phicap} T \ket{\psicap} 
= -i \hbar \iiint dr d\theta d\phi r^2 \sin\theta \phicap^\conj \left( r^2 \PD{r}{\psicap} + a r \psicap \right)
\end{align*}

So, for $T$ to be Hermitian, we require

\begin{align*}
2 r - a r = a r.
\end{align*}

So $a = 1$, and our Hermitian operator is
\begin{align*}
T = -i \hbar \left( \PD{r}{} + \frac{1}{r} \right).
\end{align*}

\subsection{6. Radial directional derivative operator.}

\subsubsection{Problem.}
Show that 

\begin{align*}
D = \Bp \cdot \rcap + \rcap \cdot \Bp,
\end{align*}

is Hermitian.  Expand this operator in spherical coordinates.  Compare result to problem 5.

\subsubsection{Solution.}

Tackling the spherical coordinates expression of of the operator $D$, we have

\begin{align*}
\inv{-i\hbar} D \Psi 
&= \left( \spacegrad \cdot \rcap + \rcap \cdot \spacegrad \right) \Psi \\
&= 
\left( \spacegrad \cdot \rcap \right) \Psi 
+ \left( \spacegrad \Psi \right) \cdot \rcap 
+ \rcap \cdot \left(\spacegrad \Psi\right) \\
&=
\left( \spacegrad \cdot \rcap \right) \Psi 
+ 2 \rcap \cdot \left( \spacegrad \Psi \right).
\end{align*}

Here braces have been used to denote the extend of the operation of the gradient.  In spherical polar coordinates, our gradient is

\begin{align*}
\spacegrad \equiv 
\rcap \PD{r}{}
+\thetacap \inv{r} \PD{\theta}{}
+\phicap \inv{r \sin\theta} \PD{\phi}{}.
\end{align*}

This gets us most of the way there, and we have

\begin{align*}
\inv{-i\hbar} D \Psi 
&=
2 \PD{r}{\Psi} 
+ 
\left( 
\rcap \cdot \PD{r}{\rcap}
+\inv{r} \thetacap \cdot \PD{\theta}{\rcap}
+\inv{r \sin\theta} \phicap \cdot \PD{\phi}{\rcap}
\right) \Psi.
\end{align*}

Since $\PDi{r}{\rcap} = 0$, we are left with evaluating $\thetacap \cdot \PDi{\theta}{\rcap}$, and $\phicap \cdot \PDi{\phi}{\rcap}$.  To do so I chose to employ the (Geometric Algebra) exponential form of the spherical unit vectors \cite{sphericalPolarUnit}

\begin{align*}
I &= \Be_1 \Be_2 \Be_3 \\
\phicap &= \Be_{2} \exp( I \Be_3 \phi ) \\
\rcap &= \Be_3 \exp( I \phicap \theta ) \\
\thetacap &= \Be_1 \Be_2 \phicap \exp( I \phicap \theta ).
\end{align*}

The partials of interest are then

\begin{align*}
\PD{\theta}{\rcap} &= \Be_3 I \phicap \exp( I \phicap \theta ) = \thetacap,
\end{align*}

and

\begin{align*}
\PD{\phi}{\rcap} 
&= \PD{\phi}{} \Be_3 \left( \cos\theta + I \phicap \sin\theta \right) \\
&= \Be_1 \Be_2 \sin\theta \PD{\phi}{\phicap} \\
&= \Be_1 \Be_2 \sin\theta \Be_2 \Be_1 \Be_2 \exp( I \Be_3 \phi ) \\
&= \sin\theta \phicap.
\end{align*}

Only after computing these, did I find exactly these results for the partials of interest, in \href{http://mathworld.wolfram.com/SphericalCoordinates.html}{mathworld's Spherical Coordinates page}, which confirms these calculations.  Note that a different angle convention is used there, so one has to exchange $\phi$, and $\theta$ and the corresponding unit vector labels.

Substitution back into our expression for the operator we have
\begin{align*}
D &= - 2 i \hbar \left( \PD{r}{} + \inv{r} \right),
\end{align*}

an operator that is exactly twice the operator of problem 5, already shown to be Hermitian.  Since the constant numerical scaling of a Hermitian operator leaves it Hermitian, this shows that $D$ is Hermitian as expected.

\subsubsection{$\thetacap$ directional momentum operator}

Let's try this for the other unit vector directions too.  We also want

\begin{align*}
\left( \spacegrad \cdot \thetacap + \thetacap \cdot \spacegrad \right) \Psi
&=
2 \thetacap \cdot (\spacegrad \Psi) + \left( \spacegrad \cdot \thetacap \right) \Psi.
\end{align*}

The work consists of evaluating

\begin{align*}
\spacegrad \cdot \thetacap 
&= \rcap \cdot \PD{r}{\thetacap}
+ \inv{r} \thetacap \cdot \PD{\theta}{\thetacap}
+ \inv{r \sin\theta} \phicap \cdot \PD{\phi}{\thetacap}.
\end{align*}

This time we need the $\PDi{\theta}{\thetacap}$, $\PDi{\phi}{\thetacap}$ partials, which are

\begin{align*}
\PD{\theta}{\thetacap} 
&=
\Be_1 \Be_2 \phicap I \phicap \exp( I \phicap \theta) \\
&=
-\Be_3 \exp( I \phicap \theta) \\
&=
- \rcap.
\end{align*}

This has no $\thetacap$ component, so does not contribute to $\spacegrad \cdot \thetacap$.  Noting that

\begin{align*}
\PD{\phi}{\phicap} &= -\Be_1 \exp( I \Be_3 \phi ) = \Be_2 \Be_1 \phicap,
\end{align*}

the $\phi$ partial is

\begin{align*}
\PD{\phi}{\thetacap} &=
\Be_1 \Be_2 \left( 
\PD{\phi}{\phicap} \exp( I \phicap \theta )
+\phicap I \sin\theta \PD{\phi}{\phicap} 
\right) \\
&=
\phicap 
\left( 
\exp( I \phicap \theta )
+I \sin\theta \Be_2 \Be_1 \phicap
\right),
\end{align*}

with $\phicap$ component
\begin{align*}
\phicap \cdot \PD{\phi}{\thetacap} &=
\gpgradezero{
\exp( I \phicap \theta )
+I \sin\theta \Be_2 \Be_1 \phicap } \\
&=
\cos\theta + \Be_3 \cdot \phicap \sin\theta \\
&=
\cos\theta.
\end{align*}

Assembling the results, and labeling this operator $\Theta$ we have

\begin{align*}
\Theta &\equiv \inv{2} \left( \Bp \cdot \thetacap + \thetacap \cdot \Bp \right)  \\
&=
-i \hbar \inv{r} \left( \PD{\theta}{} + \inv{2} \cot\theta \right).
\end{align*}

It would be reasonable to expect this operator to also be Hermitian, and checking this explicitly by comparing
$\bra{\Phi} \Theta \ket{\Psi}^\conj$ and $\bra{\Psi} \Theta \ket{\Phi}$, shows that this is in fact the case.

\subsubsection{$\phicap$ directional momentum operator}

Let's try this for the other unit vector directions too.  We also want

\begin{align*}
\left( \spacegrad \cdot \phicap + \phicap \cdot \spacegrad \right) \Psi
&=
2 \phicap \cdot (\spacegrad \Psi) + \left( \spacegrad \cdot \phicap \right) \Psi.
\end{align*}

The work consists of evaluating

\begin{align*}
\spacegrad \cdot \phicap 
&= \rcap \cdot \PD{r}{\phicap}
+ \inv{r} \thetacap \cdot \PD{\theta}{\phicap}
+ \inv{r \sin\theta} \phicap \cdot \PD{\phi}{\phicap}.
\end{align*}

This time we need the $\PDi{\theta}{\phicap}$, $\PDi{\phi}{\phicap} = \Be_2 \Be_1 \phicap$ partials.  The $\theta$ partial is

\begin{align*}
\PD{\theta}{\phicap} 
&=
\PD{\theta}{} \Be_2 \exp( I \Be_3 \phi ) \\
&= 0.
\end{align*}

We conclude that $\spacegrad \cdot \phicap = 0$, and expect that we have one more Hermitian operator

\begin{align*}
\Phi &\equiv \inv{2} \left( \Bp \cdot \phicap + \phicap \cdot \Bp \right)  \\
&=
-i \hbar \inv{r \sin\theta} \PD{\phi}{}.
\end{align*}

It is simple to confirm that this is Hermitian since the integration by parts does not involve any of the volume element.  In fact, any operator $-i\hbar f(r,\theta) \PDi{\phi}{}$ would also be Hermitian, including the simplest case $-i\hbar \PDi{\phi}{}$.  Have to dig out my Bohm text again, since I seem to recall that one used in the spherical Harmonics chapter.

\subsubsection{A note on the Hermitian test and Dirac notation.}

I've been a bit loose with my notation.  I've stated that my demonstrations of the Hermitian nature have been done by showing

\begin{align*}
\bra{\phi} A \ket{\psi}^\conj - \bra{\psi} A \ket{\phi} = 0.
\end{align*}

However, what I've actually done is show that 

\begin{align*}
\left( \int d^3 \Bx \phi^\conj (\Bx) A(\Bx) \psi(\Bx) \right)^\conj - \int d^3 \Bx \psi^\conj (\Bx) A(\Bx) \phi(\Bx) = 0.
\end{align*}

To justify this note that 

\begin{align*}
\bra{\phi} A \ket{\psi}^\conj 
&=
\left( \iint d^3 \Br d^3 \Bs \braket{\phi}{\Br} \bra{\Br} A \ket{\Bs} \braket{\Bs}{\psi} \right)^\conj \\
&=
\iint d^3 \Br d^3 \Bs \phi(\Br) \delta^3(\Br - \Bs) A^\conj(\Bs) \psi(\Bs) \\
&=
\int d^3 \Br \phi(\Br) A^\conj(\Br) \psi(\Br),
\end{align*}

and
\begin{align*}
\bra{\phi} A \ket{\psi}^\conj 
&=
\iint d^3 \Br d^3 \Bs \braket{\psi}{\Br} \bra{\Br} A \ket{\Bs} \braket{\Bs}{\phi} \\
&=
\iint d^3 \Br d^3 \Bs \bra{\Br} \psi(\Br) \delta^3(\Br - \Bs) A(\Bs) \phi(\Bs) \\
&=
\int d^3 \Br \psi(\Br) A(\Br) \phi(\Br).
\end{align*}

Working backwards one sees that the comparison of the wave function integrals in explicit inner product notation is sufficient to demonstrate the Hermitian property.

\subsection{7. Some commutators.}
\subsubsection{7. Problem.}

For $D$ in problem 6, obtain

\begin{itemize}
\item i) $[D, x_i]$
\item ii) $[D, p_i]$
\item iii) $[D, L_i]$, where $L_i = \Be_i \cdot (\Br \cross \Bp)$.
\item iv) Show that $e^{i\alpha D/\hbar} x_i e^{-i\alpha D/\hbar} = e^\alpha x_i$
\end{itemize}

\subsubsection{7. Solution.}

To start observe that we can write the operator in a form more convient for use with cartesian coordinates:

\begin{align*}
D = -2 i\hbar \inv{r}( \Br \cdot \spacegrad + 1)
\end{align*}

\begin{itemize}
\item i) 

\begin{align*}
[D, x_i] \Psi
&=
D x_i \Psi - x_i D \Psi \\
&=
-2 i \hbar \inv{r} \left( \Br \cdot \spacegrad + 1 \right) x_i \Psi
+2 i \hbar x_i \inv{r} \left( \Br \cdot \spacegrad + 1 \right) \Psi \\
&=
-2 i \hbar \inv{r} \Br \cdot \spacegrad x_i \Psi
+2 i \hbar x_i \inv{r} \Br \cdot \spacegrad \Psi \\
&=
-2 i \hbar \inv{r} \Br \cdot (\spacegrad x_i) \Psi
-2 i \hbar x_i \inv{r} \Br \cdot \spacegrad \Psi
+2 i \hbar x_i \inv{r} \Br \cdot \spacegrad \Psi \\
&=
-2 i \hbar \inv{r} \Br \cdot \Be_i \Psi.
\end{align*}

So this first commutator is:

\begin{align*}
[D, x_i] = -2 i \hbar \frac{x_i}{r}.
\end{align*}

\item ii) $[D, p_i]$
\item iii) $[D, L_i]$

\item iv) Show that $e^{i\alpha D/\hbar} x_i e^{-i\alpha D/\hbar} = e^\alpha x_i$
\end{itemize}

\subsection{8.}
\subsection{9.}
\subsection{10.}
\subsection{11.}

\EndArticle
%\EndNoBibArticle
