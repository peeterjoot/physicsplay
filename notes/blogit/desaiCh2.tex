%
% Copyright � 2015 Peeter Joot.  All Rights Reserved.
% Licenced as described in the file LICENSE under the root directory of this GIT repository.
%
\documentclass[]{eliblog}

\usepackage{amsmath}
\usepackage{mathpazo}

%
% shorthand for bold symbols, convenient for vectors and matrices
%
\newcommand{\Ba}[0]{\mathbf{a}}
\newcommand{\Bb}[0]{\mathbf{b}}
\newcommand{\Bc}[0]{\mathbf{c}}
\newcommand{\Bd}[0]{\mathbf{d}}
\newcommand{\Be}[0]{\mathbf{e}}
\newcommand{\Bf}[0]{\mathbf{f}}
\newcommand{\Bg}[0]{\mathbf{g}}
\newcommand{\Bh}[0]{\mathbf{h}}
\newcommand{\Bi}[0]{\mathbf{i}}
\newcommand{\Bj}[0]{\mathbf{j}}
\newcommand{\Bk}[0]{\mathbf{k}}
\newcommand{\Bl}[0]{\mathbf{l}}
\newcommand{\Bm}[0]{\mathbf{m}}
\newcommand{\Bn}[0]{\mathbf{n}}
\newcommand{\Bo}[0]{\mathbf{o}}
\newcommand{\Bp}[0]{\mathbf{p}}
\newcommand{\Bq}[0]{\mathbf{q}}
\newcommand{\Br}[0]{\mathbf{r}}
\newcommand{\Bs}[0]{\mathbf{s}}
\newcommand{\Bt}[0]{\mathbf{t}}
\newcommand{\Bu}[0]{\mathbf{u}}
\newcommand{\Bv}[0]{\mathbf{v}}
\newcommand{\Bw}[0]{\mathbf{w}}
\newcommand{\Bx}[0]{\mathbf{x}}
\newcommand{\By}[0]{\mathbf{y}}
\newcommand{\Bz}[0]{\mathbf{z}}
\newcommand{\BA}[0]{\mathbf{A}}
\newcommand{\BB}[0]{\mathbf{B}}
\newcommand{\BC}[0]{\mathbf{C}}
\newcommand{\BD}[0]{\mathbf{D}}
\newcommand{\BE}[0]{\mathbf{E}}
\newcommand{\BF}[0]{\mathbf{F}}
\newcommand{\BG}[0]{\mathbf{G}}
\newcommand{\BH}[0]{\mathbf{H}}
\newcommand{\BI}[0]{\mathbf{I}}
\newcommand{\BJ}[0]{\mathbf{J}}
\newcommand{\BK}[0]{\mathbf{K}}
\newcommand{\BL}[0]{\mathbf{L}}
\newcommand{\BM}[0]{\mathbf{M}}
\newcommand{\BN}[0]{\mathbf{N}}
\newcommand{\BO}[0]{\mathbf{O}}
\newcommand{\BP}[0]{\mathbf{P}}
\newcommand{\BQ}[0]{\mathbf{Q}}
\newcommand{\BR}[0]{\mathbf{R}}
\newcommand{\BS}[0]{\mathbf{S}}
\newcommand{\BT}[0]{\mathbf{T}}
\newcommand{\BU}[0]{\mathbf{U}}
\newcommand{\BV}[0]{\mathbf{V}}
\newcommand{\BW}[0]{\mathbf{W}}
\newcommand{\BX}[0]{\mathbf{X}}
\newcommand{\BY}[0]{\mathbf{Y}}
\newcommand{\BZ}[0]{\mathbf{Z}}

\newcommand{\Bzero}[0]{\mathbf{0}}
\newcommand{\Btheta}[0]{\boldsymbol{\theta}}
\newcommand{\Btau}[0]{\boldsymbol{\tau}}
\newcommand{\Bomega}[0]{\boldsymbol{\omega}}

%
% shorthand for unit vectors
%
\newcommand{\acap}[0]{\hat{\Ba}}
\newcommand{\bcap}[0]{\hat{\Bb}}
\newcommand{\ccap}[0]{\hat{\Bc}}
\newcommand{\dcap}[0]{\hat{\Bd}}
\newcommand{\ecap}[0]{\hat{\Be}}
\newcommand{\fcap}[0]{\hat{\Bf}}
\newcommand{\gcap}[0]{\hat{\Bg}}
\newcommand{\hcap}[0]{\hat{\Bh}}
\newcommand{\icap}[0]{\hat{\Bi}}
\newcommand{\jcap}[0]{\hat{\Bj}}
\newcommand{\kcap}[0]{\hat{\Bk}}
\newcommand{\lcap}[0]{\hat{\Bl}}
\newcommand{\mcap}[0]{\hat{\Bm}}
\newcommand{\ncap}[0]{\hat{\Bn}}
\newcommand{\ocap}[0]{\hat{\Bo}}
\newcommand{\pcap}[0]{\hat{\Bp}}
\newcommand{\qcap}[0]{\hat{\Bq}}
\newcommand{\rcap}[0]{\hat{\Br}}
\newcommand{\scap}[0]{\hat{\Bs}}
\newcommand{\tcap}[0]{\hat{\Bt}}
\newcommand{\ucap}[0]{\hat{\Bu}}
\newcommand{\vcap}[0]{\hat{\Bv}}
\newcommand{\wcap}[0]{\hat{\Bw}}
\newcommand{\xcap}[0]{\hat{\Bx}}
\newcommand{\ycap}[0]{\hat{\By}}
\newcommand{\zcap}[0]{\hat{\Bz}}
\newcommand{\thetacap}[0]{\hat{\Btheta}}

%
% to write R^n and C^n in a distinguishable fashion.  Perhaps change this
% to the double lined characters upon figuring out how to do so.
%
\newcommand{\C}[1]{$\mathbb{C}^{#1}$}
\newcommand{\R}[1]{$\mathbb{R}^{#1}$}

%
% various generally useful helpers
%

% derivative of #1 wrt. #2:
\newcommand{\D}[2] {\frac {d#2} {d#1}}

\newcommand{\inv}[1]{\frac{1}{#1}}
\newcommand{\cross}[0]{\times}

\newcommand{\abs}[1]{\lvert{#1}\rvert}
\newcommand{\norm}[1]{\lVert{#1}\rVert}
\newcommand{\innerprod}[2]{\langle{#1}, {#2}\rangle}
\newcommand{\dotprod}[2]{{#1} \cdot {#2}}
\newcommand{\bdotprod}[2]{\left({#1} \cdot {#2}\right)}
\newcommand{\crossprod}[2]{{#1} \cross {#2}}
\newcommand{\tripleprod}[3]{\dotprod{\left(\crossprod{#1}{#2}\right)}{#3}}

\DeclareMathOperator{\Proj}{Proj}
\DeclareMathOperator{\Span}{span}
\DeclareMathOperator{\Sgn}{sgn}
\DeclareMathOperator{\Area}{Area}
\DeclareMathOperator{\Volume}{Volume}

%
% A few miscellaneous things specific to this document
%
\newcommand{\crossop}[1]{\crossprod{#1}{}}

% R2 vector.
\newcommand{\VectorTwo}[2]{
\begin{bmatrix}
 {#1} \\
 {#2}
\end{bmatrix}
}

\newcommand{\VectorN}[1]{
\begin{bmatrix}
{#1}_1 \\
{#1}_2 \\
\vdots \\
{#1}_N \\
\end{bmatrix}
}

\newcommand{\DETuvij}[4]{
\begin{vmatrix}
 {#1}_{#3} & {#1}_{#4} \\
 {#2}_{#3} & {#2}_{#4}
\end{vmatrix}
}

\newcommand{\DETuvwijk}[6]{
\begin{vmatrix}
 {#1}_{#4} & {#1}_{#5} & {#1}_{#6} \\
 {#2}_{#4} & {#2}_{#5} & {#2}_{#6} \\
 {#3}_{#4} & {#3}_{#5} & {#3}_{#6}
\end{vmatrix}
}

\newcommand{\DETuvwxijkl}[8]{
\begin{vmatrix}
 {#1}_{#5} & {#1}_{#6} & {#1}_{#7} & {#1}_{#8} \\
 {#2}_{#5} & {#2}_{#6} & {#2}_{#7} & {#2}_{#8} \\
 {#3}_{#5} & {#3}_{#6} & {#3}_{#7} & {#3}_{#8} \\
 {#4}_{#5} & {#4}_{#6} & {#4}_{#7} & {#4}_{#8} \\
\end{vmatrix}
}

%\newcommand{\DETuvwxyijklm}[10]{
%\begin{vmatrix}
% {#1}_{#6} & {#1}_{#7} & {#1}_{#8} & {#1}_{#9} & {#1}_{#10} \\
% {#2}_{#6} & {#2}_{#7} & {#2}_{#8} & {#2}_{#9} & {#2}_{#10} \\
% {#3}_{#6} & {#3}_{#7} & {#3}_{#8} & {#3}_{#9} & {#3}_{#10} \\
% {#4}_{#6} & {#4}_{#7} & {#4}_{#8} & {#4}_{#9} & {#4}_{#10} \\
% {#5}_{#6} & {#5}_{#7} & {#5}_{#8} & {#5}_{#9} & {#5}_{#10}
%\end{vmatrix}
%}

% R3 vector.
\newcommand{\VectorThree}[3]{
\begin{bmatrix}
 {#1} \\
 {#2} \\
 {#3}
\end{bmatrix}
}



\author{Peeter Joot}
\email{peeter.joot@gmail.com}

%\documentclass[]{eliblogwidescreen}

\usepackage{amsmath}
\usepackage{mathpazo}

%
% shorthand for bold symbols, convenient for vectors and matrices
%
\newcommand{\Ba}[0]{\mathbf{a}}
\newcommand{\Bb}[0]{\mathbf{b}}
\newcommand{\Bc}[0]{\mathbf{c}}
\newcommand{\Bd}[0]{\mathbf{d}}
\newcommand{\Be}[0]{\mathbf{e}}
\newcommand{\Bf}[0]{\mathbf{f}}
\newcommand{\Bg}[0]{\mathbf{g}}
\newcommand{\Bh}[0]{\mathbf{h}}
\newcommand{\Bi}[0]{\mathbf{i}}
\newcommand{\Bj}[0]{\mathbf{j}}
\newcommand{\Bk}[0]{\mathbf{k}}
\newcommand{\Bl}[0]{\mathbf{l}}
\newcommand{\Bm}[0]{\mathbf{m}}
\newcommand{\Bn}[0]{\mathbf{n}}
\newcommand{\Bo}[0]{\mathbf{o}}
\newcommand{\Bp}[0]{\mathbf{p}}
\newcommand{\Bq}[0]{\mathbf{q}}
\newcommand{\Br}[0]{\mathbf{r}}
\newcommand{\Bs}[0]{\mathbf{s}}
\newcommand{\Bt}[0]{\mathbf{t}}
\newcommand{\Bu}[0]{\mathbf{u}}
\newcommand{\Bv}[0]{\mathbf{v}}
\newcommand{\Bw}[0]{\mathbf{w}}
\newcommand{\Bx}[0]{\mathbf{x}}
\newcommand{\By}[0]{\mathbf{y}}
\newcommand{\Bz}[0]{\mathbf{z}}
\newcommand{\BA}[0]{\mathbf{A}}
\newcommand{\BB}[0]{\mathbf{B}}
\newcommand{\BC}[0]{\mathbf{C}}
\newcommand{\BD}[0]{\mathbf{D}}
\newcommand{\BE}[0]{\mathbf{E}}
\newcommand{\BF}[0]{\mathbf{F}}
\newcommand{\BG}[0]{\mathbf{G}}
\newcommand{\BH}[0]{\mathbf{H}}
\newcommand{\BI}[0]{\mathbf{I}}
\newcommand{\BJ}[0]{\mathbf{J}}
\newcommand{\BK}[0]{\mathbf{K}}
\newcommand{\BL}[0]{\mathbf{L}}
\newcommand{\BM}[0]{\mathbf{M}}
\newcommand{\BN}[0]{\mathbf{N}}
\newcommand{\BO}[0]{\mathbf{O}}
\newcommand{\BP}[0]{\mathbf{P}}
\newcommand{\BQ}[0]{\mathbf{Q}}
\newcommand{\BR}[0]{\mathbf{R}}
\newcommand{\BS}[0]{\mathbf{S}}
\newcommand{\BT}[0]{\mathbf{T}}
\newcommand{\BU}[0]{\mathbf{U}}
\newcommand{\BV}[0]{\mathbf{V}}
\newcommand{\BW}[0]{\mathbf{W}}
\newcommand{\BX}[0]{\mathbf{X}}
\newcommand{\BY}[0]{\mathbf{Y}}
\newcommand{\BZ}[0]{\mathbf{Z}}

\newcommand{\Bzero}[0]{\mathbf{0}}
\newcommand{\Btheta}[0]{\boldsymbol{\theta}}
\newcommand{\Btau}[0]{\boldsymbol{\tau}}
\newcommand{\Bomega}[0]{\boldsymbol{\omega}}

%
% shorthand for unit vectors
%
\newcommand{\acap}[0]{\hat{\Ba}}
\newcommand{\bcap}[0]{\hat{\Bb}}
\newcommand{\ccap}[0]{\hat{\Bc}}
\newcommand{\dcap}[0]{\hat{\Bd}}
\newcommand{\ecap}[0]{\hat{\Be}}
\newcommand{\fcap}[0]{\hat{\Bf}}
\newcommand{\gcap}[0]{\hat{\Bg}}
\newcommand{\hcap}[0]{\hat{\Bh}}
\newcommand{\icap}[0]{\hat{\Bi}}
\newcommand{\jcap}[0]{\hat{\Bj}}
\newcommand{\kcap}[0]{\hat{\Bk}}
\newcommand{\lcap}[0]{\hat{\Bl}}
\newcommand{\mcap}[0]{\hat{\Bm}}
\newcommand{\ncap}[0]{\hat{\Bn}}
\newcommand{\ocap}[0]{\hat{\Bo}}
\newcommand{\pcap}[0]{\hat{\Bp}}
\newcommand{\qcap}[0]{\hat{\Bq}}
\newcommand{\rcap}[0]{\hat{\Br}}
\newcommand{\scap}[0]{\hat{\Bs}}
\newcommand{\tcap}[0]{\hat{\Bt}}
\newcommand{\ucap}[0]{\hat{\Bu}}
\newcommand{\vcap}[0]{\hat{\Bv}}
\newcommand{\wcap}[0]{\hat{\Bw}}
\newcommand{\xcap}[0]{\hat{\Bx}}
\newcommand{\ycap}[0]{\hat{\By}}
\newcommand{\zcap}[0]{\hat{\Bz}}
\newcommand{\thetacap}[0]{\hat{\Btheta}}

%
% to write R^n and C^n in a distinguishable fashion.  Perhaps change this
% to the double lined characters upon figuring out how to do so.
%
\newcommand{\C}[1]{$\mathbb{C}^{#1}$}
\newcommand{\R}[1]{$\mathbb{R}^{#1}$}

%
% various generally useful helpers
%

% derivative of #1 wrt. #2:
\newcommand{\D}[2] {\frac {d#2} {d#1}}

\newcommand{\inv}[1]{\frac{1}{#1}}
\newcommand{\cross}[0]{\times}

\newcommand{\abs}[1]{\lvert{#1}\rvert}
\newcommand{\norm}[1]{\lVert{#1}\rVert}
\newcommand{\innerprod}[2]{\langle{#1}, {#2}\rangle}
\newcommand{\dotprod}[2]{{#1} \cdot {#2}}
\newcommand{\bdotprod}[2]{\left({#1} \cdot {#2}\right)}
\newcommand{\crossprod}[2]{{#1} \cross {#2}}
\newcommand{\tripleprod}[3]{\dotprod{\left(\crossprod{#1}{#2}\right)}{#3}}

\DeclareMathOperator{\Proj}{Proj}
\DeclareMathOperator{\Span}{span}
\DeclareMathOperator{\Sgn}{sgn}
\DeclareMathOperator{\Area}{Area}
\DeclareMathOperator{\Volume}{Volume}

%
% A few miscellaneous things specific to this document
%
\newcommand{\crossop}[1]{\crossprod{#1}{}}

% R2 vector.
\newcommand{\VectorTwo}[2]{
\begin{bmatrix}
 {#1} \\
 {#2}
\end{bmatrix}
}

\newcommand{\VectorN}[1]{
\begin{bmatrix}
{#1}_1 \\
{#1}_2 \\
\vdots \\
{#1}_N \\
\end{bmatrix}
}

\newcommand{\DETuvij}[4]{
\begin{vmatrix}
 {#1}_{#3} & {#1}_{#4} \\
 {#2}_{#3} & {#2}_{#4}
\end{vmatrix}
}

\newcommand{\DETuvwijk}[6]{
\begin{vmatrix}
 {#1}_{#4} & {#1}_{#5} & {#1}_{#6} \\
 {#2}_{#4} & {#2}_{#5} & {#2}_{#6} \\
 {#3}_{#4} & {#3}_{#5} & {#3}_{#6}
\end{vmatrix}
}

\newcommand{\DETuvwxijkl}[8]{
\begin{vmatrix}
 {#1}_{#5} & {#1}_{#6} & {#1}_{#7} & {#1}_{#8} \\
 {#2}_{#5} & {#2}_{#6} & {#2}_{#7} & {#2}_{#8} \\
 {#3}_{#5} & {#3}_{#6} & {#3}_{#7} & {#3}_{#8} \\
 {#4}_{#5} & {#4}_{#6} & {#4}_{#7} & {#4}_{#8} \\
\end{vmatrix}
}

%\newcommand{\DETuvwxyijklm}[10]{
%\begin{vmatrix}
% {#1}_{#6} & {#1}_{#7} & {#1}_{#8} & {#1}_{#9} & {#1}_{#10} \\
% {#2}_{#6} & {#2}_{#7} & {#2}_{#8} & {#2}_{#9} & {#2}_{#10} \\
% {#3}_{#6} & {#3}_{#7} & {#3}_{#8} & {#3}_{#9} & {#3}_{#10} \\
% {#4}_{#6} & {#4}_{#7} & {#4}_{#8} & {#4}_{#9} & {#4}_{#10} \\
% {#5}_{#6} & {#5}_{#7} & {#5}_{#8} & {#5}_{#9} & {#5}_{#10}
%\end{vmatrix}
%}

% R3 vector.
\newcommand{\VectorThree}[3]{
\begin{bmatrix}
 {#1} \\
 {#2} \\
 {#3}
\end{bmatrix}
}



\author{Peeter Joot}
\email{peeter.joot@gmail.com}


\chapter{Desai Chapter II notes and problems.}
\label{chap:desaiCh2}
%\useCCL
\blogpage{http://sites.google.com/site/peeterjoot/math2010/desaiCh2.pdf}
\date{Sept 19, 2010}
\revisionInfo{desaiCh2.tex}

%\beginArtWithToc
\beginArtNoToc

\section{Motivation.}

Chapter II notes for \cite{desai2009quantum}.

\section{Notes}
\subsection{Canonical Commutator}

Based on the canonical relationship $[X,P] = i\hbar$, and $\braket{x'}{x} = \delta(x'-x)$, Desai determines the form of the $P$ operator in continuous space.  A consequence of this is that the matrix element of the momentum operator is found to have a delta function specification

\begin{align*}
\bra{x'} P \ket{x} = \delta(x - x') \left( -i \hbar \frac{d}{dx} \right).
\end{align*}

In particular the matrix element associated with the state $\ket{\phi}$ is found to be

\begin{align*}
\bra{x'} P \ket{\phi} = -i \hbar \frac{d}{dx'} \phi(x').
\end{align*}

Compare this to \cite{liboff2003iqm}, where this last is taken as the definition of the momentum operator, and the relationship to the delta function is not spelled out explicitly.  This canonical commuator approach, while more abstract, seems to have less black magic involved in the setup.  We do require the commutator relationship $[X,P] = i\hbar$ to be pulled out of a magic hat, but at least the magic show is a structured one based on a small set of core assumptions.

It will likely be good to come back to this later when trying to reconsile this new (for me) Dirac notation with the more basic notation I'm already comfortable with.  When trying to compare the two, it will be good to note that there is a matrix element that is implied in the more old fashioned treatment in a book such as \cite{bohm1989qt}.

There is one fundamental assumption that appears to be made in this section that isn't justified by anything except the end result.  That is the assumption that $P$ is a derivative like operator, acting with a product rule action.  That's used to obtain (2.28) and is a fairly black magic operation.  This same assumption, is also hiding, somewhat sneakily, in the manipulation for (2.44).

\subsection{Generalized momentum commutator.}

It is stated that

\begin{align*}
[P,X^n] = - n i \hbar X^{n-1}.
\end{align*}

Let's prove this.  The $n=1$ case is the canonical commutator, which is assumed.  Is there any good way to justify that from first principles, as presented in the text?  We have to prove this for $n$, given the relationship for $n-1$.  Expanding the $n$th power commuator we have

\begin{align*}
[P,X^n] 
&= P X^n - X^n P \\
&= P X^{n-1} X - X^{n } P \\
\end{align*}

Rearranging the $n-1$ result we have

\begin{align*}
P X^{n-1} = X^{n-1} P - (n-1) i \hbar X^{n-2},
\end{align*}

and can insert that in our $[P,X^n]$ expansion for

\begin{align*}
[P,X^n] 
&= \left( X^{n-1} P - (n-1) i \hbar X^{n-2} \right)X - X^{n } P \\
&= X^{n-1} (PX) - (n-1) i \hbar X^{n-1} - X^{n } P \\
&= X^{n-1} ( X P - i\hbar) - (n-1) i \hbar X^{n-1} - X^{n } P \\
&= -X^{n-1} i\hbar - (n-1) i \hbar X^{n-1} \\
&= -n i \hbar X^{n-1} 
\qquad\square
\end{align*}

\subsection{Uncertainty principle.}

The origin of the statement $[\Delta A, \Delta B] = [A, B]$ is not something that seemed obvious.  Expanding this out however is straightforward, and clarfies things.  That is

\begin{align*}
[\Delta A, \Delta B] 
&= (A - \expectation{A}) (B - \expectation{B}) - (B - \expectation{B}) (A - \expectation{A}) \\
&= 
\left( A B - \expectation{A} B - \expectation{B} A +\expectation{A} \expectation{B} \right)
-\left( B A - \expectation{B} A - \expectation{A} B +\expectation{B} \expectation{A} \right) \\
&= 
A B - B A \\
&= 
[A, B]
\qquad\square
\end{align*}

\subsection{Size of a particle}

I found it curious that using $\Delta x \Delta p \approx \hbar$ instead of $\Delta x \Delta p \ge \hbar/2$, was sufficient to obtain the hydrogen ground state energy $E_{\text{min}} = -e^2/2 a_0$, without also having to do any factor of two fudging.

\subsection{Space displacement operator.}

I'd be curious to know if others find the loose use of equality for approximation after approximation slightly disturbing too?

I also find it curious that (2.140) is written

\begin{align*}
D(x) = \exp\left( -i \frac{P}{\hbar} x \right),
\end{align*}

and not
\begin{align*}
D(x) = \exp\left( -i x \frac{P}{\hbar} \right).
\end{align*}

Is this intentional?  It doesn't seem like $P$ ought to be acting on $x$ in this case, so why order the terms that way?

Expanding the application of this operator, or at least its first order Taylor series, is helpful to get an idea about this.  Doing so, with the original $\Delta x'$ value used in the derivation of the text we have to start

\begin{align*}
D(\Delta x') \ket{\phi} 
&\approx \left(1 - i \frac{P}{\hbar} \Delta x' \right) \ket{\phi} \\
&= \left(1 - i \left( -i \hbar \delta(x -x') \frac{\partial}{\partial x} \right) \inv{\hbar} \Delta x'\right) \ket{\phi} \\
\end{align*}

This shows that the $\Delta x$ factor can be commuted with the momentum operator, as it is not a function of $x'$, so the question of $P x$, vs $x P$ above appears to be a non-issue.

Regardless of that conclusion, it seems worthy to continue an attempt at expanding this shift operator action on the state vector.  Let's do so, but do so by computing the matrix element $\bra{x'} D(\Delta x') \ket{\phi}$.  That is

\begin{align*}
\bra{x'} D(\Delta x') \ket{\phi} 
&\approx
\braket{x'}{\phi} - \bra{x'} \delta(x -x') \frac{\partial}{\partial x} \Delta x' \ket{\phi} \\
&=
\phi(x') - \int \bra{x'} \delta(x -x') \frac{\partial}{\partial x} \Delta x' \ket{x'} \braket{x'}{\phi} dx' \\
&=
\phi(x') - \Delta x' \int \delta(x -x') \frac{\partial}{\partial x} \braket{x'}{\phi} dx' \\
&=
\phi(x') - \Delta x' \frac{\partial}{\partial x'} \braket{x'}{\phi} \\
&=
\phi(x') - \Delta x' \frac{\partial}{\partial x'} \phi(x') \\
\end{align*}

This is consistent with the text.  It is interesting, and initially suprising that the space displacement operator when applied to a state vector introduces a negative shift in the wave function associated with that state vector.  In the derivation of the text, this was associated with the use of integration by parts (ie: due to the sign change in that integration).  Here we see it sneak back in, due to the $i^2$ once the momentum operator is expanded completely.

As last note and question.  The first order Taylor approximation of the momentum operator was used.  If the higher order terms are retained, as in

\begin{align*}
\exp\left( -i \Delta x' \frac{P}{\hbar} \right) = 
1 - \Delta x' \delta(x -x') \frac{\partial}{\partial x} + 
\inv{2} \left( - \Delta x' \delta(x -x') \frac{\partial}{\partial x} \right)^2 + \cdots,
\end{align*}

then how does one evaluate a squared delta function (or Nth power)?

\subsection{Time evolution operator}

The phrase ``we identify time evoution with the Hamiltonian''.  What a magic hat manuver!  Is there a way that this would be logical without already knowing the answer?

\subsection{Dispersion delta function representation.}

The Principle part notation here I found a bit unclear.  He writes

\begin{align*}
\lim_{\epsilon \rightarrow 0} 
\frac{(x'-x)}{(x'-x)^2 + \epsilon^2}
= 
P\left( \inv{x' - x} \right).
\end{align*}

In complex variables the principle part is the negative power series terms.  For example for $f(z) = \sum a_k z^k$, the principle part is

\begin{align*}
\sum_{k = -\infty}^{-1} a_k z^k
\end{align*}

This doesn't vanish at $z = 0$ as the principle part in this section is stated to.  In (2.202) he pulls the $P$ out of the integral, but I think the intention is really to keep this associated with the $1/(x'-x)$, as in

\begin{align*}
\lim_{\epsilon \rightarrow 0} 
\inv{\pi} \int_0^\infty dx' \frac{f(x')}{x'-x - i \epsilon}
= 
\inv{\pi} \int_0^\infty dx' f(x') P\left( \inv{x' - x} \right) + i f(x)
\end{align*}

Will this even have any relevance in this text?

\section{Problems.}
\subsection{1.}
\subsection{2.}
\subsection{3.}
\subsection{4.}
\subsection{5. Hermitian radial differential operator.}

Show that the operator 

\begin{align*}
R = -i \hbar \PD{r}{},
\end{align*}

is not Hermitian, and find the constant $a$ so that 

\begin{align*}
T = -i \hbar \left( \PD{r}{} + \frac{a}{r} \right),
\end{align*}

is Hermitian.

For the first part of the problem we can show that

\begin{align*}
\left( \bra{\psicap} R \ket{\phicap} \right)^\conj \ne \bra{\phicap} R \ket{\psicap}.
\end{align*}

For the RHS we have

\begin{align*}
\bra{\phicap} R \ket{\psicap} 
= -i \hbar \iiint dr d\theta d\phi r^2 \sin\theta \phicap^\conj \PD{r}{\psicap}
\end{align*}

and for the LHS we have

\begin{align*}
\left( \bra{\psicap} R \ket{\phicap} \right)^\conj
&= i \hbar \iiint dr d\theta d\phi r^2 \sin\theta \psicap \PD{r}{\phicap^\conj} \\
&= -i \hbar \iiint dr d\theta d\phi \sin\theta 
\left( 2 r \psicap 
+ r^2 \PD{\psicap}{r} 
\right)
\phicap^\conj 
\\
\end{align*}

So, unless $r\psicap = 0$, the operator $R$ is not Hermitian.

Moving on to finding the constant $a$ such that $T$ is Hermitian we calculate

\begin{align*}
\left( \bra{\psicap} T \ket{\phicap} \right)^\conj
&= i \hbar \iiint dr d\theta d\phi r^2 \sin\theta \psicap \left( \PD{r}{} + \frac{a}{r} \right) \phicap^\conj \\
&= i \hbar \iiint dr d\theta d\phi \sin\theta \psicap \left( r^2 \PD{r}{} + a r \right) \phicap^\conj \\
&= -i \hbar \iiint dr d\theta d\phi \sin\theta \left( r^2 \PD{r}{\psicap} + 2 r \psicap - a r \psicap \right) \phicap^\conj \\
\end{align*}

and

\begin{align*}
\bra{\phicap} T \ket{\psicap} 
= -i \hbar \iiint dr d\theta d\phi r^2 \sin\theta \phicap^\conj \left( r^2 \PD{r}{\psicap} + a r \psicap \right)
\end{align*}

So, for $T$ to be Hermitian, we require

\begin{align*}
2 r - a r = a r.
\end{align*}

So $a = 1$, and our Hermitian operator is
\begin{align*}
T = -i \hbar \left( \PD{r}{} + \frac{1}{r} \right).
\end{align*}

\subsection{6.}
\subsection{7.}
\subsection{8.}
\subsection{9.}
\subsection{10.}
\subsection{11.}

\EndArticle
%\EndNoBibArticle
