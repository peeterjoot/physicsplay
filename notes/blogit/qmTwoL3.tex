%
% Copyright � 2015 Peeter Joot.  All Rights Reserved.
% Licenced as described in the file LICENSE under the root directory of this GIT repository.
%
\documentclass[]{eliblog}

\usepackage{amsmath}
\usepackage{mathpazo}

%
% shorthand for bold symbols, convenient for vectors and matrices
%
\newcommand{\Ba}[0]{\mathbf{a}}
\newcommand{\Bb}[0]{\mathbf{b}}
\newcommand{\Bc}[0]{\mathbf{c}}
\newcommand{\Bd}[0]{\mathbf{d}}
\newcommand{\Be}[0]{\mathbf{e}}
\newcommand{\Bf}[0]{\mathbf{f}}
\newcommand{\Bg}[0]{\mathbf{g}}
\newcommand{\Bh}[0]{\mathbf{h}}
\newcommand{\Bi}[0]{\mathbf{i}}
\newcommand{\Bj}[0]{\mathbf{j}}
\newcommand{\Bk}[0]{\mathbf{k}}
\newcommand{\Bl}[0]{\mathbf{l}}
\newcommand{\Bm}[0]{\mathbf{m}}
\newcommand{\Bn}[0]{\mathbf{n}}
\newcommand{\Bo}[0]{\mathbf{o}}
\newcommand{\Bp}[0]{\mathbf{p}}
\newcommand{\Bq}[0]{\mathbf{q}}
\newcommand{\Br}[0]{\mathbf{r}}
\newcommand{\Bs}[0]{\mathbf{s}}
\newcommand{\Bt}[0]{\mathbf{t}}
\newcommand{\Bu}[0]{\mathbf{u}}
\newcommand{\Bv}[0]{\mathbf{v}}
\newcommand{\Bw}[0]{\mathbf{w}}
\newcommand{\Bx}[0]{\mathbf{x}}
\newcommand{\By}[0]{\mathbf{y}}
\newcommand{\Bz}[0]{\mathbf{z}}
\newcommand{\BA}[0]{\mathbf{A}}
\newcommand{\BB}[0]{\mathbf{B}}
\newcommand{\BC}[0]{\mathbf{C}}
\newcommand{\BD}[0]{\mathbf{D}}
\newcommand{\BE}[0]{\mathbf{E}}
\newcommand{\BF}[0]{\mathbf{F}}
\newcommand{\BG}[0]{\mathbf{G}}
\newcommand{\BH}[0]{\mathbf{H}}
\newcommand{\BI}[0]{\mathbf{I}}
\newcommand{\BJ}[0]{\mathbf{J}}
\newcommand{\BK}[0]{\mathbf{K}}
\newcommand{\BL}[0]{\mathbf{L}}
\newcommand{\BM}[0]{\mathbf{M}}
\newcommand{\BN}[0]{\mathbf{N}}
\newcommand{\BO}[0]{\mathbf{O}}
\newcommand{\BP}[0]{\mathbf{P}}
\newcommand{\BQ}[0]{\mathbf{Q}}
\newcommand{\BR}[0]{\mathbf{R}}
\newcommand{\BS}[0]{\mathbf{S}}
\newcommand{\BT}[0]{\mathbf{T}}
\newcommand{\BU}[0]{\mathbf{U}}
\newcommand{\BV}[0]{\mathbf{V}}
\newcommand{\BW}[0]{\mathbf{W}}
\newcommand{\BX}[0]{\mathbf{X}}
\newcommand{\BY}[0]{\mathbf{Y}}
\newcommand{\BZ}[0]{\mathbf{Z}}

\newcommand{\Bzero}[0]{\mathbf{0}}
\newcommand{\Btheta}[0]{\boldsymbol{\theta}}
\newcommand{\Btau}[0]{\boldsymbol{\tau}}
\newcommand{\Bomega}[0]{\boldsymbol{\omega}}

%
% shorthand for unit vectors
%
\newcommand{\acap}[0]{\hat{\Ba}}
\newcommand{\bcap}[0]{\hat{\Bb}}
\newcommand{\ccap}[0]{\hat{\Bc}}
\newcommand{\dcap}[0]{\hat{\Bd}}
\newcommand{\ecap}[0]{\hat{\Be}}
\newcommand{\fcap}[0]{\hat{\Bf}}
\newcommand{\gcap}[0]{\hat{\Bg}}
\newcommand{\hcap}[0]{\hat{\Bh}}
\newcommand{\icap}[0]{\hat{\Bi}}
\newcommand{\jcap}[0]{\hat{\Bj}}
\newcommand{\kcap}[0]{\hat{\Bk}}
\newcommand{\lcap}[0]{\hat{\Bl}}
\newcommand{\mcap}[0]{\hat{\Bm}}
\newcommand{\ncap}[0]{\hat{\Bn}}
\newcommand{\ocap}[0]{\hat{\Bo}}
\newcommand{\pcap}[0]{\hat{\Bp}}
\newcommand{\qcap}[0]{\hat{\Bq}}
\newcommand{\rcap}[0]{\hat{\Br}}
\newcommand{\scap}[0]{\hat{\Bs}}
\newcommand{\tcap}[0]{\hat{\Bt}}
\newcommand{\ucap}[0]{\hat{\Bu}}
\newcommand{\vcap}[0]{\hat{\Bv}}
\newcommand{\wcap}[0]{\hat{\Bw}}
\newcommand{\xcap}[0]{\hat{\Bx}}
\newcommand{\ycap}[0]{\hat{\By}}
\newcommand{\zcap}[0]{\hat{\Bz}}
\newcommand{\thetacap}[0]{\hat{\Btheta}}

%
% to write R^n and C^n in a distinguishable fashion.  Perhaps change this
% to the double lined characters upon figuring out how to do so.
%
\newcommand{\C}[1]{$\mathbb{C}^{#1}$}
\newcommand{\R}[1]{$\mathbb{R}^{#1}$}

%
% various generally useful helpers
%

% derivative of #1 wrt. #2:
\newcommand{\D}[2] {\frac {d#2} {d#1}}

\newcommand{\inv}[1]{\frac{1}{#1}}
\newcommand{\cross}[0]{\times}

\newcommand{\abs}[1]{\lvert{#1}\rvert}
\newcommand{\norm}[1]{\lVert{#1}\rVert}
\newcommand{\innerprod}[2]{\langle{#1}, {#2}\rangle}
\newcommand{\dotprod}[2]{{#1} \cdot {#2}}
\newcommand{\bdotprod}[2]{\left({#1} \cdot {#2}\right)}
\newcommand{\crossprod}[2]{{#1} \cross {#2}}
\newcommand{\tripleprod}[3]{\dotprod{\left(\crossprod{#1}{#2}\right)}{#3}}

\DeclareMathOperator{\Proj}{Proj}
\DeclareMathOperator{\Span}{span}
\DeclareMathOperator{\Sgn}{sgn}
\DeclareMathOperator{\Area}{Area}
\DeclareMathOperator{\Volume}{Volume}

%
% A few miscellaneous things specific to this document
%
\newcommand{\crossop}[1]{\crossprod{#1}{}}

% R2 vector.
\newcommand{\VectorTwo}[2]{
\begin{bmatrix}
 {#1} \\
 {#2}
\end{bmatrix}
}

\newcommand{\VectorN}[1]{
\begin{bmatrix}
{#1}_1 \\
{#1}_2 \\
\vdots \\
{#1}_N \\
\end{bmatrix}
}

\newcommand{\DETuvij}[4]{
\begin{vmatrix}
 {#1}_{#3} & {#1}_{#4} \\
 {#2}_{#3} & {#2}_{#4}
\end{vmatrix}
}

\newcommand{\DETuvwijk}[6]{
\begin{vmatrix}
 {#1}_{#4} & {#1}_{#5} & {#1}_{#6} \\
 {#2}_{#4} & {#2}_{#5} & {#2}_{#6} \\
 {#3}_{#4} & {#3}_{#5} & {#3}_{#6}
\end{vmatrix}
}

\newcommand{\DETuvwxijkl}[8]{
\begin{vmatrix}
 {#1}_{#5} & {#1}_{#6} & {#1}_{#7} & {#1}_{#8} \\
 {#2}_{#5} & {#2}_{#6} & {#2}_{#7} & {#2}_{#8} \\
 {#3}_{#5} & {#3}_{#6} & {#3}_{#7} & {#3}_{#8} \\
 {#4}_{#5} & {#4}_{#6} & {#4}_{#7} & {#4}_{#8} \\
\end{vmatrix}
}

%\newcommand{\DETuvwxyijklm}[10]{
%\begin{vmatrix}
% {#1}_{#6} & {#1}_{#7} & {#1}_{#8} & {#1}_{#9} & {#1}_{#10} \\
% {#2}_{#6} & {#2}_{#7} & {#2}_{#8} & {#2}_{#9} & {#2}_{#10} \\
% {#3}_{#6} & {#3}_{#7} & {#3}_{#8} & {#3}_{#9} & {#3}_{#10} \\
% {#4}_{#6} & {#4}_{#7} & {#4}_{#8} & {#4}_{#9} & {#4}_{#10} \\
% {#5}_{#6} & {#5}_{#7} & {#5}_{#8} & {#5}_{#9} & {#5}_{#10}
%\end{vmatrix}
%}

% R3 vector.
\newcommand{\VectorThree}[3]{
\begin{bmatrix}
 {#1} \\
 {#2} \\
 {#3}
\end{bmatrix}
}



\author{Peeter Joot}
\email{peeter.joot@gmail.com}

%\documentclass[]{eliblogwidescreen}

\usepackage{amsmath}
\usepackage{mathpazo}

%
% shorthand for bold symbols, convenient for vectors and matrices
%
\newcommand{\Ba}[0]{\mathbf{a}}
\newcommand{\Bb}[0]{\mathbf{b}}
\newcommand{\Bc}[0]{\mathbf{c}}
\newcommand{\Bd}[0]{\mathbf{d}}
\newcommand{\Be}[0]{\mathbf{e}}
\newcommand{\Bf}[0]{\mathbf{f}}
\newcommand{\Bg}[0]{\mathbf{g}}
\newcommand{\Bh}[0]{\mathbf{h}}
\newcommand{\Bi}[0]{\mathbf{i}}
\newcommand{\Bj}[0]{\mathbf{j}}
\newcommand{\Bk}[0]{\mathbf{k}}
\newcommand{\Bl}[0]{\mathbf{l}}
\newcommand{\Bm}[0]{\mathbf{m}}
\newcommand{\Bn}[0]{\mathbf{n}}
\newcommand{\Bo}[0]{\mathbf{o}}
\newcommand{\Bp}[0]{\mathbf{p}}
\newcommand{\Bq}[0]{\mathbf{q}}
\newcommand{\Br}[0]{\mathbf{r}}
\newcommand{\Bs}[0]{\mathbf{s}}
\newcommand{\Bt}[0]{\mathbf{t}}
\newcommand{\Bu}[0]{\mathbf{u}}
\newcommand{\Bv}[0]{\mathbf{v}}
\newcommand{\Bw}[0]{\mathbf{w}}
\newcommand{\Bx}[0]{\mathbf{x}}
\newcommand{\By}[0]{\mathbf{y}}
\newcommand{\Bz}[0]{\mathbf{z}}
\newcommand{\BA}[0]{\mathbf{A}}
\newcommand{\BB}[0]{\mathbf{B}}
\newcommand{\BC}[0]{\mathbf{C}}
\newcommand{\BD}[0]{\mathbf{D}}
\newcommand{\BE}[0]{\mathbf{E}}
\newcommand{\BF}[0]{\mathbf{F}}
\newcommand{\BG}[0]{\mathbf{G}}
\newcommand{\BH}[0]{\mathbf{H}}
\newcommand{\BI}[0]{\mathbf{I}}
\newcommand{\BJ}[0]{\mathbf{J}}
\newcommand{\BK}[0]{\mathbf{K}}
\newcommand{\BL}[0]{\mathbf{L}}
\newcommand{\BM}[0]{\mathbf{M}}
\newcommand{\BN}[0]{\mathbf{N}}
\newcommand{\BO}[0]{\mathbf{O}}
\newcommand{\BP}[0]{\mathbf{P}}
\newcommand{\BQ}[0]{\mathbf{Q}}
\newcommand{\BR}[0]{\mathbf{R}}
\newcommand{\BS}[0]{\mathbf{S}}
\newcommand{\BT}[0]{\mathbf{T}}
\newcommand{\BU}[0]{\mathbf{U}}
\newcommand{\BV}[0]{\mathbf{V}}
\newcommand{\BW}[0]{\mathbf{W}}
\newcommand{\BX}[0]{\mathbf{X}}
\newcommand{\BY}[0]{\mathbf{Y}}
\newcommand{\BZ}[0]{\mathbf{Z}}

\newcommand{\Bzero}[0]{\mathbf{0}}
\newcommand{\Btheta}[0]{\boldsymbol{\theta}}
\newcommand{\Btau}[0]{\boldsymbol{\tau}}
\newcommand{\Bomega}[0]{\boldsymbol{\omega}}

%
% shorthand for unit vectors
%
\newcommand{\acap}[0]{\hat{\Ba}}
\newcommand{\bcap}[0]{\hat{\Bb}}
\newcommand{\ccap}[0]{\hat{\Bc}}
\newcommand{\dcap}[0]{\hat{\Bd}}
\newcommand{\ecap}[0]{\hat{\Be}}
\newcommand{\fcap}[0]{\hat{\Bf}}
\newcommand{\gcap}[0]{\hat{\Bg}}
\newcommand{\hcap}[0]{\hat{\Bh}}
\newcommand{\icap}[0]{\hat{\Bi}}
\newcommand{\jcap}[0]{\hat{\Bj}}
\newcommand{\kcap}[0]{\hat{\Bk}}
\newcommand{\lcap}[0]{\hat{\Bl}}
\newcommand{\mcap}[0]{\hat{\Bm}}
\newcommand{\ncap}[0]{\hat{\Bn}}
\newcommand{\ocap}[0]{\hat{\Bo}}
\newcommand{\pcap}[0]{\hat{\Bp}}
\newcommand{\qcap}[0]{\hat{\Bq}}
\newcommand{\rcap}[0]{\hat{\Br}}
\newcommand{\scap}[0]{\hat{\Bs}}
\newcommand{\tcap}[0]{\hat{\Bt}}
\newcommand{\ucap}[0]{\hat{\Bu}}
\newcommand{\vcap}[0]{\hat{\Bv}}
\newcommand{\wcap}[0]{\hat{\Bw}}
\newcommand{\xcap}[0]{\hat{\Bx}}
\newcommand{\ycap}[0]{\hat{\By}}
\newcommand{\zcap}[0]{\hat{\Bz}}
\newcommand{\thetacap}[0]{\hat{\Btheta}}

%
% to write R^n and C^n in a distinguishable fashion.  Perhaps change this
% to the double lined characters upon figuring out how to do so.
%
\newcommand{\C}[1]{$\mathbb{C}^{#1}$}
\newcommand{\R}[1]{$\mathbb{R}^{#1}$}

%
% various generally useful helpers
%

% derivative of #1 wrt. #2:
\newcommand{\D}[2] {\frac {d#2} {d#1}}

\newcommand{\inv}[1]{\frac{1}{#1}}
\newcommand{\cross}[0]{\times}

\newcommand{\abs}[1]{\lvert{#1}\rvert}
\newcommand{\norm}[1]{\lVert{#1}\rVert}
\newcommand{\innerprod}[2]{\langle{#1}, {#2}\rangle}
\newcommand{\dotprod}[2]{{#1} \cdot {#2}}
\newcommand{\bdotprod}[2]{\left({#1} \cdot {#2}\right)}
\newcommand{\crossprod}[2]{{#1} \cross {#2}}
\newcommand{\tripleprod}[3]{\dotprod{\left(\crossprod{#1}{#2}\right)}{#3}}

\DeclareMathOperator{\Proj}{Proj}
\DeclareMathOperator{\Span}{span}
\DeclareMathOperator{\Sgn}{sgn}
\DeclareMathOperator{\Area}{Area}
\DeclareMathOperator{\Volume}{Volume}

%
% A few miscellaneous things specific to this document
%
\newcommand{\crossop}[1]{\crossprod{#1}{}}

% R2 vector.
\newcommand{\VectorTwo}[2]{
\begin{bmatrix}
 {#1} \\
 {#2}
\end{bmatrix}
}

\newcommand{\VectorN}[1]{
\begin{bmatrix}
{#1}_1 \\
{#1}_2 \\
\vdots \\
{#1}_N \\
\end{bmatrix}
}

\newcommand{\DETuvij}[4]{
\begin{vmatrix}
 {#1}_{#3} & {#1}_{#4} \\
 {#2}_{#3} & {#2}_{#4}
\end{vmatrix}
}

\newcommand{\DETuvwijk}[6]{
\begin{vmatrix}
 {#1}_{#4} & {#1}_{#5} & {#1}_{#6} \\
 {#2}_{#4} & {#2}_{#5} & {#2}_{#6} \\
 {#3}_{#4} & {#3}_{#5} & {#3}_{#6}
\end{vmatrix}
}

\newcommand{\DETuvwxijkl}[8]{
\begin{vmatrix}
 {#1}_{#5} & {#1}_{#6} & {#1}_{#7} & {#1}_{#8} \\
 {#2}_{#5} & {#2}_{#6} & {#2}_{#7} & {#2}_{#8} \\
 {#3}_{#5} & {#3}_{#6} & {#3}_{#7} & {#3}_{#8} \\
 {#4}_{#5} & {#4}_{#6} & {#4}_{#7} & {#4}_{#8} \\
\end{vmatrix}
}

%\newcommand{\DETuvwxyijklm}[10]{
%\begin{vmatrix}
% {#1}_{#6} & {#1}_{#7} & {#1}_{#8} & {#1}_{#9} & {#1}_{#10} \\
% {#2}_{#6} & {#2}_{#7} & {#2}_{#8} & {#2}_{#9} & {#2}_{#10} \\
% {#3}_{#6} & {#3}_{#7} & {#3}_{#8} & {#3}_{#9} & {#3}_{#10} \\
% {#4}_{#6} & {#4}_{#7} & {#4}_{#8} & {#4}_{#9} & {#4}_{#10} \\
% {#5}_{#6} & {#5}_{#7} & {#5}_{#8} & {#5}_{#9} & {#5}_{#10}
%\end{vmatrix}
%}

% R3 vector.
\newcommand{\VectorThree}[3]{
\begin{bmatrix}
 {#1} \\
 {#2} \\
 {#3}
\end{bmatrix}
}



\author{Peeter Joot}
\email{peeter.joot@gmail.com}


\chapter{PHY450H1F: Quantum Mechanics II.  Lecture 3 (Taught by Prof J.E. Sipe).  Perturbation methods}
\label{chap:qmTwoL3}
%\useCCL
\blogpage{http://sites.google.com/site/peeterjoot/math2011/qmTwoL3.pdf}
\date{Sept 19, 2011}
\revisionInfo{qmTwoL3.tex}

\beginArtWithToc
%\beginArtNoToc

Peeter's lecture notes from class.  May not be entirely coherent.

\section{States and wave functions}

Suppose we have the following non-degenerate energy eigenstates

\begin{align*}
&\vdots \\
E_1 &\sim \ket{\Psi_1} \\
E_0 &\sim \ket{\Psi_0}
\end{align*}

and consider a state that is ``very close'' to $\ket{\Psi_n}$.

\begin{equation}\label{eqn:qmTwoL3:10}
\ket{\Psi} = \ket{\Psi_n} + \ket{\delta \Psi_n}
\end{equation}

We form projections onto $\ket{\Psi_n}$ ``direction''.  The difference from this projection will be written $\ket{\Psi_{n \perp}}$, as depicted in figure (\ref{fig:qmTwoL3fig1}).  This illustration cannot not be interpreted literally, but illustrates the idea nicely.

\begin{figure}[htp]
\centering
\includegraphics[totalheight=0.4\textheight]{qmTwoL3fig1}
\caption{Pictorial illustration of ket projections}\label{fig:qmTwoL3fig1}
\end{figure}

For the amount along the projection onto $\ket{\Psi_n}$ we write

\begin{equation}\label{eqn:qmTwoL3:30}
\braket{\Psi_n}{\delta \Psi_n} = \delta \alpha
\end{equation}

so that the total deviation from the original state is

\begin{equation}\label{eqn:qmTwoL3:50}
\ket{\delta \Psi_n} 
= \delta \alpha \ket{\Psi_n} 
+ \ket{\delta \Psi_{n \perp}} .
\end{equation}

The varied ket is then
\begin{equation}\label{eqn:qmTwoL3:70}
\ket{\Psi} 
= (1 + \delta \alpha )\ket{\Psi_n} + \ket{\delta \Psi_{n \perp}} 
\end{equation}

where

\begin{equation}\label{eqn:qmTwoL3:90}
(\delta \alpha)^2, \braket{\delta \Psi_{n \perp}}{\delta \Psi_{n \perp}}  \ll 1
\end{equation}

In terms of these projections our kets magnitude is

\begin{align*}
\braket{\Psi}{\Psi} 
&= 
\Bigl(
(1 + {\delta \alpha}^\conj )\bra{\Psi_n} + \bra{\delta \Psi_{n \perp}} 
\Bigr)
\Bigl(
(1 + \delta \alpha )\ket{\Psi_n} + \ket{\delta \Psi_{n \perp}} 
\Bigr) \\
&=
\Abs{1 + \delta \alpha}^2 \braket{\Psi_n}{\Psi_n}
+ 
\braket{\delta \Psi_{n \perp}}{\delta \Psi_{n \perp}}  \\
&\quad +
(1 + {\delta \alpha}^\conj )\braket{\Psi_n}{\delta \Psi_{n \perp}} 
+
(1 + \delta \alpha )\braket{\delta \Psi_{n \perp}}{\delta \Psi_n} 
\end{align*}

Because $\braket{\delta \Psi_{n \perp}}{\delta \Psi_n} = 0$ this is

\begin{equation}\label{eqn:qmTwoL3:110}
\braket{\Psi}{\Psi}
= 
\Abs{1 + \delta \alpha }^2
\braket{\delta \Psi_{n \perp}}{\delta \Psi_{n \perp}}.
\end{equation}

Similarly for the energy expectation we have

\begin{align*}
\braket{\Psi}{\Psi} 
&= 
\Bigl(
(1 + {\delta \alpha}^\conj )\bra{\Psi_n} + \bra{\delta \Psi_{n \perp}} 
\Bigr)
H
\Bigl(
(1 + \delta \alpha )\ket{\Psi_n} + \ket{\delta \Psi_{n \perp}} 
\Bigr) \\
&=
\Abs{1 + \delta \alpha}^2 E_n \braket{\Psi_n}{\Psi_n}
+ 
\braket{\delta \Psi_{n \perp}} H {\delta \Psi_{n \perp}}  \\
&\quad + 
(1 + {\delta \alpha}^\conj ) E_n \braket{\Psi_n}{\delta \Psi_{n \perp}} 
+
(1 + \delta \alpha ) E_n \braket{\delta \Psi_{n \perp}}{\delta \Psi_n} 
\end{align*}

Or
\begin{equation}\label{eqn:qmTwoL3:130}
\bra{\Psi} H \ket{\Psi}
= 
E_n \Abs{1 + \delta \alpha }^2
+
\bra{\delta \Psi_{n \perp}} H \ket{\delta \Psi_{n \perp}}.
\end{equation}

This gives

\begin{align*}
E[\Psi] 
&= 
\frac{
\bra{\Psi} H \ket{\Psi}
}
{
\braket{\Psi}{\Psi}
} \\
&=
\frac{
E_n \Abs{1 + \delta \alpha }^2 + 
\bra{\delta \Psi_{n \perp}} H \ket{\delta \Psi_{n \perp}}
}
{
\Abs{1 + \delta \alpha }^2
\braket{\delta \Psi_{n \perp}}{\delta \Psi_{n \perp}} 
} \\
&=
\frac{
E_n 
+ 
\frac{\bra{\delta \Psi_{n \perp}} H \ket{\delta \Psi_{n \perp}} }
{\Abs{1 + \delta \alpha }^2}
}
{
1
+\frac{\braket{\delta \Psi_{n \perp}}{\delta \Psi_{n \perp}} }
{\Abs{1 + \delta \alpha }^2}
} \\
&=
E_n \left( 1 - 
\frac{\braket{\delta \Psi_{n \perp}}{\delta \Psi_{n \perp}} }
{\Abs{1 + \delta \alpha }^2}
+ \cdots \right) + \cdots \\
&=
E_n\left[1 + \mathcal{O}\left((\delta \Psi_{n \perp})^2\right)\right]
\end{align*}

where
\begin{equation}\label{eqn:qmTwoL3:150}
(\delta \Psi_{n \perp})^2
\sim
\braket{\delta \Psi_{n \perp}}{\delta \Psi_{n \perp}}
\end{equation}

\begin{figure}[htp]
\centering
\includegraphics[totalheight=0.4\textheight]{qmTwoL3fig2}
\caption{Illustration of variation of energy with variation of Hamiltonian}\label{fig:qmTwoL3fig2}
\end{figure}
%figure (\ref{fig:qmTwoL3fig2})

``small errors'' in $\ket{\Psi}$ don't lead to large errors in $E[\Psi]$

It is reasonably easy to get a good estimate and $E_0$, although it is reasonably hard to get a good estimate of $\ket{\Psi_0}$.  This is for the same reason, because $E[]$ is not terribly sensitive.

\section{Excited states.}

\begin{align*}
&\vdots \\
E_2 &\sim \ket{\Psi_2} \\
E_1 &\sim \ket{\Psi_1} \\
E_0 &\sim \ket{\Psi_0}
\end{align*}

Suppose we wanted an estimate of $E_1$.  If we knew the ground state $\ket{\Psi_0}$.  For any trial $\ket{\Psi}$ form

\begin{equation}\label{eqn:qmTwoL3:170}
\ket{\Psi'} = 
\ket{\Psi} - 
\ket{\Psi_0}  \braket{\Psi_0}{\Psi}
\end{equation}

We are taking out the projection of the ground state from an arbitrary trial function.

For a state written in terms of the basis states, allowing for an $\alpha$ degeneracy

\begin{equation}\label{eqn:qmTwoL3:190}
\ket{\Psi} = 
c_0 \ket{\Psi_0}  
+
\sum_{n> 0, \alpha} c_{n \alpha} \ket{\Psi_{n \alpha}}
\end{equation}

\begin{equation}\label{eqn:qmTwoL3:210}
\braket{\Psi_0}{\Psi} = 
c_0 
\end{equation}

and

\begin{equation}\label{eqn:qmTwoL3:230}
\ket{\Psi'} = 
\sum_{n> 0, \alpha} c_{n \alpha} \ket{\Psi_{n \alpha}}
\end{equation}

(note that there are some theorems that tell us that the ground state is generally non-degenerate).

\begin{align*}
E[\Psi'] 
&= 
\frac{
\bra{\Psi'} H \ket{\Psi'}
}
{
\braket{\Psi'}{\Psi'}
}  \\
&=
\frac{
\sum_{n> 0, \alpha} \Abs{c_{n \alpha}}^2 E_n
}
{
\sum_{m> 0, \beta} \Abs{c_{m \beta}}^2 
}
\ge E_1
\end{align*}

Often don't know the exact ground state, although we might have a guess $\ket{\tilde{\Psi}_0}$.

for

\begin{equation}\label{eqn:qmTwoL3:250}
\ket{\Psi''} = \ket{\Psi} - 
\ket{\tilde{\Psi}_0}
\braket{\tilde{\Psi}_0}{\Psi}
\end{equation}

but cannot prove that
\begin{equation}\label{eqn:qmTwoL3:270}
\frac{
\bra{\Psi''} H \ket{\Psi''}
}
{
\braket{\Psi''}{\Psi''}
} 
\ge E_1
\end{equation}

%But sometimes, even if you don't know the ground state $\ket{\Psi_0}$, can choose trial kets $\ket{\Psi'''}$ such that $\braket{\Psi_0}{\Psi'''} = 0$.

Then

FIXME: missed something here.

\begin{equation}\label{eqn:qmTwoL3:290}
\frac{
\bra{\Psi'''} H \ket{\Psi'''}
}
{
\braket{\Psi'''}{\Psi'''}
} 
\ge E_1
\end{equation}

Somewhat remarkably, this is often possible.  We talked last time about the Hydrogen atom.  In that case, you can guess that the excited state is in the $2s$ orbital and and therefore orthogonal to the $1s$ (?) orbital.  

\section{Time independent perturbation theory.}

See \S 16.1 of the text \cite{desai2009quantum}.

We can sometimes use this sort of physical insight to help construct a good approximation.  This is provided that we have some of this physical insight, or that it is good insight in the first place.

This is the no-think (turn the crank) approach.

Here we split our Hamiltonian into two parts

\begin{equation}\label{eqn:qmTwoL3:310}
H = H_0 + H'
\end{equation}

where $H_0$ is a Hamiltonian for which we know the energy eigenstates and the eigenkets.  The $H'$ is the ``perturbation'' that is supposed to be small ``in some sense''.

Prof Sipe will provide some references later that provide a more specific meaning to this ``smallness''.  From some ad-hoc discussion in the class it sounds like one has to consider sequences of operators, and look at the convergence of those sequences (is this L2 measure theory?)

\begin{figure}[htp]
\centering
\includegraphics[totalheight=0.4\textheight]{qmTwoL3fig3}
\caption{Example of small perturbation from known Hamiltonian}\label{fig:qmTwoL3fig3}
\end{figure}
%figure (\ref{fig:qmTwoL3fig3})

We'd like to consider a range of problems of the form

\begin{equation}\label{eqn:qmTwoL3:330}
H = H_0 + \lambda H'
\end{equation}

where

\begin{equation}\label{eqn:qmTwoL3:350}
\lambda \in [0,1]
\end{equation}

So that when $\lambda \rightarrow 0$ we have

\begin{equation}\label{eqn:qmTwoL3:370}
H \rightarrow H_0
\end{equation}

the problem that we already know, but for $\lambda \rightarrow 1$ we have

\begin{equation}\label{eqn:qmTwoL3:390}
H = H_0 + H'
\end{equation}

the problem that we'd like to solve.

We are assuming that we know the eigenstates and eigenvalues for $H_0$.  Assuming no degeneracy

\begin{equation}\label{eqn:qmTwoL3:410}
H_0 \ket{\Psi_s^{(0)}} = 
E_s^{(0)}
\ket{\Psi_s^{(0)}} 
\end{equation}

We seek
\begin{equation}\label{eqn:qmTwoL3:430}
(H_0 + H')\ket{\Psi_s} = 
E_s
\ket{\Psi_s} 
\end{equation}

(this is the $\lambda = 1$ case).

Once (if) found, when $\lambda \rightarrow 0$ we will have

\begin{align*}
E_s &\rightarrow E_s^{(0)} \\
\ket{\Psi_s} &\rightarrow \ket{\Psi_s^{(0)}}
\end{align*}

\begin{equation}\label{eqn:qmTwoL3:450}
E_s = E_s^{(0)}  + \lambda E_s^{(1)} + \frac{\lambda^2}{2} E_s^{(2)}
\end{equation}

\begin{equation}\label{eqn:qmTwoL3:470}
\Psi_s = \sum_n c_{ns} \ket{\Psi_n^{(0)}}
\end{equation}

This we know we can do because we are assumed to have a complete set of states.

with
\begin{equation}\label{eqn:qmTwoL3:490}
c_{ns} = c_{ns}^{(0)}  + \lambda c_{ns}^{(1)} + \frac{\lambda^2}{2} c_{ns}^{(2)}
\end{equation}

where

\begin{equation}\label{eqn:qmTwoL3:510}
c_{ns}^{(0)} = \delta_{ns}
\end{equation}

There's a subtlety here that will be treated differently from the text.  We write

\begin{align*}
\ket{\Psi_s}
&=
\ket{\Psi_s^{(0)}}
+ 
\lambda 
\sum_n
c_{ns}^{(1)} 
\ket{\Psi_n^{(0)}}
+ 
\frac{\lambda^2}{2} 
\sum_n
c_{ns}^{(2)}
\ket{\Psi_n^{(0)}}
+ \cdots \\
&=
\left(
1 + \lambda c_{ss}^{(1)} + \cdots
\right)
\ket{\Psi_s^{(0)}}
+ \lambda 
\sum_{n \ne s} c_{ns}^{(1)} 
\ket{\Psi_n^{(0)}}
+ \cdots
\end{align*}

Take
\begin{align*}
\ket{\bar{\Psi}_s}
&=
\ket{\bar{\Psi}_s^{(0)}}
+ 
\lambda
\frac{
\sum_{n \ne s} c_{ns}^{(1)} 
\ket{\Psi_n^{(0)}}
}
{
1 + \lambda c_{ss}^{(1)}
} 
+ \cdots
\\
&=
\ket{\bar{\Psi}_s^{(0)}}
+ 
\lambda
\sum_{n \ne s} \bar{c}_{ns}^{(1)} 
\ket{\Psi_n^{(0)}} + \cdots
\end{align*}

where 

\begin{equation}\label{eqn:qmTwoL3:n}
\bar{c}_{ns}^{(1)}  =
\frac{c_{ns}^{(1)} }
{
1 + \lambda c_{ss}^{(1)}
} 
\end{equation}

We have: 

\begin{align*}
\bar{c}_{ns}^{(1)} &= c_{ns}^{(1)} \\
\bar{c}_{ns}^{(2)} &\ne c_{ns}^{(2)} 
\end{align*}

FIXME: I missed something here.

Note that this is no longer normalized.

\begin{equation}\label{eqn:qmTwoL3:530}
\braket{\bar{\Psi}_s}{\bar{\Psi}_s} \ne 1
\end{equation}

\EndArticle
