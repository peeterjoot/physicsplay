%
% Copyright � 2013 Peeter Joot.  All Rights Reserved.
% Licenced as described in the file LICENSE under the root directory of this GIT repository.
%
\newcommand{\authorname}{Peeter Joot}
\newcommand{\email}{peeterjoot@protonmail.com}
\newcommand{\basename}{FIXMEbasenameUndefined}
\newcommand{\dirname}{notes/FIXMEdirnameUndefined/}

\renewcommand{\basename}{liouvilleOneDim}
\renewcommand{\dirname}{notes/phy452/}
%\newcommand{\dateintitle}{}
\newcommand{\keywords}{Liouville's theorem, Statistical mechanics, PHY452H1S, phase space, phase space current, phase space density}

\newcommand{\authorname}{Peeter Joot}
\newcommand{\onlineurl}{http://sites.google.com/site/peeterjoot2/math2013/\basename.pdf}
\newcommand{\sourcepath}{\dirname\basename.tex}
\newcommand{\generatetitle}[1]{\chapter{#1}}

\newcommand{\vcsinfo}{%
\section*{}
\noindent{\color{DarkOliveGreen}{\rule{\linewidth}{0.1mm}}}
\paragraph{Document version}
%\paragraph{\color{Maroon}{Document version}}
{
\small
\begin{itemize}
\item Available online at:\\ 
\href{\onlineurl}{\onlineurl}
\item Git Repository: \input{./.revinfo/gitRepo.tex}
\item Source: \sourcepath
\item last commit: \input{./.revinfo/gitCommitString.tex}
\item commit date: \input{./.revinfo/gitCommitDate.tex}
\end{itemize}
}
}

%\PassOptionsToPackage{dvipsnames,svgnames}{xcolor}
\PassOptionsToPackage{square,numbers}{natbib}
\documentclass{scrreprt}

\usepackage[left=2cm,right=2cm]{geometry}
\usepackage[svgnames]{xcolor}
\usepackage{peeters_layout}

\usepackage{natbib}

\usepackage[
colorlinks=true,
bookmarks=false,
pdfauthor={\authorname, \email},
backref 
]{hyperref}

% http://tex.stackexchange.com/questions/75773/how-to-reference-problems-by-the-text-label-in-an-exercise-envioronment
\usepackage[english]{cleveref}
\crefname{Exercise}{exercise}{exercises}
\Crefname{Exercise}{Exercise}{Exercises}

\RequirePackage{titlesec}
\RequirePackage{ifthen}

% http://stackoverflow.com/questions/4932910/date-in-the-tabular-environment
\makeatletter
\let\insertdate\@date
\makeatother

\titleformat{\chapter}[display]
{\bfseries\Large}
{\color{DarkSlateGrey}\filleft \authorname
\ifthenelse{\isundefined{\studentnumber}}{}{\\ \studentnumber}
\ifthenelse{\isundefined{\email}}{}{\\ \email}
\ifthenelse{\isundefined{\dateintitle}}{}{\\ \insertdate}
%\ifthenelse{\isundefined{\coursename}}{}{\\ \coursename} % put in title instead.
}
{4ex}
{\color{DarkOliveGreen}{\titlerule}\color{Maroon}
\vspace{2ex}%
\filright}
[\vspace{2ex}%
\color{DarkOliveGreen}\titlerule
]

\newcommand{\beginArtWithToc}[0]{\begin{document}\tableofcontents}
\newcommand{\beginArtNoToc}[0]{\begin{document}}
\newcommand{\EndNoBibArticle}[0]{\end{document}}
\newcommand{\EndArticle}[0]{\bibliography{Bibliography}\bibliographystyle{plainnat}\end{document}}

% 
%\newcommand{\citep}[1]{\cite{#1}}

\colorSectionsForArticle



\beginArtNoToc

\generatetitle{Liouville's theorem questions on density and current}
%\chapter{Liouville's theorem questions on density and current}
\label{chap:liouvilleOneDim}
\section{Motivation}

In the midterm today we were asked to state and prove Liouville's theorem.  I couldn't remember the proof, having only a recollection that it had something to do with the continuity equation

\begin{equation}\label{eqn:liouvilleOneDim:20}
0 = \PD{t}{\rho} + \PD{x}{j},
\end{equation}

but unfortunately couldn't remember what the $j$ was.  Looking up the proof, it's actually really simple, just the application of chain rule for a function $\rho$ that's presumed to be a function of time, position and momentum variables.  It doesn't appear to me that this proof has anything to do with any sort of notion of density, so I'm not really sure where that part comes from.  Despite not being able to answer the question on the spot, it was a good one since it highlighted some areas I have confusions about.

\section{One dimensional proof}

The proof can be distilled to one dimension, removing all the indexes that obfuscate what's being one.  For that case, application of the chain rule to a function $\rho(t, x, p)$, we have

\begin{dmath}\label{eqn:liouvilleOneDim:40}
\ddt{\rho} 
= \PD{t}{\rho} + \PD{t}{x} \PD{x}{\rho} + \PD{t}{p} \PD{p}{\rho}
= \PD{t}{\rho} + \xdot \PD{x}{\rho} + \pdot \PD{p}{\rho}
= \PD{t}{\rho} + \PD{x}{\lr{\xdot \rho}} + \PD{p}{\lr{\xdot \rho}} - \rho \lr{
\PD{x}{\xdot}
+
\PD{p}{\pdot}
}
= \PD{t}{\rho} + \PD{x}{\lr{\xdot \rho}} + \PD{p}{\lr{\xdot \rho}} - \rho 
\mathLabelBox
[
   labelstyle={xshift=2cm},
   linestyle={out=270,in=90, latex-}
]
{
\lr{
\PD{x}{} 
\lr{ \PD{p}{H} }
+
\PD{p}{} 
\lr{ -\PD{x}{H} }
}
}{$= 0$}
\end{dmath}

So, for steady state, where $d\rho/dt = 0$ we have

\begin{equation}\label{eqn:liouvilleOneDim:60}
0 = \PD{t}{\rho} + \PD{x}{\lr{\xdot \rho}} + \PD{p}{\lr{\pdot \rho}}.
\end{equation}

This answers the question of what the current is, it's this tuple

\begin{equation}\label{eqn:liouvilleOneDim:80}
\Bj = \rho (\xdot, \pdot),
\end{equation}

so if we introduce a ``phase space'' gradient

\begin{equation}\label{eqn:liouvilleOneDim:100}
\spacegrad = \lr{ \PD{x}{}, \PD{p}{} }
\end{equation}

we've got something that looks like a continuity equation

\begin{equation}\label{eqn:liouvilleOneDim:120}
0 = \PD{t}{\rho} + \spacegrad \cdot \Bj.
\end{equation}

Two questions.

\begin{enumerate}
\item This function $\rho$ appears to be pretty much arbitrary.  I don't see how this connects to any notion of density?
\item If we pick a specific Hamiltonian, say the 1D SHO, what physical interpretation do we have for this ``current'' $\Bj$?
\end{enumerate}

%\EndArticle
\EndNoBibArticle
