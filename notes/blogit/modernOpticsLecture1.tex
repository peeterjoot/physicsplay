%
% Copyright � 2012 Peeter Joot.  All Rights Reserved.
% Licenced as described in the file LICENSE under the root directory of this GIT repository.
%
\newcommand{\authorname}{Peeter Joot}
\newcommand{\email}{peeterjoot@protonmail.com}
\newcommand{\basename}{FIXMEbasenameUndefined}
\newcommand{\dirname}{notes/FIXMEdirnameUndefined/}

\renewcommand{\basename}{modernOpticsLecture1}
\renewcommand{\dirname}{notes/phy485/}
\newcommand{\keywords}{Optics, PHY485H1F, ABCD matrix, lens-makers formula, transfer matrix}

\newcommand{\authorname}{Peeter Joot}
\newcommand{\onlineurl}{http://sites.google.com/site/peeterjoot2/math2013/\basename.pdf}
\newcommand{\sourcepath}{\dirname\basename.tex}
\newcommand{\generatetitle}[1]{\chapter{#1}}

\newcommand{\vcsinfo}{%
\section*{}
\noindent{\color{DarkOliveGreen}{\rule{\linewidth}{0.1mm}}}
\paragraph{Document version}
%\paragraph{\color{Maroon}{Document version}}
{
\small
\begin{itemize}
\item Available online at:\\ 
\href{\onlineurl}{\onlineurl}
\item Git Repository: \input{./.revinfo/gitRepo.tex}
\item Source: \sourcepath
\item last commit: \input{./.revinfo/gitCommitString.tex}
\item commit date: \input{./.revinfo/gitCommitDate.tex}
\end{itemize}
}
}

%\PassOptionsToPackage{dvipsnames,svgnames}{xcolor}
\PassOptionsToPackage{square,numbers}{natbib}
\documentclass{scrreprt}

\usepackage[left=2cm,right=2cm]{geometry}
\usepackage[svgnames]{xcolor}
\usepackage{peeters_layout}

\usepackage{natbib}

\usepackage[
colorlinks=true,
bookmarks=false,
pdfauthor={\authorname, \email},
backref 
]{hyperref}

% http://tex.stackexchange.com/questions/75773/how-to-reference-problems-by-the-text-label-in-an-exercise-envioronment
\usepackage[english]{cleveref}
\crefname{Exercise}{exercise}{exercises}
\Crefname{Exercise}{Exercise}{Exercises}

\RequirePackage{titlesec}
\RequirePackage{ifthen}

% http://stackoverflow.com/questions/4932910/date-in-the-tabular-environment
\makeatletter
\let\insertdate\@date
\makeatother

\titleformat{\chapter}[display]
{\bfseries\Large}
{\color{DarkSlateGrey}\filleft \authorname
\ifthenelse{\isundefined{\studentnumber}}{}{\\ \studentnumber}
\ifthenelse{\isundefined{\email}}{}{\\ \email}
\ifthenelse{\isundefined{\dateintitle}}{}{\\ \insertdate}
%\ifthenelse{\isundefined{\coursename}}{}{\\ \coursename} % put in title instead.
}
{4ex}
{\color{DarkOliveGreen}{\titlerule}\color{Maroon}
\vspace{2ex}%
\filright}
[\vspace{2ex}%
\color{DarkOliveGreen}\titlerule
]

\newcommand{\beginArtWithToc}[0]{\begin{document}\tableofcontents}
\newcommand{\beginArtNoToc}[0]{\begin{document}}
\newcommand{\EndNoBibArticle}[0]{\end{document}}
\newcommand{\EndArticle}[0]{\bibliography{Bibliography}\bibliographystyle{plainnat}\end{document}}

% 
%\newcommand{\citep}[1]{\cite{#1}}

\colorSectionsForArticle



\beginArtNoToc

\generatetitle{PHY485H1F Modern Optics.  Lecture 1: Matrix methods in Geometric Optics.  Taught by Prof.\ Joseph Thywissen}
\label{chap:modernOpticsLecture1}

\section{Disclaimer}

Peeter's lecture notes from class.  May not be entirely coherent.

\section{Missing content}

I was late.  Snell's law and focal point formulas were covered (including a reference to $1/f = 1/s + 1/s'$ describing the location of the images in some way).  The paraxial approximation was defined (looking only near the center of a lens).

Scrounge for notes to scan from somebody else so I can fill in.  Suggested reading for this lecture is \S 5 from
\cite{hecht1998hecht}.

\section{Matrix methods}

Referring to 

FIXME: F1

we can define a transfer out matrix, or $A B C D$ matrix taking pairs of coordinates describing rays

\begin{dmath}\label{eqn:modernOpticsLecture1:10}
\begin{bmatrix}
y \\
\alpha
\end{bmatrix}
\end{dmath}

so that the transition of the ray through the interface is described as
\begin{dmath}\label{eqn:modernOpticsLecture1:30}
\begin{bmatrix}
y_f \\
\alpha_f
\end{bmatrix}
\begin{bmatrix}
A & B \\
C & D
\end{bmatrix}
\begin{bmatrix}
y_i \\
\alpha_i
\end{bmatrix}
\end{dmath}

\subsection{Free propagation}
Referring to 

FIXME: F2

we see that free propagation is described by

\begin{dmath}\label{eqn:modernOpticsLecture1:50}
\begin{bmatrix}
y_f \\
\alpha_f
\end{bmatrix}
\begin{bmatrix}
1 & L \\
0 & 1
\end{bmatrix}
\begin{bmatrix}
y_i \\
\alpha_i
\end{bmatrix}
\end{dmath}

\subsection{Refraction off of a flat lens}

Referring to 

FIXME: F3

where

\begin{dmath}\label{eqn:modernOpticsLecture1:70}
n \sin\alpha = n' \sin\alpha'.
\end{dmath}

We employ the paraxial approximation

\begin{dmath}\label{eqn:modernOpticsLecture1:90}
n \alpha \sim n' \alpha',
\end{dmath}

or 

\begin{dmath}\label{eqn:modernOpticsLecture1:110}
\alpha' = \frac{n'}{n} \alpha,
\end{dmath}

allowing for the description of refraction off of a flat lens by

\begin{dmath}\label{eqn:modernOpticsLecture1:130}
\begin{bmatrix}
y_f \\
\alpha_f
\end{bmatrix}
\begin{bmatrix}
1 & 0 \\
0 & \frac{n'}{n}
\end{bmatrix}
\begin{bmatrix}
y_i \\
\alpha_i
\end{bmatrix}
\end{dmath}

\subsection{Refraction of a curved surface}

Referring to

FIXME: F4

we see that 

\begin{dmath}\label{eqn:modernOpticsLecture1:150}
\phi \approx \frac{y}{R},
\end{dmath}

and can also employ Snell's law in the approximation

\begin{dmath}\label{eqn:modernOpticsLecture1:370}
n \theta = n' \theta'
\end{dmath}

or
\begin{dmath}\label{eqn:modernOpticsLecture1:390}
n(\alpha + \phi) = n' (\alpha + \phi),
\end{dmath}
FIXME: error in class notes here to fix.  angles shouldn't be the same on both sides should they?

so that

\begin{dmath}\label{eqn:modernOpticsLecture1:170}
\begin{bmatrix}
y_f \\
\alpha_f
\end{bmatrix}
\begin{bmatrix}
1 & 0 \\
\inv{R}\left( \frac{n}{n'} -1 \right) & \frac{n}{n'}
\end{bmatrix}
\begin{bmatrix}
y_i \\
\alpha_i
\end{bmatrix}
\end{dmath}

Observe that for $R \rightarrow \infty$ we have the same result as a flat surface.

FIXME: must have had $n/n'$ scribbled wrong in one of these.

This formula is true for both $R > 0$ and $R < 0$.  FIXME: demonstrate.  We employ the following sign convention

FIXME: F4b

where $-R$ is used for concave, and $+R$ is used for convex.

\subsection{Curvature sign convention}

The sign conventions are illustrated in 

FIXME: F5

and 

FIXME: F5b.

\subsection{$ABCD$ matrix for a lens}

Consider the lens illustrated in 

FIXME: F6

where once again we use the paraxial approximation, assuming we are considering only close enough to the middle that we can neglect any variation in thickness.  In this approximation we have a geometry of the form

FIXME: F7

where $y \ll R_1, R_2$.

Our complete transfer matrix is then given by

\begin{dmath}\label{eqn:modernOpticsLecture1:190}
\begin{bmatrix}
y_f \\
\alpha_f
\end{bmatrix}
M_3 M_2 M_1
\begin{bmatrix}
y_i \\
\alpha_i
\end{bmatrix} 
= M
\begin{bmatrix}
y_i \\
\alpha_i
\end{bmatrix} 
,
\end{dmath}

where $M_1, M_2, M_3$ are described above.  The matrix $M$ is a mess, but for a thin lens can be approximated by

\begin{dmath}\label{eqn:modernOpticsLecture1:210}
M = 
\begin{bmatrix}
1 & 0 \\
- 1/f & 0 
\end{bmatrix}
\end{dmath}

where 

\begin{dmath}\label{eqn:modernOpticsLecture1:230}
\inv{f} = \frac{n' - n}{n} \left( \inv{R_1} - \inv{R_2} \right).
\end{dmath}

This is called the \underline{Lens makers formula}.  This, and $n \approx 1.5$ is enough to construct many lens designs.  Our sign conventions for $f$ are illustrated by

FIXME: F8

where $f > 0$ is convex and $f < 0$ is concave.

\subsection{Properties of the transfer matrix}

\begin{enumerate}
\item The determinant of the transfer matrix is described by index of refraction of just the initial and final media, and $\Det M = n_0/n_f$.

FIXME: demonstrate.

Examples
\begin{enumerate}
\item Thin lens
\begin{dmath}\label{eqn:modernOpticsLecture1:250}
M = 
\begin{bmatrix}
1 & 0 \\
- 1/f & 0 
\end{bmatrix}
\end{dmath}
\item Free propagation
\begin{dmath}\label{eqn:modernOpticsLecture1:270}
M =
\begin{bmatrix}
1 & L \\
0 & 1
\end{bmatrix}
\end{dmath}
\end{enumerate}

In particular, if imaging something in air, where we have $n_0 = n_f$ we have $\Abs{M} = 1$.
\item How about $\Abs{M} = 0$.  There are a couple of cases.  One is $D = 0$ where 

\begin{dmath}\label{eqn:modernOpticsLecture1:290}
\alpha_f = C y_i + \cancel{ D \alpha_i}
\end{dmath}

output $\rightarrow$ input is focus, as illustrated in

FIXME: F9

FIXME: understand this.

\item We also have zero determinant when $A = 0$, in which case we have

\begin{dmath}\label{eqn:modernOpticsLecture1:310}
y_f = \cancel{ A y_i } + B \alpha_i,
\end{dmath}

The output location is only a function of the input angle as illustrated in 

FIXME: F10

\item How about if $B = 0$.  Now we have

\begin{dmath}\label{eqn:modernOpticsLecture1:330}
\alpha_f = \cancel{B y_i} + C \alpha_i,
\end{dmath}

so that we see that the output is an image of the input, but scaled (a magnifier or reducer).  This is illustrated in 

FIXME: F11

\item And finally if $C = 0$ we have

\begin{dmath}\label{eqn:modernOpticsLecture1:350}
\alpha_f = \cancel{C y_i} + D \alpha_i,
\end{dmath}

and we find out system is telescopic, magnifying the angle, as illustrated in

FIXME: F12
\end{enumerate}

%\vcsinfo
\EndArticle
