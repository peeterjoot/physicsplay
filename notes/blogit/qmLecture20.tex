%
% Copyright � 2015 Peeter Joot.  All Rights Reserved.
% Licenced as described in the file LICENSE under the root directory of this GIT repository.
%
\newcommand{\authorname}{Peeter Joot}
\newcommand{\email}{peeterjoot@protonmail.com}
\newcommand{\basename}{FIXMEbasenameUndefined}
\newcommand{\dirname}{notes/FIXMEdirnameUndefined/}

\renewcommand{\basename}{qmLecture20}
\renewcommand{\dirname}{notes/phy1520/}
\newcommand{\keywords}{PHY1520H}
\newcommand{\authorname}{Peeter Joot}
\newcommand{\onlineurl}{http://sites.google.com/site/peeterjoot2/math2013/\basename.pdf}
\newcommand{\sourcepath}{\dirname\basename.tex}
\newcommand{\generatetitle}[1]{\chapter{#1}}

\newcommand{\vcsinfo}{%
\section*{}
\noindent{\color{DarkOliveGreen}{\rule{\linewidth}{0.1mm}}}
\paragraph{Document version}
%\paragraph{\color{Maroon}{Document version}}
{
\small
\begin{itemize}
\item Available online at:\\ 
\href{\onlineurl}{\onlineurl}
\item Git Repository: \input{./.revinfo/gitRepo.tex}
\item Source: \sourcepath
\item last commit: \input{./.revinfo/gitCommitString.tex}
\item commit date: \input{./.revinfo/gitCommitDate.tex}
\end{itemize}
}
}

%\PassOptionsToPackage{dvipsnames,svgnames}{xcolor}
\PassOptionsToPackage{square,numbers}{natbib}
\documentclass{scrreprt}

\usepackage[left=2cm,right=2cm]{geometry}
\usepackage[svgnames]{xcolor}
\usepackage{peeters_layout}

\usepackage{natbib}

\usepackage[
colorlinks=true,
bookmarks=false,
pdfauthor={\authorname, \email},
backref 
]{hyperref}

% http://tex.stackexchange.com/questions/75773/how-to-reference-problems-by-the-text-label-in-an-exercise-envioronment
\usepackage[english]{cleveref}
\crefname{Exercise}{exercise}{exercises}
\Crefname{Exercise}{Exercise}{Exercises}

\RequirePackage{titlesec}
\RequirePackage{ifthen}

% http://stackoverflow.com/questions/4932910/date-in-the-tabular-environment
\makeatletter
\let\insertdate\@date
\makeatother

\titleformat{\chapter}[display]
{\bfseries\Large}
{\color{DarkSlateGrey}\filleft \authorname
\ifthenelse{\isundefined{\studentnumber}}{}{\\ \studentnumber}
\ifthenelse{\isundefined{\email}}{}{\\ \email}
\ifthenelse{\isundefined{\dateintitle}}{}{\\ \insertdate}
%\ifthenelse{\isundefined{\coursename}}{}{\\ \coursename} % put in title instead.
}
{4ex}
{\color{DarkOliveGreen}{\titlerule}\color{Maroon}
\vspace{2ex}%
\filright}
[\vspace{2ex}%
\color{DarkOliveGreen}\titlerule
]

\newcommand{\beginArtWithToc}[0]{\begin{document}\tableofcontents}
\newcommand{\beginArtNoToc}[0]{\begin{document}}
\newcommand{\EndNoBibArticle}[0]{\end{document}}
\newcommand{\EndArticle}[0]{\bibliography{Bibliography}\bibliographystyle{plainnat}\end{document}}

% 
%\newcommand{\citep}[1]{\cite{#1}}

\colorSectionsForArticle



%\usepackage{phy1520}
\usepackage{peeters_braket}
%\usepackage{peeters_layout_exercise}
\usepackage{peeters_figures}
\usepackage{mathtools}

\beginArtNoToc
\generatetitle{PHY1520H Graduate Quantum Mechanics.  Lecture 20: Pertubation theory.  Taught by Prof.\ Arun Paramekanti}
%\chapter{Pertubation theory}
\label{chap:qmLecture20}

\paragraph{Disclaimer}

Peeter's lecture notes from class.  These may be incoherent and rough.

These are notes for the UofT course PHY1520, Graduate Quantum Mechanics, taught by Prof. Paramekanti, covering \textchapref{{5}} \citep{sakurai2014modern} content.

\paragraph{Pertubation theory}

Given a \( 2 \times 2 \) Hamiltonian \( H = H_0 + V \), where

\begin{dmath}\label{eqn:qmLecture20:20}
H = 
\begin{bmatrix}
a & c \\
c^\conj & b
\end{bmatrix}
\end{dmath}

which has eigenvalues

\begin{dmath}\label{eqn:qmLecture20:40}
\lambda_\pm = \frac{a + b}{2} \pm \sqrt{ \lr{ \frac{a - b}{2}}^2 + \Abs{c}^2 }.
\end{dmath}

If \( c = 0 \), 

\begin{dmath}\label{eqn:qmLecture20:60}
H_0 = 
\begin{bmatrix}
a & 0 \\
0 & b
\end{bmatrix},
\end{dmath}

so 

\begin{dmath}\label{eqn:qmLecture20:80}
V = 
\begin{bmatrix}
0 & c \\
c^\conj & 0
\end{bmatrix}.
\end{dmath}

Suppose that \( \Abs{c} \ll \Abs{a - b} \), then

\begin{dmath}\label{eqn:qmLecture20:100}
\lambda_\pm \approx \frac{a + b}{2} \pm \Abs{ \frac{a - b}{2} } \lr{ 1 + 2 \frac{\Abs{c}^2}{\Abs{a - b}^2} }.
\end{dmath}

If  \( a > b \), then 

\begin{dmath}\label{eqn:qmLecture20:120}
\lambda_\pm \approx \frac{a + b}{2} \pm \frac{a - b}{2} \lr{ 1 + 2 \frac{\Abs{c}^2}{\lr{a - b}^2} }.
\end{dmath}

\begin{equation}\label{eqn:qmLecture20:140}
\begin{aligned}
\lambda_{+} &= a + \frac{\Abs{c}^2}{a - b} \\
\lambda_{-} &= b + \frac{\Abs{c}^2}{a - b}.
\end{aligned}
\end{equation}

This adiabatic evolution displays a ``level repulsion'', quadradic in \( \Abs{c} \) as sketched in \cref{fig:lecture20:lecture20Fig1}, and is described as a non-degenerate perbutation.

\imageFigure{../../figures/phy1520/lecture20Fig1}{Adiabatic (non-degenerate) pertubation}{fig:lecture20:lecture20Fig1}{0.2}

If \( \Abs{c} \gg \Abs{a -b} \), then

\begin{dmath}\label{eqn:qmLecture20:160}
\lambda_\pm \approx \frac{a + b}{2} \pm \Abs{c} \pm \frac{\lr{a - b}^2}{4 \Abs{c}} 
\end{dmath}

Here we loose the adiabaticity, and have ``level repulsion'' that is linear in \( \Abs{c} \), as sketched in \cref{fig:lecture20:lecture20Fig2}.  We no longer have the sign of \( a - b \) in the expansion.  This is described as a degenerate perbutation.

\imageFigure{../../figures/phy1520/lecture20Fig2}{Degenerate perbutation}{fig:lecture20:lecture20Fig2}{0.2}

\paragraph{General non-degenerate pertubation}

Given an unperturbed system with

\begin{equation}\label{eqn:qmLecture20:180}
H_0 \ket{n^{(0)}} = E_n^{(0)} \ket{n^{(0)}},
\end{equation}

where the pertubated Hamiltonian equation is
\begin{equation}\label{eqn:qmLecture20:200}
\lr{ H_0 + \lambda V } \ket{ n } = \lr{ E_n^{(0)} + \Delta n } \ket{n}.
\end{equation}

Here \( \Delta n \) is an energy shift as that goes to zero as \( \lambda \rightarrow 0 \).  We can write this has

\begin{equation}\label{eqn:qmLecture20:220}
\lr{ E_n^{(0)} - H_0 } \ket{ n } = \lr{ \lambda V - \Delta_n } \ket{n}.
\end{equation}

We are hoping to iterate with application of the inverse to an initial estimate of \( \ket{n} \)

\begin{equation}\label{eqn:qmLecture20:240}
\ket{n} = \lr{ E_n^{(0)} - H_0 }^{-1} \lr{ \lambda V - \Delta_n } \ket{n}.
\end{equation}

This gets us into trouble if \( \lambda \rightarrow 0 \), which can be fixed by using

\begin{equation}\label{eqn:qmLecture20:260}
\ket{n} = \lr{ E_n^{(0)} - H_0 }^{-1} \lr{ \lambda V - \Delta_n } \ket{n} + \ket{ n^{(0)} },
\end{equation}

which is also a solution to \cref{eqn:qmLecture20:220}.  We want to ask if 

\begin{equation}\label{eqn:qmLecture20:280}
\lr{ \lambda V - \Delta_n } \ket{n} ,
\end{equation}

contain a bit of \( \ket{ n^{(0)} } \)?  To determine this act with \( \bra{n^{(0)}} \) on the left

\begin{dmath}\label{eqn:qmLecture20:300}
\bra{ n^{(0)} } \lr{ \lambda V - \Delta_n } \ket{n}
=
\bra{ n^{(0)} } \lr{ E_n^{(0)} - H_0 } \ket{n}
= 
\lr{ E_n^{(0)} - E_n^{(0)} } \braket{n^{(0)}}{n}
=
0.
\end{dmath}

We see that \( \ket{n} \) is entirely orthogonal to \( \ket{n^{(0)}} \).

Define a projection operator

\begin{dmath}\label{eqn:qmLecture20:320}
P_n = \ket{n^{(0)}}\bra{n^{(0)}},
\end{dmath}

which has the idempotent property \( P_n^2 = P_n \) that we expect of a projection operator.

Define a rejection operator
\begin{dmath}\label{eqn:qmLecture20:340}
\overbar{P}_n 
= 1 - 
\ket{n^{(0)}}\bra{n^{(0)}}
= \sum_{m \ne n} 
\ket{m^{(0)}}\bra{m^{(0)}}.
\end{dmath}

This allow us to write

\begin{dmath}\label{eqn:qmLecture20:360}
\ket{n} = \lr{ E_n^{(0)} - H_0 }^{-1} \overbar{P}_n \lr{ \lambda V - \Delta_n } \ket{n} + \ket{ n^{(0)} }.
\end{dmath}

If we suppose that the unknown state and unknown energy difference operator can be expanded in a \( \lambda \) power series, say

\begin{dmath}\label{eqn:qmLecture20:380}
\ket{n} = \ket{n_0} 
+ \lambda \ket{n_1} 
+ \lambda^2 \ket{n_2} 
+ \lambda^3 \ket{n_3} + \cdots
\end{dmath}

and

\begin{dmath}\label{eqn:qmLecture20:400}
\Delta_{n} = \Delta_{n_0} 
+ \lambda \Delta_{n_1} 
+ \lambda^2 \Delta_{n_2} 
+ \lambda^3 \Delta_{n_3} + \cdots
\end{dmath}

We usually interpret functions of operators in terms of power series expansions.  In the case of \( \lr{ E_n^{(0)} - H_0 }^{-1} \), we have a concrete interpretation when acting on one of the unpertubed eigenstates

\begin{dmath}\label{eqn:qmLecture20:420}
\inv{ E_n^{(0)} - H_0 } \ket{m^{(0)}} = 
\inv{ E_n^{(0)} - E_m^0 } \ket{m^{(0)}}.
\end{dmath}

This gives

\begin{dmath}\label{eqn:qmLecture20:440}
\ket{n} 
= 
\inv{ E_n^{(0)} - H_0 }
\sum_{m \ne n} 
\ket{m^{(0)}}\bra{m^{(0)}}
 \lr{ \lambda V - \Delta_n } \ket{n} + \ket{ n^{(0)} },
\end{dmath}

or

%\begin{equation}\label{eqn:qmLecture20:460}
\boxedEquation{eqn:qmLecture20:480}{
\ket{n} 
=
 \ket{ n^{(0)} }
+
\sum_{m \ne n} 
\frac{\ket{m^{(0)}}\bra{m^{(0)}}}
{
E_n^{(0)} - E_m^{(0)}
}
 \lr{ \lambda V - \Delta_n } \ket{n}.
}
%\end{equation}

Note that

%\begin{equation}\label{eqn:qmLecture20:500}
\boxedEquation{eqn:qmLecture20:520}{
\Delta_n = \bra{n^{(0)}} \lambda V \ket{n}
}
%\end{equation}

\paragraph{to \( O(\lambda^0) \) }

\begin{dmath}\label{eqn:qmLecture20:540}
\begin{aligned}
\ket{n_0} &= \ket{n^{(0)}} \\
\Delta_{n} &= 0.
\end{aligned}
\end{dmath}

\paragraph{to \( O(\lambda^1) \) }

\begin{dmath}\label{eqn:qmLecture20:560}
\ket{n_1} 
\ket{n_1} 
= 
\sum_{m \ne n} 
\frac{
\ket{m^{(0)}} \bra{ m^{(0)}}
}
{
E_n^{(0)} - E_m^{(0)}
}
\lr{ V - \Delta_{n_1} } \ket{n_0}
\end{dmath}

So

\begin{equation}\label{eqn:qmLecture20:580}
\begin{aligned}
\ket{n_1}
&= 
\sum_{m \ne n} 
\frac{
\ket{m^{(0)}} 
}
{
E_n^{(0)} - E_m^{(0)}
}
\bra{ m^{(0)}} V \ket{n_0} \\
\Delta_{n_1} &= \bra{ n^{(0)} } V \ket{ n^0}
\end{aligned}
\end{equation}

Writing 

\begin{dmath}\label{eqn:qmLecture20:600}
V_{m n} = \bra{ m^{(0)}} V \ket{n_0} ,
\end{dmath}

\paragraph{to \( O(\lambda^2) \) }

The second order pertubation states is

\begin{dmath}\label{eqn:qmLecture20:620}
\begin{aligned}
\ket{n_2}
&= 
\sum_{l,m \ne n} 
\ket{m^{(0)}}
\frac{V_{m l} V_{l n}}
{
\lr{ E_n^{(0)} - E_m^{(0)} }
\lr{ E_n^{(0)} - E_l^{(0)} }
}
-
\sum_{m \ne n}
\ket{m^{(0)}}
\frac{V_{n n} V_{m n}}
{
\lr{ E_n^{(0)} - E_m^{(0)} }^2
} \\
\Delta_{n_2} &= 
\sum_{m \ne n} \frac{V_{n m} V_{m n} }{E_n^{(0)} - E_m^{(0)}}
\end{aligned}
\end{dmath}

\paragraph{to \( O(\lambda^3) \) }

\begin{dmath}\label{eqn:qmLecture20:640}
\Delta_{n_3} = 
\sum_{l, m \ne n} \frac{V_{n m} V_{m l} V_{l n} }{
\lr{ E_n^{(0)} - E_m^{(0)} }
\lr{ E_n^{(0)} - E_l^{(0)} }
}
-
\sum_{ m \ne n} \frac{V_{n m} V_{n n} V_{m n} }{
\lr{ E_n^{(0)} - E_m^{(0)} }^2
}
\end{dmath}

In general 

\begin{dmath}\label{eqn:qmLecture20:660}
\Delta_n^{(l)} = \bra{n^{(0)}} V \ket{n^{(l-1)}}.
\end{dmath}

\EndArticle
