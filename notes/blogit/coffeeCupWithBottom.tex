%
%
%
% Copyright � 2012 Peeter Joot
% All Rights Reserved
%
% This file may be reproduced and distributed in whole or in part, without fee, subject to the following conditions:
%
% o The copyright notice above and this permission notice must be preserved complete on all complete or partial copies.
%
% o Any translation or derived work must be approved by the author in writing before distribution.
%
% o If you distribute this work in part, instructions for obtaining the complete version of this file must be included, and a means for obtaining a complete version provided.
%
%
% Exceptions to these rules may be granted for academic purposes: Write to the author and ask.
%
%
%
%
% Copyright � 2015 Peeter Joot.  All Rights Reserved.
% Licenced as described in the file LICENSE under the root directory of this GIT repository.
%
\documentclass[]{eliblog}

\usepackage{amsmath}
\usepackage{mathpazo}

%
% shorthand for bold symbols, convenient for vectors and matrices
%
\newcommand{\Ba}[0]{\mathbf{a}}
\newcommand{\Bb}[0]{\mathbf{b}}
\newcommand{\Bc}[0]{\mathbf{c}}
\newcommand{\Bd}[0]{\mathbf{d}}
\newcommand{\Be}[0]{\mathbf{e}}
\newcommand{\Bf}[0]{\mathbf{f}}
\newcommand{\Bg}[0]{\mathbf{g}}
\newcommand{\Bh}[0]{\mathbf{h}}
\newcommand{\Bi}[0]{\mathbf{i}}
\newcommand{\Bj}[0]{\mathbf{j}}
\newcommand{\Bk}[0]{\mathbf{k}}
\newcommand{\Bl}[0]{\mathbf{l}}
\newcommand{\Bm}[0]{\mathbf{m}}
\newcommand{\Bn}[0]{\mathbf{n}}
\newcommand{\Bo}[0]{\mathbf{o}}
\newcommand{\Bp}[0]{\mathbf{p}}
\newcommand{\Bq}[0]{\mathbf{q}}
\newcommand{\Br}[0]{\mathbf{r}}
\newcommand{\Bs}[0]{\mathbf{s}}
\newcommand{\Bt}[0]{\mathbf{t}}
\newcommand{\Bu}[0]{\mathbf{u}}
\newcommand{\Bv}[0]{\mathbf{v}}
\newcommand{\Bw}[0]{\mathbf{w}}
\newcommand{\Bx}[0]{\mathbf{x}}
\newcommand{\By}[0]{\mathbf{y}}
\newcommand{\Bz}[0]{\mathbf{z}}
\newcommand{\BA}[0]{\mathbf{A}}
\newcommand{\BB}[0]{\mathbf{B}}
\newcommand{\BC}[0]{\mathbf{C}}
\newcommand{\BD}[0]{\mathbf{D}}
\newcommand{\BE}[0]{\mathbf{E}}
\newcommand{\BF}[0]{\mathbf{F}}
\newcommand{\BG}[0]{\mathbf{G}}
\newcommand{\BH}[0]{\mathbf{H}}
\newcommand{\BI}[0]{\mathbf{I}}
\newcommand{\BJ}[0]{\mathbf{J}}
\newcommand{\BK}[0]{\mathbf{K}}
\newcommand{\BL}[0]{\mathbf{L}}
\newcommand{\BM}[0]{\mathbf{M}}
\newcommand{\BN}[0]{\mathbf{N}}
\newcommand{\BO}[0]{\mathbf{O}}
\newcommand{\BP}[0]{\mathbf{P}}
\newcommand{\BQ}[0]{\mathbf{Q}}
\newcommand{\BR}[0]{\mathbf{R}}
\newcommand{\BS}[0]{\mathbf{S}}
\newcommand{\BT}[0]{\mathbf{T}}
\newcommand{\BU}[0]{\mathbf{U}}
\newcommand{\BV}[0]{\mathbf{V}}
\newcommand{\BW}[0]{\mathbf{W}}
\newcommand{\BX}[0]{\mathbf{X}}
\newcommand{\BY}[0]{\mathbf{Y}}
\newcommand{\BZ}[0]{\mathbf{Z}}

\newcommand{\Bzero}[0]{\mathbf{0}}
\newcommand{\Btheta}[0]{\boldsymbol{\theta}}
\newcommand{\Btau}[0]{\boldsymbol{\tau}}
\newcommand{\Bomega}[0]{\boldsymbol{\omega}}

%
% shorthand for unit vectors
%
\newcommand{\acap}[0]{\hat{\Ba}}
\newcommand{\bcap}[0]{\hat{\Bb}}
\newcommand{\ccap}[0]{\hat{\Bc}}
\newcommand{\dcap}[0]{\hat{\Bd}}
\newcommand{\ecap}[0]{\hat{\Be}}
\newcommand{\fcap}[0]{\hat{\Bf}}
\newcommand{\gcap}[0]{\hat{\Bg}}
\newcommand{\hcap}[0]{\hat{\Bh}}
\newcommand{\icap}[0]{\hat{\Bi}}
\newcommand{\jcap}[0]{\hat{\Bj}}
\newcommand{\kcap}[0]{\hat{\Bk}}
\newcommand{\lcap}[0]{\hat{\Bl}}
\newcommand{\mcap}[0]{\hat{\Bm}}
\newcommand{\ncap}[0]{\hat{\Bn}}
\newcommand{\ocap}[0]{\hat{\Bo}}
\newcommand{\pcap}[0]{\hat{\Bp}}
\newcommand{\qcap}[0]{\hat{\Bq}}
\newcommand{\rcap}[0]{\hat{\Br}}
\newcommand{\scap}[0]{\hat{\Bs}}
\newcommand{\tcap}[0]{\hat{\Bt}}
\newcommand{\ucap}[0]{\hat{\Bu}}
\newcommand{\vcap}[0]{\hat{\Bv}}
\newcommand{\wcap}[0]{\hat{\Bw}}
\newcommand{\xcap}[0]{\hat{\Bx}}
\newcommand{\ycap}[0]{\hat{\By}}
\newcommand{\zcap}[0]{\hat{\Bz}}
\newcommand{\thetacap}[0]{\hat{\Btheta}}

%
% to write R^n and C^n in a distinguishable fashion.  Perhaps change this
% to the double lined characters upon figuring out how to do so.
%
\newcommand{\C}[1]{$\mathbb{C}^{#1}$}
\newcommand{\R}[1]{$\mathbb{R}^{#1}$}

%
% various generally useful helpers
%

% derivative of #1 wrt. #2:
\newcommand{\D}[2] {\frac {d#2} {d#1}}

\newcommand{\inv}[1]{\frac{1}{#1}}
\newcommand{\cross}[0]{\times}

\newcommand{\abs}[1]{\lvert{#1}\rvert}
\newcommand{\norm}[1]{\lVert{#1}\rVert}
\newcommand{\innerprod}[2]{\langle{#1}, {#2}\rangle}
\newcommand{\dotprod}[2]{{#1} \cdot {#2}}
\newcommand{\bdotprod}[2]{\left({#1} \cdot {#2}\right)}
\newcommand{\crossprod}[2]{{#1} \cross {#2}}
\newcommand{\tripleprod}[3]{\dotprod{\left(\crossprod{#1}{#2}\right)}{#3}}

\DeclareMathOperator{\Proj}{Proj}
\DeclareMathOperator{\Span}{span}
\DeclareMathOperator{\Sgn}{sgn}
\DeclareMathOperator{\Area}{Area}
\DeclareMathOperator{\Volume}{Volume}

%
% A few miscellaneous things specific to this document
%
\newcommand{\crossop}[1]{\crossprod{#1}{}}

% R2 vector.
\newcommand{\VectorTwo}[2]{
\begin{bmatrix}
 {#1} \\
 {#2}
\end{bmatrix}
}

\newcommand{\VectorN}[1]{
\begin{bmatrix}
{#1}_1 \\
{#1}_2 \\
\vdots \\
{#1}_N \\
\end{bmatrix}
}

\newcommand{\DETuvij}[4]{
\begin{vmatrix}
 {#1}_{#3} & {#1}_{#4} \\
 {#2}_{#3} & {#2}_{#4}
\end{vmatrix}
}

\newcommand{\DETuvwijk}[6]{
\begin{vmatrix}
 {#1}_{#4} & {#1}_{#5} & {#1}_{#6} \\
 {#2}_{#4} & {#2}_{#5} & {#2}_{#6} \\
 {#3}_{#4} & {#3}_{#5} & {#3}_{#6}
\end{vmatrix}
}

\newcommand{\DETuvwxijkl}[8]{
\begin{vmatrix}
 {#1}_{#5} & {#1}_{#6} & {#1}_{#7} & {#1}_{#8} \\
 {#2}_{#5} & {#2}_{#6} & {#2}_{#7} & {#2}_{#8} \\
 {#3}_{#5} & {#3}_{#6} & {#3}_{#7} & {#3}_{#8} \\
 {#4}_{#5} & {#4}_{#6} & {#4}_{#7} & {#4}_{#8} \\
\end{vmatrix}
}

%\newcommand{\DETuvwxyijklm}[10]{
%\begin{vmatrix}
% {#1}_{#6} & {#1}_{#7} & {#1}_{#8} & {#1}_{#9} & {#1}_{#10} \\
% {#2}_{#6} & {#2}_{#7} & {#2}_{#8} & {#2}_{#9} & {#2}_{#10} \\
% {#3}_{#6} & {#3}_{#7} & {#3}_{#8} & {#3}_{#9} & {#3}_{#10} \\
% {#4}_{#6} & {#4}_{#7} & {#4}_{#8} & {#4}_{#9} & {#4}_{#10} \\
% {#5}_{#6} & {#5}_{#7} & {#5}_{#8} & {#5}_{#9} & {#5}_{#10}
%\end{vmatrix}
%}

% R3 vector.
\newcommand{\VectorThree}[3]{
\begin{bmatrix}
 {#1} \\
 {#2} \\
 {#3}
\end{bmatrix}
}



\author{Peeter Joot}
\email{peeter.joot@gmail.com}

%\documentclass[]{eliblogwidescreen}

\usepackage{amsmath}
\usepackage{mathpazo}

%
% shorthand for bold symbols, convenient for vectors and matrices
%
\newcommand{\Ba}[0]{\mathbf{a}}
\newcommand{\Bb}[0]{\mathbf{b}}
\newcommand{\Bc}[0]{\mathbf{c}}
\newcommand{\Bd}[0]{\mathbf{d}}
\newcommand{\Be}[0]{\mathbf{e}}
\newcommand{\Bf}[0]{\mathbf{f}}
\newcommand{\Bg}[0]{\mathbf{g}}
\newcommand{\Bh}[0]{\mathbf{h}}
\newcommand{\Bi}[0]{\mathbf{i}}
\newcommand{\Bj}[0]{\mathbf{j}}
\newcommand{\Bk}[0]{\mathbf{k}}
\newcommand{\Bl}[0]{\mathbf{l}}
\newcommand{\Bm}[0]{\mathbf{m}}
\newcommand{\Bn}[0]{\mathbf{n}}
\newcommand{\Bo}[0]{\mathbf{o}}
\newcommand{\Bp}[0]{\mathbf{p}}
\newcommand{\Bq}[0]{\mathbf{q}}
\newcommand{\Br}[0]{\mathbf{r}}
\newcommand{\Bs}[0]{\mathbf{s}}
\newcommand{\Bt}[0]{\mathbf{t}}
\newcommand{\Bu}[0]{\mathbf{u}}
\newcommand{\Bv}[0]{\mathbf{v}}
\newcommand{\Bw}[0]{\mathbf{w}}
\newcommand{\Bx}[0]{\mathbf{x}}
\newcommand{\By}[0]{\mathbf{y}}
\newcommand{\Bz}[0]{\mathbf{z}}
\newcommand{\BA}[0]{\mathbf{A}}
\newcommand{\BB}[0]{\mathbf{B}}
\newcommand{\BC}[0]{\mathbf{C}}
\newcommand{\BD}[0]{\mathbf{D}}
\newcommand{\BE}[0]{\mathbf{E}}
\newcommand{\BF}[0]{\mathbf{F}}
\newcommand{\BG}[0]{\mathbf{G}}
\newcommand{\BH}[0]{\mathbf{H}}
\newcommand{\BI}[0]{\mathbf{I}}
\newcommand{\BJ}[0]{\mathbf{J}}
\newcommand{\BK}[0]{\mathbf{K}}
\newcommand{\BL}[0]{\mathbf{L}}
\newcommand{\BM}[0]{\mathbf{M}}
\newcommand{\BN}[0]{\mathbf{N}}
\newcommand{\BO}[0]{\mathbf{O}}
\newcommand{\BP}[0]{\mathbf{P}}
\newcommand{\BQ}[0]{\mathbf{Q}}
\newcommand{\BR}[0]{\mathbf{R}}
\newcommand{\BS}[0]{\mathbf{S}}
\newcommand{\BT}[0]{\mathbf{T}}
\newcommand{\BU}[0]{\mathbf{U}}
\newcommand{\BV}[0]{\mathbf{V}}
\newcommand{\BW}[0]{\mathbf{W}}
\newcommand{\BX}[0]{\mathbf{X}}
\newcommand{\BY}[0]{\mathbf{Y}}
\newcommand{\BZ}[0]{\mathbf{Z}}

\newcommand{\Bzero}[0]{\mathbf{0}}
\newcommand{\Btheta}[0]{\boldsymbol{\theta}}
\newcommand{\Btau}[0]{\boldsymbol{\tau}}
\newcommand{\Bomega}[0]{\boldsymbol{\omega}}

%
% shorthand for unit vectors
%
\newcommand{\acap}[0]{\hat{\Ba}}
\newcommand{\bcap}[0]{\hat{\Bb}}
\newcommand{\ccap}[0]{\hat{\Bc}}
\newcommand{\dcap}[0]{\hat{\Bd}}
\newcommand{\ecap}[0]{\hat{\Be}}
\newcommand{\fcap}[0]{\hat{\Bf}}
\newcommand{\gcap}[0]{\hat{\Bg}}
\newcommand{\hcap}[0]{\hat{\Bh}}
\newcommand{\icap}[0]{\hat{\Bi}}
\newcommand{\jcap}[0]{\hat{\Bj}}
\newcommand{\kcap}[0]{\hat{\Bk}}
\newcommand{\lcap}[0]{\hat{\Bl}}
\newcommand{\mcap}[0]{\hat{\Bm}}
\newcommand{\ncap}[0]{\hat{\Bn}}
\newcommand{\ocap}[0]{\hat{\Bo}}
\newcommand{\pcap}[0]{\hat{\Bp}}
\newcommand{\qcap}[0]{\hat{\Bq}}
\newcommand{\rcap}[0]{\hat{\Br}}
\newcommand{\scap}[0]{\hat{\Bs}}
\newcommand{\tcap}[0]{\hat{\Bt}}
\newcommand{\ucap}[0]{\hat{\Bu}}
\newcommand{\vcap}[0]{\hat{\Bv}}
\newcommand{\wcap}[0]{\hat{\Bw}}
\newcommand{\xcap}[0]{\hat{\Bx}}
\newcommand{\ycap}[0]{\hat{\By}}
\newcommand{\zcap}[0]{\hat{\Bz}}
\newcommand{\thetacap}[0]{\hat{\Btheta}}

%
% to write R^n and C^n in a distinguishable fashion.  Perhaps change this
% to the double lined characters upon figuring out how to do so.
%
\newcommand{\C}[1]{$\mathbb{C}^{#1}$}
\newcommand{\R}[1]{$\mathbb{R}^{#1}$}

%
% various generally useful helpers
%

% derivative of #1 wrt. #2:
\newcommand{\D}[2] {\frac {d#2} {d#1}}

\newcommand{\inv}[1]{\frac{1}{#1}}
\newcommand{\cross}[0]{\times}

\newcommand{\abs}[1]{\lvert{#1}\rvert}
\newcommand{\norm}[1]{\lVert{#1}\rVert}
\newcommand{\innerprod}[2]{\langle{#1}, {#2}\rangle}
\newcommand{\dotprod}[2]{{#1} \cdot {#2}}
\newcommand{\bdotprod}[2]{\left({#1} \cdot {#2}\right)}
\newcommand{\crossprod}[2]{{#1} \cross {#2}}
\newcommand{\tripleprod}[3]{\dotprod{\left(\crossprod{#1}{#2}\right)}{#3}}

\DeclareMathOperator{\Proj}{Proj}
\DeclareMathOperator{\Span}{span}
\DeclareMathOperator{\Sgn}{sgn}
\DeclareMathOperator{\Area}{Area}
\DeclareMathOperator{\Volume}{Volume}

%
% A few miscellaneous things specific to this document
%
\newcommand{\crossop}[1]{\crossprod{#1}{}}

% R2 vector.
\newcommand{\VectorTwo}[2]{
\begin{bmatrix}
 {#1} \\
 {#2}
\end{bmatrix}
}

\newcommand{\VectorN}[1]{
\begin{bmatrix}
{#1}_1 \\
{#1}_2 \\
\vdots \\
{#1}_N \\
\end{bmatrix}
}

\newcommand{\DETuvij}[4]{
\begin{vmatrix}
 {#1}_{#3} & {#1}_{#4} \\
 {#2}_{#3} & {#2}_{#4}
\end{vmatrix}
}

\newcommand{\DETuvwijk}[6]{
\begin{vmatrix}
 {#1}_{#4} & {#1}_{#5} & {#1}_{#6} \\
 {#2}_{#4} & {#2}_{#5} & {#2}_{#6} \\
 {#3}_{#4} & {#3}_{#5} & {#3}_{#6}
\end{vmatrix}
}

\newcommand{\DETuvwxijkl}[8]{
\begin{vmatrix}
 {#1}_{#5} & {#1}_{#6} & {#1}_{#7} & {#1}_{#8} \\
 {#2}_{#5} & {#2}_{#6} & {#2}_{#7} & {#2}_{#8} \\
 {#3}_{#5} & {#3}_{#6} & {#3}_{#7} & {#3}_{#8} \\
 {#4}_{#5} & {#4}_{#6} & {#4}_{#7} & {#4}_{#8} \\
\end{vmatrix}
}

%\newcommand{\DETuvwxyijklm}[10]{
%\begin{vmatrix}
% {#1}_{#6} & {#1}_{#7} & {#1}_{#8} & {#1}_{#9} & {#1}_{#10} \\
% {#2}_{#6} & {#2}_{#7} & {#2}_{#8} & {#2}_{#9} & {#2}_{#10} \\
% {#3}_{#6} & {#3}_{#7} & {#3}_{#8} & {#3}_{#9} & {#3}_{#10} \\
% {#4}_{#6} & {#4}_{#7} & {#4}_{#8} & {#4}_{#9} & {#4}_{#10} \\
% {#5}_{#6} & {#5}_{#7} & {#5}_{#8} & {#5}_{#9} & {#5}_{#10}
%\end{vmatrix}
%}

% R3 vector.
\newcommand{\VectorThree}[3]{
\begin{bmatrix}
 {#1} \\
 {#2} \\
 {#3}
\end{bmatrix}
}



\author{Peeter Joot}
\email{peeter.joot@gmail.com}


%\usepackage[english]{babel}
%\usepackage{media9}
\chapter{Steady state velocity profile of stirred cup of non-bottomless coffee.}

\label{chap:coffeeCupWithBottom}
%\useCCL
\blogpage{http://sites.google.com/site/peeterjoot2/math2012/coffeeCupWithBottom.pdf}
\date{Apr 29, 2012}
\gitRevisionInfo{coffeeCupWithBottom}
\keywords{Navier-Stokes, PHY454H1S, PHY454H1}

\beginArtWithToc
%\beginArtNoToc
%\wordpresscategory{}

\section{Motivation.}

Having tackled the problem of the spin down of a bottomless cup of coffee, lets try the harder problem with a non-infinite cup.

\section{Navier-Stokes for the problem.}

Working in cylindrical coordinates is the only sensible option.  Let's look first at the steady state problem, with the cup being stirred at at constant rate with the unrealistic rotating cylinder  stir stick used previously.  We'll assume an azimuthal velocity profile

\begin{equation}\label{eqn:coffeeCupWithBottom:10}
\Bu = u(r, z, t) \phicap.
\end{equation}

Let's first verify that this satisfies our incompressible condition

\begin{equation}\label{eqn:coffeeCupWithBottom:30}
\begin{aligned}
\spacegrad \cdot \Bu
&=
\left( \rcap \partial_r + \frac{\phicap}{r} \partial_\phi + \zcap \partial_z \right) \cdot ( u \phicap ) \\
&=
\cancel{\rcap \cdot \phicap} \partial_r u 
+ \frac{\phicap}{r} \cdot (u \partial_\phi \phicap + \phicap \cancel{\partial_\phi u})
+ \cancel{\zcap \cdot \phicap} \partial_z u.
\end{aligned}
\end{equation}

The only term that survives is the $\partial_\phi \phicap = -\rcap$ but that's perpendicular to $\phicap$ so we have $0 = \spacegrad \cdot \Bu$ as desired.  This leaves us with

\begin{equation}\label{eqn:coffeeCupWithBottom:50}
\frac{u}{r} \partial_\phi ( u \phicap )
= -\inv{\rho} 
\left( \rcap \partial_r + \frac{\phicap}{r} \partial_\phi + \zcap \partial_z \right) p
+ \nu 
\left( \inv{r} \partial_r ( r \partial_r ) + \frac{1}{r^2} \partial_{\phi \phi} + \partial_{zz} \right) u \phicap + \Bg
\end{equation}

Splitting into $\phicap, \rcap, \zcap$ coordinates respectively we have

\begin{subequations}
\begin{equation}\label{eqn:coffeeCupWithBottom:70}
0 = -\inv{r \rho} \partial_\phi p + \nu \left( \inv{r} \partial_r ( r \partial_r u ) - \frac{u}{r^2} + \partial_{zz} u \right) 
\end{equation}
\begin{equation}\label{eqn:coffeeCupWithBottom:90}
- \frac{u^2}{r} = -\inv{\rho} \partial_r p 
\end{equation}
\begin{equation}\label{eqn:coffeeCupWithBottom:110}
0 = -\inv{\rho} \partial_z p - g.
\end{equation}
\end{subequations}

Demanding a symmetrical solution kills off the pressure term in \ref{eqn:coffeeCupWithBottom:70}, and we can proceed with separation of variables to find the allowable solutions for the velocity before imposing our boundary value conditions.  With $u = R(r) Z(z)$ we have

\begin{equation}\label{eqn:coffeeCupWithBottom:130}
\inv{r R} ( R' + r R'' ) - \frac{1}{r^2} = -\frac{Z''}{Z} = -\inv{a^2} = \inv{b^2}
\end{equation}

Should we pick a negative or a positive constant here (ie: trig or hyperbolic functions for $Z$)?  Allowing for both temporarily, we find for $R$

\begin{subequations}
\begin{equation}\label{eqn:coffeeCupWithBottom:150}
0 = r R' + r^2 R'' + R \left( -1 + \frac{r^2}{a^2} \right) 
\end{equation}
\begin{equation}\label{eqn:coffeeCupWithBottom:170}
0 = r R' + r^2 R'' + R \left( -1 - \frac{r^2}{b^2} \right).
\end{equation}
\end{subequations}

Using Mathematica (\href{https://raw.github.com/peeterjoot/physicsplay/master/notes/phy454/coffeeCupWithBottom.cdf}{phy454/coffeeCupWithBottom.cdf}) to look up the solutions, we find that this was a Bessel equation with solutions

\begin{subequations}
\begin{equation}\label{eqn:coffeeCupWithBottom:190}
R(r) \in \Span \{ J_1(r/a), Y_1(r/a) \} 
\end{equation}
\begin{equation}\label{eqn:coffeeCupWithBottom:210}
R(r) \in \Span \{ J_1(i r/b), Y_1(i r/b) \}.
\end{equation}
\end{subequations}

However, the second set are not real valued for $r > 0$.  This means we want the hyperbolic solutions for $Z$.  Because we also have a boundary value constrain of $u = 0$ on the bottom of the cup, we can pick only hyperbolic sine solutions.  As before, we'll ``stir'' the coffee with a cylinder at radius $s$ (with cup radius $R$), our solution has the form

\begin{equation}\label{eqn:coffeeCupWithBottom:230}
u(r, z, 0) = 
\left\{
\begin{array}{l l}
s \Omega
\sum 
C_a J_1(r/a) \sinh(z/a)
& \quad \mbox{$r < s$} \\
s \Omega
\sum 
\left( D_a J_1(r/a) + E_a Y_1(r/a) \right) \sinh(z/a)
 & \quad \mbox{$r \in [s, R]$} \\
\end{array}
\right.
\end{equation}

This is clearly going to be an easier problem to solve, if we stir the coffee on the outside edge of the cup only ($s = R$).  Lets start with the inside region and see how far we get.  We've got to match a $u(s, z, 0) = s \Omega$ boundary value constraint, where $\Omega$ is the angular velocity of the stirring.  We know how to solve that, since we had a similar problem for the $t = 0$ boundary value constraint in the bottomless spin down problem.  Writing $\lambda_i/s = 1/a$ where $\lambda_i$ are the zeros of the $J_1$ we have

\begin{equation}\label{eqn:coffeeCupWithBottom:250}
1 = \sum_i C_i J_1(\lambda_i r/s) \sinh( \lambda_i z/s)
\end{equation}

With solution

\begin{equation}\label{eqn:coffeeCupWithBottom:270}
C_i = 
\frac{
\int_0^1 w J_1 (\lambda_i w) dw
}{
\int_0^1 w J_1^2 (\lambda_i w) dw
}.
\end{equation}

We are left with only the boundary value problem on the surface to deal with, and have found in the inner portion of the stirred cup that we have

\begin{equation}\label{eqn:coffeeCupWithBottom:290}
u(r, z, 0) = 
s \Omega
\sum 
\frac{
\int_0^1 w J_1 (\lambda_i w) dw
}{
\int_0^1 w J_1^2 (\lambda_i w) dw
}
J_1( \lambda_i r/s) \sinh(\lambda_i z/s).
\end{equation}

FIXME: this is wrong.  We have a different $C_i$ for each value of $z$.  Need to build in another degree of freedom, and method of solution above does not work.

%\EndArticle
\EndNoBibArticle
