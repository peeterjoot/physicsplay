%
% Copyright � 2015 Peeter Joot.  All Rights Reserved.
% Licenced as described in the file LICENSE under the root directory of this GIT repository.
%
\newcommand{\authorname}{Peeter Joot}
\newcommand{\email}{peeterjoot@protonmail.com}
\newcommand{\basename}{FIXMEbasenameUndefined}
\newcommand{\dirname}{notes/FIXMEdirnameUndefined/}

\renewcommand{\basename}{reciprocityTheorem}
\renewcommand{\dirname}{notes/ece1229/}
%\newcommand{\dateintitle}{}
%\newcommand{\keywords}{}

\newcommand{\authorname}{Peeter Joot}
\newcommand{\onlineurl}{http://sites.google.com/site/peeterjoot2/math2013/\basename.pdf}
\newcommand{\sourcepath}{\dirname\basename.tex}
\newcommand{\generatetitle}[1]{\chapter{#1}}

\newcommand{\vcsinfo}{%
\section*{}
\noindent{\color{DarkOliveGreen}{\rule{\linewidth}{0.1mm}}}
\paragraph{Document version}
%\paragraph{\color{Maroon}{Document version}}
{
\small
\begin{itemize}
\item Available online at:\\ 
\href{\onlineurl}{\onlineurl}
\item Git Repository: \input{./.revinfo/gitRepo.tex}
\item Source: \sourcepath
\item last commit: \input{./.revinfo/gitCommitString.tex}
\item commit date: \input{./.revinfo/gitCommitDate.tex}
\end{itemize}
}
}

%\PassOptionsToPackage{dvipsnames,svgnames}{xcolor}
\PassOptionsToPackage{square,numbers}{natbib}
\documentclass{scrreprt}

\usepackage[left=2cm,right=2cm]{geometry}
\usepackage[svgnames]{xcolor}
\usepackage{peeters_layout}

\usepackage{natbib}

\usepackage[
colorlinks=true,
bookmarks=false,
pdfauthor={\authorname, \email},
backref 
]{hyperref}

% http://tex.stackexchange.com/questions/75773/how-to-reference-problems-by-the-text-label-in-an-exercise-envioronment
\usepackage[english]{cleveref}
\crefname{Exercise}{exercise}{exercises}
\Crefname{Exercise}{Exercise}{Exercises}

\RequirePackage{titlesec}
\RequirePackage{ifthen}

% http://stackoverflow.com/questions/4932910/date-in-the-tabular-environment
\makeatletter
\let\insertdate\@date
\makeatother

\titleformat{\chapter}[display]
{\bfseries\Large}
{\color{DarkSlateGrey}\filleft \authorname
\ifthenelse{\isundefined{\studentnumber}}{}{\\ \studentnumber}
\ifthenelse{\isundefined{\email}}{}{\\ \email}
\ifthenelse{\isundefined{\dateintitle}}{}{\\ \insertdate}
%\ifthenelse{\isundefined{\coursename}}{}{\\ \coursename} % put in title instead.
}
{4ex}
{\color{DarkOliveGreen}{\titlerule}\color{Maroon}
\vspace{2ex}%
\filright}
[\vspace{2ex}%
\color{DarkOliveGreen}\titlerule
]

\newcommand{\beginArtWithToc}[0]{\begin{document}\tableofcontents}
\newcommand{\beginArtNoToc}[0]{\begin{document}}
\newcommand{\EndNoBibArticle}[0]{\end{document}}
\newcommand{\EndArticle}[0]{\bibliography{Bibliography}\bibliographystyle{plainnat}\end{document}}

% 
%\newcommand{\citep}[1]{\cite{#1}}

\colorSectionsForArticle


\usepackage{peeters_layout_exercise}
\usepackage{macros_bm}

\beginArtNoToc

\generatetitle{Reciprocity theorem}
%\chapter{Reciprocity theorem}
%\label{chap:reciprocityTheorem}
%\section{Motivation}
%\section{Guts}

The class slides presented a derivation of the \textAndIndex{reciprocity theorem}, a theorem that contained the integral of

\begin{equation}\label{eqn:reciprocityTheorem:360}
\int \lr{ \BE^{(a)} \cross \BH^{(b)} - \BE^{(b)} \cross \BH^{(a)} } \cdot d\BS = \cdots
\end{equation}

over a surface, where the RHS was a volume integral involving the fields and (electric and magnetic) current sources.  
The idea was to consider two different source loading configurations of the same system, and to show that the fields and sources in the two configurations can be related.

To derive the result in question, a simple way to start is to look at the divergence of the difference of cross products above.  This will require the phasor form of the two cross product Maxwell's equations

\begin{subequations}
\label{eqn:reciprocityTheorem:99}
\begin{dmath}\label{eqn:reciprocityTheorem:100}
\spacegrad \cross \BE = - (\BM + j \omega \mu_0 \BH) % \BM^{(a)} + j \omega \mu_0 \BH^{(a)}
\end{dmath}
\begin{dmath}\label{eqn:reciprocityTheorem:120}
\spacegrad \cross \BH = \BJ + j \omega \epsilon_0 \BE, % \BJ^{(a)} + j \omega \epsilon_0 \BE^{(a)}
\end{dmath}
\end{subequations}

so the divergence is

\begin{dmath}\label{eqn:reciprocityTheorem:380}
\begin{aligned}
\spacegrad \cdot 
\lr{ \BE^{(a)} \cross \BH^{(b)} - \BE^{(b)} \cross \BH^{(a)} }
&=
\BH^{(b)} \cdot \lr{ \spacegrad \cross \BE^{(a)} } -\BE^{(a)} \cdot \lr{ \spacegrad \cross \BH^{(b)} } \\
&-\BH^{(a)} \cdot \lr{ \spacegrad \cross \BE^{(b)} } +\BE^{(b)} \cdot \lr{ \spacegrad \cross \BH^{(a)} } \\
&=
-\BH^{(b)} \cdot \lr{ \BM^{(a)} + j \omega \mu_0 \BH^{(a)} } -\BE^{(a)} \cdot \lr{ \BJ^{(b)} + j \omega \epsilon_0 \BE^{(b)} } \\
&+\BH^{(a)} \cdot \lr{ \BM^{(b)} + j \omega \mu_0 \BH^{(b)} } +\BE^{(b)} \cdot \lr{ \BJ^{(a)} + j \omega \epsilon_0 \BE^{(a)} }.
\end{aligned}
\end{dmath}

The non-source terms cancel, leaving

%\begin{dmath}\label{eqn:reciprocityTheorem:440}
\boxedEquation{eqn:reciprocityTheorem:440}{
\spacegrad \cdot 
\lr{ \BE^{(a)} \cross \BH^{(b)} - \BE^{(b)} \cross \BH^{(a)} }
=
-\BH^{(b)} \cdot \BM^{(a)} -\BE^{(a)} \cdot \BJ^{(b)} 
+\BH^{(a)} \cdot \BM^{(b)} +\BE^{(b)} \cdot \BJ^{(a)} 
}
%\end{dmath}

Should we be suprised to have a relation of this form?  Probably not, given that the energy momentum relationship between the fields and currents of a single source has the form

\begin{equation}\label{eqn:reciprocityTheorem:600}
\PD{t}{}\frac{\epsilon_0}{2} \left(\bcE^2 + c^2 \bcB^2\right) + \spacegrad \cdot \inv{\mu_0}(\bcE \cross \bcB) = -\bcE \cdot \bcJ.
\end{equation}

(this is without magnetic sources).

This initially suggests that the reciprocity theorem can be expressed more generally in terms of the energy-momentum tensor.  
However, there are some subtle differences since the time domain products lead to averages in terms of the real parts of conjugate pairs such as \( \bcE \cross \bcB \rightarrow \BE \cross \BB^\conj \), and \( \bcE \cdot \bcJ \rightarrow \BE \cdot \BJ^\conj \).

\section{far field integral form}
Employing the divergence theorem over a sphere the identity above takes the form

\begin{dmath}\label{eqn:reciprocityTheorem:480}
\int_S
\lr{ \BE^{(a)} \cross \BH^{(b)} - \BE^{(b)} \cross \BH^{(a)} } \cdot \rcap dS
=
\int_V \lr{
-\BH^{(b)} \cdot \BM^{(a)} -\BE^{(a)} \cdot \BJ^{(b)} 
+\BH^{(a)} \cdot \BM^{(b)} +\BE^{(b)} \cdot \BJ^{(a)} 
}
dV
\end{dmath}

In the far field, the cross products are strictly radial.  That surface integral can be written as

\begin{dmath}\label{eqn:reciprocityTheorem:500}
\int_S
\lr{ \BE^{(a)} \cross \BH^{(b)} - \BE^{(b)} \cross \BH^{(a)} } \cdot \rcap dS
=
\inv{\mu_0}
\int_S
\lr{ \BE^{(a)} \cross \lr{ \rcap \cross \BE^{(b)}} - \BE^{(b)} \cross \lr{ \rcap \cross \BE^{(a)}} } \cdot \rcap dS
=
\inv{\mu_0}
\int_S
\lr{ \BE^{(a)} \cdot \BE^{(b)} - \BE^{(b)} \cdot \BE^{(a)}
}
dS
= 0
\end{dmath}

The above expansions used \cref{eqn:reciprocityTheorem:540} to expand the terms of the form

\begin{dmath}\label{eqn:reciprocityTheorem:560}
\lr{ \BA \cross \lr{ \rcap \cross \BC } } \cdot \rcap 
= \BA \cdot \BC -\lr{ \BA \cdot \rcap } \lr{ \BC \cdot \rcap },
\end{dmath}

in which only the first dot product survives due to the transverse nature of the fields.

So in the far field we have a direct relation between the fields and sources of two source configurations of the same system of the form

\boxedEquation{eqn:reciprocityTheorem:580}{
%\begin{dmath}\label{eqn:reciprocityTheorem:580}
%\boxed{
\int_V \lr{
\BH^{(a)} \cdot \BM^{(b)} +\BE^{(b)} \cdot \BJ^{(a)} 
}
dV
=
\int_V \lr{
\BH^{(b)} \cdot \BM^{(a)} +\BE^{(a)} \cdot \BJ^{(b)} 
}
dV
%}
%\end{dmath}
}

\section{Application to antenna}

This is the underlying reason that we are able to pose the problem of what an antenna can recieve, in terms of what the antenna can transmit.

FIXME: More on that to come.  Describe what Prof. Eleftheriades did in terms that I understand.

\section{Identities}

\makelemma{Divergence of a cross product}{thm:polarizationReview:400}{
\begin{equation*}
\spacegrad \cdot \lr{ \BA \cross \BB } = 
\BB \lr{\spacegrad \cross \BA}
-\BA \lr{\spacegrad \cross \BB}.
\end{equation*}
}

Proof.

\begin{dmath}\label{eqn:reciprocityTheorem:420}
\spacegrad \cdot \lr{ \BA \cross \BB } 
= 
\partial_a \epsilon_{a b c} A_b B_c
= 
\epsilon_{a b c} (\partial_a A_b )B_c
-
\epsilon_{b a c} A_b (\partial_a B_c)
=
\BB \cdot (\spacegrad \cross \BA)
-\BB \cdot (\spacegrad \cross \BA).
\end{dmath}

\makelemma{Triple cross product dotted}{thm:polarizationReview:520}{
\begin{equation*}
\lr{ \BA \cross \lr{ \BB \cross \BC } } \cdot \BD
=
\lr{ \BA \cdot \BC } \lr{ \BB \cdot \BD }
-\lr{ \BA \cdot \BB } \lr{ \BC \cdot \BD }
\end{equation*}
}

Proof.

\begin{dmath}\label{eqn:reciprocityTheorem:540}
\lr{ \BA \cross \lr{ \BB \cross \BC } } \cdot \BD
=
\epsilon_{a b c} A_b \epsilon_{r s c } B_r C_s D_a
=
\delta_{[a b]}^{r s}
A_b B_r C_s D_a
=
A_s B_r C_s D_r
-A_r B_r C_s D_s
= 
\lr{ \BA \cdot \BC } \lr{ \BB \cdot \BD }
-\lr{ \BA \cdot \BB } \lr{ \BC \cdot \BD }.
\end{dmath}

%\EndArticle
\EndNoBibArticle
