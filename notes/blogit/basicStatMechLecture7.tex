%
% Copyright � 2013 Peeter Joot.  All Rights Reserved.
% Licenced as described in the file LICENSE under the root directory of this GIT repository.
%
\newcommand{\authorname}{Peeter Joot}
\newcommand{\email}{peeterjoot@protonmail.com}
\newcommand{\basename}{FIXMEbasenameUndefined}
\newcommand{\dirname}{notes/FIXMEdirnameUndefined/}

\renewcommand{\basename}{basicStatMechLecture7}
\renewcommand{\dirname}{notes/phy452/}
\newcommand{\keywords}{Statistical mechanics, PHY452H1S}
\newcommand{\authorname}{Peeter Joot}
\newcommand{\onlineurl}{http://sites.google.com/site/peeterjoot2/math2013/\basename.pdf}
\newcommand{\sourcepath}{\dirname\basename.tex}
\newcommand{\generatetitle}[1]{\chapter{#1}}

\newcommand{\vcsinfo}{%
\section*{}
\noindent{\color{DarkOliveGreen}{\rule{\linewidth}{0.1mm}}}
\paragraph{Document version}
%\paragraph{\color{Maroon}{Document version}}
{
\small
\begin{itemize}
\item Available online at:\\ 
\href{\onlineurl}{\onlineurl}
\item Git Repository: \input{./.revinfo/gitRepo.tex}
\item Source: \sourcepath
\item last commit: \input{./.revinfo/gitCommitString.tex}
\item commit date: \input{./.revinfo/gitCommitDate.tex}
\end{itemize}
}
}

%\PassOptionsToPackage{dvipsnames,svgnames}{xcolor}
\PassOptionsToPackage{square,numbers}{natbib}
\documentclass{scrreprt}

\usepackage[left=2cm,right=2cm]{geometry}
\usepackage[svgnames]{xcolor}
\usepackage{peeters_layout}

\usepackage{natbib}

\usepackage[
colorlinks=true,
bookmarks=false,
pdfauthor={\authorname, \email},
backref 
]{hyperref}

% http://tex.stackexchange.com/questions/75773/how-to-reference-problems-by-the-text-label-in-an-exercise-envioronment
\usepackage[english]{cleveref}
\crefname{Exercise}{exercise}{exercises}
\Crefname{Exercise}{Exercise}{Exercises}

\RequirePackage{titlesec}
\RequirePackage{ifthen}

% http://stackoverflow.com/questions/4932910/date-in-the-tabular-environment
\makeatletter
\let\insertdate\@date
\makeatother

\titleformat{\chapter}[display]
{\bfseries\Large}
{\color{DarkSlateGrey}\filleft \authorname
\ifthenelse{\isundefined{\studentnumber}}{}{\\ \studentnumber}
\ifthenelse{\isundefined{\email}}{}{\\ \email}
\ifthenelse{\isundefined{\dateintitle}}{}{\\ \insertdate}
%\ifthenelse{\isundefined{\coursename}}{}{\\ \coursename} % put in title instead.
}
{4ex}
{\color{DarkOliveGreen}{\titlerule}\color{Maroon}
\vspace{2ex}%
\filright}
[\vspace{2ex}%
\color{DarkOliveGreen}\titlerule
]

\newcommand{\beginArtWithToc}[0]{\begin{document}\tableofcontents}
\newcommand{\beginArtNoToc}[0]{\begin{document}}
\newcommand{\EndNoBibArticle}[0]{\end{document}}
\newcommand{\EndArticle}[0]{\bibliography{Bibliography}\bibliographystyle{plainnat}\end{document}}

% 
%\newcommand{\citep}[1]{\cite{#1}}

\colorSectionsForArticle



\beginArtNoToc
\generatetitle{PHY452H1S Basic Statistical Mechanics.  Lecture 7: Ideal gas and SHO phase space volume calculations.  Taught by Prof.\ Arun Paramekanti}
%\chapter{Ideal gas and SHO phase space volume calculations}
\label{chap:basicStatMechLecture7}

\section{Disclaimer}

Peeter's lecture notes from class.  May not be entirely coherent.

\section{Review.  Classical phase space calculation}

\begin{equation}\label{eqn:basicStatMechLecture7:20}
E_{\mathrm{ideal}} = \sum_i \frac{\Bp_i^2}{2 m}
\end{equation}

From this we calculated $\gamma(E)$, and

\begin{equation}\label{eqn:basicStatMechLecture7:40}
\frac{d\gamma(E)}{dE} = \Omega_{\mathrm{classical}}(E)
\end{equation}

Fudging with a requirement that $\Delta x \Delta p \sim h$, we corrected this as

\begin{equation}\label{eqn:basicStatMechLecture7:60}
\Omega_{\mathrm{quantum}}(E) = \frac{\Omega_{\mathrm{classical}}(E)}{N! h^{3N}}
\end{equation}

Now let's do the quantum calculation.

\paragraph{Quantum calculation}

Recall that for the solutions of the Quantum free particle in a box, as in \cref{fig:lecture7OneDSHOInBox:lecture7OneDSHOInBoxFig1}, our solutions are

\imageFigure{lecture7OneDSHOInBoxFig1}{1D Quantum free particle in a box}{fig:lecture7OneDSHOInBox:lecture7OneDSHOInBoxFig1}{0.3}

\begin{equation}\label{eqn:basicStatMechLecture7:80}
\Psi_n(x) = \sqrt{2}{L} \sin\lr{ \frac{ n \pi x}{L} },
\end{equation}

where $n = 1, 2, \cdots$, and

\begin{equation}\label{eqn:basicStatMechLecture7:100}
\epsilon_n = \frac{n^2 h^2}{8 m L^2}
\end{equation}.

In three dimensions, with $n_i = 1, 2, \cdots$ we have

\begin{equation}\label{eqn:basicStatMechLecture7:120}
\Psi_{n_1, n_2, n_3}(x, y, z) = \lr{\frac{2}{L}}^{3/2} 
\sin\lr{ \frac{ n_1 \pi x}{L} }
\sin\lr{ \frac{ n_2 \pi x}{L} }
\sin\lr{ \frac{ n_3 \pi x}{L} }
\end{equation}

and 

\begin{equation}\label{eqn:basicStatMechLecture7:140}
\epsilon_{n_1, n_2, n_3} = \frac{h^2}{8 m L^2} \lr{ n_1^2 + n_2^2 + n_3^2 }
\end{equation}

\begin{dmath}\label{eqn:basicStatMechLecture7:160}
\gamma^{3d}_{\mathrm{classical}}(E) = 
\mathLabelBox
[
   labelstyle={xshift=0.5cm},
   linestyle={out=270,in=90, latex-}
]
{V}{$L^3$}
\int d^3 p \theta \lr{ E - \frac{\Bp^2}{2m} }
= V \frac{4 \pi}{3} (2 m E)^{3/2}
\end{dmath}

so that 

\begin{equation}\label{eqn:basicStatMechLecture7:180}
\gamma^{3d}_{\mathrm{corrected}}(E) 
= V \frac{4 \pi}{3} \frac{(2 m E)^{3/2}}{h^3}
\end{equation}

\begin{equation}\label{eqn:basicStatMechLecture7:200}
\gamma^{3d}_{\mathrm{quantum}}(E) 
= 
\sum_{n_1, n_2, n_3} \Theta(E - \epsilon_{n_1, n_2, n_3} ).
\end{equation}

How do the multiplicities scale by energy?  We'll have expect something like \cref{fig:lecture7ThreeDSHOInBoxNumStates:lecture7ThreeDSHOInBoxNumStatesFig2}.

\imageFigure{lecture7ThreeDSHOInBoxNumStatesFig2}{Multiplicities for free quantum particle in a 3D box}{fig:lecture7ThreeDSHOInBoxNumStates:lecture7ThreeDSHOInBoxNumStatesFig2}{0.3}

Provided the energies $E \gg 3h^2/(8 m L)$ are large enough, we can approximate the sum with

\begin{equation}\label{eqn:basicStatMechLecture7:220}
\sum_{n_1, n_2, n_3} \sim \int_0^\infty dn_1 dn_2 dn_3
\end{equation}

So

\begin{dmath}\label{eqn:basicStatMechLecture7:240}
\gamma^{3d}_{\mathrm{quantum}} \lr{ E \gg \frac{h^2}{8 m L^2} } 
\approx
\int_0^\infty dn_1 dn_2 dn_3 \Theta \lr{
E - \frac{h^2}{8 m L^2} \lr{ n_1^2 + n_2^2 + n_3^2 }
}
=
\inv{8}
\frac{4 \pi}{3} \lr{
\frac{8 m L^2 E}{h^2}
}^{3/2}
=
L^3
\frac{4 \pi}{3} 
\frac{
\lr{2 m E}^{3/2}
}
{h^3}
\end{dmath}

\paragraph{Harmonic oscillator in 1D.}

Our phase space is of the form \cref{fig:lecture71DclassicalSHOphasespace:lecture71DclassicalSHOphasespaceFig3}.

\imageFigure{lecture71DclassicalSHOphasespaceFig3}{1D classical SHO phase space}{fig:lecture71DclassicalSHOphasespace:lecture71DclassicalSHOphasespaceFig3}{0.3}

Where the number of states in this classical picture are found with:

\begin{equation}\label{eqn:basicStatMechLecture7:260}
\gamma^{\mathrm{classical}}(E) 
= \int dx dp \theta\lr{ E - \lr{\inv{2} k x^2 + \inv{2m } p^2 }}.
\end{equation}

Rescale

\begin{subequations}
\begin{equation}\label{eqn:basicStatMechLecture7:280}
\tilde{x} = x \sqrt{ \frac{k}{2}}
\end{equation}
\begin{equation}\label{eqn:basicStatMechLecture7:300}
\tilde{p} = \frac{p}{\sqrt{2m}}
\end{equation}
\end{subequations}

so that we have

\begin{equation}\label{eqn:basicStatMechLecture7:320}
\gamma^{\mathrm{classical}}(E) 
= \int d\tilde{x} d \tilde{p} \sqrt{\frac{2 \times 2 m}{k}} \theta\lr{ E - \tilde{x}^2 - \tilde{p}^2 }
=
2 \sqrt{\frac{m}{k}} \pi E
= 2 \pi \sqrt{\frac{m}{k}} E.
\end{equation}

\begin{equation}\label{eqn:basicStatMechLecture7:340}
\gamma^{\mathrm{SHO}}_{\mathrm{corrected}}(E)  = 2 \pi \sqrt{\frac{m}{k}} \frac{E}{h}.
\end{equation}

How about the quantum calculation?

We have for the energy

\begin{subequations}
\begin{equation}\label{eqn:basicStatMechLecture7:360}
E_n^{\mathrm{SHO}} = \lr{ n + \inv{2} } \hbar \omega
\end{equation}
\begin{equation}\label{eqn:basicStatMechLecture7:380}
\omega = \sqrt{\frac{k}{m}}
\end{equation}
\begin{equation}\label{eqn:basicStatMechLecture7:400}
\hbar = \frac{h}{2 \pi}
\end{equation}
\end{subequations}

graphing the counts \cref{fig:lecture71DquantumSHOStatesPerEnergyLevel:lecture71DquantumSHOStatesPerEnergyLevelFig4}, we again have stepping as a function of energy, but no multiplicities this time

\imageFigure{lecture71DquantumSHOStatesPerEnergyLevelFig4}{1D quantum SHO states per energy level}{fig:lecture71DquantumSHOStatesPerEnergyLevel:lecture71DquantumSHOStatesPerEnergyLevelFig4}{0.3}

\begin{equation}\label{eqn:basicStatMechLecture7:420}
\gamma_{\mathrm{quantum}}(E) 
= \sum_{n = 0}^\infty \Theta\lr{ E - \lr{n + \inv{2} \hbar \omega}}
\end{equation}

we make the continuous approximation for the summation again, and throwing away the zero point energy, we have

\begin{equation}\label{eqn:basicStatMechLecture7:440}
\gamma_{\mathrm{quantum}}(E \gg \hbar \omega) 
\approx
\int_{0}^\infty dn \Theta\lr{ E - n \hbar \omega }
= 2 \pi \frac{E}{h} \sqrt{\frac{m}{k}}
\end{equation}

\paragraph{Why $N!$?}

We have a problem with out counting here.  Consider some particles in a box as in \cref{fig:lecture7threeParticlesInABox:lecture7threeParticlesInABoxFig5}.

\imageFigure{lecture7threeParticlesInABoxFig5}{Three particles in a box}{fig:lecture7threeParticlesInABox:lecture7threeParticlesInABoxFig5}{0.3}

\begin{itemize}
\item particle $1$ at $\Bx_1$
\item particle $2$ at $\Bx_2$
\item particle $3$ at $\Bx_3$
\end{itemize}

or
\begin{itemize}
\item particle $1$ at $\Bx_2$
\item particle $2$ at $\Bx_3$
\item particle $3$ at $\Bx_1$
\end{itemize}

This is fine in the classical picture, but in the quantum picture with an assumption of indistinguishability, no two particles (say electrons) cannot be labelled in this fashion.

\paragraphAndIndex{Gibbs paradox}

\begin{dmath}\label{eqn:basicStatMechLecture7:460}
\mathLabelBox
[
   labelstyle={xshift=2cm},
   linestyle={out=270,in=90, latex-}
]
{
S_{\mathrm{ideal}}^{(\mathrm{E}, \mathrm{N}, \mathrm{V})}
}{Statistical entropy}
= k_{\mathrm{B}} \ln \lr{ \frac{\Omega_{\mathrm{classical}}}{h^{3N}} }
\mathLabelBox
[
   labelstyle={below of=m\themathLableNode, below of=m\themathLableNode}
]
{\approx}{$N \gg 1$}
 k_{\mathrm{B}} \lr{
N \ln V + \frac{3 N}{2} \ln \lr{ \frac{4 \pi m E }{3 N h^2} } + \frac{3 N}{2} 
}
\end{dmath}

Suppose we double the volume as in \cref{fig:lecture7gibbsVolumeDoubling:lecture7gibbsVolumeDoublingFig6}, then our total entropy for the bigger system would be

\imageFigure{lecture7gibbsVolumeDoublingFig6}{Gibbs volume doubling argument.  Two identical systems allowed to mix}{fig:lecture7gibbsVolumeDoubling:lecture7gibbsVolumeDoublingFig6}{0.3}

\begin{dmath}\label{eqn:basicStatMechLecture7:480}
S_{\mathrm{total}}^{(\mathrm{E}, \mathrm{N}, \mathrm{V})}
= k_{\mathrm{B}} \ln \lr{ \frac{\Omega_{\mathrm{classical}}}{h^{3N}} }
\approx
 k_{\mathrm{B}} \lr{
(2 N) \ln (2 V) + \frac{3 (2 N)}{2} \ln \lr{ \frac{4 \pi m (2 E) }{2 ( 2 N) h^2} } + \frac{3 (2 N)}{2} 
}.
\end{dmath}

We have

\begin{equation}\label{eqn:basicStatMechLecture7:500}
S_{\mathrm{total}} = S_1 + S_2 + k_{\mathrm{B}} (2 N) \ln 2
= S_1 + S_2 + k_{\mathrm{B}} \ln 2^{2 N}.
\end{equation}

This is telling us that each particle could be in either the left or the right side, but we know that this uncertainty shouldn't be in the final answer.  We must drop this $k_{\mathrm{B}}$ term.

So, if we assume that these particles are identical, and divide $\Omega$ by $N!$, then we find

\begin{equation}\label{eqn:basicStatMechLecture7:520}
S_{\mathrm{ideal}} = 
k_{\mathrm{B}} 
\lr{
N \ln \frac{V}{N} + \frac{3 N}{2} \ln \lr{ \frac{4 \pi m E }{3 N h^2} } + \frac{5 N}{2} 
}
\end{equation}

\EndNoBibArticle
