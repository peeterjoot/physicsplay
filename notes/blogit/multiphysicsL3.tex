%
% Copyright � 2014 Peeter Joot.  All Rights Reserved.
% Licenced as described in the file LICENSE under the root directory of this GIT repository.
%
% for template copy, run:
%
% ~/bin/ct multiphysicsL1  multiphysicsLN tl1
%
\newcommand{\authorname}{Peeter Joot}
\newcommand{\email}{peeterjoot@protonmail.com}
\newcommand{\basename}{FIXMEbasenameUndefined}
\newcommand{\dirname}{notes/FIXMEdirnameUndefined/}

\renewcommand{\basename}{multiphysicsL3}
\renewcommand{\dirname}{notes/ece1254/}
\newcommand{\keywords}{Condensed matter physics, ECE1254H}
\newcommand{\authorname}{Peeter Joot}
\newcommand{\onlineurl}{http://sites.google.com/site/peeterjoot2/math2013/\basename.pdf}
\newcommand{\sourcepath}{\dirname\basename.tex}
\newcommand{\generatetitle}[1]{\chapter{#1}}

\newcommand{\vcsinfo}{%
\section*{}
\noindent{\color{DarkOliveGreen}{\rule{\linewidth}{0.1mm}}}
\paragraph{Document version}
%\paragraph{\color{Maroon}{Document version}}
{
\small
\begin{itemize}
\item Available online at:\\ 
\href{\onlineurl}{\onlineurl}
\item Git Repository: \input{./.revinfo/gitRepo.tex}
\item Source: \sourcepath
\item last commit: \input{./.revinfo/gitCommitString.tex}
\item commit date: \input{./.revinfo/gitCommitDate.tex}
\end{itemize}
}
}

%\PassOptionsToPackage{dvipsnames,svgnames}{xcolor}
\PassOptionsToPackage{square,numbers}{natbib}
\documentclass{scrreprt}

\usepackage[left=2cm,right=2cm]{geometry}
\usepackage[svgnames]{xcolor}
\usepackage{peeters_layout}

\usepackage{natbib}

\usepackage[
colorlinks=true,
bookmarks=false,
pdfauthor={\authorname, \email},
backref 
]{hyperref}

% http://tex.stackexchange.com/questions/75773/how-to-reference-problems-by-the-text-label-in-an-exercise-envioronment
\usepackage[english]{cleveref}
\crefname{Exercise}{exercise}{exercises}
\Crefname{Exercise}{Exercise}{Exercises}

\RequirePackage{titlesec}
\RequirePackage{ifthen}

% http://stackoverflow.com/questions/4932910/date-in-the-tabular-environment
\makeatletter
\let\insertdate\@date
\makeatother

\titleformat{\chapter}[display]
{\bfseries\Large}
{\color{DarkSlateGrey}\filleft \authorname
\ifthenelse{\isundefined{\studentnumber}}{}{\\ \studentnumber}
\ifthenelse{\isundefined{\email}}{}{\\ \email}
\ifthenelse{\isundefined{\dateintitle}}{}{\\ \insertdate}
%\ifthenelse{\isundefined{\coursename}}{}{\\ \coursename} % put in title instead.
}
{4ex}
{\color{DarkOliveGreen}{\titlerule}\color{Maroon}
\vspace{2ex}%
\filright}
[\vspace{2ex}%
\color{DarkOliveGreen}\titlerule
]

\newcommand{\beginArtWithToc}[0]{\begin{document}\tableofcontents}
\newcommand{\beginArtNoToc}[0]{\begin{document}}
\newcommand{\EndNoBibArticle}[0]{\end{document}}
\newcommand{\EndArticle}[0]{\bibliography{Bibliography}\bibliographystyle{plainnat}\end{document}}

% 
%\newcommand{\citep}[1]{\cite{#1}}

\colorSectionsForArticle



\beginArtNoToc
\generatetitle{ECE1254H Modeling of Multiphysics Systems.  Lecture 3: Nodal Analysis.  Taught by Prof.\ Piero Triverio}
%\chapter{Nodal Analysis}
\label{chap:multiphysicsL3}

\section{Disclaimer}

Peeter's lecture notes from class.  These may be incoherent and rough.

\section{Nodal Analysis}

We want to consider the same circuit \cref{fig:lecture3:lecture3Fig1}, this time considering only the node voltages.  We'd like to avoid introducing branch currents to reduce the scope of the computational problem.

\imageFigure{../../figures/ece1254/lecture3Fig1}{Resistive circuit with current sources}{fig:lecture3:lecture3Fig1}{0.3}

Unknowns: node voltages: \(V_1, V_2, \cdots V_4\)

Equations are KCL at each node except \(0\).

FIXME: verify signs.

\begin{enumerate}
\item 
		\( 
		\frac{V_1 - 0}{R_A} +
		\frac{V_1 - V_2}{R_B} = -i_{S,A}
		\)
\item 
		\( 
		\frac{V_2 - 0}{R_E} +
		\frac{V_2 - V_1}{R_B} = -i_{S,B} - i_{S,C}
		\)
\item 
		\( 
		\frac{V_3 - V_4}{R_C} = i_{S,C}
		\)
\item 
		\( 
		\frac{V_4 - 0}{R_D} 
		+\frac{V_4 - V_3}{R_C} 
		= i_{S,A} + i_{S,B}
		\)
\end{enumerate}

In matrix form this is

\begin{equation}\label{eqn:multiphysicsL3:20}
\begin{bmatrix}
	\inv{R_A} + \inv{R_B} & - \inv{R_B} & 0 & 0 \\
	-\inv{R_B} & \inv{R_B} + \inv{R_E} & 0 & 0 \\
	0 & 0 & \inv{R_C} & -\inv{R_C} \\
	0 & 0 & -\inv{R_C} & \inv{R_C} + \inv{R_D}
\end{bmatrix}
\begin{bmatrix}
	V_1 \\
	V_2 \\
	V_3 \\
	V_4 \\
\end{bmatrix}
=
\begin{bmatrix}
	-i_{S,A} \\
-i_{S,B} - i_{S,C} \\
	i_{S,C} \\
	i_{S,A} + i_{S,B}
\end{bmatrix}
\end{equation}

Introducing the nodal matrix $G$, we write this as

\begin{equation}\label{eqn:multiphysicsL3:40}
	G \overbar{V}_N = \overbar{I}_S
\end{equation}

We identify the stamp \cref{fig:lecture3:lecture3Fig2} for the resistor

\imageFigure{../../figures/ece1254/lecture3Fig2}{Resistor stamp matrix}{fig:lecture3:lecture3Fig2}{0.3}

as 

\begin{equation}\label{eqn:multiphysicsL3:60}
\begin{bmatrix}
	\inv{R} & -\inv{R} \\
	-\inv{R} & \inv{R}
\end{bmatrix},
\end{equation}

where we have a set of rows and columns for each of the node voltages \(n_1, n_2\).

Note that some care is required to use this nodal analysis method since we required the invertible relationship \(i = V/R\).  We also cannot handle short circuits \(V = 0\), or voltage sources \(V = 5\) (say).  We will also have trouble with differential terms like inductors.

\paragraph{Recap of node branch equations}

We had

\begin{itemize}
	\item KCL: \( A \cdot \overbar{I}_B = \overbar{I}_S\)
	\item Consistuative: \( \overbar{I}_B = \alpha A^\T \overbar{V}_N\),
	\item Nodal equations: \( 
\mathLabelBox
{
		A \alpha A^\T 
}
{\(G\)}
		\overbar{V}_N 
		= \overbar{I}_S \)
\end{itemize}

where \(\overbar{I}_B\) was the branch currents, \(A\) was the incidence matrix, and \(\alpha = \begin{bmatrix}\inv{R_1} & & \\ & \inv{R_2} & \\ & & \ddots \end{bmatrix} \).

The stamp can be found by multipying out the contribution for a single resistor as illustrated in \cref{fig:lecture3:lecture3Fig3}.

\imageFigure{../../figures/ece1254/lecture3Fig3}{Factoring of the stamp matrix}{fig:lecture3:lecture3Fig3}{0.3}

\paragraph{Theoretical facts}

Noting that \(\lr{ A B }^\T = B^\T A^\T \), it is clear that the nodal matrix \(G = A \alpha A^\T \) is symmetric

\begin{dmath}\label{eqn:multiphysicsL3:80}
G^\T
= 
\lr{ A \alpha A^\T }^\T
=
\lr{ A^\T }^\T \alpha^\T A^\T 
=
A \alpha A^\T 
= G
\end{dmath}

\section{Modified nodal analysis (MNA)}

This is the method that we find in software such as spice.  

To illustrate the method, consider the same circuit, augmented with an additional voltage sources as in \cref{fig:lecture3:lecture3Fig4}.

\imageFigure{../../figures/ece1254/lecture3Fig4}{Resistive circuit with current and voltage sources}{fig:lecture3:lecture3Fig4}{0.3}

We know wish to have the following unknowns

\begin{itemize}
	\item node voltages (\(N-1\)): \( V_1, V_2, \cdots V_5 \)
	\item branch currents for selected components (\(K\)): \( i_{S,C}, i_{S,D} \)
\end{itemize}

We will have two less unknowns for this system than with standard nodal analysis.  Our equations are

\begin{enumerate}
\item 
	\(
	-\frac{V_5-V_1}{R_A} 
	+\frac{V_1-V_2}{R_B} 
	+ i_{S,A} = 0
	\)
\item
	\(
	 \frac{V_2-V_5}{R_E} 
	+\frac{V_2-V_1}{R_B} 
	+ i_{S,B} 
	+ i_{S,C} 
	= 0
	\)
\item 
	\(
	-i_{S,C} + 
	\frac{V_3-V_4}{R_C} = 0
	\)
\item
	\(
	-\frac{V_4-0}{R_D} 
	+\frac{V_4-V_3}{R_C} 
	- i_{S,A} 
	- i_{S,B} 
	= 0
	\)
\item 
	\(
	\frac{V_5-V_2}{R_E} 
	+\frac{V_5-V_1}{R_A} 
	+ i_{S,D} = 0
	\)
\end{enumerate}

FIXME: check above.

Put into giant matrix form \cref{fig:lecture3:lecture3Fig5}
%\begin{dmath}\label{eqn:multiphysicsL3:n}
%	\inv{R_A} + \inv{R_B} &
%	-\inv{R_B}  &
%	0 & 
%	0 &
%	-\inv{R_A} \\
%	-\inv{R_B} & 
%	\inv{R_B} + \inv{R_E} & & & -\inv{R_E}
%\end{dmath}

\imageFigure{../../figures/ece1254/lecture3Fig5}{Nodal and voltage incidence matrices}{fig:lecture3:lecture3Fig5}{0.3}

Call the extension to the \textAndIndex{nodal matrix} \(G\), the \textAndIndex{voltage incidence matrix} \(A_V\).

%\EndArticle
\EndNoBibArticle
