%
% Copyright � 2014 Peeter Joot.  All Rights Reserved.
% Licenced as described in the file LICENSE under the root directory of this GIT repository.
%
\newcommand{\authorname}{Peeter Joot}
\newcommand{\email}{peeterjoot@protonmail.com}
\newcommand{\basename}{FIXMEbasenameUndefined}
\newcommand{\dirname}{notes/FIXMEdirnameUndefined/}

\renewcommand{\basename}{diodeRLCSample}
\renewcommand{\dirname}{notes/ece1254/}
%\newcommand{\dateintitle}{}
%\newcommand{\keywords}{}

\newcommand{\authorname}{Peeter Joot}
\newcommand{\onlineurl}{http://sites.google.com/site/peeterjoot2/math2013/\basename.pdf}
\newcommand{\sourcepath}{\dirname\basename.tex}
\newcommand{\generatetitle}[1]{\chapter{#1}}

\newcommand{\vcsinfo}{%
\section*{}
\noindent{\color{DarkOliveGreen}{\rule{\linewidth}{0.1mm}}}
\paragraph{Document version}
%\paragraph{\color{Maroon}{Document version}}
{
\small
\begin{itemize}
\item Available online at:\\ 
\href{\onlineurl}{\onlineurl}
\item Git Repository: \input{./.revinfo/gitRepo.tex}
\item Source: \sourcepath
\item last commit: \input{./.revinfo/gitCommitString.tex}
\item commit date: \input{./.revinfo/gitCommitDate.tex}
\end{itemize}
}
}

%\PassOptionsToPackage{dvipsnames,svgnames}{xcolor}
\PassOptionsToPackage{square,numbers}{natbib}
\documentclass{scrreprt}

\usepackage[left=2cm,right=2cm]{geometry}
\usepackage[svgnames]{xcolor}
\usepackage{peeters_layout}

\usepackage{natbib}

\usepackage[
colorlinks=true,
bookmarks=false,
pdfauthor={\authorname, \email},
backref 
]{hyperref}

% http://tex.stackexchange.com/questions/75773/how-to-reference-problems-by-the-text-label-in-an-exercise-envioronment
\usepackage[english]{cleveref}
\crefname{Exercise}{exercise}{exercises}
\Crefname{Exercise}{Exercise}{Exercises}

\RequirePackage{titlesec}
\RequirePackage{ifthen}

% http://stackoverflow.com/questions/4932910/date-in-the-tabular-environment
\makeatletter
\let\insertdate\@date
\makeatother

\titleformat{\chapter}[display]
{\bfseries\Large}
{\color{DarkSlateGrey}\filleft \authorname
\ifthenelse{\isundefined{\studentnumber}}{}{\\ \studentnumber}
\ifthenelse{\isundefined{\email}}{}{\\ \email}
\ifthenelse{\isundefined{\dateintitle}}{}{\\ \insertdate}
%\ifthenelse{\isundefined{\coursename}}{}{\\ \coursename} % put in title instead.
}
{4ex}
{\color{DarkOliveGreen}{\titlerule}\color{Maroon}
\vspace{2ex}%
\filright}
[\vspace{2ex}%
\color{DarkOliveGreen}\titlerule
]

\newcommand{\beginArtWithToc}[0]{\begin{document}\tableofcontents}
\newcommand{\beginArtNoToc}[0]{\begin{document}}
\newcommand{\EndNoBibArticle}[0]{\end{document}}
\newcommand{\EndArticle}[0]{\bibliography{Bibliography}\bibliographystyle{plainnat}\end{document}}

% 
%\newcommand{\citep}[1]{\cite{#1}}

\colorSectionsForArticle



\beginArtNoToc

\generatetitle{A sample diode RLC circuit}
%\chapter{A sample diode RLC circuit}
%\label{chap:diodeRLCSample}

To get a feel for how to generate the MLN equations for a circuit that has both RLC and non-linear components, consider the circuit of \cref{fig:diodeRLC:diodeRLCFig1}.

\imageFigure{../../figures/ece1254/diodeRLCFig1}{An RLC circuit with a diode.}{fig:diodeRLC:diodeRLCFig1}{0.3}

The KCL equations for this circuit are

\begin{enumerate}
\item \( 0 = i_s - i_d \)
\item \( i_L + \frac{v_2 - v_3}{R} = i_d \)
\item \( \frac{v_3 - v_2}{R} + C \frac{dv_3}{dt} = 0 \)
\item \( -v_2 + L \frac{d i_L}{dt} = 0 \)
\item \( i_d = I_0 \lr{ e^{(v_1 - v_2)/v_T} - 1} \)
\end{enumerate}

FIXME: for the diode, is that the right voltage sign with respect to the current direction?

With \( Z = 1/R \), these can be put into the standard MLN matrix form as

\begin{equation}\label{eqn:diodeRLCSample:20}
\begin{bmatrix}
0 & 0 & 0 & 0 \\
0 & Z & -Z & 1 \\
0 & -Z & Z & 0 \\
0 & -1 & 0 & 0 \\
\end{bmatrix}
\begin{bmatrix}
v_1 \\
v_2 \\
v_3 \\
i_L \\
\end{bmatrix}
+
\begin{bmatrix}
0 & 0 & 0 & 0 \\
0 & 0 & 0 & 0 \\
0 & 0 & C & 0 \\
0 & 0 & 0 & L \\
\end{bmatrix}
{\begin{bmatrix}
v_1 \\
v_2 \\
v_3 \\
i_L \\
\end{bmatrix}}'
=
\begin{bmatrix}
 I_0 & 1 \\
 -I_0 & 0 \\
 0  & 0 \\
 0  & 0 \\
\end{bmatrix}
\begin{bmatrix}
1 \\
i_s(t) \\
\end{bmatrix}
+
\begin{bmatrix}
-I_0 \\
 I_0 \\
 0  \\
 0  \\
\end{bmatrix}
\begin{bmatrix}
e^{(v_2 - v_3)/v_T}
\end{bmatrix}
\end{equation}

Let's write this as 

\begin{equation}\label{eqn:diodeRLCSample:40}
\BG \BX(t) + \BC \dot{\BX}(t) = \BB \Bu(t) + \mytilde{\BB} \Bw(t).
\end{equation}

Here \( \Bu(t) \) collects up all the unique signature sources (for example sources with each different frequency in the system), and \( \Bw(t) \) is a vector of all the unique non-linear (time dependent) terms.

Assuming a bandwidth limited periodic source we know how to express any of the time dependent variables \( v_1, ... \) in terms of their (discrete) Fourier transforms.  Suppose that 
%\( V_n^{(a)} \), \( U_n^{(b)} \), and \( W_n^{(c)} \) are 
the Fourier coefficients for \( v_a(t), u_b(t), w_c(t) \) are given by

\begin{equation}\label{eqn:diodeRLCSample:60}
\begin{aligned}
v_a(t) &= \sum_{n = -N}^N V_n^{(a)} e^{j \omega_0 n t} \\
u_b(t) &= \sum_{n = -N}^N U_n^{(b)} e^{j \omega_0 n t} \\
w_c(t) &= \sum_{n = -N}^N W_n^{(c)} e^{j \omega_0 n t}.
\end{aligned}
\end{equation}

For example, in this circuit, if the source was zero phase signal at the fundamental frequency of our Fourier basis (\( i_s(t) = e^{j \omega_0 t} \)), the only non-zero Fourier components \( U_n^{(a)} \) would be \( U_0^{(1)} = 1, U_1^{(2)} = 1 \).

\Cref{eqn:diodeRLCSample:40} then becomes

\begin{equation}\label{eqn:diodeRLCSample:80}
0 = \sum_{n=-N}^N 
e^{j n \omega_0 t}
\lr{
\lr{
\BG + j \omega_0 n \BC
}
\begin{bmatrix}
V_n^{(1)} \\
V_n^{(2)} \\
\vdots
\end{bmatrix}
- \BB
\begin{bmatrix}
U_n^{(1)} \\
U_n^{(2)} \\
\end{bmatrix}
- \mytilde{\BB}
\begin{bmatrix}
W_n^{(1)} \\
\end{bmatrix}
}
\end{equation}

The time dependence in the linear terms is nicely taken of by this transformation to the frequency domain.  However, we have a fairly messy structure with sums of Fourier components instead of the nice Fourier component vectors that we see in \S A.4 of \citep{giannini2004NonlinearMicrowaveCircuitDesign}.  That reference does consider multivariable problems like this one, so it looks like fully digesting that methodology is the next step.

\EndArticle
