\makeproblem{State counting - polymer}{basicStatMech:problemSet3:2}{ 
A typical protein is a long chain molecule made of very many elementary units called amino acids - it is an example of a class of such macromolecules called polymers. Consider a protein made $N$ amino acids, and assume each amino acid is like a sphere of radius a. As a toy model assume that the protein configuration is like a random walk with each amino acid being one ``step'', i.e., the center-to-center vector from one amino acid to the next is a random vector of length $2a$ and ignore any issues with overlapping spheres (so-called ``excluded volume'' constraints). Estimate the spatial extent of this protein in space. Typical proteins assume a compact form in order to be functional. In this case, taking the constraint of nonoverlapping spheres, estimate the radius of such a compact protein assuming it has an overall spherical shape with fully packed amino acids (ignore holes in the packing, and use only volume ratios to make this estimate). With $N = 300$ and $a \approx 5 \text{\AA}$, estimate these sizes for the random walk case as well as the compact globular case.
} % makeproblem

\makeanswer{basicStatMech:problemSet3:2}{ 

TODO.
}

