%
% Copyright � 2012 Peeter Joot.  All Rights Reserved.
% Licenced as described in the file LICENSE under the root directory of this GIT repository.
%
% pick one:
%\newcommand{\authorname}{Peeter Joot}
\newcommand{\email}{peeter.joot@utoronto.ca}
\newcommand{\studentnumber}{920798560}
\newcommand{\basename}{FIXMEbasenameUndefined}
\newcommand{\dirname}{notes/FIXMEdirnameUndefined/}

\newcommand{\authorname}{Peeter Joot}
\newcommand{\email}{peeterjoot@protonmail.com}
\newcommand{\basename}{FIXMEbasenameUndefined}
\newcommand{\dirname}{notes/FIXMEdirnameUndefined/}

\renewcommand{\basename}{kittelCh2}
\renewcommand{\dirname}{notes/phy452/}
%\newcommand{\dateintitle}{}
%\newcommand{\keywords}{}

\newcommand{\authorname}{Peeter Joot}
\newcommand{\onlineurl}{http://sites.google.com/site/peeterjoot2/math2013/\basename.pdf}
\newcommand{\sourcepath}{\dirname\basename.tex}
\newcommand{\generatetitle}[1]{\chapter{#1}}

\newcommand{\vcsinfo}{%
\section*{}
\noindent{\color{DarkOliveGreen}{\rule{\linewidth}{0.1mm}}}
\paragraph{Document version}
%\paragraph{\color{Maroon}{Document version}}
{
\small
\begin{itemize}
\item Available online at:\\ 
\href{\onlineurl}{\onlineurl}
\item Git Repository: \input{./.revinfo/gitRepo.tex}
\item Source: \sourcepath
\item last commit: \input{./.revinfo/gitCommitString.tex}
\item commit date: \input{./.revinfo/gitCommitDate.tex}
\end{itemize}
}
}

%\PassOptionsToPackage{dvipsnames,svgnames}{xcolor}
\PassOptionsToPackage{square,numbers}{natbib}
\documentclass{scrreprt}

\usepackage[left=2cm,right=2cm]{geometry}
\usepackage[svgnames]{xcolor}
\usepackage{peeters_layout}

\usepackage{natbib}

\usepackage[
colorlinks=true,
bookmarks=false,
pdfauthor={\authorname, \email},
backref 
]{hyperref}

% http://tex.stackexchange.com/questions/75773/how-to-reference-problems-by-the-text-label-in-an-exercise-envioronment
\usepackage[english]{cleveref}
\crefname{Exercise}{exercise}{exercises}
\Crefname{Exercise}{Exercise}{Exercises}

\RequirePackage{titlesec}
\RequirePackage{ifthen}

% http://stackoverflow.com/questions/4932910/date-in-the-tabular-environment
\makeatletter
\let\insertdate\@date
\makeatother

\titleformat{\chapter}[display]
{\bfseries\Large}
{\color{DarkSlateGrey}\filleft \authorname
\ifthenelse{\isundefined{\studentnumber}}{}{\\ \studentnumber}
\ifthenelse{\isundefined{\email}}{}{\\ \email}
\ifthenelse{\isundefined{\dateintitle}}{}{\\ \insertdate}
%\ifthenelse{\isundefined{\coursename}}{}{\\ \coursename} % put in title instead.
}
{4ex}
{\color{DarkOliveGreen}{\titlerule}\color{Maroon}
\vspace{2ex}%
\filright}
[\vspace{2ex}%
\color{DarkOliveGreen}\titlerule
]

\newcommand{\beginArtWithToc}[0]{\begin{document}\tableofcontents}
\newcommand{\beginArtNoToc}[0]{\begin{document}}
\newcommand{\EndNoBibArticle}[0]{\end{document}}
\newcommand{\EndArticle}[0]{\bibliography{Bibliography}\bibliographystyle{plainnat}\end{document}}

% 
%\newcommand{\citep}[1]{\cite{#1}}

\colorSectionsForArticle



\renewcommand{\QuestionNB}{\alph{Question}.\ }
\renewcommand{\theQuestion}{\alph{Question}}

\beginArtNoToc

%\generatetitle{FIXME put title here}
%\chapter{FIXME put title here}
%\label{chap:\basename}
%\section{Motivation}
%\section{Guts}

\makeoproblem{Energy and temperature}{pr:kittelCh2:1}{\citep{kittel1980thermal} problem 2.1}{
Suppose $g(U) = C U^{3N/2}$, where $C$ is a constant and $N$ is the number of particles.  This form of $g(U)$ actually applies to an ideal gas.

\makesubproblem{Show that $U = 3 N t/2$}{pr:kittelCh2:1:a}
\makesubproblem{Show that $(\partial^2 \sigma/\partial U^2)_N$ is negative.}{pr:kittelCh2:1:b}

} % makeproblem
\makeanswer{pr:kittelCh2:1}{ 

\makeSubAnswer{Temperature}{pr:kittelCh2:1:a}

We've got 

\begin{dmath}\label{eqn:kittelCh2:20}
\inv{\tau} 
= \PD{U}{\sigma} 
= \PD{U}{} \left( \ln C + \frac{3N}{2} \ln U \right) 
= \frac{3N}{2} \inv{U},
\end{dmath}

or

\begin{equation}\label{eqn:kittelCh2:40}
U = \frac{3N}{2} \tau.
\end{equation}

\makeSubAnswer{Second derivative of entropy}{pr:kittelCh2:1:b}

From above

\begin{equation}\label{eqn:kittelCh2:60}
\frac{\partial^2 \sigma}{\partial U^2}
= -\frac{3N}{2} \inv{U^2}.
\end{equation}

This doesn't seem particularly suprising if we look at the plots.  For example for $C = 1$ and $3N/2 = 1$ we have \cref{fig:kittleCh2:kittleCh2Fig1}.

\imageFigure{kittleCh2Fig1}{Plots of entropy and its derivatives for this multiplicity function}{fig:kittleCh2:kittleCh2Fig1}{0.3}

The rate of change of entropy with energy decreases monotonically and is always positive, but always has a negative slope.

} % makeanswer

\makeoproblem{Quantum harmonic oscillator}{pr:kittelCh2:3}{\citep{kittel1980thermal} problem 2.3}{

\makesubproblem{Entropy.}{pr:kittelCh2:3a}  Find the entropy of a set of $N$ oscillators of frequency $\omega$ as a function of the total quantum number $n$.  Use the multiplicity function (1.55) and make the Stirling approximation $\ln N! \approx N \ln N - N$.  Replace $N - 1$ by $N$.

\makesubproblem{Planck Energy.}{pr:kittelCh2:3b}  Let $U$ denote the total energy $n \hbar \omega$ of the oscillators.  Express the entropy as $\sigma(U, N)$.  Show that the total energy at temperature $\tau$ is

\begin{equation}\label{eqn:kittelCh2:n}
U = \frac{N \hbar \omega}{\exp\left( \hbar \omega/\tau \right) -1}
\end{equation}

This is the Planck result;  it is derived again in Chapter $4$ by a powerful method that does not require us to find the multiplicity function.

} % makeproblem

\makeanswer{pr:kittelCh2:3}{ 
\makeSubAnswer{Entropy}{pr:kittelCh2:3a}
TODO.
\makeSubAnswer{Planck Energy}{pr:kittelCh2:3b}
TODO.
} % makeanswer

% this is to produce the sites.google url and version info and so forth (for blog posts)
%\vcsinfo
\EndArticle
%\EndNoBibArticle
