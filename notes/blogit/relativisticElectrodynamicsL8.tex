%
% Copyright � 2015 Peeter Joot.  All Rights Reserved.
% Licenced as described in the file LICENSE under the root directory of this GIT repository.
%
\documentclass[]{eliblog}

\usepackage{amsmath}
\usepackage{mathpazo}

%
% shorthand for bold symbols, convenient for vectors and matrices
%
\newcommand{\Ba}[0]{\mathbf{a}}
\newcommand{\Bb}[0]{\mathbf{b}}
\newcommand{\Bc}[0]{\mathbf{c}}
\newcommand{\Bd}[0]{\mathbf{d}}
\newcommand{\Be}[0]{\mathbf{e}}
\newcommand{\Bf}[0]{\mathbf{f}}
\newcommand{\Bg}[0]{\mathbf{g}}
\newcommand{\Bh}[0]{\mathbf{h}}
\newcommand{\Bi}[0]{\mathbf{i}}
\newcommand{\Bj}[0]{\mathbf{j}}
\newcommand{\Bk}[0]{\mathbf{k}}
\newcommand{\Bl}[0]{\mathbf{l}}
\newcommand{\Bm}[0]{\mathbf{m}}
\newcommand{\Bn}[0]{\mathbf{n}}
\newcommand{\Bo}[0]{\mathbf{o}}
\newcommand{\Bp}[0]{\mathbf{p}}
\newcommand{\Bq}[0]{\mathbf{q}}
\newcommand{\Br}[0]{\mathbf{r}}
\newcommand{\Bs}[0]{\mathbf{s}}
\newcommand{\Bt}[0]{\mathbf{t}}
\newcommand{\Bu}[0]{\mathbf{u}}
\newcommand{\Bv}[0]{\mathbf{v}}
\newcommand{\Bw}[0]{\mathbf{w}}
\newcommand{\Bx}[0]{\mathbf{x}}
\newcommand{\By}[0]{\mathbf{y}}
\newcommand{\Bz}[0]{\mathbf{z}}
\newcommand{\BA}[0]{\mathbf{A}}
\newcommand{\BB}[0]{\mathbf{B}}
\newcommand{\BC}[0]{\mathbf{C}}
\newcommand{\BD}[0]{\mathbf{D}}
\newcommand{\BE}[0]{\mathbf{E}}
\newcommand{\BF}[0]{\mathbf{F}}
\newcommand{\BG}[0]{\mathbf{G}}
\newcommand{\BH}[0]{\mathbf{H}}
\newcommand{\BI}[0]{\mathbf{I}}
\newcommand{\BJ}[0]{\mathbf{J}}
\newcommand{\BK}[0]{\mathbf{K}}
\newcommand{\BL}[0]{\mathbf{L}}
\newcommand{\BM}[0]{\mathbf{M}}
\newcommand{\BN}[0]{\mathbf{N}}
\newcommand{\BO}[0]{\mathbf{O}}
\newcommand{\BP}[0]{\mathbf{P}}
\newcommand{\BQ}[0]{\mathbf{Q}}
\newcommand{\BR}[0]{\mathbf{R}}
\newcommand{\BS}[0]{\mathbf{S}}
\newcommand{\BT}[0]{\mathbf{T}}
\newcommand{\BU}[0]{\mathbf{U}}
\newcommand{\BV}[0]{\mathbf{V}}
\newcommand{\BW}[0]{\mathbf{W}}
\newcommand{\BX}[0]{\mathbf{X}}
\newcommand{\BY}[0]{\mathbf{Y}}
\newcommand{\BZ}[0]{\mathbf{Z}}

\newcommand{\Bzero}[0]{\mathbf{0}}
\newcommand{\Btheta}[0]{\boldsymbol{\theta}}
\newcommand{\Btau}[0]{\boldsymbol{\tau}}
\newcommand{\Bomega}[0]{\boldsymbol{\omega}}

%
% shorthand for unit vectors
%
\newcommand{\acap}[0]{\hat{\Ba}}
\newcommand{\bcap}[0]{\hat{\Bb}}
\newcommand{\ccap}[0]{\hat{\Bc}}
\newcommand{\dcap}[0]{\hat{\Bd}}
\newcommand{\ecap}[0]{\hat{\Be}}
\newcommand{\fcap}[0]{\hat{\Bf}}
\newcommand{\gcap}[0]{\hat{\Bg}}
\newcommand{\hcap}[0]{\hat{\Bh}}
\newcommand{\icap}[0]{\hat{\Bi}}
\newcommand{\jcap}[0]{\hat{\Bj}}
\newcommand{\kcap}[0]{\hat{\Bk}}
\newcommand{\lcap}[0]{\hat{\Bl}}
\newcommand{\mcap}[0]{\hat{\Bm}}
\newcommand{\ncap}[0]{\hat{\Bn}}
\newcommand{\ocap}[0]{\hat{\Bo}}
\newcommand{\pcap}[0]{\hat{\Bp}}
\newcommand{\qcap}[0]{\hat{\Bq}}
\newcommand{\rcap}[0]{\hat{\Br}}
\newcommand{\scap}[0]{\hat{\Bs}}
\newcommand{\tcap}[0]{\hat{\Bt}}
\newcommand{\ucap}[0]{\hat{\Bu}}
\newcommand{\vcap}[0]{\hat{\Bv}}
\newcommand{\wcap}[0]{\hat{\Bw}}
\newcommand{\xcap}[0]{\hat{\Bx}}
\newcommand{\ycap}[0]{\hat{\By}}
\newcommand{\zcap}[0]{\hat{\Bz}}
\newcommand{\thetacap}[0]{\hat{\Btheta}}

%
% to write R^n and C^n in a distinguishable fashion.  Perhaps change this
% to the double lined characters upon figuring out how to do so.
%
\newcommand{\C}[1]{$\mathbb{C}^{#1}$}
\newcommand{\R}[1]{$\mathbb{R}^{#1}$}

%
% various generally useful helpers
%

% derivative of #1 wrt. #2:
\newcommand{\D}[2] {\frac {d#2} {d#1}}

\newcommand{\inv}[1]{\frac{1}{#1}}
\newcommand{\cross}[0]{\times}

\newcommand{\abs}[1]{\lvert{#1}\rvert}
\newcommand{\norm}[1]{\lVert{#1}\rVert}
\newcommand{\innerprod}[2]{\langle{#1}, {#2}\rangle}
\newcommand{\dotprod}[2]{{#1} \cdot {#2}}
\newcommand{\bdotprod}[2]{\left({#1} \cdot {#2}\right)}
\newcommand{\crossprod}[2]{{#1} \cross {#2}}
\newcommand{\tripleprod}[3]{\dotprod{\left(\crossprod{#1}{#2}\right)}{#3}}

\DeclareMathOperator{\Proj}{Proj}
\DeclareMathOperator{\Span}{span}
\DeclareMathOperator{\Sgn}{sgn}
\DeclareMathOperator{\Area}{Area}
\DeclareMathOperator{\Volume}{Volume}

%
% A few miscellaneous things specific to this document
%
\newcommand{\crossop}[1]{\crossprod{#1}{}}

% R2 vector.
\newcommand{\VectorTwo}[2]{
\begin{bmatrix}
 {#1} \\
 {#2}
\end{bmatrix}
}

\newcommand{\VectorN}[1]{
\begin{bmatrix}
{#1}_1 \\
{#1}_2 \\
\vdots \\
{#1}_N \\
\end{bmatrix}
}

\newcommand{\DETuvij}[4]{
\begin{vmatrix}
 {#1}_{#3} & {#1}_{#4} \\
 {#2}_{#3} & {#2}_{#4}
\end{vmatrix}
}

\newcommand{\DETuvwijk}[6]{
\begin{vmatrix}
 {#1}_{#4} & {#1}_{#5} & {#1}_{#6} \\
 {#2}_{#4} & {#2}_{#5} & {#2}_{#6} \\
 {#3}_{#4} & {#3}_{#5} & {#3}_{#6}
\end{vmatrix}
}

\newcommand{\DETuvwxijkl}[8]{
\begin{vmatrix}
 {#1}_{#5} & {#1}_{#6} & {#1}_{#7} & {#1}_{#8} \\
 {#2}_{#5} & {#2}_{#6} & {#2}_{#7} & {#2}_{#8} \\
 {#3}_{#5} & {#3}_{#6} & {#3}_{#7} & {#3}_{#8} \\
 {#4}_{#5} & {#4}_{#6} & {#4}_{#7} & {#4}_{#8} \\
\end{vmatrix}
}

%\newcommand{\DETuvwxyijklm}[10]{
%\begin{vmatrix}
% {#1}_{#6} & {#1}_{#7} & {#1}_{#8} & {#1}_{#9} & {#1}_{#10} \\
% {#2}_{#6} & {#2}_{#7} & {#2}_{#8} & {#2}_{#9} & {#2}_{#10} \\
% {#3}_{#6} & {#3}_{#7} & {#3}_{#8} & {#3}_{#9} & {#3}_{#10} \\
% {#4}_{#6} & {#4}_{#7} & {#4}_{#8} & {#4}_{#9} & {#4}_{#10} \\
% {#5}_{#6} & {#5}_{#7} & {#5}_{#8} & {#5}_{#9} & {#5}_{#10}
%\end{vmatrix}
%}

% R3 vector.
\newcommand{\VectorThree}[3]{
\begin{bmatrix}
 {#1} \\
 {#2} \\
 {#3}
\end{bmatrix}
}



\author{Peeter Joot}
\email{peeter.joot@gmail.com}

%\documentclass[]{eliblogwidescreen}

\usepackage{amsmath}
\usepackage{mathpazo}

%
% shorthand for bold symbols, convenient for vectors and matrices
%
\newcommand{\Ba}[0]{\mathbf{a}}
\newcommand{\Bb}[0]{\mathbf{b}}
\newcommand{\Bc}[0]{\mathbf{c}}
\newcommand{\Bd}[0]{\mathbf{d}}
\newcommand{\Be}[0]{\mathbf{e}}
\newcommand{\Bf}[0]{\mathbf{f}}
\newcommand{\Bg}[0]{\mathbf{g}}
\newcommand{\Bh}[0]{\mathbf{h}}
\newcommand{\Bi}[0]{\mathbf{i}}
\newcommand{\Bj}[0]{\mathbf{j}}
\newcommand{\Bk}[0]{\mathbf{k}}
\newcommand{\Bl}[0]{\mathbf{l}}
\newcommand{\Bm}[0]{\mathbf{m}}
\newcommand{\Bn}[0]{\mathbf{n}}
\newcommand{\Bo}[0]{\mathbf{o}}
\newcommand{\Bp}[0]{\mathbf{p}}
\newcommand{\Bq}[0]{\mathbf{q}}
\newcommand{\Br}[0]{\mathbf{r}}
\newcommand{\Bs}[0]{\mathbf{s}}
\newcommand{\Bt}[0]{\mathbf{t}}
\newcommand{\Bu}[0]{\mathbf{u}}
\newcommand{\Bv}[0]{\mathbf{v}}
\newcommand{\Bw}[0]{\mathbf{w}}
\newcommand{\Bx}[0]{\mathbf{x}}
\newcommand{\By}[0]{\mathbf{y}}
\newcommand{\Bz}[0]{\mathbf{z}}
\newcommand{\BA}[0]{\mathbf{A}}
\newcommand{\BB}[0]{\mathbf{B}}
\newcommand{\BC}[0]{\mathbf{C}}
\newcommand{\BD}[0]{\mathbf{D}}
\newcommand{\BE}[0]{\mathbf{E}}
\newcommand{\BF}[0]{\mathbf{F}}
\newcommand{\BG}[0]{\mathbf{G}}
\newcommand{\BH}[0]{\mathbf{H}}
\newcommand{\BI}[0]{\mathbf{I}}
\newcommand{\BJ}[0]{\mathbf{J}}
\newcommand{\BK}[0]{\mathbf{K}}
\newcommand{\BL}[0]{\mathbf{L}}
\newcommand{\BM}[0]{\mathbf{M}}
\newcommand{\BN}[0]{\mathbf{N}}
\newcommand{\BO}[0]{\mathbf{O}}
\newcommand{\BP}[0]{\mathbf{P}}
\newcommand{\BQ}[0]{\mathbf{Q}}
\newcommand{\BR}[0]{\mathbf{R}}
\newcommand{\BS}[0]{\mathbf{S}}
\newcommand{\BT}[0]{\mathbf{T}}
\newcommand{\BU}[0]{\mathbf{U}}
\newcommand{\BV}[0]{\mathbf{V}}
\newcommand{\BW}[0]{\mathbf{W}}
\newcommand{\BX}[0]{\mathbf{X}}
\newcommand{\BY}[0]{\mathbf{Y}}
\newcommand{\BZ}[0]{\mathbf{Z}}

\newcommand{\Bzero}[0]{\mathbf{0}}
\newcommand{\Btheta}[0]{\boldsymbol{\theta}}
\newcommand{\Btau}[0]{\boldsymbol{\tau}}
\newcommand{\Bomega}[0]{\boldsymbol{\omega}}

%
% shorthand for unit vectors
%
\newcommand{\acap}[0]{\hat{\Ba}}
\newcommand{\bcap}[0]{\hat{\Bb}}
\newcommand{\ccap}[0]{\hat{\Bc}}
\newcommand{\dcap}[0]{\hat{\Bd}}
\newcommand{\ecap}[0]{\hat{\Be}}
\newcommand{\fcap}[0]{\hat{\Bf}}
\newcommand{\gcap}[0]{\hat{\Bg}}
\newcommand{\hcap}[0]{\hat{\Bh}}
\newcommand{\icap}[0]{\hat{\Bi}}
\newcommand{\jcap}[0]{\hat{\Bj}}
\newcommand{\kcap}[0]{\hat{\Bk}}
\newcommand{\lcap}[0]{\hat{\Bl}}
\newcommand{\mcap}[0]{\hat{\Bm}}
\newcommand{\ncap}[0]{\hat{\Bn}}
\newcommand{\ocap}[0]{\hat{\Bo}}
\newcommand{\pcap}[0]{\hat{\Bp}}
\newcommand{\qcap}[0]{\hat{\Bq}}
\newcommand{\rcap}[0]{\hat{\Br}}
\newcommand{\scap}[0]{\hat{\Bs}}
\newcommand{\tcap}[0]{\hat{\Bt}}
\newcommand{\ucap}[0]{\hat{\Bu}}
\newcommand{\vcap}[0]{\hat{\Bv}}
\newcommand{\wcap}[0]{\hat{\Bw}}
\newcommand{\xcap}[0]{\hat{\Bx}}
\newcommand{\ycap}[0]{\hat{\By}}
\newcommand{\zcap}[0]{\hat{\Bz}}
\newcommand{\thetacap}[0]{\hat{\Btheta}}

%
% to write R^n and C^n in a distinguishable fashion.  Perhaps change this
% to the double lined characters upon figuring out how to do so.
%
\newcommand{\C}[1]{$\mathbb{C}^{#1}$}
\newcommand{\R}[1]{$\mathbb{R}^{#1}$}

%
% various generally useful helpers
%

% derivative of #1 wrt. #2:
\newcommand{\D}[2] {\frac {d#2} {d#1}}

\newcommand{\inv}[1]{\frac{1}{#1}}
\newcommand{\cross}[0]{\times}

\newcommand{\abs}[1]{\lvert{#1}\rvert}
\newcommand{\norm}[1]{\lVert{#1}\rVert}
\newcommand{\innerprod}[2]{\langle{#1}, {#2}\rangle}
\newcommand{\dotprod}[2]{{#1} \cdot {#2}}
\newcommand{\bdotprod}[2]{\left({#1} \cdot {#2}\right)}
\newcommand{\crossprod}[2]{{#1} \cross {#2}}
\newcommand{\tripleprod}[3]{\dotprod{\left(\crossprod{#1}{#2}\right)}{#3}}

\DeclareMathOperator{\Proj}{Proj}
\DeclareMathOperator{\Span}{span}
\DeclareMathOperator{\Sgn}{sgn}
\DeclareMathOperator{\Area}{Area}
\DeclareMathOperator{\Volume}{Volume}

%
% A few miscellaneous things specific to this document
%
\newcommand{\crossop}[1]{\crossprod{#1}{}}

% R2 vector.
\newcommand{\VectorTwo}[2]{
\begin{bmatrix}
 {#1} \\
 {#2}
\end{bmatrix}
}

\newcommand{\VectorN}[1]{
\begin{bmatrix}
{#1}_1 \\
{#1}_2 \\
\vdots \\
{#1}_N \\
\end{bmatrix}
}

\newcommand{\DETuvij}[4]{
\begin{vmatrix}
 {#1}_{#3} & {#1}_{#4} \\
 {#2}_{#3} & {#2}_{#4}
\end{vmatrix}
}

\newcommand{\DETuvwijk}[6]{
\begin{vmatrix}
 {#1}_{#4} & {#1}_{#5} & {#1}_{#6} \\
 {#2}_{#4} & {#2}_{#5} & {#2}_{#6} \\
 {#3}_{#4} & {#3}_{#5} & {#3}_{#6}
\end{vmatrix}
}

\newcommand{\DETuvwxijkl}[8]{
\begin{vmatrix}
 {#1}_{#5} & {#1}_{#6} & {#1}_{#7} & {#1}_{#8} \\
 {#2}_{#5} & {#2}_{#6} & {#2}_{#7} & {#2}_{#8} \\
 {#3}_{#5} & {#3}_{#6} & {#3}_{#7} & {#3}_{#8} \\
 {#4}_{#5} & {#4}_{#6} & {#4}_{#7} & {#4}_{#8} \\
\end{vmatrix}
}

%\newcommand{\DETuvwxyijklm}[10]{
%\begin{vmatrix}
% {#1}_{#6} & {#1}_{#7} & {#1}_{#8} & {#1}_{#9} & {#1}_{#10} \\
% {#2}_{#6} & {#2}_{#7} & {#2}_{#8} & {#2}_{#9} & {#2}_{#10} \\
% {#3}_{#6} & {#3}_{#7} & {#3}_{#8} & {#3}_{#9} & {#3}_{#10} \\
% {#4}_{#6} & {#4}_{#7} & {#4}_{#8} & {#4}_{#9} & {#4}_{#10} \\
% {#5}_{#6} & {#5}_{#7} & {#5}_{#8} & {#5}_{#9} & {#5}_{#10}
%\end{vmatrix}
%}

% R3 vector.
\newcommand{\VectorThree}[3]{
\begin{bmatrix}
 {#1} \\
 {#2} \\
 {#3}
\end{bmatrix}
}



\author{Peeter Joot}
\email{peeter.joot@gmail.com}


\chapter{PHY450H1S.  Relativistic Electrodynamics Lecture 8 (Taught by Prof. Erich Poppitz).  Relativistic dynamics.}
\label{chap:relativisticElectrodynamicsL8}
%\useCCL
\blogpage{http://sites.google.com/site/peeterjoot/math2011/relativisticElectrodynamicsL8.pdf}
\date{Feb 1, 2011}
\revisionInfo{relativisticElectrodynamicsL8.tex}

%\beginArtWithToc
\beginArtNoToc

\section{Reading.}

Covering chapter 2 material from the text \cite{landau1980classical}.

Covering a bit more of \href{http://www.physics.utoronto.ca/~poppitz/e-poppitz/PHY450_files/RelEMpp52-56.pdf}{Professor Poppitz's lecture notes}: equation of motion, symmetries, and conserved quantities (energy-momentum 4 vector) from relativistic particle action.

Covering \href{http://www.physics.utoronto.ca/~poppitz/e-poppitz/PHY450_files/RelEMpp56.1-73.pdf}{lecture notes pp. 56.1-72}: comments on mass, energy, momentum, and massless particles (56.1-58); particles in external fields: Lorentz scalar field (59-62); reminder of a vector field under spatial rotations (63) and a Lorentz vector field (64-65) [Tuesday, Feb. 1]; the action for a relativistic particle in an external 4-vector field (65-66); the equation of motion of a relativistic particle in an external electromagnetic (4-vector) field (67,68,73) [Wednesday, Feb. 2]; mathematical interlude: (69-72): on 3x3 antisymmetric matrices, 3-vectors, and totally antisymmetric 3-index tensor - please read by yourselves, preferably by Wed., Feb. 2 class! (this is important, we�ll also soon need the 4-dimensional generalization)

\section{More on the action.}

For a free particle, our action is

\begin{align*}
S 
&= - m c \int ds  \\
&= -m c^2 \int dt \sqrt{1 - \frac{\Bv^2}{c^2}}
\end{align*}

Our Lagrangian is

\begin{equation}\label{eqn:relativisticElectrodynamicsL8:n}
\LL = -m c^2 \sqrt{1 - \frac{\Bv^2}{c^2}} .
\end{equation}

We can also make a non-relativistic velocity approximation

\begin{align*}
\LL 
&= -m c^2 \sqrt{1 - \frac{\Bv^2}{c^2}} \\
&\approx -m c^2 \left( 1 - \inv{2} \frac{\Bv^2}{c^2} \right) \\
&= \underbrace{-m c^2 }_{constant} + \underbrace{\inv{2} m \Bv^2 }_{\text{Classical Lagrangian for free particle}} + O(\Bv^2/c^2),
\end{align*}

and it is good to know that we recover the familiar Newtonian case when our velocities are small enough.

Our job is to vary the action between a pair of spacetime points

\begin{equation}\label{eqn:relativisticElectrodynamicsL8:n}
(t_a, \Bx_a) \rightarrow (t_b, \Bx_b)
\end{equation}

The equations of motion that result from this variation, or from the Euler-Lagrange equations that one can obtain from this variation are

\begin{equation}\label{eqn:relativisticElectrodynamicsL8:n}
\frac{d}{dt} ( \gamma \Bv) = 0
\end{equation}

This implies

\begin{equation}\label{eqn:relativisticElectrodynamicsL8:n}
\frac{d\Bv}{dt} = 0
\end{equation}

Since $\frac{d\gamma}{dt} = 0$ too (since $d\Bv/dt = 0$)

so we can combine the pair of equations

\begin{align}\label{eqn:relativisticElectrodynamicsL8:n}
\frac{d}{dt} ( \gamma \Bv/c ) &= 0 \\
\frac{d}{dt} ( \gamma ) &= 0
\end{align}

into

\begin{equation}\label{eqn:relativisticElectrodynamicsL8:n}
u^i = (u^0, \Bu),
\end{equation}

with 

\begin{align}\label{eqn:relativisticElectrodynamicsL8:n}
u^0 &= \gamma \\
\Bu &= \gamma \Bv/c.
\end{align}

This implies

\begin{equation}\label{eqn:relativisticElectrodynamicsL8:n}
\inv{c \sqrt{1 - \frac{\Bv^2}{c^2}}} \frac{du^i}{dt} = 0,
\end{equation}

or

\begin{equation}\label{eqn:relativisticElectrodynamicsL8:n}
\frac{u^i}{ds} = 0.
\end{equation}

The symmetries of $S$ imply conservation laws.  Our action has $SO(1,3) \times T^4 = $ Lorentz x spacetime translation $\equiv$ Poincar\'{e} group of symmetries.

Consider quantities conserved due to $T^4$ factor

\begin{align}\label{eqn:relativisticElectrodynamicsL8:n}
\Bx &\rightarrow \Bx + \Ba \\
t &\rightarrow t + \text{constant}
\end{align}

With $\Ba$ also constant.

\begin{equation}\label{eqn:relativisticElectrodynamicsL8:n}
\LL(\Bx, \Bv, t) - m c \sqrt{1 - \frac{\Bv^2}{c^2}}
= 
\LL(\Bv) - m c \sqrt{1 - \frac{\Bv^2}{c^2}}.
\end{equation}

\begin{equation}\label{eqn:relativisticElectrodynamicsL8:n}
\frac{d}{dt} \PD{\Bv}{\LL} = \PD{\Bx}{\LL} = 0,
\end{equation}

implies

\begin{align*}
0 &= \frac{d}{dt} \left( m c \gamma \Bv \right) \\
&= \frac{d}{dt} \left( m c u^{1,2,3} \right) \\
&= \frac{d}{dt} \left( m c \gamma \right) 
\end{align*}

FIXME: location?  Missed the sequencing of things here.

\begin{equation}\label{eqn:relativisticElectrodynamicsL8:n}
\frac{d}{dt} \left( m c u^i \right)  = 0
\end{equation}

Here $p^i = m u^i$ is a 4 vector.

A relativistic particle is characterizes by a conserved 4 vector quantity $p^i$ with

\begin{align}\label{eqn:relativisticElectrodynamicsL8:n}
p^0 &= m c \gamma \\
\Bp &= m \gamma \Bv \\
p^i &= (p^0, \Bp)
\end{align}

FIXME: work through this at a decent rate.

\section{Time translation invariance}

\begin{equation}\label{eqn:relativisticElectrodynamicsL8:n}
\LL(\Bx, \Bv, t) = \LL(\Bv) 
\end{equation}

However, it helps to consider the more general case

\begin{equation}\label{eqn:relativisticElectrodynamicsL8:n}
\LL(\Bx, \Bv, t) = \LL(\Bx, \Bv) 
\end{equation}

since we have no explicit time dependence.

\begin{align*}
\frac{d}{dt} \LL(\Bv) 
&= \PD{\Bx}{\LL} \cdot \dot{\Bx} + \PD{\Bv}{\LL} \cdot \dot{\Bv} \\
&= \left( \frac{d}{dt} \PD{\Bv}{\LL} \right) \cdot \Bv + \PD{\Bv}{\LL} \cdot \frac{d\Bv}{dt} \\
&= \frac{d}{dt} \left( \PD{\Bv}{\LL} \cdot \Bv\right)
\end{align*}

for 

\begin{equation}\label{eqn:relativisticElectrodynamicsL8:n}
\LL = -m c^2 \sqrt{ 1 - \frac{\Bv^2}{c^2} }
\end{equation}

\begin{align*}
E 
&= \PD{\Bv}{\LL} \cdot \Bv - \LL \\
&= \Bv \cdot \left( m \inv{\sqrt{1 - \frac{\Bv^2}{c^2}}} \Bv \right) + m c^2 \sqrt{1 - \frac{\Bv^2}{c^2}} \\
&= \frac{m \Bv^2}{\sqrt{ 1 - \frac{\Bv^2}{c^2}}} + mc^2 \sqrt{ 1 - \frac{\Bv^2}{c^2}} \\
&= \frac{\Bv^2 + m c^2 \left(1 - \frac{\Bv^2}{c^2} \right)}{\sqrt{1 - \frac{\Bv^2}{c^2}}} \\
&= \frac{m c^2 }{\sqrt{ 1 - \frac{\Bv^2}{c^2} }}
\end{align*}

We write 

\begin{equation}\label{eqn:relativisticElectrodynamicsL8:n}
\boxed{
E = \gamma m c^2 = \frac{ m c^2 }{\sqrt{1 - \frac{\Bv^2}{c^2}}}
}
\end{equation}

With $\Bv \rightarrow 0$ we have

\begin{equation}\label{eqn:relativisticElectrodynamicsL8:n}
E = m c^2
\end{equation}

Since we also know (from the spacetime translation) that $p^0 = m c \gamma = E/c$, we get another conserved quantity for free since $(p^0, \Bp)$ then is also a symmetry (i.e. thus a conserved quantity)

\begin{align*}
p^0 &= m c \gamma = \frac{E}{c} \\
\Bp &= m c \Bv
\end{align*}

\begin{equation}\label{eqn:relativisticElectrodynamicsL8:n}
p^i = ( p^0, \Bp )
\end{equation}

Note that the only ``mass'' you ever want to talk about is ``m''.  This is a Lorentz scalar, and we won't use the old notions that mass changes with velocity or ``relativistic mass''.

\section{Some properties of the four momentum.}

We have

\begin{align*}
p^i p_i 
&= (p^0)^2 - \Bp^2 \\
&= m c^2 \gamma^2 - m^2 \gamma^2 \Bv^2 \\
&= m c^2 \gamma^2 \left( 1 - \frac{\Bv^2}{c^2} \right)  \\
&= m^2 c^2
\end{align*}

So we have 

\begin{equation}\label{eqn:relativisticElectrodynamicsL8:n}
\boxed{
p^i p_i = m^2 c^2
}
\end{equation}

4-vector $p^i$ of particle with mass $m$.

Since four momentum is a conserved quantity we can use this conservation property to study relativistic collisions

PICTURE: two particles colliding with two particles resulting (particles trajectories as arrows)

\begin{equation}\label{eqn:relativisticElectrodynamicsL8:n}
\underbrace{p_1^i + p_2^i}_{\text{four momentum before}} = \underbrace{p_3^i + p_4^i}_{\text{four momentum after}}
\end{equation}

\begin{align}\label{eqn:relativisticElectrodynamicsL8:n}
\Bp &= \frac{m \Bv}{\sqrt{ 1 - \frac{\Bv^2}{c^2} }} \rightarrow 0 \mbox{when $m \rightarrow 0$} \\
E &= \frac{m c^2}{\sqrt{ 1 - \frac{\Bv^2}{c^2} }} \rightarrow 0 \mbox{when $m \rightarrow 0$}
\end{align}

except when $\Abs{\Bv} = c$, where if you take $m \rightarrow 0$ and $\Abs{\Bv} = c$ you can get anything (any values) in such a limit (limit does not exist).

However, because

\begin{equation}\label{eqn:relativisticElectrodynamicsL8:n}
\frac{E^2}{c^2} - \Bp^2 = m^2 c^2 = 0
\end{equation}

when $m \rightarrow 0$, $E$ and $\Bp$ for a massless particle must obey $E = c \Abs{\Bp}$.

Massless particles like photons (and gravitons if/when eventually measured) have lightlike 4 momentum vectors 

\begin{equation}\label{eqn:relativisticElectrodynamicsL8:n}
p^i p_i = 0
\end{equation}

Gravity waves haven't been seen yet, but the LIGO and LISA (extremely large infraferometers) experiments are expected to get some results on this in the near future.

\section{Where are we?}

In the notes there's a review (see that on one's own).  We'll also want to eventually deal with the conservation laws in four vector form, since it will illustrate how the electric and magnetic fields have to be transformed.  We'll get to that eventually.

\section{Interactions}

In classical mechanics we have 

\begin{equation}\label{eqn:relativisticElectrodynamicsL8:n}
\LL_\text{kinetic} = \inv{2} m \Bv^2 
\end{equation}

\begin{equation}\label{eqn:relativisticElectrodynamicsL8:n}
\LL = \inv{2} m \Bv^2  - U(\Br)
\end{equation}

\begin{equation}\label{eqn:relativisticElectrodynamicsL8:n}
U_\text{internal} = U(\Br)
\end{equation}

Here $U(\Br)$ is an external potential.  FIXME: what did he mean by this equality?

\begin{equation}\label{eqn:relativisticElectrodynamicsL8:n}
S = S_{\text{free}} + S_{\text{interaction}}  = \int dt m \Bv^2/2 + \int dt ( -U(\Br, t)
\end{equation}

The quantity $U(\Br, t)$ is what we call a potential field.

What's the simplest invariant field we can have?  The simplest possibility is to have a relativistic particle which interacts with an external \underline{Lorentz scalar field}.  We'd imagine that this is due to some other particle or some distribution of other fields.

Recall that the scalar field under rotations (reminder) 

PICTURE: a point with coordinates in a fixed and a rotated coordinate system

That point is 

\begin{equation}\label{eqn:relativisticElectrodynamicsL8:n}
P = (x, y) = (x', y')
\end{equation}

Similarly we can define a scalar quantity (like temperature or the Coulomb potential) is then assigned a value at each point

\begin{equation}\label{eqn:relativisticElectrodynamicsL8:n}
\phi(x, y) = \phi'(x', y')
\end{equation}

The value of this scalar in the $x,y$ coordinates system at point $P$ equals the value of this scalar in the $x',y'$ coordinates system at the same point $P$.

A Lorentz scalar field is like this, but for an event $P = (ct, x) = (ct', x')$ is the same.

So, we'd have 

\begin{equation}\label{eqn:relativisticElectrodynamicsL8:n}
\phi(ct, x) = \phi'(ct', x')
\end{equation}

The value of this scalar in the $x,ct$ coordinates system at event $P$ equals the value of this scalar in the $x',ct'$ coordinates system at the same event $P$ in the primed frame.

Our action would then be

\begin{equation}\label{eqn:relativisticElectrodynamicsL8:n}
S = - m c \int ds + g \int ds \phi(x^i)
\end{equation}

Here $g$ is a coupling constant, also called the ``charge'' of a particle under that scalar field.

Note that unfortunately nature hasn't provided us with scalar fields that are stable enough to observe in classical interactions

We do however have some scalar particles 

\begin{equation}\label{eqn:relativisticElectrodynamicsL8:n}
\pi^0, \pi^\pm, k^0, k^\pm
\end{equation}

These are unstable and short ranged.

The LHC is looking for another unstable short lived scalar field (the Higgs).  So we have to unfortunately study a more complicated field, a vector field.  We'll do that next time.

\EndArticle
