%
% Copyright � 2012 Peeter Joot.  All Rights Reserved.
% Licenced as described in the file LICENSE under the root directory of this GIT repository.
%
\newcommand{\authorname}{Peeter Joot}
\newcommand{\email}{peeterjoot@protonmail.com}
\newcommand{\basename}{FIXMEbasenameUndefined}
\newcommand{\dirname}{notes/FIXMEdirnameUndefined/}

\renewcommand{\basename}{modernOpticsLecture22}
\renewcommand{\dirname}{notes/phy485/}
\newcommand{\keywords}{Optics, PHY485H1F}
\newcommand{\authorname}{Peeter Joot}
\newcommand{\onlineurl}{http://sites.google.com/site/peeterjoot2/math2013/\basename.pdf}
\newcommand{\sourcepath}{\dirname\basename.tex}
\newcommand{\generatetitle}[1]{\chapter{#1}}

\newcommand{\vcsinfo}{%
\section*{}
\noindent{\color{DarkOliveGreen}{\rule{\linewidth}{0.1mm}}}
\paragraph{Document version}
%\paragraph{\color{Maroon}{Document version}}
{
\small
\begin{itemize}
\item Available online at:\\ 
\href{\onlineurl}{\onlineurl}
\item Git Repository: \input{./.revinfo/gitRepo.tex}
\item Source: \sourcepath
\item last commit: \input{./.revinfo/gitCommitString.tex}
\item commit date: \input{./.revinfo/gitCommitDate.tex}
\end{itemize}
}
}

%\PassOptionsToPackage{dvipsnames,svgnames}{xcolor}
\PassOptionsToPackage{square,numbers}{natbib}
\documentclass{scrreprt}

\usepackage[left=2cm,right=2cm]{geometry}
\usepackage[svgnames]{xcolor}
\usepackage{peeters_layout}

\usepackage{natbib}

\usepackage[
colorlinks=true,
bookmarks=false,
pdfauthor={\authorname, \email},
backref 
]{hyperref}

% http://tex.stackexchange.com/questions/75773/how-to-reference-problems-by-the-text-label-in-an-exercise-envioronment
\usepackage[english]{cleveref}
\crefname{Exercise}{exercise}{exercises}
\Crefname{Exercise}{Exercise}{Exercises}

\RequirePackage{titlesec}
\RequirePackage{ifthen}

% http://stackoverflow.com/questions/4932910/date-in-the-tabular-environment
\makeatletter
\let\insertdate\@date
\makeatother

\titleformat{\chapter}[display]
{\bfseries\Large}
{\color{DarkSlateGrey}\filleft \authorname
\ifthenelse{\isundefined{\studentnumber}}{}{\\ \studentnumber}
\ifthenelse{\isundefined{\email}}{}{\\ \email}
\ifthenelse{\isundefined{\dateintitle}}{}{\\ \insertdate}
%\ifthenelse{\isundefined{\coursename}}{}{\\ \coursename} % put in title instead.
}
{4ex}
{\color{DarkOliveGreen}{\titlerule}\color{Maroon}
\vspace{2ex}%
\filright}
[\vspace{2ex}%
\color{DarkOliveGreen}\titlerule
]

\newcommand{\beginArtWithToc}[0]{\begin{document}\tableofcontents}
\newcommand{\beginArtNoToc}[0]{\begin{document}}
\newcommand{\EndNoBibArticle}[0]{\end{document}}
\newcommand{\EndArticle}[0]{\bibliography{Bibliography}\bibliographystyle{plainnat}\end{document}}

% 
%\newcommand{\citep}[1]{\cite{#1}}

\colorSectionsForArticle



\usepackage{tikz}
\usepackage[draft]{fixme}
\fxusetheme{color}

\beginArtNoToc
\generatetitle{PHY485H1F Modern Optics.  Lecture 22: Number of photons per free space mode.  Taught by Prof.\ Joseph Thywissen}
%\chapter{Number of photons per free space mode}
\label{chap:modernOpticsLecture22}

%\section{Disclaimer}
%
%Peeter's lecture notes from class.  May not be entirely coherent.

\section{Number of photons per free space mode}

\fxwarning{review lecture 22}{work through this lecture in detail.}

Laser is fundamentally characterized by a large number of photons per free space mode.  Possible because photons are bosons!  Laser light has temporal and spatial coherence.

\paragraph{Review:} \underline{in a cavity}

\begin{itemize}
\item Laser light: $\expectation{n} \sim 10^7 - 10^8$ above threshold.
\item Thermal light: $\expectation{n} = \inv{ e^{\hbar \omega/k_{\mathrm{B}} T} - 1 }$.
\end{itemize}

\makeexample{Some numbers}{example:modernOpticsLecture22:1}{
\begin{equation}\label{eqn:modernOpticsLecture22:20}
\expectation{n} \sim 1
\end{equation}
\begin{equation}\label{eqn:modernOpticsLecture22:40}
\hbar \omega \sim k_{\mathrm{B}} T \ln 2
\end{equation}
\begin{equation}\label{eqn:modernOpticsLecture22:60}
300 \mathrm{K}: \lambda > 70 \mu \mathrm{m}
\end{equation}
\begin{equation}\label{eqn:modernOpticsLecture22:80}
5000 \mathrm{K}: \lambda > 4 \mu \mathrm{m}
\end{equation}
}

F1

\paragraph{3D Free space mode.}

Consider ``elementary pencil'' of light.

F2

Here 

\begin{align}\label{eqn:modernOpticsLecture22:100}
(\Delta x \Delta k_x)_{\mathrm{min}} &\sim 1 \\
(\Delta y \Delta k_y)_{\mathrm{min}} &\sim 1
\end{align}

where $1$ here means a constant of order $1$.

Along 2: consider pulse train

F3

where $\Delta z = c \tau_0$

\begin{dmath}\label{eqn:modernOpticsLecture22:n}
\Psi(t) = 
\left\{
\begin{array}{l l}
e^{-i \omega_0 t} & \quad \mbox{$t \in [-\tau_0/2, \tau_0/2]$} \\
0 & \quad \mbox{otherwise} 
\end{array}
\right.
\end{dmath}

What is $\Delta k_z$?  Making a paraxial approximation

\begin{dmath}\label{eqn:modernOpticsLecture22:120}
\Delta k_z \approx \Delta k = \frac{\Delta \omega}{c}
\end{dmath}

We can compute the Fourier transform as depicted in 
F4

\begin{dmath}\label{eqn:modernOpticsLecture22:140}
g(\omega) 
= \int e^{i \omega t} \Psi(t) dt
= z \frac{\sin \left( (\omega - \omega_0) \tau_0/2 \right) }{\omega - \omega_0}
\end{dmath}

So 

\begin{dmath}\label{eqn:modernOpticsLecture22:160}
\Delta z \Delta k_z
=
\left( c \tau_0
\right)
\left( 
\frac{2\pi}{c \tau_0}
\right)
= 2 \pi
\end{dmath}

Now consider sourcce that has width $\Gamma$.  Imagine emission of pulses of length $\tau_0 = 2 \pi/\Gamma$.

F5

\paragraph{Laser light}:

\begin{dmath}\label{eqn:modernOpticsLecture22:n}
\frac{\text{number of photons}}{\text{free space mode}}
=
\frac{P/\hbar \omega}{\Gamma_{\mathrm{laser}}} 
\sim \frac{\Gamma_{\mathrm{cav}} \expectation{n}}{\Gamma_{\mathrm{cav}}/\expectation{n}} 
\sim \expectation{n}^2
\end{dmath}

Justifying this last operation

F6

%\EndArticle
\EndNoBibArticle
