%
% Copyright � 2012 Peeter Joot.  All Rights Reserved.
% Licenced as described in the file LICENSE under the root directory of this GIT repository.
%
\newcommand{\authorname}{Peeter Joot}
\newcommand{\email}{peeterjoot@protonmail.com}
\newcommand{\basename}{FIXMEbasenameUndefined}
\newcommand{\dirname}{notes/FIXMEdirnameUndefined/}

\renewcommand{\basename}{modernOpticsLecture4}
\renewcommand{\dirname}{notes/phy485/}
\newcommand{\keywords}{Optics, PHY485H1F, diffraction, diffraction integral, Huygens-Fresnel principle}

\newcommand{\authorname}{Peeter Joot}
\newcommand{\onlineurl}{http://sites.google.com/site/peeterjoot2/math2013/\basename.pdf}
\newcommand{\sourcepath}{\dirname\basename.tex}
\newcommand{\generatetitle}[1]{\chapter{#1}}

\newcommand{\vcsinfo}{%
\section*{}
\noindent{\color{DarkOliveGreen}{\rule{\linewidth}{0.1mm}}}
\paragraph{Document version}
%\paragraph{\color{Maroon}{Document version}}
{
\small
\begin{itemize}
\item Available online at:\\ 
\href{\onlineurl}{\onlineurl}
\item Git Repository: \input{./.revinfo/gitRepo.tex}
\item Source: \sourcepath
\item last commit: \input{./.revinfo/gitCommitString.tex}
\item commit date: \input{./.revinfo/gitCommitDate.tex}
\end{itemize}
}
}

%\PassOptionsToPackage{dvipsnames,svgnames}{xcolor}
\PassOptionsToPackage{square,numbers}{natbib}
\documentclass{scrreprt}

\usepackage[left=2cm,right=2cm]{geometry}
\usepackage[svgnames]{xcolor}
\usepackage{peeters_layout}

\usepackage{natbib}

\usepackage[
colorlinks=true,
bookmarks=false,
pdfauthor={\authorname, \email},
backref 
]{hyperref}

% http://tex.stackexchange.com/questions/75773/how-to-reference-problems-by-the-text-label-in-an-exercise-envioronment
\usepackage[english]{cleveref}
\crefname{Exercise}{exercise}{exercises}
\Crefname{Exercise}{Exercise}{Exercises}

\RequirePackage{titlesec}
\RequirePackage{ifthen}

% http://stackoverflow.com/questions/4932910/date-in-the-tabular-environment
\makeatletter
\let\insertdate\@date
\makeatother

\titleformat{\chapter}[display]
{\bfseries\Large}
{\color{DarkSlateGrey}\filleft \authorname
\ifthenelse{\isundefined{\studentnumber}}{}{\\ \studentnumber}
\ifthenelse{\isundefined{\email}}{}{\\ \email}
\ifthenelse{\isundefined{\dateintitle}}{}{\\ \insertdate}
%\ifthenelse{\isundefined{\coursename}}{}{\\ \coursename} % put in title instead.
}
{4ex}
{\color{DarkOliveGreen}{\titlerule}\color{Maroon}
\vspace{2ex}%
\filright}
[\vspace{2ex}%
\color{DarkOliveGreen}\titlerule
]

\newcommand{\beginArtWithToc}[0]{\begin{document}\tableofcontents}
\newcommand{\beginArtNoToc}[0]{\begin{document}}
\newcommand{\EndNoBibArticle}[0]{\end{document}}
\newcommand{\EndArticle}[0]{\bibliography{Bibliography}\bibliographystyle{plainnat}\end{document}}

% 
%\newcommand{\citep}[1]{\cite{#1}}

\colorSectionsForArticle



\beginArtNoToc

\generatetitle{PHY485H1F Modern Optics.  Lecture 4: Diffraction.  Taught by Prof.\ Joseph Thywissen}
\label{chap:modernOpticsLecture4}

\section{Disclaimer}

Peeter's lecture notes from class.  May not be entirely coherent.

\section{Context}

We start the class with a green laser setup, where the light is displayed on the screen, then also after going through a single and double slit, as illustrated in 

FIXME: F1

We see also that a blue laser diffracts less.  The bigger the wavelength, the harder it is to ignore.  We can consider this a breakdown of geometric optics.

\section{Diffraction}

We'll want to consider systems of this sort (light source, object in between, goes some distance, then observed) mathematically.  We consider the geometry of

FIXME: F2

We have two approximations to the full problem

\begin{enumerate}
\item A scalar theory can suffice.
\item The region of interest (and source) are paraxial.
\end{enumerate}

Why a scalar theory?  If we have a plane wave polarization

\begin{dmath}\label{eqn:modernOpticsLecture4:10}
\BE(\Br, t) = \left( E_1 \xcap + E_2 \ycap \right) e^{ i \Bk \cdot \Br - i \omega t}
\end{dmath}

With the principle of superposition

\begin{enumerate}
\item Solve for the $x$ polarization.
\item Solve for the $y$ polarization.
\item Vector addition of result.
\end{enumerate}

We will assume no mixing, so that we can treat just one component.

Reading: \S 8 \cite{born2000principles}, \S \cite{jackson1975cew}.  The first goes and proves that the scalar theory is sufficient under this conditions.

We'd like to solve the wave equation with these approximations.

\begin{dmath}\label{eqn:modernOpticsLecture4:30}
\spacegrad^2 E = \inv{c^2} \PDSq{t}{E}
\end{dmath}

We will use a monochromatic wave so that we can write the electric field magnituide as a vector function times a time phase term

\begin{dmath}\label{eqn:modernOpticsLecture4:50}
E = \Psi(\Br) e^{-i \omega t}
\end{dmath}

We find

\begin{dmath}\label{eqn:modernOpticsLecture4:70}
\boxed{
(\spacegrad^2 + \Bk^2) \Psi(\Br) = 0
}
\end{dmath}

This is called the Helmholtz equation.  Reading: Appendix 2 of \cite{hecht1998hecht}.

It turns out that the solution to this equation is generally written out as the surface integral

\begin{dmath}\label{eqn:modernOpticsLecture4:90}
% FIXME had oiint ... not found.
\Psi(\Br) = \iint da' \left( \Psi(\Br') \spacegrad' G - G \spacegrad' \Psi(\Br') \right) \cdot \ncap
\end{dmath}

Here $\ncap$ is the unit normal perpendicular to the surface, and the Green function of the Helmoltz equation is

\begin{dmath}\label{eqn:modernOpticsLecture4:110}
G(\Br, \Br') = \frac{e^{i k R}}{4 \pi R}.
\end{dmath}

As illustrated in 

FIXME: F3 

this isn't an entirely unsuprising seeming Green's function for this problem.  We have the $e^{i k R}$ type of phase factor that we expected (and guessed in the geometric optics treatment), and also have the $1/R$ factor that we need to retain power at a distance $R$.

Also note that the primed gradient is taken with respect to the coordinates of $\Br'$

\begin{dmath}\label{eqn:modernOpticsLecture4:130}
\spacegrad' = \Be_m \PD{{x'}_m}{}
\end{dmath}

If we take the gradient of the Green's function we find

\begin{dmath}\label{eqn:modernOpticsLecture4:150}
\spacegrad \left( 
\frac{ e^{i k r} }{r}
\right) = \rcap \left( i k - \inv{r} \right) 
\frac{e^{i k r}}{r}
\end{dmath}

Applying this to our problem we find

\begin{dmath}\label{eqn:modernOpticsLecture4:170}
\Psi(\Br) = -\inv{4 \pi} \iint \frac{e^{i k R}}{R} 
\ncap \cdot 
\left( 
\spacegrad' \Psi + \left( i k - \inv{R} \right) \frac{\BR}{R} \Psi \right) 
da'.
\end{dmath}

Here $\BR = \Br - \Br'$ and $da' = dx' dy'$ or $\rho' d\rho' d\theta'$.  We are going to neglect the surface at $\infty$ as illustrated in 

FIXME: F4

This neglect is justified for example in Jackson, cited above.

\section{A calculated example: pinhole}

Consider

FIXME: F5

where

\begin{dmath}\label{eqn:modernOpticsLecture4:190}
\psi(\Br') = A \frac{e^{i k R_s}}{R_s}
\end{dmath}

and

\begin{dmath}\label{eqn:modernOpticsLecture4:210}
\spacegrad' \Psi = \ncap \left( i k - \inv{R_s} \right) \frac{e^{i k R_s}}{R_s}.
\end{dmath}

We will place our origin at the pinhole, so that $\Br' = 0$, $\BR = \Br$, $\BR_s = \Br_s$.  Our resulting wave function is then

\begin{dmath}\label{eqn:modernOpticsLecture4:230}
\Psi(\Br) = - \frac{A}{4 \pi} \frac{e^{i k ( R + r_s)}}{R r_s} 
\left( 
\ncap \cdot \ncap 
\left( i k - \inv{r_s} 
\right) + \ncap \cdot \rcap 
\left( i k - \inv{r}
\right)
\right)
\end{dmath}

Now, in all these $ik - 1/r_s$ we have $k$ of order $1/\lambda$ and $1/r_s$ is of order $1/r$ or $1/r_s$.

Recall from geometric optics that we used

\begin{dmath}\label{eqn:modernOpticsLecture4:250}
\spacegrad \left( \BE_0 e^{i\phi(\Br)} \right) \approx i (\BE_0 \cdot \spacegrad \phi ) e^{i \phi(\Br)},
\end{dmath}

With an assumption

\begin{dmath}\label{eqn:modernOpticsLecture4:270}
\lambda \ll r, r_s,
\end{dmath}

and

\begin{dmath}\label{eqn:modernOpticsLecture4:290}
\lambda \ll d \ll r, r_s,
\end{dmath}

where $d$ is the ``typical object size'', so that we have

\begin{dmath}\label{eqn:modernOpticsLecture4:310}
\Psi(\Br) = - \frac{A}{4 \pi} \frac{e^{i k ( R + r_s)}}{R r_s} \left( 
\ncap \cdot \ncap \left( i k - \cancel{\inv{r_s}} \right) + \ncap \cdot \rcap 
\left( i k - \cancel{\inv{r}}
\right)
\right)
\end{dmath}

or with $\theta$ as illustrated,

\begin{dmath}\label{eqn:modernOpticsLecture4:330}
\boxed{
\Psi(\Br) = \frac{k A}{2 \pi i} \frac{e^{i k ( r + r_s)}}{r r_s} k(\theta),
}
\end{dmath}

where

\begin{dmath}\label{eqn:modernOpticsLecture4:350}
k(\theta) = \inv{2} \left( 1 + \cos \theta \right),
\end{dmath}

is the ``obliquity factor''.

This is called the \underline{Huygens-Fresnel} principle.

\section{Generalizing}

Should we open up the pinhole, we have to adjust things accordingly, and find

\begin{dmath}\label{eqn:modernOpticsLecture4:370}
\boxed{
\Psi(\Br) = \frac{k A}{2 \pi r r_s} \iint e^{i k (R + R_s)} k(\theta) d^2 \Br'
}
\end{dmath}

where $R = \Abs{\Br - \Br'}$, and $R_s = \Abs{\Br_s - \Br'}$.

This is called the \underline{diffraction integral}.

\vcsinfo
\EndArticle
