\makeproblem{Surface area of a sphere in d-dimensions}{basicStatMech:problemSet3:1}{ 
Consider a $d$-dimensional sphere with radius $R$. Derive the volume of the sphere $V_d(R)$ and its surface area $S_d(R)$ using $S_d(R) = dV_d(R)/dR$.
} % makeproblem

\makeanswer{basicStatMech:problemSet3:1}{ 

Let's start with some terminology and notation, borrowing from \citep{wiki:nsphere}

\begin{itemize}
\item $1$-sphere : 2D circle, with ``volume'' $V_2$
\item $2$-sphere : 3D sphere, with volume $V_3$
\item $3$-sphere : 4D Euclidian hypersphere, with ``volume'' $V_4$
\item $4$-sphere : 5D Euclidian hypersphere, with ``volume'' $V_5$
\item $\cdots$
\end{itemize}

To calculate the volume, we require a parameterization allowing for expression of the volume element in an easy to integrate fashion.  For the $1$-sphere, we can use the usual circlular and spherical coordinate volume elements

\begin{subequations}
\begin{dmath}\label{eqn:basicStatMechProblemSet3Problem1:20}
V_2(R) = 4 \int_0^R r dr \int_0^{\pi/2} d\theta = 4 \frac{R^2}{2} \frac{\pi}{2} = \pi R^2
\end{dmath}
\begin{dmath}\label{eqn:basicStatMechProblemSet3Problem1:40}
V_3(R) = 8 \int_0^R r^2 dr \int_0^{\pi/2} \sin\theta d\theta \int_0^{\pi/2} d\phi
= 8 \frac{R^3}{3} (1) \frac{\pi}{2}
= \frac{4}{3} \pi R^3
\end{dmath}
\end{subequations}

Here, to simplify the integration ranges, we've calculated the ``volume'' integral for just one quadrant or octet of the circle and sphere respectively, as in \cref{fig:basicStatMechProblemSet3:basicStatMechProblemSet3Fig1}.

\imageFigure{basicStatMechProblemSet3Fig1}{Integrating over just one quadrant of circle or octet of sphere}{fig:basicStatMechProblemSet3:basicStatMechProblemSet3Fig1}{0.3}

How will we generalize this to higher dimensions?  To calculate the volume elements in a systematic fashion, we can introduce a parameterization and use a Jacobian to change from Cartesian coordinates.  For the $1$-volume and $2$-volume cases those parameterizations were the familiar

\begin{subequations}
\begin{equation}\label{eqn:basicStatMechProblemSet3Problem1:60}
\begin{aligned}
x_1 &= r \cos\theta \\
x_0 &= r \sin\theta
\end{aligned}
\end{equation}
\begin{equation}\label{eqn:basicStatMechProblemSet3Problem1:80}
\begin{aligned}
x_2 &= r \cos\theta \\
x_1 &= r \sin\theta \cos\phi \\
x_0 &= r \sin\theta \sin\phi
\end{aligned}
\end{equation}
\end{subequations}

Some reflection shows that this generalizes nicely.  Let's use shorthand

\begin{subequations}
\begin{equation}\label{eqn:basicStatMechProblemSet3Problem1:100}
C_k = \cos\theta_k
\end{equation}
\begin{equation}\label{eqn:basicStatMechProblemSet3Problem1:120}
S_k = \sin\theta_k,
\end{equation}
\end{subequations}

and pick $V_5$, say, a dimension bigger than the 2D or 3D cases that we can do by inspection, we can parameterize with

\begin{equation}\label{eqn:basicStatMechProblemSet3Problem1:140}
\begin{aligned}
x_4 &= r C_4 \\
x_3 &= r S_4 C_3 \\
x_2 &= r S_4 S_3 C_2 \\
x_1 &= r S_4 S_3 S_2 C_1 \\
x_0 &= r S_4 S_3 S_2 S_1.
\end{aligned}
\end{equation}

Our volume element 

\begin{equation}\label{eqn:basicStatMechProblemSet3Problem1:160}
dV_5 = 
\mathLabelBox
[
   labelstyle={xshift=2cm},
   linestyle={out=270,in=90, latex-}
]
{
\frac{\partial(x_4, x_3, x_2, x_1, x_0)}{\partial(r, \theta_4, \theta_3, \theta_2, \theta_1)} 
}{$\equiv J_5$
}
dr d\theta_4 d\theta_3 d\theta_2 d\theta_1
\end{equation}

That Jacobian is

\begin{dmath}\label{eqn:basicStatMechProblemSet3Problem1:180}
J_5 = 
\left\lvert
\begin{array}{lllll}
C_4               &          -  r S_4               & \phantom{-} 0                  & \phantom{-} 0                 & \phantom{-} 0                  \\
S_4 C_3           & \phantom{-} r C_4 C_3           &          -  r S_4 S_3          & \phantom{-} 0                 & \phantom{-} 0                  \\
S_4 S_3 C_2       & \phantom{-} r C_4 S_3 C_2       & \phantom{-} r S_4 C_3 C_2      &          -  r S_4 S_3 S_2     & \phantom{-} 0                  \\
S_4 S_3 S_2 C_1   & \phantom{-} r C_4 S_3 S_2 C_1   & \phantom{-} r S_4 C_3 S_2 C_1  & \phantom{-} r S_4 S_3 C_2 C_1 &          -  r S_4 S_3 S_2 S_1  \\
S_4 S_3 S_2 S_1   & \phantom{-} r C_4 S_3 S_2 S_1   & \phantom{-} r S_4 C_3 S_2 S_1  & \phantom{-} r S_4 S_3 C_2 S_1 & \phantom{-} r S_4 S_3 S_2 C_1   
\end{array}
\right\rvert
=
r^4 S_4^3 S_3^2 S_2^1 S_1^0
%\begin{vmatrix}
%C_4               & -S_4               &   0            &  0       &  0   \\
%S_4 C_3           &  C_4 C_3           & - S_3          &  0       &  0   \\
%S_4 S_3 C_2       &  C_4 S_3 C_2       &   C_3 C_2      & -S_2     &  0   \\
%S_4 S_3 S_2 C_1   &  C_4 S_3 S_2 C_1   &   C_3 S_2 C_1  &  C_2 C_1 & -S_1 \\
%S_4 S_3 S_2 S_1   &  C_4 S_3 S_2 S_1   &   C_3 S_2 S_1  &  C_2 S_1 &  C_1 
%\end{vmatrix}
\left\lvert
\begin{array}{lllll}
% row 1
      \begin{array}{l}
      C_4
      \end{array}
      \begin{array}{l}
      \phantom{S_3 S_2 S_1}
      \end{array}               
   & 
      \begin{array}{l}
      -S_4
      \end{array}
      \begin{array}{l}
      \phantom{S_3 S_2 S_1}
      \end{array}
   &
      \begin{array}{l}
      \phantom{-}0
      \end{array}
   &  
      \begin{array}{l}
      \phantom{-}0       
      \end{array}
   &  
      \begin{array}{l}
      \phantom{-}0
      \end{array}
\\
% row 2-5:
\begin{array}{l}
S_4 \\
S_4 \\
S_4 \\
S_4 
\end{array}
\myBoxed{
\begin{array}{l}
 C_3            \\
 S_3 C_2        \\
 S_3 S_2 C_1    \\
 S_3 S_2 S_1  
\end{array} 
}
&
\begin{array}{l}
\phantom{-}C_4 \\
\phantom{-}C_4 \\
\phantom{-}C_4 \\
\phantom{-}C_4 
\end{array}
\myBoxed{
\begin{array}{l}
C_3           \\
S_3 C_2       \\
S_3 S_2 C_1   \\
S_3 S_2 S_1   
\end{array}
}
&
\begin{array}{l}
         -  S_3           \\
\phantom{-} C_3 C_2       \\
\phantom{-} C_3 S_2 C_1   \\
\phantom{-} C_3 S_2 S_1 
\end{array}
&
\begin{array}{l}
\phantom{-} 0        \\
         -  S_2      \\
\phantom{-} C_2 C_1  \\
\phantom{-} C_2 S_1 
\end{array}
&
\begin{array}{l}
\phantom{-}0   \\
\phantom{-}0   \\
         - S_1 \\
\phantom{-}C_1 
\end{array}
\end{array}
\right\rvert
\end{dmath}

Observe above that the cofactors of both the $1,1$ and the $1,2$ elements, when expanded along the first row, have a common factor.  This allows us to work recursively

\begin{dmath}\label{eqn:basicStatMechProblemSet3Problem1:200}
J_5 
= r^4 S_4^3 S_3^2 S_2^1 S_1^0
(C_4^2 - - S_4^2)
\left\lvert
\begin{array}{llll}
C_3           &          -  S_3          & \phantom{-} 0       & \phantom{-} 0    \\
S_3 C_2       & \phantom{-} C_3 C_2      &          -  S_2     & \phantom{-} 0    \\
S_3 S_2 C_1   & \phantom{-} C_3 S_2 C_1  & \phantom{-} C_2 C_1 &          -  S_1  \\
S_3 S_2 S_1   & \phantom{-} C_3 S_2 S_1  & \phantom{-} C_2 S_1 & \phantom{-} C_1   
\end{array}
\right\rvert
=
r S_4^3 J_4
\end{dmath}

Similarily for the 4D volume

\begin{dmath}\label{eqn:basicStatMechProblemSet3Problem1:220}
J_4  
= r^3 S_3^2 S_2^1 S_1^0
\left\lvert
\begin{array}{llll}
C_3           &          -  S_3          & \phantom{-} 0       & \phantom{-} 0    \\
S_3 C_2       & \phantom{-} C_3 C_2      &          -  S_2     & \phantom{-} 0    \\
S_3 S_2 C_1   & \phantom{-} C_3 S_2 C_1  & \phantom{-} C_2 C_1 &          -  S_1  \\
S_3 S_2 S_1   & \phantom{-} C_3 S_2 S_1  & \phantom{-} C_2 S_1 & \phantom{-} C_1   
\end{array}
\right\rvert
= r^3 S_3^2 S_2^1 S_1^0 (C_2^2 + S_2^2)
\left\lvert
\begin{array}{lll}
\phantom{-} C_2      &          -  S_2     & \phantom{-} 0    \\
\phantom{-} S_2 C_1  & \phantom{-} C_2 C_1 &          -  S_1  \\
\phantom{-} S_2 S_1  & \phantom{-} C_2 S_1 & \phantom{-} C_1   
\end{array}
\right\rvert
= r S_3^2 J_3
\end{dmath}

and for the 3D volume
\begin{dmath}\label{eqn:basicStatMechProblemSet3Problem1:240}
J_3  
= r^2 S_2^1 S_1^0
\left\lvert
\begin{array}{lll}
\phantom{-} C_2      &          -  S_2     & \phantom{-} 0    \\
\phantom{-} S_2 C_1  & \phantom{-} C_2 C_1 &          -  S_1  \\
\phantom{-} S_2 S_1  & \phantom{-} C_2 S_1 & \phantom{-} C_1   
\end{array}
\right\rvert
= r^2 S_2^1 S_1^0 (C_2^2 + S_2^2)
\left\lvert
\begin{array}{ll}
C_1 &          -  S_1  \\
S_1 & \phantom{-} C_1   
\end{array}
\right\rvert
= r S_2^1 J_2,
\end{dmath}

and finally for the 2D volume

\begin{dmath}\label{eqn:basicStatMechProblemSet3Problem1:260}
J_2
= r S_1^0
\left\lvert
\begin{array}{ll}
C_1 &          -  S_1  \\
S_1 & \phantom{-} C_1   
\end{array}
\right\rvert
= r S_1^0.
\end{dmath}

Putting all the bits together, the ``volume'' element in the n-D space is

\begin{equation}\label{eqn:basicStatMechProblemSet3Problem1:280}
dV_n = dr d\theta_{n-1} \cdots d\theta_{1} r^{n-1} \prod_{k=0}^{n-2} (\sin\theta_{k+1})^{k}, 
\end{equation}

and the total volume is
\begin{dmath}\label{eqn:basicStatMechProblemSet3Problem1:300}
V_n(R) 
= 2^n 
\int_0^R r^{n-1} dr 
\prod_{k=0}^{n-2} \int_0^{\pi/2} d\theta \sin^k \theta
= 2^n \frac{R^{n}}{n}
\prod_{k=0}^{n-2} \int_0^{\pi/2} d\theta \sin^k \theta
= 
\frac{4 R^2}{n} (n-2) 2^{n-2} \frac{R^{n-2}}{n-2}
\prod_{k=n-3}^{n-2} \int_0^{\pi/2} d\theta \sin^k \theta
\prod_{k=0}^{n-4} \int_0^{\pi/2} d\theta \sin^k \theta
=
4 R^2 \frac{n-2}{n} 
V_{n-2}(R)
\int_0^{\pi/2} d\theta \sin^{n-2} \theta
\int_0^{\pi/2} d\theta \sin^{n-3} \theta
=
4 R^2 \frac{n-2}{n} 
V_{n-2}(R)
\frac{\pi}{2 (n-2)}.
\end{dmath}

Note that a single of these sine power integrals has a messy result (computed using Mathematica)
% FIXME: \nbref{statMechProblemSet3.nb}

\begin{equation}\label{eqn:basicStatMechProblemSet3Problem1:n}
\int_0^{\pi/2} d\theta \sin^{k} \theta
=
\frac{\sqrt{\pi } \Gamma \left(\frac{k+1}{2}\right)}{2 \Gamma \left(\frac{k}{2}+1\right)},
\end{equation}

but the product is simple, and that result (also computed with Mathematica) is inserted above, providing an expression for the n-D volume element as a recurrence relation

\begin{equation}\label{eqn:basicStatMechProblemSet3Problem1:n}
\myBoxed{
V_{n} = \frac{2 \pi}{n} R^2 V_{n-2}(R)
}
\end{equation}

}
