\documentclass[]{eliblog}

\usepackage{color}
%\usepackage{txfonts} % for xi

\usepackage{amsmath}
\usepackage{mathpazo}

%
% shorthand for bold symbols, convenient for vectors and matrices
%
\newcommand{\Ba}[0]{\mathbf{a}}
\newcommand{\Bb}[0]{\mathbf{b}}
\newcommand{\Bc}[0]{\mathbf{c}}
\newcommand{\Bd}[0]{\mathbf{d}}
\newcommand{\Be}[0]{\mathbf{e}}
\newcommand{\Bf}[0]{\mathbf{f}}
\newcommand{\Bg}[0]{\mathbf{g}}
\newcommand{\Bh}[0]{\mathbf{h}}
\newcommand{\Bi}[0]{\mathbf{i}}
\newcommand{\Bj}[0]{\mathbf{j}}
\newcommand{\Bk}[0]{\mathbf{k}}
\newcommand{\Bl}[0]{\mathbf{l}}
\newcommand{\Bm}[0]{\mathbf{m}}
\newcommand{\Bn}[0]{\mathbf{n}}
\newcommand{\Bo}[0]{\mathbf{o}}
\newcommand{\Bp}[0]{\mathbf{p}}
\newcommand{\Bq}[0]{\mathbf{q}}
\newcommand{\Br}[0]{\mathbf{r}}
\newcommand{\Bs}[0]{\mathbf{s}}
\newcommand{\Bt}[0]{\mathbf{t}}
\newcommand{\Bu}[0]{\mathbf{u}}
\newcommand{\Bv}[0]{\mathbf{v}}
\newcommand{\Bw}[0]{\mathbf{w}}
\newcommand{\Bx}[0]{\mathbf{x}}
\newcommand{\By}[0]{\mathbf{y}}
\newcommand{\Bz}[0]{\mathbf{z}}
\newcommand{\BA}[0]{\mathbf{A}}
\newcommand{\BB}[0]{\mathbf{B}}
\newcommand{\BC}[0]{\mathbf{C}}
\newcommand{\BD}[0]{\mathbf{D}}
\newcommand{\BE}[0]{\mathbf{E}}
\newcommand{\BF}[0]{\mathbf{F}}
\newcommand{\BG}[0]{\mathbf{G}}
\newcommand{\BH}[0]{\mathbf{H}}
\newcommand{\BI}[0]{\mathbf{I}}
\newcommand{\BJ}[0]{\mathbf{J}}
\newcommand{\BK}[0]{\mathbf{K}}
\newcommand{\BL}[0]{\mathbf{L}}
\newcommand{\BM}[0]{\mathbf{M}}
\newcommand{\BN}[0]{\mathbf{N}}
\newcommand{\BO}[0]{\mathbf{O}}
\newcommand{\BP}[0]{\mathbf{P}}
\newcommand{\BQ}[0]{\mathbf{Q}}
\newcommand{\BR}[0]{\mathbf{R}}
\newcommand{\BS}[0]{\mathbf{S}}
\newcommand{\BT}[0]{\mathbf{T}}
\newcommand{\BU}[0]{\mathbf{U}}
\newcommand{\BV}[0]{\mathbf{V}}
\newcommand{\BW}[0]{\mathbf{W}}
\newcommand{\BX}[0]{\mathbf{X}}
\newcommand{\BY}[0]{\mathbf{Y}}
\newcommand{\BZ}[0]{\mathbf{Z}}

\newcommand{\Bzero}[0]{\mathbf{0}}
\newcommand{\Btheta}[0]{\boldsymbol{\theta}}
\newcommand{\Btau}[0]{\boldsymbol{\tau}}
\newcommand{\Bomega}[0]{\boldsymbol{\omega}}

%
% shorthand for unit vectors
%
\newcommand{\acap}[0]{\hat{\Ba}}
\newcommand{\bcap}[0]{\hat{\Bb}}
\newcommand{\ccap}[0]{\hat{\Bc}}
\newcommand{\dcap}[0]{\hat{\Bd}}
\newcommand{\ecap}[0]{\hat{\Be}}
\newcommand{\fcap}[0]{\hat{\Bf}}
\newcommand{\gcap}[0]{\hat{\Bg}}
\newcommand{\hcap}[0]{\hat{\Bh}}
\newcommand{\icap}[0]{\hat{\Bi}}
\newcommand{\jcap}[0]{\hat{\Bj}}
\newcommand{\kcap}[0]{\hat{\Bk}}
\newcommand{\lcap}[0]{\hat{\Bl}}
\newcommand{\mcap}[0]{\hat{\Bm}}
\newcommand{\ncap}[0]{\hat{\Bn}}
\newcommand{\ocap}[0]{\hat{\Bo}}
\newcommand{\pcap}[0]{\hat{\Bp}}
\newcommand{\qcap}[0]{\hat{\Bq}}
\newcommand{\rcap}[0]{\hat{\Br}}
\newcommand{\scap}[0]{\hat{\Bs}}
\newcommand{\tcap}[0]{\hat{\Bt}}
\newcommand{\ucap}[0]{\hat{\Bu}}
\newcommand{\vcap}[0]{\hat{\Bv}}
\newcommand{\wcap}[0]{\hat{\Bw}}
\newcommand{\xcap}[0]{\hat{\Bx}}
\newcommand{\ycap}[0]{\hat{\By}}
\newcommand{\zcap}[0]{\hat{\Bz}}
\newcommand{\thetacap}[0]{\hat{\Btheta}}

%
% to write R^n and C^n in a distinguishable fashion.  Perhaps change this
% to the double lined characters upon figuring out how to do so.
%
\newcommand{\C}[1]{$\mathbb{C}^{#1}$}
\newcommand{\R}[1]{$\mathbb{R}^{#1}$}

%
% various generally useful helpers
%

% derivative of #1 wrt. #2:
\newcommand{\D}[2] {\frac {d#2} {d#1}}

\newcommand{\inv}[1]{\frac{1}{#1}}
\newcommand{\cross}[0]{\times}

\newcommand{\abs}[1]{\lvert{#1}\rvert}
\newcommand{\norm}[1]{\lVert{#1}\rVert}
\newcommand{\innerprod}[2]{\langle{#1}, {#2}\rangle}
\newcommand{\dotprod}[2]{{#1} \cdot {#2}}
\newcommand{\bdotprod}[2]{\left({#1} \cdot {#2}\right)}
\newcommand{\crossprod}[2]{{#1} \cross {#2}}
\newcommand{\tripleprod}[3]{\dotprod{\left(\crossprod{#1}{#2}\right)}{#3}}

\DeclareMathOperator{\Proj}{Proj}
\DeclareMathOperator{\Span}{span}
\DeclareMathOperator{\Sgn}{sgn}
\DeclareMathOperator{\Area}{Area}
\DeclareMathOperator{\Volume}{Volume}

%
% A few miscellaneous things specific to this document
%
\newcommand{\crossop}[1]{\crossprod{#1}{}}

% R2 vector.
\newcommand{\VectorTwo}[2]{
\begin{bmatrix}
 {#1} \\
 {#2}
\end{bmatrix}
}

\newcommand{\VectorN}[1]{
\begin{bmatrix}
{#1}_1 \\
{#1}_2 \\
\vdots \\
{#1}_N \\
\end{bmatrix}
}

\newcommand{\DETuvij}[4]{
\begin{vmatrix}
 {#1}_{#3} & {#1}_{#4} \\
 {#2}_{#3} & {#2}_{#4}
\end{vmatrix}
}

\newcommand{\DETuvwijk}[6]{
\begin{vmatrix}
 {#1}_{#4} & {#1}_{#5} & {#1}_{#6} \\
 {#2}_{#4} & {#2}_{#5} & {#2}_{#6} \\
 {#3}_{#4} & {#3}_{#5} & {#3}_{#6}
\end{vmatrix}
}

\newcommand{\DETuvwxijkl}[8]{
\begin{vmatrix}
 {#1}_{#5} & {#1}_{#6} & {#1}_{#7} & {#1}_{#8} \\
 {#2}_{#5} & {#2}_{#6} & {#2}_{#7} & {#2}_{#8} \\
 {#3}_{#5} & {#3}_{#6} & {#3}_{#7} & {#3}_{#8} \\
 {#4}_{#5} & {#4}_{#6} & {#4}_{#7} & {#4}_{#8} \\
\end{vmatrix}
}

%\newcommand{\DETuvwxyijklm}[10]{
%\begin{vmatrix}
% {#1}_{#6} & {#1}_{#7} & {#1}_{#8} & {#1}_{#9} & {#1}_{#10} \\
% {#2}_{#6} & {#2}_{#7} & {#2}_{#8} & {#2}_{#9} & {#2}_{#10} \\
% {#3}_{#6} & {#3}_{#7} & {#3}_{#8} & {#3}_{#9} & {#3}_{#10} \\
% {#4}_{#6} & {#4}_{#7} & {#4}_{#8} & {#4}_{#9} & {#4}_{#10} \\
% {#5}_{#6} & {#5}_{#7} & {#5}_{#8} & {#5}_{#9} & {#5}_{#10}
%\end{vmatrix}
%}

% R3 vector.
\newcommand{\VectorThree}[3]{
\begin{bmatrix}
 {#1} \\
 {#2} \\
 {#3}
\end{bmatrix}
}



\author{Peeter Joot}
\email{peeter.joot@utoronto.ca, 920798560}
%%
% Copyright � 2015 Peeter Joot.  All Rights Reserved.
% Licenced as described in the file LICENSE under the root directory of this GIT repository.
%
\documentclass[]{eliblog}

\usepackage{amsmath}
\usepackage{mathpazo}

%
% shorthand for bold symbols, convenient for vectors and matrices
%
\newcommand{\Ba}[0]{\mathbf{a}}
\newcommand{\Bb}[0]{\mathbf{b}}
\newcommand{\Bc}[0]{\mathbf{c}}
\newcommand{\Bd}[0]{\mathbf{d}}
\newcommand{\Be}[0]{\mathbf{e}}
\newcommand{\Bf}[0]{\mathbf{f}}
\newcommand{\Bg}[0]{\mathbf{g}}
\newcommand{\Bh}[0]{\mathbf{h}}
\newcommand{\Bi}[0]{\mathbf{i}}
\newcommand{\Bj}[0]{\mathbf{j}}
\newcommand{\Bk}[0]{\mathbf{k}}
\newcommand{\Bl}[0]{\mathbf{l}}
\newcommand{\Bm}[0]{\mathbf{m}}
\newcommand{\Bn}[0]{\mathbf{n}}
\newcommand{\Bo}[0]{\mathbf{o}}
\newcommand{\Bp}[0]{\mathbf{p}}
\newcommand{\Bq}[0]{\mathbf{q}}
\newcommand{\Br}[0]{\mathbf{r}}
\newcommand{\Bs}[0]{\mathbf{s}}
\newcommand{\Bt}[0]{\mathbf{t}}
\newcommand{\Bu}[0]{\mathbf{u}}
\newcommand{\Bv}[0]{\mathbf{v}}
\newcommand{\Bw}[0]{\mathbf{w}}
\newcommand{\Bx}[0]{\mathbf{x}}
\newcommand{\By}[0]{\mathbf{y}}
\newcommand{\Bz}[0]{\mathbf{z}}
\newcommand{\BA}[0]{\mathbf{A}}
\newcommand{\BB}[0]{\mathbf{B}}
\newcommand{\BC}[0]{\mathbf{C}}
\newcommand{\BD}[0]{\mathbf{D}}
\newcommand{\BE}[0]{\mathbf{E}}
\newcommand{\BF}[0]{\mathbf{F}}
\newcommand{\BG}[0]{\mathbf{G}}
\newcommand{\BH}[0]{\mathbf{H}}
\newcommand{\BI}[0]{\mathbf{I}}
\newcommand{\BJ}[0]{\mathbf{J}}
\newcommand{\BK}[0]{\mathbf{K}}
\newcommand{\BL}[0]{\mathbf{L}}
\newcommand{\BM}[0]{\mathbf{M}}
\newcommand{\BN}[0]{\mathbf{N}}
\newcommand{\BO}[0]{\mathbf{O}}
\newcommand{\BP}[0]{\mathbf{P}}
\newcommand{\BQ}[0]{\mathbf{Q}}
\newcommand{\BR}[0]{\mathbf{R}}
\newcommand{\BS}[0]{\mathbf{S}}
\newcommand{\BT}[0]{\mathbf{T}}
\newcommand{\BU}[0]{\mathbf{U}}
\newcommand{\BV}[0]{\mathbf{V}}
\newcommand{\BW}[0]{\mathbf{W}}
\newcommand{\BX}[0]{\mathbf{X}}
\newcommand{\BY}[0]{\mathbf{Y}}
\newcommand{\BZ}[0]{\mathbf{Z}}

\newcommand{\Bzero}[0]{\mathbf{0}}
\newcommand{\Btheta}[0]{\boldsymbol{\theta}}
\newcommand{\Btau}[0]{\boldsymbol{\tau}}
\newcommand{\Bomega}[0]{\boldsymbol{\omega}}

%
% shorthand for unit vectors
%
\newcommand{\acap}[0]{\hat{\Ba}}
\newcommand{\bcap}[0]{\hat{\Bb}}
\newcommand{\ccap}[0]{\hat{\Bc}}
\newcommand{\dcap}[0]{\hat{\Bd}}
\newcommand{\ecap}[0]{\hat{\Be}}
\newcommand{\fcap}[0]{\hat{\Bf}}
\newcommand{\gcap}[0]{\hat{\Bg}}
\newcommand{\hcap}[0]{\hat{\Bh}}
\newcommand{\icap}[0]{\hat{\Bi}}
\newcommand{\jcap}[0]{\hat{\Bj}}
\newcommand{\kcap}[0]{\hat{\Bk}}
\newcommand{\lcap}[0]{\hat{\Bl}}
\newcommand{\mcap}[0]{\hat{\Bm}}
\newcommand{\ncap}[0]{\hat{\Bn}}
\newcommand{\ocap}[0]{\hat{\Bo}}
\newcommand{\pcap}[0]{\hat{\Bp}}
\newcommand{\qcap}[0]{\hat{\Bq}}
\newcommand{\rcap}[0]{\hat{\Br}}
\newcommand{\scap}[0]{\hat{\Bs}}
\newcommand{\tcap}[0]{\hat{\Bt}}
\newcommand{\ucap}[0]{\hat{\Bu}}
\newcommand{\vcap}[0]{\hat{\Bv}}
\newcommand{\wcap}[0]{\hat{\Bw}}
\newcommand{\xcap}[0]{\hat{\Bx}}
\newcommand{\ycap}[0]{\hat{\By}}
\newcommand{\zcap}[0]{\hat{\Bz}}
\newcommand{\thetacap}[0]{\hat{\Btheta}}

%
% to write R^n and C^n in a distinguishable fashion.  Perhaps change this
% to the double lined characters upon figuring out how to do so.
%
\newcommand{\C}[1]{$\mathbb{C}^{#1}$}
\newcommand{\R}[1]{$\mathbb{R}^{#1}$}

%
% various generally useful helpers
%

% derivative of #1 wrt. #2:
\newcommand{\D}[2] {\frac {d#2} {d#1}}

\newcommand{\inv}[1]{\frac{1}{#1}}
\newcommand{\cross}[0]{\times}

\newcommand{\abs}[1]{\lvert{#1}\rvert}
\newcommand{\norm}[1]{\lVert{#1}\rVert}
\newcommand{\innerprod}[2]{\langle{#1}, {#2}\rangle}
\newcommand{\dotprod}[2]{{#1} \cdot {#2}}
\newcommand{\bdotprod}[2]{\left({#1} \cdot {#2}\right)}
\newcommand{\crossprod}[2]{{#1} \cross {#2}}
\newcommand{\tripleprod}[3]{\dotprod{\left(\crossprod{#1}{#2}\right)}{#3}}

\DeclareMathOperator{\Proj}{Proj}
\DeclareMathOperator{\Span}{span}
\DeclareMathOperator{\Sgn}{sgn}
\DeclareMathOperator{\Area}{Area}
\DeclareMathOperator{\Volume}{Volume}

%
% A few miscellaneous things specific to this document
%
\newcommand{\crossop}[1]{\crossprod{#1}{}}

% R2 vector.
\newcommand{\VectorTwo}[2]{
\begin{bmatrix}
 {#1} \\
 {#2}
\end{bmatrix}
}

\newcommand{\VectorN}[1]{
\begin{bmatrix}
{#1}_1 \\
{#1}_2 \\
\vdots \\
{#1}_N \\
\end{bmatrix}
}

\newcommand{\DETuvij}[4]{
\begin{vmatrix}
 {#1}_{#3} & {#1}_{#4} \\
 {#2}_{#3} & {#2}_{#4}
\end{vmatrix}
}

\newcommand{\DETuvwijk}[6]{
\begin{vmatrix}
 {#1}_{#4} & {#1}_{#5} & {#1}_{#6} \\
 {#2}_{#4} & {#2}_{#5} & {#2}_{#6} \\
 {#3}_{#4} & {#3}_{#5} & {#3}_{#6}
\end{vmatrix}
}

\newcommand{\DETuvwxijkl}[8]{
\begin{vmatrix}
 {#1}_{#5} & {#1}_{#6} & {#1}_{#7} & {#1}_{#8} \\
 {#2}_{#5} & {#2}_{#6} & {#2}_{#7} & {#2}_{#8} \\
 {#3}_{#5} & {#3}_{#6} & {#3}_{#7} & {#3}_{#8} \\
 {#4}_{#5} & {#4}_{#6} & {#4}_{#7} & {#4}_{#8} \\
\end{vmatrix}
}

%\newcommand{\DETuvwxyijklm}[10]{
%\begin{vmatrix}
% {#1}_{#6} & {#1}_{#7} & {#1}_{#8} & {#1}_{#9} & {#1}_{#10} \\
% {#2}_{#6} & {#2}_{#7} & {#2}_{#8} & {#2}_{#9} & {#2}_{#10} \\
% {#3}_{#6} & {#3}_{#7} & {#3}_{#8} & {#3}_{#9} & {#3}_{#10} \\
% {#4}_{#6} & {#4}_{#7} & {#4}_{#8} & {#4}_{#9} & {#4}_{#10} \\
% {#5}_{#6} & {#5}_{#7} & {#5}_{#8} & {#5}_{#9} & {#5}_{#10}
%\end{vmatrix}
%}

% R3 vector.
\newcommand{\VectorThree}[3]{
\begin{bmatrix}
 {#1} \\
 {#2} \\
 {#3}
\end{bmatrix}
}



\author{Peeter Joot}
\email{peeter.joot@gmail.com}


\chapter{PHY454H1S Continuum mechanics.  Problem Set 2.  XXX}
%\label{chap:continuumProblemSet2}
%\blogpage{http://sites.google.com/site/peeterjoot2/math2012/continuumProblemSet2.pdf}
%\date{Mar 2, 2012}
%\gitRevisionInfo{continuumProblemSet2}

\beginArtNoToc
%\section{Disclaimer.}
%
%This problem set is as yet ungraded.

\section{Problem Q1.}
\subsection{Statement}

Imagine a steady rectilinear blood flow of the form $\Bu = u(y) \ycap$ through a two dimensional artery.  It is driven by a constant pressure gradient $G = -dp/dx$ maintained by an external `heart'.  The top and bottom walls of the artery are $2h$ distance apart and the fluid satisfies no-slip boundary conditions at the walls.  Assuming that the fluid is Newtonian,

\begin{enumerate}
\item Show that the Navier-Stokes equation reduces to

\begin{equation}\label{eqn:continuumProblemSet2:20}
\frac{d^2 u}{dy^2} = -\frac{G}{\mu}
\end{equation}

where $\mu$ is the viscosity of the blood.

\item Show that the velocity profile of the fluid inside the artery is a parabolic profile.
\item What is the maximum speed of the fluid?  Draw the velocity profile to show where the maximum speed occurs inside the artery.
\item If due to smoking etc., the viscosity of the blood gets doubled, then what should be the new pressure gradient to be maintained by the `heart' to keep the liquid flux through the artery at the same level as the non-smoking one?
\end{enumerate}

\subsection{Solution.  Part 1.  Navier-Stokes equation for the system.}

The Navier-Stokes equation, for an incompressible unidirectional fluid $\Bu = (u, 0, 0)$, assuming that there is no $z$ dependence, takes the form

\begin{subequations}
\begin{equation}\label{eqn:continuumProblemSet2:40}
\rho \PD{t}{u} + u \PD{x}{u} = - \PD{x}{p} + \mu \left( \PDSq{x}{} + \PDSq{y}{} \right) u
\end{equation}
\begin{equation}\label{eqn:continuumProblemSet2:60}
0 = - \PD{y}{p} 
\end{equation}
\begin{equation}\label{eqn:continuumProblemSet2:80}
0 = - \PD{z}{p} 
\end{equation}
\begin{equation}\label{eqn:continuumProblemSet2:100}
0 = \PD{x}{u}.
\end{equation}
\end{subequations}

With a steady state assumption we kill the $\PDi{t}{u}$ term, and \ref{eqn:continuumProblemSet2:100} kills of the x-component of the Laplacian and our non-linear inertial term on the LHS, leaving just

\begin{subequations}
\begin{equation}\label{eqn:continuumProblemSet2:40a}
0 = - \PD{x}{p} + \mu \PDSq{y}{u} 
\end{equation}
\begin{equation}\label{eqn:continuumProblemSet2:60b}
0 = - \PD{y}{p} 
\end{equation}
\begin{equation}\label{eqn:continuumProblemSet2:80c}
0 = - \PD{z}{p}.
\end{equation}
\end{subequations}

With $\PD{z}{p} = \PD{y}{p} = 0$, we have $\PD{x}{p} = dp/dx = -G$, so \ref{eqn:continuumProblemSet2:40a} is reduced to

\begin{equation}\label{eqn:continuumProblemSet2:40a}
0 = G + \mu \PDSq{y}{u}.
\end{equation}

Finally, since we have an assumption of no z-dependence ($\PDi{z}{u} = 0$) and from the incompressibility assumption \ref{eqn:continuumProblemSet2:100} ($\PDi{x}{u} = 0$), we have

\begin{equation}\label{eqn:continuumProblemSet2:140}
\PDSq{y}{u} = \frac{d^2 u}{dy^2} = -\frac{G}{\mu},
\end{equation}

as desired.

\subsection{Solution.  Part 2.  Velocity profile.}

Integrating \ref{eqn:continuumProblemSet2:140} twice, we have

\begin{equation}\label{eqn:continuumProblemSet2:160}
u = -\frac{G}{2 \mu} y^2 + A y + B.
\end{equation}

Application of the no-slip boundary value condition $u(\pm h) = 0$, we have

\begin{align}\label{eqn:continuumProblemSet2:180}
0 &= -\frac{G}{2 \mu} h^2 + A h + B \\
0 &= -\frac{G}{2 \mu} h^2 - A h + B 
\end{align}

Adding and subtracting these, we find

\begin{subequations}
\begin{equation}\label{eqn:continuumProblemSet2:280}
A = 0 
\end{equation}
\begin{equation}\label{eqn:continuumProblemSet2:300}
B = \frac{G}{2 \mu} h^2,
\end{equation}
\end{subequations}

so the velocity is given by the parabolic function

\begin{equation}\label{eqn:continuumProblemSet2:220}
u(y) = \frac{G}{2 \mu} \left( h^2 - y^2 \right).
\end{equation}

\subsection{Solution.  Part 3.  Maximum speed of the flow.}

It is clear that the maximum speed of the fluid is found at $y = 0$

\begin{equation}\label{eqn:continuumProblemSet2:240}
u(0) = \frac{G h^2}{2 \mu}
\end{equation}

The velocity profile for this flow is drawn in figure (\ref{fig:continuumProblemSet2:continuumProblemSet2Fig1}).

\begin{figure}[htp]
   \centering
   \includegraphics[totalheight=0.2\textheight]{continuumProblemSet2Fig1}
   \caption{Velocity profile for 1D constant pressure gradient steady state flow.}\label{fig:continuumProblemSet2:continuumProblemSet2Fig1}
\end{figure}

\subsection{Solution.  Part 4.  Effects of viscosity doubling.}

With our velocity being dependent on the $G/\mu$ ratio, it is clear, even without calculating the flux, that we will need  twice the pressure gradient if the viscosity is doubled to maintain the same flux through the artery and veins.  To demonstrate this more thoroughly we can calculate this mass flux.  For an element of mass leaving a portion of the conduit, bounded by the plane normal to $\xcap$ we have

\begin{align*}
\frac{dm}{dt} 
&= \rho \frac{dV}{dt} \\
&= \rho dz dy \Bu \cdot \xcap
\end{align*}

Integrating this over a width $\Delta z$, our flux through the plane is

\begin{align*}
\text{Flux} 
&= 
\int_0^{\Delta z} dz
\int_{-h}^h dy \frac{G}{2 \mu} \left( h^2 - y^2 \right) \\
&= 
\Delta z 
\frac{G}{2 \mu} 
\evalrange{ \left( h^2 y - \inv{3} y^3 \right) }{-h}{h} \\
&=
\Delta z 
\frac{G h^3}{\mu} \left( 1 - \inv{3} \right)  \\
&=
\Delta z \frac{2 G h^3}{3 \mu}.
\end{align*}

Doubling the blood viscosity for our smoker, our respective fluxes are

\begin{align}\label{eqn:continuumProblemSet2:320}
\text{Flux}_{\text{smoker}} &= \Delta z \frac{2 G_{\text{smoker}} h^3}{3 (2 \mu)}  \\
\text{Flux}_{\text{non-smoker}} &= \Delta z \frac{2 G h^3}{3 \mu}.
\end{align}

Demanding equality before and after smoking we find

\begin{equation}\label{eqn:continuumProblemSet2:260}
G_{\text{smoker}} = 2 G.
\end{equation}

where $G$ is the magnitude of the pressure gradient before the bad habits kicked in.  The smoker's poor little heart (soon to be a big overworked and weak heart) has to generate pressure gradients that are twice as big to get the same quantity of blood distributed through the body.

\section{Problem Q2.}
\subsection{Statement}

Consider steady simple shearing flow with no imposed pressure gradient $(G = 0)$ of a two layer fluid with viscosity

\begin{equation}\label{eqn:continuumProblemSet2:120}
\mu =
\left\{
\begin{array}{l l}
\mu_1 & \quad \mbox{$-h < y < 0,$} \\
\mu_2 & \quad \mbox{$0 < y < h.$}
\end{array}
\right.
\end{equation}

The boundary conditions are no-slip at the lower plate $(y = -h)$ and at $y = 0$.  The top plate is moving with a velocity $-U$ at $y = h$ and fluid is sticking to it.  using the continuity of tangential (shear) stress at the interface ($y = 0$)

\begin{itemize}
\item Derive the velocity profile of the two fluids.
\item Calculate the maximum speed.
\item Calculate the mean speed.
\item Calculate the flux (the volume flow rate.)
\item Calculate the tangential force (per unit width) $F_x$ on the strip $0 \le x \le L$ of the wall $y = -h$.
\item Calculate the tangential force (per unit width) $F_x^0$ on the strip $0 \le x \le L$ at the interface $y = 0$ by the top fluid on the lower fluid.
\end{itemize}

\subsection{Part 1.  Derive the velocity profile of the two fluids.}
\subsection{Part 2.  Calculate the maximum speed.}
\subsection{Part 3.  Calculate the mean speed.}
\subsection{Part 4.  Calculate the flux}
\subsection{Part 5.  Tangential force on the wall.}
\subsection{Part 6.  Tangential force on the lower fluid.}

\section{Problem Q3.}
\subsection{Statement}

If on top of the problem described above a constant pressure gradient $G = -dp/dx$ is applied between the boundaries $y = \pm h$, describe qualitatively what type of flow profile you would expect in the steady state.  Draw the velocity profiles for two cases (i) $\mu_1 > \mu_2$ (ii) $\mu_1 < \mu_2$.  Explain your result.

\subsection{Solution}

%\EndArticle
\EndNoBibArticle
