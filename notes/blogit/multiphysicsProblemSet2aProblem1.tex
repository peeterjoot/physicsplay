%
% Copyright � 2014 Peeter Joot.  All Rights Reserved.
% Licenced as described in the file LICENSE under the root directory of this GIT repository.
%
\makeproblem{Conjugate gradient and Gershgorin circles}{multiphysics:problemSet2a:1}{ 
\makesubproblem{}{multiphysics:problemSet2a:1a}

Consider an arbitrary circuit made by positive resistors and independent DC current sources. Prove, mathematically, that its nodal matrix is always symmetric and positive semi-definite.

\makesubproblem{}{multiphysics:problemSet2a:1b}

Consider an electrical network made by:
\begin{itemize}
\item \( N \times N \) square grid of resistors of value \( R \), where \( N \) is the number of resistors per edge. The grid nodes are numbered from \( 1 \) to \( (N +1)^2 \) .  Node 1 is a corner node
\item resistor \( R_g \) between each node of the grid and the reference node.
\item a non-ideal voltage source connected between node 1 and ground.  The voltage source has value \( V_s \) and internal (series) resistance \( R_s \)
\item three current sources between three randomly-selected nodes of the grid and ground. The source current must flow from the grid to the reference node as shown in \cref{fig:problemSet2a:problemSet2aFig1} (and not viceversa!).  Choose the current source values randomly between 10 mA and 100 mA.

\imageFigure{../../figures/ece1254/problemSet2aFig1}{Subcircuit}{fig:problemSet2a:problemSet2aFig1}{0.3}
\end{itemize}
Write a MATLAB routine that generates a SPICE compatible netlist for this system (you can reuse the code you developed for the first problem set). Generate the modified nodal analysis equations

\begin{equation}\label{eqn:multiphysicsProblemSet2a:20}
\BG \Bx = \Bb 
\end{equation}

for a grid with \( N = 40, R = 0.1 \Omega, R_g = 1M \Omega, V_s = 2V, R_s = 0.1 \Omega\).
Then, write in MATLAB your own routine for the conjugate gradient
method. Give to the user the possibility of specifying a preconditioning
matrix P. The routine shall stop iterations when the residual norm
satisfies

\begin{equation}\label{eqn:multiphysicsProblemSet2a:40}
\frac{\Norm{ \BG \Bx - \Bb }_2}{
\Norm{\Bb}_2} < \epsilon
\end{equation}

where \( \epsilon \) is a threshold specified by the user. Use the conjugate gradient method to solve modified nodal analysis \cref{eqn:multiphysicsProblemSet2a:20} of the grid.

Does the conjugate gradient method converge?

\makesubproblem{}{multiphysics:problemSet2a:1c}
Discuss if the conjugate gradient method can be applied to the modified nodal analysis equations of the grid. Support your answer with some numerical results.

\makesubproblem{}{multiphysics:problemSet2a:1d}

Suggest a transformation of the grid that will lead to a circuit equivalent to the original one, but for which conjugate gradient can be used. We call this new circuit ``2D grid''. You will have to use it for all the questions that follow.

\makesubproblem{}{multiphysics:problemSet2a:1e}

Solve the ``2D grid'' circuit using three methods:
\begin{itemize}
\item your own LU decomposition
\item the conjugate gradient method with \( \epsilon = 10^{-3} \)
\item the conjugate gradient method with \( \epsilon = 10^{-3} \) and a tri-diagonal preconditioner \( \BP \)
\end{itemize}
for increasing N. Use \( R = 0.1 \Omega \). Plot the CPU time taken by the three
methods vs N, and the number of iterations taken by the two iterative
methods. Explore a range of N compatible with your PC speed and
memory, but make sure to reach fairly large values of N.

\makesubproblem{}{multiphysics:problemSet2a:1f}
Does preconditioning reduce the number of iterations required by conjugate gradient? Does preconditioning reduce CPU time?
\makesubproblem{}{multiphysics:problemSet2a:1g}
Try to make your code for the conjugate gradient method as efficient as possible. Show the improvements that you have obtained.
\makesubproblem{}{multiphysics:problemSet2a:1h}
Generate nodal analysis \cref{eqn:multiphysicsProblemSet2a:20} for ``2D grid'' with
\( N = 20\). Plot the eigenvalues of \( \BG \) and the Gershgorin circles in three
cases: \( R = 0.1 \Omega, R = 1 \Omega, R = 10 \Omega \). Verify Gershgoring Circle theorem
in the three cases.
\makesubproblem{}{multiphysics:problemSet2a:1i}
 Repeat the previous point for the preconditioned nodal matrix, and compare the circles obtained in the two cases.
\makesubproblem{}{multiphysics:problemSet2a:1j}

Let \( N = 20\), and solve ``2D grid'' with conjugate gradient with
and without preconditioning (use the tri-diagonal preconditioner). Plot
the normalized norm of the residual

\begin{equation}\label{eqn:multiphysicsProblemSet2a:60}
\frac{\Norm{ \BG \Bx - \Bb }_2}{
\Norm{\Bb}_2} 
\end{equation}
versus the iteration counter for three cases: \( R = 0.1\Omega, R = 1\Omega, R =
10\Omega\). Discuss your findings in the light of Gershgorin circles.
} % makeproblem

\makeanswer{multiphysics:problemSet2a:1}{ 
\makeSubAnswer{}{multiphysics:problemSet2a:1a}

We've seen in class that a circuit with resistors and voltage sources has the MNA form

\begin{equation}\label{eqn:multiphysicsProblemSet2a:80}
\begin{bmatrix}
G & -A_V^\T \\
A_V & 0 
\end{bmatrix}
\begin{bmatrix}
\BV \\
\BI
\end{bmatrix}
=
\begin{bmatrix}
\BI_S \\
\BV_S
\end{bmatrix}.
\end{equation}

First let's show that the \textAndIndex{nodal matrix} \( \BG \) is symmetric.  Recall that 
%for resistor node specification corresponding to \cref{fig:lecture2:lecture2Fig2} 
this was formed as the sum of stamp matrices of the form
%\imageFigure{../../figures/ece1254/lecture2Fig2}{Resistor node specification}{fig:lecture2:lecture2Fig2}{0.3}

\begin{equation}\label{eqn:multiphysicsProblemSet2a:n}
\kbordermatrix{
    & n_1      & n_2 \\
n_1 & \inv{R}  & -\inv{R} \\
n_2 & -\inv{R} & \inv{R}
}
\end{equation}



\makeSubAnswer{}{multiphysics:problemSet2a:1b}

TODO.
\makeSubAnswer{}{multiphysics:problemSet2a:1c}

TODO.
\makeSubAnswer{}{multiphysics:problemSet2a:1d}

TODO.
\makeSubAnswer{}{multiphysics:problemSet2a:1e}

TODO.
\makeSubAnswer{}{multiphysics:problemSet2a:1f}

TODO.
\makeSubAnswer{}{multiphysics:problemSet2a:1g}

TODO.
\makeSubAnswer{}{multiphysics:problemSet2a:1h}

TODO.
\makeSubAnswer{}{multiphysics:problemSet2a:1i}

TODO.
\makeSubAnswer{}{multiphysics:problemSet2a:1j}

TODO.
}

