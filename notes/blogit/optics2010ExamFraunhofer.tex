%
% Copyright � 2012 Peeter Joot.  All Rights Reserved.
% Licenced as described in the file LICENSE under the root directory of this GIT repository.
%
% pick one:
%\newcommand{\authorname}{Peeter Joot}
\newcommand{\email}{peeter.joot@utoronto.ca}
\newcommand{\studentnumber}{920798560}
\newcommand{\basename}{FIXMEbasenameUndefined}
\newcommand{\dirname}{notes/FIXMEdirnameUndefined/}

\newcommand{\authorname}{Peeter Joot}
\newcommand{\email}{peeterjoot@protonmail.com}
\newcommand{\basename}{FIXMEbasenameUndefined}
\newcommand{\dirname}{notes/FIXMEdirnameUndefined/}

\renewcommand{\basename}{optics2010ExamFraunhofer}
\renewcommand{\dirname}{notes/phy485/}
%\newcommand{\dateintitle}{}
%\newcommand{\keywords}{}

\newcommand{\authorname}{Peeter Joot}
\newcommand{\onlineurl}{http://sites.google.com/site/peeterjoot2/math2013/\basename.pdf}
\newcommand{\sourcepath}{\dirname\basename.tex}
\newcommand{\generatetitle}[1]{\chapter{#1}}

\newcommand{\vcsinfo}{%
\section*{}
\noindent{\color{DarkOliveGreen}{\rule{\linewidth}{0.1mm}}}
\paragraph{Document version}
%\paragraph{\color{Maroon}{Document version}}
{
\small
\begin{itemize}
\item Available online at:\\ 
\href{\onlineurl}{\onlineurl}
\item Git Repository: \input{./.revinfo/gitRepo.tex}
\item Source: \sourcepath
\item last commit: \input{./.revinfo/gitCommitString.tex}
\item commit date: \input{./.revinfo/gitCommitDate.tex}
\end{itemize}
}
}

%\PassOptionsToPackage{dvipsnames,svgnames}{xcolor}
\PassOptionsToPackage{square,numbers}{natbib}
\documentclass{scrreprt}

\usepackage[left=2cm,right=2cm]{geometry}
\usepackage[svgnames]{xcolor}
\usepackage{peeters_layout}

\usepackage{natbib}

\usepackage[
colorlinks=true,
bookmarks=false,
pdfauthor={\authorname, \email},
backref 
]{hyperref}

% http://tex.stackexchange.com/questions/75773/how-to-reference-problems-by-the-text-label-in-an-exercise-envioronment
\usepackage[english]{cleveref}
\crefname{Exercise}{exercise}{exercises}
\Crefname{Exercise}{Exercise}{Exercises}

\RequirePackage{titlesec}
\RequirePackage{ifthen}

% http://stackoverflow.com/questions/4932910/date-in-the-tabular-environment
\makeatletter
\let\insertdate\@date
\makeatother

\titleformat{\chapter}[display]
{\bfseries\Large}
{\color{DarkSlateGrey}\filleft \authorname
\ifthenelse{\isundefined{\studentnumber}}{}{\\ \studentnumber}
\ifthenelse{\isundefined{\email}}{}{\\ \email}
\ifthenelse{\isundefined{\dateintitle}}{}{\\ \insertdate}
%\ifthenelse{\isundefined{\coursename}}{}{\\ \coursename} % put in title instead.
}
{4ex}
{\color{DarkOliveGreen}{\titlerule}\color{Maroon}
\vspace{2ex}%
\filright}
[\vspace{2ex}%
\color{DarkOliveGreen}\titlerule
]

\newcommand{\beginArtWithToc}[0]{\begin{document}\tableofcontents}
\newcommand{\beginArtNoToc}[0]{\begin{document}}
\newcommand{\EndNoBibArticle}[0]{\end{document}}
\newcommand{\EndArticle}[0]{\bibliography{Bibliography}\bibliographystyle{plainnat}\end{document}}

% 
%\newcommand{\citep}[1]{\cite{#1}}

\colorSectionsForArticle



\beginArtNoToc

\generatetitle{Fraunhofer diffraction pattern for four circular aperatures}
\chapter{Fraunhofer diffraction pattern for four circular aperatures}
\label{chap:\basename}
%\section{Motivation}
%\section{Guts}

\makeproblem{Fraunhofer diffraction through four circular aperatures}{pr:optics2010ExamFraunhofer:1}{ 

Calculate the diffraction pattern for the geometry of \cref{fig:optics2010ExamFraunhofer:optics2010ExamFraunhoferFig2}.

\imageFigure{optics2010ExamFraunhoferFig2}{Four circular aperatures}{fig:optics2010ExamFraunhofer:optics2010ExamFraunhoferFig2}{0.3}

} % makeproblem
\makeanswer{pr:optics2010ExamFraunhofer:1}{ 

We are working with distances illustrated in \cref{fig:optics2010ExamFraunhofer:optics2010ExamFraunhoferFig1}.

\imageFigure{optics2010ExamFraunhoferFig1}{Four aperatures with observation point and distances}{fig:optics2010ExamFraunhofer:optics2010ExamFraunhoferFig1}{0.3}

As usual we write

\begin{subequations}
\begin{dmath}\label{eqn:optics2010ExamFraunhofer:20}
\BR = \Br' - \Br_s
\end{dmath}
\begin{dmath}\label{eqn:optics2010ExamFraunhofer:40}
R 
= 
r' \left( 1 + \frac{r_s^2}{{r'}^2} - 2 \frac{\Br_s \cdot \Br'}{{r'}^2} \right)^{1/2}
\approx
r' + \frac{r_s^2}{2 {r'}} - \Br_s \cdot \rcap' 
\end{dmath}
\end{subequations}

so that

\begin{dmath}\label{eqn:optics2010ExamFraunhofer:60}
\Psi(\Br') = \frac{\Psi_0}{e^{i k r'}}{i \lambda r'} \int e^{-i k \Br_s \cdot \rcap'}
\end{dmath}

} % makeanswer
% this is to produce the sites.google url and version info and so forth (for blog posts)
%\vcsinfo
%\EndArticle
\EndNoBibArticle
