\makeproblem{Stability in a cavity}{modernOptics:problemSet4:2}{ 
Using ABCD matrices, derive the condition for stability of a Gaussian beam in a cavity. 

\makesubproblem{Assuming symmetric beams.}{modernOptics:problemSet4:2a}
%{\bf (a)}
Follow Fowles \S 10.5 to find Eq.~(10.32). 

\makesubproblem{Stability criterion.}{modernOptics:problemSet4:2b}
%{\bf (b)}
Fowles derives the stability criterion from the eigen{\em{values}}. Find the rays that at the eigen{\em{vectors}} of a single-pass ray matrix for the planar case ($L/R = 0$). Comment on why they are the only reasonable choice. 

\makesubproblem{Unequal mirror radii.}{modernOptics:problemSet4:2c}
%{\bf (c)}
Allowing for unequal mirror radii, derive Fowles Eq.~(10.33).
}

\makeanswer{modernOptics:problemSet4:2}{ 
\makeSubAnswer{Assuming symmetric beams}{modernOptics:problemSet4:2a}

Given the M\"obius transform relationship the Gaussian beams, we can first consider the stability conditions for powers of the transfer matrix itself to not diverge.  This follows \citep{fowles1989introduction}, filling in some additional details.

The matrix for a single pass of free propagation through distance $d$, and then reflection off of a curved mirror with focus $f$ is the composition

\begin{equation}\label{eqn:problemSet4Problem4:20}
M = 
\begin{bmatrix}
1 & 0 \\
-1/f & 1
\end{bmatrix}
\begin{bmatrix}
1 & d \\
0 & 1
\end{bmatrix}
=
\begin{bmatrix}
1 & d \\
-1/f &  -d/f + 1
\end{bmatrix}.
\end{equation}

Each pass of propagation and reflection adds another power of $M$ to the matrix for which we will base the final M\"obius transformation on.  We'll want to perform a diagonalization to simplify that matrix exponentiation, so that if

\begin{equation}\label{eqn:problemSet4Problem4:40}
M E = E D,
\end{equation}

or

\begin{equation}\label{eqn:problemSet4Problem4:60}
M = E D E^{-1},
\end{equation}

We can express the final matrix transformation after $n$ reflections directly

\begin{equation}\label{eqn:problemSet4Problem4:80}
M^n = E D^n E^{-1}.
\end{equation}

We'll need the eigenvalues first.  Our characteristic equation is

\begin{dmath}\label{eqn:problemSet4Problem4:100}
0 
= \Abs{ M - \lambda I } 
=
\begin{vmatrix}
1 - \lambda & d \\
-1/f & 1 - d/f - \lambda
\end{vmatrix}
=
1 - \cancel{d/f} - \lambda - \lambda( 1 - d/f - \lambda ) + \cancel{d/f}
= 
\lambda^2 - 2 \lambda \left( 1 - \frac{d}{2 f} \right) + 1.
\end{dmath}

Following Fowles, we write

\begin{dmath}\label{eqn:problemSet4Problem4:120}
\alpha = 1 - \frac{d}{2 f},
\end{dmath}

So that the characteristic equation is

\begin{dmath}\label{eqn:problemSet4Problem4:140}
0
= 
\lambda^2 - 2 \lambda \alpha + 1
= 
\left( \lambda - \alpha \right)^2 + 1 - \alpha^2
\end{dmath}

Note that this corrects a sign error in the text.  Solving for $\lambda$ we have

\begin{dmath}\label{eqn:problemSet4Problem4:160}
\lambda_{\pm} = \alpha \pm \sqrt{\alpha^2 - 1}.
\end{dmath}

Observe that these satisfy our expectation that $\lambda_{+} \lambda_{-} = 1$, so we can write these as 

\begin{align}\label{eqn:problemSet4Problem4:300}
\lambda_{+} &= \lambda \\
\lambda_{-} &= 1/\lambda,
\end{align}

for some value $\lambda$.  Our ABCD matrix for $n$ sets of propagate-and-reflect is

\begin{dmath}\label{eqn:problemSet4Problem4:260}
M^n = E 
\begin{bmatrix}
\lambda^n & 0 \\
0 & \inv{\lambda^n}
\end{bmatrix}
E^{-1}.
\end{dmath}

If $E = [ e_{i j} ]$ and $E^{-1} = [ f_{i j} ]$, then the product takes the value

\begin{dmath}\label{eqn:problemSet4Problem4:280}
E D^n E^{-1} 
= [ e_{i k} \lambda_k^n \delta_{k m } f_{m j} ]
= [ e_{i m} \lambda_m^n f_{m j} ]
= [ 
e_{i 1} \lambda^n f_{1 j} 
+
e_{i 2} \inv{\lambda^n} f_{2 j} 
],
\end{dmath}

or
\begin{dmath}\label{eqn:problemSet4Problem4:320}
M^n
= 
\lambda^n [ e_{i 1} f_{1 j} ] + \inv{\lambda^n} [ e_{i 2} f_{2 j} ].
\end{dmath}

So, if $\lambda$ is real and greater than 1, the ABCD matrix will start to grow without bound.

To consider the bounding behaviour of $\lambda^n$, lets follow Fowles, and separate the eigenvalues into real and complex as follows

\begin{dmath}\label{eqn:problemSet4Problem4:180}
\lambda = \alpha \pm 
\left\{
\begin{array}{l l}
\sqrt{\alpha^2 - 1} & \quad \mbox{if $\Abs{\alpha} > 1$} \\
i \sqrt{1 - \alpha^2} & \quad \mbox{if $\Abs{\alpha} < 1$} 
\end{array}
\right.
\end{dmath}

When $\Abs{\alpha} = 1$, we have a double eigenvalue with value $\alpha$ and may not be able to find a spanning set of eigenvectors.  For the $\Abs{\alpha} < 1$ case, we can introduce $\phi$ such that $\cos\phi = \alpha$, allowing us to write

\begin{dmath}\label{eqn:problemSet4Problem4:200}
\lambda_{\pm} = e^{\pm i \phi}
\end{dmath}

What can we say about the real valued eigenvalue case?  We know that from $\Det M = 1$ both real valued eigenvalues must have matching signs.  We can also see from

\begin{dmath}\label{eqn:problemSet4Problem4:240}
\lambda_{\pm} = \Abs{\alpha} \left( \sgn(\alpha) \pm \sqrt{1 - \inv{\alpha^2}} \right),
\end{dmath}

that if these are positive, then one of the eigenvalues is greater than 1, and if negative, at least one is less than $-1$.  If we pick $\lambda$ as the eigenvalue for which $\Abs{\lambda} > 1$, then $\lambda^n$ will clearly diverge as $n$ grows, and we the ABCD matrix of \ref{eqn:problemSet4Problem4:320} becomes unstable.

%If considering just rays, for the rays to be constrained we'd need the resulting angle to be constrained.

\makeSubAnswer{Stability criterion}{modernOptics:problemSet4:2b}

TODO.
\makeSubAnswer{Unequal mirror radii}{modernOptics:problemSet4:2c}

TODO.
}
