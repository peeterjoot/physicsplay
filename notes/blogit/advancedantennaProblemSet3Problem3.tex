%
% Copyright � 2015 Peeter Joot.  All Rights Reserved.
% Licenced as described in the file LICENSE under the root directory of this GIT rep{\textrm{LOS}}itory.
%
\makeproblem{Corner cube antenna.}{advancedantenna:problemSet3:3}{ 

Consider a symmetrically placed horizontal dipole antenna, next to a metallic corner cube.

\imageFigure{../../figures/ece1229/homework3Fig1}{A corner-cube antenna.}{fig:homework3:homework3Fig1}{0.2}

\makesubproblem{}{advancedantenna:problemSet3:3a}

Estimate the directivity enhancement of the antenna in \cref{fig:homework3:homework3Fig1} compared to the isolated antenna.

\makesubproblem{}{advancedantenna:problemSet3:3b}

Estimate the radiation resistance of the antenna in \cref{fig:homework3:homework3Fig1} compared to the isolated antenna.

\makesubproblem{}{advancedantenna:problemSet3:3c}

Calculate the array factor of the antenna in \cref{fig:homework3:homework3Fig1}.

\makesubproblem{}{advancedantenna:problemSet3:3d}

Plot the array-factor directivity pattern in the x-y plane for \( 0 < \phi \le 2 \pi \).

\makesubproblem{}{advancedantenna:problemSet3:3e}

By using numerical integration calculate the directivity of the array factor for \( h = (1/8) \lambda, h = (1/4) \lambda \) and \( h = (1/2) \lambda \).

} % makeproblem

\makeanswer{advancedantenna:problemSet3:3}{ 

To compare intensity with and without the corner cube, it probably makes sense to set the origin at the location of the radiator in the corner-cube antenna, as sketched in \cref{fig:cornerCubeGeometry:cornerCubeGeometryFig2}.

\imageFigure{../../figures/ece1229/cornerCubeGeometryFig2}{Corner cube geometry.}{fig:cornerCubeGeometry:cornerCubeGeometryFig2}{0.3}

There are three wave vectors directions of interest, \( \kcap_{\textrm{LOS}} \) for the line of sight direction, \( \kcap_{\textrm{zx}} \) for the reflection originating at the image source behind the z-x plane at \( -2 h \ycap \), and \( \kcap_{\textrm{yz}} \) for the reflection originating at the image source behind the y-z plane at \( -2 h \xcap \).

\makeSubAnswer{}{advancedantenna:problemSet3:3a}

TODO.
\makeSubAnswer{}{advancedantenna:problemSet3:3b}

TODO.
\makeSubAnswer{}{advancedantenna:problemSet3:3c}

TODO.
\makeSubAnswer{}{advancedantenna:problemSet3:3d}

TODO.
\makeSubAnswer{}{advancedantenna:problemSet3:3e}

TODO.
}
