%
% Copyright � 2015 Peeter Joot.  All Rights Reserved.
% Licenced as described in the file LICENSE under the root directory of this GIT repository.
%
\newcommand{\authorname}{Peeter Joot}
\newcommand{\email}{peeterjoot@protonmail.com}
\newcommand{\basename}{FIXMEbasenameUndefined}
\newcommand{\dirname}{notes/FIXMEdirnameUndefined/}

\renewcommand{\basename}{qmLecture13}
\renewcommand{\dirname}{notes/phy1520/}
\newcommand{\keywords}{PHY1520H}
\newcommand{\authorname}{Peeter Joot}
\newcommand{\onlineurl}{http://sites.google.com/site/peeterjoot2/math2013/\basename.pdf}
\newcommand{\sourcepath}{\dirname\basename.tex}
\newcommand{\generatetitle}[1]{\chapter{#1}}

\newcommand{\vcsinfo}{%
\section*{}
\noindent{\color{DarkOliveGreen}{\rule{\linewidth}{0.1mm}}}
\paragraph{Document version}
%\paragraph{\color{Maroon}{Document version}}
{
\small
\begin{itemize}
\item Available online at:\\ 
\href{\onlineurl}{\onlineurl}
\item Git Repository: \input{./.revinfo/gitRepo.tex}
\item Source: \sourcepath
\item last commit: \input{./.revinfo/gitCommitString.tex}
\item commit date: \input{./.revinfo/gitCommitDate.tex}
\end{itemize}
}
}

%\PassOptionsToPackage{dvipsnames,svgnames}{xcolor}
\PassOptionsToPackage{square,numbers}{natbib}
\documentclass{scrreprt}

\usepackage[left=2cm,right=2cm]{geometry}
\usepackage[svgnames]{xcolor}
\usepackage{peeters_layout}

\usepackage{natbib}

\usepackage[
colorlinks=true,
bookmarks=false,
pdfauthor={\authorname, \email},
backref 
]{hyperref}

% http://tex.stackexchange.com/questions/75773/how-to-reference-problems-by-the-text-label-in-an-exercise-envioronment
\usepackage[english]{cleveref}
\crefname{Exercise}{exercise}{exercises}
\Crefname{Exercise}{Exercise}{Exercises}

\RequirePackage{titlesec}
\RequirePackage{ifthen}

% http://stackoverflow.com/questions/4932910/date-in-the-tabular-environment
\makeatletter
\let\insertdate\@date
\makeatother

\titleformat{\chapter}[display]
{\bfseries\Large}
{\color{DarkSlateGrey}\filleft \authorname
\ifthenelse{\isundefined{\studentnumber}}{}{\\ \studentnumber}
\ifthenelse{\isundefined{\email}}{}{\\ \email}
\ifthenelse{\isundefined{\dateintitle}}{}{\\ \insertdate}
%\ifthenelse{\isundefined{\coursename}}{}{\\ \coursename} % put in title instead.
}
{4ex}
{\color{DarkOliveGreen}{\titlerule}\color{Maroon}
\vspace{2ex}%
\filright}
[\vspace{2ex}%
\color{DarkOliveGreen}\titlerule
]

\newcommand{\beginArtWithToc}[0]{\begin{document}\tableofcontents}
\newcommand{\beginArtNoToc}[0]{\begin{document}}
\newcommand{\EndNoBibArticle}[0]{\end{document}}
\newcommand{\EndArticle}[0]{\bibliography{Bibliography}\bibliographystyle{plainnat}\end{document}}

% 
%\newcommand{\citep}[1]{\cite{#1}}

\colorSectionsForArticle



%\usepackage{phy1520}
\usepackage{peeters_braket}
%\usepackage{peeters_layout_exercise}
\usepackage{peeters_figures}
\usepackage{mathtools}

\beginArtNoToc
\generatetitle{PHY1520H Graduate Quantum Mechanics.  Lecture 13: Time reversal (cont.), and angular momentum.  Taught by Prof.\ Arun Paramekanti}
%\chapter{Time reversal (cont.)}
\label{chap:qmLecture13}

\paragraph{Disclaimer}

Peeter's lecture notes from class.  These may be incoherent and rough.

These are notes for the UofT course PHY1520, Graduate Quantum Mechanics, taught by Prof. Paramekanti, covering \textchapref{{4}} \citep{sakurai2014modern} content.

\section{Time reversal (cont.)}

Given a time reversed state

\begin{dmath}\label{eqn:qmLecture13:20}
\ket{\tilde{\Psi}(t)} = \Theta \ket{\Psi(0)}
\end{dmath}

which can alternately be written

\begin{equation}\label{eqn:qmLecture13:40}
\Theta^{-1} \ket{\tilde{\Psi}(t)} = \ket{\Psi(-t)} = e^{i \hat{H} t/\Hbar} \ket{\Psi(0)}
\end{equation}

The left hand side can be expanded as the evolution of the state as found at time \( -t \)

\begin{dmath}\label{eqn:qmLecture13:60}
\Theta^{-1} \ket{\tilde{\Psi}(t)} 
=
\Theta^{-1} e^{-i \hat{H} t/\Hbar} \ket{\tilde{\Psi}(-t)} 
=
\Theta^{-1} e^{-i \hat{H} t/\Hbar} \Theta \ket{\Psi(0)}.
\end{dmath}

To first order for a small time increment \( \delta t \), we have

\begin{dmath}\label{eqn:qmLecture13:80}
\lr{ 1 + i \frac{\hat{H}}{\Hbar} \delta t } \ket{\Psi(0)} = 
\Theta^{-1} \lr{ 1 - i \frac{\hat{H}}{\Hbar} \delta t } \Theta \ket{\Psi(0)},
\end{dmath}

or

\begin{dmath}\label{eqn:qmLecture13:120}
i \frac{\hat{H}}{\Hbar} \delta t \ket{\Psi(0)}
=
\Theta^{-1} (- i) \frac{\hat{H}}{\Hbar} \delta t \Theta \ket{\Psi(0)}.
\end{dmath}

Since this holds for any state \( \ket{\Psi(0)} \), the time reversal operator satisfies

\begin{dmath}\label{eqn:qmLecture13:140}
i \hat{H} 
=
\Theta^{-1} (- i) \hat{H} \Theta.
\end{dmath}

Note that the factors of \( i \) have not been canceled on purpose, since we are allowing for the time reversal operator to not neccessarily commute with imaginary numbers.

There are two possible solutions

\begin{itemize}
\item If \( \Theta \) is unitary where \( \Theta i = i \Theta \), then

\begin{dmath}\label{eqn:qmLecture13:160}
\hat{H} 
=
-\Theta^{-1} \hat{H} \Theta,
\end{dmath}

or
\begin{dmath}\label{eqn:qmLecture13:180}
\Theta \hat{H} 
=
- \hat{H} \Theta.
\end{dmath}

Consider the implications of this on energy eigenstates
\begin{dmath}\label{eqn:qmLecture13:200}
\hat{H} \ket{\Psi_n} = E_n \ket{\Psi_n},
\end{dmath}

\begin{dmath}\label{eqn:qmLecture13:220}
\Theta \hat{H} \ket{\Psi_n} = E_n \Theta \ket{\Psi_n},
\end{dmath}

but

\begin{dmath}\label{eqn:qmLecture13:240}
-\hat{H} \Theta \ket{\Psi_n} = E_n \Theta \ket{\Psi_n},
\end{dmath}

or

\begin{dmath}\label{eqn:qmLecture13:260}
\hat{H} \lr{ \Theta \ket{\Psi_n}} = -E_n \lr{ \Theta \ket{\Psi_n} }.
\end{dmath}

This would mean that \( \lr{ \Theta \ket{\Psi_n}} \) is an eigenket of \( \hat{H} \), but with a negative energy eigenvalue.  

\item \( \Theta \) is antiunitary, where \( \Theta i = -i \Theta \).

This time
\begin{dmath}\label{eqn:qmLecture13:280}
i \hat{H} = i \Theta^{-1} \hat{H} \Theta,
\end{dmath}

so 

\begin{dmath}\label{eqn:qmLecture13:300}
\Theta \hat{H} = \hat{H} \Theta.
\end{dmath}

Acting on an energy eigenket, we've got

\begin{dmath}\label{eqn:qmLecture13:1400}
\Theta \hat{H} \ket{\Psi_n}
=
E_n \lr{ \Theta \ket{\Psi_n} },
\end{dmath}

and
\begin{dmath}\label{eqn:qmLecture13:1420}
\lr{ \hat{H} \Theta } \ket{\Psi_n} 
= 
\hat{H} \lr{ \Theta \ket{\Psi_n} },
\end{dmath}

so \( \Theta \ket{\Psi_n} \) is an eigenstate with energy \( E_n \).

\end{itemize}

\paragraph{What properties do we expect from \( \Theta \)?}

We expect
\index{time reversal!properties}
\begin{dmath}\label{eqn:qmLecture13:320}
\begin{aligned}
\hat{x} &\rightarrow \hat{x} \\
\hat{p} &\rightarrow -\hat{p} \\
\hat{\BL} &\rightarrow -\hat{\BL}
\end{aligned}
\end{dmath}

where we have a sign flip in the time dependent momentum operator (and therefore angular momentum), but not for position.  If we have

\begin{dmath}\label{eqn:qmLecture13:340}
\Theta^{-1} \hat{x} \Theta = \hat{x},
\end{dmath}

if that's true, then how about the momentum operator in the position basis
\begin{dmath}\label{eqn:qmLecture13:360}
\Theta^{-1} \hat{p} \Theta 
= 
\Theta^{-1} \lr{ -i \Hbar \PD{x}{} } \Theta 
= 
\Theta^{-1} \lr{ -i \Hbar } \Theta \PD{x}{}
= 
i \Hbar \Theta^{-1} \Theta \PD{x}{} 
= 
-\hat{p}.
\end{dmath}

How about the \( x,p \) commutator?  For that we have

\begin{dmath}\label{eqn:qmLecture13:380}
\Theta^{-1} \antisymmetric{\hat{x}}{\hat{p}} \Theta
=
\Theta^{-1} \lr{ i \Hbar } \Theta
=
-i \Hbar \Theta^{-1} \Theta
=
- \antisymmetric{\hat{x}}{\hat{p}}.
\end{dmath}

For the the angular momentum operators

\begin{dmath}\label{eqn:qmLecture13:420}
\hat{L}_i = \epsilon_{ijk} \hat{r}_j \hat{p}_k,
\end{dmath}

the time reversal operator should flip the sign due to its action on \( \hat{p}_k \).

%\begin{dmath}\label{eqn:qmLecture13:400}
%\antisymmetric{\hat{L}_i }{\hat{L}_j } = i \epsilon_{ijk} \hat{L}_k.
%\end{dmath}
%
%FIXME: lost his point about the angular momentum commutator here.

\paragraph{Time reversal acting on spin 1/2 (Fermions).  Attempt I.}

Consider two spin states \( \ket{\uparrow}, \ket{\downarrow} \).  What should the action of the time reversal operator on such a state be?  Let's (incorrectly) start by supposing that the time reversal operator effects are

\begin{dmath}\label{eqn:qmLecture13:440}
\begin{aligned}
\Theta \ket{\uparrow} &\questioneq \ket{\downarrow} \\
\Theta \ket{\downarrow} &\questioneq \ket{\uparrow}.
\end{aligned}
\end{dmath}

Given a general state
so that if

\begin{equation}\label{eqn:qmLecture13:740}
\ket{\Psi} = a \ket{\uparrow} + b \ket{\downarrow},
\end{equation}

the action of the time reversal operator would be

\begin{equation}\label{eqn:qmLecture13:760}
\Theta \ket{\Psi} = a^\conj \ket{\downarrow} + b^\conj \ket{\uparrow}.
\end{equation}

That action is:

\begin{equation}\label{eqn:qmLecture13:460}
\begin{aligned}
a \rightarrow b^\conj \\
b \rightarrow a^\conj
\end{aligned}
\end{equation}

Let's consider whether or not such an action a spin operator with properties

\begin{dmath}\label{eqn:qmLecture13:480}
\antisymmetric{\hat{S}_i}{\hat{S}_j} = i \epsilon_{ijk} \hat{S}_k.
\end{dmath}

produce the desired inversion of sign

\begin{dmath}\label{eqn:qmLecture13:500}
\Theta^{-1} \hat{S}_i \Theta = - \hat{S}_i.
\end{dmath}

The expectations of the spin operators (without any application of time reversal) are

\begin{dmath}\label{eqn:qmLecture13:1440}
\bra{\Psi} \hat{S}_x \ket{\Psi} 
= 
\frac{\Hbar}{2} 
\lr{ a^\conj \bra{\uparrow} + b^\conj \bra{\downarrow} } 
\sigma_x
\lr{ a \ket{\uparrow} + b \ket{\downarrow} } 
= 
\frac{\Hbar}{2} 
\lr{ a^\conj \bra{\uparrow} + b^\conj \bra{\downarrow} } 
\lr{ a \ket{\downarrow} + b \ket{\uparrow} } 
=
\frac{\Hbar}{2} 
\lr{ a^\conj b + b^\conj a },
\end{dmath}

\begin{dmath}\label{eqn:qmLecture13:1460}
\bra{\Psi} \hat{S}_y \ket{\Psi} 
= 
\frac{\Hbar}{2} 
\lr{ a^\conj \bra{\uparrow} + b^\conj \bra{\downarrow} } 
\sigma_y
\lr{ a \ket{\uparrow} + b \ket{\downarrow} } 
= 
\frac{i\Hbar}{2} 
\lr{ a^\conj \bra{\uparrow} + b^\conj \bra{\downarrow} } 
\lr{ a \ket{\downarrow} - b \ket{\uparrow} } 
=
\frac{\Hbar}{2 i} \lr{ a^\conj b - b^\conj a },
\end{dmath}

\begin{dmath}\label{eqn:qmLecture13:1480}
\bra{\Psi} \hat{S}_z \ket{\Psi} 
= 
\frac{\Hbar}{2} 
\lr{ a^\conj \bra{\uparrow} + b^\conj \bra{\downarrow} } 
\sigma_z
\lr{ a \ket{\uparrow} - b \ket{\downarrow} } 
= 
\frac{\Hbar}{2} \lr{ \Abs{a}^2 - \Abs{b}^2 }
\end{dmath}

The time reversed actions are

\begin{dmath}\label{eqn:qmLecture13:1560}
\bra{\Psi} \Theta^{-1} \hat{S}_x \Theta \ket{\Psi} 
= 
\frac{\Hbar}{2} 
\lr{ a^\conj \bra{\downarrow} + b^\conj \bra{\uparrow} } 
\sigma_x
\lr{ a \ket{\downarrow} + b \ket{\uparrow} } 
= 
\frac{\Hbar}{2} 
\lr{ a^\conj \bra{\downarrow} + b^\conj \bra{\uparrow} } 
\lr{ a \ket{\uparrow} + b \ket{\downarrow} } 
=
\frac{\Hbar}{2} 
\lr{ a^\conj b + b^\conj a },
\end{dmath}

\begin{dmath}\label{eqn:qmLecture13:1580}
\bra{\Psi} \Theta^{-1} \hat{S}_y \Theta \ket{\Psi} 
= 
\frac{\Hbar}{2} 
\lr{ a^\conj \bra{\downarrow} + b^\conj \bra{\uparrow} } 
\sigma_y
\lr{ a \ket{\downarrow} + b \ket{\uparrow} } 
= 
\frac{i\Hbar}{2} 
\lr{ a^\conj \bra{\downarrow} + b^\conj \bra{\uparrow} } 
\lr{ -a \ket{\uparrow} + b \ket{\downarrow} } 
=
\frac{\Hbar}{2 i} \lr{ -a^\conj b + b^\conj a },
\end{dmath}

\begin{dmath}\label{eqn:qmLecture13:1600}
\bra{\Psi} \Theta^{-1} \hat{S}_z \Theta \ket{\Psi} 
= 
\frac{\Hbar}{2} 
\lr{ a^\conj \bra{\downarrow} + b^\conj \bra{\uparrow} } 
\sigma_z
\lr{ a \ket{\downarrow} + b \ket{\uparrow} } 
= 
\frac{\Hbar}{2} 
\lr{ a^\conj \bra{\downarrow} + b^\conj \bra{\uparrow} } 
\lr{ -a \ket{\downarrow} + b \ket{\uparrow} } 
= 
\frac{\Hbar}{2} \lr{ -\Abs{a}^2 + \Abs{b}^2 }
\end{dmath}

%\begin{dmath}\label{eqn:qmLecture13:1500}
%\Theta^{-1} \hat{S}_z \Theta \ket{\uparrow}
%=
%\frac{\Hbar}{2} \Theta^{-1} \sigma_z \ket{\downarrow}
%=
%-\frac{\Hbar}{2} \Theta^{-1} \ket{\downarrow}
%=
%-\frac{\Hbar}{2} \ket{\uparrow},
%\end{dmath}
%
%\begin{dmath}\label{eqn:qmLecture13:1520}
%\Theta^{-1} \hat{S}_x \Theta \ket{\uparrow}
%=
%\frac{\Hbar}{2} \Theta^{-1} \sigma_x \ket{\downarrow}
%=
%\frac{\Hbar}{2} \Theta^{-1} \ket{\uparrow}
%=
%\frac{\Hbar}{2} \ket{\downarrow},
%\end{dmath}
%
%and
%
%\begin{dmath}\label{eqn:qmLecture13:1540}
%\Theta^{-1} \hat{S}_y \Theta \ket{\uparrow}
%=
%\frac{\Hbar}{2} \Theta^{-1} \sigma_y \ket{\downarrow}
%=
%-i \frac{\Hbar}{2} \Theta^{-1} \ket{\uparrow}
%=
%\frac{\Hbar}{2 i} \ket{\downarrow}.
%\end{dmath}
%
%Contrast this to the time reversed action on the spin operators
%
%\bra{\Psi} \Theta^{-1} \hat{S}_x \Theta \ket{\Psi} 
%= \frac{\Hbar}{2} \lr{ b^\conj a + a b^\conj } 
%
%\bra{\Psi} \Theta^{-1} \hat{S}_y \Theta \ket{\Psi} 
%= \frac{\Hbar}{2 i} \lr{ b a^\conj - a b^\conj } 
%
%\bra{\Psi} \Theta^{-1} \hat{S}_z \Theta \ket{\Psi} 
%= \frac{\Hbar}{2} \lr{ \Abs{b}^2 - \Abs{a}^2 }

We see that this is not right, because the sign for the x component has not been flipped.
% (only the sign of the y component has).

\paragraph{Spin 1/2 (Fermions).  Attempt II.}

Again assuming

\begin{equation}\label{eqn:qmLecture13:580}
\ket{\Psi} = a \ket{\uparrow} + b \ket{\downarrow},
\end{equation}

now try the action

\begin{equation}\label{eqn:qmLecture13:780}
\Theta \ket{\Psi} = a^\conj \ket{\downarrow} - b^\conj \ket{\uparrow}.
\end{equation}

This is the action:

\begin{equation}\label{eqn:qmLecture13:600}
\begin{aligned}
a \rightarrow -b^\conj \\
b \rightarrow a^\conj
\end{aligned}
\end{equation}

The correct action of time reversal on the basis states (up to a phase choice) is

\boxedEquation{eqn:qmLecture13:620}{
%\begin{boxed}\label{eqn:qmLecture13:640}
\begin{aligned}
\Theta \ket{\uparrow} &= \ket{\downarrow} \\
\Theta \ket{\downarrow} &= -\ket{\uparrow} \\
\end{aligned}
%\end{boxed}
}

Note that acting the time reversal operator twice has the effects

\begin{dmath}\label{eqn:qmLecture13:660}
\Theta^2 \ket{\uparrow} = \Theta \ket{\downarrow} = - \ket{\uparrow}
\end{dmath}
\begin{dmath}\label{eqn:qmLecture13:680}
\Theta^2 \ket{\downarrow} = \Theta (-\ket{\uparrow}) = - \ket{\uparrow}.
\end{dmath}

We end up with the same state we started with, but with the opposite sign.  This means that as an operator

\index{time reversal!squared}
%\begin{dmath}\label{eqn:qmLecture13:700}
\boxedEquation{eqn:qmLecture13:700}{
\Theta^2 = -1.
}
%\end{dmath}

This is try for half integer particles (Fermions) \( S = 1/2, 3/2, 5/2, \cdots \), but for bosons with integer spin \( S \).

%\begin{dmath}\label{eqn:qmLecture13:720}
\boxedEquation{eqn:qmLecture13:720}{
\Theta^2 = 1.
}
%\end{dmath}

\paragraph{Kramer's degeneracy for Spin 1/2 (fermions)}

\index{spin half!time reversal}

Suppose we imagine there is state for which the action of the time reversal operator productes the same state, just different in phase

\begin{equation}\label{eqn:qmLecture13:800}
\Theta \ket{\Psi_n} 
= \ket{\tilde{\Psi}_n}
= e^{i \delta} \ket{\tilde{\Psi}_n},
\end{equation}

then
\begin{dmath}\label{eqn:qmLecture13:840}
\Theta^2 \ket{\Psi_n} 
= \Theta e^{i \delta} \ket{\tilde{\Psi}_n}
= e^{i \delta} e^{i \delta} \ket{\tilde{\Psi}_n},
\end{dmath}

but 

\begin{dmath}\label{eqn:qmLecture13:860}
\Theta e^{i \delta} \ket{\tilde{\Psi}_n}
=
e^{-i \delta} \Theta \ket{\tilde{\Psi}_n}
=
e^{-i \delta} e^{i \delta} \ket{\tilde{\Psi}_n}
=
\ket{\tilde{\Psi}_n}
\ne 
- \ket{\tilde{\Psi}_n}.
\end{dmath}

This is a contradiction, so we must have at least a two-fold degeneracy.  This is called \textAndIndex{Kramer's degeneracy}.  In the homework we will show that this is not the case for integer spin particles.

\section{Angular momentum}

In classical mechanics the (orbital) angular momentum is
\index{orbital angular momentum}
\index{angular momentum operator}

\begin{dmath}\label{eqn:qmLecture13:880}
\BL = \Br \cross \Bp.
\end{dmath}

Here ``orbital'' is to distingish from spin angular momentum.

In quantum mechanics, the mapping to operators, in componont form, is

\begin{dmath}\label{eqn:qmLecture13:900}
\hat{L}_i = \epsilon_{ijk} \hat{r}_j \hat{p}_k.
\end{dmath}

These operators do not commute
\begin{dmath}\label{eqn:qmLecture13:920}
\antisymmetric{L_i}{L_j} 
= \epsilon_{i m n} \epsilon_{ijk} 
\antisymmetric{\hat{r}_m \hat{p}_n}{\hat{r}_k \hat{p}_l}
= 
\delta^{[mn]}_{jk}
\lr{
\hat{r}_m \hat{p}_n \hat{r}_k \hat{p}_l
-
\hat{r}_k \hat{p}_l
\hat{r}_m \hat{p}_n 
}
=
\lr{
\hat{r}_j \hat{p}_k \hat{r}_k \hat{p}_l
-
\hat{r}_k \hat{p}_j \hat{r}_k \hat{p}_l
-
\hat{r}_k \hat{p}_l
\hat{r}_j \hat{p}_k 
+
\hat{r}_k \hat{p}_l
\hat{r}_k \hat{p}_j 
}
=
i \epsilon_{ijk} \hat{L}_k.
\end{dmath}

FIXME: fill in the details.

\index{vector operator}
This means that we can't simultaneously determine \( \hat{L}_i \) for all \( i \).  

Aside: In quantum mechanics, we define an operator \( \hat{\BV} \) to be a vector operator if

\begin{dmath}\label{eqn:qmLecture13:940}
\antisymmetric{L_i}{V_j} 
=
i \epsilon_{ijk} \hatV_k.
\end{dmath}

Consider the commutator of the magnitude of the angular momentum operator with \( \hat{L}_x \)

\begin{dmath}\label{eqn:qmLecture13:960}
\antisymmetric{
\hat{L}_x^2 +
\hat{L}_y^2 +
\hat{L}_z^2
}
{\hat{L}_x}
=
\hat{L}_y \antisymmetric{\hat{L}_y}{\hat{L}_x}
+\antisymmetric{\hat{L}_y}{\hat{L}_x} \hat{L}_y
+\hat{L}_z \antisymmetric{\hat{L}_z}{\hat{L}_x}
+\antisymmetric{\hat{L}_z}{\hat{L}_x} \hat{L}_z
= 
0.
\end{dmath}

In fact 
\begin{dmath}\label{eqn:qmLecture13:980}
\antisymmetric{\hat{\BL}^2 }{\hat{L}_i} = 0.
\end{dmath}

Suppose we have a state \( \ket{\Psi} \) with a well defined \( \hat{L}_z \) eigenvalue and well defined \( \hat{\BL^2} \) eigenvalue, written as

\begin{dmath}\label{eqn:qmLecture13:1000}
\ket{\Psi} = \ket{a, b},
\end{dmath}

where the label \( a \) is used for the eigenvalue of \( \hat{\BL}^2 \) and \( b \) labels the eigenvalue of \( \hat{L}_z \).  Then

\begin{equation}\label{eqn:qmLecture13:1020}
\begin{aligned}
\hat{\BL}^2 \ket{a , b} &= \Hbar^2 a \ket{a ,b} \\
\hat{L_z} \ket{a , b} &= \Hbar b \ket{a ,b}.
\end{aligned}
\end{equation}

Things aren't so nice when we act with other angular momentum operators, producing a scrambled mess

\begin{equation}\label{eqn:qmLecture13:1040}
\begin{aligned}
\hat{L}_x \ket{a , b} &= \sum_{a', b'} \calA^x_{a, b, a', b'} \ket{a', b'} \\
\hat{L}_y \ket{a , b} &= \sum_{a', b'} \calA^y_{a, b, a', b'} \ket{a', b'} \\
\end{aligned}
\end{equation}

With this representation, we have

\begin{dmath}\label{eqn:qmLecture13:1060}
\hat{L}_x \hat{\BL}^2 \ket{a, b}
=
\hat{L}_x \Hbar^2 a
\sum_{a', b'} \calA^x_{a, b, a', b'} \ket{a', b'}.
\end{dmath}

\begin{dmath}\label{eqn:qmLecture13:1080}
\hat{\BL}^2 \hat{L}_x \ket{a, b}
=
\Hbar^2 
\sum_{a', b'} a' \calA^x_{a, b, a', b'} \ket{a', b'}.
\end{dmath}

Since \( \hat{\BL}^2, \hat{L}_x \) commute, we must have

\begin{dmath}\label{eqn:qmLecture13:1100}
\calA^x_{a, b, a', b'} = \delta_{a, a'} \calA^x_{a'; b, b'},
\end{dmath}

and similarily
\begin{dmath}\label{eqn:qmLecture13:1120}
\calA^y_{a, b, a', b'} = \delta_{a, a'} \calA^y_{a'; b, b'}.
\end{dmath}

Simplifying things we can write the action of \( \hat{L}_x, \hat{L}_y \) on the state as

\begin{dmath}\label{eqn:qmLecture13:1140}
\begin{aligned}
\hat{L}_x \ket{a , b} &= \sum_{ b'} \calA^x_{a; b, b'} \ket{a, b'} \\
\hat{L}_y \ket{a , b} &= \sum_{ b'} \calA^y_{a; b, b'} \ket{a, b'} \\
\end{aligned}
\end{dmath}

Let's define
\begin{dmath}\label{eqn:qmLecture13:1160}
\begin{aligned}
\hat{L}_{+} &\equiv \hat{L}_x + i \hat{L}_y \\
\hat{L}_{-} &\equiv \hat{L}_x - i \hat{L}_y \\
\end{aligned}
\end{dmath}

Because these are sums of \( \hat{L}_x, \hat{L}_y \) they must also commute
\begin{dmath}\label{eqn:qmLecture13:1180}
\antisymmetric{\hat{\BL}^2}{\hat{L}_{\pm}} = 0.
\end{dmath}

We find
\begin{dmath}\label{eqn:qmLecture13:1200}
\antisymmetric{\hat{L}_z}{\hat{L}_{\pm}} = \pm \Hbar \hat{L}_{\pm},
\end{dmath}

or
\begin{dmath}\label{eqn:qmLecture13:1220}
\begin{aligned}
\hat{L}_z \hat{L}_{+} - \hat{L}_{+} \hat{L}_z &= \Hbar \hat{L}_{+} \\
\hat{L}_z \hat{L}_{-} - \hat{L}_{-} \hat{L}_z &= -\Hbar \hat{L}_{-}
\end{aligned}
\end{dmath}

How about 

\begin{dmath}\label{eqn:qmLecture13:1240}
\hat{L}_z \hat{L}_{+} \ket{a, b} 
= 
\Hbar \hat{L}_{+} \ket{a,b} + \hat{L}_{+} \hat{L}_z \ket{a,b}
= 
\Hbar \hat{L}_{+} \ket{a,b} + \Hbar b \hat{L}_{+} \ket{a,b}
=
\Hbar \lr{ b + 1 } \hat{L}_{+} \ket{a, b} 
\end{dmath}

We find
\begin{dmath}\label{eqn:qmLecture13:1260}
\ket{\hat{L}_{\pm}} \propto \ket{a, b \pm 1},
\end{dmath}

The products of the raising and lowering operators are

\begin{dmath}\label{eqn:qmLecture13:1280}
L_m \hat{L}_{+} 
=
\lr{ \hat{L}_x - i \hat{L}_y }
\lr{ \hat{L}_x + i \hat{L}_y }
=
\lr{ \hat{\BL}^2 - \hat{L}_z^2 } + i \antisymmetric{\hat{L}_x}{\hat{L}_y}
=
\hat{\BL}^2 - \hat{L}_z^2 - \Hbar \hat{L}_z,
\end{dmath}

and
\begin{dmath}\label{eqn:qmLecture13:1300}
L_p \hat{L}_{-} 
=
\lr{ \hat{L}_x + i \hat{L}_y }
\lr{ \hat{L}_x - i \hat{L}_y }
=
\lr{ \hat{\BL}^2 - \hat{L}_z^2 } - i \antisymmetric{\hat{L}_x}{\hat{L}_y}
=
\hat{\BL}^2 - \hat{L}_z^2 + \Hbar \hat{L}_z,
\end{dmath}

So we must have

\begin{equation}\label{eqn:qmLecture13:1320}
0 
\le \bra{a, b} \hat{L}_{-} \hat{L}_{+} \ket{a, b} 
= 
\bra{a, b} 
\lr{ \hat{\BL}^2 - \hat{L}_z^2 - \Hbar \hat{L}_z }
\ket{a, b} 
=
\Hbar^2 a - \Hbar^2 b - \Hbar^2 b,
\end{equation}

and

\begin{equation}\label{eqn:qmLecture13:1340}
0 
\le \bra{a, b} \hat{L}_{+} \hat{L}_{-} \ket{a, b} 
= 
\bra{a, b} 
\lr{ \hat{\BL}^2 - \hat{L}_z^2 + \Hbar \hat{L}_z }
\ket{a, b} 
=
\Hbar^2 a - \Hbar^2 b + \Hbar^2 b.
\end{equation}

This puts constraints on \( a, b \), roughly of the form

\begin{enumerate}
\item
\begin{equation}\label{eqn:qmLecture13:1360}
a - b( b + 1) \ge 0
\end{equation}

With \( b_{\textrm{max}} > 0 \), \( b_{\textrm{max}} \approx \sqrt{a} \).

\item
\begin{equation}\label{eqn:qmLecture13:1380}
a - b( b - 1) \ge 0
\end{equation}

With \( b_{\textrm{min}} < 0 \), \( b_{\textrm{max}} \approx -\sqrt{a} \).

\end{enumerate}

%\paragraph{I Love and Desire Sofia Always!}

\EndArticle
%\EndNoBibArticle
