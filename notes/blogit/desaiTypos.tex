%
% Copyright � 2015 Peeter Joot.  All Rights Reserved.
% Licenced as described in the file LICENSE under the root directory of this GIT repository.
%
\documentclass[]{eliblog}

\usepackage{amsmath}
\usepackage{mathpazo}

%
% shorthand for bold symbols, convenient for vectors and matrices
%
\newcommand{\Ba}[0]{\mathbf{a}}
\newcommand{\Bb}[0]{\mathbf{b}}
\newcommand{\Bc}[0]{\mathbf{c}}
\newcommand{\Bd}[0]{\mathbf{d}}
\newcommand{\Be}[0]{\mathbf{e}}
\newcommand{\Bf}[0]{\mathbf{f}}
\newcommand{\Bg}[0]{\mathbf{g}}
\newcommand{\Bh}[0]{\mathbf{h}}
\newcommand{\Bi}[0]{\mathbf{i}}
\newcommand{\Bj}[0]{\mathbf{j}}
\newcommand{\Bk}[0]{\mathbf{k}}
\newcommand{\Bl}[0]{\mathbf{l}}
\newcommand{\Bm}[0]{\mathbf{m}}
\newcommand{\Bn}[0]{\mathbf{n}}
\newcommand{\Bo}[0]{\mathbf{o}}
\newcommand{\Bp}[0]{\mathbf{p}}
\newcommand{\Bq}[0]{\mathbf{q}}
\newcommand{\Br}[0]{\mathbf{r}}
\newcommand{\Bs}[0]{\mathbf{s}}
\newcommand{\Bt}[0]{\mathbf{t}}
\newcommand{\Bu}[0]{\mathbf{u}}
\newcommand{\Bv}[0]{\mathbf{v}}
\newcommand{\Bw}[0]{\mathbf{w}}
\newcommand{\Bx}[0]{\mathbf{x}}
\newcommand{\By}[0]{\mathbf{y}}
\newcommand{\Bz}[0]{\mathbf{z}}
\newcommand{\BA}[0]{\mathbf{A}}
\newcommand{\BB}[0]{\mathbf{B}}
\newcommand{\BC}[0]{\mathbf{C}}
\newcommand{\BD}[0]{\mathbf{D}}
\newcommand{\BE}[0]{\mathbf{E}}
\newcommand{\BF}[0]{\mathbf{F}}
\newcommand{\BG}[0]{\mathbf{G}}
\newcommand{\BH}[0]{\mathbf{H}}
\newcommand{\BI}[0]{\mathbf{I}}
\newcommand{\BJ}[0]{\mathbf{J}}
\newcommand{\BK}[0]{\mathbf{K}}
\newcommand{\BL}[0]{\mathbf{L}}
\newcommand{\BM}[0]{\mathbf{M}}
\newcommand{\BN}[0]{\mathbf{N}}
\newcommand{\BO}[0]{\mathbf{O}}
\newcommand{\BP}[0]{\mathbf{P}}
\newcommand{\BQ}[0]{\mathbf{Q}}
\newcommand{\BR}[0]{\mathbf{R}}
\newcommand{\BS}[0]{\mathbf{S}}
\newcommand{\BT}[0]{\mathbf{T}}
\newcommand{\BU}[0]{\mathbf{U}}
\newcommand{\BV}[0]{\mathbf{V}}
\newcommand{\BW}[0]{\mathbf{W}}
\newcommand{\BX}[0]{\mathbf{X}}
\newcommand{\BY}[0]{\mathbf{Y}}
\newcommand{\BZ}[0]{\mathbf{Z}}

\newcommand{\Bzero}[0]{\mathbf{0}}
\newcommand{\Btheta}[0]{\boldsymbol{\theta}}
\newcommand{\Btau}[0]{\boldsymbol{\tau}}
\newcommand{\Bomega}[0]{\boldsymbol{\omega}}

%
% shorthand for unit vectors
%
\newcommand{\acap}[0]{\hat{\Ba}}
\newcommand{\bcap}[0]{\hat{\Bb}}
\newcommand{\ccap}[0]{\hat{\Bc}}
\newcommand{\dcap}[0]{\hat{\Bd}}
\newcommand{\ecap}[0]{\hat{\Be}}
\newcommand{\fcap}[0]{\hat{\Bf}}
\newcommand{\gcap}[0]{\hat{\Bg}}
\newcommand{\hcap}[0]{\hat{\Bh}}
\newcommand{\icap}[0]{\hat{\Bi}}
\newcommand{\jcap}[0]{\hat{\Bj}}
\newcommand{\kcap}[0]{\hat{\Bk}}
\newcommand{\lcap}[0]{\hat{\Bl}}
\newcommand{\mcap}[0]{\hat{\Bm}}
\newcommand{\ncap}[0]{\hat{\Bn}}
\newcommand{\ocap}[0]{\hat{\Bo}}
\newcommand{\pcap}[0]{\hat{\Bp}}
\newcommand{\qcap}[0]{\hat{\Bq}}
\newcommand{\rcap}[0]{\hat{\Br}}
\newcommand{\scap}[0]{\hat{\Bs}}
\newcommand{\tcap}[0]{\hat{\Bt}}
\newcommand{\ucap}[0]{\hat{\Bu}}
\newcommand{\vcap}[0]{\hat{\Bv}}
\newcommand{\wcap}[0]{\hat{\Bw}}
\newcommand{\xcap}[0]{\hat{\Bx}}
\newcommand{\ycap}[0]{\hat{\By}}
\newcommand{\zcap}[0]{\hat{\Bz}}
\newcommand{\thetacap}[0]{\hat{\Btheta}}

%
% to write R^n and C^n in a distinguishable fashion.  Perhaps change this
% to the double lined characters upon figuring out how to do so.
%
\newcommand{\C}[1]{$\mathbb{C}^{#1}$}
\newcommand{\R}[1]{$\mathbb{R}^{#1}$}

%
% various generally useful helpers
%

% derivative of #1 wrt. #2:
\newcommand{\D}[2] {\frac {d#2} {d#1}}

\newcommand{\inv}[1]{\frac{1}{#1}}
\newcommand{\cross}[0]{\times}

\newcommand{\abs}[1]{\lvert{#1}\rvert}
\newcommand{\norm}[1]{\lVert{#1}\rVert}
\newcommand{\innerprod}[2]{\langle{#1}, {#2}\rangle}
\newcommand{\dotprod}[2]{{#1} \cdot {#2}}
\newcommand{\bdotprod}[2]{\left({#1} \cdot {#2}\right)}
\newcommand{\crossprod}[2]{{#1} \cross {#2}}
\newcommand{\tripleprod}[3]{\dotprod{\left(\crossprod{#1}{#2}\right)}{#3}}

\DeclareMathOperator{\Proj}{Proj}
\DeclareMathOperator{\Span}{span}
\DeclareMathOperator{\Sgn}{sgn}
\DeclareMathOperator{\Area}{Area}
\DeclareMathOperator{\Volume}{Volume}

%
% A few miscellaneous things specific to this document
%
\newcommand{\crossop}[1]{\crossprod{#1}{}}

% R2 vector.
\newcommand{\VectorTwo}[2]{
\begin{bmatrix}
 {#1} \\
 {#2}
\end{bmatrix}
}

\newcommand{\VectorN}[1]{
\begin{bmatrix}
{#1}_1 \\
{#1}_2 \\
\vdots \\
{#1}_N \\
\end{bmatrix}
}

\newcommand{\DETuvij}[4]{
\begin{vmatrix}
 {#1}_{#3} & {#1}_{#4} \\
 {#2}_{#3} & {#2}_{#4}
\end{vmatrix}
}

\newcommand{\DETuvwijk}[6]{
\begin{vmatrix}
 {#1}_{#4} & {#1}_{#5} & {#1}_{#6} \\
 {#2}_{#4} & {#2}_{#5} & {#2}_{#6} \\
 {#3}_{#4} & {#3}_{#5} & {#3}_{#6}
\end{vmatrix}
}

\newcommand{\DETuvwxijkl}[8]{
\begin{vmatrix}
 {#1}_{#5} & {#1}_{#6} & {#1}_{#7} & {#1}_{#8} \\
 {#2}_{#5} & {#2}_{#6} & {#2}_{#7} & {#2}_{#8} \\
 {#3}_{#5} & {#3}_{#6} & {#3}_{#7} & {#3}_{#8} \\
 {#4}_{#5} & {#4}_{#6} & {#4}_{#7} & {#4}_{#8} \\
\end{vmatrix}
}

%\newcommand{\DETuvwxyijklm}[10]{
%\begin{vmatrix}
% {#1}_{#6} & {#1}_{#7} & {#1}_{#8} & {#1}_{#9} & {#1}_{#10} \\
% {#2}_{#6} & {#2}_{#7} & {#2}_{#8} & {#2}_{#9} & {#2}_{#10} \\
% {#3}_{#6} & {#3}_{#7} & {#3}_{#8} & {#3}_{#9} & {#3}_{#10} \\
% {#4}_{#6} & {#4}_{#7} & {#4}_{#8} & {#4}_{#9} & {#4}_{#10} \\
% {#5}_{#6} & {#5}_{#7} & {#5}_{#8} & {#5}_{#9} & {#5}_{#10}
%\end{vmatrix}
%}

% R3 vector.
\newcommand{\VectorThree}[3]{
\begin{bmatrix}
 {#1} \\
 {#2} \\
 {#3}
\end{bmatrix}
}



\author{Peeter Joot}
\email{peeter.joot@gmail.com}


%\label{chap:desaiTypos}
%\useCCL
\blogpage{http://sites.google.com/site/peeterjoot/math2010/desaiTypos.pdf}
\date{Oct 31, 2010}
\revisionInfo{desaiTypos.tex}

\chapter{Believed to be typos in Desai's QM Text}

\beginArtNoToc

Vatche says that the root cause of what I've identified as a typo is in some cases incorrect, and that he's going through the text carefully himself too.

\section{Chapter I.}

\begin{itemize}
\item Page 7.  Text before (1.43).  $\alpha$ instead of $a$ used.
\item Page 19.  Equation (1.122).  $\dagger$s omitted after first equality.
\end{itemize}

\section{Chapter II.}
\begin{itemize}
\item Page 40.  Text before (2.137).  Reference to equation (2.133) should be (2.135)
\item Page 53.  Is the "Also show that" here correct?  I get a different answer.
\end{itemize}

\section{Chapter III.}
\begin{itemize}
\item Page 61.  Equation (3.51).  $1/\hbar$ missing.
\item Page 66.  Equation (3.92).  $-(d/dt \bra{\alpha}) \ket{\alpha}$ should be $+\ket{\alpha} d/dt \bra{\alpha}$.
\item Page 66.  Equation (3.93).  $H$ on wrong side of $\bra{\alpha}$ 
\end{itemize}

\section{Chapter IV.}
\begin{itemize}
\item Page 81.  Equation (4.52).  Should be $-2\alpha$ in the exponent.
\item Page 82.  Equation (4.67).  $2\alpha$ in the denominator of the normalization should be $\alpha'/\pi$.
\item Page 83.  Equation (4.74).  A normalized wave function isn't required for the discussion, but if that was intended, a $1/\sqrt{2\pi}$ factor is missing.
\item Page 86.  Equation (4.99).  Extra brace in the exponent.
\item Page 87.  Equation (4.106).  Extra brace in the exponent.
\item Page 89.  Equation (4.129), (4.130).  $\lambda - m^2/...$ should be $\lambda + ...$
\item Page 93.  Equation (4.169).  conjugation missing for $Y_{lm}$.  $Y_{l'm'}$ is missing prime on the $l$ index.
\item Page 95.  Second line of text.  Language choice?  "We now implement".  perhaps utilize would be better?
\item Page 95.  Text before (4.193).  $i$ is in bold.
\item Page 96.  Text before (4.196).  $i$ is in bold.
\item Page 97.  (4.205).  $i$ is in bold.
\item Page 97.  (4.207-209).  $\Bi$, and $\Bj$s aren't in bold like $\Bk$
\item Page 101.  (4.239-240).  The approach here is unclear.  FIXME: incorporate lecture notes from class that did this using braket notation.
\item Page 102.  (4.248-249).  Commas missing to separate $l$, and $m\pm 1$ in the kets.
\end{itemize}

\section{Chapter V.}
\begin{itemize}
\item Page 113.  (5.86). One $\sigma$ isn't in bold.
\item Page 114.  (5.100). $\chi$ is in bold.
\item Page 115.  Text before (5.106). $\alpha$ in bold.
\item Page 118.  Switch of notation in problem 5 for ensemble averages.  $[S_i]$ used instead of $\expectation{S_i}_{\text{av}}$.
\end{itemize}

\section{Chapter VI.}
\begin{itemize}
\item Page 120.  $\phi$ in bold.  $A$ not in bold.
\item Page 123.  (6.26).  $1/i \hbar$ factor missing on RHS.
\item Page 124.  Text before (6.37).  You say canonical momenta $P_k$, but call these mechanical momenta on prev page.
\item Page 125.  (6.41).  Some $\psi$s are in bold.
\item Page 126.  (6.49).  There's no mention that $\BB$ is constant, leaving it unclear how the gauge condition and how the curl of $\BA$ reproduces $\BB$.  This would also help clarify how you are able to write $\Bmu \cdot \BB = \BB \cdot \Bmu$.
\item Page 128.  (6.65).  $\Bmu \cdot \BL$ should be $\Bmu \cdot \BB$.
\item Page 129.  (6.75).  $\Bmu \cdot \BL$ should be $\Bmu \cdot \BB$.
\item Page 131.  Problem 1.  bold missing on $\BE$.
\end{itemize}

\section{Chapter 8.}
\begin{itemize}
\item Page 159.  (8.6.3).  Two references to Chapter 2 should be Chapter 4.
\item Page 160.  (8.199).  Want $\hbar^2$ not $\hbar$ in expression for $k$.
\item Page 162.  (Fig 8.9).  Figure is backwards compared to text (a bump instead of a well).
\end{itemize}

\section{Chapter 9.}
\begin{itemize}
\item Page 174.  (9.5).  Have $\hbar/2m\omega$ instead of $\hbar m \omega/2$ in expression for $P$.
\item Page 181.  (9.57).  Factor of two missing.  Want $\frac{\alpha}{2 \sqrt{\pi}}$.
\end{itemize}

\section{Chapter 10.}
\begin{itemize}
\item Page 189.  (10.22).  It would be nice to have a reference to the appendix (ie: 10.100) for the chapter so that this identity isn't pulled out of a magic hat.
\item Page 192.  (10.44, 10.45).  $2 \alpha {\alpha^\conj}'$ should be $\alpha {\alpha^\conj}' + \alpha' \alpha^\conj$
\item Page 193.  (10.51).  Application (slowly, step by step explicitly) of 10.100 to expand the $e^{\frac{i}{\hbar}(p_0 X - x_0 P)}$ in the braket gives

\begin{align*}
\bra{x} e^{\frac{i}{\hbar}(p_0 X - x_0 P)} \ket{0}
&=
\bra{x} e^{\frac{i}{\hbar}p_0 X }
e^{-\frac{i}{\hbar}x_0 P}
e^{-\frac{i}{2\hbar}x_0 p_0 \antisymmetric{X}{P}}
\ket{0} \\
&=
\bra{x} e^{\frac{i}{\hbar}p_0 X }
e^{-\frac{i}{\hbar}x_0 P}
e^{\frac{x_0 p_0}{2} }
\ket{0} \\
&=
e^{\frac{x_0 p_0}{2} }
\bra{x} e^{\frac{i}{\hbar}p_0 X} 
e^{-\frac{i}{\hbar}x_0 P}
\ket{0} \\
&=
e^{\frac{x_0 p_0}{2} }
\left(
\bra{0} 
e^{\frac{i}{\hbar}x_0 P}
e^{-\frac{i}{\hbar}p_0 X} 
\ket{x}\right)^\conj \\
&=
e^{\frac{x_0 p_0}{2} }
\left(
\bra{0} 
e^{\frac{i}{\hbar}x_0 P}
\ket{x}
e^{-\frac{i}{\hbar}p_0 x} 
\right)^\conj \\
&=
e^{\frac{x_0 p_0}{2} } e^{\frac{i}{\hbar}p_0 x} 
\left(
\bra{0} 
e^{\frac{i}{\hbar}x_0 P}
\ket{x}
\right)^\conj \\
&=
e^{\frac{x_0 p_0}{2} } e^{\frac{i}{\hbar}p_0 x} 
\left(
\braket{0}{x - x_0}
\right)^\conj \\
&=
e^{\frac{x_0 p_0}{2} } e^{\frac{i}{\hbar}p_0 x} 
\braket{x - x_0}{0} \\
&=
e^{\frac{x_0 p_0}{2} } e^{\frac{i}{\hbar}p_0 x} 
\psi_0(x - x_0, 0)
\end{align*}

This is the same as (10.51) with the exception of a real scalar constant $e^{ x_0 p_0/2}$ multiplying the wave function.  Because of this I think that (10.51) should be a proportionality statement, and not an equality as in

\begin{align*}
\bra{x} e^{\frac{i}{\hbar}(p_0 X - x_0 P)} \ket{0} \propto
e^{\frac{i}{\hbar}p_0 x} \psi_0(x - x_0, 0)
\end{align*}

(ie: building this additional factor into the wave function normalization instead).
\item Page 196.  (text after 10.76).  Looks like reference to Chapter 9, should be Chapter 9 problem 5.
\item Page 197.  (10.83).  It's not clear where this result comes from, and refering to Appendix 20.13 doesn't help.  The definition of a Green's function is missing.  As well as the appendix that discusses how to evaluate that Green's function integral once you get that far, I would have found it helpful to have at least the following sort of basic definition of a Green's function (following wikipedia)

Given a linear operator $L$, such that $L u(x) = f(x)$, we search for the Green's function $G(x,s)$ such that $L G(x,s) = \delta(x-s)$.  For such a function we have

\begin{align*}
\int L G(x,s) f(s) ds 
&= \int \delta(x-s) f(s) ds \\
&= f(x)
\end{align*}

and by linearity we also have
\begin{align*}
f(x) 
&=
\int L G(x,s) f(s) ds \\
&= L \int G(x,s) f(s) ds \\
\end{align*}

and can therefore identify $u(x) = \int G(x,s) f(s) ds$ as the desired solution to $L u(x) = f(x)$ once the Green's function $G(x,s)$ associated with operator $L$ has been determined.

\item Page 196.  It's actually unclear why Green's functions are even introduced here.  To solve this system we can use the usual trick of assuming that we can take the constant term in the homogeneous solution and allow it to vary with time to determine a non-homogeneous solution $b(t) = b_0 e^{-i \omega_0 t} \rightarrow f(t) e^{-i\omega_0 t}$.  This gives us $f' = -i \omega_0 \lambda(t) e^{i \omega_0 t}$, which we can integrate directly to find the non-homogeneous solution

\begin{align*}
b(t) = b(t_0) e^{-i \omega_0 (t - t_0)} - i \omega_0 \int_{t_0}^t \lambda(t') e^{-i \omega_0 (t-t')} dt'
\end{align*}

Setting $t_0 = -\infty$ and adding in a general homogeneous solution one has 10.92.

\item Page 197.  (text after 10.85).  Reference to Chapter 1 should be Chapter 2.



\end{itemize}

\EndArticle
