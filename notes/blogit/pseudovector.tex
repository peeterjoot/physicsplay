%
% Copyright � 2017 Peeter Joot.  All Rights Reserved.
% Licenced as described in the file LICENSE under the root directory of this GIT repository.
%
%{
\newcommand{\authorname}{Peeter Joot}
\newcommand{\email}{peeterjoot@protonmail.com}
\newcommand{\basename}{FIXMEbasenameUndefined}
\newcommand{\dirname}{notes/FIXMEdirnameUndefined/}

\renewcommand{\basename}{pseudovector}
%\renewcommand{\dirname}{notes/phy1520/}
\renewcommand{\dirname}{notes/ece1228-electromagnetic-theory/}
%\newcommand{\dateintitle}{}
%\newcommand{\keywords}{}

\newcommand{\authorname}{Peeter Joot}
\newcommand{\onlineurl}{http://sites.google.com/site/peeterjoot2/math2013/\basename.pdf}
\newcommand{\sourcepath}{\dirname\basename.tex}
\newcommand{\generatetitle}[1]{\chapter{#1}}

\newcommand{\vcsinfo}{%
\section*{}
\noindent{\color{DarkOliveGreen}{\rule{\linewidth}{0.1mm}}}
\paragraph{Document version}
%\paragraph{\color{Maroon}{Document version}}
{
\small
\begin{itemize}
\item Available online at:\\ 
\href{\onlineurl}{\onlineurl}
\item Git Repository: \input{./.revinfo/gitRepo.tex}
\item Source: \sourcepath
\item last commit: \input{./.revinfo/gitCommitString.tex}
\item commit date: \input{./.revinfo/gitCommitDate.tex}
\end{itemize}
}
}

%\PassOptionsToPackage{dvipsnames,svgnames}{xcolor}
\PassOptionsToPackage{square,numbers}{natbib}
\documentclass{scrreprt}

\usepackage[left=2cm,right=2cm]{geometry}
\usepackage[svgnames]{xcolor}
\usepackage{peeters_layout}

\usepackage{natbib}

\usepackage[
colorlinks=true,
bookmarks=false,
pdfauthor={\authorname, \email},
backref 
]{hyperref}

% http://tex.stackexchange.com/questions/75773/how-to-reference-problems-by-the-text-label-in-an-exercise-envioronment
\usepackage[english]{cleveref}
\crefname{Exercise}{exercise}{exercises}
\Crefname{Exercise}{Exercise}{Exercises}

\RequirePackage{titlesec}
\RequirePackage{ifthen}

% http://stackoverflow.com/questions/4932910/date-in-the-tabular-environment
\makeatletter
\let\insertdate\@date
\makeatother

\titleformat{\chapter}[display]
{\bfseries\Large}
{\color{DarkSlateGrey}\filleft \authorname
\ifthenelse{\isundefined{\studentnumber}}{}{\\ \studentnumber}
\ifthenelse{\isundefined{\email}}{}{\\ \email}
\ifthenelse{\isundefined{\dateintitle}}{}{\\ \insertdate}
%\ifthenelse{\isundefined{\coursename}}{}{\\ \coursename} % put in title instead.
}
{4ex}
{\color{DarkOliveGreen}{\titlerule}\color{Maroon}
\vspace{2ex}%
\filright}
[\vspace{2ex}%
\color{DarkOliveGreen}\titlerule
]

\newcommand{\beginArtWithToc}[0]{\begin{document}\tableofcontents}
\newcommand{\beginArtNoToc}[0]{\begin{document}}
\newcommand{\EndNoBibArticle}[0]{\end{document}}
\newcommand{\EndArticle}[0]{\bibliography{Bibliography}\bibliographystyle{plainnat}\end{document}}

% 
%\newcommand{\citep}[1]{\cite{#1}}

\colorSectionsForArticle



\usepackage{peeters_layout_exercise}
\usepackage{peeters_braket}
\usepackage{peeters_figures}
\usepackage{siunitx}
%\usepackage{mhchem} % \ce{}
%\usepackage{macros_bm} % \bcM
%\usepackage{macros_qed} % \qedmarker
%\usepackage{txfonts} % \ointclockwise

\beginArtNoToc

\generatetitle{XXX}
%\chapter{XXX}
%\label{chap:pseudovector}
% \citep{sakurai2014modern} pr X.Y
% \citep{pozar2009microwave}
% \citep{qftLectureNotes}
% \citep{doran2003gap}
% \citep{jackson1975cew}
% \citep{griffiths1999introduction}

In Geometric Algebra, the 3D cross product and the wedge product, can be related by an explicit multiplicative duality relationship

\begin{dmath}\label{eqn:WHAT:40}
\begin{aligned}
\Ba \wedge \Bb
&=
   \Be_1 \Be_2
\begin{vmatrix}
   a_1 & a_2 \\
   b_1 & b_2 \\
\end{vmatrix}
+
   \Be_1 \Be_3
\begin{vmatrix}
   a_1 & a_3 \\
   b_1 & b_3 \\
\end{vmatrix}
+
   \Be_2 \Be_3
\begin{vmatrix}
   a_2 & a_3 \\
   b_2 & b_3 \\
\end{vmatrix} \\
&=
   \Be_1 \Be_2 \Be_3 \Be_3
\begin{vmatrix}
   a_1 & a_2 \\
   b_1 & b_2 \\
\end{vmatrix}
+
   \Be_1 \Be_3 \Be_2 \Be_2
\begin{vmatrix}
   a_1 & a_3 \\
   b_1 & b_3 \\
\end{vmatrix}
+
   \Be_1 \Be_1 \Be_2 \Be_3
\begin{vmatrix}
   a_2 & a_3 \\
   b_2 & b_3 \\
\end{vmatrix} \\
&=
\Be_1 \Be_2 \Be_3
\lr{
      \Be_3
\begin{vmatrix}
   a_1 & a_2 \\
   b_1 & b_2 \\
\end{vmatrix}
-
   \Be_2
\begin{vmatrix}
   a_1 & a_3 \\
   b_1 & b_3 \\
\end{vmatrix}
+
   \Be_1
\begin{vmatrix}
   a_2 & a_3 \\
   b_2 & b_3 \\
\end{vmatrix}
} \\
&=
   I (\Ba \cross \Bb).
\end{aligned}
\end{dmath}

where \( I = \Be_1 \Be_2 \Be_3 \) is a pseudoscalar for the 3D space.

https://en.wikipedia.org/wiki/Bivector
The wedge product, a bivector, or grade-1 multivector, has a direct interpretation as an oriented area.  The cross product, as the dual, of that bivector, has a magnitude and direction that can also be interpretted as an oriented area, but it is a different beast since it is a grade 1 multivector.

This distinction is one that is important in some contexts.  For example, in physics the cross product isn't called a vector, but is called a pseudovector, since it has different transformation properties than a normal vector.

%}
\EndArticle
%\EndNoBibArticle
