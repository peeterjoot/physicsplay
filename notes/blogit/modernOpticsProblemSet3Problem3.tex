\makeproblem{\bf Spatial coherence and grating spectrometers}{modernOptics:problemSet3:3}{
A look inside a grating spectrometer reveals that incident light is passed through a series of slits to increase the transverse spatial coherence. In this problem, we'll try to understand why. For all of the parts below, consider a grating of $N$ slits, periodicity $a$, and width $b$. 

For parts \ref{modernOptics:problemSet3:3c} and \ref{modernOptics:problemSet3:3d}, neglect the envelope due to a finite slit width $b$, and consider only the sharp diffraction peaks.

\makesubproblem{Wavelength resolution}{modernOptics:problemSet3:3a}

For a collimated ($k_{s,y}=0$), monochromatic source illuminating $N$ slits, diffraction peaks would have an angular width of $\Delta \theta = \lambda / (N a \cos{\theta})$, for the first order of diffraction. {\bf re-derive this result} for yourself. Show that this gives a wavelength resolution is $\Delta \lambda = \lambda/N m$

\makesubproblem{Intensity}{modernOptics:problemSet3:3b}

Next, consider how the output intensity of the grating shifts if the input comes in at an angle $\theta_s$. {\bf Write an expression for $I(\theta_s, \theta)$.} You can also use the variables $k_{s,y}=k \sin \theta_s$ and $k_{y}=k \sin \theta$, as we did in class.

\makesubproblem{Resolution of spectrometer}{modernOptics:problemSet3:3c}
If the incident beam has an angular spread $\Delta \theta_s$ around normal incidence, what is the {\bf resolution $\Delta \lambda$ of the spectrometer?} Calculate this in the limit of large $N$, or $N \gg \lambda / a \Delta \theta_s$, where the angular width of the diffracted light is completely determined by the angular width of the incident light. 

\makesubproblem{Decreased coherence length at the grating}{modernOptics:problemSet3:3d}

An alternate view of part \ref{modernOptics:problemSet3:3c} is that by broadening the angular distribution of the source, we also decrease the transverse coherence length at the grating. The number of slits leading to coherent diffraction is reduced to some $N_{\mathrm{eff}}$, which is the number of slits within one coherence length $\ell_{\mathrm{tc}} = \lambda / \Delta \theta_s$. {\bf Sketch a diagram explaining this.} The frequency resolution of the spectrometer is then reduced from $\lambda/N$ to $\lambda/N_{\mathrm{eff}}$. (for order $m=1$) {\bf Compare this to the result you found in part \ref{modernOptics:problemSet3:3c}.}

} % makeproblem

\makeanswer{modernOptics:problemSet3:3}{
\makeSubAnswer{Output intensity given input angle $\theta_s$}{modernOptics:problemSet3:3b}

Let's derive the $N$ slit diffraction wave function and intensity given an off normal input.  We'll be able to use this in part \ref{modernOptics:problemSet3:3a} once we do.  We'll use a Fraunhofer geometry as in \cref{fig:modernOpticsProblemSet3Problem3:modernOpticsProblemSet3Problem3Fig1}.

\imageFigure{modernOpticsProblemSet3Problem3Fig1}{Fraunhofer geometry}{fig:modernOpticsProblemSet3Problem3:modernOpticsProblemSet3Problem3Fig1}{0.3}

\begin{subequations}
\begin{dmath}\label{eqn:modernOptics:problemSet3:3:20}
\BR + \Br' = \Br 
\end{dmath}
\begin{dmath}\label{eqn:modernOptics:problemSet3:3:40}
\BR_s + \Br' = \Br_s 
\end{dmath}
\end{subequations}

for which our path length from $\Br'$ to the observation point is

\begin{dmath}\label{eqn:modernOptics:problemSet3:3:60}
\Abs{\BR} 
= r \left( 1 + \frac{{r'}^2}{r^2} - 2 \frac{\Br \cdot \Br'}{r^2} \right)^{1/2}
\sim r + \frac{{r'}^2}{2 r^2} - \rcap \cdot \Br',
\end{dmath}

and to first order

\begin{dmath}\label{eqn:modernOptics:problemSet3:3:80}
k\Abs{\BR} 
\sim k r - k \rcap \cdot \Br'
= k r - k y' \sin\theta.
\end{dmath}

Similarily
\begin{dmath}\label{eqn:modernOptics:problemSet3:3:100}
k\Abs{\BR_s} 
\sim k r_s - k \rcap \cdot \Br'
= k r_s + \Bk \cdot \Br'
= k r_s + k \sin\theta_s y'
\end{dmath}

With 

\begin{subequations}
\begin{dmath}\label{eqn:modernOptics:problemSet3:3:120}
k_y \equiv \frac{2 \pi}{\lambda} \sin\theta,
\end{dmath}
\begin{dmath}\label{eqn:modernOptics:problemSet3:3:140}
k_{y,s} \equiv \frac{2 \pi}{\lambda} \sin\theta_s.
\end{dmath}
\end{subequations}

Our diffraction integral

\begin{dmath}\label{eqn:modernOptics:problemSet3:3:160}
\Psi \sim \int \frac{e^{i k (R + R_s)}}{ R R_s},
\end{dmath}

after pulling out and dropping the $r$ and $r_s$ dependent terms, takes the one dimensional form

\begin{dmath}\label{eqn:modernOptics:problemSet3:3:180}
\Psi(\Br) \sim \int e^{i (k_{y,s} - k_{y}) y' } dy'
=
\int e^{i \frac{2 \pi}{\lambda} (\sin\theta_s - \sin\theta) y' } dy'.
\end{dmath}

Let's write

\begin{dmath}\label{eqn:modernOptics:problemSet3:3:200}
\Delta k = k_{y} - k_{y,s},
\end{dmath}

and evaluate this over intervals $[h + m a, h + m a + b]$, for $m \in [0, N-1]$ as in \cref{fig:modernOpticsProblemSet3Problem3:modernOpticsProblemSet3Problem3Fig2}.

\imageFigure{modernOpticsProblemSet3Problem3Fig2}{N slit geometry}{fig:modernOpticsProblemSet3Problem3:modernOpticsProblemSet3Problem3Fig2}{0.3}

Integrating over the $m$th slit, we have

\begin{dmath}\label{eqn:modernOptics:problemSet3:3:220}
\int_{S_m} e^{-i \Delta k y' } dy'
=
\int_{h + m a }^{h + m a + b} e^{-i \Delta k y' } dy'
=
\evalrange{\frac{e^{-i \Delta k y' }}{-i \Delta k }}
{h + m a }{h + m a + b} 
=
\frac{e^{-i \Delta k (h + m a) }}{-i \Delta k }
\left( e^{-i \Delta k b } - 1 \right)
=
\frac{e^{-i \Delta k (h + m a ) }}{\Delta k }
e^{i \Delta k b/2 } 2 \sin( \Delta k b/2 )
=
b e^{-i \Delta k (h + m a ) }
e^{i \Delta k b/2 } \frac{\sin( \Delta k b/2 )}{ \Delta k b/2}
\end{dmath}

Adding all the slit contributions we have

\begin{dmath}\label{eqn:modernOptics:problemSet3:3:240}
\Psi = 
b 
e^{i \Delta k (b/2 -h) } 
\frac{\sin( \Delta k b/2 )}{ \Delta k b/2}
\sum_{m = 0}^{N-1}
e^{-i \Delta k m a }
=
b 
e^{i \Delta k (b/2 -h) } 
\frac{\sin( \Delta k b/2 )}{ \Delta k b/2}
\frac{1 - e^{-i \Delta k a N }}
{1 - e^{-i \Delta k a N }}
=
b 
e^{i \Delta k (b/2 -h) } 
\frac{\sin( \Delta k b/2 )}{ \Delta k b/2}
\frac
{
e^{-i \Delta k a N/2 }
}
{
e^{-i \Delta k a /2 }
}
\frac
{
e^{i \Delta k a N/2 } - e^{-i \Delta k a N/2 }
}
{
e^{i \Delta k a /2 } - e^{-i \Delta k a /2 }
}
\end{dmath}

\begin{dmath}\label{eqn:modernOptics:problemSet3:3:260}
\Psi \sim
\frac{\sin( \Delta k b/2 )}{ \Delta k b/2}
\frac
{\sin( \Delta k a N/2 )}
{N \sin( \Delta k a /2 )},
\end{dmath}

with intensity

\begin{subequations}
\begin{dmath}\label{eqn:modernOptics:problemSet3:3:280}
\boxed{
I(\theta_s, \theta)
\sim
\frac{\sin^2( \Delta k b/2 )}{ (\Delta k b/2)^2}
\frac
{\sin^2( \Delta k a N/2 )}
{N^2 \sin^2( \Delta k a /2 )},
}
\end{dmath}
\begin{dmath}\label{eqn:modernOptics:problemSet3:3:300}
\Delta k = \frac{2 \pi}{\lambda} ( \sin\theta - \sin\theta_s )
\end{dmath}
\end{subequations}

\makeSubAnswer{Peak width and wavelength resolution for normal incidence}{modernOptics:problemSet3:3a}

Here we work with a normal incident $\theta_s = 0$ plane wave source, and write

\begin{dmath}\label{eqn:modernOptics:problemSet3:3:320}
\gamma 
= \frac{\Delta k a}{2}
= \frac{k a}{2} \sin\theta
= \frac{\pi a}{\lambda} \sin\theta
\end{dmath}

and seek to understand the characteristics of the Intensity envelope

\begin{dmath}\label{eqn:modernOptics:problemSet3:3:340}
\frac{\sin^2( N \gamma / 2 )}
{N^2 \sin^2( \gamma/2) }.
\end{dmath}

To get a feel for what this may look like this is plotted for two wave lengths $\lambda = 3 \pi a$, $\lambda' = 4 \pi a$, $a = 1$ in \cref{fig:modernOpticsProblemSet3Problem3:modernOpticsProblemSet3Problem3Fig3}.

\imageFigure{modernOpticsProblemSet3Problem3Fig3}{Intensity envelope sample plot}{fig:modernOpticsProblemSet3Problem3:modernOpticsProblemSet3Problem3Fig3}{0.3}

Observe that this ratio of sines has a unit value for any $\gamma/2 = m \pi$, for integer $m$ since by H'\^{o}pital's rule we have

\begin{dmath}\label{eqn:modernOptics:problemSet3:3:360}
\lim_{\gamma/2 \rightarrow m \pi}
\frac{\sin( N \gamma / 2 )}
{N \sin( \gamma/2) }
=
\evalbar{\frac{\cos( N \gamma / 2 )}
{\cos( \gamma/2) }}{\gamma = m \pi}
= (-1)^{(N -1) m}
\end{dmath}

So for any $N \gamma/2 = l \pi$, provided $\gamma/2 \ne m \pi$ we have a zero.  We find those at

\begin{dmath}\label{eqn:modernOptics:problemSet3:3:380}
N \frac{\pi a}{\lambda} \sin\theta = l \pi,
\end{dmath}

or
\begin{dmath}\label{eqn:modernOptics:problemSet3:3:400}
\sin\theta_l = \frac{l \lambda}{N a},
\end{dmath}

For the distance between zeros past the center $\theta = 0$ lobe for a fixed wavelength, we have

\begin{dmath}\label{eqn:modernOptics:problemSet3:3:420}
\sin\theta_{l+1} - \sin\theta_l = 
\frac{(l+1) \lambda}{N a}
-
\frac{l \lambda}{N a}
=
\frac{\lambda}{N a}.
\end{dmath}

If $\Delta \theta_l$, or just $\Delta \theta$ (assuming that the peak or zero separation is about the same, although this is artifical in general as we see from the plo), we can compute this by examing the difference

\begin{dmath}\label{eqn:modernOptics:problemSet3:3:440}
\sin\theta_{l+1} - \sin\theta_l 
\sim 
\sin(\theta_l + \Delta \theta/2)
-\sin(\theta_l + \Delta \theta/2)
= 2 \cos \theta_l \sin (\Delta \theta/2)
\sim \Delta \theta \cos\theta_l.
\end{dmath}

This gives us the desired relationship (for the $l$th zero)

\begin{dmath}\label{eqn:modernOptics:problemSet3:3:460}
\boxed{
\Delta \theta_l \sim \frac{\lambda}{N a \cos\theta_l}.
}
\end{dmath}

Suppose we rather loosely identify this as the peak width, and look at the image around the $m$th peak, as in \cref{fig:modernOpticsProblemSet3Problem3:modernOpticsProblemSet3Problem3Fig4}.  This is about as close as the wavelength cans be that a superposition of the two would be distinguishable as separate (humped near center).

\imageFigure{modernOpticsProblemSet3Problem3Fig4}{Resolvable peak to peak separation}{fig:modernOpticsProblemSet3Problem3:modernOpticsProblemSet3Problem3Fig4}{0.3}

That separation of wavelength $\lambda' = \lambda + \Delta \lambda$ is

\begin{dmath}\label{eqn:modernOptics:problemSet3:3:480}
\Delta \theta
=
\sin^{-1}(m (\lambda + \Delta \lambda)/a)
-
\sin^{-1}(m \lambda/ a)
\sim
\frac{d}{d (m\lambda/a)} \left( \sin^{-1}(m \lambda/ a) \right) \frac{ m \Delta \lambda}{a}
=
\inv{\cos\sin^{-1} (m \lambda/a)}
\frac{ m \Delta \lambda}{a},
\end{dmath}

but we also have
\begin{dmath}\label{eqn:modernOptics:problemSet3:3:500}
\Delta \theta
=
\frac{\lambda}{N a \cos\theta}
=
\frac{\lambda}{N a \cos\sin^{-1}(m \lambda/a)}.
\end{dmath}

Comparing the two

\begin{dmath}\label{eqn:modernOptics:problemSet3:3:520}
\inv{\cancel{\cos\sin^{-1} (m \lambda/a)}}
\frac{ m \Delta \lambda}{\cancel{a}}
=
\frac{\lambda}{N \cancel{a} \cancel{\cos\sin^{-1}(m \lambda/a)}},
\end{dmath}

or

\begin{dmath}\label{eqn:modernOptics:problemSet3:3:540}
\boxed{
\Delta \lambda
=
\frac{\lambda}{N m},
}
\end{dmath}

which is the wavelength resolution desired.

\makeSubAnswer{XXX}{modernOptics:problemSet3:3c}
\makeSubAnswer{XXX}{modernOptics:problemSet3:3d}
%\paragraph{Part \ref{modernOptics:problemSet3:3a}.  XXX}
} % makeanswer

