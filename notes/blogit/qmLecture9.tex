%
% Copyright � 2015 Peeter Joot.  All Rights Reserved.
% Licenced as described in the file LICENSE under the root directory of this GIT repository.
%
\newcommand{\authorname}{Peeter Joot}
\newcommand{\email}{peeterjoot@protonmail.com}
\newcommand{\basename}{FIXMEbasenameUndefined}
\newcommand{\dirname}{notes/FIXMEdirnameUndefined/}

\renewcommand{\basename}{qmLecture9}
\renewcommand{\dirname}{notes/phy1520/}
\newcommand{\keywords}{PHY1520H}
\newcommand{\authorname}{Peeter Joot}
\newcommand{\onlineurl}{http://sites.google.com/site/peeterjoot2/math2013/\basename.pdf}
\newcommand{\sourcepath}{\dirname\basename.tex}
\newcommand{\generatetitle}[1]{\chapter{#1}}

\newcommand{\vcsinfo}{%
\section*{}
\noindent{\color{DarkOliveGreen}{\rule{\linewidth}{0.1mm}}}
\paragraph{Document version}
%\paragraph{\color{Maroon}{Document version}}
{
\small
\begin{itemize}
\item Available online at:\\ 
\href{\onlineurl}{\onlineurl}
\item Git Repository: \input{./.revinfo/gitRepo.tex}
\item Source: \sourcepath
\item last commit: \input{./.revinfo/gitCommitString.tex}
\item commit date: \input{./.revinfo/gitCommitDate.tex}
\end{itemize}
}
}

%\PassOptionsToPackage{dvipsnames,svgnames}{xcolor}
\PassOptionsToPackage{square,numbers}{natbib}
\documentclass{scrreprt}

\usepackage[left=2cm,right=2cm]{geometry}
\usepackage[svgnames]{xcolor}
\usepackage{peeters_layout}

\usepackage{natbib}

\usepackage[
colorlinks=true,
bookmarks=false,
pdfauthor={\authorname, \email},
backref 
]{hyperref}

% http://tex.stackexchange.com/questions/75773/how-to-reference-problems-by-the-text-label-in-an-exercise-envioronment
\usepackage[english]{cleveref}
\crefname{Exercise}{exercise}{exercises}
\Crefname{Exercise}{Exercise}{Exercises}

\RequirePackage{titlesec}
\RequirePackage{ifthen}

% http://stackoverflow.com/questions/4932910/date-in-the-tabular-environment
\makeatletter
\let\insertdate\@date
\makeatother

\titleformat{\chapter}[display]
{\bfseries\Large}
{\color{DarkSlateGrey}\filleft \authorname
\ifthenelse{\isundefined{\studentnumber}}{}{\\ \studentnumber}
\ifthenelse{\isundefined{\email}}{}{\\ \email}
\ifthenelse{\isundefined{\dateintitle}}{}{\\ \insertdate}
%\ifthenelse{\isundefined{\coursename}}{}{\\ \coursename} % put in title instead.
}
{4ex}
{\color{DarkOliveGreen}{\titlerule}\color{Maroon}
\vspace{2ex}%
\filright}
[\vspace{2ex}%
\color{DarkOliveGreen}\titlerule
]

\newcommand{\beginArtWithToc}[0]{\begin{document}\tableofcontents}
\newcommand{\beginArtNoToc}[0]{\begin{document}}
\newcommand{\EndNoBibArticle}[0]{\end{document}}
\newcommand{\EndArticle}[0]{\bibliography{Bibliography}\bibliographystyle{plainnat}\end{document}}

% 
%\newcommand{\citep}[1]{\cite{#1}}

\colorSectionsForArticle



%\usepackage{phy1520}
\usepackage{peeters_braket}
%\usepackage{peeters_layout_exercise}
\usepackage{peeters_figures}
\usepackage{mathtools}

\beginArtNoToc
\generatetitle{PHY1520H Graduate Quantum Mechanics.  Lecture 9: Dirac equation (cont.).  Taught by Prof.\ Arun Paramekanti}
%\chapter{Dirac equation (cont.)}
\label{chap:qmLecture9}

\paragraph{Disclaimer}

Peeter's lecture notes from class.  These may be incoherent and rough.

These are notes for the UofT course PHY1520, Graduate Quantum Mechanics, taught by Prof. Paramekanti.

\paragraph{Where we left off}

\begin{dmath}\label{eqn:qmLecture9:20}
-i \Hbar \PD{t}{} 
\begin{bmatrix}
\psi_1 \\
\psi_2
\end{bmatrix}
=
\begin{bmatrix}
-i \Hbar c \PD{x}{} & m c^2 \\
m c^2 & i \Hbar c \PD{x}{} \\
\end{bmatrix}.
\end{dmath}

With a potential this would be

\begin{dmath}\label{eqn:qmLecture9:40}
-i \Hbar \PD{t}{} 
\begin{bmatrix}
\psi_1 \\
\psi_2
\end{bmatrix}
=
\begin{bmatrix}
-i \Hbar c \PD{x}{} + V(x) & m c^2 \\
m c^2 & i \Hbar c \PD{x}{} + V(x) \\
\end{bmatrix}.
\end{dmath}

This means that the potential is raising the energy eigenvalue of the system.

\paragraph{Free Particle}

Assuming a form

\begin{dmath}\label{eqn:qmLecture9:60}
\begin{bmatrix}
\psi_1(x,t) \\
\psi_2(x,t)
\end{bmatrix}
=
e^{i k x}
\begin{bmatrix}
f_1(t) \\
f_2(t) \\
\end{bmatrix},
\end{dmath}

and plugging back into the Dirac equation we have

\begin{dmath}\label{eqn:qmLecture9:80}
-i \Hbar \PD{t}{} 
\begin{bmatrix}
f_1 \\
f_2
\end{bmatrix}
=
\begin{bmatrix}
k \Hbar c & m c^2 \\
m c^2 & - \Hbar k c \\
\end{bmatrix}
\begin{bmatrix}
f_1 \\
f_2
\end{bmatrix}.
\end{dmath}

We can use a diagonalizing rotation

\begin{dmath}\label{eqn:qmLecture9:100}
\begin{bmatrix}
f_1 \\
f_2
\end{bmatrix}
=
\begin{bmatrix}
\cos\theta_k & -\sin\theta_k \\
\sin\theta_k & \cos\theta_k \\
\end{bmatrix}
\begin{bmatrix}
f_{+} \\
f_{-} \\
\end{bmatrix}.
\end{dmath}

Plugging this in reduces the system to the form

\begin{dmath}\label{eqn:qmLecture9:140}
-i \Hbar \PD{t}{} 
\begin{bmatrix}
f_{+} \\
f_{-} \\
\end{bmatrix}
=
\begin{bmatrix}
E_k & 0 \\
0 & -E_k
\end{bmatrix}
\begin{bmatrix}
f_{+} \\
f_{-} \\
\end{bmatrix}.
\end{dmath}

Where the rotation angle is found to be given by

\begin{dmath}\label{eqn:qmLecture9:160}
\begin{aligned}
\sin(2 \theta_k) &= \frac{m c^2}{\sqrt{(\Hbar k c)^2 + m^2 c^4}} \\
\cos(2 \theta_k) &= \frac{\Hbar k c}{\sqrt{(\Hbar k c)^2 + m^2 c^4}} \\
E_k &= \sqrt{(\Hbar k c)^2 + m^2 c^4}
\end{aligned}
\end{dmath}

See 

F1 

for a sketch of energy vs momentum.  The asympotes are the limiting cases when \( m c^2 \rightarrow 0 \).  The \( + \) branch is what we usually associate with particles.  What about the other energy states.  For fermions Dirac argued that the lower energy states could be thought of as ``filled up'', using the Pauli principle to leave only the positive energy states available.  This was called the ``Dirac Sea''.  This isn't a good solution, and won't work for example for bosons.

Another way to rationalize this is to employ ideas from solid state theory.  For example consider a semiconductor with a valence and conduction band

F2

A photon can excite an electrom from the valence band to the conduction band, leaving all the valence band states filled except for one (a hole).  For an electron we can use almost the same picture:

F3

A photon with energy \( E_k - (-E_k) \) can create a positron-electron pair from the vacuum, where the energy of the electron and positron pair is \( E_k \).
%, and the energy of the positron is \( E_k \).  
At high enough energies, we can see this pair creation occur.

\paragraph{Zitterbewegung}

If a particle is created at a non-eigenstate such as one on the asymptote, then oscillations between the positive and negative branches are possible

F4

Only ``vertical" oscillations between the positive and negative locations on these branches is possible since those are the points that match the particle momenum.  Examining this will be the aim of one of the problem set problems.

\paragraph{Probability and current density}

If we define a probability density

\begin{dmath}\label{eqn:qmLecture9:180}
\rho(x, t) = \Abs{\psi_1}^2 + \Abs{\psi_2}^2,
\end{dmath}

does this satisfy a probability conservation relation

\begin{dmath}\label{eqn:qmLecture9:200}
\PD{t}{\rho} + \PD{x}{j} = 0,
\end{dmath}

where \( j \) is the probability current.  Plugging in the density, we have

\begin{dmath}\label{eqn:qmLecture9:220}
\PD{t}{\rho}
=
\PD{t}{\psi_1^\conj} \psi_1
+
\psi_1 \PD{t}{\psi_1^\conj} 
+
\PD{t}{\psi_2^\conj} \psi_2
+
\psi_2 \PD{t}{\psi_2^\conj} 
\end{dmath}

It turns out that the probability current has the form

\begin{dmath}\label{eqn:qmLecture9:240}
j(x,t) = c \lr{ \psi_1^\conj \psi_1 + \psi_2^\conj \psi_2 }.
\end{dmath}

FIXME: show this.

Here the speed of light \( c \) is the slope of the line in the plots above.  We can think of this current density as right movers minus the left movers.  Any state that is given can be thought of as a combination of right moving and left moving states, neither of which are eigenstates of the free particle Hamiltonian.

\paragraph{Potential step}

The next logical thing to think about, as in non-relativisitic quantum mechanics, is to think about what occurs when the particle hits a potential step, as in

F5

The approach is the same.  We write down the wave functions for the \( V = 0 \) region (I), and the higher potential region (II).

The eigenstates are found on the solid lines above the asymptotes on the right hand movers side

F6

For \( k > 0 \), an eigenstate for the incident wave is

\begin{dmath}\label{eqn:qmLecture9:260}
\Bpsi_{\textrm{inc}}(x) = 
\begin{bmatrix}
\cos\theta_k \\
\sin\theta_k
\end{bmatrix}
e^{i k x},
\end{dmath}

For the reflected wave function, we pick a function on the left moving side of the positive energy branch.

\begin{dmath}\label{eqn:qmLecture9:260}
\Bpsi_{\textrm{ref}}(x) = 
\begin{bmatrix}
? \\
?
\end{bmatrix}
e^{-i k x},
\end{dmath}

We'll go through this in more detail next time.

\makeproblem{Calculate the right going diagonalization}{problem:qmLecture9:1}{

Prove \cref{eqn:qmLecture9:160}.
} % problem

\makeanswer{problem:qmLecture9:1}{

To determine the relations for \( \theta_k \) we have to solve 

\begin{dmath}\label{eqn:qmLecture9:280}
\begin{bmatrix}
E_k & 0 \\
0 & -E_k
\end{bmatrix}
= R^{-1} H R.
\end{dmath}

Working with \( \Hbar = c = 1 \) temporarily, and \( C = \cos\theta_k, S = \sin\theta_k \), that is

\begin{dmath}\label{eqn:qmLecture9:300}
\begin{bmatrix}
E_k & 0 \\
0 & -E_k
\end{bmatrix}
=
\begin{bmatrix}
C & S \\
-S & C
\end{bmatrix}
\begin{bmatrix}
k & m \\
m & -k
\end{bmatrix}
\begin{bmatrix}
C & -S \\
S & C
\end{bmatrix}
=
\begin{bmatrix}
C & S \\
-S & C
\end{bmatrix}
\begin{bmatrix}
k C + m S & -k S + m C \\
m C - k S & -m S - k C
\end{bmatrix}
=
\begin{bmatrix}
k C^2 + m S C + m C S - k S^2   & -k S C + m C^2 -m S^2 - k C S \\
-k C S - m S^2 + m C^2 - k S C & k S^2 - m C S -m S C - k C^2
\end{bmatrix}
=
\begin{bmatrix}
k \cos(2 \theta_k) + m \sin(2 \theta_k) & m \cos(2 \theta_k) - k \sin(2 \theta_k) \\
m \cos(2 \theta_k) - k \sin(2 \theta_k) & -k \cos(2 \theta_k) - m \sin(2 \theta_k) \\
\end{bmatrix}.
\end{dmath}

This gives

\begin{dmath}\label{eqn:qmLecture9:320}
E_k 
\begin{bmatrix}
1 \\
0
\end{bmatrix}
=
\begin{bmatrix}
k \cos(2 \theta_k) + m \sin(2 \theta_k) \\
m \cos(2 \theta_k) - k \sin(2 \theta_k) \\
\end{bmatrix}
=
\begin{bmatrix}
k & m \\
m & -k
\end{bmatrix}
\begin{bmatrix}
\cos(2 \theta_k) \\
\sin(2 \theta_k) \\
\end{bmatrix}.
\end{dmath}

Adding back in the \(\Hbar\)'s and \(c\)'s this is

\begin{dmath}\label{eqn:qmLecture9:340}
\begin{bmatrix}
\cos(2 \theta_k) \\
\sin(2 \theta_k) \\
\end{bmatrix}
=
\frac{E_k}{-(\Hbar k c)^2 -(m c^2)^2}
\begin{bmatrix}
- \Hbar k c & - m c^2 \\
- m c^2     & \Hbar k c
\end{bmatrix}
\begin{bmatrix}
1 \\
0
\end{bmatrix}
=
\inv{E_k}
\begin{bmatrix}
\Hbar k c \\
m c^2
\end{bmatrix}.
\end{dmath}
} % answer
 
\EndNoBibArticle
