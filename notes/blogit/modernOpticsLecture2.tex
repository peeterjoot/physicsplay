%
% Copyright � 2012 Peeter Joot.  All Rights Reserved.
% Licenced as described in the file LICENSE under the root directory of this GIT repository.
%
% pick one:
%\newcommand{\authorname}{Peeter Joot}
\newcommand{\email}{peeter.joot@utoronto.ca}
\newcommand{\studentnumber}{920798560}
\newcommand{\basename}{FIXMEbasenameUndefined}
\newcommand{\dirname}{notes/FIXMEdirnameUndefined/}

\newcommand{\authorname}{Peeter Joot}
\newcommand{\email}{peeterjoot@protonmail.com}
\newcommand{\basename}{FIXMEbasenameUndefined}
\newcommand{\dirname}{notes/FIXMEdirnameUndefined/}

\renewcommand{\basename}{modernOpticsLecture2}
\renewcommand{\dirname}{notes/phy485/}
\newcommand{\keywords}{Optics, PHY485H1F, Rays, plane waves, index of refraction, Eikonal equation}

\newcommand{\authorname}{Peeter Joot}
\newcommand{\onlineurl}{http://sites.google.com/site/peeterjoot2/math2013/\basename.pdf}
\newcommand{\sourcepath}{\dirname\basename.tex}
\newcommand{\generatetitle}[1]{\chapter{#1}}

\newcommand{\vcsinfo}{%
\section*{}
\noindent{\color{DarkOliveGreen}{\rule{\linewidth}{0.1mm}}}
\paragraph{Document version}
%\paragraph{\color{Maroon}{Document version}}
{
\small
\begin{itemize}
\item Available online at:\\ 
\href{\onlineurl}{\onlineurl}
\item Git Repository: \input{./.revinfo/gitRepo.tex}
\item Source: \sourcepath
\item last commit: \input{./.revinfo/gitCommitString.tex}
\item commit date: \input{./.revinfo/gitCommitDate.tex}
\end{itemize}
}
}

%\PassOptionsToPackage{dvipsnames,svgnames}{xcolor}
\PassOptionsToPackage{square,numbers}{natbib}
\documentclass{scrreprt}

\usepackage[left=2cm,right=2cm]{geometry}
\usepackage[svgnames]{xcolor}
\usepackage{peeters_layout}

\usepackage{natbib}

\usepackage[
colorlinks=true,
bookmarks=false,
pdfauthor={\authorname, \email},
backref 
]{hyperref}

% http://tex.stackexchange.com/questions/75773/how-to-reference-problems-by-the-text-label-in-an-exercise-envioronment
\usepackage[english]{cleveref}
\crefname{Exercise}{exercise}{exercises}
\Crefname{Exercise}{Exercise}{Exercises}

\RequirePackage{titlesec}
\RequirePackage{ifthen}

% http://stackoverflow.com/questions/4932910/date-in-the-tabular-environment
\makeatletter
\let\insertdate\@date
\makeatother

\titleformat{\chapter}[display]
{\bfseries\Large}
{\color{DarkSlateGrey}\filleft \authorname
\ifthenelse{\isundefined{\studentnumber}}{}{\\ \studentnumber}
\ifthenelse{\isundefined{\email}}{}{\\ \email}
\ifthenelse{\isundefined{\dateintitle}}{}{\\ \insertdate}
%\ifthenelse{\isundefined{\coursename}}{}{\\ \coursename} % put in title instead.
}
{4ex}
{\color{DarkOliveGreen}{\titlerule}\color{Maroon}
\vspace{2ex}%
\filright}
[\vspace{2ex}%
\color{DarkOliveGreen}\titlerule
]

\newcommand{\beginArtWithToc}[0]{\begin{document}\tableofcontents}
\newcommand{\beginArtNoToc}[0]{\begin{document}}
\newcommand{\EndNoBibArticle}[0]{\end{document}}
\newcommand{\EndArticle}[0]{\bibliography{Bibliography}\bibliographystyle{plainnat}\end{document}}

% 
%\newcommand{\citep}[1]{\cite{#1}}

\colorSectionsForArticle



\beginArtNoToc

\generatetitle{Geometric optics: Rays.}
\label{chap:modernOpticsLecture2}
\section{Disclaimer}

Peeter's lecture notes from class.  May not be entirely coherent.

\section{Where are the rays in Maxwell's equations}

\begin{align}\label{eqn:modernOpticsLecture2:10}
\spacegrad \cdot \BD &= 0 \\
\spacegrad \cdot \BB &= 0 \\
\spacegrad \cross \BE &= - \inv{c} \PD{t}{\BB} \\
\spacegrad \cross \BB &= \inv{c} \PD{t}{\BD}
\end{align}

Assume

\begin{enumerate}
\item 
Material has no magnetic dependence ($\mu = 1$) so that we have

\begin{equation}\label{eqn:modernOpticsLecture2:5}
n = c \sqrt{\epsilon}
\end{equation}

Neglect loss, imaginary part of $n$
\item
Short wavelength limit $\lambda \ll d$, any other length scale in problem.
\end{enumerate}

If $n = \text{constant}$, we know that plane waves are solutions.  Try

\begin{equation}\label{eqn:modernOpticsLecture2:30}
\begin{bmatrix}
\BE \\
\BB \\
\end{bmatrix}
=
\begin{bmatrix}
\BE_0(\Br) \\
\BB_0(\Br) \\
\end{bmatrix}
e^{i \phi(\Br) - i \omega t}
\end{equation}

We know that for plane waves we'll have
\begin{align}\label{eqn:modernOpticsLecture2:50}
\BE_0(\Br) &\rightarrow \BE \\
\BB_0(\Br) &\rightarrow \BB \\
\phi(\Br) &\rightarrow \Bk \cdot \Br.
\end{align}

The time derivatives are
\begin{equation}\label{eqn:modernOpticsLecture2:70}
\inv{c} \PD{t}{}
\begin{bmatrix}
\BE \\
\BB
\end{bmatrix}
=
- i \frac{\omega}{c}
\begin{bmatrix}
\BE \\
\BB
\end{bmatrix}
=
- i k_0
\begin{bmatrix}
\BE \\
\BB
\end{bmatrix}.
\end{equation}

For the spatial derivatives we have 

\begin{equation}\label{eqn:modernOpticsLecture2:90}
\spacegrad \cdot \BE = 
e^{-i \omega t}
\left(
\underbrace{
e^{i \phi(\Br)}
\spacegrad \cdot \BE_0(\Br) 
}_{\text{neglect this}}
+
\BE_0(\Br) \cdot \left( \spacegrad e^{i \phi(\Br)} \right)
\right),
\end{equation}

and
\begin{equation}\label{eqn:modernOpticsLecture2:91}
\spacegrad \cross \BE = 
e^{-i \omega t}
\left(
\underbrace{
e^{i \phi(\Br)}
\spacegrad \cross \BE_0(\Br) 
}_{\text{neglect this}}
-
\BE_0(\Br) \cross \left( \spacegrad e^{i \phi(\Br)} \right)
\right).
\end{equation}

Computing the phase gradient

\begin{align*}
\spacegrad e^{i\phi}
&=
\Be_m \partial_m e^{i \phi} \\
&=
i \Be_m \partial_m \phi e^{i \phi} \\
&=
i (\spacegrad \phi) e^{i \phi} \\
\end{align*}

This leaves us with
\begin{equation}\label{eqn:modernOpticsLecture2:90a}
\spacegrad \cdot \BE \approx
i e^{i \phi -i \omega t} \BE_0(\Br) \cdot \spacegrad \phi
\end{equation}
\begin{equation}\label{eqn:modernOpticsLecture2:91b}
\spacegrad \cross \BE \approx
-i e^{i \phi -i \omega t}
\BE_0(\Br) \cross \spacegrad \phi.
\end{equation}

Note that to justify the neglect of the gradients products of $\BE_0$ we use approximations of the form

\begin{equation}\label{eqn:modernOpticsLecture2:110}
\inv{E_0^x} \frac{d E_0^x}{dx} \ll \inv{\lambda}
\end{equation}

FIXME: clarify with review.

because

\begin{equation}\label{eqn:modernOpticsLecture2:130}
\spacegrad \phi \approx k_0 \sim \frac{2 \pi}{\lambda}
\end{equation}

Maxwell's equations now take the form

\begin{subequations}
\begin{equation}\label{eqn:modernOpticsLecture2:150}
\BE_0 \cdot \spacegrad \phi = 0
\end{equation}
\begin{equation}\label{eqn:modernOpticsLecture2:170}
\BB_0 \cdot \spacegrad \phi = 0
\end{equation}
\begin{equation}\label{eqn:modernOpticsLecture2:190}
\spacegrad \phi \cross \BE_0 = k_0 \BB_0
\end{equation}
\begin{equation}\label{eqn:modernOpticsLecture2:210}
\spacegrad \phi \cross \BB_0 = - \epsilon k_0 \BE_0
\end{equation}
\end{subequations}

Crossing $\spacegrad \phi$ with \ref{eqn:modernOpticsLecture2:190} we have

\begin{align*}
\spacegrad \phi \cross (\spacegrad \phi \cross \BE_0) &= k_0 (\spacegrad \phi \cross \BB_0) \\
\spacegrad \phi \left( \cancel{\spacegrad \phi \cdot \BE_0} \right) - \BE_0 (\spacegrad \phi)^2 &= - \epsilon k_0^2 \BE_0
\end{align*}

This is called the Eikonal equation and can be written as

\begin{equation}\label{eqn:modernOpticsLecture2:230}
\Abs{\spacegrad \phi}^2 = k_0^2 \epsilon(\Br)
\end{equation}

or
\begin{equation}\label{eqn:modernOpticsLecture2:250}
\boxed{
\Abs{\spacegrad \phi} = k_0 n(\Br)
}
\end{equation}

If $n = \text{constant}$ 

\begin{equation}\label{eqn:modernOpticsLecture2:270}
\Abs{\spacegrad \phi} = k_0 n
\end{equation}

This can be illustrated as in figure (\ref{fig:modernOpticsLecture2:modernOpticsLecture2Fig1}).

\imageFigure{modernOpticsLecture2Fig1}{Plane waves for constant index of refraction}{fig:modernOpticsLecture2:modernOpticsLecture2Fig1}{0.2}

If $n \ne \text{constant}$ only locally would we have plane waves as in figure (\ref{fig:modernOpticsLecture2:modernOpticsLecture2Fig2}).

\imageFigure{modernOpticsLecture2Fig2}{Plane waves only locally with variation of index of refraction}{fig:modernOpticsLecture2:modernOpticsLecture2Fig2}{0.2}

\section{Poynting vector}

How about the Poynting vector?  This is the direction of the ``ray'', the direction of the transport of energy and momentum.  That is

\begin{equation}\label{eqn:modernOpticsLecture2:290}
\BS = \frac{c}{4 \pi} \Real{\BE} \cross \Real{\BB},
\end{equation}

and after some math, taking the average we have

\begin{equation}\label{eqn:modernOpticsLecture2:310}
\expectation{\BS}_{\text{time}} = \frac{c}{8 \pi k_0} \Abs{\BE_0}^2 \spacegrad \phi
\end{equation}

We see that the rays point along $\spacegrad \phi$.

FIXME: do this math.

\section{Ray equation}

Referring to figure (\ref{fig:modernOpticsLecture2:modernOpticsLecture2Fig3}) we let 
\imageFigure{modernOpticsLecture2Fig3}{Unit tangents on a curve}{fig:modernOpticsLecture2:modernOpticsLecture2Fig3}{0.2}

\begin{equation}\label{eqn:modernOpticsLecture2:330}
s = \text{distance along ray}
\end{equation}
\begin{equation}\label{eqn:modernOpticsLecture2:350}
\Bt = \text{tangent} = \frac{d\Br(s)}{ds}
\end{equation}

The unit vector, parallel to $\spacegrad \phi$ is

\begin{align*}
\frac{d\Br(s)}{ds} = \Bt 
&= \frac{\spacegrad \phi}{\Abs{\spacegrad \phi}} \\
&= \frac{\spacegrad \phi}{ n(\Br) k_0},
\end{align*}

So we have

\begin{equation}\label{eqn:modernOpticsLecture2:370}
n(\Br) \frac{d\Br}{ds} = \inv{k_0} \spacegrad \phi.
\end{equation}

We'd like to get rid of the pesky dependence on the phase.  Let's take another derivative to attempt to get rid of $\spacegrad \phi$.  Will this work?

\begin{align*}
\frac{d}{ds} \left( n(\Br) \frac{d\Br}{ds} \right)
&=
\inv{k_0} \frac{d}{ds} \spacegrad \phi \\
&=
\inv{k_0} \left( \frac{d\Br}{ds} \cdot \spacegrad \right) \spacegrad \phi \\
&=
\inv{k_0} \left( \inv{ k_0 n(\Br) } \spacegrad \phi \cdot \spacegrad \right) 
\spacegrad \phi
\end{align*}

It can be shown that (will be posted)

\begin{equation}\label{eqn:modernOpticsLecture2:390}
(\spacegrad \phi \cdot \spacegrad ) \cdot \spacegrad \phi = k_0^2 n \spacegrad n
\end{equation}

which gives us the \underline{Ray equation}

\begin{equation}\label{eqn:modernOpticsLecture2:410}
\boxed{
\frac{d}{ds} \left( n(\Br) \frac{d\Br}{ds} \right) = \spacegrad n(\Br)
}
\end{equation}

Note that this almost looks like a $F = m a$ type of equation with time parameterization replaced by arc length along the ray (should we ignore the index of refraction on the LHS), and also ignore the lack of a minus sign.

The lack of minus sign we can interpret as something like ``bending to higher $n$''.

\section{GRIN (Graded Refractive INdex) optics}

With a constant index we have

\begin{equation}\label{eqn:modernOpticsLecture2:430}
\frac{d}{ds} \left( n \frac{d\Br}{ds} \right) = \spacegrad n = 0
\end{equation}

So

\begin{equation}\label{eqn:modernOpticsLecture2:450}
\frac{d^2}{ds^2} \Br(s) = 0
\end{equation}

We have a straight ray with

%\dr/ds = const
\begin{equation}\label{eqn:modernOpticsLecture2:470}
\Br = s \Ba + \Br_0.
\end{equation}

Know that tangent 

\begin{equation}\label{eqn:modernOpticsLecture2:490}
\frac{d\Br}{ds} = \Ba = \inv{n k_0} \spacegrad \phi
\end{equation}

Phase along ray path?

\begin{equation}\label{eqn:modernOpticsLecture2:510}
\phi(\Br) = \omega t
\end{equation}

(of wave front)

We have

\begin{equation}\label{eqn:modernOpticsLecture2:530}
\Ba \cdot \Br = \frac{\omega t}{n \omega/c} = \frac{c}{n} t.
\end{equation}

This is for the plane wave.  

\section{Trap a ray}

Let's have some fun with non-constant $n$.  Can we trap a ray of light as in figure (\ref{fig:modernOpticsLecture2:modernOpticsLecture2Fig4})?

\imageFigure{modernOpticsLecture2Fig4}{Ray trap}{fig:modernOpticsLecture2:modernOpticsLecture2Fig4}{0.2}

If we have a circular trajectory

\begin{equation}\label{eqn:modernOpticsLecture2:550}
\Br = R
\begin{bmatrix}
\cos\theta(s) \\
\sin\theta(s) \\
0
\end{bmatrix}.
\end{equation}

We can imagine any sort of variation of $n$ with $\Br$, such as figure (\ref{fig:modernOpticsLecture2:modernOpticsLecture2Fig5}), but 
we want to figure out exactly what $n(\Br)$ has to be.  We can do so by plugging into \ref{eqn:modernOpticsLecture2:410}.  Assuming $n(\Br)$ isn't dependent on $s$ we have

\imageFigure{modernOpticsLecture2Fig5}{Imagined possible relationship between index of refraction and position.}{fig:modernOpticsLecture2:modernOpticsLecture2Fig5}{0.2}

\begin{equation}\label{eqn:modernOpticsLecture2:570}
\frac{d^2}{ds^2} \Br(s) = \frac{\spacegrad n(\Br)}{n(\Br)},
\end{equation}

FIXME: justify this assumption.

Taking derivatives, we have

\begin{equation}\label{eqn:modernOpticsLecture2:590}
\frac{d \Br}{ds} = R
\begin{bmatrix}
-\sin\theta(s) \\
\cos\theta(s) \\
0
\end{bmatrix}
\frac{d\theta}{ds}
\end{equation}

%and
%\begin{equation}\label{eqn:modernOpticsLecture2:610}
%\frac{d^2 \Br}{ds^2} = -R
%\begin{bmatrix}
%\cos\theta(s) \\
%\sin\theta(s) \\
%0
%\end{bmatrix}.
%\end{equation}

We find

\begin{equation}\label{eqn:modernOpticsLecture2:630}
\frac{d^2 \Br}{ds^2} = \frac{\spacegrad n}{n}
\end{equation}

or

\begin{equation}\label{eqn:modernOpticsLecture2:650}
\spacegrad n = -n \frac{\Br}{R^2}.
\end{equation}

Trap your own ray today!


%   \makeproblem{description}{ch1:pr1}{ }
%   \makeanswer{ch1:pr1}{ }
%
% not needed with print2.tex:
%   \shipoutAnswer

% this is to produce the sites.google url and version info and so forth (for blog posts)
\vcsinfo
%\EndArticle
\EndNoBibArticle
