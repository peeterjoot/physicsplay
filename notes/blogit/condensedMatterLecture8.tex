%
% Copyright � 2013 Peeter Joot.  All Rights Reserved.
% Licenced as described in the file LICENSE under the root directory of this GIT repository.
%
\newcommand{\authorname}{Peeter Joot}
\newcommand{\email}{peeterjoot@protonmail.com}
\newcommand{\basename}{FIXMEbasenameUndefined}
\newcommand{\dirname}{notes/FIXMEdirnameUndefined/}

\renewcommand{\basename}{condensedMatterLecture8}
\renewcommand{\dirname}{notes/phy487/}
\newcommand{\keywords}{Condensed matter physics, PHY487H1F}
\newcommand{\authorname}{Peeter Joot}
\newcommand{\onlineurl}{http://sites.google.com/site/peeterjoot2/math2013/\basename.pdf}
\newcommand{\sourcepath}{\dirname\basename.tex}
\newcommand{\generatetitle}[1]{\chapter{#1}}

\newcommand{\vcsinfo}{%
\section*{}
\noindent{\color{DarkOliveGreen}{\rule{\linewidth}{0.1mm}}}
\paragraph{Document version}
%\paragraph{\color{Maroon}{Document version}}
{
\small
\begin{itemize}
\item Available online at:\\ 
\href{\onlineurl}{\onlineurl}
\item Git Repository: \input{./.revinfo/gitRepo.tex}
\item Source: \sourcepath
\item last commit: \input{./.revinfo/gitCommitString.tex}
\item commit date: \input{./.revinfo/gitCommitDate.tex}
\end{itemize}
}
}

%\PassOptionsToPackage{dvipsnames,svgnames}{xcolor}
\PassOptionsToPackage{square,numbers}{natbib}
\documentclass{scrreprt}

\usepackage[left=2cm,right=2cm]{geometry}
\usepackage[svgnames]{xcolor}
\usepackage{peeters_layout}

\usepackage{natbib}

\usepackage[
colorlinks=true,
bookmarks=false,
pdfauthor={\authorname, \email},
backref 
]{hyperref}

% http://tex.stackexchange.com/questions/75773/how-to-reference-problems-by-the-text-label-in-an-exercise-envioronment
\usepackage[english]{cleveref}
\crefname{Exercise}{exercise}{exercises}
\Crefname{Exercise}{Exercise}{Exercises}

\RequirePackage{titlesec}
\RequirePackage{ifthen}

% http://stackoverflow.com/questions/4932910/date-in-the-tabular-environment
\makeatletter
\let\insertdate\@date
\makeatother

\titleformat{\chapter}[display]
{\bfseries\Large}
{\color{DarkSlateGrey}\filleft \authorname
\ifthenelse{\isundefined{\studentnumber}}{}{\\ \studentnumber}
\ifthenelse{\isundefined{\email}}{}{\\ \email}
\ifthenelse{\isundefined{\dateintitle}}{}{\\ \insertdate}
%\ifthenelse{\isundefined{\coursename}}{}{\\ \coursename} % put in title instead.
}
{4ex}
{\color{DarkOliveGreen}{\titlerule}\color{Maroon}
\vspace{2ex}%
\filright}
[\vspace{2ex}%
\color{DarkOliveGreen}\titlerule
]

\newcommand{\beginArtWithToc}[0]{\begin{document}\tableofcontents}
\newcommand{\beginArtNoToc}[0]{\begin{document}}
\newcommand{\EndNoBibArticle}[0]{\end{document}}
\newcommand{\EndArticle}[0]{\bibliography{Bibliography}\bibliographystyle{plainnat}\end{document}}

% 
%\newcommand{\citep}[1]{\cite{#1}}

\colorSectionsForArticle



%\citep{harald2003solid} \S x.y

% shorthand used when taking notes, replaced after with ad-hoc vim regexes
%\nai === {n \alpha i}
%\mbj === {m \beta j}
%\ai === {\alpha i}
%\bj === {\beta j}
%\n1 === {n, 1}
%\n-12 === {n-1, 2}
%\n+12 === {n+1, 2}

%\usepackage{mhchem}
\usepackage[version=3]{mhchem}

\beginArtNoToc
\generatetitle{PHY487H1F Condensed Matter Physics.  Lecture 8: Phonons (cont.).  Taught by Prof.\ Stephen Julian}
%\chapter{Phonons (cont.)}
\label{chap:condensedMatterLecture8}

\section{Disclaimer}

Peeter's lecture notes from class.  May not be entirely coherent.

\section{Phonons (cont.)}

Last time, considering a 1D linear harmonic chain

\begin{subequations}
\begin{dmath}\label{eqn:condensedMatterLecture8:20}
w_q = \sqrt{ \frac{ 4 k }{m } } \Abs{ \sin \frac{ q a }{ 2 } }.
\end{dmath}
\begin{dmath}\label{eqn:condensedMatterLecture8:40}
q = \frac{2 \pi n}{L}
\end{dmath}
\end{subequations}

These were described as wave like solutions, but these are in fact the normal modes of oscillations.

Sketching these

F1

(a) At $q = 0$ we really have uniform translation of the entire chain.
(b) At $q = a$ we have displaced, but also uniform translation of the entire chain.
(c) At $q = a/2$ we have maximum oscillation.

\section{Real solids, and potential energy.}

Reading: \citep{ibach2009solid} \S 4.1

F2

Our problems in 3D are mostly notational.  We'll write 

\begin{dmath}\label{eqn:condensedMatterLecture8:380}
u_{n \alpha i}
\end{dmath}

for the displacement of the $\alpha$th atom in the nth unit cell, in the ith ($i \in \{x, y, z\}$ direction.

The total potential energy can be written

\begin{dmath}\label{eqn:condensedMatterLecture8:400}
\Phi(
\mathLabelBox{r_{n \alpha i}}{equilibrium position}
 + 
\mathLabelBox
[
   labelstyle={below of=m\themathLableNode, below of=m\themathLableNode}
]
{u_{n \alpha i}}{displacement}
) = 
\Phi(r_{n \alpha i}) + \inv{2} \sum_{{n \alpha i}, {m \beta j}} 
\mathLabelBox{\frac{\partial^2 \Phi}{\partial r_{n \alpha i} \partial {m \beta j}} }{$\Phi^{m \beta j}_{n \alpha i}$}
\times u_{n \alpha i} u_{m \beta j}
\end{dmath}

\makeexample{1D chain}{example:condensedMatterLecture8:1}{

F3

\begin{dmath}\label{eqn:condensedMatterLecture8:60}
\Phi = \inv{2} k u_{ix}^2 + \inv{2} k u_{ix}^2 = 
\end{dmath}

\begin{dmath}\label{eqn:condensedMatterLecture8:80}
I^{nx}_{nx} = 2 k.
\end{dmath}

}

\section{Equation of motion}

Force on atom $\alpha$ in unit cell $n$, in direction $i$, due to displacement of atom $\beta$ in cell m in direction $j$ ($\BF = -\spacegrad \Phi$), is

\begin{dmath}\label{eqn:condensedMatterLecture8:100}
-\Phi^{m \beta j}_{n \alpha i} u_{m \beta j}
\end{dmath}

\begin{dmath}\label{eqn:condensedMatterLecture8:120}
M_\alpha \ddot{u}_{n \alpha i} + \sum_{m \beta j} \Phi^{m \beta j}_{n \alpha i} u_{m \beta j} = 0
\end{dmath}

example, 

\begin{dmath}\label{eqn:condensedMatterLecture8:140}
m \ddot{u}_j = 
k( u_{j + 1} - u_j )
+ k( u_{j-1} - u_j ).
\end{dmath}

Using trial solution

\begin{dmath}\label{eqn:condensedMatterLecture8:160}
u_{n \alpha i} = \inv{\sqrt{m_\alpha}} \sum_q u_{\alpha i}(q) e^{-i (\Bq \cdot \Br_n - \omega_q t ) }
\end{dmath}

This yields, after operation with $\sum_{q'} e^{i \Bq' \cdot \Br_n}$ as before, and cancelling terms \index{dynamical matrix}

\begin{dmath}\label{eqn:condensedMatterLecture8:180}
-\omega_q^2 u_{\alpha i}(q) + \sum_{\beta j} 
\mathLabelBox
[
   labelstyle={xshift=2cm},
   linestyle={out=270,in=90, latex-}
]
{
\sum_m \inv{\sqrt{ m_\alpha m_\beta} } \Phi^{m \beta j}_{n \alpha i} e^{i \Bq \cdot (\Br_m - \Br_m) } 
}{$D^{\beta j}_{\alpha i}$, the Dynamical matrix, independent of $n$}
u_{\beta j}(q) = 0,
\end{dmath}

or

\begin{dmath}\label{eqn:condensedMatterLecture8:200}
-\omega_q^2 u_{\alpha i}(q) + \sum_{\beta j} D^{\beta j}_{\alpha i} u_{\beta j}(q) = 0.
\end{dmath}

We want to solve 

\begin{dmath}\label{eqn:condensedMatterLecture8:220}
\det\lr{ 
D^{\beta j}_{\alpha i} - \omega_q^2 \BOne
} = 0.
\end{dmath}

\makeexample{diatomic linear chain}{example:condensedMatterLecture8:2}{

F4

Our force equation is now

\begin{subequations}
\begin{dmath}\label{eqn:condensedMatterLecture8:240}
M_1 \ddot{u}_{n, 1} + \Phi^{n - 1, 2}_{n, 1} u_{n - 1, 2} + \Phi^{n, 1}_{n, 1} u_{n, 1} + \Phi^{n, 2}_{n, 1} u_{n, 2} = 0
\end{dmath}
\begin{dmath}\label{eqn:condensedMatterLecture8:260}
M_2 \ddot{u}_{n, 2} + \Phi^{n, 1}_{n, 2} u_{n - 1, 2} + \Phi^{n, 2}_{n, 2} u_{n, 2} + \Phi^{n + 1, 1}_{n, 2} u_{n + 1, 1} = 0
\end{dmath}
\end{subequations}

Or

\begin{subequations}
\label{eqn:condensedMatterLecture8:280a}
\begin{dmath}\label{eqn:condensedMatterLecture8:280}
M_1 \ddot{u}_{n, 1} + f\lr{ 2 u_{n, 1} - u_{n - 1, 2} - u_{n, 2}} = 0
\end{dmath}
\begin{dmath}\label{eqn:condensedMatterLecture8:300}
M_2 \ddot{u}_{n, 2} + f\lr{ 2 u_{n, 2} - u_{n, 1} - u_{n + 1, 1}} = 0
\end{dmath}
\end{subequations}

This gives

\begin{subequations}
\begin{dmath}\label{eqn:condensedMatterLecture8:320}
\Phi^{n, 1}_{n, 1} = \Phi^{n, 2}_{n, 2} = 2 f
\end{dmath}
\begin{dmath}\label{eqn:condensedMatterLecture8:340}
\Phi^{n, 2}_{n, 1} = -f
\end{dmath}
\end{subequations}

Substitution to use is:

\begin{dmath}\label{eqn:condensedMatterLecture8:360}
u_{n, \alpha} = \inv{M_\alpha} u_\alpha(q) e^{i (
\mathLabelBox{a n}{$x_n$}
- \omega_q t)
}
\end{dmath}

Substitution into \eqnref{eqn:condensedMatterLecture8:280a} gives

\begin{subequations}
\begin{dmath}\label{eqn:condensedMatterLecture8:420}
\lr{ \frac{2 f}{M_1} - \omega_q^2 } u_1(q) - \frac{f}{\sqrt{M_1 M_2}} \lr{ 1 + e^{-i q a} } u_2(q) = 0
\end{dmath}
\begin{dmath}\label{eqn:condensedMatterLecture8:440}
- \frac{f}{\sqrt{M_1 M_2}} \lr{ 1 + e^{i q a} } u_1(q) 
+\lr{ \frac{2 f}{M_2} - \omega_q^2 } u_2(q) 
= 0
\end{dmath}
\end{subequations}

We thus want to solve

\begin{dmath}\label{eqn:condensedMatterLecture8:460}
\begin{vmatrix}
\lr{ \frac{2 f}{M_1} - \omega_q^2 } &- \frac{f}{\sqrt{M_1 M_2}} \lr{ 1 + e^{-i q a} } \\
- \frac{f}{\sqrt{M_1 M_2}} \lr{ 1 + e^{i q a} } 
&\lr{ \frac{2 f}{M_2} - \omega_q^2 } 
\end{vmatrix}
= 0,
\end{dmath}

With solution
\begin{dmath}\label{eqn:condensedMatterLecture8:480}
\omega_q^2 = f\lr{ \inv{M_1} + \inv{M_2} } \pm f \sqrt{
\inv{M_1} + \inv{M_2} - \frac{4}{M_1 M_2} \sin^2 \frac{q a}{2}
}.
\end{dmath}

If we plot these 

F5

There are two solutions for $q = 0$

$\omega^2 = 0$, or $\omega^2 = 2 f ( 1/M_1 + 1/M_2 )$.

(a)

F6

(b)

F7

(c)

F8

(d)

F9

Reading \S 4.3
}

\EndArticle
