%
% Copyright � 2013 Peeter Joot.  All Rights Reserved.
% Licenced as described in the file LICENSE under the root directory of this GIT repository.
%
\newcommand{\authorname}{Peeter Joot}
\newcommand{\email}{peeterjoot@protonmail.com}
\newcommand{\basename}{FIXMEbasenameUndefined}
\newcommand{\dirname}{notes/FIXMEdirnameUndefined/}

\renewcommand{\basename}{statMechCheatSheet}
\renewcommand{\dirname}{notes/phy452/}
%\newcommand{\dateintitle}{}
%\newcommand{\keywords}{}

\newcommand{\authorname}{Peeter Joot}
\newcommand{\onlineurl}{http://sites.google.com/site/peeterjoot2/math2013/\basename.pdf}
\newcommand{\sourcepath}{\dirname\basename.tex}
\newcommand{\generatetitle}[1]{\chapter{#1}}

\newcommand{\vcsinfo}{%
\section*{}
\noindent{\color{DarkOliveGreen}{\rule{\linewidth}{0.1mm}}}
\paragraph{Document version}
%\paragraph{\color{Maroon}{Document version}}
{
\small
\begin{itemize}
\item Available online at:\\ 
\href{\onlineurl}{\onlineurl}
\item Git Repository: \input{./.revinfo/gitRepo.tex}
\item Source: \sourcepath
\item last commit: \input{./.revinfo/gitCommitString.tex}
\item commit date: \input{./.revinfo/gitCommitDate.tex}
\end{itemize}
}
}

%\PassOptionsToPackage{dvipsnames,svgnames}{xcolor}
\PassOptionsToPackage{square,numbers}{natbib}
\documentclass{scrreprt}

\usepackage[left=2cm,right=2cm]{geometry}
\usepackage[svgnames]{xcolor}
\usepackage{peeters_layout}

\usepackage{natbib}

\usepackage[
colorlinks=true,
bookmarks=false,
pdfauthor={\authorname, \email},
backref 
]{hyperref}

% http://tex.stackexchange.com/questions/75773/how-to-reference-problems-by-the-text-label-in-an-exercise-envioronment
\usepackage[english]{cleveref}
\crefname{Exercise}{exercise}{exercises}
\Crefname{Exercise}{Exercise}{Exercises}

\RequirePackage{titlesec}
\RequirePackage{ifthen}

% http://stackoverflow.com/questions/4932910/date-in-the-tabular-environment
\makeatletter
\let\insertdate\@date
\makeatother

\titleformat{\chapter}[display]
{\bfseries\Large}
{\color{DarkSlateGrey}\filleft \authorname
\ifthenelse{\isundefined{\studentnumber}}{}{\\ \studentnumber}
\ifthenelse{\isundefined{\email}}{}{\\ \email}
\ifthenelse{\isundefined{\dateintitle}}{}{\\ \insertdate}
%\ifthenelse{\isundefined{\coursename}}{}{\\ \coursename} % put in title instead.
}
{4ex}
{\color{DarkOliveGreen}{\titlerule}\color{Maroon}
\vspace{2ex}%
\filright}
[\vspace{2ex}%
\color{DarkOliveGreen}\titlerule
]

\newcommand{\beginArtWithToc}[0]{\begin{document}\tableofcontents}
\newcommand{\beginArtNoToc}[0]{\begin{document}}
\newcommand{\EndNoBibArticle}[0]{\end{document}}
\newcommand{\EndArticle}[0]{\bibliography{Bibliography}\bibliographystyle{plainnat}\end{document}}

% 
%\newcommand{\citep}[1]{\cite{#1}}

\colorSectionsForArticle



\beginArtNoToc

\generatetitle{Cheat sheet}
%\chapter{Cheat sheet}
%\label{chap:statMechCheatSheet}

\paragraph{central limit theorem}

If $\expectation{x} = \mu$ and $\sigma^2 = \expectation{x^2} - \expectation{x}^2$, and $X = \sum x$, then in the limit

\begin{subequations}
\begin{equation}
\lim_{N \rightarrow \infty} P(X)
= \inv{\sigma \sqrt{2 \pi N}} \exp\left( - \frac{ (x - N \mu)^2}{2 N \sigma^2} \right)
\end{equation}
\begin{equation}
\expectation{X} = N \mu
\end{equation}
\begin{equation}
\expectation{X^2} - \expectation{X}^2 = N \sigma^2
\end{equation}
\end{subequations}

\paragraph{binomial distribution}
\begin{equation}
P_N(X) 
= 
\left\{
\begin{array}{l l}
\left(\inv{2}\right)^N 
\frac{N!}{
\left(\frac{N-X}{2}\right)!
\left(\frac{N+X}{2}\right)!
}
& \quad \mbox{if $X$ and $N$ have same parity} \\
0& \quad \mbox{otherwise} 
\end{array},
\right.
\end{equation}

where $X$ was something like number of Heads minus number of Tails.

\paragraph{generating function}

Given the Fourier transform of a probability distribution $\tilde{P}(k)$ we have

\begin{dmath}
\evalbar{ 
\frac{\partial^n}{\partial k^n} 
   \tilde{P}(k) 
}{k = 0}
= (-i)^n \expectation{x^n}.
\end{dmath}
\paragraph{handy mathematics}

\begin{dmath}
\ln( 1 + x ) = x - \frac{x^2}{2} + \frac{x^3}{3} - \frac{x^4}{4}
\end{dmath}

\begin{equation}
N! \approx \sqrt{ 2 \pi N} N^N e^{-N}
\end{equation}
\begin{equation}
\ln N! \approx \sqrt{ 2 \pi N} N^N e^{-N}
\end{equation}

\begin{equation}
\erf(z) = \frac{2}{\sqrt{\pi}} \int_0^z e^{-t^2} dt
\end{equation}

\begin{equation}
\Gamma(\alpha) = \int_0^\infty dy e^{-y} y^{\alpha - 1}
\end{equation}

\begin{equation}
\Gamma(\alpha + 1) = \alpha \Gamma(\alpha)
\end{equation}

\begin{equation}
\Gamma\lr{1/2} = \sqrt{\pi}
\end{equation}

\begin{subequations}
\begin{equation}
P(x, t) = \int_{-\infty}^\infty \frac{dk}{2 \pi} \tilde{P}(k, t) \exp\left( i k x \right)
\end{equation}
\begin{equation}
\tilde{P}(k, t) = \int_{-\infty}^\infty dx P(x, t) \exp\left( -i k x \right)
\end{equation}
\end{subequations}

Heavyside theta

\begin{subequations}
\begin{equation}
\Theta(x) = 
\left\{
\begin{array}{l l}
1 & \quad x \ge 0 \\
0 & \quad x < 0
\end{array}
\right.
\end{equation}
\begin{equation}
\frac{d\Theta}{dx} = \delta(x),
\end{equation}
\end{subequations}

\paragraph{Radius of gyration of a 3D polymer}

With radius $a$, we have

\begin{dmath}
r_N \approx a \sqrt{N}
\end{dmath}

\paragraph{random walk}

1D Random walk 

\begin{dmath}
\mathcal{P}( x, t ) 
= 
\inv{2} \mathcal{P}(x + \delta x, t - \delta t)
+
\inv{2} \mathcal{P}(x - \delta x, t - \delta t)
\end{dmath}

leads to
\begin{equation}
\PD{t}{\mathcal{P}}(x, t) =
\inv{2} 
\frac{(\delta x)^2}{\delta t}
\PDSq{x}{\mathcal{P}}(x, t) 
= D \PDSq{x}{\mathcal{P}}(x, t) 
= 
-\PD{x}{J},
\end{equation}

The diffusion constant relation to the probability current is referred to as Fick's law

\begin{dmath}
D = -\PD{x}{J}
\end{dmath}

with which we can cast the probability diffusion identity into a continuity equation form

\begin{dmath}
\PD{t}{\mathcal{P}} + \PD{x}{J} = 0 
\end{dmath}

In 3D (with the Maxwell distribution frictional term), this takes the form

\begin{subequations}
\begin{equation}
\Bj = -D \spacegrad_\Bv c(\Bv, t) - \eta \Bv c(\Bv, t)
\end{equation}
\begin{equation}
\PD{t}{} c(\Bv, t) + \spacegrad_\Bv \cdot \Bj(\Bv, t) = 0
\end{equation}
\end{subequations}

\paragraph{velocity random walk}

Find
\begin{equation}
\mathcal{P}_{N_{\mathrm{c}}}(\Bv) \propto e^{-
\frac{(\Bv - \Bv_0)^2}{2 N_{\mathrm{c}}}
}
\end{equation}

\paragraph{Maxwell distribution}

Add a frictional term to the velocity space diffusion current

\begin{equation}
j_v = -D \PD{v}{c}(v, t) - \eta v c(v).
\end{equation}

For steady state the continity equation $0 = \frac{dc}{dt} = -\PD{v}{j_v}$ leads to

\begin{equation}
c(v) \propto \exp
\left(
- \frac{\eta v^2}{2 D}
\right).
\end{equation}

We also find
\begin{dmath}
\expectation{v^2} = \frac{D}{\eta},
\end{dmath}

and identify
\begin{equation}
\inv{2} m \expectation{\Bv^2} = \inv{2} m \left( \frac{D}{\eta} \right) = \inv{2} \kB T
\end{equation}

\paragraph{Hamilton's equations}

\begin{subequations}
\begin{equation}
\PD{p}{H} = \xdot
\end{equation}
\begin{equation}
\PD{x}{H} = -\pdot
\end{equation}
\end{subequations}

SHO

\begin{subequations}
\begin{equation}
H = \frac{p^2}{2m} + \inv{2} k x^2
\end{equation}
\begin{equation}
\omega^2 = \frac{k}{m}
\end{equation}
\end{subequations}

Quantum energy eigenvalues
\begin{equation}
E_n = \lr{ n + \inv{2} }  \Hbar \omega
\end{equation}

\paragraph{Liouville's theorem}

\begin{dmath}
\ddt{\rho} 
%= \PD{t}{\rho} + \PD{t}{x} \PD{x}{\rho} + \PD{t}{p} \PD{p}{\rho}
= \PD{t}{\rho} + \xdot \PD{x}{\rho} + \pdot \PD{p}{\rho}
= \cdots
= \PD{t}{\rho} + \PD{x}{\lr{\xdot \rho}} + \PD{p}{\lr{\xdot \rho}} 
= \PD{t}{\rho} + \spacegrad_{x,p} \cdot (\rho \xdot, \rho \pdot)
= \PD{t}{\rho} + \spacegrad \cdot \BJ
= 0,
\end{dmath}

Regardless of whether we have a steady state system, if we sit on a region of phase space volume, the probability density in that neighbourhood will be constant.

\paragraph{Ergodic}

A system for which all accessible phase space is swept out by the trajectories.  This and Liouville's threorm allows us to assume that we can treat any given small phase space volume as if it is equally probable to the same time evolved phase space region, and switch to ensemble averaging instead of time averaging.

\paragraph{volume}

\begin{equation}
V_m
= 
\frac{ \pi^{m/2} R^{m} }
{
   \Gamma\left( m/2 + 1 \right)
}.
\end{equation}


\paragraph{thermodynamics}

\begin{subequations}
\begin{dmath}
dE = T dS - P dV + \mu dN
\end{dmath}
\begin{dmath}
\inv{T} = \PDc{E}{S}{N,V}
\end{dmath}
\begin{dmath}
\frac{P}{T} = \PDc{V}{S}{N,E}
\end{dmath}
\begin{dmath}
-\frac{\mu}{T} = \PDc{N}{S}{V,E}
\end{dmath}
\begin{dmath}
P = - \PDc{V}{E}{N,S}
\end{dmath}
\begin{dmath}
\mu = \PDc{N}{E}{V,S}
\end{dmath}
\begin{dmath}
T = \PDc{S}{E}{N,V}
\end{dmath}
\begin{dmath}
A = E - TS
\end{dmath}
\begin{equation}
G = A + P V = E - T S + P V = \mu N
\end{equation}
\begin{equation}
H = E + P V = G + T S
\end{equation}
\begin{equation}
\CV = T \PDc{T}{S}{N,V} = \PDc{T}{E}{N,V}
\end{equation}
\begin{equation}
\CP = T \PDc{T}{S}{N,P} = \PDc{T}{H}{N,P}
\end{equation}
\end{subequations}

\paragraph{microstates}

\begin{dmath}
\beta = \inv{\kB T}
\end{dmath}

\begin{dmath}
S = \kB \ln \Omega
\end{dmath}

\begin{dmath}
\Omega(N, V, E) = 
\inv{h^{3N} N!}
\int_V 
d\Bx_1 
\cdots
d\Bx_N 
\int
d\Bp_1 
\cdots
d\Bp_N 
\delta \left(
E 
- \frac{\Bp_1^2}{2 m}
\cdots
- \frac{\Bp_N^2}{2 m}
\right)
=
\frac{V^N}{h^{3N} N!}
\int
d\Bp_1 
\cdots
d\Bp_N 
\delta \left(
E 
- \frac{\Bp_1^2}{2m}
\cdots
- \frac{\Bp_N^2}{2m}
\right)
\end{dmath}

\begin{equation}
\Omega = \frac{d\gamma}{dE}
\end{equation}

\begin{equation}
\gamma
=
\frac{V^N}{h^{3N} N!}
\int
d\Bp_1 
\cdots
d\Bp_N 
\Theta \left(
E 
- \frac{\Bp_1^2}{2m}
\cdots
- \frac{\Bp_N^2}{2m}
\right)
\end{equation}

quantum

\begin{equation}
\gamma = \sum_i \Theta(E - \epsilon_i)
\end{equation}

\paragraph{ideal gas}

\begin{equation}
\Omega = \frac{V^N}{N!} \inv{h^{3N}} \frac{( 2 \pi m E)^{3 N/2 }}{E} \frac{1}{\Gamma( 3N/2 ) }
\end{equation}

\paragraph{quantum free particle in a box}

\begin{subequations}
\begin{equation}
\Psi_{n_1, n_2, n_3}(x, y, z) = \lr{\frac{2}{L}}^{3/2} 
\sin\lr{ \frac{ n_1 \pi x}{L} }
\sin\lr{ \frac{ n_2 \pi x}{L} }
\sin\lr{ \frac{ n_3 \pi x}{L} }
\end{equation}
\begin{equation}
\epsilon_{n_1, n_2, n_3} = \frac{h^2}{8 m L^2} \lr{ n_1^2 + n_2^2 + n_3^2 }
\end{equation}
\end{subequations}

%\EndArticle
\EndNoBibArticle
