%
% Copyright � 2013 Peeter Joot.  All Rights Reserved.
% Licenced as described in the file LICENSE under the root directory of this GIT repository.
%
\makeproblem{Bose-Einstein condensation (BEC) in one and two dimensions}{basicStatMech:problemSet7:1}{ 
\makesubproblem{}{basicStatMech:problemSet7:1a}

Obtain the density of states $N(\epsilon)$ in one and two dimensions for a particle with an energy-momentum relation

\begin{equation}\label{eqn:basicStatMechProblemSet7Problem1:20}
E_\Bk = \frac{\hbar^2 \Bk^2}{2 m}.
\end{equation}

\makesubproblem{}{basicStatMech:problemSet7:1b}
Using this, show that for particles whose number is conserved the BEC transition temperature vanishes in these cases - so we can always pick a chemical potential $\mu < 0$ which preserves a constant density at any temperature.

} % makeproblem

\makeanswer{basicStatMech:problemSet7:1}{ 
\makeSubAnswer{}{basicStatMech:problemSet7:1a}

We'd like to evalute

\begin{dmath}\label{eqn:basicStatMechProblemSet7Problem1:460}
N(\epsilon) \equiv
\sum_\Bk
\delta(\epsilon - \epsilon_\Bk)
\approx
\frac{V}{(2 \pi)^d} \int d^d \Bk \delta\lr{ \epsilon - \frac{\hbar^2 k^2}{2 m}},
\end{dmath}

We'll use
\begin{dmath}\label{eqn:basicStatMechLecture17:440}
\delta(g(x)) = \sum_{x_0} \frac{\delta(x - x_0)}{\Abs{g'(x_0)}},
\end{dmath}

where the roots of $g(x)$ are $x_0$.  With 

\begin{dmath}\label{eqn:basicStatMechProblemSet7Problem1:480}
g(k) = \epsilon - \frac{\hbar^2 k^2}{2 m},
\end{dmath}

the roots $k^\conj$ of $g(k) = 0$ are 

\begin{dmath}\label{eqn:basicStatMechProblemSet7Problem1:500}
k^\conj = \pm \sqrt{\frac{2 m \epsilon }{\hbar^2}}.
\end{dmath}

The derivative of $g(k)$ evaluated at these roots are

\begin{dmath}\label{eqn:basicStatMechProblemSet7Problem1:520}
g'(k^\conj) = 
-\frac{\hbar^2 k^\conj}{m}
=
\mp 
\frac{\hbar^2}{m}
\frac{\sqrt{2 m \epsilon}}{ \hbar }
=
\mp 
\frac{\hbar \sqrt{2 m \epsilon} }{m}.
\end{dmath}

In 2D, we can evaluate over a shell in $k$ space

\begin{dmath}\label{eqn:basicStatMechProblemSet7Problem1:540}
N_2(\epsilon) 
=
\frac{A}{(2 \pi)^2} \int_0^\infty 2 \pi k dk
\lr{ \delta\lr{ k - k^\conj }
+ \delta\lr{ k + k^\conj } }
\frac{m}{\hbar \sqrt{2 m \epsilon} }
=
\frac{A}{2 \pi} \cancel{k^\conj}
\frac{m}{\hbar^2 \cancel{k^\conj} }
\end{dmath}

or

\begin{dmath}\label{eqn:basicStatMechProblemSet7Problem1:560}
\myBoxed{
N_2(\epsilon) 
=
\frac{2 \pi A m}{h^2}.
}
\end{dmath}

In 1D we have

\begin{dmath}\label{eqn:basicStatMechProblemSet7Problem1:580}
N_1(\epsilon) 
=
\frac{L}{2 \pi} \int_{-\infty}^\infty dk
\lr{ \delta\lr{ k - k^\conj }
+ \delta\lr{ k + k^\conj } }
\frac{m}{\hbar \sqrt{2 m \epsilon} }
=
\frac{2 L}{2 \pi} 
\frac{m}{\hbar \sqrt{2 m \epsilon} }.
\end{dmath}

Observe that this time for 1D, unlike in 2D when we used a radial shell in $k$ space, we have contributions from both the delta function roots.  Our end result is

\begin{dmath}\label{eqn:basicStatMechProblemSet7Problem1:600}
\myBoxed{
N_1(\epsilon) 
=
\frac{2 L}{h} 
\sqrt{\frac{m}{2 \epsilon}}.
}
\end{dmath}

\makeSubAnswer{}{basicStatMech:problemSet7:1b}

TODO.
}

