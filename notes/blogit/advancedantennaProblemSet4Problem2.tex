%
% Copyright � 2015 Peeter Joot.  All Rights Reserved.
% Licenced as described in the file LICENSE under the root directory of this GIT repository.
%
\makeproblem{Schelkunoff z-axis, zero phase shifts.}{advancedantenna:problemSet4:2}{ 

Use the Schelkunoff method to design a linear array of isotropic elements placed along
the z-axis such that the zeros of the array factor are located at \( \theta = \ang{0}, \ang{60}, \ang{120} \).
The
inter-element spacing is \( d = \lambda/2 \) and the progressive phase shift is zero degrees.

\makesubproblem{}{advancedantenna:problemSet4:2a}
What is the required number of the elements?
\makesubproblem{}{advancedantenna:problemSet4:2b}
Determine the corresponding current excitation coefficients
\makesubproblem{}{advancedantenna:problemSet4:2c}
Find the array factor
\makesubproblem{}{advancedantenna:problemSet4:2d}
Plot the corresponding array factor
} % makeproblem

\makeanswer{advancedantenna:problemSet4:2}{ 
\makeSubAnswer{}{advancedantenna:problemSet4:2a}

To avoid complex numbers in \( \textrm{AF}(z) \), introduce zeros at \( \pm \ang{60}, \pm \ang{120} \), for

\begin{dmath}\label{eqn:advancedantennaProblemSet4Problem2:20}
\textrm{AF}(z) = 
z
\lr{ z - e^{j \pi/3} }
\lr{ z - e^{-j \pi/3} }
\lr{ z - e^{j 2\pi/3} }
\lr{ z - e^{-j 2\pi/3} }
=
z
\lr{ z^2 - 2 z \cos( \pi/3 ) + 1 }
\lr{ z^2 - 2 z \cos( 2\pi/3 ) + 1 }
= 
z 
\lr{ z^2 - 2 z ( 1/2 ) + 1 }
\lr{ z^2 - 2 z ( -1/2 ) + 1 }
= 
z 
\lr{ (z^2 + 1)^2 - z^2 }
= 
z 
\lr{ z^4 + z^2 + 1 }.
\end{dmath}

Normalized this is

%\begin{dmath}\label{eqn:advancedantennaProblemSet4Problem2:40}
\boxedEquation{eqn:advancedantennaProblemSet4Problem2:60}{
\textrm{AF} = \frac{z}{3}\lr{ 1 + z^2 + z^3 } = \inv{3} \lr{ z + z^3 + z^5 }.
}
%\end{dmath}

This can either be thought of as three elements spaced with separations \( d = \lambda \), or six with spacing \( d = \lambda \), but half of them having zero currents.

\makeSubAnswer{}{advancedantenna:problemSet4:2b}

With the specified \( d = \lambda/2 \) spacing, the currents at positions \( \Br_m = m d \zcap \) are

\begin{equation}\label{eqn:advancedantennaProblemSet4Problem2:80}
\begin{aligned}
I_0 &= 0 \\
I_1 &= \inv{3} \\
I_2 &= 0 \\
I_3 &= \inv{3} \\
I_4 &= 0 \\
I_5 &= \inv{3}.
\end{aligned}
\end{equation}

\makeSubAnswer{}{advancedantenna:problemSet4:2c}

Substituting \( z = e^{j \pi \cos\theta} \), the array factor is

\begin{dmath}\label{eqn:advancedantennaProblemSet4Problem2:100}
\textrm{AF} 
= \inv{3} \lr{ e^{j \pi \cos\theta} + e^{3 j \pi \cos\theta} + e^{5 j \pi \cos\theta} }
= \frac{e^{3 j \pi \cos\theta}}{3} \lr{ e^{-2 j \pi \cos\theta} + 1 + e^{2 j \pi \cos\theta} }
= \frac{e^{3 j \pi \cos\theta}}{3} \lr{ 1 + 2 \cos\lr{2 \pi \cos\theta} }.
\end{dmath}

\makeSubAnswer{}{advancedantenna:problemSet4:2d}

TODO.
}
