%
% Copyright � 2015 Peeter Joot.  All Rights Reserved.
% Licenced as described in the file LICENSE under the root directory of this GIT repository.
%
\documentclass[]{eliblog}

\usepackage{amsmath}
\usepackage{mathpazo}

%
% shorthand for bold symbols, convenient for vectors and matrices
%
\newcommand{\Ba}[0]{\mathbf{a}}
\newcommand{\Bb}[0]{\mathbf{b}}
\newcommand{\Bc}[0]{\mathbf{c}}
\newcommand{\Bd}[0]{\mathbf{d}}
\newcommand{\Be}[0]{\mathbf{e}}
\newcommand{\Bf}[0]{\mathbf{f}}
\newcommand{\Bg}[0]{\mathbf{g}}
\newcommand{\Bh}[0]{\mathbf{h}}
\newcommand{\Bi}[0]{\mathbf{i}}
\newcommand{\Bj}[0]{\mathbf{j}}
\newcommand{\Bk}[0]{\mathbf{k}}
\newcommand{\Bl}[0]{\mathbf{l}}
\newcommand{\Bm}[0]{\mathbf{m}}
\newcommand{\Bn}[0]{\mathbf{n}}
\newcommand{\Bo}[0]{\mathbf{o}}
\newcommand{\Bp}[0]{\mathbf{p}}
\newcommand{\Bq}[0]{\mathbf{q}}
\newcommand{\Br}[0]{\mathbf{r}}
\newcommand{\Bs}[0]{\mathbf{s}}
\newcommand{\Bt}[0]{\mathbf{t}}
\newcommand{\Bu}[0]{\mathbf{u}}
\newcommand{\Bv}[0]{\mathbf{v}}
\newcommand{\Bw}[0]{\mathbf{w}}
\newcommand{\Bx}[0]{\mathbf{x}}
\newcommand{\By}[0]{\mathbf{y}}
\newcommand{\Bz}[0]{\mathbf{z}}
\newcommand{\BA}[0]{\mathbf{A}}
\newcommand{\BB}[0]{\mathbf{B}}
\newcommand{\BC}[0]{\mathbf{C}}
\newcommand{\BD}[0]{\mathbf{D}}
\newcommand{\BE}[0]{\mathbf{E}}
\newcommand{\BF}[0]{\mathbf{F}}
\newcommand{\BG}[0]{\mathbf{G}}
\newcommand{\BH}[0]{\mathbf{H}}
\newcommand{\BI}[0]{\mathbf{I}}
\newcommand{\BJ}[0]{\mathbf{J}}
\newcommand{\BK}[0]{\mathbf{K}}
\newcommand{\BL}[0]{\mathbf{L}}
\newcommand{\BM}[0]{\mathbf{M}}
\newcommand{\BN}[0]{\mathbf{N}}
\newcommand{\BO}[0]{\mathbf{O}}
\newcommand{\BP}[0]{\mathbf{P}}
\newcommand{\BQ}[0]{\mathbf{Q}}
\newcommand{\BR}[0]{\mathbf{R}}
\newcommand{\BS}[0]{\mathbf{S}}
\newcommand{\BT}[0]{\mathbf{T}}
\newcommand{\BU}[0]{\mathbf{U}}
\newcommand{\BV}[0]{\mathbf{V}}
\newcommand{\BW}[0]{\mathbf{W}}
\newcommand{\BX}[0]{\mathbf{X}}
\newcommand{\BY}[0]{\mathbf{Y}}
\newcommand{\BZ}[0]{\mathbf{Z}}

\newcommand{\Bzero}[0]{\mathbf{0}}
\newcommand{\Btheta}[0]{\boldsymbol{\theta}}
\newcommand{\Btau}[0]{\boldsymbol{\tau}}
\newcommand{\Bomega}[0]{\boldsymbol{\omega}}

%
% shorthand for unit vectors
%
\newcommand{\acap}[0]{\hat{\Ba}}
\newcommand{\bcap}[0]{\hat{\Bb}}
\newcommand{\ccap}[0]{\hat{\Bc}}
\newcommand{\dcap}[0]{\hat{\Bd}}
\newcommand{\ecap}[0]{\hat{\Be}}
\newcommand{\fcap}[0]{\hat{\Bf}}
\newcommand{\gcap}[0]{\hat{\Bg}}
\newcommand{\hcap}[0]{\hat{\Bh}}
\newcommand{\icap}[0]{\hat{\Bi}}
\newcommand{\jcap}[0]{\hat{\Bj}}
\newcommand{\kcap}[0]{\hat{\Bk}}
\newcommand{\lcap}[0]{\hat{\Bl}}
\newcommand{\mcap}[0]{\hat{\Bm}}
\newcommand{\ncap}[0]{\hat{\Bn}}
\newcommand{\ocap}[0]{\hat{\Bo}}
\newcommand{\pcap}[0]{\hat{\Bp}}
\newcommand{\qcap}[0]{\hat{\Bq}}
\newcommand{\rcap}[0]{\hat{\Br}}
\newcommand{\scap}[0]{\hat{\Bs}}
\newcommand{\tcap}[0]{\hat{\Bt}}
\newcommand{\ucap}[0]{\hat{\Bu}}
\newcommand{\vcap}[0]{\hat{\Bv}}
\newcommand{\wcap}[0]{\hat{\Bw}}
\newcommand{\xcap}[0]{\hat{\Bx}}
\newcommand{\ycap}[0]{\hat{\By}}
\newcommand{\zcap}[0]{\hat{\Bz}}
\newcommand{\thetacap}[0]{\hat{\Btheta}}

%
% to write R^n and C^n in a distinguishable fashion.  Perhaps change this
% to the double lined characters upon figuring out how to do so.
%
\newcommand{\C}[1]{$\mathbb{C}^{#1}$}
\newcommand{\R}[1]{$\mathbb{R}^{#1}$}

%
% various generally useful helpers
%

% derivative of #1 wrt. #2:
\newcommand{\D}[2] {\frac {d#2} {d#1}}

\newcommand{\inv}[1]{\frac{1}{#1}}
\newcommand{\cross}[0]{\times}

\newcommand{\abs}[1]{\lvert{#1}\rvert}
\newcommand{\norm}[1]{\lVert{#1}\rVert}
\newcommand{\innerprod}[2]{\langle{#1}, {#2}\rangle}
\newcommand{\dotprod}[2]{{#1} \cdot {#2}}
\newcommand{\bdotprod}[2]{\left({#1} \cdot {#2}\right)}
\newcommand{\crossprod}[2]{{#1} \cross {#2}}
\newcommand{\tripleprod}[3]{\dotprod{\left(\crossprod{#1}{#2}\right)}{#3}}

\DeclareMathOperator{\Proj}{Proj}
\DeclareMathOperator{\Span}{span}
\DeclareMathOperator{\Sgn}{sgn}
\DeclareMathOperator{\Area}{Area}
\DeclareMathOperator{\Volume}{Volume}

%
% A few miscellaneous things specific to this document
%
\newcommand{\crossop}[1]{\crossprod{#1}{}}

% R2 vector.
\newcommand{\VectorTwo}[2]{
\begin{bmatrix}
 {#1} \\
 {#2}
\end{bmatrix}
}

\newcommand{\VectorN}[1]{
\begin{bmatrix}
{#1}_1 \\
{#1}_2 \\
\vdots \\
{#1}_N \\
\end{bmatrix}
}

\newcommand{\DETuvij}[4]{
\begin{vmatrix}
 {#1}_{#3} & {#1}_{#4} \\
 {#2}_{#3} & {#2}_{#4}
\end{vmatrix}
}

\newcommand{\DETuvwijk}[6]{
\begin{vmatrix}
 {#1}_{#4} & {#1}_{#5} & {#1}_{#6} \\
 {#2}_{#4} & {#2}_{#5} & {#2}_{#6} \\
 {#3}_{#4} & {#3}_{#5} & {#3}_{#6}
\end{vmatrix}
}

\newcommand{\DETuvwxijkl}[8]{
\begin{vmatrix}
 {#1}_{#5} & {#1}_{#6} & {#1}_{#7} & {#1}_{#8} \\
 {#2}_{#5} & {#2}_{#6} & {#2}_{#7} & {#2}_{#8} \\
 {#3}_{#5} & {#3}_{#6} & {#3}_{#7} & {#3}_{#8} \\
 {#4}_{#5} & {#4}_{#6} & {#4}_{#7} & {#4}_{#8} \\
\end{vmatrix}
}

%\newcommand{\DETuvwxyijklm}[10]{
%\begin{vmatrix}
% {#1}_{#6} & {#1}_{#7} & {#1}_{#8} & {#1}_{#9} & {#1}_{#10} \\
% {#2}_{#6} & {#2}_{#7} & {#2}_{#8} & {#2}_{#9} & {#2}_{#10} \\
% {#3}_{#6} & {#3}_{#7} & {#3}_{#8} & {#3}_{#9} & {#3}_{#10} \\
% {#4}_{#6} & {#4}_{#7} & {#4}_{#8} & {#4}_{#9} & {#4}_{#10} \\
% {#5}_{#6} & {#5}_{#7} & {#5}_{#8} & {#5}_{#9} & {#5}_{#10}
%\end{vmatrix}
%}

% R3 vector.
\newcommand{\VectorThree}[3]{
\begin{bmatrix}
 {#1} \\
 {#2} \\
 {#3}
\end{bmatrix}
}



\author{Peeter Joot}
\email{peeter.joot@gmail.com}

%\documentclass[]{eliblogwidescreen}

\usepackage{amsmath}
\usepackage{mathpazo}

%
% shorthand for bold symbols, convenient for vectors and matrices
%
\newcommand{\Ba}[0]{\mathbf{a}}
\newcommand{\Bb}[0]{\mathbf{b}}
\newcommand{\Bc}[0]{\mathbf{c}}
\newcommand{\Bd}[0]{\mathbf{d}}
\newcommand{\Be}[0]{\mathbf{e}}
\newcommand{\Bf}[0]{\mathbf{f}}
\newcommand{\Bg}[0]{\mathbf{g}}
\newcommand{\Bh}[0]{\mathbf{h}}
\newcommand{\Bi}[0]{\mathbf{i}}
\newcommand{\Bj}[0]{\mathbf{j}}
\newcommand{\Bk}[0]{\mathbf{k}}
\newcommand{\Bl}[0]{\mathbf{l}}
\newcommand{\Bm}[0]{\mathbf{m}}
\newcommand{\Bn}[0]{\mathbf{n}}
\newcommand{\Bo}[0]{\mathbf{o}}
\newcommand{\Bp}[0]{\mathbf{p}}
\newcommand{\Bq}[0]{\mathbf{q}}
\newcommand{\Br}[0]{\mathbf{r}}
\newcommand{\Bs}[0]{\mathbf{s}}
\newcommand{\Bt}[0]{\mathbf{t}}
\newcommand{\Bu}[0]{\mathbf{u}}
\newcommand{\Bv}[0]{\mathbf{v}}
\newcommand{\Bw}[0]{\mathbf{w}}
\newcommand{\Bx}[0]{\mathbf{x}}
\newcommand{\By}[0]{\mathbf{y}}
\newcommand{\Bz}[0]{\mathbf{z}}
\newcommand{\BA}[0]{\mathbf{A}}
\newcommand{\BB}[0]{\mathbf{B}}
\newcommand{\BC}[0]{\mathbf{C}}
\newcommand{\BD}[0]{\mathbf{D}}
\newcommand{\BE}[0]{\mathbf{E}}
\newcommand{\BF}[0]{\mathbf{F}}
\newcommand{\BG}[0]{\mathbf{G}}
\newcommand{\BH}[0]{\mathbf{H}}
\newcommand{\BI}[0]{\mathbf{I}}
\newcommand{\BJ}[0]{\mathbf{J}}
\newcommand{\BK}[0]{\mathbf{K}}
\newcommand{\BL}[0]{\mathbf{L}}
\newcommand{\BM}[0]{\mathbf{M}}
\newcommand{\BN}[0]{\mathbf{N}}
\newcommand{\BO}[0]{\mathbf{O}}
\newcommand{\BP}[0]{\mathbf{P}}
\newcommand{\BQ}[0]{\mathbf{Q}}
\newcommand{\BR}[0]{\mathbf{R}}
\newcommand{\BS}[0]{\mathbf{S}}
\newcommand{\BT}[0]{\mathbf{T}}
\newcommand{\BU}[0]{\mathbf{U}}
\newcommand{\BV}[0]{\mathbf{V}}
\newcommand{\BW}[0]{\mathbf{W}}
\newcommand{\BX}[0]{\mathbf{X}}
\newcommand{\BY}[0]{\mathbf{Y}}
\newcommand{\BZ}[0]{\mathbf{Z}}

\newcommand{\Bzero}[0]{\mathbf{0}}
\newcommand{\Btheta}[0]{\boldsymbol{\theta}}
\newcommand{\Btau}[0]{\boldsymbol{\tau}}
\newcommand{\Bomega}[0]{\boldsymbol{\omega}}

%
% shorthand for unit vectors
%
\newcommand{\acap}[0]{\hat{\Ba}}
\newcommand{\bcap}[0]{\hat{\Bb}}
\newcommand{\ccap}[0]{\hat{\Bc}}
\newcommand{\dcap}[0]{\hat{\Bd}}
\newcommand{\ecap}[0]{\hat{\Be}}
\newcommand{\fcap}[0]{\hat{\Bf}}
\newcommand{\gcap}[0]{\hat{\Bg}}
\newcommand{\hcap}[0]{\hat{\Bh}}
\newcommand{\icap}[0]{\hat{\Bi}}
\newcommand{\jcap}[0]{\hat{\Bj}}
\newcommand{\kcap}[0]{\hat{\Bk}}
\newcommand{\lcap}[0]{\hat{\Bl}}
\newcommand{\mcap}[0]{\hat{\Bm}}
\newcommand{\ncap}[0]{\hat{\Bn}}
\newcommand{\ocap}[0]{\hat{\Bo}}
\newcommand{\pcap}[0]{\hat{\Bp}}
\newcommand{\qcap}[0]{\hat{\Bq}}
\newcommand{\rcap}[0]{\hat{\Br}}
\newcommand{\scap}[0]{\hat{\Bs}}
\newcommand{\tcap}[0]{\hat{\Bt}}
\newcommand{\ucap}[0]{\hat{\Bu}}
\newcommand{\vcap}[0]{\hat{\Bv}}
\newcommand{\wcap}[0]{\hat{\Bw}}
\newcommand{\xcap}[0]{\hat{\Bx}}
\newcommand{\ycap}[0]{\hat{\By}}
\newcommand{\zcap}[0]{\hat{\Bz}}
\newcommand{\thetacap}[0]{\hat{\Btheta}}

%
% to write R^n and C^n in a distinguishable fashion.  Perhaps change this
% to the double lined characters upon figuring out how to do so.
%
\newcommand{\C}[1]{$\mathbb{C}^{#1}$}
\newcommand{\R}[1]{$\mathbb{R}^{#1}$}

%
% various generally useful helpers
%

% derivative of #1 wrt. #2:
\newcommand{\D}[2] {\frac {d#2} {d#1}}

\newcommand{\inv}[1]{\frac{1}{#1}}
\newcommand{\cross}[0]{\times}

\newcommand{\abs}[1]{\lvert{#1}\rvert}
\newcommand{\norm}[1]{\lVert{#1}\rVert}
\newcommand{\innerprod}[2]{\langle{#1}, {#2}\rangle}
\newcommand{\dotprod}[2]{{#1} \cdot {#2}}
\newcommand{\bdotprod}[2]{\left({#1} \cdot {#2}\right)}
\newcommand{\crossprod}[2]{{#1} \cross {#2}}
\newcommand{\tripleprod}[3]{\dotprod{\left(\crossprod{#1}{#2}\right)}{#3}}

\DeclareMathOperator{\Proj}{Proj}
\DeclareMathOperator{\Span}{span}
\DeclareMathOperator{\Sgn}{sgn}
\DeclareMathOperator{\Area}{Area}
\DeclareMathOperator{\Volume}{Volume}

%
% A few miscellaneous things specific to this document
%
\newcommand{\crossop}[1]{\crossprod{#1}{}}

% R2 vector.
\newcommand{\VectorTwo}[2]{
\begin{bmatrix}
 {#1} \\
 {#2}
\end{bmatrix}
}

\newcommand{\VectorN}[1]{
\begin{bmatrix}
{#1}_1 \\
{#1}_2 \\
\vdots \\
{#1}_N \\
\end{bmatrix}
}

\newcommand{\DETuvij}[4]{
\begin{vmatrix}
 {#1}_{#3} & {#1}_{#4} \\
 {#2}_{#3} & {#2}_{#4}
\end{vmatrix}
}

\newcommand{\DETuvwijk}[6]{
\begin{vmatrix}
 {#1}_{#4} & {#1}_{#5} & {#1}_{#6} \\
 {#2}_{#4} & {#2}_{#5} & {#2}_{#6} \\
 {#3}_{#4} & {#3}_{#5} & {#3}_{#6}
\end{vmatrix}
}

\newcommand{\DETuvwxijkl}[8]{
\begin{vmatrix}
 {#1}_{#5} & {#1}_{#6} & {#1}_{#7} & {#1}_{#8} \\
 {#2}_{#5} & {#2}_{#6} & {#2}_{#7} & {#2}_{#8} \\
 {#3}_{#5} & {#3}_{#6} & {#3}_{#7} & {#3}_{#8} \\
 {#4}_{#5} & {#4}_{#6} & {#4}_{#7} & {#4}_{#8} \\
\end{vmatrix}
}

%\newcommand{\DETuvwxyijklm}[10]{
%\begin{vmatrix}
% {#1}_{#6} & {#1}_{#7} & {#1}_{#8} & {#1}_{#9} & {#1}_{#10} \\
% {#2}_{#6} & {#2}_{#7} & {#2}_{#8} & {#2}_{#9} & {#2}_{#10} \\
% {#3}_{#6} & {#3}_{#7} & {#3}_{#8} & {#3}_{#9} & {#3}_{#10} \\
% {#4}_{#6} & {#4}_{#7} & {#4}_{#8} & {#4}_{#9} & {#4}_{#10} \\
% {#5}_{#6} & {#5}_{#7} & {#5}_{#8} & {#5}_{#9} & {#5}_{#10}
%\end{vmatrix}
%}

% R3 vector.
\newcommand{\VectorThree}[3]{
\begin{bmatrix}
 {#1} \\
 {#2} \\
 {#3}
\end{bmatrix}
}



\author{Peeter Joot}
\email{peeter.joot@gmail.com}


\chapter{Classical Mechanics Euler Angles}
\label{chap:classicalMechanicsEulerAngles}
%\useCCL
\blogpage{http://sites.google.com/site/peeterjoot2/math2012/classicalMechanicsEulerAngles.pdf}
\date{Mar 21, 2012}
\gitRevisionInfo{classicalMechanicsEulerAngles}

\beginArtWithToc
%\beginArtNoToc

\section{Pictorially.}

FIXME:

\section{Relating the two pairs of coordinate systems.}

Step 1.  Rotation of $\phi$ around $z$

\begin{equation}\label{eqn:classicalMechanicsEulerAngles:n}
\begin{bmatrix}
x' \\
y' \\
z'
\end{bmatrix}
=
\begin{bmatrix}
\cos\phi & \sin\phi & 0 \\
-\sin\phi & \cos\phi & 0 \\
0 & 0 & 1
\end{bmatrix}
\begin{bmatrix}
x \\
y \\
z
\end{bmatrix}
\end{equation}

Step 2.  Rotation around $x'$.

\begin{equation}\label{eqn:classicalMechanicsEulerAngles:n}
\begin{bmatrix}
x'' \\
y'' \\
z''
\end{bmatrix}
=
\begin{bmatrix}
1 & 0 & 0 \\
0 & \cos\theta & \sin\theta & 0 \\
0 & -\sin\theta & \cos\theta & 0 
\end{bmatrix}
\begin{bmatrix}
x' \\
y' \\
z'
\end{bmatrix}
\end{equation}

Step 3.  Rotation around $z''$.

\begin{equation}\label{eqn:classicalMechanicsEulerAngles:n}
\begin{bmatrix}
x''' \\
y''' \\
z'''
\end{bmatrix}
=
\begin{bmatrix}
\cos\psi & \sin\psi & 0 \\
-\sin\psi & \cos\psi & 0 \\
0 & 0 & 1
\end{bmatrix}
\begin{bmatrix}
x'' \\
y'' \\
z''
\end{bmatrix}
\end{equation}

So, our full rotation is the composition of the rotation matrices

\begin{equation}\label{eqn:classicalMechanicsEulerAngles:n}
\begin{bmatrix}
x''' \\
y''' \\
z'''
\end{bmatrix}
=
\begin{bmatrix}
\cos\psi & \sin\psi & 0 \\
-\sin\psi & \cos\psi & 0 \\
0 & 0 & 1
\end{bmatrix}
\begin{bmatrix}
1 & 0 & 0 \\
0 & \cos\theta & \sin\theta & 0 \\
0 & -\sin\theta & \cos\theta & 0 
\end{bmatrix}
\begin{bmatrix}
\cos\phi & \sin\phi & 0 \\
-\sin\phi & \cos\phi & 0 \\
0 & 0 & 1
\end{bmatrix}
\begin{bmatrix}
x \\
y \\
z
\end{bmatrix}
\end{equation}

Let's introduce some notation and write this as

\begin{equation}\label{eqn:classicalMechanicsEulerAngles:n}
B_z(\alpha)
=
\begin{bmatrix}
\cos\alpha & \sin\alpha & 0 \\
-\sin\alpha & \cos\alpha & 0 \\
0 & 0 & 1
\end{bmatrix}
\end{equation}

\begin{equation}\label{eqn:classicalMechanicsEulerAngles:n}
B_x(\theta) =
\begin{bmatrix}
1 & 0 & 0 \\
0 & \cos\theta & \sin\theta & 0 \\
0 & -\sin\theta & \cos\theta & 0 
\end{bmatrix}
\end{equation}

so that we have the mapping

\begin{equation}\label{eqn:classicalMechanicsEulerAngles:n}
\Br \rightarrow B_z(\psi) B_x(\theta) B_z(\phi) \Br
\end{equation}

Now let's write

\begin{equation}\label{eqn:classicalMechanicsEulerAngles:n}
\Br = 
\begin{bmatrix}
x_1 \\
x_2 \\
x_3
\end{bmatrix}
\end{equation}

We'll call

\begin{equation}\label{eqn:classicalMechanicsEulerAngles:n}
A(\psi, \theta, \phi) = B_z(\psi) B_x(\theta) B_z(\phi) 
\end{equation}

so that

\begin{equation}\label{eqn:classicalMechanicsEulerAngles:n}
x_i' = \sum_{j = 1}^3 A_{ij} x_j.
\end{equation}

We'll drop the explicit summation sign, so that the summation over repeated indexes are implied

\begin{equation}\label{eqn:classicalMechanicsEulerAngles:n}
x_i' = A_{ij} x_j.
\end{equation}

This matrix $A(\psi, \theta, \phi)$ is in fact a general parameterization of the $3 \times 3$ special orthogonal matrices. The set of three angles $\theta$, $\phi$, $\psi$ parameterizes all rotations in 3dd space.  Transformations that preserve $\Ba \cdot \Bb$ and have unit determinant.

In symbols we must have

\begin{equation}\label{eqn:classicalMechanicsEulerAngles:n}
A^\T A = 1
\end{equation}
\begin{equation}\label{eqn:classicalMechanicsEulerAngles:n}
\det A = +1.
\end{equation}

Having solved this auxillary problem, we now want to compute the angular velocity.

We want to know how to express the coordinates of a point that is fixed in the body.  i.e. We are fixing $x_i'$ and now looking for $x_i$.

The coordinates of a point that has $x'$, $y'$ and $z'$ in a body-fixed frame, in the fixed frame are $x$, $y$, $z$.  That is given by just inverting the matrix

\begin{align*}
\begin{bmatrix}
x_1 \\
x_2 \\
x_3
\end{bmatrix}
&=
A^{-1}(\psi, \theta, \phi)
\begin{bmatrix}
x_1' \\
x_2' \\
x_3'
\end{bmatrix} \\
&=
B_z^{-1}(\phi)
B_x^{-1}(\theta)
B_z^{-1}(\psi)
\begin{bmatrix}
x_1' \\
x_2' \\
x_3'
\end{bmatrix} \\
&=
B_z^{\T}(\phi)
B_x^{\T}(\theta)
B_z^{\T}(\psi)
\begin{bmatrix}
x_1' \\
x_2' \\
x_3'
\end{bmatrix}
\end{align*}

Here we've used the fact that $B_x$ and $B_z$ are orthogonal, so that their inverses are just their transposes.

We have finally

\begin{equation}\label{eqn:classicalMechanicsEulerAngles:n}
\begin{bmatrix}
x_1 \\
x_2 \\
x_3
\end{bmatrix}
=
B_z(-\phi)
B_x(-\theta)
B_z(-\psi)
\begin{bmatrix}
x_1' \\
x_2' \\
x_3'
\end{bmatrix} 
\end{equation}

If we assume that $\psi$, $\theta$ and $\phi$ are functions of time, and compute $d \Br/dt$.  Starting with

\begin{equation}\label{eqn:classicalMechanicsEulerAngles:n}
x_i = [A^{-1}(\phi, \theta, \psi)]_{ij} x_j',
\end{equation}


\begin{equation}\label{eqn:classicalMechanicsEulerAngles:n}
\Delta x_i = 
\left(
[A^{-1}(\phi + \Delta \phi, \theta + \Delta \theta, \psi + \Delta \psi)]_{ij} 
-[A^{-1}(\phi, \theta, \psi)]_{ij} 
\right)
x_j'.
\end{equation}

For small changes, we can Taylor expand and retain only the first order terms.  Doing that and dividing by $\Delta t$ we have

\begin{equation}\label{eqn:classicalMechanicsEulerAngles:n}
\frac{d x_i}{dt} =
\left(
\PD{\psi}{} A^{-1}_{ij} \dot{\psi}
+\PD{\theta}{} A^{-1}_{ij} \dot{\theta}
+\PD{\phi}{} A^{-1}_{ij} \dot{\phi}
\right) x_j'
\end{equation}

Now, we use

\begin{equation}\label{eqn:classicalMechanicsEulerAngles:n}
x_j' = A_{jl} x_l,
\end{equation}

so that we have

\begin{equation}\label{eqn:classicalMechanicsEulerAngles:n}
\frac{d x_i}{dt} =
\left(
\left(\PD{\psi}{} A^{-1}_{ij} \right) A_{jl} \dot{\psi}
+\left(\PD{\theta}{} A^{-1}_{ij} \right) A_{jl} \dot{\theta}
+\left(\PD{\phi}{} A^{-1}_{ij} \right) A_{jl} \dot{\phi}
\right) x_l.
\end{equation}

We are looking for a relation of the form

\begin{equation}\label{eqn:classicalMechanicsEulerAngles:n}
\frac{d\Br}{dt} = \BOmega \cross \Br.
\end{equation}

We can write this as

\begin{equation}\label{eqn:classicalMechanicsEulerAngles:n}
\begin{bmatrix}
v_x \\
v_y \\
v_z
\end{bmatrix} 
=
\left(
\thetadot \PD{\theta}{A^{-1}} A
+\phidot \PD{\phi}{A^{-1}} A
+\psidot \PD{\psi}{A^{-1}} A
\right)
\begin{bmatrix}
x \\
y \\
z
\end{bmatrix} 
\end{equation}

Actually doing this calculation is asked of us in HW6.  The final answer is

\begin{equation}\label{eqn:classicalMechanicsEulerAngles:n}
\frac{d x_i}{dt} = \left(
\phidot \epsilon_{ijk} j_n^\phi
+\thetadot \epsilon_{ijk} j_n^\theta
+\psidot \epsilon_{ijk} j_n^\psi
\right) x_j.
\end{equation}

Here $\epsilon_{ijk}$ is the usual fully antisymmetric tensor with properties

FIXME:
%\begin{equation}\label{eqn:classicalMechanicsEulerAngles:n}
%\epsilon_{ijk} = 0, when any of the indexes are equal
%\epsilon_{ijk} = 1, for ...
%\end{equation}

%\EndArticle
\EndNoBibArticle
