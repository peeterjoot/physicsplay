%
% Copyright � 2015 Peeter Joot.  All Rights Reserved.
% Licenced as described in the file LICENSE under the root directory of this GIT repository.
%
\newcommand{\authorname}{Peeter Joot}
\newcommand{\email}{peeterjoot@protonmail.com}
\newcommand{\basename}{FIXMEbasenameUndefined}
\newcommand{\dirname}{notes/FIXMEdirnameUndefined/}

\renewcommand{\basename}{chapter4Notes}
\renewcommand{\dirname}{notes/ece1229/}
\newcommand{\keywords}{ECE1229H}
\newcommand{\authorname}{Peeter Joot}
\newcommand{\onlineurl}{http://sites.google.com/site/peeterjoot2/math2013/\basename.pdf}
\newcommand{\sourcepath}{\dirname\basename.tex}
\newcommand{\generatetitle}[1]{\chapter{#1}}

\newcommand{\vcsinfo}{%
\section*{}
\noindent{\color{DarkOliveGreen}{\rule{\linewidth}{0.1mm}}}
\paragraph{Document version}
%\paragraph{\color{Maroon}{Document version}}
{
\small
\begin{itemize}
\item Available online at:\\ 
\href{\onlineurl}{\onlineurl}
\item Git Repository: \input{./.revinfo/gitRepo.tex}
\item Source: \sourcepath
\item last commit: \input{./.revinfo/gitCommitString.tex}
\item commit date: \input{./.revinfo/gitCommitDate.tex}
\end{itemize}
}
}

%\PassOptionsToPackage{dvipsnames,svgnames}{xcolor}
\PassOptionsToPackage{square,numbers}{natbib}
\documentclass{scrreprt}

\usepackage[left=2cm,right=2cm]{geometry}
\usepackage[svgnames]{xcolor}
\usepackage{peeters_layout}

\usepackage{natbib}

\usepackage[
colorlinks=true,
bookmarks=false,
pdfauthor={\authorname, \email},
backref 
]{hyperref}

% http://tex.stackexchange.com/questions/75773/how-to-reference-problems-by-the-text-label-in-an-exercise-envioronment
\usepackage[english]{cleveref}
\crefname{Exercise}{exercise}{exercises}
\Crefname{Exercise}{Exercise}{Exercises}

\RequirePackage{titlesec}
\RequirePackage{ifthen}

% http://stackoverflow.com/questions/4932910/date-in-the-tabular-environment
\makeatletter
\let\insertdate\@date
\makeatother

\titleformat{\chapter}[display]
{\bfseries\Large}
{\color{DarkSlateGrey}\filleft \authorname
\ifthenelse{\isundefined{\studentnumber}}{}{\\ \studentnumber}
\ifthenelse{\isundefined{\email}}{}{\\ \email}
\ifthenelse{\isundefined{\dateintitle}}{}{\\ \insertdate}
%\ifthenelse{\isundefined{\coursename}}{}{\\ \coursename} % put in title instead.
}
{4ex}
{\color{DarkOliveGreen}{\titlerule}\color{Maroon}
\vspace{2ex}%
\filright}
[\vspace{2ex}%
\color{DarkOliveGreen}\titlerule
]

\newcommand{\beginArtWithToc}[0]{\begin{document}\tableofcontents}
\newcommand{\beginArtNoToc}[0]{\begin{document}}
\newcommand{\EndNoBibArticle}[0]{\end{document}}
\newcommand{\EndArticle}[0]{\bibliography{Bibliography}\bibliographystyle{plainnat}\end{document}}

% 
%\newcommand{\citep}[1]{\cite{#1}}

\colorSectionsForArticle



\usepackage{ece1229}

\beginArtNoToc
%\generatetitle{ECE1229H Advanced Antenna Theory.  Lecture 4: XXX.  Taught by Prof.\ G.V. Eleftheriades}
%\generatetitle{XXX}
%\chapter{XXX}
\label{chap:chapter4Notes}

%\section{Disclaimer}
%
%Peeter's lecture notes from class.  These may be incoherent and rough.

These are notes for the UofT course ECE1229, Advanced Antenna Theory, taught by Prof. Eleftheriades, covering \chaptext 4 \citep{balanis2005antenna} content.

Unlike most of the other classes I have taken, I am not attempting to take comprehensive notes for this class.  The class is taught on slides that match the textbook so closely, there is little value to me taking notes that just replicate the text.  Instead, I am annotating my copy of textbook with little details instead.  My usual notes collection for the class will contain musings of details that were unclear, or in some cases, details that were provided in class, but are not in the text (and too long to pencil into my book.)

\section{Magnetic Vector Potential}

In class and in the problem set \( \BA \) was referred to as the \underlineAndIndex{Magnetic Vector Potential}.
I only recalled this referred to as the \underlineAndIndex{Vector Potential}.
Prefixing this with magnetic seemed counter intuitive to me since it is generated by electric sources (charges and currents).
This terminology can be justified due to the fact that \( \BA \) generates the magnetic field by its curl.
Some mention of this can be found in \citep{wiki:magneticPotential}, which also points out that the \underlineAndIndex{Electric Potential} refers to the scalar \( \phi \).
Prof. Eleftheriades points out that \underlineAndIndex{Electric Vector Potential} refers to the vector potential \( \BF \) generated by magnetic sources (because in that case the electric field is generated by the curl of \( \BF \).)

\EndArticle
