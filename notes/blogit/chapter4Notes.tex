%
% Copyright � 2015 Peeter Joot.  All Rights Reserved.
% Licenced as described in the file LICENSE under the root directory of this GIT repository.
%
\newcommand{\authorname}{Peeter Joot}
\newcommand{\email}{peeterjoot@protonmail.com}
\newcommand{\basename}{FIXMEbasenameUndefined}
\newcommand{\dirname}{notes/FIXMEdirnameUndefined/}

\renewcommand{\basename}{chapter4Notes}
\renewcommand{\dirname}{notes/ece1229/}
\newcommand{\keywords}{ECE1229H}
\newcommand{\authorname}{Peeter Joot}
\newcommand{\onlineurl}{http://sites.google.com/site/peeterjoot2/math2013/\basename.pdf}
\newcommand{\sourcepath}{\dirname\basename.tex}
\newcommand{\generatetitle}[1]{\chapter{#1}}

\newcommand{\vcsinfo}{%
\section*{}
\noindent{\color{DarkOliveGreen}{\rule{\linewidth}{0.1mm}}}
\paragraph{Document version}
%\paragraph{\color{Maroon}{Document version}}
{
\small
\begin{itemize}
\item Available online at:\\ 
\href{\onlineurl}{\onlineurl}
\item Git Repository: \input{./.revinfo/gitRepo.tex}
\item Source: \sourcepath
\item last commit: \input{./.revinfo/gitCommitString.tex}
\item commit date: \input{./.revinfo/gitCommitDate.tex}
\end{itemize}
}
}

%\PassOptionsToPackage{dvipsnames,svgnames}{xcolor}
\PassOptionsToPackage{square,numbers}{natbib}
\documentclass{scrreprt}

\usepackage[left=2cm,right=2cm]{geometry}
\usepackage[svgnames]{xcolor}
\usepackage{peeters_layout}

\usepackage{natbib}

\usepackage[
colorlinks=true,
bookmarks=false,
pdfauthor={\authorname, \email},
backref 
]{hyperref}

% http://tex.stackexchange.com/questions/75773/how-to-reference-problems-by-the-text-label-in-an-exercise-envioronment
\usepackage[english]{cleveref}
\crefname{Exercise}{exercise}{exercises}
\Crefname{Exercise}{Exercise}{Exercises}

\RequirePackage{titlesec}
\RequirePackage{ifthen}

% http://stackoverflow.com/questions/4932910/date-in-the-tabular-environment
\makeatletter
\let\insertdate\@date
\makeatother

\titleformat{\chapter}[display]
{\bfseries\Large}
{\color{DarkSlateGrey}\filleft \authorname
\ifthenelse{\isundefined{\studentnumber}}{}{\\ \studentnumber}
\ifthenelse{\isundefined{\email}}{}{\\ \email}
\ifthenelse{\isundefined{\dateintitle}}{}{\\ \insertdate}
%\ifthenelse{\isundefined{\coursename}}{}{\\ \coursename} % put in title instead.
}
{4ex}
{\color{DarkOliveGreen}{\titlerule}\color{Maroon}
\vspace{2ex}%
\filright}
[\vspace{2ex}%
\color{DarkOliveGreen}\titlerule
]

\newcommand{\beginArtWithToc}[0]{\begin{document}\tableofcontents}
\newcommand{\beginArtNoToc}[0]{\begin{document}}
\newcommand{\EndNoBibArticle}[0]{\end{document}}
\newcommand{\EndArticle}[0]{\bibliography{Bibliography}\bibliographystyle{plainnat}\end{document}}

% 
%\newcommand{\citep}[1]{\cite{#1}}

\colorSectionsForArticle



\usepackage{ece1229}

\beginArtNoToc
%\generatetitle{ECE1229H Advanced Antenna Theory.  Lecture 4: XXX.  Taught by Prof.\ G.V. Eleftheriades}
%\generatetitle{XXX}
%\chapter{XXX}
\label{chap:chapter4Notes}

%\section{Disclaimer}
%
%Peeter's lecture notes from class.  These may be incoherent and rough.

These are notes for the UofT course ECE1229, Advanced Antenna Theory, taught by Prof. Eleftheriades, covering \chaptext 4 \citep{balanis2005antenna} content.

Unlike most of the other classes I have taken, I am not attempting to take comprehensive notes for this class.  The class is taught on slides that match the textbook so closely, there is little value to me taking notes that just replicate the text.  Instead, I am annotating my copy of textbook with little details instead.  My usual notes collection for the class will contain musings of details that were unclear, or in some cases, details that were provided in class, but are not in the text (and too long to pencil into my book.)

\section{Magnetic Vector Potential.}

In class and in the problem set \( \BA \) was referred to as the \underlineAndIndex{Magnetic Vector Potential}.
I only recalled this referred to as the \underlineAndIndex{Vector Potential}.
Prefixing this with magnetic seemed counter intuitive to me since it is generated by electric sources (charges and currents).
This terminology can be justified due to the fact that \( \BA \) generates the magnetic field by its curl.
Some mention of this can be found in \citep{wiki:magneticPotential}, which also points out that the \underlineAndIndex{Electric Potential} refers to the scalar \( \phi \).
Prof. Eleftheriades points out that \underlineAndIndex{Electric Vector Potential} refers to the vector potential \( \BF \) generated by magnetic sources (because in that case the electric field is generated by the curl of \( \BF \).)

\section{Far field for a spherical potential.}

It is interesting to look at the far electric field associated with an arbitrary spherical magnetic vector potential, assuming all of the radial dependence is in the spherical envelope.  That is

\begin{dmath}\label{eqn:chapter4Notes:20}
\BA = \frac{e^{-j k r}}{r} \lr{ 
 \rcap a_r\lr{ \theta, \phi }
+\thetacap a_\theta\lr{ \theta, \phi }
+\phicap a_\phi\lr{ \theta, \phi }
}.
\end{dmath}

The electric field is

\begin{dmath}\label{eqn:chapter4Notes:40}
\BE = - j \omega \BA - j \frac{1}{\omega \mu_0 \epsilon_0 } \spacegrad \lr{\spacegrad \cdot \BA }.
\end{dmath}

\index{divergence!spherical coordinates}
\index{gradient!spherical coordinates}
The divergence and gradient in spherical coordinates are

\begin{subequations}
\label{eqn:chapter4Notes:60}
\begin{dmath}\label{eqn:chapter4Notes:80}
\spacegrad \cdot \BA 
=
\inv{r^2} \PD{r}{} \lr{ r^2 A_r } 
+ \inv{r \sin\theta } \PD{\theta}{} \lr{A_\theta \sin\theta} 
+ \inv{r \sin\theta } \PD{\phi}{A_\phi} 
\end{dmath}
\begin{dmath}\label{eqn:chapter4Notes:100}
\spacegrad \psi
= 
\rcap \PD{r}{\psi}
+\frac{\thetacap}{r} \PD{\theta}{\psi}
+ \frac{\phicap}{r \sin\theta} \PD{\phi}{\psi}.
\end{dmath}
\end{subequations}

For the assumed potential, the divergence is

\begin{dmath}\label{eqn:chapter4Notes:120}
\spacegrad \cdot \BA 
=
\frac{a_r}{r^2} \PD{r}{} \lr{ r^2 \frac{e^{-j k r}}{r} } 
+ \inv{r \sin\theta } \frac{e^{-j k r}}{r} \PD{\theta}{} \lr{\sin\theta a_\theta} 
+ \inv{r \sin\theta } \frac{e^{-j k r}}{r} \PD{\phi}{a_\phi} 
=
a_r
e^{-j k r} 
\lr{ 
\inv{r^2}
-j k \inv{r}
} 
+ \inv{r^2 \sin\theta } e^{-j k r} \PD{\theta}{} \lr{\sin\theta a_\theta} 
+ \inv{r^2 \sin\theta } e^{-j k r} \PD{\phi}{a_\phi} 
\approx
-j k \frac{a_r}{r}
e^{-j k r}.
\end{dmath}

The last approximation dropped all the \( 1/r^2 \) terms that will be small compared to \( 1/r \) contribution that dominates when \( r \rightarrow \infty \), the far field.

The gradient can now be computed

\begin{dmath}\label{eqn:chapter4Notes:140}
\spacegrad \lr{\spacegrad \cdot \BA } \approx
-j k 
\spacegrad 
\lr{
\frac{a_r}{r}
e^{-j k r}
}
=
-j k \lr{
\rcap \PD{r}{}
+\frac{\thetacap}{r} \PD{\theta}{}
+ \frac{\phicap}{r \sin\theta} \PD{\phi}{}
}
\frac{a_r}{r}
e^{-j k r}
=
-j k \lr{
\rcap a_r \PD{r}{} \lr{
\frac{1}{r}
e^{-j k r}
}
+\frac{\thetacap}{r^2} 
e^{-j k r}
\PD{\theta}{a_r}
+
e^{-j k r}
\frac{\phicap}{r^2 \sin\theta} 
\PD{\phi}{a_r}
}
=
-j k \lr{
-\rcap \frac{a_r}{r^2} \lr{
1
+ j k r 
}
+\frac{\thetacap}{r^2} 
\PD{\theta}{a_r}
+
\frac{\phicap}{r^2 \sin\theta} 
\PD{\phi}{a_r}
}
e^{-j k r}
\approx
- k^2 \rcap \frac{a_r}{r}
e^{-j k r}.
\end{dmath}

Again, a far field approximation has been used to kill all the \( 1/r^2 \) terms.

The far field approximation of the electric field is now possible

\begin{dmath}\label{eqn:chapter4Notes:160}
\begin{aligned}
\BE 
&= - j \omega \BA - j \frac{1}{\omega \mu_0 \epsilon_0 } \spacegrad \lr{\spacegrad \cdot \BA } \\
&= 
- j \omega 
\frac{e^{-j k r}}{r} \lr{ 
 \rcap a_r\lr{ \theta, \phi }
+\thetacap a_\theta\lr{ \theta, \phi }
+\phicap a_\phi\lr{ \theta, \phi }
}
 + j \frac{1}{\omega \mu_0 \epsilon_0 } 
 k^2 \rcap \frac{a_r}{r}
e^{-j k r} \\
&=
- j \omega 
\frac{e^{-j k r}}{r} \lr{ 
 \cancel{\rcap a_r\lr{ \theta, \phi }}
+\thetacap a_\theta\lr{ \theta, \phi }
+\phicap a_\phi\lr{ \theta, \phi }
}
 + \cancel{j \frac{c^2}{\omega } 
 \lr{\frac{\omega}{c}}^2 \rcap \frac{a_r}{r}
e^{-j k r}
} \\
&=
- j \omega 
\frac{e^{-j k r}}{r} \lr{ 
\thetacap a_\theta\lr{ \theta, \phi }
+\phicap a_\phi\lr{ \theta, \phi }
}.
\end{aligned}
\end{dmath}

Observe the perfect, somewhat miraculous seeming, cancelation of all the radial components of the field.  If \( \BA_\txtT \) is the non-radial projection of \( \BA \), the electric far field is just

\boxedEquation{eqn:chapter4Notes:n}{
%\begin{equation}\label{eqn:chapter4Notes:n}
\BE_{\textrm{ff}} = -j \omega \BA_\txtT.
%\end{equation}
}

\EndArticle
