\section{Problem 1.  Energy-momentum tensor and electromagnetic forces.}
\subsection{Statement}

In class, we argued that in the absence of charges and currents, the energy-momentum tensor (or the ``stress-energy'' tensor) of the electromagnetic field

\begin{equation}\label{eqn:relativisticElectrodynamicsPS6P1:10}
T^{k m} = -\inv{4\pi} F^{k j} {F^{m}}_j + \inv{16 \pi} g^{k m} F^{i j} F_{i j},
\end{equation}

is conserved:

\begin{equation}\label{eqn:relativisticElectrodynamicsPS6P1:20}
\partial_k T^{k m} = 0.
\end{equation}

In this problem, you will study the fate of \ref{eqn:relativisticElectrodynamicsPS6P1:10}, the law of energy and momentum conservation in the presence of charged particles and currents given by a 4-vector current $j^l$.

\subsection{1. Conservation relation in the presence of sources.}

\subsubsection{Statement.}

Use the equations of motion in the presence of sources, $\partial_l F^{l k} = \frac{4 \pi}{c} j^m$, the fact that $F^{l k} = \partial^l A^m - \partial^m A^l$, and appropriate index gymnastics to show that \ref{eqn:relativisticElectrodynamicsPS6P1:20} is now replaced by 

\begin{equation}\label{eqn:relativisticElectrodynamicsPS6P1:30}
\partial_k T^{k m} = -\inv{c} F^{m l} j_l.
\end{equation}

\subsubsection{Solution.}

\subsection{2. Timelike component of the conservation relation.}

\subsubsection{Statement.}

Consider the $m = 0$ components of \ref{eqn:relativisticElectrodynamicsPS6P1:30}.  Show that it implies the energy conservation equation already discussed in class (see notes pp. 125-127):

\begin{equation}\label{eqn:relativisticElectrodynamicsPS6P1:40}
\PD{t}{\mathcal{E}} + \spacegrad \cdot \BS = - \BE \cdot \Bj.
\end{equation}

Recall the physical interpretation of the various terms in this equation.

\subsubsection{Solution.}

\subsection{3. Spacelike component of the conservation relation.}

\subsubsection{Statement.}

Consider the $m = \alpha$ components of \ref{eqn:relativisticElectrodynamicsPS6P1:30}.  Show that it implies that:

\begin{equation}\label{eqn:relativisticElectrodynamicsPS6P1:40}
\PD{t}{}\left( \frac{S^\alpha}{c^2} \right) + \PD{x^\beta}{} T^{\beta \alpha}
= - \left( \rho E^\alpha + \inv{c} \left( \Bj \cross \BB \right)^\alpha \right) \equiv - f^\alpha
\end{equation}

Give a physical interpretation of $f^\alpha$.

\subsubsection{Solution.}

\subsection{4. Integrated over a volume.}

\subsubsection{Statement.}

Integrate \ref{eqn:relativisticElectrodynamicsPS6P1:40} over a closed volume $V$ and use integration by parts to obtain

\begin{equation}\label{eqn:relativisticElectrodynamicsPS6P1:50}
\PD{t}{} \int_V d^3 \Bx \frac{S^\alpha}{c^2} 
= 
- \int_{\partial V = S} d \sigma^\beta T^{\beta \alpha} - \int_V d^3 \Bx f^\alpha 
\end{equation}

Give a physical interpretation of \ref{eqn:relativisticElectrodynamicsPS6P1:50} as expressing momentum conservation.  In particular, explain how, if the volume $V$ is that of a body (made of charged particles -- bound or otherwise), this implies that:

\begin{equation}\label{eqn:relativisticElectrodynamicsPS6P1:60}
\begin{aligned}
&\ddt{} \left( 
\Bp_{\text{EM field in $V$}} + 
\Bp_{\text{charged particles in $V$}} + 
\right)^\alpha \\
&\qquad = \int_{\text{surface of body}} \left( 
(\text{surface force})^\alpha \text{on body due to shears and pressures}
\right)
\end{aligned}
\end{equation}

(Note that here $d \sigma^\beta$ is an outward normal vector to the surface of the body, so the surface has a relative minus signs w.r.t the one from class, where an inward normal was used.)

\subsubsection{Solution.}

\subsection{5. Pressure and shear of linearly polarized EM wave.}

\subsubsection{Statement.}

Imagine that a place linearly polarized electromagnetic wave is falling on a flat surface at an angle of incidence $\alpha$, and is completely absorbeed by the body.  Find the pressure and shear on a unit area of the surface using the Maxwell stress tensor.

\subsubsection{Solution.}

