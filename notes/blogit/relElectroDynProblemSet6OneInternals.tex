\section{Problem 1.  Energy-momentum tensor and electromagnetic forces.}
\subsection{Statement}

In class, we argued that in the absence of charges and currents, the energy-momentum tensor (or the ``stress-energy'' tensor) of the electromagnetic field

\begin{equation}\label{eqn:relativisticElectrodynamicsPS6P1:10}
T^{k m} = -\inv{4\pi} F^{k j} {F^{m}}_j + \inv{16 \pi} g^{k m} F^{i j} F_{i j},
\end{equation}

is conserved:

\begin{equation}\label{eqn:relativisticElectrodynamicsPS6P1:20}
\partial_k T^{k m} = 0.
\end{equation}

In this problem, you will study the fate of \ref{eqn:relativisticElectrodynamicsPS6P1:10}, the law of energy and momentum conservation in the presence of charged particles and currents given by a 4-vector current $j^l$.

\subsection{1. Conservation relation in the presence of sources.}

\subsubsection{Statement.}

Use the equations of motion in the presence of sources, $\partial_l F^{l k} = \frac{4 \pi}{c} j^m$, the fact that $F^{l k} = \partial^l A^m - \partial^m A^l$, and appropriate index gymnastics to show that \ref{eqn:relativisticElectrodynamicsPS6P1:20} is now replaced by 

\begin{equation}\label{eqn:relativisticElectrodynamicsPS6P1:30}
\partial_k T^{k m} = -\inv{c} F^{m l} j_l.
\end{equation}

\subsubsection{Solution.}

We can proceed directly, contracting the four gradient with the tensor

\begin{align*}
\partial_k T^{k m} 
&=
\inv{4 \pi} \left( 
-\partial_k( F^{k j} {F^{m}}_j ) 
+ \inv{4} g^{k m} \partial_k 
(F^{i j} F_{i j}) 
\right) \\
&=
\inv{4 \pi} \left( 
-\partial^k( F_{k j} F^{m j} ) + \inv{2} F_{i j} \partial^m F^{i j} 
\right) \\
&=
\inv{4 \pi} \left( 
-F^{m j} \underbrace{\partial^k F_{k j} }_{= 4 \pi j_j/c}
-\underbrace{F_{k j} \partial^k F^{m j} }_{\text{rename $k \rightarrow i$}}
+ \inv{2} F_{i j} \partial^m F^{i j} 
\right) \\
&=
- \inv{c} F^{m a} j_a
+\frac{F_{i j}}{4 \pi} \left( 
-\partial^i F^{m j} + \inv{2} \partial^m F^{i j} 
\right) \\
\end{align*}

To complete the task we have to show that this second term is zero.  First lets get rid of the $1/2$ by writing $1 = 1/2 + 1/2$ using the index swapping trick

\begin{align*}
F_{i j} \partial^i F^{m j} 
&= 
\inv{2} F_{i j} \partial^i F^{m j} + \inv{2} F_{j i} \partial^j F^{m i} \\
&= 
\inv{2} F_{i j} \left( \partial^i F^{m j} - \partial^j F^{m i} \right)
\end{align*}

This gives us for the second term

\begin{align*}
\frac{F_{i j}}{4 \pi} \left( -\partial^i F^{m j} + \inv{2} \partial^m F^{i j} \right) 
&=
\frac{F_{i j}}{8 \pi} \left( \partial^i F^{j m} + \partial^j F^{m i} + \partial^m F^{i j} \right) \\
&=
\frac{F_{i j}}{8 \pi} \left( 
\partial^i \partial^j A^m
-\partial^i \partial^m A^j
+\partial^j \partial^m A^i
-\partial^j \partial^i A^m
+\partial^m \partial^i A^j
-\partial^m \partial^j A^i
\right) \\
\end{align*}

By commuting derivatives, assuming the typical sufficient continuity of the fields, we see that each of the six terms in braces cancel, which completes the exersize.

\subsection{2. Timelike component of the conservation relation.}

\subsubsection{Statement.}

Consider the $m = 0$ components of \ref{eqn:relativisticElectrodynamicsPS6P1:30}.  Show that it implies the energy conservation equation already discussed in class (see notes pp. 125-127):

\begin{equation}\label{eqn:relativisticElectrodynamicsPS6P1:40}
\PD{t}{\mathcal{E}} + \spacegrad \cdot \BS = - \BE \cdot \Bj.
\end{equation}

Recall the physical interpretation of the various terms in this equation.

\subsubsection{Solution.}

We want to express

\begin{equation}\label{eqn:relativisticElectrodynamicsPS6P1:45}
\partial_k T^{k 0} = -\inv{c} F^{0 a} j_a,
\end{equation}

in terms of $\BE$, $\BB$, and $\Bj$.  On the RHS, because $F^{0 0} = 0$ we have

\begin{equation}\label{eqn:relativisticElectrodynamicsPS6P1:80}
F^{0 a} j_a = F^{0 \alpha} j_\alpha = -F^{0 \alpha} (\Bj)^\alpha.
\end{equation}

The tensor factor is

\begin{align*}
F^{0 \alpha} 
&= \partial^0 A^\alpha - \partial^\alpha A^0 \\
&= \inv{c} \partial_t A^\alpha + \partial_\alpha A^0 \\
&= -(\BE)^\alpha,
\end{align*}

so we have for the RHS of \ref{eqn:relativisticElectrodynamicsPS6P1:45}

\begin{equation}\label{eqn:relativisticElectrodynamicsPS6P1:100}
-\inv{c} F^{0 a} j_a = -\inv{c} \BE \cdot \Bj.
\end{equation}

Now let's expand the LHS, noting that we have

\begin{align}\label{eqn:relativisticElectrodynamicsPS6P1:180}
T^{ 0 0 } &= \inv{8 \pi} ( \BE^2 + \BB^2 ) = \mathcal{E} \\
T^{\alpha 0} &= \inv{4 \pi} (\BE \cross \BB)^\alpha = \frac{\BS^\alpha}{c}.
\end{align}

With $\partial_0 = \partial_t/c$ we have

\begin{equation}\label{eqn:relativisticElectrodynamicsPS6P1:160}
\partial_k T^{k 0} = \inv{c} \PD{t}{} \mathcal{E} + \PD{x^\alpha}{} \frac{\BS^\alpha}{c} = -\inv{c} \BE \cdot \Bj.
\end{equation}

Multiplying through by $c$ recovers \ref{eqn:relativisticElectrodynamicsPS6P1:40} as desired.

\subsection{3. Spacelike component of the conservation relation.}

\subsubsection{Statement.}

Consider the $m = \alpha$ components of \ref{eqn:relativisticElectrodynamicsPS6P1:30}.  Show that it implies that:

\begin{equation}\label{eqn:relativisticElectrodynamicsPS6P1:41}
\PD{t}{}\left( \frac{S^\alpha}{c^2} \right) + \PD{x^\beta}{} T^{\beta \alpha}
= - \left( \rho E^\alpha + \inv{c} \left( \Bj \cross \BB \right)^\alpha \right) \equiv - f^\alpha
\end{equation}

Give a physical interpretation of $f^\alpha$.

\subsubsection{Solution.}

We want to expand

\begin{equation}\label{eqn:relativisticElectrodynamicsPS6P1:42}
\partial_k T^{k \alpha} = -\inv{c} F^{\alpha l} j_l.
\end{equation}

On the RHS we have

\begin{align*}
-\inv{c} F^{\alpha l} j_l
&=
-\inv{c} 
\left( 
F^{\alpha 0} j_0 
+F^{\alpha \beta} j_\beta 
\right) \\
&= 
- \BE^\alpha \rho - \inv{c} (- \epsilon^{\sigma \alpha \beta} \BB^\sigma) (-\Bj^\beta) \\
&= 
- \BE^\alpha \rho - \inv{c} \epsilon^{\alpha \beta \sigma} \BB^\sigma \Bj^\beta \\
&= 
- (\rho \BE + \frac{\Bj}{c} \cross \BB)^\alpha
\end{align*}

For the LHS of \ref{eqn:relativisticElectrodynamicsPS6P1:42}, using

\begin{equation}\label{eqn:relativisticElectrodynamicsPS6P1:200}
T^{0 \alpha} = \frac{\BS^\alpha}{c}.
\end{equation}


\subsection{4. Integrated over a volume.}

\subsubsection{Statement.}

Integrate \ref{eqn:relativisticElectrodynamicsPS6P1:41} over a closed volume $V$ and use integration by parts to obtain

\begin{equation}\label{eqn:relativisticElectrodynamicsPS6P1:50}
\PD{t}{} \int_V d^3 \Bx \frac{S^\alpha}{c^2} 
= 
- \int_{\partial V = S} d \sigma^\beta T^{\beta \alpha} - \int_V d^3 \Bx f^\alpha 
\end{equation}

Give a physical interpretation of \ref{eqn:relativisticElectrodynamicsPS6P1:50} as expressing momentum conservation.  In particular, explain how, if the volume $V$ is that of a body (made of charged particles -- bound or otherwise), this implies that:

\begin{equation}\label{eqn:relativisticElectrodynamicsPS6P1:60}
\begin{aligned}
&\ddt{} \left( 
\Bp_{\text{EM field in $V$}} + 
\Bp_{\text{charged particles in $V$}} + 
\right)^\alpha \\
&\qquad = \int_{\text{surface of body}} \left( 
(\text{surface force})^\alpha \text{on body due to shears and pressures}
\right)
\end{aligned}
\end{equation}

(Note that here $d \sigma^\beta$ is an outward normal vector to the surface of the body, so the surface has a relative minus signs w.r.t the one from class, where an inward normal was used.)

\subsubsection{Solution.}

\subsection{5. Pressure and shear of linearly polarized EM wave.}

\subsubsection{Statement.}

Imagine that a place linearly polarized electromagnetic wave is falling on a flat surface at an angle of incidence $\alpha$, and is completely absorbed by the body.  Find the pressure and shear on a unit area of the surface using the Maxwell stress tensor.

\subsubsection{Solution.}

\section{Appendix.  Move to lecture notes for missed lecture.}

\subsection{Energy term of the stress energy tensor.}

\begin{align*}
T^{ 0 0 } 
&=
-\inv{4 \pi} F^{ 0 j} {F^0}_j + \inv{16 \pi} F^{i j} F_{i j} \\
&=
-\inv{4 \pi} F^{ 0 \alpha} {F^0}_\alpha + \inv{16 \pi} \left(
F^{0 j} F_{0 j} 
+F^{\alpha j} F_{\alpha j} 
\right)
\\
&=
\inv{4 \pi} F^{ 0 \alpha} F^{0 \alpha} + \inv{16 \pi} 
\left(
F^{0 \alpha} F_{0 \alpha} 
+F^{\alpha 0} F_{\alpha 0} 
+F^{\alpha \beta} F_{\alpha \beta} 
\right)
\\
&=
\inv{4 \pi} \BE^2 + \inv{16 \pi} \left(
-2 \BE^2 +F^{\alpha \beta} F^{\alpha \beta} 
\right)
\end{align*}

We have 
\begin{align*}
F^{\alpha \beta} 
&= \partial^\alpha A^\beta - \partial^\beta A^\alpha \\
&= -\partial_\alpha A^\beta + \partial_\beta A^\alpha \\
&= -\epsilon^{\sigma \alpha \beta} (\BB)^\sigma
\end{align*}

So

\begin{align*}
F^{\alpha \beta} F^{\alpha \beta} 
&= 
\epsilon^{\sigma \alpha \beta} (\BB)^\sigma
\epsilon^{\mu \alpha \beta} (\BB)^\mu \\
&= (2!) \delta^{\sigma \mu} 
(\BB)^\sigma
(\BB)^\mu \\
&= 2 \BB^2
\end{align*}

A final bit of assembly gives us $T^{0 0}$

\begin{equation}\label{eqn:relativisticElectrodynamicsPS6P1:120}
\boxed{
T^{ 0 0 } = \inv{8 \pi} ( \BE^2 + \BB^2 ) = \mathcal{E}
}
\end{equation}

\subsection{Momentum terms of the stress energy tensor.}

For the spatial $T^{k 0}$ components we have

\begin{align*}
T^{\alpha 0} 
&= 
-\inv{4 \pi} F^{\alpha j} {F^0}_j + \inv{16 \pi} g^{\alpha 0} F^{i j} F_{i j} \\
&= 
-\inv{4 \pi} F^{\alpha j} {F^0}_j \\
&= 
-\inv{4 \pi} 
\left( 
F^{\alpha 0} {F^0}_0 
+F^{\alpha \beta} {F^0}_\beta 
\right) \\
&= 
\inv{4 \pi} F^{\alpha \beta} F^{0 \beta} \\
&= 
\inv{4 \pi} (-\epsilon^{\sigma \alpha \beta} (\BB)^\sigma) (-(\BE)^\beta) \\
&= 
\inv{4 \pi} \epsilon^{\alpha \beta \sigma} 
(\BE)^\beta 
(\BB)^\sigma
\\
\end{align*}

So we have

\begin{equation}\label{eqn:relativisticElectrodynamicsPS6P1:140}
\boxed{
T^{\alpha 0} = \inv{4 \pi} (\BE \cross \BB)^\alpha = \frac{\BS^\alpha}{c}.
}
\end{equation}

\subsection{Symmetry}

It is simple to show that $T^{k m}$ is symmetric

\begin{align*}
T^{m k} 
&= -\inv{4\pi} F^{m j} {F^{k}}_j + \inv{16 \pi} g^{m k} F^{i j} F_{i j} \\
&= -\inv{4\pi} {F^{m}}_j F^{k j} + \inv{16 \pi} g^{k m} F^{i j} F_{i j} \\
&= T^{k m}
\end{align*}

\subsection{Pressure and shear terms.}

Let's now expand $T^{\beta \alpha}$, starting with the diagonal terms $T^{\alpha\alpha}$.  Because this repeated index isn't summed over, things get slightly irregular, so it's easier to drop the abstraction and just pick a specific $\alpha$, say, $\alpha = 1$.  Then we have

\begin{align*}
T^{1 1} 
&= \inv{4 \pi} \left( - F^{1 k} {F^1}_k - \inv{2} (\BB^2 - \BE^2) \right) \\
&= \inv{4 \pi} \left( 
- F^{1 0} F^{1 0}
+ F^{1 \alpha} F^{1 \alpha}
 - \inv{2} (\BB^2 - \BE^2) \right) \\
&= \inv{4 \pi} \left( 
- E_x^2
+ F^{1 2} F^{1 2}
+ F^{1 3} F^{1 3}
 - \inv{2} (\BB^2 - \BE^2) \right) \\
\end{align*}

For the magnetic components above we have for example

\begin{align*}
F^{1 2} F^{1 2} 
&=
(\partial^1 A^2 - \partial^2 A^2) (\partial^1 A^2 - \partial^2 A^2) \\
&=
(\partial_1 A^2 - \partial_2 A^2) (\partial_1 A^2 - \partial_2 A^2) \\
&=
B_z^2
\end{align*}

So we have

\begin{equation}\label{eqn:relativisticElectrodynamicsPS6P1:n}
T^{1 1} 
= \inv{4 \pi} \left( 
- E_x^2 + B_y^2 + B_z^2
- \inv{2} (\BB^2 - \BE^2) \right)
\end{equation}

Or

\begin{equation}\label{eqn:relativisticElectrodynamicsPS6P1:n}
T^{1 1} 
= \inv{8 \pi} \left( 
- E_x^2 + E_y^2 + E_z^2
- B_x^2 + B_y^2 + B_z^2
\right) 
\end{equation}
