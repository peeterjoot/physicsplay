\section{Problem 1.  Energy-momentum tensor and electromagnetic forces.}
\subsection{Statement}

In class, it was argued that in the absence of charges and currents, the energy-momentum tensor (or the ``stress-energy'' tensor) of the electromagnetic field

\begin{equation}\label{eqn:relativisticElectrodynamicsPS6P1:10}
T^{k m} = -\inv{4\pi} F^{k j} {F^{m}}_j + \inv{16 \pi} g^{k m} F^{i j} F_{i j},
\end{equation}

is conserved:

\begin{equation}\label{eqn:relativisticElectrodynamicsPS6P1:20}
\partial_k T^{k m} = 0.
\end{equation}

In this problem, you will study the fate of \ref{eqn:relativisticElectrodynamicsPS6P1:10}, the law of energy and momentum conservation in the presence of charged particles and currents given by a 4-vector current $j^l$.

\subsection{1. Conservation relation in the presence of sources.}

\subsubsection{Statement.}

Use the equations of motion in the presence of sources, $\partial_l F^{l k} = \frac{4 \pi}{c} j^m$, the fact that $F^{l k} = \partial^l A^m - \partial^m A^l$, and appropriate index gymnastics to show that \ref{eqn:relativisticElectrodynamicsPS6P1:20} is now replaced by 

\begin{equation}\label{eqn:relativisticElectrodynamicsPS6P1:30}
\partial_k T^{k m} = -\inv{c} F^{m l} j_l.
\end{equation}

\subsubsection{Solution.}

Diving straight in, a contraction of the coordinates of the four gradient with the stress energy tensor appears to produce most of the desired result

\begin{align*}
\partial_k T^{k m} 
&=
\inv{4 \pi} \left( 
-\partial_k( F^{k j} {F^{m}}_j ) 
+ \inv{4} g^{k m} \partial_k 
(F^{i j} F_{i j}) 
\right) \\
&=
\inv{4 \pi} \left( 
-\partial^k( F_{k j} F^{m j} ) + \inv{2} F_{i j} \partial^m F^{i j} 
\right) \\
&=
\inv{4 \pi} \left( 
-F^{m j} \underbrace{\partial^k F_{k j} }_{= 4 \pi j_j/c}
-\underbrace{F_{k j} \partial^k F^{m j} }_{\text{rename $k \rightarrow i$}}
+ \inv{2} F_{i j} \partial^m F^{i j} 
\right) \\
&=
- \inv{c} F^{m a} j_a
+\frac{F_{i j}}{4 \pi} \left( 
-\partial^i F^{m j} + \inv{2} \partial^m F^{i j} 
\right) \\
\end{align*}

To complete the task, it only remains to show that this second term is zero.  First lets get rid of the $1/2$ by writing $1 = 1/2 + 1/2$ using the index swapping trick

\begin{align*}
F_{i j} \partial^i F^{m j} 
&= 
\inv{2} F_{i j} \partial^i F^{m j} + \inv{2} F_{j i} \partial^j F^{m i} \\
&= 
\inv{2} F_{i j} \left( \partial^i F^{m j} - \partial^j F^{m i} \right)
\end{align*}

This gives us for the second term

\begin{align*}
\frac{F_{i j}}{4 \pi} \left( -\partial^i F^{m j} + \inv{2} \partial^m F^{i j} \right) 
&=
\frac{F_{i j}}{8 \pi} \left( \partial^i F^{j m} + \partial^j F^{m i} + \partial^m F^{i j} \right) \\
&=
\frac{F_{i j}}{8 \pi} \left( 
\partial^i \partial^j A^m
-\partial^i \partial^m A^j
+\partial^j \partial^m A^i
-\partial^j \partial^i A^m
+\partial^m \partial^i A^j
-\partial^m \partial^j A^i
\right) \\
\end{align*}

By commuting derivatives, assuming the typical sufficient continuity of the fields, all of these six terms in braces cancel.  This completes this portion of the exercise.

\subsection{2. Timelike component of the conservation relation.}

\subsubsection{Statement.}

Consider the $m = 0$ components of \ref{eqn:relativisticElectrodynamicsPS6P1:30}.  Show that it implies the energy conservation equation already discussed in class (see notes pp. 125-127):

\begin{equation}\label{eqn:relativisticElectrodynamicsPS6P1:40}
\PD{t}{\mathcal{E}} + \spacegrad \cdot \BS = - \BE \cdot \Bj.
\end{equation}

Recall the physical interpretation of the various terms in this equation.

\subsubsection{Solution.}

The goal is to express the four divergence

\begin{equation}\label{eqn:relativisticElectrodynamicsPS6P1:45}
\partial_k T^{k 0} = -\inv{c} F^{0 a} j_a,
\end{equation}

explicitly utilizing a space time split from some stationary frame where the fields and currents are observed as $\BE$, $\BB$, $\Bj$, and $\rho$.  On the RHS, because $F^{0 0} = 0$ the summation is reduced to three indexes

\begin{equation}\label{eqn:relativisticElectrodynamicsPS6P1:80}
F^{0 a} j_a = F^{0 \alpha} j_\alpha = -F^{0 \alpha} (\Bj)^\alpha.
\end{equation}

In this the tensor factor is

\begin{align*}
F^{0 \alpha} 
&= \partial^0 A^\alpha - \partial^\alpha A^0 \\
&= \inv{c} \partial_t A^\alpha + \partial_\alpha A^0 \\
&= -(\BE)^\alpha,
\end{align*}

and the RHS of \ref{eqn:relativisticElectrodynamicsPS6P1:45} is reduced to

\begin{equation}\label{eqn:relativisticElectrodynamicsPS6P1:100}
-\inv{c} F^{0 a} j_a = -\inv{c} \BE \cdot \Bj.
\end{equation}

Now let's expand the LHS.  Recall that

\begin{align}\label{eqn:relativisticElectrodynamicsPS6P1:180}
T^{ 0 0 } &= \inv{8 \pi} ( \BE^2 + \BB^2 ) = \mathcal{E} \\
T^{\alpha 0} &= \inv{4 \pi} (\BE \cross \BB)^\alpha = \frac{\BS^\alpha}{c}.
\end{align}

With $\partial_0 = \partial_t/c$, our equation becomes

\begin{equation}\label{eqn:relativisticElectrodynamicsPS6P1:160}
\partial_k T^{k 0} = \inv{c} \PD{t}{} \mathcal{E} + \PD{x^\alpha}{} \frac{\BS^\alpha}{c} = -\inv{c} \BE \cdot \Bj.
\end{equation}

Multiplying through by $c$ recovers \ref{eqn:relativisticElectrodynamicsPS6P1:40} as desired.

\subsection{3. Spacelike component of the conservation relation.}

\subsubsection{Statement.}

Consider the $m = \alpha$ components of \ref{eqn:relativisticElectrodynamicsPS6P1:30}.  Show that it implies that:

\begin{equation}\label{eqn:relativisticElectrodynamicsPS6P1:41}
\PD{t}{}\left( \frac{S^\alpha}{c^2} \right) + \PD{x^\beta}{} T^{\beta \alpha}
= - \left( \rho E^\alpha + \inv{c} \left( \Bj \cross \BB \right)^\alpha \right) \equiv - f^\alpha
\end{equation}

Give a physical interpretation of $f^\alpha$.

\subsubsection{Solution.}

The goal is to expand

\begin{equation}\label{eqn:relativisticElectrodynamicsPS6P1:42}
\partial_k T^{k \alpha} = -\inv{c} F^{\alpha l} j_l.
\end{equation}

On the RHS is 

\begin{align*}
-\inv{c} F^{\alpha l} j_l
&=
-\inv{c} 
\left( 
F^{\alpha 0} j_0 
+F^{\alpha \beta} j_\beta 
\right) \\
&= 
- \BE^\alpha \rho - \inv{c} (- \epsilon^{\sigma \alpha \beta} \BB^\sigma) (-\Bj^\beta) \\
&= 
- \BE^\alpha \rho - \inv{c} \epsilon^{\alpha \beta \sigma} \BB^\sigma \Bj^\beta \\
&= 
- (\rho \BE + \frac{\Bj}{c} \cross \BB)^\alpha
\end{align*}

For the LHS of \ref{eqn:relativisticElectrodynamicsPS6P1:42}, using

\begin{equation}\label{eqn:relativisticElectrodynamicsPS6P1:200}
T^{0 \alpha} = \frac{\BS^\alpha}{c}.
\end{equation}

Putting the pieces together leaves us with the desired relationship

\begin{equation}\label{eqn:relativisticElectrodynamicsPS6P1:201}
\inv{c} \PD{t}{} \frac{S^\alpha}{c} + \PD{x^\beta}{T^{\beta \alpha}} = 
- (\rho \BE + \frac{\Bj}{c} \cross \BB)^\alpha
\end{equation}

The RHS can be seen to be the (negated) Lorentz force per unit volume.  Introducing discrete charge and current densities utilizing delta functions and integrating, gives us exactly the spatial (non-energy) components of the Lorentz force equation (this is done in detail in the next portion of this problem below).

This is a rather interesting result.  In \S 33 of \cite{landau1980classical} the energy momentum tensor was found to be closely related to the spacetime translation symmetries for the charge and current free Lagrangian density for the field (although this produced a non-symmetric tensor and a special value of zero had to be added to get it into symmetric form).  So without any requirement to perform variation of the interaction action

\begin{equation}\label{eqn:relativisticElectrodynamicsPS6P1:202}
S = -m c \int ds - \frac{e}{c} \int ds u^i A_i,
\end{equation}

one still ends up with all the components of the Lorentz force equation.  Only the Lagrangian density for the field was required to obtain the result (which was also indirectly used to obtain the relation of the field to the charge and current densities).  The interaction action (and thus the Lorentz force equation itself) seems to be \underline{almost} redundant.  What it does provide, however, is excellent motivation for the labeling of

\begin{equation}\label{eqn:relativisticElectrodynamicsPS6P1:203}
\PD{t}{} \frac{S^\alpha}{c^2},
\end{equation}

as momentum density for the EM field (a much better motivation than the dimensional analysis arguments used in class when the Poynting vector $\BS$ was introduced).  With force per volume on the RHS and the time derivative of a ``something'' $S^\alpha/c^2$ on the LHS, one is forced to conclude that this ``something'' is a momentum density.

\subsection{4. Integrated over a volume.}

\subsubsection{Statement.}

Integrate \ref{eqn:relativisticElectrodynamicsPS6P1:41} over a closed volume $V$ and use integration by parts to obtain

\begin{equation}\label{eqn:relativisticElectrodynamicsPS6P1:50}
\PD{t}{} \int_V d^3 \Bx \frac{S^\alpha}{c^2} 
= 
- \int_{\partial V = S} d \sigma^\beta T^{\beta \alpha} - \int_V d^3 \Bx f^\alpha 
\end{equation}

Give a physical interpretation of \ref{eqn:relativisticElectrodynamicsPS6P1:50} as expressing momentum conservation.  In particular, explain how, if the volume $V$ is that of a body (made of charged particles -- bound or otherwise), this implies that:

\begin{equation}\label{eqn:relativisticElectrodynamicsPS6P1:60}
\begin{aligned}
&\ddt{} \left( 
\Bp_{\text{EM field in $V$}} + 
\Bp_{\text{charged particles in $V$}} 
\right)^\alpha \\
&\qquad = \int_{\text{surface of body}} \left( 
(\text{surface force})^\alpha \text{on body due to shears and pressures}
\right)
\end{aligned}
\end{equation}

(Note that here $d \sigma^\beta$ is an outward normal vector to the surface of the body, so the surface has a relative minus signs w.r.t the one from class, where an inward normal was used.)

\subsubsection{Solution.}

Integrating \ref{eqn:relativisticElectrodynamicsPS6P1:41} over a closed volume $V$ gives

\begin{align*}
0 
&=
\int_V d^3 \Bx \PD{t}{}\left( \frac{S^\alpha}{c^2} \right) 
+ 
\int_V d^3 \Bx \PD{x^\beta}{} T^{\beta \alpha} 
+
\int_V d^3 \Bx \rho E^\alpha + \inv{c} \left( \Bj \cross \BB \right)^\alpha  \\
&=
\PD{t}{} \int_V d^3 \Bx \frac{S^\alpha}{c^2} 
+ 
\int_V d^3 \Bx \spacegrad \cdot (\Be_\beta T^{\beta \alpha})
+
\sum_b q_b \int_V d^3 \Bx \left( E^\alpha + \inv{c} \left( \Bv_b(t) \cross \BB \right)^\alpha \right) \delta^3(\Bx - \Bx_b(t)) \\
&=
\PD{t}{} \int_V d^3 \Bx \frac{S^\alpha}{c^2} 
+ 
\int_{\partial V} d \sigma (\Bn \cdot \Be_\beta) T^{\beta \alpha}
+
\sum_b q_b \left( E^\alpha(\Bx_b) + \left( \frac{\Bv_b(t)}{c} \cross \BB(\Bx_b) \right)^\alpha \right)
\end{align*}

In the first integral, the integration and time derivative operational order was exchanged.  In the second integral the contraction was written as a spatial divergence $\partial_\beta T^{\beta \alpha} = \spacegrad \cdot (\Be_\beta T^{\beta \alpha})$, so that Stokes theorem could be used to express this integral as the integral over the boundary of the surface, with outward normal $\Bn$.  In the last, the charge and current densities were expressed in terms of discrete particles

\begin{align}\label{eqn:relativisticElectrodynamicsPS6P1:230}
\rho &= \sum_b q_b \delta^3(\Bx - \Bx_b(t)) \\
\Bj &= \sum_b q_b \Bv_b(t) \delta^3(\Bx - \Bx_b(t)).
\end{align}

So with the surface area element $d \sigma$, and the outward normal $\Bn$ on that surface, an indexed normal area element can be introduced as in the problem statement

\begin{equation}\label{eqn:relativisticElectrodynamicsPS6P1:250}
d \sigma^\beta \equiv d \sigma (\Bn \cdot \Be_\beta).
\end{equation}

So our integrated conservation relationship is left in the form

\begin{equation}\label{eqn:relativisticElectrodynamicsPS6P1:270}
\PD{t}{} \int_V d^3 \Bx \frac{S^\alpha}{c^2} 
+ 
\int_{\partial V} d \sigma^\beta T^{\beta \alpha}
= -
\sum_b q_b \left( \BE(\Bx_b) + \left( \frac{\Bv_b(t)}{c} \cross \BB(\Bx_b) \right) \right)^\alpha 
\end{equation}

Observe that the RHS is the $\alpha$ component of the (negated) Lorentz force $f_\alpha$ on the particles from the field, so the RHS represents the force of the charge distribution on the field.  Looking at the LHS of the equation where the time derivative of $\int d^3\Bx S^\alpha/c^2$ appears, there is finally an excellent justification for calling $S^\alpha/c^2$ the momentum density (much better than the original dimensional analysis argument given in class).

Once this Lorentz force is expressed as a rate of change of momentum

\begin{equation}\label{eqn:relativisticElectrodynamicsPS6P1:290}
\frac{d}{dt} \Bp_{\text{charges}} = \sum_b q_b \left( \BE(\Bx_b) + \frac{\Bv_b(t)}{c} \cross \BB(\Bx_b) \right),
\end{equation}

and the field momentum is also expressed in terms of the momentum density

\begin{equation}\label{eqn:relativisticElectrodynamicsPS6P1:310}
\Bp_{\text{EM field}} = \int d^3 \Bx \frac{\BS}{c^2},
\end{equation}

the desired result is produced

\begin{equation}\label{eqn:relativisticElectrodynamicsPS6P1:330}
\ddt{} \left( \Bp_{\text{EM field}} + \Bp_{\text{charges}} \right)^\alpha = -\int_{\partial V} d \sigma^\beta T^{\beta \alpha}.
\end{equation}

Any change in the momentum of the field or the charges acted on by the field in a volume, is found to equal a force per unit area, acting on the surface of that volume.  Those components of this force that are normal to the surface can be call pressure, and just as in mechanics, the portion of this force per unit area acting tangentially along the surface, can be called shear.

\subsection{5. Pressure and shear of linearly polarized EM wave.}

\subsubsection{Statement.}

Imagine that a place linearly polarized electromagnetic wave is falling on a flat surface at an angle of incidence $\alpha$, and is completely absorbed by the body.  Find the pressure and shear on a unit area of the surface using the Maxwell stress tensor.

\subsubsection{Solution.}

