%
% Copyright � 2013 Peeter Joot.  All Rights Reserved.
% Licenced as described in the file LICENSE under the root directory of this GIT repository.
%
\makeproblem{Temperature dependent carrier density of a doped semiconductor}{condensedMatter:problemSet10:2}{ 
Germanium has an energy gap of $E_{\mathrm{C}} - E_{\mathrm{V}} = 0.67 \Unit{e V}$, and an intrinsic carrier density $n_i = 2.5 � 10^{19} \Unit{m^{-3}}$ at room temperature.  A sample of germanium is doped with arsenic, which has a donor level located at $E_{\mathrm{d}} = 0.0127 \Unit{e V}$ below $E_{\mathrm{C}}$. The concentration of arsenic atoms is $N_{\mathrm{D}} = 1.0 \times 10^{22} \Unit{m^{-3}}$.

Show the temperature dependence of the carrier density in this sample by drawing a plot of $\ln(n)$ vs. $1/T$, between $1 \Unit{K}$ and $293 \Unit{K}$ (room temperature), and a second plot covering the temperature range from $100 \Unit{K}$ to $1000 \Unit{K}$. Explain the main features of the plots.
} % makeproblem

\makeanswer{condensedMatter:problemSet10:2}{ 

TODO.
}
