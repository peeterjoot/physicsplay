%
% Copyright � 2016 Peeter Joot.  All Rights Reserved.
% Licenced as described in the file LICENSE under the root directory of this GIT repository.
%
%{
\newcommand{\authorname}{Peeter Joot}
\newcommand{\email}{peeterjoot@protonmail.com}
\newcommand{\basename}{FIXMEbasenameUndefined}
\newcommand{\dirname}{notes/FIXMEdirnameUndefined/}

\renewcommand{\basename}{magneticMomentJackson}
%\renewcommand{\dirname}{notes/phy1520/}
\renewcommand{\dirname}{notes/ece1228-electromagnetic-theory/}
%\newcommand{\dateintitle}{}
%\newcommand{\keywords}{}

\newcommand{\authorname}{Peeter Joot}
\newcommand{\onlineurl}{http://sites.google.com/site/peeterjoot2/math2013/\basename.pdf}
\newcommand{\sourcepath}{\dirname\basename.tex}
\newcommand{\generatetitle}[1]{\chapter{#1}}

\newcommand{\vcsinfo}{%
\section*{}
\noindent{\color{DarkOliveGreen}{\rule{\linewidth}{0.1mm}}}
\paragraph{Document version}
%\paragraph{\color{Maroon}{Document version}}
{
\small
\begin{itemize}
\item Available online at:\\ 
\href{\onlineurl}{\onlineurl}
\item Git Repository: \input{./.revinfo/gitRepo.tex}
\item Source: \sourcepath
\item last commit: \input{./.revinfo/gitCommitString.tex}
\item commit date: \input{./.revinfo/gitCommitDate.tex}
\end{itemize}
}
}

%\PassOptionsToPackage{dvipsnames,svgnames}{xcolor}
\PassOptionsToPackage{square,numbers}{natbib}
\documentclass{scrreprt}

\usepackage[left=2cm,right=2cm]{geometry}
\usepackage[svgnames]{xcolor}
\usepackage{peeters_layout}

\usepackage{natbib}

\usepackage[
colorlinks=true,
bookmarks=false,
pdfauthor={\authorname, \email},
backref 
]{hyperref}

% http://tex.stackexchange.com/questions/75773/how-to-reference-problems-by-the-text-label-in-an-exercise-envioronment
\usepackage[english]{cleveref}
\crefname{Exercise}{exercise}{exercises}
\Crefname{Exercise}{Exercise}{Exercises}

\RequirePackage{titlesec}
\RequirePackage{ifthen}

% http://stackoverflow.com/questions/4932910/date-in-the-tabular-environment
\makeatletter
\let\insertdate\@date
\makeatother

\titleformat{\chapter}[display]
{\bfseries\Large}
{\color{DarkSlateGrey}\filleft \authorname
\ifthenelse{\isundefined{\studentnumber}}{}{\\ \studentnumber}
\ifthenelse{\isundefined{\email}}{}{\\ \email}
\ifthenelse{\isundefined{\dateintitle}}{}{\\ \insertdate}
%\ifthenelse{\isundefined{\coursename}}{}{\\ \coursename} % put in title instead.
}
{4ex}
{\color{DarkOliveGreen}{\titlerule}\color{Maroon}
\vspace{2ex}%
\filright}
[\vspace{2ex}%
\color{DarkOliveGreen}\titlerule
]

\newcommand{\beginArtWithToc}[0]{\begin{document}\tableofcontents}
\newcommand{\beginArtNoToc}[0]{\begin{document}}
\newcommand{\EndNoBibArticle}[0]{\end{document}}
\newcommand{\EndArticle}[0]{\bibliography{Bibliography}\bibliographystyle{plainnat}\end{document}}

% 
%\newcommand{\citep}[1]{\cite{#1}}

\colorSectionsForArticle



\usepackage{peeters_layout_exercise}
\usepackage{peeters_braket}
\usepackage{peeters_figures}
\usepackage{siunitx}
%\usepackage{txfonts} % \ointclockwise

\beginArtNoToc

\generatetitle{Magnetic moment for a localized magnetostatic current}
%\chapter{Magnetic moment for a localized magnetostatic current}
%\label{chap:magneticMomentJackson}
% \citep{sakurai2014modern} pr X.Y
% \citep{pozar2009microwave}
% \citep{qftLectureNotes}
% \citep{doran2003gap}
\paragraph{Motivation.}

I was once again reading my Jackson \citep{jackson1975cew}.  This time I found that his 
presentation of magnetic moment didn't really make sense to me.  Here's my own pass through it, filling in a number of details.  As I did last time, I'll also translate into SI units as I go.

\paragraph{Vector potential.}

The Biot-Savart expression for the magnetic field can be factored into a curl expression using the usual tricks

\begin{dmath}\label{eqn:magneticMomentJackson:20}
\BB 
= \frac{\mu_0}{4\pi} \int \frac{\BJ(\Bx') \cross (\Bx - \Bx')}{\Abs{\Bx - \Bx'}^3} d^3 x'
= -\frac{\mu_0}{4\pi} \int \BJ(\Bx') \cross \spacegrad \inv{\Abs{\Bx - \Bx'}} d^3 x'
= \frac{\mu_0}{4\pi} \spacegrad \cross \int \frac{\BJ(\Bx')}{\Abs{\Bx - \Bx'}} d^3 x',
\end{dmath}

so the vector potential, through its curl, defines the magnetic field \( \BB = \spacegrad \cross \BA \) is given by

\begin{dmath}\label{eqn:magneticMomentJackson:40}
\BA(\Bx) = \frac{\mu_0}{4 \pi} \int \frac{J(\Bx')}{\Abs{\Bx - \Bx'}} d^3 x'.
\end{dmath}

If the current source is localized (zero outside of some finite region), then there will always be a region for which \( \Abs{\Bx} \gg \Abs{\Bx'} \), so the denominator yields to Taylor expansion

\begin{dmath}\label{eqn:magneticMomentJackson:60}
\inv{\Abs{\Bx - \Bx'}}
=
\inv{\Abs{\Bx}} \lr{1 + \frac{\Abs{\Bx'}^2}{\Abs{\Bx}^2} - 2 \frac{\Bx \cdot \Bx'}{\Abs{\Bx}^2} }^{-1/2}
\approx
\inv{\Abs{\Bx}} \lr{ 1 + \frac{\Bx \cdot \Bx'}{\Abs{\Bx}^2} }
=
\inv{\Abs{\Bx}} + \frac{\Bx \cdot \Bx'}{\Abs{\Bx}^3}.
\end{dmath}

so the vector potential, far enough away from the current source is
\begin{dmath}\label{eqn:magneticMomentJackson:80}
\BB(\Bx) 
=
\frac{\mu_0}{4 \pi} \int \frac{J(\Bx')}{\Abs{\Bx}} d^3 x'
+\frac{\mu_0}{4 \pi} \int \frac{(\Bx \cdot \Bx')J(\Bx')}{\Abs{\Bx}^3} d^3 x'.
\end{dmath}

Jackson uses a sneaky trick to show that the first integral is killed for a localized source.  That trick appears to be based on evaluating the following divergence

\begin{dmath}\label{eqn:magneticMomentJackson:100}
\spacegrad \cdot (\BJ(\Bx) x_i)
=
(\spacegrad \cdot \BJ) x_i
+
(\spacegrad x_i) \cdot \BJ
=
(\Be_k \partial_k x_i) \cdot\BJ
=
\delta_{ki} J_k
=
J_i.
\end{dmath}

Note that this made use of the fact that \( \spacegrad \cdot \BJ = 0 \) for magnetostatics.  This provides a way to rewrite the current density as a divergence

\begin{dmath}\label{eqn:magneticMomentJackson:120}
\int \frac{J(\Bx')}{\Abs{\Bx}} d^3 x'
=
\Be_i \int \frac{\spacegrad' \cdot (x_i' \BJ(\Bx'))}{\Abs{\Bx}} d^3 x'
=
\frac{\Be_i}{\Abs{\Bx}} \int \spacegrad' \cdot (x_i' \BJ(\Bx')) d^3 x'
=
\frac{1}{\Abs{\Bx}} \oint \Bx' (d\Ba \cdot \BJ(\Bx')).
\end{dmath}

When \( \BJ \) is localized, this is zero provided we pick the integration surface for the volume outside of that localization region.

It is now desired to rewrite \( \int \Bx \cdot \Bx' \BJ \) as a triple cross product since the dot product of such a triple cross product has exactly this term in it

\begin{dmath}\label{eqn:magneticMomentJackson:140}
- \Bx \cross \int \Bx' \cross \BJ
=
\int (\Bx \cdot \Bx') \BJ
-
\int (\Bx \cdot \BJ) \Bx'
=
\int (\Bx \cdot \Bx') \BJ
-
\Be_k x_i \int J_i x_k',
\end{dmath}

so
\begin{dmath}\label{eqn:magneticMomentJackson:160}
\int (\Bx \cdot \Bx') \BJ
=
- \Bx \cross \int \Bx' \cross \BJ
+
\Be_k x_i \int J_i x_k'.
\end{dmath}

To get of this second term, the next sneaky trick is to consider the following divergence

\begin{dmath}\label{eqn:magneticMomentJackson:180}
\oint d\Ba' \cdot (\BJ(\Bx') x_i' x_j')
=
\int dV' \spacegrad' \cdot (\BJ(\Bx') x_i' x_j')
=
\int dV' (\spacegrad' \cdot \BJ)
+
\int dV' \BJ \cdot \spacegrad' (x_i' x_j')
=
\int dV' J_k \cdot \lr{ x_i' \partial_k x_j' + x_j' \partial_k x_i' }
=
\int dV' \lr{ J_k x_i' \delta_{kj} + J_k x_j' \delta_{ki} }
=
\int dV' \lr{ J_j x_i' + J_i x_j'}.
\end{dmath}

The surface integral is once again zero, which means that we have an antisymmetric relationship in integrals of the form

\begin{dmath}\label{eqn:magneticMomentJackson:200}
\int J_j x_i' = -\int J_i x_j'.
\end{dmath}

Now we can use the tensor algebra trick of writing \( y = (y + y)/2 \),

\begin{dmath}\label{eqn:magneticMomentJackson:220}
\int (\Bx \cdot \Bx') \BJ
=
- \Bx \cross \int \Bx' \cross \BJ
+
\Be_k x_i \int J_i x_k'
=
- \Bx \cross \int \Bx' \cross \BJ
+
\inv{2} \Be_k x_i \int \lr{ J_i x_k' + J_i x_k' }
=
- \Bx \cross \int \Bx' \cross \BJ
+
\inv{2} \Be_k x_i \int \lr{ J_i x_k' - J_k x_i' }
=
- \Bx \cross \int \Bx' \cross \BJ
+
\inv{2} \Be_k x_i \int (\BJ \cross \Bx')_j \epsilon_{ikj}
=
- \Bx \cross \int \Bx' \cross \BJ
-
\inv{2} \epsilon_{kij} \Be_k x_i \int (\BJ \cross \Bx')_j 
=
- \Bx \cross \int \Bx' \cross \BJ
-
\inv{2} \Bx \cross \int \BJ \cross \Bx'
=
- \Bx \cross \int \Bx' \cross \BJ
+
\inv{2} \Bx \cross \int \Bx' \cross \BJ
=
-\inv{2} \Bx \cross \int \Bx' \cross \BJ,
\end{dmath}

so

\begin{dmath}\label{eqn:magneticMomentJackson:240}
\BA(\Bx) \approx \frac{\mu_0}{4 \pi \Abs{\Bx}^3} \lr{ -\frac{\Bx}{2} } \int \Bx' \cross \BJ(\Bx') d^3 x'.
\end{dmath}

Letting 

%\begin{dmath}\label{eqn:magneticMomentJackson:260}
\boxedEquation{eqn:magneticMomentJackson:260}{
\Bm = \inv{2} \int \Bx' \cross \BJ(\Bx') d^3 x',
}
%\end{dmath}

the far field approximation of the vector potential is
%\begin{dmath}\label{eqn:magneticMomentJackson:280}
\boxedEquation{eqn:magneticMomentJackson:280}{
\BA(\Bx) = \frac{\mu_0}{4 \pi} \frac{\Bm \cross \Bx}{\Abs{\Bx}^3}.
}
%\end{dmath}

Note that when the current is restricted to an infintisimally thin loop, the magnetic moment reduces to

\begin{dmath}\label{eqn:magneticMomentJackson:300}
\Bm(\Bx) = \frac{I}{2} \int \Bx \cross d\Bl'.
\end{dmath}

Refering to \citep{griffiths1999introduction} (pr. 1.60), this can be seen to be \( I \) times the ``vector-area'' integral.

%}
\EndArticle
