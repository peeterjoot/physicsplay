%
% Copyright � 2013 Peeter Joot.  All Rights Reserved.
% Licenced as described in the file LICENSE under the root directory of this GIT repository.
%
\makeproblem{Geometrical construction of the reciprocal lattice}{condensedMatter:problemSet3:1}{ 
\makesubproblem{}{condensedMatter:problemSet3:1a}
  
\Cref{fig:reciprocal_lattice:reciprocal_latticeFig1}.
shows a two-dimensional lattice together with two 
  vectors.  Demonstrate that $\Ba_1$ and $\Ba_2$ 
  are basis vectors by showing that you can reach the sites numbered 
  1 through 5 by a combination of $\Ba_1$ and $\Ba_2$. 
(e.g. to reach site 1 use $0*\Ba_1 + 1*\vec{a_2}$, etc.. )

\imageFigure{reciprocal_latticeFig1}{A sample lattice}{fig:reciprocal_lattice:reciprocal_latticeFig1}{0.3}

\makesubproblem{}{condensedMatter:problemSet3:1b}
On the page and using a ruler and/or protractor and/or any other 
  drafting tools you may require, draw the two basis vectors $\Bg_1$ and 
  $\Bg_2$ of the corresponding reciprocal lattice. 
Draw $\Bg_1$ and $\Bg_2$ to scale (so they have the correct 
  length relative to each other). 

\makesubproblem{}{condensedMatter:problemSet3:1c}
Consider three waves, with wave-vectors $\Bk_a = \Bg_1$, 
  $\Bk_b = \Bg_2$ and $\Bk_c = \Bg_1 + 2\Bg_2$, 
  draw lines on the diagram to indicate the positions of the wave-crests, 
  assuming that one of the crests passes through the point ``0".  
  If these lines correspond to ``planes" in the two-dimensional crystal, 
  give the Miller indices of the planes.

} % makeproblem

\makeanswer{condensedMatter:problemSet3:1}{ 
\makeSubAnswer{}{condensedMatter:problemSet3:1a}

TODO.
\makeSubAnswer{}{condensedMatter:problemSet3:1b}

TODO.
\makeSubAnswer{}{condensedMatter:problemSet3:1c}

TODO.
}

