%
% Copyright � 2014 Peeter Joot.  All Rights Reserved.
% Licenced as described in the file LICENSE under the root directory of this GIT repository.
%
\makeproblem{Modified Nodal Analysis}{multiphysics:problemSet1:1}{ 
\makesubproblem{
Write a MATLAB routine [G,b]=NodalAnalysis(filename)
that generates the modified nodal analysis (MNA) equations.
}{multiphysics:problemSet1:1a}
\makesubproblem{Explain how did you include the controlled source into the mod-
ified nodal analysis formulation. Which general rule can be given to
``stamp'' a voltage-controlled voltage source into MNA?}{multiphysics:problemSet1:1b}
\makesubproblem{Consider the circuit shown in the figure \cref{fig:ps1Orig:ps1OrigFig1}. Write an input file for the netlist parser developed in the previous point, and use it to generate the matrices G and b for the circuit. The operational amplifiers have an input resistance of \(1 M \Omega\)? and a gain of 10^6 . Model them with a resistor and a voltage-controlled voltage source. Use the MATLAB command \(\backslash\) to solve the linear system and determine the voltage \(V_\circ\) shown in the figure.}{multiphysics:problemSet1:1c}
\imageFigure{../../figures/ece1254/ps1OrigFig1}{Circuit to solve}{fig:ps1Orig:ps1OrigFig1}{0.4}
\makesubproblem{Implement your own LU factorization routine. Repeat the previous point using your own LU factorization and forward/backward substitution routines to solve the circuit equations. Report the computed \(V_\circ\).}{multiphysics:problemSet1:1d}
} % makeproblem

\makeanswer{multiphysics:problemSet1:1}{ 
\makeSubAnswer{}{multiphysics:problemSet1:1a}

TODO.
\makeSubAnswer{}{multiphysics:problemSet1:1b}

TODO.
\makeSubAnswer{}{multiphysics:problemSet1:1c}

Fig 2.3 of \citep{sedra1982microelectronic} illustrates a model of an ideal op amp
%\cref{fig:idealOpAmp:idealOpAmpFig1}.
\imageFigure{../../figures/ece1254/idealOpAmpFig1}{Ideal op amp}{fig:idealOpAmp:idealOpAmpFig1}{0.3}

The model for this problem differs only by having a current and resistance between the input terminals.

\makeSubAnswer{}{multiphysics:problemSet1:1d}

TODO.
}

