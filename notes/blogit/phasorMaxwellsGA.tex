%
% Copyright � 2015 Peeter Joot.  All Rights Reserved.
% Licenced as described in the file LICENSE under the root directory of this GIT repository.
%
\newcommand{\authorname}{Peeter Joot}
\newcommand{\email}{peeterjoot@protonmail.com}
\newcommand{\basename}{FIXMEbasenameUndefined}
\newcommand{\dirname}{notes/FIXMEdirnameUndefined/}

\renewcommand{\basename}{phasorMaxwellsGA}
\renewcommand{\dirname}{notes/FIXMEwheretodirname/}
%\newcommand{\dateintitle}{}
%\newcommand{\keywords}{}

\newcommand{\authorname}{Peeter Joot}
\newcommand{\onlineurl}{http://sites.google.com/site/peeterjoot2/math2013/\basename.pdf}
\newcommand{\sourcepath}{\dirname\basename.tex}
\newcommand{\generatetitle}[1]{\chapter{#1}}

\newcommand{\vcsinfo}{%
\section*{}
\noindent{\color{DarkOliveGreen}{\rule{\linewidth}{0.1mm}}}
\paragraph{Document version}
%\paragraph{\color{Maroon}{Document version}}
{
\small
\begin{itemize}
\item Available online at:\\ 
\href{\onlineurl}{\onlineurl}
\item Git Repository: \input{./.revinfo/gitRepo.tex}
\item Source: \sourcepath
\item last commit: \input{./.revinfo/gitCommitString.tex}
\item commit date: \input{./.revinfo/gitCommitDate.tex}
\end{itemize}
}
}

%\PassOptionsToPackage{dvipsnames,svgnames}{xcolor}
\PassOptionsToPackage{square,numbers}{natbib}
\documentclass{scrreprt}

\usepackage[left=2cm,right=2cm]{geometry}
\usepackage[svgnames]{xcolor}
\usepackage{peeters_layout}

\usepackage{natbib}

\usepackage[
colorlinks=true,
bookmarks=false,
pdfauthor={\authorname, \email},
backref 
]{hyperref}

% http://tex.stackexchange.com/questions/75773/how-to-reference-problems-by-the-text-label-in-an-exercise-envioronment
\usepackage[english]{cleveref}
\crefname{Exercise}{exercise}{exercises}
\Crefname{Exercise}{Exercise}{Exercises}

\RequirePackage{titlesec}
\RequirePackage{ifthen}

% http://stackoverflow.com/questions/4932910/date-in-the-tabular-environment
\makeatletter
\let\insertdate\@date
\makeatother

\titleformat{\chapter}[display]
{\bfseries\Large}
{\color{DarkSlateGrey}\filleft \authorname
\ifthenelse{\isundefined{\studentnumber}}{}{\\ \studentnumber}
\ifthenelse{\isundefined{\email}}{}{\\ \email}
\ifthenelse{\isundefined{\dateintitle}}{}{\\ \insertdate}
%\ifthenelse{\isundefined{\coursename}}{}{\\ \coursename} % put in title instead.
}
{4ex}
{\color{DarkOliveGreen}{\titlerule}\color{Maroon}
\vspace{2ex}%
\filright}
[\vspace{2ex}%
\color{DarkOliveGreen}\titlerule
]

\newcommand{\beginArtWithToc}[0]{\begin{document}\tableofcontents}
\newcommand{\beginArtNoToc}[0]{\begin{document}}
\newcommand{\EndNoBibArticle}[0]{\end{document}}
\newcommand{\EndArticle}[0]{\bibliography{Bibliography}\bibliographystyle{plainnat}\end{document}}

% 
%\newcommand{\citep}[1]{\cite{#1}}

\colorSectionsForArticle



\usepackage{ece1229}

\beginArtNoToc

\generatetitle{Maxwell's (phasor) equations in Geometric Algebra}
%\chapter{Maxwell's (phasor) equations in Geometric Algebra}
%\label{chap:phasorMaxwellsGA}
%\section{Motivation}


In \citep{balanis2005antenna} \S 3.2 is a demonstration of the required (curl) form for the magnetic field, and potential form for the electric field.

I was wondering how this derivation would proceed using the Geometric Algebra (GA) formalism.

\section{Maxwell's equation in GA phasor form.}

Maxwell's equations, omitting magnetic charges and currents, are

\begin{subequations}
\begin{dmath}\label{eqn:phasorMaxwellsGA:20}
\spacegrad \cross \bcE = -\PD{t}{\bcB}
\end{dmath}
\begin{dmath}\label{eqn:phasorMaxwellsGA:40}
\spacegrad \cross \bcH = \bcJ + \PD{t}{\bcD}
\end{dmath}
\begin{dmath}\label{eqn:phasorMaxwellsGA:60}
\spacegrad \cdot \bcD = \rho
\end{dmath}
\begin{dmath}\label{eqn:phasorMaxwellsGA:80}
\spacegrad \cdot \bcB = 0.
\end{dmath}
\end{subequations}

Assuming linear media \( \bcB = \mu_0 \bcH \), \( \bcD = \epsilon_0 \bcE \), and phasor relationships of the form \( \bcE = \Real \lr{ \BE(\Br) e^{j \omega t}} \) for the fields and the currents, these reduce to

\begin{subequations}
\label{eqn:phasorMaxwellsGA:99}
\begin{dmath}\label{eqn:phasorMaxwellsGA:100}
\spacegrad \cross \BE = - j \omega \BB
\end{dmath}
\begin{dmath}\label{eqn:phasorMaxwellsGA:120}
\spacegrad \cross \BB = \mu_0 \BJ + j \omega \epsilon_0 \mu_0 \BE
\end{dmath}
\begin{dmath}\label{eqn:phasorMaxwellsGA:140}
\spacegrad \cdot \BE = \rho/\epsilon_0
\end{dmath}
\begin{dmath}\label{eqn:phasorMaxwellsGA:160}
\spacegrad \cdot \BB = 0.
\end{dmath}
\end{subequations}

These four equations can be assembled into a single equation form using the GA identities

\begin{subequations}
\label{eqn:phasorMaxwellsGA:180}
\begin{equation}\label{eqn:phasorMaxwellsGA:200}
\Bf \Bg 
= \Bf \cdot \Bg + \Bf \wedge \Bg
= \Bf \cdot \Bg + I \Bf \cross \Bg.
\end{equation}
\begin{dmath}\label{eqn:phasorMaxwellsGA:220}
I = \xcap \ycap \zcap.
\end{dmath}
\end{subequations}

The electric and magnetic field equations, respectively, are

\begin{subequations}
\label{eqn:phasorMaxwellsGA:240}
\begin{equation}\label{eqn:phasorMaxwellsGA:260}
\spacegrad \BE = \rho/\epsilon_0 -j k c \BB I
\end{equation}
\begin{equation}\label{eqn:phasorMaxwellsGA:280}
\spacegrad c \BB = \frac{I}{\epsilon_0 c} \BJ + j k \BE I
\end{equation}
\end{subequations}

where \( \omega = k c \), and \( 1 = c^2 \epsilon_0 \mu_0 \) have also been used to eliminate some of the mess of constants.

Summing these (first scaling \cref{eqn:phasorMaxwellsGA:280} by \( I \)), gives Maxwell's equation in its GA phasor form

\boxedEquation{eqn:phasorMaxwellsGA:300}{
%\begin{equation}\label{eqn:phasorMaxwellsGA:300}
\lr{ \spacegrad + j k } \lr{ \BE + I c \BB } = \inv{\epsilon_0 c}\lr{c \rho - \BJ}.
%\end{equation}
}

\section{Constructing a potential representation.}

\EndArticle
