%
%
%
% Copyright � 2012 Peeter Joot
% All Rights Reserved
%
% This file may be reproduced and distributed in whole or in part, without fee, subject to the following conditions:
%
% o The copyright notice above and this permission notice must be preserved complete on all complete or partial copies.
%
% o Any translation or derived work must be approved by the author in writing before distribution.
%
% o If you distribute this work in part, instructions for obtaining the complete version of this file must be included, and a means for obtaining a complete version provided.
%
%
% Exceptions to these rules may be granted for academic purposes: Write to the author and ask.
%
%
%
%
% Copyright � 2015 Peeter Joot.  All Rights Reserved.
% Licenced as described in the file LICENSE under the root directory of this GIT repository.
%
\documentclass[]{eliblog}

\usepackage{amsmath}
\usepackage{mathpazo}

%
% shorthand for bold symbols, convenient for vectors and matrices
%
\newcommand{\Ba}[0]{\mathbf{a}}
\newcommand{\Bb}[0]{\mathbf{b}}
\newcommand{\Bc}[0]{\mathbf{c}}
\newcommand{\Bd}[0]{\mathbf{d}}
\newcommand{\Be}[0]{\mathbf{e}}
\newcommand{\Bf}[0]{\mathbf{f}}
\newcommand{\Bg}[0]{\mathbf{g}}
\newcommand{\Bh}[0]{\mathbf{h}}
\newcommand{\Bi}[0]{\mathbf{i}}
\newcommand{\Bj}[0]{\mathbf{j}}
\newcommand{\Bk}[0]{\mathbf{k}}
\newcommand{\Bl}[0]{\mathbf{l}}
\newcommand{\Bm}[0]{\mathbf{m}}
\newcommand{\Bn}[0]{\mathbf{n}}
\newcommand{\Bo}[0]{\mathbf{o}}
\newcommand{\Bp}[0]{\mathbf{p}}
\newcommand{\Bq}[0]{\mathbf{q}}
\newcommand{\Br}[0]{\mathbf{r}}
\newcommand{\Bs}[0]{\mathbf{s}}
\newcommand{\Bt}[0]{\mathbf{t}}
\newcommand{\Bu}[0]{\mathbf{u}}
\newcommand{\Bv}[0]{\mathbf{v}}
\newcommand{\Bw}[0]{\mathbf{w}}
\newcommand{\Bx}[0]{\mathbf{x}}
\newcommand{\By}[0]{\mathbf{y}}
\newcommand{\Bz}[0]{\mathbf{z}}
\newcommand{\BA}[0]{\mathbf{A}}
\newcommand{\BB}[0]{\mathbf{B}}
\newcommand{\BC}[0]{\mathbf{C}}
\newcommand{\BD}[0]{\mathbf{D}}
\newcommand{\BE}[0]{\mathbf{E}}
\newcommand{\BF}[0]{\mathbf{F}}
\newcommand{\BG}[0]{\mathbf{G}}
\newcommand{\BH}[0]{\mathbf{H}}
\newcommand{\BI}[0]{\mathbf{I}}
\newcommand{\BJ}[0]{\mathbf{J}}
\newcommand{\BK}[0]{\mathbf{K}}
\newcommand{\BL}[0]{\mathbf{L}}
\newcommand{\BM}[0]{\mathbf{M}}
\newcommand{\BN}[0]{\mathbf{N}}
\newcommand{\BO}[0]{\mathbf{O}}
\newcommand{\BP}[0]{\mathbf{P}}
\newcommand{\BQ}[0]{\mathbf{Q}}
\newcommand{\BR}[0]{\mathbf{R}}
\newcommand{\BS}[0]{\mathbf{S}}
\newcommand{\BT}[0]{\mathbf{T}}
\newcommand{\BU}[0]{\mathbf{U}}
\newcommand{\BV}[0]{\mathbf{V}}
\newcommand{\BW}[0]{\mathbf{W}}
\newcommand{\BX}[0]{\mathbf{X}}
\newcommand{\BY}[0]{\mathbf{Y}}
\newcommand{\BZ}[0]{\mathbf{Z}}

\newcommand{\Bzero}[0]{\mathbf{0}}
\newcommand{\Btheta}[0]{\boldsymbol{\theta}}
\newcommand{\Btau}[0]{\boldsymbol{\tau}}
\newcommand{\Bomega}[0]{\boldsymbol{\omega}}

%
% shorthand for unit vectors
%
\newcommand{\acap}[0]{\hat{\Ba}}
\newcommand{\bcap}[0]{\hat{\Bb}}
\newcommand{\ccap}[0]{\hat{\Bc}}
\newcommand{\dcap}[0]{\hat{\Bd}}
\newcommand{\ecap}[0]{\hat{\Be}}
\newcommand{\fcap}[0]{\hat{\Bf}}
\newcommand{\gcap}[0]{\hat{\Bg}}
\newcommand{\hcap}[0]{\hat{\Bh}}
\newcommand{\icap}[0]{\hat{\Bi}}
\newcommand{\jcap}[0]{\hat{\Bj}}
\newcommand{\kcap}[0]{\hat{\Bk}}
\newcommand{\lcap}[0]{\hat{\Bl}}
\newcommand{\mcap}[0]{\hat{\Bm}}
\newcommand{\ncap}[0]{\hat{\Bn}}
\newcommand{\ocap}[0]{\hat{\Bo}}
\newcommand{\pcap}[0]{\hat{\Bp}}
\newcommand{\qcap}[0]{\hat{\Bq}}
\newcommand{\rcap}[0]{\hat{\Br}}
\newcommand{\scap}[0]{\hat{\Bs}}
\newcommand{\tcap}[0]{\hat{\Bt}}
\newcommand{\ucap}[0]{\hat{\Bu}}
\newcommand{\vcap}[0]{\hat{\Bv}}
\newcommand{\wcap}[0]{\hat{\Bw}}
\newcommand{\xcap}[0]{\hat{\Bx}}
\newcommand{\ycap}[0]{\hat{\By}}
\newcommand{\zcap}[0]{\hat{\Bz}}
\newcommand{\thetacap}[0]{\hat{\Btheta}}

%
% to write R^n and C^n in a distinguishable fashion.  Perhaps change this
% to the double lined characters upon figuring out how to do so.
%
\newcommand{\C}[1]{$\mathbb{C}^{#1}$}
\newcommand{\R}[1]{$\mathbb{R}^{#1}$}

%
% various generally useful helpers
%

% derivative of #1 wrt. #2:
\newcommand{\D}[2] {\frac {d#2} {d#1}}

\newcommand{\inv}[1]{\frac{1}{#1}}
\newcommand{\cross}[0]{\times}

\newcommand{\abs}[1]{\lvert{#1}\rvert}
\newcommand{\norm}[1]{\lVert{#1}\rVert}
\newcommand{\innerprod}[2]{\langle{#1}, {#2}\rangle}
\newcommand{\dotprod}[2]{{#1} \cdot {#2}}
\newcommand{\bdotprod}[2]{\left({#1} \cdot {#2}\right)}
\newcommand{\crossprod}[2]{{#1} \cross {#2}}
\newcommand{\tripleprod}[3]{\dotprod{\left(\crossprod{#1}{#2}\right)}{#3}}

\DeclareMathOperator{\Proj}{Proj}
\DeclareMathOperator{\Span}{span}
\DeclareMathOperator{\Sgn}{sgn}
\DeclareMathOperator{\Area}{Area}
\DeclareMathOperator{\Volume}{Volume}

%
% A few miscellaneous things specific to this document
%
\newcommand{\crossop}[1]{\crossprod{#1}{}}

% R2 vector.
\newcommand{\VectorTwo}[2]{
\begin{bmatrix}
 {#1} \\
 {#2}
\end{bmatrix}
}

\newcommand{\VectorN}[1]{
\begin{bmatrix}
{#1}_1 \\
{#1}_2 \\
\vdots \\
{#1}_N \\
\end{bmatrix}
}

\newcommand{\DETuvij}[4]{
\begin{vmatrix}
 {#1}_{#3} & {#1}_{#4} \\
 {#2}_{#3} & {#2}_{#4}
\end{vmatrix}
}

\newcommand{\DETuvwijk}[6]{
\begin{vmatrix}
 {#1}_{#4} & {#1}_{#5} & {#1}_{#6} \\
 {#2}_{#4} & {#2}_{#5} & {#2}_{#6} \\
 {#3}_{#4} & {#3}_{#5} & {#3}_{#6}
\end{vmatrix}
}

\newcommand{\DETuvwxijkl}[8]{
\begin{vmatrix}
 {#1}_{#5} & {#1}_{#6} & {#1}_{#7} & {#1}_{#8} \\
 {#2}_{#5} & {#2}_{#6} & {#2}_{#7} & {#2}_{#8} \\
 {#3}_{#5} & {#3}_{#6} & {#3}_{#7} & {#3}_{#8} \\
 {#4}_{#5} & {#4}_{#6} & {#4}_{#7} & {#4}_{#8} \\
\end{vmatrix}
}

%\newcommand{\DETuvwxyijklm}[10]{
%\begin{vmatrix}
% {#1}_{#6} & {#1}_{#7} & {#1}_{#8} & {#1}_{#9} & {#1}_{#10} \\
% {#2}_{#6} & {#2}_{#7} & {#2}_{#8} & {#2}_{#9} & {#2}_{#10} \\
% {#3}_{#6} & {#3}_{#7} & {#3}_{#8} & {#3}_{#9} & {#3}_{#10} \\
% {#4}_{#6} & {#4}_{#7} & {#4}_{#8} & {#4}_{#9} & {#4}_{#10} \\
% {#5}_{#6} & {#5}_{#7} & {#5}_{#8} & {#5}_{#9} & {#5}_{#10}
%\end{vmatrix}
%}

% R3 vector.
\newcommand{\VectorThree}[3]{
\begin{bmatrix}
 {#1} \\
 {#2} \\
 {#3}
\end{bmatrix}
}



\author{Peeter Joot}
\email{peeter.joot@gmail.com}

%\documentclass[]{eliblogwidescreen}

\usepackage{amsmath}
\usepackage{mathpazo}

%
% shorthand for bold symbols, convenient for vectors and matrices
%
\newcommand{\Ba}[0]{\mathbf{a}}
\newcommand{\Bb}[0]{\mathbf{b}}
\newcommand{\Bc}[0]{\mathbf{c}}
\newcommand{\Bd}[0]{\mathbf{d}}
\newcommand{\Be}[0]{\mathbf{e}}
\newcommand{\Bf}[0]{\mathbf{f}}
\newcommand{\Bg}[0]{\mathbf{g}}
\newcommand{\Bh}[0]{\mathbf{h}}
\newcommand{\Bi}[0]{\mathbf{i}}
\newcommand{\Bj}[0]{\mathbf{j}}
\newcommand{\Bk}[0]{\mathbf{k}}
\newcommand{\Bl}[0]{\mathbf{l}}
\newcommand{\Bm}[0]{\mathbf{m}}
\newcommand{\Bn}[0]{\mathbf{n}}
\newcommand{\Bo}[0]{\mathbf{o}}
\newcommand{\Bp}[0]{\mathbf{p}}
\newcommand{\Bq}[0]{\mathbf{q}}
\newcommand{\Br}[0]{\mathbf{r}}
\newcommand{\Bs}[0]{\mathbf{s}}
\newcommand{\Bt}[0]{\mathbf{t}}
\newcommand{\Bu}[0]{\mathbf{u}}
\newcommand{\Bv}[0]{\mathbf{v}}
\newcommand{\Bw}[0]{\mathbf{w}}
\newcommand{\Bx}[0]{\mathbf{x}}
\newcommand{\By}[0]{\mathbf{y}}
\newcommand{\Bz}[0]{\mathbf{z}}
\newcommand{\BA}[0]{\mathbf{A}}
\newcommand{\BB}[0]{\mathbf{B}}
\newcommand{\BC}[0]{\mathbf{C}}
\newcommand{\BD}[0]{\mathbf{D}}
\newcommand{\BE}[0]{\mathbf{E}}
\newcommand{\BF}[0]{\mathbf{F}}
\newcommand{\BG}[0]{\mathbf{G}}
\newcommand{\BH}[0]{\mathbf{H}}
\newcommand{\BI}[0]{\mathbf{I}}
\newcommand{\BJ}[0]{\mathbf{J}}
\newcommand{\BK}[0]{\mathbf{K}}
\newcommand{\BL}[0]{\mathbf{L}}
\newcommand{\BM}[0]{\mathbf{M}}
\newcommand{\BN}[0]{\mathbf{N}}
\newcommand{\BO}[0]{\mathbf{O}}
\newcommand{\BP}[0]{\mathbf{P}}
\newcommand{\BQ}[0]{\mathbf{Q}}
\newcommand{\BR}[0]{\mathbf{R}}
\newcommand{\BS}[0]{\mathbf{S}}
\newcommand{\BT}[0]{\mathbf{T}}
\newcommand{\BU}[0]{\mathbf{U}}
\newcommand{\BV}[0]{\mathbf{V}}
\newcommand{\BW}[0]{\mathbf{W}}
\newcommand{\BX}[0]{\mathbf{X}}
\newcommand{\BY}[0]{\mathbf{Y}}
\newcommand{\BZ}[0]{\mathbf{Z}}

\newcommand{\Bzero}[0]{\mathbf{0}}
\newcommand{\Btheta}[0]{\boldsymbol{\theta}}
\newcommand{\Btau}[0]{\boldsymbol{\tau}}
\newcommand{\Bomega}[0]{\boldsymbol{\omega}}

%
% shorthand for unit vectors
%
\newcommand{\acap}[0]{\hat{\Ba}}
\newcommand{\bcap}[0]{\hat{\Bb}}
\newcommand{\ccap}[0]{\hat{\Bc}}
\newcommand{\dcap}[0]{\hat{\Bd}}
\newcommand{\ecap}[0]{\hat{\Be}}
\newcommand{\fcap}[0]{\hat{\Bf}}
\newcommand{\gcap}[0]{\hat{\Bg}}
\newcommand{\hcap}[0]{\hat{\Bh}}
\newcommand{\icap}[0]{\hat{\Bi}}
\newcommand{\jcap}[0]{\hat{\Bj}}
\newcommand{\kcap}[0]{\hat{\Bk}}
\newcommand{\lcap}[0]{\hat{\Bl}}
\newcommand{\mcap}[0]{\hat{\Bm}}
\newcommand{\ncap}[0]{\hat{\Bn}}
\newcommand{\ocap}[0]{\hat{\Bo}}
\newcommand{\pcap}[0]{\hat{\Bp}}
\newcommand{\qcap}[0]{\hat{\Bq}}
\newcommand{\rcap}[0]{\hat{\Br}}
\newcommand{\scap}[0]{\hat{\Bs}}
\newcommand{\tcap}[0]{\hat{\Bt}}
\newcommand{\ucap}[0]{\hat{\Bu}}
\newcommand{\vcap}[0]{\hat{\Bv}}
\newcommand{\wcap}[0]{\hat{\Bw}}
\newcommand{\xcap}[0]{\hat{\Bx}}
\newcommand{\ycap}[0]{\hat{\By}}
\newcommand{\zcap}[0]{\hat{\Bz}}
\newcommand{\thetacap}[0]{\hat{\Btheta}}

%
% to write R^n and C^n in a distinguishable fashion.  Perhaps change this
% to the double lined characters upon figuring out how to do so.
%
\newcommand{\C}[1]{$\mathbb{C}^{#1}$}
\newcommand{\R}[1]{$\mathbb{R}^{#1}$}

%
% various generally useful helpers
%

% derivative of #1 wrt. #2:
\newcommand{\D}[2] {\frac {d#2} {d#1}}

\newcommand{\inv}[1]{\frac{1}{#1}}
\newcommand{\cross}[0]{\times}

\newcommand{\abs}[1]{\lvert{#1}\rvert}
\newcommand{\norm}[1]{\lVert{#1}\rVert}
\newcommand{\innerprod}[2]{\langle{#1}, {#2}\rangle}
\newcommand{\dotprod}[2]{{#1} \cdot {#2}}
\newcommand{\bdotprod}[2]{\left({#1} \cdot {#2}\right)}
\newcommand{\crossprod}[2]{{#1} \cross {#2}}
\newcommand{\tripleprod}[3]{\dotprod{\left(\crossprod{#1}{#2}\right)}{#3}}

\DeclareMathOperator{\Proj}{Proj}
\DeclareMathOperator{\Span}{span}
\DeclareMathOperator{\Sgn}{sgn}
\DeclareMathOperator{\Area}{Area}
\DeclareMathOperator{\Volume}{Volume}

%
% A few miscellaneous things specific to this document
%
\newcommand{\crossop}[1]{\crossprod{#1}{}}

% R2 vector.
\newcommand{\VectorTwo}[2]{
\begin{bmatrix}
 {#1} \\
 {#2}
\end{bmatrix}
}

\newcommand{\VectorN}[1]{
\begin{bmatrix}
{#1}_1 \\
{#1}_2 \\
\vdots \\
{#1}_N \\
\end{bmatrix}
}

\newcommand{\DETuvij}[4]{
\begin{vmatrix}
 {#1}_{#3} & {#1}_{#4} \\
 {#2}_{#3} & {#2}_{#4}
\end{vmatrix}
}

\newcommand{\DETuvwijk}[6]{
\begin{vmatrix}
 {#1}_{#4} & {#1}_{#5} & {#1}_{#6} \\
 {#2}_{#4} & {#2}_{#5} & {#2}_{#6} \\
 {#3}_{#4} & {#3}_{#5} & {#3}_{#6}
\end{vmatrix}
}

\newcommand{\DETuvwxijkl}[8]{
\begin{vmatrix}
 {#1}_{#5} & {#1}_{#6} & {#1}_{#7} & {#1}_{#8} \\
 {#2}_{#5} & {#2}_{#6} & {#2}_{#7} & {#2}_{#8} \\
 {#3}_{#5} & {#3}_{#6} & {#3}_{#7} & {#3}_{#8} \\
 {#4}_{#5} & {#4}_{#6} & {#4}_{#7} & {#4}_{#8} \\
\end{vmatrix}
}

%\newcommand{\DETuvwxyijklm}[10]{
%\begin{vmatrix}
% {#1}_{#6} & {#1}_{#7} & {#1}_{#8} & {#1}_{#9} & {#1}_{#10} \\
% {#2}_{#6} & {#2}_{#7} & {#2}_{#8} & {#2}_{#9} & {#2}_{#10} \\
% {#3}_{#6} & {#3}_{#7} & {#3}_{#8} & {#3}_{#9} & {#3}_{#10} \\
% {#4}_{#6} & {#4}_{#7} & {#4}_{#8} & {#4}_{#9} & {#4}_{#10} \\
% {#5}_{#6} & {#5}_{#7} & {#5}_{#8} & {#5}_{#9} & {#5}_{#10}
%\end{vmatrix}
%}

% R3 vector.
\newcommand{\VectorThree}[3]{
\begin{bmatrix}
 {#1} \\
 {#2} \\
 {#3}
\end{bmatrix}
}



\author{Peeter Joot}
\email{peeter.joot@gmail.com}


%\usepackage{media9}
\chapter{Continuum mechanics fluids review.}

\label{chap:continuumFluidsReview}
%\useCCL
\blogpage{http://sites.google.com/site/peeterjoot2/math2012/continuumFluidsReview.pdf}
\date{Apr 21, 2012}
\gitRevisionInfo{continuumFluidsReview}
\keywords{PHY454H1S, PHY454H1, strain, displacement vector, stress, constitutive relation, Navier-Stokes, incompressible fluid}
%The text introduces the capillary constant
%\section{Surface tension.  Laplace pressure.}
%\section{Hydrostatics.}
%\section{Mass conservation through aperatures.}
%\section{Non-dimensionality and scaling.}
%\section{Reynold's number.}
%\section{Boundary layers.}
%\subsection{Impulsive flow.}
%\subsection{Oscillatory flow.}
%\subsection{Universal behaviour.}
%\subsection{Bernoulli equation.}
%\subsection{Blassius problem (boundary layer thickness in flow over plate).}
%\section{Singular pertubation theory.}
%\section{Stability.}
%\subsection{Thermal stability: Rayleigh-Benard problem.}
%\section{Relative change in volume}
%\section{Conservation of mass}
%\section{Constitutive relation}
%\section{Conservation of momentum (Navier-Stokes)}
%\section{No slip condition}
%\section{Traction vector matching at an interface.}
%\subsection{Flux}
%\begin{itemize}
%\item channel flow with externl pressure gradient
%\item shear flow
%\item pipe (Poiseuille) flow
%\item steady state gravity driven film flow down a slope.
%\end{itemize}

\beginArtWithToc
%\beginArtNoToc
%\wordpresscategory{}

\section{Motivation.}

Review of key ideas and equations from the fluid dynamics portion of the class.

\section{Vector displacements.}

Those portions of the theory of elasticity that we did cover have the appearance of providing some logical context for the derivation of the Navier-Stokes equation.  Our starting point is almost identical, but we now look at displacements that vary with time, forming

\begin{equation}\label{eqn:continuumFluidsReview:830}
d\Bx' = d\Bx + d\Bu \delta t.
\end{equation}

We compute a first order Taylor expansion of this differential, defining a symmetric strain and antisymmetric vorticity tensor

\begin{subequations}
\begin{equation}\label{eqn:continuumFluidsReview:850}
e_{ij} = \inv{2} \left(
\PD{x_j}{u_i} +
\PD{x_i}{u_j} \right).
\end{equation}
\begin{equation}\label{eqn:continuumFluidsReview:870}
\omega_{ij} = \inv{2} \left(
\PD{x_j}{u_i}
-\PD{x_i}{u_j} \right)
\end{equation}
\end{subequations}

Allowing us to write

\begin{equation}\label{eqn:continuumFluidsReview:890}
dx_i' = dx_i + e_{ij} dx_j \delta t + \omega_{ij} dx_j \delta t.
\end{equation}

We introduced vector and dual vector forms of the vorticity tensor with

\begin{subequations}
\begin{equation}\label{eqn:continuumFluidsReview:970}
\Omega_k = \inv{2} \partial_i u_j \epsilon_{i j k}
\end{equation}
\begin{equation}\label{eqn:continuumFluidsReview:990}
\omega_{i j} = -\Omega_k \epsilon_{i j k},
\end{equation}
\end{subequations}

or

\begin{subequations}
\begin{equation}\label{eqn:continuumFluidsReview:910}
\Bomega = \spacegrad \cross \Bu
\end{equation}
\begin{equation}\label{eqn:continuumFluidsReview:930}
\BOmega = \inv{2} (\Bomega)_a \Be_a.
\end{equation}
\end{subequations}

We were then able to put our displacement differential into a partial vector form

\begin{equation}\label{eqn:continuumFluidsReview:950}
d\Bx' = d\Bx + \left( \Be_i (e_{ij} \Be_j) \cdot d\Bx + \BOmega \cross d\Bx \right) \delta t.
\end{equation}

\section{Relative change in volume}

We are able to identity the divergence of the displacement as the relative change in volume per unit time in terms of the strain tensor trace (in the basis for which the strain is diagonal at a given point)

\begin{equation}\label{eqn:continuumFluidsReview:1010}
\frac{dV' - dV}{dV \delta t} = \spacegrad \cdot \Bu.
\end{equation}

\section{Conservation of mass}

Utilizing Green's theorem we argued that 

\begin{equation}\label{eqn:continuumFluidsReview:1030}
\int \left( \PD{t}{\rho} + \spacegrad \cdot (\rho \Bu) \right) dV = 0.
\end{equation}

We were able to relate this to rate of change of density, computing

\begin{equation}\label{eqn:continuumFluidsReview:1050}
\frac{d\rho}{dt} = \PD{t}{\rho} + \Bu \cdot \spacegrad \rho =
- \rho \spacegrad \cdot \Bu.
\end{equation}

An important consequence of this is that for incompressible fluids (the only types of fluids considered in this course) the divergence of the displacement $\spacegrad \cdot \Bu = 0$.

\section{Constitutive relation}

We consider only Newtonian fluids, for which the stress is linearly related to the strain.  We will model fluids as disjoint sets of hydrostatic materials for which the constitutive relation was previously found to be

\begin{equation}\label{eqn:continuumFluidsReview:1070}
\sigma_{ij} = - p \delta_{ij} + 2 \mu e_{ij}.
\end{equation}

\section{Conservation of momentum (Navier-Stokes)}

As in elasticity, our momentum conservation equation had the form

\begin{equation}\label{eqn:continuumFluidsReview:1090}
\rho \frac{du_i}{dt} = \PD{x_j}{\sigma_{ij}} + \rho f_i,
\end{equation}

where $f_i$ are the components of the external (body) forces per unit volume acting on the fluid.

Utilizing the constitutive relation and explicitly evaluating the stress tensor divergence $\PDi{x_j}{\sigma_{ij}}$ we find

\begin{equation}\label{eqn:continuumFluidsReview:1110}
\rho \frac{d\Bu}{dt} 
=
\rho \PD{t}{\Bu} + \rho (\Bu \cdot \spacegrad) \Bu
= -\spacegrad p + \mu \spacegrad^2 \Bu
+ \mu \spacegrad (\spacegrad \cdot \Bu) + \rho \Bf.
\end{equation}

Since we treat only incompressible fluids in this course we can decompose this into a pair of equations

\begin{subequations}
\begin{equation}\label{eqn:continuumFluidsReview:1130}
\rho \PD{t}{\Bu} + \rho (\Bu \cdot \spacegrad) \Bu
= -\spacegrad p 
+ \mu \spacegrad^2 \Bu
+ \rho \Bf.
\end{equation}
\begin{equation}\label{eqn:continuumFluidsReview:1150}
\spacegrad \cdot \Bu = 0
\end{equation}
\end{subequations}

\section{No slip condition}

We'll find in general that we have to solve for our boundary value conditions.  One of the important constraints that we have to do so will be a requirement (experimentally motivated) that our velocities match at an interface.  This was illustrated with a rocker tank video in class.

This is the no-slip condition, and includes a requirement that the fluid velocity at the boundary of a non-moving surface is zero, and that the fluid velocity on the boundary of a moving surface matches the rate of the surface itself.

For fluids $A$ and $B$ separated at an interface with unit normal $\ncap$ and unit tangent $\taucap$ we wrote the no-slip condition as

\begin{subequations}
\begin{equation}\label{eqn:continuumFluidsReview:1170}
\Bu_A \cdot \taucap = \Bu_B \cdot \taucap
\end{equation}
\begin{equation}\label{eqn:continuumFluidsReview:1190}
\Bu_A \cdot \ncap = \Bu_B \cdot \ncap.
\end{equation}
\end{subequations}

For the problems we attempt, it will often be enough to consider only the tangential component of the velocity.

\section{Traction vector matching at an interface.}

As well as matching velocities, we have a force balance requirement at any interface.  This will be expressed in terms of the traction vector

\begin{equation}\label{eqn:continuumFluidsReview:1210}
\BT = \Be_i \sigma_{ij} n_j = \Bsigma \cdot \ncap
\end{equation}

where $\ncap = n_j \Be_i$ is the normal pointing from the interface into the fluid (so the traction vector represents the force of the interface on the fluid).  When that interface is another fluid, we are able to calculate the force of one fluid on the other.

In addition the the constraints provided by the no-slip condition, we'll often have to constrain our solutions according to the equality of the tangential components of the traction vector

\begin{equation}\label{eqn:continuumFluidsReview:1230}
\evalbar{\tau_i (\sigma_{ij} n_j)}{A} =
\evalbar{\tau_i (\sigma_{ij} n_j)}{B},
\end{equation}

We'll sometimes also have to consider, especially when solving for the pressure, the force balance for the normal component of the traction vector at the interface too

\begin{equation}\label{eqn:continuumFluidsReview:1250}
\evalbar{n_i (\sigma_{ij} n_j)}{A} =
\evalbar{n_i (\sigma_{ij} n_j)}{B}.
\end{equation}

As well as having a messy non-linear PDE to start with, our boundary value constraints can be very complicated, making the subject rich and tricky.

\subsection{Flux}

A number of problems we did asked for the flux rate.  A slightly more sensible physical quantity is the mass flux, which adds the density into the mix

\begin{equation}\label{eqn:continuumFluidsReview:1450}
\int \frac{dm}{dt} = \rho \int \frac{dV}{dt} = \rho \int (\Bu \cdot \ncap) dA
\end{equation}

\section{Worked problems from class.}

A number of problems were tackled in class

\begin{itemize}
\item channel flow with external pressure gradient
\item shear flow
\item pipe (Poiseuille) flow
\item steady state gravity driven film flow down a slope.
\end{itemize}

\subsection{Channel and shear flow}

An example that shows many of the features of the above problems is rectilinear flow problem with a pressure gradient and shearing surface.  As a review let's consider fluid flowing between surfaces at $z = \pm h$, the lower surface moving at velocity $v$ and pressure gradient $dp/dx = -G$ we find that Navier-Stokes for an assumed flow of $\Bu = u(z) \xcap$ takes the form

\begin{align}\label{eqn:continuumFluidsReview:1270}
0 &= \partial_x u + \partial_y (0) + \partial_z (0) \\
u \cancel{\partial_x u} &= - \partial_x p + \mu \partial_{zz} u \\
0 &= -\partial_y p \\
0 &= -\partial_z p
\end{align}

We find that this reduces to 

\begin{equation}\label{eqn:continuumFluidsReview:1290}
\frac{d^2 u}{dz^2} = -\frac{G}{\mu}
\end{equation}

with solution

\begin{equation}\label{eqn:continuumFluidsReview:1310}
u(z) = \frac{G}{2\mu}(h^2 - z^2) + A (z + h) + B.
\end{equation}

Application of the no-slip velocity matching constraint gives us in short order

\begin{equation}\label{eqn:continuumFluidsReview:1330}
u(z) = \frac{G}{2\mu}(h^2 - z^2) + v \left( 1 - \inv{2h} (z + h) \right).
\end{equation}

With $v = 0$ this is the channel flow solution, and with $G = 0$ this is the shearing flow solution.

Having solved for the velocity at any height, we can also solve for the mass or volume flux through a slice of the channel.  For the mass flux $\rho Q$ per unit time (given volume flux $Q$)

\begin{equation}\label{eqn:continuumFluidsReview:1350}
\int \frac{dm}{dt} 
=
\int \rho \frac{dV}{dt} 
=
\rho (\Delta A) \int \Bu \cdot \taucap,
\end{equation}

we find 

\begin{equation}\label{eqn:continuumFluidsReview:1370}
\rho Q =
\rho (\Delta y) \left( 
\frac{2 G h^3}{3 \mu} + h v
\right).
\end{equation}

We can also calculate the force of the boundaries on the fluid.  For example, the force per unit volume of the boundary at $z = \pm h$ on the fluid is found by calculating the tangential component of the traction vector taken with normal $\ncap = \mp \zcap$.  That tangent vector is found to be

\begin{equation}\label{eqn:continuumFluidsReview:1390}
\Bsigma \cdot (\pm \ncap) = -p \zcap \pm 2 \mu \Be_i e_{ij} \delta_{j 3} = - p \zcap \pm \xcap \mu \PD{z}{u}.
\end{equation}

The tangential component is the $\xcap$ component evaluated at $z = \pm h$, so for the lower and upper interfaces we have

\begin{align}\label{eqn:continuumFluidsReview:1410}
\evalbar{(\Bsigma \cdot \ncap) \cdot \xcap}{z = -h} &= -G (-h) - \frac{v \mu}{2 h} \\
\evalbar{(\Bsigma \cdot -\ncap) \cdot \xcap}{z = +h} &= -G (+h) + \frac{v \mu}{2 h},
\end{align}

so the force per unit area that the boundary applies to the fluid is

\begin{align}\label{eqn:continuumFluidsReview:1430}
\text{force per unit length of lower interface on fluid} &= L \left( G h - \frac{v \mu}{2 h} \right) \\
\text{force per unit length of upper interface on fluid} 
&= L \left( -G h + \frac{v \mu}{2 h} \right).
\end{align}

Does the sign of the velocity term make sense?  Let's consider the case where we have a zero pressure gradient and look at the lower interface.  This is the force of the interface on the fluid, so the force of the fluid on the interface would have the opposite sign

\begin{equation}\label{eqn:continuumFluidsReview:1470}
\frac{v \mu}{2 h}.
\end{equation}

This does seem reasonable.  Our fluid flowing along with a positive velocity is imparting a force on what it is flowing over in the same direction.

\section{Hydrostatics.}
\section{Mass conservation through apertures.}
\section{Surface tension.  Laplace pressure.}

We covered some aspects of this topic in class.  \S 2.4.9-2.4.10 of \cite{granger1995fluid} also has a nice treatment.  Also recommended in the text is the ``Surface Tension in Fluid Mechanics'' movie which can be found on youtube in three parts \youtubehref{DkEhPltiqmo}, \youtubehref{yiixltf\_HKw}, \youtubehref{5d6efCcwkWs}.

The first of these films points out that the surface tension equation we were shown in class

\begin{equation}\label{eqn:continuumFluidsReview:1490}
\Delta p = \frac{2 \sigma}{R},
\end{equation}

is only for spherical objects that have a single radius of curvature.  This formula can in fact be derived with a simple physical argument, stating that the force generated by the surface tension $\sigma$ along the equator of a bubble (as in \ref{fig:continuumFluidsReview:continuumFluidsReviewFig1}), in a fluid would be balanced by the difference in pressure times the area of that equatorial cross section.  That is

\imageFigure{continuumFluidsReviewFig1}{Spherical bubble in liquid.}{fig:continuumFluidsReview:continuumFluidsReviewFig1}{0.2}

\begin{equation}\label{eqn:continuumFluidsReview:1510}
\sigma 2 \pi R = \Delta p \pi R^2
\end{equation}

Observe that we obtain \ref{eqn:continuumFluidsReview:1490} after dividing through by the area.

\subsection{A sample problem.}

To get a better feeling for this, let's look to a worked problem.  The most obvious one to try to attempt is the shape of a meniscus of water against a wall.  This problem is worked in \cite{landau1987course}, but it is worth some extra notes.  As in the text we'll work with $z$ axis up, and the fluid up against a wall at $x = 0$ as illustrated in figure (\ref{fig:continuumFluidsReview:continuumFluidsReviewFig2}).

\imageFigure{continuumFluidsReviewFig2}{Curvature of fluid against a wall.}{fig:continuumFluidsReview:continuumFluidsReviewFig2}{0.3}

The starting point is a variation of what we have in class

\begin{equation}\label{eqn:continuumFluidsReview:1530}
p_1 - p_2 = \sigma \left( \inv{R_1} + \inv{R_2} \right),
\end{equation}

where $p_2$ is the atmospheric pressure, $p_1$ is the fluid pressure, and the (signed!) radius of curvatures positive if pointing into medium 1 (the fluid).

For fluid at rest, Navier-Stokes takes the form

\begin{equation}\label{eqn:continuumFluidsReview:1630}
0 = -\spacegrad p_1 + \rho \Bg.
\end{equation}

With $\Bg = -g \zcap$ we have

\begin{equation}\label{eqn:continuumFluidsReview:1650}
0 = -\PD{z}{p_1} - \rho g,
\end{equation}

or

\begin{equation}\label{eqn:continuumFluidsReview:1670}
p_1 = \text{constant} - \rho g z.
\end{equation}

%At a height $z$ from the base of the surface (i.e. the bottom of the meniscous), our pressure is 
%
%\begin{equation}\label{eqn:continuumFluidsReview:1550}
%p_1 = p_a + \rho g z.
%\end{equation}

We have $p_2 = p_a$, the atmospheric pressure, so our pressure difference is

\begin{equation}\label{eqn:continuumFluidsReview:1570}
p_1 - p_2 = \text{constant} - \rho g z.
\end{equation}

We have then

\begin{equation}\label{eqn:continuumFluidsReview:1590}
\text{constant} -\frac{\rho g z}{\sigma} = \inv{R_1} + \inv{R_2}.
\end{equation}

One of our axis of curvature directions is directly along the $y$ axis so that curvature is zero $1/R_1 = 0$.  We can fix the constant by noting that at $x = \infty$, $z = 0$, we have no curvature $1/R_2 = 0$.  This gives

\begin{equation}\label{eqn:continuumFluidsReview:1690}
\text{constant} -0 = 0 + 0.
\end{equation}

That leaves just the second curvature to determine.  For a curve $z = z(x)$ our absolute curvature, according to \cite{wiki:curvature} is

\begin{equation}\label{eqn:continuumFluidsReview:1610}
\Abs{\inv{R_2}} = \frac{\Abs{z''}}{(1 + (z')^2)^{3/2}}.
\end{equation}

Now we have to fix the sign.  I didn't recall any sort of notion of a signed radius of curvature, but there's a blurb about it on the curvature article above, including a nice illustration of signed radius of curvatures can be found in this \href{http://goo.gl/Wqzz2}{wikipedia radius of curvature figure for a Lemniscate}.  Following that definition for a curve such as $z(x) = (1-x)^2$ we'd have a positive curvature, but the text explicitly points out that the curvatures are will be set positive if pointing into the medium.  For us to point the normal into the medium as in the figure, we have to invert the sign, so our equation to solve for $z$ is given by
% shorten to get rid of _ that latex doesn't like.
%http://upload.wikimedia.org/wikipedia/commons/b/b2/Lemniscate_nebeneinander_animated.gif

\begin{equation}\label{eqn:continuumFluidsReview:1710}
-\frac{\rho g z}{\sigma} = -\frac{z''}{(1 + (z')^2)^{3/2}}.
\end{equation}

The text introduces the capillary constant

\begin{equation}\label{eqn:continuumFluidsReview:1730}
a = \sqrt{2 \sigma/ g \rho}.
\end{equation}

Using that capillary constant $a$ to tidy up a bit and multiplying by a $z'$ integrating factor we have

\begin{equation}\label{eqn:continuumFluidsReview:1750}
-\frac{2 z z'}{a^2} = -\frac{z'' z'}{(1 + (z')^2)^{3/2}},
\end{equation}

we can integrate to find

\begin{equation}\label{eqn:continuumFluidsReview:1770}
A - \frac{z^2}{a^2} = \frac{1}{(1 + (z')^2)^{1/2}}.
\end{equation}

Again for $x = \infty$ we have $z = 0$, $z' = 0$, so $A = 1$.  Rearranging we have

\begin{equation}\label{eqn:continuumFluidsReview:1790}
\int dx = \int dz \left( \inv{(1 - z^2/a^2)^2} - 1 \right)^{-1/2}.
\end{equation}

Integrating this with Mathematica I get

\begin{equation}\label{eqn:continuumFluidsReview:1810}
x - x_0 =
\sqrt{2 a^2-z^2} \sgn(a-z)+ \frac{a}{\sqrt{2}} \ln \left(\frac{a \left(2 a-\sqrt{4 a^2-2 z^2} \sgn(a-z)\right)}{z}\right).
\end{equation}

It looks like the constant would have to be fixed numerically.  We require at $x = 0$

\begin{equation}\label{eqn:continuumFluidsReview:1830}
z'(0) = \frac{-\cos\theta}{\sin\theta} = -\cot \theta,
\end{equation}

but we don't have an explicit function for $z$.

\section{Non-dimensionality and scaling.}
\section{Reynold's number.}
\section{Boundary layers.}
\subsection{Impulsive flow.}
\subsection{Oscillatory flow.}
\subsection{Universal behavior.}
\subsection{Bernoulli equation.}
\subsection{Blassius problem (boundary layer thickness in flow over plate).}
\section{Singular perturbation theory.}
\section{Stability.}
\subsection{Thermal stability: Rayleigh-Benard problem.}

\EndArticle
