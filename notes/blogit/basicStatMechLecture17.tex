%
% Copyright � 2013 Peeter Joot.  All Rights Reserved.
% Licenced as described in the file LICENSE under the root directory of this GIT repository.
%
\newcommand{\authorname}{Peeter Joot}
\newcommand{\email}{peeterjoot@protonmail.com}
\newcommand{\basename}{FIXMEbasenameUndefined}
\newcommand{\dirname}{notes/FIXMEdirnameUndefined/}

\renewcommand{\basename}{basicStatMechLecture17}
\renewcommand{\dirname}{notes/phy452/}
\newcommand{\keywords}{Statistical mechanics, PHY452H1S}
\newcommand{\authorname}{Peeter Joot}
\newcommand{\onlineurl}{http://sites.google.com/site/peeterjoot2/math2013/\basename.pdf}
\newcommand{\sourcepath}{\dirname\basename.tex}
\newcommand{\generatetitle}[1]{\chapter{#1}}

\newcommand{\vcsinfo}{%
\section*{}
\noindent{\color{DarkOliveGreen}{\rule{\linewidth}{0.1mm}}}
\paragraph{Document version}
%\paragraph{\color{Maroon}{Document version}}
{
\small
\begin{itemize}
\item Available online at:\\ 
\href{\onlineurl}{\onlineurl}
\item Git Repository: \input{./.revinfo/gitRepo.tex}
\item Source: \sourcepath
\item last commit: \input{./.revinfo/gitCommitString.tex}
\item commit date: \input{./.revinfo/gitCommitDate.tex}
\end{itemize}
}
}

%\PassOptionsToPackage{dvipsnames,svgnames}{xcolor}
\PassOptionsToPackage{square,numbers}{natbib}
\documentclass{scrreprt}

\usepackage[left=2cm,right=2cm]{geometry}
\usepackage[svgnames]{xcolor}
\usepackage{peeters_layout}

\usepackage{natbib}

\usepackage[
colorlinks=true,
bookmarks=false,
pdfauthor={\authorname, \email},
backref 
]{hyperref}

% http://tex.stackexchange.com/questions/75773/how-to-reference-problems-by-the-text-label-in-an-exercise-envioronment
\usepackage[english]{cleveref}
\crefname{Exercise}{exercise}{exercises}
\Crefname{Exercise}{Exercise}{Exercises}

\RequirePackage{titlesec}
\RequirePackage{ifthen}

% http://stackoverflow.com/questions/4932910/date-in-the-tabular-environment
\makeatletter
\let\insertdate\@date
\makeatother

\titleformat{\chapter}[display]
{\bfseries\Large}
{\color{DarkSlateGrey}\filleft \authorname
\ifthenelse{\isundefined{\studentnumber}}{}{\\ \studentnumber}
\ifthenelse{\isundefined{\email}}{}{\\ \email}
\ifthenelse{\isundefined{\dateintitle}}{}{\\ \insertdate}
%\ifthenelse{\isundefined{\coursename}}{}{\\ \coursename} % put in title instead.
}
{4ex}
{\color{DarkOliveGreen}{\titlerule}\color{Maroon}
\vspace{2ex}%
\filright}
[\vspace{2ex}%
\color{DarkOliveGreen}\titlerule
]

\newcommand{\beginArtWithToc}[0]{\begin{document}\tableofcontents}
\newcommand{\beginArtNoToc}[0]{\begin{document}}
\newcommand{\EndNoBibArticle}[0]{\end{document}}
\newcommand{\EndArticle}[0]{\bibliography{Bibliography}\bibliographystyle{plainnat}\end{document}}

% 
%\newcommand{\citep}[1]{\cite{#1}}

\colorSectionsForArticle



\beginArtNoToc
\generatetitle{PHY452H1S Basic Statistical Mechanics.  Lecture 17: Fermi gas thermodynamics.  Taught by Prof.\ Arun Paramekanti}
%\chapter{Fermi gas thermodynamics}
\label{chap:basicStatMechLecture17}

%\section{Disclaimer}
%
%Peeter's lecture notes from class.  May not be entirely coherent.
%
%\section{Fermi gas thermodynamics}
%
%\begin{enumerate}
%\item Energy was found to be
%
%\begin{dmath}\label{eqn:basicStatMechLecture17:20}
%\frac{E}{N} = \frac{3}{5} \epsilon_{\mathrm{F}}\qquad T = 0.
%\end{dmath}
%
%\item Pressure was found to have the form
%
%F1
%
%\item The chemical potential was found to have the form
%
%F2
%
%We found that 
%
%\begin{subequations}
%\begin{dmath}\label{eqn:basicStatMechLecture17:40}
%e^{\beta \mu} = \rho \lambda_T^3,
%\end{dmath}
%\begin{dmath}\label{eqn:basicStatMechLecture17:60}
%\lambda_T = \frac{h}{\sqrt{ 2 \pi m \kB T}}
%\end{dmath}
%\end{subequations}
%
%so that the zero crossing is approximately when 
%
%\begin{subequations}
%e^{\beta \times 0} = 1 = \rho \lambda_T^3,
%\end{subequations}
%
%which gives $T \sim T_{\mathrm{F}}$.
%
%\end{enumerate}
%
%\paragraph{How about at other temperatures?}
%
%\begin{enumerate}
%\item $\mu(T) = ?$
%\item $E(T) = ?$
%\item $\CV(T) = ?$
%\end{enumerate}
%
%We had
%\begin{dmath}\label{eqn:basicStatMechLecture17:80}
%N = \sum_k \inv{e^{\beta (\epsilon_k - \mu)} + 1 = \sum_{\Bk} n_{\mathrm{F}}(\epsilon_\Bk).
%\end{dmath}
%\begin{dmath}\label{eqn:basicStatMechLecture17:100}
%E(T) = 
%\sum_k \epsilon_\Bk n_{\mathrm{F}}(\epsilon_\Bk).
%\end{dmath}
%
%We can define a density of states 
%
%\begin{dmath}\label{eqn:basicStatMechLecture17:120}
%\sum_\Bk 
%= \sum_\Bk \int_{-\infty}^\infty d\epsilon \delta(\epsilon - \epsilon_\Bk)
%= 
%\int_{-\infty}^\infty d\epsilon 
%\sum_\Bk 
%\delta(\epsilon - \epsilon_\Bk),
%\end{dmath}
%
%where the liberty to informally switch the order of differentiation and integration has been used.  This construction allows us to write a more general sum
%
%XX
\begin{dmath}\label{eqn:basicStatMechLecture17:140}
\sum_\Bk f(\epsilon_\Bk) = 
= \sum_\Bk \int_{-\infty}^\infty d\epsilon \delta(\epsilon - \epsilon_\Bk) i
f(\epsilon_\Bk)
= 
\sum_\Bk 
\int_{-\infty}^\infty d\epsilon 
\delta(\epsilon - \epsilon_\Bk)
f(\epsilon_\Bk)
= 
\int_{-\infty}^\infty d\epsilon 
f(\epsilon_\Bk)
\lr{
\sum_\Bk 
\delta(\epsilon - \epsilon_\Bk)
}
\end{dmath}

\begin{dmath}\label{eqn:basicStatMechLecture17:160}
N(\epsilon) \equiv
\sum_\Bk 
\delta(\epsilon - \epsilon_\Bk)
=
\frac{V}{(2 \pi)^3} \int d^3 \Bk \delta\lr{ \epsilon - \frac{\hbar^2 k^2}{2 m}}
=
\frac{V}{(2 \pi)^3} 4 \pi \int_0^\infty k^2 dk \delta\lr{ \epsilon - \frac{\hbar^2 k^2}{2 m}}
=
\cdots
V \lr{\frac{2 m}{\hbar^2}}^{3/2} \inv{4 \pi^2} \sqrt{\epsilon}
\end{dmath}

To make that last evaluation a change of vars and $k \sim \sqrt{\epsilon}$ so that $\epsilon^{3/2} /\epsilon \sim \sqrt{\epsilon}$.  FIXME: chug through and understand this.

In \underline{2D} this would be

\begin{dmath}\label{eqn:basicStatMechLecture17:180}
N(\epsilon) \sim V \int dk k \delta \lr{ \epsilon - \frac{\hbar^2 k^2}{2m} } \sim V
\end{dmath}

and in \underline{1D}

\begin{dmath}\label{eqn:basicStatMechLecture17:200}
N(\epsilon) \sim V \int dk k \delta \lr{ \epsilon - \frac{\hbar^2 k^2}{2m} } \sim \inv{\sqrt{\epsilon}}.
\end{dmath}

\paragraph{What happens when we have linear energy momentum relationships}

\begin{dmath}\label{eqn:basicStatMechLecture17:220}
\epsilon_\Bk = v \Abs{\Bk}
\end{dmath}

At high velocities energy and momentum of particles are also of this form

\begin{dmath}\label{eqn:basicStatMechLecture17:240}
\epsilon_\Bk = \sqrt{ m_0^2 c^4 + p^2 c^2 } \sim \Abs{\Bp} c.
\end{dmath}

Another example is graphene, a carbon structure of the form 

F3

where the energy momentum is related in roughly the following form

F4

where 

\begin{dmath}\label{eqn:basicStatMechLecture17:260}
\epsilon_\Bk = \pm v_{\mathrm{F}} \Abs{\Bk}.
\end{dmath}

Continuing with the \underline{3D} case we have

\begin{dmath}\label{eqn:basicStatMechLecture17:280}
N = V \int_0^\infty 
\mathLabelBox{
n_{\mathrm{F}}(\epsilon)
}{$1/(e^{\beta (\epsilon - \mu)} + 1)$}
\mathLabelBox
[
   labelstyle={below of=m\themathLableNode, below of=m\themathLableNode}
]
{
N(\epsilon)
}{
$\epsilon^{1/2}$
}
\end{dmath}

\begin{dmath}\label{eqn:basicStatMechLecture17:300}
\rho
=
\frac{N}{V}
=
\lr{ \frac{2m}{\hbar^2 } }
^{3/2} \inv{ 4 \pi^2} 
\int_0^\infty d\epsilon \frac{\epsilon^{1/2}}{z^{-1} e^{\beta \epsilon} + 1}
=
\lr{ \frac{2m}{\hbar^2 } }
^{3/2} \inv{ 4 \pi^2} 
\lr{\kB T}
^{3/2}
\int_0^\infty dx \frac{x^{1/2}}{z^{-1} e^{x} + 1}
\end{dmath}

where $z = e^{\beta \mu}$ as usual, and we write $x = \beta \epsilon$.   For the low temperature aysmtotic behaviour see one of the appendices (FIXME:lookup) of \citep{pathriastatistical}.  For $z$ large it can be shown that this is

\begin{dmath}\label{eqn:basicStatMechLecture17:320}
\int_0^\infty dx \frac{x^{1/2}}{z^{-1} e^{x} + 1}
\approx
\frac{2}{3} 
\lr{\ln z}
^{3/2}
\lr{
1 + \frac{\pi^2}{8} \inv{(\ln z)^2}
},
\end{dmath}

so that

\begin{dmath}\label{eqn:basicStatMechLecture17:340}
\rho \approx
\lr{ \frac{2m}{\hbar^2 } }
^{3/2} \inv{ 4 \pi^2} 
\lr{\kB T}
^{3/2}
\frac{2}{3} 
\lr{\ln z}
^{3/2}
\lr{
1 + \frac{\pi^2}{8} \inv{(\ln z)^2}
}
=
\lr{ \frac{2m}{\hbar^2 } }
^{3/2} \inv{ 4 \pi^2} 
\frac{2}{3}
\mu^{3/2}
\lr{
1 + \frac{\pi^2}{8} \inv{(\beta \mu)^2}
}
=
\lr{ \frac{2m}{\hbar^2 } }
^{3/2} \inv{ 4 \pi^2} 
\frac{2}{3}
\mu^{3/2}
\lr{
1 + \frac{\pi^2}{8} 
\lr{ \frac{\kB T}{\mu}}^2
}
= \rho_{T = 0} 
\lr{ \frac{\mu}{ \ee_F } }
^{3/2}
\lr{
1 + \frac{\pi^2}{8} 
\lr{ \frac{\kB T}{\mu}}^2
}
\end{dmath}

Assuming a quadratic form for the chemical potential at low temperature

\begin{dmath}\label{eqn:basicStatMechLecture17:n}
1 = 
\lr{ \frac{\mu}{ \ee_F } }
^{3/2}
\lr{
1 + \frac{\pi^2}{8} 
\lr{ \frac{\kB T}{\mu}}^2
}
=
\lr{ \frac{\ee_F - a T^2}{ \ee_F } }
^{3/2}
\lr{
1 + \frac{\pi^2}{8} 
\lr{ \frac{\kB T}{\ee_F - a T^2}}^2
}
\approx
\lr{
1 - \frac{3}{2} a \frac{T^2}{\ee_F}
}
\lr{
1 + \frac{\pi^2}{8} \frac{(\kB T)^2}{\ee_F^2}
}
= 
1 - \frac{3}{2} a \frac{T^2}{\ee_F} + \frac{\pi^2}{8} \frac{(\kB T)^2}{\ee_F^2}
\end{dmath}

We have used a Taylor expansion $(1 + x)^n \approx 1 + n x$ for small $x$.  With

\begin{dmath}\label{eqn:basicStatMechLecture17:n}
a = \frac{\pi^2}{12} \frac{\kB^2}{\ee_F} 
\end{dmath}

we have 

\begin{dmath}\label{eqn:basicStatMechLecture17:n}
\mu = \ee_F - \frac{\pi^2}{12} \frac{(\kB T)^2}{\ee_F}
\end{dmath}

\EndArticle
