\makeproblem{Polymer stretching - ``entropic forces''}{basicStatMech:problemSet5:1}{
%(3 points)
Consider a toy model of a polymer in one dimension which is made of $N$ steps (amino acids) of unit length, going left or right like a random walk. Let one end of this polymer be at the origin and the other end be at a point $X = \sqrt{N}$ (viz. the rms size of the polymer) , so $1 \ll X \ll N$. We have previously calculated the number of configurations corresponding to this condition (approximate the binomial distribution by a Gaussian). 

\makesubproblem{}{pr:basicStatMechProblemSet5Problem1:a}
Using this, find the entropy of this polymer as $S = k_{\mathrm{B}} \ln \Omega$. The free energy of this polymer, even in the absence of any other interactions, thus has an entropic contribution, $F = -T S$. If we stretch this polymer, we expect to have fewer available configurations, and thus a smaller entropy and a higher free energy. 

\makesubproblem{}{pr:basicStatMechProblemSet5Problem1:b}
Find the change in free energy of this polymer if we stretch this polymer from its end being at $X$ to a larger distance $X + \Delta X$. 

\makesubproblem{}{pr:basicStatMechProblemSet5Problem1:c}
Show that the change in free energy is quadradic in the displacement for small $\Delta X$, and hence find the temperature dependent ``entropic spring constant'' of this polymer.  (This entropic force is important to overcome for packing DNA into the nucleus, and in many biological processes.)
} % makeproblem

\makeanswer{basicStatMech:problemSet5:1}{

\makeSubAnswer{}{pr:basicStatMechProblemSet5Problem1:a}
TODO.
\makeSubAnswer{}{pr:basicStatMechProblemSet5Problem1:b}
TODO.
\makeSubAnswer{}{pr:basicStatMechProblemSet5Problem1:c}
TODO.
}
