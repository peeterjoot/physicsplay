%
% Copyright � 2013 Peeter Joot.  All Rights Reserved.
% Licenced as described in the file LICENSE under the root directory of this GIT repository.
%
\newcommand{\authorname}{Peeter Joot}
\newcommand{\email}{peeterjoot@protonmail.com}
\newcommand{\basename}{FIXMEbasenameUndefined}
\newcommand{\dirname}{notes/FIXMEdirnameUndefined/}

\renewcommand{\basename}{condensedMatterLecture17}
\renewcommand{\dirname}{notes/phy487/}
\newcommand{\keywords}{Condensed matter physics, PHY487H1F}
\newcommand{\authorname}{Peeter Joot}
\newcommand{\onlineurl}{http://sites.google.com/site/peeterjoot2/math2013/\basename.pdf}
\newcommand{\sourcepath}{\dirname\basename.tex}
\newcommand{\generatetitle}[1]{\chapter{#1}}

\newcommand{\vcsinfo}{%
\section*{}
\noindent{\color{DarkOliveGreen}{\rule{\linewidth}{0.1mm}}}
\paragraph{Document version}
%\paragraph{\color{Maroon}{Document version}}
{
\small
\begin{itemize}
\item Available online at:\\ 
\href{\onlineurl}{\onlineurl}
\item Git Repository: \input{./.revinfo/gitRepo.tex}
\item Source: \sourcepath
\item last commit: \input{./.revinfo/gitCommitString.tex}
\item commit date: \input{./.revinfo/gitCommitDate.tex}
\end{itemize}
}
}

%\PassOptionsToPackage{dvipsnames,svgnames}{xcolor}
\PassOptionsToPackage{square,numbers}{natbib}
\documentclass{scrreprt}

\usepackage[left=2cm,right=2cm]{geometry}
\usepackage[svgnames]{xcolor}
\usepackage{peeters_layout}

\usepackage{natbib}

\usepackage[
colorlinks=true,
bookmarks=false,
pdfauthor={\authorname, \email},
backref 
]{hyperref}

% http://tex.stackexchange.com/questions/75773/how-to-reference-problems-by-the-text-label-in-an-exercise-envioronment
\usepackage[english]{cleveref}
\crefname{Exercise}{exercise}{exercises}
\Crefname{Exercise}{Exercise}{Exercises}

\RequirePackage{titlesec}
\RequirePackage{ifthen}

% http://stackoverflow.com/questions/4932910/date-in-the-tabular-environment
\makeatletter
\let\insertdate\@date
\makeatother

\titleformat{\chapter}[display]
{\bfseries\Large}
{\color{DarkSlateGrey}\filleft \authorname
\ifthenelse{\isundefined{\studentnumber}}{}{\\ \studentnumber}
\ifthenelse{\isundefined{\email}}{}{\\ \email}
\ifthenelse{\isundefined{\dateintitle}}{}{\\ \insertdate}
%\ifthenelse{\isundefined{\coursename}}{}{\\ \coursename} % put in title instead.
}
{4ex}
{\color{DarkOliveGreen}{\titlerule}\color{Maroon}
\vspace{2ex}%
\filright}
[\vspace{2ex}%
\color{DarkOliveGreen}\titlerule
]

\newcommand{\beginArtWithToc}[0]{\begin{document}\tableofcontents}
\newcommand{\beginArtNoToc}[0]{\begin{document}}
\newcommand{\EndNoBibArticle}[0]{\end{document}}
\newcommand{\EndArticle}[0]{\bibliography{Bibliography}\bibliographystyle{plainnat}\end{document}}

% 
%\newcommand{\citep}[1]{\cite{#1}}

\colorSectionsForArticle



%\citep{harald2003solid} \S x.y
%\citep{ibach2009solid} \S x.y

%\usepackage{mhchem}
\usepackage[version=3]{mhchem}
\newcommand{\nought}[0]{\circ}
\newcommand{\EF}[0]{\epsilon_{\mathrm{F}}}
\newcommand{\kF}[0]{k_{\mathrm{F}}}

\beginArtNoToc
\generatetitle{PHY487H1F Condensed Matter Physics.  Lecture 17: 3 dimensional band structures, Fermi surfaces of real metals.  Taught by Prof.\ Stephen Julian}
%\chapter{3 dimensional band structures, Fermi surfaces of real metals}
\label{chap:condensedMatterLecture17}

%\section{Disclaimer}
%
%Peeter's lecture notes from class.  May not be entirely coherent.

\section{3 dimensional band structures, Fermi surfaces of real metals}

READING: \citep{ibach2009solid} \S 7.4

Consider a nearly free electron metal, and a hypothetical simple cubic system

F1.  Simple cubic Brillion zone
F2.  
F3

We can ask some questions

\begin{itemize}
\item 
What is the occupancy?
\item 
Where is $\EF$ (the Fermi energy)?
\end{itemize}

Consider alkalai metals, such as \ce{Li} $1s^2 2 s^1$

Is tight binding like \underline{fully occupied}.
$2 \times 1 s$ electrons per atom.

In 1 band

\begin{dmath}\label{eqn:condensedMatterLecture17:20}
\mathLabelBox
[
   labelstyle={xshift=-2cm},
   linestyle={out=270,in=90, latex-}
]
{\frac{ \cancel{(2 \pi)^3}}{a^3} }{
volume of k space }
\times 
\mathLabelBox
[
   labelstyle={below of=m\themathLableNode, below of=m\themathLableNode}
]
{2}{spin} 
\mathLabelBox
[
   labelstyle={xshift=2cm},
   linestyle={out=270,in=90, latex-}
]
{\frac{ V}{ \cancel{(2 \pi)^3} }}{density of k points}
\end{dmath}

F4
F5

This gives

\begin{dmath}\label{eqn:condensedMatterLecture17:40}
2 \frac{V}{a^3} = 2 N,
\end{dmath}

where $N$ is the number of unit cells.

The $2 s$ orbitals stick out a long way, so this is free electron like.
We have \underline{half filled} orbitals.  Where $\EF$ crosses $E(k)$ is a Fermi surface.  It never gets close to the Brillion zone boundary.

This is nearly spherical.  See slides for the true Brillion zone diagram for \ce{Li}.

\paragraph{Valence 2}

For alkalai earth's \ce{Mg}, \ce{Ca}, $\cdots$, we have $2 \times 2 s$ valence electrons.  This \underline{doesn't} fill the band, because of dispersion.

Define a free electron sphere.  It extends beyond the first Brillion zone.  Incorrectly sketching this as a simple cubic (actual is perhaps FCC)

F6
F7
F8 ... Fermi surfaces for \ce{Cu} like simple cubic.

\examhint{For a picture like this, understand what happens at the boundary, and how it reconstructs.}

\paragraph{Valence 3}

Considering a material such as \ce{Al}, which is in valance 3.  See slide.
\examhint{Not examinable}

\paragraph{d electron systems}

We have two kinds of valence electrons.

\begin{itemize}
\item s electrons.  Free electron like.
\item d electrons.  compact orbitals that are tight binding like.
\end{itemize}

Looking with an experienced eye, we see two types of bands.  The first are the s-bands that are free electron like and rapidly disburse.  The others are the d orbital band that disperse a bit, but not very much.  

F9

What actually happens in here where they cross is like 

F10

The \underline{occpancy} for \ce{Cu} is $3 d^{10} 4 s^1$.  The d orbitals are \underline{filled}.

$\EF$ intersects $4 s$ bands, we have a nearly free electron Fermi surface.

The reason that copper is copper coloured is because there's a high density of states, with obsorbtion in blue, resulting in a brown look.

We have something similar for \ce{Ag} and \ce{Au}.

From \ce{Sc} to \ce{Ni}, or \ce{Y} to \ce{Pd}, or \ce{La} to \ce{Pt}, the d orbitals are partially occupied, and $\EF$ is among the d bands.  This ends up being a complicated Fermi surface, and there is a high density of states at $\EF$ (see table in slides).

These atoms with $3d$ valence electrons are very prone to magnetism.
The atoms with $4d$ valence electrons are very prone to superconductivity.

A high density of states in physics makes for interesting effects.

\EndArticle
