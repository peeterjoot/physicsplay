%
% Copyright � 2015 Peeter Joot.  All Rights Reserved.
% Licenced as described in the file LICENSE under the root directory of this GIT repository.
%
\documentclass[]{eliblog}

\usepackage{amsmath}
\usepackage{mathpazo}

%
% shorthand for bold symbols, convenient for vectors and matrices
%
\newcommand{\Ba}[0]{\mathbf{a}}
\newcommand{\Bb}[0]{\mathbf{b}}
\newcommand{\Bc}[0]{\mathbf{c}}
\newcommand{\Bd}[0]{\mathbf{d}}
\newcommand{\Be}[0]{\mathbf{e}}
\newcommand{\Bf}[0]{\mathbf{f}}
\newcommand{\Bg}[0]{\mathbf{g}}
\newcommand{\Bh}[0]{\mathbf{h}}
\newcommand{\Bi}[0]{\mathbf{i}}
\newcommand{\Bj}[0]{\mathbf{j}}
\newcommand{\Bk}[0]{\mathbf{k}}
\newcommand{\Bl}[0]{\mathbf{l}}
\newcommand{\Bm}[0]{\mathbf{m}}
\newcommand{\Bn}[0]{\mathbf{n}}
\newcommand{\Bo}[0]{\mathbf{o}}
\newcommand{\Bp}[0]{\mathbf{p}}
\newcommand{\Bq}[0]{\mathbf{q}}
\newcommand{\Br}[0]{\mathbf{r}}
\newcommand{\Bs}[0]{\mathbf{s}}
\newcommand{\Bt}[0]{\mathbf{t}}
\newcommand{\Bu}[0]{\mathbf{u}}
\newcommand{\Bv}[0]{\mathbf{v}}
\newcommand{\Bw}[0]{\mathbf{w}}
\newcommand{\Bx}[0]{\mathbf{x}}
\newcommand{\By}[0]{\mathbf{y}}
\newcommand{\Bz}[0]{\mathbf{z}}
\newcommand{\BA}[0]{\mathbf{A}}
\newcommand{\BB}[0]{\mathbf{B}}
\newcommand{\BC}[0]{\mathbf{C}}
\newcommand{\BD}[0]{\mathbf{D}}
\newcommand{\BE}[0]{\mathbf{E}}
\newcommand{\BF}[0]{\mathbf{F}}
\newcommand{\BG}[0]{\mathbf{G}}
\newcommand{\BH}[0]{\mathbf{H}}
\newcommand{\BI}[0]{\mathbf{I}}
\newcommand{\BJ}[0]{\mathbf{J}}
\newcommand{\BK}[0]{\mathbf{K}}
\newcommand{\BL}[0]{\mathbf{L}}
\newcommand{\BM}[0]{\mathbf{M}}
\newcommand{\BN}[0]{\mathbf{N}}
\newcommand{\BO}[0]{\mathbf{O}}
\newcommand{\BP}[0]{\mathbf{P}}
\newcommand{\BQ}[0]{\mathbf{Q}}
\newcommand{\BR}[0]{\mathbf{R}}
\newcommand{\BS}[0]{\mathbf{S}}
\newcommand{\BT}[0]{\mathbf{T}}
\newcommand{\BU}[0]{\mathbf{U}}
\newcommand{\BV}[0]{\mathbf{V}}
\newcommand{\BW}[0]{\mathbf{W}}
\newcommand{\BX}[0]{\mathbf{X}}
\newcommand{\BY}[0]{\mathbf{Y}}
\newcommand{\BZ}[0]{\mathbf{Z}}

\newcommand{\Bzero}[0]{\mathbf{0}}
\newcommand{\Btheta}[0]{\boldsymbol{\theta}}
\newcommand{\Btau}[0]{\boldsymbol{\tau}}
\newcommand{\Bomega}[0]{\boldsymbol{\omega}}

%
% shorthand for unit vectors
%
\newcommand{\acap}[0]{\hat{\Ba}}
\newcommand{\bcap}[0]{\hat{\Bb}}
\newcommand{\ccap}[0]{\hat{\Bc}}
\newcommand{\dcap}[0]{\hat{\Bd}}
\newcommand{\ecap}[0]{\hat{\Be}}
\newcommand{\fcap}[0]{\hat{\Bf}}
\newcommand{\gcap}[0]{\hat{\Bg}}
\newcommand{\hcap}[0]{\hat{\Bh}}
\newcommand{\icap}[0]{\hat{\Bi}}
\newcommand{\jcap}[0]{\hat{\Bj}}
\newcommand{\kcap}[0]{\hat{\Bk}}
\newcommand{\lcap}[0]{\hat{\Bl}}
\newcommand{\mcap}[0]{\hat{\Bm}}
\newcommand{\ncap}[0]{\hat{\Bn}}
\newcommand{\ocap}[0]{\hat{\Bo}}
\newcommand{\pcap}[0]{\hat{\Bp}}
\newcommand{\qcap}[0]{\hat{\Bq}}
\newcommand{\rcap}[0]{\hat{\Br}}
\newcommand{\scap}[0]{\hat{\Bs}}
\newcommand{\tcap}[0]{\hat{\Bt}}
\newcommand{\ucap}[0]{\hat{\Bu}}
\newcommand{\vcap}[0]{\hat{\Bv}}
\newcommand{\wcap}[0]{\hat{\Bw}}
\newcommand{\xcap}[0]{\hat{\Bx}}
\newcommand{\ycap}[0]{\hat{\By}}
\newcommand{\zcap}[0]{\hat{\Bz}}
\newcommand{\thetacap}[0]{\hat{\Btheta}}

%
% to write R^n and C^n in a distinguishable fashion.  Perhaps change this
% to the double lined characters upon figuring out how to do so.
%
\newcommand{\C}[1]{$\mathbb{C}^{#1}$}
\newcommand{\R}[1]{$\mathbb{R}^{#1}$}

%
% various generally useful helpers
%

% derivative of #1 wrt. #2:
\newcommand{\D}[2] {\frac {d#2} {d#1}}

\newcommand{\inv}[1]{\frac{1}{#1}}
\newcommand{\cross}[0]{\times}

\newcommand{\abs}[1]{\lvert{#1}\rvert}
\newcommand{\norm}[1]{\lVert{#1}\rVert}
\newcommand{\innerprod}[2]{\langle{#1}, {#2}\rangle}
\newcommand{\dotprod}[2]{{#1} \cdot {#2}}
\newcommand{\bdotprod}[2]{\left({#1} \cdot {#2}\right)}
\newcommand{\crossprod}[2]{{#1} \cross {#2}}
\newcommand{\tripleprod}[3]{\dotprod{\left(\crossprod{#1}{#2}\right)}{#3}}

\DeclareMathOperator{\Proj}{Proj}
\DeclareMathOperator{\Span}{span}
\DeclareMathOperator{\Sgn}{sgn}
\DeclareMathOperator{\Area}{Area}
\DeclareMathOperator{\Volume}{Volume}

%
% A few miscellaneous things specific to this document
%
\newcommand{\crossop}[1]{\crossprod{#1}{}}

% R2 vector.
\newcommand{\VectorTwo}[2]{
\begin{bmatrix}
 {#1} \\
 {#2}
\end{bmatrix}
}

\newcommand{\VectorN}[1]{
\begin{bmatrix}
{#1}_1 \\
{#1}_2 \\
\vdots \\
{#1}_N \\
\end{bmatrix}
}

\newcommand{\DETuvij}[4]{
\begin{vmatrix}
 {#1}_{#3} & {#1}_{#4} \\
 {#2}_{#3} & {#2}_{#4}
\end{vmatrix}
}

\newcommand{\DETuvwijk}[6]{
\begin{vmatrix}
 {#1}_{#4} & {#1}_{#5} & {#1}_{#6} \\
 {#2}_{#4} & {#2}_{#5} & {#2}_{#6} \\
 {#3}_{#4} & {#3}_{#5} & {#3}_{#6}
\end{vmatrix}
}

\newcommand{\DETuvwxijkl}[8]{
\begin{vmatrix}
 {#1}_{#5} & {#1}_{#6} & {#1}_{#7} & {#1}_{#8} \\
 {#2}_{#5} & {#2}_{#6} & {#2}_{#7} & {#2}_{#8} \\
 {#3}_{#5} & {#3}_{#6} & {#3}_{#7} & {#3}_{#8} \\
 {#4}_{#5} & {#4}_{#6} & {#4}_{#7} & {#4}_{#8} \\
\end{vmatrix}
}

%\newcommand{\DETuvwxyijklm}[10]{
%\begin{vmatrix}
% {#1}_{#6} & {#1}_{#7} & {#1}_{#8} & {#1}_{#9} & {#1}_{#10} \\
% {#2}_{#6} & {#2}_{#7} & {#2}_{#8} & {#2}_{#9} & {#2}_{#10} \\
% {#3}_{#6} & {#3}_{#7} & {#3}_{#8} & {#3}_{#9} & {#3}_{#10} \\
% {#4}_{#6} & {#4}_{#7} & {#4}_{#8} & {#4}_{#9} & {#4}_{#10} \\
% {#5}_{#6} & {#5}_{#7} & {#5}_{#8} & {#5}_{#9} & {#5}_{#10}
%\end{vmatrix}
%}

% R3 vector.
\newcommand{\VectorThree}[3]{
\begin{bmatrix}
 {#1} \\
 {#2} \\
 {#3}
\end{bmatrix}
}



\author{Peeter Joot}
\email{peeter.joot@gmail.com}

%\documentclass[]{eliblogwidescreen}

\usepackage{amsmath}
\usepackage{mathpazo}

%
% shorthand for bold symbols, convenient for vectors and matrices
%
\newcommand{\Ba}[0]{\mathbf{a}}
\newcommand{\Bb}[0]{\mathbf{b}}
\newcommand{\Bc}[0]{\mathbf{c}}
\newcommand{\Bd}[0]{\mathbf{d}}
\newcommand{\Be}[0]{\mathbf{e}}
\newcommand{\Bf}[0]{\mathbf{f}}
\newcommand{\Bg}[0]{\mathbf{g}}
\newcommand{\Bh}[0]{\mathbf{h}}
\newcommand{\Bi}[0]{\mathbf{i}}
\newcommand{\Bj}[0]{\mathbf{j}}
\newcommand{\Bk}[0]{\mathbf{k}}
\newcommand{\Bl}[0]{\mathbf{l}}
\newcommand{\Bm}[0]{\mathbf{m}}
\newcommand{\Bn}[0]{\mathbf{n}}
\newcommand{\Bo}[0]{\mathbf{o}}
\newcommand{\Bp}[0]{\mathbf{p}}
\newcommand{\Bq}[0]{\mathbf{q}}
\newcommand{\Br}[0]{\mathbf{r}}
\newcommand{\Bs}[0]{\mathbf{s}}
\newcommand{\Bt}[0]{\mathbf{t}}
\newcommand{\Bu}[0]{\mathbf{u}}
\newcommand{\Bv}[0]{\mathbf{v}}
\newcommand{\Bw}[0]{\mathbf{w}}
\newcommand{\Bx}[0]{\mathbf{x}}
\newcommand{\By}[0]{\mathbf{y}}
\newcommand{\Bz}[0]{\mathbf{z}}
\newcommand{\BA}[0]{\mathbf{A}}
\newcommand{\BB}[0]{\mathbf{B}}
\newcommand{\BC}[0]{\mathbf{C}}
\newcommand{\BD}[0]{\mathbf{D}}
\newcommand{\BE}[0]{\mathbf{E}}
\newcommand{\BF}[0]{\mathbf{F}}
\newcommand{\BG}[0]{\mathbf{G}}
\newcommand{\BH}[0]{\mathbf{H}}
\newcommand{\BI}[0]{\mathbf{I}}
\newcommand{\BJ}[0]{\mathbf{J}}
\newcommand{\BK}[0]{\mathbf{K}}
\newcommand{\BL}[0]{\mathbf{L}}
\newcommand{\BM}[0]{\mathbf{M}}
\newcommand{\BN}[0]{\mathbf{N}}
\newcommand{\BO}[0]{\mathbf{O}}
\newcommand{\BP}[0]{\mathbf{P}}
\newcommand{\BQ}[0]{\mathbf{Q}}
\newcommand{\BR}[0]{\mathbf{R}}
\newcommand{\BS}[0]{\mathbf{S}}
\newcommand{\BT}[0]{\mathbf{T}}
\newcommand{\BU}[0]{\mathbf{U}}
\newcommand{\BV}[0]{\mathbf{V}}
\newcommand{\BW}[0]{\mathbf{W}}
\newcommand{\BX}[0]{\mathbf{X}}
\newcommand{\BY}[0]{\mathbf{Y}}
\newcommand{\BZ}[0]{\mathbf{Z}}

\newcommand{\Bzero}[0]{\mathbf{0}}
\newcommand{\Btheta}[0]{\boldsymbol{\theta}}
\newcommand{\Btau}[0]{\boldsymbol{\tau}}
\newcommand{\Bomega}[0]{\boldsymbol{\omega}}

%
% shorthand for unit vectors
%
\newcommand{\acap}[0]{\hat{\Ba}}
\newcommand{\bcap}[0]{\hat{\Bb}}
\newcommand{\ccap}[0]{\hat{\Bc}}
\newcommand{\dcap}[0]{\hat{\Bd}}
\newcommand{\ecap}[0]{\hat{\Be}}
\newcommand{\fcap}[0]{\hat{\Bf}}
\newcommand{\gcap}[0]{\hat{\Bg}}
\newcommand{\hcap}[0]{\hat{\Bh}}
\newcommand{\icap}[0]{\hat{\Bi}}
\newcommand{\jcap}[0]{\hat{\Bj}}
\newcommand{\kcap}[0]{\hat{\Bk}}
\newcommand{\lcap}[0]{\hat{\Bl}}
\newcommand{\mcap}[0]{\hat{\Bm}}
\newcommand{\ncap}[0]{\hat{\Bn}}
\newcommand{\ocap}[0]{\hat{\Bo}}
\newcommand{\pcap}[0]{\hat{\Bp}}
\newcommand{\qcap}[0]{\hat{\Bq}}
\newcommand{\rcap}[0]{\hat{\Br}}
\newcommand{\scap}[0]{\hat{\Bs}}
\newcommand{\tcap}[0]{\hat{\Bt}}
\newcommand{\ucap}[0]{\hat{\Bu}}
\newcommand{\vcap}[0]{\hat{\Bv}}
\newcommand{\wcap}[0]{\hat{\Bw}}
\newcommand{\xcap}[0]{\hat{\Bx}}
\newcommand{\ycap}[0]{\hat{\By}}
\newcommand{\zcap}[0]{\hat{\Bz}}
\newcommand{\thetacap}[0]{\hat{\Btheta}}

%
% to write R^n and C^n in a distinguishable fashion.  Perhaps change this
% to the double lined characters upon figuring out how to do so.
%
\newcommand{\C}[1]{$\mathbb{C}^{#1}$}
\newcommand{\R}[1]{$\mathbb{R}^{#1}$}

%
% various generally useful helpers
%

% derivative of #1 wrt. #2:
\newcommand{\D}[2] {\frac {d#2} {d#1}}

\newcommand{\inv}[1]{\frac{1}{#1}}
\newcommand{\cross}[0]{\times}

\newcommand{\abs}[1]{\lvert{#1}\rvert}
\newcommand{\norm}[1]{\lVert{#1}\rVert}
\newcommand{\innerprod}[2]{\langle{#1}, {#2}\rangle}
\newcommand{\dotprod}[2]{{#1} \cdot {#2}}
\newcommand{\bdotprod}[2]{\left({#1} \cdot {#2}\right)}
\newcommand{\crossprod}[2]{{#1} \cross {#2}}
\newcommand{\tripleprod}[3]{\dotprod{\left(\crossprod{#1}{#2}\right)}{#3}}

\DeclareMathOperator{\Proj}{Proj}
\DeclareMathOperator{\Span}{span}
\DeclareMathOperator{\Sgn}{sgn}
\DeclareMathOperator{\Area}{Area}
\DeclareMathOperator{\Volume}{Volume}

%
% A few miscellaneous things specific to this document
%
\newcommand{\crossop}[1]{\crossprod{#1}{}}

% R2 vector.
\newcommand{\VectorTwo}[2]{
\begin{bmatrix}
 {#1} \\
 {#2}
\end{bmatrix}
}

\newcommand{\VectorN}[1]{
\begin{bmatrix}
{#1}_1 \\
{#1}_2 \\
\vdots \\
{#1}_N \\
\end{bmatrix}
}

\newcommand{\DETuvij}[4]{
\begin{vmatrix}
 {#1}_{#3} & {#1}_{#4} \\
 {#2}_{#3} & {#2}_{#4}
\end{vmatrix}
}

\newcommand{\DETuvwijk}[6]{
\begin{vmatrix}
 {#1}_{#4} & {#1}_{#5} & {#1}_{#6} \\
 {#2}_{#4} & {#2}_{#5} & {#2}_{#6} \\
 {#3}_{#4} & {#3}_{#5} & {#3}_{#6}
\end{vmatrix}
}

\newcommand{\DETuvwxijkl}[8]{
\begin{vmatrix}
 {#1}_{#5} & {#1}_{#6} & {#1}_{#7} & {#1}_{#8} \\
 {#2}_{#5} & {#2}_{#6} & {#2}_{#7} & {#2}_{#8} \\
 {#3}_{#5} & {#3}_{#6} & {#3}_{#7} & {#3}_{#8} \\
 {#4}_{#5} & {#4}_{#6} & {#4}_{#7} & {#4}_{#8} \\
\end{vmatrix}
}

%\newcommand{\DETuvwxyijklm}[10]{
%\begin{vmatrix}
% {#1}_{#6} & {#1}_{#7} & {#1}_{#8} & {#1}_{#9} & {#1}_{#10} \\
% {#2}_{#6} & {#2}_{#7} & {#2}_{#8} & {#2}_{#9} & {#2}_{#10} \\
% {#3}_{#6} & {#3}_{#7} & {#3}_{#8} & {#3}_{#9} & {#3}_{#10} \\
% {#4}_{#6} & {#4}_{#7} & {#4}_{#8} & {#4}_{#9} & {#4}_{#10} \\
% {#5}_{#6} & {#5}_{#7} & {#5}_{#8} & {#5}_{#9} & {#5}_{#10}
%\end{vmatrix}
%}

% R3 vector.
\newcommand{\VectorThree}[3]{
\begin{bmatrix}
 {#1} \\
 {#2} \\
 {#3}
\end{bmatrix}
}



\author{Peeter Joot}
\email{peeter.joot@gmail.com}


\chapter{PHY450H1S.  Relativistic Electrodynamics Lecture 15 (Taught by Prof. Erich Poppitz).  FIXME.}
\label{chap:relativisticElectrodynamicsL15}
%\useCCL
\blogpage{http://sites.google.com/site/peeterjoot/math2011/relativisticElectrodynamicsL15.pdf}
\date{Feb 16, 2011}
\revisionInfo{relativisticElectrodynamicsL15.tex}

%\beginArtWithToc
\beginArtNoToc

\section{Reading.}

Covering chapter 4 material from the text \cite{landau1980classical}.

Covering \href{http://www.physics.utoronto.ca/~poppitz/e-poppitz/PHY450_files/RelEMpp115-127.pdf}{lecture notes pp. 115-127}: reminder on wave equations (115); reminder on Fourier series and integral (115-117); Fourier expansion of the EM potential in Coulomb gauge and equation of motion for the spatial Fourier components (118-119); the general solution of Maxwell's equations in vacuum (120-121) [Tuesday, Mar. 1]; properties of monochromatic plane EM waves (122-124); energy and energy flux of the EM field and energy conservation from the equations of motion (125-127)  [Wednesday, Mar. 2]

\section{Review of wave equation results obtained.}

Maxwell's equations in vacuum lead to Coulomb gauge and the Lorentz gauge.

Coulomb gauge

\begin{align}\label{eqn:relativisticElectrodynamicsL15:n}
A^0 &= 0 \\
\spacegrad \cdot \BA &= 0
\left( \inv{c^2} \PDSq{t}{} - \Delta \right) \BA &= 0
\end{align}

Lorentz gauge

\begin{align}\label{eqn:relativisticElectrodynamicsL15:n}
\partial_i A^i &= 0 \\
\left( \inv{c^2} \PDSq{t}{} - \Delta \right) A^i &= 0
\end{align}

Note that $\partial_i A^i = 0$ is invariant under gauge transformations

\begin{equation}\label{eqn:relativisticElectrodynamicsL15:n}
A^i \rightarrow A^i + \partial^i \chi
\end{equation}

where 

\begin{equation}\label{eqn:relativisticElectrodynamicsL15:n}
\partial_i \partial^i \chi = 0
\end{equation}

i.e.  If one uses the Lorentz gauge, this has to be fixed.

However, in both cases we have 

\begin{equation}\label{eqn:relativisticElectrodynamicsL15:n}
\left( \inv{c^2} \PDSq{t}{} - \Delta \right) f = 0
\end{equation}

where 

\begin{equation}\label{eqn:relativisticElectrodynamicsL15:n}
\inv{c^2} \PDSq{t}{} - \Delta 
\end{equation}

is the wave operator.

Consider 

\begin{equation}\label{eqn:relativisticElectrodynamicsL15:n}
\Delta = \PDSq{x}{}
\end{equation}

where we are looking for a solution that is independent of $y, z$.  Recall that the general solution for this equation has the form

\begin{equation}\label{eqn:relativisticElectrodynamicsL15:n}
f(t, x) = 
F_1 \left(t - \frac{x}{c}\right)
+F_2 \left(t + \frac{x}{c}\right)
\end{equation}

PICTURE: superposition of two waves with $F_1$ moving along the x-axis in the positive direction, and $F_2$ in the negative x direction.

\section{Review of Fourier methods.}

It is often convienent to impose periodic boundary conditions

\begin{equation}\label{eqn:relativisticElectrodynamicsL15:n}
\BA(\Bx + \Be_i L) = \BA(\Bx), i = 1,2,3
\end{equation}

\subsection{In one dimension}

\begin{equation}\label{eqn:relativisticElectrodynamicsL15:n}
f(x + L) = f(x)
\end{equation}

\begin{equation}\label{eqn:relativisticElectrodynamicsL15:n}
f(x) = \sum_{n=-\infty}^\infty e^{i \frac{2 \pi n}{L} x} \tilde{f}_n
\end{equation}

When $f(x)$ is real we also have

\begin{equation}\label{eqn:relativisticElectrodynamicsL15:n}
f^\conj(x) = \sum_{n = -\infty}^\infty e^{-i \frac{2 \pi n}{L} x} (\tilde{f}_n)^\conj
\end{equation}

which implies

FIXME: DIY: 

\begin{equation}\label{eqn:relativisticElectrodynamicsL15:n}
{\tilde{f}^\conj}_{n} = \tilde{f}_{-n}
\end{equation}

Write 

\begin{equation}\label{eqn:relativisticElectrodynamicsL15:n}
k_n = \frac{2 \pi n}{L}
\end{equation}

This is the wave number, giving us

\begin{equation}\label{eqn:relativisticElectrodynamicsL15:n}
f(x) = \sum_{n=-\infty}^\infty e^{i k_n x} \tilde{f}_{k_n}
\end{equation}

The inverse transform is obtained by integration

\begin{equation}\label{eqn:relativisticElectrodynamicsL15:n}
f_{k_n} = \inv{L} \int_{-L/2}^{L/2} dx e^{-i k_n x} f(x)
\end{equation}

Verify: 
\begin{align*}
f_{k_n} 
&= \inv{L} \int_{-L/2}^{L/2} dx e^{-i k_n x} \sum_{m=-\infty}^\infty e^{i k_m x} \tilde{f}_{k_m} \\
&= ...  \\
&DIY: \\
&= \sum_{m = -\infty}^\infty \tilde{f}_{k_m} \delta_{mn} = \tilde{f}_{k_m}
\end{align*}

It is conventional to absorb $\tilde{f}_{k_n} = \tilde{f}(k_n)$ for

\begin{equation}\label{eqn:relativisticElectrodynamicsL15:n}
f(x) = \inv{L} \sum_n \tilde{f}(k_n) e^{i k_n x}
\tilde{f}(k_n) = \int_{-L/2}^{L/2} dx f(x) e^{-i k_n x}
\end{equation}

To take $L -> \infty$ notice

\begin{equation}\label{eqn:relativisticElectrodynamicsL15:n}
k_n = \frac{2 \pi}{L} n
\end{equation}

when $n$ changes by $\Delta n = 1$, $k_n$ changes by $\Delta k_n = \frac{2 \pi}{L} \Delta n$

Using this 

\begin{equation}\label{eqn:relativisticElectrodynamicsL15:n}
f(x) = \inv{2\pi} \sum_n \left( \frac{2\pi}{L} \delta n \right) \tilde{f}(k_n) e^{i k_n x}
\end{equation}

With $L -> \infty$, and $\Delta k_n \rightarrow 0$

\begin{equation}\label{eqn:relativisticElectrodynamicsL15:n}
f(x) = \frac{dk}{2\pi} \tilde{f}(k_n) e^{i k_n x}
\end{equation}

Verify:

\begin{equation}\label{eqn:relativisticElectrodynamicsL15:n}
f(x) = \int \frac{dk}{2\pi} \tilde{f}(k_n) e^{i k x}
\end{equation}

\begin{equation}\label{eqn:relativisticElectrodynamicsL15:n}
\tilde{f}(k) = \int dx f(x) e^{-i k x}
\end{equation}

With
\begin{equation}\label{eqn:relativisticElectrodynamicsL15:n}
\int dk e^{i k (x - y)} = 2 \pi \delta(x - y)
\end{equation}

we verify the inverse relation.

=> FIXME:DIY:

\subsection{In three dimensions}

\begin{align}\label{eqn:relativisticElectrodynamicsL15:n}
\BA(\Bx, t) &= \iiint_{-\infty}^\infty \tilde{\BA}(\Bk, t) e^{i \Bk \cdot \Bx} \\
\tilde{\BA}(\Bx, t) &= \iiint_{-\infty}^\infty \BA(\Bx, t) e^{-i \Bk \cdot \Bx}
\end{align}

\subsection{Application to the wave equation}

\begin{align*}
0 &= 
\left( \inv{c^2} \PDSq{t}{} - \Delta \right) \BA(\Bx, t) \\
&=
\left( \inv{c^2} \PDSq{t}{} - \Delta \right) \BA(\Bx, t)
\int \tilde{\BA}(\Bk, t) e^{i \Bk \cdot \Bx} \\
&=
\int 
\left( 
\inv{c^2} \partial_{tt} \tilde{\BA}(\Bk, t) + \Bk^2 \BA(\Bk, t)
\right)
e^{i \Bk \cdot \Bx} 
\end{align*}

Now operate with $\int d^3 \Bx e^{-i \Bp \cdot \Bx }$

\begin{align*}
0 &=
\int d^3 \Bx e^{-i \Bp \cdot \Bx }
\int 
\left( 
\inv{c^2} \partial_{tt} \tilde{\BA}(\Bk, t) + \Bk^2 \BA(\Bk, t)
\right)
e^{i \Bk \cdot \Bx}  \\
&=
\int \frac{d^3 \Bk}{2\pi} (2\pi)^3 \delta^3(\Bp -\Bk) 
\left( 
\inv{c^2} \partial_{tt} \tilde{\BA}(\Bk, t) + \Bk^2 \BA(\Bk, t)
\right)
\end{align*}

Since this is true for all $\Bp$ we have

\begin{equation}\label{eqn:relativisticElectrodynamicsL15:n}
\partial_{tt} \tilde{\BA}(\Bp, t) = -c^2 \Bp^2 \tilde{\BA}(\Bp, t) 
\end{equation}

For every value of momentum we have a harmonic osciallator!

\begin{equation}\label{eqn:relativisticElectrodynamicsL15:n}
\ddot{x} = -\omega^2 x
\end{equation}

Fourier modes of EM potential in vacuum obey

\partial_{tt} \tilde{\BA}(\Bk, t) = -c^2 \Bk^2 \tilde{\BA}(\Bk, t)

FIXME: DIY:

\spacegrad \cdot \BA(\Bx, t) = 0

=> 

\begin{align*}
0 
&= \int \frac{d^3 \Bk }{(2 \pi)^3} \spacegrad e^{i \Bk \cdot \Bx} \cdot \TBA(\Bk, t)
\end{align*}

e^{i \Bk \cdot \Bx} i \Bk \cdot \BA(\Bk, t)

=> 

\begin{align}\label{eqn:relativisticElectrodynamicsL15:n}
\TBA(-\Bk, t) &= \TBA^\conj(\Bk, t)
\Bk \cdot \TBA(\Bk, t) &= 0
\end{align}

\BA(\Bx, t) = \int d3k/2p^3 e^{i \Bk \cdot \Bx} \left( \inv{2} \TBA(\Bk, t) + \inv{2} \TBA^\conj(- \Bk, t) \right)

Since we have

\partial_{tt} \TBA(\Bk, t) = - \omega_k^2 \TBA(\Bk, t)

gives general solution 

\TBA(\Bk, t) = e^{i \omega_k t} \Ba_{+}(\Bk) +e^{-i \omega_k t} \Ba_{-}(\Bk)
\TBA^\conj(\Bk, t) = e^{-i \omega_k t} \Ba_{+}^\conj(\Bk) +e^{i \omega_k t} \Ba_{-}^\conj(\Bk)


\BA(\Bx, t) 
= \int d3k/2pi^3 e^{i \Bk \cdot \Bx} 
\inv{2} \left( 
e^{i \omega_k t} (\Ba_p(\Bk) + \Ba_m^\conj(-\Bk)) 
+e^{-i \omega_k t} (\Ba_m(\Bk) + \Ba_p^\conj(-\Bk)) 
\right)

Define

\Bbeta(\Bk) \equiv \Ba_m(\Bk) + \Ba_p^\conj(-\Bk) 

so that

\Bbeta(-\Bk) = \Ba_p(\Bk) + \Ba_p^\conj(-\Bk)

This is now manifestly real

\begin{equation}\label{eqn:relativisticElectrodynamicsL15:n}
\BA(\Bx, t) = \int d3k/2pi^3 \left( 
e^{i (\Bk \cdot \Bx + \omega_k t) \Bbeta^\conj(-\Bk)
+e^{i (\Bk \cdot \Bx - \omega_k t) \Bbeta(\Bk)
\end{equation}

With $\Bk \cdot \Bbeta(\Bk)  = 0$

CHANGE: EXAM: 7-9 MONDAY!!!

\EndArticle
