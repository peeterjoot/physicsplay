%
% Copyright � 2015 Peeter Joot.  All Rights Reserved.
% Licenced as described in the file LICENSE under the root directory of this GIT repository.
%
\newcommand{\authorname}{Peeter Joot}
\newcommand{\email}{peeterjoot@protonmail.com}
\newcommand{\basename}{FIXMEbasenameUndefined}
\newcommand{\dirname}{notes/FIXMEdirnameUndefined/}

\renewcommand{\basename}{moreKet}
\renewcommand{\dirname}{notes/phy1520/}
%\newcommand{\dateintitle}{}
%\newcommand{\keywords}{}

\newcommand{\authorname}{Peeter Joot}
\newcommand{\onlineurl}{http://sites.google.com/site/peeterjoot2/math2013/\basename.pdf}
\newcommand{\sourcepath}{\dirname\basename.tex}
\newcommand{\generatetitle}[1]{\chapter{#1}}

\newcommand{\vcsinfo}{%
\section*{}
\noindent{\color{DarkOliveGreen}{\rule{\linewidth}{0.1mm}}}
\paragraph{Document version}
%\paragraph{\color{Maroon}{Document version}}
{
\small
\begin{itemize}
\item Available online at:\\ 
\href{\onlineurl}{\onlineurl}
\item Git Repository: \input{./.revinfo/gitRepo.tex}
\item Source: \sourcepath
\item last commit: \input{./.revinfo/gitCommitString.tex}
\item commit date: \input{./.revinfo/gitCommitDate.tex}
\end{itemize}
}
}

%\PassOptionsToPackage{dvipsnames,svgnames}{xcolor}
\PassOptionsToPackage{square,numbers}{natbib}
\documentclass{scrreprt}

\usepackage[left=2cm,right=2cm]{geometry}
\usepackage[svgnames]{xcolor}
\usepackage{peeters_layout}

\usepackage{natbib}

\usepackage[
colorlinks=true,
bookmarks=false,
pdfauthor={\authorname, \email},
backref 
]{hyperref}

% http://tex.stackexchange.com/questions/75773/how-to-reference-problems-by-the-text-label-in-an-exercise-envioronment
\usepackage[english]{cleveref}
\crefname{Exercise}{exercise}{exercises}
\Crefname{Exercise}{Exercise}{Exercises}

\RequirePackage{titlesec}
\RequirePackage{ifthen}

% http://stackoverflow.com/questions/4932910/date-in-the-tabular-environment
\makeatletter
\let\insertdate\@date
\makeatother

\titleformat{\chapter}[display]
{\bfseries\Large}
{\color{DarkSlateGrey}\filleft \authorname
\ifthenelse{\isundefined{\studentnumber}}{}{\\ \studentnumber}
\ifthenelse{\isundefined{\email}}{}{\\ \email}
\ifthenelse{\isundefined{\dateintitle}}{}{\\ \insertdate}
%\ifthenelse{\isundefined{\coursename}}{}{\\ \coursename} % put in title instead.
}
{4ex}
{\color{DarkOliveGreen}{\titlerule}\color{Maroon}
\vspace{2ex}%
\filright}
[\vspace{2ex}%
\color{DarkOliveGreen}\titlerule
]

\newcommand{\beginArtWithToc}[0]{\begin{document}\tableofcontents}
\newcommand{\beginArtNoToc}[0]{\begin{document}}
\newcommand{\EndNoBibArticle}[0]{\end{document}}
\newcommand{\EndArticle}[0]{\bibliography{Bibliography}\bibliographystyle{plainnat}\end{document}}

% 
%\newcommand{\citep}[1]{\cite{#1}}

\colorSectionsForArticle



\usepackage{peeters_layout_exercise}
\usepackage{peeters_braket}
\usepackage{peeters_figures}

\beginArtNoToc

\generatetitle{More ket problems}
%\chapter{More ket problems}
%\label{chap:moreKet}


\makeoproblem{degenerate ket space example}{problem:moreKet:23}{\citep{sakurai2014modern} pr. X.23}{

Consider operators with representation

\begin{equation}\label{eqn:moreKet:20}
A = 
\begin{bmatrix}
a & 0 & 0 \\
0 & -a & 0 \\
0 & 0 & -a
\end{bmatrix}
,
\qquad
B = 
\begin{bmatrix}
b & 0 & 0 \\
0 & 0 & -ib \\
0 & ib & 0
\end{bmatrix}.
\end{equation}

Show that these both have degeneracies, commute, and compute a simultaneous ket space for both operators.

} % problem

\makeanswer{problem:moreKet:23}{
The eigenvalues and eigenvectors for \( A \) can be read off by inspection, with values of \( a, -a, -a \), and kets

\begin{equation}\label{eqn:moreKet:40}
\ket{a_1} = 
\begin{bmatrix}
1 \\
0 \\
0
\end{bmatrix},
\ket{a_2} = 
\begin{bmatrix}
0 \\
1 \\
0
\end{bmatrix},
\ket{a_3} = 
\begin{bmatrix}
0 \\
0 \\
1 \\
\end{bmatrix}
\end{equation}

Notice that the lower-right \( 2 \times 2 \) submatrix of \( B \) is proportional to \( \sigma_y \), so it's eigenvalues can be formed by inspection

\begin{equation}\label{eqn:moreKet:60}
\ket{b_1} = 
\begin{bmatrix}
1 \\
0 \\
0
\end{bmatrix},
\ket{b_2} = 
\inv{\sqrt{2}}
\begin{bmatrix}
0 \\
1 \\
i
\end{bmatrix},
\ket{b_3} = 
\inv{\sqrt{2}}
\begin{bmatrix}
0 \\
1 \\
-i \\
\end{bmatrix}.
\end{equation}

Computing \( B \ket{b_i} \) shows that the eigenvalues are \( b, b, -b \) respectively.

Because of the two-fold degeneracy in the \( -a \) eigenvalues of \( A \), any linear combination of \( \ket{a_2}, \ket{a_3} \) will also be an eigenket.   In particular,

\begin{equation}\label{eqn:moreKet:80}
\begin{aligned}
\ket{a_2} + i \ket{a_3} &= \ket{b_2} \\
\ket{a_2} - i \ket{a_3} &= \ket{b_3},
\end{aligned}
\end{equation}

so the basis \( \setlr{ \ket{b_i}} \) is a simulaneous eigenspace for both \( A \) and \(B\).  Because there is a simulaneous eigenspace, the matrices must commute.  This can be confirmed with direct computation

\begin{dmath}\label{eqn:moreKet:100}
A B = a b 
\begin{bmatrix}
1 & 0 & 0 \\
0 & -1 & 0 \\
0 & 0 & -1
\end{bmatrix}
\begin{bmatrix}
1 & 0 & 0 \\
0 & 0 & -i \\
0 & i & 0
\end{bmatrix}
=
a b
\begin{bmatrix}
1 & 0 & 0 \\
0 & 0 & i \\
0 & -i & 0
\end{bmatrix},
\end{dmath}

and

\begin{dmath}\label{eqn:moreKet:120}
B A = a b 
\begin{bmatrix}
1 & 0 & 0 \\
0 & 0 & -i \\
0 & i & 0
\end{bmatrix}
\begin{bmatrix}
1 & 0 & 0 \\
0 & -1 & 0 \\
0 & 0 & -1
\end{bmatrix}
=
a b
\begin{bmatrix}
1 & 0 & 0 \\
0 & 0 & i \\
0 & -i & 0
\end{bmatrix}.
\end{dmath}

} % answer

\EndArticle
