%
% Copyright � 2013 Peeter Joot.  All Rights Reserved.
% Licenced as described in the file LICENSE under the root directory of this GIT repository.
%
\newcommand{\authorname}{Peeter Joot}
\newcommand{\email}{peeterjoot@protonmail.com}
\newcommand{\basename}{FIXMEbasenameUndefined}
\newcommand{\dirname}{notes/FIXMEdirnameUndefined/}

\renewcommand{\basename}{basicStatMechLecture4}
\renewcommand{\dirname}{notes/phy452/}
\newcommand{\keywords}{Statistical mechanics, PHY452H1S}
\newcommand{\authorname}{Peeter Joot}
\newcommand{\onlineurl}{http://sites.google.com/site/peeterjoot2/math2013/\basename.pdf}
\newcommand{\sourcepath}{\dirname\basename.tex}
\newcommand{\generatetitle}[1]{\chapter{#1}}

\newcommand{\vcsinfo}{%
\section*{}
\noindent{\color{DarkOliveGreen}{\rule{\linewidth}{0.1mm}}}
\paragraph{Document version}
%\paragraph{\color{Maroon}{Document version}}
{
\small
\begin{itemize}
\item Available online at:\\ 
\href{\onlineurl}{\onlineurl}
\item Git Repository: \input{./.revinfo/gitRepo.tex}
\item Source: \sourcepath
\item last commit: \input{./.revinfo/gitCommitString.tex}
\item commit date: \input{./.revinfo/gitCommitDate.tex}
\end{itemize}
}
}

%\PassOptionsToPackage{dvipsnames,svgnames}{xcolor}
\PassOptionsToPackage{square,numbers}{natbib}
\documentclass{scrreprt}

\usepackage[left=2cm,right=2cm]{geometry}
\usepackage[svgnames]{xcolor}
\usepackage{peeters_layout}

\usepackage{natbib}

\usepackage[
colorlinks=true,
bookmarks=false,
pdfauthor={\authorname, \email},
backref 
]{hyperref}

% http://tex.stackexchange.com/questions/75773/how-to-reference-problems-by-the-text-label-in-an-exercise-envioronment
\usepackage[english]{cleveref}
\crefname{Exercise}{exercise}{exercises}
\Crefname{Exercise}{Exercise}{Exercises}

\RequirePackage{titlesec}
\RequirePackage{ifthen}

% http://stackoverflow.com/questions/4932910/date-in-the-tabular-environment
\makeatletter
\let\insertdate\@date
\makeatother

\titleformat{\chapter}[display]
{\bfseries\Large}
{\color{DarkSlateGrey}\filleft \authorname
\ifthenelse{\isundefined{\studentnumber}}{}{\\ \studentnumber}
\ifthenelse{\isundefined{\email}}{}{\\ \email}
\ifthenelse{\isundefined{\dateintitle}}{}{\\ \insertdate}
%\ifthenelse{\isundefined{\coursename}}{}{\\ \coursename} % put in title instead.
}
{4ex}
{\color{DarkOliveGreen}{\titlerule}\color{Maroon}
\vspace{2ex}%
\filright}
[\vspace{2ex}%
\color{DarkOliveGreen}\titlerule
]

\newcommand{\beginArtWithToc}[0]{\begin{document}\tableofcontents}
\newcommand{\beginArtNoToc}[0]{\begin{document}}
\newcommand{\EndNoBibArticle}[0]{\end{document}}
\newcommand{\EndArticle}[0]{\bibliography{Bibliography}\bibliographystyle{plainnat}\end{document}}

% 
%\newcommand{\citep}[1]{\cite{#1}}

\colorSectionsForArticle



%\usepackage[draft]{fixme}
%\fxusetheme{color}
%\fxwarning{review lecture 4}{work through this lecture in detail.}

\beginArtNoToc
\generatetitle{PHY452H1S Basic Statistical Mechanics.  Lecture 4: Maxwell distribution for gases.  Taught by Prof.\ Arun Paramekanti}
%\chapter{Maxwell distribution for gases}
\label{chap:basicStatMechLecture4}

%\section{Disclaimer}
%
%Peeter's lecture notes from class.  May not be entirely coherent.

\section{Review: Lead up to Maxwell distribution for gases}

After a number of collisions, our $i$th particle will have a velocity

\begin{equation}\label{eqn:basicStatMechLecture4:20}
\Bv_i(N_{\mathrm{c}}) = \Bv_i(0) + \sum_{l = 1}^{N_{\mathrm{c}}} \Delta \Bv_i(l)
\end{equation}

We argued that the probability distribution for finding a velocity $\Bv$ was as in

\begin{equation}\label{eqn:basicStatMechLecture4:40}
\mathcal{P}_{N_{\mathrm{c}}}(\Bv_i) \propto \exp
\left(
-\frac{(\Bv_i - \Bv_i(0))^2}{ 2 N_{\mathrm{c}}}
\right)
\end{equation}

%\cref{fig:basicStatMechLecture4:basicStatMechLecture4Fig1}.
\imageFigure{basicStatMechLecture4Fig1}{Velocity distribution found without considering kinetic energy conservation}{fig:basicStatMechLecture4:basicStatMechLecture4Fig1}{0.3}

\section{What went wrong?}

However, we know that this must be wrong, since we require

\begin{equation}\label{eqn:basicStatMechLecture4:60}
T = \inv{2} \sum_{i = 1}^{n} \Bv_i^2 = \mbox{conserved}
\end{equation}

Where our arguement went wrong is that when the particle has a greater than average velocity, the effect of a collision will be to slow it down.  We have to account for 

\begin{itemize}
\item Fluctuations $\rightarrow$ ``random walk''
\item Dissipation $\rightarrow$ ``slowing down''
\end{itemize}

There were two ingredients to diffusion (the random walk), these were

\begin{itemize}
\item 
Conservation of particles

\begin{equation}\label{eqn:basicStatMechLecture4:80}
\PD{t}{c} + \PD{x}{j} = 0
\end{equation}

We can also think about a conservation of a particles in a velocity space
\begin{equation}\label{eqn:basicStatMechLecture4:100}
\PD{t}{c}(v, t) + \PD{v}{j_v} = 0
\end{equation}

where $j_v$ is a probability current in this velocity space.
\item Fick's law in velocity space takes the form

\begin{equation}\label{eqn:basicStatMechLecture4:120}
j_v = -D \PD{v}{c}(v, t)
\end{equation}
\end{itemize}

The diffusion results in an ``attempt'' to flatten the distribution of the concentration as in \cref{fig:basicStatMechLecture4:basicStatMechLecture4Fig2}.

\imageFigure{basicStatMechLecture4Fig2}{A friction like term is require to oppose the diffusion pressure}{fig:basicStatMechLecture4:basicStatMechLecture4Fig2}{0.3}

We'd like to add to the diffusion current an extra frictional like term

\begin{equation}\label{eqn:basicStatMechLecture4:140}
j_v = -D \PD{v}{c}(v, t) - \eta v c(v)
\end{equation}

We want something directed opposite to the velocity and the concentration

\begin{subequations}
\begin{equation}\label{eqn:basicStatMechLecture4:180}
\text{Diffusion current} \equiv -D \PD{v}{c}(v, t)
\end{equation}
\begin{equation}\label{eqn:basicStatMechLecture4:200}
\text{Dissipation current} \equiv - \eta v c(v, t)
\end{equation}
\end{subequations}

This gives

\begin{dmath}\label{eqn:basicStatMechLecture4:160}
\PD{t}{c}(v, t) 
= -\PD{v}{j_v}
= -\PD{v}{} 
\left(
-D \PD{v}{c}(v, t) - \eta v c(v)
\right)
= D \PDSq{v}{c}(v, t) + \eta \PD{v}{}
\left(
v c(v, t)
\right)
\end{dmath}

Can we find a steady state solution to this equation when $t \rightarrow \infty$?  For such a steady state we have

\begin{equation}\label{eqn:basicStatMechLecture4:220}
0 = \frac{d^2 c}{dv^2} + \eta \frac{d}{dv} 
\left(
v c
\right)
\end{equation}

Integrating once we have

\begin{equation}\label{eqn:basicStatMechLecture4:240}
\frac{d c}{dv} = -\eta v c + \text{constant}
\end{equation}

supposing that $dc/dv = 0$ at $v = 0$, integrating once more we have

\begin{equation}\label{eqn:basicStatMechLecture4:260}
\boxed{
c(v) \propto \exp
\left(
- \frac{\eta v^2}{2 D}
\right).
}
\end{equation}

This is the \underline{Maxwell-Boltzman} distribution, illustrated in \cref{fig:basicStatMechLecture4:basicStatMechLecture4Fig3}.

\imageFigure{basicStatMechLecture4Fig3}{Maxwell-Boltzman distribution}{fig:basicStatMechLecture4:basicStatMechLecture4Fig3}{0.3}

The concentration $c(v)$ has a probability distribution.

Calculating $\expectation{v^2}$ from this distribution we can identify the $D/\eta$ factor.

\begin{dmath}\label{eqn:basicStatMechLecture4:520}
\expectation{v^2} 
= \frac{\int v^2 e^{- \eta v^2/2D} dv}{
\int e^{- \eta v^2/2D} dv
}
= 
\frac{
\frac{D}{\eta} 
\int v \frac{d}{dv} \left( 
-e^{- \eta v^2/2D} \right) dv
}{
\int e^{- \eta v^2/2D} dv
}
= -
\frac{D}{\eta} 
\frac{
\int -e^{- \eta v^2/2D} dv
}{
\int e^{- \eta v^2/2D} dv
}
= 
\frac{D}{\eta}.
\end{dmath}

This also happens to be the energy in terms of temperature (we can view this as a definition of the temperature for now), writing

\begin{equation}\label{eqn:basicStatMechLecture4:280}
\inv{2} m \expectation{\Bv^2} = \inv{2} m \left( \frac{D}{\eta} \right) = \inv{2} k_{\mathrm{B}} T
\end{equation}

Here

\begin{subequations}
\begin{equation}\label{eqn:basicStatMechLecture4:300}
k_{\mathrm{B}} = \mbox{Boltzmann constant}
\end{equation}
\begin{equation}\label{eqn:basicStatMechLecture4:320}
T = \mbox{absolute temperature}
\end{equation}
\end{subequations}

\section{Equilibrium steady states}

Fluctations $\leftrightarrow$ Dissipation

\begin{equation}\label{eqn:basicStatMechLecture4:340}
\boxed{
\frac{D}{\eta} = \frac{ k_{\mathrm{B}} T}{m}
}
\end{equation}

This is a specific example of the more general \underline{Fluctuation-Dissipation theorem}.

\paragraph{Generalizing to 3D}

Fick's law and the continuity equation in 3D are respectively

\begin{subequations}
\begin{equation}\label{eqn:basicStatMechLecture4:360}
\Bj = -D \spacegrad_\Bv c(\Bv, t) - \eta \Bv c(\Bv, t)
\end{equation}
\begin{equation}\label{eqn:basicStatMechLecture4:380}
\PD{t}{} c(\Bv, t) + \spacegrad_\Bv \cdot \Bj(\Bv, t) = 0
\end{equation}
\end{subequations}

As above we have for the steady state

\begin{dmath}\label{eqn:basicStatMechLecture4:540}
0 = \PD{t}{} c(\Bv, t) = \spacegrad_\Bv \cdot 
\left(
 -D \spacegrad_\Bv c(\Bv, t) - \eta \Bv c(\Bv, t)
\right)
= -D \spacegrad^2_\Bv c - \eta \spacegrad_\Bv \cdot (\Bv c)
\end{dmath}

Integrating once over all space

\begin{equation}\label{eqn:basicStatMechLecture4:560}
D \spacegrad_\Bv c = -\eta \Bv c + \text{vector constant, assumed zero}
\end{equation}

This is three sets of equations, one for each component $v_\alpha$ of $\Bv$

\begin{equation}\label{eqn:basicStatMechLecture4:580}
\PD{v_\alpha}{c} = -\frac{\eta}{D} v_\alpha c 
\end{equation}

So that our steady state equation is

\begin{equation}\label{eqn:basicStatMechLecture4:400}
c(\Bv, t \rightarrow \infty) \propto \exp
\left(
-\frac{
v_x^2 
+
v_y^2 
+
v_z^2 
}{ 2 (D/\eta) }
\right).
\end{equation}

and we can find

\begin{equation}\label{eqn:basicStatMechLecture4:420}
\inv{2} m \expectation{\Bv \cdot \Bv} = \frac{3}{2} k_{\mathrm{B}} T
\end{equation}

\section{Phase space}

Now let's switch directions a bit and look at how to examine a more general system described by the phase space of generalized coordinates

\begin{equation}\label{eqn:basicStatMechLecture4:440}
\{ 
x_{i_\alpha}(t),
p_{i_\alpha}(t)
 \}
\end{equation}

Here
\begin{itemize}
\item $i = \mbox{molecule or particle number}$
\item $\alpha \in \{x, y, z\}$
\item $\mbox{Dimension} = N_{\text{particles} \times 2 \times d}$, where $d$ is the physical space dimension.
\end{itemize}

The motion in phase space will be governed by the knowledge of how each of these coordinates change for ever particle

\begin{subequations}
\begin{equation}\label{eqn:basicStatMechLecture4:460}
\ddt{} x_{i_\alpha}(t) 
= \PD{p_{i_\alpha}}{H}
\end{equation}
\begin{equation}\label{eqn:basicStatMechLecture4:480}
\ddt{} p_{i_\alpha}(t)
= -\PD{x_{i_\alpha}}{H}
\end{equation}
\end{subequations}

Example, 1D SHO

\begin{equation}\label{eqn:basicStatMechLecture4:500}
H = \inv{2} \frac{p^2}{m} + \inv{2} k x^2
\end{equation}

This has phase space trajectories as in \cref{fig:basicStatMechLecture4:basicStatMechLecture4Fig4}.

\imageFigure{basicStatMechLecture4Fig4}{Classical SHO phase space trajectories}{fig:basicStatMechLecture4:basicStatMechLecture4Fig4}{0.3}

%\EndArticle
\EndNoBibArticle
