%
% Copyright � 2016 Peeter Joot.  All Rights Reserved.
% Licenced as described in the file LICENSE under the root directory of this GIT repository.
%
%{
\newcommand{\authorname}{Peeter Joot}
\newcommand{\email}{peeterjoot@protonmail.com}
\newcommand{\basename}{FIXMEbasenameUndefined}
\newcommand{\dirname}{notes/FIXMEdirnameUndefined/}

\renewcommand{\basename}{curlAndDivergenceVectorUniqueness}
\renewcommand{\dirname}{notes/phy1520/}
%\newcommand{\dateintitle}{}
%\newcommand{\keywords}{}

\newcommand{\authorname}{Peeter Joot}
\newcommand{\onlineurl}{http://sites.google.com/site/peeterjoot2/math2013/\basename.pdf}
\newcommand{\sourcepath}{\dirname\basename.tex}
\newcommand{\generatetitle}[1]{\chapter{#1}}

\newcommand{\vcsinfo}{%
\section*{}
\noindent{\color{DarkOliveGreen}{\rule{\linewidth}{0.1mm}}}
\paragraph{Document version}
%\paragraph{\color{Maroon}{Document version}}
{
\small
\begin{itemize}
\item Available online at:\\ 
\href{\onlineurl}{\onlineurl}
\item Git Repository: \input{./.revinfo/gitRepo.tex}
\item Source: \sourcepath
\item last commit: \input{./.revinfo/gitCommitString.tex}
\item commit date: \input{./.revinfo/gitCommitDate.tex}
\end{itemize}
}
}

%\PassOptionsToPackage{dvipsnames,svgnames}{xcolor}
\PassOptionsToPackage{square,numbers}{natbib}
\documentclass{scrreprt}

\usepackage[left=2cm,right=2cm]{geometry}
\usepackage[svgnames]{xcolor}
\usepackage{peeters_layout}

\usepackage{natbib}

\usepackage[
colorlinks=true,
bookmarks=false,
pdfauthor={\authorname, \email},
backref 
]{hyperref}

% http://tex.stackexchange.com/questions/75773/how-to-reference-problems-by-the-text-label-in-an-exercise-envioronment
\usepackage[english]{cleveref}
\crefname{Exercise}{exercise}{exercises}
\Crefname{Exercise}{Exercise}{Exercises}

\RequirePackage{titlesec}
\RequirePackage{ifthen}

% http://stackoverflow.com/questions/4932910/date-in-the-tabular-environment
\makeatletter
\let\insertdate\@date
\makeatother

\titleformat{\chapter}[display]
{\bfseries\Large}
{\color{DarkSlateGrey}\filleft \authorname
\ifthenelse{\isundefined{\studentnumber}}{}{\\ \studentnumber}
\ifthenelse{\isundefined{\email}}{}{\\ \email}
\ifthenelse{\isundefined{\dateintitle}}{}{\\ \insertdate}
%\ifthenelse{\isundefined{\coursename}}{}{\\ \coursename} % put in title instead.
}
{4ex}
{\color{DarkOliveGreen}{\titlerule}\color{Maroon}
\vspace{2ex}%
\filright}
[\vspace{2ex}%
\color{DarkOliveGreen}\titlerule
]

\newcommand{\beginArtWithToc}[0]{\begin{document}\tableofcontents}
\newcommand{\beginArtNoToc}[0]{\begin{document}}
\newcommand{\EndNoBibArticle}[0]{\end{document}}
\newcommand{\EndArticle}[0]{\bibliography{Bibliography}\bibliographystyle{plainnat}\end{document}}

% 
%\newcommand{\citep}[1]{\cite{#1}}

\colorSectionsForArticle



\usepackage{peeters_layout_exercise}
\usepackage{peeters_braket}
\usepackage{peeters_figures}
\usepackage{siunitx}

\beginArtNoToc

\generatetitle{Does the divergence and curl uniquely determine the vector}
%\chapter{Does the divergence and curl uniquely determine the vector}
%\label{chap:curlAndDivergenceVectorUniqueness}
% \citep{sakurai2014modern} pr X.Y
% \citep{pozar2009microwave}
% \citep{qftLectureNotes}
% \citep{griffiths1999introduction}

Here's a homework question from ece1228 that screams for solution using Geometric Algebra techniques.  If I submitted such a solution, then my Prof probably wouldn't know what I was doing, so I'll probably also have to try to solve it another way too.

\makeproblem{Helmholtz theorem}{emt:problemSet1:5}{ 
Prove the first Helmholtz's theorem, i.e. if vector \(\BM_1\) is defined by its divergence

\begin{dmath}\label{eqn:emtProblemSet1Problem5:20}
\spacegrad \cdot \BM_1 = s
\end{dmath}

and its curl
\begin{dmath}\label{eqn:emtProblemSet1Problem5:40}
\spacegrad \cross \BM_1 = \BC 
\end{dmath}

within a region and its normal component \( \BM_{1\txtn} \) over the boundary, then \( \BM_1 \) is 
uniquely specified.

Note: Assume there is a vector \( \BM_2 \) with its divergence and curl equal to \( s \) and \( \BC \)
respectively, then show that \( \BM_1 = \BM_2 \) .
} % makeproblem

\makeanswer{emt:problemSet1:5}{ 

Dropping the suffix from \( \BM_1 \), the gradient of the vector can be written as a single even grade multivector

\begin{dmath}\label{eqn:curlAndDivergenceVectorUniqueness:60}
\spacegrad \BM
= \spacegrad \cdot \BM + I \spacegrad \cross \BM
= s + I \BC.
\end{dmath}

Unlike the divergence or curl, the gradient is invertible, with the \R{3} Green's function

\begin{dmath}\label{eqn:curlAndDivergenceVectorUniqueness:80}
\begin{aligned}
G(\Bx ; \By) &= \inv{4 \pi} \frac{ \Bx - \By }{\Abs{\Bx - \By}^3} \\
\spacegrad \BG &= \spacegrad \cdot \BG = \delta(\Bx - \By).
\end{aligned}
\end{dmath}

This Green's function result is taken from \citep{doran2003gap}, where it is used to generalize the Cauchy integral equations to higher dimensions.

The inversion equation is an application of the Fundamental Theorem of (Geometric) Calculus

\begin{dmath}\label{eqn:curlAndDivergenceVectorUniqueness:100}
\oint_{\partial V} G d^2 \Bx \BM
=
\int_V G d^3 \Bx \lrspacegrad \BM 
=
\int_V d^3 \Bx (G \spacegrad) \BM 
+
\int_V d^3 \Bx G (\spacegrad \BM)
=
\int_V d^3 \Bx \delta(\Bx - \By) \BM 
+
\int_V d^3 \Bx G \lr{ s + I \BC }.
\end{dmath}

In the first volume integral the gradient operates bidirectionally on both \( G \) and \( \BM \).  Because \( d^3 \Bx \propto I \) in \R{3} we can commute it with any grades.  With \( d^3 \Bx = I dV \), and \( d^2 \Bx \ncap = I dA \), we have

\begin{dmath}\label{eqn:curlAndDivergenceVectorUniqueness:120}
\BM(\By)
=
\inv{4\pi} \oint_{\partial V} dA \frac{ \Bx - \By }{\Abs{\Bx - \By}^3} \ncap \BM
-
\inv{4\pi} \int_V dV \frac{ \Bx - \By }{\Abs{\Bx - \By}^3} \lr{ s + I \BC }.
\end{dmath}

The trivectors grades on the RHS must sum to zero.  Omitting those for now, 
%Since \( \BM \) is a vector the vector and trivector grades on the RHS must sum to zero.  
with \( \Br = \Bx - \By \), the vector grade selection is

\begin{dmath}\label{eqn:curlAndDivergenceVectorUniqueness:120}
\BM(\By)
=
\inv{4\pi} \oint_{\partial V} dA \inv{ r^2 } \gpgradeone{ \rcap \ncap \BM }
-
\inv{4\pi} \int_V dV \frac{ s \rcap }{ r^2 } 
-
\inv{4\pi} \int_V dV \frac{ \rcap \cdot (I\BC) }{ r^2 }
=
\inv{4\pi} \oint_{\partial V} dA \inv{ r^2 } \gpgradeone{ \rcap \ncap \BM }
+
\inv{4\pi} \int_V dV \frac{ \rcap \cross \BC - s \rcap }{ r^2 },
\end{dmath}

where the cross product comes from the expansion

\begin{dmath}\label{eqn:curlAndDivergenceVectorUniqueness:140}
\rcap \cdot (I\BC)
=
\gpgradeone{ \rcap I\BC }
=
I (\rcap \wedge \BC)
=
-\rcap \cross \BC.
\end{dmath}

The vector grade selection can be expanded as 

\begin{dmath}\label{eqn:curlAndDivergenceVectorUniqueness:220}
\gpgradeone{ \rcap \ncap \BM }
=
\lr{ \rcap (\ncap \cdot \BM) + \rcap \cdot (\ncap \wedge \BM) }
=
\rcap (\ncap \cdot \BM) + (\rcap \cdot \ncap) \BM - \ncap ( \rcap \cdot \BM ).
\end{dmath}

This looks like it is probably largest when all of \( \rcap, \ncap, \BM \) are colinear, so will likely have magnitude \( \gpgradeone{ \rcap \ncap \BM } < \Abs{\BM} \).  What we can say for sure is 

\begin{dmath}\label{eqn:curlAndDivergenceVectorUniqueness:240}
\Abs{\lr{ \inv{4\pi} \oint_{\partial V} dA \inv{ r^2 } \gpgradeone{ \rcap \ncap \BM } }}
\ge
\inv{4\pi} \oint_{\partial V} dA \inv{ r^2 } \Abs{ \gpgradeone{ \rcap \ncap \BM } }
\ge
\inv{4\pi} \oint_{\partial V} dA \inv{ r^2 } 3 \Abs{ \BM },
\end{dmath}

so if the magnitude \( \Abs{\BM} \ll r^2 \) as \( r \rightarrow \infty \), this whole surface integral will be killed,
leaving \( \BM \) uniquely determined by

%\begin{boxed}\label{eqn:curlAndDivergenceVectorUniqueness:200}
\boxedEquation{eqn:curlAndDivergenceVectorUniqueness:200}{
\BM(\Bx)
=
\inv{4\pi} \int_V dV' \frac{ \BC(\Bx') \cross (\Bx - \Bx') + (\Bx - \Bx') s(\Bx')}{ \Abs{\Bx - \Bx'}^3 }.
}
%\end{boxed}

How about the trivector grade component of \cref{eqn:curlAndDivergenceVectorUniqueness:120}?  For that to be zero means that we must have

\begin{dmath}\label{eqn:curlAndDivergenceVectorUniqueness:160}
\int_V dV \frac{ \rcap \cdot \BC }{ r^2 }
=
\oint_{\partial V} dA \inv{ r^2 } I ( \rcap \wedge \ncap \wedge \BM )
=
\oint_{\partial V} dA \inv{ r^2 } (\ncap \cross \rcap) \cdot \BM.
\end{dmath}

With the same limiting argument as above, the area integral is killed in the limit, meaning that \( \rcap \cdot \BC = 0 \) is required for the trivector grade selection to be zero.  I'm not quite sure what to make of that. 
}

%}
\EndArticle
