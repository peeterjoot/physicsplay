%
% Copyright � 2016 Peeter Joot.  All Rights Reserved.
% Licenced as described in the file LICENSE under the root directory of this GIT repository.
%
%{
\newcommand{\authorname}{Peeter Joot}
\newcommand{\email}{peeterjoot@protonmail.com}
\newcommand{\basename}{FIXMEbasenameUndefined}
\newcommand{\dirname}{notes/FIXMEdirnameUndefined/}

\renewcommand{\basename}{trivectorBivectorProduct}
\renewcommand{\dirname}{notes/phy1520/}
%\newcommand{\dateintitle}{}
%\newcommand{\keywords}{}

\newcommand{\authorname}{Peeter Joot}
\newcommand{\onlineurl}{http://sites.google.com/site/peeterjoot2/math2013/\basename.pdf}
\newcommand{\sourcepath}{\dirname\basename.tex}
\newcommand{\generatetitle}[1]{\chapter{#1}}

\newcommand{\vcsinfo}{%
\section*{}
\noindent{\color{DarkOliveGreen}{\rule{\linewidth}{0.1mm}}}
\paragraph{Document version}
%\paragraph{\color{Maroon}{Document version}}
{
\small
\begin{itemize}
\item Available online at:\\ 
\href{\onlineurl}{\onlineurl}
\item Git Repository: \input{./.revinfo/gitRepo.tex}
\item Source: \sourcepath
\item last commit: \input{./.revinfo/gitCommitString.tex}
\item commit date: \input{./.revinfo/gitCommitDate.tex}
\end{itemize}
}
}

%\PassOptionsToPackage{dvipsnames,svgnames}{xcolor}
\PassOptionsToPackage{square,numbers}{natbib}
\documentclass{scrreprt}

\usepackage[left=2cm,right=2cm]{geometry}
\usepackage[svgnames]{xcolor}
\usepackage{peeters_layout}

\usepackage{natbib}

\usepackage[
colorlinks=true,
bookmarks=false,
pdfauthor={\authorname, \email},
backref 
]{hyperref}

% http://tex.stackexchange.com/questions/75773/how-to-reference-problems-by-the-text-label-in-an-exercise-envioronment
\usepackage[english]{cleveref}
\crefname{Exercise}{exercise}{exercises}
\Crefname{Exercise}{Exercise}{Exercises}

\RequirePackage{titlesec}
\RequirePackage{ifthen}

% http://stackoverflow.com/questions/4932910/date-in-the-tabular-environment
\makeatletter
\let\insertdate\@date
\makeatother

\titleformat{\chapter}[display]
{\bfseries\Large}
{\color{DarkSlateGrey}\filleft \authorname
\ifthenelse{\isundefined{\studentnumber}}{}{\\ \studentnumber}
\ifthenelse{\isundefined{\email}}{}{\\ \email}
\ifthenelse{\isundefined{\dateintitle}}{}{\\ \insertdate}
%\ifthenelse{\isundefined{\coursename}}{}{\\ \coursename} % put in title instead.
}
{4ex}
{\color{DarkOliveGreen}{\titlerule}\color{Maroon}
\vspace{2ex}%
\filright}
[\vspace{2ex}%
\color{DarkOliveGreen}\titlerule
]

\newcommand{\beginArtWithToc}[0]{\begin{document}\tableofcontents}
\newcommand{\beginArtNoToc}[0]{\begin{document}}
\newcommand{\EndNoBibArticle}[0]{\end{document}}
\newcommand{\EndArticle}[0]{\bibliography{Bibliography}\bibliographystyle{plainnat}\end{document}}

% 
%\newcommand{\citep}[1]{\cite{#1}}

\colorSectionsForArticle



\usepackage{peeters_layout_exercise}
\usepackage{peeters_braket}
\usepackage{peeters_figures}
\usepackage{siunitx}

\beginArtNoToc

\generatetitle{XXX}
%\chapter{XXX}
%\label{chap:trivectorBivectorProduct}
% \citep{sakurai2014modern} pr X.Y
% \citep{pozar2009microwave}
% \citep{qftLectureNotes}

I can only guess that by verify you mean to expand the product in terms of it's component grades.  That expansion takes the form

\begin{dmath}\label{eqn:trivectorBivectorProduct:20}
\lr{ a \wedge b \wedge c } \lr{ d \wedge e }
=
\lr{ a \wedge b \wedge c } \cdot \lr{ d \wedge e }
+
\gpgradethree{ \lr{ a \wedge b \wedge c } \lr{ d \wedge e } }
+
a \wedge b \wedge c \wedge d \wedge e 
\end{dmath}

The first term is a vector and expands as

\begin{dmath}\label{eqn:trivectorBivectorProduct:40}
\lr{ a \wedge b \wedge c } \cdot \lr{ d \wedge e }
=
\lr{ \lr{ a \wedge b \wedge c } \cdot d } \cdot e
=
\lr{ 
 a \wedge b (c \cdot d) 
+b \wedge c (a \cdot d) 
+c \wedge a (b \cdot d) 
} \cdot e
=
 a (b \cdot e) (c \cdot d) 
+b (c \cdot e) (a \cdot d) 
+c (a \cdot e) (b \cdot d) 
-(a\cdot e) b (c \cdot d) 
-(b\cdot e) c (a \cdot d) 
-(c\cdot e) a (b \cdot d) 
=
 a \lr{ (b \cdot e) (c \cdot d) -(c\cdot e) (b \cdot d) }
+b \lr{ (c \cdot e) (a \cdot d) -(a\cdot e) (c \cdot d) }
+c \lr{ (a \cdot e) (b \cdot d) -(b\cdot e) (a \cdot d) }.
\end{dmath}

%}
\EndArticle
%\EndNoBibArticle
