%
% Copyright � 2015 Peeter Joot.  All Rights Reserved.
% Licenced as described in the file LICENSE under the root directory of this GIT repository.
%
\newcommand{\authorname}{Peeter Joot}
\newcommand{\email}{peeterjoot@protonmail.com}
\newcommand{\basename}{FIXMEbasenameUndefined}
\newcommand{\dirname}{notes/FIXMEdirnameUndefined/}

\renewcommand{\basename}{translation}
\renewcommand{\dirname}{notes/phy1520/}
%\newcommand{\dateintitle}{}
%\newcommand{\keywords}{}

\newcommand{\authorname}{Peeter Joot}
\newcommand{\onlineurl}{http://sites.google.com/site/peeterjoot2/math2013/\basename.pdf}
\newcommand{\sourcepath}{\dirname\basename.tex}
\newcommand{\generatetitle}[1]{\chapter{#1}}

\newcommand{\vcsinfo}{%
\section*{}
\noindent{\color{DarkOliveGreen}{\rule{\linewidth}{0.1mm}}}
\paragraph{Document version}
%\paragraph{\color{Maroon}{Document version}}
{
\small
\begin{itemize}
\item Available online at:\\ 
\href{\onlineurl}{\onlineurl}
\item Git Repository: \input{./.revinfo/gitRepo.tex}
\item Source: \sourcepath
\item last commit: \input{./.revinfo/gitCommitString.tex}
\item commit date: \input{./.revinfo/gitCommitDate.tex}
\end{itemize}
}
}

%\PassOptionsToPackage{dvipsnames,svgnames}{xcolor}
\PassOptionsToPackage{square,numbers}{natbib}
\documentclass{scrreprt}

\usepackage[left=2cm,right=2cm]{geometry}
\usepackage[svgnames]{xcolor}
\usepackage{peeters_layout}

\usepackage{natbib}

\usepackage[
colorlinks=true,
bookmarks=false,
pdfauthor={\authorname, \email},
backref 
]{hyperref}

% http://tex.stackexchange.com/questions/75773/how-to-reference-problems-by-the-text-label-in-an-exercise-envioronment
\usepackage[english]{cleveref}
\crefname{Exercise}{exercise}{exercises}
\Crefname{Exercise}{Exercise}{Exercises}

\RequirePackage{titlesec}
\RequirePackage{ifthen}

% http://stackoverflow.com/questions/4932910/date-in-the-tabular-environment
\makeatletter
\let\insertdate\@date
\makeatother

\titleformat{\chapter}[display]
{\bfseries\Large}
{\color{DarkSlateGrey}\filleft \authorname
\ifthenelse{\isundefined{\studentnumber}}{}{\\ \studentnumber}
\ifthenelse{\isundefined{\email}}{}{\\ \email}
\ifthenelse{\isundefined{\dateintitle}}{}{\\ \insertdate}
%\ifthenelse{\isundefined{\coursename}}{}{\\ \coursename} % put in title instead.
}
{4ex}
{\color{DarkOliveGreen}{\titlerule}\color{Maroon}
\vspace{2ex}%
\filright}
[\vspace{2ex}%
\color{DarkOliveGreen}\titlerule
]

\newcommand{\beginArtWithToc}[0]{\begin{document}\tableofcontents}
\newcommand{\beginArtNoToc}[0]{\begin{document}}
\newcommand{\EndNoBibArticle}[0]{\end{document}}
\newcommand{\EndArticle}[0]{\bibliography{Bibliography}\bibliographystyle{plainnat}\end{document}}

% 
%\newcommand{\citep}[1]{\cite{#1}}

\colorSectionsForArticle



\usepackage{peeters_layout_exercise}
\usepackage{peeters_braket}
\usepackage{peeters_figures}

\beginArtNoToc

\generatetitle{Translation operator problems}
%\chapter{Translation operator problems}
%\label{chap:translation}

\makeoproblem{Polynomial commutators.}{problem:translation:29}{\citep{sakurai2014modern} pr. 1.29}{

\makesubproblem{}{problem:translation:29:a}
For power series \( F, G \), verify

\begin{equation}\label{eqn:translation:180}
\antisymmetric{x_i}{G(\Bp)} = i \Hbar \PD{p_i}{G}, \qquad
\antisymmetric{p_i}{F(\Bx)} = -i \Hbar \PD{x_i}{F}.
\end{equation}

\makesubproblem{}{problem:translation:29:b}

Evaluate \( \antisymmetric{x^2}{p^2} \), and compare to the classical Poisson bracket \( \antisymmetric{x^2}{p^2}_{\textrm{classical}} \).

} % problem

\makeanswer{problem:translation:29}{
\makeSubAnswer{}{problem:translation:29:a}
\makeSubAnswer{}{problem:translation:29:b}
} % answer

\makeoproblem{Translation operator and position expectation.}{problem:translation:30}{\citep{sakurai2014modern} pr. 1.30}{

The translation operator for a finite spatial displacement is given by

\begin{dmath}\label{eqn:translation:20}
J(\Bl) = \exp\lr{ -i \Bp \cdot \Bl/\Hbar },
\end{dmath}

where \( \Bp \) is the momentum operator.

\makesubproblem{}{problem:translation:1.30:a}

Evaluate 

\begin{dmath}\label{eqn:translation:40}
\antisymmetric{x_i}{J(\Bl)}.
\end{dmath}

\makesubproblem{}{problem:translation:1.30:b}
Demonstrate how the expectation value \( \expectation{\Bx} \) changes under translation.

} % problem

\makeanswer{problem:translation:30}{

\makeSubAnswer{}{problem:translation:1.30:a}

For clarity, let's set \( x_i = y \).  The general result will be clear despite doing so.

\begin{dmath}\label{eqn:translation:60}
\antisymmetric{y}{J(\Bl)}
=
\sum_{k= 0} \inv{k!} \lr{\frac{-i}{\Hbar}} 
\antisymmetric{y}{
\lr{ \Bp \cdot \Bl }^k
}.
\end{dmath}

The commutator expands as

\begin{dmath}\label{eqn:translation:80}
\antisymmetric{y}{
\lr{ \Bp \cdot \Bl }^k 
}
+ \lr{ \Bp \cdot \Bl }^k y
=
y \lr{ \Bp \cdot \Bl }^k 
=
y \lr{ p_x l_x + p_y l_y + p_z l_z } \lr{ \Bp \cdot \Bl }^{k-1} 
=
\lr{ p_x l_x y + y p_y l_y + p_z l_z y } \lr{ \Bp \cdot \Bl }^{k-1} 
=
\lr{ p_x l_x y + l_y \lr{ p_y y + i \Hbar } + p_z l_z y } \lr{ \Bp \cdot \Bl }^{k-1} 
=
\lr{ \Bp \cdot \Bl } y \lr{ \Bp \cdot \Bl }^{k-1} 
+ i \Hbar l_y \lr{ \Bp \cdot \Bl }^{k-1} 
= \cdots
=
\lr{ \Bp \cdot \Bl }^{k-1} y \lr{ \Bp \cdot \Bl }^{k-(k-1)} 
+ (k-1) i \Hbar l_y \lr{ \Bp \cdot \Bl }^{k-1} 
=
\lr{ \Bp \cdot \Bl }^{k} y 
+ k i \Hbar l_y \lr{ \Bp \cdot \Bl }^{k-1}.
\end{dmath}

In the above expansion, the commutation of \( y \) with \( p_x, p_z \) has been used.  This gives, for \( k \ne 0 \),

\begin{dmath}\label{eqn:translation:100}
\antisymmetric{y}{
\lr{ \Bp \cdot \Bl }^k 
}
=
k i \Hbar l_y \lr{ \Bp \cdot \Bl }^{k-1}.
\end{dmath}

Note that this also holds for the \( k = 0 \) case, since \( y \) commutes with the identity operator.  Plugging back into the \( J \) commutator, we have

\begin{dmath}\label{eqn:translation:120}
\antisymmetric{y}{J(\Bl)}
=
\sum_{k = 1} \inv{k!} \lr{\frac{-i}{\Hbar}} 
k i \Hbar l_y \lr{ \Bp \cdot \Bl }^{k-1}
=
l_y \sum_{k = 1} \inv{(k-1)!} \lr{\frac{-i}{\Hbar}} 
\lr{ \Bp \cdot \Bl }^{k-1}
=
l_y J(\Bl).
\end{dmath}

The same pattern clearly applies with the other \( x_i \) values, providing the desired relation.

\begin{equation}\label{eqn:translation:140}
\antisymmetric{\Bx}{J(\Bl)} = \sum_{m = 1}^3 \Be_m l_m J(\Bl) = \Bl J(\Bl).
\end{equation}

\makeSubAnswer{}{problem:translation:1.30:b}

Suppose that the translated state is defined as \( \ket{\alpha_{\Bl}} = J(\Bl) \ket{\alpha} \).  The expectation value with respect to this state is

\begin{dmath}\label{eqn:translation:160}
\expectation{\Bx'} 
= 
\bra{\alpha_{\Bl}} \Bx \ket{\alpha_{\Bl}}
= 
\bra{\alpha} J^\dagger(\Bl) \Bx J(\Bl) \ket{\alpha}
= 
\bra{\alpha} J^\dagger(\Bl) \lr{ \Bx J(\Bl) } \ket{\alpha}
= 
\bra{\alpha} J^\dagger(\Bl) \lr{ J(\Bl) \Bx + \Bl J(\Bl) } \ket{\alpha}
= 
\bra{\alpha} J^\dagger J \Bx + \Bl J^\dagger J \ket{\alpha}
= 
\bra{\alpha} \Bx \ket{\alpha} + \Bl \braket{\alpha}{\alpha}
= 
\expectation{\Bx} + \Bl.
\end{dmath}

} % answer

\EndArticle
