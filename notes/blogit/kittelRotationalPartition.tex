%
% Copyright � 2013 Peeter Joot.  All Rights Reserved.
% Licenced as described in the file LICENSE under the root directory of this GIT repository.
%
% pick one:
%\newcommand{\authorname}{Peeter Joot}
\newcommand{\email}{peeter.joot@utoronto.ca}
\newcommand{\studentnumber}{920798560}
\newcommand{\basename}{FIXMEbasenameUndefined}
\newcommand{\dirname}{notes/FIXMEdirnameUndefined/}

\newcommand{\authorname}{Peeter Joot}
\newcommand{\email}{peeterjoot@protonmail.com}
\newcommand{\basename}{FIXMEbasenameUndefined}
\newcommand{\dirname}{notes/FIXMEdirnameUndefined/}

\renewcommand{\basename}{kittelRotationalPartition}
\renewcommand{\dirname}{notes/phy452/}
%\newcommand{\dateintitle}{}
%\newcommand{\keywords}{}

\newcommand{\authorname}{Peeter Joot}
\newcommand{\onlineurl}{http://sites.google.com/site/peeterjoot2/math2013/\basename.pdf}
\newcommand{\sourcepath}{\dirname\basename.tex}
\newcommand{\generatetitle}[1]{\chapter{#1}}

\newcommand{\vcsinfo}{%
\section*{}
\noindent{\color{DarkOliveGreen}{\rule{\linewidth}{0.1mm}}}
\paragraph{Document version}
%\paragraph{\color{Maroon}{Document version}}
{
\small
\begin{itemize}
\item Available online at:\\ 
\href{\onlineurl}{\onlineurl}
\item Git Repository: \input{./.revinfo/gitRepo.tex}
\item Source: \sourcepath
\item last commit: \input{./.revinfo/gitCommitString.tex}
\item commit date: \input{./.revinfo/gitCommitDate.tex}
\end{itemize}
}
}

%\PassOptionsToPackage{dvipsnames,svgnames}{xcolor}
\PassOptionsToPackage{square,numbers}{natbib}
\documentclass{scrreprt}

\usepackage[left=2cm,right=2cm]{geometry}
\usepackage[svgnames]{xcolor}
\usepackage{peeters_layout}

\usepackage{natbib}

\usepackage[
colorlinks=true,
bookmarks=false,
pdfauthor={\authorname, \email},
backref 
]{hyperref}

% http://tex.stackexchange.com/questions/75773/how-to-reference-problems-by-the-text-label-in-an-exercise-envioronment
\usepackage[english]{cleveref}
\crefname{Exercise}{exercise}{exercises}
\Crefname{Exercise}{Exercise}{Exercises}

\RequirePackage{titlesec}
\RequirePackage{ifthen}

% http://stackoverflow.com/questions/4932910/date-in-the-tabular-environment
\makeatletter
\let\insertdate\@date
\makeatother

\titleformat{\chapter}[display]
{\bfseries\Large}
{\color{DarkSlateGrey}\filleft \authorname
\ifthenelse{\isundefined{\studentnumber}}{}{\\ \studentnumber}
\ifthenelse{\isundefined{\email}}{}{\\ \email}
\ifthenelse{\isundefined{\dateintitle}}{}{\\ \insertdate}
%\ifthenelse{\isundefined{\coursename}}{}{\\ \coursename} % put in title instead.
}
{4ex}
{\color{DarkOliveGreen}{\titlerule}\color{Maroon}
\vspace{2ex}%
\filright}
[\vspace{2ex}%
\color{DarkOliveGreen}\titlerule
]

\newcommand{\beginArtWithToc}[0]{\begin{document}\tableofcontents}
\newcommand{\beginArtNoToc}[0]{\begin{document}}
\newcommand{\EndNoBibArticle}[0]{\end{document}}
\newcommand{\EndArticle}[0]{\bibliography{Bibliography}\bibliographystyle{plainnat}\end{document}}

% 
%\newcommand{\citep}[1]{\cite{#1}}

\colorSectionsForArticle



\beginArtNoToc

\generatetitle{Rotation of diatomic molecules}
%\chapter{Rotation of diatomic molecules}
%\label{chap:kittelRotationalPartition}

\makeoproblem{Rotation of diatomic molecules}{pr:kittelRotationalPartition:3:6}{\citep{kittel1980thermal} problem 3.6}{
In our first look at the ideal gas we considered only the translational energy of the particles.  But modecules can rotate, with kinetic energy.  The rotation motion is quantized; and the energy levels of a diatomic molecule are of the form

\begin{equation}\label{eqn:kittelRotationalPartition:10}
\epsilon(j) = j(j + 1) \epsilon_0
\end{equation}

where $j$ is any positive integer including zero: $j = 0, 1, 2, \cdots$.  The multiplicity of each rotation level is $g(j) = 2 j + 1$.

\makesubproblem{}{pr:kittelRotationalPartition:3:6:a}

Find the partition function $Z_R(\tau)$ for the rotational states of one molecule.  Remember that $Z$ is a sum over all states, not over all levels -- this makes a difference.
\makesubproblem{}{pr:kittelRotationalPartition:3:6:b}
Evaluate $Z_R(\tau)$ approximately for $\tau \gg \epsilon_0$, by converting the sum to an integral.
\makesubproblem{}{pr:kittelRotationalPartition:3:6:c}

Do the same for $\tau \ll \epsilon_0$, by truncating the sum after the second term.
\makesubproblem{}{pr:kittelRotationalPartition:3:6:d}

Give expressions for the energy $U$ and the heat capacity $C$, as functions of $\tau$, in both limits.  Observe that the rotational contribution to the heat capacity of a diatomic molecule approaches 1 (or, in conventional units, $k_{\mathrm{B}}$) when $\tau \gg \epsilon_0$.
\makesubproblem{}{pr:kittelRotationalPartition:3:6:e}

Sketch the behaviour of $U(\tau)$ and $C(\tau)$, showing the limiting behaviors for $\tau \rightarrow \infty$ and $\tau \rightarrow 0$.
} % makeproblem

\makeanswer{pr:kittelRotationalPartition:3:6}{ 
\makeSubAnswer{Partition function $Z_R(\tau)$}{pr:kittelRotationalPartition:3:6:a}

To understand the reference to multiplicity recall (\S 4.13 \citep{desai2009quantum}) that the rotational Hamiltonian was of the form

\begin{equation}\label{eqn:kittelRotationalPartition:30}
H = \frac{\BL^2}{2 M r^2},
\end{equation}

where the $\BL^2$ eigenvectors satisfied

\begin{subequations}
\begin{equation}\label{eqn:kittelRotationalPartition:50}
\BL^2 \ket{l m} = l (l + 1) \hbar^2 \ket{l m}
\end{equation}
\begin{equation}\label{eqn:kittelRotationalPartition:70}
L_z \ket{l m} = m \hbar \ket{l m}
\end{equation}
\end{subequations}

and $-l \le m \le l$, where $l \ge 0$ is a positive integer.  We see that $\epsilon_0$ is of the form

\begin{equation}\label{eqn:kittelRotationalPartition:90}
\epsilon_0 = \frac{\hbar^2}{2 M R_l(r)},
\end{equation}

and our partition function is

\begin{equation}\label{eqn:kittelRotationalPartition:110}
Z_R(\tau) 
= \sum_{l = 0}^\infty \sum_{m = -l}^l e^{-l (l + 1)\epsilon_0/\tau}
= \sum_{l = 0}^\infty (2 l + 1) e^{-l (l + 1)\epsilon_0/\tau}.
\end{equation}

We have no dependence on $m$ in the sum, and just have to sum terms like \cref{fig:kittelCh3Pr6:kittelCh3Pr6Fig1}, and are able to sum over $m$ trivially.

\imageFigure{kittelCh3Pr6Fig1}{Summation over $m$}{fig:kittelCh3Pr6:kittelCh3Pr6Fig1}{0.3}

That's where the multiplicity comes from.

\makeSubAnswer{Evaluate partition function for large temperatures}{pr:kittelRotationalPartition:3:6:b}
TODO.
\makeSubAnswer{Evaluate partition function for small temperatures}{pr:kittelRotationalPartition:3:6:c}
TODO.
\makeSubAnswer{Energy and heat capacity}{pr:kittelRotationalPartition:3:6:d}
TODO.
\makeSubAnswer{Sketch}{pr:kittelRotationalPartition:3:6:e}
TODO.
} % makeanswer

\EndArticle
