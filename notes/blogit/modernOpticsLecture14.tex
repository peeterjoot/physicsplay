%
% Copyright � 2012 Peeter Joot.  All Rights Reserved.
% Licenced as described in the file LICENSE under the root directory of this GIT repository.
%
\newcommand{\authorname}{Peeter Joot}
\newcommand{\email}{peeterjoot@protonmail.com}
\newcommand{\basename}{FIXMEbasenameUndefined}
\newcommand{\dirname}{notes/FIXMEdirnameUndefined/}

\renewcommand{\basename}{modernOpticsLecture14}
\renewcommand{\dirname}{notes/phy485/}
\newcommand{\keywords}{Optics, PHY485H1F}
\newcommand{\authorname}{Peeter Joot}
\newcommand{\onlineurl}{http://sites.google.com/site/peeterjoot2/math2013/\basename.pdf}
\newcommand{\sourcepath}{\dirname\basename.tex}
\newcommand{\generatetitle}[1]{\chapter{#1}}

\newcommand{\vcsinfo}{%
\section*{}
\noindent{\color{DarkOliveGreen}{\rule{\linewidth}{0.1mm}}}
\paragraph{Document version}
%\paragraph{\color{Maroon}{Document version}}
{
\small
\begin{itemize}
\item Available online at:\\ 
\href{\onlineurl}{\onlineurl}
\item Git Repository: \input{./.revinfo/gitRepo.tex}
\item Source: \sourcepath
\item last commit: \input{./.revinfo/gitCommitString.tex}
\item commit date: \input{./.revinfo/gitCommitDate.tex}
\end{itemize}
}
}

%\PassOptionsToPackage{dvipsnames,svgnames}{xcolor}
\PassOptionsToPackage{square,numbers}{natbib}
\documentclass{scrreprt}

\usepackage[left=2cm,right=2cm]{geometry}
\usepackage[svgnames]{xcolor}
\usepackage{peeters_layout}

\usepackage{natbib}

\usepackage[
colorlinks=true,
bookmarks=false,
pdfauthor={\authorname, \email},
backref 
]{hyperref}

% http://tex.stackexchange.com/questions/75773/how-to-reference-problems-by-the-text-label-in-an-exercise-envioronment
\usepackage[english]{cleveref}
\crefname{Exercise}{exercise}{exercises}
\Crefname{Exercise}{Exercise}{Exercises}

\RequirePackage{titlesec}
\RequirePackage{ifthen}

% http://stackoverflow.com/questions/4932910/date-in-the-tabular-environment
\makeatletter
\let\insertdate\@date
\makeatother

\titleformat{\chapter}[display]
{\bfseries\Large}
{\color{DarkSlateGrey}\filleft \authorname
\ifthenelse{\isundefined{\studentnumber}}{}{\\ \studentnumber}
\ifthenelse{\isundefined{\email}}{}{\\ \email}
\ifthenelse{\isundefined{\dateintitle}}{}{\\ \insertdate}
%\ifthenelse{\isundefined{\coursename}}{}{\\ \coursename} % put in title instead.
}
{4ex}
{\color{DarkOliveGreen}{\titlerule}\color{Maroon}
\vspace{2ex}%
\filright}
[\vspace{2ex}%
\color{DarkOliveGreen}\titlerule
]

\newcommand{\beginArtWithToc}[0]{\begin{document}\tableofcontents}
\newcommand{\beginArtNoToc}[0]{\begin{document}}
\newcommand{\EndNoBibArticle}[0]{\end{document}}
\newcommand{\EndArticle}[0]{\bibliography{Bibliography}\bibliographystyle{plainnat}\end{document}}

% 
%\newcommand{\citep}[1]{\cite{#1}}

\colorSectionsForArticle


\beginArtNoToc
\generatetitle{PHY485H1F Modern Optics.  Lecture 14: XXX.  Taught by Prof.\ Joseph Thywissen}
%\chapter{XXX}
\label{chap:modernOpticsLecture14}

\section{XXX}

We've been discussing a Fabry-Perot Etalon as in

F2

with input that is of a discrete frequency (a spectral line).  We'll get something like

F1

something that is an idealization of

F3

since we have widening due to smaller than ideal reflectivity.  It's not clear to me what the measurement mechanism here is.  We are plotting something against frequency, but only sending in dicrete frequencies.

We found two frequencies $\overbar{\omega} \pm \Delta \omega/2$ are resolved when

\begin{dmath}\label{eqn:modernOpticsLecture14:n}
\frac{\Delta \omega}{\overbar{\omega}} = \inv{ m \FF }
\end{dmath}

Here $m = \text{order of interference}$ so 

\begin{dmath}\label{eqn:modernOpticsLecture14:n}
\Delta = m 2 \pi
\end{dmath}

\begin{dmath}\label{eqn:modernOpticsLecture14:n}
\FF = \pi \frac{\sqrt{R}}{1 - R} = 
\text{Finess of Etalon}
\end{dmath}

F4

Example numbers

\begin{subequations}
\begin{dmath}\label{eqn:modernOpticsLecture14:n}
L = 10 \text{cm} \rightarrow \text{RST} = 10^{10} s^-1 \text{or} 1.5 \text{GHz}
\end{dmath}
\begin{dmath}\label{eqn:modernOpticsLecture14:n}
R = 97 \% \rightarrow F = 100
\end{dmath}
\begin{dmath}\label{eqn:modernOpticsLecture14:n}
\overbar{\lambda} \sim 0.6 \mu m \rightarrow \text{visible light}
\end{dmath}
\begin{dmath}\label{eqn:modernOpticsLecture14:n}
\overbar{\omega} = 3 \times 10^{15} s^{-1}
\end{dmath}
\begin{dmath}\label{eqn:modernOpticsLecture14:n}
\overbar{\nu} = 500 \text{THz}
\end{dmath}
\begin{dmath}\label{eqn:modernOpticsLecture14:n}
\frac{\Delta \omega}{\overbar{\omega} = 3 \times 10^{-8}
\end{dmath}
\begin{dmath}\label{eqn:modernOpticsLecture14:n}
m = \frac{\overbar{\omega}}{\text{FSR}} 
= \frac{3 \times 10^{15} s^{-1}}{ 10^10 s^{-1}} = 3 \times 10^{5}
\end{dmath}
\end{subequations}

$\Delta \omega$ is the smallest separation of two frequencies that we can measure.

\section{Cavity (or Etalon) (Fabry-Perot) as an oscillator}

Why are we talking so much about a specific interferometer, when this is a class on Advanced Classical Optics.  It turns out that the interaction with light in a cavity, as in a large setup 

F5

is basically the same idea as in an implementation of a laser

F6

If we are saying that something is an oscillator, then we can ask a couple questions:

\begin{itemize}
\item 
What is the resonant frequency?
\item 
What is the alignment?
\end{itemize}

The resonant frequency occurs every time that we can get an integer number of half wavelengths in the cavity.

We could actually ask what are the resonant frequencies, since we could have a ``comb'' of resonances

F7

(transmission of the Etalon: see slides)

Answering our question of what are the resonant frequencies, our answer is

\begin{dmath}\label{eqn:modernOpticsLecture14:n}
\omega_m = (\text{offset}) + \text{FSR} m
\end{dmath}

where $m$ is an integer.  For the question of line width, consider a \underline{Lorenztian}

\begin{dmath}\label{eqn:modernOpticsLecture14:n}
\Gamma = \frac{\text{FSR}}{2 \FF}
\end{dmath}

as in

F8

We've got

\begin{dmath}\label{eqn:modernOpticsLecture14:n}
\frac{I}{I_0} = \inv{
1 + \frac{4 \FF^2}{\pi^2} \sin^2 \left( \Delta/2 \right)
}
\end{dmath}

Consider the plot of $\sin^2(\Delta/2)$ as in \cref{fig:modernOpticsLecture14:modernOpticsLecture14Fig9}.

\imageFigure{modernOpticsLecture14Fig9}{Squared sine plot}{fig:modernOpticsLecture14:modernOpticsLecture14Fig9}{0.3}

Our phase offset from the resonance is

\begin{dmath}\label{eqn:modernOpticsLecture14:n}
\Delta = 2 \pi m + \eta
\end{dmath}

\begin{dmath}\label{eqn:modernOpticsLecture14:n}
\eta \ll 1
\end{dmath}

In terms of $\delta = \omega - \omega_m$

\begin{dmath}\label{eqn:modernOpticsLecture14:n}
\eta = \frac{2 L}{c} \delta
\end{dmath}

(because $\Delta = \frac{2 L}{c} \omega + \text{offset}$), we are left with

\begin{dmath}\label{eqn:modernOpticsLecture14:n}
\frac{I}{I_0} 
= \inv{
1 + \frac{4 \FF^2}{\pi^2} \left( \frac{\eta}{2} \right)^2
}
= \inv{
1 + \left( \frac{ 2 L \FF \delta}{\pi c} \right)^2
}
\end{dmath}

or
\begin{dmath}\label{eqn:modernOpticsLecture14:n}
\boxed{
\frac{I}{I_0} 
= \inv{ 1 = \frac{\delta^2}{\Gamma^2}},
}
\end{dmath}

if 

\begin{dmath}\label{eqn:modernOpticsLecture14:n}
\Gamma = \frac{ \pi c }{2 L \FF} = \frac{\text{FSR}}{ 2 \FF }
\end{dmath}

What's the meaning of all of this?  It means that the Fabry-Perot oscillator is a device that traps light, and the resonance looks like a Lorentzian.  

We need a very high Finess (high reflectivity) to get a good Lorentzian.

Recall that the Lorentzian is a Fourier transform of a damped exponential time domain signal

F10

Light is trapped in the cavity for a time $\tau = 1/\Gamma$.

%\EndArticle
\EndNoBibArticle
