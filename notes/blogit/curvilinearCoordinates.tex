%
% Copyright � 2013 Peeter Joot.  All Rights Reserved.
% Licenced as described in the file LICENSE under the root directory of this GIT repository.
%
\newcommand{\authorname}{Peeter Joot}
\newcommand{\email}{peeterjoot@protonmail.com}
\newcommand{\basename}{FIXMEbasenameUndefined}
\newcommand{\dirname}{notes/FIXMEdirnameUndefined/}

\renewcommand{\basename}{curvilinearCoordinates}
\renewcommand{\dirname}{notes/gabook/}
%\newcommand{\dateintitle}{}
\newcommand{\keywords}{curvilinear coordinates, geometric algebra, dual basis, reciprocal coordinates}

\newcommand{\authorname}{Peeter Joot}
\newcommand{\onlineurl}{http://sites.google.com/site/peeterjoot2/math2013/\basename.pdf}
\newcommand{\sourcepath}{\dirname\basename.tex}
\newcommand{\generatetitle}[1]{\chapter{#1}}

\newcommand{\vcsinfo}{%
\section*{}
\noindent{\color{DarkOliveGreen}{\rule{\linewidth}{0.1mm}}}
\paragraph{Document version}
%\paragraph{\color{Maroon}{Document version}}
{
\small
\begin{itemize}
\item Available online at:\\ 
\href{\onlineurl}{\onlineurl}
\item Git Repository: \input{./.revinfo/gitRepo.tex}
\item Source: \sourcepath
\item last commit: \input{./.revinfo/gitCommitString.tex}
\item commit date: \input{./.revinfo/gitCommitDate.tex}
\end{itemize}
}
}

%\PassOptionsToPackage{dvipsnames,svgnames}{xcolor}
\PassOptionsToPackage{square,numbers}{natbib}
\documentclass{scrreprt}

\usepackage[left=2cm,right=2cm]{geometry}
\usepackage[svgnames]{xcolor}
\usepackage{peeters_layout}

\usepackage{natbib}

\usepackage[
colorlinks=true,
bookmarks=false,
pdfauthor={\authorname, \email},
backref 
]{hyperref}

% http://tex.stackexchange.com/questions/75773/how-to-reference-problems-by-the-text-label-in-an-exercise-envioronment
\usepackage[english]{cleveref}
\crefname{Exercise}{exercise}{exercises}
\Crefname{Exercise}{Exercise}{Exercises}

\RequirePackage{titlesec}
\RequirePackage{ifthen}

% http://stackoverflow.com/questions/4932910/date-in-the-tabular-environment
\makeatletter
\let\insertdate\@date
\makeatother

\titleformat{\chapter}[display]
{\bfseries\Large}
{\color{DarkSlateGrey}\filleft \authorname
\ifthenelse{\isundefined{\studentnumber}}{}{\\ \studentnumber}
\ifthenelse{\isundefined{\email}}{}{\\ \email}
\ifthenelse{\isundefined{\dateintitle}}{}{\\ \insertdate}
%\ifthenelse{\isundefined{\coursename}}{}{\\ \coursename} % put in title instead.
}
{4ex}
{\color{DarkOliveGreen}{\titlerule}\color{Maroon}
\vspace{2ex}%
\filright}
[\vspace{2ex}%
\color{DarkOliveGreen}\titlerule
]

\newcommand{\beginArtWithToc}[0]{\begin{document}\tableofcontents}
\newcommand{\beginArtNoToc}[0]{\begin{document}}
\newcommand{\EndNoBibArticle}[0]{\end{document}}
\newcommand{\EndArticle}[0]{\bibliography{Bibliography}\bibliographystyle{plainnat}\end{document}}

% 
%\newcommand{\citep}[1]{\cite{#1}}

\colorSectionsForArticle


\usepackage{peeters_figures}
\usepackage{peeters_layout_exercise}

\beginArtNoToc

\generatetitle{Curvilinear coordinates and reciprocal basis}
%\chapter{Curvilinear coordinates and reciprocal basis}
%\label{chap:curvilinearCoordinates}
%\section{Guts}

\section{Motivation}

Here I'd like to explore some ideas from \citep{aMacdonaldVAGC} where curvilinear coordinates, manifolds, and the vector derivative are introduced.  %The vector derivative is the projection of the gradient on the manifold's tangent space.

\section{Notation}

For simplicity, let's consider the concrete example of a 2D manifold, a surface in an \(n\) dimensional vector space, parameterized by two variables

\begin{equation}\label{eqn:curvilinearCoordinates:20}
\Bx = \Bx(a,b) = \Bx(u^1, u^2).
\end{equation}

Note that the indices here do not represent exponentiation.  We can construct a basis for the manifold as

\begin{equation}\label{eqn:curvilinearCoordinates:40}
\Bx_i = \PD{u^i}{\Bx}.
\end{equation}

On the manifold we can calculate a reciprocal basis \(\{\Bx^i\}\), defined by requiring, at each point on the surface

\begin{equation}\label{eqn:curvilinearCoordinates:80}
\Bx^i \cdot \Bx_j = {\delta^i}_j.
\end{equation}

Associated implicitly with this basis is a curvilinear coordinate representation defined by the projection operation

\begin{equation}\label{eqn:curvilinearCoordinates:60}
\Bx = x^i \Bx_i,
\end{equation}

(sums over mixed indices are implied).   These coordinates can be calculated by taking dot products with the reciprocal frame vectors

\begin{dmath}\label{eqn:curvilinearCoordinates:100}
\Bx \cdot \Bx^i
= x^j \Bx_j \cdot \Bx^i
= x^j {\delta_j}^i
= x^i.
\end{dmath}

\section{Examples}

Let's pause for a couple examples that have interesting aspects.

\makeexample{Circular coordinates on a disk}{example:curvilinearCoordinates:1}{
Consider an infinite disk at height \(z_0\), with the origin omitted, parameterized by circular coordinates as in \cref{fig:diskManifoldSurface:diskManifoldSurfaceFig1}.

\imageFigure{../../figures/gabook/diskManifoldSurfaceFig1}{Plane with circular coordinates}{fig:diskManifoldSurface:diskManifoldSurfaceFig1}{0.3}

Points on this surface are

\begin{equation}\label{eqn:curvilinearCoordinates:120}
\Bx(r, \theta) = (r \cos\theta, r \sin\theta, z_0).
\end{equation}

The manifold basis vectors, defined by \eqnref{eqn:curvilinearCoordinates:40} are

\begin{equation}\label{eqn:curvilinearCoordinates:140}
\begin{aligned}
\Bx_r &= (\cos\theta, \sin\theta, 0) \\
\Bx_\theta &= r (-\sin\theta, \cos\theta, 0).
\end{aligned}
\end{equation}

By inspection, the reciprocal basis is

\begin{equation}\label{eqn:curvilinearCoordinates:200}
\begin{aligned}
\Bx^r &= (\cos\theta, \sin\theta, 0) \\
\Bx^\theta &= \inv{r} (-\sin\theta, \cos\theta, 0).
\end{aligned}
\end{equation}

The first thing to note here is that we cannot reach the points \(\Bx\) of \eqnref{eqn:curvilinearCoordinates:120} by linear combination of these basis vectors.  Instead these basis vectors only allow us to reach other points on the surface, when already there.  For example we cannot actually write

\begin{equation}\label{eqn:curvilinearCoordinates:160}
\Bx = x^r \Bx_r + x^\theta \Bx_\theta,
\end{equation}

unless \(z_0 = 0\).  This is why \eqnref{eqn:curvilinearCoordinates:60} was described as a projective operation (and probably deserves an alternate notation).  To recover the original parameterized form of the position vector on the surface, we require

\begin{equation}\label{eqn:curvilinearCoordinates:180}
\Bx = x^r \Bx_r + x^\theta \Bx_\theta + z_0 \zcap.
\end{equation}

The coordinates \(x^r, x^\theta\) follow by taking dot products

\begin{subequations}
\begin{dmath}\label{eqn:curvilinearCoordinates:220}
x^r
= \Bx \cdot \Bx^r
=
(r \cos\theta, r \sin\theta, z_0) \cdot
(\cos\theta, \sin\theta, 0)
=
r \lr{ \cos^2 \theta + \sin^2 \theta}
= r
\end{dmath}
\begin{dmath}\label{eqn:curvilinearCoordinates:240}
x^\theta
= \Bx \cdot \Bx^\theta
=
(r \cos\theta, r \sin\theta, z_0) \cdot
\inv{r} (-\sin\theta, \cos\theta, 0)
=
0.
\end{dmath}
\end{subequations}

Therefore, a point on the plane, relative to the origin of the plane, in this case, requires just one of the tangent plane basis vectors

\begin{dmath}\label{eqn:curvilinearCoordinates:260}
\Bx = r \Bx_r.
\end{dmath}
}

\makeexample{Circumference of a circle}{example:curvilinearCoordinates:2}{

Now consider a circular perimeter, as illustrated in \cref{fig:circularPerimeterManifoldSurface:circularPerimeterManifoldSurfaceFig2}, with the single variable parameterization

\imageFigure{../../figures/gabook/circularPerimeterManifoldSurfaceFig2}{Circular perimeter}{fig:circularPerimeterManifoldSurface:circularPerimeterManifoldSurfaceFig2}{0.2}

\begin{dmath}\label{eqn:curvilinearCoordinates:280}
\Bx = r_0 \lr{\cos\theta, \sin\theta}.
\end{dmath}

Our tangent space basis is
\begin{dmath}\label{eqn:curvilinearCoordinates:300}
\Bx_\theta = r_0 \lr{-\sin\theta, \cos\theta},
\end{dmath}

with, by inspection, a reciprocal basis
\begin{dmath}\label{eqn:curvilinearCoordinates:320}
\Bx^\theta = \inv{r_0} \lr{-\sin\theta, \cos\theta}.
\end{dmath}

Here we have a curious condition, since the tangent space basis vector is perpendicular to the position vector for the points on the circular surface.  So, should we attempt to calculate coordinates using \eqnref{eqn:curvilinearCoordinates:60}, we just get zero

\begin{dmath}\label{eqn:curvilinearCoordinates:340}
x^\theta
=
\Bx \cdot \Bx^\theta
=
r_0 \lr{\cos\theta, \sin\theta} \cdot
\inv{r_0} \lr{-\sin\theta, \cos\theta}
= 0.
\end{dmath}

It's perhaps notable that a coordinate representation using the tangent space basis is possible, but we need to utilize a complex geometry.  Assuming

\begin{dmath}\label{eqn:curvilinearCoordinates:360}
\Bx = x^\theta \Bx_\theta,
\end{dmath}

and writing \(i = \Be_1 \Be_2\) for the pseudoscalar, we can write

\begin{dmath}\label{eqn:curvilinearCoordinates:380}
\begin{aligned}
\Bx &= r_0 \Be_1 e^{i\theta} \\
\Bx_\theta &= r_0 \Be_2 e^{i\theta},
\end{aligned}
\end{dmath}

so that, by inversion, the \(\theta\) coordinate is

\begin{dmath}\label{eqn:curvilinearCoordinates:400}
x^\theta
=
\Bx
\inv{\Bx_\theta}
=
\lr{ r_0 \Be_1 e^{i\theta} }
\lr{ \frac{ e^{-i\theta } \Be_2 }{r_0} }
= i,
\end{dmath}

or

\begin{equation}\label{eqn:curvilinearCoordinates:440}
\Bx = i \Bx_\theta.
\end{equation}
}

\makeexample{Surface of a sphere}{example:curvilinearCoordinates:3}{
It is also clear that any parameterization that has radial symmetry will suffer the same issue.  For example, for a radial surface in 3D with radius \(r_0\) we have

\begin{equation}\label{eqn:curvilinearCoordinates:420}
\begin{aligned}
\Bx &= r_0 \lr{ \sin\theta \cos\phi, \sin\theta \sin\phi, \cos\theta } \\
\Bx_\theta &= r_0 \lr{ \cos\theta \cos\phi, \cos\theta \sin\phi, -\sin\theta } \\
\Bx_\phi &= r_0 \lr{ -\sin\theta \sin\phi, \sin\theta \cos\phi, 0 } \\
\Bx^\theta &= \inv{r_0} \lr{ \cos\theta \cos\phi, \cos\theta \sin\phi, -\sin\theta } \\
\Bx^\phi &= \inv{r_0 \sin\theta} \lr{ -\sin\phi, \cos\phi, 0 }.
\end{aligned}
\end{equation}

The reciprocals here were computed using the mathematica reciprocalFrameSphericalSurface.nb notebook.

Do we have a bivector parameterization of the surface using the tangent space basis?  Let's try

\begin{equation}\label{eqn:curvilinearCoordinates:460}
\Bx = x^\theta \Bx_\theta + x^\phi \Bx_\phi.
\end{equation}

Wedging with \(\Bx_\theta\) and \(\Bx_\phi\), and writing \(i = \Be_1 \Be_2\), respectively yields

\begin{dmath}\label{eqn:curvilinearCoordinates:480}
x^\theta = \Bx \wedge \Bx_\phi \inv{\Bx_\theta \wedge \Bx_\phi}
=
-\Be_1 \Be_3 \cos \phi - \Be_2 \Be_3 \sin \phi
=
\Be_{31} e^{ i \phi}.
\end{dmath}

\begin{dmath}\label{eqn:curvilinearCoordinates:500}
x^\phi
= -\Bx \wedge \Bx_\theta \inv{\Bx_\theta \wedge \Bx_\phi}
=
\Be_1 \Be_3 \cot \theta \sin \phi + i -\Be_2 \Be_3 \cot \theta \cos \phi
=
\Be_2 \Be_3 \cot \theta e^{i \phi} + i.
\end{dmath}

However, substitution back into \eqnref{eqn:curvilinearCoordinates:460} shows either pair parameterizes the radial position vector

\begin{equation}\label{eqn:curvilinearCoordinates:520}
\Bx = x^\theta \Bx_\theta = x^\phi \Bx_\phi.
\end{equation}

It is interesting that duality relationships seem to naturally arise attempting to describe points on a surface using the tangent space basis for that surface.
}

\EndArticle
