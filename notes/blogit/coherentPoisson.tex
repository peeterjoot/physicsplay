%
% Copyright � 2015 Peeter Joot.  All Rights Reserved.
% Licenced as described in the file LICENSE under the root directory of this GIT repository.
%
\newcommand{\authorname}{Peeter Joot}
\newcommand{\email}{peeterjoot@protonmail.com}
\newcommand{\basename}{FIXMEbasenameUndefined}
\newcommand{\dirname}{notes/FIXMEdirnameUndefined/}

\renewcommand{\basename}{coherentPoisson}
\renewcommand{\dirname}{notes/phy1520/}
%\newcommand{\dateintitle}{}
%\newcommand{\keywords}{}

\newcommand{\authorname}{Peeter Joot}
\newcommand{\onlineurl}{http://sites.google.com/site/peeterjoot2/math2013/\basename.pdf}
\newcommand{\sourcepath}{\dirname\basename.tex}
\newcommand{\generatetitle}[1]{\chapter{#1}}

\newcommand{\vcsinfo}{%
\section*{}
\noindent{\color{DarkOliveGreen}{\rule{\linewidth}{0.1mm}}}
\paragraph{Document version}
%\paragraph{\color{Maroon}{Document version}}
{
\small
\begin{itemize}
\item Available online at:\\ 
\href{\onlineurl}{\onlineurl}
\item Git Repository: \input{./.revinfo/gitRepo.tex}
\item Source: \sourcepath
\item last commit: \input{./.revinfo/gitCommitString.tex}
\item commit date: \input{./.revinfo/gitCommitDate.tex}
\end{itemize}
}
}

%\PassOptionsToPackage{dvipsnames,svgnames}{xcolor}
\PassOptionsToPackage{square,numbers}{natbib}
\documentclass{scrreprt}

\usepackage[left=2cm,right=2cm]{geometry}
\usepackage[svgnames]{xcolor}
\usepackage{peeters_layout}

\usepackage{natbib}

\usepackage[
colorlinks=true,
bookmarks=false,
pdfauthor={\authorname, \email},
backref 
]{hyperref}

% http://tex.stackexchange.com/questions/75773/how-to-reference-problems-by-the-text-label-in-an-exercise-envioronment
\usepackage[english]{cleveref}
\crefname{Exercise}{exercise}{exercises}
\Crefname{Exercise}{Exercise}{Exercises}

\RequirePackage{titlesec}
\RequirePackage{ifthen}

% http://stackoverflow.com/questions/4932910/date-in-the-tabular-environment
\makeatletter
\let\insertdate\@date
\makeatother

\titleformat{\chapter}[display]
{\bfseries\Large}
{\color{DarkSlateGrey}\filleft \authorname
\ifthenelse{\isundefined{\studentnumber}}{}{\\ \studentnumber}
\ifthenelse{\isundefined{\email}}{}{\\ \email}
\ifthenelse{\isundefined{\dateintitle}}{}{\\ \insertdate}
%\ifthenelse{\isundefined{\coursename}}{}{\\ \coursename} % put in title instead.
}
{4ex}
{\color{DarkOliveGreen}{\titlerule}\color{Maroon}
\vspace{2ex}%
\filright}
[\vspace{2ex}%
\color{DarkOliveGreen}\titlerule
]

\newcommand{\beginArtWithToc}[0]{\begin{document}\tableofcontents}
\newcommand{\beginArtNoToc}[0]{\begin{document}}
\newcommand{\EndNoBibArticle}[0]{\end{document}}
\newcommand{\EndArticle}[0]{\bibliography{Bibliography}\bibliographystyle{plainnat}\end{document}}

% 
%\newcommand{\citep}[1]{\cite{#1}}

\colorSectionsForArticle



\usepackage{peeters_layout_exercise}
\usepackage{peeters_braket}
\usepackage{peeters_figures}

\beginArtNoToc

\generatetitle{More on (SHO) coherent states}
%\chapter{more on coherent states}
%\label{chap:coherentPoisson}

\paragraph{\citep{sakurai2014modern} pr. 2.19(c)}

Show that \( \Abs{f(n)}^2 \) for a coherent state written as

\begin{dmath}\label{eqn:gradQuantumProblemSet2Problem1:561}
\ket{z} = \sum_{n=0}^\infty f(n) \ket{n}
\end{dmath}

has the form of a Poisson distribution, and find the most probable value of \( n\), and thus the most probable energy.

\paragraph{A:}

The Poisson distribution has the form

\begin{dmath}\label{eqn:gradQuantumProblemSet2Problem1:581}
P(n) = \frac{\mu^{n} e^{-\mu}}{n!}.
\end{dmath}

Here \( \mu \) is the mean of the distribution

\begin{dmath}\label{eqn:gradQuantumProblemSet2Problem1:601}
\expectation{n} 
= \sum_{n=0}^\infty n P(n) 
= \sum_{n=1}^\infty n \frac{\mu^{n} e^{-\mu}}{n!}
= \mu e^{-\mu} \sum_{n=1}^\infty \frac{\mu^{n-1}}{(n-1)!}
= \mu e^{-\mu} e^{\mu}
= \mu.
\end{dmath}

We found that the coherent state had the form

\begin{dmath}\label{eqn:gradQuantumProblemSet2Problem1:621}
\ket{z} = c_0 \sum_{n=0} \frac{z^n}{\sqrt{n!}} \ket{n},
\end{dmath}

so the probability coefficients for \( \ket{n} \) are

\begin{dmath}\label{eqn:gradQuantumProblemSet2Problem1:641}
P(n) 
= c_0^2 \frac{\Abs{z^n}^2}{n!}
= e^{-\Abs{z}^2} \frac{\Abs{z^n}^2}{n!}.
\end{dmath}

This has the structure of the Poisson distribution with mean \( \mu = \Abs{z}^2 \).  The most probable value of \( n \) is that for which \( \Abs{f(n)}^2 \) is the largest.  This is, in general, hard to compute, since we have a maximization problem in the integer domain that falls outside the normal toolbox.  If we assume that \( n \) is large, so that Stirling's approximation can be used to approximate the factorial, and also seek a non-integer value that maximizes the distribution, the most probable value will be the closest integer to that, and this can be computed.  Let 

\begin{dmath}\label{eqn:gradQuantumProblemSet2Problem1:661}
g(n) 
= \Abs{f(n)}^2
= \frac{e^{-\mu} \mu^n}{n!}
= \frac{e^{-\mu} \mu^n}{e^{\ln n!}}
\approx e^{-\mu - n \ln n + n } \mu^n.
= e^{-\mu - n \ln n + n + n \ln \mu }
\end{dmath}

This is maximized when

\begin{dmath}\label{eqn:gradQuantumProblemSet2Problem1:681}
0 
= \frac{dg}{dn}
= \lr{ - \ln n - 1 + 1 + \ln \mu } g(n),
\end{dmath}

which is maximized at \( n = \mu \).  One of the integers \( n = \lfloor \mu \rfloor \) or \( n = \lceil \mu \rceil \) that brackets this value \( \mu = \Abs{z}^2 \) is the most probable.  So, if an energy measurement is made of a coherent state \( \ket{z} \), the most probable value will be one of

\begin{dmath}\label{eqn:gradQuantumProblemSet2Problem1:701}
E = \Hbar \lr{ 
\largestIntLessThan{\Abs{z}^2}
 + \inv{2} },
\end{dmath}

or

\begin{dmath}\label{eqn:gradQuantumProblemSet2Problem1:721}
E = \Hbar \lr{ 
\largestIntGreaterThan{\Abs{z}^2}
 + \inv{2} },
\end{dmath}

\EndArticle
%\EndNoBibArticle
