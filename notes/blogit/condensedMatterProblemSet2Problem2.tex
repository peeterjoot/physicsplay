%
% Copyright � 2013 Peeter Joot.  All Rights Reserved.
% Licenced as described in the file LICENSE under the root directory of this GIT repository.
%
\makeproblem{Packing of spheres on lattices}{condensedMatter:problemSet2:2}{ 
(based on question 2.7 of Ibach and Luth) 
Supposing the atoms to be rigid spheres in contact, 
calculate the fraction of space that is filled by atoms in the 
primitive cubic, fcc, hcp and bcc lattices. 

} % makeproblem

\makeanswer{condensedMatter:problemSet2:2}{ 

For fcc we refer to \cref{fig:problemSet2Problem2faceCenteredCubic:problemSet2Problem2faceCenteredCubicFig4} to see that if the cube width is $a$ the radius of any of the spherical pieces is $\sqrt{2} a/4$, and the volume occupied is

\begin{dmath}\label{eqn:condensedMatterProblemSet2Problem2:n}
\frac{4}{3} \pi \lr{ \sqrt{2} \frac{a}{4} }^3
\lr{ 
\inv{2} \times 6 
+ \inv{8} \times 8
}
=
\frac{\cancel{4}}{3} \pi a^3 \frac{2 \sqrt{2}}{ 4 \cancel{4} \cancel{4} } (\cancel{3 + 1})
= \pi a^3 \frac{\sqrt{2}}{6},
\end{dmath}

so that the occupied fraction for fcc is

\begin{dmath}\label{eqn:condensedMatterProblemSet2Problem2:n}
\frac{\sqrt{2} \pi}{6} \approx 0.740.
\end{dmath}

\imageFigure{../phy452/figures/problemSet2Problem2faceCenteredCubicFig4b}{Element of a face centered cubic}{fig:problemSet2Problem2faceCenteredCubic:problemSet2Problem2faceCenteredCubicFig4}{0.2}

For bcc we refer to \cref{fig:bccPacking:bccPackingFig1}.

\imageFigure{../phy487/figures/bccPackingFig1}{Element of a body centered cubic}{fig:bccPacking:bccPackingFig1}{0.2}

For the bcc element twice our diameter is the diagonal cross length of $\sqrt{3 a^2}$, so our filling fraction is

\begin{dmath}\label{eqn:condensedMatterProblemSet2Problem2:n}
\frac{4}{3} \pi \lr{\frac{\sqrt{3}}{4}}^3 \lr{ 1 + \inv{8} \times 8}
=
\frac{\sqrt{3} \pi }{8} 
\approx
0.680.
\end{dmath}

For hcp it's hard to find a good pictoral representation of the unit cell to calculate from.  From \href{http://youtu.be/GAd18wfbXfY}{http://youtu.be/GAd18wfbXfY} I frame grabbed 
\cref{fig:hcpUnitHexCell:hcpUnitHexCellFig1}.

\imageFigure{hcpUnitHexCellFig1}{Hex Close Packed unit cell}{fig:hcpUnitHexCell:hcpUnitHexCellFig1}{0.3}

Some authors appear to call this structure the unit cell for hcp, since it is clearly possible to construct a periodic structure using it.  However, others divide it into three parts and call each of those parts the primtive unit cell for hcp as in \cref{fig:hcpUnitHexCellInThreePieces:hcpUnitHexCellInThreePiecesFig4}.

\imageFigure{hcpUnitHexCellInThreePiecesFig4}{hcp primitive unit cell}{fig:hcpUnitHexCellInThreePieces:hcpUnitHexCellInThreePiecesFig4}{0.3}

It is apparently possible to use repetition of this structure (likely after various transformations) to do the 3D hcp tiling, but that's not obvious when viewing a 2D diagram of such a unit cell.  We can do the packing density calculation using the hexagonal structure instead.

An attempt to actually construct this structure shows that the representation of \cref{fig:hcpUnitHexCell:hcpUnitHexCellFig1} is actually deceptive, since it make it appear that the three interior spheres are tidily constrained by the walls of the hexagonal prism.  We find instead that there is portions of those three interior spheres are sheared off by the unit cell walls, but those sheared portions can be fit back in.  This is illustrated in \cref{fig:hcpSecondLayerFitting:hcpSecondLayerFittingFig2} (note that I glued in the relocated piece slightly incorrectly, and it should be shifted up as illustrated).

\imageFigure{hcpSecondLayerFittingFig2}{Second layer overlap packing in hcp.}{fig:hcpSecondLayerFitting:hcpSecondLayerFittingFig2}{0.3}

Finally, the key to the density calculation is the observation that we have three tetrahodrons formed between the bottom and middle layers of the unit cell.  Two of these three tetrahedrons are marked in \cref{fig:hcpTetrahedralUnit:hcpTetrahedralUnitFig3}.

\imageFigure{hcpTetrahedralUnitFig3}{Tetrahedral packing of two hcp layers}{fig:hcpTetrahedralUnit:hcpTetrahedralUnitFig3}{0.3}

Moving on the calculation, first consider the base of the tetrahedron, which has the form illustrated in \cref{fig:tetrahedralBase:tetrahedralBaseFig5}.

\imageFigure{tetrahedralBaseFig5}{Tetrahedral base}{fig:tetrahedralBase:tetrahedralBaseFig5}{0.3}

... TODO ...

Next we calculate the height of the tetrahedron, referring to \cref{fig:tetrahedralHeight:tetrahedralHeightFig6}.

\imageFigure{tetrahedralHeightFig6}{Tetrahedral height}{fig:tetrahedralHeight:tetrahedralHeightFig6}{0.3}

... TODO.
}
