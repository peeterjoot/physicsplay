%
% Copyright � 2013 Peeter Joot.  All Rights Reserved.
% Licenced as described in the file LICENSE under the root directory of this GIT repository.
%
\newcommand{\authorname}{Peeter Joot}
\newcommand{\email}{peeterjoot@protonmail.com}
\newcommand{\basename}{FIXMEbasenameUndefined}
\newcommand{\dirname}{notes/FIXMEdirnameUndefined/}

\renewcommand{\basename}{condensedMatterLecture12}
\renewcommand{\dirname}{notes/phy487/}
\newcommand{\keywords}{Condensed matter physics, PHY487H1F}
\newcommand{\authorname}{Peeter Joot}
\newcommand{\onlineurl}{http://sites.google.com/site/peeterjoot2/math2013/\basename.pdf}
\newcommand{\sourcepath}{\dirname\basename.tex}
\newcommand{\generatetitle}[1]{\chapter{#1}}

\newcommand{\vcsinfo}{%
\section*{}
\noindent{\color{DarkOliveGreen}{\rule{\linewidth}{0.1mm}}}
\paragraph{Document version}
%\paragraph{\color{Maroon}{Document version}}
{
\small
\begin{itemize}
\item Available online at:\\ 
\href{\onlineurl}{\onlineurl}
\item Git Repository: \input{./.revinfo/gitRepo.tex}
\item Source: \sourcepath
\item last commit: \input{./.revinfo/gitCommitString.tex}
\item commit date: \input{./.revinfo/gitCommitDate.tex}
\end{itemize}
}
}

%\PassOptionsToPackage{dvipsnames,svgnames}{xcolor}
\PassOptionsToPackage{square,numbers}{natbib}
\documentclass{scrreprt}

\usepackage[left=2cm,right=2cm]{geometry}
\usepackage[svgnames]{xcolor}
\usepackage{peeters_layout}

\usepackage{natbib}

\usepackage[
colorlinks=true,
bookmarks=false,
pdfauthor={\authorname, \email},
backref 
]{hyperref}

% http://tex.stackexchange.com/questions/75773/how-to-reference-problems-by-the-text-label-in-an-exercise-envioronment
\usepackage[english]{cleveref}
\crefname{Exercise}{exercise}{exercises}
\Crefname{Exercise}{Exercise}{Exercises}

\RequirePackage{titlesec}
\RequirePackage{ifthen}

% http://stackoverflow.com/questions/4932910/date-in-the-tabular-environment
\makeatletter
\let\insertdate\@date
\makeatother

\titleformat{\chapter}[display]
{\bfseries\Large}
{\color{DarkSlateGrey}\filleft \authorname
\ifthenelse{\isundefined{\studentnumber}}{}{\\ \studentnumber}
\ifthenelse{\isundefined{\email}}{}{\\ \email}
\ifthenelse{\isundefined{\dateintitle}}{}{\\ \insertdate}
%\ifthenelse{\isundefined{\coursename}}{}{\\ \coursename} % put in title instead.
}
{4ex}
{\color{DarkOliveGreen}{\titlerule}\color{Maroon}
\vspace{2ex}%
\filright}
[\vspace{2ex}%
\color{DarkOliveGreen}\titlerule
]

\newcommand{\beginArtWithToc}[0]{\begin{document}\tableofcontents}
\newcommand{\beginArtNoToc}[0]{\begin{document}}
\newcommand{\EndNoBibArticle}[0]{\end{document}}
\newcommand{\EndArticle}[0]{\bibliography{Bibliography}\bibliographystyle{plainnat}\end{document}}

% 
%\newcommand{\citep}[1]{\cite{#1}}

\colorSectionsForArticle



%\citep{harald2003solid} \S x.y
%\citep{ibach2009solid} \S x.y

%\usepackage{mhchem}
\usepackage[version=3]{mhchem}
\newcommand{\nought}[0]{\circ}
\newcommand{\EF}[0]{E_{\mathrm{F}}}
\newcommand{\CV}[0]{C_{\mathrm{V}}}

\beginArtNoToc
\generatetitle{PHY487H1F Condensed Matter Physics.  Lecture 12: Free electron model (cont.).  Taught by Prof.\ Stephen Julian}
%\chapter{Free electron model (cont.)}
\label{chap:condensedMatterLecture12}

%\section{Disclaimer}
%
%Peeter's lecture notes from class.  May not be entirely coherent.

\paragraph{Last time}

We want to calculate the thermal properties of the free electron gas, characterized by energies of the form \cref{fig:qmSolidsL12:qmSolidsL12Fig1}.  At $T = 0$, the free electron gas fills up states up to $\EF$

\imageFigure{qmSolidsL12Fig1}{Free electron energy distribution}{fig:qmSolidsL12:qmSolidsL12Fig1}{0.2}


\section{Fermi dirac distribution for $T > 0$}


Note that we are following Einstein here, not the text.

Consider a closed system and reservoir $r$, as in \cref{fig:qmSolidsL12:qmSolidsL12Fig4}, exchanging energy and particles, with a smaller system that has energy $\epsilon$ and $n$ particles.  The total energy and number of particles are respectively $U$ and $N$.

\imageFigure{qmSolidsL12Fig4}{Two systems in contact}{fig:qmSolidsL12:qmSolidsL12Fig4}{0.2}

%%\cref{fig:qmSolidsL12:qmSolidsL12Fig3}.
%\imageFigure{qmSolidsL12Fig3}{k space region}{fig:qmSolidsL12:qmSolidsL12Fig3}{0.3}
%%\cref{fig:qmSolidsL12:qmSolidsL12Fig2}.
%\imageFigure{qmSolidsL12Fig2}{Fermi energy in the energy distribution range}{fig:qmSolidsL12:qmSolidsL12Fig2}{0.2}

The probability of eigenstate $\epsilon$ with $n$ particles

\begin{dmath}\label{eqn:condensedMatterLecture12:20}
P(\epsilon, n) \propto g_r(U - \epsilon, N - n),
\end{dmath}

where $g_r$ is the number of ``microstates of the reservoir when it has energy $U - \epsilon$, and $N - n$ particles.  We have 

\begin{dmath}\label{eqn:condensedMatterLecture12:40}
g_r(U - \epsilon, N - n)
=
e^{ \ln g_r(U - \epsilon, N - n) }
\approx
\exp\lr{ 
\ln g_r(U, N) 
- \epsilon \PDc{U}{\ln g_r}{N}
- n \PDc{N}{\ln g_r}{U}
}.
\end{dmath}

Introducing the Boltzman entropy 

\begin{dmath}\label{eqn:condensedMatterLecture12:60}
S_r = \kB \ln g_r,
\end{dmath}

we have approximately

\begin{dmath}\label{eqn:condensedMatterLecture12:80}
P(\epsilon, n) \propto 
\exp\lr{ 
\ln g_r(U, N) 
- \frac{\epsilon}{\kB} \PDc{U}{S_r}{N}
- \frac{n}{\kB} \PDc{N}{S_r}{U}
}.
\end{dmath}

Recall the thermodynamic relationship

\begin{dmath}\label{eqn:condensedMatterLecture12:100}
dU_r = T dS_r - \cancel{p dV_r} + \mu dN_r,
\end{dmath}

from which we can define

\begin{subequations}
\begin{dmath}\label{eqn:condensedMatterLecture12:120}
\PDc{U_r}{S_r}{N} = \inv{T}
\end{dmath}
\begin{dmath}\label{eqn:condensedMatterLecture12:140}
\PDc{N}{S_r}{U,V} = -\frac{\mu}{T},
\end{dmath}
\end{subequations}

giving approximately

\begin{dmath}\label{eqn:condensedMatterLecture12:160}
P(\epsilon, n) \propto e^{-(\epsilon - \mu n)/\kB T}.
\end{dmath}

This is the \underlineAndIndex{Boltzman-Gibbs distribution}.  Normalizing we have

\begin{dmath}\label{eqn:condensedMatterLecture12:180}
P(\epsilon, n) = 
\frac{e^{-(\epsilon - \mu n)/\kB T}}{
\sum_n \sum_{\epsilon(n)} e^{-(\epsilon - \mu n)/\kB T}
}.
\end{dmath}

This method is deemed old fashioned because it relies on being able to calculate the eigenstates of the system.  Imagine the impossibility of this for, say, a room full of air.

It seems that we are labelling the system as having an energy eigenstate since we imagine that it can be characterized as having a single energy level.

\paragraph{Appication to free electrons}

Select one energy level $(*)$, \cref{fig:qmSolidsL12:qmSolidsL12Fig5}, as the `system', and treat all other levels as the `reservoir'.

\imageFigure{qmSolidsL12Fig5}{Selected energy level for system}{fig:qmSolidsL12:qmSolidsL12Fig5}{0.15}

Note that $n_i \in \{0, 1\}$ due to the Pauli exclusion principle, and $\epsilon \in \{0, \epsilon_i\}$. 

\begin{subequations}
\begin{dmath}\label{eqn:condensedMatterLecture12:200}
P(\epsilon = 0, n = 0) = \inv{ 1 + e^{-(\epsilon_i - \mu)}}
\end{dmath}
\begin{dmath}\label{eqn:condensedMatterLecture12:220}
P(\epsilon_i, n = 1) = \frac{ e^{-(\epsilon_i - \mu)/\kB T }
}{ 1 + e^{-(\epsilon_i - \mu)} }
\end{dmath}
\end{subequations}

Average occupancy 

\begin{dmath}\label{eqn:condensedMatterLecture12:240}
\expectation{n_i} = \frac{ 0 \times 1 + 1 \times e^{-(\epsilon_i - \mu)/ \kB T } }
{ 1 + e^{-(\epsilon_i - \mu)/\kB T} },
\end{dmath}

or

\begin{dmath}\label{eqn:condensedMatterLecture12:260}
\myBoxed{
\expectation{n_i} = \frac{ e^{-(\epsilon_i - \mu)/ \kB T } }
{ e^{(\epsilon_i - \mu)/\kB T} + 1 }.
}
\end{dmath}

This is the \underlineAndIndex{Fermi-Dirac distribution}, as sketched roughly in
\cref{fig:qmSolidsL12:qmSolidsL12Fig6}.

\imageFigure{qmSolidsL12Fig6}{Fermi dirac distribution}{fig:qmSolidsL12:qmSolidsL12Fig6}{0.2}

%F7
%\cref{fig:qmSolidsL12:qmSolidsL12Fig7}.
%\imageFigure{qmSolidsL12Fig7}{7: CAPTION}{fig:qmSolidsL12:qmSolidsL12Fig7}{0.2}
%
This solved a big mystery, since the equipartition theorem says 

\begin{dmath}\label{eqn:condensedMatterLecture12:280}
U = \frac{3}{2} \kB T \times 
\mathLabelBox
{n}
{electron density},
\end{dmath}

so that 

\begin{dmath}\label{eqn:condensedMatterLecture12:300}
\CV = \PD{T}{U} \sim 10^{28} \frac{\text{electrons}}{m^3} \times \frac{3}{2} \kB,
\end{dmath}

however the value of $\CV$ that was measured is $1/100$ times too small.  Because of the Pauli exclusion prinipcle, most electrons are trapped far below $\EF$, and can't accept thermal energy.

\section{Heat capacity of free electrons}

Estimate

\begin{dmath}\label{eqn:condensedMatterLecture12:320}
U(T) - U(0)
\sim 
\mathLabelBox
{\kB T \times D(\EF)}{from F.D. distribution, number of thermally excited electrons}
\times 
\mathLabelBox
[
   labelstyle={below of=m\themathLableNode, below of=m\themathLableNode}
]
{ \kB T}
{thermal energy per thermally excited electron}
\end{dmath}

F7

\begin{dmath}\label{eqn:condensedMatterLecture12:340}
C(T) 
= \frac{dU}{dT} 
\sim 2 \kB^2 D(\EF) T
\sim \frac{2 \kB^2}{2 \pi^2} \lr{\frac{2 m}{\Hbar}}^{3/2} \sqrt{\EF} T
\sim ...
\sim 3 \kB n 
\mathLabelBox
{\frac{ \kB T}{ \EF }}
{reduction from classical values}
\end{dmath}

%\cref{fig:qmSolidsL12:qmSolidsL12Fig8}.
\imageFigure{qmSolidsL12Fig8}{Specific heat}{fig:qmSolidsL12:qmSolidsL12Fig8}{0.2}

%\EndArticle
\EndNoBibArticle
