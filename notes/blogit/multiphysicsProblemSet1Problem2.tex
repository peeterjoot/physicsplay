%
% Copyright � 2014 Peeter Joot.  All Rights Reserved.
% Licenced as described in the file LICENSE under the root directory of this GIT repository.
%
\makeproblem{Resistor Mesh}{multiphysics:problemSet1:2}{ 
\makesubproblem{
Write a small matlab function that generates a netlist for a network made by:\(\cdots\)
%� a N�N square grid of resistors of value R, where N is the number of
%resistors per edge. The grid nodes are numbered from 1 to (N +1) 2 ;
%� a voltage source V = 1V connected between node 1 and ground;
%� three current sources, each one connected between a randomly-
%selected node of the grid and the reference node. The source current
%flows from the grid node to the reference node. Choose their value
%randomly between 10 mA and 100 mA;
%Generate the modified nodal analysis equations (1) for a grid with N =
%50, R = 0.2� and solve them with your LU routine to find the node
%voltages. Plot the result with the MATLAB command surf.
}{multiphysics:problemSet1:2a}
\makesubproblem{
}{multiphysics:problemSet1:2b}
Plot the CPU time taken by your system solver (LU factorization + forward/backward substitution) as a function of the size \(n\) of the 3 modified nodal analysis matrix \(\BG\). Note: do not exploit the sparsity of the matrix.
\makesubproblem{
Determine experimentally how the CPU time scales for large \(n\).  Fit the CPU times you observe with a power law like \(t_{\mathrm{cpu}} (n) \simeq K n^\alpha \) where \(K\) is a constant in order to determine the exponent \(\alpha\).
}{multiphysics:problemSet1:2c}
\makesubproblem{
Comment on the result. Discuss your findings in light of what we discussed in class
}{multiphysics:problemSet1:2d}
} % makeproblem

\makeanswer{multiphysics:problemSet1:2}{ 
\makeSubAnswer{}{multiphysics:problemSet1:2a}

TODO.
\makeSubAnswer{}{multiphysics:problemSet1:2b}

TODO.
\makeSubAnswer{}{multiphysics:problemSet1:2c}

TODO.
\makeSubAnswer{}{multiphysics:problemSet1:2d}

TODO.
}
