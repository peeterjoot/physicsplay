%
% Copyright � 2012 Peeter Joot.  All Rights Reserved.
% Licenced as described in the file LICENSE under the root directory of this GIT repository.
%
\newcommand{\authorname}{Peeter Joot}
\newcommand{\email}{peeterjoot@protonmail.com}
\newcommand{\basename}{FIXMEbasenameUndefined}
\newcommand{\dirname}{notes/FIXMEdirnameUndefined/}

\renewcommand{\basename}{modernOpticsLecture20}
\renewcommand{\dirname}{notes/phy485/}
\newcommand{\keywords}{Optics, PHY485H1F}
\newcommand{\authorname}{Peeter Joot}
\newcommand{\onlineurl}{http://sites.google.com/site/peeterjoot2/math2013/\basename.pdf}
\newcommand{\sourcepath}{\dirname\basename.tex}
\newcommand{\generatetitle}[1]{\chapter{#1}}

\newcommand{\vcsinfo}{%
\section*{}
\noindent{\color{DarkOliveGreen}{\rule{\linewidth}{0.1mm}}}
\paragraph{Document version}
%\paragraph{\color{Maroon}{Document version}}
{
\small
\begin{itemize}
\item Available online at:\\ 
\href{\onlineurl}{\onlineurl}
\item Git Repository: \input{./.revinfo/gitRepo.tex}
\item Source: \sourcepath
\item last commit: \input{./.revinfo/gitCommitString.tex}
\item commit date: \input{./.revinfo/gitCommitDate.tex}
\end{itemize}
}
}

%\PassOptionsToPackage{dvipsnames,svgnames}{xcolor}
\PassOptionsToPackage{square,numbers}{natbib}
\documentclass{scrreprt}

\usepackage[left=2cm,right=2cm]{geometry}
\usepackage[svgnames]{xcolor}
\usepackage{peeters_layout}

\usepackage{natbib}

\usepackage[
colorlinks=true,
bookmarks=false,
pdfauthor={\authorname, \email},
backref 
]{hyperref}

% http://tex.stackexchange.com/questions/75773/how-to-reference-problems-by-the-text-label-in-an-exercise-envioronment
\usepackage[english]{cleveref}
\crefname{Exercise}{exercise}{exercises}
\Crefname{Exercise}{Exercise}{Exercises}

\RequirePackage{titlesec}
\RequirePackage{ifthen}

% http://stackoverflow.com/questions/4932910/date-in-the-tabular-environment
\makeatletter
\let\insertdate\@date
\makeatother

\titleformat{\chapter}[display]
{\bfseries\Large}
{\color{DarkSlateGrey}\filleft \authorname
\ifthenelse{\isundefined{\studentnumber}}{}{\\ \studentnumber}
\ifthenelse{\isundefined{\email}}{}{\\ \email}
\ifthenelse{\isundefined{\dateintitle}}{}{\\ \insertdate}
%\ifthenelse{\isundefined{\coursename}}{}{\\ \coursename} % put in title instead.
}
{4ex}
{\color{DarkOliveGreen}{\titlerule}\color{Maroon}
\vspace{2ex}%
\filright}
[\vspace{2ex}%
\color{DarkOliveGreen}\titlerule
]

\newcommand{\beginArtWithToc}[0]{\begin{document}\tableofcontents}
\newcommand{\beginArtNoToc}[0]{\begin{document}}
\newcommand{\EndNoBibArticle}[0]{\end{document}}
\newcommand{\EndArticle}[0]{\bibliography{Bibliography}\bibliographystyle{plainnat}\end{document}}

% 
%\newcommand{\citep}[1]{\cite{#1}}

\colorSectionsForArticle



\usepackage[draft]{fixme}
\fxusetheme{color}

\beginArtNoToc
\generatetitle{PHY485H1F Modern Optics.  Lecture 20: XXX.  Taught by Prof.\ Joseph Thywissen}
%\chapter{XXX}
\label{chap:modernOpticsLecture20}

%\section{Disclaimer}
%
%Peeter's lecture notes from class.  May not be entirely coherent.

\section{Gaussian beams (cont)}

\fxwarning{review lecture 20}{work through this lecture in detail.}

\begin{dmath}\label{eqn:modernOpticsLecture20:10}
u_{00} = \frac{w_0}{w(z)} \exp
\left(
-\frac{r^2}{w^2(z) + \frac{i k r^2}{R(z)} - i \phi(z) 
\right)
\end{dmath}

\begin{dmath}\label{eqn:modernOpticsLecture20:30}
w^2(z) = w_0^2 \left( 
1 + \frac{z^2}{z_0^2}
\right)
\end{dmath}

\begin{dmath}\label{eqn:modernOpticsLecture20:50}
z_0 = \frac{\pi w_0^2}{\lambda}
\end{dmath}

\begin{dmath}\label{eqn:modernOpticsLecture20:70}
\inv{R(z)} = \frac{z}{z^2 + z_0^2}
\end{dmath}

\begin{dmath}\label{eqn:modernOpticsLecture20:90}
\phi(z) = \Atan\left(
\frac{z}{z_0}
\right)
\end{dmath}

\paragraph{GUOY phase shift}

FIXME: GUOY: a name?

Recall

\begin{dmath}\label{eqn:modernOpticsLecture20:110}
\Psi(x, y, z) = \Psi_0 u(x, y, z) e^{i k z - i \omega t}
\end{dmath}

\begin{dmath}\label{eqn:modernOpticsLecture20:130}
\Psi(0, 0, z) = \Psi_0 \frac{w_0}{w(z)} e^{i k z - i \omega t}
\end{dmath}

What is the \underline{phase velocity} at $z \ll z_0$?

F1

Want

\begin{dmath}\label{eqn:modernOpticsLecture20:150}
-i \phi(z) + i k z - i \omega t = i \text{constant}
\end{dmath}

Using 
\begin{subequations}
\begin{dmath}\label{eqn:modernOpticsLecture20:170}
\phi = \Atan(z/z_0)
\end{dmath}
\begin{dmath}\label{eqn:modernOpticsLecture20:190}
\ddz{\phi} = \frac{z_0}{z^2 + z_0^2}
\end{dmath}
\begin{dmath}\label{eqn:modernOpticsLecture20:210}
\ddt{\phi} = \ddt{z} \ddz{\phi}
\end{dmath}
\end{subequations}

Take time derivatives

\begin{dmath}\label{eqn:modernOpticsLecture20:230}
-i \ddt{\phi(z)} + i k \ddt{z} - i \omega = 0
\end{dmath}

Solve for $dz/dt$

\begin{equation}\label{eqn:modernOpticsLecture20:250}
\boxed{
\begin{aligned}
\ddt{z} &= V_ph = \frac{\omega}{k_eff} \\
k_eff &= k - \frac{z_0}{z^2 + z_0^2}
\end{aligned}
}
\end{equation}

\begin{equation}\label{eqn:modernOpticsLecture20:270}
V_ph > c
\end{equation}

At $z = 0$

\begin{dmath}\label{eqn:modernOpticsLecture20:290}
k_eff 
= k - \frac{1}{z_0}
= k - \frac{2}{k \omega_0^2}
\end{dmath}

\paragraph{Cavity}

\begin{itemize}
\item Mode to be a solution of cavity.  Mirror ``undoes'' propagation.
\item Round trip phase shift is $2 \pi (\text{integer})$ for resonance.
\end{itemize}

F2

\paragraph{Higher order modes}

\begin{dmath}\label{eqn:modernOpticsLecture20:310}
u_{lm} (x, y, z) \sim \frac{w_0}{w(z)} \exp
\left(
-\frac{r^2}{w^2(z) + \frac{i k r^2}{R(z)} - i (m + l + 1) \phi(z) 
\right)
\times
H_l \left(
\frac{\sqrt{2} x}{w(z)}
\right)
H_m \left(
\frac{\sqrt{2} y}{w(z)}
\right)
\end{dmath}

\begin{equation}\label{eqn:modernOpticsLecture20:330}
\begin{aligned}
H_0(x) &= 1 \\
H_1(x) &= 2 x \\
H_2(x) &= 4 x^2 - 1
\end{aligned}
\end{equation}

Now get

\begin{equation}\label{eqn:modernOpticsLecture20:350}
k_eff = k - ( M + l + 1) \frac{z_0}{z^2 + z_0^2}
\end{equation}

\paragraph{Beam parameter}

We want to look at how the Gaussian beam interacts with mirrors to get an idea of how the beam will behave in a cavity (without starting over at the Helmholtz equation).  Bringing back in our $q$ notation

\begin{equation}\label{eqn:modernOpticsLecture20:370}
\inv{q(z)} = \inv{R(z)} + i \frac{\lambda}{\pi w^2(z)}
\end{equation}

where $\Real \inv{q(z)}$ gives curvature, and $\Imag \inv{q(z)}$ gives beam radius.

Now

\begin{dmath}\label{eqn:modernOpticsLecture20:390}
u_{lm} 
= \frac{c_{lm}}{w(z)} H_l H_m e^{\frac{i k r^2}{2 q}} e^{-i (l + m + 1) \phi}
\end{dmath}
\begin{dmath}\label{eqn:modernOpticsLecture20:410}
u_{00} 
= \frac{w_0}{w(z)} e^{\frac{i k r^2}{2 q}} e^{-i \phi}
\end{dmath}

Found in uniform medium

\begin{equation}\label{eqn:modernOpticsLecture20:430}
q(z) = z - i z_0
\end{equation}

Know that if $q_1$ at some position $z_1$ then at $z_2$ 

\begin{equation}\label{eqn:modernOpticsLecture20:450}
q_2 = q_1 + (z_2 - z_1)
\end{equation}

% latex o umlaut
\paragraph{Mobius Transform}

\begin{equation}\label{eqn:modernOpticsLecture20:470}
q' = \frac{ A q + B }{C q + D}
\end{equation}

where coefficients same as we used in geometric optics 

i.e. $A$, $B$, $C$, $D$ transformation, such as that of a lens: 

\begin{equation}\label{eqn:modernOpticsLecture20:490}
M = 
\begin{bmatrix}
1 & 0 \\
-\inv{f} & 1
\end{bmatrix}
\end{equation}

($A = 1$, $B = 0$, $C = -1/f$, $D = 1$).

This happens to be (not to be proven) that this is exactly how a Gaussian lens behaves when it encounters a lens/mirror/...

For a lens interaction we have

\begin{equation}\label{eqn:modernOpticsLecture20:530}
q' = \frac{(1) q + (0)}{(-1/f) q + (1)} = \inv{-\inv{f} + \inv{q}}
\end{equation}

\paragraph{Check for free propagation}

M = 
\begin{bmatrix}
1 & L \\
0 & 1
\end{bmatrix}

\begin{equation}\label{eqn:modernOpticsLecture20:510}
q' = \frac{(1) q + (L)}{(0)q + (1)} = q + L
\end{equation}

which is what we know from \ref{eqn:modernOpticsLecture20:450}.

For the lens transformation of \ref{eqn:modernOpticsLecture20:530} we have

\begin{equation}\label{eqn:modernOpticsLecture20:n}
\inv{q'} = \inv{q} - \inv{f}
\end{equation}

so that 

\begin{equation}\label{eqn:modernOpticsLecture20:n}
u = e^{\frac{i k r^2}{2 q}} \rightarrow u e^{-\frac{i k r^2}{2 f}}
\end{equation}

%\EndArticle
\EndNoBibArticle
