%
% Copyright � 2014 Peeter Joot.  All Rights Reserved.
% Licenced as described in the file LICENSE under the root directory of this GIT repository.
%
\makeproblem{Heat conduction}{multiphysics:problemSet1:3}{ 
\makesubproblem
{
In this problem we will examine the heat conducting bar basic example \(\cdots\)
%, but will consider the case of a �leaky� bar to give you prac-
%tice developing a numerical technique for a new physical problem. With
%an appropriate input file, the simulator you developed in problem 1 can
%be used to solve numerically the one-dimensional Poisson equation with
%arbitrary boundary conditions. The Poisson equation can be used to de-
%termine steady-state temperature distribution in a heat-conducting bar,
%as in
%� 2 T(x)
%�x 2
%=
%� a
%� m
%(T(x) � T 0 ) �
%H(x)
%� m
%(2)
%where T(x) is the temperature at a point in space x, H(x) is the heat
%generated at x, � m is the thermal conductivity along the metal bar, and
%� a is the thermal conductivity from the bar to the surrounding air. The
%temperature T 0 is the surrounding air temperature. The ratio
%� a
%� m
%will be
%small as heat moves much more easily along the bar than dissipates from
%the bar into the surrounding air.
%Use your Matlab simulator to numerically solve the above Poisson equa-
%tion for T(x), x � [0,1], given H(x) = 50(sin(2�x)) 2 for x � [0,1],
%� a = 0.001, and � m = 0.1. In addition, assume the ambient air tem-
%perature is T 0 = 400, and T(0) = 250 and T(1) = 250. The boundary
%conditions at x = 0 and x = 1 model heat sink connections to a cool
%metal cabinet at both ends of the package. That is, it is assumed that
%the heat sink connections will insure both ends of the package are fixed
%at near room temperature.
%4
%To represent equation (2) as a circuit, you must first discretize your
%bar along the spatial variable x in small sections of length �x, and
%approximate the derivatives using a finite difference formula, e.g.,
%� 2 T(x)
%�x 2
%�
%T(x+�x)�T(x)
%�x
%�
%T(x)�T(x��x)
%�x
%�x
%(3)
Then, interpret the discretized equation as a KCL using the electrothermal analogy where temperature corresponds to node voltage, and heat flow to current. Draw the equivalent circuit you obtained.
}{multiphysics:problemSet1:3a}

\makesubproblem
{
Plot \(T(x)\) in \(x \in [0,1]\).
}{multiphysics:problemSet1:3b}

\makesubproblem
{
In your numerical calculation, how did you choose \(\Delta x\)? Justify the choice of \(\Delta x\).
}{multiphysics:problemSet1:3c}

\makesubproblem
{
Now use your simulator to numerically solve the above equation \(\cdots\)
%for T(x), x � [0,1], given H(x) = 50 for x � [0,1], � a = 0.001, and
%� m = 0.1. In addition, assume the ambient air temperature is T 0 = 400,
%and there is not heat flow at both ends of the bar. The zero heat flow
%boundary condition at x = 0 and x = 1 implies that there are no heat
%sinks at the ends of the package. Since heat flow is given by
%heatflow = � �T
%�x
%zero heat flow at the boundaries means that T(0) and T(1) are unknown,
%but
%�T
%�x (0) = 0 and
%�T
%�x (1) = 0.
%Given the zero-heat-flow boundary condition, what is the new equivalent
%circuit? How the different boundary condition maps into the equivalent circuit?
}{multiphysics:problemSet1:3d}

\makesubproblem
{
Plot the new temperature profile.
}{multiphysics:problemSet1:3e}

\makesubproblem
{
Explain the temperature distributions that you obtained from a physical standpoint.
}{multiphysics:problemSet1:3f}
} % makeproblem


\makeanswer{multiphysics:problemSet1:3}{ 

\makeSubAnswer{ }{multiphysics:problemSet1:3a}

TODO.
\makeSubAnswer{ }{multiphysics:problemSet1:3b}

TODO.
\makeSubAnswer{ }{multiphysics:problemSet1:3c}

TODO.
\makeSubAnswer{ }{multiphysics:problemSet1:3d}

TODO.
\makeSubAnswer{ }{multiphysics:problemSet1:3e}

TODO.
\makeSubAnswer{ }{multiphysics:problemSet1:3f}

TODO.
}

