\documentclass[]{eliblog}

\usepackage{amsmath}
\usepackage{mathpazo}

%
% shorthand for bold symbols, convenient for vectors and matrices
%
\newcommand{\Ba}[0]{\mathbf{a}}
\newcommand{\Bb}[0]{\mathbf{b}}
\newcommand{\Bc}[0]{\mathbf{c}}
\newcommand{\Bd}[0]{\mathbf{d}}
\newcommand{\Be}[0]{\mathbf{e}}
\newcommand{\Bf}[0]{\mathbf{f}}
\newcommand{\Bg}[0]{\mathbf{g}}
\newcommand{\Bh}[0]{\mathbf{h}}
\newcommand{\Bi}[0]{\mathbf{i}}
\newcommand{\Bj}[0]{\mathbf{j}}
\newcommand{\Bk}[0]{\mathbf{k}}
\newcommand{\Bl}[0]{\mathbf{l}}
\newcommand{\Bm}[0]{\mathbf{m}}
\newcommand{\Bn}[0]{\mathbf{n}}
\newcommand{\Bo}[0]{\mathbf{o}}
\newcommand{\Bp}[0]{\mathbf{p}}
\newcommand{\Bq}[0]{\mathbf{q}}
\newcommand{\Br}[0]{\mathbf{r}}
\newcommand{\Bs}[0]{\mathbf{s}}
\newcommand{\Bt}[0]{\mathbf{t}}
\newcommand{\Bu}[0]{\mathbf{u}}
\newcommand{\Bv}[0]{\mathbf{v}}
\newcommand{\Bw}[0]{\mathbf{w}}
\newcommand{\Bx}[0]{\mathbf{x}}
\newcommand{\By}[0]{\mathbf{y}}
\newcommand{\Bz}[0]{\mathbf{z}}
\newcommand{\BA}[0]{\mathbf{A}}
\newcommand{\BB}[0]{\mathbf{B}}
\newcommand{\BC}[0]{\mathbf{C}}
\newcommand{\BD}[0]{\mathbf{D}}
\newcommand{\BE}[0]{\mathbf{E}}
\newcommand{\BF}[0]{\mathbf{F}}
\newcommand{\BG}[0]{\mathbf{G}}
\newcommand{\BH}[0]{\mathbf{H}}
\newcommand{\BI}[0]{\mathbf{I}}
\newcommand{\BJ}[0]{\mathbf{J}}
\newcommand{\BK}[0]{\mathbf{K}}
\newcommand{\BL}[0]{\mathbf{L}}
\newcommand{\BM}[0]{\mathbf{M}}
\newcommand{\BN}[0]{\mathbf{N}}
\newcommand{\BO}[0]{\mathbf{O}}
\newcommand{\BP}[0]{\mathbf{P}}
\newcommand{\BQ}[0]{\mathbf{Q}}
\newcommand{\BR}[0]{\mathbf{R}}
\newcommand{\BS}[0]{\mathbf{S}}
\newcommand{\BT}[0]{\mathbf{T}}
\newcommand{\BU}[0]{\mathbf{U}}
\newcommand{\BV}[0]{\mathbf{V}}
\newcommand{\BW}[0]{\mathbf{W}}
\newcommand{\BX}[0]{\mathbf{X}}
\newcommand{\BY}[0]{\mathbf{Y}}
\newcommand{\BZ}[0]{\mathbf{Z}}

\newcommand{\Bzero}[0]{\mathbf{0}}
\newcommand{\Btheta}[0]{\boldsymbol{\theta}}
\newcommand{\Btau}[0]{\boldsymbol{\tau}}
\newcommand{\Bomega}[0]{\boldsymbol{\omega}}

%
% shorthand for unit vectors
%
\newcommand{\acap}[0]{\hat{\Ba}}
\newcommand{\bcap}[0]{\hat{\Bb}}
\newcommand{\ccap}[0]{\hat{\Bc}}
\newcommand{\dcap}[0]{\hat{\Bd}}
\newcommand{\ecap}[0]{\hat{\Be}}
\newcommand{\fcap}[0]{\hat{\Bf}}
\newcommand{\gcap}[0]{\hat{\Bg}}
\newcommand{\hcap}[0]{\hat{\Bh}}
\newcommand{\icap}[0]{\hat{\Bi}}
\newcommand{\jcap}[0]{\hat{\Bj}}
\newcommand{\kcap}[0]{\hat{\Bk}}
\newcommand{\lcap}[0]{\hat{\Bl}}
\newcommand{\mcap}[0]{\hat{\Bm}}
\newcommand{\ncap}[0]{\hat{\Bn}}
\newcommand{\ocap}[0]{\hat{\Bo}}
\newcommand{\pcap}[0]{\hat{\Bp}}
\newcommand{\qcap}[0]{\hat{\Bq}}
\newcommand{\rcap}[0]{\hat{\Br}}
\newcommand{\scap}[0]{\hat{\Bs}}
\newcommand{\tcap}[0]{\hat{\Bt}}
\newcommand{\ucap}[0]{\hat{\Bu}}
\newcommand{\vcap}[0]{\hat{\Bv}}
\newcommand{\wcap}[0]{\hat{\Bw}}
\newcommand{\xcap}[0]{\hat{\Bx}}
\newcommand{\ycap}[0]{\hat{\By}}
\newcommand{\zcap}[0]{\hat{\Bz}}
\newcommand{\thetacap}[0]{\hat{\Btheta}}

%
% to write R^n and C^n in a distinguishable fashion.  Perhaps change this
% to the double lined characters upon figuring out how to do so.
%
\newcommand{\C}[1]{$\mathbb{C}^{#1}$}
\newcommand{\R}[1]{$\mathbb{R}^{#1}$}

%
% various generally useful helpers
%

% derivative of #1 wrt. #2:
\newcommand{\D}[2] {\frac {d#2} {d#1}}

\newcommand{\inv}[1]{\frac{1}{#1}}
\newcommand{\cross}[0]{\times}

\newcommand{\abs}[1]{\lvert{#1}\rvert}
\newcommand{\norm}[1]{\lVert{#1}\rVert}
\newcommand{\innerprod}[2]{\langle{#1}, {#2}\rangle}
\newcommand{\dotprod}[2]{{#1} \cdot {#2}}
\newcommand{\bdotprod}[2]{\left({#1} \cdot {#2}\right)}
\newcommand{\crossprod}[2]{{#1} \cross {#2}}
\newcommand{\tripleprod}[3]{\dotprod{\left(\crossprod{#1}{#2}\right)}{#3}}

\DeclareMathOperator{\Proj}{Proj}
\DeclareMathOperator{\Span}{span}
\DeclareMathOperator{\Sgn}{sgn}
\DeclareMathOperator{\Area}{Area}
\DeclareMathOperator{\Volume}{Volume}

%
% A few miscellaneous things specific to this document
%
\newcommand{\crossop}[1]{\crossprod{#1}{}}

% R2 vector.
\newcommand{\VectorTwo}[2]{
\begin{bmatrix}
 {#1} \\
 {#2}
\end{bmatrix}
}

\newcommand{\VectorN}[1]{
\begin{bmatrix}
{#1}_1 \\
{#1}_2 \\
\vdots \\
{#1}_N \\
\end{bmatrix}
}

\newcommand{\DETuvij}[4]{
\begin{vmatrix}
 {#1}_{#3} & {#1}_{#4} \\
 {#2}_{#3} & {#2}_{#4}
\end{vmatrix}
}

\newcommand{\DETuvwijk}[6]{
\begin{vmatrix}
 {#1}_{#4} & {#1}_{#5} & {#1}_{#6} \\
 {#2}_{#4} & {#2}_{#5} & {#2}_{#6} \\
 {#3}_{#4} & {#3}_{#5} & {#3}_{#6}
\end{vmatrix}
}

\newcommand{\DETuvwxijkl}[8]{
\begin{vmatrix}
 {#1}_{#5} & {#1}_{#6} & {#1}_{#7} & {#1}_{#8} \\
 {#2}_{#5} & {#2}_{#6} & {#2}_{#7} & {#2}_{#8} \\
 {#3}_{#5} & {#3}_{#6} & {#3}_{#7} & {#3}_{#8} \\
 {#4}_{#5} & {#4}_{#6} & {#4}_{#7} & {#4}_{#8} \\
\end{vmatrix}
}

%\newcommand{\DETuvwxyijklm}[10]{
%\begin{vmatrix}
% {#1}_{#6} & {#1}_{#7} & {#1}_{#8} & {#1}_{#9} & {#1}_{#10} \\
% {#2}_{#6} & {#2}_{#7} & {#2}_{#8} & {#2}_{#9} & {#2}_{#10} \\
% {#3}_{#6} & {#3}_{#7} & {#3}_{#8} & {#3}_{#9} & {#3}_{#10} \\
% {#4}_{#6} & {#4}_{#7} & {#4}_{#8} & {#4}_{#9} & {#4}_{#10} \\
% {#5}_{#6} & {#5}_{#7} & {#5}_{#8} & {#5}_{#9} & {#5}_{#10}
%\end{vmatrix}
%}

% R3 vector.
\newcommand{\VectorThree}[3]{
\begin{bmatrix}
 {#1} \\
 {#2} \\
 {#3}
\end{bmatrix}
}



\author{Peeter Joot}
\email{peeter.joot@gmail.com}


\chapter{Electrodynamic field energy for vacuum.}
\label{chap:fourierMaxVac}
%\useCCL
\blogpage{http://sites.google.com/site/peeterjoot/math2009/fourierMaxVac.pdf}
\date{Dec 16, 2009}
\revisionInfo{fourierMaxVac.tex}

\beginArtWithToc
%\beginArtNoToc

\section{Motivation.}

We now know how to formulate the energy momentum tensor for complex vector fields (ie. phasors) in the Geometric Algebra formalism.  To recap, for the field $F = \BE + I c \BB$, where $\BE$ and $\BB$ may be complex vectors we have for Maxwell's equation

\begin{align}\label{eqn:fourierMaxVac:1}
\grad F = J/\epsilon_0 c.
\end{align}

This is a doubly complex representation, with the four vector pseudoscalar $I = \gamma_0 \gamma_1 \gamma_2 \gamma_3$ acting as a non-commutatitive imaginary, as well as real and imaginary parts for the electric and magnetic field vectors.  We take the real part (not the scalar part) of any bivector solution $F$ of Maxwell's equation as the actual solution, but allow ourself the freedom to work with the complex phasor representation when convenient.  In these phasor vectors, the imaginary $i$, as in $\BE = \Real(\BE) + i \Imag(\BE)$, is a commuting imaginary, commuting with all the multivector elements in the algebra.

The real valued, four vector, energy momentum tensor $T(a)$ was found to be

\begin{align}\label{eqn:fourierMaxVac:2}
T(a) = \frac{\epsilon_0}{4} \Bigl( \conjugateStar{F} a \tilde{F} + \tilde{F} a \conjugateStar{F} \Bigr) = 
-\frac{\epsilon_0}{2} \Real \Bigl( \conjugateStar{F} a F \Bigr).
\end{align}

To supply some context that gives meaning to this tensor the associated conservation relationship was found to be

\begin{align}\label{eqn:fourierMaxVac:3}
\grad \cdot T(a) &= a \cdot \inv{ c } \Real \left( J \cdot \conjugateStar{F} \right).
\end{align}

and in particular for $a = \gamma^0$, this four vector divergence takes the form

\begin{align}\label{eqn:fourierMaxVac:4}
\PD{t}{}\frac{\epsilon_0}{2}(\BE \cdot \conjugateStar{\BE} + c^2 \BB \cdot \conjugateStar{\BB})
+ \spacegrad \cdot \inv{\mu_0} \Real (\BE \cross \conjugateStar{\BB} )
+ \Real( \BJ \cdot \conjugateStar{\BE} ) 
= 0,
\end{align}

relating the energy term $T^{00} = T(\gamma^0) \cdot \gamma^0$ and the Poynting spatial vector $T(\gamma^0) \wedge \gamma^0$ with the current density and electric field product that constitutes the energy portion of the Lorentz force density.

Let's apply this to calculating the energy associated with the field that is periodic within a rectangular prism as done by Bohm in \cite{bohm1989qt}.  We do not necessarily need the Geometric Algebra formalism for this calculation, but this will be a fun way to attempt it.

\section{Setup}

Let's assume a Fourier representation for the four vector potential $A$ for the field $F = \grad \wedge A$.  That is

\begin{align}
\label{eqn:fourierMaxVac:5}
A = \sum_{\Bk} A_\Bk(t) e^{2 \pi i \Bk \cdot \Bx},
\end{align}

where summation is over all wave number triplets $\Bk = (p/\lambda_1,q/\lambda_2,r/\lambda_3)$.  The Fourier coefficients $A_\Bk = {A_\Bk}^\mu \gamma_\mu$ are allowed to be complex valued, as is the resulting four vector $A$, and the associated bivector field $F$.

Fourier inversion follows from

\begin{align}\label{eqn:fourierMaxVac:6}
\delta_{\Bk', \Bk} =
\inv{ \lambda_1 \lambda_2 \lambda_3 }
\int_0^{\lambda_1}
\int_0^{\lambda_2}
\int_0^{\lambda_3} 
e^{2 \pi i \Bk' \cdot \Bx} 
e^{-2 \pi i \Bk \cdot \Bx} dx^1 dx^2 dx^3,
\end{align}

but only this orthogonality relationship and not the Fourier coefficients themselves

\begin{align}
\label{eqn:fourierMaxVac:7}
A_\Bk = 
\inv{ \lambda_1 \lambda_2 \lambda_3 }
\int_0^{\lambda_1}
\int_0^{\lambda_2}
\int_0^{\lambda_3} A(\Bx, t) e^{-2 \pi i \Bk \cdot \Bx} dx^1 dx^2 dx^3,
\end{align}

will be of interest here.  Evaluating the curl for this potential yields

\begin{align}\label{eqn:fourierMaxVac:8}
F = \grad \wedge A
= \sum_{\Bk} \left( \inv{c} \gamma^0 \wedge \dot{A}_\Bk + \sum_{m=1}^3 \gamma^m \wedge A_\Bk \frac{2 \pi i k_m}{\lambda_m} \right) e^{2 \pi i \Bk \cdot \Bx}.
\end{align}

We can now form the energy density

\begin{align}\label{eqn:fourierMaxVac:9}
U = T(\gamma^0) \cdot \gamma^0
=
-\frac{\epsilon_0}{2} \Real \Bigl( \conjugateStar{F} \gamma^0 F \gamma^0 \Bigr).
\end{align}

With implied summation over all repeated integer indexes (even without matching uppers and lowers), this is

\begin{align}\label{eqn:fourierMaxVac:10}
U =
-\frac{\epsilon_0}{2} \sum_{\Bk', \Bk} \Real \gpgradezero{
\left( \inv{c} \gamma^0 \wedge \conjugateStar{\dot{A}_{\Bk'}} - \gamma^m \wedge \conjugateStar{A_{\Bk'}} \frac{2 \pi i k_m'}{\lambda_m} \right) e^{-2 \pi i \Bk' \cdot \Bx}
\gamma^0
\left( \inv{c} \gamma^0 \wedge \dot{A}_\Bk + \gamma^n \wedge A_\Bk \frac{2 \pi i k_n}{\lambda_n} \right) e^{2 \pi i \Bk \cdot \Bx}
\gamma^0
}.
\end{align}

The grade selection used here doesn't change the result since we already have a scalar, but will just make it convenient to filter out any higher order products that will cancel anyways.  Integrating over the volume element and taking advantage of the orthogonality relationship \autoref{eqn:fourierMaxVac:6}, the exponentials are removed, leaving the energy contained in the volume

\begin{align}
\label{eqn:fourierMaxVac:11}
H = %\int U d^3 x =
-\frac{\epsilon_0 \lambda_1 \lambda_2 \lambda_3}{2}
\sum_{\Bk} \Real 
\gpgradezero{
\left( \inv{c} \gamma^0 \wedge \conjugateStar{\dot{A}_{\Bk}} - \gamma^m \wedge \conjugateStar{A_{\Bk}} \frac{2 \pi i k_m}{\lambda_m} \right) 
\gamma^0
\left( \inv{c} \gamma^0 \wedge \dot{A}_\Bk + \gamma^n \wedge A_\Bk \frac{2 \pi i k_n}{\lambda_n} \right) 
\gamma^0
}.
\end{align}

\section{First reduction of the Hamiltonian.}

Let's take the products involved in sequence one at a time, and evaluate, later adding and taking real parts if required all of

\begin{subequations}
\label{eqn:fourierMaxVac:12}
\begin{align}
\inv{c^2}
\gpgradezero{ (\gamma^0 \wedge \conjugateStar{\dot{A}_{\Bk}} ) \gamma^0 (\gamma^0 \wedge \dot{A}_\Bk) \gamma^0 } &=
-\inv{c^2}
\gpgradezero{ (\gamma^0 \wedge \conjugateStar{\dot{A}_{\Bk}} ) (\gamma^0 \wedge \dot{A}_\Bk) } 
\label{eqn:fourierMaxVac:12a}
\\
- \frac{2 \pi i k_m}{c \lambda_m} 
\gpgradezero{ 
(\gamma^m \wedge \conjugateStar{A_{\Bk}} ) \gamma^0 ( \gamma^0 \wedge \dot{A}_\Bk ) \gamma^0
} 
&=
\frac{2 \pi i k_m}{c \lambda_m} 
\gpgradezero{ 
(\gamma^m \wedge \conjugateStar{A_{\Bk}} ) ( \gamma^0 \wedge \dot{A}_\Bk ) 
} 
\label{eqn:fourierMaxVac:12b}
\\
\frac{2 \pi i k_n}{c \lambda_n} 
\gpgradezero{ 
( \gamma^0 \wedge \conjugateStar{\dot{A}_{\Bk}} ) \gamma^0 ( \gamma^n \wedge A_\Bk ) \gamma^0
} 
&=
-\frac{2 \pi i k_n}{c \lambda_n} 
\gpgradezero{ 
( \gamma^0 \wedge \conjugateStar{\dot{A}_{\Bk}} ) ( \gamma^n \wedge A_\Bk ) 
} 
\label{eqn:fourierMaxVac:12c}
\\
-\frac{4 \pi^2 k_m k_n}{\lambda_m \lambda_n}
\gpgradezero{ 
(\gamma^m \wedge \conjugateStar{A_{\Bk}} ) \gamma^0
(\gamma^n \wedge A_\Bk ) \gamma^0
}. &
\label{eqn:fourierMaxVac:12d}
\end{align}
\end{subequations}

The expectation is to obtain a Hamiltonian for the field that has the structure of harmonic oscillators, where the middle two products would have to be zero or sum to zero or have real parts that sum to zero.  The first is expected to contain only products of $\Abs{{\dot{A}_\Bk}^m}^2$, and the last only products of $\Abs{{A_\Bk}^m}^2$.

While initially guessing that \autoref{eqn:fourierMaxVac:12b} and \autoref{eqn:fourierMaxVac:12c} may cancel, this isn't so obviously the case.  The use of cyclic permutation of multivectors within the scalar grade selection operator $\gpgradezero{A B} = \gpgradezero{B A}$ plus a change of dummy summation indexes in one of the two shows that this sum is of the form $Z + \conjugateStar{Z}$.  This sum is intrinsically real, so we can neglect one of the two doubling the other, but we will still be required to show that the real part of either is zero.

Lets reduce these one at a time starting with \autoref{eqn:fourierMaxVac:12a}, and write $\dot{A}_\Bk = \kappa$ temporarily

\begin{align*}
\gpgradezero{ (\gamma^0 \wedge \conjugateStar{\kappa} ) (\gamma^0 \wedge \kappa } 
&=
{\kappa^m}^{\conj} \kappa^{m'}
\gpgradezero{ \gamma^0 \gamma_m \gamma^0 \gamma_{m'} } \\
&=
-{\kappa^m}^{\conj} \kappa^{m'}
\gpgradezero{ \gamma_m \gamma_{m'} }  \\
&=
{\kappa^m}^{\conj} \kappa^{m'}
\delta_{m m'}.
\end{align*}

So the first of our Hamiltonian terms is

\begin{align}\label{eqn:fourierMaxVac:13}
\frac{\epsilon_0 \lambda_1 \lambda_2 \lambda_3}{2 c^2}
\gpgradezero{ (\gamma^0 \wedge \conjugateStar{\dot{A}_\Bk} ) (\gamma^0 \wedge \dot{A}_\Bk } 
&=
\frac{\epsilon_0 \lambda_1 \lambda_2 \lambda_3}{2 c^2}
\Abs{{{\dot{A}}_{\Bk}}^m}^2.
\end{align}

Note that summation over $m$ is still implied here, so we'd be better off with a spatial vector representation of the Fourier coeffienents $\BA_\Bk = A_\Bk \wedge \gamma_0$.  With such a notation, this contribution to the Hamiltonian is

\begin{align}\label{eqn:fourierMaxVac:13b}
\frac{\epsilon_0 \lambda_1 \lambda_2 \lambda_3}{2 c^2} \dot{\BA}_\Bk \cdot \conjugateStar{\dot{\BA}_\Bk}.
\end{align}

To reduce \autoref{eqn:fourierMaxVac:12b} and \autoref{eqn:fourierMaxVac:12b}, this time writing $\kappa = A_\Bk$, we can start with just the scalar selection

\begin{align*}
\gpgradezero{ (\gamma^m \wedge \conjugateStar{\kappa} ) ( \gamma^0 \wedge \dot{\kappa} ) } 
&=
\Bigl( \gamma^m \conjugateStar{(\kappa^0)} - \conjugateStar{\kappa} \underbrace{(\gamma^m \cdot \gamma^0)}_{=0} \Bigr) \cdot \dot{\kappa} \\
&=
\conjugateStar{(\kappa^0)} \dot{\kappa}^m
\end{align*}

Thus the contribution to the Hamiltonian from \autoref{eqn:fourierMaxVac:12b} and \autoref{eqn:fourierMaxVac:12b} is

\begin{align}\label{eqn:fourierMaxVac:14}
\frac{2 \epsilon_0 \lambda_1 \lambda_2 \lambda_3 \pi k_m}{c \lambda_m} \Real \Bigl( i \conjugateStar{(A_\Bk^0)} \dot{A_\Bk}^m \Bigl)
=
\frac{2 \pi \epsilon_0 \lambda_1 \lambda_2 \lambda_3}{c} \Real \Bigl( i \conjugateStar{(A_\Bk^0)} \Bk \cdot \dot{\BA}_\Bk \Bigl).
\end{align}

Most definitively not zero in general.  Our final expansion \autoref{eqn:fourierMaxVac:12d} is the messiest.  Again with $A_\Bk = \kappa$ for short, the grade selection of this term in coordinates is

\begin{align}\label{eqn:fourierMaxVac:15}
\gpgradezero{ (\gamma^m \wedge \conjugateStar{\kappa} ) \gamma^0 (\gamma^n \wedge \kappa ) \gamma^0 }
&=
- \conjugateStar{\kappa_\mu} \kappa^\nu
   \gpgradezero{ (\gamma^m \wedge \gamma^\mu) \gamma^0 (\gamma_n \wedge \gamma_\nu) \gamma^0 }
\end{align}

Expanding this out yields

\begin{align}\label{eqn:fourierMaxVac:15b}
\gpgradezero{ (\gamma^m \wedge \conjugateStar{\kappa} ) \gamma^0 (\gamma^n \wedge \kappa ) \gamma^0 }
&=
- ( \Abs{\kappa_0}^2 - \Abs{A^a}^2 ) \delta_{m n} + \conjugateStar{A^n} A^m.
\end{align}

The contribution to the Hamiltonian from this, with $\phi_\Bk = A^0_\Bk$, is then

\begin{align}\label{eqn:fourierMaxVac:16}
2 \pi^2 \epsilon_0 \lambda_1 \lambda_2 \lambda_3 
\Bigl(
-\Bk^2 \conjugateStar{\phi_\Bk} \phi_\Bk 
+ \Bk^2 (\conjugateStar{\BA_\Bk} \cdot \BA_\Bk)
+ (\Bk \cdot \conjugateStar{\BA_k}) (\Bk \cdot \BA_k)
\Bigr).
\end{align}

A final reassembly of the Hamiltonian from the parts \autoref{eqn:fourierMaxVac:13b} and \autoref{eqn:fourierMaxVac:14} and \autoref{eqn:fourierMaxVac:16} is then

\begin{align}
\label{eqn:fourierMaxVac:17}
H = 
\epsilon_0 \lambda_1 \lambda_2 \lambda_3 \sum_\Bk
\left(
\inv{2 c^2} \Abs{\dot{\BA}_\Bk}^2
+\frac{2 \pi}{c} \Real \Bigl( i \conjugateStar{ \phi_\Bk } (\Bk \cdot \dot{\BA}_\Bk) \Bigl)
+2 \pi^2 
\Bigl(
\Bk^2 ( -\Abs{\phi_\Bk}^2 + \Abs{\BA_\Bk}^2 ) + \Abs{\Bk \cdot \BA_\Bk}^2
\Bigr)
\right).
\end{align}

This is finally reduced to a completely real expression, and one without any explicit Geometric Algebra.  All the four vector Fourier vector potentials written out explicitly in terms of the spacetime split $A_\Bk = (\phi_\Bk, \BA_\Bk)$, which is natural since an explicit time and space split was the starting point.  

\section{Gauge transformation to simplify the Hamiltonian.}

While \autoref{eqn:fourierMaxVac:17} has considerably simpler form than \autoref{eqn:fourierMaxVac:11}, what was expected, was something that looked like the Harmonic oscillator.  On the surface this does not appear to be such a beast.  Exploitation of gauge freedom is required to make the simplification that puts things into the Harmonic oscillator form.

If we are to change our four vector potential $A \rightarrow A + \grad \psi$, then Maxwell's equation takes the form

\begin{align}\label{eqn:fourierMaxVac:30}
J/\epsilon_0 c = \grad (\grad \wedge (A + \grad \psi) = \grad (\grad \wedge A) + \grad (\underbrace{\grad \wedge \grad \psi}_{=0}),
\end{align}

which is unchanged by the addition of the gradient to any original potential solution to the equation.  In coordinates this is a transformation of the form

\begin{align}\label{eqn:fourierMaxVac:31}
A^\mu \rightarrow A^\mu + \partial_\mu \psi,
\end{align}

and we can use this to force any one of the potential coordinates to zero.  For this problem, it appears that it is desirable to seek a $\psi$ such that $A^0 + \partial_0 \psi = 0$.  That is

\begin{align}\label{eqn:fourierMaxVac:32}
\sum_\Bk \phi_\Bk(t) e^{2 \pi i \Bk \cdot \Bx} + \inv{c} \partial_t \psi = 0.
\end{align}

Or,

\begin{align}\label{eqn:fourierMaxVac:33}
\psi(\Bx,t) = \psi(\Bx,0) -\inv{c} \sum_\Bk e^{2 \pi i \Bk \cdot \Bx} \int_{\tau=0}^t \phi_\Bk(\tau).
\end{align}

With such a transformation, the $\phi_\Bk$ and $\dot{\BA}_\Bk$ cross term in the Hamiltonian \autoref{eqn:fourierMaxVac:17} vanishes, as does the $\phi_\Bk$ term in the four vector square of the last term, leaving just

\begin{align}
\label{eqn:fourierMaxVac:17b}
H = 
\frac{\epsilon_0}{c^2} \lambda_1 \lambda_2 \lambda_3 \sum_\Bk
\left(
\inv{2} \Abs{\dot{\BA}_\Bk}^2
+
\inv{2} \Bigl(
(2 \pi c \Bk)^2 \Abs{\BA_\Bk}^2 + \Abs{ ( 2 \pi c \Bk) \cdot \BA_\Bk}^2
\Bigr)
\right).
\end{align}

Additionally, wedging \autoref{eqn:fourierMaxVac:5} with $\gamma_0$ now does not loose any information so our potential Fourier series is reduced to just

\begin{subequations}
\label{eqn:fourierMaxVac:5b}
\begin{align}
\BA &= \sum_{\Bk} \BA_\Bk(t) e^{2 \pi i \Bk \cdot \Bx} \\
\BA_\Bk &= 
\inv{ \lambda_1 \lambda_2 \lambda_3 }
\int_0^{\lambda_1}
\int_0^{\lambda_2}
\int_0^{\lambda_3} \BA(\Bx, t) e^{-2 \pi i \Bk \cdot \Bx} dx^1 dx^2 dx^3.
\end{align}
\end{subequations}

The desired harmonic oscillator form would be had in \autoref{eqn:fourierMaxVac:17b} if it were not for the $\Bk \cdot \BA_\Bk$ term.  Does that vanish?  Returning to Maxwell's equation should answer that question, but first it has to be expressed in terms of the vector potential.  While $\BA = A \wedge \gamma_0$, the lack of an $A^0$ component means that this can be inverted as

\begin{align}\label{eqn:fourierMaxVac:41}
A = \BA \gamma_0 = -\gamma_0 \BA.
\end{align}

The gradient can also be factored scalar and spatial vector components

\begin{align}\label{eqn:fourierMaxVac:42}
\grad = \gamma^0 ( \partial_0 + \spacegrad ) = ( \partial_0 - \spacegrad ) \gamma^0.
\end{align}

So, with this $A^0 = 0$ gauge choice the bivector field $F$ is

\begin{align}\label{eqn:fourierMaxVac:43}
F = \grad \wedge A = \inv{2} \left( \rgrad A - A \lgrad \right) 
\end{align}

From the left the gradient action on $A$ is

\begin{align*}
\rgrad A 
&= ( \partial_0 - \spacegrad ) \gamma^0 (-\gamma_0 \BA) \\
&= ( -\partial_0 + \rspacegrad ) \BA,
%&= \partial_0 \BA 
%+ \spacegrad \cdot \BA 
%+ \spacegrad \wedge \BA 
%\\
\end{align*}

and from the right

\begin{align*}
A \lgrad
&= 
\BA \gamma_0 \gamma^0 ( \partial_0 + \spacegrad ) \\
&= 
\BA ( \partial_0 + \spacegrad ) \\
&= 
\partial_0 \BA 
+ \BA \lspacegrad 
\end{align*}

Taking the difference we have

\begin{align*}
F 
&= 
\inv{2} \Bigl( -\partial_0 \BA + \rspacegrad \BA -  \partial_0 \BA - \BA \lspacegrad \Bigr).
\end{align*}

Which is just

\begin{align}\label{eqn:fourierMaxVac:44}
F = -\partial_0 \BA + \spacegrad \wedge \BA.
\end{align}

For this vacuum case, premultiplication of Maxwell's equation by $\gamma_0$ gives

\begin{align*}
0 
&= \gamma_0 \grad ( -\partial_0 \BA + \spacegrad \wedge \BA ) \\
&= (\partial_0 + \spacegrad)( -\partial_0 \BA + \spacegrad \wedge \BA ) \\
&= -\inv{c^2} \partial_{tt} \BA 
- \partial_0 \spacegrad \cdot \BA 
- \partial_0 \spacegrad \wedge \BA 
+ \partial_0 ( \spacegrad \wedge \BA ) 
+ \underbrace{\spacegrad \cdot ( \spacegrad \wedge \BA ) }_{\spacegrad^2 \BA - \spacegrad (\spacegrad \cdot \BA)}
+ \underbrace{\spacegrad \wedge ( \spacegrad \wedge \BA )}_{=0} \\
\end{align*}

The spatial bivector and trivector grades are all zero.  Equating the remaining scalar and vector components to zero separately yields a pair of equations in $\BA$

\begin{subequations}
\label{eqn:fourierMaxVac:45}
\begin{align}
0 &= \partial_t (\spacegrad \cdot \BA) \\
0 &= -\inv{c^2} \partial_{tt} \BA + \spacegrad^2 \BA + \spacegrad (\spacegrad \cdot \BA) 
\end{align}
\end{subequations}

If the divergence of the vector potential is constant we have just a wave equation.  Let's see what that divergence is with the assumed Fourier representation

\begin{align*}
\spacegrad \cdot \BA 
&=
\sum_{k \ne (0,0,0)} {\BA_\Bk}^m 2 \pi i \frac{k_m}{\lambda_m} e^{2\pi i \Bk \cdot \Bx} \\
&=
2 \pi i \sum_{k \ne (0,0,0)} (\BA_\Bk \cdot \Bk) e^{2\pi i \Bk \cdot \Bx} \\
\end{align*}

Since $\BA_\Bk = \BA_\Bk(t)$, there are two ways for $\partial_t (\spacegrad \cdot \BA) = 0$.  For each $\Bk \ne 0$ there must be a requirement for either $\BA_\Bk \cdot \Bk = 0$ or $\BA_\Bk = \text{constant}$.  The constant $\BA_\Bk$ solution to the first equation appears to represent a standing spatial wave with no time dependence.  Is that of any interest?

The more interesting seeming case is where we have some non-static time varing state.  In this case, if $\BA_\Bk \cdot \Bk$ for all $\Bk \ne 0$ the second of these Maxwell's equations is just the vector potential wave equation, since the divergence is zero.  That is

\begin{align}\label{eqn:fourierMaxVac:50}
0 &= -\inv{c^2} \partial_{tt} \BA + \spacegrad^2 \BA 
\end{align}

Solving this isn't really what is of interest, since the objective was just to determine if the divergence could be assumed to be zero.  This shows then, that if the transverse solution to Maxwell's equation is picked, the Hamiltonian for this field, with this gauge choice, becomes

\begin{align}
\label{eqn:fourierMaxVac:17c}
H = 
\frac{\epsilon_0}{c^2} \lambda_1 \lambda_2 \lambda_3 \sum_\Bk
\left(
\inv{2} \Abs{\dot{\BA}_\Bk}^2
+
\inv{2} 
(2 \pi c \Bk)^2 \Abs{\BA_\Bk}^2 
\right).
\end{align}

\section{Conclusions and followup.}

The objective was met, a reproduction of Bohm's Harmonic oscillator result using a complex exponential Fourier series instead of separate sine and cosines.

The reason for Bohm's choice to fix zero divergence as the gauge choice upfront is now clear.  That automatically cuts complexity from the results.  Figuring out how to work this problem with complex valued potentials and also using the Geometric Algebra formulation probably also made the work a bit more difficult since blundering through both simultaneously was required instead of just one at a time.

This was an interesting exersize though, since doing it this way I am able to understand all the intermediate steps.  Bohm employed some subtler argumentation to eliminate the scalar potential $\phi$ upfront, and I have to admit I did not follow his logic, whereas blindly following where the math leads me all makes sense.

As a bit of followup, I'd like to consider the constant $\BA_\Bk$ case, and any implications of the freedom to pick $\BA_0$.  I'd also like to construct the Poynting vector $T(\gamma^0) \wedge \gamma_0$, and see what the structure of that is with this Fourier representation.

\EndArticle
%\EndNoBibArticle
