%
% Copyright � 2016 Peeter Joot.  All Rights Reserved.
% Licenced as described in the file LICENSE under the root directory of this GIT repository.
%
%{
\newcommand{\authorname}{Peeter Joot}
\newcommand{\email}{peeterjoot@protonmail.com}
\newcommand{\basename}{FIXMEbasenameUndefined}
\newcommand{\dirname}{notes/FIXMEdirnameUndefined/}

\renewcommand{\basename}{fresnelSumAndDifferenceAngleFormulas}
%\renewcommand{\dirname}{notes/phy1520/}
\renewcommand{\dirname}{notes/ece1228-electromagnetic-theory/}
%\newcommand{\dateintitle}{}
%\newcommand{\keywords}{}

\newcommand{\authorname}{Peeter Joot}
\newcommand{\onlineurl}{http://sites.google.com/site/peeterjoot2/math2013/\basename.pdf}
\newcommand{\sourcepath}{\dirname\basename.tex}
\newcommand{\generatetitle}[1]{\chapter{#1}}

\newcommand{\vcsinfo}{%
\section*{}
\noindent{\color{DarkOliveGreen}{\rule{\linewidth}{0.1mm}}}
\paragraph{Document version}
%\paragraph{\color{Maroon}{Document version}}
{
\small
\begin{itemize}
\item Available online at:\\ 
\href{\onlineurl}{\onlineurl}
\item Git Repository: \input{./.revinfo/gitRepo.tex}
\item Source: \sourcepath
\item last commit: \input{./.revinfo/gitCommitString.tex}
\item commit date: \input{./.revinfo/gitCommitDate.tex}
\end{itemize}
}
}

%\PassOptionsToPackage{dvipsnames,svgnames}{xcolor}
\PassOptionsToPackage{square,numbers}{natbib}
\documentclass{scrreprt}

\usepackage[left=2cm,right=2cm]{geometry}
\usepackage[svgnames]{xcolor}
\usepackage{peeters_layout}

\usepackage{natbib}

\usepackage[
colorlinks=true,
bookmarks=false,
pdfauthor={\authorname, \email},
backref 
]{hyperref}

% http://tex.stackexchange.com/questions/75773/how-to-reference-problems-by-the-text-label-in-an-exercise-envioronment
\usepackage[english]{cleveref}
\crefname{Exercise}{exercise}{exercises}
\Crefname{Exercise}{Exercise}{Exercises}

\RequirePackage{titlesec}
\RequirePackage{ifthen}

% http://stackoverflow.com/questions/4932910/date-in-the-tabular-environment
\makeatletter
\let\insertdate\@date
\makeatother

\titleformat{\chapter}[display]
{\bfseries\Large}
{\color{DarkSlateGrey}\filleft \authorname
\ifthenelse{\isundefined{\studentnumber}}{}{\\ \studentnumber}
\ifthenelse{\isundefined{\email}}{}{\\ \email}
\ifthenelse{\isundefined{\dateintitle}}{}{\\ \insertdate}
%\ifthenelse{\isundefined{\coursename}}{}{\\ \coursename} % put in title instead.
}
{4ex}
{\color{DarkOliveGreen}{\titlerule}\color{Maroon}
\vspace{2ex}%
\filright}
[\vspace{2ex}%
\color{DarkOliveGreen}\titlerule
]

\newcommand{\beginArtWithToc}[0]{\begin{document}\tableofcontents}
\newcommand{\beginArtNoToc}[0]{\begin{document}}
\newcommand{\EndNoBibArticle}[0]{\end{document}}
\newcommand{\EndArticle}[0]{\bibliography{Bibliography}\bibliographystyle{plainnat}\end{document}}

% 
%\newcommand{\citep}[1]{\cite{#1}}

\colorSectionsForArticle



\usepackage{peeters_layout_exercise}
\usepackage{peeters_braket}
\usepackage{peeters_figures}
\usepackage{siunitx}
%\usepackage{mhchem} % \ce{}
%\usepackage{macros_bm} % \bcM
%\usepackage{txfonts} % \ointclockwise

\beginArtNoToc

\generatetitle{Fresnel angular sum and difference formulas}
%\chapter{Fresnel angular sum and difference formulas}
%\label{chap:fresnelSumAndDifferenceAngleFormulas}

In \citep{hecht1998hecht} are some sum and angle difference formulations for the Fresnel formulas given a \( \mu_1 = \mu_2 \) constraint.  The proof of these trig Fresnel equations is left to an exercise, and will be derived here.

We need a couple trig identities to start with.

\begin{dmath}\label{eqn:fresnelSumAndDifferenceAngleFormulas:20}
\begin{aligned}
\sin(a + b)
&=
\Imag\lr{ e^{j(a + b)} } \\
&=
\Imag\lr{
e^{ja} e^{+ jb}
} \\
&=
\Imag\lr{
(\cos a + j \sin a) (\cos b + j \sin b)
} \\
&=
\sin a \cos b + \cos a \sin b.
\end{aligned}
\end{dmath}

Allowing for both signs we have

\begin{dmath}\label{eqn:fresnelSumAndDifferenceAngleFormulas:240}
\begin{aligned}
\sin(a + b) &= \sin a \cos b + \cos a \sin b \\
\sin(a - b) &= \sin a \cos b - \cos a \sin b.
\end{aligned}
\end{dmath}

The mixed sine and cosine product can be expressed as a sum of sines

\begin{dmath}\label{eqn:fresnelSumAndDifferenceAngleFormulas:40}
2 \sin a \cos b = \sin(a + b) + \sin(a - b).
\end{dmath}

With \( 2 x = a + b, 2 y = a - b \), or \( a = x + y, b = x - y \), we find

\begin{dmath}\label{eqn:fresnelSumAndDifferenceAngleFormulas:60}
\begin{aligned}
2 \sin(x + y) \cos (x - y) &= \sin( 2 x ) + \sin( 2 y ) \\
2 \sin(x - y) \cos (x + y) &= \sin( 2 x ) - \sin( 2 y ).
\end{aligned}
\end{dmath}

Returning to the problem.  When \( \mu_1 = \mu_2 \) the Fresnel equations were found to be

\begin{dmath}\label{eqn:fresnelSumAndDifferenceAngleFormulas:100}
\begin{aligned}
r^{\textrm{TE}} &= \frac { n_1 \cos\theta_i - n_2 \cos\theta_t } { n_1 \cos\theta_i + n_2 \cos\theta_t } \\
r^{\textrm{TM}} &= \frac{n_2 \cos\theta_i - n_1 \cos\theta_t }{ n_2 \cos\theta_i + n_1 \cos\theta_t } \\
t^{\textrm{TE}} &= \frac{ 2 n_1 \cos\theta_i } { n_1 \cos\theta_i + n_2 \cos\theta_t } \\
t^{\textrm{TM}} &= \frac{2 n_1 \cos\theta_i }{ n_2 \cos\theta_i + n_1 \cos\theta_t }.
\end{aligned}
\end{dmath}

Using Snell's law, one of \( n_1, n_2 \) can be eliminated, for example

\begin{dmath}\label{eqn:fresnelSumAndDifferenceAngleFormulas:120}
n_1 = n_2 \frac{\sin \theta_t}{\sin\theta_i}.
\end{dmath}

Inserting this and proceeding with the application of the trig identities above, we have

\begin{dmath}\label{eqn:fresnelSumAndDifferenceAngleFormulas:160}
\begin{aligned}
r^{\textrm{TE}}
&= \frac { n_2 \frac{\sin\theta_t}{\sin\theta_i} \cos\theta_i - n_2 \cos\theta_t } { n_2 \frac{\sin\theta_t}{\sin\theta_i} \cos\theta_i + n_2 \cos\theta_t } \\
&=
\frac {
\sin\theta_t \cos\theta_i - \cos\theta_t \sin\theta_i
} {
\sin\theta_t \cos\theta_i + \cos\theta_t \sin\theta_i
} \\
&=
\frac {
\sin( \theta_t - \theta_i )
} {
\sin( \theta_t + \theta_i )
}
\end{aligned}
\end{dmath}
\begin{dmath}\label{eqn:fresnelSumAndDifferenceAngleFormulas:180}
\begin{aligned}
r^{\textrm{TM}}
&= \frac{n_2 \cos\theta_i - n_2 \frac{\sin\theta_t}{\sin\theta_i} \cos\theta_t }{ n_2 \cos\theta_i + n_2 \frac{\sin\theta_t}{\sin\theta_i} \cos\theta_t } \\
&= \frac{
\sin\theta_i \cos\theta_i - \sin\theta_t \cos\theta_t
}{
\sin\theta_i \cos\theta_i + \sin\theta_t \cos\theta_t
} \\
&= \frac{\inv{2} \sin(2 \theta_i) -  \inv{2} \sin(2 \theta_t) }{ \inv{2} \sin(2 \theta_i) +  \inv{2} \sin(2 \theta_t) } \\
&= \frac
{\sin(\theta_i - \theta_t)\cos(\theta_i + \theta_t) }
{\sin(\theta_i + \theta_t)\cos(\theta_i - \theta_t) } \\
&=
\frac
{\tan(\theta_i -\theta_t)}
{\tan(\theta_i +\theta_t)}
\end{aligned}
\end{dmath}
\begin{dmath}\label{eqn:fresnelSumAndDifferenceAngleFormulas:200}
\begin{aligned}
t^{\textrm{TE}}
&= \frac{ 2 n_2 \frac{\sin\theta_t}{\sin\theta_i} \cos\theta_i } { n_2 \frac{\sin\theta_t}{\sin\theta_i} \cos\theta_i + n_2 \cos\theta_t } \\
&= \frac{ 2  \sin\theta_t \cos\theta_i } { \sin\theta_t \cos\theta_i + \cos\theta_t \sin\theta_i } \\
&= \frac{ 2  \sin\theta_t \cos\theta_i }
{ \sin(\theta_i + \theta_t) }
\end{aligned}
\end{dmath}
\begin{dmath}\label{eqn:fresnelSumAndDifferenceAngleFormulas:220}
\begin{aligned}
t^{\textrm{TM}}
&= \frac{2 n_2 \frac{\sin\theta_t}{\sin\theta_i} \cos\theta_i }{ n_2 \cos\theta_i + n_2 \frac{\sin\theta_t}{\sin\theta_i} \cos\theta_t } \\
&= \frac{2  \sin\theta_t \cos\theta_i }{ \sin\theta_i \cos\theta_i +  \sin\theta_t \cos\theta_t } \\
&= \frac{2  \sin\theta_t \cos\theta_i }
{ \inv{2} \sin(2 \theta_i) +  \inv{2} \sin(2 \theta_t) } \\
&= \frac{2 \sin\theta_t \cos\theta_i }
{ \sin(\theta_i + \theta_t) \cos(\theta_i - \theta_t) }
\end{aligned}
\end{dmath}

%}
\EndArticle
