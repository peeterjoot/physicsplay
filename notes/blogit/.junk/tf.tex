%
% Copyright � 2016 Peeter Joot.  All Rights Reserved.
% Licenced as described in the file LICENSE under the root directory of this GIT repository.
%
%{
\newcommand{\authorname}{Peeter Joot}
\newcommand{\email}{peeterjoot@protonmail.com}
\newcommand{\basename}{FIXMEbasenameUndefined}
\newcommand{\dirname}{notes/FIXMEdirnameUndefined/}

\renewcommand{\basename}{fieldsAtInterface}
\renewcommand{\dirname}{notes/ece1228-electromagnetic-theory/}
%\newcommand{\dateintitle}{}
%\newcommand{\keywords}{}

\newcommand{\authorname}{Peeter Joot}
\newcommand{\onlineurl}{http://sites.google.com/site/peeterjoot2/math2013/\basename.pdf}
\newcommand{\sourcepath}{\dirname\basename.tex}
\newcommand{\generatetitle}[1]{\chapter{#1}}

\newcommand{\vcsinfo}{%
\section*{}
\noindent{\color{DarkOliveGreen}{\rule{\linewidth}{0.1mm}}}
\paragraph{Document version}
%\paragraph{\color{Maroon}{Document version}}
{
\small
\begin{itemize}
\item Available online at:\\ 
\href{\onlineurl}{\onlineurl}
\item Git Repository: \input{./.revinfo/gitRepo.tex}
\item Source: \sourcepath
\item last commit: \input{./.revinfo/gitCommitString.tex}
\item commit date: \input{./.revinfo/gitCommitDate.tex}
\end{itemize}
}
}

%\PassOptionsToPackage{dvipsnames,svgnames}{xcolor}
\PassOptionsToPackage{square,numbers}{natbib}
\documentclass{scrreprt}

\usepackage[left=2cm,right=2cm]{geometry}
\usepackage[svgnames]{xcolor}
\usepackage{peeters_layout}

\usepackage{natbib}

\usepackage[
colorlinks=true,
bookmarks=false,
pdfauthor={\authorname, \email},
backref 
]{hyperref}

% http://tex.stackexchange.com/questions/75773/how-to-reference-problems-by-the-text-label-in-an-exercise-envioronment
\usepackage[english]{cleveref}
\crefname{Exercise}{exercise}{exercises}
\Crefname{Exercise}{Exercise}{Exercises}

\RequirePackage{titlesec}
\RequirePackage{ifthen}

% http://stackoverflow.com/questions/4932910/date-in-the-tabular-environment
\makeatletter
\let\insertdate\@date
\makeatother

\titleformat{\chapter}[display]
{\bfseries\Large}
{\color{DarkSlateGrey}\filleft \authorname
\ifthenelse{\isundefined{\studentnumber}}{}{\\ \studentnumber}
\ifthenelse{\isundefined{\email}}{}{\\ \email}
\ifthenelse{\isundefined{\dateintitle}}{}{\\ \insertdate}
%\ifthenelse{\isundefined{\coursename}}{}{\\ \coursename} % put in title instead.
}
{4ex}
{\color{DarkOliveGreen}{\titlerule}\color{Maroon}
\vspace{2ex}%
\filright}
[\vspace{2ex}%
\color{DarkOliveGreen}\titlerule
]

\newcommand{\beginArtWithToc}[0]{\begin{document}\tableofcontents}
\newcommand{\beginArtNoToc}[0]{\begin{document}}
\newcommand{\EndNoBibArticle}[0]{\end{document}}
\newcommand{\EndArticle}[0]{\bibliography{Bibliography}\bibliographystyle{plainnat}\end{document}}

% 
%\newcommand{\citep}[1]{\cite{#1}}

\colorSectionsForArticle



\usepackage{peeters_layout_exercise}
\usepackage{peeters_braket}
\usepackage{peeters_figures}
\usepackage{siunitx}

\beginArtNoToc

\generatetitle{Electric and magnetic fields at an interface}
%\chapter{Electric and magnetic fields at an interface}
%\label{chap:fieldsAtInterface}

As pointed out in \citep{balanis1989advanced} the fields at an interface that is not a perfect conductor on either side are related by

\begin{equation}\label{eqn:fieldsAtInterface:20}
\begin{aligned}
\ncap \cdot \lr{ \BD_2 - \BD_1 } &= \rho_{es} \\
\ncap \cross \lr{ \BE_2 - \BE_1 } &= -\BM_s \\
\ncap \cdot \lr{ \BB_2 - \BB_1 } &= \rho_{ms} \\
\ncap \cross \lr{ \BH_2 - \BH_1 } &= \BJ_s.
\end{aligned}
\end{equation}

Given the fields in medium 1, assuming that boths sets of media are linear, we can use these relationships to determine the fields in the other medium.

\begin{equation}\label{eqn:fieldsAtInterface:40}
\begin{aligned}
\ncap \cdot \BE_2 &= \inv{\epsilon_2} \lr{ \epsilon_1 \ncap \cdot \BE_1 + \rho_{es} } \\
\ncap \wedge \BE_2 &= \ncap \wedge \BE_1 -I \BM_s \\
\ncap \cdot \BB_2 &= \ncap \cdot \BB_1 + \rho_{ms} \\
\ncap \wedge \BB_2 &= \mu_2 \lr{ \inv{\mu_1} \ncap \wedge \BB_1 + I \BJ_s}.
\end{aligned}
\end{equation}

Now the fields in interface 2 can be obtained by adding the normal and tangential projections.  For the electric field

\begin{dmath}\label{eqn:fieldsAtInterface:60}
\begin{aligned}
\BE_2
&=
\ncap (\ncap \cdot \BE_2 )
+ \ncap \cdot (\ncap \wedge \BE_2) \\
&=
\inv{\epsilon_2} \ncap \lr{ \epsilon_1 \ncap \cdot \BE_1 + \rho_{es} }
+
\ncap \cdot (\ncap \wedge \BE_1 -I \BM_s).
\end{aligned}
\end{dmath}

Note that this manipulation can also be done without Geometric Algebra by writing \( \BE_2 = \ncap (\ncap \cdot \BE_2 ) - \ncap \cross (\ncap \cross \BE_2) \)).
Expanding \( \ncap \cdot (\ncap \wedge \BE_1) = \BE_1 - \ncap (\ncap \cdot \BE_1) \), and \( \ncap \cdot (I \BM_s) = -\ncap \cross \BM_s \), that is

\begin{dmath}\label{eqn:fieldsAtInterface:80}
\boxed{
\BE_2
=
\BE_1
+ \ncap (\ncap \cdot \BE_1) \lr{ \frac{\epsilon_1}{\epsilon_2} - 1 }
+ \frac{\rho_{es}}{\epsilon_2}
+ \ncap \cross \BM_s.
}
\end{dmath}

For the magnetic field

\begin{dmath}\label{eqn:fieldsAtInterface:100}
\begin{aligned}
\BB_2
&=
\ncap (\ncap \cdot \BB_2 )
+
\ncap \cdot (\ncap \wedge \BB_2) \\
&=
\ncap \lr{ \ncap \cdot \BB_1 + \rho_{ms} }
+
\mu_2 \ncap \cdot \lr{ \lr{ \inv{\mu_1} \ncap \wedge \BB_1 + I \BJ_s} },
\end{aligned}
\end{dmath}

which is

\begin{dmath}\label{eqn:fieldsAtInterface:120}
\boxed{
\BB_2
=
\frac{\mu_2}{\mu_1} \BB_1
+
\ncap (\ncap \cdot \BB_1) \lr{ 1 - \frac{\mu_2}{\mu_1} }
+ \ncap \rho_{ms}
- \ncap \cross \BJ_s.
}
\end{dmath}

These are kind of pretty results, having none of the explicit angle dependence that we see in the Fresnel relationships.  In this analysis, it is assumed there is only a transmitted component of the ray in question, and no reflected component.  Can we do a purely vectoral treatment of the Fresnel equations along these same lines?

%}
\EndArticle
%\EndNoBibArticle
