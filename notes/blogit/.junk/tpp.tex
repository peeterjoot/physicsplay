%
% Copyright � 2016 Peeter Joot.  All Rights Reserved.
% Licenced as described in the file LICENSE under the root directory of this GIT repository.
%
%{
\newcommand{\authorname}{Peeter Joot}
\newcommand{\email}{peeterjoot@protonmail.com}
\newcommand{\basename}{FIXMEbasenameUndefined}
\newcommand{\dirname}{notes/FIXMEdirnameUndefined/}

\renewcommand{\basename}{poynting}
%\renewcommand{\dirname}{notes/phy1520/}
\renewcommand{\dirname}{notes/ece1228-electromagnetic-theory/}
%\newcommand{\dateintitle}{}
%\newcommand{\keywords}{}

\newcommand{\authorname}{Peeter Joot}
\newcommand{\onlineurl}{http://sites.google.com/site/peeterjoot2/math2013/\basename.pdf}
\newcommand{\sourcepath}{\dirname\basename.tex}
\newcommand{\generatetitle}[1]{\chapter{#1}}

\newcommand{\vcsinfo}{%
\section*{}
\noindent{\color{DarkOliveGreen}{\rule{\linewidth}{0.1mm}}}
\paragraph{Document version}
%\paragraph{\color{Maroon}{Document version}}
{
\small
\begin{itemize}
\item Available online at:\\ 
\href{\onlineurl}{\onlineurl}
\item Git Repository: \input{./.revinfo/gitRepo.tex}
\item Source: \sourcepath
\item last commit: \input{./.revinfo/gitCommitString.tex}
\item commit date: \input{./.revinfo/gitCommitDate.tex}
\end{itemize}
}
}

%\PassOptionsToPackage{dvipsnames,svgnames}{xcolor}
\PassOptionsToPackage{square,numbers}{natbib}
\documentclass{scrreprt}

\usepackage[left=2cm,right=2cm]{geometry}
\usepackage[svgnames]{xcolor}
\usepackage{peeters_layout}

\usepackage{natbib}

\usepackage[
colorlinks=true,
bookmarks=false,
pdfauthor={\authorname, \email},
backref 
]{hyperref}

% http://tex.stackexchange.com/questions/75773/how-to-reference-problems-by-the-text-label-in-an-exercise-envioronment
\usepackage[english]{cleveref}
\crefname{Exercise}{exercise}{exercises}
\Crefname{Exercise}{Exercise}{Exercises}

\RequirePackage{titlesec}
\RequirePackage{ifthen}

% http://stackoverflow.com/questions/4932910/date-in-the-tabular-environment
\makeatletter
\let\insertdate\@date
\makeatother

\titleformat{\chapter}[display]
{\bfseries\Large}
{\color{DarkSlateGrey}\filleft \authorname
\ifthenelse{\isundefined{\studentnumber}}{}{\\ \studentnumber}
\ifthenelse{\isundefined{\email}}{}{\\ \email}
\ifthenelse{\isundefined{\dateintitle}}{}{\\ \insertdate}
%\ifthenelse{\isundefined{\coursename}}{}{\\ \coursename} % put in title instead.
}
{4ex}
{\color{DarkOliveGreen}{\titlerule}\color{Maroon}
\vspace{2ex}%
\filright}
[\vspace{2ex}%
\color{DarkOliveGreen}\titlerule
]

\newcommand{\beginArtWithToc}[0]{\begin{document}\tableofcontents}
\newcommand{\beginArtNoToc}[0]{\begin{document}}
\newcommand{\EndNoBibArticle}[0]{\end{document}}
\newcommand{\EndArticle}[0]{\bibliography{Bibliography}\bibliographystyle{plainnat}\end{document}}

% 
%\newcommand{\citep}[1]{\cite{#1}}

\colorSectionsForArticle



\usepackage{peeters_layout_exercise}
\usepackage{peeters_braket}
\usepackage{peeters_figures}
\usepackage{siunitx}
%\usepackage{txfonts} % \ointclockwise

\beginArtNoToc

\generatetitle{Poynting relationship}
%\chapter{Poynting relationship}
%\label{chap:poynting}

\paragraph{Problem:}

Given
\begin{equation}\label{eqn:poynting:20}
\spacegrad \cross \BE
= -\BM_i - \PD{t}{\BB},
\end{equation}

and
\begin{equation}\label{eqn:poynting:40}
\spacegrad \cross \BH
= \BJ_i + \BJ_c + \PD{t}{\BD},
\end{equation}

expand the divergence of \( \BE \cross \BH \) to find the form of the Poynting theorem.

\paragraph{Solution:}

First we need the chain rule for of this sort of divergence.  Using primes to indicate the scope of the gradient operation

\begin{dmath}\label{eqn:poynting:60}
\begin{aligned}
\spacegrad \cdot \lr{ \BE \cross \BH }
&=
\spacegrad' \cdot \lr{ \BE' \cross \BH }
-
\spacegrad' \cdot \lr{ \BH' \cross \BE }  \\
&=
\BH \cdot \lr{ \spacegrad' \cross \BE' }
-
\BH \cdot \lr{ \spacegrad' \cross \BH' }  \\
&=
\BH \cdot \lr{ \spacegrad \cross \BE }
-
\BE \cdot \lr{ \spacegrad \cross \BH }.
\end{aligned}
\end{dmath}

In the second step, cyclic permutation of the triple product was used.
This checks against the inside front cover of Jackson \citep{jackson1975cew}.  Now we can plug in the Maxwell equation cross products.

\begin{dmath}\label{eqn:poynting:80}
\begin{aligned}
\spacegrad \cdot \lr{ \BE \cross \BH }
&=
\BH \cdot \lr{ -\BM_i - \PD{t}{\BB} }
-
\BE \cdot \lr{ \BJ_i + \BJ_c + \PD{t}{\BD} } \\
&=
-\BH \cdot \BM_i
-\mu \BH \cdot \PD{t}{\BH}
-
\BE \cdot \BJ_i
-
\BE \cdot \BJ_c
-
\epsilon \BE \cdot \PD{t}{\BE},
\end{aligned}
\end{dmath}

or

\begin{dmath}\label{eqn:poynting:120}
\boxed{
0
=
\spacegrad \cdot \lr{ \BE \cross \BH }
+ \frac{\epsilon}{2} \PD{t}{} \Abs{ \BE }^2
+ \frac{\mu}{2} \PD{t}{} \Abs{ \BH }^2
+ \BH \cdot \BM_i
+ \BE \cdot \BJ_i
+ \sigma \Abs{\BE}^2.
}
\end{dmath}

%}
\EndArticle
%\EndNoBibArticle
