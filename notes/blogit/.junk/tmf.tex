%
% Copyright � 2016 Peeter Joot.  All Rights Reserved.
% Licenced as described in the file LICENSE under the root directory of this GIT repository.
%
%{
\newcommand{\authorname}{Peeter Joot}
\newcommand{\email}{peeterjoot@protonmail.com}
\newcommand{\basename}{FIXMEbasenameUndefined}
\newcommand{\dirname}{notes/FIXMEdirnameUndefined/}

\renewcommand{\basename}{magneticFieldFromMoment}
%\renewcommand{\dirname}{notes/phy1520/}
\renewcommand{\dirname}{notes/ece1228-electromagnetic-theory/}
%\newcommand{\dateintitle}{}
%\newcommand{\keywords}{}

\newcommand{\authorname}{Peeter Joot}
\newcommand{\onlineurl}{http://sites.google.com/site/peeterjoot2/math2013/\basename.pdf}
\newcommand{\sourcepath}{\dirname\basename.tex}
\newcommand{\generatetitle}[1]{\chapter{#1}}

\newcommand{\vcsinfo}{%
\section*{}
\noindent{\color{DarkOliveGreen}{\rule{\linewidth}{0.1mm}}}
\paragraph{Document version}
%\paragraph{\color{Maroon}{Document version}}
{
\small
\begin{itemize}
\item Available online at:\\ 
\href{\onlineurl}{\onlineurl}
\item Git Repository: \input{./.revinfo/gitRepo.tex}
\item Source: \sourcepath
\item last commit: \input{./.revinfo/gitCommitString.tex}
\item commit date: \input{./.revinfo/gitCommitDate.tex}
\end{itemize}
}
}

%\PassOptionsToPackage{dvipsnames,svgnames}{xcolor}
\PassOptionsToPackage{square,numbers}{natbib}
\documentclass{scrreprt}

\usepackage[left=2cm,right=2cm]{geometry}
\usepackage[svgnames]{xcolor}
\usepackage{peeters_layout}

\usepackage{natbib}

\usepackage[
colorlinks=true,
bookmarks=false,
pdfauthor={\authorname, \email},
backref 
]{hyperref}

% http://tex.stackexchange.com/questions/75773/how-to-reference-problems-by-the-text-label-in-an-exercise-envioronment
\usepackage[english]{cleveref}
\crefname{Exercise}{exercise}{exercises}
\Crefname{Exercise}{Exercise}{Exercises}

\RequirePackage{titlesec}
\RequirePackage{ifthen}

% http://stackoverflow.com/questions/4932910/date-in-the-tabular-environment
\makeatletter
\let\insertdate\@date
\makeatother

\titleformat{\chapter}[display]
{\bfseries\Large}
{\color{DarkSlateGrey}\filleft \authorname
\ifthenelse{\isundefined{\studentnumber}}{}{\\ \studentnumber}
\ifthenelse{\isundefined{\email}}{}{\\ \email}
\ifthenelse{\isundefined{\dateintitle}}{}{\\ \insertdate}
%\ifthenelse{\isundefined{\coursename}}{}{\\ \coursename} % put in title instead.
}
{4ex}
{\color{DarkOliveGreen}{\titlerule}\color{Maroon}
\vspace{2ex}%
\filright}
[\vspace{2ex}%
\color{DarkOliveGreen}\titlerule
]

\newcommand{\beginArtWithToc}[0]{\begin{document}\tableofcontents}
\newcommand{\beginArtNoToc}[0]{\begin{document}}
\newcommand{\EndNoBibArticle}[0]{\end{document}}
\newcommand{\EndArticle}[0]{\bibliography{Bibliography}\bibliographystyle{plainnat}\end{document}}

% 
%\newcommand{\citep}[1]{\cite{#1}}

\colorSectionsForArticle



\usepackage{peeters_layout_exercise}
\usepackage{peeters_braket}
\usepackage{peeters_figures}
\usepackage{siunitx}
%\usepackage{mhchem} % \ce{}
%\usepackage{macros_bm} % \bcM
%\usepackage{txfonts} % \ointclockwise

\beginArtNoToc

\generatetitle{Calculating the magnetostatic field from the moment}
%\chapter{Calculating the magnetostatic field from the moment}
%\label{chap:magneticFieldFromMoment}
% \citep{jackson1975cew}

The vector potential, to first order, for a magnetostatic localized current distribution was found to be

\begin{dmath}\label{eqn:magneticFieldFromMoment:20}
\BA(\Bx) = \frac{\mu_0}{4 \pi} \frac{\Bm \cross \Bx}{\Abs{\Bx}^3}.
\end{dmath}

Initially, I tried to calculate the magnetic field from this, but ran into trouble.  Here's a new try.

\begin{dmath}\label{eqn:magneticFieldFromMoment:40}
\begin{aligned}
\BB
&=
\frac{\mu_0}{4 \pi}
\spacegrad \cross \lr{ \Bm \cross \frac{\Bx}{r^3} } \\
&=
-\frac{\mu_0}{4 \pi}
\spacegrad \cdot \lr{ \Bm \wedge \frac{\Bx}{r^3} } \\
&=
-\frac{\mu_0}{4 \pi}
\lr{
(\Bm \cdot \spacegrad) \frac{\Bx}{r^3}
-\Bm \spacegrad \cdot \frac{\Bx}{r^3}
} \\
&=
\frac{\mu_0}{4 \pi}
\lr{
-\frac{(\Bm \cdot \spacegrad) \Bx}{r^3}
- \lr{ \Bm \cdot \lr{\spacegrad \inv{r^3} }} \Bx
+\Bm (\spacegrad \cdot \Bx) \inv{r^3}
+\Bm \lr{\spacegrad \inv{r^3} } \cdot \Bx
}.
\end{aligned}
\end{dmath}

Here I've used \( \Ba \cross \lr{ \Bb \cross \Bc } = -\Ba \cdot \lr{ \Bb \wedge \Bc } \), and then expanded that with \( \Ba \cdot \lr{ \Bb \wedge \Bc } = (\Ba \cdot \Bb) \Bc - (\Ba \cdot \Bc) \Bb \).  Since one of these vectors is the gradient, care must be taken to have it operate on the appropriate terms in such an expansion.

Since we have \( \spacegrad \cdot \Bx = 3 \), \( (\Bm \cdot \spacegrad) \Bx = \Bm \), and \( \spacegrad 1/r^n = -n \Bx/r^{n+2} \), this reduces to

\begin{dmath}\label{eqn:magneticFieldFromMoment:60}
\begin{aligned}
\BB
&=
\frac{\mu_0}{4 \pi}
\lr{
- \frac{\Bm}{r^3}
+ 3 \frac{(\Bm \cdot \Bx) \Bx}{r^5} %
+ 3 \Bm \inv{r^3}
-3 \Bm \frac{\Bx}{r^5} \cdot \Bx
} \\
&=
\frac{\mu_0}{4 \pi}
\frac{3 (\Bm \cdot \ncap) \ncap -\Bm}{r^3},
\end{aligned}
\end{dmath}

which is the desired result.

%}
%\EndArticle
\EndNoBibArticle
