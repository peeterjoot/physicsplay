%
% Copyright � 2012 Peeter Joot.  All Rights Reserved.
% Licenced as described in the file LICENSE under the root directory of this GIT repository.
%
% pick one:
%\newcommand{\authorname}{Peeter Joot}
\newcommand{\email}{peeter.joot@utoronto.ca}
\newcommand{\studentnumber}{920798560}
\newcommand{\basename}{FIXMEbasenameUndefined}
\newcommand{\dirname}{notes/FIXMEdirnameUndefined/}

\newcommand{\authorname}{Peeter Joot}
\newcommand{\email}{peeterjoot@protonmail.com}
\newcommand{\basename}{FIXMEbasenameUndefined}
\newcommand{\dirname}{notes/FIXMEdirnameUndefined/}

\renewcommand{\basename}{fresnelAlternatePolarization}
\renewcommand{\dirname}{notes/phy485/}
\newcommand{\keywords}{Fresnel equations}

\newcommand{\authorname}{Peeter Joot}
\newcommand{\onlineurl}{http://sites.google.com/site/peeterjoot2/math2013/\basename.pdf}
\newcommand{\sourcepath}{\dirname\basename.tex}
\newcommand{\generatetitle}[1]{\chapter{#1}}

\newcommand{\vcsinfo}{%
\section*{}
\noindent{\color{DarkOliveGreen}{\rule{\linewidth}{0.1mm}}}
\paragraph{Document version}
%\paragraph{\color{Maroon}{Document version}}
{
\small
\begin{itemize}
\item Available online at:\\ 
\href{\onlineurl}{\onlineurl}
\item Git Repository: \input{./.revinfo/gitRepo.tex}
\item Source: \sourcepath
\item last commit: \input{./.revinfo/gitCommitString.tex}
\item commit date: \input{./.revinfo/gitCommitDate.tex}
\end{itemize}
}
}

%\PassOptionsToPackage{dvipsnames,svgnames}{xcolor}
\PassOptionsToPackage{square,numbers}{natbib}
\documentclass{scrreprt}

\usepackage[left=2cm,right=2cm]{geometry}
\usepackage[svgnames]{xcolor}
\usepackage{peeters_layout}

\usepackage{natbib}

\usepackage[
colorlinks=true,
bookmarks=false,
pdfauthor={\authorname, \email},
backref 
]{hyperref}

% http://tex.stackexchange.com/questions/75773/how-to-reference-problems-by-the-text-label-in-an-exercise-envioronment
\usepackage[english]{cleveref}
\crefname{Exercise}{exercise}{exercises}
\Crefname{Exercise}{Exercise}{Exercises}

\RequirePackage{titlesec}
\RequirePackage{ifthen}

% http://stackoverflow.com/questions/4932910/date-in-the-tabular-environment
\makeatletter
\let\insertdate\@date
\makeatother

\titleformat{\chapter}[display]
{\bfseries\Large}
{\color{DarkSlateGrey}\filleft \authorname
\ifthenelse{\isundefined{\studentnumber}}{}{\\ \studentnumber}
\ifthenelse{\isundefined{\email}}{}{\\ \email}
\ifthenelse{\isundefined{\dateintitle}}{}{\\ \insertdate}
%\ifthenelse{\isundefined{\coursename}}{}{\\ \coursename} % put in title instead.
}
{4ex}
{\color{DarkOliveGreen}{\titlerule}\color{Maroon}
\vspace{2ex}%
\filright}
[\vspace{2ex}%
\color{DarkOliveGreen}\titlerule
]

\newcommand{\beginArtWithToc}[0]{\begin{document}\tableofcontents}
\newcommand{\beginArtNoToc}[0]{\begin{document}}
\newcommand{\EndNoBibArticle}[0]{\end{document}}
\newcommand{\EndArticle}[0]{\bibliography{Bibliography}\bibliographystyle{plainnat}\end{document}}

% 
%\newcommand{\citep}[1]{\cite{#1}}

\colorSectionsForArticle



\beginArtNoToc

\generatetitle{Derivation of Fresnel equations for any polarity}
\label{chap:\basename}
\section{Motivation}

In \cite{hecht1998hecht} we have a derivation of the Fresnel equations for the TE and TM polarization modes.  Can we do this for an arbitrary polarization angles?

\section{Setup}

The task at hand is to find evaluate the boundary value constraints.  Following the interface plane conventions of \cite{griffith1981introduction}, and his notation that is

\begin{subequations}
\begin{equation}\label{eqn:fresnelAlternatePolarization:10}
\epsilon_1 ( \BE_i + \BE_r )_z = \epsilon_2 ( \BE_t )_z
\end{equation}
\begin{equation}\label{eqn:fresnelAlternatePolarization:30}
( \BB_i + \BB_r )_z = ( \BB_t )_z
\end{equation}
\begin{equation}\label{eqn:fresnelAlternatePolarization:50}
( \BE_i + \BE_r )_{x,y} = ( \BE_t )_{x,y}
\end{equation}
\begin{equation}\label{eqn:fresnelAlternatePolarization:70}
\inv{\mu_1} ( \BB_i + \BB_r )_{x,y} = \inv{\mu_2} ( \BB_t )_{x,y}
\end{equation}
\end{subequations}

I'll work here with a phasor representation directly and not bother with taking real parts, or using tilde notation to mark the vectors as complex.

Our complex magnetic field phasors are related to the electric fields with

\begin{equation}\label{eqn:fresnelAlternatePolarization:90}
\BB = \inv{v} \kcap \cross \BE.
\end{equation}

Referring to figure (\ref{fig:fresnelAlternatePolarization:reflectionAtInterfaceUnlabelledFig1}) shows the geometrical task to tackle, since we've got to express all the various unit vectors algebraically.  I'll use Geometric Algebra here to do that for its compact expression of rotations.  With

\pdfTexFigure{reflectionAtInterfaceUnlabelledFig1.pdf_tex}{Reflection and transmission of light at an interface}{fig:fresnelAlternatePolarization:reflectionAtInterfaceUnlabelledFig1}{0.8}

\begin{equation}\label{eqn:fresnelAlternatePolarization:110}
j = \Be_3 \Be_1,
\end{equation}

we can express each of the $k$ vector directions by inspection.  Those are 

\begin{subequations}
\begin{equation}\label{eqn:fresnelAlternatePolarization:130}
\kcap_i = \Be_3 e^{j \theta_i}
\end{equation}
\begin{equation}\label{eqn:fresnelAlternatePolarization:150}
\kcap_r = \Be_1 e^{j (\pi/2 - \theta_r)} = -\Be_3 e^{-j \theta_r}
\end{equation}
\begin{equation}\label{eqn:fresnelAlternatePolarization:170}
\kcap_t = \Be_3 e^{j \theta_t}.
\end{equation}
\end{subequations}

Similarily, our vectors perpendicular to these and $\Be_2$ are respectively
\begin{subequations}
\begin{equation}\label{eqn:fresnelAlternatePolarization:190}
\mcap_i = -\Be_3 e^{-j (\pi/2 - \theta_i)} = \Be_1 e^{j \theta_i}
\end{equation}
\begin{equation}\label{eqn:fresnelAlternatePolarization:210}
\mcap_r = -\Be_3 e^{-j (\pi/2 - \theta_r)} = -\Be_1 e^{j \theta_r}
\end{equation}
\begin{equation}\label{eqn:fresnelAlternatePolarization:230}
\mcap_t = \Be_1 e^{j \theta_t}
\end{equation}
\end{subequations}

One of the Griffith's problems was to show that the polarization angles must be the same as a result of matching the boundary constraints, but this problem was restricted to showing this for one of TE or TM modes (I forget), but for radiation that was fully perpendicular to the interface.  Can we also tackle that problem for both this more general angle of incidence and a general polarization?  Let's try so, allowing temporarily for different polarizations of the reflected and transmitted components of the light, calling those polarization angles $\phi_i$, $\phi_r$, and $\phi_t$ respectively.  With $\phi_i = 0$ let's align $\BE_i$, $\BB_i$ with the $\mcap_i$, and $\Be_2$ directions respectively.  Our phasors are then

\begin{subequations}
\begin{equation}\label{eqn:fresnelAlternatePolarization:250}
\begin{bmatrix}
\BE_i \\
\BB_i \\
\end{bmatrix}
=
\begin{bmatrix}
\mcap_i \\
\Be_2 \\
\end{bmatrix}
e^{ \mcap_i \Be_2 \phi_i }
\end{equation}
\begin{equation}\label{eqn:fresnelAlternatePolarization:270}
\begin{bmatrix}
\BE_r \\
\BB_r \\
\end{bmatrix}
=
\begin{bmatrix}
\mcap_r \\
\Be_2 \\
\end{bmatrix}
e^{ \mcap_r \Be_2 \phi_r }
\end{equation}
\begin{equation}\label{eqn:fresnelAlternatePolarization:290}
\begin{bmatrix}
\BE_t \\
\BB_t \\
\end{bmatrix}
=
\begin{bmatrix}
\mcap_t \\
\Be_2 \\
\end{bmatrix}
e^{ \mcap_t \Be_2 \phi_t }
\end{equation}.
\end{subequations}

We are now set to at least express our boundary value constraints
\begin{subequations}
\begin{equation}\label{eqn:fresnelAlternatePolarization:310}
\epsilon_1 \left( \mcap_i E_{0 i} e^{ \mcap_i \Be_2 \phi_i } + \mcap_r E_{0 r} e^{ \mcap_r \Be_2 \phi_r } \right)_z = \epsilon_2 \left( \mcap_r E_{0 t} e^{ \mcap_t \Be_2 \phi_t } \right)_z
\end{equation}
\begin{equation}\label{eqn:fresnelAlternatePolarization:330}
\inv{v_1} \left( \Be_2 E_{0 i} e^{ \mcap_i \Be_2 \phi_i } + \Be_2 E_{0 r} e^{ \mcap_r \Be_2 \phi_r } \right)_z = \inv{v_2} \left( \Be_2 E_{0 t} e^{ \mcap_t \Be_2 \phi_t } \right)_z
\end{equation}
\begin{equation}\label{eqn:fresnelAlternatePolarization:350}
\left( \mcap_i E_{0 i} e^{ \mcap_i \Be_2 \phi_i } + \mcap_r E_{0 r} e^{ \mcap_r \Be_2 \phi_r } \right)_{x,y} = \left( \mcap_r E_{0 t} e^{ \mcap_t \Be_2 \phi_t } \right)_{x,y}
\end{equation}
\begin{equation}\label{eqn:fresnelAlternatePolarization:370}
\inv{\mu_1 v_1} \left( \Be_2 E_{0 i} e^{ \mcap_i \Be_2 \phi_i } + \Be_2 E_{0 r} e^{ \mcap_r \Be_2 \phi_r } \right)_{x,y} = \inv{\mu_2 v_2} \left( \Be_2 E_{0 t} e^{ \mcap_t \Be_2 \phi_t } \right)_{x,y}.
\end{equation}
\end{subequations}

%   \makeproblem{description}{ch1:pr1}{ }
%   \makeanswer{ch1:pr1}{ }
%
% not needed with print2.tex:
%   \shipoutAnswer

% this is to produce the sites.google url and version info and so forth (for blog posts)
%\vcsinfo
\EndArticle
%\EndNoBibArticle
