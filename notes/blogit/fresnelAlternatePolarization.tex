%
% Copyright � 2012 Peeter Joot.  All Rights Reserved.
% Licenced as described in the file LICENSE under the root directory of this GIT repository.
%
% pick one:
%\newcommand{\authorname}{Peeter Joot}
\newcommand{\email}{peeter.joot@utoronto.ca}
\newcommand{\studentnumber}{920798560}
\newcommand{\basename}{FIXMEbasenameUndefined}
\newcommand{\dirname}{notes/FIXMEdirnameUndefined/}

\newcommand{\authorname}{Peeter Joot}
\newcommand{\email}{peeterjoot@protonmail.com}
\newcommand{\basename}{FIXMEbasenameUndefined}
\newcommand{\dirname}{notes/FIXMEdirnameUndefined/}

\renewcommand{\basename}{fresnelAlternatePolarization}
\renewcommand{\dirname}{notes/phy485/}
\newcommand{\keywords}{Fresnel equations}

\newcommand{\authorname}{Peeter Joot}
\newcommand{\onlineurl}{http://sites.google.com/site/peeterjoot2/math2013/\basename.pdf}
\newcommand{\sourcepath}{\dirname\basename.tex}
\newcommand{\generatetitle}[1]{\chapter{#1}}

\newcommand{\vcsinfo}{%
\section*{}
\noindent{\color{DarkOliveGreen}{\rule{\linewidth}{0.1mm}}}
\paragraph{Document version}
%\paragraph{\color{Maroon}{Document version}}
{
\small
\begin{itemize}
\item Available online at:\\ 
\href{\onlineurl}{\onlineurl}
\item Git Repository: \input{./.revinfo/gitRepo.tex}
\item Source: \sourcepath
\item last commit: \input{./.revinfo/gitCommitString.tex}
\item commit date: \input{./.revinfo/gitCommitDate.tex}
\end{itemize}
}
}

%\PassOptionsToPackage{dvipsnames,svgnames}{xcolor}
\PassOptionsToPackage{square,numbers}{natbib}
\documentclass{scrreprt}

\usepackage[left=2cm,right=2cm]{geometry}
\usepackage[svgnames]{xcolor}
\usepackage{peeters_layout}

\usepackage{natbib}

\usepackage[
colorlinks=true,
bookmarks=false,
pdfauthor={\authorname, \email},
backref 
]{hyperref}

% http://tex.stackexchange.com/questions/75773/how-to-reference-problems-by-the-text-label-in-an-exercise-envioronment
\usepackage[english]{cleveref}
\crefname{Exercise}{exercise}{exercises}
\Crefname{Exercise}{Exercise}{Exercises}

\RequirePackage{titlesec}
\RequirePackage{ifthen}

% http://stackoverflow.com/questions/4932910/date-in-the-tabular-environment
\makeatletter
\let\insertdate\@date
\makeatother

\titleformat{\chapter}[display]
{\bfseries\Large}
{\color{DarkSlateGrey}\filleft \authorname
\ifthenelse{\isundefined{\studentnumber}}{}{\\ \studentnumber}
\ifthenelse{\isundefined{\email}}{}{\\ \email}
\ifthenelse{\isundefined{\dateintitle}}{}{\\ \insertdate}
%\ifthenelse{\isundefined{\coursename}}{}{\\ \coursename} % put in title instead.
}
{4ex}
{\color{DarkOliveGreen}{\titlerule}\color{Maroon}
\vspace{2ex}%
\filright}
[\vspace{2ex}%
\color{DarkOliveGreen}\titlerule
]

\newcommand{\beginArtWithToc}[0]{\begin{document}\tableofcontents}
\newcommand{\beginArtNoToc}[0]{\begin{document}}
\newcommand{\EndNoBibArticle}[0]{\end{document}}
\newcommand{\EndArticle}[0]{\bibliography{Bibliography}\bibliographystyle{plainnat}\end{document}}

% 
%\newcommand{\citep}[1]{\cite{#1}}

\colorSectionsForArticle



\beginArtNoToc

\generatetitle{Derivation of Fresnel equations for any polarity}
\label{chap:\basename}
\section{Motivation}

In \citep{hecht1998hecht} we have a derivation of the Fresnel equations for the TE and TM polarization modes.  Can we do this for an arbitrary polarization angles?

\section{Setup}

The task at hand is to find evaluate the boundary value constraints.  Following the interface plane conventions of \citep{griffith1981introduction}, and his notation that is

\begin{subequations}
\begin{equation}\label{eqn:fresnelAlternatePolarization:10}
\epsilon_1 ( \BE_i + \BE_r )_z = \epsilon_2 ( \BE_t )_z
\end{equation}
\begin{equation}\label{eqn:fresnelAlternatePolarization:30}
( \BB_i + \BB_r )_z = ( \BB_t )_z
\end{equation}
\begin{equation}\label{eqn:fresnelAlternatePolarization:50}
( \BE_i + \BE_r )_{x,y} = ( \BE_t )_{x,y}
\end{equation}
\begin{equation}\label{eqn:fresnelAlternatePolarization:70}
\inv{\mu_1} ( \BB_i + \BB_r )_{x,y} = \inv{\mu_2} ( \BB_t )_{x,y}
\end{equation}
\end{subequations}

I'll work here with a phasor representation directly and not bother with taking real parts, or using tilde notation to mark the vectors as complex.

Our complex magnetic field phasors are related to the electric fields with

\begin{equation}\label{eqn:fresnelAlternatePolarization:90}
\BB = \inv{v} \kcap \cross \BE.
\end{equation}

Referring to figure (\ref{fig:fresnelAlternatePolarization:reflectionAtInterfaceLabelledFig2}) shows the geometrical task to tackle, since we've got to express all the various unit vectors algebraically.  I'll use Geometric Algebra here to do that for its compact expression of rotations.  With

\pdfTexFigure{reflectionAtInterfaceLabelledFig2.pdf_tex}{Reflection and transmission of light at an interface}{fig:fresnelAlternatePolarization:reflectionAtInterfaceLabelledFig2}{0.8}

\begin{equation}\label{eqn:fresnelAlternatePolarization:110}
j = \Be_3 \Be_1,
\end{equation}

we can express each of the $k$ vector directions by inspection.  Those are 

\begin{subequations}
\begin{equation}\label{eqn:fresnelAlternatePolarization:130}
\kcap_i = \Be_3 e^{j \theta_i} = \Be_3 \cos\theta_i + \Be_1 \sin\theta_i
\end{equation}
\begin{equation}\label{eqn:fresnelAlternatePolarization:150}
\kcap_r = -\Be_3 e^{-j \theta_r} = -\Be_3 \cos\theta_r +\Be_1 \sin\theta_r
\end{equation}
\begin{equation}\label{eqn:fresnelAlternatePolarization:170}
\kcap_t = \Be_3 e^{j \theta_t} = \Be_3 \cos\theta_t + \Be_1 \sin\theta_t.
\end{equation}
\end{subequations}

Similarly, the perpendiculars $\mcap_\alpha = \Be_2 \cross \kcap_\alpha$ for each of $\alpha \in (i, r, t)$ are
\begin{subequations}
\begin{equation}\label{eqn:fresnelAlternatePolarization:190}
\mcap_i 
%= -\Be_{1 2 3 2 3} e^{j \theta_i}
= \Be_{1} e^{j \theta_i}
= \Be_1 \cos\theta_i - \Be_3 \sin\theta_i
= \Be_3 j e^{j \theta_i}
\end{equation}
\begin{equation}\label{eqn:fresnelAlternatePolarization:210}
\mcap_r 
%= -\Be_{1 2 3 2 3} -e^{-j \theta_r}
= -\Be_{1} e^{-j \theta_r}
= -\Be_1 \cos\theta_r - \Be_3 \sin\theta_r
= -\Be_3 j e^{-j \theta_r}
\end{equation}
\begin{equation}\label{eqn:fresnelAlternatePolarization:230}
\mcap_t 
%= -\Be_{1 2 3 2 3} e^{j \theta_t}
= \Be_{1 } e^{j \theta_t} = \Be_1 \cos\theta_t - \Be_3 \sin\theta_t
= \Be_3 j e^{j \theta_t}
\end{equation}
\end{subequations}

One of the Griffiths' problems was to show that the polarization angles must be the same as a result of matching the boundary constraints, but this problem was restricted to showing this for one of TE or TM modes (FIXME: I forget), but for radiation that was fully perpendicular to the interface.  Can we also tackle that problem for both this more general angle of incidence and a general polarization?  Let's try so, allowing temporarily for different polarizations of the reflected and transmitted components of the light, calling those polarization angles $\phi_i$, $\phi_r$, and $\phi_t$ respectively.  Let's set the $\phi_i = 0$ polarization aligned such that $\BE_i$, $\BB_i$ are aligned with the $\Be_2$ and $-\mcap_i$ directions respectively, so that the generally polarized phasors for $\alpha \in \{i, r, t\}$ are

\begin{equation}\label{eqn:fresnelAlternatePolarization:250}
\begin{bmatrix}
\BE_\alpha \\
\BB_\alpha \\
\end{bmatrix}
=
\begin{bmatrix}
\Be_2 \\
-\mcap_\alpha \\
\end{bmatrix}
e^{ \mcap_\alpha \Be_2 \phi_\alpha }
\end{equation}

We are now set to at least express our boundary value constraints
\begin{subequations}
\label{eqn:fresnelAlternatePolarization:300}
\begin{equation}\label{eqn:fresnelAlternatePolarization:310}
\epsilon_1 \left( \Be_2 E_{0 i} e^{ \mcap_i \Be_2 \phi_i } + \Be_2 E_{0 r} e^{ \mcap_r \Be_2 \phi_r } \right) \cdot \Be_3 = \epsilon_2 \left( \Be_2 E_{0 t} e^{ \mcap_t \Be_2 \phi_t } \right) \cdot \Be_3
\end{equation}
\begin{equation}\label{eqn:fresnelAlternatePolarization:330}
\inv{v_1} \left( -\mcap_i E_{0 i} e^{ \mcap_i \Be_2 \phi_i } - \mcap_r E_{0 r} e^{ \mcap_r \Be_2 \phi_r } \right) \cdot \Be_3 = \inv{v_2} \left( -\mcap_t E_{0 t} e^{ \mcap_t \Be_2 \phi_t } \right) \cdot \Be_3
\end{equation}
\begin{equation}\label{eqn:fresnelAlternatePolarization:350}
\left( \Be_2 E_{0 i} e^{ \mcap_i \Be_2 \phi_i } + \Be_2 E_{0 r} e^{ \mcap_r \Be_2 \phi_r } \right) \wedge \Be_3 = \left( \Be_2 E_{0 t} e^{ \mcap_t \Be_2 \phi_t } \right) \wedge \Be_3
\end{equation}
\begin{equation}\label{eqn:fresnelAlternatePolarization:370}
\inv{\mu_1 v_1} \left( -\mcap_i E_{0 i} e^{ \mcap_i \Be_2 \phi_i } - \mcap_r E_{0 r} e^{ \mcap_r \Be_2 \phi_r } \right) \wedge \Be_3 = \inv{\mu_2 v_2} \left( -\mcap_t E_{0 t} e^{ \mcap_t \Be_2 \phi_t } \right) \wedge \Be_3
\end{equation}
\end{subequations}

Let's try this in a couple of steps.  First with polarization angles set so that one of the fields lies in the plane of the interface (with both variations), and then attempt the general case.

\makeproblem{Sanity check.  Verify for $\BE$ parallel to the interface.}{fresnelAlternatePolarization:pr1}{ }
\makeproblem{Sanity check.  Verify for $\BB$ parallel to the interface.}{fresnelAlternatePolarization:pr2}{ }
\makeproblem{General case.  Arbitrary polarization angle.}{fresnelAlternatePolarization:pr3}{ Determine the set of simulaneous equations that would have to be solved for if the incident polarization angle was allowed to be neither TE nor TM mode.}
\makeproblem{Using superposition determine the Fresnel equations for an arbitrary incident polarization angle.}{fresnelAlternatePolarization:pr4}{ }

\makeanswer{fresnelAlternatePolarization:pr1}{
For the $\BE_\alpha \parallel \Be_2$ polarization ($\phi_i = \phi_r = \phi_t$), and $\alpha \in \{i, r, t\}$ our phasors are

\begin{subequations}
\begin{dmath}\label{eqn:fresnelAlternatePolarization:390}
\BE_\alpha = \Be_2 E_{0\alpha}
\end{dmath}
\begin{dmath}\label{eqn:fresnelAlternatePolarization:410}
\BB_\alpha = -\inv{v_\alpha} \mcap_\alpha E_{0\alpha}
\end{dmath}
\end{subequations}

Our boundary value constraints then become
\begin{subequations}
\label{eqn:fresnelAlternatePolarization:300a}
\begin{equation}\label{eqn:fresnelAlternatePolarization:310a}
\epsilon_1 \left( \Be_2 E_{0 i}  + \Be_2 E_{0 r}  \right) \cdot \Be_3 = \epsilon_2 \left( \Be_2 E_{0 t}  \right) \cdot \Be_3
\end{equation}
\begin{equation}\label{eqn:fresnelAlternatePolarization:330a}
\inv{v_1} \left( \mcap_i E_{0 i} + \mcap_r E_{0 r}  \right) \cdot \Be_3 = \inv{v_2} \left( \mcap_t E_{0 t}  \right) \cdot \Be_3
\end{equation}
\begin{equation}\label{eqn:fresnelAlternatePolarization:350a}
\left( \Be_2 E_{0 i}  + \Be_2 E_{0 r}  \right) \wedge \Be_3 = \left( \Be_2 E_{0 t}  \right) \wedge \Be_3
\end{equation}
\begin{equation}\label{eqn:fresnelAlternatePolarization:370a}
\inv{\mu_1 v_1} \left( \mcap_i E_{0 i}  + \mcap_r E_{0 r}  \right) \wedge \Be_3 = \inv{\mu_2 v_2} \left( \mcap_t E_{0 t}  \right) \wedge \Be_3.
\end{equation}
\end{subequations}

With $\mcap_\alpha$ substitution this is
\begin{subequations}
\label{eqn:fresnelAlternatePolarization:300b}
\begin{equation}\label{eqn:fresnelAlternatePolarization:310b}
\epsilon_1 \gpgradezero{ \Be_3 \left( \Be_2 E_{0 i}  + \Be_2 E_{0 r}  \right) } = \epsilon_2 \gpgradezero{ \Be_3 \left( \Be_2 E_{0 t}  \right) }
\end{equation}
\begin{equation}\label{eqn:fresnelAlternatePolarization:330b}
\inv{v_1} \gpgradezero{ \Be_3 \left( \Be_1 e^{j \theta_i} E_{0 i}  - \Be_1 e^{-j \theta_r} E_{0 r}  \right) } = \inv{v_2} \gpgradezero{ \Be_3 \left( \Be_1 e^{j \theta_t} E_{0 t}  \right) }
\end{equation}
\begin{equation}\label{eqn:fresnelAlternatePolarization:350b}
\gpgradetwo{ \Be_3 \left( \Be_2 E_{0 i}  + \Be_2 E_{0 r}  \right) } = \gpgradetwo{ \Be_3 \left( \Be_2 E_{0 t}  \right) }
\end{equation}
\begin{equation}\label{eqn:fresnelAlternatePolarization:370b}
\inv{\mu_1 v_1} \gpgradetwo{ \Be_3 \left( \Be_1 e^{j \theta_i} E_{0 i}  -\Be_1 e^{-j \theta_r} E_{0 r}  \right) } = \inv{\mu_2 v_2} \gpgradetwo{ \Be_3 \left( \Be_1 e^{j \theta_t} E_{0 t}  \right) }.
\end{equation}
\end{subequations}

Evaluating the grade selections we have a separation into an analogue of real and imaginary parts for
\begin{subequations}
\label{eqn:fresnelAlternatePolarization:300c}
\begin{equation}\label{eqn:fresnelAlternatePolarization:310c}
0 = 0
\end{equation}
\begin{equation}\label{eqn:fresnelAlternatePolarization:330c}
\inv{v_1} \left( -\sin\theta_i E_{0 i}  - \sin\theta_r E_{0 r}  \right) = \inv{v_2} \left( -\sin\theta_t E_{0 t}  \right)
\end{equation}
\begin{equation}\label{eqn:fresnelAlternatePolarization:350c}
E_{0 i} + E_{0 r} = E_{0 t}
\end{equation}
\begin{equation}\label{eqn:fresnelAlternatePolarization:370c}
\inv{\mu_1 v_1} \left( \cos{\theta_i} E_{0 i}  - \cos{\theta_r} E_{0 r}  \right) = \inv{\mu_2 v_2} \left( \cos{ \theta_t} E_{0 t}  \right).
\end{equation}
\end{subequations}

With $\theta_i = \theta_r$ and $\sin\theta_t/\sin\theta_i = n_1/n_2$ \ref{eqn:fresnelAlternatePolarization:330c} becomes

\begin{dmath}\label{eqn:fresnelAlternatePolarization:430}
E_{0 i} + E_{0 r} 
= \frac{n_1 v_1}{n_2 v_2} E_{0 t} 
= \frac{v_2 v_1}{v_1 v_2} E_{0 t} 
= E_{0 t},
\end{dmath}

so that we find \ref{eqn:fresnelAlternatePolarization:330c} and \ref{eqn:fresnelAlternatePolarization:350c} are dependent.  We are left with a pair of equations
\begin{subequations}
\label{eqn:fresnelAlternatePolarization:450x}
\begin{dmath}\label{eqn:fresnelAlternatePolarization:450}
E_{0 i} + E_{0 r} = E_{0 t}
\end{dmath}
\begin{dmath}\label{eqn:fresnelAlternatePolarization:470}
E_{0 i} - E_{0 r} = \frac{\mu_1 v_1}{\mu_2 v_2} \frac{\cos{ \theta_t}}{\cos\theta_i} E_{0 t},
\end{dmath}
\end{subequations}

Adding and subtracting we have

\begin{subequations}
\begin{dmath}\label{eqn:fresnelAlternatePolarization:490}
2 E_{0 i} = \left( 1 + \frac{\mu_1 v_1}{\mu_2 v_2} \frac{\cos{ \theta_t}}{\cos\theta_i} \right) E_{0 t}
\end{dmath}
\begin{dmath}\label{eqn:fresnelAlternatePolarization:510}
2 E_{0 r} = \left( 1 - \frac{\mu_1 v_1}{\mu_2 v_2} \frac{\cos{ \theta_t}}{\cos\theta_i} \right) E_{0 t},
\end{dmath}
\end{subequations}

with a final rearrangement to yield

\begin{subequations}
\begin{dmath}\label{eqn:fresnelAlternatePolarization:530}
\frac{E_{0t}}{E_{0i}}
=
\frac{2 \mu_2 v_2 \cos\theta_i}
{
\mu_2 v_2 \cos\theta_i
+\mu_1 v_1 \cos\theta_t
}
\end{dmath}
\begin{dmath}\label{eqn:fresnelAlternatePolarization:550}
\frac{E_{0r}}{E_{0i}}
=
\frac
{
\mu_2 v_2 \cos\theta_i
-\mu_1 v_1 \cos\theta_t
}
{
\mu_2 v_2 \cos\theta_i
+\mu_1 v_1 \cos\theta_t
}
\end{dmath}
\end{subequations}

With

\begin{subequations}
\begin{dmath}\label{eqn:fresnelAlternatePolarization:750}
\alpha = \frac{\mu_2 v_2} {\mu_1 v_1} 
\end{dmath}
\begin{dmath}\label{eqn:fresnelAlternatePolarization:770}
\beta = \frac{\cos\theta_t}{\cos\theta_i},
\end{dmath}
\end{subequations}

(FIXME: check $\alpha, \beta$ notation against Griffiths) we have

\begin{subequations}
\begin{dmath}\label{eqn:fresnelAlternatePolarization:530a}
\frac{E_{0t}}{E_{0i}}
=
\frac
{
2 \alpha
}
{
\alpha + \beta
}
\end{dmath}
\begin{dmath}\label{eqn:fresnelAlternatePolarization:550a}
\frac{E_{0r}}{E_{0i}}
=
\frac
{
\alpha - \beta
}
{
\alpha + \beta
}
\end{dmath}
\end{subequations}

FIXME: check against Hecht and/or Griffiths.
}

\makeanswer{fresnelAlternatePolarization:pr2}{ 

As a second sanity check let's rotate our field polarizations by applying a rotation $e^{\Be_2 \mcap_\alpha \pi/2} = \Be_2 \mcap_\alpha$ ($\phi_i = \phi_r = \phi_t = -\pi/2$) so that

\begin{subequations}
\begin{dmath}\label{eqn:fresnelAlternatePolarization:570}
-\mcap_\alpha \rightarrow -\mcap_\alpha \Be_2 \mcap_\alpha = \Be_2
\end{dmath}
\begin{dmath}\label{eqn:fresnelAlternatePolarization:590}
\Be_2 \rightarrow \Be_2 \Be_2 \mcap_\alpha = \mcap_\alpha
\end{dmath}
\end{subequations}

This time we have $\BE_\alpha \parallel \mcap_\alpha$ and $\BB_\alpha \parallel \Be_2$.  Our boundary value equations become

\begin{subequations}
\label{eqn:fresnelAlternatePolarization:610}
\begin{equation}\label{eqn:fresnelAlternatePolarization:630}
\epsilon_1 \gpgradezero{ \Be_3 \left( \mcap_i E_{0 i}  + \mcap_r E_{0 r}  \right) } = \epsilon_2 \gpgradezero{ \Be_3 \left( \mcap_t E_{0 t}  \right) }
\end{equation}
\begin{equation}\label{eqn:fresnelAlternatePolarization:650}
\inv{v_1} \gpgradezero{ \Be_3 \left( \Be_2 E_{0 i} + \Be_2 E_{0 r}  \right) } = \inv{v_2} \gpgradezero{ \Be_3 \left( \Be_2 E_{0 t}  \right) }
\end{equation}
\begin{equation}\label{eqn:fresnelAlternatePolarization:670}
\gpgradetwo{ \Be_3 \left( \mcap_i E_{0 i}  + \mcap_r E_{0 r}  \right) } = \gpgradetwo{ \Be_3 \left( \mcap_t E_{0 t}  \right) }
\end{equation}
\begin{equation}\label{eqn:fresnelAlternatePolarization:690}
\inv{\mu_1 v_1} \gpgradetwo{ \Be_3 \left( \Be_2 E_{0 i}  + \Be_2 E_{0 r}  \right) } = \inv{\mu_2 v_2} \gpgradetwo{ \Be_3 \left( \Be_2 E_{0 t}  \right) }.
\end{equation}
\end{subequations}

This second equation \ref{eqn:fresnelAlternatePolarization:650} is a $0 = 0$ identity, and the remaining after $\mcap_\alpha$ substitution are

\begin{subequations}
\label{eqn:fresnelAlternatePolarization:610a}
\begin{equation}\label{eqn:fresnelAlternatePolarization:630a}
\epsilon_1 \gpgradezero{ \Be_3 \left( \Be_3 j e^{j \theta_i} E_{0 i}  + (-\Be_3) j e^{-j \theta_r} E_{0 r}  \right) } = \epsilon_2 \gpgradezero{ \Be_3 \left( \Be_3 j e^{j \theta_t} E_{0 t}  \right) }
\end{equation}
\begin{equation}\label{eqn:fresnelAlternatePolarization:670a}
\gpgradetwo{ \Be_3 \left( \Be_3 j e^{j \theta_i} E_{0 i}  + (-\Be_3) j e^{-j \theta_r} E_{0 r}  \right) } = \gpgradetwo{ \Be_3 \left( \Be_3 j e^{j \theta_t} E_{0 t}  \right) }
\end{equation}
\begin{equation}\label{eqn:fresnelAlternatePolarization:690a}
\inv{\mu_1 v_1} \gpgradetwo{ \Be_3 \left( \Be_2 E_{0 i}  + \Be_2 E_{0 r}  \right) } = \inv{\mu_2 v_2} \gpgradetwo{ \Be_3 \left( \Be_2 E_{0 t}  \right) }.
\end{equation}
\end{subequations}

Simplifying we have

\begin{subequations}
\label{eqn:fresnelAlternatePolarization:610b}
\begin{equation}\label{eqn:fresnelAlternatePolarization:630b}
\epsilon_1 \left(  -\sin \theta_i E_{0 i}  - \sin{\theta_r} E_{0 r}  \right) = - \epsilon_2 \sin{\theta_t} E_{0 t}  
\end{equation}
\begin{equation}\label{eqn:fresnelAlternatePolarization:670b}
 \cos{ \theta_i} E_{0 i}  - \cos{ \theta_r} E_{0 r} = \cos{ \theta_t} E_{0 t}
\end{equation}
\begin{equation}\label{eqn:fresnelAlternatePolarization:690b}
E_{0 i}  + E_{0 r} = \frac{\mu_1 v_1} {\mu_2 v_2} E_{0 t}
\end{equation}
\end{subequations}

We expect an equality

\begin{dmath}\label{eqn:fresnelAlternatePolarization:710}
\frac{\epsilon_2 \sin\theta_t}{\epsilon_1 \sin\theta_i} =  \frac{\mu_1 v_1} {\mu_2 v_2},
\end{dmath}

Noting that $\epsilon_\alpha v_\alpha = 1/(v_\alpha \mu_\alpha)$ we find that to be true

\begin{dmath}\label{eqn:fresnelAlternatePolarization:730}
\frac{\epsilon_2 \sin\theta_t}{\epsilon_1 \sin\theta_i} 
= 
\frac{\epsilon_2 n_1}{\epsilon_1 n_2} 
= 
\frac{\epsilon_2 v_2}{\epsilon_1 v_1} 
= 
\frac{\mu_1 v_1}{\mu_2 v_2} 
\end{dmath}

we see that \ref{eqn:fresnelAlternatePolarization:630b} and \ref{eqn:fresnelAlternatePolarization:690b} are dependent.  We are left with the system

\begin{subequations}
\label{eqn:fresnelAlternatePolarization:610d}
\begin{equation}\label{eqn:fresnelAlternatePolarization:670d}
E_{0 i} - E_{0 r} = \beta E_{0 t}
\end{equation}
\begin{equation}\label{eqn:fresnelAlternatePolarization:690d}
\alpha (E_{0 i} + E_{0 r}) = E_{0 t},
\end{equation}
\end{subequations}

with solution
\begin{subequations}
\begin{dmath}\label{eqn:fresnelAlternatePolarization:790}
\frac{E_t}{E_i} = \frac{2 \alpha}{1 + \alpha \beta}
\end{dmath}
\begin{dmath}\label{eqn:fresnelAlternatePolarization:810}
\frac{E_r}{E_i} = \frac{1 - \alpha \beta}{1 + \alpha \beta}
\end{dmath}
\end{subequations}
}

\makeanswer{fresnelAlternatePolarization:pr3}{

Substituting our $\mcap_\alpha$ vector expressions into the boundary value constraints we have

\begin{subequations}
\label{eqn:fresnelAlternatePolarization:830}
\begin{equation}\label{eqn:fresnelAlternatePolarization:850}
\epsilon_1 \gpgradezero{ \Be_3 \Be_2 \left( E_{0 i} e^{ \mcap_i \Be_2 \phi_i } + E_{0 r} e^{ \mcap_r \Be_2 \phi_r } \right) } = \epsilon_2 \gpgradezero{ \Be_3 \Be_2 E_{0 t} e^{ \mcap_t \Be_2 \phi_t } } 
\end{equation}
\begin{equation}\label{eqn:fresnelAlternatePolarization:870}
\inv{v_1} \gpgradezero{ j e^{j \theta_i} E_{0 i} e^{ \mcap_i \Be_2 \phi_i } - j e^{-j \theta_r} E_{0 r} e^{ \mcap_r \Be_2 \phi_r } } = \inv{v_2} \gpgradezero{ j e^{j \theta_t} E_{0 t} e^{ \mcap_t \Be_2 \phi_t } }
\end{equation}
\begin{equation}\label{eqn:fresnelAlternatePolarization:890}
\gpgradetwo{ \Be_3 \Be_2 \left( E_{0 i} e^{ \mcap_i \Be_2 \phi_i } + E_{0 r} e^{ \mcap_r \Be_2 \phi_r } \right) } = \gpgradetwo{ \Be_3 \Be_2 E_{0 t} e^{ \mcap_t \Be_2 \phi_t } }
\end{equation}
\begin{equation}\label{eqn:fresnelAlternatePolarization:910}
\inv{\mu_1 v_1} \gpgradetwo{ j e^{j \theta_i} E_{0 i} e^{ \mcap_i \Be_2 \phi_i } - j e^{-j \theta_r} E_{0 r} e^{ \mcap_r \Be_2 \phi_r } } = \inv{\mu_2 v_2} \gpgradetwo{ j e^{j \theta_t} E_{0 t} e^{ \mcap_t \Be_2 \phi_t } }
\end{equation}
\end{subequations}

With $\beta \in \{i,r\}$ we want to expand some intermediate multivector products

\begin{dmath}\label{eqn:fresnelAlternatePolarization:930}
\Be_{32} e^{\mcap_\beta \Be_2 \phi_\beta}
=
\Be_{32} \cos \phi_\beta
+\Be_{32} \mcap_\beta \Be_2 \sin{\phi_\beta}
=
\Be_{32} \cos \phi_\beta
+\Be_{32} \Be_3 j e^{j \theta_\beta} \Be_2 \sin{\phi_\beta}
=
\Be_{32} \cos \phi_\beta
-j e^{j \theta_\beta} \sin{\phi_\beta}
=
\Be_{32} \cos \phi_\beta - \Be_{31} \cos\theta_\beta \sin\phi_\beta
+ \sin\theta_\beta \sin{\phi_\beta}
\end{dmath}

\begin{dmath}\label{eqn:fresnelAlternatePolarization:930b}
\Be_{32} e^{\mcap_r \Be_2 \phi_r}
=
\Be_{32} \cos \phi_r
+\Be_{32} \mcap_r \Be_2 \sin{\phi_r}
=
\Be_{32} \cos \phi_r
+\Be_{32} (-\Be_3) j e^{-j \theta_r} \Be_2 \sin{\phi_r}
=
\Be_{32} \cos \phi_r
+ j e^{-j \theta_r} \sin{\phi_r}
=
\Be_{32} \cos \phi_r + \Be_{31} \cos\theta_r \sin{\phi_r}
+ \sin \theta_r \sin{\phi_r}
\end{dmath}

\begin{dmath}\label{eqn:fresnelAlternatePolarization:950}
j e^{j \theta_\beta} e^{\mcap_\beta \Be_2 \phi_\beta}
=
j e^{j \theta_\beta} \left( 
\cos \phi_\beta
+ \mcap_\beta \Be_2 \sin{\phi_\beta}
\right)
=
j e^{j \theta_\beta} \left( 
\cos \phi_\beta
+ \Be_3 j e^{j \theta_\beta} \Be_2 \sin{\phi_\beta}
\right)
=
j e^{j \theta_\beta} \left( 
\cos \phi_\beta
- j e^{-j \theta_\beta} \Be_{32} \sin{\phi_\beta}
\right)
=
j e^{j \theta_\beta} \cos \phi_\beta
+ \Be_{32} \sin{\phi_\beta}
=
\Be_{31} \cos {j \theta_\beta} \cos \phi_\beta
+ \Be_{32} \sin{\phi_\beta}
- \sin{ \theta_\beta} \cos \phi_\beta
\end{dmath}

\begin{dmath}\label{eqn:fresnelAlternatePolarization:970}
-j e^{-j \theta_r} e^{\mcap_r \Be_2 \phi_r}
=
-j e^{-j \theta_r} \left(
\cos\phi_r + \mcap_r \Be_2 \sin\phi_r
\right)
=
-j e^{-j \theta_r} \left(
\cos\phi_r - \Be_3 j e^{-j \theta_r} \Be_2 \sin\phi_r
\right)
=
-j e^{-j \theta_r} \left(
\cos\phi_r + j e^{j \theta_r} \Be_{32} \sin\phi_r
\right)
=
-j e^{-j \theta_r} \cos\phi_r 
+ \Be_{32} \sin\phi_r
=
-\Be_{31} \cos{\theta_r} \cos\phi_r 
+ \Be_{32} \sin\phi_r
- \sin{ \theta_r} \cos\phi_r 
\end{dmath}

Our boundary value conditions are then

\begin{subequations}
\begin{dmath}\label{eqn:fresnelAlternatePolarization:990}
\epsilon_1 \left( 
E_{0 i} 
\sin\theta_i \sin{\phi_i}
+ E_{0 r} 
\sin \theta_r \sin{\phi_r}
\right) 
= \epsilon_2 E_{0 t} 
\sin\theta_t \sin{\phi_t}
\end{dmath}
\begin{dmath}\label{eqn:fresnelAlternatePolarization:1010}
\inv{v_1}
\left(
E_{0 i} 
\sin{ \theta_i} \cos \phi_i
+
E_{0 r} 
\sin{ \theta_r} \cos\phi_r 
\right)
=
\inv{v_2}
E_{0 t} 
\sin{ \theta_t} \cos \phi_t
\end{dmath}
%\begin{dmath}\label{eqn:fresnelAlternatePolarization:1030}
%E_{0 i} 
%\left(
%\Be_{2} \cos \phi_i - \Be_{1} \cos\theta_i \sin\phi_i
%\right)
%+ 
%E_{0 r} 
%\left(
%\Be_{2} \cos \phi_r + \Be_{1} \cos\theta_r \sin{\phi_r}
%\right)
%=
%E_{0 t} 
%\left(
%\Be_{2} \cos \phi_t - \Be_{1} \cos\theta_t \sin\phi_t
%\right)
%\end{dmath}
%\begin{dmath}\label{eqn:fresnelAlternatePolarization:1050}
%\inv{\mu_1 v_1}
%\left(
%E_{0 i}
%\left(
%\Be_{1} \cos { \theta_i} \cos \phi_i + \Be_{2} \sin{\phi_i}
%\right)
%+
%E_{0 r}
%\left(
%-\Be_{1} \cos{\theta_r} \cos\phi_r + \Be_{2} \sin\phi_r
%\right)
%\right)
%=
%\inv{\mu_2 v_2}
%E_{0 t}
%\left(
%\Be_{1} \cos { \theta_t} \cos \phi_t + \Be_{2} \sin{\phi_t}
%\right)
%\end{dmath}
\begin{dmath}\label{eqn:fresnelAlternatePolarization:1030a}
E_{0 i} 
\cos \phi_i 
+ 
E_{0 r} 
\cos \phi_r 
=
E_{0 t} 
\cos \phi_t 
\end{dmath}

\begin{dmath}\label{eqn:fresnelAlternatePolarization:1030b}
-E_{0 i} 
\cos\theta_i \sin\phi_i
+ 
E_{0 r} 
\cos\theta_r \sin{\phi_r}
=
-E_{0 t} 
\cos\theta_t \sin\phi_t
\end{dmath}
\begin{dmath}\label{eqn:fresnelAlternatePolarization:1050a}
\inv{\mu_1 v_1}
\left(
E_{0 i}
\cos { \theta_i} \cos \phi_i 
-
E_{0 r}
\cos{\theta_r} \cos\phi_r 
\right)
=
\inv{\mu_2 v_2}
E_{0 t}
 \cos { \theta_t} \cos \phi_t 
\end{dmath}

\begin{dmath}\label{eqn:fresnelAlternatePolarization:1050b}
\inv{\mu_1 v_1}
\left(
E_{0 i}
\sin{\phi_i}
+
E_{0 r}
\sin\phi_r
\right)
=
\inv{\mu_2 v_2}
E_{0 t}
\sin{\phi_t}
\end{dmath}
\end{subequations}

Note that the wedge product equations above have been separated into $\Be_3 \Be_1$ and $\Be_3 \Be_2$ components, yielding two equations each.  Because of \ref{eqn:fresnelAlternatePolarization:710}, we see that \ref{eqn:fresnelAlternatePolarization:990} and \ref{eqn:fresnelAlternatePolarization:1050b} are dependent.  Also, as demonstrated in \ref{eqn:fresnelAlternatePolarization:430} we see that \ref{eqn:fresnelAlternatePolarization:1010} and \ref{eqn:fresnelAlternatePolarization:1030a} are also dependent.  We can therefore consider only the last four equations (and still have additional linear dependencies to be discovered.)

Let's write these as
\begin{subequations}
\begin{dmath}\label{eqn:fresnelAlternatePolarization:1070}
E_{0 i} 
\cos \phi_i 
+ 
E_{0 r} 
\cos \phi_r 
=
E_{0 t} 
\cos \phi_t 
\end{dmath}

\begin{dmath}\label{eqn:fresnelAlternatePolarization:1090}
-E_{0 i} 
\sin\phi_i
+ 
E_{0 r} 
\sin{\phi_r}
=
-E_{0 t} \beta
\sin\phi_t
\end{dmath}
\begin{dmath}\label{eqn:fresnelAlternatePolarization:1110}
E_{0 i}
\cos \phi_i 
-
E_{0 r}
\cos\phi_r 
=
\frac{\beta}{\alpha}
E_{0 t}
\cos \phi_t 
\end{dmath}

\begin{dmath}\label{eqn:fresnelAlternatePolarization:1130}
E_{0 i}
\sin{\phi_i}
+
E_{0 r}
\sin\phi_r
=
\inv{\alpha}
E_{0 t}
\sin{\phi_t}
\end{dmath}
\end{subequations}

Observe that if $\phi_i = \phi_r = \phi_t = 0$ (killing all the sine terms) we recover \ref{eqn:fresnelAlternatePolarization:450x}, and with $\phi_i = \phi_r = \phi_t = \pi/2$ (killing all the cosines) we recover \ref{eqn:fresnelAlternatePolarization:610d}.

Now, if $\phi_i n \pi/2$ we've got a different story.  Specifically it appears that should we wish to solve for the reflected and transmitted magnitudes, we also have to simulaneously solve for the polarization angles in the reflected and transmitted directions.  This is now a problem of solving four simulaneous equations in two linear and two non-linear variables.  

Does it make sense that we would have polarization rotation should our initial polarization angle be rotated?  I think so.  In dicusssing this problem with Prof Thywissen, he strongly suggested treating the problem as a superposition of two light waves.  If we consider that, even without attempting to solve the problem, we see that we must have different reflected and transmitted magnitudes associated with the pair of incident waves since we have to calculate each of these with different Fresnel equations.  This would have an effect of scaling and rotating the superimposed reflected and transmitted waves.
}

\makeanswer{ch:fresnelAlternatePolarization:pr4}{ 
FIXME: ... TODO.
}

%\vcsinfo
\EndArticle
