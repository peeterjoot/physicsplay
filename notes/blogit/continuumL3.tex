%
% Copyright � 2015 Peeter Joot.  All Rights Reserved.
% Licenced as described in the file LICENSE under the root directory of this GIT repository.
%
\documentclass[]{eliblog}

\usepackage{amsmath}
\usepackage{mathpazo}

%
% shorthand for bold symbols, convenient for vectors and matrices
%
\newcommand{\Ba}[0]{\mathbf{a}}
\newcommand{\Bb}[0]{\mathbf{b}}
\newcommand{\Bc}[0]{\mathbf{c}}
\newcommand{\Bd}[0]{\mathbf{d}}
\newcommand{\Be}[0]{\mathbf{e}}
\newcommand{\Bf}[0]{\mathbf{f}}
\newcommand{\Bg}[0]{\mathbf{g}}
\newcommand{\Bh}[0]{\mathbf{h}}
\newcommand{\Bi}[0]{\mathbf{i}}
\newcommand{\Bj}[0]{\mathbf{j}}
\newcommand{\Bk}[0]{\mathbf{k}}
\newcommand{\Bl}[0]{\mathbf{l}}
\newcommand{\Bm}[0]{\mathbf{m}}
\newcommand{\Bn}[0]{\mathbf{n}}
\newcommand{\Bo}[0]{\mathbf{o}}
\newcommand{\Bp}[0]{\mathbf{p}}
\newcommand{\Bq}[0]{\mathbf{q}}
\newcommand{\Br}[0]{\mathbf{r}}
\newcommand{\Bs}[0]{\mathbf{s}}
\newcommand{\Bt}[0]{\mathbf{t}}
\newcommand{\Bu}[0]{\mathbf{u}}
\newcommand{\Bv}[0]{\mathbf{v}}
\newcommand{\Bw}[0]{\mathbf{w}}
\newcommand{\Bx}[0]{\mathbf{x}}
\newcommand{\By}[0]{\mathbf{y}}
\newcommand{\Bz}[0]{\mathbf{z}}
\newcommand{\BA}[0]{\mathbf{A}}
\newcommand{\BB}[0]{\mathbf{B}}
\newcommand{\BC}[0]{\mathbf{C}}
\newcommand{\BD}[0]{\mathbf{D}}
\newcommand{\BE}[0]{\mathbf{E}}
\newcommand{\BF}[0]{\mathbf{F}}
\newcommand{\BG}[0]{\mathbf{G}}
\newcommand{\BH}[0]{\mathbf{H}}
\newcommand{\BI}[0]{\mathbf{I}}
\newcommand{\BJ}[0]{\mathbf{J}}
\newcommand{\BK}[0]{\mathbf{K}}
\newcommand{\BL}[0]{\mathbf{L}}
\newcommand{\BM}[0]{\mathbf{M}}
\newcommand{\BN}[0]{\mathbf{N}}
\newcommand{\BO}[0]{\mathbf{O}}
\newcommand{\BP}[0]{\mathbf{P}}
\newcommand{\BQ}[0]{\mathbf{Q}}
\newcommand{\BR}[0]{\mathbf{R}}
\newcommand{\BS}[0]{\mathbf{S}}
\newcommand{\BT}[0]{\mathbf{T}}
\newcommand{\BU}[0]{\mathbf{U}}
\newcommand{\BV}[0]{\mathbf{V}}
\newcommand{\BW}[0]{\mathbf{W}}
\newcommand{\BX}[0]{\mathbf{X}}
\newcommand{\BY}[0]{\mathbf{Y}}
\newcommand{\BZ}[0]{\mathbf{Z}}

\newcommand{\Bzero}[0]{\mathbf{0}}
\newcommand{\Btheta}[0]{\boldsymbol{\theta}}
\newcommand{\Btau}[0]{\boldsymbol{\tau}}
\newcommand{\Bomega}[0]{\boldsymbol{\omega}}

%
% shorthand for unit vectors
%
\newcommand{\acap}[0]{\hat{\Ba}}
\newcommand{\bcap}[0]{\hat{\Bb}}
\newcommand{\ccap}[0]{\hat{\Bc}}
\newcommand{\dcap}[0]{\hat{\Bd}}
\newcommand{\ecap}[0]{\hat{\Be}}
\newcommand{\fcap}[0]{\hat{\Bf}}
\newcommand{\gcap}[0]{\hat{\Bg}}
\newcommand{\hcap}[0]{\hat{\Bh}}
\newcommand{\icap}[0]{\hat{\Bi}}
\newcommand{\jcap}[0]{\hat{\Bj}}
\newcommand{\kcap}[0]{\hat{\Bk}}
\newcommand{\lcap}[0]{\hat{\Bl}}
\newcommand{\mcap}[0]{\hat{\Bm}}
\newcommand{\ncap}[0]{\hat{\Bn}}
\newcommand{\ocap}[0]{\hat{\Bo}}
\newcommand{\pcap}[0]{\hat{\Bp}}
\newcommand{\qcap}[0]{\hat{\Bq}}
\newcommand{\rcap}[0]{\hat{\Br}}
\newcommand{\scap}[0]{\hat{\Bs}}
\newcommand{\tcap}[0]{\hat{\Bt}}
\newcommand{\ucap}[0]{\hat{\Bu}}
\newcommand{\vcap}[0]{\hat{\Bv}}
\newcommand{\wcap}[0]{\hat{\Bw}}
\newcommand{\xcap}[0]{\hat{\Bx}}
\newcommand{\ycap}[0]{\hat{\By}}
\newcommand{\zcap}[0]{\hat{\Bz}}
\newcommand{\thetacap}[0]{\hat{\Btheta}}

%
% to write R^n and C^n in a distinguishable fashion.  Perhaps change this
% to the double lined characters upon figuring out how to do so.
%
\newcommand{\C}[1]{$\mathbb{C}^{#1}$}
\newcommand{\R}[1]{$\mathbb{R}^{#1}$}

%
% various generally useful helpers
%

% derivative of #1 wrt. #2:
\newcommand{\D}[2] {\frac {d#2} {d#1}}

\newcommand{\inv}[1]{\frac{1}{#1}}
\newcommand{\cross}[0]{\times}

\newcommand{\abs}[1]{\lvert{#1}\rvert}
\newcommand{\norm}[1]{\lVert{#1}\rVert}
\newcommand{\innerprod}[2]{\langle{#1}, {#2}\rangle}
\newcommand{\dotprod}[2]{{#1} \cdot {#2}}
\newcommand{\bdotprod}[2]{\left({#1} \cdot {#2}\right)}
\newcommand{\crossprod}[2]{{#1} \cross {#2}}
\newcommand{\tripleprod}[3]{\dotprod{\left(\crossprod{#1}{#2}\right)}{#3}}

\DeclareMathOperator{\Proj}{Proj}
\DeclareMathOperator{\Span}{span}
\DeclareMathOperator{\Sgn}{sgn}
\DeclareMathOperator{\Area}{Area}
\DeclareMathOperator{\Volume}{Volume}

%
% A few miscellaneous things specific to this document
%
\newcommand{\crossop}[1]{\crossprod{#1}{}}

% R2 vector.
\newcommand{\VectorTwo}[2]{
\begin{bmatrix}
 {#1} \\
 {#2}
\end{bmatrix}
}

\newcommand{\VectorN}[1]{
\begin{bmatrix}
{#1}_1 \\
{#1}_2 \\
\vdots \\
{#1}_N \\
\end{bmatrix}
}

\newcommand{\DETuvij}[4]{
\begin{vmatrix}
 {#1}_{#3} & {#1}_{#4} \\
 {#2}_{#3} & {#2}_{#4}
\end{vmatrix}
}

\newcommand{\DETuvwijk}[6]{
\begin{vmatrix}
 {#1}_{#4} & {#1}_{#5} & {#1}_{#6} \\
 {#2}_{#4} & {#2}_{#5} & {#2}_{#6} \\
 {#3}_{#4} & {#3}_{#5} & {#3}_{#6}
\end{vmatrix}
}

\newcommand{\DETuvwxijkl}[8]{
\begin{vmatrix}
 {#1}_{#5} & {#1}_{#6} & {#1}_{#7} & {#1}_{#8} \\
 {#2}_{#5} & {#2}_{#6} & {#2}_{#7} & {#2}_{#8} \\
 {#3}_{#5} & {#3}_{#6} & {#3}_{#7} & {#3}_{#8} \\
 {#4}_{#5} & {#4}_{#6} & {#4}_{#7} & {#4}_{#8} \\
\end{vmatrix}
}

%\newcommand{\DETuvwxyijklm}[10]{
%\begin{vmatrix}
% {#1}_{#6} & {#1}_{#7} & {#1}_{#8} & {#1}_{#9} & {#1}_{#10} \\
% {#2}_{#6} & {#2}_{#7} & {#2}_{#8} & {#2}_{#9} & {#2}_{#10} \\
% {#3}_{#6} & {#3}_{#7} & {#3}_{#8} & {#3}_{#9} & {#3}_{#10} \\
% {#4}_{#6} & {#4}_{#7} & {#4}_{#8} & {#4}_{#9} & {#4}_{#10} \\
% {#5}_{#6} & {#5}_{#7} & {#5}_{#8} & {#5}_{#9} & {#5}_{#10}
%\end{vmatrix}
%}

% R3 vector.
\newcommand{\VectorThree}[3]{
\begin{bmatrix}
 {#1} \\
 {#2} \\
 {#3}
\end{bmatrix}
}



\author{Peeter Joot}
\email{peeter.joot@gmail.com}

%\documentclass[]{eliblogwidescreen}

\usepackage{amsmath}
\usepackage{mathpazo}

%
% shorthand for bold symbols, convenient for vectors and matrices
%
\newcommand{\Ba}[0]{\mathbf{a}}
\newcommand{\Bb}[0]{\mathbf{b}}
\newcommand{\Bc}[0]{\mathbf{c}}
\newcommand{\Bd}[0]{\mathbf{d}}
\newcommand{\Be}[0]{\mathbf{e}}
\newcommand{\Bf}[0]{\mathbf{f}}
\newcommand{\Bg}[0]{\mathbf{g}}
\newcommand{\Bh}[0]{\mathbf{h}}
\newcommand{\Bi}[0]{\mathbf{i}}
\newcommand{\Bj}[0]{\mathbf{j}}
\newcommand{\Bk}[0]{\mathbf{k}}
\newcommand{\Bl}[0]{\mathbf{l}}
\newcommand{\Bm}[0]{\mathbf{m}}
\newcommand{\Bn}[0]{\mathbf{n}}
\newcommand{\Bo}[0]{\mathbf{o}}
\newcommand{\Bp}[0]{\mathbf{p}}
\newcommand{\Bq}[0]{\mathbf{q}}
\newcommand{\Br}[0]{\mathbf{r}}
\newcommand{\Bs}[0]{\mathbf{s}}
\newcommand{\Bt}[0]{\mathbf{t}}
\newcommand{\Bu}[0]{\mathbf{u}}
\newcommand{\Bv}[0]{\mathbf{v}}
\newcommand{\Bw}[0]{\mathbf{w}}
\newcommand{\Bx}[0]{\mathbf{x}}
\newcommand{\By}[0]{\mathbf{y}}
\newcommand{\Bz}[0]{\mathbf{z}}
\newcommand{\BA}[0]{\mathbf{A}}
\newcommand{\BB}[0]{\mathbf{B}}
\newcommand{\BC}[0]{\mathbf{C}}
\newcommand{\BD}[0]{\mathbf{D}}
\newcommand{\BE}[0]{\mathbf{E}}
\newcommand{\BF}[0]{\mathbf{F}}
\newcommand{\BG}[0]{\mathbf{G}}
\newcommand{\BH}[0]{\mathbf{H}}
\newcommand{\BI}[0]{\mathbf{I}}
\newcommand{\BJ}[0]{\mathbf{J}}
\newcommand{\BK}[0]{\mathbf{K}}
\newcommand{\BL}[0]{\mathbf{L}}
\newcommand{\BM}[0]{\mathbf{M}}
\newcommand{\BN}[0]{\mathbf{N}}
\newcommand{\BO}[0]{\mathbf{O}}
\newcommand{\BP}[0]{\mathbf{P}}
\newcommand{\BQ}[0]{\mathbf{Q}}
\newcommand{\BR}[0]{\mathbf{R}}
\newcommand{\BS}[0]{\mathbf{S}}
\newcommand{\BT}[0]{\mathbf{T}}
\newcommand{\BU}[0]{\mathbf{U}}
\newcommand{\BV}[0]{\mathbf{V}}
\newcommand{\BW}[0]{\mathbf{W}}
\newcommand{\BX}[0]{\mathbf{X}}
\newcommand{\BY}[0]{\mathbf{Y}}
\newcommand{\BZ}[0]{\mathbf{Z}}

\newcommand{\Bzero}[0]{\mathbf{0}}
\newcommand{\Btheta}[0]{\boldsymbol{\theta}}
\newcommand{\Btau}[0]{\boldsymbol{\tau}}
\newcommand{\Bomega}[0]{\boldsymbol{\omega}}

%
% shorthand for unit vectors
%
\newcommand{\acap}[0]{\hat{\Ba}}
\newcommand{\bcap}[0]{\hat{\Bb}}
\newcommand{\ccap}[0]{\hat{\Bc}}
\newcommand{\dcap}[0]{\hat{\Bd}}
\newcommand{\ecap}[0]{\hat{\Be}}
\newcommand{\fcap}[0]{\hat{\Bf}}
\newcommand{\gcap}[0]{\hat{\Bg}}
\newcommand{\hcap}[0]{\hat{\Bh}}
\newcommand{\icap}[0]{\hat{\Bi}}
\newcommand{\jcap}[0]{\hat{\Bj}}
\newcommand{\kcap}[0]{\hat{\Bk}}
\newcommand{\lcap}[0]{\hat{\Bl}}
\newcommand{\mcap}[0]{\hat{\Bm}}
\newcommand{\ncap}[0]{\hat{\Bn}}
\newcommand{\ocap}[0]{\hat{\Bo}}
\newcommand{\pcap}[0]{\hat{\Bp}}
\newcommand{\qcap}[0]{\hat{\Bq}}
\newcommand{\rcap}[0]{\hat{\Br}}
\newcommand{\scap}[0]{\hat{\Bs}}
\newcommand{\tcap}[0]{\hat{\Bt}}
\newcommand{\ucap}[0]{\hat{\Bu}}
\newcommand{\vcap}[0]{\hat{\Bv}}
\newcommand{\wcap}[0]{\hat{\Bw}}
\newcommand{\xcap}[0]{\hat{\Bx}}
\newcommand{\ycap}[0]{\hat{\By}}
\newcommand{\zcap}[0]{\hat{\Bz}}
\newcommand{\thetacap}[0]{\hat{\Btheta}}

%
% to write R^n and C^n in a distinguishable fashion.  Perhaps change this
% to the double lined characters upon figuring out how to do so.
%
\newcommand{\C}[1]{$\mathbb{C}^{#1}$}
\newcommand{\R}[1]{$\mathbb{R}^{#1}$}

%
% various generally useful helpers
%

% derivative of #1 wrt. #2:
\newcommand{\D}[2] {\frac {d#2} {d#1}}

\newcommand{\inv}[1]{\frac{1}{#1}}
\newcommand{\cross}[0]{\times}

\newcommand{\abs}[1]{\lvert{#1}\rvert}
\newcommand{\norm}[1]{\lVert{#1}\rVert}
\newcommand{\innerprod}[2]{\langle{#1}, {#2}\rangle}
\newcommand{\dotprod}[2]{{#1} \cdot {#2}}
\newcommand{\bdotprod}[2]{\left({#1} \cdot {#2}\right)}
\newcommand{\crossprod}[2]{{#1} \cross {#2}}
\newcommand{\tripleprod}[3]{\dotprod{\left(\crossprod{#1}{#2}\right)}{#3}}

\DeclareMathOperator{\Proj}{Proj}
\DeclareMathOperator{\Span}{span}
\DeclareMathOperator{\Sgn}{sgn}
\DeclareMathOperator{\Area}{Area}
\DeclareMathOperator{\Volume}{Volume}

%
% A few miscellaneous things specific to this document
%
\newcommand{\crossop}[1]{\crossprod{#1}{}}

% R2 vector.
\newcommand{\VectorTwo}[2]{
\begin{bmatrix}
 {#1} \\
 {#2}
\end{bmatrix}
}

\newcommand{\VectorN}[1]{
\begin{bmatrix}
{#1}_1 \\
{#1}_2 \\
\vdots \\
{#1}_N \\
\end{bmatrix}
}

\newcommand{\DETuvij}[4]{
\begin{vmatrix}
 {#1}_{#3} & {#1}_{#4} \\
 {#2}_{#3} & {#2}_{#4}
\end{vmatrix}
}

\newcommand{\DETuvwijk}[6]{
\begin{vmatrix}
 {#1}_{#4} & {#1}_{#5} & {#1}_{#6} \\
 {#2}_{#4} & {#2}_{#5} & {#2}_{#6} \\
 {#3}_{#4} & {#3}_{#5} & {#3}_{#6}
\end{vmatrix}
}

\newcommand{\DETuvwxijkl}[8]{
\begin{vmatrix}
 {#1}_{#5} & {#1}_{#6} & {#1}_{#7} & {#1}_{#8} \\
 {#2}_{#5} & {#2}_{#6} & {#2}_{#7} & {#2}_{#8} \\
 {#3}_{#5} & {#3}_{#6} & {#3}_{#7} & {#3}_{#8} \\
 {#4}_{#5} & {#4}_{#6} & {#4}_{#7} & {#4}_{#8} \\
\end{vmatrix}
}

%\newcommand{\DETuvwxyijklm}[10]{
%\begin{vmatrix}
% {#1}_{#6} & {#1}_{#7} & {#1}_{#8} & {#1}_{#9} & {#1}_{#10} \\
% {#2}_{#6} & {#2}_{#7} & {#2}_{#8} & {#2}_{#9} & {#2}_{#10} \\
% {#3}_{#6} & {#3}_{#7} & {#3}_{#8} & {#3}_{#9} & {#3}_{#10} \\
% {#4}_{#6} & {#4}_{#7} & {#4}_{#8} & {#4}_{#9} & {#4}_{#10} \\
% {#5}_{#6} & {#5}_{#7} & {#5}_{#8} & {#5}_{#9} & {#5}_{#10}
%\end{vmatrix}
%}

% R3 vector.
\newcommand{\VectorThree}[3]{
\begin{bmatrix}
 {#1} \\
 {#2} \\
 {#3}
\end{bmatrix}
}



\author{Peeter Joot}
\email{peeter.joot@gmail.com}


\chapter{PHY454H1S\\Continuum Mechanics.  Lecture 3.  Strain tensor review.  Stress tensor.  Taught by Prof. K. Das.}
\label{chap:continuumL3}
%\useCCL
\blogpage{http://sites.google.com/site/peeterjoot2/math2012/continuumL3.pdf}
\date{Jan 18, 2012}
\revisionInfo{continuumL3.tex}

\beginArtWithToc
%\beginArtNoToc

%\section{Disclaimer.}
%
%Peeter's lecture notes from class.  May not be entirely coherent.
%
\section{Review.  Strain.}

Strain is the measure of stretching.  This is illustrated pictorially in figure (\ref{fig:continuumL3:continuumL3fig1})
\begin{figure}[htp]
   \centering
   \includegraphics[totalheight=0.2\textheight]{continuumL3fig1}
   \caption{Stretched line elements.}\label{fig:continuumL3:continuumL3fig1}
\end{figure}

\begin{equation}\label{eqn:continuumL3:10}
{ds'}^2 - ds^2 = 2 e_{ik} dx_i dx_k,
\end{equation}

where $e_{ik}$ is the strain tensor.  We found

\begin{equation}\label{eqn:continuumL3:30}
e_{ik} = \inv{2} \left( 
\PD{x_k}{e_i} 
+\PD{x_i}{e_k} 
+
\PD{x_i}{e_l} 
\PD{x_k}{e_l} 
\right)
\end{equation}

Why do we have a factor two?  Observe that if the deformation is small we can write

\begin{align*}
{ds'}^2 - ds^2 
&= (ds' - ds)(ds' + ds) \\
&\approx
 (ds' - ds) 2 ds
\end{align*}

so that we find 

\begin{equation}\label{eqn:continuumL3:50}
\frac{{ds'}^2 - ds^2 }{ds^2}
\approx
\frac{ds' - ds }{ds}
\end{equation}

Suppose for example, that we have a diagonalized strain tensor, then we find

\begin{equation}\label{eqn:continuumL3:70}
{ds'}^2 - ds^2 
= 2 e_{ii} \left(\frac{dx_i}{ds}\right)^2
\end{equation}

so that

\begin{equation}\label{eqn:continuumL3:90}
\frac{
{ds'}^2 - ds^2 
}{ds^2}
= 2 e_{ii} dx_i^2
\end{equation}

Observe that here again we see this factor of two.

If we have a diagonalized strain tensor, the tensor is of the form

\begin{equation}\label{eqn:continuumL3:110}
\begin{bmatrix}
e_{11} & 0 & 0 \\
0 & e_{22} & 0 \\
0 & 0 & e_{33} 
\end{bmatrix}
\end{equation}

we have

\begin{equation}\label{eqn:continuumL3:130}
{dx_i'}^2 - dx_i^2 = 2 e_{ii} dx_i^2
\end{equation}

\begin{equation}\label{eqn:continuumL3:150}
{ds'}^2 = 
(1 + 2 e_{11}) dx_1^2
+(1 + 2 e_{22}) dx_2^2
+(1 + 2 e_{33}) dx_3^2
\end{equation}

\begin{equation}\label{eqn:continuumL3:170}
ds^2 = 
dx_1^2
+dx_2^2
+dx_3^2
\end{equation}

so 

\begin{align}\label{eqn:continuumL3:190}
dx_1' &= \sqrt{1 + 2 e_{11}} dx_1 \sim ( 1 + e_{11}) dx_1 \\
dx_2' &= \sqrt{1 + 2 e_{22}} dx_2 \sim ( 1 + e_{22}) dx_2 \\
dx_3' &= \sqrt{1 + 2 e_{33}} dx_3 \sim ( 1 + e_{33}) dx_3
\end{align}

Observe that the change in the volume element becomes the trace

\begin{equation}\label{eqn:continuumL3:210}
dV' = 
dx_1'
dx_2'
dx_3'
= dV(1 + e_{ii})
\end{equation}

How do we use this?  Suppose that you are given a strain tensor.  This should allow you to compute the stretch in any given direction.

FIXME: find problem and try this.

\section{Stress tensor.}

Reading for this section is \S 2 from the text associated with the prepared notes \cite{landau1960theory}.

We'd like to consider a macroscopic model that contains the net effects of all the internal forces in the object as depicted in figure (\ref{fig:continuumL3:continuumL3fig2})

\begin{figure}[htp]
   \centering
   \includegraphics[totalheight=0.2\textheight]{continuumL3fig2}
   \caption{Internal forces.}\label{fig:continuumL3:continuumL3fig2}
\end{figure}

We will consider a volume big enough that we won't have to consider the individual atomic interactions, only the average effects of those interactions.  Will will look at the force per unit volume on a differential volume element

The total force on the body is 

\begin{equation}\label{eqn:continuumL3:230}
\iiint \BF dV,
\end{equation}

where $\BF$ is the force per unit volume.  We will evaluate this by utilizing the divergence theorem.  Recall that this was

\begin{equation}\label{eqn:continuumL3:250}
\iiint (\spacegrad \cdot \BA) dV
= \iint \BA \cdot d\Bs
\end{equation}

We have a small problem, since we have a non-divergence expression of the force here, and it is not immediately obvious that we can apply the divergence theorem.  We can deal with this by assuming that we can find a vector valued tensor, so that if we take the divergence of this tensor, we end up with the force.  We introduce the quantity

\begin{equation}\label{eqn:continuumL3:270}
\BF = \PD{x_k}{\sigma_{ik}},
\end{equation}

and require this to be a vector.  We can then apply the divergence theorem

\begin{equation}\label{eqn:continuumL3:290}
\iiint \BF dV 
= \iiint \PD{x_k}{\sigma_{ik}} d\Bx^3 
\iint \sigma_{ik} ds_k,
\end{equation}

where $ds_k$ is a surface element.  We identify this tensor

\begin{equation}\label{eqn:continuumL3:310}
\sigma_{ik} = \frac{\text{Force}}{\text{Unit Area}}
\end{equation}

and 

\begin{equation}\label{eqn:continuumL3:330}
f_i = \sigma_{ik} ds_k,
\end{equation}

as the force on the surface element $ds_k$.  In two dimensions this is illustrated in the following figures (\ref{fig:continuumL3:continuumL3fig3})
\begin{figure}[htp]
   \centering
   \includegraphics[totalheight=0.2\textheight]{continuumL3fig3}
   \caption{2D strain tensor.}\label{fig:continuumL3:continuumL3fig3}
\end{figure}

Observe that we use the index $i$ above as the direction of the force, and index $k$ as the direction normal to the surface.

Note that the strain tensor has the matrix form

\begin{equation}\label{eqn:continuumL3:350}
\begin{bmatrix}
\sigma_{11} & \sigma_{12} & \sigma_{13} \\
\sigma_{21} & \sigma_{22} & \sigma_{23} \\
\sigma_{31} & \sigma_{32} & \sigma_{33}
\end{bmatrix}
\end{equation}

We will show later that this tensor is in fact symmetric.

FIXME: given some 3D forces, compute the stress tensor that is associated with it.

\subsection{Examples of the stress tensor}

\subsubsection{Example 1.  stretch in two opposing directions.}

\begin{figure}[htp]
   \centering
   \includegraphics[totalheight=0.2\textheight]{continuumL3fig4}
   \caption{Opposing stresses in one direction.}\label{fig:continuumL3:continuumL3fig4}
\end{figure}

Here, as illustrated in figure (\ref{fig:continuumL3:continuumL3fig4}), the associated (2D) stress tensor takes the simple form

\begin{equation}\label{eqn:continuumL3:370}
\begin{bmatrix}
\sigma_{11} & 0 \\
0 & 0
\end{bmatrix}
\end{equation}

\subsubsection{Example 2.  stretch in a pair of mutually perpendicular directions}

For a pair of perpendicular forces applied in two dimensions, as illustrated in figure (\ref{fig:continuumL3:continuumL3fig5})
\begin{figure}[htp]
   \centering
   \includegraphics[totalheight=0.2\textheight]{continuumL3fig5}
   \caption{Mutually perpendicular forces}\label{fig:continuumL3:continuumL3fig5}
\end{figure}

our stress tensor now just takes the form

\begin{equation}\label{eqn:continuumL3:390}
\begin{bmatrix}
\sigma_{11} & 0 \\
0 & \sigma_{22}
\end{bmatrix}
\end{equation}

It's easy to imagine now how to get some more general stress tensors, should we make a change of basis that rotates our frame.

\subsubsection{Example 3.  radial stretch}

Suppose we have a fire fighter's safety net, used to catch somebody jumping from a burning building (do they ever do that outside of movies?), as in figure (\ref{fig:continuumL3:continuumL3fig6}).  Each of the firefighters contributes to the stretch.  

\begin{figure}[htp]
   \centering
   \includegraphics[totalheight=0.2\textheight]{continuumL3fig6}
   \caption{Radial forces.}\label{fig:continuumL3:continuumL3fig6}
\end{figure}

FIXME: what form would the tensor take for this?  Would we have to use a radial form of the tensor?  What would that be?

\EndArticle
