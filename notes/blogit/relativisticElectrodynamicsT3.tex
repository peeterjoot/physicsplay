%
% Copyright � 2015 Peeter Joot.  All Rights Reserved.
% Licenced as described in the file LICENSE under the root directory of this GIT repository.
%
\documentclass[]{eliblog}

\usepackage{amsmath}
\usepackage{mathpazo}

%
% shorthand for bold symbols, convenient for vectors and matrices
%
\newcommand{\Ba}[0]{\mathbf{a}}
\newcommand{\Bb}[0]{\mathbf{b}}
\newcommand{\Bc}[0]{\mathbf{c}}
\newcommand{\Bd}[0]{\mathbf{d}}
\newcommand{\Be}[0]{\mathbf{e}}
\newcommand{\Bf}[0]{\mathbf{f}}
\newcommand{\Bg}[0]{\mathbf{g}}
\newcommand{\Bh}[0]{\mathbf{h}}
\newcommand{\Bi}[0]{\mathbf{i}}
\newcommand{\Bj}[0]{\mathbf{j}}
\newcommand{\Bk}[0]{\mathbf{k}}
\newcommand{\Bl}[0]{\mathbf{l}}
\newcommand{\Bm}[0]{\mathbf{m}}
\newcommand{\Bn}[0]{\mathbf{n}}
\newcommand{\Bo}[0]{\mathbf{o}}
\newcommand{\Bp}[0]{\mathbf{p}}
\newcommand{\Bq}[0]{\mathbf{q}}
\newcommand{\Br}[0]{\mathbf{r}}
\newcommand{\Bs}[0]{\mathbf{s}}
\newcommand{\Bt}[0]{\mathbf{t}}
\newcommand{\Bu}[0]{\mathbf{u}}
\newcommand{\Bv}[0]{\mathbf{v}}
\newcommand{\Bw}[0]{\mathbf{w}}
\newcommand{\Bx}[0]{\mathbf{x}}
\newcommand{\By}[0]{\mathbf{y}}
\newcommand{\Bz}[0]{\mathbf{z}}
\newcommand{\BA}[0]{\mathbf{A}}
\newcommand{\BB}[0]{\mathbf{B}}
\newcommand{\BC}[0]{\mathbf{C}}
\newcommand{\BD}[0]{\mathbf{D}}
\newcommand{\BE}[0]{\mathbf{E}}
\newcommand{\BF}[0]{\mathbf{F}}
\newcommand{\BG}[0]{\mathbf{G}}
\newcommand{\BH}[0]{\mathbf{H}}
\newcommand{\BI}[0]{\mathbf{I}}
\newcommand{\BJ}[0]{\mathbf{J}}
\newcommand{\BK}[0]{\mathbf{K}}
\newcommand{\BL}[0]{\mathbf{L}}
\newcommand{\BM}[0]{\mathbf{M}}
\newcommand{\BN}[0]{\mathbf{N}}
\newcommand{\BO}[0]{\mathbf{O}}
\newcommand{\BP}[0]{\mathbf{P}}
\newcommand{\BQ}[0]{\mathbf{Q}}
\newcommand{\BR}[0]{\mathbf{R}}
\newcommand{\BS}[0]{\mathbf{S}}
\newcommand{\BT}[0]{\mathbf{T}}
\newcommand{\BU}[0]{\mathbf{U}}
\newcommand{\BV}[0]{\mathbf{V}}
\newcommand{\BW}[0]{\mathbf{W}}
\newcommand{\BX}[0]{\mathbf{X}}
\newcommand{\BY}[0]{\mathbf{Y}}
\newcommand{\BZ}[0]{\mathbf{Z}}

\newcommand{\Bzero}[0]{\mathbf{0}}
\newcommand{\Btheta}[0]{\boldsymbol{\theta}}
\newcommand{\Btau}[0]{\boldsymbol{\tau}}
\newcommand{\Bomega}[0]{\boldsymbol{\omega}}

%
% shorthand for unit vectors
%
\newcommand{\acap}[0]{\hat{\Ba}}
\newcommand{\bcap}[0]{\hat{\Bb}}
\newcommand{\ccap}[0]{\hat{\Bc}}
\newcommand{\dcap}[0]{\hat{\Bd}}
\newcommand{\ecap}[0]{\hat{\Be}}
\newcommand{\fcap}[0]{\hat{\Bf}}
\newcommand{\gcap}[0]{\hat{\Bg}}
\newcommand{\hcap}[0]{\hat{\Bh}}
\newcommand{\icap}[0]{\hat{\Bi}}
\newcommand{\jcap}[0]{\hat{\Bj}}
\newcommand{\kcap}[0]{\hat{\Bk}}
\newcommand{\lcap}[0]{\hat{\Bl}}
\newcommand{\mcap}[0]{\hat{\Bm}}
\newcommand{\ncap}[0]{\hat{\Bn}}
\newcommand{\ocap}[0]{\hat{\Bo}}
\newcommand{\pcap}[0]{\hat{\Bp}}
\newcommand{\qcap}[0]{\hat{\Bq}}
\newcommand{\rcap}[0]{\hat{\Br}}
\newcommand{\scap}[0]{\hat{\Bs}}
\newcommand{\tcap}[0]{\hat{\Bt}}
\newcommand{\ucap}[0]{\hat{\Bu}}
\newcommand{\vcap}[0]{\hat{\Bv}}
\newcommand{\wcap}[0]{\hat{\Bw}}
\newcommand{\xcap}[0]{\hat{\Bx}}
\newcommand{\ycap}[0]{\hat{\By}}
\newcommand{\zcap}[0]{\hat{\Bz}}
\newcommand{\thetacap}[0]{\hat{\Btheta}}

%
% to write R^n and C^n in a distinguishable fashion.  Perhaps change this
% to the double lined characters upon figuring out how to do so.
%
\newcommand{\C}[1]{$\mathbb{C}^{#1}$}
\newcommand{\R}[1]{$\mathbb{R}^{#1}$}

%
% various generally useful helpers
%

% derivative of #1 wrt. #2:
\newcommand{\D}[2] {\frac {d#2} {d#1}}

\newcommand{\inv}[1]{\frac{1}{#1}}
\newcommand{\cross}[0]{\times}

\newcommand{\abs}[1]{\lvert{#1}\rvert}
\newcommand{\norm}[1]{\lVert{#1}\rVert}
\newcommand{\innerprod}[2]{\langle{#1}, {#2}\rangle}
\newcommand{\dotprod}[2]{{#1} \cdot {#2}}
\newcommand{\bdotprod}[2]{\left({#1} \cdot {#2}\right)}
\newcommand{\crossprod}[2]{{#1} \cross {#2}}
\newcommand{\tripleprod}[3]{\dotprod{\left(\crossprod{#1}{#2}\right)}{#3}}

\DeclareMathOperator{\Proj}{Proj}
\DeclareMathOperator{\Span}{span}
\DeclareMathOperator{\Sgn}{sgn}
\DeclareMathOperator{\Area}{Area}
\DeclareMathOperator{\Volume}{Volume}

%
% A few miscellaneous things specific to this document
%
\newcommand{\crossop}[1]{\crossprod{#1}{}}

% R2 vector.
\newcommand{\VectorTwo}[2]{
\begin{bmatrix}
 {#1} \\
 {#2}
\end{bmatrix}
}

\newcommand{\VectorN}[1]{
\begin{bmatrix}
{#1}_1 \\
{#1}_2 \\
\vdots \\
{#1}_N \\
\end{bmatrix}
}

\newcommand{\DETuvij}[4]{
\begin{vmatrix}
 {#1}_{#3} & {#1}_{#4} \\
 {#2}_{#3} & {#2}_{#4}
\end{vmatrix}
}

\newcommand{\DETuvwijk}[6]{
\begin{vmatrix}
 {#1}_{#4} & {#1}_{#5} & {#1}_{#6} \\
 {#2}_{#4} & {#2}_{#5} & {#2}_{#6} \\
 {#3}_{#4} & {#3}_{#5} & {#3}_{#6}
\end{vmatrix}
}

\newcommand{\DETuvwxijkl}[8]{
\begin{vmatrix}
 {#1}_{#5} & {#1}_{#6} & {#1}_{#7} & {#1}_{#8} \\
 {#2}_{#5} & {#2}_{#6} & {#2}_{#7} & {#2}_{#8} \\
 {#3}_{#5} & {#3}_{#6} & {#3}_{#7} & {#3}_{#8} \\
 {#4}_{#5} & {#4}_{#6} & {#4}_{#7} & {#4}_{#8} \\
\end{vmatrix}
}

%\newcommand{\DETuvwxyijklm}[10]{
%\begin{vmatrix}
% {#1}_{#6} & {#1}_{#7} & {#1}_{#8} & {#1}_{#9} & {#1}_{#10} \\
% {#2}_{#6} & {#2}_{#7} & {#2}_{#8} & {#2}_{#9} & {#2}_{#10} \\
% {#3}_{#6} & {#3}_{#7} & {#3}_{#8} & {#3}_{#9} & {#3}_{#10} \\
% {#4}_{#6} & {#4}_{#7} & {#4}_{#8} & {#4}_{#9} & {#4}_{#10} \\
% {#5}_{#6} & {#5}_{#7} & {#5}_{#8} & {#5}_{#9} & {#5}_{#10}
%\end{vmatrix}
%}

% R3 vector.
\newcommand{\VectorThree}[3]{
\begin{bmatrix}
 {#1} \\
 {#2} \\
 {#3}
\end{bmatrix}
}



\author{Peeter Joot}
\email{peeter.joot@gmail.com}

%\documentclass[]{eliblogwidescreen}

\usepackage{amsmath}
\usepackage{mathpazo}

%
% shorthand for bold symbols, convenient for vectors and matrices
%
\newcommand{\Ba}[0]{\mathbf{a}}
\newcommand{\Bb}[0]{\mathbf{b}}
\newcommand{\Bc}[0]{\mathbf{c}}
\newcommand{\Bd}[0]{\mathbf{d}}
\newcommand{\Be}[0]{\mathbf{e}}
\newcommand{\Bf}[0]{\mathbf{f}}
\newcommand{\Bg}[0]{\mathbf{g}}
\newcommand{\Bh}[0]{\mathbf{h}}
\newcommand{\Bi}[0]{\mathbf{i}}
\newcommand{\Bj}[0]{\mathbf{j}}
\newcommand{\Bk}[0]{\mathbf{k}}
\newcommand{\Bl}[0]{\mathbf{l}}
\newcommand{\Bm}[0]{\mathbf{m}}
\newcommand{\Bn}[0]{\mathbf{n}}
\newcommand{\Bo}[0]{\mathbf{o}}
\newcommand{\Bp}[0]{\mathbf{p}}
\newcommand{\Bq}[0]{\mathbf{q}}
\newcommand{\Br}[0]{\mathbf{r}}
\newcommand{\Bs}[0]{\mathbf{s}}
\newcommand{\Bt}[0]{\mathbf{t}}
\newcommand{\Bu}[0]{\mathbf{u}}
\newcommand{\Bv}[0]{\mathbf{v}}
\newcommand{\Bw}[0]{\mathbf{w}}
\newcommand{\Bx}[0]{\mathbf{x}}
\newcommand{\By}[0]{\mathbf{y}}
\newcommand{\Bz}[0]{\mathbf{z}}
\newcommand{\BA}[0]{\mathbf{A}}
\newcommand{\BB}[0]{\mathbf{B}}
\newcommand{\BC}[0]{\mathbf{C}}
\newcommand{\BD}[0]{\mathbf{D}}
\newcommand{\BE}[0]{\mathbf{E}}
\newcommand{\BF}[0]{\mathbf{F}}
\newcommand{\BG}[0]{\mathbf{G}}
\newcommand{\BH}[0]{\mathbf{H}}
\newcommand{\BI}[0]{\mathbf{I}}
\newcommand{\BJ}[0]{\mathbf{J}}
\newcommand{\BK}[0]{\mathbf{K}}
\newcommand{\BL}[0]{\mathbf{L}}
\newcommand{\BM}[0]{\mathbf{M}}
\newcommand{\BN}[0]{\mathbf{N}}
\newcommand{\BO}[0]{\mathbf{O}}
\newcommand{\BP}[0]{\mathbf{P}}
\newcommand{\BQ}[0]{\mathbf{Q}}
\newcommand{\BR}[0]{\mathbf{R}}
\newcommand{\BS}[0]{\mathbf{S}}
\newcommand{\BT}[0]{\mathbf{T}}
\newcommand{\BU}[0]{\mathbf{U}}
\newcommand{\BV}[0]{\mathbf{V}}
\newcommand{\BW}[0]{\mathbf{W}}
\newcommand{\BX}[0]{\mathbf{X}}
\newcommand{\BY}[0]{\mathbf{Y}}
\newcommand{\BZ}[0]{\mathbf{Z}}

\newcommand{\Bzero}[0]{\mathbf{0}}
\newcommand{\Btheta}[0]{\boldsymbol{\theta}}
\newcommand{\Btau}[0]{\boldsymbol{\tau}}
\newcommand{\Bomega}[0]{\boldsymbol{\omega}}

%
% shorthand for unit vectors
%
\newcommand{\acap}[0]{\hat{\Ba}}
\newcommand{\bcap}[0]{\hat{\Bb}}
\newcommand{\ccap}[0]{\hat{\Bc}}
\newcommand{\dcap}[0]{\hat{\Bd}}
\newcommand{\ecap}[0]{\hat{\Be}}
\newcommand{\fcap}[0]{\hat{\Bf}}
\newcommand{\gcap}[0]{\hat{\Bg}}
\newcommand{\hcap}[0]{\hat{\Bh}}
\newcommand{\icap}[0]{\hat{\Bi}}
\newcommand{\jcap}[0]{\hat{\Bj}}
\newcommand{\kcap}[0]{\hat{\Bk}}
\newcommand{\lcap}[0]{\hat{\Bl}}
\newcommand{\mcap}[0]{\hat{\Bm}}
\newcommand{\ncap}[0]{\hat{\Bn}}
\newcommand{\ocap}[0]{\hat{\Bo}}
\newcommand{\pcap}[0]{\hat{\Bp}}
\newcommand{\qcap}[0]{\hat{\Bq}}
\newcommand{\rcap}[0]{\hat{\Br}}
\newcommand{\scap}[0]{\hat{\Bs}}
\newcommand{\tcap}[0]{\hat{\Bt}}
\newcommand{\ucap}[0]{\hat{\Bu}}
\newcommand{\vcap}[0]{\hat{\Bv}}
\newcommand{\wcap}[0]{\hat{\Bw}}
\newcommand{\xcap}[0]{\hat{\Bx}}
\newcommand{\ycap}[0]{\hat{\By}}
\newcommand{\zcap}[0]{\hat{\Bz}}
\newcommand{\thetacap}[0]{\hat{\Btheta}}

%
% to write R^n and C^n in a distinguishable fashion.  Perhaps change this
% to the double lined characters upon figuring out how to do so.
%
\newcommand{\C}[1]{$\mathbb{C}^{#1}$}
\newcommand{\R}[1]{$\mathbb{R}^{#1}$}

%
% various generally useful helpers
%

% derivative of #1 wrt. #2:
\newcommand{\D}[2] {\frac {d#2} {d#1}}

\newcommand{\inv}[1]{\frac{1}{#1}}
\newcommand{\cross}[0]{\times}

\newcommand{\abs}[1]{\lvert{#1}\rvert}
\newcommand{\norm}[1]{\lVert{#1}\rVert}
\newcommand{\innerprod}[2]{\langle{#1}, {#2}\rangle}
\newcommand{\dotprod}[2]{{#1} \cdot {#2}}
\newcommand{\bdotprod}[2]{\left({#1} \cdot {#2}\right)}
\newcommand{\crossprod}[2]{{#1} \cross {#2}}
\newcommand{\tripleprod}[3]{\dotprod{\left(\crossprod{#1}{#2}\right)}{#3}}

\DeclareMathOperator{\Proj}{Proj}
\DeclareMathOperator{\Span}{span}
\DeclareMathOperator{\Sgn}{sgn}
\DeclareMathOperator{\Area}{Area}
\DeclareMathOperator{\Volume}{Volume}

%
% A few miscellaneous things specific to this document
%
\newcommand{\crossop}[1]{\crossprod{#1}{}}

% R2 vector.
\newcommand{\VectorTwo}[2]{
\begin{bmatrix}
 {#1} \\
 {#2}
\end{bmatrix}
}

\newcommand{\VectorN}[1]{
\begin{bmatrix}
{#1}_1 \\
{#1}_2 \\
\vdots \\
{#1}_N \\
\end{bmatrix}
}

\newcommand{\DETuvij}[4]{
\begin{vmatrix}
 {#1}_{#3} & {#1}_{#4} \\
 {#2}_{#3} & {#2}_{#4}
\end{vmatrix}
}

\newcommand{\DETuvwijk}[6]{
\begin{vmatrix}
 {#1}_{#4} & {#1}_{#5} & {#1}_{#6} \\
 {#2}_{#4} & {#2}_{#5} & {#2}_{#6} \\
 {#3}_{#4} & {#3}_{#5} & {#3}_{#6}
\end{vmatrix}
}

\newcommand{\DETuvwxijkl}[8]{
\begin{vmatrix}
 {#1}_{#5} & {#1}_{#6} & {#1}_{#7} & {#1}_{#8} \\
 {#2}_{#5} & {#2}_{#6} & {#2}_{#7} & {#2}_{#8} \\
 {#3}_{#5} & {#3}_{#6} & {#3}_{#7} & {#3}_{#8} \\
 {#4}_{#5} & {#4}_{#6} & {#4}_{#7} & {#4}_{#8} \\
\end{vmatrix}
}

%\newcommand{\DETuvwxyijklm}[10]{
%\begin{vmatrix}
% {#1}_{#6} & {#1}_{#7} & {#1}_{#8} & {#1}_{#9} & {#1}_{#10} \\
% {#2}_{#6} & {#2}_{#7} & {#2}_{#8} & {#2}_{#9} & {#2}_{#10} \\
% {#3}_{#6} & {#3}_{#7} & {#3}_{#8} & {#3}_{#9} & {#3}_{#10} \\
% {#4}_{#6} & {#4}_{#7} & {#4}_{#8} & {#4}_{#9} & {#4}_{#10} \\
% {#5}_{#6} & {#5}_{#7} & {#5}_{#8} & {#5}_{#9} & {#5}_{#10}
%\end{vmatrix}
%}

% R3 vector.
\newcommand{\VectorThree}[3]{
\begin{bmatrix}
 {#1} \\
 {#2} \\
 {#3}
\end{bmatrix}
}



\author{Peeter Joot}
\email{peeter.joot@gmail.com}


\chapter{PHY450H1S.  Relativistic Electrodynamics Tutorial 3 (TA: Simon Freedman).  XXX FIXME TITLE XXX.}
\label{chap:relativisticElectrodynamicsT3}
%\useCCL
\blogpage{http://sites.google.com/site/peeterjoot/math2011/relativisticElectrodynamicsT3.pdf}
\date{Feb 3, 2011}
\revisionInfo{relativisticElectrodynamicsT3.tex}

\beginArtWithToc
%\beginArtNoToc

\section{Motion in an constant uniform Electric field.}

Given
\begin{equation}\label{eqn:relativisticElectrodynamicsT3:n}
\BE = E \xcap,
\end{equation}

We want to solve the problem
\begin{equation}\label{eqn:relativisticElectrodynamicsT3:n}
\BF = \frac{d\Bp}{dt} =
e ( \BE + \frac{\Bv}{c} \cross \BH ) = e ( \BE ).
\end{equation}

Unlike second year classical physics, we will use relativistic momentum, so for only a constant electric field, our Lorentz force equation to solve becomes

\begin{equation}\label{eqn:relativisticElectrodynamicsT3:n}
\frac{d\Bp}{dt} = \frac{d m \gamma \Bv}{dt} = e \BE.
\end{equation}

In components this is

\begin{align}\label{eqn:relativisticElectrodynamicsT3:n}
\pdot_x &= e E \\
\pdot_y &= \text{constant}
\end{align}

\begin{align*}
e E t + p_x(0)
&=
\frac{m \xdot}{\sqrt{1 - (\xdot^2 + \ydot^2)/c^2}} = e E t \\
&= 
\frac{m^2}{(e E t)^2} \xdot^2 = 1 - \frac{\xdot^2 - \ydot^2}{c^2}
\end{align*}

where we let $p_x(0) = 0$

\begin{equation}\label{eqn:relativisticElectrodynamicsT3:n}
\xdot^2 = \frac{c^2 - \ydot^2}{1 + (\frac{mc}{eEt})^2}
\end{equation}

With $p_y(0) = p_0$

\begin{align*}
\frac{m \ydot}{\sqrt{1 - (\xdot^2 + \ydot^2)/c^2}} &= p_0 \\
\implies \\
\frac{c^2 m^2}{p_0^2} \ydot^2 &= c^2 - \xdot^2 - \ydot^2 \\
\implies \\
\ydot^2 &= \frac{ p_0^2 (c^2 - \xdot^2)}{1 + \frac{m^2 c^2}{p_0^2}}
\end{align*}

Observe that 

\begin{equation}\label{eqn:relativisticElectrodynamicsT3:n}
\mathcal{E}^2 = p_0^2 c^2 + m^2 c^4
\end{equation}

so we can write

\begin{equation}\label{eqn:relativisticElectrodynamicsT3:n}
\ydot^2 = \frac{ c^2 p_0^2 (c^2 - \xdot^2)}{ \mathcal{E}_0^2 }
\end{equation}

Substitution yields

\begin{align}\label{eqn:relativisticElectrodynamicsT3:n}
\xdot &= \frac{c^2 e \mathcal{E} t}{\sqrt{ \mathcal{E}_0^2 + (e c \mathcal{E} t)^2 }} \\
\ydot &= \frac{c p_0^2 }{\sqrt{ \mathcal{E}_0^2 + (e c \mathcal{E} t)^2 }}
\end{align}

There's also a tricky way (as in the text), with 

\begin{align}\label{eqn:relativisticElectrodynamicsT3:n}
\Bp &= m \gamma \Bv  \\
\mathcal{E} &= \gamma m c^2 
\end{align}

Solve for $\Bp$ and get

\begin{equation}\label{eqn:relativisticElectrodynamicsT3:n}
\Bp = \frac{\mathcal{E} \Bv}{c^2},
\end{equation}

which implies

\begin{equation}\label{eqn:relativisticElectrodynamicsT3:n}
\xdot = \frac{c^2 p_x}{\mathcal{E}}.
\end{equation}

Solving for $x$ we have

\begin{equation}\label{eqn:relativisticElectrodynamicsT3:n}
x(t) = c^2 e E \int \frac{dt' t'}{\sqrt{ \mathcal{E}_0^2 + (e c E t)^2 }}
\end{equation}

Can solve with hyperbolic substitution or

\begin{equation}\label{eqn:relativisticElectrodynamicsT3:n}
x(t) = c^2 e E \int \frac{dt' t'}{\sqrt{ \mathcal{E}_0^2 + (e c E t)^2 }}
\end{equation}

\begin{equation}\label{eqn:relativisticElectrodynamicsT3:n}
du^2 = 2 u du \implies u du = \inv{2} d(u^2)
\end{equation}

\begin{equation}\label{eqn:relativisticElectrodynamicsT3:n}
x(t) = \frac{c^2 e E}{2 \mathcal{E}_0} \int \frac{d (u^2)}{\sqrt{ 1 + \left(\frac{e c E}{\mathcal{E}_0}\right)^2 u^2 }}
\end{equation}

Solving this we get

\begin{equation}\label{eqn:relativisticElectrodynamicsT3:n}
x(t) = \frac{e E}{\sqrt{ \mathcal{E}_0^2 + (e c E t)^2 }}
\end{equation}

or

\begin{equation}\label{eqn:relativisticElectrodynamicsT3:n}
x^2 - c^2 t^2 = \frac{E_0^2}{ e^2 E^2 } = a^{-2}
\end{equation}

\begin{equation}\label{eqn:relativisticElectrodynamicsT3:n}
y(t) = c^2 p_0^2 \int \frac{dt}{ \sqrt{\mathcal{E}_0^2 + (e c E t)^2 }}
\end{equation}

\begin{equation}\label{eqn:relativisticElectrodynamicsT3:n}
t = \frac{E_0}{ e c E} \sinh(u) 
\end{equation}

\begin{equation}\label{eqn:relativisticElectrodynamicsT3:n}
dt = \frac{E_0}{ e c E} \cosh(u) du
\end{equation}

\begin{align*}
y(t) 
&= \frac{c^2 p_0^2}{\mathcal{E}_0} \int \frac{dt}{\sqrt{1 + (\frac{e c E}{\mathcal{E}_0)})^2 t^2 }} \\
&= \frac{c^2 p_0^2}{\mathcal{E}_0} 
\frac{E_0}{ e c E} \cosh(u) du
\int \frac{ du \cosh u }{\sqrt{1 + (\sinh u)^2 }} \\
&= \frac{c p_0}{ e E} u 
\end{align*}

for

\begin{equation}\label{eqn:relativisticElectrodynamicsT3:n}
\boxed{
\begin{aligned}
y(t) &= \frac{c p_0}{ e E} \sinh^{-1} \left( \frac{e c E t}{\mathcal{E}_0} \right) \\
x(y) &= \frac{\mathcal{E}_0}{c E} \cosh( \frac{y e E }{ c p_0} )
\end{aligned}
}
\end{equation}

Checks:
\begin{equation}\label{eqn:relativisticElectrodynamicsT3:n}
v \rightarrow c, t \rightarrow \infty
\end{equation}

\begin{equation}\label{eqn:relativisticElectrodynamicsT3:n}
v << c, t \rightarrow 0
\end{equation}

\begin{align*}
m v_x &= e E t + ...
x &\sim t^2 
\end{align*}

\begin{equation}\label{eqn:relativisticElectrodynamicsT3:n}
m v_y = p_0 \rightarrow y \sim t
\end{equation}

\begin{equation}\label{eqn:relativisticElectrodynamicsT3:n}
x(y) \sim y^2
\end{equation}

(a parabola)

\section{Motion in an constant uniform Magnetic field.}

Note that the magnetic field does no work

\begin{equation}\label{eqn:relativisticElectrodynamicsT3:n}
\BF = \frac{e}{c} \Bv \cross \BH
\end{equation}

\begin{align*}
dW = 
\BF \cdot d\Bl
&=
\frac{e}{c} (\Bv \cross \BH) \cdot d\Bl
&= 0
\end{align*}

(perpendicular)

So our energy is only the initial time value

\begin{equation}\label{eqn:relativisticElectrodynamicsT3:n}
\mathcal{E} = .... + e A^0
\end{equation}

\begin{equation}\label{eqn:relativisticElectrodynamicsT3:n}
\mathcal{E}(t) = \mathcal{E}_0
\end{equation}

\begin{equation}\label{eqn:relativisticElectrodynamicsT3:n}
\BH = H \zcap
\end{equation}

\begin{equation}\label{eqn:relativisticElectrodynamicsT3:n}
\delta_{\alpha 3} H = H_\alpha
\end{equation}

\begin{equation}\label{eqn:relativisticElectrodynamicsT3:n}
\Bp = \mathcal{E} \frac{\Bv}{c^2} = \mathcal{E}_0 \frac{\Bv}{c^2}
\end{equation}

implies

\begin{equation}\label{eqn:relativisticElectrodynamicsT3:n}
\Bv = \Bp \frac{c^2}{\mathcal{E}_0}
\end{equation}

\begin{equation}\label{eqn:relativisticElectrodynamicsT3:n}
\dot{\Bv} = \dot{\Bp} \frac{c^2}{\mathcal{E}_0}
\end{equation}

\begin{equation}\label{eqn:relativisticElectrodynamicsT3:n}
\vdot_\alpha = \frac{e c}{\mathcal{E}_0} \epsilon_{\alpha \beta \gamma} v^\beta H_\gamma
\end{equation}

write

\begin{equation}\label{eqn:relativisticElectrodynamicsT3:n}
\omega = \frac{e c H}{\mathcal{E}_0}
\end{equation}

Evaluating the delta 
\begin{equation}\label{eqn:relativisticElectrodynamicsT3:n}
\vdot_\alpha = \omega \epsilon_{\alpha \beta 3} v_\beta 
\end{equation}
FIXME: SHOW.

\begin{align}\label{eqn:relativisticElectrodynamicsT3:n}
\vdot_1 &= \omega \epsilon_{1 \beta 3} v_\beta = \omega v_2 \\
\vdot_2 &= \omega \epsilon_{2 \beta 3} v_\beta = - \omega v_1 \\
\vdot_3 &= \omega \epsilon_{3 \beta 3} v_\beta = 0
\end{align}

Looks like circular motion, so it's natural to use complex variables.  With

\begin{equation}\label{eqn:relativisticElectrodynamicsT3:n}
z = v_1 + i v_2 
\end{equation}

We can show
FIXME: do this:

\begin{equation}\label{eqn:relativisticElectrodynamicsT3:n}
ddt ( v_1 + i v_2 ) = -i \omega ( v_1 + i v_2 ),
\end{equation}

for
\begin{equation}\label{eqn:relativisticElectrodynamicsT3:n}
z = V_0 e^{-i \omega z t + i \alpha}
\end{equation}

Real and imaginary parts

\begin{align}\label{eqn:relativisticElectrodynamicsT3:n}
v_1(t) &= V_0 \cos( \omega z t + \alpha) \\
v_2(t) &= -V_0 \sin( \omega z t + \alpha)
\end{align}

Integrating
\begin{align}\label{eqn:relativisticElectrodynamicsT3:n}
x_1(t) &= x_1(0) + V_0 \sin( \omega z t + \alpha) \\
x_2(t) &= x_2(0) + V_0 \cos( \omega z t + \alpha)
\end{align}

Which is a helix.
FIXME: PICTURE.

%\EndArticle
\EndNoBibArticle
