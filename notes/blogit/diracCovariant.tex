%
% Copyright � 2015 Peeter Joot.  All Rights Reserved.
% Licenced as described in the file LICENSE under the root directory of this GIT repository.
%
\documentclass[]{eliblog}

\usepackage{amsmath}
\usepackage{mathpazo}

%
% shorthand for bold symbols, convenient for vectors and matrices
%
\newcommand{\Ba}[0]{\mathbf{a}}
\newcommand{\Bb}[0]{\mathbf{b}}
\newcommand{\Bc}[0]{\mathbf{c}}
\newcommand{\Bd}[0]{\mathbf{d}}
\newcommand{\Be}[0]{\mathbf{e}}
\newcommand{\Bf}[0]{\mathbf{f}}
\newcommand{\Bg}[0]{\mathbf{g}}
\newcommand{\Bh}[0]{\mathbf{h}}
\newcommand{\Bi}[0]{\mathbf{i}}
\newcommand{\Bj}[0]{\mathbf{j}}
\newcommand{\Bk}[0]{\mathbf{k}}
\newcommand{\Bl}[0]{\mathbf{l}}
\newcommand{\Bm}[0]{\mathbf{m}}
\newcommand{\Bn}[0]{\mathbf{n}}
\newcommand{\Bo}[0]{\mathbf{o}}
\newcommand{\Bp}[0]{\mathbf{p}}
\newcommand{\Bq}[0]{\mathbf{q}}
\newcommand{\Br}[0]{\mathbf{r}}
\newcommand{\Bs}[0]{\mathbf{s}}
\newcommand{\Bt}[0]{\mathbf{t}}
\newcommand{\Bu}[0]{\mathbf{u}}
\newcommand{\Bv}[0]{\mathbf{v}}
\newcommand{\Bw}[0]{\mathbf{w}}
\newcommand{\Bx}[0]{\mathbf{x}}
\newcommand{\By}[0]{\mathbf{y}}
\newcommand{\Bz}[0]{\mathbf{z}}
\newcommand{\BA}[0]{\mathbf{A}}
\newcommand{\BB}[0]{\mathbf{B}}
\newcommand{\BC}[0]{\mathbf{C}}
\newcommand{\BD}[0]{\mathbf{D}}
\newcommand{\BE}[0]{\mathbf{E}}
\newcommand{\BF}[0]{\mathbf{F}}
\newcommand{\BG}[0]{\mathbf{G}}
\newcommand{\BH}[0]{\mathbf{H}}
\newcommand{\BI}[0]{\mathbf{I}}
\newcommand{\BJ}[0]{\mathbf{J}}
\newcommand{\BK}[0]{\mathbf{K}}
\newcommand{\BL}[0]{\mathbf{L}}
\newcommand{\BM}[0]{\mathbf{M}}
\newcommand{\BN}[0]{\mathbf{N}}
\newcommand{\BO}[0]{\mathbf{O}}
\newcommand{\BP}[0]{\mathbf{P}}
\newcommand{\BQ}[0]{\mathbf{Q}}
\newcommand{\BR}[0]{\mathbf{R}}
\newcommand{\BS}[0]{\mathbf{S}}
\newcommand{\BT}[0]{\mathbf{T}}
\newcommand{\BU}[0]{\mathbf{U}}
\newcommand{\BV}[0]{\mathbf{V}}
\newcommand{\BW}[0]{\mathbf{W}}
\newcommand{\BX}[0]{\mathbf{X}}
\newcommand{\BY}[0]{\mathbf{Y}}
\newcommand{\BZ}[0]{\mathbf{Z}}

\newcommand{\Bzero}[0]{\mathbf{0}}
\newcommand{\Btheta}[0]{\boldsymbol{\theta}}
\newcommand{\Btau}[0]{\boldsymbol{\tau}}
\newcommand{\Bomega}[0]{\boldsymbol{\omega}}

%
% shorthand for unit vectors
%
\newcommand{\acap}[0]{\hat{\Ba}}
\newcommand{\bcap}[0]{\hat{\Bb}}
\newcommand{\ccap}[0]{\hat{\Bc}}
\newcommand{\dcap}[0]{\hat{\Bd}}
\newcommand{\ecap}[0]{\hat{\Be}}
\newcommand{\fcap}[0]{\hat{\Bf}}
\newcommand{\gcap}[0]{\hat{\Bg}}
\newcommand{\hcap}[0]{\hat{\Bh}}
\newcommand{\icap}[0]{\hat{\Bi}}
\newcommand{\jcap}[0]{\hat{\Bj}}
\newcommand{\kcap}[0]{\hat{\Bk}}
\newcommand{\lcap}[0]{\hat{\Bl}}
\newcommand{\mcap}[0]{\hat{\Bm}}
\newcommand{\ncap}[0]{\hat{\Bn}}
\newcommand{\ocap}[0]{\hat{\Bo}}
\newcommand{\pcap}[0]{\hat{\Bp}}
\newcommand{\qcap}[0]{\hat{\Bq}}
\newcommand{\rcap}[0]{\hat{\Br}}
\newcommand{\scap}[0]{\hat{\Bs}}
\newcommand{\tcap}[0]{\hat{\Bt}}
\newcommand{\ucap}[0]{\hat{\Bu}}
\newcommand{\vcap}[0]{\hat{\Bv}}
\newcommand{\wcap}[0]{\hat{\Bw}}
\newcommand{\xcap}[0]{\hat{\Bx}}
\newcommand{\ycap}[0]{\hat{\By}}
\newcommand{\zcap}[0]{\hat{\Bz}}
\newcommand{\thetacap}[0]{\hat{\Btheta}}

%
% to write R^n and C^n in a distinguishable fashion.  Perhaps change this
% to the double lined characters upon figuring out how to do so.
%
\newcommand{\C}[1]{$\mathbb{C}^{#1}$}
\newcommand{\R}[1]{$\mathbb{R}^{#1}$}

%
% various generally useful helpers
%

% derivative of #1 wrt. #2:
\newcommand{\D}[2] {\frac {d#2} {d#1}}

\newcommand{\inv}[1]{\frac{1}{#1}}
\newcommand{\cross}[0]{\times}

\newcommand{\abs}[1]{\lvert{#1}\rvert}
\newcommand{\norm}[1]{\lVert{#1}\rVert}
\newcommand{\innerprod}[2]{\langle{#1}, {#2}\rangle}
\newcommand{\dotprod}[2]{{#1} \cdot {#2}}
\newcommand{\bdotprod}[2]{\left({#1} \cdot {#2}\right)}
\newcommand{\crossprod}[2]{{#1} \cross {#2}}
\newcommand{\tripleprod}[3]{\dotprod{\left(\crossprod{#1}{#2}\right)}{#3}}

\DeclareMathOperator{\Proj}{Proj}
\DeclareMathOperator{\Span}{span}
\DeclareMathOperator{\Sgn}{sgn}
\DeclareMathOperator{\Area}{Area}
\DeclareMathOperator{\Volume}{Volume}

%
% A few miscellaneous things specific to this document
%
\newcommand{\crossop}[1]{\crossprod{#1}{}}

% R2 vector.
\newcommand{\VectorTwo}[2]{
\begin{bmatrix}
 {#1} \\
 {#2}
\end{bmatrix}
}

\newcommand{\VectorN}[1]{
\begin{bmatrix}
{#1}_1 \\
{#1}_2 \\
\vdots \\
{#1}_N \\
\end{bmatrix}
}

\newcommand{\DETuvij}[4]{
\begin{vmatrix}
 {#1}_{#3} & {#1}_{#4} \\
 {#2}_{#3} & {#2}_{#4}
\end{vmatrix}
}

\newcommand{\DETuvwijk}[6]{
\begin{vmatrix}
 {#1}_{#4} & {#1}_{#5} & {#1}_{#6} \\
 {#2}_{#4} & {#2}_{#5} & {#2}_{#6} \\
 {#3}_{#4} & {#3}_{#5} & {#3}_{#6}
\end{vmatrix}
}

\newcommand{\DETuvwxijkl}[8]{
\begin{vmatrix}
 {#1}_{#5} & {#1}_{#6} & {#1}_{#7} & {#1}_{#8} \\
 {#2}_{#5} & {#2}_{#6} & {#2}_{#7} & {#2}_{#8} \\
 {#3}_{#5} & {#3}_{#6} & {#3}_{#7} & {#3}_{#8} \\
 {#4}_{#5} & {#4}_{#6} & {#4}_{#7} & {#4}_{#8} \\
\end{vmatrix}
}

%\newcommand{\DETuvwxyijklm}[10]{
%\begin{vmatrix}
% {#1}_{#6} & {#1}_{#7} & {#1}_{#8} & {#1}_{#9} & {#1}_{#10} \\
% {#2}_{#6} & {#2}_{#7} & {#2}_{#8} & {#2}_{#9} & {#2}_{#10} \\
% {#3}_{#6} & {#3}_{#7} & {#3}_{#8} & {#3}_{#9} & {#3}_{#10} \\
% {#4}_{#6} & {#4}_{#7} & {#4}_{#8} & {#4}_{#9} & {#4}_{#10} \\
% {#5}_{#6} & {#5}_{#7} & {#5}_{#8} & {#5}_{#9} & {#5}_{#10}
%\end{vmatrix}
%}

% R3 vector.
\newcommand{\VectorThree}[3]{
\begin{bmatrix}
 {#1} \\
 {#2} \\
 {#3}
\end{bmatrix}
}



\author{Peeter Joot}
\email{peeter.joot@gmail.com}

%\documentclass[]{eliblogwidescreen}

\usepackage{amsmath}
\usepackage{mathpazo}

%
% shorthand for bold symbols, convenient for vectors and matrices
%
\newcommand{\Ba}[0]{\mathbf{a}}
\newcommand{\Bb}[0]{\mathbf{b}}
\newcommand{\Bc}[0]{\mathbf{c}}
\newcommand{\Bd}[0]{\mathbf{d}}
\newcommand{\Be}[0]{\mathbf{e}}
\newcommand{\Bf}[0]{\mathbf{f}}
\newcommand{\Bg}[0]{\mathbf{g}}
\newcommand{\Bh}[0]{\mathbf{h}}
\newcommand{\Bi}[0]{\mathbf{i}}
\newcommand{\Bj}[0]{\mathbf{j}}
\newcommand{\Bk}[0]{\mathbf{k}}
\newcommand{\Bl}[0]{\mathbf{l}}
\newcommand{\Bm}[0]{\mathbf{m}}
\newcommand{\Bn}[0]{\mathbf{n}}
\newcommand{\Bo}[0]{\mathbf{o}}
\newcommand{\Bp}[0]{\mathbf{p}}
\newcommand{\Bq}[0]{\mathbf{q}}
\newcommand{\Br}[0]{\mathbf{r}}
\newcommand{\Bs}[0]{\mathbf{s}}
\newcommand{\Bt}[0]{\mathbf{t}}
\newcommand{\Bu}[0]{\mathbf{u}}
\newcommand{\Bv}[0]{\mathbf{v}}
\newcommand{\Bw}[0]{\mathbf{w}}
\newcommand{\Bx}[0]{\mathbf{x}}
\newcommand{\By}[0]{\mathbf{y}}
\newcommand{\Bz}[0]{\mathbf{z}}
\newcommand{\BA}[0]{\mathbf{A}}
\newcommand{\BB}[0]{\mathbf{B}}
\newcommand{\BC}[0]{\mathbf{C}}
\newcommand{\BD}[0]{\mathbf{D}}
\newcommand{\BE}[0]{\mathbf{E}}
\newcommand{\BF}[0]{\mathbf{F}}
\newcommand{\BG}[0]{\mathbf{G}}
\newcommand{\BH}[0]{\mathbf{H}}
\newcommand{\BI}[0]{\mathbf{I}}
\newcommand{\BJ}[0]{\mathbf{J}}
\newcommand{\BK}[0]{\mathbf{K}}
\newcommand{\BL}[0]{\mathbf{L}}
\newcommand{\BM}[0]{\mathbf{M}}
\newcommand{\BN}[0]{\mathbf{N}}
\newcommand{\BO}[0]{\mathbf{O}}
\newcommand{\BP}[0]{\mathbf{P}}
\newcommand{\BQ}[0]{\mathbf{Q}}
\newcommand{\BR}[0]{\mathbf{R}}
\newcommand{\BS}[0]{\mathbf{S}}
\newcommand{\BT}[0]{\mathbf{T}}
\newcommand{\BU}[0]{\mathbf{U}}
\newcommand{\BV}[0]{\mathbf{V}}
\newcommand{\BW}[0]{\mathbf{W}}
\newcommand{\BX}[0]{\mathbf{X}}
\newcommand{\BY}[0]{\mathbf{Y}}
\newcommand{\BZ}[0]{\mathbf{Z}}

\newcommand{\Bzero}[0]{\mathbf{0}}
\newcommand{\Btheta}[0]{\boldsymbol{\theta}}
\newcommand{\Btau}[0]{\boldsymbol{\tau}}
\newcommand{\Bomega}[0]{\boldsymbol{\omega}}

%
% shorthand for unit vectors
%
\newcommand{\acap}[0]{\hat{\Ba}}
\newcommand{\bcap}[0]{\hat{\Bb}}
\newcommand{\ccap}[0]{\hat{\Bc}}
\newcommand{\dcap}[0]{\hat{\Bd}}
\newcommand{\ecap}[0]{\hat{\Be}}
\newcommand{\fcap}[0]{\hat{\Bf}}
\newcommand{\gcap}[0]{\hat{\Bg}}
\newcommand{\hcap}[0]{\hat{\Bh}}
\newcommand{\icap}[0]{\hat{\Bi}}
\newcommand{\jcap}[0]{\hat{\Bj}}
\newcommand{\kcap}[0]{\hat{\Bk}}
\newcommand{\lcap}[0]{\hat{\Bl}}
\newcommand{\mcap}[0]{\hat{\Bm}}
\newcommand{\ncap}[0]{\hat{\Bn}}
\newcommand{\ocap}[0]{\hat{\Bo}}
\newcommand{\pcap}[0]{\hat{\Bp}}
\newcommand{\qcap}[0]{\hat{\Bq}}
\newcommand{\rcap}[0]{\hat{\Br}}
\newcommand{\scap}[0]{\hat{\Bs}}
\newcommand{\tcap}[0]{\hat{\Bt}}
\newcommand{\ucap}[0]{\hat{\Bu}}
\newcommand{\vcap}[0]{\hat{\Bv}}
\newcommand{\wcap}[0]{\hat{\Bw}}
\newcommand{\xcap}[0]{\hat{\Bx}}
\newcommand{\ycap}[0]{\hat{\By}}
\newcommand{\zcap}[0]{\hat{\Bz}}
\newcommand{\thetacap}[0]{\hat{\Btheta}}

%
% to write R^n and C^n in a distinguishable fashion.  Perhaps change this
% to the double lined characters upon figuring out how to do so.
%
\newcommand{\C}[1]{$\mathbb{C}^{#1}$}
\newcommand{\R}[1]{$\mathbb{R}^{#1}$}

%
% various generally useful helpers
%

% derivative of #1 wrt. #2:
\newcommand{\D}[2] {\frac {d#2} {d#1}}

\newcommand{\inv}[1]{\frac{1}{#1}}
\newcommand{\cross}[0]{\times}

\newcommand{\abs}[1]{\lvert{#1}\rvert}
\newcommand{\norm}[1]{\lVert{#1}\rVert}
\newcommand{\innerprod}[2]{\langle{#1}, {#2}\rangle}
\newcommand{\dotprod}[2]{{#1} \cdot {#2}}
\newcommand{\bdotprod}[2]{\left({#1} \cdot {#2}\right)}
\newcommand{\crossprod}[2]{{#1} \cross {#2}}
\newcommand{\tripleprod}[3]{\dotprod{\left(\crossprod{#1}{#2}\right)}{#3}}

\DeclareMathOperator{\Proj}{Proj}
\DeclareMathOperator{\Span}{span}
\DeclareMathOperator{\Sgn}{sgn}
\DeclareMathOperator{\Area}{Area}
\DeclareMathOperator{\Volume}{Volume}

%
% A few miscellaneous things specific to this document
%
\newcommand{\crossop}[1]{\crossprod{#1}{}}

% R2 vector.
\newcommand{\VectorTwo}[2]{
\begin{bmatrix}
 {#1} \\
 {#2}
\end{bmatrix}
}

\newcommand{\VectorN}[1]{
\begin{bmatrix}
{#1}_1 \\
{#1}_2 \\
\vdots \\
{#1}_N \\
\end{bmatrix}
}

\newcommand{\DETuvij}[4]{
\begin{vmatrix}
 {#1}_{#3} & {#1}_{#4} \\
 {#2}_{#3} & {#2}_{#4}
\end{vmatrix}
}

\newcommand{\DETuvwijk}[6]{
\begin{vmatrix}
 {#1}_{#4} & {#1}_{#5} & {#1}_{#6} \\
 {#2}_{#4} & {#2}_{#5} & {#2}_{#6} \\
 {#3}_{#4} & {#3}_{#5} & {#3}_{#6}
\end{vmatrix}
}

\newcommand{\DETuvwxijkl}[8]{
\begin{vmatrix}
 {#1}_{#5} & {#1}_{#6} & {#1}_{#7} & {#1}_{#8} \\
 {#2}_{#5} & {#2}_{#6} & {#2}_{#7} & {#2}_{#8} \\
 {#3}_{#5} & {#3}_{#6} & {#3}_{#7} & {#3}_{#8} \\
 {#4}_{#5} & {#4}_{#6} & {#4}_{#7} & {#4}_{#8} \\
\end{vmatrix}
}

%\newcommand{\DETuvwxyijklm}[10]{
%\begin{vmatrix}
% {#1}_{#6} & {#1}_{#7} & {#1}_{#8} & {#1}_{#9} & {#1}_{#10} \\
% {#2}_{#6} & {#2}_{#7} & {#2}_{#8} & {#2}_{#9} & {#2}_{#10} \\
% {#3}_{#6} & {#3}_{#7} & {#3}_{#8} & {#3}_{#9} & {#3}_{#10} \\
% {#4}_{#6} & {#4}_{#7} & {#4}_{#8} & {#4}_{#9} & {#4}_{#10} \\
% {#5}_{#6} & {#5}_{#7} & {#5}_{#8} & {#5}_{#9} & {#5}_{#10}
%\end{vmatrix}
%}

% R3 vector.
\newcommand{\VectorThree}[3]{
\begin{bmatrix}
 {#1} \\
 {#2} \\
 {#3}
\end{bmatrix}
}



\author{Peeter Joot}
\email{peeter.joot@gmail.com}


\chapter{Covariant recovery of spin adjusted Klein-Gordon equation from Dirac's}
\label{chap:diracCovariant}
%\useCCL
\blogpage{http://sites.google.com/site/peeterjoot/math2011/diracCovariant.pdf}
\date{Sept 1, 2011}
\revisionInfo{diracCovariant.tex}

\newcommand{\pslash}[0]{\cancel{p}}
\newcommand{\aslash}[0]{\cancel{a}}
\newcommand{\bslash}[0]{\cancel{b}}
\newcommand{\Dslash}[0]{\cancel{D}}
\newcommand{\Aslash}[0]{\cancel{A}}
\newcommand{\partialslash}[0]{\cancel{\partial}}

\beginArtWithToc
%\beginArtNoToc

\section{Motivation.}

In \S 36.4 of \cite{desai2009quantum} is a covariant treatment of the gauge transformed Dirac equation, with the end goal of finding the Klien-Gordon relation with the spin terms required for electromagnetic phenomina.

Two typos to start things off, and (36.63-64) should be respectively

\begin{align}\label{eqn:diracCovariant:10}
p_\mu &\rightarrow p_\mu - e A_\mu \\
D_\mu &= \partial_\mu + i e A_\mu.
\end{align}

Other than this there are no typos till the end where a factor of two is lost, and we should have

\begin{equation}\label{eqn:diracCovariant:30}
\left( D^\mu D_\mu - \frac{e}{2} \sigma^{\mu \nu} F_{\mu \nu} + m^2 \right) \psi = 0
\end{equation}

It's slightly tempting to re-derive this, with the inclusion of the $\hbar$ and $c$ factors.  However, a bit of play using the Geometric Algebra (GA) operators discussed in \cite{doran2003gap} proves more productive.  This text introduces a purely GA formalism for the Dirac equation which I'll not use here.  Instead I'll use the more conventional Feynman slash notation so that a four vector with coordinates $a^\mu$ is written with its basis as

\begin{equation}\label{eqn:diracCovariant:50}
\aslash = a^\mu \gamma_\mu = a_\mu \gamma^\mu
\end{equation}

\section{Geometric Algebra notation.}

We require the GA dot, and wedge operators

\begin{equation}\label{eqn:diracCovariant:70}
\aslash \cdot \bslash = \inv{2}( \aslash \bslash + \bslash \aslash ) = a^\mu b_\mu
\end{equation}
\begin{equation}\label{eqn:diracCovariant:90}
\aslash \wedge \bslash = \inv{2}( \aslash \bslash - \bslash \aslash ) = \inv{2} a^\mu b^\nu \gamma_{[\mu} \gamma_{\nu]}.
\end{equation}

In contrast to the matrix notation, the product of two identical gamma matrices is written as the unit scalar value (or grade zero product), instead of a using an explicit four by four identity matrix representation.  We similarily label the product of different basis elements, for example $\gamma_0 \gamma_1$ as a grade two element, or bivector.  Thus the dot product is the grade zero term of the multivector product $\aslash \bslash$, and the wedge product is the grade zero term of the same.  More generally, for a product of multivectors $A$ and $B$ the grade selection operator is defined as

\begin{equation}\label{eqn:diracCovariant:110}
\gpgrade{A B}{n}.
\end{equation}

This is an abstract notation encoding the instructions to select just the $n$ grade elements of the multivector product if they exist.  In this notation, the product of vectors splits into scalar and bivector terms which may also be expressed as the dot and wedge products

\begin{equation}\label{eqn:diracCovariant:130}
\aslash \bslash = \gpgrade{\aslash \bslash}{0} + \gpgrade{\aslash \bslash}{2} = \aslash \cdot \bslash + \aslash \wedge \bslash.
\end{equation}

\section{Gauge transforming the Dirac equation.}

Our electron equation is

\begin{equation}\label{eqn:diracCovariant:150}
\left(\pslash - m c \right) \psi = 0
\end{equation}

After a gauge transformation

\begin{equation}\label{eqn:diracCovariant:170}
\pslash \rightarrow \pslash - \frac{e}{c} \Aslash,
\end{equation}

this is

\begin{equation}\label{eqn:diracCovariant:190}
\left(\pslash - \frac{e}{c}\Aslash - m c \right) \psi = 0.
\end{equation}

We left multiply by the conjugate quantity $\pslash - \frac{e}{c}\Aslash + m c$, and our task is now to reduce operator equation

\begin{equation}\label{eqn:diracCovariant:210}
\begin{aligned}
0 
&= \left(\pslash - \frac{e}{c}\Aslash + m c \right) \left(\pslash - \frac{e}{c}\Aslash - m c \right) \psi \\
&= 
\gpgrade{
\left(\pslash - \frac{e}{c}\Aslash + m c \right) \left(\pslash - \frac{e}{c}\Aslash - m c \right) \psi
}{0} \\
&\quad +\gpgradeone{
\left(\pslash - \frac{e}{c}\Aslash + m c \right) \left(\pslash - \frac{e}{c}\Aslash - m c \right) \psi
} \\
&\quad +\gpgradetwo{
\left(\pslash - \frac{e}{c}\Aslash + m c \right) \left(\pslash - \frac{e}{c}\Aslash - m c \right) \psi
}.
\end{aligned}
\end{equation}

We have two contributions for the scalar parts, one is the product of $mc$ scalars, and the other is the dot product of the vectors

\begin{equation}\label{eqn:diracCovariant:230}
\gpgrade{
\left(\pslash - \frac{e}{c}\Aslash + m c \right) \left(\pslash - \frac{e}{c}\Aslash - m c \right) \psi
}{0}
=
\left(\pslash - \frac{e}{c}\Aslash \right) \cdot \left(\pslash - \frac{e}{c}\Aslash \right) \psi
- (m c)^2 \psi.
\end{equation}

The grade one (four vector) components sum to zero since those are only the scalar times four vector portions and have opposing signs

\begin{equation}\label{eqn:diracCovariant:250}
\gpgradeone{
\left(\pslash - \frac{e}{c}\Aslash + m c \right) \left(\pslash - \frac{e}{c}\Aslash - m c \right) \psi
} 
=
\left(\pslash - \frac{e}{c}\Aslash \right) \left( - m c \right) \psi
+ \left( m c \right) \left(\pslash - \frac{e}{c}\Aslash \right) \psi
= 0.
\end{equation}

Only the vector products can contribute to the grade two portion of the multivector product, so retain only the wedges between those

\begin{align*}
\gpgradetwo{
\left(\pslash - \frac{e}{c}\Aslash + m c \right) \left(\pslash - \frac{e}{c}\Aslash - m c \right) \psi
}
&=
\left(\pslash - \frac{e}{c}\Aslash \right) \wedge \left(\pslash - \frac{e}{c}\Aslash \right) \psi \\
&=
\pslash \wedge \pslash \psi 
- \frac{e}{c} \left( \pslash \wedge \Aslash \psi + \Aslash \wedge \pslash \psi \right)
+ \frac{e^2}{c^2} \Aslash \wedge \Aslash \psi \\
&=
- \frac{e}{c} \left( (\pslash \wedge \Aslash) \psi + (\pslash \psi) \wedge \Aslash + \Aslash \wedge \pslash \psi \right) \\
&=
- \frac{e}{c} (\pslash \wedge \Aslash) \psi.
\end{align*}

Here braces have been used to denote the range of operation of our differential operator $\pslash$.  With $\partialslash = \gamma^\mu \partial_\mu$, our momentum operator is

\begin{equation}\label{eqn:diracCovariant:270}
\pslash = i \hbar \partialslash,
\end{equation}

allowing us to write 

\begin{align*}
\pslash \wedge \Aslash
&=
i \hbar \partialslash \wedge \Aslash \\
\end{align*}

For the electromagnetic field bivector lets write

\begin{equation}\label{eqn:diracCovariant:290}
F = \partialslash \wedge \Aslash = \inv{2} F_{\mu \nu} \gamma^\mu \wedge \gamma^\nu.
\end{equation}

So the grade two terms of \ref{eqn:diracCovariant:210} are

\begin{equation}\label{eqn:diracCovariant:300}
\gpgradetwo{
\left(\pslash - \frac{e}{c}\Aslash + m c \right) \left(\pslash - \frac{e}{c}\Aslash - m c \right) \psi
}
=
-\frac{ie \hbar }{c } F \psi.
\end{equation}

Assembling all results we have

\begin{equation}\label{eqn:diracCovariant:320}
\left(
\left(\pslash - \frac{e}{c}\Aslash \right) \cdot \left(\pslash - \frac{e}{c}\Aslash \right) 
- \frac{ie \hbar }{c } F 
- (m c)^2 \right)  \psi = 0.
\end{equation}

\section{Some checks}

There are two tasks that remain.  One is to verify that this reproduces the result expressed in terms of the gauge covariant derivative $D_\mu$.  The other is to verify that this also reproduces the earlier result with an explicit split into energy and momentum operators.

\subsection{Gauge covariant derivative form}

With $\pslash = i \hbar \gamma^\mu \partial_\mu$ our dot product takes the form

\begin{align*}
\left(\pslash - \frac{e}{c}\Aslash \right) \cdot \left(\pslash - \frac{e}{c}\Aslash \right) 
&=
\left(i \hbar \partial^\mu - \frac{e}{c}A^\mu \right) \left(i \hbar \partial_\mu - \frac{e}{c}A_\mu \right)  \\
&=
-\hbar^2 \left(\partial^\mu + \frac{i e}{c \hbar}A^\mu \right) \left(\partial_\mu + \frac{i e}{ \hbar c}A_\mu \right)  \\
\end{align*}

We thus write 

\begin{equation}\label{eqn:diracCovariant:340}
D_\mu = \partial_\mu + \frac{i e}{c \hbar}A_\mu,
\end{equation}

differing from the text only by the inclusion of $\hbar$ and $c$ factors.  Since we have

\begin{equation}\label{eqn:diracCovariant:360}
\sigma^{\mu \nu} = \inv{i} \gamma^\mu \wedge \gamma^\nu,
\end{equation}

we have

\begin{equation}\label{eqn:diracCovariant:380}
\left( -\hbar^2 D^\mu D_\mu + \frac{i e \hbar}{ 2 c } \sigma^{\mu\nu} F_{\mu\nu} - (m c)^2 \right) \psi  = 0
\end{equation}

This reproduces \ref{eqn:diracCovariant:30}, the (corrected) result (36.78) from the text.

\subsection{Verifying the space time split into energy and spatial momentum operators.}

\EndArticle
