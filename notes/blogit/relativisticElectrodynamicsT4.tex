%
% Copyright � 2015 Peeter Joot.  All Rights Reserved.
% Licenced as described in the file LICENSE under the root directory of this GIT repository.
%
\documentclass[]{eliblog}

\usepackage{amsmath}
\usepackage{mathpazo}

%
% shorthand for bold symbols, convenient for vectors and matrices
%
\newcommand{\Ba}[0]{\mathbf{a}}
\newcommand{\Bb}[0]{\mathbf{b}}
\newcommand{\Bc}[0]{\mathbf{c}}
\newcommand{\Bd}[0]{\mathbf{d}}
\newcommand{\Be}[0]{\mathbf{e}}
\newcommand{\Bf}[0]{\mathbf{f}}
\newcommand{\Bg}[0]{\mathbf{g}}
\newcommand{\Bh}[0]{\mathbf{h}}
\newcommand{\Bi}[0]{\mathbf{i}}
\newcommand{\Bj}[0]{\mathbf{j}}
\newcommand{\Bk}[0]{\mathbf{k}}
\newcommand{\Bl}[0]{\mathbf{l}}
\newcommand{\Bm}[0]{\mathbf{m}}
\newcommand{\Bn}[0]{\mathbf{n}}
\newcommand{\Bo}[0]{\mathbf{o}}
\newcommand{\Bp}[0]{\mathbf{p}}
\newcommand{\Bq}[0]{\mathbf{q}}
\newcommand{\Br}[0]{\mathbf{r}}
\newcommand{\Bs}[0]{\mathbf{s}}
\newcommand{\Bt}[0]{\mathbf{t}}
\newcommand{\Bu}[0]{\mathbf{u}}
\newcommand{\Bv}[0]{\mathbf{v}}
\newcommand{\Bw}[0]{\mathbf{w}}
\newcommand{\Bx}[0]{\mathbf{x}}
\newcommand{\By}[0]{\mathbf{y}}
\newcommand{\Bz}[0]{\mathbf{z}}
\newcommand{\BA}[0]{\mathbf{A}}
\newcommand{\BB}[0]{\mathbf{B}}
\newcommand{\BC}[0]{\mathbf{C}}
\newcommand{\BD}[0]{\mathbf{D}}
\newcommand{\BE}[0]{\mathbf{E}}
\newcommand{\BF}[0]{\mathbf{F}}
\newcommand{\BG}[0]{\mathbf{G}}
\newcommand{\BH}[0]{\mathbf{H}}
\newcommand{\BI}[0]{\mathbf{I}}
\newcommand{\BJ}[0]{\mathbf{J}}
\newcommand{\BK}[0]{\mathbf{K}}
\newcommand{\BL}[0]{\mathbf{L}}
\newcommand{\BM}[0]{\mathbf{M}}
\newcommand{\BN}[0]{\mathbf{N}}
\newcommand{\BO}[0]{\mathbf{O}}
\newcommand{\BP}[0]{\mathbf{P}}
\newcommand{\BQ}[0]{\mathbf{Q}}
\newcommand{\BR}[0]{\mathbf{R}}
\newcommand{\BS}[0]{\mathbf{S}}
\newcommand{\BT}[0]{\mathbf{T}}
\newcommand{\BU}[0]{\mathbf{U}}
\newcommand{\BV}[0]{\mathbf{V}}
\newcommand{\BW}[0]{\mathbf{W}}
\newcommand{\BX}[0]{\mathbf{X}}
\newcommand{\BY}[0]{\mathbf{Y}}
\newcommand{\BZ}[0]{\mathbf{Z}}

\newcommand{\Bzero}[0]{\mathbf{0}}
\newcommand{\Btheta}[0]{\boldsymbol{\theta}}
\newcommand{\Btau}[0]{\boldsymbol{\tau}}
\newcommand{\Bomega}[0]{\boldsymbol{\omega}}

%
% shorthand for unit vectors
%
\newcommand{\acap}[0]{\hat{\Ba}}
\newcommand{\bcap}[0]{\hat{\Bb}}
\newcommand{\ccap}[0]{\hat{\Bc}}
\newcommand{\dcap}[0]{\hat{\Bd}}
\newcommand{\ecap}[0]{\hat{\Be}}
\newcommand{\fcap}[0]{\hat{\Bf}}
\newcommand{\gcap}[0]{\hat{\Bg}}
\newcommand{\hcap}[0]{\hat{\Bh}}
\newcommand{\icap}[0]{\hat{\Bi}}
\newcommand{\jcap}[0]{\hat{\Bj}}
\newcommand{\kcap}[0]{\hat{\Bk}}
\newcommand{\lcap}[0]{\hat{\Bl}}
\newcommand{\mcap}[0]{\hat{\Bm}}
\newcommand{\ncap}[0]{\hat{\Bn}}
\newcommand{\ocap}[0]{\hat{\Bo}}
\newcommand{\pcap}[0]{\hat{\Bp}}
\newcommand{\qcap}[0]{\hat{\Bq}}
\newcommand{\rcap}[0]{\hat{\Br}}
\newcommand{\scap}[0]{\hat{\Bs}}
\newcommand{\tcap}[0]{\hat{\Bt}}
\newcommand{\ucap}[0]{\hat{\Bu}}
\newcommand{\vcap}[0]{\hat{\Bv}}
\newcommand{\wcap}[0]{\hat{\Bw}}
\newcommand{\xcap}[0]{\hat{\Bx}}
\newcommand{\ycap}[0]{\hat{\By}}
\newcommand{\zcap}[0]{\hat{\Bz}}
\newcommand{\thetacap}[0]{\hat{\Btheta}}

%
% to write R^n and C^n in a distinguishable fashion.  Perhaps change this
% to the double lined characters upon figuring out how to do so.
%
\newcommand{\C}[1]{$\mathbb{C}^{#1}$}
\newcommand{\R}[1]{$\mathbb{R}^{#1}$}

%
% various generally useful helpers
%

% derivative of #1 wrt. #2:
\newcommand{\D}[2] {\frac {d#2} {d#1}}

\newcommand{\inv}[1]{\frac{1}{#1}}
\newcommand{\cross}[0]{\times}

\newcommand{\abs}[1]{\lvert{#1}\rvert}
\newcommand{\norm}[1]{\lVert{#1}\rVert}
\newcommand{\innerprod}[2]{\langle{#1}, {#2}\rangle}
\newcommand{\dotprod}[2]{{#1} \cdot {#2}}
\newcommand{\bdotprod}[2]{\left({#1} \cdot {#2}\right)}
\newcommand{\crossprod}[2]{{#1} \cross {#2}}
\newcommand{\tripleprod}[3]{\dotprod{\left(\crossprod{#1}{#2}\right)}{#3}}

\DeclareMathOperator{\Proj}{Proj}
\DeclareMathOperator{\Span}{span}
\DeclareMathOperator{\Sgn}{sgn}
\DeclareMathOperator{\Area}{Area}
\DeclareMathOperator{\Volume}{Volume}

%
% A few miscellaneous things specific to this document
%
\newcommand{\crossop}[1]{\crossprod{#1}{}}

% R2 vector.
\newcommand{\VectorTwo}[2]{
\begin{bmatrix}
 {#1} \\
 {#2}
\end{bmatrix}
}

\newcommand{\VectorN}[1]{
\begin{bmatrix}
{#1}_1 \\
{#1}_2 \\
\vdots \\
{#1}_N \\
\end{bmatrix}
}

\newcommand{\DETuvij}[4]{
\begin{vmatrix}
 {#1}_{#3} & {#1}_{#4} \\
 {#2}_{#3} & {#2}_{#4}
\end{vmatrix}
}

\newcommand{\DETuvwijk}[6]{
\begin{vmatrix}
 {#1}_{#4} & {#1}_{#5} & {#1}_{#6} \\
 {#2}_{#4} & {#2}_{#5} & {#2}_{#6} \\
 {#3}_{#4} & {#3}_{#5} & {#3}_{#6}
\end{vmatrix}
}

\newcommand{\DETuvwxijkl}[8]{
\begin{vmatrix}
 {#1}_{#5} & {#1}_{#6} & {#1}_{#7} & {#1}_{#8} \\
 {#2}_{#5} & {#2}_{#6} & {#2}_{#7} & {#2}_{#8} \\
 {#3}_{#5} & {#3}_{#6} & {#3}_{#7} & {#3}_{#8} \\
 {#4}_{#5} & {#4}_{#6} & {#4}_{#7} & {#4}_{#8} \\
\end{vmatrix}
}

%\newcommand{\DETuvwxyijklm}[10]{
%\begin{vmatrix}
% {#1}_{#6} & {#1}_{#7} & {#1}_{#8} & {#1}_{#9} & {#1}_{#10} \\
% {#2}_{#6} & {#2}_{#7} & {#2}_{#8} & {#2}_{#9} & {#2}_{#10} \\
% {#3}_{#6} & {#3}_{#7} & {#3}_{#8} & {#3}_{#9} & {#3}_{#10} \\
% {#4}_{#6} & {#4}_{#7} & {#4}_{#8} & {#4}_{#9} & {#4}_{#10} \\
% {#5}_{#6} & {#5}_{#7} & {#5}_{#8} & {#5}_{#9} & {#5}_{#10}
%\end{vmatrix}
%}

% R3 vector.
\newcommand{\VectorThree}[3]{
\begin{bmatrix}
 {#1} \\
 {#2} \\
 {#3}
\end{bmatrix}
}



\author{Peeter Joot}
\email{peeter.joot@gmail.com}

%\documentclass[]{eliblogwidescreen}

\usepackage{amsmath}
\usepackage{mathpazo}

%
% shorthand for bold symbols, convenient for vectors and matrices
%
\newcommand{\Ba}[0]{\mathbf{a}}
\newcommand{\Bb}[0]{\mathbf{b}}
\newcommand{\Bc}[0]{\mathbf{c}}
\newcommand{\Bd}[0]{\mathbf{d}}
\newcommand{\Be}[0]{\mathbf{e}}
\newcommand{\Bf}[0]{\mathbf{f}}
\newcommand{\Bg}[0]{\mathbf{g}}
\newcommand{\Bh}[0]{\mathbf{h}}
\newcommand{\Bi}[0]{\mathbf{i}}
\newcommand{\Bj}[0]{\mathbf{j}}
\newcommand{\Bk}[0]{\mathbf{k}}
\newcommand{\Bl}[0]{\mathbf{l}}
\newcommand{\Bm}[0]{\mathbf{m}}
\newcommand{\Bn}[0]{\mathbf{n}}
\newcommand{\Bo}[0]{\mathbf{o}}
\newcommand{\Bp}[0]{\mathbf{p}}
\newcommand{\Bq}[0]{\mathbf{q}}
\newcommand{\Br}[0]{\mathbf{r}}
\newcommand{\Bs}[0]{\mathbf{s}}
\newcommand{\Bt}[0]{\mathbf{t}}
\newcommand{\Bu}[0]{\mathbf{u}}
\newcommand{\Bv}[0]{\mathbf{v}}
\newcommand{\Bw}[0]{\mathbf{w}}
\newcommand{\Bx}[0]{\mathbf{x}}
\newcommand{\By}[0]{\mathbf{y}}
\newcommand{\Bz}[0]{\mathbf{z}}
\newcommand{\BA}[0]{\mathbf{A}}
\newcommand{\BB}[0]{\mathbf{B}}
\newcommand{\BC}[0]{\mathbf{C}}
\newcommand{\BD}[0]{\mathbf{D}}
\newcommand{\BE}[0]{\mathbf{E}}
\newcommand{\BF}[0]{\mathbf{F}}
\newcommand{\BG}[0]{\mathbf{G}}
\newcommand{\BH}[0]{\mathbf{H}}
\newcommand{\BI}[0]{\mathbf{I}}
\newcommand{\BJ}[0]{\mathbf{J}}
\newcommand{\BK}[0]{\mathbf{K}}
\newcommand{\BL}[0]{\mathbf{L}}
\newcommand{\BM}[0]{\mathbf{M}}
\newcommand{\BN}[0]{\mathbf{N}}
\newcommand{\BO}[0]{\mathbf{O}}
\newcommand{\BP}[0]{\mathbf{P}}
\newcommand{\BQ}[0]{\mathbf{Q}}
\newcommand{\BR}[0]{\mathbf{R}}
\newcommand{\BS}[0]{\mathbf{S}}
\newcommand{\BT}[0]{\mathbf{T}}
\newcommand{\BU}[0]{\mathbf{U}}
\newcommand{\BV}[0]{\mathbf{V}}
\newcommand{\BW}[0]{\mathbf{W}}
\newcommand{\BX}[0]{\mathbf{X}}
\newcommand{\BY}[0]{\mathbf{Y}}
\newcommand{\BZ}[0]{\mathbf{Z}}

\newcommand{\Bzero}[0]{\mathbf{0}}
\newcommand{\Btheta}[0]{\boldsymbol{\theta}}
\newcommand{\Btau}[0]{\boldsymbol{\tau}}
\newcommand{\Bomega}[0]{\boldsymbol{\omega}}

%
% shorthand for unit vectors
%
\newcommand{\acap}[0]{\hat{\Ba}}
\newcommand{\bcap}[0]{\hat{\Bb}}
\newcommand{\ccap}[0]{\hat{\Bc}}
\newcommand{\dcap}[0]{\hat{\Bd}}
\newcommand{\ecap}[0]{\hat{\Be}}
\newcommand{\fcap}[0]{\hat{\Bf}}
\newcommand{\gcap}[0]{\hat{\Bg}}
\newcommand{\hcap}[0]{\hat{\Bh}}
\newcommand{\icap}[0]{\hat{\Bi}}
\newcommand{\jcap}[0]{\hat{\Bj}}
\newcommand{\kcap}[0]{\hat{\Bk}}
\newcommand{\lcap}[0]{\hat{\Bl}}
\newcommand{\mcap}[0]{\hat{\Bm}}
\newcommand{\ncap}[0]{\hat{\Bn}}
\newcommand{\ocap}[0]{\hat{\Bo}}
\newcommand{\pcap}[0]{\hat{\Bp}}
\newcommand{\qcap}[0]{\hat{\Bq}}
\newcommand{\rcap}[0]{\hat{\Br}}
\newcommand{\scap}[0]{\hat{\Bs}}
\newcommand{\tcap}[0]{\hat{\Bt}}
\newcommand{\ucap}[0]{\hat{\Bu}}
\newcommand{\vcap}[0]{\hat{\Bv}}
\newcommand{\wcap}[0]{\hat{\Bw}}
\newcommand{\xcap}[0]{\hat{\Bx}}
\newcommand{\ycap}[0]{\hat{\By}}
\newcommand{\zcap}[0]{\hat{\Bz}}
\newcommand{\thetacap}[0]{\hat{\Btheta}}

%
% to write R^n and C^n in a distinguishable fashion.  Perhaps change this
% to the double lined characters upon figuring out how to do so.
%
\newcommand{\C}[1]{$\mathbb{C}^{#1}$}
\newcommand{\R}[1]{$\mathbb{R}^{#1}$}

%
% various generally useful helpers
%

% derivative of #1 wrt. #2:
\newcommand{\D}[2] {\frac {d#2} {d#1}}

\newcommand{\inv}[1]{\frac{1}{#1}}
\newcommand{\cross}[0]{\times}

\newcommand{\abs}[1]{\lvert{#1}\rvert}
\newcommand{\norm}[1]{\lVert{#1}\rVert}
\newcommand{\innerprod}[2]{\langle{#1}, {#2}\rangle}
\newcommand{\dotprod}[2]{{#1} \cdot {#2}}
\newcommand{\bdotprod}[2]{\left({#1} \cdot {#2}\right)}
\newcommand{\crossprod}[2]{{#1} \cross {#2}}
\newcommand{\tripleprod}[3]{\dotprod{\left(\crossprod{#1}{#2}\right)}{#3}}

\DeclareMathOperator{\Proj}{Proj}
\DeclareMathOperator{\Span}{span}
\DeclareMathOperator{\Sgn}{sgn}
\DeclareMathOperator{\Area}{Area}
\DeclareMathOperator{\Volume}{Volume}

%
% A few miscellaneous things specific to this document
%
\newcommand{\crossop}[1]{\crossprod{#1}{}}

% R2 vector.
\newcommand{\VectorTwo}[2]{
\begin{bmatrix}
 {#1} \\
 {#2}
\end{bmatrix}
}

\newcommand{\VectorN}[1]{
\begin{bmatrix}
{#1}_1 \\
{#1}_2 \\
\vdots \\
{#1}_N \\
\end{bmatrix}
}

\newcommand{\DETuvij}[4]{
\begin{vmatrix}
 {#1}_{#3} & {#1}_{#4} \\
 {#2}_{#3} & {#2}_{#4}
\end{vmatrix}
}

\newcommand{\DETuvwijk}[6]{
\begin{vmatrix}
 {#1}_{#4} & {#1}_{#5} & {#1}_{#6} \\
 {#2}_{#4} & {#2}_{#5} & {#2}_{#6} \\
 {#3}_{#4} & {#3}_{#5} & {#3}_{#6}
\end{vmatrix}
}

\newcommand{\DETuvwxijkl}[8]{
\begin{vmatrix}
 {#1}_{#5} & {#1}_{#6} & {#1}_{#7} & {#1}_{#8} \\
 {#2}_{#5} & {#2}_{#6} & {#2}_{#7} & {#2}_{#8} \\
 {#3}_{#5} & {#3}_{#6} & {#3}_{#7} & {#3}_{#8} \\
 {#4}_{#5} & {#4}_{#6} & {#4}_{#7} & {#4}_{#8} \\
\end{vmatrix}
}

%\newcommand{\DETuvwxyijklm}[10]{
%\begin{vmatrix}
% {#1}_{#6} & {#1}_{#7} & {#1}_{#8} & {#1}_{#9} & {#1}_{#10} \\
% {#2}_{#6} & {#2}_{#7} & {#2}_{#8} & {#2}_{#9} & {#2}_{#10} \\
% {#3}_{#6} & {#3}_{#7} & {#3}_{#8} & {#3}_{#9} & {#3}_{#10} \\
% {#4}_{#6} & {#4}_{#7} & {#4}_{#8} & {#4}_{#9} & {#4}_{#10} \\
% {#5}_{#6} & {#5}_{#7} & {#5}_{#8} & {#5}_{#9} & {#5}_{#10}
%\end{vmatrix}
%}

% R3 vector.
\newcommand{\VectorThree}[3]{
\begin{bmatrix}
 {#1} \\
 {#2} \\
 {#3}
\end{bmatrix}
}



\author{Peeter Joot}
\email{peeter.joot@gmail.com}


\chapter{PHY450H1S.  Relativistic Electrodynamics Tutorial 4 (TA: Simon Freedman).  Waveguides: confined EM waves.}
\label{chap:relativisticElectrodynamicsT4}
%\useCCL
\blogpage{http://sites.google.com/site/peeterjoot/math2011/relativisticElectrodynamicsT4.pdf}
\date{Mar 3, 2011}
\revisionInfo{relativisticElectrodynamicsT4.tex}

\beginArtWithToc
%\beginArtNoToc

\section{Motivation}

While this isn't part of the course, the topic of waveguides is one of so many applications that it is worth a mention, and that will be done in this tutorial.

We will setup our system with a waveguide (conducting surface that confines the radiation) oriented in the $\zcap$ direction.  The shape can be arbitrary

PICTURE: cross section of wacky shape.

\subsection{At the surface of a conductor.}

At the surface of the conductor (I presume this means the interior surface where there is no charge or current enclosed) we have

\begin{align}\label{eqn:relativisticElectrodynamicsT4:10}
\spacegrad \cross \BE &= - \inv{c} \PD{t}{\BB} \\
\spacegrad \cross \BB &= \inv{c} \PD{t}{\BE} \\
\spacegrad \cdot \BB &= 0 \\
\spacegrad \cdot \BE &= 0
\end{align}

If we are talking about the exterior surface, do we need to make any other assumptions (perfect conductors, or constant potentials)?

\subsection{Wave equations.}

For electric and magnetic fields in vacuum, we can show easily that these, like the potentials, separately satisfy the wave equation

Taking curls of the Maxwell curl equations above we have

\begin{align}\label{eqn:relativisticElectrodynamicsT4:30}
\spacegrad \cross (\spacegrad \cross \BE) &= - \inv{c^2} \PDSq{t}{\BE} \\
\spacegrad \cross (\spacegrad \cross \BB) &= - \inv{c^2} \PDSq{t}{\BB},
\end{align}

but we have for vector $\BM$

\begin{equation}\label{eqn:relativisticElectrodynamicsT4:50}
\spacegrad \cross (\spacegrad \cross \BM)
=
\spacegrad (\spacegrad \cdot \BM) - \Delta \BM,
\end{equation}

which gives us a pair of wave equations

\begin{align}\label{eqn:relativisticElectrodynamicsT4:70}
\square \BE &= 0 \\
\square \BB &= 0.
\end{align}

We still have the original constraints of Maxwell's equations to deal with, but we are free now to pick the complex exponentials as fundamental solutions, as our starting point

\begin{align}\label{eqn:relativisticElectrodynamicsT4:90}
\BE &= \BE_0 e^{i k^a x_a} = \BE_0 e^{ i (k^0 x_0 - \Bk \cdot \Bx) } \\
\BB &= \BB_0 e^{i k^a x_a} = \BB_0 e^{ i (k^0 x_0 - \Bk \cdot \Bx) },
\end{align}

With $k_0 = \omega/c$ and $x_0 = c t$ this is

\begin{align}\label{eqn:relativisticElectrodynamicsT4:110}
\BE &= \BE_0 e^{ i (\omega t - \Bk \cdot \Bx) } \\
\BB &= \BB_0 e^{ i (\omega t - \Bk \cdot \Bx) }.
\end{align}

For the vacuum case, with monochromatic light, we treated the amplitudes as constants.  Let's see what happens if we relax this assumption, and allow for spatial dependence (but no time dependence) of $\BE_0$ and $\BB_0$.  For the LHS of the electric field curl equation we have

\begin{align*}
0 
&= \spacegrad \cross \BE_0 e^{i k_a x^a} \\
&= (\spacegrad \cross \BE_0 - \BE_0 \cross \spacegrad) e^{i k_a x^a} \\
&= (\spacegrad \cross \BE_0 - \BE_0 \cross \Be^\alpha i k_a \partial_\alpha x^a) e^{i k_a x^a} \\
&= (\spacegrad \cross \BE_0 + \BE_0 \cross \Be^\alpha i k^a {\delta_\alpha}^a ) e^{i k_a x^a} \\
&= (\spacegrad \cross \BE_0 + i \BE_0 \cross \Bk ) e^{i k_a x^a}.
\end{align*}

Similarly for the divergence we have

\begin{align*}
0 
&= \spacegrad \cdot \BE_0 e^{i k_a x^a} \\
&= (\spacegrad \cdot \BE_0 + \BE_0 \cdot \spacegrad) e^{i k_a x^a} \\
&= (\spacegrad \cdot \BE_0 + \BE_0 \cdot \Be^\alpha i k_a \partial_\alpha x^a) e^{i k_a x^a} \\
&= (\spacegrad \cdot \BE_0 - \BE_0 \cdot \Be^\alpha i k^a {\delta_\alpha}^a ) e^{i k_a x^a} \\
&= (\spacegrad \cdot \BE_0 - i \Bk \cdot \BE_0 ) e^{i k_a x^a}.
\end{align*}

This provides constraints on the amplitudes

\begin{align}\label{eqn:relativisticElectrodynamicsT4:130}
\spacegrad \cross \BE_0 - i \Bk \cross \BE_0 &= -i \frac{\omega}{c} \BB_0 \\
\spacegrad \cross \BB_0 - i \Bk \cross \BB_0 &= i \frac{\omega}{c} \BE_0 \\
\spacegrad \cdot \BE_0 - i \Bk \cdot \BE_0 &= 0 \\
\spacegrad \cdot \BB_0 - i \Bk \cdot \BB_0 &= 0
\end{align}

%Having seen that taking curls of the curl equations gave us the wave equation, let's see what happens when we do so in momentum space.  It is worth recalling that
%
%\begin{equation}\label{eqn:relativisticElectrodynamicsT4:150}
%\spacegrad \cross (\Bk \cross \BM) = \Bk (\spacegrad \cdot \BM) - (\Bk \cdot \spacegrad) \BM,
%\end{equation}
%
%where $\Bk$ is a constant, and $\BM$ is not.  We have
%
%\begin{align*}
%\spacegrad (\spacegrad \cdot \BE_0) - \Delta \BE_0 - i (
%\Bk (\spacegrad \cdot \BE_0) - (\Bk \cdot \spacegrad) \BE_0
%) &= -i \frac{\omega}{c} \spacegrad \cross \BB_0.
%\end{align*}
%
%This doesn't look like it will be terribly helpful.  Backing up, and 
Applying the wave equation operator to our phasor we get

\begin{align*}
0 &=
\left(\inv{c^2} \partial_{tt} - \spacegrad^2 \right) \BE_0 e^{i (\omega t - \Bk \cdot \Bx)} \\
&=
\left(-\frac{\omega^2}{c^2} - \spacegrad^2 + \Bk^2 \right) \BE_0 e^{i (\omega t - \Bk \cdot \Bx)}
\end{align*}

So the momentum space equivalents of the wave equations are

\begin{align}\label{eqn:relativisticElectrodynamicsT4:170}
\left( \spacegrad^2 +\frac{\omega^2}{c^2} - \Bk^2 \right) \BE_0 &= 0 \\
\left( \spacegrad^2 +\frac{\omega^2}{c^2} - \Bk^2 \right) \BB_0 &= 0.
\end{align}

Observe that if $c^2 \Bk^2 = \omega^2$, then these amplitudes are harmonic functions (solutions to the Laplacian equation).  However, it doesn't appear that we require such a light like relation for the four vector $k^a = (\omega/c, \Bk)$.

\section{Back to the tutorial notes.}

In class we went straight to an assumed solution of the form

\begin{align}\label{eqn:relativisticElectrodynamicsT4:190}
\BE &= \BE_0(x, y) e^{ i(\omega t - k z) } \\
\BB &= \BB_0(x, y) e^{ i(\omega t - k z) },
\end{align}

where $\Bk = k \zcap$.  Our Laplacian was also written as the sum of components in the propagation and perpendicular directions

\begin{equation}\label{eqn:relativisticElectrodynamicsT4:210}
\spacegrad^2 = \PDSq{x_\perp}{} + \PDSq{z}{}.
\end{equation}

With no $z$ dependence in the amplitudes we have

\begin{align}\label{eqn:relativisticElectrodynamicsT4:170b}
\left( \PDSq{x_\perp}{} +\frac{\omega^2}{c^2} - \Bk^2 \right) \BE_0 &= 0 \\
\left( \PDSq{x_\perp}{} +\frac{\omega^2}{c^2} - \Bk^2 \right) \BB_0 &= 0.
\end{align}

\section{Separation into components.}

It was left as an exersize to separate out our Maxwell equations, so that our field components $\BE_0 = \BE_\perp + \BE_z$ and $\BB_0 = \BB_\perp + \BB_z$ in the propagation direction, and components in the perpendicular direction are separated

\begin{align*}
\spacegrad \cross \BE_0 
&=
(\spacegrad_\perp + \zcap\partial_z) \cross \BE_0 \\
&=
\spacegrad_\perp \cross \BE_0 \\
&=
\spacegrad_\perp \cross (\BE_\perp + \BE_z) \\
&=
\spacegrad_\perp \cross \BE_\perp 
+\spacegrad_\perp \cross \BE_z \\
&=
( \xcap \partial_x +\ycap \partial_y ) \cross ( \xcap E_x +\ycap E_y ) 
+\spacegrad_\perp \cross \BE_z \\
&=
\zcap (\partial_x E_y - \partial_z E_z) 
+\spacegrad_\perp \cross \BE_z.
\end{align*}

We can do something similar for $\BB_0$.  This allows for a split of \ref{eqn:relativisticElectrodynamicsT4:130} into $\zcap$ and perpendicular components

\begin{align}\label{eqn:relativisticElectrodynamicsT4:130b}
% zcap terms
\spacegrad_\perp \cross \BE_\perp &= -i \frac{\omega}{c} \BB_z \\
\spacegrad_\perp \cross \BB_\perp &= i \frac{\omega}{c} \BE_z \\
% perp terms
\spacegrad_\perp \cross \BE_z - i \Bk \cross \BE_\perp &= -i \frac{\omega}{c} \BB_\perp \\
\spacegrad_\perp \cross \BB_z - i \Bk \cross \BB_\perp &= i \frac{\omega}{c} \BE_\perp \\
\spacegrad_\perp \cdot \BE_\perp &= i k E_z - \partial_z E_z \\
\spacegrad_\perp \cdot \BB_\perp &= i k B_z - \partial_z B_z.
\end{align}

So we see that once we have a solution for $\BE_z$ and $\BB_z$ (by solving the wave equation above for those components), the components for the fields in terms of those components can be found.  Alternately, if one solves for the perpendicular components of the fields, these propagation components are available immediately with only differentiation.

In the case where the perpendicular components are taken as given

\begin{align}\label{eqn:relativisticElectrodynamicsT4:300}
\BB_z &= i \frac{ c  }{\omega} \spacegrad_\perp \cross \BE_\perp \\
\BE_z &= -i \frac{ c  }{\omega} \spacegrad_\perp \cross \BB_\perp,
\end{align}

we can express the remaining ones strictly in terms of the perpendicular fields

\begin{align}\label{eqn:relativisticElectrodynamicsT4:320}
\frac{\omega}{c} \BB_\perp &= \frac{c}{\omega} \spacegrad_\perp \cross (\spacegrad_\perp \cross \BB_\perp) + \Bk \cross \BE_\perp \\
\frac{\omega}{c} \BE_\perp &= \frac{c}{\omega} \spacegrad_\perp \cross (\spacegrad_\perp \cross \BE_\perp) - \Bk \cross \BB_\perp \\
\spacegrad_\perp \cdot \BE_\perp &= -i \frac{c}{\omega} (i k - \partial_z) \zcap \cdot (\spacegrad_\perp \cross \BB_\perp) \\
\spacegrad_\perp \cdot \BB_\perp &= i \frac{c}{\omega} (i k - \partial_z) \zcap \cdot (\spacegrad_\perp \cross \BE_\perp).
\end{align}

Is it at all helpful to expand the double cross products?

\begin{align*}
\frac{\omega^2}{c^2} \BB_\perp 
&= 
\spacegrad_\perp (\spacegrad_\perp \cdot \BB_\perp) -{\spacegrad_\perp}^2 \BB_\perp + \frac{\omega}{c} \Bk \cross \BE_\perp \\
&= 
i \frac{c}{\omega}
(i k - \partial_z)
\spacegrad_\perp \zcap \cdot (\spacegrad_\perp \cross \BE_\perp)
-{\spacegrad_\perp}^2 \BB_\perp + \frac{\omega}{c} \Bk \cross \BE_\perp 
\end{align*}
%\frac{\omega}{c} \BE_\perp &= 
%\frac{c}{\omega} (
%\spacegrad_\perp (\spacegrad_\perp \cdot \BE_\perp)
%-{\spacegrad_\perp}^2 \BE_\perp
%)
%- \Bk \cross \BB_\perp  \\

This gives us
\begin{align}\label{eqn:relativisticElectrodynamicsT4:340}
\left( {\spacegrad_\perp}^2 + \frac{\omega^2}{c^2} \right) \BB_\perp 
&= - \frac{c}{\omega} (k + i\partial_z) \spacegrad_\perp \zcap \cdot (\spacegrad_\perp \cross \BE_\perp) + \frac{\omega}{c} \Bk \cross \BE_\perp \\
\left( {\spacegrad_\perp}^2 + \frac{\omega^2}{c^2} \right) \BE_\perp 
&= -\frac{c}{\omega} (k + i\partial_z) \spacegrad_\perp \zcap \cdot (\spacegrad_\perp \cross \BB_\perp) - \frac{\omega}{c} \Bk \cross \BB_\perp,
\end{align}

but that doesn't seem particularly useful for completely solving the system?  It appears fairly messy to try to solve for $\BE_\perp$ and $\BB_\perp$ given the propagation direction fields.  I wonder if there is a simplification available that I am missing?

\section{Solving the momentum space wave equations.}

Back to the class notes.  We proceeded to solve for $\BE_z$ and $\BB_z$ from the wave equations by separation of variables.  We wish to solve equations of the form

\begin{equation}\label{eqn:relativisticElectrodynamicsT4:360}
\left( \PDSq{x}{} + \PDSq{y}{} + \frac{\omega^2}{c^2} - \Bk^2 \right) \phi(x,y) = 0
\end{equation}

Write $\phi(x,y) = X(x) Y(y)$, so that we have

\begin{equation}\label{eqn:relativisticElectrodynamicsT4:380}
\frac{X''}{X} + \frac{Y''}{Y} = \Bk^2 - \frac{\omega^2}{c^2}
\end{equation}

One solution is sinusoidal
\begin{align}\label{eqn:relativisticElectrodynamicsT4:400}
\frac{X''}{X} &= -k_1^2 \\
\frac{Y''}{Y} &= -k_2^2 \\
-k_1^2 - k_2^2
&= \Bk^2 - \frac{\omega^2}{c^2}.
\end{align}

The example in the tutorial now switche to a rectangular waveguide, still oriented with the propagation direction down the z-axis, but with lengths $a$ and $b$ along the $x$ and $y$ axis respectively.

Writing $k_1 = 2\pi m/a$, and $k_2 = 2 \pi n/ b$, we have

\begin{equation}\label{eqn:relativisticElectrodynamicsT4:420}
\phi(x, y) = \sum_{mn} a_{mn} 
\exp\left( \frac{2 \pi i m}{a} x \right)
\exp\left( \frac{2 \pi i n}{b} y \right)
\end{equation}

We were also provided with some definitions

\begin{definition}TE (Transverse Electric)

$\BE_3 = 0$.
\end{definition}
\begin{definition}
TM (Transverse Magnetic)

$\BB_3 = 0$.
\end{definition}
\begin{definition}
TM (Transverse Electromagnetic)

$\BE_3 = \BB_3 = 0$.
\end{definition}

\begin{claim}TEM do not existin in a hollow waveguide.
\end{claim}

Why: I had in my notes
\begin{align*}
\spacegrad \cross \BE = 0 & \implies \PD{x^1}{E_2} -\PD{x^2}{E_1} = 0 \\
\spacegrad \cdot \BE = 0 & \implies \PD{x^1}{E_1} +\PD{x^2}{E_2} = 0
\end{align*}

and then

\begin{align*}
\spacegrad^2 \phi &= 0 \\
\phi &= \text{const}
\end{align*}

In retrospect I fail to see how these are connected?  What happened to the $\partial_t \BB$ term in the curl equation above?

It was argued that we have $\BE_\parallel = \BB_\perp = 0$ on the boundary.

So for the TE case, where $\BE_3 = 0$, we have from the separation of variables argument

\begin{equation}\label{eqn:relativisticElectrodynamicsT4:440}
\zcap \cdot \BB_0(x, y) =
\sum_{mn} a_{mn} 
\cos\left( \frac{2 \pi i m}{a} x \right)
\cos\left( \frac{2 \pi i n}{b} y \right).
\end{equation}

No sines because 

\begin{equation}\label{eqn:relativisticElectrodynamicsT4:460}
B_1 \propto \PD{x_a}{B_3} \rightarrow \cos(k_1 x^1).
\end{equation}

The quantity

\begin{equation}\label{eqn:relativisticElectrodynamicsT4:480}
a_mn 
\cos\left( \frac{2 \pi i m}{a} x \right)
\cos\left( \frac{2 \pi i n}{b} y \right).
\end{equation}

is called the $TE_{mn}$ mode.  Note that since $B = \text{const}$ an ampere loop requires $\BB = 0$ since there is no current.

Writing 

\begin{align}\label{eqn:relativisticElectrodynamicsT4:500}
k &= \frac{\omega}{c} \sqrt{ 1 - \left(\frac{\omega_{mn}}{\omega}\right)^2 } \\
\omega_{mn} &= 2 \pi c \sqrt{ \left(\frac{m}{a} \right)^2 + \left(\frac{n}{b} \right)^2 }
\end{align}

Since $\omega < \omega_{mn}$ we have $k$ purely imaginary, and the term

\begin{equation}\label{eqn:relativisticElectrodynamicsT4:520}
e^{-i k z} = e^{- \Abs{k} z}
\end{equation}

represents the die off.

$\omega_{10}$ is the smallest.

Note that the convention is that the $m$ in $TE_{mn}$ is the bigger of the two indexes, so $\omega > \omega_{10}$.

The phase velocity 

\begin{equation}\label{eqn:relativisticElectrodynamicsT4:530}
V_\phi = \frac{\omega}{k} = \frac{c}{\sqrt{ 1 - \left(\frac{\omega_{mn}}{\omega}\right)^2 }} \ge c
\end{equation}

However, energy is transmitted with the group velocity, the ratio of the Poynting vector and energy density

\begin{equation}\label{eqn:relativisticElectrodynamicsT4:540}
\frac{\expectation{\BS}}{\expectation{U}} = V_g = \PD{k}{\omega} = 1/\PD{\omega}{k}
\end{equation}

(This can be shown).

Since

\begin{equation}\label{eqn:relativisticElectrodynamicsT4:560}
\left(\PD{\omega}{k}\right)^{-1} = 
\left(
\sqrt{ (\omega/c)^2 - (\omega_{mn}/c)^2 }
\right)^{-1} = c \sqrt{ 1 - (\omega_{mn}/\omega)^2 } \le c
\end{equation}

We see that the energy is transmitted at less than the speed of light as expected.

\section{Final remarks.}

I'd started converting my handwritten scrawl for this tutorial into an attempt at working through these ideas with enough detail that they self contained, but gave up part way.  This appears to me to be too big of a subdisipline to give it justice in one hours class.  As is, it is enough to at least get an concept of some of the ideas involved.  I think were I to learn this for real, I'd need a good text as a reference (or the time to attempt to blunder through the ideas in much much more detail).

%\EndArticle
\EndNoBibArticle
