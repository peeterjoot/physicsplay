%
% Copyright � 2015 Peeter Joot.  All Rights Reserved.
% Licenced as described in the file LICENSE under the root directory of this GIT repository.
%
\newcommand{\authorname}{Peeter Joot}
\newcommand{\email}{peeterjoot@protonmail.com}
\newcommand{\basename}{FIXMEbasenameUndefined}
\newcommand{\dirname}{notes/FIXMEdirnameUndefined/}

\renewcommand{\basename}{diracAlternate}
\renewcommand{\dirname}{notes/phy1520/}
%\newcommand{\dateintitle}{}
%\newcommand{\keywords}{}

\newcommand{\authorname}{Peeter Joot}
\newcommand{\onlineurl}{http://sites.google.com/site/peeterjoot2/math2013/\basename.pdf}
\newcommand{\sourcepath}{\dirname\basename.tex}
\newcommand{\generatetitle}[1]{\chapter{#1}}

\newcommand{\vcsinfo}{%
\section*{}
\noindent{\color{DarkOliveGreen}{\rule{\linewidth}{0.1mm}}}
\paragraph{Document version}
%\paragraph{\color{Maroon}{Document version}}
{
\small
\begin{itemize}
\item Available online at:\\ 
\href{\onlineurl}{\onlineurl}
\item Git Repository: \input{./.revinfo/gitRepo.tex}
\item Source: \sourcepath
\item last commit: \input{./.revinfo/gitCommitString.tex}
\item commit date: \input{./.revinfo/gitCommitDate.tex}
\end{itemize}
}
}

%\PassOptionsToPackage{dvipsnames,svgnames}{xcolor}
\PassOptionsToPackage{square,numbers}{natbib}
\documentclass{scrreprt}

\usepackage[left=2cm,right=2cm]{geometry}
\usepackage[svgnames]{xcolor}
\usepackage{peeters_layout}

\usepackage{natbib}

\usepackage[
colorlinks=true,
bookmarks=false,
pdfauthor={\authorname, \email},
backref 
]{hyperref}

% http://tex.stackexchange.com/questions/75773/how-to-reference-problems-by-the-text-label-in-an-exercise-envioronment
\usepackage[english]{cleveref}
\crefname{Exercise}{exercise}{exercises}
\Crefname{Exercise}{Exercise}{Exercises}

\RequirePackage{titlesec}
\RequirePackage{ifthen}

% http://stackoverflow.com/questions/4932910/date-in-the-tabular-environment
\makeatletter
\let\insertdate\@date
\makeatother

\titleformat{\chapter}[display]
{\bfseries\Large}
{\color{DarkSlateGrey}\filleft \authorname
\ifthenelse{\isundefined{\studentnumber}}{}{\\ \studentnumber}
\ifthenelse{\isundefined{\email}}{}{\\ \email}
\ifthenelse{\isundefined{\dateintitle}}{}{\\ \insertdate}
%\ifthenelse{\isundefined{\coursename}}{}{\\ \coursename} % put in title instead.
}
{4ex}
{\color{DarkOliveGreen}{\titlerule}\color{Maroon}
\vspace{2ex}%
\filright}
[\vspace{2ex}%
\color{DarkOliveGreen}\titlerule
]

\newcommand{\beginArtWithToc}[0]{\begin{document}\tableofcontents}
\newcommand{\beginArtNoToc}[0]{\begin{document}}
\newcommand{\EndNoBibArticle}[0]{\end{document}}
\newcommand{\EndArticle}[0]{\bibliography{Bibliography}\bibliographystyle{plainnat}\end{document}}

% 
%\newcommand{\citep}[1]{\cite{#1}}

\colorSectionsForArticle



\usepackage{peeters_layout_exercise}
\usepackage{peeters_braket}
\usepackage{peeters_figures}

\beginArtNoToc

\generatetitle{Alternate Dirac equation representation}
%\chapter{AlternateDiracEquation}
%\label{chap:diracAlternate}

Given an alternate representation of the Dirac equation

\begin{dmath}\label{eqn:diracAlternate:20}
H = 
\begin{bmatrix}
m c^2 + V_0 & c \hat{p} \\
c \hat{p} & - m c^2 + V_0
\end{bmatrix},
\end{dmath}

calculate the plane wave solutions, the Heisenberg velocity operator \( \hat{v} \), and find the form of the probability density current.

\paragraph{Plane wave solutions}

The action of the Hamiltonian on

\begin{dmath}\label{eqn:diracAlternate:40}
\psi = 
e^{i k x}
\begin{bmatrix}
\psi_1 \\
\psi_2
\end{bmatrix}
\end{dmath}

is
\begin{dmath}\label{eqn:diracAlternate:60}
H \psi
=
\begin{bmatrix}
m c^2 + V_0 & c (-i \Hbar) i k \\
c (-i \Hbar) i k & - m c^2 + V_0
\end{bmatrix}
e^{i k x}
\begin{bmatrix}
\psi_1 \\
\psi_2
\end{bmatrix}
=
\begin{bmatrix}
m c^2 + V_0 & c \Hbar k \\
c \Hbar k & - m c^2 + V_0
\end{bmatrix}
e^{i k x}
\begin{bmatrix}
\psi_1 \\
\psi_2
\end{bmatrix}
\end{dmath}

Writing

\begin{dmath}\label{eqn:diracAlternate:80}
H_k 
=
\begin{bmatrix}
m c^2 + V_0 & c \Hbar k \\
c \Hbar k & - m c^2 + V_0
\end{bmatrix}
\end{dmath}

the characteristic equation is

\begin{dmath}\label{eqn:diracAlternate:100}
0 = 
(m c^2 + V_0 - \lambda)
(-m c^2 + V_0 - \lambda)
%(-m c^2 - V_0 + \lambda)
%(m c^2 - V_0 + \lambda)
- (c \Hbar k)^2
=
\lr{ (\lambda - V_0)^2 - (m c^2)^2 } - (c \Hbar k)^2,
\end{dmath}

so 

\begin{dmath}\label{eqn:diracAlternate:120}
\lambda = V_0 \pm \epsilon,
\end{dmath}

where
\begin{dmath}\label{eqn:diracAlternate:140}
\epsilon^2 = (m c^2)^2 + (c \Hbar k)^2.
\end{dmath}

We've got

\begin{dmath}\label{eqn:diracAlternate:160}
\begin{aligned}
H - ( V_0 + \epsilon )
&=
\begin{bmatrix}
m c^2 - \epsilon & c \Hbar k \\
c \Hbar k & - m c^2 - \epsilon
\end{bmatrix} \\
H - ( V_0 - \epsilon )
&=
\begin{bmatrix}
m c^2 + \epsilon & c \Hbar k \\
c \Hbar k & - m c^2 + \epsilon
\end{bmatrix} \\
\end{aligned}
\end{dmath}

The eigenkets are

\begin{dmath}\label{eqn:diracAlternate:180}
\begin{aligned}
\ket{V_0+\epsilon}
&\propto
\begin{bmatrix}
-c \Hbar k \\
m c^2 - \epsilon 
\end{bmatrix} \\
\ket{V_0-\epsilon}
&\propto
\begin{bmatrix}
-c \Hbar k \\
m c^2 + \epsilon 
\end{bmatrix}.
\end{aligned}
\end{dmath}

% (c \Hbar k)^2 + (m c^2 - \epsilon)^2 = 2 \epsilon^2 - 2 m c^2 \epsilon = 2 \epsilon ( \epsilon - m c^2)
% (c \Hbar k)^2 + (m c^2 + \epsilon)^2 = 2 \epsilon^2 + 2 m c^2 \epsilon = 2 \epsilon ( \epsilon + m c^2)

Normalized, these are

\begin{dmath}\label{eqn:diracAlternate:200}
\begin{aligned}
\ket{V_0 + \epsilon}
&=
\inv{\sqrt{ 2 \epsilon ( \epsilon - m c^2) }}
\begin{bmatrix}
c \Hbar k \\
\epsilon -m c^2 
\end{bmatrix} \\
\ket{V_0 - \epsilon}
&=
\inv{\sqrt{ 2 \epsilon ( \epsilon + m c^2) }}
\begin{bmatrix}
-c \Hbar k \\
\epsilon + m c^2
\end{bmatrix} \\
\end{aligned}
\end{dmath}

We can now write

\begin{dmath}\label{eqn:diracAlternate:220}
H_k = 
E 
\begin{bmatrix}
V_0 + \epsilon & 0 \\
0         & V_0 - \epsilon 
\end{bmatrix}
E^{-1},
\end{dmath}

where
\begin{dmath}\label{eqn:diracAlternate:240}
E = 
\inv{\sqrt{2 \epsilon} }
\begin{bmatrix}
\frac{c \Hbar k}{ \sqrt{ \epsilon - m c^2 } } & -\frac{c \Hbar k}{ \sqrt{ \epsilon + m c^2 } } \\
\sqrt{ \epsilon - m c^2 } & \sqrt{ \epsilon + m c^2 }
\end{bmatrix}
\end{dmath}

Observe that there's redundancy in this matrix since \( \ifrac{c \Hbar k}{ \sqrt{ \epsilon - m c^2 } } = \sqrt{ \epsilon + m c^2 } \), and \( \ifrac{c \Hbar k}{ \sqrt{ \epsilon + m c^2 } } = \sqrt{ \epsilon - m c^2 } \), which allows the transformation matrix to be written in the form of a rotation matrix

\begin{dmath}\label{eqn:diracAlternate:260}
E = 
\inv{\sqrt{2 \epsilon} }
\begin{bmatrix}
\frac{c \Hbar k}{ \sqrt{ \epsilon - m c^2 } } & -\frac{c \Hbar k}{ \sqrt{ \epsilon + m c^2 } } \\
\frac{c \Hbar k}{ \sqrt{ \epsilon + m c^2 } } & \frac{c \Hbar k}{ \sqrt{ \epsilon - m c^2 } }
\end{bmatrix}.
\end{dmath}

This has unit determinant, so with

\begin{equation}\label{eqn:diracAlternate:280}
\begin{aligned}
\cos\theta &= \frac{c \Hbar k}{ \sqrt{ 2 \epsilon( \epsilon - m c^2) } } \\
\sin\theta &= \frac{c \Hbar k}{ \sqrt{ 2 \epsilon( \epsilon + m c^2) } }
\end{aligned},
\end{equation}

the transformation matrix is
\begin{equation}\label{eqn:diracAlternate:300}
E = 
\begin{bmatrix}
\ket{V_0 + \epsilon} & \ket{V_0 - \epsilon} 
\end{bmatrix}
=
\begin{bmatrix}
\cos\theta & -\sin\theta \\
\sin\theta & \cos\theta
\end{bmatrix}.
\end{equation}

%\EndArticle
\EndNoBibArticle
