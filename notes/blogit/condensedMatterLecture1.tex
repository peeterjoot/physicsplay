%
% Copyright � 2013 Peeter Joot.  All Rights Reserved.
% Licenced as described in the file LICENSE under the root directory of this GIT repository.
%
\newcommand{\authorname}{Peeter Joot}
\newcommand{\email}{peeterjoot@protonmail.com}
\newcommand{\basename}{FIXMEbasenameUndefined}
\newcommand{\dirname}{notes/FIXMEdirnameUndefined/}

\renewcommand{\basename}{condensedMatterLecture1}
\renewcommand{\dirname}{notes/phy487/}
\newcommand{\keywords}{Condensed matter physics, PHY487H1S}
\newcommand{\authorname}{Peeter Joot}
\newcommand{\onlineurl}{http://sites.google.com/site/peeterjoot2/math2013/\basename.pdf}
\newcommand{\sourcepath}{\dirname\basename.tex}
\newcommand{\generatetitle}[1]{\chapter{#1}}

\newcommand{\vcsinfo}{%
\section*{}
\noindent{\color{DarkOliveGreen}{\rule{\linewidth}{0.1mm}}}
\paragraph{Document version}
%\paragraph{\color{Maroon}{Document version}}
{
\small
\begin{itemize}
\item Available online at:\\ 
\href{\onlineurl}{\onlineurl}
\item Git Repository: \input{./.revinfo/gitRepo.tex}
\item Source: \sourcepath
\item last commit: \input{./.revinfo/gitCommitString.tex}
\item commit date: \input{./.revinfo/gitCommitDate.tex}
\end{itemize}
}
}

%\PassOptionsToPackage{dvipsnames,svgnames}{xcolor}
\PassOptionsToPackage{square,numbers}{natbib}
\documentclass{scrreprt}

\usepackage[left=2cm,right=2cm]{geometry}
\usepackage[svgnames]{xcolor}
\usepackage{peeters_layout}

\usepackage{natbib}

\usepackage[
colorlinks=true,
bookmarks=false,
pdfauthor={\authorname, \email},
backref 
]{hyperref}

% http://tex.stackexchange.com/questions/75773/how-to-reference-problems-by-the-text-label-in-an-exercise-envioronment
\usepackage[english]{cleveref}
\crefname{Exercise}{exercise}{exercises}
\Crefname{Exercise}{Exercise}{Exercises}

\RequirePackage{titlesec}
\RequirePackage{ifthen}

% http://stackoverflow.com/questions/4932910/date-in-the-tabular-environment
\makeatletter
\let\insertdate\@date
\makeatother

\titleformat{\chapter}[display]
{\bfseries\Large}
{\color{DarkSlateGrey}\filleft \authorname
\ifthenelse{\isundefined{\studentnumber}}{}{\\ \studentnumber}
\ifthenelse{\isundefined{\email}}{}{\\ \email}
\ifthenelse{\isundefined{\dateintitle}}{}{\\ \insertdate}
%\ifthenelse{\isundefined{\coursename}}{}{\\ \coursename} % put in title instead.
}
{4ex}
{\color{DarkOliveGreen}{\titlerule}\color{Maroon}
\vspace{2ex}%
\filright}
[\vspace{2ex}%
\color{DarkOliveGreen}\titlerule
]

\newcommand{\beginArtWithToc}[0]{\begin{document}\tableofcontents}
\newcommand{\beginArtNoToc}[0]{\begin{document}}
\newcommand{\EndNoBibArticle}[0]{\end{document}}
\newcommand{\EndArticle}[0]{\bibliography{Bibliography}\bibliographystyle{plainnat}\end{document}}

% 
%\newcommand{\citep}[1]{\cite{#1}}

\colorSectionsForArticle



\beginArtNoToc
\generatetitle{PHY487H1S Condensed Matter Physics.  Lecture 1: Course overview.  Taught by Prof.\ Stephen Julian}
%\chapter{Course overview}
\label{chap:condensedMatterLecture1}
% classmates: Seth (eng sci, tall and lanky)

\section{Disclaimer}

Peeter's lecture notes from class.  May not be entirely coherent.

\section{Course overview}

\begin{itemize}
\item Bonding and crystal structure: Why solids form, describing periodic order mathematically, diffraction.
\item Lattice vibrations: ``elementary excitation'' of a periodic array of atoms (periodicity allows a $10^{23}$ body problem to be solved).
\item Electrons in solids: Explaining this introduces a need for quantum mechanics.  
%Periodicity -> band structure -> insulators vs metals.
\item Electrical conduction in solids.  (metals and semiconductors).
\end{itemize}

\section{Chemical bonding in solids}

% Reading: L&L chapter 1.

\begin{itemize}
\item Different types of chemical bonds explain many of the differences between solids.
\item Differences in solids: hard/soft.  Example: Lithium, so soft that a pure sample will flow if set on a desk ; metal vs insulating, melting points
\end{itemize}

FIXME: Fig1: periodic table annoted with orbital filling notes.

\S 1.1 \underline{The periodic table}.  Elements are classified according to which type of orbital is being filled.

FIXME: Fig2: Hydrogenic atom (only one electron).  Hydrogen, or other super ionized material (example: iron with all but one electron observed in supernova spectra).

FIXME: Fig3: Two electron atom. eg: $\text{Ti}_{22}$.

Read: \S 1.1.

FIXME: Fig4.  s, m, l ranges.

\begin{itemize}
\item upper right hand of periodic table: covalent bonding
\item lower left hand of periodic table: metallic bonding
\item mixed left hand with right hand of periodic table: ionic bonding
\end{itemize}

\section{Covalent bonding}

Consider a pair of hydrogen nuclei sharing one electron.  \S 1.2 \citep{ibach2009solid} has a mathematical description (not examinable)

FIXME: Fig5

F6:  far apart

F7: close together.  bonding

F8: close together.  anti-bonding

F9: Chemistry diagram.

%\paragraph{Adding more electrons}

\EndArticle
