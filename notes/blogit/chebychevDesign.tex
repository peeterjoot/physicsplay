%
% Copyright � 2015 Peeter Joot.  All Rights Reserved.
% Licenced as described in the file LICENSE under the root directory of this GIT repository.
%
\newcommand{\authorname}{Peeter Joot}
\newcommand{\email}{peeterjoot@protonmail.com}
\newcommand{\basename}{FIXMEbasenameUndefined}
\newcommand{\dirname}{notes/FIXMEdirnameUndefined/}

\renewcommand{\basename}{chebychevDesign}
\renewcommand{\dirname}{notes/FIXMEwheretodirname/}
%\newcommand{\dateintitle}{}
%\newcommand{\keywords}{}

\newcommand{\authorname}{Peeter Joot}
\newcommand{\onlineurl}{http://sites.google.com/site/peeterjoot2/math2013/\basename.pdf}
\newcommand{\sourcepath}{\dirname\basename.tex}
\newcommand{\generatetitle}[1]{\chapter{#1}}

\newcommand{\vcsinfo}{%
\section*{}
\noindent{\color{DarkOliveGreen}{\rule{\linewidth}{0.1mm}}}
\paragraph{Document version}
%\paragraph{\color{Maroon}{Document version}}
{
\small
\begin{itemize}
\item Available online at:\\ 
\href{\onlineurl}{\onlineurl}
\item Git Repository: \input{./.revinfo/gitRepo.tex}
\item Source: \sourcepath
\item last commit: \input{./.revinfo/gitCommitString.tex}
\item commit date: \input{./.revinfo/gitCommitDate.tex}
\end{itemize}
}
}

%\PassOptionsToPackage{dvipsnames,svgnames}{xcolor}
\PassOptionsToPackage{square,numbers}{natbib}
\documentclass{scrreprt}

\usepackage[left=2cm,right=2cm]{geometry}
\usepackage[svgnames]{xcolor}
\usepackage{peeters_layout}

\usepackage{natbib}

\usepackage[
colorlinks=true,
bookmarks=false,
pdfauthor={\authorname, \email},
backref 
]{hyperref}

% http://tex.stackexchange.com/questions/75773/how-to-reference-problems-by-the-text-label-in-an-exercise-envioronment
\usepackage[english]{cleveref}
\crefname{Exercise}{exercise}{exercises}
\Crefname{Exercise}{Exercise}{Exercises}

\RequirePackage{titlesec}
\RequirePackage{ifthen}

% http://stackoverflow.com/questions/4932910/date-in-the-tabular-environment
\makeatletter
\let\insertdate\@date
\makeatother

\titleformat{\chapter}[display]
{\bfseries\Large}
{\color{DarkSlateGrey}\filleft \authorname
\ifthenelse{\isundefined{\studentnumber}}{}{\\ \studentnumber}
\ifthenelse{\isundefined{\email}}{}{\\ \email}
\ifthenelse{\isundefined{\dateintitle}}{}{\\ \insertdate}
%\ifthenelse{\isundefined{\coursename}}{}{\\ \coursename} % put in title instead.
}
{4ex}
{\color{DarkOliveGreen}{\titlerule}\color{Maroon}
\vspace{2ex}%
\filright}
[\vspace{2ex}%
\color{DarkOliveGreen}\titlerule
]

\newcommand{\beginArtWithToc}[0]{\begin{document}\tableofcontents}
\newcommand{\beginArtNoToc}[0]{\begin{document}}
\newcommand{\EndNoBibArticle}[0]{\end{document}}
\newcommand{\EndArticle}[0]{\bibliography{Bibliography}\bibliographystyle{plainnat}\end{document}}

% 
%\newcommand{\citep}[1]{\cite{#1}}

\colorSectionsForArticle



\beginArtNoToc

\generatetitle{Chebychev antenna design}
%\chapter{Chebychev antenna design}
%\label{chap:chebychevDesign}

In our text \citep{balanis2005antenna} is a design procedure that applies Chebychev polynomials to the selection of current magnitudes for an evenly spaced array of identical antennas placed along the z-axis.

For an even number \( 2 M \) of identical antennas placed at positions \( \Br_m = (d/2) \lr{2 m -1} \Be_3 \), the array factor is

\begin{dmath}\label{eqn:chebychevDesign:20}
\textrm{AF} 
= 
\sum_{m=-N}^N I_m e^{-j k \rcap \cdot \Br_m }.
\end{dmath}

Assuming the currents are symmetric \( I_{-m} = I_m \), with \( \rcap = (\sin\theta \cos\phi, \sin\theta \sin\phi, \cos\theta ) \), and \( u = \frac{\pi d}{\lambda} \cos\theta \), this is

\begin{dmath}\label{eqn:chebychevDesign:40}
\textrm{AF}
= 
\sum_{m=-N}^N I_m e^{-j k (d/2) (  2 m -1 )\cos\theta }
= 
2 \sum_{m=1}^N I_m \cos\lr{ k (d/2) ( 2 m -1)\cos\theta } % 
=
2 \sum_{m=1}^N I_m \cos\lr{ (2 m -1) u }.
\end{dmath}

This is a sum of only odd cosines, and can be expanded as a sum that includes all the odd powers of \( \cos u \).
%  The odd degree Chebychev polynomials \( T_m(\cos\theta) = \cos(m \theta) \) are similar, since the expansion of that in \( \cos\theta \) has all the that can be expanded 

\EndArticle
%\EndNoBibArticle
