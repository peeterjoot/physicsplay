%
% Copyright � 2015 Peeter Joot.  All Rights Reserved.
% Licenced as described in the file LICENSE under the root directory of this GIT repository.
%
\newcommand{\authorname}{Peeter Joot}
\newcommand{\email}{peeterjoot@protonmail.com}
\newcommand{\basename}{FIXMEbasenameUndefined}
\newcommand{\dirname}{notes/FIXMEdirnameUndefined/}

\renewcommand{\basename}{chebychevDesign}
\renewcommand{\dirname}{notes/ece1229/}

\newcommand{\authorname}{Peeter Joot}
\newcommand{\onlineurl}{http://sites.google.com/site/peeterjoot2/math2013/\basename.pdf}
\newcommand{\sourcepath}{\dirname\basename.tex}
\newcommand{\generatetitle}[1]{\chapter{#1}}

\newcommand{\vcsinfo}{%
\section*{}
\noindent{\color{DarkOliveGreen}{\rule{\linewidth}{0.1mm}}}
\paragraph{Document version}
%\paragraph{\color{Maroon}{Document version}}
{
\small
\begin{itemize}
\item Available online at:\\ 
\href{\onlineurl}{\onlineurl}
\item Git Repository: \input{./.revinfo/gitRepo.tex}
\item Source: \sourcepath
\item last commit: \input{./.revinfo/gitCommitString.tex}
\item commit date: \input{./.revinfo/gitCommitDate.tex}
\end{itemize}
}
}

%\PassOptionsToPackage{dvipsnames,svgnames}{xcolor}
\PassOptionsToPackage{square,numbers}{natbib}
\documentclass{scrreprt}

\usepackage[left=2cm,right=2cm]{geometry}
\usepackage[svgnames]{xcolor}
\usepackage{peeters_layout}

\usepackage{natbib}

\usepackage[
colorlinks=true,
bookmarks=false,
pdfauthor={\authorname, \email},
backref 
]{hyperref}

% http://tex.stackexchange.com/questions/75773/how-to-reference-problems-by-the-text-label-in-an-exercise-envioronment
\usepackage[english]{cleveref}
\crefname{Exercise}{exercise}{exercises}
\Crefname{Exercise}{Exercise}{Exercises}

\RequirePackage{titlesec}
\RequirePackage{ifthen}

% http://stackoverflow.com/questions/4932910/date-in-the-tabular-environment
\makeatletter
\let\insertdate\@date
\makeatother

\titleformat{\chapter}[display]
{\bfseries\Large}
{\color{DarkSlateGrey}\filleft \authorname
\ifthenelse{\isundefined{\studentnumber}}{}{\\ \studentnumber}
\ifthenelse{\isundefined{\email}}{}{\\ \email}
\ifthenelse{\isundefined{\dateintitle}}{}{\\ \insertdate}
%\ifthenelse{\isundefined{\coursename}}{}{\\ \coursename} % put in title instead.
}
{4ex}
{\color{DarkOliveGreen}{\titlerule}\color{Maroon}
\vspace{2ex}%
\filright}
[\vspace{2ex}%
\color{DarkOliveGreen}\titlerule
]

\newcommand{\beginArtWithToc}[0]{\begin{document}\tableofcontents}
\newcommand{\beginArtNoToc}[0]{\begin{document}}
\newcommand{\EndNoBibArticle}[0]{\end{document}}
\newcommand{\EndArticle}[0]{\bibliography{Bibliography}\bibliographystyle{plainnat}\end{document}}

% 
%\newcommand{\citep}[1]{\cite{#1}}

\colorSectionsForArticle



\usepackage{peeters_layout_exercise}
\usepackage{ece1229}
\usepackage{siunitx}
\usepackage{esint} % \oiint

\beginArtNoToc

\generatetitle{Chebychev antenna design}
%\chapter{Chebychev antenna design}
%\label{chap:chebychevDesign}

In our text \citep{balanis2005antenna} is a design procedure that applies Chebychev polynomials to the selection of current magnitudes for an evenly spaced array of identical antennas placed along the z-axis.

For an even number \( 2 M \) of identical antennas placed at positions \( \Br_m = (d/2) \lr{2 m -1} \Be_3 \), the array factor is

\begin{dmath}\label{eqn:chebychevDesign:20}
\textrm{AF} 
= 
\sum_{m=-N}^N I_m e^{-j k \rcap \cdot \Br_m }.
\end{dmath}

Assuming the currents are symmetric \( I_{-m} = I_m \), with \( \rcap = (\sin\theta \cos\phi, \sin\theta \sin\phi, \cos\theta ) \), and \( u = \frac{\pi d}{\lambda} \cos\theta \), this is

\begin{dmath}\label{eqn:chebychevDesign:40}
\textrm{AF}
= 
\sum_{m=-N}^N I_m e^{-j k (d/2) (  2 m -1 )\cos\theta }
= 
2 \sum_{m=1}^N I_m \cos\lr{ k (d/2) ( 2 m -1)\cos\theta } % 
=
2 \sum_{m=1}^N I_m \cos\lr{ (2 m -1) u }.
\end{dmath}

This is a sum of only odd cosines, and can be expanded as a sum that includes all the odd powers of \( \cos u \).  Suppose for example that this is a four element array with \( N = 2 \).  In this case the array factor has the form

\begin{dmath}\label{eqn:chebychevDesign:60}
\textrm{AF}
=
2 \lr{ I_1 \cos u + I_2 \lr{ 4 \cos^3 u - 3 \cos u } }
=
2 \lr{ \lr{ I_1 - 3 I_2 } \cos u + 4 I_2 \cos^3 u }.
\end{dmath}

The design procedure in the text sets \( \cos u = z/z_0 \), and then equates this to \( T_3(z) = 4 z^3 - 3 z \) to determine the current amplitudes \( I_m \).  That is

\begin{dmath}\label{eqn:chebychevDesign:80}
\frac{ 2 I_1 - 6 I_2 }{z_0} z  + \frac{8 I_2}{z_0^3} z^3 = -3 z + 4 z^3,
\end{dmath}
%\frac{ 2 I_1 - 3 z_0^3 }{z_0} z  + \frac{8 I_2}{z_0^3} z^3 = -3 z + 4 z^3,
%2 I_1 = 3 (z_0^2 -1) z_0

or

\begin{dmath}\label{eqn:chebychevDesign:100}
\begin{bmatrix}
I_1 \\
I_2
\end{bmatrix}
=
{\begin{bmatrix}
2/z_0 & -6/z_0 \\
0 & 8/z_0^3
\end{bmatrix}}^{-1}
\begin{bmatrix}
-3 \\
4
\end{bmatrix}
=
\frac{z_0}{2}
\begin{bmatrix}
3 (z_0^2 -1) \\
z_0^2
\end{bmatrix}.
\end{dmath}

The currents in the array factor are fully determined up to a scale factor, reducing the array factor to

\begin{dmath}\label{eqn:chebychevDesign:140}
\textrm{AF} = 4 z_0^3 \cos^3 u - 3 z_0 \cos u.
\end{dmath}

The zeros of this array factor are located at the zeros of

\begin{dmath}\label{eqn:chebychevDesign:120}
T_3( z_0 \cos u ) = \cos( 3 \cos^{-1} \lr{ z_0 \cos u } ),
\end{dmath}

which are at \( 3 \cos^{-1} \lr{ z_0 \cos u } = \pi/2 + m \pi = \pi \lr{ m + \inv{2} } \)

\begin{dmath}\label{eqn:chebychevDesign:160}
\cos u = \inv{z_0} \cos\lr{ \frac{\pi}{3} \lr{ m + \inv{2} } } = \setlr{ 0, \pm \frac{\sqrt{3}}{2 z_0} }.
\end{dmath}

showing that the scaling factor \( z_0 \) effects the locations of the zeros.  It also allows the values at the extremes \( \cos u = \pm 1 \), to increase past the \( \pm 1 \) non-scaled limit values.  These effects can be explored in \url{http://goo.gl/KPqcjX}, but can also be seen in \cref{fig:ChebyChevThreeScaled:ChebyChevThreeScaledFig2}.

\imageFigure{../../figures/ece1229/ChebyChevThreeScaledFig2}{\( T_3( z_0 x) \) for a few different scale factors \( z_0 \).}{fig:ChebyChevThreeScaled:ChebyChevThreeScaledFig2}{0.2}

The scale factor can be fixed for a desired maximum power gain.  For \( R \si{dB} \), that will be when

\begin{dmath}\label{eqn:chebychevDesign:180}
20 \log_{10} \cosh( 3 \cosh^{-1} z_0 ) = R \si{dB},
\end{dmath}

or

\begin{dmath}\label{eqn:chebychevDesign:200}
z_0 = \cosh \lr{ \inv{3} \cosh^{-1} \lr{ 10^{\frac{R}{20}} } }.
\end{dmath}

For \( R = 30 \) dB (say), we have \( z_0 = 2.1 \), and

\begin{dmath}\label{eqn:chebychevDesign:220}
\textrm{AF}
= 40 \cos^3 \lr{ \frac{\pi d}{\lambda} \cos\theta }  - 6.4 \cos \lr{ \frac{\pi d}{\lambda} \cos\theta }.
\end{dmath}

\EndArticle
%\EndNoBibArticle
