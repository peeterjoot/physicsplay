\documentclass[]{eliblog}

\usepackage{color}
%\usepackage{txfonts} % for xi

\usepackage{amsmath}
\usepackage{mathpazo}

%
% shorthand for bold symbols, convenient for vectors and matrices
%
\newcommand{\Ba}[0]{\mathbf{a}}
\newcommand{\Bb}[0]{\mathbf{b}}
\newcommand{\Bc}[0]{\mathbf{c}}
\newcommand{\Bd}[0]{\mathbf{d}}
\newcommand{\Be}[0]{\mathbf{e}}
\newcommand{\Bf}[0]{\mathbf{f}}
\newcommand{\Bg}[0]{\mathbf{g}}
\newcommand{\Bh}[0]{\mathbf{h}}
\newcommand{\Bi}[0]{\mathbf{i}}
\newcommand{\Bj}[0]{\mathbf{j}}
\newcommand{\Bk}[0]{\mathbf{k}}
\newcommand{\Bl}[0]{\mathbf{l}}
\newcommand{\Bm}[0]{\mathbf{m}}
\newcommand{\Bn}[0]{\mathbf{n}}
\newcommand{\Bo}[0]{\mathbf{o}}
\newcommand{\Bp}[0]{\mathbf{p}}
\newcommand{\Bq}[0]{\mathbf{q}}
\newcommand{\Br}[0]{\mathbf{r}}
\newcommand{\Bs}[0]{\mathbf{s}}
\newcommand{\Bt}[0]{\mathbf{t}}
\newcommand{\Bu}[0]{\mathbf{u}}
\newcommand{\Bv}[0]{\mathbf{v}}
\newcommand{\Bw}[0]{\mathbf{w}}
\newcommand{\Bx}[0]{\mathbf{x}}
\newcommand{\By}[0]{\mathbf{y}}
\newcommand{\Bz}[0]{\mathbf{z}}
\newcommand{\BA}[0]{\mathbf{A}}
\newcommand{\BB}[0]{\mathbf{B}}
\newcommand{\BC}[0]{\mathbf{C}}
\newcommand{\BD}[0]{\mathbf{D}}
\newcommand{\BE}[0]{\mathbf{E}}
\newcommand{\BF}[0]{\mathbf{F}}
\newcommand{\BG}[0]{\mathbf{G}}
\newcommand{\BH}[0]{\mathbf{H}}
\newcommand{\BI}[0]{\mathbf{I}}
\newcommand{\BJ}[0]{\mathbf{J}}
\newcommand{\BK}[0]{\mathbf{K}}
\newcommand{\BL}[0]{\mathbf{L}}
\newcommand{\BM}[0]{\mathbf{M}}
\newcommand{\BN}[0]{\mathbf{N}}
\newcommand{\BO}[0]{\mathbf{O}}
\newcommand{\BP}[0]{\mathbf{P}}
\newcommand{\BQ}[0]{\mathbf{Q}}
\newcommand{\BR}[0]{\mathbf{R}}
\newcommand{\BS}[0]{\mathbf{S}}
\newcommand{\BT}[0]{\mathbf{T}}
\newcommand{\BU}[0]{\mathbf{U}}
\newcommand{\BV}[0]{\mathbf{V}}
\newcommand{\BW}[0]{\mathbf{W}}
\newcommand{\BX}[0]{\mathbf{X}}
\newcommand{\BY}[0]{\mathbf{Y}}
\newcommand{\BZ}[0]{\mathbf{Z}}

\newcommand{\Bzero}[0]{\mathbf{0}}
\newcommand{\Btheta}[0]{\boldsymbol{\theta}}
\newcommand{\Btau}[0]{\boldsymbol{\tau}}
\newcommand{\Bomega}[0]{\boldsymbol{\omega}}

%
% shorthand for unit vectors
%
\newcommand{\acap}[0]{\hat{\Ba}}
\newcommand{\bcap}[0]{\hat{\Bb}}
\newcommand{\ccap}[0]{\hat{\Bc}}
\newcommand{\dcap}[0]{\hat{\Bd}}
\newcommand{\ecap}[0]{\hat{\Be}}
\newcommand{\fcap}[0]{\hat{\Bf}}
\newcommand{\gcap}[0]{\hat{\Bg}}
\newcommand{\hcap}[0]{\hat{\Bh}}
\newcommand{\icap}[0]{\hat{\Bi}}
\newcommand{\jcap}[0]{\hat{\Bj}}
\newcommand{\kcap}[0]{\hat{\Bk}}
\newcommand{\lcap}[0]{\hat{\Bl}}
\newcommand{\mcap}[0]{\hat{\Bm}}
\newcommand{\ncap}[0]{\hat{\Bn}}
\newcommand{\ocap}[0]{\hat{\Bo}}
\newcommand{\pcap}[0]{\hat{\Bp}}
\newcommand{\qcap}[0]{\hat{\Bq}}
\newcommand{\rcap}[0]{\hat{\Br}}
\newcommand{\scap}[0]{\hat{\Bs}}
\newcommand{\tcap}[0]{\hat{\Bt}}
\newcommand{\ucap}[0]{\hat{\Bu}}
\newcommand{\vcap}[0]{\hat{\Bv}}
\newcommand{\wcap}[0]{\hat{\Bw}}
\newcommand{\xcap}[0]{\hat{\Bx}}
\newcommand{\ycap}[0]{\hat{\By}}
\newcommand{\zcap}[0]{\hat{\Bz}}
\newcommand{\thetacap}[0]{\hat{\Btheta}}

%
% to write R^n and C^n in a distinguishable fashion.  Perhaps change this
% to the double lined characters upon figuring out how to do so.
%
\newcommand{\C}[1]{$\mathbb{C}^{#1}$}
\newcommand{\R}[1]{$\mathbb{R}^{#1}$}

%
% various generally useful helpers
%

% derivative of #1 wrt. #2:
\newcommand{\D}[2] {\frac {d#2} {d#1}}

\newcommand{\inv}[1]{\frac{1}{#1}}
\newcommand{\cross}[0]{\times}

\newcommand{\abs}[1]{\lvert{#1}\rvert}
\newcommand{\norm}[1]{\lVert{#1}\rVert}
\newcommand{\innerprod}[2]{\langle{#1}, {#2}\rangle}
\newcommand{\dotprod}[2]{{#1} \cdot {#2}}
\newcommand{\bdotprod}[2]{\left({#1} \cdot {#2}\right)}
\newcommand{\crossprod}[2]{{#1} \cross {#2}}
\newcommand{\tripleprod}[3]{\dotprod{\left(\crossprod{#1}{#2}\right)}{#3}}

\DeclareMathOperator{\Proj}{Proj}
\DeclareMathOperator{\Span}{span}
\DeclareMathOperator{\Sgn}{sgn}
\DeclareMathOperator{\Area}{Area}
\DeclareMathOperator{\Volume}{Volume}

%
% A few miscellaneous things specific to this document
%
\newcommand{\crossop}[1]{\crossprod{#1}{}}

% R2 vector.
\newcommand{\VectorTwo}[2]{
\begin{bmatrix}
 {#1} \\
 {#2}
\end{bmatrix}
}

\newcommand{\VectorN}[1]{
\begin{bmatrix}
{#1}_1 \\
{#1}_2 \\
\vdots \\
{#1}_N \\
\end{bmatrix}
}

\newcommand{\DETuvij}[4]{
\begin{vmatrix}
 {#1}_{#3} & {#1}_{#4} \\
 {#2}_{#3} & {#2}_{#4}
\end{vmatrix}
}

\newcommand{\DETuvwijk}[6]{
\begin{vmatrix}
 {#1}_{#4} & {#1}_{#5} & {#1}_{#6} \\
 {#2}_{#4} & {#2}_{#5} & {#2}_{#6} \\
 {#3}_{#4} & {#3}_{#5} & {#3}_{#6}
\end{vmatrix}
}

\newcommand{\DETuvwxijkl}[8]{
\begin{vmatrix}
 {#1}_{#5} & {#1}_{#6} & {#1}_{#7} & {#1}_{#8} \\
 {#2}_{#5} & {#2}_{#6} & {#2}_{#7} & {#2}_{#8} \\
 {#3}_{#5} & {#3}_{#6} & {#3}_{#7} & {#3}_{#8} \\
 {#4}_{#5} & {#4}_{#6} & {#4}_{#7} & {#4}_{#8} \\
\end{vmatrix}
}

%\newcommand{\DETuvwxyijklm}[10]{
%\begin{vmatrix}
% {#1}_{#6} & {#1}_{#7} & {#1}_{#8} & {#1}_{#9} & {#1}_{#10} \\
% {#2}_{#6} & {#2}_{#7} & {#2}_{#8} & {#2}_{#9} & {#2}_{#10} \\
% {#3}_{#6} & {#3}_{#7} & {#3}_{#8} & {#3}_{#9} & {#3}_{#10} \\
% {#4}_{#6} & {#4}_{#7} & {#4}_{#8} & {#4}_{#9} & {#4}_{#10} \\
% {#5}_{#6} & {#5}_{#7} & {#5}_{#8} & {#5}_{#9} & {#5}_{#10}
%\end{vmatrix}
%}

% R3 vector.
\newcommand{\VectorThree}[3]{
\begin{bmatrix}
 {#1} \\
 {#2} \\
 {#3}
\end{bmatrix}
}



\author{Peeter Joot}
\email{peeter.joot@utoronto.ca, 920798560}
%%
% Copyright � 2015 Peeter Joot.  All Rights Reserved.
% Licenced as described in the file LICENSE under the root directory of this GIT repository.
%
\documentclass[]{eliblog}

\usepackage{amsmath}
\usepackage{mathpazo}

%
% shorthand for bold symbols, convenient for vectors and matrices
%
\newcommand{\Ba}[0]{\mathbf{a}}
\newcommand{\Bb}[0]{\mathbf{b}}
\newcommand{\Bc}[0]{\mathbf{c}}
\newcommand{\Bd}[0]{\mathbf{d}}
\newcommand{\Be}[0]{\mathbf{e}}
\newcommand{\Bf}[0]{\mathbf{f}}
\newcommand{\Bg}[0]{\mathbf{g}}
\newcommand{\Bh}[0]{\mathbf{h}}
\newcommand{\Bi}[0]{\mathbf{i}}
\newcommand{\Bj}[0]{\mathbf{j}}
\newcommand{\Bk}[0]{\mathbf{k}}
\newcommand{\Bl}[0]{\mathbf{l}}
\newcommand{\Bm}[0]{\mathbf{m}}
\newcommand{\Bn}[0]{\mathbf{n}}
\newcommand{\Bo}[0]{\mathbf{o}}
\newcommand{\Bp}[0]{\mathbf{p}}
\newcommand{\Bq}[0]{\mathbf{q}}
\newcommand{\Br}[0]{\mathbf{r}}
\newcommand{\Bs}[0]{\mathbf{s}}
\newcommand{\Bt}[0]{\mathbf{t}}
\newcommand{\Bu}[0]{\mathbf{u}}
\newcommand{\Bv}[0]{\mathbf{v}}
\newcommand{\Bw}[0]{\mathbf{w}}
\newcommand{\Bx}[0]{\mathbf{x}}
\newcommand{\By}[0]{\mathbf{y}}
\newcommand{\Bz}[0]{\mathbf{z}}
\newcommand{\BA}[0]{\mathbf{A}}
\newcommand{\BB}[0]{\mathbf{B}}
\newcommand{\BC}[0]{\mathbf{C}}
\newcommand{\BD}[0]{\mathbf{D}}
\newcommand{\BE}[0]{\mathbf{E}}
\newcommand{\BF}[0]{\mathbf{F}}
\newcommand{\BG}[0]{\mathbf{G}}
\newcommand{\BH}[0]{\mathbf{H}}
\newcommand{\BI}[0]{\mathbf{I}}
\newcommand{\BJ}[0]{\mathbf{J}}
\newcommand{\BK}[0]{\mathbf{K}}
\newcommand{\BL}[0]{\mathbf{L}}
\newcommand{\BM}[0]{\mathbf{M}}
\newcommand{\BN}[0]{\mathbf{N}}
\newcommand{\BO}[0]{\mathbf{O}}
\newcommand{\BP}[0]{\mathbf{P}}
\newcommand{\BQ}[0]{\mathbf{Q}}
\newcommand{\BR}[0]{\mathbf{R}}
\newcommand{\BS}[0]{\mathbf{S}}
\newcommand{\BT}[0]{\mathbf{T}}
\newcommand{\BU}[0]{\mathbf{U}}
\newcommand{\BV}[0]{\mathbf{V}}
\newcommand{\BW}[0]{\mathbf{W}}
\newcommand{\BX}[0]{\mathbf{X}}
\newcommand{\BY}[0]{\mathbf{Y}}
\newcommand{\BZ}[0]{\mathbf{Z}}

\newcommand{\Bzero}[0]{\mathbf{0}}
\newcommand{\Btheta}[0]{\boldsymbol{\theta}}
\newcommand{\Btau}[0]{\boldsymbol{\tau}}
\newcommand{\Bomega}[0]{\boldsymbol{\omega}}

%
% shorthand for unit vectors
%
\newcommand{\acap}[0]{\hat{\Ba}}
\newcommand{\bcap}[0]{\hat{\Bb}}
\newcommand{\ccap}[0]{\hat{\Bc}}
\newcommand{\dcap}[0]{\hat{\Bd}}
\newcommand{\ecap}[0]{\hat{\Be}}
\newcommand{\fcap}[0]{\hat{\Bf}}
\newcommand{\gcap}[0]{\hat{\Bg}}
\newcommand{\hcap}[0]{\hat{\Bh}}
\newcommand{\icap}[0]{\hat{\Bi}}
\newcommand{\jcap}[0]{\hat{\Bj}}
\newcommand{\kcap}[0]{\hat{\Bk}}
\newcommand{\lcap}[0]{\hat{\Bl}}
\newcommand{\mcap}[0]{\hat{\Bm}}
\newcommand{\ncap}[0]{\hat{\Bn}}
\newcommand{\ocap}[0]{\hat{\Bo}}
\newcommand{\pcap}[0]{\hat{\Bp}}
\newcommand{\qcap}[0]{\hat{\Bq}}
\newcommand{\rcap}[0]{\hat{\Br}}
\newcommand{\scap}[0]{\hat{\Bs}}
\newcommand{\tcap}[0]{\hat{\Bt}}
\newcommand{\ucap}[0]{\hat{\Bu}}
\newcommand{\vcap}[0]{\hat{\Bv}}
\newcommand{\wcap}[0]{\hat{\Bw}}
\newcommand{\xcap}[0]{\hat{\Bx}}
\newcommand{\ycap}[0]{\hat{\By}}
\newcommand{\zcap}[0]{\hat{\Bz}}
\newcommand{\thetacap}[0]{\hat{\Btheta}}

%
% to write R^n and C^n in a distinguishable fashion.  Perhaps change this
% to the double lined characters upon figuring out how to do so.
%
\newcommand{\C}[1]{$\mathbb{C}^{#1}$}
\newcommand{\R}[1]{$\mathbb{R}^{#1}$}

%
% various generally useful helpers
%

% derivative of #1 wrt. #2:
\newcommand{\D}[2] {\frac {d#2} {d#1}}

\newcommand{\inv}[1]{\frac{1}{#1}}
\newcommand{\cross}[0]{\times}

\newcommand{\abs}[1]{\lvert{#1}\rvert}
\newcommand{\norm}[1]{\lVert{#1}\rVert}
\newcommand{\innerprod}[2]{\langle{#1}, {#2}\rangle}
\newcommand{\dotprod}[2]{{#1} \cdot {#2}}
\newcommand{\bdotprod}[2]{\left({#1} \cdot {#2}\right)}
\newcommand{\crossprod}[2]{{#1} \cross {#2}}
\newcommand{\tripleprod}[3]{\dotprod{\left(\crossprod{#1}{#2}\right)}{#3}}

\DeclareMathOperator{\Proj}{Proj}
\DeclareMathOperator{\Span}{span}
\DeclareMathOperator{\Sgn}{sgn}
\DeclareMathOperator{\Area}{Area}
\DeclareMathOperator{\Volume}{Volume}

%
% A few miscellaneous things specific to this document
%
\newcommand{\crossop}[1]{\crossprod{#1}{}}

% R2 vector.
\newcommand{\VectorTwo}[2]{
\begin{bmatrix}
 {#1} \\
 {#2}
\end{bmatrix}
}

\newcommand{\VectorN}[1]{
\begin{bmatrix}
{#1}_1 \\
{#1}_2 \\
\vdots \\
{#1}_N \\
\end{bmatrix}
}

\newcommand{\DETuvij}[4]{
\begin{vmatrix}
 {#1}_{#3} & {#1}_{#4} \\
 {#2}_{#3} & {#2}_{#4}
\end{vmatrix}
}

\newcommand{\DETuvwijk}[6]{
\begin{vmatrix}
 {#1}_{#4} & {#1}_{#5} & {#1}_{#6} \\
 {#2}_{#4} & {#2}_{#5} & {#2}_{#6} \\
 {#3}_{#4} & {#3}_{#5} & {#3}_{#6}
\end{vmatrix}
}

\newcommand{\DETuvwxijkl}[8]{
\begin{vmatrix}
 {#1}_{#5} & {#1}_{#6} & {#1}_{#7} & {#1}_{#8} \\
 {#2}_{#5} & {#2}_{#6} & {#2}_{#7} & {#2}_{#8} \\
 {#3}_{#5} & {#3}_{#6} & {#3}_{#7} & {#3}_{#8} \\
 {#4}_{#5} & {#4}_{#6} & {#4}_{#7} & {#4}_{#8} \\
\end{vmatrix}
}

%\newcommand{\DETuvwxyijklm}[10]{
%\begin{vmatrix}
% {#1}_{#6} & {#1}_{#7} & {#1}_{#8} & {#1}_{#9} & {#1}_{#10} \\
% {#2}_{#6} & {#2}_{#7} & {#2}_{#8} & {#2}_{#9} & {#2}_{#10} \\
% {#3}_{#6} & {#3}_{#7} & {#3}_{#8} & {#3}_{#9} & {#3}_{#10} \\
% {#4}_{#6} & {#4}_{#7} & {#4}_{#8} & {#4}_{#9} & {#4}_{#10} \\
% {#5}_{#6} & {#5}_{#7} & {#5}_{#8} & {#5}_{#9} & {#5}_{#10}
%\end{vmatrix}
%}

% R3 vector.
\newcommand{\VectorThree}[3]{
\begin{bmatrix}
 {#1} \\
 {#2} \\
 {#3}
\end{bmatrix}
}



\author{Peeter Joot}
\email{peeter.joot@gmail.com}


\chapter{PHY454H1S Continuum mechanics.  Problem Set 3.  Velocity scaling, non-dimensionalisation, boundary layers.}
%\label{chap:continuumProblemSet3}
%\blogpage{http://sites.google.com/site/peeterjoot2/math2012/continuumProblemSet3.pdf}
\date{Mar 23, 2012}
%\gitRevisionInfo{continuumProblemSet3}

\beginArtNoToc
\section{Disclaimer.}

This problem set is as yet ungraded.

\section{Problem Q1.}
\subsection{Background.}

In fluid convection problems one can make several choices for characteristic velocity scales.  Some choices are given below for example:

\begin{enumerate}
\item $U_1 = \frac{g \alpha d^2 \nabla T}{\nu}$
\item $U_2 = \frac{\nu}{d}$
\item $U_3 = \sqrt{ g \alpha d \nabla T }$
\item $U_4 = \kappa/ d$
\end{enumerate}

where $g$ is the acceleration due to gravity, $\alpha = (\PDi{T}{V})/V$ is the coefficient of volume expansion, $d$ length scale associated with the problem, $\nabla T$ is the applied temperature difference, $\nu$ is the kinematic viscosity and $\kappa$ is the thermal diffusivity.  

\begin{enumerate}
\item For $U_1$:

Observing that 

\begin{equation}\label{eqn:continuumProblemSet3:n}
\left[ \PD{t}{\Bu} \right] = [ \nu \spacegrad^2 \Bu ]
\end{equation}

we must have

\begin{equation}\label{eqn:continuumProblemSet3:n}
[\nu] = \inv{[t] [\spacegrad^2]} = \inv{T} L^2
\end{equation}

We also find

\begin{equation}\label{eqn:continuumProblemSet3:n}
[\alpha] = \inv{[V]} \left[ \PD{T}{V} \right] = \inv{[K]},
\end{equation}

so that

\begin{equation}\label{eqn:continuumProblemSet3:n}
[U_1] = \frac{L}{T \cancel{T}} \cancel{\inv{K}} L^2 \cancel{K} \frac{\cancel{T}}{L^2} = \frac{L}{T}
\end{equation}

\item For $U_2$:

\begin{equation}\label{eqn:continuumProblemSet3:n}
[U_2] = \frac{L^2}{T} \inv{L} = \frac{L}{T}.
\end{equation}

\item For $U_3$

\begin{equation}\label{eqn:continuumProblemSet3:n}
[U_3] = \sqrt{ \frac{L}{T^2} \inv{K} L K } = \frac{L}{T}
\end{equation}

\item For $U_4$

According to \href{http://scienceworld.wolfram.com/physics/ThermalDiffusivity.html}{http://scienceworld.wolfram.com/physics/ThermalDiffusivity.html}, the thermal diffusivity is defined by

\begin{equation}\label{eqn:continuumProblemSet3:n}
\PD{t}{T} = \kappa \spacegrad^2 T
\end{equation}

so that 

\begin{equation}\label{eqn:continuumProblemSet3:n}
[\kappa] = \inv{[t][\spacegrad^2]} = \frac{L^2}{T}
\end{equation}

That gives us

\begin{equation}\label{eqn:continuumProblemSet3:n}
[U_4] = \frac{L^2}{T} \inv{L} = \frac{L}{T}.
\end{equation}
\end{enumerate}

We've verified that all of these have dimensions of velocity.

\subsection{Part 1.  Statement.  Check whether the dimensions match in each case above.}
\subsection{Solution.}

\subsection{Part 2.  Statement. Pure liquid.}

For pure liquid, say pure water at room temperature, one has the following estimates in cgs units:

\begin{align*}
\alpha &\sim 10^{-4} \\
\kappa &\sim 10^{-3} \\
\nu &\sim 10^{-2}
\end{align*}

For a $d \sim 1 \text{cm}$ layer depth and a ten degree temperature drop convective velocities have been experimentally measured of about $10^{-2}$.
% \text{cm}/\text{s}  
With $g \sim 10^{-3}$, calculate the values of $U_1$, $U_2$, $U_3$, and $U_4$.  Which ones of the characteristic velocities $(U_1$, $U_2$, $U_3, U_4)$ do you think are suitable for nondimensionalising the velocity in Navier-Stokes/Energy equation describing the water convection problem?

We have

\begin{itemize}
\item 

\begin{equation}\label{eqn:continuumProblemSet3:n}
U_1 \sim 10^{-3} 10^{-4} (1)^2 10^1 \inv{10^{-2}} = 10^{-4}
\end{equation}

\item

\begin{equation}\label{eqn:continuumProblemSet3:n}
U_2 \sim 10^{-2}/1 = 10^{-2}
\end{equation}

\item

\begin{equation}\label{eqn:continuumProblemSet3:n}
U_3 \sim \sqrt{ 10^{-3} 10^{-4} (1) 10^1 } = 10^{-3}
\end{equation}

\item

\begin{equation}\label{eqn:continuumProblemSet3:n}
U_4 \sim 10^{-3}/1 = 10^{-3}
\end{equation}
\end{itemize}

Use of $U_2 = \nu/d$ gives the closest match to the measured characteristic velocity of $10^{-2}$.

\subsection{Solution.}

\subsection{Part 3.  Statement.  Mantle convection.}

For mantle convection, we have

\begin{align*}
\alpha &\sim 10^{-5} \\
\nu &\sim 10^{21} \\
\kappa &\sim 10^{-2} \\
d &\sim 10^8 \\
\nabla T &\sim 10^3,
\end{align*}

and the actual convective mantle velocity is $10^{-8}$.  Which of the characteristic velocities should we use to nondimensionalise Navier-Stokes/Energy equations describing mantle convection?
\subsection{Solution}

Let's compute the characteristic velocities again with the mantle numbers

\begin{align}\label{eqn:continuumProblemSet3:n}
U_1 &\sim \frac{10^{-3} 10^{-5} 10^{16} 10^3 }{10^{21}} = 10^{-10} \\
U_2 &\sim \frac{10^{21}}{10^8} = 10^{13} \\
U_3 &\sim \sqrt{ 10^{-3} 10^{-5} 10^8 10^3 } \sim 10^1 \\
U_4 &\sim \frac{10^{-2}}{10^8} = 10^{-10}
\end{align}

Both $U_1$ and $U_4$ come close to the actual convective mantle velocity of $10^{-8}$.  Use of $U_1$ to nondimensionalise is probably best, since it has more degrees of freedom, and includes the gravity term that is probably important for such large masses.

\section{Problem Q2.}
\subsection{Statement}

Nondimensionalise N-S equation

\begin{equation}\label{eqn:continuumL19:n}
\rho \PD{t}{\Bu} + \rho (\Bu \cdot \spacegrad) \Bu = - \spacegrad p + \mu \spacegrad^2 \Bu + \rho g \zcap
\end{equation}

where $\zcap$ is the unit vector in the $z$ direction.  You may scale:

\begin{itemize}
\item velocity with the characteristic velocity $U$,
\item time with $R/U$, where $R$ is the characteristic length scale,
\item pressure with $\rho U^2$,
\end{itemize}

Reynolds number $\text{Re} = R U \rho/ \mu$ and Froude number $\text{Fr} = g R/U$.

\subsection{Solution}

\section{Problem Q3.}
\subsection{Statement}

In case of Stokes' boundary layer problem (see class note) calculate shear stress on the plate $y = 0$.   What is the phase difference between the velocity of the plate $U(t) = U_0 \cos\omega t$ and the shear stress on the plate?

\subsection{Solution}

We found in class that the velocity of the fluid was given by

\begin{equation}\label{eqn:continuumL19:n}
u(y, t) = U_0 e^{-\lambda y} \cos(\lambda y - \omega t)
\end{equation}

where

\begin{equation}\label{eqn:continuumL19:n}
\lambda = \sqrt{\frac{\omega}{2 \nu}}
\end{equation}

Calculating our shear stress we find

\begin{align*}
\mu \PD{y}{u} 
&= U_0 \lambda \mu e^{-\lambda y}
\left(
-1
-
 \sin(\lambda y - \omega t)
\right)
\end{align*}

and on the plate ($y = 0$) this is just

\begin{equation}\label{eqn:continuumL19:n}
\evalbar{\mu \PD{y}{u}}{y = 0} = U_0 \lambda \mu (-1 + \sin(\omega t)).
\end{equation}

We've got a constant term, plus one that is sinusoidal.  Observing that 

\begin{align}\label{eqn:continuumProblemSet3:n}
\cos x &= \Real( e^{ix} )  \\
\sin x &= \Real( -i e^{ix} ) = \Real( e^{i (x - \pi/2)} ),
\end{align}

The phase difference between the non-constant portions of the shear stress at the plate, and the plate velocity $U(t) = U_0 \cos\omega t$ is just $-\pi/2$.  The shear stress at that points lags by 90 degrees.

%\EndArticle
\EndNoBibArticle
