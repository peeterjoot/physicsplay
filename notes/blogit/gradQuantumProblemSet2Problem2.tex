%
% Copyright � 2015 Peeter Joot.  All Rights Reserved.
% Licenced as described in the file LICENSE under the root directory of this GIT repository.
%
\makeproblem{Virial theorem}{gradQuantum:problemSet2:2}{ 

Consider a three-dimensional Hamiltonian 

\begin{dmath}\label{eqn:gradQuantumProblemSet2Problem2:20}
H = \frac{\Bp^2}{2m} + V(\Bx).
\end{dmath}

Calculate \( \antisymmetric{\Bx \cdot \Bp}{H} \) and show that

\begin{dmath}\label{eqn:gradQuantumProblemSet2Problem2:40}
\ddt{} \expectation{ \Bx \cdot \Bp } = \expectation{ \frac{\Bp^2}{2m} } - \expectation{ \Bx \cdot \spacegrad V }.
\end{dmath}


\makesubproblem{}{gradQuantum:problemSet2:2a}
Under what conditions does the left-hand side vanish?  

\makesubproblem{}{gradQuantum:problemSet2:2b}
When the l.h.s. vanishes, the result that the r.h.s. is zero is called the quantum virial theorem. 
Consider the 3D isotropic harmonic oscillator, and show explicitly that its eigenstates obey the virial theorem. 

\makesubproblem{}{gradQuantum:problemSet2:2c}
Evaluate the r.h.s. for the superposition state \( \ket{0,0,0} + \ket{0,0,2} \) where the notation stands for \( \ket{ n_x, n_y, n_z } \) occupation numbers.

} % makeproblem

%%%%%%%%%%%%%%%%%
%
% ../phy1520/qmVirialTheorem.tex
%
%
\makeanswer{gradQuantum:problemSet2:2}{ 

TODO.
\makeSubAnswer{}{gradQuantum:problemSet2:2a}

TODO.
\makeSubAnswer{}{gradQuantum:problemSet2:2b}

TODO.
\makeSubAnswer{}{gradQuantum:problemSet2:2c}

TODO.
}
