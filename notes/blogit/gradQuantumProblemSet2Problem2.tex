%
% Copyright � 2015 Peeter Joot.  All Rights Reserved.
% Licenced as described in the file LICENSE under the root directory of this GIT repository.
%

% Augmented version of problem 2.7 from the text.
\makeproblem{Virial theorem}{gradQuantum:problemSet2:2}{ 

Consider a three-dimensional Hamiltonian 

\begin{dmath}\label{eqn:gradQuantumProblemSet2Problem2:21}
H = \frac{\Bp^2}{2m} + V(\Bx).
\end{dmath}

\makesubproblem{}{gradQuantum:problemSet2:2a}

Calculate \( \antisymmetric{\Bx \cdot \Bp}{H} \) and show that

\begin{dmath}\label{eqn:gradQuantumProblemSet2Problem2:41}
\ddt{} \expectation{ \Bx \cdot \Bp } = \expectation{ \frac{\Bp^2}{2m} } - \expectation{ \Bx \cdot \spacegrad V }.
\end{dmath}


\makesubproblem{}{gradQuantum:problemSet2:2b}
Under what conditions does the left-hand side vanish?  

\makesubproblem{}{gradQuantum:problemSet2:2c}
When the l.h.s. vanishes, the result that the r.h.s. is zero is called the quantum virial theorem. 
Consider the 3D isotropic harmonic oscillator, and show explicitly that its eigenstates obey the virial theorem. 

\makesubproblem{}{gradQuantum:problemSet2:2d}
Evaluate the r.h.s. for the superposition state \( \ket{0,0,0} + \ket{0,0,2} \) where the notation stands for \( \ket{ n_x, n_y, n_z } \) occupation numbers.

} % makeproblem

%%%%%%%%%%%%%%%%%
%
% from: ../phy1520/gradQuantumProblemSet2Problem2.tex
%
%
\makeanswer{gradQuantum:problemSet2:2}{ 

\makeSubAnswer{}{gradQuantum:problemSet2:2a}

We'll need various commutators to evaluate the Heisenberg equation of motion.  
\begin{dmath}\label{eqn:gradQuantumProblemSet2Problem2:40}
\antisymmetric{\Bx \cdot \Bp}{H}
=
\inv{2 m} \antisymmetric{\Bx \cdot \Bp}{\Bp^2} + \antisymmetric{\Bx \cdot \Bp}{V(\Bx)}
=
\inv{2 m} \lr{ x_r p_r \Bp^2 - \Bp^2 x_r p_r} 
+ 
\lr{ x_r p_r V(\Bx) - V(\Bx) x_r p_r }
=
\inv{2 m} \antisymmetric{ x_r }{\Bp^2} p_r
+ 
x_r \antisymmetric{ p_r}{ V(\Bx)},
\end{dmath}

Evaluating those commutators separately, gives

\begin{dmath}\label{eqn:gradQuantumProblemSet2Problem2:60}
\begin{aligned}
\antisymmetric{ x_r }{\Bp^2}
&=
\antisymmetric{ x_r }{p_r^2}\qquad \text{no sum} \\
&=
2 i \Hbar p_r,
\end{aligned}
\end{dmath}

and

\begin{dmath}\label{eqn:gradQuantumProblemSet2Problem2:80}
\antisymmetric{ p_r}{ V(\Bx)}
= -i \Hbar \PD{x_r}{V(\Bx)},
\end{dmath}

so
\begin{dmath}\label{eqn:gradQuantumProblemSet2Problem2:100}
\ddt{}\lr{\Bx \cdot \Bp}
=
\inv{i \Hbar}
\antisymmetric{\Bx \cdot \Bp}{H}
=
\inv{2 m} 2 p_r p_r - x_r \PD{x_r}{V(\Bx)}
=
\frac{\Bp^2}{m} - \Bx \cdot \spacegrad V(\Bx).
\end{dmath}

Evaluating the expectation of this identity with respect to stationary states (i.e. states that are independent of time), we have

\begin{dmath}\label{eqn:gradQuantumProblemSet2Problem2:120}
\ddt{} \expectation{ \Bx \cdot \Bp } 
= \expectation{\frac{\Bp^2}{m}} - \expectation{\Bx \cdot \spacegrad V(\Bx)}.
\end{dmath}

Note that taking the expectation with respect to stationary states was required to reverse the order of the time derivative and the expectation operation.

\makeSubAnswer{}{gradQuantum:problemSet2:2b}
TODO.

\makeSubAnswer{}{gradQuantum:problemSet2:2c}
When \cref{eqn:gradQuantumProblemSet2Problem2:120} is zero, we have the quantum equivalent of the virial theorem, relating the average kinetic energy to the potential

\begin{dmath}\label{eqn:gradQuantumProblemSet2Problem2:140}
2 \expectation{T} = \expectation{\Bx \cdot \spacegrad V(\Bx)}
\end{dmath}

Let's now evaluate these expectations operations with respect to the 3D SHO eigenstates.

TODO.

\makeSubAnswer{}{gradQuantum:problemSet2:2d}

TODO.
}
