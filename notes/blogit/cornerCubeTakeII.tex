%
% Copyright � 2015 Peeter Joot.  All Rights Reserved.
% Licenced as described in the file LICENSE under the root directory of this GIT repository.
%
\newcommand{\authorname}{Peeter Joot}
\newcommand{\email}{peeterjoot@protonmail.com}
\newcommand{\basename}{FIXMEbasenameUndefined}
\newcommand{\dirname}{notes/FIXMEdirnameUndefined/}

\renewcommand{\basename}{cornerCubeTakeII}
\renewcommand{\dirname}{notes/ece1229/}
%\newcommand{\dateintitle}{}
%\newcommand{\keywords}{}

\newcommand{\authorname}{Peeter Joot}
\newcommand{\onlineurl}{http://sites.google.com/site/peeterjoot2/math2013/\basename.pdf}
\newcommand{\sourcepath}{\dirname\basename.tex}
\newcommand{\generatetitle}[1]{\chapter{#1}}

\newcommand{\vcsinfo}{%
\section*{}
\noindent{\color{DarkOliveGreen}{\rule{\linewidth}{0.1mm}}}
\paragraph{Document version}
%\paragraph{\color{Maroon}{Document version}}
{
\small
\begin{itemize}
\item Available online at:\\ 
\href{\onlineurl}{\onlineurl}
\item Git Repository: \input{./.revinfo/gitRepo.tex}
\item Source: \sourcepath
\item last commit: \input{./.revinfo/gitCommitString.tex}
\item commit date: \input{./.revinfo/gitCommitDate.tex}
\end{itemize}
}
}

%\PassOptionsToPackage{dvipsnames,svgnames}{xcolor}
\PassOptionsToPackage{square,numbers}{natbib}
\documentclass{scrreprt}

\usepackage[left=2cm,right=2cm]{geometry}
\usepackage[svgnames]{xcolor}
\usepackage{peeters_layout}

\usepackage{natbib}

\usepackage[
colorlinks=true,
bookmarks=false,
pdfauthor={\authorname, \email},
backref 
]{hyperref}

% http://tex.stackexchange.com/questions/75773/how-to-reference-problems-by-the-text-label-in-an-exercise-envioronment
\usepackage[english]{cleveref}
\crefname{Exercise}{exercise}{exercises}
\Crefname{Exercise}{Exercise}{Exercises}

\RequirePackage{titlesec}
\RequirePackage{ifthen}

% http://stackoverflow.com/questions/4932910/date-in-the-tabular-environment
\makeatletter
\let\insertdate\@date
\makeatother

\titleformat{\chapter}[display]
{\bfseries\Large}
{\color{DarkSlateGrey}\filleft \authorname
\ifthenelse{\isundefined{\studentnumber}}{}{\\ \studentnumber}
\ifthenelse{\isundefined{\email}}{}{\\ \email}
\ifthenelse{\isundefined{\dateintitle}}{}{\\ \insertdate}
%\ifthenelse{\isundefined{\coursename}}{}{\\ \coursename} % put in title instead.
}
{4ex}
{\color{DarkOliveGreen}{\titlerule}\color{Maroon}
\vspace{2ex}%
\filright}
[\vspace{2ex}%
\color{DarkOliveGreen}\titlerule
]

\newcommand{\beginArtWithToc}[0]{\begin{document}\tableofcontents}
\newcommand{\beginArtNoToc}[0]{\begin{document}}
\newcommand{\EndNoBibArticle}[0]{\end{document}}
\newcommand{\EndArticle}[0]{\bibliography{Bibliography}\bibliographystyle{plainnat}\end{document}}

% 
%\newcommand{\citep}[1]{\cite{#1}}

\colorSectionsForArticle



\usepackage{peeters_layout_exercise}
%\usepackage{macros_mathematica}
\usepackage{ece1229}
\usepackage{siunitx}
\usepackage{esint} % \oiint

\renewcommand{\QuestionNB}{\alph{Question}.\ }
\renewcommand{\theQuestion}{\alph{Question}}

\beginArtNoToc

\generatetitle{Corner cube image factor}
%\chapter{Corner cube image factor}
%\label{chap:cornerCubeTakeII}

\makeproblem{Corner cube antenna.}{advancedantenna:problemSet3TakeII:3}{ 

Consider a symmetrically placed horizontal dipole antenna, next to a metallic corner cube.

\imageFigure{../../figures/ece1229/homework3Fig1}{A corner-cube antenna.}{fig:homework3:homework3Fig1}{0.2}

\makesubproblem{}{advancedantenna:problemSet3TakeII:3c}

Calculate the array factor of the antenna in \cref{fig:homework3:homework3Fig1}.

\makesubproblem{}{advancedantenna:problemSet3TakeII:3a}

Estimate the directivity enhancement of the antenna in \cref{fig:homework3:homework3Fig1} compared to the isolated antenna.

\makesubproblem{}{advancedantenna:problemSet3TakeII:3b}

Estimate the radiation resistance of the antenna in \cref{fig:homework3:homework3Fig1} compared to the isolated antenna.


\makesubproblem{}{advancedantenna:problemSet3TakeII:3d}

Plot the array-factor directivity pattern in the x-y plane for \( 0 < \phi \le 2 \pi \).

\makesubproblem{}{advancedantenna:problemSet3TakeII:3e}

By using numerical integration calculate the directivity of the array factor for \( h = (1/8) \lambda, h = (1/4) \lambda \) and \( h = (1/2) \lambda \).

} % makeproblem

\makeanswer{advancedantenna:problemSet3TakeII:3}{ 

\makeSubAnswer{}{advancedantenna:problemSet3TakeII:3c}

This problem can be tackled with the image theorem, which requires placement of sources as in \cref{fig:cornerCubeImageSourcePlacement:cornerCubeImageSourcePlacementFig3}.

% FIXME: duplicate.
\imageFigure{../../figures/ece1229/cornerCubeImageSourcePlacementFig3}{Correct image source placement for the corner cube.}{fig:cornerCubeImageSourcePlacement:cornerCubeImageSourcePlacementFig3}{0.2}

The sources are located one in each quadrant

\begin{equation}\label{eqn:cornerCubeTakeII:800}
\begin{aligned}
\Bs_1 &= h \lr{ 1,1,0} \\
\Bs_2 &= h \lr{ -1,1,0} \\
\Bs_3 &= h \lr{ -1,-1,0} \\
\Bs_4 &= h \lr{ 1,-1,0}
\end{aligned}
\end{equation}

and the point of measurement at \( \Br = r \rcap = r ( \sin\theta \cos\phi, \sin\theta \sin\phi, \cos\theta ) \).  If \( \Br_m = \Br - \Bs_m \) is the distance from the \( m \)th source to the observation point, then the squared distance is

\begin{dmath}\label{eqn:cornerCubeTakeII:840}
r_m 
= \Abs{ \Br - \Bs_m }
= \lr{ r^2 + \Bs_m^2 - 2 \Br \cdot \Bs_m }^{1/2}
= r \lr{ 1 + \frac{\Bs_m^2}{r^2} - 2 \frac{\rcap}{r} \cdot \Bs_m }^{1/2}
\approx r \lr{ 1 + \inv{2} \frac{\Bs_m^2}{r^2} - \frac{\rcap}{r} \cdot \Bs_m }
=
r + \inv{2} \frac{\Bs_m^2}{r} - \rcap \cdot \Bs_m
\approx
r - \rcap \cdot \Bs_m.
\end{dmath}

Those distances are
\begin{subequations}
\label{eqn:cornerCubeTakeII:860}
\begin{equation}\label{eqn:cornerCubeTakeII:880}
\rcap \cdot \Bs_1 = \sin\theta \lr{ \cos\phi+ \sin\phi } = \sqrt{2} \sin\theta \cos\lr{ \phi - \pi/4 }
\end{equation}
\begin{equation}\label{eqn:cornerCubeTakeII:900}
\rcap \cdot \Bs_2 = \sin\theta \lr{ -\cos\phi+ \sin\phi } = -\sqrt{2} \sin\theta \cos\lr{ \phi + \pi/4 }
\end{equation}
\begin{equation}\label{eqn:cornerCubeTakeII:920}
\rcap \cdot \Bs_3 = -\sin\theta \lr{ \cos\phi+ \sin\phi } = -\sqrt{2} \sin\theta \cos\lr{ \phi - \pi/4 }
\end{equation}
\begin{equation}\label{eqn:cornerCubeTakeII:940}
\rcap \cdot \Bs_4 = \sin\theta \lr{ \cos\phi- \sin\phi } = \sqrt{2} \sin\theta \cos\lr{ \phi + \pi/4 }
\end{equation}
\end{subequations}

Suppose the magnetic vector potential has the structure of an infinitesimal dipole

\begin{equation}\label{eqn:cornerCubeTakeII:20}
\BA_m = \frac{\mu_0 I_0 }{4 \pi r_m} e^{-j k r_m} \zcap.
\end{equation}

In the far field, the direction vectors for all the fields will be approximately

\begin{dmath}\label{eqn:cornerCubeTakeII:60}
\zcap - \lr{ \zcap \cdot \rcap} \rcap
=
\cos\theta \rcap - \sin\theta \thetacap - \cos\theta \rcap
= 
- \sin\theta \thetacap.
\end{dmath}

The far field electric field for each image source is approximately

\begin{dmath}\label{eqn:cornerCubeTakeII:80}
\BE_m
= -j \omega \BA_T
=  j \omega \frac{\mu_0 I_0 }{4 \pi r} e^{-j k r_m} \sin\theta \thetacap
=  j \eta k \frac{ I_0 }{4 \pi r} e^{-j k r_m} \sin\theta \thetacap.
\end{dmath}

The superposition of all the image sources is

\begin{dmath}\label{eqn:cornerCubeTakeII:480}
\BE
=  j \eta k \frac{ I_0 }{4 \pi r} e^{-j k r} \sin\theta 
\lr{ 
e^{j k h \sqrt{2} \sin\theta \cos\lr{ \phi + \pi/4}}
e^{-j k h \sqrt{2} \sin\theta \cos\lr{ \phi + \pi/4}}
e^{j k h \sqrt{2} \sin\theta \cos\lr{ \phi - \pi/4}}
e^{-j k h \sqrt{2} \sin\theta \cos\lr{ \phi - \pi/4}}
 } \thetacap.
\end{dmath}

Writing \( s = \sqrt{2} h \) for the distance from the origin to each of the image sources, this is

%\begin{equation}\label{eqn:cornerCubeTakeII:960}
\boxedEquation{eqn:cornerCubeTakeII:960}{
\BE
=  2 j \eta k \frac{ I_0 }{4 \pi r} e^{-j k r} \sin\theta 
\cos\lr{j k s \sin\theta \cos\lr{ \phi + \pi/4}}
+\cos\lr{j k s \sin\theta \cos\lr{ \phi - \pi/4}}
}
%\end{equation}

This provides the array factor

%\begin{dmath}\label{eqn:cornerCubeTakeII:500}
\boxedEquation{eqn:cornerCubeTakeII:500}{
\textrm{AF} 
= 2
\lr{
\cos\lr{ k s \sin\theta \cos\lr{ \phi + \pi/4}}
+\cos\lr{ k s \sin\theta \cos\lr{ \phi - \pi/4}}
}.
}
%\end{dmath}

\makeSubAnswer{}{advancedantenna:problemSet3TakeII:3a}

The radiation intensity is

\begin{equation}\label{eqn:cornerCubeTakeII:540}
U 
= \inv{2} \eta \lr{ \frac{k I_0  }{4 \pi} }^2 \sin^2 \theta \Abs{\textrm{AF}}^2
= B_0 \sin^2 \theta 
\Abs{\textrm{AF}}^2.
\end{equation}

This holds for both isolated antenna with \( \textrm{AF} = 1 \), and the corner cube with \( \Abs{\textrm{AF}}^2 \) given by \cref{eqn:cornerCubeTakeII:500}.

For the isolated antenna, the radiation intensity is maximized at \( \theta = \pi/2 \), so

\begin{dmath}\label{eqn:cornerCubeTakeII:560}
D_{0,\textrm{iso}} 
= \frac{4 \pi \times 1}{2 \pi \int_0^\pi \sin^3\theta d\theta}
= \frac{2}{4/3}
= \frac{3}{2}.
\end{dmath}

For the corner cube the maximization problem is trickier.  As an approximation, if \( k s \) is assumed to be small, then all the cosines in \cref{eqn:cornerCubeTakeII:500} are close to unity, so again at \( \theta = \pi/2 \)

\begin{equation}\label{eqn:cornerCubeTakeII:580}
\max \sin^2\theta \Abs{\textrm{AF}}^2 \approx 1 \times 16 = 16.
\end{equation}

Despite this being equal to the maximum radiation intensity of the isolated radiator, the radiated power will be reduced by the corner cube configuration, which should increase the directivity.

That radiated power is

\begin{equation}\label{eqn:cornerCubeTakeII:600}
P_{\textrm{rad}}
=
B_0
\int_0^{\pi/2} d\phi
\int_0^{\pi} d\theta \sin^3 \theta
\Abs{\textrm{AF}}^2
\end{equation}

A first order expansion \( \cos (k h \alpha) \approx 1 - (k h \alpha)^2/2 \) gives

\begin{dmath}\label{eqn:cornerCubeTakeII:620}
\Abs{\textrm{AF}}^2
\approx
1 + (k h)^2 \sin^2 \theta \lr{ 
\cos^2\phi + \sin^2 \phi - (\cos\phi - \sin\phi)^2 
}
=
1 + 2 (k h)^2 \sin^2 \theta \cos\phi \sin\phi.
=
1 + (k h)^2 \sin^2 \theta \sin( 2 \phi ).
\end{dmath}

With 

\begin{dmath}\label{eqn:cornerCubeTakeII:640}
\int_0^{\pi/2} \sin (2\phi) d\phi = 1,
\end{dmath}

and

\begin{dmath}\label{eqn:cornerCubeTakeII:660}
\int_0^{\pi} \sin^5 \theta d\theta = 16/15,
\end{dmath}

the radiated power is

\begin{dmath}\label{eqn:cornerCubeTakeII:680}
P_{\textrm{rad}} \approx B_0 \lr{ \frac{\pi}{2} \frac{4}{3} + (k h)^2 \frac{15}{16}}.
\end{dmath}

Provided that \( k h \ll \sqrt{(2 \pi/3)(16/15)} \approx 1.5 \), the approximate directivity of the corner cube is

\begin{equation}\label{eqn:cornerCubeTakeII:700}
D_{0,\textrm{ccube}} \approx \frac{4 \pi \times 1}{2 \pi/3} = 6,
\end{equation}

a factor of 4 greater than the directivity of the isolated radiator.

\makeSubAnswer{}{advancedantenna:problemSet3TakeII:3b}

The radiation resistance was defined implicitly by the relation

\begin{equation}\label{eqn:cornerCubeTakeII:720}
P_{\textrm{rad}} = \inv{2} \Abs{I_0}^2 R_r,
\end{equation}

so the ratio of radiation resistance will just be the ratio of the radiated powers

\begin{equation}\label{eqn:cornerCubeTakeII:740}
\frac{R_{r,\textrm{ccube}}}{R_{r,\textrm{iso}}}
=
\frac{P_{\textrm{rad},\textrm{ccube}}}{P_{\textrm{rad},\textrm{iso}}}
=
\frac{2 \pi/3}{8 \pi/3}
=
\inv{4}.
\end{equation}

\makeSubAnswer{}{advancedantenna:problemSet3TakeII:3d}

The x-y plane is found at \( \theta = \pi/2 \) where the absolute square array factor is

\begin{equation}\label{eqn:cornerCubeTakeII:760}
\Abs{\textrm{AF}}^2 = 3 
- 2 \cos\lr{ k h \cos\phi }
- 2 \cos\lr{ k h \sin\phi }
+ 2 \cos\lr{ k h (\cos\phi - \sin\phi) }.
\end{equation}

With \( h = \alpha \lambda \), this is

\begin{equation}\label{eqn:cornerCubeTakeII:780}
\Abs{\textrm{AF}}^2 = 3 
- 2 \cos\lr{ 2 \pi \alpha \cos\phi }
- 2 \cos\lr{ 2 \pi \alpha \sin\phi }
+ 2 \cos\lr{ 2 \pi \alpha (\cos\phi - \sin\phi) }.
\end{equation}

A plot against both \( \alpha, \phi \) is found in \cref{fig:arrayFactorXY:arrayFactorXYCorrectedFig4}, which shows that there are generally four lobes for any value of \( s \), except for small values where the pattern has circular symmetry.

% FIXME: was this a plot of AF or AF^2.

\mathImageFigure{../../figures/ece1229/arrayFactorXYCorrectedFig4}{Plot of \( \Abs{\textrm{AF}}^2 \) in XY plane with \( \alpha = h/\lambda \).}{fig:arrayFactorXY:arrayFactorXYCorrectedFig4}{0.3}{ps3:ps3Q3plotsCorrected.nb}

This is also plotted in
\cref{fig:arrayFactorXY:arrayFactorXYCorrectedFig5} for a few selected values of \( h = \alpha \lambda \).

\mathImageFigure{../../figures/ece1229/arrayFactorXYCorrectedFig5}{Polar plot of \( \textrm{AF} \) in XY plane for various values of \( \alpha = h/\lambda \).}{fig:arrayFactorXY:arrayFactorXYCorrectedFig5}{0.3}{ps3:ps3Q3plotsCorrected.nb}

The squared array factor is plotted in \cref{fig:arrayFactorXYSqCorrected:arrayFactorXYSqCorrectedFig5}.

\mathImageFigure{../../figures/ece1229/arrayFactorXYSqCorrectedFig5}{Polar plot of \( \Abs{\textrm{AF}}^2 \) for \( \theta = 0.\)}{fig:arrayFactorXYSqCorrected:arrayFactorXYSqCorrectedFig5}{0.3}{ps3:ps3Q3plotsCorrected.nb}

It's more fun to visualize this in 3D as in, and a manipulate control for visualizing \( \Abs{\textrm{AF}}^2 \) is available at \href{http://goo.gl/IVaiw2}{http://goo.gl/IVaiw2}.

% FIXME: where's my 3D plot?
% 0.645 Fig6

\makeSubAnswer{}{advancedantenna:problemSet3TakeII:3e}

The code for the numerical calculations can be found in \nbref{ps3:ps3Q3plotsCorrected.nb}.  The results are
%The listing of \cref{mat:cornerCubeTakeII:20} shows the code used for the numerical calculation.  The results are

\begin{equation}\label{eqn:cornerCubeTakeII:820}
\begin{aligned}
D_0[h = \lambda/8] &= 5.93207 \\
D_0[h = \lambda/4] &= 9.07849 \\
D_0[h = \lambda/2] &= 18.3207.
\end{aligned}
\end{equation}

}

\section{Mathematica Sources}

The Mathematica code associated with these notes is available under 
\href{https://github.com/peeterjoot/mathematica/tree/master/ece1229/ps3}{ece1229/ps3/}
within the github repository:

git@github.com:peeterjoot/mathematica.git

The notebooks for this problem set are

\input{ps3mathematica.tex}

%The data and figures referenced in these notes were generated with versions not greater than:
%
%FIXME: 
%\begin{itemize}
%\item commit 5641358d1f397dab8a4053f7fb9681b0d532cad6
%\end{itemize}

%\EndArticle
\EndNoBibArticle
