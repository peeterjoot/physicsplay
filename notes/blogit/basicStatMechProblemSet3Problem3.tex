\makeproblem{State counting - spins}{basicStatMech:problemSet3:3}{ 
Consider a toy model of a magnet where the net magnetization arises from electronic spins on each atom which can point in one of only two possible directions - Up/North or Down/South.  If we have a system with $N$ spins, and if the magnetization can only take on values $\pm 1$ (Up = $+1$, Down = $-1$), how many configurations are there which have a total magnetization $m$, where $m = (N_\uparrow - N_\downarrow)/N$ (note that $N_\uparrow + N_\downarrow = N$)?  Simplify this result assuming $N \gg 1$ and a generic $m$ (assume we are not interested in the extreme case of a fully magnetized system where $m = \pm 1$).  Find the value of the magnetization $m$ for which the number of such microscopic states is a maximum.  For $N = 20$, make a numerical plot of the number of states as a function of the magnetization (note: $-1 \le m \le 1$) without making the $N \gg 1$ assumption.
} % makeproblem

\makeanswer{basicStatMech:problemSet3:3}{ 

For the first couple values of $N$, lets enumerate the spin sample spaces, their magnetization.

\paragraph{$N = 1$}

\begin{itemize}
\item $\uparrow$ : $m = 1$
\item $\downarrow$, $m = -1$
\end{itemize}

\paragraph{$N = 2$}

\begin{itemize}
\item $\uparrow \uparrow$ : $m = 1$
\item $\uparrow \downarrow$ : $m = 0$
\item $\downarrow \uparrow$, $m = 0$
\item $\downarrow \downarrow$, $m = -1$
\end{itemize}

\paragraph{$N = 3$}

\begin{itemize}
\item $\uparrow \uparrow \uparrow$ : $m = 1$
\item $\uparrow \uparrow \downarrow$ : $m = 1/3$
\item $\uparrow \downarrow \uparrow$ : $m = 1/3$
\item $\uparrow \downarrow \downarrow$ : $m = -1/3$
\item $\downarrow \uparrow \uparrow$ : $m = 1/3$
\item $\downarrow \uparrow \downarrow$ : $m = -1/3$
\item $\downarrow \downarrow \uparrow$ : $m = -1/3$
\item $\downarrow \downarrow \downarrow$ : $m = 1$
\end{itemize}

The respective multiplicities for $N = 1,2,3$ are $\{1\}$, $\{1, 2, 1\}$, $\{1, 3, 3, 1\}$.  It's clear that these are just the binomial coefficients.  Let's write for the multiplicities

\begin{equation}\label{eqn:basicStatMechProblemSet3Problem3:20}
g(N, m) = \binom{N}{i(m)}
\end{equation}

where $i(m)$ is a function that maps from the magnetization values $m$ to the integers $[0, N]$.  Assuming

\begin{equation}\label{eqn:basicStatMechProblemSet3Problem3:40}
i(m) = a m + b,
\end{equation}

where $i(-1) = 0$ and $i(1) = N$, we solve

\begin{equation}\label{eqn:basicStatMechProblemSet3Problem3:60}
\begin{aligned}
a (-1) + b &= 0 \\
a (1) + b &= N,
\end{aligned}
\end{equation}

so

\begin{equation}\label{eqn:basicStatMechProblemSet3Problem3:80}
i(m) = \frac{N}{2}(m + 1)
\end{equation}

and

\begin{equation}\label{eqn:basicStatMechProblemSet3Problem3:100}
g(N, m) = \binom{N}{\frac{N}{2}(1 + m)} = \frac{N!}{
\left(\frac{N}{2}(1 + m)\right)!
\left(\frac{N}{2}(1 - m)\right)!
}
\end{equation}

From 
\begin{equation}\label{eqn:basicStatMechProblemSet3Problem3:120}
\begin{aligned}
2 m &= N_{\uparrow} - N_{\downarrow} \\
N &= N_{\uparrow} + N_{\downarrow},
\end{aligned}
\end{equation}

we see that this can also be written
\begin{equation}\label{eqn:basicStatMechProblemSet3Problem3:140}
\myBoxed{
g(N, m) = 
\frac{N!}{
(N_{\uparrow})!
(N_{\downarrow})!
}
}
\end{equation}

\paragraph{Simplification for large $N$}

We can think of $g(N, m)$ as an unnormalized probability density function (with a $2^N$ normalization factor), so to find a large $N$ approximation we can apply the central limit theorem.  We need the mean and variance for the $N = 1$ case

\begin{equation}\label{eqn:basicStatMechProblemSet3Problem3:n}
\expecation{m} = \inv{2}(-1) + \inv{2}(+1) = 0
\end{equation}
\begin{equation}\label{eqn:basicStatMechProblemSet3Problem3:n}
\expecation{m^2} = \inv{2}(1) + \inv{2}(1) = 1
\end{equation}

So, by the central limit theorem, a pdf of $g(N, m)/2^N$ would have a limiting value of approximately

\begin{equation}\label{eqn:basicStatMechProblemSet3Problem3:n}
g(N, m) = 2^N \inv{2 \pi N} \exp\left( - \frac{ m^2 }{2 N } \right)
\end{equation}

%This is plotted in \cref{fig:statMechProblemSet3Problem3:statMechProblemSet3Problem3Fig1}.
%
%\imageFigure{statMechProblemSet3Problem3Fig1}{Distribution of number of configurations of $N = 20$ magnets}{fig:statMechProblemSet3Problem3:statMechProblemSet3Problem3Fig1}{0.3}
}
