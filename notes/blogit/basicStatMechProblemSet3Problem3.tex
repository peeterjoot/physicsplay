\makeproblem{State counting - spins}{basicStatMech:problemSet3:3}{ 
Consider a toy model of a magnet where the net magnetization arises from electronic spins on each atom which can point in one of only two possible directions - Up/North or Down/South.  If we have a system with $N$ spins, and if the magnetization can only take on values $\pm 1$ (Up = $+1$, Down = $-1$), how many configurations are there which have a total magnetization $m$, where $m = (N_\uparrow + N_\downarrow)/(2N)$ (note that $N_\uparrow + N_\downarrow = N$)?  Simplify this result assuming $N \gg 1$ and a generic $m$ (assume we are not interested in the extreme case of a fully magnetized system where $m = \pm 1$).  Find the value of the magnetization $m$ for which the number of such microscopic states is a maximum.  For $N = 20$, make a numerical plot of the number of states as a function of the magnetization (note: $-1 \le m \le 1$) without making the $N \gg 1$ assumption.
} % makeproblem

\makeanswer{basicStatMech:problemSet3:3}{ 

TODO.
}

