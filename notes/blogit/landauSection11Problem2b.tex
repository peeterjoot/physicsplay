%
% Copyright � 2013 Peeter Joot.  All Rights Reserved.
% Licenced as described in the file LICENSE under the root directory of this GIT repository.
%
% pick one:
%\newcommand{\authorname}{Peeter Joot}
\newcommand{\email}{peeter.joot@utoronto.ca}
\newcommand{\studentnumber}{920798560}
\newcommand{\basename}{FIXMEbasenameUndefined}
\newcommand{\dirname}{notes/FIXMEdirnameUndefined/}

\newcommand{\authorname}{Peeter Joot}
\newcommand{\email}{peeterjoot@protonmail.com}
\newcommand{\basename}{FIXMEbasenameUndefined}
\newcommand{\dirname}{notes/FIXMEdirnameUndefined/}

\renewcommand{\basename}{landauSection11Problem2b}
\renewcommand{\dirname}{notes/classicalmechanics/}
\renewcommand{\email}{}
\renewcommand{\authorname}{Alexandre L\'eonard, Peeter Joot}
%\newcommand{\dateintitle}{}
%\newcommand{\keywords}{}

\newcommand{\authorname}{Peeter Joot}
\newcommand{\onlineurl}{http://sites.google.com/site/peeterjoot2/math2013/\basename.pdf}
\newcommand{\sourcepath}{\dirname\basename.tex}
\newcommand{\generatetitle}[1]{\chapter{#1}}

\newcommand{\vcsinfo}{%
\section*{}
\noindent{\color{DarkOliveGreen}{\rule{\linewidth}{0.1mm}}}
\paragraph{Document version}
%\paragraph{\color{Maroon}{Document version}}
{
\small
\begin{itemize}
\item Available online at:\\ 
\href{\onlineurl}{\onlineurl}
\item Git Repository: \input{./.revinfo/gitRepo.tex}
\item Source: \sourcepath
\item last commit: \input{./.revinfo/gitCommitString.tex}
\item commit date: \input{./.revinfo/gitCommitDate.tex}
\end{itemize}
}
}

%\PassOptionsToPackage{dvipsnames,svgnames}{xcolor}
\PassOptionsToPackage{square,numbers}{natbib}
\documentclass{scrreprt}

\usepackage[left=2cm,right=2cm]{geometry}
\usepackage[svgnames]{xcolor}
\usepackage{peeters_layout}

\usepackage{natbib}

\usepackage[
colorlinks=true,
bookmarks=false,
pdfauthor={\authorname, \email},
backref 
]{hyperref}

% http://tex.stackexchange.com/questions/75773/how-to-reference-problems-by-the-text-label-in-an-exercise-envioronment
\usepackage[english]{cleveref}
\crefname{Exercise}{exercise}{exercises}
\Crefname{Exercise}{Exercise}{Exercises}

\RequirePackage{titlesec}
\RequirePackage{ifthen}

% http://stackoverflow.com/questions/4932910/date-in-the-tabular-environment
\makeatletter
\let\insertdate\@date
\makeatother

\titleformat{\chapter}[display]
{\bfseries\Large}
{\color{DarkSlateGrey}\filleft \authorname
\ifthenelse{\isundefined{\studentnumber}}{}{\\ \studentnumber}
\ifthenelse{\isundefined{\email}}{}{\\ \email}
\ifthenelse{\isundefined{\dateintitle}}{}{\\ \insertdate}
%\ifthenelse{\isundefined{\coursename}}{}{\\ \coursename} % put in title instead.
}
{4ex}
{\color{DarkOliveGreen}{\titlerule}\color{Maroon}
\vspace{2ex}%
\filright}
[\vspace{2ex}%
\color{DarkOliveGreen}\titlerule
]

\newcommand{\beginArtWithToc}[0]{\begin{document}\tableofcontents}
\newcommand{\beginArtNoToc}[0]{\begin{document}}
\newcommand{\EndNoBibArticle}[0]{\end{document}}
\newcommand{\EndArticle}[0]{\bibliography{Bibliography}\bibliographystyle{plainnat}\end{document}}

% 
%\newcommand{\citep}[1]{\cite{#1}}

\colorSectionsForArticle



\beginArtNoToc

\generatetitle{A period determination problem from Landau and Lifshitz}
%\chapter{A period determination problem from Landau and Lifshitz}
\label{chap:landauSection11Problem2b}

\makeproblem{description}{pr:landauSection11Problem2b:1}{A period problem from \citep{landau1960classical} \S 11 (problem 2b)}{ 

Determine the period of oscillation, as a function of the energy, when a particle of mass $m$ moves in a field for which the potential energy is 

\begin{equation}\label{eqn:landauSection11Problem2b:20}
U = U_0 \tan^2\alpha x
\end{equation}

} % makeproblem

\makeanswer{pr:landauSection11Problem2b:1}{ 

The answer is (according to Landau and Lifshitz):

\begin{equation}\label{eqn:landauSection11Problem2b:40}
T = \frac{\pi}{\alpha}\sqrt{\frac{2m}{E+U_0}}
\end{equation}

Starting from the formula

\begin{equation}\label{eqn:landauSection11Problem2b:60}
T(E) = \sqrt{2m}\int_{x_1(E)}^{x_2(E)}\frac{dx}{\sqrt{E-U(x)}},
\end{equation}

where $x_1(E)$ and $x_2(E)$ are the limits of the motion. From the symmetry of our potential, it clear that we have:

\begin{equation}\label{eqn:landauSection11Problem2b:80}
T(E) = 2\sqrt{2m}\int_{0}^{x_2(E)}\frac{dx}{\sqrt{E-U(x)}}.
\end{equation}

Now we can easily find $x_2(E)$:

\begin{equation}\label{eqn:landauSection11Problem2b:100}
E = U_0\tan^2\alpha x_2 \quad \rightarrow \quad x_2 = \frac{1}{\alpha}\Atan{\sqrt{\frac{E}{U_0}}},
\end{equation}

and we are left with the following integral:

\begin{equation}\label{eqn:landauSection11Problem2b:120}
T(E) = 2\sqrt{2m}\int_{0}^{\frac{1}{\alpha}\Atan{\sqrt{\frac{E}{U_0}}}}\frac{dx}{\sqrt{E-U_0\tan^2\alpha x}}.
\end{equation}

First obvious change of variable is $y = \alpha x ~ \rightarrow ~ dy = \alpha dx$ which gives

\begin{equation}\label{eqn:landauSection11Problem2b:140}
T(E) = \frac{2}{\alpha}\sqrt{\frac{2m}{E}}\int_{0}^{\Atan{\sqrt{\frac{E}{U_0}}}}\frac{dy}{\sqrt{1 - \frac{U_0}{E}\tan^2 y}}.
\end{equation}

Now write $a^2 = U_0/E$ and make a change of variables that will eliminate the square root, yielding a cosine

\begin{equation}\label{eqn:landauSection11Problem2b:160}
a \tan y = \sin z.
\end{equation}

\begin{equation}\label{eqn:landauSection11Problem2b:180}
\cos z dz
=
a \sec^2 y dy 
\end{equation}

\begin{equation}\label{eqn:landauSection11Problem2b:200}
I 
= 
\int \frac{dy}{\sqrt{1 - \frac{U_0}{E}\tan^2 y}}
=
\inv{a} \int \frac{\cos^2 y}{\cos z} dz.
\end{equation}

From \ref{eqn:landauSection11Problem2b:160}

\begin{equation}\label{eqn:landauSection11Problem2b:220}
a^2 (1 - \cos^2 y) = \sin^2 z \cos^2 y,
\end{equation}

or

\begin{equation}\label{eqn:landauSection11Problem2b:240}
\frac{a^2}{a^2 + \sin^2 z } = \cos^2 y.
\end{equation}

Yielding 

\begin{equation}\label{eqn:landauSection11Problem2b:260}
I
=
a \int \frac{dz}{(a^2 + \sin^2 z)\cos z} 
\end{equation}

With $u = \sin z$, and $du = \cos z dz$ we have

\begin{dmath}\label{eqn:landauSection11Problem2b:280}
I
=
a \int \frac{du}{(a^2 + u^2)(1 - u^2)} 
=
\frac{a}{1 + a^2}
\int du \left(
\inv{a^2 + u^2} 
- \frac{1/2}{u - 1}
+ \frac{1/2}{u + 1}
\right)
=
\frac{a}{1 + a^2}
\left(
\inv{a} \Atan\left( \frac{u}{a} \right) 
- \inv{2} \ln (u - 1)
+ \inv{2} \ln (u + 1)
\right)
=
\frac{a}{1 + a^2}
\left(
\inv{a} \Atan\left( \frac{u}{a} \right) 
+ \inv{2} \ln \frac{u + 1}{u - 1}
\right)
\end{dmath}

Now plug back in $u = \sqrt{U_0/E} \tan \alpha x$.  Does this give the expected answer after simplification?
} % makeanswer

\EndArticle
