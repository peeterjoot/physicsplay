%
% Copyright � 2013 Peeter Joot.  All Rights Reserved.
% Licenced as described in the file LICENSE under the root directory of this GIT repository.
%
\newcommand{\authorname}{Peeter Joot}
\newcommand{\email}{peeterjoot@protonmail.com}
\newcommand{\basename}{FIXMEbasenameUndefined}
\newcommand{\dirname}{notes/FIXMEdirnameUndefined/}

\renewcommand{\basename}{condensedMatterLecture16}
\renewcommand{\dirname}{notes/phy487/}
\newcommand{\keywords}{Condensed matter physics, PHY487H1F}
\newcommand{\authorname}{Peeter Joot}
\newcommand{\onlineurl}{http://sites.google.com/site/peeterjoot2/math2013/\basename.pdf}
\newcommand{\sourcepath}{\dirname\basename.tex}
\newcommand{\generatetitle}[1]{\chapter{#1}}

\newcommand{\vcsinfo}{%
\section*{}
\noindent{\color{DarkOliveGreen}{\rule{\linewidth}{0.1mm}}}
\paragraph{Document version}
%\paragraph{\color{Maroon}{Document version}}
{
\small
\begin{itemize}
\item Available online at:\\ 
\href{\onlineurl}{\onlineurl}
\item Git Repository: \input{./.revinfo/gitRepo.tex}
\item Source: \sourcepath
\item last commit: \input{./.revinfo/gitCommitString.tex}
\item commit date: \input{./.revinfo/gitCommitDate.tex}
\end{itemize}
}
}

%\PassOptionsToPackage{dvipsnames,svgnames}{xcolor}
\PassOptionsToPackage{square,numbers}{natbib}
\documentclass{scrreprt}

\usepackage[left=2cm,right=2cm]{geometry}
\usepackage[svgnames]{xcolor}
\usepackage{peeters_layout}

\usepackage{natbib}

\usepackage[
colorlinks=true,
bookmarks=false,
pdfauthor={\authorname, \email},
backref 
]{hyperref}

% http://tex.stackexchange.com/questions/75773/how-to-reference-problems-by-the-text-label-in-an-exercise-envioronment
\usepackage[english]{cleveref}
\crefname{Exercise}{exercise}{exercises}
\Crefname{Exercise}{Exercise}{Exercises}

\RequirePackage{titlesec}
\RequirePackage{ifthen}

% http://stackoverflow.com/questions/4932910/date-in-the-tabular-environment
\makeatletter
\let\insertdate\@date
\makeatother

\titleformat{\chapter}[display]
{\bfseries\Large}
{\color{DarkSlateGrey}\filleft \authorname
\ifthenelse{\isundefined{\studentnumber}}{}{\\ \studentnumber}
\ifthenelse{\isundefined{\email}}{}{\\ \email}
\ifthenelse{\isundefined{\dateintitle}}{}{\\ \insertdate}
%\ifthenelse{\isundefined{\coursename}}{}{\\ \coursename} % put in title instead.
}
{4ex}
{\color{DarkOliveGreen}{\titlerule}\color{Maroon}
\vspace{2ex}%
\filright}
[\vspace{2ex}%
\color{DarkOliveGreen}\titlerule
]

\newcommand{\beginArtWithToc}[0]{\begin{document}\tableofcontents}
\newcommand{\beginArtNoToc}[0]{\begin{document}}
\newcommand{\EndNoBibArticle}[0]{\end{document}}
\newcommand{\EndArticle}[0]{\bibliography{Bibliography}\bibliographystyle{plainnat}\end{document}}

% 
%\newcommand{\citep}[1]{\cite{#1}}

\colorSectionsForArticle



%\citep{harald2003solid} \S x.y
%\citep{ibach2009solid} \S x.y

%\usepackage{mhchem}
\usepackage[version=3]{mhchem}
\newcommand{\nought}[0]{\circ}
\newcommand{\EF}[0]{\epsilon_{\mathrm{F}}}
\newcommand{\kF}[0]{k_{\mathrm{F}}}

\beginArtNoToc
\generatetitle{PHY487H1F Condensed Matter Physics.  Lecture 16: Tight binding model.  Taught by Prof.\ Stephen Julian}
%\chapter{Tight binding model}
\label{chap:condensedMatterLecture16}

%\section{Disclaimer}
%
%Peeter's lecture notes from class.  May not be entirely coherent.

\section{Tight binding model}

\paragraph{Reading:} \citep{ibach2009solid} \S 7.3

Assume a periodic lattice with large lattice parameter $a$, so that atomic poential $V_a(\Br - \Br_n)$ and wave functions $\phi(\Br- \Br_n)$ don't overlap much between neighbours.  The potential $\phi_i(\Br - \Br_n)$ satisfies

\begin{dmath}\label{eqn:condensedMatterLecture16:n}
\hat{H}_A \phi_i(\Brn) = E_i \phi(\Brn)
\end{dmath}

where 

\begin{dmath}\label{eqn:condensedMatterLecture16:n}
\Ha = - \hb2m \spacegrad + V_A(\Brn)
\end{dmath}

F1

The one electron Hamiltonian

\begin{dmath}\label{eqn:condensedMatterLecture16:n}
\Hc 
= -\hb2m \spacegrad^2 + \sum_{n'} V_A(\Brn')
= -\hb2m \spacegrad^2 + V_A(\Brn) + \sum_{n' \ne n} V_A(\Brn')
= \Ha(\Brn) + v(\Brn)
\end{dmath}

Look for a solution that is a Linear Combination of Atomic Orbitals (LCAO).

\begin{dmath}\label{eqn:condensedMatterLecture16:n}
\Psi_\Bk(\Br) \sim \Phi_\Bk(\Br) 
= \sum_n a_n \phi_i(\Brn)
= \sum_n e^{i \Bk \cdot \Br_n} \phi_i(\Brn),
\end{dmath}

so by Bloch's theorem

\begin{dmath}\label{eqn:condensedMatterLecture16:n}
\Phi_\Bk(\Br) = U_\Bk e^{i \Bk \cdot \Br},
\end{dmath}

implies

\begin{dmath}\label{eqn:condensedMatterLecture16:n}
\Phi_\Bk(\Br +\Br_m) = \Phi_\Bk(\Br) e^{i \Bk \cdot \Br_m},
\end{dmath}

\begin{dmath}\label{eqn:condensedMatterLecture16:n}
\Phi_\Bk(\Br +\Br_m) 
= 
\sum_n
e^{i \Bk \cdot \Br_n}
\phi_i( \Br + \Br_m - \Br_n)
= 
e^{i \Bk \cdot \Br_m}
\sum_n
e^{i \Bk \cdot (\Br_n - \Br_m)}
\phi_i( \Br + (\Br_n - \Br_m))
\end{dmath}

\begin{dmath}\label{eqn:condensedMatterLecture16:n}
\Phi_{\Bk + \BG} = \sum_n e^{i \Bk \cdot \Br_n} 
\mathLabelBox
{
e^{i \BG \cdot \Br_n}
}
{$1$}
\phi_i(\Br - \Br_n)
= \Phi_\Bk(\Br)
\end{dmath}

\paragraph{Normalization}

Calculate

E(\Bk) = 
\frac{
	\bra{
	\Phi_\Bk}
	\Hc \ket{\Phi_\Bk}
}
{
	\braket{\Phi_\Bk}{\Phi_\Bk}
}.

With

\begin{dmath}\label{eqn:condensedMatterLecture16:n}
\braket{\Phi_\Bk}{\Phi_\Bk}
 = 
\sum_{n, m} e^{i \Bk \cdot (\Br_n - \Br_m)}
\times
\int d\Br 
\phi_i^\conj( \Brm)
\phi_i( \Brn)
=
\left\{
\begin{array}{l l}
1 & \quad \mbox{if $m = n$} \\
0 & \quad \mbox{otherwise} 
\end{array}
\right.
\approx N
\end{dmath}

\begin{dmath}\label{eqn:condensedMatterLecture16:n}
E(\Bk) \approx \inv{N} 
\sum_{n, m} e^{i \Bk \cdot (\Br_n - \Br_m)}
\int d\Br 
\phi_i^\conj( \Brm)
(
\mathLabelBox{\Ha(\Brn) + v(\Brn)}{exact}
)
\phi_i( \Brn)
\approx
\sum_{n, m} e^{i \Bk \cdot (\Br_n - \Br_m)}
\int d\Br 
\phi_i^\conj( \Brm)
\lr{
E_i
+ v(\Brn)
}
\phi_i( \Brn)
\end{dmath}

In the integral we have from $\Ha$, a value of $E_i$ if $m = n$f, and zero otherwise.  For the $v$ contribution to the integral, we have

\begin{itemize}
\item $m = n$.  Large $\phi_i^\conj( \Brm) \phi_i( \Brn)$.  We've got $v(\Brn)$ small near $\Br_n$.
\item $m = n \pm 1$.  Near $\Br_m$, $\phi_i$ and $v$ are both large, and $\phi_i(\Brn)$ is small.
\end{itemize}

In short we have to keep both terms.

With

\begin{dmath}\label{eqn:condensedMatterLecture16:n}
\begin{aligned}
- A &= 
\int d\Br 
\phi_i^\conj( \Brn)
+ v(\Brn)
\phi_i( \Brn)  \\
- B &= 
\int d\Br 
\phi_i^\conj( \Brm)
+ v(\Brn)
\phi_i( \Brn) 
\end{aligned}
\end{dmath}

So

\begin{dmath}\label{eqn:condensedMatterLecture16:n}
E(\Bk) \approx \inv{N}
\lr{
\sum_n E_i - \sum_n A - \sum_{n, m} e^{i \Bk \cdot (\Br_n - \Br_m)} B
},
\end{dmath}

or

\begin{dmath}\label{eqn:condensedMatterLecture16:n}
E(\Bk) \approx E_i - A - B \sum_{m = nn of n} e^{i \Bk \cdot (\Br_n - \Br_m)}
\end{dmath}

In one dimension

\makeexample{1d lattice}{example:condensedMatterLecture16:1}{

For 1D we have

\begin{dmath}\label{eqn:condensedMatterLecture16:n}
\Br_n - \Br_m = \pm a,
\end{dmath}

which implies the nearest neighbour sum is

\begin{dmath}\label{eqn:condensedMatterLecture16:n}
\sum_{nn} \lr{ 
e^{i k a} 
+e^{-i k a} 
}
= 2 \cos k a
\end{dmath}

So

\begin{dmath}\label{eqn:condensedMatterLecture16:n}
E(k) \approx E_i - A - 
\mathLabelBox
{2 B \cos k a
}
{The hopping term}
\end{dmath}

F2

%\EndArticle
\EndNoBibArticle
