%
% Copyright � 2013 Peeter Joot.  All Rights Reserved.
% Licenced as described in the file LICENSE under the root directory of this GIT repository.
%
%\newcommand{\authorname}{Peeter Joot}
\newcommand{\email}{peeterjoot@protonmail.com}
\newcommand{\basename}{FIXMEbasenameUndefined}
\newcommand{\dirname}{notes/FIXMEdirnameUndefined/}

%\renewcommand{\basename}{pathriaCh2P5.tex}
%\renewcommand{\dirname}{notes/phy452/}
%\newcommand{\dateintitle}{}
%\newcommand{\keywords}{Statistical mechanics, PHY452H1S, phase space, one dimensional well, WKB, connection formulas}
%
%\newcommand{\authorname}{Peeter Joot}
\newcommand{\onlineurl}{http://sites.google.com/site/peeterjoot2/math2013/\basename.pdf}
\newcommand{\sourcepath}{\dirname\basename.tex}
\newcommand{\generatetitle}[1]{\chapter{#1}}

\newcommand{\vcsinfo}{%
\section*{}
\noindent{\color{DarkOliveGreen}{\rule{\linewidth}{0.1mm}}}
\paragraph{Document version}
%\paragraph{\color{Maroon}{Document version}}
{
\small
\begin{itemize}
\item Available online at:\\ 
\href{\onlineurl}{\onlineurl}
\item Git Repository: \input{./.revinfo/gitRepo.tex}
\item Source: \sourcepath
\item last commit: \input{./.revinfo/gitCommitString.tex}
\item commit date: \input{./.revinfo/gitCommitDate.tex}
\end{itemize}
}
}

%\PassOptionsToPackage{dvipsnames,svgnames}{xcolor}
\PassOptionsToPackage{square,numbers}{natbib}
\documentclass{scrreprt}

\usepackage[left=2cm,right=2cm]{geometry}
\usepackage[svgnames]{xcolor}
\usepackage{peeters_layout}

\usepackage{natbib}

\usepackage[
colorlinks=true,
bookmarks=false,
pdfauthor={\authorname, \email},
backref 
]{hyperref}

% http://tex.stackexchange.com/questions/75773/how-to-reference-problems-by-the-text-label-in-an-exercise-envioronment
\usepackage[english]{cleveref}
\crefname{Exercise}{exercise}{exercises}
\Crefname{Exercise}{Exercise}{Exercises}

\RequirePackage{titlesec}
\RequirePackage{ifthen}

% http://stackoverflow.com/questions/4932910/date-in-the-tabular-environment
\makeatletter
\let\insertdate\@date
\makeatother

\titleformat{\chapter}[display]
{\bfseries\Large}
{\color{DarkSlateGrey}\filleft \authorname
\ifthenelse{\isundefined{\studentnumber}}{}{\\ \studentnumber}
\ifthenelse{\isundefined{\email}}{}{\\ \email}
\ifthenelse{\isundefined{\dateintitle}}{}{\\ \insertdate}
%\ifthenelse{\isundefined{\coursename}}{}{\\ \coursename} % put in title instead.
}
{4ex}
{\color{DarkOliveGreen}{\titlerule}\color{Maroon}
\vspace{2ex}%
\filright}
[\vspace{2ex}%
\color{DarkOliveGreen}\titlerule
]

\newcommand{\beginArtWithToc}[0]{\begin{document}\tableofcontents}
\newcommand{\beginArtNoToc}[0]{\begin{document}}
\newcommand{\EndNoBibArticle}[0]{\end{document}}
\newcommand{\EndArticle}[0]{\bibliography{Bibliography}\bibliographystyle{plainnat}\end{document}}

% 
%\newcommand{\citep}[1]{\cite{#1}}

\colorSectionsForArticle


%
%\beginArtNoToc
%
%\generatetitle{One dimensional well problem from Pathria chapter II}
%\chapter{One dimensional well problem from Pathria chapter II}
%\label{chap:pathriaCh2P5.tex}

\makeoproblem{Bohr-Somerfelt quantization condition}{pr:pathriaCh2P5:1}{\citep{pathriastatistical}, problem 2.5}{

Show that

\begin{equation}\label{eqn:pathriaCh2P5:20}
\oint p dq = \lr{ n + \inv{2} } h,
\end{equation}

provided the particle's potential is such that

\begin{equation}\label{eqn:pathriaCh2P5:40}
m \Hbar \Abs{ \frac{dV}{dq} } \ll \lr{ m ( E - V ) }^{3/2}.
\end{equation}

} % makeoproblem

\makeanswer{pr:pathriaCh2P5:1}{

I took a guess that this was actually the WKB condition

\begin{equation}\label{eqn:pathriaCh2P5:80}
\frac{k'}{k^2} \ll 1,
\end{equation}

where the WKB solution was of the form

\begin{subequations}
\begin{equation}\label{eqn:pathriaCh2P5:60}
k^2(q) = 2 m (E - V(q))/\Hbar^2
\end{equation}
\begin{equation}\label{eqn:pathriaCh2P5:100}
\psi(q) = \inv{\sqrt{k}} e^{\pm i \int k(q) dq}.
\end{equation}
\end{subequations}

The WKB validity condition is

\begin{equation}\label{eqn:pathriaCh2P5:140}
1 \gg \frac{-2 m V'}{\Hbar} \inv{2} \inv{\sqrt{2 m (E - V)}} \frac{\Hbar^2}{2 m(E - V)}
\end{equation}

or

\begin{equation}\label{eqn:pathriaCh2P5:160}
m \Hbar \Abs{V'} \ll
\lr{2 m (E - V)}^{3/2}.
\end{equation}

This differs by a factor of \(2 \sqrt{2}\) from the constraint specified in the problem, but I'm guessing that constant factors of that sort have just been dropped.

Even after figuring out that this question was referring to WKB, I didn't know what to make of the oriented integral \(\int p dq\).  With \(p\) being an operator in the QM context, what did this even mean.  I found the answer in \citep{bohm1989qt} \S 12.12.  Here \(p\) just means

\begin{equation}\label{eqn:pathriaCh2P5:180}
p(q) = \Hbar k(q),
\end{equation}

where \(k(q)\) is given by \eqnref{eqn:pathriaCh2P5:60}.  The rest of the problem can also be found there and relies on the WKB connection formulas, which aren't derived in any text that I own.  Quoting results based on other results that I don't know the origin of it's worthwhile, so that's as far as I'll attempt this question (but do plan to eventually look up and understand those WKB connection formulas, and then see how they can be applied in a problem like this).
} % makeanswer

%\EndArticle
