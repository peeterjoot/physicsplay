%
% Copyright � 2012 Peeter Joot.  All Rights Reserved.
% Licenced as described in the file LICENSE under the root directory of this GIT repository.
%

%
%
%\chapter{Learning Geometric Algebra/Clifford Algebra}
\chapter{Further reading}
\label{chap:gabookmark}

%\setlength{\textwidth}{13in}
%\setlength{\oddsidemargin}{0in}
%\setlength{\evensidemargin}{0in}

%\section{Learning Geometric Algebra and its applications}

There is a wealth of information on the subject available online, but finding information at an appropriate level may be difficult.  Not all resources use the same notation or nomenclature, and one can get lost in a sea of product operators.
Some of the introductory material also assumes knowledge of various levels of physics.  This is natural since the algebra can be utilized well to expresses many physics concepts.  While natural, this can also be intimidating if one is unprepared, so mathematics that one could potentially understand may be presented
in a fashion that is inaccessible.

%Colected here is an attempt to collect some of the available online information.

\section{Geometric Algebra for Computer Science Book}

The book
\href{http://www.geometricalgebra.net/tour.html}{Geometric Algebra For Computer Science.}
by Dorst, Fontijne, and Mann has one of the best introductions to the subject that I have seen.  It is also fairly inexpensive (\\(60 Canadian).  Compared for example to Hestenes's ``From Clifford Algebra to Geometric Calculus'' which I have seen listed on amazon.com with a default price of \\)250, discounted to \$150.

This book contains particularly good introductions to the dot and wedge products, both for vectors, and the generalizations.  How these can be applied and what they can be used to model is covered excellently.

Compromises have been made in this book on the order to present information, and what level of detail to use and when.  Many proofs are deferred or placed only in the appendix.  For example, they introduce (define) a scalar product initially (denoted with an asterisk (*)), and define this using a determinant without motivation.  This allows for development of a working knowledge of how to apply the subject.

Once an ability to apply has been developed they proceed with an axiomatic development.  I would consider an axiomatic approach to the subject very important since there is a sea of identities associated with the algebra.  Figuring out which ones are consequences of the others can be difficult, if one starts with definitions that are not fundamental.  One can easily go in circles and wonder really are the basic rules (this was my first impression starting with the Hestenes book ``New Foundations for Classical Mechanics''.   The book ``Geometric Algebra for Physicists'' has an excellent axiomatic development.  It however notably makes a similar compromise first introducing the algebra with a dot plus wedge product formulation to develop some familiarity.

This book has three parts.  The first is on the algebra, covering the generalized dot and wedge products, rotors, projections, join, linear transformations as outermorphisms, and all the rest of the basic material that one would expect.  It does this excellently.

The second portion of this book is on the use of a 5D conformal model for 3D graphics (adding a point at infinity on top of the normal extra viewport dimension that traditional graphics applications use).  I can not comment too much on this part of the book since I loaned it to a friend after reading the first and last parts of the book.

The last part of the book is on implementation, and makes for an interesting read.  Details on their Gaigen implementation are discussed, as are performance and code size implications of their implementation.

The only thing negative I have to say about this book is the unfortunate introduction of an alternate notation for the generalized dot product (L and backwards L).  This is distracting if one started, like I did, with the Hestenes, Cambridge, or Baylis papers or books, and their notation dominates the literature as far as I can tell.  This does not take too long to adjust, since one mostly just has to mentally substitute dots for L's (although there are some subtle differences where this transposition does not necessarily work).

\section{GAViewer}

Performing the GAViewer tutorial exercises is a great way to build some intuition to go along with the math (putting the geometric back in the algebra).

There are specific GAViewer exercises that you can do independent of the book, and there is also an excellent interactive tutorial 2003 Game Developer Lecture available here:

\href{http://www.science.uva.nl/ga/tutorials/}{Interactive GA tutorial. UvA GA Website: Tutorials}

 (they have hijacked GAViewer here to use as presentation software, but you can go through things at your own pace, and do things such as rotating viewpoints). Quite neat, and worth doing just to play with the graphical cross product manipulation even if you decide not to learn GA.
\section{Other resources from Dorst, Fontijne, and Mann}

There are other web resources available associated with this book that are quite good. The best of these is GAViewer, a graphical geometric calculator that was the product of some of the research that generated this book.

See
%\href{http://staff.science.uva.nl/~fontijne/phd.html}{Daniel Fontijne PhD thesis}
, or his paper itself
%\href{http://staff.science.uva.nl/~fontijne/phd/fontijne_phd.pdf}{fontijne_phd.pdf}
.

Some other links:

\href{http://staff.science.uva.nl/~leo/clifford/index.html}{Geometric algebra (Clifford algebra)}


This is
\href{http://staff.science.uva.nl/~leo/clifford/dorst-mann-I.pdf}{a good tutorial}
, as it focuses on the geometrical rather than have any tie to physics (fun but more to know).  The following looks like a slightly longer updated version:

\href{http://staff.science.uva.nl/~leo/clifford/dorst-mann-I.pdf}{GA: a practical tool for efficient geometric representation (Dorst)}

\section{Learning GA}
Of the various GA primers and workbooks above, here are a couple specific documents that are noteworthy, and some direct links to a few things that can be found by browsing that were noteworthy.
This is an
\href{http://www.science.uva.nl/ga/tutorials/}{interactive GA tutorial/presentation for a game programmers conference}

that provides a really good intro and has a lot of examples that I found helpful to get an intuitive feel for all the various product operations and object types.
Even if you weare not trying to learn GA, if you have done any traditional vector algebra/calculus, IMO its worthwhile to download this just to just to see the animation of how the old cross product varies with changes to the vectors.
You have to download the GAViewer program (graphical vector calculator) to run the presentation. Once you do that you can use it for other calculation examples, such as those available in these examples of how to use GAViewer as a standalone tool.. Note that the book the drills are from use a different notation for dot product (with a slightly different meaning and uses an oriented L symbol dependent on the grades of the blades.

\href{http://www.lomont.org/Math/GeometricAlgebra/Geometric%20Algebra%20Primer%20-%20Suter%20-%202003.pdf}{Jaap Suter's GA primer}.
\href{http://www.jaapsuter.com/}{His website}, which is referenced in various GA papers no longer (at least obviously) has this primer on it any longer (Sept/2008).

\href{http://www.iancgbell.clara.net/maths/geoalg.htm}{Ian Bell's introduction to GA}

    This author has a wide range of GA information, but looking at it will probably give you a headache.


\href{http://en.wikipedia.org/wiki/Geometric_algebra.}{GA wikipedia}

There are a number of comparisons here between GA identities and traditional vector identities, that may be helpful to get oriented.

- Maths - Clifford / Geometric Algebra - Martin Baker

A GA intro, a small part in the much larger Euclidean space website.

- As mentioned above there is a lot of learning GA content available in the Cambridge/Baylis/Hestenes/Dorst/... sites.

\section{Cambridge}
The Cambridge GA group has a number of Geometric Algebra publications, including the book

\href{http://www.mrao.cam.ac.uk/~cjld1/pages/book.htm}{Geometric Algebra for Physicists}

This book has an excellent introductory treatment of a number of basic GA concepts, a number of which are much easier to follow than similar content in Hestenes's "New Foundations for Classical Mechanics".  When it comes to physics content in this book there are a lot of details left out, so it is not the best for learning the physics itself if you are new to the topic in question.

Much of the content of their book
is actually available online in their publications above, but it is hard to beat coherent organization and a paper version that you can mark up.

Some other online learning content from the Cambridge group includes

\href{http://www.mrao.cam.ac.uk/~clifford/introduction/index.html}{Introduction to Geometric Algebra}

This is an HTML version of the
\href{http://www.mrao.cam.ac.uk/~clifford/publications/ps/imag_numbs.pdf}{Imaginary numbers are not real paper.}

A
nice starting point is lect1.pdf from the
\href{http://www.mrao.cam.ac.uk/~clifford/ptIIIcourse/GeometricAlgebraLectures.zip}{Cambridge PartIII physics course on GA applications}
. Only at the very end of this first pdf is any real physics content.
taught to what sounds like final year undergrad physics students.  The first parts of this do not need much physics knowledge.

\section{Baylis}

\href{http://www.uwindsor.ca/users/b/baylis/main.nsf}{Wiliam Baylis GA page}

He uses a scalar plus vector multivector representation for relativity (APS, Algebra of Physical Space), and an associated conjugate length operation.  You will find an intro relativity, GA workbook, and some papers on GA applied to physics here.
Also based on his APS approach is the following wikibook:

\href{http://en.wikibooks.org/wiki/Physics_in_the_Language_of_Geometric_Algebra._An_Approach_with_the_Algebra_of_Physical_Space}{Physics in the Language of Geometric Algebra. An Approach with the Algebra of Physical Space}

\section{Hestenes}
Hestenes main page for GA is
\href{http://modelingnts.la.asu.edu/}{Geometric Calculus R \& D Home Page}

This includes a number of primers and introductions to the subject such as
\href{http://modelingnts.la.asu.edu/pdf/PrimerGeometricAlgebra.pdf}{Geometric Algebra Primer.}
As described in the
\href{http://modelingnts.la.asu.edu/html/IntroPrimerGeometricAlgebra.html}{Introduction page for this primer}, this is a workbook, and reading should not be passive.

Also available is his
\href{http://modelingnts.la.asu.edu/pdf/OerstedMedalLecture.pdf}{Oersted Lecture}, which contains a good introduction.

If you do not have his ``New Foundations of Classical Mechanics'' book, you can find some of the dot-product/wedge-product reduction formulas in the following
\href{http://modelingnts.la.asu.edu/pdf/UGA.pdf}{non-metric treatment of GA.}

Also interesting is this
\href{http://modelingnts.la.asu.edu/pdf/GTG.w.GC.FP.pdf}{Gauge Theory Gravity with Geometric Algebra}
paper.  This has an introduction to STA (Space Time Algebra) as used in the Cambridge books.  This also shows at a high level where one can go with a lot of these ideas (like the grad F = J formulation of Maxwell's equation, a multivector form that incorporates all of the traditional four vector Maxwell's equations).  Nice teaser document if you intend to use GA for physics study, but hard to read even the consumable bits because they are buried in among a lot of other higher level math and physics.


Hestenes, Li and Rockwood in their paper
\href{http://modelingnts.la.asu.edu/pdf/CompGeom-ch1.pdf}{ New Algebraic Tools for Classical Geometry}
in G. Sommer (ed.) Geom.
Computing with Clifford Algebras (Springer, 2001) treat outermorphisms
and determinants in a separate subsection entitled "Outermorphism"
of section 1.3 Linear Transformations:

This is a comprehensive doc.  Content includes:
\begin{itemize}
\item GA intro boilerplate.
\item Projection and Rejection.
\item Meet and Join.
\item Reciprocal vectors (dual frame).
\item Vector differentiation.
\item Linear transformations.
\item Determinants and outermorphisms.
\item Rotations.
\item Simplexes and boundaries
\item Dual quaternions.
\end{itemize}

%\section{Peeter's GA/Physics Topics}
%
%\href{http://en.wikipedia.org/wiki/Geometric_algebra}{GA wikipedia}
%
%I dumped a bunch of info in this doc as I was puzzling things out (initially before I had purchased any books).  I have since stopped contributing since it was getting too big and was no longer appropriate (ie: getting bookish instead of encyclopedic).
%
%\href{http://sites.google.com/site/peeterjoot/}{Instead I have got a bunch of standalone latex writeups of my notes here.}
%
%I have a number of smallish GA related documents, which in reverse chronological order document my own learning/relearning roadmap (for GA and physics).  Hopefully useful for others too.  Please email peeterjoot@protonmail.com if any mistakes are found (other than ones that are already described in the index that I have not gone back to fix).
%

\section{Eckhard M. S. Hitzer (University of Fukui)}

From
\href{http://sinai.mech.fukui-u.ac.jp/gala2/}{Eckhard's Geometric Algebra Topics.}

Since these are all specific documents, and all at a fairly consumable level for a new learner, I have listed them here specifically:

\begin{itemize}
\item
\href{http://sinai.mech.fukui-u.ac.jp/gala2/GAtopics/axioms.pdf}{Axioms of geometric algebra}
\item
\href{http://sinai.mech.fukui-u.ac.jp/gala2/GAtopics/qform.pdf}{The use of quadratic forms in geometric algebra}
\item
\href{http://sinai.mech.fukui-u.ac.jp/gala2/GAtopics/products.pdf}{The geometric product and derived products}
\item
\href{http://sinai.mech.fukui-u.ac.jp/gala2/GAtopics/det.pdf}{Determinants in geometric algebra}
\item
\href{http://sinai.mech.fukui-u.ac.jp/gala2/GAtopics/GS.pdf}{Gram-Schmidt orthogonalization in geometric algebra}
\item
\href{http://sinai.mech.fukui-u.ac.jp/gala2/GAtopics/WhatIsi.pdf}{What is an imaginary number?}
\item
\href{http://sinai.mech.fukui-u.ac.jp/gcj/publications/mvdifcalc/mvdc.pdf}{Simplical calculus:}
\end{itemize}

\section{Electrodynamics}

John Denker has a number of GA docs that all appear very readable.  One such doc is:

\href{http://www.av8n.com/physics/maxwell-ga.pdf}{Electromagnetism using Geometric Algebra versus Components}

This is a nice little doc (there is also an HTML version, but it is very hard to read, and the first time I saw it I actually missed a lot of content).

The oft repeated introduction to GA is not in this doc, so you have to know the basics first.  Denker takes the \(\grad F = J/c \epsilon_0\) equation and unpacks it in a brute force but understandable fashion, and shows that these are identical to the vector differential form of Maxwell's equations.  A few other E\&M constructs are shown in their GA form (covariant form of Lorentz force equation, Lagrangian density, Stress tensor, Poynting Vector.  There are also many good comments on notation issues.

A cautionary note if you have read any of the Cambridge papers.  This doc uses a -+++ metric instead of the +--- used in those docs.

Some other Denker GA papers:
\begin{itemize}
\item
\href{http://www.av8n.com/physics/straight-wire.pdf}{Magnetic field of a straight wire.}
\item
\href{http://www.av8n.com/physics/clifford-intro.pdf}{Clifford Intro.}

Very nice axiomatic introduction with excellent commentary.  Also includes an STA intro.

\item
\href{http://www.av8n.com/physics/complex-clifford.pdf}{Complex numbers.}
\item
\href{http://www.av8n.com/physics/area-volume.pdf}{Area and Volume.}
\item
\href{http://www.av8n.com/physics/rotations.pdf}{Rotations.}
\end{itemize}
(have not read all these yet).


Richard E. Harke,
\href{http://www.harke.org/ps/intro.ps.gz}{An Introduction to the Mathematics of the Space-Time Algebra}

This is a nice complete little doc (\~40 pages), where many basic GA constructs are developed axiomatically with associated proofs.  This includes some simplical calculus and outermorphism content, and eventually moves on to STA and Lorentz rotations.



\section{Misc}

\begin{itemize}
\item
A blog like
\href{http://gaupdate.wordpress.com/}{subscription service that carries abstracts}
for various papers on or using Geometric Algebra.
\end{itemize}

\section{Collections of other GA bookmarks}

\begin{itemize}
\item
\href{http://www.geomerics.com/geometric-algebra.htm}{Geomerics.  Graphics software for Games, Geometric Algebra references and description.}
\item
\href{http://www.xtec.es/~rgonzal1/links.htm}{Ramon Gonz�lez Calvet us GA links.}
\item
\href{http://www.rwgrayprojects.com/GeometricAlgebra/references.html}{R. W. Gray's GA links.}
\item
\href{http://www.mrao.cam.ac.uk/~clifford/pages/links.htm}{Cambridge groups GA urls.}
\end{itemize}

\section{Exterior Algebra and differential forms}

\begin{itemize}
\item
\href{http://www.grassmannalgebra.info/grassmannalgebra/book/index.htm}{Grassmann Algebra Book}

Pdf files of a book draft entitled Grassmann Algebra: Exploring applications of extended vector algebra with Mathematica.

This has some useful info.  In particular, a great example of solving linear systems with the wedge product.
\item
The Cornell Library Historical Mathematics Monographs -
\href{http://historical.library.cornell.edu/cgi-bin/cul.math/docviewer?did=00540001&seq=15&frames=0&view=50}{hyde on grassman}
\item
\href{http://www.math.boun.edu.tr/instructors/ozturk/eskiders/fall04math488/bachman.pdf}{A Geometric Approach to Differential Forms by David Bachman}
\end{itemize}

\section{Software}

\begin{itemize}
\item
\href{http://staff.science.uva.nl/~fontijne/gaigen2.html}{Gaigen 2}
\item
\href{http://users.tkk.fi/~ppuska/mirror/Lounesto/CLICAL.htm}{CLICAL for Clifford Algebra Calculations}
\item
\href{http://www.nklein.com/products/geoma/}{nklein software.  Geoma.}
\end{itemize}
