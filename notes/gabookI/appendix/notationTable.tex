%
% Copyright � 2012 Peeter Joot.  All Rights Reserved.
% Licenced as described in the file LICENSE under the root directory of this GIT repository.
%

%
%
\chapter{Notation and definitions}\label{chap:notationTable}

Here is a summary of the notation and definitions that will be used.

The following tables summarize a lot of the notation used in these notes.
This largely follows conventions from \citep{doran2003gap}.

\section{Coordinates and basis vectors}

Greek letters range over all indices and English indices range over \(1,2,3\).

Bold vectors are spatial entities and non-bold is used for four vectors and scalars.

Summation convention \index{summation convention} is often used (less so in earlier notes).  This is
summation over all sets of matched upper and lower indices is implied.

While many things could be formulated in a metric signature independent
fashion,
a time positive
\((+,-,-,-)\)
metric signature should be assumed in most cases.  Specifically, that is \((\gamma_0)^2 = 1\), and \((\gamma_k)^2 = -1\).

\begin{equation*}
\begin{array}{l l l}
\gamma_{\mu} & \gamma_{\mu} \cdot \gamma_{\nu} = \pm {\delta^{\mu}}_{\nu} & \quad \mbox{Four vector basis vector} \\
& & \quad \mbox{(\(\gamma_{\mu} \cdot \gamma_{\nu} = \pm {\delta^{\mu}}_{\nu}\))} \\
{(\gamma_0)}^2 {(\gamma_k)}^2 &= -1 & \quad \mbox{Minkowski metric} \\
\sigma_k = \sigma^k &= \gamma_{k} \wedge \gamma_0 & \quad \mbox{Spatial basis bivector. (\(\sigma_k \cdot \sigma_j = \delta_{kj}\))} \\
                    &= \gamma_{k} \gamma_0 & \\
I &= \gamma_{0} \wedge \gamma_1 \wedge \gamma_{2} \wedge \gamma_3 & \quad \mbox{Four-vector pseudoscalar} \\
  &= \gamma_{0} \gamma_1 \gamma_{2} \gamma_3 & \\
  &= \gamma_{0123} \\
\gamma^{\mu} \cdot \gamma_{\nu} &= {\delta^{\mu}}_{\nu} & \quad \mbox{Reciprocal basis vectors} \\
x^{\mu} &= x \cdot \gamma^{\mu} & \quad \mbox{Vector coordinate} \\
x_{\mu} &= x \cdot \gamma_{\mu} & \quad \mbox{Coordinate for reciprocal basis} \\
x &= \gamma_{\mu} x^{\mu} & \quad \mbox{Four vector in terms of coordinates} \\
  &= \gamma^{\mu} x_{\mu} \\
x^{0} &= x \cdot \gamma^0 & \quad \mbox{Time coordinate (length dim.)} \\
      &= c t \\
\Bx &= x \wedge \gamma_0 & \quad \mbox{Spatial vector} \\
    &= x^k \sigma_k \\
x^2 &= x \cdot x & \quad \mbox{Four vector square. } \\
    &= x^\mu x_\mu \\
\Bx^2 &= \Bx \cdot \Bx & \quad \mbox{Spatial vector square. } \\
    &= \sum_{k=1}^3 (x^k)^2 \\
    &= \Abs{\Bx}^2 \\
\end{array}
\end{equation*}

If convient sometimes \(i\) will be used for the pseudoscalar.

\section{Electromagnetism}

SI units are
used in most places, but occasionally natural units are used.  In
some cases, when working with material such as \citep{bohm1989qt},
CGS modifications of the notation are employed.

\begin{equation*}
\begin{array}{l l l}
\BE &= \sum E^k \sigma_k & \quad \mbox{Electric field spatial vector} \\
\BB &= \sum B^k \sigma_k & \quad \mbox{Magnetic field spatial vector} \\
\bcE &= E^k \sigma_k & \quad \mbox{(CGS)Electric field spatial vector} \\
\bcH &= H^k \sigma_k & \quad \mbox{(CGS)Magnetic field spatial vector} \\
J &= \gamma_{\mu} J^{\mu} & \quad \mbox{Current density four vector.} \\
  &= \gamma^{\mu} J_{\mu} \\
F &= \BE + I c \BB & \quad \mbox{Electromagnetic (Faraday) bivector} \\
  &= F^{\mu\nu} \gamma_\mu \wedge \gamma_\nu & \quad \mbox{in terms of Faraday tensor} \\
  &= \bcE + I \bcH & \quad \mbox{(CGS)} \\
J^{0} &= J \cdot \gamma^0 & \quad \mbox{Charge density.} \\
      &= c \rho & \quad \mbox{(current density dimensions.)} \\
      &= \rho & \quad \mbox{(CGS) (current density dimensions.)} \\
\BJ &= J \wedge \gamma_0 & \quad \mbox{Current density spatial vector} \\
    &= J^k \sigma_k \\
\end{array}
\end{equation*}

\section{Differential operators}
\begin{equation*}
\begin{array}{l l l}
\partial_{\mu} &= \PDi{x^\mu}{} & \quad \mbox{Index up partial.} \\
\partial^{\mu} &= \PDi{x_\mu}{} & \quad \mbox{Index down partial.} \\
\partial_{\mu\nu} &= \PDi{x^\mu}{}\PDi{x^\nu}{} & \quad \mbox{Index up partial.} \\
\grad &= \sum \gamma^{\mu} \partial/\partial {x^{\mu}} & \quad \mbox{Spacetime gradient} \\
      &= \gamma^{\mu}\partial_{\mu} \\
      &= \sum \gamma_{\mu} \partial/\partial {x_{\mu}} \\
      &= \gamma_{\mu}\partial^{\mu} \\
\spacegrad &= \sigma^{k} \partial_k & \quad \mbox{Spatial gradient} \\
\hat{A}_{\Bk} &= \hat{A}_{k_1,k_2,k_3} & \quad \mbox{Fourier coefficient, integer indices.} \\
\grad^2 A
   &= (\grad \cdot \grad) A & \quad \mbox{Four Laplacian. } \\
   &= (\partial_{00} - \sum_k \partial_{kk}) A & \\
d^3 x &= dx^1 dx^2 dx^3 & \quad \mbox{Spatial volume element. } \\
d^4 x &= dx^0 dx^1 dx^2 dx^3 & \quad \mbox{Four volume element. } \\
\int_{\partial I} &= \int_{a}^{b} & \quad \mbox{Integration range \(I = [a,b]\) } \\
\text{STA} & & \quad \mbox{Space Time Algebra} \\
(xyz)^{\tilde{}} &= \widetilde{xyz} = z y x & \quad \mbox{Reverse of a vector product.} \\
\end{array}
\end{equation*}

\section{Misc}

The \(\PV\) notation is taken from \citep{lepage1980cva} where the author uses it in his Riemann integral proof of the inverse Fourier integral.

\begin{equation*}
\begin{array}{l l l}
\PV \IIinf &= \lim_{R\rightarrow \infty} \int_{R}^R & \quad \mbox{Integral Principle value} \\
\hat{A}(k) &= \calF(A(x)) & \quad \mbox{Fourier transform of \(A\)} \\
{A}(x) &= \calF^{-1}(A(k)) & \quad \mbox{Inverse Fourier transform} \\
\exp(i\Bk\phi) &=
\cos(\Abs{\Bk}\phi) + \frac{i \Bk}{\Abs{i\Bk}} \sin(\Abs{\Bk}\phi) & \quad \mbox{bivector exponential. } \\
\end{array}
\end{equation*}
