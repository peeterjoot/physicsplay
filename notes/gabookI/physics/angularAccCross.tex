%
% Copyright � 2012 Peeter Joot.  All Rights Reserved.
% Licenced as described in the file LICENSE under the root directory of this GIT repository.
%

%
%
\chapter{Cross product Radial decomposition}\label{chap:PJAngAccCross}
%\date{July 8, 2008.  angularAccCross.tex}

We have seen how to use GA constructs to perform a radial
decomposition of a velocity and acceleration vector.  Is it that
much harder to do this with straight vector algebra.  This shows that
the answer is no, but we need to at least assume some additional
identities that can take work to separately prove.  Here is a quick
demonstration for comparision purposes how a radial decomposition
can be performed entirely without any GA usage.

\section{Starting point}

Starting point is taking derivatives of:

\begin{equation}\label{eqn:angularAccCross:20}
\Br = r \rcap
\end{equation}

\begin{equation}\label{eqn:angularAccCross:40}
\Bv = r' \rcap + r \rcap'
\end{equation}

It can be shown without any Geometric Algebra use (see for example \citep{salas1990coa}) that the unit vector derivative can be expressed using the cross product:

\begin{equation}\label{eqn:angularAccCross:60}
\rcap' = \inv{r} \left(\rcap \cross \frac{d\Br}{dt}\right) \cross \rcap.
\end{equation}

Now, one can express \(r'\) in terms of \(\Br\) as well as follows:

\begin{equation}\label{eqn:angularAccCross:80}
\left(\Br \cdot \Br\right)' = 2 \Bv \cdot \Br = 2 r r'.
\end{equation}

Thus the derivative of the vector magnitude is part of a projective term:

\begin{equation}\label{eqn:angularAccCross:100}
r' = \rcap \cdot \Bv.
\end{equation}

Putting this together one has velocity in terms of projective and rejective
components along a radial direction:

\begin{equation}\label{eqn:angularAccCross:120}
\Bv = \left(\rcap \cdot \Bv\right) \rcap + \left(\rcap \cross \frac{d\Br}{dt}\right) \cross \rcap.
\end{equation}

Now \(\Bomega = \frac{\Br \cross \Bv}{r^2}\) term is what we call the angular velocity.  The magnitude of this
is the rate of change of the angle between the radial arm and the direction of rotation.  The direction of this
cross product is normal to the plane of rotation and encodes both the rotational plane and the direction of the
rotation.  Putting these together one has the total velocity expressed radially:

\begin{equation}\label{eqn:angularAccCross:140}
\Bv = \left(\rcap \cdot \Bv\right) \rcap + \Bomega \cross \Br.
\end{equation}

\section{Acceleration}

Acceleration follows in the same fashion.

\begin{equation}\label{eqn:angularAccCross:160}
\begin{aligned}
\Bv'
&= (
\mathLabelBox{\left(\rcap \cdot \Bv\right) \rcap}{\(r'\rcap\)}
)' + (
\mathLabelBox{\Bomega \cross \Br}{\((\Br \cross \Bv) \cross \frac{\Br}{r^2}\)}
)' \\
&= r'' \rcap
 + r' \frac{\Bomega \cross \Br}{r}
 + (\rcap \cross \Ba) \cross \rcap
 + (\mathLabelBox{\Bv \cross \Bv)}{\(=0\)} \cross \frac{\Br}{r^2}
 + (\Br \cross \Bv) \cross \left(\frac{\Br}{r^2}\right)'
\end{aligned}
\end{equation}

That last derivative is

\begin{equation}\label{eqn:angularAccCross:180}
\begin{aligned}
\left(\frac{\Br}{r^2}\right)'
&= \left(\frac{\rcap}{r}\right)' \\
&= \frac{\rcap'}{r} - \frac{\rcap r'}{r^2} \\
&= \frac{\Bomega \cross \Br}{r^2} - \frac{\rcap r'}{r^2},
\end{aligned}
\end{equation}

and back substitution gives:
\begin{equation}\label{eqn:angularAccCross:200}
\begin{aligned}
\Bv'
&= r'' \rcap
 + r' \frac{\Bomega \cross \Br}{r}
 + (\rcap \cross \Ba) \cross \rcap
 + (\Br \cross \Bv) \cross \left( \frac{\Bomega \cross \Br}{r^2} - \frac{\rcap r'}{r^2} \right).
\end{aligned}
\end{equation}

Canceling terms and collecting we have the final result for acceleration expressed radially:

\begin{equation}
\Bv' = \Ba = r'' \rcap + \Bomega \cross \left( {\Bomega \cross \Br} \right) + (\rcap \cross \Ba) \cross \rcap
\end{equation}

Now, applying the angular velocity via cross product takes the vector back to the original plane, but inverts it.  Thus we can write the acceleration completely in terms of the radially directed components, and the perpendicular component.

\begin{equation}
\Ba = r'' \rcap -\Br \omega^2 + (\rcap \cross \Ba) \cross \rcap
\end{equation}

An alternate way to express this is in terms of radial scalar acceleration:

\begin{equation}
\Ba \cdot \rcap = r'' - r\omega^2.
\end{equation}

This is the acceleration analogue of the scalar radial velocity component demonstrated above:
\begin{equation}
\Bv \cdot \rcap = r'.
\end{equation}
