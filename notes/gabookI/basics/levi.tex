%
% Copyright � 2012 Peeter Joot.  All Rights Reserved.
% Licenced as described in the file LICENSE under the root directory of this GIT repository.
%

%
%
\chapter{Levi-Civitica summation identity}
\index{Levi-Civitica tensor}
\label{chap:levi}
%\date{March 13, 2009.  levi.tex}

\section{Motivation}

In \citep{byron1992mca} it is left to the reader to show

\index{contraction!Levi-Civitica tensor}
\begin{equation}\label{eqn:levi:20}
\begin{aligned}
\sum_k \epsilon_{ijk} \epsilon_{klm} = \delta_{il}\delta_{jm} - \delta_{jl}\delta_{im}
\end{aligned}
\end{equation}

\section{A mechanical proof}

Although it is not mathematical, this is easy to prove, at least for 3D.  The
following perl code does the trick

\lstinputlisting{listings/levi.pl}

The output produced has all the variations of indices, such as

\begin{equation}\label{eqn:levi:40}
\begin{aligned}
0 &= \sum_{k=1}^{3} \epsilon_{11k} \epsilon_{k11} = \delta_{11}\delta_{11} - \delta_{11}\delta_{11} \\
0 &= \sum_{k=1}^{3} \epsilon_{11k} \epsilon_{k12} = \delta_{11}\delta_{12} - \delta_{11}\delta_{12} \\
\vdots \\
0 &= \sum_{k=1}^{3} \epsilon_{11k} \epsilon_{k33} = \delta_{13}\delta_{13} - \delta_{13}\delta_{13} \\
0 &= \sum_{k=1}^{3} \epsilon_{12k} \epsilon_{k11} = \delta_{11}\delta_{21} - \delta_{21}\delta_{11} \\
1 &= \sum_{k=1}^{3} \epsilon_{12k} \epsilon_{k12} = \delta_{11}\delta_{22} - \delta_{21}\delta_{12} \\
0 &= \sum_{k=1}^{3} \epsilon_{12k} \epsilon_{k13} = \delta_{11}\delta_{23} - \delta_{21}\delta_{13} \\
-1 &= \sum_{k=1}^{3} \epsilon_{12k} \epsilon_{k21} = \delta_{12}\delta_{21} - \delta_{22}\delta_{11} \\
\vdots \\
\end{aligned}
\end{equation}

\section{Proof using bivector dot product}

This identity can also be derived from an expansion of the bivector
dot product in two different ways.

\begin{equation}\label{eqn:levi:60}
\begin{aligned}
( \Be_i \wedge \Be_j ) \cdot ( \Be_m \wedge \Be_n )
&=
( ( \Be_i \wedge \Be_j ) \cdot \Be_m ) \cdot \Be_n  \\
&=
(
\Be_i ( \Be_j \cdot \Be_m )
-\Be_j ( \Be_i \cdot \Be_m )
) \cdot \Be_n  \\
&=
( \Be_i \delta_{jm} -\Be_j \delta_{im} ) \cdot \Be_n  \\
&=
\delta_{in} \delta_{jm} -\delta_{jn} \delta_{im}
\end{aligned}
\end{equation}

Expressing the wedge product in terms duality, using the pseudoscalar
\(I = \Be_1 \Be_2 \Be_3\), we have

\begin{equation}\label{eqn:levi:80}
\begin{aligned}
(\Be_i \wedge \Be_j ) \Be_k = I \epsilon_{ijk}
\end{aligned}
\end{equation}

Or
\begin{equation}\label{eqn:levi:100}
\begin{aligned}
\Be_i \wedge \Be_j = I \sum_k \epsilon_{ijk} \Be_k
\end{aligned}
\end{equation}

Then the bivector dot product is
\begin{equation}\label{eqn:levi:120}
\begin{aligned}
( \Be_i \wedge \Be_j ) \cdot ( \Be_m \wedge \Be_n )
&=
\gpgradezero{
I \sum_k \epsilon_{ijk} \Be_k I \sum_p \epsilon_{mnp} \Be_p
} \\
&=
I^2 \sum_{k,p} \epsilon_{ijk} \epsilon_{mnp} \gpgradezero{ \Be_k \Be_p } \\
&=
- \sum_{k,p} \epsilon_{ijk} \epsilon_{mnp} \delta_{kp} \\
&=
- \sum_{k} \epsilon_{ijk} \epsilon_{mnk} \\
\end{aligned}
\end{equation}

Comparing the two expansions we have

\begin{equation}\label{eqn:levi:140}
\begin{aligned}
\sum_{k} \epsilon_{ijk} \epsilon_{mnk} &= \delta_{jn} \delta_{im} - \delta_{in} \delta_{jm}
\end{aligned}
\end{equation}

Which is equivalent to the original identity (after an index switcheroo).
Note both the dimension and metric dependencies in this proof.
