%
% Copyright � 2012 Peeter Joot.  All Rights Reserved.
% Licenced as described in the file LICENSE under the root directory of this GIT repository.
%

%
%
\chapter{Rotor interpolation calculation}
\index{rotor!interpolation}
\label{chap:slerp}
%\date{Nov 30, 2008.  slerp.tex}

The aim is to compute the interpolating rotor \(r\) that takes an object
from one position to another in \(n\) steps.
Here the initial and final positions are given by two rotors \(R_1\), and \(R_2\)
like so

\begin{equation}\label{eqn:slerp:20}
\begin{aligned}
X_1 &= R_1 X {R_1}^\dagger \\
X_2 &= R_2 X {R_2}^\dagger = r^n R_1 X {R_1}^\dagger {r^n}^\dagger
\end{aligned}
\end{equation}

So, writing

\begin{equation}\label{eqn:slerp:40}
\begin{aligned}
%r^n R_1 = R_2
a = r^n = R_2 \inv{R_1} = \frac{R_2 {R_1}^\dagger}{R_1 {{R_1}^\dagger}} = \cos\theta + I \sin\theta
\end{aligned}
\end{equation}

So,

\begin{equation}\label{eqn:slerp:60}
\begin{aligned}
\frac{\gpgradetwo{a}}{\gpgradezero{a}} &=
\frac{\gpgradetwo{a}}{\Abs{\gpgradetwo{a}}} \frac{\Abs{\gpgradetwo{a}}}{\gpgradezero{a}} \\
&= I \tan\theta
\end{aligned}
\end{equation}

Therefore the interpolating rotor is:
\begin{equation}\label{eqn:slerp:80}
\begin{aligned}
I &= \frac{\gpgradetwo{a}}{\Abs{\gpgradetwo{a}}} \\
\theta &= \atan2\left(\Abs{\gpgradetwo{a}}, \gpgradezero{a}\right) \\
r &= \cos(\theta/n) + I \sin(\theta/n)
\end{aligned}
\end{equation}

In \citep{dorst2007gac}, equation \(10.15\), they have got something like this
for a fractional angle, but then say that they do not use that in software,
instead using \(r\) directly, with a comment about designing more sophisticated
algorithms (bivector splines).  That spline comment in particular sounds
interesting.  Sounds like the details on that are to be found in the journals
mentioned in Further Reading section of chapter 10.
