%
% Copyright � 2012 Peeter Joot.  All Rights Reserved.
% Licenced as described in the file LICENSE under the root directory of this GIT repository.
%

% 
% 
\chapter{Magnetic field between two parallel wires}        
\label{chap:sgMx41}
\date{July 20, 2008}

\section{Student's guide to Maxwell's' equations.  problem 4.1}        

The 
\href{http://www4.wittenberg.edu/maxwell/chapter4/problem1/}{problem is}:

Two parallel wires carry currents I1 and 2I1 in opposite directions.  Use Ampere�s law to find the magnetic field at a point midway between the wires.

Do this instead (visualizing the cross section through the wires) for N wires
located at points $P_k$, with currents $I_k$.

\begin{figure}[htp]
\centering
\includegraphics[totalheight=0.4\textheight]{p41}
\caption{Currents through parallel wires}\label{fig:2wires}
\end{figure}

This is illustrated for two wires in figure \ref{fig:2wires}.

\subsection{}

Consider first just the magnetic field for one wire, temporarily putting
the origin at the point of the current.

\begin{equation*}
\int \BB \cdot d\Bl = \mu_0 I
\end{equation*}

At a point $\Br$ from the local origin the tangent vector is obtained by 
rotation of the unit vector:

\begin{equation*}
\ycap \exp{\left(\xcap\ycap \log{\left(\frac{\Br}{\norm{\Br}}\right)}\right)}
= \ycap {\left(\frac{\Br}{\norm{\Br}}\right)}^{\xcap\ycap}
\end{equation*}

Thus the magnetic field at the point $\Br$ due to this particular current is:

\begin{equation*}
\BB(\Br) 
= \frac{\mu_0 I \ycap}{2\pi \norm{\Br}} {\left(\frac{\Br}{\norm{\Br}}\right)}^{\xcap\ycap}
\end{equation*}

Considering additional currents with the wire centers at points $P_k$, and measurement of the field at point $\BR$ we have for each of those:

\begin{equation*}
\Br = \BR - \BP
\end{equation*}

Thus the total field at point $\BR$ is:

\begin{equation}
\BB(\BR) = \frac{\mu_0 \ycap}{2\pi} \sum_k \frac{I_k}{\norm{\BR - \BP_k}} {\left(\frac{\BR - \BP_k}{\norm{\BR - \BP_k}}\right)}^{\xcap\ycap}
\end{equation}

\subsection{Original problem}

For the problem as stated, put the origin between the two points with those two points on the x-axis.

\begin{align*}
\BP_1 &= - \xcap d/2 \\
\BP_2 &= \xcap d/2 
\end{align*}

Here $\BR$ = 0, so $\Br_1 = \BR - \BP_1 = \xcap d/2 $ and $\Br_2 = - \xcap d/2$.  With $\xcap\ycap = i$, this is:

\begin{align*}
\BB(0)
&= \frac{\mu_0 \ycap}{\pi d} \left( I_1 {(-\xcap)}^i + I_2 {\xcap^i} \right) \\
&= \frac{\mu_0 \ycap}{\pi d} \left( -I -2 I\right) \\
&= \frac{-3 I \mu_0 \ycap}{\pi d}
\end{align*}

Here unit vectors exponentials were evaluated with the equivalent complex number manipulations:

\begin{align*}
(-1)^i &= x \\
i \log{(-1)} &= \log{x} \\
i \pi &= \log{x} \\
\exp{(i \pi)} &= \log{x} \\
x &= -1
\end{align*}

\begin{align*}
(1)^i &= x \\
i \log{(1)} &= \log{x} \\
0 &= \log{x} \\
x &= 1 
\end{align*}
