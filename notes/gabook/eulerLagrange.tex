\documentclass{article}

\usepackage{amsmath}
\usepackage{mathpazo}

%
% shorthand for bold symbols, convenient for vectors and matrices
%
\newcommand{\Ba}[0]{\mathbf{a}}
\newcommand{\Bb}[0]{\mathbf{b}}
\newcommand{\Bc}[0]{\mathbf{c}}
\newcommand{\Bd}[0]{\mathbf{d}}
\newcommand{\Be}[0]{\mathbf{e}}
\newcommand{\Bf}[0]{\mathbf{f}}
\newcommand{\Bg}[0]{\mathbf{g}}
\newcommand{\Bh}[0]{\mathbf{h}}
\newcommand{\Bi}[0]{\mathbf{i}}
\newcommand{\Bj}[0]{\mathbf{j}}
\newcommand{\Bk}[0]{\mathbf{k}}
\newcommand{\Bl}[0]{\mathbf{l}}
\newcommand{\Bm}[0]{\mathbf{m}}
\newcommand{\Bn}[0]{\mathbf{n}}
\newcommand{\Bo}[0]{\mathbf{o}}
\newcommand{\Bp}[0]{\mathbf{p}}
\newcommand{\Bq}[0]{\mathbf{q}}
\newcommand{\Br}[0]{\mathbf{r}}
\newcommand{\Bs}[0]{\mathbf{s}}
\newcommand{\Bt}[0]{\mathbf{t}}
\newcommand{\Bu}[0]{\mathbf{u}}
\newcommand{\Bv}[0]{\mathbf{v}}
\newcommand{\Bw}[0]{\mathbf{w}}
\newcommand{\Bx}[0]{\mathbf{x}}
\newcommand{\By}[0]{\mathbf{y}}
\newcommand{\Bz}[0]{\mathbf{z}}
\newcommand{\BA}[0]{\mathbf{A}}
\newcommand{\BB}[0]{\mathbf{B}}
\newcommand{\BC}[0]{\mathbf{C}}
\newcommand{\BD}[0]{\mathbf{D}}
\newcommand{\BE}[0]{\mathbf{E}}
\newcommand{\BF}[0]{\mathbf{F}}
\newcommand{\BG}[0]{\mathbf{G}}
\newcommand{\BH}[0]{\mathbf{H}}
\newcommand{\BI}[0]{\mathbf{I}}
\newcommand{\BJ}[0]{\mathbf{J}}
\newcommand{\BK}[0]{\mathbf{K}}
\newcommand{\BL}[0]{\mathbf{L}}
\newcommand{\BM}[0]{\mathbf{M}}
\newcommand{\BN}[0]{\mathbf{N}}
\newcommand{\BO}[0]{\mathbf{O}}
\newcommand{\BP}[0]{\mathbf{P}}
\newcommand{\BQ}[0]{\mathbf{Q}}
\newcommand{\BR}[0]{\mathbf{R}}
\newcommand{\BS}[0]{\mathbf{S}}
\newcommand{\BT}[0]{\mathbf{T}}
\newcommand{\BU}[0]{\mathbf{U}}
\newcommand{\BV}[0]{\mathbf{V}}
\newcommand{\BW}[0]{\mathbf{W}}
\newcommand{\BX}[0]{\mathbf{X}}
\newcommand{\BY}[0]{\mathbf{Y}}
\newcommand{\BZ}[0]{\mathbf{Z}}

\newcommand{\Bzero}[0]{\mathbf{0}}
\newcommand{\Btheta}[0]{\boldsymbol{\theta}}
\newcommand{\Btau}[0]{\boldsymbol{\tau}}
\newcommand{\Bomega}[0]{\boldsymbol{\omega}}

%
% shorthand for unit vectors
%
\newcommand{\acap}[0]{\hat{\Ba}}
\newcommand{\bcap}[0]{\hat{\Bb}}
\newcommand{\ccap}[0]{\hat{\Bc}}
\newcommand{\dcap}[0]{\hat{\Bd}}
\newcommand{\ecap}[0]{\hat{\Be}}
\newcommand{\fcap}[0]{\hat{\Bf}}
\newcommand{\gcap}[0]{\hat{\Bg}}
\newcommand{\hcap}[0]{\hat{\Bh}}
\newcommand{\icap}[0]{\hat{\Bi}}
\newcommand{\jcap}[0]{\hat{\Bj}}
\newcommand{\kcap}[0]{\hat{\Bk}}
\newcommand{\lcap}[0]{\hat{\Bl}}
\newcommand{\mcap}[0]{\hat{\Bm}}
\newcommand{\ncap}[0]{\hat{\Bn}}
\newcommand{\ocap}[0]{\hat{\Bo}}
\newcommand{\pcap}[0]{\hat{\Bp}}
\newcommand{\qcap}[0]{\hat{\Bq}}
\newcommand{\rcap}[0]{\hat{\Br}}
\newcommand{\scap}[0]{\hat{\Bs}}
\newcommand{\tcap}[0]{\hat{\Bt}}
\newcommand{\ucap}[0]{\hat{\Bu}}
\newcommand{\vcap}[0]{\hat{\Bv}}
\newcommand{\wcap}[0]{\hat{\Bw}}
\newcommand{\xcap}[0]{\hat{\Bx}}
\newcommand{\ycap}[0]{\hat{\By}}
\newcommand{\zcap}[0]{\hat{\Bz}}
\newcommand{\thetacap}[0]{\hat{\Btheta}}

%
% to write R^n and C^n in a distinguishable fashion.  Perhaps change this
% to the double lined characters upon figuring out how to do so.
%
\newcommand{\C}[1]{$\mathbb{C}^{#1}$}
\newcommand{\R}[1]{$\mathbb{R}^{#1}$}

%
% various generally useful helpers
%

% derivative of #1 wrt. #2:
\newcommand{\D}[2] {\frac {d#2} {d#1}}

\newcommand{\inv}[1]{\frac{1}{#1}}
\newcommand{\cross}[0]{\times}

\newcommand{\abs}[1]{\lvert{#1}\rvert}
\newcommand{\norm}[1]{\lVert{#1}\rVert}
\newcommand{\innerprod}[2]{\langle{#1}, {#2}\rangle}
\newcommand{\dotprod}[2]{{#1} \cdot {#2}}
\newcommand{\bdotprod}[2]{\left({#1} \cdot {#2}\right)}
\newcommand{\crossprod}[2]{{#1} \cross {#2}}
\newcommand{\tripleprod}[3]{\dotprod{\left(\crossprod{#1}{#2}\right)}{#3}}

\DeclareMathOperator{\Proj}{Proj}
\DeclareMathOperator{\Span}{span}
\DeclareMathOperator{\Sgn}{sgn}
\DeclareMathOperator{\Area}{Area}
\DeclareMathOperator{\Volume}{Volume}

%
% A few miscellaneous things specific to this document
%
\newcommand{\crossop}[1]{\crossprod{#1}{}}

% R2 vector.
\newcommand{\VectorTwo}[2]{
\begin{bmatrix}
 {#1} \\
 {#2}
\end{bmatrix}
}

\newcommand{\VectorN}[1]{
\begin{bmatrix}
{#1}_1 \\
{#1}_2 \\
\vdots \\
{#1}_N \\
\end{bmatrix}
}

\newcommand{\DETuvij}[4]{
\begin{vmatrix}
 {#1}_{#3} & {#1}_{#4} \\
 {#2}_{#3} & {#2}_{#4}
\end{vmatrix}
}

\newcommand{\DETuvwijk}[6]{
\begin{vmatrix}
 {#1}_{#4} & {#1}_{#5} & {#1}_{#6} \\
 {#2}_{#4} & {#2}_{#5} & {#2}_{#6} \\
 {#3}_{#4} & {#3}_{#5} & {#3}_{#6}
\end{vmatrix}
}

\newcommand{\DETuvwxijkl}[8]{
\begin{vmatrix}
 {#1}_{#5} & {#1}_{#6} & {#1}_{#7} & {#1}_{#8} \\
 {#2}_{#5} & {#2}_{#6} & {#2}_{#7} & {#2}_{#8} \\
 {#3}_{#5} & {#3}_{#6} & {#3}_{#7} & {#3}_{#8} \\
 {#4}_{#5} & {#4}_{#6} & {#4}_{#7} & {#4}_{#8} \\
\end{vmatrix}
}

%\newcommand{\DETuvwxyijklm}[10]{
%\begin{vmatrix}
% {#1}_{#6} & {#1}_{#7} & {#1}_{#8} & {#1}_{#9} & {#1}_{#10} \\
% {#2}_{#6} & {#2}_{#7} & {#2}_{#8} & {#2}_{#9} & {#2}_{#10} \\
% {#3}_{#6} & {#3}_{#7} & {#3}_{#8} & {#3}_{#9} & {#3}_{#10} \\
% {#4}_{#6} & {#4}_{#7} & {#4}_{#8} & {#4}_{#9} & {#4}_{#10} \\
% {#5}_{#6} & {#5}_{#7} & {#5}_{#8} & {#5}_{#9} & {#5}_{#10}
%\end{vmatrix}
%}

% R3 vector.
\newcommand{\VectorThree}[3]{
\begin{bmatrix}
 {#1} \\
 {#2} \\
 {#3}
\end{bmatrix}
}


\newcommand{\LL}[0]{\mathcal{L}}

\newcommand{\qdot}[0]{\dot{q}}
\newcommand{\ndot}[0]{\dot{n}}
\newcommand{\qbar}[0]{\bar{q}}
\newcommand{\qdotbar}[0]{\dot{\bar{q}}}
\newcommand{\qddot}[0]{\ddot{q}}
\newcommand{\PD}[2]{\frac{\partial {#2}}{\partial {#1}}}
\newcommand{\grad}[0]{\nabla}

\usepackage[bookmarks=true]{hyperref}


\title{Euler Lagrange Equations.}
\author{Peeter Joot}
\date{ October 13, 2008.  Last Revision: $Date: 2008/10/15 03:03:39 $ }

\begin{document}

\maketitle{}

\tableofcontents

\section{Scalar form of Euler-Lagrange equations.}

\cite{lasenby1993mda} presents a multivector Lagrangian treatment.  To
preparation for understanding that I've gone 
back and derived the scalar
case myself.  As in my recent field lagrangian derivations Feynmans
\cite{feynman1963flp} simple action procedure will be used.

Write 

\begin{align*}
\LL &= \LL(q^i, \qdot^i, \lambda) \\
q^i &= \qbar^i + n^i \\
S &= \int_{\partial \lambda} \LL d\lambda
\end{align*}

Here $\qbar^i$ are the desired optimial solutions, and the functions $n^i$
are all zero at the end points of the integration range $\partial \lambda$.

A first order taylor expansion of a multivariable function
$f(a^i) = f(a^1, a^2, \cdots, a^n)$
takes the form

\begin{align*}
f(a^i + x^i) \approx f(a^i) + \sum_i (a^i + x^i) \left. \PD{x^i}{f} \right\vert_{x^i = a^i}
\end{align*}

In this case the $x^i$ take the values $q^i$, and $\qdot^i$, so the first
order Lagrangian approximation requires summation over differential contributions for both sets of terms

\begin{align}\label{eqn:linearizedLagrangian}
\LL(q^i, \qdot^i, \lambda) 
&\approx \LL(\qbar^i, \qdotbar^i, \lambda) 
+ \sum_i (\qbar^i + n^i) \left. \PD{q^i}{\LL} \right\vert_{q^i = \qbar^i}
+ \sum_i (\qdotbar^i + \ndot^i) \left. \PD{\qdot^i}{\LL} \right\vert_{q^i = \qbar^i}
\end{align}

%Here subscript $0$ represents evaluation of the integrals at $q_j = \qbar_j$, 
%or $\qdot_j = \qdotbar_j$.
%
Now form the action, and group the terms in fixed and variable sets

\begin{align*}
S &= \int \LL d\lambda \\
&\approx
\int d\lambda 
\left(
\LL(\qbar^i, \qdotbar^i, \lambda)
+ \sum_i \qbar^i \left. \PD{q^i}{\LL} \right\vert_{q^i = \qbar^i}
+ \sum_i \qdotbar^i \left. \PD{\qdot^i}{\LL} \right\vert_{q^i = \qbar^i}
\right) \\
&+
\underbrace{
\sum_i \int d\lambda
\left(
n^i \left. \PD{q^i}{\LL} \right\vert_{q^i = \qbar^i}
+\ndot^i \left. \PD{\qdot^i}{\LL} \right\vert_{q^i = \qbar^i}
\right)
}_{\delta S}
\end{align*}

For the optimal solution we want $\delta S = 0$ for all possible paths $n^i$.  Now do the integration by parts writing
$u' = \ndot^i$, and $v = \partial \LL/{\partial \qdot^i}$ 

\begin{align*}
\int u' v = u v - \int u v'
\end{align*}

The action variation is then

\begin{align*}
\delta S =
+ \sum_i \left. \left( n^i \PD{\qdot^i}{\LL} \right) \right\vert_{\partial \lambda}
+ \sum_i \int d\lambda n^i
\left(
\left. \PD{q^i}{\LL} \right\vert_{q^i = \qbar^i}
-\frac{d}{d\lambda} \left. \PD{\qdot^i}{\LL} \right\vert_{q^i = \qbar^i}
\right)
\end{align*}

The non-integral term is zero since by definition $n^i = 0$ on the boundary of the desired integration region, so for the
total variation to equal zero for all possible paths $n^i$ one must have

\begin{align}\label{eqn:eulerlag}
\PD{q^i}{\LL} -\frac{d}{d\lambda} \PD{\qdot^i}{\LL} = 0.
\end{align}

Evaluation of these derivatives at the optimial desired paths has been supressed since these equations now define that path.

\subsection{ Some comparision to the Goldstein approach. }

\cite{goldstein1951cm} calls the quantity \ref{eqn:eulerlag} the functional derivative

\begin{align*}
\frac{\delta S}{\delta q^i} = \PD{q^i}{\LL} -\frac{d}{d\lambda} \PD{\qdot^i}{\LL}
\end{align*}

(with higher order derivatives if the Lagrangian has dependencies on more than generalized position and velocity terms).  Goldstein's 
approach is also harder to follow than Feynman's (Goldstein introduces a parameter $\epsilon$, writing

\begin{align}\label{eqn:epsilonvariation}
q^i = \qbar^i + \epsilon n^i
\end{align}

He then takes derivatives under the integral sign for the end result.

While his approach is a bit harder to follow initially, that additional $\epsilon$ parameterization of the variation path also fits nicely with this
linearization procedure.
After the integration by parts and subsequent differentiation under integral sign nicely does the job of
discarding all the ``fixed'' $\qbar^i$ contributions to the action leaving:

\begin{align*}
\frac{dS}{d\epsilon} = \int d\lambda \sum_i n^i \left. \frac{\delta S}{\delta q^i} \right\vert_{q^i = \qbar^i}
\end{align*}

Introducing this idea does firm things up, eliminating some handwaving.  To obtain the extremal solution it does
make sense to set the derivate of the action equal to zero, and introducing an additional scalar variational control
in the paths from the optimimal solution provides that something to take derivatives with respect to.

Goldstein also writes that this action derivative is then evaluated at $\epsilon = 0$.  This really says the same
thing as Feynman... toss all the higher order terms, since factors of epsilon will be left associated with of these.
With my initial read of Goldstein this wasn't the least bit clear... it was really yet another example of the classic
physics approach of solving something with a first order linear approximation.

\section{ Vector formulation. }

\subsection{ Simple case.  Unforced purely kinetic Lagrangian. }

Before considering multivector Lagrangians, a step back to the simplest vector Lagrangrian is in order

\begin{align*}
\LL = \inv{2}m \dot{\Bx} \cdot \dot{\Bx}
\end{align*}

Writing $\Bx(\lambda) = \bar{\Bx} + \epsilon \Bn$, and using the variational technique directly the equation of motion for this unforced path should follow directly
in vector form

\begin{align*}
S = \int d\lambda \inv{2} m \dot{\bar{\Bx}}^2 + \int m d\lambda \epsilon \dot{\bar{\Bx}} \cdot \dot{\Bn} + \int d\lambda \inv{2} m \epsilon^2 \dot{\bar{\Bn}}^2
\end{align*}

Integration by parts operating directly on the vector function we have

\begin{align*}
\left.\frac{d S}{d\epsilon}\right\vert_{\epsilon=0} 
&= \left.{m \dot{\bar{\Bx}} \cdot {\Bn}}\right\vert_{\partial \lambda} - \int m d\lambda \ddot{\bar{\Bx}} \cdot {\Bn} \\
&= - \int m d\lambda \ddot{\bar{\Bx}} \cdot {\Bn} \\
\end{align*}

Introducing shorthand $\delta S/\delta \Bx$, for a vector functional derivative, we have
\begin{align*}
\left.\frac{d S}{d\epsilon}\right\vert_{\epsilon=0} &= \int d\lambda \Bn \cdot \frac{\delta S}{\delta \Bx},
\end{align*}

where the extremal condition is
\begin{align*}
\frac{\delta S}{\delta \Bx} = - m \ddot{\bar{\Bx}} = 0.
\end{align*}

Here the expected and desired Euler Lagrange equation for the Lagrangian (constant velocity in some direction dependent on initial conditions) is arrived at directly in vector form without dropping down to coordinates and reassembling them to get back the vector expression.

\subsection{ Position and velocity gradients in the configuration space. }

Having tackled the simplest case, to generalize this we need a construct to do first order taylor series expansion in the neighbourhood of a vector
position.  The (multivector) gradient is the obvious candidate operator to do the job.
Before going down that road consider the scalar Lagrangian case once more, where we will see that it is natural to define position and velocity gradients
in the configuration space.  It will also be observed that the chain rule essentially motivates the initially somewhat odd seeming reciprocal basis
used to express the gradient when operating in a non-orthonormal frame.

In equation \ref{eqn:linearizedLagrangian}, the linear differential increment in the neighbourhood of the optimal solution had the form

\begin{align}\label{eqn:gradientMotivator}
\Delta \LL &=
+ \sum_i (\qbar^i + n^i) \left. \PD{q^i}{\LL} \right\vert_{q^i = \qbar^i}
+ \sum^i (\qdotbar^i + \ndot^i) \left. \PD{\qdot^i}{\LL} \right\vert_{q^i = \qbar^i}
\end{align}

If one defines a configuration space position and velocity gradients respectively as

\begin{align*}
\grad_{\Bq} &= \left(\PD{q^1}{}, \PD{q^2}{}, \cdots, \PD{q^n}{}\right) = f_k \PD{q^k}{} \\
\grad_{\dot{\Bq}} &= \left(\PD{\qdot^1}{}, \PD{\qdot^2}{}, \cdots, \PD{\qdot^n}{}\right) = f_k \PD{\qdot^k}{}
\end{align*}

and forms a configuration space vector with respect to some linearly independent, but not neccessarily orthonormal, basis

\begin{align*}
\Bq = q^i e_i
\end{align*}

then the chain rule dictates the relationship between the configuration vector basis and the basis with which the gradient must be expressed.  In
particular, if we wish to write \ref{eqn:gradientMotivator} in terms of the configuration space gradients

\begin{align*}
\Delta \LL &=
(\bar{\Bq} + \Bn) \cdot \left.\grad_\Bq \LL \right\vert_{\Bq = \bar{\Bq}}
+ (\dot{\bar{\Bq}} + \dot{\Bn}) \cdot \left.\grad_{\dot{\Bq}} \LL \right\vert_{\dot{\Bq} = \dot{\bar{\Bq}}}
\end{align*}

Then we must have a reciprocal relationship between the basis vector for the configuration space vectors $e_i$, and the corresponding vectors
from which the gradient was formed

\begin{align*}
e_i \cdot f_j &= \delta_{i j} \\
\implies \\
f_j = e^j
\end{align*}

This gives us the position and velocity gradients in the configuration space

\begin{align}
\grad_{\Bq} &= e^k \PD{q^k}{} \\
\grad_{\dot{\Bq}} &= e^k \PD{\qdot^k}{}.
\end{align}

Note also that the size of this configuration space does not have to be the same space as the problem.  With this definitions completion of the integration
by parts yields the Euler-Lagrange equations in a hybrid configuration space vector form

\begin{align}
\grad_{\Bq} \LL = \frac{d}{d\lambda} \grad_{\dot{\Bq}} \LL
\end{align}

When the configuration space equals the geometrical space being operated in (ie: generalized coordinates are regular old coordinates), this 
provides a nice explaination for why we must have the funny pairing of upper index coordinates in the partials of the gradient and reciprocal frame
vectors multiplying all these partials.  Contrast to a text like \cite{doran2003gap} where the gradient (and spacetime gradient) are defined in this
fashion instead, and one gradually sees that this does in fact work out.

That said, the negative side of this vector notation is that 
it obscures somewhat the Euler-Lagrange equations, which are not terribly complicated to begin with.  However, since this appears to be the form
of the multivector form of the Euler-Lagrange equations it is likely worthwhile to see how this also expresses the simpler familar scalar case too.

\bibliographystyle{plainnat} % supposed to allow for \url use.
\bibliography{myrefs}      % expects file "myrefs.bib"

\end{document}               % End of document.
