\documentclass{article}      % Specifies the document class

\usepackage{amsmath}
\usepackage{mathpazo}

%
% shorthand for bold symbols, convenient for vectors and matrices
%
\newcommand{\Ba}[0]{\mathbf{a}}
\newcommand{\Bb}[0]{\mathbf{b}}
\newcommand{\Bc}[0]{\mathbf{c}}
\newcommand{\Bd}[0]{\mathbf{d}}
\newcommand{\Be}[0]{\mathbf{e}}
\newcommand{\Bf}[0]{\mathbf{f}}
\newcommand{\Bg}[0]{\mathbf{g}}
\newcommand{\Bh}[0]{\mathbf{h}}
\newcommand{\Bi}[0]{\mathbf{i}}
\newcommand{\Bj}[0]{\mathbf{j}}
\newcommand{\Bk}[0]{\mathbf{k}}
\newcommand{\Bl}[0]{\mathbf{l}}
\newcommand{\Bm}[0]{\mathbf{m}}
\newcommand{\Bn}[0]{\mathbf{n}}
\newcommand{\Bo}[0]{\mathbf{o}}
\newcommand{\Bp}[0]{\mathbf{p}}
\newcommand{\Bq}[0]{\mathbf{q}}
\newcommand{\Br}[0]{\mathbf{r}}
\newcommand{\Bs}[0]{\mathbf{s}}
\newcommand{\Bt}[0]{\mathbf{t}}
\newcommand{\Bu}[0]{\mathbf{u}}
\newcommand{\Bv}[0]{\mathbf{v}}
\newcommand{\Bw}[0]{\mathbf{w}}
\newcommand{\Bx}[0]{\mathbf{x}}
\newcommand{\By}[0]{\mathbf{y}}
\newcommand{\Bz}[0]{\mathbf{z}}
\newcommand{\BA}[0]{\mathbf{A}}
\newcommand{\BB}[0]{\mathbf{B}}
\newcommand{\BC}[0]{\mathbf{C}}
\newcommand{\BD}[0]{\mathbf{D}}
\newcommand{\BE}[0]{\mathbf{E}}
\newcommand{\BF}[0]{\mathbf{F}}
\newcommand{\BG}[0]{\mathbf{G}}
\newcommand{\BH}[0]{\mathbf{H}}
\newcommand{\BI}[0]{\mathbf{I}}
\newcommand{\BJ}[0]{\mathbf{J}}
\newcommand{\BK}[0]{\mathbf{K}}
\newcommand{\BL}[0]{\mathbf{L}}
\newcommand{\BM}[0]{\mathbf{M}}
\newcommand{\BN}[0]{\mathbf{N}}
\newcommand{\BO}[0]{\mathbf{O}}
\newcommand{\BP}[0]{\mathbf{P}}
\newcommand{\BQ}[0]{\mathbf{Q}}
\newcommand{\BR}[0]{\mathbf{R}}
\newcommand{\BS}[0]{\mathbf{S}}
\newcommand{\BT}[0]{\mathbf{T}}
\newcommand{\BU}[0]{\mathbf{U}}
\newcommand{\BV}[0]{\mathbf{V}}
\newcommand{\BW}[0]{\mathbf{W}}
\newcommand{\BX}[0]{\mathbf{X}}
\newcommand{\BY}[0]{\mathbf{Y}}
\newcommand{\BZ}[0]{\mathbf{Z}}

\newcommand{\Bzero}[0]{\mathbf{0}}
\newcommand{\Btheta}[0]{\boldsymbol{\theta}}
\newcommand{\Btau}[0]{\boldsymbol{\tau}}
\newcommand{\Bomega}[0]{\boldsymbol{\omega}}

%
% shorthand for unit vectors
%
\newcommand{\acap}[0]{\hat{\Ba}}
\newcommand{\bcap}[0]{\hat{\Bb}}
\newcommand{\ccap}[0]{\hat{\Bc}}
\newcommand{\dcap}[0]{\hat{\Bd}}
\newcommand{\ecap}[0]{\hat{\Be}}
\newcommand{\fcap}[0]{\hat{\Bf}}
\newcommand{\gcap}[0]{\hat{\Bg}}
\newcommand{\hcap}[0]{\hat{\Bh}}
\newcommand{\icap}[0]{\hat{\Bi}}
\newcommand{\jcap}[0]{\hat{\Bj}}
\newcommand{\kcap}[0]{\hat{\Bk}}
\newcommand{\lcap}[0]{\hat{\Bl}}
\newcommand{\mcap}[0]{\hat{\Bm}}
\newcommand{\ncap}[0]{\hat{\Bn}}
\newcommand{\ocap}[0]{\hat{\Bo}}
\newcommand{\pcap}[0]{\hat{\Bp}}
\newcommand{\qcap}[0]{\hat{\Bq}}
\newcommand{\rcap}[0]{\hat{\Br}}
\newcommand{\scap}[0]{\hat{\Bs}}
\newcommand{\tcap}[0]{\hat{\Bt}}
\newcommand{\ucap}[0]{\hat{\Bu}}
\newcommand{\vcap}[0]{\hat{\Bv}}
\newcommand{\wcap}[0]{\hat{\Bw}}
\newcommand{\xcap}[0]{\hat{\Bx}}
\newcommand{\ycap}[0]{\hat{\By}}
\newcommand{\zcap}[0]{\hat{\Bz}}
\newcommand{\thetacap}[0]{\hat{\Btheta}}

%
% to write R^n and C^n in a distinguishable fashion.  Perhaps change this
% to the double lined characters upon figuring out how to do so.
%
\newcommand{\C}[1]{$\mathbb{C}^{#1}$}
\newcommand{\R}[1]{$\mathbb{R}^{#1}$}

%
% various generally useful helpers
%

% derivative of #1 wrt. #2:
\newcommand{\D}[2] {\frac {d#2} {d#1}}

\newcommand{\inv}[1]{\frac{1}{#1}}
\newcommand{\cross}[0]{\times}

\newcommand{\abs}[1]{\lvert{#1}\rvert}
\newcommand{\norm}[1]{\lVert{#1}\rVert}
\newcommand{\innerprod}[2]{\langle{#1}, {#2}\rangle}
\newcommand{\dotprod}[2]{{#1} \cdot {#2}}
\newcommand{\bdotprod}[2]{\left({#1} \cdot {#2}\right)}
\newcommand{\crossprod}[2]{{#1} \cross {#2}}
\newcommand{\tripleprod}[3]{\dotprod{\left(\crossprod{#1}{#2}\right)}{#3}}

\DeclareMathOperator{\Proj}{Proj}
\DeclareMathOperator{\Span}{span}
\DeclareMathOperator{\Sgn}{sgn}
\DeclareMathOperator{\Area}{Area}
\DeclareMathOperator{\Volume}{Volume}

%
% A few miscellaneous things specific to this document
%
\newcommand{\crossop}[1]{\crossprod{#1}{}}

% R2 vector.
\newcommand{\VectorTwo}[2]{
\begin{bmatrix}
 {#1} \\
 {#2}
\end{bmatrix}
}

\newcommand{\VectorN}[1]{
\begin{bmatrix}
{#1}_1 \\
{#1}_2 \\
\vdots \\
{#1}_N \\
\end{bmatrix}
}

\newcommand{\DETuvij}[4]{
\begin{vmatrix}
 {#1}_{#3} & {#1}_{#4} \\
 {#2}_{#3} & {#2}_{#4}
\end{vmatrix}
}

\newcommand{\DETuvwijk}[6]{
\begin{vmatrix}
 {#1}_{#4} & {#1}_{#5} & {#1}_{#6} \\
 {#2}_{#4} & {#2}_{#5} & {#2}_{#6} \\
 {#3}_{#4} & {#3}_{#5} & {#3}_{#6}
\end{vmatrix}
}

\newcommand{\DETuvwxijkl}[8]{
\begin{vmatrix}
 {#1}_{#5} & {#1}_{#6} & {#1}_{#7} & {#1}_{#8} \\
 {#2}_{#5} & {#2}_{#6} & {#2}_{#7} & {#2}_{#8} \\
 {#3}_{#5} & {#3}_{#6} & {#3}_{#7} & {#3}_{#8} \\
 {#4}_{#5} & {#4}_{#6} & {#4}_{#7} & {#4}_{#8} \\
\end{vmatrix}
}

%\newcommand{\DETuvwxyijklm}[10]{
%\begin{vmatrix}
% {#1}_{#6} & {#1}_{#7} & {#1}_{#8} & {#1}_{#9} & {#1}_{#10} \\
% {#2}_{#6} & {#2}_{#7} & {#2}_{#8} & {#2}_{#9} & {#2}_{#10} \\
% {#3}_{#6} & {#3}_{#7} & {#3}_{#8} & {#3}_{#9} & {#3}_{#10} \\
% {#4}_{#6} & {#4}_{#7} & {#4}_{#8} & {#4}_{#9} & {#4}_{#10} \\
% {#5}_{#6} & {#5}_{#7} & {#5}_{#8} & {#5}_{#9} & {#5}_{#10}
%\end{vmatrix}
%}

% R3 vector.
\newcommand{\VectorThree}[3]{
\begin{bmatrix}
 {#1} \\
 {#2} \\
 {#3}
\end{bmatrix}
}


\newcommand{\T}[0]{\text{T}}

%
% The real thing:
%

                             % The preamble begins here.
\title{ Oblique projection and reciprocal frame vectors. }
\author{Peeter Joot}         % Declares the author's name.
%\date{}        % Deleting this command produces today's date.

\begin{document}             % End of preamble and beginning of text.

\maketitle{}

\section{ Motivation. }

Followup on wikipedia projection article's description of an oblique
projection.  Calculate this myself.

\section{ Using GA.  Oblique projection onto a line. }

INSERT DIAGRAM.

Problem is to project a vector $\Bx$ onto a line with direction $\pcap$, along a direction vector $\dcap$.

Write:

\[
\Bx + \alpha \dcap = \beta \pcap
\]

and solve for $\Bp = \beta \pcap$.  Wedging with $\dcap$ provides the solution:

\[
\Bx \wedge \dcap + \alpha \underbrace{\dcap \wedge \dcap}_{=0} = \beta \pcap \wedge \dcap
\]
\[
\implies
\beta = \frac{\Bx \wedge \dcap}{\pcap \wedge \dcap}
\]

So the ``oblique'' projection onto this line (using direction $\dcap$) is:

\begin{equation}
\Proj_{\dcap \rightarrow \pcap}(\Bx) = 
\frac{\Bx \wedge \dcap}{\pcap \wedge \dcap} \pcap
\end{equation}

This also shows that we do not need unit vectors for this sort of projection
operation, since we can scale these two vectors by any quantity since they are
in both the numerator and denominator.

Let $\BD$, and $\BP$ be vectors in the directions of $\dcap$, and $\pcap$ respectively.  Then the projection can also be written:

\begin{equation}\label{eqn:obliqueGAproj}
\Proj_{\BD \rightarrow \BP}(\Bx) = 
\frac{\Bx \wedge \BD}{\BP \wedge \BD} \BP
\end{equation}

It's interesting to see projection expressed here without any sort of dot
product when all our previous projection calculations had intrinsic
requirements for a metric.

Because this result is intrisically non-metric, if convienient one can introduce one and express this result with that too.  Such an expansion is:

\begin{align*}
\frac{\Bx \wedge \BD}{\BP \wedge \BD} \BP
&=
\Bx \wedge \BD 
\frac{\BD \wedge \BP}{\BD \wedge \BP}
\inv{\BP \wedge \BD}
\BP \\
&=
(\Bx \wedge \BD) \cdot (\BD \wedge \BP)
\inv{\abs{\BP \wedge \BD}^2}
\BP \\
&=
((\Bx \wedge \BD) \cdot \BD) \cdot \BP
\inv{\abs{\BP \wedge \BD}^2}
\BP \\
&=
(
\Bx \BD^2 - \Bx \cdot \BD \BD
) \cdot \BP
\inv{\abs{\BP \wedge \BD}^2}
\BP \\
&=
\frac{\Bx \cdot \BP \BD^2 - \Bx \cdot \BD \BD \cdot \BP}
{\BP^2 \BD^2 - (\BP \cdot \BD)^2
%p ^ d . d ^ p = p ^ d . d . p = (p d^2 - d.p d) . p = p^2 d^2 - (d.p)^2
}
\BP \\
&=
\Bx \cdot \frac{\BP \BD^2 - \BD \BD \cdot \BP}{\BP^2 \BD^2 - (\BP \cdot \BD)^2}
\BP \\
\end{align*}

Now, this doesn't really simplify things at all, but my aim was to take the GA result from equation \ref{eqn:obliqueGAproj} and put it into
matrix form.

Assuming a euclidian metrix, and a bit of playing shows that the denominator can be written more simply as:

\[
\BP^2 \BD^2 - (\BP \cdot \BD)^2 = 
\begin{vmatrix}
U^\T U
\end{vmatrix}
\]

where:

\[
U = 
\begin{bmatrix}
P & D
\end{bmatrix}
\]

Similarily the numerator can be written:

\[
\Bx \cdot \BP \BD^2 - \Bx \cdot \BD \BD \cdot \BP = 
D^\T U 
\begin{bmatrix}
0 & -1 \\
1 & 0 \\
\end{bmatrix}
U^\T \Bx.
\]

Combining these yields a projection matrix:

\begin{equation}
\Proj_{\BD \rightarrow \BP}(\Bx) = 
\left(
\BP
\inv{
\begin{vmatrix}
U^\T U
\end{vmatrix}
}
\BD^\T U 
\begin{bmatrix}
0 & -1 \\
1 & 0 \\
\end{bmatrix}
U^\T\right) \Bx.
\end{equation}

The alternation above suggests that this is related to the matrix inverse of something.  Let's try to calculate this directly instead.

\section{ Oblique projection onto a line using matrices. }

\end{document}               % End of document.
