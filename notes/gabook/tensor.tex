\documentclass{article}      % Specifies the document class

\usepackage{amsmath}

%
% shorthand for bold symbols, convenient for vectors and matrices
%
\newcommand{\Ba}[0]{\mathbf{a}}
\newcommand{\Bb}[0]{\mathbf{b}}
\newcommand{\Bc}[0]{\mathbf{c}}
\newcommand{\Bd}[0]{\mathbf{d}}
\newcommand{\Be}[0]{\mathbf{e}}
\newcommand{\Bf}[0]{\mathbf{f}}
\newcommand{\Bg}[0]{\mathbf{g}}
\newcommand{\Bh}[0]{\mathbf{h}}
\newcommand{\Bi}[0]{\mathbf{i}}
\newcommand{\Bj}[0]{\mathbf{j}}
\newcommand{\Bk}[0]{\mathbf{k}}
\newcommand{\Bl}[0]{\mathbf{l}}
\newcommand{\Bm}[0]{\mathbf{m}}
\newcommand{\Bn}[0]{\mathbf{n}}
\newcommand{\Bo}[0]{\mathbf{o}}
\newcommand{\Bp}[0]{\mathbf{p}}
\newcommand{\Bq}[0]{\mathbf{q}}
\newcommand{\Br}[0]{\mathbf{r}}
\newcommand{\Bs}[0]{\mathbf{s}}
\newcommand{\Bt}[0]{\mathbf{t}}
\newcommand{\Bu}[0]{\mathbf{u}}
\newcommand{\Bv}[0]{\mathbf{v}}
\newcommand{\Bw}[0]{\mathbf{w}}
\newcommand{\Bx}[0]{\mathbf{x}}
\newcommand{\By}[0]{\mathbf{y}}
\newcommand{\Bz}[0]{\mathbf{z}}
\newcommand{\BA}[0]{\mathbf{A}}
\newcommand{\BB}[0]{\mathbf{B}}
\newcommand{\BC}[0]{\mathbf{C}}
\newcommand{\BD}[0]{\mathbf{D}}
\newcommand{\BE}[0]{\mathbf{E}}
\newcommand{\BF}[0]{\mathbf{F}}
\newcommand{\BG}[0]{\mathbf{G}}
\newcommand{\BH}[0]{\mathbf{H}}
\newcommand{\BI}[0]{\mathbf{I}}
\newcommand{\BJ}[0]{\mathbf{J}}
\newcommand{\BK}[0]{\mathbf{K}}
\newcommand{\BL}[0]{\mathbf{L}}
\newcommand{\BM}[0]{\mathbf{M}}
\newcommand{\BN}[0]{\mathbf{N}}
\newcommand{\BO}[0]{\mathbf{O}}
\newcommand{\BP}[0]{\mathbf{P}}
\newcommand{\BQ}[0]{\mathbf{Q}}
\newcommand{\BR}[0]{\mathbf{R}}
\newcommand{\BS}[0]{\mathbf{S}}
\newcommand{\BT}[0]{\mathbf{T}}
\newcommand{\BU}[0]{\mathbf{U}}
\newcommand{\BV}[0]{\mathbf{V}}
\newcommand{\BW}[0]{\mathbf{W}}
\newcommand{\BX}[0]{\mathbf{X}}
\newcommand{\BY}[0]{\mathbf{Y}}
\newcommand{\BZ}[0]{\mathbf{Z}}

\newcommand{\Bzero}[0]{\mathbf{0}}
\newcommand{\Btheta}[0]{\boldsymbol{\theta}}
\newcommand{\Btau}[0]{\boldsymbol{\tau}}
\newcommand{\Bomega}[0]{\boldsymbol{\omega}}

%
% shorthand for unit vectors
%
\newcommand{\acap}[0]{\hat{\Ba}}
\newcommand{\bcap}[0]{\hat{\Bb}}
\newcommand{\ccap}[0]{\hat{\Bc}}
\newcommand{\dcap}[0]{\hat{\Bd}}
\newcommand{\ecap}[0]{\hat{\Be}}
\newcommand{\fcap}[0]{\hat{\Bf}}
\newcommand{\gcap}[0]{\hat{\Bg}}
\newcommand{\hcap}[0]{\hat{\Bh}}
\newcommand{\icap}[0]{\hat{\Bi}}
\newcommand{\jcap}[0]{\hat{\Bj}}
\newcommand{\kcap}[0]{\hat{\Bk}}
\newcommand{\lcap}[0]{\hat{\Bl}}
\newcommand{\mcap}[0]{\hat{\Bm}}
\newcommand{\ncap}[0]{\hat{\Bn}}
\newcommand{\ocap}[0]{\hat{\Bo}}
\newcommand{\pcap}[0]{\hat{\Bp}}
\newcommand{\qcap}[0]{\hat{\Bq}}
\newcommand{\rcap}[0]{\hat{\Br}}
\newcommand{\scap}[0]{\hat{\Bs}}
\newcommand{\tcap}[0]{\hat{\Bt}}
\newcommand{\ucap}[0]{\hat{\Bu}}
\newcommand{\vcap}[0]{\hat{\Bv}}
\newcommand{\wcap}[0]{\hat{\Bw}}
\newcommand{\xcap}[0]{\hat{\Bx}}
\newcommand{\ycap}[0]{\hat{\By}}
\newcommand{\zcap}[0]{\hat{\Bz}}
\newcommand{\thetacap}[0]{\hat{\Btheta}}

%
% to write R^n and C^n in a distinguishable fashion.  Perhaps change this
% to the double lined characters upon figuring out how to do so.
%
\newcommand{\C}[1]{${\BC}^{#1}$}
\newcommand{\R}[1]{${\BR}^{#1}$}

%
% various generally useful helpers
%

% derivative of #1 wrt. #2:
\newcommand{\D}[2] {\frac {d#2} {d#1}}

\newcommand{\inv}[1]{\frac{1}{#1}}
\newcommand{\cross}[0]{\times}

\newcommand{\abs}[1]{\lvert#1\rvert}
\newcommand{\norm}[1]{\lVert#1\rVert}
\newcommand{\innerprod}[2]{\langle{#1}, {#2}\rangle}
\newcommand{\dotprod}[2]{#1 \cdot #2}
\newcommand{\crossprod}[2]{#1 \cross #2}
\newcommand{\tripleprod}[3]{\dotprod{\crossprod{#1}{#2}}{#3}}

%
% A few miscellaneous things specific to this document
%
\newcommand{\crossop}[1]{\crossprod{#1}{}}

\newcommand{\PDP}[2]{\BP^{#1}\BD{\BP^{#2}}}
\newcommand{\PDPDP}[3]{\Bv^T\BP^{#1}\BD\BP^{#2}\BD\BP^{#3}\Bv}

\newcommand{\Mp}[0]{
\begin{bmatrix}
0 & 1 & 0 & 0 \\
0 & 0 & 1 & 0 \\
0 & 0 & 0 & 1 \\
1 & 0 & 0 & 0
\end{bmatrix}
}
\newcommand{\Mpp}[0]{
\begin{bmatrix}
0 & 0 & 1 & 0 \\
0 & 0 & 0 & 1 \\
1 & 0 & 0 & 0 \\
0 & 1 & 0 & 0
\end{bmatrix}
}
\newcommand{\Mppp}[0]{
\begin{bmatrix}
0 & 0 & 0 & 1 \\
1 & 0 & 0 & 0 \\
0 & 1 & 0 & 0 \\
0 & 0 & 1 & 0
\end{bmatrix}
}
\newcommand{\Mpu}[0]{
\begin{bmatrix}
u_1 & 0 & 0 & 0 \\
0 & u_2 & 0 & 0 \\
0 & 0 & u_3 & 0 \\
0 & 0 & 0 & u_4
\end{bmatrix}
}

%
% The real thing:
%

                             % The preamble begins here.
\title{ Covarient/vector derivativate notes, plus notes on raised and lowered indexes.  }
\author{Peeter Joot}         % Declares the author's name.

%\date{}        % Deleting this command produces today's date.

\begin{document}             % End of preamble and beginning of text.

\maketitle{}

\section{Introduction/Abstract}

\subsection{ Raised and lowered indexes. Coordinates of vectors with non-orthonormal frames. }

Let $\{ e_i \}$ represent a frame of not necessarily orthonormal basis vectors for a metric space, and $\{ e^i \}$ represent the reciprocal frame.

The reciprocal frame vectors are defined by the relation:

\begin{equation}
e_i \cdot e^j = {\delta_i}^j.
\end{equation}

Lets compute the coordinates of a vector $x$ in terms of both frames:

\[
x = \sum \alpha_j e_j = \sum \beta_j e^j
\]

Forming $x \cdot e^i$, and $x \cdot e_i$ respectively solves for the $\alpha$, and $\beta$ coefficients

\[
x \cdot e^i = \sum \alpha_j e_j \cdot e^i = \sum \alpha_j {\delta_j}^i = \alpha_i
\]

\[
x \cdot e_i = \sum \beta_j e^j \cdot e_i = \sum \beta_j {\delta_i}^j = \beta_i
\]

Thus, the reciprocal frame vectors allow for simple determination of coordinates for an arbitrary frame. We can summarize this as follows:

\[
x = \sum ( x \cdot e^i ) e_i = \sum ( x \cdot e_i ) e^i
\]

Now, for orthonormal frames we are used to writing:

\[
x = \sum x_i e_i,
\]

however for non-orthonormal frames the convention is to mix raised and lowered indexes as follows:

\[
x = \sum x^i e_i = \sum x_i e^i.
\]

Where, as demonstrated above these generalized coordinates have the values, $x^i = x \cdot e^i$, and $x_i = x \cdot e_i$.

\subsection{ Metric tensor. }

It is customary in tensor formulations of physics to utilize a metric tensor to express the dot product.

Compute the dot product using the coordinate vectors

\[
x \cdot y = \left(\sum x^i e_i \right)\left(\sum y^i e_i \right) = \sum x^i y^i \left( e_i \cdot e_j \right)
\]

\[
x \cdot y = \left(\sum x_i e^i \right)\left(\sum y_i e^i \right) = \sum x_i y_i \left( e^i \cdot e^j \right)
\]

Introducing second rank (symmetric) tensors for the dot product pairs $ e_i \cdot e_j = g_{ij}$, and $ g^{ij} = e^i \cdot e^j $ we have

\[
x \cdot y = \sum x_i y_i g^{ij} = \sum x^i y^i g_{ij}
\]

We see that the metric tensor provides a way to specify the dot product in index notation, and removes the explicit references to the original frame vectors.

Note that it is also common to see Einstein summation convention employed, which omits the $\sum$:

\[
x \cdot y = x_i y_i g^{ij} = x^i y^i g_{ij}
\]

Summation over all upper, lower index pairs is implied.

\subsection{ metric tensor relations to coordinates. }

Given a coordinate expression of a vector, we dot that with the frame vectors to observe the relation between coordinates and the metric tensor:

\[
x \cdot e_i = \sum x^j e_j \cdot e_i = \sum x^j g_{ij}
\]

\[
x \cdot e^i = \sum x_j e^j \cdot e^i = \sum x_j g^{ij}
\]

The metric tensors can therefore be used be used to express the relations between the upper and lower index coordinates:

\begin{align}
x_i &= \sum g_{ij} x^j \label{eqn:metric_upper_to_lower} \\
x^i &= \sum g^{ij} x_j \label{eqn:metric_lower_to_upper}
\end{align}

It is therefore apparent that the matrix of the index lowered metric tensor $g_{ij}$ is the inverse of the matrix for the raised index metric tensor $g^{ij}$.

\subsection{ Metric tensor as a Jacobian }

The relations of equations \ref{eqn:metric_upper_to_lower}, and \ref{eqn:metric_lower_to_upper} show that the metric tensor can be expressed in terms of partial derivatives:

\begin{align}
\frac{\partial x_i }{\partial x^j } &= g_{ij} \\
\frac{\partial x^i }{\partial x_j } &= g^{ij}
\end{align}

Therefore the metric tensors can also be expressed as Jacobian matrixes (not Jacobian determinants) :

\begin{align}
g_{ij} &= \frac{\partial (x_1, \cdots, x_n) }{\partial (x^1, \cdots, x^n) } \\
g^{ij} &= \frac{\partial (x^1, \cdots, x^n) }{\partial (x_1, \cdots, x_n) }
\end{align}

This is expected to be useful in some differential

\subsection{ computation of reciprocal frame vectors }

The reciprocal frame vectors can be computed with matrix methods or using geometric algebra quotients.

%...

\end{document}               % End of document.
