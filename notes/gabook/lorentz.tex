\documentclass{article}      % Specifies the document class

\usepackage{amsmath}
\usepackage{mathpazo}

%
% shorthand for bold symbols, convenient for vectors and matrices
%
\newcommand{\Ba}[0]{\mathbf{a}}
\newcommand{\Bb}[0]{\mathbf{b}}
\newcommand{\Bc}[0]{\mathbf{c}}
\newcommand{\Bd}[0]{\mathbf{d}}
\newcommand{\Be}[0]{\mathbf{e}}
\newcommand{\Bf}[0]{\mathbf{f}}
\newcommand{\Bg}[0]{\mathbf{g}}
\newcommand{\Bh}[0]{\mathbf{h}}
\newcommand{\Bi}[0]{\mathbf{i}}
\newcommand{\Bj}[0]{\mathbf{j}}
\newcommand{\Bk}[0]{\mathbf{k}}
\newcommand{\Bl}[0]{\mathbf{l}}
\newcommand{\Bm}[0]{\mathbf{m}}
\newcommand{\Bn}[0]{\mathbf{n}}
\newcommand{\Bo}[0]{\mathbf{o}}
\newcommand{\Bp}[0]{\mathbf{p}}
\newcommand{\Bq}[0]{\mathbf{q}}
\newcommand{\Br}[0]{\mathbf{r}}
\newcommand{\Bs}[0]{\mathbf{s}}
\newcommand{\Bt}[0]{\mathbf{t}}
\newcommand{\Bu}[0]{\mathbf{u}}
\newcommand{\Bv}[0]{\mathbf{v}}
\newcommand{\Bw}[0]{\mathbf{w}}
\newcommand{\Bx}[0]{\mathbf{x}}
\newcommand{\By}[0]{\mathbf{y}}
\newcommand{\Bz}[0]{\mathbf{z}}
\newcommand{\BA}[0]{\mathbf{A}}
\newcommand{\BB}[0]{\mathbf{B}}
\newcommand{\BC}[0]{\mathbf{C}}
\newcommand{\BD}[0]{\mathbf{D}}
\newcommand{\BE}[0]{\mathbf{E}}
\newcommand{\BF}[0]{\mathbf{F}}
\newcommand{\BG}[0]{\mathbf{G}}
\newcommand{\BH}[0]{\mathbf{H}}
\newcommand{\BI}[0]{\mathbf{I}}
\newcommand{\BJ}[0]{\mathbf{J}}
\newcommand{\BK}[0]{\mathbf{K}}
\newcommand{\BL}[0]{\mathbf{L}}
\newcommand{\BM}[0]{\mathbf{M}}
\newcommand{\BN}[0]{\mathbf{N}}
\newcommand{\BO}[0]{\mathbf{O}}
\newcommand{\BP}[0]{\mathbf{P}}
\newcommand{\BQ}[0]{\mathbf{Q}}
\newcommand{\BR}[0]{\mathbf{R}}
\newcommand{\BS}[0]{\mathbf{S}}
\newcommand{\BT}[0]{\mathbf{T}}
\newcommand{\BU}[0]{\mathbf{U}}
\newcommand{\BV}[0]{\mathbf{V}}
\newcommand{\BW}[0]{\mathbf{W}}
\newcommand{\BX}[0]{\mathbf{X}}
\newcommand{\BY}[0]{\mathbf{Y}}
\newcommand{\BZ}[0]{\mathbf{Z}}

\newcommand{\Bzero}[0]{\mathbf{0}}
\newcommand{\Btheta}[0]{\boldsymbol{\theta}}
\newcommand{\Btau}[0]{\boldsymbol{\tau}}
\newcommand{\Bomega}[0]{\boldsymbol{\omega}}

%
% shorthand for unit vectors
%
\newcommand{\acap}[0]{\hat{\Ba}}
\newcommand{\bcap}[0]{\hat{\Bb}}
\newcommand{\ccap}[0]{\hat{\Bc}}
\newcommand{\dcap}[0]{\hat{\Bd}}
\newcommand{\ecap}[0]{\hat{\Be}}
\newcommand{\fcap}[0]{\hat{\Bf}}
\newcommand{\gcap}[0]{\hat{\Bg}}
\newcommand{\hcap}[0]{\hat{\Bh}}
\newcommand{\icap}[0]{\hat{\Bi}}
\newcommand{\jcap}[0]{\hat{\Bj}}
\newcommand{\kcap}[0]{\hat{\Bk}}
\newcommand{\lcap}[0]{\hat{\Bl}}
\newcommand{\mcap}[0]{\hat{\Bm}}
\newcommand{\ncap}[0]{\hat{\Bn}}
\newcommand{\ocap}[0]{\hat{\Bo}}
\newcommand{\pcap}[0]{\hat{\Bp}}
\newcommand{\qcap}[0]{\hat{\Bq}}
\newcommand{\rcap}[0]{\hat{\Br}}
\newcommand{\scap}[0]{\hat{\Bs}}
\newcommand{\tcap}[0]{\hat{\Bt}}
\newcommand{\ucap}[0]{\hat{\Bu}}
\newcommand{\vcap}[0]{\hat{\Bv}}
\newcommand{\wcap}[0]{\hat{\Bw}}
\newcommand{\xcap}[0]{\hat{\Bx}}
\newcommand{\ycap}[0]{\hat{\By}}
\newcommand{\zcap}[0]{\hat{\Bz}}
\newcommand{\thetacap}[0]{\hat{\Btheta}}

%
% to write R^n and C^n in a distinguishable fashion.  Perhaps change this
% to the double lined characters upon figuring out how to do so.
%
\newcommand{\C}[1]{$\mathbb{C}^{#1}$}
\newcommand{\R}[1]{$\mathbb{R}^{#1}$}

%
% various generally useful helpers
%

% derivative of #1 wrt. #2:
\newcommand{\D}[2] {\frac {d#2} {d#1}}

\newcommand{\inv}[1]{\frac{1}{#1}}
\newcommand{\cross}[0]{\times}

\newcommand{\abs}[1]{\lvert{#1}\rvert}
\newcommand{\norm}[1]{\lVert{#1}\rVert}
\newcommand{\innerprod}[2]{\langle{#1}, {#2}\rangle}
\newcommand{\dotprod}[2]{{#1} \cdot {#2}}
\newcommand{\bdotprod}[2]{\left({#1} \cdot {#2}\right)}
\newcommand{\crossprod}[2]{{#1} \cross {#2}}
\newcommand{\tripleprod}[3]{\dotprod{\left(\crossprod{#1}{#2}\right)}{#3}}

\DeclareMathOperator{\Proj}{Proj}
\DeclareMathOperator{\Span}{span}
\DeclareMathOperator{\Sgn}{sgn}
\DeclareMathOperator{\Area}{Area}
\DeclareMathOperator{\Volume}{Volume}

%
% A few miscellaneous things specific to this document
%
\newcommand{\crossop}[1]{\crossprod{#1}{}}

% R2 vector.
\newcommand{\VectorTwo}[2]{
\begin{bmatrix}
 {#1} \\
 {#2}
\end{bmatrix}
}

\newcommand{\VectorN}[1]{
\begin{bmatrix}
{#1}_1 \\
{#1}_2 \\
\vdots \\
{#1}_N \\
\end{bmatrix}
}

\newcommand{\DETuvij}[4]{
\begin{vmatrix}
 {#1}_{#3} & {#1}_{#4} \\
 {#2}_{#3} & {#2}_{#4}
\end{vmatrix}
}

\newcommand{\DETuvwijk}[6]{
\begin{vmatrix}
 {#1}_{#4} & {#1}_{#5} & {#1}_{#6} \\
 {#2}_{#4} & {#2}_{#5} & {#2}_{#6} \\
 {#3}_{#4} & {#3}_{#5} & {#3}_{#6}
\end{vmatrix}
}

\newcommand{\DETuvwxijkl}[8]{
\begin{vmatrix}
 {#1}_{#5} & {#1}_{#6} & {#1}_{#7} & {#1}_{#8} \\
 {#2}_{#5} & {#2}_{#6} & {#2}_{#7} & {#2}_{#8} \\
 {#3}_{#5} & {#3}_{#6} & {#3}_{#7} & {#3}_{#8} \\
 {#4}_{#5} & {#4}_{#6} & {#4}_{#7} & {#4}_{#8} \\
\end{vmatrix}
}

%\newcommand{\DETuvwxyijklm}[10]{
%\begin{vmatrix}
% {#1}_{#6} & {#1}_{#7} & {#1}_{#8} & {#1}_{#9} & {#1}_{#10} \\
% {#2}_{#6} & {#2}_{#7} & {#2}_{#8} & {#2}_{#9} & {#2}_{#10} \\
% {#3}_{#6} & {#3}_{#7} & {#3}_{#8} & {#3}_{#9} & {#3}_{#10} \\
% {#4}_{#6} & {#4}_{#7} & {#4}_{#8} & {#4}_{#9} & {#4}_{#10} \\
% {#5}_{#6} & {#5}_{#7} & {#5}_{#8} & {#5}_{#9} & {#5}_{#10}
%\end{vmatrix}
%}

% R3 vector.
\newcommand{\VectorThree}[3]{
\begin{bmatrix}
 {#1} \\
 {#2} \\
 {#3}
\end{bmatrix}
}



\newcommand{\laplacian}[0]{\nabla^2}

%
% The real thing:
%

                             % The preamble begins here.
\title{Derive Lorentz transformation from wave equation.} % Declares the document's title.
\author{Peeter Joot \quad peeter.joot@gmail.com}         % Declares the author's name.
\date{ June 25, 2008 }        % Deleting this command produces today's date.

\begin{document}             % End of preamble and beginning of text.

\maketitle{}

\section{Intro.}

My old electrodynamics book did a Lorentz transformation derivation using a requirement for invariance of a spherical light shell.  ie:

\[
x^2 - c^2 t^2 = {x'}^2 - c^2 {t'}^2
\]

Such an approach does not require any sofisticated math, but I never understood why that invariance condition could be assumed.
To understand that intuitively, requires that you understand how the speed of light is constant.  There are some subtlies 
involved in understanding that which are not neccessarily obvious to me.  A good illustration of this is Feynman's question
about what speed to expect light to be
going from a rocket ship going 100000 miles per second is a good example (ref: book: Six not so easy parts).
Many people who would say "yes, the speed of light is constant" would still answer 280000 miles per second for that question.

I present below an alternate approach to deriving the Lorentz transformation.  This has a bit more math (ie: partial differentials for 
change of variables in the wave equation).  However, compared to really understanding that the speed of light is constant,
I think it is easier to
to conceptualize the idea that light is wavelike regardless of the motion of the observer since it (ie: an electrodynamic field)
must satisfy the wave equation (ie: Maxwell's equations) regardless of the parameterization.  I'm curious if somebody
else also new to the subject of relativity would agree?

The motivation for this is the fact that many introductory relativity texts mention how Lorentz observed that
while Maxwell's equations were not invariant with respect to Galilean
transformation, they were with his modified transformation.

I found it interesting to consider this statement with a bit of detail.  The result is what I think is an interesting approach
to introducing the Lorentz transformation.

\section{ The wave equation for Electrodynamic fields (light).}

From Maxwell's equations one can show that in a charge and current free region
the electric field and magnetic field both satisfy the wave equation:

\begin{equation}
\laplacian - \inv{c^2}\frac{\partial^2}{\partial t^2} = 0
\end{equation}

I believe this is the specific case where there are the light contains enough
photons that the bulk (wavelike) phenomena dominate and quantum effects don't have to be considered.

The wikipedia article Electromagnetic radiation (under Derivation)

%\htmladdnormallink{<URL>} { http://en.wikipedia.org/wiki/Electromagnetic_radiation#Derivation }

goes over this nicely.

Although this can be solved separately for either $\BE$ or $\BB$ the two are not independent.
This dependence is nicely expressed by writing the electromagnetic field as a complete
bivector $\BF = \BE + i c \BB$, and in that form the
general solution to this equation for the combined electromagnetic
field is:

\begin{equation}
\BF = (\BE_0 + \kcap \wedge \BE_0) f( \kcap \cdot \Br \pm c t)
\end{equation}

Here f is any function, and represents the amplitude of the waveform.

\section{Verifying Lorentz invariance.}

The Lorentz transform for a moving (primed) frame where the motion is
along the x axis is ($\beta = v/c$, $\gamma^{-2} = 1 -\beta^2$).

\begin{equation*}
\begin{bmatrix}
x' \\
c t' \\
\end{bmatrix}
=
\gamma
\begin{bmatrix}
1 & -\beta \\
-\beta & 1 \\
\end{bmatrix}
\end{equation*}

Or,
\begin{equation*}
\begin{bmatrix}
x \\
c t \\
\end{bmatrix}
=
\gamma
\begin{bmatrix}
1 & \beta \\
\beta & 1 \\
\end{bmatrix}
\end{equation*}

Using this we can express the partials of the wave equation in the
primed frame.  Starting with the first derivatives:

\begin{align*}
\frac{\partial}{\partial x}
&= \frac{\partial x'}{\partial x} \frac{\partial}{\partial x'} + \frac{\partial c t'}{\partial x} \frac{\partial}{\partial c t'} \\
&= \gamma \frac{\partial}{\partial x'} - \gamma \beta \frac{\partial}{\partial c t'} \\
\end{align*}

And:

\begin{align*}
\frac{\partial}{\partial ct}
&= \frac{\partial x'}{\partial ct} \frac{\partial}{\partial x'} + \frac{\partial c t'}{\partial ct} \frac{\partial}{\partial c t'} \\
&= -\beta \gamma \frac{\partial}{\partial x'} + \gamma \frac{\partial}{\partial c t'} \\
\end{align*}

Thus the second partials in terms of the primed frame are:

\begin{align*}
\frac{\partial^2}{\partial x^2}
&= \gamma^2
\left(\frac{\partial}{\partial x'} - \beta \frac{\partial}{\partial c t'} \right)
\left(\frac{\partial}{\partial x'} - \beta \frac{\partial}{\partial c t'} \right)
\\
&= \gamma^2
\left(
\frac{\partial^2}{\partial x'\partial x'} + \beta^2 \frac{\partial^2}{\partial c t'\partial c t'}
- \beta \left(
\frac{\partial^2}{\partial x' \partial c t'}
\frac{\partial^2}{\partial c t' \partial x'}
\right)
\right)
\\
\end{align*}

\begin{align*}
\frac{\partial^2}{\partial ct \partial ct}
&= \gamma^2
\left(
\beta^2 \frac{\partial^2}{\partial x'\partial x'} + \frac{\partial^2}{\partial c t'\partial c t'}
- \beta \left(
\frac{\partial^2}{\partial x' \partial c t'}
\frac{\partial^2}{\partial c t' \partial x'}
\right)
\right)
\\
\end{align*}

Thus the wave equation transforms as:

\begin{align*}
\frac{\partial^2}{\partial x^2} - \frac{\partial^2}{\partial ct \partial ct}
&=
\gamma^2
\left(
(1 - \beta^2) \frac{\partial^2}{\partial x'\partial x'} + (\beta^2 -1)\frac{\partial^2}{\partial c t'\partial c t'}
\right) \\
&=
\frac{\partial^2}{\partial x'\partial x'} - \frac{\partial^2}{\partial c t'\partial c t'}
\end{align*}

which is what we expect but nice to see written out in full without having to introduce Minkowski space, and its invariant norm,
or use Einstein's subtle arguments from his "Relativity, the special and general theory" (the latter requires actual understanding
whereas the former and this just require math).

\section{ Derive Lorentz Transformation requiring invariance of the wave equation. }

Now, lets look at a general change of variables for the wave equation for the electromagnetic field.  This will include
the Galilean transformation, as well as the Lorentz transformation above, as special cases.

Consider a two variable, scaled Laplacian:

\begin{equation}
\laplacian = m \frac{\partial^2}{\partial u^2} + n \frac{\partial^2}{\partial v^2}
\end{equation}

and a linear change of variables defined by:

\begin{equation}
\begin{bmatrix}
u \\
v \\
\end{bmatrix}
=
\begin{bmatrix}
e & f \\
g & h \\
\end{bmatrix}
\begin{bmatrix}
x \\
y \\
\end{bmatrix}
=
A
\begin{bmatrix}
x \\
y \\
\end{bmatrix}
\end{equation}

To perform the change of variables we need to evaluate the following:

\begin{align}
\frac{\partial}{\partial u}
&= \frac{\partial x}{\partial u} \frac{\partial}{\partial x} + \frac{\partial y}{\partial u} \frac{\partial}{\partial y} \\
\frac{\partial}{\partial v}
&= \frac{\partial x}{\partial v} \frac{\partial}{\partial x} + \frac{\partial y}{\partial v} \frac{\partial}{\partial y}
\end{align}

To compute the partials we must invert $A$.  Writing

\[
J =
\begin{vmatrix}
e & f \\
g & h \\
\end{vmatrix}
=
\frac{\partial(u,v)}{\partial(x,y)},
\]

that inverse is

\[
A^{-1} =
\inv
{
\begin{vmatrix}
e & f \\
g & h \\
\end{vmatrix}
}
\begin{bmatrix}
h & -f \\
-g & e \\
\end{bmatrix}.
\]

The first partials are therefore:

\begin{align}
\frac{\partial}{\partial u}
&= \inv{J} \left(h \frac{\partial}{\partial x} - g \frac{\partial}{\partial y}\right) \\
\frac{\partial}{\partial v}
&= \inv{J} \left(-f \frac{\partial}{\partial x} + e \frac{\partial}{\partial y}\right).
\end{align}

Repeating for the second partials yields:

\begin{align}
\frac{\partial^2}{\partial u^2}
&= \inv{J^2} \left(
h^2 \frac{\partial^2}{\partial x^2} + g^2 \frac{\partial^2}{\partial y^2}
-g h \left( \frac{\partial^2}{\partial x \partial y} + \frac{\partial^2}{\partial y \partial x} \right)
\right) \\
\frac{\partial^2}{\partial v^2}
&= \inv{J^2} \left(
f^2 \frac{\partial^2}{\partial x^2} + e^2 \frac{\partial^2}{\partial y^2}
-e f \left( \frac{\partial^2}{\partial x \partial y} + \frac{\partial^2}{\partial y \partial x} \right)
\right)
\end{align}

That's the last calculation required to compute the transformed Laplacian:

\begin{equation}\label{eqn:laplaciantx}
\laplacian = \inv{J^2} 
\left(
(m h^2 + n f^2)\partial_{xx} 
+(m g^2 + n e^2)\partial_{yy} 
-(m g h + n e f)( \partial_{xy} + \partial_{yx}  )
\right)
\end{equation}

\subsection{ Galilean transformation.}

Lets apply this to the electrodynamics wave equation, first using a Galilean transformation $x = x' + v t$, $t = t'$, $\beta = v/c$.

\begin{equation}
\begin{bmatrix}
x \\
ct \\
\end{bmatrix}
=
\begin{bmatrix}
1 & \beta \\
0 & 1 \\
\end{bmatrix}
\begin{bmatrix}
x' \\
c t' \\
\end{bmatrix}
\end{equation}

\begin{equation}
\partial_{xx} -\inv{c^2}\partial_{tt} =
(1 - \beta^2)\partial_{x'x'} 
-\inv{c^2}\partial_{t't'} 
+\inv{c}\beta( \partial_{x't'} + \partial_{t'x'} )
\end{equation}

Thus we see that the equations of light when subjected to a Galalian transformation have a different form after such a
transformation.  If this was correct we should see the effects of the mixed product terms and the reduced effect of the
spatial component when there is any motion.  However, light comes in a wave form regardless of the motion, so there
is something wrong with application of this transformation to the equations of light.  This was the big problem of physics
over a hundred years ago before Einstein introduced relativity to explain all this.

\subsection{ Determine the transformation of coordinates that retains the form of the equations of light. }

Before Einstein, Lorentz worked out the transformation that left Maxwell's equation ``invariant''.  I haven't seen 
any text that actually showed this.  Lorentz may have showed that his transformations left Maxwell's equations 
invariant in their full generality, however that complexity is not required to derive the transformation itself.  Instead
this can be done considering only the wave equation for light in source free space.

Let's define the matrix $A$ for a general change of space and time variables in one spatial dimension:

\begin{equation}
\begin{bmatrix}
x \\
ct \\
\end{bmatrix}
=
\begin{bmatrix}
e & f \\
g & h \\
\end{bmatrix}
\begin{bmatrix}
x' \\
c t' \\
\end{bmatrix}
\end{equation}

Application of this to equation \ref{eqn:laplaciantx} gives:

\begin{equation}\label{eqn:transformedlightwave}
\partial_{xx} -\partial_{ct,ct} =
\inv{J^2} 
\left(
(h^2 - f^2)\partial_{x'x'} 
+(g^2 - e^2)\partial_{ct',ct'} 
-(g h - e f)( \partial_{x',ct'} + \partial_{ct',x'}  )
\right)
\end{equation}

Now, we observe that light has wavelike behavior regardless of our velocity (we do observe frequency variation with 
velocity but the fundamental waviness doesn't change).  Once that is accepted as a requirement for a transformation
of coordinates of the wave equation for light we get the Lorentz transformation.  

Expressed mathematically, this means that we want equation \ref{eqn:transformedlightwave} to have the form:

\begin{equation}
\partial_{xx} -\partial_{ct,ct} = \partial_{x'x'} -\partial_{ct',ct'}
\end{equation}

This requirement is equivalent to the following system of equations:

\begin{align}
J &= eh - fg \\
h^2 - f^2 &= J^2 \\
g^2 - e^2 &= -J^2 \\
g h &= e f.
\end{align}

Attempting to solve this in full generality for any J gets messy (ie: non-linear).
To simplify things, it is not unreasonable to
require $J = 1$, which is consistent with Galilean transformation, in particular for the limiting case as $v \rightarrow 0$.

Additionally, we want to give physical significance to these values $e,f,g,h$.  Following Einstein's simple derivation
of the Lorentz transformation, we do this by defining $x'=0$ as the origin of the moving frame:

\[
x'
=
\inv{J}
\begin{bmatrix}
h & -f \\
\end{bmatrix}
\begin{bmatrix}
x \\
c t \\
\end{bmatrix}
= 0
\]

This allows us to relate $f,h$ to the velocity:

\[
x h = f c t
\]
\[
\implies
\frac{dx}{dt} = \frac{f c}{h} = v,
\]

and provides physical meaning to the first of the elements of the linear transformation:

\begin{equation}\label{eqn:betaterm}
f = h \frac{v}{c} = h \beta.
\end{equation}

The significance and values of $e,g,h$ remain to be determined.  Substituting equation \ref{eqn:betaterm} into our
system of equations we have:

\begin{align*}
h^2 - h^2 \beta^2 &= 1 \\
g^2 - e^2 &= -1 \\
g h &= e h \beta.
\end{align*}

From the first equation we have $h^2 = \inv{1 -\beta^2}$, which is what is usually designated $\gamma^2$.  Considering
the limiting case again of $v \rightarrow 0$, we want to take the positive root.  Summarizing what has been found 
so far we have:

\begin{align}
h &= \inv{\sqrt{1 - \beta^2}} = \gamma \\
f &= \gamma \beta \\
g^2 - e^2 &= -1 \\
g &= e \beta.
\end{align}

Substitution of the last yields

\[
e^2 (\beta^2 - 1) = -1
\]

which means that $e^2 = \gamma^2$, or $e = \gamma$, and $g = \gamma \beta$ (again taking the positive root to avoid a reflective transformation in the limiting case).  This completely specifies the linear
transformation required to maintain the wave equation in wave equation form after a change of variables that includes
a velocity transformation in one direction:

\begin{equation}
\begin{bmatrix}
x \\
ct \\
\end{bmatrix}
=
\gamma
\begin{bmatrix}
1 & \beta \\
\beta & 1 \\
\end{bmatrix}
\begin{bmatrix}
x' \\
c t' \\
\end{bmatrix}
\end{equation}

Inversion of this yields the typical one dimensional Lorentz transformation where the position and time of a moving
frame is specified in terms of the inertial frame:

\begin{equation}
\begin{bmatrix}
x' \\
ct' \\
\end{bmatrix}
=
\gamma
\begin{bmatrix}
1 & -\beta \\
-\beta & 1 \\
\end{bmatrix}
\begin{bmatrix}
x \\
c t \\
\end{bmatrix}.
\end{equation}

That's perhaps more evident when this is written out explicitly in terms of velocity:

\begin{align}
x' &= \frac{x - v t}{\sqrt{1 - v^2/c^2}} \\
t' &= \frac{t - (v/c^2) x}{\sqrt{1 - v^2/c^2}}.
\end{align}

\section{ Light sphere, and relativistic metric. }

TBD.

My old E\&M book did this derivation using a requirement for invariance of a spherical light shell.  ie:

$x^2 - c^2 t^2 = x'^2 - c^2 t'^2$.

That approach requires less math (ie: to partial derivives or change of variables), but I never understood why that invariance condition could be assumed (to understand that intuitively, you have to understand the constancy of light phenomena, which has a few subtlies that aren't obvious in my opinion).

I like my approach, which has a bit more math, but I think is easier (vs. light constancy) to conceptualize the idea that light is wavelike regardless of the motion of the observer since it (ie: an electrodynamic field) must satisfy the wave equation (ie: Maxwell's equations).  I'm curious if somebody else also new to the subject of relativity would agree?

\section{ Derive relativistic doppler shift.}

TBD.

This is something I think would make sense to do considering solutions
to the wave
equation instead of utilizing more abstract wave number, and frequency
four vector concepts.  Haven't yet done the calculations for this part.

\end{document}               % End of document.
