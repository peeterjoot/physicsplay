\documentclass{article}

\usepackage{amsmath}
\usepackage{mathpazo}

%
% shorthand for bold symbols, convenient for vectors and matrices
%
\newcommand{\Ba}[0]{\mathbf{a}}
\newcommand{\Bb}[0]{\mathbf{b}}
\newcommand{\Bc}[0]{\mathbf{c}}
\newcommand{\Bd}[0]{\mathbf{d}}
\newcommand{\Be}[0]{\mathbf{e}}
\newcommand{\Bf}[0]{\mathbf{f}}
\newcommand{\Bg}[0]{\mathbf{g}}
\newcommand{\Bh}[0]{\mathbf{h}}
\newcommand{\Bi}[0]{\mathbf{i}}
\newcommand{\Bj}[0]{\mathbf{j}}
\newcommand{\Bk}[0]{\mathbf{k}}
\newcommand{\Bl}[0]{\mathbf{l}}
\newcommand{\Bm}[0]{\mathbf{m}}
\newcommand{\Bn}[0]{\mathbf{n}}
\newcommand{\Bo}[0]{\mathbf{o}}
\newcommand{\Bp}[0]{\mathbf{p}}
\newcommand{\Bq}[0]{\mathbf{q}}
\newcommand{\Br}[0]{\mathbf{r}}
\newcommand{\Bs}[0]{\mathbf{s}}
\newcommand{\Bt}[0]{\mathbf{t}}
\newcommand{\Bu}[0]{\mathbf{u}}
\newcommand{\Bv}[0]{\mathbf{v}}
\newcommand{\Bw}[0]{\mathbf{w}}
\newcommand{\Bx}[0]{\mathbf{x}}
\newcommand{\By}[0]{\mathbf{y}}
\newcommand{\Bz}[0]{\mathbf{z}}
\newcommand{\BA}[0]{\mathbf{A}}
\newcommand{\BB}[0]{\mathbf{B}}
\newcommand{\BC}[0]{\mathbf{C}}
\newcommand{\BD}[0]{\mathbf{D}}
\newcommand{\BE}[0]{\mathbf{E}}
\newcommand{\BF}[0]{\mathbf{F}}
\newcommand{\BG}[0]{\mathbf{G}}
\newcommand{\BH}[0]{\mathbf{H}}
\newcommand{\BI}[0]{\mathbf{I}}
\newcommand{\BJ}[0]{\mathbf{J}}
\newcommand{\BK}[0]{\mathbf{K}}
\newcommand{\BL}[0]{\mathbf{L}}
\newcommand{\BM}[0]{\mathbf{M}}
\newcommand{\BN}[0]{\mathbf{N}}
\newcommand{\BO}[0]{\mathbf{O}}
\newcommand{\BP}[0]{\mathbf{P}}
\newcommand{\BQ}[0]{\mathbf{Q}}
\newcommand{\BR}[0]{\mathbf{R}}
\newcommand{\BS}[0]{\mathbf{S}}
\newcommand{\BT}[0]{\mathbf{T}}
\newcommand{\BU}[0]{\mathbf{U}}
\newcommand{\BV}[0]{\mathbf{V}}
\newcommand{\BW}[0]{\mathbf{W}}
\newcommand{\BX}[0]{\mathbf{X}}
\newcommand{\BY}[0]{\mathbf{Y}}
\newcommand{\BZ}[0]{\mathbf{Z}}

\newcommand{\Bzero}[0]{\mathbf{0}}
\newcommand{\Btheta}[0]{\boldsymbol{\theta}}
\newcommand{\Btau}[0]{\boldsymbol{\tau}}
\newcommand{\Bomega}[0]{\boldsymbol{\omega}}

%
% shorthand for unit vectors
%
\newcommand{\acap}[0]{\hat{\Ba}}
\newcommand{\bcap}[0]{\hat{\Bb}}
\newcommand{\ccap}[0]{\hat{\Bc}}
\newcommand{\dcap}[0]{\hat{\Bd}}
\newcommand{\ecap}[0]{\hat{\Be}}
\newcommand{\fcap}[0]{\hat{\Bf}}
\newcommand{\gcap}[0]{\hat{\Bg}}
\newcommand{\hcap}[0]{\hat{\Bh}}
\newcommand{\icap}[0]{\hat{\Bi}}
\newcommand{\jcap}[0]{\hat{\Bj}}
\newcommand{\kcap}[0]{\hat{\Bk}}
\newcommand{\lcap}[0]{\hat{\Bl}}
\newcommand{\mcap}[0]{\hat{\Bm}}
\newcommand{\ncap}[0]{\hat{\Bn}}
\newcommand{\ocap}[0]{\hat{\Bo}}
\newcommand{\pcap}[0]{\hat{\Bp}}
\newcommand{\qcap}[0]{\hat{\Bq}}
\newcommand{\rcap}[0]{\hat{\Br}}
\newcommand{\scap}[0]{\hat{\Bs}}
\newcommand{\tcap}[0]{\hat{\Bt}}
\newcommand{\ucap}[0]{\hat{\Bu}}
\newcommand{\vcap}[0]{\hat{\Bv}}
\newcommand{\wcap}[0]{\hat{\Bw}}
\newcommand{\xcap}[0]{\hat{\Bx}}
\newcommand{\ycap}[0]{\hat{\By}}
\newcommand{\zcap}[0]{\hat{\Bz}}
\newcommand{\thetacap}[0]{\hat{\Btheta}}

%
% to write R^n and C^n in a distinguishable fashion.  Perhaps change this
% to the double lined characters upon figuring out how to do so.
%
\newcommand{\C}[1]{$\mathbb{C}^{#1}$}
\newcommand{\R}[1]{$\mathbb{R}^{#1}$}

%
% various generally useful helpers
%

% derivative of #1 wrt. #2:
\newcommand{\D}[2] {\frac {d#2} {d#1}}

\newcommand{\inv}[1]{\frac{1}{#1}}
\newcommand{\cross}[0]{\times}

\newcommand{\abs}[1]{\lvert{#1}\rvert}
\newcommand{\norm}[1]{\lVert{#1}\rVert}
\newcommand{\innerprod}[2]{\langle{#1}, {#2}\rangle}
\newcommand{\dotprod}[2]{{#1} \cdot {#2}}
\newcommand{\bdotprod}[2]{\left({#1} \cdot {#2}\right)}
\newcommand{\crossprod}[2]{{#1} \cross {#2}}
\newcommand{\tripleprod}[3]{\dotprod{\left(\crossprod{#1}{#2}\right)}{#3}}

\DeclareMathOperator{\Proj}{Proj}
\DeclareMathOperator{\Span}{span}
\DeclareMathOperator{\Sgn}{sgn}
\DeclareMathOperator{\Area}{Area}
\DeclareMathOperator{\Volume}{Volume}

%
% A few miscellaneous things specific to this document
%
\newcommand{\crossop}[1]{\crossprod{#1}{}}

% R2 vector.
\newcommand{\VectorTwo}[2]{
\begin{bmatrix}
 {#1} \\
 {#2}
\end{bmatrix}
}

\newcommand{\VectorN}[1]{
\begin{bmatrix}
{#1}_1 \\
{#1}_2 \\
\vdots \\
{#1}_N \\
\end{bmatrix}
}

\newcommand{\DETuvij}[4]{
\begin{vmatrix}
 {#1}_{#3} & {#1}_{#4} \\
 {#2}_{#3} & {#2}_{#4}
\end{vmatrix}
}

\newcommand{\DETuvwijk}[6]{
\begin{vmatrix}
 {#1}_{#4} & {#1}_{#5} & {#1}_{#6} \\
 {#2}_{#4} & {#2}_{#5} & {#2}_{#6} \\
 {#3}_{#4} & {#3}_{#5} & {#3}_{#6}
\end{vmatrix}
}

\newcommand{\DETuvwxijkl}[8]{
\begin{vmatrix}
 {#1}_{#5} & {#1}_{#6} & {#1}_{#7} & {#1}_{#8} \\
 {#2}_{#5} & {#2}_{#6} & {#2}_{#7} & {#2}_{#8} \\
 {#3}_{#5} & {#3}_{#6} & {#3}_{#7} & {#3}_{#8} \\
 {#4}_{#5} & {#4}_{#6} & {#4}_{#7} & {#4}_{#8} \\
\end{vmatrix}
}

%\newcommand{\DETuvwxyijklm}[10]{
%\begin{vmatrix}
% {#1}_{#6} & {#1}_{#7} & {#1}_{#8} & {#1}_{#9} & {#1}_{#10} \\
% {#2}_{#6} & {#2}_{#7} & {#2}_{#8} & {#2}_{#9} & {#2}_{#10} \\
% {#3}_{#6} & {#3}_{#7} & {#3}_{#8} & {#3}_{#9} & {#3}_{#10} \\
% {#4}_{#6} & {#4}_{#7} & {#4}_{#8} & {#4}_{#9} & {#4}_{#10} \\
% {#5}_{#6} & {#5}_{#7} & {#5}_{#8} & {#5}_{#9} & {#5}_{#10}
%\end{vmatrix}
%}

% R3 vector.
\newcommand{\VectorThree}[3]{
\begin{bmatrix}
 {#1} \\
 {#2} \\
 {#3}
\end{bmatrix}
}


\newcommand{\gpgrade}[2] {{\left\langle{{#1}}\right\rangle}_{#2}}
\newcommand{\gpgradezero}[1] {\gpgrade{#1}{0}}
\newcommand{\gpgradetwo}[1] {\gpgrade{#1}{2}}
\newcommand{\gpgradefour}[1] {\gpgrade{#1}{4}}

\title{ Metric signature dependencies for electromagnetic equations. }
\author{Peeter Joot}
\date{ Last Revision: $Date: 2008/09/05 01:37:54 $ }

\begin{document}

\maketitle{}

\section{ Motivation. }

Doran/Lasenby use a $+,-,-,-$ signature, and I had gotten used to that.  On first seeing the alternate signature used by John Denker's excellent 
explainatory paper:

http://www.av8n.com/physics/maxwell-ga.pdf

I found myself disoriented.  How many of the identities that I was used to were metric dependent?   Here are some notes that explore the 
metric dependencies of STA, in particular observing which identities are metric dependent and which aren't.

\section{}

\subsection{ spatial basis }

Our spatial (bivector) basis:

\begin{equation*}
\sigma_i = \gamma_i \wedge \gamma_0 = \gamma_{i0},
\end{equation*}

that behaves like Euclidean vectors (positive square) still behave as desired, regardless of the signature:

\begin{align*}
\sigma_i \cdot \sigma_j 
&= \gpgradezero{\gamma_{i0j0}}  \\
&= - \gpgradezero{\gamma_{ij}} \gamma_{00}  \\
&= -\delta_{ij} {\gamma_i}^2 \gamma_{00}
\end{align*}

Regardless of the signature the pair of products ${\gamma_i}^2 \gamma_{00} = -1$, so our spatial bivectors are metric invariant.

\subsection{ How about commutation? }

Commutation with 
\begin{equation*}
i \gamma_{\mu} = \gamma_{0123\mu} = \gamma_{\mu0123}
\end{equation*}

$\mu$ has to "pass" three indexes regardless of metric, so anticommutes for any $\mu$.

\begin{equation*}
\sigma_k \gamma_{\mu} = \gamma_{k0\mu}
\end{equation*}

If $k = \mu$, or $0 = \mu$, then we get a sign inversion, and otherwise commute (pass two indexes).  This is also metric invariant.

\subsection{ Electrodynamic tensor. }

John Denker's paper writes:

\begin{equation*}
F = (\BE + ic\BB) \gamma_0
\end{equation*}

with
\begin{align*}
\BE &= E^i \gamma_i \\
\BB &= B^i \gamma_i
\end{align*}

Since he uses the postive end of the metric for spatial indexes this works fine.  Contrast to Doran/Lasenby who write:

\begin{equation*}
F = \BE + ic\BB
\end{equation*}

with the following implied spatial bivector representation:
\begin{align*}
\BE &= E^i \sigma_i = E^i \gamma_{i0} \\
\BB &= B^i \sigma_i = B^i \gamma_{i0}
\end{align*}

(that implied representation wasn't obvious to me, but I eventually figured out what they meant).

The end result in both cases is a pure bivector representation for the complete field:

\begin{equation*}
F = E^j \gamma_{j0} + icB^j \gamma_{j0}
\end{equation*}

Let's look at the $B^j$ basis bivectors a bit more closely:

\begin{equation*}
i\gamma_{j0} 
= \gamma_{0123j0}
= -\gamma_{01230j}
= +\gamma_{00123j}
= (\gamma_0)^2 \gamma_{123j}
\end{equation*}

Where, 
\begin{equation*}
\gamma_{123j} = 
(\gamma_{j})^2
\left\{ 
\begin{array}{l l}
& \gamma_{31} \quad \mbox{if $j = 1$} \\
& \gamma_{23} \quad \mbox{if $j = 2$} \\
& \gamma_{12} \quad \mbox{if $j = 3$} \\
\end{array} \right.
\end{equation*}

Combining these results we have a $(\gamma_0)^2 (\gamma_{j})^2 = -1$ coefficient that is metric invariant, and can write:

\begin{equation*}
i \gamma_{j0} = 
\left\{ 
\begin{array}{l l}
& \gamma_{13} \quad \mbox{if $j = 1$} \\
& \gamma_{32} \quad \mbox{if $j = 2$} \\
& \gamma_{21} \quad \mbox{if $j = 3$} \\
\end{array} \right.
\end{equation*}

\end{document}
