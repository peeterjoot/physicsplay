\documentclass{article}      % Specifies the document class

\usepackage{amsmath}
\usepackage{mathpazo}

%
% shorthand for bold symbols, convenient for vectors and matrices
%
\newcommand{\Ba}[0]{\mathbf{a}}
\newcommand{\Bb}[0]{\mathbf{b}}
\newcommand{\Bc}[0]{\mathbf{c}}
\newcommand{\Bd}[0]{\mathbf{d}}
\newcommand{\Be}[0]{\mathbf{e}}
\newcommand{\Bf}[0]{\mathbf{f}}
\newcommand{\Bg}[0]{\mathbf{g}}
\newcommand{\Bh}[0]{\mathbf{h}}
\newcommand{\Bi}[0]{\mathbf{i}}
\newcommand{\Bj}[0]{\mathbf{j}}
\newcommand{\Bk}[0]{\mathbf{k}}
\newcommand{\Bl}[0]{\mathbf{l}}
\newcommand{\Bm}[0]{\mathbf{m}}
\newcommand{\Bn}[0]{\mathbf{n}}
\newcommand{\Bo}[0]{\mathbf{o}}
\newcommand{\Bp}[0]{\mathbf{p}}
\newcommand{\Bq}[0]{\mathbf{q}}
\newcommand{\Br}[0]{\mathbf{r}}
\newcommand{\Bs}[0]{\mathbf{s}}
\newcommand{\Bt}[0]{\mathbf{t}}
\newcommand{\Bu}[0]{\mathbf{u}}
\newcommand{\Bv}[0]{\mathbf{v}}
\newcommand{\Bw}[0]{\mathbf{w}}
\newcommand{\Bx}[0]{\mathbf{x}}
\newcommand{\By}[0]{\mathbf{y}}
\newcommand{\Bz}[0]{\mathbf{z}}
\newcommand{\BA}[0]{\mathbf{A}}
\newcommand{\BB}[0]{\mathbf{B}}
\newcommand{\BC}[0]{\mathbf{C}}
\newcommand{\BD}[0]{\mathbf{D}}
\newcommand{\BE}[0]{\mathbf{E}}
\newcommand{\BF}[0]{\mathbf{F}}
\newcommand{\BG}[0]{\mathbf{G}}
\newcommand{\BH}[0]{\mathbf{H}}
\newcommand{\BI}[0]{\mathbf{I}}
\newcommand{\BJ}[0]{\mathbf{J}}
\newcommand{\BK}[0]{\mathbf{K}}
\newcommand{\BL}[0]{\mathbf{L}}
\newcommand{\BM}[0]{\mathbf{M}}
\newcommand{\BN}[0]{\mathbf{N}}
\newcommand{\BO}[0]{\mathbf{O}}
\newcommand{\BP}[0]{\mathbf{P}}
\newcommand{\BQ}[0]{\mathbf{Q}}
\newcommand{\BR}[0]{\mathbf{R}}
\newcommand{\BS}[0]{\mathbf{S}}
\newcommand{\BT}[0]{\mathbf{T}}
\newcommand{\BU}[0]{\mathbf{U}}
\newcommand{\BV}[0]{\mathbf{V}}
\newcommand{\BW}[0]{\mathbf{W}}
\newcommand{\BX}[0]{\mathbf{X}}
\newcommand{\BY}[0]{\mathbf{Y}}
\newcommand{\BZ}[0]{\mathbf{Z}}

\newcommand{\Bzero}[0]{\mathbf{0}}
\newcommand{\Btheta}[0]{\boldsymbol{\theta}}
\newcommand{\Btau}[0]{\boldsymbol{\tau}}
\newcommand{\Bomega}[0]{\boldsymbol{\omega}}

%
% shorthand for unit vectors
%
\newcommand{\acap}[0]{\hat{\Ba}}
\newcommand{\bcap}[0]{\hat{\Bb}}
\newcommand{\ccap}[0]{\hat{\Bc}}
\newcommand{\dcap}[0]{\hat{\Bd}}
\newcommand{\ecap}[0]{\hat{\Be}}
\newcommand{\fcap}[0]{\hat{\Bf}}
\newcommand{\gcap}[0]{\hat{\Bg}}
\newcommand{\hcap}[0]{\hat{\Bh}}
\newcommand{\icap}[0]{\hat{\Bi}}
\newcommand{\jcap}[0]{\hat{\Bj}}
\newcommand{\kcap}[0]{\hat{\Bk}}
\newcommand{\lcap}[0]{\hat{\Bl}}
\newcommand{\mcap}[0]{\hat{\Bm}}
\newcommand{\ncap}[0]{\hat{\Bn}}
\newcommand{\ocap}[0]{\hat{\Bo}}
\newcommand{\pcap}[0]{\hat{\Bp}}
\newcommand{\qcap}[0]{\hat{\Bq}}
\newcommand{\rcap}[0]{\hat{\Br}}
\newcommand{\scap}[0]{\hat{\Bs}}
\newcommand{\tcap}[0]{\hat{\Bt}}
\newcommand{\ucap}[0]{\hat{\Bu}}
\newcommand{\vcap}[0]{\hat{\Bv}}
\newcommand{\wcap}[0]{\hat{\Bw}}
\newcommand{\xcap}[0]{\hat{\Bx}}
\newcommand{\ycap}[0]{\hat{\By}}
\newcommand{\zcap}[0]{\hat{\Bz}}
\newcommand{\thetacap}[0]{\hat{\Btheta}}

%
% to write R^n and C^n in a distinguishable fashion.  Perhaps change this
% to the double lined characters upon figuring out how to do so.
%
\newcommand{\C}[1]{$\mathbb{C}^{#1}$}
\newcommand{\R}[1]{$\mathbb{R}^{#1}$}

%
% various generally useful helpers
%

% derivative of #1 wrt. #2:
\newcommand{\D}[2] {\frac {d#2} {d#1}}

\newcommand{\inv}[1]{\frac{1}{#1}}
\newcommand{\cross}[0]{\times}

\newcommand{\abs}[1]{\lvert{#1}\rvert}
\newcommand{\norm}[1]{\lVert{#1}\rVert}
\newcommand{\innerprod}[2]{\langle{#1}, {#2}\rangle}
\newcommand{\dotprod}[2]{{#1} \cdot {#2}}
\newcommand{\bdotprod}[2]{\left({#1} \cdot {#2}\right)}
\newcommand{\crossprod}[2]{{#1} \cross {#2}}
\newcommand{\tripleprod}[3]{\dotprod{\left(\crossprod{#1}{#2}\right)}{#3}}

\DeclareMathOperator{\Proj}{Proj}
\DeclareMathOperator{\Span}{span}
\DeclareMathOperator{\Sgn}{sgn}
\DeclareMathOperator{\Area}{Area}
\DeclareMathOperator{\Volume}{Volume}

%
% A few miscellaneous things specific to this document
%
\newcommand{\crossop}[1]{\crossprod{#1}{}}

% R2 vector.
\newcommand{\VectorTwo}[2]{
\begin{bmatrix}
 {#1} \\
 {#2}
\end{bmatrix}
}

\newcommand{\VectorN}[1]{
\begin{bmatrix}
{#1}_1 \\
{#1}_2 \\
\vdots \\
{#1}_N \\
\end{bmatrix}
}

\newcommand{\DETuvij}[4]{
\begin{vmatrix}
 {#1}_{#3} & {#1}_{#4} \\
 {#2}_{#3} & {#2}_{#4}
\end{vmatrix}
}

\newcommand{\DETuvwijk}[6]{
\begin{vmatrix}
 {#1}_{#4} & {#1}_{#5} & {#1}_{#6} \\
 {#2}_{#4} & {#2}_{#5} & {#2}_{#6} \\
 {#3}_{#4} & {#3}_{#5} & {#3}_{#6}
\end{vmatrix}
}

\newcommand{\DETuvwxijkl}[8]{
\begin{vmatrix}
 {#1}_{#5} & {#1}_{#6} & {#1}_{#7} & {#1}_{#8} \\
 {#2}_{#5} & {#2}_{#6} & {#2}_{#7} & {#2}_{#8} \\
 {#3}_{#5} & {#3}_{#6} & {#3}_{#7} & {#3}_{#8} \\
 {#4}_{#5} & {#4}_{#6} & {#4}_{#7} & {#4}_{#8} \\
\end{vmatrix}
}

%\newcommand{\DETuvwxyijklm}[10]{
%\begin{vmatrix}
% {#1}_{#6} & {#1}_{#7} & {#1}_{#8} & {#1}_{#9} & {#1}_{#10} \\
% {#2}_{#6} & {#2}_{#7} & {#2}_{#8} & {#2}_{#9} & {#2}_{#10} \\
% {#3}_{#6} & {#3}_{#7} & {#3}_{#8} & {#3}_{#9} & {#3}_{#10} \\
% {#4}_{#6} & {#4}_{#7} & {#4}_{#8} & {#4}_{#9} & {#4}_{#10} \\
% {#5}_{#6} & {#5}_{#7} & {#5}_{#8} & {#5}_{#9} & {#5}_{#10}
%\end{vmatrix}
%}

% R3 vector.
\newcommand{\VectorThree}[3]{
\begin{bmatrix}
 {#1} \\
 {#2} \\
 {#3}
\end{bmatrix}
}



%
% The real thing:
%

                             % The preamble begins here.
\title{Four vector dot product invariance.} % Declares the document's title.
\author{Peeter Joot}         % Declares the author's name.
\date{August 1, 2008}        % Deleting this command produces today's date.

\begin{document}             % End of preamble and beginning of text.

\maketitle{}

\section{}

Ramamurti Shankar's lectures indicate that the four vector dot product
is a Lorentz invarient.  This makes some logical sense, but lets demonstrate it explicitly.

Start with a Lorentz transform matrix between coordinates for two four vectors (omitting the components perperendicular  to the motion) :

\begin{equation*}
{
\begin{bmatrix}
x^1 \\
x^0 \\
\end{bmatrix}
}'
=
\gamma
\begin{bmatrix}
1 & -\beta \\
-\beta & 1
\end{bmatrix}
\begin{bmatrix}
x^1 \\
x^0 \\
\end{bmatrix}
\end{equation*}

\begin{equation*}
{
\begin{bmatrix}
y^1 \\
y^0 \\
\end{bmatrix}
}'
=
\gamma
\begin{bmatrix}
1 & -\beta \\
-\beta & 1
\end{bmatrix}
\begin{bmatrix}
y^1 \\
y^0 \\
\end{bmatrix}
\end{equation*}

Now write out the dot product between the two vectors given the percieved length and time measurements for the same events in the moving frame:

\begin{align*}
X' \cdot Y' 
&= \gamma^2 \left( (-\beta x^1 + x^0)(-\beta y^1 + y^0) -(x^1 -\beta x^0) (y^1 -\beta y^0) \right) \\
&= \gamma^2 \left( (\beta^2 x^1 y^1 + x^0 y^0) + x^0 y^1( -\beta + \beta ) + x^1 y^0( -\beta + \beta ) -(x^1 y^1 + \beta^2 x^0 y^0) \right) \\
&= \gamma^2 \left( x^0 y^0 (1-\beta^2) - (1-\beta^2) x^1 y^1 \right) \\
&= x^0 y^0 - x^1 y^1 \\
&= X \cdot Y
\end{align*}

This completes the proof of dot product Lorentz invariance.  An automatic consequence of this is invariance
of the Minkowski length.

\subsection{ Invariance shown with hyperbolic trig functions. }

Dot product or length invariance can also be shown with the hyperbolic representation of the Lorentz transformation:

\begin{equation}\label{eqn:hyperbolicmatrix}
{
\begin{bmatrix}
x^1 \\
x^0 \\
\end{bmatrix}
}'
=
\begin{bmatrix}
\cosh(\alpha) & -\sinh(\alpha) \\
-\sinh(\alpha) & \cosh(\alpha)
\end{bmatrix}
\begin{bmatrix}
x^1 \\
x^0 \\
\end{bmatrix}
\end{equation}

Writing $S=\sinh(\alpha)$, and $C=\cosh(\alpha)$ for short, this gives:

\begin{align*}
X' \cdot Y' 
&= \left( (-S x^1 + C x^0)(-S y^1 + C y^0) -(C x^1 -S x^0) (C y^1 -S y^0) \right) \\
&= \left( (S^2  x^1 y^1 + C^2  x^0 y^0) + x^0 y^1( -SC + SC ) + x^1 y^0( -SC + SC ) -(C^2  x^1 y^1 + S^2  x^0 y^0) \right) \\
&= \left( x^0 y^0 (C^2  -S^2 ) - (C^2 -S^2 ) x^1 y^1 \right) \\
&= x^0 y^0 - x^1 y^1 \\
&= X \cdot Y
\end{align*}

This is not really any less work.

\section{ Geometric product formulation of Lorentz transform. }

We can show the above invariance almost trivially when we write the Lorentz boost in exponential form.  However we first have to show
how to do so.

Writing the spacetime bivector $\gamma_{10} = \gamma_1 \wedge \gamma_0$ for short, lets calculate the exponential of this spacetime bivector, as scaled with a rapidity 
angle $\alpha$ :

\begin{equation}\label{eqn:bivecexponential}
\exp(\gamma_{10}\alpha) = \sum \frac{(\gamma_{10}\alpha)^k}{k!}
\end{equation}

Now, the spacetime bivector has a unit square:

\begin{equation*}
{\gamma_{10}}^2 = \gamma_{1010} = -\gamma_{1001} = -\gamma_{11} = 1
\end{equation*}

so, we can split the sum of equation \ref{eqn:bivecexponential} into even and odd parts, and pull out the common bivector factor:

\begin{equation}\label{eqn:bivechyper}
\exp(\gamma_{10}\alpha) 
= \sum \frac{\alpha^{2k}}{(2k)!} + \gamma_{10}\sum \frac{\alpha^{2k+1}}{(2k+1)!}
= \cosh(\alpha) + \gamma_{10} \sinh(\alpha)
\end{equation}

\subsection{ Spatial rotation }

So, this quite a similar form as bivector exponential with a Euclidian metric.  For such a space the bivector had a negative square, just like the complex unit imaginary,
which allowed for the normal trigonometric split of the exponential: 

\begin{equation}
\exp(\Be_{12}\theta) 
= \sum (-1)^k\frac{\theta^{2k}}{(2k)!} + \Be_{12}\sum (-1)^k\frac{\theta^{2k+1}}{(2k+1)!}
= \cos(\theta) + \Be_{12} \sin(\theta)
\end{equation}

Now, with the Minkowski metric having a negative square for purely spatial components, how does a purely spacial bivector behave when squared?  Let's try it with

\begin{equation*}
{\gamma_{12}}^2 
= \gamma_{1212}
= -\gamma_{1221}
= \gamma_{11}
= -1
\end{equation*}

This also has a square that behaves like the unit imaginary, so we can do spacial rotations with rotors like we can with euclidian space.  However, we have to invert the sign of the angle when using a Minkowski metric.  Take a specific example of a 90 degree rotation in the x-y plane, expressed in complex form:

\begin{align*}
R_{\pi/2}(\gamma_1)
&= \gamma_1 \exp({ \gamma_{12} \pi/2 }) \\
&= \gamma_1 (0 + \gamma_{12}) \\
&= -\gamma_2 \\
\end{align*}

In general our Rotor equation with a Minkowski $(+,-,-,-)$ metric will be thus be:

\begin{equation}\label{eqn:spacerot}
R_{\theta}(x) = \exp( i\theta/2) x \exp( -i\theta/2)
\end{equation}

Here $i$ is a spatial bivector (a bivector with negative square), such as $\gamma_{1}\wedge\gamma_{2}$, and the rotation sense is with increasing angle from $\gamma_1$ towards $\gamma_2$.

\subsection{ Validity of the double sided spatial rotor formula. }

To demonstrate the validity of equation \ref{eqn:spacerot} one has to observe how the unit vectors $\gamma_{\mu}$ behave with respect to commutation, and how that behaviour results in either commutation or conjugate commutation with the exponential rotor.  Without any loss of generality one can restrict attention to a specific example, such as bivector $\gamma_{12}$.  By inspection, $\gamma_0$, and $\gamma_3$ both commute since an even number of exchanges in position is required for either:

\begin{align*}
\gamma_{0} \gamma_{12} 
&= \gamma_{0} \wedge \gamma_{1} \wedge \gamma_{2} \\
&= \gamma_{1} \wedge \gamma_{2} \wedge \gamma_{0} \\
&= \gamma_{12} \gamma_0
\end{align*}

For this reason, application of the double sided rotation does not change any such (perpendicular) vector that commutes with the rotor:

\begin{align*}
R_{\theta}(x_{\perp}) 
&= \exp( i\theta/2) x_{\perp} \exp( -i\theta/2) \\
&= x_{\perp} \exp( i\theta/2) \exp( -i\theta/2) \\
&= x_{\perp}
\end{align*}

Now for the basis vectors that lie in the plane of the spatial rotation we have anticommutation:

\begin{align*}
\gamma_{1} \gamma_{12} 
&= -\gamma_{1} \gamma_{21}  \\
&= -\gamma_{121} \\
&= -\gamma_{12} \gamma_{1}
\end{align*}

\begin{align*}
\gamma_{2} \gamma_{12} 
&= \gamma_{21}\gamma_{2} \\
&= -\gamma_{12}\gamma_{2}
\end{align*}

Given an understanding of how the unit vectors either commute or anticommute with the bivector for the plane of rotation, one can now see how these behave when multiplied by a rotor expressed exponentially:

\begin{equation*}
\gamma_{\mu}\exp(i\theta)
= \gamma_{\mu}\left( \cos(\theta) + i\sin(\theta) \right)
=
\left\{ 
\begin{array}{l l}
\left( \cos(\theta) + i\sin(\theta) \right) \gamma_{\mu} & \quad \mbox{if $\gamma_{\mu} \cdot i = 0$} \\
\left( \cos(\theta) - i\sin(\theta) \right) \gamma_{\mu} & \quad \mbox{if $\gamma_{\mu} \cdot i \ne 0$} \\
\end{array} \right.
\end{equation*}

The condition $\gamma_{\mu} \cdot i = 0$ corresponds to a spacelike vector perpendicular to the plane of rotation, or a timelike vector, or any combination of the two, whereas
$\gamma_{\mu} \cdot i \ne 0$ is true for any spacelike vector that lies completely in the plane of rotation.

Putting this information all together, we now complete the verification that the double sided rotor formula leaves the perpendicular spacelike or the timelike components untouched.  For for purely spacelike vectors in the plane of rotation we recover the single sided complex form rotation as illustrated by the following x-y plane rotation:

\begin{align*}
R_{\theta}(x_{\parallel}) 
&= \exp( \gamma_{12}\theta/2) x_{\parallel} \exp( -\gamma_{12}\theta/2) \\
&= x_{\parallel} \exp( -\gamma_{12}\theta/2) \exp( -\gamma_{12}\theta/2) \\
&= x_{\parallel} \exp( -\gamma_{12}\theta) \\
\end{align*}

\subsection{ Back to time space rotation. }

Now, like we can express a spatial rotation in exponential form, we can do the same for the hyperbolic ``rotation'' matrix of equation \ref{eqn:hyperbolicmatrix}.  Direct expansion
\footnote{
The paper ``Generalized relativistic velocity addition with spacetime algebra'', http://arxiv.org/pdf/physics/0511247.pdf derives the bivector form of this Lorentz boost directly in an interesting fashion.  Simple relativistic arguments are used that are quite similar to those of Einstein in his ``Relativity, the special and general theory'' appendix.  This paper is written in a form that requires you to work out many of the details yourself (likely for brievity).  However, once that extra work is done, I found the first half of that paper quite readable.
}
of the product is the easiest way to see that this is the case:

\begin{align*}
\left(\gamma_{1} x^1 + \gamma_{0} x^0 \right)\exp(\gamma_{10}\alpha)
&= 
\left(\gamma_{1} x^1 + \gamma_{0} x^0 \right) \left( \cosh(\alpha) +\gamma_{10}\sinh(\alpha) \right) \\
&= 
  \gamma_1\left( x^1 \cosh(\alpha) - x^0 \sinh(\alpha)\right) \\
& \quad + \gamma_0\left( x^0 \cosh(\alpha) - x^1 \sinh(\alpha)\right)
\end{align*}

As with the spatial rotation, full characterization of this expontial rotation operator, in both single and double sided form requires that one looks at how the various unit vectors commute with the unit bivector.  Without loss of generality one can restrict attention to a specific case, as done with the $\gamma_{10}$ above.



\end{document}               % End of document.
