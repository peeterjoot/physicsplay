\documentclass{article}      % Specifies the document class

\usepackage{amsmath}
\usepackage{mathpazo}

%
% shorthand for bold symbols, convenient for vectors and matrices
%
\newcommand{\Ba}[0]{\mathbf{a}}
\newcommand{\Bb}[0]{\mathbf{b}}
\newcommand{\Bc}[0]{\mathbf{c}}
\newcommand{\Bd}[0]{\mathbf{d}}
\newcommand{\Be}[0]{\mathbf{e}}
\newcommand{\Bf}[0]{\mathbf{f}}
\newcommand{\Bg}[0]{\mathbf{g}}
\newcommand{\Bh}[0]{\mathbf{h}}
\newcommand{\Bi}[0]{\mathbf{i}}
\newcommand{\Bj}[0]{\mathbf{j}}
\newcommand{\Bk}[0]{\mathbf{k}}
\newcommand{\Bl}[0]{\mathbf{l}}
\newcommand{\Bm}[0]{\mathbf{m}}
\newcommand{\Bn}[0]{\mathbf{n}}
\newcommand{\Bo}[0]{\mathbf{o}}
\newcommand{\Bp}[0]{\mathbf{p}}
\newcommand{\Bq}[0]{\mathbf{q}}
\newcommand{\Br}[0]{\mathbf{r}}
\newcommand{\Bs}[0]{\mathbf{s}}
\newcommand{\Bt}[0]{\mathbf{t}}
\newcommand{\Bu}[0]{\mathbf{u}}
\newcommand{\Bv}[0]{\mathbf{v}}
\newcommand{\Bw}[0]{\mathbf{w}}
\newcommand{\Bx}[0]{\mathbf{x}}
\newcommand{\By}[0]{\mathbf{y}}
\newcommand{\Bz}[0]{\mathbf{z}}
\newcommand{\BA}[0]{\mathbf{A}}
\newcommand{\BB}[0]{\mathbf{B}}
\newcommand{\BC}[0]{\mathbf{C}}
\newcommand{\BD}[0]{\mathbf{D}}
\newcommand{\BE}[0]{\mathbf{E}}
\newcommand{\BF}[0]{\mathbf{F}}
\newcommand{\BG}[0]{\mathbf{G}}
\newcommand{\BH}[0]{\mathbf{H}}
\newcommand{\BI}[0]{\mathbf{I}}
\newcommand{\BJ}[0]{\mathbf{J}}
\newcommand{\BK}[0]{\mathbf{K}}
\newcommand{\BL}[0]{\mathbf{L}}
\newcommand{\BM}[0]{\mathbf{M}}
\newcommand{\BN}[0]{\mathbf{N}}
\newcommand{\BO}[0]{\mathbf{O}}
\newcommand{\BP}[0]{\mathbf{P}}
\newcommand{\BQ}[0]{\mathbf{Q}}
\newcommand{\BR}[0]{\mathbf{R}}
\newcommand{\BS}[0]{\mathbf{S}}
\newcommand{\BT}[0]{\mathbf{T}}
\newcommand{\BU}[0]{\mathbf{U}}
\newcommand{\BV}[0]{\mathbf{V}}
\newcommand{\BW}[0]{\mathbf{W}}
\newcommand{\BX}[0]{\mathbf{X}}
\newcommand{\BY}[0]{\mathbf{Y}}
\newcommand{\BZ}[0]{\mathbf{Z}}

\newcommand{\Bzero}[0]{\mathbf{0}}
\newcommand{\Btheta}[0]{\boldsymbol{\theta}}
\newcommand{\Btau}[0]{\boldsymbol{\tau}}
\newcommand{\Bomega}[0]{\boldsymbol{\omega}}

%
% shorthand for unit vectors
%
\newcommand{\acap}[0]{\hat{\Ba}}
\newcommand{\bcap}[0]{\hat{\Bb}}
\newcommand{\ccap}[0]{\hat{\Bc}}
\newcommand{\dcap}[0]{\hat{\Bd}}
\newcommand{\ecap}[0]{\hat{\Be}}
\newcommand{\fcap}[0]{\hat{\Bf}}
\newcommand{\gcap}[0]{\hat{\Bg}}
\newcommand{\hcap}[0]{\hat{\Bh}}
\newcommand{\icap}[0]{\hat{\Bi}}
\newcommand{\jcap}[0]{\hat{\Bj}}
\newcommand{\kcap}[0]{\hat{\Bk}}
\newcommand{\lcap}[0]{\hat{\Bl}}
\newcommand{\mcap}[0]{\hat{\Bm}}
\newcommand{\ncap}[0]{\hat{\Bn}}
\newcommand{\ocap}[0]{\hat{\Bo}}
\newcommand{\pcap}[0]{\hat{\Bp}}
\newcommand{\qcap}[0]{\hat{\Bq}}
\newcommand{\rcap}[0]{\hat{\Br}}
\newcommand{\scap}[0]{\hat{\Bs}}
\newcommand{\tcap}[0]{\hat{\Bt}}
\newcommand{\ucap}[0]{\hat{\Bu}}
\newcommand{\vcap}[0]{\hat{\Bv}}
\newcommand{\wcap}[0]{\hat{\Bw}}
\newcommand{\xcap}[0]{\hat{\Bx}}
\newcommand{\ycap}[0]{\hat{\By}}
\newcommand{\zcap}[0]{\hat{\Bz}}
\newcommand{\thetacap}[0]{\hat{\Btheta}}

%
% to write R^n and C^n in a distinguishable fashion.  Perhaps change this
% to the double lined characters upon figuring out how to do so.
%
\newcommand{\C}[1]{$\mathbb{C}^{#1}$}
\newcommand{\R}[1]{$\mathbb{R}^{#1}$}

%
% various generally useful helpers
%

% derivative of #1 wrt. #2:
\newcommand{\D}[2] {\frac {d#2} {d#1}}

\newcommand{\inv}[1]{\frac{1}{#1}}
\newcommand{\cross}[0]{\times}

\newcommand{\abs}[1]{\lvert{#1}\rvert}
\newcommand{\norm}[1]{\lVert{#1}\rVert}
\newcommand{\innerprod}[2]{\langle{#1}, {#2}\rangle}
\newcommand{\dotprod}[2]{{#1} \cdot {#2}}
\newcommand{\bdotprod}[2]{\left({#1} \cdot {#2}\right)}
\newcommand{\crossprod}[2]{{#1} \cross {#2}}
\newcommand{\tripleprod}[3]{\dotprod{\left(\crossprod{#1}{#2}\right)}{#3}}

\DeclareMathOperator{\Proj}{Proj}
\DeclareMathOperator{\Span}{span}
\DeclareMathOperator{\Sgn}{sgn}
\DeclareMathOperator{\Area}{Area}
\DeclareMathOperator{\Volume}{Volume}

%
% A few miscellaneous things specific to this document
%
\newcommand{\crossop}[1]{\crossprod{#1}{}}

% R2 vector.
\newcommand{\VectorTwo}[2]{
\begin{bmatrix}
 {#1} \\
 {#2}
\end{bmatrix}
}

\newcommand{\VectorN}[1]{
\begin{bmatrix}
{#1}_1 \\
{#1}_2 \\
\vdots \\
{#1}_N \\
\end{bmatrix}
}

\newcommand{\DETuvij}[4]{
\begin{vmatrix}
 {#1}_{#3} & {#1}_{#4} \\
 {#2}_{#3} & {#2}_{#4}
\end{vmatrix}
}

\newcommand{\DETuvwijk}[6]{
\begin{vmatrix}
 {#1}_{#4} & {#1}_{#5} & {#1}_{#6} \\
 {#2}_{#4} & {#2}_{#5} & {#2}_{#6} \\
 {#3}_{#4} & {#3}_{#5} & {#3}_{#6}
\end{vmatrix}
}

\newcommand{\DETuvwxijkl}[8]{
\begin{vmatrix}
 {#1}_{#5} & {#1}_{#6} & {#1}_{#7} & {#1}_{#8} \\
 {#2}_{#5} & {#2}_{#6} & {#2}_{#7} & {#2}_{#8} \\
 {#3}_{#5} & {#3}_{#6} & {#3}_{#7} & {#3}_{#8} \\
 {#4}_{#5} & {#4}_{#6} & {#4}_{#7} & {#4}_{#8} \\
\end{vmatrix}
}

%\newcommand{\DETuvwxyijklm}[10]{
%\begin{vmatrix}
% {#1}_{#6} & {#1}_{#7} & {#1}_{#8} & {#1}_{#9} & {#1}_{#10} \\
% {#2}_{#6} & {#2}_{#7} & {#2}_{#8} & {#2}_{#9} & {#2}_{#10} \\
% {#3}_{#6} & {#3}_{#7} & {#3}_{#8} & {#3}_{#9} & {#3}_{#10} \\
% {#4}_{#6} & {#4}_{#7} & {#4}_{#8} & {#4}_{#9} & {#4}_{#10} \\
% {#5}_{#6} & {#5}_{#7} & {#5}_{#8} & {#5}_{#9} & {#5}_{#10}
%\end{vmatrix}
%}

% R3 vector.
\newcommand{\VectorThree}[3]{
\begin{bmatrix}
 {#1} \\
 {#2} \\
 {#3}
\end{bmatrix}
}



\newcommand{\laplacian}[0]{\nabla^2}
\newcommand{\Dsq}[2] {\frac {\partial^2 {#1}} {\partial {#2}^2}}
\newcommand{\dxj}[2] {\frac {\partial {#1}} {\partial {x_{#2}}}}
\newcommand{\dsqxj}[2] {\frac {\partial^2 {#1}} {\partial {x_{#2}}^2}}
\DeclareMathOperator{\Exp}{e}
\newcommand{\gpgrade}[2] {{\left\langle{{#1}}\right\rangle}_{#2}}

%
% The real thing:
%

                             % The preamble begins here.
\title{exponential solutions to laplace equation}
\author{Peeter Joot}         % Declares the author's name.
%\date{}        % Deleting this command produces today's date.

\begin{document}             % End of preamble and beginning of text.

\maketitle{}

\section{ The problem. }

Want solutions of

\[
\laplacian f = \sum_k \dsqxj{f}{k} = 0
\]

For real f.

\subsection{ One dimension. }

Here the problem is easy, we just integrate twice:

\[
f = cx + d
\]

\subsection{ Two dimensions. }

Solve:

\[
\dsqxj{f}{1} + \dsqxj{f}{2} = 0
\]

Can find solutions of the form $f = X(x_1)Y(x_2)$.  Differentiating we have:

\[
X''Y + XY'' = 0
\]

So, for $X \ne 0$, and $Y \ne 0$:
\[
\frac{X''}{X} = -\frac{Y''}{Y} = k^2
\]

\[
\implies
X = \Exp^{kx}
\]
\[
Y = \Exp^{k\Bi y}
\]

\[
\implies
f = XY = \Exp^{k(x + \Bi y)}
\]

Here $\Bi$ is anything that squares to -1.  Traditionally this is the
complex unit imaginary, but we are also free to use a geometric product unit bivector such as $\Bi = \Be_1 \wedge \Be_2 = \Be_1\Be_2 = \Be_{12}$, or $\Bi = \Be_{21}$.

With $\Bi = \Be_{12}$ for example we have:

\begin{align*}
f = XY = \Exp^{k(x + \Bi y)} 
&= \Exp^{k(x + \Be_{12} y)} \\
&= \Exp^{k(x\Be_{1}\Be_1 + \Be_{12} y)} \\
&= \Exp^{k\Be_1(x\Be_1 + \Be_2 y)} \\
\end{align*}

Writing $\Br = \sum x_i \Be_i$, all of the following are solutions
of the laplacian

\begin{align*}
\Exp^{k\Be_1\Br} \\
\Exp^{\Br k\Be_1} \\
\Exp^{k\Be_2\Br} \\
\Exp^{\Br k\Be_2} \\
\end{align*}

Now there isn't anything special about the use of the x and y axis so it is reasonable to expect that, given any constant vector $\Bv$, 
the the following are also solutions

\begin{align*}
\Exp^{\Br\Bv} &= \Exp^{\Br \cdot \Bv + \Br \wedge \Bv} \\
\Exp^{\Bv\Br} &= \Exp^{\Br \cdot \Bv - \Br \wedge \Bv} \\
\end{align*}

Provided $\Br$, and $\Bv$ aren't colinear, the wedge product component of the above can be written in terms of a unit bivector
$\Bi = \frac{\Br \wedge \Bv}{\abs{\Br \wedge \Bv}}$:

\begin{align*}
\Exp^{\Br\Bv} &= \Exp^{\Br \cdot \Bv + \Br \wedge \Bv} \\
&= \Exp^{\Br \cdot \Bv + \left( \frac{\Br \wedge \Bv}{\abs{\Br \wedge \Bv}} \right) {\abs{\Br \wedge \Bv}}} \\
&= \Exp^{\Br \cdot \Bv} \left( \cos{\abs{\Br \wedge \Bv}} + \left(\frac{\Br \wedge \Bv}{\abs{\Br \wedge \Bv}}\right) \sin{\abs{\Br \wedge \Bv}} \right) \\
&= \Exp^{\Br \cdot \Bv} \left( \cos{\abs{\Br \wedge \Bv}} + \Bi \sin{\abs{\Br \wedge \Bv}} \right) \\
\end{align*}

And, for the reverse:
\begin{align*}
(\Exp^{\Br\Bv})^\dagger = \Exp^{\Bv\Br}
&= \Exp^{\Br \cdot \Bv} \left( \cos{\abs{\Br \wedge \Bv}} - \Bi \sin{\abs{\Br \wedge \Bv}} \right) \\
\end{align*}

This exponential however has both scalar and bivector parts, and we are looking for real solutions, so we can use linear combinations of the
exponential and its reverse 
to write strictly real solutions for the $\Br \wedge \Bv \ne 0$ cases:

\begin{align*}
\inv{2}\left(\Exp^{\Br\Bv} + \Exp^{\Bv\Br}\right) = \Exp^{\Br\cdot\Bv}\cos{\abs{\Br \wedge \Bv}} \\
\inv{2\Bi}\left(\Exp^{\Br\Bv} - \Exp^{\Bv\Br}\right) = \Exp^{\Br\cdot\Bv}\sin{\abs{\Br \wedge \Bv}} \\
\end{align*}

Also note that further linear combinations (with positive and negative values for $\Bv$) can be taken, so we can write solutions to the laplacian in 
the symmetric, real numbered, coordinate free, form

\begin{align*}
\cosh(\Br\cdot\Bv)\cos{\abs{\Br \wedge \Bv}} \\
\sinh(\Br\cdot\Bv)\cos{\abs{\Br \wedge \Bv}} \\
\cosh(\Br\cdot\Bv)\sin{\abs{\Br \wedge \Bv}} \\
\sinh(\Br\cdot\Bv)\sin{\abs{\Br \wedge \Bv}} \\
\end{align*}

\subsection{ Verify the expontial solution. }

It is simple enough to verify that the vector product exponential solution above is valid in \R{2}, and it's tempting to assume that this
is also a solution for \R{N}.  That isn't the case, but to demonstrate this requires a few interesting geometric product manipulations (perhaps
obvious if tackled differently or to somebody with a better grasp of all the concepts).

Let's calculate one of the partials of the exponential solution

\begin{align*}
\dxj{}{j}\Exp^{\Br\Bv} 
&= \dxj{}{j}\Exp^{\Br \cdot \Bv + \Br \wedge \Bv} \\
&= \left(\dxj{}{j}\Exp^{\Br \cdot \Bv}\right) \Exp^{\Br \wedge \Bv} + \Exp^{\Br \cdot \Bv}\dxj{}{j}{\Exp^{\Br \wedge \Bv}} \\
&= \dxj{(\Br \cdot \Bv)}{j}\Exp^{\Br\Bv} + \Exp^{\Br \cdot \Bv}\dxj{}{j}{\Exp^{\Br \wedge \Bv}} \\
&= (\Be_j \cdot \Bv)\Exp^{\Br\Bv} + \Exp^{\Br \cdot \Bv}\dxj{}{j}{\Exp^{\Br \wedge \Bv}} \\
&= v_j\Exp^{\Br\Bv} + \Exp^{\Br \cdot \Bv}\dxj{}{j}{\Exp^{\Br \wedge \Bv}} \\
\end{align*}

Now, how do we differentiate this remaining bivector exponential?  Since bivectors do not generally commute, loose application of the 
chain rule $\frac{df(g)}{dx} = \frac{df}{dg}\frac{dg}{dx} = \frac{dg}{dx}\frac{df}{dg}$ may not be appropriate (what order would one use).  Instead
write this expontial in terms of scalar and bivector parts and differentiate that

\begin{align*}
\dxj{}{j}{\Exp^{\Br \wedge \Bv}} 
&= \dxj{}{j} \left( \cos{\abs{\Br \wedge \Bv}} + \Bi \sin{\abs{\Br \wedge \Bv}} \right) \\
&= \left(-\sin{\abs{\Br \wedge \Bv}} + \Bi \cos{\abs{\Br \wedge \Bv}} \right)\dxj{\abs{\Br \wedge \Bv}}{j} + \dxj{\Bi}{j} \sin{\abs{\Br \wedge \Bv}} \\
&= \Bi\left(\Bi\sin{\abs{\Br \wedge \Bv}} + \cos{\abs{\Br \wedge \Bv}} \right)\dxj{\abs{\Br \wedge \Bv}}{j} + \dxj{\Bi}{j} \sin{\abs{\Br \wedge \Bv}} \\
&= \Bi\Exp^{\Br \wedge \Bv} \dxj{\abs{\Br \wedge \Bv}}{j} + \dxj{\Bi}{j} \sin{\abs{\Br \wedge \Bv}} \\
&= \Exp^{\Br \wedge \Bv} \dxj{\abs{\Br \wedge \Bv}}{j} + \dxj{\Bi}{j} \sin{\abs{\Br \wedge \Bv}} \\
\end{align*}

%\subsubsection{ partial of the unit bivector term }
For \R{2}, it is fairly simple to confirm that 
\[
\dxj{\Bi}{j} = 0 
\]

This makes some intuitive sense (would the unit bivector for the plane be changed by varing one of the length of one of the components).  To prove this
in general takes a bit more work.

\begin{align*}
\dxj{\Bi}{j}
&= \dxj{}{j} \frac{\Br \wedge \Bv}{\abs{\Br \wedge \Bv}} \\
&= \dxj{(\Br \wedge \Bv)}{j} \inv{\abs{\Br \wedge \Bv}} + {(\Br \wedge \Bv)} \dxj{}{j} \inv{\abs{\Br \wedge \Bv}} \\
&= (\Be_j \wedge \Bv) \inv{\abs{\Br \wedge \Bv}} + (\Br \wedge \Bv) \dxj{}{j} {(\Br \wedge \Bv \Bv \wedge \Br)}^{-1/2} \\
&= (\Be_j \wedge \Bv) \inv{\abs{\Br \wedge \Bv}} -\inv{2 \abs{\Br \wedge \Bv}^3} {\Br \wedge \Bv} \dxj{}{j} ({\Br \wedge \Bv}{\Bv \wedge \Br}) \\
&= (\Be_j \wedge \Bv) \inv{\abs{\Br \wedge \Bv}} -\inv{2 \abs{\Br \wedge \Bv}^3} {\Br \wedge \Bv} (\Be_j \wedge \Bv\Bv \wedge \Br + \Br \wedge \Bv\Bv \wedge \Be_j) \\
&= (\Be_j \wedge \Bv) \inv{\abs{\Br \wedge \Bv}} +\inv{2 \abs{\Br \wedge \Bv}} \Bi ((\Be_j \wedge \Bv) \Bi + \Bi (\Be_j \wedge \Bv)) \\
&= \inv{\abs{\Br \wedge \Bv}} \left( \Be_j \wedge \Bv  +\inv{2}\left( \Bi(\Be_j \wedge \Bv)\Bi - \Be_j \wedge \Bv \right) \right) \\
&= \inv{2 \abs{\Br \wedge \Bv}} \left( \Be_j \wedge \Bv  + \Bi(\Be_j \wedge \Bv)\Bi \right) \\
&= \inv{2 \abs{\Br \wedge \Bv}} \left( \Be_j \wedge \Bv  - \Exp^{-\Bi \pi/2}(\Be_j \wedge \Bv)\Exp^{\Bi \pi/2} \right) \\
\end{align*}

This last term is a rotation by $\pi$ in the plane of $\Bi$.

If $\BB = \Be_j \wedge \Bv$ lies in the plane ($\BB = k\Bi$) we have:

\[
\BB + \Bi (k\Bi) \Bi = \BB - k\Bi = \BB - \BB = 0
\]

How about for other orientations of $\BB$?  In general this won't be zero.  As a linear operator, the rotation of $\BB$ will be the sum of the
rotations of the components that are in the plane of rotation and the components out of the plane, so taking the difference, all the components in the plane
will be zero.

Example, in \R{3}:

\[
(1,1,1)\wedge(-1,1,1) = 
\begin{vmatrix}
 1 & 1 \\
 -1 & 1 \\
\end{vmatrix}(\Be_{13} + \Be_{12}) = 2(\Be_{13} + \Be_{12})
\]

Rotation of the direction vectors for the plane by $\pi$ in the $\Be_{12}$ plane we have:

\[
-\Be_{12}(1,1,1)\Be_{12}= 
(\Be_{2}
-\Be_{1}
+\Be_{213})\Be_{12}
= (-\Be_{1} -\Be_{2} + \Be_{3})
\]

and

\[
-\Be_{12}(-1,1,1)\Be_{12}= 
(-\Be_{2}
-\Be_{1}
+\Be_{213})\Be_{12}
= (\Be_{1} -\Be_{2} + \Be_{3})
\]

So the bivector for the plane with the direction vectors rotated is:
\[
%(-1,-1,1)
%( 1,-1,1) = 
(-1,-1,1)\wedge(1,-1,1) = 
\begin{vmatrix}
 -1 & 1 \\
 -1 & 1 \\
\end{vmatrix}
\Be_{23}
+
\begin{vmatrix}
 -1 & 1 \\
 1 & 1 \\
\end{vmatrix}
\Be_{13}
+
\begin{vmatrix}
 -1 & -1 \\
 1 & -1 \\
\end{vmatrix}
\Be_{12}
=
-2\Be_{13} + 2\Be_{12}
\]

Taking the difference, the component in the plane of rotation vanishes as expected, and in this particular case we have:

\[
\BB + \Bi\BB\Bi = \BB - \Bi^\dagger\BB\Bi = 
2(\Be_{13} + \Be_{12}) -(-2\Be_{13} + 2\Be_{12}) = 4\Be_{13}
\]

In the general case to calculate the remainder, we have to know how to compute the projection of a bivector onto a plane.  The procedure for this is similar to a vector
projection onto a space, and we calculate

\begin{align*}
\BB\inv{\BA}\BA 
&= \left(\BB \cdot \inv{\BA} +{\langle{\BB \wedge \inv{\BA}}\rangle}_2 +\BB \wedge \inv{\BA}\right) \BA \\
&= 
\BB \cdot \inv{\BA} \BA \\
&+{\langle{ {\BB \inv{\BA}}\rangle}_2} \cdot \BA 
+{\langle{{\langle{ {\BB \inv{\BA}}\rangle}_2} \BA}\rangle}_2 
+{\langle{ {\BB \inv{\BA}}\rangle}_2} \wedge \BA  \\
&+(\BB \wedge \inv{\BA}) \cdot \BA 
+{\langle{\BB \wedge \inv{\BA} \BA}\rangle}_4 
+\BB \wedge \inv{\BA} \wedge \BA \\
\end{align*}

Since the LHS is a bivector, all the 0-grade, 4-grade and 6-grade terms must be zero (though the 6-grade term $\BB \wedge \inv{\BA} \wedge \BA$ is plainly zero anyhow).  That leaves:

\begin{equation}\label{eqn:bivectorprojbivector}
\BB
= 
\BB \cdot \inv{\BA} \BA \\
%+{\langle{{\langle{ {\BB \inv{\BA}}\rangle}_2} \BA}\rangle}_2 
+\gpgrade{\gpgrade{\BB\inv\BA}{2} \BA}{2}
+\left(\BB \wedge \inv{\BA}\right) \cdot \BA 
\end{equation}
\newcommand{\grade}[2] {{\left\langle{{#1}}\right\rangle}_{#2}}

%\begin{align*}
%\BB + \Bi\BB\Bi
%&= \left(\BB\cdot\inv{\Bi}\right)\Bi + \left(\BB\wedge\inv{\Bi}\right)\Bi - \Bi^\dagger\left( \left(\BB\cdot\inv{\Bi}\right)\Bi + \left(\BB\wedge\inv{\Bi}\right)\Bi \right) \Bi \\
%&= \left(\BB\cdot\inv{\Bi}\right)\Bi + \left(\BB\wedge\inv{\Bi}\right)\Bi - \Bi^\dagger \left( \left(\BB\cdot\inv{\Bi}\right)\Bi + \left(\BB\wedge\inv{\Bi}\right)\Bi\right) \Bi \\
%&= \left(\BB\wedge\inv{\Bi}\right)\Bi + \Bi \left(\BB\wedge\inv{\Bi}\right) \\
%&= -\left(\BB\wedge{\Bi}\right)\Bi - \Bi \left(\BB\wedge{\Bi}\right) \\
%\end{align*}
%
%Check:
%\begin{align*}
%\BB \wedge \Bi =
%2\left(\Be_{13} + \Be_{12}\right) \wedge \Be_{12}
%= 0
%\end{align*}
%
%Must be a mistake here somewhere, since these aren't consistent (pretty sure the numerical calculation is right).

\end{document}               % End of document.
