\documentclass{article}      % Specifies the document class

\usepackage{amsmath}
\usepackage{mathpazo}

%
% shorthand for bold symbols, convenient for vectors and matrices
%
\newcommand{\Ba}[0]{\mathbf{a}}
\newcommand{\Bb}[0]{\mathbf{b}}
\newcommand{\Bc}[0]{\mathbf{c}}
\newcommand{\Bd}[0]{\mathbf{d}}
\newcommand{\Be}[0]{\mathbf{e}}
\newcommand{\Bf}[0]{\mathbf{f}}
\newcommand{\Bg}[0]{\mathbf{g}}
\newcommand{\Bh}[0]{\mathbf{h}}
\newcommand{\Bi}[0]{\mathbf{i}}
\newcommand{\Bj}[0]{\mathbf{j}}
\newcommand{\Bk}[0]{\mathbf{k}}
\newcommand{\Bl}[0]{\mathbf{l}}
\newcommand{\Bm}[0]{\mathbf{m}}
\newcommand{\Bn}[0]{\mathbf{n}}
\newcommand{\Bo}[0]{\mathbf{o}}
\newcommand{\Bp}[0]{\mathbf{p}}
\newcommand{\Bq}[0]{\mathbf{q}}
\newcommand{\Br}[0]{\mathbf{r}}
\newcommand{\Bs}[0]{\mathbf{s}}
\newcommand{\Bt}[0]{\mathbf{t}}
\newcommand{\Bu}[0]{\mathbf{u}}
\newcommand{\Bv}[0]{\mathbf{v}}
\newcommand{\Bw}[0]{\mathbf{w}}
\newcommand{\Bx}[0]{\mathbf{x}}
\newcommand{\By}[0]{\mathbf{y}}
\newcommand{\Bz}[0]{\mathbf{z}}
\newcommand{\BA}[0]{\mathbf{A}}
\newcommand{\BB}[0]{\mathbf{B}}
\newcommand{\BC}[0]{\mathbf{C}}
\newcommand{\BD}[0]{\mathbf{D}}
\newcommand{\BE}[0]{\mathbf{E}}
\newcommand{\BF}[0]{\mathbf{F}}
\newcommand{\BG}[0]{\mathbf{G}}
\newcommand{\BH}[0]{\mathbf{H}}
\newcommand{\BI}[0]{\mathbf{I}}
\newcommand{\BJ}[0]{\mathbf{J}}
\newcommand{\BK}[0]{\mathbf{K}}
\newcommand{\BL}[0]{\mathbf{L}}
\newcommand{\BM}[0]{\mathbf{M}}
\newcommand{\BN}[0]{\mathbf{N}}
\newcommand{\BO}[0]{\mathbf{O}}
\newcommand{\BP}[0]{\mathbf{P}}
\newcommand{\BQ}[0]{\mathbf{Q}}
\newcommand{\BR}[0]{\mathbf{R}}
\newcommand{\BS}[0]{\mathbf{S}}
\newcommand{\BT}[0]{\mathbf{T}}
\newcommand{\BU}[0]{\mathbf{U}}
\newcommand{\BV}[0]{\mathbf{V}}
\newcommand{\BW}[0]{\mathbf{W}}
\newcommand{\BX}[0]{\mathbf{X}}
\newcommand{\BY}[0]{\mathbf{Y}}
\newcommand{\BZ}[0]{\mathbf{Z}}

\newcommand{\Bzero}[0]{\mathbf{0}}
\newcommand{\Btheta}[0]{\boldsymbol{\theta}}
\newcommand{\Btau}[0]{\boldsymbol{\tau}}
\newcommand{\Bomega}[0]{\boldsymbol{\omega}}

%
% shorthand for unit vectors
%
\newcommand{\acap}[0]{\hat{\Ba}}
\newcommand{\bcap}[0]{\hat{\Bb}}
\newcommand{\ccap}[0]{\hat{\Bc}}
\newcommand{\dcap}[0]{\hat{\Bd}}
\newcommand{\ecap}[0]{\hat{\Be}}
\newcommand{\fcap}[0]{\hat{\Bf}}
\newcommand{\gcap}[0]{\hat{\Bg}}
\newcommand{\hcap}[0]{\hat{\Bh}}
\newcommand{\icap}[0]{\hat{\Bi}}
\newcommand{\jcap}[0]{\hat{\Bj}}
\newcommand{\kcap}[0]{\hat{\Bk}}
\newcommand{\lcap}[0]{\hat{\Bl}}
\newcommand{\mcap}[0]{\hat{\Bm}}
\newcommand{\ncap}[0]{\hat{\Bn}}
\newcommand{\ocap}[0]{\hat{\Bo}}
\newcommand{\pcap}[0]{\hat{\Bp}}
\newcommand{\qcap}[0]{\hat{\Bq}}
\newcommand{\rcap}[0]{\hat{\Br}}
\newcommand{\scap}[0]{\hat{\Bs}}
\newcommand{\tcap}[0]{\hat{\Bt}}
\newcommand{\ucap}[0]{\hat{\Bu}}
\newcommand{\vcap}[0]{\hat{\Bv}}
\newcommand{\wcap}[0]{\hat{\Bw}}
\newcommand{\xcap}[0]{\hat{\Bx}}
\newcommand{\ycap}[0]{\hat{\By}}
\newcommand{\zcap}[0]{\hat{\Bz}}
\newcommand{\thetacap}[0]{\hat{\Btheta}}

%
% to write R^n and C^n in a distinguishable fashion.  Perhaps change this
% to the double lined characters upon figuring out how to do so.
%
\newcommand{\C}[1]{$\mathbb{C}^{#1}$}
\newcommand{\R}[1]{$\mathbb{R}^{#1}$}

%
% various generally useful helpers
%

% derivative of #1 wrt. #2:
\newcommand{\D}[2] {\frac {d#2} {d#1}}

\newcommand{\inv}[1]{\frac{1}{#1}}
\newcommand{\cross}[0]{\times}

\newcommand{\abs}[1]{\lvert{#1}\rvert}
\newcommand{\norm}[1]{\lVert{#1}\rVert}
\newcommand{\innerprod}[2]{\langle{#1}, {#2}\rangle}
\newcommand{\dotprod}[2]{{#1} \cdot {#2}}
\newcommand{\bdotprod}[2]{\left({#1} \cdot {#2}\right)}
\newcommand{\crossprod}[2]{{#1} \cross {#2}}
\newcommand{\tripleprod}[3]{\dotprod{\left(\crossprod{#1}{#2}\right)}{#3}}

\DeclareMathOperator{\Proj}{Proj}
\DeclareMathOperator{\Span}{span}
\DeclareMathOperator{\Sgn}{sgn}
\DeclareMathOperator{\Area}{Area}
\DeclareMathOperator{\Volume}{Volume}

%
% A few miscellaneous things specific to this document
%
\newcommand{\crossop}[1]{\crossprod{#1}{}}

% R2 vector.
\newcommand{\VectorTwo}[2]{
\begin{bmatrix}
 {#1} \\
 {#2}
\end{bmatrix}
}

\newcommand{\VectorN}[1]{
\begin{bmatrix}
{#1}_1 \\
{#1}_2 \\
\vdots \\
{#1}_N \\
\end{bmatrix}
}

\newcommand{\DETuvij}[4]{
\begin{vmatrix}
 {#1}_{#3} & {#1}_{#4} \\
 {#2}_{#3} & {#2}_{#4}
\end{vmatrix}
}

\newcommand{\DETuvwijk}[6]{
\begin{vmatrix}
 {#1}_{#4} & {#1}_{#5} & {#1}_{#6} \\
 {#2}_{#4} & {#2}_{#5} & {#2}_{#6} \\
 {#3}_{#4} & {#3}_{#5} & {#3}_{#6}
\end{vmatrix}
}

\newcommand{\DETuvwxijkl}[8]{
\begin{vmatrix}
 {#1}_{#5} & {#1}_{#6} & {#1}_{#7} & {#1}_{#8} \\
 {#2}_{#5} & {#2}_{#6} & {#2}_{#7} & {#2}_{#8} \\
 {#3}_{#5} & {#3}_{#6} & {#3}_{#7} & {#3}_{#8} \\
 {#4}_{#5} & {#4}_{#6} & {#4}_{#7} & {#4}_{#8} \\
\end{vmatrix}
}

%\newcommand{\DETuvwxyijklm}[10]{
%\begin{vmatrix}
% {#1}_{#6} & {#1}_{#7} & {#1}_{#8} & {#1}_{#9} & {#1}_{#10} \\
% {#2}_{#6} & {#2}_{#7} & {#2}_{#8} & {#2}_{#9} & {#2}_{#10} \\
% {#3}_{#6} & {#3}_{#7} & {#3}_{#8} & {#3}_{#9} & {#3}_{#10} \\
% {#4}_{#6} & {#4}_{#7} & {#4}_{#8} & {#4}_{#9} & {#4}_{#10} \\
% {#5}_{#6} & {#5}_{#7} & {#5}_{#8} & {#5}_{#9} & {#5}_{#10}
%\end{vmatrix}
%}

% R3 vector.
\newcommand{\VectorThree}[3]{
\begin{bmatrix}
 {#1} \\
 {#2} \\
 {#3}
\end{bmatrix}
}



\newcommand{\laplacian}[0]{\nabla^2}
\newcommand{\Dsq}[2] {\frac {\partial^2 {#1}} {\partial {#2}^2}}
\newcommand{\dxj}[2] {\frac {\partial {#1}} {\partial {x_{#2}}}}
\newcommand{\dsqxj}[2] {\frac {\partial^2 {#1}} {\partial {x_{#2}}^2}}
\DeclareMathOperator{\Exp}{e}
\DeclareMathOperator{\Rej}{Rej}
\newcommand{\gpgrade}[2] {{\left\langle{{#1}}\right\rangle}_{#2}}
\newcommand{\gpgradezero}[1] {\gpgrade{#1}{0}}
\newcommand{\gpgradetwo}[1] {\gpgrade{#1}{2}}
\newcommand{\gpgradefour}[1] {\gpgrade{#1}{4}}

%
% The real thing:
%

                             % The preamble begins here.
\title{Exponential Solutions to Laplace Equation in \R{N}}
\author{Peeter Joot}         % Declares the author's name.
%\date{}        % Deleting this command produces today's date.

\begin{document}             % End of preamble and beginning of text.

\maketitle{}

\section{ The problem. }

Want solutions of

\begin{equation}\label{eqn:laplacian}
\laplacian f = \sum_k \dsqxj{f}{k} = 0
\end{equation}

For real f.

\subsection{ One dimension. }

Here the problem is easy, just integrate twice:

\[
f = cx + d.
\]

\subsection{ Two dimensions. }

For the two dimensional case we want to solve:

\[
\dsqxj{f}{1} + \dsqxj{f}{2} = 0
\]

Using separation of variables one can find solutions of the form $f = X(x_1)Y(x_2)$.  Differentiating we have:

\[
X''Y + XY'' = 0
\]

So, for $X \ne 0$, and $Y \ne 0$:
\[
\frac{X''}{X} = -\frac{Y''}{Y} = k^2
\]

\[
\implies
X = \Exp^{kx}
\]
\[
Y = \Exp^{k\Bi y}
\]

\[
\implies
f = XY = \Exp^{k(x + \Bi y)}
\]

Here $\Bi$ is anything that squares to -1.  Traditionally this is the
complex unit imaginary, but we are also free to use a geometric product unit bivector such as $\Bi = \Be_1 \wedge \Be_2 = \Be_1\Be_2 = \Be_{12}$, or $\Bi = \Be_{21}$.

With $\Bi = \Be_{12}$ for example we have:

\begin{align*}
f = XY = \Exp^{k(x + \Bi y)}
&= \Exp^{k(x + \Be_{12} y)} \\
&= \Exp^{k(x\Be_{1}\Be_1 + \Be_{12} y)} \\
&= \Exp^{k\Be_1(x\Be_1 + \Be_2 y)} \\
\end{align*}

Writing $\Bx = \sum x_i \Be_i$, all of the following are solutions
of the laplacian

\begin{align*}
\Exp^{k\Be_1\Bx} \\
\Exp^{\Bx k\Be_1} \\
\Exp^{k\Be_2\Bx} \\
\Exp^{\Bx k\Be_2} \\
\end{align*}

Now there isn't anything special about the use of the x and y axis so it is reasonable to expect that, given any constant vector $\Bk$,
the the following may also be solutions to the two dimensional Laplacian problem

\begin{equation}\label{eqn:expgeo1}
\Exp^{\Bx\Bk} = \Exp^{\Bx \cdot \Bk + \Bx \wedge \Bk}
\end{equation}
\begin{equation}\label{eqn:expgeo2}
\Exp^{\Bk\Bx} = \Exp^{\Bx \cdot \Bk - \Bx \wedge \Bk}
\end{equation}

\subsection{ Assuming it's a solution, characterize in real numbers }

Verification that equations \ref{eqn:expgeo1} and \ref{eqn:expgeo2} are solutions of equation \ref{eqn:laplacian}
will be shown below a little bit further on, where we
look to see if (or under what conditions) this GA product exponential form is a solution for the \R{N} variation of Laplace's equation.

Assuming this exponential geometric product is a solution, let's first characterize this in terms of real numbers.

Provided $\Bx$, and $\Bk$ aren't colinear, the wedge product component of the above can be written in terms of a unit bivector
$\Bi = \frac{\Bx \wedge \Bk}{\abs{\Bx \wedge \Bk}}$:

\begin{align*}
\Exp^{\Bx\Bk} &= \Exp^{\Bx \cdot \Bk + \Bx \wedge \Bk} \\
&= \Exp^{\Bx \cdot \Bk + \left( \frac{\Bx \wedge \Bk}{\abs{\Bx \wedge \Bk}} \right) {\abs{\Bx \wedge \Bk}}} \\
&= \Exp^{\Bx \cdot \Bk} \left( \cos{\abs{\Bx \wedge \Bk}} + \left(\frac{\Bx \wedge \Bk}{\abs{\Bx \wedge \Bk}}\right) \sin{\abs{\Bx \wedge \Bk}} \right) \\
&= \Exp^{\Bx \cdot \Bk} \left( \cos{\abs{\Bx \wedge \Bk}} + \Bi \sin{\abs{\Bx \wedge \Bk}} \right) \\
&= \Exp^{\Bx \cdot \Bk} \left( \cos{\frac{\Bx \wedge \Bk}{\Bi}} + \Bi \sin{\frac{\Bx \wedge \Bk}{\Bi}} \right) \\
\end{align*}

And, for the reverse:
\begin{align*}
(\Exp^{\Bx\Bk})^\dagger = \Exp^{\Bk\Bx}
&= \Exp^{\Bx \cdot \Bk} \left( \cos{\abs{\Bx \wedge \Bk}} - \Bi \sin{\abs{\Bx \wedge \Bk}} \right) \\
&= \Exp^{\Bx \cdot \Bk} \left( \cos{\frac{\Bx \wedge \Bk}{\Bi}} - \Bi \sin{\frac{\Bx \wedge \Bk}{\Bi}} \right) \\
\end{align*}

This exponential however has both scalar and bivector parts, and we are looking for a strictly scalar result, so we can use linear combinations of the
exponential and its reverse to form a strictly real sum for the $\Bx \wedge \Bk \ne 0$ cases:

\begin{align*}
\inv{2}\left(\Exp^{\Bx\Bk} + \Exp^{\Bk\Bx}\right) = \Exp^{\Bx\cdot\Bk}\cos{\frac{\Bx \wedge \Bk}{\Bi}} \\
\inv{2\Bi}\left(\Exp^{\Bx\Bk} - \Exp^{\Bk\Bx}\right) = \Exp^{\Bx\cdot\Bk}\sin{\frac{\Bx \wedge \Bk}{\Bi}} \\
\end{align*}

Also note that further linear combinations (with positive and negative variations of $\Bk$) can be taken, so we can
combine equations \ref{eqn:expgeo1} and \ref{eqn:expgeo2} into the following real numbered, coordinate free, form:

\begin{align*}
\cosh(\Bx\cdot\Bk)\cos{\frac{\Bx \wedge \Bk}{\Bi}} \\
\sinh(\Bx\cdot\Bk)\cos{\frac{\Bx \wedge \Bk}{\Bi}} \\
\cosh(\Bx\cdot\Bk)\sin{\frac{\Bx \wedge \Bk}{\Bi}} \\
\sinh(\Bx\cdot\Bk)\sin{\frac{\Bx \wedge \Bk}{\Bi}} \\
\end{align*}

\subsection{ Verify the expontial solution. }

%It is simple enough to verify that the vector product exponential solution above is valid in \R{2}, and it's tempting to assume that this is also a solution for \R{N}, however that does not appear to be the case.  One has to restrict the vector $\Bx$ and the constant free parameter vector $\Bk$ to a plane.

To verify that equations \ref{eqn:expgeo1} and \ref{eqn:expgeo2} are Laplacian solutions, start with taking the first order partial with one
of the coordinates.
%into the following real numbered, coordinate free, form:
%Let's calculate one of the partials of the exponential solution

\begin{align*}
\dxj{}{j}\Exp^{\Bx\Bk}
&= \dxj{}{j}\Exp^{\Bx \cdot \Bk + \Bx \wedge \Bk} \\
&= \left(\dxj{}{j}\Exp^{\Bx \cdot \Bk}\right) \Exp^{\Bx \wedge \Bk} + \Exp^{\Bx \cdot \Bk}\dxj{}{j}{\Exp^{\Bx \wedge \Bk}} \\
&= \dxj{(\Bx \cdot \Bk)}{j}\Exp^{\Bx\Bk} + \Exp^{\Bx \cdot \Bk}\dxj{}{j}{\Exp^{\Bx \wedge \Bk}} \\
&= (\Be_j \cdot \Bk)\Exp^{\Bx\Bk} + \Exp^{\Bx \cdot \Bk}\dxj{}{j}{\Exp^{\Bx \wedge \Bk}} \\
&= k_j\Exp^{\Bx\Bk} + \Exp^{\Bx \cdot \Bk}\dxj{}{j}{\Exp^{\Bx \wedge \Bk}} \\
\end{align*}

%Now, how do we differentiate this remaining bivector exponential?  Since bivectors do not generally commute, loose application of the
%chain rule $\frac{df(g)}{dx} = \frac{df}{dg}\frac{dg}{dx} = \frac{dg}{dx}\frac{df}{dg}$ may not be appropriate (what order would one use).
To differentiate the bivector exponential write it first in terms of scalar and bivector parts:

\begin{align*}
\dxj{}{j}{\Exp^{\Bx \wedge \Bk}}
&= \dxj{}{j} \left( \cos{\abs{\Bx \wedge \Bk}} + \Bi \sin{\abs{\Bx \wedge \Bk} } \right) \\
&= \dxj{\abs{\Bx \wedge \Bk}}{j}
      \left(-\sin{\abs{\Bx \wedge \Bk}} + \Bi \cos{\abs{\Bx \wedge \Bk} } \right)
 + \dxj{\Bi}{j} \sin{\abs{\Bx \wedge \Bk}} \\
&=
\dxj{\abs{\Bx \wedge \Bk}}{j}
\Bi \left(\Bi\sin{\abs{\Bx \wedge \Bk}} + \cos{\abs{\Bx \wedge \Bk}} \right)
 + \dxj{\Bi}{j} \sin{\abs{\Bx \wedge \Bk}} \\
&=
\dxj{\abs{\Bx \wedge \Bk}}{j}
\Bi
\Exp^{\Bx \wedge \Bk}
 + \dxj{\Bi}{j} \sin{\abs{\Bx \wedge \Bk}} \\
&= 
\dxj{\abs{\Bx \wedge \Bk}}{j}\Bi \Exp^{\Bx \wedge \Bk} 
 + \dxj{\Bi}{j} \sin{\abs{\Bx \wedge \Bk}} \\
\end{align*}

Combining the results we have

\begin{equation}\label{eqn:firstpartial}
\dxj{}{j}\Exp^{\Bx\Bk}
= 
%k_j\Exp^{\Bx\Bk} + \Exp^{\Bx \cdot \Bk}
%(
%\dxj{\abs{\Bx \wedge \Bk}}{j}\Bi \Exp^{\Bx \wedge \Bk} 
% + \dxj{\Bi}{j} \sin{\abs{\Bx \wedge \Bk}} \\
%)
\left(k_j + \dxj{\abs{\Bx \wedge \Bk}}{j}\Bi\right) \Exp^{\Bx\Bk} + \dxj{\Bi}{j} \Exp^{\Bx \cdot \Bk} \sin{\abs{\Bx \wedge \Bk}}
\end{equation}

\subsection{ collecting some intermediate derivatives }

To evalate equation \ref{eqn:firstpartial}, we need a couple intermediate results:

\begin{align*}
\dxj{\abs{\Bx \wedge \Bk}}{j} \\
\dxj{\Bi}{j} = \dxj{}{j} \frac{\Bx \wedge \Bk}{\abs{\Bx \wedge \Bk}} \\
\end{align*}

To evaluate the first start with the squared magnitude
\begin{align*}
\dxj{\abs{\Bx \wedge \Bk}^2}{j} 
&= 2 {\abs{\Bx \wedge \Bk}} \dxj{\abs{\Bx \wedge \Bk}}{j} \\
&= -\dxj{(\Bx \wedge \Bk)^2}{j} \\
&= - \dxj{(\Bx \wedge \Bk)}{j} (\Bx \wedge \Bk) - (\Bx \wedge \Bk) \dxj{(\Bx \wedge \Bk)}{j} \\
&= - (\Be_j \wedge \Bk) (\Bx \wedge \Bk) - (\Bx \wedge \Bk) (\Be_j \wedge \Bk) \\
&= - 2 (\Be_j \wedge \Bk) \cdot (\Bx \wedge \Bk) \\
\end{align*}

Thus we have
\begin{equation}\label{eqn:partialabs}
\dxj{\abs{\Bx \wedge \Bk}}{j} = - (\Be_j \wedge \Bk) \cdot \Bi
\end{equation}

We want 
\[
\dxj{\abs{\Bx \wedge \Bk}}{j}\Bi
%= - (\Be_j \wedge \Bk) \cdot \Bi \Bi
= \inv{2} ( \Be_j \wedge \Bk - \Bi (\Be_j \wedge \Bk) \Bi )
\]

Next evaluate the derivative of the unit bivector
\begin{align*}
\dxj{\Bi}{j}
&= \dxj{ \frac{\Bx \wedge \Bk}{\abs{\Bx \wedge \Bk}} }{j} \\
&= \frac{\Be_j \wedge \Bk}{ \abs{\Bx \wedge \Bk} } + {\Bx \wedge \Bk} \dxj{}{j} \inv{\abs{\Bx \wedge \Bk}} \\
&= \frac{\Be_j \wedge \Bk}{\abs{\Bx \wedge \Bk}} 
 - \frac{\Bx \wedge \Bk}{ \abs{\Bx \wedge \Bk}^2 } \dxj{\abs{\Bx \wedge \Bk}}{j} \\
&= \inv{\abs{\Bx \wedge \Bk}} ( \Be_j \wedge \Bk - \Bi\dxj{\abs{\Bx \wedge \Bk}}{j}) \\
%&= \inv{\abs{\Bx \wedge \Bk}} ( \Be_j \wedge \Bk + \Bi (\Be_j \wedge \Bk) \cdot \Bi ) \\
&= \inv{2 \abs{\Bx \wedge \Bk}} ( \Be_j \wedge \Bk + \Bi (\Be_j \wedge \Bk) \Bi ) \\
&= \inv{\abs{\Bx \wedge \Bk}} \inv{2}( - (\Be_j \wedge \Bk) \Bi + \Bi (\Be_j \wedge \Bk) ) \Bi \\
\end{align*}
\begin{equation}\label{eqn:ipartial} \dxj{\Bi}{j}
= \frac{\gpgradetwo{\Bi (\Be_j \wedge \Bk)} \Bi}{\abs{\Bx \wedge \Bk}} 
= \frac{\Rej_{\Bi}(\Be_j \wedge \Bk)}{\abs{\Bx \wedge \Bk}}
\end{equation}

This grade two term must be zero for \R{2} since there is only one plane possible.

%\subsection{ partial of the unit bivector term }
%For \R{2}, one can confirm that
%\[
%\dxj{\Bi}{j} = 0
%\]
%This makes some intuitive sense (would the unit bivector for the plane be changed by varing one of the length of one of the components).
%That, again in \R{2}, is required to confirm that the exponential of the vector product is in fact a solution to Laplaces equation.
%However, to determine the conditions where this is generally true for \R{N} takes more work.
%
%... snip ...

\subsection{ First partial derivatives expanded. }

We are now equipt to expand equation \ref{eqn:firstpartial}:

\begin{equation}
\dxj{}{j}\Exp^{\Bx\Bk}
= 
\left(k_j - ((\Be_j \wedge \Bk) \cdot \Bi) \Bi \right) \Exp^{\Bx\Bk} + 
\frac{\gpgradetwo{\Bi (\Be_j \wedge \Bk)} \Bi}{\abs{\Bx \wedge \Bk}} 
\Exp^{\Bx \cdot \Bk} \sin{\abs{\Bx \wedge \Bk}}
\end{equation}



%\begin{align*}
%\dxj{}{j}\Exp^{\Bx\Bk}
%&= \left(k_j + \dxj{\abs{\Bx \wedge \Bk}}{j} \Bi \right) \Exp^{\Bx\Bk}
%  +\Exp^{\Bx \cdot \Bk} \sin{\abs{\Bx \wedge \Bk}}\dxj{\Bi}{j} \\
%\end{align*}

%With the second partial being
%
%\begin{align*}
%\dsqxj{}{j}\Exp^{\Bx\Bk}
%&=
%\left(
%      \left(k_j + \dxj{\abs{\Bx \wedge \Bk}}{j}\right)^2
%    + \dsqxj{\abs{\Bx \wedge \Bk}}{j}
%\right)
%   \Exp^{\Bx\Bk}
%+\dxj{}{j}\left(\Exp^{\Bx \cdot \Bk} \sin{\abs{\Bx \wedge \Bk}}\dxj{\Bi}{j}\right) \\
%\end{align*}


%\begin{align*}
%\dxj{\abs{\Bx \wedge \Bk}}{j}
%&=
%\dxj{ ({\Bx \wedge \Bk}{\Bk \wedge \Bx})^{1/2} }{j} \\
%&= \inv{ 2 {\abs{\Bx \wedge \Bk}} } \left( \left(\Be_j \wedge \Bk\right)\left(\Bk \wedge \Bx\right) + \left(\Bx \wedge \Bk\right)\left(\Bk \wedge \Be_j\right)\right) \\
%&= -\inv{ 2 } \left(\left(\Be_j \wedge \Bk\right)\Bi + \Bi\left(\Bk \wedge \Be_j\right)\right) \\
%\end{align*}


%In this last term we have the symmetric product of two bivectors and the LHS is a scalar, so all the grade-2 and grade-4 terms must cancel out.  This leaves just
%
%\begin{align*}
%\dxj{\abs{\Bx \wedge \Bk}}{j}
%&= -\inv{ 2 }\left( \left(\Be_j \wedge \Bk\right) \cdot \Bi + \Bi \cdot \left(\Be_j \wedge \Bk\right) \right) \\
%&= x_j \frac{\abs{\Be_j \wedge \Bk}^2}
%\end{align*}

%\[
%\dxj{}{j}\Exp^{\Bx\Bk}
%=
%\left(k_j + x_j \frac{\abs{\Be_j \wedge \Bk}^2}{\abs{\Bx \wedge \Bk}}\right)\Exp^{\Bx\Bk}
%+ \Exp^{\Bx \cdot \Bk} \sin{\abs{\Bx \wedge \Bk}} \gpgradetwo{ \gpgradetwo{ \frac{\Be_j \wedge \Bk}{\abs{\Bx \wedge \Bk}} \Bi^\dagger } \Bi }
%\]
%
%And the second partial is:
%
%\begin{align*}
%\dsqxj{}{j}\Exp^{\Bx\Bk}
%&=
%\left(
%\left(k_j + x_j \frac{\abs{\Be_j \wedge \Bk}^2}{\abs{\Bx \wedge \Bk}}\right)^2
%+ {\abs{\Be_j \wedge \Bk}^2}\dxj{\frac{x_j}{\abs{\Bx \wedge \Bk}}}{j}\right)
%      \Exp^{\Bx\Bk} \\
%&+
%\dsqxj{}{j}
%\Exp^{\Bx \cdot \Bk} \sin{\abs{\Bx \wedge \Bk}} \gpgradetwo{ \gpgradetwo{ \frac{\Be_j \wedge \Bk}{\abs{\Bx \wedge \Bk}} \Bi^\dagger } \Bi } \\
%&=
%\left(
%\left(k_j + x_j \frac{\abs{\Be_j \wedge \Bk}^2}{\abs{\Bx \wedge \Bk}}\right)^2
%+ \frac{\abs{\Be_j \wedge \Bk}^2}{\abs{\Bx \wedge \Bk}}
%+ -x_j^2 \frac{\abs{\Be_j \wedge \Bk}^4}{\abs{\Bx \wedge \Bk}^3}
%\right)
%      \Exp^{\Bx\Bk} \\
%&+
%\dsqxj{}{j}
%\Exp^{\Bx \cdot \Bk} \sin{\abs{\Bx \wedge \Bk}} \gpgradetwo{ \gpgradetwo{ \frac{\Be_j \wedge \Bk}{\abs{\Bx \wedge \Bk}} \Bi^\dagger } \Bi }
%\end{align*}

\end{document}               % End of document.
