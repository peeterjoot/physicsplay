%\documentclass[]{eliblog}

\usepackage{amsmath}
\usepackage{mathpazo}

%
% shorthand for bold symbols, convenient for vectors and matrices
%
\newcommand{\Ba}[0]{\mathbf{a}}
\newcommand{\Bb}[0]{\mathbf{b}}
\newcommand{\Bc}[0]{\mathbf{c}}
\newcommand{\Bd}[0]{\mathbf{d}}
\newcommand{\Be}[0]{\mathbf{e}}
\newcommand{\Bf}[0]{\mathbf{f}}
\newcommand{\Bg}[0]{\mathbf{g}}
\newcommand{\Bh}[0]{\mathbf{h}}
\newcommand{\Bi}[0]{\mathbf{i}}
\newcommand{\Bj}[0]{\mathbf{j}}
\newcommand{\Bk}[0]{\mathbf{k}}
\newcommand{\Bl}[0]{\mathbf{l}}
\newcommand{\Bm}[0]{\mathbf{m}}
\newcommand{\Bn}[0]{\mathbf{n}}
\newcommand{\Bo}[0]{\mathbf{o}}
\newcommand{\Bp}[0]{\mathbf{p}}
\newcommand{\Bq}[0]{\mathbf{q}}
\newcommand{\Br}[0]{\mathbf{r}}
\newcommand{\Bs}[0]{\mathbf{s}}
\newcommand{\Bt}[0]{\mathbf{t}}
\newcommand{\Bu}[0]{\mathbf{u}}
\newcommand{\Bv}[0]{\mathbf{v}}
\newcommand{\Bw}[0]{\mathbf{w}}
\newcommand{\Bx}[0]{\mathbf{x}}
\newcommand{\By}[0]{\mathbf{y}}
\newcommand{\Bz}[0]{\mathbf{z}}
\newcommand{\BA}[0]{\mathbf{A}}
\newcommand{\BB}[0]{\mathbf{B}}
\newcommand{\BC}[0]{\mathbf{C}}
\newcommand{\BD}[0]{\mathbf{D}}
\newcommand{\BE}[0]{\mathbf{E}}
\newcommand{\BF}[0]{\mathbf{F}}
\newcommand{\BG}[0]{\mathbf{G}}
\newcommand{\BH}[0]{\mathbf{H}}
\newcommand{\BI}[0]{\mathbf{I}}
\newcommand{\BJ}[0]{\mathbf{J}}
\newcommand{\BK}[0]{\mathbf{K}}
\newcommand{\BL}[0]{\mathbf{L}}
\newcommand{\BM}[0]{\mathbf{M}}
\newcommand{\BN}[0]{\mathbf{N}}
\newcommand{\BO}[0]{\mathbf{O}}
\newcommand{\BP}[0]{\mathbf{P}}
\newcommand{\BQ}[0]{\mathbf{Q}}
\newcommand{\BR}[0]{\mathbf{R}}
\newcommand{\BS}[0]{\mathbf{S}}
\newcommand{\BT}[0]{\mathbf{T}}
\newcommand{\BU}[0]{\mathbf{U}}
\newcommand{\BV}[0]{\mathbf{V}}
\newcommand{\BW}[0]{\mathbf{W}}
\newcommand{\BX}[0]{\mathbf{X}}
\newcommand{\BY}[0]{\mathbf{Y}}
\newcommand{\BZ}[0]{\mathbf{Z}}

\newcommand{\Bzero}[0]{\mathbf{0}}
\newcommand{\Btheta}[0]{\boldsymbol{\theta}}
\newcommand{\Btau}[0]{\boldsymbol{\tau}}
\newcommand{\Bomega}[0]{\boldsymbol{\omega}}

%
% shorthand for unit vectors
%
\newcommand{\acap}[0]{\hat{\Ba}}
\newcommand{\bcap}[0]{\hat{\Bb}}
\newcommand{\ccap}[0]{\hat{\Bc}}
\newcommand{\dcap}[0]{\hat{\Bd}}
\newcommand{\ecap}[0]{\hat{\Be}}
\newcommand{\fcap}[0]{\hat{\Bf}}
\newcommand{\gcap}[0]{\hat{\Bg}}
\newcommand{\hcap}[0]{\hat{\Bh}}
\newcommand{\icap}[0]{\hat{\Bi}}
\newcommand{\jcap}[0]{\hat{\Bj}}
\newcommand{\kcap}[0]{\hat{\Bk}}
\newcommand{\lcap}[0]{\hat{\Bl}}
\newcommand{\mcap}[0]{\hat{\Bm}}
\newcommand{\ncap}[0]{\hat{\Bn}}
\newcommand{\ocap}[0]{\hat{\Bo}}
\newcommand{\pcap}[0]{\hat{\Bp}}
\newcommand{\qcap}[0]{\hat{\Bq}}
\newcommand{\rcap}[0]{\hat{\Br}}
\newcommand{\scap}[0]{\hat{\Bs}}
\newcommand{\tcap}[0]{\hat{\Bt}}
\newcommand{\ucap}[0]{\hat{\Bu}}
\newcommand{\vcap}[0]{\hat{\Bv}}
\newcommand{\wcap}[0]{\hat{\Bw}}
\newcommand{\xcap}[0]{\hat{\Bx}}
\newcommand{\ycap}[0]{\hat{\By}}
\newcommand{\zcap}[0]{\hat{\Bz}}
\newcommand{\thetacap}[0]{\hat{\Btheta}}

%
% to write R^n and C^n in a distinguishable fashion.  Perhaps change this
% to the double lined characters upon figuring out how to do so.
%
\newcommand{\C}[1]{$\mathbb{C}^{#1}$}
\newcommand{\R}[1]{$\mathbb{R}^{#1}$}

%
% various generally useful helpers
%

% derivative of #1 wrt. #2:
\newcommand{\D}[2] {\frac {d#2} {d#1}}

\newcommand{\inv}[1]{\frac{1}{#1}}
\newcommand{\cross}[0]{\times}

\newcommand{\abs}[1]{\lvert{#1}\rvert}
\newcommand{\norm}[1]{\lVert{#1}\rVert}
\newcommand{\innerprod}[2]{\langle{#1}, {#2}\rangle}
\newcommand{\dotprod}[2]{{#1} \cdot {#2}}
\newcommand{\bdotprod}[2]{\left({#1} \cdot {#2}\right)}
\newcommand{\crossprod}[2]{{#1} \cross {#2}}
\newcommand{\tripleprod}[3]{\dotprod{\left(\crossprod{#1}{#2}\right)}{#3}}

\DeclareMathOperator{\Proj}{Proj}
\DeclareMathOperator{\Span}{span}
\DeclareMathOperator{\Sgn}{sgn}
\DeclareMathOperator{\Area}{Area}
\DeclareMathOperator{\Volume}{Volume}

%
% A few miscellaneous things specific to this document
%
\newcommand{\crossop}[1]{\crossprod{#1}{}}

% R2 vector.
\newcommand{\VectorTwo}[2]{
\begin{bmatrix}
 {#1} \\
 {#2}
\end{bmatrix}
}

\newcommand{\VectorN}[1]{
\begin{bmatrix}
{#1}_1 \\
{#1}_2 \\
\vdots \\
{#1}_N \\
\end{bmatrix}
}

\newcommand{\DETuvij}[4]{
\begin{vmatrix}
 {#1}_{#3} & {#1}_{#4} \\
 {#2}_{#3} & {#2}_{#4}
\end{vmatrix}
}

\newcommand{\DETuvwijk}[6]{
\begin{vmatrix}
 {#1}_{#4} & {#1}_{#5} & {#1}_{#6} \\
 {#2}_{#4} & {#2}_{#5} & {#2}_{#6} \\
 {#3}_{#4} & {#3}_{#5} & {#3}_{#6}
\end{vmatrix}
}

\newcommand{\DETuvwxijkl}[8]{
\begin{vmatrix}
 {#1}_{#5} & {#1}_{#6} & {#1}_{#7} & {#1}_{#8} \\
 {#2}_{#5} & {#2}_{#6} & {#2}_{#7} & {#2}_{#8} \\
 {#3}_{#5} & {#3}_{#6} & {#3}_{#7} & {#3}_{#8} \\
 {#4}_{#5} & {#4}_{#6} & {#4}_{#7} & {#4}_{#8} \\
\end{vmatrix}
}

%\newcommand{\DETuvwxyijklm}[10]{
%\begin{vmatrix}
% {#1}_{#6} & {#1}_{#7} & {#1}_{#8} & {#1}_{#9} & {#1}_{#10} \\
% {#2}_{#6} & {#2}_{#7} & {#2}_{#8} & {#2}_{#9} & {#2}_{#10} \\
% {#3}_{#6} & {#3}_{#7} & {#3}_{#8} & {#3}_{#9} & {#3}_{#10} \\
% {#4}_{#6} & {#4}_{#7} & {#4}_{#8} & {#4}_{#9} & {#4}_{#10} \\
% {#5}_{#6} & {#5}_{#7} & {#5}_{#8} & {#5}_{#9} & {#5}_{#10}
%\end{vmatrix}
%}

% R3 vector.
\newcommand{\VectorThree}[3]{
\begin{bmatrix}
 {#1} \\
 {#2} \\
 {#3}
\end{bmatrix}
}



\author{Peeter Joot}
\email{peeter.joot@gmail.com}


\chapter{Relativistic Doppler formula.}
\label{chap:frequencyTx}
\blogpage{http://sites.google.com/site/peeterjoot/math2009/frequencyTx.pdf}
%\date{June 27, 2009}
%\revisionInfo{$RCSfile: frequencyTx.tex,v $ Last $Revision: 1.5 $ $Date: 2009/07/11 05:49:02 $}

\date{June 27, 2009 $RCSfile: frequencyTx.tex,v $ Last $Revision: 1.5 $ $Date: 2009/07/11 05:49:02 $}

\beginArtWithToc

\section{Transform of angular velocity four vector.}

It was possible to derive the Lorentz boost matrix by requiring that the wave equation operator

\begin{align}
\grad^2 = \inv{c^2}\frac{\partial^2}{\partial t^2} - \spacegrad^2
\end{align}

retain its form under linear transformation (\cite{PJLorentzWave}).  Applying spatial Fourier transforms (\cite{PJwaveFourier}), one finds that solutions to the wave equation 

\begin{align}
\grad^2 \psi(t,\Bx) = 0
\end{align}

Have the form

\begin{align}
\psi(t, \Bx) = \int A(\Bk) e^{i(\Bk \cdot \Bx - \omega t)} d^3 k
\end{align}

Provided that $\omega = \pm c \Abs{\Bk}$.  Wave equation solutions can therefore be thought of as continuously weighted superpositions of constrained fundamental solutions

\begin{align}
\psi &= e^{i(\Bk \cdot \Bx - \omega t)} \\
c^2 \Bk^2 &= \omega^2
\end{align}

The constraint on frequency and wave number has the look of a Lorentz square

\begin{align}
\omega^2 - c^2 \Bk^2 = 0
\end{align}

Which suggests that in additional to the spacetime vector

\begin{align}
X = (ct, \Bx) = x^\mu \gamma_\mu
\end{align}

evident in the wave equation fundamental solution, we also have a frequency-wavenumber four vector

\begin{align}
K = (\omega/c, \Bk) = k^\mu \gamma_\mu
\end{align}

The pair of four vectors above allow the fundamental solutions to be put explicitly into covariant form

\begin{align}
K \cdot X = \omega t - \Bk \cdot \Bx = k_\mu x^\mu
\end{align}

\begin{align}
\psi = e^{-i K \cdot X}
\end{align}

Let's also examine the transformation properties of this fundamental solution, and see as a side effect that $K$
has transforms appropriately as a four vector.

\begin{align*}
0 &= \grad^2 \psi(t,\Bx) \\
&= {\grad'}^2 \psi(t',\Bx') \\
&= {\grad'}^2 e^{i(\Bx' \cdot \Bk' - \omega' t')} \\
&= -\left(\frac{{\omega'}^2}{c^2} - {\Bk'}^2 \right) e^{i(\Bx' \cdot \Bk' - \omega' t')} \\
\end{align*}

We therefore have the same form of frequency wave number constraint in the transformed frame (if we require that
the wave function for light is unchanged under transformation)

\begin{align}
{\omega'}^2 = c^2 {\Bk'}^2 
\end{align}

Writing this as

\begin{align}
0 = {\omega}^2 - c^2 {\Bk}^2 = {\omega'}^2 - c^2 {\Bk'}^2 
\end{align}

singles out the Lorentz invariant nature of the $(\omega, \Bk)$ pairing, and we conclude that this pairing 
does indeed transform as a four vector.

\section{Application of one dimensional boost.}

Having attempted to justify the four vector nature of the wave number vector $K$, now move on to application of a boost along the x-axis to this vector.

\begin{align*}
\begin{bmatrix}
\omega' \\
c k' \\
\end{bmatrix}
&=
\gamma
\begin{bmatrix}
1 & -\beta \\
-\beta& 1 \\
\end{bmatrix}
\begin{bmatrix}
\omega \\
c k \\
\end{bmatrix} 
\\
&=
\begin{bmatrix}
\omega - v k \\
c k - \beta \omega
\end{bmatrix} 
\end{align*}

We can take ratios of the frequencies if we make use of the dependency between $\omega$ and $k$.  Namely, $\omega = \pm c k$.  We then have

\begin{align*}
\frac{\omega'}{\omega}
%&= \omega - (v/c) (\pm \omega)
&= \gamma(1 \mp \beta) \\
&= \frac{1 \mp \beta}{\sqrt{1 - \beta^2}} \\
&= \frac{1 \mp \beta}{\sqrt{1 - \beta}\sqrt{1 + \beta}} \\
\end{align*}

For the positive angular frequency this is

\begin{align*}
\frac{\omega'}{\omega}
&= \frac{\sqrt{1 - \beta}}{\sqrt{1 + \beta}} 
\\
\end{align*}

and for the negative frequency the reciprocal.

Deriving this with a Lorentz boost is much simpler than the time dilation argument in wikipedia doppler article (\cite{wiki:relDoppler}).  EDIT: Later found exactly the above boost argument in the wiki k-vector article (\cite{wiki:kvector}).

What's missing here is putting this in a physical context properly with source and reciever frequencies spelled out.  That would make this more than just math.

\EndArticle
