\documentclass{article}

\usepackage{amsmath}
\usepackage{mathpazo}

%
% shorthand for bold symbols, convenient for vectors and matrices
%
\newcommand{\Ba}[0]{\mathbf{a}}
\newcommand{\Bb}[0]{\mathbf{b}}
\newcommand{\Bc}[0]{\mathbf{c}}
\newcommand{\Bd}[0]{\mathbf{d}}
\newcommand{\Be}[0]{\mathbf{e}}
\newcommand{\Bf}[0]{\mathbf{f}}
\newcommand{\Bg}[0]{\mathbf{g}}
\newcommand{\Bh}[0]{\mathbf{h}}
\newcommand{\Bi}[0]{\mathbf{i}}
\newcommand{\Bj}[0]{\mathbf{j}}
\newcommand{\Bk}[0]{\mathbf{k}}
\newcommand{\Bl}[0]{\mathbf{l}}
\newcommand{\Bm}[0]{\mathbf{m}}
\newcommand{\Bn}[0]{\mathbf{n}}
\newcommand{\Bo}[0]{\mathbf{o}}
\newcommand{\Bp}[0]{\mathbf{p}}
\newcommand{\Bq}[0]{\mathbf{q}}
\newcommand{\Br}[0]{\mathbf{r}}
\newcommand{\Bs}[0]{\mathbf{s}}
\newcommand{\Bt}[0]{\mathbf{t}}
\newcommand{\Bu}[0]{\mathbf{u}}
\newcommand{\Bv}[0]{\mathbf{v}}
\newcommand{\Bw}[0]{\mathbf{w}}
\newcommand{\Bx}[0]{\mathbf{x}}
\newcommand{\By}[0]{\mathbf{y}}
\newcommand{\Bz}[0]{\mathbf{z}}
\newcommand{\BA}[0]{\mathbf{A}}
\newcommand{\BB}[0]{\mathbf{B}}
\newcommand{\BC}[0]{\mathbf{C}}
\newcommand{\BD}[0]{\mathbf{D}}
\newcommand{\BE}[0]{\mathbf{E}}
\newcommand{\BF}[0]{\mathbf{F}}
\newcommand{\BG}[0]{\mathbf{G}}
\newcommand{\BH}[0]{\mathbf{H}}
\newcommand{\BI}[0]{\mathbf{I}}
\newcommand{\BJ}[0]{\mathbf{J}}
\newcommand{\BK}[0]{\mathbf{K}}
\newcommand{\BL}[0]{\mathbf{L}}
\newcommand{\BM}[0]{\mathbf{M}}
\newcommand{\BN}[0]{\mathbf{N}}
\newcommand{\BO}[0]{\mathbf{O}}
\newcommand{\BP}[0]{\mathbf{P}}
\newcommand{\BQ}[0]{\mathbf{Q}}
\newcommand{\BR}[0]{\mathbf{R}}
\newcommand{\BS}[0]{\mathbf{S}}
\newcommand{\BT}[0]{\mathbf{T}}
\newcommand{\BU}[0]{\mathbf{U}}
\newcommand{\BV}[0]{\mathbf{V}}
\newcommand{\BW}[0]{\mathbf{W}}
\newcommand{\BX}[0]{\mathbf{X}}
\newcommand{\BY}[0]{\mathbf{Y}}
\newcommand{\BZ}[0]{\mathbf{Z}}

\newcommand{\Bzero}[0]{\mathbf{0}}
\newcommand{\Btheta}[0]{\boldsymbol{\theta}}
\newcommand{\Btau}[0]{\boldsymbol{\tau}}
\newcommand{\Bomega}[0]{\boldsymbol{\omega}}

%
% shorthand for unit vectors
%
\newcommand{\acap}[0]{\hat{\Ba}}
\newcommand{\bcap}[0]{\hat{\Bb}}
\newcommand{\ccap}[0]{\hat{\Bc}}
\newcommand{\dcap}[0]{\hat{\Bd}}
\newcommand{\ecap}[0]{\hat{\Be}}
\newcommand{\fcap}[0]{\hat{\Bf}}
\newcommand{\gcap}[0]{\hat{\Bg}}
\newcommand{\hcap}[0]{\hat{\Bh}}
\newcommand{\icap}[0]{\hat{\Bi}}
\newcommand{\jcap}[0]{\hat{\Bj}}
\newcommand{\kcap}[0]{\hat{\Bk}}
\newcommand{\lcap}[0]{\hat{\Bl}}
\newcommand{\mcap}[0]{\hat{\Bm}}
\newcommand{\ncap}[0]{\hat{\Bn}}
\newcommand{\ocap}[0]{\hat{\Bo}}
\newcommand{\pcap}[0]{\hat{\Bp}}
\newcommand{\qcap}[0]{\hat{\Bq}}
\newcommand{\rcap}[0]{\hat{\Br}}
\newcommand{\scap}[0]{\hat{\Bs}}
\newcommand{\tcap}[0]{\hat{\Bt}}
\newcommand{\ucap}[0]{\hat{\Bu}}
\newcommand{\vcap}[0]{\hat{\Bv}}
\newcommand{\wcap}[0]{\hat{\Bw}}
\newcommand{\xcap}[0]{\hat{\Bx}}
\newcommand{\ycap}[0]{\hat{\By}}
\newcommand{\zcap}[0]{\hat{\Bz}}
\newcommand{\thetacap}[0]{\hat{\Btheta}}

%
% to write R^n and C^n in a distinguishable fashion.  Perhaps change this
% to the double lined characters upon figuring out how to do so.
%
\newcommand{\C}[1]{$\mathbb{C}^{#1}$}
\newcommand{\R}[1]{$\mathbb{R}^{#1}$}

%
% various generally useful helpers
%

% derivative of #1 wrt. #2:
\newcommand{\D}[2] {\frac {d#2} {d#1}}

\newcommand{\inv}[1]{\frac{1}{#1}}
\newcommand{\cross}[0]{\times}

\newcommand{\abs}[1]{\lvert{#1}\rvert}
\newcommand{\norm}[1]{\lVert{#1}\rVert}
\newcommand{\innerprod}[2]{\langle{#1}, {#2}\rangle}
\newcommand{\dotprod}[2]{{#1} \cdot {#2}}
\newcommand{\bdotprod}[2]{\left({#1} \cdot {#2}\right)}
\newcommand{\crossprod}[2]{{#1} \cross {#2}}
\newcommand{\tripleprod}[3]{\dotprod{\left(\crossprod{#1}{#2}\right)}{#3}}

\DeclareMathOperator{\Proj}{Proj}
\DeclareMathOperator{\Span}{span}
\DeclareMathOperator{\Sgn}{sgn}
\DeclareMathOperator{\Area}{Area}
\DeclareMathOperator{\Volume}{Volume}

%
% A few miscellaneous things specific to this document
%
\newcommand{\crossop}[1]{\crossprod{#1}{}}

% R2 vector.
\newcommand{\VectorTwo}[2]{
\begin{bmatrix}
 {#1} \\
 {#2}
\end{bmatrix}
}

\newcommand{\VectorN}[1]{
\begin{bmatrix}
{#1}_1 \\
{#1}_2 \\
\vdots \\
{#1}_N \\
\end{bmatrix}
}

\newcommand{\DETuvij}[4]{
\begin{vmatrix}
 {#1}_{#3} & {#1}_{#4} \\
 {#2}_{#3} & {#2}_{#4}
\end{vmatrix}
}

\newcommand{\DETuvwijk}[6]{
\begin{vmatrix}
 {#1}_{#4} & {#1}_{#5} & {#1}_{#6} \\
 {#2}_{#4} & {#2}_{#5} & {#2}_{#6} \\
 {#3}_{#4} & {#3}_{#5} & {#3}_{#6}
\end{vmatrix}
}

\newcommand{\DETuvwxijkl}[8]{
\begin{vmatrix}
 {#1}_{#5} & {#1}_{#6} & {#1}_{#7} & {#1}_{#8} \\
 {#2}_{#5} & {#2}_{#6} & {#2}_{#7} & {#2}_{#8} \\
 {#3}_{#5} & {#3}_{#6} & {#3}_{#7} & {#3}_{#8} \\
 {#4}_{#5} & {#4}_{#6} & {#4}_{#7} & {#4}_{#8} \\
\end{vmatrix}
}

%\newcommand{\DETuvwxyijklm}[10]{
%\begin{vmatrix}
% {#1}_{#6} & {#1}_{#7} & {#1}_{#8} & {#1}_{#9} & {#1}_{#10} \\
% {#2}_{#6} & {#2}_{#7} & {#2}_{#8} & {#2}_{#9} & {#2}_{#10} \\
% {#3}_{#6} & {#3}_{#7} & {#3}_{#8} & {#3}_{#9} & {#3}_{#10} \\
% {#4}_{#6} & {#4}_{#7} & {#4}_{#8} & {#4}_{#9} & {#4}_{#10} \\
% {#5}_{#6} & {#5}_{#7} & {#5}_{#8} & {#5}_{#9} & {#5}_{#10}
%\end{vmatrix}
%}

% R3 vector.
\newcommand{\VectorThree}[3]{
\begin{bmatrix}
 {#1} \\
 {#2} \\
 {#3}
\end{bmatrix}
}


%<misc>
%
\newcommand{\Abs}[1]{{\left\lvert{#1}\right\rvert}}
\newcommand{\spacegrad}[0]{\boldsymbol{\nabla}}
\newcommand{\grad}[0]{\nabla}
\newcommand{\LL}[0]{\mathcal{L}}

% == \partial_{#1} {#2}
\newcommand{\PD}[2]{\frac{\partial {#2}}{\partial {#1}}}
% inline variant
\newcommand{\PDi}[2]{{\partial {#2}}/{\partial {#1}}}

\newcommand{\PDD}[3]{\frac{\partial^2 {#3}}{\partial {#1}\partial {#2}}}
%\newcommand{\PDd}[2]{\frac{\partial^2 {#2}}{{\partial{#1}}^2}}
\newcommand{\PDsq}[2]{\frac{\partial^2 {#2}}{(\partial {#1})^2}}

\newcommand{\Partial}[2]{\frac{\partial {#1}}{\partial {#2}}}
\DeclareMathOperator{\RejName}{Rej}
\newcommand{\Rej}[2]{\RejName_{#1}\left( {#2} \right)}
\newcommand{\Rm}[1]{\mathbb{R}^{#1}}
\newcommand{\Cm}[1]{\mathbb{C}^{#1}}
\newcommand{\conj}[0]{{*}}

%</misc>

% <grade selection>
%
\newcommand{\gpgrade}[2] {{\left\langle{{#1}}\right\rangle}_{#2}}

\newcommand{\gpgradezero}[1] {\gpgrade{#1}{}}
%\newcommand{\gpscalargrade}[1] {{\left\langle{{#1}}\right\rangle}}
%\newcommand{\gpgradezero}[1] {\gpgrade{#1}{0}}

%\newcommand{\gpgradeone}[1] {{\left\langle{{#1}}\right\rangle}_{1}}
\newcommand{\gpgradeone}[1] {\gpgrade{#1}{1}}

\newcommand{\gpgradetwo}[1] {\gpgrade{#1}{2}}
\newcommand{\gpgradethree}[1] {\gpgrade{#1}{3}}
\newcommand{\gpgradefour}[1] {\gpgrade{#1}{4}}
%
% </grade selection>



\newcommand{\adot}[0]{{\dot{a}}}
\newcommand{\bdot}[0]{{\dot{b}}}
% taken for centered dot:
%\newcommand{\cdot}[0]{{\dot{c}}}
%\newcommand{\ddot}[0]{{\dot{d}}}
\newcommand{\edot}[0]{{\dot{e}}}
\newcommand{\fdot}[0]{{\dot{f}}}
\newcommand{\gdot}[0]{{\dot{g}}}
\newcommand{\hdot}[0]{{\dot{h}}}
\newcommand{\idot}[0]{{\dot{i}}}
\newcommand{\jdot}[0]{{\dot{j}}}
\newcommand{\kdot}[0]{{\dot{k}}}
\newcommand{\ldot}[0]{{\dot{l}}}
\newcommand{\mdot}[0]{{\dot{m}}}
\newcommand{\ndot}[0]{{\dot{n}}}
%\newcommand{\odot}[0]{{\dot{o}}}
\newcommand{\pdot}[0]{{\dot{p}}}
\newcommand{\qdot}[0]{{\dot{q}}}
\newcommand{\rdot}[0]{{\dot{r}}}
\newcommand{\sdot}[0]{{\dot{s}}}
\newcommand{\tdot}[0]{{\dot{t}}}
\newcommand{\udot}[0]{{\dot{u}}}
\newcommand{\vdot}[0]{{\dot{v}}}
\newcommand{\wdot}[0]{{\dot{w}}}
\newcommand{\xdot}[0]{{\dot{x}}}
\newcommand{\ydot}[0]{{\dot{y}}}
\newcommand{\zdot}[0]{{\dot{z}}}
\newcommand{\addot}[0]{{\ddot{a}}}
\newcommand{\bddot}[0]{{\ddot{b}}}
\newcommand{\cddot}[0]{{\ddot{c}}}
%\newcommand{\dddot}[0]{{\ddot{d}}}
\newcommand{\eddot}[0]{{\ddot{e}}}
\newcommand{\fddot}[0]{{\ddot{f}}}
\newcommand{\gddot}[0]{{\ddot{g}}}
\newcommand{\hddot}[0]{{\ddot{h}}}
\newcommand{\iddot}[0]{{\ddot{i}}}
\newcommand{\jddot}[0]{{\ddot{j}}}
\newcommand{\kddot}[0]{{\ddot{k}}}
\newcommand{\lddot}[0]{{\ddot{l}}}
\newcommand{\mddot}[0]{{\ddot{m}}}
\newcommand{\nddot}[0]{{\ddot{n}}}
\newcommand{\oddot}[0]{{\ddot{o}}}
\newcommand{\pddot}[0]{{\ddot{p}}}
\newcommand{\qddot}[0]{{\ddot{q}}}
\newcommand{\rddot}[0]{{\ddot{r}}}
\newcommand{\sddot}[0]{{\ddot{s}}}
\newcommand{\tddot}[0]{{\ddot{t}}}
\newcommand{\uddot}[0]{{\ddot{u}}}
\newcommand{\vddot}[0]{{\ddot{v}}}
\newcommand{\wddot}[0]{{\ddot{w}}}
\newcommand{\xddot}[0]{{\ddot{x}}}
\newcommand{\yddot}[0]{{\ddot{y}}}
\newcommand{\zddot}[0]{{\ddot{z}}}

%<bold and dot greek symbols>
%

\newcommand{\Deltadot}[0]{{\dot{\Delta}}}
\newcommand{\Gammadot}[0]{{\dot{\Gamma}}}
\newcommand{\Lambdadot}[0]{{\dot{\Lambda}}}
\newcommand{\Omegadot}[0]{{\dot{\Omega}}}
\newcommand{\Phidot}[0]{{\dot{\Phi}}}
\newcommand{\Pidot}[0]{{\dot{\Pi}}}
\newcommand{\Psidot}[0]{{\dot{\Psi}}}
\newcommand{\Sigmadot}[0]{{\dot{\Sigma}}}
\newcommand{\Thetadot}[0]{{\dot{\Theta}}}
\newcommand{\Upsilondot}[0]{{\dot{\Upsilon}}}
\newcommand{\Xidot}[0]{{\dot{\Xi}}}
\newcommand{\alphadot}[0]{{\dot{\alpha}}}
\newcommand{\betadot}[0]{{\dot{\beta}}}
\newcommand{\chidot}[0]{{\dot{\chi}}}
\newcommand{\deltadot}[0]{{\dot{\delta}}}
\newcommand{\epsilondot}[0]{{\dot{\epsilon}}}
\newcommand{\etadot}[0]{{\dot{\eta}}}
\newcommand{\gammadot}[0]{{\dot{\gamma}}}
\newcommand{\kappadot}[0]{{\dot{\kappa}}}
\newcommand{\lambdadot}[0]{{\dot{\lambda}}}
\newcommand{\mudot}[0]{{\dot{\mu}}}
\newcommand{\nudot}[0]{{\dot{\nu}}}
\newcommand{\omegadot}[0]{{\dot{\omega}}}
\newcommand{\phidot}[0]{{\dot{\phi}}}
\newcommand{\pidot}[0]{{\dot{\pi}}}
\newcommand{\psidot}[0]{{\dot{\psi}}}
\newcommand{\rhodot}[0]{{\dot{\rho}}}
\newcommand{\sigmadot}[0]{{\dot{\sigma}}}
\newcommand{\taudot}[0]{{\dot{\tau}}}
\newcommand{\thetadot}[0]{{\dot{\theta}}}
\newcommand{\upsilondot}[0]{{\dot{\upsilon}}}
\newcommand{\varepsilondot}[0]{{\dot{\varepsilon}}}
\newcommand{\varphidot}[0]{{\dot{\varphi}}}
\newcommand{\varpidot}[0]{{\dot{\varpi}}}
\newcommand{\varrhodot}[0]{{\dot{\varrho}}}
\newcommand{\varsigmadot}[0]{{\dot{\varsigma}}}
\newcommand{\varthetadot}[0]{{\dot{\vartheta}}}
\newcommand{\xidot}[0]{{\dot{\xi}}}
\newcommand{\zetadot}[0]{{\dot{\zeta}}}

\newcommand{\Deltaddot}[0]{{\ddot{\Delta}}}
\newcommand{\Gammaddot}[0]{{\ddot{\Gamma}}}
\newcommand{\Lambdaddot}[0]{{\ddot{\Lambda}}}
\newcommand{\Omegaddot}[0]{{\ddot{\Omega}}}
\newcommand{\Phiddot}[0]{{\ddot{\Phi}}}
\newcommand{\Piddot}[0]{{\ddot{\Pi}}}
\newcommand{\Psiddot}[0]{{\ddot{\Psi}}}
\newcommand{\Sigmaddot}[0]{{\ddot{\Sigma}}}
\newcommand{\Thetaddot}[0]{{\ddot{\Theta}}}
\newcommand{\Upsilonddot}[0]{{\ddot{\Upsilon}}}
\newcommand{\Xiddot}[0]{{\ddot{\Xi}}}
\newcommand{\alphaddot}[0]{{\ddot{\alpha}}}
\newcommand{\betaddot}[0]{{\ddot{\beta}}}
\newcommand{\chiddot}[0]{{\ddot{\chi}}}
\newcommand{\deltaddot}[0]{{\ddot{\delta}}}
\newcommand{\epsilonddot}[0]{{\ddot{\epsilon}}}
\newcommand{\etaddot}[0]{{\ddot{\eta}}}
\newcommand{\gammaddot}[0]{{\ddot{\gamma}}}
\newcommand{\kappaddot}[0]{{\ddot{\kappa}}}
\newcommand{\lambdaddot}[0]{{\ddot{\lambda}}}
\newcommand{\muddot}[0]{{\ddot{\mu}}}
\newcommand{\nuddot}[0]{{\ddot{\nu}}}
\newcommand{\omegaddot}[0]{{\ddot{\omega}}}
\newcommand{\phiddot}[0]{{\ddot{\phi}}}
\newcommand{\piddot}[0]{{\ddot{\pi}}}
\newcommand{\psiddot}[0]{{\ddot{\psi}}}
\newcommand{\rhoddot}[0]{{\ddot{\rho}}}
\newcommand{\sigmaddot}[0]{{\ddot{\sigma}}}
\newcommand{\tauddot}[0]{{\ddot{\tau}}}
\newcommand{\thetaddot}[0]{{\ddot{\theta}}}
\newcommand{\upsilonddot}[0]{{\ddot{\upsilon}}}
\newcommand{\varepsilonddot}[0]{{\ddot{\varepsilon}}}
\newcommand{\varphiddot}[0]{{\ddot{\varphi}}}
\newcommand{\varpiddot}[0]{{\ddot{\varpi}}}
\newcommand{\varrhoddot}[0]{{\ddot{\varrho}}}
\newcommand{\varsigmaddot}[0]{{\ddot{\varsigma}}}
\newcommand{\varthetaddot}[0]{{\ddot{\vartheta}}}
\newcommand{\xiddot}[0]{{\ddot{\xi}}}
\newcommand{\zetaddot}[0]{{\ddot{\zeta}}}

\newcommand{\BDelta}[0]{\boldsymbol{\Delta}}
\newcommand{\BGamma}[0]{\boldsymbol{\Gamma}}
\newcommand{\BLambda}[0]{\boldsymbol{\Lambda}}
\newcommand{\BOmega}[0]{\boldsymbol{\Omega}}
\newcommand{\BPhi}[0]{\boldsymbol{\Phi}}
\newcommand{\BPi}[0]{\boldsymbol{\Pi}}
\newcommand{\BPsi}[0]{\boldsymbol{\Psi}}
\newcommand{\BSigma}[0]{\boldsymbol{\Sigma}}
\newcommand{\BTheta}[0]{\boldsymbol{\Theta}}
\newcommand{\BUpsilon}[0]{\boldsymbol{\Upsilon}}
\newcommand{\BXi}[0]{\boldsymbol{\Xi}}
\newcommand{\Balpha}[0]{\boldsymbol{\alpha}}
\newcommand{\Bbeta}[0]{\boldsymbol{\beta}}
\newcommand{\Bchi}[0]{\boldsymbol{\chi}}
\newcommand{\Bdelta}[0]{\boldsymbol{\delta}}
\newcommand{\Bepsilon}[0]{\boldsymbol{\epsilon}}
\newcommand{\Beta}[0]{\boldsymbol{\eta}}
\newcommand{\Bgamma}[0]{\boldsymbol{\gamma}}
\newcommand{\Bkappa}[0]{\boldsymbol{\kappa}}
\newcommand{\Blambda}[0]{\boldsymbol{\lambda}}
\newcommand{\Bmu}[0]{\boldsymbol{\mu}}
\newcommand{\Bnu}[0]{\boldsymbol{\nu}}
%\newcommand{\Bomega}[0]{\boldsymbol{\omega}}
\newcommand{\Bphi}[0]{\boldsymbol{\phi}}
\newcommand{\Bpi}[0]{\boldsymbol{\pi}}
\newcommand{\Bpsi}[0]{\boldsymbol{\psi}}
\newcommand{\Brho}[0]{\boldsymbol{\rho}}
\newcommand{\Bsigma}[0]{\boldsymbol{\sigma}}
%\newcommand{\Btau}[0]{\boldsymbol{\tau}}
%\newcommand{\Btheta}[0]{\boldsymbol{\theta}}
\newcommand{\Bupsilon}[0]{\boldsymbol{\upsilon}}
\newcommand{\Bvarepsilon}[0]{\boldsymbol{\varepsilon}}
\newcommand{\Bvarphi}[0]{\boldsymbol{\varphi}}
\newcommand{\Bvarpi}[0]{\boldsymbol{\varpi}}
\newcommand{\Bvarrho}[0]{\boldsymbol{\varrho}}
\newcommand{\Bvarsigma}[0]{\boldsymbol{\varsigma}}
\newcommand{\Bvartheta}[0]{\boldsymbol{\vartheta}}
\newcommand{\Bxi}[0]{\boldsymbol{\xi}}
\newcommand{\Bzeta}[0]{\boldsymbol{\zeta}}
%
%</bold and dot greek symbols>
%<infrequent>
%
%\newcommand{\AreaOp}[1]{\AName_{#1}}
%\newcommand{\Babs}[0]{\abs{\BB}}
%\newcommand{\Bcap}[0]{\hat{\BB}}
%\newcommand{\BrPrimeRej}[0]{\rcap(\rcap \wedge \Br')}
%\newcommand{\CA}[0]{\mathcal{A}}
%\newcommand{\Cos}[1]{\cos{\left({#1}\right)}}
%\newcommand{\Det}[1] {\abs{#1}}
%\newcommand{\Dsq}[2] {\frac {\partial^2 {#1}} {\partial {#2}^2}}
%\newcommand{\Exp}[1]{\exp{\left({#1}\right)}}
%\newcommand{\Norm}[1]{\left\lVert{#1}\right\rVert}
%\newcommand{\Sin}[1]{\sin{\left({#1}\right)}}
%\newcommand{\T}[0]{\text{T}}
%\newcommand{\VolumeOp}[1]{\VName_{#1}}
%\newcommand{\agrad}[0]{\Ba \cdot \nabla}
%\newcommand{\alphacap}[0]{\hat{\boldsymbol{\alpha}}}
%\newcommand{\Fcap}[0]{\hat{\BF}}
%\newcommand{\bithree}[0]{{\Bi}_3}
%\newcommand{\bxa}[0]{\Bx\Ba}
%\newcommand{\coordvec}[2]{
%\newcommand{\costheta}[0]{\acap \cdot \xcap}
%\newcommand{\ddt}[1]{\ddot{#1}}
%\newcommand{\ddu}[1] {\frac {d{#1}} {du}}
%\newcommand{\dsqxj}[2] {\frac {\partial^2 {#1}} {\partial {x_{#2}}^2}}
%\newcommand{\dtheta}[1]{\frac{d {#1}}{d \theta}}
%\newcommand{\dt}[1]{\dot{#1}}
%\newcommand{\dt}[1]{\frac{d {#1}}{dt}}
%\newcommand{\dxj}[2] {\frac {\partial {#1}} {\partial {x_{#2}}}}
%\newcommand{\halfPhi}[0]{\frac{\phi}{2}}
%\newcommand{\half}[0]{\inv{2}}
%\newcommand{\inv}[1]{\frac{1}{#1}}
%\newcommand{\laplacian}[0]{\nabla^2}
%\newcommand{\matrixoftx}[3]{
%\newcommand{\nrrp}[0]{\norm{\rcap \wedge \Br'}}
%\newcommand{\oiint}{\bigcirc \hspace{-1.4em} \int \hspace{-.8em} \int}
%\newcommand{\transpose}[1]{{#1}^{\text{T}}}
%\newcommand{\transpose}[1]{{{#1}^{\TextTranspose}}}
%\newcommand{\transpose}[1]{{{#1}^{\text{T}}}}
%\newcommand{\barA}[0]{\bar{A}}
%\newcommand{\qbar}[0]{\bar{q}}
%\newcommand{\qdotbar}[0]{\dot{\bar{q}}}
%
%</infrequent>




\newcommand{\EE}[0]{\boldsymbol{\mathcal{E}}}
\newcommand{\HH}[0]{\boldsymbol{\mathcal{H}}}

\usepackage[bookmarks=true]{hyperref}

\usepackage{color,cite,graphicx}
   % use colour in the document, put your citations as [1-4]
   % rather than [1,2,3,4] (it looks nicer, and the extended LaTeX2e
   % graphics package. 
\usepackage{latexsym,amssymb,epsf} % don't remember if these are
   % needed, but their inclusion can't do any damage


\title{ Plane wave Fourier series solutions to the Maxwell vacuum equation. }
\author{Peeter Joot}
\date{ Feb 08, 2009.  Last Revision: $Date: 2009/02/10 05:26:46 $ }

\begin{document}
\maketitle{}
\tableofcontents

\section{ Motivation. }

In \cite{PJFourierVacuum} an exploration of spatially periodic solutions to the electrodynamic vacuum equation was performed using a multivector formulation 
of a 3D Fourier series.
Here a summary of the results obtained will be presented in a more
coherent fashion, followed by an attempt to build on them.
In particular a complete
description of the field energy and momentum is desired.

A conclusion from the first analysis was that the
orientation of both the electric and magnetic field components
must be perpendicular to the angular velocity and wave number vectors 
within the entire spatial volume.  This was a requirement for the field
solutions to retain a bivector grade (STA/Dirac basis).

Here a specific orientation of the Fourier volume so that two of the axis
lie in the direction of the initial time electric and magnetic fields will be
used.  This is expected to simplify the treatment.

Also note that having obtained some results in a first attempt hindsight
now allows a few choices of variables that will be seen to be appropriate.
The natural motivation for any such choices can be found in the initial
treatment.

\subsection{ Notation. }

Conventions, definitions, and notation used here will largely follow
\cite{PJFourierVacuum}.  Also of possible aid in that document is a 
a table of symbols and their definitions.

\section{ A consise review of results. }

\subsection{ Fourier series and coefficients. }

A notation for a 3D Fourier series for a spatially periodic function and its Fourier coefficients was developed

\begin{align}
f(\Bx) &= \sum_{\Bk} \hat{f}_{\Bk} e^{ - i \Bk \cdot \Bx } \\
\hat{f}_{\Bk} &= \inv{V} \int f(\Bx) e^{ i \Bk \cdot \Bx } d^3 x
\end{align}

In the vector context $\Bk$ is

\begin{align}
\Bk = 2 \pi \sum_m \sigma^m \frac{k_m}{\lambda_m}
\end{align}

Where $\lambda_m$ are the dimensions of the volume of integration, 
$V = \lambda_1 \lambda_2 \lambda_3$ is the volume, and
in an index context $\Bk = \{k_1, k_2, k_3\}$ is a triplet of integers,
positive, negative or zero.

\subsection{ Vacuum solution and constraints. }

We want to find (STA) bivector solutions $F$ to the vacuum Maxwell equation

\begin{align}\label{eqn:maxwell}
\grad F = \gamma_0 (\partial_0 + \spacegrad) F = 0
\end{align}

We start by assuming a Fourier series solution of the form

\begin{align}
F(\Bx,t) &= \sum_{\Bk} \hat{F}_{\Bk}(t) e^{-i \Bk \cdot \Bx} 
\end{align}

For a solution term by term identity is required

\begin{align*}
\PD{t}{} \hat{F}_{\Bk}(t) e^{-i \Bk \cdot \Bx} 
&= -c \sigma^m \hat{F}_{\Bk}(t) \PD{x^m}{} \exp\left(-i 2 \pi \frac{k_j x^j}{ \lambda_j}\right) \\
&= i c \Bk \hat{F}_{\Bk}(t) e^{-i \Bk \cdot \Bx} 
\end{align*}

With $\Bomega = c \Bk$, we have a simple first order single variable differential equation

\begin{align*}
\hat{F}_{\Bk}'(t) = i \Bomega \hat{F}_{\Bk}(t) 
\end{align*}

with solution

\begin{align}
\hat{F}_{\Bk}(t) = e^{i \Bomega t} \hat{F}_{\Bk}
\end{align}

Here, the constant was written as $\hat{F}_{\Bk}$ given prior knowledge that this is will be the fourier coefficient of the
initial time field.  Our assumed solution is now

\begin{align}
F(\Bx,t) &= \sum_{\Bk} e^{i \Bomega t} \hat{F}_{\Bk} e^{-i \Bk \cdot \Bx} 
\end{align}

Observe that for $t = 0$, we have

\begin{align*}
F(\Bx,0) &= \sum_{\Bk} \hat{F}_{\Bk} e^{-i \Bk \cdot \Bx} 
\end{align*}

which is confirmation of the Fourier coefficient role of $\hat{F}_{\Bk}$

\begin{align}
\hat{F}_{\Bk} &= \inv{V} \int F(\Bx', 0) e^{ i \Bk \cdot \Bx' } d^3 x'
\end{align}

\begin{align}\label{eqn:solutionFromInitialConditions}
F(\Bx,t) &= \inv{V} \sum_{\Bk} \int e^{i \Bomega t} F(\Bx', 0) e^{i \Bk \cdot (\Bx'-\Bx) } d^3 x'
\end{align}

It is straightforward to show that $F(\Bx, 0)$, and pseudoscalar exponentials commute.  Specifically we have

\begin{align}
F e^{ i \Bk \cdot \Bx } = e^{ i \Bk \cdot \Bx } F
\end{align}

This follows from the (STA) bivector nature of $F$.

Another commutivity relation of note is between our time phase exponential and the pseudoscalar exponentials.  This one is also straightforward to show
and won't be done again here

\begin{align}
e^{ i \Bomega t} e^{ i \Bk \cdot \Bx } = e^{ i \Bk \cdot \Bx } e^{ i \Bomega t} 
\end{align}

Lastly, and most importantly of the commutitivity relations,
it was also found that the initial field $F(\Bx,0)$ must have both electric and magnetic field components perpendicular to all $\Bomega \propto \Bk$ at all points
$\Bx$ in the integration volume.  
This was because the vacuum Maxwell equation \ref{eqn:maxwell} by itself does not impose any grade requirement on the solution in isolation.  An
additional requirement that the solution have bivector only values imposes this inherent planar nature in a charge free region, at least for solutions
with spatial periodicity.  Some revisiting of previous Fourier transform solutions attempts at the vacuum equation is required since similar constraints are
expected there too.

The planar constraint can be expressed in terms of dot products of the field components, but an alternate way of expressing the same thing was seen to be
a statement of conjugate commutivity between this dual spatial vector exponential and the complete field

\begin{align}\label{eqn:restriction}
e^{ i \Bomega t} F &= F e^{ -i \Bomega t} 
\end{align}

The set of fourier coefficients considered in the sum must be restricted to those values that equation \ref{eqn:restriction} holds.  An effective 
way to achieve this is to 
pick a specific orientation of the coordinate system so the angular
velocity bivector is quantized in the same plane as the field.  This means that
the angular velocity takes on integer multiples $k$ of this value

\begin{align}
i \Bomega_k = 2 \pi i c k \frac{\Bsigma}{\lambda}
\end{align}

Here $\Bsigma$ is a unit vector describing the perpendicular to the plane of the field, or equivalently via a duality relationship $i \Bsigma$ is a unit bivector with the same orientation as the field.

\subsection{ Conjugate operations. }

In order to tackle expansion of energy and momentum in terms of Fourier coefficients, some conjugation operations will be required.

Such a conjugation is found when computing electric and magnetic field components and also in the $T(\gamma_0) \propto F \gamma_0 F$ energy 
momentum four vector.  In both cases it involves products with $\gamma_0$.

\subsection{ Electric and magnetic fields. }

From the total field one can 
obtain the electric and magnetic fields via coordinates as in

%\begin{align*}
%F = E^m \sigma_m + i H^m \sigma_m
%\end{align*}
%
%From which we could write
\begin{align*}
%E^m &= F \cdot \sigma_m \\
%H^m &= (-i F) \cdot \sigma_m
\EE &= \sigma^m (F \cdot \sigma_m) \\
\HH &= \sigma^m ((-i F) \cdot \sigma_m)
\end{align*}

However, due to the conjugation effect of $\gamma_0$ 
(a particular observer's time basis vector)
on $F$, 
we can compute the electric and magnetic field components without resorting to coordinates

\begin{align}
\EE &= \inv{2}(F - \gamma_0 F \gamma_0 ) \\
\HH &= \inv{2i}(F + \gamma_0 F \gamma_0 )
\end{align}

Such a split is expected to show up when examining the energy and momentum of our Fourier expressed field in detail.
%Observe the conjugation effect of the multiplications with $\gamma_0$.

\subsection{ Conjugate effects on the exponentials. }

Now, since $\gamma_0$ anticommutes with $i$ we have a conguation operation on percolation of $\gamma_0$ through the products of an exponential

\begin{align}
\gamma_0 e^{i \Bk \cdot \Bx} = e^{-i \Bk \cdot \Bx} \gamma_0 
\end{align}

However, since $\gamma_0$ also anticommutes with any spatial basis vector $\sigma_k = \gamma_k \gamma_0$, we have for a dual spatial vector exponential

\begin{align}
\gamma_0 e^{i \Bomega t} &= e^{i \Bomega t} \gamma_0
\end{align}

We should now be armed to consider the energy momentum questions that were the desired goal of the initial treatment.

\section{ Energy momentum four vector. }

To obtain the energy component $U$ of the energy momentum four
vector (given here in cgs units)

\begin{align*}
T(\gamma_0) &= \frac{-1}{8\pi}(F \gamma_0 F) \\
\end{align*}

we want a calculation of

\begin{align*}
U 
&= T(\gamma_0) \cdot \gamma_0 \\
&= -\inv{16 \pi} ( F \gamma_0 F \gamma_0 + \gamma_0 F \gamma_0 F )
\end{align*}

%Ignoring the factor of $-16\pi$ for now, 
We can calculate this using the
observed commutativity relationships and equation
\ref{eqn:solutionFromInitialConditions}.  Let's write this using two different 
sets of wave number indexes

\begin{align*}
F(\Bx,t) 
&= \inv{V} \sum_{\Bk} \int e^{i \Bomega_k t + i \Bk \cdot (\Ba-\Bx) } F(\Ba, 0) d^3 a \\
%&= \inv{V} \sum_{\Bk} \int F(\Ba, 0) e^{-i \Bomega_k t + i \Bk \cdot (\Ba-\Bx) } d^3 a \\
&= \inv{V} \sum_{\Bm} \int e^{i \Bomega_m t + i \Bm \cdot (\Bb-\Bx) } F(\Bb, 0) d^3 b 
%&= \inv{V} \sum_{\Bm} \int F(\Bb,0) e^{-i \Bomega_m t + i \Bm \cdot (\Bb-\Bx) } F(\Bb, 0) d^3 b 
\end{align*}

Let's expand out the products in pieces 

\begin{align*}
F \gamma_0 F \gamma_0 
&=
\inv{V^2} \sum_{\Bk,\Bm} 
\int e^{i \Bomega_m t + i \Bm \cdot (\Bb-\Bx) } F(\Bb, 0) d^3 b \gamma_0
e^{i \Bomega_k t + i \Bk \cdot (\Ba-\Bx) } F(\Ba, 0) d^3 a \gamma_0 \\
&=
\inv{V^2} \sum_{\Bk,\Bm} 
\int e^{i \Bomega_m t + i \Bm \cdot (\Bb-\Bx) } F(\Bb, 0) d^3 b 
e^{i \Bomega_k t - i \Bk \cdot (\Ba-\Bx) } \gamma_0 F(\Ba, 0) d^3 a \gamma_0 \\
&=
\inv{V^2} \sum_{\Bk,\Bm} 
\int e^{i \Bomega_m t + i \Bm \cdot (\Bb-\Bx) -i \Bomega_k t - i \Bk \cdot (\Ba-\Bx) } F(\Bb, 0) \gamma_0 F(\Ba, 0) \gamma_0 d^3 a d^3 b \\
\end{align*}

For the second term we have
\begin{align*}
\gamma_0 F \gamma_0 F 
%&=
%\inv{V^2} \sum_{\Bk,\Bm} \int
%\gamma_0 e^{i \Bomega_k t + i \Bk \cdot (\Ba-\Bx) } F(\Ba, 0) d^3 a \gamma_0 e^{i \Bomega_m t + i \Bm \cdot (\Bb-\Bx) } F(\Bb, 0) d^3 b  \\
%&=
%\inv{V^2} \sum_{\Bk,\Bm} \int
%e^{i \Bomega_k t - i \Bk \cdot (\Ba-\Bx) } \gamma_0 F(\Ba, 0) d^3 a e^{i \Bomega_m t - i \Bm \cdot (\Bb-\Bx) } \gamma_0 F(\Bb, 0) d^3 b  \\
%&=
%\inv{V^2} \sum_{\Bk,\Bm} \int
%e^{i \Bomega_k t - i \Bk \cdot (\Ba-\Bx) } \gamma_0 d^3 a e^{-i \Bomega_m t - i \Bm \cdot (\Bb-\Bx) } F(\Ba, 0) \gamma_0 F(\Bb, 0) d^3 b  \\
%&=
%\inv{V^2} \sum_{\Bk,\Bm} \int
%e^{i \Bomega_k t - i \Bk \cdot (\Ba-\Bx) -i \Bomega_m t + i \Bm \cdot (\Bb-\Bx) } \gamma_0 F(\Ba, 0) \gamma_0 F(\Bb, 0) d^3 a d^3 b  \\
&=
\inv{V^2} \sum_{\Bk,\Bm} \int
\gamma_0 e^{i \Bomega_m t + i \Bm \cdot (\Bb-\Bx) } F(\Bb, 0) d^3 b \gamma_0 e^{i \Bomega_k t + i \Bk \cdot (\Ba-\Bx) } F(\Ba, 0) d^3 a  \\
&=
\inv{V^2} \sum_{\Bk,\Bm} \int
e^{i \Bomega_m t - i \Bm \cdot (\Bb-\Bx) } \gamma_0 F(\Bb, 0) d^3 b e^{i \Bomega_k t - i \Bk \cdot (\Ba-\Bx) } \gamma_0 F(\Ba, 0) d^3 a  \\
&=
\inv{V^2} \sum_{\Bk,\Bm} \int
e^{i \Bomega_m t - i \Bm \cdot (\Bb-\Bx) } \gamma_0 d^3 b e^{-i \Bomega_k t - i \Bk \cdot (\Ba-\Bx) } F(\Bb, 0) \gamma_0 F(\Ba, 0) d^3 a  \\
&=
\inv{V^2} \sum_{\Bk,\Bm} \int
e^{i \Bomega_m t - i \Bm \cdot (\Bb-\Bx) -i \Bomega_k t + i \Bk \cdot (\Ba-\Bx) } \gamma_0 F(\Bb, 0) \gamma_0 F(\Ba, 0) d^3 a d^3 b  \\
\end{align*}

Summing these we have
\begin{align*}
F &\gamma_0 F \gamma_0 + \gamma_0 F \gamma_0 F \\
&=\inv{V^2} \sum_{\Bk,\Bm} \int d^3 a d^3 b e^{i (\Bomega_m -\Bomega_k) t } \times \\
&\quad (e^{   i \Bm \cdot (\Bb-\Bx) - i \Bk \cdot (\Ba-\Bx) } F(\Bb, 0) \gamma_0 F(\Ba, 0) \gamma_0  
+e^{ - i \Bm \cdot (\Bb-\Bx) + i \Bk \cdot (\Ba-\Bx) } \gamma_0 F(\Bb, 0) \gamma_0 F(\Ba, 0))
\end{align*}

It is clear that all the $\Bk = \Bm$ terms are considerably less complex.  Those reduce to

\begin{align*}
\inv{V^2} &\sum_{\Bk} \int d^3 a d^3 b (e^{   i \Bk \cdot (\Bb-\Bx) - i \Bk \cdot (\Ba-\Bx) } F(\Bb, 0) \gamma_0 F(\Ba, 0) \gamma_0  
+e^{ - i \Bk \cdot (\Bb-\Bx) + i \Bk \cdot (\Ba-\Bx) } \gamma_0 F(\Bb, 0) \gamma_0 F(\Ba, 0)) \\
&=
\inv{V^2} \sum_{\Bk} \int d^3 a d^3 b (e^{   i \Bk \cdot (\Bb - \Ba) } F(\Bb, 0) \gamma_0 F(\Ba, 0) \gamma_0  
+e^{ i \Bk \cdot (\Ba -\Bb) } \gamma_0 F(\Bb, 0) \gamma_0 F(\Ba, 0)) \\
&=
\sum_{\Bk} ( \hat{F}_k \gamma_0 \hat{F}_k \gamma_0 + \gamma_0 \hat{F}_k \gamma_0 \hat{F}_k ) \\
\end{align*}

All the time dependence has been eliminated and we have something with a structure roughtly equal to the sum of the Fourier coefficients 
times their conjugates.  This is strikingly like a discrete version of the Rayleigh energy theorem seen in 
a Fourier transform context in \cite{haykin1994cs} (with notes on this in \cite{PJqmFourier}).
Before we put nail down the meaning of this precisely, it is expected that 
the $\Bk \ne \Bm$ terms all cancel, and we should verify this.

Having restricted the orientations of the allowed angular velocity bivectors to scalar multiples of the plane formed by the (wedge of) the electric 
and magnetic fields, we have only a single set of indexes to sum over (ie: $\Bk = 2 \pi \sigma k/ \lambda$).
In particular we can sum over $k < m$, and $k > m$ cases separately and add these
with expectation of cancelation.  Let's see if this works out.

Write $\Bomega = 2 \pi \sigma / \lambda$, this is

\begin{align*}
&=\inv{V^2} \sum_{\{k <m\} \cup \{ k > m \}} \int d^3 a d^3 b e^{i \Bomega (m -k) t } \times \\
&\quad (e^{   i m \Bomega \cdot (\Bb-\Bx)/c - i k \Bomega \cdot (\Ba-\Bx)/c } F(\Bb, 0) \gamma_0 F(\Ba, 0) \gamma_0  
+e^{ - i m \Bomega \cdot (\Bb-\Bx)/c + i k \Bomega \cdot (\Ba-\Bx)/c } \gamma_0 F(\Bb, 0) \gamma_0 F(\Ba, 0)) \\
&=\inv{V^2} \sum_{ k <m } \int d^3 a d^3 b \\
&e^{i \Bomega (m -k) t } \times \\
&\quad (e^{   i m \Bomega \cdot (\Bb-\Bx)/c - i k \Bomega \cdot (\Ba-\Bx)/c } F(\Bb, 0) \gamma_0 F(\Ba, 0) \gamma_0  
+e^{ - i m \Bomega \cdot (\Bb-\Bx)/c + i k \Bomega \cdot (\Ba-\Bx)/c } \gamma_0 F(\Bb, 0) \gamma_0 F(\Ba, 0)) \\
&+e^{-i \Bomega (m -k) t } \times \\
&\quad (e^{   i k \Bomega \cdot (\Bb-\Bx)/c - i m \Bomega \cdot (\Ba-\Bx)/c } F(\Bb, 0) \gamma_0 F(\Ba, 0) \gamma_0  
+e^{ - i k \Bomega \cdot (\Bb-\Bx)/c + i m \Bomega \cdot (\Ba-\Bx)/c } \gamma_0 F(\Bb, 0) \gamma_0 F(\Ba, 0)) \\
&=\inv{V^2} \sum_{ k <m } \int d^3 a d^3 b \\
&e^{i \Bomega (m -k) t } \times \\
&\quad (e^{   i m \Bomega \cdot \Bb/c - i k \Bomega \cdot \Ba/c + i (k-m) \Bomega \cdot \Bx/c } F(\Bb, 0) \gamma_0 F(\Ba, 0) \gamma_0  
+e^{ i k \Bomega \cdot \Ba/c - i m \Bomega \cdot \Bb/c - i (k -m) \Bomega \cdot \Bx/c } \gamma_0 F(\Bb, 0) \gamma_0 F(\Ba, 0)) \\
&+e^{-i \Bomega (m -k) t } \times \\
&\quad (e^{   i k \Bomega \cdot \Bb/c - i m \Bomega \cdot \Ba/c -i (k-m) \Bomega \cdot \Bx/c } F(\Bb, 0) \gamma_0 F(\Ba, 0) \gamma_0  
+e^{ - i k \Bomega \cdot \Bb/c + i m \Bomega \cdot \Ba/c + i (k-m) \Bomega \cdot \Bx/c } \gamma_0 F(\Bb, 0) \gamma_0 F(\Ba, 0)) \\
&=\inv{V^2} \sum_{ k <m } \int d^3 a d^3 b \times \\
&(e^{
  i \Bomega (m -k) t
+ i m \Bomega \cdot \Bb/c 
- i k \Bomega \cdot \Ba/c
+ i (k-m) \Bomega \cdot \Bx/c
}
+e^{
- i \Bomega (m -k) t
- i m \Bomega \cdot \Ba/c 
+ i k \Bomega \cdot \Bb/c 
- i (k-m) \Bomega \cdot \Bx/c
} 
)
F(\Bb, 0) \gamma_0 F(\Ba, 0) \gamma_0  \\
&+(e^{  
  i \Bomega (m -k) t 
+ i k \Bomega \cdot \Ba/c 
- i m \Bomega \cdot \Bb/c 
- i (k -m) \Bomega \cdot \Bx/c
} 
+e^{
- i \Bomega (m -k) t  
- i k \Bomega \cdot \Bb/c 
+ i m \Bomega \cdot \Ba/c 
+ i (k-m) \Bomega \cdot \Bx/c
} 
) \gamma_0 F(\Bb, 0) \gamma_0 F(\Ba, 0)  \\
\end{align*}

Damn.  What an unholy mess, and doesn't obviously simplify.  Let's start over for this part.

\begin{align*}
\sum_{\Bk \ne \Bm} &(
e^{i \Bomega_k t} 
\hat{F}_{\Bk} 
e^{-i \Bk \cdot \Bx} 
\gamma_0 
e^{i \Bomega_m t} 
\hat{F}_{\Bm} 
e^{-i \Bm \cdot \Bx} 
\gamma_0
+ \gamma_0 
e^{i \Bomega_k t} 
\hat{F}_{\Bk} 
e^{-i \Bk \cdot \Bx} 
\gamma_0 
e^{i \Bomega_m t} 
\hat{F}_{\Bm} 
e^{-i \Bm \cdot \Bx} 
) \\
&=
\sum_{\Bk \ne \Bm} 
e^{i (\Bomega_k -\Bomega_m) t}
(
e^{ i (\Bm -\Bk)\cdot \Bx} 
\hat{F}_{\Bk} 
\gamma_0 
\hat{F}_{\Bm} 
\gamma_0
+ 
e^{ i (\Bk -\Bm) \cdot \Bx} 
\gamma_0 
\hat{F}_{\Bk} 
\gamma_0 
\hat{F}_{\Bm} 
) \\
&=
\sum_{\Bk \ne \Bm} 
e^{i (k-m) \Bomega t}
(
e^{ i (m -k) \Bomega \cdot \Bx/c} 
\hat{F}_{\Bk} 
\gamma_0 
\hat{F}_{\Bm} 
\gamma_0
+ 
e^{ i (k -m) \Bomega \cdot \Bx/c} 
\gamma_0 
\hat{F}_{\Bk} 
\gamma_0 
\hat{F}_{\Bm} 
) \\
\end{align*}

To progress it appears that we have to examine the commutation behaviour of $\gamma_0$ and $\hat{F}_{\Bk}$, and probably also to 
see if there is a relationship between 
$\hat{F}_{\Bk}$, and
$\hat{F}_{\Bm}$ that can be exploited.

\bibliographystyle{plainnat}
\bibliography{myrefs}

\end{document}
