\documentclass{article}      % Specifies the document class

\usepackage{amsmath}
\usepackage{mathpazo}

%
% shorthand for bold symbols, convenient for vectors and matrices
%
\newcommand{\Ba}[0]{\mathbf{a}}
\newcommand{\Bb}[0]{\mathbf{b}}
\newcommand{\Bc}[0]{\mathbf{c}}
\newcommand{\Bd}[0]{\mathbf{d}}
\newcommand{\Be}[0]{\mathbf{e}}
\newcommand{\Bf}[0]{\mathbf{f}}
\newcommand{\Bg}[0]{\mathbf{g}}
\newcommand{\Bh}[0]{\mathbf{h}}
\newcommand{\Bi}[0]{\mathbf{i}}
\newcommand{\Bj}[0]{\mathbf{j}}
\newcommand{\Bk}[0]{\mathbf{k}}
\newcommand{\Bl}[0]{\mathbf{l}}
\newcommand{\Bm}[0]{\mathbf{m}}
\newcommand{\Bn}[0]{\mathbf{n}}
\newcommand{\Bo}[0]{\mathbf{o}}
\newcommand{\Bp}[0]{\mathbf{p}}
\newcommand{\Bq}[0]{\mathbf{q}}
\newcommand{\Br}[0]{\mathbf{r}}
\newcommand{\Bs}[0]{\mathbf{s}}
\newcommand{\Bt}[0]{\mathbf{t}}
\newcommand{\Bu}[0]{\mathbf{u}}
\newcommand{\Bv}[0]{\mathbf{v}}
\newcommand{\Bw}[0]{\mathbf{w}}
\newcommand{\Bx}[0]{\mathbf{x}}
\newcommand{\By}[0]{\mathbf{y}}
\newcommand{\Bz}[0]{\mathbf{z}}
\newcommand{\BA}[0]{\mathbf{A}}
\newcommand{\BB}[0]{\mathbf{B}}
\newcommand{\BC}[0]{\mathbf{C}}
\newcommand{\BD}[0]{\mathbf{D}}
\newcommand{\BE}[0]{\mathbf{E}}
\newcommand{\BF}[0]{\mathbf{F}}
\newcommand{\BG}[0]{\mathbf{G}}
\newcommand{\BH}[0]{\mathbf{H}}
\newcommand{\BI}[0]{\mathbf{I}}
\newcommand{\BJ}[0]{\mathbf{J}}
\newcommand{\BK}[0]{\mathbf{K}}
\newcommand{\BL}[0]{\mathbf{L}}
\newcommand{\BM}[0]{\mathbf{M}}
\newcommand{\BN}[0]{\mathbf{N}}
\newcommand{\BO}[0]{\mathbf{O}}
\newcommand{\BP}[0]{\mathbf{P}}
\newcommand{\BQ}[0]{\mathbf{Q}}
\newcommand{\BR}[0]{\mathbf{R}}
\newcommand{\BS}[0]{\mathbf{S}}
\newcommand{\BT}[0]{\mathbf{T}}
\newcommand{\BU}[0]{\mathbf{U}}
\newcommand{\BV}[0]{\mathbf{V}}
\newcommand{\BW}[0]{\mathbf{W}}
\newcommand{\BX}[0]{\mathbf{X}}
\newcommand{\BY}[0]{\mathbf{Y}}
\newcommand{\BZ}[0]{\mathbf{Z}}

\newcommand{\Bzero}[0]{\mathbf{0}}
\newcommand{\Btheta}[0]{\boldsymbol{\theta}}
\newcommand{\Btau}[0]{\boldsymbol{\tau}}
\newcommand{\Bomega}[0]{\boldsymbol{\omega}}

%
% shorthand for unit vectors
%
\newcommand{\acap}[0]{\hat{\Ba}}
\newcommand{\bcap}[0]{\hat{\Bb}}
\newcommand{\ccap}[0]{\hat{\Bc}}
\newcommand{\dcap}[0]{\hat{\Bd}}
\newcommand{\ecap}[0]{\hat{\Be}}
\newcommand{\fcap}[0]{\hat{\Bf}}
\newcommand{\gcap}[0]{\hat{\Bg}}
\newcommand{\hcap}[0]{\hat{\Bh}}
\newcommand{\icap}[0]{\hat{\Bi}}
\newcommand{\jcap}[0]{\hat{\Bj}}
\newcommand{\kcap}[0]{\hat{\Bk}}
\newcommand{\lcap}[0]{\hat{\Bl}}
\newcommand{\mcap}[0]{\hat{\Bm}}
\newcommand{\ncap}[0]{\hat{\Bn}}
\newcommand{\ocap}[0]{\hat{\Bo}}
\newcommand{\pcap}[0]{\hat{\Bp}}
\newcommand{\qcap}[0]{\hat{\Bq}}
\newcommand{\rcap}[0]{\hat{\Br}}
\newcommand{\scap}[0]{\hat{\Bs}}
\newcommand{\tcap}[0]{\hat{\Bt}}
\newcommand{\ucap}[0]{\hat{\Bu}}
\newcommand{\vcap}[0]{\hat{\Bv}}
\newcommand{\wcap}[0]{\hat{\Bw}}
\newcommand{\xcap}[0]{\hat{\Bx}}
\newcommand{\ycap}[0]{\hat{\By}}
\newcommand{\zcap}[0]{\hat{\Bz}}
\newcommand{\thetacap}[0]{\hat{\Btheta}}

%
% to write R^n and C^n in a distinguishable fashion.  Perhaps change this
% to the double lined characters upon figuring out how to do so.
%
\newcommand{\C}[1]{$\mathbb{C}^{#1}$}
\newcommand{\R}[1]{$\mathbb{R}^{#1}$}

%
% various generally useful helpers
%

% derivative of #1 wrt. #2:
\newcommand{\D}[2] {\frac {d#2} {d#1}}

\newcommand{\inv}[1]{\frac{1}{#1}}
\newcommand{\cross}[0]{\times}

\newcommand{\abs}[1]{\lvert{#1}\rvert}
\newcommand{\norm}[1]{\lVert{#1}\rVert}
\newcommand{\innerprod}[2]{\langle{#1}, {#2}\rangle}
\newcommand{\dotprod}[2]{{#1} \cdot {#2}}
\newcommand{\bdotprod}[2]{\left({#1} \cdot {#2}\right)}
\newcommand{\crossprod}[2]{{#1} \cross {#2}}
\newcommand{\tripleprod}[3]{\dotprod{\left(\crossprod{#1}{#2}\right)}{#3}}

\DeclareMathOperator{\Proj}{Proj}
\DeclareMathOperator{\Span}{span}
\DeclareMathOperator{\Sgn}{sgn}
\DeclareMathOperator{\Area}{Area}
\DeclareMathOperator{\Volume}{Volume}

%
% A few miscellaneous things specific to this document
%
\newcommand{\crossop}[1]{\crossprod{#1}{}}

% R2 vector.
\newcommand{\VectorTwo}[2]{
\begin{bmatrix}
 {#1} \\
 {#2}
\end{bmatrix}
}

\newcommand{\VectorN}[1]{
\begin{bmatrix}
{#1}_1 \\
{#1}_2 \\
\vdots \\
{#1}_N \\
\end{bmatrix}
}

\newcommand{\DETuvij}[4]{
\begin{vmatrix}
 {#1}_{#3} & {#1}_{#4} \\
 {#2}_{#3} & {#2}_{#4}
\end{vmatrix}
}

\newcommand{\DETuvwijk}[6]{
\begin{vmatrix}
 {#1}_{#4} & {#1}_{#5} & {#1}_{#6} \\
 {#2}_{#4} & {#2}_{#5} & {#2}_{#6} \\
 {#3}_{#4} & {#3}_{#5} & {#3}_{#6}
\end{vmatrix}
}

\newcommand{\DETuvwxijkl}[8]{
\begin{vmatrix}
 {#1}_{#5} & {#1}_{#6} & {#1}_{#7} & {#1}_{#8} \\
 {#2}_{#5} & {#2}_{#6} & {#2}_{#7} & {#2}_{#8} \\
 {#3}_{#5} & {#3}_{#6} & {#3}_{#7} & {#3}_{#8} \\
 {#4}_{#5} & {#4}_{#6} & {#4}_{#7} & {#4}_{#8} \\
\end{vmatrix}
}

%\newcommand{\DETuvwxyijklm}[10]{
%\begin{vmatrix}
% {#1}_{#6} & {#1}_{#7} & {#1}_{#8} & {#1}_{#9} & {#1}_{#10} \\
% {#2}_{#6} & {#2}_{#7} & {#2}_{#8} & {#2}_{#9} & {#2}_{#10} \\
% {#3}_{#6} & {#3}_{#7} & {#3}_{#8} & {#3}_{#9} & {#3}_{#10} \\
% {#4}_{#6} & {#4}_{#7} & {#4}_{#8} & {#4}_{#9} & {#4}_{#10} \\
% {#5}_{#6} & {#5}_{#7} & {#5}_{#8} & {#5}_{#9} & {#5}_{#10}
%\end{vmatrix}
%}

% R3 vector.
\newcommand{\VectorThree}[3]{
\begin{bmatrix}
 {#1} \\
 {#2} \\
 {#3}
\end{bmatrix}
}



%
% The real thing:
%

                             % The preamble begins here.
\title{} % Declares the document's title.
\author{Peeter Joot}         % Declares the author's name.
%\date{}        % Deleting this command produces today's date.

\begin{document}             % End of preamble and beginning of text.

\maketitle{}

\section{ Motivation. }

Both the NFCM and GAFP books have axiomatic introductions of the 
generalized (vector, blade) dot and wedge products, but there are
elements of both that I was unsatisfied with.  Perhaps the biggest
issue with both is that they aren't presented in a dumb enough fashion.

NFCM presents but
does not prove the generalized dot and wedge product operations
in terms of symmetric and antisymmetric sums, but it is really the
grade operation that is fundamental.  You need that to define the
dot product of two bivectors for example.

GAFP axiomatic presentation is much clearer, but the definition of
generalized wedge product as the totally antisymmetric sum is a bit
strange when all the differential forms book give such a different
definition.

Here I collect some of my notes on how one starts with the geometric
product action on colinear and perpendicular vectors and gets the 
familiar results for two and three vector products.  I may not try to
generalize this, but just want to see things presented in a fashion
that makes sense to me.

\section{ axioms }

The axioms for the euclician geometric product of vectors are:

\begin{enumerate}
\item{ linearity }
\item{ associativity }
\item{ contraction }

The product of colinear vectors is the product of their magnitudes
(square of a vector is the squared length).
\end{enumerate}

%This last property is weakened in some circumstances.

As justification of the contraction property one could 
consider a set of colinear vectors and the real number line to be
isomorphic.

The product of two positive numbers is a positive number.  Multiplying
by $-1$ (the unit negative) produces a rotatation by 180 degrees.  Two negative multiplications
produces a rotation of 360.  This can be thought of as a justification
of the grade school ``rule'' that a negative times a negative is positive.

It seems natural to have the rules for vector multiplication reduce to something
like the rules for numbers when those vectors are restricted to a linear subspace.

In analogy with numbers, the contraction rule gives us such similar properties.  Namely, the
product of same facing vectors is positive, and the product of opposite facing
vectors is negative, both scaled by their magnitudes.

\section{ dot product }

One can express the dot product in terms of these axioms.  This follows by calculating the 
length of a sum or difference of vectors, starting with the requirement that the vector square is the squared length of that vector.

Given two vectors $\Ba$ and $\Bb$, their sum
$\Bc = \Ba + \Bb$ has squared length:

\begin{equation}\label{eqn:absquared}
\Bc^2 = (\Ba + \Bb)(\Ba + \Bb) = \Ba^2 + \Bb\Ba + \Ba\Bb + \Bb^2.
\end{equation}

We do not have any specific meaning for the product of vectors, but equation \ref{eqn:absquared}
shows that the symmetric sum of such a product:

\begin{equation}
\Bb\Ba + \Ba\Bb = \text{scalar}
\end{equation}

since the RHS is also a scalar.

Additionally, if $\Ba$ and $\Bb$ are perpendicular, then we must also have:

\[
\Ba^2 + \Bb^2 = a^2 + b^2.
\]

This implies a rule for vector multiplication of perpendicular vectors

\[
\Bb\Ba + \Ba\Bb = 0
\]

Or, 

\begin{equation}\label{eqn:perpabcommutesign}
\Bb\Ba = -\Ba\Bb.
\end{equation}

Note that equation \ref{eqn:perpabcommutesign} doesn't assign any meaning to this product of vectors when they perpendicular.
Whatever that meaning is, the entity such a perpendicular vector product produces changes sign
with commutation.

Performing the same length calculation using standard vector algebra shows that we can identify the symmetric 
sum of vector products with the dot product:

\begin{equation}\label{eqn:standarddot}
\norm{\Bc}^2 = (\Ba + \Bb) \cdot (\Ba + \Bb) = \norm{\Ba}^2 + 2 \Ba \cdot \Bb + \norm{\Bb}^2.
\end{equation}

Thus we can make the identity:

\begin{equation}\label{eqn:dotprod}
\Ba \cdot \Bb = \inv{2}(\Ba \Bb + \Bb \Ba)
\end{equation}

\section{ Dot product in terms of components. }

A powerful feature of geometric algebra is that it allows for component free results, and the avoidance of basis selection
that such components require.  While this is true, I found it helpful for myself to demonstrate that the results
do provide the standard component formulations too.

As an example, one can expand equation \ref{eqn:dotprod} in terms of a standard basis.

Writing $\Ba = \sum{a_i \Be_i}$ and $\Bb = \sum{b_i \Be_i}$, this is

\begin{align*}
\Bb\Ba + \Ba\Bb
        &= \sum{b_i a_j \Be_i \Be_j + a_i b_j \Be_i \Be_j} \\
        &= 
          \sum_{i = j} {b_i a_j \Be_i \Be_j + a_i b_j \Be_i \Be_j} 
        + \sum_{i < j} {b_i a_j \Be_i \Be_j + a_i b_j \Be_i \Be_j} 
        + \sum_{i > j} {b_i a_j \Be_i \Be_j + a_i b_j \Be_i \Be_j} \\
        &= 
        2 \sum_{i} {a_i b_i \Be_i \Be_i} 
        + \sum_{i < j} {b_i a_j \Be_i \Be_j + a_i b_j \Be_i \Be_j
        + b_j a_i \Be_j \Be_i + a_j b_i \Be_j \Be_i} \\
        &=
        2 \sum_{i} {a_i b_i \Be_i \Be_i} 
        + \sum_{i < j} (b_i a_j + a_i b_j - b_j a_i - a_j b_i ) \Be_i \Be_j \\
        &=
        2 \sum_{i} {a_i b_i} 
\end{align*}

Thus without requiring the ``triangle law'' expansion of \ref{eqn:standarddot}, we have a second
verification of the identity of this symmetric sum

\begin{equation}\label{eqn:dotprodcomp}
\frac{1}{2}(\Ba\Bb + \Bb\Ba) = \sum{a_i b_i}
\end{equation}

with the dot product since the RHS of equation \ref{eqn:dotprodcomp} is the standard form for the definition of the euclidian dot product in terms
of components.

\section{ Symmetric and antisymmetric parts of the geometric product. }

Having identified the symmetric sum of vector products with the dot product we can write the geometric product of two arbitrary vectors
in terms of this and it's difference

\begin{align*}
\Ba \Bb 
&= \inv{2}(\Ba \Bb + \Bb \Ba) + \inv{2}(\Ba \Bb - \Bb \Ba) \\
&= \Ba \cdot \Bb + f(\Ba, \Bb) \\
\end{align*}

Let's examine this second term, a mapping of a pair of vectors into a different sort of object of yet unknown properties.

\[
f(\Ba, k\Ba) = \inv{2}(\Ba k\Ba - k\Ba \Ba) = 0
\]

Property: Zero when the vectors are colinear.

\[
f(\Ba, k\Ba + \Bb) = \inv{2}(\Ba (k\Ba + \Bb) - (k\Ba + m\Bb)\Ba) = f(\Ba, \Bb)
\]

Property: colinear contributions are rejected.

\[
f(\alpha \Ba, \beta \Bb) = \inv{2}(\alpha \Ba \beta \Bb - \beta \Bb \alpha \Ba) = \alpha \beta f(\Ba, \Bb)
\]

Property: bilinearity.

\[
f(\Bb, \Ba) 
= \inv{2}(\Bb \Ba - \Ba\Bb) 
= -\inv{2}(\Ba \Bb - \Bb\Ba) 
= -f(\Ba, \Bb) 
\]

Property: Interchange inverts.

These are in fact the properties of what in the calculus of differential forms is called the wedge product (uncoincidentally, these are also all properties of the cross product as well.)

Because the properties are identical the notation from differential forms is stolen, and we write

\begin{equation}\label{eqn:wedge}
\Ba \wedge \Bb = \inv{2}(\Ba \Bb - \Bb \Ba)
\end{equation}

We call the wedge product of two vectors a bivector.

Strictly speaking the 
wedge product of differential calculus is defined as an alternating, associative, multilinear form.  We have here bilinear, not multilinear and associativity is
not meaningful until more than two factors are introduced, however when we get to the product of more than three
vectors, we will find that the geometric vector product produces an entity with all of these properties.

Returning to the product of two vectors we can now write

\begin{equation}\label{eqn:gaproddotwedge}
\Ba \Bb = \Ba \cdot \Bb + \Ba \wedge \Bb
\end{equation}

This is often be used as the initial definition of the geometric product.

\section{ Yes, but what is that wedge product thing. }

Combination of the symmetric and antisymmetric decomposition in equation \ref{eqn:gaproddotwedge} shows that the product of two vectors according to the axioms
has a scalar part and a bivector part.  What is this bivector part geometrically?

One can show that the equation of a plane can be written in terms of bivectors.  One can also show that
the area of the parallelogram spanned by two vectors can be expressed in terms of the ``magnitude'' of a bivector.  Both of these
show that a bivector characterizes a plane and can be thought of loosely as a ``plane vector''.

Neither the plane equation or the area result are hard to show, but we will get to those later.  A more direct way to get an
intuitive feel for the geometric properties of the bivector can be obtained by first examining the
square of a bivector.

By subtracting the projection of one vector $\Ba$ from another $\Bb$, one can form the rejection of $\Ba$ from $\Bb$:

\[
\Bb' = \Bb - (\Bb \cdot \acap)\acap
\]

With respect to the dot product, this vector is orthogonal to $\Ba$.  Since $\Ba \wedge \acap = 0$, this allows us to 
write the wedge product of vectors $\Ba$ and $\Bb$ as the direct product of two othogonal vectors

\begin{align*}
\Ba \wedge \Bb 
&= \Ba \wedge (\Bb - (\Bb \cdot \acap)\acap)) \\
&= \Ba \wedge \Bb' \\
&= \Ba \Bb' \\
&= -\Bb' \Ba \\
\end{align*}

The square of the bivector can then be written

\begin{align*}
(\Ba \wedge \Bb)^2
&= (\Ba \Bb')(-\Bb'\Ba) \\
&= -\Ba^2 (\Bb')^2.
\end{align*}

Thus the square of a bivector is negative.  Looking back to equation \ref{eqn:gaproddotwedge} one can assign additional meaning to the two parts.  The first, the dot product, is a scalar (ie: a real number), and a second part, the wedge product, is an entity that acts like an imaginary number.

The geometric product of two vectors is thus very much like a complex number.  The complex number system
is the algebra of the plane, and the geometric product of two vectors can be used to completely characterize the algebra of an arbitrarily oriented plane in a higher
order vector space.  It actually will be very natural to define complex numbers in terms of the geometric product, and we will see later that
the geometric product allows for the ad-hoc definition of ``complex number'' systems according to convienience in many ways.  
We will also see that generalizations of complex numbers such as quaternion algebras also find their natural place as specific instances of geometric products.
Concepts familar from
complex numbers such as conjugate, inverse, exponentials as rotations, and even things like the residue theory of contour integration, will all have a natural
analogue in this geometric algebra.

We will return to this, but first some more detailed initial examination of the wedge product properties is in order.

\end{document}               % End of document.
