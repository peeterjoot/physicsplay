\documentclass{article}

\usepackage{amsmath}
\usepackage{mathpazo}

%
% shorthand for bold symbols, convenient for vectors and matrices
%
\newcommand{\Ba}[0]{\mathbf{a}}
\newcommand{\Bb}[0]{\mathbf{b}}
\newcommand{\Bc}[0]{\mathbf{c}}
\newcommand{\Bd}[0]{\mathbf{d}}
\newcommand{\Be}[0]{\mathbf{e}}
\newcommand{\Bf}[0]{\mathbf{f}}
\newcommand{\Bg}[0]{\mathbf{g}}
\newcommand{\Bh}[0]{\mathbf{h}}
\newcommand{\Bi}[0]{\mathbf{i}}
\newcommand{\Bj}[0]{\mathbf{j}}
\newcommand{\Bk}[0]{\mathbf{k}}
\newcommand{\Bl}[0]{\mathbf{l}}
\newcommand{\Bm}[0]{\mathbf{m}}
\newcommand{\Bn}[0]{\mathbf{n}}
\newcommand{\Bo}[0]{\mathbf{o}}
\newcommand{\Bp}[0]{\mathbf{p}}
\newcommand{\Bq}[0]{\mathbf{q}}
\newcommand{\Br}[0]{\mathbf{r}}
\newcommand{\Bs}[0]{\mathbf{s}}
\newcommand{\Bt}[0]{\mathbf{t}}
\newcommand{\Bu}[0]{\mathbf{u}}
\newcommand{\Bv}[0]{\mathbf{v}}
\newcommand{\Bw}[0]{\mathbf{w}}
\newcommand{\Bx}[0]{\mathbf{x}}
\newcommand{\By}[0]{\mathbf{y}}
\newcommand{\Bz}[0]{\mathbf{z}}
\newcommand{\BA}[0]{\mathbf{A}}
\newcommand{\BB}[0]{\mathbf{B}}
\newcommand{\BC}[0]{\mathbf{C}}
\newcommand{\BD}[0]{\mathbf{D}}
\newcommand{\BE}[0]{\mathbf{E}}
\newcommand{\BF}[0]{\mathbf{F}}
\newcommand{\BG}[0]{\mathbf{G}}
\newcommand{\BH}[0]{\mathbf{H}}
\newcommand{\BI}[0]{\mathbf{I}}
\newcommand{\BJ}[0]{\mathbf{J}}
\newcommand{\BK}[0]{\mathbf{K}}
\newcommand{\BL}[0]{\mathbf{L}}
\newcommand{\BM}[0]{\mathbf{M}}
\newcommand{\BN}[0]{\mathbf{N}}
\newcommand{\BO}[0]{\mathbf{O}}
\newcommand{\BP}[0]{\mathbf{P}}
\newcommand{\BQ}[0]{\mathbf{Q}}
\newcommand{\BR}[0]{\mathbf{R}}
\newcommand{\BS}[0]{\mathbf{S}}
\newcommand{\BT}[0]{\mathbf{T}}
\newcommand{\BU}[0]{\mathbf{U}}
\newcommand{\BV}[0]{\mathbf{V}}
\newcommand{\BW}[0]{\mathbf{W}}
\newcommand{\BX}[0]{\mathbf{X}}
\newcommand{\BY}[0]{\mathbf{Y}}
\newcommand{\BZ}[0]{\mathbf{Z}}

\newcommand{\Bzero}[0]{\mathbf{0}}
\newcommand{\Btheta}[0]{\boldsymbol{\theta}}
\newcommand{\Btau}[0]{\boldsymbol{\tau}}
\newcommand{\Bomega}[0]{\boldsymbol{\omega}}

%
% shorthand for unit vectors
%
\newcommand{\acap}[0]{\hat{\Ba}}
\newcommand{\bcap}[0]{\hat{\Bb}}
\newcommand{\ccap}[0]{\hat{\Bc}}
\newcommand{\dcap}[0]{\hat{\Bd}}
\newcommand{\ecap}[0]{\hat{\Be}}
\newcommand{\fcap}[0]{\hat{\Bf}}
\newcommand{\gcap}[0]{\hat{\Bg}}
\newcommand{\hcap}[0]{\hat{\Bh}}
\newcommand{\icap}[0]{\hat{\Bi}}
\newcommand{\jcap}[0]{\hat{\Bj}}
\newcommand{\kcap}[0]{\hat{\Bk}}
\newcommand{\lcap}[0]{\hat{\Bl}}
\newcommand{\mcap}[0]{\hat{\Bm}}
\newcommand{\ncap}[0]{\hat{\Bn}}
\newcommand{\ocap}[0]{\hat{\Bo}}
\newcommand{\pcap}[0]{\hat{\Bp}}
\newcommand{\qcap}[0]{\hat{\Bq}}
\newcommand{\rcap}[0]{\hat{\Br}}
\newcommand{\scap}[0]{\hat{\Bs}}
\newcommand{\tcap}[0]{\hat{\Bt}}
\newcommand{\ucap}[0]{\hat{\Bu}}
\newcommand{\vcap}[0]{\hat{\Bv}}
\newcommand{\wcap}[0]{\hat{\Bw}}
\newcommand{\xcap}[0]{\hat{\Bx}}
\newcommand{\ycap}[0]{\hat{\By}}
\newcommand{\zcap}[0]{\hat{\Bz}}
\newcommand{\thetacap}[0]{\hat{\Btheta}}

%
% to write R^n and C^n in a distinguishable fashion.  Perhaps change this
% to the double lined characters upon figuring out how to do so.
%
\newcommand{\C}[1]{$\mathbb{C}^{#1}$}
\newcommand{\R}[1]{$\mathbb{R}^{#1}$}

%
% various generally useful helpers
%

% derivative of #1 wrt. #2:
\newcommand{\D}[2] {\frac {d#2} {d#1}}

\newcommand{\inv}[1]{\frac{1}{#1}}
\newcommand{\cross}[0]{\times}

\newcommand{\abs}[1]{\lvert{#1}\rvert}
\newcommand{\norm}[1]{\lVert{#1}\rVert}
\newcommand{\innerprod}[2]{\langle{#1}, {#2}\rangle}
\newcommand{\dotprod}[2]{{#1} \cdot {#2}}
\newcommand{\bdotprod}[2]{\left({#1} \cdot {#2}\right)}
\newcommand{\crossprod}[2]{{#1} \cross {#2}}
\newcommand{\tripleprod}[3]{\dotprod{\left(\crossprod{#1}{#2}\right)}{#3}}

\DeclareMathOperator{\Proj}{Proj}
\DeclareMathOperator{\Span}{span}
\DeclareMathOperator{\Sgn}{sgn}
\DeclareMathOperator{\Area}{Area}
\DeclareMathOperator{\Volume}{Volume}

%
% A few miscellaneous things specific to this document
%
\newcommand{\crossop}[1]{\crossprod{#1}{}}

% R2 vector.
\newcommand{\VectorTwo}[2]{
\begin{bmatrix}
 {#1} \\
 {#2}
\end{bmatrix}
}

\newcommand{\VectorN}[1]{
\begin{bmatrix}
{#1}_1 \\
{#1}_2 \\
\vdots \\
{#1}_N \\
\end{bmatrix}
}

\newcommand{\DETuvij}[4]{
\begin{vmatrix}
 {#1}_{#3} & {#1}_{#4} \\
 {#2}_{#3} & {#2}_{#4}
\end{vmatrix}
}

\newcommand{\DETuvwijk}[6]{
\begin{vmatrix}
 {#1}_{#4} & {#1}_{#5} & {#1}_{#6} \\
 {#2}_{#4} & {#2}_{#5} & {#2}_{#6} \\
 {#3}_{#4} & {#3}_{#5} & {#3}_{#6}
\end{vmatrix}
}

\newcommand{\DETuvwxijkl}[8]{
\begin{vmatrix}
 {#1}_{#5} & {#1}_{#6} & {#1}_{#7} & {#1}_{#8} \\
 {#2}_{#5} & {#2}_{#6} & {#2}_{#7} & {#2}_{#8} \\
 {#3}_{#5} & {#3}_{#6} & {#3}_{#7} & {#3}_{#8} \\
 {#4}_{#5} & {#4}_{#6} & {#4}_{#7} & {#4}_{#8} \\
\end{vmatrix}
}

%\newcommand{\DETuvwxyijklm}[10]{
%\begin{vmatrix}
% {#1}_{#6} & {#1}_{#7} & {#1}_{#8} & {#1}_{#9} & {#1}_{#10} \\
% {#2}_{#6} & {#2}_{#7} & {#2}_{#8} & {#2}_{#9} & {#2}_{#10} \\
% {#3}_{#6} & {#3}_{#7} & {#3}_{#8} & {#3}_{#9} & {#3}_{#10} \\
% {#4}_{#6} & {#4}_{#7} & {#4}_{#8} & {#4}_{#9} & {#4}_{#10} \\
% {#5}_{#6} & {#5}_{#7} & {#5}_{#8} & {#5}_{#9} & {#5}_{#10}
%\end{vmatrix}
%}

% R3 vector.
\newcommand{\VectorThree}[3]{
\begin{bmatrix}
 {#1} \\
 {#2} \\
 {#3}
\end{bmatrix}
}


%<misc>
%
\newcommand{\Abs}[1]{{\left\lvert{#1}\right\rvert}}
\newcommand{\spacegrad}[0]{\boldsymbol{\nabla}}
\newcommand{\grad}[0]{\nabla}
\newcommand{\LL}[0]{\mathcal{L}}

% == \partial_{#1} {#2}
\newcommand{\PD}[2]{\frac{\partial {#2}}{\partial {#1}}}
% inline variant
\newcommand{\PDi}[2]{{\partial {#2}}/{\partial {#1}}}

\newcommand{\PDD}[3]{\frac{\partial^2 {#3}}{\partial {#1}\partial {#2}}}
%\newcommand{\PDd}[2]{\frac{\partial^2 {#2}}{{\partial{#1}}^2}}
\newcommand{\PDsq}[2]{\frac{\partial^2 {#2}}{(\partial {#1})^2}}

\newcommand{\Partial}[2]{\frac{\partial {#1}}{\partial {#2}}}
\DeclareMathOperator{\RejName}{Rej}
\newcommand{\Rej}[2]{\RejName_{#1}\left( {#2} \right)}
\newcommand{\Rm}[1]{\mathbb{R}^{#1}}
\newcommand{\Cm}[1]{\mathbb{C}^{#1}}
\newcommand{\conj}[0]{{*}}

%</misc>

% <grade selection>
%
\newcommand{\gpgrade}[2] {{\left\langle{{#1}}\right\rangle}_{#2}}

\newcommand{\gpgradezero}[1] {\gpgrade{#1}{}}
%\newcommand{\gpscalargrade}[1] {{\left\langle{{#1}}\right\rangle}}
%\newcommand{\gpgradezero}[1] {\gpgrade{#1}{0}}

%\newcommand{\gpgradeone}[1] {{\left\langle{{#1}}\right\rangle}_{1}}
\newcommand{\gpgradeone}[1] {\gpgrade{#1}{1}}

\newcommand{\gpgradetwo}[1] {\gpgrade{#1}{2}}
\newcommand{\gpgradethree}[1] {\gpgrade{#1}{3}}
\newcommand{\gpgradefour}[1] {\gpgrade{#1}{4}}
%
% </grade selection>



\newcommand{\adot}[0]{{\dot{a}}}
\newcommand{\bdot}[0]{{\dot{b}}}
% taken for centered dot:
%\newcommand{\cdot}[0]{{\dot{c}}}
%\newcommand{\ddot}[0]{{\dot{d}}}
\newcommand{\edot}[0]{{\dot{e}}}
\newcommand{\fdot}[0]{{\dot{f}}}
\newcommand{\gdot}[0]{{\dot{g}}}
\newcommand{\hdot}[0]{{\dot{h}}}
\newcommand{\idot}[0]{{\dot{i}}}
\newcommand{\jdot}[0]{{\dot{j}}}
\newcommand{\kdot}[0]{{\dot{k}}}
\newcommand{\ldot}[0]{{\dot{l}}}
\newcommand{\mdot}[0]{{\dot{m}}}
\newcommand{\ndot}[0]{{\dot{n}}}
%\newcommand{\odot}[0]{{\dot{o}}}
\newcommand{\pdot}[0]{{\dot{p}}}
\newcommand{\qdot}[0]{{\dot{q}}}
\newcommand{\rdot}[0]{{\dot{r}}}
\newcommand{\sdot}[0]{{\dot{s}}}
\newcommand{\tdot}[0]{{\dot{t}}}
\newcommand{\udot}[0]{{\dot{u}}}
\newcommand{\vdot}[0]{{\dot{v}}}
\newcommand{\wdot}[0]{{\dot{w}}}
\newcommand{\xdot}[0]{{\dot{x}}}
\newcommand{\ydot}[0]{{\dot{y}}}
\newcommand{\zdot}[0]{{\dot{z}}}
\newcommand{\addot}[0]{{\ddot{a}}}
\newcommand{\bddot}[0]{{\ddot{b}}}
\newcommand{\cddot}[0]{{\ddot{c}}}
%\newcommand{\dddot}[0]{{\ddot{d}}}
\newcommand{\eddot}[0]{{\ddot{e}}}
\newcommand{\fddot}[0]{{\ddot{f}}}
\newcommand{\gddot}[0]{{\ddot{g}}}
\newcommand{\hddot}[0]{{\ddot{h}}}
\newcommand{\iddot}[0]{{\ddot{i}}}
\newcommand{\jddot}[0]{{\ddot{j}}}
\newcommand{\kddot}[0]{{\ddot{k}}}
\newcommand{\lddot}[0]{{\ddot{l}}}
\newcommand{\mddot}[0]{{\ddot{m}}}
\newcommand{\nddot}[0]{{\ddot{n}}}
\newcommand{\oddot}[0]{{\ddot{o}}}
\newcommand{\pddot}[0]{{\ddot{p}}}
\newcommand{\qddot}[0]{{\ddot{q}}}
\newcommand{\rddot}[0]{{\ddot{r}}}
\newcommand{\sddot}[0]{{\ddot{s}}}
\newcommand{\tddot}[0]{{\ddot{t}}}
\newcommand{\uddot}[0]{{\ddot{u}}}
\newcommand{\vddot}[0]{{\ddot{v}}}
\newcommand{\wddot}[0]{{\ddot{w}}}
\newcommand{\xddot}[0]{{\ddot{x}}}
\newcommand{\yddot}[0]{{\ddot{y}}}
\newcommand{\zddot}[0]{{\ddot{z}}}

%<bold and dot greek symbols>
%

\newcommand{\Deltadot}[0]{{\dot{\Delta}}}
\newcommand{\Gammadot}[0]{{\dot{\Gamma}}}
\newcommand{\Lambdadot}[0]{{\dot{\Lambda}}}
\newcommand{\Omegadot}[0]{{\dot{\Omega}}}
\newcommand{\Phidot}[0]{{\dot{\Phi}}}
\newcommand{\Pidot}[0]{{\dot{\Pi}}}
\newcommand{\Psidot}[0]{{\dot{\Psi}}}
\newcommand{\Sigmadot}[0]{{\dot{\Sigma}}}
\newcommand{\Thetadot}[0]{{\dot{\Theta}}}
\newcommand{\Upsilondot}[0]{{\dot{\Upsilon}}}
\newcommand{\Xidot}[0]{{\dot{\Xi}}}
\newcommand{\alphadot}[0]{{\dot{\alpha}}}
\newcommand{\betadot}[0]{{\dot{\beta}}}
\newcommand{\chidot}[0]{{\dot{\chi}}}
\newcommand{\deltadot}[0]{{\dot{\delta}}}
\newcommand{\epsilondot}[0]{{\dot{\epsilon}}}
\newcommand{\etadot}[0]{{\dot{\eta}}}
\newcommand{\gammadot}[0]{{\dot{\gamma}}}
\newcommand{\kappadot}[0]{{\dot{\kappa}}}
\newcommand{\lambdadot}[0]{{\dot{\lambda}}}
\newcommand{\mudot}[0]{{\dot{\mu}}}
\newcommand{\nudot}[0]{{\dot{\nu}}}
\newcommand{\omegadot}[0]{{\dot{\omega}}}
\newcommand{\phidot}[0]{{\dot{\phi}}}
\newcommand{\pidot}[0]{{\dot{\pi}}}
\newcommand{\psidot}[0]{{\dot{\psi}}}
\newcommand{\rhodot}[0]{{\dot{\rho}}}
\newcommand{\sigmadot}[0]{{\dot{\sigma}}}
\newcommand{\taudot}[0]{{\dot{\tau}}}
\newcommand{\thetadot}[0]{{\dot{\theta}}}
\newcommand{\upsilondot}[0]{{\dot{\upsilon}}}
\newcommand{\varepsilondot}[0]{{\dot{\varepsilon}}}
\newcommand{\varphidot}[0]{{\dot{\varphi}}}
\newcommand{\varpidot}[0]{{\dot{\varpi}}}
\newcommand{\varrhodot}[0]{{\dot{\varrho}}}
\newcommand{\varsigmadot}[0]{{\dot{\varsigma}}}
\newcommand{\varthetadot}[0]{{\dot{\vartheta}}}
\newcommand{\xidot}[0]{{\dot{\xi}}}
\newcommand{\zetadot}[0]{{\dot{\zeta}}}

\newcommand{\Deltaddot}[0]{{\ddot{\Delta}}}
\newcommand{\Gammaddot}[0]{{\ddot{\Gamma}}}
\newcommand{\Lambdaddot}[0]{{\ddot{\Lambda}}}
\newcommand{\Omegaddot}[0]{{\ddot{\Omega}}}
\newcommand{\Phiddot}[0]{{\ddot{\Phi}}}
\newcommand{\Piddot}[0]{{\ddot{\Pi}}}
\newcommand{\Psiddot}[0]{{\ddot{\Psi}}}
\newcommand{\Sigmaddot}[0]{{\ddot{\Sigma}}}
\newcommand{\Thetaddot}[0]{{\ddot{\Theta}}}
\newcommand{\Upsilonddot}[0]{{\ddot{\Upsilon}}}
\newcommand{\Xiddot}[0]{{\ddot{\Xi}}}
\newcommand{\alphaddot}[0]{{\ddot{\alpha}}}
\newcommand{\betaddot}[0]{{\ddot{\beta}}}
\newcommand{\chiddot}[0]{{\ddot{\chi}}}
\newcommand{\deltaddot}[0]{{\ddot{\delta}}}
\newcommand{\epsilonddot}[0]{{\ddot{\epsilon}}}
\newcommand{\etaddot}[0]{{\ddot{\eta}}}
\newcommand{\gammaddot}[0]{{\ddot{\gamma}}}
\newcommand{\kappaddot}[0]{{\ddot{\kappa}}}
\newcommand{\lambdaddot}[0]{{\ddot{\lambda}}}
\newcommand{\muddot}[0]{{\ddot{\mu}}}
\newcommand{\nuddot}[0]{{\ddot{\nu}}}
\newcommand{\omegaddot}[0]{{\ddot{\omega}}}
\newcommand{\phiddot}[0]{{\ddot{\phi}}}
\newcommand{\piddot}[0]{{\ddot{\pi}}}
\newcommand{\psiddot}[0]{{\ddot{\psi}}}
\newcommand{\rhoddot}[0]{{\ddot{\rho}}}
\newcommand{\sigmaddot}[0]{{\ddot{\sigma}}}
\newcommand{\tauddot}[0]{{\ddot{\tau}}}
\newcommand{\thetaddot}[0]{{\ddot{\theta}}}
\newcommand{\upsilonddot}[0]{{\ddot{\upsilon}}}
\newcommand{\varepsilonddot}[0]{{\ddot{\varepsilon}}}
\newcommand{\varphiddot}[0]{{\ddot{\varphi}}}
\newcommand{\varpiddot}[0]{{\ddot{\varpi}}}
\newcommand{\varrhoddot}[0]{{\ddot{\varrho}}}
\newcommand{\varsigmaddot}[0]{{\ddot{\varsigma}}}
\newcommand{\varthetaddot}[0]{{\ddot{\vartheta}}}
\newcommand{\xiddot}[0]{{\ddot{\xi}}}
\newcommand{\zetaddot}[0]{{\ddot{\zeta}}}

\newcommand{\BDelta}[0]{\boldsymbol{\Delta}}
\newcommand{\BGamma}[0]{\boldsymbol{\Gamma}}
\newcommand{\BLambda}[0]{\boldsymbol{\Lambda}}
\newcommand{\BOmega}[0]{\boldsymbol{\Omega}}
\newcommand{\BPhi}[0]{\boldsymbol{\Phi}}
\newcommand{\BPi}[0]{\boldsymbol{\Pi}}
\newcommand{\BPsi}[0]{\boldsymbol{\Psi}}
\newcommand{\BSigma}[0]{\boldsymbol{\Sigma}}
\newcommand{\BTheta}[0]{\boldsymbol{\Theta}}
\newcommand{\BUpsilon}[0]{\boldsymbol{\Upsilon}}
\newcommand{\BXi}[0]{\boldsymbol{\Xi}}
\newcommand{\Balpha}[0]{\boldsymbol{\alpha}}
\newcommand{\Bbeta}[0]{\boldsymbol{\beta}}
\newcommand{\Bchi}[0]{\boldsymbol{\chi}}
\newcommand{\Bdelta}[0]{\boldsymbol{\delta}}
\newcommand{\Bepsilon}[0]{\boldsymbol{\epsilon}}
\newcommand{\Beta}[0]{\boldsymbol{\eta}}
\newcommand{\Bgamma}[0]{\boldsymbol{\gamma}}
\newcommand{\Bkappa}[0]{\boldsymbol{\kappa}}
\newcommand{\Blambda}[0]{\boldsymbol{\lambda}}
\newcommand{\Bmu}[0]{\boldsymbol{\mu}}
\newcommand{\Bnu}[0]{\boldsymbol{\nu}}
%\newcommand{\Bomega}[0]{\boldsymbol{\omega}}
\newcommand{\Bphi}[0]{\boldsymbol{\phi}}
\newcommand{\Bpi}[0]{\boldsymbol{\pi}}
\newcommand{\Bpsi}[0]{\boldsymbol{\psi}}
\newcommand{\Brho}[0]{\boldsymbol{\rho}}
\newcommand{\Bsigma}[0]{\boldsymbol{\sigma}}
%\newcommand{\Btau}[0]{\boldsymbol{\tau}}
%\newcommand{\Btheta}[0]{\boldsymbol{\theta}}
\newcommand{\Bupsilon}[0]{\boldsymbol{\upsilon}}
\newcommand{\Bvarepsilon}[0]{\boldsymbol{\varepsilon}}
\newcommand{\Bvarphi}[0]{\boldsymbol{\varphi}}
\newcommand{\Bvarpi}[0]{\boldsymbol{\varpi}}
\newcommand{\Bvarrho}[0]{\boldsymbol{\varrho}}
\newcommand{\Bvarsigma}[0]{\boldsymbol{\varsigma}}
\newcommand{\Bvartheta}[0]{\boldsymbol{\vartheta}}
\newcommand{\Bxi}[0]{\boldsymbol{\xi}}
\newcommand{\Bzeta}[0]{\boldsymbol{\zeta}}
%
%</bold and dot greek symbols>
%<infrequent>
%
%\newcommand{\AreaOp}[1]{\AName_{#1}}
%\newcommand{\Babs}[0]{\abs{\BB}}
%\newcommand{\Bcap}[0]{\hat{\BB}}
%\newcommand{\BrPrimeRej}[0]{\rcap(\rcap \wedge \Br')}
%\newcommand{\CA}[0]{\mathcal{A}}
%\newcommand{\Cos}[1]{\cos{\left({#1}\right)}}
%\newcommand{\Det}[1] {\abs{#1}}
%\newcommand{\Dsq}[2] {\frac {\partial^2 {#1}} {\partial {#2}^2}}
%\newcommand{\Exp}[1]{\exp{\left({#1}\right)}}
%\newcommand{\Norm}[1]{\left\lVert{#1}\right\rVert}
%\newcommand{\Sin}[1]{\sin{\left({#1}\right)}}
%\newcommand{\T}[0]{\text{T}}
%\newcommand{\VolumeOp}[1]{\VName_{#1}}
%\newcommand{\agrad}[0]{\Ba \cdot \nabla}
%\newcommand{\alphacap}[0]{\hat{\boldsymbol{\alpha}}}
%\newcommand{\Fcap}[0]{\hat{\BF}}
%\newcommand{\bithree}[0]{{\Bi}_3}
%\newcommand{\bxa}[0]{\Bx\Ba}
%\newcommand{\coordvec}[2]{
%\newcommand{\costheta}[0]{\acap \cdot \xcap}
%\newcommand{\ddt}[1]{\ddot{#1}}
%\newcommand{\ddu}[1] {\frac {d{#1}} {du}}
%\newcommand{\dsqxj}[2] {\frac {\partial^2 {#1}} {\partial {x_{#2}}^2}}
%\newcommand{\dtheta}[1]{\frac{d {#1}}{d \theta}}
%\newcommand{\dt}[1]{\dot{#1}}
%\newcommand{\dt}[1]{\frac{d {#1}}{dt}}
%\newcommand{\dxj}[2] {\frac {\partial {#1}} {\partial {x_{#2}}}}
%\newcommand{\halfPhi}[0]{\frac{\phi}{2}}
%\newcommand{\half}[0]{\inv{2}}
%\newcommand{\inv}[1]{\frac{1}{#1}}
%\newcommand{\laplacian}[0]{\nabla^2}
%\newcommand{\matrixoftx}[3]{
%\newcommand{\nrrp}[0]{\norm{\rcap \wedge \Br'}}
%\newcommand{\oiint}{\bigcirc \hspace{-1.4em} \int \hspace{-.8em} \int}
%\newcommand{\transpose}[1]{{#1}^{\text{T}}}
%\newcommand{\transpose}[1]{{{#1}^{\TextTranspose}}}
%\newcommand{\transpose}[1]{{{#1}^{\text{T}}}}
%\newcommand{\barA}[0]{\bar{A}}
%\newcommand{\qbar}[0]{\bar{q}}
%\newcommand{\qdotbar}[0]{\dot{\bar{q}}}
%
%</infrequent>





\newcommand{\EE}[0]{\boldsymbol{\mathcal{E}}}
\newcommand{\HH}[0]{\boldsymbol{\mathcal{H}}}

\usepackage[bookmarks=true]{hyperref}

\usepackage{color,cite,graphicx}
   % use colour in the document, put your citations as [1-4]
   % rather than [1,2,3,4] (it looks nicer, and the extended LaTeX2e
   % graphics package. 
\usepackage{latexsym,amssymb,epsf} % don't remember if these are
   % needed, but their inclusion can't do any damage


\title{ Auxillary notes for Bohm's Rayleigh-Jeans QM intro topic. }
\author{Peeter Joot}
\date{ Dec 27, 2008.  Last Revision: $Date: 2008/12/29 03:23:22 $ }

\begin{document}
\maketitle{}

%\tableofcontents

\section{ Motivation. }

Fill in the gaps for a reading of the
initial parts of the Rayleigh-Jeans discussion of \cite{bohm1989qt}.

\section{ 2. Electromagnetic energy. }

Energy of the field given to be:

\begin{align*}
E = \inv{8\pi} \int (\EE + \HH)
\end{align*}

I still don't really know where this comes from.
Could perhaps justify this with a Hamiltonian of a field (although this is
uncomfortably abstract).

With the particle Hamiltonian we have

\begin{align*}
H = \qdot_i p_i -\LL
\end{align*}

What is the field equivalent of this?  Try to get the feel for this with some simple fields (such as the one dimensional vibrating string), and the Coulomb field.  For the physical case, do this with both the Hamitonian approach and a physical limiting argument.

\section{ 3. Electromagnetic Potentials. }

Bohm writes maxwell's equations in non-SI units, and also, naturally, not in STA form which would be somewhat more natural for a gauge
discussion.

\begin{align*}
\spacegrad \cross \EE &= -\inv{c} \partial_t \HH \\
\spacegrad \cdot \EE &= 4 \pi \rho \\
\spacegrad \cross \HH &= \inv{c} \partial_t \EE + 4 \pi \Bj \\
\spacegrad \cdot \HH &= 0
\end{align*}

In STA form this is

\begin{align*}
\spacegrad \EE &= - \partial_0 i\HH + 4 \pi \rho \\
\spacegrad i\HH &= -\partial_0 \EE - 4 \pi \Bj \\
\end{align*}

Or
\begin{align}\label{eqn:maxwellNotQuiteCovariant}
\spacegrad (\EE + i\HH) + \partial_0 (\EE + i\HH) &= 4 \pi (\rho - \Bj) \\
\end{align}

Left multiplying by $\gamma_0$ gives

\begin{align*}
\gamma_0 \spacegrad 
&= \gamma_0 \sum_k \sigma_k \partial_k \\
&= \gamma_0 \sum_k \gamma_k \gamma_0 \partial_k \\
&= -\sum_k \gamma_k \partial_k \\
&= \gamma^k \partial_k \\
\end{align*}

and

\begin{align*}
\gamma_0 \Bj 
&= \sum_k \gamma_0 \sigma_k j^k \\
&= -\sum_k \gamma_k j^k,
\end{align*}

so with $J^0 = \rho$, $J^k = j^k$ and $J = \gamma_\mu J^\mu$, we have

\begin{align*}
\gamma^\mu \partial_\mu (\EE + i\HH) &= 4 \pi J
\end{align*}

and finally with $F = \EE + i\HH$, we have Maxwell's equation in covariant form

\begin{align*}
\grad F &= 4 \pi J.
\end{align*}

Next it is stated that general solutions can be expressed as

\begin{align}\label{eqn:fieldsFromPotentials}
\HH &= \spacegrad \cross \Ba \\
\EE &= - \inv{c} \PD{t}{\Ba} - \spacegrad \phi
\end{align}

Let's double check that this jives with the bivector potential solution $F = \grad \wedge A = \EE + i\HH$.  Let's split our bivector
into spacetime and spatial components by the conjugate operation

\begin{align*}
F^\dagger &=\gamma_0 F \gamma_0 \\
&= \gamma_0 \gamma^\mu \wedge \gamma^\nu \partial_\mu A_\mu \gamma_0 \\
&=
\left\{
\begin{array}{l l}
0 & \quad \mbox{if $\mu = \nu$} \\
\gamma^\mu \gamma^\nu \partial_\mu A_\nu & \quad \mbox{if $\mu \in \{1,2,3\}$, and $\nu \in \{1,2,3\}$} \\
-\gamma^\mu \gamma^\nu \partial_\mu A_\nu & \quad \mbox{one of $\mu = 0$ or $\nu = 0$ } \\
\end{array} \right.
\end{align*}

\begin{align*}
F 
&= \EE + i\HH \\
&= \inv{2}(F - F^\dagger) + \inv{2}(F + F^\dagger) \\
&= \left(\gamma^k \wedge \gamma^0 \partial_k A_0 +\gamma^0 \wedge \gamma^k \partial_0 A_k\right) + \left(\gamma^a \wedge \gamma^b \partial_a A_b\right) \\
&= -\left(\sum_k \sigma_k \partial_k A^0 + \partial_0 \sigma_k A^k\right) + i\left(\epsilon_{abc}\sigma_a \partial_b A^c\right) \\
\end{align*}

So, with $\Ba = \sigma_k A^k$, and $\phi = A^0$, we do have equations \ref{eqn:fieldsFromPotentials} as identical to $F = \grad \wedge A$.

Now how about the gauge variations of the fields?  Bohm writes that we can alter the potentials by

\begin{align}\label{eqn:gauge}
\Ba' &= \Ba - \spacegrad \psi \\
\phi' &= \phi + \inv{c}\PD{t}{\psi}
\end{align}

How does this translate to an alteration of the four potential?  For the vector potential we have

\begin{align*}
\sigma_k {A^k}' &= \sigma_k A^k - \sigma_k \partial \psi \\
\gamma_k \gamma_0 {A^k}' &= \gamma_k \gamma_0 A^k - \gamma_k \gamma_0 \partial_k \psi \\
-\gamma_0 \gamma_k {A^k}' &= -\gamma_0 \gamma_k A^k - \gamma_0 \gamma^k \partial_k \psi \\
\gamma_k {A^k}' &= \gamma_k A^k + \gamma^k \partial_k \psi \\
\end{align*}

with $\phi = A^0$, add in the $\phi$ term

\begin{align*}
\gamma_0 \phi' &= \gamma_0 \phi + \gamma_0 \PD{x^0}{\psi} \\
\gamma_0 \phi' &= \gamma_0 \phi + \gamma^0 \PD{x^0}{\psi}
\end{align*}

For
\begin{align*}
\gamma_\mu {A^\mu}' &= \gamma_\mu A^\mu + \gamma^\mu \partial_\mu \psi \\
\end{align*}

Which is just a statement that we can add a spacetime gradient to our vector potential without altering the field equation:

\begin{align*}
A' &= A + \grad \psi
\end{align*}

Let's verify that this does in fact not alter Maxwell's equation.

\begin{align*}
\grad (\grad \wedge (A + \grad \psi) &= 4 \pi J
\grad (\grad \wedge A) + \grad (\grad \wedge \grad \psi) &= 
\end{align*}

Since $\grad \wedge \grad = 0$ we have

\begin{align*}
\grad (\grad \wedge A') = \grad (\grad \wedge A)
\end{align*}

Now the statement that $\grad \wedge \grad$ as an operator equals zero, just by virtue of $\grad$ being a vector is worth explicit
confirmation.  Let's expand that to verify

\begin{align*}
\grad \wedge \grad \psi 
&= \gamma^\mu \wedge \gamma^\nu \partial_\mu \partial_\nu \psi \\
&= \left(\sum_{\mu < \nu} + \sum_{\nu < \mu}\right) \gamma^\mu \wedge \gamma^\nu \partial_\mu \partial_\nu \psi \\
&= \sum_{\mu < \nu} \gamma^\mu \wedge \gamma^\nu (\partial_\mu \partial_\nu \psi - \partial_\nu \partial_\mu \psi) \\
\end{align*}

So, we see that we additionally need the field variable $\psi$ to be sufficently continuous for mixed partial equality for the
statement that $\grad \wedge \grad = 0$ to be valid.  Assuming that continuity is taken as a given the confirmation of the invariance under this transformation is thus complete.

Now, Bohm says it is possible to pick $\spacegrad \cdot \Ba' = 0$.  From \ref{eqn:gauge} that implies

\begin{align*}
\spacegrad \cdot \Ba'
&= \spacegrad \cdot \Ba - \spacegrad \cdot \spacegrad \psi \\
&= \spacegrad \cdot \Ba - \spacegrad^2 \psi = 0 \\
\end{align*}

So, provided we can find a solution to the Poisson equation

\begin{align}\label{eqn:psiForDivAPrimeEqZero}
\spacegrad^2 \psi = \spacegrad \cdot \Ba
\end{align}

one can find a $\psi, \Ba$ gauge transformation that has the particular quality that $\spacegrad \cdot \Ba' = 0$.

That solution, from equation \ref{eqn:laplacianOfPoisson} is

\begin{align*}
\psi(\Br) = -\inv{4\pi}\int (\spacegrad' \cdot \Ba(\Br')) dV' \inv{\Abs{\Br-\Br'}}
\end{align*}

The corrallary to this
is that one may similarily impose a requirement that $\spacegrad \cdot \Ba = 0$, since if that is not the case, some $\Ba'$ can be added to the vector potential to make that the case.

FIXME: handwaving description here.  Show with a math statement with $\Ba \rightarrow \Ba'$.

\subsection{ Free space solutions. }

From \ref{eqn:maxwellNotQuiteCovariant} and \ref{eqn:fieldsFromPotentials}
the free space solution to Maxwell's
equation must satisfy 

\begin{align*}
0 
&= \left(\spacegrad + \partial_0\right) (\EE + i\HH) \\
&= \left(\spacegrad + \partial_0\right) \left(- \partial_0{\Ba} - \spacegrad \phi + \spacegrad \wedge \Ba \right) \\
&= - \spacegrad \partial_0{\Ba} - \spacegrad^2 \phi + \spacegrad (\spacegrad \wedge \Ba) 
 - \partial_{00}{\Ba} - \partial_0 \spacegrad \phi + \partial_0 (\spacegrad \wedge \Ba) \\
&= - \spacegrad \cdot \partial_0{\Ba} - \spacegrad^2 \phi + \spacegrad \cdot (\spacegrad \wedge \Ba) 
 - \partial_{00}{\Ba} - \partial_0 \spacegrad \phi  \\
\end{align*}

Since the scalar and vector parts of this equation must separately equal zero we have

\begin{align*}
0 &= - \partial_0 \spacegrad \cdot {\Ba} - \spacegrad^2 \phi \\
0 &= \spacegrad \cdot (\spacegrad \wedge \Ba) - \partial_{00}{\Ba} - \partial_0 \spacegrad \phi  \\
\end{align*}

If one picks a gauge transformation such that $\spacegrad \cdot \Ba = 0$ we then have
\begin{align*}
0 &= \spacegrad^2 \phi \\
0 &= \spacegrad^2 \Ba - \partial_{00}{\Ba} - \partial_0 \spacegrad \phi  \\
\end{align*}

For the first Bohm argues that ``It is well known that the only solution of this equation that is regular over all space is $\phi = 0$'', and anything else implies charge in the region.  What does regular mean here?  I suppose this seems like a reasonable enough statement, but I think the proper way to think about this is really that one has picked the covariant gauge $\grad \cdot A = 0$ (that's simpler anyhow).  With an acceptance of the $\phi =0$ argument one is left with the vector potential wave equation which was the desired goal of that section.

\subsection{ Doing this all directly. }

Now, the whole point of the gauge transformation appears to be to show that one can find the four wave equation solutions for 
Maxwell's equation by picking a specific gauge.  This is actually trivial to do from the STA Maxwell equation:

\begin{align*}
\grad (\grad \wedge A) = \grad( \grad A - \grad \cdot A ) = \grad^2 A - \grad (\grad \cdot A) = 4 \pi J
\end{align*}

So, if one picks a gauge transformation with $\grad \cdot A = 0$, one has

\begin{align*}
\grad^2 A = 4 \pi J
\end{align*}

This is precisely the four wave equations desired
\begin{align*}
\partial_\nu\partial^\nu A^\mu = 4 \pi J^\mu
\end{align*}

FIXME: show the precise gauge transformation $A \rightarrow A'$ that leads to $\grad \cdot A = 0$.

\section{ Auxillary details. }

\subsection{ Confirm Poisson solution to Laplacian. }

Bohm lists the solution for \ref{eqn:psiForDivAPrimeEqZero} (a Poisson integral), but I forget how one shows this.  I can't figure out how to integrate this Laplacian, but it is simple enough to confirm this by back substuition.

Suppose one has

\begin{align*}
\psi = \int \frac{\rho(\Br')}{\Abs{\Br - \Br'}} dV'
\end{align*}

We can take the Laplacian by direct differentiation under the integration sign

\begin{align*}
\spacegrad^2 \psi = \int {\rho(\Br')} dV' \spacegrad^2 \inv{\Abs{\Br - \Br'}}
\end{align*}

To evaluate the Laplacian we need

\begin{align*}
\PD{x_i}{\Abs{\Br - \Br'}^k} 
&= \PD{x_i}{} \left(\sum_j (x_j - x_j')^2 \right)^{k/2} \\
&= k { 2\Abs{\Br - \Br'}^{k-2}} \PD{x_i}{} \left(\sum_j (x_j - x_j')^2 \right) \\
%&= k { 2\Abs{\Br - \Br'}^{k-2}} 2 (x_i - x_i')
&= k { \Abs{\Br - \Br'}^{k-2}} (x_i - x_i')
\end{align*}

So we have
\begin{align*}
\PD{x_i}{} \PD{x_i}{} {\Abs{\Br - \Br'}^{-1}} 
&= 
- (x_i - x_i') \PD{x_i}{}{ \inv{ \Abs{\Br - \Br'}^3} } 
- \inv{ \Abs{\Br - \Br'}^3} \PD{x_i}{ (x_i - x_i') } \\
&= 
3 (x_i - x_i')^2 { \inv{ \Abs{\Br - \Br'}^5} } 
- \inv{ \Abs{\Br - \Br'}^3} \\
\end{align*}

So, provided $\Br \ne \Br'$ we have

\begin{align*}
\spacegrad^2 \psi &= 
3 (\Br - \Br')^2 \inv{ \Abs{\Br - \Br'}^5} 
- 3 \inv{ \Abs{\Br - \Br'}^3} \\
&= 0 
\end{align*}

Observe that this is true only for \R{3}.  Now, one is left with only an integral around a neighbourhood around the point $\Br$ which can be made small enough that $\rho(\Br') = \rho(\Br)$ in that volume can be taken as constant.

\begin{align*}
\spacegrad^2 \psi 
&= \rho(\Br) \int dV' \spacegrad^2 \inv{\Abs{\Br - \Br'}} \\
&= \rho(\Br) \int dV' \spacegrad \cdot \spacegrad \inv{\Abs{\Br - \Br'}} \\
&= -\rho(\Br) \int dV' \spacegrad \cdot \frac{(\Br -\Br')}{\Abs{\Br - \Br'}^3} \\
\end{align*}

Now, if the divergence in this integral was with respect to the primed variable that ranges over the infinitesimal volume, then this could be converted to a surface integral.  
Observe that a radial expansion of this divergence allows for convient change of variables to the primed $x_i'$ coordinates

\begin{align*}
\spacegrad \cdot \frac{(\Br -\Br')}{\Abs{\Br - \Br'}^3}
&= 
\left(\frac{\Br - \Br'}{\Abs{\Br-\Br'}} \PD{\Abs{\Br-\Br'}}{}\right) \cdot
\left(\frac{\Br - \Br'}{\Abs{\Br-\Br'}} \inv{\Abs{\Br-\Br'}^2}\right) \\
&= 
\PD{\Abs{\Br'-\Br}}{} {\Abs{\Br'-\Br}^{-2}} \\
&= 
\left(\frac{\Br' - \Br}{\Abs{\Br'-\Br}} \PD{\Abs{\Br'-\Br}}{}\right) \cdot
\left(\frac{\Br' - \Br}{\Abs{\Br'-\Br}} \inv{\Abs{\Br'-\Br}^2}\right) \\
&= \spacegrad' \cdot \frac{(\Br'-\Br)}{\Abs{\Br' - \Br}^3}
\end{align*}

Now, since $\Br'-\Br$ is in the direction of the outwards normal the divergence theorem can be used

\begin{align*}
\spacegrad^2 \psi 
&= -\rho(\Br) \int dV' \spacegrad' \cdot \frac{(\Br' -\Br)}{\Abs{\Br' - \Br}^3} \\
&= -\rho(\Br) \int_{\partial V'} dA' \inv{\Abs{\Br' - \Br}^2} \\
\end{align*}

Picking a spherical integration volume, for which the radius is constant $R = \Abs{\Br'-\Br}$, we have

\begin{align*}
\spacegrad^2 \psi 
&= -\rho(\Br) 4 \pi R^2 \inv{R^2} \\
\end{align*}

In summary this is

\begin{align}\label{eqn:laplacianOfPoisson}
\psi &= \int \frac{\rho(\Br')}{\Abs{\Br - \Br'}} dV' \\
\spacegrad^2 \psi &= - 4 \pi \rho(\Br)
\end{align}

Having written this out I recall that the same approach was used in
\cite{schwartz1987pe} (there it was to calculate $\spacegrad \cdot \BE$ in terms of the charge density, but the ideas are all the same.)

\bibliographystyle{plainnat}
\bibliography{myrefs}

\end{document}
