\documentclass{article}      % Specifies the document class

\usepackage{amsmath}
\usepackage{mathpazo}

%
% shorthand for bold symbols, convenient for vectors and matrices
%
\newcommand{\Ba}[0]{\mathbf{a}}
\newcommand{\Bb}[0]{\mathbf{b}}
\newcommand{\Bc}[0]{\mathbf{c}}
\newcommand{\Bd}[0]{\mathbf{d}}
\newcommand{\Be}[0]{\mathbf{e}}
\newcommand{\Bf}[0]{\mathbf{f}}
\newcommand{\Bg}[0]{\mathbf{g}}
\newcommand{\Bh}[0]{\mathbf{h}}
\newcommand{\Bi}[0]{\mathbf{i}}
\newcommand{\Bj}[0]{\mathbf{j}}
\newcommand{\Bk}[0]{\mathbf{k}}
\newcommand{\Bl}[0]{\mathbf{l}}
\newcommand{\Bm}[0]{\mathbf{m}}
\newcommand{\Bn}[0]{\mathbf{n}}
\newcommand{\Bo}[0]{\mathbf{o}}
\newcommand{\Bp}[0]{\mathbf{p}}
\newcommand{\Bq}[0]{\mathbf{q}}
\newcommand{\Br}[0]{\mathbf{r}}
\newcommand{\Bs}[0]{\mathbf{s}}
\newcommand{\Bt}[0]{\mathbf{t}}
\newcommand{\Bu}[0]{\mathbf{u}}
\newcommand{\Bv}[0]{\mathbf{v}}
\newcommand{\Bw}[0]{\mathbf{w}}
\newcommand{\Bx}[0]{\mathbf{x}}
\newcommand{\By}[0]{\mathbf{y}}
\newcommand{\Bz}[0]{\mathbf{z}}
\newcommand{\BA}[0]{\mathbf{A}}
\newcommand{\BB}[0]{\mathbf{B}}
\newcommand{\BC}[0]{\mathbf{C}}
\newcommand{\BD}[0]{\mathbf{D}}
\newcommand{\BE}[0]{\mathbf{E}}
\newcommand{\BF}[0]{\mathbf{F}}
\newcommand{\BG}[0]{\mathbf{G}}
\newcommand{\BH}[0]{\mathbf{H}}
\newcommand{\BI}[0]{\mathbf{I}}
\newcommand{\BJ}[0]{\mathbf{J}}
\newcommand{\BK}[0]{\mathbf{K}}
\newcommand{\BL}[0]{\mathbf{L}}
\newcommand{\BM}[0]{\mathbf{M}}
\newcommand{\BN}[0]{\mathbf{N}}
\newcommand{\BO}[0]{\mathbf{O}}
\newcommand{\BP}[0]{\mathbf{P}}
\newcommand{\BQ}[0]{\mathbf{Q}}
\newcommand{\BR}[0]{\mathbf{R}}
\newcommand{\BS}[0]{\mathbf{S}}
\newcommand{\BT}[0]{\mathbf{T}}
\newcommand{\BU}[0]{\mathbf{U}}
\newcommand{\BV}[0]{\mathbf{V}}
\newcommand{\BW}[0]{\mathbf{W}}
\newcommand{\BX}[0]{\mathbf{X}}
\newcommand{\BY}[0]{\mathbf{Y}}
\newcommand{\BZ}[0]{\mathbf{Z}}

\newcommand{\Bzero}[0]{\mathbf{0}}
\newcommand{\Btheta}[0]{\boldsymbol{\theta}}
\newcommand{\Btau}[0]{\boldsymbol{\tau}}
\newcommand{\Bomega}[0]{\boldsymbol{\omega}}

%
% shorthand for unit vectors
%
\newcommand{\acap}[0]{\hat{\Ba}}
\newcommand{\bcap}[0]{\hat{\Bb}}
\newcommand{\ccap}[0]{\hat{\Bc}}
\newcommand{\dcap}[0]{\hat{\Bd}}
\newcommand{\ecap}[0]{\hat{\Be}}
\newcommand{\fcap}[0]{\hat{\Bf}}
\newcommand{\gcap}[0]{\hat{\Bg}}
\newcommand{\hcap}[0]{\hat{\Bh}}
\newcommand{\icap}[0]{\hat{\Bi}}
\newcommand{\jcap}[0]{\hat{\Bj}}
\newcommand{\kcap}[0]{\hat{\Bk}}
\newcommand{\lcap}[0]{\hat{\Bl}}
\newcommand{\mcap}[0]{\hat{\Bm}}
\newcommand{\ncap}[0]{\hat{\Bn}}
\newcommand{\ocap}[0]{\hat{\Bo}}
\newcommand{\pcap}[0]{\hat{\Bp}}
\newcommand{\qcap}[0]{\hat{\Bq}}
\newcommand{\rcap}[0]{\hat{\Br}}
\newcommand{\scap}[0]{\hat{\Bs}}
\newcommand{\tcap}[0]{\hat{\Bt}}
\newcommand{\ucap}[0]{\hat{\Bu}}
\newcommand{\vcap}[0]{\hat{\Bv}}
\newcommand{\wcap}[0]{\hat{\Bw}}
\newcommand{\xcap}[0]{\hat{\Bx}}
\newcommand{\ycap}[0]{\hat{\By}}
\newcommand{\zcap}[0]{\hat{\Bz}}
\newcommand{\thetacap}[0]{\hat{\Btheta}}

%
% to write R^n and C^n in a distinguishable fashion.  Perhaps change this
% to the double lined characters upon figuring out how to do so.
%
\newcommand{\C}[1]{$\mathbb{C}^{#1}$}
\newcommand{\R}[1]{$\mathbb{R}^{#1}$}

%
% various generally useful helpers
%

% derivative of #1 wrt. #2:
\newcommand{\D}[2] {\frac {d#2} {d#1}}

\newcommand{\inv}[1]{\frac{1}{#1}}
\newcommand{\cross}[0]{\times}

\newcommand{\abs}[1]{\lvert{#1}\rvert}
\newcommand{\norm}[1]{\lVert{#1}\rVert}
\newcommand{\innerprod}[2]{\langle{#1}, {#2}\rangle}
\newcommand{\dotprod}[2]{{#1} \cdot {#2}}
\newcommand{\bdotprod}[2]{\left({#1} \cdot {#2}\right)}
\newcommand{\crossprod}[2]{{#1} \cross {#2}}
\newcommand{\tripleprod}[3]{\dotprod{\left(\crossprod{#1}{#2}\right)}{#3}}

\DeclareMathOperator{\Proj}{Proj}
\DeclareMathOperator{\Span}{span}
\DeclareMathOperator{\Sgn}{sgn}
\DeclareMathOperator{\Area}{Area}
\DeclareMathOperator{\Volume}{Volume}

%
% A few miscellaneous things specific to this document
%
\newcommand{\crossop}[1]{\crossprod{#1}{}}

% R2 vector.
\newcommand{\VectorTwo}[2]{
\begin{bmatrix}
 {#1} \\
 {#2}
\end{bmatrix}
}

\newcommand{\VectorN}[1]{
\begin{bmatrix}
{#1}_1 \\
{#1}_2 \\
\vdots \\
{#1}_N \\
\end{bmatrix}
}

\newcommand{\DETuvij}[4]{
\begin{vmatrix}
 {#1}_{#3} & {#1}_{#4} \\
 {#2}_{#3} & {#2}_{#4}
\end{vmatrix}
}

\newcommand{\DETuvwijk}[6]{
\begin{vmatrix}
 {#1}_{#4} & {#1}_{#5} & {#1}_{#6} \\
 {#2}_{#4} & {#2}_{#5} & {#2}_{#6} \\
 {#3}_{#4} & {#3}_{#5} & {#3}_{#6}
\end{vmatrix}
}

\newcommand{\DETuvwxijkl}[8]{
\begin{vmatrix}
 {#1}_{#5} & {#1}_{#6} & {#1}_{#7} & {#1}_{#8} \\
 {#2}_{#5} & {#2}_{#6} & {#2}_{#7} & {#2}_{#8} \\
 {#3}_{#5} & {#3}_{#6} & {#3}_{#7} & {#3}_{#8} \\
 {#4}_{#5} & {#4}_{#6} & {#4}_{#7} & {#4}_{#8} \\
\end{vmatrix}
}

%\newcommand{\DETuvwxyijklm}[10]{
%\begin{vmatrix}
% {#1}_{#6} & {#1}_{#7} & {#1}_{#8} & {#1}_{#9} & {#1}_{#10} \\
% {#2}_{#6} & {#2}_{#7} & {#2}_{#8} & {#2}_{#9} & {#2}_{#10} \\
% {#3}_{#6} & {#3}_{#7} & {#3}_{#8} & {#3}_{#9} & {#3}_{#10} \\
% {#4}_{#6} & {#4}_{#7} & {#4}_{#8} & {#4}_{#9} & {#4}_{#10} \\
% {#5}_{#6} & {#5}_{#7} & {#5}_{#8} & {#5}_{#9} & {#5}_{#10}
%\end{vmatrix}
%}

% R3 vector.
\newcommand{\VectorThree}[3]{
\begin{bmatrix}
 {#1} \\
 {#2} \\
 {#3}
\end{bmatrix}
}



\newcommand{\Fcap}[0]{\hat{\BF}}
\newcommand{\bithree}[0]{{\Bi}_3}
\newcommand{\nrrp}[0]{\norm{\rcap \wedge \Br'}}

%
% The real thing:
%

                             % The preamble begins here.
\title{Torque expressed with geometric algebra} % Declares the document's title.
\author{Peeter Joot}         % Declares the author's name.
%\date{}        % Deleting this command produces today's date.

\begin{document}             % End of preamble and beginning of text.

\maketitle{}

\section{Torque}

Torque is generally defined as the magnitude of the perpendicular force component times distance, or work per unit angle.

Suppose a circular path in an arbitrary plane containing orthonormal vectors $\ucap$ and $\vcap$ is parameterized by angle.

\[
\Br = r(\ucap \cos \theta + \vcap \sin \theta) = r \ucap(\cos \theta + \ucap \vcap \sin \theta)
\]

By designating the unit bivector of this plane as the imaginary number

\[
\Bi  = \ucap \vcap = \ucap \wedge \vcap
\]
\[
\Bi ^2 = -1
\]

this path vector can be conveniently written in complex exponential form

\[
\Br = r \ucap e^{\Bi  \theta}
\]

and the derivative with respect to angle is

\[
\frac{d \Br}{d\theta} = r \ucap \Bi  e^{\Bi  \theta} = \Br  \Bi 
\]

So the torque, the rate of change of work $W$, due to a force $F$, is

\[
\tau = \frac{dW}{d\theta} = \BF \cdot \frac{d \Br}{d\theta} = \BF \cdot (\Br  \Bi )
\]

Unlike the cross product description of torque, $\Btau = \Br \times \BF$ no vector in a normal direction had to be introduced, a normal that doesn't exist in two dimensions or in greater than three dimensions.  The unit bivector describes the plane and the orientation of the rotation, and the sense of the rotation is relative to the angle between the vectors $\ucap$ and $\vcap$.

\subsection{Expanding the result in terms of components }

At a glance this doesn't appear much like the familiar torque as a determinant or cross product, but this can be expanded to demonstrate its equivalance (the cross product is hiding there in the bivector $\Bi = \ucap \wedge \vcap$).  Expanding the position vector in terms of the planar unit vectors 

\[
\Br \Bi =
\left(
r_u \ucap + r_v \vcap
\right)
\ucap \vcap
= 
r_u \vcap  
- r_v \ucap
\]

and expanding the force by components in the same direction plus the possible perpendicular remainder term

\[
\BF  = F_u \ucap + F_v \vcap + \BF _{\perp \ucap,\vcap}
\]

and then taking dot products yields is the torque

\[
\tau = \BF \cdot (\Br  \Bi ) = r_u F_v - r_v F_u
\]

This determinant may be familiar from derivations with $\ucap = \Be _1$, and $\vcap = \Be _2$ (See the Feynman lectures Volume I for example).

\subsection{Geometrical description }

When the magnitude of the "rotational arm" is factored out, the torque can be written as

\[
\tau = \BF \cdot (\Br  \Bi ) = |\Br |  (\BF \cdot (\rcap \Bi ))
\]

The vector $\rcap \Bi $ is the unit vector perpendicular to the $\Br$.  Thus the torque can also be described as the product of the magnitude of the rotational arm times the component of the force that is in the direction of the rotation (ie: the work done rotating something depends on length of the lever, and the size of the useful part of the force pushing on it).

\subsection{Slight generalization.  Application of the force to a lever not in the plane. }

If the rotational arm that the force is applied to is not in the plane of rotation then only the components of the lever arm direction and the component of the force that are in the plane will contribute to the work done.  The calculation above allowed for a force applied in an arbitrary direction, so to generalize this, a calculation that discards the component of the level arm direction not in the plane.

When $\Br $ is allowed to lie outside of the plane of rotation the component in the plane (bivector) $\Bi $ can be described with the geometric product nicely

\[
\Br _{\Bi } =  (\Br  \cdot \Bi ) \frac{1}{\Bi } =  -(\Br  \cdot \Bi ) \Bi 
\]

Thus, the vector with this magnitude that is perpendicular to this in the plane of the rotation  is

\[
\Br _{\Bi } \Bi  
=  -(\Br  \cdot \Bi ) \Bi ^2
=  (\Br  \cdot \Bi ) 
\]

So, the most general for torque for rotation constrained to the plane $i$ is:

\[
\tau
=  \BF  \cdot (\Br  \cdot \Bi ) 
\]

This makes sense when once considers that only the dot product part of $\Br  \Bi  = \Br  \cdot \Bi  + \Br  \wedge \Bi $ contributes to the component of $\Br $ in the plane, and when the lever is in the rotational plane this wedge product component of 
$\Br \Bi $ is zero.

\subsection{expressing torque as a bivector}

The general expression for torque for a rotation constrained to a plane has been found to be:

\[
\tau
=  \BF  \cdot (\Br  \cdot \Bi ) 
\]

We have an expectation that torque should have a form similar to the traditional vector torque
\[
\Btau = \Br \times \BF = -\bithree (\Br \wedge \BF)
\]

Note that here $\bithree = \Be_1 \Be_2 \Be_3 $ is the unit pseudoscalar for $\mathbb{R}^3$, not the unit bivector for the rotational plane.
We should be able to express torque in a form related to $\Br \wedge \BF$, but modified
in a fashion that results in a scalar value.

When the rotation isn't constrained to a specific plane the motion will be in

\[
\Bi = \frac{\rcap \wedge \Br'}{\nrrp}
\]

The lever arm component in this plane is

\begin{align*}
\Br \cdot \Bi
   &= \frac{1}{2}           (\Br \Bi - \Bi \Br) \\
   &= \frac{1}{2\nrrp} (\Br (\rcap \wedge \Br') - (\rcap \wedge \Br') \Br) \\
   &= \frac{1}{\nrrp}   \Br (\rcap \wedge \Br') \\
\end{align*}

So the torque in this natural plane of rotation is

\begin{align*}
\tau 
   &=  \BF  \cdot (\Br  \cdot \Bi )  \\
   &=  \frac{1}{\nrrp}     \BF \cdot ( \Br (\rcap \wedge \Br') ) \\
   &=  \frac{1}{2 \nrrp} \left(    \BF \Br (\rcap \wedge \Br') + (\Br' \wedge \rcap) \Br \BF    \right)\\
   &=  \frac{1}{2} ( \BF \Br \Bi + (\BF \Br \Bi)^\dagger ) = \frac{1}{2} ( \Bi \Br \BF + (\Bi \Br \BF)^\dagger ) \\
   &=  {\langle \Bi \Br \BF \rangle}_0
\end{align*}

The torque is the scalar part of $\Bi (\Br \BF)$.

\[
\tau 
   =  {\langle \Bi (\Br \cdot \BF + \Br \wedge \BF) \rangle}_0 \\
\]

Since the bivector scalar product $\Bi (\Br \cdot \BF)$ here contributes only a bivector part the scalar part comes only from the $\Bi (\Br \wedge \BF)$ component,
and one can write the torque in a fashion that's very similar to the vector cross product torque.  Here's both for comparision

\begin{align*}
\tau &=  {\langle \Bi (\Br \wedge \BF) \rangle}_0 \\
\Btau &= -\bithree (\Br \wedge \BF)
\end{align*}

Note again that $\Bi$ here is the unit bivector for the plane of rotation and not the unit 3D pseudoscalar $\bithree$.

\subsection{Plane common to force and vector}
Physical intuition provides one further way to express this.  Namely, the unit bivector for the rotational plane should also be in the plane common to $\BF$ and $\Br$

\[
\Bi = \frac{\BF \wedge \Br}{\sqrt{-(\BF \wedge \Br)^2}}
\]

So the torque is
\begin{align*}
\tau 
   &=  \frac{1}{\sqrt{-(\BF \wedge \Br)^2}} {\langle (\BF \wedge \Br)(\Br \wedge \BF) \rangle}_0 \\
   &=  \frac{1}{\sqrt{-(\BF \wedge \Br)^2}} {(\BF \wedge \Br)(\Br \wedge \BF) } \\
   &=  \frac {-(\Br \wedge \BF)^2 } {\sqrt{-(\Br \wedge \BF)^2}} \\
   &=  \sqrt {-(\Br \wedge \BF)^2 } \\
   &=  \norm {\Br \wedge \BF}
\end{align*}

Above the ${\langle \cdots \rangle}_0$ could be dropped because the quantity has only a scalar part.
The fact that the sign of the square root can be either plus or minus follows from the fact that the orientation of the unit bivector in the $\Br$, $\BF$ plane has two possibilites.  The positive root selection here is due to the orientation picked for $\Bi$.  

For comparision, this can also be expressed with the cross product:
\begin{align*}
\tau 
   &=  \sqrt {-(\Br \wedge \BF)^2 } \\
   &=  \sqrt {-(\Br \wedge \BF)(\Br \wedge \BF) } \\
   &=  \sqrt {-((\Br \times \BF)\bithree)(\bithree(\Br \times \BF)) } \\
   &=  \sqrt {(\Br \times \BF)^2} \\
   &=  \norm{\Br \times \BF} \\
   &=  \norm{\Btau} \\
\end{align*}

\end{document}               % End of document.
