\chapter{Time rate of change of the Poynting vector, and its conservation law.}\label{chap:PJpoyntingRate}
\date{ Jan 18, 2009.  $RCSfile: poyntingRate.tex,v $ Last $Revision: 1.16 $ $Date: 2009/06/11 16:45:58 $ }

\section{Motivation }

Derive the conservation laws for the time rate of change of the Poynting vector, which appears to be a momentum density like quantity.

The Poynting conservation relationship has been derived previously.  Additionally a starting
exploration
\ref{chap:PJemstresstensor}
of the related four vector quantity has been related to a subset of the energy momentum stress tensor.
This was incomplete since the meaning of the $T_{kj}$ terms of the tensor were unknown and the expected
Lorentz transform relationships had not been determined.  The aim here is to try to figure out this remainder.

\section{Calculation }

Repeating again from \ref{chap:PJpoynting}, the electrodynamic energy density $U$ and momentum flux density vectors are related as follows

\begin{align}\label{eqn:poynting_rate:fromPoyntingNotes}
U &= \frac{\epsilon_0}{2}\left( \BE^2 + c^2 \BB^2 \right) \\
\BP &= \inv{\mu_0}\BE \cross \BB = \inv{\mu_0} (i \BB) \cdot \BE \\
0 &= \PD{t}{U} + \spacegrad \cdot \BP + \BE \cdot \Bj
\end{align}

We want to now calculate the time rate of change of this Poynting (field momentum density) vector.

\begin{align*}
\PD{t}{\BP}
&= \PD{t}{} \left( \inv{\mu_0} \BE \cross \BB \right) \\
&= \PD{t}{} \left( \inv{\mu_0}(i\BB) \cdot \BE \right) \\
&= \partial_0 \left( \inv{\mu_0}(i c\BB) \cdot \BE \right) \\
&= \inv{\mu_0} \left( \partial_0 (i c\BB) \cdot \BE  + (i c\BB) \cdot \partial_0 \BE  \right)
\end{align*}

%Let's ignore the $\mu_0$ factor for now, and focus just on the field dot products.
We will want to express these time derivatives in
terms of the current and spatial derivatives to determine the conservation identity.  To do this let's go back to Maxwell's equation
once more, with a premultiplication by $\gamma_0$ to provide us with an observer dependent spacetime split

\begin{align*}
\gamma_0 \grad F &= \gamma_0 J / \epsilon_0 c \\
(\partial_0 + \spacegrad) ( \BE + i c \BB ) &= \rho/\epsilon_0 - \Bj/\epsilon_0 c \\
\end{align*}

We want the grade one and grade two components for the time derivative terms.  For grade one we have

\begin{align*}
- \Bj/\epsilon_0 c
&= \gpgradeone{(\partial_0 + \spacegrad) ( \BE + i c \BB )} \\
&= \partial_0 \BE + \spacegrad \cdot (ic \BB)
\end{align*}

and for grade two
\begin{align*}
0
&= \gpgradetwo{(\partial_0 + \spacegrad) ( \BE + i c \BB )} \\
&= \partial_0 (i c \BB) + \spacegrad \wedge \BE
\end{align*}

Using these we can express the time derivatives for back substitution
\begin{align*}
\partial_0 \BE &= - \Bj/\epsilon_0 c - \spacegrad \cdot (ic \BB) \\
\partial_0 (i c \BB) &= -\spacegrad \wedge \BE
\end{align*}

yielding
\begin{align*}
\mu_0 \PD{t}{\BP}
&= \partial_0 (i c\BB) \cdot \BE  + (i c\BB) \cdot \partial_0 \BE \\
&= -(\spacegrad \wedge \BE) \cdot \BE  - (i c\BB) \cdot \left( \Bj/\epsilon_0 c + \spacegrad \cdot (ic \BB) \right) \\
\end{align*}

Or
\begin{align*}
%\PD{t}{((i\BB) \cdot \BE)} +(\spacegrad \wedge \BE) \cdot \BE + (i c\BB) \cdot (\spacegrad \cdot (ic \BB)) &= - (i c\BB) \cdot \Bj/\epsilon_0 c \\
0
&= \partial_0 {((ic \BB) \cdot \BE)} +(\spacegrad \wedge \BE) \cdot \BE + (i c\BB) \cdot (\spacegrad \cdot (ic \BB)) + (i c\BB) \cdot \Bj/\epsilon_0 c \\
&= \gpgradeone{ \partial_0 (ic \BB \BE) +(\spacegrad \wedge \BE) \BE + i c\BB (\spacegrad \cdot (ic \BB)) + i c\BB \Bj/\epsilon_0 c } \\
&= \gpgradeone{ \partial_0 (ic \BB \BE) +(\spacegrad \wedge \BE) \BE + (\spacegrad \wedge (c \BB)) c \BB + i c\BB \Bj/\epsilon_0 c } \\
0 &= i\partial_0 (c \BB \wedge \BE) + (\spacegrad \wedge \BE) \cdot \BE + (\spacegrad \wedge (c \BB)) \cdot (c \BB) + i( c\BB \wedge \Bj)/\epsilon_0 c
\end{align*}

This appears to be the conservation law that is expected for the change in vector field momentum density.

\begin{align}\label{eqn:poynting_rate:momCons}
\partial_t (\BE \cross \BB) + (\spacegrad \wedge \BE) \cdot \BE + c^2 (\spacegrad \wedge \BB) \cdot \BB = (\BB \cross \Bj)/\epsilon_0
\end{align}

In terms of the original Poynting vector this is
\begin{align}\label{eqn:poynting_rate:withPoynting}
\PD{t}{\BP} + \inv{\mu_0}(\spacegrad \wedge \BE) \cdot \BE + c^2 \inv{\mu_0}(\spacegrad \wedge \BB) \cdot \BB = c^2 (\BB \cross \Bj)
\end{align}

%For clarity let's temporarily switch to natural units $c = \epsilon_0 = \mu_0$
%
%\begin{align}
%0 = \partial_0 (\BB \cross \BE) + (\spacegrad \wedge \BE) \cdot \BE + (\spacegrad \wedge \BB) \cdot \BB + \BB \wedge \Bj
%\end{align}
%
%The SI units can be restored here easily enough since we just have to put a $c$ with each $\BB$, and divide our current density by $\epsilon_0 c$.

Now, there are a few things to pursue here.

\begin{itemize}
\item How to or can we put this in four vector divergence form.
\item Relate this to the wikipedia result which is very different looking.
\item Find the relation to the stress energy tensor.
\item Lorentz transformation relation to Poynting energy momentum conservation law.
\end{itemize}

\subsection{Four vector form? }

If $\BP = P^m \sigma_m$, then each of the $P^m$ coordinates could be thought of as the zero coordinate of a four vector.  Can we get a four vector
divergence out of equation \ref{eqn:poynting_rate:momCons}?

Let's expand the wedge-dot term in coordinates.

\begin{align*}
( (\spacegrad \wedge \BE ) \cdot \BE ) \cdot \sigma_m
&= ((\sigma^a \wedge \sigma_b) \cdot \sigma_k ) \cdot \sigma_m (\partial_a E^b) E^k \\
&= (\delta^a_m \delta_{bk} - \delta_{bm} \delta^a_k) (\partial_a E^b) E^k \\
&= \sum_k (\partial_m E^k - \partial_k E^m) E^k \\
&= \partial_m \frac{\BE^2}{2} - (\BE \cdot \spacegrad) E^m \\
\end{align*}

So we have three equations, one for each $m = \{1,2,3\}$

\begin{align}\label{eqn:poynting_rate:coordinates}
%0 &= \partial_0 (c \BB \wedge \BE) + (\spacegrad \wedge \BE) \cdot \BE + (\spacegrad \wedge (c \BB)) \cdot (c \BB) + c\BB \wedge \Bj/\epsilon_0 c
\PD{t}{P^m} + c^2 \PD{x^m}{U} - \inv{\mu_0}( (\BE \cdot \spacegrad) E^m + c^2 (\BB \cdot \spacegrad) B^m ) &= c^2 (\BB \cross \Bj)_m
\end{align}

Damn.  This doesn't look anything like the four vector divergence that we had with the Poynting conservation equation.  In the second last line
of the wedge dot expansion we do see that we only have to sum over the $k \ne m$ terms.  Can that help simplify this?

\subsection{Compare to wikipedia form. }

To compare equation \ref{eqn:poynting_rate:withPoynting} with the
\href{http://en.wikipedia.org/wiki/Electromagnetic_stress-energy_tensor#Conservation_laws}{wikipedia article}
, the first thing we have to do is eliminate the wedge
products.

This can be done in a couple different ways.  One, is conversion to cross products

\begin{align*}
(\spacegrad \wedge \Ba) \cdot \Ba
&= \gpgradeone{(\spacegrad \wedge \Ba) \Ba } \\
&= \gpgradeone{i (\spacegrad \cross \Ba) \Ba } \\
&= \gpgradeone{i ((\spacegrad \cross \Ba) \cdot \Ba) + i ((\spacegrad \cross \Ba) \wedge \Ba) } \\
&= \gpgradeone{i ((\spacegrad \cross \Ba) \wedge \Ba) } \\
&= i^2 ((\spacegrad \cross \Ba) \cross \Ba) \\
\end{align*}

So we have
\begin{align}
(\spacegrad \wedge \Ba) \cdot \Ba &= \Ba \cross (\spacegrad \cross \Ba)
\end{align}

so we can rewrite the Poynting time change equation \ref{eqn:poynting_rate:withPoynting} as
\begin{align}
\PD{t}{\BP} + \inv{\mu_0}
\left( \BE \cross (\spacegrad \cross \BE) + c^2 \BB \cross (\spacegrad \cross \BB) \right)
 = c^2 (\BB \cross \Bj)
\end{align}

However, the wikipedia article has $\rho \BE$ terms, which suggests that a $\spacegrad \cdot \BE$ based expansion has been used.  Take II.

Let's try expanding this wedge dot differently, and to track what's being operated on write $\Bx$ as a variable vector, and
$\Ba$ as a constant vector.  Now expand

\begin{align*}
(\spacegrad \wedge \Bx) \cdot \Ba
&= - \Ba \cdot (\spacegrad \wedge \Bx) \\
&= \spacegrad (\Ba \cdot \Bx) - (\Ba \cdot \spacegrad) \wedge \Bx \\
%&= - \gpgradeone{\Ba (\spacegrad \wedge \Bx)} \\
%&= - \gpgradeone{\Ba (\spacegrad \Bx - \spacegrad \cdot \Bx)} \\
%&= \Ba (\spacegrad \cdot \Bx) - \gpgradeone{\Ba \spacegrad \Bx} \\
%&= \Ba (\spacegrad \cdot \Bx) - (\Ba \cdot \spacegrad) \Bx - \gpgradeone{ (\Ba \wedge \spacegrad) \Bx } \\
%&= \Ba (\spacegrad \cdot \Bx) - (\Ba \cdot \spacegrad) \Bx - (\Ba \wedge \spacegrad) \cdot \Bx \\
%&= \Ba (\spacegrad \cdot \Bx) - (\Ba \cdot \spacegrad) \Bx - \Ba (\spacegrad \cdot \Bx) + \spacegrad (\Ba \cdot \Bx) \\
\end{align*}

%Now it looks like we are going in circles a bit here, and there's probably a more direct way.  We do however have as a final
%result
%
%\begin{align*}
%(\spacegrad \wedge \Bx) \cdot \Ba &= \spacegrad (\Ba \cdot \Bx) - (\Ba \cdot \spacegrad) \Bx \\
%\end{align*}

What we really want is an expansion of $(\spacegrad \wedge \Bx) \cdot \Bx$.  To get there consider

\begin{align*}
\spacegrad \Bx^2
&= \dot{\spacegrad} \dot{\Bx} \cdot \Bx + \dot{\spacegrad} {\Bx} \cdot \dot{\Bx} \\
&= 2 \dot{\spacegrad} {\Bx} \cdot \dot{\Bx} \\
\end{align*}

This has the same form as the first term above.  We take the gradient and apply it to a dot product where one of the vectors is kept constant, so we can write

\begin{align*}
{\spacegrad} {\Bx} \cdot \dot{\Bx} &= \inv{2} \spacegrad \Bx^2
\end{align*}

and finally
\begin{align}\label{eqn:poynting_rate:wedgedot}
(\spacegrad \wedge \Bx) \cdot \Bx
%&= \dot{\spacegrad} (\Bx \cdot \dot{\Bx}) - (\Bx \cdot \spacegrad) \Bx \\
&= \inv{2} \spacegrad \Bx^2 - (\Bx \cdot \spacegrad) \Bx
\end{align}

We can now reassemble the equations and write

\begin{align*}
(\spacegrad \wedge \BE) \cdot \BE + c^2 (\spacegrad \wedge \BB) \cdot \BB
&=
\inv{2} \spacegrad \BE^2 - (\BE \cdot \spacegrad) \BE
+ c^2 \left(\inv{2} \spacegrad \BB^2 - (\BB \cdot \spacegrad) \BB \right) \\
&= \inv{\epsilon_0} \spacegrad U - (\BE \cdot \spacegrad) \BE - c^2 (\BB \cdot \spacegrad) \BB \\
\end{align*}

Now, we have the time derivative of momentum and the spatial derivative of the energy grouped together in a nice
relativistic seeming pairing.  For comparison let's also put the energy density rate change equation with this
to observe them together

\begin{align}\label{eqn:poynting_rate:withPoyntingAndEnergy}
\PD{t}{U} + \spacegrad \cdot \BP &= -\Bj \cdot \BE \\
\PD{t}{\BP} + c^2 \spacegrad U &= -c^2 (\Bj \cross \BB) + \inv{\mu_0} \left( (\BE \cdot \spacegrad) \BE + c^2 (\BB \cdot \spacegrad) \BB \right) 
\end{align}

The second equation here is exactly what we worked out above by coordinate expansion when looking for a four vector formulation
of this equation.  This however, appears much closer to the desired result, which wasn't actually clear looking at the
coordinate expansion.

These equations aren't tidy enough seeming, so one can intuit that there's some more natural way to express those
misfit seeming $(\Bx \cdot \spacegrad) \Bx$ terms.
It would be logically tidier if we could express those both in terms of charge and current densities.

Now, it is too bad that it isn't true that
\begin{align*}
(\BE \cdot \spacegrad) \BE &= \BE (\spacegrad \cdot \BE) 
\end{align*}

If that were the case then we would have on the right hand side
\begin{align*}
-c^2(\Bj \cross \BB)
+
\inv{\mu} \left(
\BE (\spacegrad \cdot \BE) + c^2 \BB (\spacegrad \cdot \BB) 
\right) 
&=
-c^2(\Bj \cross \BB) +
\inv{\mu_0} (\BE \rho + c^2 \BB (0)) \\
&= -c^2(\Bj \cross \BB) + \inv{\mu_0} \rho \BE \\
\end{align*}

This has a striking similarity to the Lorentz force law, and is also fairly close to the wikipedia equation, with the exception that the
$\Bj \cross \BB$ and $\rho \BE$ terms have opposing signs.

Lets instead adding and subtracting this term so that the conservation equation remains correct

\begin{align*}
\inv{c^2} \PD{t}{\BP} + \spacegrad U &-\epsilon_0 \left( 
\BE (\spacegrad \cdot \BE) +(\BE \cdot \spacegrad) \BE 
+ c^2 \BB (\spacegrad \cdot \BB) + c^2 (\BB \cdot \spacegrad) \BB 
\right) \\
&= -(\Bj \cross \BB) - \epsilon_0 \rho \BE 
\end{align*}
% c^2 \mu_0 = 1/\e_0

Now we are left with quantities of the following form.

\begin{align*}
\Bx (\spacegrad \cdot \Bx) +(\Bx \cdot \spacegrad) \Bx 
\end{align*}

The sum of these for the electric and magnetic fields appears to be what the 
wiki article calls $\spacegrad \cdot \sigma$, although it appears
there that $\sigma$ is a scalar so this doesn't quite make sense.

It appears that 
we should therefore be looking to
express these in terms of a gradient of the squared fields?  We have such $\BE^2$ and $\BB^2$ terms in the energy so it would make some logical sense if this
could be done.

The essence of the desired reduction is to see if we can find a scalar function $\sigma(\Bx)$ such that

\begin{align*}
\spacegrad \sigma(\Bx) &= \inv{2} \spacegrad \Bx^2 - \left(\Bx (\spacegrad \cdot \Bx) + (\Bx \cdot \spacegrad) \Bx )\right)
\end{align*}

\subsection{stress tensor. }

From \cite{doran2003gap} we expect that there is a relationship between
the equations \ref{eqn:poynting_rate:coordinates}, and $F \gamma_k F$.  Let's see 
if we can find exactly how these relate.

%First.  In the spatial basis, as a three dimensional cross product, we can express the Poynting vector as an antisymmetric tensor.

%\BP = \BP ...

TODO: ...

\section{Take II. }

After going in circles and having a better idea now where I'm going, time to restart and make sure that errors aren't compounding.

The starting point will be

\begin{align*}
\PD{t}{\BP} &= \inv{\mu_0} \left( \partial_0 (i c\BB) \cdot \BE  + (i c\BB) \cdot \partial_0 \BE  \right) \\
\partial_0 \BE &= - \Bj/\epsilon_0 c - \spacegrad \cdot (ic \BB) \\
\partial_0 (i c \BB) &= -\spacegrad \wedge \BE
\end{align*}

Assembling we have

\begin{align*}
\PD{t}{\BP} + \inv{\mu_0} \left( (\spacegrad \wedge \BE) \cdot \BE + (i c\BB) \cdot ( \Bj/\epsilon_0 c + \spacegrad \cdot (ic \BB) ) \right) &= 0 \\
\end{align*}

This is

\begin{align*}
\PD{t}{\BP} + \inv{\mu_0} \left( (\spacegrad \wedge \BE) \cdot \BE + (i c\BB) \cdot ( \spacegrad \cdot (ic \BB) ) \right) &= -c^2 (i \BB) \cdot \Bj.
\end{align*}

Now get rid of the pseudoscalars
\begin{align*}
(i \BB) \cdot \Bj
&= \gpgradeone{ i \BB \Bj } \\
&= i (\BB \wedge \Bj) \\
&= i^2 (\BB \cross \Bj) \\
&= -(\BB \cross \Bj) \\
\end{align*}

and 
\begin{align*}
(i c\BB) \cdot ( \spacegrad \cdot (ic \BB) )  
&= c^2 \gpgradeone{ i \BB ( \spacegrad \cdot (i\BB) ) } \\
&= c^2 \gpgradeone{ i \BB \gpgradeone{\spacegrad i\BB} } \\
&= c^2 \gpgradeone{ i \BB i (\spacegrad \wedge \BB) } \\
&= -c^2 \gpgradeone{ \BB (\spacegrad \wedge \BB) } \\
&= -c^2 \BB \cdot (\spacegrad \wedge \BB) \\
\end{align*}

So we have
\begin{align*}
\PD{t}{\BP} - \inv{\mu_0} \left( \BE \cdot (\spacegrad \wedge \BE) +c^2 \BB \cdot (\spacegrad \wedge \BB) \right) &= c^2 (\BB \cross \Bj) \\
\end{align*}

Now we subtract $(\BE (\spacegrad \cdot \BE) + c^2 \BB (\spacegrad \cdot \BB))/\mu_0 = \BE \rho/\epsilon_0\mu_0$ from both sides yielding

\begin{align*}
\PD{t}{\BP} - \inv{\mu_0} \left( 
\BE \cdot (\spacegrad \wedge \BE) + \BE (\spacegrad \cdot \BE) 
+ c^2 \BB \cdot (\spacegrad \wedge \BB) 
+ c^2 \BB (\spacegrad \cdot \BB)
\right) &= -c^2 (\Bj \cross \BB + \rho \BE) \\
\end{align*}

Regrouping slightly 

\begin{align*}
0 = \inv{c^2}\PD{t}{\BP} &+ (\Bj \cross \BB + \rho \BE) 
\\
&-{\epsilon_0} \left( 
\BE \cdot (\spacegrad \wedge \BE) + \BE (\spacegrad \cdot \BE) 
+ c^2 \BB \cdot (\spacegrad \wedge \BB) 
+ c^2 \BB (\spacegrad \cdot \BB)
\right) 
\end{align*}

Now, let's write the $\BE$ gradient terms here explicitly in coordinates.

\begin{align*}
-\BE \cdot (\spacegrad \wedge \BE) - \BE (\spacegrad \cdot \BE) 
&= -\sigma_k \cdot (\sigma^m \wedge \sigma_n) E^k \partial_m E^n 
- E^k \sigma_k \partial_m E^m \\
&= 
-\delta_{k}^m \sigma_n E^k \partial_m E^n 
+\delta_{kn} \sigma^m E^k \partial_m E^n 
- E^k \sigma_k \partial_m E^m \\
&= 
- \sigma_n E^k \partial_k E^n 
+ \sigma^m E^k \partial_m E^k 
- E^k \sigma_k \partial_m E^m \\
&= \sum_{k,m} \sigma_k \left( -E^m \partial_m E^k +E^m \partial_k E^m -E^k \partial_m E^m \right) \\
\end{align*}

We could do the $\BB$ terms too, but they will have the same form.  Now \cite{schwartz1987pe} contains a relativistic treatment of
the stress tensor that would take some notation study to digest, but the end result appears to have the divergence
result that is desired.  It is a second rank tensor which probably explains the $\grad \cdot \sigma$ notation in wikipedia.

For the $x$ coordinate of the $\PDi{t}{\BP}$ vector the book says we have a vector of the form

\begin{align*}
\BT_x = \inv{2}(- E_x^2 + E_y^2 + E_z^2)\sigma_1 - E_x E_y \sigma_2 - E_x E_z \sigma_3
\end{align*}

and it looks like the divergence of this should give us our desired mess.  Let's try this, writing $k,m,n$ as distinct indexes.

\begin{align*}
\BT_k &= \inv{2}(- (E^k)^2 + (E^m)^2 + (E^n)^2)\sigma_k - E^k E^m \sigma_m - E^k E^n \sigma_n
\end{align*}

\begin{align*}
\spacegrad \cdot \BT_k
&= \inv{2} \partial_k (-(E^k)^2 + (E^m)^2 + (E^n)^2) 
- \partial_m (E^k E^m) 
- \partial_n(E^k E^n) \\
&= 
-E^k \partial_k E^k 
+ E^m \partial_k E^m
+ E^n \partial_k E^n
- E^k \partial_m E^m
- E^m \partial_m E^k 
- E^k \partial_n E^n
- E^n \partial_n E^k \\
&= 
- E^k \partial_k E^k 
- E^k \partial_m E^m
- E^k \partial_n E^n \\
&- E^m \partial_m E^k  
+ E^m \partial_k E^m \\
&- E^n \partial_n E^k 
+ E^n \partial_k E^n
\\
\end{align*}

Does this match?  Let's expand our $k$ term above to see if it looks the same.  That is

\begin{align*}
\sum_m (-E^m \partial_m E^k +E^m \partial_k E^m -E^k \partial_m E^m) 
&=
-E^k \partial_k E^k 
+E^k \partial_k E^k 
-E^k \partial_k E^k \\
&-E^m \partial_m E^k 
+E^m \partial_k E^m 
-E^k \partial_m E^m \\
&-E^n \partial_n E^k 
+E^n \partial_k E^n 
-E^k \partial_n E^n \\
&=
-E^k \partial_k E^k 
-E^k \partial_m E^m
-E^k \partial_n E^n \\
&-E^m \partial_m E^k 
+E^m \partial_k E^m \\
&-E^n \partial_n E^k 
+E^n \partial_k E^n \\
\end{align*}

Yeah!  Finally have a form of the momentum conservation equation that is strictly in terms of gradients and time partials.  Summarizing the results, this is

\begin{align}
\inv{c^2}\PD{t}{\BP} + \Bj \cross \BB + \rho \BE + \sum_k \sigma_k \spacegrad \cdot \BT_k = 0
\end{align}

Where
\begin{align*}
\sum_k \sigma_k \spacegrad \cdot \BT_k
&=
-\epsilon_0 \left(
\BE \cdot (\spacegrad \wedge \BE) + \BE (\spacegrad \cdot \BE) 
+ c^2 \BB \cdot (\spacegrad \wedge \BB) 
+ c^2 \BB (\spacegrad \cdot \BB) \right)
\end{align*}

For $\BT_k$ itself, with $k \ne m \ne n$ we have

\begin{align*}
\BT_k &= 
\epsilon_0 \left( \inv{2}(-(E^k)^2 + (E^m)^2 + (E^n)^2)\sigma_k - E^k E^m \sigma_m - E^k E^n \sigma_n \right) \\
&+ \inv{\mu_0}\left(\inv{2}(-(B^k)^2 + (B^m)^2 + (B^n)^2)\sigma_k - B^k B^m \sigma_m - B^k B^n \sigma_n \right)
\end{align*}
