\documentclass{article}

\usepackage{amsmath}
\usepackage{mathpazo}

%
% shorthand for bold symbols, convenient for vectors and matrices
%
\newcommand{\Ba}[0]{\mathbf{a}}
\newcommand{\Bb}[0]{\mathbf{b}}
\newcommand{\Bc}[0]{\mathbf{c}}
\newcommand{\Bd}[0]{\mathbf{d}}
\newcommand{\Be}[0]{\mathbf{e}}
\newcommand{\Bf}[0]{\mathbf{f}}
\newcommand{\Bg}[0]{\mathbf{g}}
\newcommand{\Bh}[0]{\mathbf{h}}
\newcommand{\Bi}[0]{\mathbf{i}}
\newcommand{\Bj}[0]{\mathbf{j}}
\newcommand{\Bk}[0]{\mathbf{k}}
\newcommand{\Bl}[0]{\mathbf{l}}
\newcommand{\Bm}[0]{\mathbf{m}}
\newcommand{\Bn}[0]{\mathbf{n}}
\newcommand{\Bo}[0]{\mathbf{o}}
\newcommand{\Bp}[0]{\mathbf{p}}
\newcommand{\Bq}[0]{\mathbf{q}}
\newcommand{\Br}[0]{\mathbf{r}}
\newcommand{\Bs}[0]{\mathbf{s}}
\newcommand{\Bt}[0]{\mathbf{t}}
\newcommand{\Bu}[0]{\mathbf{u}}
\newcommand{\Bv}[0]{\mathbf{v}}
\newcommand{\Bw}[0]{\mathbf{w}}
\newcommand{\Bx}[0]{\mathbf{x}}
\newcommand{\By}[0]{\mathbf{y}}
\newcommand{\Bz}[0]{\mathbf{z}}
\newcommand{\BA}[0]{\mathbf{A}}
\newcommand{\BB}[0]{\mathbf{B}}
\newcommand{\BC}[0]{\mathbf{C}}
\newcommand{\BD}[0]{\mathbf{D}}
\newcommand{\BE}[0]{\mathbf{E}}
\newcommand{\BF}[0]{\mathbf{F}}
\newcommand{\BG}[0]{\mathbf{G}}
\newcommand{\BH}[0]{\mathbf{H}}
\newcommand{\BI}[0]{\mathbf{I}}
\newcommand{\BJ}[0]{\mathbf{J}}
\newcommand{\BK}[0]{\mathbf{K}}
\newcommand{\BL}[0]{\mathbf{L}}
\newcommand{\BM}[0]{\mathbf{M}}
\newcommand{\BN}[0]{\mathbf{N}}
\newcommand{\BO}[0]{\mathbf{O}}
\newcommand{\BP}[0]{\mathbf{P}}
\newcommand{\BQ}[0]{\mathbf{Q}}
\newcommand{\BR}[0]{\mathbf{R}}
\newcommand{\BS}[0]{\mathbf{S}}
\newcommand{\BT}[0]{\mathbf{T}}
\newcommand{\BU}[0]{\mathbf{U}}
\newcommand{\BV}[0]{\mathbf{V}}
\newcommand{\BW}[0]{\mathbf{W}}
\newcommand{\BX}[0]{\mathbf{X}}
\newcommand{\BY}[0]{\mathbf{Y}}
\newcommand{\BZ}[0]{\mathbf{Z}}

\newcommand{\Bzero}[0]{\mathbf{0}}
\newcommand{\Btheta}[0]{\boldsymbol{\theta}}
\newcommand{\Btau}[0]{\boldsymbol{\tau}}
\newcommand{\Bomega}[0]{\boldsymbol{\omega}}

%
% shorthand for unit vectors
%
\newcommand{\acap}[0]{\hat{\Ba}}
\newcommand{\bcap}[0]{\hat{\Bb}}
\newcommand{\ccap}[0]{\hat{\Bc}}
\newcommand{\dcap}[0]{\hat{\Bd}}
\newcommand{\ecap}[0]{\hat{\Be}}
\newcommand{\fcap}[0]{\hat{\Bf}}
\newcommand{\gcap}[0]{\hat{\Bg}}
\newcommand{\hcap}[0]{\hat{\Bh}}
\newcommand{\icap}[0]{\hat{\Bi}}
\newcommand{\jcap}[0]{\hat{\Bj}}
\newcommand{\kcap}[0]{\hat{\Bk}}
\newcommand{\lcap}[0]{\hat{\Bl}}
\newcommand{\mcap}[0]{\hat{\Bm}}
\newcommand{\ncap}[0]{\hat{\Bn}}
\newcommand{\ocap}[0]{\hat{\Bo}}
\newcommand{\pcap}[0]{\hat{\Bp}}
\newcommand{\qcap}[0]{\hat{\Bq}}
\newcommand{\rcap}[0]{\hat{\Br}}
\newcommand{\scap}[0]{\hat{\Bs}}
\newcommand{\tcap}[0]{\hat{\Bt}}
\newcommand{\ucap}[0]{\hat{\Bu}}
\newcommand{\vcap}[0]{\hat{\Bv}}
\newcommand{\wcap}[0]{\hat{\Bw}}
\newcommand{\xcap}[0]{\hat{\Bx}}
\newcommand{\ycap}[0]{\hat{\By}}
\newcommand{\zcap}[0]{\hat{\Bz}}
\newcommand{\thetacap}[0]{\hat{\Btheta}}

%
% to write R^n and C^n in a distinguishable fashion.  Perhaps change this
% to the double lined characters upon figuring out how to do so.
%
\newcommand{\C}[1]{$\mathbb{C}^{#1}$}
\newcommand{\R}[1]{$\mathbb{R}^{#1}$}

%
% various generally useful helpers
%

% derivative of #1 wrt. #2:
\newcommand{\D}[2] {\frac {d#2} {d#1}}

\newcommand{\inv}[1]{\frac{1}{#1}}
\newcommand{\cross}[0]{\times}

\newcommand{\abs}[1]{\lvert{#1}\rvert}
\newcommand{\norm}[1]{\lVert{#1}\rVert}
\newcommand{\innerprod}[2]{\langle{#1}, {#2}\rangle}
\newcommand{\dotprod}[2]{{#1} \cdot {#2}}
\newcommand{\bdotprod}[2]{\left({#1} \cdot {#2}\right)}
\newcommand{\crossprod}[2]{{#1} \cross {#2}}
\newcommand{\tripleprod}[3]{\dotprod{\left(\crossprod{#1}{#2}\right)}{#3}}

\DeclareMathOperator{\Proj}{Proj}
\DeclareMathOperator{\Span}{span}
\DeclareMathOperator{\Sgn}{sgn}
\DeclareMathOperator{\Area}{Area}
\DeclareMathOperator{\Volume}{Volume}

%
% A few miscellaneous things specific to this document
%
\newcommand{\crossop}[1]{\crossprod{#1}{}}

% R2 vector.
\newcommand{\VectorTwo}[2]{
\begin{bmatrix}
 {#1} \\
 {#2}
\end{bmatrix}
}

\newcommand{\VectorN}[1]{
\begin{bmatrix}
{#1}_1 \\
{#1}_2 \\
\vdots \\
{#1}_N \\
\end{bmatrix}
}

\newcommand{\DETuvij}[4]{
\begin{vmatrix}
 {#1}_{#3} & {#1}_{#4} \\
 {#2}_{#3} & {#2}_{#4}
\end{vmatrix}
}

\newcommand{\DETuvwijk}[6]{
\begin{vmatrix}
 {#1}_{#4} & {#1}_{#5} & {#1}_{#6} \\
 {#2}_{#4} & {#2}_{#5} & {#2}_{#6} \\
 {#3}_{#4} & {#3}_{#5} & {#3}_{#6}
\end{vmatrix}
}

\newcommand{\DETuvwxijkl}[8]{
\begin{vmatrix}
 {#1}_{#5} & {#1}_{#6} & {#1}_{#7} & {#1}_{#8} \\
 {#2}_{#5} & {#2}_{#6} & {#2}_{#7} & {#2}_{#8} \\
 {#3}_{#5} & {#3}_{#6} & {#3}_{#7} & {#3}_{#8} \\
 {#4}_{#5} & {#4}_{#6} & {#4}_{#7} & {#4}_{#8} \\
\end{vmatrix}
}

%\newcommand{\DETuvwxyijklm}[10]{
%\begin{vmatrix}
% {#1}_{#6} & {#1}_{#7} & {#1}_{#8} & {#1}_{#9} & {#1}_{#10} \\
% {#2}_{#6} & {#2}_{#7} & {#2}_{#8} & {#2}_{#9} & {#2}_{#10} \\
% {#3}_{#6} & {#3}_{#7} & {#3}_{#8} & {#3}_{#9} & {#3}_{#10} \\
% {#4}_{#6} & {#4}_{#7} & {#4}_{#8} & {#4}_{#9} & {#4}_{#10} \\
% {#5}_{#6} & {#5}_{#7} & {#5}_{#8} & {#5}_{#9} & {#5}_{#10}
%\end{vmatrix}
%}

% R3 vector.
\newcommand{\VectorThree}[3]{
\begin{bmatrix}
 {#1} \\
 {#2} \\
 {#3}
\end{bmatrix}
}


%<misc>
%
\newcommand{\Abs}[1]{{\left\lvert{#1}\right\rvert}}
\newcommand{\spacegrad}[0]{\boldsymbol{\nabla}}
\newcommand{\grad}[0]{\nabla}
\newcommand{\LL}[0]{\mathcal{L}}

% == \partial_{#1} {#2}
\newcommand{\PD}[2]{\frac{\partial {#2}}{\partial {#1}}}
% inline variant
\newcommand{\PDi}[2]{{\partial {#2}}/{\partial {#1}}}

\newcommand{\PDD}[3]{\frac{\partial^2 {#3}}{\partial {#1}\partial {#2}}}
%\newcommand{\PDd}[2]{\frac{\partial^2 {#2}}{{\partial{#1}}^2}}
\newcommand{\PDsq}[2]{\frac{\partial^2 {#2}}{(\partial {#1})^2}}

\newcommand{\Partial}[2]{\frac{\partial {#1}}{\partial {#2}}}
\DeclareMathOperator{\RejName}{Rej}
\newcommand{\Rej}[2]{\RejName_{#1}\left( {#2} \right)}
\newcommand{\Rm}[1]{\mathbb{R}^{#1}}
\newcommand{\Cm}[1]{\mathbb{C}^{#1}}
\newcommand{\conj}[0]{{*}}

%</misc>

% <grade selection>
%
\newcommand{\gpgrade}[2] {{\left\langle{{#1}}\right\rangle}_{#2}}

\newcommand{\gpgradezero}[1] {\gpgrade{#1}{}}
%\newcommand{\gpscalargrade}[1] {{\left\langle{{#1}}\right\rangle}}
%\newcommand{\gpgradezero}[1] {\gpgrade{#1}{0}}

%\newcommand{\gpgradeone}[1] {{\left\langle{{#1}}\right\rangle}_{1}}
\newcommand{\gpgradeone}[1] {\gpgrade{#1}{1}}

\newcommand{\gpgradetwo}[1] {\gpgrade{#1}{2}}
\newcommand{\gpgradethree}[1] {\gpgrade{#1}{3}}
\newcommand{\gpgradefour}[1] {\gpgrade{#1}{4}}
%
% </grade selection>



\newcommand{\adot}[0]{{\dot{a}}}
\newcommand{\bdot}[0]{{\dot{b}}}
% taken for centered dot:
%\newcommand{\cdot}[0]{{\dot{c}}}
%\newcommand{\ddot}[0]{{\dot{d}}}
\newcommand{\edot}[0]{{\dot{e}}}
\newcommand{\fdot}[0]{{\dot{f}}}
\newcommand{\gdot}[0]{{\dot{g}}}
\newcommand{\hdot}[0]{{\dot{h}}}
\newcommand{\idot}[0]{{\dot{i}}}
\newcommand{\jdot}[0]{{\dot{j}}}
\newcommand{\kdot}[0]{{\dot{k}}}
\newcommand{\ldot}[0]{{\dot{l}}}
\newcommand{\mdot}[0]{{\dot{m}}}
\newcommand{\ndot}[0]{{\dot{n}}}
%\newcommand{\odot}[0]{{\dot{o}}}
\newcommand{\pdot}[0]{{\dot{p}}}
\newcommand{\qdot}[0]{{\dot{q}}}
\newcommand{\rdot}[0]{{\dot{r}}}
\newcommand{\sdot}[0]{{\dot{s}}}
\newcommand{\tdot}[0]{{\dot{t}}}
\newcommand{\udot}[0]{{\dot{u}}}
\newcommand{\vdot}[0]{{\dot{v}}}
\newcommand{\wdot}[0]{{\dot{w}}}
\newcommand{\xdot}[0]{{\dot{x}}}
\newcommand{\ydot}[0]{{\dot{y}}}
\newcommand{\zdot}[0]{{\dot{z}}}
\newcommand{\addot}[0]{{\ddot{a}}}
\newcommand{\bddot}[0]{{\ddot{b}}}
\newcommand{\cddot}[0]{{\ddot{c}}}
%\newcommand{\dddot}[0]{{\ddot{d}}}
\newcommand{\eddot}[0]{{\ddot{e}}}
\newcommand{\fddot}[0]{{\ddot{f}}}
\newcommand{\gddot}[0]{{\ddot{g}}}
\newcommand{\hddot}[0]{{\ddot{h}}}
\newcommand{\iddot}[0]{{\ddot{i}}}
\newcommand{\jddot}[0]{{\ddot{j}}}
\newcommand{\kddot}[0]{{\ddot{k}}}
\newcommand{\lddot}[0]{{\ddot{l}}}
\newcommand{\mddot}[0]{{\ddot{m}}}
\newcommand{\nddot}[0]{{\ddot{n}}}
\newcommand{\oddot}[0]{{\ddot{o}}}
\newcommand{\pddot}[0]{{\ddot{p}}}
\newcommand{\qddot}[0]{{\ddot{q}}}
\newcommand{\rddot}[0]{{\ddot{r}}}
\newcommand{\sddot}[0]{{\ddot{s}}}
\newcommand{\tddot}[0]{{\ddot{t}}}
\newcommand{\uddot}[0]{{\ddot{u}}}
\newcommand{\vddot}[0]{{\ddot{v}}}
\newcommand{\wddot}[0]{{\ddot{w}}}
\newcommand{\xddot}[0]{{\ddot{x}}}
\newcommand{\yddot}[0]{{\ddot{y}}}
\newcommand{\zddot}[0]{{\ddot{z}}}

%<bold and dot greek symbols>
%

\newcommand{\Deltadot}[0]{{\dot{\Delta}}}
\newcommand{\Gammadot}[0]{{\dot{\Gamma}}}
\newcommand{\Lambdadot}[0]{{\dot{\Lambda}}}
\newcommand{\Omegadot}[0]{{\dot{\Omega}}}
\newcommand{\Phidot}[0]{{\dot{\Phi}}}
\newcommand{\Pidot}[0]{{\dot{\Pi}}}
\newcommand{\Psidot}[0]{{\dot{\Psi}}}
\newcommand{\Sigmadot}[0]{{\dot{\Sigma}}}
\newcommand{\Thetadot}[0]{{\dot{\Theta}}}
\newcommand{\Upsilondot}[0]{{\dot{\Upsilon}}}
\newcommand{\Xidot}[0]{{\dot{\Xi}}}
\newcommand{\alphadot}[0]{{\dot{\alpha}}}
\newcommand{\betadot}[0]{{\dot{\beta}}}
\newcommand{\chidot}[0]{{\dot{\chi}}}
\newcommand{\deltadot}[0]{{\dot{\delta}}}
\newcommand{\epsilondot}[0]{{\dot{\epsilon}}}
\newcommand{\etadot}[0]{{\dot{\eta}}}
\newcommand{\gammadot}[0]{{\dot{\gamma}}}
\newcommand{\kappadot}[0]{{\dot{\kappa}}}
\newcommand{\lambdadot}[0]{{\dot{\lambda}}}
\newcommand{\mudot}[0]{{\dot{\mu}}}
\newcommand{\nudot}[0]{{\dot{\nu}}}
\newcommand{\omegadot}[0]{{\dot{\omega}}}
\newcommand{\phidot}[0]{{\dot{\phi}}}
\newcommand{\pidot}[0]{{\dot{\pi}}}
\newcommand{\psidot}[0]{{\dot{\psi}}}
\newcommand{\rhodot}[0]{{\dot{\rho}}}
\newcommand{\sigmadot}[0]{{\dot{\sigma}}}
\newcommand{\taudot}[0]{{\dot{\tau}}}
\newcommand{\thetadot}[0]{{\dot{\theta}}}
\newcommand{\upsilondot}[0]{{\dot{\upsilon}}}
\newcommand{\varepsilondot}[0]{{\dot{\varepsilon}}}
\newcommand{\varphidot}[0]{{\dot{\varphi}}}
\newcommand{\varpidot}[0]{{\dot{\varpi}}}
\newcommand{\varrhodot}[0]{{\dot{\varrho}}}
\newcommand{\varsigmadot}[0]{{\dot{\varsigma}}}
\newcommand{\varthetadot}[0]{{\dot{\vartheta}}}
\newcommand{\xidot}[0]{{\dot{\xi}}}
\newcommand{\zetadot}[0]{{\dot{\zeta}}}

\newcommand{\Deltaddot}[0]{{\ddot{\Delta}}}
\newcommand{\Gammaddot}[0]{{\ddot{\Gamma}}}
\newcommand{\Lambdaddot}[0]{{\ddot{\Lambda}}}
\newcommand{\Omegaddot}[0]{{\ddot{\Omega}}}
\newcommand{\Phiddot}[0]{{\ddot{\Phi}}}
\newcommand{\Piddot}[0]{{\ddot{\Pi}}}
\newcommand{\Psiddot}[0]{{\ddot{\Psi}}}
\newcommand{\Sigmaddot}[0]{{\ddot{\Sigma}}}
\newcommand{\Thetaddot}[0]{{\ddot{\Theta}}}
\newcommand{\Upsilonddot}[0]{{\ddot{\Upsilon}}}
\newcommand{\Xiddot}[0]{{\ddot{\Xi}}}
\newcommand{\alphaddot}[0]{{\ddot{\alpha}}}
\newcommand{\betaddot}[0]{{\ddot{\beta}}}
\newcommand{\chiddot}[0]{{\ddot{\chi}}}
\newcommand{\deltaddot}[0]{{\ddot{\delta}}}
\newcommand{\epsilonddot}[0]{{\ddot{\epsilon}}}
\newcommand{\etaddot}[0]{{\ddot{\eta}}}
\newcommand{\gammaddot}[0]{{\ddot{\gamma}}}
\newcommand{\kappaddot}[0]{{\ddot{\kappa}}}
\newcommand{\lambdaddot}[0]{{\ddot{\lambda}}}
\newcommand{\muddot}[0]{{\ddot{\mu}}}
\newcommand{\nuddot}[0]{{\ddot{\nu}}}
\newcommand{\omegaddot}[0]{{\ddot{\omega}}}
\newcommand{\phiddot}[0]{{\ddot{\phi}}}
\newcommand{\piddot}[0]{{\ddot{\pi}}}
\newcommand{\psiddot}[0]{{\ddot{\psi}}}
\newcommand{\rhoddot}[0]{{\ddot{\rho}}}
\newcommand{\sigmaddot}[0]{{\ddot{\sigma}}}
\newcommand{\tauddot}[0]{{\ddot{\tau}}}
\newcommand{\thetaddot}[0]{{\ddot{\theta}}}
\newcommand{\upsilonddot}[0]{{\ddot{\upsilon}}}
\newcommand{\varepsilonddot}[0]{{\ddot{\varepsilon}}}
\newcommand{\varphiddot}[0]{{\ddot{\varphi}}}
\newcommand{\varpiddot}[0]{{\ddot{\varpi}}}
\newcommand{\varrhoddot}[0]{{\ddot{\varrho}}}
\newcommand{\varsigmaddot}[0]{{\ddot{\varsigma}}}
\newcommand{\varthetaddot}[0]{{\ddot{\vartheta}}}
\newcommand{\xiddot}[0]{{\ddot{\xi}}}
\newcommand{\zetaddot}[0]{{\ddot{\zeta}}}

\newcommand{\BDelta}[0]{\boldsymbol{\Delta}}
\newcommand{\BGamma}[0]{\boldsymbol{\Gamma}}
\newcommand{\BLambda}[0]{\boldsymbol{\Lambda}}
\newcommand{\BOmega}[0]{\boldsymbol{\Omega}}
\newcommand{\BPhi}[0]{\boldsymbol{\Phi}}
\newcommand{\BPi}[0]{\boldsymbol{\Pi}}
\newcommand{\BPsi}[0]{\boldsymbol{\Psi}}
\newcommand{\BSigma}[0]{\boldsymbol{\Sigma}}
\newcommand{\BTheta}[0]{\boldsymbol{\Theta}}
\newcommand{\BUpsilon}[0]{\boldsymbol{\Upsilon}}
\newcommand{\BXi}[0]{\boldsymbol{\Xi}}
\newcommand{\Balpha}[0]{\boldsymbol{\alpha}}
\newcommand{\Bbeta}[0]{\boldsymbol{\beta}}
\newcommand{\Bchi}[0]{\boldsymbol{\chi}}
\newcommand{\Bdelta}[0]{\boldsymbol{\delta}}
\newcommand{\Bepsilon}[0]{\boldsymbol{\epsilon}}
\newcommand{\Beta}[0]{\boldsymbol{\eta}}
\newcommand{\Bgamma}[0]{\boldsymbol{\gamma}}
\newcommand{\Bkappa}[0]{\boldsymbol{\kappa}}
\newcommand{\Blambda}[0]{\boldsymbol{\lambda}}
\newcommand{\Bmu}[0]{\boldsymbol{\mu}}
\newcommand{\Bnu}[0]{\boldsymbol{\nu}}
%\newcommand{\Bomega}[0]{\boldsymbol{\omega}}
\newcommand{\Bphi}[0]{\boldsymbol{\phi}}
\newcommand{\Bpi}[0]{\boldsymbol{\pi}}
\newcommand{\Bpsi}[0]{\boldsymbol{\psi}}
\newcommand{\Brho}[0]{\boldsymbol{\rho}}
\newcommand{\Bsigma}[0]{\boldsymbol{\sigma}}
%\newcommand{\Btau}[0]{\boldsymbol{\tau}}
%\newcommand{\Btheta}[0]{\boldsymbol{\theta}}
\newcommand{\Bupsilon}[0]{\boldsymbol{\upsilon}}
\newcommand{\Bvarepsilon}[0]{\boldsymbol{\varepsilon}}
\newcommand{\Bvarphi}[0]{\boldsymbol{\varphi}}
\newcommand{\Bvarpi}[0]{\boldsymbol{\varpi}}
\newcommand{\Bvarrho}[0]{\boldsymbol{\varrho}}
\newcommand{\Bvarsigma}[0]{\boldsymbol{\varsigma}}
\newcommand{\Bvartheta}[0]{\boldsymbol{\vartheta}}
\newcommand{\Bxi}[0]{\boldsymbol{\xi}}
\newcommand{\Bzeta}[0]{\boldsymbol{\zeta}}
%
%</bold and dot greek symbols>
%<infrequent>
%
%\newcommand{\AreaOp}[1]{\AName_{#1}}
%\newcommand{\Babs}[0]{\abs{\BB}}
%\newcommand{\Bcap}[0]{\hat{\BB}}
%\newcommand{\BrPrimeRej}[0]{\rcap(\rcap \wedge \Br')}
%\newcommand{\CA}[0]{\mathcal{A}}
%\newcommand{\Cos}[1]{\cos{\left({#1}\right)}}
%\newcommand{\Det}[1] {\abs{#1}}
%\newcommand{\Dsq}[2] {\frac {\partial^2 {#1}} {\partial {#2}^2}}
%\newcommand{\Exp}[1]{\exp{\left({#1}\right)}}
%\newcommand{\Norm}[1]{\left\lVert{#1}\right\rVert}
%\newcommand{\Sin}[1]{\sin{\left({#1}\right)}}
%\newcommand{\T}[0]{\text{T}}
%\newcommand{\VolumeOp}[1]{\VName_{#1}}
%\newcommand{\agrad}[0]{\Ba \cdot \nabla}
%\newcommand{\alphacap}[0]{\hat{\boldsymbol{\alpha}}}
%\newcommand{\Fcap}[0]{\hat{\BF}}
%\newcommand{\bithree}[0]{{\Bi}_3}
%\newcommand{\bxa}[0]{\Bx\Ba}
%\newcommand{\coordvec}[2]{
%\newcommand{\costheta}[0]{\acap \cdot \xcap}
%\newcommand{\ddt}[1]{\ddot{#1}}
%\newcommand{\ddu}[1] {\frac {d{#1}} {du}}
%\newcommand{\dsqxj}[2] {\frac {\partial^2 {#1}} {\partial {x_{#2}}^2}}
%\newcommand{\dtheta}[1]{\frac{d {#1}}{d \theta}}
%\newcommand{\dt}[1]{\dot{#1}}
%\newcommand{\dt}[1]{\frac{d {#1}}{dt}}
%\newcommand{\dxj}[2] {\frac {\partial {#1}} {\partial {x_{#2}}}}
%\newcommand{\halfPhi}[0]{\frac{\phi}{2}}
%\newcommand{\half}[0]{\inv{2}}
%\newcommand{\inv}[1]{\frac{1}{#1}}
%\newcommand{\laplacian}[0]{\nabla^2}
%\newcommand{\matrixoftx}[3]{
%\newcommand{\nrrp}[0]{\norm{\rcap \wedge \Br'}}
%\newcommand{\oiint}{\bigcirc \hspace{-1.4em} \int \hspace{-.8em} \int}
%\newcommand{\transpose}[1]{{#1}^{\text{T}}}
%\newcommand{\transpose}[1]{{{#1}^{\TextTranspose}}}
%\newcommand{\transpose}[1]{{{#1}^{\text{T}}}}
%\newcommand{\barA}[0]{\bar{A}}
%\newcommand{\qbar}[0]{\bar{q}}
%\newcommand{\qdotbar}[0]{\dot{\bar{q}}}
%
%</infrequent>




\newcommand{\PDSq}[2]{\frac{\partial^2 {#2}}{\partial {#1}^2}}
\DeclareMathOperator{\sinc}{sinc}
\newcommand{\FF}[0]{\mathcal{F}}
\newcommand{\IIinf}[0]{ \int_{-\infty}^\infty }

\usepackage[bookmarks=true]{hyperref}

\usepackage{color,cite,graphicx}
   % use colour in the document, put your citations as [1-4]
   % rather than [1,2,3,4] (it looks nicer, and the extended LaTeX2e
   % graphics package. 
\usepackage{latexsym,amssymb,epsf} % don't remember if these are
   % needed, but their inclusion can't do any damage


\title{ Fourier Solutions to Heat and Wave equations. }
\author{Peeter Joot}
\date{ Jan 19, 2009.  Last Revision: $Date: 2009/01/21 03:39:10 $ }

\begin{document}

\maketitle{}

%\tableofcontents

\section{ Motivation. }

Stanford iTunesU has some Fourier transform lectures by Prof. Brad Osgood.
He starts with Fourier series and by Lecture 5 has covered this and
the solution of the Heat equation on a ring as an example.

Now, for these lectures I get only sound on my ipod.  I can listen along and
pick up most of the lectures since this is review material, but here's some
notes to firm things up.

Since this heat equation

\begin{align}
\grad^2 u = \kappa \partial_t u
\end{align}

is also the Schr\"{o}dinger equation for a free particle in one 
dimension (once the 
constant is fixed appropriately), we can also apply the Fourier
technique to a particle
constrained to a circle.  It would be interesting afterwards to 
contrast this with Susskind's solution of the
same problem (where he used the Fourier transform and algebraic techniques
instead).

\section{ Preliminaries. }

\subsection{ Laplacian. }

Osgood wrote the heat equation for the ring as

\begin{align*}
\inv{2} u_{xx} = u_t
\end{align*}

where $x$ represented an angular position on the ring, and where
he set the heat diffusion constant to $1/2$ for convienience.
To apply this to the Schr\"{o}dinger equation retaining all the desired
units we want to be a bit more careful, so let's start with the Laplacian
in polar coordinates.

In polar coordinates our gradient is

\begin{align*}
\grad = \thetacap \inv{r} \PD{\theta}{} +\rcap \PD{r}{} 
\end{align*}

squaring this we have

\begin{align*}
\grad^2 = \grad \cdot \grad
&= 
\thetacap \inv{r} \PD{\theta}{} \cdot \left(\thetacap \inv{r} \PD{\theta}{}\right)
 +
\rcap \PD{r}{} \cdot \left(\rcap \PD{r}{} \right) \\
&= 
\frac{-1}{r^3} \PD{\theta}{r} \PD{\theta}{}
+\inv{r^2} \PDSq{\theta}{}
+ \PDSq{r}{}
\\
&= \inv{r^2} \PDSq{\theta}{} + \PDSq{r}{} \\
\end{align*}

So for the circularly constrained where $r$ is constant case we have simply

\begin{align}
\grad^2 = \inv{r^2} \PDSq{\theta}{}
\end{align}

and our heat equation to solve becomes

\begin{align}
\PDSq{\theta}{u(\theta, t)} = (r^2 \kappa) \PD{t}{u(\theta, t)}
\end{align}

\subsection{ Fourier series. }

Now we also want Fourier series for a given period.  Assuming the absence of the "Rigor Police" as Osgood puts it
we write for a periodic function $f(x)$ known on the interval $I = [a, a+T]$

\begin{align*}
f(x) = \sum c_k e^{2\pi i k x/T}
\end{align*}

\begin{align*}
\int_{\partial I} f(x) e^{- 2 \pi i n x /T} 
&= \sum c_k \int_{\partial I} e^{2\pi i (k -n) x/T} \\
&= c_n T
\end{align*}

So our Fourier coefficient is
\begin{align*}
\hat{f}(n) = c_n = \inv{T} \int_{\partial I} f(x) e^{- 2 \pi i n x /T} 
\end{align*}

\section{ Solution of heat equation. } 

\subsection{ Basic solution. }

Now we are ready to solve the radial heat equation

\begin{align}\label{eqn:heatRadial}
u_{\theta\theta} = r^2 \kappa u_t,
\end{align}

by assuming a Fourier series solution.

Suppose

\begin{align*}
u(\theta, t) 
&= \sum c_n(t) e^{2 \pi i n \theta / T} \\
&= \sum c_n(t) e^{i n \theta} \\
\end{align*}

Taking derivatives of this assumed solution we have
\begin{align*}
u_{\theta\theta} &= \sum (i n)^2 c_n e^{i n \theta} \\
u_{t} &= \sum c_n' e^{i n \theta}
\end{align*}

Substituting this back into \ref{eqn:heatRadial} we have

\begin{align*}
\sum - n^2 c_n e^{ i n \theta} = \sum c_n' r^2 \kappa e^{i n \theta}
\end{align*}

equating components we have 

\begin{align*}
c_n' = - \frac{n^2}{ r^2 \kappa } c_n 
\end{align*}

which is also just an exponential.

\begin{align*}
c_n = A_n \exp\left(- \frac{n^2}{ r^2 \kappa } t \right)
\end{align*}

Reassembling we have the time variation of the solution now fixed and can write

\begin{align}
u(\theta, t) = \sum A_n \exp\left(- \frac{n^2}{ r^2 \kappa } t + i n \theta\right)
\end{align}

\subsection{ As initial value problem. }

For the heat equation case, we can assume a known initial heat distribution 
$f(\theta)$.
For an initial time $t=0$ we can then write

\begin{align*}
u(\theta, 0) = \sum A_n e^{i n \theta} = f(\theta)
\end{align*}

This is just another Fourier series, with Fourier coefficients

\begin{align*}
A_n = \inv{2\pi} \int_{\partial I} f(v) e^{-i n v} dv
\end{align*}

Final reassembly of the results gives us

\begin{align}
u(\theta, t) = \sum \exp\left(- \frac{n^2}{ r^2 \kappa } t + i n \theta\right) \inv{2\pi} \int_{\partial I} f(v) e^{-i n v} dv
\end{align}

\subsection{ Convolution. }

Osgood's next step, also with the rigor police in hiding, was to exchange orders of integration and summation, to write

\begin{align*}
u(\theta, t) 
&= 
\int_{\partial I} f(v) dv \inv{2 \pi} \sum_{n=-\infty}^{\infty} \exp\left(- \frac{n^2}{ r^2 \kappa } t -i n (v -\theta)\right) \\
\end{align*}

Introducing a Green's function $g(v, t)$, we then have the complete solution in terms of convolution

\begin{align}\label{eqn:seriesGreens}
g( v , t ) &= \inv{2 \pi} \sum_{n=-\infty}^\infty \exp\left(- \frac{n^2}{ r^2 \kappa } t -i n v \right) \\
u(\theta, t) &= \int_{\partial I} f(v) g(v - \theta, t) dv 
\end{align}

Now, this Green's function is fairly interesting.  By summing over paired negative and positive indexes, we have a set of
weighted Gaussians.

\begin{align*}
g( v , t ) &= \inv{2 \pi} + \sum_{n=1}^\infty \exp\left(- \frac{n^2}{ r^2 \kappa } t \right) \frac{\cos(n v )}{\pi} \\
\end{align*}

Recalling that the delta function can be expressed as a limit of a $\sinc$ function, seeing something similar
in this Green's function is not entirely unsuprising seeming.

\section{ Wave equation. }

The QM equation for a free particle is

\begin{align}\label{eqn:schro}
-\frac{\hbar^2}{2m} \grad^2 \psi = i \hbar \partial_t \psi
\end{align}

This has the same form of the heat equation, so for the free particle on a circle our wave equation is

\begin{align*}
\psi_{\theta\theta} = - \frac{2 m i r^2 }{\hbar} \partial_t \psi \quad \mbox{ ie: $\kappa = - 2 m i /\hbar$ }
\end{align*}

So, if the wave equation was known at an initial time $\psi(\theta, 0) = \phi(\theta)$, we therefore have by comparision the time evolution of the particle's wave function is

\begin{align*}
g( w, t ) &= \inv{2 \pi} + \sum_{n=1}^\infty \exp\left(- \frac{i \hbar n^2 t}{ 2 m r^2 } \right) \frac{\cos(n w )}{\pi} \\
\psi(\theta, t) &= \int_{\partial I} \phi(v) g(v - \theta, t) dv 
\end{align*}

%TODO: contrast this to a Fourier transform solution.  Also write this in terms of circular angular momentum since that appears natually in the Green's function.

\section{ Fourier transform solution. }

% also see example (brief on details)
%\href{http://zakuski.utsa.edu/~gokhman/ftp//courses/notes/heat.pdf}{ example of Fourier tx solution. }
Now, lets try this one dimensional heat problem with a Fourier transform instead to compare.  Here we don't try to start with an
assumed solution, but instead take the Fourier transform of both sides of the equation directly.

\begin{align*}
\FF(u_{xx}) = \kappa \FF(u_t)
\end{align*}

Let's start with the left hand side, where we can evaluate by integrating by parts

\begin{align*}
\FF(u_{xx}) 
&= \inv{\sqrt{2\pi}} \IIinf u_{xx}(x, t) e^{- 2 \pi i s x } dx \\
&= \inv{\sqrt{2\pi}} \IIinf \PD{x}{u_x(x, t)} e^{- 2 \pi i s x } dx \\
&= \inv{\sqrt{2\pi}} 
\left(
{\left. u_x(x, t) e^{- 2 \pi i s x } \right\vert}_{x= -\infty}^\infty
-( - 2 \pi i s ) \IIinf u_x(x, t) e^{- 2 \pi i s x } dx 
\right) \\
\end{align*}

So if we assume (or require) that the derivative of our unknown function $u$ is zero at infinity, and then similarily
require the function itself to be zero there, we have

\begin{align*}
\FF(u_{xx}) 
&= \inv{\sqrt{2\pi}} ( 2 \pi i s ) \IIinf \PD{x}{u_x(x, t)} e^{- 2 \pi i s x } dx  \\
&= \inv{\sqrt{2\pi}} ( 2 \pi i s )^2 \IIinf u(x, t) e^{- 2 \pi i s x } dx  \\
&= ( 2 \pi i s )^2 \FF(u)
\end{align*}

Now, for the time derivative.  We want

\begin{align*}
\FF(u_t) &= \inv{\sqrt{2\pi}} \IIinf u_t(x, t) e^{- 2 \pi i s x } dx \\
\end{align*}

But can pull the derivative out of the integral for
\begin{align*}
\FF(u_t)
&= \PD{t}{} \left(\inv{\sqrt{2\pi}} \IIinf u(x, t) e^{- 2 \pi i s x } dx \right) \\
&= \PD{t}{\FF(u)} 
\end{align*}

So, now we have an equation relating time derivatives only of the Fourier transformed solution.

Writing $\FF(u) = \hat{u}$ this is

\begin{align}\label{eqn:toSolveFreq}
( 2 \pi i s )^2 \hat{u} = \kappa \PD{t}{\hat{u}}
\end{align}

With a solution of

\begin{align*}
\hat{u} = A(s) e^{ -4 \pi^2 s^2 t/ \kappa }
\end{align*}

Here $A(s)$ is an arbitrary constant in time integration constant, which may depend on $s$ since it is a solution of our simpler freqency domain partial differential equation
\ref{eqn:toSolveFreq}.

Performing an inverse transform to recover $u(x,t)$ we thus have

\begin{align*}
u(x,t) 
&= \inv{\sqrt{2\pi}} \IIinf \hat{u} e^{2 \pi i x s } ds  \\
&= \inv{\sqrt{2\pi}} \IIinf A(s) e^{ -4 \pi^2 s^2 t/ \kappa } e^{2 \pi i x s } ds  \\
\end{align*}

Now, how about initial conditions.  Suppose we have $u(x,0) = f(x)$, then 

\begin{align*}
f(x) &= \inv{\sqrt{2\pi}} \IIinf A(s) e^{2 \pi i x s } ds \\
\end{align*}

Which is just an inverse Fourier transform in terms of the integration ``constant'' $A(s)$.  We can therefore write the $A(s)$ in terms of the
initial time domain conditions.

\begin{align*}
A(s) &= \inv{\sqrt{2\pi}} \IIinf f(x) e^{-2 \pi i s x } dx \\
&= \hat{f}(s)
\end{align*}

and finally have a complete solution of the one dimensional Heat equation.  That is

\begin{align*}
u(x,t) &= \inv{\sqrt{2\pi}} \IIinf \hat{f}(s) e^{ -4 \pi^2 s^2 t/ \kappa } e^{2 \pi i x s } ds  \\
\end{align*}

\subsection{ With Green's function? }

If we put in the integral for $\hat{f}(s)$ explicitly and switch the order as was done with the Fourier series will we get a similar result?   Let's try

\begin{align*}
u(x,t) 
&= \inv{\sqrt{2\pi}} \IIinf \left( \inv{\sqrt{2\pi}} \IIinf f(u) e^{-2 \pi i s u } du \right) e^{ -4 \pi^2 s^2 t/ \kappa } e^{2 \pi i x s } ds  \\
&= \inv{\sqrt{2\pi}} \IIinf du f(u) \inv{\sqrt{2\pi}} \IIinf e^{ -4 \pi^2 s^2 t/ \kappa } e^{2 \pi i (x - u) s } ds  \\
\end{align*}

Cool.  So, with the introduction of a Green's function $g(w,t)$ for the fundamental solution of the heat equation, we therefore have
our solution in terms of convolution with the initial conditions.  It doesn't get any more general than this!

\begin{align}
g(w,t) &= \inv{{2\pi}} \IIinf \exp\left( -\frac{4 \pi^2 s^2 t}{\kappa} + 2 \pi i w s \right) ds \\
u(x,t) &= \IIinf f(u) g( x - u, t) du
\end{align}

Compare this to \ref{eqn:seriesGreens}, the solution in terms of Fourier series.  The form is almost identical, but the requirement for periodicity has been removed by switch to the continuous frequency domain!

\subsection{ Wave equation. }

With only a change of variables, setting $\kappa = - 2 m i /\hbar$ we have the general solution to the one dimensional zero potential wave equation 
\ref{eqn:schro}
in terms of an initial wave function.

Our solution is 

\begin{align}
g(w,t) &= \inv{{2\pi}} \IIinf \exp\left( 2 \pi i s \left(w -\frac{\pi s t \hbar}{m} \right)\right) ds \\
u(x,t) &= \IIinf f(u) g( x - u, t) du
\end{align}

%\bibliographystyle{plainnat}
%\bibliography{myrefs}

\end{document}
