\documentclass{article}

\usepackage{amsmath}
\usepackage{mathpazo}

%
% shorthand for bold symbols, convenient for vectors and matrices
%
\newcommand{\Ba}[0]{\mathbf{a}}
\newcommand{\Bb}[0]{\mathbf{b}}
\newcommand{\Bc}[0]{\mathbf{c}}
\newcommand{\Bd}[0]{\mathbf{d}}
\newcommand{\Be}[0]{\mathbf{e}}
\newcommand{\Bf}[0]{\mathbf{f}}
\newcommand{\Bg}[0]{\mathbf{g}}
\newcommand{\Bh}[0]{\mathbf{h}}
\newcommand{\Bi}[0]{\mathbf{i}}
\newcommand{\Bj}[0]{\mathbf{j}}
\newcommand{\Bk}[0]{\mathbf{k}}
\newcommand{\Bl}[0]{\mathbf{l}}
\newcommand{\Bm}[0]{\mathbf{m}}
\newcommand{\Bn}[0]{\mathbf{n}}
\newcommand{\Bo}[0]{\mathbf{o}}
\newcommand{\Bp}[0]{\mathbf{p}}
\newcommand{\Bq}[0]{\mathbf{q}}
\newcommand{\Br}[0]{\mathbf{r}}
\newcommand{\Bs}[0]{\mathbf{s}}
\newcommand{\Bt}[0]{\mathbf{t}}
\newcommand{\Bu}[0]{\mathbf{u}}
\newcommand{\Bv}[0]{\mathbf{v}}
\newcommand{\Bw}[0]{\mathbf{w}}
\newcommand{\Bx}[0]{\mathbf{x}}
\newcommand{\By}[0]{\mathbf{y}}
\newcommand{\Bz}[0]{\mathbf{z}}
\newcommand{\BA}[0]{\mathbf{A}}
\newcommand{\BB}[0]{\mathbf{B}}
\newcommand{\BC}[0]{\mathbf{C}}
\newcommand{\BD}[0]{\mathbf{D}}
\newcommand{\BE}[0]{\mathbf{E}}
\newcommand{\BF}[0]{\mathbf{F}}
\newcommand{\BG}[0]{\mathbf{G}}
\newcommand{\BH}[0]{\mathbf{H}}
\newcommand{\BI}[0]{\mathbf{I}}
\newcommand{\BJ}[0]{\mathbf{J}}
\newcommand{\BK}[0]{\mathbf{K}}
\newcommand{\BL}[0]{\mathbf{L}}
\newcommand{\BM}[0]{\mathbf{M}}
\newcommand{\BN}[0]{\mathbf{N}}
\newcommand{\BO}[0]{\mathbf{O}}
\newcommand{\BP}[0]{\mathbf{P}}
\newcommand{\BQ}[0]{\mathbf{Q}}
\newcommand{\BR}[0]{\mathbf{R}}
\newcommand{\BS}[0]{\mathbf{S}}
\newcommand{\BT}[0]{\mathbf{T}}
\newcommand{\BU}[0]{\mathbf{U}}
\newcommand{\BV}[0]{\mathbf{V}}
\newcommand{\BW}[0]{\mathbf{W}}
\newcommand{\BX}[0]{\mathbf{X}}
\newcommand{\BY}[0]{\mathbf{Y}}
\newcommand{\BZ}[0]{\mathbf{Z}}

\newcommand{\Bzero}[0]{\mathbf{0}}
\newcommand{\Btheta}[0]{\boldsymbol{\theta}}
\newcommand{\Btau}[0]{\boldsymbol{\tau}}
\newcommand{\Bomega}[0]{\boldsymbol{\omega}}

%
% shorthand for unit vectors
%
\newcommand{\acap}[0]{\hat{\Ba}}
\newcommand{\bcap}[0]{\hat{\Bb}}
\newcommand{\ccap}[0]{\hat{\Bc}}
\newcommand{\dcap}[0]{\hat{\Bd}}
\newcommand{\ecap}[0]{\hat{\Be}}
\newcommand{\fcap}[0]{\hat{\Bf}}
\newcommand{\gcap}[0]{\hat{\Bg}}
\newcommand{\hcap}[0]{\hat{\Bh}}
\newcommand{\icap}[0]{\hat{\Bi}}
\newcommand{\jcap}[0]{\hat{\Bj}}
\newcommand{\kcap}[0]{\hat{\Bk}}
\newcommand{\lcap}[0]{\hat{\Bl}}
\newcommand{\mcap}[0]{\hat{\Bm}}
\newcommand{\ncap}[0]{\hat{\Bn}}
\newcommand{\ocap}[0]{\hat{\Bo}}
\newcommand{\pcap}[0]{\hat{\Bp}}
\newcommand{\qcap}[0]{\hat{\Bq}}
\newcommand{\rcap}[0]{\hat{\Br}}
\newcommand{\scap}[0]{\hat{\Bs}}
\newcommand{\tcap}[0]{\hat{\Bt}}
\newcommand{\ucap}[0]{\hat{\Bu}}
\newcommand{\vcap}[0]{\hat{\Bv}}
\newcommand{\wcap}[0]{\hat{\Bw}}
\newcommand{\xcap}[0]{\hat{\Bx}}
\newcommand{\ycap}[0]{\hat{\By}}
\newcommand{\zcap}[0]{\hat{\Bz}}
\newcommand{\thetacap}[0]{\hat{\Btheta}}

%
% to write R^n and C^n in a distinguishable fashion.  Perhaps change this
% to the double lined characters upon figuring out how to do so.
%
\newcommand{\C}[1]{$\mathbb{C}^{#1}$}
\newcommand{\R}[1]{$\mathbb{R}^{#1}$}

%
% various generally useful helpers
%

% derivative of #1 wrt. #2:
\newcommand{\D}[2] {\frac {d#2} {d#1}}

\newcommand{\inv}[1]{\frac{1}{#1}}
\newcommand{\cross}[0]{\times}

\newcommand{\abs}[1]{\lvert{#1}\rvert}
\newcommand{\norm}[1]{\lVert{#1}\rVert}
\newcommand{\innerprod}[2]{\langle{#1}, {#2}\rangle}
\newcommand{\dotprod}[2]{{#1} \cdot {#2}}
\newcommand{\bdotprod}[2]{\left({#1} \cdot {#2}\right)}
\newcommand{\crossprod}[2]{{#1} \cross {#2}}
\newcommand{\tripleprod}[3]{\dotprod{\left(\crossprod{#1}{#2}\right)}{#3}}

\DeclareMathOperator{\Proj}{Proj}
\DeclareMathOperator{\Span}{span}
\DeclareMathOperator{\Sgn}{sgn}
\DeclareMathOperator{\Area}{Area}
\DeclareMathOperator{\Volume}{Volume}

%
% A few miscellaneous things specific to this document
%
\newcommand{\crossop}[1]{\crossprod{#1}{}}

% R2 vector.
\newcommand{\VectorTwo}[2]{
\begin{bmatrix}
 {#1} \\
 {#2}
\end{bmatrix}
}

\newcommand{\VectorN}[1]{
\begin{bmatrix}
{#1}_1 \\
{#1}_2 \\
\vdots \\
{#1}_N \\
\end{bmatrix}
}

\newcommand{\DETuvij}[4]{
\begin{vmatrix}
 {#1}_{#3} & {#1}_{#4} \\
 {#2}_{#3} & {#2}_{#4}
\end{vmatrix}
}

\newcommand{\DETuvwijk}[6]{
\begin{vmatrix}
 {#1}_{#4} & {#1}_{#5} & {#1}_{#6} \\
 {#2}_{#4} & {#2}_{#5} & {#2}_{#6} \\
 {#3}_{#4} & {#3}_{#5} & {#3}_{#6}
\end{vmatrix}
}

\newcommand{\DETuvwxijkl}[8]{
\begin{vmatrix}
 {#1}_{#5} & {#1}_{#6} & {#1}_{#7} & {#1}_{#8} \\
 {#2}_{#5} & {#2}_{#6} & {#2}_{#7} & {#2}_{#8} \\
 {#3}_{#5} & {#3}_{#6} & {#3}_{#7} & {#3}_{#8} \\
 {#4}_{#5} & {#4}_{#6} & {#4}_{#7} & {#4}_{#8} \\
\end{vmatrix}
}

%\newcommand{\DETuvwxyijklm}[10]{
%\begin{vmatrix}
% {#1}_{#6} & {#1}_{#7} & {#1}_{#8} & {#1}_{#9} & {#1}_{#10} \\
% {#2}_{#6} & {#2}_{#7} & {#2}_{#8} & {#2}_{#9} & {#2}_{#10} \\
% {#3}_{#6} & {#3}_{#7} & {#3}_{#8} & {#3}_{#9} & {#3}_{#10} \\
% {#4}_{#6} & {#4}_{#7} & {#4}_{#8} & {#4}_{#9} & {#4}_{#10} \\
% {#5}_{#6} & {#5}_{#7} & {#5}_{#8} & {#5}_{#9} & {#5}_{#10}
%\end{vmatrix}
%}

% R3 vector.
\newcommand{\VectorThree}[3]{
\begin{bmatrix}
 {#1} \\
 {#2} \\
 {#3}
\end{bmatrix}
}


\newcommand{\grad}[0]{\nabla}
\newcommand{\spacegrad}[0]{\boldsymbol{\nabla}}
\newcommand{\LL}[0]{\mathcal{L}}
\newcommand{\xdot}[0]{\dot{x}}
\newcommand{\xddot}[0]{\ddot{x}}
\newcommand{\pdot}[0]{\dot{p}}
\newcommand{\pddot}[0]{\ddot{p}}

\usepackage[bookmarks=true]{hyperref}

\title{ Revisit Lorentz force from Lagrangian. }
\author{Peeter Joot}
\date{ October 8, 2008.  Last Revision: $Date: 2008/10/09 04:24:58 $ }

\begin{document}

\maketitle{}
%\tableofcontents
%\section{ Motivation }

In \cite{PJSrLagrangian} a derivation of the Lorentz force in covariant
form was performed.  Intuition says that result, because of the squared
proper velocity, was dependent on the 
positive time Minkowski signature.  This signature is the common convention
in \cite{doran2003gap}, using Hestenes' STA relativistic formulation.  With many
GR references using the opposite signature, it seems worthwhile to understand
what results are signature dependent and put them in a signature invariant form.

Here the result will be rederived without assuming this signature.

Assume a Lagrangian of the following form

\begin{align}
\LL = \inv{2}m v^2 + \kappa A \cdot v
\end{align}

where $v$ is the proper velocity.  Here $A(x^\mu,\xdot^\nu) = A(x^\mu)$ is a position but not velocity dependent four vector potential.  The constant $\kappa$ includes the charge of the test mass, and will be determined exactly in due course.

As observed in \cite{PJCanMomentum} the equations of motion can be written

\begin{align}
\grad \LL = \frac{d}{d\tau}(\grad_v \LL)
\end{align}

We have

\begin{align*}
\grad v^2 &= 0 \\
\grad (A \cdot v)
&= \grad A_\mu \xdot^\mu \\
&= \gamma^\nu \xdot^\mu \partial_\nu A_\mu \\
\inv{2} \grad_v v^2 
&= \inv{2} \grad_v (\gamma_\mu)^2 (\xdot^\mu)^2 \\
&= \gamma^\nu (\gamma_\mu)^2 \partial_{\xdot^\nu} \xdot^\mu \\
&= \gamma^\mu (\gamma_\mu)^2 \xdot^\mu \\
&= \gamma_\mu \xdot^\mu \\
&= v \\
\grad_v (A \cdot v)
&= \gamma^\nu \partial_{\xdot^\nu} A_\mu \xdot^\mu \\
&= \gamma^\nu A_\mu {\delta^\mu}_\nu \\
&= \gamma^\mu A_\mu \\
&= A \\
\frac{d}{d\tau} &= \xdot^\mu \partial_\mu
\end{align*}

Putting all this back together 
\begin{align*}
\grad \LL &= \frac{d}{d\tau}(\grad_v \LL) \\
\kappa \gamma^\nu \xdot^\mu \partial_\nu A_\mu &= \frac{d}{d\tau}\left( m v + \kappa A \right) \\
\implies \\
\pdot 
&= \kappa \left( \gamma^\nu \xdot^\mu \partial_\nu A_\mu - \xdot^\nu \partial_\nu \gamma^\mu A_\mu \right) \\
&= \kappa \partial_\nu A_\mu \left( \gamma^\nu \xdot^\mu - \xdot^\nu \gamma^\mu \right) \\
\end{align*}

We know this will be related to $F \cdot v$, where $F = \grad \wedge A$.  Expanding that for comparision

\begin{align*}
F \cdot v
&= (\grad \wedge A) \cdot v \\
&= (\gamma^\mu \wedge \gamma^\nu) \cdot \gamma_\alpha \xdot^\alpha \partial_\mu A_\nu \\
&= \left( \gamma^\mu {\delta^\nu}_\alpha -\gamma^\nu {\delta^\mu}_\alpha \right) \xdot^\alpha \partial_\mu A_\nu \\
&= 
\gamma^\mu \xdot^\nu \partial_\mu A_\nu 
-\gamma^\nu \xdot^\mu \partial_\mu A_\nu \\
&= \partial_\nu A_\mu \left( \gamma^\nu \xdot^\mu -\gamma^\mu \xdot^\nu \right) \\
\end{align*}

Comparison shows that we therefore have

\begin{align}\label{eqn:lorentzUndeterminedConst}
\pdot = \kappa F \cdot v
\end{align}

A reasonable approach to fix the constant $\kappa$ is to put this into correspondance with the classical
vector form of the Lorentz force equation.

Introduce a rest observer, with worldline $x = ct e_0$.  Computation of the spatial parts of the four vector force equation \ref{eqn:lorentzUndeterminedConst} for this rest observer requires taking the wedge product
with the observer velocity $v = c \gamma e_0$.  This will discard the timelike components of the force equation with
respect to this observer rest frame, and leave only the 
purely spatial (using Euclidian vector-like spatial bivector basis $\{\sigma_i = e_i \wedge e_0\}$) components
for that observer.  For clarity, for the observer frame we use a different set of basis vectors $\{e_\mu\}$, to point
out that $\gamma_0$ of the derivation above does not have to equal $e_0$.  Since the end result of the Lagrangian calculation
ended up being coordinate and signature free, this is perhaps superfluous.

Omitting the scale factor $\gamma = dt/d\tau$ for now, application of a wedge with $e_0$ operation to both sides 
will suffice to determine this observer dependent expression of the force

\bibliographystyle{plainnat} % supposed to allow for \url use.
\bibliography{myrefs}      % expects file "myrefs.bib"

\end{document}               % End of document.
