\documentclass{article}      % Specifies the document class

\usepackage{color,cite,graphicx}
   % use colour in the document, put your citations as [1-4]
   % rather than [1,2,3,4] (it looks nicer, and the extended LaTeX2e
   % graphics package. 
\usepackage{latexsym,amssymb,epsf} % don't remember if these are
   % needed, but their inclusion can't do any damage

\usepackage{amsmath}
\usepackage{mathpazo}

%
% shorthand for bold symbols, convenient for vectors and matrices
%
\newcommand{\Ba}[0]{\mathbf{a}}
\newcommand{\Bb}[0]{\mathbf{b}}
\newcommand{\Bc}[0]{\mathbf{c}}
\newcommand{\Bd}[0]{\mathbf{d}}
\newcommand{\Be}[0]{\mathbf{e}}
\newcommand{\Bf}[0]{\mathbf{f}}
\newcommand{\Bg}[0]{\mathbf{g}}
\newcommand{\Bh}[0]{\mathbf{h}}
\newcommand{\Bi}[0]{\mathbf{i}}
\newcommand{\Bj}[0]{\mathbf{j}}
\newcommand{\Bk}[0]{\mathbf{k}}
\newcommand{\Bl}[0]{\mathbf{l}}
\newcommand{\Bm}[0]{\mathbf{m}}
\newcommand{\Bn}[0]{\mathbf{n}}
\newcommand{\Bo}[0]{\mathbf{o}}
\newcommand{\Bp}[0]{\mathbf{p}}
\newcommand{\Bq}[0]{\mathbf{q}}
\newcommand{\Br}[0]{\mathbf{r}}
\newcommand{\Bs}[0]{\mathbf{s}}
\newcommand{\Bt}[0]{\mathbf{t}}
\newcommand{\Bu}[0]{\mathbf{u}}
\newcommand{\Bv}[0]{\mathbf{v}}
\newcommand{\Bw}[0]{\mathbf{w}}
\newcommand{\Bx}[0]{\mathbf{x}}
\newcommand{\By}[0]{\mathbf{y}}
\newcommand{\Bz}[0]{\mathbf{z}}
\newcommand{\BA}[0]{\mathbf{A}}
\newcommand{\BB}[0]{\mathbf{B}}
\newcommand{\BC}[0]{\mathbf{C}}
\newcommand{\BD}[0]{\mathbf{D}}
\newcommand{\BE}[0]{\mathbf{E}}
\newcommand{\BF}[0]{\mathbf{F}}
\newcommand{\BG}[0]{\mathbf{G}}
\newcommand{\BH}[0]{\mathbf{H}}
\newcommand{\BI}[0]{\mathbf{I}}
\newcommand{\BJ}[0]{\mathbf{J}}
\newcommand{\BK}[0]{\mathbf{K}}
\newcommand{\BL}[0]{\mathbf{L}}
\newcommand{\BM}[0]{\mathbf{M}}
\newcommand{\BN}[0]{\mathbf{N}}
\newcommand{\BO}[0]{\mathbf{O}}
\newcommand{\BP}[0]{\mathbf{P}}
\newcommand{\BQ}[0]{\mathbf{Q}}
\newcommand{\BR}[0]{\mathbf{R}}
\newcommand{\BS}[0]{\mathbf{S}}
\newcommand{\BT}[0]{\mathbf{T}}
\newcommand{\BU}[0]{\mathbf{U}}
\newcommand{\BV}[0]{\mathbf{V}}
\newcommand{\BW}[0]{\mathbf{W}}
\newcommand{\BX}[0]{\mathbf{X}}
\newcommand{\BY}[0]{\mathbf{Y}}
\newcommand{\BZ}[0]{\mathbf{Z}}

\newcommand{\Bzero}[0]{\mathbf{0}}
\newcommand{\Btheta}[0]{\boldsymbol{\theta}}
\newcommand{\Btau}[0]{\boldsymbol{\tau}}
\newcommand{\Bomega}[0]{\boldsymbol{\omega}}

%
% shorthand for unit vectors
%
\newcommand{\acap}[0]{\hat{\Ba}}
\newcommand{\bcap}[0]{\hat{\Bb}}
\newcommand{\ccap}[0]{\hat{\Bc}}
\newcommand{\dcap}[0]{\hat{\Bd}}
\newcommand{\ecap}[0]{\hat{\Be}}
\newcommand{\fcap}[0]{\hat{\Bf}}
\newcommand{\gcap}[0]{\hat{\Bg}}
\newcommand{\hcap}[0]{\hat{\Bh}}
\newcommand{\icap}[0]{\hat{\Bi}}
\newcommand{\jcap}[0]{\hat{\Bj}}
\newcommand{\kcap}[0]{\hat{\Bk}}
\newcommand{\lcap}[0]{\hat{\Bl}}
\newcommand{\mcap}[0]{\hat{\Bm}}
\newcommand{\ncap}[0]{\hat{\Bn}}
\newcommand{\ocap}[0]{\hat{\Bo}}
\newcommand{\pcap}[0]{\hat{\Bp}}
\newcommand{\qcap}[0]{\hat{\Bq}}
\newcommand{\rcap}[0]{\hat{\Br}}
\newcommand{\scap}[0]{\hat{\Bs}}
\newcommand{\tcap}[0]{\hat{\Bt}}
\newcommand{\ucap}[0]{\hat{\Bu}}
\newcommand{\vcap}[0]{\hat{\Bv}}
\newcommand{\wcap}[0]{\hat{\Bw}}
\newcommand{\xcap}[0]{\hat{\Bx}}
\newcommand{\ycap}[0]{\hat{\By}}
\newcommand{\zcap}[0]{\hat{\Bz}}
\newcommand{\thetacap}[0]{\hat{\Btheta}}

%
% to write R^n and C^n in a distinguishable fashion.  Perhaps change this
% to the double lined characters upon figuring out how to do so.
%
\newcommand{\C}[1]{$\mathbb{C}^{#1}$}
\newcommand{\R}[1]{$\mathbb{R}^{#1}$}

%
% various generally useful helpers
%

% derivative of #1 wrt. #2:
\newcommand{\D}[2] {\frac {d#2} {d#1}}

\newcommand{\inv}[1]{\frac{1}{#1}}
\newcommand{\cross}[0]{\times}

\newcommand{\abs}[1]{\lvert{#1}\rvert}
\newcommand{\norm}[1]{\lVert{#1}\rVert}
\newcommand{\innerprod}[2]{\langle{#1}, {#2}\rangle}
\newcommand{\dotprod}[2]{{#1} \cdot {#2}}
\newcommand{\bdotprod}[2]{\left({#1} \cdot {#2}\right)}
\newcommand{\crossprod}[2]{{#1} \cross {#2}}
\newcommand{\tripleprod}[3]{\dotprod{\left(\crossprod{#1}{#2}\right)}{#3}}

\DeclareMathOperator{\Proj}{Proj}
\DeclareMathOperator{\Span}{span}
\DeclareMathOperator{\Sgn}{sgn}
\DeclareMathOperator{\Area}{Area}
\DeclareMathOperator{\Volume}{Volume}

%
% A few miscellaneous things specific to this document
%
\newcommand{\crossop}[1]{\crossprod{#1}{}}

% R2 vector.
\newcommand{\VectorTwo}[2]{
\begin{bmatrix}
 {#1} \\
 {#2}
\end{bmatrix}
}

\newcommand{\VectorN}[1]{
\begin{bmatrix}
{#1}_1 \\
{#1}_2 \\
\vdots \\
{#1}_N \\
\end{bmatrix}
}

\newcommand{\DETuvij}[4]{
\begin{vmatrix}
 {#1}_{#3} & {#1}_{#4} \\
 {#2}_{#3} & {#2}_{#4}
\end{vmatrix}
}

\newcommand{\DETuvwijk}[6]{
\begin{vmatrix}
 {#1}_{#4} & {#1}_{#5} & {#1}_{#6} \\
 {#2}_{#4} & {#2}_{#5} & {#2}_{#6} \\
 {#3}_{#4} & {#3}_{#5} & {#3}_{#6}
\end{vmatrix}
}

\newcommand{\DETuvwxijkl}[8]{
\begin{vmatrix}
 {#1}_{#5} & {#1}_{#6} & {#1}_{#7} & {#1}_{#8} \\
 {#2}_{#5} & {#2}_{#6} & {#2}_{#7} & {#2}_{#8} \\
 {#3}_{#5} & {#3}_{#6} & {#3}_{#7} & {#3}_{#8} \\
 {#4}_{#5} & {#4}_{#6} & {#4}_{#7} & {#4}_{#8} \\
\end{vmatrix}
}

%\newcommand{\DETuvwxyijklm}[10]{
%\begin{vmatrix}
% {#1}_{#6} & {#1}_{#7} & {#1}_{#8} & {#1}_{#9} & {#1}_{#10} \\
% {#2}_{#6} & {#2}_{#7} & {#2}_{#8} & {#2}_{#9} & {#2}_{#10} \\
% {#3}_{#6} & {#3}_{#7} & {#3}_{#8} & {#3}_{#9} & {#3}_{#10} \\
% {#4}_{#6} & {#4}_{#7} & {#4}_{#8} & {#4}_{#9} & {#4}_{#10} \\
% {#5}_{#6} & {#5}_{#7} & {#5}_{#8} & {#5}_{#9} & {#5}_{#10}
%\end{vmatrix}
%}

% R3 vector.
\newcommand{\VectorThree}[3]{
\begin{bmatrix}
 {#1} \\
 {#2} \\
 {#3}
\end{bmatrix}
}



\DeclareMathOperator{\Rej}{Rej}
\DeclareMathOperator{\Exp}{e}

%
% The real thing:
%

                             % The preamble begins here.
\title{ Rotor Notes. } % Declares the document's title.
\author{Peeter Joot}         % Declares the author's name.
%\date{}        % Deleting this command produces today's date.

\begin{document}             % End of preamble and beginning of text.

\maketitle{}

\section{ Rotations strictly in a plane. }

For a plane rotation, a rotation does not have to
be expressed in terms of left and right half angle rotations, as is the case
with complex numbers.  Starting with this ``natural'' one sided rotation
we will see why the half angle double sided Rotor formula works.

\subsection{ Identifying a plane with a bivector.  Justification. }
Given a bivector $\BB$, we can say this defines the orientation of a plane
(through the origin)
since for any vector in the plane we have $\BB \wedge \Bx = 0$, or any vector
strictly normal to the plane $\BB \cdot \Bx = 0$.

Note that this naturally compares
to the equation of a line (through the origin) expressed in terms of a
direction vector $\Bb$,
where $\Bb \wedge \Bx=0$ if $\Bx$ lies on the line, and $\Bb \cdot \Bx = 0$
if $\Bx$ is normal to the line.

Given this it's not unreasonable to identify the plane with its bivector.  This
will be done below, and it should be clear that 
loose language such as ``the plane $\BB$'', should really be interpretted
as ``the plane with direction bivector $\BB$'', where the direction bivector
has the wedge and dot product properties noted above.

\subsection{ Components of a vector in and out of a plane. }

To calculate the components of a vector in and out of a plane, we can form
the product

\[
\Bx = \Bx \BB \inv{\BB} = \Bx \cdot \BB \inv{\BB} + \Bx \wedge \BB \inv{\BB}
\]

This is an orthogonal decomposition of the vector $\Bx$ where the first
part is the projective term onto the plane $\BB$, and the second is the rejective
term, the component not in the plane.  Let's verify this.

Write $\Bx = \Bx_\parallel + \Bx_\perp$, where $\Bx_\parallel$, and $\Bx_\perp$ are the components of $\Bx$ parallel and perpenidular to the plane.  Also write
$\BB = \Bb_1 \wedge \Bb_2$, where $\Bb_i$ are non-colinear vectors in the plane $\BB$.

If $\Bx = \Bx_\parallel$, a vector entirely in the plane $\BB$, then one can
write

\[
\Bx = a_1\Bb_1 + a_2\Bb_2
\]

and the wedge product term is zero

\begin{align*}
\Bx \wedge \BB 
&= \left( a_1\Bb_1 + a_2\Bb_2 \right) \wedge \Bb_1 \wedge \Bb_2 \\
&= a_1 ( \Bb_1 \wedge \Bb_1 ) \wedge \Bb_2
 - a_2 ( \Bb_2 \wedge \Bb_2 ) \wedge \Bb_1 \\
&= 0
\end{align*}

Thus the component parallel to the plane is composed strictly of the dot
product term

\begin{equation}
\Bx_\parallel = \Bx \cdot \BB \inv{\BB}
\end{equation}

Or for a general vector not neccessarily in the plane the component
of that vector in the plane, it's projection onto the plane is,

\[
\Proj_{\BB}(\Bx) = \Bx \cdot \BB \inv{\BB}
= \inv{\abs{\BB}^2}(\BB \cdot \Bx)\BB
= (\hat{\BB} \cdot \Bx)\hat{\BB}
\]

Now, for a vector that lies completely perpendicular to the plane $\Bx = \Bx_\perp$, the dot product term with the plane is zero.  To verify this observe

\begin{align*}
\Bx_\perp \cdot \BB 
&= \Bx_\perp \cdot (\Bb_1 \wedge \Bb_2) \\
&= (\Bx_\perp \cdot \Bb_1) \Bb_2 - (\Bx_\perp \cdot \Bb_2) \Bb_1 \\
\end{align*}

Each of these dot products are zero since $\Bx$ has no components that lie
in the plane (those components if they existed could be expressed as linear
combinations of $\Bb_i$).

Thus only the component perpendicular to the plane is composed strictly of the 
wedge product term

\begin{equation}
\Bx_\perp = \Bx \wedge \BB \inv{\BB}
\end{equation}

And again for a general vector we can write it's component that lies out
of the plane as, the rejection of the plane from the vector as,

\[
\Rej_{\BB}(\Bx) 
= \Bx \wedge \BB \inv{\BB}
= -\inv{\abs{\BB}^2} \Bx \wedge \BB {\BB}
= -\Bx \wedge \hat{\BB} \hat{\BB}
\]

\section{ Rotation around normal to arbitrarily oriented plane through origin }

Having established the preliminaries, we can now express a rotation around
the normal to a plane (with the plane and that normal through the origin).

\begin{figure}[htp]
\centering
\includegraphics[totalheight=0.4\textheight]{rotor}
\caption{Rotation of Vector}\label{fig:rotor}
\end{figure}

Such a rotation is illustrated in figure \ref{fig:rotor}
preserves all components of the vector that are perpendicular
to the plane, and operates only on the components parallel to the plane.

Expressed in terms of exponentials and the projective and rejective decompositions above, this is

\begin{align*}
R_\theta(\Bx) 
&= \Bx \wedge \BB \inv{\BB} + \left(\Bx \cdot \BB \inv{\BB}\right)\Exp^{\hat{\BB}\theta} \\
&= \Bx \wedge \BB \inv{\BB} + \Exp^{-\hat{\BB}\theta}\left(\Bx \cdot \BB \inv{\BB}\right) \\
\end{align*}

Where we have made explicit note that a plane rotation does not commute with a vector in a plane (it's reverse is required).

To demonstrate this write $i = \Be_2 \Be_1$, a unit bivector in some plane with unit vectors $\Be_i$ also in the plane.  If a vector
lies in that plane we can write the rotation 

\begin{align*}
\Bx \Exp^{i\theta} 
&= \left(a_1\Be_1 + a_2\Be_2\right)\left(\cos\theta + i\sin\theta\right) \\
&= \cos\theta\left(a_1\Be_1 + a_2\Be_2\right) + \left(a_1\Be_1 + a_2\Be_2\right)\left(\Be_2 \Be_1\sin\theta\right) \\
&= \cos\theta\left(a_1\Be_1 + a_2\Be_2\right) + \sin\theta \left(-a_1\Be_2 + a_2\Be_1\right) \\
&= \cos\theta\left(a_1\Be_1 + a_2\Be_2\right) -\Be_2 \Be_1\sin\theta \left(a_1\Be_1 + a_2\Be_2\right) \\
&= \Exp^{-i\theta}\Bx \\
\end{align*}

Similarily for a vector that lies outside of the plane we can write

\begin{align*}
\Bx \Exp^{i\theta} 
&= (\sum_{j \ne 1,2} a_j \Be_j)(\cos\theta + \Be_2 \Be_1\sin\theta) \\
&= (\cos\theta + \Be_2 \Be_1\sin\theta) (\sum_{j \ne 1,2} a_j \Be_j) \\
&= \Exp^{i\theta}\Bx
\end{align*}

The multivector for a rotation in a plane perpentidular to a vector commutes with that vector.  The properties of the 
exponential allow us to factor a rotation

\[
R(\theta) = R(\alpha\theta) R((1-\alpha)\theta)
\]

where $\alpha <= 1$, and in particular we can set $\alpha = 1/2$, and write

\begin{align*}
R_\theta(\Bx) 
&= \Bx \wedge \BB \inv{\BB} + \left(\Bx \cdot \BB \inv{\BB}\right)\Exp^{\hat{\BB}\theta} \\
&= \left(\Bx \wedge \BB \inv{\BB}\right) \Exp^{-\hat{\BB}\theta/2} \Exp^{\hat{\BB}\theta/2} 
 + \left(\Bx \cdot \BB \inv{\BB} \right) \Exp^{\hat{\BB}\theta/2} \Exp^{\hat{\BB}\theta/2} \\
&= \Exp^{-\hat{\BB}\theta/2} \left(\Bx \wedge \BB \inv{\BB}\right) \Exp^{\hat{\BB}\theta/2}  
+ \Exp^{-\hat{\BB}\theta/2} \left(\Bx \cdot \BB \inv{\BB}\right)\Exp^{\hat{\BB}\theta/2} \\
&= \Exp^{-\hat{\BB}\theta/2} \left(\Bx \wedge \BB + \Bx \cdot \BB\right) \inv{\BB} \Exp^{\hat{\BB}\theta/2} \\
&= \Exp^{-\hat{\BB}\theta/2} \left(\Bx \BB \inv{\BB} \right) \Exp^{\hat{\BB}\theta/2}  
\end{align*}

This takes us full circle from dot and wedge products back to $\Bx$, and allows us to express the rotated vector as:

\begin{equation}\label{eqn:rotor}
R_\theta(\Bx) 
= \Exp^{-\hat{\BB}\theta/2} \Bx \Exp^{\hat{\BB}\theta/2}
\end{equation}

Only when the vector lies in the plane ($\Bx = \Bx_\parallel$, or $\Bx \wedge \BB = 0$) can be written using the familiar left or right ``full angle'' rotation exponential that we are used to from complex arithmetric:

\[
R_\theta(\Bx) = \Exp^{-\hat{\BB}\theta} \Bx = \Bx \Exp^{\hat{\BB}\theta}
\]

\section{ Rotor equation in terms of normal to plane. }

The rotor equation above is valid for any number of dimensions.  For \R{3} we can alternatively parameterize the plane in terms of
a unit normal $\Bn$:

\[
\BB = k i\Bn
\]

Here $i$ is the \R{3} pseudoscalar $\Be_1 \Be_2 \Be_3$.

Thus we can write

\[
\hat{\BB} = i\Bn
\]

and expressing \ref{eqn:rotor} in terms of the unit normal becomes trivial

\begin{equation}
R_\theta(\Bx) 
= \Exp^{- i {\Bn}\theta/2} \Bx \Exp^{i{\Bn}\theta/2}
\end{equation}

Expressing this in terms of components and the unit normal is a bit harder

\begin{align*}
R_\theta(\Bx) 
&= \Bx \wedge \BB \inv{\BB} + \left(\Bx \cdot \BB \inv{\BB}\right)\Exp^{\hat{\BB}\theta} \\
&= \Bx \wedge (i\Bn) \inv{i\Bn} + \left(\Bx \cdot (i\Bn) \inv{i\Bn}\right)\Exp^{{i\Bn}\theta} \\
\end{align*}

Now, 

\begin{align*}
\Bx \wedge (i\Bn) 
&= \inv{2}(\Bx i \Bn + i \Bn \Bx) \\
&= \frac{i}{2}(\Bx \Bn + \Bn \Bx) \\
&= (\Bx \cdot \Bn) i
\end{align*}

And

\begin{align*}
\inv{i\Bn} 
&= \inv{i\Bn} \inv{\Bn i} \Bn i \\
&= - i \Bn \\
\end{align*}

So the rejective term becomes
\begin{align*}
\Bx \wedge \BB \inv{\BB} 
&= \Bx \wedge (i\Bn) \inv{i\Bn} \\
&= \Bx \wedge (i\Bn) \inv{i\Bn} \\
&= (\Bx \cdot \Bn) i (-i) \Bn \\
&= (\Bx \cdot \Bn) \Bn \\
&= \Proj_{\Bn}(\Bx) \\
\end{align*}

Now, for the dot product with the plane term, we have

\begin{align*}
\Bx \cdot \BB
&= \Bx \cdot (i \Bn) \\
&= \inv{2}(\Bx i \Bn - i \Bn \Bx) \\
&= (\Bx \wedge \Bn)i \\
\end{align*}

Putting it all together we have

\begin{equation}\label{eqn:rotexp}
R_\theta(\Bx) 
= (\Bx \cdot \Bn) \Bn + (\Bx \wedge \Bn)\Bn \Exp^{{i\Bn}\theta}
\end{equation}

In terms of explicit sine and cosine terms this is (observe that $(i\Bn)^2 = -1$),

\begin{align*}
R_\theta(\Bx) 
&= \left(\Bx \cdot \Bn\right) \Bn + \left(\Bx \wedge \Bn\right)\Bn \left(\cos\theta + i\Bn \sin\theta\right) \\
\end{align*}

\begin{equation}\label{eqn:rotnorm}
R_\theta(\Bx) =
\left(\Bx \cdot \Bn\right) \Bn + \left(\Bx \wedge \Bn\right)\Bn \cos\theta + (\Bx \wedge \Bn) i \sin\theta 
\end{equation}

\begin{figure}[htp]
\centering
\includegraphics[totalheight=0.4\textheight]{normalRot}
\caption{Direction vectors associated with rotation}\label{fig:normalRot}
\end{figure}

This triplet of mutually orthogonal direction vectors, 
$\Bn$, $(\Bx \wedge \Bn)\Bn$, and $(\Bx \wedge \Bn) i$
are illustrated in figure \ref{fig:normalRot}.  The component of the vector in the direction of the normal
$\Proj_\Bn(\Bx) = \Bx \cdot \Bn \Bn$ is unaltered by the rotation.
The rotation is applied to the remaining component of $\Bx$, $\Rej_{\Bn}(\Bx) = (\Bx \wedge \Bn)\Bn$, and we rotate
in the direction $(\Bx \wedge \Bn) i$

\subsection{ Vector rotation in terms of dot and cross products only. }

Expression of this rotation formula \ref{eqn:rotnorm} in terms of ``vector'' relations is also possible, by removing the wedge
products and the pseudoscalar references.

First the rejective term

\begin{align*}
(\Bx \wedge \Bn) \Bn 
&= ((\Bx \cross \Bn) i) \Bn \\
&= ((\Bx \cross \Bn) i) \cdot \Bn \\
&= \inv{2} ( ((\Bx \cross \Bn) i) \Bn - \Bn ((\Bx \cross \Bn) i)) \\
&= \frac{i}{2} ( (\Bx \cross \Bn) \Bn - \Bn (\Bx \cross \Bn) ) \\
&= i ( (\Bx \cross \Bn) \wedge \Bn ) \\
&= i^2 ( (\Bx \cross \Bn) \cross \Bn ) \\
&= \Bn \cross (\Bx \cross \Bn) \\
\end{align*}

The next term expressed in terms of the cross product is

\begin{align*}
(\Bx \wedge \Bn) i
&=
(\Bx \cross \Bn) i^2 \\
&= \Bn \cross \Bx \\
\end{align*}

And putting it all together we have

\begin{equation}\label{eqn:rotcross}
R_\theta(\Bx) =
\left(\Bx \cdot \Bn\right) \Bn 
 + \left(\Bn \cross \Bx\right) \cross \Bn \cos\theta 
 + \Bn \cross \Bx \sin\theta 
\end{equation}

Compare equation \ref{eqn:rotcross} to \ref{eqn:rotnorm} and \ref{eqn:rotexp}, and then back to \ref{eqn:rotor}.

\section{ Giving a meaning to the sign of the bivector. }

For a rotation between two vectors in the plane containing those vectors, we can write the rotation 
in terms of the exponential as either a left or right rotation operator:

\[
\Bb = \Ba\Exp^{\Bi\theta} = \Exp^{-\Bi\theta}\Ba 
\]
\[
\Bb = \Exp^{\Bj\theta}\Ba = \Ba\Exp^{-\Bj\theta/2}
\]

Here both $\Bi$ and $\Bj=-\Bi$ are unit bivectors with the property $\Bi^2 = \Bj^2 = -1$.
Thus in order to write a rotation in exponential form a meaning must be assigned to the sign of the unit bivector that describes the
plane and the orientation of the rotation.

Consider for example the case of a rotation by $\pi/2$.  For this is the exponential is:

\[
\Exp^{\Bi\pi/2} = \cos(\pi/2) + \Bi \sin(\pi/2) = \Bi
\]

Thus for perpendicular unit vectors $\Bu$ and $\Bv$, if we wish $\Bi$ to act as a $\pi/2$ rotation left acting operator on $\Bu$
towards $\Bv$ it's value must be:

\[
\Bi = \Bu \wedge \Bv
\]
\[
\Bu\Bi = \Bu \Bu \wedge \Bv = \Bu\Bu\Bv = \Bv
\]

For that same rotation if the bivector is employed as a right acting operator, the reverse is required:

\[
\Bj = \Bv \wedge \Bu
\]
\[
\Bj\Bu = \Bv \wedge \Bu = \Bv\Bu\Bu = \Bv
\]

\begin{figure}[htp]
\centering
\includegraphics[totalheight=0.4\textheight]{imaginaryorientation}
\caption{Orientation of unit imaginary}\label{fig:imaginaryorientation}
\end{figure}

In general, for any two vectors, one can find an angle $\theta$ in the range $0 \le \theta \le \pi$ between those vectors.  
If one lets that angle define the orientation of the rotation between the vectors, and implicitly
define a sort of ``imaginary axis'' for that plane, that imaginary axis will have direction

\[
\inv{\Ba} \Ba \wedge \Bb = \Bb \wedge \Ba \inv {\Ba}.
\]

This is illustrated in figure \ref{fig:imaginaryorientation}.

Thus the bivector

\[
\Bi = \frac{\Ba \wedge \Bb}{\abs{\Ba \wedge \Bb}}
\]

When acting as an operator to the left ($\Ba \Bi$) with a vector in the plane can be interpreted as acting as a rotation by $\pi/2$ towards $\Bb$.

Similarily the bivector

\[
\Bj = \Bi^\dagger = -\Bi = \frac{\Bb \wedge \Ba}{\abs{\Bb \wedge \Ba}}
\]

also applied to a vector in the plane produces the same rotation when 
acting as an operator to the right.  Thus, in general we can write
a rotation by theta in the plane containing non-colinear vectors $\Ba$ and $\Bb$ in the direction of minimal angle
from $\Ba$ towards $\Bb$ in one of the three forms:

\[
R_{\theta : \Ba \rightarrow \Bb}(\Ba)
= \Ba \Exp^{ \frac{\Ba \wedge \Bb}{\abs{\Ba \wedge \Bb}} \theta }
= \Exp^{ \frac{\Bb \wedge \Ba}{\abs{\Bb \wedge \Ba}} \theta } \Ba
\]

Or,
\[
R_{\theta : \Ba \rightarrow \Bb}(\Bx)
= \Exp^{ \frac{\Bb \wedge \Ba}{\abs{\Bb \wedge \Ba}} \theta/2 } \Bx \Exp^{ \frac{\Ba \wedge \Bb}{\abs{\Ba \wedge \Bb}} \theta/2 }
\]

This last (writing $\Bx$ instead of $\Ba$ since it also applies to vectors that lie outside of the $\Ba \wedge \Bb$ plane), 
is our rotor formula \ref{eqn:rotor}, reexpressed in a way that removes the sign ambiguity of the bivector $\Bi$ in that equation.

\section{ Rotation between two unit vectors. }

\begin{figure}[htp]
\centering
\includegraphics[totalheight=0.4\textheight]{parallelogramvec}
\caption{Sum of unit vectors bisects angle between.}\label{fig:parallelogramvec}
\end{figure}

As illustrated in figure \ref{fig:parallelogramvec}, when the angle between two vectors is less than $\pi$
the fact that the sum of two arbitrarily oriented unit vectors bisects those vectors provides a convienient 
way to compute the half angle rotation exponential.

Thus we can write

\begin{equation*}
\frac{\Ba + \Bb}{\abs{\Ba + \Bb}} = \Ba \Exp^{\Bi\theta/2} = \Exp^{\Bj\theta/2} \Ba
\end{equation*}

Where $\Bi = \Bj^\dagger$ are unit bivectors of appropriate sign.  Multiplication through by $\Ba$ gives

\begin{equation*}
\Exp^{\Bi\theta/2} = 
\frac{1 + \Ba\Bb}{\abs{\Ba + \Bb}}
\end{equation*}

Or,
\begin{equation*}
\Exp^{\Bj\theta/2} =
\frac{1 + \Bb\Ba}{\abs{\Ba + \Bb}}
\end{equation*}

Thus we can write the total rotation from $\Ba$ to $\Bb$ as

\begin{equation*}
\Bb 
= \Exp^{-\Bi\theta/2} \Ba \Exp^{\Bi\theta/2}
= \Exp^{\Bj\theta/2} \Ba \Exp^{-\Bj\theta/2}
= \left(\frac{1 + \Bb\Ba}{\abs{\Ba + \Bb}}\right) \Ba \left(\frac{1 + \Ba\Bb}{\abs{\Ba + \Bb}}\right)
\end{equation*}

For the case where the rotation is through an angle $\theta$ where $\pi < \theta < 2\pi$, again employing a left acting
exponential operator we have

\begin{align*}
\frac{\Ba + \Bb}{\abs{\Ba + \Bb}} 
&= \Bb \Exp^{\Bi(2\pi - \theta)/2} \\
&= \Bb \Exp^{\Bi \pi} \Exp^{- \Bi\theta/2} \\
&= -\Bb \Exp^{- \Bi\theta/2} \\
\end{align*}

Or,
\[
\Exp^{- \Bi\theta/2} = -\frac{\Bb\Ba + 1}{\abs{\Ba + \Bb}} 
\]

Thus 

\begin{equation}\label{eqn:rotunit}
\Bb = \Exp^{- \Bi\theta/2} \Ba \Exp^{ \Bi\theta/2} =
\left(-\frac{1 + \Bb\Ba}{\abs{\Ba + \Bb}}\right) \Ba \left(-\frac{1 + \Ba\Bb}{\abs{\Ba + \Bb}}\right)
\end{equation}

Note that the two negatives cancel, giving the same result as in the $\theta < \pi$ case.  Thus equation \ref{eqn:rotunit} is valid for all vectors $\Ba \ne -\Bb$ (this can be verified by direct multiplication.)

These
half angle exponentials are called rotors, writing the rotor as

\[
R = \frac{1 + \Ba\Bb}{\abs{\Ba + \Bb}}
\]

and the rotation in terms of rotors is:

\[
\Bb = R^\dagger \Ba R
\]

The angle associated with this rotor $R$ is the minimal angle between the two vectors ($0 < \theta < \pi$), and is directed from $\Ba$ to $\Bb$.  Inverting the rotor will not change the net effect of the rotation, but has the geometric meaning that the rotation from $\Ba$ to $\Bb$ 
rotates in the opposite direction through the larger angle ($\pi < \theta < 2\pi$) between the vectors.

\end{document}               % End of document.
