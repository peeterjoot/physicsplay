\documentclass{article}

\usepackage{amsmath}
\usepackage{mathpazo}

%
% shorthand for bold symbols, convenient for vectors and matrices
%
\newcommand{\Ba}[0]{\mathbf{a}}
\newcommand{\Bb}[0]{\mathbf{b}}
\newcommand{\Bc}[0]{\mathbf{c}}
\newcommand{\Bd}[0]{\mathbf{d}}
\newcommand{\Be}[0]{\mathbf{e}}
\newcommand{\Bf}[0]{\mathbf{f}}
\newcommand{\Bg}[0]{\mathbf{g}}
\newcommand{\Bh}[0]{\mathbf{h}}
\newcommand{\Bi}[0]{\mathbf{i}}
\newcommand{\Bj}[0]{\mathbf{j}}
\newcommand{\Bk}[0]{\mathbf{k}}
\newcommand{\Bl}[0]{\mathbf{l}}
\newcommand{\Bm}[0]{\mathbf{m}}
\newcommand{\Bn}[0]{\mathbf{n}}
\newcommand{\Bo}[0]{\mathbf{o}}
\newcommand{\Bp}[0]{\mathbf{p}}
\newcommand{\Bq}[0]{\mathbf{q}}
\newcommand{\Br}[0]{\mathbf{r}}
\newcommand{\Bs}[0]{\mathbf{s}}
\newcommand{\Bt}[0]{\mathbf{t}}
\newcommand{\Bu}[0]{\mathbf{u}}
\newcommand{\Bv}[0]{\mathbf{v}}
\newcommand{\Bw}[0]{\mathbf{w}}
\newcommand{\Bx}[0]{\mathbf{x}}
\newcommand{\By}[0]{\mathbf{y}}
\newcommand{\Bz}[0]{\mathbf{z}}
\newcommand{\BA}[0]{\mathbf{A}}
\newcommand{\BB}[0]{\mathbf{B}}
\newcommand{\BC}[0]{\mathbf{C}}
\newcommand{\BD}[0]{\mathbf{D}}
\newcommand{\BE}[0]{\mathbf{E}}
\newcommand{\BF}[0]{\mathbf{F}}
\newcommand{\BG}[0]{\mathbf{G}}
\newcommand{\BH}[0]{\mathbf{H}}
\newcommand{\BI}[0]{\mathbf{I}}
\newcommand{\BJ}[0]{\mathbf{J}}
\newcommand{\BK}[0]{\mathbf{K}}
\newcommand{\BL}[0]{\mathbf{L}}
\newcommand{\BM}[0]{\mathbf{M}}
\newcommand{\BN}[0]{\mathbf{N}}
\newcommand{\BO}[0]{\mathbf{O}}
\newcommand{\BP}[0]{\mathbf{P}}
\newcommand{\BQ}[0]{\mathbf{Q}}
\newcommand{\BR}[0]{\mathbf{R}}
\newcommand{\BS}[0]{\mathbf{S}}
\newcommand{\BT}[0]{\mathbf{T}}
\newcommand{\BU}[0]{\mathbf{U}}
\newcommand{\BV}[0]{\mathbf{V}}
\newcommand{\BW}[0]{\mathbf{W}}
\newcommand{\BX}[0]{\mathbf{X}}
\newcommand{\BY}[0]{\mathbf{Y}}
\newcommand{\BZ}[0]{\mathbf{Z}}

\newcommand{\Bzero}[0]{\mathbf{0}}
\newcommand{\Btheta}[0]{\boldsymbol{\theta}}
\newcommand{\Btau}[0]{\boldsymbol{\tau}}
\newcommand{\Bomega}[0]{\boldsymbol{\omega}}

%
% shorthand for unit vectors
%
\newcommand{\acap}[0]{\hat{\Ba}}
\newcommand{\bcap}[0]{\hat{\Bb}}
\newcommand{\ccap}[0]{\hat{\Bc}}
\newcommand{\dcap}[0]{\hat{\Bd}}
\newcommand{\ecap}[0]{\hat{\Be}}
\newcommand{\fcap}[0]{\hat{\Bf}}
\newcommand{\gcap}[0]{\hat{\Bg}}
\newcommand{\hcap}[0]{\hat{\Bh}}
\newcommand{\icap}[0]{\hat{\Bi}}
\newcommand{\jcap}[0]{\hat{\Bj}}
\newcommand{\kcap}[0]{\hat{\Bk}}
\newcommand{\lcap}[0]{\hat{\Bl}}
\newcommand{\mcap}[0]{\hat{\Bm}}
\newcommand{\ncap}[0]{\hat{\Bn}}
\newcommand{\ocap}[0]{\hat{\Bo}}
\newcommand{\pcap}[0]{\hat{\Bp}}
\newcommand{\qcap}[0]{\hat{\Bq}}
\newcommand{\rcap}[0]{\hat{\Br}}
\newcommand{\scap}[0]{\hat{\Bs}}
\newcommand{\tcap}[0]{\hat{\Bt}}
\newcommand{\ucap}[0]{\hat{\Bu}}
\newcommand{\vcap}[0]{\hat{\Bv}}
\newcommand{\wcap}[0]{\hat{\Bw}}
\newcommand{\xcap}[0]{\hat{\Bx}}
\newcommand{\ycap}[0]{\hat{\By}}
\newcommand{\zcap}[0]{\hat{\Bz}}
\newcommand{\thetacap}[0]{\hat{\Btheta}}

%
% to write R^n and C^n in a distinguishable fashion.  Perhaps change this
% to the double lined characters upon figuring out how to do so.
%
\newcommand{\C}[1]{$\mathbb{C}^{#1}$}
\newcommand{\R}[1]{$\mathbb{R}^{#1}$}

%
% various generally useful helpers
%

% derivative of #1 wrt. #2:
\newcommand{\D}[2] {\frac {d#2} {d#1}}

\newcommand{\inv}[1]{\frac{1}{#1}}
\newcommand{\cross}[0]{\times}

\newcommand{\abs}[1]{\lvert{#1}\rvert}
\newcommand{\norm}[1]{\lVert{#1}\rVert}
\newcommand{\innerprod}[2]{\langle{#1}, {#2}\rangle}
\newcommand{\dotprod}[2]{{#1} \cdot {#2}}
\newcommand{\bdotprod}[2]{\left({#1} \cdot {#2}\right)}
\newcommand{\crossprod}[2]{{#1} \cross {#2}}
\newcommand{\tripleprod}[3]{\dotprod{\left(\crossprod{#1}{#2}\right)}{#3}}

\DeclareMathOperator{\Proj}{Proj}
\DeclareMathOperator{\Span}{span}
\DeclareMathOperator{\Sgn}{sgn}
\DeclareMathOperator{\Area}{Area}
\DeclareMathOperator{\Volume}{Volume}

%
% A few miscellaneous things specific to this document
%
\newcommand{\crossop}[1]{\crossprod{#1}{}}

% R2 vector.
\newcommand{\VectorTwo}[2]{
\begin{bmatrix}
 {#1} \\
 {#2}
\end{bmatrix}
}

\newcommand{\VectorN}[1]{
\begin{bmatrix}
{#1}_1 \\
{#1}_2 \\
\vdots \\
{#1}_N \\
\end{bmatrix}
}

\newcommand{\DETuvij}[4]{
\begin{vmatrix}
 {#1}_{#3} & {#1}_{#4} \\
 {#2}_{#3} & {#2}_{#4}
\end{vmatrix}
}

\newcommand{\DETuvwijk}[6]{
\begin{vmatrix}
 {#1}_{#4} & {#1}_{#5} & {#1}_{#6} \\
 {#2}_{#4} & {#2}_{#5} & {#2}_{#6} \\
 {#3}_{#4} & {#3}_{#5} & {#3}_{#6}
\end{vmatrix}
}

\newcommand{\DETuvwxijkl}[8]{
\begin{vmatrix}
 {#1}_{#5} & {#1}_{#6} & {#1}_{#7} & {#1}_{#8} \\
 {#2}_{#5} & {#2}_{#6} & {#2}_{#7} & {#2}_{#8} \\
 {#3}_{#5} & {#3}_{#6} & {#3}_{#7} & {#3}_{#8} \\
 {#4}_{#5} & {#4}_{#6} & {#4}_{#7} & {#4}_{#8} \\
\end{vmatrix}
}

%\newcommand{\DETuvwxyijklm}[10]{
%\begin{vmatrix}
% {#1}_{#6} & {#1}_{#7} & {#1}_{#8} & {#1}_{#9} & {#1}_{#10} \\
% {#2}_{#6} & {#2}_{#7} & {#2}_{#8} & {#2}_{#9} & {#2}_{#10} \\
% {#3}_{#6} & {#3}_{#7} & {#3}_{#8} & {#3}_{#9} & {#3}_{#10} \\
% {#4}_{#6} & {#4}_{#7} & {#4}_{#8} & {#4}_{#9} & {#4}_{#10} \\
% {#5}_{#6} & {#5}_{#7} & {#5}_{#8} & {#5}_{#9} & {#5}_{#10}
%\end{vmatrix}
%}

% R3 vector.
\newcommand{\VectorThree}[3]{
\begin{bmatrix}
 {#1} \\
 {#2} \\
 {#3}
\end{bmatrix}
}


\newcommand{\grad}[0]{\nabla}
\newcommand{\PD}[2]{\frac{\partial {#2}}{\partial {#1}}}
\newcommand{\Rm}[1]{\mathbb{R}^{#1}}
%\newcommand{\spacegrad}[0]{\boldsymbol{\nabla}}
\newcommand{\gpgrade}[2] {{\left\langle{{#1}}\right\rangle}_{#2}}
\newcommand{\gpgradezero}[1] {\gpgrade{#1}{0}}
\newcommand{\gpgradeone}[1] {\gpgrade{#1}{1}}
\newcommand{\gpgradetwo}[1] {\gpgrade{#1}{2}}
\newcommand{\gpgradethree}[1] {\gpgrade{#1}{3}}

\usepackage{color,cite,graphicx}
   % use colour in the document, put your citations as [1-4]
   % rather than [1,2,3,4] (it looks nicer, and the extended LaTeX2e
   % graphics package. 
\usepackage{latexsym,amssymb,epsf} % don't remember if these are
   % needed, but their inclusion can't do any damage


% ointclockwise, ointctrclockwise
\usepackage{txfonts}

\usepackage[bookmarks=true]{hyperref}

\title{ Revisit Stokes derivation. }
\author{Peeter Joot}
\date{ Sept 27, 2008.  Last Revision: $Date: 2008/09/28 02:26:12 $ }

\begin{document}

\maketitle{}

\tableofcontents

\section{ Algebraic description of oriented boundaries. }

Having used pictorial methods to enumerate the bounding loop and area elements 
\href{http://www.geocities.com/peeter_joot/geometric_algebra/vector_integral_relations.pdf}{
in the previous derivation} of the vector and bivector forms of Stokes's, makes the application
of these formulas harder.  Here this will be revisited, with the aim of remedying this, as well as
obtaining a proof for the general case, which was not possible because of a lack of exactly this
algebraic formulation.

\subsection{ Parallelogram parameterization. }

\begin{figure}[htp]
\centering
\includegraphics[totalheight=0.4\textheight]{parallelogram_parameterized}
\caption{Two variable parameterization of \R{n} parallelogram}\label{fig:parallelogram}
\end{figure}

An oriented curve around a parallelogram in \R{n} is illustrated in figure 
\ref{fig:parallelogram}.  We want to evaluate the line integral around this
path

\begin{align}\label{eqn:lineprep}
\ointclockwise \Bf \cdot d\Br
&=
\int du_1 \left.{\Bf \cdot \PD{u_1}{\Br} }\right\vert_{u_2(0)}^{u_2(1)}
-\int du_2 \left.{\Bf \cdot \PD{u_2}{\Br} }\right\vert_{u_1(0)}^{u_1(1)} \\
\end{align}

Now, we can put this in a more symmetric form utilizing a reciprocal 
frame to enumerate the alternation.  Write

\begin{align*}
\Br_{u_i} &= \PD{u_i}{\Br} \\
\Br^{u_j} \cdot \Br_{u_i} &= {\delta^j}_i \\
I &= \Br_{u_1} \wedge \Br_{u_2} \\
I \Br^{u_1} &=
I \cdot \Br^{u_1} =
% (\Br_{u_1} \wedge \Br_{u_2}) \cdot \Br^{u_1}
- \Br_{u_2} \\
I \Br^{u_2} &=
I \cdot \Br^{u_2} =
% (\Br_{u_1} \wedge \Br_{u_2}) \cdot \Br^{u_2}
 \Br_{u_1}.
\end{align*}

We don't care to actually calculate the reciprocal frame vectors.  They just work well to describe the alternation in terms of the pseudoscalar for the plane.

Substituiting back into \ref{eqn:lineprep} we have

\begin{align*}
\ointclockwise \Bf \cdot d\Br
&=
\int du_1 \left.{\Bf \cdot \left(I \Br^{u_2}\right) }\right\vert_{u_2(0)}^{u_2(1)}
+\int du_2 \left.{\Bf \cdot \left(I \Br^{u_1}\right) }\right\vert_{u_1(0)}^{u_1(1)} \\
\end{align*}

Or
\begin{equation}\label{eqn:lineintegral}
\ointclockwise \Bf \cdot d\Br
=
\sum_{i} \int \frac{du_1 du_2}{du_i} \left.{\Bf \cdot \left(I \Br^{u_i}\right) }\right\vert_{u_i(0)}^{u_i(1)}
\end{equation}

This completes the goal of expressing the line integral in a fashion that doesn't require drawing any pictures,
and gives a hint about how to do the same for general ${\bigwedge}^k \Rm{n}$ case.

As before this can be written in terms of its integrals

\begin{align*}
\ointclockwise \Bf \cdot d\Br
&= \sum_{i, j \ne i} 
\int_{u_j(0)}^{u_j(1)} du_j
\int_{u_i(0)}^{u_i(1)}
 \PD{u_i}{} {\Bf \cdot \left(I \Br^{u_i}\right)} du_i \\
&= \iint du_1 du_2 \sum \PD{u_i}{} {\Bf \cdot \left(I \Br^{u_i}\right)}
\end{align*}

Evaluating the derivatives to prove the Stokes/Green's result will be deferred for now (may instead proving
the general case once formulated).

\subsection{ Parallelopiped parameterization. }

\begin{figure}[htp]
\centering
\includegraphics[totalheight=0.4\textheight]{parallelopiped_parameterized}
\caption{Three variable parameterization of \R{n} parallelopiped}\label{fig:parallelopiped}
\end{figure}

Next, lets evaluate the bivector area dot products, as in figure \ref{fig:parallelopiped}.

\begin{align*}
\oiintclockwise \BF \cdot d\BA
&= \iint du_2 du_1 \left.{F \cdot (\Br_{u_2} \wedge \Br_{u_1})}\right\vert_{u_3(1)}^{u_3(0)} \\
& +\iint du_3 du_1 \left.{F \cdot (\Br_{u_3} \wedge \Br_{u_1})}\right\vert_{u_2(0)}^{u_2(1)} \\
& +\iint du_3 du_2 \left.{F \cdot (\Br_{u_3} \wedge -\Br_{u_2})}\right\vert_{u_1(0)}^{u_1(1)} \\
\end{align*}

Again introducing reciprocal vectors to enumerate the alternation, but now write $I$ as a pseudoscalar
for the parallelopiped subspace that the area bounds

\begin{align*}
I &= \Br_{u_1} \wedge \Br_{u_2} \wedge \Br_{u_3} \\
I \Br^{u_1} &= \Br_{u_2} \wedge \Br_{u_3} \\
I \Br^{u_2} &= -\Br_{u_1} \wedge \Br_{u_3} \\
I \Br^{u_3} &= \Br_{u_1} \wedge \Br_{u_2} \\
\end{align*}

Substituting we have a form almost identical to the line integral of equation \ref{eqn:lineintegral}.

\begin{align}
\oiintclockwise \BF \cdot d\BA 
&= \sum \iint \frac{du_1 du_2 du_3}{du_i} \left.{F \cdot (I \Br^{u_i})}\right\vert_{u_i(0)}^{u_i(1)} \\
&= \iiint du_1 du_2 du_3 \sum \PD{u_i}{} F \cdot (I \Br^{u_i})
\end{align}

\subsection{ General case. }

Having found that the line integral and oriented area integrals can be expressed uniformly in the same algebraic form, it
is reasonable to define an integral with such structure as a directed hypervolume boundary for any grade blade, and then verify that
this yields the expected generalized Stokes result that has been proven for only the vector and area cases.

Writing
\begin{align*}
F &\in {\bigwedge}^k \Rm{n} \\
d^k \Bx &= \PD{u_1}{\Br} \wedge \PD{u_2}{\Br} \wedge \cdots \wedge \PD{u_k}{\Br} du_1 du_2 \cdots du_k = I du_1 du_2 \cdots du_k \\
\end{align*}

We wish to prove the general Stokes equation for a hyper-parallopiped volume

\begin{equation}
\int_V F \cdot d^{k}\Bx = \int_{\partial V} F \cdot d^{k-1}\Bx 
\end{equation}

With the presumption that this will algebraically be identical to the line integral and area integral cases for vectors and bivectors respectively
we want to evaluate

\begin{align}
\int_{\partial V} F \cdot d^{k-1}\Bx 
&= \sum \int \frac{du_1 du_2 \cdots du_k}{du_i} \left.{F \cdot (I \Br^{u_i})}\right\vert_{u_i(0)}^{u_i(1)} \\
&= \int du_1 du_2 \cdots du_k \sum \PD{u_i}{} F \cdot (I \Br^{u_i}).
\end{align}

The partials here can likely be expanded by coordinates as done in the previous line and area proofs, but is there a more elegant way to do this?

\end{document}
