\documentclass[]{eliblog}

\usepackage{amsmath}
\usepackage{mathpazo}

%
% shorthand for bold symbols, convenient for vectors and matrices
%
\newcommand{\Ba}[0]{\mathbf{a}}
\newcommand{\Bb}[0]{\mathbf{b}}
\newcommand{\Bc}[0]{\mathbf{c}}
\newcommand{\Bd}[0]{\mathbf{d}}
\newcommand{\Be}[0]{\mathbf{e}}
\newcommand{\Bf}[0]{\mathbf{f}}
\newcommand{\Bg}[0]{\mathbf{g}}
\newcommand{\Bh}[0]{\mathbf{h}}
\newcommand{\Bi}[0]{\mathbf{i}}
\newcommand{\Bj}[0]{\mathbf{j}}
\newcommand{\Bk}[0]{\mathbf{k}}
\newcommand{\Bl}[0]{\mathbf{l}}
\newcommand{\Bm}[0]{\mathbf{m}}
\newcommand{\Bn}[0]{\mathbf{n}}
\newcommand{\Bo}[0]{\mathbf{o}}
\newcommand{\Bp}[0]{\mathbf{p}}
\newcommand{\Bq}[0]{\mathbf{q}}
\newcommand{\Br}[0]{\mathbf{r}}
\newcommand{\Bs}[0]{\mathbf{s}}
\newcommand{\Bt}[0]{\mathbf{t}}
\newcommand{\Bu}[0]{\mathbf{u}}
\newcommand{\Bv}[0]{\mathbf{v}}
\newcommand{\Bw}[0]{\mathbf{w}}
\newcommand{\Bx}[0]{\mathbf{x}}
\newcommand{\By}[0]{\mathbf{y}}
\newcommand{\Bz}[0]{\mathbf{z}}
\newcommand{\BA}[0]{\mathbf{A}}
\newcommand{\BB}[0]{\mathbf{B}}
\newcommand{\BC}[0]{\mathbf{C}}
\newcommand{\BD}[0]{\mathbf{D}}
\newcommand{\BE}[0]{\mathbf{E}}
\newcommand{\BF}[0]{\mathbf{F}}
\newcommand{\BG}[0]{\mathbf{G}}
\newcommand{\BH}[0]{\mathbf{H}}
\newcommand{\BI}[0]{\mathbf{I}}
\newcommand{\BJ}[0]{\mathbf{J}}
\newcommand{\BK}[0]{\mathbf{K}}
\newcommand{\BL}[0]{\mathbf{L}}
\newcommand{\BM}[0]{\mathbf{M}}
\newcommand{\BN}[0]{\mathbf{N}}
\newcommand{\BO}[0]{\mathbf{O}}
\newcommand{\BP}[0]{\mathbf{P}}
\newcommand{\BQ}[0]{\mathbf{Q}}
\newcommand{\BR}[0]{\mathbf{R}}
\newcommand{\BS}[0]{\mathbf{S}}
\newcommand{\BT}[0]{\mathbf{T}}
\newcommand{\BU}[0]{\mathbf{U}}
\newcommand{\BV}[0]{\mathbf{V}}
\newcommand{\BW}[0]{\mathbf{W}}
\newcommand{\BX}[0]{\mathbf{X}}
\newcommand{\BY}[0]{\mathbf{Y}}
\newcommand{\BZ}[0]{\mathbf{Z}}

\newcommand{\Bzero}[0]{\mathbf{0}}
\newcommand{\Btheta}[0]{\boldsymbol{\theta}}
\newcommand{\Btau}[0]{\boldsymbol{\tau}}
\newcommand{\Bomega}[0]{\boldsymbol{\omega}}

%
% shorthand for unit vectors
%
\newcommand{\acap}[0]{\hat{\Ba}}
\newcommand{\bcap}[0]{\hat{\Bb}}
\newcommand{\ccap}[0]{\hat{\Bc}}
\newcommand{\dcap}[0]{\hat{\Bd}}
\newcommand{\ecap}[0]{\hat{\Be}}
\newcommand{\fcap}[0]{\hat{\Bf}}
\newcommand{\gcap}[0]{\hat{\Bg}}
\newcommand{\hcap}[0]{\hat{\Bh}}
\newcommand{\icap}[0]{\hat{\Bi}}
\newcommand{\jcap}[0]{\hat{\Bj}}
\newcommand{\kcap}[0]{\hat{\Bk}}
\newcommand{\lcap}[0]{\hat{\Bl}}
\newcommand{\mcap}[0]{\hat{\Bm}}
\newcommand{\ncap}[0]{\hat{\Bn}}
\newcommand{\ocap}[0]{\hat{\Bo}}
\newcommand{\pcap}[0]{\hat{\Bp}}
\newcommand{\qcap}[0]{\hat{\Bq}}
\newcommand{\rcap}[0]{\hat{\Br}}
\newcommand{\scap}[0]{\hat{\Bs}}
\newcommand{\tcap}[0]{\hat{\Bt}}
\newcommand{\ucap}[0]{\hat{\Bu}}
\newcommand{\vcap}[0]{\hat{\Bv}}
\newcommand{\wcap}[0]{\hat{\Bw}}
\newcommand{\xcap}[0]{\hat{\Bx}}
\newcommand{\ycap}[0]{\hat{\By}}
\newcommand{\zcap}[0]{\hat{\Bz}}
\newcommand{\thetacap}[0]{\hat{\Btheta}}

%
% to write R^n and C^n in a distinguishable fashion.  Perhaps change this
% to the double lined characters upon figuring out how to do so.
%
\newcommand{\C}[1]{$\mathbb{C}^{#1}$}
\newcommand{\R}[1]{$\mathbb{R}^{#1}$}

%
% various generally useful helpers
%

% derivative of #1 wrt. #2:
\newcommand{\D}[2] {\frac {d#2} {d#1}}

\newcommand{\inv}[1]{\frac{1}{#1}}
\newcommand{\cross}[0]{\times}

\newcommand{\abs}[1]{\lvert{#1}\rvert}
\newcommand{\norm}[1]{\lVert{#1}\rVert}
\newcommand{\innerprod}[2]{\langle{#1}, {#2}\rangle}
\newcommand{\dotprod}[2]{{#1} \cdot {#2}}
\newcommand{\bdotprod}[2]{\left({#1} \cdot {#2}\right)}
\newcommand{\crossprod}[2]{{#1} \cross {#2}}
\newcommand{\tripleprod}[3]{\dotprod{\left(\crossprod{#1}{#2}\right)}{#3}}

\DeclareMathOperator{\Proj}{Proj}
\DeclareMathOperator{\Span}{span}
\DeclareMathOperator{\Sgn}{sgn}
\DeclareMathOperator{\Area}{Area}
\DeclareMathOperator{\Volume}{Volume}

%
% A few miscellaneous things specific to this document
%
\newcommand{\crossop}[1]{\crossprod{#1}{}}

% R2 vector.
\newcommand{\VectorTwo}[2]{
\begin{bmatrix}
 {#1} \\
 {#2}
\end{bmatrix}
}

\newcommand{\VectorN}[1]{
\begin{bmatrix}
{#1}_1 \\
{#1}_2 \\
\vdots \\
{#1}_N \\
\end{bmatrix}
}

\newcommand{\DETuvij}[4]{
\begin{vmatrix}
 {#1}_{#3} & {#1}_{#4} \\
 {#2}_{#3} & {#2}_{#4}
\end{vmatrix}
}

\newcommand{\DETuvwijk}[6]{
\begin{vmatrix}
 {#1}_{#4} & {#1}_{#5} & {#1}_{#6} \\
 {#2}_{#4} & {#2}_{#5} & {#2}_{#6} \\
 {#3}_{#4} & {#3}_{#5} & {#3}_{#6}
\end{vmatrix}
}

\newcommand{\DETuvwxijkl}[8]{
\begin{vmatrix}
 {#1}_{#5} & {#1}_{#6} & {#1}_{#7} & {#1}_{#8} \\
 {#2}_{#5} & {#2}_{#6} & {#2}_{#7} & {#2}_{#8} \\
 {#3}_{#5} & {#3}_{#6} & {#3}_{#7} & {#3}_{#8} \\
 {#4}_{#5} & {#4}_{#6} & {#4}_{#7} & {#4}_{#8} \\
\end{vmatrix}
}

%\newcommand{\DETuvwxyijklm}[10]{
%\begin{vmatrix}
% {#1}_{#6} & {#1}_{#7} & {#1}_{#8} & {#1}_{#9} & {#1}_{#10} \\
% {#2}_{#6} & {#2}_{#7} & {#2}_{#8} & {#2}_{#9} & {#2}_{#10} \\
% {#3}_{#6} & {#3}_{#7} & {#3}_{#8} & {#3}_{#9} & {#3}_{#10} \\
% {#4}_{#6} & {#4}_{#7} & {#4}_{#8} & {#4}_{#9} & {#4}_{#10} \\
% {#5}_{#6} & {#5}_{#7} & {#5}_{#8} & {#5}_{#9} & {#5}_{#10}
%\end{vmatrix}
%}

% R3 vector.
\newcommand{\VectorThree}[3]{
\begin{bmatrix}
 {#1} \\
 {#2} \\
 {#3}
\end{bmatrix}
}



\author{Peeter Joot}
\email{peeter.joot@gmail.com}


\chapter{Bivector form of quantum angular momentum operator}
\label{chap:qmAngularMom}
%\useCCL
\blogpage{http://sites.google.com/site/peeterjoot/math2009/qmAngularMom.pdf}
\date{July 27, 2009}
\revisionInfo{$RCSfile: qmAngularMom.tex,v $ Last $Revision: 1.2 $ $Date: 2009/07/28 04:35:02 $}

\beginArtWithToc

\section{Spatial bivector representation of the angular momentum operator.}

Reading (\cite{bohm1989qt}) on the angular momentum operator, the form of the operator is suggested by analogy where components of $\Bx \cross \Bp$ with 
the position representation $\Bp \sim -i \hbar \spacegrad$ used to expand the coordinate representation of the operator.

The result is the following coordinate representation of the operator

%\begin{align*}
%L_x &= -i \hbar( y \partial_z - z \partial_y ) \\
%L_y &= -i \hbar( z \partial_x - x \partial_z ) \\
%L_z &= -i \hbar( x \partial_y - y \partial_x ) \\
%\end{align*}
\begin{align*}
L_1 &= -i \hbar( x_2 \partial_3 - x_3 \partial_2 ) \\
L_2 &= -i \hbar( x_3 \partial_1 - x_1 \partial_3 ) \\
L_3 &= -i \hbar( x_1 \partial_2 - x_2 \partial_1 ) \\
\end{align*}

It is interesting to put these in vector form, and then employ the freedom to use for $i = \sigma_1 \sigma_2 \sigma_3$ the spatial pseudoscalar.

\begin{align*}
\BL 
&= 
-\sigma_1 (\sigma_1 \sigma_2 \sigma_3) \hbar( x_2 \partial_3 - x_3 \partial_2 ) 
-\sigma_2 (\sigma_2 \sigma_3 \sigma_1) \hbar( x_3 \partial_1 - x_1 \partial_3 ) 
-\sigma_3 (\sigma_3 \sigma_1 \sigma_2) \hbar( x_1 \partial_2 - x_2 \partial_1 ) \\
&= 
-\sigma_2 \sigma_3 \hbar( x_2 \partial_3 - x_3 \partial_2 ) 
-\sigma_3 \sigma_1 \hbar( x_3 \partial_1 - x_1 \partial_3 ) 
-\sigma_1 \sigma_2 \hbar( x_1 \partial_2 - x_2 \partial_1 ) \\
&=
-\hbar ( \sigma_1 x_1 +\sigma_2 x_2 +\sigma_3 x_3 ) \wedge ( \sigma_1 \partial_1 +\sigma_2 \partial_2 +\sigma_3 \partial_3 ) \\
\end{align*}

The choice to use the pseudoscalar for this imaginary seems a logical one and the end result is a pure bivector representation of angular momentum operator

\begin{align}\label{eqn:ang}
\BL &= - \hbar \Bx \wedge \spacegrad
\end{align}

The choice to represent angular momentum as a bivector $\Bx \wedge \Bp$ is also natural in classical mechanics (encoding the orientation of the plane and the magnitude of the momentum in the bivector), although its dual form the axial vector $\Bx \cross \Bp$ is more common, at least in introductory mechanics.  Observe that there is no longer any explicit imaginary in (\ref{eqn:ang}), since the bivector itself has an implicit complex structure.

\section{Factoring the gradient and Laplacian.}

The form of (\ref{eqn:ang}) suggests a more direct way to extract the angular momentum operator from the Hamiltonian (i.e. from the Laplacian).  Bohm uses the spherical polar representation of the Laplacian as the starting point.  Instead let's project the gradient itself in a specific constant direction $\Ba$, much as we can do to find the polar form angular velocity and acceleration components.

Write 

\begin{align*}
\spacegrad 
&=
\inv{\Ba} \Ba \spacegrad \\
&=
\inv{\Ba} (\Ba \cdot \spacegrad + \Ba \wedge \spacegrad) \\
\end{align*}

Or
\begin{align*}
\spacegrad 
&=
\spacegrad \Ba \inv{\Ba} \\
&=
(\spacegrad \cdot \Ba + \spacegrad \wedge \Ba) \inv{\Ba} \\
&=
(\Ba \cdot \spacegrad - \Ba \wedge \spacegrad) \inv{\Ba} \\
\end{align*}

The Laplacian is therefore

\begin{align*}
\spacegrad^2 
&=
\gpgradezero{ \spacegrad^2 } \\
&=
\gpgradezero{ 
(\Ba \cdot \spacegrad - \Ba \wedge \spacegrad) \inv{\Ba} \inv{\Ba} (\Ba \cdot \spacegrad + \Ba \wedge \spacegrad) 
} \\
&=
\inv{\Ba^2} \gpgradezero{ 
(\Ba \cdot \spacegrad - \Ba \wedge \spacegrad) (\Ba \cdot \spacegrad + \Ba \wedge \spacegrad) 
} \\
&=
\inv{\Ba^2} ((\Ba \cdot \spacegrad)^2 - (\Ba \wedge \spacegrad)^2 ) \\
\end{align*}

So we have for the Laplacian a representation in terms of projection and rejection components 

\begin{align*}
\spacegrad^2
&=
(\acap \cdot \spacegrad)^2 - \inv{\Ba^2} (\Ba \wedge \spacegrad)^2 \\
\end{align*}

The vector $\Ba$ was arbitrary, and just needed to be constant with respect to the factorization operations.  Setting $\Ba = \Bx$, the radial position from the origin, we have

\begin{align}
\spacegrad^2 &= \frac{\partial^2 }{\partial r^2} - \inv{\Bx^2} (\Bx \wedge \spacegrad)^2 
\end{align}

So in polar form the bivector form of the angular momentum operator is quite evident, just by application of projection of the gradient onto the radial direction and the tangential plane to the sphere at the radial point

\begin{align*}
-\frac{\hbar^2}{2m} \spacegrad^2 + V
&=
-\frac{\hbar^2}{2m} \frac{\partial^2 }{\partial r^2} + \frac{\hbar^2}{2m \Bx^2} (\Bx \wedge \spacegrad)^2 + V
\end{align*}

\EndArticle
