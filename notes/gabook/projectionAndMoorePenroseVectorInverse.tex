\documentclass{article}      % Specifies the document class

\usepackage{amsmath}
\usepackage{mathpazo}

%
% shorthand for bold symbols, convenient for vectors and matrices
%
\newcommand{\Ba}[0]{\mathbf{a}}
\newcommand{\Bb}[0]{\mathbf{b}}
\newcommand{\Bc}[0]{\mathbf{c}}
\newcommand{\Bd}[0]{\mathbf{d}}
\newcommand{\Be}[0]{\mathbf{e}}
\newcommand{\Bf}[0]{\mathbf{f}}
\newcommand{\Bg}[0]{\mathbf{g}}
\newcommand{\Bh}[0]{\mathbf{h}}
\newcommand{\Bi}[0]{\mathbf{i}}
\newcommand{\Bj}[0]{\mathbf{j}}
\newcommand{\Bk}[0]{\mathbf{k}}
\newcommand{\Bl}[0]{\mathbf{l}}
\newcommand{\Bm}[0]{\mathbf{m}}
\newcommand{\Bn}[0]{\mathbf{n}}
\newcommand{\Bo}[0]{\mathbf{o}}
\newcommand{\Bp}[0]{\mathbf{p}}
\newcommand{\Bq}[0]{\mathbf{q}}
\newcommand{\Br}[0]{\mathbf{r}}
\newcommand{\Bs}[0]{\mathbf{s}}
\newcommand{\Bt}[0]{\mathbf{t}}
\newcommand{\Bu}[0]{\mathbf{u}}
\newcommand{\Bv}[0]{\mathbf{v}}
\newcommand{\Bw}[0]{\mathbf{w}}
\newcommand{\Bx}[0]{\mathbf{x}}
\newcommand{\By}[0]{\mathbf{y}}
\newcommand{\Bz}[0]{\mathbf{z}}
\newcommand{\BA}[0]{\mathbf{A}}
\newcommand{\BB}[0]{\mathbf{B}}
\newcommand{\BC}[0]{\mathbf{C}}
\newcommand{\BD}[0]{\mathbf{D}}
\newcommand{\BE}[0]{\mathbf{E}}
\newcommand{\BF}[0]{\mathbf{F}}
\newcommand{\BG}[0]{\mathbf{G}}
\newcommand{\BH}[0]{\mathbf{H}}
\newcommand{\BI}[0]{\mathbf{I}}
\newcommand{\BJ}[0]{\mathbf{J}}
\newcommand{\BK}[0]{\mathbf{K}}
\newcommand{\BL}[0]{\mathbf{L}}
\newcommand{\BM}[0]{\mathbf{M}}
\newcommand{\BN}[0]{\mathbf{N}}
\newcommand{\BO}[0]{\mathbf{O}}
\newcommand{\BP}[0]{\mathbf{P}}
\newcommand{\BQ}[0]{\mathbf{Q}}
\newcommand{\BR}[0]{\mathbf{R}}
\newcommand{\BS}[0]{\mathbf{S}}
\newcommand{\BT}[0]{\mathbf{T}}
\newcommand{\BU}[0]{\mathbf{U}}
\newcommand{\BV}[0]{\mathbf{V}}
\newcommand{\BW}[0]{\mathbf{W}}
\newcommand{\BX}[0]{\mathbf{X}}
\newcommand{\BY}[0]{\mathbf{Y}}
\newcommand{\BZ}[0]{\mathbf{Z}}

\newcommand{\Bzero}[0]{\mathbf{0}}
\newcommand{\Btheta}[0]{\boldsymbol{\theta}}
\newcommand{\Btau}[0]{\boldsymbol{\tau}}
\newcommand{\Bomega}[0]{\boldsymbol{\omega}}

%
% shorthand for unit vectors
%
\newcommand{\acap}[0]{\hat{\Ba}}
\newcommand{\bcap}[0]{\hat{\Bb}}
\newcommand{\ccap}[0]{\hat{\Bc}}
\newcommand{\dcap}[0]{\hat{\Bd}}
\newcommand{\ecap}[0]{\hat{\Be}}
\newcommand{\fcap}[0]{\hat{\Bf}}
\newcommand{\gcap}[0]{\hat{\Bg}}
\newcommand{\hcap}[0]{\hat{\Bh}}
\newcommand{\icap}[0]{\hat{\Bi}}
\newcommand{\jcap}[0]{\hat{\Bj}}
\newcommand{\kcap}[0]{\hat{\Bk}}
\newcommand{\lcap}[0]{\hat{\Bl}}
\newcommand{\mcap}[0]{\hat{\Bm}}
\newcommand{\ncap}[0]{\hat{\Bn}}
\newcommand{\ocap}[0]{\hat{\Bo}}
\newcommand{\pcap}[0]{\hat{\Bp}}
\newcommand{\qcap}[0]{\hat{\Bq}}
\newcommand{\rcap}[0]{\hat{\Br}}
\newcommand{\scap}[0]{\hat{\Bs}}
\newcommand{\tcap}[0]{\hat{\Bt}}
\newcommand{\ucap}[0]{\hat{\Bu}}
\newcommand{\vcap}[0]{\hat{\Bv}}
\newcommand{\wcap}[0]{\hat{\Bw}}
\newcommand{\xcap}[0]{\hat{\Bx}}
\newcommand{\ycap}[0]{\hat{\By}}
\newcommand{\zcap}[0]{\hat{\Bz}}
\newcommand{\thetacap}[0]{\hat{\Btheta}}

%
% to write R^n and C^n in a distinguishable fashion.  Perhaps change this
% to the double lined characters upon figuring out how to do so.
%
\newcommand{\C}[1]{$\mathbb{C}^{#1}$}
\newcommand{\R}[1]{$\mathbb{R}^{#1}$}

%
% various generally useful helpers
%

% derivative of #1 wrt. #2:
\newcommand{\D}[2] {\frac {d#2} {d#1}}

\newcommand{\inv}[1]{\frac{1}{#1}}
\newcommand{\cross}[0]{\times}

\newcommand{\abs}[1]{\lvert{#1}\rvert}
\newcommand{\norm}[1]{\lVert{#1}\rVert}
\newcommand{\innerprod}[2]{\langle{#1}, {#2}\rangle}
\newcommand{\dotprod}[2]{{#1} \cdot {#2}}
\newcommand{\bdotprod}[2]{\left({#1} \cdot {#2}\right)}
\newcommand{\crossprod}[2]{{#1} \cross {#2}}
\newcommand{\tripleprod}[3]{\dotprod{\left(\crossprod{#1}{#2}\right)}{#3}}

\DeclareMathOperator{\Proj}{Proj}
\DeclareMathOperator{\Span}{span}
\DeclareMathOperator{\Sgn}{sgn}
\DeclareMathOperator{\Area}{Area}
\DeclareMathOperator{\Volume}{Volume}

%
% A few miscellaneous things specific to this document
%
\newcommand{\crossop}[1]{\crossprod{#1}{}}

% R2 vector.
\newcommand{\VectorTwo}[2]{
\begin{bmatrix}
 {#1} \\
 {#2}
\end{bmatrix}
}

\newcommand{\VectorN}[1]{
\begin{bmatrix}
{#1}_1 \\
{#1}_2 \\
\vdots \\
{#1}_N \\
\end{bmatrix}
}

\newcommand{\DETuvij}[4]{
\begin{vmatrix}
 {#1}_{#3} & {#1}_{#4} \\
 {#2}_{#3} & {#2}_{#4}
\end{vmatrix}
}

\newcommand{\DETuvwijk}[6]{
\begin{vmatrix}
 {#1}_{#4} & {#1}_{#5} & {#1}_{#6} \\
 {#2}_{#4} & {#2}_{#5} & {#2}_{#6} \\
 {#3}_{#4} & {#3}_{#5} & {#3}_{#6}
\end{vmatrix}
}

\newcommand{\DETuvwxijkl}[8]{
\begin{vmatrix}
 {#1}_{#5} & {#1}_{#6} & {#1}_{#7} & {#1}_{#8} \\
 {#2}_{#5} & {#2}_{#6} & {#2}_{#7} & {#2}_{#8} \\
 {#3}_{#5} & {#3}_{#6} & {#3}_{#7} & {#3}_{#8} \\
 {#4}_{#5} & {#4}_{#6} & {#4}_{#7} & {#4}_{#8} \\
\end{vmatrix}
}

%\newcommand{\DETuvwxyijklm}[10]{
%\begin{vmatrix}
% {#1}_{#6} & {#1}_{#7} & {#1}_{#8} & {#1}_{#9} & {#1}_{#10} \\
% {#2}_{#6} & {#2}_{#7} & {#2}_{#8} & {#2}_{#9} & {#2}_{#10} \\
% {#3}_{#6} & {#3}_{#7} & {#3}_{#8} & {#3}_{#9} & {#3}_{#10} \\
% {#4}_{#6} & {#4}_{#7} & {#4}_{#8} & {#4}_{#9} & {#4}_{#10} \\
% {#5}_{#6} & {#5}_{#7} & {#5}_{#8} & {#5}_{#9} & {#5}_{#10}
%\end{vmatrix}
%}

% R3 vector.
\newcommand{\VectorThree}[3]{
\begin{bmatrix}
 {#1} \\
 {#2} \\
 {#3}
\end{bmatrix}
}



\newcommand{\T}[0]{{\text{T}}}

%
% The real thing:
%

\usepackage[bookmarks=true]{hyperref}

                             % The preamble begins here.
\title{ Projection and Moore-Penrose vector inverse. }
\author{Peeter Joot \quad peeter.joot@gmail.com}         % Declares the author's name.
\date{ May 16, 2008.  Last Revision: $Date: 2009/02/22 15:11:52 $ }

\begin{document}             % End of preamble and beginning of text.

\maketitle{}

\section{ Projection and Moore-Penrose vector inverse. }

One can observe that the moore penrose left vector inverse $\Bv^+$ shows up in the projection matrix for a projection onto a line with a direction vector $\Bv$:

\begin{equation}
\Proj_\Bv(\Bx) = \Bv \underbrace{\inv{ \Bv^\T \Bv} \Bv^\T}_{\Bv^+} \Bx
\end{equation}

I don't know of any other ``application'' of this Moore-Penrose vector inverse in traditional matrix algebra.  As stated it's an interesting mathematical curiousity that yes one can define a vector inverse, however what would you do with it?

In geometric algebra we also have a vector inverse, but it plays a much more fundamental role, and does not have the restriction of only acting from the left and 
producing a scalar result.  As an example consider the projection, and rejection decomposition of a vector:

\begin{align*}
\Bx 
&= \Bv \inv{\Bv} \Bx \\
&= \Bv \left(\inv{\Bv} \cdot \Bx\right) + \Bv \left(\inv{\Bv} \wedge \Bx\right) \\
&= \Bv 
\left(
\frac{\Bv}{\Bv^2} \cdot 
 \Bx\right)
 + \Bv \left(\frac{\Bv}{\Bv^2} \wedge \Bx\right) \\
\end{align*}

In the above, $\frac{\Bv}{\Bv^2} \cdot = \frac{\Bv^\T}{\Bv^\T \Bv} = \Bv^+$.  We can therefore describe the moore penrose vector left inverse as the matrix of the GA linear transformation $\inv{\Bv} \cdot$.

Unlike the GA vector inverse, whos associativity allowed for the projection/rejection derivation above, this Moore-Penrose vector inverse has only left action, so in the above, you can't further write:

\[
\Bv \Bv^{+} = 1
\]

(ie: $\Bv \Bv^{+}$ is a projection matrix not scalar or matrix unity).

\subsection{ matrix of wedge project transformation? }

Q: What's the matrix of the linear transformation $\inv{\Bv} \wedge$?

In rigid body dynamics we see the matrix of the linear transformation $T_\Bv(\Bx) = (\Bv \cross)(\Bx)$.  This is the completely antisymetric matrix as follows:

\begin{equation}
\Bv \times \Bx = 
\begin{bmatrix}
0 & -v_3 & v_2 \\
v_3 & 0 & -v_1 \\
-v_2 & v_1 & 0 \\
\end{bmatrix}
\begin{bmatrix}
x_1 \\
x_2 \\
x_3 \\
\end{bmatrix}
\end{equation}

In order to specify the matrix of a vector-vector wedge product linear transformation we must introduce bivector coordinate vectors.  For the matrix of the cross product linear transformation the standard vector basis was the obvious choice.

Let's pick the following orthonormal basis:

\[
\sigma = \{ \sigma_{ij} = \Be_i \wedge \Be_j \}_{i<j}
\]

and construct the matrix of the wedge project $T_\Bv : \mathbb{R}^N \rightarrow {\bigwedge}^2$

\[
T_\Bv(\Bx) = \Bv \wedge \Bx = \sum_{\mu = ij, i<j} \DETuvij{v}{x}{i}{j} \sigma_{\mu}
\]
\[
\implies
T_\Bv(\Be_k) \cdot {\sigma_{ij}}^\dagger = 
\sum_{k \in ij, i<j} \DETuvij{v}{x}{i}{j} 
= %\sum_{k \in ij, i<j} 
v_i \delta_{kj} - v_j \delta_{ki}
\]

Since $k$ cannot be simultaneously equal to both $i$, and $j$, this is:

\[
T_\Bv(\Be_k) \cdot {\sigma_{ij}}^\dagger = 
\left\{
\begin{array}{rl}
v_i & k=j \\
-v_j & k=i \\
0 & k \ne i,j \\
\end{array}
\right\}
\]

Unlike the left Moore-Penrose vector inverse that we find as the matrix of the linear transformation $v \cdot ( \cdot )$, except for \R{3} where we have the cross product, I do not recognize this as the matrix of any common linear transformation.

\end{document}               % End of document.
