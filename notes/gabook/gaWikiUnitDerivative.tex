\documentclass{article}      % Specifies the document class

\usepackage{amsmath}
\usepackage{mathpazo}

%
% shorthand for bold symbols, convenient for vectors and matrices
%
\newcommand{\Ba}[0]{\mathbf{a}}
\newcommand{\Bb}[0]{\mathbf{b}}
\newcommand{\Bc}[0]{\mathbf{c}}
\newcommand{\Bd}[0]{\mathbf{d}}
\newcommand{\Be}[0]{\mathbf{e}}
\newcommand{\Bf}[0]{\mathbf{f}}
\newcommand{\Bg}[0]{\mathbf{g}}
\newcommand{\Bh}[0]{\mathbf{h}}
\newcommand{\Bi}[0]{\mathbf{i}}
\newcommand{\Bj}[0]{\mathbf{j}}
\newcommand{\Bk}[0]{\mathbf{k}}
\newcommand{\Bl}[0]{\mathbf{l}}
\newcommand{\Bm}[0]{\mathbf{m}}
\newcommand{\Bn}[0]{\mathbf{n}}
\newcommand{\Bo}[0]{\mathbf{o}}
\newcommand{\Bp}[0]{\mathbf{p}}
\newcommand{\Bq}[0]{\mathbf{q}}
\newcommand{\Br}[0]{\mathbf{r}}
\newcommand{\Bs}[0]{\mathbf{s}}
\newcommand{\Bt}[0]{\mathbf{t}}
\newcommand{\Bu}[0]{\mathbf{u}}
\newcommand{\Bv}[0]{\mathbf{v}}
\newcommand{\Bw}[0]{\mathbf{w}}
\newcommand{\Bx}[0]{\mathbf{x}}
\newcommand{\By}[0]{\mathbf{y}}
\newcommand{\Bz}[0]{\mathbf{z}}
\newcommand{\BA}[0]{\mathbf{A}}
\newcommand{\BB}[0]{\mathbf{B}}
\newcommand{\BC}[0]{\mathbf{C}}
\newcommand{\BD}[0]{\mathbf{D}}
\newcommand{\BE}[0]{\mathbf{E}}
\newcommand{\BF}[0]{\mathbf{F}}
\newcommand{\BG}[0]{\mathbf{G}}
\newcommand{\BH}[0]{\mathbf{H}}
\newcommand{\BI}[0]{\mathbf{I}}
\newcommand{\BJ}[0]{\mathbf{J}}
\newcommand{\BK}[0]{\mathbf{K}}
\newcommand{\BL}[0]{\mathbf{L}}
\newcommand{\BM}[0]{\mathbf{M}}
\newcommand{\BN}[0]{\mathbf{N}}
\newcommand{\BO}[0]{\mathbf{O}}
\newcommand{\BP}[0]{\mathbf{P}}
\newcommand{\BQ}[0]{\mathbf{Q}}
\newcommand{\BR}[0]{\mathbf{R}}
\newcommand{\BS}[0]{\mathbf{S}}
\newcommand{\BT}[0]{\mathbf{T}}
\newcommand{\BU}[0]{\mathbf{U}}
\newcommand{\BV}[0]{\mathbf{V}}
\newcommand{\BW}[0]{\mathbf{W}}
\newcommand{\BX}[0]{\mathbf{X}}
\newcommand{\BY}[0]{\mathbf{Y}}
\newcommand{\BZ}[0]{\mathbf{Z}}

\newcommand{\Bzero}[0]{\mathbf{0}}
\newcommand{\Btheta}[0]{\boldsymbol{\theta}}
\newcommand{\Btau}[0]{\boldsymbol{\tau}}
\newcommand{\Bomega}[0]{\boldsymbol{\omega}}

%
% shorthand for unit vectors
%
\newcommand{\acap}[0]{\hat{\Ba}}
\newcommand{\bcap}[0]{\hat{\Bb}}
\newcommand{\ccap}[0]{\hat{\Bc}}
\newcommand{\dcap}[0]{\hat{\Bd}}
\newcommand{\ecap}[0]{\hat{\Be}}
\newcommand{\fcap}[0]{\hat{\Bf}}
\newcommand{\gcap}[0]{\hat{\Bg}}
\newcommand{\hcap}[0]{\hat{\Bh}}
\newcommand{\icap}[0]{\hat{\Bi}}
\newcommand{\jcap}[0]{\hat{\Bj}}
\newcommand{\kcap}[0]{\hat{\Bk}}
\newcommand{\lcap}[0]{\hat{\Bl}}
\newcommand{\mcap}[0]{\hat{\Bm}}
\newcommand{\ncap}[0]{\hat{\Bn}}
\newcommand{\ocap}[0]{\hat{\Bo}}
\newcommand{\pcap}[0]{\hat{\Bp}}
\newcommand{\qcap}[0]{\hat{\Bq}}
\newcommand{\rcap}[0]{\hat{\Br}}
\newcommand{\scap}[0]{\hat{\Bs}}
\newcommand{\tcap}[0]{\hat{\Bt}}
\newcommand{\ucap}[0]{\hat{\Bu}}
\newcommand{\vcap}[0]{\hat{\Bv}}
\newcommand{\wcap}[0]{\hat{\Bw}}
\newcommand{\xcap}[0]{\hat{\Bx}}
\newcommand{\ycap}[0]{\hat{\By}}
\newcommand{\zcap}[0]{\hat{\Bz}}
\newcommand{\thetacap}[0]{\hat{\Btheta}}

%
% to write R^n and C^n in a distinguishable fashion.  Perhaps change this
% to the double lined characters upon figuring out how to do so.
%
\newcommand{\C}[1]{$\mathbb{C}^{#1}$}
\newcommand{\R}[1]{$\mathbb{R}^{#1}$}

%
% various generally useful helpers
%

% derivative of #1 wrt. #2:
\newcommand{\D}[2] {\frac {d#2} {d#1}}

\newcommand{\inv}[1]{\frac{1}{#1}}
\newcommand{\cross}[0]{\times}

\newcommand{\abs}[1]{\lvert{#1}\rvert}
\newcommand{\norm}[1]{\lVert{#1}\rVert}
\newcommand{\innerprod}[2]{\langle{#1}, {#2}\rangle}
\newcommand{\dotprod}[2]{{#1} \cdot {#2}}
\newcommand{\bdotprod}[2]{\left({#1} \cdot {#2}\right)}
\newcommand{\crossprod}[2]{{#1} \cross {#2}}
\newcommand{\tripleprod}[3]{\dotprod{\left(\crossprod{#1}{#2}\right)}{#3}}

\DeclareMathOperator{\Proj}{Proj}
\DeclareMathOperator{\Span}{span}
\DeclareMathOperator{\Sgn}{sgn}
\DeclareMathOperator{\Area}{Area}
\DeclareMathOperator{\Volume}{Volume}

%
% A few miscellaneous things specific to this document
%
\newcommand{\crossop}[1]{\crossprod{#1}{}}

% R2 vector.
\newcommand{\VectorTwo}[2]{
\begin{bmatrix}
 {#1} \\
 {#2}
\end{bmatrix}
}

\newcommand{\VectorN}[1]{
\begin{bmatrix}
{#1}_1 \\
{#1}_2 \\
\vdots \\
{#1}_N \\
\end{bmatrix}
}

\newcommand{\DETuvij}[4]{
\begin{vmatrix}
 {#1}_{#3} & {#1}_{#4} \\
 {#2}_{#3} & {#2}_{#4}
\end{vmatrix}
}

\newcommand{\DETuvwijk}[6]{
\begin{vmatrix}
 {#1}_{#4} & {#1}_{#5} & {#1}_{#6} \\
 {#2}_{#4} & {#2}_{#5} & {#2}_{#6} \\
 {#3}_{#4} & {#3}_{#5} & {#3}_{#6}
\end{vmatrix}
}

\newcommand{\DETuvwxijkl}[8]{
\begin{vmatrix}
 {#1}_{#5} & {#1}_{#6} & {#1}_{#7} & {#1}_{#8} \\
 {#2}_{#5} & {#2}_{#6} & {#2}_{#7} & {#2}_{#8} \\
 {#3}_{#5} & {#3}_{#6} & {#3}_{#7} & {#3}_{#8} \\
 {#4}_{#5} & {#4}_{#6} & {#4}_{#7} & {#4}_{#8} \\
\end{vmatrix}
}

%\newcommand{\DETuvwxyijklm}[10]{
%\begin{vmatrix}
% {#1}_{#6} & {#1}_{#7} & {#1}_{#8} & {#1}_{#9} & {#1}_{#10} \\
% {#2}_{#6} & {#2}_{#7} & {#2}_{#8} & {#2}_{#9} & {#2}_{#10} \\
% {#3}_{#6} & {#3}_{#7} & {#3}_{#8} & {#3}_{#9} & {#3}_{#10} \\
% {#4}_{#6} & {#4}_{#7} & {#4}_{#8} & {#4}_{#9} & {#4}_{#10} \\
% {#5}_{#6} & {#5}_{#7} & {#5}_{#8} & {#5}_{#9} & {#5}_{#10}
%\end{vmatrix}
%}

% R3 vector.
\newcommand{\VectorThree}[3]{
\begin{bmatrix}
 {#1} \\
 {#2} \\
 {#3}
\end{bmatrix}
}



\newcommand{\dt}[1]{\frac{d {#1}}{dt}}

%
% The real thing:
%

                             % The preamble begins here.
\title{Derivatives of a unit vector}
\author{Peeter Joot}         % Declares the author's name.
%\date{}        % Deleting this command produces today's date.

\begin{document}             % End of preamble and beginning of text.

\maketitle{}

\section{First derivative of a unit vector}

\subsection{Expressed with the cross product.}

It can be shown that a unit vector derivative can be expressed using the cross product.  Two cross product operations are required to get the result back into the plane of the rotation, since a unit vector is constrained to circular (really perpendicular to itself) motion.

\[
\dt{}\left(\frac{\Br}{\Vert \Br \Vert}\right)
= \frac{1}{{\Vert \Br \Vert}^3}\left(\Br \times \dt{\Br}\right) \times \Br
= \left(\rcap \times \frac{1}{{\Vert \Br \Vert}} \dt{\Br}\right) \times \rcap
\]

This derivative is the rejective component of $\dt{\Br}$ with respect to $\rcap$, but is scaled by $1/\Vert \Br \Vert$.

How to calculate this result can be found in other places (Salus, and Hille Calculus for example).

\section{Equivalent result utilizing the geometric product}

The equivalent geometric product result can be obtained by calculating the derivative of a vector $\Br = r \rcap$.

\[
\dt{\Br} = r \dt{\rcap} + \rcap \dt{r} 
\]

\subsection{Taking dot products.}
One trick is required first (as was also the case in the Salus and Hille derivation), which is expressing $\dt{r}$ via the dot product.

\begin{align*}
\dt{(r^2)} &= 2r \dt{r} \\
\dt{(\Br \cdot \Br)} &= 2 \Br \cdot \dt{\Br} \\
\end{align*}

Thus,
\[
\dt{r} = \rcap \cdot \dt{\Br}
\]

Taking dot products of the derivative above yields

\begin{align*}
\rcap \cdot \dt{\Br} &= \rcap \cdot r \dt{\rcap} + \rcap \cdot \rcap \dt{r} \\
                            &= \Br \cdot \dt{\rcap} + \dt{r} \\
                            &= \Br \cdot \dt{\rcap} + \rcap \cdot \dt{\Br}
\end{align*}

\[
\implies
\Br \cdot \dt{\rcap} = \Bzero
\]

One could alternatively prove this with a diagram.


\subsection{Taking wedge products.}

As in linear equation solution, the $\rcap$ component can be eliminated by taking a wedge product

\begin{align*}
\rcap \wedge \dt{\Br} &= \rcap \wedge r \dt{\rcap} + \rcap \wedge \rcap \dt{r} \\
                             &= r \rcap \wedge \dt{\rcap} \\
                             &= \Br \wedge \dt{\rcap}  \\
                             &= \Br \wedge \dt{\rcap} + \Br \cdot \dt{\rcap} \\
                             &= \Br \dt{\rcap}
\end{align*}

This allows expression of $\dt{\rcap}$ in terms of $\dt{\Br}$ in various ways (compare to the cross product results above)

\begin{align*}
\dt{\rcap} &= \frac{1}{{ \Br }}\left(\rcap \wedge \dt{\Br}\right) \\
%                   &= \frac{1}{\Vert \Br \Vert}{     \frac{1}{\rcap} \left(\rcap \wedge \dt{\Br}\right)       } \\
                   &= \frac{1}{\Vert \Br \Vert}{     {\rcap} \left(\rcap \wedge \dt{\Br}\right)       } \\
%                   &= \frac{1}{{\Vert \Br \Vert}^3}{     {\Br} \left(\Br \wedge \dt{\Br}\right)       } \\
                   &= \frac{1}{\Vert \Br \Vert}\left({ \dt{\Br} - \rcap (\rcap \cdot \dt{\Br}) }\right) \\
\end{align*}

Thus this derivative is the component of
$\frac{1}{{\Vert \Br \Vert}}\dt{\Br}$
in the direction perpendicular to 
$\Br$.

\subsection{Another view.}

When the objective isn't comparing to the cross product, it's also notable that this unit vector derivative can be written

\[
{{ \Br }} \dt{\rcap}
= \rcap \wedge \dt{\Br}
\]

\subsection{A more direct route.}

Like a lot of stuff in math, once you know the answer you can get the answer more directly.  There's an unfortunate tendancy
in some math texts to skip the logical sequence and go straight to the end result by the quickest route.  This is more 
elegant

\begin{align*}
r \dt{\rcap} 
   &= \dt{\Br} - \rcap \dt{r} \\
   &= \dt{\Br} - \rcap\left(\rcap \cdot \dt{\Br}\right) \\
   &= \rcap \left(\rcap \dt{\Br} - \rcap \cdot \dt{\Br}\right) \\
   &= \rcap \left(\rcap \wedge \dt{\Br}\right) \\
\end{align*}

and gives the appearance of being clever, but it's easy to be clever when you already know the answer.

\subsection{total first derivative of a vector.}

Utilizing the previous result, we have the derivative of a radially expressed vector:

\begin{align*}
(r\rcap)'  
   &= r'\rcap  + r\rcap' \\
   &= r'\rcap  + \rcap(\rcap \wedge \Br') \\
\end{align*}

There are two components.  One is in the $\rcap$ direction (linear component)
and the other perpendicular to that (a rotational component) in the direction of the rejection
of $\rcap$ from $\Br'$.

\section{Second derivative of a vector}

Taking second derivatives of a radially expressed vector, we have

\begin{align*}
(r\rcap)'' 
   &= (r'\rcap + r{\rcap}')' \\
   &= r''\rcap + r'\rcap' + (r\rcap')' \\
   &= r''\rcap + (r'/r)\rcap(\rcap \wedge \Br') + (r\rcap')' \\
\end{align*}

Expanding the last term takes a bit more work
\begin{align*}
(r\rcap')' 
   &= (\rcap(\rcap \wedge \Br'))' \\
   &= 
\rcap'(\rcap \wedge \Br') +
\rcap(\rcap' \wedge \Br') +
\rcap(\rcap \wedge \Br'') \\
   &= 
(1/r)(\rcap(\rcap \wedge \Br'))(\rcap \wedge \Br') +
\rcap(\rcap' \wedge \Br') +
\rcap(\rcap \wedge \Br'') \\
   &= 
(1/r)\rcap(\rcap \wedge \Br')^2 +
\rcap(\rcap' \wedge \Br') +
\rcap(\rcap \wedge \Br'') \\
\end{align*}

There are three terms to this.  One a scalar (negative) multiple of $\rcap$, and another, the rejection of $\rcap$ from $\Br''$.  The middle term here remains to be expanded.  In particular,

\begin{align*}
\rcap' \wedge \Br' 
   &= \rcap' \wedge (r\rcap' + r'\rcap) \\
   &= r' \rcap' \wedge \rcap \\
   &= r'/2 (\rcap'\rcap - \rcap\rcap') \\
   &= r'/2r ((\Br' \wedge \rcap)\rcap\rcap - \rcap\rcap(\rcap \wedge \Br')) \\
   &= r'/2r (\Br' \wedge \rcap - \rcap \wedge \Br') \\
   &= -(r'/r) \rcap \wedge \Br' \\
\end{align*}

\begin{align*}
\implies
(r\rcap')' 
   &= 
(1/r)\rcap(\rcap \wedge \Br')^2 
-(r'/r)\rcap(\rcap \wedge \Br') 
+\rcap(\rcap \wedge \Br'') \\
\end{align*}

\begin{align*}
\implies
(r\rcap)'' 
   &= r''\rcap 
+(r'/r)\rcap(\rcap \wedge \Br')
+(1/r)\rcap(\rcap \wedge \Br')^2 
-(r'/r)\rcap(\rcap \wedge \Br')
+\rcap(\rcap \wedge \Br'') \\
   &= r''\rcap 
    +(1/r)\rcap(\rcap \wedge \Br')^2 
    +\rcap(\rcap \wedge \Br'') \\
   &= 
\rcap \left(  r'' +(1/r)(\rcap \wedge \Br')^2\right) +    \rcap(\rcap \wedge \Br'') \\
\end{align*}

There are two terms here that are in the $\rcap$ direction (the bivector square is a negative scalar), and
one rejective term in the direction of the component perpendicular to $\rcap$ relative to $\Br''$.

\end{document}               % End of document.
