\documentclass{article}

\usepackage{amsmath}
\usepackage{mathpazo}

%
% shorthand for bold symbols, convenient for vectors and matrices
%
\newcommand{\Ba}[0]{\mathbf{a}}
\newcommand{\Bb}[0]{\mathbf{b}}
\newcommand{\Bc}[0]{\mathbf{c}}
\newcommand{\Bd}[0]{\mathbf{d}}
\newcommand{\Be}[0]{\mathbf{e}}
\newcommand{\Bf}[0]{\mathbf{f}}
\newcommand{\Bg}[0]{\mathbf{g}}
\newcommand{\Bh}[0]{\mathbf{h}}
\newcommand{\Bi}[0]{\mathbf{i}}
\newcommand{\Bj}[0]{\mathbf{j}}
\newcommand{\Bk}[0]{\mathbf{k}}
\newcommand{\Bl}[0]{\mathbf{l}}
\newcommand{\Bm}[0]{\mathbf{m}}
\newcommand{\Bn}[0]{\mathbf{n}}
\newcommand{\Bo}[0]{\mathbf{o}}
\newcommand{\Bp}[0]{\mathbf{p}}
\newcommand{\Bq}[0]{\mathbf{q}}
\newcommand{\Br}[0]{\mathbf{r}}
\newcommand{\Bs}[0]{\mathbf{s}}
\newcommand{\Bt}[0]{\mathbf{t}}
\newcommand{\Bu}[0]{\mathbf{u}}
\newcommand{\Bv}[0]{\mathbf{v}}
\newcommand{\Bw}[0]{\mathbf{w}}
\newcommand{\Bx}[0]{\mathbf{x}}
\newcommand{\By}[0]{\mathbf{y}}
\newcommand{\Bz}[0]{\mathbf{z}}
\newcommand{\BA}[0]{\mathbf{A}}
\newcommand{\BB}[0]{\mathbf{B}}
\newcommand{\BC}[0]{\mathbf{C}}
\newcommand{\BD}[0]{\mathbf{D}}
\newcommand{\BE}[0]{\mathbf{E}}
\newcommand{\BF}[0]{\mathbf{F}}
\newcommand{\BG}[0]{\mathbf{G}}
\newcommand{\BH}[0]{\mathbf{H}}
\newcommand{\BI}[0]{\mathbf{I}}
\newcommand{\BJ}[0]{\mathbf{J}}
\newcommand{\BK}[0]{\mathbf{K}}
\newcommand{\BL}[0]{\mathbf{L}}
\newcommand{\BM}[0]{\mathbf{M}}
\newcommand{\BN}[0]{\mathbf{N}}
\newcommand{\BO}[0]{\mathbf{O}}
\newcommand{\BP}[0]{\mathbf{P}}
\newcommand{\BQ}[0]{\mathbf{Q}}
\newcommand{\BR}[0]{\mathbf{R}}
\newcommand{\BS}[0]{\mathbf{S}}
\newcommand{\BT}[0]{\mathbf{T}}
\newcommand{\BU}[0]{\mathbf{U}}
\newcommand{\BV}[0]{\mathbf{V}}
\newcommand{\BW}[0]{\mathbf{W}}
\newcommand{\BX}[0]{\mathbf{X}}
\newcommand{\BY}[0]{\mathbf{Y}}
\newcommand{\BZ}[0]{\mathbf{Z}}

\newcommand{\Bzero}[0]{\mathbf{0}}
\newcommand{\Btheta}[0]{\boldsymbol{\theta}}
\newcommand{\Btau}[0]{\boldsymbol{\tau}}
\newcommand{\Bomega}[0]{\boldsymbol{\omega}}

%
% shorthand for unit vectors
%
\newcommand{\acap}[0]{\hat{\Ba}}
\newcommand{\bcap}[0]{\hat{\Bb}}
\newcommand{\ccap}[0]{\hat{\Bc}}
\newcommand{\dcap}[0]{\hat{\Bd}}
\newcommand{\ecap}[0]{\hat{\Be}}
\newcommand{\fcap}[0]{\hat{\Bf}}
\newcommand{\gcap}[0]{\hat{\Bg}}
\newcommand{\hcap}[0]{\hat{\Bh}}
\newcommand{\icap}[0]{\hat{\Bi}}
\newcommand{\jcap}[0]{\hat{\Bj}}
\newcommand{\kcap}[0]{\hat{\Bk}}
\newcommand{\lcap}[0]{\hat{\Bl}}
\newcommand{\mcap}[0]{\hat{\Bm}}
\newcommand{\ncap}[0]{\hat{\Bn}}
\newcommand{\ocap}[0]{\hat{\Bo}}
\newcommand{\pcap}[0]{\hat{\Bp}}
\newcommand{\qcap}[0]{\hat{\Bq}}
\newcommand{\rcap}[0]{\hat{\Br}}
\newcommand{\scap}[0]{\hat{\Bs}}
\newcommand{\tcap}[0]{\hat{\Bt}}
\newcommand{\ucap}[0]{\hat{\Bu}}
\newcommand{\vcap}[0]{\hat{\Bv}}
\newcommand{\wcap}[0]{\hat{\Bw}}
\newcommand{\xcap}[0]{\hat{\Bx}}
\newcommand{\ycap}[0]{\hat{\By}}
\newcommand{\zcap}[0]{\hat{\Bz}}
\newcommand{\thetacap}[0]{\hat{\Btheta}}

%
% to write R^n and C^n in a distinguishable fashion.  Perhaps change this
% to the double lined characters upon figuring out how to do so.
%
\newcommand{\C}[1]{$\mathbb{C}^{#1}$}
\newcommand{\R}[1]{$\mathbb{R}^{#1}$}

%
% various generally useful helpers
%

% derivative of #1 wrt. #2:
\newcommand{\D}[2] {\frac {d#2} {d#1}}

\newcommand{\inv}[1]{\frac{1}{#1}}
\newcommand{\cross}[0]{\times}

\newcommand{\abs}[1]{\lvert{#1}\rvert}
\newcommand{\norm}[1]{\lVert{#1}\rVert}
\newcommand{\innerprod}[2]{\langle{#1}, {#2}\rangle}
\newcommand{\dotprod}[2]{{#1} \cdot {#2}}
\newcommand{\bdotprod}[2]{\left({#1} \cdot {#2}\right)}
\newcommand{\crossprod}[2]{{#1} \cross {#2}}
\newcommand{\tripleprod}[3]{\dotprod{\left(\crossprod{#1}{#2}\right)}{#3}}

\DeclareMathOperator{\Proj}{Proj}
\DeclareMathOperator{\Span}{span}
\DeclareMathOperator{\Sgn}{sgn}
\DeclareMathOperator{\Area}{Area}
\DeclareMathOperator{\Volume}{Volume}

%
% A few miscellaneous things specific to this document
%
\newcommand{\crossop}[1]{\crossprod{#1}{}}

% R2 vector.
\newcommand{\VectorTwo}[2]{
\begin{bmatrix}
 {#1} \\
 {#2}
\end{bmatrix}
}

\newcommand{\VectorN}[1]{
\begin{bmatrix}
{#1}_1 \\
{#1}_2 \\
\vdots \\
{#1}_N \\
\end{bmatrix}
}

\newcommand{\DETuvij}[4]{
\begin{vmatrix}
 {#1}_{#3} & {#1}_{#4} \\
 {#2}_{#3} & {#2}_{#4}
\end{vmatrix}
}

\newcommand{\DETuvwijk}[6]{
\begin{vmatrix}
 {#1}_{#4} & {#1}_{#5} & {#1}_{#6} \\
 {#2}_{#4} & {#2}_{#5} & {#2}_{#6} \\
 {#3}_{#4} & {#3}_{#5} & {#3}_{#6}
\end{vmatrix}
}

\newcommand{\DETuvwxijkl}[8]{
\begin{vmatrix}
 {#1}_{#5} & {#1}_{#6} & {#1}_{#7} & {#1}_{#8} \\
 {#2}_{#5} & {#2}_{#6} & {#2}_{#7} & {#2}_{#8} \\
 {#3}_{#5} & {#3}_{#6} & {#3}_{#7} & {#3}_{#8} \\
 {#4}_{#5} & {#4}_{#6} & {#4}_{#7} & {#4}_{#8} \\
\end{vmatrix}
}

%\newcommand{\DETuvwxyijklm}[10]{
%\begin{vmatrix}
% {#1}_{#6} & {#1}_{#7} & {#1}_{#8} & {#1}_{#9} & {#1}_{#10} \\
% {#2}_{#6} & {#2}_{#7} & {#2}_{#8} & {#2}_{#9} & {#2}_{#10} \\
% {#3}_{#6} & {#3}_{#7} & {#3}_{#8} & {#3}_{#9} & {#3}_{#10} \\
% {#4}_{#6} & {#4}_{#7} & {#4}_{#8} & {#4}_{#9} & {#4}_{#10} \\
% {#5}_{#6} & {#5}_{#7} & {#5}_{#8} & {#5}_{#9} & {#5}_{#10}
%\end{vmatrix}
%}

% R3 vector.
\newcommand{\VectorThree}[3]{
\begin{bmatrix}
 {#1} \\
 {#2} \\
 {#3}
\end{bmatrix}
}


\newcommand{\grad}[0]{\nabla}
\newcommand{\spacegrad}[0]{\boldsymbol{\nabla}}
\newcommand{\gpgrade}[2] {{\left\langle{{#1}}\right\rangle}_{#2}}
\newcommand{\gpgradezero}[1] {\gpgrade{#1}{0}}
\newcommand{\gpgradeone}[1] {\gpgrade{#1}{1}}
\newcommand{\gpgradetwo}[1] {\gpgrade{#1}{2}}
\newcommand{\gpgradethree}[1] {\gpgrade{#1}{3}}

% ointclockwise, ointctrclockwise
\usepackage{txfonts}

\usepackage[bookmarks=true]{hyperref}

\title{ Application of stokes integrals to Maxwell equation. }
\author{Peeter Joot}
\date{ Sept 26, 2008.  Last Revision: $Date: 2008/09/28 02:26:12 $ }

\begin{document}             % End of preamble and beginning of text.

\maketitle{}

\tableofcontents

\section{ Putting Maxwell's equation in curl form. }

These notes are followup for

\href{http://www.geocities.com/peeter_joot/geometric_algebra/vector_integral_relations.pdf}{vector\_integral\_relations.pdf}
, applying the bivector stokes volume equation

\begin{equation}\label{eqn:summaryStokesVolume}
\iiint (\grad \wedge F) \cdot d^3\Bx = \oiintclockwise F \cdot d^2\Bx,
\end{equation}

to the electromagnetic field bivector equation

\begin{equation}\label{eqn:maxwell}
\grad F = J/\epsilon_0 c.
\end{equation}

taking vector and trivector parts we have two equations
\begin{equation}\label{eqn:maxwellv}
\grad \cdot F = J/\epsilon_0 c,
\end{equation}

and
\begin{equation}\label{eqn:maxwellt}
\grad \wedge F = 0.
\end{equation}

\subsection{ Trivector equation part. }

The second of these, equation \ref{eqn:maxwellt}, we can apply Stokes to
directly:

\begin{equation}
\iiint (\grad \wedge F) \cdot d^3 \Bx = \oiintclockwise F \cdot d^2\Bx = 0.
\end{equation}

This area integral is a flux like quantity.  Suppose we call this the field flux, then this says
says
the flux of the combined electromagnetic field through any surface is zero
independent of the charge or current densities.
Note that here $d^3\Bx$
can be a regular spatial volume trivector element, but one can also pick
a spacetime (area times time) ``volume'' to integrate over, in which case
$d^2\Bx$ are the oriented ``surfaces'' of such a spacetime volume.

This doesn't seem like a result that I'm familiar with based on the traditional
vector forms of Maxwell�s equation.  Perhaps it is recognizable in terms of
$\BE$ and $\BB$ explicitly:

\begin{equation}\label{eqn:gaussmagnetostatics}
\oiintclockwise \BE \cdot d^2\Bx = - c \oiintclockwise \BB \cdot (d^2\Bx I)
\end{equation}

On the surface this doesn't look like a familiar identity.  It is in fact Gauss's law for magnetostatics, which will be shown
later.

Note also the subtle difference from traditional vector treatments where
$\BE$ and $\BB$ were spatial vectors.  Here they are writen as spacetime
bivectors,
$\BE = E^i \sigma_i = E^i \gamma_i \wedge \gamma_0$,
$\BB = B^i \sigma_i = B^i \gamma_i \wedge \gamma_0$.


\subsection{ Vector part. }

Moving on to the charge and current dependent vector terms
of Maxwell�s equation, we want express equation \ref{eqn:maxwellv} as a
spacetime curl so that we can apply stokes to it.

We can do this by temporarily writing our field in terms of a potential, and a
bivector dual to that.

\begin{align*}
F = \grad \wedge A = I D
\end{align*}
\begin{align*}
\grad F
&= \grad (\grad \wedge A) \\
&= \grad \cdot (\grad \wedge A) + \grad \wedge (\grad \wedge A) \\
&= \grad \cdot (I D) \\
&= \gpgradeone{ \grad I D } \\
&= -\gpgradeone{ I (\underbrace{\grad \cdot D}_{1-\text{vector}} + \underbrace{\grad \wedge D}_{3-\text{vector}}) } \\
&= - I (\grad \wedge D) \\
\end{align*}

or
\begin{align*}
I \grad F = \grad \wedge D.
\end{align*}

Applying stokes we have
\begin{align*}
\int (\grad \wedge D) \cdot d^3\Bx &= \oiintclockwise D \cdot d^2\Bx \\
\int (I \grad F) \cdot d^3\Bx
&= \oiintclockwise (-I F) \cdot d^2\Bx \\
&= \oiintclockwise \gpgradezero{ - F d^2\Bx I } \\
&= -\oiintclockwise F \cdot (d^2\Bx I) \\
\inv{\epsilon_0 c} \int (I J) \cdot d^3\Bx &= \\
\inv{\epsilon_0 c} \int \gpgradezero{ I J d^3\Bx } &= \\
\inv{\epsilon_0 c} \int \gpgradezero{ J d^3\Bx I } &= \\
\inv{\epsilon_0 c} \int J \cdot (d^3\Bx I) &= \\
\end{align*}

Or
\begin{equation}
\oiintctrclockwise F \cdot (d^2\Bx I) = \int \frac{J}{\epsilon_0 c} \cdot (d^3\Bx I)
\end{equation}

This is the integral form of the vector part of Maxwells equation \ref{eqn:maxwell}.
This doens't look terribly familiar, but we aren't used to
seeing Maxwell's equations in a non-disassembled form.
Hiding in there should be a subset of the 
traditional eight Maxwell's equations in integral form.  It will be 
possible to extract these by considering variations of current and charge density and different 
volume and surface integration regions.

\section{ Extracting the vector integral forms of Maxwell's equations. }

One can extract the integral forms of Maxwell's equations 
from \ref{eqn:maxwell}, by first extracting the differential vector
equations, and then using the spatial
divergence and stokes equations.
However, having formulated Stokes equation in its bivector form
we can go directly to those equations by appropriate selection of spatial
or spacetime volumes.
Of course we also now have new tools to work with the field in its entirety,
but lets use this as an exercise to verify that all the previous computation
that led to Stokes equation gives us the expected results.  In particular
this should be a good way to verify that
sign mistakes or other similar small errors (which would not be too hard)
have not been made.

\subsection{ Zero current density. Gauss's law for Electrostatics. }

With $J = c \rho \gamma_0$, the integral form of Maxwell's equation becomes

\begin{align*}
\oiintctrclockwise F \cdot (d^2\Bx I) 
&= \int \frac{\rho}{\epsilon_0} \gpgradezero{\gamma_0 d^3\Bx I} \\
&= \int \frac{\rho}{\epsilon_0} \gpgradezero{\gamma_{0123} \gamma_0 d^3\Bx} \\
&= -\inv{\epsilon_0} {\gamma_0}^2 \int \frac{\rho}{\epsilon_0} \gpgradezero{\gamma_{123} d^3\Bx} \\
\end{align*}

From this we see that, in the absence of currents the LHS integral must be zero unless the volume is purely spatial.  Denoting the boundary of a spacetime volume as $\partial A c t$, this is

\begin{align*}
\oiintctrclockwise_{\partial {A c t}} F \cdot (d^2\Bx I) &= 0.
\end{align*}

For a purely spatial volume the dual surfaces $d^2\Bx I$ always includes a spacetime bivector, therefore the magnetic field contributes nothing

\begin{equation*}
\oiintctrclockwise_{\partial V} I c B \cdot (d^2\Bx I) = 
-c \oiintctrclockwise_{\partial V} B \cdot d^2\Bx = 0
\end{equation*}

Although this looks similar to the integral equivalent of $\spacegrad \cdot B = 0$, we should look elsewhere for that since
that is true for the non-zero current density case too.

That leaves

\begin{align*}
\oiintctrclockwise E \cdot (d^2\Bx I) &= -\inv{\epsilon_0} {\gamma_0}^2 \int_{V} \rho \gpgradezero{\gamma_{123} d^3\Bx} \\
\end{align*}

Letting $d^3 \Bx = dx^1 dx^2 dx^3 \gamma_{123}$.  Within the charge integral becomes

\begin{align*}
-\inv{\epsilon_0} {\gamma_0}^2 \int_{V} \rho \gpgradezero{\gamma_{123} d^3\Bx} 
&=
\inv{\epsilon_0}
\underbrace{{\gamma_0}^2 {\gamma_1}^2}_{=-1}
\underbrace{{\gamma_2}^2 {\gamma_3}^2}_{=(\pm 1)^2}
 \int_{V} \rho dx^1 dx^2 dx^3
&= -\inv{\epsilon_0} \int_{V} \rho dx^1 dx^2 dx^3
\end{align*}

To put this in correspondence with the forms we are used to consider the surfaces separately.  For the dual to the
\href{http://www.geocities.com/peeter_joot/geometric_algebra/vector_integral_relations.pdf}{front surface} we have

\begin{align*}
d^2 \Bx I
&= dx^1 dx^2 \gamma_{12} I \\
&= dx^1 dx^2 \gamma_{120123} \\
&= dx^1 dx^2 \gamma_{112023} \\
&= -dx^1 dx^2 \gamma_{112203} \\
&= -(\pm 1)^2 dx^1 dx^2 \gamma_{03} \\
&= dx^1 dx^2 \sigma_3
\end{align*}

For the left surface
\begin{align*}
d^2 \Bx I
&= dx^3 dx^2 \gamma_{32} I \\
&= dx^3 dx^2 \gamma_{320123} \\
&= dx^3 dx^2 \gamma_{332012} \\
&= dx^3 dx^2 \gamma_{332201} \\
&= dx^3 dx^2 (\pm 1)^2 \gamma_{01} \\
&= -dx^3 dx^2 \sigma_1 \\
\end{align*}

and for the top
\begin{align*}
d^2 \Bx I
&= dx^1 dx^3 \gamma_{13} I \\
&= dx^1 dx^3 \gamma_{130123} \\
&= dx^1 dx^3 \gamma_{113023} \\
&= dx^1 dx^3 \gamma_{113302} \\
&= -dx^1 dx^3 \sigma_2 \\
\end{align*}

Assembling results, writing $(x^1, x^2, x^3) = (x,y,z)$ we have
%\begin{align*}
%\iint 
%(E_x(x, y, z_0) - E_x(x, y, z_1)) dx dy \\
%-\iint 
%(E_y(x_1, y, z) - E_y(x_0, y, z)) dy dz \\
%-\iint 
%(E_z(x, y_1, z) - E_z(x, y_0, z)) dx dz \\
%&= -\inv{\epsilon_0} \int_{V} \rho dx dy dz
%\end{align*}
\begin{align*}
\inv{\epsilon_0} \int_{V} \rho dx dy dz 
&=
\iint 
(E_x(x, y, z_1) - E_x(x, y, z_0)) dx dy \\
&+\iint 
(E_y(x_1, y, z) - E_y(x_0, y, z)) dy dz \\
&+\iint 
(E_z(x, y_1, z) - E_z(x, y_0, z)) dx dz \\
\end{align*}

This is Gauss's law for electrostatics in integral form

\begin{equation}
\iint \BE \cdot \ncap dA = \iiint \frac{\rho}{\epsilon_0} dV
\end{equation}

Although this extraction method is easy to understand, it is apparent that having only a pictorial way of enumerating the
oriented bivector
area elements is not efficient for high level computation.  Revisiting the stokes derivation with a more algebraic enumeration
of the surfaces should be done!

\subsection{ Gausses law for magnetostatics }

Return now to equation \ref{eqn:gaussmagnetostatics}, which resulted from considering the trivector part of Maxwell's equation

\begin{equation}
\oiintclockwise \BE \cdot d^2\Bx = - c \oiintclockwise \BB \cdot (d^2\Bx I).
\end{equation}

To start some observations can be made.

Only the spacetime surfaces of the volume
contribute to the LHS integral since $\sigma_i \cdot (\gamma_j \wedge \gamma_k) = 0$.

For the RHS, only the purely spatial surfaces contribute to that $\BB$ integral, since the dual surface $d^2\Bx I$ must have a spacetime component for that dot product to be non-zero.  We have also just enumerated these dual surface area elements $d^2 \Bx I$ for a purely 
spatial surface, therefore with a $E,B$ substutition we must have

\begin{align*}
0 &=
\iint 
(B_x(x, y, z_1) - B_x(x, y, z_0)) dx dy  \\
&+\iint 
(B_y(x_1, y, z) - B_y(x_0, y, z)) dy dz  \\
&+\iint 
(B_z(x, y_1, z) - B_z(x, y_0, z)) dx dz
\end{align*}

or, more compactly

\begin{equation}
\iint \BB \cdot \ncap dA = 0
\end{equation}

For any current or charge distribution.  We have therefore obtained two of the eight Maxwell's equations.

\end{document}               % End of document.
