\documentclass{article}      % Specifies the document class

\usepackage{amsmath}
\usepackage{mathpazo}

%
% shorthand for bold symbols, convenient for vectors and matrices
%
\newcommand{\Ba}[0]{\mathbf{a}}
\newcommand{\Bb}[0]{\mathbf{b}}
\newcommand{\Bc}[0]{\mathbf{c}}
\newcommand{\Bd}[0]{\mathbf{d}}
\newcommand{\Be}[0]{\mathbf{e}}
\newcommand{\Bf}[0]{\mathbf{f}}
\newcommand{\Bg}[0]{\mathbf{g}}
\newcommand{\Bh}[0]{\mathbf{h}}
\newcommand{\Bi}[0]{\mathbf{i}}
\newcommand{\Bj}[0]{\mathbf{j}}
\newcommand{\Bk}[0]{\mathbf{k}}
\newcommand{\Bl}[0]{\mathbf{l}}
\newcommand{\Bm}[0]{\mathbf{m}}
\newcommand{\Bn}[0]{\mathbf{n}}
\newcommand{\Bo}[0]{\mathbf{o}}
\newcommand{\Bp}[0]{\mathbf{p}}
\newcommand{\Bq}[0]{\mathbf{q}}
\newcommand{\Br}[0]{\mathbf{r}}
\newcommand{\Bs}[0]{\mathbf{s}}
\newcommand{\Bt}[0]{\mathbf{t}}
\newcommand{\Bu}[0]{\mathbf{u}}
\newcommand{\Bv}[0]{\mathbf{v}}
\newcommand{\Bw}[0]{\mathbf{w}}
\newcommand{\Bx}[0]{\mathbf{x}}
\newcommand{\By}[0]{\mathbf{y}}
\newcommand{\Bz}[0]{\mathbf{z}}
\newcommand{\BA}[0]{\mathbf{A}}
\newcommand{\BB}[0]{\mathbf{B}}
\newcommand{\BC}[0]{\mathbf{C}}
\newcommand{\BD}[0]{\mathbf{D}}
\newcommand{\BE}[0]{\mathbf{E}}
\newcommand{\BF}[0]{\mathbf{F}}
\newcommand{\BG}[0]{\mathbf{G}}
\newcommand{\BH}[0]{\mathbf{H}}
\newcommand{\BI}[0]{\mathbf{I}}
\newcommand{\BJ}[0]{\mathbf{J}}
\newcommand{\BK}[0]{\mathbf{K}}
\newcommand{\BL}[0]{\mathbf{L}}
\newcommand{\BM}[0]{\mathbf{M}}
\newcommand{\BN}[0]{\mathbf{N}}
\newcommand{\BO}[0]{\mathbf{O}}
\newcommand{\BP}[0]{\mathbf{P}}
\newcommand{\BQ}[0]{\mathbf{Q}}
\newcommand{\BR}[0]{\mathbf{R}}
\newcommand{\BS}[0]{\mathbf{S}}
\newcommand{\BT}[0]{\mathbf{T}}
\newcommand{\BU}[0]{\mathbf{U}}
\newcommand{\BV}[0]{\mathbf{V}}
\newcommand{\BW}[0]{\mathbf{W}}
\newcommand{\BX}[0]{\mathbf{X}}
\newcommand{\BY}[0]{\mathbf{Y}}
\newcommand{\BZ}[0]{\mathbf{Z}}

\newcommand{\Bzero}[0]{\mathbf{0}}
\newcommand{\Btheta}[0]{\boldsymbol{\theta}}
\newcommand{\Btau}[0]{\boldsymbol{\tau}}
\newcommand{\Bomega}[0]{\boldsymbol{\omega}}

%
% shorthand for unit vectors
%
\newcommand{\acap}[0]{\hat{\Ba}}
\newcommand{\bcap}[0]{\hat{\Bb}}
\newcommand{\ccap}[0]{\hat{\Bc}}
\newcommand{\dcap}[0]{\hat{\Bd}}
\newcommand{\ecap}[0]{\hat{\Be}}
\newcommand{\fcap}[0]{\hat{\Bf}}
\newcommand{\gcap}[0]{\hat{\Bg}}
\newcommand{\hcap}[0]{\hat{\Bh}}
\newcommand{\icap}[0]{\hat{\Bi}}
\newcommand{\jcap}[0]{\hat{\Bj}}
\newcommand{\kcap}[0]{\hat{\Bk}}
\newcommand{\lcap}[0]{\hat{\Bl}}
\newcommand{\mcap}[0]{\hat{\Bm}}
\newcommand{\ncap}[0]{\hat{\Bn}}
\newcommand{\ocap}[0]{\hat{\Bo}}
\newcommand{\pcap}[0]{\hat{\Bp}}
\newcommand{\qcap}[0]{\hat{\Bq}}
\newcommand{\rcap}[0]{\hat{\Br}}
\newcommand{\scap}[0]{\hat{\Bs}}
\newcommand{\tcap}[0]{\hat{\Bt}}
\newcommand{\ucap}[0]{\hat{\Bu}}
\newcommand{\vcap}[0]{\hat{\Bv}}
\newcommand{\wcap}[0]{\hat{\Bw}}
\newcommand{\xcap}[0]{\hat{\Bx}}
\newcommand{\ycap}[0]{\hat{\By}}
\newcommand{\zcap}[0]{\hat{\Bz}}
\newcommand{\thetacap}[0]{\hat{\Btheta}}

%
% to write R^n and C^n in a distinguishable fashion.  Perhaps change this
% to the double lined characters upon figuring out how to do so.
%
\newcommand{\C}[1]{$\mathbb{C}^{#1}$}
\newcommand{\R}[1]{$\mathbb{R}^{#1}$}

%
% various generally useful helpers
%

% derivative of #1 wrt. #2:
\newcommand{\D}[2] {\frac {d#2} {d#1}}

\newcommand{\inv}[1]{\frac{1}{#1}}
\newcommand{\cross}[0]{\times}

\newcommand{\abs}[1]{\lvert{#1}\rvert}
\newcommand{\norm}[1]{\lVert{#1}\rVert}
\newcommand{\innerprod}[2]{\langle{#1}, {#2}\rangle}
\newcommand{\dotprod}[2]{{#1} \cdot {#2}}
\newcommand{\bdotprod}[2]{\left({#1} \cdot {#2}\right)}
\newcommand{\crossprod}[2]{{#1} \cross {#2}}
\newcommand{\tripleprod}[3]{\dotprod{\left(\crossprod{#1}{#2}\right)}{#3}}

\DeclareMathOperator{\Proj}{Proj}
\DeclareMathOperator{\Span}{span}
\DeclareMathOperator{\Sgn}{sgn}
\DeclareMathOperator{\Area}{Area}
\DeclareMathOperator{\Volume}{Volume}

%
% A few miscellaneous things specific to this document
%
\newcommand{\crossop}[1]{\crossprod{#1}{}}

% R2 vector.
\newcommand{\VectorTwo}[2]{
\begin{bmatrix}
 {#1} \\
 {#2}
\end{bmatrix}
}

\newcommand{\VectorN}[1]{
\begin{bmatrix}
{#1}_1 \\
{#1}_2 \\
\vdots \\
{#1}_N \\
\end{bmatrix}
}

\newcommand{\DETuvij}[4]{
\begin{vmatrix}
 {#1}_{#3} & {#1}_{#4} \\
 {#2}_{#3} & {#2}_{#4}
\end{vmatrix}
}

\newcommand{\DETuvwijk}[6]{
\begin{vmatrix}
 {#1}_{#4} & {#1}_{#5} & {#1}_{#6} \\
 {#2}_{#4} & {#2}_{#5} & {#2}_{#6} \\
 {#3}_{#4} & {#3}_{#5} & {#3}_{#6}
\end{vmatrix}
}

\newcommand{\DETuvwxijkl}[8]{
\begin{vmatrix}
 {#1}_{#5} & {#1}_{#6} & {#1}_{#7} & {#1}_{#8} \\
 {#2}_{#5} & {#2}_{#6} & {#2}_{#7} & {#2}_{#8} \\
 {#3}_{#5} & {#3}_{#6} & {#3}_{#7} & {#3}_{#8} \\
 {#4}_{#5} & {#4}_{#6} & {#4}_{#7} & {#4}_{#8} \\
\end{vmatrix}
}

%\newcommand{\DETuvwxyijklm}[10]{
%\begin{vmatrix}
% {#1}_{#6} & {#1}_{#7} & {#1}_{#8} & {#1}_{#9} & {#1}_{#10} \\
% {#2}_{#6} & {#2}_{#7} & {#2}_{#8} & {#2}_{#9} & {#2}_{#10} \\
% {#3}_{#6} & {#3}_{#7} & {#3}_{#8} & {#3}_{#9} & {#3}_{#10} \\
% {#4}_{#6} & {#4}_{#7} & {#4}_{#8} & {#4}_{#9} & {#4}_{#10} \\
% {#5}_{#6} & {#5}_{#7} & {#5}_{#8} & {#5}_{#9} & {#5}_{#10}
%\end{vmatrix}
%}

% R3 vector.
\newcommand{\VectorThree}[3]{
\begin{bmatrix}
 {#1} \\
 {#2} \\
 {#3}
\end{bmatrix}
}


\newcommand{\LL}[0]{\mathcal{L}}
\newcommand{\grad}[0]{\nabla}
\newcommand{\PD}[2]{\frac{\partial {#2}}{\partial {#1}}}
\newcommand{\xdot}[0]{\dot{x}}
\newcommand{\vdot}[0]{\dot{v}}
\newcommand{\mdot}[0]{\dot{m}}
\newcommand{\xddot}[0]{\ddot{x}}

%
% The real thing:
%

                             % The preamble begins here.
\title{ Equations of motion given mass variation with spacetime position. } % Declares the document's title.
\author{Peeter Joot}         % Declares the author's name.
\date{August 28, 2008}        % Deleting this command produces today's date.

\begin{document}             % End of preamble and beginning of text.

\maketitle{}

\section{}

Let
\begin{align*}
x &= \sum \gamma_{\mu} {x}^{\mu} \\
v &= \frac{dx}{d\tau} = \sum \gamma_{\mu} \xdot^{\mu}
\end{align*}

Where whatever spacetime basis you pick has a corresponding reciprocal frame defined implicitly by:

\begin{equation*}
\gamma^{\mu} \cdot \gamma_{\nu} = {\delta^{\mu}}_{\nu}
\end{equation*}

You could for example pick these so that these are othornormal with:

\begin{align*}
\gamma_{i}^2 &= \gamma_i \cdot \gamma_i = -1 \\
\gamma^{i} &= -\gamma_{i} \\
\gamma^{0} &= \gamma_{0} \\
\gamma_{0}^2 &= 1 \\
\gamma_{i} \cdot \gamma_0 &= 0
\end{align*}

ie: the frame vectors define the metric tensor implicitly:

\begin{equation*}
g_{\mu\nu} = \gamma_{\mu} \cdot \gamma_{\nu} =
\begin{bmatrix}
1 & 0 & 0 & 0 \\
0 & -1 & 0 & 0 \\
0 & 0 & -1 & 0 \\
0 & 0 & 0 & -1 \\
\end{bmatrix}
\end{equation*}

Now, my assumption is that given a Lagrangian of the form:

\begin{equation*}
\LL = \inv{2} m v^2 - \phi
\end{equation*}

That the equations of motion follow by computation of:

\begin{equation*}
\PD{x^{\mu}}{\LL} = \frac{d}{d\tau} \PD{\xdot^{\mu}}{\LL}
\end{equation*}

I don't have any proof of this (I don't yet know any calculus of variations, and this is a guess based on intuition).  It does however work out to get the covariant form of the Lorentz force law, so I think it is right.

To get the EOM we need the squared proper velocity.  This is just $c^2$.  Example: for an orthonormal spacetime frame one has:

\begin{align*}
v^2 &= 
\left(\gamma^0 c dt/d\tau + \sum \gamma_i dx/d\tau\right)^2  \\
&= \gamma \left(\gamma_0 c + \sum \gamma_i dx/dt\right)^2 \\
&= \gamma^2 \left(c^2 - \Bv^2\right) = c^2
\end{align*}

but if we leave this expressed in terms of coordinates (also don't have to assume the diagonal metric tensor, since we can use non-orthonormal basis vectors if desired) we have:

\begin{align*}
v^2 
&= \left(\sum \gamma_{\mu} \xdot^{\mu}\right) \cdot \left(\sum \gamma_{\nu} \xdot^{\nu}\right) \\
&= \sum \gamma_{\mu} \cdot \gamma_{\nu} \xdot^{\mu} \xdot^{\nu} \\
&= \sum g_{\mu\nu} \xdot^{\mu} \xdot^{\nu}
\end{align*}

Therefore the Lagrangian to minimize is:

\begin{equation*}
\LL = \inv{2} m \sum g_{\mu\nu} \xdot^{\mu} \xdot^{\nu} - \phi.
\end{equation*}

Performing the calculations for the EOM, and in this case, also allowing mass to be a function of space or time position ($m = m(x^{\mu})$)

\begin{align*}
\PD{x^{\mu}}{\LL} &= \frac{d}{d\tau} \PD{\xdot^{\mu}}{\LL} \\
- \PD{x^{\mu}}{\phi} + \inv{2} \PD{x^{\mu}}{m} \sum g_{\alpha\beta} \xdot^{\alpha} \xdot^{\beta} &= \\
- \PD{x^{\mu}}{\phi} + \inv{2} \PD{x^{\mu}}{m} v^2 &= \\
%- \PD{x^{\mu}}{\phi} + \inv{2} \PD{x^{\mu}}{m} c^2 &= \\
&= \inv{2} \frac{d}{d\tau} m \sum g_{\alpha\beta} \PD{x^{\mu}}{}\left(\xdot^{\alpha} \xdot^{\beta}\right) \\
&= \inv{2} \frac{d}{d\tau} m \sum g_{\alpha\beta} \left(\delta^{\mu\alpha} \xdot^{\beta} + \xdot^{\alpha} \delta^{\mu\beta}\right) \\
&= \frac{d}{d\tau} m \sum g_{\alpha\mu} \xdot^{\alpha} \\
&= \sum \PD{x^{\beta}}{m} \xdot^{\beta} g_{\alpha\mu} \xdot^{\alpha} + m g_{\alpha\mu} \xddot^{\alpha} \\
\end{align*}

Now, the metric tensor values can be removed by summing since they can be used to switch upper and lower indexes of the frame vectors:

\begin{align*}
\gamma_{\mu} &= \sum a^{\nu} \gamma^{\nu} \\
\gamma_{\mu} \cdot \gamma_{\beta}
&= \sum a^{\nu} \gamma^{\nu} \cdot \gamma_{\beta} \\
&= \sum a^{\nu} {\delta^{\nu}}_{\beta} \\
&= a^{\beta} \\
\implies \\
\gamma_{\mu}
&= \sum \gamma_{\mu} \cdot \gamma_{\nu} \gamma^{\nu} \\
&= \sum g_{\mu\nu} \gamma^{\nu} \\
\end{align*}

If you are already familiar with tensors then this may be obvious to you (but wasn't to me with only vector background).

Multiplying throughout by $\gamma^{\mu}$, and summing over $\mu$ one has:

\begin{align*}
\sum \gamma^{\mu} \left( - \PD{x^{\mu}}{\phi} + \inv{2} \PD{x^{\mu}}{m} v^2 \right) 
&= \sum \gamma^{\mu} \left(\PD{x^{\beta}}{m} \xdot^{\beta} g_{\alpha\mu} \xdot^{\alpha} + m g_{\alpha\mu} \xddot^{\alpha} \right) \\
- \left(\sum \gamma^{\mu} \PD{x^{\mu}}{}\right) \phi + \inv{2} v^2 \left(\sum \gamma^{\mu} \PD{x^{\mu}}{}\right) m &= \\
&= \sum \PD{x^{\beta}}{m} \xdot^{\beta} \gamma^{\mu} \gamma_{\alpha} \cdot \gamma_{\mu} \xdot^{\alpha} + m \gamma^{\mu} \gamma_{\alpha} \cdot \gamma_{\mu} \xddot^{\alpha}  \\
&= \sum \PD{x^{\beta}}{m} \xdot^{\beta} \gamma_{\alpha} \xdot^{\alpha} + m \gamma_{\alpha} \xddot^{\alpha}  \\
\end{align*}

Writing:
\begin{equation*}
\nabla = \sum \gamma^{\mu} \frac{\partial}{\partial x^{\mu}}
\end{equation*}

This is:
\begin{equation*}
- \grad \phi + \inv{2} v^2 \grad m = v \sum \PD{x^{\beta}}{m} \xdot^{\beta} + m \vdot 
\end{equation*}

However, 
\begin{align*}
(\grad m) \cdot v 
&= 
\left(\sum \gamma^{\mu} \PD{x^{\mu}}{m}\right) \cdot \left( \sum \gamma_{\nu} \xdot^{\nu} \right) \\
&= \sum \gamma^{\mu} \cdot \gamma_{\nu} \PD{x^{\mu}}{m} \xdot^{\nu} \\
&= \sum {\delta^{\mu}}_{\nu} \PD{x^{\mu}}{m} \xdot^{\nu} \\
&= \sum \PD{x^{\mu}}{m} \xdot^{\mu} = \frac{dm}{d\tau}
\end{align*}

That allows for expressing the EOM in strict vector form:
\begin{equation*}
- \grad \phi + \inv{2} v^2 \grad m = v \grad m \cdot v + m \vdot.
\end{equation*}

However, there is still an asymmetry here, as one would expect a $\mdot v$ term.  Regrouping slightly, and using some algebraic vector
manipulation we have:

\begin{align*}
m \vdot + v \grad m \cdot v - \inv{2} v^2 \grad m &= - \grad \phi \\
m \vdot + \inv{2} v ( \underbrace{2 \grad m \cdot v - v \grad m}_{2 a \cdot b - b a = a b}) &= \\
m \vdot + \inv{2} v (\grad m) v &= \\
m \vdot + \inv{2} (v \grad m) v &= \\
m \vdot + \inv{2} (2 v \cdot \grad m - \grad m v) v &= \\
m \vdot + (v \cdot \grad m) v - \inv{2} (\grad m v) v &= \\
m \vdot + \mdot v - \inv{2} \grad m (v v) &= \\
\implies \\
\frac{d (m v)}{d\tau} = m \vdot + \mdot v
&= \inv{2} \grad m c^2 -\grad \phi \\
&= -\grad \left(\phi - \inv{2} m c^2 \right) \\
&= -\grad \left(\phi - \inv{2} m v^2 \right) \\
\end{align*}

So, after a whole wack of algebra, the end result is to show the proper time varient of the Lagrangian equations imply that our 
proper force can be expressed as a (spacetime) gradient.

The cavaet is that if the mass is allowed to vary, it also needs to be
included in the generalized potential associated with the equation of motion.

\subsection{ Summarizing. }

We took this Lagrangian with kinetic energy and non-velocity dependent potential terms, where the
mass in the kinetic energy term is allowed to vary with position or time.  That plus the
presumed proper-time Lagrange equations:

\begin{align*}
\LL &= \inv{2} m v^2 - \phi \\
\PD{x^{\mu}}{\LL} &= \frac{d}{d\tau} \PD{\xdot^{\mu}}{\LL},
\end{align*}

when followed to their algebraic conclusion together imply that the equation of motion is:

\begin{equation*}
\frac{d (m v)}{d\tau} = \grad \LL,
\end{equation*}

\end{document}               % End of document.
