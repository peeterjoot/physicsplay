\documentclass[]{eliblog}

\usepackage{amsmath}
\usepackage{mathpazo}

%
% shorthand for bold symbols, convenient for vectors and matrices
%
\newcommand{\Ba}[0]{\mathbf{a}}
\newcommand{\Bb}[0]{\mathbf{b}}
\newcommand{\Bc}[0]{\mathbf{c}}
\newcommand{\Bd}[0]{\mathbf{d}}
\newcommand{\Be}[0]{\mathbf{e}}
\newcommand{\Bf}[0]{\mathbf{f}}
\newcommand{\Bg}[0]{\mathbf{g}}
\newcommand{\Bh}[0]{\mathbf{h}}
\newcommand{\Bi}[0]{\mathbf{i}}
\newcommand{\Bj}[0]{\mathbf{j}}
\newcommand{\Bk}[0]{\mathbf{k}}
\newcommand{\Bl}[0]{\mathbf{l}}
\newcommand{\Bm}[0]{\mathbf{m}}
\newcommand{\Bn}[0]{\mathbf{n}}
\newcommand{\Bo}[0]{\mathbf{o}}
\newcommand{\Bp}[0]{\mathbf{p}}
\newcommand{\Bq}[0]{\mathbf{q}}
\newcommand{\Br}[0]{\mathbf{r}}
\newcommand{\Bs}[0]{\mathbf{s}}
\newcommand{\Bt}[0]{\mathbf{t}}
\newcommand{\Bu}[0]{\mathbf{u}}
\newcommand{\Bv}[0]{\mathbf{v}}
\newcommand{\Bw}[0]{\mathbf{w}}
\newcommand{\Bx}[0]{\mathbf{x}}
\newcommand{\By}[0]{\mathbf{y}}
\newcommand{\Bz}[0]{\mathbf{z}}
\newcommand{\BA}[0]{\mathbf{A}}
\newcommand{\BB}[0]{\mathbf{B}}
\newcommand{\BC}[0]{\mathbf{C}}
\newcommand{\BD}[0]{\mathbf{D}}
\newcommand{\BE}[0]{\mathbf{E}}
\newcommand{\BF}[0]{\mathbf{F}}
\newcommand{\BG}[0]{\mathbf{G}}
\newcommand{\BH}[0]{\mathbf{H}}
\newcommand{\BI}[0]{\mathbf{I}}
\newcommand{\BJ}[0]{\mathbf{J}}
\newcommand{\BK}[0]{\mathbf{K}}
\newcommand{\BL}[0]{\mathbf{L}}
\newcommand{\BM}[0]{\mathbf{M}}
\newcommand{\BN}[0]{\mathbf{N}}
\newcommand{\BO}[0]{\mathbf{O}}
\newcommand{\BP}[0]{\mathbf{P}}
\newcommand{\BQ}[0]{\mathbf{Q}}
\newcommand{\BR}[0]{\mathbf{R}}
\newcommand{\BS}[0]{\mathbf{S}}
\newcommand{\BT}[0]{\mathbf{T}}
\newcommand{\BU}[0]{\mathbf{U}}
\newcommand{\BV}[0]{\mathbf{V}}
\newcommand{\BW}[0]{\mathbf{W}}
\newcommand{\BX}[0]{\mathbf{X}}
\newcommand{\BY}[0]{\mathbf{Y}}
\newcommand{\BZ}[0]{\mathbf{Z}}

\newcommand{\Bzero}[0]{\mathbf{0}}
\newcommand{\Btheta}[0]{\boldsymbol{\theta}}
\newcommand{\Btau}[0]{\boldsymbol{\tau}}
\newcommand{\Bomega}[0]{\boldsymbol{\omega}}

%
% shorthand for unit vectors
%
\newcommand{\acap}[0]{\hat{\Ba}}
\newcommand{\bcap}[0]{\hat{\Bb}}
\newcommand{\ccap}[0]{\hat{\Bc}}
\newcommand{\dcap}[0]{\hat{\Bd}}
\newcommand{\ecap}[0]{\hat{\Be}}
\newcommand{\fcap}[0]{\hat{\Bf}}
\newcommand{\gcap}[0]{\hat{\Bg}}
\newcommand{\hcap}[0]{\hat{\Bh}}
\newcommand{\icap}[0]{\hat{\Bi}}
\newcommand{\jcap}[0]{\hat{\Bj}}
\newcommand{\kcap}[0]{\hat{\Bk}}
\newcommand{\lcap}[0]{\hat{\Bl}}
\newcommand{\mcap}[0]{\hat{\Bm}}
\newcommand{\ncap}[0]{\hat{\Bn}}
\newcommand{\ocap}[0]{\hat{\Bo}}
\newcommand{\pcap}[0]{\hat{\Bp}}
\newcommand{\qcap}[0]{\hat{\Bq}}
\newcommand{\rcap}[0]{\hat{\Br}}
\newcommand{\scap}[0]{\hat{\Bs}}
\newcommand{\tcap}[0]{\hat{\Bt}}
\newcommand{\ucap}[0]{\hat{\Bu}}
\newcommand{\vcap}[0]{\hat{\Bv}}
\newcommand{\wcap}[0]{\hat{\Bw}}
\newcommand{\xcap}[0]{\hat{\Bx}}
\newcommand{\ycap}[0]{\hat{\By}}
\newcommand{\zcap}[0]{\hat{\Bz}}
\newcommand{\thetacap}[0]{\hat{\Btheta}}

%
% to write R^n and C^n in a distinguishable fashion.  Perhaps change this
% to the double lined characters upon figuring out how to do so.
%
\newcommand{\C}[1]{$\mathbb{C}^{#1}$}
\newcommand{\R}[1]{$\mathbb{R}^{#1}$}

%
% various generally useful helpers
%

% derivative of #1 wrt. #2:
\newcommand{\D}[2] {\frac {d#2} {d#1}}

\newcommand{\inv}[1]{\frac{1}{#1}}
\newcommand{\cross}[0]{\times}

\newcommand{\abs}[1]{\lvert{#1}\rvert}
\newcommand{\norm}[1]{\lVert{#1}\rVert}
\newcommand{\innerprod}[2]{\langle{#1}, {#2}\rangle}
\newcommand{\dotprod}[2]{{#1} \cdot {#2}}
\newcommand{\bdotprod}[2]{\left({#1} \cdot {#2}\right)}
\newcommand{\crossprod}[2]{{#1} \cross {#2}}
\newcommand{\tripleprod}[3]{\dotprod{\left(\crossprod{#1}{#2}\right)}{#3}}

\DeclareMathOperator{\Proj}{Proj}
\DeclareMathOperator{\Span}{span}
\DeclareMathOperator{\Sgn}{sgn}
\DeclareMathOperator{\Area}{Area}
\DeclareMathOperator{\Volume}{Volume}

%
% A few miscellaneous things specific to this document
%
\newcommand{\crossop}[1]{\crossprod{#1}{}}

% R2 vector.
\newcommand{\VectorTwo}[2]{
\begin{bmatrix}
 {#1} \\
 {#2}
\end{bmatrix}
}

\newcommand{\VectorN}[1]{
\begin{bmatrix}
{#1}_1 \\
{#1}_2 \\
\vdots \\
{#1}_N \\
\end{bmatrix}
}

\newcommand{\DETuvij}[4]{
\begin{vmatrix}
 {#1}_{#3} & {#1}_{#4} \\
 {#2}_{#3} & {#2}_{#4}
\end{vmatrix}
}

\newcommand{\DETuvwijk}[6]{
\begin{vmatrix}
 {#1}_{#4} & {#1}_{#5} & {#1}_{#6} \\
 {#2}_{#4} & {#2}_{#5} & {#2}_{#6} \\
 {#3}_{#4} & {#3}_{#5} & {#3}_{#6}
\end{vmatrix}
}

\newcommand{\DETuvwxijkl}[8]{
\begin{vmatrix}
 {#1}_{#5} & {#1}_{#6} & {#1}_{#7} & {#1}_{#8} \\
 {#2}_{#5} & {#2}_{#6} & {#2}_{#7} & {#2}_{#8} \\
 {#3}_{#5} & {#3}_{#6} & {#3}_{#7} & {#3}_{#8} \\
 {#4}_{#5} & {#4}_{#6} & {#4}_{#7} & {#4}_{#8} \\
\end{vmatrix}
}

%\newcommand{\DETuvwxyijklm}[10]{
%\begin{vmatrix}
% {#1}_{#6} & {#1}_{#7} & {#1}_{#8} & {#1}_{#9} & {#1}_{#10} \\
% {#2}_{#6} & {#2}_{#7} & {#2}_{#8} & {#2}_{#9} & {#2}_{#10} \\
% {#3}_{#6} & {#3}_{#7} & {#3}_{#8} & {#3}_{#9} & {#3}_{#10} \\
% {#4}_{#6} & {#4}_{#7} & {#4}_{#8} & {#4}_{#9} & {#4}_{#10} \\
% {#5}_{#6} & {#5}_{#7} & {#5}_{#8} & {#5}_{#9} & {#5}_{#10}
%\end{vmatrix}
%}

% R3 vector.
\newcommand{\VectorThree}[3]{
\begin{bmatrix}
 {#1} \\
 {#2} \\
 {#3}
\end{bmatrix}
}



\author{Peeter Joot}
\email{peeter.joot@gmail.com}


\chapter{Relativistic classical proton electron interaction.}
\label{chap:nuclearInteraction}
%\useCCL
\blogpage{http://sites.google.com/site/peeterjoot/math2009/nuclearInteraction.pdf}
\date{Sept 13, 2009}
\revisionInfo{$RCSfile: nuclearInteraction.tex,v $ Last $Revision: 1.4 $ $Date: 2009/09/14 14:34:31 $}

\beginArtWithToc
%\beginArtNoToc

\section{Motivation}

The problem of a solving for the relativistically correct tragectories of classically interacting proton and electron is one that I've wanted to try for a while.  Conceptually this is just about the simplest interaction problem in electrodynamics (other than motion of a particle in a field), but it is not obvious to me how to even set up the right equations to solve.

Familiarity with Geometric Algebra, and the STA form of the Maxwell and Lorentz force equation will be assumed.  Writing $F = \BE + c I \BB$ for the Faraday bivector, these equations are respectively

\begin{align}\label{eqn:nuclearInteraction:boo1}
\grad F &= J/\epsilon_0 c \\
m\frac{d^2 X}{d\tau} &= \frac{q}{c} F \cdot \frac{dX}{d\tau}
\end{align}

%  To avoid confusion no use of $F$ or $\BP$ for force will be used here, instead using $d\BP/d\tau$.
The possibility of self interaction will also be ignored here.  From what I have read this self interaction is more complex than regular two particle interaction.

\section{With only Coulomb interaction.}

With just Coulomb (non-relativistic) interaction setup of the equations of motion for the relative vector difference between the particles is straightforward.  Let's write this out as a reference.  Whatever we come up with for the relativistic case should reduce to this at small velocities.

Fixing notation, lets write the proton and electron positions respectively by $\Br_p$ and $\Br_e$, the proton charge as $Z e$, and the electron charge $-e$.  For the forces we have

FIXME: picture

\begin{align}\label{eqn:nuclearInteraction:hoo1}
\text{Force on electron} &= m_e \frac{d^2 \Br_e}{dt^2} = - \inv{4 \pi \epsilon_0} Z e^2 \frac{\Br_e - \Br_p}{\Abs{\Br_e - \Br_p}^3} \\
\text{Force on proton} &= m_p \frac{d^2 \Br_p}{dt^2} = \inv{4 \pi \epsilon_0} Z e^2 \frac{\Br_e - \Br_p}{\Abs{\Br_e - \Br_p}^3}
\end{align}

Subtracting the two after mass division yields the reduced mass equation for the relative motion

\begin{align}\label{eqn:nuclearInteraction:hoo2}
\frac{d^2 (\Br_e -\Br_p)}{dt^2} = - \inv{4 \pi \epsilon_0} Z e^2 \left( \inv{m_e} + \inv{m_p}\right) \frac{\Br_e - \Br_p}{\Abs{\Br_e - \Br_p}^3} 
\end{align}

This is now of the same form as the classical problem of two particle gravitational interaction, with the well known conic solutions.

\section{Using the divergence equation instead.}

While use of the Coulomb force above provides the equation of motion for the relative motion of the charges, how to generalize this to the relativistic case is not entirely clear.  For the relativistic case we need to consider all of Maxwell's equations, and not just the divergence equation.  Let's back up a step and setup the problem using the divergence equation instead of Coulomb's law.  This is a bit closer to the use of all of Maxwell's equations.

To start off we need a discrete charge expression for the charge density, and can use the delta distribution to express this.

\begin{align}\label{eqn:nuclearInteraction:boo2}
0 = \int d^3 x \left( \spacegrad \cdot \BE - \inv{\epsilon_0} \left( Z e \delta^3(\Bx - \Br_p) - e \delta^3(\Bx - \Br_e) \right) \right)
\end{align}

Picking a volume element that only encloses one of the respective charges gives us the Coulomb law for the field produced by those charges as above

\begin{align}\label{eqn:nuclearInteraction:boo3}
0 &= \int_{\text{Volume around proton only}} d^3 x \left( \spacegrad \cdot \BE_p - \inv{\epsilon_0} Z e \delta^3(\Bx - \Br_p) \right) \\
0 &= \int_{\text{Volume around electron only}} d^3 x \left( \spacegrad \cdot \BE_e + \inv{\epsilon_0} e \delta^3(\Bx - \Br_e) \right)
\end{align}

Here $\BE_p$ and $\BE_e$ denote the electric fields due to the proton and electron respectively.  Ignoring the possibility of self interaction the Lorentz forces on the particles are

\begin{align*}
\text{Force on proton/electron} = \text{charge of proton/electron times field due to electron/proton}
\end{align*}

In symbols, this is

\begin{align}\label{eqn:nuclearInteraction:boo4}
m_p \frac{d^2 \Br_p}{dt^2} &= Z e \BE_e \\
m_e \frac{d^2 \Br_e}{dt^2} &= - e \BE_p
\end{align}

If we were to substite back into the volume integrals we'd have

\begin{align}\label{eqn:nuclearInteraction:boo5}
0 &= \int_{\text{Volume around proton only}} d^3 x \left( -\frac{m_e}{e}\spacegrad \cdot \frac{d^2 \Br_e}{dt^2} - \inv{\epsilon_0} Z e \delta^3(\Bx - \Br_p) \right) \\
0 &= \int_{\text{Volume around electron only}} d^3 x \left( \frac{m_p}{Z e}\spacegrad \cdot \frac{d^2 \Br_p}{dt^2} + \inv{\epsilon_0} e \delta^3(\Bx - \Br_e) \right)
\end{align}

It is tempting to take the differences of these two equations so that we can write this in terms of the relative acceleration $d^2 (\Br_e - \Br_p)/dt^2$.  I did just this initially, and was suprised by a mass term of the form $1/m_e - 1/m_p$ instead of reduced mass, which cannot be right.  The key to avoiding this mistake is the proper considerations of the integration volumes.  Since the volumes are different and can in fact be entirely disjoint, subtracting these is not possible.  For this reason we have to be especially careful if a differential form of the divergence integrals (\ref{eqn:nuclearInteraction:boo4}) were to be used, as in

\begin{align}\label{eqn:nuclearInteraction:boo6}
\spacegrad \cdot \BE_p &= \inv{\epsilon_0} Z e \delta^3(\Bx - \Br_p) \\
\spacegrad \cdot \BE_e &= -\inv{\epsilon_0} e \delta^3(\Bx - \Br_e) 
\end{align}

The domain of applicability of these equations is no longer explicit, since each has to omit a neighbourhood around the other charge.  When using a delta distribution to express the point charge density it is probably best to stick with an explicit integral form.

Comparing how far we can get starting with the Gauss's law instead of the Coulomb force, and looking forward to the relativistic case, it seems likely that solving the field equations due to the respective current densities will be the first required step.  Only then can we substitute that field solution back into the Lorentz force equation to complete the search for the particle trajectories.

\section{Relativistic interaction.}

Any self interaction effects will not be o
Since we are not considering the effect of 
The superposition of the fields from the two point particles wou
We seek the individual fields 
For the relativistic case, lets write the proton and electron worldlines in the observer frame (the origin) respectively by $X_p = (ct, \Br_p)$ and $X_e = (ct,\Br_e)$, the observer frame proper time as $\tau$.  We should be able to calculate the total field at any point in space with superposition of the individual fields

\begin{align}\label{eqn:nuclearInteraction:hoo3}
\text{Field due to electron acting on proton} &= F_e \equiv \BE_e + c I\BB_e \\
\text{Field due to proton acting on electron} &= F_p \equiv \BE_p + c I\BB_p
\end{align}

but for this problem, I believe that only the effect of the field on the opposing charge is required (we don't have to consider any other points in space).

That field effect 
can Each of the particles has an associated field, say
The Lorentz force equations, in covariant Geometric Algebra form, are a pair of proper force equations

\begin{align}\label{eqn:nuclearInteraction:hoo6}
\text{proper Force on electron} &= m_e \frac{d^2 X_e}{d\tau^2} = - e F_e \cdot \frac{d X_e}{c d\tau} \\
\text{proper Force on proton} &= m_p \frac{d^2 X_p}{d\tau^2} = Z e F_p \cdot \frac{d X_p}{c d\tau}
\end{align}


It should be possible to obtain the fields $F_e$ and $F_p$ by solving for the pair of Maxwell's equations

\begin{align}\label{eqn:nuclearInteraction:hoo4}
\grad F_e &= J_e/\epsilon_0 c \\
\grad F_p &= J_p/\epsilon_0 c
\end{align}

Here the $J$'s are the four vector current densities, each dependent on the particle trajectories, also to be determined.

Unfortunately we don't want the gradients of the fields, but the fields themselves, so life is made more complex.  That issue was avoided in the Coulomb case since we started not with $\spacegrad \cdot \BE = \rho/\epsilon_0$, but the solution to this divergence equation (the only part of Maxwell's equations left in the static limit).





With the question of how to solve or even express the respective fields sidestepped for now, we can at least express the proper force Lorentz interactions.  That is

\begin{align}\label{eqn:nuclearInteraction:hoo5}
J = Q \int d\tau' \frac{dX'}{d\tau'} \delta^4 (X' - X(\tau))
\end{align}


How to express $J_e$ and $J_p$ will have to be considered more carefully, and 
This is a set of second order four-vector equations, really eight equations, further coupled by the addition pair of eight equations for each of the fields.  What a mess!  So, where do we start?

One possible starting point is to encode all of the particle tragectory in an active Lorentz transformation and solve for that transformation.  This technique was used successfully in \cite{doran2003gap} for the single particle in a field problem, so it seems worthwhile to at least give it a try.

Suppose we relate our electron and proton event vector at two different proper times by a Lorentz transforms

\begin{align*}
X_e' &= X_e(\tau_0 + d\tau) = \tilde{R_e} X_e(\tau_0) R_e = X_e(\tau_0) + d\tau {\left. \frac{dX_e}{d\tau} \right\vert}_{\tau = \tau_0} + \cdots \\
X_p' &= X_p(\tau_0 + d\tau) = \tilde{R_p} X_p(\tau_0) R_p = X_p(\tau_0) + d\tau {\left. \frac{dX_p}{d\tau} \right\vert}_{\tau = \tau_0} + \cdots
\end{align*}

%\begin{align}\label{eqn:nuclearInteraction:hoo6}
%X' = X(\tau_0 + d\tau) = \tilde{R}(\tau) X(\tau_0) R(\tau)
%\end{align}
%
Inverting and taking derivatives, also utilizing $\tilde{R}{R} = 1$, we find the proper velocity expressed in terms of the commutator of the bivector $\Omega = (dR/d\tau) \tilde{R}$.   That is

\begin{align*}
\frac{dX}{d\tau} 
&= \frac{d}{d\tau} \left( R X' \tilde{R} \right) \\
&= \frac{dR }{d\tau} \tilde{R} X + X \underbrace{R \frac{d\tilde{R}}{d\tau}}_{= - (dR/d\tau) \tilde{R}} \\
&= \antisymmetric{ \Omega }{X}
\end{align*}



\EndArticle
%\EndNoBibArticle
