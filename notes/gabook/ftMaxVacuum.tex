\documentclass[]{eliblog}

\usepackage{amsmath}
\usepackage{mathpazo}

%
% shorthand for bold symbols, convenient for vectors and matrices
%
\newcommand{\Ba}[0]{\mathbf{a}}
\newcommand{\Bb}[0]{\mathbf{b}}
\newcommand{\Bc}[0]{\mathbf{c}}
\newcommand{\Bd}[0]{\mathbf{d}}
\newcommand{\Be}[0]{\mathbf{e}}
\newcommand{\Bf}[0]{\mathbf{f}}
\newcommand{\Bg}[0]{\mathbf{g}}
\newcommand{\Bh}[0]{\mathbf{h}}
\newcommand{\Bi}[0]{\mathbf{i}}
\newcommand{\Bj}[0]{\mathbf{j}}
\newcommand{\Bk}[0]{\mathbf{k}}
\newcommand{\Bl}[0]{\mathbf{l}}
\newcommand{\Bm}[0]{\mathbf{m}}
\newcommand{\Bn}[0]{\mathbf{n}}
\newcommand{\Bo}[0]{\mathbf{o}}
\newcommand{\Bp}[0]{\mathbf{p}}
\newcommand{\Bq}[0]{\mathbf{q}}
\newcommand{\Br}[0]{\mathbf{r}}
\newcommand{\Bs}[0]{\mathbf{s}}
\newcommand{\Bt}[0]{\mathbf{t}}
\newcommand{\Bu}[0]{\mathbf{u}}
\newcommand{\Bv}[0]{\mathbf{v}}
\newcommand{\Bw}[0]{\mathbf{w}}
\newcommand{\Bx}[0]{\mathbf{x}}
\newcommand{\By}[0]{\mathbf{y}}
\newcommand{\Bz}[0]{\mathbf{z}}
\newcommand{\BA}[0]{\mathbf{A}}
\newcommand{\BB}[0]{\mathbf{B}}
\newcommand{\BC}[0]{\mathbf{C}}
\newcommand{\BD}[0]{\mathbf{D}}
\newcommand{\BE}[0]{\mathbf{E}}
\newcommand{\BF}[0]{\mathbf{F}}
\newcommand{\BG}[0]{\mathbf{G}}
\newcommand{\BH}[0]{\mathbf{H}}
\newcommand{\BI}[0]{\mathbf{I}}
\newcommand{\BJ}[0]{\mathbf{J}}
\newcommand{\BK}[0]{\mathbf{K}}
\newcommand{\BL}[0]{\mathbf{L}}
\newcommand{\BM}[0]{\mathbf{M}}
\newcommand{\BN}[0]{\mathbf{N}}
\newcommand{\BO}[0]{\mathbf{O}}
\newcommand{\BP}[0]{\mathbf{P}}
\newcommand{\BQ}[0]{\mathbf{Q}}
\newcommand{\BR}[0]{\mathbf{R}}
\newcommand{\BS}[0]{\mathbf{S}}
\newcommand{\BT}[0]{\mathbf{T}}
\newcommand{\BU}[0]{\mathbf{U}}
\newcommand{\BV}[0]{\mathbf{V}}
\newcommand{\BW}[0]{\mathbf{W}}
\newcommand{\BX}[0]{\mathbf{X}}
\newcommand{\BY}[0]{\mathbf{Y}}
\newcommand{\BZ}[0]{\mathbf{Z}}

\newcommand{\Bzero}[0]{\mathbf{0}}
\newcommand{\Btheta}[0]{\boldsymbol{\theta}}
\newcommand{\Btau}[0]{\boldsymbol{\tau}}
\newcommand{\Bomega}[0]{\boldsymbol{\omega}}

%
% shorthand for unit vectors
%
\newcommand{\acap}[0]{\hat{\Ba}}
\newcommand{\bcap}[0]{\hat{\Bb}}
\newcommand{\ccap}[0]{\hat{\Bc}}
\newcommand{\dcap}[0]{\hat{\Bd}}
\newcommand{\ecap}[0]{\hat{\Be}}
\newcommand{\fcap}[0]{\hat{\Bf}}
\newcommand{\gcap}[0]{\hat{\Bg}}
\newcommand{\hcap}[0]{\hat{\Bh}}
\newcommand{\icap}[0]{\hat{\Bi}}
\newcommand{\jcap}[0]{\hat{\Bj}}
\newcommand{\kcap}[0]{\hat{\Bk}}
\newcommand{\lcap}[0]{\hat{\Bl}}
\newcommand{\mcap}[0]{\hat{\Bm}}
\newcommand{\ncap}[0]{\hat{\Bn}}
\newcommand{\ocap}[0]{\hat{\Bo}}
\newcommand{\pcap}[0]{\hat{\Bp}}
\newcommand{\qcap}[0]{\hat{\Bq}}
\newcommand{\rcap}[0]{\hat{\Br}}
\newcommand{\scap}[0]{\hat{\Bs}}
\newcommand{\tcap}[0]{\hat{\Bt}}
\newcommand{\ucap}[0]{\hat{\Bu}}
\newcommand{\vcap}[0]{\hat{\Bv}}
\newcommand{\wcap}[0]{\hat{\Bw}}
\newcommand{\xcap}[0]{\hat{\Bx}}
\newcommand{\ycap}[0]{\hat{\By}}
\newcommand{\zcap}[0]{\hat{\Bz}}
\newcommand{\thetacap}[0]{\hat{\Btheta}}

%
% to write R^n and C^n in a distinguishable fashion.  Perhaps change this
% to the double lined characters upon figuring out how to do so.
%
\newcommand{\C}[1]{$\mathbb{C}^{#1}$}
\newcommand{\R}[1]{$\mathbb{R}^{#1}$}

%
% various generally useful helpers
%

% derivative of #1 wrt. #2:
\newcommand{\D}[2] {\frac {d#2} {d#1}}

\newcommand{\inv}[1]{\frac{1}{#1}}
\newcommand{\cross}[0]{\times}

\newcommand{\abs}[1]{\lvert{#1}\rvert}
\newcommand{\norm}[1]{\lVert{#1}\rVert}
\newcommand{\innerprod}[2]{\langle{#1}, {#2}\rangle}
\newcommand{\dotprod}[2]{{#1} \cdot {#2}}
\newcommand{\bdotprod}[2]{\left({#1} \cdot {#2}\right)}
\newcommand{\crossprod}[2]{{#1} \cross {#2}}
\newcommand{\tripleprod}[3]{\dotprod{\left(\crossprod{#1}{#2}\right)}{#3}}

\DeclareMathOperator{\Proj}{Proj}
\DeclareMathOperator{\Span}{span}
\DeclareMathOperator{\Sgn}{sgn}
\DeclareMathOperator{\Area}{Area}
\DeclareMathOperator{\Volume}{Volume}

%
% A few miscellaneous things specific to this document
%
\newcommand{\crossop}[1]{\crossprod{#1}{}}

% R2 vector.
\newcommand{\VectorTwo}[2]{
\begin{bmatrix}
 {#1} \\
 {#2}
\end{bmatrix}
}

\newcommand{\VectorN}[1]{
\begin{bmatrix}
{#1}_1 \\
{#1}_2 \\
\vdots \\
{#1}_N \\
\end{bmatrix}
}

\newcommand{\DETuvij}[4]{
\begin{vmatrix}
 {#1}_{#3} & {#1}_{#4} \\
 {#2}_{#3} & {#2}_{#4}
\end{vmatrix}
}

\newcommand{\DETuvwijk}[6]{
\begin{vmatrix}
 {#1}_{#4} & {#1}_{#5} & {#1}_{#6} \\
 {#2}_{#4} & {#2}_{#5} & {#2}_{#6} \\
 {#3}_{#4} & {#3}_{#5} & {#3}_{#6}
\end{vmatrix}
}

\newcommand{\DETuvwxijkl}[8]{
\begin{vmatrix}
 {#1}_{#5} & {#1}_{#6} & {#1}_{#7} & {#1}_{#8} \\
 {#2}_{#5} & {#2}_{#6} & {#2}_{#7} & {#2}_{#8} \\
 {#3}_{#5} & {#3}_{#6} & {#3}_{#7} & {#3}_{#8} \\
 {#4}_{#5} & {#4}_{#6} & {#4}_{#7} & {#4}_{#8} \\
\end{vmatrix}
}

%\newcommand{\DETuvwxyijklm}[10]{
%\begin{vmatrix}
% {#1}_{#6} & {#1}_{#7} & {#1}_{#8} & {#1}_{#9} & {#1}_{#10} \\
% {#2}_{#6} & {#2}_{#7} & {#2}_{#8} & {#2}_{#9} & {#2}_{#10} \\
% {#3}_{#6} & {#3}_{#7} & {#3}_{#8} & {#3}_{#9} & {#3}_{#10} \\
% {#4}_{#6} & {#4}_{#7} & {#4}_{#8} & {#4}_{#9} & {#4}_{#10} \\
% {#5}_{#6} & {#5}_{#7} & {#5}_{#8} & {#5}_{#9} & {#5}_{#10}
%\end{vmatrix}
%}

% R3 vector.
\newcommand{\VectorThree}[3]{
\begin{bmatrix}
 {#1} \\
 {#2} \\
 {#3}
\end{bmatrix}
}



\author{Peeter Joot}
\email{peeter.joot@gmail.com}


\chapter{Energy and momentum for assumed Fourier transform solutions to the homogeneous Maxwell equation.}
\label{chap:ftMaxVacuum}
\blogpage{http://sites.google.com/site/peeterjoot/math2009/ftMaxVacuum.pdf}
\date{Dec 21, 2009}
\revisionInfo{ftMaxVacuum.tex}

%\beginArtWithToc
\beginArtNoToc

\section{Motivation and notation.}

In \chapcite{fourierMaxVac}, building on \cite{complexFieldEnergy} a derivation for the energy and momentum density was derived for an assumed Fourier series solution to the homogeneous Maxwell's equation.  Here we move to the continuous case examining Fourier transform solutions and the associated energy and momentum density.

A complex (phasor) representation is implied, so taking real parts when all is said and done is required of the fields.  For the energy momentum tensor the Geometric Algebra form, modified for complex fields, is used

\begin{align}
\label{eqn:ftMaxVacuum:1}
T(a) = -\frac{\epsilon_0}{2} \Real \Bigl( \conjugateStar{F} a F \Bigr).
\end{align}

The assumed four vector potential will be written

\begin{align}
\label{eqn:ftMaxVacuum:2}
A(\Bx, t) = A^\mu(\Bx, t) \gamma_\mu = \inv{(\sqrt{2 \pi})^3} \int A(\Bk, t) e^{i \Bk \cdot \Bx } d^3 \Bk.
\end{align}

Subject to the requirement that $A$ is a solution of Maxwell's equation

\begin{align}
\label{eqn:ftMaxVacuum:3}
\grad (\grad \wedge A) = 0.
\end{align}

To avoid latex hell, no special notation will be used for the Fourier coefficients,

\begin{align}
\label{eqn:ftMaxVacuum:3a}
A(\Bk, t) = \inv{(\sqrt{2 \pi})^3} \int A(\Bx, t) e^{-i \Bk \cdot \Bx } d^3 \Bx.
\end{align}

When convenient and unambiguous, this $(\Bk,t)$ dependence will be implied.

Having picked a time and space representation for the field, it will be natural to express both the four potential and the gradient as scalar plus spatial vector, instead of using the Dirac basis.  For the gradient this is

\begin{align}
\label{eqn:ftMaxVacuum:4}
\grad &= \gamma^\mu \partial_\mu = (\partial_0 - \spacegrad) \gamma_0 = \gamma_0 (\partial_0 + \spacegrad),
\end{align}

and for the four potential (or the Fourier transform functions), this is

\begin{align}
\label{eqn:ftMaxVacuum:5}
A &= \gamma_\mu A^\mu = (\phi + \BA) \gamma_0 = \gamma_0 (\phi - \BA).
\end{align}

\section{Setup}

The field bivector $F = \grad \wedge A$ is required for the energy momentum tensor.  This is

\begin{align*}
\grad \wedge A
&= \inv{2}\left( \rgrad A - A \lgrad \right) \\
&= \inv{2}\left( (\rpartial_0 - \rspacegrad) \gamma_0 \gamma_0 (\phi - \BA)
-
(\phi + \BA) \gamma_0 \gamma_0 (\lpartial_0 + \lspacegrad)
\right) \\
&= -\spacegrad \phi -\partial_0 \BA + \inv{2}(\rspacegrad \BA - \BA \lspacegrad)
\end{align*}

This last term is a spatial curl and the field is then

\begin{align}
\label{eqn:ftMaxVacuum:6}
F = -\spacegrad \phi -\partial_0 \BA + \spacegrad \wedge \BA
\end{align}

Applied to the Fourier representation this is

\begin{align}
\label{eqn:ftMaxVacuum:7}
F =
\inv{(\sqrt{2 \pi})^3} \int
\left(
- \inv{c} \dot{\BA}
- i \Bk \phi
+ i \Bk \wedge \BA
\right)
e^{i \Bk \cdot \Bx } d^3 \Bk.
\end{align}

It is only the real parts of this that we are actually interested in, unless physical meaning can be assigned to the complete complex vector field.

\section{Constraints supplied by Maxwell's equation.}

A Fourier transform solution of Maxwell's vacuum equation $\grad F = 0$ has been assumed.  Having expressed the Faraday bivector in terms of spatial vector quantities, it is more convienient to do this back substutition into after pre-multiplying Maxwell's equation by $\gamma_0$, namely

\begin{align}
\label{eqn:ftMaxVacuum:20}
0
&= \gamma_0 \grad F \\
&= (\partial_0 + \spacegrad) F.
\end{align}

Applied to the spatially decomposed field as specified in \autoref{eqn:ftMaxVacuum:6}, this is

\begin{align*}
0
&=
-\partial_0 \spacegrad \phi
-\partial_{00} \BA
+ \partial_0 \spacegrad \wedge \BA
-\spacegrad^2 \phi
- \spacegrad \partial_0 \BA
+ \spacegrad \cdot (\spacegrad \wedge \BA ) \\
\end{align*}

All grades of this equation must simulataneously equal zero, but the bivector grades cancel (assuming commuting space and time partials), leaving two equations of constraint for the system

\begin{subequations}
\label{eqn:ftMaxVacuum:22}
\begin{align}
0 &=
-\spacegrad^2 \phi - \spacegrad \cdot \partial_0 \BA
\label{eqn:ftMaxVacuum:22a}
\\
0 &=
\partial_{00} \BA - \spacegrad^2 \BA
- \spacegrad (\partial_0 \phi - \spacegrad \cdot \BA )
\label{eqn:ftMaxVacuum:22b}
\end{align}
\end{subequations}

It is immediately evident that a gauge transformation could be immediately helpful to simplify things.  In \cite{bohm1989qt} the gauge choice $\spacegrad \cdot \BA = 0$ is used.  From \autoref{eqn:ftMaxVacuum:22a} this implies that $\spacegrad^2 \phi = 0$.  Bohm argues that for this current and charge free case this implies $\phi = 0$, but he also has a periodicity constraint.  Without a periodicity constraint it is easy to manufacture non-zero counterexamples.  One is a linear function in the space and time coordinates

\begin{align}
\label{eqn:ftMaxVacuum:23}
\phi = p x + q y + r z + s t
\end{align}

This is a valid scalar potential provided that the wave equation for the vector potential is also a solution.  We can however, force $\phi = 0$ by making the transformation $A^\mu \rightarrow A^\mu + \partial^\mu \psi$, which in non-covariant notation is

\begin{align}
\label{eqn:ftMaxVacuum:24}
\phi &\rightarrow \phi + \inv{c} \partial_t \psi \\
\BA &\rightarrow \phi - \spacegrad \psi
\end{align}

If the transformed field $\phi' = \phi + \partial_t \psi/c$ can be forced to zero, then the complexity of the associated Maxwell equations are reduced.  In particular, antidifferentiation of $\phi = -(1/c) \partial_t \psi$, yields

\begin{align}
\label{eqn:ftMaxVacuum:25}
\psi(\Bx,t) = \psi(\Bx, 0) - c \int_{\tau=0}^t \phi(\Bx, \tau) d\tau.
\end{align}

Dropping primes, the transformed Maxwell equations now take the form

\begin{subequations}
\label{eqn:ftMaxVacuum:26}
\begin{align}
0 &= \partial_t( \spacegrad \cdot \BA )
\label{eqn:ftMaxVacuum:26a}
\\
0 &=
\partial_{00} \BA - \spacegrad^2 \BA + \spacegrad (\spacegrad \cdot \BA ).
\label{eqn:ftMaxVacuum:26b}
\end{align}
\end{subequations}

There are two classes of solutions that stand out for these equations.  If the vector potential is constant in time $\BA(\Bx,t) = \BA(\Bx)$, Maxwell's equations are reduced to the single equation

\begin{align}
\label{eqn:ftMaxVacuum:28}
0
&= - \spacegrad^2 \BA + \spacegrad (\spacegrad \cdot \BA ).
\end{align}

Observe that a gradient can be factored out of this equation

\begin{align*}
- \spacegrad^2 \BA + \spacegrad (\spacegrad \cdot \BA )
&=
\spacegrad (-\spacegrad \BA + \spacegrad \cdot \BA ) \\
&=
-\spacegrad (\spacegrad \wedge \BA).
\end{align*}

The solutions are then those $\BA$s that satisify both
\begin{subequations}
\label{eqn:ftMaxVacuum:28b}
\begin{align}
0 &= \partial_t \BA \\
0 &= \spacegrad (\spacegrad \wedge \BA).
\end{align}
\end{subequations}

In particular any non-time dependent potential $\BA$ with constant curl provides a solution to Maxwell's equations.  There may be other solutions to \autoref{eqn:ftMaxVacuum:28} too that are more general.  Returning to \autoref{eqn:ftMaxVacuum:26} a second way to satify these equations stands out.  Instead of requiring of $\BA$ constant curl, constant divergence with respect to the time partial eliminates \autoref{eqn:ftMaxVacuum:26a}.  The simplest resulting equations are those for which the divergence is a constant in time and space (such as zero).  The solution set are then spanned by the vectors $\BA$ for which

\begin{subequations}
\label{eqn:ftMaxVacuum:29}
\begin{align}
\text{constant} &= \spacegrad \cdot \BA \\
0 &= \inv{c^2} \partial_{tt} \BA - \spacegrad^2 \BA.
\end{align}
\end{subequations}

Any $\BA$ that both has constant divergence and satisfies the wave equation will via \autoref{eqn:ftMaxVacuum:6} then produce a solution to Maxwell's equation.

%Also consider the gauge transformation here.  Note that the gauge transformation shows that the $\Abs{\BA}^2$ belongs in the potential, while the $\Abs{\dot{\BA}/c}^2$ belongs in the kinetic.  Assume that the remainder of the position only and velocity only terms are grouped that way and see where it leads.

\section{Maxwell equation constraints applied to the assumed Fourier solutions.}

Let's consider Maxwell's equations in all three forms, \autoref{eqn:ftMaxVacuum:22}, \autoref{eqn:ftMaxVacuum:28b}, and \autoref{eqn:ftMaxVacuum:29} and apply these constraints to the assumed Fourier solution.

In all cases the starting point is a pair of fourier transform relationships, where the Fourier transforms are the functions to be determined

\begin{subequations}
\label{eqn:ftMaxVacuum:40}
\begin{align}
\phi(\Bx, t) &= (2 \pi)^{-3/2} \int \phi(\Bk, t) e^{i \Bk \cdot \Bx } d^3 \Bk 
\label{eqn:ftMaxVacuum:40a}
\\
\BA(\Bx, t) &= (2 \pi)^{-3/2} \int \BA(\Bk, t) e^{i \Bk \cdot \Bx } d^3 \Bk 
\label{eqn:ftMaxVacuum:40b}
\end{align}
\end{subequations}

\subsection{Case I.  Constant time vector potential.  Scalar potential eliminated by gauge transformation.}

From \autoref{eqn:ftMaxVacuum:40a} we require

\begin{align}\label{eqn:ftMaxVacuum:50}
0 = (2 \pi)^{-3/2} \int \partial_t \BA(\Bk, t) e^{i \Bk \cdot \Bx } d^3 \Bk.
\end{align}

So the fourier transform also cannot have any time dependence, and we have

\begin{align}\label{eqn:ftMaxVacuum:51}
\BA(\Bx, t) &= (2 \pi)^{-3/2} \int \BA(\Bk) e^{i \Bk \cdot \Bx } d^3 \Bk 
\end{align}

What is the curl of this?  Temporarily falling back to coordinates is easiest for this calculation

\begin{align*}
\spacegrad \wedge \BA(\Bk) e^{i\Bk \cdot \Bx}
&=
\sigma_m \partial_m \wedge \sigma_n A^n(\Bk) e^{i \Bx \cdot \Bx} \\
&=
\sigma_m \wedge \sigma_n A^n(\Bk) i k^m e^{i \Bx \cdot \Bx} \\
&=
i\Bk \wedge \BA(\Bk) e^{i \Bx \cdot \Bx} \\
\end{align*}

This gives

\begin{align}\label{eqn:ftMaxVacuum:52}
\spacegrad \wedge \BA(\Bx, t) &= (2 \pi)^{-3/2} \int i \Bk \wedge \BA(\Bk) e^{i \Bk \cdot \Bx } d^3 \Bk.
\end{align}

We want to equate the divergence of this to zero.  Neglecting the integral and constant factor this requires

\begin{align*}
0 
&= 
\spacegrad \cdot \left( i \Bk \wedge \BA e^{i\Bk \cdot \Bx} \right) \\
&= 
\gpgradeone{ \sigma_m \partial_m i (\Bk \wedge \BA) e^{i\Bk \cdot \Bx} } \\
&= 
-\gpgradeone{ \sigma_m (\Bk \wedge \BA) k^m e^{i\Bk \cdot \Bx} } \\
&= 
-\Bk \cdot (\Bk \wedge \BA) e^{i\Bk \cdot \Bx} \\
\end{align*}

Requiring that the plane spanned by $\Bk$ and $\BA(\Bk)$ be perpendicular to $\Bk$ implies that $\BA \propto \Bk$.  The solution set is then completely described by functions of the form

\begin{align}\label{eqn:ftMaxVacuum:53}
\BA(\Bx, t) &= (2 \pi)^{-3/2} \int \Bk \psi(\Bk) e^{i \Bk \cdot \Bx } d^3 \Bk,
\end{align}

where $\psi(\Bk)$ is an arbitrary scalar valued function.  This is however, an extremely uninteresting solution since the curl is uniformly zero

\begin{align*}
F 
&= \spacegrad \wedge \BA \\
&= (2 \pi)^{-3/2} \int (i \Bk) \wedge \Bk \psi(\Bk) e^{i \Bk \cdot \Bx } d^3 \Bk.
\end{align*}

Since $\Bk \wedge \Bk = 0$, when all is said and done the $\phi = 0$, $\partial_t \BA = 0$ case appears to have no non-trivial (zero) solutions.  Moving on, ...

\subsection{Case II.  Constant vector potential divergence.  Scalar potential eliminated by gauge transformation.}

Case: \autoref{eqn:ftMaxVacuum:29}

\subsection{Case III.  General case.  No gauge transformation.}

Case: \autoref{eqn:ftMaxVacuum:22}

\section{The energy momentum tensor}

The energy momentum tensor is then

\begin{align}
\label{eqn:ftMaxVacuum:8}
T(a) &= -\frac{\epsilon_0}{2 (2 \pi)^3} \Real \iint
\left(
- \inv{c} \conjugateStar{\dot{\BA}}(\Bk',t)
+ i \Bk' \conjugateStar{\phi}(\Bk', t)
- i \Bk' \wedge \conjugateStar{\BA}(\Bk', t)
\right)
a
\left(
- \inv{c} \dot{\BA}(\Bk, t)
- i \Bk \phi(\Bk, t)
+ i \Bk \wedge \BA(\Bk, t)
\right)
e^{i (\Bk -\Bk') \cdot \Bx } d^3 \Bk d^3 \Bk'.
\end{align}

Observing that $\gamma_0$ commutes with spatial bivectors and anticommutes with spatial vectors, and writing $\sigma_\mu = \gamma_\mu \gamma_0$, the tensor splits neatly into scalar and spatial vector components

\begin{subequations}
\label{eqn:ftMaxVacuum:16}
\begin{align}
T(\gamma_\mu) \cdot \gamma_0 &= \frac{\epsilon_0}{2 (2 \pi)^3} \Real \iint
\gpgradezero{
\left(
\inv{c} \conjugateStar{\dot{\BA}}(\Bk',t)
- i \Bk' \conjugateStar{\phi}(\Bk', t)
+ i \Bk' \wedge \conjugateStar{\BA}(\Bk', t)
\right)
\sigma_\mu
\left(
\inv{c} \dot{\BA}(\Bk, t)
+ i \Bk \phi(\Bk, t)
+ i \Bk \wedge \BA(\Bk, t)
\right)
}
e^{i (\Bk -\Bk') \cdot \Bx } d^3 \Bk d^3 \Bk' \\
T(\gamma_\mu) \wedge \gamma_0 &= \frac{\epsilon_0}{2 (2 \pi)^3} \Real \iint
\gpgradeone{
\left(
\inv{c} \conjugateStar{\dot{\BA}}(\Bk',t)
- i \Bk' \conjugateStar{\phi}(\Bk', t)
+ i \Bk' \wedge \conjugateStar{\BA}(\Bk', t)
\right)
\sigma_\mu
\left(
\inv{c} \dot{\BA}(\Bk, t)
+ i \Bk \phi(\Bk, t)
+ i \Bk \wedge \BA(\Bk, t)
\right)
}
e^{i (\Bk -\Bk') \cdot \Bx } d^3 \Bk d^3 \Bk'.
\end{align}
\end{subequations}

In particular for $\mu = 0$, we have

\begin{subequations}
\label{eqn:ftMaxVacuum:17}
\begin{align}
H &\equiv
T(\gamma_0) \cdot \gamma_0 = \frac{\epsilon_0}{2 (2 \pi)^3} \Real \iint
\left(
\left(
\inv{c} \conjugateStar{\dot{\BA}}(\Bk',t)
- i \Bk' \conjugateStar{\phi}(\Bk', t)
\right)
\cdot
\left(
\inv{c} \dot{\BA}(\Bk, t)
+ i \Bk \phi(\Bk, t)
\right)
- (\Bk' \wedge \conjugateStar{\BA}(\Bk', t)) \cdot (\Bk \wedge \BA(\Bk, t))
\right)
e^{i (\Bk -\Bk') \cdot \Bx } d^3 \Bk d^3 \Bk' \\
\BP &\equiv
T(\gamma_\mu) \wedge \gamma_0 = \frac{\epsilon_0}{2 (2 \pi)^3} \Real \iint
\left(
i
\left(
\inv{c} \conjugateStar{\dot{\BA}}(\Bk',t)
- i \Bk' \conjugateStar{\phi}(\Bk', t)
\right) \cdot
\left(
\Bk \wedge \BA(\Bk, t)
\right)
-i
\left(
\inv{c} \dot{\BA}(\Bk, t)
+ i \Bk \phi(\Bk, t)
\right)
\cdot
\left(
\Bk' \wedge \conjugateStar{\BA}(\Bk', t)
\right)
\right)
e^{i (\Bk -\Bk') \cdot \Bx } d^3 \Bk d^3 \Bk'.
\end{align}
\end{subequations}

Integrating this over all space and identification of the delta function

\begin{align}
\label{eqn:ftMaxVacuum:9}
\delta(\Bk) \equiv \inv{(2 \pi)^3} \int e^{i \Bk \cdot \Bx} d^3 \Bx,
\end{align}

reduces the tensor to a single integral in the continuous angular wave number space of $\Bk$.

\begin{align}
\label{eqn:ftMaxVacuum:10}
\int T(a) d^3 \Bx &= -\frac{\epsilon_0}{2} \Real \int
\left(
- \inv{c} \conjugateStar{\dot{\BA}}
+ i \Bk \conjugateStar{\phi}
- i \Bk \wedge \conjugateStar{\BA}
\right)
a
\left(
- \inv{c} \dot{\BA}
- i \Bk \phi
+ i \Bk \wedge \BA
\right)
d^3 \Bk.
\end{align}

XX: cut

\begin{align}
\label{eqn:ftMaxVacuum:12}
\int T(\gamma_\mu) \gamma_0 d^3 \Bx =
\frac{\epsilon_0}{2} \Real \int
\gpgrade{
\left(
\inv{c} \conjugateStar{\dot{\BA}}
- i \Bk \conjugateStar{\phi}
+ i \Bk \wedge \conjugateStar{\BA}
\right)
\sigma_\mu
\left(
\inv{c} \dot{\BA}
+ i \Bk \phi
+ i \Bk \wedge \BA
\right)
}{0,1}
d^3 \Bk.
\end{align}

The scalar and spatial vector grade selection operator has been added for convenience and does not change the result since those are necessarily the only grades anyhow.  The post multiplication by the observer frame time basis vector $\gamma_0$ serves to separate the energy and momentum like components of the tensor nicely into scalar and vector aspects.  In particular for $T(\gamma^0)$, one could write

\begin{align}
\label{eqn:ftMaxVacuum:13}
\int T(\gamma^0) d^3 \Bx = (H + \BP) \gamma_0,
\end{align}

If these are correctly identified with energy and momentum then it also ought to be true that we have the conservation relationship

\begin{align}
\label{eqn:ftMaxVacuum:13b}
\PD{t}{H} + \spacegrad \cdot (c \BP) = 0.
\end{align}

However, multiplying out \autoref{eqn:ftMaxVacuum:12} yields for $H$

\begin{align}
\label{eqn:ftMaxVacuum:14}
H &=
\frac{\epsilon_0}{2} \int d^3 \Bk \left(
\inv{c^2} \Abs{\dot{\BA}}^2 + \Bk^2 (\Abs{\phi}^2 + \Abs{\BA}^2 )
- \Abs{\Bk \cdot \BA}^2
+ 2 \frac{\Bk}{c} \cdot \Real( i \conjugateStar{\phi} \dot{\BA} )
\right)
\end{align}

The vector component takes a bit more work to reduce
\begin{align*}
\BP &=
\frac{\epsilon_0}{2} \int d^3 \Bk \Real \left(
\frac{i}{c} (\conjugateStar{\dot{\BA}} \cdot (\Bk \wedge \BA)
+ \conjugateStar{\phi} \Bk \cdot (\Bk \wedge \BA)
+ \frac{i}{c} (\Bk \wedge \conjugateStar{\BA}) \cdot \dot{\BA}
- \phi (\Bk \wedge \conjugateStar{\BA}) \cdot \Bk
\right) \\
&=
\frac{\epsilon_0}{2} \int d^3 \Bk \Real \left(
\frac{i}{c} \left( (\conjugateStar{\dot{\BA}} \cdot \Bk) \BA -(\conjugateStar{\dot{\BA}} \cdot \BA) \Bk \right)
+ \conjugateStar{\phi} \left( \Bk^2 \BA - (\Bk \cdot \BA) \Bk \right)
+ \frac{i}{c} \left( (\conjugateStar{\BA} \cdot \dot{\BA}) \Bk - (\Bk \cdot \dot{\BA}) \conjugateStar{\BA} \right)
+ \phi \left( \Bk^2 \conjugateStar{\BA} -(\conjugateStar{\BA} \cdot \Bk) \Bk \right)
\right).
\end{align*}

Canceling and regrouping leaves

\begin{align}
\label{eqn:ftMaxVacuum:15}
\BP
&=
\epsilon_0 \int d^3 \Bk \Real \left(
\BA \left( \Bk^2 \conjugateStar{\phi} + \Bk \cdot \conjugateStar{\dot{\BA}} \right)
+ \Bk \left( -\conjugateStar{\phi} (\Bk \cdot \BA) + \frac{i}{c} (\conjugateStar{\BA} \cdot \dot{\BA})
\right)
\right).
\end{align}

This has no explicit $\Bx$ dependence, so the conservation relation \autoref{eqn:ftMaxVacuum:13b} is violated unless $\PDi{t}{H} = 0$.  There is no reason to assume that will be the case.  In the discrete Fourier series treatment, a gauge transformation allowed for elimination of $\phi$, and this implied $\Bk \cdot \BA_\Bk = 0$ or $\BA_\Bk$ constant.  We will probably have a similar result here, eliminating most of the terms in \autoref{eqn:ftMaxVacuum:14} and \autoref{eqn:ftMaxVacuum:15}.  Except for the constant $\BA_\Bk$ solution of the field equations there is no obvious way that such a simplified energy expression will have zero derivative.

A more reasonable conclusion is that this approach is flawed.  We ought to be looking at the divergence relation as a starting point, and instead of integrating over all space, instead employing Gauss's theorem to convert the divergence integral into a surface integral.  Without math, the conservation relationship probably ought to be expressed as energy change in a volume is matched by the momentum change through the surface.  However, without an integral over all space, we do not get the nice delta function cancellation observed above.  How to proceed is not immediately clear.  Stepping back to review applications of Gauss's theorem is probably a good first step.

\section{Energy and Momentum change across the surface of a volume.}

Let's scrap the previous attempt at integration over all space.

Evaluating the time derivative of $H$ is simple since all the time dependence is in the Fourier transforms $\phi$, and $\BA$.  The divergence of $\BP$ is an operation of the following form

\begin{align*}
\spacegrad \cdot \int \BX e^{i \Bb \cdot \Bx } d^3 \Bk
&=
\int X^m (i k^m) e^{i \Bb \cdot \Bx } d^3 \Bk \\
&=
\int i \Bk \cdot \BX e^{i \Bb \cdot \Bx } d^3 \Bk \\
\end{align*}

The momentum divergence is then

\begin{align}
\label{eqn:ftMaxVacuum:18}
\spacegrad \cdot \BP &=
\frac{\epsilon_0}{2 (2 \pi)^3} \Real \iint
(\Bk - \Bk') \cdot
\left(
\left(
\inv{c} \dot{\BA}(\Bk, t)
+ i \Bk \phi(\Bk, t)
\right)
\cdot
\left(
\Bk' \wedge \conjugateStar{\BA}(\Bk', t)
\right)
-
\left(
\inv{c} \conjugateStar{\dot{\BA}}(\Bk',t)
- i \Bk' \conjugateStar{\phi}(\Bk', t)
\right) \cdot
\left(
\Bk \wedge \BA(\Bk, t)
\right)
\right)
e^{i (\Bk -\Bk') \cdot \Bx } d^3 \Bk d^3 \Bk'.
\end{align}

It appears that it is natural to rewrite this slightly, utilizing the identity $\Ba \cdot (\Bb \cdot (\Bc \wedge \Bd)) = (\Ba \wedge \Bb) \cdot (\Bc \wedge \Bd)$.  Because $\Ba \wedge \Ba = 0$ we loose a few terms doing so, but things may get messier, hopefully just temporarily.

First term:
\begin{align*}
(\Bk - \Bk') \cdot
\left(
\left(
\inv{c} \dot{\BA}(\Bk, t)
+ i \Bk \phi(\Bk, t)
\right)
\cdot
\left(
\Bk' \wedge \conjugateStar{\BA}(\Bk', t)
\right)
\right)
&=
\left( \inv{c} (\Bk \wedge \dot{\BA}(\Bk, t) )
- \inv{c} (\Bk' \wedge \dot{\BA}(\Bk, t) )
- i \phi(\Bk, t) (\Bk' \wedge \Bk) \right) \cdot \left( \Bk' \wedge \conjugateStar{\BA}(\Bk', t) \right)
\end{align*}

Second term:
\begin{align*}
-
(\Bk - \Bk') \cdot
\left(
\left( \inv{c} \conjugateStar{\dot{\BA}}(\Bk',t) - i \Bk' \conjugateStar{\phi}(\Bk', t) \right) \cdot
\left( \Bk \wedge \BA(\Bk, t) \right)
\right)
&=
\left( -\inv{c} (\Bk \wedge \conjugateStar{\dot{\BA}}(\Bk',t) )
+ i \conjugateStar{\phi}(\Bk', t) (\Bk \wedge \Bk')
+ \inv{c} (\Bk' \wedge \conjugateStar{\dot{\BA}}(\Bk',t) ) \right) \cdot \left( \Bk \wedge \BA(\Bk, t) \right)
\end{align*}

Doesn't appear helpful after all.  This only killed two of the products.  I could expand all the dot products, but that could be done just as easily using \autoref{eqn:ftMaxVacuum:18} as the starting point.  Will have to step back and consider exactly what I'm trying to accomplish here.


FIXME: Given an energy H and a set of generalized coordinates, how does one construct the canonical momenta without the Lagrangian.  How can one find the Lagrangian from the Hamiltonian, when the separate potential and kinetic terms are not known.  What is the potential and the Kinetic in this system (ie: $H = E^2 + B^2$).  Looks like there is a separation into velocity and non-velocity parts.  Using that could potentially produce a Lagrangian.

\EndArticle
