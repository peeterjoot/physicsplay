\documentclass{article}      % Specifies the document class

\usepackage{amsmath}
\usepackage{mathpazo}

%
% shorthand for bold symbols, convenient for vectors and matrices
%
\newcommand{\Ba}[0]{\mathbf{a}}
\newcommand{\Bb}[0]{\mathbf{b}}
\newcommand{\Bc}[0]{\mathbf{c}}
\newcommand{\Bd}[0]{\mathbf{d}}
\newcommand{\Be}[0]{\mathbf{e}}
\newcommand{\Bf}[0]{\mathbf{f}}
\newcommand{\Bg}[0]{\mathbf{g}}
\newcommand{\Bh}[0]{\mathbf{h}}
\newcommand{\Bi}[0]{\mathbf{i}}
\newcommand{\Bj}[0]{\mathbf{j}}
\newcommand{\Bk}[0]{\mathbf{k}}
\newcommand{\Bl}[0]{\mathbf{l}}
\newcommand{\Bm}[0]{\mathbf{m}}
\newcommand{\Bn}[0]{\mathbf{n}}
\newcommand{\Bo}[0]{\mathbf{o}}
\newcommand{\Bp}[0]{\mathbf{p}}
\newcommand{\Bq}[0]{\mathbf{q}}
\newcommand{\Br}[0]{\mathbf{r}}
\newcommand{\Bs}[0]{\mathbf{s}}
\newcommand{\Bt}[0]{\mathbf{t}}
\newcommand{\Bu}[0]{\mathbf{u}}
\newcommand{\Bv}[0]{\mathbf{v}}
\newcommand{\Bw}[0]{\mathbf{w}}
\newcommand{\Bx}[0]{\mathbf{x}}
\newcommand{\By}[0]{\mathbf{y}}
\newcommand{\Bz}[0]{\mathbf{z}}
\newcommand{\BA}[0]{\mathbf{A}}
\newcommand{\BB}[0]{\mathbf{B}}
\newcommand{\BC}[0]{\mathbf{C}}
\newcommand{\BD}[0]{\mathbf{D}}
\newcommand{\BE}[0]{\mathbf{E}}
\newcommand{\BF}[0]{\mathbf{F}}
\newcommand{\BG}[0]{\mathbf{G}}
\newcommand{\BH}[0]{\mathbf{H}}
\newcommand{\BI}[0]{\mathbf{I}}
\newcommand{\BJ}[0]{\mathbf{J}}
\newcommand{\BK}[0]{\mathbf{K}}
\newcommand{\BL}[0]{\mathbf{L}}
\newcommand{\BM}[0]{\mathbf{M}}
\newcommand{\BN}[0]{\mathbf{N}}
\newcommand{\BO}[0]{\mathbf{O}}
\newcommand{\BP}[0]{\mathbf{P}}
\newcommand{\BQ}[0]{\mathbf{Q}}
\newcommand{\BR}[0]{\mathbf{R}}
\newcommand{\BS}[0]{\mathbf{S}}
\newcommand{\BT}[0]{\mathbf{T}}
\newcommand{\BU}[0]{\mathbf{U}}
\newcommand{\BV}[0]{\mathbf{V}}
\newcommand{\BW}[0]{\mathbf{W}}
\newcommand{\BX}[0]{\mathbf{X}}
\newcommand{\BY}[0]{\mathbf{Y}}
\newcommand{\BZ}[0]{\mathbf{Z}}

\newcommand{\Bzero}[0]{\mathbf{0}}
\newcommand{\Btheta}[0]{\boldsymbol{\theta}}
\newcommand{\Btau}[0]{\boldsymbol{\tau}}
\newcommand{\Bomega}[0]{\boldsymbol{\omega}}

%
% shorthand for unit vectors
%
\newcommand{\acap}[0]{\hat{\Ba}}
\newcommand{\bcap}[0]{\hat{\Bb}}
\newcommand{\ccap}[0]{\hat{\Bc}}
\newcommand{\dcap}[0]{\hat{\Bd}}
\newcommand{\ecap}[0]{\hat{\Be}}
\newcommand{\fcap}[0]{\hat{\Bf}}
\newcommand{\gcap}[0]{\hat{\Bg}}
\newcommand{\hcap}[0]{\hat{\Bh}}
\newcommand{\icap}[0]{\hat{\Bi}}
\newcommand{\jcap}[0]{\hat{\Bj}}
\newcommand{\kcap}[0]{\hat{\Bk}}
\newcommand{\lcap}[0]{\hat{\Bl}}
\newcommand{\mcap}[0]{\hat{\Bm}}
\newcommand{\ncap}[0]{\hat{\Bn}}
\newcommand{\ocap}[0]{\hat{\Bo}}
\newcommand{\pcap}[0]{\hat{\Bp}}
\newcommand{\qcap}[0]{\hat{\Bq}}
\newcommand{\rcap}[0]{\hat{\Br}}
\newcommand{\scap}[0]{\hat{\Bs}}
\newcommand{\tcap}[0]{\hat{\Bt}}
\newcommand{\ucap}[0]{\hat{\Bu}}
\newcommand{\vcap}[0]{\hat{\Bv}}
\newcommand{\wcap}[0]{\hat{\Bw}}
\newcommand{\xcap}[0]{\hat{\Bx}}
\newcommand{\ycap}[0]{\hat{\By}}
\newcommand{\zcap}[0]{\hat{\Bz}}
\newcommand{\thetacap}[0]{\hat{\Btheta}}

%
% to write R^n and C^n in a distinguishable fashion.  Perhaps change this
% to the double lined characters upon figuring out how to do so.
%
\newcommand{\C}[1]{$\mathbb{C}^{#1}$}
\newcommand{\R}[1]{$\mathbb{R}^{#1}$}

%
% various generally useful helpers
%

% derivative of #1 wrt. #2:
\newcommand{\D}[2] {\frac {d#2} {d#1}}

\newcommand{\inv}[1]{\frac{1}{#1}}
\newcommand{\cross}[0]{\times}

\newcommand{\abs}[1]{\lvert{#1}\rvert}
\newcommand{\norm}[1]{\lVert{#1}\rVert}
\newcommand{\innerprod}[2]{\langle{#1}, {#2}\rangle}
\newcommand{\dotprod}[2]{{#1} \cdot {#2}}
\newcommand{\bdotprod}[2]{\left({#1} \cdot {#2}\right)}
\newcommand{\crossprod}[2]{{#1} \cross {#2}}
\newcommand{\tripleprod}[3]{\dotprod{\left(\crossprod{#1}{#2}\right)}{#3}}

\DeclareMathOperator{\Proj}{Proj}
\DeclareMathOperator{\Span}{span}
\DeclareMathOperator{\Sgn}{sgn}
\DeclareMathOperator{\Area}{Area}
\DeclareMathOperator{\Volume}{Volume}

%
% A few miscellaneous things specific to this document
%
\newcommand{\crossop}[1]{\crossprod{#1}{}}

% R2 vector.
\newcommand{\VectorTwo}[2]{
\begin{bmatrix}
 {#1} \\
 {#2}
\end{bmatrix}
}

\newcommand{\VectorN}[1]{
\begin{bmatrix}
{#1}_1 \\
{#1}_2 \\
\vdots \\
{#1}_N \\
\end{bmatrix}
}

\newcommand{\DETuvij}[4]{
\begin{vmatrix}
 {#1}_{#3} & {#1}_{#4} \\
 {#2}_{#3} & {#2}_{#4}
\end{vmatrix}
}

\newcommand{\DETuvwijk}[6]{
\begin{vmatrix}
 {#1}_{#4} & {#1}_{#5} & {#1}_{#6} \\
 {#2}_{#4} & {#2}_{#5} & {#2}_{#6} \\
 {#3}_{#4} & {#3}_{#5} & {#3}_{#6}
\end{vmatrix}
}

\newcommand{\DETuvwxijkl}[8]{
\begin{vmatrix}
 {#1}_{#5} & {#1}_{#6} & {#1}_{#7} & {#1}_{#8} \\
 {#2}_{#5} & {#2}_{#6} & {#2}_{#7} & {#2}_{#8} \\
 {#3}_{#5} & {#3}_{#6} & {#3}_{#7} & {#3}_{#8} \\
 {#4}_{#5} & {#4}_{#6} & {#4}_{#7} & {#4}_{#8} \\
\end{vmatrix}
}

%\newcommand{\DETuvwxyijklm}[10]{
%\begin{vmatrix}
% {#1}_{#6} & {#1}_{#7} & {#1}_{#8} & {#1}_{#9} & {#1}_{#10} \\
% {#2}_{#6} & {#2}_{#7} & {#2}_{#8} & {#2}_{#9} & {#2}_{#10} \\
% {#3}_{#6} & {#3}_{#7} & {#3}_{#8} & {#3}_{#9} & {#3}_{#10} \\
% {#4}_{#6} & {#4}_{#7} & {#4}_{#8} & {#4}_{#9} & {#4}_{#10} \\
% {#5}_{#6} & {#5}_{#7} & {#5}_{#8} & {#5}_{#9} & {#5}_{#10}
%\end{vmatrix}
%}

% R3 vector.
\newcommand{\VectorThree}[3]{
\begin{bmatrix}
 {#1} \\
 {#2} \\
 {#3}
\end{bmatrix}
}



\newcommand{\laplacian}[0]{\nabla^2}
\newcommand{\Dsq}[2] {\frac {\partial^2 {#1}} {\partial {#2}^2}}
\newcommand{\dxj}[2] {\frac {\partial {#1}} {\partial {x_{#2}}}}
\newcommand{\dsqxj}[2] {\frac {\partial^2 {#1}} {\partial {x_{#2}}^2}}
\DeclareMathOperator{\Exp}{e}
\DeclareMathOperator{\Rej}{Rej}
\newcommand{\gpgrade}[2] {{\left\langle{{#1}}\right\rangle}_{#2}}
\newcommand{\gpgradezero}[1] {\gpgrade{#1}{0}}
\newcommand{\gpgradetwo}[1] {\gpgrade{#1}{2}}
\newcommand{\gpgradefour}[1] {\gpgrade{#1}{4}}

%
% The real thing:
%

                             % The preamble begins here.
\title{Geometry of intersecting bivectors}
\author{Peeter Joot}         % Declares the author's name.
%\date{}        % Deleting this command produces today's date.

\begin{document}             % End of preamble and beginning of text.

\maketitle{}

\section{ The problem. }

Examination of exponential solutions for Laplace's equation leads one to
a requirement to examine the product of intersecting bivectors such as

\[
\left(\abs{\Bx \wedge \Bk}^2\right)' = -\left(
(\Bx' \wedge \Bv)(\Bx \wedge \Bv)
+(\Bx \wedge \Bv)(\Bx' \wedge \Bv)
\right)
\]

Here we see that the symmetric sum of bivectors $\Bx \wedge \Bk$ and $\Bx' \wedge \Bk$ is a scalar quantity.  This we will identify later as a quantity
related to the bivector dot product.

It is worthwhile to systematically examine the
general products of intersecting bivectors, that is planes that share a common line, in this case the line directed along the vector $\Bk$.
It is also notable that since all non coplanar bivectors in \R{3} intersect
this
examination will cover the important special case of three dimensional
plane geometry.

A result of this examination is that many of the concepts familiar from
vector geometry such as
orthagonality, projection, and rejection will have direct bivector
equivalents.

General bivector geometry, in spaces where non-coplanar bivectors do not 
neccessarily intersect (such as in \R{4}), will need to be treated separately,
but some of the grade 4 product terms will be carried below to explicitly
hightlight the point where the intersecting bivector space requirement
effects the results.

\section{The meat.}

The geometric product of two bivectors can be written:

\begin{equation}\label{eqn:ABprod}
\BA \BB = 
\gpgrade{\BA \BB}{0}
+\gpgrade{\BA \BB}{2}
+\gpgrade{\BA \BB}{4}
= 
{\BA \cdot \BB}
+\gpgrade{\BA \BB}{2}
+{\BA \wedge \BB}
\end{equation}
\begin{equation}\label{eqn:BAprod}
\BB \BA = 
\gpgrade{\BB \BA}{0}
+\gpgrade{\BB \BA}{2}
+\gpgrade{\BB \BA}{4}
= 
{\BB \cdot \BA}
+\gpgrade{\BB \BA}{2}
+{\BB \wedge \BA}
\end{equation}

Because we have three terms involved, unlike the vector dot and wedge product
we cannot generally separate these terms by 
symmetric and antisymmetric parts.  However forming those sums
will still worthwhile, especially for the case of interecting bivectors
since the last term will be zero in that case.

\subsection{ Sign change of each grade term with commutation. }

Starting with the last term we can first observe that

\begin{equation}\label{eqn:wedgesign}
\BA \wedge \BB = \BB \wedge \BA
\end{equation}

To show this let $\BA = \Ba \wedge \Bb$, and $\BB = \Bc \wedge \Bd$.  When

$\BA \wedge \BB \ne 0$, one can write:

\begin{align*}
\BA \wedge \BB 
&= \Ba \wedge \Bb \wedge \Bc \wedge \Bd \\
&= - \Bb \wedge \Bc \wedge \Bd \wedge \Ba \\
&= \Bc \wedge \Bd \wedge \Ba \wedge \Bb \\
&= \BB \wedge \BA \\
\end{align*}

To see how the signs of the remaining two terms vary with commutation
form:

\begin{align*}
(\BA + \BB)^2
&= (\BA + \BB)(\BA + \BB) \\
&= \BA^2 + \BB^2 + \BA \BB + \BB \BA \\
\end{align*}

When $\BA$ and $\BB$ interect we can write
$\BA = \Ba \wedge \Bx$, and $\BB = \Bb \wedge \Bx$, thus the sum is a bivector

\[
(\BA + \BB)
= (\Ba + \Bb) \wedge \Bx
\]

And so, the square of the two is a scalar.  When $\BA$ and $\BB$ have only
non intersecting components, such as the grade two \R{4} multivector
$\Be_{12} + \Be_{34}$, the square of this sum will have both grade four and
scalar parts.

Since the LHS = RHS, and the grades of the two also must be the same.
This implies that the quantity

\[
\BA \BB + \BB \BA = 
\BA \cdot \BB + \BB \cdot \BA
+\gpgradetwo{\BA \BB} + \gpgradetwo{\BB \BA}
+\BA \wedge \BB + \BB \wedge \BA
\]

is a scalar $\iff$ 
$\BA + \BB$ is a bivector, and in general has scalar and grade four terms.
Because this symmetric sum has no grade two terms, 
regardless of whether $\BA$, and $\BB$ intersect, we have:

\[
\gpgradetwo{\BA \BB} + \gpgradetwo{\BB \BA} = 0
\]
\begin{equation}\label{eqn:signgradetwo}
\implies
\gpgradetwo{\BA \BB} = -\gpgradetwo{\BB \BA}
\end{equation}

One would intuitively expect $\BA \cdot \BB = \BB \cdot \BA$.  This can be
demonstrated by forming the complete symmetric sum

\begin{align*}
\BA \BB + \BB \BA 
&= 
{\BA \cdot \BB} +{\BB \cdot \BA}
+\gpgrade{\BA \BB}{2} +\gpgrade{\BB \BA}{2}
+{\BA \wedge \BB} + {\BB \wedge \BA} \\
&= 
{\BA \cdot \BB} +{\BB \cdot \BA}
+\gpgrade{\BA \BB}{2} -\gpgrade{\BA \BB}{2}
+{\BA \wedge \BB} + {\BA \wedge \BB} \\
&= 
{\BA \cdot \BB} +{\BB \cdot \BA}
+2{\BA \wedge \BB} \\
\end{align*}

The LHS commutes with interchange of $\BA$ and $\BB$, as does
${\BA \wedge \BB}$.  So for the RHS to also commute, the remaining grade 0 term
must also:

\begin{equation}\label{eqn:dotsign}
\BA \cdot \BB = \BB \cdot \BA
\end{equation}

\subsection{ Dot, wedge and grade two terms of bivector product. }

Collecting the results of the previous section and substituiting back
into equation \ref{eqn:ABprod} we have:

\begin{equation}\label{eqn:AdotB}
\BA \cdot \BB = \gpgrade{\frac{\BA \BB + \BB\BA}{2}}{0}
\end{equation}

\begin{equation}\label{eqn:AtwoB}
\gpgradetwo{\BA \BB} = \frac{\BA \BB - \BB\BA}{2}
\end{equation}

\begin{equation}\label{eqn:AwedgeB}
\BA \wedge \BB = \gpgrade{\frac{\BA \BB + \BB\BA}{2}}{4}
\end{equation}

When these intersect in a line the wedge term is zero, so for that special case we can write:

\begin{equation*}
\BA \cdot \BB = \frac{\BA \BB + \BB\BA}{2}
\end{equation*}

\begin{equation*}
\gpgradetwo{\BA \BB} = \frac{\BA \BB - \BB\BA}{2}
\end{equation*}

\begin{equation*}
\BA \wedge \BB = 0
\end{equation*}

(note that this is always the case for \R{3}).

\section{ Intersection of planes. }

Starting with two planes specified parametrically, each in terms of two direction vectors and a point on the plane:

\begin{align}\label{eqn:twoplanes}
\Bx &= \Bp + \alpha \Bu + \beta \Bv \\
\By &= \Bq + a \Bw + b \Bz \\
\end{align}

If these intersect then all points on the line must satisify $\Bx = \By$, so the
solution requires:

\[
\Bp + \alpha \Bu + \beta \Bv = \Bq + a \Bw + b \Bz
\]
\[
\implies
(\Bp + \alpha \Bu + \beta \Bv) \wedge \Bw \wedge \Bz = (\Bq + a \Bw + b \Bz) \wedge \Bw \wedge \Bz = \Bq \wedge \Bw \wedge \Bz
\]

Rearranging for $\beta$, and writing $\BB = \Bw \wedge \Bz$:

\[
\beta = \frac{\Bq \wedge \BB - (\Bp + \alpha \Bu) \wedge \BB}{\Bv \wedge \BB}
\]

Note that when the solution exists the left vs right order of the division by $\Bv \wedge \BB$ should not matter since the numerator will be proportional to this bivector (or else the $\beta$ would not be a scalar).

Substitution of $\beta$ back into $\Bx = \Bp + \alpha \Bu + \beta \Bv$ (all points in the first plane) gives you a parametric equation for a line:

\[
\Bx = \Bp + \frac{(\Bq-\Bp)\wedge \BB}{\Bv \wedge \BB}\Bv + \alpha\frac{1}{\Bv \wedge \BB}((\Bv \wedge \BB) \Bu - (\Bu \wedge \BB)\Bv)
\]

Where a point on the line is:

\[
\Bp + \frac{(\Bq-\Bp)\wedge \BB}{\Bv \wedge \BB}\Bv 
%= \frac{1}{\Bv \wedge \BB}((\Bv \wedge \BB)\Bp + ((\Bq-\Bp)\wedge \BB)\Bv)
\]

And a direction vector for the line is:

\[
\frac{1}{\Bv \wedge \BB}((\Bv \wedge \BB) \Bu - (\Bu \wedge \BB)\Bv)
\]
\[
\propto
(\Bv \wedge \BB)^2 \Bu - (\Bv \wedge \BB)(\Bu \wedge \BB)\Bv
\]

Now, this result is only valid if $\Bv \wedge \BB \ne 0$ (ie: line of intersection is not directed along $\Bv$), but if that is the case the second form will be zero.  Thus we can add the results (or any non-zero linear combination of) allowing for either of $\Bu$, or $\Bv$ to be directed along the line of intersection:

\begin{equation}\label{eqn:dirvecintersection}
a\left( (\Bv \wedge \BB)^2 \Bu
- (\Bv \wedge \BB)(\Bu \wedge \BB)\Bv \right)
+ b\left((\Bu \wedge \BB)^2 \Bv 
- (\Bu \wedge \BB)(\Bv \wedge \BB)\Bu\right)
\end{equation}

Alternately, one could formulate this in terms of $\BA = \Bu \wedge \Bv$, $\Bw$, and $\Bz$.  Is there a more symetrical form for this direction vector?

\subsection{ Vector along line of intersection in \R{3}}

For \R{3} one can solve the intersection problem using the normals to the planes.  For simplicity put the origin on the line of intersection (and all planes through a common point in \R{3} have at least a line of intersection).  In this case, for bivectors $\BA$ and $\BB$, normals to those planes are $i\BA$, and $i\BB$ respectively.  The plane through both of those normals is:

\begin{align*}
(i\BA) \wedge (i\BB)
= \frac{(i\BA)(i\BB) - (i\BB)(i\BA)}{2} 
= \frac{\BB\BA - \BA\BB}{2} 
= \gpgradetwo{\BB\BA}
\end{align*}

The normal to this plane

\begin{equation}\label{eqn:r3planeintersect}
i\gpgradetwo{\BB\BA}
\end{equation}

is directed along the line of interesection.  This result is more appealing than
the general \R{N} result of equation \ref{eqn:dirvecintersection}, not
just because it is simpler, but also because it is a function of only the
bivectors for the planes, without a requirement to find or calculate
two specific independent direction vectors in one of the planes.

\subsection{ Applying this result to \R{N} }

If you reject the component of $\BA$ from $\BB$ for two intersecting bivectors:

\[
\Rej_{\BA}(\BB) = \frac{1}{\BA}\gpgradetwo{\BA\BB}
\]

the line of intersection remains the same ... that operation rotates $\BB$ so that the two are mutually perpendicular.  This essentially reduces the problem to that of the three dimensional case, so the solution has to be of the same form... you just need to calculate a ``pseudoscalar'' (what you are calling the join), for the subspace spanned by the two bivectors.

That can be computed by taking any direction vector that is on one plane, but isn't in the second.  For example, pick a vector $\Bu$ in the plane $\BA$ that is not on the intersection of $\BA$ and $\BB$.  In mathese that is $\Bu = \inv{\BA}(\BA\cdot \Bu)$ (or $\Bu \wedge \BA = 0$), where $\Bu \wedge \BB \ne 0$.  Thus a pseudoscalar for this subspace is:

\[
\Bi = \frac{\Bu \wedge \BB}{\abs{\Bu \wedge \BB}}
\]

To calculate the direction vector along the intersection we don't care about the scaling above.  Also note that provided $\Bu$ has a component in the plane $\BA$, $\Bu \cdot \BA$ is also in the plane (it's rotated $\pi/2$ from $\inv{\BA}(\BA \cdot \Bu)$.

Thus, provided that $\Bu \cdot \BA$ isn't on the intersection, a scaled ``pseudoscalar''
for the subspace can be calculated by taking from any vector $\Bu$ with a component in the plane $\BA$:

\[
\Bi \propto (\Bu \cdot \BA) \wedge \BB
\]

Thus a vector along the intersection is:

\begin{equation}\label{eqn:pseudoscalarinter}
\Bd = ((\Bu \cdot \BA) \wedge \BB) \gpgradetwo{\BA\BB}
\end{equation}

(an interchange of $\BA$ and $\BB$ above would also work).

Without showing the steps one can write the complete parametric solution of the line through the planes of equation \ref{eqn:twoplanes} in terms of this direction vector:

\begin{equation}\label{eqn:finalsolnofRNplaneintersection}
\Bx = \Bp + \left(\frac{(\Bq - \Bp)\wedge \BB}{(\Bd \cdot \BA) \wedge \BB}\right) (\Bd \cdot \BA) + \alpha \Bd
\end{equation}

Since $(\Bd \cdot \BA) \ne 0$ and $(\Bd \cdot \BA) \wedge \BB \ne 0$ (unless $\BA$ and $\BB$ are coplanar), observe that this is a natural generator
of the pseudoscalar for the subspace, and as such shows up in the expression
above.

\section{ Grade components of a trivector product. }

While trying to put equation \ref{eqn:dirvecintersection} into a form
that eliminated $\Bu$, and $\Bv$ in favour of $\BA = \Bu \wedge \Bv$
symmetric and antisymmetric formulations for the various grade terms
of a trivector product looked like they could be handy.  Here's a summary
of those results.

\subsection{ Grade 6 term. }

Writing two trivectors in terms
of mutually orthogonal components

\[
\BA = \Bx \wedge \By \wedge \Bz = \Bx\By\Bz
\]

and

\[
\BB = \Bu \wedge \Bv \wedge \Bw =\Bu\Bv\Bw
\]

Assuming that there is no common vector between the two, the 
wedge of these is

\begin{align*}
\BA \wedge \BB 
&= \gpgrade{\BA\BB}{6} \\
&= \gpgrade{\Bx\By\Bz\Bu\Bv\Bw}{6} \\
&= \gpgrade{\By\Bz(\Bx\Bu)\Bv\Bw}{6} \\
&= \gpgrade{\By\Bz(-\Bu\Bx + 2\Bu \cdot \Bx)\Bv\Bw}{6} \\
&= -\gpgrade{\By\Bz\Bu(\Bx\Bv)\Bw}{6} \\
&= -\gpgrade{\By\Bz\Bu(-\Bv\Bx + 2\Bv \cdot \Bx)\Bw}{6} \\
&= \gpgrade{\By\Bz\Bu\Bv(\Bx\Bw)}{6} \\
&= \cdots \\
&= -\gpgrade{\Bu\Bv\Bw\Bx\By\Bz}{6} \\
&= -\gpgrade{\BB\BA}{6} \\
&= -\BB \wedge \BA
\end{align*}

Note above that any interchange of terms inverts the sign (demonstrated 
explicitly for all the $\Bx$ interchanges).

As an aside, this
sign change on interchange is taken as the defining property of the 
wedge product in differential forms.  That property also
implies also that the wedge product is
zero when a vector is wedged with itself since zero is the only
value that is the negation of itself.  Thus we see explicitly
how the notation of using the wedge for the highest grade term
of two blades is consistent with the traditional
wedge product definition.

The end result here is that the grade 6 term of a trivector trivector product
changes sign on interchange of the trivectors:

\begin{equation}\label{eqn:trivecgpgrade6}
\gpgrade{\BA\BB}{6} = -\gpgrade{\BB\BA}{6}
\end{equation}

\subsection{ Grade 4 term. }

For a trivector product to have a grade 4 term there must be a common
vector between the two

\[
\BA = \Bx \wedge \By \wedge \Bz = \Bx\By\Bz
\]

and

\[
\BB = \Bu \wedge \Bv \wedge \Bz =\Bu\Bv\Bz
\]

The grade four term of the product is

\begin{align*}
\gpgrade{\BB \BA}{4}
&= \gpgrade{ \Bu\Bv\Bz \Bx\By\Bz }{4} \\
&= \gpgrade{ \Bu\Bv\Bz \Bz\Bx\By }{4} \\
&= \Bz^2\gpgrade{ \Bu\Bv\Bx\By }{4} \\
&= \Bz^2\gpgrade{ \Bu(\Bv\Bx)\By }{4} \\
&= \Bz^2\gpgrade{ \Bu(-\Bx\Bv + 2 \Bx \cdot \Bv)\By }{4} \\
&= -\Bz^2\gpgrade{ \Bu\Bx\Bv\By }{4} \\
&= \cdots \\
&= \Bz^2\gpgrade{ \Bx\By\Bu\Bv }{4} \\
&= \gpgrade{ \Bx\By\Bz\Bz\Bu\Bv }{4} \\
&= \gpgrade{ \Bx\By\Bz\Bu\Bv\Bz }{4} \\
&= \gpgrade{ \Bx\By\Bz\Bu\Bv\Bz }{4} \\
&= \gpgrade{\BA \BB}{4}
\end{align*}

Thus the grade 4 term commutes on interchange:

\begin{equation}\label{eqn:trivecgpgrade4}
\gpgrade{\BA\BB}{4} = \gpgrade{\BB\BA}{4}
\end{equation}

\subsection{ Grade 2 term. }

Similar to above, 
for a trivector product to have a grade 2 term there must be two common
vectors between the two

\[
\BA = \Bx \wedge \By \wedge \Bz = \Bx\By\Bz
\]

and

\[
\BB = \Bu \wedge \By \wedge \Bz =\Bu\By\Bz
\]

The grade two term of the product is

\begin{align*}
\gpgrade{\BA \BB}{2}
&= \gpgrade{ \Bx\By\Bz \Bu\By\Bz }{2} \\
&= \gpgrade{ \Bx\By\Bz \By\Bz \Bu}{2} \\
&= (\By\Bz)^2\gpgrade{ \Bx \Bu}{2} \\
&= -(\By\Bz)^2\gpgrade{ \Bu \Bx}{2} \\
&= -\gpgrade{ \BB \BA }{2} \\
\end{align*}

The grade 2 term anticommutes on interchange:

\begin{equation}\label{eqn:trivecgpgrade2}
\gpgrade{\BA\BB}{2} = -\gpgrade{\BB\BA}{2}
\end{equation}

\subsection{ Grade 0 term. }

Any grade 0 terms are due to products of the form $\BA = k\BB$

\begin{align*}
\gpgrade{\BA \BB}{0}
&= \gpgrade{k\BB \BB}{0} \\
&= \gpgrade{\BB k\BB}{0} \\
&= \gpgrade{\BB \BA}{0} \\
\end{align*}

The grade 2 term commutes on interchange:

\begin{equation}\label{eqn:trivecgpgrade0}
\gpgrade{\BA\BB}{0} = \gpgrade{\BB\BA}{0}
\end{equation}

\subsection{ combining results. }

\begin{equation*}
\BA \BB
=\gpgrade{\BA\BB}{0}
+\gpgrade{\BA\BB}{2}
+\gpgrade{\BA\BB}{4}
+\gpgrade{\BA\BB}{6}
\end{equation*}

\begin{align*}
\BB\BA
&=\gpgrade{\BB\BA}{0}
+\gpgrade{\BB\BA}{2}
+\gpgrade{\BB\BA}{4}
+\gpgrade{\BB\BA}{6} \\
&=\gpgrade{\BA\BB}{0}
-\gpgrade{\BA\BB}{2}
+\gpgrade{\BA\BB}{4}
-\gpgrade{\BA\BB}{6} \\
\end{align*}

These can be combined to express each of the grade terms as subsets
of the symmetric and antisymmetric parts:

\begin{align*}
\BA \cdot \BB = \gpgrade{\BA\BB}{0} &= \gpgrade{\frac{\BA\BB + \BB\BA}{2}}{0} \\
\gpgrade{\BA\BB}{2} &= \gpgrade{\frac{\BA\BB - \BB\BA}{2}}{2} \\
\gpgrade{\BA\BB}{4} &= \gpgrade{\frac{\BA\BB + \BB\BA}{2}}{4} \\
\BA \wedge \BB = \gpgrade{\BA\BB}{6} &= \gpgrade{\frac{\BA\BB - \BB\BA}{2}}{6} \\
\end{align*}

Note that above I've been somewhat loose with the argument above.  A grade three vector
will have the following form:

\[
\sum_{i<j<k} D_{ijk} \Be_{ijk}
\]

Where $D_{ijk}$ is the determinant of $ijk$ components of the vectors being wedged.  Thus the product
of two trivectors will be of the following form:

\[
\sum_{i<j<k} \sum_{i'<j'<k'} D_{ijk} D'_{i'j'k'} (\Be_{ijk} \Be_{i'j'k'})
\]

It's really each of these $\Be_{ijk} \Be_{i'j'k'}$ products that have to be considered in the grade 
and sign arguments above.  The end result will be the same though... one would just have to present
it a bit more carefully for a true proof.

\subsection{ Intersecting trivector cases. }

As with the intersecting bivector case, when there is a line of intersection between the two volumes one can
write:

\begin{align*}
\BA \cdot \BB = \gpgrade{\BA\BB}{0} &= \gpgrade{\frac{\BA\BB + \BB\BA}{2}}{0} \\
\gpgrade{\BA\BB}{2} &= \frac{\BA\BB - \BB\BA}{2} \\
\gpgrade{\BA\BB}{4} &= \gpgrade{\frac{\BA\BB + \BB\BA}{2}}{4} \\
\BA \wedge \BB = \gpgrade{\BA\BB}{6} &= 0 \\
\end{align*}

And if these volumes intersect in a plane a further simplification is possible:
\begin{align*}
\BA \cdot \BB = \gpgrade{\BA\BB}{0} &= \frac{\BA\BB + \BB\BA}{2} \\
\gpgrade{\BA\BB}{2} &= \frac{\BA\BB - \BB\BA}{2} \\
\gpgrade{\BA\BB}{4} &= 0 \\
\BA \wedge \BB = \gpgrade{\BA\BB}{6} &= 0 \\
\end{align*}

\end{document}               % End of document.
