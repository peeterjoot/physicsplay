\documentclass{article}      % Specifies the document class

\usepackage{amsmath}
\usepackage{mathpazo}

%
% shorthand for bold symbols, convenient for vectors and matrices
%
\newcommand{\Ba}[0]{\mathbf{a}}
\newcommand{\Bb}[0]{\mathbf{b}}
\newcommand{\Bc}[0]{\mathbf{c}}
\newcommand{\Bd}[0]{\mathbf{d}}
\newcommand{\Be}[0]{\mathbf{e}}
\newcommand{\Bf}[0]{\mathbf{f}}
\newcommand{\Bg}[0]{\mathbf{g}}
\newcommand{\Bh}[0]{\mathbf{h}}
\newcommand{\Bi}[0]{\mathbf{i}}
\newcommand{\Bj}[0]{\mathbf{j}}
\newcommand{\Bk}[0]{\mathbf{k}}
\newcommand{\Bl}[0]{\mathbf{l}}
\newcommand{\Bm}[0]{\mathbf{m}}
\newcommand{\Bn}[0]{\mathbf{n}}
\newcommand{\Bo}[0]{\mathbf{o}}
\newcommand{\Bp}[0]{\mathbf{p}}
\newcommand{\Bq}[0]{\mathbf{q}}
\newcommand{\Br}[0]{\mathbf{r}}
\newcommand{\Bs}[0]{\mathbf{s}}
\newcommand{\Bt}[0]{\mathbf{t}}
\newcommand{\Bu}[0]{\mathbf{u}}
\newcommand{\Bv}[0]{\mathbf{v}}
\newcommand{\Bw}[0]{\mathbf{w}}
\newcommand{\Bx}[0]{\mathbf{x}}
\newcommand{\By}[0]{\mathbf{y}}
\newcommand{\Bz}[0]{\mathbf{z}}
\newcommand{\BA}[0]{\mathbf{A}}
\newcommand{\BB}[0]{\mathbf{B}}
\newcommand{\BC}[0]{\mathbf{C}}
\newcommand{\BD}[0]{\mathbf{D}}
\newcommand{\BE}[0]{\mathbf{E}}
\newcommand{\BF}[0]{\mathbf{F}}
\newcommand{\BG}[0]{\mathbf{G}}
\newcommand{\BH}[0]{\mathbf{H}}
\newcommand{\BI}[0]{\mathbf{I}}
\newcommand{\BJ}[0]{\mathbf{J}}
\newcommand{\BK}[0]{\mathbf{K}}
\newcommand{\BL}[0]{\mathbf{L}}
\newcommand{\BM}[0]{\mathbf{M}}
\newcommand{\BN}[0]{\mathbf{N}}
\newcommand{\BO}[0]{\mathbf{O}}
\newcommand{\BP}[0]{\mathbf{P}}
\newcommand{\BQ}[0]{\mathbf{Q}}
\newcommand{\BR}[0]{\mathbf{R}}
\newcommand{\BS}[0]{\mathbf{S}}
\newcommand{\BT}[0]{\mathbf{T}}
\newcommand{\BU}[0]{\mathbf{U}}
\newcommand{\BV}[0]{\mathbf{V}}
\newcommand{\BW}[0]{\mathbf{W}}
\newcommand{\BX}[0]{\mathbf{X}}
\newcommand{\BY}[0]{\mathbf{Y}}
\newcommand{\BZ}[0]{\mathbf{Z}}

\newcommand{\Bzero}[0]{\mathbf{0}}
\newcommand{\Btheta}[0]{\boldsymbol{\theta}}
\newcommand{\Btau}[0]{\boldsymbol{\tau}}
\newcommand{\Bomega}[0]{\boldsymbol{\omega}}

%
% shorthand for unit vectors
%
\newcommand{\acap}[0]{\hat{\Ba}}
\newcommand{\bcap}[0]{\hat{\Bb}}
\newcommand{\ccap}[0]{\hat{\Bc}}
\newcommand{\dcap}[0]{\hat{\Bd}}
\newcommand{\ecap}[0]{\hat{\Be}}
\newcommand{\fcap}[0]{\hat{\Bf}}
\newcommand{\gcap}[0]{\hat{\Bg}}
\newcommand{\hcap}[0]{\hat{\Bh}}
\newcommand{\icap}[0]{\hat{\Bi}}
\newcommand{\jcap}[0]{\hat{\Bj}}
\newcommand{\kcap}[0]{\hat{\Bk}}
\newcommand{\lcap}[0]{\hat{\Bl}}
\newcommand{\mcap}[0]{\hat{\Bm}}
\newcommand{\ncap}[0]{\hat{\Bn}}
\newcommand{\ocap}[0]{\hat{\Bo}}
\newcommand{\pcap}[0]{\hat{\Bp}}
\newcommand{\qcap}[0]{\hat{\Bq}}
\newcommand{\rcap}[0]{\hat{\Br}}
\newcommand{\scap}[0]{\hat{\Bs}}
\newcommand{\tcap}[0]{\hat{\Bt}}
\newcommand{\ucap}[0]{\hat{\Bu}}
\newcommand{\vcap}[0]{\hat{\Bv}}
\newcommand{\wcap}[0]{\hat{\Bw}}
\newcommand{\xcap}[0]{\hat{\Bx}}
\newcommand{\ycap}[0]{\hat{\By}}
\newcommand{\zcap}[0]{\hat{\Bz}}
\newcommand{\thetacap}[0]{\hat{\Btheta}}

%
% to write R^n and C^n in a distinguishable fashion.  Perhaps change this
% to the double lined characters upon figuring out how to do so.
%
\newcommand{\C}[1]{$\mathbb{C}^{#1}$}
\newcommand{\R}[1]{$\mathbb{R}^{#1}$}

%
% various generally useful helpers
%

% derivative of #1 wrt. #2:
\newcommand{\D}[2] {\frac {d#2} {d#1}}

\newcommand{\inv}[1]{\frac{1}{#1}}
\newcommand{\cross}[0]{\times}

\newcommand{\abs}[1]{\lvert{#1}\rvert}
\newcommand{\norm}[1]{\lVert{#1}\rVert}
\newcommand{\innerprod}[2]{\langle{#1}, {#2}\rangle}
\newcommand{\dotprod}[2]{{#1} \cdot {#2}}
\newcommand{\bdotprod}[2]{\left({#1} \cdot {#2}\right)}
\newcommand{\crossprod}[2]{{#1} \cross {#2}}
\newcommand{\tripleprod}[3]{\dotprod{\left(\crossprod{#1}{#2}\right)}{#3}}

\DeclareMathOperator{\Proj}{Proj}
\DeclareMathOperator{\Span}{span}
\DeclareMathOperator{\Sgn}{sgn}
\DeclareMathOperator{\Area}{Area}
\DeclareMathOperator{\Volume}{Volume}

%
% A few miscellaneous things specific to this document
%
\newcommand{\crossop}[1]{\crossprod{#1}{}}

% R2 vector.
\newcommand{\VectorTwo}[2]{
\begin{bmatrix}
 {#1} \\
 {#2}
\end{bmatrix}
}

\newcommand{\VectorN}[1]{
\begin{bmatrix}
{#1}_1 \\
{#1}_2 \\
\vdots \\
{#1}_N \\
\end{bmatrix}
}

\newcommand{\DETuvij}[4]{
\begin{vmatrix}
 {#1}_{#3} & {#1}_{#4} \\
 {#2}_{#3} & {#2}_{#4}
\end{vmatrix}
}

\newcommand{\DETuvwijk}[6]{
\begin{vmatrix}
 {#1}_{#4} & {#1}_{#5} & {#1}_{#6} \\
 {#2}_{#4} & {#2}_{#5} & {#2}_{#6} \\
 {#3}_{#4} & {#3}_{#5} & {#3}_{#6}
\end{vmatrix}
}

\newcommand{\DETuvwxijkl}[8]{
\begin{vmatrix}
 {#1}_{#5} & {#1}_{#6} & {#1}_{#7} & {#1}_{#8} \\
 {#2}_{#5} & {#2}_{#6} & {#2}_{#7} & {#2}_{#8} \\
 {#3}_{#5} & {#3}_{#6} & {#3}_{#7} & {#3}_{#8} \\
 {#4}_{#5} & {#4}_{#6} & {#4}_{#7} & {#4}_{#8} \\
\end{vmatrix}
}

%\newcommand{\DETuvwxyijklm}[10]{
%\begin{vmatrix}
% {#1}_{#6} & {#1}_{#7} & {#1}_{#8} & {#1}_{#9} & {#1}_{#10} \\
% {#2}_{#6} & {#2}_{#7} & {#2}_{#8} & {#2}_{#9} & {#2}_{#10} \\
% {#3}_{#6} & {#3}_{#7} & {#3}_{#8} & {#3}_{#9} & {#3}_{#10} \\
% {#4}_{#6} & {#4}_{#7} & {#4}_{#8} & {#4}_{#9} & {#4}_{#10} \\
% {#5}_{#6} & {#5}_{#7} & {#5}_{#8} & {#5}_{#9} & {#5}_{#10}
%\end{vmatrix}
%}

% R3 vector.
\newcommand{\VectorThree}[3]{
\begin{bmatrix}
 {#1} \\
 {#2} \\
 {#3}
\end{bmatrix}
}



\newcommand{\laplacian}[0]{\nabla^2}
\newcommand{\Dsq}[2] {\frac {\partial^2 {#1}} {\partial {#2}^2}}
\newcommand{\dxj}[2] {\frac {\partial {#1}} {\partial {x_{#2}}}}
\newcommand{\dsqxj}[2] {\frac {\partial^2 {#1}} {\partial {x_{#2}}^2}}
\DeclareMathOperator{\Exp}{e}
\newcommand{\gpgrade}[2] {{\left\langle{{#1}}\right\rangle}_{#2}}
\newcommand{\gpgradezero}[1] {\gpgrade{#1}{0}}
\newcommand{\gpgradetwo}[1] {\gpgrade{#1}{2}}
\newcommand{\gpgradefour}[1] {\gpgrade{#1}{4}}

%
% The real thing:
%

                             % The preamble begins here.
\title{Geometry of intersecting bivectors}
\author{Peeter Joot}         % Declares the author's name.
%\date{}        % Deleting this command produces today's date.

\begin{document}             % End of preamble and beginning of text.

\maketitle{}

\section{ The problem. }

Examination of exponential solutions for Laplace's equation leads one to
a requirement to examine the product of intersecting bivectors such as

\[
\left(\abs{\Bx \wedge \Bk}^2\right)' = -\left(
(\Bx' \wedge \Bv)(\Bx \wedge \Bv)
+(\Bx \wedge \Bv)(\Bx' \wedge \Bv)
\right)
\]

Here we see that the symmetric sum of bivectors $\Bx \wedge \Bk$ and $\Bx' \wedge \Bk$ is a scalar quantity.  This we will identify later as a quantity
related to the bivector dot product.

It is worthwhile to systematically examine the
general products of intersecting bivectors, that is planes that share a common line, in this case the line directed along the vector $\Bk$.
It is also notable that since all non coplanar bivectors in \R{3} intersect
this
examination will cover the important special case of three dimensional
plane geometry.

A result of this examination is that many of the concepts familiar from
vector geometry such as
orthagonality, projection, and rejection will have direct bivector
equivalents.

General bivector geometry, in spaces where non-coplanar bivectors do not 
neccessarily intersect (such as in \R{4}), will need to be treated separately,
but some of the grade 4 product terms will be carried below to explicitly
hightlight the point where the intersecting bivector space requirement
effects the results.

\section{The meat.}

The geometric product of two bivectors can be written:

\begin{equation}\label{eqn:ABprod}
\BA \BB = 
\gpgrade{\BA \BB}{0}
+\gpgrade{\BA \BB}{2}
+\gpgrade{\BA \BB}{4}
= 
{\BA \cdot \BB}
+\gpgrade{\BA \BB}{2}
+{\BA \wedge \BB}
\end{equation}
\begin{equation}\label{eqn:BAprod}
\BB \BA = 
\gpgrade{\BB \BA}{0}
+\gpgrade{\BB \BA}{2}
+\gpgrade{\BB \BA}{4}
= 
{\BB \cdot \BA}
+\gpgrade{\BB \BA}{2}
+{\BB \wedge \BA}
\end{equation}

Because we have three terms involved, unlike the vector dot and wedge product
we cannot generally separate these terms by 
symmetric and antisymmetric parts.  However forming those sums
will still worthwhile, especially for the case of interecting bivectors
since the last term will be zero in that case.

\subsection{ Sign change of each grade term with commutation. }

Starting with the last term we can first observe that

\begin{equation}\label{eqn:wedgesign}
\BA \wedge \BB = \BB \wedge \BA
\end{equation}

To show this let $\BA = \Ba \wedge \Bb$, and $\BB = \Bc \wedge \Bd$.  When

$\BA \wedge \BB \ne 0$, one can write:

\begin{align*}
\BA \wedge \BB 
&= \Ba \wedge \Bb \wedge \Bc \wedge \Bd \\
&= - \Bb \wedge \Bc \wedge \Bd \wedge \Ba \\
&= \Bc \wedge \Bd \wedge \Ba \wedge \Bb \\
&= \BB \wedge \BA \\
\end{align*}

To see how the signs of the remaining two terms vary with commutation
form:

\begin{align*}
(\BA + \BB)^2
&= (\BA + \BB)(\BA + \BB) \\
&= \BA^2 + \BB^2 + \BA \BB + \BB \BA \\
\end{align*}

When $\BA$ and $\BB$ interect we can write
$\BA = \Ba \wedge \Bx$, and $\BB = \Bb \wedge \Bx$, thus the sum is a bivector

\[
(\BA + \BB)
= (\Ba + \Bb) \wedge \Bx
\]

And so, the square of the two is a scalar.  When $\BA$ and $\BB$ have only
non intersecting components, such as the grade two \R{4} multivector
$\Be_{12} + \Be_{34}$, the square of this sum will have both grade four and
scalar parts.

Since the LHS = RHS, and the grades of the two also must be the same.
This implies that the quantity

\[
\BA \BB + \BB \BA = 
\BA \cdot \BB + \BB \cdot \BA
+\gpgradetwo{\BA \BB} + \gpgradetwo{\BB \BA}
+\BA \wedge \BB + \BB \wedge \BA
\]

is a scalar $\iff$ 
$\BA + \BB$ is a bivector, and in general has scalar and grade four terms.
Because this symmetric sum has no grade two terms, 
regardless of whether $\BA$, and $\BB$ intersect, we have:

\[
\gpgradetwo{\BA \BB} + \gpgradetwo{\BB \BA} = 0
\]
\begin{equation}\label{eqn:signgradetwo}
\implies
\gpgradetwo{\BA \BB} = -\gpgradetwo{\BB \BA}
\end{equation}

One would intuitively expect $\BA \cdot \BB = \BB \cdot \BA$.  This can be
demonstrated by forming the complete symmetric sum

\begin{align*}
\BA \BB + \BB \BA 
&= 
{\BA \cdot \BB} +{\BB \cdot \BA}
+\gpgrade{\BA \BB}{2} +\gpgrade{\BB \BA}{2}
+{\BA \wedge \BB} + {\BB \wedge \BA} \\
&= 
{\BA \cdot \BB} +{\BB \cdot \BA}
+\gpgrade{\BA \BB}{2} -\gpgrade{\BA \BB}{2}
+{\BA \wedge \BB} + {\BA \wedge \BB} \\
&= 
{\BA \cdot \BB} +{\BB \cdot \BA}
+2{\BA \wedge \BB} \\
\end{align*}

The LHS commutes with interchange of $\BA$ and $\BB$, as does
${\BA \wedge \BB}$.  So for the RHS to also commute, the remaining grade 0 term
must also:

\begin{equation}\label{eqn:dotsign}
\BA \cdot \BB = \BB \cdot \BA
\end{equation}

\subsection{ Dot, wedge and grade two terms of bivector product. }

Collecting the results of the previous section and substituiting back
into equation \ref{eqn:ABprod} we have:

\begin{equation}\label{eqn:AdotB}
\BA \cdot \BB = \gpgrade{\frac{\BA \BB + \BB\BA}{2}}{0}
\end{equation}

\begin{equation}\label{eqn:AtwoB}
\gpgradetwo{\BA \BB} = \frac{\BA \BB - \BB\BA}{2}
\end{equation}

\begin{equation}\label{eqn:AwedgeB}
\BA \wedge \BB = \gpgrade{\frac{\BA \BB + \BB\BA}{2}}{4}
\end{equation}

\end{document}               % End of document.
