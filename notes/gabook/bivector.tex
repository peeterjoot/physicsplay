\documentclass{article}      % Specifies the document class

\usepackage{amsmath}
\usepackage{mathpazo}

%
% shorthand for bold symbols, convenient for vectors and matrices
%
\newcommand{\Ba}[0]{\mathbf{a}}
\newcommand{\Bb}[0]{\mathbf{b}}
\newcommand{\Bc}[0]{\mathbf{c}}
\newcommand{\Bd}[0]{\mathbf{d}}
\newcommand{\Be}[0]{\mathbf{e}}
\newcommand{\Bf}[0]{\mathbf{f}}
\newcommand{\Bg}[0]{\mathbf{g}}
\newcommand{\Bh}[0]{\mathbf{h}}
\newcommand{\Bi}[0]{\mathbf{i}}
\newcommand{\Bj}[0]{\mathbf{j}}
\newcommand{\Bk}[0]{\mathbf{k}}
\newcommand{\Bl}[0]{\mathbf{l}}
\newcommand{\Bm}[0]{\mathbf{m}}
\newcommand{\Bn}[0]{\mathbf{n}}
\newcommand{\Bo}[0]{\mathbf{o}}
\newcommand{\Bp}[0]{\mathbf{p}}
\newcommand{\Bq}[0]{\mathbf{q}}
\newcommand{\Br}[0]{\mathbf{r}}
\newcommand{\Bs}[0]{\mathbf{s}}
\newcommand{\Bt}[0]{\mathbf{t}}
\newcommand{\Bu}[0]{\mathbf{u}}
\newcommand{\Bv}[0]{\mathbf{v}}
\newcommand{\Bw}[0]{\mathbf{w}}
\newcommand{\Bx}[0]{\mathbf{x}}
\newcommand{\By}[0]{\mathbf{y}}
\newcommand{\Bz}[0]{\mathbf{z}}
\newcommand{\BA}[0]{\mathbf{A}}
\newcommand{\BB}[0]{\mathbf{B}}
\newcommand{\BC}[0]{\mathbf{C}}
\newcommand{\BD}[0]{\mathbf{D}}
\newcommand{\BE}[0]{\mathbf{E}}
\newcommand{\BF}[0]{\mathbf{F}}
\newcommand{\BG}[0]{\mathbf{G}}
\newcommand{\BH}[0]{\mathbf{H}}
\newcommand{\BI}[0]{\mathbf{I}}
\newcommand{\BJ}[0]{\mathbf{J}}
\newcommand{\BK}[0]{\mathbf{K}}
\newcommand{\BL}[0]{\mathbf{L}}
\newcommand{\BM}[0]{\mathbf{M}}
\newcommand{\BN}[0]{\mathbf{N}}
\newcommand{\BO}[0]{\mathbf{O}}
\newcommand{\BP}[0]{\mathbf{P}}
\newcommand{\BQ}[0]{\mathbf{Q}}
\newcommand{\BR}[0]{\mathbf{R}}
\newcommand{\BS}[0]{\mathbf{S}}
\newcommand{\BT}[0]{\mathbf{T}}
\newcommand{\BU}[0]{\mathbf{U}}
\newcommand{\BV}[0]{\mathbf{V}}
\newcommand{\BW}[0]{\mathbf{W}}
\newcommand{\BX}[0]{\mathbf{X}}
\newcommand{\BY}[0]{\mathbf{Y}}
\newcommand{\BZ}[0]{\mathbf{Z}}

\newcommand{\Bzero}[0]{\mathbf{0}}
\newcommand{\Btheta}[0]{\boldsymbol{\theta}}
\newcommand{\Btau}[0]{\boldsymbol{\tau}}
\newcommand{\Bomega}[0]{\boldsymbol{\omega}}

%
% shorthand for unit vectors
%
\newcommand{\acap}[0]{\hat{\Ba}}
\newcommand{\bcap}[0]{\hat{\Bb}}
\newcommand{\ccap}[0]{\hat{\Bc}}
\newcommand{\dcap}[0]{\hat{\Bd}}
\newcommand{\ecap}[0]{\hat{\Be}}
\newcommand{\fcap}[0]{\hat{\Bf}}
\newcommand{\gcap}[0]{\hat{\Bg}}
\newcommand{\hcap}[0]{\hat{\Bh}}
\newcommand{\icap}[0]{\hat{\Bi}}
\newcommand{\jcap}[0]{\hat{\Bj}}
\newcommand{\kcap}[0]{\hat{\Bk}}
\newcommand{\lcap}[0]{\hat{\Bl}}
\newcommand{\mcap}[0]{\hat{\Bm}}
\newcommand{\ncap}[0]{\hat{\Bn}}
\newcommand{\ocap}[0]{\hat{\Bo}}
\newcommand{\pcap}[0]{\hat{\Bp}}
\newcommand{\qcap}[0]{\hat{\Bq}}
\newcommand{\rcap}[0]{\hat{\Br}}
\newcommand{\scap}[0]{\hat{\Bs}}
\newcommand{\tcap}[0]{\hat{\Bt}}
\newcommand{\ucap}[0]{\hat{\Bu}}
\newcommand{\vcap}[0]{\hat{\Bv}}
\newcommand{\wcap}[0]{\hat{\Bw}}
\newcommand{\xcap}[0]{\hat{\Bx}}
\newcommand{\ycap}[0]{\hat{\By}}
\newcommand{\zcap}[0]{\hat{\Bz}}
\newcommand{\thetacap}[0]{\hat{\Btheta}}

%
% to write R^n and C^n in a distinguishable fashion.  Perhaps change this
% to the double lined characters upon figuring out how to do so.
%
\newcommand{\C}[1]{$\mathbb{C}^{#1}$}
\newcommand{\R}[1]{$\mathbb{R}^{#1}$}

%
% various generally useful helpers
%

% derivative of #1 wrt. #2:
\newcommand{\D}[2] {\frac {d#2} {d#1}}

\newcommand{\inv}[1]{\frac{1}{#1}}
\newcommand{\cross}[0]{\times}

\newcommand{\abs}[1]{\lvert{#1}\rvert}
\newcommand{\norm}[1]{\lVert{#1}\rVert}
\newcommand{\innerprod}[2]{\langle{#1}, {#2}\rangle}
\newcommand{\dotprod}[2]{{#1} \cdot {#2}}
\newcommand{\bdotprod}[2]{\left({#1} \cdot {#2}\right)}
\newcommand{\crossprod}[2]{{#1} \cross {#2}}
\newcommand{\tripleprod}[3]{\dotprod{\left(\crossprod{#1}{#2}\right)}{#3}}

\DeclareMathOperator{\Proj}{Proj}
\DeclareMathOperator{\Span}{span}
\DeclareMathOperator{\Sgn}{sgn}
\DeclareMathOperator{\Area}{Area}
\DeclareMathOperator{\Volume}{Volume}

%
% A few miscellaneous things specific to this document
%
\newcommand{\crossop}[1]{\crossprod{#1}{}}

% R2 vector.
\newcommand{\VectorTwo}[2]{
\begin{bmatrix}
 {#1} \\
 {#2}
\end{bmatrix}
}

\newcommand{\VectorN}[1]{
\begin{bmatrix}
{#1}_1 \\
{#1}_2 \\
\vdots \\
{#1}_N \\
\end{bmatrix}
}

\newcommand{\DETuvij}[4]{
\begin{vmatrix}
 {#1}_{#3} & {#1}_{#4} \\
 {#2}_{#3} & {#2}_{#4}
\end{vmatrix}
}

\newcommand{\DETuvwijk}[6]{
\begin{vmatrix}
 {#1}_{#4} & {#1}_{#5} & {#1}_{#6} \\
 {#2}_{#4} & {#2}_{#5} & {#2}_{#6} \\
 {#3}_{#4} & {#3}_{#5} & {#3}_{#6}
\end{vmatrix}
}

\newcommand{\DETuvwxijkl}[8]{
\begin{vmatrix}
 {#1}_{#5} & {#1}_{#6} & {#1}_{#7} & {#1}_{#8} \\
 {#2}_{#5} & {#2}_{#6} & {#2}_{#7} & {#2}_{#8} \\
 {#3}_{#5} & {#3}_{#6} & {#3}_{#7} & {#3}_{#8} \\
 {#4}_{#5} & {#4}_{#6} & {#4}_{#7} & {#4}_{#8} \\
\end{vmatrix}
}

%\newcommand{\DETuvwxyijklm}[10]{
%\begin{vmatrix}
% {#1}_{#6} & {#1}_{#7} & {#1}_{#8} & {#1}_{#9} & {#1}_{#10} \\
% {#2}_{#6} & {#2}_{#7} & {#2}_{#8} & {#2}_{#9} & {#2}_{#10} \\
% {#3}_{#6} & {#3}_{#7} & {#3}_{#8} & {#3}_{#9} & {#3}_{#10} \\
% {#4}_{#6} & {#4}_{#7} & {#4}_{#8} & {#4}_{#9} & {#4}_{#10} \\
% {#5}_{#6} & {#5}_{#7} & {#5}_{#8} & {#5}_{#9} & {#5}_{#10}
%\end{vmatrix}
%}

% R3 vector.
\newcommand{\VectorThree}[3]{
\begin{bmatrix}
 {#1} \\
 {#2} \\
 {#3}
\end{bmatrix}
}



\usepackage{color,cite,graphicx}
   % use colour in the document, put your citations as [1-4]
   % rather than [1,2,3,4] (it looks nicer, and the extended LaTeX2e
   % graphics package. 
\usepackage{latexsym,amssymb,epsf} % don't remember if these are
   % needed, but their inclusion can't do any damage

\newcommand{\laplacian}[0]{\nabla^2}
\newcommand{\Dsq}[2] {\frac {\partial^2 {#1}} {\partial {#2}^2}}
\newcommand{\dxj}[2] {\frac {\partial {#1}} {\partial {x_{#2}}}}
\newcommand{\dsqxj}[2] {\frac {\partial^2 {#1}} {\partial {x_{#2}}^2}}
\DeclareMathOperator{\Exp}{e}
\DeclareMathOperator{\Rej}{Rej}
\DeclareMathOperator{\Rot}{R}
\newcommand{\gpgrade}[2] {{\left\langle{{#1}}\right\rangle}_{#2}}
\newcommand{\gpgradezero}[1] {\gpgrade{#1}{0}}
\newcommand{\gpgradetwo}[1] {\gpgrade{#1}{2}}
\newcommand{\gpgradefour}[1] {\gpgrade{#1}{4}}

%
% The real thing:
%

                             % The preamble begins here.
\title{Bivector Geometry.}
\author{Peeter Joot}         % Declares the author's name.
%\date{}        % Deleting this command produces today's date.

\begin{document}             % End of preamble and beginning of text.

\maketitle{}

\section{ The problem. }

Examination of exponential solutions for Laplace's equation leads one to
a requirement to examine the product of intersecting bivectors such as

\[
\left(\abs{\Bx \wedge \Bk}^2\right)' = -\left(
(\Bx' \wedge \Bv)(\Bx \wedge \Bv)
+(\Bx \wedge \Bv)(\Bx' \wedge \Bv)
\right)
\]

Here we see that the symmetric sum of bivectors $\Bx \wedge \Bk$ and $\Bx' \wedge \Bk$ is a scalar quantity.  This we will identify later as a quantity
related to the bivector dot product.

It is worthwhile to systematically examine the
general products of intersecting bivectors, that is planes that share a common line, in this case the line directed along the vector $\Bk$.
It is also notable that since all non coplanar bivectors in \R{3} intersect
this
examination will cover the important special case of three dimensional
plane geometry.

A result of this examination is that many of the concepts familiar from
vector geometry such as
orthagonality, projection, and rejection will have direct bivector
equivalents.

General bivector geometry, in spaces where non-coplanar bivectors do not 
neccessarily intersect (such as in \R{4}), will need to be treated separately,
but some of the grade 4 product terms will be carried below to explicitly
hightlight the point where the intersecting bivector space requirement
effects the results.

\section{Components of grade two multivector product.}

The geometric product of two bivectors can be written:

\begin{equation}\label{eqn:ABprod}
\BA \BB = 
\gpgrade{\BA \BB}{0}
+\gpgrade{\BA \BB}{2}
+\gpgrade{\BA \BB}{4}
= 
{\BA \cdot \BB}
+\gpgrade{\BA \BB}{2}
+{\BA \wedge \BB}
\end{equation}
\begin{equation}\label{eqn:BAprod}
\BB \BA = 
\gpgrade{\BB \BA}{0}
+\gpgrade{\BB \BA}{2}
+\gpgrade{\BB \BA}{4}
= 
{\BB \cdot \BA}
+\gpgrade{\BB \BA}{2}
+{\BB \wedge \BA}
\end{equation}

Because we have three terms involved, unlike the vector dot and wedge product
we cannot generally separate these terms by 
symmetric and antisymmetric parts.  However forming those sums
will still worthwhile, especially for the case of interecting bivectors
since the last term will be zero in that case.

\subsection{ Sign change of each grade term with commutation. }

Starting with the last term we can first observe that

\begin{equation}\label{eqn:wedgesign}
\BA \wedge \BB = \BB \wedge \BA
\end{equation}

To show this let $\BA = \Ba \wedge \Bb$, and $\BB = \Bc \wedge \Bd$.  When

$\BA \wedge \BB \ne 0$, one can write:

\begin{align*}
\BA \wedge \BB 
&= \Ba \wedge \Bb \wedge \Bc \wedge \Bd \\
&= - \Bb \wedge \Bc \wedge \Bd \wedge \Ba \\
&= \Bc \wedge \Bd \wedge \Ba \wedge \Bb \\
&= \BB \wedge \BA \\
\end{align*}

To see how the signs of the remaining two terms vary with commutation
form:

\begin{align*}
(\BA + \BB)^2
&= (\BA + \BB)(\BA + \BB) \\
&= \BA^2 + \BB^2 + \BA \BB + \BB \BA \\
\end{align*}

When $\BA$ and $\BB$ interect we can write
$\BA = \Ba \wedge \Bx$, and $\BB = \Bb \wedge \Bx$, thus the sum is a bivector

\[
(\BA + \BB)
= (\Ba + \Bb) \wedge \Bx
\]

And so, the square of the two is a scalar.  When $\BA$ and $\BB$ have only
non intersecting components, such as the grade two \R{4} multivector
$\Be_{12} + \Be_{34}$, the square of this sum will have both grade four and
scalar parts.

Since the LHS = RHS, and the grades of the two also must be the same.
This implies that the quantity

\[
\BA \BB + \BB \BA = 
\BA \cdot \BB + \BB \cdot \BA
+\gpgradetwo{\BA \BB} + \gpgradetwo{\BB \BA}
+\BA \wedge \BB + \BB \wedge \BA
\]

is a scalar $\iff$ 
$\BA + \BB$ is a bivector, and in general has scalar and grade four terms.
Because this symmetric sum has no grade two terms, 
regardless of whether $\BA$, and $\BB$ intersect, we have:

\[
\gpgradetwo{\BA \BB} + \gpgradetwo{\BB \BA} = 0
\]
\begin{equation}\label{eqn:signgradetwo}
\implies
\gpgradetwo{\BA \BB} = -\gpgradetwo{\BB \BA}
\end{equation}

One would intuitively expect $\BA \cdot \BB = \BB \cdot \BA$.  This can be
demonstrated by forming the complete symmetric sum

\begin{align*}
\BA \BB + \BB \BA 
&= 
{\BA \cdot \BB} +{\BB \cdot \BA}
+\gpgrade{\BA \BB}{2} +\gpgrade{\BB \BA}{2}
+{\BA \wedge \BB} + {\BB \wedge \BA} \\
&= 
{\BA \cdot \BB} +{\BB \cdot \BA}
+\gpgrade{\BA \BB}{2} -\gpgrade{\BA \BB}{2}
+{\BA \wedge \BB} + {\BA \wedge \BB} \\
&= 
{\BA \cdot \BB} +{\BB \cdot \BA}
+2{\BA \wedge \BB} \\
\end{align*}

The LHS commutes with interchange of $\BA$ and $\BB$, as does
${\BA \wedge \BB}$.  So for the RHS to also commute, the remaining grade 0 term
must also:

\begin{equation}\label{eqn:dotsign}
\BA \cdot \BB = \BB \cdot \BA
\end{equation}

\subsection{ Dot, wedge and grade two terms of bivector product. }

Collecting the results of the previous section and substituiting back
into equation \ref{eqn:ABprod} we have:

\begin{equation}\label{eqn:AdotB}
\BA \cdot \BB = \gpgrade{\frac{\BA \BB + \BB\BA}{2}}{0}
\end{equation}

\begin{equation}\label{eqn:AtwoB}
\gpgradetwo{\BA \BB} = \frac{\BA \BB - \BB\BA}{2}
\end{equation}

\begin{equation}\label{eqn:AwedgeB}
\BA \wedge \BB = \gpgrade{\frac{\BA \BB + \BB\BA}{2}}{4}
\end{equation}

When these intersect in a line the wedge term is zero, so for that special case we can write:

\begin{equation*}
\BA \cdot \BB = \frac{\BA \BB + \BB\BA}{2}
\end{equation*}

\begin{equation*}
\gpgradetwo{\BA \BB} = \frac{\BA \BB - \BB\BA}{2}
\end{equation*}

\begin{equation*}
\BA \wedge \BB = 0
\end{equation*}

(note that this is always the case for \R{3}).

\section{ Intersection of planes. }

Starting with two planes specified parametrically, each in terms of two direction vectors and a point on the plane:

\begin{align}
\Bx &= \Bp + \alpha \Bu + \beta \Bv \label{eqn:firstplane} \\
\By &= \Bq + a \Bw + b \Bz \label{eqn:secondplane} \\
\end{align}

If these intersect then all points on the line must satisify $\Bx = \By$, so the
solution requires:

\[
\Bp + \alpha \Bu + \beta \Bv = \Bq + a \Bw + b \Bz
\]
\[
\implies
(\Bp + \alpha \Bu + \beta \Bv) \wedge \Bw \wedge \Bz = (\Bq + a \Bw + b \Bz) \wedge \Bw \wedge \Bz = \Bq \wedge \Bw \wedge \Bz
\]

Rearranging for $\beta$, and writing $\BB = \Bw \wedge \Bz$:

\[
\beta = \frac{\Bq \wedge \BB - (\Bp + \alpha \Bu) \wedge \BB}{\Bv \wedge \BB}
\]

Note that when the solution exists the left vs right order of the division by $\Bv \wedge \BB$ should not matter since the numerator will be proportional to this bivector (or else the $\beta$ would not be a scalar).

Substitution of $\beta$ back into $\Bx = \Bp + \alpha \Bu + \beta \Bv$ (all points in the first plane) gives you a parametric equation for a line:

\[
\Bx = \Bp + \frac{(\Bq-\Bp)\wedge \BB}{\Bv \wedge \BB}\Bv + \alpha\frac{1}{\Bv \wedge \BB}((\Bv \wedge \BB) \Bu - (\Bu \wedge \BB)\Bv)
\]

Where a point on the line is:

\[
\Bp + \frac{(\Bq-\Bp)\wedge \BB}{\Bv \wedge \BB}\Bv 
%= \frac{1}{\Bv \wedge \BB}((\Bv \wedge \BB)\Bp + ((\Bq-\Bp)\wedge \BB)\Bv)
\]

And a direction vector for the line is:

\[
\frac{1}{\Bv \wedge \BB}((\Bv \wedge \BB) \Bu - (\Bu \wedge \BB)\Bv)
\]
\[
\propto
(\Bv \wedge \BB)^2 \Bu - (\Bv \wedge \BB)(\Bu \wedge \BB)\Bv
\]

Now, this result is only valid if $\Bv \wedge \BB \ne 0$ (ie: line of intersection is not directed along $\Bv$), but if that is the case the second form will be zero.  Thus we can add the results (or any non-zero linear combination of) allowing for either of $\Bu$, or $\Bv$ to be directed along the line of intersection:

\begin{equation}\label{eqn:dirvecintersection}
a\left( (\Bv \wedge \BB)^2 \Bu
- (\Bv \wedge \BB)(\Bu \wedge \BB)\Bv \right)
+ b\left((\Bu \wedge \BB)^2 \Bv 
- (\Bu \wedge \BB)(\Bv \wedge \BB)\Bu\right)
\end{equation}

Alternately, one could formulate this in terms of $\BA = \Bu \wedge \Bv$, $\Bw$, and $\Bz$.  Is there a more symetrical form for this direction vector?

\subsection{ Vector along line of intersection in \R{3}}

For \R{3} one can solve the intersection problem using the normals to the planes.  For simplicity put the origin on the line of intersection (and all planes through a common point in \R{3} have at least a line of intersection).  In this case, for bivectors $\BA$ and $\BB$, normals to those planes are $i\BA$, and $i\BB$ respectively.  The plane through both of those normals is:

\begin{align*}
(i\BA) \wedge (i\BB)
= \frac{(i\BA)(i\BB) - (i\BB)(i\BA)}{2} 
= \frac{\BB\BA - \BA\BB}{2} 
= \gpgradetwo{\BB\BA}
\end{align*}

The normal to this plane

\begin{equation}\label{eqn:r3planeintersect}
i\gpgradetwo{\BB\BA}
\end{equation}

is directed along the line of interesection.  This result is more appealing than
the general \R{N} result of equation \ref{eqn:dirvecintersection}, not
just because it is simpler, but also because it is a function of only the
bivectors for the planes, without a requirement to find or calculate
two specific independent direction vectors in one of the planes.

\subsection{ Applying this result to \R{N} }

If you reject the component of $\BA$ from $\BB$ for two intersecting bivectors:

\[
\Rej_{\BA}(\BB) = \frac{1}{\BA}\gpgradetwo{\BA\BB}
\]

the line of intersection remains the same ... that operation rotates $\BB$ so that the two are mutually perpendicular.  This essentially reduces the problem to that of the three dimensional case, so the solution has to be of the same form... you just need to calculate a ``pseudoscalar'' (what you are calling the join), for the subspace spanned by the two bivectors.

That can be computed by taking any direction vector that is on one plane, but isn't in the second.  For example, pick a vector $\Bu$ in the plane $\BA$ that is not on the intersection of $\BA$ and $\BB$.  In mathese that is $\Bu = \inv{\BA}(\BA\cdot \Bu)$ (or $\Bu \wedge \BA = 0$), where $\Bu \wedge \BB \ne 0$.  Thus a pseudoscalar for this subspace is:

\[
\Bi = \frac{\Bu \wedge \BB}{\abs{\Bu \wedge \BB}}
\]

To calculate the direction vector along the intersection we don't care about the scaling above.  Also note that provided $\Bu$ has a component in the plane $\BA$, $\Bu \cdot \BA$ is also in the plane (it's rotated $\pi/2$ from $\inv{\BA}(\BA \cdot \Bu)$.

Thus, provided that $\Bu \cdot \BA$ isn't on the intersection, a scaled ``pseudoscalar''
for the subspace can be calculated by taking from any vector $\Bu$ with a component in the plane $\BA$:

\[
\Bi \propto (\Bu \cdot \BA) \wedge \BB
\]

Thus a vector along the intersection is:

\begin{equation}\label{eqn:pseudoscalarinter}
\Bd = ((\Bu \cdot \BA) \wedge \BB) \gpgradetwo{\BA\BB}
\end{equation}

Interchange of $\BA$ and $\BB$ in either the trivector or bivector terms above would also work.

Without showing the steps one can write the complete parametric solution of the line through the planes of equations \ref{eqn:firstplane} and \ref{eqn:secondplane} in terms of this direction vector:

\begin{equation}\label{eqn:finalsolnofRNplaneintersection}
\Bx = \Bp + \left(\frac{(\Bq - \Bp)\wedge \BB}{(\Bd \cdot \BA) \wedge \BB}\right) (\Bd \cdot \BA) + \alpha \Bd
\end{equation}

Since $(\Bd \cdot \BA) \ne 0$ and $(\Bd \cdot \BA) \wedge \BB \ne 0$ (unless $\BA$ and $\BB$ are coplanar), observe that this is a natural generator
of the pseudoscalar for the subspace, and as such shows up in the expression
above.

Also observe the non-coincidental similarity of the $\Bq-\Bp$ term to Cramer's rule (a ration of determinants).

\section{ Components of a grade two multivector }

The procedure to calculate projections and rejections of planes onto planes
is similar to a vector projection onto a space.

To arrive at that result we can consider the product of a grade two multivector $\BA$ with a bivector $\BB$ and its inverse (
the restriction that $\BB$ be a bivector, a grade two multivector that can be written as a wedge product of two vectors, is required for general invertability).

\begin{align*}
\BA\inv{\BB}\BB 
&= \left(\BA \cdot \inv{\BB} + \gpgradetwo{ \BA \inv{\BB} } + \BA \wedge \inv{\BB}\right) \BB \\
&= 
\BA \cdot \inv{\BB} \BB \\
&
+\gpgradetwo{ \BA \inv{\BB} } \cdot \BB 
+\gpgradetwo{ \gpgradetwo{ \BA \inv{\BB} } \BB }
+\gpgradetwo{ \BA \inv{\BB} } \wedge \BB \\
&
+\left(\BA \wedge \inv{\BB}\right) \cdot \BB 
+\gpgradefour{\BA \wedge \inv{\BB} \BB}
+\BA \wedge \inv{\BB} \wedge \BB \\
\end{align*}

Since $\inv{\BB} = -\frac{\BB}{{\abs{\BB}}^2}$, this implies that the 6-grade term $\BA \wedge \inv{\BB} \wedge \BB$ is zero.  Since the LHS has grade 2, this
implies that the 0-grade and 4-grade terms are zero (also independently implies that the 6-grade term is zero).  This leaves:

\begin{equation}\label{eqn:bivectorprojbivector}
\BA
= 
\BA \cdot \inv{\BB} \BB \\
+\gpgradetwo{\gpgradetwo{\BA\inv\BB} \BB}
+\left(\BA \wedge \inv{\BB}\right) \cdot \BB 
\end{equation}

This could be written somewhat more symmetrically as
\begin{align*}
\BA
&=\sum_{i=0,2,4}\gpgradetwo{\gpgrade{\BA \inv{\BB}}{i} \BB} \\
&= \gpgradetwo{ \gpgradezero{\BA \inv{\BB}} \BB +\gpgradetwo{\BA \inv{\BB}} \BB +\gpgradefour{\BA \inv{\BB}} \BB } \\
\end{align*}

This is also a more direct way to derive the result in retrospect.

Looking at equation \ref{eqn:bivectorprojbivector} we have three terms:

\begin{enumerate}
\item \[ \BA \cdot \inv{\BB} \BB \]

This is the component of $\BA$ that lies in the plane $\BB$ (the projection
of $\BA$ onto $\BB$).

\item
\begin{equation}\label{eqn:gpgradetwo}
\gpgradetwo{\gpgradetwo{\BA\inv\BB} \BB}
\end{equation}

If $\BB$ and $\BA$ have any intersecting components, this is the components
of $\BA$ from the intersection that are perpendicular to $\BB$ with respect to the bivector dot product.  ie: This is the rejective term.

\item \[ \left(\BA \wedge \inv{\BB}\right) \cdot \BB \]

This is the remainder, the non-projective and non-coplanar terms.  
Greater than three dimensions is required to generate such a term.  Example:

\begin{align*}
\BA &= \Be_{12} + \Be_{23} + \Be_{43} \\
\BB &= \Be_{34} \\
\end{align*}

Product terms for these are:

\begin{align*}
\BA \cdot \BB &= 1 \\
\gpgradetwo{\BA \BB} &= \Be_{24} \\
\BA \wedge \BB &= \Be_{1234} \\
\end{align*}

The decomposition is thus:
\[
\BA = \left(\BA \cdot \BB + \gpgradetwo{\BA \BB} + \BA \wedge \BB\right) \inv{\BB} = (1 + \Be_{24} + \Be_{1234}) \Be_{43}
\]

\end{enumerate}

\subsection{ Closer look at the grade two term }

The grade two term of equation \ref{eqn:gpgradetwo} can be expanded using its antisymmetric bivector product representation

\begin{align*}
\gpgradetwo{\BA\inv\BB} \BB
&= \inv{2}\left(\BA\inv{\BB} - \inv{\BB}\BA\right) \BB \\
&= \inv{2}\left(\BA - \inv{\BB}\BA \BB\right) \\
&= \inv{2}\left(\BA - \inv{\hat{\BB}}\BA \hat{\BB}\right) \\
\end{align*}

Observe here one can restrict the examination to the case where $\BB$ is a unit bivector without loss of generality.

\begin{align*}
\gpgradetwo{\BA\inv\Bi} \Bi
&= \inv{2}\left(\BA + \Bi\BA\Bi\right) \\
&= \inv{2}\left(\BA - \Bi^\dagger\BA\Bi\right) \\
\end{align*}

The second term is a rotation in the plane $\Bi$, by 180 degrees:

\[
\Bi^\dagger\BA\Bi = \Exp^{-\Bi \pi/2}\BA \Exp^{\Bi \pi/2}
\]

So, any components of $\BA$ that are completely in the plane cancel out (ie: the $\BA \cdot \inv{\Bi}\Bi$ component).

Also, if $\gpgradefour{\BA \Bi} \ne 0$ then those components of $\BA \Bi$ commute so

\begin{align*}
\gpgradefour{\BA - \Bi^\dagger\BA\Bi}
&= \gpgradefour{\BA} - \gpgradefour{\Bi^\dagger\BA\Bi} \\
&= \gpgradefour{\BA} - \gpgradefour{\Bi^\dagger\Bi\BA} \\
&= \gpgradefour{\BA} - \gpgradefour{\BA} \\
&= 0 \\
\end{align*}

This implies that we have only grade two terms, and the final grade selection in equation \ref{eqn:gpgradetwo} can be dropped:

\begin{equation} \label{eqn:simplergpgradetwo}
\gpgradetwo{\gpgradetwo{\BA\inv\BB} \BB} = \gpgradetwo{\BA\inv\BB} \BB
\end{equation}

It's also possible to write this in a few alternate variations which are useful to list explicitly so that one
can recognize them in other contexts:

\begin{align*}
\gpgradetwo{\BA\inv\BB} \BB
&= \inv{2}\left(\BA - \inv{\BB}\BA\BB\right)  \\
&= \inv{2}\left(\BA + \hat{\BB}\BA\hat{\BB}\right)  \\
&= \inv{2}\left( \hat{\BB}\BA -\BA\hat{\BB} \right)\hat{\BB} \\
&= \gpgradetwo{\hat{\BB}\BA}\hat{\BB} \\
&= \hat{\BB}\gpgradetwo{{\BA}\hat{\BB}} \\
\end{align*}

\subsection{ Projection and Rejection }

Equation \ref{eqn:simplergpgradetwo} can be substuited back into equation \ref{eqn:bivectorprojbivector} yeilding:

\begin{equation}\label{eqn:simplerbivectorprojbivector}
\BA =
\BA \cdot \inv{\BB} \BB \\
+\gpgradetwo{\BA\inv\BB} \BB
+\left(\BA \wedge \inv{\BB}\right) \cdot \BB 
\end{equation}

Now, for the special case where $\BA \wedge \BB = 0$ (all bivector components of the grade two multivector $\BA$ have a common vector with bivector $\BB$) we can write

\begin{align*}
\BA 
&= \BA \cdot \inv{\BB} \BB +\gpgradetwo{\BA\inv{\BB}} \BB \\
&= \BB \inv{\BB} \cdot {\BA} + \BB \gpgradetwo{\inv{\BB}\BA} \\
&= \Proj_{\BB}(\BA) + \Rej_{\BB}(\BA) \label{eqn:projrejbivectorbivector} \\
\end{align*}

It's worth verifying that these two terms are orthogonal (with respect to the grade two vector dot product)
\begin{align*}
\Proj_{\BB}(\BA) \cdot \Rej_{\BB}(\BA)
&= \gpgradezero{ \Proj_{\BB}(\BA) \Rej_{\BB}(\BA) } \\
&= \gpgradezero{ \BA \cdot \inv{\BB} \BB \BB \gpgradetwo{\inv{\BB}\BA} } \\
&= \inv{4\BB^2}\gpgradezero{ (\BA\BB + \BB\BA)(\BB\BA - \BA\BB) } \\
&= \inv{4\BB^2}\gpgradezero{ \BA\BB\BB\BA -\BA\BB\BA\BB +\BB\BA\BB\BA -\BB\BA\BA\BB } \\
&= \inv{4\BB^2}\gpgradezero{ -\BA\BB\BA\BB +\BB\BA\BB\BA } \\
\end{align*}

Since we have introduced the restriction 
$\BA \wedge \BB \ne 0$, we can 
use the dot product to reorder product terms:

\[
\BA\BB = -\BB\BA + 2 \BA \cdot \BB
\]

This can be used to reduce the grade zero term above:
\begin{align*}
\gpgradezero{ \BB\BA\BB\BA -\BA\BB\BA\BB }
&= \gpgradezero{ \BB\BA(-\BA\BB + 2 \BA \cdot \BB) -(-\BB\BA + 2 \BA \cdot \BB)\BA\BB } \\
&= + 2 (\BA \cdot \BB)\gpgradezero{\BB\BA - \BA\BB } \\
&= + 4 (\BA \cdot \BB)\gpgradezero{\gpgradetwo{\BB\BA}} \\
&= 0 \\
\end{align*}

This proves orthogonality as expected.

\subsection{ Grade two term as a generator of rotations. }

\begin{figure}[htp]
\centering
\includegraphics[totalheight=0.4\textheight]{planerejection}
\caption{Bivector rejection.  Perpendicular component of plane.}\label{fig:planerejection}
\end{figure}

Figure \ref{fig:planerejection} illustrates how the grade 2 component of the
bivector product acts as a rotation in the rejection operation.

Provided that $\BA$ and $\BB$ are not coplanar, $\gpgradetwo{\BA\BB}$ is a plane mutually perpendicular to both.


Given two mutually perpendicular unit bivectors ${\BA}$ and ${\BB}$, we can in fact write: 

\[
{\BB} = {\BA}\gpgradetwo{{\BB}{\BA}}
\]
\[
{\BB} = \gpgradetwo{{\BA}{\BB}}{\BA}
\]

Compare this to a unit bivector for two mutually perpendicular vectors:

\[
\Bb = \Ba (\Ba \wedge \Bb)
\]
\[
\Bb = (\Bb \wedge \Ba) \Ba
\]

In both cases, the unit bivector functions as an imaginary number,
applying a rotation of $\pi/2$ rotating one of the 
perpendicular entities onto the other.

As with vectors one can split the rotation of the unit bivector into half
angle left and right rotations.  For example, for the same mutually
perpendicular pair of bivectors one can write

\begin{align*}
\BB 
&= \BA\gpgradetwo{\BB \BA} \\
&= \BA \Exp^{\gpgradetwo{\BB \BA}\pi/2} \\
&= \Exp^{-\gpgradetwo{\BB \BA}\pi/4} \BA \Exp^{\gpgradetwo{\BB \BA}\pi/4} \\
&= \left(\inv{\sqrt{2}}(1 - \BB \BA)\right) \BA \left(\inv{\sqrt{2}}(1 + \BB \BA) \right) \\
\end{align*}
%&= \inv{2}(\BA + \BB )(1 + \BB \BA)
%&= \inv{2}(\BA + \BB + \BA\BB\BA + \BB\BB\BA)
%&= \inv{2}(\BA + \BB - \BA\BA\BB + \BB\BB\BA)
%&= \inv{2}(\BA + \BB + \BB - \BA)

Direct multiplication can be used to verify that this does in fact produce the desired result.

In general, writing 

\[
\Bi = \frac{\gpgradetwo{\BB \BA}}{\abs{\gpgradetwo{\BB \BA}}}
\]

the rotation of plane $\BB$ towards $\BA$ by angle $\theta$ can be expressed with either a single sided full angle

\begin{align*}
\Rot_{\theta: \BA \rightarrow \BB}(\BA) 
&= \BA \Exp^{\Bi \theta} \\
&= \Exp^{-\Bi \theta} \BA \\
\end{align*}

or double sided the half angle rotor formulas:

\begin{equation}\label{eqn:rotor}
\Rot_{\theta: \BA \rightarrow \BB}(\BA) = \Exp^{-\Bi \theta/2} \BA \Exp^{\Bi \theta/2} = \BR^\dagger \BA \BR
\end{equation}

Where:
\begin{align*}
\BR 
&= \Exp^{\Bi\theta/2} \\
&= \cos(\theta/2) + \frac{\gpgradetwo{\BB \BA}}{\abs{\gpgradetwo{\BB \BA}}}\sin(\theta/2) \\
\end{align*}

As with half angle rotors applied to vectors, there are two possible orientations to rotate.  Here the orientation of the rotation is such that the angle is measured along the minimal arc betwen the two, where the angle between the two is in the range $(0,\pi)$ as opposed to the $(\pi,2\pi)$ rotational direction.

\subsection{ Angle between two intersecting planes. }

Worth pointing out for comparison to the vector result, one can use the bivector dot product to calculate the angle between two
interecting planes.
This angle of separation $\theta$ between the two can be expressed using the exponential:

\[
\hat{\BB} = \hat{\BA} \Exp^{ \frac{\gpgradetwo{\BB \BA}}{\abs{\gpgradetwo{\BB \BA}}} \theta}
\]
\[
\implies
-\hat{\BA} \hat{\BB} = \Exp^{ \frac{\gpgradetwo{\BB \BA}}{\abs{\gpgradetwo{\BB \BA}}} \theta}
\]

Taking the grade zero terms of both sides we have:
\[
-\gpgradezero{\hat{\BA} \hat{\BB}} = \gpgradezero{ \Exp^{ \frac{\gpgradetwo{\BB \BA}}{\abs{\gpgradetwo{\BB \BA}}} \theta} }
\]
\[
\implies
\cos(\theta) = - \frac{\BA \cdot \BB}{\abs{\BA}\abs{\BB}}
\]

%The negative value here is expected since A and B coplanar have A . A = -|A|^2

The sine can be obtained by selecting the grade two terms

\[
-\gpgradetwo{\hat{\BA} \hat{\BB}} = \frac{\gpgradetwo{\BB \BA}}{\abs{\gpgradetwo{\BB \BA}}} \sin(\theta)
\]
\[
%\inv{\abs{\BA}\abs{\BB}} = \frac{1}{\abs{\gpgradetwo{\BB \BA}}} \sin(\theta)
\implies
\sin(\theta) = \frac{\abs{\gpgradetwo{\BB \BA}}}{ \abs{\BA}\abs{\BB} }
\]

Note that the strictly positive sine result here is consistent with the fact that the angle is being measured such that it is in the
$(0,\pi)$ range.

\subsection{ Application of half angle Rotors to general planes. }

A couple observations can be made about the application of equation \ref{eqn:rotor} on an arbitrarily oriented plane.  Express the rotor
in terms of a unit bivector $\Bi$:

\[
\BR = \Exp^{\Bi\theta/2}
\]

where it acts on $\BA$ via $\BR^\dagger \BA \BR$ to apply a rotation of $\theta$ radians in the plane with bivector $\Bi$.

Let's express $\BA$ in terms of it's components with respect to $\Bi$:

\begin{equation}\label{eqn:Aicomponents}
\BA = \BA \cdot \inv{\Bi} \Bi + \gpgradetwo{\BA \inv{\Bi}} \Bi + (\BA \wedge \inv{\Bi}) \cdot \Bi
\end{equation}

We should expect the action of the rotation to not effect the first term (rotating any pair of vectors in a plane does not change the
relative angle between those vectors or their magnitudes).  This can be confirmed with straight multiplication:

\begin{align*}
\BR^\dagger (\alpha \Bi) \BR
&= (\cos(\theta/2) - \Bi \sin(\theta/2))(\alpha \Bi) (\cos(\theta/2) + \Bi \sin(\theta/2)) \\
&= (\cos(\theta/2) - \Bi \sin(\theta/2))(\cos(\theta/2) + \sin(\theta/2) \Bi ) \Bi \alpha \\
&= (\cos^2(\theta/2) - \Bi^2 \sin^2(\theta/2)) \Bi \alpha \\
&= \alpha \Bi \\
\end{align*}

The action of the rotor on the second term of equation \ref{eqn:Aicomponents} has already been examined (that's what we started with rotation of planes perpendicular to the plane of rotation).

Because the third term of equation \ref{eqn:Aicomponents} is symmetric like the projection term intuition would says the rotor equation will
also leave that untouched.  Replacement of $\alpha \Bi$ for the action of the rotor on the projective term does in fact prove this if
one first observes that the $(\BA \wedge \inv{\Bi}) \cdot \Bi$ term also commutes with $\Bi$, thus also commutes with the rotor $\BR$.

Now, constrast this to the effects of rotating a pair of vectors.  Is the bivector produced by the rotated vectors consistent with applying
the rotor equation to the bivector directly?

% A <BA>_2 = A/2(BA-AB) = 1/2(ABA-AAB) = 1/2(ABA -BAA) = 1/2(AB -BA)A = <AB>_2 A = -<BA>_2 A
% if <AB>_2 ^ C^D \ne 0 -> <AB>_2 ^ C ^ D = C ^ <AB>_2 ^ D = C ^ D ^ <AB>_2

\end{document}               % End of document.
