\documentclass{article}

\usepackage{amsmath}
\usepackage{mathpazo}

%
% shorthand for bold symbols, convenient for vectors and matrices
%
\newcommand{\Ba}[0]{\mathbf{a}}
\newcommand{\Bb}[0]{\mathbf{b}}
\newcommand{\Bc}[0]{\mathbf{c}}
\newcommand{\Bd}[0]{\mathbf{d}}
\newcommand{\Be}[0]{\mathbf{e}}
\newcommand{\Bf}[0]{\mathbf{f}}
\newcommand{\Bg}[0]{\mathbf{g}}
\newcommand{\Bh}[0]{\mathbf{h}}
\newcommand{\Bi}[0]{\mathbf{i}}
\newcommand{\Bj}[0]{\mathbf{j}}
\newcommand{\Bk}[0]{\mathbf{k}}
\newcommand{\Bl}[0]{\mathbf{l}}
\newcommand{\Bm}[0]{\mathbf{m}}
\newcommand{\Bn}[0]{\mathbf{n}}
\newcommand{\Bo}[0]{\mathbf{o}}
\newcommand{\Bp}[0]{\mathbf{p}}
\newcommand{\Bq}[0]{\mathbf{q}}
\newcommand{\Br}[0]{\mathbf{r}}
\newcommand{\Bs}[0]{\mathbf{s}}
\newcommand{\Bt}[0]{\mathbf{t}}
\newcommand{\Bu}[0]{\mathbf{u}}
\newcommand{\Bv}[0]{\mathbf{v}}
\newcommand{\Bw}[0]{\mathbf{w}}
\newcommand{\Bx}[0]{\mathbf{x}}
\newcommand{\By}[0]{\mathbf{y}}
\newcommand{\Bz}[0]{\mathbf{z}}
\newcommand{\BA}[0]{\mathbf{A}}
\newcommand{\BB}[0]{\mathbf{B}}
\newcommand{\BC}[0]{\mathbf{C}}
\newcommand{\BD}[0]{\mathbf{D}}
\newcommand{\BE}[0]{\mathbf{E}}
\newcommand{\BF}[0]{\mathbf{F}}
\newcommand{\BG}[0]{\mathbf{G}}
\newcommand{\BH}[0]{\mathbf{H}}
\newcommand{\BI}[0]{\mathbf{I}}
\newcommand{\BJ}[0]{\mathbf{J}}
\newcommand{\BK}[0]{\mathbf{K}}
\newcommand{\BL}[0]{\mathbf{L}}
\newcommand{\BM}[0]{\mathbf{M}}
\newcommand{\BN}[0]{\mathbf{N}}
\newcommand{\BO}[0]{\mathbf{O}}
\newcommand{\BP}[0]{\mathbf{P}}
\newcommand{\BQ}[0]{\mathbf{Q}}
\newcommand{\BR}[0]{\mathbf{R}}
\newcommand{\BS}[0]{\mathbf{S}}
\newcommand{\BT}[0]{\mathbf{T}}
\newcommand{\BU}[0]{\mathbf{U}}
\newcommand{\BV}[0]{\mathbf{V}}
\newcommand{\BW}[0]{\mathbf{W}}
\newcommand{\BX}[0]{\mathbf{X}}
\newcommand{\BY}[0]{\mathbf{Y}}
\newcommand{\BZ}[0]{\mathbf{Z}}

\newcommand{\Bzero}[0]{\mathbf{0}}
\newcommand{\Btheta}[0]{\boldsymbol{\theta}}
\newcommand{\Btau}[0]{\boldsymbol{\tau}}
\newcommand{\Bomega}[0]{\boldsymbol{\omega}}

%
% shorthand for unit vectors
%
\newcommand{\acap}[0]{\hat{\Ba}}
\newcommand{\bcap}[0]{\hat{\Bb}}
\newcommand{\ccap}[0]{\hat{\Bc}}
\newcommand{\dcap}[0]{\hat{\Bd}}
\newcommand{\ecap}[0]{\hat{\Be}}
\newcommand{\fcap}[0]{\hat{\Bf}}
\newcommand{\gcap}[0]{\hat{\Bg}}
\newcommand{\hcap}[0]{\hat{\Bh}}
\newcommand{\icap}[0]{\hat{\Bi}}
\newcommand{\jcap}[0]{\hat{\Bj}}
\newcommand{\kcap}[0]{\hat{\Bk}}
\newcommand{\lcap}[0]{\hat{\Bl}}
\newcommand{\mcap}[0]{\hat{\Bm}}
\newcommand{\ncap}[0]{\hat{\Bn}}
\newcommand{\ocap}[0]{\hat{\Bo}}
\newcommand{\pcap}[0]{\hat{\Bp}}
\newcommand{\qcap}[0]{\hat{\Bq}}
\newcommand{\rcap}[0]{\hat{\Br}}
\newcommand{\scap}[0]{\hat{\Bs}}
\newcommand{\tcap}[0]{\hat{\Bt}}
\newcommand{\ucap}[0]{\hat{\Bu}}
\newcommand{\vcap}[0]{\hat{\Bv}}
\newcommand{\wcap}[0]{\hat{\Bw}}
\newcommand{\xcap}[0]{\hat{\Bx}}
\newcommand{\ycap}[0]{\hat{\By}}
\newcommand{\zcap}[0]{\hat{\Bz}}
\newcommand{\thetacap}[0]{\hat{\Btheta}}

%
% to write R^n and C^n in a distinguishable fashion.  Perhaps change this
% to the double lined characters upon figuring out how to do so.
%
\newcommand{\C}[1]{$\mathbb{C}^{#1}$}
\newcommand{\R}[1]{$\mathbb{R}^{#1}$}

%
% various generally useful helpers
%

% derivative of #1 wrt. #2:
\newcommand{\D}[2] {\frac {d#2} {d#1}}

\newcommand{\inv}[1]{\frac{1}{#1}}
\newcommand{\cross}[0]{\times}

\newcommand{\abs}[1]{\lvert{#1}\rvert}
\newcommand{\norm}[1]{\lVert{#1}\rVert}
\newcommand{\innerprod}[2]{\langle{#1}, {#2}\rangle}
\newcommand{\dotprod}[2]{{#1} \cdot {#2}}
\newcommand{\bdotprod}[2]{\left({#1} \cdot {#2}\right)}
\newcommand{\crossprod}[2]{{#1} \cross {#2}}
\newcommand{\tripleprod}[3]{\dotprod{\left(\crossprod{#1}{#2}\right)}{#3}}

\DeclareMathOperator{\Proj}{Proj}
\DeclareMathOperator{\Span}{span}
\DeclareMathOperator{\Sgn}{sgn}
\DeclareMathOperator{\Area}{Area}
\DeclareMathOperator{\Volume}{Volume}

%
% A few miscellaneous things specific to this document
%
\newcommand{\crossop}[1]{\crossprod{#1}{}}

% R2 vector.
\newcommand{\VectorTwo}[2]{
\begin{bmatrix}
 {#1} \\
 {#2}
\end{bmatrix}
}

\newcommand{\VectorN}[1]{
\begin{bmatrix}
{#1}_1 \\
{#1}_2 \\
\vdots \\
{#1}_N \\
\end{bmatrix}
}

\newcommand{\DETuvij}[4]{
\begin{vmatrix}
 {#1}_{#3} & {#1}_{#4} \\
 {#2}_{#3} & {#2}_{#4}
\end{vmatrix}
}

\newcommand{\DETuvwijk}[6]{
\begin{vmatrix}
 {#1}_{#4} & {#1}_{#5} & {#1}_{#6} \\
 {#2}_{#4} & {#2}_{#5} & {#2}_{#6} \\
 {#3}_{#4} & {#3}_{#5} & {#3}_{#6}
\end{vmatrix}
}

\newcommand{\DETuvwxijkl}[8]{
\begin{vmatrix}
 {#1}_{#5} & {#1}_{#6} & {#1}_{#7} & {#1}_{#8} \\
 {#2}_{#5} & {#2}_{#6} & {#2}_{#7} & {#2}_{#8} \\
 {#3}_{#5} & {#3}_{#6} & {#3}_{#7} & {#3}_{#8} \\
 {#4}_{#5} & {#4}_{#6} & {#4}_{#7} & {#4}_{#8} \\
\end{vmatrix}
}

%\newcommand{\DETuvwxyijklm}[10]{
%\begin{vmatrix}
% {#1}_{#6} & {#1}_{#7} & {#1}_{#8} & {#1}_{#9} & {#1}_{#10} \\
% {#2}_{#6} & {#2}_{#7} & {#2}_{#8} & {#2}_{#9} & {#2}_{#10} \\
% {#3}_{#6} & {#3}_{#7} & {#3}_{#8} & {#3}_{#9} & {#3}_{#10} \\
% {#4}_{#6} & {#4}_{#7} & {#4}_{#8} & {#4}_{#9} & {#4}_{#10} \\
% {#5}_{#6} & {#5}_{#7} & {#5}_{#8} & {#5}_{#9} & {#5}_{#10}
%\end{vmatrix}
%}

% R3 vector.
\newcommand{\VectorThree}[3]{
\begin{bmatrix}
 {#1} \\
 {#2} \\
 {#3}
\end{bmatrix}
}


\newcommand{\gpgrade}[2] {{\left\langle{{#1}}\right\rangle}_{#2}}
\newcommand{\gpgradeone}[1] {\gpgrade{#1}{1}}
\newcommand{\gpgradezero}[1] {\gpgrade{#1}{}}
\newcommand{\grad}[0] {\nabla}
\newcommand{\spacegrad}[0]{\boldsymbol{\nabla}}

\title{ Covariant Lagrangian, and electrodynamic potential. }
\author{Peeter Joot}
\date{August 21, 2008}

\begin{document}

\maketitle{}

\section{}

Previously, it was observed that insertion of $F = \grad \wedge A$ into
the covariant form of the Lorentz force:

\begin{equation}\label{eqn:lorentz}
\dot{p} = q (F \cdot v/c)
\end{equation}

allowed this law to be expressed as a gradient equation:

\begin{equation}
\dot{p} = q \grad (A \cdot v/c).
\end{equation}

Now, this suggests the possibility of a covariant potential that could be 
used in a Lagrangian to produce equation \ref{eqn:lorentz} directly.  An
initial incorrect guess at what this Lagrangian would be was done, and
here some better guesses are made as well as a bit of raw algebra to verify
that it works out.

\section{ Guess at the Lagrange equations for relativistic correctness. }

Now, the author does not at the moment know any variational calculus worth
speaking of, but can guess at what the Lagrangian equations that would 
solve the relativistic minimization problem.  Specifically, use proper
time in place of any local time derivatives:

\begin{equation}
\frac{\partial \mathcal{L}}{\partial x^{\mu}} = 
\frac{d}{d\tau} \frac{\partial \mathcal{L}}{\partial \dot{x}^{\mu}}.
\end{equation}

Note that in this equation $\dot{x}^{\mu} = \frac{d x^{\mu}}{d\tau}$.

\subsection{ Try it with a non-velocity dependant potential. }

Lets see if this works as expected, by applying it to the simplest general
kinetic and potential Lagrangian.

\begin{equation}
\mathcal{L} = \inv{2} m v^2 - \phi
\end{equation}

Calculate the lagrangian equations:

\begin{align*}
\frac{\partial \mathcal{L}}{\partial x^{\mu}} &= \frac{d}{d\tau} \frac{\partial \mathcal{L}}{\partial \dot{x}^{\mu}} \\
- \frac{\partial \phi}{\partial x^{\mu}} 
&= \inv{2} m \frac{d}{d\tau} \frac{\partial}{\partial \dot{x}^{\mu}} \sum \gamma_{\alpha} \cdot \gamma_{\beta} \dot{x}^{\alpha} \dot{x}^{\beta} \\
&= \inv{2} m \sum \gamma_{\alpha} \cdot \gamma_{\beta} \frac{d}{d\tau} \left({\delta^{\alpha}}_{\mu} \dot{x}^{\beta} + \dot{x}^{\alpha} {\delta^{\beta}}_{\mu}\right) \\
&= \inv{2} m \sum \frac{d}{d\tau}
\left(\gamma_{\mu} \cdot \gamma_{\beta} \dot{x}^{\beta} + \gamma_{\alpha} \cdot \gamma_{\mu} \dot{x}^{\alpha} \right) \\
&= m \sum \frac{d}{d\tau} \gamma_{\mu} \cdot \gamma_{\alpha} \dot{x}^{\alpha} \\
&= m \sum \gamma_{\mu} \cdot \gamma_{\alpha} \ddot{x}^{\alpha} \\
\end{align*}

Now, as in the Newtonian case, where we could show the correct form of the gradient for non-orthonormal frames could be derived from the lagrangian equations using appropriate reciprocal vector pairs, we do the same thing here, summing the product of this last result with the reciprocal frame vectors:

\begin{align*}
\sum \gamma^{\mu} \left(- \frac{\partial \phi}{\partial x^{\mu}}\right) &= \sum \gamma^{\mu} \left(m \gamma_{\mu} \cdot \gamma_{\alpha} \ddot{x}^{\alpha}\right) \\
- \left(\sum \gamma^{\mu} \frac{\partial}{\partial x^{\mu}}\right) \phi 
&= m \sum \gamma_{\alpha} \ddot{x}^{\alpha} \\
&= m \ddot{x}
\end{align*}

Now, this left hand operator quantity is exactly our spacetime gradient:

\begin{equation}
\grad = \sum \gamma^{\mu} \frac{\partial}{\partial x^{\mu}},
\end{equation}

and the right hand side is our proper momentum.  Therefore the result of following through with the assumed Lagrangian equations yield the expected result:

\begin{equation}
\dot{p} = -\grad \phi.
\end{equation}

Additionally, this demonstrates that the spacetime gradient used in GAFP is appropriate for any spacetime basis, regardless of whether the chosen basis vectors are othnonormal.

An interesting feature here is that there is also no requirement for a mixed signature metric for spacetime in any of this.  That has to come another source.

\subsection{ Velocity dependent potential. }

\section{ Appendix.  Omitted details. }

\subsection{ calculation of reciprocal frame vector over sum. }

show:

\begin{equation*}
\gamma_{\alpha} = \sum \gamma^{\mu} \gamma_{\mu} \cdot \gamma_{\alpha}
\end{equation*}

\end{document}
