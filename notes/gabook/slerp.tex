% 
% 
% 
% Copyright � 2012 Peeter Joot
% All Rights Reserved
% 
% This file may be reproduced and distributed in whole or in part, without fee, subject to the following conditions:
% 
% o The copyright notice above and this permission notice must be preserved complete on all complete or partial copies.
% 
% o Any translation or derived work must be approved by the author in writing before distribution.
% 
% o If you distribute this work in part, instructions for obtaining the complete version of this file must be included, and a means for obtaining a complete version provided.
% 
% 
% Exceptions to these rules may be granted for academic purposes: Write to the author and ask.
% 
% 
% 
\chapter{Rotor interpolation calculation.}
\label{chap:slerp}
\date{Nov 30, 2008.  slerp.tex}

The aim is to compute the interpolating rotor $r$ that takes an object
from one position to another in $n$ steps.
Here the initial and final positions are given by two rotors $R_1$, and $R_2$
like so

\begin{align*}
X_1 &= R_1 X {R_1}^\dagger \\
X_2 &= R_2 X {R_2}^\dagger = r^n R_1 X {R_1}^\dagger {r^n}^\dagger
\end{align*}

So, writing 

\begin{align*}
%r^n R_1 = R_2 
a = r^n = R_2 \inv{R_1} = \frac{R_2 {R_1}^\dagger}{R_1 {{R_1}^\dagger}} = \cos\theta + I \sin\theta
\end{align*}

So, 

\begin{align*}
\frac{\gpgradetwo{a}}{\gpgradezero{a}} &= 
\frac{\gpgradetwo{a}}{\Abs{\gpgradetwo{a}}} \frac{\Abs{\gpgradetwo{a}}}{\gpgradezero{a}} \\
&= I \tan\theta
\end{align*}

Therefore the interpolating rotor is:
\begin{align*}
I &= \frac{\gpgradetwo{a}}{\Abs{\gpgradetwo{a}}} \\
\theta &= \atan2\left(\Abs{\gpgradetwo{a}}, \gpgradezero{a}\right) \\
r &= \cos(\theta/n) + I \sin(\theta/n)
\end{align*}

In \citep{dorst2007gac}, equation $10.15$, they've got something like this
for a fractional angle, but then say that they don't use that in software, 
instead using $r$ directly, with a comment about designing more sophisticated
algorithms (bivector splines).  That spline comment in particular sounds
interesting.  Sounds like the details on that are to be found in the journals
mentioned in Further Reading section of chapter 10.
