\documentclass{article}      % Specifies the document class

\usepackage{amsmath}
\usepackage{mathpazo}

%
% shorthand for bold symbols, convenient for vectors and matrices
%
\newcommand{\Ba}[0]{\mathbf{a}}
\newcommand{\Bb}[0]{\mathbf{b}}
\newcommand{\Bc}[0]{\mathbf{c}}
\newcommand{\Bd}[0]{\mathbf{d}}
\newcommand{\Be}[0]{\mathbf{e}}
\newcommand{\Bf}[0]{\mathbf{f}}
\newcommand{\Bg}[0]{\mathbf{g}}
\newcommand{\Bh}[0]{\mathbf{h}}
\newcommand{\Bi}[0]{\mathbf{i}}
\newcommand{\Bj}[0]{\mathbf{j}}
\newcommand{\Bk}[0]{\mathbf{k}}
\newcommand{\Bl}[0]{\mathbf{l}}
\newcommand{\Bm}[0]{\mathbf{m}}
\newcommand{\Bn}[0]{\mathbf{n}}
\newcommand{\Bo}[0]{\mathbf{o}}
\newcommand{\Bp}[0]{\mathbf{p}}
\newcommand{\Bq}[0]{\mathbf{q}}
\newcommand{\Br}[0]{\mathbf{r}}
\newcommand{\Bs}[0]{\mathbf{s}}
\newcommand{\Bt}[0]{\mathbf{t}}
\newcommand{\Bu}[0]{\mathbf{u}}
\newcommand{\Bv}[0]{\mathbf{v}}
\newcommand{\Bw}[0]{\mathbf{w}}
\newcommand{\Bx}[0]{\mathbf{x}}
\newcommand{\By}[0]{\mathbf{y}}
\newcommand{\Bz}[0]{\mathbf{z}}
\newcommand{\BA}[0]{\mathbf{A}}
\newcommand{\BB}[0]{\mathbf{B}}
\newcommand{\BC}[0]{\mathbf{C}}
\newcommand{\BD}[0]{\mathbf{D}}
\newcommand{\BE}[0]{\mathbf{E}}
\newcommand{\BF}[0]{\mathbf{F}}
\newcommand{\BG}[0]{\mathbf{G}}
\newcommand{\BH}[0]{\mathbf{H}}
\newcommand{\BI}[0]{\mathbf{I}}
\newcommand{\BJ}[0]{\mathbf{J}}
\newcommand{\BK}[0]{\mathbf{K}}
\newcommand{\BL}[0]{\mathbf{L}}
\newcommand{\BM}[0]{\mathbf{M}}
\newcommand{\BN}[0]{\mathbf{N}}
\newcommand{\BO}[0]{\mathbf{O}}
\newcommand{\BP}[0]{\mathbf{P}}
\newcommand{\BQ}[0]{\mathbf{Q}}
\newcommand{\BR}[0]{\mathbf{R}}
\newcommand{\BS}[0]{\mathbf{S}}
\newcommand{\BT}[0]{\mathbf{T}}
\newcommand{\BU}[0]{\mathbf{U}}
\newcommand{\BV}[0]{\mathbf{V}}
\newcommand{\BW}[0]{\mathbf{W}}
\newcommand{\BX}[0]{\mathbf{X}}
\newcommand{\BY}[0]{\mathbf{Y}}
\newcommand{\BZ}[0]{\mathbf{Z}}

\newcommand{\Bzero}[0]{\mathbf{0}}
\newcommand{\Btheta}[0]{\boldsymbol{\theta}}
\newcommand{\Btau}[0]{\boldsymbol{\tau}}
\newcommand{\Bomega}[0]{\boldsymbol{\omega}}

%
% shorthand for unit vectors
%
\newcommand{\acap}[0]{\hat{\Ba}}
\newcommand{\bcap}[0]{\hat{\Bb}}
\newcommand{\ccap}[0]{\hat{\Bc}}
\newcommand{\dcap}[0]{\hat{\Bd}}
\newcommand{\ecap}[0]{\hat{\Be}}
\newcommand{\fcap}[0]{\hat{\Bf}}
\newcommand{\gcap}[0]{\hat{\Bg}}
\newcommand{\hcap}[0]{\hat{\Bh}}
\newcommand{\icap}[0]{\hat{\Bi}}
\newcommand{\jcap}[0]{\hat{\Bj}}
\newcommand{\kcap}[0]{\hat{\Bk}}
\newcommand{\lcap}[0]{\hat{\Bl}}
\newcommand{\mcap}[0]{\hat{\Bm}}
\newcommand{\ncap}[0]{\hat{\Bn}}
\newcommand{\ocap}[0]{\hat{\Bo}}
\newcommand{\pcap}[0]{\hat{\Bp}}
\newcommand{\qcap}[0]{\hat{\Bq}}
\newcommand{\rcap}[0]{\hat{\Br}}
\newcommand{\scap}[0]{\hat{\Bs}}
\newcommand{\tcap}[0]{\hat{\Bt}}
\newcommand{\ucap}[0]{\hat{\Bu}}
\newcommand{\vcap}[0]{\hat{\Bv}}
\newcommand{\wcap}[0]{\hat{\Bw}}
\newcommand{\xcap}[0]{\hat{\Bx}}
\newcommand{\ycap}[0]{\hat{\By}}
\newcommand{\zcap}[0]{\hat{\Bz}}
\newcommand{\thetacap}[0]{\hat{\Btheta}}

%
% to write R^n and C^n in a distinguishable fashion.  Perhaps change this
% to the double lined characters upon figuring out how to do so.
%
\newcommand{\C}[1]{$\mathbb{C}^{#1}$}
\newcommand{\R}[1]{$\mathbb{R}^{#1}$}

%
% various generally useful helpers
%

% derivative of #1 wrt. #2:
\newcommand{\D}[2] {\frac {d#2} {d#1}}

\newcommand{\inv}[1]{\frac{1}{#1}}
\newcommand{\cross}[0]{\times}

\newcommand{\abs}[1]{\lvert{#1}\rvert}
\newcommand{\norm}[1]{\lVert{#1}\rVert}
\newcommand{\innerprod}[2]{\langle{#1}, {#2}\rangle}
\newcommand{\dotprod}[2]{{#1} \cdot {#2}}
\newcommand{\bdotprod}[2]{\left({#1} \cdot {#2}\right)}
\newcommand{\crossprod}[2]{{#1} \cross {#2}}
\newcommand{\tripleprod}[3]{\dotprod{\left(\crossprod{#1}{#2}\right)}{#3}}

\DeclareMathOperator{\Proj}{Proj}
\DeclareMathOperator{\Span}{span}
\DeclareMathOperator{\Sgn}{sgn}
\DeclareMathOperator{\Area}{Area}
\DeclareMathOperator{\Volume}{Volume}

%
% A few miscellaneous things specific to this document
%
\newcommand{\crossop}[1]{\crossprod{#1}{}}

% R2 vector.
\newcommand{\VectorTwo}[2]{
\begin{bmatrix}
 {#1} \\
 {#2}
\end{bmatrix}
}

\newcommand{\VectorN}[1]{
\begin{bmatrix}
{#1}_1 \\
{#1}_2 \\
\vdots \\
{#1}_N \\
\end{bmatrix}
}

\newcommand{\DETuvij}[4]{
\begin{vmatrix}
 {#1}_{#3} & {#1}_{#4} \\
 {#2}_{#3} & {#2}_{#4}
\end{vmatrix}
}

\newcommand{\DETuvwijk}[6]{
\begin{vmatrix}
 {#1}_{#4} & {#1}_{#5} & {#1}_{#6} \\
 {#2}_{#4} & {#2}_{#5} & {#2}_{#6} \\
 {#3}_{#4} & {#3}_{#5} & {#3}_{#6}
\end{vmatrix}
}

\newcommand{\DETuvwxijkl}[8]{
\begin{vmatrix}
 {#1}_{#5} & {#1}_{#6} & {#1}_{#7} & {#1}_{#8} \\
 {#2}_{#5} & {#2}_{#6} & {#2}_{#7} & {#2}_{#8} \\
 {#3}_{#5} & {#3}_{#6} & {#3}_{#7} & {#3}_{#8} \\
 {#4}_{#5} & {#4}_{#6} & {#4}_{#7} & {#4}_{#8} \\
\end{vmatrix}
}

%\newcommand{\DETuvwxyijklm}[10]{
%\begin{vmatrix}
% {#1}_{#6} & {#1}_{#7} & {#1}_{#8} & {#1}_{#9} & {#1}_{#10} \\
% {#2}_{#6} & {#2}_{#7} & {#2}_{#8} & {#2}_{#9} & {#2}_{#10} \\
% {#3}_{#6} & {#3}_{#7} & {#3}_{#8} & {#3}_{#9} & {#3}_{#10} \\
% {#4}_{#6} & {#4}_{#7} & {#4}_{#8} & {#4}_{#9} & {#4}_{#10} \\
% {#5}_{#6} & {#5}_{#7} & {#5}_{#8} & {#5}_{#9} & {#5}_{#10}
%\end{vmatrix}
%}

% R3 vector.
\newcommand{\VectorThree}[3]{
\begin{bmatrix}
 {#1} \\
 {#2} \\
 {#3}
\end{bmatrix}
}


\newcommand{\grad}[0]{\nabla}

%
% The real thing:
%

                             % The preamble begins here.
\title{} % Declares the document's title.
\author{Peeter Joot}         % Declares the author's name.
\date{ July 12, 2008 }        % Deleting this command produces today's date.

\begin{document}             % End of preamble and beginning of text.

\maketitle{}

\section{ Back to Maxwell's equations }

Having observed and demonstrated that the Lorentz transformation is a natural consequence of requiring the electromagnetic wave equation retains the
form of the wave equation under change of space and time variables that includes a velocity change in one spacial direction.

Lets step back and look at Maxwell's equations to see where the wave equation comes from.  We start with the equations in SI units:

\begin{align*}
\int_{S(\text{closed boundary of V})} \BE \cdot \ncap dA &= \inv{\epsilon_0} \int_V \rho dV \\
\int_{S(\text{any closed surface})} \BB \cdot \ncap dA &= 0 \\
\int_{C(\text{boundary of S})} \BE \cdot d\Bx &= - \int_{S} \frac{\partial \BB}{\partial t} \cdot \ncap dA \\
\int_{C(\text{boundary of S})} \BB \cdot d\Bx &= \mu_0\left(I + \epsilon_0\int_{S} \frac{\partial \BE}{\partial t} \cdot \ncap dA\right) \\
\end{align*}

As the surfaces and corresponding loops or volumes are made infinitely small, these equations (FIXME: demonstrate), can be written in differential form:

\begin{align*}
\grad \cdot \BE &= \frac{\rho}{\epsilon_0} \\
\grad \cdot \BB &= 0 \\
\grad \cross \BE &= - \frac{\partial \BB}{\partial t} \\
\grad \cross \BB &= \mu_0\left(\BJ + \epsilon_0 \frac{\partial \BE}{\partial t}\right) \\
\end{align*}

These are respectively, Gauss's Law for E, Gauss's Law for B, Faraday's Law, and the Ampere/Maxwell's Law.

This differential form can be manipulated to derive the wave equation for free space, or the wave equation with charge and current forcing terms in other space.

\subsection{ Regrouping terms for dimensional consistency. }

Derivation of the wave equation can be done nicely using geometric algebra, but first is it helpful to put these equations in a more dimensionally pleasant form.
Lets relate the dimensions of the electric and magnetic fields and the constants $\mu_0, \epsilon_0$.

From Faraday's equation we can relate the dimensions of
$\BB$, and $\BE$:

\begin{equation}
\frac{[\BE]}{[d]} = \frac{[\BB]}{[t]}
\end{equation}

We therefore see that $\BB$, and $\BE$ are related dimensionally by a velocity factor.

Looking at the dimensions of the displacement current density in the Ampere/Maxwell equation we see:

\begin{equation}
\frac{[\BB]}{[d]} = [\mu_0\epsilon_0] \frac{[\BE]}{[t]}
\end{equation}

From the two of these the dimensions of the $\mu_0\epsilon_0$ product can be seen to be:

\begin{equation}
[\mu_0\epsilon_0] = \frac{{[t]}^2}{{[d]}^2}
\end{equation}

So, we see that we have a velocity factor relating $\BE$, and $\BB$, and we also see that we have a squared velocity coefficient in Ampere/Maxwell's law.  Let's factor this out explicitly so that $\BE$ and $\BB$ take dimensionally consistent form:

\begin{align}
\tau &= \frac{t}{\sqrt{\mu_0\epsilon_0}}  \\
\grad \cdot \BE &= \frac{\rho}{\epsilon_0} \\
\grad \cdot \frac{\BB}{\sqrt{\mu_0\epsilon_0}} &= 0 \\
\grad \cross \BE &= - \frac{\partial}{\partial \tau} \frac{\BB}{\sqrt{\mu_0\epsilon_0}} \\
\grad \cross \frac{\BB}{\sqrt{\mu_0\epsilon_0}} &= \sqrt{\frac{\mu_0}{\epsilon_0}} \BJ + \frac{\partial \BE}{\partial \tau}
\end{align}

\subsection{ Refactoring the equations with the geometric product. }

Now that things are dimensionally consistent, we are ready to group these equations using the geometric product

\begin{equation}
\BA \BB = \BA \cdot \BB + \BA \wedge \BB = \BA \cdot \BB + i \BA \cross \BB
\end{equation}

where $i = \Be_1\Be_2\Be_3$ is the spatial pseudoscalar.  By grouping the divergence and curl terms for each of $\BB$, and $\BE$ we can write vector gradient equations
for each of the Electric and Magnetic fields:

\begin{align}
\grad \BE = \frac{\rho}{\epsilon_0} - i \frac{\partial}{\partial \tau} \frac{\BB}{\sqrt{\mu_0\epsilon_0}} \label{eqn:grad_e} \\
\grad \frac{\BB}{\sqrt{\mu_0\epsilon_0}} = i\sqrt{\frac{\mu_0}{\epsilon_0}} \BJ + i\frac{\partial \BE}{\partial \tau} \label{eqn:grad_b}
\end{align}

Multiplication of equation \ref{eqn:grad_b} with $i$, and adding to \ref{eqn:grad_e}, we have Maxwell's equations consolidated into:

\begin{equation}
\grad \left(\BE + i \frac{\BB}{\sqrt{\mu_0\epsilon_0}}\right) =
\left(\frac{\rho}{\epsilon_0} - \sqrt{\frac{\mu_0}{\epsilon_0}} \BJ\right)
- \frac{\partial}{\partial \tau} \left(\BE + \frac{i\BB}{\sqrt{\mu_0\epsilon_0}} \right)
\end{equation}

We see that we have a natural combined Electrodynamic field:

\begin{equation}
\BF = \BE + i \frac{\BB}{\sqrt{\mu_0\epsilon_0}},
\end{equation}

and can use this to write Maxwell's equations as:

\begin{equation}
\left(\grad + \sqrt{\mu_0\epsilon_0}\frac{\partial}{\partial t} \right) \BF = \frac{\rho}{\epsilon_0} - \sqrt{\frac{\mu_0}{\epsilon_0}} \BJ.  \label{eqn:maxwell}
\end{equation}

These are still four equations, and the originals can be recovered by taking scalar, vector, bivector and trivector parts.  However, in this
consolidated form, we are able to see the structure more easily.

\subsection{ Wave equation for light. }

In particular can do some operations, such as deriving the wave equation
with special ease.

Let's do just that, by taking the gradient of equation \ref{eqn:maxwell}:

\begin{equation*}
\grad^2 \BF + \sqrt{\mu_0\epsilon_0}\grad \frac{\partial}{\partial t} \BF =
\grad \left(\frac{\rho}{\epsilon_0} - \sqrt{\frac{\mu_0}{\epsilon_0}} \BJ\right)
\end{equation*}

%\grad \BF = - \sqrt{\mu_0\epsilon_0}\frac{\partial \BF}{\partial t} + \left(\frac{\rho}{\epsilon_0} - \sqrt{\frac{\mu_0}{\epsilon_0}} \BJ \right)

Assuming continuity suffienct for mixed partial equality, this is

\begin{align*}
\grad^2 \BF
&= - \sqrt{\mu_0\epsilon_0}\frac{\partial}{\partial t} \grad \BF + \grad \left(\frac{\rho}{\epsilon_0} - \sqrt{\frac{\mu_0}{\epsilon_0}} \BJ\right) \\
&= + \sqrt{\mu_0\epsilon_0}\frac{\partial}{\partial t}
\left(\sqrt{\mu_0\epsilon_0}\frac{\partial \BF}{\partial t} + \left(\frac{\rho}{\epsilon_0} - \sqrt{\frac{\mu_0}{\epsilon_0}} \BJ \right)\right)
 + \grad \left(\frac{\rho}{\epsilon_0} - \sqrt{\frac{\mu_0}{\epsilon_0}} \BJ\right) \\
\end{align*}

Or,

\begin{equation}
\left(\grad^2 - \mu_0\epsilon_0 \frac{\partial^2}{\partial t^2} \right) \BF =
\left(\grad + \sqrt{\mu_0\epsilon_0}\frac{\partial}{\partial t}\right)
\left(\frac{\rho}{\epsilon_0} - \sqrt{\frac{\mu_0}{\epsilon_0}} \BJ \right)
\end{equation}

Now there are a number of things that can be read out of this equation.  The first is that in a charge and current free region the electromagnetic field is described by the
wave equation:

\begin{equation}
\left(\grad^2 - \mu_0\epsilon_0 \frac{\partial^2}{\partial t^2} \right) \BF = 0.
\end{equation}

Dimensional analysis told us that $1/{\sqrt{\mu_0\epsilon_0}}$ had dimensions of velocity.  We now have a specific meaning for it, namely the
wave velocity $c$ for an electromagnetic wave in free space:

\begin{equation}
c = \inv{\sqrt{\mu_0\epsilon_0}}
\end{equation}

We can utilize this to tidy up many of the relations above, replacing $\mu_0$ with $\epsilon_0$ and $c$ since these three constants are dependent.

\begin{equation}
\BF = \BE + i c \BB
\end{equation}

\begin{equation}
\left(\grad + \partial_{ct}\right) \BF =
\inv{c \epsilon_0}\left( c \rho - \BJ \right)
\end{equation}

\begin{equation} \label{eqn:wave}
\left(\grad^2 - \partial_{ct, ct}\right) \BF =
\left(\grad + \partial_{c t}\right) \inv{c \epsilon_0}\left( c \rho - \BJ \right)
\end{equation}

Now, the left hand side of equation \ref{eqn:wave} has only vector and bivector parts.  This implies that the scalar components of the right hand side are zero.  Specifically:

\begin{equation*}
\partial_t \rho - \grad \cdot \BJ = 0
\end{equation*}

This is a statement of charge conservation, and is more easily interpretted in integral form:
%\partial_t \rho = \grad \cdot \BJ

\begin{equation}
\int_{S(\text{closed boundary of V})} \BJ \cdot \ncap dA = \frac{\partial}{\partial t} \int_V \rho dV = \frac{\partial Q_{enc}}{\partial t}
\end{equation}

The flux of the current density vector through a closed surface equals the time rate of change of the charge enclosed by that volume (ie: the current).  This could perhaps be viewed as the definition of the current density itself.  This fact would probably be more obvious if I did the math myself to demonstate exactly how to take Maxwells equations in integral form and convert those to their differential form.  In liew of having done that proof myself I can at least determine this as a side effect of a bit of math.

\end{document}               % End of document.
