\documentclass{article}

\usepackage{amsmath}
\usepackage{mathpazo}

%
% shorthand for bold symbols, convenient for vectors and matrices
%
\newcommand{\Ba}[0]{\mathbf{a}}
\newcommand{\Bb}[0]{\mathbf{b}}
\newcommand{\Bc}[0]{\mathbf{c}}
\newcommand{\Bd}[0]{\mathbf{d}}
\newcommand{\Be}[0]{\mathbf{e}}
\newcommand{\Bf}[0]{\mathbf{f}}
\newcommand{\Bg}[0]{\mathbf{g}}
\newcommand{\Bh}[0]{\mathbf{h}}
\newcommand{\Bi}[0]{\mathbf{i}}
\newcommand{\Bj}[0]{\mathbf{j}}
\newcommand{\Bk}[0]{\mathbf{k}}
\newcommand{\Bl}[0]{\mathbf{l}}
\newcommand{\Bm}[0]{\mathbf{m}}
\newcommand{\Bn}[0]{\mathbf{n}}
\newcommand{\Bo}[0]{\mathbf{o}}
\newcommand{\Bp}[0]{\mathbf{p}}
\newcommand{\Bq}[0]{\mathbf{q}}
\newcommand{\Br}[0]{\mathbf{r}}
\newcommand{\Bs}[0]{\mathbf{s}}
\newcommand{\Bt}[0]{\mathbf{t}}
\newcommand{\Bu}[0]{\mathbf{u}}
\newcommand{\Bv}[0]{\mathbf{v}}
\newcommand{\Bw}[0]{\mathbf{w}}
\newcommand{\Bx}[0]{\mathbf{x}}
\newcommand{\By}[0]{\mathbf{y}}
\newcommand{\Bz}[0]{\mathbf{z}}
\newcommand{\BA}[0]{\mathbf{A}}
\newcommand{\BB}[0]{\mathbf{B}}
\newcommand{\BC}[0]{\mathbf{C}}
\newcommand{\BD}[0]{\mathbf{D}}
\newcommand{\BE}[0]{\mathbf{E}}
\newcommand{\BF}[0]{\mathbf{F}}
\newcommand{\BG}[0]{\mathbf{G}}
\newcommand{\BH}[0]{\mathbf{H}}
\newcommand{\BI}[0]{\mathbf{I}}
\newcommand{\BJ}[0]{\mathbf{J}}
\newcommand{\BK}[0]{\mathbf{K}}
\newcommand{\BL}[0]{\mathbf{L}}
\newcommand{\BM}[0]{\mathbf{M}}
\newcommand{\BN}[0]{\mathbf{N}}
\newcommand{\BO}[0]{\mathbf{O}}
\newcommand{\BP}[0]{\mathbf{P}}
\newcommand{\BQ}[0]{\mathbf{Q}}
\newcommand{\BR}[0]{\mathbf{R}}
\newcommand{\BS}[0]{\mathbf{S}}
\newcommand{\BT}[0]{\mathbf{T}}
\newcommand{\BU}[0]{\mathbf{U}}
\newcommand{\BV}[0]{\mathbf{V}}
\newcommand{\BW}[0]{\mathbf{W}}
\newcommand{\BX}[0]{\mathbf{X}}
\newcommand{\BY}[0]{\mathbf{Y}}
\newcommand{\BZ}[0]{\mathbf{Z}}

\newcommand{\Bzero}[0]{\mathbf{0}}
\newcommand{\Btheta}[0]{\boldsymbol{\theta}}
\newcommand{\Btau}[0]{\boldsymbol{\tau}}
\newcommand{\Bomega}[0]{\boldsymbol{\omega}}

%
% shorthand for unit vectors
%
\newcommand{\acap}[0]{\hat{\Ba}}
\newcommand{\bcap}[0]{\hat{\Bb}}
\newcommand{\ccap}[0]{\hat{\Bc}}
\newcommand{\dcap}[0]{\hat{\Bd}}
\newcommand{\ecap}[0]{\hat{\Be}}
\newcommand{\fcap}[0]{\hat{\Bf}}
\newcommand{\gcap}[0]{\hat{\Bg}}
\newcommand{\hcap}[0]{\hat{\Bh}}
\newcommand{\icap}[0]{\hat{\Bi}}
\newcommand{\jcap}[0]{\hat{\Bj}}
\newcommand{\kcap}[0]{\hat{\Bk}}
\newcommand{\lcap}[0]{\hat{\Bl}}
\newcommand{\mcap}[0]{\hat{\Bm}}
\newcommand{\ncap}[0]{\hat{\Bn}}
\newcommand{\ocap}[0]{\hat{\Bo}}
\newcommand{\pcap}[0]{\hat{\Bp}}
\newcommand{\qcap}[0]{\hat{\Bq}}
\newcommand{\rcap}[0]{\hat{\Br}}
\newcommand{\scap}[0]{\hat{\Bs}}
\newcommand{\tcap}[0]{\hat{\Bt}}
\newcommand{\ucap}[0]{\hat{\Bu}}
\newcommand{\vcap}[0]{\hat{\Bv}}
\newcommand{\wcap}[0]{\hat{\Bw}}
\newcommand{\xcap}[0]{\hat{\Bx}}
\newcommand{\ycap}[0]{\hat{\By}}
\newcommand{\zcap}[0]{\hat{\Bz}}
\newcommand{\thetacap}[0]{\hat{\Btheta}}

%
% to write R^n and C^n in a distinguishable fashion.  Perhaps change this
% to the double lined characters upon figuring out how to do so.
%
\newcommand{\C}[1]{$\mathbb{C}^{#1}$}
\newcommand{\R}[1]{$\mathbb{R}^{#1}$}

%
% various generally useful helpers
%

% derivative of #1 wrt. #2:
\newcommand{\D}[2] {\frac {d#2} {d#1}}

\newcommand{\inv}[1]{\frac{1}{#1}}
\newcommand{\cross}[0]{\times}

\newcommand{\abs}[1]{\lvert{#1}\rvert}
\newcommand{\norm}[1]{\lVert{#1}\rVert}
\newcommand{\innerprod}[2]{\langle{#1}, {#2}\rangle}
\newcommand{\dotprod}[2]{{#1} \cdot {#2}}
\newcommand{\bdotprod}[2]{\left({#1} \cdot {#2}\right)}
\newcommand{\crossprod}[2]{{#1} \cross {#2}}
\newcommand{\tripleprod}[3]{\dotprod{\left(\crossprod{#1}{#2}\right)}{#3}}

\DeclareMathOperator{\Proj}{Proj}
\DeclareMathOperator{\Span}{span}
\DeclareMathOperator{\Sgn}{sgn}
\DeclareMathOperator{\Area}{Area}
\DeclareMathOperator{\Volume}{Volume}

%
% A few miscellaneous things specific to this document
%
\newcommand{\crossop}[1]{\crossprod{#1}{}}

% R2 vector.
\newcommand{\VectorTwo}[2]{
\begin{bmatrix}
 {#1} \\
 {#2}
\end{bmatrix}
}

\newcommand{\VectorN}[1]{
\begin{bmatrix}
{#1}_1 \\
{#1}_2 \\
\vdots \\
{#1}_N \\
\end{bmatrix}
}

\newcommand{\DETuvij}[4]{
\begin{vmatrix}
 {#1}_{#3} & {#1}_{#4} \\
 {#2}_{#3} & {#2}_{#4}
\end{vmatrix}
}

\newcommand{\DETuvwijk}[6]{
\begin{vmatrix}
 {#1}_{#4} & {#1}_{#5} & {#1}_{#6} \\
 {#2}_{#4} & {#2}_{#5} & {#2}_{#6} \\
 {#3}_{#4} & {#3}_{#5} & {#3}_{#6}
\end{vmatrix}
}

\newcommand{\DETuvwxijkl}[8]{
\begin{vmatrix}
 {#1}_{#5} & {#1}_{#6} & {#1}_{#7} & {#1}_{#8} \\
 {#2}_{#5} & {#2}_{#6} & {#2}_{#7} & {#2}_{#8} \\
 {#3}_{#5} & {#3}_{#6} & {#3}_{#7} & {#3}_{#8} \\
 {#4}_{#5} & {#4}_{#6} & {#4}_{#7} & {#4}_{#8} \\
\end{vmatrix}
}

%\newcommand{\DETuvwxyijklm}[10]{
%\begin{vmatrix}
% {#1}_{#6} & {#1}_{#7} & {#1}_{#8} & {#1}_{#9} & {#1}_{#10} \\
% {#2}_{#6} & {#2}_{#7} & {#2}_{#8} & {#2}_{#9} & {#2}_{#10} \\
% {#3}_{#6} & {#3}_{#7} & {#3}_{#8} & {#3}_{#9} & {#3}_{#10} \\
% {#4}_{#6} & {#4}_{#7} & {#4}_{#8} & {#4}_{#9} & {#4}_{#10} \\
% {#5}_{#6} & {#5}_{#7} & {#5}_{#8} & {#5}_{#9} & {#5}_{#10}
%\end{vmatrix}
%}

% R3 vector.
\newcommand{\VectorThree}[3]{
\begin{bmatrix}
 {#1} \\
 {#2} \\
 {#3}
\end{bmatrix}
}


%<misc>
%
\newcommand{\Abs}[1]{{\left\lvert{#1}\right\rvert}}
\newcommand{\spacegrad}[0]{\boldsymbol{\nabla}}
\newcommand{\grad}[0]{\nabla}
\newcommand{\LL}[0]{\mathcal{L}}

% == \partial_{#1} {#2}
\newcommand{\PD}[2]{\frac{\partial {#2}}{\partial {#1}}}
% inline variant
\newcommand{\PDi}[2]{{\partial {#2}}/{\partial {#1}}}

\newcommand{\PDD}[3]{\frac{\partial^2 {#3}}{\partial {#1}\partial {#2}}}
%\newcommand{\PDd}[2]{\frac{\partial^2 {#2}}{{\partial{#1}}^2}}
\newcommand{\PDsq}[2]{\frac{\partial^2 {#2}}{(\partial {#1})^2}}

\newcommand{\Partial}[2]{\frac{\partial {#1}}{\partial {#2}}}
\DeclareMathOperator{\RejName}{Rej}
\newcommand{\Rej}[2]{\RejName_{#1}\left( {#2} \right)}
\newcommand{\Rm}[1]{\mathbb{R}^{#1}}
\newcommand{\Cm}[1]{\mathbb{C}^{#1}}
\newcommand{\conj}[0]{{*}}

%</misc>

% <grade selection>
%
\newcommand{\gpgrade}[2] {{\left\langle{{#1}}\right\rangle}_{#2}}

\newcommand{\gpgradezero}[1] {\gpgrade{#1}{}}
%\newcommand{\gpscalargrade}[1] {{\left\langle{{#1}}\right\rangle}}
%\newcommand{\gpgradezero}[1] {\gpgrade{#1}{0}}

%\newcommand{\gpgradeone}[1] {{\left\langle{{#1}}\right\rangle}_{1}}
\newcommand{\gpgradeone}[1] {\gpgrade{#1}{1}}

\newcommand{\gpgradetwo}[1] {\gpgrade{#1}{2}}
\newcommand{\gpgradethree}[1] {\gpgrade{#1}{3}}
\newcommand{\gpgradefour}[1] {\gpgrade{#1}{4}}
%
% </grade selection>



\newcommand{\adot}[0]{{\dot{a}}}
\newcommand{\bdot}[0]{{\dot{b}}}
% taken for centered dot:
%\newcommand{\cdot}[0]{{\dot{c}}}
%\newcommand{\ddot}[0]{{\dot{d}}}
\newcommand{\edot}[0]{{\dot{e}}}
\newcommand{\fdot}[0]{{\dot{f}}}
\newcommand{\gdot}[0]{{\dot{g}}}
\newcommand{\hdot}[0]{{\dot{h}}}
\newcommand{\idot}[0]{{\dot{i}}}
\newcommand{\jdot}[0]{{\dot{j}}}
\newcommand{\kdot}[0]{{\dot{k}}}
\newcommand{\ldot}[0]{{\dot{l}}}
\newcommand{\mdot}[0]{{\dot{m}}}
\newcommand{\ndot}[0]{{\dot{n}}}
%\newcommand{\odot}[0]{{\dot{o}}}
\newcommand{\pdot}[0]{{\dot{p}}}
\newcommand{\qdot}[0]{{\dot{q}}}
\newcommand{\rdot}[0]{{\dot{r}}}
\newcommand{\sdot}[0]{{\dot{s}}}
\newcommand{\tdot}[0]{{\dot{t}}}
\newcommand{\udot}[0]{{\dot{u}}}
\newcommand{\vdot}[0]{{\dot{v}}}
\newcommand{\wdot}[0]{{\dot{w}}}
\newcommand{\xdot}[0]{{\dot{x}}}
\newcommand{\ydot}[0]{{\dot{y}}}
\newcommand{\zdot}[0]{{\dot{z}}}
\newcommand{\addot}[0]{{\ddot{a}}}
\newcommand{\bddot}[0]{{\ddot{b}}}
\newcommand{\cddot}[0]{{\ddot{c}}}
%\newcommand{\dddot}[0]{{\ddot{d}}}
\newcommand{\eddot}[0]{{\ddot{e}}}
\newcommand{\fddot}[0]{{\ddot{f}}}
\newcommand{\gddot}[0]{{\ddot{g}}}
\newcommand{\hddot}[0]{{\ddot{h}}}
\newcommand{\iddot}[0]{{\ddot{i}}}
\newcommand{\jddot}[0]{{\ddot{j}}}
\newcommand{\kddot}[0]{{\ddot{k}}}
\newcommand{\lddot}[0]{{\ddot{l}}}
\newcommand{\mddot}[0]{{\ddot{m}}}
\newcommand{\nddot}[0]{{\ddot{n}}}
\newcommand{\oddot}[0]{{\ddot{o}}}
\newcommand{\pddot}[0]{{\ddot{p}}}
\newcommand{\qddot}[0]{{\ddot{q}}}
\newcommand{\rddot}[0]{{\ddot{r}}}
\newcommand{\sddot}[0]{{\ddot{s}}}
\newcommand{\tddot}[0]{{\ddot{t}}}
\newcommand{\uddot}[0]{{\ddot{u}}}
\newcommand{\vddot}[0]{{\ddot{v}}}
\newcommand{\wddot}[0]{{\ddot{w}}}
\newcommand{\xddot}[0]{{\ddot{x}}}
\newcommand{\yddot}[0]{{\ddot{y}}}
\newcommand{\zddot}[0]{{\ddot{z}}}

%<bold and dot greek symbols>
%

\newcommand{\Deltadot}[0]{{\dot{\Delta}}}
\newcommand{\Gammadot}[0]{{\dot{\Gamma}}}
\newcommand{\Lambdadot}[0]{{\dot{\Lambda}}}
\newcommand{\Omegadot}[0]{{\dot{\Omega}}}
\newcommand{\Phidot}[0]{{\dot{\Phi}}}
\newcommand{\Pidot}[0]{{\dot{\Pi}}}
\newcommand{\Psidot}[0]{{\dot{\Psi}}}
\newcommand{\Sigmadot}[0]{{\dot{\Sigma}}}
\newcommand{\Thetadot}[0]{{\dot{\Theta}}}
\newcommand{\Upsilondot}[0]{{\dot{\Upsilon}}}
\newcommand{\Xidot}[0]{{\dot{\Xi}}}
\newcommand{\alphadot}[0]{{\dot{\alpha}}}
\newcommand{\betadot}[0]{{\dot{\beta}}}
\newcommand{\chidot}[0]{{\dot{\chi}}}
\newcommand{\deltadot}[0]{{\dot{\delta}}}
\newcommand{\epsilondot}[0]{{\dot{\epsilon}}}
\newcommand{\etadot}[0]{{\dot{\eta}}}
\newcommand{\gammadot}[0]{{\dot{\gamma}}}
\newcommand{\kappadot}[0]{{\dot{\kappa}}}
\newcommand{\lambdadot}[0]{{\dot{\lambda}}}
\newcommand{\mudot}[0]{{\dot{\mu}}}
\newcommand{\nudot}[0]{{\dot{\nu}}}
\newcommand{\omegadot}[0]{{\dot{\omega}}}
\newcommand{\phidot}[0]{{\dot{\phi}}}
\newcommand{\pidot}[0]{{\dot{\pi}}}
\newcommand{\psidot}[0]{{\dot{\psi}}}
\newcommand{\rhodot}[0]{{\dot{\rho}}}
\newcommand{\sigmadot}[0]{{\dot{\sigma}}}
\newcommand{\taudot}[0]{{\dot{\tau}}}
\newcommand{\thetadot}[0]{{\dot{\theta}}}
\newcommand{\upsilondot}[0]{{\dot{\upsilon}}}
\newcommand{\varepsilondot}[0]{{\dot{\varepsilon}}}
\newcommand{\varphidot}[0]{{\dot{\varphi}}}
\newcommand{\varpidot}[0]{{\dot{\varpi}}}
\newcommand{\varrhodot}[0]{{\dot{\varrho}}}
\newcommand{\varsigmadot}[0]{{\dot{\varsigma}}}
\newcommand{\varthetadot}[0]{{\dot{\vartheta}}}
\newcommand{\xidot}[0]{{\dot{\xi}}}
\newcommand{\zetadot}[0]{{\dot{\zeta}}}

\newcommand{\Deltaddot}[0]{{\ddot{\Delta}}}
\newcommand{\Gammaddot}[0]{{\ddot{\Gamma}}}
\newcommand{\Lambdaddot}[0]{{\ddot{\Lambda}}}
\newcommand{\Omegaddot}[0]{{\ddot{\Omega}}}
\newcommand{\Phiddot}[0]{{\ddot{\Phi}}}
\newcommand{\Piddot}[0]{{\ddot{\Pi}}}
\newcommand{\Psiddot}[0]{{\ddot{\Psi}}}
\newcommand{\Sigmaddot}[0]{{\ddot{\Sigma}}}
\newcommand{\Thetaddot}[0]{{\ddot{\Theta}}}
\newcommand{\Upsilonddot}[0]{{\ddot{\Upsilon}}}
\newcommand{\Xiddot}[0]{{\ddot{\Xi}}}
\newcommand{\alphaddot}[0]{{\ddot{\alpha}}}
\newcommand{\betaddot}[0]{{\ddot{\beta}}}
\newcommand{\chiddot}[0]{{\ddot{\chi}}}
\newcommand{\deltaddot}[0]{{\ddot{\delta}}}
\newcommand{\epsilonddot}[0]{{\ddot{\epsilon}}}
\newcommand{\etaddot}[0]{{\ddot{\eta}}}
\newcommand{\gammaddot}[0]{{\ddot{\gamma}}}
\newcommand{\kappaddot}[0]{{\ddot{\kappa}}}
\newcommand{\lambdaddot}[0]{{\ddot{\lambda}}}
\newcommand{\muddot}[0]{{\ddot{\mu}}}
\newcommand{\nuddot}[0]{{\ddot{\nu}}}
\newcommand{\omegaddot}[0]{{\ddot{\omega}}}
\newcommand{\phiddot}[0]{{\ddot{\phi}}}
\newcommand{\piddot}[0]{{\ddot{\pi}}}
\newcommand{\psiddot}[0]{{\ddot{\psi}}}
\newcommand{\rhoddot}[0]{{\ddot{\rho}}}
\newcommand{\sigmaddot}[0]{{\ddot{\sigma}}}
\newcommand{\tauddot}[0]{{\ddot{\tau}}}
\newcommand{\thetaddot}[0]{{\ddot{\theta}}}
\newcommand{\upsilonddot}[0]{{\ddot{\upsilon}}}
\newcommand{\varepsilonddot}[0]{{\ddot{\varepsilon}}}
\newcommand{\varphiddot}[0]{{\ddot{\varphi}}}
\newcommand{\varpiddot}[0]{{\ddot{\varpi}}}
\newcommand{\varrhoddot}[0]{{\ddot{\varrho}}}
\newcommand{\varsigmaddot}[0]{{\ddot{\varsigma}}}
\newcommand{\varthetaddot}[0]{{\ddot{\vartheta}}}
\newcommand{\xiddot}[0]{{\ddot{\xi}}}
\newcommand{\zetaddot}[0]{{\ddot{\zeta}}}

\newcommand{\BDelta}[0]{\boldsymbol{\Delta}}
\newcommand{\BGamma}[0]{\boldsymbol{\Gamma}}
\newcommand{\BLambda}[0]{\boldsymbol{\Lambda}}
\newcommand{\BOmega}[0]{\boldsymbol{\Omega}}
\newcommand{\BPhi}[0]{\boldsymbol{\Phi}}
\newcommand{\BPi}[0]{\boldsymbol{\Pi}}
\newcommand{\BPsi}[0]{\boldsymbol{\Psi}}
\newcommand{\BSigma}[0]{\boldsymbol{\Sigma}}
\newcommand{\BTheta}[0]{\boldsymbol{\Theta}}
\newcommand{\BUpsilon}[0]{\boldsymbol{\Upsilon}}
\newcommand{\BXi}[0]{\boldsymbol{\Xi}}
\newcommand{\Balpha}[0]{\boldsymbol{\alpha}}
\newcommand{\Bbeta}[0]{\boldsymbol{\beta}}
\newcommand{\Bchi}[0]{\boldsymbol{\chi}}
\newcommand{\Bdelta}[0]{\boldsymbol{\delta}}
\newcommand{\Bepsilon}[0]{\boldsymbol{\epsilon}}
\newcommand{\Beta}[0]{\boldsymbol{\eta}}
\newcommand{\Bgamma}[0]{\boldsymbol{\gamma}}
\newcommand{\Bkappa}[0]{\boldsymbol{\kappa}}
\newcommand{\Blambda}[0]{\boldsymbol{\lambda}}
\newcommand{\Bmu}[0]{\boldsymbol{\mu}}
\newcommand{\Bnu}[0]{\boldsymbol{\nu}}
%\newcommand{\Bomega}[0]{\boldsymbol{\omega}}
\newcommand{\Bphi}[0]{\boldsymbol{\phi}}
\newcommand{\Bpi}[0]{\boldsymbol{\pi}}
\newcommand{\Bpsi}[0]{\boldsymbol{\psi}}
\newcommand{\Brho}[0]{\boldsymbol{\rho}}
\newcommand{\Bsigma}[0]{\boldsymbol{\sigma}}
%\newcommand{\Btau}[0]{\boldsymbol{\tau}}
%\newcommand{\Btheta}[0]{\boldsymbol{\theta}}
\newcommand{\Bupsilon}[0]{\boldsymbol{\upsilon}}
\newcommand{\Bvarepsilon}[0]{\boldsymbol{\varepsilon}}
\newcommand{\Bvarphi}[0]{\boldsymbol{\varphi}}
\newcommand{\Bvarpi}[0]{\boldsymbol{\varpi}}
\newcommand{\Bvarrho}[0]{\boldsymbol{\varrho}}
\newcommand{\Bvarsigma}[0]{\boldsymbol{\varsigma}}
\newcommand{\Bvartheta}[0]{\boldsymbol{\vartheta}}
\newcommand{\Bxi}[0]{\boldsymbol{\xi}}
\newcommand{\Bzeta}[0]{\boldsymbol{\zeta}}
%
%</bold and dot greek symbols>
%<infrequent>
%
%\newcommand{\AreaOp}[1]{\AName_{#1}}
%\newcommand{\Babs}[0]{\abs{\BB}}
%\newcommand{\Bcap}[0]{\hat{\BB}}
%\newcommand{\BrPrimeRej}[0]{\rcap(\rcap \wedge \Br')}
%\newcommand{\CA}[0]{\mathcal{A}}
%\newcommand{\Cos}[1]{\cos{\left({#1}\right)}}
%\newcommand{\Det}[1] {\abs{#1}}
%\newcommand{\Dsq}[2] {\frac {\partial^2 {#1}} {\partial {#2}^2}}
%\newcommand{\Exp}[1]{\exp{\left({#1}\right)}}
%\newcommand{\Norm}[1]{\left\lVert{#1}\right\rVert}
%\newcommand{\Sin}[1]{\sin{\left({#1}\right)}}
%\newcommand{\T}[0]{\text{T}}
%\newcommand{\VolumeOp}[1]{\VName_{#1}}
%\newcommand{\agrad}[0]{\Ba \cdot \nabla}
%\newcommand{\alphacap}[0]{\hat{\boldsymbol{\alpha}}}
%\newcommand{\Fcap}[0]{\hat{\BF}}
%\newcommand{\bithree}[0]{{\Bi}_3}
%\newcommand{\bxa}[0]{\Bx\Ba}
%\newcommand{\coordvec}[2]{
%\newcommand{\costheta}[0]{\acap \cdot \xcap}
%\newcommand{\ddt}[1]{\ddot{#1}}
%\newcommand{\ddu}[1] {\frac {d{#1}} {du}}
%\newcommand{\dsqxj}[2] {\frac {\partial^2 {#1}} {\partial {x_{#2}}^2}}
%\newcommand{\dtheta}[1]{\frac{d {#1}}{d \theta}}
%\newcommand{\dt}[1]{\dot{#1}}
%\newcommand{\dt}[1]{\frac{d {#1}}{dt}}
%\newcommand{\dxj}[2] {\frac {\partial {#1}} {\partial {x_{#2}}}}
%\newcommand{\halfPhi}[0]{\frac{\phi}{2}}
%\newcommand{\half}[0]{\inv{2}}
%\newcommand{\inv}[1]{\frac{1}{#1}}
%\newcommand{\laplacian}[0]{\nabla^2}
%\newcommand{\matrixoftx}[3]{
%\newcommand{\nrrp}[0]{\norm{\rcap \wedge \Br'}}
%\newcommand{\oiint}{\bigcirc \hspace{-1.4em} \int \hspace{-.8em} \int}
%\newcommand{\transpose}[1]{{#1}^{\text{T}}}
%\newcommand{\transpose}[1]{{{#1}^{\TextTranspose}}}
%\newcommand{\transpose}[1]{{{#1}^{\text{T}}}}
%\newcommand{\barA}[0]{\bar{A}}
%\newcommand{\qbar}[0]{\bar{q}}
%\newcommand{\qdotbar}[0]{\dot{\bar{q}}}
%
%</infrequent>





\usepackage[bookmarks=true]{hyperref}

\usepackage{color,cite,graphicx}
   % use colour in the document, put your citations as [1-4]
   % rather than [1,2,3,4] (it looks nicer, and the extended LaTeX2e
   % graphics package.
\usepackage{latexsym,amssymb,epsf} % don't remember if these are
   % needed, but their inclusion can't do any damage


\title{ Energy momentum tensor }
\author{Peeter Joot}
\date{ Jan 01, 2009.  Last Revision: $Date: 2009/01/02 06:35:02 $ }

\begin{document}
\maketitle{}

\tableofcontents
\section{ Expanding out the stress energy vector in tensor form. }

\cite{doran2003gap} defines (with $\epsilon_0$ ommitted),
the energy momentum stress tensor as a vector to
vector mapping of the following form:

\begin{align}
T(a)
&= \frac{\epsilon_0}{2} F a \tilde{F}
= - \frac{\epsilon_0}{2} F a F
\end{align}

This quantity can only have vector, trivector, and five vector grades.  The
grade five term must be zero

\begin{align*}
\gpgrade{T(a)}{5}
&= \frac{\epsilon_0}{2} F \wedge a \wedge \tilde{F} \\
&= \frac{\epsilon_0}{2} a \wedge (F \wedge \tilde{F}) \\
&= 0
\end{align*}

Since $(T(a))^{\tilde{}} = T(a)$, the grade three term is also zero (trivectors invert on reversion), so this must therefore be a vector.

As a vector this can be expanded in coordinates

\begin{align*}
T(a)
&= \left(T(a) \cdot \gamma^\nu \right) \gamma_\nu \\
&= \left(T(a^\mu \gamma_\mu) \cdot \gamma^\nu \right) \gamma_\nu \\
&= a^\mu \gamma_\nu \left(T(\gamma_\mu) \cdot \gamma^\nu \right) \\
\end{align*}

It's this last bit that has the form of a traditional tensor, so we can write

\begin{align}
T(a) &= a^\mu \gamma_\nu {T_\mu}^{\nu} \\
{T_\mu}^{\nu} &= T(\gamma_\mu) \cdot \gamma^\nu
\end{align}

Let's expand this tensor ${T_\mu}^{\nu}$ explicitly to verify its form.

We want to expand, and dot with $\gamma^\nu$, the following

\begin{align*}
-2 \inv{\epsilon_0} \left(T(\gamma_\mu) \cdot \gamma^\nu \right) \gamma_\nu
&= \gpgradeone{(\grad \wedge A) \gamma_\mu (\grad \wedge A)} \\
&= \gpgradeone{
(\grad \wedge A) \cdot \gamma_\mu (\grad \wedge A)
+ (\grad \wedge A) \wedge \gamma^\mu (\grad \wedge A)
} \\
&=
((\grad \wedge A) \cdot \gamma_\mu) \cdot (\grad \wedge A)
+ ((\grad \wedge A) \wedge \gamma_\mu) \cdot (\grad \wedge A)
\\
\end{align*}

Both of these will get temporarily messy, so let's do them in parts.  Starting
with

\begin{align*}
(\grad \wedge A) \cdot \gamma_\mu
&= (\gamma^{\alpha} \wedge \gamma^{\beta}) \cdot \gamma_{\mu} \partial_{\alpha} A_{\beta} \\
&= (\gamma^{\alpha} {\delta^{\beta}}_{\mu} -\gamma^{\beta} {\delta^{\alpha}}_{\mu} ) \partial_{\alpha} A_{\beta} \\
&=
\gamma^{\alpha} \partial_{\alpha} A_{\mu}
-\gamma^{\beta} \partial_{\mu} A_{\beta} \\
&= \gamma^{\alpha} (\partial_{\alpha} A_{\mu} -\partial_{\mu} A_{\alpha} ) \\
&= \gamma^{\alpha} F_{\alpha \mu} \\
\end{align*}


\begin{align*}
((\grad \wedge A) \cdot \gamma_\mu) \cdot (\grad \wedge A)
&= (\gamma^{\alpha} F_{\alpha \mu}) \cdot (\gamma_{\beta} \wedge \gamma_{\lambda}) \partial^{\beta} A^{\lambda} \\
%&=
%\partial^{\beta} A^{\lambda} F_{\alpha \mu}
%\gamma^{\alpha} \cdot (\gamma_{\beta} \wedge \gamma_{\lambda})
%\\
&=
\partial^{\beta} A^{\lambda} F_{\alpha \mu}
(
{\delta^{\alpha}}_{\beta} \gamma_{\lambda}
-{\delta^{\alpha}}_{\lambda} \gamma_{\beta}
)
\\
&=
(\partial^{\alpha} A^{\beta} F_{\alpha \mu} -\partial^{\beta} A^{\alpha} F_{\alpha \mu} )\gamma_{\beta}
\\
&= F^{\alpha \beta} F_{\alpha \mu} \gamma_{\beta} \\
\end{align*}

So, by dotting with $\gamma^\nu$ we have

\begin{align}\label{eqn:firstPartDone}
((\grad \wedge A) \cdot \gamma_\mu) \cdot (\grad \wedge A) \cdot \gamma^{\nu}
&= F^{\alpha \nu} F_{\alpha \mu}
\end{align}

Moving on to the next bit,
$(((\grad \wedge A) \wedge \gamma^\mu) \cdot (\grad \wedge A)) \cdot \gamma^\nu$.� By inspection the first part of this is

\begin{align*}
(\grad \wedge A) \wedge \gamma_\mu
&= (\gamma_\mu)^2 (\gamma^{\alpha} \wedge \gamma^{\beta}) \wedge \gamma^{\mu} \partial_{\alpha} A_{\beta} \\
\end{align*}

so dotting with $\grad \wedge A$, we have

\begin{align*}
((\grad \wedge A) \wedge \gamma_\mu) \cdot (\grad \wedge A)
&=
(\gamma_\mu)^2
\partial_{\alpha} A_{\beta}
\partial^{\lambda} A^{\delta}
(\gamma^{\alpha} \wedge \gamma^{\beta} \wedge \gamma^{\mu}) \cdot
(\gamma_{\lambda} \wedge \gamma_{\delta})
\\
&=
(\gamma_\mu)^2
\partial_{\alpha} A_{\beta}
\partial^{\lambda} A^{\delta}
((\gamma^{\alpha} \wedge \gamma^{\beta} \wedge \gamma^{\mu}) \cdot \gamma_{\lambda} ) \cdot \gamma_{\delta}
\\
\end{align*}

Expanding just the dot product parts of this we have
\begin{align*}
&(((\gamma^{\alpha} \wedge \gamma^{\beta}) \wedge \gamma^{\mu}) \cdot \gamma_{\lambda} ) \cdot \gamma_{\delta} \\
&=
(\gamma^{\alpha} \wedge \gamma^{\beta}) {\delta^{\mu}}_{\lambda}
-(\gamma^{\alpha} \wedge \gamma^{\mu}) {\delta^{\beta}}_{\lambda}
+(\gamma^{\beta} \wedge \gamma^{\mu}) {\delta^{\alpha}}_{\lambda}
) \cdot \gamma_{\delta}
\\
%&=
%(
%  \gamma^{\alpha} {\delta^{\beta}}_{\delta} {\delta^{\mu}}_{\lambda}
%- \gamma^{\alpha} {\delta^{\mu}}_{\delta} {\delta^{\beta}}_{\lambda}
%+ \gamma^{\beta} {\delta^{\mu}}_{\delta} {\delta^{\alpha}}_{\lambda}
%- \gamma^{\beta} {\delta^{\alpha}}_{\delta} {\delta^{\mu}}_{\lambda}
%+ \gamma^{\mu} {\delta^{\alpha}}_{\delta} {\delta^{\beta}}_{\lambda}
%- \gamma^{\mu} {\delta^{\beta}}_{\delta} {\delta^{\alpha}}_{\lambda}
%)
%\\
&=
  \gamma^{\alpha} ({\delta^{\beta}}_{\delta} {\delta^{\mu}}_{\lambda}
-            {\delta^{\mu}}_{\delta} {\delta^{\beta}}_{\lambda})
+ \gamma^{\beta} ({\delta^{\mu}}_{\delta} {\delta^{\alpha}}_{\lambda}
-            {\delta^{\alpha}}_{\delta} {\delta^{\mu}}_{\lambda})
+ \gamma^{\mu} ({\delta^{\alpha}}_{\delta} {\delta^{\beta}}_{\lambda}
-                 {\delta^{\beta}}_{\delta} {\delta^{\alpha}}_{\lambda})
\\
\end{align*}

This can now be applied to $\partial^{\lambda} A^{\delta}$

\begin{align*}
\partial^{\lambda} A^{\delta} &(((\gamma^{\alpha} \wedge \gamma^{\beta}) \wedge \gamma^{\mu}) \cdot \gamma_{\lambda} ) \cdot \gamma_{\delta} \\
&=
  \partial^{\mu} A^{\beta} \gamma^{\alpha}
- \partial^{\beta} A^{\mu} \gamma^{\alpha}
+ \partial^{\alpha} A^{\mu} \gamma^{\beta}
- \partial^{\mu} A^{\alpha} \gamma^{\beta}
+ \partial^{\beta} A^{\alpha} \gamma^{\mu}
- \partial^{\alpha} A^{\beta} \gamma^{\mu}
\\
&=
(  \partial^{\mu} A^{\beta}
- \partial^{\beta} A^{\mu} ) \gamma^{\alpha}
+( \partial^{\alpha} A^{\mu}
- \partial^{\mu} A^{\alpha} ) \gamma^{\beta}
+( \partial^{\beta} A^{\alpha}
- \partial^{\alpha} A^{\beta} ) \gamma^{\mu}
\\
&=
%(  \partial^{\mu} A^{\beta} - \partial^{\beta} A^{\mu} )
F^{\mu \beta}
\gamma^{\alpha}
+
%( \partial^{\alpha} A^{\mu} - \partial^{\mu} A^{\alpha} )
F^{\alpha \mu}
\gamma^{\beta}
+
%( \partial^{\beta} A^{\alpha} - \partial^{\alpha} A^{\beta} )
F^{\beta \alpha}
\gamma^{\mu}
\\
\end{align*}

This is getting closer, and we can now write
\begin{align*}
((\grad \wedge A) \wedge \gamma_\mu) \cdot (\grad \wedge A) &=
(\gamma_\mu)^2 \partial_{\alpha} A_{\beta}
(
  F^{\mu \beta} \gamma^{\alpha}
+ F^{\alpha \mu} \gamma^{\beta}
+ F^{\beta \alpha} \gamma^{\mu}
) \\
&=
  (\gamma_\mu)^2 \partial_{\beta} A_{\alpha} F^{\mu \alpha} \gamma^{\beta}
+ (\gamma_\mu)^2 \partial_{\alpha} A_{\beta} F^{\alpha \mu} \gamma^{\beta}
+ (\gamma_\mu)^2 \partial_{\alpha} A_{\beta} F^{\beta \alpha} \gamma^{\mu}
\\
&=
  F^{\beta \alpha} F_{\mu \alpha} \gamma_{\beta}
+ \partial_{\alpha} A_{\beta} F^{\beta \alpha} \gamma_{\mu}
\\
\end{align*}

This can now be dotted with $\gamma^\nu$,

\begin{align*}
((\grad \wedge A) \wedge \gamma_\mu) \cdot (\grad \wedge A) \cdot \gamma^\nu
&=
 F^{\beta \alpha} F_{\mu \alpha} {\delta_{\beta}}^\nu
+ \partial_{\alpha} A_{\beta} F^{\beta \alpha} {\delta_{\mu}}^\nu
\\
%&= F^{\nu \alpha} F_{\mu \alpha} + \inv{2} F_{\alpha \beta} F^{\beta \alpha} {\delta_{\mu}}^\nu \\
\end{align*}

which is
\begin{align}\label{eqn:secondPart}
((\grad \wedge A) \wedge \gamma_\mu) \cdot (\grad \wedge A) \cdot \gamma^\nu
&= F^{\nu \alpha} F_{\mu \alpha} + \inv{2} F_{\alpha \beta} F^{\beta \alpha} {\delta_{\mu}}^\nu
%F^{\nu \beta} F_{\mu \beta} +\inv{2} F_{\alpha \beta} F^{\beta \alpha} {\delta_{\mu}}^\nu
\end{align}

The final combination of results \ref{eqn:firstPartDone}, and
\ref{eqn:secondPart} gives

\begin{align*}
(F \gamma_\mu F ) \cdot \gamma^\nu
&=
2 F^{\alpha \nu} F_{\alpha \mu}
+\inv{2} F_{\alpha \beta} F^{\beta \alpha} {\delta_{\mu}}^\nu
\end{align*}

Yielding the tensor

\begin{align}\label{eqn:messyTensor}
{T_\mu}^{\nu}
&=
\epsilon_0 \left(
\inv{4} F_{\alpha \beta} F^{\alpha \beta} {\delta_{\mu}}^\nu
-
F_{\alpha \mu}
F^{\alpha \nu}
\right)
\end{align}

\section{ Validate against previously calculated Poynting result. }

In \cite{PJpoynting}, the electrodynamic energy density $U$ and momentum flux density vectors were related as follows

\begin{align}\label{eqn:fromPoyntingNotes}
U &= \frac{\epsilon_0}{2}\left( \BE^2 + c^2 \BB^2 \right) \\
\BP &= \epsilon_0 c^2 \BE \cross \BB = \epsilon_0 c (i c \BB) \cdot \BE \\
0 &= \PD{t}{}\frac{\epsilon_0}{2} \left(\BE^2 + c^2 \BB^2\right) + c^2 \epsilon_0 \spacegrad \cdot (\BE \cross \BB) + \BE \cdot \Bj 
\end{align}

Additionally the energy and momentum flux densities are components of this stress tensor four vector

\begin{align*}
T(\gamma_0) &= U \gamma_0 + \inv{c} \BP \gamma_0 \\
\end{align*}

From this we can read the first row of the tensor elements

\begin{align*}
{T_0}^0 &= U 
= \frac{\epsilon_0}{2}\left( \BE^2 + c^2 \BB^2 \right) \\
{T_0}^k &= \inv{c} (\BP \gamma_0) \cdot \gamma^k = \epsilon_0 c E^a B^b \epsilon_{k a b}
\end{align*}

Let's compare these to \ref{eqn:messyTensor}, which gives
\begin{align*}
{T_0}^{0} 
&= \epsilon_0 \left( \inv{4} F_{\alpha \beta} F^{\alpha \beta} - F_{\alpha 0} F^{\alpha 0} \right) \\
&= \frac{\epsilon_0}{4} \left( F_{\alpha j} F^{\alpha j} - {3} F_{j 0} F^{j 0} \right) \\
&= \frac{\epsilon_0}{4} \left( F_{m j} F^{m j} +F_{0 j} F^{0 j} - {3} F_{j 0} F^{j 0} \right) \\
&= \frac{\epsilon_0}{4} \left( F_{m j} F^{m j} - {2} F_{j 0} F^{j 0} \right) \\
{T_0}^{k} 
&= -\epsilon_0 F_{\alpha 0} F^{\alpha k} \\
&= -\epsilon_0 F_{j 0} F^{j k} \\
\end{align*}

Now, our field in terms of electric and magnetic coordinates is

\begin{align*}
F &= \BE + i c \BB \\
  &= E^k \gamma_k \gamma_0 + i c B^k \gamma_k \gamma_0 \\
  &= E^k \gamma_k \gamma_0 - c \epsilon_{a b k} B^k \gamma_a \gamma_b
\end{align*}

so the electric field tensor components are

\begin{align*}
F^{j 0} 
&= (F \cdot \gamma^0) \cdot \gamma^j \\
&= E^k {\delta_k}^j \\
&= E^j
\end{align*}

and
\begin{align*}
F_{j 0} &= (\gamma_j)^2 (\gamma_0)^2 F^{j 0} \\
&= -E^j
\end{align*}

and the magnetic tensor components are

\begin{align*}
F^{m j} &= F_{m j} \\
&= - c \epsilon_{a b k} B^k ((\gamma_a \gamma_b) \cdot \gamma_{j}) \cdot \gamma_m \\
&= - c \epsilon_{m j k} B^k
\end{align*}

This gives
\begin{align*}
{T_0}^{0} 
&= \frac{\epsilon_0}{4} \left( 2 c^2 B^k B^k + {2} E^j E^{j} \right) \\
&= \frac{\epsilon_0}{2} \left( c^2 \BB^2 + \BE^2 \right) \\
{T_0}^{k} 
&= \epsilon_0 E^{j} F^{j k} \\
&= \epsilon_0 c \epsilon_{k e f} E^e B^f \\
&= \epsilon_0 (c \BE \cross \BB)_k \\
&= \inv{c} (\BP \cdot \sigma_k)
\end{align*}

Okay, good.  This checks 4 of the elements of \ref{eqn:messyTensor} against the explicit $\BE$ and $\BB$ based representation of $T(\gamma_0)$ in \ref{eqn:fromPoyntingNotes}, leaving only 6 unique elements in the remaining parts of the (symmetric) tensor to verify.

\section{ Validate with relativistic transformation. }

As a relativistic quantity we should be able to verify the messy tensor relationship 
by Lorentz transforming the energy density from a rest frame to a 
moving frame.  Also observe that the momentum flux density, the Poynting vector,
is zero in the rest frame, which makes sense since there is no magnetic field
for a static charge distribution.

Now let's try the Lorentz transformation of the energy density.

FIXME: TODO.

\bibliographystyle{plainnat}
\bibliography{myrefs}

\end{document}
