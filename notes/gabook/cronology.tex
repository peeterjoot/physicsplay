\chapter{Cronological Index}
\label{chap:Cronology}

\begin{itemize}

\item October 13, 2007 \ref{chap:gaWiki} Comparison of many traditional vector and GA identities

\item October 13, 2007 \ref{chap:gaWikiTorque} Torque

\item October 16, 2007 \ref{chap:PJUnitDer} Derivatives of a unit vector

\item October 16, 2007 \ref{chap:gaWikiCramers} Cramer's rule

\item October 22, 2007 \ref{chap:PJRadialDer} Radial components of vector derivatives

\item January 1, 2008 \ref{chap:plane} More details on NFCM plane formulation

\item January 29, 2008 \ref{chap:PJAngVel} Rotational dynamics

\item January 29, 2008 \ref{chap:maxwellsGa} Maxwell's equations expressed with Geometric Algebra

\item February 2, 2008 \ref{chap:quaternion} Quaternions

\item February 4, 2008 \ref{chap:legendre} Legendre Polynomials

\item February 15, 2008 \ref{chap:inertialTensor} Inertia Tensor

\item February 19, 2008 \ref{chap:rotor} Rotor Notes

\item February 28, 2008 \ref{chap:laplace} Exponential Solutions to Laplace Equation in \R{N}

\item March 9, 2008 \ref{chap:bivector} Bivector Geometry

\item March 9, 2008 \ref{chap:trivector} Trivector geometry

\item March 12, 2008 \ref{chap:kvectorExponential} Exponential of a blade

\item March 16, 2008 \ref{chap:scalarCommutes} Multivector product grade zero terms

\item March 17, 2008 \ref{chap:angleBetweenLineAndPlane} Angle between geometric elements

\item March 17, 2008 \ref{chap:gaGradeDotWedge} An earlier attempt to intuitively introduce the dot, wedge, cross, and geometric products

\item March 25, 2008 \ref{chap:bladegradereduction} Blade grade reduction

\item March 29, 2008 \ref{chap:reciprocalFrame} Reciprocal Frame Vectors

\item March 31, 2008 \ref{chap:gradientAndForms} Exterior derivative and chain rule components of the gradient

\item April 1, 2008 \ref{chap:orthodecomp} Orthogonal decomposition take II

\item April 11, 2008 \ref{chap:PJMatrixReview} Matrix review

\item April 13, 2008 \ref{chap:locateSatellite} Satellite triangulation over sphere

\item April 30, 2008 \ref{chap:PJKeRot} Kinetic Energy in rotational frame

\item May 7, 2008 \ref{chap:lorentzRotation} Lorentz Force Trajectory

\item May 16, 2008 \ref{chap:obliqueProj} Oblique projection and reciprocal frame vectors

\item May 16, 2008 \ref{chap:matrixOfLinearTx} Matrix of grade k multivector linear transformations

\item May 16, 2008 \ref{chap:projectionAndMoorePenroseVectorInverse} Projection and Moore-Penrose vector inverse

\item May 17, 2008 \ref{chap:PJprojGen} Projection with generalized dot product

\item June 6, 2008 \ref{chap:tensor} Gradient and tensor notes

\item June 10, 2008 \ref{chap:PJAngAcc} Angular Velocity and Acceleration.  Again

\item June 25, 2008 \ref{chap:PJLorentzWave} Wave equation based Lorentz transformation derivation

A derivation of the Lorentz transformation requiring invariance of electrodynamic wave equation.  A mechanical approach very similar to the usual spherical light shell invariance, but one that doesn't require the difficult conceptualization of speed of light invariance.\item July 8, 2008 \ref{chap:PJAngAccCross} Cross product Radial decomposition

\item July 12, 2008 \ref{chap:PJMaxwell2} Back to Maxwell's equations

\item July 16, 2008 \ref{chap:spacetimegrad} Lorentz transformation of spacetime gradient

\item July 20, 2008 \ref{chap:sgMx41} Magnetic field between two parallel wires

\item August 1, 2008 \ref{chap:fourvecDotinvariance} Four vector dot product invariance and Lorentz rotors

Rotor form of the Lorentz boost, and invariance of four vector dot product.\item August 9, 2008 \ref{chap:newtonianLagrangianAndGradient} Newton's Law from Lagrangian

\item August 13, 2008 \ref{chap:cauchyGradient} Cauchy Equations expressed as a gradient

\item August 13, 2008 \ref{chap:velocityTx} Understanding four velocity transform from rest frame

Explicit expansion of Lorentz boost applied to rest frame event vector.\item August 15, 2008 \ref{chap:emPotential} Four vector potential

\item August 16, 2008 \ref{chap:PJSrGAFPLorentzForce} Lorentz force Law

\item August 21, 2008 \ref{chap:PJSrLagrangian} Covariant Lagrangian, and electrodynamic potential

\item August 25, 2008 \ref{chap:PJTongMf1} Solutions to David Tong's mf1 Lagrangian problems

\item August 28, 2008 \ref{chap:massVaryLagrangian} Equations of motion given mass variation with spacetime position

\item September 1, 2008 \ref{chap:PJCanMomentum} Vector canonical momentum

\item September 2, 2008 \ref{chap:outermorphismDet} OuterMorphism Question 

\item September 5, 2008 \ref{chap:emBivectorMetricDependencies} Metric signature dependencies

Examination of metric dependencies in STA and relationships to tensor expression\item September 7, 2008 \ref{chap:PJMaxwellTensor} Tensor relations from bivector field equation

\item September 8, 2008 \ref{chap:PJMaxwellLagrangian} Direct variation of Maxwell equations

\item September 9, 2008 \ref{chap:PJMaxwellProj} Vector forms of Maxwell's equations as projection and rejection operations

\item September 18, 2008 \ref{chap:PJStokes1} Stokes law in wedge product form

\item September 26, 2008 \ref{chap:stokesMaxwellApplication} Application of Stokes Integrals to Maxwell's Equation

\item September 27, 2008 \ref{chap:PJStokes2} Stokes Law revisited with algebraic enumeration of boundary

\item October 8, 2008 \ref{chap:PJSrLorentzForce} Revisit Lorentz force from Lagrangian

\item October 10, 2008 \ref{chap:PJFieldLagrangian} Derivation of Euler-Lagrange field equations

Derivation of the field form of the Euler Lagrange equations, with applications including Schrodinger's and Klien-Gordan field equations\item October 12, 2008 \ref{chap:maxwellTensorLagrangian} Tensor Derivation of Covariant Lorentz Force from Lagrangian

\item October 13, 2008 \ref{chap:PJEulerLagrange} Euler Lagrange Equations

\item October 19, 2008 \ref{chap:PJBoostMaxwell} Lorentz Invariance of Maxwell Lagrangian

\item October 22, 2008 \ref{chap:PJLorentzTxInteraction} Lorentz transform Noether current for interaction Lagrangian

\item October 26, 2008 \ref{chap:gem} GravitoElectroMagnetism

Rough notes (mostly questions) about GravitoElectroMagnetism.\item October 29, 2008 \ref{chap:PJNoethersField} Field form of Noether's Law

\item November 1, 2008 \ref{chap:PJEulerAngle} Euler Angle Notes

\item November 8, 2008 \ref{chap:complex} Hyper complex numbers and symplectic structure

\item November 13, 2008 \ref{chap:sphericalPolar} Spherical polar coordinates

\item November 22, 2008 \ref{chap:gaussianSurface} Gaussian Surface invariance for radial field

\item November 23, 2008 \ref{chap:chargeArcElement} Field due to line charge in arc

\item November 23, 2008 \ref{chap:chargeLineElement} Charge line element

\item November 27, 2008 \ref{chap:nfcmCh2} Some NFCM exercise solutions and notes

\item November 30, 2008 \ref{chap:PJwaveFourVector} Expressing wave equation exponential solutions using four vectors

Four vector exponential solutions of arbitrary velocity wave equations.\item November 30, 2008 \ref{chap:slerp} Rotor interpolation calculation

\item December 6, 2008 \ref{chap:pauliMatrix} Pauli Matrixes in Clifford Algebra

Pauli algebra notes.  Apply the Pauli algebra in a GA like fashion for spatial relationships.  Wedge, dot and cross products expressed in terms of commutator and anticommutators.\item December 11, 2008 \ref{chap:bohr} Bohr Model

Derivation and notes on the Bohr model.\item December 13, 2008 \ref{chap:PJDiracGamma} Gamma Matrices

\item December 21, 2008 \ref{chap:diracLagrangian} Dirac Lagrangian

An attempt to decode the Dirac equation Lagrangians found in wikipedia.  Calculate the field equations from the Lagrangians once all the terms were understood.  Includes a translation between the matrix and Doran/Lasenby notations for dagger and Dirac adjoint.\item December 27, 2008 \ref{chap:PJrayleighJeans} Rayleigh-Jeans Law Notes

\item December 29, 2008 \ref{chap:PJpoynting} Poynting vector and Electromagnetic Energy conservation

\item January 1, 2009 \ref{chap:PJemstresstensor} Energy momentum tensor

As well as some brute force notes on expanding the tensor, the spacetime divergence of the rest frame elements of this tensor is used to derive, in a particularily slick fashion IMO, the Poynting energy momentum current conservation equation.  Want to also followup on what's here with a relativistic transformation approach, but will have to think it through. \item January 3, 2009 \ref{chap:PJelectricFieldEnergy} Field and wave energy and momentum

Start working out for myself the electrostatic and magnetostatic energy relationships.  Got the electrostatic part done, and got as far as a from first principles Biot-Savart derivation using the STA formalism.  Next work out the magnetostatic energy relationship.  Also intend to tackle wave energy and momentum here, but in the end, may split that into a separate set of notes.  Relate the energy-density-rate + Poynting divergence equation to the Lorentz force and discuss.  Also relates the various terms of the stress energy tensor to the Lorentz force.  See now how the covariant Lorentz force and the stress energy tensor is related, and also have some intuitive justification now for why we call $E^2 + B^2$ the field energy density.  Want to justify in terms of work done against Lorentz force. \item January 5, 2009 \ref{chap:vectorDifferentialIdentities} Vector Differential Identities

 Translate some identities from the Feynman lectures into GA form.  These apply in higher dimensions with the GA formalism, and proofs of the generalized idenities are derived.  Make a note of the last two identities that I wanted to work through.  This is an incomplete attempt at them.  It was trickier than I expected, and probably why they were omitted from Feynman's text. \item January 6, 2009 \ref{chap:dcPower} DC Power consumption formula for resistive load

Work out $P = I V$ from first principles since I forgot it.  Well, from second principles I suppose, since I utilize my recent Poynting derivation. \item January 9, 2009 \ref{chap:PJqmFourier} Some Fourier transform notes

QM formulation, with hbar's, of the Fourier transform pair, and Rayleigh Energy theorem, as seen in the book "Quantum Mechanics Demystified".  Very non-rigourous treatment, good only for intuition.  Also derive the Rayleigh Energy theorem used (but not proved) in this text. \item January 11, 2009 \ref{chap:schCurrent} Schr\"{o}dinger equation probability conservation

Schrodinger probability density and current conservation equation, and comparison of four-vectorized current to Dirac Lagrangian.

Calculating the rate of change of probability, and using Schrodinger's equation and its conjugate allows for the definition of a probability current, and an electromagnetic like probability-density/current-density conservation law. 

What I thought was interesting was that if you put this into a four vector form as a spacetime divergence (ie: the Lorentz gauge of electrodynamics), the resulting 'four-component' current vector needs only a $\gamma^0 \partial_0$ term to be added to it, for that current itself to be the Dirac Lagrangian (omitting the local-gauge term eA).  So it looks like taking the spacetime divergence of the Dirac Lagrangian essentially gives you the probability/current conservation equation (except now this would also produce an extra timelike term not there in the original Schrodinger's equation.)  There are some notational differences with the wikipedia form of the Dirac Lagrangian, but I believe all the basic content is there once those differences are accounted for.  Very suprising to see the Dirac Lagrangian fall so naturally out of the Schrodinger (non-relativistic) equation. 

I also observe that the probability wave function is perhaps naturally expressed as a relativistic four vector (with a $\gamma_0$ term factored out).  I still don't understand how Maxwell's equation and QM fit together, but with Maxwell's equation or Lagrangian expressable strictly in terms of four vectors (or the four-gradient and four-curl of such four vectors), there would be a logical cleanliness if one could also express the (relativistic) QM laws strictly in terms of four vectors.  Definitely worth playing with. 
\item January 13, 2009 \ref{chap:radial} Polar velocity and acceleration

Straight up column matrix vectors and complex number variants of radial motion derivatives. \item January 18, 2009 \ref{chap:PJpoyntingRate} Time rate of change of the Poynting vector, and its conservation law

These notes contain the conservation calculation itself, and verify the end result of Schwartz's tricky relativistic argument, that I have yet to understand, to put the conservation into a divergence form that is volume integrable. 

The derivation itself is not too hard.  Reconciling all the different notations is actually the tricky bit.  Schwartz does this in terms of the dual field tensors F and G, Doran/Lasenby have their GA $F \gamma_k F$ formulation, wikipedia had something different either of than those, and I'd seen in another paper that Jackson used something completely different.  At the time I did not have Jackson to see how he did it. 

Very interesting here is that we end up with what looks like the Lorentz force law by only looking at conservation requirements based on Maxwell's equation itself.  Calling the Poynting vector a field momentum density by analogy (because it showed up in what appeared to be an Energy/momentum (density) four vector) is then seen to be very justifiable.  Previously I'd seen that it took two Lagrangians for electrodynamics.  One for the fields and one for the interaction term.  But now it looks like the interaction term follows from the fields (in a handwaving, fuzzy, not yet fully understood way).  Quite interesting, and worth more thought, but seeing how one gets the interaction term from the QM field equation should probably take precedence. 
\item January 19, 2009 \ref{chap:PJheatFourier} Fourier Solutions to Heat and Wave equations

Apply the series technique to solve for the general time evolution of a wave function for a free (no potential) particle constrained to a circle, and the transform method for a one dimensional linear (non-periodic) scenerio. \item January 21, 2009 \ref{chap:fourierNotation} A cheatsheet for Fourier transform conventions

\item January 25, 2009 \ref{chap:PJemWave} Electrodynamic wave equation solutions

Carry the separation of variables to a reasonable point of completion, deriving a tidy relativistic solution for $F_{\mu\nu}$.  After this try generalizing that a bit with some intuition that turned out to be busted.  Left my dead ends as a marker pointing where not to go in the future. \item January 26, 2009 \ref{chap:PJwaveFourier} Fourier transform solutions to the wave equation

Produces the $f(x,t) = g( x - vt )$ solution quite nicely!  This works in a fashion for the 2 and 3D cases too, but there the Green's function doesn't reduce nicely to a delta function as in the 1D case. \item January 29, 2009 \ref{chap:PJfourierMaxwellSecondOrder} Fourier transform solutions to Maxwell's equation

Work out a Green's function solution of sorts for the non-homogenious Maxwell's equation. \item January 31, 2009 \ref{chap:PJfirstOrderMaxwell} First order Fourier transform solution of Maxwell's equation

Application of the Fourier transform to the spacetime split of the gradient term of Maxwell's equation allows for a complete solution of both the vacuum and current forced fields without requiring any computation with four vector potentials.  Presuming I got all the math right, this is a beautiful application of both Fourier theory and the STA algebra.  Note that the Rigor police are thoroughly away on vacation in this particular set of notes! \item February 1, 2009 \ref{chap:PJ4dFourier} 4D Fourier transforms applied to Maxwell's equation

Wow, using a spacetime Fourier transform for a Maxwell's solution is much simpler.  This is a neat result.\item February 3, 2009 \ref{chap:PJFourierVacuum} Fourier series Vacuum Maxwell's equations

Go through Bohm's treatment that preps for the Rayleigh-Jeans result in his quantum book in a more natural way.  I use complex exponentials, with the STA pseudoscalar for i, and use the much simpler STA maxwell vacuum equation as the base. \item February 7, 2009 \ref{chap:potentialFourier} Lorentz Gauge Fourier Vacuum potential solutions

Split from the first order treatment. \item February 8, 2009 \ref{chap:PJplaneWave} Plane wave Fourier series solutions to the Maxwell vacuum equation

My first attempt is getting confusing, especially after seeing after the fact that plane wave constraints on the solution are required for the solution to maintain a grade two form.  Summarizes results from the first attempt in a more coherent, albeit denser, form. \item February 13, 2009 \ref{chap:PJstressEnergyLorentz} Lorentz force relation to the energy momentum tensor

Express the energy momentum tensor in terms of the four vector Lorentz force.  This builds on the previous observation that the $T(\gamma_0)$ is related to the work done against the Lorentz force. \item February 17, 2009 \ref{chap:PJenMtensor} Energy momentum tensor relation to Lorentz force

\item February 18, 2009 \ref{chap:PJpoisson} Poisson and retarded Potential Green's functions from Fourier kernels

Work through the details of how to derive the Poisson integral kernel starting with the Fourier transform derived Green's function.  Do the same thing with the wave equation, and produce the retarded and advanced form solutions.  A few years in the works since seeing them in Feynman and wondering where they came from.  Feb 25.  Did a reduction of the 1D forced wave equation's Green function to a difference of unit step functions.  Have to compute derivatives to see if this really works. \item February 26, 2009 \ref{chap:nvolume} Spherical and hyperspherical parametrization

Volume calculations for 1-sphere (circle), 2-sphere (sphere), 3-sphere (hypersphere).  Followup with a calculation of the differential volume element for the hypersphere (ie: Minkowski spaces of signature (+,-,-,-).  Plan to use these results in an attempt to reduce the 4D hyperbolic Green's functions that we get from Fourier transforming Maxwell's equation. \item March 13, 2009 \ref{chap:levi} Levi-Civitica summation identity

A summation identity given in Byron and Fuller, ch 1.  Initial proof with a perl script, then note equivalence to bivector dot product. \item March 18, 2009 \ref{chap:electronRotor} Lorentz force rotor formulation

Time evolution of a particle in a field as a bivector differential equation, solving for the active Lorentz transformation on the rest frame worldline.  Work it out at my own pace in both the GA and tensor formalism.   \item April 15, 2009 \ref{chap:lorentzForcePQA} Lorentz force Lagrangian with conjugate momentum

The Lagrangian can be expressed in a QM like form in terms of a sum of mechanical and electromagnetic momentum, mv + qA/c.  The end result is the same and it works out to just be a factorization of the original Lorentz force covariant Lagrangian. \item April 18, 2009 \ref{chap:biotSavart} Biot Savart Derivation

\item April 20, 2009 \ref{chap:maxwellTensorFromLagrangian} Tensor derivation of non-dual Maxwell equation from Lagrangian

A tensor only derivation. \item April 28, 2009 \ref{chap:PJmultiTaylors} Developing some intuition for Multivariable and Multivector Taylor Series

Explicit expansion and Hessian matrix connection.  Factor out the gradient from the direction derivative for a few different multivector spaces. \item May 23, 2009 \ref{chap:lorentzForceTx} Lorentz boost of Lorentz force equations

My own attempt to walk through the Lorentz transformation of the pair of Lorentz force and power equations, as done in Bohm's 'The Special Theory of Relativity'.  Bohm's text left out a number of details, as well as had a number of sign typos and some dropped terms.  Try to get it right.  Was able to do some of it, but part of the final "the reader can verify bits" have me stumped.  How to do those last bits is not obvious to me, which is likely why Bohm left this out of this pseudo-layman book.  This set of notes starts off with a large digression on how to express and translate from the GA hyperbolic exponential Lorentz boost formulation to the "classical" coordinate and vector representations used in the Bohm text and other places.  My initial reason for writing that up for myself all in one place was that I intended to try the Lorentz force boost procedure of the Bohm text completely in GA form, but I also have not gotten to attempting that.  My goal was to finish the details of the "old-fashioned" way first, but the algebra for that way is so messy I don't see how to do it. \item May 28, 2009 \ref{chap:macroscopicMaxwell} Macroscopic Maxwell's equation

Got my "new" second hand 2nd ed. of Jackson's Classical Electrodynamics in the mail, and got distracted reading the introduction.  Turns out that a trivector "current" term (with basis vectors in the Dirac vector space) to supplement the four-vector current completely summarizes the mess of $B,D,H,E,M,P,J,\rho$ variables nicely in a fashion very similar to the $\grad F = J$ variation of Maxwell's equation for the microscopic case.\item June 1, 2009 \ref{chap:poincareTx} Poincare transformations

A paper used a specific antisymmetric object for linearized Poincare transformations.  Try to figure this out.  Turns out to be a representation of the bivector that encodes the plane of rotation or spacetime boost plane.\item June 5, 2009 \ref{chap:stressEnergyNoethers} Canonical energy momentum tensor and Lagrangian translation

Examine symmetries under translation and spacetime translation and relate to energy and momentum conservation where possible.\item June 17, 2009 \ref{chap:lForceLag2} Comparison of two covariant Lorentz force Lagrangians

The Lorentz force Lagrangian for a single particle can be expressed in a quadratic fashion much like the classical Kinetic energy based Lagrangian.  Compare to the proper time, non quadratic action.\item June 21, 2009 \ref{chap:emVacWave} Wave equation form of Maxwell's equations

Fill in missing details from Jackson, and find the wave equation from Maxwell's equations with and without Geometric Algebra\item June 27, 2009 \ref{chap:frequencyTx} Relativistic Doppler formula

Deriving the Doppler shift result with a Lorentz boost is much simpler than the time dilation argument in wikipedia.\item July 2, 2009 \ref{chap:maxwellVacuum} Space time algebra solutions of the Maxwell equation for discrete frequencies

Exploring vacuum maxwell solutions using Geometric Algebra formalism.  Motivate with fourier transform techniques, and examine the result and constraints required for solution.\item July 17, 2009 \ref{chap:stokesGradeTwo} Stokes theorem applied to vector and bivector fields

Tackle Stokes theorem again.\item July 21, 2009 \ref{chap:stokesNoTensor} Stokes theorem derivation without tensor expansion of the blade

Yet another attack at Stokes theorem.  Finally get one that I like here.\item July 27, 2009 \ref{chap:qmAngularMom} Bivector form of quantum angular momentum operator

Exploring a wedge product formulation of the angular momentum operator in Cartesian and spherical polar representations.  Lots of good stuff here!\item July 30, 2009 \ref{chap:transverseField} Transverse electric and magnetic fields

Coupling between transverse and propagation direction components of wave guide solutions is examined using Geometric Algebra.\item Aug 6, 2009 \ref{chap:transverseWave} Comparing phasor and geometric transverse solutions to the Maxwell equation

Attempting to use the pseudoscalar as the imaginary in a wave equation phasor expression leads to specific results.  Examine these and contrast to scalar imaginary phasors.\item Aug 10, 2009 \ref{chap:covariantMedia} Covariant Maxwell equation in media

Formulate the Maxwell equation in media (from Jackson) without an explicit spacetime split.\item Aug 14, 2009 \ref{chap:radiationGeometry} (INCOMPLETE) Geometry of Maxwell radiation solutions

After having some trouble with pseudoscalar phasor representations of the wave equation, step back and examine the geometry that these require.  Find that the use of $I\zcap$ for the imaginary means that only transverse solutions can be encoded.\item Aug 16, 2009 \ref{chap:L1Associated} Graphical representation of Spherical Harmonics for $l=1$

Observations that the first set of spherical harmonic associated Legendre eigenfunctions have a natural representation as projections from rotated spherical polar rotation points.\item Aug 31, 2009 \ref{chap:rotationGenerator} Generator of rotations in arbitrary dimensions.

Similar to the exponential translation operator, the exponential operator that generates rotations is derived.  Geometric Algebra is used (with an attempt to make this somewhat understandable without a lot of GA background).  Explicit coordinate expansion is also covered, as well as a comparison to how the same derivation technique could be done with matrix only methods.  The results obtained apply to Euclidean and other metrics and also to all dimensions, both 2D and greater or equal to 3D (unlike the cross product form).\item Sept 4, 2009 \ref{chap:rotationCurrents} Translation and rotation Noether field currents.

\item Sept 6, 2009 \ref{chap:bivectorSelect} Bivector grades of the squared angular momentum operator.

\item Sept 13, 2009 \ref{chap:nuclearInteraction} Relativistic classical proton electron interaction.

\item Sept 20, 2009 \ref{chap:sphericalPolarUnit} Spherical Polar unit vectors in exponential form.

\item Sept 22, 2009 \ref{chap:jackson12Dash9} Lorentz force from Lagrangian (non-covariant)

\item Sept 24, 2009 \ref{chap:jackson12Dash1Gauge} Electromagnetic Gauge invariance.

\end{itemize}
