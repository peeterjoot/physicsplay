%
% Copyright � 2012 Peeter Joot.  All Rights Reserved.
% Licenced as described in the file LICENSE under the root directory of this GIT repository.
%

% 
% 
\chapter{Plane wave Fourier series solutions to the Maxwell vacuum equation}\label{chap:PJplaneWave}
\index{Maxwell equation!Fourier series}
%\date{Feb 08, 2009.  planewave.tex}

\section{Motivation}

In \chapcite{PJFourierVacuum} an exploration of spatially periodic solutions to the electrodynamic vacuum equation was performed using a multivector formulation
of a 3D Fourier series.
Here a summary of the results obtained will be presented in a more
coherent fashion, followed by an attempt to build on them.
In particular a complete
description of the field energy and momentum is desired.

A conclusion from the first analysis was that the
orientation of both the electric and magnetic field components
must be perpendicular to the angular velocity and wave number vectors
within the entire spatial volume.  This was a requirement for the field
solutions to retain a bivector grade (STA/Dirac basis).

Here a specific orientation of the Fourier volume so that two of the axis
lie in the direction of the initial time electric and magnetic fields will be
used.  This is expected to simplify the treatment.

Also note that having obtained some results in a first attempt hindsight
now allows a few choices of variables that will be seen to be appropriate.
The natural motivation for any such choices can be found in the initial
treatment.

\subsection{Notation}

Conventions, definitions, and notation used here will largely follow
\chapcite{PJFourierVacuum}.  Also of possible aid in that document is a
a table of symbols and their definitions.

\section{A concise review of results}

\subsection{Fourier series and coefficients}

A notation for a 3D Fourier series for a spatially periodic function and its Fourier coefficients was developed

\begin{equation}\label{eqn:planewave:20}
\begin{aligned}
f(\Bx) &= \sum_{\Bk} \hat{f}_{\Bk} e^{ - i \Bk \cdot \Bx } \\
\hat{f}_{\Bk} &= \inv{V} \int f(\Bx) e^{ i \Bk \cdot \Bx } d^3 x
\end{aligned}
\end{equation}

In the vector context \(\Bk\) is

\begin{equation}\label{eqn:planewave:40}
\begin{aligned}
\Bk = 2 \pi \sum_m \sigma^m \frac{k_m}{\lambda_m}
\end{aligned}
\end{equation}

Where \(\lambda_m\) are the dimensions of the volume of integration,
\(V = \lambda_1 \lambda_2 \lambda_3\) is the volume, and
in an index context \(\Bk = \{k_1, k_2, k_3\}\) is a triplet of integers,
positive, negative or zero.

\subsection{Vacuum solution and constraints}

We want to find (STA) bivector solutions \(F\) to the vacuum Maxwell equation

\begin{equation}\label{eqn:planewave:maxwell}
\begin{aligned}
\grad F = \gamma_0 (\partial_0 + \spacegrad) F = 0
\end{aligned}
\end{equation}

We start by assuming a Fourier series solution of the form

\begin{equation}\label{eqn:planewave:60}
\begin{aligned}
F(\Bx,t) &= \sum_{\Bk} \hat{F}_{\Bk}(t) e^{-i \Bk \cdot \Bx}
\end{aligned}
\end{equation}

For a solution term by term identity is required

\begin{equation}\label{eqn:planewave:80}
\begin{aligned}
\PD{t}{} \hat{F}_{\Bk}(t) e^{-i \Bk \cdot \Bx}
&= -c \sigma^m \hat{F}_{\Bk}(t) \PD{x^m}{} \exp\left(-i 2 \pi \frac{k_j x^j}{ \lambda_j}\right) \\
&= i c \Bk \hat{F}_{\Bk}(t) e^{-i \Bk \cdot \Bx}
\end{aligned}
\end{equation}

With \(\Bomega = c \Bk\), we have a simple first order single variable differential equation

\begin{equation}\label{eqn:planewave:100}
\begin{aligned}
\hat{F}_{\Bk}'(t) = i \Bomega \hat{F}_{\Bk}(t)
\end{aligned}
\end{equation}

with solution

\begin{equation}\label{eqn:planewave:120}
\begin{aligned}
\hat{F}_{\Bk}(t) = e^{i \Bomega t} \hat{F}_{\Bk}
\end{aligned}
\end{equation}

Here, the constant was written as \(\hat{F}_{\Bk}\) given prior knowledge that this is will be the Fourier coefficient of the
initial time field.  Our assumed solution is now

\begin{equation}\label{eqn:planewave:assummedSolutionNewStartingPoint}
\begin{aligned}
F(\Bx,t) &= \sum_{\Bk} e^{i \Bomega t} \hat{F}_{\Bk} e^{-i \Bk \cdot \Bx}
\end{aligned}
\end{equation}

Observe that for \(t = 0\), we have

\begin{equation}\label{eqn:planewave:140}
\begin{aligned}
F(\Bx,0) &= \sum_{\Bk} \hat{F}_{\Bk} e^{-i \Bk \cdot \Bx}
\end{aligned}
\end{equation}

which is confirmation of the Fourier coefficient role of \(\hat{F}_{\Bk}\)

\begin{equation}\label{eqn:planewave:160}
\begin{aligned}
\hat{F}_{\Bk} &= \inv{V} \int F(\Bx', 0) e^{ i \Bk \cdot \Bx' } d^3 x'
\end{aligned}
\end{equation}

\begin{equation}\label{eqn:planewave:solutionFromInitialConditions}
\begin{aligned}
F(\Bx,t) &= \inv{V} \sum_{\Bk} \int e^{i \Bomega t} F(\Bx', 0) e^{i \Bk \cdot (\Bx'-\Bx) } d^3 x'
\end{aligned}
\end{equation}

It is straightforward to show that \(F(\Bx, 0)\), and pseudoscalar exponentials commute.  Specifically we have

\begin{equation}\label{eqn:planewave:180}
\begin{aligned}
F e^{ i \Bk \cdot \Bx } = e^{ i \Bk \cdot \Bx } F
\end{aligned}
\end{equation}

This follows from the (STA) bivector nature of \(F\).

Another commutativity relation of note is between our time phase exponential and the pseudoscalar exponentials.  This one is also straightforward to show
and will not be done again here

\begin{equation}\label{eqn:planewave:200}
\begin{aligned}
e^{ i \Bomega t} e^{ i \Bk \cdot \Bx } = e^{ i \Bk \cdot \Bx } e^{ i \Bomega t}
\end{aligned}
\end{equation}

Lastly, and most importantly of the commutativity relations,
it was also found that the initial field \(F(\Bx,0)\) must have both electric and magnetic field components perpendicular to all \(\Bomega \propto \Bk\) at all points
\(\Bx\) in the integration volume.
This was because the vacuum Maxwell equation \eqnref{eqn:planewave:maxwell} by itself does not impose any grade requirement on the solution in isolation.  An
additional requirement that the solution have bivector only values imposes this inherent planar nature in a charge free region, at least for solutions
with spatial periodicity.  Some revisiting of previous Fourier transform solutions attempts at the vacuum equation is required since similar constraints are
expected there too.

The planar constraint can be expressed in terms of dot products of the field components, but an alternate way of expressing the same thing was seen to be
a statement of conjugate commutativity between this dual spatial vector exponential and the complete field

\begin{equation}\label{eqn:planewave:restriction}
\begin{aligned}
e^{ i \Bomega t} F &= F e^{ -i \Bomega t}
\end{aligned}
\end{equation}

The set of Fourier coefficients considered in the sum must be restricted to those values that \eqnref{eqn:planewave:restriction} holds.  An effective
way to achieve this is to
pick a specific orientation of the coordinate system so the angular
velocity bivector is quantized in the same plane as the field.  This means that
the angular velocity takes on integer multiples \(k\) of this value

\begin{equation}\label{eqn:planewave:220}
\begin{aligned}
i \Bomega_k = 2 \pi i c k \frac{\Bsigma}{\lambda}
\end{aligned}
\end{equation}

Here \(\Bsigma\) is a unit vector describing the perpendicular to the plane of the field, or equivalently via a duality relationship \(i \Bsigma\) is a unit bivector with the same orientation as the field.

\subsection{Conjugate operations}

In order to tackle expansion of energy and momentum in terms of Fourier coefficients, some conjugation operations will be required.

Such a conjugation is found when computing electric and magnetic field components and also in the \(T(\gamma_0) \propto F \gamma_0 F\) energy
momentum four vector.  In both cases it involves products with \(\gamma_0\).

\subsection{Electric and magnetic fields}

From the total field one can
obtain the electric and magnetic fields via coordinates as in

%\begin{align*}
%F = E^m \sigma_m + i H^m \sigma_m
%\end{align*}
%
%From which we could write
\begin{equation}\label{eqn:planewave:240}
\begin{aligned}
%E^m &= F \cdot \sigma_m \\
%H^m &= (-i F) \cdot \sigma_m
\EE &= \sigma^m (F \cdot \sigma_m) \\
\HH &= \sigma^m ((-i F) \cdot \sigma_m)
\end{aligned}
\end{equation}

However, due to the conjugation effect of \(\gamma_0\)
(a particular observer's time basis vector)
on \(F\),
we can compute the electric and magnetic field components without resorting to coordinates

\begin{equation}\label{eqn:planewave:260}
\begin{aligned}
\EE &= \inv{2}(F - \gamma_0 F \gamma_0 ) \\
\HH &= \inv{2i}(F + \gamma_0 F \gamma_0 )
\end{aligned}
\end{equation}

Such a split is expected to show up when examining the energy and momentum of our Fourier expressed field in detail.
%Observe the conjugation effect of the multiplications with \(\gamma_0\).

\subsection{Conjugate effects on the exponentials}

Now, since \(\gamma_0\) anticommutes with \(i\) we have a conjugation operation on percolation of \(\gamma_0\) through the products of an exponential

\begin{equation}\label{eqn:planewave:280}
\begin{aligned}
\gamma_0 e^{i \Bk \cdot \Bx} = e^{-i \Bk \cdot \Bx} \gamma_0
\end{aligned}
\end{equation}

However, since \(\gamma_0\) also anticommutes with any spatial basis vector \(\sigma_k = \gamma_k \gamma_0\), we have for a dual spatial vector exponential

\begin{equation}\label{eqn:planewave:300}
\begin{aligned}
\gamma_0 e^{i \Bomega t} &= e^{i \Bomega t} \gamma_0
\end{aligned}
\end{equation}

We should now be armed to consider the energy momentum questions that were the desired goal of the initial treatment.

\section{Plane wave Energy and Momentum in terms of Fourier coefficients}

\subsection{Energy momentum four vector}
\index{energy momentum}

To obtain the energy component \(U\) of the energy momentum four
vector (given here in CGS units)

\begin{equation}\label{eqn:planewave:320}
\begin{aligned}
T(\gamma_0) &= \inv{8\pi} F \gamma_0 \tilde{F} = \frac{-1}{8\pi}(F \gamma_0 F)
\end{aligned}
\end{equation}

we want a calculation of the field energy for the plane wave solutions of Maxwell's equation

\begin{equation}\label{eqn:planewave:340}
\begin{aligned}
U
&= T(\gamma_0) \cdot \gamma_0 \\
&= -\inv{16 \pi} ( F \gamma_0 F \gamma_0 + \gamma_0 F \gamma_0 F )
\end{aligned}
\end{equation}

Given the
observed commutativity relationships, at
least some parts of this calculation can be performed by direct multiplication of
\eqnref{eqn:planewave:solutionFromInitialConditions} summed over two sets of wave
number vector indices as in.

\begin{equation}\label{eqn:planewave:360}
\begin{aligned}
F(\Bx,t)
&= \inv{V} \sum_{\Bk} \int e^{i \Bomega_k t + i \Bk \cdot (\Ba-\Bx) } F(\Ba, 0) d^3 a \\
%&= \inv{V} \sum_{\Bk} \int F(\Ba, 0) e^{-i \Bomega_k t + i \Bk \cdot (\Ba-\Bx) } d^3 a \\
&= \inv{V} \sum_{\Bm} \int e^{i \Bomega_m t + i \Bm \cdot (\Bb-\Bx) } F(\Bb, 0) d^3 b
%&= \inv{V} \sum_{\Bm} \int F(\Bb,0) e^{-i \Bomega_m t + i \Bm \cdot (\Bb-\Bx) } F(\Bb, 0) d^3 b
\end{aligned}
\end{equation}

However, this gets messy fast.  Looking for an alternate approach requires some mechanism for encoding the effect
of the \(\gamma_0\) sandwich on the Fourier coefficients of the field bivector.  It has been observed that this operation has a conjugate
effect.  The form of the stress energy four vector suggests that a natural conjugate definition will be

\begin{equation}\label{eqn:planewave:380}
\begin{aligned}
F^\dagger &= \gamma_0 \tilde{F} \gamma_0
\end{aligned}
\end{equation}

where \(\tilde{F}\) is the multivector reverse operation.

This notation for conjugation is in fact what 
, for Quantum Mechanics, \citep{doran2003gap} calls the Hermitian adjoint.

In this form our stress energy vector is

\begin{equation}\label{eqn:planewave:400}
\begin{aligned}
T(\gamma_0) &= \inv{8 \pi} F F^\dagger \gamma_0
\end{aligned}
\end{equation}

While the trailing \(\gamma_0\) term here may look a bit out of place,
the energy density and the Poynting vector end up with a very complementary structure

\begin{equation}\label{eqn:planewave:420}
\begin{aligned}
U &= \inv{16 \pi} \left(F F^\dagger + (F F^\dagger)^{\tilde{}} \right) \\
\BP &= \inv{16 \pi c} \left(F F^\dagger - (F F^\dagger)^{\tilde{}} \right)
\end{aligned}
\end{equation}

Having this conjugate operation defined it can also be applied to the
spacetime split of the electric and the magnetic fields.  That can also now be written in a form
that calls out the inherent complex nature of the fields

\begin{equation}\label{eqn:planewave:440}
\begin{aligned}
\EE &= \inv{2}(F + F^\dagger ) \\
\HH &= \inv{2i}(F - F^\dagger )
\end{aligned}
\end{equation}

\subsection{Aside.  Applications for the conjugate in non-QM contexts}

Despite the existence of the QM notation, it does not appear used in the text or ptIII notes outside of that context.
For example,
in addition to the
stress energy tensor and the spacetime split of the fields, an additional non-QM example
where the conjugate operation could be used, is in the ptIII hout8 where Rotors that satisfy

\begin{equation}\label{eqn:planewave:460}
\begin{aligned}
v \cdot \gamma_0 = \gpgradezero{\gamma_0 R \gamma_0 \tilde{R}} = \gpgradezero{R^\dagger R} > 0
\end{aligned}
\end{equation}

are called proper orthochronous.  There are likely other places involving a time centric projections where this
conjugation operator would have a natural fit.


\subsection{Energy density. Take II}
\index{energy density}

For the Fourier coefficient energy calculation we now take \eqnref{eqn:planewave:assummedSolutionNewStartingPoint} as the starting point.

We will need the conjugate of the field

\begin{equation}\label{eqn:planewave:480}
\begin{aligned}
F^\dagger 
&= \gamma_0 
\left(
\sum_{\Bk} 
e^{i \Bomega t}
\hat{F}_{\Bk}
e^{-i \Bk \cdot \Bx}
\right)
^{\tilde{}}
\gamma_0 \\
&= \gamma_0 \sum_{\Bk} 
(e^{-i \Bk \cdot \Bx})^{\tilde{}}
(-\hat{F}_{\Bk})
(e^{i \Bomega t})^{\tilde{}}
\gamma_0 \\
&= -\gamma_0 \sum_{\Bk} 
e^{-i \Bk \cdot \Bx}
\hat{F}_{\Bk}
e^{-i \Bomega t}
\gamma_0 \\
&= -\sum_{\Bk} 
e^{i \Bk \cdot \Bx}
\gamma_0
\hat{F}_{\Bk}
\gamma_0
e^{-i \Bomega t}
\\
\end{aligned}
\end{equation}
% reverse i \Bk = \Bk~ i~ = -\Bk i = -i\Bk

This is
\begin{equation}\label{eqn:planewave:conjugateField}
\begin{aligned}
F^\dagger 
&= \sum_{\Bk} 
e^{i \Bk \cdot \Bx}
(\hat{F}_{\Bk})^{\dagger}
e^{-i \Bomega t}
\end{aligned}
\end{equation}

So for the energy we have

\begin{equation}\label{eqn:planewave:500}
\begin{aligned}
FF^\dagger + F^\dagger F
&= 
\sum_{\Bm, \Bk} 
e^{i \Bomega_m t}
\hat{F}_{\Bm}
%e^{-i \Bm \cdot \Bx}
e^{i (\Bk-\Bm) \cdot \Bx}
(\hat{F}_{\Bk})^{\dagger}
e^{-i \Bomega_k t}
+
e^{i \Bk \cdot \Bx}
(\hat{F}_{\Bk})^{\dagger}
%e^{-i \Bomega_k t}
e^{i (\Bomega_m-\Bomega_k) t}
\hat{F}_{\Bm}
e^{-i \Bm \cdot \Bx} 
\\
&= 
\sum_{\Bm, \Bk} 
e^{i \Bomega_m t}
\hat{F}_{\Bm}
(\hat{F}_{\Bk})^{\dagger}
e^{i (\Bk-\Bm) \cdot \Bx -i \Bomega_k t}
+
e^{i \Bk \cdot \Bx}
(\hat{F}_{\Bk})^{\dagger}
\hat{F}_{\Bm}
e^{-i (\Bomega_m-\Bomega_k) t -i \Bm \cdot \Bx}
\\
&= 
\sum_{\Bm, \Bk} 
\hat{F}_{\Bm}
(\hat{F}_{\Bk})^{\dagger}
e^{i (\Bk-\Bm) \cdot \Bx -i (\Bomega_k-\Bomega_m) t}
+
(\hat{F}_{\Bk})^{\dagger}
\hat{F}_{\Bm}
e^{i (\Bomega_k-\Bomega_m) t +i (\Bk -\Bm) \cdot \Bx}
\\
&= 
\sum_{\Bk} 
\hat{F}_{\Bk}
(\hat{F}_{\Bk})^{\dagger}
+
(\hat{F}_{\Bk})^{\dagger}
\hat{F}_{\Bk} \\
&\quad+ 
\sum_{\Bm \ne \Bk} 
\hat{F}_{\Bm}
(\hat{F}_{\Bk})^{\dagger}
e^{i (\Bk-\Bm) \cdot \Bx -i (\Bomega_k-\Bomega_m) t}
+
(\hat{F}_{\Bk})^{\dagger}
\hat{F}_{\Bm}
e^{i (\Bomega_k-\Bomega_m) t +i (\Bk -\Bm) \cdot \Bx}
\\
\end{aligned}
\end{equation}

In the first sum all the time dependence and all the spatial dependence 
that is not embedded in the Fourier coefficients themselves has been eliminated.
What is left is something that looks like it is a real quantity (to be verified)
Assuming (also to be verified) that \(\hat{F}_{\Bk}\) commutes with its conjugate
we have something that looks 
like a discrete version of what \citep{haykin1994cs} calls
the Rayleigh energy theorem

\begin{equation}\label{eqn:planewave:520}
\begin{aligned}
\IIinf f(x)f^\conj(x) dx &= \IIinf \hat{f}(k)\hat{f}^\conj(k) dk
\end{aligned}
\end{equation}

Here \(\hat{f}(k)\) is the Fourier transform of \(f(x)\).

Before going on it is expected that
the \(\Bk \ne \Bm\) terms all cancel.
Having restricted the orientations of the allowed angular velocity bivectors to scalar multiples of the plane formed by the (wedge of) the electric
and magnetic fields, we have only a single set of indices to sum over (ie: \(\Bk = 2 \pi \sigma k/ \lambda\)).
In particular we can sum over \(k < m\), and \(k > m\) cases separately and add these
with expectation of cancellation.  Let us see if this works out.

Write \(\Bomega = 2 \pi \sigma / \lambda\), \(\Bomega_k = k \Bomega\), and \(\Bk = \Bomega/c\) then we have for these terms

\begin{equation}\label{eqn:planewave:540}
\begin{aligned}
%\sum_{\{m < k \} \cup \{ k < m \}} &
\sum_{\Bm \ne \Bk} 
e^{ i (k-m) \Bomega \cdot \Bx/c } \left( 
\hat{F}_{\Bm} (\hat{F}_{\Bk})^{\dagger} e^{ -i (k-m) \Bomega t }
+
(\hat{F}_{\Bk})^{\dagger} \hat{F}_{\Bm} e^{ i (k-m)\Bomega t }
\right)
\\
\end{aligned}
\end{equation}

\subsubsection{Hermitian conjugate identities}
\index{Hermitian conjugate}

To get comfortable with the required manipulations, let us 
find the Hermitian conjugate equivalents to some of the familiar complex number relationships.

Not all of these will be the same as in ``normal'' complex numbers.  For instance, while for complex numbers, the identities

\begin{equation}\label{eqn:planewave:560}
\begin{aligned}
z + \overbar{z} &= 2 \Re(z) \\
\inv{i}( z - \overbar{z} ) &= 2 \Im(z)
\end{aligned}
\end{equation}

are both real numbers, we have seen for the electric and magnetic fields that we do not get scalars from the Hermitian conjugates, instead get a spatial vector where we would get a real number in complex arithmetic.  Similarly we get a (bi)vector in the dual
space for the field minus its conjugate.

Some properties:
\begin{itemize}
\item Hermitian conjugate of a product

\begin{equation}\label{eqn:planewave:580}
\begin{aligned}
(ab)^\dagger 
&= \gamma_0 (ab)^{\tilde{}} \gamma_0 \\
&= \gamma_0 
(b)^{\tilde{}}
(a)^{\tilde{}}
\gamma_0 \\
&= \left(\gamma_0 (b)^{\tilde{}} \gamma0\right) \left(\gamma_0 (a)^{\tilde{}} \gamma_0\right) \\
\end{aligned}
\end{equation}

This is our familiar conjugate of a product is the inverted order product of conjugates.
\begin{equation}\label{eqn:planewave:600}
\begin{aligned}
(ab)^\dagger &= b^\dagger a^\dagger 
\end{aligned}
\end{equation}

\item conjugate of a pure pseudoscalar exponential

\begin{equation}\label{eqn:planewave:620}
\begin{aligned}
\left(e^{i\alpha}\right)^\dagger
&=
\gamma_0
\left(
\cos(\alpha)
+ i \sin(\alpha)
\right)^{\tilde{}}
\gamma_0 \\
&=
\cos(\alpha)
- i 
\gamma_0
\sin(\alpha)
\gamma_0 \\
\end{aligned}
\end{equation}

But that is just
\begin{equation}\label{eqn:planewave:640}
\begin{aligned}
\left(e^{i\alpha}\right)^\dagger &= e^{-i\alpha}
\end{aligned}
\end{equation}

Again in sync with complex analysis.  Good.

\item conjugate of a dual spatial vector exponential
\begin{equation}\label{eqn:planewave:660}
\begin{aligned}
\left(e^{i\Bk}\right)^\dagger
&=
\gamma_0
\left(
\cos(\Bk)
+ i \sin(\Bk)
\right)^{\tilde{}}
\gamma_0 \\
&=
\gamma_0
\left(
\cos(\Bk)
- \sin(\Bk) i
\right)
\gamma_0 \\
&=
\cos(\Bk)
- i\sin(\Bk) 
\end{aligned}
\end{equation}

So, we have

\begin{equation}\label{eqn:planewave:680}
\begin{aligned}
\left(e^{i\Bk}\right)^\dagger &= e^{-i\Bk}
\end{aligned}
\end{equation}

Again, consistent with complex numbers for this type of multivector object.

\item dual spatial vector exponential product with a conjugate.

\begin{equation}\label{eqn:planewave:700}
\begin{aligned}
F^\dagger  e^{i\Bk} 
&= \gamma_0 \tilde{F} \gamma_0 e^{i\Bk} \\
&= \gamma_0 \tilde{F} e^{i\Bk} \gamma_0 \\
&= \gamma_0 e^{-i\Bk} \tilde{F} \gamma_0 \\
&= e^{i\Bk} \gamma_0 \tilde{F} \gamma_0 \\
\end{aligned}
\end{equation}

So we have conjugate commutation for both the field and its conjugate
\begin{equation}\label{eqn:planewave:720}
\begin{aligned}
F^\dagger e^{i\Bk} &= e^{-i\Bk} F^\dagger \\
F e^{i\Bk} &= e^{-i\Bk} F
\end{aligned}
\end{equation}

\item pseudoscalar exponential product with a conjugate.

For scalar \(\alpha\)

\begin{equation}\label{eqn:planewave:740}
\begin{aligned}
F^\dagger  e^{i\alpha} 
&= \gamma_0 \tilde{F} \gamma_0 e^{i\alpha} \\
&= \gamma_0 \tilde{F} e^{-i\alpha} \gamma_0 \\
&= \gamma_0 e^{-i\alpha} \tilde{F} \gamma_0 \\
&= e^{i\alpha} \gamma_0 \tilde{F} \gamma_0 \\
\end{aligned}
\end{equation}

In opposition to the dual spatial vector exponential, the plain old pseudoscalar exponentials commute with 
both the field and its conjugate.

\begin{equation}\label{eqn:planewave:760}
\begin{aligned}
F^\dagger e^{i\alpha} &= e^{i\alpha} F^\dagger \\
F e^{i\alpha} &= e^{i\alpha} F
\end{aligned}
\end{equation}

\item Pauli vector conjugate.

\begin{equation}\label{eqn:planewave:780}
\begin{aligned}
(\sigma_k)^\dagger &= \gamma_0 \gamma_0 \gamma_k \gamma_0 = \sigma_k
\end{aligned}
\end{equation}

Jives with the fact that these in matrix form are called Hermitian.

\item pseudoscalar conjugate.

\begin{equation}\label{eqn:planewave:800}
\begin{aligned}
i^\dagger = \gamma_{0} i \gamma_0 = -i
\end{aligned}
\end{equation}

\item Field Fourier coefficient conjugate.

\begin{equation}\label{eqn:planewave:820}
\begin{aligned}
(\hat{F}_{\Bk})^\dagger 
&=
\inv{V} \int e^{-i \Bk \cdot \Bx } F^\dagger(\Bx,0) d^3x = \widehat{F^\dagger}_{-\Bk}
\end{aligned}
\end{equation}

The conjugate of the \(\Bk\) Fourier coefficient is the \(-\Bk\) Fourier coefficient of the conjugate field.

\end{itemize}

Observe that the first three of these properties would have allowed for calculation of 
\eqnref{eqn:planewave:conjugateField} by inspection.

\subsection{Products of Fourier coefficient with another conjugate coefficient}

To progress a relationship between the conjugate products of Fourier coefficients may be required.

\section{FIXME: finish this}

I am getting tired of trying to show (using Latex as a tool and also on paper)
that the \(\Bk \ne \Bm\) terms vanish and am going to take a break, and move on for a bit.  Come back to this later, but start
with a electric field and magnetic field expansion of the $
(\hat{F}_\Bk)^\dagger \hat{F}_\Bk
+
\hat{F}_\Bk (\hat{F}_\Bk)^\dagger 
$ term to verify that this ends up being a scalar as desired and expected
(this is perhaps an easier first step than showing the cross terms are zero).
