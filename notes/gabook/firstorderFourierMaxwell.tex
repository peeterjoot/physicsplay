\documentclass{article}

\usepackage{amsmath}
\usepackage{mathpazo}

%
% shorthand for bold symbols, convenient for vectors and matrices
%
\newcommand{\Ba}[0]{\mathbf{a}}
\newcommand{\Bb}[0]{\mathbf{b}}
\newcommand{\Bc}[0]{\mathbf{c}}
\newcommand{\Bd}[0]{\mathbf{d}}
\newcommand{\Be}[0]{\mathbf{e}}
\newcommand{\Bf}[0]{\mathbf{f}}
\newcommand{\Bg}[0]{\mathbf{g}}
\newcommand{\Bh}[0]{\mathbf{h}}
\newcommand{\Bi}[0]{\mathbf{i}}
\newcommand{\Bj}[0]{\mathbf{j}}
\newcommand{\Bk}[0]{\mathbf{k}}
\newcommand{\Bl}[0]{\mathbf{l}}
\newcommand{\Bm}[0]{\mathbf{m}}
\newcommand{\Bn}[0]{\mathbf{n}}
\newcommand{\Bo}[0]{\mathbf{o}}
\newcommand{\Bp}[0]{\mathbf{p}}
\newcommand{\Bq}[0]{\mathbf{q}}
\newcommand{\Br}[0]{\mathbf{r}}
\newcommand{\Bs}[0]{\mathbf{s}}
\newcommand{\Bt}[0]{\mathbf{t}}
\newcommand{\Bu}[0]{\mathbf{u}}
\newcommand{\Bv}[0]{\mathbf{v}}
\newcommand{\Bw}[0]{\mathbf{w}}
\newcommand{\Bx}[0]{\mathbf{x}}
\newcommand{\By}[0]{\mathbf{y}}
\newcommand{\Bz}[0]{\mathbf{z}}
\newcommand{\BA}[0]{\mathbf{A}}
\newcommand{\BB}[0]{\mathbf{B}}
\newcommand{\BC}[0]{\mathbf{C}}
\newcommand{\BD}[0]{\mathbf{D}}
\newcommand{\BE}[0]{\mathbf{E}}
\newcommand{\BF}[0]{\mathbf{F}}
\newcommand{\BG}[0]{\mathbf{G}}
\newcommand{\BH}[0]{\mathbf{H}}
\newcommand{\BI}[0]{\mathbf{I}}
\newcommand{\BJ}[0]{\mathbf{J}}
\newcommand{\BK}[0]{\mathbf{K}}
\newcommand{\BL}[0]{\mathbf{L}}
\newcommand{\BM}[0]{\mathbf{M}}
\newcommand{\BN}[0]{\mathbf{N}}
\newcommand{\BO}[0]{\mathbf{O}}
\newcommand{\BP}[0]{\mathbf{P}}
\newcommand{\BQ}[0]{\mathbf{Q}}
\newcommand{\BR}[0]{\mathbf{R}}
\newcommand{\BS}[0]{\mathbf{S}}
\newcommand{\BT}[0]{\mathbf{T}}
\newcommand{\BU}[0]{\mathbf{U}}
\newcommand{\BV}[0]{\mathbf{V}}
\newcommand{\BW}[0]{\mathbf{W}}
\newcommand{\BX}[0]{\mathbf{X}}
\newcommand{\BY}[0]{\mathbf{Y}}
\newcommand{\BZ}[0]{\mathbf{Z}}

\newcommand{\Bzero}[0]{\mathbf{0}}
\newcommand{\Btheta}[0]{\boldsymbol{\theta}}
\newcommand{\Btau}[0]{\boldsymbol{\tau}}
\newcommand{\Bomega}[0]{\boldsymbol{\omega}}

%
% shorthand for unit vectors
%
\newcommand{\acap}[0]{\hat{\Ba}}
\newcommand{\bcap}[0]{\hat{\Bb}}
\newcommand{\ccap}[0]{\hat{\Bc}}
\newcommand{\dcap}[0]{\hat{\Bd}}
\newcommand{\ecap}[0]{\hat{\Be}}
\newcommand{\fcap}[0]{\hat{\Bf}}
\newcommand{\gcap}[0]{\hat{\Bg}}
\newcommand{\hcap}[0]{\hat{\Bh}}
\newcommand{\icap}[0]{\hat{\Bi}}
\newcommand{\jcap}[0]{\hat{\Bj}}
\newcommand{\kcap}[0]{\hat{\Bk}}
\newcommand{\lcap}[0]{\hat{\Bl}}
\newcommand{\mcap}[0]{\hat{\Bm}}
\newcommand{\ncap}[0]{\hat{\Bn}}
\newcommand{\ocap}[0]{\hat{\Bo}}
\newcommand{\pcap}[0]{\hat{\Bp}}
\newcommand{\qcap}[0]{\hat{\Bq}}
\newcommand{\rcap}[0]{\hat{\Br}}
\newcommand{\scap}[0]{\hat{\Bs}}
\newcommand{\tcap}[0]{\hat{\Bt}}
\newcommand{\ucap}[0]{\hat{\Bu}}
\newcommand{\vcap}[0]{\hat{\Bv}}
\newcommand{\wcap}[0]{\hat{\Bw}}
\newcommand{\xcap}[0]{\hat{\Bx}}
\newcommand{\ycap}[0]{\hat{\By}}
\newcommand{\zcap}[0]{\hat{\Bz}}
\newcommand{\thetacap}[0]{\hat{\Btheta}}

%
% to write R^n and C^n in a distinguishable fashion.  Perhaps change this
% to the double lined characters upon figuring out how to do so.
%
\newcommand{\C}[1]{$\mathbb{C}^{#1}$}
\newcommand{\R}[1]{$\mathbb{R}^{#1}$}

%
% various generally useful helpers
%

% derivative of #1 wrt. #2:
\newcommand{\D}[2] {\frac {d#2} {d#1}}

\newcommand{\inv}[1]{\frac{1}{#1}}
\newcommand{\cross}[0]{\times}

\newcommand{\abs}[1]{\lvert{#1}\rvert}
\newcommand{\norm}[1]{\lVert{#1}\rVert}
\newcommand{\innerprod}[2]{\langle{#1}, {#2}\rangle}
\newcommand{\dotprod}[2]{{#1} \cdot {#2}}
\newcommand{\bdotprod}[2]{\left({#1} \cdot {#2}\right)}
\newcommand{\crossprod}[2]{{#1} \cross {#2}}
\newcommand{\tripleprod}[3]{\dotprod{\left(\crossprod{#1}{#2}\right)}{#3}}

\DeclareMathOperator{\Proj}{Proj}
\DeclareMathOperator{\Span}{span}
\DeclareMathOperator{\Sgn}{sgn}
\DeclareMathOperator{\Area}{Area}
\DeclareMathOperator{\Volume}{Volume}

%
% A few miscellaneous things specific to this document
%
\newcommand{\crossop}[1]{\crossprod{#1}{}}

% R2 vector.
\newcommand{\VectorTwo}[2]{
\begin{bmatrix}
 {#1} \\
 {#2}
\end{bmatrix}
}

\newcommand{\VectorN}[1]{
\begin{bmatrix}
{#1}_1 \\
{#1}_2 \\
\vdots \\
{#1}_N \\
\end{bmatrix}
}

\newcommand{\DETuvij}[4]{
\begin{vmatrix}
 {#1}_{#3} & {#1}_{#4} \\
 {#2}_{#3} & {#2}_{#4}
\end{vmatrix}
}

\newcommand{\DETuvwijk}[6]{
\begin{vmatrix}
 {#1}_{#4} & {#1}_{#5} & {#1}_{#6} \\
 {#2}_{#4} & {#2}_{#5} & {#2}_{#6} \\
 {#3}_{#4} & {#3}_{#5} & {#3}_{#6}
\end{vmatrix}
}

\newcommand{\DETuvwxijkl}[8]{
\begin{vmatrix}
 {#1}_{#5} & {#1}_{#6} & {#1}_{#7} & {#1}_{#8} \\
 {#2}_{#5} & {#2}_{#6} & {#2}_{#7} & {#2}_{#8} \\
 {#3}_{#5} & {#3}_{#6} & {#3}_{#7} & {#3}_{#8} \\
 {#4}_{#5} & {#4}_{#6} & {#4}_{#7} & {#4}_{#8} \\
\end{vmatrix}
}

%\newcommand{\DETuvwxyijklm}[10]{
%\begin{vmatrix}
% {#1}_{#6} & {#1}_{#7} & {#1}_{#8} & {#1}_{#9} & {#1}_{#10} \\
% {#2}_{#6} & {#2}_{#7} & {#2}_{#8} & {#2}_{#9} & {#2}_{#10} \\
% {#3}_{#6} & {#3}_{#7} & {#3}_{#8} & {#3}_{#9} & {#3}_{#10} \\
% {#4}_{#6} & {#4}_{#7} & {#4}_{#8} & {#4}_{#9} & {#4}_{#10} \\
% {#5}_{#6} & {#5}_{#7} & {#5}_{#8} & {#5}_{#9} & {#5}_{#10}
%\end{vmatrix}
%}

% R3 vector.
\newcommand{\VectorThree}[3]{
\begin{bmatrix}
 {#1} \\
 {#2} \\
 {#3}
\end{bmatrix}
}


%<misc>
%
\newcommand{\Abs}[1]{{\left\lvert{#1}\right\rvert}}
\newcommand{\spacegrad}[0]{\boldsymbol{\nabla}}
\newcommand{\grad}[0]{\nabla}
\newcommand{\LL}[0]{\mathcal{L}}

% == \partial_{#1} {#2}
\newcommand{\PD}[2]{\frac{\partial {#2}}{\partial {#1}}}
% inline variant
\newcommand{\PDi}[2]{{\partial {#2}}/{\partial {#1}}}

\newcommand{\PDD}[3]{\frac{\partial^2 {#3}}{\partial {#1}\partial {#2}}}
%\newcommand{\PDd}[2]{\frac{\partial^2 {#2}}{{\partial{#1}}^2}}
\newcommand{\PDsq}[2]{\frac{\partial^2 {#2}}{(\partial {#1})^2}}

\newcommand{\Partial}[2]{\frac{\partial {#1}}{\partial {#2}}}
\DeclareMathOperator{\RejName}{Rej}
\newcommand{\Rej}[2]{\RejName_{#1}\left( {#2} \right)}
\newcommand{\Rm}[1]{\mathbb{R}^{#1}}
\newcommand{\Cm}[1]{\mathbb{C}^{#1}}
\newcommand{\conj}[0]{{*}}

%</misc>

% <grade selection>
%
\newcommand{\gpgrade}[2] {{\left\langle{{#1}}\right\rangle}_{#2}}

\newcommand{\gpgradezero}[1] {\gpgrade{#1}{}}
%\newcommand{\gpscalargrade}[1] {{\left\langle{{#1}}\right\rangle}}
%\newcommand{\gpgradezero}[1] {\gpgrade{#1}{0}}

%\newcommand{\gpgradeone}[1] {{\left\langle{{#1}}\right\rangle}_{1}}
\newcommand{\gpgradeone}[1] {\gpgrade{#1}{1}}

\newcommand{\gpgradetwo}[1] {\gpgrade{#1}{2}}
\newcommand{\gpgradethree}[1] {\gpgrade{#1}{3}}
\newcommand{\gpgradefour}[1] {\gpgrade{#1}{4}}
%
% </grade selection>



\newcommand{\adot}[0]{{\dot{a}}}
\newcommand{\bdot}[0]{{\dot{b}}}
% taken for centered dot:
%\newcommand{\cdot}[0]{{\dot{c}}}
%\newcommand{\ddot}[0]{{\dot{d}}}
\newcommand{\edot}[0]{{\dot{e}}}
\newcommand{\fdot}[0]{{\dot{f}}}
\newcommand{\gdot}[0]{{\dot{g}}}
\newcommand{\hdot}[0]{{\dot{h}}}
\newcommand{\idot}[0]{{\dot{i}}}
\newcommand{\jdot}[0]{{\dot{j}}}
\newcommand{\kdot}[0]{{\dot{k}}}
\newcommand{\ldot}[0]{{\dot{l}}}
\newcommand{\mdot}[0]{{\dot{m}}}
\newcommand{\ndot}[0]{{\dot{n}}}
%\newcommand{\odot}[0]{{\dot{o}}}
\newcommand{\pdot}[0]{{\dot{p}}}
\newcommand{\qdot}[0]{{\dot{q}}}
\newcommand{\rdot}[0]{{\dot{r}}}
\newcommand{\sdot}[0]{{\dot{s}}}
\newcommand{\tdot}[0]{{\dot{t}}}
\newcommand{\udot}[0]{{\dot{u}}}
\newcommand{\vdot}[0]{{\dot{v}}}
\newcommand{\wdot}[0]{{\dot{w}}}
\newcommand{\xdot}[0]{{\dot{x}}}
\newcommand{\ydot}[0]{{\dot{y}}}
\newcommand{\zdot}[0]{{\dot{z}}}
\newcommand{\addot}[0]{{\ddot{a}}}
\newcommand{\bddot}[0]{{\ddot{b}}}
\newcommand{\cddot}[0]{{\ddot{c}}}
%\newcommand{\dddot}[0]{{\ddot{d}}}
\newcommand{\eddot}[0]{{\ddot{e}}}
\newcommand{\fddot}[0]{{\ddot{f}}}
\newcommand{\gddot}[0]{{\ddot{g}}}
\newcommand{\hddot}[0]{{\ddot{h}}}
\newcommand{\iddot}[0]{{\ddot{i}}}
\newcommand{\jddot}[0]{{\ddot{j}}}
\newcommand{\kddot}[0]{{\ddot{k}}}
\newcommand{\lddot}[0]{{\ddot{l}}}
\newcommand{\mddot}[0]{{\ddot{m}}}
\newcommand{\nddot}[0]{{\ddot{n}}}
\newcommand{\oddot}[0]{{\ddot{o}}}
\newcommand{\pddot}[0]{{\ddot{p}}}
\newcommand{\qddot}[0]{{\ddot{q}}}
\newcommand{\rddot}[0]{{\ddot{r}}}
\newcommand{\sddot}[0]{{\ddot{s}}}
\newcommand{\tddot}[0]{{\ddot{t}}}
\newcommand{\uddot}[0]{{\ddot{u}}}
\newcommand{\vddot}[0]{{\ddot{v}}}
\newcommand{\wddot}[0]{{\ddot{w}}}
\newcommand{\xddot}[0]{{\ddot{x}}}
\newcommand{\yddot}[0]{{\ddot{y}}}
\newcommand{\zddot}[0]{{\ddot{z}}}

%<bold and dot greek symbols>
%

\newcommand{\Deltadot}[0]{{\dot{\Delta}}}
\newcommand{\Gammadot}[0]{{\dot{\Gamma}}}
\newcommand{\Lambdadot}[0]{{\dot{\Lambda}}}
\newcommand{\Omegadot}[0]{{\dot{\Omega}}}
\newcommand{\Phidot}[0]{{\dot{\Phi}}}
\newcommand{\Pidot}[0]{{\dot{\Pi}}}
\newcommand{\Psidot}[0]{{\dot{\Psi}}}
\newcommand{\Sigmadot}[0]{{\dot{\Sigma}}}
\newcommand{\Thetadot}[0]{{\dot{\Theta}}}
\newcommand{\Upsilondot}[0]{{\dot{\Upsilon}}}
\newcommand{\Xidot}[0]{{\dot{\Xi}}}
\newcommand{\alphadot}[0]{{\dot{\alpha}}}
\newcommand{\betadot}[0]{{\dot{\beta}}}
\newcommand{\chidot}[0]{{\dot{\chi}}}
\newcommand{\deltadot}[0]{{\dot{\delta}}}
\newcommand{\epsilondot}[0]{{\dot{\epsilon}}}
\newcommand{\etadot}[0]{{\dot{\eta}}}
\newcommand{\gammadot}[0]{{\dot{\gamma}}}
\newcommand{\kappadot}[0]{{\dot{\kappa}}}
\newcommand{\lambdadot}[0]{{\dot{\lambda}}}
\newcommand{\mudot}[0]{{\dot{\mu}}}
\newcommand{\nudot}[0]{{\dot{\nu}}}
\newcommand{\omegadot}[0]{{\dot{\omega}}}
\newcommand{\phidot}[0]{{\dot{\phi}}}
\newcommand{\pidot}[0]{{\dot{\pi}}}
\newcommand{\psidot}[0]{{\dot{\psi}}}
\newcommand{\rhodot}[0]{{\dot{\rho}}}
\newcommand{\sigmadot}[0]{{\dot{\sigma}}}
\newcommand{\taudot}[0]{{\dot{\tau}}}
\newcommand{\thetadot}[0]{{\dot{\theta}}}
\newcommand{\upsilondot}[0]{{\dot{\upsilon}}}
\newcommand{\varepsilondot}[0]{{\dot{\varepsilon}}}
\newcommand{\varphidot}[0]{{\dot{\varphi}}}
\newcommand{\varpidot}[0]{{\dot{\varpi}}}
\newcommand{\varrhodot}[0]{{\dot{\varrho}}}
\newcommand{\varsigmadot}[0]{{\dot{\varsigma}}}
\newcommand{\varthetadot}[0]{{\dot{\vartheta}}}
\newcommand{\xidot}[0]{{\dot{\xi}}}
\newcommand{\zetadot}[0]{{\dot{\zeta}}}

\newcommand{\Deltaddot}[0]{{\ddot{\Delta}}}
\newcommand{\Gammaddot}[0]{{\ddot{\Gamma}}}
\newcommand{\Lambdaddot}[0]{{\ddot{\Lambda}}}
\newcommand{\Omegaddot}[0]{{\ddot{\Omega}}}
\newcommand{\Phiddot}[0]{{\ddot{\Phi}}}
\newcommand{\Piddot}[0]{{\ddot{\Pi}}}
\newcommand{\Psiddot}[0]{{\ddot{\Psi}}}
\newcommand{\Sigmaddot}[0]{{\ddot{\Sigma}}}
\newcommand{\Thetaddot}[0]{{\ddot{\Theta}}}
\newcommand{\Upsilonddot}[0]{{\ddot{\Upsilon}}}
\newcommand{\Xiddot}[0]{{\ddot{\Xi}}}
\newcommand{\alphaddot}[0]{{\ddot{\alpha}}}
\newcommand{\betaddot}[0]{{\ddot{\beta}}}
\newcommand{\chiddot}[0]{{\ddot{\chi}}}
\newcommand{\deltaddot}[0]{{\ddot{\delta}}}
\newcommand{\epsilonddot}[0]{{\ddot{\epsilon}}}
\newcommand{\etaddot}[0]{{\ddot{\eta}}}
\newcommand{\gammaddot}[0]{{\ddot{\gamma}}}
\newcommand{\kappaddot}[0]{{\ddot{\kappa}}}
\newcommand{\lambdaddot}[0]{{\ddot{\lambda}}}
\newcommand{\muddot}[0]{{\ddot{\mu}}}
\newcommand{\nuddot}[0]{{\ddot{\nu}}}
\newcommand{\omegaddot}[0]{{\ddot{\omega}}}
\newcommand{\phiddot}[0]{{\ddot{\phi}}}
\newcommand{\piddot}[0]{{\ddot{\pi}}}
\newcommand{\psiddot}[0]{{\ddot{\psi}}}
\newcommand{\rhoddot}[0]{{\ddot{\rho}}}
\newcommand{\sigmaddot}[0]{{\ddot{\sigma}}}
\newcommand{\tauddot}[0]{{\ddot{\tau}}}
\newcommand{\thetaddot}[0]{{\ddot{\theta}}}
\newcommand{\upsilonddot}[0]{{\ddot{\upsilon}}}
\newcommand{\varepsilonddot}[0]{{\ddot{\varepsilon}}}
\newcommand{\varphiddot}[0]{{\ddot{\varphi}}}
\newcommand{\varpiddot}[0]{{\ddot{\varpi}}}
\newcommand{\varrhoddot}[0]{{\ddot{\varrho}}}
\newcommand{\varsigmaddot}[0]{{\ddot{\varsigma}}}
\newcommand{\varthetaddot}[0]{{\ddot{\vartheta}}}
\newcommand{\xiddot}[0]{{\ddot{\xi}}}
\newcommand{\zetaddot}[0]{{\ddot{\zeta}}}

\newcommand{\BDelta}[0]{\boldsymbol{\Delta}}
\newcommand{\BGamma}[0]{\boldsymbol{\Gamma}}
\newcommand{\BLambda}[0]{\boldsymbol{\Lambda}}
\newcommand{\BOmega}[0]{\boldsymbol{\Omega}}
\newcommand{\BPhi}[0]{\boldsymbol{\Phi}}
\newcommand{\BPi}[0]{\boldsymbol{\Pi}}
\newcommand{\BPsi}[0]{\boldsymbol{\Psi}}
\newcommand{\BSigma}[0]{\boldsymbol{\Sigma}}
\newcommand{\BTheta}[0]{\boldsymbol{\Theta}}
\newcommand{\BUpsilon}[0]{\boldsymbol{\Upsilon}}
\newcommand{\BXi}[0]{\boldsymbol{\Xi}}
\newcommand{\Balpha}[0]{\boldsymbol{\alpha}}
\newcommand{\Bbeta}[0]{\boldsymbol{\beta}}
\newcommand{\Bchi}[0]{\boldsymbol{\chi}}
\newcommand{\Bdelta}[0]{\boldsymbol{\delta}}
\newcommand{\Bepsilon}[0]{\boldsymbol{\epsilon}}
\newcommand{\Beta}[0]{\boldsymbol{\eta}}
\newcommand{\Bgamma}[0]{\boldsymbol{\gamma}}
\newcommand{\Bkappa}[0]{\boldsymbol{\kappa}}
\newcommand{\Blambda}[0]{\boldsymbol{\lambda}}
\newcommand{\Bmu}[0]{\boldsymbol{\mu}}
\newcommand{\Bnu}[0]{\boldsymbol{\nu}}
%\newcommand{\Bomega}[0]{\boldsymbol{\omega}}
\newcommand{\Bphi}[0]{\boldsymbol{\phi}}
\newcommand{\Bpi}[0]{\boldsymbol{\pi}}
\newcommand{\Bpsi}[0]{\boldsymbol{\psi}}
\newcommand{\Brho}[0]{\boldsymbol{\rho}}
\newcommand{\Bsigma}[0]{\boldsymbol{\sigma}}
%\newcommand{\Btau}[0]{\boldsymbol{\tau}}
%\newcommand{\Btheta}[0]{\boldsymbol{\theta}}
\newcommand{\Bupsilon}[0]{\boldsymbol{\upsilon}}
\newcommand{\Bvarepsilon}[0]{\boldsymbol{\varepsilon}}
\newcommand{\Bvarphi}[0]{\boldsymbol{\varphi}}
\newcommand{\Bvarpi}[0]{\boldsymbol{\varpi}}
\newcommand{\Bvarrho}[0]{\boldsymbol{\varrho}}
\newcommand{\Bvarsigma}[0]{\boldsymbol{\varsigma}}
\newcommand{\Bvartheta}[0]{\boldsymbol{\vartheta}}
\newcommand{\Bxi}[0]{\boldsymbol{\xi}}
\newcommand{\Bzeta}[0]{\boldsymbol{\zeta}}
%
%</bold and dot greek symbols>
%<infrequent>
%
%\newcommand{\AreaOp}[1]{\AName_{#1}}
%\newcommand{\Babs}[0]{\abs{\BB}}
%\newcommand{\Bcap}[0]{\hat{\BB}}
%\newcommand{\BrPrimeRej}[0]{\rcap(\rcap \wedge \Br')}
%\newcommand{\CA}[0]{\mathcal{A}}
%\newcommand{\Cos}[1]{\cos{\left({#1}\right)}}
%\newcommand{\Det}[1] {\abs{#1}}
%\newcommand{\Dsq}[2] {\frac {\partial^2 {#1}} {\partial {#2}^2}}
%\newcommand{\Exp}[1]{\exp{\left({#1}\right)}}
%\newcommand{\Norm}[1]{\left\lVert{#1}\right\rVert}
%\newcommand{\Sin}[1]{\sin{\left({#1}\right)}}
%\newcommand{\T}[0]{\text{T}}
%\newcommand{\VolumeOp}[1]{\VName_{#1}}
%\newcommand{\agrad}[0]{\Ba \cdot \nabla}
%\newcommand{\alphacap}[0]{\hat{\boldsymbol{\alpha}}}
%\newcommand{\Fcap}[0]{\hat{\BF}}
%\newcommand{\bithree}[0]{{\Bi}_3}
%\newcommand{\bxa}[0]{\Bx\Ba}
%\newcommand{\coordvec}[2]{
%\newcommand{\costheta}[0]{\acap \cdot \xcap}
%\newcommand{\ddt}[1]{\ddot{#1}}
%\newcommand{\ddu}[1] {\frac {d{#1}} {du}}
%\newcommand{\dsqxj}[2] {\frac {\partial^2 {#1}} {\partial {x_{#2}}^2}}
%\newcommand{\dtheta}[1]{\frac{d {#1}}{d \theta}}
%\newcommand{\dt}[1]{\dot{#1}}
%\newcommand{\dt}[1]{\frac{d {#1}}{dt}}
%\newcommand{\dxj}[2] {\frac {\partial {#1}} {\partial {x_{#2}}}}
%\newcommand{\halfPhi}[0]{\frac{\phi}{2}}
%\newcommand{\half}[0]{\inv{2}}
%\newcommand{\inv}[1]{\frac{1}{#1}}
%\newcommand{\laplacian}[0]{\nabla^2}
%\newcommand{\matrixoftx}[3]{
%\newcommand{\nrrp}[0]{\norm{\rcap \wedge \Br'}}
%\newcommand{\oiint}{\bigcirc \hspace{-1.4em} \int \hspace{-.8em} \int}
%\newcommand{\transpose}[1]{{#1}^{\text{T}}}
%\newcommand{\transpose}[1]{{{#1}^{\TextTranspose}}}
%\newcommand{\transpose}[1]{{{#1}^{\text{T}}}}
%\newcommand{\barA}[0]{\bar{A}}
%\newcommand{\qbar}[0]{\bar{q}}
%\newcommand{\qdotbar}[0]{\dot{\bar{q}}}
%
%</infrequent>





\newcommand{\PDSq}[2]{\frac{\partial^2 {#2}}{\partial {#1}^2}}
\DeclareMathOperator{\sinc}{sinc}
\DeclareMathOperator{\PV}{PV}
\newcommand{\FF}[0]{\mathcal{F}}
\newcommand{\IIinf}[0]{ \int_{-\infty}^\infty }

\usepackage[bookmarks=true]{hyperref}

\usepackage{color,cite,graphicx}
   % use colour in the document, put your citations as [1-4]
   % rather than [1,2,3,4] (it looks nicer, and the extended LaTeX2e
   % graphics package. 
\usepackage{latexsym,amssymb,epsf} % don't remember if these are
   % needed, but their inclusion can't do any damage


\title{ First order Fourier transform solution attempt for Maxwell's equation. }
\author{Peeter Joot}
\date{ Jan 31, 2009.  Last Revision: $Date: 2009/01/31 22:13:12 $ }

\begin{document}

\maketitle{}

\tableofcontents

\section{ Motivation. }

In \cite{PJfourierMaxwellSecondOrder} solutions of Maxwell's equation
via Fourier transformation of the four potential forced wave equations were
explored.

Here a first order solution is attempted, by directly Fourier transforming
the Maxwell's equation in bivector form.

\section{ Setup. }

Again using a 3D spatial fourier transform, we want to put Maxwell's equation into an explicit time dependent form, and can do so by
premultiplying by our observer's time basis vector $\gamma_0$

\begin{align*}
\gamma_0 \grad F &= \gamma_0 \frac{J }{\epsilon_0 c}
\end{align*}

On the left hand side we have
\begin{align*}
\gamma_0 \grad 
&= \gamma_0 \left( \gamma^0 \partial_0 + \gamma^k \partial_k \right) \\
&= \partial_0 - \gamma^k \gamma_0 \partial_k \\
&= \partial_0 + \sigma^k \partial_k \\
&= \partial_0 + \spacegrad \\
\end{align*}

and on the right hand side we have
\begin{align*}
\gamma_0 \frac{J }{\epsilon_0 c}
&= \gamma_0 \frac{c \rho \gamma_0 + J^k \gamma_k }{\epsilon_0 c} \\
&= \frac{c \rho - J^k \sigma_k }{\epsilon_0 c} \\
&= \frac{\rho}{\epsilon_0} - \frac{\Bj}{\epsilon_0 c} \\
\end{align*}

Both the spacetime gradient and the current density four vector have been put in a quaternionic form with scalar and bivector grades in the 
STA basis.  This leaves us with the time centric formulation of Maxwell's equation

\begin{align}
\left(\partial_0 + \spacegrad\right) F &= \frac{\rho}{\epsilon_0} - \frac{\Bj}{\epsilon_0 c} 
\end{align}

Except for the fact that we have objects of various grades here, and that this is a first instead of second order equation,
these equations have the same form as in the previous Fourier transform attacks.
Those were Fourier transform application for the homogeneous and inhomogeneous wave equations, and the heat and 
Schr\"{o}dinger equation.

\section{ Fourier transforming a mixed grade object. }

Now, here we make the assumption that we can apply 3D Fourier transform pairs to mixed grade objects, as in

\begin{align}\label{eqn:FourierTxDefinition}
\hat{\psi}(\Bk, t) &= \inv{(\sqrt{2\pi})^3} \IIinf \psi(\Bx, t) \exp\left( -i \Bk \cdot \Bx \right) d^3 x \\
{\psi}(\Bx, t) &= \PV \inv{(\sqrt{2\pi})^3} \IIinf \hat{\psi}(\Bk, t) \exp\left( i \Bk \cdot \Bx \right) d^3 k
\end{align}

Now, because of linearity, is it clear enough that this will work, provided this is a valid transform pair for any specific grade.
We do however want to be careful of the order of the factors since we want the flexibility to use any particular convienient representation
of $i$, in particular $i = \gamma_0 \gamma_1 \gamma_2 \gamma_3 = \sigma_1 \sigma_2 \sigma_3$.

Let's repeat our an ad-hoc verification that this transform pair works as desired, being careful with the order of products and specifically
allowing for $\psi$ to be a non-scalar function.  
Writing $\Bk = k_m \sigma^m$, $\Br = \sigma_m r^m$, $\Bx = \sigma_m x^m$, that is an expansion of

\begin{align*}
\PV &\inv{(\sqrt{2\pi})^3} \int 
\left( \inv{(\sqrt{2\pi})^3} \int \psi(\Br, t) \exp\left( -i \Bk \cdot \Br \right) d^3 r \right)
\exp\left( i \Bk \cdot \Bx \right) d^3 k \\
&= \int \psi(\Br, t) d^3 r \PV \inv{({2\pi})^3} \int \exp\left( i \Bk \cdot (\Bx -\Br) \right) d^3 k \\
&= \int \psi(\Br, t) d^3 r \Pi_{m=1}^3 \PV \inv{{2\pi}} \int \exp\left( i k_m (x^m -r^m) \right) dk_m \\
&= \int \psi(\Br, t) d^3 r \Pi_{m=1}^3 \lim_{R\rightarrow \infty} \frac{\sin\left( R (x^m -r^m) \right)}{\pi(x^m - r^m)} \\
&\sim \int \psi(\Br, t) \delta(x^1-r^1) \delta(x^2-r^2) \delta(x^3-r^3) d^3 r \\
&= \psi(\Bx, t)
\end{align*}

In the second last step above we make the ad-hoc identification of that $\sinc$ limit with the dirac delta function, and recover
our original function as desired.

\subsection{ Rotor form of the Fourier transform? }

Although the formulation picked above appears to work, it isn't the only choice to potentially make for the Fourier transform
of multivector.  Would it be more natural to pick an explicit Rotor formulation?  This perhaps makes more sense since it is then
automatically grade preserving.

\begin{align}
\hat{\psi}(\Bk, t) &= \inv{(\sqrt{2\pi})^3} \IIinf \exp\left( \inv{2} i \Bk \cdot \Bx \right) \psi(\Bx, t) \exp\left( - \inv{2} i \Bk \cdot \Bx \right) d^3 x \\
{\psi}(\Bx, t) &= \PV \inv{(\sqrt{2\pi})^3} \IIinf \exp\left( -\inv{2} i \Bk \cdot \Bx \right) \hat{\psi}(\Bk, t) \exp\left( \inv{2} i \Bk \cdot \Bx \right) d^3 k
\end{align}

Later I tried to make an assumption that the grade of a transformed object equals the original grade, and that doesn't work with the
Fourier transform definition that has been picked in equation \ref{eqn:FourierTxDefinition}.  It may be neccessary to revamp the complete treatment, but
for now an observation that the grades of transform pairs do not neccessarily match is required.

\section{ Fourier transforming the spacetime split gradient equation. }

Now, suppose we have a Maxwell like equation of the form

\begin{align}\label{eqn:spacetimeGradientEquation}
\left(\partial_0 + \spacegrad \right) \psi = \rho
\end{align}

Let's take the Fourier transform of this equation.  This gives us

\begin{align*}
\partial_0 \hat{\psi} + \sigma^m \FF(\partial_m \psi) = \hat{\rho}
\end{align*}

Now, we need to look at the middle term in a bit more detail.  For the wave, and heat equations this was evaluated with just an integration
by parts.  Was there any commutation assumption in that previous treatment?  Let's write this out in full to make sure we are cool.

\begin{align*}
\FF(\partial_m \psi) 
&= \inv{(\sqrt{2\pi})^3} \int \left(\partial_m \psi(\Bx, t)\right) \exp\left( -i \Bk \cdot \Bx \right) d^3 x 
\end{align*}

Let's also expand the integral completely, employing a permutation of indexes $\pi(1,2,3) = (m,n,p)$.

\begin{align*}
\FF(\partial_m \psi) 
&= 
\inv{(\sqrt{2\pi})^3} 
\int_{\partial x^p} dx^p
\int_{\partial x^n} dx^n
\int_{\partial x^m} dx^m
\left(\partial_m \psi(\Bx, t)\right) \exp\left( -i \Bk \cdot \Bx \right) \\
\end{align*}

Okay, now we are ready for the integration by parts.  We want a derivative substitution, based on

\begin{align*}
\partial_m &\left( \psi(\Bx, t) \exp\left( -i \Bk \cdot \Bx \right) \right) \\
&= (\partial_m \psi(\Bx, t)) \exp\left( -i \Bk \cdot \Bx \right) + \psi(\Bx, t) \partial_m \exp\left( -i \Bk \cdot \Bx \right) \\
&= (\partial_m \psi(\Bx, t)) \exp\left( -i \Bk \cdot \Bx \right) + \psi(\Bx, t) ( -i k_m ) \exp\left( -i \Bk \cdot \Bx \right) \\
\end{align*}

Observe that we do not wish to assume that the pseudoscalar $i$ commutes with anything except the exponential term, so we have to leave
it sandwiched or on the far right.  We also must take care to not neccessarily commute the exponential itself with $\psi$ or its derivative.
Having noted this we can rearrange as desired for the integration by parts

\begin{align*}
(\partial_m \psi(\Bx, t)) \exp\left( -i \Bk \cdot \Bx \right)
&=
\partial_m \left( \psi(\Bx, t) \exp\left( -i \Bk \cdot \Bx \right) \right) - \psi(\Bx, t) ( -i k_m ) \exp\left( -i \Bk \cdot \Bx \right) \\
\end{align*}

and substitute back into the integral

\begin{align*}
\sigma^m \FF(\partial_m \psi) 
&= 
\inv{(\sqrt{2\pi})^3} 
\int_{\partial x^p} dx^p
\int_{\partial x^n} dx^n
{\left. {\left(\sigma^m \psi(\Bx, t) \exp\left( -i \Bk \cdot \Bx \right) \right)} \right\vert}_{\partial x^m} \\
&- 
\inv{(\sqrt{2\pi})^3} 
\int_{\partial x^p} dx^p
\int_{\partial x^n} dx^n
\int_{\partial x^m} dx^m
\sigma^m \psi(\Bx, t) ( -i k_m )
\exp\left( -i \Bk \cdot \Bx \right) 
\\
\end{align*}

So, we find that the Fourier transform of our spatial gradient is

\begin{align*}
\FF(\grad \psi) = \Bk \hat{\psi} i
\end{align*}

This has the specific ordering of the vector products for our possiblility of non-communative factors.

From this, without making any assumptions about grade, we have the wave number domain equivalent
for the spacetime split of the gradient equation \ref{eqn:spacetimeGradientEquation}

\begin{align}\label{eqn:waveDomainGeneral}
\partial_0 \hat{\psi} + \Bk \hat{\psi} i = \hat{\rho}
\end{align}

\section{ Back to specifics.  Maxwell's equation in wave number domain. }

For Maxwell's equation our field variable $F$ is grade two in the STA basis, and our specific transform pair is:

\begin{align}
\left(\partial_0 + \spacegrad \right) F &= \gamma_0 J/\epsilon_0 c \\
\partial_0 \hat{F} + \Bk \hat{F} i &= \gamma_0 \hat{J}/ \epsilon_0 c
\end{align}

Observe that in the transition from
\ref{eqn:waveDomainGeneral} we do not neccessarily have the freedom to commute $\hat{F}$ with $i = \gamma_0 \gamma_1 \gamma_2 \gamma_3$.
If we knew for sure that $F$ and its transform had the same grade then this commutation would be possible.  Since $F$ is a grade
two multivector, and two sign interchange would be required to commute with each of its vector factors in the STA basis, and one could write $F i = i F$.

We want to find a solution to this second equation.  If the objects in question were all scalars this would be simple enough, and is
a problem of the form

\begin{align*}
f' + a f &= g
\end{align*}

Let's remind ourselves what the form of the solution of this inhomogeneous scalar equation is, using the variation of parameters method.

\bibliographystyle{plainnat}
\bibliography{myrefs}

\end{document}
