%
% Copyright � 2012 Peeter Joot.  All Rights Reserved.
% Licenced as described in the file LICENSE under the root directory of this GIT repository.
%

% 
% 
%\documentclass[]{eliblog}

\usepackage{amsmath}
\usepackage{mathpazo}

%
% shorthand for bold symbols, convenient for vectors and matrices
%
\newcommand{\Ba}[0]{\mathbf{a}}
\newcommand{\Bb}[0]{\mathbf{b}}
\newcommand{\Bc}[0]{\mathbf{c}}
\newcommand{\Bd}[0]{\mathbf{d}}
\newcommand{\Be}[0]{\mathbf{e}}
\newcommand{\Bf}[0]{\mathbf{f}}
\newcommand{\Bg}[0]{\mathbf{g}}
\newcommand{\Bh}[0]{\mathbf{h}}
\newcommand{\Bi}[0]{\mathbf{i}}
\newcommand{\Bj}[0]{\mathbf{j}}
\newcommand{\Bk}[0]{\mathbf{k}}
\newcommand{\Bl}[0]{\mathbf{l}}
\newcommand{\Bm}[0]{\mathbf{m}}
\newcommand{\Bn}[0]{\mathbf{n}}
\newcommand{\Bo}[0]{\mathbf{o}}
\newcommand{\Bp}[0]{\mathbf{p}}
\newcommand{\Bq}[0]{\mathbf{q}}
\newcommand{\Br}[0]{\mathbf{r}}
\newcommand{\Bs}[0]{\mathbf{s}}
\newcommand{\Bt}[0]{\mathbf{t}}
\newcommand{\Bu}[0]{\mathbf{u}}
\newcommand{\Bv}[0]{\mathbf{v}}
\newcommand{\Bw}[0]{\mathbf{w}}
\newcommand{\Bx}[0]{\mathbf{x}}
\newcommand{\By}[0]{\mathbf{y}}
\newcommand{\Bz}[0]{\mathbf{z}}
\newcommand{\BA}[0]{\mathbf{A}}
\newcommand{\BB}[0]{\mathbf{B}}
\newcommand{\BC}[0]{\mathbf{C}}
\newcommand{\BD}[0]{\mathbf{D}}
\newcommand{\BE}[0]{\mathbf{E}}
\newcommand{\BF}[0]{\mathbf{F}}
\newcommand{\BG}[0]{\mathbf{G}}
\newcommand{\BH}[0]{\mathbf{H}}
\newcommand{\BI}[0]{\mathbf{I}}
\newcommand{\BJ}[0]{\mathbf{J}}
\newcommand{\BK}[0]{\mathbf{K}}
\newcommand{\BL}[0]{\mathbf{L}}
\newcommand{\BM}[0]{\mathbf{M}}
\newcommand{\BN}[0]{\mathbf{N}}
\newcommand{\BO}[0]{\mathbf{O}}
\newcommand{\BP}[0]{\mathbf{P}}
\newcommand{\BQ}[0]{\mathbf{Q}}
\newcommand{\BR}[0]{\mathbf{R}}
\newcommand{\BS}[0]{\mathbf{S}}
\newcommand{\BT}[0]{\mathbf{T}}
\newcommand{\BU}[0]{\mathbf{U}}
\newcommand{\BV}[0]{\mathbf{V}}
\newcommand{\BW}[0]{\mathbf{W}}
\newcommand{\BX}[0]{\mathbf{X}}
\newcommand{\BY}[0]{\mathbf{Y}}
\newcommand{\BZ}[0]{\mathbf{Z}}

\newcommand{\Bzero}[0]{\mathbf{0}}
\newcommand{\Btheta}[0]{\boldsymbol{\theta}}
\newcommand{\Btau}[0]{\boldsymbol{\tau}}
\newcommand{\Bomega}[0]{\boldsymbol{\omega}}

%
% shorthand for unit vectors
%
\newcommand{\acap}[0]{\hat{\Ba}}
\newcommand{\bcap}[0]{\hat{\Bb}}
\newcommand{\ccap}[0]{\hat{\Bc}}
\newcommand{\dcap}[0]{\hat{\Bd}}
\newcommand{\ecap}[0]{\hat{\Be}}
\newcommand{\fcap}[0]{\hat{\Bf}}
\newcommand{\gcap}[0]{\hat{\Bg}}
\newcommand{\hcap}[0]{\hat{\Bh}}
\newcommand{\icap}[0]{\hat{\Bi}}
\newcommand{\jcap}[0]{\hat{\Bj}}
\newcommand{\kcap}[0]{\hat{\Bk}}
\newcommand{\lcap}[0]{\hat{\Bl}}
\newcommand{\mcap}[0]{\hat{\Bm}}
\newcommand{\ncap}[0]{\hat{\Bn}}
\newcommand{\ocap}[0]{\hat{\Bo}}
\newcommand{\pcap}[0]{\hat{\Bp}}
\newcommand{\qcap}[0]{\hat{\Bq}}
\newcommand{\rcap}[0]{\hat{\Br}}
\newcommand{\scap}[0]{\hat{\Bs}}
\newcommand{\tcap}[0]{\hat{\Bt}}
\newcommand{\ucap}[0]{\hat{\Bu}}
\newcommand{\vcap}[0]{\hat{\Bv}}
\newcommand{\wcap}[0]{\hat{\Bw}}
\newcommand{\xcap}[0]{\hat{\Bx}}
\newcommand{\ycap}[0]{\hat{\By}}
\newcommand{\zcap}[0]{\hat{\Bz}}
\newcommand{\thetacap}[0]{\hat{\Btheta}}

%
% to write R^n and C^n in a distinguishable fashion.  Perhaps change this
% to the double lined characters upon figuring out how to do so.
%
\newcommand{\C}[1]{$\mathbb{C}^{#1}$}
\newcommand{\R}[1]{$\mathbb{R}^{#1}$}

%
% various generally useful helpers
%

% derivative of #1 wrt. #2:
\newcommand{\D}[2] {\frac {d#2} {d#1}}

\newcommand{\inv}[1]{\frac{1}{#1}}
\newcommand{\cross}[0]{\times}

\newcommand{\abs}[1]{\lvert{#1}\rvert}
\newcommand{\norm}[1]{\lVert{#1}\rVert}
\newcommand{\innerprod}[2]{\langle{#1}, {#2}\rangle}
\newcommand{\dotprod}[2]{{#1} \cdot {#2}}
\newcommand{\bdotprod}[2]{\left({#1} \cdot {#2}\right)}
\newcommand{\crossprod}[2]{{#1} \cross {#2}}
\newcommand{\tripleprod}[3]{\dotprod{\left(\crossprod{#1}{#2}\right)}{#3}}

\DeclareMathOperator{\Proj}{Proj}
\DeclareMathOperator{\Span}{span}
\DeclareMathOperator{\Sgn}{sgn}
\DeclareMathOperator{\Area}{Area}
\DeclareMathOperator{\Volume}{Volume}

%
% A few miscellaneous things specific to this document
%
\newcommand{\crossop}[1]{\crossprod{#1}{}}

% R2 vector.
\newcommand{\VectorTwo}[2]{
\begin{bmatrix}
 {#1} \\
 {#2}
\end{bmatrix}
}

\newcommand{\VectorN}[1]{
\begin{bmatrix}
{#1}_1 \\
{#1}_2 \\
\vdots \\
{#1}_N \\
\end{bmatrix}
}

\newcommand{\DETuvij}[4]{
\begin{vmatrix}
 {#1}_{#3} & {#1}_{#4} \\
 {#2}_{#3} & {#2}_{#4}
\end{vmatrix}
}

\newcommand{\DETuvwijk}[6]{
\begin{vmatrix}
 {#1}_{#4} & {#1}_{#5} & {#1}_{#6} \\
 {#2}_{#4} & {#2}_{#5} & {#2}_{#6} \\
 {#3}_{#4} & {#3}_{#5} & {#3}_{#6}
\end{vmatrix}
}

\newcommand{\DETuvwxijkl}[8]{
\begin{vmatrix}
 {#1}_{#5} & {#1}_{#6} & {#1}_{#7} & {#1}_{#8} \\
 {#2}_{#5} & {#2}_{#6} & {#2}_{#7} & {#2}_{#8} \\
 {#3}_{#5} & {#3}_{#6} & {#3}_{#7} & {#3}_{#8} \\
 {#4}_{#5} & {#4}_{#6} & {#4}_{#7} & {#4}_{#8} \\
\end{vmatrix}
}

%\newcommand{\DETuvwxyijklm}[10]{
%\begin{vmatrix}
% {#1}_{#6} & {#1}_{#7} & {#1}_{#8} & {#1}_{#9} & {#1}_{#10} \\
% {#2}_{#6} & {#2}_{#7} & {#2}_{#8} & {#2}_{#9} & {#2}_{#10} \\
% {#3}_{#6} & {#3}_{#7} & {#3}_{#8} & {#3}_{#9} & {#3}_{#10} \\
% {#4}_{#6} & {#4}_{#7} & {#4}_{#8} & {#4}_{#9} & {#4}_{#10} \\
% {#5}_{#6} & {#5}_{#7} & {#5}_{#8} & {#5}_{#9} & {#5}_{#10}
%\end{vmatrix}
%}

% R3 vector.
\newcommand{\VectorThree}[3]{
\begin{bmatrix}
 {#1} \\
 {#2} \\
 {#3}
\end{bmatrix}
}



\author{Peeter Joot}
\email{peeter.joot@gmail.com}

%\usepackage{txfonts}

\chapter{Stokes theorem applied to vector and bivector fields}
\label{chap:stokesGradeTwo}

%%\date{July 17, 2009}
%%\revisionInfo{$RCSfile: stokesGradeTwo.tex,v $ Last $Revision: 1.5 $ $Date: 2009/08/01 22:12:18 $}

%\blogpage{http://sites.google.com/site/peeterjoot/math2009/stokesGradeTwo.pdf}
%\date{July 17, 2009.  $RCSfile: stokesGradeTwo.tex,v $ Last $Revision: 1.5 $ $Date: 2009/08/01 22:12:18 $}

\beginArtWithToc

\section{Vector Stokes Theorem}

I found my self forgetting stokes theorem once again.  Redo this for the simplest case of a parallelogram area element.

What I recall is that we have on one side the curl dotted into the plane of the surface area element

\begin{align}
\int ( \grad \wedge A ) \cdot d^2 x
\end{align}

and on the other side a loop integral

\begin{align}
\ointctrclockwise A \cdot dx
\end{align}

Comparing the two we should end up with the same form and thus determine the form of the grade two Stokes equation (i.e. for curl of a vector).

\subsection{Bivector product part}

\begin{align*}
( \grad \wedge A ) \cdot d^2 x 
&=
( \grad \wedge A ) \cdot \left(\PD{\alpha}{x} \wedge \PD{\beta}{x}\right) 
d\alpha d\beta \\
&=
\partial_\mu A_\nu \PD{\alpha}{x^\sigma} \PD{\beta}{x^\epsilon} (\gamma^\mu \wedge \gamma^\nu) \cdot (\gamma_\sigma \wedge \gamma_\epsilon) 
d\alpha d\beta \\
&=
\partial_\mu A_\nu \PD{\alpha}{x^\sigma} \PD{\beta}{x^\epsilon} ( {\delta^\mu}_\epsilon {\delta^\nu}_\sigma - {\delta^\mu}_\sigma {\delta^\nu}_\epsilon ) 
d\alpha d\beta \\
&=
\partial_\mu A_\nu \left( \PD{\alpha}{x^\nu} \PD{\beta}{x^\mu} - \PD{\alpha}{x^\mu} \PD{\beta}{x^\nu} \right) 
d\alpha d\beta \\
\end{align*}

So we have
\begin{align}\label{eqn:stokesGradeTwo:withJac}
( \grad \wedge A ) \cdot d^2 x &= -\partial_\mu A_\nu \frac{\partial (x^\mu, x^\nu)}{\partial (\alpha, \beta)} d\alpha d\beta
\end{align}

\subsection{Loop integral part}

Integrating around a parallelogram spacetime area element with sides $d\alpha \partial x/\partial \alpha$ and $d\beta \partial x/\partial \beta$, as
depicted in figure (\ref{fig:surface_area_element}), we have

\imageFigure{figures/surface_area_element}{Surface area element}{fig:surface_area_element}{0.4}

\begin{align*}
\ointctrclockwise 
A \cdot dx
&=
\int
{\left. A \right\vert}_{\beta=\beta_0} \cdot \PD{\alpha}{x} d\alpha
+ {\left. A \right\vert}_{\alpha=\alpha_1} \cdot \PD{\beta}{x} d\beta
+ {\left. A \right\vert}_{\beta=\beta_1} \cdot \left( -\PD{\alpha}{x} d\alpha \right)
+ {\left. A \right\vert}_{\alpha=\alpha_0} \cdot \left( -\PD{\beta}{x} d\beta \right) 
\\
&=
\int
\left( {\left. A \right\vert}_{\alpha=\alpha_1} - {\left. A \right\vert}_{\alpha=\alpha_0} \right) \cdot \PD{\beta}{x} d\beta
-\left( {\left. A \right\vert}_{\beta=\beta_1} - {\left. A \right\vert}_{\beta=\beta_0} \right) \cdot \PD{\alpha}{x} d\alpha 
\\
&=
\int
\PD{\alpha}{A} \cdot \PD{\beta}{x} d\alpha d\beta
-\PD{\beta}{A} \cdot \PD{\alpha}{x} d\beta d\alpha
\end{align*}

Expanding the derivatives in terms of coordinates we have

\begin{align*}
\PD{\sigma}{A} 
&=
\PD{\sigma}{A_\mu} \gamma^\mu \\
&= 
\PD{x^\nu}{A_\mu}\PD{\sigma}{x^\nu} \gamma^\mu \\
&= 
\partial_\nu A_\mu \PD{\sigma}{x^\nu} \gamma^\mu \\
\end{align*}

and
\begin{align*}
\PD{\sigma}{x} &= \PD{\sigma}{x^\nu} \gamma_\nu
\end{align*}

Assembling we have
\begin{align*}
\ointctrclockwise 
A \cdot dx
&=
\int
\partial_\nu A_\mu \left( \PD{\alpha}{x^\nu} \PD{\beta}{x^\mu} - \PD{\beta}{x^\nu} \PD{\alpha}{x^\mu} \right) d\alpha d\beta
\end{align*}

In terms of the Jacobian used in (\ref{eqn:stokesGradeTwo:withJac}) we have

\begin{align*}
\ointctrclockwise 
A \cdot dx &= \int \partial_\mu A_\nu \frac{\partial (x^\mu, x^\nu)}{\partial (\alpha, \beta)} d\alpha d\beta
\end{align*}

Comparing the two we have only a sign difference so the conclusion is that Stokes for a vector field (considering only a flat parallelogram area element) is

\begin{align}
\int ( \grad \wedge A ) \cdot d^2 x &= \ointclockwise A \cdot dx
\end{align}

Observe that there's an implied orientation of the area element on the LHS, required to match up with the orientation of the RHS integral.

\section{Bivector Stokes Theorem}

A parallelepiped volume element is depicted in figure (\ref{fig:volume_element}).  Three parameters $\alpha$, $\beta$, $\sigma$ generate a set of differential vector displacements spanning the three dimensional subspace

\imageFigure{figures/volume_element}{Differential volume element}{fig:volume_element}{0.4}

Writing the displacements 

\begin{align*}
dx_\alpha &= \PD{\alpha}{x} d\alpha \\
dx_\beta &= \PD{\beta}{x} d\beta \\
dx_\sigma &= \PD{\sigma}{x} d\sigma
\end{align*}

We have for the front, right and top face area elements 

\begin{align*}
dA_F &= dx_\alpha \wedge dx_\beta \\
dA_R &= dx_\beta \wedge dx_\sigma \\
dA_T &= dx_\sigma \wedge dx_\alpha \\
\end{align*}

These are the surfaces of constant parametrization, respectively, $\sigma = \sigma_1$, $\alpha = \alpha_1$, and $\beta = \beta_1$.  For a bivector, the flux through the surface is therefore

\begin{align*}
\int B \cdot dA 
&=
(B_{\sigma_1} \cdot dA_F - B_{\sigma_0} \cdot dA_P )
+ (B_{\alpha_1} \cdot dA_R - B_{\alpha_0} \cdot dA_L)
+ (B_{\beta_1} \cdot dA_T - B_{\beta_0} \cdot dA_B) \\
&=
d \sigma \PD{\sigma}{B} \cdot (dx_\alpha \wedge dx_\beta )
+ d \alpha \PD{\alpha}{B} \cdot (dx_\beta \wedge dx_\sigma) 
+ d \beta \PD{\beta}{B} \cdot (dx_\sigma \wedge dx_\alpha ) \\
\end{align*}

Written out in full this is a bit of a mess
\begin{align}\label{eqn:stokesGradeTwo:mess}
\int B \cdot dA 
&=
d \alpha d\beta d\sigma 
\partial_\mu B \cdot
\left(
\left(
- \PD{\sigma}{x^\mu} \PD{\beta}{x^\nu} \PD{\alpha}{x^\epsilon} 
+ \PD{\alpha}{x^\mu} \PD{\beta}{x^\nu} \PD{\sigma}{x^\epsilon} 
+ \PD{\beta}{x^\mu} \PD{\sigma}{x^\nu} \PD{\alpha}{x^\epsilon} 
\right) 
(\gamma_\nu \wedge \gamma_\epsilon )
\right) 
\end{align}

It should equal, at least up to a sign, $\int (\grad \wedge B) \cdot d^3 x$.  Expanding the latter is probably easier than regrouping the mess, and doing so we have

\begin{align*}
(\grad \wedge B) \cdot d^3 x
&=
d\alpha d\beta d\sigma ( \gamma^\mu \wedge \partial_\mu B)  \cdot \left( \PD{\alpha}{x} \wedge \PD{\beta}{x} \wedge \PD{\sigma}{x} \right) \\
&= 
d\alpha d\beta d\sigma \inv{2} ( \gamma^\mu \partial_\mu B + \partial_\mu B \gamma^\mu )  \cdot \left( \PD{\alpha}{x} \wedge \PD{\beta}{x} \wedge \PD{\sigma}{x} \right) \\
&=
d\alpha d\beta d\sigma \inv{2} \gpgradezero{
( \gamma^\mu \partial_\mu B + \partial_\mu B \gamma^\mu )  \left( \PD{\alpha}{x} \wedge \PD{\beta}{x} \wedge \PD{\sigma}{x} \right) }
\\
&=
d\alpha d\beta d\sigma \inv{2} 
\partial_\mu B \cdot
\gpgradetwo{
\left( \PD{\alpha}{x} \wedge \PD{\beta}{x} \wedge \PD{\sigma}{x} \right) \gamma^\mu 
+ \gamma^\mu \left( \PD{\alpha}{x} \wedge \PD{\beta}{x} \wedge \PD{\sigma}{x} \right) }
\\
&=
d\alpha d\beta d\sigma 
\partial_\mu B \cdot
\left( \left( \PD{\alpha}{x} \wedge \PD{\beta}{x} \wedge \PD{\sigma}{x} \right) \cdot \gamma^\mu \right)
\\
\end{align*}

Expanding just that trivector-vector dot product

\begin{align*}
\left( \PD{\alpha}{x} \wedge \PD{\beta}{x} \wedge \PD{\sigma}{x} \right) \cdot \gamma^\mu 
&=
\PD{\alpha}{x^\lambda} \PD{\beta}{x^\nu} \PD{\sigma}{x^\epsilon} \left( \gamma_\lambda \wedge \gamma_\nu \wedge \gamma_\epsilon \right) \cdot \gamma^\mu  \\
&=
\PD{\alpha}{x^\lambda} \PD{\beta}{x^\nu} \PD{\sigma}{x^\epsilon} \left( 
\gamma_\lambda \wedge \gamma_\nu {\delta_\epsilon}^\mu
-\gamma_\lambda \wedge \gamma_\epsilon {\delta_\nu}^\mu
+\gamma_\nu \wedge \gamma_\epsilon {\delta_\lambda}^\mu
\right) 
\end{align*}

So we have
\begin{align*}
(\grad \wedge B) \cdot d^3 x
&=
d\alpha d\beta d\sigma \PD{\alpha}{x^\lambda} \PD{\beta}{x^\nu} \PD{\sigma}{x^\epsilon} \partial_\mu B \cdot \left( 
\gamma_\lambda \wedge \gamma_\nu {\delta_\epsilon}^\mu
-\gamma_\lambda \wedge \gamma_\epsilon {\delta_\nu}^\mu
+\gamma_\nu \wedge \gamma_\epsilon {\delta_\lambda}^\mu
\right) 
\\
&=
d\alpha d\beta d\sigma 
\partial_\mu B \cdot \left( 
  \PD{\alpha}{x^\lambda} \PD{\beta}{x^\nu} \PD{\sigma}{x^\mu} \gamma_\lambda \wedge \gamma_\nu 
%- \PD{\alpha}{x^\lambda} \PD{\beta}{x^\mu} \PD{\sigma}{x^\epsilon} \gamma_\lambda \wedge \gamma_\epsilon 
+ \PD{\alpha}{x^\lambda} \PD{\beta}{x^\mu} \PD{\sigma}{x^\epsilon} \gamma_\epsilon \wedge \gamma_\lambda 
+ \PD{\alpha}{x^\mu} \PD{\beta}{x^\nu} \PD{\sigma}{x^\epsilon} \gamma_\nu \wedge \gamma_\epsilon 
\right) 
\\
&=
d\alpha d\beta d\sigma 
\partial_\mu B \cdot 
\left( 
\left( 
  \PD{\alpha}{x^\nu} \PD{\beta}{x^\epsilon} \PD{\sigma}{x^\mu} 
+ \PD{\alpha}{x^\epsilon} \PD{\beta}{x^\mu} \PD{\sigma}{x^\nu} 
+ \PD{\alpha}{x^\mu} \PD{\beta}{x^\nu} \PD{\sigma}{x^\epsilon} 
\right) 
\gamma_\nu \wedge \gamma_\epsilon 
\right) 
\\
\end{align*}

Noting that an $\epsilon$, $\nu$ interchange in the first term inverts the sign, we have an exact match with (\ref{eqn:stokesGradeTwo:mess}), thus fixing the sign for the
bivector form of Stokes theorem for the orientation picked in this diagram

\begin{align}
\int (\grad \wedge B) \cdot d^3 x &= \int B \cdot d^2 x
\end{align}

Like the vector case, there is a requirement to be very specific about the meaning given to the oriented surfaces, and the corresponding oriented volume element (which could be a volume subspace of a greater than three dimensional space).

%\EndNoBibArticle
