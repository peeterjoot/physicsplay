\documentclass{article}      

\usepackage{amsmath}
\usepackage{mathpazo}

%
% shorthand for bold symbols, convenient for vectors and matrices
%
\newcommand{\Ba}[0]{\mathbf{a}}
\newcommand{\Bb}[0]{\mathbf{b}}
\newcommand{\Bc}[0]{\mathbf{c}}
\newcommand{\Bd}[0]{\mathbf{d}}
\newcommand{\Be}[0]{\mathbf{e}}
\newcommand{\Bf}[0]{\mathbf{f}}
\newcommand{\Bg}[0]{\mathbf{g}}
\newcommand{\Bh}[0]{\mathbf{h}}
\newcommand{\Bi}[0]{\mathbf{i}}
\newcommand{\Bj}[0]{\mathbf{j}}
\newcommand{\Bk}[0]{\mathbf{k}}
\newcommand{\Bl}[0]{\mathbf{l}}
\newcommand{\Bm}[0]{\mathbf{m}}
\newcommand{\Bn}[0]{\mathbf{n}}
\newcommand{\Bo}[0]{\mathbf{o}}
\newcommand{\Bp}[0]{\mathbf{p}}
\newcommand{\Bq}[0]{\mathbf{q}}
\newcommand{\Br}[0]{\mathbf{r}}
\newcommand{\Bs}[0]{\mathbf{s}}
\newcommand{\Bt}[0]{\mathbf{t}}
\newcommand{\Bu}[0]{\mathbf{u}}
\newcommand{\Bv}[0]{\mathbf{v}}
\newcommand{\Bw}[0]{\mathbf{w}}
\newcommand{\Bx}[0]{\mathbf{x}}
\newcommand{\By}[0]{\mathbf{y}}
\newcommand{\Bz}[0]{\mathbf{z}}
\newcommand{\BA}[0]{\mathbf{A}}
\newcommand{\BB}[0]{\mathbf{B}}
\newcommand{\BC}[0]{\mathbf{C}}
\newcommand{\BD}[0]{\mathbf{D}}
\newcommand{\BE}[0]{\mathbf{E}}
\newcommand{\BF}[0]{\mathbf{F}}
\newcommand{\BG}[0]{\mathbf{G}}
\newcommand{\BH}[0]{\mathbf{H}}
\newcommand{\BI}[0]{\mathbf{I}}
\newcommand{\BJ}[0]{\mathbf{J}}
\newcommand{\BK}[0]{\mathbf{K}}
\newcommand{\BL}[0]{\mathbf{L}}
\newcommand{\BM}[0]{\mathbf{M}}
\newcommand{\BN}[0]{\mathbf{N}}
\newcommand{\BO}[0]{\mathbf{O}}
\newcommand{\BP}[0]{\mathbf{P}}
\newcommand{\BQ}[0]{\mathbf{Q}}
\newcommand{\BR}[0]{\mathbf{R}}
\newcommand{\BS}[0]{\mathbf{S}}
\newcommand{\BT}[0]{\mathbf{T}}
\newcommand{\BU}[0]{\mathbf{U}}
\newcommand{\BV}[0]{\mathbf{V}}
\newcommand{\BW}[0]{\mathbf{W}}
\newcommand{\BX}[0]{\mathbf{X}}
\newcommand{\BY}[0]{\mathbf{Y}}
\newcommand{\BZ}[0]{\mathbf{Z}}

\newcommand{\Bzero}[0]{\mathbf{0}}
\newcommand{\Btheta}[0]{\boldsymbol{\theta}}
\newcommand{\Btau}[0]{\boldsymbol{\tau}}
\newcommand{\Bomega}[0]{\boldsymbol{\omega}}

%
% shorthand for unit vectors
%
\newcommand{\acap}[0]{\hat{\Ba}}
\newcommand{\bcap}[0]{\hat{\Bb}}
\newcommand{\ccap}[0]{\hat{\Bc}}
\newcommand{\dcap}[0]{\hat{\Bd}}
\newcommand{\ecap}[0]{\hat{\Be}}
\newcommand{\fcap}[0]{\hat{\Bf}}
\newcommand{\gcap}[0]{\hat{\Bg}}
\newcommand{\hcap}[0]{\hat{\Bh}}
\newcommand{\icap}[0]{\hat{\Bi}}
\newcommand{\jcap}[0]{\hat{\Bj}}
\newcommand{\kcap}[0]{\hat{\Bk}}
\newcommand{\lcap}[0]{\hat{\Bl}}
\newcommand{\mcap}[0]{\hat{\Bm}}
\newcommand{\ncap}[0]{\hat{\Bn}}
\newcommand{\ocap}[0]{\hat{\Bo}}
\newcommand{\pcap}[0]{\hat{\Bp}}
\newcommand{\qcap}[0]{\hat{\Bq}}
\newcommand{\rcap}[0]{\hat{\Br}}
\newcommand{\scap}[0]{\hat{\Bs}}
\newcommand{\tcap}[0]{\hat{\Bt}}
\newcommand{\ucap}[0]{\hat{\Bu}}
\newcommand{\vcap}[0]{\hat{\Bv}}
\newcommand{\wcap}[0]{\hat{\Bw}}
\newcommand{\xcap}[0]{\hat{\Bx}}
\newcommand{\ycap}[0]{\hat{\By}}
\newcommand{\zcap}[0]{\hat{\Bz}}
\newcommand{\thetacap}[0]{\hat{\Btheta}}

%
% to write R^n and C^n in a distinguishable fashion.  Perhaps change this
% to the double lined characters upon figuring out how to do so.
%
\newcommand{\C}[1]{$\mathbb{C}^{#1}$}
\newcommand{\R}[1]{$\mathbb{R}^{#1}$}

%
% various generally useful helpers
%

% derivative of #1 wrt. #2:
\newcommand{\D}[2] {\frac {d#2} {d#1}}

\newcommand{\inv}[1]{\frac{1}{#1}}
\newcommand{\cross}[0]{\times}

\newcommand{\abs}[1]{\lvert{#1}\rvert}
\newcommand{\norm}[1]{\lVert{#1}\rVert}
\newcommand{\innerprod}[2]{\langle{#1}, {#2}\rangle}
\newcommand{\dotprod}[2]{{#1} \cdot {#2}}
\newcommand{\bdotprod}[2]{\left({#1} \cdot {#2}\right)}
\newcommand{\crossprod}[2]{{#1} \cross {#2}}
\newcommand{\tripleprod}[3]{\dotprod{\left(\crossprod{#1}{#2}\right)}{#3}}

\DeclareMathOperator{\Proj}{Proj}
\DeclareMathOperator{\Span}{span}
\DeclareMathOperator{\Sgn}{sgn}
\DeclareMathOperator{\Area}{Area}
\DeclareMathOperator{\Volume}{Volume}

%
% A few miscellaneous things specific to this document
%
\newcommand{\crossop}[1]{\crossprod{#1}{}}

% R2 vector.
\newcommand{\VectorTwo}[2]{
\begin{bmatrix}
 {#1} \\
 {#2}
\end{bmatrix}
}

\newcommand{\VectorN}[1]{
\begin{bmatrix}
{#1}_1 \\
{#1}_2 \\
\vdots \\
{#1}_N \\
\end{bmatrix}
}

\newcommand{\DETuvij}[4]{
\begin{vmatrix}
 {#1}_{#3} & {#1}_{#4} \\
 {#2}_{#3} & {#2}_{#4}
\end{vmatrix}
}

\newcommand{\DETuvwijk}[6]{
\begin{vmatrix}
 {#1}_{#4} & {#1}_{#5} & {#1}_{#6} \\
 {#2}_{#4} & {#2}_{#5} & {#2}_{#6} \\
 {#3}_{#4} & {#3}_{#5} & {#3}_{#6}
\end{vmatrix}
}

\newcommand{\DETuvwxijkl}[8]{
\begin{vmatrix}
 {#1}_{#5} & {#1}_{#6} & {#1}_{#7} & {#1}_{#8} \\
 {#2}_{#5} & {#2}_{#6} & {#2}_{#7} & {#2}_{#8} \\
 {#3}_{#5} & {#3}_{#6} & {#3}_{#7} & {#3}_{#8} \\
 {#4}_{#5} & {#4}_{#6} & {#4}_{#7} & {#4}_{#8} \\
\end{vmatrix}
}

%\newcommand{\DETuvwxyijklm}[10]{
%\begin{vmatrix}
% {#1}_{#6} & {#1}_{#7} & {#1}_{#8} & {#1}_{#9} & {#1}_{#10} \\
% {#2}_{#6} & {#2}_{#7} & {#2}_{#8} & {#2}_{#9} & {#2}_{#10} \\
% {#3}_{#6} & {#3}_{#7} & {#3}_{#8} & {#3}_{#9} & {#3}_{#10} \\
% {#4}_{#6} & {#4}_{#7} & {#4}_{#8} & {#4}_{#9} & {#4}_{#10} \\
% {#5}_{#6} & {#5}_{#7} & {#5}_{#8} & {#5}_{#9} & {#5}_{#10}
%\end{vmatrix}
%}

% R3 vector.
\newcommand{\VectorThree}[3]{
\begin{bmatrix}
 {#1} \\
 {#2} \\
 {#3}
\end{bmatrix}
}


\newcommand{\grad}[0]{\nabla}
\newcommand{\PD}[2]{ \frac{\partial{#1}}{\partial {#2}} }

\title{ Derivation of Newton's Law from Lagrangian and general gradient.} 
\author{Peeter Joot}         
%\date{}        % Deleting this command produces today's date.

\begin{document}             

\maketitle{}

\section{}

In the classical limit the Lagrangian action for a point particle in a general 
position dependent field is:

\begin{equation}
S = \inv{2} m\Bv^2 - \varphi
\end{equation}

Given the Lagrange equations that minimize the action, it is fairly simple
to derive the Newtonian force law.

\begin{align*}
0
&= \PD{S}{x^i} - \frac{d}{dt}\PD{S}{\dot{x}^i} \\
&= -\PD{\varphi}{x^i} - \frac{d}{dt}\left(m \dot{x}^i\right)
\end{align*}

Multiplication of this result with the unit vector $\Be_i$, and summing over 
all unit vectors we have:

\begin{equation*}
\sum \Be_i \frac{d}{dt}\left(m \dot{x}^i\right) = - \sum \Be_i \PD{\varphi}{x^i} 
\end{equation*}

Or, using the gradient operator, and writing $\Bv = \sum \Be_i \dot{x}^i$, we have:

\begin{equation}
\frac{d m \Bv}{dt} = - \grad \varphi
\end{equation}

\subsection{ The mistake hiding above. }

Now, despite the use of upper and lower pairs of indexes for the basis vectors and coordinates, this 
result is not valid for a general set of basis vectors.  This initially confused the author, since the RHS
sum $\Bv = \sum \Be_i v^i$ is valid for any set of basis vectors independent of the orthonormality of that 
set of basis vectors.  This is assuming that these coordinate pairs follow the usual reciprocal relationships:

\begin{equation*}
\Bx = \sum \Be_i x^i
\end{equation*}
\begin{equation*}
x^i = \Bx \cdot \Be^i
\end{equation*}
\begin{equation*}
\Be^i \cdot \Be_j = {\delta^i}_j
\end{equation*}

However, the LHS that implicitly defines the gradient as:
\begin{equation*}
\grad = \sum \Be_i \PD{}{x^i} 
\end{equation*}

is a result that is only valid when the set of basis vectors $\Be_i$ is orthonormal.  The general result is
expected instead to be:

\begin{equation*}
\grad = \sum \Be^i \PD{}{x^i} 
\end{equation*}

This is how the gradient is defined (without motivation) in Doran/Lasenby.  One can however demonstrate that this definition, and not $\grad = \sum \Be_i \PD{}{x^i}$ is required, by doing a computation of something like $\grad \norm{\Bx}^\alpha$ with $\Bx = \sum x^i \Be_i$ for a general basis $\Be_i$ to demonstrate this.

So where did things go wrong?  It was in one of the ``obvious'' skipped steps: $\Bv = \sum {\dot{x^i}}^2$.  It is in that
spot where there is a hidden orthonormal frame vector requirement since a general basis will have mixed product terms too
(ie: non-diagonal metric tensor).

Expressed in full for general frame vectors the action to minimize is the following:

\begin{equation}
S = \inv{2} m \sum \dot{x}^i \dot{x}^j \Be_i \cdot \Be_j -\varphi
\end{equation}

Or, expressed using a metric tensor $g_{ij} = \Be_i \cdot \Be_j$, this is:

\begin{equation}
S = \inv{2} m \sum \dot{x}^i \dot{x}^j g_{ij} -\varphi
\end{equation}

Now we are in shape to calculate the equations of motion from the Lagrangian action minimization equations.

\subsection{ Equations of motion for vectors in a general frame. }

However, this 
This is out familiar Newtonian force law for a point particle in a conservative potential field:

\begin{equation}\label{eqn:newtonionmotion}
F = \frac{d (m \Bv)}{dt} = -\grad \varphi.
\end{equation}

Observe that the gradient and velocity vectors as expressed here have no requirement that the basis vectors to be an
othonormal set.  The only requirement is that the coordinates $x^i$ and the basis vectors $\Be_i$ be reciprocal pairs, defined by the following relationships:


The requirement for reciprocal pairs of coordinates and basis frame vectors is due to the 
summation $\Bv = \sum \Be_i \dot{x}^i$.  The Lagrange equations that minimize the action still generate equations of
motion that hold when the coordinate and basis vectors cannot be summed in this fashion.
In such a case, however, the ability to merge the generalized coordinate equations of motion into a single
vector relationship will not be possible.

Having expressed the velocity vector $\Bv$ in terms of a
reciprocal frame on the LHS of equation \ref{eqn:newtonionmotion}, a 
desire to express the LHS:

\begin{equation*}
-\sum \Be_i \PD{\varphi}{x^i}
\end{equation*}

as a gradient implies a requirement to define:

\begin{equation}
\grad = \sum \Be_i \PD{}{x^i}
\end{equation}

provided we want the LHS to be called a gradient when an orthonormal frame
is used.

FIXME: reorganize this text.  Flow doesn't make sense.

\end{document}               
