\documentclass{article}      % Specifies the document class

\usepackage{amsmath}
\usepackage{mathpazo}

%
% shorthand for bold symbols, convenient for vectors and matrices
%
\newcommand{\Ba}[0]{\mathbf{a}}
\newcommand{\Bb}[0]{\mathbf{b}}
\newcommand{\Bc}[0]{\mathbf{c}}
\newcommand{\Bd}[0]{\mathbf{d}}
\newcommand{\Be}[0]{\mathbf{e}}
\newcommand{\Bf}[0]{\mathbf{f}}
\newcommand{\Bg}[0]{\mathbf{g}}
\newcommand{\Bh}[0]{\mathbf{h}}
\newcommand{\Bi}[0]{\mathbf{i}}
\newcommand{\Bj}[0]{\mathbf{j}}
\newcommand{\Bk}[0]{\mathbf{k}}
\newcommand{\Bl}[0]{\mathbf{l}}
\newcommand{\Bm}[0]{\mathbf{m}}
\newcommand{\Bn}[0]{\mathbf{n}}
\newcommand{\Bo}[0]{\mathbf{o}}
\newcommand{\Bp}[0]{\mathbf{p}}
\newcommand{\Bq}[0]{\mathbf{q}}
\newcommand{\Br}[0]{\mathbf{r}}
\newcommand{\Bs}[0]{\mathbf{s}}
\newcommand{\Bt}[0]{\mathbf{t}}
\newcommand{\Bu}[0]{\mathbf{u}}
\newcommand{\Bv}[0]{\mathbf{v}}
\newcommand{\Bw}[0]{\mathbf{w}}
\newcommand{\Bx}[0]{\mathbf{x}}
\newcommand{\By}[0]{\mathbf{y}}
\newcommand{\Bz}[0]{\mathbf{z}}
\newcommand{\BA}[0]{\mathbf{A}}
\newcommand{\BB}[0]{\mathbf{B}}
\newcommand{\BC}[0]{\mathbf{C}}
\newcommand{\BD}[0]{\mathbf{D}}
\newcommand{\BE}[0]{\mathbf{E}}
\newcommand{\BF}[0]{\mathbf{F}}
\newcommand{\BG}[0]{\mathbf{G}}
\newcommand{\BH}[0]{\mathbf{H}}
\newcommand{\BI}[0]{\mathbf{I}}
\newcommand{\BJ}[0]{\mathbf{J}}
\newcommand{\BK}[0]{\mathbf{K}}
\newcommand{\BL}[0]{\mathbf{L}}
\newcommand{\BM}[0]{\mathbf{M}}
\newcommand{\BN}[0]{\mathbf{N}}
\newcommand{\BO}[0]{\mathbf{O}}
\newcommand{\BP}[0]{\mathbf{P}}
\newcommand{\BQ}[0]{\mathbf{Q}}
\newcommand{\BR}[0]{\mathbf{R}}
\newcommand{\BS}[0]{\mathbf{S}}
\newcommand{\BT}[0]{\mathbf{T}}
\newcommand{\BU}[0]{\mathbf{U}}
\newcommand{\BV}[0]{\mathbf{V}}
\newcommand{\BW}[0]{\mathbf{W}}
\newcommand{\BX}[0]{\mathbf{X}}
\newcommand{\BY}[0]{\mathbf{Y}}
\newcommand{\BZ}[0]{\mathbf{Z}}

\newcommand{\Bzero}[0]{\mathbf{0}}
\newcommand{\Btheta}[0]{\boldsymbol{\theta}}
\newcommand{\Btau}[0]{\boldsymbol{\tau}}
\newcommand{\Bomega}[0]{\boldsymbol{\omega}}

%
% shorthand for unit vectors
%
\newcommand{\acap}[0]{\hat{\Ba}}
\newcommand{\bcap}[0]{\hat{\Bb}}
\newcommand{\ccap}[0]{\hat{\Bc}}
\newcommand{\dcap}[0]{\hat{\Bd}}
\newcommand{\ecap}[0]{\hat{\Be}}
\newcommand{\fcap}[0]{\hat{\Bf}}
\newcommand{\gcap}[0]{\hat{\Bg}}
\newcommand{\hcap}[0]{\hat{\Bh}}
\newcommand{\icap}[0]{\hat{\Bi}}
\newcommand{\jcap}[0]{\hat{\Bj}}
\newcommand{\kcap}[0]{\hat{\Bk}}
\newcommand{\lcap}[0]{\hat{\Bl}}
\newcommand{\mcap}[0]{\hat{\Bm}}
\newcommand{\ncap}[0]{\hat{\Bn}}
\newcommand{\ocap}[0]{\hat{\Bo}}
\newcommand{\pcap}[0]{\hat{\Bp}}
\newcommand{\qcap}[0]{\hat{\Bq}}
\newcommand{\rcap}[0]{\hat{\Br}}
\newcommand{\scap}[0]{\hat{\Bs}}
\newcommand{\tcap}[0]{\hat{\Bt}}
\newcommand{\ucap}[0]{\hat{\Bu}}
\newcommand{\vcap}[0]{\hat{\Bv}}
\newcommand{\wcap}[0]{\hat{\Bw}}
\newcommand{\xcap}[0]{\hat{\Bx}}
\newcommand{\ycap}[0]{\hat{\By}}
\newcommand{\zcap}[0]{\hat{\Bz}}
\newcommand{\thetacap}[0]{\hat{\Btheta}}

%
% to write R^n and C^n in a distinguishable fashion.  Perhaps change this
% to the double lined characters upon figuring out how to do so.
%
\newcommand{\C}[1]{$\mathbb{C}^{#1}$}
\newcommand{\R}[1]{$\mathbb{R}^{#1}$}

%
% various generally useful helpers
%

% derivative of #1 wrt. #2:
\newcommand{\D}[2] {\frac {d#2} {d#1}}

\newcommand{\inv}[1]{\frac{1}{#1}}
\newcommand{\cross}[0]{\times}

\newcommand{\abs}[1]{\lvert{#1}\rvert}
\newcommand{\norm}[1]{\lVert{#1}\rVert}
\newcommand{\innerprod}[2]{\langle{#1}, {#2}\rangle}
\newcommand{\dotprod}[2]{{#1} \cdot {#2}}
\newcommand{\bdotprod}[2]{\left({#1} \cdot {#2}\right)}
\newcommand{\crossprod}[2]{{#1} \cross {#2}}
\newcommand{\tripleprod}[3]{\dotprod{\left(\crossprod{#1}{#2}\right)}{#3}}

\DeclareMathOperator{\Proj}{Proj}
\DeclareMathOperator{\Span}{span}
\DeclareMathOperator{\Sgn}{sgn}
\DeclareMathOperator{\Area}{Area}
\DeclareMathOperator{\Volume}{Volume}

%
% A few miscellaneous things specific to this document
%
\newcommand{\crossop}[1]{\crossprod{#1}{}}

% R2 vector.
\newcommand{\VectorTwo}[2]{
\begin{bmatrix}
 {#1} \\
 {#2}
\end{bmatrix}
}

\newcommand{\VectorN}[1]{
\begin{bmatrix}
{#1}_1 \\
{#1}_2 \\
\vdots \\
{#1}_N \\
\end{bmatrix}
}

\newcommand{\DETuvij}[4]{
\begin{vmatrix}
 {#1}_{#3} & {#1}_{#4} \\
 {#2}_{#3} & {#2}_{#4}
\end{vmatrix}
}

\newcommand{\DETuvwijk}[6]{
\begin{vmatrix}
 {#1}_{#4} & {#1}_{#5} & {#1}_{#6} \\
 {#2}_{#4} & {#2}_{#5} & {#2}_{#6} \\
 {#3}_{#4} & {#3}_{#5} & {#3}_{#6}
\end{vmatrix}
}

\newcommand{\DETuvwxijkl}[8]{
\begin{vmatrix}
 {#1}_{#5} & {#1}_{#6} & {#1}_{#7} & {#1}_{#8} \\
 {#2}_{#5} & {#2}_{#6} & {#2}_{#7} & {#2}_{#8} \\
 {#3}_{#5} & {#3}_{#6} & {#3}_{#7} & {#3}_{#8} \\
 {#4}_{#5} & {#4}_{#6} & {#4}_{#7} & {#4}_{#8} \\
\end{vmatrix}
}

%\newcommand{\DETuvwxyijklm}[10]{
%\begin{vmatrix}
% {#1}_{#6} & {#1}_{#7} & {#1}_{#8} & {#1}_{#9} & {#1}_{#10} \\
% {#2}_{#6} & {#2}_{#7} & {#2}_{#8} & {#2}_{#9} & {#2}_{#10} \\
% {#3}_{#6} & {#3}_{#7} & {#3}_{#8} & {#3}_{#9} & {#3}_{#10} \\
% {#4}_{#6} & {#4}_{#7} & {#4}_{#8} & {#4}_{#9} & {#4}_{#10} \\
% {#5}_{#6} & {#5}_{#7} & {#5}_{#8} & {#5}_{#9} & {#5}_{#10}
%\end{vmatrix}
%}

% R3 vector.
\newcommand{\VectorThree}[3]{
\begin{bmatrix}
 {#1} \\
 {#2} \\
 {#3}
\end{bmatrix}
}



%
% The real thing:
%

                             % The preamble begins here.
\title{Maxwell's equations expressed with Geometric Algebra.} % Declares the document's title.
\author{Peeter Joot}         % Declares the author's name.
%\date{}        % Deleting this command produces today's date.

\begin{document}             % End of preamble and beginning of text.

\maketitle{}

\section{On different ways of expressing Maxwell's equations.}

One of the most striking applications of the geometric product is the ability to formulate the eight Maxwell's equations in a coherent fashion as a single equation.

This isn't a new idea, and this has been done historically using formulations based on quaterions (~1910.  dig up citation).  A formulation in terms of antisymetric second rank tensors $F_{\mu \nu}$ and $G_{\mu \nu}$ (See: wiki:Formulation of Maxwell's equations in special relativity) reduces the eight equations to two, but also introduces complexity and obfuscates the connection to the physically measurable quantities.

A formulation in terms of differential forms (See: wiki:Maxwell's equations) is also possible.  This doesn't have the complexity of the tensor formulation, but requires the electromagnetic field to be expressed as a differential form.  This is arguably strange given a traditional vector calculus education.  One also doesn't have to integrate a field in any fashion, so what meaning should be given to a electrodynamic field as a differential form?

\subsection{Introduction of complex vector electromagnetic field }

To explore the ideas, the starting point is the traditional set of Maxwell's equations

\[
\nabla \cdot \BE  = \frac {\rho} {\epsilon_0}
\]
\[
\nabla \cdot \BB  = 0
\]
\[
\nabla \times \BE  +\frac{\partial \BB } {\partial t} = 0
\]
\[
c^2 \nabla \times \BB  - \frac{\partial \BE } {\partial t}
= \frac{\BJ}{\epsilon_0}
\]

It is customary in relativistic treatments of electrodynamics to introduce a four vector $(x, y, z, ict)$.  Using this as a hint, one can write the time partials in terms of $ict$ and regrouping slightly

\[
\nabla \cdot \BE  = \frac {\rho} {\epsilon_0}
\]
\[
\nabla \cdot (ic\BB ) = 0
\]
\[
\nabla \times \BE  +\frac{\partial (ic\BB )} {\partial (ict)} = 0
\]
\[
\nabla \times (ic\BB ) + \frac{\partial \BE } {\partial (ict)}
= i\frac{\BJ}{\epsilon_0 c}
\]

There is no use of geometric or wedge products here, but the opposing signs in the two sets of curl and time partial equations is removed.  The pairs of equations can be added together without loss of information since the original equations can be recovered by taking real and imaginary parts.

\[
\nabla \cdot (\BE + ic \BB) = \frac {\rho} {\epsilon_0}
\]
\[
\nabla \times (\BE + ic \BB) + \frac{\partial (\BE + ic \BB)} {\partial (ict)}
= i\frac{\BJ}{\epsilon_0 c}
\]

It is thus natural to define a combined electrodynamic field as a complex vector, expressing the natural orthogality of the electric and magnetic fields

\[
\BF = \BE + ic \BB
\]

The electric and magnetic fields can be recovered from this composite field by taking real and imaginary parts respectively, and we can now write write Maxwell's equations in terms of this single electrodyamic field

\[
\nabla \cdot \BF = \frac {\rho} {\epsilon_0}
\]
\[
\nabla \times \BF + \frac{\partial \BF} {\partial (ict)}
= i\frac{\BJ}{\epsilon_0 c}
\]

\subsection{Converting the curls in the pair of Maxwell's equations for the electrodynamic field to wedge and geometric products }

With multiplication of the second by a $i$ factor to convert to a wedge product representation the remaining pair of equations can be written

\[
\nabla \cdot \BF = \frac {\rho} {\epsilon_0}
\]
\[
i\nabla \times \BF + \frac{1}{c} \frac{\partial \BF}{\partial t} 
= -\frac{\BJ}{\epsilon_0 c}  
\]

This last, in terms of the geometric product is,
\[
\nabla \wedge \BF + \frac{1}{c} \frac{\partial \BF}{\partial t} 
= -\frac{\BJ}{\epsilon_0 c}  
\]

These equations can be added without loss

\[
\nabla \cdot \BF + \nabla \wedge \BF + \frac{1}{c} \frac{\partial \BF}{\partial t} = \frac {\rho} {\epsilon_0} - \frac{\BJ}{\epsilon_0 c}
\]

Leading to the end result

\[
(\frac{1}{c} \frac{\partial}{\partial t} + \nabla)\BF = \frac {1} {\epsilon_0}(\rho - \frac{\BJ}{c})
\]

Here we have all of Maxwell's equations as a single differential equation.
This gives a hint why it is hard to separately solve these equations for the electric or magnetic field components (the partials of which are scattered accross the original eight different equations.)  Logically the electric and magnetic field components have to be kept together.

Solution of this equation will require some new tools.  Minimally, some relearning of existing vector calculus tools is required.

\subsection{Components of the geometric product Maxwell equation }

Explicit expansion of this equation, using $i={\Be}_1{\Be}_2{\Be}_3$, will yield a scalar, vector, bivector, and pseudoscalar components, and is an interesting exercise to verify the simpler field equation really describes the same thing.

\[
(\frac{1}{c} \frac{\partial}{\partial t} + \nabla)\BF
= \frac{1}{c} \frac{\partial \BE}{\partial t} + i\frac{1}{c} \frac{\partial \BB}{\partial t}
+ \nabla \cdot \BE + \nabla \wedge \BE + \nabla \cdot \mathbf{iB} + \nabla \wedge \mathbf{iB}
\]

The imaginary part of the field can be multiplied out as bivector components explicitly

\[
i \BB = \Be _1 \Be _2 \Be _3 ( \Be _1 B_1 + \Be _2 B_2 + \Be _3 B_3 )
= \Be _2 \Be _3 B_1 + \Be _2 \Be _3 B_1 + \Be _1 \Be _2 B_3
\]

which allows for direct calculation of the following

\[
\nabla \wedge i\BB = i\nabla \cdot \BB
\]
\[
\nabla \wedge \BE = -\nabla \times \BB
\]

That, plus writing the electric field curl term in terms of the cross product

\[
\nabla \wedge \BE = -\nabla \times \BB
\]

This allows for grouping of real and imaginary scalar and real and imaginary vector (bivector) components

\[
   (\nabla \cdot \BE) + i(\nabla \cdot \BB)
+
   (\frac{1}{c} \frac{\partial \BE}{\partial t} - \nabla \times \BB)
+ i(\frac{1}{c} \frac{\partial \BB}{\partial t} + \nabla \times \BE)
\]
\[
= \frac{\rho}{\epsilon_0} + i(0) + (-\frac{\BJ}{\epsilon_0 c}) + i \Bzero
\]

Comparing each of the left and right side components recovers the original set of four (or eight depending on your point of view) Maxwell's equations.

\end{document}               % End of document.
