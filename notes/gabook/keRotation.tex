\documentclass{article}      % Specifies the document class

\usepackage{amsmath}
\usepackage{mathpazo}

%
% shorthand for bold symbols, convenient for vectors and matrices
%
\newcommand{\Ba}[0]{\mathbf{a}}
\newcommand{\Bb}[0]{\mathbf{b}}
\newcommand{\Bc}[0]{\mathbf{c}}
\newcommand{\Bd}[0]{\mathbf{d}}
\newcommand{\Be}[0]{\mathbf{e}}
\newcommand{\Bf}[0]{\mathbf{f}}
\newcommand{\Bg}[0]{\mathbf{g}}
\newcommand{\Bh}[0]{\mathbf{h}}
\newcommand{\Bi}[0]{\mathbf{i}}
\newcommand{\Bj}[0]{\mathbf{j}}
\newcommand{\Bk}[0]{\mathbf{k}}
\newcommand{\Bl}[0]{\mathbf{l}}
\newcommand{\Bm}[0]{\mathbf{m}}
\newcommand{\Bn}[0]{\mathbf{n}}
\newcommand{\Bo}[0]{\mathbf{o}}
\newcommand{\Bp}[0]{\mathbf{p}}
\newcommand{\Bq}[0]{\mathbf{q}}
\newcommand{\Br}[0]{\mathbf{r}}
\newcommand{\Bs}[0]{\mathbf{s}}
\newcommand{\Bt}[0]{\mathbf{t}}
\newcommand{\Bu}[0]{\mathbf{u}}
\newcommand{\Bv}[0]{\mathbf{v}}
\newcommand{\Bw}[0]{\mathbf{w}}
\newcommand{\Bx}[0]{\mathbf{x}}
\newcommand{\By}[0]{\mathbf{y}}
\newcommand{\Bz}[0]{\mathbf{z}}
\newcommand{\BA}[0]{\mathbf{A}}
\newcommand{\BB}[0]{\mathbf{B}}
\newcommand{\BC}[0]{\mathbf{C}}
\newcommand{\BD}[0]{\mathbf{D}}
\newcommand{\BE}[0]{\mathbf{E}}
\newcommand{\BF}[0]{\mathbf{F}}
\newcommand{\BG}[0]{\mathbf{G}}
\newcommand{\BH}[0]{\mathbf{H}}
\newcommand{\BI}[0]{\mathbf{I}}
\newcommand{\BJ}[0]{\mathbf{J}}
\newcommand{\BK}[0]{\mathbf{K}}
\newcommand{\BL}[0]{\mathbf{L}}
\newcommand{\BM}[0]{\mathbf{M}}
\newcommand{\BN}[0]{\mathbf{N}}
\newcommand{\BO}[0]{\mathbf{O}}
\newcommand{\BP}[0]{\mathbf{P}}
\newcommand{\BQ}[0]{\mathbf{Q}}
\newcommand{\BR}[0]{\mathbf{R}}
\newcommand{\BS}[0]{\mathbf{S}}
\newcommand{\BT}[0]{\mathbf{T}}
\newcommand{\BU}[0]{\mathbf{U}}
\newcommand{\BV}[0]{\mathbf{V}}
\newcommand{\BW}[0]{\mathbf{W}}
\newcommand{\BX}[0]{\mathbf{X}}
\newcommand{\BY}[0]{\mathbf{Y}}
\newcommand{\BZ}[0]{\mathbf{Z}}

\newcommand{\Bzero}[0]{\mathbf{0}}
\newcommand{\Btheta}[0]{\boldsymbol{\theta}}
\newcommand{\Btau}[0]{\boldsymbol{\tau}}
\newcommand{\Bomega}[0]{\boldsymbol{\omega}}

%
% shorthand for unit vectors
%
\newcommand{\acap}[0]{\hat{\Ba}}
\newcommand{\bcap}[0]{\hat{\Bb}}
\newcommand{\ccap}[0]{\hat{\Bc}}
\newcommand{\dcap}[0]{\hat{\Bd}}
\newcommand{\ecap}[0]{\hat{\Be}}
\newcommand{\fcap}[0]{\hat{\Bf}}
\newcommand{\gcap}[0]{\hat{\Bg}}
\newcommand{\hcap}[0]{\hat{\Bh}}
\newcommand{\icap}[0]{\hat{\Bi}}
\newcommand{\jcap}[0]{\hat{\Bj}}
\newcommand{\kcap}[0]{\hat{\Bk}}
\newcommand{\lcap}[0]{\hat{\Bl}}
\newcommand{\mcap}[0]{\hat{\Bm}}
\newcommand{\ncap}[0]{\hat{\Bn}}
\newcommand{\ocap}[0]{\hat{\Bo}}
\newcommand{\pcap}[0]{\hat{\Bp}}
\newcommand{\qcap}[0]{\hat{\Bq}}
\newcommand{\rcap}[0]{\hat{\Br}}
\newcommand{\scap}[0]{\hat{\Bs}}
\newcommand{\tcap}[0]{\hat{\Bt}}
\newcommand{\ucap}[0]{\hat{\Bu}}
\newcommand{\vcap}[0]{\hat{\Bv}}
\newcommand{\wcap}[0]{\hat{\Bw}}
\newcommand{\xcap}[0]{\hat{\Bx}}
\newcommand{\ycap}[0]{\hat{\By}}
\newcommand{\zcap}[0]{\hat{\Bz}}
\newcommand{\thetacap}[0]{\hat{\Btheta}}

%
% to write R^n and C^n in a distinguishable fashion.  Perhaps change this
% to the double lined characters upon figuring out how to do so.
%
\newcommand{\C}[1]{$\mathbb{C}^{#1}$}
\newcommand{\R}[1]{$\mathbb{R}^{#1}$}

%
% various generally useful helpers
%

% derivative of #1 wrt. #2:
\newcommand{\D}[2] {\frac {d#2} {d#1}}

\newcommand{\inv}[1]{\frac{1}{#1}}
\newcommand{\cross}[0]{\times}

\newcommand{\abs}[1]{\lvert{#1}\rvert}
\newcommand{\norm}[1]{\lVert{#1}\rVert}
\newcommand{\innerprod}[2]{\langle{#1}, {#2}\rangle}
\newcommand{\dotprod}[2]{{#1} \cdot {#2}}
\newcommand{\bdotprod}[2]{\left({#1} \cdot {#2}\right)}
\newcommand{\crossprod}[2]{{#1} \cross {#2}}
\newcommand{\tripleprod}[3]{\dotprod{\left(\crossprod{#1}{#2}\right)}{#3}}

\DeclareMathOperator{\Proj}{Proj}
\DeclareMathOperator{\Span}{span}
\DeclareMathOperator{\Sgn}{sgn}
\DeclareMathOperator{\Area}{Area}
\DeclareMathOperator{\Volume}{Volume}

%
% A few miscellaneous things specific to this document
%
\newcommand{\crossop}[1]{\crossprod{#1}{}}

% R2 vector.
\newcommand{\VectorTwo}[2]{
\begin{bmatrix}
 {#1} \\
 {#2}
\end{bmatrix}
}

\newcommand{\VectorN}[1]{
\begin{bmatrix}
{#1}_1 \\
{#1}_2 \\
\vdots \\
{#1}_N \\
\end{bmatrix}
}

\newcommand{\DETuvij}[4]{
\begin{vmatrix}
 {#1}_{#3} & {#1}_{#4} \\
 {#2}_{#3} & {#2}_{#4}
\end{vmatrix}
}

\newcommand{\DETuvwijk}[6]{
\begin{vmatrix}
 {#1}_{#4} & {#1}_{#5} & {#1}_{#6} \\
 {#2}_{#4} & {#2}_{#5} & {#2}_{#6} \\
 {#3}_{#4} & {#3}_{#5} & {#3}_{#6}
\end{vmatrix}
}

\newcommand{\DETuvwxijkl}[8]{
\begin{vmatrix}
 {#1}_{#5} & {#1}_{#6} & {#1}_{#7} & {#1}_{#8} \\
 {#2}_{#5} & {#2}_{#6} & {#2}_{#7} & {#2}_{#8} \\
 {#3}_{#5} & {#3}_{#6} & {#3}_{#7} & {#3}_{#8} \\
 {#4}_{#5} & {#4}_{#6} & {#4}_{#7} & {#4}_{#8} \\
\end{vmatrix}
}

%\newcommand{\DETuvwxyijklm}[10]{
%\begin{vmatrix}
% {#1}_{#6} & {#1}_{#7} & {#1}_{#8} & {#1}_{#9} & {#1}_{#10} \\
% {#2}_{#6} & {#2}_{#7} & {#2}_{#8} & {#2}_{#9} & {#2}_{#10} \\
% {#3}_{#6} & {#3}_{#7} & {#3}_{#8} & {#3}_{#9} & {#3}_{#10} \\
% {#4}_{#6} & {#4}_{#7} & {#4}_{#8} & {#4}_{#9} & {#4}_{#10} \\
% {#5}_{#6} & {#5}_{#7} & {#5}_{#8} & {#5}_{#9} & {#5}_{#10}
%\end{vmatrix}
%}

% R3 vector.
\newcommand{\VectorThree}[3]{
\begin{bmatrix}
 {#1} \\
 {#2} \\
 {#3}
\end{bmatrix}
}



\newcommand{\dt}[1]{\dot{#1}}
\newcommand{\transpose}[1]{{#1}^{\text{T}}}

\newcommand{\gpgrade}[2] {{\left\langle{{#1}}\right\rangle}_{#2}}
\newcommand{\gpgradeone}[1] {{\left\langle{{#1}}\right\rangle}_{1}}
\newcommand{\gpscalargrade}[1] {{\left\langle{{#1}}\right\rangle}}


%
% The real thing:
%

                             % The preamble begins here.
\title{ Kinetic Energy in rotational frame. } % Declares the document's title.
\author{Peeter Joot}         % Declares the author's name.
%\date{}        % Deleting this command produces today's date.

\begin{document}             % End of preamble and beginning of text.

\maketitle{}

\section{ Motivation. }

Fill in the missing details of the rotational KE derivation in Tong's classical
dynamics paper and contrast matrix and GA approach.

Given a change of coordinates to a rotating frame

\begin{equation}\label{eqn:rotcoord}
\Br' = R \Br
\end{equation}

Where

\[
R = 
\begin{bmatrix}
\cos \theta & -\sin \theta & 0 \\
\sin \theta & \cos \theta & 0 \\
0 & 0 & 1 \\
\end{bmatrix}
\]

Express the Kinetic energy of a particle in terms of the rotating frame
coordinate system.

\[
K = 
\inv{2} m \dt{\Br}^2 %= \inv{2} m \dt{\Br}^{'2}
\]

\section{ With matrix formulation. }

The tong paper does this for a z axis rotation with $\theta = \omega t$.
Constant angular frequency is assumed.

First we calculate our position vector in terms of the rotational frame

\[
\Br = R^{-1}\Br'
\]

Where

\[
R_\theta^{-1} = R_{-\theta} =
\begin{bmatrix}
\cos \theta & \sin \theta & 0 \\
-\sin \theta & \cos \theta & 0 \\
0 & 0 & 1 \\
\end{bmatrix}
\]

The velocity is:

\[
\dt{\Br} = \dt{R_{-\theta}}\Br' + R_{-\theta}\dt{\Br'}.
\]

Taking the matrix time derivative we have:

\[
\dt{R_{-\theta}} =
-\dt{\theta}
\begin{bmatrix}
\sin \theta & -\cos \theta & 0 \\
\cos \theta & \sin \theta & 0 \\
0 & 0 & 0 \\
\end{bmatrix}.
\]

Taking magnitudes of the velocity we have three terms

\[
\dt{\Br}^2 = 
\transpose{\Br'} \transpose{\dt{R_{-\theta}}} \dt{R_{-\theta}}\Br'
+2 \transpose{{\Br'}} \transpose{\dt{R_{-\theta}}} R_{-\theta} \dt{\Br'}
+\transpose{\dt{\Br'}} \transpose{R_{-\theta}} R_{-\theta}\dt{\Br'}
\]

We need to calculate all the intermediate matrix products.  The last is 
identity, and the first is:

\[
\transpose{\dt{R_{-\theta}}} \dt{R_{-\theta}}
=
{\dt{\theta}}^2
\begin{bmatrix}
\sin \theta & \cos \theta & 0 \\
-\cos \theta & \sin \theta & 0 \\
0 & 0 & 0 \\
\end{bmatrix}
\begin{bmatrix}
\sin \theta & -\cos \theta & 0 \\
\cos \theta & \sin \theta & 0 \\
0 & 0 & 0 \\
\end{bmatrix}
\]
\[
=
{\dt{\theta}}^2
\begin{bmatrix}
1 & 0 & 0 \\
0 & 1 & 0 \\
0 & 0 & 0 \\
\end{bmatrix}
\]

This leaves just the mixed term

\[
\transpose{\dt{R_{-\theta}}} {R_{-\theta}}
=
-{\dt{\theta}}
\begin{bmatrix}
\sin \theta & \cos \theta & 0 \\
-\cos \theta & \sin \theta & 0 \\
0 & 0 & 0 \\
\end{bmatrix}
\begin{bmatrix}
\cos \theta & \sin \theta & 0 \\
-\sin \theta & \cos \theta & 0 \\
0 & 0 & 1 \\
\end{bmatrix}
\]
\[
=
-{\dt{\theta}}
\begin{bmatrix}
0 & 1 & 0 \\
-1 & 0 & 0 \\
0 & 0 & 0 \\
\end{bmatrix}
\]

With $\dt{\theta} = \omega$, the total magnitude of the velocity is thus

\[
\dt{\Br}^2 = 
\transpose{\Br'} 
\omega^2
\begin{bmatrix}
1 & 0 & 0 \\
0 & 1 & 0 \\
0 & 0 & 0 \\
\end{bmatrix}
\Br'
-2 \omega \transpose{{\Br'}} 
\begin{bmatrix}
0 & 1 & 0 \\
-1 & 0 & 0 \\
0 & 0 & 0 \\
\end{bmatrix}
\dt{\Br'}
+ {\dt{\Br'}}^2
\]

Tong's paper presents this expanded out in terms of coordinates:

\[
\dt{\Br}^2 = 
\omega^2( {x'}^{2} + {y'}^{2} )
+ 2 \omega ( -x' \dt{y'} +y' \dt{x'} )
+ ( \dt{x'}^{2} + \dt{y'}^{2} + \dt{z'}^{2} )
\]

Or,
\begin{equation}\label{eqn:vmagwithmatrix}
\dt{\Br}^2 = 
+( \omega y' + \dt{x'} )^2 
+( \omega x' - \dt{y'} )^2 
+ \dt{z'}^2 
\end{equation}

Hmm.  Sign above is reversed from the text?  He also then goes on to
show that this can be written, with $\Bomega = \omega \zcap$, as 

\[
\dt{\Br}^2 = ( \dt{\Br'} + \Bomega \cross \Br')^2
\]

The implication here is that this is a valid result for any rotationing
coordinate system.   How one would actually prove this in general is 
a different story.

\section{ With rotor. }

The equivalent to equation \ref{eqn:rotcoord} using a rotor is:

\begin{equation}
\Br' = R^\dagger \Br R
\end{equation}

Where $R = \exp( i\theta/2 )$.

Unlike the 
matrix formulation above we are free to pick any constant unit bivector
for $i$ if we want to generalize this to any rotational axis, but if we
want an equivalent to the above rotation matrix we just have to take
$i = \Be_1 \wedge \Be_2$.

We need a double sided inversion to get our unprimed vector:

\[
\Br = R \Br' R^\dagger
\]

and can then take derivatives:

\[
\dt{\Br} = 
\dt{R} \Br' R^\dagger
+{R} {\Br'} \dt{R}^\dagger
+{R} \dt{\Br'} R^\dagger
\]
\[
= 
i\omega \inv{2} {R} \Br' R^\dagger
- {R} \Br' R^\dagger i\omega\inv{2}
+{R} \dt{\Br'} R^\dagger
\]
\[
= 
\omega i \cdot ({R} \Br' R^\dagger) +  {R} \dt{\Br'} R^\dagger
\]

One can put this into the traditional cross product form by introducing
a normal vector for the rotational axis in the usual way:

\[
\Bomega = \omega i / \BI_3
\]

\begin{align*}
\omega i \cdot ({R} \Br' R^\dagger)
&= \gpgradeone{ \omega i {R} \Br' R^\dagger } \\
&= \gpgradeone{ {R} \omega i \Br' R^\dagger } \\
&= {R} (\omega i) \cdot \Br' R^\dagger \\
&= {R} (\Bomega \BI_3) \cdot \Br' R^\dagger \\
&= {R} (\Bomega \cross \Br') R^\dagger \\
\end{align*}

Summarizing, we can write our velocity using rotational frame components
as: 
\begin{equation}\label{eqn:vrotcross}
\dt{\Br} = {R} \left( \Bomega \cross \Br' + \dt{\Br'} \right) R^\dagger
\end{equation}
Or
\begin{equation}
\dt{\Br} = {R} \left( \omega i \cdot \Br' + \dt{\Br'} \right) R^\dagger
\end{equation}

Using the result above from equation \ref{eqn:vrotcross}, we can calculate
the squared magnitude directly:

\begin{align*}
\dt{\Br} ^2 
&= \gpscalargrade{ 
{R} \left( \Bomega \cross \Br' + \dt{\Br'} \right) R^\dagger
{R} \left( \Bomega \cross \Br' + \dt{\Br'} \right) R^\dagger
} \\
&= \gpscalargrade{ 
{R} ( \Bomega \cross \Br' + \dt{\Br'} ) ^2 R^\dagger
} \\
&= ( \Bomega \cross \Br' + \dt{\Br'} ) ^2 \\
\end{align*}

We are able to go straight to the end result this way without the mess
of sine and cosine terms in the rotation matrix.  This is something that
we can expand by components if desired:

\begin{align*}
\Bomega \cross \Br' + \dt{\Br'}
&= 
\begin{vmatrix}
\Be_1 & \Be_2 & \Be_3 \\
0 & 0 & \omega \\
x' & y' & z' \\
\end{vmatrix}
+ \dt{\Br'} \\
&=
\begin{bmatrix}
-\omega y' + \dt{x'} \\
\omega x' + \dt{y'} \\
 \dt{z'} \\
\end{bmatrix}
\end{align*}

This verifies the second part of Tong's equation 2.19, and has a sign that
doesn't match with equation \ref{eqn:vmagwithmatrix}, which confirms that
there's likely a sign mixup above leading to that original result.

\end{document}               % End of document.
