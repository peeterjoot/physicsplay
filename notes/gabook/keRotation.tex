\documentclass{article}      % Specifies the document class

\usepackage{amsmath}
\usepackage{mathpazo}

%
% shorthand for bold symbols, convenient for vectors and matrices
%
\newcommand{\Ba}[0]{\mathbf{a}}
\newcommand{\Bb}[0]{\mathbf{b}}
\newcommand{\Bc}[0]{\mathbf{c}}
\newcommand{\Bd}[0]{\mathbf{d}}
\newcommand{\Be}[0]{\mathbf{e}}
\newcommand{\Bf}[0]{\mathbf{f}}
\newcommand{\Bg}[0]{\mathbf{g}}
\newcommand{\Bh}[0]{\mathbf{h}}
\newcommand{\Bi}[0]{\mathbf{i}}
\newcommand{\Bj}[0]{\mathbf{j}}
\newcommand{\Bk}[0]{\mathbf{k}}
\newcommand{\Bl}[0]{\mathbf{l}}
\newcommand{\Bm}[0]{\mathbf{m}}
\newcommand{\Bn}[0]{\mathbf{n}}
\newcommand{\Bo}[0]{\mathbf{o}}
\newcommand{\Bp}[0]{\mathbf{p}}
\newcommand{\Bq}[0]{\mathbf{q}}
\newcommand{\Br}[0]{\mathbf{r}}
\newcommand{\Bs}[0]{\mathbf{s}}
\newcommand{\Bt}[0]{\mathbf{t}}
\newcommand{\Bu}[0]{\mathbf{u}}
\newcommand{\Bv}[0]{\mathbf{v}}
\newcommand{\Bw}[0]{\mathbf{w}}
\newcommand{\Bx}[0]{\mathbf{x}}
\newcommand{\By}[0]{\mathbf{y}}
\newcommand{\Bz}[0]{\mathbf{z}}
\newcommand{\BA}[0]{\mathbf{A}}
\newcommand{\BB}[0]{\mathbf{B}}
\newcommand{\BC}[0]{\mathbf{C}}
\newcommand{\BD}[0]{\mathbf{D}}
\newcommand{\BE}[0]{\mathbf{E}}
\newcommand{\BF}[0]{\mathbf{F}}
\newcommand{\BG}[0]{\mathbf{G}}
\newcommand{\BH}[0]{\mathbf{H}}
\newcommand{\BI}[0]{\mathbf{I}}
\newcommand{\BJ}[0]{\mathbf{J}}
\newcommand{\BK}[0]{\mathbf{K}}
\newcommand{\BL}[0]{\mathbf{L}}
\newcommand{\BM}[0]{\mathbf{M}}
\newcommand{\BN}[0]{\mathbf{N}}
\newcommand{\BO}[0]{\mathbf{O}}
\newcommand{\BP}[0]{\mathbf{P}}
\newcommand{\BQ}[0]{\mathbf{Q}}
\newcommand{\BR}[0]{\mathbf{R}}
\newcommand{\BS}[0]{\mathbf{S}}
\newcommand{\BT}[0]{\mathbf{T}}
\newcommand{\BU}[0]{\mathbf{U}}
\newcommand{\BV}[0]{\mathbf{V}}
\newcommand{\BW}[0]{\mathbf{W}}
\newcommand{\BX}[0]{\mathbf{X}}
\newcommand{\BY}[0]{\mathbf{Y}}
\newcommand{\BZ}[0]{\mathbf{Z}}

\newcommand{\Bzero}[0]{\mathbf{0}}
\newcommand{\Btheta}[0]{\boldsymbol{\theta}}
\newcommand{\Btau}[0]{\boldsymbol{\tau}}
\newcommand{\Bomega}[0]{\boldsymbol{\omega}}

%
% shorthand for unit vectors
%
\newcommand{\acap}[0]{\hat{\Ba}}
\newcommand{\bcap}[0]{\hat{\Bb}}
\newcommand{\ccap}[0]{\hat{\Bc}}
\newcommand{\dcap}[0]{\hat{\Bd}}
\newcommand{\ecap}[0]{\hat{\Be}}
\newcommand{\fcap}[0]{\hat{\Bf}}
\newcommand{\gcap}[0]{\hat{\Bg}}
\newcommand{\hcap}[0]{\hat{\Bh}}
\newcommand{\icap}[0]{\hat{\Bi}}
\newcommand{\jcap}[0]{\hat{\Bj}}
\newcommand{\kcap}[0]{\hat{\Bk}}
\newcommand{\lcap}[0]{\hat{\Bl}}
\newcommand{\mcap}[0]{\hat{\Bm}}
\newcommand{\ncap}[0]{\hat{\Bn}}
\newcommand{\ocap}[0]{\hat{\Bo}}
\newcommand{\pcap}[0]{\hat{\Bp}}
\newcommand{\qcap}[0]{\hat{\Bq}}
\newcommand{\rcap}[0]{\hat{\Br}}
\newcommand{\scap}[0]{\hat{\Bs}}
\newcommand{\tcap}[0]{\hat{\Bt}}
\newcommand{\ucap}[0]{\hat{\Bu}}
\newcommand{\vcap}[0]{\hat{\Bv}}
\newcommand{\wcap}[0]{\hat{\Bw}}
\newcommand{\xcap}[0]{\hat{\Bx}}
\newcommand{\ycap}[0]{\hat{\By}}
\newcommand{\zcap}[0]{\hat{\Bz}}
\newcommand{\thetacap}[0]{\hat{\Btheta}}

%
% to write R^n and C^n in a distinguishable fashion.  Perhaps change this
% to the double lined characters upon figuring out how to do so.
%
\newcommand{\C}[1]{$\mathbb{C}^{#1}$}
\newcommand{\R}[1]{$\mathbb{R}^{#1}$}

%
% various generally useful helpers
%

% derivative of #1 wrt. #2:
\newcommand{\D}[2] {\frac {d#2} {d#1}}

\newcommand{\inv}[1]{\frac{1}{#1}}
\newcommand{\cross}[0]{\times}

\newcommand{\abs}[1]{\lvert{#1}\rvert}
\newcommand{\norm}[1]{\lVert{#1}\rVert}
\newcommand{\innerprod}[2]{\langle{#1}, {#2}\rangle}
\newcommand{\dotprod}[2]{{#1} \cdot {#2}}
\newcommand{\bdotprod}[2]{\left({#1} \cdot {#2}\right)}
\newcommand{\crossprod}[2]{{#1} \cross {#2}}
\newcommand{\tripleprod}[3]{\dotprod{\left(\crossprod{#1}{#2}\right)}{#3}}

\DeclareMathOperator{\Proj}{Proj}
\DeclareMathOperator{\Span}{span}
\DeclareMathOperator{\Sgn}{sgn}
\DeclareMathOperator{\Area}{Area}
\DeclareMathOperator{\Volume}{Volume}

%
% A few miscellaneous things specific to this document
%
\newcommand{\crossop}[1]{\crossprod{#1}{}}

% R2 vector.
\newcommand{\VectorTwo}[2]{
\begin{bmatrix}
 {#1} \\
 {#2}
\end{bmatrix}
}

\newcommand{\VectorN}[1]{
\begin{bmatrix}
{#1}_1 \\
{#1}_2 \\
\vdots \\
{#1}_N \\
\end{bmatrix}
}

\newcommand{\DETuvij}[4]{
\begin{vmatrix}
 {#1}_{#3} & {#1}_{#4} \\
 {#2}_{#3} & {#2}_{#4}
\end{vmatrix}
}

\newcommand{\DETuvwijk}[6]{
\begin{vmatrix}
 {#1}_{#4} & {#1}_{#5} & {#1}_{#6} \\
 {#2}_{#4} & {#2}_{#5} & {#2}_{#6} \\
 {#3}_{#4} & {#3}_{#5} & {#3}_{#6}
\end{vmatrix}
}

\newcommand{\DETuvwxijkl}[8]{
\begin{vmatrix}
 {#1}_{#5} & {#1}_{#6} & {#1}_{#7} & {#1}_{#8} \\
 {#2}_{#5} & {#2}_{#6} & {#2}_{#7} & {#2}_{#8} \\
 {#3}_{#5} & {#3}_{#6} & {#3}_{#7} & {#3}_{#8} \\
 {#4}_{#5} & {#4}_{#6} & {#4}_{#7} & {#4}_{#8} \\
\end{vmatrix}
}

%\newcommand{\DETuvwxyijklm}[10]{
%\begin{vmatrix}
% {#1}_{#6} & {#1}_{#7} & {#1}_{#8} & {#1}_{#9} & {#1}_{#10} \\
% {#2}_{#6} & {#2}_{#7} & {#2}_{#8} & {#2}_{#9} & {#2}_{#10} \\
% {#3}_{#6} & {#3}_{#7} & {#3}_{#8} & {#3}_{#9} & {#3}_{#10} \\
% {#4}_{#6} & {#4}_{#7} & {#4}_{#8} & {#4}_{#9} & {#4}_{#10} \\
% {#5}_{#6} & {#5}_{#7} & {#5}_{#8} & {#5}_{#9} & {#5}_{#10}
%\end{vmatrix}
%}

% R3 vector.
\newcommand{\VectorThree}[3]{
\begin{bmatrix}
 {#1} \\
 {#2} \\
 {#3}
\end{bmatrix}
}



\newcommand{\dt}[1]{\dot{#1}}
\newcommand{\ddt}[1]{\ddot{#1}}
\newcommand{\transpose}[1]{{#1}^{\text{T}}}
\newcommand{\Balpha}[0]{\boldsymbol{\alpha}}

\newcommand{\gpgrade}[2] {{\left\langle{{#1}}\right\rangle}_{#2}}
\newcommand{\gpgradeone}[1] {{\left\langle{{#1}}\right\rangle}_{1}}
\newcommand{\gpscalargrade}[1] {{\left\langle{{#1}}\right\rangle}}
\newcommand{\BOmega}[0]{\boldsymbol{\Omega}}


%
% The real thing:
%

                             % The preamble begins here.
\title{ Kinetic Energy in rotational frame. } % Declares the document's title.
\author{Peeter Joot \quad peeter.joot@gmail.com}         % Declares the author's name.
\date{ April 30, 2008.  Last Revision: $Date: 2009/02/22 15:35:07 $ }

\begin{document}             % End of preamble and beginning of text.

\maketitle{}

\section{ Motivation. }

Fill in the missing details of the rotational KE derivation in Tong's classical
dynamics paper and contrast matrix and GA approach.

Generalize acceleration in terms
of rotating frame coordinates without unproved extrapolation that the z axis result
of Tong's paper is good unconditionally (his cross products are kind of pulled out of
a magic hat and this write up will show a couple ways to see where they come from).

Given coordinates for a point in a rotating frame $\Br'$, the coordinate vector for that point
in a rest frame is:

\begin{equation}\label{eqn:rotcoord}
\Br = R \Br'
\end{equation}

Where the rotating frame moves according to the following z-axis rotation matrix:

\[
R = 
\begin{bmatrix}
\cos \theta & -\sin \theta & 0 \\
\sin \theta & \cos \theta & 0 \\
0 & 0 & 1 \\
\end{bmatrix}
\]

To compute the Lagrangian we want to reexpress the 
kinetic energy of a particle:

\[
K = 
\inv{2} m \dt{\Br}^2
\]

in terms of the rotating frame coordinate system.

\section{ With matrix formulation. }

The Tong paper does this for a z axis rotation with $\theta = \omega t$.
Constant angular frequency is assumed.

First we calculate our position vector in terms of the rotational frame

\[
\Br = R\Br'
\]

%Where
%
%\[
%R_\theta^{-1} = R_{-\theta} =
%\begin{bmatrix}
%\cos \theta & \sin \theta & 0 \\
%-\sin \theta & \cos \theta & 0 \\
%0 & 0 & 1 \\
%\end{bmatrix}
%\]

The rest frame velocity is:

\[
\dt{\Br} = \dt{R}_{\theta} \Br' + R_{\theta} \dt{\Br'}.
\]

Taking the matrix time derivative we have:

\[
\dt{R}_{\theta} =
-\dt{\theta}
\begin{bmatrix}
\sin \theta & \cos \theta & 0 \\
-\cos \theta & \sin \theta & 0 \\
0 & 0 & 0 \\
\end{bmatrix}.
\]

Taking magnitudes of the velocity we have three terms

\begin{align*}
\dt{\Br}^2 
&= 
(\dt{R}_{\theta} \Br') \cdot (\dt{R}_{\theta} \Br')
+2 (\dt{R}_{\theta} \Br') \cdot (R_{\theta} \dt{\Br'})
+(R_{\theta} \dt{\Br'}) \cdot (R_{\theta} \dt{\Br'}) \\
&= 
\transpose{\Br'}\transpose{\dt{R}_{\theta}} \dt{R}_{\theta} \Br'
+2 \transpose{\Br'} \transpose{\dt{R}_{\theta}} R_{\theta} \dt{\Br'}
+\dt{\Br'}^2 \\
\end{align*}

We need to calculate all the intermediate matrix products.  The last was 
identity, and the first is:

\[
\transpose{\dt{R}_{\theta}} \dt{R}_{\theta}
=
{\dt{\theta}}^2
\begin{bmatrix}
\sin \theta & -\cos \theta & 0 \\
\cos \theta & \sin \theta & 0 \\
0 & 0 & 0 \\
\end{bmatrix}
\begin{bmatrix}
\sin \theta & \cos \theta & 0 \\
-\cos \theta & \sin \theta & 0 \\
0 & 0 & 0 \\
\end{bmatrix}
\]
\[
=
{\dt{\theta}}^2
\begin{bmatrix}
1 & 0 & 0 \\
0 & 1 & 0 \\
0 & 0 & 0 \\
\end{bmatrix}
\]

This leaves just the mixed term

\[
\transpose{\dt{R}_{\theta}} {R_{\theta}}
=
-{\dt{\theta}}
\begin{bmatrix}
\sin \theta & -\cos \theta & 0 \\
\cos \theta & \sin \theta & 0 \\
0 & 0 & 0 \\
\end{bmatrix}
\begin{bmatrix}
\cos \theta & -\sin \theta & 0 \\
\sin \theta & \cos \theta & 0 \\
0 & 0 & 1 \\
\end{bmatrix}
\]
\[
=
-{\dt{\theta}}
\begin{bmatrix}
0 & -1 & 0 \\
1 & 0 & 0 \\
0 & 0 & 0 \\
\end{bmatrix}
\]

With $\dt{\theta} = \omega$, the total magnitude of the velocity is thus

\[
\dt{\Br}^2 = 
\transpose{\Br'} 
\omega^2
\begin{bmatrix}
1 & 0 & 0 \\
0 & 1 & 0 \\
0 & 0 & 0 \\
\end{bmatrix}
\Br'
-2 \omega \transpose{{\Br'}} 
\begin{bmatrix}
0 & -1 & 0 \\
1 & 0 & 0 \\
0 & 0 & 0 \\
\end{bmatrix}
\dt{\Br'}
+ {\dt{\Br'}}^2
\]

Tong's paper presents this expanded out in terms of coordinates:

\[
\dt{\Br}^2 = 
\omega^2\left( {x'}^{2} + {y'}^{2} \right)
+ 2 \omega \left( x' \dt{y'} -y' \dt{x'} \right)
+ \left( \dt{x'}^{2} + \dt{y'}^{2} + \dt{z'}^{2} \right)
\]

Or,
\begin{equation}\label{eqn:vmagwithmatrix}
\dt{\Br}^2 = 
\left( -\omega y' + \dt{x'} \right)^2 
+\left( \omega x' + \dt{y'} \right)^2 
+ \dt{z'}^2 
\end{equation}

He also then goes on to
show that this can be written, with $\Bomega = \omega \zcap$, as 

\[
\dt{\Br}^2 = ( \dt{\Br'} + \Bomega \cross \Br')^2
\]

The implication here is that this is a valid result for any rotating
coordinate system.   How to prove this in the general rotation case, is shown much later
in his treatment of rigid bodies.

\section{ With rotor. }

The equivalent to equation \ref{eqn:rotcoord} using a rotor is:

\begin{equation}
\Br' = R^\dagger \Br R
\end{equation}

Where $R = \exp( i\theta/2 )$.

Unlike the 
matrix formulation above we are free to pick any constant unit bivector
for $i$ if we want to generalize this to any rotational axis, but if we
want an equivalent to the above rotation matrix we just have to take
$i = \Be_1 \wedge \Be_2$.

We need a double sided inversion to get our unprimed vector:

\[
\Br = R \Br' R^\dagger
\]

and can then take derivatives:

\[
\dt{\Br} = 
\dt{R} \Br' R^\dagger
+{R} {\Br'} \dt{R}^\dagger
+{R} \dt{\Br'} R^\dagger
\]
\[
= 
i\omega \inv{2} {R} \Br' R^\dagger
- {R} \Br' R^\dagger i\omega\inv{2}
+{R} \dt{\Br'} R^\dagger
\]
\begin{equation}\label{eqn:velocityfixedrotplane}
\implies
\dt{\Br} = \omega i \cdot ({R} \Br' R^\dagger) +  {R} \dt{\Br'} R^\dagger
\end{equation}

One can put this into the traditional cross product form by introducing
a normal vector for the rotational axis in the usual way:

\[
\BOmega = \omega i
\]
\[
\Bomega = \BOmega / \BI_3
\]

We can describe the angular velocity by a scaled normal vector $(\Bomega)$ to the rotational plane, or by a scaled bivector for the plane itself ($\BOmega$).

\begin{align*}
\BOmega \cdot ({R} \Br' R^\dagger)
&= \gpgradeone{ \BOmega {R} \Br' R^\dagger } \\
&= \gpgradeone{ {R} \BOmega \Br' R^\dagger } \\
&= {R} \BOmega \cdot \Br' R^\dagger \\
&= {R} (\Bomega \BI_3) \cdot \Br' R^\dagger \\
&= {R} (\Bomega \cross \Br') R^\dagger \\
\end{align*}

Note that here as before this is valid only when the rotational plane orientation is constant (ie: no wobble), since only then can we assume $i$, and thus $\BOmega$ will commute with the rotor $R$.

Summarizing, we can write our velocity using rotational frame components
as: 
\begin{equation}\label{eqn:vrotcross}
\dt{\Br} = {R} \left( \Bomega \cross \Br' + \dt{\Br'} \right) R^\dagger
\end{equation}
Or
\begin{equation}
\dt{\Br} = {R} \left( \BOmega \cdot \Br' + \dt{\Br'} \right) R^\dagger
\end{equation}

Using the result above from equation \ref{eqn:vrotcross}, we can calculate
the squared magnitude directly:

\begin{align*}
\dt{\Br} ^2 
&= \gpscalargrade{ 
{R} \left( \Bomega \cross \Br' + \dt{\Br'} \right) R^\dagger
{R} \left( \Bomega \cross \Br' + \dt{\Br'} \right) R^\dagger
} \\
&= \gpscalargrade{ 
{R} ( \Bomega \cross \Br' + \dt{\Br'} ) ^2 R^\dagger
} \\
&= ( \Bomega \cross \Br' + \dt{\Br'} ) ^2 \\
\end{align*}

We are able to go straight to the end result this way without the mess
of sine and cosine terms in the rotation matrix.  This is something that
we can expand by components if desired:

\begin{align*}
\Bomega \cross \Br' + \dt{\Br'}
&= 
\begin{vmatrix}
\Be_1 & \Be_2 & \Be_3 \\
0 & 0 & \omega \\
x' & y' & z' \\
\end{vmatrix}
+ \dt{\Br'} \\
&=
\begin{bmatrix}
-\omega y' + \dt{x'} \\
\omega x' + \dt{y'} \\
 \dt{z'} \\
\end{bmatrix}
\end{align*}

This verifies the second part of Tong's equation 2.19, also consistent with the
derivation of equation \ref{eqn:vmagwithmatrix}.

\section{ Acceleration in rotating coordinates. }

Having calculated velocity in terms of rotational frame coordinates, acceleration is the next
logical step.

The starting point is the velocity

\[
\dt{\Br} = R ( \BOmega \cdot \Br' + \dt{\Br}' ) R^\dagger
\]

Taking deriviatives we have
\[
\ddt{\Br} = i \omega /2 \dt{\Br} - \dt{\Br} i \omega /2 + R \left( \dot{\BOmega} \cdot \Br' + \BOmega \cdot \dt{\Br}' + \ddt{\Br}' \right) R^\dagger
\]

The first two terms are a bivector vector dot product and we can simplify this as follows

\begin{align*}
i \omega /2 \dt{\Br} - \dt{\Br} i \omega /2
&= \BOmega /2 \dt{\Br} - \dt{\Br} \BOmega \\
&= \BOmega \cdot \dt{\Br} \\
&= \gpgradeone{ \BOmega R ( \BOmega \cdot \Br' + \dt{\Br}' ) R^\dagger } \\
&= \gpgradeone{ R ( \BOmega (\BOmega \cdot \Br' + \dt{\Br}') ) R^\dagger } \\
&= R ( \BOmega \cdot (\BOmega \cdot \Br') + \BOmega \cdot \dt{\Br}') R^\dagger \\
\end{align*}

Thus the total acceleration is

\begin{equation}
\ddt{\Br} = R \left( \BOmega \cdot (\BOmega \cdot \Br') +\dot{\BOmega} \cdot \Br' + 2 \BOmega \cdot \dt{\Br}' + \ddt{\Br}' \right) R^\dagger
\end{equation}

Or, in terms of cross products, and angular velocity and acceleration vectors $\Bomega$, and $\Balpha$ respectively, this is

\begin{equation}\label{eqn:accelerationfixedrotplane}
\ddt{\Br} = R \left( \Bomega \cross (\Bomega \cross \Br') + \Balpha \cross \Br' + 2 \Bomega \cross \dt{\Br}' + \ddt{\Br}' \right) R^\dagger
\end{equation}

\section{ Allow for a wobble in rotational plane. }

A calculation similar to this can be found in GAFP, but for strictly rigid motion.  It doesn't take too much to combine the two for a generalized result that
expresses the total acceleration expressed in rotating frame coordinates, but also allowing for general rotation where the frame rotation and the angular velocity
bivector don't have to be coplanar (ie: commute as above).

Since the primes and dots are kind of cumbersome switch to the GAFP notation where the position of a particle is expressed in terns of a rotational component $\Bx$
and origin translation $\Bx_0$:

\[
\By = R \Bx R^\dagger + \Bx_0
\]

Taking derivatives for velocity

\begin{equation}\label{eqn:velocity}
\dt{\By} = \dt{R} \Bx R^\dagger +R \Bx \dt{R^\dagger} +R \dt{\Bx} R^\dagger + \dt{\Bx}_0
\end{equation}

Now use the same observation that the derivative of $R R^\dagger = 1$ is zero:

\begin{equation*}
\frac{d (R R^\dagger)}{dt} = \dt{R}R^\dagger + R \dt{R^\dagger} = 0
\end{equation*}
\begin{equation}\label{eqn:rotdtrot}
\implies
\dt{R}R^\dagger = 
%-R \dt{R^\dagger} =
 - R \dt{R}^\dagger = -{\left( \dt{R} R^\dagger \right)}^\dagger
\end{equation}

Since $R$ has only grade 0 and 2 terms, so does its derivative.  Thus the product of the two has grade 0, 2, and 4 terms, but 
equation \ref{eqn:rotdtrot} shows that the product $\dt{R} R^\dagger$ has a value that is the negative of its reverse, so it must have
only grade 2 terms (the reverse of the grade 0 and 4 terms would not change sign).

As in equation \ref{eqn:velocityfixedrotplane} we want to write $\dt{R}$ as a bivector/rotor product and equation \ref{eqn:rotdtrot} gives us a means to do so.
This would have been clearer in GAFP if they had done the simple example first with the orientation of the rotational plane fixed.

So, write:

\[
\dt{R}R^\dagger = \inv{2}\BOmega
\]
\[
\dt{R} = \inv{2}\BOmega R
\]
\[
\dt{R}^\dagger = - \inv{2} R^\dagger \BOmega
\]

(including the $1/2$ here is a bit of a cheat ... it's here because having done the calculation on paper first one sees that it's natural to do so).

With this we can substitute back into equation \ref{eqn:velocity}, writing $\By_0 = \By - \Bx_0$ :

\begin{align*}
\dt{\By}
&= \inv{2} \BOmega {R} \Bx R^\dagger - \inv{2} R \Bx {R^\dagger} \BOmega +R \dt{\Bx} R^\dagger + \dt{\Bx}_0 \\
&= \inv{2}\left(\BOmega \By_ - \By_0 \BOmega\right) +R \dt{\Bx} R^\dagger + \dt{\Bx}_0 \\
&= \BOmega \cdot \By_0 +R \dt{\Bx} R^\dagger + \dt{\Bx}_0 \\
\end{align*}

We also want to pull in this $\BOmega$ into the rotor as in the fixed orientation
case, but cannot use commutivity this time since the rotor and angular velocity bivector aren't neccessarily in the same plane.

This is where GAFP introduces their body angular velocity, which applies an inverse rotation to the angular velocity.

Let:
\[
\BOmega = R \BOmega_B R^\dagger
\]

Computing this bivector dot product with $\By$ we have

\begin{align*}
\BOmega \cdot \By_0
&= (R \BOmega_B R^\dagger) \cdot (R \Bx R^\dagger) \\
&= \gpgradeone{R \BOmega_B R^\dagger R \Bx R^\dagger} \\
&= \gpgradeone{R \BOmega_B \Bx R^\dagger} \\
&= \gpgradeone{R (\BOmega_B \cdot \Bx + \BOmega_B \wedge \Bx) R^\dagger} \\
&= {R \BOmega_B \cdot \Bx R^\dagger} \\
\end{align*}

Thus the total velocity is:

\begin{equation}
\dt{\By} = {R (\BOmega_B \cdot \Bx + \dt{\Bx} )R^\dagger} + \dt{\Bx}_0
\end{equation}

Thus given any vector $\Bx$ in the rotating frame coordinate system, we have the relationship for the inertial frame velocity.  We can apply this a second
time to compute the inertial (rest frame) acceleration in terms of rotating coordinates.  Write $\Bv = \BOmega_B \cdot \Bx + \dt{\Bx}$, 

\begin{equation*}
\dt{\By} = {R \Bv R^\dagger} + \dt{\Bx}_0
\end{equation*}
\begin{equation*}
\implies
\ddt{\By} = {R (\BOmega_B \cdot \Bv + \dt{\Bv} ) R^\dagger} + \ddt{\Bx}_0
\end{equation*}

\[
\dt{\Bv} = 
\dt{\BOmega}_B \cdot \Bx 
+\BOmega_B \cdot \dt{\Bx}
+ \ddt{\Bx}
\]

Combining these we have:
\begin{align*}
\ddt{\By} 
&= {R (\BOmega_B \cdot ( \BOmega_B \cdot \Bx + \dt{\Bx} ) + \dt{\BOmega}_B \cdot \Bx +\BOmega_B \cdot \dt{\Bx} + \ddt{\Bx}) R^\dagger} + \ddt{\Bx}_0 \\
\end{align*}

\begin{equation}
\implies
\ddt{\By} 
= {R (\BOmega_B \cdot ( \BOmega_B \cdot \Bx ) + \dt{\BOmega}_B \cdot \Bx + 2\BOmega_B \cdot \dt{\Bx} + \ddt{\Bx}) R^\dagger} + \ddt{\Bx}_0
\end{equation}

This generalizes equation \ref{eqn:accelerationfixedrotplane}, providing the rest frame acceleration in terms of rotational frame coordinates, with centrifugal acceleration, euler force acceleration, and corolis force acceleration terms that accompany the plain old acceleration term $\ddt{\Bx}$.  The only
requirement for the generality of allowing the orientation of the rotational plane to potentially vary is the use of the ``body angular velocity''
$\BOmega_B$, replacing the angular velocity as seen from the rest frame $\BOmega$.

\subsection{ Body angular acceleration in terms of rest frame. }

Since we know the relationship between the body angular velocity $\BOmega_B$ with the Rotor (rest frame) angular velocity bivector, for
completeness, lets compute the body angular acceleration bivector $\dt{\BOmega}_B$ in terms of the rest frame angular acceleration $\dt{\BOmega}$.

\[
\BOmega_B = R^\dagger \BOmega R
\]
\begin{align*}
\implies
\dt{\BOmega}_B 
&= \dt{R}^\dagger \BOmega R +R^\dagger \dt{\BOmega} R +R^\dagger \BOmega \dt{R} \\
&= -\inv{2}R^\dagger \BOmega^2 R +R^\dagger \dt{\BOmega} R +R^\dagger \BOmega^2 R \inv{2} \\
&= \inv{2} \left(R^\dagger \BOmega^2 R - R^\dagger \BOmega^2 R\right) +R^\dagger \dt{\BOmega} R \\
&= R^\dagger \dt{\BOmega} R \\
\end{align*}

This shows that the body angular acceleration is just an inverse rotation of the rest frame angular acceleration like the angular velocities are.

\section{ Revisit general rotation using matrixes. }

Having fully calculated velocity and acceleration in terms of rotating frame coordinates, lets
go back and revisit this with matrixes and see how one would do the same for a general rotation.

Following GAFP express the rest frame coordinates for a point $\By$ in terms of a rotation
applied to a rotating frame position $\Bx$ (this is easier than the mess of primes and dots
used in Tong's paper).  Also omit the origin translation (that can be added in later if desired
easily enough)

\[
\By = R \Bx
\]

Thus the deriviative is:

\[
\dt{\By} = \dt{R} \Bx + R \dt{\Bx}.
\]

As in the GA case we want to factor this so that we have a rotation applied to a something
that is completely specified in the rotating frame.  This
is quite easy with matrixes, as we just have to factor out a rotation matrix from $\dt{R}$:

\begin{align*}
\dt{\By} 
&= R \transpose{R}\dt{R} \Bx + R \dt{\Bx} \\
&= R \left(\transpose{R}\dt{R} \Bx + \dt{\Bx} \right) \\
\end{align*}

This new product $\transpose{R}\dt{R} \Bx$ we have seen above in the special case of z-axis
rotation as a cross product.  In the GA general rotation case, we've seen that this as a
bivector-vector dot product.  Both of these are fundamentally antisymmetric operations,
so we expect this of the matrix operator too.  Verification of this antisymmetry follows
in almost the same fashion as the GA case, by observing that the deriviative of an identity
matrix $I = \transpose{R}R$ is zero:

\[
\dt{I} = 0
\]
\[
\implies
\transpose{\dt{R}}R + \transpose{R}\dt{R} = 0
\]
\[
\implies
\transpose{R}\dt{R} = -\transpose{\dt{R}}R = -\transpose{\transpose{R}\dt{R}}
\]

Thus if one writes:

\begin{equation}\label{eqn:bodyangularvelocitymatrix}
\BOmega = \transpose{R}\dt{R}
\end{equation}

the antisymetric property of this matrix can be summarized as:

\[
\BOmega = -\transpose{\BOmega}.
\]

Let's write out the form of this matrix in the first few dimensions:

\begin{itemize}
\item \R{2}

\[
\BOmega = 
\begin{bmatrix}
0 & -a \\
a & 0  \\
\end{bmatrix}
\]

For some $a$.

\item \R{3}

\[
\BOmega = 
\begin{bmatrix}
0 & -a & -b \\
a &  0 & -c \\
b &  c &  0 \\
\end{bmatrix}
\]

For some $a, b, c$.

\item \R{4}

\[
\BOmega = 
\begin{bmatrix}
0 & -a & -b & -d \\
a & 0 & -c & -e \\
b & c & 0 & -f \\
d & e & f & 0 \\
\end{bmatrix}
\]

For some $a, b, c, d, e, f$.
\end{itemize}

For \R{N} we have $(N^2-N)/2$ degrees of freedom.  It's noteworthy to observe that this is exactly the number of basis elements of a bivector.  For example, in \R{4}, such a bivector basis is
$\Be_{12}, \Be_{13}, \Be_{14}, \Be_{23}, \Be_{24}, \Be_{34}$.

For \R{3} we have three degrees of freedom and because of the antisymmetry 
can express this matrix-vector product using the cross product.  Let

\[
(a,b,c) = (\omega_3, -\omega_2, \omega_1)
\]

One has:

\[
\BOmega \Bx = 
\begin{bmatrix}
0 & -\omega_3 & \omega_2 \\
\omega_3 &  0 & -\omega_1 \\
-\omega_2 & \omega_1 &  0 \\
\end{bmatrix}
\begin{bmatrix}
x_1 \\
x_2 \\
x_3 \\
\end{bmatrix}
=
\begin{bmatrix}
-\omega_3 x_2 +\omega_2 x_3 \\
+\omega_3 x_1 -\omega_1 x_3 \\
-\omega_2 x_1 +\omega_1 x_2 \\
\end{bmatrix}
= \Bomega \cross \Bx
\]

Summarizing the velocity result we have, using $\BOmega$ from equation \ref{eqn:bodyangularvelocitymatrix}:

\begin{equation}
\dt{\By} = R \left( \BOmega \Bx + \dt{\Bx} \right)
\end{equation}

Or, for \R{3}, we can define a body angular velocity vector

\begin{equation}
\Bomega = 
\begin{bmatrix}
\BOmega_{32} \\
\BOmega_{13} \\
\BOmega_{21} \\
\end{bmatrix}
\end{equation}

and thus write the velocity as:

\begin{equation}
\dt{\By} = R \left( \Bomega \cross \Bx + \dt{\Bx} \right)
\end{equation}

This, like the GA result is good for general rotations.  Then don't have to be constant
rotation rates, and it allows for arbitrarily
oriented as well as wobbly motion of the rotating frame.

As with the GA general velocity calculation, this general form also allows us to calculate
the squared velocity easily, since the rotation matrixes will
cancel after transposition:

\[
\dt{\By}^2 = 
\left(R \left( \Bomega \cross \Bx + \dt{\Bx} \right)\right) \cdot
\left(R \left( \Bomega \cross \Bx + \dt{\Bx} \right)\right)
=
\transpose{\left( \Bomega \cross \Bx + \dt{\Bx} \right)} \transpose{R}
R \left( \Bomega \cross \Bx + \dt{\Bx} \right)
\]
\[
\implies
\dt{\By}^2 = 
{\left( \Bomega \cross \Bx + \dt{\Bx} \right)}^2
\]

\section{ Equations of motion from Lagrange partials. }

TBD.  Do this using the Rotor formulation.  How?

\end{document}               % End of document.
