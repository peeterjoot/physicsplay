\documentclass{article}

\usepackage{amsmath}
\usepackage{mathpazo}

%
% shorthand for bold symbols, convenient for vectors and matrices
%
\newcommand{\Ba}[0]{\mathbf{a}}
\newcommand{\Bb}[0]{\mathbf{b}}
\newcommand{\Bc}[0]{\mathbf{c}}
\newcommand{\Bd}[0]{\mathbf{d}}
\newcommand{\Be}[0]{\mathbf{e}}
\newcommand{\Bf}[0]{\mathbf{f}}
\newcommand{\Bg}[0]{\mathbf{g}}
\newcommand{\Bh}[0]{\mathbf{h}}
\newcommand{\Bi}[0]{\mathbf{i}}
\newcommand{\Bj}[0]{\mathbf{j}}
\newcommand{\Bk}[0]{\mathbf{k}}
\newcommand{\Bl}[0]{\mathbf{l}}
\newcommand{\Bm}[0]{\mathbf{m}}
\newcommand{\Bn}[0]{\mathbf{n}}
\newcommand{\Bo}[0]{\mathbf{o}}
\newcommand{\Bp}[0]{\mathbf{p}}
\newcommand{\Bq}[0]{\mathbf{q}}
\newcommand{\Br}[0]{\mathbf{r}}
\newcommand{\Bs}[0]{\mathbf{s}}
\newcommand{\Bt}[0]{\mathbf{t}}
\newcommand{\Bu}[0]{\mathbf{u}}
\newcommand{\Bv}[0]{\mathbf{v}}
\newcommand{\Bw}[0]{\mathbf{w}}
\newcommand{\Bx}[0]{\mathbf{x}}
\newcommand{\By}[0]{\mathbf{y}}
\newcommand{\Bz}[0]{\mathbf{z}}
\newcommand{\BA}[0]{\mathbf{A}}
\newcommand{\BB}[0]{\mathbf{B}}
\newcommand{\BC}[0]{\mathbf{C}}
\newcommand{\BD}[0]{\mathbf{D}}
\newcommand{\BE}[0]{\mathbf{E}}
\newcommand{\BF}[0]{\mathbf{F}}
\newcommand{\BG}[0]{\mathbf{G}}
\newcommand{\BH}[0]{\mathbf{H}}
\newcommand{\BI}[0]{\mathbf{I}}
\newcommand{\BJ}[0]{\mathbf{J}}
\newcommand{\BK}[0]{\mathbf{K}}
\newcommand{\BL}[0]{\mathbf{L}}
\newcommand{\BM}[0]{\mathbf{M}}
\newcommand{\BN}[0]{\mathbf{N}}
\newcommand{\BO}[0]{\mathbf{O}}
\newcommand{\BP}[0]{\mathbf{P}}
\newcommand{\BQ}[0]{\mathbf{Q}}
\newcommand{\BR}[0]{\mathbf{R}}
\newcommand{\BS}[0]{\mathbf{S}}
\newcommand{\BT}[0]{\mathbf{T}}
\newcommand{\BU}[0]{\mathbf{U}}
\newcommand{\BV}[0]{\mathbf{V}}
\newcommand{\BW}[0]{\mathbf{W}}
\newcommand{\BX}[0]{\mathbf{X}}
\newcommand{\BY}[0]{\mathbf{Y}}
\newcommand{\BZ}[0]{\mathbf{Z}}

\newcommand{\Bzero}[0]{\mathbf{0}}
\newcommand{\Btheta}[0]{\boldsymbol{\theta}}
\newcommand{\Btau}[0]{\boldsymbol{\tau}}
\newcommand{\Bomega}[0]{\boldsymbol{\omega}}

%
% shorthand for unit vectors
%
\newcommand{\acap}[0]{\hat{\Ba}}
\newcommand{\bcap}[0]{\hat{\Bb}}
\newcommand{\ccap}[0]{\hat{\Bc}}
\newcommand{\dcap}[0]{\hat{\Bd}}
\newcommand{\ecap}[0]{\hat{\Be}}
\newcommand{\fcap}[0]{\hat{\Bf}}
\newcommand{\gcap}[0]{\hat{\Bg}}
\newcommand{\hcap}[0]{\hat{\Bh}}
\newcommand{\icap}[0]{\hat{\Bi}}
\newcommand{\jcap}[0]{\hat{\Bj}}
\newcommand{\kcap}[0]{\hat{\Bk}}
\newcommand{\lcap}[0]{\hat{\Bl}}
\newcommand{\mcap}[0]{\hat{\Bm}}
\newcommand{\ncap}[0]{\hat{\Bn}}
\newcommand{\ocap}[0]{\hat{\Bo}}
\newcommand{\pcap}[0]{\hat{\Bp}}
\newcommand{\qcap}[0]{\hat{\Bq}}
\newcommand{\rcap}[0]{\hat{\Br}}
\newcommand{\scap}[0]{\hat{\Bs}}
\newcommand{\tcap}[0]{\hat{\Bt}}
\newcommand{\ucap}[0]{\hat{\Bu}}
\newcommand{\vcap}[0]{\hat{\Bv}}
\newcommand{\wcap}[0]{\hat{\Bw}}
\newcommand{\xcap}[0]{\hat{\Bx}}
\newcommand{\ycap}[0]{\hat{\By}}
\newcommand{\zcap}[0]{\hat{\Bz}}
\newcommand{\thetacap}[0]{\hat{\Btheta}}

%
% to write R^n and C^n in a distinguishable fashion.  Perhaps change this
% to the double lined characters upon figuring out how to do so.
%
\newcommand{\C}[1]{$\mathbb{C}^{#1}$}
\newcommand{\R}[1]{$\mathbb{R}^{#1}$}

%
% various generally useful helpers
%

% derivative of #1 wrt. #2:
\newcommand{\D}[2] {\frac {d#2} {d#1}}

\newcommand{\inv}[1]{\frac{1}{#1}}
\newcommand{\cross}[0]{\times}

\newcommand{\abs}[1]{\lvert{#1}\rvert}
\newcommand{\norm}[1]{\lVert{#1}\rVert}
\newcommand{\innerprod}[2]{\langle{#1}, {#2}\rangle}
\newcommand{\dotprod}[2]{{#1} \cdot {#2}}
\newcommand{\bdotprod}[2]{\left({#1} \cdot {#2}\right)}
\newcommand{\crossprod}[2]{{#1} \cross {#2}}
\newcommand{\tripleprod}[3]{\dotprod{\left(\crossprod{#1}{#2}\right)}{#3}}

\DeclareMathOperator{\Proj}{Proj}
\DeclareMathOperator{\Span}{span}
\DeclareMathOperator{\Sgn}{sgn}
\DeclareMathOperator{\Area}{Area}
\DeclareMathOperator{\Volume}{Volume}

%
% A few miscellaneous things specific to this document
%
\newcommand{\crossop}[1]{\crossprod{#1}{}}

% R2 vector.
\newcommand{\VectorTwo}[2]{
\begin{bmatrix}
 {#1} \\
 {#2}
\end{bmatrix}
}

\newcommand{\VectorN}[1]{
\begin{bmatrix}
{#1}_1 \\
{#1}_2 \\
\vdots \\
{#1}_N \\
\end{bmatrix}
}

\newcommand{\DETuvij}[4]{
\begin{vmatrix}
 {#1}_{#3} & {#1}_{#4} \\
 {#2}_{#3} & {#2}_{#4}
\end{vmatrix}
}

\newcommand{\DETuvwijk}[6]{
\begin{vmatrix}
 {#1}_{#4} & {#1}_{#5} & {#1}_{#6} \\
 {#2}_{#4} & {#2}_{#5} & {#2}_{#6} \\
 {#3}_{#4} & {#3}_{#5} & {#3}_{#6}
\end{vmatrix}
}

\newcommand{\DETuvwxijkl}[8]{
\begin{vmatrix}
 {#1}_{#5} & {#1}_{#6} & {#1}_{#7} & {#1}_{#8} \\
 {#2}_{#5} & {#2}_{#6} & {#2}_{#7} & {#2}_{#8} \\
 {#3}_{#5} & {#3}_{#6} & {#3}_{#7} & {#3}_{#8} \\
 {#4}_{#5} & {#4}_{#6} & {#4}_{#7} & {#4}_{#8} \\
\end{vmatrix}
}

%\newcommand{\DETuvwxyijklm}[10]{
%\begin{vmatrix}
% {#1}_{#6} & {#1}_{#7} & {#1}_{#8} & {#1}_{#9} & {#1}_{#10} \\
% {#2}_{#6} & {#2}_{#7} & {#2}_{#8} & {#2}_{#9} & {#2}_{#10} \\
% {#3}_{#6} & {#3}_{#7} & {#3}_{#8} & {#3}_{#9} & {#3}_{#10} \\
% {#4}_{#6} & {#4}_{#7} & {#4}_{#8} & {#4}_{#9} & {#4}_{#10} \\
% {#5}_{#6} & {#5}_{#7} & {#5}_{#8} & {#5}_{#9} & {#5}_{#10}
%\end{vmatrix}
%}

% R3 vector.
\newcommand{\VectorThree}[3]{
\begin{bmatrix}
 {#1} \\
 {#2} \\
 {#3}
\end{bmatrix}
}


\newcommand{\LL}[0]{\mathcal{L}}
\newcommand{\gpgrade}[2] {{\left\langle{{#1}}\right\rangle}_{#2}}
\newcommand{\gpgradezero}[1] {\gpgrade{#1}{0}}
\newcommand{\gpgradetwo}[1] {\gpgrade{#1}{2}}
\newcommand{\gpgradefour}[1] {\gpgrade{#1}{4}}
\newcommand{\grad}[0]{\nabla}
\newcommand{\spacegrad}[0]{\boldsymbol{\nabla}}

\title{ An attempt at evaluating the Lagrangian em field equations. }
\author{Peeter Joot}
\date{ Sept 8, 2008.  Last Revision: $Date: 2008/09/09 03:18:37 $ }

\begin{document}

\maketitle{}

\section{ Evaluating Lagragian equations. }

We have seen previously that the Lagrangian equations expressed in vector 
form can be written

\begin{equation}\label{eqn:lagrangian}
\grad \LL = \frac{d}{d\tau} \grad_v \LL
\end{equation}

There is a constant (at least sign) in the Lagrangian that I am not sure of
at the moment, so the exersize becomes to recover the field equation 
given a Lagrangian of the form

\begin{equation*}
\LL = \kappa \gpgradezero{ (\grad \wedge A)^2 } + J \cdot A
\end{equation*}

and determine the constant factor $\kappa$.  Splitting the Lagrangian temporarily as

\begin{equation*}
\LL = \LL_1 + \LL_2
\end{equation*}

the easy part of this evaluation is application of equation \ref{eqn:lagrangian} to $\LL_2 = J \cdot A$

\begin{equation}\label{eqn:maybewrong}
\grad \LL_2 = J (\grad \cdot A) = \frac{d}{d\tau} \grad_v \LL_2
\end{equation}

Therefore the expectation is that evaluation of the equations on $\LL_2$ will 
result in:

\begin{align*}
\grad \LL_1 
&= \kappa \grad \gpgradezero{ (\grad \wedge A)^2 } \\
&\propto (\grad (\grad \wedge A)) (\grad \cdot A)
\end{align*}

So, the some algebra to fill in the middle steps is required.  My
first attempt to do so has been unsuccessful, so perhaps I'll fall back
to components and give it a try that way first (or try that with $A = \phi \gamma_0$ perhaps as a simple case).

\section{ In coordinates. }

Is equation \ref{eqn:maybewrong} correct?  Let's try the complete Lagrangian calculation in coordinates instead (adding an extra factor of $1/2$
having tried some of this on paper).

\begin{equation*}
\LL = \frac{\kappa}{4} ( \partial_{\mu} A_{\nu} - \partial_{\nu} A_{\mu} ) ( \partial^{\mu} A^{\nu} - \partial^{\nu} A^{\mu} ) + J_{\alpha} A^{\alpha}
\end{equation*}

We have no $\dot{x}_{\mu}$ factors so the field equation is just the result of evaluating $\partial_{\mu} \LL = 0$, but before doing so, the $F^{\mu\nu}F_{\mu\nu}$ term can be simplified:

\begin{align*}
\inv{2} F^{\mu\nu}F_{\mu\nu}
&=
\inv{2} \left(
 \partial_{\mu} A_{\nu} \partial^{\mu} A^{\nu}
-\partial_{\mu} A_{\nu} \partial^{\nu} A^{\mu}
-\partial_{\nu} A_{\mu} \partial^{\mu} A^{\nu}
+\partial_{\nu} A_{\mu} \partial^{\nu} A^{\mu} \right) \\
&=
\partial_{\mu} A_{\nu} \partial^{\mu} A^{\nu}
-\partial_{\mu} A_{\nu} \partial^{\nu} A^{\mu} \\
&= \partial_{\mu} A_{\nu} \left( \partial^{\mu} A^{\nu} -\partial^{\nu} A^{\mu} \right) \\
\end{align*}

%This amounts to writing $F^{\mu\nu} = \inv{2}( F^{\mu\nu} - F^{\nu\mu} )$. NO.

So we want to evaluate the equations for:
\begin{equation*}
\LL = \frac{\kappa}{2} \partial_{\mu} A_{\nu} ( \partial^{\mu} A^{\nu} - \partial^{\nu} A^{\mu} ) + J_{\alpha} A^{\alpha}
\end{equation*}

\begin{align*}
\partial_{\beta} \LL 
&= 
\frac{\kappa}{2}
\partial_{\beta\mu} A_{\nu} ( \partial^{\mu} A^{\nu} - \partial^{\nu} A^{\mu} )
+ \frac{\kappa}{2}
\partial_{\mu} A_{\nu} ( \partial_{\beta}^{\mu} A^{\nu} - \partial_{\beta}^{\nu} A^{\mu} )
+ J_{\alpha} \partial_{\beta}A^{\alpha} \\
&= 
\frac{\kappa}{2} \partial_{\beta}^{\mu} A^{\nu} ( \partial_{\mu} A_{\nu} - \partial_{\nu} A_{\mu} )
+ \frac{\kappa}{2} \partial_{\mu} A_{\nu} ( \partial_{\beta}^{\mu} A^{\nu} - \partial_{\beta}^{\nu} A^{\mu} )
+ J_{\alpha} \partial_{\beta}A^{\alpha} \\
&= 
{\kappa} \partial_{\beta}^{\mu} A^{\nu} \partial_{\mu} A_{\nu}
- {\kappa} \partial_{\beta}^{\mu} A^{\nu} \partial_{\nu} A_{\mu}
+ J_{\alpha} \partial_{\beta}A^{\alpha} \\
\implies \\
-J_{\alpha} \partial_{\beta}A^{\alpha} 
&= {\kappa} \partial_{\beta}^{\mu} A^{\nu} \left( \partial_{\mu} A_{\nu} - \partial_{\nu} A_{\mu} \right) \\
&= {\kappa} (\partial^{\mu} \partial_{\beta}A^{\nu} ) F^{\mu\nu} \\
&= {\kappa} (\partial_{\mu} \partial_{\beta}A_{\nu} ) F_{\mu\nu} \\
\end{align*}

\subsection{ Field equations in tensor form. }

Hmm.  I don't see how to reduce the mess above.  Let's compare to the field equations in the form we know.

\begin{align*}
J/c \epsilon_0 
&= \grad (\grad \wedge A) \\
&= \grad \cdot ( \grad \wedge A ) \\
&= \gamma^{\alpha} \partial_{\alpha} \cdot ( \gamma^{\mu} \wedge \gamma_{\nu} \partial_{\mu} A^{\nu} ) \\
&= (\gamma^{\alpha} \cdot \gamma_{\mu\nu}) \partial_{\alpha} \partial^{\mu} A^{\nu} \\
&= (
%%\gamma^{\alpha} \cdot \gamma_{\mu}_{\nu}
\delta^{\alpha}_{\mu} \gamma_{\nu}
-\delta^{\alpha}_{\nu} \gamma_{\mu}
) \partial_{\alpha} \partial^{\mu} A^{\nu} \\
&= ( \gamma_{\nu} \partial_{\mu} - \gamma_{\mu} \partial_{\nu} ) \partial^{\mu} A^{\nu} \\
&= \gamma_{\nu} \partial_{\mu} (\partial^{\mu} A^{\nu} -\partial^{\nu} A^{\mu} ) \\
&= \gamma_{\nu} \partial_{\mu} F_{\mu\nu}
\end{align*}

Dotting the LHS with $\gamma^{\alpha}$ we have
\begin{align*}
\gamma^{\alpha} \cdot J/c \epsilon_0 
&= \gamma^{\alpha} \cdot \gamma_{\beta }J^{\beta}/c \epsilon_0 \\
&= \delta^{\alpha}_{\beta }J^{\beta}/c \epsilon_0 \\
&= J^{\alpha}/c \epsilon_0 \\
\end{align*}

and for the RHS 
\begin{align*}
\gamma^{\alpha} \cdot \gamma_{\nu} \partial_{\mu} F_{\mu\nu}
&= \partial_{\mu} F_{\mu\alpha} \\
\end{align*}

Or,
\begin{equation}
\partial_{\mu} F_{\mu\alpha} = J^{\alpha}/c \epsilon_0
\end{equation}

This is almost the tensor equation in wikipedia's Electromagnetic\_tensor
%http://en.wikipedia.org/wiki/Electromagnetic_tensor
article, especially since their $F^{\mu\nu}$ is my $F_{\mu\nu}$ (see em\_bivector\_metric\_dependencies.pdf),
but even with that adjustment, this is off by a factor of -1.

\end{document}               % End of document.
