\documentclass{article}

\usepackage{amsmath}
\usepackage{mathpazo}

%
% shorthand for bold symbols, convenient for vectors and matrices
%
\newcommand{\Ba}[0]{\mathbf{a}}
\newcommand{\Bb}[0]{\mathbf{b}}
\newcommand{\Bc}[0]{\mathbf{c}}
\newcommand{\Bd}[0]{\mathbf{d}}
\newcommand{\Be}[0]{\mathbf{e}}
\newcommand{\Bf}[0]{\mathbf{f}}
\newcommand{\Bg}[0]{\mathbf{g}}
\newcommand{\Bh}[0]{\mathbf{h}}
\newcommand{\Bi}[0]{\mathbf{i}}
\newcommand{\Bj}[0]{\mathbf{j}}
\newcommand{\Bk}[0]{\mathbf{k}}
\newcommand{\Bl}[0]{\mathbf{l}}
\newcommand{\Bm}[0]{\mathbf{m}}
\newcommand{\Bn}[0]{\mathbf{n}}
\newcommand{\Bo}[0]{\mathbf{o}}
\newcommand{\Bp}[0]{\mathbf{p}}
\newcommand{\Bq}[0]{\mathbf{q}}
\newcommand{\Br}[0]{\mathbf{r}}
\newcommand{\Bs}[0]{\mathbf{s}}
\newcommand{\Bt}[0]{\mathbf{t}}
\newcommand{\Bu}[0]{\mathbf{u}}
\newcommand{\Bv}[0]{\mathbf{v}}
\newcommand{\Bw}[0]{\mathbf{w}}
\newcommand{\Bx}[0]{\mathbf{x}}
\newcommand{\By}[0]{\mathbf{y}}
\newcommand{\Bz}[0]{\mathbf{z}}
\newcommand{\BA}[0]{\mathbf{A}}
\newcommand{\BB}[0]{\mathbf{B}}
\newcommand{\BC}[0]{\mathbf{C}}
\newcommand{\BD}[0]{\mathbf{D}}
\newcommand{\BE}[0]{\mathbf{E}}
\newcommand{\BF}[0]{\mathbf{F}}
\newcommand{\BG}[0]{\mathbf{G}}
\newcommand{\BH}[0]{\mathbf{H}}
\newcommand{\BI}[0]{\mathbf{I}}
\newcommand{\BJ}[0]{\mathbf{J}}
\newcommand{\BK}[0]{\mathbf{K}}
\newcommand{\BL}[0]{\mathbf{L}}
\newcommand{\BM}[0]{\mathbf{M}}
\newcommand{\BN}[0]{\mathbf{N}}
\newcommand{\BO}[0]{\mathbf{O}}
\newcommand{\BP}[0]{\mathbf{P}}
\newcommand{\BQ}[0]{\mathbf{Q}}
\newcommand{\BR}[0]{\mathbf{R}}
\newcommand{\BS}[0]{\mathbf{S}}
\newcommand{\BT}[0]{\mathbf{T}}
\newcommand{\BU}[0]{\mathbf{U}}
\newcommand{\BV}[0]{\mathbf{V}}
\newcommand{\BW}[0]{\mathbf{W}}
\newcommand{\BX}[0]{\mathbf{X}}
\newcommand{\BY}[0]{\mathbf{Y}}
\newcommand{\BZ}[0]{\mathbf{Z}}

\newcommand{\Bzero}[0]{\mathbf{0}}
\newcommand{\Btheta}[0]{\boldsymbol{\theta}}
\newcommand{\Btau}[0]{\boldsymbol{\tau}}
\newcommand{\Bomega}[0]{\boldsymbol{\omega}}

%
% shorthand for unit vectors
%
\newcommand{\acap}[0]{\hat{\Ba}}
\newcommand{\bcap}[0]{\hat{\Bb}}
\newcommand{\ccap}[0]{\hat{\Bc}}
\newcommand{\dcap}[0]{\hat{\Bd}}
\newcommand{\ecap}[0]{\hat{\Be}}
\newcommand{\fcap}[0]{\hat{\Bf}}
\newcommand{\gcap}[0]{\hat{\Bg}}
\newcommand{\hcap}[0]{\hat{\Bh}}
\newcommand{\icap}[0]{\hat{\Bi}}
\newcommand{\jcap}[0]{\hat{\Bj}}
\newcommand{\kcap}[0]{\hat{\Bk}}
\newcommand{\lcap}[0]{\hat{\Bl}}
\newcommand{\mcap}[0]{\hat{\Bm}}
\newcommand{\ncap}[0]{\hat{\Bn}}
\newcommand{\ocap}[0]{\hat{\Bo}}
\newcommand{\pcap}[0]{\hat{\Bp}}
\newcommand{\qcap}[0]{\hat{\Bq}}
\newcommand{\rcap}[0]{\hat{\Br}}
\newcommand{\scap}[0]{\hat{\Bs}}
\newcommand{\tcap}[0]{\hat{\Bt}}
\newcommand{\ucap}[0]{\hat{\Bu}}
\newcommand{\vcap}[0]{\hat{\Bv}}
\newcommand{\wcap}[0]{\hat{\Bw}}
\newcommand{\xcap}[0]{\hat{\Bx}}
\newcommand{\ycap}[0]{\hat{\By}}
\newcommand{\zcap}[0]{\hat{\Bz}}
\newcommand{\thetacap}[0]{\hat{\Btheta}}

%
% to write R^n and C^n in a distinguishable fashion.  Perhaps change this
% to the double lined characters upon figuring out how to do so.
%
\newcommand{\C}[1]{$\mathbb{C}^{#1}$}
\newcommand{\R}[1]{$\mathbb{R}^{#1}$}

%
% various generally useful helpers
%

% derivative of #1 wrt. #2:
\newcommand{\D}[2] {\frac {d#2} {d#1}}

\newcommand{\inv}[1]{\frac{1}{#1}}
\newcommand{\cross}[0]{\times}

\newcommand{\abs}[1]{\lvert{#1}\rvert}
\newcommand{\norm}[1]{\lVert{#1}\rVert}
\newcommand{\innerprod}[2]{\langle{#1}, {#2}\rangle}
\newcommand{\dotprod}[2]{{#1} \cdot {#2}}
\newcommand{\bdotprod}[2]{\left({#1} \cdot {#2}\right)}
\newcommand{\crossprod}[2]{{#1} \cross {#2}}
\newcommand{\tripleprod}[3]{\dotprod{\left(\crossprod{#1}{#2}\right)}{#3}}

\DeclareMathOperator{\Proj}{Proj}
\DeclareMathOperator{\Span}{span}
\DeclareMathOperator{\Sgn}{sgn}
\DeclareMathOperator{\Area}{Area}
\DeclareMathOperator{\Volume}{Volume}

%
% A few miscellaneous things specific to this document
%
\newcommand{\crossop}[1]{\crossprod{#1}{}}

% R2 vector.
\newcommand{\VectorTwo}[2]{
\begin{bmatrix}
 {#1} \\
 {#2}
\end{bmatrix}
}

\newcommand{\VectorN}[1]{
\begin{bmatrix}
{#1}_1 \\
{#1}_2 \\
\vdots \\
{#1}_N \\
\end{bmatrix}
}

\newcommand{\DETuvij}[4]{
\begin{vmatrix}
 {#1}_{#3} & {#1}_{#4} \\
 {#2}_{#3} & {#2}_{#4}
\end{vmatrix}
}

\newcommand{\DETuvwijk}[6]{
\begin{vmatrix}
 {#1}_{#4} & {#1}_{#5} & {#1}_{#6} \\
 {#2}_{#4} & {#2}_{#5} & {#2}_{#6} \\
 {#3}_{#4} & {#3}_{#5} & {#3}_{#6}
\end{vmatrix}
}

\newcommand{\DETuvwxijkl}[8]{
\begin{vmatrix}
 {#1}_{#5} & {#1}_{#6} & {#1}_{#7} & {#1}_{#8} \\
 {#2}_{#5} & {#2}_{#6} & {#2}_{#7} & {#2}_{#8} \\
 {#3}_{#5} & {#3}_{#6} & {#3}_{#7} & {#3}_{#8} \\
 {#4}_{#5} & {#4}_{#6} & {#4}_{#7} & {#4}_{#8} \\
\end{vmatrix}
}

%\newcommand{\DETuvwxyijklm}[10]{
%\begin{vmatrix}
% {#1}_{#6} & {#1}_{#7} & {#1}_{#8} & {#1}_{#9} & {#1}_{#10} \\
% {#2}_{#6} & {#2}_{#7} & {#2}_{#8} & {#2}_{#9} & {#2}_{#10} \\
% {#3}_{#6} & {#3}_{#7} & {#3}_{#8} & {#3}_{#9} & {#3}_{#10} \\
% {#4}_{#6} & {#4}_{#7} & {#4}_{#8} & {#4}_{#9} & {#4}_{#10} \\
% {#5}_{#6} & {#5}_{#7} & {#5}_{#8} & {#5}_{#9} & {#5}_{#10}
%\end{vmatrix}
%}

% R3 vector.
\newcommand{\VectorThree}[3]{
\begin{bmatrix}
 {#1} \\
 {#2} \\
 {#3}
\end{bmatrix}
}


\newcommand{\LL}[0]{\mathcal{L}}
\newcommand{\gpgrade}[2] {{\left\langle{{#1}}\right\rangle}_{#2}}
\newcommand{\gpgradezero}[1] {\gpgrade{#1}{0}}
\newcommand{\gpgradetwo}[1] {\gpgrade{#1}{2}}
\newcommand{\gpgradefour}[1] {\gpgrade{#1}{4}}
\newcommand{\grad}[0]{\nabla}
\newcommand{\spacegrad}[0]{\boldsymbol{\nabla}}
\newcommand{\PD}[2]{\frac{\partial {#2}}{\partial {#1}}}

\usepackage[
bookmarks=true
%,pdffitwindow
%,pdfcenterwindow
]{hyperref}

\title{ An attempt at evaluating the Lagrangian em field equations. }
\author{Peeter Joot}
\date{ Sept 8, 2008.  Last Revision: $Date: 2008/09/13 21:47:31 $ }

\begin{document}

\tableofcontents

\maketitle{}

\section{ Motivation. }

This document will attempt to calculate Maxwells equation, which in multivector form is

\begin{equation}
\grad F = J/\epsilon_0 c
\end{equation}

using a Lagrangian energy density variational approach.

\subsection{ Notation and definitions. }

For standalone purposes, here is a summary of the notation and definitions that will be used.  Greek letters range over all indexes and
english indexes range over $1,2,3$.  Bold vectors are spatial enties and non-bold is used for four vectors.

\begin{equation*}
\begin{array}{l l l}
\gamma_{\mu} & & \quad \mbox{Four vector basis vector)} \\
& & \quad \mbox{($\gamma_{\mu} \cdot \gamma_{\nu} = \pm {\delta^{\mu}}_{\nu}$)} \\
{(\gamma_0)}^2 {(\gamma_i)}^2 &= -1 & \quad \mbox{Minkowski metric} \\
\sigma_i = \sigma^i &= \gamma_{i} \wedge \gamma_0 & \quad \mbox{Spatial basis bivector. ($\sigma_i \cdot \sigma_j = \delta_{ij}$)} \\
                    &= \gamma_{i0} \\
I &= \gamma_{0} \wedge \gamma_1 \wedge \gamma_{2} \wedge \gamma_3 & \quad \mbox{Four-vector pseudoscalar} \\
  &= \gamma_{0123} \\
\gamma^{\mu} \cdot \gamma_{\nu} &= {\delta^{\mu}}_{\nu} & \quad \mbox{Reciprocal basis vectors} \\
x^{\mu} &= x \cdot \gamma^{\mu} & \quad \mbox{Vector coordinate} \\
x_{\mu} &= x \cdot \gamma_{\mu} & \quad \mbox{Coordinate for reciprocal basis} \\
x &= \sum \gamma_{\mu} x^{\mu} & \quad \mbox{Four vector in terms of coordinates} \\
  &= \sum \gamma^{\mu} x_{\mu} \\
\BE &= \sum E^i \sigma_i & \quad \mbox{Electric field spatial vector} \\
\BB &= \sum B^i \sigma_i & \quad \mbox{Magnetic field spatial vector} \\
J &= \sum \gamma_{\mu} J^{\mu} & \quad \mbox{Current density four vector.} \\
  &= \sum \gamma^{\mu} J_{\mu} \\
F &= \BE + I c \BB & \quad \mbox{Electromagnetic bivector} \\
x^{0} &= x \cdot \gamma^0 & \quad \mbox{Time coordinate (length dim.)} \\
      &= c t \\
\Bx &= x \wedge \gamma_0 & \quad \mbox{Spatial vector} \\
    &= x^i \sigma_i \\
J^{0} &= J \cdot \gamma^0 & \quad \mbox{Charge density.} \\
      &= c \rho & \quad \mbox{(current density dimensions.)} \\
\BJ &= J \wedge \gamma_0 & \quad \mbox{Current density spatial vector} \\
    &= \sum J^i \sigma_i \\
\grad &= \sum \gamma^{\mu} \PD{x^{\mu}}{} & \quad \mbox{Spacetime gradient} \\
      &= \sum \gamma^{\mu}\partial_{\mu} \\
      &= \sum \gamma_{\mu} \PD{x_{\mu}}{} \\
      &= \sum \gamma_{\mu}\partial^{\mu} \\
\spacegrad &= \sum \sigma^{i} \PD{x^{i}}{} & \quad \mbox{Spatial gradient} \\
           &= \sum \sigma^{i}\partial_{i} \\
\end{array}
\end{equation*}

Summation convention, where summation over all sets of matched upper and lower indexes is implied, will be in effect from this point on.

\subsection{ Tensor form of the field. } 

Explicit expansion of the field bivector in terms of coordinates one has

\begin{align*}
F 
&= \BE + I c \BB \\
&= E^k \gamma_{k0} + \gamma_{0123k0} c B^k \\
&= E^k \gamma_{k0} + {(\gamma_{0})}^2 {(\gamma_{k})}^2 {\epsilon^{ij}}_k c \gamma_{ij} B^k \\
\end{align*}

Or,
\begin{equation}
F = E^k \gamma_{k0} - c {\epsilon^{ij}}_k B^k \gamma_{ij}
\end{equation}

When this bivector is expressed in terms of basis bivectors $\gamma_{\mu\nu}$ we have

\begin{align*}
F 
= \sum_{\mu<\nu} (F \cdot \gamma^{\nu\mu}) \gamma_{\mu\nu}
= \inv{2} (F \cdot \gamma^{\nu\mu}) \gamma_{\mu\nu} 
\end{align*}

As shorthand for the coordinates the field can be expressed with respect to various bivector basis sets in tensor form

\begin{equation*}
\begin{array}{l l l}
F^{\mu\nu} &= F \cdot \gamma^{\nu\mu} & \quad F = (1/2) F^{\mu\nu} \gamma_{\mu\nu} \\
F_{\mu\nu} &= F \cdot \gamma_{\nu\mu} & \quad F = (1/2) F_{\mu\nu} \gamma^{\mu\nu} \\
{F_{\mu}}^\nu &= F \cdot {\gamma_{\nu}}^{\mu} & \quad F = (1/2) {F_{\mu}}^{\nu} {\gamma^{\mu}}_{\nu} \\
{F^{\mu}}_\nu &= F \cdot {\gamma^{\nu}}_{\mu} & \quad F = (1/2) {F^{\mu}}_{\nu} {\gamma_{\mu}}^{\nu}
\end{array}
\end{equation*}

In particular, we can extract the electric field components by dotting with a spacetime mix of indexes

\begin{equation*}
F^{i0} = E^k \gamma_{k0} \cdot \gamma^{0i} = E^i = -F_{i0}
\end{equation*}

and the magnetic field components by dotting with the bivectors having a pure spatial mix of indexes

\begin{equation*}
F^{ij} = - c {\epsilon^{a b}}_k B^k \gamma_{a b} \cdot \gamma^{ji} = - c {\epsilon^{i j}}_k B^k = F_{ij}
\end{equation*}

It is customary to summarize these tensors in matrix form
\begin{equation}\label{eqn:matrixtensor}
F^{\mu\nu} =
\begin{bmatrix}
0   & -E^1 & -E^2 & -E^3 \\
E^1 &   0  & -c B^3 &  c B^2 \\
E^2 &  c B^3 &   0  & -c B^1 \\
E^3 & -c B^2 &  c B^1 &   0  \\
\end{bmatrix}
\end{equation}

\begin{equation}
F_{\mu\nu} =
\begin{bmatrix}
0   & E^1 & E^2 & E^3 \\
-E^1 &   0  & -c B^3 &  c B^2 \\
-E^2 &  c B^3 &   0  & -c B^1 \\
-E^3 & -c B^2 &  c B^1 &   0  \\
\end{bmatrix}.
\end{equation}

We will not need either of these matrixes explcitly, but include them for comparison since there is some variation in the sign conventions and units used
for the field tensor.

\subsubsection{ Potential form. }

With the assumption that the field can be expressed in terms of the curl of a potential vector

\begin{equation}\label{eqn:potentialdef}
F = \grad \wedge A
\end{equation}

the tensor expression of the field becomes

\begin{align*}\label{eqn:tensorpot}
F^{\mu\nu} &= F \cdot (\gamma^{\nu} \wedge \gamma^{\mu}) = \partial^{\mu} A^{\nu} - \partial^{\nu} A^{\mu} \\
F_{\mu\nu} &= F \cdot (\gamma_{\nu} \wedge \gamma_{\mu}) = \partial_{\mu} A_{\nu} - \partial_{\nu} A_{\mu} \\
{F^{\mu}}_{\nu} &= F \cdot (\gamma^{\nu} \wedge \gamma_{\mu}) = \partial^{\mu} A_{\nu} - \partial_{\nu} A^{\mu} \\
{F_{\mu}}^{\nu} &= F \cdot (\gamma_{\nu} \wedge \gamma^{\mu}) = \partial_{\mu} A^{\nu} - \partial^{\nu} A_{\mu}
\end{align*}

These field bivector coordinates will be used in the Lagrangian calculations.

\subsection{ Field square. }

Our Lagrangian will be formed from the scalar part (will the pseudoscalar part of the field also play a part?) of the squared bivector

\begin{align*}
F^2 
&= (\BE + I c \BB) (\BE + I c \BB) \\
&= \BE^2 - c^2 \BB^2 + c \left( I \BB \BE + \BE I \BB \right) \\
&= \BE^2 - c^2 \BB^2 + c I \left( \BB \BE + \BE \BB \right) \\
&= \BE^2 - c^2 \BB^2 + 2 c I \BE \cdot \BB
\end{align*}

\subsubsection{ Scalar part. }

One can also show that the following are all identical representations.

\begin{equation}
\inv{2} F_{\mu\nu}F^{\mu\nu} = -\gpgradezero{F^2} = c^2 \BB^2 -\BE^2
\end{equation}

In particular, we will use the tensor form with the field defined in terms of the vector potential of equation \ref{eqn:potentialdef}.

\begin{align}
\LL &= -\frac{\kappa}{4} \gpgradezero{F^2} + J \cdot A \\
&= \frac{\kappa}{4} F_{\mu\nu} F^{\mu\nu} + J_{\alpha} A^{\alpha} \\
&= \frac{\kappa}{4} ( \partial_{\mu} A_{\nu} - \partial_{\nu} A_{\mu} ) ( \partial^{\mu} A^{\nu} - \partial^{\nu} A^{\mu} ) + J_{\alpha} A^{\alpha}
\end{align}

It will be convient to write the density term in a slightly different fashion

\begin{align*}
F^{\mu\nu} F_{\mu\nu}
&=
\partial_{\mu} A_{\nu} \partial^{\mu} A^{\nu}
-\partial_{\mu} A_{\nu} \partial^{\nu} A^{\mu}
-\partial_{\nu} A_{\mu} \partial^{\mu} A^{\nu}
+\partial_{\nu} A_{\mu} \partial^{\nu} A^{\mu} \\
&= 2 \left( \partial_{\mu} A_{\nu} \partial^{\mu} A^{\nu} -\partial_{\mu} A_{\nu} \partial^{\nu} A^{\mu} \right) \\
&= 2 \partial_{\mu} A_{\nu} \left( \partial^{\mu} A^{\nu} -\partial^{\nu} A^{\mu} \right) \\
\end{align*}

So we want to evaluate the equations for:
\begin{equation}\label{eqn:density}
\LL = \frac{\kappa}{2} \partial_{\mu} A_{\nu} ( \partial^{\mu} A^{\nu} - \partial^{\nu} A^{\mu} ) + J_{\alpha} A^{\alpha}
\end{equation}

\subsubsection{ Pseudoscalar part. }

FIXME: write the $\BE \cdot \BB$ term in tensor notation and see if the second half of Maxwells equation follows from that.

\section{ Variational background. }

Trying to blindly plug into the proper time variation of the Euler-Lagrange equations that can be used to derive the Lorentz force law from a $A \cdot v$ based Lagrangian

\begin{equation}\label{eqn:eulerlag}
\PD{x^\mu}{\LL} = \frac{d}{d\tau} \PD{\dot{x}^\mu}{\LL}
\end{equation}

did not really come close to producing Maxwell's equations from the Lagrangian in equation \ref{eqn:density}.  Whatever the equivalent of the Euler-Lagrange equations is for an energy density Lagrangian they aren't what is in equation \ref{eqn:eulerlag}.

However, what did work was Feynman's way from the second volume of the Lectures (the ``entertainment'' chapter on Principle of Least Action).
which uses some slightly ad-hoc seeming variational techniques directly.  To demonstrate the technique some simple examples
will be calculated to get the feel for the method.  After this we move on to the more complex case of trying with the electrodynamic Lagrange
density of equation \ref{eqn:density}.

\subsection{ One dimensional purely kinentic Lagrangian. }

Here is pretty much the simplest case, and illustrates the technique well.

Suppose we have an action associated with a kinetic Lagrangian density $(1/2) m v^2$

\begin{equation}\label{eqn:oneDimKinetic}
S = \int_a^b \frac{m}{2} { \left(\frac{dx}{dt}\right) }^2 dt
\end{equation}

where $x = x(t)$ is the undetermined function to solve for.  Feynman's technique is similar to Goldstein's way of deriving the Euler Lagrange equations, but instead of writing

\begin{equation*}
x(t, \epsilon) = x(t, 0) + \epsilon n(t)
\end{equation*}

and taking derivatives under the integral sign with respect to $\epsilon$, instead he just writes

\begin{equation}\label{eqn:xbarplusn}
x = \bar{x} + n
\end{equation}

In either case, the function $n = n(t)$ is zero at the boundaries of the integration region, and is allowed to take any value in between.

Substitution of \ref{eqn:xbarplusn} into \ref{eqn:oneDimKinetic} we have

\begin{equation}
S = 
\int_a^b \frac{m}{2} { \left(\frac{d\bar{x}}{dt}\right) }^2 dt
+ 2 \int_a^b \frac{m}{2} \frac{d\bar{x}}{dt} \frac{d n}{dt} dt
+ \int_a^b \frac{m}{2} { \left(\frac{d n}{dt}\right) }^2 dt
\end{equation}

The last term being quadratic and presumed small is just dropped.  The first term is strictly positive and doesn't vary with $n$ in any way.  The middle term, just as in 
Goldstein is integrated by parts

\begin{align*}
\int_a^b \underbrace{ \frac{d\bar{x}}{dt} \frac{d n}{dt} }_{fg'} dt
&= \underbrace{ \left. \frac{d\bar{x}}{dt} n \right\vert_a^b }_{fg} - \int_a^b \underbrace{\frac{d^2\bar{x}}{dt^2} n}_{f'g} dt.
\end{align*}

Since $n(a) = n(b) = 0$ the first term is zero.  For the remainder, for it to be independent of path (independent of $n$) is set to zero, leaving

\begin{equation*}
\frac{d^2\bar{x}}{dt^2} = 0
\end{equation*}

As the solution to the extreme value problem.  This is nothing but the equation for a straight line, which is what we expect if there are no external forces

\begin{equation*}
\bar{x} - x_0 = v(t - t_0)
\end{equation*}

\subsection{ Electrostatic potential in one dimension. }

Next is to apply the same idea to a field Lagrangian, but we will also start with the absolute simplest case.

%\subsection{ Electrostatic potential in two dimensions. }

% for the area here I think we want both type I and type II (elipse, square, ...).  Can probably generalize to
% other topological forms like donuts but will require that the variational function vanish on _any_ boundary.

\end{document}               % End of document.
