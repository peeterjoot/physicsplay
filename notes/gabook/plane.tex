\documentclass{article}      

\usepackage{amsmath}
\usepackage{mathpazo}

%
% shorthand for bold symbols, convenient for vectors and matrices
%
\newcommand{\Ba}[0]{\mathbf{a}}
\newcommand{\Bb}[0]{\mathbf{b}}
\newcommand{\Bc}[0]{\mathbf{c}}
\newcommand{\Bd}[0]{\mathbf{d}}
\newcommand{\Be}[0]{\mathbf{e}}
\newcommand{\Bf}[0]{\mathbf{f}}
\newcommand{\Bg}[0]{\mathbf{g}}
\newcommand{\Bh}[0]{\mathbf{h}}
\newcommand{\Bi}[0]{\mathbf{i}}
\newcommand{\Bj}[0]{\mathbf{j}}
\newcommand{\Bk}[0]{\mathbf{k}}
\newcommand{\Bl}[0]{\mathbf{l}}
\newcommand{\Bm}[0]{\mathbf{m}}
\newcommand{\Bn}[0]{\mathbf{n}}
\newcommand{\Bo}[0]{\mathbf{o}}
\newcommand{\Bp}[0]{\mathbf{p}}
\newcommand{\Bq}[0]{\mathbf{q}}
\newcommand{\Br}[0]{\mathbf{r}}
\newcommand{\Bs}[0]{\mathbf{s}}
\newcommand{\Bt}[0]{\mathbf{t}}
\newcommand{\Bu}[0]{\mathbf{u}}
\newcommand{\Bv}[0]{\mathbf{v}}
\newcommand{\Bw}[0]{\mathbf{w}}
\newcommand{\Bx}[0]{\mathbf{x}}
\newcommand{\By}[0]{\mathbf{y}}
\newcommand{\Bz}[0]{\mathbf{z}}
\newcommand{\BA}[0]{\mathbf{A}}
\newcommand{\BB}[0]{\mathbf{B}}
\newcommand{\BC}[0]{\mathbf{C}}
\newcommand{\BD}[0]{\mathbf{D}}
\newcommand{\BE}[0]{\mathbf{E}}
\newcommand{\BF}[0]{\mathbf{F}}
\newcommand{\BG}[0]{\mathbf{G}}
\newcommand{\BH}[0]{\mathbf{H}}
\newcommand{\BI}[0]{\mathbf{I}}
\newcommand{\BJ}[0]{\mathbf{J}}
\newcommand{\BK}[0]{\mathbf{K}}
\newcommand{\BL}[0]{\mathbf{L}}
\newcommand{\BM}[0]{\mathbf{M}}
\newcommand{\BN}[0]{\mathbf{N}}
\newcommand{\BO}[0]{\mathbf{O}}
\newcommand{\BP}[0]{\mathbf{P}}
\newcommand{\BQ}[0]{\mathbf{Q}}
\newcommand{\BR}[0]{\mathbf{R}}
\newcommand{\BS}[0]{\mathbf{S}}
\newcommand{\BT}[0]{\mathbf{T}}
\newcommand{\BU}[0]{\mathbf{U}}
\newcommand{\BV}[0]{\mathbf{V}}
\newcommand{\BW}[0]{\mathbf{W}}
\newcommand{\BX}[0]{\mathbf{X}}
\newcommand{\BY}[0]{\mathbf{Y}}
\newcommand{\BZ}[0]{\mathbf{Z}}

\newcommand{\Bzero}[0]{\mathbf{0}}
\newcommand{\Btheta}[0]{\boldsymbol{\theta}}
\newcommand{\Btau}[0]{\boldsymbol{\tau}}
\newcommand{\Bomega}[0]{\boldsymbol{\omega}}

%
% shorthand for unit vectors
%
\newcommand{\acap}[0]{\hat{\Ba}}
\newcommand{\bcap}[0]{\hat{\Bb}}
\newcommand{\ccap}[0]{\hat{\Bc}}
\newcommand{\dcap}[0]{\hat{\Bd}}
\newcommand{\ecap}[0]{\hat{\Be}}
\newcommand{\fcap}[0]{\hat{\Bf}}
\newcommand{\gcap}[0]{\hat{\Bg}}
\newcommand{\hcap}[0]{\hat{\Bh}}
\newcommand{\icap}[0]{\hat{\Bi}}
\newcommand{\jcap}[0]{\hat{\Bj}}
\newcommand{\kcap}[0]{\hat{\Bk}}
\newcommand{\lcap}[0]{\hat{\Bl}}
\newcommand{\mcap}[0]{\hat{\Bm}}
\newcommand{\ncap}[0]{\hat{\Bn}}
\newcommand{\ocap}[0]{\hat{\Bo}}
\newcommand{\pcap}[0]{\hat{\Bp}}
\newcommand{\qcap}[0]{\hat{\Bq}}
\newcommand{\rcap}[0]{\hat{\Br}}
\newcommand{\scap}[0]{\hat{\Bs}}
\newcommand{\tcap}[0]{\hat{\Bt}}
\newcommand{\ucap}[0]{\hat{\Bu}}
\newcommand{\vcap}[0]{\hat{\Bv}}
\newcommand{\wcap}[0]{\hat{\Bw}}
\newcommand{\xcap}[0]{\hat{\Bx}}
\newcommand{\ycap}[0]{\hat{\By}}
\newcommand{\zcap}[0]{\hat{\Bz}}
\newcommand{\thetacap}[0]{\hat{\Btheta}}

%
% to write R^n and C^n in a distinguishable fashion.  Perhaps change this
% to the double lined characters upon figuring out how to do so.
%
\newcommand{\C}[1]{$\mathbb{C}^{#1}$}
\newcommand{\R}[1]{$\mathbb{R}^{#1}$}

%
% various generally useful helpers
%

% derivative of #1 wrt. #2:
\newcommand{\D}[2] {\frac {d#2} {d#1}}

\newcommand{\inv}[1]{\frac{1}{#1}}
\newcommand{\cross}[0]{\times}

\newcommand{\abs}[1]{\lvert{#1}\rvert}
\newcommand{\norm}[1]{\lVert{#1}\rVert}
\newcommand{\innerprod}[2]{\langle{#1}, {#2}\rangle}
\newcommand{\dotprod}[2]{{#1} \cdot {#2}}
\newcommand{\bdotprod}[2]{\left({#1} \cdot {#2}\right)}
\newcommand{\crossprod}[2]{{#1} \cross {#2}}
\newcommand{\tripleprod}[3]{\dotprod{\left(\crossprod{#1}{#2}\right)}{#3}}

\DeclareMathOperator{\Proj}{Proj}
\DeclareMathOperator{\Span}{span}
\DeclareMathOperator{\Sgn}{sgn}
\DeclareMathOperator{\Area}{Area}
\DeclareMathOperator{\Volume}{Volume}

%
% A few miscellaneous things specific to this document
%
\newcommand{\crossop}[1]{\crossprod{#1}{}}

% R2 vector.
\newcommand{\VectorTwo}[2]{
\begin{bmatrix}
 {#1} \\
 {#2}
\end{bmatrix}
}

\newcommand{\VectorN}[1]{
\begin{bmatrix}
{#1}_1 \\
{#1}_2 \\
\vdots \\
{#1}_N \\
\end{bmatrix}
}

\newcommand{\DETuvij}[4]{
\begin{vmatrix}
 {#1}_{#3} & {#1}_{#4} \\
 {#2}_{#3} & {#2}_{#4}
\end{vmatrix}
}

\newcommand{\DETuvwijk}[6]{
\begin{vmatrix}
 {#1}_{#4} & {#1}_{#5} & {#1}_{#6} \\
 {#2}_{#4} & {#2}_{#5} & {#2}_{#6} \\
 {#3}_{#4} & {#3}_{#5} & {#3}_{#6}
\end{vmatrix}
}

\newcommand{\DETuvwxijkl}[8]{
\begin{vmatrix}
 {#1}_{#5} & {#1}_{#6} & {#1}_{#7} & {#1}_{#8} \\
 {#2}_{#5} & {#2}_{#6} & {#2}_{#7} & {#2}_{#8} \\
 {#3}_{#5} & {#3}_{#6} & {#3}_{#7} & {#3}_{#8} \\
 {#4}_{#5} & {#4}_{#6} & {#4}_{#7} & {#4}_{#8} \\
\end{vmatrix}
}

%\newcommand{\DETuvwxyijklm}[10]{
%\begin{vmatrix}
% {#1}_{#6} & {#1}_{#7} & {#1}_{#8} & {#1}_{#9} & {#1}_{#10} \\
% {#2}_{#6} & {#2}_{#7} & {#2}_{#8} & {#2}_{#9} & {#2}_{#10} \\
% {#3}_{#6} & {#3}_{#7} & {#3}_{#8} & {#3}_{#9} & {#3}_{#10} \\
% {#4}_{#6} & {#4}_{#7} & {#4}_{#8} & {#4}_{#9} & {#4}_{#10} \\
% {#5}_{#6} & {#5}_{#7} & {#5}_{#8} & {#5}_{#9} & {#5}_{#10}
%\end{vmatrix}
%}

% R3 vector.
\newcommand{\VectorThree}[3]{
\begin{bmatrix}
 {#1} \\
 {#2} \\
 {#3}
\end{bmatrix}
}



                             % The preamble begins here.
\title{More details on NFCM plane formulation} % Declares the document's title.
\author{Peeter Joot}         % Declares the author's name.
%\date{}        % Deleting this command produces today's date.

\begin{document}             % End of preamble and beginning of text.

%\maketitle{}

\section{Wedge product formula for a plane.}

The equation of the plane with bivector $\BU$ through point $\Ba$ is given
by

\[
(\Bx - \Ba) \wedge \BU = 0
\]

or

\[
\Bx \wedge \BU = \Ba \wedge \BU = \BT
\]

\subsection{ Examining this equation in more details. }

Without any loss of generality one can express this plane equation
in terms of a unit bivector $\Bi$

\[
\Bx \wedge \Bi = \Ba \wedge \Bi
\]

As with the line equation, to express this in the ``standard'' parametric
form, right multiplication with $1/\Bi$ is required.

\[
(\Bx \wedge \Bi)\frac{1}{\Bi} = (\Ba \wedge \Bi)\frac{1}{\Bi}
\]

We have a trivector bivector product here, which in general has a vector,
trivector, and 5-vector component.  Since $\Bi \wedge \Bi = 0$, the
5-vector component is zero:

\[
\Bx \wedge \Bi \wedge -\Bi = 0
\]

and intuition says that the trivector component will also be zero.  However,
as well as providing verification of this, expansion of this product will also
demonstrate how to find the projective and rejective components of a vector
with respect to a plane (ie: components in and out of the plane).

\subsection{Rejection from a plane product expansion.}

Here's an explicit expansion of the rejective term above

\begin{align*}
(\Bx \wedge \Bi)\frac{1}{\Bi} 
&= -(\Bx \wedge \Bi){\Bi} \\ 
&= -\frac{1}{2}(\Bx\Bi + \Bi\Bx){\Bi} \\ 
&= \frac{1}{2}(\Bx - \Bi\Bx\Bi) \\ 
&= \frac{1}{2}(\Bx - (\Bx \Bi + 2 \Bi \cdot \Bx)\Bi) \\ 
&= \Bx - (\Bi \cdot \Bx)\Bi \\ 
\end{align*}

In this last term the quantity $\Bi \cdot \Bx$ is a vector in the plane.
This can be demonstrated by writing $\Bi$ in terms of a pair of orthonormal
vectors $\Bi = \ucap\vcap = \ucap \wedge \vcap$.

\begin{align*}
\Bi \cdot \Bx &= (\ucap \wedge \vcap) \cdot \Bx \\
              &= \ucap (\vcap \cdot \Bx) - \vcap (\ucap \cdot \Bx) \\
\end{align*}

Thus, $(\Bi \cdot \Bx) \wedge \Bi = 0$, 
and $(\Bi \cdot \Bx) \Bi = (\Bi \cdot \Bx) \cdot \Bi$.  Inserting this above
we have the end result

\begin{align*}
(\Bx \wedge \Bi)\frac{1}{\Bi} 
&= \Bx - (\Bi \cdot \Bx) \cdot \Bi \\ 
&= \Ba - (\Bi \cdot \Ba) \cdot \Bi \\ 
\end{align*}

Or
\begin{align*}
\Bx  - \Ba 
&= (\Bi \cdot (\Bx - \Ba)) \cdot \Bi \\ 
\end{align*}

This is actually the standard parametric equation of a plane, but expressed
in terms of a unit bivector that describes the plane instead of in terms
of a pair of vectors in the plane.

To demonstrate this expansion of the right hand side is required

\begin{align*}
(\Bi \cdot \Bx) \cdot \Bi
&= (\ucap (\vcap \cdot \Bx) - \vcap (\ucap \cdot \Bx)) \ucap \vcap \\
&= \vcap (\vcap \cdot \Bx) + \ucap (\ucap \cdot \Bx) \\
\end{align*}

Substituting this back yields:

\begin{align*}
\Bx 
&= \Ba + \ucap (\ucap \cdot (\Bx - \Ba)) + \vcap (\vcap \cdot (\Bx - \Ba)) \\
&= \Ba + s \ucap + t \vcap
\end{align*}

In words this says that the plane is specified by a point in the plane,
and the span
of a pair of orthonormal vectors directed in that plane.

This (but perhaps without neccessariliy using orthornomal direction vectors)
is often how the plane is defined to start with.

It isn't neccessarily obvious that the bivector wedge product formula for
a plane that we started with:

\[
\Bx \wedge \BU = \Ba \wedge \BU
\]

can also be used to express this parametric representation.

\subsection{ Orthonormal decomposition of a vector with respect to a plane. }

With the expansion above we have a separation of a vector into two
components, and these can be demonstrated to be the components that are
directed entirely within and out of the plane.

Rearranging terms from above we have:

\begin{align*}
\Bx 
&= 
(\Bx \cdot \Bi) \cdot \frac{1}{\Bi} + (\Bx \wedge \Bi) \cdot \frac{1}{\Bi} \\
&= 
(\Bx \cdot \Bi) \frac{1}{\Bi} + (\Bx \wedge \Bi) \frac{1}{\Bi} \\
\end{align*}

% write x = x_perp + x_parallel to show that this is a ortho decomp.
% can then write formula for directrix of plane.

\subsection{ Alternate derivation of orthonormal planar decomposition }

This could alternately be derived by expanding the vector unit bivector
product directly

\begin{align*}
\Bx \Bi \frac{1}{\Bi} 
&= ( \Bx \cdot \Bi + \Bx \wedge \Bi ) \frac{1}{\Bi} \\
&= 
- {(\Bx \cdot \Bi) \cdot \Bi} - {(\Bx \cdot \Bi) \wedge \Bi} - {(\Bx \wedge \Bi) \Bi} \\
&= 
- {(\Bx \cdot \Bi) \cdot \Bi} - {(\Bx \wedge \Bi) \cdot \Bi } - {<(\Bx \wedge \Bi) \Bi>_3} - {(\Bx \wedge \Bi) \wedge \Bi} \\
&= 
{(\Bx \cdot \Bi) \cdot \frac{1}{\Bi}} + {(\Bx \wedge \Bi) \cdot \frac{1}{\Bi}} - {<(\Bx \wedge \Bi) \Bi>_3} \\
\end{align*}

Since the LHS of this equation is the vector $\Bx$, the right hand side must
also be a vector, which demonstrates that the term

\[
<(\Bx \wedge \Bi) \Bi>_3 = 0
\]

So, one has

\begin{align*}
\Bx 
&=
{(\Bx \cdot \Bi) \cdot \frac{1}{\Bi}} + {(\Bx \wedge \Bi) \cdot \frac{1}{\Bi}} \\
&=
{(\Bx \cdot \Bi) \frac{1}{\Bi}} + {(\Bx \wedge \Bi) \frac{1}{\Bi}} \\
\end{align*}


\end{document}
