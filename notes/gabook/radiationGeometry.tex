\documentclass[]{eliblog}

\usepackage{amsmath}
\usepackage{mathpazo}

%
% shorthand for bold symbols, convenient for vectors and matrices
%
\newcommand{\Ba}[0]{\mathbf{a}}
\newcommand{\Bb}[0]{\mathbf{b}}
\newcommand{\Bc}[0]{\mathbf{c}}
\newcommand{\Bd}[0]{\mathbf{d}}
\newcommand{\Be}[0]{\mathbf{e}}
\newcommand{\Bf}[0]{\mathbf{f}}
\newcommand{\Bg}[0]{\mathbf{g}}
\newcommand{\Bh}[0]{\mathbf{h}}
\newcommand{\Bi}[0]{\mathbf{i}}
\newcommand{\Bj}[0]{\mathbf{j}}
\newcommand{\Bk}[0]{\mathbf{k}}
\newcommand{\Bl}[0]{\mathbf{l}}
\newcommand{\Bm}[0]{\mathbf{m}}
\newcommand{\Bn}[0]{\mathbf{n}}
\newcommand{\Bo}[0]{\mathbf{o}}
\newcommand{\Bp}[0]{\mathbf{p}}
\newcommand{\Bq}[0]{\mathbf{q}}
\newcommand{\Br}[0]{\mathbf{r}}
\newcommand{\Bs}[0]{\mathbf{s}}
\newcommand{\Bt}[0]{\mathbf{t}}
\newcommand{\Bu}[0]{\mathbf{u}}
\newcommand{\Bv}[0]{\mathbf{v}}
\newcommand{\Bw}[0]{\mathbf{w}}
\newcommand{\Bx}[0]{\mathbf{x}}
\newcommand{\By}[0]{\mathbf{y}}
\newcommand{\Bz}[0]{\mathbf{z}}
\newcommand{\BA}[0]{\mathbf{A}}
\newcommand{\BB}[0]{\mathbf{B}}
\newcommand{\BC}[0]{\mathbf{C}}
\newcommand{\BD}[0]{\mathbf{D}}
\newcommand{\BE}[0]{\mathbf{E}}
\newcommand{\BF}[0]{\mathbf{F}}
\newcommand{\BG}[0]{\mathbf{G}}
\newcommand{\BH}[0]{\mathbf{H}}
\newcommand{\BI}[0]{\mathbf{I}}
\newcommand{\BJ}[0]{\mathbf{J}}
\newcommand{\BK}[0]{\mathbf{K}}
\newcommand{\BL}[0]{\mathbf{L}}
\newcommand{\BM}[0]{\mathbf{M}}
\newcommand{\BN}[0]{\mathbf{N}}
\newcommand{\BO}[0]{\mathbf{O}}
\newcommand{\BP}[0]{\mathbf{P}}
\newcommand{\BQ}[0]{\mathbf{Q}}
\newcommand{\BR}[0]{\mathbf{R}}
\newcommand{\BS}[0]{\mathbf{S}}
\newcommand{\BT}[0]{\mathbf{T}}
\newcommand{\BU}[0]{\mathbf{U}}
\newcommand{\BV}[0]{\mathbf{V}}
\newcommand{\BW}[0]{\mathbf{W}}
\newcommand{\BX}[0]{\mathbf{X}}
\newcommand{\BY}[0]{\mathbf{Y}}
\newcommand{\BZ}[0]{\mathbf{Z}}

\newcommand{\Bzero}[0]{\mathbf{0}}
\newcommand{\Btheta}[0]{\boldsymbol{\theta}}
\newcommand{\Btau}[0]{\boldsymbol{\tau}}
\newcommand{\Bomega}[0]{\boldsymbol{\omega}}

%
% shorthand for unit vectors
%
\newcommand{\acap}[0]{\hat{\Ba}}
\newcommand{\bcap}[0]{\hat{\Bb}}
\newcommand{\ccap}[0]{\hat{\Bc}}
\newcommand{\dcap}[0]{\hat{\Bd}}
\newcommand{\ecap}[0]{\hat{\Be}}
\newcommand{\fcap}[0]{\hat{\Bf}}
\newcommand{\gcap}[0]{\hat{\Bg}}
\newcommand{\hcap}[0]{\hat{\Bh}}
\newcommand{\icap}[0]{\hat{\Bi}}
\newcommand{\jcap}[0]{\hat{\Bj}}
\newcommand{\kcap}[0]{\hat{\Bk}}
\newcommand{\lcap}[0]{\hat{\Bl}}
\newcommand{\mcap}[0]{\hat{\Bm}}
\newcommand{\ncap}[0]{\hat{\Bn}}
\newcommand{\ocap}[0]{\hat{\Bo}}
\newcommand{\pcap}[0]{\hat{\Bp}}
\newcommand{\qcap}[0]{\hat{\Bq}}
\newcommand{\rcap}[0]{\hat{\Br}}
\newcommand{\scap}[0]{\hat{\Bs}}
\newcommand{\tcap}[0]{\hat{\Bt}}
\newcommand{\ucap}[0]{\hat{\Bu}}
\newcommand{\vcap}[0]{\hat{\Bv}}
\newcommand{\wcap}[0]{\hat{\Bw}}
\newcommand{\xcap}[0]{\hat{\Bx}}
\newcommand{\ycap}[0]{\hat{\By}}
\newcommand{\zcap}[0]{\hat{\Bz}}
\newcommand{\thetacap}[0]{\hat{\Btheta}}

%
% to write R^n and C^n in a distinguishable fashion.  Perhaps change this
% to the double lined characters upon figuring out how to do so.
%
\newcommand{\C}[1]{$\mathbb{C}^{#1}$}
\newcommand{\R}[1]{$\mathbb{R}^{#1}$}

%
% various generally useful helpers
%

% derivative of #1 wrt. #2:
\newcommand{\D}[2] {\frac {d#2} {d#1}}

\newcommand{\inv}[1]{\frac{1}{#1}}
\newcommand{\cross}[0]{\times}

\newcommand{\abs}[1]{\lvert{#1}\rvert}
\newcommand{\norm}[1]{\lVert{#1}\rVert}
\newcommand{\innerprod}[2]{\langle{#1}, {#2}\rangle}
\newcommand{\dotprod}[2]{{#1} \cdot {#2}}
\newcommand{\bdotprod}[2]{\left({#1} \cdot {#2}\right)}
\newcommand{\crossprod}[2]{{#1} \cross {#2}}
\newcommand{\tripleprod}[3]{\dotprod{\left(\crossprod{#1}{#2}\right)}{#3}}

\DeclareMathOperator{\Proj}{Proj}
\DeclareMathOperator{\Span}{span}
\DeclareMathOperator{\Sgn}{sgn}
\DeclareMathOperator{\Area}{Area}
\DeclareMathOperator{\Volume}{Volume}

%
% A few miscellaneous things specific to this document
%
\newcommand{\crossop}[1]{\crossprod{#1}{}}

% R2 vector.
\newcommand{\VectorTwo}[2]{
\begin{bmatrix}
 {#1} \\
 {#2}
\end{bmatrix}
}

\newcommand{\VectorN}[1]{
\begin{bmatrix}
{#1}_1 \\
{#1}_2 \\
\vdots \\
{#1}_N \\
\end{bmatrix}
}

\newcommand{\DETuvij}[4]{
\begin{vmatrix}
 {#1}_{#3} & {#1}_{#4} \\
 {#2}_{#3} & {#2}_{#4}
\end{vmatrix}
}

\newcommand{\DETuvwijk}[6]{
\begin{vmatrix}
 {#1}_{#4} & {#1}_{#5} & {#1}_{#6} \\
 {#2}_{#4} & {#2}_{#5} & {#2}_{#6} \\
 {#3}_{#4} & {#3}_{#5} & {#3}_{#6}
\end{vmatrix}
}

\newcommand{\DETuvwxijkl}[8]{
\begin{vmatrix}
 {#1}_{#5} & {#1}_{#6} & {#1}_{#7} & {#1}_{#8} \\
 {#2}_{#5} & {#2}_{#6} & {#2}_{#7} & {#2}_{#8} \\
 {#3}_{#5} & {#3}_{#6} & {#3}_{#7} & {#3}_{#8} \\
 {#4}_{#5} & {#4}_{#6} & {#4}_{#7} & {#4}_{#8} \\
\end{vmatrix}
}

%\newcommand{\DETuvwxyijklm}[10]{
%\begin{vmatrix}
% {#1}_{#6} & {#1}_{#7} & {#1}_{#8} & {#1}_{#9} & {#1}_{#10} \\
% {#2}_{#6} & {#2}_{#7} & {#2}_{#8} & {#2}_{#9} & {#2}_{#10} \\
% {#3}_{#6} & {#3}_{#7} & {#3}_{#8} & {#3}_{#9} & {#3}_{#10} \\
% {#4}_{#6} & {#4}_{#7} & {#4}_{#8} & {#4}_{#9} & {#4}_{#10} \\
% {#5}_{#6} & {#5}_{#7} & {#5}_{#8} & {#5}_{#9} & {#5}_{#10}
%\end{vmatrix}
%}

% R3 vector.
\newcommand{\VectorThree}[3]{
\begin{bmatrix}
 {#1} \\
 {#2} \\
 {#3}
\end{bmatrix}
}



\author{Peeter Joot}
\email{peeter.joot@gmail.com}


\chapter{Geometry of Maxwell radiation solutions}
\label{chap:radiationGeometry}
%\useCCL
\blogpage{http://sites.google.com/site/peeterjoot/math2009/radiationGeometry.pdf}
\date{Aug 14, 2009}
\revisionInfo{$RCSfile: radiationGeometry.tex,v $ Last $Revision: 1.2 $ $Date: 2009/08/15 04:16:35 $}

\beginArtWithToc
%\beginArtNoToc

\section{Motivation}

We have in GA multiple possible ways to parameterize an oscillatory time dependence for a radiation field.  
\section{Setup.  The eigenvalue problem.}

Again following Jackson (\cite{jackson1975cew}), we use CGS units.  Maxwell's equation in these units, with $F = \BE + I\BB/\sqrt{\mu\epsilon}$ is

\begin{align}\label{eqn:foo1}
0 &= (\spacegrad + \sqrt{\mu\epsilon} \partial_0) F 
\end{align}

With an assumed oscillatory time dependence 

\begin{align}\label{eqn:foo2}
F = \FF e^{-i\omega t}
\end{align}

Maxwell's equation reduces to a multivariable eigenvalue problem

\begin{align}\label{eqn:foo3}
\spacegrad \FF &= \lambda \FF \\
\lambda &= i\sqrt{\mu\epsilon} \frac{\omega}{c} 
\end{align}

We have some flexibility in picking the imaginary.  As well as a non-geometric imaginary $i$ typically used for a phasor representation where we take real parts of the field, we have additional possibilities, two of which are

\begin{align}\label{eqn:foo4}
i &= \xcap\ycap\zcap = I \\
i &= \xcap \ycap = I \zcap
\end{align}

The first is the spatial pseudoscalar, which commutes with all vectors and bivectors.  The second is the unit bivector for the transverse plane, here parameterized by duality using the perpendicular to the plane direction $\zcap$.

Let's examine the geometry required of the object $\FF$ for each of these two geometric modelling choices.

\section{Using the transverse plane bivector for the imaginary.}

Assuming no prior assumptions about $\FF$ let's allow for the possibility of scalar, vector, bivector and pseudoscalar components 

\begin{align}\label{eqn:foo5}
F = e^{-I\zcap \omega t} ( F_0 + F_1 + F_2 + F_3 )
\end{align}

Writing $e^{-I\zcap \omega t} = \cos(\omega t) -I \zcap \sin(\omega t) = C_\omega -I \zcap S_\omega$, an expansion of this product separated into grades is

\begin{align*}
F &= 
  C_\omega F_0 - I S_\omega (\zcap \wedge F_2) \\
&+ C_\omega F_1 - \zcap S_\omega (I F_3) + S_\omega (\zcap \cross F_1)  \\
&+ C_\omega F_2 - I \zcap S_\omega F_0 - I S_\omega (\zcap \cdot F_2) \\
&+ C_\omega F_3 - I S_\omega (\zcap \cdot F_1)
\end{align*}

By construction $F$ has only vector and bivector grades, so a requirement for zero scalar and pseudoscalar for all $t$ means that we have four immediate constraints

\begin{align*}
\begin{array}{l l l}
F_0 &= 0 & \\
F_3 &= 0 & \\
F_2 &= \zcap \wedge \Bm \\
F_1 &= \Bn & \quad \mbox{Where $\Bn \perp \zcap$.} \\
\end{array}
\end{align*}

Since we have the flexiblity to add or subtract any scalar multiple of $\zcap$ to $\Bm$ we can write $F_2 = \zcap \Bm$ where $\Bm \perp \zcap$.  Our field can now be written as just

\begin{align*}
F &= 
 C_\omega \Bn - I S_\omega (\zcap \wedge \Bn)  \\
&+ C_\omega \zcap \Bm - I S_\omega (\zcap \cdot (\zcap \Bm)) \\
\end{align*}

We can similarily require $\Bn \perp \zcap$, leaving

\begin{align}\label{eqn:foo7}
F &= (C_\omega - I \zcap S_\omega ) \Bn  + (C_\omega - I \zcap S_\omega) \Bm \zcap
\end{align}

So, just the geometrical constraints give us

\begin{align}\label{eqn:foo6}
F &= e^{-I\zcap \omega t}(\Bn + \Bm \zcap)
\end{align}

The first thing to be noted is that this phasor representation utilizing for the imaginary the transverse plane bivector $I\zcap$ cannot be the most general.  This representation allows for only transverse fields!  This can be seen two ways.  Computing the transverse and propagation field components we have

\begin{align*}
F_z 
&= \inv{2}(F + \zcap F \zcap) \\
&= 
\inv{2} e^{-I\zcap \omega t}( \Bn + \Bm \zcap + \zcap \Bn \zcap + \zcap \Bm \zcap \zcap) \\
&= 
\inv{2} e^{-I\zcap \omega t}( \Bn + \Bm \zcap - \Bn - \Bm \zcap ) \\
&= 0
\end{align*}

The computation for the transverse field $F_t = (F - \zcap F \zcap)/2$ shows that $F = F_t$ as expected since the propagation component is zero.

Another way to observe this is from the split of $F$ into electric and magnetic field components.  From (\ref{eqn:foo7}) we have

\begin{align}\label{eqn:foo8}
\BE &= \cos(\omega t) \Bm + \sin(\omega t) (\zcap \cross \Bm) \\
\BB &= \cos(\omega t) (\zcap \cross \Bn) - \sin(\omega t) \Bn
\end{align}

The space containing each of the $\BE$ and $\BB$ vectors lies in the span of the transverse plane.  We also see that there's some potential redundancy in the respresentation visible here since we have four vectors describing this span $\Bm$, $\Bn$, $\zcap \cross \Bm$, and $\zcap \cross \Bn$, instead of just two.

While this may not lead to the most general solution to the radiation problem, the transverse only propagation problem is still one of interest.  Let's see where this leads.  In order to reduce the scope of the problem by one degree of freedom, let's split out the $\zcap$ component of the gradient, writing

\begin{align}\label{eqn:foo9}
\spacegrad = \spacegrad_t + \zcap \partial_z
\end{align}

Also introduce a product split for separation of variables for the $z$ dependence.  That is

\begin{align}\label{eqn:foo10}
\FF = G(x,y) Z(z)
\end{align}

Again we are faced with the problem of too many choices for the grades of each of these factors.  We can pick one of these, say $Z$, to have only scalar and pseudoscalar grades so that the two factors commute.  Then we have

\begin{align*}
(\spacegrad_t + \spacegrad_z) \FF = (\spacegrad_t G) Z + \zcap G \partial_z Z = I \zcap \lambda G Z
\end{align*}

With $Z$ in an algebra isomorphic to the complex numbers, it is neccessarily invertable (and commutes with it's derivative).  So with some rearranging we have

\begin{align*}
-\zcap (\spacegrad_t G) + I \lambda G = G (\partial_z Z)\inv{Z} 
\end{align*}

Similar arguments to the grade fixing for $\FF$ show that $G$ has only vector and bivector grades, but does $G$ have the inverse required to do the separation of variables?  Let's blindly suppose that we can do this, which gives us

\begin{align}\label{eqn:foo11}
-\zcap (\spacegrad_t G) \inv{G} + I \lambda  = (\partial_z Z)\inv{Z} = \text{constant}
\end{align}

In order to commute these factors we've only required that $Z$ have only scalar and pseudoscalar grades, so for the constant let's pick an arbitrary element in this subspace.  That is

\begin{align}\label{eqn:foo12}
(\partial_z Z)\inv{Z} = \alpha + k I
\end{align}

The solution for the $Z$ factor in the separation of variables is thus

\begin{align}\label{eqn:foo13}
Z \propto e^{(\alpha + k I)z}
\end{align}

The remainder of the separation of variables gives us

\begin{align}\label{eqn:foo14}
\spacegrad_t G = \zcap (I (\lambda -k) -\alpha) G
\end{align}

We've now reduced the problem to solve to a two variable eigenvalue problem, where the differential operator to find eigenvectors for is the transverse gradient $\spacegrad_t$.

\EndArticle
%\EndNoBibArticle
