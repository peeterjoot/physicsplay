\chapter{Quaternions.} % Declares the document's title.

Like complex numbers, quaternions may be written as a multivector with scalar and bivector components (a 0,2-multivector).

\[
q = \alpha + \mathbf{B}
\]

Where the complex number has one bivector component, and the quaternions have three.

One can describe quaternions as 0,2-multivectors where the basis for the bivector part is left handed.  There isn't really anything special about quaternion multiplication, or complex number multiplication, for that matter.  Both are just a specific examples of a 0,2-multivector multiplication.  Other quaternion operations can also be found to have natural multivector equivalents.  The most important of which is likely the quaternion conjugate, since it implies the norm and the inverse.  As a multivector, like complex numbers, the conjugate operation is reversal:

\[
\overline{q} = q^\dagger = \alpha - \mathbf{B}
\]

Thus $\abs{q}^2 = q\overline{q} = \alpha^2 - \mathbf{B}^2$.  Note that this norm is a positive definite as expected since a bivector square is negative.

To be more specific about the left handed basis property of quaternions one can note that the quaternion bivector basis is usually defined in terms of the following properties

\[
\mathbf{i}^2 = \mathbf{j}^2 = \mathbf{k}^2 = -1
\]
\[
\mathbf{i}\mathbf{j} = -\mathbf{j}\mathbf{i}, \mathbf{i}\mathbf{k} = -\mathbf{k}\mathbf{i}, \mathbf{j}\mathbf{k} = -\mathbf{k}\mathbf{j}
\]
\[
\mathbf{i}\mathbf{j} = \mathbf{k}
\]

The first two properties are satisified by any set of orthogonal unit bivectors for the space.  The last property, which could also be written $\mathbf{i}\mathbf{j}\mathbf{k} = -1$, amounts to a choice for the orientation of this bivector basis of the 2-vector part of the quaternion.

As an example suppose one picks

\[
\mathbf{i} = \mathbf{e}_2\mathbf{e}_3
\]
\[
\mathbf{j} = \mathbf{e}_3\mathbf{e}_1
\]

Then the third bivector required to complete the basis set subject to the properties above is

\[
\mathbf{i}\mathbf{j} = \mathbf{e}_2\mathbf{e}_1 = \mathbf{k}
\].

Suppose that, instead of the above, one picked a slightly more natural bivector basis, the duals of the unit vectors obtained by multiplication with the pseudoscalar ($\mathbf{e}_1\mathbf{e}_2\mathbf{e}_3\mathbf{e}_i$).  These bivectors are

\[
\mathbf{i}=\mathbf{e}_2\mathbf{e}_3, \mathbf{j}=\mathbf{e}_3\mathbf{e}_1, \mathbf{k}=\mathbf{e}_1\mathbf{e}_2
\].

A 0,2-multivector with this as the basis for the bivector part would have properties similar to the standard quaternions (anti-commutative unit quaternions, negation for unit quaternion square, same congugate, norm and inversion operations, ...), however the triple product would have the value $\mathbf{i}\mathbf{j}\mathbf{k} = 1$, instead of $-1$.

\section{quaternion as generator of dot and cross product. }

The product of pure quaternions is noted as being a generator of dot and cross products.  This is also true
of a vector bivector product.

Writing a vector $\Bx$ as

\[
\Bx = \sum_i x_i \Be_i = x_1 \Be_1 + x_2 \Be_2 + x_3 \Be_3
\]

And a bivector $\BB$ (where for short, $\Be_{ij} = \Be_i \Be_j = \Be_i \wedge \Be_j$) as:

\[
\BB = \sum_i b_i \Be_i I = b_1 \Be_{23} + b_2 \Be_{31} + b_3 \Be_{12}
\]

The product of these two is
\begin{align*}
\Bx \BB 
&= (x_1 \Be_1 + x_2 \Be_2 + x_3 \Be_3)(b_1 \Be_{23} + b_2 \Be_{31} + b_3 \Be_{12}) \\
&= (x_3 b_2 - x_2 b_3) \Be_1 + (x_1 b_3 - x_3 b_1) \Be_2 + (x_2 b_1 - x_1 b_2) \Be_3 \\
&+ (x_1 b_1 + x_2 b_2 + x_3 b_3) \Be_{123} \\
\end{align*}

Looking at the vector and trivector components of this we recognize the dot product and negated cross product
immediately (as with multiplication of pure quaternions).

Those products are, in fact, $\Bx \cdot \BB$ and $\Bx \wedge \BB$ respectively.

Introducing a vector and bivector basis $\alpha = \{ \Be_i \}$, and $\beta = \{ \Be_i I \}$, we can
express the dot product and cross product of the associated coordinate vectors
in terms of vector bivectors products as follows:

\[
[\Bx]_\alpha \cdot [\BB]_\beta = \frac{\BB \wedge \Bx}{I}
\]
\[
[\Bx]_\alpha \cross [\BB]_\beta = [\BB \cdot \Bx]_\alpha
\]

