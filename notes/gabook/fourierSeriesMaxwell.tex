\documentclass{article}

\usepackage{amsmath}
\usepackage{mathpazo}

%
% shorthand for bold symbols, convenient for vectors and matrices
%
\newcommand{\Ba}[0]{\mathbf{a}}
\newcommand{\Bb}[0]{\mathbf{b}}
\newcommand{\Bc}[0]{\mathbf{c}}
\newcommand{\Bd}[0]{\mathbf{d}}
\newcommand{\Be}[0]{\mathbf{e}}
\newcommand{\Bf}[0]{\mathbf{f}}
\newcommand{\Bg}[0]{\mathbf{g}}
\newcommand{\Bh}[0]{\mathbf{h}}
\newcommand{\Bi}[0]{\mathbf{i}}
\newcommand{\Bj}[0]{\mathbf{j}}
\newcommand{\Bk}[0]{\mathbf{k}}
\newcommand{\Bl}[0]{\mathbf{l}}
\newcommand{\Bm}[0]{\mathbf{m}}
\newcommand{\Bn}[0]{\mathbf{n}}
\newcommand{\Bo}[0]{\mathbf{o}}
\newcommand{\Bp}[0]{\mathbf{p}}
\newcommand{\Bq}[0]{\mathbf{q}}
\newcommand{\Br}[0]{\mathbf{r}}
\newcommand{\Bs}[0]{\mathbf{s}}
\newcommand{\Bt}[0]{\mathbf{t}}
\newcommand{\Bu}[0]{\mathbf{u}}
\newcommand{\Bv}[0]{\mathbf{v}}
\newcommand{\Bw}[0]{\mathbf{w}}
\newcommand{\Bx}[0]{\mathbf{x}}
\newcommand{\By}[0]{\mathbf{y}}
\newcommand{\Bz}[0]{\mathbf{z}}
\newcommand{\BA}[0]{\mathbf{A}}
\newcommand{\BB}[0]{\mathbf{B}}
\newcommand{\BC}[0]{\mathbf{C}}
\newcommand{\BD}[0]{\mathbf{D}}
\newcommand{\BE}[0]{\mathbf{E}}
\newcommand{\BF}[0]{\mathbf{F}}
\newcommand{\BG}[0]{\mathbf{G}}
\newcommand{\BH}[0]{\mathbf{H}}
\newcommand{\BI}[0]{\mathbf{I}}
\newcommand{\BJ}[0]{\mathbf{J}}
\newcommand{\BK}[0]{\mathbf{K}}
\newcommand{\BL}[0]{\mathbf{L}}
\newcommand{\BM}[0]{\mathbf{M}}
\newcommand{\BN}[0]{\mathbf{N}}
\newcommand{\BO}[0]{\mathbf{O}}
\newcommand{\BP}[0]{\mathbf{P}}
\newcommand{\BQ}[0]{\mathbf{Q}}
\newcommand{\BR}[0]{\mathbf{R}}
\newcommand{\BS}[0]{\mathbf{S}}
\newcommand{\BT}[0]{\mathbf{T}}
\newcommand{\BU}[0]{\mathbf{U}}
\newcommand{\BV}[0]{\mathbf{V}}
\newcommand{\BW}[0]{\mathbf{W}}
\newcommand{\BX}[0]{\mathbf{X}}
\newcommand{\BY}[0]{\mathbf{Y}}
\newcommand{\BZ}[0]{\mathbf{Z}}

\newcommand{\Bzero}[0]{\mathbf{0}}
\newcommand{\Btheta}[0]{\boldsymbol{\theta}}
\newcommand{\Btau}[0]{\boldsymbol{\tau}}
\newcommand{\Bomega}[0]{\boldsymbol{\omega}}

%
% shorthand for unit vectors
%
\newcommand{\acap}[0]{\hat{\Ba}}
\newcommand{\bcap}[0]{\hat{\Bb}}
\newcommand{\ccap}[0]{\hat{\Bc}}
\newcommand{\dcap}[0]{\hat{\Bd}}
\newcommand{\ecap}[0]{\hat{\Be}}
\newcommand{\fcap}[0]{\hat{\Bf}}
\newcommand{\gcap}[0]{\hat{\Bg}}
\newcommand{\hcap}[0]{\hat{\Bh}}
\newcommand{\icap}[0]{\hat{\Bi}}
\newcommand{\jcap}[0]{\hat{\Bj}}
\newcommand{\kcap}[0]{\hat{\Bk}}
\newcommand{\lcap}[0]{\hat{\Bl}}
\newcommand{\mcap}[0]{\hat{\Bm}}
\newcommand{\ncap}[0]{\hat{\Bn}}
\newcommand{\ocap}[0]{\hat{\Bo}}
\newcommand{\pcap}[0]{\hat{\Bp}}
\newcommand{\qcap}[0]{\hat{\Bq}}
\newcommand{\rcap}[0]{\hat{\Br}}
\newcommand{\scap}[0]{\hat{\Bs}}
\newcommand{\tcap}[0]{\hat{\Bt}}
\newcommand{\ucap}[0]{\hat{\Bu}}
\newcommand{\vcap}[0]{\hat{\Bv}}
\newcommand{\wcap}[0]{\hat{\Bw}}
\newcommand{\xcap}[0]{\hat{\Bx}}
\newcommand{\ycap}[0]{\hat{\By}}
\newcommand{\zcap}[0]{\hat{\Bz}}
\newcommand{\thetacap}[0]{\hat{\Btheta}}

%
% to write R^n and C^n in a distinguishable fashion.  Perhaps change this
% to the double lined characters upon figuring out how to do so.
%
\newcommand{\C}[1]{$\mathbb{C}^{#1}$}
\newcommand{\R}[1]{$\mathbb{R}^{#1}$}

%
% various generally useful helpers
%

% derivative of #1 wrt. #2:
\newcommand{\D}[2] {\frac {d#2} {d#1}}

\newcommand{\inv}[1]{\frac{1}{#1}}
\newcommand{\cross}[0]{\times}

\newcommand{\abs}[1]{\lvert{#1}\rvert}
\newcommand{\norm}[1]{\lVert{#1}\rVert}
\newcommand{\innerprod}[2]{\langle{#1}, {#2}\rangle}
\newcommand{\dotprod}[2]{{#1} \cdot {#2}}
\newcommand{\bdotprod}[2]{\left({#1} \cdot {#2}\right)}
\newcommand{\crossprod}[2]{{#1} \cross {#2}}
\newcommand{\tripleprod}[3]{\dotprod{\left(\crossprod{#1}{#2}\right)}{#3}}

\DeclareMathOperator{\Proj}{Proj}
\DeclareMathOperator{\Span}{span}
\DeclareMathOperator{\Sgn}{sgn}
\DeclareMathOperator{\Area}{Area}
\DeclareMathOperator{\Volume}{Volume}

%
% A few miscellaneous things specific to this document
%
\newcommand{\crossop}[1]{\crossprod{#1}{}}

% R2 vector.
\newcommand{\VectorTwo}[2]{
\begin{bmatrix}
 {#1} \\
 {#2}
\end{bmatrix}
}

\newcommand{\VectorN}[1]{
\begin{bmatrix}
{#1}_1 \\
{#1}_2 \\
\vdots \\
{#1}_N \\
\end{bmatrix}
}

\newcommand{\DETuvij}[4]{
\begin{vmatrix}
 {#1}_{#3} & {#1}_{#4} \\
 {#2}_{#3} & {#2}_{#4}
\end{vmatrix}
}

\newcommand{\DETuvwijk}[6]{
\begin{vmatrix}
 {#1}_{#4} & {#1}_{#5} & {#1}_{#6} \\
 {#2}_{#4} & {#2}_{#5} & {#2}_{#6} \\
 {#3}_{#4} & {#3}_{#5} & {#3}_{#6}
\end{vmatrix}
}

\newcommand{\DETuvwxijkl}[8]{
\begin{vmatrix}
 {#1}_{#5} & {#1}_{#6} & {#1}_{#7} & {#1}_{#8} \\
 {#2}_{#5} & {#2}_{#6} & {#2}_{#7} & {#2}_{#8} \\
 {#3}_{#5} & {#3}_{#6} & {#3}_{#7} & {#3}_{#8} \\
 {#4}_{#5} & {#4}_{#6} & {#4}_{#7} & {#4}_{#8} \\
\end{vmatrix}
}

%\newcommand{\DETuvwxyijklm}[10]{
%\begin{vmatrix}
% {#1}_{#6} & {#1}_{#7} & {#1}_{#8} & {#1}_{#9} & {#1}_{#10} \\
% {#2}_{#6} & {#2}_{#7} & {#2}_{#8} & {#2}_{#9} & {#2}_{#10} \\
% {#3}_{#6} & {#3}_{#7} & {#3}_{#8} & {#3}_{#9} & {#3}_{#10} \\
% {#4}_{#6} & {#4}_{#7} & {#4}_{#8} & {#4}_{#9} & {#4}_{#10} \\
% {#5}_{#6} & {#5}_{#7} & {#5}_{#8} & {#5}_{#9} & {#5}_{#10}
%\end{vmatrix}
%}

% R3 vector.
\newcommand{\VectorThree}[3]{
\begin{bmatrix}
 {#1} \\
 {#2} \\
 {#3}
\end{bmatrix}
}


%<misc>
%
\newcommand{\Abs}[1]{{\left\lvert{#1}\right\rvert}}
\newcommand{\spacegrad}[0]{\boldsymbol{\nabla}}
\newcommand{\grad}[0]{\nabla}
\newcommand{\LL}[0]{\mathcal{L}}

% == \partial_{#1} {#2}
\newcommand{\PD}[2]{\frac{\partial {#2}}{\partial {#1}}}
% inline variant
\newcommand{\PDi}[2]{{\partial {#2}}/{\partial {#1}}}

\newcommand{\PDD}[3]{\frac{\partial^2 {#3}}{\partial {#1}\partial {#2}}}
%\newcommand{\PDd}[2]{\frac{\partial^2 {#2}}{{\partial{#1}}^2}}
\newcommand{\PDsq}[2]{\frac{\partial^2 {#2}}{(\partial {#1})^2}}

\newcommand{\Partial}[2]{\frac{\partial {#1}}{\partial {#2}}}
\DeclareMathOperator{\RejName}{Rej}
\newcommand{\Rej}[2]{\RejName_{#1}\left( {#2} \right)}
\newcommand{\Rm}[1]{\mathbb{R}^{#1}}
\newcommand{\Cm}[1]{\mathbb{C}^{#1}}
\newcommand{\conj}[0]{{*}}

%</misc>

% <grade selection>
%
\newcommand{\gpgrade}[2] {{\left\langle{{#1}}\right\rangle}_{#2}}

\newcommand{\gpgradezero}[1] {\gpgrade{#1}{}}
%\newcommand{\gpscalargrade}[1] {{\left\langle{{#1}}\right\rangle}}
%\newcommand{\gpgradezero}[1] {\gpgrade{#1}{0}}

%\newcommand{\gpgradeone}[1] {{\left\langle{{#1}}\right\rangle}_{1}}
\newcommand{\gpgradeone}[1] {\gpgrade{#1}{1}}

\newcommand{\gpgradetwo}[1] {\gpgrade{#1}{2}}
\newcommand{\gpgradethree}[1] {\gpgrade{#1}{3}}
\newcommand{\gpgradefour}[1] {\gpgrade{#1}{4}}
%
% </grade selection>



\newcommand{\adot}[0]{{\dot{a}}}
\newcommand{\bdot}[0]{{\dot{b}}}
% taken for centered dot:
%\newcommand{\cdot}[0]{{\dot{c}}}
%\newcommand{\ddot}[0]{{\dot{d}}}
\newcommand{\edot}[0]{{\dot{e}}}
\newcommand{\fdot}[0]{{\dot{f}}}
\newcommand{\gdot}[0]{{\dot{g}}}
\newcommand{\hdot}[0]{{\dot{h}}}
\newcommand{\idot}[0]{{\dot{i}}}
\newcommand{\jdot}[0]{{\dot{j}}}
\newcommand{\kdot}[0]{{\dot{k}}}
\newcommand{\ldot}[0]{{\dot{l}}}
\newcommand{\mdot}[0]{{\dot{m}}}
\newcommand{\ndot}[0]{{\dot{n}}}
%\newcommand{\odot}[0]{{\dot{o}}}
\newcommand{\pdot}[0]{{\dot{p}}}
\newcommand{\qdot}[0]{{\dot{q}}}
\newcommand{\rdot}[0]{{\dot{r}}}
\newcommand{\sdot}[0]{{\dot{s}}}
\newcommand{\tdot}[0]{{\dot{t}}}
\newcommand{\udot}[0]{{\dot{u}}}
\newcommand{\vdot}[0]{{\dot{v}}}
\newcommand{\wdot}[0]{{\dot{w}}}
\newcommand{\xdot}[0]{{\dot{x}}}
\newcommand{\ydot}[0]{{\dot{y}}}
\newcommand{\zdot}[0]{{\dot{z}}}
\newcommand{\addot}[0]{{\ddot{a}}}
\newcommand{\bddot}[0]{{\ddot{b}}}
\newcommand{\cddot}[0]{{\ddot{c}}}
%\newcommand{\dddot}[0]{{\ddot{d}}}
\newcommand{\eddot}[0]{{\ddot{e}}}
\newcommand{\fddot}[0]{{\ddot{f}}}
\newcommand{\gddot}[0]{{\ddot{g}}}
\newcommand{\hddot}[0]{{\ddot{h}}}
\newcommand{\iddot}[0]{{\ddot{i}}}
\newcommand{\jddot}[0]{{\ddot{j}}}
\newcommand{\kddot}[0]{{\ddot{k}}}
\newcommand{\lddot}[0]{{\ddot{l}}}
\newcommand{\mddot}[0]{{\ddot{m}}}
\newcommand{\nddot}[0]{{\ddot{n}}}
\newcommand{\oddot}[0]{{\ddot{o}}}
\newcommand{\pddot}[0]{{\ddot{p}}}
\newcommand{\qddot}[0]{{\ddot{q}}}
\newcommand{\rddot}[0]{{\ddot{r}}}
\newcommand{\sddot}[0]{{\ddot{s}}}
\newcommand{\tddot}[0]{{\ddot{t}}}
\newcommand{\uddot}[0]{{\ddot{u}}}
\newcommand{\vddot}[0]{{\ddot{v}}}
\newcommand{\wddot}[0]{{\ddot{w}}}
\newcommand{\xddot}[0]{{\ddot{x}}}
\newcommand{\yddot}[0]{{\ddot{y}}}
\newcommand{\zddot}[0]{{\ddot{z}}}

%<bold and dot greek symbols>
%

\newcommand{\Deltadot}[0]{{\dot{\Delta}}}
\newcommand{\Gammadot}[0]{{\dot{\Gamma}}}
\newcommand{\Lambdadot}[0]{{\dot{\Lambda}}}
\newcommand{\Omegadot}[0]{{\dot{\Omega}}}
\newcommand{\Phidot}[0]{{\dot{\Phi}}}
\newcommand{\Pidot}[0]{{\dot{\Pi}}}
\newcommand{\Psidot}[0]{{\dot{\Psi}}}
\newcommand{\Sigmadot}[0]{{\dot{\Sigma}}}
\newcommand{\Thetadot}[0]{{\dot{\Theta}}}
\newcommand{\Upsilondot}[0]{{\dot{\Upsilon}}}
\newcommand{\Xidot}[0]{{\dot{\Xi}}}
\newcommand{\alphadot}[0]{{\dot{\alpha}}}
\newcommand{\betadot}[0]{{\dot{\beta}}}
\newcommand{\chidot}[0]{{\dot{\chi}}}
\newcommand{\deltadot}[0]{{\dot{\delta}}}
\newcommand{\epsilondot}[0]{{\dot{\epsilon}}}
\newcommand{\etadot}[0]{{\dot{\eta}}}
\newcommand{\gammadot}[0]{{\dot{\gamma}}}
\newcommand{\kappadot}[0]{{\dot{\kappa}}}
\newcommand{\lambdadot}[0]{{\dot{\lambda}}}
\newcommand{\mudot}[0]{{\dot{\mu}}}
\newcommand{\nudot}[0]{{\dot{\nu}}}
\newcommand{\omegadot}[0]{{\dot{\omega}}}
\newcommand{\phidot}[0]{{\dot{\phi}}}
\newcommand{\pidot}[0]{{\dot{\pi}}}
\newcommand{\psidot}[0]{{\dot{\psi}}}
\newcommand{\rhodot}[0]{{\dot{\rho}}}
\newcommand{\sigmadot}[0]{{\dot{\sigma}}}
\newcommand{\taudot}[0]{{\dot{\tau}}}
\newcommand{\thetadot}[0]{{\dot{\theta}}}
\newcommand{\upsilondot}[0]{{\dot{\upsilon}}}
\newcommand{\varepsilondot}[0]{{\dot{\varepsilon}}}
\newcommand{\varphidot}[0]{{\dot{\varphi}}}
\newcommand{\varpidot}[0]{{\dot{\varpi}}}
\newcommand{\varrhodot}[0]{{\dot{\varrho}}}
\newcommand{\varsigmadot}[0]{{\dot{\varsigma}}}
\newcommand{\varthetadot}[0]{{\dot{\vartheta}}}
\newcommand{\xidot}[0]{{\dot{\xi}}}
\newcommand{\zetadot}[0]{{\dot{\zeta}}}

\newcommand{\Deltaddot}[0]{{\ddot{\Delta}}}
\newcommand{\Gammaddot}[0]{{\ddot{\Gamma}}}
\newcommand{\Lambdaddot}[0]{{\ddot{\Lambda}}}
\newcommand{\Omegaddot}[0]{{\ddot{\Omega}}}
\newcommand{\Phiddot}[0]{{\ddot{\Phi}}}
\newcommand{\Piddot}[0]{{\ddot{\Pi}}}
\newcommand{\Psiddot}[0]{{\ddot{\Psi}}}
\newcommand{\Sigmaddot}[0]{{\ddot{\Sigma}}}
\newcommand{\Thetaddot}[0]{{\ddot{\Theta}}}
\newcommand{\Upsilonddot}[0]{{\ddot{\Upsilon}}}
\newcommand{\Xiddot}[0]{{\ddot{\Xi}}}
\newcommand{\alphaddot}[0]{{\ddot{\alpha}}}
\newcommand{\betaddot}[0]{{\ddot{\beta}}}
\newcommand{\chiddot}[0]{{\ddot{\chi}}}
\newcommand{\deltaddot}[0]{{\ddot{\delta}}}
\newcommand{\epsilonddot}[0]{{\ddot{\epsilon}}}
\newcommand{\etaddot}[0]{{\ddot{\eta}}}
\newcommand{\gammaddot}[0]{{\ddot{\gamma}}}
\newcommand{\kappaddot}[0]{{\ddot{\kappa}}}
\newcommand{\lambdaddot}[0]{{\ddot{\lambda}}}
\newcommand{\muddot}[0]{{\ddot{\mu}}}
\newcommand{\nuddot}[0]{{\ddot{\nu}}}
\newcommand{\omegaddot}[0]{{\ddot{\omega}}}
\newcommand{\phiddot}[0]{{\ddot{\phi}}}
\newcommand{\piddot}[0]{{\ddot{\pi}}}
\newcommand{\psiddot}[0]{{\ddot{\psi}}}
\newcommand{\rhoddot}[0]{{\ddot{\rho}}}
\newcommand{\sigmaddot}[0]{{\ddot{\sigma}}}
\newcommand{\tauddot}[0]{{\ddot{\tau}}}
\newcommand{\thetaddot}[0]{{\ddot{\theta}}}
\newcommand{\upsilonddot}[0]{{\ddot{\upsilon}}}
\newcommand{\varepsilonddot}[0]{{\ddot{\varepsilon}}}
\newcommand{\varphiddot}[0]{{\ddot{\varphi}}}
\newcommand{\varpiddot}[0]{{\ddot{\varpi}}}
\newcommand{\varrhoddot}[0]{{\ddot{\varrho}}}
\newcommand{\varsigmaddot}[0]{{\ddot{\varsigma}}}
\newcommand{\varthetaddot}[0]{{\ddot{\vartheta}}}
\newcommand{\xiddot}[0]{{\ddot{\xi}}}
\newcommand{\zetaddot}[0]{{\ddot{\zeta}}}

\newcommand{\BDelta}[0]{\boldsymbol{\Delta}}
\newcommand{\BGamma}[0]{\boldsymbol{\Gamma}}
\newcommand{\BLambda}[0]{\boldsymbol{\Lambda}}
\newcommand{\BOmega}[0]{\boldsymbol{\Omega}}
\newcommand{\BPhi}[0]{\boldsymbol{\Phi}}
\newcommand{\BPi}[0]{\boldsymbol{\Pi}}
\newcommand{\BPsi}[0]{\boldsymbol{\Psi}}
\newcommand{\BSigma}[0]{\boldsymbol{\Sigma}}
\newcommand{\BTheta}[0]{\boldsymbol{\Theta}}
\newcommand{\BUpsilon}[0]{\boldsymbol{\Upsilon}}
\newcommand{\BXi}[0]{\boldsymbol{\Xi}}
\newcommand{\Balpha}[0]{\boldsymbol{\alpha}}
\newcommand{\Bbeta}[0]{\boldsymbol{\beta}}
\newcommand{\Bchi}[0]{\boldsymbol{\chi}}
\newcommand{\Bdelta}[0]{\boldsymbol{\delta}}
\newcommand{\Bepsilon}[0]{\boldsymbol{\epsilon}}
\newcommand{\Beta}[0]{\boldsymbol{\eta}}
\newcommand{\Bgamma}[0]{\boldsymbol{\gamma}}
\newcommand{\Bkappa}[0]{\boldsymbol{\kappa}}
\newcommand{\Blambda}[0]{\boldsymbol{\lambda}}
\newcommand{\Bmu}[0]{\boldsymbol{\mu}}
\newcommand{\Bnu}[0]{\boldsymbol{\nu}}
%\newcommand{\Bomega}[0]{\boldsymbol{\omega}}
\newcommand{\Bphi}[0]{\boldsymbol{\phi}}
\newcommand{\Bpi}[0]{\boldsymbol{\pi}}
\newcommand{\Bpsi}[0]{\boldsymbol{\psi}}
\newcommand{\Brho}[0]{\boldsymbol{\rho}}
\newcommand{\Bsigma}[0]{\boldsymbol{\sigma}}
%\newcommand{\Btau}[0]{\boldsymbol{\tau}}
%\newcommand{\Btheta}[0]{\boldsymbol{\theta}}
\newcommand{\Bupsilon}[0]{\boldsymbol{\upsilon}}
\newcommand{\Bvarepsilon}[0]{\boldsymbol{\varepsilon}}
\newcommand{\Bvarphi}[0]{\boldsymbol{\varphi}}
\newcommand{\Bvarpi}[0]{\boldsymbol{\varpi}}
\newcommand{\Bvarrho}[0]{\boldsymbol{\varrho}}
\newcommand{\Bvarsigma}[0]{\boldsymbol{\varsigma}}
\newcommand{\Bvartheta}[0]{\boldsymbol{\vartheta}}
\newcommand{\Bxi}[0]{\boldsymbol{\xi}}
\newcommand{\Bzeta}[0]{\boldsymbol{\zeta}}
%
%</bold and dot greek symbols>
%<infrequent>
%
%\newcommand{\AreaOp}[1]{\AName_{#1}}
%\newcommand{\Babs}[0]{\abs{\BB}}
%\newcommand{\Bcap}[0]{\hat{\BB}}
%\newcommand{\BrPrimeRej}[0]{\rcap(\rcap \wedge \Br')}
%\newcommand{\CA}[0]{\mathcal{A}}
%\newcommand{\Cos}[1]{\cos{\left({#1}\right)}}
%\newcommand{\Det}[1] {\abs{#1}}
%\newcommand{\Dsq}[2] {\frac {\partial^2 {#1}} {\partial {#2}^2}}
%\newcommand{\Exp}[1]{\exp{\left({#1}\right)}}
%\newcommand{\Norm}[1]{\left\lVert{#1}\right\rVert}
%\newcommand{\Sin}[1]{\sin{\left({#1}\right)}}
%\newcommand{\T}[0]{\text{T}}
%\newcommand{\VolumeOp}[1]{\VName_{#1}}
%\newcommand{\agrad}[0]{\Ba \cdot \nabla}
%\newcommand{\alphacap}[0]{\hat{\boldsymbol{\alpha}}}
%\newcommand{\Fcap}[0]{\hat{\BF}}
%\newcommand{\bithree}[0]{{\Bi}_3}
%\newcommand{\bxa}[0]{\Bx\Ba}
%\newcommand{\coordvec}[2]{
%\newcommand{\costheta}[0]{\acap \cdot \xcap}
%\newcommand{\ddt}[1]{\ddot{#1}}
%\newcommand{\ddu}[1] {\frac {d{#1}} {du}}
%\newcommand{\dsqxj}[2] {\frac {\partial^2 {#1}} {\partial {x_{#2}}^2}}
%\newcommand{\dtheta}[1]{\frac{d {#1}}{d \theta}}
%\newcommand{\dt}[1]{\dot{#1}}
%\newcommand{\dt}[1]{\frac{d {#1}}{dt}}
%\newcommand{\dxj}[2] {\frac {\partial {#1}} {\partial {x_{#2}}}}
%\newcommand{\halfPhi}[0]{\frac{\phi}{2}}
%\newcommand{\half}[0]{\inv{2}}
%\newcommand{\inv}[1]{\frac{1}{#1}}
%\newcommand{\laplacian}[0]{\nabla^2}
%\newcommand{\matrixoftx}[3]{
%\newcommand{\nrrp}[0]{\norm{\rcap \wedge \Br'}}
%\newcommand{\oiint}{\bigcirc \hspace{-1.4em} \int \hspace{-.8em} \int}
%\newcommand{\transpose}[1]{{#1}^{\text{T}}}
%\newcommand{\transpose}[1]{{{#1}^{\TextTranspose}}}
%\newcommand{\transpose}[1]{{{#1}^{\text{T}}}}
%\newcommand{\barA}[0]{\bar{A}}
%\newcommand{\qbar}[0]{\bar{q}}
%\newcommand{\qdotbar}[0]{\dot{\bar{q}}}
%
%</infrequent>




\newcommand{\EE}[0]{\boldsymbol{\mathcal{E}}}
\newcommand{\HH}[0]{\boldsymbol{\mathcal{H}}}

\usepackage[bookmarks=true]{hyperref}

\usepackage{color,cite,graphicx}
   % use colour in the document, put your citations as [1-4]
   % rather than [1,2,3,4] (it looks nicer, and the extended LaTeX2e
   % graphics package. 
\usepackage{latexsym,amssymb,epsf} % don't remember if these are
   % needed, but their inclusion can't do any damage


\title{ Fourier series Vacuum Maxwell's equations. }
\author{Peeter Joot}
\date{ Feb 03, 2009.  Last Revision: $Date: 2009/02/04 13:45:02 $ }

\begin{document}

\maketitle{}

\tableofcontents

\section{ Motivation. }

In \cite{bohm1989qt}, 
after finding a formulation of Maxwell's equations that he likes, his next
step is to assume the electric and magnetic fields can be expressed in 
a 3D Fourier series form, with periodicity in some repeated volume 
of space, and then procedes to evaluate the energy of the 
field.

\section{ Setup. }

Let's try this.  Instead of using the sine and cosine fourier series
which looks more complex than it ought to be, use of a complex exponential
ought to be cleaner.

\subsection{ 3D Fourier series in complex exponential form. }

For a multivector function $f(\Bx, t)$, periodic in some rectangular spatial volume, let's assume that we have a
3D Fourier series representation.

Define the element of volume for our fundamental wavelengths to be the region bounded by three intervals in the $x^1, x^2, x^3$ directions respectively

\begin{align*}
I_1 &= [ a^1, a^1 + \lambda_1 ] \\
I_2 &= [ a^2, a^2 + \lambda_2 ] \\
I_3 &= [ a^3, a^3 + \lambda_3 ] \\
\end{align*}

Our assumed Fourier representation is then

\begin{align*}
f(\Bx, t) &= \sum_{\Bk} \hat{f}_{\Bk}(t) \exp\left( - \sum_j \frac{2 \pi i k_j x^j}{\lambda_j} \right)
\end{align*}

Here $\hat{f}_{\Bk} = \hat{f}_{\{k_1, k_2, k_3\}}$ is indexed over a triplet of integer values, and the $k_1, k_2, k_3$ indexes take on all integer values in the $[-\infty, \infty]$ range.

Note that we also wish to allow $i$ to not just be a generic complex number, but allow for the use of either the Euclidian or Minkowski pseudoscalar

\begin{align*}
i = \gamma_0 \gamma_1 \gamma_2 \gamma_3 = \sigma_1 \sigma_2 \sigma_3
\end{align*}

Because of this we should not assume that we can commute $i$, or our exponentials with the functions $f(\Bx,t)$, or $\hat{f}_{\Bk}(t)$.

\begin{align*}
\int_{x^1 = \partial I_1} &\int_{x^2 = \partial I_2} \int_{x^3 = \partial I_3} f(\Bx, t) 
e^{ 2 \pi i m_j x^j/\lambda_j}
dx^1 dx^2 dx^3 \\
&= \sum_{\Bk} \hat{f}_{\Bk}(t) \int_{x^1 = \partial I_1} \int_{x^2 = \partial I_2} \int_{x^3 = \partial I_3} dx^1 dx^2 dx^3 e^{ 2 \pi i (m_j - k_j) x^j/\lambda_j} dx^1 dx^2 dx^3
\end{align*}

But each of these integrals is just $\delta_{\Bk,\Bm} \lambda_1 \lambda_2 \lambda_3$, giving us

\begin{align*}
\hat{f}_{\Bk}(t)
&= \inv{\lambda_1 \lambda_2 \lambda_3 } \int_{x^1 = \partial I_1} \int_{x^2 = \partial I_2} \int_{x^3 = \partial I_3} f(\Bx, t) \exp\left( \sum_j \frac{2 \pi i k_j x^j}{\lambda_j} \right) dx^1 dx^2 dx^3 \\
\end{align*}

For short lets write this as

\begin{align}
f(\Bx, t) &= \sum_{\Bk} \hat{f}_{\Bk}(t) \exp\left( - \frac{2 \pi i k_j x^j }{\lambda_j} \right) \\
\hat{f}_{\Bk}( t) &= \inv{V} \int f(\Bx, t) \exp\left( \frac{2 \pi i k_j x^j}{\lambda_j} \right) d^3 x
\end{align}

\subsection{ Vacuum equation. }

Now that we have a desirable seeming Fourier series representation, we 
want to apply this to Maxwell's equation for the vacuum.  We will use the 
STA formulation of Maxwell's equation, but use the unit convention of Bohm's
book.

In \cite{PJrayleighJeans} the STA equivalent to Bohm's notation 
for Maxwell's equations was found to be

\begin{align}
F &= \EE + i\HH \\
J &= (\rho + \Bj) \gamma_0 \\
\grad F &= 4 \pi J
\end{align}

This is the cgs form of Maxwell's equation, but with the old style $\HH$ for $c\BB$, and $\EE$ for $\BE$.  In more recent texts $\EE$ is reserved for electromotive flux.  In this set of notes I use Bohm's notation, since the aim is to clarify for myself aspects of his treatment.

For the vacuum equation, we make an explicit spacetime split by premultiplying with $\gamma_0$

\begin{align*}
\gamma_0 \grad 
&= \gamma_0 (\gamma^0 \partial_0 + \gamma^k \partial_k) \\
&= \partial_0 - \gamma^k \gamma_0 \partial_k \\
&= \partial_0 + \gamma_k \gamma_0 \partial_k \\
&= \partial_0 + \sigma_k \partial_k \\
&= \partial_0 + \spacegrad \\
\end{align*}

So our vacuum equation is just

\begin{align}\label{eqn:vacuumMaxwell}
(\partial_0 + \spacegrad) F = 0
\end{align}

\section{ First order vacuum solution with Fourier series. }

Now that a notation for the 3D Fourier series has been established, we
can assume a series solution for our field of the form

\begin{align}\label{eqn:assumed}
F(\Bx,t) = \sum_{\Bk} \hat{F}_{\Bk}(t) e^{-2\pi i k_j x^j/\lambda_j}
\end{align}

can now apply this to the vacuum Maxwell equation \ref{eqn:vacuumMaxwell}.
This gives us

\begin{align*}
\sum_{\Bk} \left(\partial_t \hat{F}_{\Bk}(t) \right) e^{-2\pi i k_j x^j/\lambda_j}
&= -c \sum_{\Bk, m} \sigma^m \hat{F}_{\Bk}(t) \PD{x^m}{} e^{-2\pi i k_j x^j/\lambda_j} \\
&= -c \sum_{\Bk, m} \sigma^m \hat{F}_{\Bk}(t) \left(-2 \pi \frac{k_m}{\lambda_m}\right) e^{-2\pi i k_j x^j/\lambda_j} \\
&= 2 \pi c \sum_{\Bk} \sum_m \frac{\sigma^m k_m}{\lambda_m} \hat{F}_{\Bk}(t) i e^{-2\pi i k_j x^j/\lambda_j} \\
\end{align*}

Now lets invent (perhaps abuse) some notation to tidy things up.  As a subscript on our Fourier coefficients we've used $\Bk$ as an index.  Let's also use it as a vector, and write

\begin{align*}
\Bk = 2 \pi \sum_m \frac{\sigma^m k_m}{\lambda_m}
\end{align*}

It was also previously implied that we had

\begin{align*}
\Bx = \sum_m \sigma_m x^m
\end{align*}

Also noting that $i$ commutes with $\Bk$ and since $F$ is also an STA bivector $i$ commutes with $F$.  Putting all this together we have

\begin{align*}
\sum_{\Bk} \left(\partial_t \hat{F}_{\Bk}(t) \right) e^{-i \Bk \cdot \Bx }
&= i c \sum_{\Bk} \Bk \hat{F}_{\Bk}(t) e^{- i \Bk \cdot \Bx } \\
\end{align*}

Term by term we now have a (big ass, triple infinite) set of very simple first order differential equations, one for each $\Bk$ triplet of indexes.  Specifically this is

\begin{align*}
\hat{F}_{\Bk}' &= i c \Bk \hat{F}_{\Bk}
\end{align*}

With solutions

\begin{align*}
\hat{F}_{0} &= C_{0} \\
\hat{F}_{\Bk} &= \exp\left(i c \Bk t \right) C_{\Bk} \\
\end{align*}

Here $C_{\Bk}$ is an undetermined STA bivector.  Note that we have to keep this undetermined coefficient on the right hand side of the exponential since we cannot assume it commutes with a factor of the form $\exp(i\Bk\phi)$.  Substitution back into our assumed solution sum we have a solution to Maxwell's equation, in terms of a set of as yet undetermined (bivector) coefficients

\begin{align}\label{eqn:undetermined}
F(\Bx,t) = C_0 + \sum_{\Bk \ne 0} \exp\left(i c \Bk t \right) C_{\Bk} \exp(-i \Bk \cdot \Bx )
\end{align}

Now, observe the form of this sum for $t=0$.  This is

\begin{align*}
F(\Bx,0) 
&= C_0 + \sum_{\Bk \ne 0} C_{\Bk} \exp(-i \Bk \cdot \Bx ) \\
&= \sum_{\Bk} C_{\Bk} \exp(-i \Bk \cdot \Bx ) \\
\end{align*}

So, the $C_k$ coefficients are precisely the Fourier coefficients of $F(\Bx,0)$.  This is to be expected having repeatedly seen similar results in the Fourier transform treatments of 
\cite{PJfourierMaxwellSecondOrder}, \cite{PJfirstOrderMaxwell}, and \cite{PJ4dFourier}.
If we write 
$
C_{\Bk}
=
{\left.\hat{F}_{\Bk}\right\vert}_{t=0}
$, we then have an equation for the complete time evolution of any spatially periodic electrodynamic field in terms of the field value at all points in the region at some initial time.  That is

\begin{align}
F(\Bx,t) &= \sum_{\Bk} \exp\left(i c \Bk t \right) {\left.\hat{F}_{\Bk}\right\vert}_{t=0} \exp(-i \Bk \cdot \Bx) \\
{\left.\hat{F}_{\Bk}\right\vert}_{t=0} &= \inv{V} \int F(\Bx', 0) \exp\left( i\Bk \cdot \Bx' \right) d^3 x'
\end{align}

Regrouping slightly, taking advantage that we can commute our complex exponential to the left (cannot do so for the $i\Bk$ exponential)
we can also write this as a convolution with a Fourier kernel (a Green's function).  That is

\begin{align}
F(\Bx,t) &= \inv{V} \int \sum_k e^{ - i (\Bk \cdot (\Bx' - \Bx) - \Bk c t) } F(\Bx', 0) d^3 x'
\end{align}

Or
\begin{align}
F(\Bx,t) &= \int G(\Bx' - \Bx, t) F(\Bx', 0) d^3 x' \\
G(\Bx,t) &= \inv{V} \sum_k e^{ - i (\Bk \cdot \Bx - \Bk c t) }
\end{align}

Okay, that's cool.  We've now got the basic periodicity result directly from Maxwell's equation in one shot.  No need to drop down to
potentials, or even the separate electric or magnetic components of our field $F = \EE + i \HH$.

\subsection{ Electric and magnetic field components. }

Given $F$ we can compute the electric and magnetic field components using a spacetime observer split

\begin{align*}
\EE &= \inv{2}(F - \gamma_0 F \gamma_0 ) \\
\HH &= \inv{2i}(F + \gamma_0 F \gamma_0 ) \\
\end{align*}

FIXME: thought I once wrote a method of doing this with only $i$?  I don't see how to do that now?  Go back in older notes and fix or understand what I did.
%Alternately separation using the pseudoscalar is possible
%FIXME: ...
%F = \EE + i\HH
%F = \EE + i\HH

Since we are used to separate electric and magnetic fields, as measurable and experiencable quantities, reexpressing the previous results 
in terms of separate fields out to take some of the abstraction out of the picture.  Additionally, in the Fourier components we expect, or at least ought to be able to show, a $\Bk$ dependence between $\EE$ and $\HH$.

FIXME:TODO...

\subsection{ Separate electric and magnetic fields with the boundary conditions unspecified. }

Fixing the boundary value in terms of the initial field at $t=0$ is not the only option.  We see similar things in classical mechanics in constant acceleration problems where one can use an initial position and velocity, or positions at two different times, and so forth.  Bohm leaves his equivalents to the integration constants $C_{\Bk}$ unspecified.  If we do so too we have equation \ref{eqn:undetermined}.  How does that look with separate fields?

FIXME:TODO:...

\section{ Second order treatment with potentials. }

\subsection{ With the Lorentz gauge. }

% FIXME:TODO:
%Next in the sequence of understanding
%Bohm's Rayleigh-Jeans result will be to consider the energy and momentum density of the field, but that's a job for a different day.

\bibliographystyle{plainnat}
\bibliography{myrefs}

\end{document}
