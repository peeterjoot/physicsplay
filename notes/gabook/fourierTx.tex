\chapter{Some Fourier transform notes.}\label{chap:PJqmFourier}
\date{ Jan 09, 2009.  Last Revision: $Date: 2009/06/04 13:13:27 $ }

\section{Motivation. }

In \cite{mcmahon2005qmd} the Fourier transform pairs are written in a somewhat 
non-orthodox seeming way.

\begin{align*}
\phi(p) &= \FM \Iinf{\psi(x) e^{-ipx/\hbar} dx} \\
\psi(x) &= \FM \Iinf{\phi(p) e^{ipx/\hbar} dp} \\
\end{align*}

The aim here is to do verify this form and do a couple associated calculations (like the Rayleigh energy theorem).

\section{Verify transform pair. }

As an exercise to verify, in a not particularly rigorous fashion, that we get back our original function applying the forward and reverse transformations
in sequence.  Specifically, let's compute

\begin{align*}
\FF^{-1}(\FF(\psi(x)))
&= \PV \FM \Iinf{\left(\FM \Iinf{\psi(u) e^{-ipu/\hbar} du}\right) e^{ipx/\hbar} dp} \\
\end{align*}

Here $\PV$ is the principle value of the integral, which is the specifically symmetric integration

\begin{align*}
\PV \Iinf{} &= \lim_{R \rightarrow \infty} \int_{-R}^R
\end{align*}

We have for the integration
\begin{align*}
\FF^{-1}(\FF(\psi(x)))
&= \PV \inv{{2\pi\hbar}} \int du \psi(u) \int e^{i p (x-u)/\hbar} dp \\
\end{align*}

Now, let $v = (x-u)/\hbar$, or $u=x-v\hbar$ for

\begin{align*}
\inv{{2\pi}} \int dv \psi(x -v\hbar) \int_{-R}^R e^{i p v} dp 
&= \inv{{2\pi}} \int dv \psi(x - v\hbar) \left. \inv{i v} e^{i p v} \right\vert_{p=-R}^R \\
&= \int dv \psi(x - v\hbar) \frac{\sin(R v)}{\pi v} \\
\end{align*}

In a hand-waving (aka. Engineering) fashion, one can identify the limit of $\sin(Rv)/\pi v$ as the Dirac delta function and
then declare that this does in fact recover the value of $\psi(x)$ by a Dirac delta filtering around the point $v=0$.

This does in fact work out, but as a strict integration exercise one ought to be able to do better.
Observe that the integral performed here wasn't really valid for $v=0$ in which case the exponential takes the value of one, so it would be
better to treat the neighborhood of $v=0$ more carefully.  Doing so

\begin{align*}
\inv{{2\pi}} \int dv \psi(x -v\hbar) \int_{-R}^R e^{i p v} dp 
&= \int_{v=-\infty}^{-\epsilon} dv \psi(x - v\hbar) \frac{\sin(R v)}{\pi v} \\
&+ \int_{v=\epsilon}^\infty dv \psi(x - v\hbar) \frac{\sin(R v)}{\pi v} \\
&+ \inv{{2\pi}} \int_{v=-\epsilon}^\epsilon dv \psi(x - v\hbar) \int_{-R}^R e^{i p v} dp \\
&= \int_{v=\epsilon}^\infty dv \left( \psi(x - v\hbar) +\psi(x + v\hbar) \right) \frac{\sin(R v)}{\pi v} \\
&+ \inv{{2\pi}} \int_{v=-\epsilon}^\epsilon dv \psi(x - v\hbar) \int_{-R}^R e^{i p v} dp \\
\end{align*}

Now, evaluating this with $\epsilon$ allowing to tend to zero and $R$ tending to infinity simultaneously is troublesome seeming.  I seem to recall that one
can do something to the effect of setting $\epsilon=1/R$, and then carefully take the limit, but it isn't obvious to me how exactly to do this without
pulling out an old text.
While some kind of ad-hoc limit process can likely be done and justified in some fashion, one can see why the hard core mathematicians had to invent
an alternate stricter mathematics to deal with this stuff rigorously.

That said, from an intuitive point of view, it is fairly clear that the filtering involved here will recover the average of
$\psi(x)$ in the neighborhood assuming that it is piecewise continuous:

\begin{align*}
\FF^{-1}(\FF(\psi(x))) &= \inv{2} \left( \psi(x -\epsilon) + \psi(x + \epsilon) \right)
\end{align*}

After digging through my old texts I found a treatment of the Fourier integral very similar to what I've done above
in \cite{lepage1980cva}, but the important details aren't omitted (like integrability conditions).
I'd read that and some of my treatment is obviously was based on that.  That text 
treats this still with Riemann (and not Lebesgue) integration, but very carefully.

\section{Parseval's theorem. }

In \cite{mcmahon2005qmd} he notes that Parseval's theorem tells us

\begin{align*}
\Iinf{f(x)g(x) dx} &= \Iinf{F(k)G^\conj(k) dk} \\
\Iinf{\Abs{f(x)}^2 dx} &= \Iinf{\Abs{F(k)}^2 dk}
\end{align*}

The last of these in \cite{haykin1994cs} is called Rayleigh's energy theorem.  
As a refresher in 
Fourier manipulation, and to translate to the QM Fourier transform notation, let's go through the arguments
required to prove these.

\subsection{Convolution. }

We'll need convolution in the QM notation as a first step to express
the transform of a product.

Suppose we have two functions 
$\phi_i(x)$
, and their transform
pairs 
$\tilde{\phi}_i(x) = \FF(\phi_i)$, then the transform of the product is

\begin{align*}
\tilde{\Phi}_{12}(p) = \FF(\phi_1(x)\phi_2(x)) &= \FM \Iinf{\phi_1(x) \phi_2(x) e^{-ipx/\hbar} dx}
\end{align*}

Now write $\phi_2(x)$ in terms of its inverse transform

\begin{align*}
\phi_2(x)) &= \FM \Iinf{\tilde{\phi}_2(u) e^{iux/\hbar} du}
\end{align*}

The product transform is now
\begin{align*}
\tilde{\Phi}_{12}(p) 
&= \FM \Iinf{ \phi_1(x)             \FM \Iinf{\tilde{\phi}_2(u) e^{iux/\hbar} du} e^{-ipx/\hbar} dx} \\
&= \FM \Iinf{ du \tilde{\phi}_2(u)  \FM \Iinf{ \phi_1(x) e^{-ix(p-u)/\hbar} dx } } \\
&= \FM \Iinf{ du \tilde{\phi}_2(u)  \tilde{\phi}_1(p-u) } \\
&= \FM \Iinf{ dv \tilde{\phi}_1(v)  \tilde{\phi}_2(p-v)}  \\
\end{align*}

So we have product transform expressed by the convolution integral, but have an extra $1/\sqrt{2\pi\hbar}$ factor in this form

\begin{align}
\phi_1(x) \phi_2(x) \Leftrightarrow \FM \Iinf{ dv \tilde{\phi}_1(v)  \tilde{\phi}_2(p-v) } \\
\end{align}

\subsection{Conjugation. }

Next we need to see how the conjugate transforms.  This is pretty straight forward

\begin{align*}
\phi^\conj(x) 
&\Leftrightarrow \FM \Iinf{ \phi^\conj(x) e^{-ipx/\hbar dx}} \\
&= \left(\FM \Iinf{ \phi(x) e^{ipx/\hbar dx}}\right)^\conj \\
\end{align*}

So we have
\begin{align*}
\phi^\conj(x) \Leftrightarrow \left(\tilde{\phi}(-p) \right)^\conj \\
\end{align*}

\subsection{Rayleigh's Energy Theorem. }

Now, we should be set to prove the energy theorem.  Let's start with the 
momentum domain integral and translate back to position basis

\begin{align*}
\Iinf{ dp \tilde{\phi}(p){\tilde{\phi}}^\conj(p) } 
&= \Iinf{ dp \FM \Iinf{ dx \phi(x) e^{-ipx/\hbar}} {\tilde{\phi}}^\conj(p) }  \\
&= \Iinf{ dx \phi(x) \FM \Iinf{ dp {\tilde{\phi}}^\conj(p) } e^{-ipx/\hbar}} \\
&= \Iinf{ dx \phi(x) \FM \Iinf{ dp {\tilde{\phi}}^\conj(-p) } e^{ipx/\hbar}} \\
&= \Iinf{ dx \phi(x) \FF^{-1}( {\tilde{\phi}}^\conj(-p) )}  \\
\end{align*}

This is exactly our desired result
\begin{align*}
\Iinf{ dp \tilde{\phi}(p){\tilde{\phi}}^\conj(p) } &= 
\Iinf{ dx \phi(x) \phi^\conj(x)}
\end{align*}

Hmm.  Didn't even need the convolution as the systems book did.  Will have to look over how they did this more closely.  Regardless, this method was nicely direct.
