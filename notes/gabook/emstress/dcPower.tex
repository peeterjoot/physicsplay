%
% Copyright � 2012 Peeter Joot.  All Rights Reserved.
% Licenced as described in the file LICENSE under the root directory of this GIT repository.
%

% 
% 
\chapter{DC Power consumption formula for resistive load}
\label{chap:dcPower}
%\date{Jan 06, 2009.  dcPower.tex}

\section{Motivation}

Despite a lot of recent study of electrodynamics, faced with a simple electrical problem:

``What capacity generator would be required for an arc welder on a 30 Amp breaker using a 220 volt circuit''.

I couldn't think of how to answer this off the top of my head.  Back in school without hesitation I would have
been able to plug into $P = I V$ to get a capacity estimation for the generator.

Having forgotten the formula, let's see how we get that $P = I V$ relationship from Maxwell's equations.

\section{}

Having just derived the Poynting energy momentum density relationship from Maxwell's equations, let that be the starting
point

\begin{align*}
\frac{d}{dt}\left(\frac{\epsilon_0}{2}\left(\BE^2 + c^2\BB^2 \right) \right) = - \inv{\mu_0} \left(\BE \cross \BB \right) - \BE \cdot \Bj
\end{align*}

The left hand side is the energy density time variation, which is power per unit volume, so we can integrate this
over a volume to determine the power associated with a change in the field.

\begin{align*}
P = -\int dV \left( \inv{\mu_0} \left(\BE \cross \BB \right) + \BE \cdot \Bj \right)
\end{align*}

As a reminder, let's write the magnetic and electric fields in terms of potentials.

In terms of the ``native'' four potential our field is

\begin{align*}
F 
&= \BE + ic \BB \\
&= \grad \wedge A \\
&= \gamma^0 \gamma_k \partial_0 A^k + \gamma^j \gamma_0 \partial_j A^0 + \gamma^m \wedge \gamma_n \partial_m A^n \\
\end{align*}

The electric field is

\begin{align*}
\BE &= \sum_k (\grad \wedge A) \cdot (\gamma^0 \gamma^k) \gamma_k \gamma_0 \\
\end{align*}

From this, with $\phi = A^0$, and $\BA = \sigma_k A^k$ we have
\begin{align*}
\BE &= -\inv{c} \PD{t}{\BA} - \grad \phi \\
i\BB &= \spacegrad \wedge \BA
\end{align*}

Now, the arc welder is (I think) a DC device, and to 
get a rough idea of what it requires lets just assume that its a rectifier that outputs RMS DC.
So if we make this simplification, and assume that we have a 
purely resistive load (ie: no inductance and therefore no magnetic fields) and a DC supply and constant current, then
we eliminate the vector potential terms.

This wipes out the $\BB$ and the Poynting vector, and leaves our electric field specified in terms
of the potential difference across the load $\BE = -\spacegrad \phi$.

That is
\begin{align*}
P &= \int dV (\spacegrad \phi) \cdot \Bj
\end{align*}

Suppose we are integrating over the length of a uniformly resistive load with some fixed cross sectional area.  $\Bj dV$ is then the magnitude of the current directed along the wire for its length.  This basically leaves us with a line integral over the length of the wire that we are measuring our potential drop over so we are left with just

\begin{align*}
P &= (\delta \phi) I
\end{align*}

This $\delta \phi$ is just our voltage drop $V$, and this gives us the desired $P = I V$ equation.
Now, I also recall from school
now that I think about it that $P = I V$ also applied to inductive loads, but it required that $I$ and $V$ be phasors that
represented the sinusoidal currents and sources.  A good followup exercise would be to show from Maxwell's equations
that this is in fact valid.  Eventually I'll know the origin of all the formulas from my old engineering courses.
