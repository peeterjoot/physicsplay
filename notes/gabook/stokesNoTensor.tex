\documentclass[]{eliblog}

\usepackage{amsmath}
\usepackage{mathpazo}

%
% shorthand for bold symbols, convenient for vectors and matrices
%
\newcommand{\Ba}[0]{\mathbf{a}}
\newcommand{\Bb}[0]{\mathbf{b}}
\newcommand{\Bc}[0]{\mathbf{c}}
\newcommand{\Bd}[0]{\mathbf{d}}
\newcommand{\Be}[0]{\mathbf{e}}
\newcommand{\Bf}[0]{\mathbf{f}}
\newcommand{\Bg}[0]{\mathbf{g}}
\newcommand{\Bh}[0]{\mathbf{h}}
\newcommand{\Bi}[0]{\mathbf{i}}
\newcommand{\Bj}[0]{\mathbf{j}}
\newcommand{\Bk}[0]{\mathbf{k}}
\newcommand{\Bl}[0]{\mathbf{l}}
\newcommand{\Bm}[0]{\mathbf{m}}
\newcommand{\Bn}[0]{\mathbf{n}}
\newcommand{\Bo}[0]{\mathbf{o}}
\newcommand{\Bp}[0]{\mathbf{p}}
\newcommand{\Bq}[0]{\mathbf{q}}
\newcommand{\Br}[0]{\mathbf{r}}
\newcommand{\Bs}[0]{\mathbf{s}}
\newcommand{\Bt}[0]{\mathbf{t}}
\newcommand{\Bu}[0]{\mathbf{u}}
\newcommand{\Bv}[0]{\mathbf{v}}
\newcommand{\Bw}[0]{\mathbf{w}}
\newcommand{\Bx}[0]{\mathbf{x}}
\newcommand{\By}[0]{\mathbf{y}}
\newcommand{\Bz}[0]{\mathbf{z}}
\newcommand{\BA}[0]{\mathbf{A}}
\newcommand{\BB}[0]{\mathbf{B}}
\newcommand{\BC}[0]{\mathbf{C}}
\newcommand{\BD}[0]{\mathbf{D}}
\newcommand{\BE}[0]{\mathbf{E}}
\newcommand{\BF}[0]{\mathbf{F}}
\newcommand{\BG}[0]{\mathbf{G}}
\newcommand{\BH}[0]{\mathbf{H}}
\newcommand{\BI}[0]{\mathbf{I}}
\newcommand{\BJ}[0]{\mathbf{J}}
\newcommand{\BK}[0]{\mathbf{K}}
\newcommand{\BL}[0]{\mathbf{L}}
\newcommand{\BM}[0]{\mathbf{M}}
\newcommand{\BN}[0]{\mathbf{N}}
\newcommand{\BO}[0]{\mathbf{O}}
\newcommand{\BP}[0]{\mathbf{P}}
\newcommand{\BQ}[0]{\mathbf{Q}}
\newcommand{\BR}[0]{\mathbf{R}}
\newcommand{\BS}[0]{\mathbf{S}}
\newcommand{\BT}[0]{\mathbf{T}}
\newcommand{\BU}[0]{\mathbf{U}}
\newcommand{\BV}[0]{\mathbf{V}}
\newcommand{\BW}[0]{\mathbf{W}}
\newcommand{\BX}[0]{\mathbf{X}}
\newcommand{\BY}[0]{\mathbf{Y}}
\newcommand{\BZ}[0]{\mathbf{Z}}

\newcommand{\Bzero}[0]{\mathbf{0}}
\newcommand{\Btheta}[0]{\boldsymbol{\theta}}
\newcommand{\Btau}[0]{\boldsymbol{\tau}}
\newcommand{\Bomega}[0]{\boldsymbol{\omega}}

%
% shorthand for unit vectors
%
\newcommand{\acap}[0]{\hat{\Ba}}
\newcommand{\bcap}[0]{\hat{\Bb}}
\newcommand{\ccap}[0]{\hat{\Bc}}
\newcommand{\dcap}[0]{\hat{\Bd}}
\newcommand{\ecap}[0]{\hat{\Be}}
\newcommand{\fcap}[0]{\hat{\Bf}}
\newcommand{\gcap}[0]{\hat{\Bg}}
\newcommand{\hcap}[0]{\hat{\Bh}}
\newcommand{\icap}[0]{\hat{\Bi}}
\newcommand{\jcap}[0]{\hat{\Bj}}
\newcommand{\kcap}[0]{\hat{\Bk}}
\newcommand{\lcap}[0]{\hat{\Bl}}
\newcommand{\mcap}[0]{\hat{\Bm}}
\newcommand{\ncap}[0]{\hat{\Bn}}
\newcommand{\ocap}[0]{\hat{\Bo}}
\newcommand{\pcap}[0]{\hat{\Bp}}
\newcommand{\qcap}[0]{\hat{\Bq}}
\newcommand{\rcap}[0]{\hat{\Br}}
\newcommand{\scap}[0]{\hat{\Bs}}
\newcommand{\tcap}[0]{\hat{\Bt}}
\newcommand{\ucap}[0]{\hat{\Bu}}
\newcommand{\vcap}[0]{\hat{\Bv}}
\newcommand{\wcap}[0]{\hat{\Bw}}
\newcommand{\xcap}[0]{\hat{\Bx}}
\newcommand{\ycap}[0]{\hat{\By}}
\newcommand{\zcap}[0]{\hat{\Bz}}
\newcommand{\thetacap}[0]{\hat{\Btheta}}

%
% to write R^n and C^n in a distinguishable fashion.  Perhaps change this
% to the double lined characters upon figuring out how to do so.
%
\newcommand{\C}[1]{$\mathbb{C}^{#1}$}
\newcommand{\R}[1]{$\mathbb{R}^{#1}$}

%
% various generally useful helpers
%

% derivative of #1 wrt. #2:
\newcommand{\D}[2] {\frac {d#2} {d#1}}

\newcommand{\inv}[1]{\frac{1}{#1}}
\newcommand{\cross}[0]{\times}

\newcommand{\abs}[1]{\lvert{#1}\rvert}
\newcommand{\norm}[1]{\lVert{#1}\rVert}
\newcommand{\innerprod}[2]{\langle{#1}, {#2}\rangle}
\newcommand{\dotprod}[2]{{#1} \cdot {#2}}
\newcommand{\bdotprod}[2]{\left({#1} \cdot {#2}\right)}
\newcommand{\crossprod}[2]{{#1} \cross {#2}}
\newcommand{\tripleprod}[3]{\dotprod{\left(\crossprod{#1}{#2}\right)}{#3}}

\DeclareMathOperator{\Proj}{Proj}
\DeclareMathOperator{\Span}{span}
\DeclareMathOperator{\Sgn}{sgn}
\DeclareMathOperator{\Area}{Area}
\DeclareMathOperator{\Volume}{Volume}

%
% A few miscellaneous things specific to this document
%
\newcommand{\crossop}[1]{\crossprod{#1}{}}

% R2 vector.
\newcommand{\VectorTwo}[2]{
\begin{bmatrix}
 {#1} \\
 {#2}
\end{bmatrix}
}

\newcommand{\VectorN}[1]{
\begin{bmatrix}
{#1}_1 \\
{#1}_2 \\
\vdots \\
{#1}_N \\
\end{bmatrix}
}

\newcommand{\DETuvij}[4]{
\begin{vmatrix}
 {#1}_{#3} & {#1}_{#4} \\
 {#2}_{#3} & {#2}_{#4}
\end{vmatrix}
}

\newcommand{\DETuvwijk}[6]{
\begin{vmatrix}
 {#1}_{#4} & {#1}_{#5} & {#1}_{#6} \\
 {#2}_{#4} & {#2}_{#5} & {#2}_{#6} \\
 {#3}_{#4} & {#3}_{#5} & {#3}_{#6}
\end{vmatrix}
}

\newcommand{\DETuvwxijkl}[8]{
\begin{vmatrix}
 {#1}_{#5} & {#1}_{#6} & {#1}_{#7} & {#1}_{#8} \\
 {#2}_{#5} & {#2}_{#6} & {#2}_{#7} & {#2}_{#8} \\
 {#3}_{#5} & {#3}_{#6} & {#3}_{#7} & {#3}_{#8} \\
 {#4}_{#5} & {#4}_{#6} & {#4}_{#7} & {#4}_{#8} \\
\end{vmatrix}
}

%\newcommand{\DETuvwxyijklm}[10]{
%\begin{vmatrix}
% {#1}_{#6} & {#1}_{#7} & {#1}_{#8} & {#1}_{#9} & {#1}_{#10} \\
% {#2}_{#6} & {#2}_{#7} & {#2}_{#8} & {#2}_{#9} & {#2}_{#10} \\
% {#3}_{#6} & {#3}_{#7} & {#3}_{#8} & {#3}_{#9} & {#3}_{#10} \\
% {#4}_{#6} & {#4}_{#7} & {#4}_{#8} & {#4}_{#9} & {#4}_{#10} \\
% {#5}_{#6} & {#5}_{#7} & {#5}_{#8} & {#5}_{#9} & {#5}_{#10}
%\end{vmatrix}
%}

% R3 vector.
\newcommand{\VectorThree}[3]{
\begin{bmatrix}
 {#1} \\
 {#2} \\
 {#3}
\end{bmatrix}
}



\author{Peeter Joot}
\email{peeter.joot@gmail.com}


\chapter{stokesToTensor}
\label{chap:stokesToTensor}
%\useCCL
\date{July 21, 2009}
\blogpage{http://sites.google.com/site/peeterjoot/math2009/stokesToTensor.pdf}
\revisionInfo{$RCSfile: stokesNoTensor.tex,v $ Last $Revision: 1.3 $ $Date: 2009/07/22 13:10:23 $}

\beginArtWithToc

\section{Motivation}

Relying on pictorial means and a brute force ugly comparison of left and right hand sides, a verification of Stokes theorem for the vector and bivector cases was performed.  This was more of a confirmation than a derivation, and the technique fails the transition to the trivector case.  The trivector case is of particular interest in electromagnetism since it provides the four-vector divergence theorem equivalent via a duality transformation.

The fact that the pictorial means of defining the boundary surface doens't work well in four vector space is not the only unsatisfactory aspect of the previous treatment.  The fact that a coordinate expansion of the hypervolume element and hypersurface element was performed in the LHS and RHS comparisions was required is particularily ugly.  It is a lot of work and essentially has to be undone on the opposing side of the equation.  Comparing to previous attempts to come to terms with Stokes theorem in (\cite{PJStokes1}) and (\cite{PJStokes2}) this more recent attempt at least avoids the requirement for a tensor expansion of the vector or bivector.  It should be possible to build on this and minimize the amount of coordinate expansion required and go directly from the volume integral to the expression of the boundary surface.

\section{Notation and Setup.}

The desire is to relate the curl hypervolume integral to a hypersurface integral on the boundary

\begin{align}\label{eqn:stokes}
\int (\grad \wedge F) \cdot d^k x = \int F \cdot d^{k-1} x
\end{align}

In order to put meaning to this statement the volume and surface elements need to be properly defined.  In order that this be a scalar equation, the object $F$ in the integral is required to be of grade $k-1$, and $k \le n$ where $n$ is the dimension of the vector space that generates the object $F$.

\subsection{Reciprocal frames.}

As evident in equation (\ref{eqn:stokes}) a metric is required to define the dot product.  If an affine non-metric formulation
of Stokes theorem is possible it will not be attempted here.  A reciprocal basis pair will be utilized, defined by

\begin{align}
\gamma^\mu \cdot \gamma_\nu = {\delta^\mu}_\nu
\end{align}

Both of the sets $\{\gamma_\mu\}$ and $\{\gamma^\mu\}$ are taken to span the space, but are not required to be orthogonal.  The notation is consistent with the Dirac reciprocal basis, and there will not be anything in this treatment that prohibits the Minkowski metric signature required for such a relativistic space.

Vector decomposition in terms of coordinates follows by taking dot products.  We write

\begin{align}
x = x^\mu \gamma_\mu = x_\nu \gamma^\nu
\end{align}

\subsection{Gradient.}

When working with a non-orthonormal basis, use of the reciprocal frame can be utilized to express the gradient.

\begin{align}
\grad \equiv \gamma^\mu \partial_\mu \equiv \sum_\mu \gamma^\mu \PD{x^\mu}{}
\end{align}

This contains what may perhaps seem like an odd seeming mix of upper and lower indexes in this definition.  This is how the gradient is defined in \cite{doran2003gap}.  Although it is possible to accept this definition and work with it, this form can be justified by require of the gradient consistency with the the definition of directional derivative.  A definition of the directional derivative that works for single and multivector functions, in \R{3} and other more general spaces is

\begin{align}
a \cdot \grad F \equiv \lim_{\tau \rightarrow 0} \frac{F(x + a\tau) - F(x)}{\tau} = {\left.\PD{\tau}{F(x + a\tau)} \right\vert}_{\tau=0}
\end{align}

Taylor expanding about $\tau=0$ in terms of coordinates we have

\begin{align*}
{\left.\PD{\tau}{F(x + a\tau)} \right\vert}_{\tau=0}
&= a^\mu \PD{x^\mu}{F} \\
%&= a^\mu \partial_\mu F \\
&= (a^\nu \gamma_\nu) \cdot (\gamma^\mu \partial_\mu) F \\
&= a \cdot \grad F \quad\quad\quad\square
\end{align*}

The lower index representation of the vector coordinates could also have been used, so using the directional derivative to imply a definition of the gradient, we have an additional alternate representation of the gradient

\begin{align}
\grad \equiv \gamma_\mu \partial^\mu \equiv \sum_\mu \gamma_\mu \PD{x_\mu}{}
\end{align}

\subsection{Volume element}
We define the hypervolume in terms of parameterized vector displacements $x = x(a_1, a_2, ... a_k)$.  For the vector x we can form a pseudoscalar for the subspace spanned by this parameterization by wedging the displacements in each of the directions defined by variation of the parameters.  Let

\begin{align*}
dx_1 & = \PD{a_1}{x} da_1 = \gamma_\mu \PD{a_1}{x^\mu} da_1 \\
dx_2 & = \PD{a_2}{x} da_2 = \gamma_\mu \PD{a_2}{x^\mu} da_2 \\
\hdots \\
dx_k & = \PD{a_k}{x} da_k = \gamma_\mu \PD{a_k}{x^\mu} da_k \\
\end{align*}

The hypervolume element is then
\begin{align}
d^k x \equiv dx_1 \wedge dx_2 \cdots dx_k
\end{align}

This can be expanded explicitly in coordinates

\begin{align*}
d^k x 
&= da_1 da_2 \cdots da_k 
\left(
\PD{a_1}{x^{\alpha_1}} 
\PD{a_2}{x^{\alpha_2}} 
\cdots
\PD{a_k}{x^{\alpha_k}} 
\right)
( \gamma_{\alpha_1} \wedge \gamma_{\alpha_2} \wedge \cdots \wedge \gamma_{\alpha_k} ) \\
&= 
da_1 da_2 \cdots da_k 
\sum_{\alpha_1 < \alpha_2 < \cdots < \alpha_k}
\PD{(a_1, a_2, \cdots, a_k)}{(x^{\alpha_1}, x^{\alpha_2}, \cdots, x^{\alpha_k})}
( \gamma_{\alpha_1} \wedge \gamma_{\alpha_2} \wedge \cdots \wedge \gamma_{\alpha_k} ) \\
&= 
\inv{k!}
da_1 da_2 \cdots da_k
\PD{(a_1, a_2, \cdots, a_k)}{(x^{\alpha_1}, x^{\alpha_2}, \cdots, x^{\alpha_k})}
( \gamma_{\alpha_1} \wedge \gamma_{\alpha_2} \wedge \cdots \wedge \gamma_{\alpha_k} ) \\
\end{align*}

However, we don't wish to use these Jacobians since we've seen how messy this makes things.  Even when the volume element for the subspace was three dimensional, this was starting to get untractable.

\section{Guts}

\EndArticle
