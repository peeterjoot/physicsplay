%
% Copyright � 2012 Peeter Joot.  All Rights Reserved.
% Licenced as described in the file LICENSE under the root directory of this GIT repository.
%

% 
% 
%\documentclass[]{eliblog}

\usepackage{amsmath}
\usepackage{mathpazo}

%
% shorthand for bold symbols, convenient for vectors and matrices
%
\newcommand{\Ba}[0]{\mathbf{a}}
\newcommand{\Bb}[0]{\mathbf{b}}
\newcommand{\Bc}[0]{\mathbf{c}}
\newcommand{\Bd}[0]{\mathbf{d}}
\newcommand{\Be}[0]{\mathbf{e}}
\newcommand{\Bf}[0]{\mathbf{f}}
\newcommand{\Bg}[0]{\mathbf{g}}
\newcommand{\Bh}[0]{\mathbf{h}}
\newcommand{\Bi}[0]{\mathbf{i}}
\newcommand{\Bj}[0]{\mathbf{j}}
\newcommand{\Bk}[0]{\mathbf{k}}
\newcommand{\Bl}[0]{\mathbf{l}}
\newcommand{\Bm}[0]{\mathbf{m}}
\newcommand{\Bn}[0]{\mathbf{n}}
\newcommand{\Bo}[0]{\mathbf{o}}
\newcommand{\Bp}[0]{\mathbf{p}}
\newcommand{\Bq}[0]{\mathbf{q}}
\newcommand{\Br}[0]{\mathbf{r}}
\newcommand{\Bs}[0]{\mathbf{s}}
\newcommand{\Bt}[0]{\mathbf{t}}
\newcommand{\Bu}[0]{\mathbf{u}}
\newcommand{\Bv}[0]{\mathbf{v}}
\newcommand{\Bw}[0]{\mathbf{w}}
\newcommand{\Bx}[0]{\mathbf{x}}
\newcommand{\By}[0]{\mathbf{y}}
\newcommand{\Bz}[0]{\mathbf{z}}
\newcommand{\BA}[0]{\mathbf{A}}
\newcommand{\BB}[0]{\mathbf{B}}
\newcommand{\BC}[0]{\mathbf{C}}
\newcommand{\BD}[0]{\mathbf{D}}
\newcommand{\BE}[0]{\mathbf{E}}
\newcommand{\BF}[0]{\mathbf{F}}
\newcommand{\BG}[0]{\mathbf{G}}
\newcommand{\BH}[0]{\mathbf{H}}
\newcommand{\BI}[0]{\mathbf{I}}
\newcommand{\BJ}[0]{\mathbf{J}}
\newcommand{\BK}[0]{\mathbf{K}}
\newcommand{\BL}[0]{\mathbf{L}}
\newcommand{\BM}[0]{\mathbf{M}}
\newcommand{\BN}[0]{\mathbf{N}}
\newcommand{\BO}[0]{\mathbf{O}}
\newcommand{\BP}[0]{\mathbf{P}}
\newcommand{\BQ}[0]{\mathbf{Q}}
\newcommand{\BR}[0]{\mathbf{R}}
\newcommand{\BS}[0]{\mathbf{S}}
\newcommand{\BT}[0]{\mathbf{T}}
\newcommand{\BU}[0]{\mathbf{U}}
\newcommand{\BV}[0]{\mathbf{V}}
\newcommand{\BW}[0]{\mathbf{W}}
\newcommand{\BX}[0]{\mathbf{X}}
\newcommand{\BY}[0]{\mathbf{Y}}
\newcommand{\BZ}[0]{\mathbf{Z}}

\newcommand{\Bzero}[0]{\mathbf{0}}
\newcommand{\Btheta}[0]{\boldsymbol{\theta}}
\newcommand{\Btau}[0]{\boldsymbol{\tau}}
\newcommand{\Bomega}[0]{\boldsymbol{\omega}}

%
% shorthand for unit vectors
%
\newcommand{\acap}[0]{\hat{\Ba}}
\newcommand{\bcap}[0]{\hat{\Bb}}
\newcommand{\ccap}[0]{\hat{\Bc}}
\newcommand{\dcap}[0]{\hat{\Bd}}
\newcommand{\ecap}[0]{\hat{\Be}}
\newcommand{\fcap}[0]{\hat{\Bf}}
\newcommand{\gcap}[0]{\hat{\Bg}}
\newcommand{\hcap}[0]{\hat{\Bh}}
\newcommand{\icap}[0]{\hat{\Bi}}
\newcommand{\jcap}[0]{\hat{\Bj}}
\newcommand{\kcap}[0]{\hat{\Bk}}
\newcommand{\lcap}[0]{\hat{\Bl}}
\newcommand{\mcap}[0]{\hat{\Bm}}
\newcommand{\ncap}[0]{\hat{\Bn}}
\newcommand{\ocap}[0]{\hat{\Bo}}
\newcommand{\pcap}[0]{\hat{\Bp}}
\newcommand{\qcap}[0]{\hat{\Bq}}
\newcommand{\rcap}[0]{\hat{\Br}}
\newcommand{\scap}[0]{\hat{\Bs}}
\newcommand{\tcap}[0]{\hat{\Bt}}
\newcommand{\ucap}[0]{\hat{\Bu}}
\newcommand{\vcap}[0]{\hat{\Bv}}
\newcommand{\wcap}[0]{\hat{\Bw}}
\newcommand{\xcap}[0]{\hat{\Bx}}
\newcommand{\ycap}[0]{\hat{\By}}
\newcommand{\zcap}[0]{\hat{\Bz}}
\newcommand{\thetacap}[0]{\hat{\Btheta}}

%
% to write R^n and C^n in a distinguishable fashion.  Perhaps change this
% to the double lined characters upon figuring out how to do so.
%
\newcommand{\C}[1]{$\mathbb{C}^{#1}$}
\newcommand{\R}[1]{$\mathbb{R}^{#1}$}

%
% various generally useful helpers
%

% derivative of #1 wrt. #2:
\newcommand{\D}[2] {\frac {d#2} {d#1}}

\newcommand{\inv}[1]{\frac{1}{#1}}
\newcommand{\cross}[0]{\times}

\newcommand{\abs}[1]{\lvert{#1}\rvert}
\newcommand{\norm}[1]{\lVert{#1}\rVert}
\newcommand{\innerprod}[2]{\langle{#1}, {#2}\rangle}
\newcommand{\dotprod}[2]{{#1} \cdot {#2}}
\newcommand{\bdotprod}[2]{\left({#1} \cdot {#2}\right)}
\newcommand{\crossprod}[2]{{#1} \cross {#2}}
\newcommand{\tripleprod}[3]{\dotprod{\left(\crossprod{#1}{#2}\right)}{#3}}

\DeclareMathOperator{\Proj}{Proj}
\DeclareMathOperator{\Span}{span}
\DeclareMathOperator{\Sgn}{sgn}
\DeclareMathOperator{\Area}{Area}
\DeclareMathOperator{\Volume}{Volume}

%
% A few miscellaneous things specific to this document
%
\newcommand{\crossop}[1]{\crossprod{#1}{}}

% R2 vector.
\newcommand{\VectorTwo}[2]{
\begin{bmatrix}
 {#1} \\
 {#2}
\end{bmatrix}
}

\newcommand{\VectorN}[1]{
\begin{bmatrix}
{#1}_1 \\
{#1}_2 \\
\vdots \\
{#1}_N \\
\end{bmatrix}
}

\newcommand{\DETuvij}[4]{
\begin{vmatrix}
 {#1}_{#3} & {#1}_{#4} \\
 {#2}_{#3} & {#2}_{#4}
\end{vmatrix}
}

\newcommand{\DETuvwijk}[6]{
\begin{vmatrix}
 {#1}_{#4} & {#1}_{#5} & {#1}_{#6} \\
 {#2}_{#4} & {#2}_{#5} & {#2}_{#6} \\
 {#3}_{#4} & {#3}_{#5} & {#3}_{#6}
\end{vmatrix}
}

\newcommand{\DETuvwxijkl}[8]{
\begin{vmatrix}
 {#1}_{#5} & {#1}_{#6} & {#1}_{#7} & {#1}_{#8} \\
 {#2}_{#5} & {#2}_{#6} & {#2}_{#7} & {#2}_{#8} \\
 {#3}_{#5} & {#3}_{#6} & {#3}_{#7} & {#3}_{#8} \\
 {#4}_{#5} & {#4}_{#6} & {#4}_{#7} & {#4}_{#8} \\
\end{vmatrix}
}

%\newcommand{\DETuvwxyijklm}[10]{
%\begin{vmatrix}
% {#1}_{#6} & {#1}_{#7} & {#1}_{#8} & {#1}_{#9} & {#1}_{#10} \\
% {#2}_{#6} & {#2}_{#7} & {#2}_{#8} & {#2}_{#9} & {#2}_{#10} \\
% {#3}_{#6} & {#3}_{#7} & {#3}_{#8} & {#3}_{#9} & {#3}_{#10} \\
% {#4}_{#6} & {#4}_{#7} & {#4}_{#8} & {#4}_{#9} & {#4}_{#10} \\
% {#5}_{#6} & {#5}_{#7} & {#5}_{#8} & {#5}_{#9} & {#5}_{#10}
%\end{vmatrix}
%}

% R3 vector.
\newcommand{\VectorThree}[3]{
\begin{bmatrix}
 {#1} \\
 {#2} \\
 {#3}
\end{bmatrix}
}



\author{Peeter Joot}
\email{peeter.joot@gmail.com}


\chapter{Space time algebra solutions of the Maxwell equation for discrete frequencies}
\label{chap:maxwellVacuum}
%\date{July 2, 2009 $RCSfile: maxwellVacuum.tex,v $ Last $Revision: 1.8 $ $Date: 2009/08/06 09:35:17 $}
%%\date{July 2, 2009}
%%\revisionInfo{$RCSfile: maxwellVacuum.tex,v $ Last $Revision: 1.8 $ $Date: 2009/08/06 09:35:17 $}
%\blogpage{http://sites.google.com/site/peeterjoot/math2009/maxwellVacuum.pdf}

\beginArtWithToc

\section{Motivation}

How to obtain solutions to Maxwell's equations in vacuum is well known.  The aim here is to explore the same problem starting with the Geometric Algebra (GA) formalism (\citep{doran2003gap}) of the Maxwell equation.

\begin{align}\label{eqn:maxwellVacuum:maxwell}
\grad F &= J/\epsilon_0 c \\
F &= \grad \wedge A = \BE + i c \BB
\end{align}

A Fourier transformation attack on the equation should be possible, so let us see what falls out doing so.

\subsection{Fourier problem}

Picking an observer bias for the gradient by premultiplying with $\gamma_0$ the vacuum equation for light can therefore also be written as

\begin{align*}
0
&= \gamma_0 \grad F \\
&= \gamma_0 (\gamma^0 \partial_0 + \gamma^k \partial_k) F \\
&= (\partial_0 - \gamma^k \gamma_0 \partial_k) F \\
&= (\partial_0 + \sigma^k \partial_k) F \\
&= \left(\inv{c}\partial_t + \spacegrad \right) F \\
\end{align*}

A Fourier transformation of this equation produces

\begin{align*}
0 &= \inv{c} \frac{\partial F}{\partial t}(\Bk,t) + \inv{(\sqrt{2\pi})^3} \int \sigma^m \partial_m F(\Bx,t) e^{-i \Bk \cdot \Bx} d^3 x
\end{align*}

and with a single integration by parts one has

\begin{align*}
0
&= \inv{c} \frac{\partial F}{\partial t}(\Bk,t) - \inv{(\sqrt{2\pi})^3} \int \sigma^m F(\Bx,t) (-i k_m) e^{-i \Bk \cdot \Bx} d^3 x \\
&= \inv{c} \frac{\partial F}{\partial t}(\Bk,t) + \inv{(\sqrt{2\pi})^3} \int \Bk F(\Bx,t) i e^{-i \Bk \cdot \Bx} d^3 x \\
&= \inv{c} \frac{\partial F}{\partial t}(\Bk,t) + i \Bk \hat{F}(\Bk,t)
\end{align*}

The flexibility to employ the pseudoscalar as the imaginary $i = \gamma_0 \gamma_1 \gamma_2 \gamma_3$ has been employed above, so it should be noted that pseudoscalar commutation with Dirac bivectors was implied above, but also that we do not have the flexibility to commute $\Bk$ with $F$.

Having done this, the problem to solve is now Maxwell's vacuum equation in the frequency domain

\begin{align*}
\frac{\partial F}{\partial t}(\Bk,t) = -i c \Bk \hat{F}(\Bk,t)
\end{align*}

Introducing an angular frequency (spatial) bivector, and its vector dual

\begin{align}
\Omega &= -i c \Bk \\
\Bomega &= c \Bk
\end{align}

This becomes

\begin{align}\label{eqn:maxwellVacuum:MaxwellFreq}
\hat{F}' = \Omega F
\end{align}

With solution

\begin{align}
\hat{F} = e^{\Omega t} \hat{F}(\Bk,0)
\end{align}

Differentiation with respect to time verifies that the ordering of the terms is correct and this does in fact solve (\ref{eqn:maxwellVacuum:MaxwellFreq}).  This is something we have to be careful of due to the possibility of non-commuting variables.

Back substitution into the inverse transform now supplies the time evolution of the field given the initial time specification

\begin{align*}
F(\Bx,t)
&= \inv{(\sqrt{2\pi})^3} \int e^{\Omega t} \hat{F}(\Bk,0) e^{i \Bk \cdot \Bx} d^3 k \\
&= \inv{(2\pi)^3} \int e^{\Omega t} \left( \int {F}(\Bx',0) e^{-i \Bk \cdot \Bx'} d^3 x' \right) e^{i \Bk \cdot \Bx} d^3 k
\end{align*}

Observe that Pseudoscalar exponentials commute with the field because $i$ commutes with spatial vectors and itself

\begin{align*}
F e^{i\theta}
&= (\BE + i c \BB) (C + iS) \\
&=
C (\BE + i c \BB)
+ S (\BE + i c \BB) i  \\
&=
C (\BE + i c \BB)
+ S i (\BE + i c \BB) \\
&=
e^{i\theta} F
\end{align*}

This allows the specifics of the initial time conditions to be suppressed

\begin{align}
F(\Bx,t) &= \int d^3 k e^{\Omega t} e^{i \Bk \cdot \Bx} \int \inv{(2\pi)^3} {F}(\Bx',0) e^{-i \Bk \cdot \Bx'}  d^3 x'
\end{align}

The interior integral has the job of a weighting function over plane wave solutions, and this can be made explicit writing

\begin{align}
D(\Bk) &= \inv{(2\pi)^3} \int {F}(\Bx',0) e^{-i \Bk \cdot \Bx'}  d^3 x' \\
F(\Bx,t) &= \int e^{\Omega t} e^{i \Bk \cdot \Bx} D(\Bk) d^3 k
\end{align}

Many assumptions have been made here, not the least of which was a requirement for the Fourier transform of a bivector valued function to be meaningful, and have an inverse.  It is therefore reasonable to verify that this weighted plane wave result is in fact a solution to the original Maxwell vacuum equation.  Differentiation verifies that things are okay so far

\begin{align*}
\gamma_0 \grad F(\Bx,t)
&=
\left(\inv{c}\partial_t + \spacegrad \right)\int e^{\Omega t} e^{i \Bk \cdot \Bx} D(\Bk) d^3 k \\
&=
\int \left(\inv{c}\Omega e^{\Omega t} + \sigma^m e^{\Omega t} i k_m \right) e^{i \Bk \cdot \Bx} D(\Bk) d^3 k \\
&=
\int \left(\inv{c}(-i \Bk c) + i \Bk \right) e^{\Omega t} e^{i \Bk \cdot \Bx} D(\Bk) d^3 k \\
&= 0 \quad\quad\quad\square
\end{align*}

\subsection{Discretizing and grade restrictions}

The fact that it the integral has zero gradient does not mean that it is a bivector, so there must also be at least also be restrictions on the grades of $D(\Bk)$.

To simplify discussion, let us discretize the integral writing

\begin{align*}
D(\Bk') = D_\Bk \delta^3 (\Bk - \Bk')
\end{align*}

So we have

\begin{align*}
F(\Bx,t)
&= \int e^{\Omega t} e^{i \Bk' \cdot \Bx} D(\Bk') d^3 k' \\
&= \int e^{\Omega t} e^{i \Bk' \cdot \Bx} D_\Bk \delta^3(\Bk - \Bk') d^3 k' \\
\end{align*}

This produces something planewave-ish

\begin{align}\label{eqn:maxwellVacuum:planewaveish}
F(\Bx,t) &= e^{\Omega t} e^{i \Bk \cdot \Bx} D_\Bk
\end{align}

Observe that at $t=0$ we have

\begin{align*}
F(\Bx,0)
&= e^{i \Bk \cdot \Bx} D_\Bk  \\
&= (\cos (\Bk \cdot \Bx) + i \sin(\Bk \cdot \Bx)) D_\Bk  \\
\end{align*}

There is therefore a requirement for $D_\Bk$ to be either a spatial vector or its dual, a spatial bivector.  For example taking $D_k$ to be a spatial vector we can then identify the electric and magnetic components of the field

\begin{align*}
\BE(\Bx,0) &= \cos (\Bk \cdot \Bx) D_\Bk \\
c \BB(\Bx,0) &= \sin (\Bk \cdot \Bx) D_\Bk
\end{align*}

and if $D_k$ is taken to be a spatial bivector, this pair of identifications would be inverted.

Considering (\ref{eqn:maxwellVacuum:planewaveish}) at $\Bx=0$, we have

\begin{align*}
F(0, t)
&= e^{\Omega t} D_\Bk \\
&= (\cos(\Abs{\Omega} t) + \hat{\Omega} \sin(\Abs{\Omega} t)) D_\Bk \\
&= (\cos(\Abs{\Omega} t) - i \hat{\Bk} \sin(\Abs{\Omega} t)) D_\Bk \\
\end{align*}

If $D_\Bk$ is first assumed to be a spatial vector, then $F$ would have a pseudoscalar component if $D_\Bk$ has any component parallel to $\hat{\Bk}$.

\begin{align}\label{eqn:maxwellVacuum:commutationRequirementVector}
D_\Bk \in \span\{\sigma^m\} \implies D_\Bk \cdot \hat{\Bk} = 0
\end{align}
\begin{align}\label{eqn:maxwellVacuum:commutationRequirementBiVector}
D_\Bk \in \span\{\sigma^a \wedge \sigma^b\} \implies D_\Bk \cdot (i\hat{\Bk}) = 0
\end{align}

Since we can convert between the spatial vector and bivector cases using a duality transformation, there may not appear to be any loss of generality imposing a spatial vector restriction on $D_\Bk$, at least in this current free case.  However, an attempt to do so leads to trouble.  In particular, this leads to collinear electric and magnetic fields, and thus the odd seeming condition where the field energy density is non-zero but the field momentum density (Poynting vector $\BP \propto \BE \cross \BB$) is zero.  In retrospect being forced down the path of including both grades is not unreasonable, especially since this gives $D_\Bk$ precisely the form of the field itself $F = \BE + i c \BB$.

\section{Electric and Magnetic field split}

With the basic form of the Maxwell vacuum solution determined, we are now ready to start extracting information from the solution and making comparisons with the more familiar vector form.  To start doing the phasor form of the fundamental solution can be expanded explicitly in terms of two arbitrary spatial parametrization vectors $\BE_\Bk$ and $\BB_\Bk$.

\begin{align}\label{eqn:maxwellVacuum:phasor}
F &= e^{-i\Bomega t} e^{i \Bk \cdot \Bx} (\BE_\Bk + i c \BB_\Bk)
\end{align}

Whether these parametrization vectors have any relation to electric and magnetic fields respectively will have to be determined, but making that assumption for now to label these uniquely does not seem unreasonable.

From (\ref{eqn:maxwellVacuum:phasor}) we can compute the electric and magnetic fields by the conjugate relations (\ref{eqn:maxwellVacuum:conjuagateSplit}).  Our conjugate is

\begin{align*}
F^\dagger
&= (\BE_\Bk - i c \BB_\Bk) e^{-i \Bk \cdot \Bx} e^{i\Bomega t} \\
&=
e^{-i\Bomega t}
e^{-i \Bk \cdot \Bx}
(\BE_\Bk - i c \BB_\Bk)
\end{align*}

Thus for the electric field

\begin{align*}
F + F^\dagger
&=
e^{-i\Bomega t} \left(
 e^{i \Bk \cdot \Bx} (\BE_\Bk + i c \BB_\Bk)
+e^{-i \Bk \cdot \Bx} (\BE_\Bk - i c \BB_\Bk)
\right) \\
&=
e^{-i\Bomega t} \left(
 2 \cos(\Bk \cdot \Bx) \BE_\Bk
+ i c (2 i) \sin(\Bk \cdot \Bx) \BB_\Bk
\right) \\
&=
2 \cos(\omega t) \left(
 \cos(\Bk \cdot \Bx) \BE_\Bk
- c \sin(\Bk \cdot \Bx) \BB_\Bk
\right) \\
&+ 2
\sin(\omega t)
\kcap \cross
\left(
 \cos(\Bk \cdot \Bx) \BE_\Bk
- c \sin(\Bk \cdot \Bx) \BB_\Bk
\right) \\
\end{align*}

So for the electric field $\BE = \inv{2}(F + F^\dagger)$ we have

\begin{align}\label{eqn:maxwellVacuum:electricSplit}
\BE &=
\left( \cos(\omega t) + \sin(\omega t) \kcap \cross \right)
\left(
 \cos(\Bk \cdot \Bx) \BE_\Bk
- c \sin(\Bk \cdot \Bx) \BB_\Bk
\right)
\end{align}

Similarly for the magnetic field we have
\begin{align*}
F - F^\dagger
&=
e^{-i\Bomega t} \left(
 e^{i \Bk \cdot \Bx} (\BE_\Bk + i c \BB_\Bk)
-e^{-i \Bk \cdot \Bx} (\BE_\Bk - i c \BB_\Bk)
\right) \\
&=
e^{-i\Bomega t} \left(
 2 i \sin(\Bk \cdot \Bx) \BE_\Bk
+ 2 i c \cos(\Bk \cdot \Bx) \BB_\Bk
\right) \\
\end{align*}

This gives $c \BB = \inv{2i}(F - F^\dagger)$ we have

\begin{align}\label{eqn:maxwellVacuum:magneticSplit}
c \BB &=
\left( \cos(\omega t) + \sin(\omega t) \kcap \cross \right)
\left(
 \sin(\Bk \cdot \Bx) \BE_\Bk
+ c \cos(\Bk \cdot \Bx) \BB_\Bk
\right)
\end{align}

Observe that the action of the time dependent phasor has been expressed, somewhat abusively and sneakily, in a scalar plus cross product operator form.  The end result, when applied to a vector perpendicular to $\kcap$, is still a vector

\begin{align*}
e^{-i\Bomega t} \Ba
&=
\left( \cos(\omega t) + \sin(\omega t) \kcap \cross \right) \Ba
\end{align*}

Also observe that the Hermitian conjugate split of the total field bivector $F$ produces vectors $\BE$ and $\BB$, not phasors.  There is no further need to take real or imaginary parts nor treat the phasor (\ref{eqn:maxwellVacuum:phasor}) as an artificial mathematical construct used for convenience only.

With $\BE \cdot \kcap = \BB \cdot \kcap = 0$, we have here what Jackson (\citep{jackson1975cew}, ch7), calls a transverse wave.

\subsection{Polar Form}

Suppose an explicit polar form is introduced for the plane vectors $\BE_\Bk$, and $\BB_\Bk$.  Let

\begin{align*}
\BE_\Bk &= E {\hat{\BE}_k} \\
\BB_\Bk &= B {\hat{\BE}_k} e^{i\kcap \theta}
\end{align*}

Then for the field we have

\begin{align}\label{eqn:maxwellVacuum:phasorPolar}
F &= e^{-i\Bomega t} e^{i \Bk \cdot \Bx} (E + i c B e^{-i\kcap \theta}) \hat{\BE}_k
\end{align}

For the conjugate
\begin{align*}
F^\dagger
&=
\hat{\BE}_k
(E - i c B e^{i\kcap \theta})
e^{-i \Bk \cdot \Bx}
e^{i\Bomega t} \\
&=
e^{-i\Bomega t} e^{-i \Bk \cdot \Bx} (E - i c B e^{-i\kcap \theta}) \hat{\BE}_k
\end{align*}

So, in the polar form we have for the electric, and magnetic fields

\begin{align}\label{eqn:maxwellVacuum:fieldsPolar}
\BE &= e^{-i\Bomega t} (E \cos(\Bk \cdot \Bx) - c B \sin(\Bk \cdot \Bx) e^{-i \kcap\theta}) \hat{\BE}_k \\
c \BB &= e^{-i\Bomega t} (E \sin(\Bk \cdot \Bx) + c B \cos(\Bk \cdot \Bx) e^{-i \kcap\theta}) \hat{\BE}_k
\end{align}

Observe when $\theta$ is an integer multiple of $\pi$, $\BE$ and $\BB$ are colinear, having the zero Poynting vector mentioned previously.
Now, for arbitrary $\theta$ it does not appear that there is any inherent perpendicularity between the electric and magnetic fields.  It is common
to read of light being the propagation of perpendicular fields, both perpendicular to the propagation direction.  We have perpendicularity to the
propagation direction by virtue of requiring that the field be a (Dirac) bivector, but it does not look like the solution requires any inherent perpendicularity for the field components.  It appears that a normal triplet of field vectors and propagation directions must actually be a special case.
Intuition says that this freedom to pick different magnitude or angle between $\BE_\Bk$ and $\BB_\Bk$ in the plane perpendicular to the transmission direction may correspond to different mixes of linear, circular, and elliptic polarization, but this has to be confirmed.

Working towards confirming (or disproving) this intuition, lets find the constraints on the fields that lead to normal electric and magnetic fields.  This should follow by taking dot products

\begin{align*}
\BE \cdot \BB c
&=
%\gpgradezero{
\left\langle{
e^{-i\Bomega t} (E \cos(\Bk \cdot \Bx) - c B \sin(\Bk \cdot \Bx) e^{-i \kcap\theta}) \hat{\BE}_k
\hat{\BE}_k
e^{i\Bomega t} (E \sin(\Bk \cdot \Bx) + c B \cos(\Bk \cdot \Bx) e^{i \kcap\theta})
%} \\
}\right\rangle \\
&=
%\gpgradezero{
\left\langle{
(E \cos(\Bk \cdot \Bx) - c B \sin(\Bk \cdot \Bx) e^{-i \kcap\theta})
(E \sin(\Bk \cdot \Bx) + c B \cos(\Bk \cdot \Bx) e^{i \kcap\theta})
%} \\
}\right\rangle \\
&=
(E^2 - c^2 B^2) \cos(\Bk \cdot \Bx) \sin(\Bk \cdot \Bx)
+ c E B
%\gpgradezero{
\left\langle{
\cos^2(\Bk \cdot \Bx) e^{i \kcap \theta}
-\sin^2(\Bk \cdot \Bx) e^{-i \kcap \theta}
%} \\
}\right\rangle \\
&=
(E^2 - c^2 B^2) \cos(\Bk \cdot \Bx) \sin(\Bk \cdot \Bx)
+ c E B \cos(\theta) ( \cos^2(\Bk \cdot \Bx) -\sin^2(\Bk \cdot \Bx) ) \\
&=
(E^2 - c^2 B^2) \cos(\Bk \cdot \Bx) \sin(\Bk \cdot \Bx)
+ c E B \cos(\theta) ( \cos^2(\Bk \cdot \Bx) -\sin^2(\Bk \cdot \Bx) ) \\
&=
\inv{2} (E^2 - c^2 B^2) \sin(2 \Bk \cdot \Bx)
+ c E B \cos(\theta) \cos(2 \Bk \cdot \Bx) \\
\end{align*}

The only way this can be zero for any $\Bx$ is if the left and right terms are separately zero, which means

\begin{align*}
\Abs{\BE_k} &= c \Abs{\BB_k} \\
\theta &= \frac{\pi}{2} + n \pi
\end{align*}

This simplifies the phasor considerably, leaving

\begin{align*}
E + i c B e^{-i\kcap \theta}
&=
E(1 + i (\mp i\kcap )) \\
&=
E(1 \pm \kcap)
\end{align*}

So the field is just

\begin{align}
F = e^{-i \Bomega t} e^{i \Bk \cdot \Bx} (1 \pm \kcap) \BE_\Bk
\end{align}

Using this, and some regrouping, a calculation of the field components yields

\begin{align}
\BE &= e^{i \kcap( \pm \Bk \cdot \Bx -\omega t )} \BE_\Bk \\
c \BB &= \pm e^{i \kcap( \pm \Bk \cdot \Bx -\omega t )} i \Bk \BE_\Bk
\end{align}

Observe that $i\Bk$ rotates any vector in the plane perpendicular to $\kcap$ by 90 degrees, so we have here $c \BB = \pm \kcap \cross \BE$.  This is consistent with the transverse wave restriction (7.11) of Jackson (\citep{jackson1975cew}), where he says, the ``curl equations provide a further restriction, namely'', and 

\begin{align}\label{eqn:fooX}
\mathcal{B} = \sqrt{\mu\epsilon} \Bn \cross \mathcal{E}
\end{align}

He works in explicit complex phasor form and CGS units.  He also allows $\Bn$ to be complex.  With real $\Bk$, and no $\BE \cdot \BB = 0$ constraint, it appears that we cannot have such a simple coupling between the field components?  Is it possible that allowing $\Bk$ to be complex allows this cross product coupling constraint on the fields without the explicit 90 degree phase difference between the electric and magnetic fields?

\section{Energy and momentum for the phasor}

To calculate the field energy density we can work with the two fields of equations (\ref{eqn:maxwellVacuum:fieldsPolar}), or work with the phasor (\ref{eqn:maxwellVacuum:phasor}) directly.  From the phasor and the energy-momentum four vector (\ref{eqn:maxwellVacuum:emFourVect}) we have for the energy density 

\begin{align*}
U &= T(\gamma_0) \cdot \gamma_0 \\
&= \frac{-\epsilon_0}{2}\gpgradezero{ F \gamma_0 F \gamma_0 } \\
&= \frac{-\epsilon_0}{2}
%\gpgradezero{ 
\left\langle{
e^{-i\Bomega t} e^{i \Bk \cdot \Bx} (\BE_\Bk + i c \BB_\Bk) \gamma_0 e^{-i\Bomega t} e^{i \Bk \cdot \Bx} (\BE_\Bk + i c \BB_\Bk) \gamma_0 
%} \\
}\right\rangle \\
&= \frac{-\epsilon_0}{2}
%\gpgradezero{ 
\left\langle{
e^{-i\Bomega t} e^{i \Bk \cdot \Bx} (\BE_\Bk + i c \BB_\Bk) (\gamma_0)^2 e^{-i\Bomega t} e^{-i \Bk \cdot \Bx} (-\BE_\Bk + i c \BB_\Bk) 
%} \\
}\right\rangle \\
&= \frac{-\epsilon_0}{2}
%\gpgradezero{ 
\left\langle{
e^{-i\Bomega t} (\BE_\Bk + i c \BB_\Bk) e^{-i\Bomega t} (-\BE_\Bk + i c \BB_\Bk) 
%} \\
}\right\rangle \\
&= \frac{\epsilon_0}{2}\gpgradezero{ (\BE_\Bk + i c \BB_\Bk) (\BE_\Bk - i c \BB_\Bk) } \\
&= 
\frac{\epsilon_0}{2} \left( (\BE_k)^2 + c^2 (\BB_\Bk)^2\right) + {c \epsilon_0} \gpgradezero{ i \BE_\Bk \wedge \BB_\Bk } \\
&= 
\frac{\epsilon_0}{2} \left( (\BE_k)^2 + c^2 (\BB_\Bk)^2\right) + {c \epsilon_0} \gpgradezero{ \BB_\Bk \cross \BE_\Bk } \\
\end{align*}

Quite anticlimactically we have for the energy the sum of the energies associated with the parametrization constants, lending some justification for the initial choice to label these as electric and magnetic fields

\begin{align}
U = \frac{\epsilon_0}{2} \left( (\BE_k)^2 + c^2 (\BB_\Bk)^2\right)
\end{align}

For the momentum, we want the difference of $F F^\dagger$, and $F^\dagger F$

\begin{align*}
F F^\dagger 
&= e^{-i\Bomega t} e^{i \Bk \cdot \Bx} (\BE_\Bk + i c \BB_\Bk) (\BE_\Bk - i c \BB_\Bk) e^{-i \Bk \cdot \Bx} e^{i\Bomega t}  \\
&= (\BE_\Bk + i c \BB_\Bk) (\BE_\Bk - i c \BB_\Bk) \\
&= (\BE_\Bk)^2 + c^2 (\BB_\Bk)^2 - 2 c \BB_\Bk \cross \BE_\Bk
\end{align*}

\begin{align*}
F F^\dagger 
&= (\BE_\Bk - i c \BB_\Bk) e^{-i \Bk \cdot \Bx} e^{i\Bomega t}  e^{-i\Bomega t} e^{i \Bk \cdot \Bx} (\BE_\Bk + i c \BB_\Bk)  \\
&= (\BE_\Bk - i c \BB_\Bk) (\BE_\Bk + i c \BB_\Bk) \\
&= (\BE_\Bk)^2 + c^2 (\BB_\Bk)^2 + 2 c \BB_\Bk \cross \BE_\Bk
\end{align*}

So we have for the momentum, also anticlimactically

\begin{align}
\BP = \inv{c} T(\gamma_0) \wedge \gamma_0 = \epsilon_0 \BE_\Bk \cross \BB_\Bk 
\end{align}

\section{Followup}

Well, that is enough for one day.  Understanding how to express circular and elliptic polarization is one of the logical next steps.  I seem to recall from Susskind's QM lectures that these can be considered superpositions of linearly polarized waves, so examining a sum of two co-directionally propagating fields would seem to be in order.  Also there ought to be a more natural way to express the perpendicularity requirement for the field and the propagation direction.  The fact that the field components and propagation direction when all multiplied is proportional to the spatial pseudoscalar can probably be utilized to tidy this up and also produce a form that allows for simpler summation of fields in different propagation directions.  It also seems reasonable to consider a planar Fourier decomposition of the field components, perhaps framing the superposition of multiple fields in that context.

Reconsilation of the Jackson's (7.11) restriction for perpendicularity of the fields noted above has not been done.  If such a restriction is required with an explicit dot and cross product split of Maxwell's equation, it would make sense to also have this required of a GA based solution.  Is this just a conquense of the differences between his explicit phasor representation, and this geometric approach where the phasor has an explicit representation in terms of the transverse plane?

\section{Appendix.  Background details}

\subsection{Conjugate split}

The Hermitian conjugate is defined as

\begin{align}
A^\dagger = \gamma_0 \tilde{A} \gamma_0
\end{align}

The conjugate action on a multivector product is straightforward to calculate

\begin{align*}
(A B)^\dagger
&= \gamma_0 (A B)^{\tilde{}} \gamma_0 \\
&= \gamma_0 \tilde{B} \tilde{A} \gamma_0 \\
&= \gamma_0 \tilde{B} {\gamma_0}^2 \tilde{A} \gamma_0 \\
&= B^\dagger A^\dagger
\end{align*}

For a spatial vector Hermitian conjugation leaves the vector unaltered

\begin{align*}
\Ba
&= \gamma_0 (\gamma_k \gamma_0)^{\tilde{}} a^k \gamma_0 \\
&= \gamma_0 (\gamma_0 \gamma_k) a^k \gamma_0 \\
&= \gamma_k a^k \gamma_0 \\
&= \Ba
\end{align*}

But the pseudoscalar is negated

\begin{align*}
i^\dagger
&=
\gamma_0 \tilde{i} \gamma_0 \\
&=
\gamma_0 i \gamma_0 \\
&=
-\gamma_0 \gamma_0 i \\
&=
- i \\
\end{align*}

This allows for a split by conjugation of the field into its electric and magnetic field components.

\begin{align*}
F^\dagger
&= -\gamma_0 ( \BE + i c \BB) \gamma_0 \\
&= -\gamma_0^2 ( -\BE + i c \BB) \\
&= \BE - i c\BB \\
\end{align*}

So we have

\begin{align}\label{eqn:maxwellVacuum:conjuagateSplit}
\BE &= \inv{2}(F + F^\dagger) \\
c \BB &= \inv{2i}(F - F^\dagger)
\end{align}

\subsection{Field Energy Momentum density four vector}

In the GA formalism the energy momentum tensor is

\begin{align}
T(a) = \frac{\epsilon_0}{2} F a \tilde{F}
\end{align}

It is not necessarily obvious this bivector-vector-bivector product construction is even a vector quantity.  Expansion of $T(\gamma_0)$ in terms of the electric and magnetic fields demonstrates this vectorial nature.

\begin{align*}
F \gamma_0 \tilde{F}
&=
-(\BE + i c \BB) \gamma_0 (\BE + i c \BB) \\
&=
-\gamma_0 (-\BE + i c \BB) (\BE + i c \BB) \\
&=
-\gamma_0 (-\BE^2 - c^2 \BB^2 + i c (\BB \BE - \BE \BB) ) \\
&=
\gamma_0 (\BE^2 + c^2 \BB^2) - 2 \gamma_0 i c (\BB \wedge \BE) ) \\
&=
\gamma_0 (\BE^2 + c^2 \BB^2) + 2 \gamma_0 c (\BB \cross \BE) \\
&=
\gamma_0 (\BE^2 + c^2 \BB^2) + 2 \gamma_0 c \gamma_k \gamma_0 (\BB \cross \BE)^k \\
&=
\gamma_0 (\BE^2 + c^2 \BB^2) + 2 \gamma_k (\BE \cross (c \BB))^k \\
\end{align*}

Therefore, $T(\gamma_0)$, the energy momentum tensor biased towards a particular observer frame $\gamma_0$
is

\begin{align}\label{eqn:maxwellVacuum:emFourVect}
T(\gamma_0)
&=
\gamma_0 \frac{\epsilon_0}{2} (\BE^2 + c^2 \BB^2) + \gamma_k \epsilon_0 (\BE \cross (c \BB))^k
\end{align}

Recognizable here in the components $T(\gamma_0)$ are the field energy density and momentum density.  In particular the energy density can be obtained by dotting with $\gamma_0$, whereas the (spatial vector) momentum by wedging with $\gamma_0$.

These are

\begin{align}
U \equiv T(\gamma_0) \cdot \gamma_0 &= \frac{1}{2} \left( \epsilon_0 \BE^2 + \inv{\mu_0} \BB^2 \right) \\
c \BP \equiv T(\gamma_0) \wedge \gamma_0 &= \inv{\mu_0} \BE \cross \BB
\end{align}

In terms of the combined field these are

\begin{align}
U &= \frac{-\epsilon_0}{2}( F \gamma_0 F \gamma_0 + \gamma_0 F \gamma_0 F) \\
c \BP &= \frac{-\epsilon_0}{2}( F \gamma_0 F \gamma_0 - \gamma_0 F \gamma_0 F)
\end{align}

Summarizing with the Hermitian conjugate

\begin{align}
U &= \frac{\epsilon_0}{2}( F F^\dagger + F^\dagger F) \\
c \BP &= \frac{\epsilon_0}{2}( F F^\dagger - F^\dagger F)
\end{align}

\subsubsection{Divergence}

Calculation of the divergence produces the components of the Lorentz force densities

\begin{align*}
\grad \cdot T(a)
&= \frac{\epsilon_0}{2} \gpgradezero{ \grad (F a F) } \\
&= \frac{\epsilon_0}{2} \gpgradezero{ (\grad F) a F + (F \grad) F a } \\
\end{align*}

Here the gradient is used implicitly in bidirectional form, where the direction is implied by context.  From Maxwell's equation we have

\begin{align*}
J/\epsilon_0 c
&= (\grad F)^{\tilde{}} \\
&= (\tilde{F} \tilde{\grad}) \\
&= -(F \grad)
\end{align*}

and continuing the expansion

\begin{align*}
\grad \cdot T(a)
&= \frac{1}{2c} \gpgradezero{ J a F - J F a } \\
&= \frac{1}{2c} \gpgradezero{ F J a - J F a } \\
&= \frac{1}{2c} \gpgradezero{ (F J - J F) a } \\
\end{align*}

Wrapping up, the divergence and the adjoint of the energy momentum tensor are

\begin{align}
\grad \cdot T(a) &= \frac{1}{c} (F \cdot J) \cdot a \\
\overbar{T}(\grad) &= F \cdot J/c
\end{align}

When integrated over a volume, the quantities $F \cdot J/c$ are the components of the RHS of the Lorentz force equation $\dot{p} = q F \cdot v/c$.

%\EndArticle
