\documentclass{article}

\usepackage{amsmath}
\usepackage{mathpazo}

%
% shorthand for bold symbols, convenient for vectors and matrices
%
\newcommand{\Ba}[0]{\mathbf{a}}
\newcommand{\Bb}[0]{\mathbf{b}}
\newcommand{\Bc}[0]{\mathbf{c}}
\newcommand{\Bd}[0]{\mathbf{d}}
\newcommand{\Be}[0]{\mathbf{e}}
\newcommand{\Bf}[0]{\mathbf{f}}
\newcommand{\Bg}[0]{\mathbf{g}}
\newcommand{\Bh}[0]{\mathbf{h}}
\newcommand{\Bi}[0]{\mathbf{i}}
\newcommand{\Bj}[0]{\mathbf{j}}
\newcommand{\Bk}[0]{\mathbf{k}}
\newcommand{\Bl}[0]{\mathbf{l}}
\newcommand{\Bm}[0]{\mathbf{m}}
\newcommand{\Bn}[0]{\mathbf{n}}
\newcommand{\Bo}[0]{\mathbf{o}}
\newcommand{\Bp}[0]{\mathbf{p}}
\newcommand{\Bq}[0]{\mathbf{q}}
\newcommand{\Br}[0]{\mathbf{r}}
\newcommand{\Bs}[0]{\mathbf{s}}
\newcommand{\Bt}[0]{\mathbf{t}}
\newcommand{\Bu}[0]{\mathbf{u}}
\newcommand{\Bv}[0]{\mathbf{v}}
\newcommand{\Bw}[0]{\mathbf{w}}
\newcommand{\Bx}[0]{\mathbf{x}}
\newcommand{\By}[0]{\mathbf{y}}
\newcommand{\Bz}[0]{\mathbf{z}}
\newcommand{\BA}[0]{\mathbf{A}}
\newcommand{\BB}[0]{\mathbf{B}}
\newcommand{\BC}[0]{\mathbf{C}}
\newcommand{\BD}[0]{\mathbf{D}}
\newcommand{\BE}[0]{\mathbf{E}}
\newcommand{\BF}[0]{\mathbf{F}}
\newcommand{\BG}[0]{\mathbf{G}}
\newcommand{\BH}[0]{\mathbf{H}}
\newcommand{\BI}[0]{\mathbf{I}}
\newcommand{\BJ}[0]{\mathbf{J}}
\newcommand{\BK}[0]{\mathbf{K}}
\newcommand{\BL}[0]{\mathbf{L}}
\newcommand{\BM}[0]{\mathbf{M}}
\newcommand{\BN}[0]{\mathbf{N}}
\newcommand{\BO}[0]{\mathbf{O}}
\newcommand{\BP}[0]{\mathbf{P}}
\newcommand{\BQ}[0]{\mathbf{Q}}
\newcommand{\BR}[0]{\mathbf{R}}
\newcommand{\BS}[0]{\mathbf{S}}
\newcommand{\BT}[0]{\mathbf{T}}
\newcommand{\BU}[0]{\mathbf{U}}
\newcommand{\BV}[0]{\mathbf{V}}
\newcommand{\BW}[0]{\mathbf{W}}
\newcommand{\BX}[0]{\mathbf{X}}
\newcommand{\BY}[0]{\mathbf{Y}}
\newcommand{\BZ}[0]{\mathbf{Z}}

\newcommand{\Bzero}[0]{\mathbf{0}}
\newcommand{\Btheta}[0]{\boldsymbol{\theta}}
\newcommand{\Btau}[0]{\boldsymbol{\tau}}
\newcommand{\Bomega}[0]{\boldsymbol{\omega}}

%
% shorthand for unit vectors
%
\newcommand{\acap}[0]{\hat{\Ba}}
\newcommand{\bcap}[0]{\hat{\Bb}}
\newcommand{\ccap}[0]{\hat{\Bc}}
\newcommand{\dcap}[0]{\hat{\Bd}}
\newcommand{\ecap}[0]{\hat{\Be}}
\newcommand{\fcap}[0]{\hat{\Bf}}
\newcommand{\gcap}[0]{\hat{\Bg}}
\newcommand{\hcap}[0]{\hat{\Bh}}
\newcommand{\icap}[0]{\hat{\Bi}}
\newcommand{\jcap}[0]{\hat{\Bj}}
\newcommand{\kcap}[0]{\hat{\Bk}}
\newcommand{\lcap}[0]{\hat{\Bl}}
\newcommand{\mcap}[0]{\hat{\Bm}}
\newcommand{\ncap}[0]{\hat{\Bn}}
\newcommand{\ocap}[0]{\hat{\Bo}}
\newcommand{\pcap}[0]{\hat{\Bp}}
\newcommand{\qcap}[0]{\hat{\Bq}}
\newcommand{\rcap}[0]{\hat{\Br}}
\newcommand{\scap}[0]{\hat{\Bs}}
\newcommand{\tcap}[0]{\hat{\Bt}}
\newcommand{\ucap}[0]{\hat{\Bu}}
\newcommand{\vcap}[0]{\hat{\Bv}}
\newcommand{\wcap}[0]{\hat{\Bw}}
\newcommand{\xcap}[0]{\hat{\Bx}}
\newcommand{\ycap}[0]{\hat{\By}}
\newcommand{\zcap}[0]{\hat{\Bz}}
\newcommand{\thetacap}[0]{\hat{\Btheta}}

%
% to write R^n and C^n in a distinguishable fashion.  Perhaps change this
% to the double lined characters upon figuring out how to do so.
%
\newcommand{\C}[1]{$\mathbb{C}^{#1}$}
\newcommand{\R}[1]{$\mathbb{R}^{#1}$}

%
% various generally useful helpers
%

% derivative of #1 wrt. #2:
\newcommand{\D}[2] {\frac {d#2} {d#1}}

\newcommand{\inv}[1]{\frac{1}{#1}}
\newcommand{\cross}[0]{\times}

\newcommand{\abs}[1]{\lvert{#1}\rvert}
\newcommand{\norm}[1]{\lVert{#1}\rVert}
\newcommand{\innerprod}[2]{\langle{#1}, {#2}\rangle}
\newcommand{\dotprod}[2]{{#1} \cdot {#2}}
\newcommand{\bdotprod}[2]{\left({#1} \cdot {#2}\right)}
\newcommand{\crossprod}[2]{{#1} \cross {#2}}
\newcommand{\tripleprod}[3]{\dotprod{\left(\crossprod{#1}{#2}\right)}{#3}}

\DeclareMathOperator{\Proj}{Proj}
\DeclareMathOperator{\Span}{span}
\DeclareMathOperator{\Sgn}{sgn}
\DeclareMathOperator{\Area}{Area}
\DeclareMathOperator{\Volume}{Volume}

%
% A few miscellaneous things specific to this document
%
\newcommand{\crossop}[1]{\crossprod{#1}{}}

% R2 vector.
\newcommand{\VectorTwo}[2]{
\begin{bmatrix}
 {#1} \\
 {#2}
\end{bmatrix}
}

\newcommand{\VectorN}[1]{
\begin{bmatrix}
{#1}_1 \\
{#1}_2 \\
\vdots \\
{#1}_N \\
\end{bmatrix}
}

\newcommand{\DETuvij}[4]{
\begin{vmatrix}
 {#1}_{#3} & {#1}_{#4} \\
 {#2}_{#3} & {#2}_{#4}
\end{vmatrix}
}

\newcommand{\DETuvwijk}[6]{
\begin{vmatrix}
 {#1}_{#4} & {#1}_{#5} & {#1}_{#6} \\
 {#2}_{#4} & {#2}_{#5} & {#2}_{#6} \\
 {#3}_{#4} & {#3}_{#5} & {#3}_{#6}
\end{vmatrix}
}

\newcommand{\DETuvwxijkl}[8]{
\begin{vmatrix}
 {#1}_{#5} & {#1}_{#6} & {#1}_{#7} & {#1}_{#8} \\
 {#2}_{#5} & {#2}_{#6} & {#2}_{#7} & {#2}_{#8} \\
 {#3}_{#5} & {#3}_{#6} & {#3}_{#7} & {#3}_{#8} \\
 {#4}_{#5} & {#4}_{#6} & {#4}_{#7} & {#4}_{#8} \\
\end{vmatrix}
}

%\newcommand{\DETuvwxyijklm}[10]{
%\begin{vmatrix}
% {#1}_{#6} & {#1}_{#7} & {#1}_{#8} & {#1}_{#9} & {#1}_{#10} \\
% {#2}_{#6} & {#2}_{#7} & {#2}_{#8} & {#2}_{#9} & {#2}_{#10} \\
% {#3}_{#6} & {#3}_{#7} & {#3}_{#8} & {#3}_{#9} & {#3}_{#10} \\
% {#4}_{#6} & {#4}_{#7} & {#4}_{#8} & {#4}_{#9} & {#4}_{#10} \\
% {#5}_{#6} & {#5}_{#7} & {#5}_{#8} & {#5}_{#9} & {#5}_{#10}
%\end{vmatrix}
%}

% R3 vector.
\newcommand{\VectorThree}[3]{
\begin{bmatrix}
 {#1} \\
 {#2} \\
 {#3}
\end{bmatrix}
}


%<misc>
%
\newcommand{\Abs}[1]{{\left\lvert{#1}\right\rvert}}
\newcommand{\spacegrad}[0]{\boldsymbol{\nabla}}
\newcommand{\grad}[0]{\nabla}
\newcommand{\LL}[0]{\mathcal{L}}

% == \partial_{#1} {#2}
\newcommand{\PD}[2]{\frac{\partial {#2}}{\partial {#1}}}
% inline variant
\newcommand{\PDi}[2]{{\partial {#2}}/{\partial {#1}}}

\newcommand{\PDD}[3]{\frac{\partial^2 {#3}}{\partial {#1}\partial {#2}}}
%\newcommand{\PDd}[2]{\frac{\partial^2 {#2}}{{\partial{#1}}^2}}
\newcommand{\PDsq}[2]{\frac{\partial^2 {#2}}{(\partial {#1})^2}}

\newcommand{\Partial}[2]{\frac{\partial {#1}}{\partial {#2}}}
\DeclareMathOperator{\RejName}{Rej}
\newcommand{\Rej}[2]{\RejName_{#1}\left( {#2} \right)}
\newcommand{\Rm}[1]{\mathbb{R}^{#1}}
\newcommand{\Cm}[1]{\mathbb{C}^{#1}}
\newcommand{\conj}[0]{{*}}

%</misc>

% <grade selection>
%
\newcommand{\gpgrade}[2] {{\left\langle{{#1}}\right\rangle}_{#2}}

\newcommand{\gpgradezero}[1] {\gpgrade{#1}{}}
%\newcommand{\gpscalargrade}[1] {{\left\langle{{#1}}\right\rangle}}
%\newcommand{\gpgradezero}[1] {\gpgrade{#1}{0}}

%\newcommand{\gpgradeone}[1] {{\left\langle{{#1}}\right\rangle}_{1}}
\newcommand{\gpgradeone}[1] {\gpgrade{#1}{1}}

\newcommand{\gpgradetwo}[1] {\gpgrade{#1}{2}}
\newcommand{\gpgradethree}[1] {\gpgrade{#1}{3}}
\newcommand{\gpgradefour}[1] {\gpgrade{#1}{4}}
%
% </grade selection>



\newcommand{\adot}[0]{{\dot{a}}}
\newcommand{\bdot}[0]{{\dot{b}}}
% taken for centered dot:
%\newcommand{\cdot}[0]{{\dot{c}}}
%\newcommand{\ddot}[0]{{\dot{d}}}
\newcommand{\edot}[0]{{\dot{e}}}
\newcommand{\fdot}[0]{{\dot{f}}}
\newcommand{\gdot}[0]{{\dot{g}}}
\newcommand{\hdot}[0]{{\dot{h}}}
\newcommand{\idot}[0]{{\dot{i}}}
\newcommand{\jdot}[0]{{\dot{j}}}
\newcommand{\kdot}[0]{{\dot{k}}}
\newcommand{\ldot}[0]{{\dot{l}}}
\newcommand{\mdot}[0]{{\dot{m}}}
\newcommand{\ndot}[0]{{\dot{n}}}
%\newcommand{\odot}[0]{{\dot{o}}}
\newcommand{\pdot}[0]{{\dot{p}}}
\newcommand{\qdot}[0]{{\dot{q}}}
\newcommand{\rdot}[0]{{\dot{r}}}
\newcommand{\sdot}[0]{{\dot{s}}}
\newcommand{\tdot}[0]{{\dot{t}}}
\newcommand{\udot}[0]{{\dot{u}}}
\newcommand{\vdot}[0]{{\dot{v}}}
\newcommand{\wdot}[0]{{\dot{w}}}
\newcommand{\xdot}[0]{{\dot{x}}}
\newcommand{\ydot}[0]{{\dot{y}}}
\newcommand{\zdot}[0]{{\dot{z}}}
\newcommand{\addot}[0]{{\ddot{a}}}
\newcommand{\bddot}[0]{{\ddot{b}}}
\newcommand{\cddot}[0]{{\ddot{c}}}
%\newcommand{\dddot}[0]{{\ddot{d}}}
\newcommand{\eddot}[0]{{\ddot{e}}}
\newcommand{\fddot}[0]{{\ddot{f}}}
\newcommand{\gddot}[0]{{\ddot{g}}}
\newcommand{\hddot}[0]{{\ddot{h}}}
\newcommand{\iddot}[0]{{\ddot{i}}}
\newcommand{\jddot}[0]{{\ddot{j}}}
\newcommand{\kddot}[0]{{\ddot{k}}}
\newcommand{\lddot}[0]{{\ddot{l}}}
\newcommand{\mddot}[0]{{\ddot{m}}}
\newcommand{\nddot}[0]{{\ddot{n}}}
\newcommand{\oddot}[0]{{\ddot{o}}}
\newcommand{\pddot}[0]{{\ddot{p}}}
\newcommand{\qddot}[0]{{\ddot{q}}}
\newcommand{\rddot}[0]{{\ddot{r}}}
\newcommand{\sddot}[0]{{\ddot{s}}}
\newcommand{\tddot}[0]{{\ddot{t}}}
\newcommand{\uddot}[0]{{\ddot{u}}}
\newcommand{\vddot}[0]{{\ddot{v}}}
\newcommand{\wddot}[0]{{\ddot{w}}}
\newcommand{\xddot}[0]{{\ddot{x}}}
\newcommand{\yddot}[0]{{\ddot{y}}}
\newcommand{\zddot}[0]{{\ddot{z}}}

%<bold and dot greek symbols>
%

\newcommand{\Deltadot}[0]{{\dot{\Delta}}}
\newcommand{\Gammadot}[0]{{\dot{\Gamma}}}
\newcommand{\Lambdadot}[0]{{\dot{\Lambda}}}
\newcommand{\Omegadot}[0]{{\dot{\Omega}}}
\newcommand{\Phidot}[0]{{\dot{\Phi}}}
\newcommand{\Pidot}[0]{{\dot{\Pi}}}
\newcommand{\Psidot}[0]{{\dot{\Psi}}}
\newcommand{\Sigmadot}[0]{{\dot{\Sigma}}}
\newcommand{\Thetadot}[0]{{\dot{\Theta}}}
\newcommand{\Upsilondot}[0]{{\dot{\Upsilon}}}
\newcommand{\Xidot}[0]{{\dot{\Xi}}}
\newcommand{\alphadot}[0]{{\dot{\alpha}}}
\newcommand{\betadot}[0]{{\dot{\beta}}}
\newcommand{\chidot}[0]{{\dot{\chi}}}
\newcommand{\deltadot}[0]{{\dot{\delta}}}
\newcommand{\epsilondot}[0]{{\dot{\epsilon}}}
\newcommand{\etadot}[0]{{\dot{\eta}}}
\newcommand{\gammadot}[0]{{\dot{\gamma}}}
\newcommand{\kappadot}[0]{{\dot{\kappa}}}
\newcommand{\lambdadot}[0]{{\dot{\lambda}}}
\newcommand{\mudot}[0]{{\dot{\mu}}}
\newcommand{\nudot}[0]{{\dot{\nu}}}
\newcommand{\omegadot}[0]{{\dot{\omega}}}
\newcommand{\phidot}[0]{{\dot{\phi}}}
\newcommand{\pidot}[0]{{\dot{\pi}}}
\newcommand{\psidot}[0]{{\dot{\psi}}}
\newcommand{\rhodot}[0]{{\dot{\rho}}}
\newcommand{\sigmadot}[0]{{\dot{\sigma}}}
\newcommand{\taudot}[0]{{\dot{\tau}}}
\newcommand{\thetadot}[0]{{\dot{\theta}}}
\newcommand{\upsilondot}[0]{{\dot{\upsilon}}}
\newcommand{\varepsilondot}[0]{{\dot{\varepsilon}}}
\newcommand{\varphidot}[0]{{\dot{\varphi}}}
\newcommand{\varpidot}[0]{{\dot{\varpi}}}
\newcommand{\varrhodot}[0]{{\dot{\varrho}}}
\newcommand{\varsigmadot}[0]{{\dot{\varsigma}}}
\newcommand{\varthetadot}[0]{{\dot{\vartheta}}}
\newcommand{\xidot}[0]{{\dot{\xi}}}
\newcommand{\zetadot}[0]{{\dot{\zeta}}}

\newcommand{\Deltaddot}[0]{{\ddot{\Delta}}}
\newcommand{\Gammaddot}[0]{{\ddot{\Gamma}}}
\newcommand{\Lambdaddot}[0]{{\ddot{\Lambda}}}
\newcommand{\Omegaddot}[0]{{\ddot{\Omega}}}
\newcommand{\Phiddot}[0]{{\ddot{\Phi}}}
\newcommand{\Piddot}[0]{{\ddot{\Pi}}}
\newcommand{\Psiddot}[0]{{\ddot{\Psi}}}
\newcommand{\Sigmaddot}[0]{{\ddot{\Sigma}}}
\newcommand{\Thetaddot}[0]{{\ddot{\Theta}}}
\newcommand{\Upsilonddot}[0]{{\ddot{\Upsilon}}}
\newcommand{\Xiddot}[0]{{\ddot{\Xi}}}
\newcommand{\alphaddot}[0]{{\ddot{\alpha}}}
\newcommand{\betaddot}[0]{{\ddot{\beta}}}
\newcommand{\chiddot}[0]{{\ddot{\chi}}}
\newcommand{\deltaddot}[0]{{\ddot{\delta}}}
\newcommand{\epsilonddot}[0]{{\ddot{\epsilon}}}
\newcommand{\etaddot}[0]{{\ddot{\eta}}}
\newcommand{\gammaddot}[0]{{\ddot{\gamma}}}
\newcommand{\kappaddot}[0]{{\ddot{\kappa}}}
\newcommand{\lambdaddot}[0]{{\ddot{\lambda}}}
\newcommand{\muddot}[0]{{\ddot{\mu}}}
\newcommand{\nuddot}[0]{{\ddot{\nu}}}
\newcommand{\omegaddot}[0]{{\ddot{\omega}}}
\newcommand{\phiddot}[0]{{\ddot{\phi}}}
\newcommand{\piddot}[0]{{\ddot{\pi}}}
\newcommand{\psiddot}[0]{{\ddot{\psi}}}
\newcommand{\rhoddot}[0]{{\ddot{\rho}}}
\newcommand{\sigmaddot}[0]{{\ddot{\sigma}}}
\newcommand{\tauddot}[0]{{\ddot{\tau}}}
\newcommand{\thetaddot}[0]{{\ddot{\theta}}}
\newcommand{\upsilonddot}[0]{{\ddot{\upsilon}}}
\newcommand{\varepsilonddot}[0]{{\ddot{\varepsilon}}}
\newcommand{\varphiddot}[0]{{\ddot{\varphi}}}
\newcommand{\varpiddot}[0]{{\ddot{\varpi}}}
\newcommand{\varrhoddot}[0]{{\ddot{\varrho}}}
\newcommand{\varsigmaddot}[0]{{\ddot{\varsigma}}}
\newcommand{\varthetaddot}[0]{{\ddot{\vartheta}}}
\newcommand{\xiddot}[0]{{\ddot{\xi}}}
\newcommand{\zetaddot}[0]{{\ddot{\zeta}}}

\newcommand{\BDelta}[0]{\boldsymbol{\Delta}}
\newcommand{\BGamma}[0]{\boldsymbol{\Gamma}}
\newcommand{\BLambda}[0]{\boldsymbol{\Lambda}}
\newcommand{\BOmega}[0]{\boldsymbol{\Omega}}
\newcommand{\BPhi}[0]{\boldsymbol{\Phi}}
\newcommand{\BPi}[0]{\boldsymbol{\Pi}}
\newcommand{\BPsi}[0]{\boldsymbol{\Psi}}
\newcommand{\BSigma}[0]{\boldsymbol{\Sigma}}
\newcommand{\BTheta}[0]{\boldsymbol{\Theta}}
\newcommand{\BUpsilon}[0]{\boldsymbol{\Upsilon}}
\newcommand{\BXi}[0]{\boldsymbol{\Xi}}
\newcommand{\Balpha}[0]{\boldsymbol{\alpha}}
\newcommand{\Bbeta}[0]{\boldsymbol{\beta}}
\newcommand{\Bchi}[0]{\boldsymbol{\chi}}
\newcommand{\Bdelta}[0]{\boldsymbol{\delta}}
\newcommand{\Bepsilon}[0]{\boldsymbol{\epsilon}}
\newcommand{\Beta}[0]{\boldsymbol{\eta}}
\newcommand{\Bgamma}[0]{\boldsymbol{\gamma}}
\newcommand{\Bkappa}[0]{\boldsymbol{\kappa}}
\newcommand{\Blambda}[0]{\boldsymbol{\lambda}}
\newcommand{\Bmu}[0]{\boldsymbol{\mu}}
\newcommand{\Bnu}[0]{\boldsymbol{\nu}}
%\newcommand{\Bomega}[0]{\boldsymbol{\omega}}
\newcommand{\Bphi}[0]{\boldsymbol{\phi}}
\newcommand{\Bpi}[0]{\boldsymbol{\pi}}
\newcommand{\Bpsi}[0]{\boldsymbol{\psi}}
\newcommand{\Brho}[0]{\boldsymbol{\rho}}
\newcommand{\Bsigma}[0]{\boldsymbol{\sigma}}
%\newcommand{\Btau}[0]{\boldsymbol{\tau}}
%\newcommand{\Btheta}[0]{\boldsymbol{\theta}}
\newcommand{\Bupsilon}[0]{\boldsymbol{\upsilon}}
\newcommand{\Bvarepsilon}[0]{\boldsymbol{\varepsilon}}
\newcommand{\Bvarphi}[0]{\boldsymbol{\varphi}}
\newcommand{\Bvarpi}[0]{\boldsymbol{\varpi}}
\newcommand{\Bvarrho}[0]{\boldsymbol{\varrho}}
\newcommand{\Bvarsigma}[0]{\boldsymbol{\varsigma}}
\newcommand{\Bvartheta}[0]{\boldsymbol{\vartheta}}
\newcommand{\Bxi}[0]{\boldsymbol{\xi}}
\newcommand{\Bzeta}[0]{\boldsymbol{\zeta}}
%
%</bold and dot greek symbols>
%<infrequent>
%
%\newcommand{\AreaOp}[1]{\AName_{#1}}
%\newcommand{\Babs}[0]{\abs{\BB}}
%\newcommand{\Bcap}[0]{\hat{\BB}}
%\newcommand{\BrPrimeRej}[0]{\rcap(\rcap \wedge \Br')}
%\newcommand{\CA}[0]{\mathcal{A}}
%\newcommand{\Cos}[1]{\cos{\left({#1}\right)}}
%\newcommand{\Det}[1] {\abs{#1}}
%\newcommand{\Dsq}[2] {\frac {\partial^2 {#1}} {\partial {#2}^2}}
%\newcommand{\Exp}[1]{\exp{\left({#1}\right)}}
%\newcommand{\Norm}[1]{\left\lVert{#1}\right\rVert}
%\newcommand{\Sin}[1]{\sin{\left({#1}\right)}}
%\newcommand{\T}[0]{\text{T}}
%\newcommand{\VolumeOp}[1]{\VName_{#1}}
%\newcommand{\agrad}[0]{\Ba \cdot \nabla}
%\newcommand{\alphacap}[0]{\hat{\boldsymbol{\alpha}}}
%\newcommand{\Fcap}[0]{\hat{\BF}}
%\newcommand{\bithree}[0]{{\Bi}_3}
%\newcommand{\bxa}[0]{\Bx\Ba}
%\newcommand{\coordvec}[2]{
%\newcommand{\costheta}[0]{\acap \cdot \xcap}
%\newcommand{\ddt}[1]{\ddot{#1}}
%\newcommand{\ddu}[1] {\frac {d{#1}} {du}}
%\newcommand{\dsqxj}[2] {\frac {\partial^2 {#1}} {\partial {x_{#2}}^2}}
%\newcommand{\dtheta}[1]{\frac{d {#1}}{d \theta}}
%\newcommand{\dt}[1]{\dot{#1}}
%\newcommand{\dt}[1]{\frac{d {#1}}{dt}}
%\newcommand{\dxj}[2] {\frac {\partial {#1}} {\partial {x_{#2}}}}
%\newcommand{\halfPhi}[0]{\frac{\phi}{2}}
%\newcommand{\half}[0]{\inv{2}}
%\newcommand{\inv}[1]{\frac{1}{#1}}
%\newcommand{\laplacian}[0]{\nabla^2}
%\newcommand{\matrixoftx}[3]{
%\newcommand{\nrrp}[0]{\norm{\rcap \wedge \Br'}}
%\newcommand{\oiint}{\bigcirc \hspace{-1.4em} \int \hspace{-.8em} \int}
%\newcommand{\transpose}[1]{{#1}^{\text{T}}}
%\newcommand{\transpose}[1]{{{#1}^{\TextTranspose}}}
%\newcommand{\transpose}[1]{{{#1}^{\text{T}}}}
%\newcommand{\barA}[0]{\bar{A}}
%\newcommand{\qbar}[0]{\bar{q}}
%\newcommand{\qdotbar}[0]{\dot{\bar{q}}}
%
%</infrequent>





\newcommand{\PDSq}[2]{\frac{\partial^2 {#2}}{\partial {#1}^2}}
\DeclareMathOperator{\sinc}{sinc}
\DeclareMathOperator{\PV}{PV}
\newcommand{\FF}[0]{\mathcal{F}}
\newcommand{\IIinf}[0]{ \int_{-\infty}^\infty }

\usepackage[bookmarks=true]{hyperref}

\usepackage{color,cite,graphicx}
   % use colour in the document, put your citations as [1-4]
   % rather than [1,2,3,4] (it looks nicer, and the extended LaTeX2e
   % graphics package. 
\usepackage{latexsym,amssymb,epsf} % don't remember if these are
   % needed, but their inclusion can't do any damage


\title{ 4D Fourier transforms applied to Maxwell's equation. }
\author{Peeter Joot}
\date{ Feb 1, 2009.  Last Revision: $Date: 2009/02/06 00:44:19 $ }

\begin{document}

\maketitle{}
\tableofcontents

\section{ Notation. }

Please see \ref{eqn:notation} below for a summary of prerequisite notation used here.

\section{ Motivation. }

In \cite{PJfirstOrderMaxwell}, a solution of the first order Maxwell equation

\begin{align}
\grad F &= \frac{J }{\epsilon_0 c}
\end{align}

was found to be

\begin{align}
F(\Bx,t) 
&=
\inv{({2\pi})^3} \int 
e^{-i c \Bk t}
\left(
F(\Bu, 0) + \inv{\epsilon_0} \int_{\tau = -\infty}^{t} e^{i c \Bk \tau } \gamma_0 J(\Bu,\tau)  d\tau  
\right)
e^{i \Bk \cdot (\Bx-\Bu)} 
d^3 u
d^3 k
\end{align}

This doesn't have the spacetime uniformity that is expected for a solution of a Lorentz invariant equation.

Similarily, in \cite{PJfourierMaxwellSecondOrder} solutions of the second order Maxwell equation in the Lorentz gauge
$\grad \cdot A = 0$ 

\begin{align*}
F &= \grad \wedge A \\
\grad^2 A &= J/\epsilon_0 c
\end{align*}

were found to be
\begin{align}\label{eqn:fourVectorPotentials}
{A^\mu}(x)
&= \inv{\epsilon_0 c} \int J^\mu(x') G(x - x') d^4 x' \\
G(x)
&= 
\frac{u(x \cdot \gamma_0)}{ (2\pi)^3 }
\int
\sin( \Abs{\Bk} x \cdot \gamma_0 )
\exp\left( -i (\Bk \gamma_0) \cdot x \right)
\frac{d^3 k}{ \Abs{\Bk} }
\end{align}

Here our convolution kernel $G$ also doesn't exhibit a uniform four vector form that one could logically expect.

In these notes an attempt to rework these problems using a 4D spacetime Fourier transform will be made.

\section{ 4D Fourier transform. }

As before we want a multivector friendly Fourier transform pair, and choose the following

\begin{align}\label{eqn:FourierTxDefinition}
\hat{\psi}(k) &= \inv{(\sqrt{2\pi})^4} \IIinf \psi(x) \exp\left( -i k \cdot x \right) d^4 x \\
{\psi}(x) &= \PV \inv{(\sqrt{2\pi})^4} \IIinf \hat{\psi}(k) \exp\left( i k \cdot x \right) d^4 k
\end{align}

Here we use $i = \gamma_0 \gamma_1 \gamma_2 \gamma_3$ as our pseudoscalar, and have to therefore be careful of order
of operations since this does not neccessarily commute with multivector $\psi$ or $\hat{\psi}$ functions.

For our dot product and vectors, with summation over matched upstairs downstairs indexes implied, we write

\begin{align*}
x &= x^\mu \gamma_\mu = x_\mu \gamma^\mu \\
k &= k^\mu \gamma_\mu = k_\mu \gamma^\mu \\
x \cdot k &= x^\mu k_\mu = x_\mu k^\mu
\end{align*}

Finally our differential volume elements are defined to be

\begin{align*}
d^4 x &= dx^0 dx^1 dx^2 dx^3 \\
d^4 k &= dk_0 dk_1 dk_2 dk_3 \\
\end{align*}

Note the opposite pairing of upstairs and downstairs indexes in the coordinates.

\section{ Potential equations. }

\subsection{ Inhomogeneous case. } 

First for the attack is the Maxwell potential equations.  As well as using a 4D transform, having learned how to do Fourier
transformations of multivectors, we will attack this one in vector form as well.  Our equation to invert is

\begin{align*}
\grad^2 A = J/\epsilon_0 c
\end{align*}

There is nothing special to do for the transformation of the current term, but the left hand side will require two integration
parts

\begin{align*}
\FF(\grad^2 A )
&= \inv{(2 \pi)^2} \IIinf \left(\left(\partial_{00} - \sum_m \partial_{mm}\right) A \right) e^{ -i k_\mu x^\mu } d^4 x \\
&= \inv{(2 \pi)^2} \IIinf A \left( (-i k_0)^2 - \sum_m (-i k_m)^2 \right) e^{ -i k_\mu x^\mu } d^4 x \\
\end{align*}

As usual it is required that $A$ and $\partial_\mu A$ vanish at infinity.  Now for the scalar in the interior we have

\begin{align*}
(-i k_0)^2 - \sum_m (-i k_m)^2 
&= -(k_0)^2 + \sum_m (k_m)^2 \\
\end{align*}

But this is just the (negation) of the square of our wave number vector
\begin{align*}
k^2 
&= k_\mu \gamma^\mu \cdot k_\nu \gamma^\nu \\
&= k_\mu k_\nu \gamma^\mu \cdot \gamma^\nu \\
&= 
k_0 k_0 \gamma_0 \cdot \gamma^0 
-\sum_{a,b} k_a k_b \gamma_a \cdot \gamma^b \\
&= (k_0)^2 - \sum_a (k_a)^2
\end{align*}

Putting things back together we have for our potential vector in the wave number domain

\begin{align*}
\hat{A} &= \frac{\hat{J}}{- k^2 \epsilon_0 c}
\end{align*}

Inverting, and substitution for $\hat{J}$ gives us our spacetime domain potential vector in one fell swoop

\begin{align*}
A(x)
&= 
\inv{(\sqrt{2\pi})^4} \IIinf 
\left(
\inv{- k^2 \epsilon_0 c} \inv{(\sqrt{2\pi})^4} \IIinf {J}(x') e^{-i k \cdot x' } d^4 x'
\right)
e^{ i k \cdot x } d^4 k \\
&= 
\inv{({2\pi})^4} \IIinf {J}(x') \inv{- k^2 \epsilon_0 c} e^{ i k \cdot (x - x') } d^4 k d^4 x'
\\
\end{align*}

This allows us to write this entire specific solution to the forced wave equation problem as a convolution integral

\begin{align}\label{eqn:potentialInHomogeneous}
A(x) &= \inv{\epsilon_0 c} \IIinf {J}(x') G(x-x') d^4 x' \\
G(x) &= \frac{-1}{ ({2\pi})^4} \IIinf \frac{e^{ i k \cdot x }}{ k^2 } d^4 k 
\end{align}

Pretty slick looking, but actually also problematic if one thinks about it.  Since $k^2$ is null in some cases
$G(x)$ may blow up in some conditions.  My assumption however, is that a well defined meaning can be associated
with this integral, I just do not know what it is yet.  A way to define this more exactly may require
picking a more specific orthonormal basis once the exact character of $J$ is known.

\subsection{ The homogeneous case. }

The missing element here is the addition of any allowed homogeneous solutions to the wave equation.
The form of such solutions cannot be obtained with the 4D transform since that produces

\begin{align*}
-k^2 \hat{A} = 0
\end{align*}

and no meaningful inversion of that is possible.

For the homogeneous problem we are forced to reexpress the spacetime Laplacian with an explicit bias towards either time or a specific direction in space, and attack with a Fourier transform on the remaining coordinates.  This has been done previously, but we can
revisit this using our new vector transform.

Now we switch to a spatial Fourier transform

\begin{align}\label{eqn:3DFourierTxDefinition}
\hat{\psi}(\Bk, t) &= \inv{(\sqrt{2\pi})^3} \IIinf \psi(\Bx, t) \exp\left( -i \Bk \cdot \Bx \right) d^3 x \\
{\psi}(\Bx, t) &= \PV \inv{(\sqrt{2\pi})^3} \IIinf \hat{\psi}(\Bk, t) \exp\left( i \Bk \cdot \Bx \right) d^3 k
\end{align}

Using a spatial transform we have

\begin{align*}
\FF((\partial_{00} - \sum_m \partial_{mm}) A)
&= \partial_{00} \hat{A} - \sum_m \hat{A} (-i k_m)^2 
\end{align*}

Carefully keeping the pseudoscalar factors all on the right of our vector as the integration by parts was performed doesn't make a difference since we just end up with a scalar in the end.  Our equation in the wave number domain is then just

\begin{align*}
\partial_{tt} \hat{A}(\Bk,t) + (c^2 \Bk^2) \hat{A}(\Bk,t) &= 0 %c \hat{J}(\Bk,t)/\epsilon_0
\end{align*}

with exponential solutions

\begin{align*}
\hat{A}(\Bk, t) &= C(\Bk) \exp(\pm i c \Abs{\Bk} t)
\end{align*}

In particular, for $t = 0$ we have

\begin{align*}
\hat{A}(\Bk, 0) &= C(\Bk)
\end{align*}

Reassembling then gives us our homogeneous solution

\begin{align*}
{A}(\Bx, t)
&= 
\inv{(\sqrt{2\pi})^3} \IIinf 
\left( \inv{(\sqrt{2\pi})^3} \IIinf A(\Bx', 0) e^{ -i \Bk \cdot \Bx' } d^3 x' \right) e^{\pm i c \Abs{\Bk} t}
e^{ i \Bk \cdot \Bx } d^3 k
\end{align*}

This is

\begin{align}\label{eqn:potentialHomogeneous}
{A}(\Bx, t) &= \IIinf A(\Bx', 0) G( \Bx - \Bx' ) d^3 x' \\
G(\Bx) &= \inv{({2\pi})^3} \IIinf \exp\left( i \Bk \cdot \Bx \pm i c \Abs{\Bk} t \right) d^3 k
\end{align}

Here also we have to be careful to keep the Green's function on the right hand side of $A$ since they 
won't generally commute.

\subsection{ Summarizing. }

Assembling both the homogeneous and inhomogeneous parts for a complete solution we have for the Maxwell
four vector potential

\begin{align}
A(x) &= \IIinf \left( A(\Bx', 0) H( \Bx - \Bx' ) + \inv{\epsilon_0 c} \IIinf {J}(x') G(x-x') dx^0 \right) dx^1 dx^2 dx^3 \\
H(\Bx) &= \inv{({2\pi})^3} \IIinf \exp\left( i \Bk \cdot \Bx  \pm i c \Abs{\Bk} t \right) d^3 k \\
G(x) &= \frac{-1}{ ({2\pi})^4} \IIinf \frac{e^{ i k \cdot x }}{ k^2 } d^4 k
\end{align}

Here for convienence both four vectors and spatial vectors were used with

\begin{align*}
x &= x^\mu \gamma_\mu \\
\Bx &= x^m \sigma_m = x \wedge \gamma_0
\end{align*}

As expected, operating where possible in a Four vector context does produce a simpler convolution kernel for the vector potential.

\section{ First order Maxwell equation treatment. }

Now we want to Fourier transform Maxwell's equation directly.  That is

\begin{align*}
\FF(\grad F = J/\epsilon_0 c)
\end{align*}

For the LHS we have

\begin{align*}
\FF(\grad F)
&= \FF(\gamma^\mu \partial_\mu F) \\
&= \gamma^\mu \inv{(2\pi)^2} \IIinf (\partial_\mu F) e^{ - i k \cdot x } d^4 x \\
&= -\gamma^\mu \inv{(2\pi)^2} \IIinf F \partial_\mu (e^{ - i k_\sigma x^\sigma }) d^4 x \\
&= -\gamma^\mu \inv{(2\pi)^2} \IIinf F (-i k_\mu ) e^{ - i k \cdot x } d^4 x \\
&= -i \gamma^\mu k_\mu \inv{(2\pi)^2} \IIinf F e^{ - i k \cdot x } d^4 x \\
&= -i k \hat{F}
\end{align*}

This gives us

\begin{align*}
-i k \hat{F} = \hat{J}/\epsilon_0 c
\end{align*}

So to solve the forced Maxwell equation we have only to inverse transform the following

\begin{align*}
\hat{F} = \inv{ -i k \epsilon_0 c} \hat{J}
\end{align*}

This is 

\begin{align*}
{F} 
&= \inv{(\sqrt{2\pi})^4} \IIinf \inv{ -i k \epsilon_0 c} \left( \inv{(\sqrt{2\pi})^4} \IIinf J(x') e^{ -i k \cdot x' } d^4 x' \right) e^{ i k \cdot x } d^4 k \\
\end{align*}

Adding to this a solution to the homogeneous equation we now have a complete solution in terms of the given four current density and an
initial field wave packet

\begin{align*}
{F} &= 
\inv{({2\pi})^3} \int e^{ -i c \Bk t } F(\Bx', 0) e^{ i \Bk \cdot (\Bx-\Bx') } d^3 x' d^3 k  
+
\inv{ ({2\pi})^4 \epsilon_0 c} \int \frac{i}{ k } J(x') e^{ i k \cdot (x - x') } d^4 k d^4 x' \\
\end{align*}

Observe that we can't make a single sided Green's function to convolve $J$ with since the vectors $k$ and $J$ may not commute.

As expected working in a relativistic context for our inherently relativistic equation turns out to be much simpler and produce a simpler result.  As before 
trying to actually evaluate these integrals is a different story.

\section{ Appendix.  Summary of Notation used. }

% 
% NOTATION BOILERPLATE BASE STOLEN FROM ../geometric-algebra/lagrangian_field_density.ltx
%
For standalone purposes, here is a summary of the notation and definitions that will be used.  Greek letters range over all indexes and
english indexes range over $1,2,3$.  Bold vectors are spatial enties and non-bold is used for four vectors.

\begin{equation*}\label{eqn:notation}
\begin{array}{l l l}
\gamma_{\mu} & & \quad \mbox{Four vector basis vector)} \\
& & \quad \mbox{($\gamma_{\mu} \cdot \gamma_{\nu} = \pm {\delta^{\mu}}_{\nu}$)} \\
{(\gamma_0)}^2 {(\gamma_i)}^2 &= -1 & \quad \mbox{Minkowski metric} \\
\sigma_i = \sigma^i &= \gamma_{i} \wedge \gamma_0 & \quad \mbox{Spatial basis bivector. ($\sigma_i \cdot \sigma_j = \delta_{ij}$)} \\
%                    &= \gamma_{i0} \\
i &= \gamma_{0} \wedge \gamma_1 \wedge \gamma_{2} \wedge \gamma_3 & \quad \mbox{Four-vector pseudoscalar} \\
%  &= \gamma_{0123} \\
\gamma^{\mu} \cdot \gamma_{\nu} &= {\delta^{\mu}}_{\nu} & \quad \mbox{Reciprocal basis vectors} \\
x^{\mu} &= x \cdot \gamma^{\mu} & \quad \mbox{Vector coordinate} \\
x_{\mu} &= x \cdot \gamma_{\mu} & \quad \mbox{Coordinate for reciprocal basis} \\
x &= \sum \gamma_{\mu} x^{\mu} & \quad \mbox{Four vector in terms of coordinates} \\
  &= \sum \gamma^{\mu} x_{\mu} \\
\BE &= \sum E^i \sigma_i & \quad \mbox{Electric field spatial vector} \\
\BB &= \sum B^i \sigma_i & \quad \mbox{Magnetic field spatial vector} \\
J &= \sum \gamma_{\mu} J^{\mu} & \quad \mbox{Current density four vector.} \\
  &= \sum \gamma^{\mu} J_{\mu} \\
F &= \BE + i c \BB & \quad \mbox{Faraday bivector} \\
  &= F^{\mu\nu} \gamma_\mu \wedge \gamma_\nu & \quad \mbox{in terms of Faraday tensor} \\
x^{0} &= x \cdot \gamma^0 & \quad \mbox{Time coordinate (length dim.)} \\
      &= c t \\
\Bx &= x \wedge \gamma_0 & \quad \mbox{Spatial vector} \\
    &= x^i \sigma_i \\
J^{0} &= J \cdot \gamma^0 & \quad \mbox{Charge density.} \\
      &= c \rho & \quad \mbox{(current density dimensions.)} \\
\BJ &= J \wedge \gamma_0 & \quad \mbox{Current density spatial vector} \\
    &= \sum J^i \sigma_i \\
\partial_{\mu} &= \PDi{x^\mu}{} & \quad \mbox{Partial of index up coord.} \\
\grad &= \sum \gamma^{\mu} \partial/\partial {x^{\mu}} & \quad \mbox{Spacetime gradient} \\
      &= \sum \gamma^{\mu}\partial_{\mu} \\
      &= \sum \gamma_{\mu} \partial/\partial {x_{\mu}} \\
      &= \sum \gamma_{\mu}\partial^{\mu} \\
\spacegrad &= \sum \sigma^{i} \partial/\partial{x^{i}} & \quad \mbox{Spatial gradient} \\
           &= \sum \sigma^{i}\partial_{i} \\
\PV \IIinf &= \lim_{R\rightarrow \infty} \int_{R}^R & \quad \mbox{Integral Principle value} \\
\hat{A}(k) &= \FF(A(x)) & \quad \mbox{Fourier transform of $A$} \\ 
{A}(x) &= \FF^{-1}(A(k)) & \quad \mbox{Inverse Fourier transform} \\ 
\grad^2 A
   &= (\grad \cdot \grad) A & \quad \mbox{Four Laplacian. } \\
   &= (\partial_{00} - \sum_k \partial_{kk}) A & \\
x^2 &= x \cdot x & \quad \mbox{Four vector square. } \\
    &= x^\mu x_\mu \\
\Bx^2 &= \Bx \cdot \Bx & \quad \mbox{Spatial vector square. } \\
    &= {x^k}^2 \\
    &= \Abs{\Bx}^2 \\
d^4 x &= dx^0 dx^1 dx^2 dx^3 & \quad \mbox{Four volume element. } \\
d^3 x &= dx^1 dx^2 dx^3 & \quad \mbox{Spatial volume element. } \\
\exp(i\Bk\phi) &= 
\cos(\Abs{\Bk}\phi) + \frac{i \Bk}{\Abs{i\Bk}} \sin(\Abs{\Bk}\phi) & \quad \mbox{bivector exponential. } \\
\end{array}
\end{equation*}

Summation convention, where summation over all sets of matched upper and lower indexes is implied. %, will be in effect from this point on.

\bibliographystyle{plainnat}
\bibliography{myrefs}

\end{document}
