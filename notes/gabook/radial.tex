\chapter{Polar velocity and acceleration.}

\section{Motivation. }

Have previously worked out the radial velocity and acceleration components a pile of different ways in
\ref{chap:PJAngAcc}, 
\ref{chap:PJAngAccCross}, 
\ref{chap:PJAngVel}, 
\ref{chap:PJKeRot}, 
\ref{chap:PJRadialDer}, and
\ref{chap:PJUnitDer}.

So, what's a couple more?

When the motion is strictly restricted to a plane we can get away with doing this either in complex numbers
(used in a number of the Tong Lagrangian solutions), or with a polar form \R{2} vector (a polar representation
I haven't seen since High School).

\section{With complex numbers. }

Let
\begin{align*}
z = r e^{i\theta}
\end{align*}

So our velocity is

\begin{align*}
\zdot = \rdot e^{i\theta} + i r \thetadot e^{i\theta}
\end{align*}

and the acceleration is
\begin{align*}
\ddot{z}
&= \ddot{r} e^{i\theta} + i \dot{r} \thetadot e^{i\theta}
 + i \rdot \thetadot e^{i\theta}
 + i r \ddot{\theta} e^{i\theta}
 - r \thetadot^2 e^{i\theta} \\
&= (\ddot{r} - r \thetadot^2 ) e^{i\theta} + (2 \dot{r} \thetadot + r \ddot{\theta} ) i e^{i\theta}
\end{align*}

\section{Plane vector representation. }

Also can do this with polar vector representation directly (without involving the complexity of rotation matrices or anything fancy)

\begin{align*}
\Br 
&= r 
\begin{bmatrix}
\cos\theta \\
\sin\theta
\end{bmatrix}
\end{align*}

Velocity is then
\begin{align*}
\Bv 
&= 
\rdot 
\begin{bmatrix}
\cos\theta \\
\sin\theta
\end{bmatrix}
+r \thetadot
\begin{bmatrix}
-\sin\theta \\
\cos\theta
\end{bmatrix}
\end{align*}

and for acceleration we have

\begin{align*}
\Ba 
&= 
\ddot{r}
\begin{bmatrix}
\cos\theta \\
\sin\theta
\end{bmatrix}
+\rdot \thetadot
\begin{bmatrix}
-\sin\theta \\
\cos\theta
\end{bmatrix}
+\rdot \thetadot
\begin{bmatrix}
-\sin\theta \\
\cos\theta
\end{bmatrix}
+r \ddot{\theta}
\begin{bmatrix}
-\sin\theta \\
\cos\theta
\end{bmatrix}
-r \thetadot^2
\begin{bmatrix}
\cos\theta \\
\sin\theta 
\end{bmatrix} \\
&=
(\ddot{r} -r \thetadot^2)
\begin{bmatrix}
\cos\theta \\
\sin\theta
\end{bmatrix}
+(2\rdot \thetadot +r \ddot{\theta})
\begin{bmatrix}
-\sin\theta \\
\cos\theta
\end{bmatrix}
\end{align*}
