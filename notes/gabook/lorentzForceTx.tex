\documentclass{article}

\usepackage{amsmath}
\usepackage{mathpazo}

%
% shorthand for bold symbols, convenient for vectors and matrices
%
\newcommand{\Ba}[0]{\mathbf{a}}
\newcommand{\Bb}[0]{\mathbf{b}}
\newcommand{\Bc}[0]{\mathbf{c}}
\newcommand{\Bd}[0]{\mathbf{d}}
\newcommand{\Be}[0]{\mathbf{e}}
\newcommand{\Bf}[0]{\mathbf{f}}
\newcommand{\Bg}[0]{\mathbf{g}}
\newcommand{\Bh}[0]{\mathbf{h}}
\newcommand{\Bi}[0]{\mathbf{i}}
\newcommand{\Bj}[0]{\mathbf{j}}
\newcommand{\Bk}[0]{\mathbf{k}}
\newcommand{\Bl}[0]{\mathbf{l}}
\newcommand{\Bm}[0]{\mathbf{m}}
\newcommand{\Bn}[0]{\mathbf{n}}
\newcommand{\Bo}[0]{\mathbf{o}}
\newcommand{\Bp}[0]{\mathbf{p}}
\newcommand{\Bq}[0]{\mathbf{q}}
\newcommand{\Br}[0]{\mathbf{r}}
\newcommand{\Bs}[0]{\mathbf{s}}
\newcommand{\Bt}[0]{\mathbf{t}}
\newcommand{\Bu}[0]{\mathbf{u}}
\newcommand{\Bv}[0]{\mathbf{v}}
\newcommand{\Bw}[0]{\mathbf{w}}
\newcommand{\Bx}[0]{\mathbf{x}}
\newcommand{\By}[0]{\mathbf{y}}
\newcommand{\Bz}[0]{\mathbf{z}}
\newcommand{\BA}[0]{\mathbf{A}}
\newcommand{\BB}[0]{\mathbf{B}}
\newcommand{\BC}[0]{\mathbf{C}}
\newcommand{\BD}[0]{\mathbf{D}}
\newcommand{\BE}[0]{\mathbf{E}}
\newcommand{\BF}[0]{\mathbf{F}}
\newcommand{\BG}[0]{\mathbf{G}}
\newcommand{\BH}[0]{\mathbf{H}}
\newcommand{\BI}[0]{\mathbf{I}}
\newcommand{\BJ}[0]{\mathbf{J}}
\newcommand{\BK}[0]{\mathbf{K}}
\newcommand{\BL}[0]{\mathbf{L}}
\newcommand{\BM}[0]{\mathbf{M}}
\newcommand{\BN}[0]{\mathbf{N}}
\newcommand{\BO}[0]{\mathbf{O}}
\newcommand{\BP}[0]{\mathbf{P}}
\newcommand{\BQ}[0]{\mathbf{Q}}
\newcommand{\BR}[0]{\mathbf{R}}
\newcommand{\BS}[0]{\mathbf{S}}
\newcommand{\BT}[0]{\mathbf{T}}
\newcommand{\BU}[0]{\mathbf{U}}
\newcommand{\BV}[0]{\mathbf{V}}
\newcommand{\BW}[0]{\mathbf{W}}
\newcommand{\BX}[0]{\mathbf{X}}
\newcommand{\BY}[0]{\mathbf{Y}}
\newcommand{\BZ}[0]{\mathbf{Z}}

\newcommand{\Bzero}[0]{\mathbf{0}}
\newcommand{\Btheta}[0]{\boldsymbol{\theta}}
\newcommand{\Btau}[0]{\boldsymbol{\tau}}
\newcommand{\Bomega}[0]{\boldsymbol{\omega}}

%
% shorthand for unit vectors
%
\newcommand{\acap}[0]{\hat{\Ba}}
\newcommand{\bcap}[0]{\hat{\Bb}}
\newcommand{\ccap}[0]{\hat{\Bc}}
\newcommand{\dcap}[0]{\hat{\Bd}}
\newcommand{\ecap}[0]{\hat{\Be}}
\newcommand{\fcap}[0]{\hat{\Bf}}
\newcommand{\gcap}[0]{\hat{\Bg}}
\newcommand{\hcap}[0]{\hat{\Bh}}
\newcommand{\icap}[0]{\hat{\Bi}}
\newcommand{\jcap}[0]{\hat{\Bj}}
\newcommand{\kcap}[0]{\hat{\Bk}}
\newcommand{\lcap}[0]{\hat{\Bl}}
\newcommand{\mcap}[0]{\hat{\Bm}}
\newcommand{\ncap}[0]{\hat{\Bn}}
\newcommand{\ocap}[0]{\hat{\Bo}}
\newcommand{\pcap}[0]{\hat{\Bp}}
\newcommand{\qcap}[0]{\hat{\Bq}}
\newcommand{\rcap}[0]{\hat{\Br}}
\newcommand{\scap}[0]{\hat{\Bs}}
\newcommand{\tcap}[0]{\hat{\Bt}}
\newcommand{\ucap}[0]{\hat{\Bu}}
\newcommand{\vcap}[0]{\hat{\Bv}}
\newcommand{\wcap}[0]{\hat{\Bw}}
\newcommand{\xcap}[0]{\hat{\Bx}}
\newcommand{\ycap}[0]{\hat{\By}}
\newcommand{\zcap}[0]{\hat{\Bz}}
\newcommand{\thetacap}[0]{\hat{\Btheta}}

%
% to write R^n and C^n in a distinguishable fashion.  Perhaps change this
% to the double lined characters upon figuring out how to do so.
%
\newcommand{\C}[1]{$\mathbb{C}^{#1}$}
\newcommand{\R}[1]{$\mathbb{R}^{#1}$}

%
% various generally useful helpers
%

% derivative of #1 wrt. #2:
\newcommand{\D}[2] {\frac {d#2} {d#1}}

\newcommand{\inv}[1]{\frac{1}{#1}}
\newcommand{\cross}[0]{\times}

\newcommand{\abs}[1]{\lvert{#1}\rvert}
\newcommand{\norm}[1]{\lVert{#1}\rVert}
\newcommand{\innerprod}[2]{\langle{#1}, {#2}\rangle}
\newcommand{\dotprod}[2]{{#1} \cdot {#2}}
\newcommand{\bdotprod}[2]{\left({#1} \cdot {#2}\right)}
\newcommand{\crossprod}[2]{{#1} \cross {#2}}
\newcommand{\tripleprod}[3]{\dotprod{\left(\crossprod{#1}{#2}\right)}{#3}}

\DeclareMathOperator{\Proj}{Proj}
\DeclareMathOperator{\Span}{span}
\DeclareMathOperator{\Sgn}{sgn}
\DeclareMathOperator{\Area}{Area}
\DeclareMathOperator{\Volume}{Volume}

%
% A few miscellaneous things specific to this document
%
\newcommand{\crossop}[1]{\crossprod{#1}{}}

% R2 vector.
\newcommand{\VectorTwo}[2]{
\begin{bmatrix}
 {#1} \\
 {#2}
\end{bmatrix}
}

\newcommand{\VectorN}[1]{
\begin{bmatrix}
{#1}_1 \\
{#1}_2 \\
\vdots \\
{#1}_N \\
\end{bmatrix}
}

\newcommand{\DETuvij}[4]{
\begin{vmatrix}
 {#1}_{#3} & {#1}_{#4} \\
 {#2}_{#3} & {#2}_{#4}
\end{vmatrix}
}

\newcommand{\DETuvwijk}[6]{
\begin{vmatrix}
 {#1}_{#4} & {#1}_{#5} & {#1}_{#6} \\
 {#2}_{#4} & {#2}_{#5} & {#2}_{#6} \\
 {#3}_{#4} & {#3}_{#5} & {#3}_{#6}
\end{vmatrix}
}

\newcommand{\DETuvwxijkl}[8]{
\begin{vmatrix}
 {#1}_{#5} & {#1}_{#6} & {#1}_{#7} & {#1}_{#8} \\
 {#2}_{#5} & {#2}_{#6} & {#2}_{#7} & {#2}_{#8} \\
 {#3}_{#5} & {#3}_{#6} & {#3}_{#7} & {#3}_{#8} \\
 {#4}_{#5} & {#4}_{#6} & {#4}_{#7} & {#4}_{#8} \\
\end{vmatrix}
}

%\newcommand{\DETuvwxyijklm}[10]{
%\begin{vmatrix}
% {#1}_{#6} & {#1}_{#7} & {#1}_{#8} & {#1}_{#9} & {#1}_{#10} \\
% {#2}_{#6} & {#2}_{#7} & {#2}_{#8} & {#2}_{#9} & {#2}_{#10} \\
% {#3}_{#6} & {#3}_{#7} & {#3}_{#8} & {#3}_{#9} & {#3}_{#10} \\
% {#4}_{#6} & {#4}_{#7} & {#4}_{#8} & {#4}_{#9} & {#4}_{#10} \\
% {#5}_{#6} & {#5}_{#7} & {#5}_{#8} & {#5}_{#9} & {#5}_{#10}
%\end{vmatrix}
%}

% R3 vector.
\newcommand{\VectorThree}[3]{
\begin{bmatrix}
 {#1} \\
 {#2} \\
 {#3}
\end{bmatrix}
}


%<misc>
%
\newcommand{\Abs}[1]{{\left\lvert{#1}\right\rvert}}
\newcommand{\spacegrad}[0]{\boldsymbol{\nabla}}
\newcommand{\grad}[0]{\nabla}
\newcommand{\LL}[0]{\mathcal{L}}

% == \partial_{#1} {#2}
\newcommand{\PD}[2]{\frac{\partial {#2}}{\partial {#1}}}
% inline variant
\newcommand{\PDi}[2]{{\partial {#2}}/{\partial {#1}}}

\newcommand{\PDD}[3]{\frac{\partial^2 {#3}}{\partial {#1}\partial {#2}}}
%\newcommand{\PDd}[2]{\frac{\partial^2 {#2}}{{\partial{#1}}^2}}
\newcommand{\PDsq}[2]{\frac{\partial^2 {#2}}{(\partial {#1})^2}}

\newcommand{\Partial}[2]{\frac{\partial {#1}}{\partial {#2}}}
\DeclareMathOperator{\RejName}{Rej}
\newcommand{\Rej}[2]{\RejName_{#1}\left( {#2} \right)}
\newcommand{\Rm}[1]{\mathbb{R}^{#1}}
\newcommand{\Cm}[1]{\mathbb{C}^{#1}}
\newcommand{\conj}[0]{{*}}

%</misc>

% <grade selection>
%
\newcommand{\gpgrade}[2] {{\left\langle{{#1}}\right\rangle}_{#2}}

\newcommand{\gpgradezero}[1] {\gpgrade{#1}{}}
%\newcommand{\gpscalargrade}[1] {{\left\langle{{#1}}\right\rangle}}
%\newcommand{\gpgradezero}[1] {\gpgrade{#1}{0}}

%\newcommand{\gpgradeone}[1] {{\left\langle{{#1}}\right\rangle}_{1}}
\newcommand{\gpgradeone}[1] {\gpgrade{#1}{1}}

\newcommand{\gpgradetwo}[1] {\gpgrade{#1}{2}}
\newcommand{\gpgradethree}[1] {\gpgrade{#1}{3}}
\newcommand{\gpgradefour}[1] {\gpgrade{#1}{4}}
%
% </grade selection>



\newcommand{\adot}[0]{{\dot{a}}}
\newcommand{\bdot}[0]{{\dot{b}}}
% taken for centered dot:
%\newcommand{\cdot}[0]{{\dot{c}}}
%\newcommand{\ddot}[0]{{\dot{d}}}
\newcommand{\edot}[0]{{\dot{e}}}
\newcommand{\fdot}[0]{{\dot{f}}}
\newcommand{\gdot}[0]{{\dot{g}}}
\newcommand{\hdot}[0]{{\dot{h}}}
\newcommand{\idot}[0]{{\dot{i}}}
\newcommand{\jdot}[0]{{\dot{j}}}
\newcommand{\kdot}[0]{{\dot{k}}}
\newcommand{\ldot}[0]{{\dot{l}}}
\newcommand{\mdot}[0]{{\dot{m}}}
\newcommand{\ndot}[0]{{\dot{n}}}
%\newcommand{\odot}[0]{{\dot{o}}}
\newcommand{\pdot}[0]{{\dot{p}}}
\newcommand{\qdot}[0]{{\dot{q}}}
\newcommand{\rdot}[0]{{\dot{r}}}
\newcommand{\sdot}[0]{{\dot{s}}}
\newcommand{\tdot}[0]{{\dot{t}}}
\newcommand{\udot}[0]{{\dot{u}}}
\newcommand{\vdot}[0]{{\dot{v}}}
\newcommand{\wdot}[0]{{\dot{w}}}
\newcommand{\xdot}[0]{{\dot{x}}}
\newcommand{\ydot}[0]{{\dot{y}}}
\newcommand{\zdot}[0]{{\dot{z}}}
\newcommand{\addot}[0]{{\ddot{a}}}
\newcommand{\bddot}[0]{{\ddot{b}}}
\newcommand{\cddot}[0]{{\ddot{c}}}
%\newcommand{\dddot}[0]{{\ddot{d}}}
\newcommand{\eddot}[0]{{\ddot{e}}}
\newcommand{\fddot}[0]{{\ddot{f}}}
\newcommand{\gddot}[0]{{\ddot{g}}}
\newcommand{\hddot}[0]{{\ddot{h}}}
\newcommand{\iddot}[0]{{\ddot{i}}}
\newcommand{\jddot}[0]{{\ddot{j}}}
\newcommand{\kddot}[0]{{\ddot{k}}}
\newcommand{\lddot}[0]{{\ddot{l}}}
\newcommand{\mddot}[0]{{\ddot{m}}}
\newcommand{\nddot}[0]{{\ddot{n}}}
\newcommand{\oddot}[0]{{\ddot{o}}}
\newcommand{\pddot}[0]{{\ddot{p}}}
\newcommand{\qddot}[0]{{\ddot{q}}}
\newcommand{\rddot}[0]{{\ddot{r}}}
\newcommand{\sddot}[0]{{\ddot{s}}}
\newcommand{\tddot}[0]{{\ddot{t}}}
\newcommand{\uddot}[0]{{\ddot{u}}}
\newcommand{\vddot}[0]{{\ddot{v}}}
\newcommand{\wddot}[0]{{\ddot{w}}}
\newcommand{\xddot}[0]{{\ddot{x}}}
\newcommand{\yddot}[0]{{\ddot{y}}}
\newcommand{\zddot}[0]{{\ddot{z}}}

%<bold and dot greek symbols>
%

\newcommand{\Deltadot}[0]{{\dot{\Delta}}}
\newcommand{\Gammadot}[0]{{\dot{\Gamma}}}
\newcommand{\Lambdadot}[0]{{\dot{\Lambda}}}
\newcommand{\Omegadot}[0]{{\dot{\Omega}}}
\newcommand{\Phidot}[0]{{\dot{\Phi}}}
\newcommand{\Pidot}[0]{{\dot{\Pi}}}
\newcommand{\Psidot}[0]{{\dot{\Psi}}}
\newcommand{\Sigmadot}[0]{{\dot{\Sigma}}}
\newcommand{\Thetadot}[0]{{\dot{\Theta}}}
\newcommand{\Upsilondot}[0]{{\dot{\Upsilon}}}
\newcommand{\Xidot}[0]{{\dot{\Xi}}}
\newcommand{\alphadot}[0]{{\dot{\alpha}}}
\newcommand{\betadot}[0]{{\dot{\beta}}}
\newcommand{\chidot}[0]{{\dot{\chi}}}
\newcommand{\deltadot}[0]{{\dot{\delta}}}
\newcommand{\epsilondot}[0]{{\dot{\epsilon}}}
\newcommand{\etadot}[0]{{\dot{\eta}}}
\newcommand{\gammadot}[0]{{\dot{\gamma}}}
\newcommand{\kappadot}[0]{{\dot{\kappa}}}
\newcommand{\lambdadot}[0]{{\dot{\lambda}}}
\newcommand{\mudot}[0]{{\dot{\mu}}}
\newcommand{\nudot}[0]{{\dot{\nu}}}
\newcommand{\omegadot}[0]{{\dot{\omega}}}
\newcommand{\phidot}[0]{{\dot{\phi}}}
\newcommand{\pidot}[0]{{\dot{\pi}}}
\newcommand{\psidot}[0]{{\dot{\psi}}}
\newcommand{\rhodot}[0]{{\dot{\rho}}}
\newcommand{\sigmadot}[0]{{\dot{\sigma}}}
\newcommand{\taudot}[0]{{\dot{\tau}}}
\newcommand{\thetadot}[0]{{\dot{\theta}}}
\newcommand{\upsilondot}[0]{{\dot{\upsilon}}}
\newcommand{\varepsilondot}[0]{{\dot{\varepsilon}}}
\newcommand{\varphidot}[0]{{\dot{\varphi}}}
\newcommand{\varpidot}[0]{{\dot{\varpi}}}
\newcommand{\varrhodot}[0]{{\dot{\varrho}}}
\newcommand{\varsigmadot}[0]{{\dot{\varsigma}}}
\newcommand{\varthetadot}[0]{{\dot{\vartheta}}}
\newcommand{\xidot}[0]{{\dot{\xi}}}
\newcommand{\zetadot}[0]{{\dot{\zeta}}}

\newcommand{\Deltaddot}[0]{{\ddot{\Delta}}}
\newcommand{\Gammaddot}[0]{{\ddot{\Gamma}}}
\newcommand{\Lambdaddot}[0]{{\ddot{\Lambda}}}
\newcommand{\Omegaddot}[0]{{\ddot{\Omega}}}
\newcommand{\Phiddot}[0]{{\ddot{\Phi}}}
\newcommand{\Piddot}[0]{{\ddot{\Pi}}}
\newcommand{\Psiddot}[0]{{\ddot{\Psi}}}
\newcommand{\Sigmaddot}[0]{{\ddot{\Sigma}}}
\newcommand{\Thetaddot}[0]{{\ddot{\Theta}}}
\newcommand{\Upsilonddot}[0]{{\ddot{\Upsilon}}}
\newcommand{\Xiddot}[0]{{\ddot{\Xi}}}
\newcommand{\alphaddot}[0]{{\ddot{\alpha}}}
\newcommand{\betaddot}[0]{{\ddot{\beta}}}
\newcommand{\chiddot}[0]{{\ddot{\chi}}}
\newcommand{\deltaddot}[0]{{\ddot{\delta}}}
\newcommand{\epsilonddot}[0]{{\ddot{\epsilon}}}
\newcommand{\etaddot}[0]{{\ddot{\eta}}}
\newcommand{\gammaddot}[0]{{\ddot{\gamma}}}
\newcommand{\kappaddot}[0]{{\ddot{\kappa}}}
\newcommand{\lambdaddot}[0]{{\ddot{\lambda}}}
\newcommand{\muddot}[0]{{\ddot{\mu}}}
\newcommand{\nuddot}[0]{{\ddot{\nu}}}
\newcommand{\omegaddot}[0]{{\ddot{\omega}}}
\newcommand{\phiddot}[0]{{\ddot{\phi}}}
\newcommand{\piddot}[0]{{\ddot{\pi}}}
\newcommand{\psiddot}[0]{{\ddot{\psi}}}
\newcommand{\rhoddot}[0]{{\ddot{\rho}}}
\newcommand{\sigmaddot}[0]{{\ddot{\sigma}}}
\newcommand{\tauddot}[0]{{\ddot{\tau}}}
\newcommand{\thetaddot}[0]{{\ddot{\theta}}}
\newcommand{\upsilonddot}[0]{{\ddot{\upsilon}}}
\newcommand{\varepsilonddot}[0]{{\ddot{\varepsilon}}}
\newcommand{\varphiddot}[0]{{\ddot{\varphi}}}
\newcommand{\varpiddot}[0]{{\ddot{\varpi}}}
\newcommand{\varrhoddot}[0]{{\ddot{\varrho}}}
\newcommand{\varsigmaddot}[0]{{\ddot{\varsigma}}}
\newcommand{\varthetaddot}[0]{{\ddot{\vartheta}}}
\newcommand{\xiddot}[0]{{\ddot{\xi}}}
\newcommand{\zetaddot}[0]{{\ddot{\zeta}}}

\newcommand{\BDelta}[0]{\boldsymbol{\Delta}}
\newcommand{\BGamma}[0]{\boldsymbol{\Gamma}}
\newcommand{\BLambda}[0]{\boldsymbol{\Lambda}}
\newcommand{\BOmega}[0]{\boldsymbol{\Omega}}
\newcommand{\BPhi}[0]{\boldsymbol{\Phi}}
\newcommand{\BPi}[0]{\boldsymbol{\Pi}}
\newcommand{\BPsi}[0]{\boldsymbol{\Psi}}
\newcommand{\BSigma}[0]{\boldsymbol{\Sigma}}
\newcommand{\BTheta}[0]{\boldsymbol{\Theta}}
\newcommand{\BUpsilon}[0]{\boldsymbol{\Upsilon}}
\newcommand{\BXi}[0]{\boldsymbol{\Xi}}
\newcommand{\Balpha}[0]{\boldsymbol{\alpha}}
\newcommand{\Bbeta}[0]{\boldsymbol{\beta}}
\newcommand{\Bchi}[0]{\boldsymbol{\chi}}
\newcommand{\Bdelta}[0]{\boldsymbol{\delta}}
\newcommand{\Bepsilon}[0]{\boldsymbol{\epsilon}}
\newcommand{\Beta}[0]{\boldsymbol{\eta}}
\newcommand{\Bgamma}[0]{\boldsymbol{\gamma}}
\newcommand{\Bkappa}[0]{\boldsymbol{\kappa}}
\newcommand{\Blambda}[0]{\boldsymbol{\lambda}}
\newcommand{\Bmu}[0]{\boldsymbol{\mu}}
\newcommand{\Bnu}[0]{\boldsymbol{\nu}}
%\newcommand{\Bomega}[0]{\boldsymbol{\omega}}
\newcommand{\Bphi}[0]{\boldsymbol{\phi}}
\newcommand{\Bpi}[0]{\boldsymbol{\pi}}
\newcommand{\Bpsi}[0]{\boldsymbol{\psi}}
\newcommand{\Brho}[0]{\boldsymbol{\rho}}
\newcommand{\Bsigma}[0]{\boldsymbol{\sigma}}
%\newcommand{\Btau}[0]{\boldsymbol{\tau}}
%\newcommand{\Btheta}[0]{\boldsymbol{\theta}}
\newcommand{\Bupsilon}[0]{\boldsymbol{\upsilon}}
\newcommand{\Bvarepsilon}[0]{\boldsymbol{\varepsilon}}
\newcommand{\Bvarphi}[0]{\boldsymbol{\varphi}}
\newcommand{\Bvarpi}[0]{\boldsymbol{\varpi}}
\newcommand{\Bvarrho}[0]{\boldsymbol{\varrho}}
\newcommand{\Bvarsigma}[0]{\boldsymbol{\varsigma}}
\newcommand{\Bvartheta}[0]{\boldsymbol{\vartheta}}
\newcommand{\Bxi}[0]{\boldsymbol{\xi}}
\newcommand{\Bzeta}[0]{\boldsymbol{\zeta}}
%
%</bold and dot greek symbols>
%<infrequent>
%
%\newcommand{\AreaOp}[1]{\AName_{#1}}
%\newcommand{\Babs}[0]{\abs{\BB}}
%\newcommand{\Bcap}[0]{\hat{\BB}}
%\newcommand{\BrPrimeRej}[0]{\rcap(\rcap \wedge \Br')}
%\newcommand{\CA}[0]{\mathcal{A}}
%\newcommand{\Cos}[1]{\cos{\left({#1}\right)}}
%\newcommand{\Det}[1] {\abs{#1}}
%\newcommand{\Dsq}[2] {\frac {\partial^2 {#1}} {\partial {#2}^2}}
%\newcommand{\Exp}[1]{\exp{\left({#1}\right)}}
%\newcommand{\Norm}[1]{\left\lVert{#1}\right\rVert}
%\newcommand{\Sin}[1]{\sin{\left({#1}\right)}}
%\newcommand{\T}[0]{\text{T}}
%\newcommand{\VolumeOp}[1]{\VName_{#1}}
%\newcommand{\agrad}[0]{\Ba \cdot \nabla}
%\newcommand{\alphacap}[0]{\hat{\boldsymbol{\alpha}}}
%\newcommand{\Fcap}[0]{\hat{\BF}}
%\newcommand{\bithree}[0]{{\Bi}_3}
%\newcommand{\bxa}[0]{\Bx\Ba}
%\newcommand{\coordvec}[2]{
%\newcommand{\costheta}[0]{\acap \cdot \xcap}
%\newcommand{\ddt}[1]{\ddot{#1}}
%\newcommand{\ddu}[1] {\frac {d{#1}} {du}}
%\newcommand{\dsqxj}[2] {\frac {\partial^2 {#1}} {\partial {x_{#2}}^2}}
%\newcommand{\dtheta}[1]{\frac{d {#1}}{d \theta}}
%\newcommand{\dt}[1]{\dot{#1}}
%\newcommand{\dt}[1]{\frac{d {#1}}{dt}}
%\newcommand{\dxj}[2] {\frac {\partial {#1}} {\partial {x_{#2}}}}
%\newcommand{\halfPhi}[0]{\frac{\phi}{2}}
%\newcommand{\half}[0]{\inv{2}}
%\newcommand{\inv}[1]{\frac{1}{#1}}
%\newcommand{\laplacian}[0]{\nabla^2}
%\newcommand{\matrixoftx}[3]{
%\newcommand{\nrrp}[0]{\norm{\rcap \wedge \Br'}}
%\newcommand{\oiint}{\bigcirc \hspace{-1.4em} \int \hspace{-.8em} \int}
%\newcommand{\transpose}[1]{{#1}^{\text{T}}}
%\newcommand{\transpose}[1]{{{#1}^{\TextTranspose}}}
%\newcommand{\transpose}[1]{{{#1}^{\text{T}}}}
%\newcommand{\barA}[0]{\bar{A}}
%\newcommand{\qbar}[0]{\bar{q}}
%\newcommand{\qdotbar}[0]{\dot{\bar{q}}}
%
%</infrequent>





\newcommand{\EE}[0]{\boldsymbol{\mathcal{E}}}
\newcommand{\HH}[0]{\boldsymbol{\mathcal{H}}}
\newcommand{\Vcap}[0]{\hat{\BV}}

%\usepackage{listings}
%\usepackage{txfonts} % for ointctr... (also appears to make "prettier" \int and \sum's)
% makes \grad look funny though (almost like spacegrad, but narrower)
\usepackage[bookmarks=true]{hyperref}

\usepackage{color,cite,graphicx}
   % use colour in the document, put your citations as [1-4]
   % rather than [1,2,3,4] (it looks nicer, and the extended LaTeX2e
   % graphics package. 
\usepackage{latexsym,amssymb,epsf} % don't remember if these are
   % needed, but their inclusion can't do any damage


\title{ Lorentz boost of Lorentz force equations. }
\author{Peeter Joot \quad peeter.joot@gmail.com }
\date{ May 23, 2009.  Last Revision: $Date: 2009/05/26 05:34:48 $ }

\begin{document}

\maketitle{}
\tableofcontents
\section{ Motivation. }

Reading of \cite{bohm1996str} is a treatment of the Lorentz transform
properties of the Lorentz force equation.  This isn't clear to me
without working through it myself, so do this.

I also have the urge to 
try this with the GA formulation of the Lorentz transformation.  That may not end up being simpler
if one works with the non-covariant form of the Lorentz force equation, but only trying it will tell.

\section{ Compare forms of the Lorentz Boost. }

Working from the Geometric Algebra form of the Lorentz boost, show equivalence to the standard
coordinate matrix form and the vector form from Bohm.

\subsection{ Exponential form. }

Write the Lorentz boost of a four vector $x = x^\mu \gamma_\mu = ct \gamma_0 + x^k \gamma_k$ as

\begin{align}\label{eqn:LorentzBoost}
L(x) &= 
e^{-\alpha \vcap/2}
x
e^{\alpha \vcap/2}
\end{align}

\subsection{ Invariance property. }

A Lorentz transformation (boost or rotation) can be defined as those transformation that leave the four vector square unchanged.

Following \cite{doran2003gap}, work with a $+---$ metric signature ($1 = \gamma_0^2 = -\gamma_k^2$), and $\sigma_k = \gamma_k \gamma_0$.  Our four vector square in this representation has the familiar invariant form

\begin{align*}
x^2 
&= (ct \gamma_0 + x^m \gamma_m) (ct \gamma_0 + x^k \gamma_k) \\
&= (ct \gamma_0 + x^m \gamma_m) \gamma_0^2 (ct \gamma_0 + x^k \gamma_k) \\
&= (ct + x^m \sigma_m) (ct - x^k \sigma_k) \\
&= (ct + \Bx) (ct - \Bx) \\
&= (ct)^2 - \Bx^2
\end{align*}

and we expect this of the Lorentz boost of equation \ref{eqn:LorentzBoost}.  To verify we have

\begin{align*}
L(x)^2 
&=
e^{-\alpha \vcap/2}
x
e^{\alpha \vcap/2}
e^{-\alpha \vcap/2}
x
e^{\alpha \vcap/2} \\
&=
e^{-\alpha \vcap/2}
x
x
e^{\alpha \vcap/2} \\
&=
x^2
e^{-\alpha \vcap/2}
e^{\alpha \vcap/2} \\
&=
x^2
\end{align*}

\subsection{ Sign of the rapidity angle. }

The factor $\alpha$ will be the rapidity angle, but what sign do we want for a boost along the positive $\vcap$ direction?

Dropping to coordinates is an easy way to determine the sign convention in effect.  Write $\vcap = \sigma_1$

\begin{align*}
L(x) &= 
e^{-\alpha \vcap/2}
x
e^{\alpha \vcap/2} \\
&=
(\cosh(\alpha/2) - \sigma_1 \sinh(\alpha/2))
(
x^0 \gamma_0
+x^1 \gamma_1
+x^2 \gamma_2
+x^3 \gamma_3
)
(\cosh(\alpha/2) + \sigma_1 \sinh(\alpha/2))
\end{align*}

$\sigma_1$ commutes with $\gamma_2$ and $\gamma_3$ and anticommutes otherwise, so we have

\begin{align*}
L(x) &= 
(
x^2 \gamma_2
+x^3 \gamma_3
) 
e^{-\alpha \vcap/2}
e^{\alpha \vcap/2}
+
(
x^0 \gamma_0
+x^1 \gamma_1
)
e^{\alpha \vcap} \\
&=
x^2 \gamma_2
+x^3 \gamma_3
+(
x^0 \gamma_0
+x^1 \gamma_1
)
e^{\alpha \vcap} \\
&=
x^2 \gamma_2
+x^3 \gamma_3
+(
x^0 \gamma_0
+x^1 \gamma_1
)
(\cosh(\alpha) + \sigma_1 \sinh(\alpha))
\end{align*}

Expanding out just the $0,1$ terms changed by the transformation we have

\begin{align*}
(
&x^0 \gamma_0
+x^1 \gamma_1
)
(\cosh(\alpha) + \sigma_1 \sinh(\alpha)) \\
&=
x^0 \gamma_0 \cosh(\alpha) 
+x^1 \gamma_1 \cosh(\alpha) 
+x^0 \gamma_0 \sigma_1 \sinh(\alpha)
+x^1 \gamma_1 \sigma_1 \sinh(\alpha) \\
&=
x^0 \gamma_0 \cosh(\alpha) 
+x^1 \gamma_1 \cosh(\alpha) 
+x^0 \gamma_0 \gamma_1 \gamma_0 \sinh(\alpha)
+x^1 \gamma_1 \gamma_1 \gamma_0 \sinh(\alpha) \\
&=
x^0 \gamma_0 \cosh(\alpha) 
+x^1 \gamma_1 \cosh(\alpha) 
-x^0 \gamma_1 \sinh(\alpha)
-x^1 \gamma_0 \sinh(\alpha) \\
&=
\gamma_0 (x^0 \cosh(\alpha) -x^1 \sinh(\alpha) )
+\gamma_1 (x^1 \cosh(\alpha) -x^0 \sinh(\alpha) )
\\
\end{align*}

Writing ${x^\mu}' = L(x) \cdot \gamma^\mu$, and $x^\mu = x \cdot \gamma^\mu$,
and a substitution of $\cosh(\alpha) = 1/\sqrt{1 - \Bv^2/c^2}$, and $\alpha \vcap = \tanh^{-1}(\Bv/c)$,
we have the traditional coordinate
expression for the one directional Lorentz boost

\begin{align}
\begin{bmatrix}
{x^0}' \\
{x^1}' \\
{x^2}' \\
{x^3}'
\end{bmatrix}
&=
\begin{bmatrix}
\cosh\alpha & -\sinh\alpha & 0 & 0 \\
-\sinh\alpha & \cosh\alpha & 0 & 0 \\
0 & 0 & 1 & 0 \\
0 & 0 & 0 & 1 \\
\end{bmatrix}
\begin{bmatrix}
x^0 \\
x^1 \\
x^2 \\
x^3
\end{bmatrix}
\end{align}

Performing this expansion showed initially showed that I had the wrong sign for $\alpha$ in the exponentials and I went back and
adjusted it all accordingly.

\subsection{ Expanding out the Lorentz boost for projective and rejective directions. }

Two forms of Lorentz boost representations have been compared above.  An additional one is used in the Bohm text (a 
vector form of the Lorentz transformation not using coordinates).  Let's see
if we can derive that from the exponential form.

Start with computation of components of a four vector relative to an observer timelike unit vector $\gamma_0$.

\begin{align*}
x 
&= x \gamma_0 \gamma_0 \\
&= (x \gamma_0) \gamma_0 \\
&= (x \cdot \gamma_0 + x \wedge \gamma_0) \gamma_0 \\
\end{align*}

For the spatial vector factor above write $\Bx = x \wedge \gamma_0$, for

\begin{align*}
x 
&= (x \cdot \gamma_0) \gamma_0 + \Bx \gamma_0 \\
&= (x \cdot \gamma_0) \gamma_0 + \Bx \vcap \vcap \gamma_0 \\
&= (x \cdot \gamma_0) \gamma_0 + (\Bx \cdot \vcap) \vcap \gamma_0 + (\Bx \wedge \vcap) \vcap \gamma_0 \\
\end{align*}

We have the following commutation relations for the various components
\begin{align*}
\vcap (\gamma_0) &= - \gamma_0 \vcap \\
\vcap (\vcap \gamma_0) &= - (\vcap \gamma_0) \vcap \\
\vcap ((\Bx \wedge \vcap) \vcap \gamma_0 ) 
%&= -(\Bx \wedge \vcap) \vcap \vcap \gamma_0 
&= ((\Bx \wedge \vcap) \vcap \gamma_0) \vcap
\end{align*}

For a four vector $u$ that commutes with $\vcap$ we have $e^{-\alpha \vcap/2} u = u e^{-\alpha \vcap/2}$, and if it anticommutes
we have the conjugate relation
$e^{-\alpha \vcap/2} u = u e^{\alpha \vcap/2}$.  This gives us

\begin{align*}
L(x) 
&= 
(\Bx \wedge \vcap) \vcap \gamma_0 +
\left( (x \cdot \gamma_0) \gamma_0 + (\Bx \cdot \vcap) \vcap \gamma_0 \right) e^{\alpha \vcap} \\
\end{align*}

Now write the exponential as a scalar and spatial vector sum
\begin{align*}
e^{\alpha \vcap}
&= 
\cosh\alpha 
+\vcap \sinh\alpha 
\\
&= 
\gamma (1 +\vcap \tanh\alpha )
\\
&= 
\gamma (1 +\vcap \beta)
\\
&= 
\gamma (1 + \Bv/c )
\\
\end{align*}

Expanding out the exponential product above, also writing $x^0 = ct = x \cdot \gamma_0$, we have

\begin{align*}
( &x^0 \gamma_0 + (\Bx \cdot \vcap) \vcap \gamma_0 ) e^{\alpha \vcap} \\
&=
\gamma ( x^0 \gamma_0 + (\Bx \cdot \vcap) \vcap \gamma_0 ) ( 1 + \Bv/c ) \\
&=
\gamma ( 
x^0 \gamma_0 
+ (\Bx \cdot \vcap) \vcap \gamma_0 
+x^0 \gamma_0 \Bv/c  
+ (\Bx \cdot \vcap) \vcap \gamma_0 \Bv/c
) \\
\end{align*}

So for the total Lorentz boost in vector form we have

\begin{align}\label{eqn:fourVectorExpanded}
L(x)
&=
(\Bx \wedge \vcap) \vcap \gamma_0 +
\gamma \left(x^0 - \Bx \cdot \frac{\Bv}{c} \right) \gamma_0
+ \gamma \left( \Bx \cdot \inv{\Bv/c} - x^0 \right) \frac{\Bv}{c} \gamma_0
\end{align}

Now a visual inspection shows that this does match
equation (15-12) from the text:

\begin{align}\label{eqn:explicitSpaceAndTimeTransformed}
\Bx' 
&= \Bx - (\vcap \cdot \Bx) \vcap + \frac{ (\vcap \cdot \Bx )\vcap - \Bv t }{ \sqrt{1 - (v^2/c^2)} } \\
t' 
&= \frac{ t - (\Bv \cdot \Bx )/c^2 }{ \sqrt{1 - (v^2/c^2)} } 
\end{align}

but the equivalence of these is perhaps not so obvious without familiarity
with the GA constructs.

\subsection{ differential form. }

Bohm utilizes a vector differential form of the Lorentz transformation
for both the spacetime and energy-momentum vectors.  From equation
\ref{eqn:explicitSpaceAndTimeTransformed}
we can derive the expressions used.  In particular for the transformed
spatial component we have

\begin{align*}
\Bx' 
&= \Bx + \gamma \left( -(\vcap \cdot \Bx) \vcap \inv{\gamma} + (\vcap \cdot \Bx )\vcap - \Bv t \right) \\
&= \Bx + \gamma \left( (\vcap \cdot \Bx) \vcap \left(1-\inv{\gamma} \right) - \Bv t \right) \\
&= \Bx + (\gamma-1)(\vcap \cdot \Bx) \vcap - \gamma \Bv t \\
\end{align*}

So in differential vector form we have
\begin{align}\label{eqn:differentialSpaceTime}
d\Bx'
&= d\Bx + (\gamma-1)(\vcap \cdot d\Bx) \vcap - \gamma \Bv dt \\
dt' 
&= \gamma ( dt - (\Bv \cdot d\Bx )/c^2 )
\end{align}

and by analogy with $dx^0 = cdt \rightarrow dE/c$, and $d\Bx \rightarrow d\Bp$, we also have the energy momentum transformation 

\begin{align}\label{eqn:differentialEnergyMomentum}
d\Bp'
&= d\Bp + (\gamma-1)(\vcap \cdot d\Bp) \vcap - \gamma \Bv dE/c^2 \\
dE' 
&= \gamma ( dE - \Bv \cdot d\Bp )
\end{align}

Reflecting on these forms of the Lorentz transformation, they are quite
natural ways to express the vector results.  The terms with $\gamma$ factors
are exactly what we are used to in the coordinate representation (transformation
of only the time component and the projection of the spatial vector in the
velocity direction), while the $-1$ part of the $(\gamma-1)$ term just
subtracts off the projection unaltered, leaving
$d\Bx - (d\Bx \cdot \vcap) \vcap = (d\Bx \wedge \vcap) \vcap$, the rejection
from the $\vcap$ direction.

\section{ Lorentz force transformation. }

Preliminaries out of the way, now we want to examine the tranform of the electric and magnetic field as used in the Lorentz force equation.  In
csg units as in the text we have

\begin{align}
\frac{d\Bp}{dt} &= q \left( \EE + \frac{\Bv}{c} \cross \HH \right) \\
\frac{dE}{dt} &= q \EE \cdot \Bv
\end{align}

After writing this
in differential form
\begin{align}\label{eqn:untransformedLorentzForce}
d\Bp &= q \left( \EE dt + \frac{d\Bx}{c} \cross \HH \right) \\
dE &= q \EE \cdot d\Bx
\end{align}

and the transformed variation of this equation, also in differential form
\begin{align}\label{eqn:transformedLorentzForce}
d\Bp' &= q \left( \EE' dt' + \frac{d\Bx'}{c} \cross \HH' \right) \\
dE' &= q \EE' \cdot d\Bx'
\end{align}

A brute force insertion of the transform results of equations
\ref{eqn:differentialSpaceTime}, and \ref{eqn:differentialEnergyMomentum} into these is performed.  This is mostly
a mess of algebra.  If it wasn't for mechanical algebra and calculus physics
would be much easier!

While the Bohm book covers some of this, other parts are left for the reader.  Do the whole thing here as an exersize.

Begining the substuition into 
\ref{eqn:transformedLorentzForce} we have for a boost by velocity $\BV$, 

\begin{align*}
d\Bp &+ (\gamma-1)(\Vcap \cdot d\Bp) \Vcap - \gamma \BV dE/c^2 = \\
&q \left( \EE' \left(\gamma ( dt - (\Vcap \cdot d\Bx )/c^2 )\right) + \frac{d\Bx + (\gamma-1)(\Vcap \cdot d\Bx) \Vcap - \gamma \BV dt}{c} \cross \HH' \right)  \\
\end{align*}

and

\begin{align*}
\gamma ( dE - \BV \cdot d\Bp ) &= q \EE' \cdot \left(d\Bx + (\gamma-1)(\Vcap \cdot d\Bx) \Vcap - \gamma \BV dt\right)
\end{align*}

For the right hand sides a substitution for $d\Bp$ and $dE$ in terms of the fields using equation
\ref{eqn:untransformedLorentzForce} we have

%d\Bp &= q \left( \EE dt + \frac{d\Bx}{c} \cross \HH \right) \\
%dE &= q \EE \cdot d\Bx

\begin{align*}
d\Bp &+ (\gamma-1)(\Vcap \cdot d\Bp) \Vcap - \gamma \BV dE/c^2 = \\
&q \left( \EE dt + \frac{d\Bx}{c} \cross \HH \right) + (\gamma-1)\left(\Vcap \cdot \left(q \left( \EE dt + \frac{d\Bx}{c} \cross \HH \right)\right)\right) \Vcap - \gamma \BV \left( q \EE \cdot d\Bx \right)/c^2  \\
\gamma &( dE - \BV \cdot d\Bp ) = \\
&\gamma \left( q \EE \cdot d\Bx - \BV \cdot \left(q \left( \EE dt + \frac{d\Bx}{c} \cross \HH \right)\right) \right) 
\end{align*}

Combining the two sets of messes we have
\begin{align*}
q &\left( \EE dt + \frac{d\Bx}{c} \cross \HH \right) + (\gamma-1)\left(\Vcap \cdot \left(q \left( \EE dt + \frac{d\Bx}{c} \cross \HH \right)\right)\right) \Vcap - \gamma \BV \left( q \EE \cdot d\Bx \right)/c^2  \\
&=q \left( \EE' \left(\gamma ( dt - (\Vcap \cdot d\Bx )/c^2 )\right) + \frac{d\Bx + (\gamma-1)(\Vcap \cdot d\Bx) \Vcap - \gamma \BV dt}{c} \cross \HH' \right)  \\
\end{align*}

and
\begin{align*}
\gamma &\left( q \EE \cdot d\Bx - \BV \cdot \left(q \left( \EE dt + \frac{d\Bx}{c} \cross \HH \right)\right) \right) 
=
q \EE' \cdot \left(d\Bx + (\gamma-1)(\Vcap \cdot d\Bx) \Vcap - \gamma \BV dt\right)
\end{align*}

cancelling $q$

\begin{align*}
\EE &dt 
+ \inv{c} d\Bx \cross (\HH - \HH')
+ (\gamma-1)(\Vcap \cdot \EE) \Vcap dt 
+ \frac{\gamma-1}{c}\left(\Vcap \cdot \left( d\Bx \cross \HH \right)\right) \Vcap 
- \gamma \BV \left( \EE \cdot d\Bx \right)/c^2  \\
&=
\EE' \gamma dt 
-\EE' (\Vcap \cdot d\Bx )/c^2 
+ \frac{\gamma-1}{c} (\Vcap \cdot d\Bx) \Vcap \cross \HH' 
- \frac{\gamma-1}{c}\gamma dt \BV \cross \HH' 
\end{align*}

and for the energy equation using a cyclic permutation on the triple product $V \cdot (d\Bx \cross \HH)$ we have
\begin{align*}
d\Bx \cdot \left( \gamma \EE 
- \gamma \inv{c} (\HH \cross \BV)
-\EE' - (\gamma-1) \Vcap (\EE' \cdot \Vcap)
\right)
= \gamma \BV \cdot (\EE -\EE')dt 
\end{align*}

%Now we want a cyclic permutation of the triple product $\Vcap \cdot (d\Bx \cross \HH) = d\Bx \cdot (\HH \cross \Vcap)$, to prepare for grouping by $dt$ and $d\Bx$.  That is
%
%\begin{align*}
%\EE &dt 
%+ (\gamma-1)\left(
%\Vcap \cdot \EE dt 
%+\frac{d\Bx}{c} \cdot (\HH \cross \Vcap)
%\right) \Vcap - \gamma \BV \left( \EE \cdot d\Bx \right)/c^2  \\
%&=
%\EE' \left(\gamma ( dt - (\Vcap \cdot d\Bx )/c^2 )\right) + \frac{(\gamma-1)(\Vcap \cdot d\Bx) \Vcap - \gamma \BV dt}{c} \cross \HH' 
%\end{align*}
%
%and
%\begin{align*}
%\gamma &\left( \EE \cdot d\Bx - 
%\Vcap \cdot \EE dt 
%-\frac{d\Bx}{c} (\HH \cross \Vcap)
%\right) \\
%&=
%\EE' \cdot \left(d\Bx + (\gamma-1)(\Vcap \cdot d\Bx) \Vcap - \gamma \BV dt\right)
%\end{align*}
%
%Grouping by $d\Bx$ and $dt$ we have
%
%\begin{align*}
%dt &\left(
%\EE 
%+ (\gamma-1) \Vcap (\Vcap \cdot \EE)
%- \gamma \EE' 
%+ \gamma \BV \cross \HH'/c 
%\right)
%\\
%&=
%d\Bx \cdot \left(
%(\gamma-1) \Vcap/c (\Vcap \cross \HH)
%+ \gamma \BV/c^2 \EE 
%- \gamma \EE' \Vcap/c^2 
%\right)
%+ (\gamma-1)(\Vcap \cdot d\Bx) \Vcap \cross \HH'/c
%\end{align*}
%
%% FIXME: group by dt and dx
%%and
%%\begin{align*}
%%\gamma &\left( \EE \cdot d\Bx - 
%%\Vcap \cdot \EE dt 
%%-\frac{d\Bx}{c} (\HH \cross \Vcap)
%%\right) \\
%%&=
%%\EE' \cdot \left(d\Bx + (\gamma-1)(\Vcap \cdot d\Bx) \Vcap - \gamma \BV dt\right)
%%\end{align*}

\bibliographystyle{plainnat}
\bibliography{myrefs}

\end{document}
