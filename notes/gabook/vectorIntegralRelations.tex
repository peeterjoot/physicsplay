\documentclass{article}

\usepackage{amsmath}
\usepackage{mathpazo}

%
% shorthand for bold symbols, convenient for vectors and matrices
%
\newcommand{\Ba}[0]{\mathbf{a}}
\newcommand{\Bb}[0]{\mathbf{b}}
\newcommand{\Bc}[0]{\mathbf{c}}
\newcommand{\Bd}[0]{\mathbf{d}}
\newcommand{\Be}[0]{\mathbf{e}}
\newcommand{\Bf}[0]{\mathbf{f}}
\newcommand{\Bg}[0]{\mathbf{g}}
\newcommand{\Bh}[0]{\mathbf{h}}
\newcommand{\Bi}[0]{\mathbf{i}}
\newcommand{\Bj}[0]{\mathbf{j}}
\newcommand{\Bk}[0]{\mathbf{k}}
\newcommand{\Bl}[0]{\mathbf{l}}
\newcommand{\Bm}[0]{\mathbf{m}}
\newcommand{\Bn}[0]{\mathbf{n}}
\newcommand{\Bo}[0]{\mathbf{o}}
\newcommand{\Bp}[0]{\mathbf{p}}
\newcommand{\Bq}[0]{\mathbf{q}}
\newcommand{\Br}[0]{\mathbf{r}}
\newcommand{\Bs}[0]{\mathbf{s}}
\newcommand{\Bt}[0]{\mathbf{t}}
\newcommand{\Bu}[0]{\mathbf{u}}
\newcommand{\Bv}[0]{\mathbf{v}}
\newcommand{\Bw}[0]{\mathbf{w}}
\newcommand{\Bx}[0]{\mathbf{x}}
\newcommand{\By}[0]{\mathbf{y}}
\newcommand{\Bz}[0]{\mathbf{z}}
\newcommand{\BA}[0]{\mathbf{A}}
\newcommand{\BB}[0]{\mathbf{B}}
\newcommand{\BC}[0]{\mathbf{C}}
\newcommand{\BD}[0]{\mathbf{D}}
\newcommand{\BE}[0]{\mathbf{E}}
\newcommand{\BF}[0]{\mathbf{F}}
\newcommand{\BG}[0]{\mathbf{G}}
\newcommand{\BH}[0]{\mathbf{H}}
\newcommand{\BI}[0]{\mathbf{I}}
\newcommand{\BJ}[0]{\mathbf{J}}
\newcommand{\BK}[0]{\mathbf{K}}
\newcommand{\BL}[0]{\mathbf{L}}
\newcommand{\BM}[0]{\mathbf{M}}
\newcommand{\BN}[0]{\mathbf{N}}
\newcommand{\BO}[0]{\mathbf{O}}
\newcommand{\BP}[0]{\mathbf{P}}
\newcommand{\BQ}[0]{\mathbf{Q}}
\newcommand{\BR}[0]{\mathbf{R}}
\newcommand{\BS}[0]{\mathbf{S}}
\newcommand{\BT}[0]{\mathbf{T}}
\newcommand{\BU}[0]{\mathbf{U}}
\newcommand{\BV}[0]{\mathbf{V}}
\newcommand{\BW}[0]{\mathbf{W}}
\newcommand{\BX}[0]{\mathbf{X}}
\newcommand{\BY}[0]{\mathbf{Y}}
\newcommand{\BZ}[0]{\mathbf{Z}}

\newcommand{\Bzero}[0]{\mathbf{0}}
\newcommand{\Btheta}[0]{\boldsymbol{\theta}}
\newcommand{\Btau}[0]{\boldsymbol{\tau}}
\newcommand{\Bomega}[0]{\boldsymbol{\omega}}

%
% shorthand for unit vectors
%
\newcommand{\acap}[0]{\hat{\Ba}}
\newcommand{\bcap}[0]{\hat{\Bb}}
\newcommand{\ccap}[0]{\hat{\Bc}}
\newcommand{\dcap}[0]{\hat{\Bd}}
\newcommand{\ecap}[0]{\hat{\Be}}
\newcommand{\fcap}[0]{\hat{\Bf}}
\newcommand{\gcap}[0]{\hat{\Bg}}
\newcommand{\hcap}[0]{\hat{\Bh}}
\newcommand{\icap}[0]{\hat{\Bi}}
\newcommand{\jcap}[0]{\hat{\Bj}}
\newcommand{\kcap}[0]{\hat{\Bk}}
\newcommand{\lcap}[0]{\hat{\Bl}}
\newcommand{\mcap}[0]{\hat{\Bm}}
\newcommand{\ncap}[0]{\hat{\Bn}}
\newcommand{\ocap}[0]{\hat{\Bo}}
\newcommand{\pcap}[0]{\hat{\Bp}}
\newcommand{\qcap}[0]{\hat{\Bq}}
\newcommand{\rcap}[0]{\hat{\Br}}
\newcommand{\scap}[0]{\hat{\Bs}}
\newcommand{\tcap}[0]{\hat{\Bt}}
\newcommand{\ucap}[0]{\hat{\Bu}}
\newcommand{\vcap}[0]{\hat{\Bv}}
\newcommand{\wcap}[0]{\hat{\Bw}}
\newcommand{\xcap}[0]{\hat{\Bx}}
\newcommand{\ycap}[0]{\hat{\By}}
\newcommand{\zcap}[0]{\hat{\Bz}}
\newcommand{\thetacap}[0]{\hat{\Btheta}}

%
% to write R^n and C^n in a distinguishable fashion.  Perhaps change this
% to the double lined characters upon figuring out how to do so.
%
\newcommand{\C}[1]{$\mathbb{C}^{#1}$}
\newcommand{\R}[1]{$\mathbb{R}^{#1}$}

%
% various generally useful helpers
%

% derivative of #1 wrt. #2:
\newcommand{\D}[2] {\frac {d#2} {d#1}}

\newcommand{\inv}[1]{\frac{1}{#1}}
\newcommand{\cross}[0]{\times}

\newcommand{\abs}[1]{\lvert{#1}\rvert}
\newcommand{\norm}[1]{\lVert{#1}\rVert}
\newcommand{\innerprod}[2]{\langle{#1}, {#2}\rangle}
\newcommand{\dotprod}[2]{{#1} \cdot {#2}}
\newcommand{\bdotprod}[2]{\left({#1} \cdot {#2}\right)}
\newcommand{\crossprod}[2]{{#1} \cross {#2}}
\newcommand{\tripleprod}[3]{\dotprod{\left(\crossprod{#1}{#2}\right)}{#3}}

\DeclareMathOperator{\Proj}{Proj}
\DeclareMathOperator{\Span}{span}
\DeclareMathOperator{\Sgn}{sgn}
\DeclareMathOperator{\Area}{Area}
\DeclareMathOperator{\Volume}{Volume}

%
% A few miscellaneous things specific to this document
%
\newcommand{\crossop}[1]{\crossprod{#1}{}}

% R2 vector.
\newcommand{\VectorTwo}[2]{
\begin{bmatrix}
 {#1} \\
 {#2}
\end{bmatrix}
}

\newcommand{\VectorN}[1]{
\begin{bmatrix}
{#1}_1 \\
{#1}_2 \\
\vdots \\
{#1}_N \\
\end{bmatrix}
}

\newcommand{\DETuvij}[4]{
\begin{vmatrix}
 {#1}_{#3} & {#1}_{#4} \\
 {#2}_{#3} & {#2}_{#4}
\end{vmatrix}
}

\newcommand{\DETuvwijk}[6]{
\begin{vmatrix}
 {#1}_{#4} & {#1}_{#5} & {#1}_{#6} \\
 {#2}_{#4} & {#2}_{#5} & {#2}_{#6} \\
 {#3}_{#4} & {#3}_{#5} & {#3}_{#6}
\end{vmatrix}
}

\newcommand{\DETuvwxijkl}[8]{
\begin{vmatrix}
 {#1}_{#5} & {#1}_{#6} & {#1}_{#7} & {#1}_{#8} \\
 {#2}_{#5} & {#2}_{#6} & {#2}_{#7} & {#2}_{#8} \\
 {#3}_{#5} & {#3}_{#6} & {#3}_{#7} & {#3}_{#8} \\
 {#4}_{#5} & {#4}_{#6} & {#4}_{#7} & {#4}_{#8} \\
\end{vmatrix}
}

%\newcommand{\DETuvwxyijklm}[10]{
%\begin{vmatrix}
% {#1}_{#6} & {#1}_{#7} & {#1}_{#8} & {#1}_{#9} & {#1}_{#10} \\
% {#2}_{#6} & {#2}_{#7} & {#2}_{#8} & {#2}_{#9} & {#2}_{#10} \\
% {#3}_{#6} & {#3}_{#7} & {#3}_{#8} & {#3}_{#9} & {#3}_{#10} \\
% {#4}_{#6} & {#4}_{#7} & {#4}_{#8} & {#4}_{#9} & {#4}_{#10} \\
% {#5}_{#6} & {#5}_{#7} & {#5}_{#8} & {#5}_{#9} & {#5}_{#10}
%\end{vmatrix}
%}

% R3 vector.
\newcommand{\VectorThree}[3]{
\begin{bmatrix}
 {#1} \\
 {#2} \\
 {#3}
\end{bmatrix}
}


\newcommand{\grad}[0]{\nabla}
\newcommand{\PD}[2]{\frac{\partial {#2}}{\partial {#1}}}
\newcommand{\Abs}[1]{\left\lvert{#1}\right\rvert}

\usepackage[bookmarks=true]{hyperref}

\title{ Reconciling vector integral relations. }
\author{Peeter Joot}
\date{ Sept. 18, 2008.  Last Revision: $Date: 2008/09/19 01:38:16 $ }

\begin{document}

\maketitle{}

\tableofcontents

\section{ A hodge podge of relations. }

%I was never satisfied with how vector integral relationships were presented
%to me in my Calculus classes.  Some of these were presented as equations to
%memorize instead of with proof.  

The aim of these notes is to work through proofs of the following 
integral equations

\begin{itemize}

\item Gradient line integral. 

\begin{equation}\label{eqn:lineintegral}
\int_C (\grad f) \cdot d\Br = f \vert_{\partial C}
\end{equation}

\item Jacobian area determinants. 

Change of variables for a double integral

\begin{equation}
dA = dx dy =
\begin{vmatrix}
\PD{u}{x} & \PD{u}{y} \\
\PD{v}{x} & \PD{v}{y} \\
\end{vmatrix}
du dv
= \Abs{ \PD{(u,v)}{(x,y)} } du dv
\end{equation}

In Salus and Hille this is proved using Green's theorem, despite it 
being seeming like the more basic operation.  The greater than two
dimensional cases are not proved at all.

\item Green's theorem. 

\begin{equation}\label{eqn:greens}
\int\int \left(\PD{y}{Q} \PD{x}{P}\right) dx dy = \oint P dx + Q dy
\end{equation}

\item Divergence theorem. 

\begin{equation}\label{eqn:divergenceplane}
\int\int \grad \cdot \Bv\, dx dy = \oint \Bv \cdot \ncap\, ds
\end{equation}

\begin{equation}\label{eqn:divergencevolume}
\int\int\int_V \grad \cdot \Bv\, dx dy dz = \int\int_S \Bv \cdot \ncap\, dA
\end{equation}

\begin{equation}\label{eqn:divergencegrad}
\int\int\int_V \grad \phi\, dV \int\int_S \ncap \phi\, dA
\end{equation}

\begin{equation}\label{eqn:divergencegradcross}
\int\int\int_V \grad \cross \Bv\, dV \int\int_S \Bv \cross \ncap\, dA
\end{equation}

\item Stokes theorem. 

\begin{equation}\label{eqn:stokes}
\int\int (\grad \cross \Bv) \cdot \ncap\, dx dy = \oint \Bv \cdot d\Br
\end{equation}

\end{itemize}

In particular I'd like to relate these to the geometrical concepts
of Clifford algebra now that I know how to work with that in a 
differential and algebraic fashion for many sorts of problems.  I am hoping
that working through proofs of these basic identities 
will be enough that I can go on to the more general approaches in 
differential forms and the geometric calculus of Hestenes.

John Denker's 
\href{ http://www.av8n.com/physics/straight-wire.pdf }{ article on the
magnetic field of a straight wire }
gives a simple looking high level description of vector form of Stokes'
theorem in it's Clifford formulation

\begin{equation}\label{eqn:stokesGA}
\int_S \grad \wedge F = \int_{\partial S} F
\end{equation}

This is simple enough looking, but there are some important details left
out.  In particular the grades do not match, so there must be some sort of
implied projection or dot product operations too.

I'd say this suffers from some of the things that I had trouble with in
attempting to study differential forms.

The basic ideas of how to formulate
the curve, surface, volume, ... of integration is not specified.  How to do
that in greater than three dimensions is not trivial seeming to me since
none of the traditional methods of dotting with a normal will not work.

Knowing now about how subspaces can be expressed using blades is likely the
key.  The Clifford algebra ideas seem particularly suited to this as many
of these ideas can be formulated independent of the calculus applications.
One can learn the geometric and algebraic concepts first and then move on
to the Calculus.

\section{ Gradient line integral. }

This is the easiest of the identities to prove.  Introduction of a reciprocal frame $\gamma^{\mu} \cdot \gamma_{\nu} = {\delta^{\mu}}_{\nu}$
also means that we can do in full generalitity with a possibly
non-orthonormal basis of any dimension, and an arbitrary metric.

Write the gradient as normal

\begin{equation*}
\grad = \sum \gamma^{\mu} \PD{x^{\mu}}{} = \gamma^{\mu} \partial_{\mu}
\end{equation*}

Here summation convention with implied sum over mixed upper and lower indexes is employed.

Express the position vector along the curve as
a parameterized path $\Br = \Br(\lambda) = \gamma_{\mu} x^{\mu}$, and use
this to form the element of vector length along the path
%This can be used to form the line integral element that we also need
%the differential element of vector length.  

\begin{equation*}
d\Br = \gamma_{\mu} \frac{d x^{\mu}}{d\lambda} d\lambda
\end{equation*}

Dotting the gradient and the path element we have
\begin{align*}
\grad f \cdot d\Br 
&= \left(\gamma^{\mu} \partial_{\mu} f\right) \cdot \left(\gamma_{\nu} \frac{d x^{\nu}}{d\lambda} \right) d\lambda \\
&= {\delta^{\mu}}_{\nu} \PD{x^{\mu}}{f} \frac{d x^{\nu}}{d\lambda} d\lambda \\
&= \sum \PD{x^{\mu}}{f} \frac{d x^{\mu}}{d\lambda} d\lambda \\
&= \frac{d f}{d \lambda} d\lambda
\end{align*}

Equation \ref{eqn:lineintegral} follows immediately.

\section{ Jacobian area determinants. }

\section{ Green's theorem. }

\section{ Divergence theorem. }

\section{ Stokes theorem. }

\end{document}               % End of document.
