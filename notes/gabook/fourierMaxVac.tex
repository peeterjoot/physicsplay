\documentclass[]{eliblog}

\usepackage{amsmath}
\usepackage{mathpazo}

%
% shorthand for bold symbols, convenient for vectors and matrices
%
\newcommand{\Ba}[0]{\mathbf{a}}
\newcommand{\Bb}[0]{\mathbf{b}}
\newcommand{\Bc}[0]{\mathbf{c}}
\newcommand{\Bd}[0]{\mathbf{d}}
\newcommand{\Be}[0]{\mathbf{e}}
\newcommand{\Bf}[0]{\mathbf{f}}
\newcommand{\Bg}[0]{\mathbf{g}}
\newcommand{\Bh}[0]{\mathbf{h}}
\newcommand{\Bi}[0]{\mathbf{i}}
\newcommand{\Bj}[0]{\mathbf{j}}
\newcommand{\Bk}[0]{\mathbf{k}}
\newcommand{\Bl}[0]{\mathbf{l}}
\newcommand{\Bm}[0]{\mathbf{m}}
\newcommand{\Bn}[0]{\mathbf{n}}
\newcommand{\Bo}[0]{\mathbf{o}}
\newcommand{\Bp}[0]{\mathbf{p}}
\newcommand{\Bq}[0]{\mathbf{q}}
\newcommand{\Br}[0]{\mathbf{r}}
\newcommand{\Bs}[0]{\mathbf{s}}
\newcommand{\Bt}[0]{\mathbf{t}}
\newcommand{\Bu}[0]{\mathbf{u}}
\newcommand{\Bv}[0]{\mathbf{v}}
\newcommand{\Bw}[0]{\mathbf{w}}
\newcommand{\Bx}[0]{\mathbf{x}}
\newcommand{\By}[0]{\mathbf{y}}
\newcommand{\Bz}[0]{\mathbf{z}}
\newcommand{\BA}[0]{\mathbf{A}}
\newcommand{\BB}[0]{\mathbf{B}}
\newcommand{\BC}[0]{\mathbf{C}}
\newcommand{\BD}[0]{\mathbf{D}}
\newcommand{\BE}[0]{\mathbf{E}}
\newcommand{\BF}[0]{\mathbf{F}}
\newcommand{\BG}[0]{\mathbf{G}}
\newcommand{\BH}[0]{\mathbf{H}}
\newcommand{\BI}[0]{\mathbf{I}}
\newcommand{\BJ}[0]{\mathbf{J}}
\newcommand{\BK}[0]{\mathbf{K}}
\newcommand{\BL}[0]{\mathbf{L}}
\newcommand{\BM}[0]{\mathbf{M}}
\newcommand{\BN}[0]{\mathbf{N}}
\newcommand{\BO}[0]{\mathbf{O}}
\newcommand{\BP}[0]{\mathbf{P}}
\newcommand{\BQ}[0]{\mathbf{Q}}
\newcommand{\BR}[0]{\mathbf{R}}
\newcommand{\BS}[0]{\mathbf{S}}
\newcommand{\BT}[0]{\mathbf{T}}
\newcommand{\BU}[0]{\mathbf{U}}
\newcommand{\BV}[0]{\mathbf{V}}
\newcommand{\BW}[0]{\mathbf{W}}
\newcommand{\BX}[0]{\mathbf{X}}
\newcommand{\BY}[0]{\mathbf{Y}}
\newcommand{\BZ}[0]{\mathbf{Z}}

\newcommand{\Bzero}[0]{\mathbf{0}}
\newcommand{\Btheta}[0]{\boldsymbol{\theta}}
\newcommand{\Btau}[0]{\boldsymbol{\tau}}
\newcommand{\Bomega}[0]{\boldsymbol{\omega}}

%
% shorthand for unit vectors
%
\newcommand{\acap}[0]{\hat{\Ba}}
\newcommand{\bcap}[0]{\hat{\Bb}}
\newcommand{\ccap}[0]{\hat{\Bc}}
\newcommand{\dcap}[0]{\hat{\Bd}}
\newcommand{\ecap}[0]{\hat{\Be}}
\newcommand{\fcap}[0]{\hat{\Bf}}
\newcommand{\gcap}[0]{\hat{\Bg}}
\newcommand{\hcap}[0]{\hat{\Bh}}
\newcommand{\icap}[0]{\hat{\Bi}}
\newcommand{\jcap}[0]{\hat{\Bj}}
\newcommand{\kcap}[0]{\hat{\Bk}}
\newcommand{\lcap}[0]{\hat{\Bl}}
\newcommand{\mcap}[0]{\hat{\Bm}}
\newcommand{\ncap}[0]{\hat{\Bn}}
\newcommand{\ocap}[0]{\hat{\Bo}}
\newcommand{\pcap}[0]{\hat{\Bp}}
\newcommand{\qcap}[0]{\hat{\Bq}}
\newcommand{\rcap}[0]{\hat{\Br}}
\newcommand{\scap}[0]{\hat{\Bs}}
\newcommand{\tcap}[0]{\hat{\Bt}}
\newcommand{\ucap}[0]{\hat{\Bu}}
\newcommand{\vcap}[0]{\hat{\Bv}}
\newcommand{\wcap}[0]{\hat{\Bw}}
\newcommand{\xcap}[0]{\hat{\Bx}}
\newcommand{\ycap}[0]{\hat{\By}}
\newcommand{\zcap}[0]{\hat{\Bz}}
\newcommand{\thetacap}[0]{\hat{\Btheta}}

%
% to write R^n and C^n in a distinguishable fashion.  Perhaps change this
% to the double lined characters upon figuring out how to do so.
%
\newcommand{\C}[1]{$\mathbb{C}^{#1}$}
\newcommand{\R}[1]{$\mathbb{R}^{#1}$}

%
% various generally useful helpers
%

% derivative of #1 wrt. #2:
\newcommand{\D}[2] {\frac {d#2} {d#1}}

\newcommand{\inv}[1]{\frac{1}{#1}}
\newcommand{\cross}[0]{\times}

\newcommand{\abs}[1]{\lvert{#1}\rvert}
\newcommand{\norm}[1]{\lVert{#1}\rVert}
\newcommand{\innerprod}[2]{\langle{#1}, {#2}\rangle}
\newcommand{\dotprod}[2]{{#1} \cdot {#2}}
\newcommand{\bdotprod}[2]{\left({#1} \cdot {#2}\right)}
\newcommand{\crossprod}[2]{{#1} \cross {#2}}
\newcommand{\tripleprod}[3]{\dotprod{\left(\crossprod{#1}{#2}\right)}{#3}}

\DeclareMathOperator{\Proj}{Proj}
\DeclareMathOperator{\Span}{span}
\DeclareMathOperator{\Sgn}{sgn}
\DeclareMathOperator{\Area}{Area}
\DeclareMathOperator{\Volume}{Volume}

%
% A few miscellaneous things specific to this document
%
\newcommand{\crossop}[1]{\crossprod{#1}{}}

% R2 vector.
\newcommand{\VectorTwo}[2]{
\begin{bmatrix}
 {#1} \\
 {#2}
\end{bmatrix}
}

\newcommand{\VectorN}[1]{
\begin{bmatrix}
{#1}_1 \\
{#1}_2 \\
\vdots \\
{#1}_N \\
\end{bmatrix}
}

\newcommand{\DETuvij}[4]{
\begin{vmatrix}
 {#1}_{#3} & {#1}_{#4} \\
 {#2}_{#3} & {#2}_{#4}
\end{vmatrix}
}

\newcommand{\DETuvwijk}[6]{
\begin{vmatrix}
 {#1}_{#4} & {#1}_{#5} & {#1}_{#6} \\
 {#2}_{#4} & {#2}_{#5} & {#2}_{#6} \\
 {#3}_{#4} & {#3}_{#5} & {#3}_{#6}
\end{vmatrix}
}

\newcommand{\DETuvwxijkl}[8]{
\begin{vmatrix}
 {#1}_{#5} & {#1}_{#6} & {#1}_{#7} & {#1}_{#8} \\
 {#2}_{#5} & {#2}_{#6} & {#2}_{#7} & {#2}_{#8} \\
 {#3}_{#5} & {#3}_{#6} & {#3}_{#7} & {#3}_{#8} \\
 {#4}_{#5} & {#4}_{#6} & {#4}_{#7} & {#4}_{#8} \\
\end{vmatrix}
}

%\newcommand{\DETuvwxyijklm}[10]{
%\begin{vmatrix}
% {#1}_{#6} & {#1}_{#7} & {#1}_{#8} & {#1}_{#9} & {#1}_{#10} \\
% {#2}_{#6} & {#2}_{#7} & {#2}_{#8} & {#2}_{#9} & {#2}_{#10} \\
% {#3}_{#6} & {#3}_{#7} & {#3}_{#8} & {#3}_{#9} & {#3}_{#10} \\
% {#4}_{#6} & {#4}_{#7} & {#4}_{#8} & {#4}_{#9} & {#4}_{#10} \\
% {#5}_{#6} & {#5}_{#7} & {#5}_{#8} & {#5}_{#9} & {#5}_{#10}
%\end{vmatrix}
%}

% R3 vector.
\newcommand{\VectorThree}[3]{
\begin{bmatrix}
 {#1} \\
 {#2} \\
 {#3}
\end{bmatrix}
}



\author{Peeter Joot}
\email{peeter.joot@gmail.com}


\chapter{Electrodynamic field energy for vacuum.}
\label{chap:fourierMaxVac}
%\useCCL
\blogpage{http://sites.google.com/site/peeterjoot/math2009/fourierMaxVac.pdf}
\date{Dec 16, 2009}
\revisionInfo{fourierMaxVac.tex}

%\beginArtWithToc
\beginArtNoToc

\section{Motivation.}

From \urlchapcite{http://peeterjoot.wordpress.com/2009/12/15/energy-and-momentum-for-complex-electric-and-magnetic-field-phasors}{Energy and momentum for Complex electric and magnetic field phasors}{fourierMaxVac} how to formulate the energy momentum tensor for complex vector fields (ie. phasors) in the Geometric Algebra formalism is now understood.  To recap, for the field $F = \BE + I c \BB$, where $\BE$ and $\BB$ may be complex vectors we have for Maxwell's equation

\begin{align}\label{eqn:fourierMaxVac:1}
\grad F = J/\epsilon_0 c.
\end{align}

This is a doubly complex representation, with the four vector pseudoscalar $I = \gamma_0 \gamma_1 \gamma_2 \gamma_3$ acting as a non-commutatitive imaginary, as well as real and imaginary parts for the electric and magnetic field vectors.  We take the real part (not the scalar part) of any bivector solution $F$ of Maxwell's equation as the actual solution, but allow ourself the freedom to work with the complex phasor representation when convenient.  In these phasor vectors, the imaginary $i$, as in $\BE = \Real(\BE) + i \Imag(\BE)$, is a commuting imaginary, commuting with all the multivector elements in the algebra.

The real valued, four vector, energy momentum tensor $T(a)$ was found to be

\begin{align}\label{eqn:fourierMaxVac:2}
T(a) = \frac{\epsilon_0}{4} \Bigl( \conjugateStar{F} a \tilde{F} + \tilde{F} a \conjugateStar{F} \Bigr) = 
-\frac{\epsilon_0}{2} \Real \Bigl( \conjugateStar{F} a F \Bigr).
\end{align}

To supply some context that gives meaning to this tensor the associated conservation relationship was found to be

\begin{align}\label{eqn:fourierMaxVac:3}
\grad \cdot T(a) &= a \cdot \inv{ c } \Real \left( J \cdot \conjugateStar{F} \right).
\end{align}

and in particular for $a = \gamma^0$, this four vector divergence takes the form

\begin{align}\label{eqn:fourierMaxVac:4}
\PD{t}{}\frac{\epsilon_0}{2}(\BE \cdot \conjugateStar{\BE} + c^2 \BB \cdot \conjugateStar{\BB})
+ \spacegrad \cdot \inv{\mu_0} \Real (\BE \cross \conjugateStar{\BB} )
+ \Real( \BJ \cdot \conjugateStar{\BE} ) 
= 0,
\end{align}

relating the energy term $T^{00} = T(\gamma^0) \cdot \gamma^0$ and the Poynting spatial vector $T(\gamma^0) \wedge \gamma^0$ with the current density and electric field product that constitutes the energy portion of the Lorentz force density.

Let's apply this to calculating the energy associated with the field that is periodic within a rectangular prism as done by Bohm in \cite{bohm1989qt}.  We do not necessarily need the Geometric Algebra formalism for this calculation, but this will be a fun way to attempt it.

\section{Setup}

Let's assume a Fourier representation for the four vector potential $A$ for the field $F = \grad \wedge A$.  That is

\begin{align}
\label{eqn:fourierMaxVac:5}
A = \sum_{\Bk} A_\Bk(t) e^{i \Bk \cdot \Bx},
\end{align}

where summation is over all angular wave number triplets $\Bk = 2 \pi (k_1/\lambda_1, k_2/\lambda_2, k_3/\lambda_3)$.  The Fourier coefficients $A_\Bk = {A_\Bk}^\mu \gamma_\mu$ are allowed to be complex valued, as is the resulting four vector $A$, and the associated bivector field $F$.

Fourier inversion, with $V = \lambda_1 \lambda_2 \lambda_3$, follows from

\begin{align}\label{eqn:fourierMaxVac:6}
\delta_{\Bk', \Bk} =
\inv{ V }
\int_0^{\lambda_1}
\int_0^{\lambda_2}
\int_0^{\lambda_3} 
e^{ i \Bk' \cdot \Bx} 
e^{-i \Bk \cdot \Bx} dx^1 dx^2 dx^3,
\end{align}

but only this orthogonality relationship and not the Fourier coefficients themselves

\begin{align}
\label{eqn:fourierMaxVac:7}
A_\Bk = 
\inv{ V }
\int_0^{\lambda_1}
\int_0^{\lambda_2}
\int_0^{\lambda_3} A(\Bx, t) e^{- i \Bk \cdot \Bx} dx^1 dx^2 dx^3,
\end{align}

will be of interest here.  Evaluating the curl for this potential yields

\begin{align}\label{eqn:fourierMaxVac:8}
F = \grad \wedge A
= \sum_{\Bk} \left( \inv{c} \gamma^0 \wedge \dot{A}_\Bk + \gamma^m \wedge A_\Bk \frac{2 \pi i k_m}{\lambda_m} \right) e^{i \Bk \cdot \Bx}.
\end{align}

Since the four vector potential has been expressed using an explicit split into time and space components it will be natural to reexpress the bivector field in terms of scalar and (spatial) vector potentials, with the Fourier coefficents.  Writing $\sigma_m = \gamma_m \gamma_0$ for the spatial basis vectors, ${A_\Bk}^0 = \phi_\Bk$, and $\BA = A^k \sigma_k$, this is

\begin{align}\label{eqn:fourierMaxVac:71}
A_\Bk = (\phi_\Bk + \BA_\Bk) \gamma_0.
\end{align}

The faraday bivector field $F$ is then

\begin{align}\label{eqn:fourierMaxVac:72}
F = \sum_\Bk \left( -\inv{c} \dot{\BA}_\Bk - i \Bk \phi_\Bk + i \Bk \wedge \BA_\Bk \right) e^{i \Bk \cdot \Bx}.
\end{align}

This is now enough to express the energy momentum tensor $T(\gamma^\mu)$

\begin{align}\label{eqn:fourierMaxVac:73}
T(\gamma^\mu) 
&= -\frac{\epsilon_0}{2} \sum_{\Bk,\Bk'}
\Real \left(
\left( -\inv{c} \conjugateStar{(\dot{\BA}_{\Bk'})} + i \Bk' \conjugateStar{\phi_{\Bk'}} - i \Bk' \wedge \conjugateStar{\BA_{\Bk'}} \right) 
\gamma^\mu 
\left( -\inv{c} \dot{\BA}_\Bk - i \Bk \phi_\Bk + i \Bk \wedge \BA_\Bk \right) e^{i (\Bk -\Bk') \cdot \Bx}
\right).
\end{align}

It will be more convienient to work with a scalar plus bivector (spatial vector) form of this tensor, and right multiplication by $\gamma_0$ produces such a split

\begin{align}\label{eqn:fourierMaxVac:74}
T(\gamma^\mu) \gamma_0 = \gpgradezero{T(\gamma^\mu) \gamma_0} + \sigma_a \gpgradezero{ \sigma_a T(\gamma^\mu) \gamma_0 }
\end{align}

The primary object of this treatment will be consideration of the $\mu = 0$ components of the tensor, which provide a split into energy density $T(\gamma^0) \cdot \gamma_0$, and Poynting vector (momentum density) $T(\gamma^0) \wedge \gamma_0$.

Our first step is to integrate \autoref{eqn:fourierMaxVac:74} over the volume $V$.  This integration and the orthogonality relationship \autoref{eqn:fourierMaxVac:6}, removes the exponentials, leaving

\begin{subequations}
\label{eqn:fourierMaxVac:75}
\begin{align}
\int T(\gamma^\mu) \cdot \gamma_0
&= -\frac{\epsilon_0 V}{2} \sum_{\Bk}
\Real \gpgradezero{
\left( -\inv{c} \conjugateStar{(\dot{\BA}_{\Bk})} + i \Bk \conjugateStar{\phi_{\Bk}} - i \Bk \wedge \conjugateStar{\BA_{\Bk}} \right) 
\gamma^\mu 
\left( -\inv{c} \dot{\BA}_\Bk - i \Bk \phi_\Bk + i \Bk \wedge \BA_\Bk \right) 
\gamma_0 } \\
\int T(\gamma^\mu) \wedge \gamma_0
&= -\frac{\epsilon_0 V}{2} \sum_{\Bk}
\Real \sigma_a \gpgradezero{ \sigma_a
\left( -\inv{c} \conjugateStar{(\dot{\BA}_{\Bk})} + i \Bk \conjugateStar{\phi_{\Bk}} - i \Bk \wedge \conjugateStar{\BA_{\Bk}} \right) 
\gamma^\mu 
\left( -\inv{c} \dot{\BA}_\Bk - i \Bk \phi_\Bk + i \Bk \wedge \BA_\Bk \right) \gamma_0
} 
\end{align}
\end{subequations}

Because $\gamma_0$ commutes with the spatial bivectors, and anticommutes with the spatial vectors, the remainder of the Dirac basis vectors in these expressions can be eliminated

\begin{subequations}
\label{eqn:fourierMaxVac:77}
\begin{align}
\int T(\gamma^0) \cdot \gamma_0
&= -\frac{\epsilon_0 V }{2} \sum_{\Bk}
\Real \gpgradezero{
\left( -\inv{c} \conjugateStar{(\dot{\BA}_{\Bk})} + i \Bk \conjugateStar{\phi_{\Bk}} - i \Bk \wedge \conjugateStar{\BA_{\Bk}} \right) 
\left( \inv{c} \dot{\BA}_\Bk + i \Bk \phi_\Bk + i \Bk \wedge \BA_\Bk \right) 
} 
\label{eqn:fourierMaxVac:77a}
\\
\int T(\gamma^0) \wedge \gamma_0
&= -\frac{\epsilon_0 V}{2} \sum_{\Bk}
\Real \sigma_a \gpgradezero{ \sigma_a
\left( -\inv{c} \conjugateStar{(\dot{\BA}_{\Bk})} + i \Bk \conjugateStar{\phi_{\Bk}} - i \Bk \wedge \conjugateStar{\BA_{\Bk}} \right) 
\left( \inv{c} \dot{\BA}_\Bk + i \Bk \phi_\Bk + i \Bk \wedge \BA_\Bk \right) 
} 
\label{eqn:fourierMaxVac:77b}
\\
\int T(\gamma^m) \cdot \gamma_0
&= \frac{\epsilon_0 V }{2} \sum_{\Bk}
\Real \gpgradezero{
\left( -\inv{c} \conjugateStar{(\dot{\BA}_{\Bk})} + i \Bk \conjugateStar{\phi_{\Bk}} - i \Bk \wedge \conjugateStar{\BA_{\Bk}} \right) 
\sigma_m
\left( \inv{c} \dot{\BA}_\Bk + i \Bk \phi_\Bk + i \Bk \wedge \BA_\Bk \right) 
} 
\label{eqn:fourierMaxVac:77c}
\\
\int T(\gamma^m) \wedge \gamma_0
&= \frac{\epsilon_0 V}{2} \sum_{\Bk}
\Real \sigma_a \gpgradezero{ \sigma_a
\left( -\inv{c} \conjugateStar{(\dot{\BA}_{\Bk})} + i \Bk \conjugateStar{\phi_{\Bk}} - i \Bk \wedge \conjugateStar{\BA_{\Bk}} \right) 
\sigma_m
\left( \inv{c} \dot{\BA}_\Bk + i \Bk \phi_\Bk + i \Bk \wedge \BA_\Bk \right) 
}
\label{eqn:fourierMaxVac:77d}
.
\end{align}
\end{subequations}

\section{Expanding the energy momentum tensor components.}

\subsection{Energy}

In \autoref{eqn:fourierMaxVac:77a} only the bivector-bivector and vector-vector products produce any scalar grades.  Except for the bivector product this can be done by inspection.  For that part we utilize the identity

\begin{align}\label{eqn:fourierMaxVac:90}
\gpgradezero{ (\Bk \wedge \Ba) (\Bk \wedge \Bb) }
= (\Ba \cdot \Bk) (\Bb \cdot \Bk) - \Bk^2 (\Ba \cdot \Bb).
\end{align}

This leaves for the energy $H = \int T(\gamma^0) \cdot \gamma_0$ in the volume

\begin{align}\label{eqn:fourierMaxVac:80}
H = 
\frac{\epsilon_0 V}{2} \sum_\Bk \left(
\inv{c^2} \Abs{\dot{\BA}_\Bk}^2 
+\Bk^2 \left( \Abs{\phi_\Bk}^2 + \Abs{\BA_\Bk}^2 \right) - \Abs{\Bk \cdot \BA_\Bk}^2
+ \frac{2}{c} \Real \left( i \conjugateStar{\phi_\Bk} \cdot \dot{\BA}_\Bk \right)
\right)
\end{align}

We are left with a completely real expression, and one without any explicit Geometric Algebra.  This does not look like the Harmonic oscillator Hamiltonian that was expected.  A gauge transformation to eliminate $\phi_\Bk$ and an observation about when $\Bk \cdot \BA_\Bk$ equals zero will give us that, but first lets get the mechanical jobs done, and reduce the products for the field momentum.

\subsection{Momentum}

Now move on to \autoref{eqn:fourierMaxVac:77b}.  For the factors other than $\sigma_a$ only the vector-bivector products can contribute to the scalar product.  We have two such products, one of the form

\begin{align*}
\sigma_a \gpgradezero{ \sigma_a \Ba (\Bk \wedge \Bc) }
&=
\sigma_a (\Bc \cdot \sigma_a) (\Ba \cdot \Bk) - \sigma_a (\Bk \cdot \sigma_a) (\Ba \cdot \Bc) \\
&=
\Bc (\Ba \cdot \Bk) - \Bk (\Ba \cdot \Bc),
\end{align*}

and the other
\begin{align*}
\sigma_a \gpgradezero{ \sigma_a (\Bk \wedge \Bc) \Ba }
&=
\sigma_a (\Bk \cdot \sigma_a) (\Ba \cdot \Bc) - \sigma_a (\Bc \cdot \sigma_a) (\Ba \cdot \Bk) \\
&=
\Bk (\Ba \cdot \Bc) - \Bc (\Ba \cdot \Bk).
\end{align*}

The momentum $\BP = \int T(\gamma^0) \wedge \gamma_0$ in this volume follows by computation of

\begin{align*}
&\sigma_a \gpgradezero{ \sigma_a
\left( -\inv{c} \conjugateStar{(\dot{\BA}_{\Bk})} + i \Bk \conjugateStar{\phi_{\Bk}} - i \Bk \wedge \conjugateStar{\BA_{\Bk}} \right) 
\left( \inv{c} \dot{\BA}_\Bk + i \Bk \phi_\Bk + i \Bk \wedge \BA_\Bk \right) 
} \\
&=
  i \BA_\Bk \left( \left( -\inv{c} \conjugateStar{(\dot{\BA}_{\Bk})} + i \Bk \conjugateStar{\phi_{\Bk}} \right) \cdot \Bk \right)  
- i \Bk \left( \left( -\inv{c} \conjugateStar{(\dot{\BA}_{\Bk})} + i \Bk \conjugateStar{\phi_{\Bk}} \right) \cdot \BA_\Bk \right)  \\
&- i \Bk \left( \left( \inv{c} \dot{\BA}_\Bk + i \Bk \phi_\Bk \right) \cdot \conjugateStar{\BA_\Bk} \right)  
+ i \conjugateStar{\BA_{\Bk}} \left( \left( \inv{c} \dot{\BA}_\Bk + i \Bk \phi_\Bk \right) \cdot \Bk \right)
\end{align*}

All the products are paired in nice conjugates.  Doubling and taking real parts the momentum is left as

\begin{align}\label{eqn:fourierMaxVac:100}
\BP &= \epsilon_0 V \sum_{\Bk}
\Real \left(
i \BA_\Bk \left( \inv{c} \conjugateStar{\dot{\BA}_\Bk} \cdot \Bk \right)
+ 
\Bk^2 \phi_\Bk \conjugateStar{ \BA_\Bk }
+ i \Bk 
\left( \inv{c} \dot{\BA}_\Bk \cdot \conjugateStar{\BA_\Bk} \right)
- \Bk \conjugateStar{\phi_\Bk} (\Bk \cdot \BA_\Bk)
\right)
\end{align}

FIXME: with $\Bk \cdot \BA_\Bk = 0$ all but the third term vanishes, yet this third term is purely imaginary:

\begin{align*}
\Real ( \dot{\BA}_\Bk \cdot \conjugateStar{\BA_\Bk} ) = \inv{2} \frac{d}{dt} \left( \BA_\Bk \cdot \conjugateStar{\BA_\Bk} \right)
\end{align*}

That leaves the total momentum equal zero but something proportional to $\Bk$ seems more reasonable?

\subsection{Gauge transformation to simplify the Hamiltonian.}

In \autoref{eqn:fourierMaxVac:80} something that looked like the Harmonic oscillator was expected.  On the surface this does not appear to be such a beast.  Exploitation of gauge freedom is required to make the simplification that puts things into the Harmonic oscillator form.

If we are to change our four vector potential $A \rightarrow A + \grad \psi$, then Maxwell's equation takes the form

\begin{align}\label{eqn:fourierMaxVac:30}
J/\epsilon_0 c = \grad (\grad \wedge (A + \grad \psi) = \grad (\grad \wedge A) + \grad (\underbrace{\grad \wedge \grad \psi}_{=0}),
\end{align}

which is unchanged by the addition of the gradient to any original potential solution to the equation.  In coordinates this is a transformation of the form

\begin{align}\label{eqn:fourierMaxVac:31}
A^\mu \rightarrow A^\mu + \partial_\mu \psi,
\end{align}

and we can use this to force any one of the potential coordinates to zero.  For this problem, it appears that it is desirable to seek a $\psi$ such that $A^0 + \partial_0 \psi = 0$.  That is

\begin{align}\label{eqn:fourierMaxVac:32}
\sum_\Bk \phi_\Bk(t) e^{i \Bk \cdot \Bx} + \inv{c} \partial_t \psi = 0.
\end{align}

Or,

\begin{align}\label{eqn:fourierMaxVac:33}
\psi(\Bx,t) = \psi(\Bx,0) -\inv{c} \sum_\Bk e^{i \Bk \cdot \Bx} \int_{\tau=0}^t \phi_\Bk(\tau).
\end{align}

With such a transformation, the $\phi_\Bk$ and $\dot{\BA}_\Bk$ cross term in the Hamiltonian \autoref{eqn:fourierMaxVac:80} vanishes, as does the $\phi_\Bk$ term in the four vector square of the last term, leaving just

\begin{align}
\label{eqn:fourierMaxVac:17b}
H = 
\frac{\epsilon_0}{c^2} V \sum_\Bk
\left(
\inv{2} \Abs{\dot{\BA}_\Bk}^2
+
\inv{2} \Bigl(
(c \Bk)^2 \Abs{\BA_\Bk}^2 + \Abs{ ( c \Bk) \cdot \BA_\Bk}^2
+ \Abs{ c \Bk \cdot \BA_\Bk}^2
\Bigr)
\right).
\end{align}

Additionally, wedging \autoref{eqn:fourierMaxVac:5} with $\gamma_0$ now does not loose any information so our potential Fourier series is reduced to just

\begin{subequations}
\label{eqn:fourierMaxVac:5b}
\begin{align}
\BA &= \sum_{\Bk} \BA_\Bk(t) e^{2 \pi i \Bk \cdot \Bx} \\
\BA_\Bk &= 
\inv{ V }
\int_0^{\lambda_1}
\int_0^{\lambda_2}
\int_0^{\lambda_3} \BA(\Bx, t) e^{-i \Bk \cdot \Bx} dx^1 dx^2 dx^3.
\end{align}
\end{subequations}

The desired harmonic oscillator form would be had in \autoref{eqn:fourierMaxVac:17b} if it were not for the $\Bk \cdot \BA_\Bk$ term.  Does that vanish?  Returning to Maxwell's equation should answer that question, but first it has to be expressed in terms of the vector potential.  While $\BA = A \wedge \gamma_0$, the lack of an $A^0$ component means that this can be inverted as

\begin{align}\label{eqn:fourierMaxVac:41}
A = \BA \gamma_0 = -\gamma_0 \BA.
\end{align}

The gradient can also be factored scalar and spatial vector components

\begin{align}\label{eqn:fourierMaxVac:42}
\grad = \gamma^0 ( \partial_0 + \spacegrad ) = ( \partial_0 - \spacegrad ) \gamma^0.
\end{align}

So, with this $A^0 = 0$ gauge choice the bivector field $F$ is

\begin{align}\label{eqn:fourierMaxVac:43}
F = \grad \wedge A = \inv{2} \left( \rgrad A - A \lgrad \right) 
\end{align}

From the left the gradient action on $A$ is

\begin{align*}
\rgrad A 
&= ( \partial_0 - \spacegrad ) \gamma^0 (-\gamma_0 \BA) \\
&= ( -\partial_0 + \rspacegrad ) \BA,
%&= \partial_0 \BA 
%+ \spacegrad \cdot \BA 
%+ \spacegrad \wedge \BA 
%\\
\end{align*}

and from the right

\begin{align*}
A \lgrad
&= 
\BA \gamma_0 \gamma^0 ( \partial_0 + \spacegrad ) \\
&= 
\BA ( \partial_0 + \spacegrad ) \\
&= 
\partial_0 \BA 
+ \BA \lspacegrad 
\end{align*}

Taking the difference we have

\begin{align*}
F 
&= 
\inv{2} \Bigl( -\partial_0 \BA + \rspacegrad \BA -  \partial_0 \BA - \BA \lspacegrad \Bigr).
\end{align*}

Which is just

\begin{align}\label{eqn:fourierMaxVac:44}
F = -\partial_0 \BA + \spacegrad \wedge \BA.
\end{align}

For this vacuum case, premultiplication of Maxwell's equation by $\gamma_0$ gives

\begin{align*}
0 
&= \gamma_0 \grad ( -\partial_0 \BA + \spacegrad \wedge \BA ) \\
&= (\partial_0 + \spacegrad)( -\partial_0 \BA + \spacegrad \wedge \BA ) \\
&= -\inv{c^2} \partial_{tt} \BA 
- \partial_0 \spacegrad \cdot \BA 
- \partial_0 \spacegrad \wedge \BA 
+ \partial_0 ( \spacegrad \wedge \BA ) 
+ \underbrace{\spacegrad \cdot ( \spacegrad \wedge \BA ) }_{\spacegrad^2 \BA - \spacegrad (\spacegrad \cdot \BA)}
+ \underbrace{\spacegrad \wedge ( \spacegrad \wedge \BA )}_{=0} \\
\end{align*}

The spatial bivector and trivector grades are all zero.  Equating the remaining scalar and vector components to zero separately yields a pair of equations in $\BA$

\begin{subequations}
\label{eqn:fourierMaxVac:45}
\begin{align}
0 &= \partial_t (\spacegrad \cdot \BA) \\
0 &= -\inv{c^2} \partial_{tt} \BA + \spacegrad^2 \BA + \spacegrad (\spacegrad \cdot \BA) 
\end{align}
\end{subequations}

If the divergence of the vector potential is constant we have just a wave equation.  Let's see what that divergence is with the assumed Fourier representation

\begin{align*}
\spacegrad \cdot \BA 
&=
\sum_{\Bk \ne (0,0,0)} {\BA_\Bk}^m 2 \pi i \frac{k_m}{\lambda_m} e^{i \Bk \cdot \Bx} \\
&=
i \sum_{\Bk \ne (0,0,0)} (\BA_\Bk \cdot \Bk) e^{i \Bk \cdot \Bx} \\
&=
i \sum_\Bk (\BA_\Bk \cdot \Bk) e^{i \Bk \cdot \Bx} 
\end{align*}

Since $\BA_\Bk = \BA_\Bk(t)$, there are two ways for $\partial_t (\spacegrad \cdot \BA) = 0$.  For each $\Bk$ there must be a requirement for either $\BA_\Bk \cdot \Bk = 0$ or $\BA_\Bk = \text{constant}$.  The constant $\BA_\Bk$ solution to the first equation appears to represent a standing spatial wave with no time dependence.  Is that of any interest?

The more interesting seeming case is where we have some non-static time varying state.  In this case, if $\BA_\Bk \cdot \Bk$, the second of these Maxwell's equations is just the vector potential wave equation, since the divergence is zero.  That is

\begin{align}\label{eqn:fourierMaxVac:50}
0 &= -\inv{c^2} \partial_{tt} \BA + \spacegrad^2 \BA 
\end{align}

Solving this isn't really what is of interest, since the objective was just to determine if the divergence could be assumed to be zero.  This shows then, that if the transverse solution to Maxwell's equation is picked, the Hamiltonian for this field, with this gauge choice, becomes

\begin{align}
\label{eqn:fourierMaxVac:17c}
H = 
\frac{\epsilon_0}{c^2} V \sum_\Bk
\left(
\inv{2} \Abs{\dot{\BA}_\Bk}^2
+
\inv{2} 
(c \Bk)^2 \Abs{\BA_\Bk}^2 
\right).
\end{align}

\section{Conclusions and followup.}

The objective was met, a reproduction of Bohm's Harmonic oscillator result using a complex exponential Fourier series instead of separate sine and cosines.

The reason for Bohm's choice to fix zero divergence as the gauge choice upfront is now clear.  That automatically cuts complexity from the results.  Figuring out how to work this problem with complex valued potentials and also using the Geometric Algebra formulation probably also made the work a bit more difficult since blundering through both simultaneously was required instead of just one at a time.

This was an interesting exercise though, since doing it this way I am able to understand all the intermediate steps.  Bohm employed some subtler argumentation to eliminate the scalar potential $\phi$ upfront, and I have to admit I did not follow his logic, whereas blindly following where the math leads me all makes sense.

As a bit of followup, I'd like to consider the constant $\BA_\Bk$ case, and any implications of the freedom to pick $\BA_0$.  I'd also like to construct the Poynting vector $T(\gamma^0) \wedge \gamma_0$, and see what the structure of that is with this Fourier representation.

A general calculation of $T^{\mu\nu}$ for an assumed Fourier solution should be possible too, but working in spatial quantities for the general case is probably torture.  A four dimensional Fourier series is likely a superior option for the general case.

FIXME: check units.  What's up with the $1/c^2$ in the Hamiltonian?

FIXME: discuss energy and momentum and Hamiltonian equations once done.  Perhaps repost since equation numbers and citations are shuffled.

\EndArticle
