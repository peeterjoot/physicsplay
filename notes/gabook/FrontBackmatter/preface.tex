%
% Copyright � 2012 Peeter Joot.  All Rights Reserved.
% Licenced as described in the file LICENSE under the root directory of this GIT repository.
%

% 
% 
%\chapter{Preface}
% this suppresses an explicit chapter number for the preface.
\chapter*{Preface}%\normalsize
  \addcontentsline{toc}{chapter}{Preface}

This is a somewhat hodge podge, and very exploratory, compilation of Geometric (or Clifford) Algebra related notes on mathematics and Physics.

Most of what appear here as chapters were originally disjoint standalone notes.  I eventually accumulated enough of these individual notes that assembling them into a bookish form made some sense, even if only for personal organizational purposes.  Since my original notes were disconnected, this assembled form is not necessarily in a logical sequence, so in some cases reading in a chronological sequence (\chapcite{Chronology}) may be helpful.

Because of the journaling nature of many of these notes, a reader will find that I do not always know where I am going or what the final result will be ahead of time.  This is much different than what you will find in a polished textbook where the author knows the subject like the back of his hand.  I sometimes hit dead ends, mistakes, or unproductive paths.  
%Not all of these have been removed (although mistakes should at least be pointed out if I am aware of them).  
You will find repetition and rework of topics that were not initially covered satisfactorily, and unlike a carefully crafted text, these false starts have not all been purged.
%.  Eventually I would like to revisit much of this too verbose and too lengthly collection and shorten it to make it more useful.  It is more work to make a good short book, than to make a long poor one.

%What can be found here is an exploratory record of learning.  I have found the process of attempting to write notes of what I am learning to be very educational.  This process feeds back on itself, so I learn by writing as well as writing on what I learn.  This process often highlights holes in my understanding or errors in my original messy paper scribblings.  I find that an attempt to coherently write what one has learned makes it easier to move on.  This summarization exersize is also an excellent way to observe further topics and ideas worthy of followup.

%At the time that I wrote most of this I had little formal education in Physics, having been schooled in, as well as employed in, software engineering work.  My engineering undergrad education supplied me basic background in elementary mechanics, electromagnetism and mathematics.  This has been enough to allow me to pursue a part time ``home schooling'' project furthering my understanding of Physics.  The Physics as well as the Geometric Algebra (GA) used to explore Physics are both subjects that I find fascinating, enjoyable and complementary.  These notes are the product of concurrent study of both.

The use of this algebra in Physics could be said to be still in its infancy.  There is a fair amount Geometric Algebra in advanced treatments like the work of the Cambridge group (\citep{doran2003gap}).  There is much less that is easily accessible to someone with undergrad level education.  Even a text like Hestenes's New Foundations (\citep{hestenes1999nfc}), which has a more elementary target audience is fairly difficult to read.  These notes attempt to bridge some of that gap.
%Reading these leaves one having to do a fair amount of figure it out yourself.  
%This prompted a fair number of the notes in this compilation.  Somebody who has studied Physics instead of engineering would probably be better equipt for the subject as it is currently presented.

I can not promise that I have explained things in a way that is good for anybody else.  My audience was essentially myself as I existed at the time of writing, so the prerequisites, both for the mathematics and the Physics, have evolved continually.

Peeter Joot  \quad peeter.joot@gmail.com 
