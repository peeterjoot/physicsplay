\documentclass[12pt,leqno]{book}
\usepackage{amsmath,amssymb,amsfonts} % Typical maths resource packages
%\usepackage{graphics}                 % Packages to allow inclusion of graphics
\usepackage{graphicx}
\usepackage{color}                   % For creating coloured text and background

% for ointctr... (also appears to make "prettier" \int and \sum's)
% ... but messes up other stuff (grad vs \spacegrad).
\usepackage{txfonts} 

% levi.tex:
\usepackage{listings}

%\usepackage{latexsym,epsf}

\usepackage[bookmarks=true]{hyperref}

\parindent 1cm
\parskip 0.2cm
\topmargin 0.2cm
\oddsidemargin 1cm
\evensidemargin 0.5cm
\textwidth 15cm
\textheight 21cm

\newtheorem{theorem}{Theorem}[section]
\newtheorem{proposition}[theorem]{Proposition}
\newtheorem{corollary}[theorem]{Corollary}
\newtheorem{lemma}[theorem]{Lemma}
\newtheorem{remark}[theorem]{Remark}
\newtheorem{definition}[theorem]{Definition}

\usepackage{amsmath}
\usepackage{mathpazo}

%
% shorthand for bold symbols, convenient for vectors and matrices
%
\newcommand{\Ba}[0]{\mathbf{a}}
\newcommand{\Bb}[0]{\mathbf{b}}
\newcommand{\Bc}[0]{\mathbf{c}}
\newcommand{\Bd}[0]{\mathbf{d}}
\newcommand{\Be}[0]{\mathbf{e}}
\newcommand{\Bf}[0]{\mathbf{f}}
\newcommand{\Bg}[0]{\mathbf{g}}
\newcommand{\Bh}[0]{\mathbf{h}}
\newcommand{\Bi}[0]{\mathbf{i}}
\newcommand{\Bj}[0]{\mathbf{j}}
\newcommand{\Bk}[0]{\mathbf{k}}
\newcommand{\Bl}[0]{\mathbf{l}}
\newcommand{\Bm}[0]{\mathbf{m}}
\newcommand{\Bn}[0]{\mathbf{n}}
\newcommand{\Bo}[0]{\mathbf{o}}
\newcommand{\Bp}[0]{\mathbf{p}}
\newcommand{\Bq}[0]{\mathbf{q}}
\newcommand{\Br}[0]{\mathbf{r}}
\newcommand{\Bs}[0]{\mathbf{s}}
\newcommand{\Bt}[0]{\mathbf{t}}
\newcommand{\Bu}[0]{\mathbf{u}}
\newcommand{\Bv}[0]{\mathbf{v}}
\newcommand{\Bw}[0]{\mathbf{w}}
\newcommand{\Bx}[0]{\mathbf{x}}
\newcommand{\By}[0]{\mathbf{y}}
\newcommand{\Bz}[0]{\mathbf{z}}
\newcommand{\BA}[0]{\mathbf{A}}
\newcommand{\BB}[0]{\mathbf{B}}
\newcommand{\BC}[0]{\mathbf{C}}
\newcommand{\BD}[0]{\mathbf{D}}
\newcommand{\BE}[0]{\mathbf{E}}
\newcommand{\BF}[0]{\mathbf{F}}
\newcommand{\BG}[0]{\mathbf{G}}
\newcommand{\BH}[0]{\mathbf{H}}
\newcommand{\BI}[0]{\mathbf{I}}
\newcommand{\BJ}[0]{\mathbf{J}}
\newcommand{\BK}[0]{\mathbf{K}}
\newcommand{\BL}[0]{\mathbf{L}}
\newcommand{\BM}[0]{\mathbf{M}}
\newcommand{\BN}[0]{\mathbf{N}}
\newcommand{\BO}[0]{\mathbf{O}}
\newcommand{\BP}[0]{\mathbf{P}}
\newcommand{\BQ}[0]{\mathbf{Q}}
\newcommand{\BR}[0]{\mathbf{R}}
\newcommand{\BS}[0]{\mathbf{S}}
\newcommand{\BT}[0]{\mathbf{T}}
\newcommand{\BU}[0]{\mathbf{U}}
\newcommand{\BV}[0]{\mathbf{V}}
\newcommand{\BW}[0]{\mathbf{W}}
\newcommand{\BX}[0]{\mathbf{X}}
\newcommand{\BY}[0]{\mathbf{Y}}
\newcommand{\BZ}[0]{\mathbf{Z}}

\newcommand{\Bzero}[0]{\mathbf{0}}
\newcommand{\Btheta}[0]{\boldsymbol{\theta}}
\newcommand{\Btau}[0]{\boldsymbol{\tau}}
\newcommand{\Bomega}[0]{\boldsymbol{\omega}}

%
% shorthand for unit vectors
%
\newcommand{\acap}[0]{\hat{\Ba}}
\newcommand{\bcap}[0]{\hat{\Bb}}
\newcommand{\ccap}[0]{\hat{\Bc}}
\newcommand{\dcap}[0]{\hat{\Bd}}
\newcommand{\ecap}[0]{\hat{\Be}}
\newcommand{\fcap}[0]{\hat{\Bf}}
\newcommand{\gcap}[0]{\hat{\Bg}}
\newcommand{\hcap}[0]{\hat{\Bh}}
\newcommand{\icap}[0]{\hat{\Bi}}
\newcommand{\jcap}[0]{\hat{\Bj}}
\newcommand{\kcap}[0]{\hat{\Bk}}
\newcommand{\lcap}[0]{\hat{\Bl}}
\newcommand{\mcap}[0]{\hat{\Bm}}
\newcommand{\ncap}[0]{\hat{\Bn}}
\newcommand{\ocap}[0]{\hat{\Bo}}
\newcommand{\pcap}[0]{\hat{\Bp}}
\newcommand{\qcap}[0]{\hat{\Bq}}
\newcommand{\rcap}[0]{\hat{\Br}}
\newcommand{\scap}[0]{\hat{\Bs}}
\newcommand{\tcap}[0]{\hat{\Bt}}
\newcommand{\ucap}[0]{\hat{\Bu}}
\newcommand{\vcap}[0]{\hat{\Bv}}
\newcommand{\wcap}[0]{\hat{\Bw}}
\newcommand{\xcap}[0]{\hat{\Bx}}
\newcommand{\ycap}[0]{\hat{\By}}
\newcommand{\zcap}[0]{\hat{\Bz}}
\newcommand{\thetacap}[0]{\hat{\Btheta}}

%
% to write R^n and C^n in a distinguishable fashion.  Perhaps change this
% to the double lined characters upon figuring out how to do so.
%
\newcommand{\C}[1]{$\mathbb{C}^{#1}$}
\newcommand{\R}[1]{$\mathbb{R}^{#1}$}

%
% various generally useful helpers
%

% derivative of #1 wrt. #2:
\newcommand{\D}[2] {\frac {d#2} {d#1}}

\newcommand{\inv}[1]{\frac{1}{#1}}
\newcommand{\cross}[0]{\times}

\newcommand{\abs}[1]{\lvert{#1}\rvert}
\newcommand{\norm}[1]{\lVert{#1}\rVert}
\newcommand{\innerprod}[2]{\langle{#1}, {#2}\rangle}
\newcommand{\dotprod}[2]{{#1} \cdot {#2}}
\newcommand{\bdotprod}[2]{\left({#1} \cdot {#2}\right)}
\newcommand{\crossprod}[2]{{#1} \cross {#2}}
\newcommand{\tripleprod}[3]{\dotprod{\left(\crossprod{#1}{#2}\right)}{#3}}

\DeclareMathOperator{\Proj}{Proj}
\DeclareMathOperator{\Span}{span}
\DeclareMathOperator{\Sgn}{sgn}
\DeclareMathOperator{\Area}{Area}
\DeclareMathOperator{\Volume}{Volume}

%
% A few miscellaneous things specific to this document
%
\newcommand{\crossop}[1]{\crossprod{#1}{}}

% R2 vector.
\newcommand{\VectorTwo}[2]{
\begin{bmatrix}
 {#1} \\
 {#2}
\end{bmatrix}
}

\newcommand{\VectorN}[1]{
\begin{bmatrix}
{#1}_1 \\
{#1}_2 \\
\vdots \\
{#1}_N \\
\end{bmatrix}
}

\newcommand{\DETuvij}[4]{
\begin{vmatrix}
 {#1}_{#3} & {#1}_{#4} \\
 {#2}_{#3} & {#2}_{#4}
\end{vmatrix}
}

\newcommand{\DETuvwijk}[6]{
\begin{vmatrix}
 {#1}_{#4} & {#1}_{#5} & {#1}_{#6} \\
 {#2}_{#4} & {#2}_{#5} & {#2}_{#6} \\
 {#3}_{#4} & {#3}_{#5} & {#3}_{#6}
\end{vmatrix}
}

\newcommand{\DETuvwxijkl}[8]{
\begin{vmatrix}
 {#1}_{#5} & {#1}_{#6} & {#1}_{#7} & {#1}_{#8} \\
 {#2}_{#5} & {#2}_{#6} & {#2}_{#7} & {#2}_{#8} \\
 {#3}_{#5} & {#3}_{#6} & {#3}_{#7} & {#3}_{#8} \\
 {#4}_{#5} & {#4}_{#6} & {#4}_{#7} & {#4}_{#8} \\
\end{vmatrix}
}

%\newcommand{\DETuvwxyijklm}[10]{
%\begin{vmatrix}
% {#1}_{#6} & {#1}_{#7} & {#1}_{#8} & {#1}_{#9} & {#1}_{#10} \\
% {#2}_{#6} & {#2}_{#7} & {#2}_{#8} & {#2}_{#9} & {#2}_{#10} \\
% {#3}_{#6} & {#3}_{#7} & {#3}_{#8} & {#3}_{#9} & {#3}_{#10} \\
% {#4}_{#6} & {#4}_{#7} & {#4}_{#8} & {#4}_{#9} & {#4}_{#10} \\
% {#5}_{#6} & {#5}_{#7} & {#5}_{#8} & {#5}_{#9} & {#5}_{#10}
%\end{vmatrix}
%}

% R3 vector.
\newcommand{\VectorThree}[3]{
\begin{bmatrix}
 {#1} \\
 {#2} \\
 {#3}
\end{bmatrix}
}


%<misc>
%
\newcommand{\Abs}[1]{{\left\lvert{#1}\right\rvert}}
\newcommand{\spacegrad}[0]{\boldsymbol{\nabla}}
\newcommand{\grad}[0]{\nabla}
\newcommand{\LL}[0]{\mathcal{L}}

% == \partial_{#1} {#2}
\newcommand{\PD}[2]{\frac{\partial {#2}}{\partial {#1}}}
% inline variant
\newcommand{\PDi}[2]{{\partial {#2}}/{\partial {#1}}}

\newcommand{\PDD}[3]{\frac{\partial^2 {#3}}{\partial {#1}\partial {#2}}}
%\newcommand{\PDd}[2]{\frac{\partial^2 {#2}}{{\partial{#1}}^2}}
\newcommand{\PDsq}[2]{\frac{\partial^2 {#2}}{(\partial {#1})^2}}

\newcommand{\Partial}[2]{\frac{\partial {#1}}{\partial {#2}}}
\DeclareMathOperator{\RejName}{Rej}
\newcommand{\Rej}[2]{\RejName_{#1}\left( {#2} \right)}
\newcommand{\Rm}[1]{\mathbb{R}^{#1}}
\newcommand{\Cm}[1]{\mathbb{C}^{#1}}
\newcommand{\conj}[0]{{*}}

%</misc>

% <grade selection>
%
\newcommand{\gpgrade}[2] {{\left\langle{{#1}}\right\rangle}_{#2}}

\newcommand{\gpgradezero}[1] {\gpgrade{#1}{}}
%\newcommand{\gpscalargrade}[1] {{\left\langle{{#1}}\right\rangle}}
%\newcommand{\gpgradezero}[1] {\gpgrade{#1}{0}}

%\newcommand{\gpgradeone}[1] {{\left\langle{{#1}}\right\rangle}_{1}}
\newcommand{\gpgradeone}[1] {\gpgrade{#1}{1}}

\newcommand{\gpgradetwo}[1] {\gpgrade{#1}{2}}
\newcommand{\gpgradethree}[1] {\gpgrade{#1}{3}}
\newcommand{\gpgradefour}[1] {\gpgrade{#1}{4}}
%
% </grade selection>



\newcommand{\adot}[0]{{\dot{a}}}
\newcommand{\bdot}[0]{{\dot{b}}}
% taken for centered dot:
%\newcommand{\cdot}[0]{{\dot{c}}}
%\newcommand{\ddot}[0]{{\dot{d}}}
\newcommand{\edot}[0]{{\dot{e}}}
\newcommand{\fdot}[0]{{\dot{f}}}
\newcommand{\gdot}[0]{{\dot{g}}}
\newcommand{\hdot}[0]{{\dot{h}}}
\newcommand{\idot}[0]{{\dot{i}}}
\newcommand{\jdot}[0]{{\dot{j}}}
\newcommand{\kdot}[0]{{\dot{k}}}
\newcommand{\ldot}[0]{{\dot{l}}}
\newcommand{\mdot}[0]{{\dot{m}}}
\newcommand{\ndot}[0]{{\dot{n}}}
%\newcommand{\odot}[0]{{\dot{o}}}
\newcommand{\pdot}[0]{{\dot{p}}}
\newcommand{\qdot}[0]{{\dot{q}}}
\newcommand{\rdot}[0]{{\dot{r}}}
\newcommand{\sdot}[0]{{\dot{s}}}
\newcommand{\tdot}[0]{{\dot{t}}}
\newcommand{\udot}[0]{{\dot{u}}}
\newcommand{\vdot}[0]{{\dot{v}}}
\newcommand{\wdot}[0]{{\dot{w}}}
\newcommand{\xdot}[0]{{\dot{x}}}
\newcommand{\ydot}[0]{{\dot{y}}}
\newcommand{\zdot}[0]{{\dot{z}}}
\newcommand{\addot}[0]{{\ddot{a}}}
\newcommand{\bddot}[0]{{\ddot{b}}}
\newcommand{\cddot}[0]{{\ddot{c}}}
%\newcommand{\dddot}[0]{{\ddot{d}}}
\newcommand{\eddot}[0]{{\ddot{e}}}
\newcommand{\fddot}[0]{{\ddot{f}}}
\newcommand{\gddot}[0]{{\ddot{g}}}
\newcommand{\hddot}[0]{{\ddot{h}}}
\newcommand{\iddot}[0]{{\ddot{i}}}
\newcommand{\jddot}[0]{{\ddot{j}}}
\newcommand{\kddot}[0]{{\ddot{k}}}
\newcommand{\lddot}[0]{{\ddot{l}}}
\newcommand{\mddot}[0]{{\ddot{m}}}
\newcommand{\nddot}[0]{{\ddot{n}}}
\newcommand{\oddot}[0]{{\ddot{o}}}
\newcommand{\pddot}[0]{{\ddot{p}}}
\newcommand{\qddot}[0]{{\ddot{q}}}
\newcommand{\rddot}[0]{{\ddot{r}}}
\newcommand{\sddot}[0]{{\ddot{s}}}
\newcommand{\tddot}[0]{{\ddot{t}}}
\newcommand{\uddot}[0]{{\ddot{u}}}
\newcommand{\vddot}[0]{{\ddot{v}}}
\newcommand{\wddot}[0]{{\ddot{w}}}
\newcommand{\xddot}[0]{{\ddot{x}}}
\newcommand{\yddot}[0]{{\ddot{y}}}
\newcommand{\zddot}[0]{{\ddot{z}}}

%<bold and dot greek symbols>
%

\newcommand{\Deltadot}[0]{{\dot{\Delta}}}
\newcommand{\Gammadot}[0]{{\dot{\Gamma}}}
\newcommand{\Lambdadot}[0]{{\dot{\Lambda}}}
\newcommand{\Omegadot}[0]{{\dot{\Omega}}}
\newcommand{\Phidot}[0]{{\dot{\Phi}}}
\newcommand{\Pidot}[0]{{\dot{\Pi}}}
\newcommand{\Psidot}[0]{{\dot{\Psi}}}
\newcommand{\Sigmadot}[0]{{\dot{\Sigma}}}
\newcommand{\Thetadot}[0]{{\dot{\Theta}}}
\newcommand{\Upsilondot}[0]{{\dot{\Upsilon}}}
\newcommand{\Xidot}[0]{{\dot{\Xi}}}
\newcommand{\alphadot}[0]{{\dot{\alpha}}}
\newcommand{\betadot}[0]{{\dot{\beta}}}
\newcommand{\chidot}[0]{{\dot{\chi}}}
\newcommand{\deltadot}[0]{{\dot{\delta}}}
\newcommand{\epsilondot}[0]{{\dot{\epsilon}}}
\newcommand{\etadot}[0]{{\dot{\eta}}}
\newcommand{\gammadot}[0]{{\dot{\gamma}}}
\newcommand{\kappadot}[0]{{\dot{\kappa}}}
\newcommand{\lambdadot}[0]{{\dot{\lambda}}}
\newcommand{\mudot}[0]{{\dot{\mu}}}
\newcommand{\nudot}[0]{{\dot{\nu}}}
\newcommand{\omegadot}[0]{{\dot{\omega}}}
\newcommand{\phidot}[0]{{\dot{\phi}}}
\newcommand{\pidot}[0]{{\dot{\pi}}}
\newcommand{\psidot}[0]{{\dot{\psi}}}
\newcommand{\rhodot}[0]{{\dot{\rho}}}
\newcommand{\sigmadot}[0]{{\dot{\sigma}}}
\newcommand{\taudot}[0]{{\dot{\tau}}}
\newcommand{\thetadot}[0]{{\dot{\theta}}}
\newcommand{\upsilondot}[0]{{\dot{\upsilon}}}
\newcommand{\varepsilondot}[0]{{\dot{\varepsilon}}}
\newcommand{\varphidot}[0]{{\dot{\varphi}}}
\newcommand{\varpidot}[0]{{\dot{\varpi}}}
\newcommand{\varrhodot}[0]{{\dot{\varrho}}}
\newcommand{\varsigmadot}[0]{{\dot{\varsigma}}}
\newcommand{\varthetadot}[0]{{\dot{\vartheta}}}
\newcommand{\xidot}[0]{{\dot{\xi}}}
\newcommand{\zetadot}[0]{{\dot{\zeta}}}

\newcommand{\Deltaddot}[0]{{\ddot{\Delta}}}
\newcommand{\Gammaddot}[0]{{\ddot{\Gamma}}}
\newcommand{\Lambdaddot}[0]{{\ddot{\Lambda}}}
\newcommand{\Omegaddot}[0]{{\ddot{\Omega}}}
\newcommand{\Phiddot}[0]{{\ddot{\Phi}}}
\newcommand{\Piddot}[0]{{\ddot{\Pi}}}
\newcommand{\Psiddot}[0]{{\ddot{\Psi}}}
\newcommand{\Sigmaddot}[0]{{\ddot{\Sigma}}}
\newcommand{\Thetaddot}[0]{{\ddot{\Theta}}}
\newcommand{\Upsilonddot}[0]{{\ddot{\Upsilon}}}
\newcommand{\Xiddot}[0]{{\ddot{\Xi}}}
\newcommand{\alphaddot}[0]{{\ddot{\alpha}}}
\newcommand{\betaddot}[0]{{\ddot{\beta}}}
\newcommand{\chiddot}[0]{{\ddot{\chi}}}
\newcommand{\deltaddot}[0]{{\ddot{\delta}}}
\newcommand{\epsilonddot}[0]{{\ddot{\epsilon}}}
\newcommand{\etaddot}[0]{{\ddot{\eta}}}
\newcommand{\gammaddot}[0]{{\ddot{\gamma}}}
\newcommand{\kappaddot}[0]{{\ddot{\kappa}}}
\newcommand{\lambdaddot}[0]{{\ddot{\lambda}}}
\newcommand{\muddot}[0]{{\ddot{\mu}}}
\newcommand{\nuddot}[0]{{\ddot{\nu}}}
\newcommand{\omegaddot}[0]{{\ddot{\omega}}}
\newcommand{\phiddot}[0]{{\ddot{\phi}}}
\newcommand{\piddot}[0]{{\ddot{\pi}}}
\newcommand{\psiddot}[0]{{\ddot{\psi}}}
\newcommand{\rhoddot}[0]{{\ddot{\rho}}}
\newcommand{\sigmaddot}[0]{{\ddot{\sigma}}}
\newcommand{\tauddot}[0]{{\ddot{\tau}}}
\newcommand{\thetaddot}[0]{{\ddot{\theta}}}
\newcommand{\upsilonddot}[0]{{\ddot{\upsilon}}}
\newcommand{\varepsilonddot}[0]{{\ddot{\varepsilon}}}
\newcommand{\varphiddot}[0]{{\ddot{\varphi}}}
\newcommand{\varpiddot}[0]{{\ddot{\varpi}}}
\newcommand{\varrhoddot}[0]{{\ddot{\varrho}}}
\newcommand{\varsigmaddot}[0]{{\ddot{\varsigma}}}
\newcommand{\varthetaddot}[0]{{\ddot{\vartheta}}}
\newcommand{\xiddot}[0]{{\ddot{\xi}}}
\newcommand{\zetaddot}[0]{{\ddot{\zeta}}}

\newcommand{\BDelta}[0]{\boldsymbol{\Delta}}
\newcommand{\BGamma}[0]{\boldsymbol{\Gamma}}
\newcommand{\BLambda}[0]{\boldsymbol{\Lambda}}
\newcommand{\BOmega}[0]{\boldsymbol{\Omega}}
\newcommand{\BPhi}[0]{\boldsymbol{\Phi}}
\newcommand{\BPi}[0]{\boldsymbol{\Pi}}
\newcommand{\BPsi}[0]{\boldsymbol{\Psi}}
\newcommand{\BSigma}[0]{\boldsymbol{\Sigma}}
\newcommand{\BTheta}[0]{\boldsymbol{\Theta}}
\newcommand{\BUpsilon}[0]{\boldsymbol{\Upsilon}}
\newcommand{\BXi}[0]{\boldsymbol{\Xi}}
\newcommand{\Balpha}[0]{\boldsymbol{\alpha}}
\newcommand{\Bbeta}[0]{\boldsymbol{\beta}}
\newcommand{\Bchi}[0]{\boldsymbol{\chi}}
\newcommand{\Bdelta}[0]{\boldsymbol{\delta}}
\newcommand{\Bepsilon}[0]{\boldsymbol{\epsilon}}
\newcommand{\Beta}[0]{\boldsymbol{\eta}}
\newcommand{\Bgamma}[0]{\boldsymbol{\gamma}}
\newcommand{\Bkappa}[0]{\boldsymbol{\kappa}}
\newcommand{\Blambda}[0]{\boldsymbol{\lambda}}
\newcommand{\Bmu}[0]{\boldsymbol{\mu}}
\newcommand{\Bnu}[0]{\boldsymbol{\nu}}
%\newcommand{\Bomega}[0]{\boldsymbol{\omega}}
\newcommand{\Bphi}[0]{\boldsymbol{\phi}}
\newcommand{\Bpi}[0]{\boldsymbol{\pi}}
\newcommand{\Bpsi}[0]{\boldsymbol{\psi}}
\newcommand{\Brho}[0]{\boldsymbol{\rho}}
\newcommand{\Bsigma}[0]{\boldsymbol{\sigma}}
%\newcommand{\Btau}[0]{\boldsymbol{\tau}}
%\newcommand{\Btheta}[0]{\boldsymbol{\theta}}
\newcommand{\Bupsilon}[0]{\boldsymbol{\upsilon}}
\newcommand{\Bvarepsilon}[0]{\boldsymbol{\varepsilon}}
\newcommand{\Bvarphi}[0]{\boldsymbol{\varphi}}
\newcommand{\Bvarpi}[0]{\boldsymbol{\varpi}}
\newcommand{\Bvarrho}[0]{\boldsymbol{\varrho}}
\newcommand{\Bvarsigma}[0]{\boldsymbol{\varsigma}}
\newcommand{\Bvartheta}[0]{\boldsymbol{\vartheta}}
\newcommand{\Bxi}[0]{\boldsymbol{\xi}}
\newcommand{\Bzeta}[0]{\boldsymbol{\zeta}}
%
%</bold and dot greek symbols>
%<infrequent>
%
%\newcommand{\AreaOp}[1]{\AName_{#1}}
%\newcommand{\Babs}[0]{\abs{\BB}}
%\newcommand{\Bcap}[0]{\hat{\BB}}
%\newcommand{\BrPrimeRej}[0]{\rcap(\rcap \wedge \Br')}
%\newcommand{\CA}[0]{\mathcal{A}}
%\newcommand{\Cos}[1]{\cos{\left({#1}\right)}}
%\newcommand{\Det}[1] {\abs{#1}}
%\newcommand{\Dsq}[2] {\frac {\partial^2 {#1}} {\partial {#2}^2}}
%\newcommand{\Exp}[1]{\exp{\left({#1}\right)}}
%\newcommand{\Norm}[1]{\left\lVert{#1}\right\rVert}
%\newcommand{\Sin}[1]{\sin{\left({#1}\right)}}
%\newcommand{\T}[0]{\text{T}}
%\newcommand{\VolumeOp}[1]{\VName_{#1}}
%\newcommand{\agrad}[0]{\Ba \cdot \nabla}
%\newcommand{\alphacap}[0]{\hat{\boldsymbol{\alpha}}}
%\newcommand{\Fcap}[0]{\hat{\BF}}
%\newcommand{\bithree}[0]{{\Bi}_3}
%\newcommand{\bxa}[0]{\Bx\Ba}
%\newcommand{\coordvec}[2]{
%\newcommand{\costheta}[0]{\acap \cdot \xcap}
%\newcommand{\ddt}[1]{\ddot{#1}}
%\newcommand{\ddu}[1] {\frac {d{#1}} {du}}
%\newcommand{\dsqxj}[2] {\frac {\partial^2 {#1}} {\partial {x_{#2}}^2}}
%\newcommand{\dtheta}[1]{\frac{d {#1}}{d \theta}}
%\newcommand{\dt}[1]{\dot{#1}}
%\newcommand{\dt}[1]{\frac{d {#1}}{dt}}
%\newcommand{\dxj}[2] {\frac {\partial {#1}} {\partial {x_{#2}}}}
%\newcommand{\halfPhi}[0]{\frac{\phi}{2}}
%\newcommand{\half}[0]{\inv{2}}
%\newcommand{\inv}[1]{\frac{1}{#1}}
%\newcommand{\laplacian}[0]{\nabla^2}
%\newcommand{\matrixoftx}[3]{
%\newcommand{\nrrp}[0]{\norm{\rcap \wedge \Br'}}
%\newcommand{\oiint}{\bigcirc \hspace{-1.4em} \int \hspace{-.8em} \int}
%\newcommand{\transpose}[1]{{#1}^{\text{T}}}
%\newcommand{\transpose}[1]{{{#1}^{\TextTranspose}}}
%\newcommand{\transpose}[1]{{{#1}^{\text{T}}}}
%\newcommand{\barA}[0]{\bar{A}}
%\newcommand{\qbar}[0]{\bar{q}}
%\newcommand{\qdotbar}[0]{\dot{\bar{q}}}
%
%</infrequent>





%-------------------------------------------------------

\newcommand{\symmetric}[2]{{\left\{{#1},{#2}\right\}}}
\newcommand{\antisymmetric}[2]{\left[{#1},{#2}\right]}
\DeclareMathOperator{\sgn}{sgn}
\DeclareMathOperator{\something}{something}

\newcommand{\uDETuvij}[4]{
\begin{vmatrix}
 {#1}^{#3} & {#1}^{#4} \\
 {#2}^{#3} & {#2}^{#4}
\end{vmatrix}
}

\newcommand{\PDSq}[2]{\frac{\partial^2 {#2}}{\partial {#1}^2}}
\newcommand{\transpose}[1]{{#1}^{\mathrm{T}}}
\newcommand{\stardot}[0]{{*}}

% bivector.tex:
\newcommand{\laplacian}[0]{\nabla^2}
\newcommand{\Dsq}[2] {\frac {\partial^2 {#1}} {\partial {#2}^2}}
\newcommand{\dxj}[2] {\frac {\partial {#1}} {\partial {x_{#2}}}}
\newcommand{\dsqxj}[2] {\frac {\partial^2 {#1}} {\partial {x_{#2}}^2}}
\DeclareMathOperator{\ExpName}{e}
%\DeclareMathOperator{\Exp}{e}
%\newcommand{\Exp}[1]{\exp{\left({#1}\right)}}
%\DeclareMathOperator{\Rej}{Rej}
\DeclareMathOperator{\Rot}{R}
%\newcommand{\gpgrade}[2] {{\left\langle{{#1}}\right\rangle}_{#2}}
%\newcommand{\gpgradezero}[1] {\gpgrade{#1}{0}}
%\newcommand{\gpgradetwo}[1] {\gpgrade{#1}{2}}
%\newcommand{\gpgradefour}[1] {\gpgrade{#1}{4}}

% ga_wiki_torque.tex:
\newcommand{\Fcap}[0]{\hat{\BF}}
\newcommand{\bithree}[0]{{\Bi}_3}
\newcommand{\nrrp}[0]{\norm{\rcap \wedge \Br'}}
\newcommand{\dtheta}[1]{\frac{d {#1}}{d \theta}}

% ga_wiki_unit_derivative.tex
\newcommand{\dt}[1]{\frac{d {#1}}{dt}}
\newcommand{\BrPrimeRej}[0]{\rcap(\rcap \wedge \Br')}

% radial_vector_derivatives.tex:
%\newcommand{\BrPrimeRej}[0]{\rcap(\rcap \wedge \Br')}

% angular_velocity.tex

%\newcommand{\dt}[1]{\frac{d {#1}}{dt}}
%\newcommand{\Norm}[1]{\left\lVert{#1}\right\rVert}
%\newcommand{\dtheta}[1]{\frac{d {#1}}{d \theta}}

% reciprocal_frame.tex
\DeclareMathOperator{\AbsName}{abs}

%\DeclareMathOperator{\RejName}{Rej}
%\newcommand{\Rej}[2]{\RejName_{#1}\left( {#2} \right)}

\DeclareMathOperator{\AName}{A}
\newcommand{\AreaOp}[1]{\AName_{#1}}

\DeclareMathOperator{\VName}{V}
\newcommand{\VolumeOp}[1]{\VName_{#1}}

%\newcommand{\gpgrade}[2] {{\left\langle{{#1}}\right\rangle}_{#2}}
%\newcommand{\gpgradeone}[1] {{\left\langle{{#1}}\right\rangle}_{1}}


% projection_with_matrix_comparison.tex
%\DeclareMathOperator{\Transpose}{T}
\DeclareMathOperator{\rank}{rank}
%\newcommand{\transpose}[1]{{{#1}^{\TextTranspose}}}
%\newcommand{\transpose}[1]{{{#1}^{\text{T}}}}
\newcommand{\T}[0]{{\text{T}}}
%\newcommand{\BOmega}[0]{\boldsymbol{\Omega}}

%\newcommand{\Det}[1] {\abs{#1}}

% oblique_proj.tex
%\newcommand{\T}[0]{\text{T}}
%\newcommand{\Bbeta}[0]{\boldsymbol{\beta}}

% spherical_polar.tex
\newcommand{\phicap}[0]{\hat{\boldsymbol{\phi}}}
\newcommand{\Lor}[2]{{{\Lambda^{#1}}_{#2}}}
\newcommand{\ILor}[2]{{{ \{{\Lambda^{-1}\} }^{#1}}_{#2}}}

% slerp.tex
\DeclareMathOperator{\atan2}{atan2}

% kvector_exponential.tex
%\DeclareMathOperator{\Exp}{e}
%\DeclareMathOperator{\Rej}{Rej}
\newcommand{\Bcap}[0]{\hat{\BB}}
\newcommand{\Babs}[0]{\abs{\BB}}
%\newcommand{\gpgrade}[2] {{\left\langle{{#1}}\right\rangle}_{#2}}
%\newcommand{\gpgradezero}[1] {\gpgrade{#1}{0}}
%\newcommand{\gpgradetwo}[1] {\gpgrade{#1}{2}}
%\newcommand{\gpgradefour}[1] {\gpgrade{#1}{4}}

\newcommand{\ddu}[1] {\frac {d{#1}} {du}}

% vector_integral_relations.tex
%\newcommand{\Oiint}{\bigcirc \hspace{-1.4em} \int \hspace{-.8em} \int}

% legendre.tex
\newcommand{\agrad}[0]{\Ba \cdot \nabla}
\newcommand{\bxa}[0]{\Bx\Ba}
\newcommand{\costheta}[0]{\acap \cdot \xcap}
%\newcommand{\inv}[1]{\frac{1}{#1}}
\newcommand{\half}[0]{\inv{2}}

% ke_rotation.tex
\newcommand{\DotT}[1]{\dot{#1}}
\newcommand{\DDotT}[1]{\ddot{#1}}
%\newcommand{\transpose}[1]{{#1}^{\text{T}}}
%\newcommand{\Balpha}[0]{\boldsymbol{\alpha}}

%\newcommand{\gpgrade}[2] {{\left\langle{{#1}}\right\rangle}_{#2}}
%\newcommand{\gpgradeone}[1] {{\left\langle{{#1}}\right\rangle}_{1}}
\newcommand{\gpscalargrade}[1] {{\left\langle{{#1}}\right\rangle}}
%\newcommand{\BOmega}[0]{\boldsymbol{\Omega}}

% gaussian_surface.tex
%\newcommand{\phicap}[0]{\hat{\Bphi}}

% newtonian_lagrangian_and_gradient.tex
% PD macro that is backwards from current in macros2:
\newcommand{\PDb}[2]{ \frac{\partial{#1}}{\partial {#2}} }

% inertial_tensor.tex
\newcommand{\matrixoftx}[3]{
{
\begin{bmatrix}
{#1}
\end{bmatrix}
}_{#2}^{#3}
}

\newcommand{\coordvec}[2]{
{
\begin{bmatrix}
{#1}
\end{bmatrix}
}_{#2}
}

% bohr.tex
\newcommand{\K}[0]{\inv{4 \pi \epsilon_0}}

% euler_lagrange.tex
\newcommand{\qbar}[0]{\bar{q}}
\newcommand{\qdotbar}[0]{\dot{\bar{q}}}
\newcommand{\DD}[2]{\frac{d{#2}}{d{#1}}}
\newcommand{\Xdot}[0]{\dot{X}}

% rayleigh_jeans.tex
\newcommand{\EE}[0]{\boldsymbol{\mathcal{E}}}
\newcommand{\HH}[0]{\boldsymbol{\mathcal{H}}}

% 4d_fourier.tex

%\newcommand{\PDSq}[2]{\frac{\partial^2 {#2}}{\partial {#1}^2}}
\DeclareMathOperator{\sinc}{sinc}
\DeclareMathOperator{\PV}{PV}
\newcommand{\FF}[0]{\mathcal{F}}
\newcommand{\IIinf}[0]{ \int_{-\infty}^\infty }

% poisson.tex
%\newcommand{\PDSq}[2]{\frac{\partial^2 {#2}}{\partial {#1}^2}}
%\DeclareMathOperator{\sinc}{sinc}
%\DeclareMathOperator{\PV}{PV}
%\newcommand{\FF}[0]{\mathcal{F}}
%\newcommand{\IIinf}[0]{ \int_{-\infty}^\infty }

% fourier_maxwell.tex
%\newcommand{\PDSq}[2]{\frac{\partial^2 {#2}}{\partial {#1}^2}}
%\DeclareMathOperator{\sinc}{sinc}
%\DeclareMathOperator{\sgn}{sgn}
%\DeclareMathOperator{\PV}{PV}
%\newcommand{\FF}[0]{\mathcal{F}}
%\newcommand{\IIinf}[0]{ \int_{-\infty}^\infty }

% firstorder_fourier_maxwell.tex
%\newcommand{\PDSq}[2]{\frac{\partial^2 {#2}}{\partial {#1}^2}}
%\DeclareMathOperator{\sinc}{sinc}
%\DeclareMathOperator{\PV}{PV}
%\newcommand{\FF}[0]{\mathcal{F}}
%\newcommand{\IIinf}[0]{ \int_{-\infty}^\infty }

% wave_fourier.tex
%\newcommand{\PDSq}[2]{\frac{\partial^2 {#2}}{\partial {#1}^2}}
%\DeclareMathOperator{\sinc}{sinc}
%\DeclareMathOperator{\PV}{PV}
%\newcommand{\FF}[0]{\mathcal{F}}
%\newcommand{\IIinf}[0]{ \int_{-\infty}^\infty }

% heat_fourier.tex
%\newcommand{\PDSq}[2]{\frac{\partial^2 {#2}}{\partial {#1}^2}}
%\DeclareMathOperator{\sinc}{sinc}
%\newcommand{\FF}[0]{\mathcal{F}}
%\newcommand{\IIinf}[0]{ \int_{-\infty}^\infty }

% proj_generalized_dot_prod.tex
%\newcommand{\T}[0]{\text{T}}

% fourier_tx.tex
%\newcommand{\FF}[0]{\mathcal{F}}
\newcommand{\FM}[0]{\inv{\sqrt{2\pi\hbar}}}
\newcommand{\Iinf}[1]{ \int_{-\infty}^\infty {#1}}
%\DeclareMathOperator{\PV}{PV}

% fourier_notation.tex
%\newcommand{\FF}[0]{\mathcal{F}}
%\newcommand{\IIinf}[0]{ \int_{-\infty}^\infty }
%\DeclareMathOperator{\PV}{PV}
%\DeclareMathOperator{\sinc}{sinc}

% planewave.tex
%\newcommand{\EE}[0]{\boldsymbol{\mathcal{E}}}
%\newcommand{\HH}[0]{\boldsymbol{\mathcal{H}}}
%\newcommand{\IIinf}[0]{ \int_{-\infty}^\infty }

% dirac_lagrangian.tex
\newcommand{\Dslash}[0]{ \not\!D }

% pauli_matrix.tex
\newcommand{\Clifford}[2]{\mathcal{C}_{\{{#1},{#2}\}}}
\DeclareMathOperator{\tr}{Tr}
%\DeclareMathOperator{\Scalar}{Scalar}
\DeclareMathOperator{\Real}{Re}
\DeclareMathOperator{\Imag}{Im}
\newcommand{\trace}[1]{\tr{#1}}
\newcommand{\scalarProduct}[2]{{#1} \bullet {#2}}
\newcommand{\traceB}[1]{\tr\left({#1}\right)}
%\newcommand{\symmetric}[2]{{\left\{{#1},{#2}\right\}}}
%\newcommand{\antisymmetric}[2]{\left[{#1},{#2}\right]}
%\newcommand{\Bcap}[0]{\hat{\BB}}

\newcommand{\xhat}[0]{\hat{x}}

\newcommand{\PauliI}[0]{
\begin{bmatrix}
1 & 0 \\
0 & 1 \\
\end{bmatrix}
}

\newcommand{\PauliX}[0]{
\begin{bmatrix}
0 & 1 \\
1 & 0 \\
\end{bmatrix}
}

\newcommand{\PauliY}[0]{
\begin{bmatrix}
0 & -i \\
i & 0 \\
\end{bmatrix}
}

\newcommand{\PauliYNoI}[0]{
\begin{bmatrix}
0 & -1 \\
1 & 0 \\
\end{bmatrix}
}

\newcommand{\PauliZ}[0]{
\begin{bmatrix}
1 & 0 \\
0 & -1 \\
\end{bmatrix}
}

% gamma.tex
%\newcommand{\scalarProduct}[2]{{#1} \bullet {#2}}
%\newcommand{\symmetric}[2]{{\left\{{#1},{#2}\right\}}}
%\newcommand{\antisymmetric}[2]{\left[{#1},{#2}\right]}

%\newcommand{\PauliX}[0]{
%\begin{bmatrix}
%0 & 1 \\
%1 & 0 \\
%\end{bmatrix}
%}

%\newcommand{\PauliY}[0]{
%\begin{bmatrix}
%0 & -i \\
%i & 0 \\
%\end{bmatrix}
%}

%\newcommand{\PauliYNoI}[0]{
%\begin{bmatrix}
%0 & -1 \\
%1 & 0 \\
%\end{bmatrix}
%}

%\newcommand{\PauliZ}[0]{
%\begin{bmatrix}
%1 & 0 \\
%0 & -1 \\
%\end{bmatrix}
%}

% em_bivector_metric_dependencies.tex

%\newcommand{\LL}[0]{\mathcal{L}}
%\newcommand{\gpgrade}[2] {{\left\langle{{#1}}\right\rangle}_{#2}}
%\newcommand{\gpgradezero}[1] {\gpgrade{#1}{0}}
%\newcommand{\gpgradetwo}[1] {\gpgrade{#1}{2}}
%\newcommand{\gpgradeone}[1] {\gpgrade{#1}{1}}
%\newcommand{\gpgradefour}[1] {\gpgrade{#1}{4}}
%\newcommand{\grad}[0]{\nabla}
%\newcommand{\spacegrad}[0]{\boldsymbol{\nabla}}
% == \partial_{#1} {#2}
%\newcommand{\PD}[2]{\frac{\partial {#2}}{\partial {#1}}}
%\newcommand{\PDD}[3]{\frac{\partial^2 {#3}}{\partial {#1}\partial {#2}}}
\newcommand{\PDsQ}[2]{\frac{\partial^2 {#2}}{\partial^2 {#1}}}

% gem.tex
\newcommand{\barh}[0]{\bar{h}}

% mass_vary_lagrangian.tex
%\newcommand{\LL}[0]{\mathcal{L}}
%\newcommand{\grad}[0]{\nabla}
%\newcommand{\PD}[2]{\frac{\partial {#2}}{\partial {#1}}}
%\newcommand{\xdot}[0]{\dot{x}}
%\newcommand{\vdot}[0]{\dot{v}}
%\newcommand{\mdot}[0]{\dot{m}}
%\newcommand{\xddot}[0]{\ddot{x}}
%\newcommand{\spacegrad}[0]{\boldsymbol{\nabla}}

% fourvec_dotinvariance.tex
%\newcommand{\Balpha}[0]{\boldsymbol{\alpha}}
\newcommand{\alphacap}[0]{\hat{\boldsymbol{\alpha}}}
%\newcommand{\Bcap}[0]{\hat{\BB}}
%\newcommand{\gpgrade}[2] {{\left\langle{{#1}}\right\rangle}_{#2}}
%\newcommand{\gpgradezero}[1] {\gpgrade{#1}{0}}

% lorentz.tex
%\newcommand{\laplacian}[0]{\nabla^2}

% field_lagrangian.tex
%\newcommand{\LL}[0]{\mathcal{L}}
%\newcommand{\PD}[2]{\frac{\partial {#2}}{\partial {#1}}}
\newcommand{\barA}[0]{\bar{A}}
%\newcommand{\grad}[0]{\nabla}
%\newcommand{\conj}[0]{{*}}

%\newcommand{\spacegrad}[0]{\boldsymbol{\nabla}}

%\newcommand{\gpgrade}[2] {{\left\langle{{#1}}\right\rangle}_{#2}}
%\newcommand{\gpgradezero}[1] {\gpgrade{#1}{0}}
%\newcommand{\gpgradetwo}[1] {\gpgrade{#1}{2}}
%\newcommand{\gpgradefour}[1] {\gpgrade{#1}{4}}

% lagrangian_field_density.tex
%\newcommand{\LL}[0]{\mathcal{L}}
%\newcommand{\gpgrade}[2] {{\left\langle{{#1}}\right\rangle}_{#2}}
%\newcommand{\gpgradezero}[1] {\gpgrade{#1}{0}}
%\newcommand{\gpgradetwo}[1] {\gpgrade{#1}{2}}
%\newcommand{\gpgradefour}[1] {\gpgrade{#1}{4}}
%\newcommand{\grad}[0]{\nabla}
%\newcommand{\spacegrad}[0]{\boldsymbol{\nabla}}
%\newcommand{\PD}[2]{\frac{\partial {#2}}{\partial {#1}}}
\newcommand{\PDd}[2]{\frac{\partial^2 {#2}}{{\partial{#1}}^2}}
%\newcommand{\PDD}[3]{\frac{\partial^2 {#3}}{\partial {#1}\partial {#2}}}

%\newcommand{\barA}[0]{\bar{A}}

% lorentz_force.tex
%\newcommand{\grad}[0]{\nabla}
%\newcommand{\spacegrad}[0]{\boldsymbol{\nabla}}
%\newcommand{\LL}[0]{\mathcal{L}}
%\newcommand{\xdot}[0]{\dot{x}}
%\newcommand{\xddot}[0]{\ddot{x}}
%\newcommand{\pdot}[0]{\dot{p}}
%\newcommand{\pddot}[0]{\ddot{p}}
%\newcommand{\fdot}[0]{\dot{f}}
%\newcommand{\fddot}[0]{\ddot{f}}

%\newcommand{\gpgrade}[2] {{\left\langle{{#1}}\right\rangle}_{#2}}
%\newcommand{\gpgradeone}[1] {\gpgrade{#1}{1}}
%\newcommand{\gpgradezero}[1] {\gpgrade{#1}{}}
%\newcommand{\grad}[0] {\nabla}
%\newcommand{\spacegrad}[0]{\boldsymbol{\nabla}}

%\newcommand{\pdot}[0]{\dot{p}}
%\newcommand{\pddot}[0]{\ddot{p}}

%\newcommand{\xdot}[0]{\dot{x}}
%\newcommand{\xddot}[0]{\ddot{x}}
%\newcommand{\PD}[2]{\frac{\partial {#2}}{\partial {#1}}}

% stokes_maxwell_application.tex
%\newcommand{\grad}[0]{\nabla}
%\newcommand{\PD}[2]{\frac{\partial {#2}}{\partial {#1}}}
%\newcommand{\spacegrad}[0]{\boldsymbol{\nabla}}
%\newcommand{\gpgrade}[2] {{\left\langle{{#1}}\right\rangle}_{#2}}
%\newcommand{\gpgradezero}[1] {\gpgrade{#1}{0}}
%\newcommand{\gpgradeone}[1] {\gpgrade{#1}{1}}
%\newcommand{\gpgradetwo}[1] {\gpgrade{#1}{2}}
%\newcommand{\gpgradethree}[1] {\gpgrade{#1}{3}}

% lorentz_rotation.tex
%\DeclareMathOperator{\Transpose}{T}
%\newcommand{\T}[0]{\text{T}}

% electron_rotor.tex
\newcommand{\reverse}[1]{\tilde{{#1}}}
%\newcommand{\ILambda}[0]{{(\Lambda^{-1})}}
\newcommand{\ILambda}[0]{\Pi}

% em_potential.tex
%\newcommand{\spacegrad}[0]{\boldsymbol{\nabla}}
%\newcommand{\grad}[0]{\nabla}
\newcommand{\CA}[0]{\mathcal{A}}
 
% maxwell_to_tensor.tex
%\newcommand{\LL}[0]{\mathcal{L}}
%\newcommand{\gpgrade}[2] {{\left\langle{{#1}}\right\rangle}_{#2}}
%\newcommand{\gpgradezero}[1] {\gpgrade{#1}{0}}
%\newcommand{\gpgradetwo}[1] {\gpgrade{#1}{2}}
%\newcommand{\gpgradeone}[1] {\gpgrade{#1}{1}}
%\newcommand{\gpgradefour}[1] {\gpgrade{#1}{4}}
%\newcommand{\grad}[0]{\nabla}
%\newcommand{\spacegrad}[0]{\boldsymbol{\nabla}}
% == \partial_{#1} {#2}
%\newcommand{\PD}[2]{\frac{\partial {#2}}{\partial {#1}}}
%\newcommand{\PDD}[3]{\frac{\partial^2 {#3}}{\partial {#1}\partial {#2}}}
%\newcommand{\PDsQ}[2]{\frac{\partial^2 {#2}}{\partial^2 {#1}}}

%\newcommand{\EE}[0]{\boldsymbol{\mathcal{E}}}
%\newcommand{\HH}[0]{\boldsymbol{\mathcal{H}}}
\newcommand{\Vcap}[0]{\hat{\BV}}

% END COMMANDS.
%-------------------------------------------------------

\makeindex

\title{Applied Geometric Algebra}

\author{Peeter Joot  \quad peeter.joot@gmail.com \\
{\small\em \copyright \  Draft date \today }}

\date{ May 31, 2009.  Last Revision: $Date: 2009/06/02 22:11:27 $ }
\begin{document}
\maketitle
 \addcontentsline{toc}{chapter}{Contents}
\pagenumbering{roman}
\tableofcontents
\listoffigures
\listoftables
\chapter*{Preface}\normalsize
  \addcontentsline{toc}{chapter}{Preface}
\pagestyle{plain}

Start compiling my hodge podge set of Geometric Algebra notes into a coherent unit.  Want a treatment that is more
understandable than that of 
\cite{doran2003gap} or 
\cite{hestenes1999nfc}.
This may be hard since so many of these notes are disconnected, and as
is do not necessarily build in a logical sequence.  Many of these 
notes were an attempt to provide motivation for things presented in
an unmotivated fashion in other locations, but ironically this probably
means that much of the content here will be unmotivated since this 
is the gaps but not the meat of other treatments.

%Note the tag used to make an index entry. You may need to consult Lamport's
%book~\cite{lamport} for details of the procedure to make the index input
%file; \LaTeX \ will create a pre-index by listing all the tagged
%items in the file {\tt bookex.idx} then you edit this into
%a {\tt theindex} environment, as {\tt index.tex}.

\pagestyle{headings}
\pagenumbering{arabic}

%-------------------------------------------------------

\part{Basics and Geometry.}
\include{intro_ga}
\include{ga_wiki}
\include{ga_wiki_cramers}
\include{ga_wiki_torque}
\include{ga_wiki_unit_derivative}
\include{radial_vector_derivatives}
\include{angular_velocity}
\documentclass{article}      % Specifies the document class

\usepackage{amsmath}
\usepackage{mathpazo}

%
% shorthand for bold symbols, convenient for vectors and matrices
%
\newcommand{\Ba}[0]{\mathbf{a}}
\newcommand{\Bb}[0]{\mathbf{b}}
\newcommand{\Bc}[0]{\mathbf{c}}
\newcommand{\Bd}[0]{\mathbf{d}}
\newcommand{\Be}[0]{\mathbf{e}}
\newcommand{\Bf}[0]{\mathbf{f}}
\newcommand{\Bg}[0]{\mathbf{g}}
\newcommand{\Bh}[0]{\mathbf{h}}
\newcommand{\Bi}[0]{\mathbf{i}}
\newcommand{\Bj}[0]{\mathbf{j}}
\newcommand{\Bk}[0]{\mathbf{k}}
\newcommand{\Bl}[0]{\mathbf{l}}
\newcommand{\Bm}[0]{\mathbf{m}}
\newcommand{\Bn}[0]{\mathbf{n}}
\newcommand{\Bo}[0]{\mathbf{o}}
\newcommand{\Bp}[0]{\mathbf{p}}
\newcommand{\Bq}[0]{\mathbf{q}}
\newcommand{\Br}[0]{\mathbf{r}}
\newcommand{\Bs}[0]{\mathbf{s}}
\newcommand{\Bt}[0]{\mathbf{t}}
\newcommand{\Bu}[0]{\mathbf{u}}
\newcommand{\Bv}[0]{\mathbf{v}}
\newcommand{\Bw}[0]{\mathbf{w}}
\newcommand{\Bx}[0]{\mathbf{x}}
\newcommand{\By}[0]{\mathbf{y}}
\newcommand{\Bz}[0]{\mathbf{z}}
\newcommand{\BA}[0]{\mathbf{A}}
\newcommand{\BB}[0]{\mathbf{B}}
\newcommand{\BC}[0]{\mathbf{C}}
\newcommand{\BD}[0]{\mathbf{D}}
\newcommand{\BE}[0]{\mathbf{E}}
\newcommand{\BF}[0]{\mathbf{F}}
\newcommand{\BG}[0]{\mathbf{G}}
\newcommand{\BH}[0]{\mathbf{H}}
\newcommand{\BI}[0]{\mathbf{I}}
\newcommand{\BJ}[0]{\mathbf{J}}
\newcommand{\BK}[0]{\mathbf{K}}
\newcommand{\BL}[0]{\mathbf{L}}
\newcommand{\BM}[0]{\mathbf{M}}
\newcommand{\BN}[0]{\mathbf{N}}
\newcommand{\BO}[0]{\mathbf{O}}
\newcommand{\BP}[0]{\mathbf{P}}
\newcommand{\BQ}[0]{\mathbf{Q}}
\newcommand{\BR}[0]{\mathbf{R}}
\newcommand{\BS}[0]{\mathbf{S}}
\newcommand{\BT}[0]{\mathbf{T}}
\newcommand{\BU}[0]{\mathbf{U}}
\newcommand{\BV}[0]{\mathbf{V}}
\newcommand{\BW}[0]{\mathbf{W}}
\newcommand{\BX}[0]{\mathbf{X}}
\newcommand{\BY}[0]{\mathbf{Y}}
\newcommand{\BZ}[0]{\mathbf{Z}}

\newcommand{\Bzero}[0]{\mathbf{0}}
\newcommand{\Btheta}[0]{\boldsymbol{\theta}}
\newcommand{\Btau}[0]{\boldsymbol{\tau}}
\newcommand{\Bomega}[0]{\boldsymbol{\omega}}

%
% shorthand for unit vectors
%
\newcommand{\acap}[0]{\hat{\Ba}}
\newcommand{\bcap}[0]{\hat{\Bb}}
\newcommand{\ccap}[0]{\hat{\Bc}}
\newcommand{\dcap}[0]{\hat{\Bd}}
\newcommand{\ecap}[0]{\hat{\Be}}
\newcommand{\fcap}[0]{\hat{\Bf}}
\newcommand{\gcap}[0]{\hat{\Bg}}
\newcommand{\hcap}[0]{\hat{\Bh}}
\newcommand{\icap}[0]{\hat{\Bi}}
\newcommand{\jcap}[0]{\hat{\Bj}}
\newcommand{\kcap}[0]{\hat{\Bk}}
\newcommand{\lcap}[0]{\hat{\Bl}}
\newcommand{\mcap}[0]{\hat{\Bm}}
\newcommand{\ncap}[0]{\hat{\Bn}}
\newcommand{\ocap}[0]{\hat{\Bo}}
\newcommand{\pcap}[0]{\hat{\Bp}}
\newcommand{\qcap}[0]{\hat{\Bq}}
\newcommand{\rcap}[0]{\hat{\Br}}
\newcommand{\scap}[0]{\hat{\Bs}}
\newcommand{\tcap}[0]{\hat{\Bt}}
\newcommand{\ucap}[0]{\hat{\Bu}}
\newcommand{\vcap}[0]{\hat{\Bv}}
\newcommand{\wcap}[0]{\hat{\Bw}}
\newcommand{\xcap}[0]{\hat{\Bx}}
\newcommand{\ycap}[0]{\hat{\By}}
\newcommand{\zcap}[0]{\hat{\Bz}}
\newcommand{\thetacap}[0]{\hat{\Btheta}}

%
% to write R^n and C^n in a distinguishable fashion.  Perhaps change this
% to the double lined characters upon figuring out how to do so.
%
\newcommand{\C}[1]{$\mathbb{C}^{#1}$}
\newcommand{\R}[1]{$\mathbb{R}^{#1}$}

%
% various generally useful helpers
%

% derivative of #1 wrt. #2:
\newcommand{\D}[2] {\frac {d#2} {d#1}}

\newcommand{\inv}[1]{\frac{1}{#1}}
\newcommand{\cross}[0]{\times}

\newcommand{\abs}[1]{\lvert{#1}\rvert}
\newcommand{\norm}[1]{\lVert{#1}\rVert}
\newcommand{\innerprod}[2]{\langle{#1}, {#2}\rangle}
\newcommand{\dotprod}[2]{{#1} \cdot {#2}}
\newcommand{\bdotprod}[2]{\left({#1} \cdot {#2}\right)}
\newcommand{\crossprod}[2]{{#1} \cross {#2}}
\newcommand{\tripleprod}[3]{\dotprod{\left(\crossprod{#1}{#2}\right)}{#3}}

\DeclareMathOperator{\Proj}{Proj}
\DeclareMathOperator{\Span}{span}
\DeclareMathOperator{\Sgn}{sgn}
\DeclareMathOperator{\Area}{Area}
\DeclareMathOperator{\Volume}{Volume}

%
% A few miscellaneous things specific to this document
%
\newcommand{\crossop}[1]{\crossprod{#1}{}}

% R2 vector.
\newcommand{\VectorTwo}[2]{
\begin{bmatrix}
 {#1} \\
 {#2}
\end{bmatrix}
}

\newcommand{\VectorN}[1]{
\begin{bmatrix}
{#1}_1 \\
{#1}_2 \\
\vdots \\
{#1}_N \\
\end{bmatrix}
}

\newcommand{\DETuvij}[4]{
\begin{vmatrix}
 {#1}_{#3} & {#1}_{#4} \\
 {#2}_{#3} & {#2}_{#4}
\end{vmatrix}
}

\newcommand{\DETuvwijk}[6]{
\begin{vmatrix}
 {#1}_{#4} & {#1}_{#5} & {#1}_{#6} \\
 {#2}_{#4} & {#2}_{#5} & {#2}_{#6} \\
 {#3}_{#4} & {#3}_{#5} & {#3}_{#6}
\end{vmatrix}
}

\newcommand{\DETuvwxijkl}[8]{
\begin{vmatrix}
 {#1}_{#5} & {#1}_{#6} & {#1}_{#7} & {#1}_{#8} \\
 {#2}_{#5} & {#2}_{#6} & {#2}_{#7} & {#2}_{#8} \\
 {#3}_{#5} & {#3}_{#6} & {#3}_{#7} & {#3}_{#8} \\
 {#4}_{#5} & {#4}_{#6} & {#4}_{#7} & {#4}_{#8} \\
\end{vmatrix}
}

%\newcommand{\DETuvwxyijklm}[10]{
%\begin{vmatrix}
% {#1}_{#6} & {#1}_{#7} & {#1}_{#8} & {#1}_{#9} & {#1}_{#10} \\
% {#2}_{#6} & {#2}_{#7} & {#2}_{#8} & {#2}_{#9} & {#2}_{#10} \\
% {#3}_{#6} & {#3}_{#7} & {#3}_{#8} & {#3}_{#9} & {#3}_{#10} \\
% {#4}_{#6} & {#4}_{#7} & {#4}_{#8} & {#4}_{#9} & {#4}_{#10} \\
% {#5}_{#6} & {#5}_{#7} & {#5}_{#8} & {#5}_{#9} & {#5}_{#10}
%\end{vmatrix}
%}

% R3 vector.
\newcommand{\VectorThree}[3]{
\begin{bmatrix}
 {#1} \\
 {#2} \\
 {#3}
\end{bmatrix}
}



\newcommand{\laplacian}[0]{\nabla^2}
\newcommand{\Dsq}[2] {\frac {\partial^2 {#1}} {\partial {#2}^2}}
\newcommand{\dxj}[2] {\frac {\partial {#1}} {\partial {x_{#2}}}}
\newcommand{\dsqxj}[2] {\frac {\partial^2 {#1}} {\partial {x_{#2}}^2}}
\DeclareMathOperator{\Exp}{e}
\DeclareMathOperator{\Rej}{Rej}
\newcommand{\gpgrade}[2] {{\left\langle{{#1}}\right\rangle}_{#2}}
\newcommand{\gpgradezero}[1] {\gpgrade{#1}{0}}
\newcommand{\gpgradetwo}[1] {\gpgrade{#1}{2}}
\newcommand{\gpgradefour}[1] {\gpgrade{#1}{4}}

%
% The real thing:
%

                             % The preamble begins here.
\title{Geometry of intersecting bivectors}
\author{Peeter Joot}         % Declares the author's name.
%\date{}        % Deleting this command produces today's date.

\begin{document}             % End of preamble and beginning of text.

\maketitle{}

\section{ The problem. }

Examination of exponential solutions for Laplace's equation leads one to
a requirement to examine the product of intersecting bivectors such as

\[
\left(\abs{\Bx \wedge \Bk}^2\right)' = -\left(
(\Bx' \wedge \Bv)(\Bx \wedge \Bv)
+(\Bx \wedge \Bv)(\Bx' \wedge \Bv)
\right)
\]

Here we see that the symmetric sum of bivectors $\Bx \wedge \Bk$ and $\Bx' \wedge \Bk$ is a scalar quantity.  This we will identify later as a quantity
related to the bivector dot product.

It is worthwhile to systematically examine the
general products of intersecting bivectors, that is planes that share a common line, in this case the line directed along the vector $\Bk$.
It is also notable that since all non coplanar bivectors in \R{3} intersect
this
examination will cover the important special case of three dimensional
plane geometry.

A result of this examination is that many of the concepts familiar from
vector geometry such as
orthagonality, projection, and rejection will have direct bivector
equivalents.

General bivector geometry, in spaces where non-coplanar bivectors do not 
neccessarily intersect (such as in \R{4}), will need to be treated separately,
but some of the grade 4 product terms will be carried below to explicitly
hightlight the point where the intersecting bivector space requirement
effects the results.

\section{The meat.}

The geometric product of two bivectors can be written:

\begin{equation}\label{eqn:ABprod}
\BA \BB = 
\gpgrade{\BA \BB}{0}
+\gpgrade{\BA \BB}{2}
+\gpgrade{\BA \BB}{4}
= 
{\BA \cdot \BB}
+\gpgrade{\BA \BB}{2}
+{\BA \wedge \BB}
\end{equation}
\begin{equation}\label{eqn:BAprod}
\BB \BA = 
\gpgrade{\BB \BA}{0}
+\gpgrade{\BB \BA}{2}
+\gpgrade{\BB \BA}{4}
= 
{\BB \cdot \BA}
+\gpgrade{\BB \BA}{2}
+{\BB \wedge \BA}
\end{equation}

Because we have three terms involved, unlike the vector dot and wedge product
we cannot generally separate these terms by 
symmetric and antisymmetric parts.  However forming those sums
will still worthwhile, especially for the case of interecting bivectors
since the last term will be zero in that case.

\subsection{ Sign change of each grade term with commutation. }

Starting with the last term we can first observe that

\begin{equation}\label{eqn:wedgesign}
\BA \wedge \BB = \BB \wedge \BA
\end{equation}

To show this let $\BA = \Ba \wedge \Bb$, and $\BB = \Bc \wedge \Bd$.  When

$\BA \wedge \BB \ne 0$, one can write:

\begin{align*}
\BA \wedge \BB 
&= \Ba \wedge \Bb \wedge \Bc \wedge \Bd \\
&= - \Bb \wedge \Bc \wedge \Bd \wedge \Ba \\
&= \Bc \wedge \Bd \wedge \Ba \wedge \Bb \\
&= \BB \wedge \BA \\
\end{align*}

To see how the signs of the remaining two terms vary with commutation
form:

\begin{align*}
(\BA + \BB)^2
&= (\BA + \BB)(\BA + \BB) \\
&= \BA^2 + \BB^2 + \BA \BB + \BB \BA \\
\end{align*}

When $\BA$ and $\BB$ interect we can write
$\BA = \Ba \wedge \Bx$, and $\BB = \Bb \wedge \Bx$, thus the sum is a bivector

\[
(\BA + \BB)
= (\Ba + \Bb) \wedge \Bx
\]

And so, the square of the two is a scalar.  When $\BA$ and $\BB$ have only
non intersecting components, such as the grade two \R{4} multivector
$\Be_{12} + \Be_{34}$, the square of this sum will have both grade four and
scalar parts.

Since the LHS = RHS, and the grades of the two also must be the same.
This implies that the quantity

\[
\BA \BB + \BB \BA = 
\BA \cdot \BB + \BB \cdot \BA
+\gpgradetwo{\BA \BB} + \gpgradetwo{\BB \BA}
+\BA \wedge \BB + \BB \wedge \BA
\]

is a scalar $\iff$ 
$\BA + \BB$ is a bivector, and in general has scalar and grade four terms.
Because this symmetric sum has no grade two terms, 
regardless of whether $\BA$, and $\BB$ intersect, we have:

\[
\gpgradetwo{\BA \BB} + \gpgradetwo{\BB \BA} = 0
\]
\begin{equation}\label{eqn:signgradetwo}
\implies
\gpgradetwo{\BA \BB} = -\gpgradetwo{\BB \BA}
\end{equation}

One would intuitively expect $\BA \cdot \BB = \BB \cdot \BA$.  This can be
demonstrated by forming the complete symmetric sum

\begin{align*}
\BA \BB + \BB \BA 
&= 
{\BA \cdot \BB} +{\BB \cdot \BA}
+\gpgrade{\BA \BB}{2} +\gpgrade{\BB \BA}{2}
+{\BA \wedge \BB} + {\BB \wedge \BA} \\
&= 
{\BA \cdot \BB} +{\BB \cdot \BA}
+\gpgrade{\BA \BB}{2} -\gpgrade{\BA \BB}{2}
+{\BA \wedge \BB} + {\BA \wedge \BB} \\
&= 
{\BA \cdot \BB} +{\BB \cdot \BA}
+2{\BA \wedge \BB} \\
\end{align*}

The LHS commutes with interchange of $\BA$ and $\BB$, as does
${\BA \wedge \BB}$.  So for the RHS to also commute, the remaining grade 0 term
must also:

\begin{equation}\label{eqn:dotsign}
\BA \cdot \BB = \BB \cdot \BA
\end{equation}

\subsection{ Dot, wedge and grade two terms of bivector product. }

Collecting the results of the previous section and substituiting back
into equation \ref{eqn:ABprod} we have:

\begin{equation}\label{eqn:AdotB}
\BA \cdot \BB = \gpgrade{\frac{\BA \BB + \BB\BA}{2}}{0}
\end{equation}

\begin{equation}\label{eqn:AtwoB}
\gpgradetwo{\BA \BB} = \frac{\BA \BB - \BB\BA}{2}
\end{equation}

\begin{equation}\label{eqn:AwedgeB}
\BA \wedge \BB = \gpgrade{\frac{\BA \BB + \BB\BA}{2}}{4}
\end{equation}

When these intersect in a line the wedge term is zero, so for that special case we can write:

\begin{equation*}
\BA \cdot \BB = \frac{\BA \BB + \BB\BA}{2}
\end{equation*}

\begin{equation*}
\gpgradetwo{\BA \BB} = \frac{\BA \BB - \BB\BA}{2}
\end{equation*}

\begin{equation*}
\BA \wedge \BB = 0
\end{equation*}

(note that this is always the case for \R{3}).

\section{ Intersection of planes. }

Starting with two planes specified parametrically, each in terms of two direction vectors and a point on the plane:

\begin{align}\label{eqn:twoplanes}
\Bx &= \Bp + \alpha \Bu + \beta \Bv \\
\By &= \Bq + a \Bw + b \Bz \\
\end{align}

If these intersect then all points on the line must satisify $\Bx = \By$, so the
solution requires:

\[
\Bp + \alpha \Bu + \beta \Bv = \Bq + a \Bw + b \Bz
\]
\[
\implies
(\Bp + \alpha \Bu + \beta \Bv) \wedge \Bw \wedge \Bz = (\Bq + a \Bw + b \Bz) \wedge \Bw \wedge \Bz = \Bq \wedge \Bw \wedge \Bz
\]

Rearranging for $\beta$, and writing $\BB = \Bw \wedge \Bz$:

\[
\beta = \frac{\Bq \wedge \BB - (\Bp + \alpha \Bu) \wedge \BB}{\Bv \wedge \BB}
\]

Note that when the solution exists the left vs right order of the division by $\Bv \wedge \BB$ should not matter since the numerator will be proportional to this bivector (or else the $\beta$ would not be a scalar).

Substitution of $\beta$ back into $\Bx = \Bp + \alpha \Bu + \beta \Bv$ (all points in the first plane) gives you a parametric equation for a line:

\[
\Bx = \Bp + \frac{(\Bq-\Bp)\wedge \BB}{\Bv \wedge \BB}\Bv + \alpha\frac{1}{\Bv \wedge \BB}((\Bv \wedge \BB) \Bu - (\Bu \wedge \BB)\Bv)
\]

Where a point on the line is:

\[
\Bp + \frac{(\Bq-\Bp)\wedge \BB}{\Bv \wedge \BB}\Bv 
%= \frac{1}{\Bv \wedge \BB}((\Bv \wedge \BB)\Bp + ((\Bq-\Bp)\wedge \BB)\Bv)
\]

And a direction vector for the line is:

\[
\frac{1}{\Bv \wedge \BB}((\Bv \wedge \BB) \Bu - (\Bu \wedge \BB)\Bv)
\]
\[
\propto
(\Bv \wedge \BB)^2 \Bu - (\Bv \wedge \BB)(\Bu \wedge \BB)\Bv
\]

Now, this result is only valid if $\Bv \wedge \BB \ne 0$ (ie: line of intersection is not directed along $\Bv$), but if that is the case the second form will be zero.  Thus we can add the results (or any non-zero linear combination of) allowing for either of $\Bu$, or $\Bv$ to be directed along the line of intersection:

\begin{equation}\label{eqn:dirvecintersection}
a\left( (\Bv \wedge \BB)^2 \Bu
- (\Bv \wedge \BB)(\Bu \wedge \BB)\Bv \right)
+ b\left((\Bu \wedge \BB)^2 \Bv 
- (\Bu \wedge \BB)(\Bv \wedge \BB)\Bu\right)
\end{equation}

Alternately, one could formulate this in terms of $\BA = \Bu \wedge \Bv$, $\Bw$, and $\Bz$.  Is there a more symetrical form for this direction vector?

\subsection{ Vector along line of intersection in \R{3}}

For \R{3} one can solve the intersection problem using the normals to the planes.  For simplicity put the origin on the line of intersection (and all planes through a common point in \R{3} have at least a line of intersection).  In this case, for bivectors $\BA$ and $\BB$, normals to those planes are $i\BA$, and $i\BB$ respectively.  The plane through both of those normals is:

\begin{align*}
(i\BA) \wedge (i\BB)
= \frac{(i\BA)(i\BB) - (i\BB)(i\BA)}{2} 
= \frac{\BB\BA - \BA\BB}{2} 
= \gpgradetwo{\BB\BA}
\end{align*}

The normal to this plane

\begin{equation}\label{eqn:r3planeintersect}
i\gpgradetwo{\BB\BA}
\end{equation}

is directed along the line of interesection.  This result is more appealing than
the general \R{N} result of equation \ref{eqn:dirvecintersection}, not
just because it is simpler, but also because it is a function of only the
bivectors for the planes, without a requirement to find or calculate
two specific independent direction vectors in one of the planes.

\subsection{ Applying this result to \R{N} }

If you reject the component of $\BA$ from $\BB$ for two intersecting bivectors:

\[
\Rej_{\BA}(\BB) = \frac{1}{\BA}\gpgradetwo{\BA\BB}
\]

the line of intersection remains the same ... that operation rotates $\BB$ so that the two are mutually perpendicular.  This essentially reduces the problem to that of the three dimensional case, so the solution has to be of the same form... you just need to calculate a ``pseudoscalar'' (what you are calling the join), for the subspace spanned by the two bivectors.

That can be computed by taking any direction vector that is on one plane, but isn't in the second.  For example, pick a vector $\Bu$ in the plane $\BA$ that is not on the intersection of $\BA$ and $\BB$.  In mathese that is $\Bu = \inv{\BA}(\BA\cdot \Bu)$ (or $\Bu \wedge \BA = 0$), where $\Bu \wedge \BB \ne 0$.  Thus a pseudoscalar for this subspace is:

\[
\Bi = \frac{\Bu \wedge \BB}{\abs{\Bu \wedge \BB}}
\]

To calculate the direction vector along the intersection we don't care about the scaling above.  Also note that provided $\Bu$ has a component in the plane $\BA$, $\Bu \cdot \BA$ is also in the plane (it's rotated $\pi/2$ from $\inv{\BA}(\BA \cdot \Bu)$.

Thus, provided that $\Bu \cdot \BA$ isn't on the intersection, a scaled ``pseudoscalar''
for the subspace can be calculated by taking from any vector $\Bu$ with a component in the plane $\BA$:

\[
\Bi \propto (\Bu \cdot \BA) \wedge \BB
\]

Thus a vector along the intersection is:

\begin{equation}\label{eqn:pseudoscalarinter}
\Bd = ((\Bu \cdot \BA) \wedge \BB) \gpgradetwo{\BA\BB}
\end{equation}

(an interchange of $\BA$ and $\BB$ above would also work).

Without showing the steps one can write the complete parametric solution of the line through the planes of equation \ref{eqn:twoplanes} in terms of this direction vector:

\begin{equation}\label{eqn:finalsolnofRNplaneintersection}
\Bx = \Bp + \left(\frac{(\Bq - \Bp)\wedge \BB}{(\Bd \cdot \BA) \wedge \BB}\right) (\Bd \cdot \BA) + \alpha \Bd
\end{equation}

Since $(\Bd \cdot \BA) \ne 0$ and $(\Bd \cdot \BA) \wedge \BB \ne 0$ (unless $\BA$ and $\BB$ are coplanar), observe that this is a natural generator
of the pseudoscalar for the subspace, and as such shows up in the expression
above.

\section{ Grade components of a trivector product. }

While trying to put equation \ref{eqn:dirvecintersection} into a form
that eliminated $\Bu$, and $\Bv$ in favour of $\BA = \Bu \wedge \Bv$
symmetric and antisymmetric formulations for the various grade terms
of a trivector product looked like they could be handy.  Here's a summary
of those results.

\subsection{ Grade 6 term. }

Writing two trivectors in terms
of mutually orthogonal components

\[
\BA = \Bx \wedge \By \wedge \Bz = \Bx\By\Bz
\]

and

\[
\BB = \Bu \wedge \Bv \wedge \Bw =\Bu\Bv\Bw
\]

Assuming that there is no common vector between the two, the 
wedge of these is

\begin{align*}
\BA \wedge \BB 
&= \gpgrade{\BA\BB}{6} \\
&= \gpgrade{\Bx\By\Bz\Bu\Bv\Bw}{6} \\
&= \gpgrade{\By\Bz(\Bx\Bu)\Bv\Bw}{6} \\
&= \gpgrade{\By\Bz(-\Bu\Bx + 2\Bu \cdot \Bx)\Bv\Bw}{6} \\
&= -\gpgrade{\By\Bz\Bu(\Bx\Bv)\Bw}{6} \\
&= -\gpgrade{\By\Bz\Bu(-\Bv\Bx + 2\Bv \cdot \Bx)\Bw}{6} \\
&= \gpgrade{\By\Bz\Bu\Bv(\Bx\Bw)}{6} \\
&= \cdots \\
&= -\gpgrade{\Bu\Bv\Bw\Bx\By\Bz}{6} \\
&= -\gpgrade{\BB\BA}{6} \\
&= -\BB \wedge \BA
\end{align*}

Note above that any interchange of terms inverts the sign (demonstrated 
explicitly for all the $\Bx$ interchanges).

As an aside, this
sign change on interchange is taken as the defining property of the 
wedge product in differential forms.  That property also
implies also that the wedge product is
zero when a vector is wedged with itself since zero is the only
value that is the negation of itself.  Thus we see explicitly
how the notation of using the wedge for the highest grade term
of two blades is consistent with the traditional
wedge product definition.

The end result here is that the grade 6 term of a trivector trivector product
changes sign on interchange of the trivectors:

\begin{equation}\label{eqn:trivecgpgrade6}
\gpgrade{\BA\BB}{6} = -\gpgrade{\BB\BA}{6}
\end{equation}

\subsection{ Grade 4 term. }

For a trivector product to have a grade 4 term there must be a common
vector between the two

\[
\BA = \Bx \wedge \By \wedge \Bz = \Bx\By\Bz
\]

and

\[
\BB = \Bu \wedge \Bv \wedge \Bz =\Bu\Bv\Bz
\]

The grade four term of the product is

\begin{align*}
\gpgrade{\BB \BA}{4}
&= \gpgrade{ \Bu\Bv\Bz \Bx\By\Bz }{4} \\
&= \gpgrade{ \Bu\Bv\Bz \Bz\Bx\By }{4} \\
&= \Bz^2\gpgrade{ \Bu\Bv\Bx\By }{4} \\
&= \Bz^2\gpgrade{ \Bu(\Bv\Bx)\By }{4} \\
&= \Bz^2\gpgrade{ \Bu(-\Bx\Bv + 2 \Bx \cdot \Bv)\By }{4} \\
&= -\Bz^2\gpgrade{ \Bu\Bx\Bv\By }{4} \\
&= \cdots \\
&= \Bz^2\gpgrade{ \Bx\By\Bu\Bv }{4} \\
&= \gpgrade{ \Bx\By\Bz\Bz\Bu\Bv }{4} \\
&= \gpgrade{ \Bx\By\Bz\Bu\Bv\Bz }{4} \\
&= \gpgrade{ \Bx\By\Bz\Bu\Bv\Bz }{4} \\
&= \gpgrade{\BA \BB}{4}
\end{align*}

Thus the grade 4 term commutes on interchange:

\begin{equation}\label{eqn:trivecgpgrade4}
\gpgrade{\BA\BB}{4} = \gpgrade{\BB\BA}{4}
\end{equation}

\subsection{ Grade 2 term. }

Similar to above, 
for a trivector product to have a grade 2 term there must be two common
vectors between the two

\[
\BA = \Bx \wedge \By \wedge \Bz = \Bx\By\Bz
\]

and

\[
\BB = \Bu \wedge \By \wedge \Bz =\Bu\By\Bz
\]

The grade two term of the product is

\begin{align*}
\gpgrade{\BA \BB}{2}
&= \gpgrade{ \Bx\By\Bz \Bu\By\Bz }{2} \\
&= \gpgrade{ \Bx\By\Bz \By\Bz \Bu}{2} \\
&= (\By\Bz)^2\gpgrade{ \Bx \Bu}{2} \\
&= -(\By\Bz)^2\gpgrade{ \Bu \Bx}{2} \\
&= -\gpgrade{ \BB \BA }{2} \\
\end{align*}

The grade 2 term anticommutes on interchange:

\begin{equation}\label{eqn:trivecgpgrade2}
\gpgrade{\BA\BB}{2} = -\gpgrade{\BB\BA}{2}
\end{equation}

\subsection{ Grade 0 term. }

Any grade 0 terms are due to products of the form $\BA = k\BB$

\begin{align*}
\gpgrade{\BA \BB}{0}
&= \gpgrade{k\BB \BB}{0} \\
&= \gpgrade{\BB k\BB}{0} \\
&= \gpgrade{\BB \BA}{0} \\
\end{align*}

The grade 2 term commutes on interchange:

\begin{equation}\label{eqn:trivecgpgrade0}
\gpgrade{\BA\BB}{0} = \gpgrade{\BB\BA}{0}
\end{equation}

\subsection{ combining results. }

\begin{equation*}
\BA \BB
=\gpgrade{\BA\BB}{0}
+\gpgrade{\BA\BB}{2}
+\gpgrade{\BA\BB}{4}
+\gpgrade{\BA\BB}{6}
\end{equation*}

\begin{align*}
\BB\BA
&=\gpgrade{\BB\BA}{0}
+\gpgrade{\BB\BA}{2}
+\gpgrade{\BB\BA}{4}
+\gpgrade{\BB\BA}{6} \\
&=\gpgrade{\BA\BB}{0}
-\gpgrade{\BA\BB}{2}
+\gpgrade{\BA\BB}{4}
-\gpgrade{\BA\BB}{6} \\
\end{align*}

These can be combined to express each of the grade terms as subsets
of the symmetric and antisymmetric parts:

\begin{align*}
\BA \cdot \BB = \gpgrade{\BA\BB}{0} &= \gpgrade{\frac{\BA\BB + \BB\BA}{2}}{0} \\
\gpgrade{\BA\BB}{2} &= \gpgrade{\frac{\BA\BB - \BB\BA}{2}}{2} \\
\gpgrade{\BA\BB}{4} &= \gpgrade{\frac{\BA\BB + \BB\BA}{2}}{4} \\
\BA \wedge \BB = \gpgrade{\BA\BB}{6} &= \gpgrade{\frac{\BA\BB - \BB\BA}{2}}{6} \\
\end{align*}

Note that above I've been somewhat loose with the argument above.  A grade three vector
will have the following form:

\[
\sum_{i<j<k} D_{ijk} \Be_{ijk}
\]

Where $D_{ijk}$ is the determinant of $ijk$ components of the vectors being wedged.  Thus the product
of two trivectors will be of the following form:

\[
\sum_{i<j<k} \sum_{i'<j'<k'} D_{ijk} D'_{i'j'k'} (\Be_{ijk} \Be_{i'j'k'})
\]

It's really each of these $\Be_{ijk} \Be_{i'j'k'}$ products that have to be considered in the grade 
and sign arguments above.  The end result will be the same though... one would just have to present
it a bit more carefully for a true proof.

\subsection{ Intersecting trivector cases. }

As with the intersecting bivector case, when there is a line of intersection between the two volumes one can
write:

\begin{align*}
\BA \cdot \BB = \gpgrade{\BA\BB}{0} &= \gpgrade{\frac{\BA\BB + \BB\BA}{2}}{0} \\
\gpgrade{\BA\BB}{2} &= \frac{\BA\BB - \BB\BA}{2} \\
\gpgrade{\BA\BB}{4} &= \gpgrade{\frac{\BA\BB + \BB\BA}{2}}{4} \\
\BA \wedge \BB = \gpgrade{\BA\BB}{6} &= 0 \\
\end{align*}

And if these volumes intersect in a plane a further simplification is possible:
\begin{align*}
\BA \cdot \BB = \gpgrade{\BA\BB}{0} &= \frac{\BA\BB + \BB\BA}{2} \\
\gpgrade{\BA\BB}{2} &= \frac{\BA\BB - \BB\BA}{2} \\
\gpgrade{\BA\BB}{4} &= 0 \\
\BA \wedge \BB = \gpgrade{\BA\BB}{6} &= 0 \\
\end{align*}

\end{document}               % End of document.

%
% Copyright � 2012 Peeter Joot.  All Rights Reserved.
% Licenced as described in the file LICENSE under the root directory of this GIT repository.
%

%
%
\chapter{Trivector geometry}
\index{trivector}
\label{chap:trivector}
%\date{Mar 9, 2008.  trivector.tex}

\section{Motivation}

The direction vector for two intersecting planes can be found to have the
form:

\begin{equation}\label{eqn:trivector:dirvecintersection}
a\left( (\Bv \wedge \BB)^2 \Bu
- (\Bv \wedge \BB)(\Bu \wedge \BB)\Bv \right)
+ b\left((\Bu \wedge \BB)^2 \Bv
- (\Bu \wedge \BB)(\Bv \wedge \BB)\Bu\right)
\end{equation}

While trying to put \eqnref{eqn:trivector:dirvecintersection} into a form
that eliminated \(\Bu\), and \(\Bv\) in favor of \(\BA = \Bu \wedge \Bv\)
symmetric and antisymmetric formulations for the various grade terms
of a trivector product looked like they could be handy.  Here is a summary
of those results.

\section{Grade components of a trivector product}

\subsection{Grade 6 term}

Writing two trivectors in terms
of mutually orthogonal components

\begin{equation}\label{eqn:trivector:26}
\BA = \Bx \wedge \By \wedge \Bz = \Bx\By\Bz
\end{equation}

and

\begin{equation}\label{eqn:trivector:46}
\BB = \Bu \wedge \Bv \wedge \Bw =\Bu\Bv\Bw
\end{equation}

Assuming that there is no common vector between the two, the
wedge of these is

\begin{equation}\label{eqn:trivector:186}
\begin{aligned}
\BA \wedge \BB
&= \gpgrade{\BA\BB}{6} \\
&= \gpgrade{\Bx\By\Bz\Bu\Bv\Bw}{6} \\
&= \gpgrade{\By\Bz(\Bx\Bu)\Bv\Bw}{6} \\
&= \gpgrade{\By\Bz(-\Bu\Bx + 2\Bu \cdot \Bx)\Bv\Bw}{6} \\
&= -\gpgrade{\By\Bz\Bu(\Bx\Bv)\Bw}{6} \\
&= -\gpgrade{\By\Bz\Bu(-\Bv\Bx + 2\Bv \cdot \Bx)\Bw}{6} \\
&= \gpgrade{\By\Bz\Bu\Bv(\Bx\Bw)}{6} \\
&= \cdots \\
&= -\gpgrade{\Bu\Bv\Bw\Bx\By\Bz}{6} \\
&= -\gpgrade{\BB\BA}{6} \\
&= -\BB \wedge \BA
\end{aligned}
\end{equation}

Note above that any interchange of terms inverts the sign (demonstrated
explicitly for all the \(\Bx\) interchanges).

As an aside, this
sign change on interchange is taken as the defining property of the
wedge product in differential forms.  That property also
implies also that the wedge product is
zero when a vector is wedged with itself since zero is the only
value that is the negation of itself.  Thus we see explicitly
how the notation of using the wedge for the highest grade term
of two blades is consistent with the traditional
wedge product definition.

The end result here is that the grade 6 term of a trivector trivector product
changes sign on interchange of the trivectors:

\begin{equation}\label{eqn:trivector:trivecgpgrade6}
\gpgrade{\BA\BB}{6} = -\gpgrade{\BB\BA}{6}
\end{equation}

\subsection{Grade 4 term}

For a trivector product to have a grade 4 term there must be a common
vector between the two

\begin{equation}\label{eqn:trivector:66}
\BA = \Bx \wedge \By \wedge \Bz = \Bx\By\Bz
\end{equation}

and

\begin{equation}\label{eqn:trivector:86}
\BB = \Bu \wedge \Bv \wedge \Bz =\Bu\Bv\Bz
\end{equation}

The grade four term of the product is

\begin{equation}\label{eqn:trivector:206}
\begin{aligned}
\gpgrade{\BB \BA}{4}
&= \gpgrade{ \Bu\Bv\Bz \Bx\By\Bz }{4} \\
&= \gpgrade{ \Bu\Bv\Bz \Bz\Bx\By }{4} \\
&= \Bz^2\gpgrade{ \Bu\Bv\Bx\By }{4} \\
&= \Bz^2\gpgrade{ \Bu(\Bv\Bx)\By }{4} \\
&= \Bz^2\gpgrade{ \Bu(-\Bx\Bv + 2 \Bx \cdot \Bv)\By }{4} \\
&= -\Bz^2\gpgrade{ \Bu\Bx\Bv\By }{4} \\
&= \cdots \\
&= \Bz^2\gpgrade{ \Bx\By\Bu\Bv }{4} \\
&= \gpgrade{ \Bx\By\Bz\Bz\Bu\Bv }{4} \\
&= \gpgrade{ \Bx\By\Bz\Bu\Bv\Bz }{4} \\
&= \gpgrade{ \Bx\By\Bz\Bu\Bv\Bz }{4} \\
&= \gpgrade{\BA \BB}{4}
\end{aligned}
\end{equation}

Thus the grade 4 term commutes on interchange:

\begin{equation}\label{eqn:trivector:trivecgpgrade4}
\gpgrade{\BA\BB}{4} = \gpgrade{\BB\BA}{4}
\end{equation}

\subsection{Grade 2 term}

Similar to above,
for a trivector product to have a grade 2 term there must be two common
vectors between the two

\begin{equation}\label{eqn:trivector:106}
\BA = \Bx \wedge \By \wedge \Bz = \Bx\By\Bz
\end{equation}

and

\begin{equation}\label{eqn:trivector:126}
\BB = \Bu \wedge \By \wedge \Bz =\Bu\By\Bz
\end{equation}

The grade two term of the product is

\begin{equation}\label{eqn:trivector:226}
\begin{aligned}
\gpgrade{\BA \BB}{2}
&= \gpgrade{ \Bx\By\Bz \Bu\By\Bz }{2} \\
&= \gpgrade{ \Bx\By\Bz \By\Bz \Bu}{2} \\
&= (\By\Bz)^2\gpgrade{ \Bx \Bu}{2} \\
&= -(\By\Bz)^2\gpgrade{ \Bu \Bx}{2} \\
&= -\gpgrade{ \BB \BA }{2} \\
\end{aligned}
\end{equation}

The grade 2 term anticommutes on interchange:

\begin{equation}\label{eqn:trivector:trivecgpgrade2}
\gpgrade{\BA\BB}{2} = -\gpgrade{\BB\BA}{2}
\end{equation}

\subsection{Grade 0 term}

Any grade 0 terms are due to products of the form \(\BA = k\BB\)

\begin{equation}\label{eqn:trivector:246}
\begin{aligned}
\gpgrade{\BA \BB}{0}
&= \gpgrade{k\BB \BB}{0} \\
&= \gpgrade{\BB k\BB}{0} \\
&= \gpgrade{\BB \BA}{0} \\
\end{aligned}
\end{equation}

The grade 2 term commutes on interchange:

\begin{equation}\label{eqn:trivector:trivecgpgrade0}
\gpgrade{\BA\BB}{0} = \gpgrade{\BB\BA}{0}
\end{equation}

\subsection{combining results}

\begin{equation*}
\BA \BB
=\gpgrade{\BA\BB}{0}
+\gpgrade{\BA\BB}{2}
+\gpgrade{\BA\BB}{4}
+\gpgrade{\BA\BB}{6}
\end{equation*}

\begin{equation}\label{eqn:trivector:266}
\begin{aligned}
\BB\BA
&=\gpgrade{\BB\BA}{0}
+\gpgrade{\BB\BA}{2}
+\gpgrade{\BB\BA}{4}
+\gpgrade{\BB\BA}{6} \\
&=\gpgrade{\BA\BB}{0}
-\gpgrade{\BA\BB}{2}
+\gpgrade{\BA\BB}{4}
-\gpgrade{\BA\BB}{6} \\
\end{aligned}
\end{equation}

These can be combined to express each of the grade terms as subsets
of the symmetric and antisymmetric parts:

\begin{equation}\label{eqn:trivector:286}
\begin{aligned}
\BA \cdot \BB = \gpgrade{\BA\BB}{0} &= \gpgrade{\frac{\BA\BB + \BB\BA}{2}}{0} \\
\gpgrade{\BA\BB}{2} &= \gpgrade{\frac{\BA\BB - \BB\BA}{2}}{2} \\
\gpgrade{\BA\BB}{4} &= \gpgrade{\frac{\BA\BB + \BB\BA}{2}}{4} \\
\BA \wedge \BB = \gpgrade{\BA\BB}{6} &= \gpgrade{\frac{\BA\BB - \BB\BA}{2}}{6} \\
\end{aligned}
\end{equation}

Note that above I have been somewhat loose with the argument above.  A grade three vector
will have the following form:

\begin{equation}\label{eqn:trivector:146}
\sum_{i<j<k} D_{ijk} \Be_{ijk}
\end{equation}

Where \(D_{ijk}\) is the determinant of \(ijk\) components of the vectors being wedged.  Thus the product
of two trivectors will be of the following form:

\begin{equation}\label{eqn:trivector:166}
\sum_{i<j<k} \sum_{i'<j'<k'} D_{ijk} D'_{i'j'k'} (\Be_{ijk} \Be_{i'j'k'})
\end{equation}

It is really each of these \(\Be_{ijk} \Be_{i'j'k'}\) products that have to be considered in the grade
and sign arguments above.  The end result will be the same though... one would just have to present
it a bit more carefully for a true proof.

\subsection{Intersecting trivector cases}

As with the intersecting bivector case, when there is a line of intersection between the two volumes one can
write:

\begin{equation}\label{eqn:trivector:306}
\begin{aligned}
\BA \cdot \BB = \gpgrade{\BA\BB}{0} &= \gpgrade{\frac{\BA\BB + \BB\BA}{2}}{0} \\
\gpgrade{\BA\BB}{2} &= \frac{\BA\BB - \BB\BA}{2} \\
\gpgrade{\BA\BB}{4} &= \gpgrade{\frac{\BA\BB + \BB\BA}{2}}{4} \\
\BA \wedge \BB = \gpgrade{\BA\BB}{6} &= 0 \\
\end{aligned}
\end{equation}

And if these volumes intersect in a plane a further simplification is possible:
\begin{equation}\label{eqn:trivector:326}
\begin{aligned}
\BA \cdot \BB = \gpgrade{\BA\BB}{0} &= \frac{\BA\BB + \BB\BA}{2} \\
\gpgrade{\BA\BB}{2} &= \frac{\BA\BB - \BB\BA}{2} \\
\gpgrade{\BA\BB}{4} &= 0 \\
\BA \wedge \BB = \gpgrade{\BA\BB}{6} &= 0 \\
\end{aligned}
\end{equation}


\include{scalar_commutes}
%
% Copyright � 2012 Peeter Joot.  All Rights Reserved.
% Licenced as described in the file LICENSE under the root directory of this GIT repository.
%

%
%
\chapter{Blade grade reduction}
\index{grade reduction}
\label{chap:bladegradereduction}
%\date{Mar 25, 2008.  bladegradereduction.tex}

\section{General triple product reduction formula}

Consideration of the reciprocal frame bivector decomposition required the following identity

\begin{equation}
(\BA_a \wedge \BA_b) \cdot \BA_c =
\BA_a \cdot (\BA_b \cdot \BA_c)
\end{equation}

This holds when \(a + b \le c\), and \(a <= b\).  Similar equations for vector wedge blade dot blade reduction can be found in NFCM, but intuition let me to believe the above generalization was valid.

To prove this use the definition of the generalized dot product of two blades:

\begin{equation}\label{eqn:bladegradereduction:282}
\begin{aligned}
(\BA_a \wedge \BA_b) \cdot \BA_c
&= \gpgrade{ (\BA_a \wedge \BA_b) \BA_c }{\abs{c-(a+b)}} \\
\end{aligned}
\end{equation}

The subsequent discussion
is restricted to the \(b \ge a\) case.  Would have to think whether this restriction is required.

\begin{equation}
\label{eqn:bladegradereduction:bladewedge}
\begin{aligned}
\BA_a \wedge \BA_b
&= \BA_a \BA_b - \sum_{i=\abs{b-a},i+=2}^{a+b}\gpgrade{\BA_a\BA_b}{i} \\
&= \BA_a \BA_b - \sum_{k=0}^{a-1}\gpgrade{\BA_a\BA_b}{2k + b - a} \\
\end{aligned}
\end{equation}

Back substitution gives:

\begin{equation}\label{eqn:bladegradereduction:322}
\begin{aligned}
\gpgrade{ (\BA_a \wedge \BA_b) \BA_c }{\abs{c-(a+b)}}
&=
\gpgrade{ \BA_a \BA_b \BA_c }{\abs{c-(a+b)}}
-
\sum_{k=0}^{a-1}
\gpgrade{ \gpgrade{\BA_a\BA_b}{2k + b - a} \BA_c }{c-a-b}
\end{aligned}
\end{equation}

Temporarily writing \(\gpgrade{\BA_a\BA_b}{2k + b - a} = \BC_i\),
\begin{equation}\label{eqn:bladegradereduction:342}
\begin{aligned}
\gpgrade{\BA_a\BA_b}{2k + b - a} \BA_c
&= \sum_{j=c-i,j+=2}^{c+i} \gpgrade{ \BC_i \BA_c }{j} \\
&= \sum_{r=0}^{i} \gpgrade{ \BC_i \BA_c }{c-i+2r} \\
&= \sum_{r=0}^{2k+b-a} \gpgrade{ \BC_i \BA_c }{c-2k-b+a+2r} \\
&= \sum_{r=0}^{2k+b-a} \gpgrade{ \BC_i \BA_c }{c-b+a +2(r-k)} \\
\end{aligned}
\end{equation}

We want the only the following grade terms:

\begin{equation}\label{eqn:bladegradereduction:42}
c-b+a+2(r-k) = c - b - a
\implies
r=k-a
\end{equation}

There are many such \(k,r\) combinations, but we have a \(k \in [0,a-1]\) constraint, which implies \(r \in [-a,-1]\).  This contradicts with \(r\) strictly
positive,
so there are no such grade elements.

This gives an intermediate result, the reduction of the triple product to a direct product, removing the explicit wedge:

\begin{equation}
(\BA_a \wedge \BA_b) \cdot \BA_c =
\gpgrade{\BA_a \BA_b \BA_c}{c-a-b}
\end{equation}

\begin{equation}\label{eqn:bladegradereduction:362}
\begin{aligned}
\gpgrade{\BA_a \BA_b \BA_c}{c-a-b}
&= \gpgrade{\BA_a (\BA_b \BA_c)}{c-a-b} \\
&= \gpgrade{\BA_a \sum_{i}\gpgrade{\BA_b \BA_c}{i}}{c-a-b} \\
&= \gpgrade{\sum_{j}\gpgrade{\BA_a \sum_{i}\gpgrade{\BA_b \BA_c}{i}}{j}}{c-a-b} \\
\end{aligned}
\end{equation}

Explicitly specifying the grades here is omitted for simplicity.  The lowest grade of these is \((c-b)-a\), and all others are higher,
so grade selection excludes them.

By definition

\begin{equation}\label{eqn:bladegradereduction:62}
\gpgrade{\BA_b \BA_c}{c-b} = \BA_b \cdot \BA_c
\end{equation}

so that lowest grade term is thus

\begin{equation}\label{eqn:bladegradereduction:82}
\gpgrade{\BA_a \gpgrade{\BA_b \BA_c}{c-b}}{c-a-b}
= \gpgrade{\BA_a (\BA_b \cdot \BA_c)}{c-a-b}
= \BA_a \cdot (\BA_b \cdot \BA_c)
\end{equation}

This completes the proof.

\section{reduction of grade of dot product of two blades}

The result above can be applied to reducing the dot product of two blades.  For \(k<=s\):

\begin{equation}\label{eqn:bladegradereduction:102}
(\Ba_1 \wedge \Ba_2 \wedge \Ba_3 \cdots \wedge \Ba_k) \cdot (\Bb_1 \wedge \Bb_2 \cdots \wedge \Bb_s)
\end{equation}
\begin{equation}\label{eqn:bladegradereduction:382}
\begin{aligned}
&= (\Ba_1 \wedge (\Ba_2 \wedge \Ba_3 \cdots \wedge \Ba_k)) \cdot (\Bb_1 \wedge \Bb_2 \cdots \wedge \Bb_s) \\
&= (\Ba_1 \cdot ((\Ba_2 \wedge \Ba_3 \cdots \wedge \Ba_k)) \cdot (\Bb_1 \wedge \Bb_2 \cdots \wedge \Bb_s)) \\
&= (\Ba_1 \cdot (\Ba_2 \cdot (\Ba_3 \cdots \wedge \Ba_k)) \cdot (\Bb_1 \wedge \Bb_2 \cdots \wedge \Bb_s)) \\
&= \cdots \\
&= \Ba_1 \cdot (\Ba_2 \cdot (\Ba_3 \cdot (\cdots \cdot (\Ba_k \cdot (\Bb_1 \wedge \Bb_2 \cdots \wedge \Bb_s))))) \\
\end{aligned}
\end{equation}

This can be reduced to a single determinant, as is done in
the Flanders' differential forms book definition of the
\({\bigwedge}^k\) inner product (which is then used to define the Hodge dual).

The first such product is:

\begin{equation}\label{eqn:bladegradereduction:122}
\Ba_k \cdot (\Bb_1 \wedge \Bb_2 \cdots \wedge \Bb_k)
= \sum (-1)^{u-1} (\Ba_k \cdot \Bb_u) \Bb_1 \wedge \cdots \check{\Bb_u} \cdots \wedge \Bb_k
\end{equation}

Next, take dot product with \(\Ba_{k-1}\):

\begin{enumerate}
\item \(k = 2\)

\begin{equation}\label{eqn:bladegradereduction:402}
\begin{aligned}
&\Ba_{k-1} \cdot (\Ba_k \cdot (\Bb_1 \wedge \Bb_2 \cdots \wedge \Bb_k)) \\
&= \sum_{v \ne u} (-1)^{u-1} (\Ba_k \cdot \Bb_u) (\Ba_1 \cdot \Bb_v) \\
&=
 \sum_{u < v} (-1)^{v-1} (\Ba_k \cdot \Bb_v) (\Ba_1 \cdot \Bb_u)
+\sum_{u < v} (-1)^{u-1} (\Ba_k \cdot \Bb_u) (\Ba_1 \cdot \Bb_v) \\
&=
+\sum_{u < v} (\Ba_k \cdot \Bb_u) (\Ba_1 \cdot \Bb_v)
-\sum_{u < v} (\Ba_k \cdot \Bb_v) (\Ba_1 \cdot \Bb_u) \\
&=
+\sum_{u< v} (\Ba_k \cdot \Bb_u) (\Ba_1 \cdot \Bb_v)
- (\Ba_k \cdot \Bb_v) (\Ba_1 \cdot \Bb_u) \\
\end{aligned}
\end{equation}
\begin{equation}\label{eqn:bladegradereduction:k2dot}
-\sum_{u< v}
\begin{vmatrix}
\Ba_{k-1} \cdot \Bb_u & \Ba_{k-1} \cdot \Bb_v \\
\Ba_k \cdot \Bb_u & \Ba_k \cdot \Bb_v \\
\end{vmatrix}
\end{equation}

\item \(k>2\)
\end{enumerate}

\begin{equation}\label{eqn:bladegradereduction:142}
\Ba_{k-1} \cdot (\Ba_k \cdot (\Bb_1 \wedge \Bb_2 \cdots \wedge \Bb_k))
\end{equation}
\begin{equation}\label{eqn:bladegradereduction:422}
\begin{aligned}
&= \sum (-1)^{u-1} (\Ba_k \cdot \Bb_u) \Ba_{k-1} \cdot (\Bb_1 \wedge \cdots \check{\Bb_u} \cdots \wedge \Bb_k) \\
&= \sum_{v<u} (-1)^{u-1} (\Ba_k \cdot \Bb_u) (-1)^{v-1} (\Ba_{k-1} \cdot \Bb_v) (\Bb_1 \wedge \cdots \check{\Bb_v} \cdots \check{\Bb_u} \cdots \wedge \Bb_k) \\
&+ \sum_{v>u} (-1)^{u-1} (\Ba_k \cdot \Bb_u) (-1)^{v} (\Ba_{k-1} \cdot \Bb_v) (\Bb_1 \wedge \cdots \check{\Bb_u} \cdots \check{\Bb_v} \cdots \wedge \Bb_k) \\
\end{aligned}
\end{equation}

Add negation exponents, and use a change of variables for the first sum
\begin{equation}\label{eqn:bladegradereduction:442}
\begin{aligned}
&= \sum_{u<v} (-1)^{v+u} (\Ba_k \cdot \Bb_v) (\Ba_{k-1} \cdot \Bb_u) (\Bb_1 \wedge \cdots \check{\Bb_u} \cdots \check{\Bb_v} \cdots \wedge \Bb_k) \\
&- \sum_{u<v} (-1)^{u+v} (\Ba_k \cdot \Bb_u) (\Ba_{k-1} \cdot \Bb_v) (\Bb_1 \wedge \cdots \check{\Bb_u} \cdots \check{\Bb_v} \cdots \wedge \Bb_k) \\
\end{aligned}
\end{equation}

Merge sums:
\begin{equation}\label{eqn:bladegradereduction:462}
\begin{aligned}
&= \sum_{u<v} (-1)^{u+v}
\left(
(\Ba_k \cdot \Bb_v) (\Ba_{k-1} \cdot \Bb_u)
-(\Ba_k \cdot \Bb_u) (\Ba_{k-1} \cdot \Bb_v)
\right) \\
& \; (\Bb_1 \wedge \cdots \check{\Bb_u} \cdots \check{\Bb_v} \cdots \wedge \Bb_k)
\end{aligned}
\end{equation}

\begin{equation}\label{eqn:bladegradereduction:bivectordotkvector}
\Ba_{k-1} \cdot (\Ba_k \cdot (\Bb_1 \wedge \Bb_2 \cdots \wedge \Bb_k))
=
\end{equation}
\begin{equation*}
\sum_{u<v} (-1)^{u+v}
\begin{vmatrix}
\Ba_{k-1} \cdot \Bb_u & \Ba_{k-1} \cdot \Bb_v \\
\Ba_k \cdot \Bb_u & \Ba_k \cdot \Bb_v \\
\end{vmatrix}
(\Bb_1 \wedge \cdots \check{\Bb_u} \cdots \check{\Bb_v} \cdots \wedge \Bb_k) \\
\end{equation*}

Note that special casing \(k=2\) does not seem to be required because in that
case \(-1^{u+v} = -1^{1+2}=-1\), so this is identical to \eqnref{eqn:bladegradereduction:k2dot} after all.

\subsection{Pause to reflect}

Although my initial aim was to show that \(\BA_k \cdot \BB_k\) could be
expressed as a determinant as in the differential forms book (different
sign though), and to determine exactly what that determinant is, there
are some useful identities that fall out of this even just for this
bivector kvector dot product expansion.

Here is a summary of some of the things figured out so far

\begin{enumerate}
\item Dot product of grade one blades.

Here we have a result that can be expressed as a one by one determinant.  Worth mentioning to explicitly show the sign.

\begin{equation}\label{eqn:bladegradereduction:dotoneblades}
\Ba \cdot \Bb = \det[\Ba \cdot \Bb]
\end{equation}

%(Used \(\det{}\) here instead of \(\Det{}\) to avoid confusing with absolute value).
\item Dot product of grade two blades.

\begin{equation}\label{eqn:bladegradereduction:k2k2dot}
(\Ba_1 \wedge \Ba_2) \cdot (\Bb_1 \wedge \Bb_2)
=
-
\begin{vmatrix}
\Ba_1 \cdot \Bb_1 & \Ba_1 \cdot \Bb_2 \\
\Ba_2 \cdot \Bb_1 & \Ba_2 \cdot \Bb_2 \\
\end{vmatrix}
=
-\det[\Ba_i \cdot \Bb_j]
\end{equation}

\item Dot product of grade two blade with grade \(>2\) blade.

\begin{equation*}
(\Ba_{1} \wedge \Ba_2) \cdot (\Bb_1 \wedge \Bb_2 \cdots \wedge \Bb_k)
\end{equation*}
\begin{equation}\label{eqn:bladegradereduction:bivectordot}
=
\sum_{u<v} (-1)^{u+v-1}
(\Ba_1 \wedge \Ba_2) \cdot (\Bb_u \wedge \Bb_v)
(\Bb_1 \wedge \cdots \check{\Bb_u} \cdots \check{\Bb_v} \cdots \wedge \Bb_k)
\end{equation}
\end{enumerate}

Observe how similar this is to the vector blade dot product expansion:

\begin{equation}\label{eqn:bladegradereduction:vectordot}
\Ba \cdot (\Bb_1 \wedge \Bb_2 \cdots \wedge \Bb_k)
=
\sum (-1)^{i-1}
(\Ba \cdot \Bb_i) (\Bb_1 \wedge \cdots \check{\Bb_i} \cdots \wedge \Bb_k)
\end{equation}

\subsubsection{Expand it for \texorpdfstring{\(k=3\)}{k equal 3}}

Explicit expansion of \eqnref{eqn:bladegradereduction:bivectordot} for the \(k=3\) case, is also helpful to get a feel for
the equation:

\begin{equation}\label{eqn:bladegradereduction:482}
\begin{aligned}
(\Ba_{1} \wedge \Ba_2) \cdot (\Bb_1 \wedge \Bb_2 \wedge \Bb_3)
&=
(\Ba_1 \wedge \Ba_2) \cdot (\Bb_1 \wedge \Bb_2) \Bb_3 \\
&+(\Ba_1 \wedge \Ba_2) \cdot (\Bb_3 \wedge \Bb_1) \Bb_2 \\
&+(\Ba_1 \wedge \Ba_2) \cdot (\Bb_2 \wedge \Bb_3) \Bb_1
\end{aligned}
\end{equation}

Observe the cross product like alternation in sign and indices.
This suggests that a more natural way to express the sign coefficient may be via a \(\Sgn(\pi)\) expression for the sign of the
permutation of indices.

\section{trivector dot product}

With the result of \eqnref{eqn:bladegradereduction:bivectordot}, or the earlier equivalent determinant expression in equation
\eqnref{eqn:bladegradereduction:bivectordotkvector} we are now in a position to evaluate the dot product of a trivector and a greater or equal grade blade.

\begin{equation*}
\Ba_1 \cdot ((\Ba_{2} \wedge \Ba_3) \cdot (\Bb_1 \wedge \Bb_2 \cdots \wedge \Bb_k))
\end{equation*}
\begin{equation}\label{eqn:bladegradereduction:502}
\begin{aligned}
&=
\sum_{u<v} (-1)^{u+v-1}
(\Ba_2 \wedge \Ba_3) \cdot (\Bb_u \wedge \Bb_v)
\Ba_1 \cdot (\Bb_1 \wedge \cdots \check{\Bb_u} \cdots \check{\Bb_v} \cdots \wedge \Bb_k)  \\
&=
\sum_{w<u<v} (-1)^{u+v+w}
(\Ba_2 \wedge \Ba_3) \cdot (\Bb_u \wedge \Bb_v)
(\Ba_1 \cdot \Bb_w) (\Bb_1 \wedge \cdots \check{\Bb_w} \cdots \check{\Bb_u} \cdots \check{\Bb_v} \cdots \wedge \Bb_k)  \\
&+\sum_{u<w<v} (-1)^{u+v+w-1}
(\Ba_2 \wedge \Ba_3) \cdot (\Bb_u \wedge \Bb_v)
(\Ba_1 \cdot \Bb_w) (\Bb_1 \wedge \cdots \check \Bb_u \cdots \check{\Bb_w} \cdots \check{\Bb_v} \cdots \wedge \Bb_k)  \\
&+\sum_{u<v<w} (-1)^{u+v+w}
(\Ba_2 \wedge \Ba_3) \cdot (\Bb_u \wedge \Bb_v)
(\Ba_1 \cdot \Bb_w) (\Bb_1 \wedge \cdots \check \Bb_u \cdots \check{\Bb_v} \cdots \check{\Bb_w} \cdots \wedge \Bb_k)  \\
\end{aligned}
\end{equation}

Change the indices of summation and grouping like terms we have:
\begin{equation}\label{eqn:bladegradereduction:522}
\begin{aligned}
\sum_{u<v<w} (-1)^{u+v+w}
(
&(\Ba_2 \wedge \Ba_3) \cdot (\Bb_v \wedge \Bb_w) (\Ba_1 \cdot \Bb_u)  \\
&-(\Ba_2 \wedge \Ba_3) \cdot (\Bb_u \wedge \Bb_w) (\Ba_1 \cdot \Bb_v)  \\
&+(\Ba_2 \wedge \Ba_3) \cdot (\Bb_u \wedge \Bb_v) (\Ba_1 \cdot \Bb_w)  \\
)
(\Bb_1 \wedge \cdots \check \Bb_u \cdots \check{\Bb_v} \cdots \check{\Bb_w} \cdots \wedge \Bb_k)  \\
\end{aligned}
\end{equation}

Now, each of the embedded dot products were in fact determinants:
\begin{equation}\label{eqn:bladegradereduction:162}
(\Ba_2 \wedge \Ba_3) \cdot (\Bb_x \wedge \Bb_y)
=
-
\begin{vmatrix}
\Ba_2 \cdot \Bb_x & \Ba_2 \cdot \Bb_y \\
\Ba_3 \cdot \Bb_x & \Ba_3 \cdot \Bb_y \\
\end{vmatrix}
\end{equation}

Thus, we can expand these triple dot products like so (factor of \(-1\) omitted):
\begin{equation}\label{eqn:bladegradereduction:542}
\begin{aligned}
&(\Ba_2 \wedge \Ba_3) \cdot (\Bb_v \wedge \Bb_w) (\Ba_1 \cdot \Bb_u) \\
&-(\Ba_2 \wedge \Ba_3) \cdot (\Bb_u \wedge \Bb_w) (\Ba_1 \cdot \Bb_v) \\
&+(\Ba_2 \wedge \Ba_3) \cdot (\Bb_u \wedge \Bb_v) (\Ba_1 \cdot \Bb_w)  \\
&=
(\Ba_1 \cdot \Bb_u)
\begin{vmatrix}
\Ba_2 \cdot \Bb_v & \Ba_2 \cdot \Bb_w \\
\Ba_3 \cdot \Bb_v & \Ba_3 \cdot \Bb_w \\
\end{vmatrix} \\
&-
(\Ba_1 \cdot \Bb_v)
\begin{vmatrix}
\Ba_2 \cdot \Bb_u & \Ba_2 \cdot \Bb_w \\
\Ba_3 \cdot \Bb_u & \Ba_3 \cdot \Bb_w \\
\end{vmatrix} \\
&+
(\Ba_1 \cdot \Bb_w)
\begin{vmatrix}
\Ba_2 \cdot \Bb_u & \Ba_2 \cdot \Bb_v \\
\Ba_3 \cdot \Bb_u & \Ba_3 \cdot \Bb_v \\
\end{vmatrix} \\
%&=
%\begin{vmatrix}
%\Ba_1 \cdot \Bb_u & 0 & 0 \\
%0 & \Ba_2 \cdot \Bb_v & \Ba_2 \cdot \Bb_w \\
%0 & \Ba_3 \cdot \Bb_v & \Ba_3 \cdot \Bb_w \\
%\end{vmatrix} \\
%&+
%\begin{vmatrix}
%0 & \Ba_1 \cdot \Bb_v & 0 \\
%\Ba_2 \cdot \Bb_u & 0 & \Ba_2 \cdot \Bb_w \\
%\Ba_3 \cdot \Bb_u & 0 & \Ba_3 \cdot \Bb_w \\
%\end{vmatrix} \\
%&+
%\begin{vmatrix}
%0 & 0 & \Ba_1 \cdot \Bb_w \\
%\Ba_2 \cdot \Bb_u & \Ba_2 \cdot \Bb_v & 0 \\
%\Ba_3 \cdot \Bb_u & \Ba_3 \cdot \Bb_v & 0 \\
%\end{vmatrix} \\
&=
\begin{vmatrix}
\Ba_1 \cdot \Bb_u & \Ba_1 \cdot \Bb_v & \Ba_1 \cdot \Bb_w \\
\Ba_2 \cdot \Bb_u & \Ba_2 \cdot \Bb_v & \Ba_2 \cdot \Bb_w \\
\Ba_3 \cdot \Bb_u & \Ba_3 \cdot \Bb_v & \Ba_3 \cdot \Bb_w \\
\end{vmatrix} \\
\end{aligned}
\end{equation}

Final back substitution gives:

\begin{equation*}
(\Ba_1 \wedge \Ba_{2} \wedge \Ba_3) \cdot (\Bb_1 \wedge \Bb_2 \cdots \wedge \Bb_k)
\end{equation*}
\begin{equation}\label{eqn:bladegradereduction:trivectordotdet}
=
\sum_{u<v<w} (-1)^{u+v+w-1}
\begin{vmatrix}
\Ba_1 \cdot \Bb_u & \Ba_1 \cdot \Bb_v & \Ba_1 \cdot \Bb_w \\
\Ba_2 \cdot \Bb_u & \Ba_2 \cdot \Bb_v & \Ba_2 \cdot \Bb_w \\
\Ba_3 \cdot \Bb_u & \Ba_3 \cdot \Bb_v & \Ba_3 \cdot \Bb_w \\
\end{vmatrix}
(\Bb_1 \wedge \cdots \check \Bb_u \cdots \check{\Bb_v} \cdots \check{\Bb_w} \cdots \wedge \Bb_k)  \\
\end{equation}

In particular for \(k=3\) we have
\begin{equation*}
(\Ba_1 \wedge \Ba_{2} \wedge \Ba_3) \cdot (\Bb_1 \wedge \Bb_2 \wedge \Bb_3)
\end{equation*}
\begin{equation}\label{eqn:bladegradereduction:trivectordotdettri}
=
-\begin{vmatrix}
\Ba_1 \cdot \Bb_1 & \Ba_1 \cdot \Bb_2 & \Ba_1 \cdot \Bb_3 \\
\Ba_2 \cdot \Bb_1 & \Ba_2 \cdot \Bb_2 & \Ba_2 \cdot \Bb_3 \\
\Ba_3 \cdot \Bb_1 & \Ba_3 \cdot \Bb_2 & \Ba_3 \cdot \Bb_3 \\
\end{vmatrix}
=
-\det[\Ba_i \cdot \Bb_j]
\end{equation}

This can be substituted back into \eqnref{eqn:bladegradereduction:trivectordotdet} to put it in a non determinant form.

\begin{equation*}
(\Ba_1 \wedge \Ba_{2} \wedge \Ba_3) \cdot (\Bb_1 \wedge \Bb_2 \cdots \wedge \Bb_k)
\end{equation*}
\begin{equation}\label{eqn:bladegradereduction:trivectordotnondet}
=
\sum_{u<v<w} (-1)^{u+v+w}
(\Ba_1 \wedge \Ba_{2} \wedge \Ba_3) \cdot (\Bb_u \wedge \Bb_v \wedge \Bb_w)
(\Bb_1 \wedge \cdots \check \Bb_u \cdots \check{\Bb_v} \cdots \check{\Bb_w} \cdots \wedge \Bb_k)  \\
\end{equation}

\section{Induction on the result}

It is pretty clear that recursively performing these calculations will yield similar determinant and inner dot product reduction
results.

\subsection{dot product of like grade terms as determinant}

Let us consider the equal grade case first, summarizing the results so far

\begin{equation}\label{eqn:bladegradereduction:562}
\begin{aligned}
\Ba \cdot \Bb &= \det[\Ba \cdot \Bb] \\
(\Ba_1 \wedge \Ba_2) \cdot (\Bb_1 \wedge \Bb_2) &= -\det[\Ba_i \cdot \Bb_j] \\
(\Ba_1 \wedge \Ba_2 \wedge \Ba_3) \cdot (\Bb_1 \wedge \Bb_2 \wedge \Bb_3) &= -\det[\Ba_i \cdot \Bb_j] \\
\end{aligned}
\end{equation}

What will the sign be for the higher grade equivalents?  It has the appearance of being related to the sign associated with blade
reversion.  To verify this calculate the dot product of a blade formed from a set of perpendicular unit vectors with itself.

\begin{equation}\label{eqn:bladegradereduction:582}
\begin{aligned}
&(\Be_1 \wedge \cdots \wedge \Be_k) \cdot (\Be_1 \wedge \Be_2 \wedge \cdots \wedge \Be_k) \\
&= (-1)^{k(k-1)/2}(\Be_1 \wedge \cdots \wedge \Be_k) \cdot (\Be_k \wedge \cdots \wedge \Be_2 \wedge \Be_1) \\
&= (-1)^{k(k-1)/2}\Be_1 \cdot (\Be_2 \cdots (\Be_k \cdot (\Be_k \wedge \cdots \wedge \Be_2 \wedge \Be_1))) \\
&= (-1)^{k(k-1)/2}\Be_1 \cdot (\Be_2 \cdots (\Be_{k-1} \cdot (\Be_{k-1} \wedge \cdots \wedge \Be_2 \wedge \Be_1))) \\
&= \cdots \\
&= (-1)^{k(k-1)/2}
\end{aligned}
\end{equation}

This fixes the sign, and provides the induction hypothesis for the general case:

\begin{equation}\label{eqn:bladegradereduction:bladedothyp}
(\Ba_1 \wedge \cdots \wedge \Ba_k) \cdot (\Bb_1 \wedge \Bb_2 \wedge \cdots \wedge \Bb_k) = (-1)^{k(k-1)/2}\det[\Ba_i \cdot \Bb_j]
\end{equation}

Alternately, one can remove the sign change coefficient with reversion of one of the blades:

\begin{equation}\label{eqn:bladegradereduction:bladedothyprev}
(\Ba_1 \wedge \cdots \wedge \Ba_k) \cdot (\Bb_k \wedge \Bb_{k-1} \wedge \cdots \wedge \Bb_1) = \det[\Ba_i \cdot \Bb_j]
\end{equation}

\subsection{Unlike grades}

Let us summarize the results for unlike grades at the same time reformulating the previous results in terms of index
permutation, also writing for brevity \(\BA_s = \Ba_1 \wedge \cdots \wedge \Ba_s\), and \(\BB_k = \Bb_1 \wedge \cdots \wedge \Bb_k\):

\begin{equation}\label{eqn:bladegradereduction:182}
\BA_1 \cdot \BB_k =
\sum_i \Sgn(\pi(i,1,2,\cdots\check{i}\cdots,k)) (\BA_1 \cdot \Bb_i) (\Bb_1 \wedge \cdots \check{\Bb_i} \cdots \wedge \Bb_k)
\end{equation}

\begin{equation}\label{eqn:bladegradereduction:202}
\BA_2 \cdot \BB_k =
\sum_{i_1<i_2} \Sgn(\pi(i_1,i_2,1,2,\cdots\check{i_1}\cdots\check{i_2}\cdots,k))
\end{equation}
\begin{equation}\label{eqn:bladegradereduction:222}
   \BA_2 \cdot (\Bb_{i_1} \wedge \Bb_{i_2})
   (\Bb_1 \wedge \cdots \check{\Bb_{i_1}} \cdots \check{\Bb_{i_2}} \cdots \wedge \Bb_k)
\end{equation}

\begin{equation}\label{eqn:bladegradereduction:242}
\BA_3 \cdot \BB_k =
\sum_{i_1<i_2<i_3} \Sgn(\pi(i_1,i_2,i_3,1,2,\cdots\check{i_1}\cdots\check{i_2}\cdots\check{i_3}\cdots,k))
\end{equation}
\begin{equation}\label{eqn:bladegradereduction:262}
\BA_3 \cdot (\Bb_{i_1} \wedge \Bb_{i_2} \wedge \Bb_{i_3})
(\Bb_1 \wedge \cdots \check{\Bb_{i_1}} \cdots \check{\Bb_{i_2}} \cdots \check{\Bb_{i_3}} \cdots \wedge \Bb_k)
\end{equation}

We see that the dot product consumes any of the excess sign variation not described by the sign of the permutation of indices.

The induction hypothesis is basically described above (change \(3\) to \(s\), and add extra dots):

\begin{equation*}
\BA_s \cdot \BB_k =
\sum_{i_1<i_2\cdots<i_s} \Sgn(\pi(i_1,i_2\cdots,i_s,1,2,\cdots\check{i_1}\cdots\check{i_2}\cdots\check{i_s}\cdots,k))
\end{equation*}
\begin{equation}\label{eqn:bladegradereduction:inductionbigdotblade}
\BA_s \cdot (\Bb_{i_1} \wedge \Bb_{i_2} \cdots \wedge \Bb_{i_s})
(\Bb_1 \wedge \cdots \check{\Bb_{i_1}} \cdots \check{\Bb_{i_2}} \cdots \check{\Bb_{i_s}} \cdots \wedge \Bb_k)
\end{equation}

\subsection{Perform the induction}

In a sense this has already been done.  The steps will be pretty much the same as the logic that produced the bivector and trivector
results.  Thinking about typing this up in latex is not fun, so this will be left for a paper proof.

\documentclass{article}      

\usepackage{amsmath}
\usepackage{mathpazo}

%
% shorthand for bold symbols, convenient for vectors and matrices
%
\newcommand{\Ba}[0]{\mathbf{a}}
\newcommand{\Bb}[0]{\mathbf{b}}
\newcommand{\Bc}[0]{\mathbf{c}}
\newcommand{\Bd}[0]{\mathbf{d}}
\newcommand{\Be}[0]{\mathbf{e}}
\newcommand{\Bf}[0]{\mathbf{f}}
\newcommand{\Bg}[0]{\mathbf{g}}
\newcommand{\Bh}[0]{\mathbf{h}}
\newcommand{\Bi}[0]{\mathbf{i}}
\newcommand{\Bj}[0]{\mathbf{j}}
\newcommand{\Bk}[0]{\mathbf{k}}
\newcommand{\Bl}[0]{\mathbf{l}}
\newcommand{\Bm}[0]{\mathbf{m}}
\newcommand{\Bn}[0]{\mathbf{n}}
\newcommand{\Bo}[0]{\mathbf{o}}
\newcommand{\Bp}[0]{\mathbf{p}}
\newcommand{\Bq}[0]{\mathbf{q}}
\newcommand{\Br}[0]{\mathbf{r}}
\newcommand{\Bs}[0]{\mathbf{s}}
\newcommand{\Bt}[0]{\mathbf{t}}
\newcommand{\Bu}[0]{\mathbf{u}}
\newcommand{\Bv}[0]{\mathbf{v}}
\newcommand{\Bw}[0]{\mathbf{w}}
\newcommand{\Bx}[0]{\mathbf{x}}
\newcommand{\By}[0]{\mathbf{y}}
\newcommand{\Bz}[0]{\mathbf{z}}
\newcommand{\BA}[0]{\mathbf{A}}
\newcommand{\BB}[0]{\mathbf{B}}
\newcommand{\BC}[0]{\mathbf{C}}
\newcommand{\BD}[0]{\mathbf{D}}
\newcommand{\BE}[0]{\mathbf{E}}
\newcommand{\BF}[0]{\mathbf{F}}
\newcommand{\BG}[0]{\mathbf{G}}
\newcommand{\BH}[0]{\mathbf{H}}
\newcommand{\BI}[0]{\mathbf{I}}
\newcommand{\BJ}[0]{\mathbf{J}}
\newcommand{\BK}[0]{\mathbf{K}}
\newcommand{\BL}[0]{\mathbf{L}}
\newcommand{\BM}[0]{\mathbf{M}}
\newcommand{\BN}[0]{\mathbf{N}}
\newcommand{\BO}[0]{\mathbf{O}}
\newcommand{\BP}[0]{\mathbf{P}}
\newcommand{\BQ}[0]{\mathbf{Q}}
\newcommand{\BR}[0]{\mathbf{R}}
\newcommand{\BS}[0]{\mathbf{S}}
\newcommand{\BT}[0]{\mathbf{T}}
\newcommand{\BU}[0]{\mathbf{U}}
\newcommand{\BV}[0]{\mathbf{V}}
\newcommand{\BW}[0]{\mathbf{W}}
\newcommand{\BX}[0]{\mathbf{X}}
\newcommand{\BY}[0]{\mathbf{Y}}
\newcommand{\BZ}[0]{\mathbf{Z}}

\newcommand{\Bzero}[0]{\mathbf{0}}
\newcommand{\Btheta}[0]{\boldsymbol{\theta}}
\newcommand{\Btau}[0]{\boldsymbol{\tau}}
\newcommand{\Bomega}[0]{\boldsymbol{\omega}}

%
% shorthand for unit vectors
%
\newcommand{\acap}[0]{\hat{\Ba}}
\newcommand{\bcap}[0]{\hat{\Bb}}
\newcommand{\ccap}[0]{\hat{\Bc}}
\newcommand{\dcap}[0]{\hat{\Bd}}
\newcommand{\ecap}[0]{\hat{\Be}}
\newcommand{\fcap}[0]{\hat{\Bf}}
\newcommand{\gcap}[0]{\hat{\Bg}}
\newcommand{\hcap}[0]{\hat{\Bh}}
\newcommand{\icap}[0]{\hat{\Bi}}
\newcommand{\jcap}[0]{\hat{\Bj}}
\newcommand{\kcap}[0]{\hat{\Bk}}
\newcommand{\lcap}[0]{\hat{\Bl}}
\newcommand{\mcap}[0]{\hat{\Bm}}
\newcommand{\ncap}[0]{\hat{\Bn}}
\newcommand{\ocap}[0]{\hat{\Bo}}
\newcommand{\pcap}[0]{\hat{\Bp}}
\newcommand{\qcap}[0]{\hat{\Bq}}
\newcommand{\rcap}[0]{\hat{\Br}}
\newcommand{\scap}[0]{\hat{\Bs}}
\newcommand{\tcap}[0]{\hat{\Bt}}
\newcommand{\ucap}[0]{\hat{\Bu}}
\newcommand{\vcap}[0]{\hat{\Bv}}
\newcommand{\wcap}[0]{\hat{\Bw}}
\newcommand{\xcap}[0]{\hat{\Bx}}
\newcommand{\ycap}[0]{\hat{\By}}
\newcommand{\zcap}[0]{\hat{\Bz}}
\newcommand{\thetacap}[0]{\hat{\Btheta}}

%
% to write R^n and C^n in a distinguishable fashion.  Perhaps change this
% to the double lined characters upon figuring out how to do so.
%
\newcommand{\C}[1]{$\mathbb{C}^{#1}$}
\newcommand{\R}[1]{$\mathbb{R}^{#1}$}

%
% various generally useful helpers
%

% derivative of #1 wrt. #2:
\newcommand{\D}[2] {\frac {d#2} {d#1}}

\newcommand{\inv}[1]{\frac{1}{#1}}
\newcommand{\cross}[0]{\times}

\newcommand{\abs}[1]{\lvert{#1}\rvert}
\newcommand{\norm}[1]{\lVert{#1}\rVert}
\newcommand{\innerprod}[2]{\langle{#1}, {#2}\rangle}
\newcommand{\dotprod}[2]{{#1} \cdot {#2}}
\newcommand{\bdotprod}[2]{\left({#1} \cdot {#2}\right)}
\newcommand{\crossprod}[2]{{#1} \cross {#2}}
\newcommand{\tripleprod}[3]{\dotprod{\left(\crossprod{#1}{#2}\right)}{#3}}

\DeclareMathOperator{\Proj}{Proj}
\DeclareMathOperator{\Span}{span}
\DeclareMathOperator{\Sgn}{sgn}
\DeclareMathOperator{\Area}{Area}
\DeclareMathOperator{\Volume}{Volume}

%
% A few miscellaneous things specific to this document
%
\newcommand{\crossop}[1]{\crossprod{#1}{}}

% R2 vector.
\newcommand{\VectorTwo}[2]{
\begin{bmatrix}
 {#1} \\
 {#2}
\end{bmatrix}
}

\newcommand{\VectorN}[1]{
\begin{bmatrix}
{#1}_1 \\
{#1}_2 \\
\vdots \\
{#1}_N \\
\end{bmatrix}
}

\newcommand{\DETuvij}[4]{
\begin{vmatrix}
 {#1}_{#3} & {#1}_{#4} \\
 {#2}_{#3} & {#2}_{#4}
\end{vmatrix}
}

\newcommand{\DETuvwijk}[6]{
\begin{vmatrix}
 {#1}_{#4} & {#1}_{#5} & {#1}_{#6} \\
 {#2}_{#4} & {#2}_{#5} & {#2}_{#6} \\
 {#3}_{#4} & {#3}_{#5} & {#3}_{#6}
\end{vmatrix}
}

\newcommand{\DETuvwxijkl}[8]{
\begin{vmatrix}
 {#1}_{#5} & {#1}_{#6} & {#1}_{#7} & {#1}_{#8} \\
 {#2}_{#5} & {#2}_{#6} & {#2}_{#7} & {#2}_{#8} \\
 {#3}_{#5} & {#3}_{#6} & {#3}_{#7} & {#3}_{#8} \\
 {#4}_{#5} & {#4}_{#6} & {#4}_{#7} & {#4}_{#8} \\
\end{vmatrix}
}

%\newcommand{\DETuvwxyijklm}[10]{
%\begin{vmatrix}
% {#1}_{#6} & {#1}_{#7} & {#1}_{#8} & {#1}_{#9} & {#1}_{#10} \\
% {#2}_{#6} & {#2}_{#7} & {#2}_{#8} & {#2}_{#9} & {#2}_{#10} \\
% {#3}_{#6} & {#3}_{#7} & {#3}_{#8} & {#3}_{#9} & {#3}_{#10} \\
% {#4}_{#6} & {#4}_{#7} & {#4}_{#8} & {#4}_{#9} & {#4}_{#10} \\
% {#5}_{#6} & {#5}_{#7} & {#5}_{#8} & {#5}_{#9} & {#5}_{#10}
%\end{vmatrix}
%}

% R3 vector.
\newcommand{\VectorThree}[3]{
\begin{bmatrix}
 {#1} \\
 {#2} \\
 {#3}
\end{bmatrix}
}



                             % The preamble begins here.
\title{More details on NFCM plane formulation} % Declares the document's title.
\author{Peeter Joot}         % Declares the author's name.
%\date{}        % Deleting this command produces today's date.

\begin{document}             % End of preamble and beginning of text.

%\maketitle{}

\section{Wedge product formula for a plane.}

The equation of the plane with bivector $\BU$ through point $\Ba$ is given
by

\[
(\Bx - \Ba) \wedge \BU = 0
\]

or

\[
\Bx \wedge \BU = \Ba \wedge \BU = \BT
\]

\subsection{ Examining this equation in more details. }

Without any loss of generality one can express this plane equation
in terms of a unit bivector $\Bi$

\[
\Bx \wedge \Bi = \Ba \wedge \Bi
\]

As with the line equation, to express this in the ``standard'' parametric
form, right multiplication with $1/\Bi$ is required.

\[
(\Bx \wedge \Bi)\frac{1}{\Bi} = (\Ba \wedge \Bi)\frac{1}{\Bi}
\]

We have a trivector bivector product here, which in general has a vector,
trivector, and 5-vector component.  Since $\Bi \wedge \Bi = 0$, the
5-vector component is zero:

\[
\Bx \wedge \Bi \wedge -\Bi = 0
\]

and intuition says that the trivector component will also be zero.  However,
as well as providing verification of this, expansion of this product will also
demonstrate how to find the projective and rejective components of a vector
with respect to a plane (ie: components in and out of the plane).

\subsection{Rejection from a plane product expansion.}

Here's an explicit expansion of the rejective term above

\begin{align*}
(\Bx \wedge \Bi)\frac{1}{\Bi} 
&= -(\Bx \wedge \Bi){\Bi} \\ 
&= -\frac{1}{2}(\Bx\Bi + \Bi\Bx){\Bi} \\ 
&= \frac{1}{2}(\Bx - \Bi\Bx\Bi) \\ 
&= \frac{1}{2}(\Bx - (\Bx \Bi + 2 \Bi \cdot \Bx)\Bi) \\ 
&= \Bx - (\Bi \cdot \Bx)\Bi \\ 
\end{align*}

In this last term the quantity $\Bi \cdot \Bx$ is a vector in the plane.
This can be demonstrated by writing $\Bi$ in terms of a pair of orthonormal
vectors $\Bi = \ucap\vcap = \ucap \wedge \vcap$.

\begin{align*}
\Bi \cdot \Bx &= (\ucap \wedge \vcap) \cdot \Bx \\
              &= \ucap (\vcap \cdot \Bx) - \vcap (\ucap \cdot \Bx) \\
\end{align*}

Thus, $(\Bi \cdot \Bx) \wedge \Bi = 0$, 
and $(\Bi \cdot \Bx) \Bi = (\Bi \cdot \Bx) \cdot \Bi$.  Inserting this above
we have the end result

\begin{align*}
(\Bx \wedge \Bi)\frac{1}{\Bi} 
&= \Bx - (\Bi \cdot \Bx) \cdot \Bi \\ 
&= \Ba - (\Bi \cdot \Ba) \cdot \Bi \\ 
\end{align*}

Or
\begin{align*}
\Bx  - \Ba 
&= (\Bi \cdot (\Bx - \Ba)) \cdot \Bi \\ 
\end{align*}

This is actually the standard parametric equation of a plane, but expressed
in terms of a unit bivector that describes the plane instead of in terms
of a pair of vectors in the plane.

To demonstrate this expansion of the right hand side is required

\begin{align*}
(\Bi \cdot \Bx) \cdot \Bi
&= (\ucap (\vcap \cdot \Bx) - \vcap (\ucap \cdot \Bx)) \ucap \vcap \\
&= \vcap (\vcap \cdot \Bx) + \ucap (\ucap \cdot \Bx) \\
\end{align*}

Substituting this back yields:

\begin{align*}
\Bx 
&= \Ba + \ucap (\ucap \cdot (\Bx - \Ba)) + \vcap (\vcap \cdot (\Bx - \Ba)) \\
&= \Ba + s \ucap + t \vcap
\end{align*}

In words this says that the plane is specified by a point in the plane,
and the span
of a pair of orthonormal vectors directed in that plane.

This (but perhaps without neccessariliy using orthornomal direction vectors)
is often how the plane is defined to start with.

It isn't neccessarily obvious that the bivector wedge product formula for
a plane that we started with:

\[
\Bx \wedge \BU = \Ba \wedge \BU
\]

can also be used to express this parametric representation.

\subsection{ Orthonormal decomposition of a vector with respect to a plane. }

With the expansion above we have a separation of a vector into two
components, and these can be demonstrated to be the components that are
directed entirely within and out of the plane.

Rearranging terms from above we have:

\begin{align*}
\Bx 
&= 
(\Bx \cdot \Bi) \cdot \frac{1}{\Bi} + (\Bx \wedge \Bi) \cdot \frac{1}{\Bi} \\
&= 
(\Bx \cdot \Bi) \frac{1}{\Bi} + (\Bx \wedge \Bi) \frac{1}{\Bi} \\
\end{align*}

% write x = x_perp + x_parallel to show that this is a ortho decomp.
% can then write formula for directrix of plane.

\subsection{ Alternate derivation of orthonormal planar decomposition }

This could alternately be derived by expanding the vector unit bivector
product directly

\begin{align*}
\Bx \Bi \frac{1}{\Bi} 
&= ( \Bx \cdot \Bi + \Bx \wedge \Bi ) \frac{1}{\Bi} \\
&= 
- {(\Bx \cdot \Bi) \cdot \Bi} - {(\Bx \cdot \Bi) \wedge \Bi} - {(\Bx \wedge \Bi) \Bi} \\
&= 
- {(\Bx \cdot \Bi) \cdot \Bi} - {(\Bx \wedge \Bi) \cdot \Bi } - {<(\Bx \wedge \Bi) \Bi>_3} - {(\Bx \wedge \Bi) \wedge \Bi} \\
&= 
{(\Bx \cdot \Bi) \cdot \frac{1}{\Bi}} + {(\Bx \wedge \Bi) \cdot \frac{1}{\Bi}} - {<(\Bx \wedge \Bi) \Bi>_3} \\
\end{align*}

Since the LHS of this equation is the vector $\Bx$, the right hand side must
also be a vector, which demonstrates that the term

\[
<(\Bx \wedge \Bi) \Bi>_3 = 0
\]

So, one has

\begin{align*}
\Bx 
&=
{(\Bx \cdot \Bi) \cdot \frac{1}{\Bi}} + {(\Bx \wedge \Bi) \cdot \frac{1}{\Bi}} \\
&=
{(\Bx \cdot \Bi) \frac{1}{\Bi}} + {(\Bx \wedge \Bi) \frac{1}{\Bi}} \\
\end{align*}


\end{document}

%
% Copyright � 2012 Peeter Joot.  All Rights Reserved.
% Licenced as described in the file LICENSE under the root directory of this GIT repository.
%

%
%
\chapter{Quaternions}
\index{quaternion}
\label{chap:quaternion}
%\date{Feb 2, 2008.  quaternion.tex}

Like complex numbers, quaternions may be written as a multivector with scalar and bivector components (a 0,2-multivector).

\begin{equation}\label{eqn:quaternion:20}
q = \alpha + \mathbf{B}
\end{equation}

Where the complex number has one bivector component, and the quaternions have three.

One can describe quaternions as 0,2-multivectors where the basis for the bivector part is left handed.  There is not really anything special about quaternion multiplication, or complex number multiplication, for that matter.  Both are just a specific examples of a 0,2-multivector multiplication.  Other quaternion operations can also be found to have natural multivector equivalents.  The most important of which is likely the quaternion conjugate, since it implies the norm and the inverse.  As a multivector, like complex numbers, the conjugate operation is reversal:

\begin{equation}\label{eqn:quaternion:40}
\overline{q} = q^\dagger = \alpha - \mathbf{B}
\end{equation}

Thus \(\abs{q}^2 = q\overline{q} = \alpha^2 - \mathbf{B}^2\).  Note that this norm is a positive definite as expected since a bivector square is negative.

To be more specific about the left handed basis property of quaternions one can note that the quaternion bivector basis is usually defined in terms of the following properties

\begin{equation}\label{eqn:quaternion:60}
\mathbf{i}^2 = \mathbf{j}^2 = \mathbf{k}^2 = -1
\end{equation}
\begin{equation}\label{eqn:quaternion:80}
\mathbf{i}\mathbf{j} = -\mathbf{j}\mathbf{i}, \mathbf{i}\mathbf{k} = -\mathbf{k}\mathbf{i}, \mathbf{j}\mathbf{k} = -\mathbf{k}\mathbf{j}
\end{equation}
\begin{equation}\label{eqn:quaternion:100}
\mathbf{i}\mathbf{j} = \mathbf{k}
\end{equation}

The first two properties are satisfied by any set of orthogonal unit bivectors for the space.  The last property, which could also be written \(\mathbf{i}\mathbf{j}\mathbf{k} = -1\), amounts to a choice for the orientation of this bivector basis of the 2-vector part of the quaternion.

As an example suppose one picks

\begin{equation}\label{eqn:quaternion:120}
\mathbf{i} = \mathbf{e}_2\mathbf{e}_3
\end{equation}
\begin{equation}\label{eqn:quaternion:140}
\mathbf{j} = \mathbf{e}_3\mathbf{e}_1
\end{equation}

Then the third bivector required to complete the basis set subject to the properties above is

\begin{equation}\label{eqn:quaternion:160}
\mathbf{i}\mathbf{j} = \mathbf{e}_2\mathbf{e}_1 = \mathbf{k}
\end{equation}.

Suppose that, instead of the above, one picked a slightly more natural bivector basis, the duals of the unit vectors obtained by multiplication with the pseudoscalar (\(\mathbf{e}_1\mathbf{e}_2\mathbf{e}_3\mathbf{e}_i\)).  These bivectors are

\begin{equation}\label{eqn:quaternion:180}
\mathbf{i}=\mathbf{e}_2\mathbf{e}_3, \mathbf{j}=\mathbf{e}_3\mathbf{e}_1, \mathbf{k}=\mathbf{e}_1\mathbf{e}_2
\end{equation}.

A 0,2-multivector with this as the basis for the bivector part would have properties similar to the standard quaternions (anti-commutative unit quaternions, negation for unit quaternion square, same conjugate, norm and inversion operations, ...), however the triple product would have the value \(\mathbf{i}\mathbf{j}\mathbf{k} = 1\), instead of \(-1\).

\section{quaternion as generator of dot and cross product}

The product of pure quaternions is noted as being a generator of dot and cross products.  This is also true
of a vector bivector product.

Writing a vector \(\Bx\) as

\begin{equation}\label{eqn:quaternion:200}
\Bx = \sum_i x_i \Be_i = x_1 \Be_1 + x_2 \Be_2 + x_3 \Be_3
\end{equation}

And a bivector \(\BB\) (where for short, \(\Be_{ij} = \Be_i \Be_j = \Be_i \wedge \Be_j\)) as:

\begin{equation}\label{eqn:quaternion:220}
\BB = \sum_i b_i \Be_i I = b_1 \Be_{23} + b_2 \Be_{31} + b_3 \Be_{12}
\end{equation}

The product of these two is
\begin{equation}\label{eqn:quaternion:280}
\begin{aligned}
\Bx \BB
&= (x_1 \Be_1 + x_2 \Be_2 + x_3 \Be_3)(b_1 \Be_{23} + b_2 \Be_{31} + b_3 \Be_{12}) \\
&= (x_3 b_2 - x_2 b_3) \Be_1 + (x_1 b_3 - x_3 b_1) \Be_2 + (x_2 b_1 - x_1 b_2) \Be_3 \\
&+ (x_1 b_1 + x_2 b_2 + x_3 b_3) \Be_{123} \\
\end{aligned}
\end{equation}

Looking at the vector and trivector components of this we recognize the dot product and negated cross product
immediately (as with multiplication of pure quaternions).

Those products are, in fact, \(\Bx \cdot \BB\) and \(\Bx \wedge \BB\) respectively.

Introducing a vector and bivector basis \(\alpha = \{ \Be_i \}\), and \(\beta = \{ \Be_i I \}\), we can
express the dot product and cross product of the associated coordinate vectors
in terms of vector bivectors products as follows:

\begin{equation}\label{eqn:quaternion:240}
[\Bx]_\alpha \cdot [\BB]_\beta = \frac{\BB \wedge \Bx}{I}
\end{equation}
\begin{equation}\label{eqn:quaternion:260}
[\Bx]_\alpha \cross [\BB]_\beta = [\BB \cdot \Bx]_\alpha
\end{equation}


\include{cauchy_gradient}
%
% Copyright � 2012 Peeter Joot.  All Rights Reserved.
% Licenced as described in the file LICENSE under the root directory of this GIT repository.
%

%
%
\chapter{Legendre Polynomials}
\index{Legendre polynomial}
\label{chap:legendre}
%\date{Feb 4, 2008.  legendre.tex}

Exercise 8.4, from \citep{hestenes1999nfc}.

Find the first couple terms of the Legendre polynomial expansion of

\begin{equation}\label{eqn:legendre:20}
\inv{\abs{\Bx - \Ba}}
\end{equation}

Write

\begin{equation}\label{eqn:legendre:40}
f(x) = \inv{\abs{\Bx}}
\end{equation}

Expanding \(f(\Bx - \Ba)\) about \(\Bx\) we have

\begin{equation}\label{eqn:legendre:60}
\inv{\abs{\Bx - \Ba}} =
\sum_{k=0}{ \inv{k!} (-\agrad)^k} \inv{\abs{\Bx}}
\end{equation}

Expanding the first term we have

\begin{equation}\label{eqn:legendre:200}
\begin{aligned}
-\agrad \inv{\abs{\Bx}}
&=
\inv{{\abs{\Bx}}^2} \agrad {\abs{\Bx}} \\
&=
\inv{{\abs{\Bx}}^2} \agrad (\Bx^2)^{1/2} \\
&=
\inv{{\abs{\Bx}}^2} \frac{(1/2)}{({\abs{\Bx}}^2)^{1/2}}\agrad \Bx^2 \\
&=
\frac{\Ba \cdot \Bx}{{\abs{\Bx}}^3}
\end{aligned}
\end{equation}

Expansion of the second derivative term is
\begin{equation}\label{eqn:legendre:220}
\begin{aligned}
\frac{(-\agrad)}{2}\frac{(-\agrad)}{1}\inv{\abs{\Bx}}
&=
\frac{\agrad}{2} \left(\frac{-\Ba \cdot \Bx}{{\abs{\Bx}}^3}\right) \\
&=
\frac{-1}{2}
\left(
\frac{\agrad {(\Ba \cdot \Bx)}}{{\abs{\Bx}}^3} + {(\Ba \cdot \Bx)}\agrad \inv{{\abs{\Bx}}^3} \right) \\
\end{aligned}
\end{equation}

For this we need
\begin{equation}\label{eqn:legendre:80}
\agrad {(\Ba \cdot \Bx)} =
\Ba \cdot (\agrad {\Bx}) = \Ba^2
\end{equation}

And
\begin{equation}\label{eqn:legendre:240}
\begin{aligned}
\agrad \inv{{\abs{\Bx}}^k}
&=
k \inv{{\abs{\Bx}}^{k-1}} \agrad \inv{{\abs{\Bx}}} \\
&=
k \inv{{\abs{\Bx}}^{k-1}} \frac{- \Ba \cdot \Bx }{{\abs{\Bx}}^3} \\
&=
-k \frac{\Ba \cdot \Bx }{{\abs{\Bx}}^{k+2}} \\
\end{aligned}
\end{equation}

Thus the second derivative term is
\begin{equation}\label{eqn:legendre:260}
\begin{aligned}
\frac{-1}{2}
\left(
\frac{\Ba^2}{{\abs{\Bx}}^3} -3 \frac{(\Ba \cdot \Bx)^2} {{\abs{\Bx}}^5} \right)
=
\frac{ (1/2)\left( 3 (\Ba \cdot \Bx)^2 - \Ba^2 \Bx^2 \right) }
{ {{\abs{\Bx}}^5} }
\end{aligned}
\end{equation}

Summing these terms we have

\begin{equation}\label{eqn:legendre:100}
\inv{\abs{\Bx -\Ba}} =
\inv{\abs{\Bx}} +
\frac{ \Ba \cdot \Bx } { {\abs{\Bx}}^3 } +
\frac{ (1/2)\left( 3 (\Ba \cdot \Bx)^2 - \Ba^2 \Bx^2 \right) } { {{\abs{\Bx}}^5} } + \cdots
\end{equation}

NFCM writes this as
\begin{equation}\label{eqn:legendre:120}
\inv{\abs{\Bx -\Ba}} =
\frac{ P_0(\bxa) } {  \abs{\Bx}} +
\frac{ P_1(\bxa) } { {\abs{\Bx}}^3 } +
\frac{ P_2(\bxa) } { {\abs{\Bx}}^5 } + \cdots
\end{equation}

And calls \(P_i = P_i(\bxa)\) terms the Legendre polynomials.  This is not terribly clear since one expects a different form for the Legendre polynomials.

Using the Taylor formula one can derive a recurrence relation for these that makes the calculation a bit
simpler

\begin{equation}\label{eqn:legendre:280}
\begin{aligned}
\frac{P_{k+1}}{\abs{\Bx}^{2(k+1)+1}}
&= \frac{-\agrad}{k+1}\left(\frac{P_k}{\abs{\Bx}^{2k+1}}\right) \\
&=
\frac{-1}{k+1}
\left(
\frac{\agrad({P_k}}
{\abs{\Bx}^{2k+1}}
+
{P_k}\frac{\agrad}
{\abs{\Bx}^{2k+1}}
\right) \\
&=
\inv{k+1}
\left(
{P_k}(2k+1) \frac{\Ba \cdot \Bx}
{\abs{\Bx}^{2k+3}}
-\Bx^2 \frac{\agrad{P_k}}
{\abs{\Bx}^{2k+3}}
\right) \\
\end{aligned}
\end{equation}

Or
\begin{equation}\label{eqn:legendre:300}
\begin{aligned}
(k+1){P_{k+1}}
=
{P_k}(2k+1) {\Ba \cdot \Bx}
-\Bx^2 {\agrad{P_k}}
\end{aligned}
\end{equation}

Some of these have been calculated

\begin{equation}\label{eqn:legendre:320}
\begin{aligned}
P_0 &= 1 \\
P_1 &= \Ba \cdot \Bx \\
P_2 &= \half(3(\Ba \cdot \Bx)^2 -\Ba^2\Bx^2) \\
\end{aligned}
\end{equation}

And for the derivatives

\begin{equation}\label{eqn:legendre:340}
\begin{aligned}
\agrad P_0 &= 0 \\
\agrad P_1 &= \Ba^2 \\
\agrad P_2 &= \half((3)(2)(\Ba \cdot \Bx)\Ba^2 - 2\Ba^2\Bx \cdot \Ba) \\
           &= 2\Ba^2(\Bx \cdot \Ba) \\
\end{aligned}
\end{equation}

Using the recurrence relation one can calculate \(P_3\) for example.

\begin{equation}\label{eqn:legendre:360}
\begin{aligned}
P_3
%(k+1){P_{k+1}} ; k=2
&=
(1/3)\left(
\frac{5}{2}(3(\Ba \cdot \Bx)^2 -\Ba^2\Bx^2)({\Ba \cdot \Bx})
- 2 \Bx^2 \Ba^2(\Bx \cdot \Ba) \right) \\
&=
(1/3) ({\Ba \cdot \Bx}) \left(
\frac{5}{2}(3(\Ba \cdot \Bx)^2 -\Ba^2\Bx^2)
- 2 \Bx^2 \Ba^2 \right) \\
&=
({\Ba \cdot \Bx}) \left( \frac{5}{2}((\Ba \cdot \Bx)^2 ) - 3/2 \Bx^2 \Ba^2 \right) \\
&=
\half({\Ba \cdot \Bx}) ( {5}(\Ba \cdot \Bx)^2 - 3 \Bx^2 \Ba^2 ) \\
\end{aligned}
\end{equation}

\section{ Putting things in standard Legendre polynomial form}

This is still pretty laborious to calculate, especially because of not having a closed form recurrence
relation for \(\agrad P_k\).  Let us relate these to the standard Legendre polynomial form.

Observe that we can write

\begin{equation}\label{eqn:legendre:380}
\begin{aligned}
P_0(\bxa) &= 1 \\
\frac{P_1(\bxa)}{\abs{\Bx} \abs{\Ba}} &= \costheta \\
\frac{P_2(\bxa)}{\abs{\Bx}^2 \abs{\Ba}^2} &= \half(3(\costheta)^2 - 1) \\
\frac{P_3(\bxa)}{\abs{\Bx}^3 \abs{\Ba}^3} &= \half ( {5}(\costheta)^3 - 3 {(\costheta)} ) \\
\end{aligned}
\end{equation}

With this scaling, we have the standard form for the Legendre polynomials, and can write

\begin{equation}\label{eqn:legendre:140}
\inv{\Bx-\Ba} = \inv{\abs{\Bx}}\left(
P_0
+ \frac{\abs{\Ba}}{\abs{\Bx}} P_1(\costheta)
+ \left(\frac{\abs{\Ba}}{\abs{\Bx}}\right)^2 P_2(\costheta)
+ \left(\frac{\abs{\Ba}}{\abs{\Bx}}\right)^3 P_3(\costheta)
+ \cdots \right)
\end{equation}

\section{ Scaling standard form Legendre polynomials}

Since the odd Legendre polynomials have only odd terms and even have only even terms this allows for
the scaled form that NFCM uses.

\begin{equation}\label{eqn:legendre:400}
\begin{aligned}
P_0(\bxa) &= P_0(\costheta) \\
P_1(\bxa) &= \abs{\Bx}\abs{\Ba} P_1(\costheta) = \Ba \cdot \Bx \\
P_2(\bxa) &= \abs{\Bx}^2\abs{\Ba}^2 P_2(\costheta) = \half(3(\Ba \cdot \Bx)^2 - \Bx^2\Ba^2) \\
P_3(\bxa) &= \abs{\Bx}^3\abs{\Ba}^3 P_3(\costheta) = \half(5(\Ba \cdot \Bx)^3 - 3(\Ba \cdot \Bx) \Bx^2\Ba^2) \\
\end{aligned}
\end{equation}

Every term for the \(k^{th}\) polynomial is a permutation of the geometric product \(\Bx^k\Ba^k\).

This allows for writing some of these terms using the wedge product.  Using the product expansion:

\begin{equation}\label{eqn:legendre:160}
%\Ba \Bx \Bx \Ba = \Ba^2 \Bx^2 = (\Ba \cdot \Bx + \Ba \wedge \Bx)(\Bx \cdot \Ba + \Bx \wedge \Ba) = (\Ba \cdot \Bx)^2 - ( \Ba \wedge \Bx )^2
%\Ba^2 \Bx^2 = (\Ba \cdot \Bx)^2 - ( \Ba \wedge \Bx )^2
(\Ba \cdot \Bx)^2 = ( \Ba \wedge \Bx )^2 + \Ba^2 \Bx^2
\end{equation}

Thus we have:
\begin{equation}\label{eqn:legendre:420}
\begin{aligned}
P_2(\bxa)
&= (\Ba \cdot \Bx)^2 + \half(\Ba \wedge \Bx)^2 \\
&= (\Ba \cdot \Bx)^2 - \half\abs{\Ba \wedge \Bx}^2 \\
\end{aligned}
\end{equation}

This is nice geometrically since the directional dependence of this term on the co-linearity and
perpendicularity of the vectors \(\Ba\) and \(\Bx\) is clear.

Doing the same for the \(P_3\):

\begin{equation}\label{eqn:legendre:440}
\begin{aligned}
P_3(\bxa) &= (\Ba \cdot \Bx)\half(5(\Ba \cdot \Bx)^2 - 3\Bx^2\Ba^2) \\
          &= (\Ba \cdot \Bx)\half(2(\Ba \cdot \Bx)^2 + 3(\Ba \wedge \Bx)^2) \\
          &= (\Ba \cdot \Bx)((\Ba \cdot \Bx)^2 - \frac{3}{2}\abs{\Ba \wedge \Bx}^2) \\
\end{aligned}
\end{equation}

I suppose that one could get the same geometrical interpretation with a standard Legendre expansion in terms of \(\costheta = cos(\theta)\) terms, by collect both \(sin(\theta)\) and \(cos(\theta)\) powers, but one
can see the power of writing things explicitly in terms of the original vectors.

\section{ Note on NFCM Legendre polynomial notation}

In NFCM's slightly abusive notation \(P_k\) was used with various meanings.  He wrote \(P_k(\costheta) = \frac{P_k(\bxa)}{\abs{\Bx}^k \abs{\Ba}^k}\).

Note for example that the standard first degree Legendre polynomial \(P_1(x) = x\) evaluated with a \(\bxa\) value:

\begin{equation}\label{eqn:legendre:460}
\begin{aligned}
\inv {\abs{\Bx}\abs{\Ba}} {P_1(x) \vert_{x=\bxa}} &= \xcap \acap \\
&= \xcap \cdot \acap + \xcap \wedge \acap \\
\end{aligned}
\end{equation}

This has a bivector component in addition to the component identical to the standard Legendre polynomial
term (the first part).

By luck it happens that the scalar part of this equals \(P_1(\costheta)\), but this
is not the case for other terms.  Example, \(P_2(\bxa)\):

\begin{equation}\label{eqn:legendre:480}
\begin{aligned}
{P_2(x) \vert_{x=\bxa}}
&= \half( 3(\Bx \Ba)^2 - 1 ) \\
&= \half( 3(-\Ba \Bx + 2 \Ba \cdot \Bx )(\Bx \Ba) - 1 ) \\
&= \half( 3(-\Ba^2 \Bx^2 + 2(\Ba \cdot \Bx)^2 + 2(\Ba \cdot \Bx)(\Bx \wedge \Ba)) - 1 ) \\
&=  -(3/2)\Ba^2 \Bx^2 + 3(\Ba \cdot \Bx)^2 + 3(\Ba \cdot \Bx)(\Bx \wedge \Ba) - 1/2  \\
\end{aligned}
\end{equation}

Scaling this by \(1/(\Ba^2\Bx^2)\) is
\begin{equation}\label{eqn:legendre:180}
-\frac{3}{2} + 3(\costheta)^2 + 3(\costheta)(\xcap \wedge \acap) - \inv{\Ba^2\Bx^2} \\
\end{equation}

The scalar part of this is not anything recognizable.

\part{Projection.}
\include{reciprocal_frame}
\include{projection_with_matrix_comparison}
\include{oblique_proj}
\include{projection_and_moore_penrose_vector_inverse}
\include{angle_between_line_and_plane}
% 
% 
% 
% Copyright � 2012 Peeter Joot
% All Rights Reserved
% 
% This file may be reproduced and distributed in whole or in part, without fee, subject to the following conditions:
% 
% o The copyright notice above and this permission notice must be preserved complete on all complete or partial copies.
% 
% o Any translation or derived work must be approved by the author in writing before distribution.
% 
% o If you distribute this work in part, instructions for obtaining the complete version of this file must be included, and a means for obtaining a complete version provided.
% 
% 
% Exceptions to these rules may be granted for academic purposes: Write to the author and ask.
% 
% 
% 
\chapter{Orthogonal decomposition take II.} 
\label{chap:orthodecomp}
\date{April 1, 2008.  orthodecomp.tex}
\section{Lemma.  Orthogonal decomposition.}
To do so we first need to be able to express a vector $\Bx$ in terms
of components parallel and perpendicular to the blade $\BA \in \wedge^k$.

\begin{align*}
\Bx 
&= \Bx \BA \inv{\BA} \\
&= (\Bx \cdot \BA + \Bx \wedge \BA) \inv{\BA} \\
&= 
(\Bx \cdot \BA) \cdot \inv{\BA}
+ \sum_{i=3,5,\cdots,2k-1} \gpgrade{(\Bx \cdot \BA) \inv{\BA}}{i} \\
&+ 
(\Bx \wedge \BA) \cdot \inv{\BA}
+ \sum_{i=3,5,\cdots,2k-1} \gpgrade{(\Bx \wedge \BA) \inv{\BA}}{i} 
+ \underbrace{(\Bx \wedge \BA) \wedge \inv{\BA}}_{=0}
\end{align*}

Since the LHS and RHS must both be vectors all the non-grade one terms
are either zero or cancel out.  This can be observed directly since:

\begin{align*}
\gpgrade{\Bx \cdot \BA \inv{\BA}}{i}
&= \gpgrade{ \frac{\Bx \BA - (-1)^{k}\BA\Bx}{2}\inv{\BA} }{i}  \\
&= -\frac{(-1)^{k}}{2} \gpgrade{ \BA\Bx \inv{\BA} }{i}  \\
\end{align*}

and

\begin{align*}
\gpgrade{\Bx \wedge \BA \inv{\BA}}{i} 
&= \gpgrade{ \frac{\Bx \BA + (-1)^{k}\BA\Bx}{2}\inv{\BA} }{i}  \\
&= +\frac{(-1)^{k}}{2} \gpgrade{ \BA\Bx \inv{\BA} }{i}  \\
\end{align*}

Thus all of the grade $3, \cdots ,2k-1$ terms cancel each other out.  Some terms
like $(\Bx \cdot \BA) \wedge \inv{\BA}$ are also independently zero.

(This is a result I've got in other places, but I thought it's worth
 writing down since I thought the direct cancellation is elegant).

\part{Rotation.}
%
% Copyright � 2012 Peeter Joot.  All Rights Reserved.
% Licenced as described in the file LICENSE under the root directory of this GIT repository.
%

%
%
\chapter{Rotor Notes}\label{chap:rotor}
\index{rotor}
%\date{Feb 19, 2008.  rotor.tex}

\section{Rotations strictly in a plane}

For a plane rotation, a rotation does not have to
be expressed in terms of left and right half angle rotations, as is the case
with complex numbers.  Starting with this ``natural'' one sided rotation
we will see why the half angle double sided Rotor formula works.

\subsection{Identifying a plane with a bivector.  Justification}
Given a bivector \(\BB\), we can say this defines the orientation of a plane
(through the origin)
since for any vector in the plane we have \(\BB \wedge \Bx = 0\), or any vector
strictly normal to the plane \(\BB \cdot \Bx = 0\).

Note that this naturally compares
to the equation of a line (through the origin) expressed in terms of a
direction vector \(\Bb\),
where \(\Bb \wedge \Bx=0\) if \(\Bx\) lies on the line, and \(\Bb \cdot \Bx = 0\)
if \(\Bx\) is normal to the line.

Given this it is not unreasonable to identify the plane with its bivector.  This
will be done below, and it should be clear that
loose language such as ``the plane \(\BB\)'', should really be interpreted
as ``the plane with direction bivector \(\BB\)'', where the direction bivector
has the wedge and dot product properties noted above.

\subsection{Components of a vector in and out of a plane}

To calculate the components of a vector in and out of a plane, we can form
the product

\begin{equation}\label{eqn:rotor:20}
\Bx = \Bx \BB \inv{\BB} = \Bx \cdot \BB \inv{\BB} + \Bx \wedge \BB \inv{\BB}
\end{equation}

This is an orthogonal decomposition of the vector \(\Bx\) where the first
part is the projective term onto the plane \(\BB\), and the second is the rejective
term, the component not in the plane.  Let us verify this.

Write \(\Bx = \Bx_\parallel + \Bx_\perp\), where \(\Bx_\parallel\), and \(\Bx_\perp\) are the components of \(\Bx\) parallel and perpendicular to the plane.  Also write
\(\BB = \Bb_1 \wedge \Bb_2\), where \(\Bb_i\) are non-colinear vectors in the plane \(\BB\).

If \(\Bx = \Bx_\parallel\), a vector entirely in the plane \(\BB\), then one can
write

\begin{equation}\label{eqn:rotor:40}
\Bx = a_1\Bb_1 + a_2\Bb_2
\end{equation}

and the wedge product term is zero

\begin{equation}\label{eqn:rotor:740}
\begin{aligned}
\Bx \wedge \BB
&= \left( a_1\Bb_1 + a_2\Bb_2 \right) \wedge \Bb_1 \wedge \Bb_2 \\
&= a_1 ( \Bb_1 \wedge \Bb_1 ) \wedge \Bb_2
 - a_2 ( \Bb_2 \wedge \Bb_2 ) \wedge \Bb_1 \\
&= 0
\end{aligned}
\end{equation}

Thus the component parallel to the plane is composed strictly of the dot
product term

\begin{equation}
\Bx_\parallel = \Bx \cdot \BB \inv{\BB}
\end{equation}

Or for a general vector not necessarily in the plane the component
of that vector in the plane, its projection onto the plane is,

\begin{equation}\label{eqn:rotor:60}
\Proj_{\BB}(\Bx) = \Bx \cdot \BB \inv{\BB}
= \inv{\abs{\BB}^2}(\BB \cdot \Bx)\BB
= (\hat{\BB} \cdot \Bx)\hat{\BB}
\end{equation}

Now, for a vector that lies completely perpendicular to the plane \(\Bx = \Bx_\perp\), the dot product term with the plane is zero.  To verify this observe

\begin{equation}\label{eqn:rotor:760}
\begin{aligned}
\Bx_\perp \cdot \BB
&= \Bx_\perp \cdot (\Bb_1 \wedge \Bb_2) \\
&= (\Bx_\perp \cdot \Bb_1) \Bb_2 - (\Bx_\perp \cdot \Bb_2) \Bb_1 \\
\end{aligned}
\end{equation}

Each of these dot products are zero since \(\Bx\) has no components that lie
in the plane (those components if they existed could be expressed as linear
combinations of \(\Bb_i\)).

Thus only the component perpendicular to the plane is composed strictly of the
wedge product term

\begin{equation}
\Bx_\perp = \Bx \wedge \BB \inv{\BB}
\end{equation}

And again for a general vector the component that lies out
of the plane as, the rejection of the plane from the vector is

\begin{equation}\label{eqn:rotor:80}
\RejName_{\BB}(\Bx)
= \Bx \wedge \BB \inv{\BB}
= -\inv{\abs{\BB}^2} \Bx \wedge \BB {\BB}
= -\Bx \wedge \hat{\BB} \hat{\BB}
\end{equation}

\section{Rotation around normal to arbitrarily oriented plane through origin}

Having established the preliminaries, we can now express a rotation around
the normal to a plane (with the plane and that normal through the origin).

\imageFigure{../../figures/gabook/rotor}{Rotation of Vector}{fig:rotor}{0.4}

Such a rotation is illustrated in \cref{fig:rotor}
preserves all components of the vector that are perpendicular
to the plane, and operates only on the components parallel to the plane.

Expressed in terms of exponentials and the projective and rejective decompositions above, this is

\begin{equation}\label{eqn:rotor:780}
\begin{aligned}
R_\theta(\Bx)
&= \Bx \wedge \BB \inv{\BB} + \left(\Bx \cdot \BB \inv{\BB}\right)e^{\hat{\BB}\theta} \\
&= \Bx \wedge \BB \inv{\BB} + e^{-\hat{\BB}\theta}\left(\Bx \cdot \BB \inv{\BB}\right) \\
\end{aligned}
\end{equation}

Where we have made explicit note that a plane rotation does not commute with a vector in a plane (its reverse is required).

To demonstrate this write \(i = \Be_2 \Be_1\), a unit bivector in some plane with unit vectors \(\Be_i\) also in the plane.  If a vector
lies in that plane we can write the rotation

\begin{equation}\label{eqn:rotor:800}
\begin{aligned}
\Bx e^{i\theta}
&= \left(a_1\Be_1 + a_2\Be_2\right)\left(\cos\theta + i\sin\theta\right) \\
&= \cos\theta\left(a_1\Be_1 + a_2\Be_2\right) + \left(a_1\Be_1 + a_2\Be_2\right)\left(\Be_2 \Be_1\sin\theta\right) \\
&= \cos\theta\left(a_1\Be_1 + a_2\Be_2\right) + \sin\theta \left(-a_1\Be_2 + a_2\Be_1\right) \\
&= \cos\theta\left(a_1\Be_1 + a_2\Be_2\right) -\Be_2 \Be_1\sin\theta \left(a_1\Be_1 + a_2\Be_2\right) \\
&= e^{-i\theta}\Bx \\
\end{aligned}
\end{equation}

Similarly for a vector that lies outside of the plane we can write

\begin{equation}\label{eqn:rotor:820}
\begin{aligned}
\Bx e^{i\theta}
&= (\sum_{j \ne 1,2} a_j \Be_j)(\cos\theta + \Be_2 \Be_1\sin\theta) \\
&= (\cos\theta + \Be_2 \Be_1\sin\theta) (\sum_{j \ne 1,2} a_j \Be_j) \\
&= e^{i\theta}\Bx
\end{aligned}
\end{equation}

The multivector for a rotation in a plane perpendicular to a vector commutes with that vector.  The properties of the
exponential allow us to factor a rotation

\begin{equation}\label{eqn:rotor:100}
R(\theta) = R(\alpha\theta) R((1-\alpha)\theta)
\end{equation}

where \(\alpha <= 1\), and in particular we can set \(\alpha = 1/2\), and write

\begin{equation}\label{eqn:rotor:840}
\begin{aligned}
R_\theta(\Bx)
&= \Bx \wedge \BB \inv{\BB} + \left(\Bx \cdot \BB \inv{\BB}\right)e^{\hat{\BB}\theta} \\
&= \left(\Bx \wedge \BB \inv{\BB}\right) e^{-\hat{\BB}\theta/2} e^{\hat{\BB}\theta/2}
 + \left(\Bx \cdot \BB \inv{\BB} \right) e^{\hat{\BB}\theta/2} e^{\hat{\BB}\theta/2} \\
&= e^{-\hat{\BB}\theta/2} \left(\Bx \wedge \BB \inv{\BB}\right) e^{\hat{\BB}\theta/2}
+ e^{-\hat{\BB}\theta/2} \left(\Bx \cdot \BB \inv{\BB}\right)e^{\hat{\BB}\theta/2} \\
&= e^{-\hat{\BB}\theta/2} \left(\Bx \wedge \BB + \Bx \cdot \BB\right) \inv{\BB} e^{\hat{\BB}\theta/2} \\
&= e^{-\hat{\BB}\theta/2} \left(\Bx \BB \inv{\BB} \right) e^{\hat{\BB}\theta/2}
\end{aligned}
\end{equation}

This takes us full circle from dot and wedge products back to \(\Bx\), and allows us to express the rotated vector as:

\begin{equation}\label{eqn:rotor:rotor}
R_\theta(\Bx)
= e^{-\hat{\BB}\theta/2} \Bx e^{\hat{\BB}\theta/2}
\end{equation}

Only when the vector lies in the plane (\(\Bx = \Bx_\parallel\), or \(\Bx \wedge \BB = 0\)) can be written using the familiar left or right ``full angle'' rotation exponential that we are used to from complex arithmetic:

\begin{equation}\label{eqn:rotor:120}
R_\theta(\Bx) = e^{-\hat{\BB}\theta} \Bx = \Bx e^{\hat{\BB}\theta}
\end{equation}

\section{Rotor equation in terms of normal to plane}

The rotor equation above is valid for any number of dimensions.  For \R{3} we can alternatively parametrize the plane in terms of
a unit normal \(\Bn\):

\begin{equation}\label{eqn:rotor:140}
\BB = k i\Bn
\end{equation}

Here \(i\) is the \R{3} pseudoscalar \(\Be_1 \Be_2 \Be_3\).

Thus we can write

\begin{equation}\label{eqn:rotor:160}
\hat{\BB} = i\Bn
\end{equation}

and expressing \eqnref{eqn:rotor:rotor} in terms of the unit normal becomes trivial

\begin{equation}
R_\theta(\Bx)
= e^{- i {\Bn}\theta/2} \Bx e^{i{\Bn}\theta/2}
\end{equation}

Expressing this in terms of components and the unit normal is a bit harder

\begin{equation}\label{eqn:rotor:860}
\begin{aligned}
R_\theta(\Bx)
&= \Bx \wedge \BB \inv{\BB} + \left(\Bx \cdot \BB \inv{\BB}\right)e^{\hat{\BB}\theta} \\
&= \Bx \wedge (i\Bn) \inv{i\Bn} + \left(\Bx \cdot (i\Bn) \inv{i\Bn}\right)e^{{i\Bn}\theta} \\
\end{aligned}
\end{equation}

Now,

\begin{equation}\label{eqn:rotor:880}
\begin{aligned}
\Bx \wedge (i\Bn)
&= \inv{2}(\Bx i \Bn + i \Bn \Bx) \\
&= \frac{i}{2}(\Bx \Bn + \Bn \Bx) \\
&= (\Bx \cdot \Bn) i
\end{aligned}
\end{equation}

And

\begin{equation}\label{eqn:rotor:900}
\begin{aligned}
\inv{i\Bn}
&= \inv{i\Bn} \inv{\Bn i} \Bn i \\
&= - i \Bn \\
\end{aligned}
\end{equation}

So the rejective term becomes
\begin{equation}\label{eqn:rotor:920}
\begin{aligned}
\Bx \wedge \BB \inv{\BB}
&= \Bx \wedge (i\Bn) \inv{i\Bn} \\
&= \Bx \wedge (i\Bn) \inv{i\Bn} \\
&= (\Bx \cdot \Bn) i (-i) \Bn \\
&= (\Bx \cdot \Bn) \Bn \\
&= \Proj_{\Bn}(\Bx) \\
\end{aligned}
\end{equation}

Now, for the dot product with the plane term, we have

\begin{equation}\label{eqn:rotor:940}
\begin{aligned}
\Bx \cdot \BB
&= \Bx \cdot (i \Bn) \\
&= \inv{2}(\Bx i \Bn - i \Bn \Bx) \\
&= (\Bx \wedge \Bn)i \\
\end{aligned}
\end{equation}

Putting it all together we have

\begin{equation}\label{eqn:rotor:rotexp}
R_\theta(\Bx)
= (\Bx \cdot \Bn) \Bn + (\Bx \wedge \Bn)\Bn e^{{i\Bn}\theta}
\end{equation}

In terms of explicit sine and cosine terms this is (observe that \((i\Bn)^2 = -1\)),

\begin{equation}\label{eqn:rotor:960}
\begin{aligned}
R_\theta(\Bx)
&= \left(\Bx \cdot \Bn\right) \Bn + \left(\Bx \wedge \Bn\right)\Bn \left(\cos\theta + i\Bn \sin\theta\right) \\
\end{aligned}
\end{equation}

\begin{equation}\label{eqn:rotor:rotnorm}
R_\theta(\Bx) =
\left(\Bx \cdot \Bn\right) \Bn + \left(\Bx \wedge \Bn\right)\Bn \cos\theta + (\Bx \wedge \Bn) i \sin\theta
\end{equation}

\imageFigure{../../figures/gabook/normalRot}{Direction vectors associated with rotation}{fig:normalRot}{0.4}

This triplet of mutually orthogonal direction vectors,
\(\Bn\), \((\Bx \wedge \Bn)\Bn\), and \((\Bx \wedge \Bn) i\)
are illustrated in \cref{fig:normalRot}.  The component of the vector in the direction of the normal
\(\Proj_\Bn(\Bx) = \Bx \cdot \Bn \Bn\) is unaltered by the rotation.
The rotation is applied to the remaining component of \(\Bx\), \(\RejName_{\Bn}(\Bx) = (\Bx \wedge \Bn)\Bn\), and we rotate
in the direction \((\Bx \wedge \Bn) i\)

\subsection{Vector rotation in terms of dot and cross products only}

Expression of this rotation formula \eqnref{eqn:rotor:rotnorm} in terms of ``vector'' relations is also possible, by removing the wedge
products and the pseudoscalar references.

First the rejective term

\begin{equation}\label{eqn:rotor:980}
\begin{aligned}
(\Bx \wedge \Bn) \Bn
&= ((\Bx \cross \Bn) i) \Bn \\
&= ((\Bx \cross \Bn) i) \cdot \Bn \\
&= \inv{2} ( ((\Bx \cross \Bn) i) \Bn - \Bn ((\Bx \cross \Bn) i)) \\
&= \frac{i}{2} ( (\Bx \cross \Bn) \Bn - \Bn (\Bx \cross \Bn) ) \\
&= i ( (\Bx \cross \Bn) \wedge \Bn ) \\
&= i^2 ( (\Bx \cross \Bn) \cross \Bn ) \\
&= \Bn \cross (\Bx \cross \Bn) \\
\end{aligned}
\end{equation}

The next term expressed in terms of the cross product is

\begin{equation}\label{eqn:rotor:1000}
\begin{aligned}
(\Bx \wedge \Bn) i
&=
(\Bx \cross \Bn) i^2 \\
&= \Bn \cross \Bx \\
\end{aligned}
\end{equation}

And putting it all together we have

\begin{equation}\label{eqn:rotor:rotcross}
R_\theta(\Bx) =
\left(\Bx \cdot \Bn\right) \Bn
 + \left(\Bn \cross \Bx\right) \cross \Bn \cos\theta
 + \Bn \cross \Bx \sin\theta
\end{equation}

Compare \eqnref{eqn:rotor:rotcross} to \eqnref{eqn:rotor:rotnorm} and \eqnref{eqn:rotor:rotexp}, and then back to \eqnref{eqn:rotor:rotor}.

\section{Giving a meaning to the sign of the bivector}

For a rotation between two vectors in the plane containing those vectors, we can write the rotation
in terms of the exponential as either a left or right rotation operator:

\begin{equation}\label{eqn:rotor:180}
\Bb = \Ba e^{\Bi\theta} = e^{-\Bi\theta}\Ba
\end{equation}
\begin{equation}\label{eqn:rotor:200}
\Bb = e^{\Bj\theta}\Ba = \Ba e^{-\Bj\theta/2}
\end{equation}

Here both \(\Bi\) and \(\Bj=-\Bi\) are unit bivectors with the property \(\Bi^2 = \Bj^2 = -1\).
Thus in order to write a rotation in exponential form a meaning must be assigned to the sign of the unit bivector that describes the
plane and the orientation of the rotation.

Consider for example the case of a rotation by \(\pi/2\).  For this is the exponential is:

\begin{equation}\label{eqn:rotor:220}
e^{\Bi\pi/2} = \cos(\pi/2) + \Bi \sin(\pi/2) = \Bi
\end{equation}

Thus for perpendicular unit vectors \(\Bu\) and \(\Bv\), if we wish \(\Bi\) to act as a \(\pi/2\) rotation left acting operator on \(\Bu\)
towards \(\Bv\) its value must be:

\begin{equation}\label{eqn:rotor:240}
\Bi = \Bu \wedge \Bv
\end{equation}
\begin{equation}\label{eqn:rotor:260}
\Bu\Bi = \Bu \Bu \wedge \Bv = \Bu\Bu\Bv = \Bv
\end{equation}

For that same rotation if the bivector is employed as a right acting operator, the reverse is required:

\begin{equation}\label{eqn:rotor:280}
\Bj = \Bv \wedge \Bu
\end{equation}
\begin{equation}\label{eqn:rotor:300}
\Bj\Bu = \Bv \wedge \Bu = \Bv\Bu\Bu = \Bv
\end{equation}

\imageFigure{../../figures/gabook/imaginaryorientation}{Orientation of unit imaginary}{fig:imaginaryorientation}{0.4}

In general, for any two vectors, one can find an angle \(\theta\) in the range \(0 \le \theta \le \pi\) between those vectors.
If one lets that angle define the orientation of the rotation between the vectors, and implicitly
define a sort of ``imaginary axis'' for that plane, that imaginary axis will have direction

\begin{equation}\label{eqn:rotor:320}
\inv{\Ba} \Ba \wedge \Bb = \Bb \wedge \Ba \inv {\Ba}.
\end{equation}

This is illustrated in \cref{fig:imaginaryorientation}.

Thus the bivector

\begin{equation}\label{eqn:rotor:340}
\Bi = \frac{\Ba \wedge \Bb}{\abs{\Ba \wedge \Bb}}
\end{equation}

When acting as an operator to the left (\(\Ba \Bi\)) with a vector in the plane can be interpreted as acting as a rotation by \(\pi/2\) towards \(\Bb\).

Similarly the bivector

\begin{equation}\label{eqn:rotor:360}
\Bj = \Bi^\dagger = -\Bi = \frac{\Bb \wedge \Ba}{\abs{\Bb \wedge \Ba}}
\end{equation}

also applied to a vector in the plane produces the same rotation when
acting as an operator to the right.  Thus, in general we can write
a rotation by theta in the plane containing non-colinear vectors \(\Ba\) and \(\Bb\) in the direction of minimal angle
from \(\Ba\) towards \(\Bb\) in one of the three forms:

\begin{equation}\label{eqn:rotor:380}
R_{\theta : \Ba \rightarrow \Bb}(\Ba)
= \Ba e^{ \frac{\Ba \wedge \Bb}{\abs{\Ba \wedge \Bb}} \theta }
= e^{ \frac{\Bb \wedge \Ba}{\abs{\Bb \wedge \Ba}} \theta } \Ba
\end{equation}

Or,
\begin{equation}\label{eqn:rotor:400}
R_{\theta : \Ba \rightarrow \Bb}(\Bx)
= e^{ \frac{\Bb \wedge \Ba}{\abs{\Bb \wedge \Ba}} \theta/2 } \Bx e^{ \frac{\Ba \wedge \Bb}{\abs{\Ba \wedge \Bb}} \theta/2 }
\end{equation}

This last (writing \(\Bx\) instead of \(\Ba\) since it also applies to vectors that lie outside of the \(\Ba \wedge \Bb\) plane),
is our rotor formula \eqnref{eqn:rotor:rotor}, reexpressed in a way that removes the sign ambiguity of the bivector \(\Bi\) in that equation.

\section{Rotation between two unit vectors}

\imageFigure{../../figures/gabook/parallelogramvec}{Sum of unit vectors bisects angle between}{fig:parallelogramvec}{0.4}

As illustrated in \cref{fig:parallelogramvec}, when the angle between two vectors is less than \(\pi\)
the fact that the sum of two arbitrarily oriented unit vectors bisects those vectors provides a convenient
way to compute the half angle rotation exponential.

Thus we can write

\begin{equation*}
\frac{\Ba + \Bb}{\abs{\Ba + \Bb}} = \Ba e^{\Bi\theta/2} = e^{\Bj\theta/2} \Ba
\end{equation*}

Where \(\Bi = \Bj^\dagger\) are unit bivectors of appropriate sign.  Multiplication through by \(\Ba\) gives

\begin{equation*}
e^{\Bi\theta/2} =
\frac{1 + \Ba\Bb}{\abs{\Ba + \Bb}}
\end{equation*}

Or,
\begin{equation*}
e^{\Bj\theta/2} =
\frac{1 + \Bb\Ba}{\abs{\Ba + \Bb}}
\end{equation*}

Thus we can write the total rotation from \(\Ba\) to \(\Bb\) as

\begin{equation*}
\Bb
= e^{-\Bi\theta/2} \Ba e^{\Bi\theta/2}
= e^{\Bj\theta/2} \Ba e^{-\Bj\theta/2}
= \left(\frac{1 + \Bb\Ba}{\abs{\Ba + \Bb}}\right) \Ba \left(\frac{1 + \Ba\Bb}{\abs{\Ba + \Bb}}\right)
\end{equation*}

For the case where the rotation is through an angle \(\theta\) where \(\pi < \theta < 2\pi\), again employing a left acting
exponential operator we have

\begin{equation}\label{eqn:rotor:1020}
\begin{aligned}
\frac{\Ba + \Bb}{\abs{\Ba + \Bb}}
&= \Bb e^{\Bi(2\pi - \theta)/2} \\
&= \Bb e^{\Bi \pi} e^{- \Bi\theta/2} \\
&= -\Bb e^{- \Bi\theta/2} \\
\end{aligned}
\end{equation}

Or,
\begin{equation}\label{eqn:rotor:420}
e^{- \Bi\theta/2} = -\frac{\Bb\Ba + 1}{\abs{\Ba + \Bb}}
\end{equation}

Thus

\begin{equation}\label{eqn:rotor:rotunit}
\Bb = e^{- \Bi\theta/2} \Ba e^{ \Bi\theta/2} =
\left(-\frac{1 + \Bb\Ba}{\abs{\Ba + \Bb}}\right) \Ba \left(-\frac{1 + \Ba\Bb}{\abs{\Ba + \Bb}}\right)
\end{equation}

Note that the two negatives cancel, giving the same result as in the \(\theta < \pi\) case.  Thus \eqnref{eqn:rotor:rotunit} is valid for all vectors \(\Ba \ne -\Bb\) (this can be verified by direct multiplication.)

These
half angle exponentials are called rotors, writing the rotor as

\begin{equation}\label{eqn:rotor:440}
R = \frac{1 + \Ba\Bb}{\abs{\Ba + \Bb}}
\end{equation}

and the rotation in terms of rotors is:

\begin{equation}\label{eqn:rotor:460}
\Bb = R^\dagger \Ba R
\end{equation}

The angle associated with this rotor \(R\) is the minimal angle between the two vectors (\(0 < \theta < \pi\)), and is directed from \(\Ba\) to \(\Bb\).  Inverting the rotor will not change the net effect of the rotation, but has the geometric meaning that the rotation from \(\Ba\) to \(\Bb\)
rotates in the opposite direction through the larger angle (\(\pi < \theta < 2\pi\)) between the vectors.

\section{Eigenvalues, vectors and coordinate vector and matrix of the rotation linear transformation}

Given the plane containing two orthogonal vectors \(\Bu\) and \(\Bv\), we can form a unit bivector for the plane

\begin{equation}\label{eqn:rotor:480}
\BB = \Bu\Bv
\end{equation}

A normal to this plane is \(\Bn = \Bv\Bu I\).

The rotation operator for a rotation around \(\Bn\) in that plane (directed from \(\Bu\) towards \(\Bv\)) is

\begin{equation}\label{eqn:rotor:500}
R_\theta(\Bx) = e^{\Bv\Bu \theta/2} \Bx e^{\Bu\Bv \theta/2}
\end{equation}

To form the matrix of this linear transformation assume an orthonormal basis \(\sigma = \{ \Be_i \}\).

In terms of these basis vectors we can write

\begin{equation}\label{eqn:rotor:520}
R_\theta(\Be_j) =
e^{-\Bv\Bu \theta/2} \Be_j e^{\Bu\Bv \theta/2}
=
\sum_i \left(e^{-\Bv\Bu \theta/2} \Be_j e^{\Bu\Bv \theta/2}\right) \cdot \Be_i \Be_i
\end{equation}

Thus the coordinate vector for this basis is

\begin{equation}\label{eqn:rotor:540}
{
\begin{bmatrix}
R_\theta(\Be_j)
\end{bmatrix}
}_\sigma
=
\begin{bmatrix}
\left(e^{-\Bv\Bu \theta/2} \Be_j e^{\Bu\Bv \theta/2}\right) \cdot \Be_1 \\
\vdots \\
\left(e^{-\Bv\Bu \theta/2} \Be_j e^{\Bu\Bv \theta/2}\right) \cdot \Be_n \\
\end{bmatrix}
\end{equation}

We can use this to form the matrix for the linear operator that takes coordinate vectors from
the basis \(\sigma\) to \(\sigma\):

\begin{equation}\label{eqn:rotor:560}
{
\begin{bmatrix}
R_\theta(\Bx)
\end{bmatrix}
}_\sigma
=
{
\begin{bmatrix}
R_\theta
\end{bmatrix}
}_\sigma^\sigma
{
\begin{bmatrix}
\Bx
\end{bmatrix}
}_\sigma
\end{equation}

Where
\begin{equation}\label{eqn:rotor:rotcoords}
{
\begin{bmatrix}
R_\theta
\end{bmatrix}
}_\sigma^\sigma
=
\begin{bmatrix}
{
\begin{bmatrix}
R_\theta(\Be_1)
\end{bmatrix}
}_\sigma
\hdots
{
\begin{bmatrix}
R_\theta(\Be_n)
\end{bmatrix}
}_\sigma
\end{bmatrix}
=
{
\begin{bmatrix}
\left(e^{-\Bv\Bu \theta/2} \Be_j e^{\Bu\Bv \theta/2}\right) \cdot \Be_i \\
\end{bmatrix}
}_{ij}
\end{equation}

If one uses the plane and its normal to form an alternate orthonormal basis
\(\alpha = \{\Bu, \Bv, \Bn\}\).

The transformation matrix for coordinate vectors in this basis is

\begin{equation}\label{eqn:rotor:580}
{
\begin{bmatrix}
R_\theta
\end{bmatrix}
}_\alpha^\alpha
=
\begin{bmatrix}
\left(\Bu e^{\Bu\Bv \theta}\right) \cdot \Bu & \left(\Bv e^{\Bu\Bv \theta}\right) \cdot \Bu & 0 \\
\left(\Bu e^{\Bu\Bv \theta}\right) \cdot \Bv & \left(\Bv e^{\Bu\Bv \theta}\right) \cdot \Bv & 0 \\
0 & 0 & \Bn\cdot\Bn \\
\end{bmatrix}
=
\begin{bmatrix}
\cos\theta & -\sin\theta & 0 \\
\sin\theta & \cos\theta & 0 \\
0 & 0 & 1 \\
\end{bmatrix}
\end{equation}

This matrix has eigenvalues \(e^{i\theta}, e^{-i\theta}, 1\), with (coordinate) eigenvectors

\begin{equation}\label{eqn:rotor:600}
\inv{\sqrt{2}}
\begin{bmatrix}
1 \\
-i \\
0 \\
\end{bmatrix},
\inv{\sqrt{2}}
\begin{bmatrix}
1 \\
i \\
0 \\
\end{bmatrix},
\begin{bmatrix}
0 \\
0 \\
1 \\
\end{bmatrix}
\end{equation}

Its interesting to observe that without introducing coordinate vectors an eigensolution is possible directly from
the linear transformation itself.

The rotation linear operator has right and left eigenvalues \(e^{\Bu\Bv \theta}\), \(e^{\Bv\Bu \theta}\) (respectively), where the eigenvectors for these are any vectors in the plane.  There is also a scalar eigenvalue \(1\) (both left and right eigenvalue), for the eigenvector \(\Bn\):

\begin{equation}\label{eqn:rotor:1040}
\begin{aligned}
R_\theta(\Bu) &= e^{\Bv \Bu \theta} \Bx = \Bx e^{\Bu \Bv \theta} \\
R_\theta(\Bu) &= e^{\Bv \Bu \theta} \Bx = \Bx e^{\Bu \Bv \theta} \\
R_\theta(\Bn) &= \Bn (1) \\
\end{aligned}
\end{equation}

Observe that the eigenvalues here are not all scalars, which is likely related
to the fact that the coordinate matrix was not diagonalizable with real vectors.

the matrix of the linear transformation.
Given this, one can write:

\begin{equation}\label{eqn:rotor:1060}
\begin{aligned}
\begin{bmatrix}
R_\theta(\Bu) & R_\theta(\Bv) & R_\theta(\Bn) \\
\end{bmatrix}
&=
\begin{bmatrix}
\Bu & \Bv & \Bn \\
\end{bmatrix}
\begin{bmatrix}
e^{\Bu \Bv \theta} & 0 & 0 \\
0 & e^{\Bu \Bv \theta} & 0 \\
0 & 0 & 1 \\
\end{bmatrix} \\
&=
\begin{bmatrix}
e^{\Bv \Bu \theta} & 0 & 0 \\
0 & e^{\Bv \Bu \theta} & 0 \\
0 & 0 & 1 \\
\end{bmatrix}
\begin{bmatrix}
\Bu & \Bv & \Bn \\
\end{bmatrix}
\end{aligned}
\end{equation}

But neither of these can be used to diagonalize the matrix of the transformation.  To do that
we require dot products that span the matrix product to form the coordinate vector columns.

Observe that interestingly
enough the left and right eigenvalues of the operator in the plane are of complex exponential form (\(e^{\pm \Bn I \theta}\)) just as the eigenvalues for
coordinate vectors restricted to the plane are complex exponentials (\(e^{\pm i\theta}\)).
%This suggests that a basis for a quaternion
%like space (0-2 multivectors) will be required to diagonalize a rotation operator.

\section{matrix for rotation linear transformation}

Let us expand the terms in \eqnref{eqn:rotor:rotcoords} to calculate explicitly the rotation matrix for an arbitrary
rotation.  Also, as before, write \(\Bn = \Bv\Bu I\), and parametrize the Rotor as follows:

\begin{equation}\label{eqn:rotor:620}
R = e^{\Bn I \theta/2} = \cos{\theta/2} + \Bn I \sin{\theta/2} = \alpha + I\Bbeta
\end{equation}

Thus the \(ij\) terms in the matrix are:

\begin{equation}\label{eqn:rotor:1080}
\begin{aligned}
\Be_i \cdot \left(e^{-\Bn I \theta/2} \Be_j e^{\Bn I \theta/2}\right)
&= \langle{ \Be_i (\alpha -I\Bbeta) \Be_j (\alpha +I\Bbeta) } \rangle \\
&= \langle{ \Be_i (\Be_j \alpha -I\Bbeta\Be_j) (\alpha +I\Bbeta) } \rangle \\
&= \langle{ \Be_i \left( \Be_j \alpha^2 -I\alpha(\Bbeta\Be_j - \Be_j\Bbeta) + \Bbeta\Be_j\Bbeta \right) } \rangle \\
&= \delta_{ij}\alpha^2 + \langle{ \Be_i \left( -2I\alpha(\Bbeta \wedge \Be_j) + \Bbeta\Be_j\Bbeta \right) } \rangle \\
&= \delta_{ij}\alpha^2 + 2\alpha \Be_i \cdot (\Bbeta \cross \Be_j) + \langle{ \Be_i \Bbeta \Be_j \Bbeta } \rangle \\
\end{aligned}
\end{equation}

Lets expand the last term separately:
\begin{equation}\label{eqn:rotor:1100}
\begin{aligned}
\langle{ \Be_i \Bbeta \Be_j \Bbeta } \rangle
&= \langle{ ( \Be_i \cdot \Bbeta + \Be_i \wedge \Bbeta) ( \Be_j \cdot \Bbeta + \Be_j \wedge \Bbeta) } \rangle  \\
&= (\Be_i \cdot \Bbeta)(\Be_j \cdot \Bbeta) + \langle{ (\Be_i \wedge \Bbeta) ( \Be_j \wedge \Bbeta) } \rangle  \\
\end{aligned}
\end{equation}

And once more considering first the \(i=j\) case (writing \(s \ne i \ne t\)).

\begin{equation}\label{eqn:rotor:1120}
\begin{aligned}
\langle{ (\Be_i \wedge \Bbeta)^2 }\rangle
&= \lr{ \sum_{k \ne i}{ \Be_{ik} \beta_k} }^2 \\
&= ( \Be_{is} \beta_s + \Be_{it} \beta_t ) ( \Be_{is} \beta_s + \Be_{it} \beta_t ) \\
&= -\beta_s^2 -\beta_t^2 -  \Be_{st} \beta_s \beta_t + \Be_{ts} \beta_t \beta_s  \\
&= -\beta_s^2 -\beta_t^2 \\
&= -\Bbeta^2 + \beta_i^2 \\
\end{aligned}
\end{equation}

For the \(i \ne j\) term, writing \(i \ne j \ne k\)
\begin{equation}\label{eqn:rotor:1140}
\begin{aligned}
\langle{(\Be_i \wedge \Bbeta) (\Be_j \wedge \Bbeta)}\rangle
&= \langle{\sum_{s \ne i} \Be_{is} \beta_s\sum_{t \ne i} \Be_{it} \beta_t}\rangle \\
&= \langle{( \Be_{ij} \beta_j + \Be_{ik} \beta_k) ( \Be_{ji} \beta_i + \Be_{jk} \beta_k)}\rangle \\
&= \beta_i\beta_j + \langle{ \Be_{ji} \beta_k^2 +\Be_{ik} \beta_j \beta_k +\Be_{kj} \beta_k \beta_i }\rangle \\
&= \beta_i\beta_j \\
\end{aligned}
\end{equation}

Thus
\begin{equation}\label{eqn:rotor:640}
\langle{ (\Be_i \wedge \Bbeta) ( \Be_j \wedge \Bbeta) } \rangle
= \delta_{ij}(-\Bbeta^2 + \beta_i^2) + (1-\delta_{ij})\beta_i\beta_j
= \beta_i\beta_j -\delta_{ij}\Bbeta^2
\end{equation}

And putting it all back together
\begin{equation}\label{eqn:rotor:rotmgreek}
\Be_i \cdot \left(e^{-\Bn I \theta/2} \Be_j e^{\Bn I \theta/2}\right)
= \delta_{ij}(\alpha^2 -\Bbeta^2) + 2\alpha \Be_i \cdot (\Bbeta \cross \Be_j) + 2\beta_i\beta_j
\end{equation}


The \(\alpha\) and \(\beta\) terms can be expanded in terms of \(\theta\).
we see that The \(\delta_{ij}\) coefficient is

\begin{equation}\label{eqn:rotor:660}
\alpha^2 -\Bbeta^2 = 2{\cos}^2{\theta} -1 = \cos\theta.
\end{equation}

The triple product \(\Be_i \cdot (\Bbeta \cross \Be_j)\) is zero along the diagonal where \(i=j\) since an \(\Be_j=\Be_i\) cross has no \(\Be_i\) component, so
for \(k \ne i \ne j\), the triple product term is

\begin{equation}\label{eqn:rotor:1160}
\begin{aligned}
2\alpha \Be_i \cdot (\Bbeta \cross \Be_j)
&= 2\alpha \beta_k \Be_i \cdot (\Be_k \cross \Be_j) \\
&= 2\alpha \beta_k \Sgn{\pi_{ikj}} \\
&= 2 n_k \cos({\theta/2})\sin({\theta/2}) \Sgn{\pi_{ikj}} \\
&= n_k \sin{\theta} \Sgn{\pi_{ikj}} \\
\end{aligned}
\end{equation}

The last term is:
\begin{equation}\label{eqn:rotor:680}
2\beta_i\beta_j
= 2 n_i n_j {\sin}^2({\theta/2})
= n_i n_j (1-\cos\theta)
\end{equation}

Thus we can alternatively write \eqnref{eqn:rotor:rotmgreek}

\begin{equation}\label{eqn:rotor:rotmn}
\Be_i \cdot \left(e^{-\Bn I \theta/2} \Be_j e^{\Bn I \theta/2}\right)
= \delta_{ij}\cos\theta
+ n_k \sin{\theta} \epsilon_{ikj} + n_i n_j (1-\cos\theta)
\end{equation}

This is enough to easily and explicitly write out the complete rotation matrix for a rotation about unit vector \(\Bn = (n_1, n_2, n_3)\):
(with basis \(\sigma = \{\Be_i\}\)):

\begin{equation}\label{eqn:rotor:700}
[
R_\theta
]_\sigma^\sigma
=
\begin{bmatrix}
\cos\theta(1 -n_1^2) + n_1^2 & n_1 n_2 (1-\cos\theta) - n_3 \sin\theta & n_1 n_3 (1-\cos\theta) + n_2 \sin\theta \\
n_1 n_2 (1-\cos\theta) + n_3 \sin\theta & \cos\theta(1 -n_2^2) + n_2^2 & n_2 n_3 (1-\cos\theta) - n_1 \sin\theta \\
n_1 n_3 (1-\cos\theta) - n_2 \sin\theta & n_2 n_3 (1-\cos\theta) + n_1 \sin\theta & \cos\theta(1 -n_3^2) + n_3^2 \\
\end{bmatrix}
\end{equation}

Note also that the \(n_i\) terms are the direction cosines of the unit normal for the rotation, so all the terms above
are really strictly sums of sine and cosine products, so we have the rotation matrix completely described in terms of four
angles.  Also observe how much additional complexity we have to express a rotation in terms of the matrix.  This representation also
does not work for plane rotations, just vectors (whereas that is not the case for the rotor form).

It is actually somewhat simpler looking to leave things in terms of the \(\alpha\), and \(\beta\) parameters.  We can rewrite
\eqnref{eqn:rotor:rotmgreek} as:

\begin{equation}
\Be_i \cdot \left(e^{-\Bn I \theta/2} \Be_j e^{\Bn I \theta/2}\right)
= \delta_{ij}(2\alpha^2 -1)
+2\alpha \beta_k \epsilon_{ikj} + 2\beta_i\beta_j
\end{equation}

and the rotation matrix:

\begin{equation}\label{eqn:rotor:720}
[
R_\theta
]_\sigma^\sigma
=
2
\begin{bmatrix}
\alpha^2 -\frac{1}{2} + \beta_1^2 & \beta_1 \beta_2  - \beta_3 \alpha & \beta_1 \beta_3  + \beta_2 \alpha \\
\beta_1 \beta_2  + \beta_3 \alpha & \alpha^2 -\frac{1}{2} + \beta_2^2 & \beta_2 \beta_3  - \beta_1 \alpha \\
\beta_1 \beta_3  - \beta_2 \alpha & \beta_2 \beta_3  + \beta_1 \alpha & \alpha^2 -\frac{1}{2} + \beta_3^2 \\
\end{bmatrix}
\end{equation}

Not really that much simpler, but a bit.  The trade off is that the similarity to the standard \(2x2\) rotation matrix is not obvious.


\documentclass{article}      % Specifies the document class

\usepackage{amsmath}
\usepackage{mathpazo}

%
% shorthand for bold symbols, convenient for vectors and matrices
%
\newcommand{\Ba}[0]{\mathbf{a}}
\newcommand{\Bb}[0]{\mathbf{b}}
\newcommand{\Bc}[0]{\mathbf{c}}
\newcommand{\Bd}[0]{\mathbf{d}}
\newcommand{\Be}[0]{\mathbf{e}}
\newcommand{\Bf}[0]{\mathbf{f}}
\newcommand{\Bg}[0]{\mathbf{g}}
\newcommand{\Bh}[0]{\mathbf{h}}
\newcommand{\Bi}[0]{\mathbf{i}}
\newcommand{\Bj}[0]{\mathbf{j}}
\newcommand{\Bk}[0]{\mathbf{k}}
\newcommand{\Bl}[0]{\mathbf{l}}
\newcommand{\Bm}[0]{\mathbf{m}}
\newcommand{\Bn}[0]{\mathbf{n}}
\newcommand{\Bo}[0]{\mathbf{o}}
\newcommand{\Bp}[0]{\mathbf{p}}
\newcommand{\Bq}[0]{\mathbf{q}}
\newcommand{\Br}[0]{\mathbf{r}}
\newcommand{\Bs}[0]{\mathbf{s}}
\newcommand{\Bt}[0]{\mathbf{t}}
\newcommand{\Bu}[0]{\mathbf{u}}
\newcommand{\Bv}[0]{\mathbf{v}}
\newcommand{\Bw}[0]{\mathbf{w}}
\newcommand{\Bx}[0]{\mathbf{x}}
\newcommand{\By}[0]{\mathbf{y}}
\newcommand{\Bz}[0]{\mathbf{z}}
\newcommand{\BA}[0]{\mathbf{A}}
\newcommand{\BB}[0]{\mathbf{B}}
\newcommand{\BC}[0]{\mathbf{C}}
\newcommand{\BD}[0]{\mathbf{D}}
\newcommand{\BE}[0]{\mathbf{E}}
\newcommand{\BF}[0]{\mathbf{F}}
\newcommand{\BG}[0]{\mathbf{G}}
\newcommand{\BH}[0]{\mathbf{H}}
\newcommand{\BI}[0]{\mathbf{I}}
\newcommand{\BJ}[0]{\mathbf{J}}
\newcommand{\BK}[0]{\mathbf{K}}
\newcommand{\BL}[0]{\mathbf{L}}
\newcommand{\BM}[0]{\mathbf{M}}
\newcommand{\BN}[0]{\mathbf{N}}
\newcommand{\BO}[0]{\mathbf{O}}
\newcommand{\BP}[0]{\mathbf{P}}
\newcommand{\BQ}[0]{\mathbf{Q}}
\newcommand{\BR}[0]{\mathbf{R}}
\newcommand{\BS}[0]{\mathbf{S}}
\newcommand{\BT}[0]{\mathbf{T}}
\newcommand{\BU}[0]{\mathbf{U}}
\newcommand{\BV}[0]{\mathbf{V}}
\newcommand{\BW}[0]{\mathbf{W}}
\newcommand{\BX}[0]{\mathbf{X}}
\newcommand{\BY}[0]{\mathbf{Y}}
\newcommand{\BZ}[0]{\mathbf{Z}}

\newcommand{\Bzero}[0]{\mathbf{0}}
\newcommand{\Btheta}[0]{\boldsymbol{\theta}}
\newcommand{\Btau}[0]{\boldsymbol{\tau}}
\newcommand{\Bomega}[0]{\boldsymbol{\omega}}

%
% shorthand for unit vectors
%
\newcommand{\acap}[0]{\hat{\Ba}}
\newcommand{\bcap}[0]{\hat{\Bb}}
\newcommand{\ccap}[0]{\hat{\Bc}}
\newcommand{\dcap}[0]{\hat{\Bd}}
\newcommand{\ecap}[0]{\hat{\Be}}
\newcommand{\fcap}[0]{\hat{\Bf}}
\newcommand{\gcap}[0]{\hat{\Bg}}
\newcommand{\hcap}[0]{\hat{\Bh}}
\newcommand{\icap}[0]{\hat{\Bi}}
\newcommand{\jcap}[0]{\hat{\Bj}}
\newcommand{\kcap}[0]{\hat{\Bk}}
\newcommand{\lcap}[0]{\hat{\Bl}}
\newcommand{\mcap}[0]{\hat{\Bm}}
\newcommand{\ncap}[0]{\hat{\Bn}}
\newcommand{\ocap}[0]{\hat{\Bo}}
\newcommand{\pcap}[0]{\hat{\Bp}}
\newcommand{\qcap}[0]{\hat{\Bq}}
\newcommand{\rcap}[0]{\hat{\Br}}
\newcommand{\scap}[0]{\hat{\Bs}}
\newcommand{\tcap}[0]{\hat{\Bt}}
\newcommand{\ucap}[0]{\hat{\Bu}}
\newcommand{\vcap}[0]{\hat{\Bv}}
\newcommand{\wcap}[0]{\hat{\Bw}}
\newcommand{\xcap}[0]{\hat{\Bx}}
\newcommand{\ycap}[0]{\hat{\By}}
\newcommand{\zcap}[0]{\hat{\Bz}}
\newcommand{\thetacap}[0]{\hat{\Btheta}}

%
% to write R^n and C^n in a distinguishable fashion.  Perhaps change this
% to the double lined characters upon figuring out how to do so.
%
\newcommand{\C}[1]{$\mathbb{C}^{#1}$}
\newcommand{\R}[1]{$\mathbb{R}^{#1}$}

%
% various generally useful helpers
%

% derivative of #1 wrt. #2:
\newcommand{\D}[2] {\frac {d#2} {d#1}}

\newcommand{\inv}[1]{\frac{1}{#1}}
\newcommand{\cross}[0]{\times}

\newcommand{\abs}[1]{\lvert{#1}\rvert}
\newcommand{\norm}[1]{\lVert{#1}\rVert}
\newcommand{\innerprod}[2]{\langle{#1}, {#2}\rangle}
\newcommand{\dotprod}[2]{{#1} \cdot {#2}}
\newcommand{\bdotprod}[2]{\left({#1} \cdot {#2}\right)}
\newcommand{\crossprod}[2]{{#1} \cross {#2}}
\newcommand{\tripleprod}[3]{\dotprod{\left(\crossprod{#1}{#2}\right)}{#3}}

\DeclareMathOperator{\Proj}{Proj}
\DeclareMathOperator{\Span}{span}
\DeclareMathOperator{\Sgn}{sgn}
\DeclareMathOperator{\Area}{Area}
\DeclareMathOperator{\Volume}{Volume}

%
% A few miscellaneous things specific to this document
%
\newcommand{\crossop}[1]{\crossprod{#1}{}}

% R2 vector.
\newcommand{\VectorTwo}[2]{
\begin{bmatrix}
 {#1} \\
 {#2}
\end{bmatrix}
}

\newcommand{\VectorN}[1]{
\begin{bmatrix}
{#1}_1 \\
{#1}_2 \\
\vdots \\
{#1}_N \\
\end{bmatrix}
}

\newcommand{\DETuvij}[4]{
\begin{vmatrix}
 {#1}_{#3} & {#1}_{#4} \\
 {#2}_{#3} & {#2}_{#4}
\end{vmatrix}
}

\newcommand{\DETuvwijk}[6]{
\begin{vmatrix}
 {#1}_{#4} & {#1}_{#5} & {#1}_{#6} \\
 {#2}_{#4} & {#2}_{#5} & {#2}_{#6} \\
 {#3}_{#4} & {#3}_{#5} & {#3}_{#6}
\end{vmatrix}
}

\newcommand{\DETuvwxijkl}[8]{
\begin{vmatrix}
 {#1}_{#5} & {#1}_{#6} & {#1}_{#7} & {#1}_{#8} \\
 {#2}_{#5} & {#2}_{#6} & {#2}_{#7} & {#2}_{#8} \\
 {#3}_{#5} & {#3}_{#6} & {#3}_{#7} & {#3}_{#8} \\
 {#4}_{#5} & {#4}_{#6} & {#4}_{#7} & {#4}_{#8} \\
\end{vmatrix}
}

%\newcommand{\DETuvwxyijklm}[10]{
%\begin{vmatrix}
% {#1}_{#6} & {#1}_{#7} & {#1}_{#8} & {#1}_{#9} & {#1}_{#10} \\
% {#2}_{#6} & {#2}_{#7} & {#2}_{#8} & {#2}_{#9} & {#2}_{#10} \\
% {#3}_{#6} & {#3}_{#7} & {#3}_{#8} & {#3}_{#9} & {#3}_{#10} \\
% {#4}_{#6} & {#4}_{#7} & {#4}_{#8} & {#4}_{#9} & {#4}_{#10} \\
% {#5}_{#6} & {#5}_{#7} & {#5}_{#8} & {#5}_{#9} & {#5}_{#10}
%\end{vmatrix}
%}

% R3 vector.
\newcommand{\VectorThree}[3]{
\begin{bmatrix}
 {#1} \\
 {#2} \\
 {#3}
\end{bmatrix}
}


%<misc>
%
\newcommand{\Abs}[1]{{\left\lvert{#1}\right\rvert}}
\newcommand{\spacegrad}[0]{\boldsymbol{\nabla}}
\newcommand{\grad}[0]{\nabla}
\newcommand{\LL}[0]{\mathcal{L}}

% == \partial_{#1} {#2}
\newcommand{\PD}[2]{\frac{\partial {#2}}{\partial {#1}}}
% inline variant
\newcommand{\PDi}[2]{{\partial {#2}}/{\partial {#1}}}

\newcommand{\PDD}[3]{\frac{\partial^2 {#3}}{\partial {#1}\partial {#2}}}
%\newcommand{\PDd}[2]{\frac{\partial^2 {#2}}{{\partial{#1}}^2}}
\newcommand{\PDsq}[2]{\frac{\partial^2 {#2}}{(\partial {#1})^2}}

\newcommand{\Partial}[2]{\frac{\partial {#1}}{\partial {#2}}}
\DeclareMathOperator{\RejName}{Rej}
\newcommand{\Rej}[2]{\RejName_{#1}\left( {#2} \right)}
\newcommand{\Rm}[1]{\mathbb{R}^{#1}}
\newcommand{\Cm}[1]{\mathbb{C}^{#1}}
\newcommand{\conj}[0]{{*}}

%</misc>

% <grade selection>
%
\newcommand{\gpgrade}[2] {{\left\langle{{#1}}\right\rangle}_{#2}}

\newcommand{\gpgradezero}[1] {\gpgrade{#1}{}}
%\newcommand{\gpscalargrade}[1] {{\left\langle{{#1}}\right\rangle}}
%\newcommand{\gpgradezero}[1] {\gpgrade{#1}{0}}

%\newcommand{\gpgradeone}[1] {{\left\langle{{#1}}\right\rangle}_{1}}
\newcommand{\gpgradeone}[1] {\gpgrade{#1}{1}}

\newcommand{\gpgradetwo}[1] {\gpgrade{#1}{2}}
\newcommand{\gpgradethree}[1] {\gpgrade{#1}{3}}
\newcommand{\gpgradefour}[1] {\gpgrade{#1}{4}}
%
% </grade selection>



\newcommand{\adot}[0]{{\dot{a}}}
\newcommand{\bdot}[0]{{\dot{b}}}
% taken for centered dot:
%\newcommand{\cdot}[0]{{\dot{c}}}
%\newcommand{\ddot}[0]{{\dot{d}}}
\newcommand{\edot}[0]{{\dot{e}}}
\newcommand{\fdot}[0]{{\dot{f}}}
\newcommand{\gdot}[0]{{\dot{g}}}
\newcommand{\hdot}[0]{{\dot{h}}}
\newcommand{\idot}[0]{{\dot{i}}}
\newcommand{\jdot}[0]{{\dot{j}}}
\newcommand{\kdot}[0]{{\dot{k}}}
\newcommand{\ldot}[0]{{\dot{l}}}
\newcommand{\mdot}[0]{{\dot{m}}}
\newcommand{\ndot}[0]{{\dot{n}}}
%\newcommand{\odot}[0]{{\dot{o}}}
\newcommand{\pdot}[0]{{\dot{p}}}
\newcommand{\qdot}[0]{{\dot{q}}}
\newcommand{\rdot}[0]{{\dot{r}}}
\newcommand{\sdot}[0]{{\dot{s}}}
\newcommand{\tdot}[0]{{\dot{t}}}
\newcommand{\udot}[0]{{\dot{u}}}
\newcommand{\vdot}[0]{{\dot{v}}}
\newcommand{\wdot}[0]{{\dot{w}}}
\newcommand{\xdot}[0]{{\dot{x}}}
\newcommand{\ydot}[0]{{\dot{y}}}
\newcommand{\zdot}[0]{{\dot{z}}}
\newcommand{\addot}[0]{{\ddot{a}}}
\newcommand{\bddot}[0]{{\ddot{b}}}
\newcommand{\cddot}[0]{{\ddot{c}}}
%\newcommand{\dddot}[0]{{\ddot{d}}}
\newcommand{\eddot}[0]{{\ddot{e}}}
\newcommand{\fddot}[0]{{\ddot{f}}}
\newcommand{\gddot}[0]{{\ddot{g}}}
\newcommand{\hddot}[0]{{\ddot{h}}}
\newcommand{\iddot}[0]{{\ddot{i}}}
\newcommand{\jddot}[0]{{\ddot{j}}}
\newcommand{\kddot}[0]{{\ddot{k}}}
\newcommand{\lddot}[0]{{\ddot{l}}}
\newcommand{\mddot}[0]{{\ddot{m}}}
\newcommand{\nddot}[0]{{\ddot{n}}}
\newcommand{\oddot}[0]{{\ddot{o}}}
\newcommand{\pddot}[0]{{\ddot{p}}}
\newcommand{\qddot}[0]{{\ddot{q}}}
\newcommand{\rddot}[0]{{\ddot{r}}}
\newcommand{\sddot}[0]{{\ddot{s}}}
\newcommand{\tddot}[0]{{\ddot{t}}}
\newcommand{\uddot}[0]{{\ddot{u}}}
\newcommand{\vddot}[0]{{\ddot{v}}}
\newcommand{\wddot}[0]{{\ddot{w}}}
\newcommand{\xddot}[0]{{\ddot{x}}}
\newcommand{\yddot}[0]{{\ddot{y}}}
\newcommand{\zddot}[0]{{\ddot{z}}}

%<bold and dot greek symbols>
%

\newcommand{\Deltadot}[0]{{\dot{\Delta}}}
\newcommand{\Gammadot}[0]{{\dot{\Gamma}}}
\newcommand{\Lambdadot}[0]{{\dot{\Lambda}}}
\newcommand{\Omegadot}[0]{{\dot{\Omega}}}
\newcommand{\Phidot}[0]{{\dot{\Phi}}}
\newcommand{\Pidot}[0]{{\dot{\Pi}}}
\newcommand{\Psidot}[0]{{\dot{\Psi}}}
\newcommand{\Sigmadot}[0]{{\dot{\Sigma}}}
\newcommand{\Thetadot}[0]{{\dot{\Theta}}}
\newcommand{\Upsilondot}[0]{{\dot{\Upsilon}}}
\newcommand{\Xidot}[0]{{\dot{\Xi}}}
\newcommand{\alphadot}[0]{{\dot{\alpha}}}
\newcommand{\betadot}[0]{{\dot{\beta}}}
\newcommand{\chidot}[0]{{\dot{\chi}}}
\newcommand{\deltadot}[0]{{\dot{\delta}}}
\newcommand{\epsilondot}[0]{{\dot{\epsilon}}}
\newcommand{\etadot}[0]{{\dot{\eta}}}
\newcommand{\gammadot}[0]{{\dot{\gamma}}}
\newcommand{\kappadot}[0]{{\dot{\kappa}}}
\newcommand{\lambdadot}[0]{{\dot{\lambda}}}
\newcommand{\mudot}[0]{{\dot{\mu}}}
\newcommand{\nudot}[0]{{\dot{\nu}}}
\newcommand{\omegadot}[0]{{\dot{\omega}}}
\newcommand{\phidot}[0]{{\dot{\phi}}}
\newcommand{\pidot}[0]{{\dot{\pi}}}
\newcommand{\psidot}[0]{{\dot{\psi}}}
\newcommand{\rhodot}[0]{{\dot{\rho}}}
\newcommand{\sigmadot}[0]{{\dot{\sigma}}}
\newcommand{\taudot}[0]{{\dot{\tau}}}
\newcommand{\thetadot}[0]{{\dot{\theta}}}
\newcommand{\upsilondot}[0]{{\dot{\upsilon}}}
\newcommand{\varepsilondot}[0]{{\dot{\varepsilon}}}
\newcommand{\varphidot}[0]{{\dot{\varphi}}}
\newcommand{\varpidot}[0]{{\dot{\varpi}}}
\newcommand{\varrhodot}[0]{{\dot{\varrho}}}
\newcommand{\varsigmadot}[0]{{\dot{\varsigma}}}
\newcommand{\varthetadot}[0]{{\dot{\vartheta}}}
\newcommand{\xidot}[0]{{\dot{\xi}}}
\newcommand{\zetadot}[0]{{\dot{\zeta}}}

\newcommand{\Deltaddot}[0]{{\ddot{\Delta}}}
\newcommand{\Gammaddot}[0]{{\ddot{\Gamma}}}
\newcommand{\Lambdaddot}[0]{{\ddot{\Lambda}}}
\newcommand{\Omegaddot}[0]{{\ddot{\Omega}}}
\newcommand{\Phiddot}[0]{{\ddot{\Phi}}}
\newcommand{\Piddot}[0]{{\ddot{\Pi}}}
\newcommand{\Psiddot}[0]{{\ddot{\Psi}}}
\newcommand{\Sigmaddot}[0]{{\ddot{\Sigma}}}
\newcommand{\Thetaddot}[0]{{\ddot{\Theta}}}
\newcommand{\Upsilonddot}[0]{{\ddot{\Upsilon}}}
\newcommand{\Xiddot}[0]{{\ddot{\Xi}}}
\newcommand{\alphaddot}[0]{{\ddot{\alpha}}}
\newcommand{\betaddot}[0]{{\ddot{\beta}}}
\newcommand{\chiddot}[0]{{\ddot{\chi}}}
\newcommand{\deltaddot}[0]{{\ddot{\delta}}}
\newcommand{\epsilonddot}[0]{{\ddot{\epsilon}}}
\newcommand{\etaddot}[0]{{\ddot{\eta}}}
\newcommand{\gammaddot}[0]{{\ddot{\gamma}}}
\newcommand{\kappaddot}[0]{{\ddot{\kappa}}}
\newcommand{\lambdaddot}[0]{{\ddot{\lambda}}}
\newcommand{\muddot}[0]{{\ddot{\mu}}}
\newcommand{\nuddot}[0]{{\ddot{\nu}}}
\newcommand{\omegaddot}[0]{{\ddot{\omega}}}
\newcommand{\phiddot}[0]{{\ddot{\phi}}}
\newcommand{\piddot}[0]{{\ddot{\pi}}}
\newcommand{\psiddot}[0]{{\ddot{\psi}}}
\newcommand{\rhoddot}[0]{{\ddot{\rho}}}
\newcommand{\sigmaddot}[0]{{\ddot{\sigma}}}
\newcommand{\tauddot}[0]{{\ddot{\tau}}}
\newcommand{\thetaddot}[0]{{\ddot{\theta}}}
\newcommand{\upsilonddot}[0]{{\ddot{\upsilon}}}
\newcommand{\varepsilonddot}[0]{{\ddot{\varepsilon}}}
\newcommand{\varphiddot}[0]{{\ddot{\varphi}}}
\newcommand{\varpiddot}[0]{{\ddot{\varpi}}}
\newcommand{\varrhoddot}[0]{{\ddot{\varrho}}}
\newcommand{\varsigmaddot}[0]{{\ddot{\varsigma}}}
\newcommand{\varthetaddot}[0]{{\ddot{\vartheta}}}
\newcommand{\xiddot}[0]{{\ddot{\xi}}}
\newcommand{\zetaddot}[0]{{\ddot{\zeta}}}

\newcommand{\BDelta}[0]{\boldsymbol{\Delta}}
\newcommand{\BGamma}[0]{\boldsymbol{\Gamma}}
\newcommand{\BLambda}[0]{\boldsymbol{\Lambda}}
\newcommand{\BOmega}[0]{\boldsymbol{\Omega}}
\newcommand{\BPhi}[0]{\boldsymbol{\Phi}}
\newcommand{\BPi}[0]{\boldsymbol{\Pi}}
\newcommand{\BPsi}[0]{\boldsymbol{\Psi}}
\newcommand{\BSigma}[0]{\boldsymbol{\Sigma}}
\newcommand{\BTheta}[0]{\boldsymbol{\Theta}}
\newcommand{\BUpsilon}[0]{\boldsymbol{\Upsilon}}
\newcommand{\BXi}[0]{\boldsymbol{\Xi}}
\newcommand{\Balpha}[0]{\boldsymbol{\alpha}}
\newcommand{\Bbeta}[0]{\boldsymbol{\beta}}
\newcommand{\Bchi}[0]{\boldsymbol{\chi}}
\newcommand{\Bdelta}[0]{\boldsymbol{\delta}}
\newcommand{\Bepsilon}[0]{\boldsymbol{\epsilon}}
\newcommand{\Beta}[0]{\boldsymbol{\eta}}
\newcommand{\Bgamma}[0]{\boldsymbol{\gamma}}
\newcommand{\Bkappa}[0]{\boldsymbol{\kappa}}
\newcommand{\Blambda}[0]{\boldsymbol{\lambda}}
\newcommand{\Bmu}[0]{\boldsymbol{\mu}}
\newcommand{\Bnu}[0]{\boldsymbol{\nu}}
%\newcommand{\Bomega}[0]{\boldsymbol{\omega}}
\newcommand{\Bphi}[0]{\boldsymbol{\phi}}
\newcommand{\Bpi}[0]{\boldsymbol{\pi}}
\newcommand{\Bpsi}[0]{\boldsymbol{\psi}}
\newcommand{\Brho}[0]{\boldsymbol{\rho}}
\newcommand{\Bsigma}[0]{\boldsymbol{\sigma}}
%\newcommand{\Btau}[0]{\boldsymbol{\tau}}
%\newcommand{\Btheta}[0]{\boldsymbol{\theta}}
\newcommand{\Bupsilon}[0]{\boldsymbol{\upsilon}}
\newcommand{\Bvarepsilon}[0]{\boldsymbol{\varepsilon}}
\newcommand{\Bvarphi}[0]{\boldsymbol{\varphi}}
\newcommand{\Bvarpi}[0]{\boldsymbol{\varpi}}
\newcommand{\Bvarrho}[0]{\boldsymbol{\varrho}}
\newcommand{\Bvarsigma}[0]{\boldsymbol{\varsigma}}
\newcommand{\Bvartheta}[0]{\boldsymbol{\vartheta}}
\newcommand{\Bxi}[0]{\boldsymbol{\xi}}
\newcommand{\Bzeta}[0]{\boldsymbol{\zeta}}
%
%</bold and dot greek symbols>
%<infrequent>
%
%\newcommand{\AreaOp}[1]{\AName_{#1}}
%\newcommand{\Babs}[0]{\abs{\BB}}
%\newcommand{\Bcap}[0]{\hat{\BB}}
%\newcommand{\BrPrimeRej}[0]{\rcap(\rcap \wedge \Br')}
%\newcommand{\CA}[0]{\mathcal{A}}
%\newcommand{\Cos}[1]{\cos{\left({#1}\right)}}
%\newcommand{\Det}[1] {\abs{#1}}
%\newcommand{\Dsq}[2] {\frac {\partial^2 {#1}} {\partial {#2}^2}}
%\newcommand{\Exp}[1]{\exp{\left({#1}\right)}}
%\newcommand{\Norm}[1]{\left\lVert{#1}\right\rVert}
%\newcommand{\Sin}[1]{\sin{\left({#1}\right)}}
%\newcommand{\T}[0]{\text{T}}
%\newcommand{\VolumeOp}[1]{\VName_{#1}}
%\newcommand{\agrad}[0]{\Ba \cdot \nabla}
%\newcommand{\alphacap}[0]{\hat{\boldsymbol{\alpha}}}
%\newcommand{\Fcap}[0]{\hat{\BF}}
%\newcommand{\bithree}[0]{{\Bi}_3}
%\newcommand{\bxa}[0]{\Bx\Ba}
%\newcommand{\coordvec}[2]{
%\newcommand{\costheta}[0]{\acap \cdot \xcap}
%\newcommand{\ddt}[1]{\ddot{#1}}
%\newcommand{\ddu}[1] {\frac {d{#1}} {du}}
%\newcommand{\dsqxj}[2] {\frac {\partial^2 {#1}} {\partial {x_{#2}}^2}}
%\newcommand{\dtheta}[1]{\frac{d {#1}}{d \theta}}
%\newcommand{\dt}[1]{\dot{#1}}
%\newcommand{\dt}[1]{\frac{d {#1}}{dt}}
%\newcommand{\dxj}[2] {\frac {\partial {#1}} {\partial {x_{#2}}}}
%\newcommand{\halfPhi}[0]{\frac{\phi}{2}}
%\newcommand{\half}[0]{\inv{2}}
%\newcommand{\inv}[1]{\frac{1}{#1}}
%\newcommand{\laplacian}[0]{\nabla^2}
%\newcommand{\matrixoftx}[3]{
%\newcommand{\nrrp}[0]{\norm{\rcap \wedge \Br'}}
%\newcommand{\oiint}{\bigcirc \hspace{-1.4em} \int \hspace{-.8em} \int}
%\newcommand{\transpose}[1]{{#1}^{\text{T}}}
%\newcommand{\transpose}[1]{{{#1}^{\TextTranspose}}}
%\newcommand{\transpose}[1]{{{#1}^{\text{T}}}}
%\newcommand{\barA}[0]{\bar{A}}
%\newcommand{\qbar}[0]{\bar{q}}
%\newcommand{\qdotbar}[0]{\dot{\bar{q}}}
%
%</infrequent>





%
% The real thing:
%

\usepackage[bookmarks=true]{hyperref}

                             % The preamble begins here.
\title{Some notes on Euler Angles.} % Declares the document's title.
\author{Peeter Joot}         % Declares the author's name.
\date{ November 1, 2008. Last Revision: $Date: 2008/11/02 04:15:55 $ } % Deleting this command produces today's date.

\begin{document}             % End of preamble and beginning of text.

\maketitle{}
\tableofcontents

\section{ Removing the rotors from the exponentials. }

In \cite{doran2003gap} section 2.7.5 the euler angle formula is 
developed for $\{z,x',z''\}$ axis rotations by $\{\phi, \theta, \psi\}$
respectively.

Other than a few details the derivation is pretty straightforward.  Equation
2.153 would be clearer with a series expansion hint like

\begin{align*}
\exp(R \alpha i R^\dagger) 
&= \sum_k \inv{k!} (R \alpha i R^\dagger)^k \\
&= \sum_k \inv{k!} R (\alpha i)^k R^\dagger \\
&= R \exp(\alpha i) R^\dagger
\end{align*}

where $i$ is a bivector, and $R$ is a rotor.

\section{ In matrix form. }

The end result of the composite Euler rotations is that the rotation is

\begin{align*}
R(x) &= R x R^\dagger \\
R &= \exp(-e_{12}\phi/2) \exp(-e_{23}\theta/2) \exp(-e_{12}\psi/2)
\end{align*}

Then there are notes saying this is easier to visualize and work with than
the equivalent matrix formula.  Let's see what the equivalent matrix formula
is



\bibliographystyle{plainnat} % supposed to allow for \url use.
\bibliography{myrefs}      % expects file "myrefs.bib"

\end{document}               % End of document.

\include{spherical_polar}
% 
% 
% 
% Copyright � 2012 Peeter Joot
% All Rights Reserved
% 
% This file may be reproduced and distributed in whole or in part, without fee, subject to the following conditions:
% 
% o The copyright notice above and this permission notice must be preserved complete on all complete or partial copies.
% 
% o Any translation or derived work must be approved by the author in writing before distribution.
% 
% o If you distribute this work in part, instructions for obtaining the complete version of this file must be included, and a means for obtaining a complete version provided.
% 
% 
% Exceptions to these rules may be granted for academic purposes: Write to the author and ask.
% 
% 
% 
\chapter{Rotor interpolation calculation.}
\label{chap:slerp}
\date{Nov 30, 2008.  slerp.tex}

The aim is to compute the interpolating rotor $r$ that takes an object
from one position to another in $n$ steps.
Here the initial and final positions are given by two rotors $R_1$, and $R_2$
like so

\begin{align*}
X_1 &= R_1 X {R_1}^\dagger \\
X_2 &= R_2 X {R_2}^\dagger = r^n R_1 X {R_1}^\dagger {r^n}^\dagger
\end{align*}

So, writing 

\begin{align*}
%r^n R_1 = R_2 
a = r^n = R_2 \inv{R_1} = \frac{R_2 {R_1}^\dagger}{R_1 {{R_1}^\dagger}} = \cos\theta + I \sin\theta
\end{align*}

So, 

\begin{align*}
\frac{\gpgradetwo{a}}{\gpgradezero{a}} &= 
\frac{\gpgradetwo{a}}{\Abs{\gpgradetwo{a}}} \frac{\Abs{\gpgradetwo{a}}}{\gpgradezero{a}} \\
&= I \tan\theta
\end{align*}

Therefore the interpolating rotor is:
\begin{align*}
I &= \frac{\gpgradetwo{a}}{\Abs{\gpgradetwo{a}}} \\
\theta &= \atan2\left(\Abs{\gpgradetwo{a}}, \gpgradezero{a}\right) \\
r &= \cos(\theta/n) + I \sin(\theta/n)
\end{align*}

In \citep{dorst2007gac}, equation $10.15$, they've got something like this
for a fractional angle, but then say that they don't use that in software, 
instead using $r$ directly, with a comment about designing more sophisticated
algorithms (bivector splines).  That spline comment in particular sounds
interesting.  Sounds like the details on that are to be found in the journals
mentioned in Further Reading section of chapter 10.

\include{kvector_exponential}
\part{Calculus.}
\include{multivector_taylors}
\include{gradient_and_forms}
\include{vector_integral_relations}
\include{stokes_revisited}
\documentclass{article}

\usepackage{amsmath}
\usepackage{mathpazo}

%
% shorthand for bold symbols, convenient for vectors and matrices
%
\newcommand{\Ba}[0]{\mathbf{a}}
\newcommand{\Bb}[0]{\mathbf{b}}
\newcommand{\Bc}[0]{\mathbf{c}}
\newcommand{\Bd}[0]{\mathbf{d}}
\newcommand{\Be}[0]{\mathbf{e}}
\newcommand{\Bf}[0]{\mathbf{f}}
\newcommand{\Bg}[0]{\mathbf{g}}
\newcommand{\Bh}[0]{\mathbf{h}}
\newcommand{\Bi}[0]{\mathbf{i}}
\newcommand{\Bj}[0]{\mathbf{j}}
\newcommand{\Bk}[0]{\mathbf{k}}
\newcommand{\Bl}[0]{\mathbf{l}}
\newcommand{\Bm}[0]{\mathbf{m}}
\newcommand{\Bn}[0]{\mathbf{n}}
\newcommand{\Bo}[0]{\mathbf{o}}
\newcommand{\Bp}[0]{\mathbf{p}}
\newcommand{\Bq}[0]{\mathbf{q}}
\newcommand{\Br}[0]{\mathbf{r}}
\newcommand{\Bs}[0]{\mathbf{s}}
\newcommand{\Bt}[0]{\mathbf{t}}
\newcommand{\Bu}[0]{\mathbf{u}}
\newcommand{\Bv}[0]{\mathbf{v}}
\newcommand{\Bw}[0]{\mathbf{w}}
\newcommand{\Bx}[0]{\mathbf{x}}
\newcommand{\By}[0]{\mathbf{y}}
\newcommand{\Bz}[0]{\mathbf{z}}
\newcommand{\BA}[0]{\mathbf{A}}
\newcommand{\BB}[0]{\mathbf{B}}
\newcommand{\BC}[0]{\mathbf{C}}
\newcommand{\BD}[0]{\mathbf{D}}
\newcommand{\BE}[0]{\mathbf{E}}
\newcommand{\BF}[0]{\mathbf{F}}
\newcommand{\BG}[0]{\mathbf{G}}
\newcommand{\BH}[0]{\mathbf{H}}
\newcommand{\BI}[0]{\mathbf{I}}
\newcommand{\BJ}[0]{\mathbf{J}}
\newcommand{\BK}[0]{\mathbf{K}}
\newcommand{\BL}[0]{\mathbf{L}}
\newcommand{\BM}[0]{\mathbf{M}}
\newcommand{\BN}[0]{\mathbf{N}}
\newcommand{\BO}[0]{\mathbf{O}}
\newcommand{\BP}[0]{\mathbf{P}}
\newcommand{\BQ}[0]{\mathbf{Q}}
\newcommand{\BR}[0]{\mathbf{R}}
\newcommand{\BS}[0]{\mathbf{S}}
\newcommand{\BT}[0]{\mathbf{T}}
\newcommand{\BU}[0]{\mathbf{U}}
\newcommand{\BV}[0]{\mathbf{V}}
\newcommand{\BW}[0]{\mathbf{W}}
\newcommand{\BX}[0]{\mathbf{X}}
\newcommand{\BY}[0]{\mathbf{Y}}
\newcommand{\BZ}[0]{\mathbf{Z}}

\newcommand{\Bzero}[0]{\mathbf{0}}
\newcommand{\Btheta}[0]{\boldsymbol{\theta}}
\newcommand{\Btau}[0]{\boldsymbol{\tau}}
\newcommand{\Bomega}[0]{\boldsymbol{\omega}}

%
% shorthand for unit vectors
%
\newcommand{\acap}[0]{\hat{\Ba}}
\newcommand{\bcap}[0]{\hat{\Bb}}
\newcommand{\ccap}[0]{\hat{\Bc}}
\newcommand{\dcap}[0]{\hat{\Bd}}
\newcommand{\ecap}[0]{\hat{\Be}}
\newcommand{\fcap}[0]{\hat{\Bf}}
\newcommand{\gcap}[0]{\hat{\Bg}}
\newcommand{\hcap}[0]{\hat{\Bh}}
\newcommand{\icap}[0]{\hat{\Bi}}
\newcommand{\jcap}[0]{\hat{\Bj}}
\newcommand{\kcap}[0]{\hat{\Bk}}
\newcommand{\lcap}[0]{\hat{\Bl}}
\newcommand{\mcap}[0]{\hat{\Bm}}
\newcommand{\ncap}[0]{\hat{\Bn}}
\newcommand{\ocap}[0]{\hat{\Bo}}
\newcommand{\pcap}[0]{\hat{\Bp}}
\newcommand{\qcap}[0]{\hat{\Bq}}
\newcommand{\rcap}[0]{\hat{\Br}}
\newcommand{\scap}[0]{\hat{\Bs}}
\newcommand{\tcap}[0]{\hat{\Bt}}
\newcommand{\ucap}[0]{\hat{\Bu}}
\newcommand{\vcap}[0]{\hat{\Bv}}
\newcommand{\wcap}[0]{\hat{\Bw}}
\newcommand{\xcap}[0]{\hat{\Bx}}
\newcommand{\ycap}[0]{\hat{\By}}
\newcommand{\zcap}[0]{\hat{\Bz}}
\newcommand{\thetacap}[0]{\hat{\Btheta}}

%
% to write R^n and C^n in a distinguishable fashion.  Perhaps change this
% to the double lined characters upon figuring out how to do so.
%
\newcommand{\C}[1]{$\mathbb{C}^{#1}$}
\newcommand{\R}[1]{$\mathbb{R}^{#1}$}

%
% various generally useful helpers
%

% derivative of #1 wrt. #2:
\newcommand{\D}[2] {\frac {d#2} {d#1}}

\newcommand{\inv}[1]{\frac{1}{#1}}
\newcommand{\cross}[0]{\times}

\newcommand{\abs}[1]{\lvert{#1}\rvert}
\newcommand{\norm}[1]{\lVert{#1}\rVert}
\newcommand{\innerprod}[2]{\langle{#1}, {#2}\rangle}
\newcommand{\dotprod}[2]{{#1} \cdot {#2}}
\newcommand{\bdotprod}[2]{\left({#1} \cdot {#2}\right)}
\newcommand{\crossprod}[2]{{#1} \cross {#2}}
\newcommand{\tripleprod}[3]{\dotprod{\left(\crossprod{#1}{#2}\right)}{#3}}

\DeclareMathOperator{\Proj}{Proj}
\DeclareMathOperator{\Span}{span}
\DeclareMathOperator{\Sgn}{sgn}
\DeclareMathOperator{\Area}{Area}
\DeclareMathOperator{\Volume}{Volume}

%
% A few miscellaneous things specific to this document
%
\newcommand{\crossop}[1]{\crossprod{#1}{}}

% R2 vector.
\newcommand{\VectorTwo}[2]{
\begin{bmatrix}
 {#1} \\
 {#2}
\end{bmatrix}
}

\newcommand{\VectorN}[1]{
\begin{bmatrix}
{#1}_1 \\
{#1}_2 \\
\vdots \\
{#1}_N \\
\end{bmatrix}
}

\newcommand{\DETuvij}[4]{
\begin{vmatrix}
 {#1}_{#3} & {#1}_{#4} \\
 {#2}_{#3} & {#2}_{#4}
\end{vmatrix}
}

\newcommand{\DETuvwijk}[6]{
\begin{vmatrix}
 {#1}_{#4} & {#1}_{#5} & {#1}_{#6} \\
 {#2}_{#4} & {#2}_{#5} & {#2}_{#6} \\
 {#3}_{#4} & {#3}_{#5} & {#3}_{#6}
\end{vmatrix}
}

\newcommand{\DETuvwxijkl}[8]{
\begin{vmatrix}
 {#1}_{#5} & {#1}_{#6} & {#1}_{#7} & {#1}_{#8} \\
 {#2}_{#5} & {#2}_{#6} & {#2}_{#7} & {#2}_{#8} \\
 {#3}_{#5} & {#3}_{#6} & {#3}_{#7} & {#3}_{#8} \\
 {#4}_{#5} & {#4}_{#6} & {#4}_{#7} & {#4}_{#8} \\
\end{vmatrix}
}

%\newcommand{\DETuvwxyijklm}[10]{
%\begin{vmatrix}
% {#1}_{#6} & {#1}_{#7} & {#1}_{#8} & {#1}_{#9} & {#1}_{#10} \\
% {#2}_{#6} & {#2}_{#7} & {#2}_{#8} & {#2}_{#9} & {#2}_{#10} \\
% {#3}_{#6} & {#3}_{#7} & {#3}_{#8} & {#3}_{#9} & {#3}_{#10} \\
% {#4}_{#6} & {#4}_{#7} & {#4}_{#8} & {#4}_{#9} & {#4}_{#10} \\
% {#5}_{#6} & {#5}_{#7} & {#5}_{#8} & {#5}_{#9} & {#5}_{#10}
%\end{vmatrix}
%}

% R3 vector.
\newcommand{\VectorThree}[3]{
\begin{bmatrix}
 {#1} \\
 {#2} \\
 {#3}
\end{bmatrix}
}


%<misc>
%
\newcommand{\Abs}[1]{{\left\lvert{#1}\right\rvert}}
\newcommand{\spacegrad}[0]{\boldsymbol{\nabla}}
\newcommand{\grad}[0]{\nabla}
\newcommand{\LL}[0]{\mathcal{L}}

% == \partial_{#1} {#2}
\newcommand{\PD}[2]{\frac{\partial {#2}}{\partial {#1}}}
% inline variant
\newcommand{\PDi}[2]{{\partial {#2}}/{\partial {#1}}}

\newcommand{\PDD}[3]{\frac{\partial^2 {#3}}{\partial {#1}\partial {#2}}}
%\newcommand{\PDd}[2]{\frac{\partial^2 {#2}}{{\partial{#1}}^2}}
\newcommand{\PDsq}[2]{\frac{\partial^2 {#2}}{(\partial {#1})^2}}

\newcommand{\Partial}[2]{\frac{\partial {#1}}{\partial {#2}}}
\DeclareMathOperator{\RejName}{Rej}
\newcommand{\Rej}[2]{\RejName_{#1}\left( {#2} \right)}
\newcommand{\Rm}[1]{\mathbb{R}^{#1}}
\newcommand{\Cm}[1]{\mathbb{C}^{#1}}
\newcommand{\conj}[0]{{*}}

%</misc>

% <grade selection>
%
\newcommand{\gpgrade}[2] {{\left\langle{{#1}}\right\rangle}_{#2}}

\newcommand{\gpgradezero}[1] {\gpgrade{#1}{}}
%\newcommand{\gpscalargrade}[1] {{\left\langle{{#1}}\right\rangle}}
%\newcommand{\gpgradezero}[1] {\gpgrade{#1}{0}}

%\newcommand{\gpgradeone}[1] {{\left\langle{{#1}}\right\rangle}_{1}}
\newcommand{\gpgradeone}[1] {\gpgrade{#1}{1}}

\newcommand{\gpgradetwo}[1] {\gpgrade{#1}{2}}
\newcommand{\gpgradethree}[1] {\gpgrade{#1}{3}}
\newcommand{\gpgradefour}[1] {\gpgrade{#1}{4}}
%
% </grade selection>



\newcommand{\adot}[0]{{\dot{a}}}
\newcommand{\bdot}[0]{{\dot{b}}}
% taken for centered dot:
%\newcommand{\cdot}[0]{{\dot{c}}}
%\newcommand{\ddot}[0]{{\dot{d}}}
\newcommand{\edot}[0]{{\dot{e}}}
\newcommand{\fdot}[0]{{\dot{f}}}
\newcommand{\gdot}[0]{{\dot{g}}}
\newcommand{\hdot}[0]{{\dot{h}}}
\newcommand{\idot}[0]{{\dot{i}}}
\newcommand{\jdot}[0]{{\dot{j}}}
\newcommand{\kdot}[0]{{\dot{k}}}
\newcommand{\ldot}[0]{{\dot{l}}}
\newcommand{\mdot}[0]{{\dot{m}}}
\newcommand{\ndot}[0]{{\dot{n}}}
%\newcommand{\odot}[0]{{\dot{o}}}
\newcommand{\pdot}[0]{{\dot{p}}}
\newcommand{\qdot}[0]{{\dot{q}}}
\newcommand{\rdot}[0]{{\dot{r}}}
\newcommand{\sdot}[0]{{\dot{s}}}
\newcommand{\tdot}[0]{{\dot{t}}}
\newcommand{\udot}[0]{{\dot{u}}}
\newcommand{\vdot}[0]{{\dot{v}}}
\newcommand{\wdot}[0]{{\dot{w}}}
\newcommand{\xdot}[0]{{\dot{x}}}
\newcommand{\ydot}[0]{{\dot{y}}}
\newcommand{\zdot}[0]{{\dot{z}}}
\newcommand{\addot}[0]{{\ddot{a}}}
\newcommand{\bddot}[0]{{\ddot{b}}}
\newcommand{\cddot}[0]{{\ddot{c}}}
%\newcommand{\dddot}[0]{{\ddot{d}}}
\newcommand{\eddot}[0]{{\ddot{e}}}
\newcommand{\fddot}[0]{{\ddot{f}}}
\newcommand{\gddot}[0]{{\ddot{g}}}
\newcommand{\hddot}[0]{{\ddot{h}}}
\newcommand{\iddot}[0]{{\ddot{i}}}
\newcommand{\jddot}[0]{{\ddot{j}}}
\newcommand{\kddot}[0]{{\ddot{k}}}
\newcommand{\lddot}[0]{{\ddot{l}}}
\newcommand{\mddot}[0]{{\ddot{m}}}
\newcommand{\nddot}[0]{{\ddot{n}}}
\newcommand{\oddot}[0]{{\ddot{o}}}
\newcommand{\pddot}[0]{{\ddot{p}}}
\newcommand{\qddot}[0]{{\ddot{q}}}
\newcommand{\rddot}[0]{{\ddot{r}}}
\newcommand{\sddot}[0]{{\ddot{s}}}
\newcommand{\tddot}[0]{{\ddot{t}}}
\newcommand{\uddot}[0]{{\ddot{u}}}
\newcommand{\vddot}[0]{{\ddot{v}}}
\newcommand{\wddot}[0]{{\ddot{w}}}
\newcommand{\xddot}[0]{{\ddot{x}}}
\newcommand{\yddot}[0]{{\ddot{y}}}
\newcommand{\zddot}[0]{{\ddot{z}}}

%<bold and dot greek symbols>
%

\newcommand{\Deltadot}[0]{{\dot{\Delta}}}
\newcommand{\Gammadot}[0]{{\dot{\Gamma}}}
\newcommand{\Lambdadot}[0]{{\dot{\Lambda}}}
\newcommand{\Omegadot}[0]{{\dot{\Omega}}}
\newcommand{\Phidot}[0]{{\dot{\Phi}}}
\newcommand{\Pidot}[0]{{\dot{\Pi}}}
\newcommand{\Psidot}[0]{{\dot{\Psi}}}
\newcommand{\Sigmadot}[0]{{\dot{\Sigma}}}
\newcommand{\Thetadot}[0]{{\dot{\Theta}}}
\newcommand{\Upsilondot}[0]{{\dot{\Upsilon}}}
\newcommand{\Xidot}[0]{{\dot{\Xi}}}
\newcommand{\alphadot}[0]{{\dot{\alpha}}}
\newcommand{\betadot}[0]{{\dot{\beta}}}
\newcommand{\chidot}[0]{{\dot{\chi}}}
\newcommand{\deltadot}[0]{{\dot{\delta}}}
\newcommand{\epsilondot}[0]{{\dot{\epsilon}}}
\newcommand{\etadot}[0]{{\dot{\eta}}}
\newcommand{\gammadot}[0]{{\dot{\gamma}}}
\newcommand{\kappadot}[0]{{\dot{\kappa}}}
\newcommand{\lambdadot}[0]{{\dot{\lambda}}}
\newcommand{\mudot}[0]{{\dot{\mu}}}
\newcommand{\nudot}[0]{{\dot{\nu}}}
\newcommand{\omegadot}[0]{{\dot{\omega}}}
\newcommand{\phidot}[0]{{\dot{\phi}}}
\newcommand{\pidot}[0]{{\dot{\pi}}}
\newcommand{\psidot}[0]{{\dot{\psi}}}
\newcommand{\rhodot}[0]{{\dot{\rho}}}
\newcommand{\sigmadot}[0]{{\dot{\sigma}}}
\newcommand{\taudot}[0]{{\dot{\tau}}}
\newcommand{\thetadot}[0]{{\dot{\theta}}}
\newcommand{\upsilondot}[0]{{\dot{\upsilon}}}
\newcommand{\varepsilondot}[0]{{\dot{\varepsilon}}}
\newcommand{\varphidot}[0]{{\dot{\varphi}}}
\newcommand{\varpidot}[0]{{\dot{\varpi}}}
\newcommand{\varrhodot}[0]{{\dot{\varrho}}}
\newcommand{\varsigmadot}[0]{{\dot{\varsigma}}}
\newcommand{\varthetadot}[0]{{\dot{\vartheta}}}
\newcommand{\xidot}[0]{{\dot{\xi}}}
\newcommand{\zetadot}[0]{{\dot{\zeta}}}

\newcommand{\Deltaddot}[0]{{\ddot{\Delta}}}
\newcommand{\Gammaddot}[0]{{\ddot{\Gamma}}}
\newcommand{\Lambdaddot}[0]{{\ddot{\Lambda}}}
\newcommand{\Omegaddot}[0]{{\ddot{\Omega}}}
\newcommand{\Phiddot}[0]{{\ddot{\Phi}}}
\newcommand{\Piddot}[0]{{\ddot{\Pi}}}
\newcommand{\Psiddot}[0]{{\ddot{\Psi}}}
\newcommand{\Sigmaddot}[0]{{\ddot{\Sigma}}}
\newcommand{\Thetaddot}[0]{{\ddot{\Theta}}}
\newcommand{\Upsilonddot}[0]{{\ddot{\Upsilon}}}
\newcommand{\Xiddot}[0]{{\ddot{\Xi}}}
\newcommand{\alphaddot}[0]{{\ddot{\alpha}}}
\newcommand{\betaddot}[0]{{\ddot{\beta}}}
\newcommand{\chiddot}[0]{{\ddot{\chi}}}
\newcommand{\deltaddot}[0]{{\ddot{\delta}}}
\newcommand{\epsilonddot}[0]{{\ddot{\epsilon}}}
\newcommand{\etaddot}[0]{{\ddot{\eta}}}
\newcommand{\gammaddot}[0]{{\ddot{\gamma}}}
\newcommand{\kappaddot}[0]{{\ddot{\kappa}}}
\newcommand{\lambdaddot}[0]{{\ddot{\lambda}}}
\newcommand{\muddot}[0]{{\ddot{\mu}}}
\newcommand{\nuddot}[0]{{\ddot{\nu}}}
\newcommand{\omegaddot}[0]{{\ddot{\omega}}}
\newcommand{\phiddot}[0]{{\ddot{\phi}}}
\newcommand{\piddot}[0]{{\ddot{\pi}}}
\newcommand{\psiddot}[0]{{\ddot{\psi}}}
\newcommand{\rhoddot}[0]{{\ddot{\rho}}}
\newcommand{\sigmaddot}[0]{{\ddot{\sigma}}}
\newcommand{\tauddot}[0]{{\ddot{\tau}}}
\newcommand{\thetaddot}[0]{{\ddot{\theta}}}
\newcommand{\upsilonddot}[0]{{\ddot{\upsilon}}}
\newcommand{\varepsilonddot}[0]{{\ddot{\varepsilon}}}
\newcommand{\varphiddot}[0]{{\ddot{\varphi}}}
\newcommand{\varpiddot}[0]{{\ddot{\varpi}}}
\newcommand{\varrhoddot}[0]{{\ddot{\varrho}}}
\newcommand{\varsigmaddot}[0]{{\ddot{\varsigma}}}
\newcommand{\varthetaddot}[0]{{\ddot{\vartheta}}}
\newcommand{\xiddot}[0]{{\ddot{\xi}}}
\newcommand{\zetaddot}[0]{{\ddot{\zeta}}}

\newcommand{\BDelta}[0]{\boldsymbol{\Delta}}
\newcommand{\BGamma}[0]{\boldsymbol{\Gamma}}
\newcommand{\BLambda}[0]{\boldsymbol{\Lambda}}
\newcommand{\BOmega}[0]{\boldsymbol{\Omega}}
\newcommand{\BPhi}[0]{\boldsymbol{\Phi}}
\newcommand{\BPi}[0]{\boldsymbol{\Pi}}
\newcommand{\BPsi}[0]{\boldsymbol{\Psi}}
\newcommand{\BSigma}[0]{\boldsymbol{\Sigma}}
\newcommand{\BTheta}[0]{\boldsymbol{\Theta}}
\newcommand{\BUpsilon}[0]{\boldsymbol{\Upsilon}}
\newcommand{\BXi}[0]{\boldsymbol{\Xi}}
\newcommand{\Balpha}[0]{\boldsymbol{\alpha}}
\newcommand{\Bbeta}[0]{\boldsymbol{\beta}}
\newcommand{\Bchi}[0]{\boldsymbol{\chi}}
\newcommand{\Bdelta}[0]{\boldsymbol{\delta}}
\newcommand{\Bepsilon}[0]{\boldsymbol{\epsilon}}
\newcommand{\Beta}[0]{\boldsymbol{\eta}}
\newcommand{\Bgamma}[0]{\boldsymbol{\gamma}}
\newcommand{\Bkappa}[0]{\boldsymbol{\kappa}}
\newcommand{\Blambda}[0]{\boldsymbol{\lambda}}
\newcommand{\Bmu}[0]{\boldsymbol{\mu}}
\newcommand{\Bnu}[0]{\boldsymbol{\nu}}
%\newcommand{\Bomega}[0]{\boldsymbol{\omega}}
\newcommand{\Bphi}[0]{\boldsymbol{\phi}}
\newcommand{\Bpi}[0]{\boldsymbol{\pi}}
\newcommand{\Bpsi}[0]{\boldsymbol{\psi}}
\newcommand{\Brho}[0]{\boldsymbol{\rho}}
\newcommand{\Bsigma}[0]{\boldsymbol{\sigma}}
%\newcommand{\Btau}[0]{\boldsymbol{\tau}}
%\newcommand{\Btheta}[0]{\boldsymbol{\theta}}
\newcommand{\Bupsilon}[0]{\boldsymbol{\upsilon}}
\newcommand{\Bvarepsilon}[0]{\boldsymbol{\varepsilon}}
\newcommand{\Bvarphi}[0]{\boldsymbol{\varphi}}
\newcommand{\Bvarpi}[0]{\boldsymbol{\varpi}}
\newcommand{\Bvarrho}[0]{\boldsymbol{\varrho}}
\newcommand{\Bvarsigma}[0]{\boldsymbol{\varsigma}}
\newcommand{\Bvartheta}[0]{\boldsymbol{\vartheta}}
\newcommand{\Bxi}[0]{\boldsymbol{\xi}}
\newcommand{\Bzeta}[0]{\boldsymbol{\zeta}}
%
%</bold and dot greek symbols>
%<infrequent>
%
%\newcommand{\AreaOp}[1]{\AName_{#1}}
%\newcommand{\Babs}[0]{\abs{\BB}}
%\newcommand{\Bcap}[0]{\hat{\BB}}
%\newcommand{\BrPrimeRej}[0]{\rcap(\rcap \wedge \Br')}
%\newcommand{\CA}[0]{\mathcal{A}}
%\newcommand{\Cos}[1]{\cos{\left({#1}\right)}}
%\newcommand{\Det}[1] {\abs{#1}}
%\newcommand{\Dsq}[2] {\frac {\partial^2 {#1}} {\partial {#2}^2}}
%\newcommand{\Exp}[1]{\exp{\left({#1}\right)}}
%\newcommand{\Norm}[1]{\left\lVert{#1}\right\rVert}
%\newcommand{\Sin}[1]{\sin{\left({#1}\right)}}
%\newcommand{\T}[0]{\text{T}}
%\newcommand{\VolumeOp}[1]{\VName_{#1}}
%\newcommand{\agrad}[0]{\Ba \cdot \nabla}
%\newcommand{\alphacap}[0]{\hat{\boldsymbol{\alpha}}}
%\newcommand{\Fcap}[0]{\hat{\BF}}
%\newcommand{\bithree}[0]{{\Bi}_3}
%\newcommand{\bxa}[0]{\Bx\Ba}
%\newcommand{\coordvec}[2]{
%\newcommand{\costheta}[0]{\acap \cdot \xcap}
%\newcommand{\ddt}[1]{\ddot{#1}}
%\newcommand{\ddu}[1] {\frac {d{#1}} {du}}
%\newcommand{\dsqxj}[2] {\frac {\partial^2 {#1}} {\partial {x_{#2}}^2}}
%\newcommand{\dtheta}[1]{\frac{d {#1}}{d \theta}}
%\newcommand{\dt}[1]{\dot{#1}}
%\newcommand{\dt}[1]{\frac{d {#1}}{dt}}
%\newcommand{\dxj}[2] {\frac {\partial {#1}} {\partial {x_{#2}}}}
%\newcommand{\halfPhi}[0]{\frac{\phi}{2}}
%\newcommand{\half}[0]{\inv{2}}
%\newcommand{\inv}[1]{\frac{1}{#1}}
%\newcommand{\laplacian}[0]{\nabla^2}
%\newcommand{\matrixoftx}[3]{
%\newcommand{\nrrp}[0]{\norm{\rcap \wedge \Br'}}
%\newcommand{\oiint}{\bigcirc \hspace{-1.4em} \int \hspace{-.8em} \int}
%\newcommand{\transpose}[1]{{#1}^{\text{T}}}
%\newcommand{\transpose}[1]{{{#1}^{\TextTranspose}}}
%\newcommand{\transpose}[1]{{{#1}^{\text{T}}}}
%\newcommand{\barA}[0]{\bar{A}}
%\newcommand{\qbar}[0]{\bar{q}}
%\newcommand{\qdotbar}[0]{\dot{\bar{q}}}
%
%</infrequent>





\usepackage[bookmarks=true]{hyperref}

\usepackage{color,cite,graphicx}
   % use colour in the document, put your citations as [1-4]
   % rather than [1,2,3,4] (it looks nicer, and the extended LaTeX2e
   % graphics package. 
\usepackage{latexsym,amssymb,epsf} % don't remember if these are
   % needed, but their inclusion can't do any damage


\title{ n-sphere volume. }
\author{Peeter Joot \quad peeter.joot@gmail.com }
\date{ Feb 26, 2009.  Last Revision: $Date: 2009/02/26 23:27:03 $ }

\begin{document}

\maketitle{}
\tableofcontents

\section{ Motivation. }

In \cite{PJ4dFourier} a 4D fourier transform solution 
of Maxwell's equation yielded a Green's function of the form

\begin{align*}
G(x) = \iiiint \frac{e^{i k_\mu x^\mu}}{k_\nu k^\nu} dk_1 dk_2 dk_3 dk_4
\end{align*}

To attempt to ``evaluate'' this integral, as done in
\cite{PJpoisson}
to produce the retarded time potentials,
a hypervolume equivalent to spherical polar coordinate
parameterization is probably desirable.

Before attempting to tackle the problem of interest, the basic question
of how to do volume and weighted volume integrals over a hyperspherical volumes
must be considered.

\section{ Some hints from wikipedia. }

%\begin{figure}[htp]
%\centering
%\includegraphics[totalheight=0.4\textheight]{picturepath}
%\caption{My Caption}\label{fig:pictlabel}
%\end{figure}
%
%... see figure \ref{fig:picturepath} ...

\bibliographystyle{plainnat}
\bibliography{myrefs}

\end{document}

\include{vector_differential_identities}
\part{General Physics.}
\include{angular_acc}
\include{angular_acc_cross}
\include{ke_rotation}
\documentclass{article}

\usepackage{amsmath}
\usepackage{mathpazo}

%
% shorthand for bold symbols, convenient for vectors and matrices
%
\newcommand{\Ba}[0]{\mathbf{a}}
\newcommand{\Bb}[0]{\mathbf{b}}
\newcommand{\Bc}[0]{\mathbf{c}}
\newcommand{\Bd}[0]{\mathbf{d}}
\newcommand{\Be}[0]{\mathbf{e}}
\newcommand{\Bf}[0]{\mathbf{f}}
\newcommand{\Bg}[0]{\mathbf{g}}
\newcommand{\Bh}[0]{\mathbf{h}}
\newcommand{\Bi}[0]{\mathbf{i}}
\newcommand{\Bj}[0]{\mathbf{j}}
\newcommand{\Bk}[0]{\mathbf{k}}
\newcommand{\Bl}[0]{\mathbf{l}}
\newcommand{\Bm}[0]{\mathbf{m}}
\newcommand{\Bn}[0]{\mathbf{n}}
\newcommand{\Bo}[0]{\mathbf{o}}
\newcommand{\Bp}[0]{\mathbf{p}}
\newcommand{\Bq}[0]{\mathbf{q}}
\newcommand{\Br}[0]{\mathbf{r}}
\newcommand{\Bs}[0]{\mathbf{s}}
\newcommand{\Bt}[0]{\mathbf{t}}
\newcommand{\Bu}[0]{\mathbf{u}}
\newcommand{\Bv}[0]{\mathbf{v}}
\newcommand{\Bw}[0]{\mathbf{w}}
\newcommand{\Bx}[0]{\mathbf{x}}
\newcommand{\By}[0]{\mathbf{y}}
\newcommand{\Bz}[0]{\mathbf{z}}
\newcommand{\BA}[0]{\mathbf{A}}
\newcommand{\BB}[0]{\mathbf{B}}
\newcommand{\BC}[0]{\mathbf{C}}
\newcommand{\BD}[0]{\mathbf{D}}
\newcommand{\BE}[0]{\mathbf{E}}
\newcommand{\BF}[0]{\mathbf{F}}
\newcommand{\BG}[0]{\mathbf{G}}
\newcommand{\BH}[0]{\mathbf{H}}
\newcommand{\BI}[0]{\mathbf{I}}
\newcommand{\BJ}[0]{\mathbf{J}}
\newcommand{\BK}[0]{\mathbf{K}}
\newcommand{\BL}[0]{\mathbf{L}}
\newcommand{\BM}[0]{\mathbf{M}}
\newcommand{\BN}[0]{\mathbf{N}}
\newcommand{\BO}[0]{\mathbf{O}}
\newcommand{\BP}[0]{\mathbf{P}}
\newcommand{\BQ}[0]{\mathbf{Q}}
\newcommand{\BR}[0]{\mathbf{R}}
\newcommand{\BS}[0]{\mathbf{S}}
\newcommand{\BT}[0]{\mathbf{T}}
\newcommand{\BU}[0]{\mathbf{U}}
\newcommand{\BV}[0]{\mathbf{V}}
\newcommand{\BW}[0]{\mathbf{W}}
\newcommand{\BX}[0]{\mathbf{X}}
\newcommand{\BY}[0]{\mathbf{Y}}
\newcommand{\BZ}[0]{\mathbf{Z}}

\newcommand{\Bzero}[0]{\mathbf{0}}
\newcommand{\Btheta}[0]{\boldsymbol{\theta}}
\newcommand{\Btau}[0]{\boldsymbol{\tau}}
\newcommand{\Bomega}[0]{\boldsymbol{\omega}}

%
% shorthand for unit vectors
%
\newcommand{\acap}[0]{\hat{\Ba}}
\newcommand{\bcap}[0]{\hat{\Bb}}
\newcommand{\ccap}[0]{\hat{\Bc}}
\newcommand{\dcap}[0]{\hat{\Bd}}
\newcommand{\ecap}[0]{\hat{\Be}}
\newcommand{\fcap}[0]{\hat{\Bf}}
\newcommand{\gcap}[0]{\hat{\Bg}}
\newcommand{\hcap}[0]{\hat{\Bh}}
\newcommand{\icap}[0]{\hat{\Bi}}
\newcommand{\jcap}[0]{\hat{\Bj}}
\newcommand{\kcap}[0]{\hat{\Bk}}
\newcommand{\lcap}[0]{\hat{\Bl}}
\newcommand{\mcap}[0]{\hat{\Bm}}
\newcommand{\ncap}[0]{\hat{\Bn}}
\newcommand{\ocap}[0]{\hat{\Bo}}
\newcommand{\pcap}[0]{\hat{\Bp}}
\newcommand{\qcap}[0]{\hat{\Bq}}
\newcommand{\rcap}[0]{\hat{\Br}}
\newcommand{\scap}[0]{\hat{\Bs}}
\newcommand{\tcap}[0]{\hat{\Bt}}
\newcommand{\ucap}[0]{\hat{\Bu}}
\newcommand{\vcap}[0]{\hat{\Bv}}
\newcommand{\wcap}[0]{\hat{\Bw}}
\newcommand{\xcap}[0]{\hat{\Bx}}
\newcommand{\ycap}[0]{\hat{\By}}
\newcommand{\zcap}[0]{\hat{\Bz}}
\newcommand{\thetacap}[0]{\hat{\Btheta}}

%
% to write R^n and C^n in a distinguishable fashion.  Perhaps change this
% to the double lined characters upon figuring out how to do so.
%
\newcommand{\C}[1]{$\mathbb{C}^{#1}$}
\newcommand{\R}[1]{$\mathbb{R}^{#1}$}

%
% various generally useful helpers
%

% derivative of #1 wrt. #2:
\newcommand{\D}[2] {\frac {d#2} {d#1}}

\newcommand{\inv}[1]{\frac{1}{#1}}
\newcommand{\cross}[0]{\times}

\newcommand{\abs}[1]{\lvert{#1}\rvert}
\newcommand{\norm}[1]{\lVert{#1}\rVert}
\newcommand{\innerprod}[2]{\langle{#1}, {#2}\rangle}
\newcommand{\dotprod}[2]{{#1} \cdot {#2}}
\newcommand{\bdotprod}[2]{\left({#1} \cdot {#2}\right)}
\newcommand{\crossprod}[2]{{#1} \cross {#2}}
\newcommand{\tripleprod}[3]{\dotprod{\left(\crossprod{#1}{#2}\right)}{#3}}

\DeclareMathOperator{\Proj}{Proj}
\DeclareMathOperator{\Span}{span}
\DeclareMathOperator{\Sgn}{sgn}
\DeclareMathOperator{\Area}{Area}
\DeclareMathOperator{\Volume}{Volume}

%
% A few miscellaneous things specific to this document
%
\newcommand{\crossop}[1]{\crossprod{#1}{}}

% R2 vector.
\newcommand{\VectorTwo}[2]{
\begin{bmatrix}
 {#1} \\
 {#2}
\end{bmatrix}
}

\newcommand{\VectorN}[1]{
\begin{bmatrix}
{#1}_1 \\
{#1}_2 \\
\vdots \\
{#1}_N \\
\end{bmatrix}
}

\newcommand{\DETuvij}[4]{
\begin{vmatrix}
 {#1}_{#3} & {#1}_{#4} \\
 {#2}_{#3} & {#2}_{#4}
\end{vmatrix}
}

\newcommand{\DETuvwijk}[6]{
\begin{vmatrix}
 {#1}_{#4} & {#1}_{#5} & {#1}_{#6} \\
 {#2}_{#4} & {#2}_{#5} & {#2}_{#6} \\
 {#3}_{#4} & {#3}_{#5} & {#3}_{#6}
\end{vmatrix}
}

\newcommand{\DETuvwxijkl}[8]{
\begin{vmatrix}
 {#1}_{#5} & {#1}_{#6} & {#1}_{#7} & {#1}_{#8} \\
 {#2}_{#5} & {#2}_{#6} & {#2}_{#7} & {#2}_{#8} \\
 {#3}_{#5} & {#3}_{#6} & {#3}_{#7} & {#3}_{#8} \\
 {#4}_{#5} & {#4}_{#6} & {#4}_{#7} & {#4}_{#8} \\
\end{vmatrix}
}

%\newcommand{\DETuvwxyijklm}[10]{
%\begin{vmatrix}
% {#1}_{#6} & {#1}_{#7} & {#1}_{#8} & {#1}_{#9} & {#1}_{#10} \\
% {#2}_{#6} & {#2}_{#7} & {#2}_{#8} & {#2}_{#9} & {#2}_{#10} \\
% {#3}_{#6} & {#3}_{#7} & {#3}_{#8} & {#3}_{#9} & {#3}_{#10} \\
% {#4}_{#6} & {#4}_{#7} & {#4}_{#8} & {#4}_{#9} & {#4}_{#10} \\
% {#5}_{#6} & {#5}_{#7} & {#5}_{#8} & {#5}_{#9} & {#5}_{#10}
%\end{vmatrix}
%}

% R3 vector.
\newcommand{\VectorThree}[3]{
\begin{bmatrix}
 {#1} \\
 {#2} \\
 {#3}
\end{bmatrix}
}


%<misc>
%
\newcommand{\Abs}[1]{{\left\lvert{#1}\right\rvert}}
\newcommand{\spacegrad}[0]{\boldsymbol{\nabla}}
\newcommand{\grad}[0]{\nabla}
\newcommand{\LL}[0]{\mathcal{L}}

% == \partial_{#1} {#2}
\newcommand{\PD}[2]{\frac{\partial {#2}}{\partial {#1}}}
% inline variant
\newcommand{\PDi}[2]{{\partial {#2}}/{\partial {#1}}}

\newcommand{\PDD}[3]{\frac{\partial^2 {#3}}{\partial {#1}\partial {#2}}}
%\newcommand{\PDd}[2]{\frac{\partial^2 {#2}}{{\partial{#1}}^2}}
\newcommand{\PDsq}[2]{\frac{\partial^2 {#2}}{(\partial {#1})^2}}

\newcommand{\Partial}[2]{\frac{\partial {#1}}{\partial {#2}}}
\DeclareMathOperator{\RejName}{Rej}
\newcommand{\Rej}[2]{\RejName_{#1}\left( {#2} \right)}
\newcommand{\Rm}[1]{\mathbb{R}^{#1}}
\newcommand{\Cm}[1]{\mathbb{C}^{#1}}
\newcommand{\conj}[0]{{*}}

%</misc>

% <grade selection>
%
\newcommand{\gpgrade}[2] {{\left\langle{{#1}}\right\rangle}_{#2}}

\newcommand{\gpgradezero}[1] {\gpgrade{#1}{}}
%\newcommand{\gpscalargrade}[1] {{\left\langle{{#1}}\right\rangle}}
%\newcommand{\gpgradezero}[1] {\gpgrade{#1}{0}}

%\newcommand{\gpgradeone}[1] {{\left\langle{{#1}}\right\rangle}_{1}}
\newcommand{\gpgradeone}[1] {\gpgrade{#1}{1}}

\newcommand{\gpgradetwo}[1] {\gpgrade{#1}{2}}
\newcommand{\gpgradethree}[1] {\gpgrade{#1}{3}}
\newcommand{\gpgradefour}[1] {\gpgrade{#1}{4}}
%
% </grade selection>



\newcommand{\adot}[0]{{\dot{a}}}
\newcommand{\bdot}[0]{{\dot{b}}}
% taken for centered dot:
%\newcommand{\cdot}[0]{{\dot{c}}}
%\newcommand{\ddot}[0]{{\dot{d}}}
\newcommand{\edot}[0]{{\dot{e}}}
\newcommand{\fdot}[0]{{\dot{f}}}
\newcommand{\gdot}[0]{{\dot{g}}}
\newcommand{\hdot}[0]{{\dot{h}}}
\newcommand{\idot}[0]{{\dot{i}}}
\newcommand{\jdot}[0]{{\dot{j}}}
\newcommand{\kdot}[0]{{\dot{k}}}
\newcommand{\ldot}[0]{{\dot{l}}}
\newcommand{\mdot}[0]{{\dot{m}}}
\newcommand{\ndot}[0]{{\dot{n}}}
%\newcommand{\odot}[0]{{\dot{o}}}
\newcommand{\pdot}[0]{{\dot{p}}}
\newcommand{\qdot}[0]{{\dot{q}}}
\newcommand{\rdot}[0]{{\dot{r}}}
\newcommand{\sdot}[0]{{\dot{s}}}
\newcommand{\tdot}[0]{{\dot{t}}}
\newcommand{\udot}[0]{{\dot{u}}}
\newcommand{\vdot}[0]{{\dot{v}}}
\newcommand{\wdot}[0]{{\dot{w}}}
\newcommand{\xdot}[0]{{\dot{x}}}
\newcommand{\ydot}[0]{{\dot{y}}}
\newcommand{\zdot}[0]{{\dot{z}}}
\newcommand{\addot}[0]{{\ddot{a}}}
\newcommand{\bddot}[0]{{\ddot{b}}}
\newcommand{\cddot}[0]{{\ddot{c}}}
%\newcommand{\dddot}[0]{{\ddot{d}}}
\newcommand{\eddot}[0]{{\ddot{e}}}
\newcommand{\fddot}[0]{{\ddot{f}}}
\newcommand{\gddot}[0]{{\ddot{g}}}
\newcommand{\hddot}[0]{{\ddot{h}}}
\newcommand{\iddot}[0]{{\ddot{i}}}
\newcommand{\jddot}[0]{{\ddot{j}}}
\newcommand{\kddot}[0]{{\ddot{k}}}
\newcommand{\lddot}[0]{{\ddot{l}}}
\newcommand{\mddot}[0]{{\ddot{m}}}
\newcommand{\nddot}[0]{{\ddot{n}}}
\newcommand{\oddot}[0]{{\ddot{o}}}
\newcommand{\pddot}[0]{{\ddot{p}}}
\newcommand{\qddot}[0]{{\ddot{q}}}
\newcommand{\rddot}[0]{{\ddot{r}}}
\newcommand{\sddot}[0]{{\ddot{s}}}
\newcommand{\tddot}[0]{{\ddot{t}}}
\newcommand{\uddot}[0]{{\ddot{u}}}
\newcommand{\vddot}[0]{{\ddot{v}}}
\newcommand{\wddot}[0]{{\ddot{w}}}
\newcommand{\xddot}[0]{{\ddot{x}}}
\newcommand{\yddot}[0]{{\ddot{y}}}
\newcommand{\zddot}[0]{{\ddot{z}}}

%<bold and dot greek symbols>
%

\newcommand{\Deltadot}[0]{{\dot{\Delta}}}
\newcommand{\Gammadot}[0]{{\dot{\Gamma}}}
\newcommand{\Lambdadot}[0]{{\dot{\Lambda}}}
\newcommand{\Omegadot}[0]{{\dot{\Omega}}}
\newcommand{\Phidot}[0]{{\dot{\Phi}}}
\newcommand{\Pidot}[0]{{\dot{\Pi}}}
\newcommand{\Psidot}[0]{{\dot{\Psi}}}
\newcommand{\Sigmadot}[0]{{\dot{\Sigma}}}
\newcommand{\Thetadot}[0]{{\dot{\Theta}}}
\newcommand{\Upsilondot}[0]{{\dot{\Upsilon}}}
\newcommand{\Xidot}[0]{{\dot{\Xi}}}
\newcommand{\alphadot}[0]{{\dot{\alpha}}}
\newcommand{\betadot}[0]{{\dot{\beta}}}
\newcommand{\chidot}[0]{{\dot{\chi}}}
\newcommand{\deltadot}[0]{{\dot{\delta}}}
\newcommand{\epsilondot}[0]{{\dot{\epsilon}}}
\newcommand{\etadot}[0]{{\dot{\eta}}}
\newcommand{\gammadot}[0]{{\dot{\gamma}}}
\newcommand{\kappadot}[0]{{\dot{\kappa}}}
\newcommand{\lambdadot}[0]{{\dot{\lambda}}}
\newcommand{\mudot}[0]{{\dot{\mu}}}
\newcommand{\nudot}[0]{{\dot{\nu}}}
\newcommand{\omegadot}[0]{{\dot{\omega}}}
\newcommand{\phidot}[0]{{\dot{\phi}}}
\newcommand{\pidot}[0]{{\dot{\pi}}}
\newcommand{\psidot}[0]{{\dot{\psi}}}
\newcommand{\rhodot}[0]{{\dot{\rho}}}
\newcommand{\sigmadot}[0]{{\dot{\sigma}}}
\newcommand{\taudot}[0]{{\dot{\tau}}}
\newcommand{\thetadot}[0]{{\dot{\theta}}}
\newcommand{\upsilondot}[0]{{\dot{\upsilon}}}
\newcommand{\varepsilondot}[0]{{\dot{\varepsilon}}}
\newcommand{\varphidot}[0]{{\dot{\varphi}}}
\newcommand{\varpidot}[0]{{\dot{\varpi}}}
\newcommand{\varrhodot}[0]{{\dot{\varrho}}}
\newcommand{\varsigmadot}[0]{{\dot{\varsigma}}}
\newcommand{\varthetadot}[0]{{\dot{\vartheta}}}
\newcommand{\xidot}[0]{{\dot{\xi}}}
\newcommand{\zetadot}[0]{{\dot{\zeta}}}

\newcommand{\Deltaddot}[0]{{\ddot{\Delta}}}
\newcommand{\Gammaddot}[0]{{\ddot{\Gamma}}}
\newcommand{\Lambdaddot}[0]{{\ddot{\Lambda}}}
\newcommand{\Omegaddot}[0]{{\ddot{\Omega}}}
\newcommand{\Phiddot}[0]{{\ddot{\Phi}}}
\newcommand{\Piddot}[0]{{\ddot{\Pi}}}
\newcommand{\Psiddot}[0]{{\ddot{\Psi}}}
\newcommand{\Sigmaddot}[0]{{\ddot{\Sigma}}}
\newcommand{\Thetaddot}[0]{{\ddot{\Theta}}}
\newcommand{\Upsilonddot}[0]{{\ddot{\Upsilon}}}
\newcommand{\Xiddot}[0]{{\ddot{\Xi}}}
\newcommand{\alphaddot}[0]{{\ddot{\alpha}}}
\newcommand{\betaddot}[0]{{\ddot{\beta}}}
\newcommand{\chiddot}[0]{{\ddot{\chi}}}
\newcommand{\deltaddot}[0]{{\ddot{\delta}}}
\newcommand{\epsilonddot}[0]{{\ddot{\epsilon}}}
\newcommand{\etaddot}[0]{{\ddot{\eta}}}
\newcommand{\gammaddot}[0]{{\ddot{\gamma}}}
\newcommand{\kappaddot}[0]{{\ddot{\kappa}}}
\newcommand{\lambdaddot}[0]{{\ddot{\lambda}}}
\newcommand{\muddot}[0]{{\ddot{\mu}}}
\newcommand{\nuddot}[0]{{\ddot{\nu}}}
\newcommand{\omegaddot}[0]{{\ddot{\omega}}}
\newcommand{\phiddot}[0]{{\ddot{\phi}}}
\newcommand{\piddot}[0]{{\ddot{\pi}}}
\newcommand{\psiddot}[0]{{\ddot{\psi}}}
\newcommand{\rhoddot}[0]{{\ddot{\rho}}}
\newcommand{\sigmaddot}[0]{{\ddot{\sigma}}}
\newcommand{\tauddot}[0]{{\ddot{\tau}}}
\newcommand{\thetaddot}[0]{{\ddot{\theta}}}
\newcommand{\upsilonddot}[0]{{\ddot{\upsilon}}}
\newcommand{\varepsilonddot}[0]{{\ddot{\varepsilon}}}
\newcommand{\varphiddot}[0]{{\ddot{\varphi}}}
\newcommand{\varpiddot}[0]{{\ddot{\varpi}}}
\newcommand{\varrhoddot}[0]{{\ddot{\varrho}}}
\newcommand{\varsigmaddot}[0]{{\ddot{\varsigma}}}
\newcommand{\varthetaddot}[0]{{\ddot{\vartheta}}}
\newcommand{\xiddot}[0]{{\ddot{\xi}}}
\newcommand{\zetaddot}[0]{{\ddot{\zeta}}}

\newcommand{\BDelta}[0]{\boldsymbol{\Delta}}
\newcommand{\BGamma}[0]{\boldsymbol{\Gamma}}
\newcommand{\BLambda}[0]{\boldsymbol{\Lambda}}
\newcommand{\BOmega}[0]{\boldsymbol{\Omega}}
\newcommand{\BPhi}[0]{\boldsymbol{\Phi}}
\newcommand{\BPi}[0]{\boldsymbol{\Pi}}
\newcommand{\BPsi}[0]{\boldsymbol{\Psi}}
\newcommand{\BSigma}[0]{\boldsymbol{\Sigma}}
\newcommand{\BTheta}[0]{\boldsymbol{\Theta}}
\newcommand{\BUpsilon}[0]{\boldsymbol{\Upsilon}}
\newcommand{\BXi}[0]{\boldsymbol{\Xi}}
\newcommand{\Balpha}[0]{\boldsymbol{\alpha}}
\newcommand{\Bbeta}[0]{\boldsymbol{\beta}}
\newcommand{\Bchi}[0]{\boldsymbol{\chi}}
\newcommand{\Bdelta}[0]{\boldsymbol{\delta}}
\newcommand{\Bepsilon}[0]{\boldsymbol{\epsilon}}
\newcommand{\Beta}[0]{\boldsymbol{\eta}}
\newcommand{\Bgamma}[0]{\boldsymbol{\gamma}}
\newcommand{\Bkappa}[0]{\boldsymbol{\kappa}}
\newcommand{\Blambda}[0]{\boldsymbol{\lambda}}
\newcommand{\Bmu}[0]{\boldsymbol{\mu}}
\newcommand{\Bnu}[0]{\boldsymbol{\nu}}
%\newcommand{\Bomega}[0]{\boldsymbol{\omega}}
\newcommand{\Bphi}[0]{\boldsymbol{\phi}}
\newcommand{\Bpi}[0]{\boldsymbol{\pi}}
\newcommand{\Bpsi}[0]{\boldsymbol{\psi}}
\newcommand{\Brho}[0]{\boldsymbol{\rho}}
\newcommand{\Bsigma}[0]{\boldsymbol{\sigma}}
%\newcommand{\Btau}[0]{\boldsymbol{\tau}}
%\newcommand{\Btheta}[0]{\boldsymbol{\theta}}
\newcommand{\Bupsilon}[0]{\boldsymbol{\upsilon}}
\newcommand{\Bvarepsilon}[0]{\boldsymbol{\varepsilon}}
\newcommand{\Bvarphi}[0]{\boldsymbol{\varphi}}
\newcommand{\Bvarpi}[0]{\boldsymbol{\varpi}}
\newcommand{\Bvarrho}[0]{\boldsymbol{\varrho}}
\newcommand{\Bvarsigma}[0]{\boldsymbol{\varsigma}}
\newcommand{\Bvartheta}[0]{\boldsymbol{\vartheta}}
\newcommand{\Bxi}[0]{\boldsymbol{\xi}}
\newcommand{\Bzeta}[0]{\boldsymbol{\zeta}}
%
%</bold and dot greek symbols>
%<infrequent>
%
%\newcommand{\AreaOp}[1]{\AName_{#1}}
%\newcommand{\Babs}[0]{\abs{\BB}}
%\newcommand{\Bcap}[0]{\hat{\BB}}
%\newcommand{\BrPrimeRej}[0]{\rcap(\rcap \wedge \Br')}
%\newcommand{\CA}[0]{\mathcal{A}}
%\newcommand{\Cos}[1]{\cos{\left({#1}\right)}}
%\newcommand{\Det}[1] {\abs{#1}}
%\newcommand{\Dsq}[2] {\frac {\partial^2 {#1}} {\partial {#2}^2}}
%\newcommand{\Exp}[1]{\exp{\left({#1}\right)}}
%\newcommand{\Norm}[1]{\left\lVert{#1}\right\rVert}
%\newcommand{\Sin}[1]{\sin{\left({#1}\right)}}
%\newcommand{\T}[0]{\text{T}}
%\newcommand{\VolumeOp}[1]{\VName_{#1}}
%\newcommand{\agrad}[0]{\Ba \cdot \nabla}
%\newcommand{\alphacap}[0]{\hat{\boldsymbol{\alpha}}}
%\newcommand{\Fcap}[0]{\hat{\BF}}
%\newcommand{\bithree}[0]{{\Bi}_3}
%\newcommand{\bxa}[0]{\Bx\Ba}
%\newcommand{\coordvec}[2]{
%\newcommand{\costheta}[0]{\acap \cdot \xcap}
%\newcommand{\ddt}[1]{\ddot{#1}}
%\newcommand{\ddu}[1] {\frac {d{#1}} {du}}
%\newcommand{\dsqxj}[2] {\frac {\partial^2 {#1}} {\partial {x_{#2}}^2}}
%\newcommand{\dtheta}[1]{\frac{d {#1}}{d \theta}}
%\newcommand{\dt}[1]{\dot{#1}}
%\newcommand{\dt}[1]{\frac{d {#1}}{dt}}
%\newcommand{\dxj}[2] {\frac {\partial {#1}} {\partial {x_{#2}}}}
%\newcommand{\halfPhi}[0]{\frac{\phi}{2}}
%\newcommand{\half}[0]{\inv{2}}
%\newcommand{\inv}[1]{\frac{1}{#1}}
%\newcommand{\laplacian}[0]{\nabla^2}
%\newcommand{\matrixoftx}[3]{
%\newcommand{\nrrp}[0]{\norm{\rcap \wedge \Br'}}
%\newcommand{\oiint}{\bigcirc \hspace{-1.4em} \int \hspace{-.8em} \int}
%\newcommand{\transpose}[1]{{#1}^{\text{T}}}
%\newcommand{\transpose}[1]{{{#1}^{\TextTranspose}}}
%\newcommand{\transpose}[1]{{{#1}^{\text{T}}}}
%\newcommand{\barA}[0]{\bar{A}}
%\newcommand{\qbar}[0]{\bar{q}}
%\newcommand{\qdotbar}[0]{\dot{\bar{q}}}
%
%</infrequent>





\usepackage[bookmarks=true]{hyperref}

\usepackage{color,cite,graphicx}
   % use colour in the document, put your citations as [1-4]
   % rather than [1,2,3,4] (it looks nicer, and the extended LaTeX2e
   % graphics package. 
\usepackage{latexsym,amssymb,epsf} % don't remember if these are
   % needed, but their inclusion can't do any damage


\title{ polar velocity and accerlation. }
\author{Peeter Joot}
\date{ Jan 13, 2009.  Last Revision: $Date: 2009/01/13 23:25:03 $ }

\begin{document}

\maketitle{}
%\tableofcontents

\section{ Motivation. }

Have previously worked out the radial velocity and acceleration components a pile of different ways in
\cite{PJAngAcc}, 
\cite{PJAngAccCross}, 
\cite{PJAngVel}, 
\cite{PJKeRot}, 
\cite{PJRadialDer}, and
\cite{PJUnitDer}.

So, what's a couple more?

When the motion is strictly restricted to a plane we can get away with doing this either in complex numbers
(used in a number of the Tong Lagrangian solutions), or with a polar form \R{2} vector (a polar representation
I haven't seen since High School).

\section{ With complex numbers. }

Let
\begin{align*}
z = r e^{i\theta}
\end{align*}

So our velocity is

\begin{align*}
\zdot = \rdot e^{i\theta} + i r \thetadot e^{i\theta}
\end{align*}

and the acceleration is
\begin{align*}
\ddot{z}
&= \ddot{r} e^{i\theta} + i \dot{r} \thetadot e^{i\theta}
 + i \rdot \thetadot e^{i\theta}
 + i r \ddot{\theta} e^{i\theta}
 - r \thetadot^2 e^{i\theta} \\
&= (\ddot{r} - r \thetadot^2 ) e^{i\theta} + (2 \dot{r} \thetadot + r \ddot{\theta} ) i e^{i\theta}
\end{align*}

\section{ Plane vector representation. }

Also can do this with polar vector representation directly (without involving the complexity of rotation matrixes or anything fancy)

\begin{align*}
\Br 
&= r 
\begin{bmatrix}
\cos\theta \\
\sin\theta
\end{bmatrix}
\end{align*}

Velocity is then
\begin{align*}
\Bv 
&= 
\rdot 
\begin{bmatrix}
\cos\theta \\
\sin\theta
\end{bmatrix}
+r \thetadot
\begin{bmatrix}
-\sin\theta \\
\cos\theta
\end{bmatrix}
\end{align*}

and for acceleration we have

\begin{align*}
\Ba 
&= 
\ddot{r}
\begin{bmatrix}
\cos\theta \\
\sin\theta
\end{bmatrix}
+\rdot \thetadot
\begin{bmatrix}
-\sin\theta \\
\cos\theta
\end{bmatrix}
+\rdot \thetadot
\begin{bmatrix}
-\sin\theta \\
\cos\theta
\end{bmatrix}
+r \ddot{\theta}
\begin{bmatrix}
-\sin\theta \\
\cos\theta
\end{bmatrix}
-r \thetadot^2
\begin{bmatrix}
\cos\theta \\
\sin\theta 
\end{bmatrix} \\
&=
(\ddot{r} -r \thetadot^2)
\begin{bmatrix}
\cos\theta \\
\sin\theta
\end{bmatrix}
+(2\rdot \thetadot +r \ddot{\theta})
\begin{bmatrix}
-\sin\theta \\
\cos\theta
\end{bmatrix}
\end{align*}

\bibliographystyle{plainnat}
\bibliography{myrefs}

\end{document}

\documentclass{article}      % Specifies the document class

\usepackage{amsmath}

%
% shorthand for bold symbols, convenient for vectors and matrices
%
\newcommand{\Ba}[0]{\mathbf{a}}
\newcommand{\Bb}[0]{\mathbf{b}}
\newcommand{\Bc}[0]{\mathbf{c}}
\newcommand{\Bd}[0]{\mathbf{d}}
\newcommand{\Be}[0]{\mathbf{e}}
\newcommand{\Bf}[0]{\mathbf{f}}
\newcommand{\Bg}[0]{\mathbf{g}}
\newcommand{\Bh}[0]{\mathbf{h}}
\newcommand{\Bi}[0]{\mathbf{i}}
\newcommand{\Bj}[0]{\mathbf{j}}
\newcommand{\Bk}[0]{\mathbf{k}}
\newcommand{\Bl}[0]{\mathbf{l}}
\newcommand{\Bm}[0]{\mathbf{m}}
\newcommand{\Bn}[0]{\mathbf{n}}
\newcommand{\Bo}[0]{\mathbf{o}}
\newcommand{\Bp}[0]{\mathbf{p}}
\newcommand{\Bq}[0]{\mathbf{q}}
\newcommand{\Br}[0]{\mathbf{r}}
\newcommand{\Bs}[0]{\mathbf{s}}
\newcommand{\Bt}[0]{\mathbf{t}}
\newcommand{\Bu}[0]{\mathbf{u}}
\newcommand{\Bv}[0]{\mathbf{v}}
\newcommand{\Bw}[0]{\mathbf{w}}
\newcommand{\Bx}[0]{\mathbf{x}}
\newcommand{\By}[0]{\mathbf{y}}
\newcommand{\Bz}[0]{\mathbf{z}}
\newcommand{\BA}[0]{\mathbf{A}}
\newcommand{\BB}[0]{\mathbf{B}}
\newcommand{\BC}[0]{\mathbf{C}}
\newcommand{\BD}[0]{\mathbf{D}}
\newcommand{\BE}[0]{\mathbf{E}}
\newcommand{\BF}[0]{\mathbf{F}}
\newcommand{\BG}[0]{\mathbf{G}}
\newcommand{\BH}[0]{\mathbf{H}}
\newcommand{\BI}[0]{\mathbf{I}}
\newcommand{\BJ}[0]{\mathbf{J}}
\newcommand{\BK}[0]{\mathbf{K}}
\newcommand{\BL}[0]{\mathbf{L}}
\newcommand{\BM}[0]{\mathbf{M}}
\newcommand{\BN}[0]{\mathbf{N}}
\newcommand{\BO}[0]{\mathbf{O}}
\newcommand{\BP}[0]{\mathbf{P}}
\newcommand{\BQ}[0]{\mathbf{Q}}
\newcommand{\BR}[0]{\mathbf{R}}
\newcommand{\BS}[0]{\mathbf{S}}
\newcommand{\BT}[0]{\mathbf{T}}
\newcommand{\BU}[0]{\mathbf{U}}
\newcommand{\BV}[0]{\mathbf{V}}
\newcommand{\BW}[0]{\mathbf{W}}
\newcommand{\BX}[0]{\mathbf{X}}
\newcommand{\BY}[0]{\mathbf{Y}}
\newcommand{\BZ}[0]{\mathbf{Z}}

\newcommand{\Bzero}[0]{\mathbf{0}}
\newcommand{\Btheta}[0]{\boldsymbol{\theta}}
\newcommand{\Btau}[0]{\boldsymbol{\tau}}
\newcommand{\Bomega}[0]{\boldsymbol{\omega}}

%
% shorthand for unit vectors
%
\newcommand{\acap}[0]{\hat{\Ba}}
\newcommand{\bcap}[0]{\hat{\Bb}}
\newcommand{\ccap}[0]{\hat{\Bc}}
\newcommand{\dcap}[0]{\hat{\Bd}}
\newcommand{\ecap}[0]{\hat{\Be}}
\newcommand{\fcap}[0]{\hat{\Bf}}
\newcommand{\gcap}[0]{\hat{\Bg}}
\newcommand{\hcap}[0]{\hat{\Bh}}
\newcommand{\icap}[0]{\hat{\Bi}}
\newcommand{\jcap}[0]{\hat{\Bj}}
\newcommand{\kcap}[0]{\hat{\Bk}}
\newcommand{\lcap}[0]{\hat{\Bl}}
\newcommand{\mcap}[0]{\hat{\Bm}}
\newcommand{\ncap}[0]{\hat{\Bn}}
\newcommand{\ocap}[0]{\hat{\Bo}}
\newcommand{\pcap}[0]{\hat{\Bp}}
\newcommand{\qcap}[0]{\hat{\Bq}}
\newcommand{\rcap}[0]{\hat{\Br}}
\newcommand{\scap}[0]{\hat{\Bs}}
\newcommand{\tcap}[0]{\hat{\Bt}}
\newcommand{\ucap}[0]{\hat{\Bu}}
\newcommand{\vcap}[0]{\hat{\Bv}}
\newcommand{\wcap}[0]{\hat{\Bw}}
\newcommand{\xcap}[0]{\hat{\Bx}}
\newcommand{\ycap}[0]{\hat{\By}}
\newcommand{\zcap}[0]{\hat{\Bz}}
\newcommand{\thetacap}[0]{\hat{\Btheta}}

%
% to write R^n and C^n in a distinguishable fashion.  Perhaps change this
% to the double lined characters upon figuring out how to do so.
%
\newcommand{\C}[1]{${\BC}^{#1}$}
\newcommand{\R}[1]{${\BR}^{#1}$}

%
% various generally useful helpers
%

% derivative of #1 wrt. #2:
\newcommand{\D}[2] {\frac {d#2} {d#1}}

\newcommand{\inv}[1]{\frac{1}{#1}}
\newcommand{\cross}[0]{\times}

\newcommand{\abs}[1]{\lvert#1\rvert}
\newcommand{\norm}[1]{\lVert#1\rVert}
\newcommand{\innerprod}[2]{\langle{#1}, {#2}\rangle}
\newcommand{\dotprod}[2]{#1 \cdot #2}
\newcommand{\crossprod}[2]{#1 \cross #2}
\newcommand{\tripleprod}[3]{\dotprod{\crossprod{#1}{#2}}{#3}}

%
% A few miscellaneous things specific to this document
%
\newcommand{\crossop}[1]{\crossprod{#1}{}}

\newcommand{\PDP}[2]{\BP^{#1}\BD{\BP^{#2}}}
\newcommand{\PDPDP}[3]{\Bv^T\BP^{#1}\BD\BP^{#2}\BD\BP^{#3}\Bv}

\newcommand{\Mp}[0]{
\begin{bmatrix}
0 & 1 & 0 & 0 \\
0 & 0 & 1 & 0 \\
0 & 0 & 0 & 1 \\
1 & 0 & 0 & 0
\end{bmatrix}
}
\newcommand{\Mpp}[0]{
\begin{bmatrix}
0 & 0 & 1 & 0 \\
0 & 0 & 0 & 1 \\
1 & 0 & 0 & 0 \\
0 & 1 & 0 & 0
\end{bmatrix}
}
\newcommand{\Mppp}[0]{
\begin{bmatrix}
0 & 0 & 0 & 1 \\
1 & 0 & 0 & 0 \\
0 & 1 & 0 & 0 \\
0 & 0 & 1 & 0
\end{bmatrix}
}
\newcommand{\Mpu}[0]{
\begin{bmatrix}
u_1 & 0 & 0 & 0 \\
0 & u_2 & 0 & 0 \\
0 & 0 & u_3 & 0 \\
0 & 0 & 0 & u_4
\end{bmatrix}
}

%
% The real thing:
%

                             % The preamble begins here.
\title{ Covariant/vector derivative notes, plus notes on raised and lowered indexes.  }
\author{Peeter Joot}         % Declares the author's name.

%\date{}        % Deleting this command produces today's date.

\begin{document}             % End of preamble and beginning of text.

\maketitle{}

\section{Motivation.}

My notes on tensors, mostly from Geometric Algebra for Physicists.  Write up enough notes for myself that I can understand the topics (if I can't explain to myself I don't understand sufficiently).  Conclude with the
solution of problem 6.1 to demonstrate the frame independence of the
covariant derivative.

\subsection{ Raised and lowered indexes. Coordinates of vectors with non-orthonormal frames. }

Let $\{ e_i \}$ represent a frame of not necessarily orthonormal basis vectors for a metric space, and $\{ e^i \}$ represent the reciprocal frame.

The reciprocal frame vectors are defined by the relation:

\begin{equation}
e_i \cdot e^j = {\delta_i}^j.
\end{equation}

Lets compute the coordinates of a vector $x$ in terms of both frames:

\[
x = \sum \alpha_j e_j = \sum \beta_j e^j
\]

Forming $x \cdot e^i$, and $x \cdot e_i$ respectively solves for the $\alpha$, and $\beta$ coefficients

\[
x \cdot e^i = \sum \alpha_j e_j \cdot e^i = \sum \alpha_j {\delta_j}^i = \alpha_i
\]

\[
x \cdot e_i = \sum \beta_j e^j \cdot e_i = \sum \beta_j {\delta_i}^j = \beta_i
\]

Thus, the reciprocal frame vectors allow for simple determination of coordinates for an arbitrary frame. We can summarize this as follows:

\[
x = \sum ( x \cdot e^i ) e_i = \sum ( x \cdot e_i ) e^i
\]

Now, for orthonormal frames, where $e_i = e^i$ we are used to writing:

\[
x = \sum x_i e_i,
\]

however for non-orthonormal frames the convention is to mix raised and lowered indexes as follows:

\[
x = \sum x^i e_i = \sum x_i e^i.
\]

Where, as demonstrated above these generalized coordinates have the values, $x^i = x \cdot e^i$, and $x_i = x \cdot e_i$.  This is a strange seeming notation at
first especially since most of linear algebra is done with always lowered (or always upper for some authors) indexes.  However one quickly gets used to it, after seeing how powerful the reciprocal frame concept is for dealing with non-orthonormal frames (otherwise one has to drag along matrices and their inverses to express the same vector decompositions).

\subsection{ Metric tensor. }

It is customary in tensor formulations of physics to utilize a metric tensor to express the dot product.

Compute the dot product using the coordinate vectors

\[
x \cdot y = \left(\sum x^i e_i \right)\left(\sum y^j e_j \right) = \sum x^i y^j \left( e_i \cdot e_j \right)
\]

\[
x \cdot y = \left(\sum x_i e^i \right)\left(\sum y_j e^j \right) = \sum x_i y_j \left( e^i \cdot e^j \right)
\]

Introducing second rank (symmetric) tensors for the dot product pairs $ e_i \cdot e_j = g_{ij}$, and $ g^{ij} = e^i \cdot e^j $ we have

\[
x \cdot y = \sum x_i y_j g^{ij} = \sum x^i y^j g_{ij} = \sum x_i y^i = \sum x^i y_i
\]

We see that the metric tensor provides a way to specify the dot product in index notation, and removes the explicit references to the original frame vectors.  Mixed indexes also removes the references to the original frame vectors, but additionally eliminates the need for either of the metric tensors.

Note that it is also common to see Einstein summation convention employed, which omits the $\sum$:

\[
x \cdot y = x_i y_j g^{ij} = x^i y^j g_{ij} = x^i y_i = x_i y^i
\]

Here summation over all matched upper, lower index pairs is implied.

\subsection{ Metric tensor relations to coordinates. }

Given a coordinate expression of a vector, we dot that with the frame vectors to observe the relation between coordinates and the metric tensor:

\[
x \cdot e_i = \sum x^j e_j \cdot e_i = \sum x^j g_{ij}
\]

\[
x \cdot e^i = \sum x_j e^j \cdot e^i = \sum x_j g^{ij}
\]

The metric tensors can therefore be used be used to express the relations between the upper and lower index coordinates:

\begin{align}
x_i &= \sum g_{ij} x^j \label{eqn:metric_upper_to_lower} \\
x^i &= \sum g^{ij} x_j \label{eqn:metric_lower_to_upper}
\end{align}

It is therefore apparent that the matrix of the index lowered metric tensor $g_{ij}$ is the inverse of the matrix for the raised index metric tensor $g^{ij}$.

\subsection{ Metric tensor as a Jacobian }

The relations of equations \ref{eqn:metric_upper_to_lower}, and \ref{eqn:metric_lower_to_upper} show that the metric tensor can be expressed in terms of partial derivatives:

\begin{align}
\frac{\partial x_i }{\partial x^j } &= g_{ij} \\
\frac{\partial x^i }{\partial x_j } &= g^{ij}
\end{align}

Therefore the metric tensors can also be expressed as Jacobian matrices (not Jacobian determinants) :

\begin{align}
g_{ij} &= \frac{\partial (x_1, \cdots, x_n) }{\partial (x^1, \cdots, x^n) } \\
g^{ij} &= \frac{\partial (x^1, \cdots, x^n) }{\partial (x_1, \cdots, x_n) }
\end{align}

Will this be useful in any way?

\subsection{ Change of basis. }

To perform a change of basis from one non-orthonormal basis $\{e_i\}$ to a second $\{f_i\}$, relations between the sets of vectors
are required.  Using Greek indexes for the $f$ frame, and English for the $e$ frame, those are:

\begin{align*}
e_i 		&= \sum f^{\mu} e_i \cdot f_{\mu} 	= \sum f_{\mu} e_i \cdot f^{\mu} \\
f_{\alpha} 	&= \sum e^k f_{\alpha} \cdot e_k 	= \sum e_k f_{\alpha} \cdot e^k \\
e^i 		&= \sum f^{\mu} e^i \cdot f_{\mu} 	= \sum f_{\mu} e^i \cdot f^{\mu} \\
f^{\alpha} 	&= \sum e^k f^{\alpha} \cdot e_k 	= \sum e_k f^{\alpha} \cdot e^k 
\end{align*}

Following GAFP we can write the dot product terms as a second order tensors $f$ (ie: matrix relation) :

\begin{align*}
e_i 		&= \sum f^{\mu} f_{i\mu}  	= \sum f_{\mu} {f_i}^{\mu} \\
f_{\alpha} 	&= \sum e^k f_{k\alpha} 	= \sum e_k {f^k}_{\alpha} \\
e^i 		&= \sum f^{\mu} {f^i}_{\mu} 	= \sum f_{\mu} f^{i \mu} \\
f^{\alpha} 	&= \sum e^k {f_k}^{\alpha}  	= \sum e_k f^{k\alpha}
\end{align*}

Looking at these relations in pairs, such as

\begin{align*}
f_{\alpha} 	&= \sum e^k f_{k\alpha} \\ 
e^i 		&= \sum f_{\mu} f^{i \mu} 
\end{align*}

and 

\begin{align*}
e_i 		&= \sum f^{\mu} f_{i\mu} \\
f^{\alpha} 	&= \sum e_k f^{k\alpha}
\end{align*}

It is clear that $f_{i\alpha}$ is the inverse of $f^{i\alpha}$.  FIXME: write this out explicitly in index notation, to specify
more exactly the inverse relationship ... that will help clarify the covariant derivative stuff later.  There are also inverse relationships for the mixed index tensors above.  Can those be used in the covariant derivative calculation to simplify things?

Note that all these various tensors are related to each other using the metric tensors for $f$ and $e$.  FIXME: show example.  Also note that using this notation the metric tensors $g_{ij}$ and $g_{\alpha\beta}$ are two completely different linear functions, and careful use of the index conventions are required to keep these straight.

\subsection{ Covariant derivative. }

GAFP exercise 6.1.  Show that the vector derivative:

\begin{equation}
\nabla = \sum e^i \frac{\partial}{\partial x^i}
\end{equation}

is not frame dependent.

To show this we will need to utilize the chain rule to rewrite the partials in terms of the alternate frame:

\begin{align*}
\frac{\partial}{\partial x^i} &= \sum \frac{\partial x^\alpha}{\partial x^i} \frac{\partial}{\partial x^\alpha} 
\end{align*}

To evaluate the first partial here, we write the coordinates of a vector in terms of both, and take dot products:

\begin{align*}
\left(\sum x^{\gamma} f_{\gamma}\right) \cdot f^{\alpha} = \left(\sum x^i e_i\right) \cdot f^{\alpha} \\
\end{align*}
\begin{align*}
x^{\alpha} &= \sum x^i {f_i}^{\alpha} \\
\end{align*}
\begin{align*}
\frac{\partial x^{\alpha}}{\partial x^i} &= {f_i}^{\alpha}
\end{align*}

Similar expressions for the other change of basis tensors is also possible, but
not required for this problem.

With this result we have the partial reexpressed in terms of coordinates
in the new frame.

\begin{align*}
\frac{\partial}{\partial x^i} &= \sum {f_i}^{\alpha} \frac{\partial}{\partial x^\alpha} 
\end{align*}

Combine this with the alternate contravariant frame vector as calculated above:

\[
e^i = \sum f^{\mu} {f^i}_{\mu}
\]

and we have:

\begin{align*}
\sum_i e^i \frac{\partial}{\partial x^i}
&= \sum_i \left(\sum_{\mu} f^{\mu} {f^i}_{\mu} \right) \left( \sum_{\alpha} {f_i}^{\alpha} \frac{\partial}{\partial x^\alpha}\right) \\
&= \sum_{\mu \alpha} \left(f^{\mu} \frac{\partial}{\partial x^\alpha} \right) \sum_i {f^i}_{\mu} {f_i}^{\alpha} \\
&= \sum_{\mu \alpha} \left(f^{\mu} \frac{\partial}{\partial x^\alpha} \right) {\delta_{\mu}}^{\alpha} \\
&= \sum_{\alpha} f^{\alpha} \frac{\partial}{\partial x^\alpha} \\
\end{align*}

FIXME: Proper justification of the step $\sum_i {f^i}_{\mu} {f_i}^{\alpha} = {\delta_{\mu}}^{\alpha}$ is missing.  This is possible due
to the imprecisely noted inverse relationships pointed out above.

Note that my original paper derivation of the above used only the tensors $f_{i\alpha}$, and $f^{i\alpha}$ instead of the mixed index versions used here.  That worked but also required a pair of metric tensors, and one more step to sum over those tensors to get at the final result.

\end{document}               % End of document.

\include{inertial_tensor}
\include{locate_satellite}
\part{Lagrangian Topics}
\include{newtonian_lagrangian_and_gradient}
\include{canonical_momentum}
\include{euler_lagrange}
\include{noethers_lorentz_force}
\part{Relativity.}
\documentclass{article}      % Specifies the document class

\usepackage{amsmath}
\usepackage{mathpazo}

%
% shorthand for bold symbols, convenient for vectors and matrices
%
\newcommand{\Ba}[0]{\mathbf{a}}
\newcommand{\Bb}[0]{\mathbf{b}}
\newcommand{\Bc}[0]{\mathbf{c}}
\newcommand{\Bd}[0]{\mathbf{d}}
\newcommand{\Be}[0]{\mathbf{e}}
\newcommand{\Bf}[0]{\mathbf{f}}
\newcommand{\Bg}[0]{\mathbf{g}}
\newcommand{\Bh}[0]{\mathbf{h}}
\newcommand{\Bi}[0]{\mathbf{i}}
\newcommand{\Bj}[0]{\mathbf{j}}
\newcommand{\Bk}[0]{\mathbf{k}}
\newcommand{\Bl}[0]{\mathbf{l}}
\newcommand{\Bm}[0]{\mathbf{m}}
\newcommand{\Bn}[0]{\mathbf{n}}
\newcommand{\Bo}[0]{\mathbf{o}}
\newcommand{\Bp}[0]{\mathbf{p}}
\newcommand{\Bq}[0]{\mathbf{q}}
\newcommand{\Br}[0]{\mathbf{r}}
\newcommand{\Bs}[0]{\mathbf{s}}
\newcommand{\Bt}[0]{\mathbf{t}}
\newcommand{\Bu}[0]{\mathbf{u}}
\newcommand{\Bv}[0]{\mathbf{v}}
\newcommand{\Bw}[0]{\mathbf{w}}
\newcommand{\Bx}[0]{\mathbf{x}}
\newcommand{\By}[0]{\mathbf{y}}
\newcommand{\Bz}[0]{\mathbf{z}}
\newcommand{\BA}[0]{\mathbf{A}}
\newcommand{\BB}[0]{\mathbf{B}}
\newcommand{\BC}[0]{\mathbf{C}}
\newcommand{\BD}[0]{\mathbf{D}}
\newcommand{\BE}[0]{\mathbf{E}}
\newcommand{\BF}[0]{\mathbf{F}}
\newcommand{\BG}[0]{\mathbf{G}}
\newcommand{\BH}[0]{\mathbf{H}}
\newcommand{\BI}[0]{\mathbf{I}}
\newcommand{\BJ}[0]{\mathbf{J}}
\newcommand{\BK}[0]{\mathbf{K}}
\newcommand{\BL}[0]{\mathbf{L}}
\newcommand{\BM}[0]{\mathbf{M}}
\newcommand{\BN}[0]{\mathbf{N}}
\newcommand{\BO}[0]{\mathbf{O}}
\newcommand{\BP}[0]{\mathbf{P}}
\newcommand{\BQ}[0]{\mathbf{Q}}
\newcommand{\BR}[0]{\mathbf{R}}
\newcommand{\BS}[0]{\mathbf{S}}
\newcommand{\BT}[0]{\mathbf{T}}
\newcommand{\BU}[0]{\mathbf{U}}
\newcommand{\BV}[0]{\mathbf{V}}
\newcommand{\BW}[0]{\mathbf{W}}
\newcommand{\BX}[0]{\mathbf{X}}
\newcommand{\BY}[0]{\mathbf{Y}}
\newcommand{\BZ}[0]{\mathbf{Z}}

\newcommand{\Bzero}[0]{\mathbf{0}}
\newcommand{\Btheta}[0]{\boldsymbol{\theta}}
\newcommand{\Btau}[0]{\boldsymbol{\tau}}
\newcommand{\Bomega}[0]{\boldsymbol{\omega}}

%
% shorthand for unit vectors
%
\newcommand{\acap}[0]{\hat{\Ba}}
\newcommand{\bcap}[0]{\hat{\Bb}}
\newcommand{\ccap}[0]{\hat{\Bc}}
\newcommand{\dcap}[0]{\hat{\Bd}}
\newcommand{\ecap}[0]{\hat{\Be}}
\newcommand{\fcap}[0]{\hat{\Bf}}
\newcommand{\gcap}[0]{\hat{\Bg}}
\newcommand{\hcap}[0]{\hat{\Bh}}
\newcommand{\icap}[0]{\hat{\Bi}}
\newcommand{\jcap}[0]{\hat{\Bj}}
\newcommand{\kcap}[0]{\hat{\Bk}}
\newcommand{\lcap}[0]{\hat{\Bl}}
\newcommand{\mcap}[0]{\hat{\Bm}}
\newcommand{\ncap}[0]{\hat{\Bn}}
\newcommand{\ocap}[0]{\hat{\Bo}}
\newcommand{\pcap}[0]{\hat{\Bp}}
\newcommand{\qcap}[0]{\hat{\Bq}}
\newcommand{\rcap}[0]{\hat{\Br}}
\newcommand{\scap}[0]{\hat{\Bs}}
\newcommand{\tcap}[0]{\hat{\Bt}}
\newcommand{\ucap}[0]{\hat{\Bu}}
\newcommand{\vcap}[0]{\hat{\Bv}}
\newcommand{\wcap}[0]{\hat{\Bw}}
\newcommand{\xcap}[0]{\hat{\Bx}}
\newcommand{\ycap}[0]{\hat{\By}}
\newcommand{\zcap}[0]{\hat{\Bz}}
\newcommand{\thetacap}[0]{\hat{\Btheta}}

%
% to write R^n and C^n in a distinguishable fashion.  Perhaps change this
% to the double lined characters upon figuring out how to do so.
%
\newcommand{\C}[1]{$\mathbb{C}^{#1}$}
\newcommand{\R}[1]{$\mathbb{R}^{#1}$}

%
% various generally useful helpers
%

% derivative of #1 wrt. #2:
\newcommand{\D}[2] {\frac {d#2} {d#1}}

\newcommand{\inv}[1]{\frac{1}{#1}}
\newcommand{\cross}[0]{\times}

\newcommand{\abs}[1]{\lvert{#1}\rvert}
\newcommand{\norm}[1]{\lVert{#1}\rVert}
\newcommand{\innerprod}[2]{\langle{#1}, {#2}\rangle}
\newcommand{\dotprod}[2]{{#1} \cdot {#2}}
\newcommand{\bdotprod}[2]{\left({#1} \cdot {#2}\right)}
\newcommand{\crossprod}[2]{{#1} \cross {#2}}
\newcommand{\tripleprod}[3]{\dotprod{\left(\crossprod{#1}{#2}\right)}{#3}}

\DeclareMathOperator{\Proj}{Proj}
\DeclareMathOperator{\Span}{span}
\DeclareMathOperator{\Sgn}{sgn}
\DeclareMathOperator{\Area}{Area}
\DeclareMathOperator{\Volume}{Volume}

%
% A few miscellaneous things specific to this document
%
\newcommand{\crossop}[1]{\crossprod{#1}{}}

% R2 vector.
\newcommand{\VectorTwo}[2]{
\begin{bmatrix}
 {#1} \\
 {#2}
\end{bmatrix}
}

\newcommand{\VectorN}[1]{
\begin{bmatrix}
{#1}_1 \\
{#1}_2 \\
\vdots \\
{#1}_N \\
\end{bmatrix}
}

\newcommand{\DETuvij}[4]{
\begin{vmatrix}
 {#1}_{#3} & {#1}_{#4} \\
 {#2}_{#3} & {#2}_{#4}
\end{vmatrix}
}

\newcommand{\DETuvwijk}[6]{
\begin{vmatrix}
 {#1}_{#4} & {#1}_{#5} & {#1}_{#6} \\
 {#2}_{#4} & {#2}_{#5} & {#2}_{#6} \\
 {#3}_{#4} & {#3}_{#5} & {#3}_{#6}
\end{vmatrix}
}

\newcommand{\DETuvwxijkl}[8]{
\begin{vmatrix}
 {#1}_{#5} & {#1}_{#6} & {#1}_{#7} & {#1}_{#8} \\
 {#2}_{#5} & {#2}_{#6} & {#2}_{#7} & {#2}_{#8} \\
 {#3}_{#5} & {#3}_{#6} & {#3}_{#7} & {#3}_{#8} \\
 {#4}_{#5} & {#4}_{#6} & {#4}_{#7} & {#4}_{#8} \\
\end{vmatrix}
}

%\newcommand{\DETuvwxyijklm}[10]{
%\begin{vmatrix}
% {#1}_{#6} & {#1}_{#7} & {#1}_{#8} & {#1}_{#9} & {#1}_{#10} \\
% {#2}_{#6} & {#2}_{#7} & {#2}_{#8} & {#2}_{#9} & {#2}_{#10} \\
% {#3}_{#6} & {#3}_{#7} & {#3}_{#8} & {#3}_{#9} & {#3}_{#10} \\
% {#4}_{#6} & {#4}_{#7} & {#4}_{#8} & {#4}_{#9} & {#4}_{#10} \\
% {#5}_{#6} & {#5}_{#7} & {#5}_{#8} & {#5}_{#9} & {#5}_{#10}
%\end{vmatrix}
%}

% R3 vector.
\newcommand{\VectorThree}[3]{
\begin{bmatrix}
 {#1} \\
 {#2} \\
 {#3}
\end{bmatrix}
}



\newcommand{\laplacian}[0]{\nabla^2}

%
% The real thing:
%

                             % The preamble begins here.
\title{Derive Lorentz transformation from wave equation.} % Declares the document's title.
\author{Peeter Joot}         % Declares the author's name.
%\date{}        % Deleting this command produces today's date.

\begin{document}             % End of preamble and beginning of text.

\maketitle{}

\section{intro.}

Many introductory relativity texts mention how Lorentz observed that 
while Maxwells equations were not invarient with respect to Galelian
transformation, they were with his modified transformation.

I found it interesting to consider this statement with a bit of detail.

\section{}

From Maxwell's equations one can show that the electric field and magnetic field both satisfy the wave equation:

\begin{equation}
\laplacian - \inv{c^2}\frac{\partial^2}{\partial t^2} = 0 
\end{equation}

The wikipedia article Electromagnetic radiation (under Derivation)

%\htmladdnormallink{<URL>} { http://en.wikipedia.org/wiki/Electromagnetic_radiation#Derivation }

goes over this nicely.

Although this can be solved separately for either $\BE$ or $\BB$ the two are not independent.
This dependence is nicely expressed by writing the electromagnetic field as a complete
bivector $\BF = \BE + i c \BB$, and in that form the 
general solution to this equation for the combined electromagnetic
field is:

\begin{equation}
\BF = (\BE_0 + \kcap \wedge \BE_0) f( \kcap \cdot \Br \pm c t)
\end{equation}

Here f is any function, and represents the amplitude of the waveform.

\section{Verifing Lorentz invarience.}

The Lorentz transfrom for a moving (primed) frame where the motion is
along the x axis is:

\begin{equation*}
\begin{bmatrix}
x' \\
c t' \\
\end{bmatrix}
=
\gamma
\begin{bmatrix}
1 & -\beta \\
-\beta & 1 \\
\end{bmatrix}
\end{equation*}

Or,
\begin{equation*}
\begin{bmatrix}
x \\
c t \\
\end{bmatrix}
=
\gamma
\begin{bmatrix}
1 & \beta \\
\beta & 1 \\
\end{bmatrix}
\end{equation*}

Using this we can express the partials of the wave equation in the 
primed frame.  Starting with the first derivatives:

\begin{align*}
\frac{\partial}{\partial x} 
&= \frac{\partial x'}{\partial x} \frac{\partial}{\partial x'} + \frac{\partial c t'}{\partial x} \frac{\partial}{\partial c t'} \\
&= \gamma \frac{\partial}{\partial x'} - \gamma \beta \frac{\partial}{\partial c t'} \\
\end{align*}

And:

\begin{align*}
\frac{\partial}{\partial ct} 
&= \frac{\partial x'}{\partial ct} \frac{\partial}{\partial x'} + \frac{\partial c t'}{\partial ct} \frac{\partial}{\partial c t'} \\
&= -\beta \gamma \frac{\partial}{\partial x'} + \gamma \frac{\partial}{\partial c t'} \\
\end{align*}

...

\section{ Derive Lorentz Transformation requiring invarience of the wave equation. }

\end{document}               % End of document.

\include{em_bivector_metric_dependencies}
\include{mass_vary_lagrangian}
\include{velocity_tx}
\include{fourvec_dotinvariance}
\documentclass{article}      % Specifies the document class

\usepackage{amsmath}
\usepackage{mathpazo}

%
% shorthand for bold symbols, convenient for vectors and matrices
%
\newcommand{\Ba}[0]{\mathbf{a}}
\newcommand{\Bb}[0]{\mathbf{b}}
\newcommand{\Bc}[0]{\mathbf{c}}
\newcommand{\Bd}[0]{\mathbf{d}}
\newcommand{\Be}[0]{\mathbf{e}}
\newcommand{\Bf}[0]{\mathbf{f}}
\newcommand{\Bg}[0]{\mathbf{g}}
\newcommand{\Bh}[0]{\mathbf{h}}
\newcommand{\Bi}[0]{\mathbf{i}}
\newcommand{\Bj}[0]{\mathbf{j}}
\newcommand{\Bk}[0]{\mathbf{k}}
\newcommand{\Bl}[0]{\mathbf{l}}
\newcommand{\Bm}[0]{\mathbf{m}}
\newcommand{\Bn}[0]{\mathbf{n}}
\newcommand{\Bo}[0]{\mathbf{o}}
\newcommand{\Bp}[0]{\mathbf{p}}
\newcommand{\Bq}[0]{\mathbf{q}}
\newcommand{\Br}[0]{\mathbf{r}}
\newcommand{\Bs}[0]{\mathbf{s}}
\newcommand{\Bt}[0]{\mathbf{t}}
\newcommand{\Bu}[0]{\mathbf{u}}
\newcommand{\Bv}[0]{\mathbf{v}}
\newcommand{\Bw}[0]{\mathbf{w}}
\newcommand{\Bx}[0]{\mathbf{x}}
\newcommand{\By}[0]{\mathbf{y}}
\newcommand{\Bz}[0]{\mathbf{z}}
\newcommand{\BA}[0]{\mathbf{A}}
\newcommand{\BB}[0]{\mathbf{B}}
\newcommand{\BC}[0]{\mathbf{C}}
\newcommand{\BD}[0]{\mathbf{D}}
\newcommand{\BE}[0]{\mathbf{E}}
\newcommand{\BF}[0]{\mathbf{F}}
\newcommand{\BG}[0]{\mathbf{G}}
\newcommand{\BH}[0]{\mathbf{H}}
\newcommand{\BI}[0]{\mathbf{I}}
\newcommand{\BJ}[0]{\mathbf{J}}
\newcommand{\BK}[0]{\mathbf{K}}
\newcommand{\BL}[0]{\mathbf{L}}
\newcommand{\BM}[0]{\mathbf{M}}
\newcommand{\BN}[0]{\mathbf{N}}
\newcommand{\BO}[0]{\mathbf{O}}
\newcommand{\BP}[0]{\mathbf{P}}
\newcommand{\BQ}[0]{\mathbf{Q}}
\newcommand{\BR}[0]{\mathbf{R}}
\newcommand{\BS}[0]{\mathbf{S}}
\newcommand{\BT}[0]{\mathbf{T}}
\newcommand{\BU}[0]{\mathbf{U}}
\newcommand{\BV}[0]{\mathbf{V}}
\newcommand{\BW}[0]{\mathbf{W}}
\newcommand{\BX}[0]{\mathbf{X}}
\newcommand{\BY}[0]{\mathbf{Y}}
\newcommand{\BZ}[0]{\mathbf{Z}}

\newcommand{\Bzero}[0]{\mathbf{0}}
\newcommand{\Btheta}[0]{\boldsymbol{\theta}}
\newcommand{\Btau}[0]{\boldsymbol{\tau}}
\newcommand{\Bomega}[0]{\boldsymbol{\omega}}

%
% shorthand for unit vectors
%
\newcommand{\acap}[0]{\hat{\Ba}}
\newcommand{\bcap}[0]{\hat{\Bb}}
\newcommand{\ccap}[0]{\hat{\Bc}}
\newcommand{\dcap}[0]{\hat{\Bd}}
\newcommand{\ecap}[0]{\hat{\Be}}
\newcommand{\fcap}[0]{\hat{\Bf}}
\newcommand{\gcap}[0]{\hat{\Bg}}
\newcommand{\hcap}[0]{\hat{\Bh}}
\newcommand{\icap}[0]{\hat{\Bi}}
\newcommand{\jcap}[0]{\hat{\Bj}}
\newcommand{\kcap}[0]{\hat{\Bk}}
\newcommand{\lcap}[0]{\hat{\Bl}}
\newcommand{\mcap}[0]{\hat{\Bm}}
\newcommand{\ncap}[0]{\hat{\Bn}}
\newcommand{\ocap}[0]{\hat{\Bo}}
\newcommand{\pcap}[0]{\hat{\Bp}}
\newcommand{\qcap}[0]{\hat{\Bq}}
\newcommand{\rcap}[0]{\hat{\Br}}
\newcommand{\scap}[0]{\hat{\Bs}}
\newcommand{\tcap}[0]{\hat{\Bt}}
\newcommand{\ucap}[0]{\hat{\Bu}}
\newcommand{\vcap}[0]{\hat{\Bv}}
\newcommand{\wcap}[0]{\hat{\Bw}}
\newcommand{\xcap}[0]{\hat{\Bx}}
\newcommand{\ycap}[0]{\hat{\By}}
\newcommand{\zcap}[0]{\hat{\Bz}}
\newcommand{\thetacap}[0]{\hat{\Btheta}}

%
% to write R^n and C^n in a distinguishable fashion.  Perhaps change this
% to the double lined characters upon figuring out how to do so.
%
\newcommand{\C}[1]{$\mathbb{C}^{#1}$}
\newcommand{\R}[1]{$\mathbb{R}^{#1}$}

%
% various generally useful helpers
%

% derivative of #1 wrt. #2:
\newcommand{\D}[2] {\frac {d#2} {d#1}}

\newcommand{\inv}[1]{\frac{1}{#1}}
\newcommand{\cross}[0]{\times}

\newcommand{\abs}[1]{\lvert{#1}\rvert}
\newcommand{\norm}[1]{\lVert{#1}\rVert}
\newcommand{\innerprod}[2]{\langle{#1}, {#2}\rangle}
\newcommand{\dotprod}[2]{{#1} \cdot {#2}}
\newcommand{\bdotprod}[2]{\left({#1} \cdot {#2}\right)}
\newcommand{\crossprod}[2]{{#1} \cross {#2}}
\newcommand{\tripleprod}[3]{\dotprod{\left(\crossprod{#1}{#2}\right)}{#3}}

\DeclareMathOperator{\Proj}{Proj}
\DeclareMathOperator{\Span}{span}
\DeclareMathOperator{\Sgn}{sgn}
\DeclareMathOperator{\Area}{Area}
\DeclareMathOperator{\Volume}{Volume}

%
% A few miscellaneous things specific to this document
%
\newcommand{\crossop}[1]{\crossprod{#1}{}}

% R2 vector.
\newcommand{\VectorTwo}[2]{
\begin{bmatrix}
 {#1} \\
 {#2}
\end{bmatrix}
}

\newcommand{\VectorN}[1]{
\begin{bmatrix}
{#1}_1 \\
{#1}_2 \\
\vdots \\
{#1}_N \\
\end{bmatrix}
}

\newcommand{\DETuvij}[4]{
\begin{vmatrix}
 {#1}_{#3} & {#1}_{#4} \\
 {#2}_{#3} & {#2}_{#4}
\end{vmatrix}
}

\newcommand{\DETuvwijk}[6]{
\begin{vmatrix}
 {#1}_{#4} & {#1}_{#5} & {#1}_{#6} \\
 {#2}_{#4} & {#2}_{#5} & {#2}_{#6} \\
 {#3}_{#4} & {#3}_{#5} & {#3}_{#6}
\end{vmatrix}
}

\newcommand{\DETuvwxijkl}[8]{
\begin{vmatrix}
 {#1}_{#5} & {#1}_{#6} & {#1}_{#7} & {#1}_{#8} \\
 {#2}_{#5} & {#2}_{#6} & {#2}_{#7} & {#2}_{#8} \\
 {#3}_{#5} & {#3}_{#6} & {#3}_{#7} & {#3}_{#8} \\
 {#4}_{#5} & {#4}_{#6} & {#4}_{#7} & {#4}_{#8} \\
\end{vmatrix}
}

%\newcommand{\DETuvwxyijklm}[10]{
%\begin{vmatrix}
% {#1}_{#6} & {#1}_{#7} & {#1}_{#8} & {#1}_{#9} & {#1}_{#10} \\
% {#2}_{#6} & {#2}_{#7} & {#2}_{#8} & {#2}_{#9} & {#2}_{#10} \\
% {#3}_{#6} & {#3}_{#7} & {#3}_{#8} & {#3}_{#9} & {#3}_{#10} \\
% {#4}_{#6} & {#4}_{#7} & {#4}_{#8} & {#4}_{#9} & {#4}_{#10} \\
% {#5}_{#6} & {#5}_{#7} & {#5}_{#8} & {#5}_{#9} & {#5}_{#10}
%\end{vmatrix}
%}

% R3 vector.
\newcommand{\VectorThree}[3]{
\begin{bmatrix}
 {#1} \\
 {#2} \\
 {#3}
\end{bmatrix}
}


\newcommand{\spacegrad}[0]{\boldsymbol{\nabla}}
\newcommand{\grad}[0]{\nabla}

%
% The real thing:
%

                             % The preamble begins here.
\title{ Lorentz transformation of spacetime gradient }
\author{Peeter Joot}         % Declares the author's name.
%\date{}        % Deleting this command produces today's date.

\begin{document}             % End of preamble and beginning of text.

\maketitle{}

\section{ Motivation. }

We have observed that the wave equation is Lorentz invarient, and conversely that invarience of the form of the wave equation under linear transformation for light can be used to calculate the Lorentz transformation.  Specifically, this means that we require the equations of light (wave equation) retain its form after a change of variables that includes
a (possibly scaled) translation.  The wave equation should have no mixed partial terms, and retain the form:

\begin{equation*}
(\spacegrad^2 - \partial_{ct}^2) F = ({\spacegrad'}^2 - \partial_{ct'}^2) F = 0
\end{equation*}

Having expressed the spacetime gradient with a (STA) Minkowski basis, and knowing that the Maxwell equation written using the spacetime gradient is Lorentz invarient:

\begin{equation*}
\grad F = J,
\end{equation*}

we therefore expect that the square root of the wave equation (Laplacian) operator is also Lorentz invarient.  Here this idea is explored, and we look at how the spacetime
gradient behaves under Lorentz transformation.

Our spacetime gradient is

\begin{equation*}
\grad = \sum \gamma^{\mu} \frac{\partial}{\partial x^{\mu}}
\end{equation*}

Under Lorentz transformation we can transform the $x^1=x$, and $x^0 = ct$ coordinates:

\begin{equation*}
\begin{bmatrix}
x' \\
ct' \\
\end{bmatrix}
=
\gamma
\begin{bmatrix}
1 & -\beta \\
-\beta & 1 \\
\end{bmatrix}
\begin{bmatrix}
x \\
ct \\
\end{bmatrix}
\end{equation*}

%\begin{equation*}
%\end{equation*}
\end{document}               % End of document.

\documentclass{article}

\usepackage{amsmath}
\usepackage{mathpazo}

%
% shorthand for bold symbols, convenient for vectors and matrices
%
\newcommand{\Ba}[0]{\mathbf{a}}
\newcommand{\Bb}[0]{\mathbf{b}}
\newcommand{\Bc}[0]{\mathbf{c}}
\newcommand{\Bd}[0]{\mathbf{d}}
\newcommand{\Be}[0]{\mathbf{e}}
\newcommand{\Bf}[0]{\mathbf{f}}
\newcommand{\Bg}[0]{\mathbf{g}}
\newcommand{\Bh}[0]{\mathbf{h}}
\newcommand{\Bi}[0]{\mathbf{i}}
\newcommand{\Bj}[0]{\mathbf{j}}
\newcommand{\Bk}[0]{\mathbf{k}}
\newcommand{\Bl}[0]{\mathbf{l}}
\newcommand{\Bm}[0]{\mathbf{m}}
\newcommand{\Bn}[0]{\mathbf{n}}
\newcommand{\Bo}[0]{\mathbf{o}}
\newcommand{\Bp}[0]{\mathbf{p}}
\newcommand{\Bq}[0]{\mathbf{q}}
\newcommand{\Br}[0]{\mathbf{r}}
\newcommand{\Bs}[0]{\mathbf{s}}
\newcommand{\Bt}[0]{\mathbf{t}}
\newcommand{\Bu}[0]{\mathbf{u}}
\newcommand{\Bv}[0]{\mathbf{v}}
\newcommand{\Bw}[0]{\mathbf{w}}
\newcommand{\Bx}[0]{\mathbf{x}}
\newcommand{\By}[0]{\mathbf{y}}
\newcommand{\Bz}[0]{\mathbf{z}}
\newcommand{\BA}[0]{\mathbf{A}}
\newcommand{\BB}[0]{\mathbf{B}}
\newcommand{\BC}[0]{\mathbf{C}}
\newcommand{\BD}[0]{\mathbf{D}}
\newcommand{\BE}[0]{\mathbf{E}}
\newcommand{\BF}[0]{\mathbf{F}}
\newcommand{\BG}[0]{\mathbf{G}}
\newcommand{\BH}[0]{\mathbf{H}}
\newcommand{\BI}[0]{\mathbf{I}}
\newcommand{\BJ}[0]{\mathbf{J}}
\newcommand{\BK}[0]{\mathbf{K}}
\newcommand{\BL}[0]{\mathbf{L}}
\newcommand{\BM}[0]{\mathbf{M}}
\newcommand{\BN}[0]{\mathbf{N}}
\newcommand{\BO}[0]{\mathbf{O}}
\newcommand{\BP}[0]{\mathbf{P}}
\newcommand{\BQ}[0]{\mathbf{Q}}
\newcommand{\BR}[0]{\mathbf{R}}
\newcommand{\BS}[0]{\mathbf{S}}
\newcommand{\BT}[0]{\mathbf{T}}
\newcommand{\BU}[0]{\mathbf{U}}
\newcommand{\BV}[0]{\mathbf{V}}
\newcommand{\BW}[0]{\mathbf{W}}
\newcommand{\BX}[0]{\mathbf{X}}
\newcommand{\BY}[0]{\mathbf{Y}}
\newcommand{\BZ}[0]{\mathbf{Z}}

\newcommand{\Bzero}[0]{\mathbf{0}}
\newcommand{\Btheta}[0]{\boldsymbol{\theta}}
\newcommand{\Btau}[0]{\boldsymbol{\tau}}
\newcommand{\Bomega}[0]{\boldsymbol{\omega}}

%
% shorthand for unit vectors
%
\newcommand{\acap}[0]{\hat{\Ba}}
\newcommand{\bcap}[0]{\hat{\Bb}}
\newcommand{\ccap}[0]{\hat{\Bc}}
\newcommand{\dcap}[0]{\hat{\Bd}}
\newcommand{\ecap}[0]{\hat{\Be}}
\newcommand{\fcap}[0]{\hat{\Bf}}
\newcommand{\gcap}[0]{\hat{\Bg}}
\newcommand{\hcap}[0]{\hat{\Bh}}
\newcommand{\icap}[0]{\hat{\Bi}}
\newcommand{\jcap}[0]{\hat{\Bj}}
\newcommand{\kcap}[0]{\hat{\Bk}}
\newcommand{\lcap}[0]{\hat{\Bl}}
\newcommand{\mcap}[0]{\hat{\Bm}}
\newcommand{\ncap}[0]{\hat{\Bn}}
\newcommand{\ocap}[0]{\hat{\Bo}}
\newcommand{\pcap}[0]{\hat{\Bp}}
\newcommand{\qcap}[0]{\hat{\Bq}}
\newcommand{\rcap}[0]{\hat{\Br}}
\newcommand{\scap}[0]{\hat{\Bs}}
\newcommand{\tcap}[0]{\hat{\Bt}}
\newcommand{\ucap}[0]{\hat{\Bu}}
\newcommand{\vcap}[0]{\hat{\Bv}}
\newcommand{\wcap}[0]{\hat{\Bw}}
\newcommand{\xcap}[0]{\hat{\Bx}}
\newcommand{\ycap}[0]{\hat{\By}}
\newcommand{\zcap}[0]{\hat{\Bz}}
\newcommand{\thetacap}[0]{\hat{\Btheta}}

%
% to write R^n and C^n in a distinguishable fashion.  Perhaps change this
% to the double lined characters upon figuring out how to do so.
%
\newcommand{\C}[1]{$\mathbb{C}^{#1}$}
\newcommand{\R}[1]{$\mathbb{R}^{#1}$}

%
% various generally useful helpers
%

% derivative of #1 wrt. #2:
\newcommand{\D}[2] {\frac {d#2} {d#1}}

\newcommand{\inv}[1]{\frac{1}{#1}}
\newcommand{\cross}[0]{\times}

\newcommand{\abs}[1]{\lvert{#1}\rvert}
\newcommand{\norm}[1]{\lVert{#1}\rVert}
\newcommand{\innerprod}[2]{\langle{#1}, {#2}\rangle}
\newcommand{\dotprod}[2]{{#1} \cdot {#2}}
\newcommand{\bdotprod}[2]{\left({#1} \cdot {#2}\right)}
\newcommand{\crossprod}[2]{{#1} \cross {#2}}
\newcommand{\tripleprod}[3]{\dotprod{\left(\crossprod{#1}{#2}\right)}{#3}}

\DeclareMathOperator{\Proj}{Proj}
\DeclareMathOperator{\Span}{span}
\DeclareMathOperator{\Sgn}{sgn}
\DeclareMathOperator{\Area}{Area}
\DeclareMathOperator{\Volume}{Volume}

%
% A few miscellaneous things specific to this document
%
\newcommand{\crossop}[1]{\crossprod{#1}{}}

% R2 vector.
\newcommand{\VectorTwo}[2]{
\begin{bmatrix}
 {#1} \\
 {#2}
\end{bmatrix}
}

\newcommand{\VectorN}[1]{
\begin{bmatrix}
{#1}_1 \\
{#1}_2 \\
\vdots \\
{#1}_N \\
\end{bmatrix}
}

\newcommand{\DETuvij}[4]{
\begin{vmatrix}
 {#1}_{#3} & {#1}_{#4} \\
 {#2}_{#3} & {#2}_{#4}
\end{vmatrix}
}

\newcommand{\DETuvwijk}[6]{
\begin{vmatrix}
 {#1}_{#4} & {#1}_{#5} & {#1}_{#6} \\
 {#2}_{#4} & {#2}_{#5} & {#2}_{#6} \\
 {#3}_{#4} & {#3}_{#5} & {#3}_{#6}
\end{vmatrix}
}

\newcommand{\DETuvwxijkl}[8]{
\begin{vmatrix}
 {#1}_{#5} & {#1}_{#6} & {#1}_{#7} & {#1}_{#8} \\
 {#2}_{#5} & {#2}_{#6} & {#2}_{#7} & {#2}_{#8} \\
 {#3}_{#5} & {#3}_{#6} & {#3}_{#7} & {#3}_{#8} \\
 {#4}_{#5} & {#4}_{#6} & {#4}_{#7} & {#4}_{#8} \\
\end{vmatrix}
}

%\newcommand{\DETuvwxyijklm}[10]{
%\begin{vmatrix}
% {#1}_{#6} & {#1}_{#7} & {#1}_{#8} & {#1}_{#9} & {#1}_{#10} \\
% {#2}_{#6} & {#2}_{#7} & {#2}_{#8} & {#2}_{#9} & {#2}_{#10} \\
% {#3}_{#6} & {#3}_{#7} & {#3}_{#8} & {#3}_{#9} & {#3}_{#10} \\
% {#4}_{#6} & {#4}_{#7} & {#4}_{#8} & {#4}_{#9} & {#4}_{#10} \\
% {#5}_{#6} & {#5}_{#7} & {#5}_{#8} & {#5}_{#9} & {#5}_{#10}
%\end{vmatrix}
%}

% R3 vector.
\newcommand{\VectorThree}[3]{
\begin{bmatrix}
 {#1} \\
 {#2} \\
 {#3}
\end{bmatrix}
}


%<misc>
%
\newcommand{\Abs}[1]{{\left\lvert{#1}\right\rvert}}
\newcommand{\spacegrad}[0]{\boldsymbol{\nabla}}
\newcommand{\grad}[0]{\nabla}
\newcommand{\LL}[0]{\mathcal{L}}

% == \partial_{#1} {#2}
\newcommand{\PD}[2]{\frac{\partial {#2}}{\partial {#1}}}
% inline variant
\newcommand{\PDi}[2]{{\partial {#2}}/{\partial {#1}}}

\newcommand{\PDD}[3]{\frac{\partial^2 {#3}}{\partial {#1}\partial {#2}}}
%\newcommand{\PDd}[2]{\frac{\partial^2 {#2}}{{\partial{#1}}^2}}
\newcommand{\PDsq}[2]{\frac{\partial^2 {#2}}{(\partial {#1})^2}}

\newcommand{\Partial}[2]{\frac{\partial {#1}}{\partial {#2}}}
\DeclareMathOperator{\RejName}{Rej}
\newcommand{\Rej}[2]{\RejName_{#1}\left( {#2} \right)}
\newcommand{\Rm}[1]{\mathbb{R}^{#1}}
\newcommand{\Cm}[1]{\mathbb{C}^{#1}}
\newcommand{\conj}[0]{{*}}

%</misc>

% <grade selection>
%
\newcommand{\gpgrade}[2] {{\left\langle{{#1}}\right\rangle}_{#2}}

\newcommand{\gpgradezero}[1] {\gpgrade{#1}{}}
%\newcommand{\gpscalargrade}[1] {{\left\langle{{#1}}\right\rangle}}
%\newcommand{\gpgradezero}[1] {\gpgrade{#1}{0}}

%\newcommand{\gpgradeone}[1] {{\left\langle{{#1}}\right\rangle}_{1}}
\newcommand{\gpgradeone}[1] {\gpgrade{#1}{1}}

\newcommand{\gpgradetwo}[1] {\gpgrade{#1}{2}}
\newcommand{\gpgradethree}[1] {\gpgrade{#1}{3}}
\newcommand{\gpgradefour}[1] {\gpgrade{#1}{4}}
%
% </grade selection>



\newcommand{\adot}[0]{{\dot{a}}}
\newcommand{\bdot}[0]{{\dot{b}}}
% taken for centered dot:
%\newcommand{\cdot}[0]{{\dot{c}}}
%\newcommand{\ddot}[0]{{\dot{d}}}
\newcommand{\edot}[0]{{\dot{e}}}
\newcommand{\fdot}[0]{{\dot{f}}}
\newcommand{\gdot}[0]{{\dot{g}}}
\newcommand{\hdot}[0]{{\dot{h}}}
\newcommand{\idot}[0]{{\dot{i}}}
\newcommand{\jdot}[0]{{\dot{j}}}
\newcommand{\kdot}[0]{{\dot{k}}}
\newcommand{\ldot}[0]{{\dot{l}}}
\newcommand{\mdot}[0]{{\dot{m}}}
\newcommand{\ndot}[0]{{\dot{n}}}
%\newcommand{\odot}[0]{{\dot{o}}}
\newcommand{\pdot}[0]{{\dot{p}}}
\newcommand{\qdot}[0]{{\dot{q}}}
\newcommand{\rdot}[0]{{\dot{r}}}
\newcommand{\sdot}[0]{{\dot{s}}}
\newcommand{\tdot}[0]{{\dot{t}}}
\newcommand{\udot}[0]{{\dot{u}}}
\newcommand{\vdot}[0]{{\dot{v}}}
\newcommand{\wdot}[0]{{\dot{w}}}
\newcommand{\xdot}[0]{{\dot{x}}}
\newcommand{\ydot}[0]{{\dot{y}}}
\newcommand{\zdot}[0]{{\dot{z}}}
\newcommand{\addot}[0]{{\ddot{a}}}
\newcommand{\bddot}[0]{{\ddot{b}}}
\newcommand{\cddot}[0]{{\ddot{c}}}
%\newcommand{\dddot}[0]{{\ddot{d}}}
\newcommand{\eddot}[0]{{\ddot{e}}}
\newcommand{\fddot}[0]{{\ddot{f}}}
\newcommand{\gddot}[0]{{\ddot{g}}}
\newcommand{\hddot}[0]{{\ddot{h}}}
\newcommand{\iddot}[0]{{\ddot{i}}}
\newcommand{\jddot}[0]{{\ddot{j}}}
\newcommand{\kddot}[0]{{\ddot{k}}}
\newcommand{\lddot}[0]{{\ddot{l}}}
\newcommand{\mddot}[0]{{\ddot{m}}}
\newcommand{\nddot}[0]{{\ddot{n}}}
\newcommand{\oddot}[0]{{\ddot{o}}}
\newcommand{\pddot}[0]{{\ddot{p}}}
\newcommand{\qddot}[0]{{\ddot{q}}}
\newcommand{\rddot}[0]{{\ddot{r}}}
\newcommand{\sddot}[0]{{\ddot{s}}}
\newcommand{\tddot}[0]{{\ddot{t}}}
\newcommand{\uddot}[0]{{\ddot{u}}}
\newcommand{\vddot}[0]{{\ddot{v}}}
\newcommand{\wddot}[0]{{\ddot{w}}}
\newcommand{\xddot}[0]{{\ddot{x}}}
\newcommand{\yddot}[0]{{\ddot{y}}}
\newcommand{\zddot}[0]{{\ddot{z}}}

%<bold and dot greek symbols>
%

\newcommand{\Deltadot}[0]{{\dot{\Delta}}}
\newcommand{\Gammadot}[0]{{\dot{\Gamma}}}
\newcommand{\Lambdadot}[0]{{\dot{\Lambda}}}
\newcommand{\Omegadot}[0]{{\dot{\Omega}}}
\newcommand{\Phidot}[0]{{\dot{\Phi}}}
\newcommand{\Pidot}[0]{{\dot{\Pi}}}
\newcommand{\Psidot}[0]{{\dot{\Psi}}}
\newcommand{\Sigmadot}[0]{{\dot{\Sigma}}}
\newcommand{\Thetadot}[0]{{\dot{\Theta}}}
\newcommand{\Upsilondot}[0]{{\dot{\Upsilon}}}
\newcommand{\Xidot}[0]{{\dot{\Xi}}}
\newcommand{\alphadot}[0]{{\dot{\alpha}}}
\newcommand{\betadot}[0]{{\dot{\beta}}}
\newcommand{\chidot}[0]{{\dot{\chi}}}
\newcommand{\deltadot}[0]{{\dot{\delta}}}
\newcommand{\epsilondot}[0]{{\dot{\epsilon}}}
\newcommand{\etadot}[0]{{\dot{\eta}}}
\newcommand{\gammadot}[0]{{\dot{\gamma}}}
\newcommand{\kappadot}[0]{{\dot{\kappa}}}
\newcommand{\lambdadot}[0]{{\dot{\lambda}}}
\newcommand{\mudot}[0]{{\dot{\mu}}}
\newcommand{\nudot}[0]{{\dot{\nu}}}
\newcommand{\omegadot}[0]{{\dot{\omega}}}
\newcommand{\phidot}[0]{{\dot{\phi}}}
\newcommand{\pidot}[0]{{\dot{\pi}}}
\newcommand{\psidot}[0]{{\dot{\psi}}}
\newcommand{\rhodot}[0]{{\dot{\rho}}}
\newcommand{\sigmadot}[0]{{\dot{\sigma}}}
\newcommand{\taudot}[0]{{\dot{\tau}}}
\newcommand{\thetadot}[0]{{\dot{\theta}}}
\newcommand{\upsilondot}[0]{{\dot{\upsilon}}}
\newcommand{\varepsilondot}[0]{{\dot{\varepsilon}}}
\newcommand{\varphidot}[0]{{\dot{\varphi}}}
\newcommand{\varpidot}[0]{{\dot{\varpi}}}
\newcommand{\varrhodot}[0]{{\dot{\varrho}}}
\newcommand{\varsigmadot}[0]{{\dot{\varsigma}}}
\newcommand{\varthetadot}[0]{{\dot{\vartheta}}}
\newcommand{\xidot}[0]{{\dot{\xi}}}
\newcommand{\zetadot}[0]{{\dot{\zeta}}}

\newcommand{\Deltaddot}[0]{{\ddot{\Delta}}}
\newcommand{\Gammaddot}[0]{{\ddot{\Gamma}}}
\newcommand{\Lambdaddot}[0]{{\ddot{\Lambda}}}
\newcommand{\Omegaddot}[0]{{\ddot{\Omega}}}
\newcommand{\Phiddot}[0]{{\ddot{\Phi}}}
\newcommand{\Piddot}[0]{{\ddot{\Pi}}}
\newcommand{\Psiddot}[0]{{\ddot{\Psi}}}
\newcommand{\Sigmaddot}[0]{{\ddot{\Sigma}}}
\newcommand{\Thetaddot}[0]{{\ddot{\Theta}}}
\newcommand{\Upsilonddot}[0]{{\ddot{\Upsilon}}}
\newcommand{\Xiddot}[0]{{\ddot{\Xi}}}
\newcommand{\alphaddot}[0]{{\ddot{\alpha}}}
\newcommand{\betaddot}[0]{{\ddot{\beta}}}
\newcommand{\chiddot}[0]{{\ddot{\chi}}}
\newcommand{\deltaddot}[0]{{\ddot{\delta}}}
\newcommand{\epsilonddot}[0]{{\ddot{\epsilon}}}
\newcommand{\etaddot}[0]{{\ddot{\eta}}}
\newcommand{\gammaddot}[0]{{\ddot{\gamma}}}
\newcommand{\kappaddot}[0]{{\ddot{\kappa}}}
\newcommand{\lambdaddot}[0]{{\ddot{\lambda}}}
\newcommand{\muddot}[0]{{\ddot{\mu}}}
\newcommand{\nuddot}[0]{{\ddot{\nu}}}
\newcommand{\omegaddot}[0]{{\ddot{\omega}}}
\newcommand{\phiddot}[0]{{\ddot{\phi}}}
\newcommand{\piddot}[0]{{\ddot{\pi}}}
\newcommand{\psiddot}[0]{{\ddot{\psi}}}
\newcommand{\rhoddot}[0]{{\ddot{\rho}}}
\newcommand{\sigmaddot}[0]{{\ddot{\sigma}}}
\newcommand{\tauddot}[0]{{\ddot{\tau}}}
\newcommand{\thetaddot}[0]{{\ddot{\theta}}}
\newcommand{\upsilonddot}[0]{{\ddot{\upsilon}}}
\newcommand{\varepsilonddot}[0]{{\ddot{\varepsilon}}}
\newcommand{\varphiddot}[0]{{\ddot{\varphi}}}
\newcommand{\varpiddot}[0]{{\ddot{\varpi}}}
\newcommand{\varrhoddot}[0]{{\ddot{\varrho}}}
\newcommand{\varsigmaddot}[0]{{\ddot{\varsigma}}}
\newcommand{\varthetaddot}[0]{{\ddot{\vartheta}}}
\newcommand{\xiddot}[0]{{\ddot{\xi}}}
\newcommand{\zetaddot}[0]{{\ddot{\zeta}}}

\newcommand{\BDelta}[0]{\boldsymbol{\Delta}}
\newcommand{\BGamma}[0]{\boldsymbol{\Gamma}}
\newcommand{\BLambda}[0]{\boldsymbol{\Lambda}}
\newcommand{\BOmega}[0]{\boldsymbol{\Omega}}
\newcommand{\BPhi}[0]{\boldsymbol{\Phi}}
\newcommand{\BPi}[0]{\boldsymbol{\Pi}}
\newcommand{\BPsi}[0]{\boldsymbol{\Psi}}
\newcommand{\BSigma}[0]{\boldsymbol{\Sigma}}
\newcommand{\BTheta}[0]{\boldsymbol{\Theta}}
\newcommand{\BUpsilon}[0]{\boldsymbol{\Upsilon}}
\newcommand{\BXi}[0]{\boldsymbol{\Xi}}
\newcommand{\Balpha}[0]{\boldsymbol{\alpha}}
\newcommand{\Bbeta}[0]{\boldsymbol{\beta}}
\newcommand{\Bchi}[0]{\boldsymbol{\chi}}
\newcommand{\Bdelta}[0]{\boldsymbol{\delta}}
\newcommand{\Bepsilon}[0]{\boldsymbol{\epsilon}}
\newcommand{\Beta}[0]{\boldsymbol{\eta}}
\newcommand{\Bgamma}[0]{\boldsymbol{\gamma}}
\newcommand{\Bkappa}[0]{\boldsymbol{\kappa}}
\newcommand{\Blambda}[0]{\boldsymbol{\lambda}}
\newcommand{\Bmu}[0]{\boldsymbol{\mu}}
\newcommand{\Bnu}[0]{\boldsymbol{\nu}}
%\newcommand{\Bomega}[0]{\boldsymbol{\omega}}
\newcommand{\Bphi}[0]{\boldsymbol{\phi}}
\newcommand{\Bpi}[0]{\boldsymbol{\pi}}
\newcommand{\Bpsi}[0]{\boldsymbol{\psi}}
\newcommand{\Brho}[0]{\boldsymbol{\rho}}
\newcommand{\Bsigma}[0]{\boldsymbol{\sigma}}
%\newcommand{\Btau}[0]{\boldsymbol{\tau}}
%\newcommand{\Btheta}[0]{\boldsymbol{\theta}}
\newcommand{\Bupsilon}[0]{\boldsymbol{\upsilon}}
\newcommand{\Bvarepsilon}[0]{\boldsymbol{\varepsilon}}
\newcommand{\Bvarphi}[0]{\boldsymbol{\varphi}}
\newcommand{\Bvarpi}[0]{\boldsymbol{\varpi}}
\newcommand{\Bvarrho}[0]{\boldsymbol{\varrho}}
\newcommand{\Bvarsigma}[0]{\boldsymbol{\varsigma}}
\newcommand{\Bvartheta}[0]{\boldsymbol{\vartheta}}
\newcommand{\Bxi}[0]{\boldsymbol{\xi}}
\newcommand{\Bzeta}[0]{\boldsymbol{\zeta}}
%
%</bold and dot greek symbols>
%<infrequent>
%
%\newcommand{\AreaOp}[1]{\AName_{#1}}
%\newcommand{\Babs}[0]{\abs{\BB}}
%\newcommand{\Bcap}[0]{\hat{\BB}}
%\newcommand{\BrPrimeRej}[0]{\rcap(\rcap \wedge \Br')}
%\newcommand{\CA}[0]{\mathcal{A}}
%\newcommand{\Cos}[1]{\cos{\left({#1}\right)}}
%\newcommand{\Det}[1] {\abs{#1}}
%\newcommand{\Dsq}[2] {\frac {\partial^2 {#1}} {\partial {#2}^2}}
%\newcommand{\Exp}[1]{\exp{\left({#1}\right)}}
%\newcommand{\Norm}[1]{\left\lVert{#1}\right\rVert}
%\newcommand{\Sin}[1]{\sin{\left({#1}\right)}}
%\newcommand{\T}[0]{\text{T}}
%\newcommand{\VolumeOp}[1]{\VName_{#1}}
%\newcommand{\agrad}[0]{\Ba \cdot \nabla}
%\newcommand{\alphacap}[0]{\hat{\boldsymbol{\alpha}}}
%\newcommand{\Fcap}[0]{\hat{\BF}}
%\newcommand{\bithree}[0]{{\Bi}_3}
%\newcommand{\bxa}[0]{\Bx\Ba}
%\newcommand{\coordvec}[2]{
%\newcommand{\costheta}[0]{\acap \cdot \xcap}
%\newcommand{\ddt}[1]{\ddot{#1}}
%\newcommand{\ddu}[1] {\frac {d{#1}} {du}}
%\newcommand{\dsqxj}[2] {\frac {\partial^2 {#1}} {\partial {x_{#2}}^2}}
%\newcommand{\dtheta}[1]{\frac{d {#1}}{d \theta}}
%\newcommand{\dt}[1]{\dot{#1}}
%\newcommand{\dt}[1]{\frac{d {#1}}{dt}}
%\newcommand{\dxj}[2] {\frac {\partial {#1}} {\partial {x_{#2}}}}
%\newcommand{\halfPhi}[0]{\frac{\phi}{2}}
%\newcommand{\half}[0]{\inv{2}}
%\newcommand{\inv}[1]{\frac{1}{#1}}
%\newcommand{\laplacian}[0]{\nabla^2}
%\newcommand{\matrixoftx}[3]{
%\newcommand{\nrrp}[0]{\norm{\rcap \wedge \Br'}}
%\newcommand{\oiint}{\bigcirc \hspace{-1.4em} \int \hspace{-.8em} \int}
%\newcommand{\transpose}[1]{{#1}^{\text{T}}}
%\newcommand{\transpose}[1]{{{#1}^{\TextTranspose}}}
%\newcommand{\transpose}[1]{{{#1}^{\text{T}}}}
%\newcommand{\barA}[0]{\bar{A}}
%\newcommand{\qbar}[0]{\bar{q}}
%\newcommand{\qdotbar}[0]{\dot{\bar{q}}}
%
%</infrequent>




\newcommand{\barh}[0]{\bar{h}}

\usepackage[bookmarks=true]{hyperref}

\title{ Some rough notes on GravitoElectroMagnetism. }
\author{Peeter Joot}
\date{ October 26, 2008.  Last Revision: $Date: 2008/10/29 04:01:33 $ }

\begin{document}

\maketitle{}
\tableofcontents

\section{ Motivation. }

I found the GEM equations interesting, and explored the surface of them slightly.  Here are some notes, mostly as a reference for myself ... looking at the
GEM equations mostly generates questions, especially since I don't have the GR
background to understand where the potentials (ie: what is that stress energy
tensor $T_{\mu\nu}$) nor the specifics of where the metric tensor 
(pertubation of the Minkowski metric) came from.

\section{ Definitions. }

The article \cite{mashhoon2003gbr} outlines the GEM equations, which in short
are

Scalar and potential fields

\begin{align}
\Phi \approx \frac{GM}{r}, \quad \BA \approx \frac{G}{c} \frac{\BJ \cross \Bx}{r^3}
\end{align}

Guage condition

\begin{align}
\inv{c}\PD{t}{\Phi} + \spacegrad \cdot \left( \inv{2} \BA \right) = 0.
\end{align}

GEM fields
\begin{align}
\BE = - \spacegrad \Phi -\inv{c} \PD{t}{}\left( \inv{2} \BB \right), \quad \BB = \spacegrad \cross \BA
\end{align}

and finally the Maxwell-like equations are

\begin{align}
\spacegrad \cross \BE &= -\inv{c} \PD{t}{}\left(\inv{2}\BB\right) \\
\spacegrad \cdot \left( \inv{2} \BB \right) &= 0 \\
\spacegrad \cdot \BE &= 4 \pi G \rho \\
\spacegrad \cross \left( \inv{2} \BB \right) &= \inv{c} \PD{t}{\BE} + \frac{4\pi G}{c}\BJ
\end{align}

\section{ STA form. }

As with Maxwell's equations a clifford algebra representation should be possible to put this into a more symmetric form.  Combining the spatial div and grads, following conventions from \cite{doran2003gap} we have

\begin{align}
\spacegrad \BE &= 4 \pi G \rho + \inv{c} \PD{t}{}\left(\inv{2}I \BB\right) \\
\spacegrad \left( \inv{2} I \BB \right) &= \inv{c} \PD{t}{\BE} + \frac{4\pi G}{c}\BJ
\end{align}

Or
\begin{align}
\left( \spacegrad -\inv{c} \PD{t}{}\right) \left( \BE + \inv{2} I \BB \right) &= \frac{4\pi G}{c} \left( c \rho + \BJ \right)
\end{align}

Left multiplication with $\gamma_0$, using a time positive metric signature ($(\gamma_0)^2=1$), 
\begin{align}
\left( \spacegrad -\inv{c} \PD{t}{}\right) \gamma_0 \left( -\BE + \inv{2} I \BB \right) &= \frac{4\pi G}{c} \left( c \rho \gamma_0 + J^i \gamma_i \right)
\end{align}

But $\left( \spacegrad -\inv{c} \PD{t}{}\right) \gamma_0 = \gamma_i \partial_i - \gamma_0 \partial_0 = -\gamma^\mu \partial_\mu = -\grad$.  Introduction of a four vector mass density $J = c\rho \gamma_0 + J^i \gamma_i = J^\mu \gamma_\mu$, and a bivector field $F = \BE -\inv{2} I \BB$ this is

\begin{align}
\grad F = -\frac{4\pi G}{c} J
\end{align}

The guage condition suggests a four potential $V = \Phi \gamma_0 + \BA \gamma_0 = V^\mu \gamma_\mu$, where $V^0 = \Phi$, and $V^i = A^i/2$.  This merges the
space and time parts of the guage condition

\begin{align*}
\grad \cdot V = \gamma^\mu \partial_\mu \cdot \gamma_\nu V^\nu = \partial_\mu V^\mu = \inv{c}\PD{t}{\Phi} + \inv{2}\partial_i A^i.
\end{align*}

It is reasonable to assume that $F = \grad \wedge V$ as in electromagnetism.  Let's see if this is the case

\begin{align*}
\BE - I\BB/2 
&= - \spacegrad \Phi -\inv{c} \PD{t}{}\left( \inv{2} \BB \right) - I\spacegrad \cross \BA/2 \\
&= - \gamma_i \partial_i \gamma_0 V^0 - \inv{2} \partial_0 A^i \gamma_i \gamma_0 + \spacegrad \wedge \BA/2 \\
&= \gamma^i \partial_i \gamma_0 V^0 + \gamma^0 \partial_0 \gamma_i A^i/2 - \gamma_i \partial_i \wedge \gamma_j V^j \\
&= \gamma^i \partial_i \gamma_0 V^0 + \gamma^0 \partial_0 \gamma_i V^i + \gamma^i \partial_i \wedge \gamma_j V^j \\
&= \gamma^\mu \partial_\mu \wedge \gamma_\nu V^\nu \\
&= \grad \wedge V
\end{align*}

Okay, so in terms of potential we have the form as Maxwell's equation

\begin{align}\label{eqn:field}
\grad (\grad \wedge V) &= -\frac{4\pi G}{c} J.
\end{align}

With the guage condition $\grad \cdot V = 0$, this produces the wave equation

\begin{align}
\grad^2 V &= -\frac{4\pi G}{c} J.
\end{align}

In terms of the author's original equation 1.2 it appears that roughly 
$V^\mu = \barh_{0\mu}$, and $J^\mu \propto T_{0\mu}$.

This is logically how he is able to go from that equation to the maxwell
form since both have the same four-vector wave equation form (when $T_{ij} \approx 0$).  To give the potentials specific values in terms of mass and current
distribution appears to be where the retarded integrals are used.

The author expresses $T^{\mu\nu}$ in terms of $\rho$, and mass current $j$, but
the field equations are in terms of $T_{\mu\nu}$.  What metric tensor is
used to translate from upper to lower indexes in this case.  ie: is it $g_{\mu\nu}$, or $\eta_{\mu\nu}$ ?

\section{ Lagrangians. }

\subsection{ Field Lagrangian. }

Since the electrodynamic equation and corresponding field Lagrangian is
\begin{align*}
\grad (\grad \wedge A) &= \frac{J}{\epsilon_0 c} \\
\LL &= -\frac{\epsilon_0 c}{2} (\grad \wedge A)^2 + A \cdot J
\end{align*}

Then, from \ref{eqn:field}, the GEM field Lagrangian in covariant form is

\begin{align*}
\LL &= \frac{c}{8 \pi G} (\grad \wedge V)^2 + V \cdot J \\
\end{align*}

Writing $F^{\mu\nu} = \partial^\mu V^\nu - \partial^\nu V^\mu$, the scalar part of this Lagrangian is:

\begin{align*}
\LL &= -\frac{c}{16 \pi G} F^{\mu\nu} F_{\mu\nu} + V^\sigma J_\sigma \\
\end{align*}

Is this expression hiding in the Einstein field equations?

What is the Lagrangian for newtonian gravity, and how do they compare?

\subsection{ Interaction Lagrangian. }

The metric (equation 1.4) in the article is given to be

\begin{align*}
ds^2 &= 
-c^2\left(1 - 2 \frac{\Phi}{c^2}\right) dt^2
+\frac{4}{c}\left(\BA \cdot d\Bx \right) dt 
+\left(1 + 2 \frac{\Phi}{c^2}\right) \delta_{ij}dx^i dx^j \\
\implies
\Abs{ds^2} = c^2 (d\tau)^2 &= (dx^0)^2 - \sum_i (dx^i)^2
-2 \frac{V_0}{c^2} (dx^0)^2
-\frac{8}{c^2} V_i dx^i dx^0
- 2 \frac{V_0}{c^2} \delta_{ij}dx^i dx^j
\end{align*}

With $v = \gamma_\mu dx^\mu/d\tau$, the Lagrangian for interaction is

\begin{align*}
\LL 
&= \inv{2} m \Abs{\frac{ds}{d\tau}}^2  \\
&= \inv{2} m c^2 \\
&= \inv{2} m v^2 -2 \frac{m V_0}{c^2} \sum_\mu (\xdot^\mu)^2 -\frac{8 m}{c^2} V_i \xdot^0 \xdot^i  \\
\end{align*}

\begin{align}\label{eqn:interactionlagrangian}
\LL &= \inv{2} m v^2 - 2m \left( V_0 \sum_\mu (\xdot^\mu / c)^2 + 4 V_i (\xdot^0/c) (\xdot^i/c) \right)
\end{align}

Now, unlike the Lorentz force Lagrangian
\begin{align*}
\LL &= \inv{2} m v^2 + q A \cdot v/c,
\end{align*}

the Lagragian of \ref{eqn:interactionlagrangian} is quadradic in powers of $\xdot^\mu$.  
There are remarks in the article saying that the non-covariant Lagrangian used to arrive at the Lorentz force equivalent was a first order approximation.
Evaluation of this interaction Lagrangian does not produce anything like the 
$\pdot_\mu = \kappa F_{\mu\nu}\xdot^\nu$ that we see in electrodynamics.

The calculation isn't interesting but the end result for reference is

\begin{align*}
\pdot
%&= \frac{4m}{c^2} \frac{d}{d\tau}\left( V_0 \gamma^\mu v^\mu + 2V_i (v^i \gamma^0 + v^0 \gamma^i) \right) \\
%&- \frac{2m}{c^2} \left( \sum_\mu (v^\mu)^2 \grad V_0 + 4 v^0 v^i \grad V_i \right) \\
&= \frac{4m}{c^2} \left( (v \cdot \grad V_0) \gamma^\mu v^\mu + 2 (v \cdot \grad V_i) (v^i \gamma^0 + v^0 \gamma^i) \right) \\
&+ \frac{4m}{c^2} \left( V_0 \gamma^\mu a^\mu + 2V_i (a^i \gamma^0 + a^0 \gamma^i) \right) \\
&- \frac{2m}{c^2} \left( \sum_\mu (v^\mu)^2 \grad V_0 + 4 v^0 v^i \grad V_i \right)
\end{align*}

This can be simplified somewhat, but no matter what it will be quadratic in the velocity coordinates.

The article also says that the line element is approximate.
Has some of what
is required for a more symmetric covariant interaction proper force been
discarded?

\section{ Conclusion. }

The ideas here are interesting.  At a high level, roughly, as I see it, the equation

\begin{align*}
\grad^2 h_{0\mu} = T_{0\mu}
\end{align*}

has exactly the same form as Maxwell's equations in covariant form, so you can define an antisymmetric field tensor equation in the same way, treating these elements of h, and the corresponding elements of T as a four vector potential and mass current.

That said, I don't have the GR background to know understand the introduction.  For example, how to actually arrive at 1.2
or how to calculated your metric tensor in equation 1.4.  I would have expected 1.4 to have a more symmetric form like the covariant Lorentz force Lagrangian ($v^2 + kA.v$), since you can get a Lorentz force like equation out of it.  Because of the quadratic velocity terms, no matter how one varies that metric with respect to s as a parameter, one cannot get anything at all close to the electrodynamics Lorentz force equation $m\ddot{x}^\mu = q F_\mu\nu \dot{x}_\nu$, so the coorrespondance between electromagnetism and GR breaks down once one considers the interaction.

\bibliographystyle{plainnat}
\bibliography{myrefs}

\end{document}

\part{Electrodynamics.}
\include{maxwells_ga}
\include{wave_eqn}
\include{gaussian_surface}
\documentclass{article}

\usepackage{amsmath}
\usepackage{mathpazo}

%
% shorthand for bold symbols, convenient for vectors and matrices
%
\newcommand{\Ba}[0]{\mathbf{a}}
\newcommand{\Bb}[0]{\mathbf{b}}
\newcommand{\Bc}[0]{\mathbf{c}}
\newcommand{\Bd}[0]{\mathbf{d}}
\newcommand{\Be}[0]{\mathbf{e}}
\newcommand{\Bf}[0]{\mathbf{f}}
\newcommand{\Bg}[0]{\mathbf{g}}
\newcommand{\Bh}[0]{\mathbf{h}}
\newcommand{\Bi}[0]{\mathbf{i}}
\newcommand{\Bj}[0]{\mathbf{j}}
\newcommand{\Bk}[0]{\mathbf{k}}
\newcommand{\Bl}[0]{\mathbf{l}}
\newcommand{\Bm}[0]{\mathbf{m}}
\newcommand{\Bn}[0]{\mathbf{n}}
\newcommand{\Bo}[0]{\mathbf{o}}
\newcommand{\Bp}[0]{\mathbf{p}}
\newcommand{\Bq}[0]{\mathbf{q}}
\newcommand{\Br}[0]{\mathbf{r}}
\newcommand{\Bs}[0]{\mathbf{s}}
\newcommand{\Bt}[0]{\mathbf{t}}
\newcommand{\Bu}[0]{\mathbf{u}}
\newcommand{\Bv}[0]{\mathbf{v}}
\newcommand{\Bw}[0]{\mathbf{w}}
\newcommand{\Bx}[0]{\mathbf{x}}
\newcommand{\By}[0]{\mathbf{y}}
\newcommand{\Bz}[0]{\mathbf{z}}
\newcommand{\BA}[0]{\mathbf{A}}
\newcommand{\BB}[0]{\mathbf{B}}
\newcommand{\BC}[0]{\mathbf{C}}
\newcommand{\BD}[0]{\mathbf{D}}
\newcommand{\BE}[0]{\mathbf{E}}
\newcommand{\BF}[0]{\mathbf{F}}
\newcommand{\BG}[0]{\mathbf{G}}
\newcommand{\BH}[0]{\mathbf{H}}
\newcommand{\BI}[0]{\mathbf{I}}
\newcommand{\BJ}[0]{\mathbf{J}}
\newcommand{\BK}[0]{\mathbf{K}}
\newcommand{\BL}[0]{\mathbf{L}}
\newcommand{\BM}[0]{\mathbf{M}}
\newcommand{\BN}[0]{\mathbf{N}}
\newcommand{\BO}[0]{\mathbf{O}}
\newcommand{\BP}[0]{\mathbf{P}}
\newcommand{\BQ}[0]{\mathbf{Q}}
\newcommand{\BR}[0]{\mathbf{R}}
\newcommand{\BS}[0]{\mathbf{S}}
\newcommand{\BT}[0]{\mathbf{T}}
\newcommand{\BU}[0]{\mathbf{U}}
\newcommand{\BV}[0]{\mathbf{V}}
\newcommand{\BW}[0]{\mathbf{W}}
\newcommand{\BX}[0]{\mathbf{X}}
\newcommand{\BY}[0]{\mathbf{Y}}
\newcommand{\BZ}[0]{\mathbf{Z}}

\newcommand{\Bzero}[0]{\mathbf{0}}
\newcommand{\Btheta}[0]{\boldsymbol{\theta}}
\newcommand{\Btau}[0]{\boldsymbol{\tau}}
\newcommand{\Bomega}[0]{\boldsymbol{\omega}}

%
% shorthand for unit vectors
%
\newcommand{\acap}[0]{\hat{\Ba}}
\newcommand{\bcap}[0]{\hat{\Bb}}
\newcommand{\ccap}[0]{\hat{\Bc}}
\newcommand{\dcap}[0]{\hat{\Bd}}
\newcommand{\ecap}[0]{\hat{\Be}}
\newcommand{\fcap}[0]{\hat{\Bf}}
\newcommand{\gcap}[0]{\hat{\Bg}}
\newcommand{\hcap}[0]{\hat{\Bh}}
\newcommand{\icap}[0]{\hat{\Bi}}
\newcommand{\jcap}[0]{\hat{\Bj}}
\newcommand{\kcap}[0]{\hat{\Bk}}
\newcommand{\lcap}[0]{\hat{\Bl}}
\newcommand{\mcap}[0]{\hat{\Bm}}
\newcommand{\ncap}[0]{\hat{\Bn}}
\newcommand{\ocap}[0]{\hat{\Bo}}
\newcommand{\pcap}[0]{\hat{\Bp}}
\newcommand{\qcap}[0]{\hat{\Bq}}
\newcommand{\rcap}[0]{\hat{\Br}}
\newcommand{\scap}[0]{\hat{\Bs}}
\newcommand{\tcap}[0]{\hat{\Bt}}
\newcommand{\ucap}[0]{\hat{\Bu}}
\newcommand{\vcap}[0]{\hat{\Bv}}
\newcommand{\wcap}[0]{\hat{\Bw}}
\newcommand{\xcap}[0]{\hat{\Bx}}
\newcommand{\ycap}[0]{\hat{\By}}
\newcommand{\zcap}[0]{\hat{\Bz}}
\newcommand{\thetacap}[0]{\hat{\Btheta}}

%
% to write R^n and C^n in a distinguishable fashion.  Perhaps change this
% to the double lined characters upon figuring out how to do so.
%
\newcommand{\C}[1]{$\mathbb{C}^{#1}$}
\newcommand{\R}[1]{$\mathbb{R}^{#1}$}

%
% various generally useful helpers
%

% derivative of #1 wrt. #2:
\newcommand{\D}[2] {\frac {d#2} {d#1}}

\newcommand{\inv}[1]{\frac{1}{#1}}
\newcommand{\cross}[0]{\times}

\newcommand{\abs}[1]{\lvert{#1}\rvert}
\newcommand{\norm}[1]{\lVert{#1}\rVert}
\newcommand{\innerprod}[2]{\langle{#1}, {#2}\rangle}
\newcommand{\dotprod}[2]{{#1} \cdot {#2}}
\newcommand{\bdotprod}[2]{\left({#1} \cdot {#2}\right)}
\newcommand{\crossprod}[2]{{#1} \cross {#2}}
\newcommand{\tripleprod}[3]{\dotprod{\left(\crossprod{#1}{#2}\right)}{#3}}

\DeclareMathOperator{\Proj}{Proj}
\DeclareMathOperator{\Span}{span}
\DeclareMathOperator{\Sgn}{sgn}
\DeclareMathOperator{\Area}{Area}
\DeclareMathOperator{\Volume}{Volume}

%
% A few miscellaneous things specific to this document
%
\newcommand{\crossop}[1]{\crossprod{#1}{}}

% R2 vector.
\newcommand{\VectorTwo}[2]{
\begin{bmatrix}
 {#1} \\
 {#2}
\end{bmatrix}
}

\newcommand{\VectorN}[1]{
\begin{bmatrix}
{#1}_1 \\
{#1}_2 \\
\vdots \\
{#1}_N \\
\end{bmatrix}
}

\newcommand{\DETuvij}[4]{
\begin{vmatrix}
 {#1}_{#3} & {#1}_{#4} \\
 {#2}_{#3} & {#2}_{#4}
\end{vmatrix}
}

\newcommand{\DETuvwijk}[6]{
\begin{vmatrix}
 {#1}_{#4} & {#1}_{#5} & {#1}_{#6} \\
 {#2}_{#4} & {#2}_{#5} & {#2}_{#6} \\
 {#3}_{#4} & {#3}_{#5} & {#3}_{#6}
\end{vmatrix}
}

\newcommand{\DETuvwxijkl}[8]{
\begin{vmatrix}
 {#1}_{#5} & {#1}_{#6} & {#1}_{#7} & {#1}_{#8} \\
 {#2}_{#5} & {#2}_{#6} & {#2}_{#7} & {#2}_{#8} \\
 {#3}_{#5} & {#3}_{#6} & {#3}_{#7} & {#3}_{#8} \\
 {#4}_{#5} & {#4}_{#6} & {#4}_{#7} & {#4}_{#8} \\
\end{vmatrix}
}

%\newcommand{\DETuvwxyijklm}[10]{
%\begin{vmatrix}
% {#1}_{#6} & {#1}_{#7} & {#1}_{#8} & {#1}_{#9} & {#1}_{#10} \\
% {#2}_{#6} & {#2}_{#7} & {#2}_{#8} & {#2}_{#9} & {#2}_{#10} \\
% {#3}_{#6} & {#3}_{#7} & {#3}_{#8} & {#3}_{#9} & {#3}_{#10} \\
% {#4}_{#6} & {#4}_{#7} & {#4}_{#8} & {#4}_{#9} & {#4}_{#10} \\
% {#5}_{#6} & {#5}_{#7} & {#5}_{#8} & {#5}_{#9} & {#5}_{#10}
%\end{vmatrix}
%}

% R3 vector.
\newcommand{\VectorThree}[3]{
\begin{bmatrix}
 {#1} \\
 {#2} \\
 {#3}
\end{bmatrix}
}


%<misc>
%
\newcommand{\Abs}[1]{{\left\lvert{#1}\right\rvert}}
\newcommand{\spacegrad}[0]{\boldsymbol{\nabla}}
\newcommand{\grad}[0]{\nabla}
\newcommand{\LL}[0]{\mathcal{L}}

% == \partial_{#1} {#2}
\newcommand{\PD}[2]{\frac{\partial {#2}}{\partial {#1}}}
% inline variant
\newcommand{\PDi}[2]{{\partial {#2}}/{\partial {#1}}}

\newcommand{\PDD}[3]{\frac{\partial^2 {#3}}{\partial {#1}\partial {#2}}}
%\newcommand{\PDd}[2]{\frac{\partial^2 {#2}}{{\partial{#1}}^2}}
\newcommand{\PDsq}[2]{\frac{\partial^2 {#2}}{(\partial {#1})^2}}

\newcommand{\Partial}[2]{\frac{\partial {#1}}{\partial {#2}}}
\DeclareMathOperator{\RejName}{Rej}
\newcommand{\Rej}[2]{\RejName_{#1}\left( {#2} \right)}
\newcommand{\Rm}[1]{\mathbb{R}^{#1}}
\newcommand{\Cm}[1]{\mathbb{C}^{#1}}
\newcommand{\conj}[0]{{*}}

%</misc>

% <grade selection>
%
\newcommand{\gpgrade}[2] {{\left\langle{{#1}}\right\rangle}_{#2}}

\newcommand{\gpgradezero}[1] {\gpgrade{#1}{}}
%\newcommand{\gpscalargrade}[1] {{\left\langle{{#1}}\right\rangle}}
%\newcommand{\gpgradezero}[1] {\gpgrade{#1}{0}}

%\newcommand{\gpgradeone}[1] {{\left\langle{{#1}}\right\rangle}_{1}}
\newcommand{\gpgradeone}[1] {\gpgrade{#1}{1}}

\newcommand{\gpgradetwo}[1] {\gpgrade{#1}{2}}
\newcommand{\gpgradethree}[1] {\gpgrade{#1}{3}}
\newcommand{\gpgradefour}[1] {\gpgrade{#1}{4}}
%
% </grade selection>



\newcommand{\adot}[0]{{\dot{a}}}
\newcommand{\bdot}[0]{{\dot{b}}}
% taken for centered dot:
%\newcommand{\cdot}[0]{{\dot{c}}}
%\newcommand{\ddot}[0]{{\dot{d}}}
\newcommand{\edot}[0]{{\dot{e}}}
\newcommand{\fdot}[0]{{\dot{f}}}
\newcommand{\gdot}[0]{{\dot{g}}}
\newcommand{\hdot}[0]{{\dot{h}}}
\newcommand{\idot}[0]{{\dot{i}}}
\newcommand{\jdot}[0]{{\dot{j}}}
\newcommand{\kdot}[0]{{\dot{k}}}
\newcommand{\ldot}[0]{{\dot{l}}}
\newcommand{\mdot}[0]{{\dot{m}}}
\newcommand{\ndot}[0]{{\dot{n}}}
%\newcommand{\odot}[0]{{\dot{o}}}
\newcommand{\pdot}[0]{{\dot{p}}}
\newcommand{\qdot}[0]{{\dot{q}}}
\newcommand{\rdot}[0]{{\dot{r}}}
\newcommand{\sdot}[0]{{\dot{s}}}
\newcommand{\tdot}[0]{{\dot{t}}}
\newcommand{\udot}[0]{{\dot{u}}}
\newcommand{\vdot}[0]{{\dot{v}}}
\newcommand{\wdot}[0]{{\dot{w}}}
\newcommand{\xdot}[0]{{\dot{x}}}
\newcommand{\ydot}[0]{{\dot{y}}}
\newcommand{\zdot}[0]{{\dot{z}}}
\newcommand{\addot}[0]{{\ddot{a}}}
\newcommand{\bddot}[0]{{\ddot{b}}}
\newcommand{\cddot}[0]{{\ddot{c}}}
%\newcommand{\dddot}[0]{{\ddot{d}}}
\newcommand{\eddot}[0]{{\ddot{e}}}
\newcommand{\fddot}[0]{{\ddot{f}}}
\newcommand{\gddot}[0]{{\ddot{g}}}
\newcommand{\hddot}[0]{{\ddot{h}}}
\newcommand{\iddot}[0]{{\ddot{i}}}
\newcommand{\jddot}[0]{{\ddot{j}}}
\newcommand{\kddot}[0]{{\ddot{k}}}
\newcommand{\lddot}[0]{{\ddot{l}}}
\newcommand{\mddot}[0]{{\ddot{m}}}
\newcommand{\nddot}[0]{{\ddot{n}}}
\newcommand{\oddot}[0]{{\ddot{o}}}
\newcommand{\pddot}[0]{{\ddot{p}}}
\newcommand{\qddot}[0]{{\ddot{q}}}
\newcommand{\rddot}[0]{{\ddot{r}}}
\newcommand{\sddot}[0]{{\ddot{s}}}
\newcommand{\tddot}[0]{{\ddot{t}}}
\newcommand{\uddot}[0]{{\ddot{u}}}
\newcommand{\vddot}[0]{{\ddot{v}}}
\newcommand{\wddot}[0]{{\ddot{w}}}
\newcommand{\xddot}[0]{{\ddot{x}}}
\newcommand{\yddot}[0]{{\ddot{y}}}
\newcommand{\zddot}[0]{{\ddot{z}}}

%<bold and dot greek symbols>
%

\newcommand{\Deltadot}[0]{{\dot{\Delta}}}
\newcommand{\Gammadot}[0]{{\dot{\Gamma}}}
\newcommand{\Lambdadot}[0]{{\dot{\Lambda}}}
\newcommand{\Omegadot}[0]{{\dot{\Omega}}}
\newcommand{\Phidot}[0]{{\dot{\Phi}}}
\newcommand{\Pidot}[0]{{\dot{\Pi}}}
\newcommand{\Psidot}[0]{{\dot{\Psi}}}
\newcommand{\Sigmadot}[0]{{\dot{\Sigma}}}
\newcommand{\Thetadot}[0]{{\dot{\Theta}}}
\newcommand{\Upsilondot}[0]{{\dot{\Upsilon}}}
\newcommand{\Xidot}[0]{{\dot{\Xi}}}
\newcommand{\alphadot}[0]{{\dot{\alpha}}}
\newcommand{\betadot}[0]{{\dot{\beta}}}
\newcommand{\chidot}[0]{{\dot{\chi}}}
\newcommand{\deltadot}[0]{{\dot{\delta}}}
\newcommand{\epsilondot}[0]{{\dot{\epsilon}}}
\newcommand{\etadot}[0]{{\dot{\eta}}}
\newcommand{\gammadot}[0]{{\dot{\gamma}}}
\newcommand{\kappadot}[0]{{\dot{\kappa}}}
\newcommand{\lambdadot}[0]{{\dot{\lambda}}}
\newcommand{\mudot}[0]{{\dot{\mu}}}
\newcommand{\nudot}[0]{{\dot{\nu}}}
\newcommand{\omegadot}[0]{{\dot{\omega}}}
\newcommand{\phidot}[0]{{\dot{\phi}}}
\newcommand{\pidot}[0]{{\dot{\pi}}}
\newcommand{\psidot}[0]{{\dot{\psi}}}
\newcommand{\rhodot}[0]{{\dot{\rho}}}
\newcommand{\sigmadot}[0]{{\dot{\sigma}}}
\newcommand{\taudot}[0]{{\dot{\tau}}}
\newcommand{\thetadot}[0]{{\dot{\theta}}}
\newcommand{\upsilondot}[0]{{\dot{\upsilon}}}
\newcommand{\varepsilondot}[0]{{\dot{\varepsilon}}}
\newcommand{\varphidot}[0]{{\dot{\varphi}}}
\newcommand{\varpidot}[0]{{\dot{\varpi}}}
\newcommand{\varrhodot}[0]{{\dot{\varrho}}}
\newcommand{\varsigmadot}[0]{{\dot{\varsigma}}}
\newcommand{\varthetadot}[0]{{\dot{\vartheta}}}
\newcommand{\xidot}[0]{{\dot{\xi}}}
\newcommand{\zetadot}[0]{{\dot{\zeta}}}

\newcommand{\Deltaddot}[0]{{\ddot{\Delta}}}
\newcommand{\Gammaddot}[0]{{\ddot{\Gamma}}}
\newcommand{\Lambdaddot}[0]{{\ddot{\Lambda}}}
\newcommand{\Omegaddot}[0]{{\ddot{\Omega}}}
\newcommand{\Phiddot}[0]{{\ddot{\Phi}}}
\newcommand{\Piddot}[0]{{\ddot{\Pi}}}
\newcommand{\Psiddot}[0]{{\ddot{\Psi}}}
\newcommand{\Sigmaddot}[0]{{\ddot{\Sigma}}}
\newcommand{\Thetaddot}[0]{{\ddot{\Theta}}}
\newcommand{\Upsilonddot}[0]{{\ddot{\Upsilon}}}
\newcommand{\Xiddot}[0]{{\ddot{\Xi}}}
\newcommand{\alphaddot}[0]{{\ddot{\alpha}}}
\newcommand{\betaddot}[0]{{\ddot{\beta}}}
\newcommand{\chiddot}[0]{{\ddot{\chi}}}
\newcommand{\deltaddot}[0]{{\ddot{\delta}}}
\newcommand{\epsilonddot}[0]{{\ddot{\epsilon}}}
\newcommand{\etaddot}[0]{{\ddot{\eta}}}
\newcommand{\gammaddot}[0]{{\ddot{\gamma}}}
\newcommand{\kappaddot}[0]{{\ddot{\kappa}}}
\newcommand{\lambdaddot}[0]{{\ddot{\lambda}}}
\newcommand{\muddot}[0]{{\ddot{\mu}}}
\newcommand{\nuddot}[0]{{\ddot{\nu}}}
\newcommand{\omegaddot}[0]{{\ddot{\omega}}}
\newcommand{\phiddot}[0]{{\ddot{\phi}}}
\newcommand{\piddot}[0]{{\ddot{\pi}}}
\newcommand{\psiddot}[0]{{\ddot{\psi}}}
\newcommand{\rhoddot}[0]{{\ddot{\rho}}}
\newcommand{\sigmaddot}[0]{{\ddot{\sigma}}}
\newcommand{\tauddot}[0]{{\ddot{\tau}}}
\newcommand{\thetaddot}[0]{{\ddot{\theta}}}
\newcommand{\upsilonddot}[0]{{\ddot{\upsilon}}}
\newcommand{\varepsilonddot}[0]{{\ddot{\varepsilon}}}
\newcommand{\varphiddot}[0]{{\ddot{\varphi}}}
\newcommand{\varpiddot}[0]{{\ddot{\varpi}}}
\newcommand{\varrhoddot}[0]{{\ddot{\varrho}}}
\newcommand{\varsigmaddot}[0]{{\ddot{\varsigma}}}
\newcommand{\varthetaddot}[0]{{\ddot{\vartheta}}}
\newcommand{\xiddot}[0]{{\ddot{\xi}}}
\newcommand{\zetaddot}[0]{{\ddot{\zeta}}}

\newcommand{\BDelta}[0]{\boldsymbol{\Delta}}
\newcommand{\BGamma}[0]{\boldsymbol{\Gamma}}
\newcommand{\BLambda}[0]{\boldsymbol{\Lambda}}
\newcommand{\BOmega}[0]{\boldsymbol{\Omega}}
\newcommand{\BPhi}[0]{\boldsymbol{\Phi}}
\newcommand{\BPi}[0]{\boldsymbol{\Pi}}
\newcommand{\BPsi}[0]{\boldsymbol{\Psi}}
\newcommand{\BSigma}[0]{\boldsymbol{\Sigma}}
\newcommand{\BTheta}[0]{\boldsymbol{\Theta}}
\newcommand{\BUpsilon}[0]{\boldsymbol{\Upsilon}}
\newcommand{\BXi}[0]{\boldsymbol{\Xi}}
\newcommand{\Balpha}[0]{\boldsymbol{\alpha}}
\newcommand{\Bbeta}[0]{\boldsymbol{\beta}}
\newcommand{\Bchi}[0]{\boldsymbol{\chi}}
\newcommand{\Bdelta}[0]{\boldsymbol{\delta}}
\newcommand{\Bepsilon}[0]{\boldsymbol{\epsilon}}
\newcommand{\Beta}[0]{\boldsymbol{\eta}}
\newcommand{\Bgamma}[0]{\boldsymbol{\gamma}}
\newcommand{\Bkappa}[0]{\boldsymbol{\kappa}}
\newcommand{\Blambda}[0]{\boldsymbol{\lambda}}
\newcommand{\Bmu}[0]{\boldsymbol{\mu}}
\newcommand{\Bnu}[0]{\boldsymbol{\nu}}
%\newcommand{\Bomega}[0]{\boldsymbol{\omega}}
\newcommand{\Bphi}[0]{\boldsymbol{\phi}}
\newcommand{\Bpi}[0]{\boldsymbol{\pi}}
\newcommand{\Bpsi}[0]{\boldsymbol{\psi}}
\newcommand{\Brho}[0]{\boldsymbol{\rho}}
\newcommand{\Bsigma}[0]{\boldsymbol{\sigma}}
%\newcommand{\Btau}[0]{\boldsymbol{\tau}}
%\newcommand{\Btheta}[0]{\boldsymbol{\theta}}
\newcommand{\Bupsilon}[0]{\boldsymbol{\upsilon}}
\newcommand{\Bvarepsilon}[0]{\boldsymbol{\varepsilon}}
\newcommand{\Bvarphi}[0]{\boldsymbol{\varphi}}
\newcommand{\Bvarpi}[0]{\boldsymbol{\varpi}}
\newcommand{\Bvarrho}[0]{\boldsymbol{\varrho}}
\newcommand{\Bvarsigma}[0]{\boldsymbol{\varsigma}}
\newcommand{\Bvartheta}[0]{\boldsymbol{\vartheta}}
\newcommand{\Bxi}[0]{\boldsymbol{\xi}}
\newcommand{\Bzeta}[0]{\boldsymbol{\zeta}}
%
%</bold and dot greek symbols>
%<infrequent>
%
%\newcommand{\AreaOp}[1]{\AName_{#1}}
%\newcommand{\Babs}[0]{\abs{\BB}}
%\newcommand{\Bcap}[0]{\hat{\BB}}
%\newcommand{\BrPrimeRej}[0]{\rcap(\rcap \wedge \Br')}
%\newcommand{\CA}[0]{\mathcal{A}}
%\newcommand{\Cos}[1]{\cos{\left({#1}\right)}}
%\newcommand{\Det}[1] {\abs{#1}}
%\newcommand{\Dsq}[2] {\frac {\partial^2 {#1}} {\partial {#2}^2}}
%\newcommand{\Exp}[1]{\exp{\left({#1}\right)}}
%\newcommand{\Norm}[1]{\left\lVert{#1}\right\rVert}
%\newcommand{\Sin}[1]{\sin{\left({#1}\right)}}
%\newcommand{\T}[0]{\text{T}}
%\newcommand{\VolumeOp}[1]{\VName_{#1}}
%\newcommand{\agrad}[0]{\Ba \cdot \nabla}
%\newcommand{\alphacap}[0]{\hat{\boldsymbol{\alpha}}}
%\newcommand{\Fcap}[0]{\hat{\BF}}
%\newcommand{\bithree}[0]{{\Bi}_3}
%\newcommand{\bxa}[0]{\Bx\Ba}
%\newcommand{\coordvec}[2]{
%\newcommand{\costheta}[0]{\acap \cdot \xcap}
%\newcommand{\ddt}[1]{\ddot{#1}}
%\newcommand{\ddu}[1] {\frac {d{#1}} {du}}
%\newcommand{\dsqxj}[2] {\frac {\partial^2 {#1}} {\partial {x_{#2}}^2}}
%\newcommand{\dtheta}[1]{\frac{d {#1}}{d \theta}}
%\newcommand{\dt}[1]{\dot{#1}}
%\newcommand{\dt}[1]{\frac{d {#1}}{dt}}
%\newcommand{\dxj}[2] {\frac {\partial {#1}} {\partial {x_{#2}}}}
%\newcommand{\halfPhi}[0]{\frac{\phi}{2}}
%\newcommand{\half}[0]{\inv{2}}
%\newcommand{\inv}[1]{\frac{1}{#1}}
%\newcommand{\laplacian}[0]{\nabla^2}
%\newcommand{\matrixoftx}[3]{
%\newcommand{\nrrp}[0]{\norm{\rcap \wedge \Br'}}
%\newcommand{\oiint}{\bigcirc \hspace{-1.4em} \int \hspace{-.8em} \int}
%\newcommand{\transpose}[1]{{#1}^{\text{T}}}
%\newcommand{\transpose}[1]{{{#1}^{\TextTranspose}}}
%\newcommand{\transpose}[1]{{{#1}^{\text{T}}}}
%\newcommand{\barA}[0]{\bar{A}}
%\newcommand{\qbar}[0]{\bar{q}}
%\newcommand{\qdotbar}[0]{\dot{\bar{q}}}
%
%</infrequent>





\usepackage[bookmarks=true]{hyperref}

\usepackage{color,cite,graphicx}
   % use colour in the document, put your citations as [1-4]
   % rather than [1,2,3,4] (it looks nicer, and the extended LaTeX2e
   % graphics package. 
\usepackage{latexsym,amssymb,epsf} % don't remember if these are
   % needed, but their inclusion can't do any damage


\title{ Poynting vector and Electromagnetic Energy conservation. }
\author{Peeter Joot}
\date{ Dec 29, 2008.  Last Revision: $Date: 2008/12/31 01:06:10 $ }

\begin{document}

\maketitle{}

\tableofcontents

\section{ Motivation. }

Clarify Poynting discussion from \cite{doran2003gap}.

Equation 7.59 and 7.60 derives a $\BE \cross \BB$ quantity, the Poynting vector, as a sort of energy flux through the surface of the containing volume.

There are a couple of magic steps here that were not at all obvious to me.  Go through this in enough detail that it makes sense to me.

\section{ Charge free case. }

In SI units the Energy density is given as

\begin{align*}
U = \frac{\epsilon_0}{2}\left( \BE^2 + c^2 \BB^2 \right)
\end{align*}

FIXME: Don't truely understand where this part comes from.  The article \href{http://farside.ph.utexas.edu/teaching/em/lectures/node89.html}{Energy Conservation} looks promising to study this.

Given this energy density the rate of change of energy in a volume is then

\begin{align*}
\frac{dU}{dt} 
&= 
\frac{d}{dt} 
\frac{\epsilon_0}{2} \int dV \left( \BE^2 + c^2 \BB^2 \right) \\
&= 
\epsilon_0 \int dV \left( \BE \cdot \PD{t}{\BE} + c^2 \BB \cdot \PD{t}{\BB} \right) \\
\end{align*}

The next (omitted in the text) step is to utilize Maxwell's equation to eliminate the time derivatives.  Since this is the
charge and current free case, we can write Maxwell's as

\begin{align*}
0
&= \gamma_0 \grad F \\
&= \gamma_0 (\gamma^0 \partial_0 + \gamma^k \partial_k) F \\
&= (\partial_0 + \gamma_k\gamma_0 \partial_k) F \\
&= (\partial_0 + \sigma_k \partial_k) F \\
&= (\partial_0 + \spacegrad)F \\
&= (\partial_0 + \spacegrad)(\BE + ic \BB) \\
&= \partial_0 \BE + ic \partial_0 \BB + \spacegrad \BE + ic \spacegrad \BB \\
\end{align*}

In the spatial ($\sigma$) basis we can separate this into even and odd grades, which are separately equal to zero

\begin{align*}
0 &= \partial_0 \BE + ic \spacegrad \BB \\
%   1                  3,1 
0 &= ic \partial_0 \BB + \spacegrad \BE 
%  2                    0,2
\end{align*}

A selection of just the vector parts is

\begin{align*}
\partial_t \BE &= - ic^2 \spacegrad \wedge \BB \\
\partial_t \BB &= i\spacegrad \wedge \BE 
\end{align*}

Which can be back substituited into the energy flux
\begin{align*}
\frac{dU}{dt} 
&= \epsilon_0 \int dV \left( \BE \cdot (-i c^2 \spacegrad \wedge \BB) + c^2 \BB \cdot (i \spacegrad \wedge \BE) \right) \\
&= \epsilon_0 c^2 \int dV \gpgradezero{ \BB i \spacegrad \wedge \BE -\BE i \spacegrad \wedge \BB } \\
\end{align*}

Since the two divergence terms are zero we can drop the wedges here for

\begin{align*}
\frac{dU}{dt} 
&= \epsilon_0 c^2 \int dV \gpgradezero{ \BB i \spacegrad \BE -\BE i \spacegrad \BB } \\
&= \epsilon_0 c^2 \int dV \gpgradezero{ (i \BB) \spacegrad \BE -\BE \spacegrad (i\BB) } \\
&= \epsilon_0 c^2 \int dV \spacegrad \cdot ( (i \BB) \cdot \BE ) \\
\end{align*}

Justification for this last step can be found below in the derivation of equation \ref{eqn:poyntingDivergence}.

We can now use Stokes theorem to change this into a surface integral for a final energy flux 

\begin{align*}
\frac{dU}{dt} 
&= \epsilon_0 c^2 \int d\BA \cdot ( (i \BB) \cdot \BE ) \\
\end{align*}

This last bivector/vector dot product is the Poynting vector

\begin{align*}
(i \BB) \cdot \BE 
&= \gpgradeone{ (i \BB) \cdot \BE } \\
&= \gpgradeone{ i \BB \BE } \\
&= \gpgradeone{ i (\BB \wedge \BE) } \\
&= i (\BB \wedge \BE) \\
&= i^2(\BB \cross \BE) \\
&= \BE \cross \BB \\
\end{align*}

So, we can identity the quantity 

\begin{align}\label{eqn:poynting}
\epsilon_0 c^2 \BE \cross \BB = \epsilon_0 c (i c \BB) \cdot \BE 
\end{align}

As a directed energy density flux through the surface of a containing volume.


\section{ With charges and currents }
 
To calculate time derivatives we want to take Maxwell's equation and put into a form with explicit time derivatives, as was done before, but this time be more careful with the handling of the four vector current term.  Starting with left factoring out of a $\gamma_0$ from the spacetime gradient. 
 
\begin{align*}
\grad &= \gamma^0 \partial_0 + \gamma^k \partial_k \\
&= \gamma^0 (\partial_0 - \gamma^k \gamma_0 \partial_k) \\
&= \gamma^0 (\partial_0 + \sigma_k \partial_k) \\
\end{align*}

Similarily, the $\gamma_0$ can be factored from the current density

\begin{align*}
J 
&= \gamma_0 c \rho + \gamma_k J^k \\
&= \gamma_0 (c \rho - \gamma_k \gamma_0 J^k) \\
&= \gamma_0 (c \rho - \sigma_k J^k) \\
&= \gamma_0 (c \rho - \Bj )
\end{align*}

With this Maxwell's equation becomes
 
\begin{align*}
\gamma_0 \grad F &= \gamma_0 J / \epsilon_0 c \\
(\partial_0 + \spacegrad) ( \BE + i c \BB ) &= \rho/\epsilon_0 - \Bj/\epsilon_0 c \\
\end{align*}
 
A split into even and odd grades including current and charge density is thus
 
\begin{align*}
\spacegrad \BE + \partial_t (i \BB) &= \rho/\epsilon_0 \\
\spacegrad (i \BB) c^2 + \partial_t \BE &= -\Bj/\epsilon_0
\end{align*}
 
Now, taking time derivatives of the energy density gives

\begin{align*}
\PD{t}{U} 
&= \PD{t}{}\inv{2} \epsilon_0 \left( \BE^2 - (ic \BB)^2 \right) \\
&= \epsilon_0 \left( \BE \cdot \partial_t \BE - c^2 (i\BB) \cdot \partial_t (i\BB) \right) \\
&= \epsilon_0 \gpgradezero{ \BE ( -\Bj/\epsilon_0 -\spacegrad (i \BB) c^2 ) - c^2 (i\BB) ( -\spacegrad \BE + \rho/\epsilon_0 ) } \\
&= -\BE \cdot \Bj + c^2 \epsilon_0 \gpgradezero{ i\BB \spacegrad \BE -\BE \spacegrad (i \BB) } \\
&= -\BE \cdot \Bj + c^2 \epsilon_0 \left( (i\BB) \cdot (\spacegrad \wedge \BE) - \BE \cdot (\spacegrad \cdot (i \BB)) \right) \\
\end{align*}

Using equation \ref{eqn:poyntingDivergence}, we now have the rate of change of
field energy for the general case including currents.  That is

\begin{align}
\PD{t}{U} &= -\BE \cdot \Bj + c^2 \epsilon_0 \spacegrad \cdot (\BE \cdot (i\BB)) 
\end{align}

Written out in full, and in terms of the Poynting vector this is

\begin{align}
\PD{t}{}\frac{\epsilon_0}{2} \left(\BE^2 + c^2 \BB^2\right) + c^2 \epsilon_0 \spacegrad \cdot (\BE \cross \BB) &= -\BE \cdot \Bj 
\end{align}

\section{ Poynting vector in terms of complete field. }

In equation \ref{eqn:poynting} the individual parts of the complete Faraday
bivector $F = \BE + i c \BB$ stand out.  How would the Poynting vector be
expressed in terms of $F$ or in tensor form?

Since
\begin{align*}
F \gamma_0 = - \gamma_0(\BE - i c \BB)
\end{align*}

we have
\begin{align*}
\gamma^0 F \gamma_0 = - (\BE - i c \BB)
\end{align*}

and
\begin{align*}
i c \BB &= \inv{2}(F + \gamma^0 F \gamma_0) \\
\BE &= \inv{2}(F - \gamma^0 F \gamma_0) \\
\end{align*}

FIXME: tried using these but messed up.

%Without justifying all the steps I think that the following is valid
%
%\begin{align*}
%(i c \BB) \cdot \BE 
%&= \gpgradeone{(i c \BB) \cdot \BE } \\
%&= \gpgradeone{i c \BB \BE } \\
%&= \inv{4} (F + \gamma_0 F \gamma_0) \cdot (F - \gamma_0 F \gamma_0) \\
%&= \inv{4} (F^2 - \gamma_0 F \gamma_0 \gamma_0 F \gamma_0 + (\gamma_0 F \gamma_0) \cdot F - F \cdot (\gamma_0 F \gamma_0) ) \\
%&= \inv{4} ( (\gamma_0 F \gamma_0) \cdot F - F \cdot (\gamma_0 F \gamma_0) ) \\
%&= \inv{2} (\gamma_0 F \gamma_0) \cdot F 
%\end{align*}
%
%  The above is wrong.  This is - <F^\dagger F>/2, which is c^2 B^2 - E^2, which isn't even vector.

\section{ Energy Density from Lagrangian. }

I didn't get too far trying to calculate the electrodynamic Hamiltonian density for the general case, so I tried it for a very 
simple special case, with just an electric field component in one direction:

\begin{align*}
\mathcal{L}
&= \frac{1}{2}(E_x)^2 \\
&= \frac{1}{2}(F_{01})^2 \\
&= \frac{1}{2}(\partial_0 A_1 - \partial_1 A_0)^2 \\
\end{align*}

Goldstein gives the Hamiltonian density as

\begin{align*}
\pi &= \frac{\partial \mathcal{L}}{\partial \dot{n}} \\
\mathcal{H} &= \dot{n} \pi - \mathcal{L}
\end{align*}

If I try calculating this I get

\begin{align*}
\pi 
&= \frac{\partial}{\partial (\partial_0 A_1)} \left( \frac{1}{2}(\partial_0 A_1 - \partial_1 A_0)^2 \right) \\
&= \partial_0 A_1 - \partial_1 A_0 \\
&= F_{01} \\
\end{align*}

So this gives a Hamiltonian of
\begin{align*}
\mathcal{H}
&= \partial_0 A_1 F_{01} - \frac{1}{2}(\partial_0 A_1 - \partial_1 A_0)F_{01} \\
&= \frac{1}{2} (\partial_0 A_1 + \partial_1 A_0 )F_{01} 
&= \frac{1}{2} ((\partial_0 A_1)^2 - (\partial_1 A_0)^2 )
\end{align*}

For a Lagrangian density of $E^2 - B^2$ we have an energy density of $E^2 + B^2$, so I'd have expected the Hamiltonian density here to stay equal to $E_x^2/2$, but it 
doesn't look like that's what I get (what I calculated isn't at all familiar seeming).

If I haven't made a mistake here, perhaps I'm incorrect in assuming that the Hamiltonian density of the electrodynamic Lagrangian should be the energy density?

\section{ Appendix.  Messy details. }

For both the charge and the charge free case, we need a proof of 

\begin{align*}
(i\BB) \cdot (\spacegrad \wedge \BE) - \BE \cdot (\spacegrad \cdot (i \BB)) 
&= \spacegrad \cdot (\BE \cdot (i\BB)) 
\end{align*}

This is relativity straightforward, albeit tedious, to do backwards.

\begin{align*}
\spacegrad \cdot ((i \BB) \cdot \BE)
&= \gpgradezero{ \spacegrad ((i \BB) \cdot \BE)} \\
&= \inv{2} \gpgradezero{ \spacegrad ( i \BB \BE - \BE i \BB ) } \\
&= \inv{2} \gpgradezero{ 
  \dot{\spacegrad} i \dot{\BB} \BE 
+ \dot{\spacegrad} i \BB \dot{\BE}
- \dot{\spacegrad} \dot{\BE} i \BB 
- \dot{\spacegrad} \BE i \dot{\BB}
} \\
&= \inv{2} \gpgradezero{ 
  \BE \spacegrad (i \BB) - (i\dot{\BB}) \dot{\spacegrad} \BE
+ \dot{\BE} \dot{\spacegrad} i \BB - i \BB \spacegrad \BE
} \\
&= \inv{2} \left(
  \BE \cdot (\spacegrad \cdot (i \BB)) - ((i\dot{\BB}) \cdot \dot{\spacegrad}) \cdot \BE
+ (\dot{\BE} \wedge \dot{\spacegrad}) \cdot (i \BB) - (i \BB) \cdot (\spacegrad \wedge \BE) 
\right)
\\
\end{align*}

Grouping the two sets of repeated terms after reordering and the associated sign adjustments we have

\begin{align}\label{eqn:poyntingDivergence}
\spacegrad \cdot ((i \BB) \cdot \BE) &= \BE \cdot (\spacegrad \cdot (i \BB)) - (i \BB) \cdot (\spacegrad \wedge \BE)
\end{align}

which is the desired identity (in negated form) that was to be proved.

There is likely some theorem that could be used to avoid some of this algebra.

\bibliographystyle{plainnat}
\bibliography{myrefs}

\end{document}

\include{poynting_rate}
\include{electric_field_energy}
\include{energy_momentum_tensor}
\include{em_wave}
\include{rayleigh_jeans}
\part{Quantum Mechanics.}
%
% Copyright � 2012 Peeter Joot.  All Rights Reserved.
% Licenced as described in the file LICENSE under the root directory of this GIT repository.
%

%
%
\chapter{Bohr Model}
\index{Bohr model}
\label{chap:bohr}
%\date{Dec 11, 2008.  bohr.tex}

\section{Motivation}

The Bohr model is taught as early as high school chemistry when the various orbitals are
discussed (or maybe it was high school physics).  I recall
that the first time I saw this I did not see where all the ideas came from.
With a bit more math under my belt now, reexamine these ideas as a lead up to
the proper wave mechanics.

\section{Calculations}

\subsection{Equations of motion}

A prerequisite to discussing electron orbits is first setting up the equations of motion
for the two charged particles (ie: the proton and electron).

With the proton position at \(\Br_p\), and the electron at \(\Br_e\), we have two equations, one
for the force on the proton from the electron and the other for the force on the proton from
the electron.  These are respectively

\begin{equation}\label{eqn:bohr:chargeEquations}
\begin{aligned}
  \K e^2 \frac { \Br_e - \Br_p } { \Abs{\Br_e - \Br_p}^3 } &= m_p \frac{d^2 \Br_p }{dt^2} \\
- \K e^2 \frac { \Br_e - \Br_p } { \Abs{\Br_e - \Br_p}^3 } &= m_e \frac{d^2 \Br_e }{dt^2}
\end{aligned}
\end{equation}

In lieu of a picture, setting \(\Br_p = 0\) works to check signs, leaving an inwards force on the electron as desired.

% FIXME: Add one.
%\begin{figure}[htp]
%\centering
%\includegraphics[totalheight=0.4\textheight]{picturepath}
%\caption{My Caption}\label{fig:pictlabel}
%\end{figure}
%
%... see \cref{fig:picturepath} ...

As usual for a two body problem, use of the difference vector and center of mass vector is desirable.  That is

\begin{equation}\label{eqn:bohr:20}
\begin{aligned}
\Bx &= \Br_e - \Br_p \\
M &= m_e + m_p \\
\BR &= \inv{M}(m_e \Br_e + m_p \Br_p)
\end{aligned}
\end{equation}

Solving for \(\Br_p\) and \(\Br_e\) in terms of \(\BR\) and \(\Bx\) we have

\begin{equation}\label{eqn:bohr:40}
\begin{aligned}
\Br_e &= \frac{m_p}{M} \Bx + \BR \\
\Br_p &= \frac{-m_e}{M} \Bx + \BR \\
\end{aligned}
\end{equation}

% check:
%r_e - r_p = M/M x
%m_e \Br_e + m_p \Br_p &= \frac{-m_e m_p}{M} \Bx + m_e \BR + \frac{m_p m_e}{M} \Bx + m_p \BR \\

Substitution back into \eqnref{eqn:bohr:chargeEquations} we have

\begin{equation}\label{eqn:bohr:60}
\begin{aligned}
  \K e^2 \frac {\Bx} { \Abs{\Bx}^3 } &= m_p \frac{d^2}{dt^2}\left( \frac{-m_e}{M} \Bx + \BR \right) \\
 -\K e^2 \frac {\Bx} { \Abs{\Bx}^3 } &= m_e \frac{d^2}{dt^2}\left( \frac{m_p}{M} \Bx + \BR \right),
\end{aligned}
\end{equation}

and sums and (scaled) differences of that give us our reduced mass equation and constant center-of-mass velocity equation
\begin{equation}\label{eqn:bohr:80}
\begin{aligned}
\frac{d^2 \Bx}{dt^2} &= -\K e^2 \frac {\Bx} { \Abs{\Bx}^3 } \left( \inv{m_e} + \inv{m_p} \right) \\
\frac{d^2 \BR}{dt^2} &= 0
\end{aligned}
\end{equation}

writing \(1/\mu = 1/m_e + 1/m_p\), and \(k = e^2/4 \pi \epsilon_0\), our difference vector equation is thus

\begin{equation}\label{eqn:bohr:reduceEOM}
\begin{aligned}
\mu \frac{d^2 \Bx}{dt^2} &= -k \frac {\Bx} { \Abs{\Bx}^3 }
\end{aligned}
\end{equation}

\subsection{Circular solution}

The Bohr model postulates that electron orbits are circular.  It is easy enough to verify that a circular orbit in the center of mass frame is a solution to equation
\eqnref{eqn:bohr:reduceEOM}.   Write the path in terms of the unit bivector for the plane of rotation \(i\) and an initial vector position \(\Bx_0\)

\begin{equation}\label{eqn:bohr:circular}
\begin{aligned}
\Bx = \Bx_0 e^{i \omega t}
\end{aligned}
\end{equation}

For constant \(i\) and \(\omega\), we have

\begin{equation}\label{eqn:bohr:100}
\begin{aligned}
\mu \Bx_0 (i\omega)^2 e^{i\omega t} = - k \frac{\Bx_0}{\Abs{\Bx_0}^3} e^{i\omega t}
\end{aligned}
\end{equation}

This provides the
angular velocity in terms of the reduced mass of the system and the charge constants

\begin{equation}\label{eqn:bohr:omegaSquared}
\begin{aligned}
\omega^2 = \frac{k}{\mu \Abs{\Bx_0}^3} = \frac{e^2}{4 \pi \epsilon_0 \mu \Abs{\Bx_0}^3}.
\end{aligned}
\end{equation}

Although not relevant to the quantum theme, it is hard not to call out the observation that this is
a Kepler's law like relation for the period of the circular orbit given the radial distance from the center of mass

\begin{equation}\label{eqn:bohr:120}
\begin{aligned}
T^2 = \frac{16 \pi^3 \epsilon_0 \mu}{e^2} \Abs{\Bx_0}^3
\end{aligned}
\end{equation}

Kepler's law also holds for elliptical orbits, but this takes more work to show.

\subsection{Angular momentum conservation}
\index{angular momentum conservation}

Now, the next step in the Bohr argument was that the angular momentum, a conserved quantity is also quantized.  To give real
meaning to the conservation statement we need the equivalent Lagrangian formulation of \eqnref{eqn:bohr:reduceEOM}.  Anti-differentiation
gives

\begin{equation}\label{eqn:bohr:140}
\begin{aligned}
\grad_\Bv \left( \inv{2} \mu \Bv^2 \right)
&= k \xcap \partial_x \inv{x} \\
&= - \grad_\Bx
\mathLabelBox
[
   labelstyle={below of=m\themathLableNode, below of=m\themathLableNode}
]
{\left(- k\inv{\Abs{\Bx}}\right)}{\(=\phi\)}
\end{aligned}
\end{equation}

So, our Lagrangian is
\begin{equation}\label{eqn:bohr:160}
\begin{aligned}
\LL = K - \phi = \inv{2} \mu \Bv^2 + k \inv{\Abs{\Bx}}
\end{aligned}
\end{equation}

The essence of the conservation argument, an application of
Noether's theorem,
is that a rotational transformation of the Lagrangian leaves this energy relationship unchanged.  Repeating
the angular momentum example from \citep{classicalmechanics:PJEulerLagrange} (which was done for the more general case of any radial potential), we
write \(\hat{B}\) for the unit bivector associated with a rotational plane.  The position vector is transformed by rotation in this plane as follows

\begin{equation}\label{eqn:bohr:180}
\begin{aligned}
\Bx &\rightarrow \Bx' \\
\Bx' &= R \Bx R^\dagger \\
R &= \exp{\hat{B}\theta/2}
\end{aligned}
\end{equation}

The magnitude of the position vector is rotation invariant

\begin{equation}\label{eqn:bohr:200}
\begin{aligned}
(\Bx')^2 &= R \Bx R^\dagger R \Bx R^\dagger = \Bx^2,
\end{aligned}
\end{equation}

as is our the square of the transformed velocity.  The transformed velocity is

\begin{equation}\label{eqn:bohr:220}
\begin{aligned}
\frac{d\Bx'}{dt} &= \dot{R} \Bx R + R \dot{\Bx} R^\dagger + R \Bx \dot{R}^\dagger
\end{aligned}
\end{equation}

but with \(\dot{\theta} = 0\), \(\dot{R} = 0\) its square is just

\begin{equation}\label{eqn:bohr:240}
\begin{aligned}
(\Bv')^2 &= R {\Bv} R^\dagger R \dot{\Bv} R^\dagger = \Bv^2.
\end{aligned}
\end{equation}

We therefore have a Lagrangian that is invariant under this rotational transformation

\begin{equation}\label{eqn:bohr:260}
\begin{aligned}
\LL \rightarrow \LL' = \LL,
\end{aligned}
\end{equation}

and by Noether's theorem (essentially application of the chain rule), we have

\begin{equation}\label{eqn:bohr:280}
\begin{aligned}
\frac{d\LL'}{d\theta}
&= \frac{d}{dt} \left(\frac{d\Bx'}{d\theta} \cdot \grad_{\Bv'} \LL \right) \\
&= \frac{d}{dt} \left( (\hat{B} \cdot \Bx') \cdot \mu \Bv' \right).
\end{aligned}
\end{equation}

But \(d\LL'/d\theta = 0\), so we have for any \(\hat{B}\)

\begin{equation}\label{eqn:bohr:300}
\begin{aligned}
(\hat{B} \cdot \Bx') \cdot (\mu \Bv') &= \hat{B} \cdot (\Bx' \wedge (\mu \Bv')) = \text{constant}
\end{aligned}
\end{equation}

Dropping primes this is

\begin{equation}\label{eqn:bohr:320}
\begin{aligned}
L = \Bx \wedge (\mu \Bv) = \text{constant},
\end{aligned}
\end{equation}

a constant bivector for the conserved center of mass (reduced-mass) angular momentum associated with the Lagrangian of this system.

\subsection{Quantized angular momentum for circular solution}

In terms of the circular solution of \eqnref{eqn:bohr:circular} the angular momentum bivector is

\begin{equation}\label{eqn:bohr:340}
\begin{aligned}
L = \Bx \wedge (\mu \Bv)
&= \gpgradetwo{ \Bx_0 e^{i \omega t} \mu \Bx_0 i \omega e^{i \omega t} } \\
&= \gpgradetwo{ e^{-i \omega t} \Bx_0 \mu \Bx_0 \omega e^{i \omega t} i } \\
&= (\Bx_0)^2 \mu \omega i \\
%&= i \frac{e \mu}{2} \sqrt{\frac{\Abs{\Bx_0}}{\pi \epsilon_0 \mu}} \\
&= i e \sqrt{\frac{\mu \Abs{\Bx_0}}{4 \pi \epsilon_0}}
\end{aligned}
\end{equation}

%\begin{align}\label{eqn:bohr:omegaSquared}
%\omega = \frac{e}{2 \sqrt{\pi \epsilon_0}} \Abs{\Bx_0}^{-3/2}

Now if this angular momentum is quantized with quantum magnitude \(l\) we have we have for the bivector angular momentum the values

\begin{equation}\label{eqn:bohr:360}
\begin{aligned}
L = i n l = i e \sqrt{\frac{\mu \Abs{\Bx_0}}{4 \pi \epsilon_0}}
\end{aligned}
\end{equation}

Which with \(l = \Hbar\) (where experiment in the form of the spectral hydrogen line values is required to fix this constant and relate it to Plank's black body constant)
is the momentum equation in terms of
the Bohr radius \(\Bx_0\) at each energy level.  Writing that radius \(r_n = \Abs{\Bx_0}\) explicitly as a function of n, we have

\begin{equation}\label{eqn:bohr:380}
\begin{aligned}
r_n = \frac{4 \pi \epsilon_0}{\mu} \left(\frac{n \Hbar}{e}\right)^2
\end{aligned}
\end{equation}

\subsubsection{Velocity}

One of the assumptions of this treatment is a \(\Abs{\Bv_e} << c\) requirement so that Coulombs law is valid (ie: slow enough that all the other Maxwell's equations can be neglected).
Let us evaluate the velocity numerically at the some of the quantization levels and see how this compares to the speed of light.

First we need an expression for the velocity itself.  This is

\begin{equation}\label{eqn:bohr:400}
\begin{aligned}
\Bv^2
&= ( \Bx_0 i \omega e^{i \omega t} )^2 \\
&= \frac{e^2}{4 \pi \epsilon_0 \mu r_n} \\
&= \frac{e^4}{(4 \pi \epsilon_0)^2 (n \Hbar)^2}.
\end{aligned}
\end{equation}

For
\begin{equation}\label{eqn:bohr:420}
\begin{aligned}
v_n
&= \frac{e^2}{4 \pi \epsilon_0 n \Hbar} \\
&= 2.1 \times 10^6 m/s
\end{aligned}
\end{equation}

This is the \(1/137\) of the speed of light value that one sees googling electron speed in hydrogen, and only decreases with quantum number so the non-relativistic speed approximation holds
(\(\gamma = 1.00002663\)).  This speed is still pretty zippy, even if it is not relativistic, so it is not unreasonable to attempt to repeat this treatment trying to incorporate the remainder
of Maxwell's equations.

Interestingly the velocity is not a function of the reduced mass at all, but just the charge and quantum numbers.  One also gets a good hint at why the Bohr theory breaks down
for larger atoms.  An electron in circular orbit around an ion of Gold would have a velocity of \(79/137\) the speed of light!

% google calculator:
%1/sqrt(1- ((elementary charge)^2 / 4 / pi / epsilon_0 /hbar/c)^2)

% - discuss connection to Sch. results?
% - try: proper maxwell's/Lorentz equations instead of just the Coulomb force.

\include{sch_current}
\include{dirac_lagrangian}
\include{pauli_matrix}
%
% Copyright � 2012 Peeter Joot.  All Rights Reserved.
% Licenced as described in the file LICENSE under the root directory of this GIT repository.
%

% 
% 
\chapter{Gamma Matrices}\label{chap:PJDiracGamma}
\index{gamma matrices}
%\date{Dec 13, 2008.  gamma.tex}

\section{Dirac matrices}

\index{Dirac!matrix}
\index{gamma matrix}
The Dirac matrices \(\gamma^\mu\) can be used as a Minkowski basis.  The basic defining relationship is the Minkowski metric, where the dot products satisfy

\begin{equation}\label{eqn:gamma:20}
\begin{aligned}
\scalarProduct{\gamma^\mu}{\gamma^\nu} &= \pm \delta_{\mu\nu} \\
(\scalarProduct{\gamma^0}{\gamma^0})(\scalarProduct{\gamma^a}{\gamma^a}) &= -1 \quad \text{where \(a \in \{1,2,3\}\)}
\end{aligned}
\end{equation}

There is freedom to pick the positive square for either \(\gamma^0\) or \(\gamma^a\), and both conventions are common.

One of the matrix representations for these vectors listed in the 
\href{http://en.wikipedia.org/wiki/Gamma_matrices}{Dirac matrix wikipedia article}
is

\begin{equation}\label{eqn:gamma:basis}
\begin{aligned}
\gamma^0 &= \begin{bmatrix}
 1  &  0  &  0  &  0  \\
 0  &  1  &  0  &  0  \\
 0  &  0  &  -1  &  0  \\
 0  &  0  &  0  &  -1  \\
\end{bmatrix} \quad
\gamma^1 = \begin{bmatrix}
 0  &  0  &  0  &  1  \\
 0  &  0  &  1  &  0  \\
 0  &  -1  &  0  &  0  \\
 -1  &  0  &  0  &  0  \\
\end{bmatrix} \\
\gamma^2 &= \begin{bmatrix}
 0  &  0  &  0  &  -i  \\
 0  &  0  &  i  &  0  \\
 0  &  i  &  0  &  0  \\
 -i  &  0  &  0  &  0  \\
\end{bmatrix}
\quad \gamma^3 = \begin{bmatrix}
 0  &  0  &  1  &  0  \\
 0  &  0  &  0  &  -1  \\
 -1  &  0  &  0  &  0  \\
 0  &  1  &  0  &  0  \\
\end{bmatrix}
\end{aligned}
\end{equation}

For this particular basis we have a \(+---\) metric signature.  In the matrix form this takes the specific meaning that \((\gamma^0)^2 = I\), and \((\gamma^a)^2 = -I\).

A table of all the possible product variants of \eqnref{eqn:gamma:basis} can be found below in the appendix.

\subsection{anticommutator product}
\index{anticommutator}

Noting that the matrices square in the fashion just described and that they reverse sign when multiplication order is reversed allows for summarizing the dot products relationships as follows

\begin{equation}\label{eqn:gamma:symmetric}
\begin{aligned}
\symmetric{\gamma^\mu}{\gamma^\nu} 
&= {\gamma^\mu}{\gamma^\nu} + {\gamma^\nu}{\gamma^\mu} \\
%&= 2 (\scalarProduct{\gamma^\mu}{\gamma^\nu}) I \\
&= 2 \eta^{\mu\nu} I,
\end{aligned}
\end{equation}

where the metric tensor \(\eta^{\mu\nu} = \scalarProduct{\gamma^\mu}{\gamma^\nu}\) is commonly summarized as coordinates of a matrix as in

\begin{equation}\label{eqn:gamma:40}
\begin{aligned}
\begin{bmatrix}
\eta^{\mu\nu}
\end{bmatrix}
&=
\begin{bmatrix}
1 & 0 & 0 & 0 \\
0 & -1 & 0 & 0 \\
0 & 0 & -1 & 0 \\
0 & 0 & 0 & -1 \\
\end{bmatrix}
\end{aligned}
\end{equation}

The relationship \eqnref{eqn:gamma:symmetric} is taken as the defining relationship for the Dirac matrices, but can be seen to be just a matricized statement of the Clifford vector dot product.

\subsection{Written as Pauli matrices}
\index{Pauli matrices}

Using the Pauli matrices

\begin{equation}\label{eqn:gamma:60}
\begin{aligned}
\sigma_1 = \PauliX \quad \sigma_2 = \PauliY \quad \sigma_3 = \PauliZ
\end{aligned}
\end{equation}

one can write the Dirac matrices and all their products (reading from the multiplication table) more concisely as

\begin{equation}\label{eqn:gamma:80}
\begin{aligned}
\gamma^0 &= 
\begin{bmatrix}
I & 0 \\
0 & -I
\end{bmatrix} \\
\gamma^a &= 
\begin{bmatrix}
0 & \sigma_a \\
-\sigma_a & 0 \\
\end{bmatrix} \\
\gamma^0 \gamma^a &=
\begin{bmatrix}
0 & \sigma_a \\
\sigma_a & 0 \\
\end{bmatrix} \\
\gamma^a \gamma^b &=
- i \epsilon_{a b c}
\begin{bmatrix}
\sigma_c & 0 \\
0 & \sigma_c \\
\end{bmatrix} \\
\gamma^1 \gamma^2 \gamma^3 &= i 
\begin{bmatrix}
0 & -I \\
I & 0
\end{bmatrix} \\
\gamma^0 \gamma^1 \gamma^2 &= i 
\begin{bmatrix}
-\sigma_1 & 0 \\
0 & \sigma_1 \\
\end{bmatrix} \\
\gamma^3 \gamma^0 \gamma^1 &= i 
\begin{bmatrix}
\sigma_2 & 0 \\
0 & -\sigma_2 \\
\end{bmatrix} \\
\gamma^0 \gamma^1 \gamma^2 &= i 
\begin{bmatrix}
-\sigma_3 & 0 \\
0 & \sigma_3 \\
\end{bmatrix}
\end{aligned}
\end{equation}

\subsection{Deriving properties using the Pauli matrices}

From the multiplication table a number of properties can be observed.  Using the Pauli matrices one can arrive at these more directly using the multiplication identity for those
matrices

\begin{equation}\label{eqn:gamma:100}
\begin{aligned}
\sigma_a \sigma_b = 2 i \epsilon_{abc} \sigma_c
\end{aligned}
\end{equation}

Actually taking the time to type this out in full does not seem worthwhile and is a fairly straightforward exercise.

\subsection{Conjugation behavior}
\index{conjugation}

Unlike the Pauli matrices, the Dirac matrices do not split nicely via conjugation.  Instead we have the time basis vector and its dual are Hermitian

\begin{equation}\label{eqn:gamma:120}
\begin{aligned}
(\gamma^0)^\conj &= \gamma^0 \\
(\gamma^1 \gamma^2 \gamma^3)^\conj &= \gamma^1 \gamma^2 \gamma^3
\end{aligned}
\end{equation}

whereas the spacelike basis vectors and their duals are all anti-Hermitian

\begin{equation}\label{eqn:gamma:140}
\begin{aligned}
(\gamma^a)^\conj &= -\gamma^a \\
(\gamma^a \gamma^b \gamma^c)^\conj &= - \gamma^a \gamma^b \gamma^c.
\end{aligned}
\end{equation}

For the scalar and the pseudoscalar parts we have a Hermitian split

\begin{equation}\label{eqn:gamma:160}
\begin{aligned}
I^\conj &= I \\
(\gamma^0 \gamma^1 \gamma^2 \gamma^3)^\conj &= -(\gamma^0 \gamma^1 \gamma^2 \gamma^3)^\conj
\end{aligned}
\end{equation}

and finally, also have a Hermitian split of the bivector parts into spacetime (relative vectors), and the purely spatial bivectors

\begin{equation}\label{eqn:gamma:180}
\begin{aligned}
(\gamma^0 \gamma^a)^\conj &= \gamma^0 \gamma^a \\
(\gamma^a \gamma^b)^\conj &= -\gamma^a \gamma^b
\end{aligned}
\end{equation}

Is there a logical and simple set of matrix operations that splits things nicely into scalar, vector, bivector, trivector, and pseudoscalar parts as there was with the Pauli
matrices?

\section{Appendix.  Table of all generated products}

A small C++ program using boost::numeric::ublas and std::complex,
plus some perl to generate part of that, was
written to generate the multiplication table for the gamma matrix products
for this particular basis.  The metric tensor and the antisymmetry of
the wedge products can be seen from these.

%% <GENERATED>




\begin{equation}\label{eqn:gamma:200}
\begin{aligned}
\gamma^0 \gamma^0 = \begin{bmatrix}
 1  &  0  &  0  &  0  \\
 0  &  1  &  0  &  0  \\
 0  &  0  &  1  &  0  \\
 0  &  0  &  0  &  1  \\
\end{bmatrix} \quad
\gamma^1 \gamma^1 = \begin{bmatrix}
 -1  &  0  &  0  &  0  \\
 0  &  -1  &  0  &  0  \\
 0  &  0  &  -1  &  0  \\
 0  &  0  &  0  &  -1  \\
\end{bmatrix}
\end{aligned}
\end{equation}

\begin{equation}\label{eqn:gamma:220}
\begin{aligned}
\gamma^2 \gamma^2 = \begin{bmatrix}
 -1  &  0  &  0  &  0  \\
 0  &  -1  &  0  &  0  \\
 0  &  0  &  -1  &  0  \\
 0  &  0  &  0  &  -1  \\
\end{bmatrix} \quad
\gamma^3 \gamma^3 = \begin{bmatrix}
 -1  &  0  &  0  &  0  \\
 0  &  -1  &  0  &  0  \\
 0  &  0  &  -1  &  0  \\
 0  &  0  &  0  &  -1  \\
\end{bmatrix}
\end{aligned}
\end{equation}

\begin{equation}\label{eqn:gamma:240}
\begin{aligned}
\gamma^0 \gamma^1 = \begin{bmatrix}
 0  &  0  &  0  &  1  \\
 0  &  0  &  1  &  0  \\
 0  &  1  &  0  &  0  \\
 1  &  0  &  0  &  0  \\
\end{bmatrix} \quad
\gamma^1 \gamma^0 = \begin{bmatrix}
 0  &  0  &  0  &  -1  \\
 0  &  0  &  -1  &  0  \\
 0  &  -1  &  0  &  0  \\
 -1  &  0  &  0  &  0  \\
\end{bmatrix}
\end{aligned}
\end{equation}

\begin{equation}\label{eqn:gamma:260}
\begin{aligned}
\gamma^0 \gamma^2 = \begin{bmatrix}
 0  &  0  &  0  &  -i  \\
 0  &  0  &  i  &  0  \\
 0  &  -i  &  0  &  0  \\
 i  &  0  &  0  &  0  \\
\end{bmatrix} \quad
\gamma^2 \gamma^0 = \begin{bmatrix}
 0  &  0  &  0  &  i  \\
 0  &  0  &  -i  &  0  \\
 0  &  i  &  0  &  0  \\
 -i  &  0  &  0  &  0  \\
\end{bmatrix}
\end{aligned}
\end{equation}

\begin{equation}\label{eqn:gamma:280}
\begin{aligned}
\gamma^0 \gamma^3 = \begin{bmatrix}
 0  &  0  &  1  &  0  \\
 0  &  0  &  0  &  -1  \\
 1  &  0  &  0  &  0  \\
 0  &  -1  &  0  &  0  \\
\end{bmatrix} \quad
\gamma^3 \gamma^0 = \begin{bmatrix}
 0  &  0  &  -1  &  0  \\
 0  &  0  &  0  &  1  \\
 -1  &  0  &  0  &  0  \\
 0  &  1  &  0  &  0  \\
\end{bmatrix}
\end{aligned}
\end{equation}

\begin{equation}\label{eqn:gamma:300}
\begin{aligned}
\gamma^1 \gamma^2 = \begin{bmatrix}
 -i  &  0  &  0  &  0  \\
 0  &  i  &  0  &  0  \\
 0  &  0  &  -i  &  0  \\
 0  &  0  &  0  &  i  \\
\end{bmatrix} \quad
\gamma^2 \gamma^1 = \begin{bmatrix}
 i  &  0  &  0  &  0  \\
 0  &  -i  &  0  &  0  \\
 0  &  0  &  i  &  0  \\
 0  &  0  &  0  &  -i  \\
\end{bmatrix}
\end{aligned}
\end{equation}

\begin{equation}\label{eqn:gamma:320}
\begin{aligned}
\gamma^1 \gamma^3 = \begin{bmatrix}
 0  &  1  &  0  &  0  \\
 -1  &  0  &  0  &  0  \\
 0  &  0  &  0  &  1  \\
 0  &  0  &  -1  &  0  \\
\end{bmatrix} \quad
\gamma^3 \gamma^1 = \begin{bmatrix}
 0  &  -1  &  0  &  0  \\
 1  &  0  &  0  &  0  \\
 0  &  0  &  0  &  -1  \\
 0  &  0  &  1  &  0  \\
\end{bmatrix}
\end{aligned}
\end{equation}

\begin{equation}\label{eqn:gamma:340}
\begin{aligned}
\gamma^2 \gamma^3 = \begin{bmatrix}
 0  &  -i  &  0  &  0  \\
 -i  &  0  &  0  &  0  \\
 0  &  0  &  0  &  -i  \\
 0  &  0  &  -i  &  0  \\
\end{bmatrix} \quad
\gamma^3 \gamma^2 = \begin{bmatrix}
 0  &  i  &  0  &  0  \\
 i  &  0  &  0  &  0  \\
 0  &  0  &  0  &  i  \\
 0  &  0  &  i  &  0  \\
\end{bmatrix}
\end{aligned}
\end{equation}

\begin{equation}\label{eqn:gamma:360}
\begin{aligned}
\gamma^1 \gamma^2 \gamma^3 = \begin{bmatrix}
 0  &  0  &  -i  &  0  \\
 0  &  0  &  0  &  -i  \\
 i  &  0  &  0  &  0  \\
 0  &  i  &  0  &  0  \\
\end{bmatrix} \quad
\gamma^2 \gamma^3 \gamma^0 = \begin{bmatrix}
 0  &  -i  &  0  &  0  \\
 -i  &  0  &  0  &  0  \\
 0  &  0  &  0  &  i  \\
 0  &  0  &  i  &  0  \\
\end{bmatrix}
\end{aligned}
\end{equation}

\begin{equation}\label{eqn:gamma:380}
\begin{aligned}
\gamma^3 \gamma^0 \gamma^1 = \begin{bmatrix}
 0  &  1  &  0  &  0  \\
 -1  &  0  &  0  &  0  \\
 0  &  0  &  0  &  -1  \\
 0  &  0  &  1  &  0  \\
\end{bmatrix} \quad
\gamma^0 \gamma^1 \gamma^2 = \begin{bmatrix}
 -i  &  0  &  0  &  0  \\
 0  &  i  &  0  &  0  \\
 0  &  0  &  i  &  0  \\
 0  &  0  &  0  &  -i  \\
\end{bmatrix}
\end{aligned}
\end{equation}

\begin{equation}\label{eqn:gamma:400}
\begin{aligned}
\gamma^0 \gamma^1 \gamma^2 \gamma^3 = \begin{bmatrix}
 0  &  0  &  -i  &  0  \\
 0  &  0  &  0  &  -i  \\
 -i  &  0  &  0  &  0  \\
 0  &  -i  &  0  &  0  \\
\end{bmatrix}
\end{aligned}
\end{equation}


%% </GENERATED>

\part{Fourier treatments}
\include{heat_fourier}
\documentclass{article}

\usepackage{amsmath}
\usepackage{mathpazo}

%
% shorthand for bold symbols, convenient for vectors and matrices
%
\newcommand{\Ba}[0]{\mathbf{a}}
\newcommand{\Bb}[0]{\mathbf{b}}
\newcommand{\Bc}[0]{\mathbf{c}}
\newcommand{\Bd}[0]{\mathbf{d}}
\newcommand{\Be}[0]{\mathbf{e}}
\newcommand{\Bf}[0]{\mathbf{f}}
\newcommand{\Bg}[0]{\mathbf{g}}
\newcommand{\Bh}[0]{\mathbf{h}}
\newcommand{\Bi}[0]{\mathbf{i}}
\newcommand{\Bj}[0]{\mathbf{j}}
\newcommand{\Bk}[0]{\mathbf{k}}
\newcommand{\Bl}[0]{\mathbf{l}}
\newcommand{\Bm}[0]{\mathbf{m}}
\newcommand{\Bn}[0]{\mathbf{n}}
\newcommand{\Bo}[0]{\mathbf{o}}
\newcommand{\Bp}[0]{\mathbf{p}}
\newcommand{\Bq}[0]{\mathbf{q}}
\newcommand{\Br}[0]{\mathbf{r}}
\newcommand{\Bs}[0]{\mathbf{s}}
\newcommand{\Bt}[0]{\mathbf{t}}
\newcommand{\Bu}[0]{\mathbf{u}}
\newcommand{\Bv}[0]{\mathbf{v}}
\newcommand{\Bw}[0]{\mathbf{w}}
\newcommand{\Bx}[0]{\mathbf{x}}
\newcommand{\By}[0]{\mathbf{y}}
\newcommand{\Bz}[0]{\mathbf{z}}
\newcommand{\BA}[0]{\mathbf{A}}
\newcommand{\BB}[0]{\mathbf{B}}
\newcommand{\BC}[0]{\mathbf{C}}
\newcommand{\BD}[0]{\mathbf{D}}
\newcommand{\BE}[0]{\mathbf{E}}
\newcommand{\BF}[0]{\mathbf{F}}
\newcommand{\BG}[0]{\mathbf{G}}
\newcommand{\BH}[0]{\mathbf{H}}
\newcommand{\BI}[0]{\mathbf{I}}
\newcommand{\BJ}[0]{\mathbf{J}}
\newcommand{\BK}[0]{\mathbf{K}}
\newcommand{\BL}[0]{\mathbf{L}}
\newcommand{\BM}[0]{\mathbf{M}}
\newcommand{\BN}[0]{\mathbf{N}}
\newcommand{\BO}[0]{\mathbf{O}}
\newcommand{\BP}[0]{\mathbf{P}}
\newcommand{\BQ}[0]{\mathbf{Q}}
\newcommand{\BR}[0]{\mathbf{R}}
\newcommand{\BS}[0]{\mathbf{S}}
\newcommand{\BT}[0]{\mathbf{T}}
\newcommand{\BU}[0]{\mathbf{U}}
\newcommand{\BV}[0]{\mathbf{V}}
\newcommand{\BW}[0]{\mathbf{W}}
\newcommand{\BX}[0]{\mathbf{X}}
\newcommand{\BY}[0]{\mathbf{Y}}
\newcommand{\BZ}[0]{\mathbf{Z}}

\newcommand{\Bzero}[0]{\mathbf{0}}
\newcommand{\Btheta}[0]{\boldsymbol{\theta}}
\newcommand{\Btau}[0]{\boldsymbol{\tau}}
\newcommand{\Bomega}[0]{\boldsymbol{\omega}}

%
% shorthand for unit vectors
%
\newcommand{\acap}[0]{\hat{\Ba}}
\newcommand{\bcap}[0]{\hat{\Bb}}
\newcommand{\ccap}[0]{\hat{\Bc}}
\newcommand{\dcap}[0]{\hat{\Bd}}
\newcommand{\ecap}[0]{\hat{\Be}}
\newcommand{\fcap}[0]{\hat{\Bf}}
\newcommand{\gcap}[0]{\hat{\Bg}}
\newcommand{\hcap}[0]{\hat{\Bh}}
\newcommand{\icap}[0]{\hat{\Bi}}
\newcommand{\jcap}[0]{\hat{\Bj}}
\newcommand{\kcap}[0]{\hat{\Bk}}
\newcommand{\lcap}[0]{\hat{\Bl}}
\newcommand{\mcap}[0]{\hat{\Bm}}
\newcommand{\ncap}[0]{\hat{\Bn}}
\newcommand{\ocap}[0]{\hat{\Bo}}
\newcommand{\pcap}[0]{\hat{\Bp}}
\newcommand{\qcap}[0]{\hat{\Bq}}
\newcommand{\rcap}[0]{\hat{\Br}}
\newcommand{\scap}[0]{\hat{\Bs}}
\newcommand{\tcap}[0]{\hat{\Bt}}
\newcommand{\ucap}[0]{\hat{\Bu}}
\newcommand{\vcap}[0]{\hat{\Bv}}
\newcommand{\wcap}[0]{\hat{\Bw}}
\newcommand{\xcap}[0]{\hat{\Bx}}
\newcommand{\ycap}[0]{\hat{\By}}
\newcommand{\zcap}[0]{\hat{\Bz}}
\newcommand{\thetacap}[0]{\hat{\Btheta}}

%
% to write R^n and C^n in a distinguishable fashion.  Perhaps change this
% to the double lined characters upon figuring out how to do so.
%
\newcommand{\C}[1]{$\mathbb{C}^{#1}$}
\newcommand{\R}[1]{$\mathbb{R}^{#1}$}

%
% various generally useful helpers
%

% derivative of #1 wrt. #2:
\newcommand{\D}[2] {\frac {d#2} {d#1}}

\newcommand{\inv}[1]{\frac{1}{#1}}
\newcommand{\cross}[0]{\times}

\newcommand{\abs}[1]{\lvert{#1}\rvert}
\newcommand{\norm}[1]{\lVert{#1}\rVert}
\newcommand{\innerprod}[2]{\langle{#1}, {#2}\rangle}
\newcommand{\dotprod}[2]{{#1} \cdot {#2}}
\newcommand{\bdotprod}[2]{\left({#1} \cdot {#2}\right)}
\newcommand{\crossprod}[2]{{#1} \cross {#2}}
\newcommand{\tripleprod}[3]{\dotprod{\left(\crossprod{#1}{#2}\right)}{#3}}

\DeclareMathOperator{\Proj}{Proj}
\DeclareMathOperator{\Span}{span}
\DeclareMathOperator{\Sgn}{sgn}
\DeclareMathOperator{\Area}{Area}
\DeclareMathOperator{\Volume}{Volume}

%
% A few miscellaneous things specific to this document
%
\newcommand{\crossop}[1]{\crossprod{#1}{}}

% R2 vector.
\newcommand{\VectorTwo}[2]{
\begin{bmatrix}
 {#1} \\
 {#2}
\end{bmatrix}
}

\newcommand{\VectorN}[1]{
\begin{bmatrix}
{#1}_1 \\
{#1}_2 \\
\vdots \\
{#1}_N \\
\end{bmatrix}
}

\newcommand{\DETuvij}[4]{
\begin{vmatrix}
 {#1}_{#3} & {#1}_{#4} \\
 {#2}_{#3} & {#2}_{#4}
\end{vmatrix}
}

\newcommand{\DETuvwijk}[6]{
\begin{vmatrix}
 {#1}_{#4} & {#1}_{#5} & {#1}_{#6} \\
 {#2}_{#4} & {#2}_{#5} & {#2}_{#6} \\
 {#3}_{#4} & {#3}_{#5} & {#3}_{#6}
\end{vmatrix}
}

\newcommand{\DETuvwxijkl}[8]{
\begin{vmatrix}
 {#1}_{#5} & {#1}_{#6} & {#1}_{#7} & {#1}_{#8} \\
 {#2}_{#5} & {#2}_{#6} & {#2}_{#7} & {#2}_{#8} \\
 {#3}_{#5} & {#3}_{#6} & {#3}_{#7} & {#3}_{#8} \\
 {#4}_{#5} & {#4}_{#6} & {#4}_{#7} & {#4}_{#8} \\
\end{vmatrix}
}

%\newcommand{\DETuvwxyijklm}[10]{
%\begin{vmatrix}
% {#1}_{#6} & {#1}_{#7} & {#1}_{#8} & {#1}_{#9} & {#1}_{#10} \\
% {#2}_{#6} & {#2}_{#7} & {#2}_{#8} & {#2}_{#9} & {#2}_{#10} \\
% {#3}_{#6} & {#3}_{#7} & {#3}_{#8} & {#3}_{#9} & {#3}_{#10} \\
% {#4}_{#6} & {#4}_{#7} & {#4}_{#8} & {#4}_{#9} & {#4}_{#10} \\
% {#5}_{#6} & {#5}_{#7} & {#5}_{#8} & {#5}_{#9} & {#5}_{#10}
%\end{vmatrix}
%}

% R3 vector.
\newcommand{\VectorThree}[3]{
\begin{bmatrix}
 {#1} \\
 {#2} \\
 {#3}
\end{bmatrix}
}


%<misc>
%
\newcommand{\Abs}[1]{{\left\lvert{#1}\right\rvert}}
\newcommand{\spacegrad}[0]{\boldsymbol{\nabla}}
\newcommand{\grad}[0]{\nabla}
\newcommand{\LL}[0]{\mathcal{L}}

% == \partial_{#1} {#2}
\newcommand{\PD}[2]{\frac{\partial {#2}}{\partial {#1}}}
% inline variant
\newcommand{\PDi}[2]{{\partial {#2}}/{\partial {#1}}}

\newcommand{\PDD}[3]{\frac{\partial^2 {#3}}{\partial {#1}\partial {#2}}}
%\newcommand{\PDd}[2]{\frac{\partial^2 {#2}}{{\partial{#1}}^2}}
\newcommand{\PDsq}[2]{\frac{\partial^2 {#2}}{(\partial {#1})^2}}

\newcommand{\Partial}[2]{\frac{\partial {#1}}{\partial {#2}}}
\DeclareMathOperator{\RejName}{Rej}
\newcommand{\Rej}[2]{\RejName_{#1}\left( {#2} \right)}
\newcommand{\Rm}[1]{\mathbb{R}^{#1}}
\newcommand{\Cm}[1]{\mathbb{C}^{#1}}
\newcommand{\conj}[0]{{*}}

%</misc>

% <grade selection>
%
\newcommand{\gpgrade}[2] {{\left\langle{{#1}}\right\rangle}_{#2}}

\newcommand{\gpgradezero}[1] {\gpgrade{#1}{}}
%\newcommand{\gpscalargrade}[1] {{\left\langle{{#1}}\right\rangle}}
%\newcommand{\gpgradezero}[1] {\gpgrade{#1}{0}}

%\newcommand{\gpgradeone}[1] {{\left\langle{{#1}}\right\rangle}_{1}}
\newcommand{\gpgradeone}[1] {\gpgrade{#1}{1}}

\newcommand{\gpgradetwo}[1] {\gpgrade{#1}{2}}
\newcommand{\gpgradethree}[1] {\gpgrade{#1}{3}}
\newcommand{\gpgradefour}[1] {\gpgrade{#1}{4}}
%
% </grade selection>



\newcommand{\adot}[0]{{\dot{a}}}
\newcommand{\bdot}[0]{{\dot{b}}}
% taken for centered dot:
%\newcommand{\cdot}[0]{{\dot{c}}}
%\newcommand{\ddot}[0]{{\dot{d}}}
\newcommand{\edot}[0]{{\dot{e}}}
\newcommand{\fdot}[0]{{\dot{f}}}
\newcommand{\gdot}[0]{{\dot{g}}}
\newcommand{\hdot}[0]{{\dot{h}}}
\newcommand{\idot}[0]{{\dot{i}}}
\newcommand{\jdot}[0]{{\dot{j}}}
\newcommand{\kdot}[0]{{\dot{k}}}
\newcommand{\ldot}[0]{{\dot{l}}}
\newcommand{\mdot}[0]{{\dot{m}}}
\newcommand{\ndot}[0]{{\dot{n}}}
%\newcommand{\odot}[0]{{\dot{o}}}
\newcommand{\pdot}[0]{{\dot{p}}}
\newcommand{\qdot}[0]{{\dot{q}}}
\newcommand{\rdot}[0]{{\dot{r}}}
\newcommand{\sdot}[0]{{\dot{s}}}
\newcommand{\tdot}[0]{{\dot{t}}}
\newcommand{\udot}[0]{{\dot{u}}}
\newcommand{\vdot}[0]{{\dot{v}}}
\newcommand{\wdot}[0]{{\dot{w}}}
\newcommand{\xdot}[0]{{\dot{x}}}
\newcommand{\ydot}[0]{{\dot{y}}}
\newcommand{\zdot}[0]{{\dot{z}}}
\newcommand{\addot}[0]{{\ddot{a}}}
\newcommand{\bddot}[0]{{\ddot{b}}}
\newcommand{\cddot}[0]{{\ddot{c}}}
%\newcommand{\dddot}[0]{{\ddot{d}}}
\newcommand{\eddot}[0]{{\ddot{e}}}
\newcommand{\fddot}[0]{{\ddot{f}}}
\newcommand{\gddot}[0]{{\ddot{g}}}
\newcommand{\hddot}[0]{{\ddot{h}}}
\newcommand{\iddot}[0]{{\ddot{i}}}
\newcommand{\jddot}[0]{{\ddot{j}}}
\newcommand{\kddot}[0]{{\ddot{k}}}
\newcommand{\lddot}[0]{{\ddot{l}}}
\newcommand{\mddot}[0]{{\ddot{m}}}
\newcommand{\nddot}[0]{{\ddot{n}}}
\newcommand{\oddot}[0]{{\ddot{o}}}
\newcommand{\pddot}[0]{{\ddot{p}}}
\newcommand{\qddot}[0]{{\ddot{q}}}
\newcommand{\rddot}[0]{{\ddot{r}}}
\newcommand{\sddot}[0]{{\ddot{s}}}
\newcommand{\tddot}[0]{{\ddot{t}}}
\newcommand{\uddot}[0]{{\ddot{u}}}
\newcommand{\vddot}[0]{{\ddot{v}}}
\newcommand{\wddot}[0]{{\ddot{w}}}
\newcommand{\xddot}[0]{{\ddot{x}}}
\newcommand{\yddot}[0]{{\ddot{y}}}
\newcommand{\zddot}[0]{{\ddot{z}}}

%<bold and dot greek symbols>
%

\newcommand{\Deltadot}[0]{{\dot{\Delta}}}
\newcommand{\Gammadot}[0]{{\dot{\Gamma}}}
\newcommand{\Lambdadot}[0]{{\dot{\Lambda}}}
\newcommand{\Omegadot}[0]{{\dot{\Omega}}}
\newcommand{\Phidot}[0]{{\dot{\Phi}}}
\newcommand{\Pidot}[0]{{\dot{\Pi}}}
\newcommand{\Psidot}[0]{{\dot{\Psi}}}
\newcommand{\Sigmadot}[0]{{\dot{\Sigma}}}
\newcommand{\Thetadot}[0]{{\dot{\Theta}}}
\newcommand{\Upsilondot}[0]{{\dot{\Upsilon}}}
\newcommand{\Xidot}[0]{{\dot{\Xi}}}
\newcommand{\alphadot}[0]{{\dot{\alpha}}}
\newcommand{\betadot}[0]{{\dot{\beta}}}
\newcommand{\chidot}[0]{{\dot{\chi}}}
\newcommand{\deltadot}[0]{{\dot{\delta}}}
\newcommand{\epsilondot}[0]{{\dot{\epsilon}}}
\newcommand{\etadot}[0]{{\dot{\eta}}}
\newcommand{\gammadot}[0]{{\dot{\gamma}}}
\newcommand{\kappadot}[0]{{\dot{\kappa}}}
\newcommand{\lambdadot}[0]{{\dot{\lambda}}}
\newcommand{\mudot}[0]{{\dot{\mu}}}
\newcommand{\nudot}[0]{{\dot{\nu}}}
\newcommand{\omegadot}[0]{{\dot{\omega}}}
\newcommand{\phidot}[0]{{\dot{\phi}}}
\newcommand{\pidot}[0]{{\dot{\pi}}}
\newcommand{\psidot}[0]{{\dot{\psi}}}
\newcommand{\rhodot}[0]{{\dot{\rho}}}
\newcommand{\sigmadot}[0]{{\dot{\sigma}}}
\newcommand{\taudot}[0]{{\dot{\tau}}}
\newcommand{\thetadot}[0]{{\dot{\theta}}}
\newcommand{\upsilondot}[0]{{\dot{\upsilon}}}
\newcommand{\varepsilondot}[0]{{\dot{\varepsilon}}}
\newcommand{\varphidot}[0]{{\dot{\varphi}}}
\newcommand{\varpidot}[0]{{\dot{\varpi}}}
\newcommand{\varrhodot}[0]{{\dot{\varrho}}}
\newcommand{\varsigmadot}[0]{{\dot{\varsigma}}}
\newcommand{\varthetadot}[0]{{\dot{\vartheta}}}
\newcommand{\xidot}[0]{{\dot{\xi}}}
\newcommand{\zetadot}[0]{{\dot{\zeta}}}

\newcommand{\Deltaddot}[0]{{\ddot{\Delta}}}
\newcommand{\Gammaddot}[0]{{\ddot{\Gamma}}}
\newcommand{\Lambdaddot}[0]{{\ddot{\Lambda}}}
\newcommand{\Omegaddot}[0]{{\ddot{\Omega}}}
\newcommand{\Phiddot}[0]{{\ddot{\Phi}}}
\newcommand{\Piddot}[0]{{\ddot{\Pi}}}
\newcommand{\Psiddot}[0]{{\ddot{\Psi}}}
\newcommand{\Sigmaddot}[0]{{\ddot{\Sigma}}}
\newcommand{\Thetaddot}[0]{{\ddot{\Theta}}}
\newcommand{\Upsilonddot}[0]{{\ddot{\Upsilon}}}
\newcommand{\Xiddot}[0]{{\ddot{\Xi}}}
\newcommand{\alphaddot}[0]{{\ddot{\alpha}}}
\newcommand{\betaddot}[0]{{\ddot{\beta}}}
\newcommand{\chiddot}[0]{{\ddot{\chi}}}
\newcommand{\deltaddot}[0]{{\ddot{\delta}}}
\newcommand{\epsilonddot}[0]{{\ddot{\epsilon}}}
\newcommand{\etaddot}[0]{{\ddot{\eta}}}
\newcommand{\gammaddot}[0]{{\ddot{\gamma}}}
\newcommand{\kappaddot}[0]{{\ddot{\kappa}}}
\newcommand{\lambdaddot}[0]{{\ddot{\lambda}}}
\newcommand{\muddot}[0]{{\ddot{\mu}}}
\newcommand{\nuddot}[0]{{\ddot{\nu}}}
\newcommand{\omegaddot}[0]{{\ddot{\omega}}}
\newcommand{\phiddot}[0]{{\ddot{\phi}}}
\newcommand{\piddot}[0]{{\ddot{\pi}}}
\newcommand{\psiddot}[0]{{\ddot{\psi}}}
\newcommand{\rhoddot}[0]{{\ddot{\rho}}}
\newcommand{\sigmaddot}[0]{{\ddot{\sigma}}}
\newcommand{\tauddot}[0]{{\ddot{\tau}}}
\newcommand{\thetaddot}[0]{{\ddot{\theta}}}
\newcommand{\upsilonddot}[0]{{\ddot{\upsilon}}}
\newcommand{\varepsilonddot}[0]{{\ddot{\varepsilon}}}
\newcommand{\varphiddot}[0]{{\ddot{\varphi}}}
\newcommand{\varpiddot}[0]{{\ddot{\varpi}}}
\newcommand{\varrhoddot}[0]{{\ddot{\varrho}}}
\newcommand{\varsigmaddot}[0]{{\ddot{\varsigma}}}
\newcommand{\varthetaddot}[0]{{\ddot{\vartheta}}}
\newcommand{\xiddot}[0]{{\ddot{\xi}}}
\newcommand{\zetaddot}[0]{{\ddot{\zeta}}}

\newcommand{\BDelta}[0]{\boldsymbol{\Delta}}
\newcommand{\BGamma}[0]{\boldsymbol{\Gamma}}
\newcommand{\BLambda}[0]{\boldsymbol{\Lambda}}
\newcommand{\BOmega}[0]{\boldsymbol{\Omega}}
\newcommand{\BPhi}[0]{\boldsymbol{\Phi}}
\newcommand{\BPi}[0]{\boldsymbol{\Pi}}
\newcommand{\BPsi}[0]{\boldsymbol{\Psi}}
\newcommand{\BSigma}[0]{\boldsymbol{\Sigma}}
\newcommand{\BTheta}[0]{\boldsymbol{\Theta}}
\newcommand{\BUpsilon}[0]{\boldsymbol{\Upsilon}}
\newcommand{\BXi}[0]{\boldsymbol{\Xi}}
\newcommand{\Balpha}[0]{\boldsymbol{\alpha}}
\newcommand{\Bbeta}[0]{\boldsymbol{\beta}}
\newcommand{\Bchi}[0]{\boldsymbol{\chi}}
\newcommand{\Bdelta}[0]{\boldsymbol{\delta}}
\newcommand{\Bepsilon}[0]{\boldsymbol{\epsilon}}
\newcommand{\Beta}[0]{\boldsymbol{\eta}}
\newcommand{\Bgamma}[0]{\boldsymbol{\gamma}}
\newcommand{\Bkappa}[0]{\boldsymbol{\kappa}}
\newcommand{\Blambda}[0]{\boldsymbol{\lambda}}
\newcommand{\Bmu}[0]{\boldsymbol{\mu}}
\newcommand{\Bnu}[0]{\boldsymbol{\nu}}
%\newcommand{\Bomega}[0]{\boldsymbol{\omega}}
\newcommand{\Bphi}[0]{\boldsymbol{\phi}}
\newcommand{\Bpi}[0]{\boldsymbol{\pi}}
\newcommand{\Bpsi}[0]{\boldsymbol{\psi}}
\newcommand{\Brho}[0]{\boldsymbol{\rho}}
\newcommand{\Bsigma}[0]{\boldsymbol{\sigma}}
%\newcommand{\Btau}[0]{\boldsymbol{\tau}}
%\newcommand{\Btheta}[0]{\boldsymbol{\theta}}
\newcommand{\Bupsilon}[0]{\boldsymbol{\upsilon}}
\newcommand{\Bvarepsilon}[0]{\boldsymbol{\varepsilon}}
\newcommand{\Bvarphi}[0]{\boldsymbol{\varphi}}
\newcommand{\Bvarpi}[0]{\boldsymbol{\varpi}}
\newcommand{\Bvarrho}[0]{\boldsymbol{\varrho}}
\newcommand{\Bvarsigma}[0]{\boldsymbol{\varsigma}}
\newcommand{\Bvartheta}[0]{\boldsymbol{\vartheta}}
\newcommand{\Bxi}[0]{\boldsymbol{\xi}}
\newcommand{\Bzeta}[0]{\boldsymbol{\zeta}}
%
%</bold and dot greek symbols>
%<infrequent>
%
%\newcommand{\AreaOp}[1]{\AName_{#1}}
%\newcommand{\Babs}[0]{\abs{\BB}}
%\newcommand{\Bcap}[0]{\hat{\BB}}
%\newcommand{\BrPrimeRej}[0]{\rcap(\rcap \wedge \Br')}
%\newcommand{\CA}[0]{\mathcal{A}}
%\newcommand{\Cos}[1]{\cos{\left({#1}\right)}}
%\newcommand{\Det}[1] {\abs{#1}}
%\newcommand{\Dsq}[2] {\frac {\partial^2 {#1}} {\partial {#2}^2}}
%\newcommand{\Exp}[1]{\exp{\left({#1}\right)}}
%\newcommand{\Norm}[1]{\left\lVert{#1}\right\rVert}
%\newcommand{\Sin}[1]{\sin{\left({#1}\right)}}
%\newcommand{\T}[0]{\text{T}}
%\newcommand{\VolumeOp}[1]{\VName_{#1}}
%\newcommand{\agrad}[0]{\Ba \cdot \nabla}
%\newcommand{\alphacap}[0]{\hat{\boldsymbol{\alpha}}}
%\newcommand{\Fcap}[0]{\hat{\BF}}
%\newcommand{\bithree}[0]{{\Bi}_3}
%\newcommand{\bxa}[0]{\Bx\Ba}
%\newcommand{\coordvec}[2]{
%\newcommand{\costheta}[0]{\acap \cdot \xcap}
%\newcommand{\ddt}[1]{\ddot{#1}}
%\newcommand{\ddu}[1] {\frac {d{#1}} {du}}
%\newcommand{\dsqxj}[2] {\frac {\partial^2 {#1}} {\partial {x_{#2}}^2}}
%\newcommand{\dtheta}[1]{\frac{d {#1}}{d \theta}}
%\newcommand{\dt}[1]{\dot{#1}}
%\newcommand{\dt}[1]{\frac{d {#1}}{dt}}
%\newcommand{\dxj}[2] {\frac {\partial {#1}} {\partial {x_{#2}}}}
%\newcommand{\halfPhi}[0]{\frac{\phi}{2}}
%\newcommand{\half}[0]{\inv{2}}
%\newcommand{\inv}[1]{\frac{1}{#1}}
%\newcommand{\laplacian}[0]{\nabla^2}
%\newcommand{\matrixoftx}[3]{
%\newcommand{\nrrp}[0]{\norm{\rcap \wedge \Br'}}
%\newcommand{\oiint}{\bigcirc \hspace{-1.4em} \int \hspace{-.8em} \int}
%\newcommand{\transpose}[1]{{#1}^{\text{T}}}
%\newcommand{\transpose}[1]{{{#1}^{\TextTranspose}}}
%\newcommand{\transpose}[1]{{{#1}^{\text{T}}}}
%\newcommand{\barA}[0]{\bar{A}}
%\newcommand{\qbar}[0]{\bar{q}}
%\newcommand{\qdotbar}[0]{\dot{\bar{q}}}
%
%</infrequent>





\usepackage[bookmarks=true]{hyperref}

\usepackage{color,cite,graphicx}
   % use colour in the document, put your citations as [1-4]
   % rather than [1,2,3,4] (it looks nicer, and the extended LaTeX2e
   % graphics package. 
\usepackage{latexsym,amssymb,epsf} % don't remember if these are
   % needed, but their inclusion can't do any damage


\title{ Attempt to make sense of fourier form of Green's function for the Poisson equation. }
\author{Peeter Joot}
\date{ Feb 18, 2009.  Last Revision: $Date: 2009/02/19 05:15:56 $ }

\begin{document}

\maketitle{}
\tableofcontents

%\section{}

Am just playing around, and 
following examples of Fourier transform solutions of the heat equation, tried the same thing for 
the electrostatics Poisson equation
\begin{align*}
\grad^2 \phi &= -\rho/\epsilon_0 \\
\end{align*}

With fourier transform pairs
\begin{align*}
\hat{f}(\mathbf{k}) &= \frac{1}{\sqrt{2\pi}} \iiint f(\mathbf{x}) e^{-i \mathbf{k} \cdot \mathbf{x} } d^3 x \\
{f}(\mathbf{x}) &= \frac{1}{\sqrt{2\pi}} \iiint \hat{f}(\mathbf{k}) e^{i \mathbf{k} \cdot \mathbf{x} } d^3 k \\
\end{align*}

one gets 

\begin{align*}
\phi(\mathbf{x}) &= \frac{1}{\epsilon_0} \int \rho(\mathbf{x}' G(\mathbf{x-x'}) d^3 x' \\
G(\mathbf{x}) &= \frac{1}{(2 \pi)^3} \iiint \frac{1}{\mathbf{k}^2} e^{ i \mathbf{k} \cdot \mathbf{x} } d^3 k
\end{align*}

Now it seems to me that this integral $G$ only has to be evaluated around a small neighbourhood of the origin.  For example if one evaluates one of
the
integrals  
\begin{align*}
\int_{-\infty}^\infty \frac{1}{{k_1}^2 + {k_2}^2 + {k_3}^3 } e^{ i k_1 x_1 } dk_1 
\end{align*}

using a an upper half plane contour the result is zero unless $k_2 = k_3 = 0$.  So one is left with something loosely like

\begin{align*}
G(\mathbf{x}) &= \lim_{\epsilon \rightarrow 0} \frac{1}{(2 \pi)^3} 
\int_{k_1 = -\epsilon}^{\epsilon} dk_1
\int_{k_2 = -\epsilon}^{\epsilon} dk_2
\int_{k_3 = -\epsilon}^{\epsilon} dk_3
 \frac{1}{\mathbf{k}^2} e^{ i \mathbf{k} \cdot \mathbf{x} } 
\end{align*}

However, from electrostatics we also know that the solution to the Poission equation means that $G(\mathbf{x}) = \frac{1}{4\pi\lvert{\mathbf{x}}\rvert}$.
Does anybody know of a technique that would reduce the integral limit expression above for $G$ to the $1/x$ form?  I've played around with this for a bit
without any success.

\bibliographystyle{plainnat}
\bibliography{myrefs}

\end{document}

\include{wave_fourier}
\include{fourier_maxwell}
\include{firstorder_fourier_maxwell}
\include{4d_fourier}
\include{fourier_series_maxwell}
\documentclass{article}

\usepackage{amsmath}
\usepackage{mathpazo}

%
% shorthand for bold symbols, convenient for vectors and matrices
%
\newcommand{\Ba}[0]{\mathbf{a}}
\newcommand{\Bb}[0]{\mathbf{b}}
\newcommand{\Bc}[0]{\mathbf{c}}
\newcommand{\Bd}[0]{\mathbf{d}}
\newcommand{\Be}[0]{\mathbf{e}}
\newcommand{\Bf}[0]{\mathbf{f}}
\newcommand{\Bg}[0]{\mathbf{g}}
\newcommand{\Bh}[0]{\mathbf{h}}
\newcommand{\Bi}[0]{\mathbf{i}}
\newcommand{\Bj}[0]{\mathbf{j}}
\newcommand{\Bk}[0]{\mathbf{k}}
\newcommand{\Bl}[0]{\mathbf{l}}
\newcommand{\Bm}[0]{\mathbf{m}}
\newcommand{\Bn}[0]{\mathbf{n}}
\newcommand{\Bo}[0]{\mathbf{o}}
\newcommand{\Bp}[0]{\mathbf{p}}
\newcommand{\Bq}[0]{\mathbf{q}}
\newcommand{\Br}[0]{\mathbf{r}}
\newcommand{\Bs}[0]{\mathbf{s}}
\newcommand{\Bt}[0]{\mathbf{t}}
\newcommand{\Bu}[0]{\mathbf{u}}
\newcommand{\Bv}[0]{\mathbf{v}}
\newcommand{\Bw}[0]{\mathbf{w}}
\newcommand{\Bx}[0]{\mathbf{x}}
\newcommand{\By}[0]{\mathbf{y}}
\newcommand{\Bz}[0]{\mathbf{z}}
\newcommand{\BA}[0]{\mathbf{A}}
\newcommand{\BB}[0]{\mathbf{B}}
\newcommand{\BC}[0]{\mathbf{C}}
\newcommand{\BD}[0]{\mathbf{D}}
\newcommand{\BE}[0]{\mathbf{E}}
\newcommand{\BF}[0]{\mathbf{F}}
\newcommand{\BG}[0]{\mathbf{G}}
\newcommand{\BH}[0]{\mathbf{H}}
\newcommand{\BI}[0]{\mathbf{I}}
\newcommand{\BJ}[0]{\mathbf{J}}
\newcommand{\BK}[0]{\mathbf{K}}
\newcommand{\BL}[0]{\mathbf{L}}
\newcommand{\BM}[0]{\mathbf{M}}
\newcommand{\BN}[0]{\mathbf{N}}
\newcommand{\BO}[0]{\mathbf{O}}
\newcommand{\BP}[0]{\mathbf{P}}
\newcommand{\BQ}[0]{\mathbf{Q}}
\newcommand{\BR}[0]{\mathbf{R}}
\newcommand{\BS}[0]{\mathbf{S}}
\newcommand{\BT}[0]{\mathbf{T}}
\newcommand{\BU}[0]{\mathbf{U}}
\newcommand{\BV}[0]{\mathbf{V}}
\newcommand{\BW}[0]{\mathbf{W}}
\newcommand{\BX}[0]{\mathbf{X}}
\newcommand{\BY}[0]{\mathbf{Y}}
\newcommand{\BZ}[0]{\mathbf{Z}}

\newcommand{\Bzero}[0]{\mathbf{0}}
\newcommand{\Btheta}[0]{\boldsymbol{\theta}}
\newcommand{\Btau}[0]{\boldsymbol{\tau}}
\newcommand{\Bomega}[0]{\boldsymbol{\omega}}

%
% shorthand for unit vectors
%
\newcommand{\acap}[0]{\hat{\Ba}}
\newcommand{\bcap}[0]{\hat{\Bb}}
\newcommand{\ccap}[0]{\hat{\Bc}}
\newcommand{\dcap}[0]{\hat{\Bd}}
\newcommand{\ecap}[0]{\hat{\Be}}
\newcommand{\fcap}[0]{\hat{\Bf}}
\newcommand{\gcap}[0]{\hat{\Bg}}
\newcommand{\hcap}[0]{\hat{\Bh}}
\newcommand{\icap}[0]{\hat{\Bi}}
\newcommand{\jcap}[0]{\hat{\Bj}}
\newcommand{\kcap}[0]{\hat{\Bk}}
\newcommand{\lcap}[0]{\hat{\Bl}}
\newcommand{\mcap}[0]{\hat{\Bm}}
\newcommand{\ncap}[0]{\hat{\Bn}}
\newcommand{\ocap}[0]{\hat{\Bo}}
\newcommand{\pcap}[0]{\hat{\Bp}}
\newcommand{\qcap}[0]{\hat{\Bq}}
\newcommand{\rcap}[0]{\hat{\Br}}
\newcommand{\scap}[0]{\hat{\Bs}}
\newcommand{\tcap}[0]{\hat{\Bt}}
\newcommand{\ucap}[0]{\hat{\Bu}}
\newcommand{\vcap}[0]{\hat{\Bv}}
\newcommand{\wcap}[0]{\hat{\Bw}}
\newcommand{\xcap}[0]{\hat{\Bx}}
\newcommand{\ycap}[0]{\hat{\By}}
\newcommand{\zcap}[0]{\hat{\Bz}}
\newcommand{\thetacap}[0]{\hat{\Btheta}}

%
% to write R^n and C^n in a distinguishable fashion.  Perhaps change this
% to the double lined characters upon figuring out how to do so.
%
\newcommand{\C}[1]{$\mathbb{C}^{#1}$}
\newcommand{\R}[1]{$\mathbb{R}^{#1}$}

%
% various generally useful helpers
%

% derivative of #1 wrt. #2:
\newcommand{\D}[2] {\frac {d#2} {d#1}}

\newcommand{\inv}[1]{\frac{1}{#1}}
\newcommand{\cross}[0]{\times}

\newcommand{\abs}[1]{\lvert{#1}\rvert}
\newcommand{\norm}[1]{\lVert{#1}\rVert}
\newcommand{\innerprod}[2]{\langle{#1}, {#2}\rangle}
\newcommand{\dotprod}[2]{{#1} \cdot {#2}}
\newcommand{\bdotprod}[2]{\left({#1} \cdot {#2}\right)}
\newcommand{\crossprod}[2]{{#1} \cross {#2}}
\newcommand{\tripleprod}[3]{\dotprod{\left(\crossprod{#1}{#2}\right)}{#3}}

\DeclareMathOperator{\Proj}{Proj}
\DeclareMathOperator{\Span}{span}
\DeclareMathOperator{\Sgn}{sgn}
\DeclareMathOperator{\Area}{Area}
\DeclareMathOperator{\Volume}{Volume}

%
% A few miscellaneous things specific to this document
%
\newcommand{\crossop}[1]{\crossprod{#1}{}}

% R2 vector.
\newcommand{\VectorTwo}[2]{
\begin{bmatrix}
 {#1} \\
 {#2}
\end{bmatrix}
}

\newcommand{\VectorN}[1]{
\begin{bmatrix}
{#1}_1 \\
{#1}_2 \\
\vdots \\
{#1}_N \\
\end{bmatrix}
}

\newcommand{\DETuvij}[4]{
\begin{vmatrix}
 {#1}_{#3} & {#1}_{#4} \\
 {#2}_{#3} & {#2}_{#4}
\end{vmatrix}
}

\newcommand{\DETuvwijk}[6]{
\begin{vmatrix}
 {#1}_{#4} & {#1}_{#5} & {#1}_{#6} \\
 {#2}_{#4} & {#2}_{#5} & {#2}_{#6} \\
 {#3}_{#4} & {#3}_{#5} & {#3}_{#6}
\end{vmatrix}
}

\newcommand{\DETuvwxijkl}[8]{
\begin{vmatrix}
 {#1}_{#5} & {#1}_{#6} & {#1}_{#7} & {#1}_{#8} \\
 {#2}_{#5} & {#2}_{#6} & {#2}_{#7} & {#2}_{#8} \\
 {#3}_{#5} & {#3}_{#6} & {#3}_{#7} & {#3}_{#8} \\
 {#4}_{#5} & {#4}_{#6} & {#4}_{#7} & {#4}_{#8} \\
\end{vmatrix}
}

%\newcommand{\DETuvwxyijklm}[10]{
%\begin{vmatrix}
% {#1}_{#6} & {#1}_{#7} & {#1}_{#8} & {#1}_{#9} & {#1}_{#10} \\
% {#2}_{#6} & {#2}_{#7} & {#2}_{#8} & {#2}_{#9} & {#2}_{#10} \\
% {#3}_{#6} & {#3}_{#7} & {#3}_{#8} & {#3}_{#9} & {#3}_{#10} \\
% {#4}_{#6} & {#4}_{#7} & {#4}_{#8} & {#4}_{#9} & {#4}_{#10} \\
% {#5}_{#6} & {#5}_{#7} & {#5}_{#8} & {#5}_{#9} & {#5}_{#10}
%\end{vmatrix}
%}

% R3 vector.
\newcommand{\VectorThree}[3]{
\begin{bmatrix}
 {#1} \\
 {#2} \\
 {#3}
\end{bmatrix}
}


%<misc>
%
\newcommand{\Abs}[1]{{\left\lvert{#1}\right\rvert}}
\newcommand{\spacegrad}[0]{\boldsymbol{\nabla}}
\newcommand{\grad}[0]{\nabla}
\newcommand{\LL}[0]{\mathcal{L}}

% == \partial_{#1} {#2}
\newcommand{\PD}[2]{\frac{\partial {#2}}{\partial {#1}}}
% inline variant
\newcommand{\PDi}[2]{{\partial {#2}}/{\partial {#1}}}

\newcommand{\PDD}[3]{\frac{\partial^2 {#3}}{\partial {#1}\partial {#2}}}
%\newcommand{\PDd}[2]{\frac{\partial^2 {#2}}{{\partial{#1}}^2}}
\newcommand{\PDsq}[2]{\frac{\partial^2 {#2}}{(\partial {#1})^2}}

\newcommand{\Partial}[2]{\frac{\partial {#1}}{\partial {#2}}}
\DeclareMathOperator{\RejName}{Rej}
\newcommand{\Rej}[2]{\RejName_{#1}\left( {#2} \right)}
\newcommand{\Rm}[1]{\mathbb{R}^{#1}}
\newcommand{\Cm}[1]{\mathbb{C}^{#1}}
\newcommand{\conj}[0]{{*}}

%</misc>

% <grade selection>
%
\newcommand{\gpgrade}[2] {{\left\langle{{#1}}\right\rangle}_{#2}}

\newcommand{\gpgradezero}[1] {\gpgrade{#1}{}}
%\newcommand{\gpscalargrade}[1] {{\left\langle{{#1}}\right\rangle}}
%\newcommand{\gpgradezero}[1] {\gpgrade{#1}{0}}

%\newcommand{\gpgradeone}[1] {{\left\langle{{#1}}\right\rangle}_{1}}
\newcommand{\gpgradeone}[1] {\gpgrade{#1}{1}}

\newcommand{\gpgradetwo}[1] {\gpgrade{#1}{2}}
\newcommand{\gpgradethree}[1] {\gpgrade{#1}{3}}
\newcommand{\gpgradefour}[1] {\gpgrade{#1}{4}}
%
% </grade selection>



\newcommand{\adot}[0]{{\dot{a}}}
\newcommand{\bdot}[0]{{\dot{b}}}
% taken for centered dot:
%\newcommand{\cdot}[0]{{\dot{c}}}
%\newcommand{\ddot}[0]{{\dot{d}}}
\newcommand{\edot}[0]{{\dot{e}}}
\newcommand{\fdot}[0]{{\dot{f}}}
\newcommand{\gdot}[0]{{\dot{g}}}
\newcommand{\hdot}[0]{{\dot{h}}}
\newcommand{\idot}[0]{{\dot{i}}}
\newcommand{\jdot}[0]{{\dot{j}}}
\newcommand{\kdot}[0]{{\dot{k}}}
\newcommand{\ldot}[0]{{\dot{l}}}
\newcommand{\mdot}[0]{{\dot{m}}}
\newcommand{\ndot}[0]{{\dot{n}}}
%\newcommand{\odot}[0]{{\dot{o}}}
\newcommand{\pdot}[0]{{\dot{p}}}
\newcommand{\qdot}[0]{{\dot{q}}}
\newcommand{\rdot}[0]{{\dot{r}}}
\newcommand{\sdot}[0]{{\dot{s}}}
\newcommand{\tdot}[0]{{\dot{t}}}
\newcommand{\udot}[0]{{\dot{u}}}
\newcommand{\vdot}[0]{{\dot{v}}}
\newcommand{\wdot}[0]{{\dot{w}}}
\newcommand{\xdot}[0]{{\dot{x}}}
\newcommand{\ydot}[0]{{\dot{y}}}
\newcommand{\zdot}[0]{{\dot{z}}}
\newcommand{\addot}[0]{{\ddot{a}}}
\newcommand{\bddot}[0]{{\ddot{b}}}
\newcommand{\cddot}[0]{{\ddot{c}}}
%\newcommand{\dddot}[0]{{\ddot{d}}}
\newcommand{\eddot}[0]{{\ddot{e}}}
\newcommand{\fddot}[0]{{\ddot{f}}}
\newcommand{\gddot}[0]{{\ddot{g}}}
\newcommand{\hddot}[0]{{\ddot{h}}}
\newcommand{\iddot}[0]{{\ddot{i}}}
\newcommand{\jddot}[0]{{\ddot{j}}}
\newcommand{\kddot}[0]{{\ddot{k}}}
\newcommand{\lddot}[0]{{\ddot{l}}}
\newcommand{\mddot}[0]{{\ddot{m}}}
\newcommand{\nddot}[0]{{\ddot{n}}}
\newcommand{\oddot}[0]{{\ddot{o}}}
\newcommand{\pddot}[0]{{\ddot{p}}}
\newcommand{\qddot}[0]{{\ddot{q}}}
\newcommand{\rddot}[0]{{\ddot{r}}}
\newcommand{\sddot}[0]{{\ddot{s}}}
\newcommand{\tddot}[0]{{\ddot{t}}}
\newcommand{\uddot}[0]{{\ddot{u}}}
\newcommand{\vddot}[0]{{\ddot{v}}}
\newcommand{\wddot}[0]{{\ddot{w}}}
\newcommand{\xddot}[0]{{\ddot{x}}}
\newcommand{\yddot}[0]{{\ddot{y}}}
\newcommand{\zddot}[0]{{\ddot{z}}}

%<bold and dot greek symbols>
%

\newcommand{\Deltadot}[0]{{\dot{\Delta}}}
\newcommand{\Gammadot}[0]{{\dot{\Gamma}}}
\newcommand{\Lambdadot}[0]{{\dot{\Lambda}}}
\newcommand{\Omegadot}[0]{{\dot{\Omega}}}
\newcommand{\Phidot}[0]{{\dot{\Phi}}}
\newcommand{\Pidot}[0]{{\dot{\Pi}}}
\newcommand{\Psidot}[0]{{\dot{\Psi}}}
\newcommand{\Sigmadot}[0]{{\dot{\Sigma}}}
\newcommand{\Thetadot}[0]{{\dot{\Theta}}}
\newcommand{\Upsilondot}[0]{{\dot{\Upsilon}}}
\newcommand{\Xidot}[0]{{\dot{\Xi}}}
\newcommand{\alphadot}[0]{{\dot{\alpha}}}
\newcommand{\betadot}[0]{{\dot{\beta}}}
\newcommand{\chidot}[0]{{\dot{\chi}}}
\newcommand{\deltadot}[0]{{\dot{\delta}}}
\newcommand{\epsilondot}[0]{{\dot{\epsilon}}}
\newcommand{\etadot}[0]{{\dot{\eta}}}
\newcommand{\gammadot}[0]{{\dot{\gamma}}}
\newcommand{\kappadot}[0]{{\dot{\kappa}}}
\newcommand{\lambdadot}[0]{{\dot{\lambda}}}
\newcommand{\mudot}[0]{{\dot{\mu}}}
\newcommand{\nudot}[0]{{\dot{\nu}}}
\newcommand{\omegadot}[0]{{\dot{\omega}}}
\newcommand{\phidot}[0]{{\dot{\phi}}}
\newcommand{\pidot}[0]{{\dot{\pi}}}
\newcommand{\psidot}[0]{{\dot{\psi}}}
\newcommand{\rhodot}[0]{{\dot{\rho}}}
\newcommand{\sigmadot}[0]{{\dot{\sigma}}}
\newcommand{\taudot}[0]{{\dot{\tau}}}
\newcommand{\thetadot}[0]{{\dot{\theta}}}
\newcommand{\upsilondot}[0]{{\dot{\upsilon}}}
\newcommand{\varepsilondot}[0]{{\dot{\varepsilon}}}
\newcommand{\varphidot}[0]{{\dot{\varphi}}}
\newcommand{\varpidot}[0]{{\dot{\varpi}}}
\newcommand{\varrhodot}[0]{{\dot{\varrho}}}
\newcommand{\varsigmadot}[0]{{\dot{\varsigma}}}
\newcommand{\varthetadot}[0]{{\dot{\vartheta}}}
\newcommand{\xidot}[0]{{\dot{\xi}}}
\newcommand{\zetadot}[0]{{\dot{\zeta}}}

\newcommand{\Deltaddot}[0]{{\ddot{\Delta}}}
\newcommand{\Gammaddot}[0]{{\ddot{\Gamma}}}
\newcommand{\Lambdaddot}[0]{{\ddot{\Lambda}}}
\newcommand{\Omegaddot}[0]{{\ddot{\Omega}}}
\newcommand{\Phiddot}[0]{{\ddot{\Phi}}}
\newcommand{\Piddot}[0]{{\ddot{\Pi}}}
\newcommand{\Psiddot}[0]{{\ddot{\Psi}}}
\newcommand{\Sigmaddot}[0]{{\ddot{\Sigma}}}
\newcommand{\Thetaddot}[0]{{\ddot{\Theta}}}
\newcommand{\Upsilonddot}[0]{{\ddot{\Upsilon}}}
\newcommand{\Xiddot}[0]{{\ddot{\Xi}}}
\newcommand{\alphaddot}[0]{{\ddot{\alpha}}}
\newcommand{\betaddot}[0]{{\ddot{\beta}}}
\newcommand{\chiddot}[0]{{\ddot{\chi}}}
\newcommand{\deltaddot}[0]{{\ddot{\delta}}}
\newcommand{\epsilonddot}[0]{{\ddot{\epsilon}}}
\newcommand{\etaddot}[0]{{\ddot{\eta}}}
\newcommand{\gammaddot}[0]{{\ddot{\gamma}}}
\newcommand{\kappaddot}[0]{{\ddot{\kappa}}}
\newcommand{\lambdaddot}[0]{{\ddot{\lambda}}}
\newcommand{\muddot}[0]{{\ddot{\mu}}}
\newcommand{\nuddot}[0]{{\ddot{\nu}}}
\newcommand{\omegaddot}[0]{{\ddot{\omega}}}
\newcommand{\phiddot}[0]{{\ddot{\phi}}}
\newcommand{\piddot}[0]{{\ddot{\pi}}}
\newcommand{\psiddot}[0]{{\ddot{\psi}}}
\newcommand{\rhoddot}[0]{{\ddot{\rho}}}
\newcommand{\sigmaddot}[0]{{\ddot{\sigma}}}
\newcommand{\tauddot}[0]{{\ddot{\tau}}}
\newcommand{\thetaddot}[0]{{\ddot{\theta}}}
\newcommand{\upsilonddot}[0]{{\ddot{\upsilon}}}
\newcommand{\varepsilonddot}[0]{{\ddot{\varepsilon}}}
\newcommand{\varphiddot}[0]{{\ddot{\varphi}}}
\newcommand{\varpiddot}[0]{{\ddot{\varpi}}}
\newcommand{\varrhoddot}[0]{{\ddot{\varrho}}}
\newcommand{\varsigmaddot}[0]{{\ddot{\varsigma}}}
\newcommand{\varthetaddot}[0]{{\ddot{\vartheta}}}
\newcommand{\xiddot}[0]{{\ddot{\xi}}}
\newcommand{\zetaddot}[0]{{\ddot{\zeta}}}

\newcommand{\BDelta}[0]{\boldsymbol{\Delta}}
\newcommand{\BGamma}[0]{\boldsymbol{\Gamma}}
\newcommand{\BLambda}[0]{\boldsymbol{\Lambda}}
\newcommand{\BOmega}[0]{\boldsymbol{\Omega}}
\newcommand{\BPhi}[0]{\boldsymbol{\Phi}}
\newcommand{\BPi}[0]{\boldsymbol{\Pi}}
\newcommand{\BPsi}[0]{\boldsymbol{\Psi}}
\newcommand{\BSigma}[0]{\boldsymbol{\Sigma}}
\newcommand{\BTheta}[0]{\boldsymbol{\Theta}}
\newcommand{\BUpsilon}[0]{\boldsymbol{\Upsilon}}
\newcommand{\BXi}[0]{\boldsymbol{\Xi}}
\newcommand{\Balpha}[0]{\boldsymbol{\alpha}}
\newcommand{\Bbeta}[0]{\boldsymbol{\beta}}
\newcommand{\Bchi}[0]{\boldsymbol{\chi}}
\newcommand{\Bdelta}[0]{\boldsymbol{\delta}}
\newcommand{\Bepsilon}[0]{\boldsymbol{\epsilon}}
\newcommand{\Beta}[0]{\boldsymbol{\eta}}
\newcommand{\Bgamma}[0]{\boldsymbol{\gamma}}
\newcommand{\Bkappa}[0]{\boldsymbol{\kappa}}
\newcommand{\Blambda}[0]{\boldsymbol{\lambda}}
\newcommand{\Bmu}[0]{\boldsymbol{\mu}}
\newcommand{\Bnu}[0]{\boldsymbol{\nu}}
%\newcommand{\Bomega}[0]{\boldsymbol{\omega}}
\newcommand{\Bphi}[0]{\boldsymbol{\phi}}
\newcommand{\Bpi}[0]{\boldsymbol{\pi}}
\newcommand{\Bpsi}[0]{\boldsymbol{\psi}}
\newcommand{\Brho}[0]{\boldsymbol{\rho}}
\newcommand{\Bsigma}[0]{\boldsymbol{\sigma}}
%\newcommand{\Btau}[0]{\boldsymbol{\tau}}
%\newcommand{\Btheta}[0]{\boldsymbol{\theta}}
\newcommand{\Bupsilon}[0]{\boldsymbol{\upsilon}}
\newcommand{\Bvarepsilon}[0]{\boldsymbol{\varepsilon}}
\newcommand{\Bvarphi}[0]{\boldsymbol{\varphi}}
\newcommand{\Bvarpi}[0]{\boldsymbol{\varpi}}
\newcommand{\Bvarrho}[0]{\boldsymbol{\varrho}}
\newcommand{\Bvarsigma}[0]{\boldsymbol{\varsigma}}
\newcommand{\Bvartheta}[0]{\boldsymbol{\vartheta}}
\newcommand{\Bxi}[0]{\boldsymbol{\xi}}
\newcommand{\Bzeta}[0]{\boldsymbol{\zeta}}
%
%</bold and dot greek symbols>
%<infrequent>
%
%\newcommand{\AreaOp}[1]{\AName_{#1}}
%\newcommand{\Babs}[0]{\abs{\BB}}
%\newcommand{\Bcap}[0]{\hat{\BB}}
%\newcommand{\BrPrimeRej}[0]{\rcap(\rcap \wedge \Br')}
%\newcommand{\CA}[0]{\mathcal{A}}
%\newcommand{\Cos}[1]{\cos{\left({#1}\right)}}
%\newcommand{\Det}[1] {\abs{#1}}
%\newcommand{\Dsq}[2] {\frac {\partial^2 {#1}} {\partial {#2}^2}}
%\newcommand{\Exp}[1]{\exp{\left({#1}\right)}}
%\newcommand{\Norm}[1]{\left\lVert{#1}\right\rVert}
%\newcommand{\Sin}[1]{\sin{\left({#1}\right)}}
%\newcommand{\T}[0]{\text{T}}
%\newcommand{\VolumeOp}[1]{\VName_{#1}}
%\newcommand{\agrad}[0]{\Ba \cdot \nabla}
%\newcommand{\alphacap}[0]{\hat{\boldsymbol{\alpha}}}
%\newcommand{\Fcap}[0]{\hat{\BF}}
%\newcommand{\bithree}[0]{{\Bi}_3}
%\newcommand{\bxa}[0]{\Bx\Ba}
%\newcommand{\coordvec}[2]{
%\newcommand{\costheta}[0]{\acap \cdot \xcap}
%\newcommand{\ddt}[1]{\ddot{#1}}
%\newcommand{\ddu}[1] {\frac {d{#1}} {du}}
%\newcommand{\dsqxj}[2] {\frac {\partial^2 {#1}} {\partial {x_{#2}}^2}}
%\newcommand{\dtheta}[1]{\frac{d {#1}}{d \theta}}
%\newcommand{\dt}[1]{\dot{#1}}
%\newcommand{\dt}[1]{\frac{d {#1}}{dt}}
%\newcommand{\dxj}[2] {\frac {\partial {#1}} {\partial {x_{#2}}}}
%\newcommand{\halfPhi}[0]{\frac{\phi}{2}}
%\newcommand{\half}[0]{\inv{2}}
%\newcommand{\inv}[1]{\frac{1}{#1}}
%\newcommand{\laplacian}[0]{\nabla^2}
%\newcommand{\matrixoftx}[3]{
%\newcommand{\nrrp}[0]{\norm{\rcap \wedge \Br'}}
%\newcommand{\oiint}{\bigcirc \hspace{-1.4em} \int \hspace{-.8em} \int}
%\newcommand{\transpose}[1]{{#1}^{\text{T}}}
%\newcommand{\transpose}[1]{{{#1}^{\TextTranspose}}}
%\newcommand{\transpose}[1]{{{#1}^{\text{T}}}}
%\newcommand{\barA}[0]{\bar{A}}
%\newcommand{\qbar}[0]{\bar{q}}
%\newcommand{\qdotbar}[0]{\dot{\bar{q}}}
%
%</infrequent>





\usepackage[bookmarks=true]{hyperref}

\usepackage{color,cite,graphicx}
   % use colour in the document, put your citations as [1-4]
   % rather than [1,2,3,4] (it looks nicer, and the extended LaTeX2e
   % graphics package. 
\usepackage{latexsym,amssymb,epsf} % don't remember if these are
   % needed, but their inclusion can't do any damage


\title{ Plane wave Fourier series solutions to the Maxwell vacuum equation. }
\author{Peeter Joot}
\date{ Feb 08, 2009.  Last Revision: $Date: 2009/02/09 02:47:16 $ }

\begin{document}
\maketitle{}
\tableofcontents

\section{ Motivation. }

In \cite{PJFourierVacuum} an exploration of spatially periodic solutions to the electrodynamic vacuum equation was performed using a multivector formulation 
of a 3D Fourier series.
Here a summary of the results obtained will be presented in a more
coherent fashion, followed by an attempt to build on them.
In particular a complete
description of the field energy and momentum is desired.

A conclusion from the first analysis was that the
orientation of both the electric and magnetic field components
must be perpendicular to the angular velocity and wave number vectors 
within the entire spatial volume.  This was a requirement for the field
solutions to retain a bivector grade (STA/Dirac basis).

Here a specific orientation of the Fourier volume so that two of the axis
lie in the direction of the initial time electric and magnetic fields will be
used.  This is expected to simplify the treatment.

Also note that having obtained some results in a first attempt hindsight
now allows a few choices of variables that will be seen to be appropriate.
The natural motivation for any such choices can be found in the initial
treatment.

\subsection{ Notation. }

Conventions, definitions, and notation used here will largely follow
\cite{PJFourierVacuum}.  Also of possible aid in that document is a 
a table of symbols and their definitions.

\section{ A consise review of results. }

A Fourier series and the Fourier coefficients are

\begin{align}
f(\Bx) &= \sum_{\Bk} \hat{f}_{\Bk} e^{ - i \Bk \cdot \Bx } \\
\hat{f}_{\Bk} &= \inv{V} \int f(\Bx) e^{ i \Bk \cdot \Bx } d^3 x
\end{align}

In the vector context $\Bk$ is

\begin{align}
\Bk = 2 \pi \sum_m \sigma^m \frac{k_m}{\lambda_m}
\end{align}

Where $\lambda_m$ are the dimensions of the volume of integration, 
$V = \lambda_1 \lambda_2 \lambda_3$ is the volume, and
in an index context $\Bk = \{k_1, k_2, k_3\}$ is a triplet of integers,
positive, negative or zero.

\subsection{ Fourier series and coefficients. }

We want to find (STA) bivector solutions $F$ to the vacuum Maxwell equation

\begin{align}
\grad F = \gamma_0 (\partial_0 + \spacegrad) F = 0
\end{align}

We start by assuming a Fourier series solution of the form

\begin{align}
F(\Bx,t) &= \sum_{\Bk} \hat{F}_{\Bk} e^{-i \Bk \cdot \Bx} 
\end{align}



\bibliographystyle{plainnat}
\bibliography{myrefs}

\end{document}

\include{potential_fourier}
\part{To Sort}
\include{maxwell_tensor_lagrangian}
\include{biot_savart}
\include{dc_power}
\include{lorentz_rotation}
\include{maxwell_tensor_from_lagrangian}
\include{boost_maxwell_lagrangian}
\include{charge_arc_element}
\include{charge_line_element}
\include{electron_rotor}
\include{em_potential}
\include{en_m_tensor}
\include{field_lagrangian}
\include{ga_maxwell}
\include{gafp_lorentz}
\include{lagrangian_field_density}
\include{lorentz_force}
\include{lorentz_force_p_qA}
\include{macroscopic_maxwell}
\include{maxwell_to_tensor}
\include{noethers_field}
\include{sg_mx_41}
\include{sr_lagrangian}
\include{stokes_maxwell_application}
\include{stress_energy_lorentz}
\include{vector_maxwells_projection}
\include{lorentz_force_tx}
%
% Copyright � 2012 Peeter Joot.  All Rights Reserved.
% Licenced as described in the file LICENSE under the root directory of this GIT repository.
%

%
%
\chapter{Levi-Civitica summation identity}
\index{Levi-Civitica tensor}
\label{chap:levi}
%\date{March 13, 2009.  levi.tex}

\section{Motivation}

In \citep{byron1992mca} it is left to the reader to show

\index{contraction!Levi-Civitica tensor}
\begin{equation}\label{eqn:levi:20}
\begin{aligned}
\sum_k \epsilon_{ijk} \epsilon_{klm} = \delta_{il}\delta_{jm} - \delta_{jl}\delta_{im}
\end{aligned}
\end{equation}

\section{A mechanical proof}

Although it is not mathematical, this is easy to prove, at least for 3D.  The
following perl code does the trick

\lstinputlisting{listings/levi.pl}

The output produced has all the variations of indices, such as

\begin{equation}\label{eqn:levi:40}
\begin{aligned}
0 &= \sum_{k=1}^{3} \epsilon_{11k} \epsilon_{k11} = \delta_{11}\delta_{11} - \delta_{11}\delta_{11} \\
0 &= \sum_{k=1}^{3} \epsilon_{11k} \epsilon_{k12} = \delta_{11}\delta_{12} - \delta_{11}\delta_{12} \\
\vdots \\
0 &= \sum_{k=1}^{3} \epsilon_{11k} \epsilon_{k33} = \delta_{13}\delta_{13} - \delta_{13}\delta_{13} \\
0 &= \sum_{k=1}^{3} \epsilon_{12k} \epsilon_{k11} = \delta_{11}\delta_{21} - \delta_{21}\delta_{11} \\
1 &= \sum_{k=1}^{3} \epsilon_{12k} \epsilon_{k12} = \delta_{11}\delta_{22} - \delta_{21}\delta_{12} \\
0 &= \sum_{k=1}^{3} \epsilon_{12k} \epsilon_{k13} = \delta_{11}\delta_{23} - \delta_{21}\delta_{13} \\
-1 &= \sum_{k=1}^{3} \epsilon_{12k} \epsilon_{k21} = \delta_{12}\delta_{21} - \delta_{22}\delta_{11} \\
\vdots \\
\end{aligned}
\end{equation}

\section{Proof using bivector dot product}

This identity can also be derived from an expansion of the bivector
dot product in two different ways.

\begin{equation}\label{eqn:levi:60}
\begin{aligned}
( \Be_i \wedge \Be_j ) \cdot ( \Be_m \wedge \Be_n )
&=
( ( \Be_i \wedge \Be_j ) \cdot \Be_m ) \cdot \Be_n  \\
&=
(
\Be_i ( \Be_j \cdot \Be_m )
-\Be_j ( \Be_i \cdot \Be_m )
) \cdot \Be_n  \\
&=
( \Be_i \delta_{jm} -\Be_j \delta_{im} ) \cdot \Be_n  \\
&=
\delta_{in} \delta_{jm} -\delta_{jn} \delta_{im}
\end{aligned}
\end{equation}

Expressing the wedge product in terms duality, using the pseudoscalar
\(I = \Be_1 \Be_2 \Be_3\), we have

\begin{equation}\label{eqn:levi:80}
\begin{aligned}
(\Be_i \wedge \Be_j ) \Be_k = I \epsilon_{ijk}
\end{aligned}
\end{equation}

Or
\begin{equation}\label{eqn:levi:100}
\begin{aligned}
\Be_i \wedge \Be_j = I \sum_k \epsilon_{ijk} \Be_k
\end{aligned}
\end{equation}

Then the bivector dot product is
\begin{equation}\label{eqn:levi:120}
\begin{aligned}
( \Be_i \wedge \Be_j ) \cdot ( \Be_m \wedge \Be_n )
&=
\gpgradezero{
I \sum_k \epsilon_{ijk} \Be_k I \sum_p \epsilon_{mnp} \Be_p
} \\
&=
I^2 \sum_{k,p} \epsilon_{ijk} \epsilon_{mnp} \gpgradezero{ \Be_k \Be_p } \\
&=
- \sum_{k,p} \epsilon_{ijk} \epsilon_{mnp} \delta_{kp} \\
&=
- \sum_{k} \epsilon_{ijk} \epsilon_{mnk} \\
\end{aligned}
\end{equation}

Comparing the two expansions we have

\begin{equation}\label{eqn:levi:140}
\begin{aligned}
\sum_{k} \epsilon_{ijk} \epsilon_{mnk} &= \delta_{jn} \delta_{im} - \delta_{in} \delta_{jm}
\end{aligned}
\end{equation}

Which is equivalent to the original identity (after an index switcheroo).
Note both the dimension and metric dependencies in this proof.

\part{Appendix.}
\include{fourier_tx}
\include{fourier_notation}
\include{proj_generalized_dot_prod}
\part{Bibliography, and Index.}
%
% Copyright � 2012 Peeter Joot.  All Rights Reserved.
% Licenced as described in the file LICENSE under the root directory of this GIT repository.
%

%
%
%\chapter{Learning Geometric Algebra/Clifford Algebra}
\chapter{Further reading}
\label{chap:gabookmark}

%\setlength{\textwidth}{13in}
%\setlength{\oddsidemargin}{0in}
%\setlength{\evensidemargin}{0in}

%\section{Learning Geometric Algebra and its applications}

There is a wealth of information on the subject available online, but finding information at an appropriate level may be difficult.  Not all resources use the same notation or nomenclature, and one can get lost in a sea of product operators.
Some of the introductory material also assumes knowledge of various levels of physics.  This is natural since the algebra can be utilized well to expresses many physics concepts.  While natural, this can also be intimidating if one is unprepared, so mathematics that one could potentially understand may be presented
in a fashion that is inaccessible.

%Colected here is an attempt to collect some of the available online information.

\section{Geometric Algebra for Computer Science Book}

The book
\href{http://www.geometricalgebra.net/tour.html}{Geometric Algebra For Computer Science.}
by Dorst, Fontijne, and Mann has one of the best introductions to the subject that I have seen.  It is also fairly inexpensive (\\(60 Canadian).  Compared for example to Hestenes's ``From Clifford Algebra to Geometric Calculus'' which I have seen listed on amazon.com with a default price of \\)250, discounted to \$150.

This book contains particularly good introductions to the dot and wedge products, both for vectors, and the generalizations.  How these can be applied and what they can be used to model is covered excellently.

Compromises have been made in this book on the order to present information, and what level of detail to use and when.  Many proofs are deferred or placed only in the appendix.  For example, they introduce (define) a scalar product initially (denoted with an asterisk (*)), and define this using a determinant without motivation.  This allows for development of a working knowledge of how to apply the subject.

Once an ability to apply has been developed they proceed with an axiomatic development.  I would consider an axiomatic approach to the subject very important since there is a sea of identities associated with the algebra.  Figuring out which ones are consequences of the others can be difficult, if one starts with definitions that are not fundamental.  One can easily go in circles and wonder really are the basic rules (this was my first impression starting with the Hestenes book ``New Foundations for Classical Mechanics''.   The book ``Geometric Algebra for Physicists'' has an excellent axiomatic development.  It however notably makes a similar compromise first introducing the algebra with a dot plus wedge product formulation to develop some familiarity.

This book has three parts.  The first is on the algebra, covering the generalized dot and wedge products, rotors, projections, join, linear transformations as outermorphisms, and all the rest of the basic material that one would expect.  It does this excellently.

The second portion of this book is on the use of a 5D conformal model for 3D graphics (adding a point at infinity on top of the normal extra viewport dimension that traditional graphics applications use).  I can not comment too much on this part of the book since I loaned it to a friend after reading the first and last parts of the book.

The last part of the book is on implementation, and makes for an interesting read.  Details on their Gaigen implementation are discussed, as are performance and code size implications of their implementation.

The only thing negative I have to say about this book is the unfortunate introduction of an alternate notation for the generalized dot product (L and backwards L).  This is distracting if one started, like I did, with the Hestenes, Cambridge, or Baylis papers or books, and their notation dominates the literature as far as I can tell.  This does not take too long to adjust, since one mostly just has to mentally substitute dots for L's (although there are some subtle differences where this transposition does not necessarily work).

\section{GAViewer}

Performing the GAViewer tutorial exercises is a great way to build some intuition to go along with the math (putting the geometric back in the algebra).

There are specific GAViewer exercises that you can do independent of the book, and there is also an excellent interactive tutorial 2003 Game Developer Lecture available here:

\href{http://www.science.uva.nl/ga/tutorials/}{Interactive GA tutorial. UvA GA Website: Tutorials}

 (they have hijacked GAViewer here to use as presentation software, but you can go through things at your own pace, and do things such as rotating viewpoints). Quite neat, and worth doing just to play with the graphical cross product manipulation even if you decide not to learn GA.
\section{Other resources from Dorst, Fontijne, and Mann}

There are other web resources available associated with this book that are quite good. The best of these is GAViewer, a graphical geometric calculator that was the product of some of the research that generated this book.

See
%\href{http://staff.science.uva.nl/~fontijne/phd.html}{Daniel Fontijne PhD thesis}
, or his paper itself
%\href{http://staff.science.uva.nl/~fontijne/phd/fontijne_phd.pdf}{fontijne_phd.pdf}
.

Some other links:

\href{http://staff.science.uva.nl/~leo/clifford/index.html}{Geometric algebra (Clifford algebra)}


This is
\href{http://staff.science.uva.nl/~leo/clifford/dorst-mann-I.pdf}{a good tutorial}
, as it focuses on the geometrical rather than have any tie to physics (fun but more to know).  The following looks like a slightly longer updated version:

\href{http://staff.science.uva.nl/~leo/clifford/dorst-mann-I.pdf}{GA: a practical tool for efficient geometric representation (Dorst)}

\section{Learning GA}
Of the various GA primers and workbooks above, here are a couple specific documents that are noteworthy, and some direct links to a few things that can be found by browsing that were noteworthy.
This is an
\href{http://www.science.uva.nl/ga/tutorials/}{interactive GA tutorial/presentation for a game programmers conference}

that provides a really good intro and has a lot of examples that I found helpful to get an intuitive feel for all the various product operations and object types.
Even if you weare not trying to learn GA, if you have done any traditional vector algebra/calculus, IMO its worthwhile to download this just to just to see the animation of how the old cross product varies with changes to the vectors.
You have to download the GAViewer program (graphical vector calculator) to run the presentation. Once you do that you can use it for other calculation examples, such as those available in these examples of how to use GAViewer as a standalone tool.. Note that the book the drills are from use a different notation for dot product (with a slightly different meaning and uses an oriented L symbol dependent on the grades of the blades.

\href{http://www.lomont.org/Math/GeometricAlgebra/Geometric%20Algebra%20Primer%20-%20Suter%20-%202003.pdf}{Jaap Suter's GA primer}.
\href{http://www.jaapsuter.com/}{His website}, which is referenced in various GA papers no longer (at least obviously) has this primer on it any longer (Sept/2008).

\href{http://www.iancgbell.clara.net/maths/geoalg.htm}{Ian Bell's introduction to GA}

    This author has a wide range of GA information, but looking at it will probably give you a headache.


\href{http://en.wikipedia.org/wiki/Geometric_algebra.}{GA wikipedia}

There are a number of comparisons here between GA identities and traditional vector identities, that may be helpful to get oriented.

- Maths - Clifford / Geometric Algebra - Martin Baker

A GA intro, a small part in the much larger Euclidean space website.

- As mentioned above there is a lot of learning GA content available in the Cambridge/Baylis/Hestenes/Dorst/... sites.

\section{Cambridge}
The Cambridge GA group has a number of Geometric Algebra publications, including the book

\href{http://www.mrao.cam.ac.uk/~cjld1/pages/book.htm}{Geometric Algebra for Physicists}

This book has an excellent introductory treatment of a number of basic GA concepts, a number of which are much easier to follow than similar content in Hestenes's "New Foundations for Classical Mechanics".  When it comes to physics content in this book there are a lot of details left out, so it is not the best for learning the physics itself if you are new to the topic in question.

Much of the content of their book
is actually available online in their publications above, but it is hard to beat coherent organization and a paper version that you can mark up.

Some other online learning content from the Cambridge group includes

\href{http://www.mrao.cam.ac.uk/~clifford/introduction/index.html}{Introduction to Geometric Algebra}

This is an HTML version of the
\href{http://www.mrao.cam.ac.uk/~clifford/publications/ps/imag_numbs.pdf}{Imaginary numbers are not real paper.}

A
nice starting point is lect1.pdf from the
\href{http://www.mrao.cam.ac.uk/~clifford/ptIIIcourse/GeometricAlgebraLectures.zip}{Cambridge PartIII physics course on GA applications}
. Only at the very end of this first pdf is any real physics content.
taught to what sounds like final year undergrad physics students.  The first parts of this do not need much physics knowledge.

\section{Baylis}

\href{http://www.uwindsor.ca/users/b/baylis/main.nsf}{Wiliam Baylis GA page}

He uses a scalar plus vector multivector representation for relativity (APS, Algebra of Physical Space), and an associated conjugate length operation.  You will find an intro relativity, GA workbook, and some papers on GA applied to physics here.
Also based on his APS approach is the following wikibook:

\href{http://en.wikibooks.org/wiki/Physics_in_the_Language_of_Geometric_Algebra._An_Approach_with_the_Algebra_of_Physical_Space}{Physics in the Language of Geometric Algebra. An Approach with the Algebra of Physical Space}

\section{Hestenes}
Hestenes main page for GA is
\href{http://modelingnts.la.asu.edu/}{Geometric Calculus R \& D Home Page}

This includes a number of primers and introductions to the subject such as
\href{http://modelingnts.la.asu.edu/pdf/PrimerGeometricAlgebra.pdf}{Geometric Algebra Primer.}
As described in the
\href{http://modelingnts.la.asu.edu/html/IntroPrimerGeometricAlgebra.html}{Introduction page for this primer}, this is a workbook, and reading should not be passive.

Also available is his
\href{http://modelingnts.la.asu.edu/pdf/OerstedMedalLecture.pdf}{Oersted Lecture}, which contains a good introduction.

If you do not have his ``New Foundations of Classical Mechanics'' book, you can find some of the dot-product/wedge-product reduction formulas in the following
\href{http://modelingnts.la.asu.edu/pdf/UGA.pdf}{non-metric treatment of GA.}

Also interesting is this
\href{http://modelingnts.la.asu.edu/pdf/GTG.w.GC.FP.pdf}{Gauge Theory Gravity with Geometric Algebra}
paper.  This has an introduction to STA (Space Time Algebra) as used in the Cambridge books.  This also shows at a high level where one can go with a lot of these ideas (like the grad F = J formulation of Maxwell's equation, a multivector form that incorporates all of the traditional four vector Maxwell's equations).  Nice teaser document if you intend to use GA for physics study, but hard to read even the consumable bits because they are buried in among a lot of other higher level math and physics.


Hestenes, Li and Rockwood in their paper
\href{http://modelingnts.la.asu.edu/pdf/CompGeom-ch1.pdf}{ New Algebraic Tools for Classical Geometry}
in G. Sommer (ed.) Geom.
Computing with Clifford Algebras (Springer, 2001) treat outermorphisms
and determinants in a separate subsection entitled "Outermorphism"
of section 1.3 Linear Transformations:

This is a comprehensive doc.  Content includes:
\begin{itemize}
\item GA intro boilerplate.
\item Projection and Rejection.
\item Meet and Join.
\item Reciprocal vectors (dual frame).
\item Vector differentiation.
\item Linear transformations.
\item Determinants and outermorphisms.
\item Rotations.
\item Simplexes and boundaries
\item Dual quaternions.
\end{itemize}

%\section{Peeter's GA/Physics Topics}
%
%\href{http://en.wikipedia.org/wiki/Geometric_algebra}{GA wikipedia}
%
%I dumped a bunch of info in this doc as I was puzzling things out (initially before I had purchased any books).  I have since stopped contributing since it was getting too big and was no longer appropriate (ie: getting bookish instead of encyclopedic).
%
%\href{http://sites.google.com/site/peeterjoot/}{Instead I have got a bunch of standalone latex writeups of my notes here.}
%
%I have a number of smallish GA related documents, which in reverse chronological order document my own learning/relearning roadmap (for GA and physics).  Hopefully useful for others too.  Please email peeterjoot@protonmail.com if any mistakes are found (other than ones that are already described in the index that I have not gone back to fix).
%

\section{Eckhard M. S. Hitzer (University of Fukui)}

From
\href{http://sinai.mech.fukui-u.ac.jp/gala2/}{Eckhard's Geometric Algebra Topics.}

Since these are all specific documents, and all at a fairly consumable level for a new learner, I have listed them here specifically:

\begin{itemize}
\item
\href{http://sinai.mech.fukui-u.ac.jp/gala2/GAtopics/axioms.pdf}{Axioms of geometric algebra}
\item
\href{http://sinai.mech.fukui-u.ac.jp/gala2/GAtopics/qform.pdf}{The use of quadratic forms in geometric algebra}
\item
\href{http://sinai.mech.fukui-u.ac.jp/gala2/GAtopics/products.pdf}{The geometric product and derived products}
\item
\href{http://sinai.mech.fukui-u.ac.jp/gala2/GAtopics/det.pdf}{Determinants in geometric algebra}
\item
\href{http://sinai.mech.fukui-u.ac.jp/gala2/GAtopics/GS.pdf}{Gram-Schmidt orthogonalization in geometric algebra}
\item
\href{http://sinai.mech.fukui-u.ac.jp/gala2/GAtopics/WhatIsi.pdf}{What is an imaginary number?}
\item
\href{http://sinai.mech.fukui-u.ac.jp/gcj/publications/mvdifcalc/mvdc.pdf}{Simplical calculus:}
\end{itemize}

\section{Electrodynamics}

John Denker has a number of GA docs that all appear very readable.  One such doc is:

\href{http://www.av8n.com/physics/maxwell-ga.pdf}{Electromagnetism using Geometric Algebra versus Components}

This is a nice little doc (there is also an HTML version, but it is very hard to read, and the first time I saw it I actually missed a lot of content).

The oft repeated introduction to GA is not in this doc, so you have to know the basics first.  Denker takes the \(\grad F = J/c \epsilon_0\) equation and unpacks it in a brute force but understandable fashion, and shows that these are identical to the vector differential form of Maxwell's equations.  A few other E\&M constructs are shown in their GA form (covariant form of Lorentz force equation, Lagrangian density, Stress tensor, Poynting Vector.  There are also many good comments on notation issues.

A cautionary note if you have read any of the Cambridge papers.  This doc uses a -+++ metric instead of the +--- used in those docs.

Some other Denker GA papers:
\begin{itemize}
\item
\href{http://www.av8n.com/physics/straight-wire.pdf}{Magnetic field of a straight wire.}
\item
\href{http://www.av8n.com/physics/clifford-intro.pdf}{Clifford Intro.}

Very nice axiomatic introduction with excellent commentary.  Also includes an STA intro.

\item
\href{http://www.av8n.com/physics/complex-clifford.pdf}{Complex numbers.}
\item
\href{http://www.av8n.com/physics/area-volume.pdf}{Area and Volume.}
\item
\href{http://www.av8n.com/physics/rotations.pdf}{Rotations.}
\end{itemize}
(have not read all these yet).


Richard E. Harke,
\href{http://www.harke.org/ps/intro.ps.gz}{An Introduction to the Mathematics of the Space-Time Algebra}

This is a nice complete little doc (\~40 pages), where many basic GA constructs are developed axiomatically with associated proofs.  This includes some simplical calculus and outermorphism content, and eventually moves on to STA and Lorentz rotations.



\section{Misc}

\begin{itemize}
\item
A blog like
\href{http://gaupdate.wordpress.com/}{subscription service that carries abstracts}
for various papers on or using Geometric Algebra.
\end{itemize}

\section{Collections of other GA bookmarks}

\begin{itemize}
\item
\href{http://www.geomerics.com/geometric-algebra.htm}{Geomerics.  Graphics software for Games, Geometric Algebra references and description.}
\item
\href{http://www.xtec.es/~rgonzal1/links.htm}{Ramon Gonz�lez Calvet us GA links.}
\item
\href{http://www.rwgrayprojects.com/GeometricAlgebra/references.html}{R. W. Gray's GA links.}
\item
\href{http://www.mrao.cam.ac.uk/~clifford/pages/links.htm}{Cambridge groups GA urls.}
\end{itemize}

\section{Exterior Algebra and differential forms}

\begin{itemize}
\item
\href{http://www.grassmannalgebra.info/grassmannalgebra/book/index.htm}{Grassmann Algebra Book}

Pdf files of a book draft entitled Grassmann Algebra: Exploring applications of extended vector algebra with Mathematica.

This has some useful info.  In particular, a great example of solving linear systems with the wedge product.
\item
The Cornell Library Historical Mathematics Monographs -
\href{http://historical.library.cornell.edu/cgi-bin/cul.math/docviewer?did=00540001&seq=15&frames=0&view=50}{hyde on grassman}
\item
\href{http://www.math.boun.edu.tr/instructors/ozturk/eskiders/fall04math488/bachman.pdf}{A Geometric Approach to Differential Forms by David Bachman}
\end{itemize}

\section{Software}

\begin{itemize}
\item
\href{http://staff.science.uva.nl/~fontijne/gaigen2.html}{Gaigen 2}
\item
\href{http://users.tkk.fi/~ppuska/mirror/Lounesto/CLICAL.htm}{CLICAL for Clifford Algebra Calculations}
\item
\href{http://www.nklein.com/products/geoma/}{nklein software.  Geoma.}
\end{itemize}


% END INCLUDES.
%-------------------------------------------------------
%\begin{thebibliography}{99}
%  \addcontentsline{toc}{chapter}{Bibliography}
%\bibitem{lamport} L. Lamport. {\bf \LaTeX \ A Document Preparation System}
%Addison-Wesley, California 1986.
%
%\end{thebibliography}

\bibliographystyle{plainnat}
  \addcontentsline{toc}{chapter}{Bibliography}
\bibliography{myrefs}

\documentclass[openany]{memoir}
\usepackage[]{makeidx}

\chapterstyle{ell}

\makeindex

\begin{document}

To solve various problems in physics, it can be advantageous
to express any arbitrary piecewise-smooth function as a Fourier Series
\index{Fourier Series}
composed of multiples of sine \index{sine} and cosine \index{cosine} functions.  These are used in \cite{acheson1990elementary}.

\printindex

\bibliography{myrefs}
\bibliographystyle{unsrturl}

\end{document}

  \addcontentsline{toc}{chapter}{Index}
\end{document}
