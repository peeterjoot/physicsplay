\documentclass[12pt,leqno]{book}
\usepackage{amsmath,amssymb,amsfonts} % Typical maths resource packages
%\usepackage{graphics}                 % Packages to allow inclusion of graphics
\usepackage{graphicx}
\usepackage{color}                    % For creating coloured text and background

% for ointctr... (also appears to make "prettier" \int and \sum's)
% ... but messes up other stuff (grad vs \spacegrad).
\usepackage{txfonts} 

%\usepackage{listings}
%\usepackage{latexsym,epsf}

\usepackage[bookmarks=true]{hyperref}

\parindent 1cm
\parskip 0.2cm
\topmargin 0.2cm
\oddsidemargin 1cm
\evensidemargin 0.5cm
\textwidth 15cm
\textheight 21cm

\newtheorem{theorem}{Theorem}[section]
\newtheorem{proposition}[theorem]{Proposition}
\newtheorem{corollary}[theorem]{Corollary}
\newtheorem{lemma}[theorem]{Lemma}
\newtheorem{remark}[theorem]{Remark}
\newtheorem{definition}[theorem]{Definition}

\usepackage{amsmath}
\usepackage{mathpazo}

%
% shorthand for bold symbols, convenient for vectors and matrices
%
\newcommand{\Ba}[0]{\mathbf{a}}
\newcommand{\Bb}[0]{\mathbf{b}}
\newcommand{\Bc}[0]{\mathbf{c}}
\newcommand{\Bd}[0]{\mathbf{d}}
\newcommand{\Be}[0]{\mathbf{e}}
\newcommand{\Bf}[0]{\mathbf{f}}
\newcommand{\Bg}[0]{\mathbf{g}}
\newcommand{\Bh}[0]{\mathbf{h}}
\newcommand{\Bi}[0]{\mathbf{i}}
\newcommand{\Bj}[0]{\mathbf{j}}
\newcommand{\Bk}[0]{\mathbf{k}}
\newcommand{\Bl}[0]{\mathbf{l}}
\newcommand{\Bm}[0]{\mathbf{m}}
\newcommand{\Bn}[0]{\mathbf{n}}
\newcommand{\Bo}[0]{\mathbf{o}}
\newcommand{\Bp}[0]{\mathbf{p}}
\newcommand{\Bq}[0]{\mathbf{q}}
\newcommand{\Br}[0]{\mathbf{r}}
\newcommand{\Bs}[0]{\mathbf{s}}
\newcommand{\Bt}[0]{\mathbf{t}}
\newcommand{\Bu}[0]{\mathbf{u}}
\newcommand{\Bv}[0]{\mathbf{v}}
\newcommand{\Bw}[0]{\mathbf{w}}
\newcommand{\Bx}[0]{\mathbf{x}}
\newcommand{\By}[0]{\mathbf{y}}
\newcommand{\Bz}[0]{\mathbf{z}}
\newcommand{\BA}[0]{\mathbf{A}}
\newcommand{\BB}[0]{\mathbf{B}}
\newcommand{\BC}[0]{\mathbf{C}}
\newcommand{\BD}[0]{\mathbf{D}}
\newcommand{\BE}[0]{\mathbf{E}}
\newcommand{\BF}[0]{\mathbf{F}}
\newcommand{\BG}[0]{\mathbf{G}}
\newcommand{\BH}[0]{\mathbf{H}}
\newcommand{\BI}[0]{\mathbf{I}}
\newcommand{\BJ}[0]{\mathbf{J}}
\newcommand{\BK}[0]{\mathbf{K}}
\newcommand{\BL}[0]{\mathbf{L}}
\newcommand{\BM}[0]{\mathbf{M}}
\newcommand{\BN}[0]{\mathbf{N}}
\newcommand{\BO}[0]{\mathbf{O}}
\newcommand{\BP}[0]{\mathbf{P}}
\newcommand{\BQ}[0]{\mathbf{Q}}
\newcommand{\BR}[0]{\mathbf{R}}
\newcommand{\BS}[0]{\mathbf{S}}
\newcommand{\BT}[0]{\mathbf{T}}
\newcommand{\BU}[0]{\mathbf{U}}
\newcommand{\BV}[0]{\mathbf{V}}
\newcommand{\BW}[0]{\mathbf{W}}
\newcommand{\BX}[0]{\mathbf{X}}
\newcommand{\BY}[0]{\mathbf{Y}}
\newcommand{\BZ}[0]{\mathbf{Z}}

\newcommand{\Bzero}[0]{\mathbf{0}}
\newcommand{\Btheta}[0]{\boldsymbol{\theta}}
\newcommand{\Btau}[0]{\boldsymbol{\tau}}
\newcommand{\Bomega}[0]{\boldsymbol{\omega}}

%
% shorthand for unit vectors
%
\newcommand{\acap}[0]{\hat{\Ba}}
\newcommand{\bcap}[0]{\hat{\Bb}}
\newcommand{\ccap}[0]{\hat{\Bc}}
\newcommand{\dcap}[0]{\hat{\Bd}}
\newcommand{\ecap}[0]{\hat{\Be}}
\newcommand{\fcap}[0]{\hat{\Bf}}
\newcommand{\gcap}[0]{\hat{\Bg}}
\newcommand{\hcap}[0]{\hat{\Bh}}
\newcommand{\icap}[0]{\hat{\Bi}}
\newcommand{\jcap}[0]{\hat{\Bj}}
\newcommand{\kcap}[0]{\hat{\Bk}}
\newcommand{\lcap}[0]{\hat{\Bl}}
\newcommand{\mcap}[0]{\hat{\Bm}}
\newcommand{\ncap}[0]{\hat{\Bn}}
\newcommand{\ocap}[0]{\hat{\Bo}}
\newcommand{\pcap}[0]{\hat{\Bp}}
\newcommand{\qcap}[0]{\hat{\Bq}}
\newcommand{\rcap}[0]{\hat{\Br}}
\newcommand{\scap}[0]{\hat{\Bs}}
\newcommand{\tcap}[0]{\hat{\Bt}}
\newcommand{\ucap}[0]{\hat{\Bu}}
\newcommand{\vcap}[0]{\hat{\Bv}}
\newcommand{\wcap}[0]{\hat{\Bw}}
\newcommand{\xcap}[0]{\hat{\Bx}}
\newcommand{\ycap}[0]{\hat{\By}}
\newcommand{\zcap}[0]{\hat{\Bz}}
\newcommand{\thetacap}[0]{\hat{\Btheta}}

%
% to write R^n and C^n in a distinguishable fashion.  Perhaps change this
% to the double lined characters upon figuring out how to do so.
%
\newcommand{\C}[1]{$\mathbb{C}^{#1}$}
\newcommand{\R}[1]{$\mathbb{R}^{#1}$}

%
% various generally useful helpers
%

% derivative of #1 wrt. #2:
\newcommand{\D}[2] {\frac {d#2} {d#1}}

\newcommand{\inv}[1]{\frac{1}{#1}}
\newcommand{\cross}[0]{\times}

\newcommand{\abs}[1]{\lvert{#1}\rvert}
\newcommand{\norm}[1]{\lVert{#1}\rVert}
\newcommand{\innerprod}[2]{\langle{#1}, {#2}\rangle}
\newcommand{\dotprod}[2]{{#1} \cdot {#2}}
\newcommand{\bdotprod}[2]{\left({#1} \cdot {#2}\right)}
\newcommand{\crossprod}[2]{{#1} \cross {#2}}
\newcommand{\tripleprod}[3]{\dotprod{\left(\crossprod{#1}{#2}\right)}{#3}}

\DeclareMathOperator{\Proj}{Proj}
\DeclareMathOperator{\Span}{span}
\DeclareMathOperator{\Sgn}{sgn}
\DeclareMathOperator{\Area}{Area}
\DeclareMathOperator{\Volume}{Volume}

%
% A few miscellaneous things specific to this document
%
\newcommand{\crossop}[1]{\crossprod{#1}{}}

% R2 vector.
\newcommand{\VectorTwo}[2]{
\begin{bmatrix}
 {#1} \\
 {#2}
\end{bmatrix}
}

\newcommand{\VectorN}[1]{
\begin{bmatrix}
{#1}_1 \\
{#1}_2 \\
\vdots \\
{#1}_N \\
\end{bmatrix}
}

\newcommand{\DETuvij}[4]{
\begin{vmatrix}
 {#1}_{#3} & {#1}_{#4} \\
 {#2}_{#3} & {#2}_{#4}
\end{vmatrix}
}

\newcommand{\DETuvwijk}[6]{
\begin{vmatrix}
 {#1}_{#4} & {#1}_{#5} & {#1}_{#6} \\
 {#2}_{#4} & {#2}_{#5} & {#2}_{#6} \\
 {#3}_{#4} & {#3}_{#5} & {#3}_{#6}
\end{vmatrix}
}

\newcommand{\DETuvwxijkl}[8]{
\begin{vmatrix}
 {#1}_{#5} & {#1}_{#6} & {#1}_{#7} & {#1}_{#8} \\
 {#2}_{#5} & {#2}_{#6} & {#2}_{#7} & {#2}_{#8} \\
 {#3}_{#5} & {#3}_{#6} & {#3}_{#7} & {#3}_{#8} \\
 {#4}_{#5} & {#4}_{#6} & {#4}_{#7} & {#4}_{#8} \\
\end{vmatrix}
}

%\newcommand{\DETuvwxyijklm}[10]{
%\begin{vmatrix}
% {#1}_{#6} & {#1}_{#7} & {#1}_{#8} & {#1}_{#9} & {#1}_{#10} \\
% {#2}_{#6} & {#2}_{#7} & {#2}_{#8} & {#2}_{#9} & {#2}_{#10} \\
% {#3}_{#6} & {#3}_{#7} & {#3}_{#8} & {#3}_{#9} & {#3}_{#10} \\
% {#4}_{#6} & {#4}_{#7} & {#4}_{#8} & {#4}_{#9} & {#4}_{#10} \\
% {#5}_{#6} & {#5}_{#7} & {#5}_{#8} & {#5}_{#9} & {#5}_{#10}
%\end{vmatrix}
%}

% R3 vector.
\newcommand{\VectorThree}[3]{
\begin{bmatrix}
 {#1} \\
 {#2} \\
 {#3}
\end{bmatrix}
}


%<misc>
%
\newcommand{\Abs}[1]{{\left\lvert{#1}\right\rvert}}
\newcommand{\spacegrad}[0]{\boldsymbol{\nabla}}
\newcommand{\grad}[0]{\nabla}
\newcommand{\LL}[0]{\mathcal{L}}

% == \partial_{#1} {#2}
\newcommand{\PD}[2]{\frac{\partial {#2}}{\partial {#1}}}
% inline variant
\newcommand{\PDi}[2]{{\partial {#2}}/{\partial {#1}}}

\newcommand{\PDD}[3]{\frac{\partial^2 {#3}}{\partial {#1}\partial {#2}}}
%\newcommand{\PDd}[2]{\frac{\partial^2 {#2}}{{\partial{#1}}^2}}
\newcommand{\PDsq}[2]{\frac{\partial^2 {#2}}{(\partial {#1})^2}}

\newcommand{\Partial}[2]{\frac{\partial {#1}}{\partial {#2}}}
\DeclareMathOperator{\RejName}{Rej}
\newcommand{\Rej}[2]{\RejName_{#1}\left( {#2} \right)}
\newcommand{\Rm}[1]{\mathbb{R}^{#1}}
\newcommand{\Cm}[1]{\mathbb{C}^{#1}}
\newcommand{\conj}[0]{{*}}

%</misc>

% <grade selection>
%
\newcommand{\gpgrade}[2] {{\left\langle{{#1}}\right\rangle}_{#2}}

\newcommand{\gpgradezero}[1] {\gpgrade{#1}{}}
%\newcommand{\gpscalargrade}[1] {{\left\langle{{#1}}\right\rangle}}
%\newcommand{\gpgradezero}[1] {\gpgrade{#1}{0}}

%\newcommand{\gpgradeone}[1] {{\left\langle{{#1}}\right\rangle}_{1}}
\newcommand{\gpgradeone}[1] {\gpgrade{#1}{1}}

\newcommand{\gpgradetwo}[1] {\gpgrade{#1}{2}}
\newcommand{\gpgradethree}[1] {\gpgrade{#1}{3}}
\newcommand{\gpgradefour}[1] {\gpgrade{#1}{4}}
%
% </grade selection>



\newcommand{\adot}[0]{{\dot{a}}}
\newcommand{\bdot}[0]{{\dot{b}}}
% taken for centered dot:
%\newcommand{\cdot}[0]{{\dot{c}}}
%\newcommand{\ddot}[0]{{\dot{d}}}
\newcommand{\edot}[0]{{\dot{e}}}
\newcommand{\fdot}[0]{{\dot{f}}}
\newcommand{\gdot}[0]{{\dot{g}}}
\newcommand{\hdot}[0]{{\dot{h}}}
\newcommand{\idot}[0]{{\dot{i}}}
\newcommand{\jdot}[0]{{\dot{j}}}
\newcommand{\kdot}[0]{{\dot{k}}}
\newcommand{\ldot}[0]{{\dot{l}}}
\newcommand{\mdot}[0]{{\dot{m}}}
\newcommand{\ndot}[0]{{\dot{n}}}
%\newcommand{\odot}[0]{{\dot{o}}}
\newcommand{\pdot}[0]{{\dot{p}}}
\newcommand{\qdot}[0]{{\dot{q}}}
\newcommand{\rdot}[0]{{\dot{r}}}
\newcommand{\sdot}[0]{{\dot{s}}}
\newcommand{\tdot}[0]{{\dot{t}}}
\newcommand{\udot}[0]{{\dot{u}}}
\newcommand{\vdot}[0]{{\dot{v}}}
\newcommand{\wdot}[0]{{\dot{w}}}
\newcommand{\xdot}[0]{{\dot{x}}}
\newcommand{\ydot}[0]{{\dot{y}}}
\newcommand{\zdot}[0]{{\dot{z}}}
\newcommand{\addot}[0]{{\ddot{a}}}
\newcommand{\bddot}[0]{{\ddot{b}}}
\newcommand{\cddot}[0]{{\ddot{c}}}
%\newcommand{\dddot}[0]{{\ddot{d}}}
\newcommand{\eddot}[0]{{\ddot{e}}}
\newcommand{\fddot}[0]{{\ddot{f}}}
\newcommand{\gddot}[0]{{\ddot{g}}}
\newcommand{\hddot}[0]{{\ddot{h}}}
\newcommand{\iddot}[0]{{\ddot{i}}}
\newcommand{\jddot}[0]{{\ddot{j}}}
\newcommand{\kddot}[0]{{\ddot{k}}}
\newcommand{\lddot}[0]{{\ddot{l}}}
\newcommand{\mddot}[0]{{\ddot{m}}}
\newcommand{\nddot}[0]{{\ddot{n}}}
\newcommand{\oddot}[0]{{\ddot{o}}}
\newcommand{\pddot}[0]{{\ddot{p}}}
\newcommand{\qddot}[0]{{\ddot{q}}}
\newcommand{\rddot}[0]{{\ddot{r}}}
\newcommand{\sddot}[0]{{\ddot{s}}}
\newcommand{\tddot}[0]{{\ddot{t}}}
\newcommand{\uddot}[0]{{\ddot{u}}}
\newcommand{\vddot}[0]{{\ddot{v}}}
\newcommand{\wddot}[0]{{\ddot{w}}}
\newcommand{\xddot}[0]{{\ddot{x}}}
\newcommand{\yddot}[0]{{\ddot{y}}}
\newcommand{\zddot}[0]{{\ddot{z}}}

%<bold and dot greek symbols>
%

\newcommand{\Deltadot}[0]{{\dot{\Delta}}}
\newcommand{\Gammadot}[0]{{\dot{\Gamma}}}
\newcommand{\Lambdadot}[0]{{\dot{\Lambda}}}
\newcommand{\Omegadot}[0]{{\dot{\Omega}}}
\newcommand{\Phidot}[0]{{\dot{\Phi}}}
\newcommand{\Pidot}[0]{{\dot{\Pi}}}
\newcommand{\Psidot}[0]{{\dot{\Psi}}}
\newcommand{\Sigmadot}[0]{{\dot{\Sigma}}}
\newcommand{\Thetadot}[0]{{\dot{\Theta}}}
\newcommand{\Upsilondot}[0]{{\dot{\Upsilon}}}
\newcommand{\Xidot}[0]{{\dot{\Xi}}}
\newcommand{\alphadot}[0]{{\dot{\alpha}}}
\newcommand{\betadot}[0]{{\dot{\beta}}}
\newcommand{\chidot}[0]{{\dot{\chi}}}
\newcommand{\deltadot}[0]{{\dot{\delta}}}
\newcommand{\epsilondot}[0]{{\dot{\epsilon}}}
\newcommand{\etadot}[0]{{\dot{\eta}}}
\newcommand{\gammadot}[0]{{\dot{\gamma}}}
\newcommand{\kappadot}[0]{{\dot{\kappa}}}
\newcommand{\lambdadot}[0]{{\dot{\lambda}}}
\newcommand{\mudot}[0]{{\dot{\mu}}}
\newcommand{\nudot}[0]{{\dot{\nu}}}
\newcommand{\omegadot}[0]{{\dot{\omega}}}
\newcommand{\phidot}[0]{{\dot{\phi}}}
\newcommand{\pidot}[0]{{\dot{\pi}}}
\newcommand{\psidot}[0]{{\dot{\psi}}}
\newcommand{\rhodot}[0]{{\dot{\rho}}}
\newcommand{\sigmadot}[0]{{\dot{\sigma}}}
\newcommand{\taudot}[0]{{\dot{\tau}}}
\newcommand{\thetadot}[0]{{\dot{\theta}}}
\newcommand{\upsilondot}[0]{{\dot{\upsilon}}}
\newcommand{\varepsilondot}[0]{{\dot{\varepsilon}}}
\newcommand{\varphidot}[0]{{\dot{\varphi}}}
\newcommand{\varpidot}[0]{{\dot{\varpi}}}
\newcommand{\varrhodot}[0]{{\dot{\varrho}}}
\newcommand{\varsigmadot}[0]{{\dot{\varsigma}}}
\newcommand{\varthetadot}[0]{{\dot{\vartheta}}}
\newcommand{\xidot}[0]{{\dot{\xi}}}
\newcommand{\zetadot}[0]{{\dot{\zeta}}}

\newcommand{\Deltaddot}[0]{{\ddot{\Delta}}}
\newcommand{\Gammaddot}[0]{{\ddot{\Gamma}}}
\newcommand{\Lambdaddot}[0]{{\ddot{\Lambda}}}
\newcommand{\Omegaddot}[0]{{\ddot{\Omega}}}
\newcommand{\Phiddot}[0]{{\ddot{\Phi}}}
\newcommand{\Piddot}[0]{{\ddot{\Pi}}}
\newcommand{\Psiddot}[0]{{\ddot{\Psi}}}
\newcommand{\Sigmaddot}[0]{{\ddot{\Sigma}}}
\newcommand{\Thetaddot}[0]{{\ddot{\Theta}}}
\newcommand{\Upsilonddot}[0]{{\ddot{\Upsilon}}}
\newcommand{\Xiddot}[0]{{\ddot{\Xi}}}
\newcommand{\alphaddot}[0]{{\ddot{\alpha}}}
\newcommand{\betaddot}[0]{{\ddot{\beta}}}
\newcommand{\chiddot}[0]{{\ddot{\chi}}}
\newcommand{\deltaddot}[0]{{\ddot{\delta}}}
\newcommand{\epsilonddot}[0]{{\ddot{\epsilon}}}
\newcommand{\etaddot}[0]{{\ddot{\eta}}}
\newcommand{\gammaddot}[0]{{\ddot{\gamma}}}
\newcommand{\kappaddot}[0]{{\ddot{\kappa}}}
\newcommand{\lambdaddot}[0]{{\ddot{\lambda}}}
\newcommand{\muddot}[0]{{\ddot{\mu}}}
\newcommand{\nuddot}[0]{{\ddot{\nu}}}
\newcommand{\omegaddot}[0]{{\ddot{\omega}}}
\newcommand{\phiddot}[0]{{\ddot{\phi}}}
\newcommand{\piddot}[0]{{\ddot{\pi}}}
\newcommand{\psiddot}[0]{{\ddot{\psi}}}
\newcommand{\rhoddot}[0]{{\ddot{\rho}}}
\newcommand{\sigmaddot}[0]{{\ddot{\sigma}}}
\newcommand{\tauddot}[0]{{\ddot{\tau}}}
\newcommand{\thetaddot}[0]{{\ddot{\theta}}}
\newcommand{\upsilonddot}[0]{{\ddot{\upsilon}}}
\newcommand{\varepsilonddot}[0]{{\ddot{\varepsilon}}}
\newcommand{\varphiddot}[0]{{\ddot{\varphi}}}
\newcommand{\varpiddot}[0]{{\ddot{\varpi}}}
\newcommand{\varrhoddot}[0]{{\ddot{\varrho}}}
\newcommand{\varsigmaddot}[0]{{\ddot{\varsigma}}}
\newcommand{\varthetaddot}[0]{{\ddot{\vartheta}}}
\newcommand{\xiddot}[0]{{\ddot{\xi}}}
\newcommand{\zetaddot}[0]{{\ddot{\zeta}}}

\newcommand{\BDelta}[0]{\boldsymbol{\Delta}}
\newcommand{\BGamma}[0]{\boldsymbol{\Gamma}}
\newcommand{\BLambda}[0]{\boldsymbol{\Lambda}}
\newcommand{\BOmega}[0]{\boldsymbol{\Omega}}
\newcommand{\BPhi}[0]{\boldsymbol{\Phi}}
\newcommand{\BPi}[0]{\boldsymbol{\Pi}}
\newcommand{\BPsi}[0]{\boldsymbol{\Psi}}
\newcommand{\BSigma}[0]{\boldsymbol{\Sigma}}
\newcommand{\BTheta}[0]{\boldsymbol{\Theta}}
\newcommand{\BUpsilon}[0]{\boldsymbol{\Upsilon}}
\newcommand{\BXi}[0]{\boldsymbol{\Xi}}
\newcommand{\Balpha}[0]{\boldsymbol{\alpha}}
\newcommand{\Bbeta}[0]{\boldsymbol{\beta}}
\newcommand{\Bchi}[0]{\boldsymbol{\chi}}
\newcommand{\Bdelta}[0]{\boldsymbol{\delta}}
\newcommand{\Bepsilon}[0]{\boldsymbol{\epsilon}}
\newcommand{\Beta}[0]{\boldsymbol{\eta}}
\newcommand{\Bgamma}[0]{\boldsymbol{\gamma}}
\newcommand{\Bkappa}[0]{\boldsymbol{\kappa}}
\newcommand{\Blambda}[0]{\boldsymbol{\lambda}}
\newcommand{\Bmu}[0]{\boldsymbol{\mu}}
\newcommand{\Bnu}[0]{\boldsymbol{\nu}}
%\newcommand{\Bomega}[0]{\boldsymbol{\omega}}
\newcommand{\Bphi}[0]{\boldsymbol{\phi}}
\newcommand{\Bpi}[0]{\boldsymbol{\pi}}
\newcommand{\Bpsi}[0]{\boldsymbol{\psi}}
\newcommand{\Brho}[0]{\boldsymbol{\rho}}
\newcommand{\Bsigma}[0]{\boldsymbol{\sigma}}
%\newcommand{\Btau}[0]{\boldsymbol{\tau}}
%\newcommand{\Btheta}[0]{\boldsymbol{\theta}}
\newcommand{\Bupsilon}[0]{\boldsymbol{\upsilon}}
\newcommand{\Bvarepsilon}[0]{\boldsymbol{\varepsilon}}
\newcommand{\Bvarphi}[0]{\boldsymbol{\varphi}}
\newcommand{\Bvarpi}[0]{\boldsymbol{\varpi}}
\newcommand{\Bvarrho}[0]{\boldsymbol{\varrho}}
\newcommand{\Bvarsigma}[0]{\boldsymbol{\varsigma}}
\newcommand{\Bvartheta}[0]{\boldsymbol{\vartheta}}
\newcommand{\Bxi}[0]{\boldsymbol{\xi}}
\newcommand{\Bzeta}[0]{\boldsymbol{\zeta}}
%
%</bold and dot greek symbols>
%<infrequent>
%
%\newcommand{\AreaOp}[1]{\AName_{#1}}
%\newcommand{\Babs}[0]{\abs{\BB}}
%\newcommand{\Bcap}[0]{\hat{\BB}}
%\newcommand{\BrPrimeRej}[0]{\rcap(\rcap \wedge \Br')}
%\newcommand{\CA}[0]{\mathcal{A}}
%\newcommand{\Cos}[1]{\cos{\left({#1}\right)}}
%\newcommand{\Det}[1] {\abs{#1}}
%\newcommand{\Dsq}[2] {\frac {\partial^2 {#1}} {\partial {#2}^2}}
%\newcommand{\Exp}[1]{\exp{\left({#1}\right)}}
%\newcommand{\Norm}[1]{\left\lVert{#1}\right\rVert}
%\newcommand{\Sin}[1]{\sin{\left({#1}\right)}}
%\newcommand{\T}[0]{\text{T}}
%\newcommand{\VolumeOp}[1]{\VName_{#1}}
%\newcommand{\agrad}[0]{\Ba \cdot \nabla}
%\newcommand{\alphacap}[0]{\hat{\boldsymbol{\alpha}}}
%\newcommand{\Fcap}[0]{\hat{\BF}}
%\newcommand{\bithree}[0]{{\Bi}_3}
%\newcommand{\bxa}[0]{\Bx\Ba}
%\newcommand{\coordvec}[2]{
%\newcommand{\costheta}[0]{\acap \cdot \xcap}
%\newcommand{\ddt}[1]{\ddot{#1}}
%\newcommand{\ddu}[1] {\frac {d{#1}} {du}}
%\newcommand{\dsqxj}[2] {\frac {\partial^2 {#1}} {\partial {x_{#2}}^2}}
%\newcommand{\dtheta}[1]{\frac{d {#1}}{d \theta}}
%\newcommand{\dt}[1]{\dot{#1}}
%\newcommand{\dt}[1]{\frac{d {#1}}{dt}}
%\newcommand{\dxj}[2] {\frac {\partial {#1}} {\partial {x_{#2}}}}
%\newcommand{\halfPhi}[0]{\frac{\phi}{2}}
%\newcommand{\half}[0]{\inv{2}}
%\newcommand{\inv}[1]{\frac{1}{#1}}
%\newcommand{\laplacian}[0]{\nabla^2}
%\newcommand{\matrixoftx}[3]{
%\newcommand{\nrrp}[0]{\norm{\rcap \wedge \Br'}}
%\newcommand{\oiint}{\bigcirc \hspace{-1.4em} \int \hspace{-.8em} \int}
%\newcommand{\transpose}[1]{{#1}^{\text{T}}}
%\newcommand{\transpose}[1]{{{#1}^{\TextTranspose}}}
%\newcommand{\transpose}[1]{{{#1}^{\text{T}}}}
%\newcommand{\barA}[0]{\bar{A}}
%\newcommand{\qbar}[0]{\bar{q}}
%\newcommand{\qdotbar}[0]{\dot{\bar{q}}}
%
%</infrequent>





\newcommand{\symmetric}[2]{{\left\{{#1},{#2}\right\}}}
\newcommand{\antisymmetric}[2]{\left[{#1},{#2}\right]}
\DeclareMathOperator{\sgn}{sgn}
\DeclareMathOperator{\something}{something}

\newcommand{\uDETuvij}[4]{
\begin{vmatrix}
 {#1}^{#3} & {#1}^{#4} \\
 {#2}^{#3} & {#2}^{#4}
\end{vmatrix}
}

\newcommand{\PDSq}[2]{\frac{\partial^2 {#2}}{\partial {#1}^2}}
\newcommand{\transpose}[1]{{#1}^{\mathrm{T}}}
\newcommand{\stardot}[0]{{*}}

% bivector.tex:
\newcommand{\laplacian}[0]{\nabla^2}
\newcommand{\Dsq}[2] {\frac {\partial^2 {#1}} {\partial {#2}^2}}
\newcommand{\dxj}[2] {\frac {\partial {#1}} {\partial {x_{#2}}}}
\newcommand{\dsqxj}[2] {\frac {\partial^2 {#1}} {\partial {x_{#2}}^2}}
\DeclareMathOperator{\ExpName}{e}
%\DeclareMathOperator{\Exp}{e}
%\newcommand{\Exp}[1]{\exp{\left({#1}\right)}}
%\DeclareMathOperator{\Rej}{Rej}
\DeclareMathOperator{\Rot}{R}
%\newcommand{\gpgrade}[2] {{\left\langle{{#1}}\right\rangle}_{#2}}
%\newcommand{\gpgradezero}[1] {\gpgrade{#1}{0}}
%\newcommand{\gpgradetwo}[1] {\gpgrade{#1}{2}}
%\newcommand{\gpgradefour}[1] {\gpgrade{#1}{4}}

% ga_wiki_torque.tex:
\newcommand{\Fcap}[0]{\hat{\BF}}
\newcommand{\bithree}[0]{{\Bi}_3}
\newcommand{\nrrp}[0]{\norm{\rcap \wedge \Br'}}
\newcommand{\dtheta}[1]{\frac{d {#1}}{d \theta}}

% ga_wiki_unit_derivative.tex
\newcommand{\dt}[1]{\frac{d {#1}}{dt}}
\newcommand{\BrPrimeRej}[0]{\rcap(\rcap \wedge \Br')}

\makeindex

\title{Applied Geometric Algebra}

\author{Peeter Joot  \quad peeter.joot@gmail.com \\
{\small\em \copyright \  Draft date \today }}

\date{ May 31, 2009.  Last Revision: $Date: 2009/05/31 16:37:23 $ }
\begin{document}
\maketitle
 \addcontentsline{toc}{chapter}{Contents}
\pagenumbering{roman}
\tableofcontents
\listoffigures
\listoftables
\chapter*{Preface}\normalsize
  \addcontentsline{toc}{chapter}{Preface}
\pagestyle{plain}

Start compiling my hodge podge set of Geometric Algebra notes into a coherent unit.  Want a treatment that is more
understandable than that of 
\cite{doran2003gap} or 
\cite{hestenes1999nfc}.
This may be hard since so many of these notes are disconnected, and as
is do not necessarily build in a logical sequence.  Many of these 
notes were an attempt to provide motivation for things presented in
an unmotivated fashion in other locations, but ironically this probably
means that much of the content here will be unmotivated since this 
is the gaps but not the meat of other treatments.

%Each
%Chapter, Appendix and the Index is made as a {\tt *.tex} file and is
%called in by the {\tt include} command---thus {\tt ch1.tex} is
%the name here of the file containing Chapter~1. The inclusion of any
%particular file can be suppressed by prefixing the line by a
%percent sign.
%
%Note the tag used to make an index entry. You may need to consult Lamport's
%book~\cite{lamport} for details of the procedure to make the index input
%file; \LaTeX \ will create a pre-index by listing all the tagged
%items in the file {\tt bookex.idx} then you edit this into
%a {\tt theindex} environment, as {\tt index.tex}.

\pagestyle{headings}
\pagenumbering{arabic}

\include{intro_ga}
\include{ga_wiki}
\include{ga_wiki_cramers}
\include{ga_wiki_torque}
\include{ga_wiki_unit_derivative}
\documentclass{article}      % Specifies the document class

\usepackage{amsmath}
\usepackage{mathpazo}

%
% shorthand for bold symbols, convenient for vectors and matrices
%
\newcommand{\Ba}[0]{\mathbf{a}}
\newcommand{\Bb}[0]{\mathbf{b}}
\newcommand{\Bc}[0]{\mathbf{c}}
\newcommand{\Bd}[0]{\mathbf{d}}
\newcommand{\Be}[0]{\mathbf{e}}
\newcommand{\Bf}[0]{\mathbf{f}}
\newcommand{\Bg}[0]{\mathbf{g}}
\newcommand{\Bh}[0]{\mathbf{h}}
\newcommand{\Bi}[0]{\mathbf{i}}
\newcommand{\Bj}[0]{\mathbf{j}}
\newcommand{\Bk}[0]{\mathbf{k}}
\newcommand{\Bl}[0]{\mathbf{l}}
\newcommand{\Bm}[0]{\mathbf{m}}
\newcommand{\Bn}[0]{\mathbf{n}}
\newcommand{\Bo}[0]{\mathbf{o}}
\newcommand{\Bp}[0]{\mathbf{p}}
\newcommand{\Bq}[0]{\mathbf{q}}
\newcommand{\Br}[0]{\mathbf{r}}
\newcommand{\Bs}[0]{\mathbf{s}}
\newcommand{\Bt}[0]{\mathbf{t}}
\newcommand{\Bu}[0]{\mathbf{u}}
\newcommand{\Bv}[0]{\mathbf{v}}
\newcommand{\Bw}[0]{\mathbf{w}}
\newcommand{\Bx}[0]{\mathbf{x}}
\newcommand{\By}[0]{\mathbf{y}}
\newcommand{\Bz}[0]{\mathbf{z}}
\newcommand{\BA}[0]{\mathbf{A}}
\newcommand{\BB}[0]{\mathbf{B}}
\newcommand{\BC}[0]{\mathbf{C}}
\newcommand{\BD}[0]{\mathbf{D}}
\newcommand{\BE}[0]{\mathbf{E}}
\newcommand{\BF}[0]{\mathbf{F}}
\newcommand{\BG}[0]{\mathbf{G}}
\newcommand{\BH}[0]{\mathbf{H}}
\newcommand{\BI}[0]{\mathbf{I}}
\newcommand{\BJ}[0]{\mathbf{J}}
\newcommand{\BK}[0]{\mathbf{K}}
\newcommand{\BL}[0]{\mathbf{L}}
\newcommand{\BM}[0]{\mathbf{M}}
\newcommand{\BN}[0]{\mathbf{N}}
\newcommand{\BO}[0]{\mathbf{O}}
\newcommand{\BP}[0]{\mathbf{P}}
\newcommand{\BQ}[0]{\mathbf{Q}}
\newcommand{\BR}[0]{\mathbf{R}}
\newcommand{\BS}[0]{\mathbf{S}}
\newcommand{\BT}[0]{\mathbf{T}}
\newcommand{\BU}[0]{\mathbf{U}}
\newcommand{\BV}[0]{\mathbf{V}}
\newcommand{\BW}[0]{\mathbf{W}}
\newcommand{\BX}[0]{\mathbf{X}}
\newcommand{\BY}[0]{\mathbf{Y}}
\newcommand{\BZ}[0]{\mathbf{Z}}

\newcommand{\Bzero}[0]{\mathbf{0}}
\newcommand{\Btheta}[0]{\boldsymbol{\theta}}
\newcommand{\Btau}[0]{\boldsymbol{\tau}}
\newcommand{\Bomega}[0]{\boldsymbol{\omega}}

%
% shorthand for unit vectors
%
\newcommand{\acap}[0]{\hat{\Ba}}
\newcommand{\bcap}[0]{\hat{\Bb}}
\newcommand{\ccap}[0]{\hat{\Bc}}
\newcommand{\dcap}[0]{\hat{\Bd}}
\newcommand{\ecap}[0]{\hat{\Be}}
\newcommand{\fcap}[0]{\hat{\Bf}}
\newcommand{\gcap}[0]{\hat{\Bg}}
\newcommand{\hcap}[0]{\hat{\Bh}}
\newcommand{\icap}[0]{\hat{\Bi}}
\newcommand{\jcap}[0]{\hat{\Bj}}
\newcommand{\kcap}[0]{\hat{\Bk}}
\newcommand{\lcap}[0]{\hat{\Bl}}
\newcommand{\mcap}[0]{\hat{\Bm}}
\newcommand{\ncap}[0]{\hat{\Bn}}
\newcommand{\ocap}[0]{\hat{\Bo}}
\newcommand{\pcap}[0]{\hat{\Bp}}
\newcommand{\qcap}[0]{\hat{\Bq}}
\newcommand{\rcap}[0]{\hat{\Br}}
\newcommand{\scap}[0]{\hat{\Bs}}
\newcommand{\tcap}[0]{\hat{\Bt}}
\newcommand{\ucap}[0]{\hat{\Bu}}
\newcommand{\vcap}[0]{\hat{\Bv}}
\newcommand{\wcap}[0]{\hat{\Bw}}
\newcommand{\xcap}[0]{\hat{\Bx}}
\newcommand{\ycap}[0]{\hat{\By}}
\newcommand{\zcap}[0]{\hat{\Bz}}
\newcommand{\thetacap}[0]{\hat{\Btheta}}

%
% to write R^n and C^n in a distinguishable fashion.  Perhaps change this
% to the double lined characters upon figuring out how to do so.
%
\newcommand{\C}[1]{$\mathbb{C}^{#1}$}
\newcommand{\R}[1]{$\mathbb{R}^{#1}$}

%
% various generally useful helpers
%

% derivative of #1 wrt. #2:
\newcommand{\D}[2] {\frac {d#2} {d#1}}

\newcommand{\inv}[1]{\frac{1}{#1}}
\newcommand{\cross}[0]{\times}

\newcommand{\abs}[1]{\lvert{#1}\rvert}
\newcommand{\norm}[1]{\lVert{#1}\rVert}
\newcommand{\innerprod}[2]{\langle{#1}, {#2}\rangle}
\newcommand{\dotprod}[2]{{#1} \cdot {#2}}
\newcommand{\bdotprod}[2]{\left({#1} \cdot {#2}\right)}
\newcommand{\crossprod}[2]{{#1} \cross {#2}}
\newcommand{\tripleprod}[3]{\dotprod{\left(\crossprod{#1}{#2}\right)}{#3}}

\DeclareMathOperator{\Proj}{Proj}
\DeclareMathOperator{\Span}{span}
\DeclareMathOperator{\Sgn}{sgn}
\DeclareMathOperator{\Area}{Area}
\DeclareMathOperator{\Volume}{Volume}

%
% A few miscellaneous things specific to this document
%
\newcommand{\crossop}[1]{\crossprod{#1}{}}

% R2 vector.
\newcommand{\VectorTwo}[2]{
\begin{bmatrix}
 {#1} \\
 {#2}
\end{bmatrix}
}

\newcommand{\VectorN}[1]{
\begin{bmatrix}
{#1}_1 \\
{#1}_2 \\
\vdots \\
{#1}_N \\
\end{bmatrix}
}

\newcommand{\DETuvij}[4]{
\begin{vmatrix}
 {#1}_{#3} & {#1}_{#4} \\
 {#2}_{#3} & {#2}_{#4}
\end{vmatrix}
}

\newcommand{\DETuvwijk}[6]{
\begin{vmatrix}
 {#1}_{#4} & {#1}_{#5} & {#1}_{#6} \\
 {#2}_{#4} & {#2}_{#5} & {#2}_{#6} \\
 {#3}_{#4} & {#3}_{#5} & {#3}_{#6}
\end{vmatrix}
}

\newcommand{\DETuvwxijkl}[8]{
\begin{vmatrix}
 {#1}_{#5} & {#1}_{#6} & {#1}_{#7} & {#1}_{#8} \\
 {#2}_{#5} & {#2}_{#6} & {#2}_{#7} & {#2}_{#8} \\
 {#3}_{#5} & {#3}_{#6} & {#3}_{#7} & {#3}_{#8} \\
 {#4}_{#5} & {#4}_{#6} & {#4}_{#7} & {#4}_{#8} \\
\end{vmatrix}
}

%\newcommand{\DETuvwxyijklm}[10]{
%\begin{vmatrix}
% {#1}_{#6} & {#1}_{#7} & {#1}_{#8} & {#1}_{#9} & {#1}_{#10} \\
% {#2}_{#6} & {#2}_{#7} & {#2}_{#8} & {#2}_{#9} & {#2}_{#10} \\
% {#3}_{#6} & {#3}_{#7} & {#3}_{#8} & {#3}_{#9} & {#3}_{#10} \\
% {#4}_{#6} & {#4}_{#7} & {#4}_{#8} & {#4}_{#9} & {#4}_{#10} \\
% {#5}_{#6} & {#5}_{#7} & {#5}_{#8} & {#5}_{#9} & {#5}_{#10}
%\end{vmatrix}
%}

% R3 vector.
\newcommand{\VectorThree}[3]{
\begin{bmatrix}
 {#1} \\
 {#2} \\
 {#3}
\end{bmatrix}
}



\newcommand{\laplacian}[0]{\nabla^2}
\newcommand{\Dsq}[2] {\frac {\partial^2 {#1}} {\partial {#2}^2}}
\newcommand{\dxj}[2] {\frac {\partial {#1}} {\partial {x_{#2}}}}
\newcommand{\dsqxj}[2] {\frac {\partial^2 {#1}} {\partial {x_{#2}}^2}}
\DeclareMathOperator{\Exp}{e}
\DeclareMathOperator{\Rej}{Rej}
\newcommand{\gpgrade}[2] {{\left\langle{{#1}}\right\rangle}_{#2}}
\newcommand{\gpgradezero}[1] {\gpgrade{#1}{0}}
\newcommand{\gpgradetwo}[1] {\gpgrade{#1}{2}}
\newcommand{\gpgradefour}[1] {\gpgrade{#1}{4}}

%
% The real thing:
%

                             % The preamble begins here.
\title{Geometry of intersecting bivectors}
\author{Peeter Joot}         % Declares the author's name.
%\date{}        % Deleting this command produces today's date.

\begin{document}             % End of preamble and beginning of text.

\maketitle{}

\section{ The problem. }

Examination of exponential solutions for Laplace's equation leads one to
a requirement to examine the product of intersecting bivectors such as

\[
\left(\abs{\Bx \wedge \Bk}^2\right)' = -\left(
(\Bx' \wedge \Bv)(\Bx \wedge \Bv)
+(\Bx \wedge \Bv)(\Bx' \wedge \Bv)
\right)
\]

Here we see that the symmetric sum of bivectors $\Bx \wedge \Bk$ and $\Bx' \wedge \Bk$ is a scalar quantity.  This we will identify later as a quantity
related to the bivector dot product.

It is worthwhile to systematically examine the
general products of intersecting bivectors, that is planes that share a common line, in this case the line directed along the vector $\Bk$.
It is also notable that since all non coplanar bivectors in \R{3} intersect
this
examination will cover the important special case of three dimensional
plane geometry.

A result of this examination is that many of the concepts familiar from
vector geometry such as
orthagonality, projection, and rejection will have direct bivector
equivalents.

General bivector geometry, in spaces where non-coplanar bivectors do not 
neccessarily intersect (such as in \R{4}), will need to be treated separately,
but some of the grade 4 product terms will be carried below to explicitly
hightlight the point where the intersecting bivector space requirement
effects the results.

\section{The meat.}

The geometric product of two bivectors can be written:

\begin{equation}\label{eqn:ABprod}
\BA \BB = 
\gpgrade{\BA \BB}{0}
+\gpgrade{\BA \BB}{2}
+\gpgrade{\BA \BB}{4}
= 
{\BA \cdot \BB}
+\gpgrade{\BA \BB}{2}
+{\BA \wedge \BB}
\end{equation}
\begin{equation}\label{eqn:BAprod}
\BB \BA = 
\gpgrade{\BB \BA}{0}
+\gpgrade{\BB \BA}{2}
+\gpgrade{\BB \BA}{4}
= 
{\BB \cdot \BA}
+\gpgrade{\BB \BA}{2}
+{\BB \wedge \BA}
\end{equation}

Because we have three terms involved, unlike the vector dot and wedge product
we cannot generally separate these terms by 
symmetric and antisymmetric parts.  However forming those sums
will still worthwhile, especially for the case of interecting bivectors
since the last term will be zero in that case.

\subsection{ Sign change of each grade term with commutation. }

Starting with the last term we can first observe that

\begin{equation}\label{eqn:wedgesign}
\BA \wedge \BB = \BB \wedge \BA
\end{equation}

To show this let $\BA = \Ba \wedge \Bb$, and $\BB = \Bc \wedge \Bd$.  When

$\BA \wedge \BB \ne 0$, one can write:

\begin{align*}
\BA \wedge \BB 
&= \Ba \wedge \Bb \wedge \Bc \wedge \Bd \\
&= - \Bb \wedge \Bc \wedge \Bd \wedge \Ba \\
&= \Bc \wedge \Bd \wedge \Ba \wedge \Bb \\
&= \BB \wedge \BA \\
\end{align*}

To see how the signs of the remaining two terms vary with commutation
form:

\begin{align*}
(\BA + \BB)^2
&= (\BA + \BB)(\BA + \BB) \\
&= \BA^2 + \BB^2 + \BA \BB + \BB \BA \\
\end{align*}

When $\BA$ and $\BB$ interect we can write
$\BA = \Ba \wedge \Bx$, and $\BB = \Bb \wedge \Bx$, thus the sum is a bivector

\[
(\BA + \BB)
= (\Ba + \Bb) \wedge \Bx
\]

And so, the square of the two is a scalar.  When $\BA$ and $\BB$ have only
non intersecting components, such as the grade two \R{4} multivector
$\Be_{12} + \Be_{34}$, the square of this sum will have both grade four and
scalar parts.

Since the LHS = RHS, and the grades of the two also must be the same.
This implies that the quantity

\[
\BA \BB + \BB \BA = 
\BA \cdot \BB + \BB \cdot \BA
+\gpgradetwo{\BA \BB} + \gpgradetwo{\BB \BA}
+\BA \wedge \BB + \BB \wedge \BA
\]

is a scalar $\iff$ 
$\BA + \BB$ is a bivector, and in general has scalar and grade four terms.
Because this symmetric sum has no grade two terms, 
regardless of whether $\BA$, and $\BB$ intersect, we have:

\[
\gpgradetwo{\BA \BB} + \gpgradetwo{\BB \BA} = 0
\]
\begin{equation}\label{eqn:signgradetwo}
\implies
\gpgradetwo{\BA \BB} = -\gpgradetwo{\BB \BA}
\end{equation}

One would intuitively expect $\BA \cdot \BB = \BB \cdot \BA$.  This can be
demonstrated by forming the complete symmetric sum

\begin{align*}
\BA \BB + \BB \BA 
&= 
{\BA \cdot \BB} +{\BB \cdot \BA}
+\gpgrade{\BA \BB}{2} +\gpgrade{\BB \BA}{2}
+{\BA \wedge \BB} + {\BB \wedge \BA} \\
&= 
{\BA \cdot \BB} +{\BB \cdot \BA}
+\gpgrade{\BA \BB}{2} -\gpgrade{\BA \BB}{2}
+{\BA \wedge \BB} + {\BA \wedge \BB} \\
&= 
{\BA \cdot \BB} +{\BB \cdot \BA}
+2{\BA \wedge \BB} \\
\end{align*}

The LHS commutes with interchange of $\BA$ and $\BB$, as does
${\BA \wedge \BB}$.  So for the RHS to also commute, the remaining grade 0 term
must also:

\begin{equation}\label{eqn:dotsign}
\BA \cdot \BB = \BB \cdot \BA
\end{equation}

\subsection{ Dot, wedge and grade two terms of bivector product. }

Collecting the results of the previous section and substituiting back
into equation \ref{eqn:ABprod} we have:

\begin{equation}\label{eqn:AdotB}
\BA \cdot \BB = \gpgrade{\frac{\BA \BB + \BB\BA}{2}}{0}
\end{equation}

\begin{equation}\label{eqn:AtwoB}
\gpgradetwo{\BA \BB} = \frac{\BA \BB - \BB\BA}{2}
\end{equation}

\begin{equation}\label{eqn:AwedgeB}
\BA \wedge \BB = \gpgrade{\frac{\BA \BB + \BB\BA}{2}}{4}
\end{equation}

When these intersect in a line the wedge term is zero, so for that special case we can write:

\begin{equation*}
\BA \cdot \BB = \frac{\BA \BB + \BB\BA}{2}
\end{equation*}

\begin{equation*}
\gpgradetwo{\BA \BB} = \frac{\BA \BB - \BB\BA}{2}
\end{equation*}

\begin{equation*}
\BA \wedge \BB = 0
\end{equation*}

(note that this is always the case for \R{3}).

\section{ Intersection of planes. }

Starting with two planes specified parametrically, each in terms of two direction vectors and a point on the plane:

\begin{align}\label{eqn:twoplanes}
\Bx &= \Bp + \alpha \Bu + \beta \Bv \\
\By &= \Bq + a \Bw + b \Bz \\
\end{align}

If these intersect then all points on the line must satisify $\Bx = \By$, so the
solution requires:

\[
\Bp + \alpha \Bu + \beta \Bv = \Bq + a \Bw + b \Bz
\]
\[
\implies
(\Bp + \alpha \Bu + \beta \Bv) \wedge \Bw \wedge \Bz = (\Bq + a \Bw + b \Bz) \wedge \Bw \wedge \Bz = \Bq \wedge \Bw \wedge \Bz
\]

Rearranging for $\beta$, and writing $\BB = \Bw \wedge \Bz$:

\[
\beta = \frac{\Bq \wedge \BB - (\Bp + \alpha \Bu) \wedge \BB}{\Bv \wedge \BB}
\]

Note that when the solution exists the left vs right order of the division by $\Bv \wedge \BB$ should not matter since the numerator will be proportional to this bivector (or else the $\beta$ would not be a scalar).

Substitution of $\beta$ back into $\Bx = \Bp + \alpha \Bu + \beta \Bv$ (all points in the first plane) gives you a parametric equation for a line:

\[
\Bx = \Bp + \frac{(\Bq-\Bp)\wedge \BB}{\Bv \wedge \BB}\Bv + \alpha\frac{1}{\Bv \wedge \BB}((\Bv \wedge \BB) \Bu - (\Bu \wedge \BB)\Bv)
\]

Where a point on the line is:

\[
\Bp + \frac{(\Bq-\Bp)\wedge \BB}{\Bv \wedge \BB}\Bv 
%= \frac{1}{\Bv \wedge \BB}((\Bv \wedge \BB)\Bp + ((\Bq-\Bp)\wedge \BB)\Bv)
\]

And a direction vector for the line is:

\[
\frac{1}{\Bv \wedge \BB}((\Bv \wedge \BB) \Bu - (\Bu \wedge \BB)\Bv)
\]
\[
\propto
(\Bv \wedge \BB)^2 \Bu - (\Bv \wedge \BB)(\Bu \wedge \BB)\Bv
\]

Now, this result is only valid if $\Bv \wedge \BB \ne 0$ (ie: line of intersection is not directed along $\Bv$), but if that is the case the second form will be zero.  Thus we can add the results (or any non-zero linear combination of) allowing for either of $\Bu$, or $\Bv$ to be directed along the line of intersection:

\begin{equation}\label{eqn:dirvecintersection}
a\left( (\Bv \wedge \BB)^2 \Bu
- (\Bv \wedge \BB)(\Bu \wedge \BB)\Bv \right)
+ b\left((\Bu \wedge \BB)^2 \Bv 
- (\Bu \wedge \BB)(\Bv \wedge \BB)\Bu\right)
\end{equation}

Alternately, one could formulate this in terms of $\BA = \Bu \wedge \Bv$, $\Bw$, and $\Bz$.  Is there a more symetrical form for this direction vector?

\subsection{ Vector along line of intersection in \R{3}}

For \R{3} one can solve the intersection problem using the normals to the planes.  For simplicity put the origin on the line of intersection (and all planes through a common point in \R{3} have at least a line of intersection).  In this case, for bivectors $\BA$ and $\BB$, normals to those planes are $i\BA$, and $i\BB$ respectively.  The plane through both of those normals is:

\begin{align*}
(i\BA) \wedge (i\BB)
= \frac{(i\BA)(i\BB) - (i\BB)(i\BA)}{2} 
= \frac{\BB\BA - \BA\BB}{2} 
= \gpgradetwo{\BB\BA}
\end{align*}

The normal to this plane

\begin{equation}\label{eqn:r3planeintersect}
i\gpgradetwo{\BB\BA}
\end{equation}

is directed along the line of interesection.  This result is more appealing than
the general \R{N} result of equation \ref{eqn:dirvecintersection}, not
just because it is simpler, but also because it is a function of only the
bivectors for the planes, without a requirement to find or calculate
two specific independent direction vectors in one of the planes.

\subsection{ Applying this result to \R{N} }

If you reject the component of $\BA$ from $\BB$ for two intersecting bivectors:

\[
\Rej_{\BA}(\BB) = \frac{1}{\BA}\gpgradetwo{\BA\BB}
\]

the line of intersection remains the same ... that operation rotates $\BB$ so that the two are mutually perpendicular.  This essentially reduces the problem to that of the three dimensional case, so the solution has to be of the same form... you just need to calculate a ``pseudoscalar'' (what you are calling the join), for the subspace spanned by the two bivectors.

That can be computed by taking any direction vector that is on one plane, but isn't in the second.  For example, pick a vector $\Bu$ in the plane $\BA$ that is not on the intersection of $\BA$ and $\BB$.  In mathese that is $\Bu = \inv{\BA}(\BA\cdot \Bu)$ (or $\Bu \wedge \BA = 0$), where $\Bu \wedge \BB \ne 0$.  Thus a pseudoscalar for this subspace is:

\[
\Bi = \frac{\Bu \wedge \BB}{\abs{\Bu \wedge \BB}}
\]

To calculate the direction vector along the intersection we don't care about the scaling above.  Also note that provided $\Bu$ has a component in the plane $\BA$, $\Bu \cdot \BA$ is also in the plane (it's rotated $\pi/2$ from $\inv{\BA}(\BA \cdot \Bu)$.

Thus, provided that $\Bu \cdot \BA$ isn't on the intersection, a scaled ``pseudoscalar''
for the subspace can be calculated by taking from any vector $\Bu$ with a component in the plane $\BA$:

\[
\Bi \propto (\Bu \cdot \BA) \wedge \BB
\]

Thus a vector along the intersection is:

\begin{equation}\label{eqn:pseudoscalarinter}
\Bd = ((\Bu \cdot \BA) \wedge \BB) \gpgradetwo{\BA\BB}
\end{equation}

(an interchange of $\BA$ and $\BB$ above would also work).

Without showing the steps one can write the complete parametric solution of the line through the planes of equation \ref{eqn:twoplanes} in terms of this direction vector:

\begin{equation}\label{eqn:finalsolnofRNplaneintersection}
\Bx = \Bp + \left(\frac{(\Bq - \Bp)\wedge \BB}{(\Bd \cdot \BA) \wedge \BB}\right) (\Bd \cdot \BA) + \alpha \Bd
\end{equation}

Since $(\Bd \cdot \BA) \ne 0$ and $(\Bd \cdot \BA) \wedge \BB \ne 0$ (unless $\BA$ and $\BB$ are coplanar), observe that this is a natural generator
of the pseudoscalar for the subspace, and as such shows up in the expression
above.

\section{ Grade components of a trivector product. }

While trying to put equation \ref{eqn:dirvecintersection} into a form
that eliminated $\Bu$, and $\Bv$ in favour of $\BA = \Bu \wedge \Bv$
symmetric and antisymmetric formulations for the various grade terms
of a trivector product looked like they could be handy.  Here's a summary
of those results.

\subsection{ Grade 6 term. }

Writing two trivectors in terms
of mutually orthogonal components

\[
\BA = \Bx \wedge \By \wedge \Bz = \Bx\By\Bz
\]

and

\[
\BB = \Bu \wedge \Bv \wedge \Bw =\Bu\Bv\Bw
\]

Assuming that there is no common vector between the two, the 
wedge of these is

\begin{align*}
\BA \wedge \BB 
&= \gpgrade{\BA\BB}{6} \\
&= \gpgrade{\Bx\By\Bz\Bu\Bv\Bw}{6} \\
&= \gpgrade{\By\Bz(\Bx\Bu)\Bv\Bw}{6} \\
&= \gpgrade{\By\Bz(-\Bu\Bx + 2\Bu \cdot \Bx)\Bv\Bw}{6} \\
&= -\gpgrade{\By\Bz\Bu(\Bx\Bv)\Bw}{6} \\
&= -\gpgrade{\By\Bz\Bu(-\Bv\Bx + 2\Bv \cdot \Bx)\Bw}{6} \\
&= \gpgrade{\By\Bz\Bu\Bv(\Bx\Bw)}{6} \\
&= \cdots \\
&= -\gpgrade{\Bu\Bv\Bw\Bx\By\Bz}{6} \\
&= -\gpgrade{\BB\BA}{6} \\
&= -\BB \wedge \BA
\end{align*}

Note above that any interchange of terms inverts the sign (demonstrated 
explicitly for all the $\Bx$ interchanges).

As an aside, this
sign change on interchange is taken as the defining property of the 
wedge product in differential forms.  That property also
implies also that the wedge product is
zero when a vector is wedged with itself since zero is the only
value that is the negation of itself.  Thus we see explicitly
how the notation of using the wedge for the highest grade term
of two blades is consistent with the traditional
wedge product definition.

The end result here is that the grade 6 term of a trivector trivector product
changes sign on interchange of the trivectors:

\begin{equation}\label{eqn:trivecgpgrade6}
\gpgrade{\BA\BB}{6} = -\gpgrade{\BB\BA}{6}
\end{equation}

\subsection{ Grade 4 term. }

For a trivector product to have a grade 4 term there must be a common
vector between the two

\[
\BA = \Bx \wedge \By \wedge \Bz = \Bx\By\Bz
\]

and

\[
\BB = \Bu \wedge \Bv \wedge \Bz =\Bu\Bv\Bz
\]

The grade four term of the product is

\begin{align*}
\gpgrade{\BB \BA}{4}
&= \gpgrade{ \Bu\Bv\Bz \Bx\By\Bz }{4} \\
&= \gpgrade{ \Bu\Bv\Bz \Bz\Bx\By }{4} \\
&= \Bz^2\gpgrade{ \Bu\Bv\Bx\By }{4} \\
&= \Bz^2\gpgrade{ \Bu(\Bv\Bx)\By }{4} \\
&= \Bz^2\gpgrade{ \Bu(-\Bx\Bv + 2 \Bx \cdot \Bv)\By }{4} \\
&= -\Bz^2\gpgrade{ \Bu\Bx\Bv\By }{4} \\
&= \cdots \\
&= \Bz^2\gpgrade{ \Bx\By\Bu\Bv }{4} \\
&= \gpgrade{ \Bx\By\Bz\Bz\Bu\Bv }{4} \\
&= \gpgrade{ \Bx\By\Bz\Bu\Bv\Bz }{4} \\
&= \gpgrade{ \Bx\By\Bz\Bu\Bv\Bz }{4} \\
&= \gpgrade{\BA \BB}{4}
\end{align*}

Thus the grade 4 term commutes on interchange:

\begin{equation}\label{eqn:trivecgpgrade4}
\gpgrade{\BA\BB}{4} = \gpgrade{\BB\BA}{4}
\end{equation}

\subsection{ Grade 2 term. }

Similar to above, 
for a trivector product to have a grade 2 term there must be two common
vectors between the two

\[
\BA = \Bx \wedge \By \wedge \Bz = \Bx\By\Bz
\]

and

\[
\BB = \Bu \wedge \By \wedge \Bz =\Bu\By\Bz
\]

The grade two term of the product is

\begin{align*}
\gpgrade{\BA \BB}{2}
&= \gpgrade{ \Bx\By\Bz \Bu\By\Bz }{2} \\
&= \gpgrade{ \Bx\By\Bz \By\Bz \Bu}{2} \\
&= (\By\Bz)^2\gpgrade{ \Bx \Bu}{2} \\
&= -(\By\Bz)^2\gpgrade{ \Bu \Bx}{2} \\
&= -\gpgrade{ \BB \BA }{2} \\
\end{align*}

The grade 2 term anticommutes on interchange:

\begin{equation}\label{eqn:trivecgpgrade2}
\gpgrade{\BA\BB}{2} = -\gpgrade{\BB\BA}{2}
\end{equation}

\subsection{ Grade 0 term. }

Any grade 0 terms are due to products of the form $\BA = k\BB$

\begin{align*}
\gpgrade{\BA \BB}{0}
&= \gpgrade{k\BB \BB}{0} \\
&= \gpgrade{\BB k\BB}{0} \\
&= \gpgrade{\BB \BA}{0} \\
\end{align*}

The grade 2 term commutes on interchange:

\begin{equation}\label{eqn:trivecgpgrade0}
\gpgrade{\BA\BB}{0} = \gpgrade{\BB\BA}{0}
\end{equation}

\subsection{ combining results. }

\begin{equation*}
\BA \BB
=\gpgrade{\BA\BB}{0}
+\gpgrade{\BA\BB}{2}
+\gpgrade{\BA\BB}{4}
+\gpgrade{\BA\BB}{6}
\end{equation*}

\begin{align*}
\BB\BA
&=\gpgrade{\BB\BA}{0}
+\gpgrade{\BB\BA}{2}
+\gpgrade{\BB\BA}{4}
+\gpgrade{\BB\BA}{6} \\
&=\gpgrade{\BA\BB}{0}
-\gpgrade{\BA\BB}{2}
+\gpgrade{\BA\BB}{4}
-\gpgrade{\BA\BB}{6} \\
\end{align*}

These can be combined to express each of the grade terms as subsets
of the symmetric and antisymmetric parts:

\begin{align*}
\BA \cdot \BB = \gpgrade{\BA\BB}{0} &= \gpgrade{\frac{\BA\BB + \BB\BA}{2}}{0} \\
\gpgrade{\BA\BB}{2} &= \gpgrade{\frac{\BA\BB - \BB\BA}{2}}{2} \\
\gpgrade{\BA\BB}{4} &= \gpgrade{\frac{\BA\BB + \BB\BA}{2}}{4} \\
\BA \wedge \BB = \gpgrade{\BA\BB}{6} &= \gpgrade{\frac{\BA\BB - \BB\BA}{2}}{6} \\
\end{align*}

Note that above I've been somewhat loose with the argument above.  A grade three vector
will have the following form:

\[
\sum_{i<j<k} D_{ijk} \Be_{ijk}
\]

Where $D_{ijk}$ is the determinant of $ijk$ components of the vectors being wedged.  Thus the product
of two trivectors will be of the following form:

\[
\sum_{i<j<k} \sum_{i'<j'<k'} D_{ijk} D'_{i'j'k'} (\Be_{ijk} \Be_{i'j'k'})
\]

It's really each of these $\Be_{ijk} \Be_{i'j'k'}$ products that have to be considered in the grade 
and sign arguments above.  The end result will be the same though... one would just have to present
it a bit more carefully for a true proof.

\subsection{ Intersecting trivector cases. }

As with the intersecting bivector case, when there is a line of intersection between the two volumes one can
write:

\begin{align*}
\BA \cdot \BB = \gpgrade{\BA\BB}{0} &= \gpgrade{\frac{\BA\BB + \BB\BA}{2}}{0} \\
\gpgrade{\BA\BB}{2} &= \frac{\BA\BB - \BB\BA}{2} \\
\gpgrade{\BA\BB}{4} &= \gpgrade{\frac{\BA\BB + \BB\BA}{2}}{4} \\
\BA \wedge \BB = \gpgrade{\BA\BB}{6} &= 0 \\
\end{align*}

And if these volumes intersect in a plane a further simplification is possible:
\begin{align*}
\BA \cdot \BB = \gpgrade{\BA\BB}{0} &= \frac{\BA\BB + \BB\BA}{2} \\
\gpgrade{\BA\BB}{2} &= \frac{\BA\BB - \BB\BA}{2} \\
\gpgrade{\BA\BB}{4} &= 0 \\
\BA \wedge \BB = \gpgrade{\BA\BB}{6} &= 0 \\
\end{align*}

\end{document}               % End of document.

%
% Copyright � 2012 Peeter Joot.  All Rights Reserved.
% Licenced as described in the file LICENSE under the root directory of this GIT repository.
%

%
%
\chapter{Trivector geometry}
\index{trivector}
\label{chap:trivector}
%\date{Mar 9, 2008.  trivector.tex}

\section{Motivation}

The direction vector for two intersecting planes can be found to have the
form:

\begin{equation}\label{eqn:trivector:dirvecintersection}
a\left( (\Bv \wedge \BB)^2 \Bu
- (\Bv \wedge \BB)(\Bu \wedge \BB)\Bv \right)
+ b\left((\Bu \wedge \BB)^2 \Bv
- (\Bu \wedge \BB)(\Bv \wedge \BB)\Bu\right)
\end{equation}

While trying to put \eqnref{eqn:trivector:dirvecintersection} into a form
that eliminated \(\Bu\), and \(\Bv\) in favor of \(\BA = \Bu \wedge \Bv\)
symmetric and antisymmetric formulations for the various grade terms
of a trivector product looked like they could be handy.  Here is a summary
of those results.

\section{Grade components of a trivector product}

\subsection{Grade 6 term}

Writing two trivectors in terms
of mutually orthogonal components

\begin{equation}\label{eqn:trivector:26}
\BA = \Bx \wedge \By \wedge \Bz = \Bx\By\Bz
\end{equation}

and

\begin{equation}\label{eqn:trivector:46}
\BB = \Bu \wedge \Bv \wedge \Bw =\Bu\Bv\Bw
\end{equation}

Assuming that there is no common vector between the two, the
wedge of these is

\begin{equation}\label{eqn:trivector:186}
\begin{aligned}
\BA \wedge \BB
&= \gpgrade{\BA\BB}{6} \\
&= \gpgrade{\Bx\By\Bz\Bu\Bv\Bw}{6} \\
&= \gpgrade{\By\Bz(\Bx\Bu)\Bv\Bw}{6} \\
&= \gpgrade{\By\Bz(-\Bu\Bx + 2\Bu \cdot \Bx)\Bv\Bw}{6} \\
&= -\gpgrade{\By\Bz\Bu(\Bx\Bv)\Bw}{6} \\
&= -\gpgrade{\By\Bz\Bu(-\Bv\Bx + 2\Bv \cdot \Bx)\Bw}{6} \\
&= \gpgrade{\By\Bz\Bu\Bv(\Bx\Bw)}{6} \\
&= \cdots \\
&= -\gpgrade{\Bu\Bv\Bw\Bx\By\Bz}{6} \\
&= -\gpgrade{\BB\BA}{6} \\
&= -\BB \wedge \BA
\end{aligned}
\end{equation}

Note above that any interchange of terms inverts the sign (demonstrated
explicitly for all the \(\Bx\) interchanges).

As an aside, this
sign change on interchange is taken as the defining property of the
wedge product in differential forms.  That property also
implies also that the wedge product is
zero when a vector is wedged with itself since zero is the only
value that is the negation of itself.  Thus we see explicitly
how the notation of using the wedge for the highest grade term
of two blades is consistent with the traditional
wedge product definition.

The end result here is that the grade 6 term of a trivector trivector product
changes sign on interchange of the trivectors:

\begin{equation}\label{eqn:trivector:trivecgpgrade6}
\gpgrade{\BA\BB}{6} = -\gpgrade{\BB\BA}{6}
\end{equation}

\subsection{Grade 4 term}

For a trivector product to have a grade 4 term there must be a common
vector between the two

\begin{equation}\label{eqn:trivector:66}
\BA = \Bx \wedge \By \wedge \Bz = \Bx\By\Bz
\end{equation}

and

\begin{equation}\label{eqn:trivector:86}
\BB = \Bu \wedge \Bv \wedge \Bz =\Bu\Bv\Bz
\end{equation}

The grade four term of the product is

\begin{equation}\label{eqn:trivector:206}
\begin{aligned}
\gpgrade{\BB \BA}{4}
&= \gpgrade{ \Bu\Bv\Bz \Bx\By\Bz }{4} \\
&= \gpgrade{ \Bu\Bv\Bz \Bz\Bx\By }{4} \\
&= \Bz^2\gpgrade{ \Bu\Bv\Bx\By }{4} \\
&= \Bz^2\gpgrade{ \Bu(\Bv\Bx)\By }{4} \\
&= \Bz^2\gpgrade{ \Bu(-\Bx\Bv + 2 \Bx \cdot \Bv)\By }{4} \\
&= -\Bz^2\gpgrade{ \Bu\Bx\Bv\By }{4} \\
&= \cdots \\
&= \Bz^2\gpgrade{ \Bx\By\Bu\Bv }{4} \\
&= \gpgrade{ \Bx\By\Bz\Bz\Bu\Bv }{4} \\
&= \gpgrade{ \Bx\By\Bz\Bu\Bv\Bz }{4} \\
&= \gpgrade{ \Bx\By\Bz\Bu\Bv\Bz }{4} \\
&= \gpgrade{\BA \BB}{4}
\end{aligned}
\end{equation}

Thus the grade 4 term commutes on interchange:

\begin{equation}\label{eqn:trivector:trivecgpgrade4}
\gpgrade{\BA\BB}{4} = \gpgrade{\BB\BA}{4}
\end{equation}

\subsection{Grade 2 term}

Similar to above,
for a trivector product to have a grade 2 term there must be two common
vectors between the two

\begin{equation}\label{eqn:trivector:106}
\BA = \Bx \wedge \By \wedge \Bz = \Bx\By\Bz
\end{equation}

and

\begin{equation}\label{eqn:trivector:126}
\BB = \Bu \wedge \By \wedge \Bz =\Bu\By\Bz
\end{equation}

The grade two term of the product is

\begin{equation}\label{eqn:trivector:226}
\begin{aligned}
\gpgrade{\BA \BB}{2}
&= \gpgrade{ \Bx\By\Bz \Bu\By\Bz }{2} \\
&= \gpgrade{ \Bx\By\Bz \By\Bz \Bu}{2} \\
&= (\By\Bz)^2\gpgrade{ \Bx \Bu}{2} \\
&= -(\By\Bz)^2\gpgrade{ \Bu \Bx}{2} \\
&= -\gpgrade{ \BB \BA }{2} \\
\end{aligned}
\end{equation}

The grade 2 term anticommutes on interchange:

\begin{equation}\label{eqn:trivector:trivecgpgrade2}
\gpgrade{\BA\BB}{2} = -\gpgrade{\BB\BA}{2}
\end{equation}

\subsection{Grade 0 term}

Any grade 0 terms are due to products of the form \(\BA = k\BB\)

\begin{equation}\label{eqn:trivector:246}
\begin{aligned}
\gpgrade{\BA \BB}{0}
&= \gpgrade{k\BB \BB}{0} \\
&= \gpgrade{\BB k\BB}{0} \\
&= \gpgrade{\BB \BA}{0} \\
\end{aligned}
\end{equation}

The grade 2 term commutes on interchange:

\begin{equation}\label{eqn:trivector:trivecgpgrade0}
\gpgrade{\BA\BB}{0} = \gpgrade{\BB\BA}{0}
\end{equation}

\subsection{combining results}

\begin{equation*}
\BA \BB
=\gpgrade{\BA\BB}{0}
+\gpgrade{\BA\BB}{2}
+\gpgrade{\BA\BB}{4}
+\gpgrade{\BA\BB}{6}
\end{equation*}

\begin{equation}\label{eqn:trivector:266}
\begin{aligned}
\BB\BA
&=\gpgrade{\BB\BA}{0}
+\gpgrade{\BB\BA}{2}
+\gpgrade{\BB\BA}{4}
+\gpgrade{\BB\BA}{6} \\
&=\gpgrade{\BA\BB}{0}
-\gpgrade{\BA\BB}{2}
+\gpgrade{\BA\BB}{4}
-\gpgrade{\BA\BB}{6} \\
\end{aligned}
\end{equation}

These can be combined to express each of the grade terms as subsets
of the symmetric and antisymmetric parts:

\begin{equation}\label{eqn:trivector:286}
\begin{aligned}
\BA \cdot \BB = \gpgrade{\BA\BB}{0} &= \gpgrade{\frac{\BA\BB + \BB\BA}{2}}{0} \\
\gpgrade{\BA\BB}{2} &= \gpgrade{\frac{\BA\BB - \BB\BA}{2}}{2} \\
\gpgrade{\BA\BB}{4} &= \gpgrade{\frac{\BA\BB + \BB\BA}{2}}{4} \\
\BA \wedge \BB = \gpgrade{\BA\BB}{6} &= \gpgrade{\frac{\BA\BB - \BB\BA}{2}}{6} \\
\end{aligned}
\end{equation}

Note that above I have been somewhat loose with the argument above.  A grade three vector
will have the following form:

\begin{equation}\label{eqn:trivector:146}
\sum_{i<j<k} D_{ijk} \Be_{ijk}
\end{equation}

Where \(D_{ijk}\) is the determinant of \(ijk\) components of the vectors being wedged.  Thus the product
of two trivectors will be of the following form:

\begin{equation}\label{eqn:trivector:166}
\sum_{i<j<k} \sum_{i'<j'<k'} D_{ijk} D'_{i'j'k'} (\Be_{ijk} \Be_{i'j'k'})
\end{equation}

It is really each of these \(\Be_{ijk} \Be_{i'j'k'}\) products that have to be considered in the grade
and sign arguments above.  The end result will be the same though... one would just have to present
it a bit more carefully for a true proof.

\subsection{Intersecting trivector cases}

As with the intersecting bivector case, when there is a line of intersection between the two volumes one can
write:

\begin{equation}\label{eqn:trivector:306}
\begin{aligned}
\BA \cdot \BB = \gpgrade{\BA\BB}{0} &= \gpgrade{\frac{\BA\BB + \BB\BA}{2}}{0} \\
\gpgrade{\BA\BB}{2} &= \frac{\BA\BB - \BB\BA}{2} \\
\gpgrade{\BA\BB}{4} &= \gpgrade{\frac{\BA\BB + \BB\BA}{2}}{4} \\
\BA \wedge \BB = \gpgrade{\BA\BB}{6} &= 0 \\
\end{aligned}
\end{equation}

And if these volumes intersect in a plane a further simplification is possible:
\begin{equation}\label{eqn:trivector:326}
\begin{aligned}
\BA \cdot \BB = \gpgrade{\BA\BB}{0} &= \frac{\BA\BB + \BB\BA}{2} \\
\gpgrade{\BA\BB}{2} &= \frac{\BA\BB - \BB\BA}{2} \\
\gpgrade{\BA\BB}{4} &= 0 \\
\BA \wedge \BB = \gpgrade{\BA\BB}{6} &= 0 \\
\end{aligned}
\end{equation}


\include{scalar_commutes}
%
% Copyright � 2012 Peeter Joot.  All Rights Reserved.
% Licenced as described in the file LICENSE under the root directory of this GIT repository.
%

%
%
\chapter{Blade grade reduction}
\index{grade reduction}
\label{chap:bladegradereduction}
%\date{Mar 25, 2008.  bladegradereduction.tex}

\section{General triple product reduction formula}

Consideration of the reciprocal frame bivector decomposition required the following identity

\begin{equation}
(\BA_a \wedge \BA_b) \cdot \BA_c =
\BA_a \cdot (\BA_b \cdot \BA_c)
\end{equation}

This holds when \(a + b \le c\), and \(a <= b\).  Similar equations for vector wedge blade dot blade reduction can be found in NFCM, but intuition let me to believe the above generalization was valid.

To prove this use the definition of the generalized dot product of two blades:

\begin{equation}\label{eqn:bladegradereduction:282}
\begin{aligned}
(\BA_a \wedge \BA_b) \cdot \BA_c
&= \gpgrade{ (\BA_a \wedge \BA_b) \BA_c }{\abs{c-(a+b)}} \\
\end{aligned}
\end{equation}

The subsequent discussion
is restricted to the \(b \ge a\) case.  Would have to think whether this restriction is required.

\begin{equation}
\label{eqn:bladegradereduction:bladewedge}
\begin{aligned}
\BA_a \wedge \BA_b
&= \BA_a \BA_b - \sum_{i=\abs{b-a},i+=2}^{a+b}\gpgrade{\BA_a\BA_b}{i} \\
&= \BA_a \BA_b - \sum_{k=0}^{a-1}\gpgrade{\BA_a\BA_b}{2k + b - a} \\
\end{aligned}
\end{equation}

Back substitution gives:

\begin{equation}\label{eqn:bladegradereduction:322}
\begin{aligned}
\gpgrade{ (\BA_a \wedge \BA_b) \BA_c }{\abs{c-(a+b)}}
&=
\gpgrade{ \BA_a \BA_b \BA_c }{\abs{c-(a+b)}}
-
\sum_{k=0}^{a-1}
\gpgrade{ \gpgrade{\BA_a\BA_b}{2k + b - a} \BA_c }{c-a-b}
\end{aligned}
\end{equation}

Temporarily writing \(\gpgrade{\BA_a\BA_b}{2k + b - a} = \BC_i\),
\begin{equation}\label{eqn:bladegradereduction:342}
\begin{aligned}
\gpgrade{\BA_a\BA_b}{2k + b - a} \BA_c
&= \sum_{j=c-i,j+=2}^{c+i} \gpgrade{ \BC_i \BA_c }{j} \\
&= \sum_{r=0}^{i} \gpgrade{ \BC_i \BA_c }{c-i+2r} \\
&= \sum_{r=0}^{2k+b-a} \gpgrade{ \BC_i \BA_c }{c-2k-b+a+2r} \\
&= \sum_{r=0}^{2k+b-a} \gpgrade{ \BC_i \BA_c }{c-b+a +2(r-k)} \\
\end{aligned}
\end{equation}

We want the only the following grade terms:

\begin{equation}\label{eqn:bladegradereduction:42}
c-b+a+2(r-k) = c - b - a
\implies
r=k-a
\end{equation}

There are many such \(k,r\) combinations, but we have a \(k \in [0,a-1]\) constraint, which implies \(r \in [-a,-1]\).  This contradicts with \(r\) strictly
positive,
so there are no such grade elements.

This gives an intermediate result, the reduction of the triple product to a direct product, removing the explicit wedge:

\begin{equation}
(\BA_a \wedge \BA_b) \cdot \BA_c =
\gpgrade{\BA_a \BA_b \BA_c}{c-a-b}
\end{equation}

\begin{equation}\label{eqn:bladegradereduction:362}
\begin{aligned}
\gpgrade{\BA_a \BA_b \BA_c}{c-a-b}
&= \gpgrade{\BA_a (\BA_b \BA_c)}{c-a-b} \\
&= \gpgrade{\BA_a \sum_{i}\gpgrade{\BA_b \BA_c}{i}}{c-a-b} \\
&= \gpgrade{\sum_{j}\gpgrade{\BA_a \sum_{i}\gpgrade{\BA_b \BA_c}{i}}{j}}{c-a-b} \\
\end{aligned}
\end{equation}

Explicitly specifying the grades here is omitted for simplicity.  The lowest grade of these is \((c-b)-a\), and all others are higher,
so grade selection excludes them.

By definition

\begin{equation}\label{eqn:bladegradereduction:62}
\gpgrade{\BA_b \BA_c}{c-b} = \BA_b \cdot \BA_c
\end{equation}

so that lowest grade term is thus

\begin{equation}\label{eqn:bladegradereduction:82}
\gpgrade{\BA_a \gpgrade{\BA_b \BA_c}{c-b}}{c-a-b}
= \gpgrade{\BA_a (\BA_b \cdot \BA_c)}{c-a-b}
= \BA_a \cdot (\BA_b \cdot \BA_c)
\end{equation}

This completes the proof.

\section{reduction of grade of dot product of two blades}

The result above can be applied to reducing the dot product of two blades.  For \(k<=s\):

\begin{equation}\label{eqn:bladegradereduction:102}
(\Ba_1 \wedge \Ba_2 \wedge \Ba_3 \cdots \wedge \Ba_k) \cdot (\Bb_1 \wedge \Bb_2 \cdots \wedge \Bb_s)
\end{equation}
\begin{equation}\label{eqn:bladegradereduction:382}
\begin{aligned}
&= (\Ba_1 \wedge (\Ba_2 \wedge \Ba_3 \cdots \wedge \Ba_k)) \cdot (\Bb_1 \wedge \Bb_2 \cdots \wedge \Bb_s) \\
&= (\Ba_1 \cdot ((\Ba_2 \wedge \Ba_3 \cdots \wedge \Ba_k)) \cdot (\Bb_1 \wedge \Bb_2 \cdots \wedge \Bb_s)) \\
&= (\Ba_1 \cdot (\Ba_2 \cdot (\Ba_3 \cdots \wedge \Ba_k)) \cdot (\Bb_1 \wedge \Bb_2 \cdots \wedge \Bb_s)) \\
&= \cdots \\
&= \Ba_1 \cdot (\Ba_2 \cdot (\Ba_3 \cdot (\cdots \cdot (\Ba_k \cdot (\Bb_1 \wedge \Bb_2 \cdots \wedge \Bb_s))))) \\
\end{aligned}
\end{equation}

This can be reduced to a single determinant, as is done in
the Flanders' differential forms book definition of the
\({\bigwedge}^k\) inner product (which is then used to define the Hodge dual).

The first such product is:

\begin{equation}\label{eqn:bladegradereduction:122}
\Ba_k \cdot (\Bb_1 \wedge \Bb_2 \cdots \wedge \Bb_k)
= \sum (-1)^{u-1} (\Ba_k \cdot \Bb_u) \Bb_1 \wedge \cdots \check{\Bb_u} \cdots \wedge \Bb_k
\end{equation}

Next, take dot product with \(\Ba_{k-1}\):

\begin{enumerate}
\item \(k = 2\)

\begin{equation}\label{eqn:bladegradereduction:402}
\begin{aligned}
&\Ba_{k-1} \cdot (\Ba_k \cdot (\Bb_1 \wedge \Bb_2 \cdots \wedge \Bb_k)) \\
&= \sum_{v \ne u} (-1)^{u-1} (\Ba_k \cdot \Bb_u) (\Ba_1 \cdot \Bb_v) \\
&=
 \sum_{u < v} (-1)^{v-1} (\Ba_k \cdot \Bb_v) (\Ba_1 \cdot \Bb_u)
+\sum_{u < v} (-1)^{u-1} (\Ba_k \cdot \Bb_u) (\Ba_1 \cdot \Bb_v) \\
&=
+\sum_{u < v} (\Ba_k \cdot \Bb_u) (\Ba_1 \cdot \Bb_v)
-\sum_{u < v} (\Ba_k \cdot \Bb_v) (\Ba_1 \cdot \Bb_u) \\
&=
+\sum_{u< v} (\Ba_k \cdot \Bb_u) (\Ba_1 \cdot \Bb_v)
- (\Ba_k \cdot \Bb_v) (\Ba_1 \cdot \Bb_u) \\
\end{aligned}
\end{equation}
\begin{equation}\label{eqn:bladegradereduction:k2dot}
-\sum_{u< v}
\begin{vmatrix}
\Ba_{k-1} \cdot \Bb_u & \Ba_{k-1} \cdot \Bb_v \\
\Ba_k \cdot \Bb_u & \Ba_k \cdot \Bb_v \\
\end{vmatrix}
\end{equation}

\item \(k>2\)
\end{enumerate}

\begin{equation}\label{eqn:bladegradereduction:142}
\Ba_{k-1} \cdot (\Ba_k \cdot (\Bb_1 \wedge \Bb_2 \cdots \wedge \Bb_k))
\end{equation}
\begin{equation}\label{eqn:bladegradereduction:422}
\begin{aligned}
&= \sum (-1)^{u-1} (\Ba_k \cdot \Bb_u) \Ba_{k-1} \cdot (\Bb_1 \wedge \cdots \check{\Bb_u} \cdots \wedge \Bb_k) \\
&= \sum_{v<u} (-1)^{u-1} (\Ba_k \cdot \Bb_u) (-1)^{v-1} (\Ba_{k-1} \cdot \Bb_v) (\Bb_1 \wedge \cdots \check{\Bb_v} \cdots \check{\Bb_u} \cdots \wedge \Bb_k) \\
&+ \sum_{v>u} (-1)^{u-1} (\Ba_k \cdot \Bb_u) (-1)^{v} (\Ba_{k-1} \cdot \Bb_v) (\Bb_1 \wedge \cdots \check{\Bb_u} \cdots \check{\Bb_v} \cdots \wedge \Bb_k) \\
\end{aligned}
\end{equation}

Add negation exponents, and use a change of variables for the first sum
\begin{equation}\label{eqn:bladegradereduction:442}
\begin{aligned}
&= \sum_{u<v} (-1)^{v+u} (\Ba_k \cdot \Bb_v) (\Ba_{k-1} \cdot \Bb_u) (\Bb_1 \wedge \cdots \check{\Bb_u} \cdots \check{\Bb_v} \cdots \wedge \Bb_k) \\
&- \sum_{u<v} (-1)^{u+v} (\Ba_k \cdot \Bb_u) (\Ba_{k-1} \cdot \Bb_v) (\Bb_1 \wedge \cdots \check{\Bb_u} \cdots \check{\Bb_v} \cdots \wedge \Bb_k) \\
\end{aligned}
\end{equation}

Merge sums:
\begin{equation}\label{eqn:bladegradereduction:462}
\begin{aligned}
&= \sum_{u<v} (-1)^{u+v}
\left(
(\Ba_k \cdot \Bb_v) (\Ba_{k-1} \cdot \Bb_u)
-(\Ba_k \cdot \Bb_u) (\Ba_{k-1} \cdot \Bb_v)
\right) \\
& \; (\Bb_1 \wedge \cdots \check{\Bb_u} \cdots \check{\Bb_v} \cdots \wedge \Bb_k)
\end{aligned}
\end{equation}

\begin{equation}\label{eqn:bladegradereduction:bivectordotkvector}
\Ba_{k-1} \cdot (\Ba_k \cdot (\Bb_1 \wedge \Bb_2 \cdots \wedge \Bb_k))
=
\end{equation}
\begin{equation*}
\sum_{u<v} (-1)^{u+v}
\begin{vmatrix}
\Ba_{k-1} \cdot \Bb_u & \Ba_{k-1} \cdot \Bb_v \\
\Ba_k \cdot \Bb_u & \Ba_k \cdot \Bb_v \\
\end{vmatrix}
(\Bb_1 \wedge \cdots \check{\Bb_u} \cdots \check{\Bb_v} \cdots \wedge \Bb_k) \\
\end{equation*}

Note that special casing \(k=2\) does not seem to be required because in that
case \(-1^{u+v} = -1^{1+2}=-1\), so this is identical to \eqnref{eqn:bladegradereduction:k2dot} after all.

\subsection{Pause to reflect}

Although my initial aim was to show that \(\BA_k \cdot \BB_k\) could be
expressed as a determinant as in the differential forms book (different
sign though), and to determine exactly what that determinant is, there
are some useful identities that fall out of this even just for this
bivector kvector dot product expansion.

Here is a summary of some of the things figured out so far

\begin{enumerate}
\item Dot product of grade one blades.

Here we have a result that can be expressed as a one by one determinant.  Worth mentioning to explicitly show the sign.

\begin{equation}\label{eqn:bladegradereduction:dotoneblades}
\Ba \cdot \Bb = \det[\Ba \cdot \Bb]
\end{equation}

%(Used \(\det{}\) here instead of \(\Det{}\) to avoid confusing with absolute value).
\item Dot product of grade two blades.

\begin{equation}\label{eqn:bladegradereduction:k2k2dot}
(\Ba_1 \wedge \Ba_2) \cdot (\Bb_1 \wedge \Bb_2)
=
-
\begin{vmatrix}
\Ba_1 \cdot \Bb_1 & \Ba_1 \cdot \Bb_2 \\
\Ba_2 \cdot \Bb_1 & \Ba_2 \cdot \Bb_2 \\
\end{vmatrix}
=
-\det[\Ba_i \cdot \Bb_j]
\end{equation}

\item Dot product of grade two blade with grade \(>2\) blade.

\begin{equation*}
(\Ba_{1} \wedge \Ba_2) \cdot (\Bb_1 \wedge \Bb_2 \cdots \wedge \Bb_k)
\end{equation*}
\begin{equation}\label{eqn:bladegradereduction:bivectordot}
=
\sum_{u<v} (-1)^{u+v-1}
(\Ba_1 \wedge \Ba_2) \cdot (\Bb_u \wedge \Bb_v)
(\Bb_1 \wedge \cdots \check{\Bb_u} \cdots \check{\Bb_v} \cdots \wedge \Bb_k)
\end{equation}
\end{enumerate}

Observe how similar this is to the vector blade dot product expansion:

\begin{equation}\label{eqn:bladegradereduction:vectordot}
\Ba \cdot (\Bb_1 \wedge \Bb_2 \cdots \wedge \Bb_k)
=
\sum (-1)^{i-1}
(\Ba \cdot \Bb_i) (\Bb_1 \wedge \cdots \check{\Bb_i} \cdots \wedge \Bb_k)
\end{equation}

\subsubsection{Expand it for \texorpdfstring{\(k=3\)}{k equal 3}}

Explicit expansion of \eqnref{eqn:bladegradereduction:bivectordot} for the \(k=3\) case, is also helpful to get a feel for
the equation:

\begin{equation}\label{eqn:bladegradereduction:482}
\begin{aligned}
(\Ba_{1} \wedge \Ba_2) \cdot (\Bb_1 \wedge \Bb_2 \wedge \Bb_3)
&=
(\Ba_1 \wedge \Ba_2) \cdot (\Bb_1 \wedge \Bb_2) \Bb_3 \\
&+(\Ba_1 \wedge \Ba_2) \cdot (\Bb_3 \wedge \Bb_1) \Bb_2 \\
&+(\Ba_1 \wedge \Ba_2) \cdot (\Bb_2 \wedge \Bb_3) \Bb_1
\end{aligned}
\end{equation}

Observe the cross product like alternation in sign and indices.
This suggests that a more natural way to express the sign coefficient may be via a \(\Sgn(\pi)\) expression for the sign of the
permutation of indices.

\section{trivector dot product}

With the result of \eqnref{eqn:bladegradereduction:bivectordot}, or the earlier equivalent determinant expression in equation
\eqnref{eqn:bladegradereduction:bivectordotkvector} we are now in a position to evaluate the dot product of a trivector and a greater or equal grade blade.

\begin{equation*}
\Ba_1 \cdot ((\Ba_{2} \wedge \Ba_3) \cdot (\Bb_1 \wedge \Bb_2 \cdots \wedge \Bb_k))
\end{equation*}
\begin{equation}\label{eqn:bladegradereduction:502}
\begin{aligned}
&=
\sum_{u<v} (-1)^{u+v-1}
(\Ba_2 \wedge \Ba_3) \cdot (\Bb_u \wedge \Bb_v)
\Ba_1 \cdot (\Bb_1 \wedge \cdots \check{\Bb_u} \cdots \check{\Bb_v} \cdots \wedge \Bb_k)  \\
&=
\sum_{w<u<v} (-1)^{u+v+w}
(\Ba_2 \wedge \Ba_3) \cdot (\Bb_u \wedge \Bb_v)
(\Ba_1 \cdot \Bb_w) (\Bb_1 \wedge \cdots \check{\Bb_w} \cdots \check{\Bb_u} \cdots \check{\Bb_v} \cdots \wedge \Bb_k)  \\
&+\sum_{u<w<v} (-1)^{u+v+w-1}
(\Ba_2 \wedge \Ba_3) \cdot (\Bb_u \wedge \Bb_v)
(\Ba_1 \cdot \Bb_w) (\Bb_1 \wedge \cdots \check \Bb_u \cdots \check{\Bb_w} \cdots \check{\Bb_v} \cdots \wedge \Bb_k)  \\
&+\sum_{u<v<w} (-1)^{u+v+w}
(\Ba_2 \wedge \Ba_3) \cdot (\Bb_u \wedge \Bb_v)
(\Ba_1 \cdot \Bb_w) (\Bb_1 \wedge \cdots \check \Bb_u \cdots \check{\Bb_v} \cdots \check{\Bb_w} \cdots \wedge \Bb_k)  \\
\end{aligned}
\end{equation}

Change the indices of summation and grouping like terms we have:
\begin{equation}\label{eqn:bladegradereduction:522}
\begin{aligned}
\sum_{u<v<w} (-1)^{u+v+w}
(
&(\Ba_2 \wedge \Ba_3) \cdot (\Bb_v \wedge \Bb_w) (\Ba_1 \cdot \Bb_u)  \\
&-(\Ba_2 \wedge \Ba_3) \cdot (\Bb_u \wedge \Bb_w) (\Ba_1 \cdot \Bb_v)  \\
&+(\Ba_2 \wedge \Ba_3) \cdot (\Bb_u \wedge \Bb_v) (\Ba_1 \cdot \Bb_w)  \\
)
(\Bb_1 \wedge \cdots \check \Bb_u \cdots \check{\Bb_v} \cdots \check{\Bb_w} \cdots \wedge \Bb_k)  \\
\end{aligned}
\end{equation}

Now, each of the embedded dot products were in fact determinants:
\begin{equation}\label{eqn:bladegradereduction:162}
(\Ba_2 \wedge \Ba_3) \cdot (\Bb_x \wedge \Bb_y)
=
-
\begin{vmatrix}
\Ba_2 \cdot \Bb_x & \Ba_2 \cdot \Bb_y \\
\Ba_3 \cdot \Bb_x & \Ba_3 \cdot \Bb_y \\
\end{vmatrix}
\end{equation}

Thus, we can expand these triple dot products like so (factor of \(-1\) omitted):
\begin{equation}\label{eqn:bladegradereduction:542}
\begin{aligned}
&(\Ba_2 \wedge \Ba_3) \cdot (\Bb_v \wedge \Bb_w) (\Ba_1 \cdot \Bb_u) \\
&-(\Ba_2 \wedge \Ba_3) \cdot (\Bb_u \wedge \Bb_w) (\Ba_1 \cdot \Bb_v) \\
&+(\Ba_2 \wedge \Ba_3) \cdot (\Bb_u \wedge \Bb_v) (\Ba_1 \cdot \Bb_w)  \\
&=
(\Ba_1 \cdot \Bb_u)
\begin{vmatrix}
\Ba_2 \cdot \Bb_v & \Ba_2 \cdot \Bb_w \\
\Ba_3 \cdot \Bb_v & \Ba_3 \cdot \Bb_w \\
\end{vmatrix} \\
&-
(\Ba_1 \cdot \Bb_v)
\begin{vmatrix}
\Ba_2 \cdot \Bb_u & \Ba_2 \cdot \Bb_w \\
\Ba_3 \cdot \Bb_u & \Ba_3 \cdot \Bb_w \\
\end{vmatrix} \\
&+
(\Ba_1 \cdot \Bb_w)
\begin{vmatrix}
\Ba_2 \cdot \Bb_u & \Ba_2 \cdot \Bb_v \\
\Ba_3 \cdot \Bb_u & \Ba_3 \cdot \Bb_v \\
\end{vmatrix} \\
%&=
%\begin{vmatrix}
%\Ba_1 \cdot \Bb_u & 0 & 0 \\
%0 & \Ba_2 \cdot \Bb_v & \Ba_2 \cdot \Bb_w \\
%0 & \Ba_3 \cdot \Bb_v & \Ba_3 \cdot \Bb_w \\
%\end{vmatrix} \\
%&+
%\begin{vmatrix}
%0 & \Ba_1 \cdot \Bb_v & 0 \\
%\Ba_2 \cdot \Bb_u & 0 & \Ba_2 \cdot \Bb_w \\
%\Ba_3 \cdot \Bb_u & 0 & \Ba_3 \cdot \Bb_w \\
%\end{vmatrix} \\
%&+
%\begin{vmatrix}
%0 & 0 & \Ba_1 \cdot \Bb_w \\
%\Ba_2 \cdot \Bb_u & \Ba_2 \cdot \Bb_v & 0 \\
%\Ba_3 \cdot \Bb_u & \Ba_3 \cdot \Bb_v & 0 \\
%\end{vmatrix} \\
&=
\begin{vmatrix}
\Ba_1 \cdot \Bb_u & \Ba_1 \cdot \Bb_v & \Ba_1 \cdot \Bb_w \\
\Ba_2 \cdot \Bb_u & \Ba_2 \cdot \Bb_v & \Ba_2 \cdot \Bb_w \\
\Ba_3 \cdot \Bb_u & \Ba_3 \cdot \Bb_v & \Ba_3 \cdot \Bb_w \\
\end{vmatrix} \\
\end{aligned}
\end{equation}

Final back substitution gives:

\begin{equation*}
(\Ba_1 \wedge \Ba_{2} \wedge \Ba_3) \cdot (\Bb_1 \wedge \Bb_2 \cdots \wedge \Bb_k)
\end{equation*}
\begin{equation}\label{eqn:bladegradereduction:trivectordotdet}
=
\sum_{u<v<w} (-1)^{u+v+w-1}
\begin{vmatrix}
\Ba_1 \cdot \Bb_u & \Ba_1 \cdot \Bb_v & \Ba_1 \cdot \Bb_w \\
\Ba_2 \cdot \Bb_u & \Ba_2 \cdot \Bb_v & \Ba_2 \cdot \Bb_w \\
\Ba_3 \cdot \Bb_u & \Ba_3 \cdot \Bb_v & \Ba_3 \cdot \Bb_w \\
\end{vmatrix}
(\Bb_1 \wedge \cdots \check \Bb_u \cdots \check{\Bb_v} \cdots \check{\Bb_w} \cdots \wedge \Bb_k)  \\
\end{equation}

In particular for \(k=3\) we have
\begin{equation*}
(\Ba_1 \wedge \Ba_{2} \wedge \Ba_3) \cdot (\Bb_1 \wedge \Bb_2 \wedge \Bb_3)
\end{equation*}
\begin{equation}\label{eqn:bladegradereduction:trivectordotdettri}
=
-\begin{vmatrix}
\Ba_1 \cdot \Bb_1 & \Ba_1 \cdot \Bb_2 & \Ba_1 \cdot \Bb_3 \\
\Ba_2 \cdot \Bb_1 & \Ba_2 \cdot \Bb_2 & \Ba_2 \cdot \Bb_3 \\
\Ba_3 \cdot \Bb_1 & \Ba_3 \cdot \Bb_2 & \Ba_3 \cdot \Bb_3 \\
\end{vmatrix}
=
-\det[\Ba_i \cdot \Bb_j]
\end{equation}

This can be substituted back into \eqnref{eqn:bladegradereduction:trivectordotdet} to put it in a non determinant form.

\begin{equation*}
(\Ba_1 \wedge \Ba_{2} \wedge \Ba_3) \cdot (\Bb_1 \wedge \Bb_2 \cdots \wedge \Bb_k)
\end{equation*}
\begin{equation}\label{eqn:bladegradereduction:trivectordotnondet}
=
\sum_{u<v<w} (-1)^{u+v+w}
(\Ba_1 \wedge \Ba_{2} \wedge \Ba_3) \cdot (\Bb_u \wedge \Bb_v \wedge \Bb_w)
(\Bb_1 \wedge \cdots \check \Bb_u \cdots \check{\Bb_v} \cdots \check{\Bb_w} \cdots \wedge \Bb_k)  \\
\end{equation}

\section{Induction on the result}

It is pretty clear that recursively performing these calculations will yield similar determinant and inner dot product reduction
results.

\subsection{dot product of like grade terms as determinant}

Let us consider the equal grade case first, summarizing the results so far

\begin{equation}\label{eqn:bladegradereduction:562}
\begin{aligned}
\Ba \cdot \Bb &= \det[\Ba \cdot \Bb] \\
(\Ba_1 \wedge \Ba_2) \cdot (\Bb_1 \wedge \Bb_2) &= -\det[\Ba_i \cdot \Bb_j] \\
(\Ba_1 \wedge \Ba_2 \wedge \Ba_3) \cdot (\Bb_1 \wedge \Bb_2 \wedge \Bb_3) &= -\det[\Ba_i \cdot \Bb_j] \\
\end{aligned}
\end{equation}

What will the sign be for the higher grade equivalents?  It has the appearance of being related to the sign associated with blade
reversion.  To verify this calculate the dot product of a blade formed from a set of perpendicular unit vectors with itself.

\begin{equation}\label{eqn:bladegradereduction:582}
\begin{aligned}
&(\Be_1 \wedge \cdots \wedge \Be_k) \cdot (\Be_1 \wedge \Be_2 \wedge \cdots \wedge \Be_k) \\
&= (-1)^{k(k-1)/2}(\Be_1 \wedge \cdots \wedge \Be_k) \cdot (\Be_k \wedge \cdots \wedge \Be_2 \wedge \Be_1) \\
&= (-1)^{k(k-1)/2}\Be_1 \cdot (\Be_2 \cdots (\Be_k \cdot (\Be_k \wedge \cdots \wedge \Be_2 \wedge \Be_1))) \\
&= (-1)^{k(k-1)/2}\Be_1 \cdot (\Be_2 \cdots (\Be_{k-1} \cdot (\Be_{k-1} \wedge \cdots \wedge \Be_2 \wedge \Be_1))) \\
&= \cdots \\
&= (-1)^{k(k-1)/2}
\end{aligned}
\end{equation}

This fixes the sign, and provides the induction hypothesis for the general case:

\begin{equation}\label{eqn:bladegradereduction:bladedothyp}
(\Ba_1 \wedge \cdots \wedge \Ba_k) \cdot (\Bb_1 \wedge \Bb_2 \wedge \cdots \wedge \Bb_k) = (-1)^{k(k-1)/2}\det[\Ba_i \cdot \Bb_j]
\end{equation}

Alternately, one can remove the sign change coefficient with reversion of one of the blades:

\begin{equation}\label{eqn:bladegradereduction:bladedothyprev}
(\Ba_1 \wedge \cdots \wedge \Ba_k) \cdot (\Bb_k \wedge \Bb_{k-1} \wedge \cdots \wedge \Bb_1) = \det[\Ba_i \cdot \Bb_j]
\end{equation}

\subsection{Unlike grades}

Let us summarize the results for unlike grades at the same time reformulating the previous results in terms of index
permutation, also writing for brevity \(\BA_s = \Ba_1 \wedge \cdots \wedge \Ba_s\), and \(\BB_k = \Bb_1 \wedge \cdots \wedge \Bb_k\):

\begin{equation}\label{eqn:bladegradereduction:182}
\BA_1 \cdot \BB_k =
\sum_i \Sgn(\pi(i,1,2,\cdots\check{i}\cdots,k)) (\BA_1 \cdot \Bb_i) (\Bb_1 \wedge \cdots \check{\Bb_i} \cdots \wedge \Bb_k)
\end{equation}

\begin{equation}\label{eqn:bladegradereduction:202}
\BA_2 \cdot \BB_k =
\sum_{i_1<i_2} \Sgn(\pi(i_1,i_2,1,2,\cdots\check{i_1}\cdots\check{i_2}\cdots,k))
\end{equation}
\begin{equation}\label{eqn:bladegradereduction:222}
   \BA_2 \cdot (\Bb_{i_1} \wedge \Bb_{i_2})
   (\Bb_1 \wedge \cdots \check{\Bb_{i_1}} \cdots \check{\Bb_{i_2}} \cdots \wedge \Bb_k)
\end{equation}

\begin{equation}\label{eqn:bladegradereduction:242}
\BA_3 \cdot \BB_k =
\sum_{i_1<i_2<i_3} \Sgn(\pi(i_1,i_2,i_3,1,2,\cdots\check{i_1}\cdots\check{i_2}\cdots\check{i_3}\cdots,k))
\end{equation}
\begin{equation}\label{eqn:bladegradereduction:262}
\BA_3 \cdot (\Bb_{i_1} \wedge \Bb_{i_2} \wedge \Bb_{i_3})
(\Bb_1 \wedge \cdots \check{\Bb_{i_1}} \cdots \check{\Bb_{i_2}} \cdots \check{\Bb_{i_3}} \cdots \wedge \Bb_k)
\end{equation}

We see that the dot product consumes any of the excess sign variation not described by the sign of the permutation of indices.

The induction hypothesis is basically described above (change \(3\) to \(s\), and add extra dots):

\begin{equation*}
\BA_s \cdot \BB_k =
\sum_{i_1<i_2\cdots<i_s} \Sgn(\pi(i_1,i_2\cdots,i_s,1,2,\cdots\check{i_1}\cdots\check{i_2}\cdots\check{i_s}\cdots,k))
\end{equation*}
\begin{equation}\label{eqn:bladegradereduction:inductionbigdotblade}
\BA_s \cdot (\Bb_{i_1} \wedge \Bb_{i_2} \cdots \wedge \Bb_{i_s})
(\Bb_1 \wedge \cdots \check{\Bb_{i_1}} \cdots \check{\Bb_{i_2}} \cdots \check{\Bb_{i_s}} \cdots \wedge \Bb_k)
\end{equation}

\subsection{Perform the induction}

In a sense this has already been done.  The steps will be pretty much the same as the logic that produced the bivector and trivector
results.  Thinking about typing this up in latex is not fun, so this will be left for a paper proof.

\documentclass{article}      

\usepackage{amsmath}
\usepackage{mathpazo}

%
% shorthand for bold symbols, convenient for vectors and matrices
%
\newcommand{\Ba}[0]{\mathbf{a}}
\newcommand{\Bb}[0]{\mathbf{b}}
\newcommand{\Bc}[0]{\mathbf{c}}
\newcommand{\Bd}[0]{\mathbf{d}}
\newcommand{\Be}[0]{\mathbf{e}}
\newcommand{\Bf}[0]{\mathbf{f}}
\newcommand{\Bg}[0]{\mathbf{g}}
\newcommand{\Bh}[0]{\mathbf{h}}
\newcommand{\Bi}[0]{\mathbf{i}}
\newcommand{\Bj}[0]{\mathbf{j}}
\newcommand{\Bk}[0]{\mathbf{k}}
\newcommand{\Bl}[0]{\mathbf{l}}
\newcommand{\Bm}[0]{\mathbf{m}}
\newcommand{\Bn}[0]{\mathbf{n}}
\newcommand{\Bo}[0]{\mathbf{o}}
\newcommand{\Bp}[0]{\mathbf{p}}
\newcommand{\Bq}[0]{\mathbf{q}}
\newcommand{\Br}[0]{\mathbf{r}}
\newcommand{\Bs}[0]{\mathbf{s}}
\newcommand{\Bt}[0]{\mathbf{t}}
\newcommand{\Bu}[0]{\mathbf{u}}
\newcommand{\Bv}[0]{\mathbf{v}}
\newcommand{\Bw}[0]{\mathbf{w}}
\newcommand{\Bx}[0]{\mathbf{x}}
\newcommand{\By}[0]{\mathbf{y}}
\newcommand{\Bz}[0]{\mathbf{z}}
\newcommand{\BA}[0]{\mathbf{A}}
\newcommand{\BB}[0]{\mathbf{B}}
\newcommand{\BC}[0]{\mathbf{C}}
\newcommand{\BD}[0]{\mathbf{D}}
\newcommand{\BE}[0]{\mathbf{E}}
\newcommand{\BF}[0]{\mathbf{F}}
\newcommand{\BG}[0]{\mathbf{G}}
\newcommand{\BH}[0]{\mathbf{H}}
\newcommand{\BI}[0]{\mathbf{I}}
\newcommand{\BJ}[0]{\mathbf{J}}
\newcommand{\BK}[0]{\mathbf{K}}
\newcommand{\BL}[0]{\mathbf{L}}
\newcommand{\BM}[0]{\mathbf{M}}
\newcommand{\BN}[0]{\mathbf{N}}
\newcommand{\BO}[0]{\mathbf{O}}
\newcommand{\BP}[0]{\mathbf{P}}
\newcommand{\BQ}[0]{\mathbf{Q}}
\newcommand{\BR}[0]{\mathbf{R}}
\newcommand{\BS}[0]{\mathbf{S}}
\newcommand{\BT}[0]{\mathbf{T}}
\newcommand{\BU}[0]{\mathbf{U}}
\newcommand{\BV}[0]{\mathbf{V}}
\newcommand{\BW}[0]{\mathbf{W}}
\newcommand{\BX}[0]{\mathbf{X}}
\newcommand{\BY}[0]{\mathbf{Y}}
\newcommand{\BZ}[0]{\mathbf{Z}}

\newcommand{\Bzero}[0]{\mathbf{0}}
\newcommand{\Btheta}[0]{\boldsymbol{\theta}}
\newcommand{\Btau}[0]{\boldsymbol{\tau}}
\newcommand{\Bomega}[0]{\boldsymbol{\omega}}

%
% shorthand for unit vectors
%
\newcommand{\acap}[0]{\hat{\Ba}}
\newcommand{\bcap}[0]{\hat{\Bb}}
\newcommand{\ccap}[0]{\hat{\Bc}}
\newcommand{\dcap}[0]{\hat{\Bd}}
\newcommand{\ecap}[0]{\hat{\Be}}
\newcommand{\fcap}[0]{\hat{\Bf}}
\newcommand{\gcap}[0]{\hat{\Bg}}
\newcommand{\hcap}[0]{\hat{\Bh}}
\newcommand{\icap}[0]{\hat{\Bi}}
\newcommand{\jcap}[0]{\hat{\Bj}}
\newcommand{\kcap}[0]{\hat{\Bk}}
\newcommand{\lcap}[0]{\hat{\Bl}}
\newcommand{\mcap}[0]{\hat{\Bm}}
\newcommand{\ncap}[0]{\hat{\Bn}}
\newcommand{\ocap}[0]{\hat{\Bo}}
\newcommand{\pcap}[0]{\hat{\Bp}}
\newcommand{\qcap}[0]{\hat{\Bq}}
\newcommand{\rcap}[0]{\hat{\Br}}
\newcommand{\scap}[0]{\hat{\Bs}}
\newcommand{\tcap}[0]{\hat{\Bt}}
\newcommand{\ucap}[0]{\hat{\Bu}}
\newcommand{\vcap}[0]{\hat{\Bv}}
\newcommand{\wcap}[0]{\hat{\Bw}}
\newcommand{\xcap}[0]{\hat{\Bx}}
\newcommand{\ycap}[0]{\hat{\By}}
\newcommand{\zcap}[0]{\hat{\Bz}}
\newcommand{\thetacap}[0]{\hat{\Btheta}}

%
% to write R^n and C^n in a distinguishable fashion.  Perhaps change this
% to the double lined characters upon figuring out how to do so.
%
\newcommand{\C}[1]{$\mathbb{C}^{#1}$}
\newcommand{\R}[1]{$\mathbb{R}^{#1}$}

%
% various generally useful helpers
%

% derivative of #1 wrt. #2:
\newcommand{\D}[2] {\frac {d#2} {d#1}}

\newcommand{\inv}[1]{\frac{1}{#1}}
\newcommand{\cross}[0]{\times}

\newcommand{\abs}[1]{\lvert{#1}\rvert}
\newcommand{\norm}[1]{\lVert{#1}\rVert}
\newcommand{\innerprod}[2]{\langle{#1}, {#2}\rangle}
\newcommand{\dotprod}[2]{{#1} \cdot {#2}}
\newcommand{\bdotprod}[2]{\left({#1} \cdot {#2}\right)}
\newcommand{\crossprod}[2]{{#1} \cross {#2}}
\newcommand{\tripleprod}[3]{\dotprod{\left(\crossprod{#1}{#2}\right)}{#3}}

\DeclareMathOperator{\Proj}{Proj}
\DeclareMathOperator{\Span}{span}
\DeclareMathOperator{\Sgn}{sgn}
\DeclareMathOperator{\Area}{Area}
\DeclareMathOperator{\Volume}{Volume}

%
% A few miscellaneous things specific to this document
%
\newcommand{\crossop}[1]{\crossprod{#1}{}}

% R2 vector.
\newcommand{\VectorTwo}[2]{
\begin{bmatrix}
 {#1} \\
 {#2}
\end{bmatrix}
}

\newcommand{\VectorN}[1]{
\begin{bmatrix}
{#1}_1 \\
{#1}_2 \\
\vdots \\
{#1}_N \\
\end{bmatrix}
}

\newcommand{\DETuvij}[4]{
\begin{vmatrix}
 {#1}_{#3} & {#1}_{#4} \\
 {#2}_{#3} & {#2}_{#4}
\end{vmatrix}
}

\newcommand{\DETuvwijk}[6]{
\begin{vmatrix}
 {#1}_{#4} & {#1}_{#5} & {#1}_{#6} \\
 {#2}_{#4} & {#2}_{#5} & {#2}_{#6} \\
 {#3}_{#4} & {#3}_{#5} & {#3}_{#6}
\end{vmatrix}
}

\newcommand{\DETuvwxijkl}[8]{
\begin{vmatrix}
 {#1}_{#5} & {#1}_{#6} & {#1}_{#7} & {#1}_{#8} \\
 {#2}_{#5} & {#2}_{#6} & {#2}_{#7} & {#2}_{#8} \\
 {#3}_{#5} & {#3}_{#6} & {#3}_{#7} & {#3}_{#8} \\
 {#4}_{#5} & {#4}_{#6} & {#4}_{#7} & {#4}_{#8} \\
\end{vmatrix}
}

%\newcommand{\DETuvwxyijklm}[10]{
%\begin{vmatrix}
% {#1}_{#6} & {#1}_{#7} & {#1}_{#8} & {#1}_{#9} & {#1}_{#10} \\
% {#2}_{#6} & {#2}_{#7} & {#2}_{#8} & {#2}_{#9} & {#2}_{#10} \\
% {#3}_{#6} & {#3}_{#7} & {#3}_{#8} & {#3}_{#9} & {#3}_{#10} \\
% {#4}_{#6} & {#4}_{#7} & {#4}_{#8} & {#4}_{#9} & {#4}_{#10} \\
% {#5}_{#6} & {#5}_{#7} & {#5}_{#8} & {#5}_{#9} & {#5}_{#10}
%\end{vmatrix}
%}

% R3 vector.
\newcommand{\VectorThree}[3]{
\begin{bmatrix}
 {#1} \\
 {#2} \\
 {#3}
\end{bmatrix}
}



                             % The preamble begins here.
\title{More details on NFCM plane formulation} % Declares the document's title.
\author{Peeter Joot}         % Declares the author's name.
%\date{}        % Deleting this command produces today's date.

\begin{document}             % End of preamble and beginning of text.

%\maketitle{}

\section{Wedge product formula for a plane.}

The equation of the plane with bivector $\BU$ through point $\Ba$ is given
by

\[
(\Bx - \Ba) \wedge \BU = 0
\]

or

\[
\Bx \wedge \BU = \Ba \wedge \BU = \BT
\]

\subsection{ Examining this equation in more details. }

Without any loss of generality one can express this plane equation
in terms of a unit bivector $\Bi$

\[
\Bx \wedge \Bi = \Ba \wedge \Bi
\]

As with the line equation, to express this in the ``standard'' parametric
form, right multiplication with $1/\Bi$ is required.

\[
(\Bx \wedge \Bi)\frac{1}{\Bi} = (\Ba \wedge \Bi)\frac{1}{\Bi}
\]

We have a trivector bivector product here, which in general has a vector,
trivector, and 5-vector component.  Since $\Bi \wedge \Bi = 0$, the
5-vector component is zero:

\[
\Bx \wedge \Bi \wedge -\Bi = 0
\]

and intuition says that the trivector component will also be zero.  However,
as well as providing verification of this, expansion of this product will also
demonstrate how to find the projective and rejective components of a vector
with respect to a plane (ie: components in and out of the plane).

\subsection{Rejection from a plane product expansion.}

Here's an explicit expansion of the rejective term above

\begin{align*}
(\Bx \wedge \Bi)\frac{1}{\Bi} 
&= -(\Bx \wedge \Bi){\Bi} \\ 
&= -\frac{1}{2}(\Bx\Bi + \Bi\Bx){\Bi} \\ 
&= \frac{1}{2}(\Bx - \Bi\Bx\Bi) \\ 
&= \frac{1}{2}(\Bx - (\Bx \Bi + 2 \Bi \cdot \Bx)\Bi) \\ 
&= \Bx - (\Bi \cdot \Bx)\Bi \\ 
\end{align*}

In this last term the quantity $\Bi \cdot \Bx$ is a vector in the plane.
This can be demonstrated by writing $\Bi$ in terms of a pair of orthonormal
vectors $\Bi = \ucap\vcap = \ucap \wedge \vcap$.

\begin{align*}
\Bi \cdot \Bx &= (\ucap \wedge \vcap) \cdot \Bx \\
              &= \ucap (\vcap \cdot \Bx) - \vcap (\ucap \cdot \Bx) \\
\end{align*}

Thus, $(\Bi \cdot \Bx) \wedge \Bi = 0$, 
and $(\Bi \cdot \Bx) \Bi = (\Bi \cdot \Bx) \cdot \Bi$.  Inserting this above
we have the end result

\begin{align*}
(\Bx \wedge \Bi)\frac{1}{\Bi} 
&= \Bx - (\Bi \cdot \Bx) \cdot \Bi \\ 
&= \Ba - (\Bi \cdot \Ba) \cdot \Bi \\ 
\end{align*}

Or
\begin{align*}
\Bx  - \Ba 
&= (\Bi \cdot (\Bx - \Ba)) \cdot \Bi \\ 
\end{align*}

This is actually the standard parametric equation of a plane, but expressed
in terms of a unit bivector that describes the plane instead of in terms
of a pair of vectors in the plane.

To demonstrate this expansion of the right hand side is required

\begin{align*}
(\Bi \cdot \Bx) \cdot \Bi
&= (\ucap (\vcap \cdot \Bx) - \vcap (\ucap \cdot \Bx)) \ucap \vcap \\
&= \vcap (\vcap \cdot \Bx) + \ucap (\ucap \cdot \Bx) \\
\end{align*}

Substituting this back yields:

\begin{align*}
\Bx 
&= \Ba + \ucap (\ucap \cdot (\Bx - \Ba)) + \vcap (\vcap \cdot (\Bx - \Ba)) \\
&= \Ba + s \ucap + t \vcap
\end{align*}

In words this says that the plane is specified by a point in the plane,
and the span
of a pair of orthonormal vectors directed in that plane.

This (but perhaps without neccessariliy using orthornomal direction vectors)
is often how the plane is defined to start with.

It isn't neccessarily obvious that the bivector wedge product formula for
a plane that we started with:

\[
\Bx \wedge \BU = \Ba \wedge \BU
\]

can also be used to express this parametric representation.

\subsection{ Orthonormal decomposition of a vector with respect to a plane. }

With the expansion above we have a separation of a vector into two
components, and these can be demonstrated to be the components that are
directed entirely within and out of the plane.

Rearranging terms from above we have:

\begin{align*}
\Bx 
&= 
(\Bx \cdot \Bi) \cdot \frac{1}{\Bi} + (\Bx \wedge \Bi) \cdot \frac{1}{\Bi} \\
&= 
(\Bx \cdot \Bi) \frac{1}{\Bi} + (\Bx \wedge \Bi) \frac{1}{\Bi} \\
\end{align*}

% write x = x_perp + x_parallel to show that this is a ortho decomp.
% can then write formula for directrix of plane.

\subsection{ Alternate derivation of orthonormal planar decomposition }

This could alternately be derived by expanding the vector unit bivector
product directly

\begin{align*}
\Bx \Bi \frac{1}{\Bi} 
&= ( \Bx \cdot \Bi + \Bx \wedge \Bi ) \frac{1}{\Bi} \\
&= 
- {(\Bx \cdot \Bi) \cdot \Bi} - {(\Bx \cdot \Bi) \wedge \Bi} - {(\Bx \wedge \Bi) \Bi} \\
&= 
- {(\Bx \cdot \Bi) \cdot \Bi} - {(\Bx \wedge \Bi) \cdot \Bi } - {<(\Bx \wedge \Bi) \Bi>_3} - {(\Bx \wedge \Bi) \wedge \Bi} \\
&= 
{(\Bx \cdot \Bi) \cdot \frac{1}{\Bi}} + {(\Bx \wedge \Bi) \cdot \frac{1}{\Bi}} - {<(\Bx \wedge \Bi) \Bi>_3} \\
\end{align*}

Since the LHS of this equation is the vector $\Bx$, the right hand side must
also be a vector, which demonstrates that the term

\[
<(\Bx \wedge \Bi) \Bi>_3 = 0
\]

So, one has

\begin{align*}
\Bx 
&=
{(\Bx \cdot \Bi) \cdot \frac{1}{\Bi}} + {(\Bx \wedge \Bi) \cdot \frac{1}{\Bi}} \\
&=
{(\Bx \cdot \Bi) \frac{1}{\Bi}} + {(\Bx \wedge \Bi) \frac{1}{\Bi}} \\
\end{align*}


\end{document}

%
% Copyright � 2012 Peeter Joot.  All Rights Reserved.
% Licenced as described in the file LICENSE under the root directory of this GIT repository.
%

%
%
\chapter{Rotor Notes}\label{chap:rotor}
\index{rotor}
%\date{Feb 19, 2008.  rotor.tex}

\section{Rotations strictly in a plane}

For a plane rotation, a rotation does not have to
be expressed in terms of left and right half angle rotations, as is the case
with complex numbers.  Starting with this ``natural'' one sided rotation
we will see why the half angle double sided Rotor formula works.

\subsection{Identifying a plane with a bivector.  Justification}
Given a bivector \(\BB\), we can say this defines the orientation of a plane
(through the origin)
since for any vector in the plane we have \(\BB \wedge \Bx = 0\), or any vector
strictly normal to the plane \(\BB \cdot \Bx = 0\).

Note that this naturally compares
to the equation of a line (through the origin) expressed in terms of a
direction vector \(\Bb\),
where \(\Bb \wedge \Bx=0\) if \(\Bx\) lies on the line, and \(\Bb \cdot \Bx = 0\)
if \(\Bx\) is normal to the line.

Given this it is not unreasonable to identify the plane with its bivector.  This
will be done below, and it should be clear that
loose language such as ``the plane \(\BB\)'', should really be interpreted
as ``the plane with direction bivector \(\BB\)'', where the direction bivector
has the wedge and dot product properties noted above.

\subsection{Components of a vector in and out of a plane}

To calculate the components of a vector in and out of a plane, we can form
the product

\begin{equation}\label{eqn:rotor:20}
\Bx = \Bx \BB \inv{\BB} = \Bx \cdot \BB \inv{\BB} + \Bx \wedge \BB \inv{\BB}
\end{equation}

This is an orthogonal decomposition of the vector \(\Bx\) where the first
part is the projective term onto the plane \(\BB\), and the second is the rejective
term, the component not in the plane.  Let us verify this.

Write \(\Bx = \Bx_\parallel + \Bx_\perp\), where \(\Bx_\parallel\), and \(\Bx_\perp\) are the components of \(\Bx\) parallel and perpendicular to the plane.  Also write
\(\BB = \Bb_1 \wedge \Bb_2\), where \(\Bb_i\) are non-colinear vectors in the plane \(\BB\).

If \(\Bx = \Bx_\parallel\), a vector entirely in the plane \(\BB\), then one can
write

\begin{equation}\label{eqn:rotor:40}
\Bx = a_1\Bb_1 + a_2\Bb_2
\end{equation}

and the wedge product term is zero

\begin{equation}\label{eqn:rotor:740}
\begin{aligned}
\Bx \wedge \BB
&= \left( a_1\Bb_1 + a_2\Bb_2 \right) \wedge \Bb_1 \wedge \Bb_2 \\
&= a_1 ( \Bb_1 \wedge \Bb_1 ) \wedge \Bb_2
 - a_2 ( \Bb_2 \wedge \Bb_2 ) \wedge \Bb_1 \\
&= 0
\end{aligned}
\end{equation}

Thus the component parallel to the plane is composed strictly of the dot
product term

\begin{equation}
\Bx_\parallel = \Bx \cdot \BB \inv{\BB}
\end{equation}

Or for a general vector not necessarily in the plane the component
of that vector in the plane, its projection onto the plane is,

\begin{equation}\label{eqn:rotor:60}
\Proj_{\BB}(\Bx) = \Bx \cdot \BB \inv{\BB}
= \inv{\abs{\BB}^2}(\BB \cdot \Bx)\BB
= (\hat{\BB} \cdot \Bx)\hat{\BB}
\end{equation}

Now, for a vector that lies completely perpendicular to the plane \(\Bx = \Bx_\perp\), the dot product term with the plane is zero.  To verify this observe

\begin{equation}\label{eqn:rotor:760}
\begin{aligned}
\Bx_\perp \cdot \BB
&= \Bx_\perp \cdot (\Bb_1 \wedge \Bb_2) \\
&= (\Bx_\perp \cdot \Bb_1) \Bb_2 - (\Bx_\perp \cdot \Bb_2) \Bb_1 \\
\end{aligned}
\end{equation}

Each of these dot products are zero since \(\Bx\) has no components that lie
in the plane (those components if they existed could be expressed as linear
combinations of \(\Bb_i\)).

Thus only the component perpendicular to the plane is composed strictly of the
wedge product term

\begin{equation}
\Bx_\perp = \Bx \wedge \BB \inv{\BB}
\end{equation}

And again for a general vector the component that lies out
of the plane as, the rejection of the plane from the vector is

\begin{equation}\label{eqn:rotor:80}
\RejName_{\BB}(\Bx)
= \Bx \wedge \BB \inv{\BB}
= -\inv{\abs{\BB}^2} \Bx \wedge \BB {\BB}
= -\Bx \wedge \hat{\BB} \hat{\BB}
\end{equation}

\section{Rotation around normal to arbitrarily oriented plane through origin}

Having established the preliminaries, we can now express a rotation around
the normal to a plane (with the plane and that normal through the origin).

\imageFigure{../../figures/gabook/rotor}{Rotation of Vector}{fig:rotor}{0.4}

Such a rotation is illustrated in \cref{fig:rotor}
preserves all components of the vector that are perpendicular
to the plane, and operates only on the components parallel to the plane.

Expressed in terms of exponentials and the projective and rejective decompositions above, this is

\begin{equation}\label{eqn:rotor:780}
\begin{aligned}
R_\theta(\Bx)
&= \Bx \wedge \BB \inv{\BB} + \left(\Bx \cdot \BB \inv{\BB}\right)e^{\hat{\BB}\theta} \\
&= \Bx \wedge \BB \inv{\BB} + e^{-\hat{\BB}\theta}\left(\Bx \cdot \BB \inv{\BB}\right) \\
\end{aligned}
\end{equation}

Where we have made explicit note that a plane rotation does not commute with a vector in a plane (its reverse is required).

To demonstrate this write \(i = \Be_2 \Be_1\), a unit bivector in some plane with unit vectors \(\Be_i\) also in the plane.  If a vector
lies in that plane we can write the rotation

\begin{equation}\label{eqn:rotor:800}
\begin{aligned}
\Bx e^{i\theta}
&= \left(a_1\Be_1 + a_2\Be_2\right)\left(\cos\theta + i\sin\theta\right) \\
&= \cos\theta\left(a_1\Be_1 + a_2\Be_2\right) + \left(a_1\Be_1 + a_2\Be_2\right)\left(\Be_2 \Be_1\sin\theta\right) \\
&= \cos\theta\left(a_1\Be_1 + a_2\Be_2\right) + \sin\theta \left(-a_1\Be_2 + a_2\Be_1\right) \\
&= \cos\theta\left(a_1\Be_1 + a_2\Be_2\right) -\Be_2 \Be_1\sin\theta \left(a_1\Be_1 + a_2\Be_2\right) \\
&= e^{-i\theta}\Bx \\
\end{aligned}
\end{equation}

Similarly for a vector that lies outside of the plane we can write

\begin{equation}\label{eqn:rotor:820}
\begin{aligned}
\Bx e^{i\theta}
&= (\sum_{j \ne 1,2} a_j \Be_j)(\cos\theta + \Be_2 \Be_1\sin\theta) \\
&= (\cos\theta + \Be_2 \Be_1\sin\theta) (\sum_{j \ne 1,2} a_j \Be_j) \\
&= e^{i\theta}\Bx
\end{aligned}
\end{equation}

The multivector for a rotation in a plane perpendicular to a vector commutes with that vector.  The properties of the
exponential allow us to factor a rotation

\begin{equation}\label{eqn:rotor:100}
R(\theta) = R(\alpha\theta) R((1-\alpha)\theta)
\end{equation}

where \(\alpha <= 1\), and in particular we can set \(\alpha = 1/2\), and write

\begin{equation}\label{eqn:rotor:840}
\begin{aligned}
R_\theta(\Bx)
&= \Bx \wedge \BB \inv{\BB} + \left(\Bx \cdot \BB \inv{\BB}\right)e^{\hat{\BB}\theta} \\
&= \left(\Bx \wedge \BB \inv{\BB}\right) e^{-\hat{\BB}\theta/2} e^{\hat{\BB}\theta/2}
 + \left(\Bx \cdot \BB \inv{\BB} \right) e^{\hat{\BB}\theta/2} e^{\hat{\BB}\theta/2} \\
&= e^{-\hat{\BB}\theta/2} \left(\Bx \wedge \BB \inv{\BB}\right) e^{\hat{\BB}\theta/2}
+ e^{-\hat{\BB}\theta/2} \left(\Bx \cdot \BB \inv{\BB}\right)e^{\hat{\BB}\theta/2} \\
&= e^{-\hat{\BB}\theta/2} \left(\Bx \wedge \BB + \Bx \cdot \BB\right) \inv{\BB} e^{\hat{\BB}\theta/2} \\
&= e^{-\hat{\BB}\theta/2} \left(\Bx \BB \inv{\BB} \right) e^{\hat{\BB}\theta/2}
\end{aligned}
\end{equation}

This takes us full circle from dot and wedge products back to \(\Bx\), and allows us to express the rotated vector as:

\begin{equation}\label{eqn:rotor:rotor}
R_\theta(\Bx)
= e^{-\hat{\BB}\theta/2} \Bx e^{\hat{\BB}\theta/2}
\end{equation}

Only when the vector lies in the plane (\(\Bx = \Bx_\parallel\), or \(\Bx \wedge \BB = 0\)) can be written using the familiar left or right ``full angle'' rotation exponential that we are used to from complex arithmetic:

\begin{equation}\label{eqn:rotor:120}
R_\theta(\Bx) = e^{-\hat{\BB}\theta} \Bx = \Bx e^{\hat{\BB}\theta}
\end{equation}

\section{Rotor equation in terms of normal to plane}

The rotor equation above is valid for any number of dimensions.  For \R{3} we can alternatively parametrize the plane in terms of
a unit normal \(\Bn\):

\begin{equation}\label{eqn:rotor:140}
\BB = k i\Bn
\end{equation}

Here \(i\) is the \R{3} pseudoscalar \(\Be_1 \Be_2 \Be_3\).

Thus we can write

\begin{equation}\label{eqn:rotor:160}
\hat{\BB} = i\Bn
\end{equation}

and expressing \eqnref{eqn:rotor:rotor} in terms of the unit normal becomes trivial

\begin{equation}
R_\theta(\Bx)
= e^{- i {\Bn}\theta/2} \Bx e^{i{\Bn}\theta/2}
\end{equation}

Expressing this in terms of components and the unit normal is a bit harder

\begin{equation}\label{eqn:rotor:860}
\begin{aligned}
R_\theta(\Bx)
&= \Bx \wedge \BB \inv{\BB} + \left(\Bx \cdot \BB \inv{\BB}\right)e^{\hat{\BB}\theta} \\
&= \Bx \wedge (i\Bn) \inv{i\Bn} + \left(\Bx \cdot (i\Bn) \inv{i\Bn}\right)e^{{i\Bn}\theta} \\
\end{aligned}
\end{equation}

Now,

\begin{equation}\label{eqn:rotor:880}
\begin{aligned}
\Bx \wedge (i\Bn)
&= \inv{2}(\Bx i \Bn + i \Bn \Bx) \\
&= \frac{i}{2}(\Bx \Bn + \Bn \Bx) \\
&= (\Bx \cdot \Bn) i
\end{aligned}
\end{equation}

And

\begin{equation}\label{eqn:rotor:900}
\begin{aligned}
\inv{i\Bn}
&= \inv{i\Bn} \inv{\Bn i} \Bn i \\
&= - i \Bn \\
\end{aligned}
\end{equation}

So the rejective term becomes
\begin{equation}\label{eqn:rotor:920}
\begin{aligned}
\Bx \wedge \BB \inv{\BB}
&= \Bx \wedge (i\Bn) \inv{i\Bn} \\
&= \Bx \wedge (i\Bn) \inv{i\Bn} \\
&= (\Bx \cdot \Bn) i (-i) \Bn \\
&= (\Bx \cdot \Bn) \Bn \\
&= \Proj_{\Bn}(\Bx) \\
\end{aligned}
\end{equation}

Now, for the dot product with the plane term, we have

\begin{equation}\label{eqn:rotor:940}
\begin{aligned}
\Bx \cdot \BB
&= \Bx \cdot (i \Bn) \\
&= \inv{2}(\Bx i \Bn - i \Bn \Bx) \\
&= (\Bx \wedge \Bn)i \\
\end{aligned}
\end{equation}

Putting it all together we have

\begin{equation}\label{eqn:rotor:rotexp}
R_\theta(\Bx)
= (\Bx \cdot \Bn) \Bn + (\Bx \wedge \Bn)\Bn e^{{i\Bn}\theta}
\end{equation}

In terms of explicit sine and cosine terms this is (observe that \((i\Bn)^2 = -1\)),

\begin{equation}\label{eqn:rotor:960}
\begin{aligned}
R_\theta(\Bx)
&= \left(\Bx \cdot \Bn\right) \Bn + \left(\Bx \wedge \Bn\right)\Bn \left(\cos\theta + i\Bn \sin\theta\right) \\
\end{aligned}
\end{equation}

\begin{equation}\label{eqn:rotor:rotnorm}
R_\theta(\Bx) =
\left(\Bx \cdot \Bn\right) \Bn + \left(\Bx \wedge \Bn\right)\Bn \cos\theta + (\Bx \wedge \Bn) i \sin\theta
\end{equation}

\imageFigure{../../figures/gabook/normalRot}{Direction vectors associated with rotation}{fig:normalRot}{0.4}

This triplet of mutually orthogonal direction vectors,
\(\Bn\), \((\Bx \wedge \Bn)\Bn\), and \((\Bx \wedge \Bn) i\)
are illustrated in \cref{fig:normalRot}.  The component of the vector in the direction of the normal
\(\Proj_\Bn(\Bx) = \Bx \cdot \Bn \Bn\) is unaltered by the rotation.
The rotation is applied to the remaining component of \(\Bx\), \(\RejName_{\Bn}(\Bx) = (\Bx \wedge \Bn)\Bn\), and we rotate
in the direction \((\Bx \wedge \Bn) i\)

\subsection{Vector rotation in terms of dot and cross products only}

Expression of this rotation formula \eqnref{eqn:rotor:rotnorm} in terms of ``vector'' relations is also possible, by removing the wedge
products and the pseudoscalar references.

First the rejective term

\begin{equation}\label{eqn:rotor:980}
\begin{aligned}
(\Bx \wedge \Bn) \Bn
&= ((\Bx \cross \Bn) i) \Bn \\
&= ((\Bx \cross \Bn) i) \cdot \Bn \\
&= \inv{2} ( ((\Bx \cross \Bn) i) \Bn - \Bn ((\Bx \cross \Bn) i)) \\
&= \frac{i}{2} ( (\Bx \cross \Bn) \Bn - \Bn (\Bx \cross \Bn) ) \\
&= i ( (\Bx \cross \Bn) \wedge \Bn ) \\
&= i^2 ( (\Bx \cross \Bn) \cross \Bn ) \\
&= \Bn \cross (\Bx \cross \Bn) \\
\end{aligned}
\end{equation}

The next term expressed in terms of the cross product is

\begin{equation}\label{eqn:rotor:1000}
\begin{aligned}
(\Bx \wedge \Bn) i
&=
(\Bx \cross \Bn) i^2 \\
&= \Bn \cross \Bx \\
\end{aligned}
\end{equation}

And putting it all together we have

\begin{equation}\label{eqn:rotor:rotcross}
R_\theta(\Bx) =
\left(\Bx \cdot \Bn\right) \Bn
 + \left(\Bn \cross \Bx\right) \cross \Bn \cos\theta
 + \Bn \cross \Bx \sin\theta
\end{equation}

Compare \eqnref{eqn:rotor:rotcross} to \eqnref{eqn:rotor:rotnorm} and \eqnref{eqn:rotor:rotexp}, and then back to \eqnref{eqn:rotor:rotor}.

\section{Giving a meaning to the sign of the bivector}

For a rotation between two vectors in the plane containing those vectors, we can write the rotation
in terms of the exponential as either a left or right rotation operator:

\begin{equation}\label{eqn:rotor:180}
\Bb = \Ba e^{\Bi\theta} = e^{-\Bi\theta}\Ba
\end{equation}
\begin{equation}\label{eqn:rotor:200}
\Bb = e^{\Bj\theta}\Ba = \Ba e^{-\Bj\theta/2}
\end{equation}

Here both \(\Bi\) and \(\Bj=-\Bi\) are unit bivectors with the property \(\Bi^2 = \Bj^2 = -1\).
Thus in order to write a rotation in exponential form a meaning must be assigned to the sign of the unit bivector that describes the
plane and the orientation of the rotation.

Consider for example the case of a rotation by \(\pi/2\).  For this is the exponential is:

\begin{equation}\label{eqn:rotor:220}
e^{\Bi\pi/2} = \cos(\pi/2) + \Bi \sin(\pi/2) = \Bi
\end{equation}

Thus for perpendicular unit vectors \(\Bu\) and \(\Bv\), if we wish \(\Bi\) to act as a \(\pi/2\) rotation left acting operator on \(\Bu\)
towards \(\Bv\) its value must be:

\begin{equation}\label{eqn:rotor:240}
\Bi = \Bu \wedge \Bv
\end{equation}
\begin{equation}\label{eqn:rotor:260}
\Bu\Bi = \Bu \Bu \wedge \Bv = \Bu\Bu\Bv = \Bv
\end{equation}

For that same rotation if the bivector is employed as a right acting operator, the reverse is required:

\begin{equation}\label{eqn:rotor:280}
\Bj = \Bv \wedge \Bu
\end{equation}
\begin{equation}\label{eqn:rotor:300}
\Bj\Bu = \Bv \wedge \Bu = \Bv\Bu\Bu = \Bv
\end{equation}

\imageFigure{../../figures/gabook/imaginaryorientation}{Orientation of unit imaginary}{fig:imaginaryorientation}{0.4}

In general, for any two vectors, one can find an angle \(\theta\) in the range \(0 \le \theta \le \pi\) between those vectors.
If one lets that angle define the orientation of the rotation between the vectors, and implicitly
define a sort of ``imaginary axis'' for that plane, that imaginary axis will have direction

\begin{equation}\label{eqn:rotor:320}
\inv{\Ba} \Ba \wedge \Bb = \Bb \wedge \Ba \inv {\Ba}.
\end{equation}

This is illustrated in \cref{fig:imaginaryorientation}.

Thus the bivector

\begin{equation}\label{eqn:rotor:340}
\Bi = \frac{\Ba \wedge \Bb}{\abs{\Ba \wedge \Bb}}
\end{equation}

When acting as an operator to the left (\(\Ba \Bi\)) with a vector in the plane can be interpreted as acting as a rotation by \(\pi/2\) towards \(\Bb\).

Similarly the bivector

\begin{equation}\label{eqn:rotor:360}
\Bj = \Bi^\dagger = -\Bi = \frac{\Bb \wedge \Ba}{\abs{\Bb \wedge \Ba}}
\end{equation}

also applied to a vector in the plane produces the same rotation when
acting as an operator to the right.  Thus, in general we can write
a rotation by theta in the plane containing non-colinear vectors \(\Ba\) and \(\Bb\) in the direction of minimal angle
from \(\Ba\) towards \(\Bb\) in one of the three forms:

\begin{equation}\label{eqn:rotor:380}
R_{\theta : \Ba \rightarrow \Bb}(\Ba)
= \Ba e^{ \frac{\Ba \wedge \Bb}{\abs{\Ba \wedge \Bb}} \theta }
= e^{ \frac{\Bb \wedge \Ba}{\abs{\Bb \wedge \Ba}} \theta } \Ba
\end{equation}

Or,
\begin{equation}\label{eqn:rotor:400}
R_{\theta : \Ba \rightarrow \Bb}(\Bx)
= e^{ \frac{\Bb \wedge \Ba}{\abs{\Bb \wedge \Ba}} \theta/2 } \Bx e^{ \frac{\Ba \wedge \Bb}{\abs{\Ba \wedge \Bb}} \theta/2 }
\end{equation}

This last (writing \(\Bx\) instead of \(\Ba\) since it also applies to vectors that lie outside of the \(\Ba \wedge \Bb\) plane),
is our rotor formula \eqnref{eqn:rotor:rotor}, reexpressed in a way that removes the sign ambiguity of the bivector \(\Bi\) in that equation.

\section{Rotation between two unit vectors}

\imageFigure{../../figures/gabook/parallelogramvec}{Sum of unit vectors bisects angle between}{fig:parallelogramvec}{0.4}

As illustrated in \cref{fig:parallelogramvec}, when the angle between two vectors is less than \(\pi\)
the fact that the sum of two arbitrarily oriented unit vectors bisects those vectors provides a convenient
way to compute the half angle rotation exponential.

Thus we can write

\begin{equation*}
\frac{\Ba + \Bb}{\abs{\Ba + \Bb}} = \Ba e^{\Bi\theta/2} = e^{\Bj\theta/2} \Ba
\end{equation*}

Where \(\Bi = \Bj^\dagger\) are unit bivectors of appropriate sign.  Multiplication through by \(\Ba\) gives

\begin{equation*}
e^{\Bi\theta/2} =
\frac{1 + \Ba\Bb}{\abs{\Ba + \Bb}}
\end{equation*}

Or,
\begin{equation*}
e^{\Bj\theta/2} =
\frac{1 + \Bb\Ba}{\abs{\Ba + \Bb}}
\end{equation*}

Thus we can write the total rotation from \(\Ba\) to \(\Bb\) as

\begin{equation*}
\Bb
= e^{-\Bi\theta/2} \Ba e^{\Bi\theta/2}
= e^{\Bj\theta/2} \Ba e^{-\Bj\theta/2}
= \left(\frac{1 + \Bb\Ba}{\abs{\Ba + \Bb}}\right) \Ba \left(\frac{1 + \Ba\Bb}{\abs{\Ba + \Bb}}\right)
\end{equation*}

For the case where the rotation is through an angle \(\theta\) where \(\pi < \theta < 2\pi\), again employing a left acting
exponential operator we have

\begin{equation}\label{eqn:rotor:1020}
\begin{aligned}
\frac{\Ba + \Bb}{\abs{\Ba + \Bb}}
&= \Bb e^{\Bi(2\pi - \theta)/2} \\
&= \Bb e^{\Bi \pi} e^{- \Bi\theta/2} \\
&= -\Bb e^{- \Bi\theta/2} \\
\end{aligned}
\end{equation}

Or,
\begin{equation}\label{eqn:rotor:420}
e^{- \Bi\theta/2} = -\frac{\Bb\Ba + 1}{\abs{\Ba + \Bb}}
\end{equation}

Thus

\begin{equation}\label{eqn:rotor:rotunit}
\Bb = e^{- \Bi\theta/2} \Ba e^{ \Bi\theta/2} =
\left(-\frac{1 + \Bb\Ba}{\abs{\Ba + \Bb}}\right) \Ba \left(-\frac{1 + \Ba\Bb}{\abs{\Ba + \Bb}}\right)
\end{equation}

Note that the two negatives cancel, giving the same result as in the \(\theta < \pi\) case.  Thus \eqnref{eqn:rotor:rotunit} is valid for all vectors \(\Ba \ne -\Bb\) (this can be verified by direct multiplication.)

These
half angle exponentials are called rotors, writing the rotor as

\begin{equation}\label{eqn:rotor:440}
R = \frac{1 + \Ba\Bb}{\abs{\Ba + \Bb}}
\end{equation}

and the rotation in terms of rotors is:

\begin{equation}\label{eqn:rotor:460}
\Bb = R^\dagger \Ba R
\end{equation}

The angle associated with this rotor \(R\) is the minimal angle between the two vectors (\(0 < \theta < \pi\)), and is directed from \(\Ba\) to \(\Bb\).  Inverting the rotor will not change the net effect of the rotation, but has the geometric meaning that the rotation from \(\Ba\) to \(\Bb\)
rotates in the opposite direction through the larger angle (\(\pi < \theta < 2\pi\)) between the vectors.

\section{Eigenvalues, vectors and coordinate vector and matrix of the rotation linear transformation}

Given the plane containing two orthogonal vectors \(\Bu\) and \(\Bv\), we can form a unit bivector for the plane

\begin{equation}\label{eqn:rotor:480}
\BB = \Bu\Bv
\end{equation}

A normal to this plane is \(\Bn = \Bv\Bu I\).

The rotation operator for a rotation around \(\Bn\) in that plane (directed from \(\Bu\) towards \(\Bv\)) is

\begin{equation}\label{eqn:rotor:500}
R_\theta(\Bx) = e^{\Bv\Bu \theta/2} \Bx e^{\Bu\Bv \theta/2}
\end{equation}

To form the matrix of this linear transformation assume an orthonormal basis \(\sigma = \{ \Be_i \}\).

In terms of these basis vectors we can write

\begin{equation}\label{eqn:rotor:520}
R_\theta(\Be_j) =
e^{-\Bv\Bu \theta/2} \Be_j e^{\Bu\Bv \theta/2}
=
\sum_i \left(e^{-\Bv\Bu \theta/2} \Be_j e^{\Bu\Bv \theta/2}\right) \cdot \Be_i \Be_i
\end{equation}

Thus the coordinate vector for this basis is

\begin{equation}\label{eqn:rotor:540}
{
\begin{bmatrix}
R_\theta(\Be_j)
\end{bmatrix}
}_\sigma
=
\begin{bmatrix}
\left(e^{-\Bv\Bu \theta/2} \Be_j e^{\Bu\Bv \theta/2}\right) \cdot \Be_1 \\
\vdots \\
\left(e^{-\Bv\Bu \theta/2} \Be_j e^{\Bu\Bv \theta/2}\right) \cdot \Be_n \\
\end{bmatrix}
\end{equation}

We can use this to form the matrix for the linear operator that takes coordinate vectors from
the basis \(\sigma\) to \(\sigma\):

\begin{equation}\label{eqn:rotor:560}
{
\begin{bmatrix}
R_\theta(\Bx)
\end{bmatrix}
}_\sigma
=
{
\begin{bmatrix}
R_\theta
\end{bmatrix}
}_\sigma^\sigma
{
\begin{bmatrix}
\Bx
\end{bmatrix}
}_\sigma
\end{equation}

Where
\begin{equation}\label{eqn:rotor:rotcoords}
{
\begin{bmatrix}
R_\theta
\end{bmatrix}
}_\sigma^\sigma
=
\begin{bmatrix}
{
\begin{bmatrix}
R_\theta(\Be_1)
\end{bmatrix}
}_\sigma
\hdots
{
\begin{bmatrix}
R_\theta(\Be_n)
\end{bmatrix}
}_\sigma
\end{bmatrix}
=
{
\begin{bmatrix}
\left(e^{-\Bv\Bu \theta/2} \Be_j e^{\Bu\Bv \theta/2}\right) \cdot \Be_i \\
\end{bmatrix}
}_{ij}
\end{equation}

If one uses the plane and its normal to form an alternate orthonormal basis
\(\alpha = \{\Bu, \Bv, \Bn\}\).

The transformation matrix for coordinate vectors in this basis is

\begin{equation}\label{eqn:rotor:580}
{
\begin{bmatrix}
R_\theta
\end{bmatrix}
}_\alpha^\alpha
=
\begin{bmatrix}
\left(\Bu e^{\Bu\Bv \theta}\right) \cdot \Bu & \left(\Bv e^{\Bu\Bv \theta}\right) \cdot \Bu & 0 \\
\left(\Bu e^{\Bu\Bv \theta}\right) \cdot \Bv & \left(\Bv e^{\Bu\Bv \theta}\right) \cdot \Bv & 0 \\
0 & 0 & \Bn\cdot\Bn \\
\end{bmatrix}
=
\begin{bmatrix}
\cos\theta & -\sin\theta & 0 \\
\sin\theta & \cos\theta & 0 \\
0 & 0 & 1 \\
\end{bmatrix}
\end{equation}

This matrix has eigenvalues \(e^{i\theta}, e^{-i\theta}, 1\), with (coordinate) eigenvectors

\begin{equation}\label{eqn:rotor:600}
\inv{\sqrt{2}}
\begin{bmatrix}
1 \\
-i \\
0 \\
\end{bmatrix},
\inv{\sqrt{2}}
\begin{bmatrix}
1 \\
i \\
0 \\
\end{bmatrix},
\begin{bmatrix}
0 \\
0 \\
1 \\
\end{bmatrix}
\end{equation}

Its interesting to observe that without introducing coordinate vectors an eigensolution is possible directly from
the linear transformation itself.

The rotation linear operator has right and left eigenvalues \(e^{\Bu\Bv \theta}\), \(e^{\Bv\Bu \theta}\) (respectively), where the eigenvectors for these are any vectors in the plane.  There is also a scalar eigenvalue \(1\) (both left and right eigenvalue), for the eigenvector \(\Bn\):

\begin{equation}\label{eqn:rotor:1040}
\begin{aligned}
R_\theta(\Bu) &= e^{\Bv \Bu \theta} \Bx = \Bx e^{\Bu \Bv \theta} \\
R_\theta(\Bu) &= e^{\Bv \Bu \theta} \Bx = \Bx e^{\Bu \Bv \theta} \\
R_\theta(\Bn) &= \Bn (1) \\
\end{aligned}
\end{equation}

Observe that the eigenvalues here are not all scalars, which is likely related
to the fact that the coordinate matrix was not diagonalizable with real vectors.

the matrix of the linear transformation.
Given this, one can write:

\begin{equation}\label{eqn:rotor:1060}
\begin{aligned}
\begin{bmatrix}
R_\theta(\Bu) & R_\theta(\Bv) & R_\theta(\Bn) \\
\end{bmatrix}
&=
\begin{bmatrix}
\Bu & \Bv & \Bn \\
\end{bmatrix}
\begin{bmatrix}
e^{\Bu \Bv \theta} & 0 & 0 \\
0 & e^{\Bu \Bv \theta} & 0 \\
0 & 0 & 1 \\
\end{bmatrix} \\
&=
\begin{bmatrix}
e^{\Bv \Bu \theta} & 0 & 0 \\
0 & e^{\Bv \Bu \theta} & 0 \\
0 & 0 & 1 \\
\end{bmatrix}
\begin{bmatrix}
\Bu & \Bv & \Bn \\
\end{bmatrix}
\end{aligned}
\end{equation}

But neither of these can be used to diagonalize the matrix of the transformation.  To do that
we require dot products that span the matrix product to form the coordinate vector columns.

Observe that interestingly
enough the left and right eigenvalues of the operator in the plane are of complex exponential form (\(e^{\pm \Bn I \theta}\)) just as the eigenvalues for
coordinate vectors restricted to the plane are complex exponentials (\(e^{\pm i\theta}\)).
%This suggests that a basis for a quaternion
%like space (0-2 multivectors) will be required to diagonalize a rotation operator.

\section{matrix for rotation linear transformation}

Let us expand the terms in \eqnref{eqn:rotor:rotcoords} to calculate explicitly the rotation matrix for an arbitrary
rotation.  Also, as before, write \(\Bn = \Bv\Bu I\), and parametrize the Rotor as follows:

\begin{equation}\label{eqn:rotor:620}
R = e^{\Bn I \theta/2} = \cos{\theta/2} + \Bn I \sin{\theta/2} = \alpha + I\Bbeta
\end{equation}

Thus the \(ij\) terms in the matrix are:

\begin{equation}\label{eqn:rotor:1080}
\begin{aligned}
\Be_i \cdot \left(e^{-\Bn I \theta/2} \Be_j e^{\Bn I \theta/2}\right)
&= \langle{ \Be_i (\alpha -I\Bbeta) \Be_j (\alpha +I\Bbeta) } \rangle \\
&= \langle{ \Be_i (\Be_j \alpha -I\Bbeta\Be_j) (\alpha +I\Bbeta) } \rangle \\
&= \langle{ \Be_i \left( \Be_j \alpha^2 -I\alpha(\Bbeta\Be_j - \Be_j\Bbeta) + \Bbeta\Be_j\Bbeta \right) } \rangle \\
&= \delta_{ij}\alpha^2 + \langle{ \Be_i \left( -2I\alpha(\Bbeta \wedge \Be_j) + \Bbeta\Be_j\Bbeta \right) } \rangle \\
&= \delta_{ij}\alpha^2 + 2\alpha \Be_i \cdot (\Bbeta \cross \Be_j) + \langle{ \Be_i \Bbeta \Be_j \Bbeta } \rangle \\
\end{aligned}
\end{equation}

Lets expand the last term separately:
\begin{equation}\label{eqn:rotor:1100}
\begin{aligned}
\langle{ \Be_i \Bbeta \Be_j \Bbeta } \rangle
&= \langle{ ( \Be_i \cdot \Bbeta + \Be_i \wedge \Bbeta) ( \Be_j \cdot \Bbeta + \Be_j \wedge \Bbeta) } \rangle  \\
&= (\Be_i \cdot \Bbeta)(\Be_j \cdot \Bbeta) + \langle{ (\Be_i \wedge \Bbeta) ( \Be_j \wedge \Bbeta) } \rangle  \\
\end{aligned}
\end{equation}

And once more considering first the \(i=j\) case (writing \(s \ne i \ne t\)).

\begin{equation}\label{eqn:rotor:1120}
\begin{aligned}
\langle{ (\Be_i \wedge \Bbeta)^2 }\rangle
&= \lr{ \sum_{k \ne i}{ \Be_{ik} \beta_k} }^2 \\
&= ( \Be_{is} \beta_s + \Be_{it} \beta_t ) ( \Be_{is} \beta_s + \Be_{it} \beta_t ) \\
&= -\beta_s^2 -\beta_t^2 -  \Be_{st} \beta_s \beta_t + \Be_{ts} \beta_t \beta_s  \\
&= -\beta_s^2 -\beta_t^2 \\
&= -\Bbeta^2 + \beta_i^2 \\
\end{aligned}
\end{equation}

For the \(i \ne j\) term, writing \(i \ne j \ne k\)
\begin{equation}\label{eqn:rotor:1140}
\begin{aligned}
\langle{(\Be_i \wedge \Bbeta) (\Be_j \wedge \Bbeta)}\rangle
&= \langle{\sum_{s \ne i} \Be_{is} \beta_s\sum_{t \ne i} \Be_{it} \beta_t}\rangle \\
&= \langle{( \Be_{ij} \beta_j + \Be_{ik} \beta_k) ( \Be_{ji} \beta_i + \Be_{jk} \beta_k)}\rangle \\
&= \beta_i\beta_j + \langle{ \Be_{ji} \beta_k^2 +\Be_{ik} \beta_j \beta_k +\Be_{kj} \beta_k \beta_i }\rangle \\
&= \beta_i\beta_j \\
\end{aligned}
\end{equation}

Thus
\begin{equation}\label{eqn:rotor:640}
\langle{ (\Be_i \wedge \Bbeta) ( \Be_j \wedge \Bbeta) } \rangle
= \delta_{ij}(-\Bbeta^2 + \beta_i^2) + (1-\delta_{ij})\beta_i\beta_j
= \beta_i\beta_j -\delta_{ij}\Bbeta^2
\end{equation}

And putting it all back together
\begin{equation}\label{eqn:rotor:rotmgreek}
\Be_i \cdot \left(e^{-\Bn I \theta/2} \Be_j e^{\Bn I \theta/2}\right)
= \delta_{ij}(\alpha^2 -\Bbeta^2) + 2\alpha \Be_i \cdot (\Bbeta \cross \Be_j) + 2\beta_i\beta_j
\end{equation}


The \(\alpha\) and \(\beta\) terms can be expanded in terms of \(\theta\).
we see that The \(\delta_{ij}\) coefficient is

\begin{equation}\label{eqn:rotor:660}
\alpha^2 -\Bbeta^2 = 2{\cos}^2{\theta} -1 = \cos\theta.
\end{equation}

The triple product \(\Be_i \cdot (\Bbeta \cross \Be_j)\) is zero along the diagonal where \(i=j\) since an \(\Be_j=\Be_i\) cross has no \(\Be_i\) component, so
for \(k \ne i \ne j\), the triple product term is

\begin{equation}\label{eqn:rotor:1160}
\begin{aligned}
2\alpha \Be_i \cdot (\Bbeta \cross \Be_j)
&= 2\alpha \beta_k \Be_i \cdot (\Be_k \cross \Be_j) \\
&= 2\alpha \beta_k \Sgn{\pi_{ikj}} \\
&= 2 n_k \cos({\theta/2})\sin({\theta/2}) \Sgn{\pi_{ikj}} \\
&= n_k \sin{\theta} \Sgn{\pi_{ikj}} \\
\end{aligned}
\end{equation}

The last term is:
\begin{equation}\label{eqn:rotor:680}
2\beta_i\beta_j
= 2 n_i n_j {\sin}^2({\theta/2})
= n_i n_j (1-\cos\theta)
\end{equation}

Thus we can alternatively write \eqnref{eqn:rotor:rotmgreek}

\begin{equation}\label{eqn:rotor:rotmn}
\Be_i \cdot \left(e^{-\Bn I \theta/2} \Be_j e^{\Bn I \theta/2}\right)
= \delta_{ij}\cos\theta
+ n_k \sin{\theta} \epsilon_{ikj} + n_i n_j (1-\cos\theta)
\end{equation}

This is enough to easily and explicitly write out the complete rotation matrix for a rotation about unit vector \(\Bn = (n_1, n_2, n_3)\):
(with basis \(\sigma = \{\Be_i\}\)):

\begin{equation}\label{eqn:rotor:700}
[
R_\theta
]_\sigma^\sigma
=
\begin{bmatrix}
\cos\theta(1 -n_1^2) + n_1^2 & n_1 n_2 (1-\cos\theta) - n_3 \sin\theta & n_1 n_3 (1-\cos\theta) + n_2 \sin\theta \\
n_1 n_2 (1-\cos\theta) + n_3 \sin\theta & \cos\theta(1 -n_2^2) + n_2^2 & n_2 n_3 (1-\cos\theta) - n_1 \sin\theta \\
n_1 n_3 (1-\cos\theta) - n_2 \sin\theta & n_2 n_3 (1-\cos\theta) + n_1 \sin\theta & \cos\theta(1 -n_3^2) + n_3^2 \\
\end{bmatrix}
\end{equation}

Note also that the \(n_i\) terms are the direction cosines of the unit normal for the rotation, so all the terms above
are really strictly sums of sine and cosine products, so we have the rotation matrix completely described in terms of four
angles.  Also observe how much additional complexity we have to express a rotation in terms of the matrix.  This representation also
does not work for plane rotations, just vectors (whereas that is not the case for the rotor form).

It is actually somewhat simpler looking to leave things in terms of the \(\alpha\), and \(\beta\) parameters.  We can rewrite
\eqnref{eqn:rotor:rotmgreek} as:

\begin{equation}
\Be_i \cdot \left(e^{-\Bn I \theta/2} \Be_j e^{\Bn I \theta/2}\right)
= \delta_{ij}(2\alpha^2 -1)
+2\alpha \beta_k \epsilon_{ikj} + 2\beta_i\beta_j
\end{equation}

and the rotation matrix:

\begin{equation}\label{eqn:rotor:720}
[
R_\theta
]_\sigma^\sigma
=
2
\begin{bmatrix}
\alpha^2 -\frac{1}{2} + \beta_1^2 & \beta_1 \beta_2  - \beta_3 \alpha & \beta_1 \beta_3  + \beta_2 \alpha \\
\beta_1 \beta_2  + \beta_3 \alpha & \alpha^2 -\frac{1}{2} + \beta_2^2 & \beta_2 \beta_3  - \beta_1 \alpha \\
\beta_1 \beta_3  - \beta_2 \alpha & \beta_2 \beta_3  + \beta_1 \alpha & \alpha^2 -\frac{1}{2} + \beta_3^2 \\
\end{bmatrix}
\end{equation}

Not really that much simpler, but a bit.  The trade off is that the similarity to the standard \(2x2\) rotation matrix is not obvious.


%
% Copyright � 2012 Peeter Joot.  All Rights Reserved.
% Licenced as described in the file LICENSE under the root directory of this GIT repository.
%

%
%
\chapter{Quaternions}
\index{quaternion}
\label{chap:quaternion}
%\date{Feb 2, 2008.  quaternion.tex}

Like complex numbers, quaternions may be written as a multivector with scalar and bivector components (a 0,2-multivector).

\begin{equation}\label{eqn:quaternion:20}
q = \alpha + \mathbf{B}
\end{equation}

Where the complex number has one bivector component, and the quaternions have three.

One can describe quaternions as 0,2-multivectors where the basis for the bivector part is left handed.  There is not really anything special about quaternion multiplication, or complex number multiplication, for that matter.  Both are just a specific examples of a 0,2-multivector multiplication.  Other quaternion operations can also be found to have natural multivector equivalents.  The most important of which is likely the quaternion conjugate, since it implies the norm and the inverse.  As a multivector, like complex numbers, the conjugate operation is reversal:

\begin{equation}\label{eqn:quaternion:40}
\overline{q} = q^\dagger = \alpha - \mathbf{B}
\end{equation}

Thus \(\abs{q}^2 = q\overline{q} = \alpha^2 - \mathbf{B}^2\).  Note that this norm is a positive definite as expected since a bivector square is negative.

To be more specific about the left handed basis property of quaternions one can note that the quaternion bivector basis is usually defined in terms of the following properties

\begin{equation}\label{eqn:quaternion:60}
\mathbf{i}^2 = \mathbf{j}^2 = \mathbf{k}^2 = -1
\end{equation}
\begin{equation}\label{eqn:quaternion:80}
\mathbf{i}\mathbf{j} = -\mathbf{j}\mathbf{i}, \mathbf{i}\mathbf{k} = -\mathbf{k}\mathbf{i}, \mathbf{j}\mathbf{k} = -\mathbf{k}\mathbf{j}
\end{equation}
\begin{equation}\label{eqn:quaternion:100}
\mathbf{i}\mathbf{j} = \mathbf{k}
\end{equation}

The first two properties are satisfied by any set of orthogonal unit bivectors for the space.  The last property, which could also be written \(\mathbf{i}\mathbf{j}\mathbf{k} = -1\), amounts to a choice for the orientation of this bivector basis of the 2-vector part of the quaternion.

As an example suppose one picks

\begin{equation}\label{eqn:quaternion:120}
\mathbf{i} = \mathbf{e}_2\mathbf{e}_3
\end{equation}
\begin{equation}\label{eqn:quaternion:140}
\mathbf{j} = \mathbf{e}_3\mathbf{e}_1
\end{equation}

Then the third bivector required to complete the basis set subject to the properties above is

\begin{equation}\label{eqn:quaternion:160}
\mathbf{i}\mathbf{j} = \mathbf{e}_2\mathbf{e}_1 = \mathbf{k}
\end{equation}.

Suppose that, instead of the above, one picked a slightly more natural bivector basis, the duals of the unit vectors obtained by multiplication with the pseudoscalar (\(\mathbf{e}_1\mathbf{e}_2\mathbf{e}_3\mathbf{e}_i\)).  These bivectors are

\begin{equation}\label{eqn:quaternion:180}
\mathbf{i}=\mathbf{e}_2\mathbf{e}_3, \mathbf{j}=\mathbf{e}_3\mathbf{e}_1, \mathbf{k}=\mathbf{e}_1\mathbf{e}_2
\end{equation}.

A 0,2-multivector with this as the basis for the bivector part would have properties similar to the standard quaternions (anti-commutative unit quaternions, negation for unit quaternion square, same conjugate, norm and inversion operations, ...), however the triple product would have the value \(\mathbf{i}\mathbf{j}\mathbf{k} = 1\), instead of \(-1\).

\section{quaternion as generator of dot and cross product}

The product of pure quaternions is noted as being a generator of dot and cross products.  This is also true
of a vector bivector product.

Writing a vector \(\Bx\) as

\begin{equation}\label{eqn:quaternion:200}
\Bx = \sum_i x_i \Be_i = x_1 \Be_1 + x_2 \Be_2 + x_3 \Be_3
\end{equation}

And a bivector \(\BB\) (where for short, \(\Be_{ij} = \Be_i \Be_j = \Be_i \wedge \Be_j\)) as:

\begin{equation}\label{eqn:quaternion:220}
\BB = \sum_i b_i \Be_i I = b_1 \Be_{23} + b_2 \Be_{31} + b_3 \Be_{12}
\end{equation}

The product of these two is
\begin{equation}\label{eqn:quaternion:280}
\begin{aligned}
\Bx \BB
&= (x_1 \Be_1 + x_2 \Be_2 + x_3 \Be_3)(b_1 \Be_{23} + b_2 \Be_{31} + b_3 \Be_{12}) \\
&= (x_3 b_2 - x_2 b_3) \Be_1 + (x_1 b_3 - x_3 b_1) \Be_2 + (x_2 b_1 - x_1 b_2) \Be_3 \\
&+ (x_1 b_1 + x_2 b_2 + x_3 b_3) \Be_{123} \\
\end{aligned}
\end{equation}

Looking at the vector and trivector components of this we recognize the dot product and negated cross product
immediately (as with multiplication of pure quaternions).

Those products are, in fact, \(\Bx \cdot \BB\) and \(\Bx \wedge \BB\) respectively.

Introducing a vector and bivector basis \(\alpha = \{ \Be_i \}\), and \(\beta = \{ \Be_i I \}\), we can
express the dot product and cross product of the associated coordinate vectors
in terms of vector bivectors products as follows:

\begin{equation}\label{eqn:quaternion:240}
[\Bx]_\alpha \cdot [\BB]_\beta = \frac{\BB \wedge \Bx}{I}
\end{equation}
\begin{equation}\label{eqn:quaternion:260}
[\Bx]_\alpha \cross [\BB]_\beta = [\BB \cdot \Bx]_\alpha
\end{equation}


\include{multivector_taylors}

%\begin{thebibliography}{99}
%  \addcontentsline{toc}{chapter}{Bibliography}
%\bibitem{lamport} L. Lamport. {\bf \LaTeX \ A Document Preparation System}
%Addison-Wesley, California 1986.
%
%\end{thebibliography}

\bibliographystyle{plainnat}
  \addcontentsline{toc}{chapter}{Bibliography}
\bibliography{myrefs}

\documentclass[openany]{memoir}
\usepackage[]{makeidx}

\chapterstyle{ell}

\makeindex

\begin{document}

To solve various problems in physics, it can be advantageous
to express any arbitrary piecewise-smooth function as a Fourier Series
\index{Fourier Series}
composed of multiples of sine \index{sine} and cosine \index{cosine} functions.  These are used in \cite{acheson1990elementary}.

\printindex

\bibliography{myrefs}
\bibliographystyle{unsrturl}

\end{document}

  \addcontentsline{toc}{chapter}{Index}
\end{document}
