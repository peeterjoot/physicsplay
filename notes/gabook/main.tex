\documentclass[12pt,leqno]{book}

\usepackage{amsmath,amssymb,amsfonts} % Typical maths resource packages
\usepackage{graphicx}
\usepackage{color}                   % For creating coloured text and background
\usepackage{txfonts} 
\usepackage{listings}
\usepackage[bookmarks=true,plainpages=false]{hyperref}

\parindent 1cm
\parskip 0.2cm
\topmargin 0.2cm
\oddsidemargin 1cm
\evensidemargin 0.5cm
\textwidth 15cm
\textheight 21cm

% how do these ones work?
\newtheorem{theorem}{Theorem}[section]
\newtheorem{proposition}[theorem]{Proposition}
\newtheorem{corollary}[theorem]{Corollary}
\newtheorem{lemma}[theorem]{Lemma}
\newtheorem{remark}[theorem]{Remark}
\newtheorem{definition}[theorem]{Definition}

\usepackage{amsmath}
\usepackage{mathpazo}

%
% shorthand for bold symbols, convenient for vectors and matrices
%
\newcommand{\Ba}[0]{\mathbf{a}}
\newcommand{\Bb}[0]{\mathbf{b}}
\newcommand{\Bc}[0]{\mathbf{c}}
\newcommand{\Bd}[0]{\mathbf{d}}
\newcommand{\Be}[0]{\mathbf{e}}
\newcommand{\Bf}[0]{\mathbf{f}}
\newcommand{\Bg}[0]{\mathbf{g}}
\newcommand{\Bh}[0]{\mathbf{h}}
\newcommand{\Bi}[0]{\mathbf{i}}
\newcommand{\Bj}[0]{\mathbf{j}}
\newcommand{\Bk}[0]{\mathbf{k}}
\newcommand{\Bl}[0]{\mathbf{l}}
\newcommand{\Bm}[0]{\mathbf{m}}
\newcommand{\Bn}[0]{\mathbf{n}}
\newcommand{\Bo}[0]{\mathbf{o}}
\newcommand{\Bp}[0]{\mathbf{p}}
\newcommand{\Bq}[0]{\mathbf{q}}
\newcommand{\Br}[0]{\mathbf{r}}
\newcommand{\Bs}[0]{\mathbf{s}}
\newcommand{\Bt}[0]{\mathbf{t}}
\newcommand{\Bu}[0]{\mathbf{u}}
\newcommand{\Bv}[0]{\mathbf{v}}
\newcommand{\Bw}[0]{\mathbf{w}}
\newcommand{\Bx}[0]{\mathbf{x}}
\newcommand{\By}[0]{\mathbf{y}}
\newcommand{\Bz}[0]{\mathbf{z}}
\newcommand{\BA}[0]{\mathbf{A}}
\newcommand{\BB}[0]{\mathbf{B}}
\newcommand{\BC}[0]{\mathbf{C}}
\newcommand{\BD}[0]{\mathbf{D}}
\newcommand{\BE}[0]{\mathbf{E}}
\newcommand{\BF}[0]{\mathbf{F}}
\newcommand{\BG}[0]{\mathbf{G}}
\newcommand{\BH}[0]{\mathbf{H}}
\newcommand{\BI}[0]{\mathbf{I}}
\newcommand{\BJ}[0]{\mathbf{J}}
\newcommand{\BK}[0]{\mathbf{K}}
\newcommand{\BL}[0]{\mathbf{L}}
\newcommand{\BM}[0]{\mathbf{M}}
\newcommand{\BN}[0]{\mathbf{N}}
\newcommand{\BO}[0]{\mathbf{O}}
\newcommand{\BP}[0]{\mathbf{P}}
\newcommand{\BQ}[0]{\mathbf{Q}}
\newcommand{\BR}[0]{\mathbf{R}}
\newcommand{\BS}[0]{\mathbf{S}}
\newcommand{\BT}[0]{\mathbf{T}}
\newcommand{\BU}[0]{\mathbf{U}}
\newcommand{\BV}[0]{\mathbf{V}}
\newcommand{\BW}[0]{\mathbf{W}}
\newcommand{\BX}[0]{\mathbf{X}}
\newcommand{\BY}[0]{\mathbf{Y}}
\newcommand{\BZ}[0]{\mathbf{Z}}

\newcommand{\Bzero}[0]{\mathbf{0}}
\newcommand{\Btheta}[0]{\boldsymbol{\theta}}
\newcommand{\Btau}[0]{\boldsymbol{\tau}}
\newcommand{\Bomega}[0]{\boldsymbol{\omega}}

%
% shorthand for unit vectors
%
\newcommand{\acap}[0]{\hat{\Ba}}
\newcommand{\bcap}[0]{\hat{\Bb}}
\newcommand{\ccap}[0]{\hat{\Bc}}
\newcommand{\dcap}[0]{\hat{\Bd}}
\newcommand{\ecap}[0]{\hat{\Be}}
\newcommand{\fcap}[0]{\hat{\Bf}}
\newcommand{\gcap}[0]{\hat{\Bg}}
\newcommand{\hcap}[0]{\hat{\Bh}}
\newcommand{\icap}[0]{\hat{\Bi}}
\newcommand{\jcap}[0]{\hat{\Bj}}
\newcommand{\kcap}[0]{\hat{\Bk}}
\newcommand{\lcap}[0]{\hat{\Bl}}
\newcommand{\mcap}[0]{\hat{\Bm}}
\newcommand{\ncap}[0]{\hat{\Bn}}
\newcommand{\ocap}[0]{\hat{\Bo}}
\newcommand{\pcap}[0]{\hat{\Bp}}
\newcommand{\qcap}[0]{\hat{\Bq}}
\newcommand{\rcap}[0]{\hat{\Br}}
\newcommand{\scap}[0]{\hat{\Bs}}
\newcommand{\tcap}[0]{\hat{\Bt}}
\newcommand{\ucap}[0]{\hat{\Bu}}
\newcommand{\vcap}[0]{\hat{\Bv}}
\newcommand{\wcap}[0]{\hat{\Bw}}
\newcommand{\xcap}[0]{\hat{\Bx}}
\newcommand{\ycap}[0]{\hat{\By}}
\newcommand{\zcap}[0]{\hat{\Bz}}
\newcommand{\thetacap}[0]{\hat{\Btheta}}

%
% to write R^n and C^n in a distinguishable fashion.  Perhaps change this
% to the double lined characters upon figuring out how to do so.
%
\newcommand{\C}[1]{$\mathbb{C}^{#1}$}
\newcommand{\R}[1]{$\mathbb{R}^{#1}$}

%
% various generally useful helpers
%

% derivative of #1 wrt. #2:
\newcommand{\D}[2] {\frac {d#2} {d#1}}

\newcommand{\inv}[1]{\frac{1}{#1}}
\newcommand{\cross}[0]{\times}

\newcommand{\abs}[1]{\lvert{#1}\rvert}
\newcommand{\norm}[1]{\lVert{#1}\rVert}
\newcommand{\innerprod}[2]{\langle{#1}, {#2}\rangle}
\newcommand{\dotprod}[2]{{#1} \cdot {#2}}
\newcommand{\bdotprod}[2]{\left({#1} \cdot {#2}\right)}
\newcommand{\crossprod}[2]{{#1} \cross {#2}}
\newcommand{\tripleprod}[3]{\dotprod{\left(\crossprod{#1}{#2}\right)}{#3}}

\DeclareMathOperator{\Proj}{Proj}
\DeclareMathOperator{\Span}{span}
\DeclareMathOperator{\Sgn}{sgn}
\DeclareMathOperator{\Area}{Area}
\DeclareMathOperator{\Volume}{Volume}

%
% A few miscellaneous things specific to this document
%
\newcommand{\crossop}[1]{\crossprod{#1}{}}

% R2 vector.
\newcommand{\VectorTwo}[2]{
\begin{bmatrix}
 {#1} \\
 {#2}
\end{bmatrix}
}

\newcommand{\VectorN}[1]{
\begin{bmatrix}
{#1}_1 \\
{#1}_2 \\
\vdots \\
{#1}_N \\
\end{bmatrix}
}

\newcommand{\DETuvij}[4]{
\begin{vmatrix}
 {#1}_{#3} & {#1}_{#4} \\
 {#2}_{#3} & {#2}_{#4}
\end{vmatrix}
}

\newcommand{\DETuvwijk}[6]{
\begin{vmatrix}
 {#1}_{#4} & {#1}_{#5} & {#1}_{#6} \\
 {#2}_{#4} & {#2}_{#5} & {#2}_{#6} \\
 {#3}_{#4} & {#3}_{#5} & {#3}_{#6}
\end{vmatrix}
}

\newcommand{\DETuvwxijkl}[8]{
\begin{vmatrix}
 {#1}_{#5} & {#1}_{#6} & {#1}_{#7} & {#1}_{#8} \\
 {#2}_{#5} & {#2}_{#6} & {#2}_{#7} & {#2}_{#8} \\
 {#3}_{#5} & {#3}_{#6} & {#3}_{#7} & {#3}_{#8} \\
 {#4}_{#5} & {#4}_{#6} & {#4}_{#7} & {#4}_{#8} \\
\end{vmatrix}
}

%\newcommand{\DETuvwxyijklm}[10]{
%\begin{vmatrix}
% {#1}_{#6} & {#1}_{#7} & {#1}_{#8} & {#1}_{#9} & {#1}_{#10} \\
% {#2}_{#6} & {#2}_{#7} & {#2}_{#8} & {#2}_{#9} & {#2}_{#10} \\
% {#3}_{#6} & {#3}_{#7} & {#3}_{#8} & {#3}_{#9} & {#3}_{#10} \\
% {#4}_{#6} & {#4}_{#7} & {#4}_{#8} & {#4}_{#9} & {#4}_{#10} \\
% {#5}_{#6} & {#5}_{#7} & {#5}_{#8} & {#5}_{#9} & {#5}_{#10}
%\end{vmatrix}
%}

% R3 vector.
\newcommand{\VectorThree}[3]{
\begin{bmatrix}
 {#1} \\
 {#2} \\
 {#3}
\end{bmatrix}
}



\newcommand{\Dslash}[0]{ \not\!D }

%-----------------------------------------
%
% stubs for article class.
%
\newcommand{\blogpage}[1]{}
\newcommand{\email}[1]{}
\newcommand{\beginArtWithToc}[0]{}
\newcommand{\beginArtNoToc}[0]{}
\newcommand{\EndArticle}[0]{}
\newcommand{\EndNoBibArticle}[0]{}
\newcommand{\revisionInfo}[1]{}
%-----------------------------------------

%\makeindex

\date{ Last $Revision: 1.71 $ }
\begin{document}
\pagenumbering{alph}

\title{Exploring physics with Geometric Algebra.}
\author{Peeter Joot  \quad peeter.joot@gmail.com \\
{\small\em \copyright \  Draft date \today }}

\maketitle
\renewcommand{\title}[1]{\chapter{}}

\clearpage\pagenumbering{roman}
\tableofcontents
\listoffigures
\listoftables

\clearpage\pagenumbering{arabic}

\pagestyle{plain}

%
% Copyright � 2015 Peeter Joot.  All Rights Reserved.
% Licenced as described in the file LICENSE under the root directory of this GIT repository.
%

% 
%\chapter{Preface}
% this suppresses an explicit chapter number for the preface.
\chapter*{Preface}%\normalsize
  \addcontentsline{toc}{chapter}{Preface}

This document was produced while taking the Spring 2016, University of Toronto Microwave Circuits course (ECE1236H), taught by Prof.\ G. V. Eleftheriades.

\paragraph{Course Syllabus}

This course outlines the principles of designing modern microwave and RF circuits.  Signal-integrity issues in high-speed digital circuits are also examined.

\begin{itemize}
\item The wave equation.
\item Ideal transmission lines.
\item Transients on transmission-lines.
\item Planar transmission lines and introduction to MMIC's.
\item Designing with scattering parameters.
\item Planar power dividers.
\item Directional couplers.
\item Microwave filters.
\item Solid-state microwave amplifiers.
\item Noise.
\item Diode-mixers.
\item RF receiver chains.
\item Oscillators.
\end{itemize}

\withproblemsetsMessage{
\textcolor{Maroon}{
\textit{THIS DOCUMENT IS REDACTED.  THE PROBLEM SET SOLUTIONS AND ASSOCIATED MATHEMATICA CODE IS NOT VISIBLE.  PLEASE EMAIL ME FOR THE FULL VERSION IF YOU ARE NOT TAKING ECE1236.}
}
}

\paragraph{This document contains:}

\begin{itemize}
\item Lecture notes.
\item Personal notes exploring auxiliary details.
\item Worked practice problems.

\ifthenelse{\boolean{redacted}}%
{%
\item Links to Mathematica notebooks associated with the course material and problems (but not problem sets).
}%
{
\item Assigned problems.%
\item Links to Mathematica notebooks associated with problems and course material.%
}
\end{itemize}

%This set of notes is significantly different from my notes for many other classes.  With the class taught on slides (and some of those slides mirroring the text closely), I did not take live notes in class.
%These notes fill in details that I felt deserved clarification, contain problem sets solutions, as well as a number of loosely related musings on Geometric Algebra equivalents to some of the generalized concepts of electromagnetic theory encountered in this class (i.e. magnetic sources).
%
My thanks go to Professor Eleftheriades for teaching this course.

Peeter Joot  \quad peeterjoot@protonmail.com 


% \addcontentsline{toc}{chapter}{Contents}
%\pagenumbering{roman}
%\chapter*{Preface}\normalsize
%  \addcontentsline{toc}{chapter}{Preface}

%-------------------------------------------------------

\part{Basics and Geometry.}
%
% Copyright � 2012 Peeter Joot.  All Rights Reserved.
% Licenced as described in the file LICENSE under the root directory of this GIT repository.
%

%
%
\chapter{Introductory concepts}
\label{chap:introGa}

%%%%%\section{My search for Geometric Algebra.}
%%%%%
%%%%%When you learned vector algebra initially, you learned how to add vectors and scale them, and you probably asked your teacher
%%%%%
%%%%%``How do we multiply vectors?''
%%%%%
%%%%%My teacher's response was something like:
%%%%%
%%%%%``It is impossible to multiply vectors, but some multiplication like operations can be defined.''
%%%%%
%%%%%This may have been followed with a lesson on the dot and cross product operators, or at least a mention that this topic would be covered later.
%%%%%
%%%%%You'll also learn how to generalize vectors from two and three dimensions to higher dimensions, and may learn to generalize vectors from real valued to complex valued.  The dot product generalizes nicely to higher dimensions and even complex valued vectors.  However, given the usefulness of the cross product, you probably find your self asking ``How does the cross product generalize?''
%%%%%
%%%%%The cross product is an explicitly three dimensional beast, and isn't even well defined in two.  You have to introduce a 3D normal direction to describe quanities like torque that are perfectly well defined in a plane.  When I was confronted with this oddity, my conclusion was that there must be a way to generalize the cross product to two dimensions or to greater than 3 dimenions.  A search for that generalization eventually led me to discover Geometric Algebra (with a stop over at differential forms on the way).
%%%%%
%%%%%Geometric Algebra answers the ``How do I multiply vectors?'' question and supplies the generalization of the cross product, among many other things.
%%%%%It also provides a mathematical toolbox that incorporates and extends many not-obviously related fields within mathematics.
%%%%%
%%%%%%Geometric Algebra is not the only answer to some of these questions.  The student of differential forms will know that the cross product can be found generalized using the wedge product of one-forms.  When I first found that, it was not obvious how to apply that generalization to many problems of geometry.  Vector objects have to be promoted to differential forms to apply the wedge product, even if a differential version of such an object does not make any sense.
%%%%%
%%%%%%Eventually, I blundered through an attempt of my own to generalize the cross product, and found a way that worked well for the generalized ``cross product'' of \( n -1 \) n-dimensional vectors.  Such a product worked nicely to define a normal to the \( n -1 \) dimensional subspace that was spanned by that set of vectors.  I was left wondering how to apply this generalization in other obvious contexts, such as a generalization of Stokes' Theorem, which is expressed in terms of the cross product in \R{3}.  I wasn't smart enough at the time to just go looking for existing generalizations of Stokes' theorem.  A friend, much smarter than I, did that search for me, and pointed me towards the field of differential forms which had a wedge product that generalized the cross product.
%%%%%
%%%%%%, and perhaps apply linear operators (i.e. matrices) to them.  One of your
%%%%%%one of your questions to the instructor was probably

Here is an attempt to provide a naturally sequenced introduction to Geometric Algebra.

\section{The Axioms}

Two basic axioms are required, contraction and associative multiplication respectively

\begin{equation}\label{eqn:introGa:40}
\begin{aligned}
a^2 &= \text{scalar} \\
a (b c) &= (a b) c = a b c
\end{aligned}
\end{equation}

Linearity and scalar multiplication should probably also be included for completeness, but even with those this is a surprisingly small set of rules.  The choice to impose these as the rules for vector multiplication will be seen to have a rich set of consequences once explored.  It will take a fair amount of work to extract all the consequences of this decision, and some of that will be done here.

\section{Contraction and the metric}

Defining \(a^2\) itself requires introduction of a metric, the specification of the multiplication rules for a particular basis for the vector space.  For Euclidean spaces, a requirement that

\begin{equation}\label{eqn:introGa:60}
\begin{aligned}
a^2 = \Abs{a}^2
\end{aligned}
\end{equation}

is sufficient to implicitly define this metric.  However, for the Minkowski spaces of special relativity one wants the squares of time and spatial basis vectors to be opposing in sign.  Deferring the discussion of metric temporarily one can work with the axioms above to discover their implications, and in particular how these relate to the coordinate vector space constructions that are so familiar.

\section{Symmetric sum of vector products}

Squaring a vector sum provides the first interesting feature of the general vector product

\begin{equation}\label{sumSquared}
\begin{aligned}
(a + b)^2 %&= (a + b)(a + b) \\
&= a^2 + b^2 + a b + b a
\end{aligned}
\end{equation}

Observe that the LHS is a scalar by the contraction identity, and on the RHS we have scalars \(a^2\) and \(b^2\) by the same.  This implies that the symmetric sum of products

\begin{equation}\label{eqn:introGa:80}
\begin{aligned}
a b + b a
\end{aligned}
\end{equation}

is also a scalar, independent of any choice of metric.  Symmetric sums of this form have a place in physics over the space of operators, often instantiated in matrix form.  There one writes this as the commutator and denotes it as

\begin{equation}\label{eqn:intro_ga:anticommutator}
\begin{aligned}
\symmetric{a}{b} \equiv a b + b a
\end{aligned}
\end{equation}

In an Euclidean space one can observe that equation \ref{sumSquared} has the same structure as the law of cosines so it should not be surprising that this symmetric sum is also related to the dot product.  For a Euclidean space where one the notion of perpendicularity can be expressed as

\begin{equation}\label{eqn:introGa:100}
\begin{aligned}
\Abs{ a + b }^2 = \Abs{a}^2 + \Abs{b}^2
\end{aligned}
\end{equation}

we can then see that an implication of the vector product is the fact that perpendicular vectors have the property

\begin{equation}\label{eqn:introGa:120}
\begin{aligned}
a b + ba = 0
\end{aligned}
\end{equation}

or

\begin{equation}\label{eqn:introGa:140}
\begin{aligned}
b a = - a b
\end{aligned}
\end{equation}

This notion of perpendicularity will also be seen to make sense for non-Euclidean spaces.

Although it retracts from a purist Geometric Algebra approach where things can be done in a coordinate free fashion, the connection between the symmetric product and the standard vector dot product can be most easily shown by considering an expansion with respect to an orthonormal basis.

Lets write two vectors in an orthonormal basis as

\begin{equation}\label{eqn:introGa:160}
\begin{aligned}
a &= \sum_\mu a^\mu e_\mu \\
b &= \sum_\mu b^\mu e_\mu
\end{aligned}
\end{equation}

Here the choice to utilize raised indices rather than lower for the coordinates is taken from physics where summation is typically implied when upper and lower indices are matched as above.

Forming the symmetric product we have

\begin{equation}\label{eqn:introGa:180}
\begin{aligned}
a b + b a
&=
\sum_{\mu,\nu} a^\mu e_\mu b^\nu e_\nu + b^\mu e_\mu a^\nu e_\nu \\
&=
\sum_{\mu,\nu} a^\mu b^\nu \left( e_\mu e_\nu + e_\nu e_\mu \right) \\
&=
2 \sum_{\mu} a^\mu b^\mu {e_\mu}^2 + \sum_{\mu \ne \nu} a^\mu b^\nu \left( e_\mu e_\nu + e_\nu e_\mu \right) \\
\end{aligned}
\end{equation}

For an Euclidean space we have \({e_\mu}^2 = 1\), and \(e_\nu e_\mu = -e_\mu e_\nu\), so we are left with

\begin{equation}\label{eqn:introGa:200}
\begin{aligned}
\sum_{\mu} a^\mu b^\mu = \inv{2} ( a b + b a)
\end{aligned}
\end{equation}

This shows that we can make an identification between the symmetric product, and the anticommutator of physics with the dot product, and then define

\begin{equation}\label{eqn:intro_ga:dotDefined}
\begin{aligned}
a \cdot b \equiv \inv{2} \symmetric{a}{b} = \inv{2} (a b + ba)
\end{aligned}
\end{equation}

\section{Antisymmetric product of two vectors (wedge product)}

Having identified or defined the symmetric product with the dot product we are now prepared to examine a general product of two vectors.  Employing a symmetric + antisymmetric decomposition we can write such a general product as

\begin{equation}\label{eqn:introGa:220}
\begin{aligned}
a b = \mathLabelBox{\inv{2}(a b + b a)}{\(a \cdot b\)} + \mathLabelBox{ \inv{2} ( a b - b a ) }{\(a\) something \(b\)}
\end{aligned}
\end{equation}

What is this remaining vector operation between the two vectors

\begin{equation}\label{eqn:introGa:240}
\begin{aligned}
a \something b = \inv{2} ( a b - b a )
\end{aligned}
\end{equation}

One can continue the comparison with the quantum mechanics, and like the
anticommutator operator that expressed our symmetric sum in equation
\eqnref{eqn:intro_ga:anticommutator} one can introduce a commutator operator

\begin{equation}\label{eqn:intro_ga:commutator}
\begin{aligned}
\antisymmetric{a}{b} \equiv a b - b a
\end{aligned}
\end{equation}

The commutator however, does not naturally extend to more than two vectors, so
as with the scalar part of the vector product (the dot product part),
it is desirable to make a different identification for this part of the vector
product.

One observation that we can make is that this vector operation changes sign when the operations are reversed.  We have

\begin{equation}\label{eqn:introGa:260}
\begin{aligned}
b \something a = \inv{2} ( b a - a b) = - a \something b
\end{aligned}
\end{equation}

Similarly, if \(a\) and \(b\) are colinear, say \(b = \alpha a\), this product is zero

\begin{equation}\label{eqn:introGa:280}
\begin{aligned}
a \something (\alpha a)
&= \inv{2} ( a  (\alpha a) - (\alpha a) a ) \\
&= 0
\end{aligned}
\end{equation}

This complete antisymmetry, aside from a potential difference in sign, are precisely the properties of the wedge product used in the mathematics of differential forms.  In this differential geometry the wedge product of \(m\) one-forms (vectors in this context) can be defined as

\begin{equation}\label{eqn:intro_ga:wedge}
\begin{aligned}
a_1 \wedge a_2 \cdots \wedge a_m
&= \inv{m!} \sum a_{i_1} a_{i_2} \cdots a_{i_m} \sgn(\pi(i_1 i_2 \cdots i_m))
\end{aligned}
\end{equation}

Here \(\sgn(\pi(\cdots))\) is the sign of the permutation of the indices.  While we have not gotten yet to products of more than two vectors it is helpful to know that the wedge product will have a place in such a general product.   An equation like \eqnref{eqn:intro_ga:wedge} makes a lot more sense after writing it out in full for a few specific cases.  For two vectors \(a_1\) and \(a_2\) this is

\begin{equation}\label{eqn:intro_ga:wedgeTwo}
\begin{aligned}
a_1 \wedge a_2 = \inv{2}
\left( a_1 a_2 (1) + a_2 a_1 (-1) \right)
\end{aligned}
\end{equation}

and for three vectors this is

\begin{equation}\label{eqn:introGa:300}
\begin{aligned}
a_1 \wedge a_2 \wedge a_3 = \inv{6}
(
&a_1 a_2 a_3 (1) + a_1 a_3 a_2 (-1) \\
+&a_2 a_1 a_3 (-1) + a_3 a_1 a_2 (1) \\
+&a_2 a_3 a_1 (1) + a_3 a_2 a_1 (-1) )
\end{aligned}
\end{equation}

We will see later that this completely antisymmetrized sum, the wedge product of differential forms will have an important place in this algebra, but like the dot product it is a specific construction of the more general vector product.  The choice to identify the antisymmetric sum with the wedge product is an action that amounts to a definition of the wedge product.  Explicitly, and complementing
the dot product definition of \eqnref{eqn:intro_ga:dotDefined} for the dot product
of two vectors, we say

\begin{equation}\label{eqn:intro_ga:wedgeDefined}
\begin{aligned}
a \wedge b \equiv \inv{2} \antisymmetric{a}{b} = \inv{2} ( a b - b a )
\end{aligned}
\end{equation}

Having made this definition, the symmetric and antisymmetric decomposition of two vectors leaves us with a peculiar looking hybrid construction:

\begin{equation}\label{eqn:intro_ga:dotPlusWedge}
\begin{aligned}
a b %&= \inv{2} (a b + b a) + \inv{2} ( a b - b a ) \\
&= a \cdot b + a \wedge b
\end{aligned}
\end{equation}

We had already seen that part of this vector product was not a vector, but was in fact a scalar.  We now see that the remainder is also not a vector but is instead something that resides in a different space.  In differential geometry this object is called a two form, or a simple element in \(\bigwedge^2\).  Various labels are available for this object are available in Geometric (or Clifford) algebra, one of which is a 2-blade.  2-vector or bivector is also used in some circumstances, but in dimensions greater than three there are reasons to reserve these labels for a slightly more general construction.

The definition of \eqnref{eqn:intro_ga:dotPlusWedge} is often used as the starting point in Geometric Algebra introductions.  While there is value to this approach I have personally found that the non-axiomatic approach becomes confusing if one attempts to sort out which of the many identities in the algebra are the fundamental ones.  That is why my preference is to treat this as a consequence rather than the starting point.

\section{Expansion of the wedge product of two vectors}

Many introductory geometric algebra treatments try very hard to avoid explicit coordinate treatment.  It is true that GA provides infrastructure for coordinate free treatment, however, this avoidance perhaps contributes to making the subject less accessible.  Since we are so used to coordinate geometry in vector and tensor algebra, let us take advantage of this comfort, and express the wedge product explicitly in coordinate form to help get some comfort for it.

Employing the definition of \eqnref{eqn:intro_ga:wedgeDefined}, and an orthonormal basis expansion in coordinates for two vectors \(a\), and \(b\), we have

\begin{equation}\label{eqn:introGa:320}
\begin{aligned}
2 (a \wedge b)
&= ( a b - b a ) \\
&=
\sum_{\mu,\nu} a^\mu b^\nu e_\mu e_\nu
-\sum_{\alpha,\beta} a^\alpha b^\beta e_\alpha e_\beta \\
&=
\mathLabelBox{\sum_{\mu} a^\mu b^\mu - \sum_{\alpha} a^\alpha b^\alpha }{\(=0\)}
+ \sum_{\mu \ne \nu} a^\mu b^\nu e_\mu e_\nu
- \sum_{\alpha \ne \beta} a^\alpha b^\beta e_\alpha e_\beta \\
&=
\sum_{\mu < \nu} (a^\mu b^\nu e_\mu e_\nu + a^\nu b^\mu e_\nu e_\mu)
- \sum_{\alpha < \beta} (a^\alpha b^\beta e_\alpha e_\beta + a^\beta b^\alpha e_\beta e_\alpha )
\\
&=
2 \sum_{\mu < \nu} ( a^\mu b^\nu - a^\nu b^\mu ) e_\mu e_\nu
\end{aligned}
\end{equation}

So we have
\begin{equation}\label{eqn:introGa:340}
\begin{aligned}
a \wedge b
&= \sum_{\mu < \nu} \uDETuvij{a}{b}{\mu}{\nu} e_\mu e_\nu
\end{aligned}
\end{equation}

The similarity to the \R{3} vector cross product is not accidental.  This similarity can be made explicit by observing the following

\begin{equation}\label{eqn:introGa:360}
\begin{aligned}
e_1 e_2 &= e_1 e_2 (e_3 e_3) = (e_1 e_2 e_3) e_3 \\
e_2 e_3 &= e_2 e_3 (e_1 e_1) = (e_1 e_2 e_3) e_1 \\
e_1 e_3 &= e_1 e_3 (e_2 e_2) = -(e_1 e_2 e_3) e_2 \\
\end{aligned}
\end{equation}

This common factor, a product of three normal vectors, or grade three blade, is called the pseudoscalar for \R{3}.  We write
\(i = e_1 e_2 e_3\), and can then express the \R{3} wedge product in terms of the cross product

\begin{equation}\label{eqn:introGa:380}
\begin{aligned}
a \wedge b
&=
\uDETuvij{a}{b}{2}{3} e_2 e_3
+\uDETuvij{a}{b}{1}{3} e_1 e_3
+\uDETuvij{a}{b}{1}{2} e_1 e_2  \\
&=
(e_1 e_2 e_3) \left( \uDETuvij{a}{b}{2}{3} e_1
-\uDETuvij{a}{b}{1}{3} e_2
+\uDETuvij{a}{b}{1}{2} e_3 \right) \\
\end{aligned}
\end{equation}

This is

\begin{equation}\label{eqn:intro_ga:wedgeAsCross}
\begin{aligned}
a \wedge b &= i (a \cross b)
\end{aligned}
\end{equation}

With this identification we now also have a curious integrated relation where the dot and cross products are united into
a single structure

\begin{equation}\label{eqn:intro_ga:scalarPlusIcross}
\begin{aligned}
a b = a \cdot b + i (a \cross b)
\end{aligned}
\end{equation}

\section{Vector product in exponential form}

One naturally expects there is an inherent connection between the dot and cross products, especially when expressed in terms of
the angle between the vectors, as in

\begin{equation}\label{eqn:introGa:400}
\begin{aligned}
a \cdot b &= \Abs{a}\Abs{b} \cos\theta_{a,b} \\
a \cross b &= \Abs{a}\Abs{b} \sin\theta_{a,b} \ncap_{a,b}
\end{aligned}
\end{equation}

However, without the structure of the geometric product the specifics of what
connection is is not obvious.  In particular the use of \eqnref{eqn:intro_ga:scalarPlusIcross} and the angle relations, one can easily
blunder upon the natural complex structure of the geometric product

\begin{equation}\label{eqn:introGa:420}
\begin{aligned}
a b
&= a \cdot b + i (a \cross b) \\
&=
\Abs{a}\Abs{b} \left( \cos\theta_{a,b} + i\ncap_{a,b} \sin\theta_{a,b} \right) \\
\end{aligned}
\end{equation}

As we have seen pseudoscalar multiplication in \R{3} provides a mapping between a grade 2 blade and a vector, so
this \(i\ncap\) product is a 2-blade.

In \R{3} we also have \(i \ncap = \ncap i\) (exercise for reader) and also \(i^2 = -1\) (again for the reader), so this
2-blade \(i\ncap\) has all the properties of the \(i\) of complex arithmetic.  We can, in fact, write

\begin{equation}\label{eqn:introGa:440}
\begin{aligned}
a b
&= a \cdot b + i (a \cross b) \\
&=
\Abs{a}\Abs{b} \exp( i\ncap_{a,b} \theta_{a,b} )
\end{aligned}
\end{equation}

In particular, for unit vectors \(a\), \(b\) one is able to quaternion exponentials of this form to rotate from one vector to the other

\begin{equation}\label{eqn:introGa:460}
\begin{aligned}
b &= a \exp( i\ncap_{a,b} \theta_{a,b} )
\end{aligned}
\end{equation}

This natural GA use
of multivector exponentials to implement rotations is not restricted to \R{3} or even Euclidean space, and is one of the most
powerful features of the algebra.

\section{Pseudoscalar}

In general the
pseudoscalar for \R{N} is a product of \(N\) normal vectors and multiplication by such an object maps m-blades to (N-m) blades.

For \R{2} the unit pseudoscalar has a negative square

\begin{equation}\label{eqn:introGa:480}
\begin{aligned}
(e_1 e_2) (e_1 e_2)
&=
- (e_2 e_1) (e_1 e_2) \\
&=
- e_2 (e_1 e_1) e_2 \\
&=
- e_2 e_2 \\
&=
-1
\end{aligned}
\end{equation}

and we have seen an example of such a planar pseudoscalar in the subspace of the span of two vectors above (where \(\ncap i\) was a pseudoscalar
for that subspace).  In general the sign of the square of the pseudoscalar depends on both the dimension and the metric of the space,
so the ``complex'' exponentials that rotate one vector into another may represent hyperbolic rotations.

For example we have for a four dimensional space the pseudoscalar square is

\begin{equation}\label{eqn:introGa:500}
\begin{aligned}
i^2 &=
(e_0 e_1 e_2 e_3) (e_0 e_1 e_2 e_3) \\
&=
- e_0 e_0 e_1 e_2 e_3 e_1 e_2 e_3 \\
&=
- e_0 e_0 e_1 e_2 e_3 e_1 e_2 e_3 \\
&=
- e_0 e_0 e_1 e_1 e_2 e_3 e_2 e_3 \\
&=
e_0 e_0 e_1 e_1 e_2 e_2 e_3 e_3 \\
\end{aligned}
\end{equation}

For a Euclidean space where each of the \({e_k}^2 = 1\), we have \(i^2 = 1\), but for a Minkowski space where one would have for \(k\ne0\), \({e_0}^2 {e_k}^2 = -1\), we have \(i^2 = -1\)

Such a mixed signature metric will allow for implementation of Lorentz transformations as exponentials (hyperbolic) rotations
in a fashion very much like the quaternionic spatial rotations for Euclidean spaces.

It is also worth pointing out that the pseudoscalar multiplication naturally provides a mapping operator into a dual space, as we have seen
in the cross product example, mapping vectors to bivectors, or bivectors to vectors.  Pseudoscalar multiplication in fact provides an
implementation of the Hodge duality operation of differential geometry.

In higher than three dimensions, such as four, this duality operation can in fact map 2-blades to orthogonal 2-blades (orthogonal in the sense
of having no common factors).  Take for example the typical example of a non-simple element from differential geometry

\begin{equation}\label{eqn:introGa:520}
\begin{aligned}
\omega = e_1 \wedge e_2 + e_3 \wedge e_4
\end{aligned}
\end{equation}

The two blades that compose this sum have no common factors and thus cannot be formed as the wedge product of two vectors.  These two blades
are orthogonal in a sense that can be made more exact later.   As this time we just wish to make the observation that
the pseudoscalar provides a natural duality operation between these two subspaces of \(\bigwedge^2\).  Take for example

\begin{equation}\label{eqn:introGa:540}
\begin{aligned}
i e_1 \wedge e_2
&=
 e_1 e_2 e_3 e_4 e_1 e_2  \\
&=
- e_1 e_1 e_2 e_3 e_4 e_2  \\
&=
- e_1 e_1 e_2 e_2 e_3 e_4 \\
&\propto
e_3 e_4 \\
\end{aligned}
\end{equation}

\section{FIXME: orphaned}
As an exercise work out axiomatically some of the key vector identities of Geometric Algebra.

Want to at least derive the vector bivector dot product distribution
identity

\begin{equation}\label{eqn:introGa:20}
\begin{aligned}
a \cdot ( b \wedge c) = (a \cdot b) c - (a \cdot c) b
\end{aligned}
\end{equation}

%\section{Higher order products}

%
% Copyright � 2012 Peeter Joot.  All Rights Reserved.
% Licenced as described in the file LICENSE under the root directory of this GIT repository.
%

%
%
\chapter{An (earlier) attempt to intuitively introduce the dot, wedge, cross, and geometric products}
\index{dot product!introduction}
\index{wedge product!introduction}
\index{cross product!introduction}
\index{geometric product!introduction}
\label{chap:gaGradeDotWedge}

%\date{March 17, 2008.  gaGradeDotWedge.tex}

\section{Motivation}

Both the NFCM and GAFP books have axiomatic introductions of the
generalized (vector, blade) dot and wedge products, but there are
elements of both that I was unsatisfied with.  Perhaps the biggest
issue with both is that they are not presented in a dumb enough fashion.

NFCM presents but
does not prove the generalized dot and wedge product operations
in terms of symmetric and antisymmetric sums, but it is really the
grade operation that is fundamental.  You need that to define the
dot product of two bivectors for example.

GAFP axiomatic presentation is much clearer, but the definition of
generalized wedge product as the totally antisymmetric sum is a bit
strange when all the differential forms book give such a different
definition.

Here I collect some of my notes on how one starts with the geometric
product action on colinear and perpendicular vectors and gets the
familiar results for two and three vector products.  I may not try to
generalize this, but just want to see things presented in a fashion
that makes sense to me.

\section{Introduction}

The aim of this document is to introduce a ``new'' powerful vector multiplication operation, the geometric product,
to a student with some traditional vector algebra background.

The geometric product, also called the Clifford product
\footnote{After William Clifford (1845-1879).}, has remained a relatively obscure mathematical subject.
This operation actually makes a great deal of vector manipulation simpler than possible with the traditional methods, and
provides a way to naturally expresses many geometric concepts.
There is a great deal of information available on the subject, however most of it is targeted for those with a
university graduate school background in physics or mathematics.  That level of mathematical sophistication
should not required to understand the subject.

It is the author's opinion that this could be dumbed down even further, so that it would be palatable for
somebody without any traditional vector algebra background.

\section{What is multiplication?}

The operations of vector addition, subtraction and numeric multiplication have the usual definitions
(addition defined in terms of addition of coordinates, and numeric multiplication as a scaling of the vector retaining its direction).  Multiplication and division of vectors is often described as ``undefined''.  It is possible however, to define a multiplication, or division operation for vectors, in a natural geometric fashion.

What meaning should be given to multiplication or division of vectors?

\subsection{Rules for multiplication of numbers}

Assuming no prior knowledge of how to multiply two vectors (such as the dot, cross, or wedge products to be introduced later) consider instead the rules for multiplication of numbers.

\begin{enumerate}
\item Product of two positive numbers is positive.  Any consideration of countable sets of objects justifies this rule.

\item Product of a positive and negative number is negative.  Example: multiplying a debt (negative number) increases the amount of the debt.

\item Product of a negative and negative number is positive.

\item Multiplication is distributive.  Product of a sum is the sum of the products.
\footnote{The name of this property is not important and no student should ever be tested on it.  It is a word like
dividand which countless countless school kids are forced to memorize.  Like dividand it is perfectly
acceptable to forget it after the test because nobody has to know it to perform division.
Since most useful sorts of multiplications have this property this is the least important
of the named multiplication properties.  This word exists mostly so that authors of math books can impress themselves writing phrases like ``a mathematical entity that behaves this way is
left and right distributive with respect to addition''.
}

\begin{equation}\label{eqn:gaGradeDotWedge:20}
a (b + c) = a b + a c
\end{equation}
\begin{equation}\label{eqn:gaGradeDotWedge:40}
(a + b) c = a c + b c
\end{equation}

\item Multiplication is associative.  Changing the order that multiplication is grouped by does not change the result.

\begin{equation}\label{eqn:gaGradeDotWedge:60}
(a b) c = a (b c)
\end{equation}

\item Multiplication is commutative.  Switching the order of multiplication does not change the result.

\begin{equation}\label{eqn:gaGradeDotWedge:80}
a b = b a
\end{equation}

\end{enumerate}

Unless the reader had an exceptionally gifted grade three teacher it is likely that rule three was presented without any sort of justification or analogy.  This can be considered as a special case of the previous rule.  Geometrically, a multiplication by -1 results in an inversion on the number line.  If one considers the number line to be a line in space, then this is a 180 degree rotation.  Two negative multiplications results in a 360 degree rotation, and thus takes the number back to its original positive or negative segment on its ``number line''.

\subsection{Rules for multiplication of vectors with the same direction}

Having identified the rules for multiplication of numbers, one can use these to define multiplication rules for a simple case, one dimensional vectors.
Conceptually a one dimensional vector space can be thought of like a number line, or the set of all numbers as the set of all scalar multiples of a unit vector of a particular direction in space.

It is reasonable to expect the rules for multiplication of two vectors with the same direction to have some of the same characteristics as multiplication of numbers.  Lets state this algebraically writing the directed distance from the origin to the points \(a\) and \(b\) in a vector notation

\begin{equation}\label{eqn:gaGradeDotWedge:540}
\begin{aligned}
\Ba &= a\Be \\
\Bb &= b\Be \\
\end{aligned}
\end{equation}

where \(\Be\) is the unit vector alone the line in question.

The product of these two vectors is

\begin{equation}\label{eqn:gaGradeDotWedge:100}
\Ba \Bb = a b \Be \Be
\end{equation}

Although no specific meaning has yet been given to the \(\Be \Be\) term yet, one can make a few observations about a product of this form.
\begin{enumerate}
\item It is commutative, since \(\Ba \Bb = \Bb \Ba = a b \Be \Be\).
\item It is distributive since numeric multiplication is.
\item The product of three such vectors is distributive (no matter the grouping of the multiplications there will be a numeric factor and a \(\Be \Be \Be\) factor.
\end{enumerate}

These properties are consistent with half the properties of numeric multiplication.  If the other half of the numeric multiplication rules are assumed to also apply we have

\begin{enumerate}
\item Product of two vectors in the same direction is positive (rules 1 and 3 above).
\item Product of two vectors pointing in opposite directions is negative (rule 2 above).
\end{enumerate}

This can only be possible by giving the following meaning to the square of a unit vector

\begin{equation}\label{eqn:gaGradeDotWedge:120}
\Be \Be = 1
\end{equation}

Alternately, one can state that the square of a vector is that vectors squared length.

\begin{equation}\label{eqn:gaGradeDotWedge:140}
\Ba \Ba = a^2
\end{equation}

This property, as well as the associative and distributive properties are the defining properties of the geometric product.

It will be shown shortly that in order to retain this squared vector length property for vectors with components in different directions it will be required to drop the commutative property of numeric multiplication:

\begin{equation}\label{eqn:gaGradeDotWedge:160}
\Ba \Bb \neq \Bb \Ba
\end{equation}

This is a choice that will later be observed to have important consequences.
There are many types of multiplications that do not have the commutative property.  Matrix multiplication is not even necessarily defined when the order is switched.  Other multiplication operations (wedge and cross products) change sign when the order is switched.

Another important choice has been made to require the product of two vectors not be a vector itself.  This also breaks
from the number line analogy since the product of two numbers is still a number.  However, it is
notable that
in order to take roots of a negative number one has to introduce a second number line
(the \(i\), or imaginary axis), and so even for numbers, products can be ``different'' than their factors.
Interestingly enough,
it will later be possible to show that the choice to not require a vector product to be a vector
allow complex numbers to be defined directly in terms of the geometric product of two vectors in a plane.

\section{Axioms}

The previous discussion attempts to justify the choice of the following set of axioms for multiplication of vectors

\begin{enumerate}
\item{ linearity }
\item{ associativity }
\item{ contraction }

Square of a vector is its squared length.
\end{enumerate}

This last property is weakened in some circumstances (for example,
an alternate definition of vector length is desirable for relativistic calculations.)

%As justification of the contraction property one could
%consider a set of colinear vectors and the real number line to be
%isomorphic.
%
%The product of two positive numbers is a positive number.  Multiplying
%by \(-1\) (the unit negative) produces a rotatation by 180 degrees.  Two negative multiplications
%produces a rotation of 360.  This can be thought of as a justification
%of the grade school ``rule'' that a negative times a negative is positive.
%
%It seems natural to have the rules for vector multiplication reduce to something
%like the rules for numbers when those vectors are restricted to a linear subspace.
%
%In analogy with numbers, the contraction rule gives us such similar properties.  Namely, the
%product of same facing vectors is positive, and the product of opposite facing
%vectors is negative, both scaled by their magnitudes.

\section{dot product}

One can express the dot product in terms of these axioms.  This follows by calculating the
length of a sum or difference of vectors, starting with the requirement that the vector square is the squared length of that vector.

Given two vectors \(\Ba\) and \(\Bb\), their sum
\(\Bc = \Ba + \Bb\) has squared length:

\begin{equation}\label{eqn:introGaFirst:absquared}
\Bc^2 = (\Ba + \Bb)(\Ba + \Bb) = \Ba^2 + \Bb\Ba + \Ba\Bb + \Bb^2.
\end{equation}

We do not have any specific meaning for the product of vectors, but \eqnref{eqn:introGaFirst:absquared}
shows that the symmetric sum of such a product:

\begin{equation}
\Bb\Ba + \Ba\Bb = \text{scalar}
\end{equation}

since the RHS is also a scalar.

Additionally, if \(\Ba\) and \(\Bb\) are perpendicular, then we must also have:

\begin{equation}\label{eqn:gaGradeDotWedge:180}
\Ba^2 + \Bb^2 = a^2 + b^2.
\end{equation}

This implies a rule for vector multiplication of perpendicular vectors

\begin{equation}\label{eqn:gaGradeDotWedge:200}
\Bb\Ba + \Ba\Bb = 0
\end{equation}

Or,

\begin{equation}\label{eqn:introGaFirst:perpabcommutesign}
\Bb\Ba = -\Ba\Bb.
\end{equation}

Note that \eqnref{eqn:introGaFirst:perpabcommutesign} does not assign any meaning to this product of vectors when they perpendicular.
Whatever that meaning is, the entity such a perpendicular vector product produces changes sign
with commutation.

Performing the same length calculation using standard vector algebra shows that we can identify the symmetric
sum of vector products with the dot product:

\begin{equation}\label{eqn:introGaFirst:standarddot}
\norm{\Bc}^2 = (\Ba + \Bb) \cdot (\Ba + \Bb) = \norm{\Ba}^2 + 2 \Ba \cdot \Bb + \norm{\Bb}^2.
\end{equation}

Thus we can make the identity:

\begin{equation}\label{eqn:introGaFirst:dotprod}
\Ba \cdot \Bb = \inv{2}(\Ba \Bb + \Bb \Ba)
\end{equation}

\section{Coordinate expansion of the geometric product}

A powerful feature of geometric algebra is that it allows for coordinate free results, and the avoidance of basis selection
that coordinates require.  While this is true, explicit coordinate expansion, especially
initially while making the transition from coordinate based vector algebra, is believed to add clarity to the subject.

Writing a pair of vectors in coordinate vector notation:

\begin{equation}\label{eqn:gaGradeDotWedge:220}
\Ba = \sum_i{a_i \Be_i}
\end{equation}
\begin{equation}\label{eqn:gaGradeDotWedge:240}
\Bb = \sum_j{b_j \Be_j}
\end{equation}

Despite not yet
knowing what meaning to give to the geometric product of two general (non-colinear) vectors,
given the axioms above and their consequences we actually have enough information to completely
expand the geometric product of two vectors in terms of these coordinates:

\begin{equation}\label{eqn:gaGradeDotWedge:560}
\begin{aligned}
\Ba\Bb
&= \sum_{ij}{a_i b_j \Be_i \Be_j} \\
&= \sum_{i = j} {a_i b_j \Be_i \Be_j}
 + \sum_{i \ne j} {a_i b_j \Be_i \Be_j} \\
&= \sum_{i} {a_i b_i \Be_i \Be_i}
 + \sum_{i < j} a_i b_j \Be_i \Be_j
 + \sum_{j < i} a_i b_j \Be_i \Be_j \\
&= \sum_{i} {a_i b_i}
 + \sum_{i < j} a_i b_j \Be_i \Be_j + a_j b_i \Be_j \Be_i \\
&= \sum_{i} {a_i b_i}
 + \sum_{i < j} (a_i b_j - b_i a_j)\Be_i \Be_j \\
\end{aligned}
\end{equation}

This can be summarized nicely in terms of determinants:

\begin{equation}\label{eqn:introGaFirst:geocoord}
\Ba\Bb = \sum_{i} {a_i b_i} + \sum_{i < j} \DETuvij{a}{b}{i}{j} \Be_i \Be_j
\end{equation}

This shows,
without requiring the ``triangle law'' expansion of \eqnref{eqn:introGaFirst:standarddot},
that the geometric product has a scalar component that we recognize as the Euclidean vector dot product.  It also shows that the remaining bit
is a ``something else'' component.  This ``something else'' is called a bivector.  We do not yet know what this bivector is or what to do with it,
but will come back to that.

Observe that an interchange of \(\Ba\) and \(\Bb\) leaves the scalar part of equation
\eqnref{eqn:introGaFirst:geocoord} unaltered (ie: it is symmetric), whereas an interchange inverts the bivector (ie: it is the antisymmetric part).

\section{Some specific examples to get a feel for things}

Moving from the abstract, consider a few specific geometric product example.

\begin{itemize}
\item Product of two non-colinear non-orthogonal vectors.
\begin{equation}\label{eqn:gaGradeDotWedge:260}
(\Be_1 + 2\Be_2) (\Be_1 - \Be_2)
= \Be_1\Be_1 -2\Be_2\Be_2 + 2\Be_2\Be_1 - \Be_1\Be_2
= -1 + 3\Be_2\Be_1
\end{equation}

Such a product produces both scalar and bivector parts.

\item Squaring a bivector
\begin{equation}\label{eqn:gaGradeDotWedge:280}
(\Be_1\Be_2)^2
=
(\Be_1\Be_2)(-\Be_2\Be_1)
=
-\Be_1(\Be_2\Be_2)\Be_1
=
-\Be_1\Be_1
=
-1
\end{equation}

This particular bivector squares to minus one very much like the imaginary number \(i\).

\item Product of two perpendicular vectors.

\begin{equation}\label{eqn:gaGradeDotWedge:300}
(\Be_1 + \Be_2) (\Be_1 - \Be_2) = 2\Be_2\Be_1
\end{equation}

Such a product generates just a bivector term.

\item Product of a bivector and a vector in the plane.

\begin{equation}\label{eqn:gaGradeDotWedge:320}
(x\Be_1 + y\Be_2) \Be_1\Be_2
=
x\Be_2 - y\Be_1
\end{equation}

This rotates the vector counterclockwise by 90 degrees.

\item General \R{3} geometric product of two vectors.

\begin{equation}\label{eqn:gaGradeDotWedge:340}
\Bx \By =
(x_1\Be_1
+x_2\Be_2
+x_3\Be_3)
(y_1\Be_1
+y_2\Be_2
+y_3\Be_3)
\end{equation}
\begin{equation}\label{eqn:gaGradeDotWedge:360}
=
\Bx \cdot \By
+\DETuvij{x}{y}{2}{3} \Be_2 \Be_3
+\DETuvij{x}{y}{1}{3} \Be_1 \Be_3
+\DETuvij{x}{y}{1}{2} \Be_1 \Be_2
\end{equation}

Or,
\begin{equation}\label{eqn:gaGradeDotWedge:380}
\Bx \By =
\Bx \cdot \By +
\begin{vmatrix}
\Be_2\Be_3 & \Be_3\Be_1 & \Be_1\Be_2 \\
x_1 & x_2 & x_3 \\
y_1 & y_2 & y_3 \\
\end{vmatrix}
\end{equation}

Observe that if one identifies
\(\Be_2\Be_3\), \(\Be_3\Be_1\), and \(\Be_1\Be_2\) with vectors
\(\Be_1\),
\(\Be_2\),
and \(\Be_3\) respectively, this second term is the cross product.  A precise way to perform this identification will be described later.

The key thing to observe here is
that the structure of the cross product is naturally associated with the geometric product.  One can think of the geometric product
as a complete product including elements of both the dot and cross product.  Unlike the cross product the geometric product is also well defined
in two dimensions and greater than three.

\end{itemize}

These examples are all somewhat random, but give a couple hints of results to come.

\section{Antisymmetric part of the geometric product}

Having identified the symmetric sum of vector products with the dot product we can write the geometric product of two arbitrary vectors
in terms of this and its difference

\begin{equation}\label{eqn:gaGradeDotWedge:580}
\begin{aligned}
\Ba \Bb
&= \inv{2}(\Ba \Bb + \Bb \Ba) + \inv{2}(\Ba \Bb - \Bb \Ba) \\
&= \Ba \cdot \Bb + f(\Ba, \Bb) \\
\end{aligned}
\end{equation}

Let us examine this second term, the bivector, a mapping of a pair of vectors into a different sort of object of yet unknown properties.

\begin{equation}\label{eqn:gaGradeDotWedge:400}
f(\Ba, k\Ba) = \inv{2}(\Ba k\Ba - k\Ba \Ba) = 0
\end{equation}

Property: Zero when the vectors are colinear.

\begin{equation}\label{eqn:gaGradeDotWedge:420}
f(\Ba, k\Ba + \Bb) = \inv{2}(\Ba (k\Ba + \Bb) - (k\Ba + m\Bb)\Ba) = f(\Ba, \Bb)
\end{equation}

Property: colinear contributions are rejected.

\begin{equation}\label{eqn:gaGradeDotWedge:440}
f(\alpha \Ba, \beta \Bb) = \inv{2}(\alpha \Ba \beta \Bb - \beta \Bb \alpha \Ba) = \alpha \beta f(\Ba, \Bb)
\end{equation}

Property: bilinearity.

\begin{equation}\label{eqn:gaGradeDotWedge:460}
f(\Bb, \Ba)
= \inv{2}(\Bb \Ba - \Ba\Bb)
= -\inv{2}(\Ba \Bb - \Bb\Ba)
= -f(\Ba, \Bb)
\end{equation}

Property: Interchange inverts.

Operationally, these are in fact the properties of what in the calculus of differential forms is called the wedge product (uncoincidentally, these are also all properties of the cross product as well.)

Because the properties are identical the notation from differential forms is stolen, and we write

\begin{equation}\label{eqn:introGaFirst:wedge}
\Ba \wedge \Bb = \inv{2}(\Ba \Bb - \Bb \Ba)
\end{equation}

And as mentioned, the object that this
wedge product produces from two vectors is called a bivector.

Strictly speaking the
wedge product of differential calculus is defined as an alternating, associative, multilinear form.  We have here bilinear, not multilinear and associativity is
not meaningful until more than two factors are introduced, however when we get to the product of more than three
vectors, we will find that the geometric vector product produces an entity with all of these properties.

Returning to the product of two vectors we can now write

\begin{equation}\label{eqn:introGaFirst:gaproddotwedge}
\Ba \Bb = \Ba \cdot \Bb + \Ba \wedge \Bb
\end{equation}

This is often used as the initial definition of the geometric product.

\section{Yes, but what is that wedge product thing}

Combination of the symmetric and antisymmetric decomposition in \eqnref{eqn:introGaFirst:gaproddotwedge} shows that the product of two vectors according to the axioms
has a scalar part and a bivector part.  What is this bivector part geometrically?

One can show that the equation of a plane can be written in terms of bivectors.  One can also show that
the area of the parallelogram spanned by two vectors can be expressed in terms of the ``magnitude'' of a bivector.  Both of these
show that a bivector characterizes a plane and can be thought of loosely as a ``plane vector''.

Neither the plane equation or the area result are hard to show, but we will get to those later.  A more direct way to get an
intuitive feel for the geometric properties of the bivector can be obtained by first examining the
square of a bivector.

By subtracting the projection of one vector \(\Ba\) from another \(\Bb\), one can form the rejection of \(\Ba\) from \(\Bb\):

\begin{equation}\label{eqn:gaGradeDotWedge:480}
\Bb' = \Bb - (\Bb \cdot \acap)\acap
\end{equation}

With respect to the dot product, this vector is orthogonal to \(\Ba\).  Since \(\Ba \wedge \acap = 0\), this allows us to
write the wedge product of vectors \(\Ba\) and \(\Bb\) as the direct product of two orthogonal vectors

\begin{equation}\label{eqn:gaGradeDotWedge:600}
\begin{aligned}
\Ba \wedge \Bb
&= \Ba \wedge (\Bb - (\Bb \cdot \acap)\acap)) \\
&= \Ba \wedge \Bb' \\
&= \Ba \Bb' \\
&= -\Bb' \Ba \\
\end{aligned}
\end{equation}

The square of the bivector can then be written

\begin{equation}\label{eqn:gaGradeDotWedge:620}
\begin{aligned}
(\Ba \wedge \Bb)^2
&= (\Ba \Bb')(-\Bb'\Ba) \\
&= -\Ba^2 (\Bb')^2.
\end{aligned}
\end{equation}

Thus the square of a bivector is negative.  It is natural to define a
bivector norm:

\begin{equation}\label{eqn:gaGradeDotWedge:500}
\abs{\Ba \wedge \Bb} = \sqrt{-(\Ba \wedge \Bb)^2} = \sqrt{ (\Ba \wedge \Bb)(\Bb \wedge \Ba) }
\end{equation}

Dividing by this norm we have an entity that acts precisely like the imaginary number \(i\).

Looking back to \eqnref{eqn:introGaFirst:gaproddotwedge} one can now assign additional meaning to the two parts.  The first, the dot product, is a scalar (ie: a real number), and a second part, the wedge product, is a pure imaginary term.  Provided \(\Ba \wedge \Bb \ne 0\), we can write \(i = \frac{\Ba \wedge \Bb}{ \abs{\Ba \wedge \Bb} }\) and express
the geometric product in complex number form:

\begin{equation}\label{eqn:gaGradeDotWedge:520}
\Ba \Bb = \Ba \cdot \Bb + i \abs{\Ba \wedge \Bb}
\end{equation}

The complex number system
is the algebra of the plane, and the geometric product of two vectors can be used to completely characterize the algebra of an arbitrarily oriented plane in a higher
order vector space.

It actually will be very natural to define complex numbers in terms of the geometric product, and we will see later that
the geometric product allows for the ad-hoc definition of ``complex number'' systems according to convenience in many ways.

We will also see that generalizations of complex numbers such as quaternion algebras also find their natural place as specific instances of geometric products.

Concepts familiar from
complex numbers such as conjugation, inversion, exponentials as rotations, and even things like the residue theory of complex contour integration, will
all have a natural geometric algebra analogue.

We will return to this, but first some more detailed initial examination of the wedge product properties is in order, as is a look at the product of greater than
two vectors.

%
% Copyright � 2012 Peeter Joot.  All Rights Reserved.
% Licenced as described in the file LICENSE under the root directory of this GIT repository.
%

%
%
%\documentclass[]{eliblog}

\usepackage{amsmath}
\usepackage{mathpazo}

%
% shorthand for bold symbols, convenient for vectors and matrices
%
\newcommand{\Ba}[0]{\mathbf{a}}
\newcommand{\Bb}[0]{\mathbf{b}}
\newcommand{\Bc}[0]{\mathbf{c}}
\newcommand{\Bd}[0]{\mathbf{d}}
\newcommand{\Be}[0]{\mathbf{e}}
\newcommand{\Bf}[0]{\mathbf{f}}
\newcommand{\Bg}[0]{\mathbf{g}}
\newcommand{\Bh}[0]{\mathbf{h}}
\newcommand{\Bi}[0]{\mathbf{i}}
\newcommand{\Bj}[0]{\mathbf{j}}
\newcommand{\Bk}[0]{\mathbf{k}}
\newcommand{\Bl}[0]{\mathbf{l}}
\newcommand{\Bm}[0]{\mathbf{m}}
\newcommand{\Bn}[0]{\mathbf{n}}
\newcommand{\Bo}[0]{\mathbf{o}}
\newcommand{\Bp}[0]{\mathbf{p}}
\newcommand{\Bq}[0]{\mathbf{q}}
\newcommand{\Br}[0]{\mathbf{r}}
\newcommand{\Bs}[0]{\mathbf{s}}
\newcommand{\Bt}[0]{\mathbf{t}}
\newcommand{\Bu}[0]{\mathbf{u}}
\newcommand{\Bv}[0]{\mathbf{v}}
\newcommand{\Bw}[0]{\mathbf{w}}
\newcommand{\Bx}[0]{\mathbf{x}}
\newcommand{\By}[0]{\mathbf{y}}
\newcommand{\Bz}[0]{\mathbf{z}}
\newcommand{\BA}[0]{\mathbf{A}}
\newcommand{\BB}[0]{\mathbf{B}}
\newcommand{\BC}[0]{\mathbf{C}}
\newcommand{\BD}[0]{\mathbf{D}}
\newcommand{\BE}[0]{\mathbf{E}}
\newcommand{\BF}[0]{\mathbf{F}}
\newcommand{\BG}[0]{\mathbf{G}}
\newcommand{\BH}[0]{\mathbf{H}}
\newcommand{\BI}[0]{\mathbf{I}}
\newcommand{\BJ}[0]{\mathbf{J}}
\newcommand{\BK}[0]{\mathbf{K}}
\newcommand{\BL}[0]{\mathbf{L}}
\newcommand{\BM}[0]{\mathbf{M}}
\newcommand{\BN}[0]{\mathbf{N}}
\newcommand{\BO}[0]{\mathbf{O}}
\newcommand{\BP}[0]{\mathbf{P}}
\newcommand{\BQ}[0]{\mathbf{Q}}
\newcommand{\BR}[0]{\mathbf{R}}
\newcommand{\BS}[0]{\mathbf{S}}
\newcommand{\BT}[0]{\mathbf{T}}
\newcommand{\BU}[0]{\mathbf{U}}
\newcommand{\BV}[0]{\mathbf{V}}
\newcommand{\BW}[0]{\mathbf{W}}
\newcommand{\BX}[0]{\mathbf{X}}
\newcommand{\BY}[0]{\mathbf{Y}}
\newcommand{\BZ}[0]{\mathbf{Z}}

\newcommand{\Bzero}[0]{\mathbf{0}}
\newcommand{\Btheta}[0]{\boldsymbol{\theta}}
\newcommand{\Btau}[0]{\boldsymbol{\tau}}
\newcommand{\Bomega}[0]{\boldsymbol{\omega}}

%
% shorthand for unit vectors
%
\newcommand{\acap}[0]{\hat{\Ba}}
\newcommand{\bcap}[0]{\hat{\Bb}}
\newcommand{\ccap}[0]{\hat{\Bc}}
\newcommand{\dcap}[0]{\hat{\Bd}}
\newcommand{\ecap}[0]{\hat{\Be}}
\newcommand{\fcap}[0]{\hat{\Bf}}
\newcommand{\gcap}[0]{\hat{\Bg}}
\newcommand{\hcap}[0]{\hat{\Bh}}
\newcommand{\icap}[0]{\hat{\Bi}}
\newcommand{\jcap}[0]{\hat{\Bj}}
\newcommand{\kcap}[0]{\hat{\Bk}}
\newcommand{\lcap}[0]{\hat{\Bl}}
\newcommand{\mcap}[0]{\hat{\Bm}}
\newcommand{\ncap}[0]{\hat{\Bn}}
\newcommand{\ocap}[0]{\hat{\Bo}}
\newcommand{\pcap}[0]{\hat{\Bp}}
\newcommand{\qcap}[0]{\hat{\Bq}}
\newcommand{\rcap}[0]{\hat{\Br}}
\newcommand{\scap}[0]{\hat{\Bs}}
\newcommand{\tcap}[0]{\hat{\Bt}}
\newcommand{\ucap}[0]{\hat{\Bu}}
\newcommand{\vcap}[0]{\hat{\Bv}}
\newcommand{\wcap}[0]{\hat{\Bw}}
\newcommand{\xcap}[0]{\hat{\Bx}}
\newcommand{\ycap}[0]{\hat{\By}}
\newcommand{\zcap}[0]{\hat{\Bz}}
\newcommand{\thetacap}[0]{\hat{\Btheta}}

%
% to write R^n and C^n in a distinguishable fashion.  Perhaps change this
% to the double lined characters upon figuring out how to do so.
%
\newcommand{\C}[1]{$\mathbb{C}^{#1}$}
\newcommand{\R}[1]{$\mathbb{R}^{#1}$}

%
% various generally useful helpers
%

% derivative of #1 wrt. #2:
\newcommand{\D}[2] {\frac {d#2} {d#1}}

\newcommand{\inv}[1]{\frac{1}{#1}}
\newcommand{\cross}[0]{\times}

\newcommand{\abs}[1]{\lvert{#1}\rvert}
\newcommand{\norm}[1]{\lVert{#1}\rVert}
\newcommand{\innerprod}[2]{\langle{#1}, {#2}\rangle}
\newcommand{\dotprod}[2]{{#1} \cdot {#2}}
\newcommand{\bdotprod}[2]{\left({#1} \cdot {#2}\right)}
\newcommand{\crossprod}[2]{{#1} \cross {#2}}
\newcommand{\tripleprod}[3]{\dotprod{\left(\crossprod{#1}{#2}\right)}{#3}}

\DeclareMathOperator{\Proj}{Proj}
\DeclareMathOperator{\Span}{span}
\DeclareMathOperator{\Sgn}{sgn}
\DeclareMathOperator{\Area}{Area}
\DeclareMathOperator{\Volume}{Volume}

%
% A few miscellaneous things specific to this document
%
\newcommand{\crossop}[1]{\crossprod{#1}{}}

% R2 vector.
\newcommand{\VectorTwo}[2]{
\begin{bmatrix}
 {#1} \\
 {#2}
\end{bmatrix}
}

\newcommand{\VectorN}[1]{
\begin{bmatrix}
{#1}_1 \\
{#1}_2 \\
\vdots \\
{#1}_N \\
\end{bmatrix}
}

\newcommand{\DETuvij}[4]{
\begin{vmatrix}
 {#1}_{#3} & {#1}_{#4} \\
 {#2}_{#3} & {#2}_{#4}
\end{vmatrix}
}

\newcommand{\DETuvwijk}[6]{
\begin{vmatrix}
 {#1}_{#4} & {#1}_{#5} & {#1}_{#6} \\
 {#2}_{#4} & {#2}_{#5} & {#2}_{#6} \\
 {#3}_{#4} & {#3}_{#5} & {#3}_{#6}
\end{vmatrix}
}

\newcommand{\DETuvwxijkl}[8]{
\begin{vmatrix}
 {#1}_{#5} & {#1}_{#6} & {#1}_{#7} & {#1}_{#8} \\
 {#2}_{#5} & {#2}_{#6} & {#2}_{#7} & {#2}_{#8} \\
 {#3}_{#5} & {#3}_{#6} & {#3}_{#7} & {#3}_{#8} \\
 {#4}_{#5} & {#4}_{#6} & {#4}_{#7} & {#4}_{#8} \\
\end{vmatrix}
}

%\newcommand{\DETuvwxyijklm}[10]{
%\begin{vmatrix}
% {#1}_{#6} & {#1}_{#7} & {#1}_{#8} & {#1}_{#9} & {#1}_{#10} \\
% {#2}_{#6} & {#2}_{#7} & {#2}_{#8} & {#2}_{#9} & {#2}_{#10} \\
% {#3}_{#6} & {#3}_{#7} & {#3}_{#8} & {#3}_{#9} & {#3}_{#10} \\
% {#4}_{#6} & {#4}_{#7} & {#4}_{#8} & {#4}_{#9} & {#4}_{#10} \\
% {#5}_{#6} & {#5}_{#7} & {#5}_{#8} & {#5}_{#9} & {#5}_{#10}
%\end{vmatrix}
%}

% R3 vector.
\newcommand{\VectorThree}[3]{
\begin{bmatrix}
 {#1} \\
 {#2} \\
 {#3}
\end{bmatrix}
}



\author{Peeter Joot}
\email{peeter.joot@gmail.com}


\chapter{Comparison of many traditional vector and GA identities}
\index{identities}
\label{chap:gaWiki}


% does not work with _ character:
%%\blogpage{http://sites.google.com/site/peeterjoot/geometric-algebra/ga_wiki.pdf}
%%\date{ Oct 13, 2007 }
%\date{Oct 13, 2007.  gaWiki.tex}
%%\revisionInfo{\(RCSfile: gaWiki.tex,v \) Last \(Revision: 1.16 \) \(Date: 2009/10/22 02:07:20 \)}

\beginArtNoToc

\section{Three dimensional vector relationships vs N dimensional equivalents}

Here are some comparisons between standard \({\mathbb R}^3\) vector relations and their corresponding wedge and geometric product equivalents.  All the wedge and geometric product equivalents here are good for more than three dimensions, and some also for two.  In two dimensions the cross product is undefined even if what it describes (like torque) is a perfectly well defined in a plane without introducing an arbitrary normal vector outside of the space.

Many of these relationships only require the introduction of the wedge product to generalize, but since that may not be familiar to somebody with only a traditional background in vector algebra and calculus, some examples are given.

\subsection{wedge and cross products are antisymmetric}
\begin{equation}\label{eqn:gaWiki:20}
\begin{aligned}
\Bv \times \Bu = - (\Bu \times \Bv)
\end{aligned}
\end{equation}
\begin{equation}\label{eqn:gaWiki:40}
\begin{aligned}
\Bv \wedge \Bu = - (\Bu \wedge \Bv)
\end{aligned}
\end{equation}

\subsection{wedge and cross products are zero when identical}
\begin{equation}\label{eqn:gaWiki:60}
\begin{aligned}
\Bu \times \Bu = 0
\end{aligned}
\end{equation}
\begin{equation}\label{eqn:gaWiki:80}
\begin{aligned}
\Bu \wedge \Bu = 0
\end{aligned}
\end{equation}

\subsection{wedge and cross products are linear}

These are both linear in the first variable
\begin{equation}\label{eqn:gaWiki:100}
\begin{aligned}
(\Bv + \Bw) \times \Bw = \Bu \times \Bw + \Bv \times \Bw
\end{aligned}
\end{equation}
\begin{equation}\label{eqn:gaWiki:120}
\begin{aligned}
(\Bv + \Bw) \wedge \Bw = \Bu \wedge \Bw + \Bv \wedge \Bw
\end{aligned}
\end{equation}

and are linear in the second variable
\begin{equation}\label{eqn:gaWiki:140}
\begin{aligned}
\Bu \times (\Bv + \Bw)= \Bu \times \Bv + \Bu \times \Bw
\end{aligned}
\end{equation}
\begin{equation}\label{eqn:gaWiki:160}
\begin{aligned}
\Bu \wedge (\Bv + \Bw)= \Bu \wedge \Bv + \Bu \wedge \Bw
\end{aligned}
\end{equation}

\subsection{In general, cross product is not associative, but the wedge product is}
\begin{equation}\label{eqn:gaWiki:180}
\begin{aligned}
(\Bu \times \Bv) \times \Bw \neq \Bu \times (\Bv \times \Bw)
\end{aligned}
\end{equation}
\begin{equation}\label{eqn:gaWiki:200}
\begin{aligned}
(\Bu \wedge \Bv) \wedge \Bw = \Bu \wedge (\Bv \wedge \Bw)
\end{aligned}
\end{equation}

\subsection{Wedge and cross product relationship to a plane}
\(\Bu \times \Bv\) is perpendicular to plane containing \(\Bu\) and \(\Bv\).
\(\Bu \wedge \Bv\) is an oriented representation of the plane containing \(\Bu\) and \(\Bv\).

\subsection{norm of a vector}

The norm (length) of a vector is defined in terms of the dot product

\begin{equation}\label{eqn:gaWiki:220}
\begin{aligned}
 {\Vert \Bu \Vert}^2 = \Bu \cdot \Bu
\end{aligned}
\end{equation}

Using the geometric product this is also true, but this can be also be expressed more compactly as

\begin{equation}\label{eqn:gaWiki:240}
\begin{aligned}
{\Vert \Bu \Vert}^2 = {\Bu}^2
\end{aligned}
\end{equation}

This follows from the definition of the geometric product and the fact that a vector wedge product with itself is zero

\begin{equation}\label{eqn:gaWiki:260}
\begin{aligned}
 \Bu \, \Bu = \Bu \cdot \Bu + \Bu \wedge \Bu = \Bu \cdot \Bu
\end{aligned}
\end{equation}

\subsection{Lagrange identity}
\index{Lagrange identity}

In three dimensions the product of two vector lengths can be expressed in terms of the dot and cross products

\begin{equation}\label{eqn:gaWiki:280}
\begin{aligned}
{\Vert \Bu  \Vert}^2 {\Vert \Bv  \Vert}^2
=
({\Bu  \cdot \Bv })^2 + {\Vert \Bu  \times \Bv  \Vert}^2
\end{aligned}
\end{equation}

The corresponding generalization expressed using the geometric product is

\begin{equation}\label{eqn:gaWiki:300}
\begin{aligned}
{\Vert \Bu  \Vert}^2 {\Vert \Bv  \Vert}^2
= ({\Bu  \cdot \Bv })^2 - (\Bu  \wedge \Bv )^2
\end{aligned}
\end{equation}

This follows from by expanding the geometric product of a pair of vectors with its reverse

\begin{equation}\label{eqn:gaWiki:320}
\begin{aligned}
(\Bu  \Bv )(\Bv  \Bu )
= ({\Bu  \cdot \Bv } + {\Bu  \wedge \Bv }) ({\Bu  \cdot \Bv } - {\Bu  \wedge \Bv })
\end{aligned}
\end{equation}

\subsection{determinant expansion of cross and wedge products}
\index{wedge!determinant expansion}

\begin{equation}\label{eqn:gaWiki:340}
\begin{aligned}
\Bu \times \Bv = \sum_{i<j}{ \begin{vmatrix}u_i & u_j\\v_i & v_j\end{vmatrix}  {\Be}_i \times {\Be}_j }
\end{aligned}
\end{equation}
\begin{equation}\label{eqn:gaWiki:360}
\begin{aligned}
\Bu \wedge \Bv = \sum_{i<j}{ \begin{vmatrix}u_i & u_j\\v_i & v_j\end{vmatrix}  {\Be}_i \wedge {\Be}_j }
\end{aligned}
\end{equation}

Without justification or historical context, traditional linear algebra texts will often define the determinant as the first step of an elaborate sequence of definitions and theorems leading up to the solution of linear systems, Cramer's rule and matrix inversion.

An alternative treatment is to axiomatically introduce the wedge product, and then demonstrate that this can be used directly to solve linear systems.  This is shown below, and does not require sophisticated math skills to understand.

It is then possible to define determinants as nothing more than the coefficients of the wedge product in terms of "unit k-vectors" (\({\Be}_i \wedge {\Be}_j\) terms) expansions as above.

A one by one determinant is the coefficient of \(\Be _1\) for an \(\mathbb R^1\) 1-vector.

A two-by-two determinant is the coefficient of \(\Be _1 \wedge \Be _2\) for an \(\mathbb R^2\) bivector

A three-by-three determinant is the coefficient of \(\Be _1 \wedge \Be _2 \wedge \Be _3\) for an \(\mathbb R^3\) trivector

When linear system solution is introduced via the wedge product, Cramer's rule follows as a side effect, and there is no need to lead up to the end results with definitions of minors, matrices, matrix invertablity, adjoints, cofactors, Laplace expansions, theorems on determinant multiplication and row column exchanges, and so forth.

\subsection{Equation of a plane}
\index{plane!equation}

For the plane of all points \({\Br}\) through the plane passing through three independent points \({\Br}_0\), \({\Br}_1\), and \({\Br}_2\), the normal form of the equation is

\begin{equation}\label{eqn:gaWiki:380}
\begin{aligned}
(({\Br}_2 - {\Br}_0) \times ({\Br}_1 - {\Br}_0)) \cdot ({\Br} - {\Br}_0) = 0
\end{aligned}
\end{equation}

The equivalent wedge product equation is
\begin{equation}\label{eqn:gaWiki:400}
\begin{aligned}
({\Br}_2 - {\Br}_0) \wedge ({\Br}_1 - {\Br}_0) \wedge ({\Br} - {\Br}_0) = 0
\end{aligned}
\end{equation}

\subsection{Projective and rejective components of a vector}

For three dimensions the projective and rejective components of a vector with respect to an arbitrary non-zero unit vector, can be expressed in terms of the dot and cross product

\begin{equation}\label{eqn:gaWiki:420}
\begin{aligned}
\Bv = (\Bv \cdot \ucap)\ucap + \ucap \times (\Bv \times \ucap)
\end{aligned}
\end{equation}

For the general case the same result can be written in terms of the dot and wedge product and the geometric product of that and the unit vector

\begin{equation}\label{eqn:gaWiki:440}
\begin{aligned}
\Bv = (\Bv \cdot \ucap)\ucap + (\Bv \wedge \ucap) \ucap
\end{aligned}
\end{equation}

It is also worthwhile to point out that this result can also be expressed using right or left vector division as defined by the geometric product

\begin{equation}\label{eqn:gaWiki:460}
\begin{aligned}
\Bv = (\Bv \cdot \Bu)\frac{1}{\Bu} + (\Bv \wedge \Bu) \frac{1}{\Bu}
\end{aligned}
\end{equation}
\begin{equation}\label{eqn:gaWiki:480}
\begin{aligned}
\Bv = \frac{1}{\Bu}(\Bu \cdot \Bv) + \frac{1}{\Bu}(\Bu \wedge \Bv)
\end{aligned}
\end{equation}

\subsection{Area (squared) of a parallelogram is norm of cross product}
\index{parallelogram!area}

\begin{equation}\label{eqn:gaWiki:500}
\begin{aligned}
A^2 = {\Vert \Bu \times \Bv \Vert}^2 = \sum_{i<j}{\begin{vmatrix}u_i & u_j\\v_i & v_j\end{vmatrix}}^2
\end{aligned}
\end{equation}

and is the negated square of a wedge product
\begin{equation}\label{eqn:gaWiki:520}
\begin{aligned}
A^2 = -(\Bu \wedge \Bv)^2 = \sum_{i<j}{\begin{vmatrix}u_i & u_j\\v_i & v_j\end{vmatrix}}^2
\end{aligned}
\end{equation}

Note that this squared bivector is a geometric product.

\subsection{Angle between two vectors}
\index{vectors!angle between}

\begin{equation}\label{eqn:gaWiki:540}
\begin{aligned}
({\sin \theta})^2 = \frac{{\Vert \Bu \times \Bv \Vert}^2}{{\Vert \Bu \Vert}^2 {\Vert \Bv \Vert}^2}
\end{aligned}
\end{equation}
\begin{equation}\label{eqn:gaWiki:560}
\begin{aligned}
({\sin \theta})^2 = -\frac{(\Bu \wedge \Bv)^2}{{ \Bu }^2 { \Bv }^2}
\end{aligned}
\end{equation}

\subsection{Volume of the parallelepiped formed by three vectors}
\index{parallelepiped!volume}

\begin{equation}\label{eqn:gaWiki:580}
\begin{aligned}
V^2 = {\Vert (\Bu \times \Bv) \cdot \Bw \Vert}^2
= {
\begin{vmatrix}
u_1 & u_2 & u_3 \\
v_1 & v_2 & v_3 \\
w_1 & w_2 & w_3 \\
\end{vmatrix}
}^2
\end{aligned}
\end{equation}

\begin{equation}\label{eqn:gaWiki:600}
\begin{aligned}
V^2 = -(\Bu \wedge \Bv \wedge \Bw)^2
= -\left(\sum_{i<j<k}
\begin{vmatrix}
u_i & u_j & u_k \\
v_i & v_j & v_k \\
w_i & w_j & w_k \\
\end{vmatrix}
\ecap_i \wedge \ecap_j \wedge \ecap_k
\right)^2
= \sum_{i<j<k}
{
\begin{vmatrix}
u_i & u_j & u_k \\
v_i & v_j & v_k \\
w_i & w_j & w_k \\
\end{vmatrix}
}^2
\end{aligned}
\end{equation}


\section{Some properties and examples}

Some fundamental geometric algebra manipulations will be provided below, showing how this vector product can be used in calculation of projections, area, and rotations.  How some of these tie together and correlate concepts from other branches of mathematics, such as complex numbers, will also be shown.

In some cases these examples provide details used above in the cross product and geometric product comparisons.

\subsection{Inversion of a vector}
\index{vector!inversion}

One of the powerful properties of the Geometric product is that it provides the capability to express the inverse of a non-zero vector.  This is expressed by:

\begin{equation}\label{eqn:gaWiki:620}
\begin{aligned}
{\Ba}^{-1} = \frac{\Ba}{\Ba \Ba} = \frac{\Ba}{{\Vert \Ba \Vert}^2}.
\end{aligned}
\end{equation}

\subsection{dot and wedge products defined in terms of the geometric product}

Given a definition of the geometric product in terms of the dot and wedge products, adding and subtracting \(\Ba  \Bb \) and \(\Bb  \Ba \) demonstrates that the dot and wedge product of two vectors can also be defined in terms of the geometric product

\subsection{The dot product}
\index{dot product}

\begin{equation}\label{eqn:gaWiki:640}
\begin{aligned}
\Ba \cdot\Bb  = \frac{1}{2}(\Ba \Bb  + \Bb \Ba )
\end{aligned}
\end{equation}

This is the symmetric component of the geometric product.  When two vectors are colinear the geometric and dot products of those vectors are equal.

As a motivation for the dot product it is normal to show that this quantity occurs in the solution of the length of a general triangle where the third side is the vector sum of the first and second sides \(\Bc  = \Ba  + \Bb \).

\begin{equation}\label{eqn:gaWiki:660}
\begin{aligned}
{\Vert \Bc  \Vert}^2 = \sum_{i}(a_i + b_i)^2 = {\Vert \Ba  \Vert}^2 + {\Vert \Bb  \Vert}^2 + 2 \sum_{i}a_i b_i
\end{aligned}
\end{equation}

The last sum is then given the name the dot product and other properties of this quantity are then shown (projection, angle between vectors, ...).

This can also be expressed using the geometric product

\begin{equation}\label{eqn:gaWiki:680}
\begin{aligned}
\Bc ^2 = (\Ba  + \Bb )(\Ba  + \Bb ) = \Ba ^2 + \Bb ^2 + (\Ba \Bb  + \Bb \Ba )
\end{aligned}
\end{equation}

By comparison, the following equality exists

\begin{equation}\label{eqn:gaWiki:700}
\begin{aligned}
\sum_{i}a_i b_i = \frac{1}{2}(\Ba \Bb  + \Bb \Ba )
\end{aligned}
\end{equation}

Without requiring expansion by components one can define the dot product exclusively in terms of the geometric product due to its properties of contraction, distribution and associativity.  This is arguably a more natural way to define the geometric product.  Addition of two similar terms is not immediately required, especially since one of those terms is the wedge product which may also be unfamiliar.

\subsection{The wedge product}
\index{wedge product}

\begin{equation}\label{eqn:gaWiki:720}
\begin{aligned}
\Ba \wedge\Bb  = \frac{1}{2}(\Ba \Bb  - \Bb \Ba )
\end{aligned}
\end{equation}

This is the antisymmetric component of the geometric product.  When two vectors are orthogonal the geometric and wedge products of those vectors are equal.

Switching the order of the vectors negates this antisymmetric geometric product component, and contraction property shows that this is zero if the vectors are equal.  These are the defining properties of the wedge product.

\subsection{Note on symmetric and antisymmetric dot and wedge product formulas}
\index{dot product!symmetric sum}
\index{wedge product!antisymmetric sum}

A generalization of the dot product that allows computation of the component of a vector "in the direction" of a plane (bivector), or other k-vectors can be found below.  Since the signs change depending on the grades of the terms being multiplied, care is required with the formulas above to ensure that they are only used for a pair of vectors.

\subsection{Reversing multiplication order.  Dot and wedge products compared to the real and imaginary parts of a complex number}

Reversing the order of multiplication of two vectors, has the effect of the inverting the sign of just the wedge product term of the product.

It is not a coincidence that this is a similar operation to the conjugate operation of complex numbers.

The reverse of a product is written in the following fashion

\begin{equation}\label{eqn:gaWiki:740}
\begin{aligned}
{\Bb  \Ba } = ({\Ba  \Bb })^\dagger
\end{aligned}
\end{equation}
\begin{equation}\label{eqn:gaWiki:760}
\begin{aligned}
{\Bc  \Bb  \Ba } = ({\Ba  \Bb  \Bc })^\dagger
\end{aligned}
\end{equation}

Expressed this way the dot and wedge products are

\begin{equation}\label{eqn:gaWiki:780}
\begin{aligned}
\Ba \cdot\Bb  = \frac{1}{2}(\Ba \Bb  + ({\Ba  \Bb })^\dagger)
\end{aligned}
\end{equation}

This is the symmetric component of the geometric product.  When two vectors are colinear the geometric and dot products of those vectors are equal.

\begin{equation}\label{eqn:gaWiki:800}
\begin{aligned}
\Ba \wedge\Bb  = \frac{1}{2}(\Ba \Bb  - ({\Ba  \Bb })^\dagger)
\end{aligned}
\end{equation}

These symmetric and antisymmetric pairs, the dot and wedge products extract the scalar and bivector components of a geometric product in the same fashion as the real and imaginary components of a complex number are also extracted by its symmetric and antisymmetric components

\begin{equation}\label{eqn:gaWiki:820}
\begin{aligned}
\mathop{Re}(z) = \frac{1}{2}(z + \overbar{z})
\end{aligned}
\end{equation}
\begin{equation}\label{eqn:gaWiki:840}
\begin{aligned}
\mathop{Im}(z) = \frac{1}{2}(z - \overbar{z})
\end{aligned}
\end{equation}

This extraction of components also applies to higher order geometric product terms.  For example

\begin{equation}\label{eqn:gaWiki:860}
\begin{aligned}
\Ba \wedge\Bb \wedge \Bc
= \frac{1}{2}(\Ba \Bb \Bc  - ({\Ba  \Bb } \Bc )^\dagger)
= \frac{1}{2}(\Bb \Bc \Ba  - ({\Bb  \Bc } \Ba )^\dagger)
= \frac{1}{2}(\Bc \Ba \Bb  - ({\Bc  \Ba } \Bb )^\dagger)
\end{aligned}
\end{equation}

\subsection{Orthogonal decomposition of a vector}

Using the \textAndIndex{Gram-Schmidt} process a single vector can be decomposed into two components with respect to a reference vector, namely the projection onto a unit vector in a reference direction, and the difference between the vector and that projection.

With, \( \ucap = \Bu / {\Vert \Bu \Vert}\), the projection of \(\Bv\) onto \( \ucap\) is

\begin{equation}\label{eqn:gaWiki:880}
\begin{aligned}
 \mathrm{Proj}_{\ucap}\,\Bv  = \ucap (\ucap \cdot \Bv)
\end{aligned}
\end{equation}

Orthogonal to that vector is the difference, designated the rejection,

\begin{equation}\label{eqn:gaWiki:900}
\begin{aligned}
 \Bv - \ucap (\ucap \cdot \Bv) = \frac{1}{{\Vert \Bu \Vert}^2} ( {\Vert \Bu \Vert}^2 \Bv - \Bu (\Bu \cdot \Bv))
\end{aligned}
\end{equation}

The rejection can be expressed as a single geometric algebraic product in a few different ways

\begin{equation}\label{eqn:gaWiki:920}
\begin{aligned}
 \frac{ \Bu }{{\Bu}^2} ( \Bu \Bv - \Bu \cdot \Bv)
= \frac{1}{\Bu} ( \Bu \wedge \Bv )
= \ucap ( \ucap \wedge \Bv )
= ( \Bv \wedge \ucap ) \ucap
\end{aligned}
\end{equation}

The similarity in form between between the projection and the rejection is notable.  The sum of these recovers the original vector

\begin{equation}\label{eqn:gaWiki:940}
\begin{aligned}%\label{eqn:gaWiki:orthoD}
 \Bv = \ucap (\ucap \cdot \Bv) + \ucap ( \ucap \wedge \Bv )
\end{aligned}
\end{equation}

Here the projection is in its customary vector form.  An alternate formulation is possible that puts the projection in a form that differs from the usual vector formulation

\begin{equation}\label{eqn:gaWiki:960}
\begin{aligned}
 \Bv
= \frac{1}{\Bu} (\Bu \cdot \Bv) + \frac{1}{\Bu} ( \Bu \wedge \Bv )
= (\Bv \cdot \Bu) \frac{1}{\Bu}  + ( \Bv \wedge \Bu ) \frac{1}{\Bu}
\end{aligned}
\end{equation}

\subsection{A quicker way to the end result}

Working backwards from the end result, it can be observed that this orthogonal decomposition result can in fact follow more directly from the definition of the geometric product itself.

\begin{equation}\label{eqn:gaWiki:980}
\begin{aligned}
\Bv = \ucap \ucap \Bv
= \ucap (\ucap \cdot \Bv + \ucap \wedge \Bv )
\end{aligned}
\end{equation}

With this approach, the original geometrical consideration is not necessarily obvious, but it is a much quicker way to get at the same algebraic result.

However, the hint that one can work backwards, coupled with the knowledge that the wedge product can be used to solve sets of linear equations,
\footnote{
http://www.grassmannalgebra.info/grassmannalgebra/book/bookpdf/TheExteriorProduct.pdf}
the problem of orthogonal decomposition can be posed directly,

Let \(\Bv = a \Bu + \Bx\), where \(\Bu \cdot \Bx = 0\).  To discard the portions of \(\Bv\) that are colinear with \(\Bu\), take the wedge product

\begin{equation}\label{eqn:gaWiki:1000}
\begin{aligned}
\Bu \wedge \Bv = \Bu \wedge (a \Bu + \Bx) = \Bu \wedge \Bx
\end{aligned}
\end{equation}

Here the geometric product can be employed

\begin{equation}\label{eqn:gaWiki:1020}
\begin{aligned}
\Bu \wedge \Bv = \Bu \wedge \Bx = \Bu \Bx - \Bu \cdot \Bx = \Bu \Bx
\end{aligned}
\end{equation}

Because the geometric product is invertible, this can be solved for x

\begin{equation}\label{eqn:gaWiki:1040}
\begin{aligned}
\Bx = \frac{1}{\Bu}(\Bu \wedge \Bv)
\end{aligned}
\end{equation}

The same techniques can be applied to similar problems, such as calculation of the component of a vector in a plane and perpendicular to the plane.

\subsection{Area of parallelogram spanned by two vectors}
\index{parallelogram!area}

\imageFigure{../../figures/gabook/parallelogramArea}{parallelogramArea}{fig:parallelogramArea}{0.4}

As depicted in \cref{fig:parallelogramArea}, one can see that the area of a parallelogram spanned by two vectors is computed from the base times height.  In the figure \(\Bu\) was picked as the base, with length \(\Norm{\Bu}\).  Designating the second vector \(\Bv\), we want the component of \(\Bv\) perpendicular to \(\ucap\) for the height.  An orthogonal decomposition of \(\Bv\) into directions parallel and perpendicular to \(\ucap\) can be performed in two ways.

\begin{equation}\label{eqn:gaWiki:1060}
\begin{aligned}
\Bv &= \Bv \ucap \ucap = (\Bv \cdot \ucap) \ucap + (\Bv \wedge \ucap) \ucap \\
    &= \ucap \ucap \Bv = \ucap (\ucap \cdot \Bv) + \ucap (\ucap \wedge \Bv)
\end{aligned}
\end{equation}

The height is the length of the perpendicular component expressed using the wedge as either \(\ucap (\ucap \wedge \Bv)\) or \((\Bv \wedge \ucap) \ucap\).

Multiplying base times height we have the parallelogram area

\begin{equation}\label{eqn:gaWiki:1080}
\begin{aligned}
A(\Bu,\Bv)
&= \Vert \Bu \Vert \Vert \ucap ( \ucap \wedge \Bv ) \Vert \\
&= \Vert \ucap ( \Bu \wedge \Bv ) \Vert \\
\end{aligned}
\end{equation}

Since the squared length of an Euclidean vector is the geometric square of that vector, we can compute the squared area of this parallogram by squaring this single scaled vector

\begin{equation}\label{eqn:gaWiki:1100}
\begin{aligned}
A^2 &= (\ucap ( \Bu \wedge \Bv ) )^2
\end{aligned}
\end{equation}

Utilizing both encodings of the perpendicular to \(\ucap\) component of \(\Bv\) computed above we have for the squared area

\begin{equation}\label{eqn:gaWiki:1120}
\begin{aligned}
A^2
&= (\ucap( \Bu \wedge {\Bv} ) )^2 \\
&= (( \Bv \wedge {\Bu} ) \ucap) (\ucap ( {\Bu} \wedge \Bv )) \\
&= ( \Bv \wedge \Bu ) ( \Bu \wedge \Bv ) \\
\end{aligned}
\end{equation}

Since \(\Bu \wedge \Bv = -\Bv \wedge \Bu\), we have finally

\begin{equation}\label{eqn:gaWiki:1140}
\begin{aligned}
A^2 = -( \Bu \wedge \Bv )^2
\end{aligned}
\end{equation}

There are a few things of note here.  One is that the parallelogram area can easily be expressed in terms of the square of a bivector.  Another is that the square of a bivector has the same property as a purely imaginary number, a negative square.

It can also be noted that a vector lying completely within a plane anticommutes with the bivector for that plane.  More generally components of vectors that lie within a plane commute with the bivector for that plane while the perpendicular components of that vector commute.  These commutation or anticommutation properties depend both on the vector and the grade of the object that one attempts to commute it with (these properties lie behind the generalized definitions of the dot and wedge product to be seen later).

% SCOTT:
% - Section 3.2.9. This is more of a comment. The commuting of the geometric product in the second line of the equation for A^2 uses the idea that u_hat is in the plane of the bivector, therefore the wedge product is zero. It would be nice if this reasoning were given. In fact, I think it would be very benificial if there were a section near the beginning that pretty much laid out what is possible with the geometric product and when. Maybe a table of sorts with columns signifying that when you fit certain criteria, orthogonal/parallel/in a plane/etc, that various properties like commutation/associativity/etc work.

\subsection{Expansion of a bivector and a vector rejection in terms of the standard basis}
\index{rejection}

If a vector is factored directly into projective and rejective terms using the geometric product \(\Bv = \frac{1}{\Bu}( \Bu \cdot \Bv + \Bu \wedge \Bv)\), then it is not necessarily obvious that the rejection term, a product of vector and bivector is even a vector.  Expansion of the vector bivector product in terms of the standard basis vectors has the following form

Let
\begin{equation}\label{eqn:gaWiki:1160}
\begin{aligned}
\Br
= \frac{1}{\Bu} ( \Bu \wedge \Bv )
= \frac{\Bu}{\Bu^2} ( \Bu \wedge \Bv )
= \frac{1}{{\Vert \Bu \Vert}^2} \Bu ( \Bu \wedge \Bv )
\end{aligned}
\end{equation}

It can be shown that
\begin{equation}\label{eqn:gaWiki:1180}
\begin{aligned}
\Br = \frac{1}{{\Vert{\Bu}\Vert}^2} \sum_{i<j}\begin{vmatrix}u_i & u_j\\v_i & v_j\end{vmatrix}
\begin{vmatrix}u_i & u_j\\ {\Be}_i & {\Be}_j\end{vmatrix}
\end{aligned}
\end{equation}

(a result that can be shown more easily straight from \(\Br = \Bv - \ucap (\ucap \cdot \Bv)\)).

The rejective term is perpendicular to \(\Bu\), since
$\begin{vmatrix}
u_i & u_j\\ u_i & u_j
\end{vmatrix}
 = 0$
implies \(\Br \cdot \Bu = \Bzero\).

The magnitude of \(\Br\), is

\begin{equation}\label{eqn:gaWiki:1200}
\begin{aligned}
{\Vert \Br \Vert}^2 = \Br \cdot \Bv = \frac{1}{{\Vert{\Bu}\Vert}^2} \sum_{i<j}\begin{vmatrix}u_i & u_j\\v_i & v_j\end{vmatrix}^2
\end{aligned}
\end{equation}.

So, the quantity

\begin{equation}\label{eqn:gaWiki:1220}
\begin{aligned}
{\Vert \Br \Vert}^2 {\Vert{\Bu}\Vert}^2 = \sum_{i<j}\begin{vmatrix}u_i & u_j\\v_i & v_j\end{vmatrix}^2
\end{aligned}
\end{equation}

is the squared area of the parallelogram formed by \(\Bu\) and \(\Bv\).

It is also noteworthy that the bivector can be expressed as

\begin{equation}\label{eqn:gaWiki:1240}
\begin{aligned}
\Bu \wedge \Bv = \sum_{i<j}{ \begin{vmatrix}u_i & u_j\\v_i & v_j\end{vmatrix}  {\Be}_i \wedge {\Be}_j }
\end{aligned}
\end{equation}.

Thus is it natural, if one considers each term \({\Be}_i \wedge {\Be}_j\) as a basis vector of the bivector space, to define the (squared) "length" of that bivector as the (squared) area.

Going back to the geometric product expression for the length of the rejection \(\frac{1}{\Bu} ( \Bu \wedge \Bv )\) we see that the length of the quotient, a vector, is in this case is the "length" of the bivector divided by the length of the divisor.

This may not be a general result for the length of the product of two \(k\)-vectors, however it is a result that may help build some intuition about the significance of the algebraic operations.  Namely,

When a vector is divided out of the plane (parallelogram span) formed from it and another vector, what remains is the perpendicular component of the remaining vector, and its length is the planar area divided by the length of the vector that was divided out.

\subsection{Projection and rejection of a vector onto and perpendicular to a plane}
\index{plane!projection}
\index{plane!rejection}

Like vector projection and rejection, higher dimensional analogs of that calculation are also possible using the geometric product.

As an example, one can calculate the component of a vector perpendicular to a plane and the projection of that vector onto the plane.

Let \(\Bw = a \Bu + b \Bv + \Bx\), where \(\Bu \cdot \Bx = \Bv \cdot \Bx = 0\).  As above, to discard the portions of \(\Bw\) that are colinear with \(\Bu\) or \(\Bu\), take the wedge product

\begin{equation}\label{eqn:gaWiki:1260}
\begin{aligned}
\Bw \wedge \Bu \wedge \Bv = (a \Bu + b \Bv + \Bx) \wedge \Bu \wedge \Bv = \Bx \wedge \Bu \wedge \Bv
\end{aligned}
\end{equation}

Having done this calculation with a vector projection, one can guess that this quantity equals \(\Bx (\Bu \wedge \Bv)\).  One can also guess there is a vector and bivector dot product like quantity such that the allows the calculation of the component of a vector that is in the "direction of a plane".  Both of these guesses are correct, and the validating these facts is worthwhile.  However, skipping ahead slightly, this to be proved fact allows for a nice closed form solution of the vector component outside of the plane:

\begin{equation}\label{eqn:gaWiki:1280}
\begin{aligned}
\Bx
= (\Bw \wedge \Bu \wedge \Bv)\frac{1}{\Bu \wedge \Bv}
= \frac{1}{\Bu \wedge \Bv}(\Bu \wedge \Bv  \wedge \Bw)
\end{aligned}
\end{equation}

Notice the similarities between this planar rejection result a the vector rejection result.  To calculation the component of a vector outside of a plane we take the volume spanned by three vectors (trivector) and "divide out" the plane.

Independent of any use of the geometric product it can be shown that this rejection in terms of the standard basis is

\begin{equation}\label{eqn:gaWiki:1300}
\begin{aligned}
\Bx = \frac{1}{(A_{u,v})^2} \sum_{i<j<k}
\begin{vmatrix}w_i & w_j & w_k \\u_i & u_j & u_k \\v_i & v_j & v_k \\\end{vmatrix}
\begin{vmatrix}u_i & u_j & u_k \\v_i & v_j & v_k \\ {\Be}_i & {\Be}_j & {\Be}_k \\ \end{vmatrix}
\end{aligned}
\end{equation}

Where

\begin{equation}\label{eqn:gaWiki:1320}
\begin{aligned}
(A_{u,v})^2
= \sum_{i<j} \begin{vmatrix}u_i & u_j\\v_i & v_j\end{vmatrix}
= -(\Bu \wedge \Bv)^2
\end{aligned}
\end{equation}

is the squared area of the parallelogram formed by \(\Bu\), and \(\Bv\).

The (squared) magnitude of \(\Bx\) is

\begin{equation}\label{eqn:gaWiki:1340}
\begin{aligned}
{\Vert \Bx \Vert}^2 =
\Bx \cdot \Bw =
\frac{1}{(A_{u,v})^2} \sum_{i<j<k}
{\begin{vmatrix}w_i & w_j & w_k \\u_i & u_j & u_k \\v_i & v_j & v_k \\\end{vmatrix}}^2
\end{aligned}
\end{equation}

Thus, the (squared) volume of the parallelepiped (base area times perpendicular height) is

\begin{equation}\label{eqn:gaWiki:1360}
\begin{aligned}
\sum_{i<j<k}
{\begin{vmatrix}w_i & w_j & w_k \\u_i & u_j & u_k \\v_i & v_j & v_k \\\end{vmatrix}}^2
\end{aligned}
\end{equation}

Note the similarity in form to the w,u,v trivector itself

\begin{equation}\label{eqn:gaWiki:1380}
\begin{aligned}
\sum_{i<j<k}
{\begin{vmatrix}w_i & w_j & w_k \\u_i & u_j & u_k \\v_i & v_j & v_k \\\end{vmatrix}} {\Be}_i \wedge {\Be}_j \wedge {\Be}_k
\end{aligned}
\end{equation}

which, if you take the set of \({\Be}_i \wedge {\Be}_j \wedge {\Be}_k\) as a basis for the trivector space, suggests this is the natural way to define the length of a trivector.  Loosely speaking the length of a vector is a length, length of a bivector is area, and the length of a trivector is volume.

\subsection{Product of a vector and bivector.  Defining the "dot product" of a plane and a vector}

In order to justify the normal to a plane result above, a general examination of the product of a vector and bivector is required.  Namely,

\begin{equation}\label{eqn:gaWiki:1400}
\begin{aligned}
\Bw (\Bu \wedge \Bv)
= \sum_{i,j<k}w_i {\Be}_i {\begin{vmatrix}u_j & u_k \\v_j & v_k \\\end{vmatrix}} {\Be}_j \wedge {\Be}_k
\end{aligned}
\end{equation}

This has two parts, the vector part where \(i=j\) or \(i=k\), and the trivector parts where no indices equal.  After some index summation trickery, and grouping terms and so forth, this is


\begin{equation}\label{eqn:gaWiki:1420}
\begin{aligned}
\Bw (\Bu \wedge \Bv) =
\sum_{i<j}(w_i {\Be}_j
- w_j {\Be}_i )
{\begin{vmatrix}u_i & u_j \\v_i & v_j \\\end{vmatrix}}
+
\sum_{i<j<k}
{\begin{vmatrix}w_i & w_j & w_k \\ u_i & u_j & u_k \\v_i & v_j & v_k \\\end{vmatrix}}
{\Be}_i \wedge {\Be}_j \wedge {\Be}_k
\end{aligned}
\end{equation}

The trivector term is \(\Bw \wedge \Bu \wedge \Bv\).  Expansion of \((\Bu \wedge \Bv) \Bw\) yields the same trivector term.  This is the completely symmetric part, and the vector term is negated.
Like the geometric product of two vectors, this geometric product can be grouped into symmetric and antisymmetric parts, one of which is a pure k-vector.  In analogy the antisymmetric part of this product can be called a generalized dot product, and is roughly speaking the dot product of a "plane" (bivector), and a vector.

The properties of this generalized dot product remain to be explored, but first here is a summary of the notation

\begin{equation}\label{eqn:gaWiki:1440}
\begin{aligned}
\Bw (\Bu \wedge \Bv) = \Bw \cdot (\Bu \wedge \Bv) + \Bw \wedge \Bu \wedge \Bv
\end{aligned}
\end{equation}

\begin{equation}\label{eqn:gaWiki:1460}
\begin{aligned}
(\Bu \wedge \Bv) \Bw = - \Bw \cdot (\Bu \wedge \Bv) + \Bw \wedge \Bu \wedge \Bv
\end{aligned}
\end{equation}

\begin{equation}\label{eqn:gaWiki:1480}
\begin{aligned}
\Bw \wedge \Bu \wedge \Bv = \frac{1}{2}(\Bw (\Bu \wedge \Bv) + (\Bu \wedge \Bv) \Bw)
\end{aligned}
\end{equation}

\begin{equation}\label{eqn:gaWiki:1500}
\begin{aligned}
\Bw \cdot (\Bu \wedge \Bv) = \frac{1}{2}(\Bw (\Bu \wedge \Bv) - (\Bu \wedge \Bv) \Bw)
\end{aligned}
\end{equation}

Let \(\Bw = \Bx + \By\), where \(\Bx = a \Bu + b \Bv\), and \(\By \cdot \Bu = \By \cdot \Bv = \Bzero\).  Expressing \(\Bw\) and the \(\Bu \wedge \Bv\), products in terms of these components is

\begin{equation}\label{eqn:gaWiki:1520}
\begin{aligned}
\Bw (\Bu \wedge \Bv) = \Bx (\Bu \wedge \Bv) + \By (\Bu \wedge \Bv)
=
\Bx \cdot (\Bu \wedge \Bv) + \By \cdot (\Bu \wedge \Bv) + \By \wedge \Bu \wedge \Bv
\end{aligned}
\end{equation}

With the conditions and definitions above, and some manipulation, it can be shown that the term \(\By \cdot (\Bu \wedge \Bv) = \Bzero\), which then justifies the previous solution of the normal to a plane problem.  Since the vector term of the vector bivector product the name dot product is zero
when the vector is perpendicular to the plane (bivector), and this vector, bivector "dot product" selects only the components that are in the plane, so in analogy to the vector-vector dot product this name itself is justified by more than the fact this is the non-wedge product term of the geometric vector-bivector product.

\subsection{Complex numbers}
\index{complex numbers}
There is a one to one correspondence between the geometric product of two \(\mathbb{R}^2\) vectors and the field of complex numbers.

Writing, a vector in terms of its components, and left multiplying by the unit vector \({\Be}_1\) yields

\begin{equation}\label{eqn:gaWiki:1540}
\begin{aligned}
 Z = {\Be}_1 \BP = {\Be}_1 ( x {\Be}_1 + y {\Be}_2)
= x (1) + y ({\Be}_1 {\Be}_2)
= x (1) + y ({\Be}_1 \wedge {\Be}_2)
\end{aligned}
\end{equation}

The unit scalar and unit bivector pair \(1, {\Be}_1 \wedge {\Be}_2\) can be considered an alternate basis for a two dimensional vector space.  This alternate vector representation is closed with respect to the geometric product

\begin{equation}\label{eqn:gaWiki:1560}
\begin{aligned}
 Z_1 Z_2
&= {\Be}_1 ( x_1 {\Be}_1 + y_1 {\Be}_2) {\Be}_1 ( x_2 {\Be}_1 + y_2 {\Be}_2) \\
&= ( x_1 + y_1 {\Be}_1 {\Be}_2) ( x_2 + y_2 {\Be}_1 {\Be}_2) \\
&= x_1 x_2 + y_1 y_2 ({\Be}_1 {\Be}_2) {\Be}_1 {\Be}_2) \\
+ (x_1 y_2 + x_2 y_1) {\Be}_1 {\Be}_2 \\
\end{aligned}
\end{equation}

This closure can be observed after calculation of the square of the unit bivector above, a quantity

\begin{equation}\label{eqn:gaWiki:1580}
\begin{aligned}
({\Be}_1 \wedge {\Be}_2)^2 = {\Be}_1 {\Be}_2 {\Be}_1 {\Be}_2 = - {\Be}_1 {\Be}_1 {\Be}_2 {\Be}_2 = -1
\end{aligned}
\end{equation}

that has the characteristics of the complex number \(i^2 = -1\).

This fact allows the simplification of the product above to

\begin{equation}\label{eqn:gaWiki:1600}
\begin{aligned}
Z_1 Z_2
= (x_1 x_2 - y_1 y_2) + (x_1 y_2 + x_2 y_1) ({\Be}_1 \wedge {\Be}_2)
\end{aligned}
\end{equation}

Thus what is traditionally the defining, and arguably arbitrary seeming, rule of complex number multiplication, is found to follow naturally from the higher order structure of the geometric product, once that is applied to a two dimensional vector space.

It is also informative to examine how the length of a vector can be represented in terms of a complex number.  Taking the square of the length

\begin{equation}\label{eqn:gaWiki:1620}
\begin{aligned}
\BP \cdot \BP &= ( x {\Be}_1 + y {\Be}_2) \cdot ( x {\Be}_1 + y {\Be}_2) \\
&= ({\Be}_1 Z) {\Be}_1 Z \\
&= (( x  - y {\Be}_1 {\Be}_2) {\Be}_1) {\Be}_1 Z \\
&= ( x  - y ({\Be}_1 \wedge {\Be}_2)) Z \\
\end{aligned}
\end{equation}

This right multiplication of a vector with \({\Be}_1\), is named the conjugate

\begin{equation}\label{eqn:gaWiki:1640}
\begin{aligned}
\overline{Z} = x  - y ({\Be}_1 \wedge {\Be}_2)
\end{aligned}
\end{equation}

And with that definition, the length of the original vector can be expressed as

\begin{equation}\label{eqn:gaWiki:1660}
\begin{aligned}
\BP \cdot \BP = \overline{Z}Z
\end{aligned}
\end{equation}

This is also a natural definition of the length of a complex number, given the fact that the complex numbers can be considered an isomorphism with the two dimensional Euclidean vector space.

\subsection{Rotation in an arbitrarily oriented plane}
\index{plane!rotation}

A point \(\BP\), of radius \(\Br\), located at an angle \(\theta\) from the vector \(\ucap\) in the direction from \(\Bu\) to \(\Bv\), can be expressed as

\begin{equation}\label{eqn:gaWiki:1680}
\begin{aligned}
\BP = r( \ucap \cos{\theta} +
\frac{\ucap (\ucap \wedge \Bv)}{\Vert \ucap (\ucap \wedge \Bv) \Vert}  \sin{\theta})
=
r \ucap
( \cos{\theta} +
\frac{(\Bu \wedge \Bv)}{\Vert \ucap (\Bu \wedge \Bv) \Vert} \sin{\theta})
\end{aligned}
\end{equation}

Writing \( {\BI}_{\Bu ,\Bv } = \frac{\Bu \wedge \Bv}{\Vert \ucap (\Bu \wedge \Bv) \Vert} \), the square of this bivector has the property \({\BI _{\Bu ,\Bv }}^2 = -1 \) of the imaginary unit complex number.

This allows the point to be specified as a complex exponential

\begin{equation}\label{eqn:gaWiki:1700}
\begin{aligned}
= \ucap r ( \cos\theta + \BI _{\Bu ,\Bv } \sin\theta )
= \ucap r \exp( \BI _{\Bu ,\Bv } \theta )
\end{aligned}
\end{equation}

Complex numbers could be expressed in terms of the \(\mathbb R^2\)unit bivector \({\Be}_1 \wedge {\Be}_2\).  However this isomorphism really only requires a pair of linearly independent vectors in a plane (of arbitrary dimension).

\subsection{Quaternions}
\index{quaternion}

Similar to complex numbers the geometric product of two \(\mathbb{R}^3\) vectors can be used to define quaternions.  Pre and Post multiplication with \({\Be}_1{\Be}_2{\Be}_3\) can be used to express a vector in terms of the quaternion unit numbers \(i, j, k\), as well as describe all the properties of those numbers.

\subsection{Cross product as outer product}

%The cross product of traditional vector algebra (on \(\mathbb{R}^3\)) find its place in geometric algebra \(\calG_3\)

Cross product can be written as a scaled outer product

\begin{equation}\label{eqn:gaWiki:1720}
\begin{aligned}
\Ba \times\Bb  = -i(\Ba \wedge\Bb )
\end{aligned}
\end{equation}

\begin{equation}\label{eqn:gaWiki:1740}
\begin{aligned}
i^2 &= ({\Be}_1{\Be}_2{\Be}_3)^2 \\
&= {\Be}_1{\Be}_2{\Be}_3{\Be}_1{\Be}_2{\Be}_3 \\
&= -{\Be}_1{\Be}_2{\Be}_1{\Be}_3{\Be}_2{\Be}_3 \\
&= {\Be}_1{\Be}_1{\Be}_2{\Be}_3{\Be}_2{\Be}_3 \\
&= -{\Be}_3{\Be}_2{\Be}_2{\Be}_3 \\
&= -1
\end{aligned}
\end{equation}

The equivalence of the \(\mathbb{R}^3\) cross product and the wedge product expression above can be confirmed by direct multiplication of \(-i = -{\Be}_1{\Be}_2{\Be}_3\) with a determinant expansion of the wedge product

\begin{equation}\label{eqn:gaWiki:1760}
\begin{aligned}
\Bu \wedge \Bv = \sum_{1<=i<j<=3}(u_i v_j - v_i u_j) {\Be}_i \wedge {\Be}_j
= \sum_{1<=i<j<=3}(u_i v_j - v_i u_j) {\Be}_i {\Be}_j
\end{aligned}
\end{equation}

%%\EndArticle
%\EndNoBibArticle

%
% Copyright � 2012 Peeter Joot.  All Rights Reserved.
% Licenced as described in the file LICENSE under the root directory of this GIT repository.
%

%
%
\chapter{Cramer's rule}
\index{Cramer's rule}
\label{chap:gaWikiCramers}
%\date{October 16, 2007.  gaWikiCramers.tex}

\section{Cramer's rule, determinants, and matrix inversion can be naturally expressed in terms of the wedge product}

The use of the wedge product in the solution of linear equations can be quite useful.

This does not require any notion of geometric algebra, only an exterior product and the concept of similar elements, and a nice example of such a treatment can be found in Solution of Linear equations section of \citep{grassmanbookExteriorProduct}.

Traditionally, instead of using the wedge product, Cramer's rule is usually presented as a generic algorithm that can be used to solve linear equations of the form \(\BA \Bx = \Bb\) (or equivalently to invert a matrix).  Namely

\begin{equation}\label{eqn:gaWikiCramers:20}
\Bx = \frac{1}{|\BA|}\operatorname{adj}(\BA)
\end{equation}

This is a useful theoretic result.  For numerical problems row reduction with pivots and other methods are more stable and efficient.

When the wedge product is coupled with the Clifford product and put into a natural geometric context, the fact that the determinants are used in the expression of \({\mathbb R}^N\) parallelogram area and parallelepiped volumes (and higher dimensional generalizations of these) also comes as a nice side effect.

As is also shown below, results such as Cramer's rule also follow directly from the property of the wedge product that it selects non identical elements.  The end result is then simple enough that it could be derived easily if required instead of having to remember or look up a rule.

\subsection{Two variables example}

\begin{equation}\label{eqn:gaWikiCramers:40}
\begin{bmatrix}
\Ba & \Bb
\end{bmatrix}
\begin{bmatrix}
x \\ y
\end{bmatrix}
= \Ba x + \Bb y = \Bc
\end{equation}

Pre and post multiplying by \(\Ba\) and \(\Bb\).

\begin{equation}\label{eqn:gaWikiCramers:60}
      ( \Ba x + \Bb y ) \wedge \Bb = (\Ba \wedge \Bb) x =       \Bc \wedge \Bb
\end{equation}
\begin{equation}\label{eqn:gaWikiCramers:80}
\Ba \wedge ( \Ba x + \Bb y )       = (\Ba \wedge \Bb) y = \Ba \wedge \Bc
\end{equation}

Provided \(\Ba \wedge \Bb \neq 0\) the solution is

\begin{equation}\label{eqn:gaWikiCramers:100}
\begin{bmatrix}x \\ y\end{bmatrix}
= \frac{1}{\Ba \wedge \Bb}
\begin{bmatrix}
\Bc \wedge \Bb \\ \Ba \wedge \Bc
\end{bmatrix}
\end{equation}

For \(\Ba, \Bb \in {\mathbb R}^2\), this is Cramer's rule since the \(\Be _1 \wedge \Be _2\) factors of the wedge products

\begin{equation}\label{eqn:gaWikiCramers:120}
\Bu \wedge \Bv = \begin{vmatrix}u_1 & u_2 \\ v_1 & v_2 \end{vmatrix} \Be _1 \wedge \Be _2
\end{equation}

divide out.

Similarly, for three, or N variables, the same ideas hold

\begin{equation}\label{eqn:gaWikiCramers:140}
\begin{bmatrix}
\Ba & \Bb & \Bc
\end{bmatrix}
\begin{bmatrix}
x \\ y \\ z
\end{bmatrix}
= \Bd
\end{equation}

\begin{equation}\label{eqn:gaWikiCramers:160}
\begin{bmatrix}
x \\ y \\ z
\end{bmatrix}
= \frac{1}{\Ba \wedge \Bb \wedge \Bc}
\begin{bmatrix}
\Bd \wedge \Bb \wedge \Bc \\
\Ba \wedge \Bd \wedge \Bc \\
\Ba \wedge \Bb \wedge \Bd
\end{bmatrix}
\end{equation}

Again, for the three variable three equation case this is Cramer's rule since the \(\Be _1 \wedge \Be _2 \wedge \Be _3\) factors of all the wedge products divide out, leaving the familiar determinants.

\subsection{A numeric example}

When there are more equations than variables case, if the equations have a solution, each of the k-vector quotients will be scalars

To illustrate here is the solution of a simple example with three equations and two unknowns.

\begin{equation}\label{eqn:gaWikiCramers:180}
\begin{bmatrix}
1 \\ 1 \\ 0
\end{bmatrix}
x
+
\begin{bmatrix}
1 \\ 1 \\ 1
\end{bmatrix}
y
=
\begin{bmatrix}
1 \\ 1 \\ 2
\end{bmatrix}
\end{equation}

The right wedge product with \((1, 1, 1)\) solves for \(x\)

\begin{equation}\label{eqn:gaWikiCramers:200}
\begin{bmatrix}
1 \\ 1 \\ 0
\end{bmatrix}
\wedge
\begin{bmatrix}
1 \\ 1 \\ 1
\end{bmatrix}
x
=
\begin{bmatrix}
1 \\ 1 \\ 2
\end{bmatrix}
\wedge
\begin{bmatrix}
1 \\ 1 \\ 1
\end{bmatrix}
\end{equation}

and a left wedge product with \((1, 1, 0)\) solves for \(y\)

\begin{equation}\label{eqn:gaWikiCramers:220}
\begin{bmatrix}
1 \\ 1 \\ 0
\end{bmatrix}
\wedge
\begin{bmatrix}
1 \\ 1 \\ 1
\end{bmatrix}
y
=
\begin{bmatrix}
1 \\ 1 \\ 0
\end{bmatrix}
\wedge
\begin{bmatrix}
1 \\ 1 \\ 2
\end{bmatrix}
\end{equation}

Observe that both of these equations have the same factor, so
one can compute this only once (if this was zero it would
indicate the system of equations has no solution).

Collection of results for
\(x\) and \(y\) yields a Cramer's rule like form
(writing \(\Be _i \wedge \Be _j = \Be _{ij}\)):

\begin{equation}\label{eqn:gaWikiCramers:240}
\begin{bmatrix}
x \\ y\end
{bmatrix}
=
\frac{1}{(1, 1, 0) \wedge (1, 1, 1)}
\begin{bmatrix}
(1, 1, 2) \wedge (1, 1, 1) \\
(1, 1, 0) \wedge (1, 1, 2)
\end{bmatrix}
=
\frac{1}{\Be_{13} + \Be_{23}}
\begin{bmatrix}
{-\Be_{13} - \Be_{23}} \\
{2\Be_{13} +2\Be_{23}} \\
\end{bmatrix}
=
\begin{bmatrix}
-1 \\ 2
\end{bmatrix}
\end{equation}


\documentclass{article}      % Specifies the document class

\usepackage{amsmath}
\usepackage{mathpazo}

%
% shorthand for bold symbols, convenient for vectors and matrices
%
\newcommand{\Ba}[0]{\mathbf{a}}
\newcommand{\Bb}[0]{\mathbf{b}}
\newcommand{\Bc}[0]{\mathbf{c}}
\newcommand{\Bd}[0]{\mathbf{d}}
\newcommand{\Be}[0]{\mathbf{e}}
\newcommand{\Bf}[0]{\mathbf{f}}
\newcommand{\Bg}[0]{\mathbf{g}}
\newcommand{\Bh}[0]{\mathbf{h}}
\newcommand{\Bi}[0]{\mathbf{i}}
\newcommand{\Bj}[0]{\mathbf{j}}
\newcommand{\Bk}[0]{\mathbf{k}}
\newcommand{\Bl}[0]{\mathbf{l}}
\newcommand{\Bm}[0]{\mathbf{m}}
\newcommand{\Bn}[0]{\mathbf{n}}
\newcommand{\Bo}[0]{\mathbf{o}}
\newcommand{\Bp}[0]{\mathbf{p}}
\newcommand{\Bq}[0]{\mathbf{q}}
\newcommand{\Br}[0]{\mathbf{r}}
\newcommand{\Bs}[0]{\mathbf{s}}
\newcommand{\Bt}[0]{\mathbf{t}}
\newcommand{\Bu}[0]{\mathbf{u}}
\newcommand{\Bv}[0]{\mathbf{v}}
\newcommand{\Bw}[0]{\mathbf{w}}
\newcommand{\Bx}[0]{\mathbf{x}}
\newcommand{\By}[0]{\mathbf{y}}
\newcommand{\Bz}[0]{\mathbf{z}}
\newcommand{\BA}[0]{\mathbf{A}}
\newcommand{\BB}[0]{\mathbf{B}}
\newcommand{\BC}[0]{\mathbf{C}}
\newcommand{\BD}[0]{\mathbf{D}}
\newcommand{\BE}[0]{\mathbf{E}}
\newcommand{\BF}[0]{\mathbf{F}}
\newcommand{\BG}[0]{\mathbf{G}}
\newcommand{\BH}[0]{\mathbf{H}}
\newcommand{\BI}[0]{\mathbf{I}}
\newcommand{\BJ}[0]{\mathbf{J}}
\newcommand{\BK}[0]{\mathbf{K}}
\newcommand{\BL}[0]{\mathbf{L}}
\newcommand{\BM}[0]{\mathbf{M}}
\newcommand{\BN}[0]{\mathbf{N}}
\newcommand{\BO}[0]{\mathbf{O}}
\newcommand{\BP}[0]{\mathbf{P}}
\newcommand{\BQ}[0]{\mathbf{Q}}
\newcommand{\BR}[0]{\mathbf{R}}
\newcommand{\BS}[0]{\mathbf{S}}
\newcommand{\BT}[0]{\mathbf{T}}
\newcommand{\BU}[0]{\mathbf{U}}
\newcommand{\BV}[0]{\mathbf{V}}
\newcommand{\BW}[0]{\mathbf{W}}
\newcommand{\BX}[0]{\mathbf{X}}
\newcommand{\BY}[0]{\mathbf{Y}}
\newcommand{\BZ}[0]{\mathbf{Z}}

\newcommand{\Bzero}[0]{\mathbf{0}}
\newcommand{\Btheta}[0]{\boldsymbol{\theta}}
\newcommand{\Btau}[0]{\boldsymbol{\tau}}
\newcommand{\Bomega}[0]{\boldsymbol{\omega}}

%
% shorthand for unit vectors
%
\newcommand{\acap}[0]{\hat{\Ba}}
\newcommand{\bcap}[0]{\hat{\Bb}}
\newcommand{\ccap}[0]{\hat{\Bc}}
\newcommand{\dcap}[0]{\hat{\Bd}}
\newcommand{\ecap}[0]{\hat{\Be}}
\newcommand{\fcap}[0]{\hat{\Bf}}
\newcommand{\gcap}[0]{\hat{\Bg}}
\newcommand{\hcap}[0]{\hat{\Bh}}
\newcommand{\icap}[0]{\hat{\Bi}}
\newcommand{\jcap}[0]{\hat{\Bj}}
\newcommand{\kcap}[0]{\hat{\Bk}}
\newcommand{\lcap}[0]{\hat{\Bl}}
\newcommand{\mcap}[0]{\hat{\Bm}}
\newcommand{\ncap}[0]{\hat{\Bn}}
\newcommand{\ocap}[0]{\hat{\Bo}}
\newcommand{\pcap}[0]{\hat{\Bp}}
\newcommand{\qcap}[0]{\hat{\Bq}}
\newcommand{\rcap}[0]{\hat{\Br}}
\newcommand{\scap}[0]{\hat{\Bs}}
\newcommand{\tcap}[0]{\hat{\Bt}}
\newcommand{\ucap}[0]{\hat{\Bu}}
\newcommand{\vcap}[0]{\hat{\Bv}}
\newcommand{\wcap}[0]{\hat{\Bw}}
\newcommand{\xcap}[0]{\hat{\Bx}}
\newcommand{\ycap}[0]{\hat{\By}}
\newcommand{\zcap}[0]{\hat{\Bz}}
\newcommand{\thetacap}[0]{\hat{\Btheta}}

%
% to write R^n and C^n in a distinguishable fashion.  Perhaps change this
% to the double lined characters upon figuring out how to do so.
%
\newcommand{\C}[1]{$\mathbb{C}^{#1}$}
\newcommand{\R}[1]{$\mathbb{R}^{#1}$}

%
% various generally useful helpers
%

% derivative of #1 wrt. #2:
\newcommand{\D}[2] {\frac {d#2} {d#1}}

\newcommand{\inv}[1]{\frac{1}{#1}}
\newcommand{\cross}[0]{\times}

\newcommand{\abs}[1]{\lvert{#1}\rvert}
\newcommand{\norm}[1]{\lVert{#1}\rVert}
\newcommand{\innerprod}[2]{\langle{#1}, {#2}\rangle}
\newcommand{\dotprod}[2]{{#1} \cdot {#2}}
\newcommand{\bdotprod}[2]{\left({#1} \cdot {#2}\right)}
\newcommand{\crossprod}[2]{{#1} \cross {#2}}
\newcommand{\tripleprod}[3]{\dotprod{\left(\crossprod{#1}{#2}\right)}{#3}}

\DeclareMathOperator{\Proj}{Proj}
\DeclareMathOperator{\Span}{span}
\DeclareMathOperator{\Sgn}{sgn}
\DeclareMathOperator{\Area}{Area}
\DeclareMathOperator{\Volume}{Volume}

%
% A few miscellaneous things specific to this document
%
\newcommand{\crossop}[1]{\crossprod{#1}{}}

% R2 vector.
\newcommand{\VectorTwo}[2]{
\begin{bmatrix}
 {#1} \\
 {#2}
\end{bmatrix}
}

\newcommand{\VectorN}[1]{
\begin{bmatrix}
{#1}_1 \\
{#1}_2 \\
\vdots \\
{#1}_N \\
\end{bmatrix}
}

\newcommand{\DETuvij}[4]{
\begin{vmatrix}
 {#1}_{#3} & {#1}_{#4} \\
 {#2}_{#3} & {#2}_{#4}
\end{vmatrix}
}

\newcommand{\DETuvwijk}[6]{
\begin{vmatrix}
 {#1}_{#4} & {#1}_{#5} & {#1}_{#6} \\
 {#2}_{#4} & {#2}_{#5} & {#2}_{#6} \\
 {#3}_{#4} & {#3}_{#5} & {#3}_{#6}
\end{vmatrix}
}

\newcommand{\DETuvwxijkl}[8]{
\begin{vmatrix}
 {#1}_{#5} & {#1}_{#6} & {#1}_{#7} & {#1}_{#8} \\
 {#2}_{#5} & {#2}_{#6} & {#2}_{#7} & {#2}_{#8} \\
 {#3}_{#5} & {#3}_{#6} & {#3}_{#7} & {#3}_{#8} \\
 {#4}_{#5} & {#4}_{#6} & {#4}_{#7} & {#4}_{#8} \\
\end{vmatrix}
}

%\newcommand{\DETuvwxyijklm}[10]{
%\begin{vmatrix}
% {#1}_{#6} & {#1}_{#7} & {#1}_{#8} & {#1}_{#9} & {#1}_{#10} \\
% {#2}_{#6} & {#2}_{#7} & {#2}_{#8} & {#2}_{#9} & {#2}_{#10} \\
% {#3}_{#6} & {#3}_{#7} & {#3}_{#8} & {#3}_{#9} & {#3}_{#10} \\
% {#4}_{#6} & {#4}_{#7} & {#4}_{#8} & {#4}_{#9} & {#4}_{#10} \\
% {#5}_{#6} & {#5}_{#7} & {#5}_{#8} & {#5}_{#9} & {#5}_{#10}
%\end{vmatrix}
%}

% R3 vector.
\newcommand{\VectorThree}[3]{
\begin{bmatrix}
 {#1} \\
 {#2} \\
 {#3}
\end{bmatrix}
}



%
% The real thing:
%

                             % The preamble begins here.
\title{Torque expressed with geometric algebra} % Declares the document's title.
\author{Peeter Joot}         % Declares the author's name.
%\date{}        % Deleting this command produces today's date.

\begin{document}             % End of preamble and beginning of text.

\maketitle{}

\section{Torque}

Torque is generally defined as the magnitude of the perpendicular force component times distance, or work per unit angle.

Suppose a circular path in an arbitrary plane containing orthonormal vectors $\ucap$ and $\vcap$ is parameterized by angle.

\[
\Br = r(\ucap \cos \theta + \vcap \sin \theta) = r \ucap(\cos \theta + \ucap \vcap \sin \theta)
\]

By designating the unit bivector of this plane as the imaginary number

\[
\Bi  = \ucap \vcap = \ucap \wedge \vcap
\]
\[
\Bi ^2 = -1
\]

this path vector can be conveniently written in complex exponential form

\[
\Br = r \ucap e^{\Bi  \theta}
\]

and the derivative with respect to angle is

\[
\frac{d \Br}{d\theta} = r \ucap \Bi  e^{\Bi  \theta} = \Br  \Bi 
\]

So the torque, the rate of change of work $W$, due to a force $F$, is

\[
\tau = \frac{dW}{d\theta} = \BF \cdot \frac{d \Br}{d\theta} = \BF \cdot (\Br  \Bi )
\]

Unlike the cross product description of torque, $\Btau = \Br \times \BF$ no vector in a normal direction had to be introduced, a normal that doesn't exist in two dimensions or in greater than three dimensions.  The unit bivector describes the plane and the orientation of the rotation, and the sense of the rotation is relative to the angle between the vectors $\ucap$ and $\vcap$.

\subsection{Expanding the result in terms of components }

At a glance this doesn't appear much like the familiar torque as a determinant or cross product, but this can be expanded to demonstrate its equivalance (the cross product is hiding there in the bivector $\Bi = \ucap \wedge \vcap$).  Expanding the position vector in terms of the planar unit vectors 

\[
\Br \Bi =
\left(
r_u \ucap + r_v \vcap
\right)
\ucap \vcap
= 
r_u \vcap  
- r_v \ucap
\]

and expanding the force by components in the same direction plus the possible perpendicular remainder term

\[
\BF  = F_u \ucap + F_v \vcap + \BF _{\perp \ucap,\vcap}
\]

and then taking dot products yields is the torque

\[
\tau = \BF \cdot (\Br  \Bi ) = r_u F_v - r_v F_u
\]

This determinant may be familiar from derivations with $\ucap = \Be _1$, and $\vcap = \Be _2$ (See the Feynman lectures Volume I for example).

\subsection{Geometrical description }

When the magnitude of the "rotational arm" is factored out, the torque can be written as

\[
\tau = \BF \cdot (\Br  \Bi ) = |\Br |  (\BF \cdot (\rcap \Bi ))
\]

The vector $\rcap \Bi $ is the unit vector perpendicular to the $\Br$.  Thus the torque can also be described as the product of the magnitude of the rotational arm times the component of the force that is in the direction of the rotation (ie: the work done rotating something depends on length of the lever, and the size of the useful part of the force pushing on it).

\subsection{Slight generalization.  Application of the force to a lever not in the plane. }

If the rotational arm that the force is applied to is not in the plane of rotation then only the components of the lever arm direction and the component of the force that are in the plane will contribute to the work done.  The calculation above allowed for a force applied in an arbitrary direction, so to generalize this, a calculation that discards the component of the level arm direction not in the plane.

When $\Br $ is allowed to lie outside of the plane of rotation the component in the plane (bivector) $\Bi $ can be described with the geometric product nicely

\[
\Br _{\Bi } =  (\Br  \cdot \Bi ) \frac{1}{\Bi } =  -(\Br  \cdot \Bi ) \Bi 
\]

Thus, the vector with this magnitude that is perpendicular to this in the plane of the rotation  is

\[
\Br _{\Bi } \Bi  
=  -(\Br  \cdot \Bi ) \Bi ^2
=  (\Br  \cdot \Bi ) 
\]

and the total torque is thus

\[
\tau
=  \BF  \cdot (\Br  \cdot \Bi ) 
\]

This makes sense when once considers that only the dot product part of $\Br  \Bi  = \Br  \cdot \Bi  + \Br  \wedge \Bi $ contributes to the component of $\Br $ in the plane, and when the lever is in the rotational plane this wedge product component of 
$\Br \Bi $ is zero.

\end{document}               % End of document.

%
% Copyright � 2012 Peeter Joot.  All Rights Reserved.
% Licenced as described in the file LICENSE under the root directory of this GIT repository.
%

%
%
\chapter{Derivatives of a unit vector}\label{chap:PJUnitDer}
\index{unit vector!derivative}
%\date{Oct 16, 2007.  gaWikiUnitDerivative.tex}

\section{First derivative of a unit vector}

\subsection{Expressed with the cross product}

It can be shown that a unit vector derivative can be expressed using the cross product.  Two cross product operations are required to get the result back into the plane of the rotation, since a unit vector is constrained to circular (really perpendicular to itself) motion.

\begin{equation}\label{eqn:gaWikiUnitDerivative:20}
\dt{}\left(\frac{\Br}{\Vert \Br \Vert}\right)
= \frac{1}{{\Vert \Br \Vert}^3}\left(\Br \times \dt{\Br}\right) \times \Br
= \left(\rcap \times \frac{1}{{\Vert \Br \Vert}} \dt{\Br}\right) \times \rcap
\end{equation}

This derivative is the rejective component of \(\dt{\Br}\) with respect to \(\rcap\), but is scaled by \(1/\Vert \Br \Vert\).

How to calculate this result can be found in other places, such as
\citep{salas1990coa}.

\section{Equivalent result utilizing the geometric product}

The equivalent geometric product result can be obtained by calculating the derivative of a vector \(\Br = r \rcap\).

\begin{equation}\label{eqn:gaWikiUnitDerivative:40}
\dt{\Br} = r \dt{\rcap} + \rcap \dt{r}
\end{equation}

\subsection{Taking dot products}
One trick is required first (as was also the case in the Salus and Hille derivation), which is expressing \(\dt{r}\) via the dot product.

\begin{equation}\label{eqn:gaWikiUnitDerivative:120}
\begin{aligned}
\dt{(r^2)} &= 2r \dt{r} \\
\dt{(\Br \cdot \Br)} &= 2 \Br \cdot \dt{\Br} \\
\end{aligned}
\end{equation}

Thus,
\begin{equation}\label{eqn:gaWikiUnitDerivative:60}
\dt{r} = \rcap \cdot \dt{\Br}
\end{equation}

Taking dot products of the derivative above yields

\begin{equation}\label{eqn:gaWikiUnitDerivative:140}
\begin{aligned}
\rcap \cdot \dt{\Br} &= \rcap \cdot r \dt{\rcap} + \rcap \cdot \rcap \dt{r} \\
                            &= \Br \cdot \dt{\rcap} + \dt{r} \\
                            &= \Br \cdot \dt{\rcap} + \rcap \cdot \dt{\Br}
\end{aligned}
\end{equation}

\begin{equation}\label{eqn:gaWikiUnitDerivative:80}
\implies
\Br \cdot \dt{\rcap} = \Bzero
\end{equation}

One could alternatively prove this with a diagram.


\subsection{Taking wedge products}

As in linear equation solution, the \(\rcap\) component can be eliminated by taking a wedge product

\begin{equation}\label{eqn:gaWikiUnitDerivative:160}
\begin{aligned}
\rcap \wedge \dt{\Br} &= \rcap \wedge r \dt{\rcap} + \rcap \wedge \rcap \dt{r} \\
                             &= r \rcap \wedge \dt{\rcap} \\
                             &= \Br \wedge \dt{\rcap}  \\
                             &= \Br \wedge \dt{\rcap} + \Br \cdot \dt{\rcap} \\
                             &= \Br \dt{\rcap}
\end{aligned}
\end{equation}

This allows expression of \(\dt{\rcap}\) in terms of \(\dt{\Br}\) in various ways (compare to the cross product results above)

\begin{equation}\label{eqn:gaWikiUnitDerivative:180}
\begin{aligned}
\dt{\rcap} &= \frac{1}{{ \Br }}\left(\rcap \wedge \dt{\Br}\right) \\
%                   &= \frac{1}{\Vert \Br \Vert}{     \frac{1}{\rcap} \left(\rcap \wedge \dt{\Br}\right)       } \\
                   &= \frac{1}{\Vert \Br \Vert}{     {\rcap} \left(\rcap \wedge \dt{\Br}\right)       } \\
%                   &= \frac{1}{{\Vert \Br \Vert}^3}{     {\Br} \left(\Br \wedge \dt{\Br}\right)       } \\
                   &= \frac{1}{\Vert \Br \Vert}\left({ \dt{\Br} - \rcap (\rcap \cdot \dt{\Br}) }\right) \\
\end{aligned}
\end{equation}

Thus this derivative is the component of
\(\frac{1}{{\Vert \Br \Vert}}\dt{\Br}\)
in the direction perpendicular to
\(\Br\).

\subsection{Another view}

When the objective is not comparing to the cross product, it is also notable that this unit vector derivative can be written

\begin{equation}\label{eqn:gaWikiUnitDerivative:100}
{{ \Br }} \dt{\rcap}
= \rcap \wedge \dt{\Br}
\end{equation}


%
% This was obvious to me at one point but is not now;)  What is the justification for the first statement?
%
%\subsection{A more direct route}
%
%Like a lot of stuff in math, once you know the answer you can get the answer more directly.  There is an unfortunate tendancy
%in some math texts to skip the logical sequence and go straight to the end result by the quickest route.  This is more
%elegant
%
%\begin{align*}
%r \dt{\rcap}
%   &= \dt{\Br} - \rcap \dt{r} \\
%   &= \dt{\Br} - \rcap\left(\rcap \cdot \dt{\Br}\right) \\
%   &= \rcap \left(\rcap \dt{\Br} - \rcap \cdot \dt{\Br}\right) \\
%   &= \rcap \left(\rcap \wedge \dt{\Br}\right) \\
%\end{align*}
%
%and gives the appearance of being clever, but it is easy to be clever when you already know the answer.


%
% Copyright � 2012 Peeter Joot.  All Rights Reserved.
% Licenced as described in the file LICENSE under the root directory of this GIT repository.
%

%
%
\chapter{Radial components of vector derivatives}\label{chap:PJRadialDer}
\index{vector!radial component}
%\date{Oct 22, 2007.  radialVectorDerivatives.tex}

\section{first derivative of a radially expressed vector}

Having calculated the derivative of a unit vector, the total
derivative of a radially expressed vector can be calculated

\begin{equation}\label{eqn:radialVectorDerivatives:20}
\begin{aligned}
(r\rcap)'
   &= r'\rcap  + r\rcap' \\
   &= r'\rcap  + \BrPrimeRej \\
\end{aligned}
\end{equation}

There are two components.  One is in the \(\rcap\) direction (linear component)
and the other perpendicular to that (a rotational component) in the direction of the rejection
of \(\rcap\) from \(\Br'\).

\section{Second derivative of a vector}

Taking second derivatives of a radially expressed vector, we have

\begin{equation}\label{eqn:radialVectorDerivatives:40}
\begin{aligned}
(r\rcap)''
   &= (r'\rcap + r{\rcap}')' \\
   &= r''\rcap + r'\rcap' + (r\rcap')' \\
   &= r''\rcap + (r'/r)\BrPrimeRej + (r\rcap')' \\
\end{aligned}
\end{equation}

Expanding the last term takes a bit more work
\begin{equation}\label{eqn:radialVectorDerivatives:60}
\begin{aligned}
(r\rcap')'
   &= (\BrPrimeRej)' \\
   &=
\rcap'(\rcap \wedge \Br') +
\rcap(\rcap' \wedge \Br') +
\rcap(\rcap \wedge \Br'') \\
   &=
(1/r)(\BrPrimeRej)(\rcap \wedge \Br') +
\rcap(\rcap' \wedge \Br') +
\rcap(\rcap \wedge \Br'') \\
   &=
(1/r)\rcap(\rcap \wedge \Br')^2 +
\rcap(\rcap' \wedge \Br') +
\rcap(\rcap \wedge \Br'') \\
\end{aligned}
\end{equation}

There are three terms to this.  One a scalar (negative) multiple of \(\rcap\), and another, the rejection of \(\rcap\) from \(\Br''\).  The middle term here remains to be expanded.  In particular,

\begin{equation}\label{eqn:radialVectorDerivatives:80}
\begin{aligned}
\rcap' \wedge \Br'
   &= \rcap' \wedge (r\rcap' + r'\rcap) \\
   &= r' \rcap' \wedge \rcap \\
   &= r'/2 (\rcap'\rcap - \rcap\rcap') \\
   &= r'/2r ((\Br' \wedge \rcap)\rcap\rcap - \rcap\rcap(\rcap \wedge \Br')) \\
   &= r'/2r (\Br' \wedge \rcap - \rcap \wedge \Br') \\
   &= -(r'/r) \rcap \wedge \Br' \\
\end{aligned}
\end{equation}

\begin{equation}\label{eqn:radialVectorDerivatives:100}
\begin{aligned}
\implies
(r\rcap')'
   &=
(1/r)\rcap(\rcap \wedge \Br')^2
-(r'/r)\BrPrimeRej
+\rcap(\rcap \wedge \Br'') \\
\end{aligned}
\end{equation}

\begin{equation}\label{eqn:radialVectorDerivatives:120}
\begin{aligned}
\implies
(r\rcap)''
   &= r''\rcap
+(r'/r)\BrPrimeRej
+(1/r)\rcap(\rcap \wedge \Br')^2
-(r'/r)\BrPrimeRej
+\rcap(\rcap \wedge \Br'') \\
   &= r''\rcap
    +(1/r)\rcap(\rcap \wedge \Br')^2
    +\rcap(\rcap \wedge \Br'') \\
   &=
\rcap \left(  r'' +(1/r)(\rcap \wedge \Br')^2\right) +    \rcap(\rcap \wedge \Br'') \\
\end{aligned}
\end{equation}

There are two terms here that are in the \(\rcap\) direction (the bivector square is a negative scalar), and
one rejective term in the direction of the component perpendicular to \(\rcap\) relative to \(\Br''\).


\chapter{Rotational dynamics}\label{chap:PJAngVel}
\date{January 29, 2008.  angularVelocity.tex}

\section{GA introduction of angular velocity}

By taking the first derivative of a radially expressed vector we have the velocity 

\[
\Bv 
   = r'\rcap + \rcap(\rcap \wedge \Br')
   = \rcap( v_r + \rcap \wedge \Bv )
\]

Or,
\[
\rcap \Bv = v_r + \rcap \wedge \Bv
\]
\[
\rcap \Bv = v_r + (1/r)\Br \wedge \Bv
\]

Put this way, the earlier calculus exercise to derive this seems a bit silly, since it is probably clear that $v_r = \rcap \cdot \Bv$.

Anyways, let's work with velocity expressed this way in a few ways.

\subsection{Speed in terms of linear and rotational components}

\[
\Abs{\Bv}^2 = v_r^2 + (\rcap(\rcap \wedge \Bv))^2
\]

And,
\begin{align*}
(\rcap(\rcap \wedge \Bv))^2 
   &= (\Bv \wedge \rcap)\rcap \rcap(\rcap \wedge \Bv) \\
   &= (\Bv \wedge \rcap)(\rcap \wedge \Bv) \\
   &= -(\rcap \wedge \Bv)^2 \\
   &= \Abs{\rcap \wedge \Bv}^2 \\
\end{align*}

\begin{align*}
\implies
\Abs{\Bv}^2 &= v_r^2 + \Abs{\rcap \wedge \Bv}^2 \\
             &= v_r^2 + \Abs{\rcap \wedge \Bv}^2 \\
\end{align*}

So, we can assign a physical significance to the bivector.

\[
\Abs{\rcap \wedge \Bv} = \abs{v_{\perp}} 
\]

The bivector $\Abs{\rcap \wedge \Bv}$ has the magnitude of the non-radial component of the velocity.  This
equals the magnitude of the component of the velocity perpendicular to its radial component (ie: the angular component of the velocity).

\subsection{angular velocity.  Prep.}

Because $\Abs{\rcap \wedge \Bv}$ is the non-radial velocity component, for small angles
${v_\perp}/r$ will equal the angle between the vector and its displacement.

This allows for the calculation of the rate of change of that angle with time, what it called the scalar
angular velocity (dimensions are $1/t$ not $x/t$).  This can be done by taking the $\sin$ as the ratio of the
length of the non-radial component of the delta to the length of the displaced vector.

\begin{align*}
\sin d\theta &= \frac{\Abs{\rcap(\rcap \wedge d\Br)}}{\Abs{\Br + d\Br}} \\
\end{align*}

With $d\Br = \dt{\Br} dt = \Bv dt$, the angular velocity is

\begin{align*}
\sin d\theta
   &= \frac{1}{\Abs{\Br + \Bv dt}} \Abs{ \rcap (\rcap \wedge \Bv) dt } \\
   &= \frac{1}{\Abs{\Br + \Bv dt}} \Abs{ (\rcap \wedge \Bv) dt } \\
\frac{\sin d\theta}{\abs{dt}}
   &= \frac{1}{\Abs{\Br + \Bv dt}} \Abs{ \rcap \wedge \Bv } \\
   &= \frac{1}{\Abs{\Br}\Abs{\Br + \Bv dt}} \Abs{ \Br \wedge \Bv } \\
\end{align*}

In the limit, taking $dt > 0$, this is
\[
\omega = \dt{\theta} = \frac{1}{\Br^2} \Abs{ \Br \wedge \Bv }
\]

\subsection{angular velocity.  Summarizing.}

Here is a summary of calculations so far involving the $\Br \wedge \Bv$ bivector

\begin{align*}
\Bv &= \rcap v_r + \frac{\rcap}{\Abs{\Br}} (\Br \wedge \Bv) \\
\dt{\rcap} &= \frac{\rcap}{\Br^2} (\Br \wedge \Bv) \\
\abs{v_{\perp}} &= \frac{1}{\Abs{\Br}} \Abs{ \Br \wedge \Bv } \\
\omega = \dt{\theta} &= \frac{1}{\Br^2} \Abs{ \Br \wedge \Bv } \\
\end{align*}

It makes sense to give the bivector a name.  Given it's magnitude the 
angular velocity bivector $\Bomega$ is designated

\[
\Bomega = \frac{ \Br \wedge \Bv }{\Br^2} 
\]

So the linear and rotational components of the velocity can thus be expressed in terms of this, as can our
unit vector derivative, scalar angular velocity, and perpendicular velocity magnitude:

\begin{align*}
\omega = \dt{\theta} &= \Abs{ \Bomega } \\
\Bv &= \rcap v_r + \Br \Bomega \\
    &= \rcap( v_r + r \Bomega ) \\
\dt{\rcap} &= \rcap \Bomega \\
\abs{v_{\perp}} &= r \Abs{ \Bomega } \\
\end{align*}

This is similar to the vector angular velocity ($\Bomega = (\Br \times \Bv)/r^2$), but instead of lying perpendicular to the
plane of rotation, it defines the plane of rotation (for a vector $\Ba$, $\Ba \wedge \Bomega$ is zero if the vector is in the plane and non-zero if the vector has a component outside of the plane).

%\begin{align*}
%\Bomega = \frac{1}{\Br^2} (\Br \wedge \Bv)
%\end{align*}
%
%Or,
%\begin{align*}
%\Br \wedge \Bv = \Br^2 \Bomega = r^2\Bomega
%\end{align*}
%
%
%\begin{align*}
%\Bv 
%   &= \rcap(v_r + (1/r)\Br \wedge \Bv) \\
%   &= \rcap(v_r + r\Bomega) \\
%   &= v_r\rcap + \Br\Bomega \\
%\end{align*}

\subsection{Explicit perpendicular unit vector.}

If one introduces a unit vector $\thetacap$ in the direction of rejection of $\Br$ from $d\Br$, the total velocity takes the symmetrical form
\begin{align*}
\Bv 
   &= v_r\rcap + r\omega\thetacap \\
   &= \dt{r}\rcap + r\dt{\theta}\thetacap \\
\end{align*}

\subsection{acceleration in terms of angular velocity bivector}

Taking derivatives of velocity, one can with a bit of work,
express acceleration in terms of
radial and non-radial components

%\Br \wedge \Bv = \Br^2 \Bomega = r^2\Bomega

\begin{align*}
\Ba 
   &= (\rcap v_r + \Br \Bomega)' \\
   &= \rcap' v_r + \rcap v_r' + \Br' \Bomega + \Br \Bomega' \\
   &= \rcap \Bomega v_r + \rcap v_r' + \Br' \Bomega + \Br \Bomega' \\
   &= \rcap \Bomega v_r + \rcap a_r + \Bv \Bomega + \Br \Bomega' \\
\end{align*}

But,
\begin{align*}
\Bomega' &= ((1/r^2) (\Br \wedge \Bv))' \\
         &= (-2/r^3) r' (\Br \wedge \Bv) + (1/r^2) (\Bv \wedge \Bv + \Br \wedge \Ba) \\
         &= -(2/r) v_r \Bomega + (1/r^2) (\Br \wedge \Ba) \\
\end{align*}

%\rcap v_r = \Bv - \Br \Bomega
So,

\begin{align*}
\Ba 
   &= \rcap a_r -\rcap \Bomega v_r + \Bv \Bomega + \rcap (\rcap \wedge \Ba) \\
   &= \rcap a_r -( \Bv - \Br \Bomega) \Bomega + \Bv \Bomega + \rcap (\rcap \wedge \Ba) \\
\\
   &= \rcap a_r + \Br \Bomega^2+ \rcap (\rcap \wedge \Ba) \\
   &= \rcap( a_r + r \Bomega^2) + \rcap (\rcap \wedge \Ba) \\
\end{align*}

Note that $\Bomega^2$ is a negative scalar, so as normal writing $\norm{\Bomega}^2 = -\Bomega^2$, we have acceleration in a fashion similar to the
traditional cross product form:

\begin{align*}
\Ba 
   &= \rcap( a_r - r \norm{\Bomega}^2) + \rcap (\rcap \wedge \Ba) \\
   &= \rcap( a_r - r \norm{\Bomega}^2 + \rcap \wedge \Ba) \\
\end{align*}

In the traditional representation, this last term, the non-radial acceleration
component, is often expressed as a derivative.

In terms of the wedge product, this can be done by noting that

\[
(\Br \wedge \Bv)' = \Bv \wedge \Bv + \Br \wedge \Ba = \Br \wedge \Ba
\]

\begin{align*}
\Ba 
   &= \rcap( a_r - r \norm{\Bomega}^2 ) + \frac{\Br}{r^2}(\Br \wedge \Bv)') \\
   &= \rcap( a_r - r \norm{\Bomega}^2 ) + \frac{1}{\Br}\dt{(\Br^2 \Bomega)} \\
\end{align*}

Expressed in terms of force (for constant mass) this is
\begin{align*}
\BF &= m \Ba \\
    &= \rcap (m a_r) + (m \Br) {\Bomega}^2
       + \frac{1}{\Br}\dt{(m \Br^2 \Bomega)} \\
    &= \BF_r + (m \Br) {\Bomega}^2
             + \frac{1}{\Br}\dt{(m \Br^2 \Bomega)} \\
\end{align*}

Alternately, the non-radial term can be expressed in terms of torque

\begin{align*}
\rcap (\rcap \wedge \Ba) 
   &= \rcap (\rcap \wedge m \Ba)  \\
   &= \frac{\Br}{r^2} (\Br \wedge \BF)  \\
   &= \frac{1}{\Br} (\Br \wedge \BF)  \\
   &= \frac{1}{\Br} \Btau \\
\end{align*}

Thus the torque bivector, which in magnitude was the angular derivative of
the work
done by the force $\norm{\Btau} = \tau = \dtheta{W} = \BF \cdot \dtheta{\Br}$
is also expressible as a time derivative

\begin{align*}
\Btau 
&= \dt{( m \Br^2 \Bomega )}  \\
&= \dt{( m \Br \wedge \Bv)}  \\
&= \dt{( \Br \wedge m \Bv)}  \\
&= \dt{( \Br \wedge \Bp  )}  \\
\end{align*}

This bivector $m \Br^2 \Bomega = \Br \wedge \Bp$ is called the angular
momentum, designated $\BJ$.  It is related to the total momentum as follows

\[
\Bp = \rcap (\rcap \cdot \Bp) + \frac{1}{\Br} \BJ 
\]

So the total force is

\begin{align*}
\BF 
    &= \BF_r + m \Br {\Bomega}^2 + \frac{1}{\Br}\dt{\BJ} \\
\end{align*}

Observe that for a purely radial (ie: central) force, we must have
$\dt{\BJ} = 0$
so, the angular
momentum must be constant.

\subsection{Kepler's laws example.}

This follows the \citep{salas1990coa} treatment, modified for the GA notation.

Consider the gravitational force 

\begin{align*}
m \Ba &= -G \frac{m M}{r^2} \rcap \\
\Ba &= - G M \frac{\rcap}{r^2} = -\rho \frac{\rcap}{r^2}
\end{align*}

Or,
\[
\frac{\rcap}{r^2} = -\frac{1}{\rho} \dt{\Bv}
\]

The unit vector derivative is

\begin{align*}
\dt{\rcap} &= \frac{\rcap}{r}(\rcap \wedge \Bv) \\
           &= \frac{\rcap}{r^2}\frac{\BJ}{m} \\
           &= -\frac{1}{m \rho} \dt{\Bv} \BJ \\
           &= \dt{(-\frac{1}{m \rho} \Bv \BJ )} \\
\end{align*}

The last because $\BJ$, $m$, and $\rho$ are all constant.

Before continuing, let's examine this funny vector bivector product term.
In general a vector
bivector product will have vector and trivector parts, but
the differential equation implies that this is a vector.  Let's confirm this

\begin{align*}
\Bv \BJ &= \Bv (\Br \wedge m \Bv) \\
        &= (m \Bv^2) \vcap (\Br \wedge \vcap) \\
        &= - (m \Bv^2) \vcap (\vcap \wedge \Br) \\
\end{align*}

So, this is in fact a vector, it is the rejective component of $\Br$ from
the direction of $\vcap$ scaled by $-m\Bv^2$.  We can also calculate
the product $\BJ \Bv$ from this:

\begin{align*}
\Bv \BJ 
        &= - (m \Bv^2) \vcap (\vcap \wedge \Br) \\
        &= - (m \Bv^2) (\Br \wedge \vcap) \vcap \\
        &= - (\Br \wedge m \Bv) \Bv \\
        &= - \BJ \Bv \\
\end{align*}

This antisymetrical result $\Bv \BJ = - \BJ \Bv$ is actually the defining 
property of the vector bivector ``dot product'' (unlike the vector dot product
which is the symmetrical parts).  This vector bivector dot product selects the
vector component, leaving the trivector part.  Since $\Bv$ lies completely in
the plane of the angular velocity bivector $\Bv \wedge \BJ = 0$ in this case.

Anyways, back to the problem, integrating 
with respect to time, and introducing a vector integration constant $\Be$
we have

\[
\rcap + \frac{1}{m \rho} \Bv \BJ = \Be
\]

Multiplying by $\Br$

\begin{align*}
r + \frac{1}{m \rho} \Br \Bv \BJ &= \Br \Be \\
r + \frac{1}{m^2 \rho} (\Br \cdot \Bp + \BJ) \BJ &= \Br \cdot \Be + \Br \wedge \Be \\
\end{align*}

This results in three equations, one for each of the scalar, vector, and bivector parts

\begin{align*}
r + \frac{\BJ^2}{m^2 \rho} &= \Br \cdot \Be \\
\frac{1}{m \rho} (\Br \cdot \Bv) \BJ &= 0 \\
\Br \wedge \Be &= 0 \\
\end{align*}

The first of these equations is the result from Salas and Hille (integration constant differs in sign though).

\begin{align*}
r - \frac{J^2}{m^2 \rho} &= \Br \cdot \Be \\
%r - \Br \cdot \Be &= \frac{J^2}{m^2 \rho} \\
%\Br \rcap - \Br \cdot \Be &= \frac{J^2}{m^2 \rho} \\
%\Br \rcap - \Br \Be &= \frac{J^2}{m^2 \rho} \\
%\Br (\rcap - \Be) &= \frac{J^2}{m^2 \rho}
\end{align*}

%\[
%\frac{\Br_0}{\norm{\Br_0}} + \frac{1}{m \rho} \Bv_0 \BJ = \Be
%\]
%
%\[
%\rcap + \frac{1}{m \rho} \Bv \BJ = \frac{\Br_0}{\norm{\Br_0}} + \frac{1}{m \rho} \Bv_0 \BJ
%\]
%
%\[
%\rcap - \rcap_0 + \frac{1}{m \rho} (\Bv - \Bv_0) \BJ = 0
%\]
%
%J = r ^ m v = r mv - r . mv = r mv 
%v = 1/(r m) J

%Diverging from the Salas and Hille treatment, instead of producing a scalar
%equation, lets remove the $\Bv$ term from the equation:
%
%\[
%\rcap + \frac{1}{m \rho} \Bv \BJ = \Be
%\]
%
%First express $\Bv$ in terms of $\Br$ and $\BJ$.
%
%\begin{align*}
%\BJ 
%   &= \Br \wedge (m \Bv) \\
%   &= m \Br \Bv - m \Br \cdot \Bv \\
%   &= m \Br \Bv \\
%\end{align*}
%
%Thus,
%\[
%\Bv = \frac{1}{m \Br}\BJ
%\]
%
%%r_0 - \Br_0 \Be = \frac{1}{m^2 \rho} J^2)
%%-r_0 + \Br_0 \Be = -\frac{1}{m^2 \rho} J^2)
%%\Br_0 \Be = r_0 - \frac{1}{m^2 \rho} J^2)
%%\Be = (1/\Br_0)(r_0 - \frac{1}{m^2 \rho} J^2))
%
%And,
%\begin{align*}
%\rcap + \frac{1}{m^2 \rho \Br} \BJ^2 &= \Be \\
%\rcap(1 - \frac{1}{m^2 \rho r} J^2) &= \Be \\
%r - \frac{1}{m^2 \rho} J^2 &= \Br \Be \\
%r - \Br \Be &= \frac{1}{m^2 \rho} J^2 \\
%r - \Br (1/\Br_0)(r_0 - \frac{1}{m^2 \rho} J^2) &= \frac{1}{m^2 \rho} J^2 \\
%r - \Br \rcap_0 + \Br (1/\Br_0) \frac{1}{m^2 \rho} J^2 &= \frac{1}{m^2 \rho} J^2 \\
%\Br \rcap - \Br \rcap_0 + \Br (1/\Br_0) \frac{1}{m^2 \rho} J^2 &= \frac{1}{m^2 \rho} J^2 \\
%\Br (\rcap - \rcap_0 + (1/\Br_0) \frac{1}{m^2 \rho} J^2) &= \frac{1}{m^2 \rho} J^2 \\
%\end{align*}
%
%Since $\Br \cdot \Bv = 0$ then:
%
%\begin{align*}
%J^2 &= -(\Br \wedge m \Bv)^2 \\
%    &= m^2 (\Br \wedge \Bv)(\Bv \wedge \Br) \\
%    &= m^2 (\Br \Bv)(\Bv \Br) \\
%    &= m^2 \Br^2 \Bv^2 \\
%    &= m^2 \Br_0^2 \Bv_0^2 \\
%\end{align*}
%
%\begin{align*}
%\Br (\rcap - \rcap_0 + (1/\Br_0) \frac{1}{\rho} \Br_0^2 \Bv_0^2) &= \frac{1}{\rho} \Br_0^2 \Bv_0^2 \\
%\Br (\rcap - \rcap_0 + \frac{1}{\rho} \Br_0 \Bv_0^2) &= \frac{1}{\rho} \Br_0^2 \Bv_0^2 \\
%\end{align*}

\subsection{Circular motion}

For circular motion $v_r = a_r = 0$, so:

\[
\Bv = \Br \Bomega
\]
\[
\Ba = \rcap \left(  -\frac{\Bv^2}{r} + \rcap \wedge \Ba \right) \\
\]

For constant circular motion:
\begin{align*}
\Ba 
   &= \Bv\Bomega + \Br\Bomega' \\
   &= \Bv\Bomega + \Br(\Bzero) \\
   &= \Br(\Bomega)^2 \\
   &= -\Br\Abs{\Bomega}^2 \\
\end{align*}

ie: the $\rcap (\rcap \wedge \Ba )$ term is zero... all acceleration is inwards.

Can also expand this in terms of $\Br$ and $\Bv$:
\begin{align*}
\Ba 
   &= \Br\left(\Bomega\right)^2 \\
   &= \Br\left(\frac{1}{\Br}\Bv\right)^2 \\
   &= -\Br\left( \Bv \frac{1}{\Br} \frac{1}{\Br}\Bv \right) \\
   &= -\Br\left( \frac{\Bv^2}{\Br^2}\right) \\
   &= -\frac{1}{\Br}\Bv^2 \\
\end{align*}


\documentclass{article}      % Specifies the document class

\usepackage{amsmath}
\usepackage{mathpazo}

%
% shorthand for bold symbols, convenient for vectors and matrices
%
\newcommand{\Ba}[0]{\mathbf{a}}
\newcommand{\Bb}[0]{\mathbf{b}}
\newcommand{\Bc}[0]{\mathbf{c}}
\newcommand{\Bd}[0]{\mathbf{d}}
\newcommand{\Be}[0]{\mathbf{e}}
\newcommand{\Bf}[0]{\mathbf{f}}
\newcommand{\Bg}[0]{\mathbf{g}}
\newcommand{\Bh}[0]{\mathbf{h}}
\newcommand{\Bi}[0]{\mathbf{i}}
\newcommand{\Bj}[0]{\mathbf{j}}
\newcommand{\Bk}[0]{\mathbf{k}}
\newcommand{\Bl}[0]{\mathbf{l}}
\newcommand{\Bm}[0]{\mathbf{m}}
\newcommand{\Bn}[0]{\mathbf{n}}
\newcommand{\Bo}[0]{\mathbf{o}}
\newcommand{\Bp}[0]{\mathbf{p}}
\newcommand{\Bq}[0]{\mathbf{q}}
\newcommand{\Br}[0]{\mathbf{r}}
\newcommand{\Bs}[0]{\mathbf{s}}
\newcommand{\Bt}[0]{\mathbf{t}}
\newcommand{\Bu}[0]{\mathbf{u}}
\newcommand{\Bv}[0]{\mathbf{v}}
\newcommand{\Bw}[0]{\mathbf{w}}
\newcommand{\Bx}[0]{\mathbf{x}}
\newcommand{\By}[0]{\mathbf{y}}
\newcommand{\Bz}[0]{\mathbf{z}}
\newcommand{\BA}[0]{\mathbf{A}}
\newcommand{\BB}[0]{\mathbf{B}}
\newcommand{\BC}[0]{\mathbf{C}}
\newcommand{\BD}[0]{\mathbf{D}}
\newcommand{\BE}[0]{\mathbf{E}}
\newcommand{\BF}[0]{\mathbf{F}}
\newcommand{\BG}[0]{\mathbf{G}}
\newcommand{\BH}[0]{\mathbf{H}}
\newcommand{\BI}[0]{\mathbf{I}}
\newcommand{\BJ}[0]{\mathbf{J}}
\newcommand{\BK}[0]{\mathbf{K}}
\newcommand{\BL}[0]{\mathbf{L}}
\newcommand{\BM}[0]{\mathbf{M}}
\newcommand{\BN}[0]{\mathbf{N}}
\newcommand{\BO}[0]{\mathbf{O}}
\newcommand{\BP}[0]{\mathbf{P}}
\newcommand{\BQ}[0]{\mathbf{Q}}
\newcommand{\BR}[0]{\mathbf{R}}
\newcommand{\BS}[0]{\mathbf{S}}
\newcommand{\BT}[0]{\mathbf{T}}
\newcommand{\BU}[0]{\mathbf{U}}
\newcommand{\BV}[0]{\mathbf{V}}
\newcommand{\BW}[0]{\mathbf{W}}
\newcommand{\BX}[0]{\mathbf{X}}
\newcommand{\BY}[0]{\mathbf{Y}}
\newcommand{\BZ}[0]{\mathbf{Z}}

\newcommand{\Bzero}[0]{\mathbf{0}}
\newcommand{\Btheta}[0]{\boldsymbol{\theta}}
\newcommand{\Btau}[0]{\boldsymbol{\tau}}
\newcommand{\Bomega}[0]{\boldsymbol{\omega}}

%
% shorthand for unit vectors
%
\newcommand{\acap}[0]{\hat{\Ba}}
\newcommand{\bcap}[0]{\hat{\Bb}}
\newcommand{\ccap}[0]{\hat{\Bc}}
\newcommand{\dcap}[0]{\hat{\Bd}}
\newcommand{\ecap}[0]{\hat{\Be}}
\newcommand{\fcap}[0]{\hat{\Bf}}
\newcommand{\gcap}[0]{\hat{\Bg}}
\newcommand{\hcap}[0]{\hat{\Bh}}
\newcommand{\icap}[0]{\hat{\Bi}}
\newcommand{\jcap}[0]{\hat{\Bj}}
\newcommand{\kcap}[0]{\hat{\Bk}}
\newcommand{\lcap}[0]{\hat{\Bl}}
\newcommand{\mcap}[0]{\hat{\Bm}}
\newcommand{\ncap}[0]{\hat{\Bn}}
\newcommand{\ocap}[0]{\hat{\Bo}}
\newcommand{\pcap}[0]{\hat{\Bp}}
\newcommand{\qcap}[0]{\hat{\Bq}}
\newcommand{\rcap}[0]{\hat{\Br}}
\newcommand{\scap}[0]{\hat{\Bs}}
\newcommand{\tcap}[0]{\hat{\Bt}}
\newcommand{\ucap}[0]{\hat{\Bu}}
\newcommand{\vcap}[0]{\hat{\Bv}}
\newcommand{\wcap}[0]{\hat{\Bw}}
\newcommand{\xcap}[0]{\hat{\Bx}}
\newcommand{\ycap}[0]{\hat{\By}}
\newcommand{\zcap}[0]{\hat{\Bz}}
\newcommand{\thetacap}[0]{\hat{\Btheta}}

%
% to write R^n and C^n in a distinguishable fashion.  Perhaps change this
% to the double lined characters upon figuring out how to do so.
%
\newcommand{\C}[1]{$\mathbb{C}^{#1}$}
\newcommand{\R}[1]{$\mathbb{R}^{#1}$}

%
% various generally useful helpers
%

% derivative of #1 wrt. #2:
\newcommand{\D}[2] {\frac {d#2} {d#1}}

\newcommand{\inv}[1]{\frac{1}{#1}}
\newcommand{\cross}[0]{\times}

\newcommand{\abs}[1]{\lvert{#1}\rvert}
\newcommand{\norm}[1]{\lVert{#1}\rVert}
\newcommand{\innerprod}[2]{\langle{#1}, {#2}\rangle}
\newcommand{\dotprod}[2]{{#1} \cdot {#2}}
\newcommand{\bdotprod}[2]{\left({#1} \cdot {#2}\right)}
\newcommand{\crossprod}[2]{{#1} \cross {#2}}
\newcommand{\tripleprod}[3]{\dotprod{\left(\crossprod{#1}{#2}\right)}{#3}}

\DeclareMathOperator{\Proj}{Proj}
\DeclareMathOperator{\Span}{span}
\DeclareMathOperator{\Sgn}{sgn}
\DeclareMathOperator{\Area}{Area}
\DeclareMathOperator{\Volume}{Volume}

%
% A few miscellaneous things specific to this document
%
\newcommand{\crossop}[1]{\crossprod{#1}{}}

% R2 vector.
\newcommand{\VectorTwo}[2]{
\begin{bmatrix}
 {#1} \\
 {#2}
\end{bmatrix}
}

\newcommand{\VectorN}[1]{
\begin{bmatrix}
{#1}_1 \\
{#1}_2 \\
\vdots \\
{#1}_N \\
\end{bmatrix}
}

\newcommand{\DETuvij}[4]{
\begin{vmatrix}
 {#1}_{#3} & {#1}_{#4} \\
 {#2}_{#3} & {#2}_{#4}
\end{vmatrix}
}

\newcommand{\DETuvwijk}[6]{
\begin{vmatrix}
 {#1}_{#4} & {#1}_{#5} & {#1}_{#6} \\
 {#2}_{#4} & {#2}_{#5} & {#2}_{#6} \\
 {#3}_{#4} & {#3}_{#5} & {#3}_{#6}
\end{vmatrix}
}

\newcommand{\DETuvwxijkl}[8]{
\begin{vmatrix}
 {#1}_{#5} & {#1}_{#6} & {#1}_{#7} & {#1}_{#8} \\
 {#2}_{#5} & {#2}_{#6} & {#2}_{#7} & {#2}_{#8} \\
 {#3}_{#5} & {#3}_{#6} & {#3}_{#7} & {#3}_{#8} \\
 {#4}_{#5} & {#4}_{#6} & {#4}_{#7} & {#4}_{#8} \\
\end{vmatrix}
}

%\newcommand{\DETuvwxyijklm}[10]{
%\begin{vmatrix}
% {#1}_{#6} & {#1}_{#7} & {#1}_{#8} & {#1}_{#9} & {#1}_{#10} \\
% {#2}_{#6} & {#2}_{#7} & {#2}_{#8} & {#2}_{#9} & {#2}_{#10} \\
% {#3}_{#6} & {#3}_{#7} & {#3}_{#8} & {#3}_{#9} & {#3}_{#10} \\
% {#4}_{#6} & {#4}_{#7} & {#4}_{#8} & {#4}_{#9} & {#4}_{#10} \\
% {#5}_{#6} & {#5}_{#7} & {#5}_{#8} & {#5}_{#9} & {#5}_{#10}
%\end{vmatrix}
%}

% R3 vector.
\newcommand{\VectorThree}[3]{
\begin{bmatrix}
 {#1} \\
 {#2} \\
 {#3}
\end{bmatrix}
}



\newcommand{\laplacian}[0]{\nabla^2}
\newcommand{\Dsq}[2] {\frac {\partial^2 {#1}} {\partial {#2}^2}}
\newcommand{\dxj}[2] {\frac {\partial {#1}} {\partial {x_{#2}}}}
\newcommand{\dsqxj}[2] {\frac {\partial^2 {#1}} {\partial {x_{#2}}^2}}
\DeclareMathOperator{\Exp}{e}
\DeclareMathOperator{\Rej}{Rej}
\newcommand{\gpgrade}[2] {{\left\langle{{#1}}\right\rangle}_{#2}}
\newcommand{\gpgradezero}[1] {\gpgrade{#1}{0}}
\newcommand{\gpgradetwo}[1] {\gpgrade{#1}{2}}
\newcommand{\gpgradefour}[1] {\gpgrade{#1}{4}}

%
% The real thing:
%

                             % The preamble begins here.
\title{Geometry of intersecting bivectors}
\author{Peeter Joot}         % Declares the author's name.
%\date{}        % Deleting this command produces today's date.

\begin{document}             % End of preamble and beginning of text.

\maketitle{}

\section{ The problem. }

Examination of exponential solutions for Laplace's equation leads one to
a requirement to examine the product of intersecting bivectors such as

\[
\left(\abs{\Bx \wedge \Bk}^2\right)' = -\left(
(\Bx' \wedge \Bv)(\Bx \wedge \Bv)
+(\Bx \wedge \Bv)(\Bx' \wedge \Bv)
\right)
\]

Here we see that the symmetric sum of bivectors $\Bx \wedge \Bk$ and $\Bx' \wedge \Bk$ is a scalar quantity.  This we will identify later as a quantity
related to the bivector dot product.

It is worthwhile to systematically examine the
general products of intersecting bivectors, that is planes that share a common line, in this case the line directed along the vector $\Bk$.
It is also notable that since all non coplanar bivectors in \R{3} intersect
this
examination will cover the important special case of three dimensional
plane geometry.

A result of this examination is that many of the concepts familiar from
vector geometry such as
orthagonality, projection, and rejection will have direct bivector
equivalents.

General bivector geometry, in spaces where non-coplanar bivectors do not 
neccessarily intersect (such as in \R{4}), will need to be treated separately,
but some of the grade 4 product terms will be carried below to explicitly
hightlight the point where the intersecting bivector space requirement
effects the results.

\section{The meat.}

The geometric product of two bivectors can be written:

\begin{equation}\label{eqn:ABprod}
\BA \BB = 
\gpgrade{\BA \BB}{0}
+\gpgrade{\BA \BB}{2}
+\gpgrade{\BA \BB}{4}
= 
{\BA \cdot \BB}
+\gpgrade{\BA \BB}{2}
+{\BA \wedge \BB}
\end{equation}
\begin{equation}\label{eqn:BAprod}
\BB \BA = 
\gpgrade{\BB \BA}{0}
+\gpgrade{\BB \BA}{2}
+\gpgrade{\BB \BA}{4}
= 
{\BB \cdot \BA}
+\gpgrade{\BB \BA}{2}
+{\BB \wedge \BA}
\end{equation}

Because we have three terms involved, unlike the vector dot and wedge product
we cannot generally separate these terms by 
symmetric and antisymmetric parts.  However forming those sums
will still worthwhile, especially for the case of interecting bivectors
since the last term will be zero in that case.

\subsection{ Sign change of each grade term with commutation. }

Starting with the last term we can first observe that

\begin{equation}\label{eqn:wedgesign}
\BA \wedge \BB = \BB \wedge \BA
\end{equation}

To show this let $\BA = \Ba \wedge \Bb$, and $\BB = \Bc \wedge \Bd$.  When

$\BA \wedge \BB \ne 0$, one can write:

\begin{align*}
\BA \wedge \BB 
&= \Ba \wedge \Bb \wedge \Bc \wedge \Bd \\
&= - \Bb \wedge \Bc \wedge \Bd \wedge \Ba \\
&= \Bc \wedge \Bd \wedge \Ba \wedge \Bb \\
&= \BB \wedge \BA \\
\end{align*}

To see how the signs of the remaining two terms vary with commutation
form:

\begin{align*}
(\BA + \BB)^2
&= (\BA + \BB)(\BA + \BB) \\
&= \BA^2 + \BB^2 + \BA \BB + \BB \BA \\
\end{align*}

When $\BA$ and $\BB$ interect we can write
$\BA = \Ba \wedge \Bx$, and $\BB = \Bb \wedge \Bx$, thus the sum is a bivector

\[
(\BA + \BB)
= (\Ba + \Bb) \wedge \Bx
\]

And so, the square of the two is a scalar.  When $\BA$ and $\BB$ have only
non intersecting components, such as the grade two \R{4} multivector
$\Be_{12} + \Be_{34}$, the square of this sum will have both grade four and
scalar parts.

Since the LHS = RHS, and the grades of the two also must be the same.
This implies that the quantity

\[
\BA \BB + \BB \BA = 
\BA \cdot \BB + \BB \cdot \BA
+\gpgradetwo{\BA \BB} + \gpgradetwo{\BB \BA}
+\BA \wedge \BB + \BB \wedge \BA
\]

is a scalar $\iff$ 
$\BA + \BB$ is a bivector, and in general has scalar and grade four terms.
Because this symmetric sum has no grade two terms, 
regardless of whether $\BA$, and $\BB$ intersect, we have:

\[
\gpgradetwo{\BA \BB} + \gpgradetwo{\BB \BA} = 0
\]
\begin{equation}\label{eqn:signgradetwo}
\implies
\gpgradetwo{\BA \BB} = -\gpgradetwo{\BB \BA}
\end{equation}

One would intuitively expect $\BA \cdot \BB = \BB \cdot \BA$.  This can be
demonstrated by forming the complete symmetric sum

\begin{align*}
\BA \BB + \BB \BA 
&= 
{\BA \cdot \BB} +{\BB \cdot \BA}
+\gpgrade{\BA \BB}{2} +\gpgrade{\BB \BA}{2}
+{\BA \wedge \BB} + {\BB \wedge \BA} \\
&= 
{\BA \cdot \BB} +{\BB \cdot \BA}
+\gpgrade{\BA \BB}{2} -\gpgrade{\BA \BB}{2}
+{\BA \wedge \BB} + {\BA \wedge \BB} \\
&= 
{\BA \cdot \BB} +{\BB \cdot \BA}
+2{\BA \wedge \BB} \\
\end{align*}

The LHS commutes with interchange of $\BA$ and $\BB$, as does
${\BA \wedge \BB}$.  So for the RHS to also commute, the remaining grade 0 term
must also:

\begin{equation}\label{eqn:dotsign}
\BA \cdot \BB = \BB \cdot \BA
\end{equation}

\subsection{ Dot, wedge and grade two terms of bivector product. }

Collecting the results of the previous section and substituiting back
into equation \ref{eqn:ABprod} we have:

\begin{equation}\label{eqn:AdotB}
\BA \cdot \BB = \gpgrade{\frac{\BA \BB + \BB\BA}{2}}{0}
\end{equation}

\begin{equation}\label{eqn:AtwoB}
\gpgradetwo{\BA \BB} = \frac{\BA \BB - \BB\BA}{2}
\end{equation}

\begin{equation}\label{eqn:AwedgeB}
\BA \wedge \BB = \gpgrade{\frac{\BA \BB + \BB\BA}{2}}{4}
\end{equation}

When these intersect in a line the wedge term is zero, so for that special case we can write:

\begin{equation*}
\BA \cdot \BB = \frac{\BA \BB + \BB\BA}{2}
\end{equation*}

\begin{equation*}
\gpgradetwo{\BA \BB} = \frac{\BA \BB - \BB\BA}{2}
\end{equation*}

\begin{equation*}
\BA \wedge \BB = 0
\end{equation*}

(note that this is always the case for \R{3}).

\section{ Intersection of planes. }

Starting with two planes specified parametrically, each in terms of two direction vectors and a point on the plane:

\begin{align}\label{eqn:twoplanes}
\Bx &= \Bp + \alpha \Bu + \beta \Bv \\
\By &= \Bq + a \Bw + b \Bz \\
\end{align}

If these intersect then all points on the line must satisify $\Bx = \By$, so the
solution requires:

\[
\Bp + \alpha \Bu + \beta \Bv = \Bq + a \Bw + b \Bz
\]
\[
\implies
(\Bp + \alpha \Bu + \beta \Bv) \wedge \Bw \wedge \Bz = (\Bq + a \Bw + b \Bz) \wedge \Bw \wedge \Bz = \Bq \wedge \Bw \wedge \Bz
\]

Rearranging for $\beta$, and writing $\BB = \Bw \wedge \Bz$:

\[
\beta = \frac{\Bq \wedge \BB - (\Bp + \alpha \Bu) \wedge \BB}{\Bv \wedge \BB}
\]

Note that when the solution exists the left vs right order of the division by $\Bv \wedge \BB$ should not matter since the numerator will be proportional to this bivector (or else the $\beta$ would not be a scalar).

Substitution of $\beta$ back into $\Bx = \Bp + \alpha \Bu + \beta \Bv$ (all points in the first plane) gives you a parametric equation for a line:

\[
\Bx = \Bp + \frac{(\Bq-\Bp)\wedge \BB}{\Bv \wedge \BB}\Bv + \alpha\frac{1}{\Bv \wedge \BB}((\Bv \wedge \BB) \Bu - (\Bu \wedge \BB)\Bv)
\]

Where a point on the line is:

\[
\Bp + \frac{(\Bq-\Bp)\wedge \BB}{\Bv \wedge \BB}\Bv 
%= \frac{1}{\Bv \wedge \BB}((\Bv \wedge \BB)\Bp + ((\Bq-\Bp)\wedge \BB)\Bv)
\]

And a direction vector for the line is:

\[
\frac{1}{\Bv \wedge \BB}((\Bv \wedge \BB) \Bu - (\Bu \wedge \BB)\Bv)
\]
\[
\propto
(\Bv \wedge \BB)^2 \Bu - (\Bv \wedge \BB)(\Bu \wedge \BB)\Bv
\]

Now, this result is only valid if $\Bv \wedge \BB \ne 0$ (ie: line of intersection is not directed along $\Bv$), but if that is the case the second form will be zero.  Thus we can add the results (or any non-zero linear combination of) allowing for either of $\Bu$, or $\Bv$ to be directed along the line of intersection:

\begin{equation}\label{eqn:dirvecintersection}
a\left( (\Bv \wedge \BB)^2 \Bu
- (\Bv \wedge \BB)(\Bu \wedge \BB)\Bv \right)
+ b\left((\Bu \wedge \BB)^2 \Bv 
- (\Bu \wedge \BB)(\Bv \wedge \BB)\Bu\right)
\end{equation}

Alternately, one could formulate this in terms of $\BA = \Bu \wedge \Bv$, $\Bw$, and $\Bz$.  Is there a more symetrical form for this direction vector?

\subsection{ Vector along line of intersection in \R{3}}

For \R{3} one can solve the intersection problem using the normals to the planes.  For simplicity put the origin on the line of intersection (and all planes through a common point in \R{3} have at least a line of intersection).  In this case, for bivectors $\BA$ and $\BB$, normals to those planes are $i\BA$, and $i\BB$ respectively.  The plane through both of those normals is:

\begin{align*}
(i\BA) \wedge (i\BB)
= \frac{(i\BA)(i\BB) - (i\BB)(i\BA)}{2} 
= \frac{\BB\BA - \BA\BB}{2} 
= \gpgradetwo{\BB\BA}
\end{align*}

The normal to this plane

\begin{equation}\label{eqn:r3planeintersect}
i\gpgradetwo{\BB\BA}
\end{equation}

is directed along the line of interesection.  This result is more appealing than
the general \R{N} result of equation \ref{eqn:dirvecintersection}, not
just because it is simpler, but also because it is a function of only the
bivectors for the planes, without a requirement to find or calculate
two specific independent direction vectors in one of the planes.

\subsection{ Applying this result to \R{N} }

If you reject the component of $\BA$ from $\BB$ for two intersecting bivectors:

\[
\Rej_{\BA}(\BB) = \frac{1}{\BA}\gpgradetwo{\BA\BB}
\]

the line of intersection remains the same ... that operation rotates $\BB$ so that the two are mutually perpendicular.  This essentially reduces the problem to that of the three dimensional case, so the solution has to be of the same form... you just need to calculate a ``pseudoscalar'' (what you are calling the join), for the subspace spanned by the two bivectors.

That can be computed by taking any direction vector that is on one plane, but isn't in the second.  For example, pick a vector $\Bu$ in the plane $\BA$ that is not on the intersection of $\BA$ and $\BB$.  In mathese that is $\Bu = \inv{\BA}(\BA\cdot \Bu)$ (or $\Bu \wedge \BA = 0$), where $\Bu \wedge \BB \ne 0$.  Thus a pseudoscalar for this subspace is:

\[
\Bi = \frac{\Bu \wedge \BB}{\abs{\Bu \wedge \BB}}
\]

To calculate the direction vector along the intersection we don't care about the scaling above.  Also note that provided $\Bu$ has a component in the plane $\BA$, $\Bu \cdot \BA$ is also in the plane (it's rotated $\pi/2$ from $\inv{\BA}(\BA \cdot \Bu)$.

Thus, provided that $\Bu \cdot \BA$ isn't on the intersection, a scaled ``pseudoscalar''
for the subspace can be calculated by taking from any vector $\Bu$ with a component in the plane $\BA$:

\[
\Bi \propto (\Bu \cdot \BA) \wedge \BB
\]

Thus a vector along the intersection is:

\begin{equation}\label{eqn:pseudoscalarinter}
\Bd = ((\Bu \cdot \BA) \wedge \BB) \gpgradetwo{\BA\BB}
\end{equation}

(an interchange of $\BA$ and $\BB$ above would also work).

Without showing the steps one can write the complete parametric solution of the line through the planes of equation \ref{eqn:twoplanes} in terms of this direction vector:

\begin{equation}\label{eqn:finalsolnofRNplaneintersection}
\Bx = \Bp + \left(\frac{(\Bq - \Bp)\wedge \BB}{(\Bd \cdot \BA) \wedge \BB}\right) (\Bd \cdot \BA) + \alpha \Bd
\end{equation}

Since $(\Bd \cdot \BA) \ne 0$ and $(\Bd \cdot \BA) \wedge \BB \ne 0$ (unless $\BA$ and $\BB$ are coplanar), observe that this is a natural generator
of the pseudoscalar for the subspace, and as such shows up in the expression
above.

\section{ Grade components of a trivector product. }

While trying to put equation \ref{eqn:dirvecintersection} into a form
that eliminated $\Bu$, and $\Bv$ in favour of $\BA = \Bu \wedge \Bv$
symmetric and antisymmetric formulations for the various grade terms
of a trivector product looked like they could be handy.  Here's a summary
of those results.

\subsection{ Grade 6 term. }

Writing two trivectors in terms
of mutually orthogonal components

\[
\BA = \Bx \wedge \By \wedge \Bz = \Bx\By\Bz
\]

and

\[
\BB = \Bu \wedge \Bv \wedge \Bw =\Bu\Bv\Bw
\]

Assuming that there is no common vector between the two, the 
wedge of these is

\begin{align*}
\BA \wedge \BB 
&= \gpgrade{\BA\BB}{6} \\
&= \gpgrade{\Bx\By\Bz\Bu\Bv\Bw}{6} \\
&= \gpgrade{\By\Bz(\Bx\Bu)\Bv\Bw}{6} \\
&= \gpgrade{\By\Bz(-\Bu\Bx + 2\Bu \cdot \Bx)\Bv\Bw}{6} \\
&= -\gpgrade{\By\Bz\Bu(\Bx\Bv)\Bw}{6} \\
&= -\gpgrade{\By\Bz\Bu(-\Bv\Bx + 2\Bv \cdot \Bx)\Bw}{6} \\
&= \gpgrade{\By\Bz\Bu\Bv(\Bx\Bw)}{6} \\
&= \cdots \\
&= -\gpgrade{\Bu\Bv\Bw\Bx\By\Bz}{6} \\
&= -\gpgrade{\BB\BA}{6} \\
&= -\BB \wedge \BA
\end{align*}

Note above that any interchange of terms inverts the sign (demonstrated 
explicitly for all the $\Bx$ interchanges).

As an aside, this
sign change on interchange is taken as the defining property of the 
wedge product in differential forms.  That property also
implies also that the wedge product is
zero when a vector is wedged with itself since zero is the only
value that is the negation of itself.  Thus we see explicitly
how the notation of using the wedge for the highest grade term
of two blades is consistent with the traditional
wedge product definition.

The end result here is that the grade 6 term of a trivector trivector product
changes sign on interchange of the trivectors:

\begin{equation}\label{eqn:trivecgpgrade6}
\gpgrade{\BA\BB}{6} = -\gpgrade{\BB\BA}{6}
\end{equation}

\subsection{ Grade 4 term. }

For a trivector product to have a grade 4 term there must be a common
vector between the two

\[
\BA = \Bx \wedge \By \wedge \Bz = \Bx\By\Bz
\]

and

\[
\BB = \Bu \wedge \Bv \wedge \Bz =\Bu\Bv\Bz
\]

The grade four term of the product is

\begin{align*}
\gpgrade{\BB \BA}{4}
&= \gpgrade{ \Bu\Bv\Bz \Bx\By\Bz }{4} \\
&= \gpgrade{ \Bu\Bv\Bz \Bz\Bx\By }{4} \\
&= \Bz^2\gpgrade{ \Bu\Bv\Bx\By }{4} \\
&= \Bz^2\gpgrade{ \Bu(\Bv\Bx)\By }{4} \\
&= \Bz^2\gpgrade{ \Bu(-\Bx\Bv + 2 \Bx \cdot \Bv)\By }{4} \\
&= -\Bz^2\gpgrade{ \Bu\Bx\Bv\By }{4} \\
&= \cdots \\
&= \Bz^2\gpgrade{ \Bx\By\Bu\Bv }{4} \\
&= \gpgrade{ \Bx\By\Bz\Bz\Bu\Bv }{4} \\
&= \gpgrade{ \Bx\By\Bz\Bu\Bv\Bz }{4} \\
&= \gpgrade{ \Bx\By\Bz\Bu\Bv\Bz }{4} \\
&= \gpgrade{\BA \BB}{4}
\end{align*}

Thus the grade 4 term commutes on interchange:

\begin{equation}\label{eqn:trivecgpgrade4}
\gpgrade{\BA\BB}{4} = \gpgrade{\BB\BA}{4}
\end{equation}

\subsection{ Grade 2 term. }

Similar to above, 
for a trivector product to have a grade 2 term there must be two common
vectors between the two

\[
\BA = \Bx \wedge \By \wedge \Bz = \Bx\By\Bz
\]

and

\[
\BB = \Bu \wedge \By \wedge \Bz =\Bu\By\Bz
\]

The grade two term of the product is

\begin{align*}
\gpgrade{\BA \BB}{2}
&= \gpgrade{ \Bx\By\Bz \Bu\By\Bz }{2} \\
&= \gpgrade{ \Bx\By\Bz \By\Bz \Bu}{2} \\
&= (\By\Bz)^2\gpgrade{ \Bx \Bu}{2} \\
&= -(\By\Bz)^2\gpgrade{ \Bu \Bx}{2} \\
&= -\gpgrade{ \BB \BA }{2} \\
\end{align*}

The grade 2 term anticommutes on interchange:

\begin{equation}\label{eqn:trivecgpgrade2}
\gpgrade{\BA\BB}{2} = -\gpgrade{\BB\BA}{2}
\end{equation}

\subsection{ Grade 0 term. }

Any grade 0 terms are due to products of the form $\BA = k\BB$

\begin{align*}
\gpgrade{\BA \BB}{0}
&= \gpgrade{k\BB \BB}{0} \\
&= \gpgrade{\BB k\BB}{0} \\
&= \gpgrade{\BB \BA}{0} \\
\end{align*}

The grade 2 term commutes on interchange:

\begin{equation}\label{eqn:trivecgpgrade0}
\gpgrade{\BA\BB}{0} = \gpgrade{\BB\BA}{0}
\end{equation}

\subsection{ combining results. }

\begin{equation*}
\BA \BB
=\gpgrade{\BA\BB}{0}
+\gpgrade{\BA\BB}{2}
+\gpgrade{\BA\BB}{4}
+\gpgrade{\BA\BB}{6}
\end{equation*}

\begin{align*}
\BB\BA
&=\gpgrade{\BB\BA}{0}
+\gpgrade{\BB\BA}{2}
+\gpgrade{\BB\BA}{4}
+\gpgrade{\BB\BA}{6} \\
&=\gpgrade{\BA\BB}{0}
-\gpgrade{\BA\BB}{2}
+\gpgrade{\BA\BB}{4}
-\gpgrade{\BA\BB}{6} \\
\end{align*}

These can be combined to express each of the grade terms as subsets
of the symmetric and antisymmetric parts:

\begin{align*}
\BA \cdot \BB = \gpgrade{\BA\BB}{0} &= \gpgrade{\frac{\BA\BB + \BB\BA}{2}}{0} \\
\gpgrade{\BA\BB}{2} &= \gpgrade{\frac{\BA\BB - \BB\BA}{2}}{2} \\
\gpgrade{\BA\BB}{4} &= \gpgrade{\frac{\BA\BB + \BB\BA}{2}}{4} \\
\BA \wedge \BB = \gpgrade{\BA\BB}{6} &= \gpgrade{\frac{\BA\BB - \BB\BA}{2}}{6} \\
\end{align*}

Note that above I've been somewhat loose with the argument above.  A grade three vector
will have the following form:

\[
\sum_{i<j<k} D_{ijk} \Be_{ijk}
\]

Where $D_{ijk}$ is the determinant of $ijk$ components of the vectors being wedged.  Thus the product
of two trivectors will be of the following form:

\[
\sum_{i<j<k} \sum_{i'<j'<k'} D_{ijk} D'_{i'j'k'} (\Be_{ijk} \Be_{i'j'k'})
\]

It's really each of these $\Be_{ijk} \Be_{i'j'k'}$ products that have to be considered in the grade 
and sign arguments above.  The end result will be the same though... one would just have to present
it a bit more carefully for a true proof.

\subsection{ Intersecting trivector cases. }

As with the intersecting bivector case, when there is a line of intersection between the two volumes one can
write:

\begin{align*}
\BA \cdot \BB = \gpgrade{\BA\BB}{0} &= \gpgrade{\frac{\BA\BB + \BB\BA}{2}}{0} \\
\gpgrade{\BA\BB}{2} &= \frac{\BA\BB - \BB\BA}{2} \\
\gpgrade{\BA\BB}{4} &= \gpgrade{\frac{\BA\BB + \BB\BA}{2}}{4} \\
\BA \wedge \BB = \gpgrade{\BA\BB}{6} &= 0 \\
\end{align*}

And if these volumes intersect in a plane a further simplification is possible:
\begin{align*}
\BA \cdot \BB = \gpgrade{\BA\BB}{0} &= \frac{\BA\BB + \BB\BA}{2} \\
\gpgrade{\BA\BB}{2} &= \frac{\BA\BB - \BB\BA}{2} \\
\gpgrade{\BA\BB}{4} &= 0 \\
\BA \wedge \BB = \gpgrade{\BA\BB}{6} &= 0 \\
\end{align*}

\end{document}               % End of document.

%
% Copyright � 2012 Peeter Joot.  All Rights Reserved.
% Licenced as described in the file LICENSE under the root directory of this GIT repository.
%

%
%
\chapter{Trivector geometry}
\index{trivector}
\label{chap:trivector}
%\date{Mar 9, 2008.  trivector.tex}

\section{Motivation}

The direction vector for two intersecting planes can be found to have the
form:

\begin{equation}\label{eqn:trivector:dirvecintersection}
a\left( (\Bv \wedge \BB)^2 \Bu
- (\Bv \wedge \BB)(\Bu \wedge \BB)\Bv \right)
+ b\left((\Bu \wedge \BB)^2 \Bv
- (\Bu \wedge \BB)(\Bv \wedge \BB)\Bu\right)
\end{equation}

While trying to put \eqnref{eqn:trivector:dirvecintersection} into a form
that eliminated \(\Bu\), and \(\Bv\) in favor of \(\BA = \Bu \wedge \Bv\)
symmetric and antisymmetric formulations for the various grade terms
of a trivector product looked like they could be handy.  Here is a summary
of those results.

\section{Grade components of a trivector product}

\subsection{Grade 6 term}

Writing two trivectors in terms
of mutually orthogonal components

\begin{equation}\label{eqn:trivector:26}
\BA = \Bx \wedge \By \wedge \Bz = \Bx\By\Bz
\end{equation}

and

\begin{equation}\label{eqn:trivector:46}
\BB = \Bu \wedge \Bv \wedge \Bw =\Bu\Bv\Bw
\end{equation}

Assuming that there is no common vector between the two, the
wedge of these is

\begin{equation}\label{eqn:trivector:186}
\begin{aligned}
\BA \wedge \BB
&= \gpgrade{\BA\BB}{6} \\
&= \gpgrade{\Bx\By\Bz\Bu\Bv\Bw}{6} \\
&= \gpgrade{\By\Bz(\Bx\Bu)\Bv\Bw}{6} \\
&= \gpgrade{\By\Bz(-\Bu\Bx + 2\Bu \cdot \Bx)\Bv\Bw}{6} \\
&= -\gpgrade{\By\Bz\Bu(\Bx\Bv)\Bw}{6} \\
&= -\gpgrade{\By\Bz\Bu(-\Bv\Bx + 2\Bv \cdot \Bx)\Bw}{6} \\
&= \gpgrade{\By\Bz\Bu\Bv(\Bx\Bw)}{6} \\
&= \cdots \\
&= -\gpgrade{\Bu\Bv\Bw\Bx\By\Bz}{6} \\
&= -\gpgrade{\BB\BA}{6} \\
&= -\BB \wedge \BA
\end{aligned}
\end{equation}

Note above that any interchange of terms inverts the sign (demonstrated
explicitly for all the \(\Bx\) interchanges).

As an aside, this
sign change on interchange is taken as the defining property of the
wedge product in differential forms.  That property also
implies also that the wedge product is
zero when a vector is wedged with itself since zero is the only
value that is the negation of itself.  Thus we see explicitly
how the notation of using the wedge for the highest grade term
of two blades is consistent with the traditional
wedge product definition.

The end result here is that the grade 6 term of a trivector trivector product
changes sign on interchange of the trivectors:

\begin{equation}\label{eqn:trivector:trivecgpgrade6}
\gpgrade{\BA\BB}{6} = -\gpgrade{\BB\BA}{6}
\end{equation}

\subsection{Grade 4 term}

For a trivector product to have a grade 4 term there must be a common
vector between the two

\begin{equation}\label{eqn:trivector:66}
\BA = \Bx \wedge \By \wedge \Bz = \Bx\By\Bz
\end{equation}

and

\begin{equation}\label{eqn:trivector:86}
\BB = \Bu \wedge \Bv \wedge \Bz =\Bu\Bv\Bz
\end{equation}

The grade four term of the product is

\begin{equation}\label{eqn:trivector:206}
\begin{aligned}
\gpgrade{\BB \BA}{4}
&= \gpgrade{ \Bu\Bv\Bz \Bx\By\Bz }{4} \\
&= \gpgrade{ \Bu\Bv\Bz \Bz\Bx\By }{4} \\
&= \Bz^2\gpgrade{ \Bu\Bv\Bx\By }{4} \\
&= \Bz^2\gpgrade{ \Bu(\Bv\Bx)\By }{4} \\
&= \Bz^2\gpgrade{ \Bu(-\Bx\Bv + 2 \Bx \cdot \Bv)\By }{4} \\
&= -\Bz^2\gpgrade{ \Bu\Bx\Bv\By }{4} \\
&= \cdots \\
&= \Bz^2\gpgrade{ \Bx\By\Bu\Bv }{4} \\
&= \gpgrade{ \Bx\By\Bz\Bz\Bu\Bv }{4} \\
&= \gpgrade{ \Bx\By\Bz\Bu\Bv\Bz }{4} \\
&= \gpgrade{ \Bx\By\Bz\Bu\Bv\Bz }{4} \\
&= \gpgrade{\BA \BB}{4}
\end{aligned}
\end{equation}

Thus the grade 4 term commutes on interchange:

\begin{equation}\label{eqn:trivector:trivecgpgrade4}
\gpgrade{\BA\BB}{4} = \gpgrade{\BB\BA}{4}
\end{equation}

\subsection{Grade 2 term}

Similar to above,
for a trivector product to have a grade 2 term there must be two common
vectors between the two

\begin{equation}\label{eqn:trivector:106}
\BA = \Bx \wedge \By \wedge \Bz = \Bx\By\Bz
\end{equation}

and

\begin{equation}\label{eqn:trivector:126}
\BB = \Bu \wedge \By \wedge \Bz =\Bu\By\Bz
\end{equation}

The grade two term of the product is

\begin{equation}\label{eqn:trivector:226}
\begin{aligned}
\gpgrade{\BA \BB}{2}
&= \gpgrade{ \Bx\By\Bz \Bu\By\Bz }{2} \\
&= \gpgrade{ \Bx\By\Bz \By\Bz \Bu}{2} \\
&= (\By\Bz)^2\gpgrade{ \Bx \Bu}{2} \\
&= -(\By\Bz)^2\gpgrade{ \Bu \Bx}{2} \\
&= -\gpgrade{ \BB \BA }{2} \\
\end{aligned}
\end{equation}

The grade 2 term anticommutes on interchange:

\begin{equation}\label{eqn:trivector:trivecgpgrade2}
\gpgrade{\BA\BB}{2} = -\gpgrade{\BB\BA}{2}
\end{equation}

\subsection{Grade 0 term}

Any grade 0 terms are due to products of the form \(\BA = k\BB\)

\begin{equation}\label{eqn:trivector:246}
\begin{aligned}
\gpgrade{\BA \BB}{0}
&= \gpgrade{k\BB \BB}{0} \\
&= \gpgrade{\BB k\BB}{0} \\
&= \gpgrade{\BB \BA}{0} \\
\end{aligned}
\end{equation}

The grade 2 term commutes on interchange:

\begin{equation}\label{eqn:trivector:trivecgpgrade0}
\gpgrade{\BA\BB}{0} = \gpgrade{\BB\BA}{0}
\end{equation}

\subsection{combining results}

\begin{equation*}
\BA \BB
=\gpgrade{\BA\BB}{0}
+\gpgrade{\BA\BB}{2}
+\gpgrade{\BA\BB}{4}
+\gpgrade{\BA\BB}{6}
\end{equation*}

\begin{equation}\label{eqn:trivector:266}
\begin{aligned}
\BB\BA
&=\gpgrade{\BB\BA}{0}
+\gpgrade{\BB\BA}{2}
+\gpgrade{\BB\BA}{4}
+\gpgrade{\BB\BA}{6} \\
&=\gpgrade{\BA\BB}{0}
-\gpgrade{\BA\BB}{2}
+\gpgrade{\BA\BB}{4}
-\gpgrade{\BA\BB}{6} \\
\end{aligned}
\end{equation}

These can be combined to express each of the grade terms as subsets
of the symmetric and antisymmetric parts:

\begin{equation}\label{eqn:trivector:286}
\begin{aligned}
\BA \cdot \BB = \gpgrade{\BA\BB}{0} &= \gpgrade{\frac{\BA\BB + \BB\BA}{2}}{0} \\
\gpgrade{\BA\BB}{2} &= \gpgrade{\frac{\BA\BB - \BB\BA}{2}}{2} \\
\gpgrade{\BA\BB}{4} &= \gpgrade{\frac{\BA\BB + \BB\BA}{2}}{4} \\
\BA \wedge \BB = \gpgrade{\BA\BB}{6} &= \gpgrade{\frac{\BA\BB - \BB\BA}{2}}{6} \\
\end{aligned}
\end{equation}

Note that above I have been somewhat loose with the argument above.  A grade three vector
will have the following form:

\begin{equation}\label{eqn:trivector:146}
\sum_{i<j<k} D_{ijk} \Be_{ijk}
\end{equation}

Where \(D_{ijk}\) is the determinant of \(ijk\) components of the vectors being wedged.  Thus the product
of two trivectors will be of the following form:

\begin{equation}\label{eqn:trivector:166}
\sum_{i<j<k} \sum_{i'<j'<k'} D_{ijk} D'_{i'j'k'} (\Be_{ijk} \Be_{i'j'k'})
\end{equation}

It is really each of these \(\Be_{ijk} \Be_{i'j'k'}\) products that have to be considered in the grade
and sign arguments above.  The end result will be the same though... one would just have to present
it a bit more carefully for a true proof.

\subsection{Intersecting trivector cases}

As with the intersecting bivector case, when there is a line of intersection between the two volumes one can
write:

\begin{equation}\label{eqn:trivector:306}
\begin{aligned}
\BA \cdot \BB = \gpgrade{\BA\BB}{0} &= \gpgrade{\frac{\BA\BB + \BB\BA}{2}}{0} \\
\gpgrade{\BA\BB}{2} &= \frac{\BA\BB - \BB\BA}{2} \\
\gpgrade{\BA\BB}{4} &= \gpgrade{\frac{\BA\BB + \BB\BA}{2}}{4} \\
\BA \wedge \BB = \gpgrade{\BA\BB}{6} &= 0 \\
\end{aligned}
\end{equation}

And if these volumes intersect in a plane a further simplification is possible:
\begin{equation}\label{eqn:trivector:326}
\begin{aligned}
\BA \cdot \BB = \gpgrade{\BA\BB}{0} &= \frac{\BA\BB + \BB\BA}{2} \\
\gpgrade{\BA\BB}{2} &= \frac{\BA\BB - \BB\BA}{2} \\
\gpgrade{\BA\BB}{4} &= 0 \\
\BA \wedge \BB = \gpgrade{\BA\BB}{6} &= 0 \\
\end{aligned}
\end{equation}


%
% Copyright � 2012 Peeter Joot.  All Rights Reserved.
% Licenced as described in the file LICENSE under the root directory of this GIT repository.
%

%
%
\chapter{Multivector product grade zero terms}
\label{chap:scalarCommutes}
%\date{Mar 16, 2008.  scalarCommutes.tex}

One can show that the grade zero component of a multivector product
is independent of the order of the terms:

\begin{equation}
\gpgradezero{\BA \BB} = \gpgradezero{\BB \BA}
\end{equation}

Doran/Lasenby has an elegant proof of this, but a dumber proof using an
explicit expansion by basis also works and highlights the similarities
with the standard component definition of the vector dot product.

Writing:

\begin{equation}\label{eqn:scalarCommutes:20}
\BA = \sum_i \gpgrade{\BA}{i}
\end{equation}
\begin{equation}\label{eqn:scalarCommutes:40}
\BB = \sum_i \gpgrade{\BB}{i}
\end{equation}

The product of \(\BA\) and \(\BB\) is:

\begin{equation}\label{eqn:scalarCommutes:180}
\begin{aligned}
\BA \BB
&= \sum_{ij} \gpgrade{\BA}{i} \gpgrade{\BB}{j} \\
&= \sum_{ij} \sum_{k=0}^{\min(i,j)}\gpgrade{\gpgrade{\BA}{i} \gpgrade{\BB}{j}}{2k + \abs{i-j}} \\
\end{aligned}
\end{equation}

\begin{equation}\label{eqn:scalar_commutes:product}
\BA \BB
= \sum_{ij} \sum_{k=0}^{\min(i,j)}\gpgrade{\gpgrade{\BA}{i} \gpgrade{\BB}{j}}{2k + \abs{i-j}}
\end{equation}

To get a better feel for this, consider an example

\begin{equation}\label{eqn:scalarCommutes:60}
\BA = \Be_1 + \Be_2 + \Be_{12} + \Be_{13} + \Be_{34} + \Be_{345}
\end{equation}
\begin{equation}\label{eqn:scalarCommutes:80}
\BB = \Be_2 + \Be_{21} + \Be_{23}
\end{equation}
\begin{equation}\label{eqn:scalarCommutes:100}
\BA \BB = ( \Be_1 + \Be_2 + \Be_{12} + \Be_{13} + \Be_{34} + \Be_{345})(\Be_2 + \Be_{21} + \Be_{23})
\end{equation}

Here are multivectors with grades ranging from zero to three.  This multiplication will include vector/vector, vector/bivector, vector/trivector, bivector/bivector, and bivector/trivector.  Some of these will be grade lowering, some grade preserving and some grade raising.

Only the like grade terms can potentially generate grade zero terms, so the grade zero terms of the product in \eqnref{eqn:scalar_commutes:product} are:

\begin{equation}\label{eqn:scalar_commutes:scalarproduct}
\BA \BB
= \sum_{i=j} \gpgradezero{\gpgrade{\BA}{i} \gpgrade{\BB}{j}}
\end{equation}

Using the example above we have

\begin{equation}\label{eqn:scalarCommutes:120}
\gpgradezero{\BA \BB}
= \gpgradezero{ (\Be_1 + \Be_2)\Be_2 }
+ \gpgradezero{ (\Be_{12} + \Be_{13} + \Be_{34})\Be_{21} }
\end{equation}

In general one can introduce an orthonormal basis
\(\sigma^k = \{\Bsigma_i^k\}_i\) for each of the \(\gpgrade{}{k}\) spaces.
Here orthonormal is with respect to the k-vector dot product

\begin{equation}\label{eqn:scalar_commutes:orthonormal}
\Bsigma_i^k \cdot \Bsigma_j^k = (-1)^{k(k-1)/2}\delta_{ij}
\end{equation}

then one can decompose each of the k-vectors with respect to that
basis:

\begin{equation}\label{eqn:scalarCommutes:140}
\gpgrade{\BA}{k} = \sum_i \left(\gpgrade{\BA}{k} \cdot \Bsigma_i^k\right) \inv{\Bsigma_i^k}
\end{equation}

\begin{equation}\label{eqn:scalarCommutes:160}
\gpgrade{\BB}{k} = \sum_{j} \left(\gpgrade{\BB}{k} \cdot \Bsigma_{j}^k\right) \inv{\Bsigma_{j}^k}
\end{equation}

Thus the scalar part of the product is

\begin{equation}\label{eqn:scalarCommutes:200}
\begin{aligned}
\gpgradezero{\BA \BB}
&= \sum_{k, i, j} \gpgradezero {
\left(\gpgrade{\BA}{k} \cdot \Bsigma_{i}^k\right) \inv{\Bsigma_{i}^k}
\left(\gpgrade{\BB}{k} \cdot \Bsigma_{j}^k\right) \inv{\Bsigma_{j}^k}
} \\
&= \sum_{k, i, j}
\gpgradezero { \Bsigma_{i}^k \Bsigma_{j}^k }
\left(\gpgrade{\BA}{k} \cdot \Bsigma_{i}^k\right)
\left(\gpgrade{\BB}{k} \cdot \Bsigma_{j}^k\right) \\
&= \sum_{k, i, j}
\left(-1\right)^{k\left(k-1\right)/2} \delta_{ij}
\left(\gpgrade{\BA}{k} \cdot \Bsigma_{i}^k\right)
\left(\gpgrade{\BB}{k} \cdot \Bsigma_{j}^k\right)
\end{aligned}
\end{equation}

Thus the complete scalar product can be written

\begin{equation}
\gpgradezero{\BA \BB} = \sum_{k, i}
\left(-1\right)^{k\left(k-1\right)/2}
\left(\gpgrade{\BA}{k} \cdot \Bsigma_{i}^k\right)
\left(\gpgrade{\BB}{k} \cdot \Bsigma_{i}^k\right)
\end{equation}

Note, compared to the vector dot product, the alternation in sign, which is
dependent on the grades involved.

Also note that this now trivially proves that the scalar product is commutative.

Perhaps more importantly we see how similar this generalized dot product is to the
standard component formulation of the vector dot product we are so used to.
At a glance the component-less geometric algebra formulation seems
so much different than the standard vector dot product expressed in terms of components, but
we see here that this is in fact not the case.


%
% Copyright � 2012 Peeter Joot.  All Rights Reserved.
% Licenced as described in the file LICENSE under the root directory of this GIT repository.
%

%
%
\chapter{Blade grade reduction}
\index{grade reduction}
\label{chap:bladegradereduction}
%\date{Mar 25, 2008.  bladegradereduction.tex}

\section{General triple product reduction formula}

Consideration of the reciprocal frame bivector decomposition required the following identity

\begin{equation}
(\BA_a \wedge \BA_b) \cdot \BA_c =
\BA_a \cdot (\BA_b \cdot \BA_c)
\end{equation}

This holds when \(a + b \le c\), and \(a <= b\).  Similar equations for vector wedge blade dot blade reduction can be found in NFCM, but intuition let me to believe the above generalization was valid.

To prove this use the definition of the generalized dot product of two blades:

\begin{equation}\label{eqn:bladegradereduction:282}
\begin{aligned}
(\BA_a \wedge \BA_b) \cdot \BA_c
&= \gpgrade{ (\BA_a \wedge \BA_b) \BA_c }{\abs{c-(a+b)}} \\
\end{aligned}
\end{equation}

The subsequent discussion
is restricted to the \(b \ge a\) case.  Would have to think whether this restriction is required.

\begin{equation}
\label{eqn:bladegradereduction:bladewedge}
\begin{aligned}
\BA_a \wedge \BA_b
&= \BA_a \BA_b - \sum_{i=\abs{b-a},i+=2}^{a+b}\gpgrade{\BA_a\BA_b}{i} \\
&= \BA_a \BA_b - \sum_{k=0}^{a-1}\gpgrade{\BA_a\BA_b}{2k + b - a} \\
\end{aligned}
\end{equation}

Back substitution gives:

\begin{equation}\label{eqn:bladegradereduction:322}
\begin{aligned}
\gpgrade{ (\BA_a \wedge \BA_b) \BA_c }{\abs{c-(a+b)}}
&=
\gpgrade{ \BA_a \BA_b \BA_c }{\abs{c-(a+b)}}
-
\sum_{k=0}^{a-1}
\gpgrade{ \gpgrade{\BA_a\BA_b}{2k + b - a} \BA_c }{c-a-b}
\end{aligned}
\end{equation}

Temporarily writing \(\gpgrade{\BA_a\BA_b}{2k + b - a} = \BC_i\),
\begin{equation}\label{eqn:bladegradereduction:342}
\begin{aligned}
\gpgrade{\BA_a\BA_b}{2k + b - a} \BA_c
&= \sum_{j=c-i,j+=2}^{c+i} \gpgrade{ \BC_i \BA_c }{j} \\
&= \sum_{r=0}^{i} \gpgrade{ \BC_i \BA_c }{c-i+2r} \\
&= \sum_{r=0}^{2k+b-a} \gpgrade{ \BC_i \BA_c }{c-2k-b+a+2r} \\
&= \sum_{r=0}^{2k+b-a} \gpgrade{ \BC_i \BA_c }{c-b+a +2(r-k)} \\
\end{aligned}
\end{equation}

We want the only the following grade terms:

\begin{equation}\label{eqn:bladegradereduction:42}
c-b+a+2(r-k) = c - b - a
\implies
r=k-a
\end{equation}

There are many such \(k,r\) combinations, but we have a \(k \in [0,a-1]\) constraint, which implies \(r \in [-a,-1]\).  This contradicts with \(r\) strictly
positive,
so there are no such grade elements.

This gives an intermediate result, the reduction of the triple product to a direct product, removing the explicit wedge:

\begin{equation}
(\BA_a \wedge \BA_b) \cdot \BA_c =
\gpgrade{\BA_a \BA_b \BA_c}{c-a-b}
\end{equation}

\begin{equation}\label{eqn:bladegradereduction:362}
\begin{aligned}
\gpgrade{\BA_a \BA_b \BA_c}{c-a-b}
&= \gpgrade{\BA_a (\BA_b \BA_c)}{c-a-b} \\
&= \gpgrade{\BA_a \sum_{i}\gpgrade{\BA_b \BA_c}{i}}{c-a-b} \\
&= \gpgrade{\sum_{j}\gpgrade{\BA_a \sum_{i}\gpgrade{\BA_b \BA_c}{i}}{j}}{c-a-b} \\
\end{aligned}
\end{equation}

Explicitly specifying the grades here is omitted for simplicity.  The lowest grade of these is \((c-b)-a\), and all others are higher,
so grade selection excludes them.

By definition

\begin{equation}\label{eqn:bladegradereduction:62}
\gpgrade{\BA_b \BA_c}{c-b} = \BA_b \cdot \BA_c
\end{equation}

so that lowest grade term is thus

\begin{equation}\label{eqn:bladegradereduction:82}
\gpgrade{\BA_a \gpgrade{\BA_b \BA_c}{c-b}}{c-a-b}
= \gpgrade{\BA_a (\BA_b \cdot \BA_c)}{c-a-b}
= \BA_a \cdot (\BA_b \cdot \BA_c)
\end{equation}

This completes the proof.

\section{reduction of grade of dot product of two blades}

The result above can be applied to reducing the dot product of two blades.  For \(k<=s\):

\begin{equation}\label{eqn:bladegradereduction:102}
(\Ba_1 \wedge \Ba_2 \wedge \Ba_3 \cdots \wedge \Ba_k) \cdot (\Bb_1 \wedge \Bb_2 \cdots \wedge \Bb_s)
\end{equation}
\begin{equation}\label{eqn:bladegradereduction:382}
\begin{aligned}
&= (\Ba_1 \wedge (\Ba_2 \wedge \Ba_3 \cdots \wedge \Ba_k)) \cdot (\Bb_1 \wedge \Bb_2 \cdots \wedge \Bb_s) \\
&= (\Ba_1 \cdot ((\Ba_2 \wedge \Ba_3 \cdots \wedge \Ba_k)) \cdot (\Bb_1 \wedge \Bb_2 \cdots \wedge \Bb_s)) \\
&= (\Ba_1 \cdot (\Ba_2 \cdot (\Ba_3 \cdots \wedge \Ba_k)) \cdot (\Bb_1 \wedge \Bb_2 \cdots \wedge \Bb_s)) \\
&= \cdots \\
&= \Ba_1 \cdot (\Ba_2 \cdot (\Ba_3 \cdot (\cdots \cdot (\Ba_k \cdot (\Bb_1 \wedge \Bb_2 \cdots \wedge \Bb_s))))) \\
\end{aligned}
\end{equation}

This can be reduced to a single determinant, as is done in
the Flanders' differential forms book definition of the
\({\bigwedge}^k\) inner product (which is then used to define the Hodge dual).

The first such product is:

\begin{equation}\label{eqn:bladegradereduction:122}
\Ba_k \cdot (\Bb_1 \wedge \Bb_2 \cdots \wedge \Bb_k)
= \sum (-1)^{u-1} (\Ba_k \cdot \Bb_u) \Bb_1 \wedge \cdots \check{\Bb_u} \cdots \wedge \Bb_k
\end{equation}

Next, take dot product with \(\Ba_{k-1}\):

\begin{enumerate}
\item \(k = 2\)

\begin{equation}\label{eqn:bladegradereduction:402}
\begin{aligned}
&\Ba_{k-1} \cdot (\Ba_k \cdot (\Bb_1 \wedge \Bb_2 \cdots \wedge \Bb_k)) \\
&= \sum_{v \ne u} (-1)^{u-1} (\Ba_k \cdot \Bb_u) (\Ba_1 \cdot \Bb_v) \\
&=
 \sum_{u < v} (-1)^{v-1} (\Ba_k \cdot \Bb_v) (\Ba_1 \cdot \Bb_u)
+\sum_{u < v} (-1)^{u-1} (\Ba_k \cdot \Bb_u) (\Ba_1 \cdot \Bb_v) \\
&=
+\sum_{u < v} (\Ba_k \cdot \Bb_u) (\Ba_1 \cdot \Bb_v)
-\sum_{u < v} (\Ba_k \cdot \Bb_v) (\Ba_1 \cdot \Bb_u) \\
&=
+\sum_{u< v} (\Ba_k \cdot \Bb_u) (\Ba_1 \cdot \Bb_v)
- (\Ba_k \cdot \Bb_v) (\Ba_1 \cdot \Bb_u) \\
\end{aligned}
\end{equation}
\begin{equation}\label{eqn:bladegradereduction:k2dot}
-\sum_{u< v}
\begin{vmatrix}
\Ba_{k-1} \cdot \Bb_u & \Ba_{k-1} \cdot \Bb_v \\
\Ba_k \cdot \Bb_u & \Ba_k \cdot \Bb_v \\
\end{vmatrix}
\end{equation}

\item \(k>2\)
\end{enumerate}

\begin{equation}\label{eqn:bladegradereduction:142}
\Ba_{k-1} \cdot (\Ba_k \cdot (\Bb_1 \wedge \Bb_2 \cdots \wedge \Bb_k))
\end{equation}
\begin{equation}\label{eqn:bladegradereduction:422}
\begin{aligned}
&= \sum (-1)^{u-1} (\Ba_k \cdot \Bb_u) \Ba_{k-1} \cdot (\Bb_1 \wedge \cdots \check{\Bb_u} \cdots \wedge \Bb_k) \\
&= \sum_{v<u} (-1)^{u-1} (\Ba_k \cdot \Bb_u) (-1)^{v-1} (\Ba_{k-1} \cdot \Bb_v) (\Bb_1 \wedge \cdots \check{\Bb_v} \cdots \check{\Bb_u} \cdots \wedge \Bb_k) \\
&+ \sum_{v>u} (-1)^{u-1} (\Ba_k \cdot \Bb_u) (-1)^{v} (\Ba_{k-1} \cdot \Bb_v) (\Bb_1 \wedge \cdots \check{\Bb_u} \cdots \check{\Bb_v} \cdots \wedge \Bb_k) \\
\end{aligned}
\end{equation}

Add negation exponents, and use a change of variables for the first sum
\begin{equation}\label{eqn:bladegradereduction:442}
\begin{aligned}
&= \sum_{u<v} (-1)^{v+u} (\Ba_k \cdot \Bb_v) (\Ba_{k-1} \cdot \Bb_u) (\Bb_1 \wedge \cdots \check{\Bb_u} \cdots \check{\Bb_v} \cdots \wedge \Bb_k) \\
&- \sum_{u<v} (-1)^{u+v} (\Ba_k \cdot \Bb_u) (\Ba_{k-1} \cdot \Bb_v) (\Bb_1 \wedge \cdots \check{\Bb_u} \cdots \check{\Bb_v} \cdots \wedge \Bb_k) \\
\end{aligned}
\end{equation}

Merge sums:
\begin{equation}\label{eqn:bladegradereduction:462}
\begin{aligned}
&= \sum_{u<v} (-1)^{u+v}
\left(
(\Ba_k \cdot \Bb_v) (\Ba_{k-1} \cdot \Bb_u)
-(\Ba_k \cdot \Bb_u) (\Ba_{k-1} \cdot \Bb_v)
\right) \\
& \; (\Bb_1 \wedge \cdots \check{\Bb_u} \cdots \check{\Bb_v} \cdots \wedge \Bb_k)
\end{aligned}
\end{equation}

\begin{equation}\label{eqn:bladegradereduction:bivectordotkvector}
\Ba_{k-1} \cdot (\Ba_k \cdot (\Bb_1 \wedge \Bb_2 \cdots \wedge \Bb_k))
=
\end{equation}
\begin{equation*}
\sum_{u<v} (-1)^{u+v}
\begin{vmatrix}
\Ba_{k-1} \cdot \Bb_u & \Ba_{k-1} \cdot \Bb_v \\
\Ba_k \cdot \Bb_u & \Ba_k \cdot \Bb_v \\
\end{vmatrix}
(\Bb_1 \wedge \cdots \check{\Bb_u} \cdots \check{\Bb_v} \cdots \wedge \Bb_k) \\
\end{equation*}

Note that special casing \(k=2\) does not seem to be required because in that
case \(-1^{u+v} = -1^{1+2}=-1\), so this is identical to \eqnref{eqn:bladegradereduction:k2dot} after all.

\subsection{Pause to reflect}

Although my initial aim was to show that \(\BA_k \cdot \BB_k\) could be
expressed as a determinant as in the differential forms book (different
sign though), and to determine exactly what that determinant is, there
are some useful identities that fall out of this even just for this
bivector kvector dot product expansion.

Here is a summary of some of the things figured out so far

\begin{enumerate}
\item Dot product of grade one blades.

Here we have a result that can be expressed as a one by one determinant.  Worth mentioning to explicitly show the sign.

\begin{equation}\label{eqn:bladegradereduction:dotoneblades}
\Ba \cdot \Bb = \det[\Ba \cdot \Bb]
\end{equation}

%(Used \(\det{}\) here instead of \(\Det{}\) to avoid confusing with absolute value).
\item Dot product of grade two blades.

\begin{equation}\label{eqn:bladegradereduction:k2k2dot}
(\Ba_1 \wedge \Ba_2) \cdot (\Bb_1 \wedge \Bb_2)
=
-
\begin{vmatrix}
\Ba_1 \cdot \Bb_1 & \Ba_1 \cdot \Bb_2 \\
\Ba_2 \cdot \Bb_1 & \Ba_2 \cdot \Bb_2 \\
\end{vmatrix}
=
-\det[\Ba_i \cdot \Bb_j]
\end{equation}

\item Dot product of grade two blade with grade \(>2\) blade.

\begin{equation*}
(\Ba_{1} \wedge \Ba_2) \cdot (\Bb_1 \wedge \Bb_2 \cdots \wedge \Bb_k)
\end{equation*}
\begin{equation}\label{eqn:bladegradereduction:bivectordot}
=
\sum_{u<v} (-1)^{u+v-1}
(\Ba_1 \wedge \Ba_2) \cdot (\Bb_u \wedge \Bb_v)
(\Bb_1 \wedge \cdots \check{\Bb_u} \cdots \check{\Bb_v} \cdots \wedge \Bb_k)
\end{equation}
\end{enumerate}

Observe how similar this is to the vector blade dot product expansion:

\begin{equation}\label{eqn:bladegradereduction:vectordot}
\Ba \cdot (\Bb_1 \wedge \Bb_2 \cdots \wedge \Bb_k)
=
\sum (-1)^{i-1}
(\Ba \cdot \Bb_i) (\Bb_1 \wedge \cdots \check{\Bb_i} \cdots \wedge \Bb_k)
\end{equation}

\subsubsection{Expand it for \texorpdfstring{\(k=3\)}{k equal 3}}

Explicit expansion of \eqnref{eqn:bladegradereduction:bivectordot} for the \(k=3\) case, is also helpful to get a feel for
the equation:

\begin{equation}\label{eqn:bladegradereduction:482}
\begin{aligned}
(\Ba_{1} \wedge \Ba_2) \cdot (\Bb_1 \wedge \Bb_2 \wedge \Bb_3)
&=
(\Ba_1 \wedge \Ba_2) \cdot (\Bb_1 \wedge \Bb_2) \Bb_3 \\
&+(\Ba_1 \wedge \Ba_2) \cdot (\Bb_3 \wedge \Bb_1) \Bb_2 \\
&+(\Ba_1 \wedge \Ba_2) \cdot (\Bb_2 \wedge \Bb_3) \Bb_1
\end{aligned}
\end{equation}

Observe the cross product like alternation in sign and indices.
This suggests that a more natural way to express the sign coefficient may be via a \(\Sgn(\pi)\) expression for the sign of the
permutation of indices.

\section{trivector dot product}

With the result of \eqnref{eqn:bladegradereduction:bivectordot}, or the earlier equivalent determinant expression in equation
\eqnref{eqn:bladegradereduction:bivectordotkvector} we are now in a position to evaluate the dot product of a trivector and a greater or equal grade blade.

\begin{equation*}
\Ba_1 \cdot ((\Ba_{2} \wedge \Ba_3) \cdot (\Bb_1 \wedge \Bb_2 \cdots \wedge \Bb_k))
\end{equation*}
\begin{equation}\label{eqn:bladegradereduction:502}
\begin{aligned}
&=
\sum_{u<v} (-1)^{u+v-1}
(\Ba_2 \wedge \Ba_3) \cdot (\Bb_u \wedge \Bb_v)
\Ba_1 \cdot (\Bb_1 \wedge \cdots \check{\Bb_u} \cdots \check{\Bb_v} \cdots \wedge \Bb_k)  \\
&=
\sum_{w<u<v} (-1)^{u+v+w}
(\Ba_2 \wedge \Ba_3) \cdot (\Bb_u \wedge \Bb_v)
(\Ba_1 \cdot \Bb_w) (\Bb_1 \wedge \cdots \check{\Bb_w} \cdots \check{\Bb_u} \cdots \check{\Bb_v} \cdots \wedge \Bb_k)  \\
&+\sum_{u<w<v} (-1)^{u+v+w-1}
(\Ba_2 \wedge \Ba_3) \cdot (\Bb_u \wedge \Bb_v)
(\Ba_1 \cdot \Bb_w) (\Bb_1 \wedge \cdots \check \Bb_u \cdots \check{\Bb_w} \cdots \check{\Bb_v} \cdots \wedge \Bb_k)  \\
&+\sum_{u<v<w} (-1)^{u+v+w}
(\Ba_2 \wedge \Ba_3) \cdot (\Bb_u \wedge \Bb_v)
(\Ba_1 \cdot \Bb_w) (\Bb_1 \wedge \cdots \check \Bb_u \cdots \check{\Bb_v} \cdots \check{\Bb_w} \cdots \wedge \Bb_k)  \\
\end{aligned}
\end{equation}

Change the indices of summation and grouping like terms we have:
\begin{equation}\label{eqn:bladegradereduction:522}
\begin{aligned}
\sum_{u<v<w} (-1)^{u+v+w}
(
&(\Ba_2 \wedge \Ba_3) \cdot (\Bb_v \wedge \Bb_w) (\Ba_1 \cdot \Bb_u)  \\
&-(\Ba_2 \wedge \Ba_3) \cdot (\Bb_u \wedge \Bb_w) (\Ba_1 \cdot \Bb_v)  \\
&+(\Ba_2 \wedge \Ba_3) \cdot (\Bb_u \wedge \Bb_v) (\Ba_1 \cdot \Bb_w)  \\
)
(\Bb_1 \wedge \cdots \check \Bb_u \cdots \check{\Bb_v} \cdots \check{\Bb_w} \cdots \wedge \Bb_k)  \\
\end{aligned}
\end{equation}

Now, each of the embedded dot products were in fact determinants:
\begin{equation}\label{eqn:bladegradereduction:162}
(\Ba_2 \wedge \Ba_3) \cdot (\Bb_x \wedge \Bb_y)
=
-
\begin{vmatrix}
\Ba_2 \cdot \Bb_x & \Ba_2 \cdot \Bb_y \\
\Ba_3 \cdot \Bb_x & \Ba_3 \cdot \Bb_y \\
\end{vmatrix}
\end{equation}

Thus, we can expand these triple dot products like so (factor of \(-1\) omitted):
\begin{equation}\label{eqn:bladegradereduction:542}
\begin{aligned}
&(\Ba_2 \wedge \Ba_3) \cdot (\Bb_v \wedge \Bb_w) (\Ba_1 \cdot \Bb_u) \\
&-(\Ba_2 \wedge \Ba_3) \cdot (\Bb_u \wedge \Bb_w) (\Ba_1 \cdot \Bb_v) \\
&+(\Ba_2 \wedge \Ba_3) \cdot (\Bb_u \wedge \Bb_v) (\Ba_1 \cdot \Bb_w)  \\
&=
(\Ba_1 \cdot \Bb_u)
\begin{vmatrix}
\Ba_2 \cdot \Bb_v & \Ba_2 \cdot \Bb_w \\
\Ba_3 \cdot \Bb_v & \Ba_3 \cdot \Bb_w \\
\end{vmatrix} \\
&-
(\Ba_1 \cdot \Bb_v)
\begin{vmatrix}
\Ba_2 \cdot \Bb_u & \Ba_2 \cdot \Bb_w \\
\Ba_3 \cdot \Bb_u & \Ba_3 \cdot \Bb_w \\
\end{vmatrix} \\
&+
(\Ba_1 \cdot \Bb_w)
\begin{vmatrix}
\Ba_2 \cdot \Bb_u & \Ba_2 \cdot \Bb_v \\
\Ba_3 \cdot \Bb_u & \Ba_3 \cdot \Bb_v \\
\end{vmatrix} \\
%&=
%\begin{vmatrix}
%\Ba_1 \cdot \Bb_u & 0 & 0 \\
%0 & \Ba_2 \cdot \Bb_v & \Ba_2 \cdot \Bb_w \\
%0 & \Ba_3 \cdot \Bb_v & \Ba_3 \cdot \Bb_w \\
%\end{vmatrix} \\
%&+
%\begin{vmatrix}
%0 & \Ba_1 \cdot \Bb_v & 0 \\
%\Ba_2 \cdot \Bb_u & 0 & \Ba_2 \cdot \Bb_w \\
%\Ba_3 \cdot \Bb_u & 0 & \Ba_3 \cdot \Bb_w \\
%\end{vmatrix} \\
%&+
%\begin{vmatrix}
%0 & 0 & \Ba_1 \cdot \Bb_w \\
%\Ba_2 \cdot \Bb_u & \Ba_2 \cdot \Bb_v & 0 \\
%\Ba_3 \cdot \Bb_u & \Ba_3 \cdot \Bb_v & 0 \\
%\end{vmatrix} \\
&=
\begin{vmatrix}
\Ba_1 \cdot \Bb_u & \Ba_1 \cdot \Bb_v & \Ba_1 \cdot \Bb_w \\
\Ba_2 \cdot \Bb_u & \Ba_2 \cdot \Bb_v & \Ba_2 \cdot \Bb_w \\
\Ba_3 \cdot \Bb_u & \Ba_3 \cdot \Bb_v & \Ba_3 \cdot \Bb_w \\
\end{vmatrix} \\
\end{aligned}
\end{equation}

Final back substitution gives:

\begin{equation*}
(\Ba_1 \wedge \Ba_{2} \wedge \Ba_3) \cdot (\Bb_1 \wedge \Bb_2 \cdots \wedge \Bb_k)
\end{equation*}
\begin{equation}\label{eqn:bladegradereduction:trivectordotdet}
=
\sum_{u<v<w} (-1)^{u+v+w-1}
\begin{vmatrix}
\Ba_1 \cdot \Bb_u & \Ba_1 \cdot \Bb_v & \Ba_1 \cdot \Bb_w \\
\Ba_2 \cdot \Bb_u & \Ba_2 \cdot \Bb_v & \Ba_2 \cdot \Bb_w \\
\Ba_3 \cdot \Bb_u & \Ba_3 \cdot \Bb_v & \Ba_3 \cdot \Bb_w \\
\end{vmatrix}
(\Bb_1 \wedge \cdots \check \Bb_u \cdots \check{\Bb_v} \cdots \check{\Bb_w} \cdots \wedge \Bb_k)  \\
\end{equation}

In particular for \(k=3\) we have
\begin{equation*}
(\Ba_1 \wedge \Ba_{2} \wedge \Ba_3) \cdot (\Bb_1 \wedge \Bb_2 \wedge \Bb_3)
\end{equation*}
\begin{equation}\label{eqn:bladegradereduction:trivectordotdettri}
=
-\begin{vmatrix}
\Ba_1 \cdot \Bb_1 & \Ba_1 \cdot \Bb_2 & \Ba_1 \cdot \Bb_3 \\
\Ba_2 \cdot \Bb_1 & \Ba_2 \cdot \Bb_2 & \Ba_2 \cdot \Bb_3 \\
\Ba_3 \cdot \Bb_1 & \Ba_3 \cdot \Bb_2 & \Ba_3 \cdot \Bb_3 \\
\end{vmatrix}
=
-\det[\Ba_i \cdot \Bb_j]
\end{equation}

This can be substituted back into \eqnref{eqn:bladegradereduction:trivectordotdet} to put it in a non determinant form.

\begin{equation*}
(\Ba_1 \wedge \Ba_{2} \wedge \Ba_3) \cdot (\Bb_1 \wedge \Bb_2 \cdots \wedge \Bb_k)
\end{equation*}
\begin{equation}\label{eqn:bladegradereduction:trivectordotnondet}
=
\sum_{u<v<w} (-1)^{u+v+w}
(\Ba_1 \wedge \Ba_{2} \wedge \Ba_3) \cdot (\Bb_u \wedge \Bb_v \wedge \Bb_w)
(\Bb_1 \wedge \cdots \check \Bb_u \cdots \check{\Bb_v} \cdots \check{\Bb_w} \cdots \wedge \Bb_k)  \\
\end{equation}

\section{Induction on the result}

It is pretty clear that recursively performing these calculations will yield similar determinant and inner dot product reduction
results.

\subsection{dot product of like grade terms as determinant}

Let us consider the equal grade case first, summarizing the results so far

\begin{equation}\label{eqn:bladegradereduction:562}
\begin{aligned}
\Ba \cdot \Bb &= \det[\Ba \cdot \Bb] \\
(\Ba_1 \wedge \Ba_2) \cdot (\Bb_1 \wedge \Bb_2) &= -\det[\Ba_i \cdot \Bb_j] \\
(\Ba_1 \wedge \Ba_2 \wedge \Ba_3) \cdot (\Bb_1 \wedge \Bb_2 \wedge \Bb_3) &= -\det[\Ba_i \cdot \Bb_j] \\
\end{aligned}
\end{equation}

What will the sign be for the higher grade equivalents?  It has the appearance of being related to the sign associated with blade
reversion.  To verify this calculate the dot product of a blade formed from a set of perpendicular unit vectors with itself.

\begin{equation}\label{eqn:bladegradereduction:582}
\begin{aligned}
&(\Be_1 \wedge \cdots \wedge \Be_k) \cdot (\Be_1 \wedge \Be_2 \wedge \cdots \wedge \Be_k) \\
&= (-1)^{k(k-1)/2}(\Be_1 \wedge \cdots \wedge \Be_k) \cdot (\Be_k \wedge \cdots \wedge \Be_2 \wedge \Be_1) \\
&= (-1)^{k(k-1)/2}\Be_1 \cdot (\Be_2 \cdots (\Be_k \cdot (\Be_k \wedge \cdots \wedge \Be_2 \wedge \Be_1))) \\
&= (-1)^{k(k-1)/2}\Be_1 \cdot (\Be_2 \cdots (\Be_{k-1} \cdot (\Be_{k-1} \wedge \cdots \wedge \Be_2 \wedge \Be_1))) \\
&= \cdots \\
&= (-1)^{k(k-1)/2}
\end{aligned}
\end{equation}

This fixes the sign, and provides the induction hypothesis for the general case:

\begin{equation}\label{eqn:bladegradereduction:bladedothyp}
(\Ba_1 \wedge \cdots \wedge \Ba_k) \cdot (\Bb_1 \wedge \Bb_2 \wedge \cdots \wedge \Bb_k) = (-1)^{k(k-1)/2}\det[\Ba_i \cdot \Bb_j]
\end{equation}

Alternately, one can remove the sign change coefficient with reversion of one of the blades:

\begin{equation}\label{eqn:bladegradereduction:bladedothyprev}
(\Ba_1 \wedge \cdots \wedge \Ba_k) \cdot (\Bb_k \wedge \Bb_{k-1} \wedge \cdots \wedge \Bb_1) = \det[\Ba_i \cdot \Bb_j]
\end{equation}

\subsection{Unlike grades}

Let us summarize the results for unlike grades at the same time reformulating the previous results in terms of index
permutation, also writing for brevity \(\BA_s = \Ba_1 \wedge \cdots \wedge \Ba_s\), and \(\BB_k = \Bb_1 \wedge \cdots \wedge \Bb_k\):

\begin{equation}\label{eqn:bladegradereduction:182}
\BA_1 \cdot \BB_k =
\sum_i \Sgn(\pi(i,1,2,\cdots\check{i}\cdots,k)) (\BA_1 \cdot \Bb_i) (\Bb_1 \wedge \cdots \check{\Bb_i} \cdots \wedge \Bb_k)
\end{equation}

\begin{equation}\label{eqn:bladegradereduction:202}
\BA_2 \cdot \BB_k =
\sum_{i_1<i_2} \Sgn(\pi(i_1,i_2,1,2,\cdots\check{i_1}\cdots\check{i_2}\cdots,k))
\end{equation}
\begin{equation}\label{eqn:bladegradereduction:222}
   \BA_2 \cdot (\Bb_{i_1} \wedge \Bb_{i_2})
   (\Bb_1 \wedge \cdots \check{\Bb_{i_1}} \cdots \check{\Bb_{i_2}} \cdots \wedge \Bb_k)
\end{equation}

\begin{equation}\label{eqn:bladegradereduction:242}
\BA_3 \cdot \BB_k =
\sum_{i_1<i_2<i_3} \Sgn(\pi(i_1,i_2,i_3,1,2,\cdots\check{i_1}\cdots\check{i_2}\cdots\check{i_3}\cdots,k))
\end{equation}
\begin{equation}\label{eqn:bladegradereduction:262}
\BA_3 \cdot (\Bb_{i_1} \wedge \Bb_{i_2} \wedge \Bb_{i_3})
(\Bb_1 \wedge \cdots \check{\Bb_{i_1}} \cdots \check{\Bb_{i_2}} \cdots \check{\Bb_{i_3}} \cdots \wedge \Bb_k)
\end{equation}

We see that the dot product consumes any of the excess sign variation not described by the sign of the permutation of indices.

The induction hypothesis is basically described above (change \(3\) to \(s\), and add extra dots):

\begin{equation*}
\BA_s \cdot \BB_k =
\sum_{i_1<i_2\cdots<i_s} \Sgn(\pi(i_1,i_2\cdots,i_s,1,2,\cdots\check{i_1}\cdots\check{i_2}\cdots\check{i_s}\cdots,k))
\end{equation*}
\begin{equation}\label{eqn:bladegradereduction:inductionbigdotblade}
\BA_s \cdot (\Bb_{i_1} \wedge \Bb_{i_2} \cdots \wedge \Bb_{i_s})
(\Bb_1 \wedge \cdots \check{\Bb_{i_1}} \cdots \check{\Bb_{i_2}} \cdots \check{\Bb_{i_s}} \cdots \wedge \Bb_k)
\end{equation}

\subsection{Perform the induction}

In a sense this has already been done.  The steps will be pretty much the same as the logic that produced the bivector and trivector
results.  Thinking about typing this up in latex is not fun, so this will be left for a paper proof.

\documentclass{article}      

\usepackage{amsmath}
\usepackage{mathpazo}

%
% shorthand for bold symbols, convenient for vectors and matrices
%
\newcommand{\Ba}[0]{\mathbf{a}}
\newcommand{\Bb}[0]{\mathbf{b}}
\newcommand{\Bc}[0]{\mathbf{c}}
\newcommand{\Bd}[0]{\mathbf{d}}
\newcommand{\Be}[0]{\mathbf{e}}
\newcommand{\Bf}[0]{\mathbf{f}}
\newcommand{\Bg}[0]{\mathbf{g}}
\newcommand{\Bh}[0]{\mathbf{h}}
\newcommand{\Bi}[0]{\mathbf{i}}
\newcommand{\Bj}[0]{\mathbf{j}}
\newcommand{\Bk}[0]{\mathbf{k}}
\newcommand{\Bl}[0]{\mathbf{l}}
\newcommand{\Bm}[0]{\mathbf{m}}
\newcommand{\Bn}[0]{\mathbf{n}}
\newcommand{\Bo}[0]{\mathbf{o}}
\newcommand{\Bp}[0]{\mathbf{p}}
\newcommand{\Bq}[0]{\mathbf{q}}
\newcommand{\Br}[0]{\mathbf{r}}
\newcommand{\Bs}[0]{\mathbf{s}}
\newcommand{\Bt}[0]{\mathbf{t}}
\newcommand{\Bu}[0]{\mathbf{u}}
\newcommand{\Bv}[0]{\mathbf{v}}
\newcommand{\Bw}[0]{\mathbf{w}}
\newcommand{\Bx}[0]{\mathbf{x}}
\newcommand{\By}[0]{\mathbf{y}}
\newcommand{\Bz}[0]{\mathbf{z}}
\newcommand{\BA}[0]{\mathbf{A}}
\newcommand{\BB}[0]{\mathbf{B}}
\newcommand{\BC}[0]{\mathbf{C}}
\newcommand{\BD}[0]{\mathbf{D}}
\newcommand{\BE}[0]{\mathbf{E}}
\newcommand{\BF}[0]{\mathbf{F}}
\newcommand{\BG}[0]{\mathbf{G}}
\newcommand{\BH}[0]{\mathbf{H}}
\newcommand{\BI}[0]{\mathbf{I}}
\newcommand{\BJ}[0]{\mathbf{J}}
\newcommand{\BK}[0]{\mathbf{K}}
\newcommand{\BL}[0]{\mathbf{L}}
\newcommand{\BM}[0]{\mathbf{M}}
\newcommand{\BN}[0]{\mathbf{N}}
\newcommand{\BO}[0]{\mathbf{O}}
\newcommand{\BP}[0]{\mathbf{P}}
\newcommand{\BQ}[0]{\mathbf{Q}}
\newcommand{\BR}[0]{\mathbf{R}}
\newcommand{\BS}[0]{\mathbf{S}}
\newcommand{\BT}[0]{\mathbf{T}}
\newcommand{\BU}[0]{\mathbf{U}}
\newcommand{\BV}[0]{\mathbf{V}}
\newcommand{\BW}[0]{\mathbf{W}}
\newcommand{\BX}[0]{\mathbf{X}}
\newcommand{\BY}[0]{\mathbf{Y}}
\newcommand{\BZ}[0]{\mathbf{Z}}

\newcommand{\Bzero}[0]{\mathbf{0}}
\newcommand{\Btheta}[0]{\boldsymbol{\theta}}
\newcommand{\Btau}[0]{\boldsymbol{\tau}}
\newcommand{\Bomega}[0]{\boldsymbol{\omega}}

%
% shorthand for unit vectors
%
\newcommand{\acap}[0]{\hat{\Ba}}
\newcommand{\bcap}[0]{\hat{\Bb}}
\newcommand{\ccap}[0]{\hat{\Bc}}
\newcommand{\dcap}[0]{\hat{\Bd}}
\newcommand{\ecap}[0]{\hat{\Be}}
\newcommand{\fcap}[0]{\hat{\Bf}}
\newcommand{\gcap}[0]{\hat{\Bg}}
\newcommand{\hcap}[0]{\hat{\Bh}}
\newcommand{\icap}[0]{\hat{\Bi}}
\newcommand{\jcap}[0]{\hat{\Bj}}
\newcommand{\kcap}[0]{\hat{\Bk}}
\newcommand{\lcap}[0]{\hat{\Bl}}
\newcommand{\mcap}[0]{\hat{\Bm}}
\newcommand{\ncap}[0]{\hat{\Bn}}
\newcommand{\ocap}[0]{\hat{\Bo}}
\newcommand{\pcap}[0]{\hat{\Bp}}
\newcommand{\qcap}[0]{\hat{\Bq}}
\newcommand{\rcap}[0]{\hat{\Br}}
\newcommand{\scap}[0]{\hat{\Bs}}
\newcommand{\tcap}[0]{\hat{\Bt}}
\newcommand{\ucap}[0]{\hat{\Bu}}
\newcommand{\vcap}[0]{\hat{\Bv}}
\newcommand{\wcap}[0]{\hat{\Bw}}
\newcommand{\xcap}[0]{\hat{\Bx}}
\newcommand{\ycap}[0]{\hat{\By}}
\newcommand{\zcap}[0]{\hat{\Bz}}
\newcommand{\thetacap}[0]{\hat{\Btheta}}

%
% to write R^n and C^n in a distinguishable fashion.  Perhaps change this
% to the double lined characters upon figuring out how to do so.
%
\newcommand{\C}[1]{$\mathbb{C}^{#1}$}
\newcommand{\R}[1]{$\mathbb{R}^{#1}$}

%
% various generally useful helpers
%

% derivative of #1 wrt. #2:
\newcommand{\D}[2] {\frac {d#2} {d#1}}

\newcommand{\inv}[1]{\frac{1}{#1}}
\newcommand{\cross}[0]{\times}

\newcommand{\abs}[1]{\lvert{#1}\rvert}
\newcommand{\norm}[1]{\lVert{#1}\rVert}
\newcommand{\innerprod}[2]{\langle{#1}, {#2}\rangle}
\newcommand{\dotprod}[2]{{#1} \cdot {#2}}
\newcommand{\bdotprod}[2]{\left({#1} \cdot {#2}\right)}
\newcommand{\crossprod}[2]{{#1} \cross {#2}}
\newcommand{\tripleprod}[3]{\dotprod{\left(\crossprod{#1}{#2}\right)}{#3}}

\DeclareMathOperator{\Proj}{Proj}
\DeclareMathOperator{\Span}{span}
\DeclareMathOperator{\Sgn}{sgn}
\DeclareMathOperator{\Area}{Area}
\DeclareMathOperator{\Volume}{Volume}

%
% A few miscellaneous things specific to this document
%
\newcommand{\crossop}[1]{\crossprod{#1}{}}

% R2 vector.
\newcommand{\VectorTwo}[2]{
\begin{bmatrix}
 {#1} \\
 {#2}
\end{bmatrix}
}

\newcommand{\VectorN}[1]{
\begin{bmatrix}
{#1}_1 \\
{#1}_2 \\
\vdots \\
{#1}_N \\
\end{bmatrix}
}

\newcommand{\DETuvij}[4]{
\begin{vmatrix}
 {#1}_{#3} & {#1}_{#4} \\
 {#2}_{#3} & {#2}_{#4}
\end{vmatrix}
}

\newcommand{\DETuvwijk}[6]{
\begin{vmatrix}
 {#1}_{#4} & {#1}_{#5} & {#1}_{#6} \\
 {#2}_{#4} & {#2}_{#5} & {#2}_{#6} \\
 {#3}_{#4} & {#3}_{#5} & {#3}_{#6}
\end{vmatrix}
}

\newcommand{\DETuvwxijkl}[8]{
\begin{vmatrix}
 {#1}_{#5} & {#1}_{#6} & {#1}_{#7} & {#1}_{#8} \\
 {#2}_{#5} & {#2}_{#6} & {#2}_{#7} & {#2}_{#8} \\
 {#3}_{#5} & {#3}_{#6} & {#3}_{#7} & {#3}_{#8} \\
 {#4}_{#5} & {#4}_{#6} & {#4}_{#7} & {#4}_{#8} \\
\end{vmatrix}
}

%\newcommand{\DETuvwxyijklm}[10]{
%\begin{vmatrix}
% {#1}_{#6} & {#1}_{#7} & {#1}_{#8} & {#1}_{#9} & {#1}_{#10} \\
% {#2}_{#6} & {#2}_{#7} & {#2}_{#8} & {#2}_{#9} & {#2}_{#10} \\
% {#3}_{#6} & {#3}_{#7} & {#3}_{#8} & {#3}_{#9} & {#3}_{#10} \\
% {#4}_{#6} & {#4}_{#7} & {#4}_{#8} & {#4}_{#9} & {#4}_{#10} \\
% {#5}_{#6} & {#5}_{#7} & {#5}_{#8} & {#5}_{#9} & {#5}_{#10}
%\end{vmatrix}
%}

% R3 vector.
\newcommand{\VectorThree}[3]{
\begin{bmatrix}
 {#1} \\
 {#2} \\
 {#3}
\end{bmatrix}
}



                             % The preamble begins here.
\title{More details on NFCM plane formulation} % Declares the document's title.
\author{Peeter Joot}         % Declares the author's name.
%\date{}        % Deleting this command produces today's date.

\begin{document}             % End of preamble and beginning of text.

%\maketitle{}

\section{Wedge product formula for a plane.}

The equation of the plane with bivector $\BU$ through point $\Ba$ is given
by

\[
(\Bx - \Ba) \wedge \BU = 0
\]

or

\[
\Bx \wedge \BU = \Ba \wedge \BU = \BT
\]

\subsection{ Examining this equation in more details. }

Without any loss of generality one can express this plane equation
in terms of a unit bivector $\Bi$

\[
\Bx \wedge \Bi = \Ba \wedge \Bi
\]

As with the line equation, to express this in the ``standard'' parametric
form, right multiplication with $1/\Bi$ is required.

\[
(\Bx \wedge \Bi)\frac{1}{\Bi} = (\Ba \wedge \Bi)\frac{1}{\Bi}
\]

We have a trivector bivector product here, which in general has a vector,
trivector, and 5-vector component.  Since $\Bi \wedge \Bi = 0$, the
5-vector component is zero:

\[
\Bx \wedge \Bi \wedge -\Bi = 0
\]

and intuition says that the trivector component will also be zero.  However,
as well as providing verification of this, expansion of this product will also
demonstrate how to find the projective and rejective components of a vector
with respect to a plane (ie: components in and out of the plane).

\subsection{Rejection from a plane product expansion.}

Here's an explicit expansion of the rejective term above

\begin{align*}
(\Bx \wedge \Bi)\frac{1}{\Bi} 
&= -(\Bx \wedge \Bi){\Bi} \\ 
&= -\frac{1}{2}(\Bx\Bi + \Bi\Bx){\Bi} \\ 
&= \frac{1}{2}(\Bx - \Bi\Bx\Bi) \\ 
&= \frac{1}{2}(\Bx - (\Bx \Bi + 2 \Bi \cdot \Bx)\Bi) \\ 
&= \Bx - (\Bi \cdot \Bx)\Bi \\ 
\end{align*}

In this last term the quantity $\Bi \cdot \Bx$ is a vector in the plane.
This can be demonstrated by writing $\Bi$ in terms of a pair of orthonormal
vectors $\Bi = \ucap\vcap = \ucap \wedge \vcap$.

\begin{align*}
\Bi \cdot \Bx &= (\ucap \wedge \vcap) \cdot \Bx \\
              &= \ucap (\vcap \cdot \Bx) - \vcap (\ucap \cdot \Bx) \\
\end{align*}

Thus, $(\Bi \cdot \Bx) \wedge \Bi = 0$, 
and $(\Bi \cdot \Bx) \Bi = (\Bi \cdot \Bx) \cdot \Bi$.  Inserting this above
we have the end result

\begin{align*}
(\Bx \wedge \Bi)\frac{1}{\Bi} 
&= \Bx - (\Bi \cdot \Bx) \cdot \Bi \\ 
&= \Ba - (\Bi \cdot \Ba) \cdot \Bi \\ 
\end{align*}

Or
\begin{align*}
\Bx  - \Ba 
&= (\Bi \cdot (\Bx - \Ba)) \cdot \Bi \\ 
\end{align*}

This is actually the standard parametric equation of a plane, but expressed
in terms of a unit bivector that describes the plane instead of in terms
of a pair of vectors in the plane.

To demonstrate this expansion of the right hand side is required

\begin{align*}
(\Bi \cdot \Bx) \cdot \Bi
&= (\ucap (\vcap \cdot \Bx) - \vcap (\ucap \cdot \Bx)) \ucap \vcap \\
&= \vcap (\vcap \cdot \Bx) + \ucap (\ucap \cdot \Bx) \\
\end{align*}

Substituting this back yields:

\begin{align*}
\Bx 
&= \Ba + \ucap (\ucap \cdot (\Bx - \Ba)) + \vcap (\vcap \cdot (\Bx - \Ba)) \\
&= \Ba + s \ucap + t \vcap
\end{align*}

In words this says that the plane is specified by a point in the plane,
and the span
of a pair of orthonormal vectors directed in that plane.

This (but perhaps without neccessariliy using orthornomal direction vectors)
is often how the plane is defined to start with.

It isn't neccessarily obvious that the bivector wedge product formula for
a plane that we started with:

\[
\Bx \wedge \BU = \Ba \wedge \BU
\]

can also be used to express this parametric representation.

\subsection{ Orthonormal decomposition of a vector with respect to a plane. }

With the expansion above we have a separation of a vector into two
components, and these can be demonstrated to be the components that are
directed entirely within and out of the plane.

Rearranging terms from above we have:

\begin{align*}
\Bx 
&= 
(\Bx \cdot \Bi) \cdot \frac{1}{\Bi} + (\Bx \wedge \Bi) \cdot \frac{1}{\Bi} \\
&= 
(\Bx \cdot \Bi) \frac{1}{\Bi} + (\Bx \wedge \Bi) \frac{1}{\Bi} \\
\end{align*}

% write x = x_perp + x_parallel to show that this is a ortho decomp.
% can then write formula for directrix of plane.

\subsection{ Alternate derivation of orthonormal planar decomposition }

This could alternately be derived by expanding the vector unit bivector
product directly

\begin{align*}
\Bx \Bi \frac{1}{\Bi} 
&= ( \Bx \cdot \Bi + \Bx \wedge \Bi ) \frac{1}{\Bi} \\
&= 
- {(\Bx \cdot \Bi) \cdot \Bi} - {(\Bx \cdot \Bi) \wedge \Bi} - {(\Bx \wedge \Bi) \Bi} \\
&= 
- {(\Bx \cdot \Bi) \cdot \Bi} - {(\Bx \wedge \Bi) \cdot \Bi } - {<(\Bx \wedge \Bi) \Bi>_3} - {(\Bx \wedge \Bi) \wedge \Bi} \\
&= 
{(\Bx \cdot \Bi) \cdot \frac{1}{\Bi}} + {(\Bx \wedge \Bi) \cdot \frac{1}{\Bi}} - {<(\Bx \wedge \Bi) \Bi>_3} \\
\end{align*}

Since the LHS of this equation is the vector $\Bx$, the right hand side must
also be a vector, which demonstrates that the term

\[
<(\Bx \wedge \Bi) \Bi>_3 = 0
\]

So, one has

\begin{align*}
\Bx 
&=
{(\Bx \cdot \Bi) \cdot \frac{1}{\Bi}} + {(\Bx \wedge \Bi) \cdot \frac{1}{\Bi}} \\
&=
{(\Bx \cdot \Bi) \frac{1}{\Bi}} + {(\Bx \wedge \Bi) \frac{1}{\Bi}} \\
\end{align*}


\end{document}

%
% Copyright � 2012 Peeter Joot.  All Rights Reserved.
% Licenced as described in the file LICENSE under the root directory of this GIT repository.
%

%
%
\chapter{Quaternions}
\index{quaternion}
\label{chap:quaternion}
%\date{Feb 2, 2008.  quaternion.tex}

Like complex numbers, quaternions may be written as a multivector with scalar and bivector components (a 0,2-multivector).

\begin{equation}\label{eqn:quaternion:20}
q = \alpha + \mathbf{B}
\end{equation}

Where the complex number has one bivector component, and the quaternions have three.

One can describe quaternions as 0,2-multivectors where the basis for the bivector part is left handed.  There is not really anything special about quaternion multiplication, or complex number multiplication, for that matter.  Both are just a specific examples of a 0,2-multivector multiplication.  Other quaternion operations can also be found to have natural multivector equivalents.  The most important of which is likely the quaternion conjugate, since it implies the norm and the inverse.  As a multivector, like complex numbers, the conjugate operation is reversal:

\begin{equation}\label{eqn:quaternion:40}
\overline{q} = q^\dagger = \alpha - \mathbf{B}
\end{equation}

Thus \(\abs{q}^2 = q\overline{q} = \alpha^2 - \mathbf{B}^2\).  Note that this norm is a positive definite as expected since a bivector square is negative.

To be more specific about the left handed basis property of quaternions one can note that the quaternion bivector basis is usually defined in terms of the following properties

\begin{equation}\label{eqn:quaternion:60}
\mathbf{i}^2 = \mathbf{j}^2 = \mathbf{k}^2 = -1
\end{equation}
\begin{equation}\label{eqn:quaternion:80}
\mathbf{i}\mathbf{j} = -\mathbf{j}\mathbf{i}, \mathbf{i}\mathbf{k} = -\mathbf{k}\mathbf{i}, \mathbf{j}\mathbf{k} = -\mathbf{k}\mathbf{j}
\end{equation}
\begin{equation}\label{eqn:quaternion:100}
\mathbf{i}\mathbf{j} = \mathbf{k}
\end{equation}

The first two properties are satisfied by any set of orthogonal unit bivectors for the space.  The last property, which could also be written \(\mathbf{i}\mathbf{j}\mathbf{k} = -1\), amounts to a choice for the orientation of this bivector basis of the 2-vector part of the quaternion.

As an example suppose one picks

\begin{equation}\label{eqn:quaternion:120}
\mathbf{i} = \mathbf{e}_2\mathbf{e}_3
\end{equation}
\begin{equation}\label{eqn:quaternion:140}
\mathbf{j} = \mathbf{e}_3\mathbf{e}_1
\end{equation}

Then the third bivector required to complete the basis set subject to the properties above is

\begin{equation}\label{eqn:quaternion:160}
\mathbf{i}\mathbf{j} = \mathbf{e}_2\mathbf{e}_1 = \mathbf{k}
\end{equation}.

Suppose that, instead of the above, one picked a slightly more natural bivector basis, the duals of the unit vectors obtained by multiplication with the pseudoscalar (\(\mathbf{e}_1\mathbf{e}_2\mathbf{e}_3\mathbf{e}_i\)).  These bivectors are

\begin{equation}\label{eqn:quaternion:180}
\mathbf{i}=\mathbf{e}_2\mathbf{e}_3, \mathbf{j}=\mathbf{e}_3\mathbf{e}_1, \mathbf{k}=\mathbf{e}_1\mathbf{e}_2
\end{equation}.

A 0,2-multivector with this as the basis for the bivector part would have properties similar to the standard quaternions (anti-commutative unit quaternions, negation for unit quaternion square, same conjugate, norm and inversion operations, ...), however the triple product would have the value \(\mathbf{i}\mathbf{j}\mathbf{k} = 1\), instead of \(-1\).

\section{quaternion as generator of dot and cross product}

The product of pure quaternions is noted as being a generator of dot and cross products.  This is also true
of a vector bivector product.

Writing a vector \(\Bx\) as

\begin{equation}\label{eqn:quaternion:200}
\Bx = \sum_i x_i \Be_i = x_1 \Be_1 + x_2 \Be_2 + x_3 \Be_3
\end{equation}

And a bivector \(\BB\) (where for short, \(\Be_{ij} = \Be_i \Be_j = \Be_i \wedge \Be_j\)) as:

\begin{equation}\label{eqn:quaternion:220}
\BB = \sum_i b_i \Be_i I = b_1 \Be_{23} + b_2 \Be_{31} + b_3 \Be_{12}
\end{equation}

The product of these two is
\begin{equation}\label{eqn:quaternion:280}
\begin{aligned}
\Bx \BB
&= (x_1 \Be_1 + x_2 \Be_2 + x_3 \Be_3)(b_1 \Be_{23} + b_2 \Be_{31} + b_3 \Be_{12}) \\
&= (x_3 b_2 - x_2 b_3) \Be_1 + (x_1 b_3 - x_3 b_1) \Be_2 + (x_2 b_1 - x_1 b_2) \Be_3 \\
&+ (x_1 b_1 + x_2 b_2 + x_3 b_3) \Be_{123} \\
\end{aligned}
\end{equation}

Looking at the vector and trivector components of this we recognize the dot product and negated cross product
immediately (as with multiplication of pure quaternions).

Those products are, in fact, \(\Bx \cdot \BB\) and \(\Bx \wedge \BB\) respectively.

Introducing a vector and bivector basis \(\alpha = \{ \Be_i \}\), and \(\beta = \{ \Be_i I \}\), we can
express the dot product and cross product of the associated coordinate vectors
in terms of vector bivectors products as follows:

\begin{equation}\label{eqn:quaternion:240}
[\Bx]_\alpha \cdot [\BB]_\beta = \frac{\BB \wedge \Bx}{I}
\end{equation}
\begin{equation}\label{eqn:quaternion:260}
[\Bx]_\alpha \cross [\BB]_\beta = [\BB \cdot \Bx]_\alpha
\end{equation}


\documentclass{article}

\usepackage{amsmath}
\usepackage{mathpazo}

%
% shorthand for bold symbols, convenient for vectors and matrices
%
\newcommand{\Ba}[0]{\mathbf{a}}
\newcommand{\Bb}[0]{\mathbf{b}}
\newcommand{\Bc}[0]{\mathbf{c}}
\newcommand{\Bd}[0]{\mathbf{d}}
\newcommand{\Be}[0]{\mathbf{e}}
\newcommand{\Bf}[0]{\mathbf{f}}
\newcommand{\Bg}[0]{\mathbf{g}}
\newcommand{\Bh}[0]{\mathbf{h}}
\newcommand{\Bi}[0]{\mathbf{i}}
\newcommand{\Bj}[0]{\mathbf{j}}
\newcommand{\Bk}[0]{\mathbf{k}}
\newcommand{\Bl}[0]{\mathbf{l}}
\newcommand{\Bm}[0]{\mathbf{m}}
\newcommand{\Bn}[0]{\mathbf{n}}
\newcommand{\Bo}[0]{\mathbf{o}}
\newcommand{\Bp}[0]{\mathbf{p}}
\newcommand{\Bq}[0]{\mathbf{q}}
\newcommand{\Br}[0]{\mathbf{r}}
\newcommand{\Bs}[0]{\mathbf{s}}
\newcommand{\Bt}[0]{\mathbf{t}}
\newcommand{\Bu}[0]{\mathbf{u}}
\newcommand{\Bv}[0]{\mathbf{v}}
\newcommand{\Bw}[0]{\mathbf{w}}
\newcommand{\Bx}[0]{\mathbf{x}}
\newcommand{\By}[0]{\mathbf{y}}
\newcommand{\Bz}[0]{\mathbf{z}}
\newcommand{\BA}[0]{\mathbf{A}}
\newcommand{\BB}[0]{\mathbf{B}}
\newcommand{\BC}[0]{\mathbf{C}}
\newcommand{\BD}[0]{\mathbf{D}}
\newcommand{\BE}[0]{\mathbf{E}}
\newcommand{\BF}[0]{\mathbf{F}}
\newcommand{\BG}[0]{\mathbf{G}}
\newcommand{\BH}[0]{\mathbf{H}}
\newcommand{\BI}[0]{\mathbf{I}}
\newcommand{\BJ}[0]{\mathbf{J}}
\newcommand{\BK}[0]{\mathbf{K}}
\newcommand{\BL}[0]{\mathbf{L}}
\newcommand{\BM}[0]{\mathbf{M}}
\newcommand{\BN}[0]{\mathbf{N}}
\newcommand{\BO}[0]{\mathbf{O}}
\newcommand{\BP}[0]{\mathbf{P}}
\newcommand{\BQ}[0]{\mathbf{Q}}
\newcommand{\BR}[0]{\mathbf{R}}
\newcommand{\BS}[0]{\mathbf{S}}
\newcommand{\BT}[0]{\mathbf{T}}
\newcommand{\BU}[0]{\mathbf{U}}
\newcommand{\BV}[0]{\mathbf{V}}
\newcommand{\BW}[0]{\mathbf{W}}
\newcommand{\BX}[0]{\mathbf{X}}
\newcommand{\BY}[0]{\mathbf{Y}}
\newcommand{\BZ}[0]{\mathbf{Z}}

\newcommand{\Bzero}[0]{\mathbf{0}}
\newcommand{\Btheta}[0]{\boldsymbol{\theta}}
\newcommand{\Btau}[0]{\boldsymbol{\tau}}
\newcommand{\Bomega}[0]{\boldsymbol{\omega}}

%
% shorthand for unit vectors
%
\newcommand{\acap}[0]{\hat{\Ba}}
\newcommand{\bcap}[0]{\hat{\Bb}}
\newcommand{\ccap}[0]{\hat{\Bc}}
\newcommand{\dcap}[0]{\hat{\Bd}}
\newcommand{\ecap}[0]{\hat{\Be}}
\newcommand{\fcap}[0]{\hat{\Bf}}
\newcommand{\gcap}[0]{\hat{\Bg}}
\newcommand{\hcap}[0]{\hat{\Bh}}
\newcommand{\icap}[0]{\hat{\Bi}}
\newcommand{\jcap}[0]{\hat{\Bj}}
\newcommand{\kcap}[0]{\hat{\Bk}}
\newcommand{\lcap}[0]{\hat{\Bl}}
\newcommand{\mcap}[0]{\hat{\Bm}}
\newcommand{\ncap}[0]{\hat{\Bn}}
\newcommand{\ocap}[0]{\hat{\Bo}}
\newcommand{\pcap}[0]{\hat{\Bp}}
\newcommand{\qcap}[0]{\hat{\Bq}}
\newcommand{\rcap}[0]{\hat{\Br}}
\newcommand{\scap}[0]{\hat{\Bs}}
\newcommand{\tcap}[0]{\hat{\Bt}}
\newcommand{\ucap}[0]{\hat{\Bu}}
\newcommand{\vcap}[0]{\hat{\Bv}}
\newcommand{\wcap}[0]{\hat{\Bw}}
\newcommand{\xcap}[0]{\hat{\Bx}}
\newcommand{\ycap}[0]{\hat{\By}}
\newcommand{\zcap}[0]{\hat{\Bz}}
\newcommand{\thetacap}[0]{\hat{\Btheta}}

%
% to write R^n and C^n in a distinguishable fashion.  Perhaps change this
% to the double lined characters upon figuring out how to do so.
%
\newcommand{\C}[1]{$\mathbb{C}^{#1}$}
\newcommand{\R}[1]{$\mathbb{R}^{#1}$}

%
% various generally useful helpers
%

% derivative of #1 wrt. #2:
\newcommand{\D}[2] {\frac {d#2} {d#1}}

\newcommand{\inv}[1]{\frac{1}{#1}}
\newcommand{\cross}[0]{\times}

\newcommand{\abs}[1]{\lvert{#1}\rvert}
\newcommand{\norm}[1]{\lVert{#1}\rVert}
\newcommand{\innerprod}[2]{\langle{#1}, {#2}\rangle}
\newcommand{\dotprod}[2]{{#1} \cdot {#2}}
\newcommand{\bdotprod}[2]{\left({#1} \cdot {#2}\right)}
\newcommand{\crossprod}[2]{{#1} \cross {#2}}
\newcommand{\tripleprod}[3]{\dotprod{\left(\crossprod{#1}{#2}\right)}{#3}}

\DeclareMathOperator{\Proj}{Proj}
\DeclareMathOperator{\Span}{span}
\DeclareMathOperator{\Sgn}{sgn}
\DeclareMathOperator{\Area}{Area}
\DeclareMathOperator{\Volume}{Volume}

%
% A few miscellaneous things specific to this document
%
\newcommand{\crossop}[1]{\crossprod{#1}{}}

% R2 vector.
\newcommand{\VectorTwo}[2]{
\begin{bmatrix}
 {#1} \\
 {#2}
\end{bmatrix}
}

\newcommand{\VectorN}[1]{
\begin{bmatrix}
{#1}_1 \\
{#1}_2 \\
\vdots \\
{#1}_N \\
\end{bmatrix}
}

\newcommand{\DETuvij}[4]{
\begin{vmatrix}
 {#1}_{#3} & {#1}_{#4} \\
 {#2}_{#3} & {#2}_{#4}
\end{vmatrix}
}

\newcommand{\DETuvwijk}[6]{
\begin{vmatrix}
 {#1}_{#4} & {#1}_{#5} & {#1}_{#6} \\
 {#2}_{#4} & {#2}_{#5} & {#2}_{#6} \\
 {#3}_{#4} & {#3}_{#5} & {#3}_{#6}
\end{vmatrix}
}

\newcommand{\DETuvwxijkl}[8]{
\begin{vmatrix}
 {#1}_{#5} & {#1}_{#6} & {#1}_{#7} & {#1}_{#8} \\
 {#2}_{#5} & {#2}_{#6} & {#2}_{#7} & {#2}_{#8} \\
 {#3}_{#5} & {#3}_{#6} & {#3}_{#7} & {#3}_{#8} \\
 {#4}_{#5} & {#4}_{#6} & {#4}_{#7} & {#4}_{#8} \\
\end{vmatrix}
}

%\newcommand{\DETuvwxyijklm}[10]{
%\begin{vmatrix}
% {#1}_{#6} & {#1}_{#7} & {#1}_{#8} & {#1}_{#9} & {#1}_{#10} \\
% {#2}_{#6} & {#2}_{#7} & {#2}_{#8} & {#2}_{#9} & {#2}_{#10} \\
% {#3}_{#6} & {#3}_{#7} & {#3}_{#8} & {#3}_{#9} & {#3}_{#10} \\
% {#4}_{#6} & {#4}_{#7} & {#4}_{#8} & {#4}_{#9} & {#4}_{#10} \\
% {#5}_{#6} & {#5}_{#7} & {#5}_{#8} & {#5}_{#9} & {#5}_{#10}
%\end{vmatrix}
%}

% R3 vector.
\newcommand{\VectorThree}[3]{
\begin{bmatrix}
 {#1} \\
 {#2} \\
 {#3}
\end{bmatrix}
}


\newcommand{\grad}[0]{\nabla}
\newcommand{\PD}[2]{ \frac{\partial{#1}}{\partial {#2}} }

\title{ Cauchy Equations expressed as a gradient. }
\author{Peeter Joot}
\date{August 13, 2008}

\begin{document}

\maketitle{}

\section{}

The complex number derivative, when it exists, is defined as:

\begin{equation*}
\frac{\delta f}{\delta z} = \frac{ f(z + \delta z) - f(z)}{\delta z}
\end{equation*}
\begin{equation*}
f'(z) = {\text{limit}}_{\abs{\delta z} \rightarrow 0} \quad \frac{\delta f}{\delta z}
\end{equation*}

Like any two variable function, this limit requires that all limiting paths produce the same result, thus it is
minimally necessary that the limits for the particular cases of $\delta z = \delta x + i \delta y$ exist for both
$\delta x = 0$, and $\delta y = 0$ independently.  Of course there are other possible ways for $\delta z \rightarrow 0$, such as spiraling inwards paths.  Apparently it can be shown that if the specific cases are satisfied, then this limit exists for any path (I'm not sure how to show that, nor will try, at least now).

Examining each of these cases separately, we have for $\delta x = 0$, and $f(z) = u(x,y) + i v(x,y)$:

\begin{align*}
\frac{\delta f}{\delta z}
&= \frac{u(x,y + \delta y) + i v(x,y + \delta y)}{i\delta y} \\
&\rightarrow -i \frac{\partial u(x,y)}{\partial y} + \frac{\partial v(x,y)}{\partial y} \\
\end{align*}

and for $\delta y = 0$
\begin{align*}
\frac{\delta f}{\delta z}
&= \frac{u(x + \delta x,y) + i v(x + \delta x, y)}{\delta x} \\
&\rightarrow \frac{\partial u(x,y)}{\partial x} + i\frac{\partial v(x,y)}{\partial x} \\
\end{align*}

If these are equal regardless of the path, then equating real and imaginary parts of these respective equations we have:

\begin{align}
\frac{\partial v}{\partial x} + \frac{\partial u}{\partial y} &= 0 \\
\frac{\partial u}{\partial x} - \frac{\partial v}{\partial y} &= 0
\end{align}

Now, these are strikingly similar to the gradient, and we make this similarily explicit using the planar
pseudoscalar
$i=\Be_1 \wedge \Be_2 = \Be_1 \Be_2$
as the unit imaginary.  For the first equation, pre multiplying by $1 = \Be_{11}$, and post multiplying by $\Be_2$ we have:

\begin{equation*}
\Be_1 \frac{\partial \Be_{12} v}{\partial x} + \Be_{2}\frac{\partial u}{\partial y} = 0,
\end{equation*}

and for the second, pre multiply by $\Be_1$, and post multiply the $\partial_y$ term by $1 = \Be_{22}$, and rearrange:
\begin{equation*}
\Be_1 \frac{\partial u}{\partial x} + \Be_{2} \frac{\partial \Be_{12} v}{\partial y} = 0.
\end{equation*}

Adding these we have:
\begin{equation*}
\Be_1 \frac{\partial u + \Be_{12}}{\partial x} + \Be_{2} \frac{\partial u + \Be_{12} v}{\partial y} = 0.
\end{equation*}

Since $f = u + i v$, this is just

\begin{equation}
\Be_1 \frac{\partial f}{\partial x} + \Be_{2} \frac{\partial f}{\partial y} = 0.
\end{equation}

Or,
\begin{equation}\label{eqn:gradf}
\grad f = 0
\end{equation}

By taking second partial derivatives and equating mixed partials we are used to seeing these Cauchy-Riemann equations
take this form as second order equations:

\begin{equation}\label{eqn:uxx}
\grad^2 u = u_{xx} + u_{yy} = 0
\end{equation}
\begin{equation}
\grad^2 v = v_{xx} + v_{yy} = 0
\end{equation}

Given this, equation \ref{eqn:gradf} is something that we could have perhaps guessed, since the square root of the Laplacian operator, is in fact the gradient (there are an infinite number of such square roots, since any rotation of the coordinate system that expresses the gradient also works).  However, a guess of this isn't required since we see this explicitly through some logical composition of relationships.

The end result is that we can make a statement that
in regions where the complex function is analytic (has a derivative), the gradient of that function is zero in that region.

This is a kind of interesting result and I expect that this will relevant when figuring out how the geometric calculus
all fits together.

\subsection{ Verify we still have the Cauchy equations hiding in the gradient. }

We have:

\begin{equation*}
\grad f \Be_1 = \grad ( \Be_1 u - \Be_2 v) = 0
\end{equation*}

If this is to be zero, both the scalar and bivector parts of this equation must also be zero.

\begin{align*}
(\grad \cdot f) \Be_1
&= \grad \cdot ( \Be_1 u - \Be_2 v) \\
&= (\Be_1 \partial_x + \Be_2 \partial_y) \cdot ( \Be_1 u - \Be_2 v) \\
&= (\partial_x u - \partial_y v) = 0
\end{align*}

\begin{align*}
(\grad \wedge f) \Be_1
&= \grad \wedge ( \Be_1 u - \Be_2 v) \\
&= (\Be_1 \partial_x + \Be_2 \partial_y) \wedge ( \Be_1 u - \Be_2 v) \\
&= -\Be_1 \wedge \Be_2 (\partial_x v + \partial_y u) = 0
\end{align*}

We therefore see that this recovers the expected pair of Cauchy equations:

\begin{align*}
\partial_x u - \partial_y v &= 0 \\
\partial_x v + \partial_y u &= 0
\end{align*}

\section{ Complex number formed from plane vector. }

Now, we form a complex number from a vector by factoring out one of the unit vectors for the plane, and introducing an
appropriate unit imaginary from the remaining plane pseudoscalar.
However, we have no requirement for orthonormal basis vectors to
express any particular vector constrained to a plane.  In terms of the usual reciprocal relationships a vector in a plane can be expressed as:

\begin{align*}
x &= x^1 \Be_1 + x^2 \Be_2 \\
\Be^i \cdot \Be_j &= {\delta^i}_{j} \\
x^i &= x \cdot \Be^i
\end{align*}

Now, similar to how we create complex numbers from vectors by factoring out a unit vector, here we can also factor out a vector to put this in complex form:

\begin{equation}\label{eqn:zgeneral}
z = \Be^1 x = x^1 + (\Be^1 \wedge \Be_2) x^2
\end{equation}

Lets verify that this funny wedge of mixed upper and lower index basis vectors behaves as expected as purely imaginary
quantity:

\begin{align*}
{(\Be^1 \wedge \Be_2)}^2
&= \Be^1 \Be_2 \Be^1 \Be_2 \\
&= - \Be^1 \Be^1 \Be_2 \Be_2 \\
&>= 0
\end{align*}

Since both $\Be^1 \Be^1 > 0$, and $\Be_2 \Be_2 > 0$, the end result operates as a pure imaginary quantity, however it is potentially
scaled since there is no requirement that $\Be^1 \Be^1 \Be_2 \Be_2 = 1$.

Going back to equation \ref{eqn:zgeneral}, lets take the gradient and verify that this is zero as expected:

\begin{align*}
\left(\Be^1 \partial_{x^1} + \Be^2 \partial_{x^2}\right) \left(x^1 + (\Be^1 \wedge \Be_2) x^2\right)
&= \Be^1 \PD{x^1}{x^1} -\Be^1 \Be^2 \Be_2 \PD{x^2}{x^2} + \Be^2\PD{x^1}{x^2} + \Be^1 \Be^1 \Be_2 \PD{x^2}{x^1} \\
&= \Be^1 - \Be^1 \\
&= 0
\end{align*}

\end{document}


%
% Copyright � 2012 Peeter Joot.  All Rights Reserved.
% Licenced as described in the file LICENSE under the root directory of this GIT repository.
%

%
%
\chapter{Legendre Polynomials}
\index{Legendre polynomial}
\label{chap:legendre}
%\date{Feb 4, 2008.  legendre.tex}

Exercise 8.4, from \citep{hestenes1999nfc}.

Find the first couple terms of the Legendre polynomial expansion of

\begin{equation}\label{eqn:legendre:20}
\inv{\abs{\Bx - \Ba}}
\end{equation}

Write

\begin{equation}\label{eqn:legendre:40}
f(x) = \inv{\abs{\Bx}}
\end{equation}

Expanding \(f(\Bx - \Ba)\) about \(\Bx\) we have

\begin{equation}\label{eqn:legendre:60}
\inv{\abs{\Bx - \Ba}} =
\sum_{k=0}{ \inv{k!} (-\agrad)^k} \inv{\abs{\Bx}}
\end{equation}

Expanding the first term we have

\begin{equation}\label{eqn:legendre:200}
\begin{aligned}
-\agrad \inv{\abs{\Bx}}
&=
\inv{{\abs{\Bx}}^2} \agrad {\abs{\Bx}} \\
&=
\inv{{\abs{\Bx}}^2} \agrad (\Bx^2)^{1/2} \\
&=
\inv{{\abs{\Bx}}^2} \frac{(1/2)}{({\abs{\Bx}}^2)^{1/2}}\agrad \Bx^2 \\
&=
\frac{\Ba \cdot \Bx}{{\abs{\Bx}}^3}
\end{aligned}
\end{equation}

Expansion of the second derivative term is
\begin{equation}\label{eqn:legendre:220}
\begin{aligned}
\frac{(-\agrad)}{2}\frac{(-\agrad)}{1}\inv{\abs{\Bx}}
&=
\frac{\agrad}{2} \left(\frac{-\Ba \cdot \Bx}{{\abs{\Bx}}^3}\right) \\
&=
\frac{-1}{2}
\left(
\frac{\agrad {(\Ba \cdot \Bx)}}{{\abs{\Bx}}^3} + {(\Ba \cdot \Bx)}\agrad \inv{{\abs{\Bx}}^3} \right) \\
\end{aligned}
\end{equation}

For this we need
\begin{equation}\label{eqn:legendre:80}
\agrad {(\Ba \cdot \Bx)} =
\Ba \cdot (\agrad {\Bx}) = \Ba^2
\end{equation}

And
\begin{equation}\label{eqn:legendre:240}
\begin{aligned}
\agrad \inv{{\abs{\Bx}}^k}
&=
k \inv{{\abs{\Bx}}^{k-1}} \agrad \inv{{\abs{\Bx}}} \\
&=
k \inv{{\abs{\Bx}}^{k-1}} \frac{- \Ba \cdot \Bx }{{\abs{\Bx}}^3} \\
&=
-k \frac{\Ba \cdot \Bx }{{\abs{\Bx}}^{k+2}} \\
\end{aligned}
\end{equation}

Thus the second derivative term is
\begin{equation}\label{eqn:legendre:260}
\begin{aligned}
\frac{-1}{2}
\left(
\frac{\Ba^2}{{\abs{\Bx}}^3} -3 \frac{(\Ba \cdot \Bx)^2} {{\abs{\Bx}}^5} \right)
=
\frac{ (1/2)\left( 3 (\Ba \cdot \Bx)^2 - \Ba^2 \Bx^2 \right) }
{ {{\abs{\Bx}}^5} }
\end{aligned}
\end{equation}

Summing these terms we have

\begin{equation}\label{eqn:legendre:100}
\inv{\abs{\Bx -\Ba}} =
\inv{\abs{\Bx}} +
\frac{ \Ba \cdot \Bx } { {\abs{\Bx}}^3 } +
\frac{ (1/2)\left( 3 (\Ba \cdot \Bx)^2 - \Ba^2 \Bx^2 \right) } { {{\abs{\Bx}}^5} } + \cdots
\end{equation}

NFCM writes this as
\begin{equation}\label{eqn:legendre:120}
\inv{\abs{\Bx -\Ba}} =
\frac{ P_0(\bxa) } {  \abs{\Bx}} +
\frac{ P_1(\bxa) } { {\abs{\Bx}}^3 } +
\frac{ P_2(\bxa) } { {\abs{\Bx}}^5 } + \cdots
\end{equation}

And calls \(P_i = P_i(\bxa)\) terms the Legendre polynomials.  This is not terribly clear since one expects a different form for the Legendre polynomials.

Using the Taylor formula one can derive a recurrence relation for these that makes the calculation a bit
simpler

\begin{equation}\label{eqn:legendre:280}
\begin{aligned}
\frac{P_{k+1}}{\abs{\Bx}^{2(k+1)+1}}
&= \frac{-\agrad}{k+1}\left(\frac{P_k}{\abs{\Bx}^{2k+1}}\right) \\
&=
\frac{-1}{k+1}
\left(
\frac{\agrad({P_k}}
{\abs{\Bx}^{2k+1}}
+
{P_k}\frac{\agrad}
{\abs{\Bx}^{2k+1}}
\right) \\
&=
\inv{k+1}
\left(
{P_k}(2k+1) \frac{\Ba \cdot \Bx}
{\abs{\Bx}^{2k+3}}
-\Bx^2 \frac{\agrad{P_k}}
{\abs{\Bx}^{2k+3}}
\right) \\
\end{aligned}
\end{equation}

Or
\begin{equation}\label{eqn:legendre:300}
\begin{aligned}
(k+1){P_{k+1}}
=
{P_k}(2k+1) {\Ba \cdot \Bx}
-\Bx^2 {\agrad{P_k}}
\end{aligned}
\end{equation}

Some of these have been calculated

\begin{equation}\label{eqn:legendre:320}
\begin{aligned}
P_0 &= 1 \\
P_1 &= \Ba \cdot \Bx \\
P_2 &= \half(3(\Ba \cdot \Bx)^2 -\Ba^2\Bx^2) \\
\end{aligned}
\end{equation}

And for the derivatives

\begin{equation}\label{eqn:legendre:340}
\begin{aligned}
\agrad P_0 &= 0 \\
\agrad P_1 &= \Ba^2 \\
\agrad P_2 &= \half((3)(2)(\Ba \cdot \Bx)\Ba^2 - 2\Ba^2\Bx \cdot \Ba) \\
           &= 2\Ba^2(\Bx \cdot \Ba) \\
\end{aligned}
\end{equation}

Using the recurrence relation one can calculate \(P_3\) for example.

\begin{equation}\label{eqn:legendre:360}
\begin{aligned}
P_3
%(k+1){P_{k+1}} ; k=2
&=
(1/3)\left(
\frac{5}{2}(3(\Ba \cdot \Bx)^2 -\Ba^2\Bx^2)({\Ba \cdot \Bx})
- 2 \Bx^2 \Ba^2(\Bx \cdot \Ba) \right) \\
&=
(1/3) ({\Ba \cdot \Bx}) \left(
\frac{5}{2}(3(\Ba \cdot \Bx)^2 -\Ba^2\Bx^2)
- 2 \Bx^2 \Ba^2 \right) \\
&=
({\Ba \cdot \Bx}) \left( \frac{5}{2}((\Ba \cdot \Bx)^2 ) - 3/2 \Bx^2 \Ba^2 \right) \\
&=
\half({\Ba \cdot \Bx}) ( {5}(\Ba \cdot \Bx)^2 - 3 \Bx^2 \Ba^2 ) \\
\end{aligned}
\end{equation}

\section{ Putting things in standard Legendre polynomial form}

This is still pretty laborious to calculate, especially because of not having a closed form recurrence
relation for \(\agrad P_k\).  Let us relate these to the standard Legendre polynomial form.

Observe that we can write

\begin{equation}\label{eqn:legendre:380}
\begin{aligned}
P_0(\bxa) &= 1 \\
\frac{P_1(\bxa)}{\abs{\Bx} \abs{\Ba}} &= \costheta \\
\frac{P_2(\bxa)}{\abs{\Bx}^2 \abs{\Ba}^2} &= \half(3(\costheta)^2 - 1) \\
\frac{P_3(\bxa)}{\abs{\Bx}^3 \abs{\Ba}^3} &= \half ( {5}(\costheta)^3 - 3 {(\costheta)} ) \\
\end{aligned}
\end{equation}

With this scaling, we have the standard form for the Legendre polynomials, and can write

\begin{equation}\label{eqn:legendre:140}
\inv{\Bx-\Ba} = \inv{\abs{\Bx}}\left(
P_0
+ \frac{\abs{\Ba}}{\abs{\Bx}} P_1(\costheta)
+ \left(\frac{\abs{\Ba}}{\abs{\Bx}}\right)^2 P_2(\costheta)
+ \left(\frac{\abs{\Ba}}{\abs{\Bx}}\right)^3 P_3(\costheta)
+ \cdots \right)
\end{equation}

\section{ Scaling standard form Legendre polynomials}

Since the odd Legendre polynomials have only odd terms and even have only even terms this allows for
the scaled form that NFCM uses.

\begin{equation}\label{eqn:legendre:400}
\begin{aligned}
P_0(\bxa) &= P_0(\costheta) \\
P_1(\bxa) &= \abs{\Bx}\abs{\Ba} P_1(\costheta) = \Ba \cdot \Bx \\
P_2(\bxa) &= \abs{\Bx}^2\abs{\Ba}^2 P_2(\costheta) = \half(3(\Ba \cdot \Bx)^2 - \Bx^2\Ba^2) \\
P_3(\bxa) &= \abs{\Bx}^3\abs{\Ba}^3 P_3(\costheta) = \half(5(\Ba \cdot \Bx)^3 - 3(\Ba \cdot \Bx) \Bx^2\Ba^2) \\
\end{aligned}
\end{equation}

Every term for the \(k^{th}\) polynomial is a permutation of the geometric product \(\Bx^k\Ba^k\).

This allows for writing some of these terms using the wedge product.  Using the product expansion:

\begin{equation}\label{eqn:legendre:160}
%\Ba \Bx \Bx \Ba = \Ba^2 \Bx^2 = (\Ba \cdot \Bx + \Ba \wedge \Bx)(\Bx \cdot \Ba + \Bx \wedge \Ba) = (\Ba \cdot \Bx)^2 - ( \Ba \wedge \Bx )^2
%\Ba^2 \Bx^2 = (\Ba \cdot \Bx)^2 - ( \Ba \wedge \Bx )^2
(\Ba \cdot \Bx)^2 = ( \Ba \wedge \Bx )^2 + \Ba^2 \Bx^2
\end{equation}

Thus we have:
\begin{equation}\label{eqn:legendre:420}
\begin{aligned}
P_2(\bxa)
&= (\Ba \cdot \Bx)^2 + \half(\Ba \wedge \Bx)^2 \\
&= (\Ba \cdot \Bx)^2 - \half\abs{\Ba \wedge \Bx}^2 \\
\end{aligned}
\end{equation}

This is nice geometrically since the directional dependence of this term on the co-linearity and
perpendicularity of the vectors \(\Ba\) and \(\Bx\) is clear.

Doing the same for the \(P_3\):

\begin{equation}\label{eqn:legendre:440}
\begin{aligned}
P_3(\bxa) &= (\Ba \cdot \Bx)\half(5(\Ba \cdot \Bx)^2 - 3\Bx^2\Ba^2) \\
          &= (\Ba \cdot \Bx)\half(2(\Ba \cdot \Bx)^2 + 3(\Ba \wedge \Bx)^2) \\
          &= (\Ba \cdot \Bx)((\Ba \cdot \Bx)^2 - \frac{3}{2}\abs{\Ba \wedge \Bx}^2) \\
\end{aligned}
\end{equation}

I suppose that one could get the same geometrical interpretation with a standard Legendre expansion in terms of \(\costheta = cos(\theta)\) terms, by collect both \(sin(\theta)\) and \(cos(\theta)\) powers, but one
can see the power of writing things explicitly in terms of the original vectors.

\section{ Note on NFCM Legendre polynomial notation}

In NFCM's slightly abusive notation \(P_k\) was used with various meanings.  He wrote \(P_k(\costheta) = \frac{P_k(\bxa)}{\abs{\Bx}^k \abs{\Ba}^k}\).

Note for example that the standard first degree Legendre polynomial \(P_1(x) = x\) evaluated with a \(\bxa\) value:

\begin{equation}\label{eqn:legendre:460}
\begin{aligned}
\inv {\abs{\Bx}\abs{\Ba}} {P_1(x) \vert_{x=\bxa}} &= \xcap \acap \\
&= \xcap \cdot \acap + \xcap \wedge \acap \\
\end{aligned}
\end{equation}

This has a bivector component in addition to the component identical to the standard Legendre polynomial
term (the first part).

By luck it happens that the scalar part of this equals \(P_1(\costheta)\), but this
is not the case for other terms.  Example, \(P_2(\bxa)\):

\begin{equation}\label{eqn:legendre:480}
\begin{aligned}
{P_2(x) \vert_{x=\bxa}}
&= \half( 3(\Bx \Ba)^2 - 1 ) \\
&= \half( 3(-\Ba \Bx + 2 \Ba \cdot \Bx )(\Bx \Ba) - 1 ) \\
&= \half( 3(-\Ba^2 \Bx^2 + 2(\Ba \cdot \Bx)^2 + 2(\Ba \cdot \Bx)(\Bx \wedge \Ba)) - 1 ) \\
&=  -(3/2)\Ba^2 \Bx^2 + 3(\Ba \cdot \Bx)^2 + 3(\Ba \cdot \Bx)(\Bx \wedge \Ba) - 1/2  \\
\end{aligned}
\end{equation}

Scaling this by \(1/(\Ba^2\Bx^2)\) is
\begin{equation}\label{eqn:legendre:180}
-\frac{3}{2} + 3(\costheta)^2 + 3(\costheta)(\xcap \wedge \acap) - \inv{\Ba^2\Bx^2} \\
\end{equation}

The scalar part of this is not anything recognizable.

%
% Copyright � 2012 Peeter Joot.  All Rights Reserved.
% Licenced as described in the file LICENSE under the root directory of this GIT repository.
%

%
%
\chapter{Levi-Civitica summation identity}
\index{Levi-Civitica tensor}
\label{chap:levi}
%\date{March 13, 2009.  levi.tex}

\section{Motivation}

In \citep{byron1992mca} it is left to the reader to show

\index{contraction!Levi-Civitica tensor}
\begin{equation}\label{eqn:levi:20}
\begin{aligned}
\sum_k \epsilon_{ijk} \epsilon_{klm} = \delta_{il}\delta_{jm} - \delta_{jl}\delta_{im}
\end{aligned}
\end{equation}

\section{A mechanical proof}

Although it is not mathematical, this is easy to prove, at least for 3D.  The
following perl code does the trick

\lstinputlisting{listings/levi.pl}

The output produced has all the variations of indices, such as

\begin{equation}\label{eqn:levi:40}
\begin{aligned}
0 &= \sum_{k=1}^{3} \epsilon_{11k} \epsilon_{k11} = \delta_{11}\delta_{11} - \delta_{11}\delta_{11} \\
0 &= \sum_{k=1}^{3} \epsilon_{11k} \epsilon_{k12} = \delta_{11}\delta_{12} - \delta_{11}\delta_{12} \\
\vdots \\
0 &= \sum_{k=1}^{3} \epsilon_{11k} \epsilon_{k33} = \delta_{13}\delta_{13} - \delta_{13}\delta_{13} \\
0 &= \sum_{k=1}^{3} \epsilon_{12k} \epsilon_{k11} = \delta_{11}\delta_{21} - \delta_{21}\delta_{11} \\
1 &= \sum_{k=1}^{3} \epsilon_{12k} \epsilon_{k12} = \delta_{11}\delta_{22} - \delta_{21}\delta_{12} \\
0 &= \sum_{k=1}^{3} \epsilon_{12k} \epsilon_{k13} = \delta_{11}\delta_{23} - \delta_{21}\delta_{13} \\
-1 &= \sum_{k=1}^{3} \epsilon_{12k} \epsilon_{k21} = \delta_{12}\delta_{21} - \delta_{22}\delta_{11} \\
\vdots \\
\end{aligned}
\end{equation}

\section{Proof using bivector dot product}

This identity can also be derived from an expansion of the bivector
dot product in two different ways.

\begin{equation}\label{eqn:levi:60}
\begin{aligned}
( \Be_i \wedge \Be_j ) \cdot ( \Be_m \wedge \Be_n )
&=
( ( \Be_i \wedge \Be_j ) \cdot \Be_m ) \cdot \Be_n  \\
&=
(
\Be_i ( \Be_j \cdot \Be_m )
-\Be_j ( \Be_i \cdot \Be_m )
) \cdot \Be_n  \\
&=
( \Be_i \delta_{jm} -\Be_j \delta_{im} ) \cdot \Be_n  \\
&=
\delta_{in} \delta_{jm} -\delta_{jn} \delta_{im}
\end{aligned}
\end{equation}

Expressing the wedge product in terms duality, using the pseudoscalar
\(I = \Be_1 \Be_2 \Be_3\), we have

\begin{equation}\label{eqn:levi:80}
\begin{aligned}
(\Be_i \wedge \Be_j ) \Be_k = I \epsilon_{ijk}
\end{aligned}
\end{equation}

Or
\begin{equation}\label{eqn:levi:100}
\begin{aligned}
\Be_i \wedge \Be_j = I \sum_k \epsilon_{ijk} \Be_k
\end{aligned}
\end{equation}

Then the bivector dot product is
\begin{equation}\label{eqn:levi:120}
\begin{aligned}
( \Be_i \wedge \Be_j ) \cdot ( \Be_m \wedge \Be_n )
&=
\gpgradezero{
I \sum_k \epsilon_{ijk} \Be_k I \sum_p \epsilon_{mnp} \Be_p
} \\
&=
I^2 \sum_{k,p} \epsilon_{ijk} \epsilon_{mnp} \gpgradezero{ \Be_k \Be_p } \\
&=
- \sum_{k,p} \epsilon_{ijk} \epsilon_{mnp} \delta_{kp} \\
&=
- \sum_{k} \epsilon_{ijk} \epsilon_{mnk} \\
\end{aligned}
\end{equation}

Comparing the two expansions we have

\begin{equation}\label{eqn:levi:140}
\begin{aligned}
\sum_{k} \epsilon_{ijk} \epsilon_{mnk} &= \delta_{jn} \delta_{im} - \delta_{in} \delta_{jm}
\end{aligned}
\end{equation}

Which is equivalent to the original identity (after an index switcheroo).
Note both the dimension and metric dependencies in this proof.

%
% Copyright � 2012 Peeter Joot.  All Rights Reserved.
% Licenced as described in the file LICENSE under the root directory of this GIT repository.
%

%
%
\chapter{Some NFCM exercise solutions and notes}
\label{chap:nfcmCh2}
%\date{Nov 27, 2008.  nfcmCh2.tex}

\paragraph{Solutions for problems in chapter 2}

I recall that some of the problems from this chapter of
\citep{hestenes1999nfc}
were fairly tricky.  Did I end up doing them all?  I intended to
revisit these and make sure I understood it all.  As I do so, write up
solutions, starting with \(1.3\), a question on the Geometric Algebra group.

Another thing I recall from the text is that I was fairly confused about
all the mass of identities by the time I got through it, and it was not clear
to me which were the fundamental ones.
Eventually I figured out that it is
really grade selection that is the fundamental operation, and
found better presentations of axiomatic treatment in \citep{doran2003gap}.

For reference the GA axioms are

\begin{itemize}
\item vector product is linear

\begin{equation}\label{eqn:nfcmCh2:20}
\begin{aligned}
a ( \alpha b + \beta c) &= \alpha a b + \beta a c \\
( \alpha a + \beta b) c &= \alpha a c + \beta b c
\end{aligned}
\end{equation}

\item distribution of vector product

\begin{equation}\label{eqn:nfcmCh2:40}
\begin{aligned}
(a b) c = a (b c) = a b c
\end{aligned}
\end{equation}

\item vector contraction

\begin{equation}\label{eqn:nfcm_ch2:contractionAxiom}
\begin{aligned}
a^2 \in \mathbb{R}
\end{aligned}
\end{equation}

For a Euclidean space, this provides the length \(a^2 = \Abs{a}^2\), but for relativity and conformal geometry this specific meaning is not required.

\end{itemize}

The definition of the generalized dot between two blades is

\begin{equation}\label{eqn:nfcm_ch2:generalDot}
\begin{aligned}
A_r \cdot B_s = \gpgrade{A B}{{\Abs{r -s}}}
\end{aligned}
\end{equation}

and the generalized wedge product definition for two blades is

\begin{equation}\label{eqn:nfcm_ch2:generalWedge}
\begin{aligned}
A_r \wedge B_s = \gpgrade{A B}{r + s}.
\end{aligned}
\end{equation}

With these definitions and the GA axioms everything else should logically follow.

I personally found it was really easy to go around in circles attempting the various proofs, and intended to revisit all of these
and prove them all for myself making sure I did not invoke any circular arguments and used only things already proven.

\subsection{Exercise 1.3}

Solve for \(x\)

\begin{equation}\label{eqn:nfcmCh2:60}
\begin{aligned}
\alpha x + a x \cdot b = c
\end{aligned}
\end{equation}

where \(\alpha\) is a scalar and all the rest are vectors.

\subsubsection{Solution}

Can dot or wedge the entire equation with the constant vectors.  In particular

\begin{equation}\label{eqn:nfcmCh2:80}
\begin{aligned}
c \cdot b &= (\alpha x + a x \cdot b) \cdot b \\
&= (\alpha + a \cdot b) x \cdot b
\end{aligned}
\end{equation}
\begin{equation}\label{eqn:nfcmCh2:100}
\begin{aligned}
\implies
x \cdot b &= \frac{c \cdot b}{\alpha + a \cdot b} \\
\end{aligned}
\end{equation}

and
\begin{equation}\label{eqn:nfcmCh2:120}
\begin{aligned}
c \wedge a &= (\alpha x + a x \cdot b) \wedge a \\
&= \alpha (x \wedge a) + \mathLabelBox
[
   labelstyle={xshift=2cm},
   linestyle={out=270,in=90, latex-}
]
{(a \wedge a)}{\(=0\)} (x \cdot b) \wedge a \\
\end{aligned}
\end{equation}
\begin{equation}\label{eqn:nfcmCh2:140}
\begin{aligned}
\implies
x \wedge a &= \inv{\alpha} (c \wedge a) \\
\end{aligned}
\end{equation}

This last can be reduced by dotting with \(b\), and then substitute the
result for \(x \cdot b\) from above

\begin{equation}\label{eqn:nfcmCh2:160}
\begin{aligned}
(x \wedge a) \cdot b
&= x (a \cdot b) - (x \cdot b) a \\
&= x (a \cdot b) - \frac{c \cdot b}{\alpha + a \cdot b} a \\
\end{aligned}
\end{equation}

Thus the final solution is

\begin{equation}\label{eqn:nfcmCh2:180}
\begin{aligned}
x = \inv{a \cdot b}\left(
\frac{c \cdot b}{\alpha + a \cdot b} a
+ \inv{\alpha} (c \wedge a) \cdot b
\right)
\end{aligned}
\end{equation}

Question: was there a geometric or physical motivation for this question.  I can not recall one?

\section{Sequential proofs of required identities}

\subsection{Split of symmetric and antisymmetric parts of the vector product}

NFCM defines the vector dot and wedge products in terms of the symmetric and antisymmetric parts, and not in terms of grade
selection.

The symmetric and antisymmetric split of a vector product takes the form

\begin{equation}\label{eqn:nfcmCh2:200}
\begin{aligned}
a b &= \inv{2}(a b + b a) + \inv{2}(a b - b a)
\end{aligned}
\end{equation}

Observe that if the two vectors are colinear, say \(b = \alpha a\), then this is

\begin{equation}\label{eqn:nfcmCh2:220}
\begin{aligned}
a b &= \frac{\alpha}{2} (a^2 + a^2) + \frac{\alpha}{2}(a^2 - a^2 )
\end{aligned}
\end{equation}

The antisymmetric part is zero for any colinear vectors, while the symmetric part is a scalar by the contraction axiom \eqnref{eqn:nfcm_ch2:contractionAxiom}.

Now, suppose that one splits the vector \(b\) into a part that is explicit
colinear with \(a\), as in \(b = \alpha a + c\).

Here one can observe that none of the colinear component of this vector
contributes to the antisymmetric part of the split

\begin{equation}\label{eqn:nfcmCh2:240}
\begin{aligned}
\inv{2}(a b - b a) &= \inv{2}(a (\alpha a + c) - (\alpha a + c) a) \\
&= \inv{2}(a c - c a)
\end{aligned}
\end{equation}

So, in a very loose fashion the symmetric part can be observed to be
due to only colinear parts of the vectors whereas colinear components of the
vectors do not contribute at all to the antisymmetric part of the product split.
One can see that there is a notion of parallelism and perpendicularity built
into this construction.

What is of interest here is to show that this symmetric and antisymmetric split
also provides the scalar and bivector parts of the product, and thus matches the
definitions of generalized dot and wedge products.

While it has been observed that the symmetric product is a scalar for colinear vectors
it has not been
demonstrated that this is necessarily a scalar in the general case.

Consideration of the square of \(a + b\) is enough to do so.

\begin{equation}\label{eqn:nfcmCh2:260}
\begin{aligned}
(a + b)^2 &= a^2 + b^2 + ab + ba \\
\implies \\
\end{aligned}
\end{equation}
\begin{equation}\label{eqn:nfcm_ch2:pythagorus}
\begin{aligned}
\inv{2}\left((a + b)^2 - a^2 - b^2\right) &= \inv{2}(ab + ba)
\end{aligned}
\end{equation}

We have only scalar terms on the LHS, which demonstrates that the symmetric product is necessarily a scalar.
This is despite the fact that the exact definition of \(a^2\) (ie: the metric for the space) has not been specified, nor even
a requirement that this vector square is even satisfies \(a^2 >= 0\).  Such an omission is valuable since it allows
for a natural formulation of relativistic four-vector algebra where both signs are allowed for the vector square.

Observe that \eqnref{eqn:nfcm_ch2:pythagorus} provides a generalization of the Pythagorean theorem.  If one defines, as in
Euclidean space, that two vectors are perpendicular by

\begin{equation}\label{eqn:nfcmCh2:280}
\begin{aligned}
(a + b)^2 = a^2 + b^2
\end{aligned}
\end{equation}

Then one necessarily has

\begin{equation}\label{eqn:nfcmCh2:300}
\begin{aligned}
\inv{2}(ab + ba) = 0
\end{aligned}
\end{equation}

So, that we have as a consequence of this perpendicularity definition a sign inversion on reversal
\begin{equation}\label{eqn:nfcmCh2:320}
\begin{aligned}
ba = -ab
\end{aligned}
\end{equation}

This equation contains the essence of the concept of grade.  The product of a pair of vectors is grade two
if reversal of the factors changes the sign, which in turn implies the two factors must be perpendicular.

Given a set of vectors that, according to the symmetric vector product (dot product) are all either mutually perpendicular or colinear, grouping by colinear sets determines the grade

\begin{equation}\label{eqn:nfcmCh2:340}
\begin{aligned}
a_1 a_2 a_3 ... a_m = (b_{j_1} b_{j_2} ... ) (b_{k_1} b_{k_2} ... ) ...  (b_{l_1} b_{l_2} ... )
\end{aligned}
\end{equation}

after grouping in pairs of colinear vectors (for which the squares are scalars) the count of the remaining elements is the grade.  By
example, suppose that \({e_i}\) is a normal basis for \R{N} \(e_i \cdot e_j \propto \delta_{ij}\), and one wishes to determine the grade
of a product.  Permuting this product so that it is ordered by index leaves it in a form that the grade can be observed by inspection

\begin{equation}\label{eqn:nfcmCh2:360}
\begin{aligned}
e_3 e_7 e_1 e_2 e_1 e_7 e_6 e_7
&= - e_3 e_1 e_7 e_2 e_1 e_7 e_6 e_7 \\
&= e_1 e_3 e_7 e_2 e_1 e_7 e_6 e_7 \\
&= ... \\
&\propto e_1 e_1 e_2 e_3 e_6 e_7 e_7 e_7 \\
&= (e_1 e_1) e_2 e_3 e_6 (e_7 e_7) e_7 \\
&\propto e_2 e_3 e_6 e_7 \\
\end{aligned}
\end{equation}

This is an example of a grade four product.  Given this implicit definition of grade, one can then see that the antisymmetric product of
two vectors is necessarily grade two.  An explicit enumeration of a vector product in terms of an explicit normal basis and associated
coordinates is helpful here to demonstrate this.

Let

\begin{equation}\label{eqn:nfcmCh2:380}
\begin{aligned}
a &= \sum_i a_i e_i \\
b &= \sum_j b_j e_j
\end{aligned}
\end{equation}

now, form the product
\begin{equation}\label{eqn:nfcmCh2:400}
\begin{aligned}
a b
&= \sum_i \sum_j a_i b_j e_i e_j \\
&=
 \sum_{i < j} a_i b_j e_i e_j
+\sum_{i = j} a_i b_j e_i e_j
+\sum_{i > j} a_i b_j e_i e_j \\
&=
 \sum_{i < j} a_i b_j e_i e_j
+\sum_{i = j} a_i b_j e_i e_j
+\sum_{j > i} a_j b_i e_j e_i \\
&=
 \sum_{i < j} a_i b_j e_i e_j
+\sum_{i = j} a_i b_j e_i e_j
-\sum_{i < j} a_j b_i e_i e_j \\
&= \sum_{i} a_i b_i (e_i)^2
+ \sum_{i < j} (a_i b_j - a_j b_i) e_i e_j  \\
\end{aligned}
\end{equation}

similarly

\begin{equation}\label{eqn:nfcmCh2:420}
\begin{aligned}
b a &= \sum_{i} a_i b_i (e_i)^2 - \sum_{i < j} (a_i b_j - a_j b_i) e_i e_j  \\
\end{aligned}
\end{equation}

Thus the symmetric and antisymmetric products are respectively

\begin{equation}\label{eqn:nfcmCh2:440}
\begin{aligned}
\inv{2}(a b + b a) &= \sum_{i} a_i b_i (e_i)^2 \\
\inv{2}(a b - b a) &= \sum_{i < j} (a_i b_j - a_j b_i) e_i e_j  \\
\end{aligned}
\end{equation}

The first part as shown above with non-coordinate arguments is a scalar.  Each term in the antisymmetric product has a grade two term, which
as a product of perpendicular vectors cannot be reduced any further, so it is therefore grade two in its entirety.

following the definitions of \eqnref{eqn:nfcm_ch2:generalDot} and \eqnref{eqn:nfcm_ch2:generalWedge} respectively, one can then write

\begin{equation}\label{eqn:nfcmCh2:460}
\begin{aligned}
a \cdot b &= \inv{2}(a b + b a) \\
a \wedge b &= \inv{2}(a b - b a)
\end{aligned}
\end{equation}

These can therefore be seen to be a consequence of the definitions and axioms rather than a required a-priori definition in their own right.  Establishing
these as derived results is important to avoid confusion when one moves on to general higher grade products.  The vector dot and wedge products are
not sufficient by themselves if taken as a fundamental definition to establish the required results for such higher grade products (in particular the useful
formulas for vector times blade dot and wedge products should be observed to be derived results as opposed to definitions).

\subsection{bivector dot with vector reduction}

In the \(1.3\) solution above the identity

\begin{equation}\label{eqn:nfcmCh2:480}
\begin{aligned}
(a \wedge b) \cdot c &= a (b \cdot c) - (a \cdot c) b \\
\end{aligned}
\end{equation}

was used.  Let us prove this.

\begin{equation}\label{eqn:nfcmCh2:500}
\begin{aligned}
(a \wedge b) \cdot c
&= \gpgradeone{(a \wedge b) c} \\
\implies \\
2 (a \wedge b) \cdot c
&= \gpgradeone{a b c - b a c} \\
&= \gpgradeone{a b c - b (- c a + 2 a \cdot c )} \\
&= \gpgradeone{a b c + b c a} - 2 b (a \cdot c ) \\
&= \gpgradeone{a (b \cdot c + b \wedge c) + (b \cdot c + b \wedge c) a} - 2 b (a \cdot c ) \\
&= 2 a (b \cdot c) + a \cdot (b \wedge c) + (b \wedge c) \cdot a - 2 b (a \cdot c ) \\
\end{aligned}
\end{equation}

To complete the proof we need \(a \cdot B = -B \cdot a\), but once that is demonstrated, one is left with the desired identity after dividing through
by \(2\).

\subsection{vector bivector dot product reversion}

Prove \(a \cdot B = -B \cdot a\).

%
% Copyright � 2012 Peeter Joot.  All Rights Reserved.
% Licenced as described in the file LICENSE under the root directory of this GIT repository.
%

%
%
\chapter{Outermorphism Question}
\index{outermorphism}
\label{chap:outermorphismDet}
%\date{Sept. 2, 2008.  outermorphismDet.tex}

\section{}

\citep{doran2003gap}
has an example of a linear operator.

\begin{equation}\label{eqn:outermorphism_det:F}
F(a) = a + \alpha(a \cdot f_1) f_2.
\end{equation}

This is used to compute the determinant without putting the operator
in matrix form.

\subsection{bivector outermorphism}

Their first step is to compute the wedge of this function applied to two vectors.  Doing this myself (not omitting steps), I get:

\begin{equation}\label{eqn:outermorphismDet:21}
\begin{aligned}
F(a \wedge b)
&= F(a) \wedge F(b) \\
&= (a + \alpha(a \cdot f_1) f_2 ) \wedge (b + \alpha(b \cdot f_1) f_2 ) \\
&= a \wedge b + \alpha(a \cdot f_1) f_2 \wedge b
+ \alpha (b \cdot f_1) a \wedge f_2
+ \alpha^2 (a \cdot f_1) (b \cdot f_1) \mathLabelBox{f_2 \wedge f_2}{\(=0\)} \\
&= a \wedge b
+ \alpha \left( (b \cdot f_1) a - (a \cdot f_1) b \right) \wedge f_2
\\
&= a \wedge b
+ \alpha \left( (a \wedge b ) \cdot f_1 \right) \wedge f_2
\end{aligned}
\end{equation}

This has a very similar form to the original function \(F\).  In particular
one can write

\begin{equation}\label{eqn:outermorphismDet:41}
\begin{aligned}
F(a)
&= a + \alpha(a \cdot f_1) f_2 \\
&= a + \gpgradeone{\alpha(a \cdot f_1) f_2} \\
&= a + \gpgrade{\alpha(a \cdot f_1) f_2}{0+1} \\
&= a + \alpha(a \cdot f_1) \wedge f_2 \\
\end{aligned}
\end{equation}

Here the fundamental definition of the wedge product as the
highest grade part of a product of blades has been used to show that the new
bivector function defined via outermorphism has the same form as the original, once we put the original in the new form that applies to bivector and vector:

\begin{equation}
F(A) = A + \alpha(A \cdot f_1) \wedge f_2
\end{equation}

\subsection{Induction}

Now, proceeding inductively, assuming that this is true for some grade \(k\) blade A, one can calculate \(F(A) \wedge F(b)\) for a vector \(b\):

\begin{equation}\label{eqn:outermorphismDet:61}
\begin{aligned}
&F(A) \wedge F(b) \\
&= (A + \alpha(A \cdot f_1) \wedge f_2) \wedge (b + \alpha(b \cdot f_1) f_2 ) \\
&= A \wedge b
+ \alpha( b \cdot f_1 ) A \wedge f_2
+ \alpha (( A \cdot f_1) \wedge f_2) \wedge b
+ \alpha^2 (b \cdot f_1) ((A \cdot f_1) \wedge f_2) \wedge f_2 \\
&= A \wedge b + \alpha \left( ( b \cdot f_1 ) A - ( A \cdot f_1) \wedge b \right) \wedge f_2 \\
&= A \wedge b + \alpha \gpgrade{ ( b \cdot f_1 ) A - ( A \cdot f_1) b}{k} \wedge f_2 \\
\end{aligned}
\end{equation}

Now, similar to the bivector case, this inner quantity can be reduced, but it is messier to do so:

\begin{equation}\label{eqn:outermorphismDet:81}
\begin{aligned}
\gpgrade{ ( b \cdot f_1 ) A - ( A \cdot f_1) b}{k}
&= \inv{2} \gpgrade{ b f_1 A - A f_1 b + f_1 (b A + (-1)^{k} A b) }{k} \\
\end{aligned}
\end{equation}
\begin{equation} \label{eqn:outermorphism_det:r1}
\implies
\gpgrade{ ( b \cdot f_1 ) A - ( A \cdot f_1) b}{k} = \inv{2} \gpgrade{ b f_1 A - A f_1 b}{k} + \gpgrade{ f_1 (b \wedge A) }{k}
\end{equation}

Consider first the right hand expression:
\begin{equation}\label{eqn:outermorphismDet:101}
\begin{aligned}
\gpgrade{ f_1 (b \wedge A) }{k}
&= f_1 \cdot (b \wedge A) \\
&= (-1)^{k} f_1 \cdot (A \wedge b) \\
&= (-1)^{k} (-1)^k (A \wedge b) \cdot f_1 \\
&= (A \wedge b) \cdot f_1 \\
\end{aligned}
\end{equation}

The right hand expression in \eqnref{eqn:outermorphism_det:r1} can be shown to equal zero.  That is messier still and the calculation can be found
at the end.

Using that equals zero result we now have:
\begin{equation}\label{eqn:outermorphismDet:121}
\begin{aligned}
F(A) \wedge F(b)
&= A \wedge b + \alpha ((A \wedge b) \cdot f_1) \wedge f_2 \\
\end{aligned}
\end{equation}

This completes the induction.

\subsection{Can the induction be avoided?}

Now, GAFP did not do this induction, nor even claim it was required.  The statement is "It follows that", after only calculating the bivector
case.  Is there a reason that they would be able to make such a statement without proof that is obvious to them perhaps but not to me?

It has been pointed out that this question is answered, ``yes, the induction can be avoided'', in \citep{aMacdonaldLAGC} page 148.

%I am guessing this would be related to the matrix concept of rank in
%some way too, but it is not clear to me exactly how.

\section{Appendix. Messy reduction for induction}

Q: Is there an easier way to do this?

Here we want to show that

\begin{equation*}
\inv{2} \gpgrade{ b f_1 A - A f_1 b}{k} = 0
\end{equation*}

Expanding the innards of this expression to group \(A\) and \(b\) parts together:

\begin{equation}\label{eqn:outermorphismDet:141}
\begin{aligned}
b f_1 A - A f_1 b
&= (f_1 b - 2 b \wedge f_1 ) A - A (b f_1 - 2 f_1 \wedge b) \\
&=
f_1 b A - A b f_1
- 2 (b \wedge f_1) A + 2 A (f_1 \wedge b) \\
&=
f_1 (b \cdot A + b \wedge A) - (A \cdot b + A \wedge b) f_1 \\
&- 2 \left( (b \wedge f_1) \cdot A + \gpgrade{(b \wedge f_1) A}{k} + (b \wedge f_1) \wedge A \right) \\
&+ 2 \left( A \cdot (f_1 \wedge b) + \gpgrade{A (f_1 \wedge b)}{k} + A \wedge (f_1 \wedge b) \right)
\end{aligned}
\end{equation}

the grade \(k-2\), and grade \(k+2\) terms of the bivector product
cancel (we are also only interested in the grade-\(k\) parts so can discard them).  This leaves:
\begin{equation*}
f_1 \wedge (b \cdot A) - (A \cdot b) \wedge f_1
+ f_1 \cdot (b \wedge A) - (A \wedge b) \cdot f_1
- 2 \gpgrade{(b \wedge f_1) A}{k}
+ 2 \gpgrade{A (f_1 \wedge b)}{k}
\end{equation*}

The bivector, blade product part of this is the antisymmetric part of that product so those two last terms can be expressed with the
commutator relationship for a bivector with blade: \(\gpgrade{B_2 A}{k} = \inv{2}(B_2 A - A B_2)\):

\begin{equation}\label{eqn:outermorphismDet:161}
\begin{aligned}
2 \gpgrade{A (f_1 \wedge b)}{k}
- 2 \gpgrade{(b \wedge f_1) A}{k}
&= A (f_1 \wedge b) - (f_1 \wedge b) A - (b \wedge f_1) A + A (b \wedge f_1) \\
&= A (f_1 \wedge b) - (f_1 \wedge b) A + (f_1 \wedge b) A - A (f_1 \wedge b) \\
&= 0
\end{aligned}
\end{equation}

So, we now have to show that we have zero for the remainder:
\begin{equation}\label{eqn:outermorphismDet:181}
\begin{aligned}
2 \gpgrade{ b f_1 A - A f_1 b}{k}
&= f_1 \wedge (b \cdot A) - (A \cdot b) \wedge f_1 \\
&\quad + f_1 \cdot (b \wedge A) - (A \wedge b) \cdot f_1 \\
&= (-1)^{k-1}f_1 \wedge (A \cdot b) - (-1)^{k-1}f_1 \wedge (A \cdot b) \\
&\quad + (-1)^{k}f_1 \cdot (A \wedge b) - (-1)^{k} f_1 \cdot (A \wedge b) \\
&= 0
\end{aligned}
\end{equation}

\section{New observation}

Looking again, I think I see one thing that I missed.  The text said they were
constructing the action on a general multivector.  So, perhaps they meant
\(b\) to be a blade.  This is a typesetting subtlety if that is the case.  Let us
assume that is what they meant, and that \(b\) is a grade \(k\) blade.  This
makes the coefficient of the scalar \(\alpha\) in equation 4.147 :

\begin{equation}\label{eqn:outermorphismDet:201}
\begin{aligned}
a \cdot f_1 f_2 \wedge b + b \cdot f_1 a \wedge f_2
&= \left( (b \cdot f_1) a + (-1)^{k} (a \cdot f_1) b \right) \wedge f_2 \\
\end{aligned}
\end{equation}

whereas they have:
\begin{equation*}
\left( (b \cdot f_1) a - (a \cdot f_1) b \right) \wedge f_2
\end{equation*}

So, no, I think they must have intended \(b\) to be a vector, not an
arbitrary grade blade.

Now, indirectly, it has been
proven here that for a vectors \(x\), \(y\), and a grade-\(k\) blade \(B\):

\begin{equation}\label{eqn:outermorphism_det:distrib}
(A \wedge x) \cdot y = A ( x \cdot y ) - ( A \cdot y ) \wedge x
\end{equation}

Or,
\begin{equation}
(A \wedge x) \cdot y = ( y \cdot x ) A + (-1)^{k}( y \cdot A ) \wedge x
\end{equation}

(changed variable names to disassociate this from the specifics of this
particular example), which is a generalization of the wedge product with
dot product distribution identity for vectors:

\begin{equation}
(a \wedge b) \cdot c = a ( b \cdot c ) - ( a \cdot c ) \wedge b
\end{equation}

I believe I have seen a still more general form of \eqnref{eqn:outermorphism_det:distrib}
in a
Hestenes paper, but did not think about using it a-priori.  Regardless, it
does not really appear the the GAFP text was treating \(b\) as anything but a
vector, since there would have to be a \((-1)^k\) factor on equation 4.147 for
it to be general.

\part{Projection.}
%
% Copyright � 2012 Peeter Joot.  All Rights Reserved.
% Licenced as described in the file LICENSE under the root directory of this GIT repository.
%

%
%
\chapter{Reciprocal Frame Vectors}
\index{reciprocal frame}
\label{chap:reciprocalFrame}
%\date{March 29, 2008.  reciprocalFrame.tex}

\section{Approach without Geometric Algebra}

Without employing geometric algebra, one can use the projection
operation expressed as a dot product and calculate the a vector
orthogonal to a set of other vectors, in the direction of a reference
vector.

Such a calculation also yields \R{N} results in terms of determinants, and as a side
effect produces equations for
parallelogram area, parallelepiped volume and higher dimensional analogues as a side effect
(without having to employ change of basis diagonalization arguments that do not work well
for higher dimensional subspaces).

\subsection{Orthogonal to one vector}

The simplest case is the vector perpendicular to another.  In anything
but \R{2} there are a whole set of such vectors, so to express this as a
non-set result a reference vector is required.

Calculation of the coordinate vector for this case follows directly from
the dot product.  Borrowing the GA term, we subtract the projection
to calculate the rejection.

\begin{equation}\label{eqn:reciprocalFrame:363}
\begin{aligned}
\Rej{\ucap}{\Bv}
&= \Bv - \Bv \cdot \ucap \ucap \\
&= \inv{\Bu^2}(\Bv\Bu^2 - \Bv \cdot \Bu \Bu) \\
&= \inv{\Bu^2}\sum{v_i\Be_i u_j u_j - v_j u_j u_i \Be_i} \\
&= \inv{\Bu^2}\sum{u_j\Be_i\DETuvij{v}{u}{i}{j}} \\
&= \inv{\Bu^2}\sum_{i<j}{(u_i \Be_j -u_j\Be_i)\DETuvij{u}{v}{i}{j}} \\
\end{aligned}
\end{equation}

Thus we can write the rejection of \(\Bv\) from \(\ucap\) as:

\begin{equation}\label{eqn:reciprocal_frame:rejonevector}
\Rej{\ucap}{\Bv} = \inv{\Bu^2}\sum_{i<j}\DETuvij{u}{v}{i}{j}\DETuvij{u}{\Be}{i}{j}
\end{equation}

Or introducing some shorthand:

\begin{equation}\label{eqn:reciprocalFrame:383}
\begin{aligned}
D_{ij}^{\Bu \Bv} &= \DETuvij{u}{v}{i}{j} \\
D_{ij}^{\Bu \Be} &= \DETuvij{u}{\Be}{i}{j} \\
\end{aligned}
\end{equation}

\eqnref{eqn:reciprocal_frame:rejonevector} can be expressed in a form that will be slightly more convenient for larger sets of
vectors:

\begin{equation}\label{eqn:reciprocal_frame:rejonevectorD}
\Rej{\ucap}{\Bv} = \inv{\Bu^2}\sum_{i<j} D_{ij}^{\Bu \Bv} D_{ij}^{\Bu \Be}
\end{equation}

Note that although the GA axiom \(\Bu^2 = \Bu \cdot \Bu\) has been used
in equations \eqnref{eqn:reciprocal_frame:rejonevector} and \eqnref{eqn:reciprocal_frame:rejonevectorD} above and the derivation, that was
not necessary to prove them.
This can, for now, be thought of as a notational convenience, to avoid having to write \(\Bu \cdot \Bu\), or
\(\norm{\Bu}^2\).

This result can be used to express the \R{N} area of a parallelogram since we just have to multiply the length
of \(\Rej{\ucap}{\Bv}\):

\begin{equation}\label{eqn:reciprocalFrame:23}
\norm{\Rej{\ucap}{\Bv}}^2 =
\Rej{\ucap}{\Bv} \cdot \Bv =
\inv{\Bu^2}\sum_{i<j} {\left(D_{ij}^{\Bu \Bv}\right)}^2
\end{equation}

with the length of the base \(\norm{\Bu}\). [FIXME: insert figure.]

Thus the area (squared) is:

\begin{equation}\label{eqn:reciprocal_frame:parallogramarea}
\AreaOp{\Bu,\Bv}^2 = \sum_{i<j} {\left(D_{ij}^{\Bu \Bv}\right)}^2
\end{equation}

For the special case of a vector in \R{2} this is
\begin{equation}\label{eqn:reciprocal_frame:parallogramarear2}
\AreaOp{\Bu,\Bv} = \abs{D_{12}^{\Bu \Bv}} = \AbsName\left(\DETuvij{u}{v}{i}{j}\right)
\end{equation}

\subsection{Vector orthogonal to two vectors in direction of a third}

The same procedure can be followed for three vectors, but the algebra gets messier.  Given three vectors \(\Bu\), \(\Bv\), and \(\Bw\)
we can calculate the component \(\Bw'\) of \(\Bw\) perpendicular to \(\Bu\) and \(\Bv\).  That is:

\begin{equation}\label{eqn:reciprocalFrame:403}
\begin{aligned}
\Bv' &= \Bv - \Bv \cdot \ucap \ucap \\
\implies & \\
\Bw' &= \Bw - \Bw \cdot \ucap \ucap - \Bw \cdot \hat{\Bv'} \hat{\Bv'}
\end{aligned}
\end{equation}

After expanding this out, a number of the terms magically cancel out and one is left with

\begin{equation}\label{eqn:reciprocalFrame:423}
\begin{aligned}
\Bw'' = \Bw' (\Bu^2\Bv^2 - (\Bu \cdot \Bv)^2)
&= \Bu \left(-\Bu \cdot \Bw \Bv^2 + (\Bu \cdot \Bv)(\Bv \cdot \Bw)\right)  \\
&+ \Bv \left(-\Bu^2(\Bv \cdot \Bw) - (\Bu \cdot \Bv)(\Bu \cdot \Bw)\right)  \\
&+ \Bw \left(\Bu^2\Bv^2 - (\Bu \cdot \Bv)^2\right) \\
\end{aligned}
\end{equation}

And this in turn can be expanded in terms of coordinates and the results collected yielding

\begin{equation}\label{eqn:reciprocalFrame:443}
\begin{aligned}
\Bw'' &= \sum \Be_i u_j v_k \left(
u_i \DETuvij{v}{w}{j}{k}
-v_i \DETuvij{u}{w}{j}{k}
w_i \DETuvij{u}{v}{j}{k}
\right) \\
&= \sum \Be_i u_j v_k \DETuvwijk{u}{v}{w}{i}{j}{k} \\
&= \sum_{i,j<k} \Be_i \DETuvij{u}{v}{j}{k} \DETuvwijk{u}{v}{w}{i}{j}{k} \\
&=
\left(\sum_{i<j<k} + \sum_{j<i<k} + \sum_{j<k<i} \right) \Be_i \DETuvij{u}{v}{j}{k} \DETuvwijk{u}{v}{w}{i}{j}{k}.
\end{aligned}
\end{equation}

Expanding the sum of the denominator in terms of coordinates:
\begin{equation}\label{eqn:reciprocalFrame:43}
\Bu^2\Bv^2 - (\Bu \cdot \Bv)^2 = \sum_{i<j} \DETuvij{u}{v}{i}{j}^2
\end{equation}

and using a change of summation indices, our final result for the vector perpendicular to two others in the direction of a third is:

\begin{equation}\label{eqn:reciprocal_frame:orthotwovectors}
\Rej{\ucap,\vcap}{\Bw} =
\frac{\sum_{i<j<k} \DETuvwijk{u}{v}{w}{i}{j}{k} \DETuvwijk{u}{v}{\Be}{i}{j}{k}}
{\sum_{i<j} \DETuvij{u}{v}{i}{j}^2}
\end{equation}

As a small aside, it is notable here to observe that
\(\Span\left\{\DETuvij{u}{\Be}{i}{j}\right\}\) is the null space for the vector \(\Bu\), and
the set \(\Span\left\{\DETuvwijk{u}{v}{\Be}{i}{j}{k}\right\}\) is the null space for the two vectors \(\Bu\) and \(\Bv\) respectively.

Since the rejection is a normal to the set of vectors it must necessarily include these cross product like determinant terms.

As in \eqnref{eqn:reciprocal_frame:rejonevectorD}, use of a \(D_{ijk}^{\Bu\Bv\Bw}\) notation allows for a more compact
result:

\begin{equation}\label{eqn:reciprocal_frame:rejtwovectorsD}
\Rej{\ucap\vcap}{\Bw} =
{\left(\sum_{i<j} \left(D_{ij}^{\Bu\Bv}\right)^2\right)}^{-1}
\sum_{i<j<k} D_{ijk}^{\Bu\Bv\Bw} D_{ijk}^{\Bu\Bv\Be}
\end{equation}

And, as before this yields the Volume of the parallelepiped by multiplying perpendicular height:

\begin{equation}\label{eqn:reciprocalFrame:63}
\norm{\Rej{\ucap\vcap}{\Bw}} =
\Rej{\ucap\vcap}{\Bw} \cdot \Bw =
{\left(\sum_{i<j} \left(D_{ij}^{\Bu\Bv}\right)^2\right)}^{-1}
\sum_{i<j<k} \left(D_{ijk}^{\Bu\Bv\Bw} \right)^2
\end{equation}

by the base area.

Thus the squared volume of a parallelepiped spanned by the three vectors is:

\begin{equation}\label{eqn:reciprocal_frame:parallopipedvolume}
\VolumeOp{\Bu,\Bv,\Bw}^2 = \sum_{i<j<k} {\left(D_{ijk}^{\Bu \Bv \Bw}\right)}^2.
\end{equation}

The simplest case is for \R{3} where we have only one summand:

\begin{equation}\label{eqn:reciprocal_frame:parallopipedvolumer3}
\VolumeOp{\Bu,\Bv,\Bw}
= \abs{D_{ijk}^{\Bu \Bv \Bw}}
= \AbsName\left(
\DETuvwijk{u}{v}{w}{1}{2}{3}
\right).
\end{equation}

\subsection{Generalization.  Inductive Hypothesis}

There are two things to prove

\begin{enumerate}
\item hypervolume of parallelepiped spanned by vectors \(\Bu_1, \Bu_2, \dots, \Bu_k\)

\begin{equation} \label{eqn:reciprocal_frame:hypervolume}
\VolumeOp{\Bu_1, \Bu_2, \cdots, \Bu_k}^2
=
\sum_{i_1 < i_2 < \cdots < i_k } \left(
D_{i_1 i_2 \cdots i_k}^{\Bu_{i_1} \Bu_{i_2} \cdots \Bu_{i_k}}
\right)^2
\end{equation}

\item Orthogonal rejection of a set of vectors in direction of another.

\begin{equation} \label{eqn:reciprocal_frame:hyperrejection}
\Rej{\ucap_1\cdots\ucap_{k-1}}{\Bu_k} =
\frac{
\sum_{i_1 < \cdots < i_{k} }
D_{i_1 \cdots i_{k}}^{\Bu_{i_1} \cdots \Bu_{i_{k}}}
D_{i_1 \cdots i_{k}}^{\Bu_{i_1} \cdots \Bu_{i_{k-1}} \Be }}
{
\sum_{i_1 < \cdots < i_{k-1} } \left(D_{i_1 \cdots i_{k-1}}^{\Bu_{i_1} \cdots \Bu_{i_{k-1}}}\right)^2
}
\end{equation}
\end{enumerate}

I cannot recall if I ever did the inductive proof for this.
Proving for the initial case is done (since it is proved for both the
two and three vector cases).  For the limiting case where \(k=n\) it can be observed that this is normal to all the others, so the
only thing to prove for that case is if the scaling provided by hypervolume \eqnref{eqn:reciprocal_frame:hypervolume} is correct.

\subsection{Scaling required for reciprocal frame vector}

Presuming an inductive proof of the general result of \eqnref{eqn:reciprocal_frame:hyperrejection} is possible, this rejection
has the property

\begin{equation*}
\Rej{\ucap_1\cdots\ucap_{k-1}}{\Bu_k} \cdot \Bu_i \propto \delta_{ki}
\end{equation*}

With the scaling factor picked so that this equals \(\delta_{ki}\), the resulting ``reciprocal frame vector'' is

\begin{equation} \label{eqn:reciprocal_frame:framevec}
\Bu^k =
\frac{
\sum_{i_1 < \cdots < i_{k} }
D_{i_1 \cdots i_{k}}^{\Bu_{i_1} \cdots \Bu_{i_{k}}}
D_{i_1 \cdots i_{k}}^{\Bu_{i_1} \cdots \Bu_{i_{k-1}} \Be }}
{
\sum_{i_1 < \cdots < i_{k} } \left(D_{i_1 \cdots i_{k}}^{\Bu_{i_1} \cdots \Bu_{i_{k}}}\right)^2
}
\end{equation}

The superscript notation is borrowed from Doran/Lasenby, and denotes not a vector raised to a power, but this
this special vector satisfying the following orthogonality and scaling criteria:

\begin{equation}\label{eqn:reciprocal_frame:reciportho}
\Bu^k \cdot \Bu_i = \delta_{ki}.
\end{equation}

Note that for \(k=n-1\), \eqnref{eqn:reciprocal_frame:framevec} reduces to

\begin{equation} \label{eqn:reciprocal_frame:framevecnminus}
\Bu^n =
\frac{ D_{1 \cdots (n-1)}^{\Bu_1 \cdots \Bu_{n-1} \Be} } { D_{1 \cdots n}^{\Bu_1 \cdots \Bu_n} }.
\end{equation}

This or some other scaled version of this is likely as close as we can come to generalizing the cross product
as an operation that takes vectors to vectors.

\subsection{Example.  \texorpdfstring{\R{3}}{3D} case.  Perpendicular to two vectors}

Observe that for \R{3}, writing \(\Bu = \Bu_1\), \(\Bv = \Bu_2\), \(\Bw = \Bu_3\), and \(\Bw' = {\Bu_3}^3\) this is:

\begin{equation}
\Bw' =
\frac{\DETuvwijk{u}{v}{\Be}{1}{2}{3}}{\DETuvwijk{u}{v}{w}{1}{2}{3}}
=
\frac{\Bu \cross \Bv}{(\Bu \cross \Bv) \cdot \Bw}
\end{equation}

This is the cross product scaled by the (signed) volume for the parallelepiped spanned by the three vectors.

\section{Derivation with GA}

Regression with respect to a set of vectors can be expressed directly.  For vectors \({\Bu_i}\) write \(\BB = \Bu_1 \wedge \Bu_2 \cdots \Bu_k\).  Then for any vector we have:

\begin{equation}\label{eqn:reciprocalFrame:463}
\begin{aligned}
\Bx
&= \Bx \BB \inv{\BB}  \\
&= \gpgradeone{ \Bx \BB \inv{\BB} } \\
&= \gpgradeone{ (\Bx \cdot \BB + \Bx \wedge \BB) \inv{\BB} }
\end{aligned}
\end{equation}

All the grade three and grade five terms are selected out by the grade one operation, leaving just

\begin{equation}
\Bx = (\Bx \cdot \BB) \cdot \inv{\BB} + (\Bx \wedge \BB) \cdot \inv{\BB}.
\end{equation}

This last term is the rejective component.

\begin{equation}\label{eqn:reciprocal_frame:bladerejection}
\Rej{\BB}{\Bx} =
(\Bx \wedge \BB) \cdot \inv{\BB}
=
\frac{
(\Bx \wedge \BB) \cdot {\BB}^\dagger
}
{
\BB \BB^\dagger
}
\end{equation}

Here we see in the denominator the squared sum of determinants in the denominator of \eqnref{eqn:reciprocal_frame:hyperrejection}:

\begin{equation}\label{eqn:reciprocalFrame:83}
\BB \BB^\dagger =
\sum_{i_1 < \cdots < i_{k} } \left(D_{i_1 \cdots i_{k}}^{\Bu_{i_1} \cdots \Bu_{i_{k}}}\right)^2
\end{equation}

In the numerator we have the dot product of two wedge products, each expressible as sums of determinants:

\begin{equation}\label{eqn:reciprocalFrame:103}
\BB^\dagger = (-1)^{k(k-1)/2}
\sum_{i_1 < \cdots < i_{k} }
D_{i_1 \cdots i_{k}}^{\Bu_{i_1} \cdots \Bu_{i_{k}}} \Be_{i_1} \Be_{i_2} \cdots \Be_{i_{k}}
\end{equation}

And
\begin{equation}\label{eqn:reciprocalFrame:123}
\Bx \wedge \BB =
\sum_{i_1 < \cdots < i_{k+1} }
D_{i_1 \cdots i_{k+1}}^{\Bx \Bu_{i_1} \cdots \Bu_{i_{k}}} \Be_{i_1} \Be_{i_2} \cdots \Be_{i_{k+1}}
\end{equation}

Dotting these is all the grade one components of the product.
Performing that calculation would likely provide an explicit confirmation of the inductive hypothesis of
\eqnref{eqn:reciprocal_frame:hyperrejection}.  This can be observed directly for the \(k+1=n\) case.  That product produces a Laplace
expansion sum.

\begin{equation}\label{eqn:reciprocalFrame:483}
\begin{aligned}
(\Bx \wedge \BB) \cdot \BB^\dagger
&=
%((-1)^{k(k-1)/2})^2
D_{1 2 \cdots n}^{\Bx \Bu_{1} \cdots \Bu_{n-1}}
\left(
\Be_{1} D_{2 3 4 \cdots n}^{\Bu_{1} \cdots \Bu_{n-1}}
-\Be_{2} D_{1 3 4 \cdots n}^{\Bu_{1} \cdots \Bu_{n-1}}
+\Be_{3} D_{1 2 4 \cdots n}^{\Bu_{1} \cdots \Bu_{n-1}}
\right)
\end{aligned}
\end{equation}

\begin{equation}\label{eqn:reciprocal_frame:hyperrejectionganminus}
(\Bx \wedge \BB) \cdot \inv{\BB}
=
\frac{
D_{1 2 \cdots n}^{\Bx \Bu_{1} \cdots \Bu_{n-1}}
D_{1 2 \cdots n}^{\Be \Bu_{1} \cdots \Bu_{n-1}}
}
{
\sum_{i_1 < \cdots < i_{k} } \left(D_{i_1 \cdots i_{k}}^{\Bu_{i_1} \cdots \Bu_{i_{k}}}\right)^2
}
\end{equation}

Thus \eqnref{eqn:reciprocal_frame:hyperrejection} for the \(k = n-1\) case is proved without induction.  A proof for the \(k+1<n\) case would be harder.
No proof is required if one picks the set of basis vectors \({\Be_i}\) such that \(\Be_i \wedge \BB = 0\) (then the \(k+1=n\) result applies).
I believe that proves the general case too if one observes that a rotation to any other basis in the span of the set of vectors only
changes the sign of the each of the determinants, and the product of the two sign changes will then have value one.

Follow through of the details for a proof of original non GA induction hypothesis is probably not worthwhile since this
reciprocal frame vector problem can be
tackled with a different approach using a subspace pseudovector.

It is notable that although this had no induction in the argument above, note that it is fundamentally required.
That is because there is an inductive proof
required to prove that the general wedge and dot product vector formulas:

\begin{equation}\label{eqn:reciprocalFrame:143}
\Bx \cdot \BB = \inv{2}(\Bx \BB - (-1)^k\BB \Bx)
\end{equation}
\begin{equation}\label{eqn:reciprocalFrame:163}
\Bx \wedge \BB = \inv{2}(\Bx \BB + (-1)^k\BB \Bx)
\end{equation}

from the GA axioms (that is an easier proof without the mass of indices and determinant products.)

\section{Pseudovector from rejection}

As noted in the previous section the reciprocal frame vector \(\Bu^k\) is the vector in the direction of \(\Bu_k\) that has no component
in \(\Span{ \Bu_1, \cdots, \Bu_{k-1}}\), normalized such that \(\Bu_k \cdot \Bu^k = 1\).  Explicitly, with
\(\BB = \Bu_1 \wedge \Bu_2 \cdots \wedge \Bu_{k-1}\) this is:

\begin{equation}\label{eqn:reciprocal_frame:reciprej}
\Bu^k =
\frac
{
(\Bu_k \wedge \BB) \cdot \BB
}
{
\Bu_k \cdot \left((\Bu_k \wedge \BB) \cdot \BB\right)
}
\end{equation}

This is derived from \eqnref{eqn:reciprocal_frame:bladerejection}, after noting that
\(\frac{\BB^\dagger}{\BB\BB^\dagger} \propto \BB\), and further
scaling to produce the desired orthonormal property of equation
\eqnref{eqn:reciprocal_frame:reciportho}
that defines the reciprocal frame vector.

\subsection{back to reciprocal result}

Now,
\eqnref{eqn:reciprocal_frame:reciprej}
looks considerably different from the Doran/Lasenby result.
Reduction to a direct pseudovector/blade product is possible since the
dot product here can be converted to a direct product.

\begin{equation}\label{eqn:reciprocalFrame:503}
\begin{aligned}
(\Bu_k \wedge \BB) \cdot \BB
&=
\mathLabelBox
[
   labelstyle={xshift=2cm},
   linestyle={out=270,in=90, latex-}
]
{(\Bx \BB)}{\(\Bx = \Bu_k - (\Bu_k \cdot \BB) \cdot \inv{\BB}\)}
\cdot \BB \\
&= \gpgradeone{
\Bx\BB \BB
} \\
&= \Bx \BB^2 \\
&= \left(\left(\Bu_k - (\Bu_k \cdot \BB) \cdot \inv{\BB}\right) \wedge \BB\right) \BB \\
&= (\Bu_k \wedge \BB) \BB \\
\end{aligned}
\end{equation}

Thus \eqnref{eqn:reciprocal_frame:reciprej} is a scaled pseudovector for the subspace
defined by \(\Span {\Bu_i}\), multiplied by a k-1 blade.

\section{Components of a vector}

The delta property of
\eqnref{eqn:reciprocal_frame:reciportho} allows one to use the reciprocal frame
vectors and the basis that generated them to calculate the coordinates
of the a vector with respect to this (not necessarily orthonormal) basis.

That is a pretty powerful result, but somewhat obscured by the Doran/Lasenby
super/sub script notation.

Suppose one writes a vector in \(\Span{\Bu_i}\) in terms of unknown coefficients

\begin{equation}\label{eqn:reciprocalFrame:183}
\Ba = \sum a_i \Bu_i
\end{equation}

Dotting with \(\Bu^j\) gives:

\begin{equation}\label{eqn:reciprocalFrame:203}
\Ba \cdot \Bu^j
= \sum a_i \Bu_i \cdot \Bu^j
= \sum a_i \delta_{ij}
= a_j
\end{equation}

Thus
\begin{equation}\label{eqn:reciprocal_frame:nonrecipcomponents}
\Ba = \sum (\Ba \cdot \Bu^i) \Bu_i
\end{equation}

Similarly, writing this vectors in terms of \(\Bu^i\) we have

\begin{equation}\label{eqn:reciprocalFrame:223}
\Ba = \sum b_i \Bu^i
\end{equation}

Dotting with \(\Bu_j\) gives:

\begin{equation}\label{eqn:reciprocalFrame:243}
\Ba \cdot \Bu_j
= \sum b_i \Bu^i \cdot \Bu_j
= \sum b_i \delta_{ij}
= b_j
\end{equation}

Thus
\begin{equation}\label{eqn:reciprocal_frame:recipcomponents}
\Ba = \sum (\Ba \cdot \Bu_i) \Bu^i
\end{equation}

We are used to seeing the equation for components of a vector in terms of a
basis in the following form:

\begin{equation}
\Ba = \sum (\Ba \cdot \Bu_i) \Bu_i
\end{equation}

This is true only when the basis vectors are orthonormal.
Equations
\eqnref{eqn:reciprocal_frame:nonrecipcomponents} and \eqnref{eqn:reciprocal_frame:recipcomponents} provide the
general decomposition of a vector in terms of a general linearly independent
set.

\subsection{Reciprocal frame vectors by solving coordinate equation}

A more natural way to these results are to take repeated wedge products.
Given a vector decomposition in terms of a basis \({\Bu_i}\), we want to solve for \(a_i\):

\begin{equation}\label{eqn:reciprocalFrame:263}
\Ba = \sum_{i=1}^k a_i \Bu_i
\end{equation}

The solution, from the wedge is:

\begin{equation}\label{eqn:reciprocalFrame:283}
\Ba \wedge (\Bu_1 \wedge \Bu_2 \cdots \check{\Bu_i} \cdots \wedge \Bu_k  = a_i (-1)^{i-1} \Bu_1 \wedge \cdots \wedge \Bu_k
\end{equation}
\begin{equation}\label{eqn:reciprocalFrame:303}
\implies
a_i =
(-1)^{i-1}
\frac{\Ba \wedge (\Bu_1 \wedge \Bu_2 \cdots \check{\Bu_i} \cdots \wedge \Bu_k}{ \Bu_1 \wedge \cdots \wedge \Bu_k })
\end{equation}

The complete vector in terms of components is thus:

\begin{equation}\label{eqn:reciprocal_frame:coordinateswedge}
\Ba =
\sum(-1)^{i-1}
\frac{\Ba \wedge \left(\Bu_1 \wedge \Bu_2 \cdots \check{\Bu_i} \cdots \wedge \Bu_k\right)}{ \Bu_1 \wedge \cdots \wedge \Bu_k } \Bu_i
\end{equation}


We are used to seeing the coordinates expressed in terms of dot products instead of wedge products.  As in \R{3} where
the pseudovector allows wedge products to be expressed in terms of the dot product we can do the same for the general case.

Writing \(\BB \in \bigwedge^{k-1}\) and \(\BI \in \bigwedge^k\) we want to reduce an equation of the following form

\begin{equation}\label{eqn:reciprocal_frame:wedgereduction}
\frac{\Ba \wedge \BB}{\BI} = \inv{\BI} \frac{\Ba\BB + (-1)^{k-1}\BB\Ba}{2}
\end{equation}

The pseudovector either commutes or anticommutes with a vector in the subspace depending on the grade

\begin{equation}\label{eqn:reciprocalFrame:523}
\begin{aligned}
\BI \Ba
&= \BI \cdot \Ba + \mathLabelBox{\BI \wedge \Ba}{\(=0\)} \\
&= (-1)^{k-1}\Ba \cdot \BI \\
&= (-1)^{k-1}\Ba \BI \\
\end{aligned}
\end{equation}

Substituting back into \eqnref{eqn:reciprocal_frame:wedgereduction} we have

\begin{equation}\label{eqn:reciprocalFrame:543}
\begin{aligned}
\frac{\Ba \wedge \BB}{\BI}
&= (-1)^{k-1} \frac{\Ba \left(\inv{\BI}\BB\right) + \left(\inv{\BI}\BB\right)\Ba}{2} \\
&= (-1)^{k-1} \Ba \cdot \left(\inv{\BI}\BB\right) \\
&= \Ba \cdot \left(\BB \inv{\BI}\right) \\
\end{aligned}
\end{equation}

With \(\BI = \Bu_1 \wedge \cdots \Bu_k\), and \(\BB = \Bu_1 \wedge \Bu_2 \cdots \check{\Bu_i} \cdots \wedge \Bu_k\),
back substitution back into \eqnref{eqn:reciprocal_frame:coordinateswedge} is thus

\begin{equation}\label{eqn:reciprocalFrame:563}
\begin{aligned}
\Ba
%&= \sum(-1)^{i-1} (-1)^{k-1} \Ba \cdot \left(\inv{\BI}\BB\right) \Bu_i \\
&= \sum
\Ba \cdot \left(
(-1)^{i-1}
\BB \inv{\BI}\right) \Bu_i
\end{aligned}
\end{equation}

The final result yields the reciprocal frame vector \(\Bu^k\), and we see how to arrive at this result naturally attempting
to answer the question of how to find the coordinates of a vector with respect to a (not necessarily orthonormal) basis.

\begin{equation}\label{eqn:reciprocal_frame:recipdot}
\Ba =
\sum
\Ba \cdot
\mathLabelBox{
\left((\Bu_1 \wedge \Bu_2 \cdots \check{\Bu_i} \cdots \wedge \Bu_k) \frac{ (-1)^{i-1} }{ \Bu_1 \wedge \cdots \wedge \Bu_k }\right)
}{\(\Bu^k\)}
\Bu_i
\end{equation}

\section{Components of a bivector}

To find the coordinates of a bivector with respect to an arbitrary basis we have a similar problem.
For a vector basis \({\Ba_i}\), introduce a bivector basis \({\Ba_i \wedge \Ba_j}\), and write

\begin{equation}\label{eqn:reciprocal_frame:bivectorcoord}
\BB = \sum_{u<v} b_{uv} \Ba_u \wedge \Ba_v
\end{equation}

Wedging with \(\Ba_i \wedge \Ba_j\) will select all but the \(ij\) component.  Specifically

\begin{equation}\label{eqn:reciprocalFrame:583}
\begin{aligned}
\BB \wedge
(\Ba_1 \wedge \cdots \check{\Ba_i} \cdots \check{\Ba_j} \cdots \wedge \Ba_k)
&= b_{ij} \Ba_i \wedge \Ba_j \wedge (\Ba_1 \wedge \cdots \check{\Ba_i} \cdots \check{\Ba_j} \cdots \wedge \Ba_k)  \\
&= b_{ij} (-1)^{j-2 + i-1}(\Ba_1 \wedge \cdots \wedge \Ba_k) \\
\end{aligned}
\end{equation}

Thus
\begin{equation}\label{eqn:reciprocal_frame:bivectorrecip}
b_{ij} = (-1)^{i+j-3}\BB \wedge
\frac{ (\Ba_1 \wedge \cdots \check{\Ba_i} \cdots \check{\Ba_j} \cdots \wedge \Ba_k) }
{\Ba_1 \wedge \cdots \wedge \Ba_k}
\end{equation}

We want to put this in dot product form like \eqnref{eqn:reciprocal_frame:recipdot}.  To do so we need a generalized grade reduction formula

\begin{equation}\label{eqn:reciprocal_frame:gradereduction}
(\BA_a \wedge \BA_b) \cdot \BA_c = \BA_a \cdot (\BA_b \cdot \BA_c)
\end{equation}

This holds when \(a + b \le c\).  Writing
\(\BA = \Ba_1 \wedge \cdots \check{\Ba_i} \cdots \check{\Ba_j} \cdots \wedge \Ba_k\), and
\(\BI = \Ba_1 \wedge \cdots \wedge \Ba_k\), we have

\begin{equation}\label{eqn:reciprocalFrame:603}
\begin{aligned}
(\BB \wedge \BA) \inv{\BI}
&= (\BB \wedge \BA) \cdot \inv{\BI} \\
&= \BB \cdot \left( \BA \cdot \inv{\BI} \right) \\
&= \BB \cdot \left( \BA \inv{\BI} \right) \\
\end{aligned}
\end{equation}

Thus the bivector in terms of its coordinates for this basis is:

\begin{equation}\label{eqn:reciprocal_frame:bivectordecomp}
\sum_{u<v}
\BB \cdot
\left(
(\Ba_1 \wedge \cdots \check{\Ba_u} \cdots \check{\Ba_v} \cdots \wedge \Ba_k)
\frac{(-1)^{u+v-2-1}}
{\Ba_1 \wedge \cdots \wedge \Ba_k}
\right)
\Ba_u \wedge \Ba_v
\end{equation}

It is easy to see how this generalizes to higher order blades since
\eqnref{eqn:reciprocal_frame:gradereduction} is good for all required grades.  In all cases, the form is going to be the same, with only differences
in sign and the number of omitted vectors in the \(\BA\) blade.

For example for a trivector

\begin{equation}\label{eqn:reciprocalFrame:323}
\BT = \sum_{u<v<w}t_{uvw} \Ba_u \wedge \Ba_v \wedge \Ba_w
\end{equation}

It is pretty straightforward to show that this can be decomposed as follows

\begin{equation}\label{eqn:reciprocal_frame:trivectordecomp}
\BT = \sum_{u<v<w} \BT \cdot
\left(
(\Ba_1 \wedge \cdots \check{\Ba_u} \cdots \check{\Ba_v} \cdots \check{\Ba_w} \cdots \wedge \Ba_k)
\frac{(-1)^{u+v+w-3-2-1}}
{\Ba_1 \wedge \cdots \wedge \Ba_k}
\right)
\Ba_u \wedge \Ba_v \wedge \Ba_w
\end{equation}

\subsection{Compare to GAFP}

Doran/Lasenby's GAFP
demonstrates \eqnref{eqn:reciprocal_frame:recipdot}, and with some incomprehensible steps skips to a generalized
result of the form
\footnote{ In retrospect I do not think that the in between steps had anything to do with logical sequence.  The authors wanted some of the results for subsequent stuff (like: rotor recovery) and sandwiched it between the vector and reciprocal frame multivector results somewhat out of sequence.}

\begin{equation}\label{eqn:reciprocal_frame:bivectordecompwithrecipbivector}
\BB = \sum_{i<j} \BB \cdot \left(\Ba^j \wedge \Ba^i\right) \Ba_i \wedge \Ba_j
\end{equation}

GAFP states this for general multivectors instead of bivectors, but the idea is the same.

This makes intuitive sense based on the very similar vector result.  This does not show that
the generalized reciprocal frame k-vectors calculated in
\eqnref{eqn:reciprocal_frame:bivectordecomp} or \eqnref{eqn:reciprocal_frame:trivectordecomp} can be produced simply
by wedging the corresponding individual reciprocal frame vectors.

To show that either takes algebraic identities that I do not know, or am not thinking of as applicable.
Alternately perhaps it would just take simple brute force.

Easier is to demonstrate the validity of the final result directly.  Then assuming my direct calculations
are correct implicitly demonstrates equivalence.

Starting with \(\BB\) as defined in \eqnref{eqn:reciprocal_frame:bivectorcoord}, take dot products with
\(\Ba^j \wedge \Ba^i\).

\begin{equation}\label{eqn:reciprocalFrame:623}
\begin{aligned}
\BB \cdot (\Ba^j \wedge \Ba^i)
 &= \sum_{u<v} b_{uv} (\Ba_u \wedge \Ba_v) \cdot (\Ba^j \wedge \Ba^i) \\
 &= \sum_{u<v} b_{uv}
\begin{vmatrix}
\Ba_u \cdot \Ba^i & \Ba_u \cdot \Ba^j \\
\Ba_v \cdot \Ba^i & \Ba_v \cdot \Ba^j \\
\end{vmatrix} \\
 &= \sum_{u<v} b_{uv}
\begin{vmatrix}
\delta_{ui} & \delta_{uj} \\
\delta_{vi} & \delta_{vj} \\
\end{vmatrix} \\
\end{aligned}
\end{equation}

Consider this determinant when \(u=i\) for example
\begin{equation}\label{eqn:reciprocalFrame:343}
\begin{vmatrix}
\delta_{ui} & \delta_{uj} \\
\delta_{vi} & \delta_{vj} \\
\end{vmatrix}
=
\begin{vmatrix}
1 & \delta_{ij} \\
\delta_{vi} & \delta_{vj} \\
\end{vmatrix}
=
\begin{vmatrix}
1 & 0 \\
\delta_{vi} & \delta_{vj} \\
\end{vmatrix}
= \delta_{vj}
\end{equation}

If any one index is common, then both must be common (\(ij=uv\)) for this determinant to have a non-zero (ie: one) value.  On the other hand, if no index is common then all the \(\delta\)'s are zero.

Like
\eqnref{eqn:reciprocal_frame:reciportho}
this demonstrates an orthonormal selection behavior like the reciprocal frame vector.  It has the action:

\begin{equation}
(\Ba_i \wedge \Ba_j) \cdot (\Ba^v \wedge \Ba^u) = \delta_{ij,uv}
\end{equation}

This means that we can write \(b_{uv}\) directly in terms of a bivector dot product

\begin{equation}\label{eqn:reciprocalFrame:643}
\begin{aligned}
b_{uv} = \BB \cdot (\Ba^v \wedge \Ba^u)
\end{aligned}
\end{equation}

and thus proves \eqnref{eqn:reciprocal_frame:bivectordecompwithrecipbivector}.  Proof of the general result
also follows from the determinant expansion of the respective blade dot products.

\subsection{Direct expansion of bivector in terms of reciprocal frame vectors}

Looking at linear operators I realized that the result for bivectors above can follow more easily from direct expansion of a bivector written in terms of vector factors:

\begin{equation}\label{eqn:reciprocalFrame:663}
\begin{aligned}
\Ba \wedge \Bb
&= \sum (\Ba \cdot \Bu_i \Bu^i) \wedge (\Bb \cdot \Bu_j \Bu^j) \\
&= \sum_{i<j} \left(\Ba \cdot \Bu_i \Bb \cdot \Bu_j - \Ba \cdot \Bu_j \Bb \cdot \Bu_i \right) \Bu^i \wedge \Bu^j \\
&= \sum_{i<j}
\begin{vmatrix}
\Ba \cdot \Bu_i  & \Ba \cdot \Bu_j  \\
\Bb \cdot \Bu_i & \Bb \cdot \Bu_j  \\
\end{vmatrix}
\Bu^i \wedge \Bu^j \\
\end{aligned}
\end{equation}

When the set of vectors \(\Bu_i = \Bu^i\) are orthonormal we have already
calculated this result when looking at the wedge product in a differential
forms context:

\begin{equation}
\Ba \wedge \Bb = \sum_{i<j} \DETuvij{a}{b}{i}{j} \Bu_i \wedge \Bu_j
\end{equation}

For this general case for possibly non-orthonormal frames, this
determinant of dot products can be recognized as the dot product of two blades

\begin{equation}\label{eqn:reciprocalFrame:683}
\begin{aligned}
( \Ba \wedge \Bb ) \cdot (\Bu_j \wedge \Bu_i)
&= \Ba \cdot (\Bb \cdot (\Bu_j \wedge \Bu_i)) \\
&= \Ba \cdot (\Bb \cdot \Bu_j \Bu_i - \Bb \cdot \Bu_i \Bu_j) \\
&= \Bb \cdot \Bu_j \Ba \cdot \Bu_i - \Bb \cdot \Bu_i \Ba \cdot \Bu_j \\
\end{aligned}
\end{equation}

Thus we have a decomposition of the bivector directly into a sum of components
for the reciprocal frame bivectors:

\begin{equation}
\Ba \wedge \Bb
= \sum_{i<j} \left((\Ba \wedge \Bb) \cdot (\Bu_j \wedge \Bu_i) \right) \Bu^i \wedge \Bu^j
\end{equation}

\documentclass{article}      % Specifies the document class


\usepackage{amsmath}
\usepackage{mathpazo}

%
% shorthand for bold symbols, convenient for vectors and matrices
%
\newcommand{\Ba}[0]{\mathbf{a}}
\newcommand{\Bb}[0]{\mathbf{b}}
\newcommand{\Bc}[0]{\mathbf{c}}
\newcommand{\Bd}[0]{\mathbf{d}}
\newcommand{\Be}[0]{\mathbf{e}}
\newcommand{\Bf}[0]{\mathbf{f}}
\newcommand{\Bg}[0]{\mathbf{g}}
\newcommand{\Bh}[0]{\mathbf{h}}
\newcommand{\Bi}[0]{\mathbf{i}}
\newcommand{\Bj}[0]{\mathbf{j}}
\newcommand{\Bk}[0]{\mathbf{k}}
\newcommand{\Bl}[0]{\mathbf{l}}
\newcommand{\Bm}[0]{\mathbf{m}}
\newcommand{\Bn}[0]{\mathbf{n}}
\newcommand{\Bo}[0]{\mathbf{o}}
\newcommand{\Bp}[0]{\mathbf{p}}
\newcommand{\Bq}[0]{\mathbf{q}}
\newcommand{\Br}[0]{\mathbf{r}}
\newcommand{\Bs}[0]{\mathbf{s}}
\newcommand{\Bt}[0]{\mathbf{t}}
\newcommand{\Bu}[0]{\mathbf{u}}
\newcommand{\Bv}[0]{\mathbf{v}}
\newcommand{\Bw}[0]{\mathbf{w}}
\newcommand{\Bx}[0]{\mathbf{x}}
\newcommand{\By}[0]{\mathbf{y}}
\newcommand{\Bz}[0]{\mathbf{z}}
\newcommand{\BA}[0]{\mathbf{A}}
\newcommand{\BB}[0]{\mathbf{B}}
\newcommand{\BC}[0]{\mathbf{C}}
\newcommand{\BD}[0]{\mathbf{D}}
\newcommand{\BE}[0]{\mathbf{E}}
\newcommand{\BF}[0]{\mathbf{F}}
\newcommand{\BG}[0]{\mathbf{G}}
\newcommand{\BH}[0]{\mathbf{H}}
\newcommand{\BI}[0]{\mathbf{I}}
\newcommand{\BJ}[0]{\mathbf{J}}
\newcommand{\BK}[0]{\mathbf{K}}
\newcommand{\BL}[0]{\mathbf{L}}
\newcommand{\BM}[0]{\mathbf{M}}
\newcommand{\BN}[0]{\mathbf{N}}
\newcommand{\BO}[0]{\mathbf{O}}
\newcommand{\BP}[0]{\mathbf{P}}
\newcommand{\BQ}[0]{\mathbf{Q}}
\newcommand{\BR}[0]{\mathbf{R}}
\newcommand{\BS}[0]{\mathbf{S}}
\newcommand{\BT}[0]{\mathbf{T}}
\newcommand{\BU}[0]{\mathbf{U}}
\newcommand{\BV}[0]{\mathbf{V}}
\newcommand{\BW}[0]{\mathbf{W}}
\newcommand{\BX}[0]{\mathbf{X}}
\newcommand{\BY}[0]{\mathbf{Y}}
\newcommand{\BZ}[0]{\mathbf{Z}}

\newcommand{\Bzero}[0]{\mathbf{0}}
\newcommand{\Btheta}[0]{\boldsymbol{\theta}}
\newcommand{\Btau}[0]{\boldsymbol{\tau}}
\newcommand{\Bomega}[0]{\boldsymbol{\omega}}

%
% shorthand for unit vectors
%
\newcommand{\acap}[0]{\hat{\Ba}}
\newcommand{\bcap}[0]{\hat{\Bb}}
\newcommand{\ccap}[0]{\hat{\Bc}}
\newcommand{\dcap}[0]{\hat{\Bd}}
\newcommand{\ecap}[0]{\hat{\Be}}
\newcommand{\fcap}[0]{\hat{\Bf}}
\newcommand{\gcap}[0]{\hat{\Bg}}
\newcommand{\hcap}[0]{\hat{\Bh}}
\newcommand{\icap}[0]{\hat{\Bi}}
\newcommand{\jcap}[0]{\hat{\Bj}}
\newcommand{\kcap}[0]{\hat{\Bk}}
\newcommand{\lcap}[0]{\hat{\Bl}}
\newcommand{\mcap}[0]{\hat{\Bm}}
\newcommand{\ncap}[0]{\hat{\Bn}}
\newcommand{\ocap}[0]{\hat{\Bo}}
\newcommand{\pcap}[0]{\hat{\Bp}}
\newcommand{\qcap}[0]{\hat{\Bq}}
\newcommand{\rcap}[0]{\hat{\Br}}
\newcommand{\scap}[0]{\hat{\Bs}}
\newcommand{\tcap}[0]{\hat{\Bt}}
\newcommand{\ucap}[0]{\hat{\Bu}}
\newcommand{\vcap}[0]{\hat{\Bv}}
\newcommand{\wcap}[0]{\hat{\Bw}}
\newcommand{\xcap}[0]{\hat{\Bx}}
\newcommand{\ycap}[0]{\hat{\By}}
\newcommand{\zcap}[0]{\hat{\Bz}}
\newcommand{\thetacap}[0]{\hat{\Btheta}}

%
% to write R^n and C^n in a distinguishable fashion.  Perhaps change this
% to the double lined characters upon figuring out how to do so.
%
\newcommand{\C}[1]{$\mathbb{C}^{#1}$}
\newcommand{\R}[1]{$\mathbb{R}^{#1}$}

%
% various generally useful helpers
%

% derivative of #1 wrt. #2:
\newcommand{\D}[2] {\frac {d#2} {d#1}}

\newcommand{\inv}[1]{\frac{1}{#1}}
\newcommand{\cross}[0]{\times}

\newcommand{\abs}[1]{\lvert{#1}\rvert}
\newcommand{\norm}[1]{\lVert{#1}\rVert}
\newcommand{\innerprod}[2]{\langle{#1}, {#2}\rangle}
\newcommand{\dotprod}[2]{{#1} \cdot {#2}}
\newcommand{\bdotprod}[2]{\left({#1} \cdot {#2}\right)}
\newcommand{\crossprod}[2]{{#1} \cross {#2}}
\newcommand{\tripleprod}[3]{\dotprod{\left(\crossprod{#1}{#2}\right)}{#3}}

\DeclareMathOperator{\Proj}{Proj}
\DeclareMathOperator{\Span}{span}
\DeclareMathOperator{\Sgn}{sgn}
\DeclareMathOperator{\Area}{Area}
\DeclareMathOperator{\Volume}{Volume}

%
% A few miscellaneous things specific to this document
%
\newcommand{\crossop}[1]{\crossprod{#1}{}}

% R2 vector.
\newcommand{\VectorTwo}[2]{
\begin{bmatrix}
 {#1} \\
 {#2}
\end{bmatrix}
}

\newcommand{\VectorN}[1]{
\begin{bmatrix}
{#1}_1 \\
{#1}_2 \\
\vdots \\
{#1}_N \\
\end{bmatrix}
}

\newcommand{\DETuvij}[4]{
\begin{vmatrix}
 {#1}_{#3} & {#1}_{#4} \\
 {#2}_{#3} & {#2}_{#4}
\end{vmatrix}
}

\newcommand{\DETuvwijk}[6]{
\begin{vmatrix}
 {#1}_{#4} & {#1}_{#5} & {#1}_{#6} \\
 {#2}_{#4} & {#2}_{#5} & {#2}_{#6} \\
 {#3}_{#4} & {#3}_{#5} & {#3}_{#6}
\end{vmatrix}
}

\newcommand{\DETuvwxijkl}[8]{
\begin{vmatrix}
 {#1}_{#5} & {#1}_{#6} & {#1}_{#7} & {#1}_{#8} \\
 {#2}_{#5} & {#2}_{#6} & {#2}_{#7} & {#2}_{#8} \\
 {#3}_{#5} & {#3}_{#6} & {#3}_{#7} & {#3}_{#8} \\
 {#4}_{#5} & {#4}_{#6} & {#4}_{#7} & {#4}_{#8} \\
\end{vmatrix}
}

%\newcommand{\DETuvwxyijklm}[10]{
%\begin{vmatrix}
% {#1}_{#6} & {#1}_{#7} & {#1}_{#8} & {#1}_{#9} & {#1}_{#10} \\
% {#2}_{#6} & {#2}_{#7} & {#2}_{#8} & {#2}_{#9} & {#2}_{#10} \\
% {#3}_{#6} & {#3}_{#7} & {#3}_{#8} & {#3}_{#9} & {#3}_{#10} \\
% {#4}_{#6} & {#4}_{#7} & {#4}_{#8} & {#4}_{#9} & {#4}_{#10} \\
% {#5}_{#6} & {#5}_{#7} & {#5}_{#8} & {#5}_{#9} & {#5}_{#10}
%\end{vmatrix}
%}

% R3 vector.
\newcommand{\VectorThree}[3]{
\begin{bmatrix}
 {#1} \\
 {#2} \\
 {#3}
\end{bmatrix}
}


%\DeclareMathOperator{\Transpose}{T}
\DeclareMathOperator{\rank}{rank}
%\newcommand{\transpose}[1]{{{#1}^{\TextTranspose}}}
%\newcommand{\transpose}[1]{{{#1}^{\text{T}}}}
\newcommand{\T}[0]{\text{T}}
\newcommand{\BOmega}[0]{\boldsymbol{\Omega}}

\newcommand{\Det}[1] {\abs{#1}}

\usepackage{color,cite,graphicx}
   % use colour in the document, put your citations as [1-4]
   % rather than [1,2,3,4] (it looks nicer, and the extended LaTeX2e
   % graphics package. 
\usepackage{latexsym,amssymb,epsf} % don't remember if these are
   % needed, but their inclusion can't do any damage


%
% The real thing:
%

                             % The preamble begins here.
\title{ Matrix review. } % Declares the document's title.
\author{Peeter Joot}         % Declares the author's name.
\date{ April 11, 2008 }        % Deleting this command produces today's date.

\begin{document}             % End of preamble and beginning of text.

\maketitle{}


\section{ Motivation. }


My initial intention for these notes was to 

get a feel for the similarities and differences between GA and matrix approaches to solution of projection.  Attempting to
write up that comparison
I found gaps in my understanding of the matrix algebra.  In particular the topic of projection as well as the related
ideas of pseudoinverses and SVD were not adequately covered in my university courses, nor my texts from those courses.
Here is my attempt to write up what I understand of these subjects and explore the gaps in my knowledge.

Particularily helpful was 
Gilbert Strang's excellent MIT lecture on subspace projection (available on the MIT opencourse website).
Much of the notes below are probably detailed in his course textbook, which I don't have.  However, 
if I can't explain the ideas to myself, or write them up in a way that I feel would explain
to others, then I obviously don't understand them sufficiently.


\section{ Subspace projection in matrix notation. }


\subsection{ Projection onto line. }


\begin{figure}[htp]

\centering
\includegraphics[totalheight=0.4\textheight]{Projection_line}
\caption{Projection onto line.}\label{fig:Projection_line}
\end{figure}

The simplest sort of projection to compute is projection onto a line.  Given a direction vector $b$, and a line with direction vector $u$
as in figure \ref{fig:Projection_line}.

The projection onto $u$ is some value:

\[
p = \alpha u
\]

and we can write

\[
b = p + e
\]

where e is the component perpendicular to the line $u$.

Expressed in terms of the dot product this relationship is described by:

\[
(b - p) \cdot a = 0
\]

Or, 

\[
b \cdot a = \alpha a \cdot a
\]

and solving for $\alpha$ and substituting we have:

\begin{equation}
p = a \frac{a \cdot b}{a \cdot a}
\end{equation}

In matrix notation that is:

\begin{equation}
p = a \left(\frac{a^\T b}{a^\T a} \right)
\end{equation}

Following Gilbert Strang's MIT lecture on subspace projection, the parenthesis can be moved to directly express this as a projection matrix operating on b.

\begin{equation}\label{eqn:projectmatrixline}
p = \left(\frac{a a^\T}{a^\T a}\right) b = P b
\end{equation}


\subsection{ Projection onto plane (or subspace). }



\begin{figure}[htp]
\centering
\includegraphics[totalheight=0.4\textheight]{Projection_plane}
\caption{Projection onto plane.}\label{fig:Projection_plane}
\end{figure}

Calculation of the projection matrix to project onto a plane is similar.  The variables to solve for are $p$, and $e$ in as
figure \ref{fig:Projection_plane}.

For projection onto a plane (or hyperplane) the idea is the same, splitting the vector into a component in the plane and an perpendicular component.
Since the
idea is the same for any dimensional subspace, explicit specification of the summation range is omitted here so the result is good for higher dimensional
subspaces as well as the plane:

\[
b - p = e
\]
\[
p = \sum \alpha_i u_i
\]

however, we get a set of equations, one for each direction vector in the plane

\[
(b - p) \cdot u_i = 0
\]

Expanding $p$ explicitly and rearranging we have the following set of equations:

\[
b \cdot u_i = (\sum_s \alpha_s u_s) \cdot u_i
\]

putting this in matrix form

\begin{align*}
[b \cdot u_i]_i 
&= 
{
\begin{bmatrix}
(\sum_s \alpha u_s) \cdot u_i
\end{bmatrix}
}_i \\
\end{align*}

Writing $U = 
\begin{bmatrix}
u_1 & u_2 & \cdots 
\end{bmatrix}$
\begin{align*}
\begin{bmatrix}
u_1^\T \\
u_2^\T \\
\vdots 
\end{bmatrix}
b
&= 
\begin{bmatrix}
(\sum_s \alpha_s u_s) \cdot u_i
\end{bmatrix} \\
&= 
{
\begin{bmatrix}
u_i \cdot u_j
\end{bmatrix}
}_{ij}
\begin{bmatrix}
\alpha_1 \\
\alpha_2 \\
\vdots \\
\end{bmatrix} \\
\end{align*}

Solving for the vector of unknown coefficients $\alpha = [\alpha_i]_i$ we have

\[
\alpha
=
{{
\begin{bmatrix}
u_i \cdot u_j
\end{bmatrix}
}_{ij}}^{-1}
U^\T b
\]

And 

\begin{equation}
p = U \alpha = U
{{
\begin{bmatrix}
u_i \cdot u_j

\end{bmatrix}
}_{ij}}^{-1}
U^\T b
\end{equation}

However, this matrix in the middle is just $U^\T U$:

\begin{align*}
\begin{bmatrix}
u_1^\T \\
u_2^\T \\
\vdots \\
\end{bmatrix}
\begin{bmatrix}
{u_1} & {u_2} & \hdots \\
\end{bmatrix}
&=
\begin{bmatrix}
u_1^\T {u_1} & u_1^\T {u_2} & \hdots \\
u_2^\T {u_1} & u_2^\T {u_2} & \hdots \\
\vdots & & \\
\end{bmatrix} \\
&=
{
\begin{bmatrix}
u_i^\T {u_j}
\end{bmatrix}
}_{ij} \\
&=
{
\begin{bmatrix}
{u_i} \cdot {u_j}
\end{bmatrix}
}_{ij} \\
\end{align*}

This provides the final result:

\begin{equation}\label{eqn:projectiongeneralmatrix}
\Proj_{U}\left(b\right) = U (U^\T U)^{-1} U^\T b
\end{equation}

\subsection{ Simplifying case.  Orthonormal basis for column space. }

To evaluate equation \ref{eqn:projectiongeneralmatrix} we need only full column rank for $U$, but this will be messy in general due to the matrix inversion required for the center product.  That can be avoided by picking an orthonormal basis for the vector space that we are projecting on.  With an orthonormal column basis that 
central product term to invert is:

\[
U^\T U = [ u_i^\T u_j ]_{ij} = [ \delta_{ij} ]_{ij} = I_{r,r}
\]

Therefore, the projection matrix can be expressed using the two exterior terms alone:

\begin{equation}\label{eqn:projOrthonormal}
\Proj_U = U (U^\T U)^{-1} U^\T = U U^\T
\end{equation}

\subsection{ Numerical expansion of left pseudoscalar matrix with matrix.}

Numerically expanding the projection matrix $A (A^\T A)^{-1}A^\T$ isn't something
that we want to do, but the simpler projection matrix of equation
\ref{eqn:projOrthonormal} that we get with an orthonormal basis makes this not so daunting.

Let's do this to get a feel for things.

\subsubsection{ \R{4} plane projection example. }

Take a simple projection onto the plane spanned by the following two orthonormal vectors

\[
u_1 = 
\frac{\sqrt{2}}{4}
\begin{bmatrix}
1 \\
2 \\
\sqrt{3} \\
0 \\
\end{bmatrix}
\]

\[
u_2 = 
\frac{\sqrt{2}}{4}
\begin{bmatrix}
-\sqrt{3} \\
0 \\
1 \\
2 \\
\end{bmatrix}
\]

Thus the projection matrix is:

\[
P =
\begin{bmatrix}
u_1 & u_2 \\
\end{bmatrix}
\begin{bmatrix}
u_1^\T \\
u_2^\T \\
\end{bmatrix}
=
\inv{8}
\begin{bmatrix}
1 & -\sqrt{3} \\
2 & 0 \\
\sqrt{3} & 1 \\
0 & 2 \\
\end{bmatrix}
\begin{bmatrix}
1 & 2 & \sqrt{3} & 0 \\
-\sqrt{3} & 0 & 1 & 2 \\
\end{bmatrix}
\]
\[
\implies
P =
\inv{8}
\begin{bmatrix}
4 & 2 & 0 & -2\sqrt{3} \\
2 & 4 & 2\sqrt{3} & 0 \\
0 & 2\sqrt{3} & 4 & 2 \\
-2\sqrt{3} & 0 & 2 & 4 \\
\end{bmatrix}
=
\begin{bmatrix}
1/2 & 1/4 & 0 & -\sqrt{3}/4 \\
1/4 & 1/2 & \sqrt{3}/4 & 0 \\
0 & \sqrt{3}/4 & 1/2 & 1/4 \\
-\sqrt{3}/4 & 0 & 1/4 & 1/2 \\
\end{bmatrix}
\]

What can be said about this just by looking at the matrix itself?

\begin{enumerate}
\item
One can verify by inspection that $P u_1 = u_1$ and $P u_2 = u_2$.  This is what we expected
so this validates all the math performed so far.  Good!

%, so this
%is has at least one of the expected properties of a projection matrix.
%We also expect that $P x = 0$ for any $x \in N(V^\T)$, so if
%it has that property too we can call it the projection matrix for the subspace
%spanned by $u_1$, and $u_2$.  We also assume that this is the projection
%matrix for $A$ (we haven't yet shown any explicit
%relationship between this first
%$r$ column vectors in the matrix $V$ and the original matrix $A$).

\item
It is symmetric.  Analytically, we know to expect this, since for a 
a full column rank matrix $A$ the transpose of the projection matrix is:

\[
P^\T = {\left(A \inv{A^\T A} A^\T \right)}^\T = P.
\]

\item
In this particular case columns 2,4 and columns 1,3 are each pairs of
perpendicular vectors.  Is something like this to be expected in general for
projection matrices?

\item
We expect this to be a rank two matrix, so the null space has dimension two.  This can be verified.

\end{enumerate}

\subsection{ Return to analytic treatment. }

Let's look at the matrix for projection onto an orthonormal basis in a bit more detail.  This simpler form allows for
some observations that are a bit harder in the general form.

Suppose we have a vector $n$ that is perpendicular to all the orthonormal vectors $u_i$ that span the subspace.  We can then write:

\[
u_i \cdot n = 0
\]

Or,
\[
u_i^\T n = 0
\]

In block matrix form for all $u_i$ that is:

\[
[u_i^\T n]_i = [u_i^\T]_i n = U^\T n = 0
\]

This is all we need to verify that our projection matrix indeed produces a zero for any vector completely outside of the subspace:

\[
\Proj_U(n) = U (U^\T n) = U 0 = 0
\]

Now we have seen numerically that $U U^\T$ is not an identity matrix despite 
operating as one on any vector that lies completely in the subspace.
%Although this matrix may 
%have blocks of identity matrixes along the diagonal in some cases.

Having seen the action of this matrix on vectors in the null space, we can now
directly examine the action of this matrix on any vector that lies in the
span of the set $\{u_i\}$.  By linearity it is sufficient to do this calcuation 
for a particular $u_i$:

\begin{align*}
U U^\T u_i
&=
U
\begin{bmatrix}
u_1^\T u_i \\
u_2^\T u_i \\
\vdots \\
u_r^\T u_i \\
\end{bmatrix}
\\
&=
\begin{bmatrix}
{u_1} & {u_2} & \cdots & {u_r} \\
\end{bmatrix}
{
\begin{bmatrix}
\delta_{si}
\end{bmatrix}
}_s \\
&= \sum_{k=1}^{r} u_k \delta_{ki} \\
&= u_i \\
\end{align*}

This now completes the validation of the properties of this matrix (in it's simpler form with an orthonormal basis for the subspace).

\section{ Matrix projection vs. Geometric Algebra. }

I found it pretty interesting how similar the projection product is to the projection matrix above from traditional matrix algebra.  It is worthwhile
to write this out and point out the similarities and differences.

\subsection{ General projection matrix. }

We've shown above, provided a matrix A is of full column rank, a projection onto it's columns can be written:

\[
\Proj_A(x) = \left( A \frac{1}{A^\T A} A^\T \right) x
\]

Now contrast this with the projection written in terms of a vector dot product with a non-null blade 

\[
\Proj_A(x) = A \left(\inv{A} \cdot x\right) = A \inv{A^\dagger A} \left(A^\dagger \cdot x\right)
\]

This is a curious correspondence and niave comparison could lead one to think that perhaps there the concepts of matrix
transposition and blade reversal are equivalent.

This isn't actually the case since the matrix transposition actually corresponds to
the adjoint operation of a linear transformation for blades.  Be that as it may, there appears to be other situations
other than projections where matrix operations of the form $A^\T A$ end up with GA equivalents in the form $A^\dagger A$.  Another example
is the rigid body equations where the body anglular velocity bivector corresponding to a rotor $R$ is of the form $\BOmega = R' R^\dagger$, whereas the matrix
form for a rotation matrix $R$ is of the form $\BOmega = R' R^\dagger$.

\subsection{ Projection matrix full column rank requirement. }

The projection matrix derivations above required full column rank.  A reformulation in terms
of a generalized matrix (Moore-Penrose) inverse, or SVD can eliminate this full column rank requirement for
the formulation of the projection matrix.

We will get to this later, but we never really proved that
full column rank implies $A^\T A$ invertability.

If one writes out the matrix $A^\T A$ in full

Now, if $A = [a_i]_i$, the matrix

\begin{equation}\label{eqn:AtA}
A^\T A
=
\begin{bmatrix}
{a_1} \cdot {a_1} & {a_1} \cdot {a_2} & \hdots \\
{a_2} \cdot {a_1} & {a_2} \cdot {a_2} & \hdots \\
\vdots & & \\
\end{bmatrix}.
\end{equation}

This is an invertible matrix provided $\{a_i\}_i$ is a linearly independent set of vectors.
For full column rank to imply invertability, it would be sufficient to prove that the 
determinant of this matrix was non-zero.

I'm not sure how to show that this is true with just matrix algebra, however
one can identify the determinant of the matrix of equation \ref{eqn:AtA}, after an adjustment
of sign for reversion, as the GA dot product of a k-blade:

\begin{equation}\label{eqn:AdagdotA}
(-1)^{k(k-1)/2} (a_1 \wedge \cdots \wedge a_k) \cdot (a_1 \wedge \cdots \wedge a_k).
\end{equation}

Linear independence means that this wedge product is non-zero, and therefore the dot product, and thus original determinant is also non-zero.

When the $A^\T A$ matrix of equation \ref{eqn:AtA} is invertable that inverse can be written using a cofactor matrix (adjugate expansion):

Let's write this out, where $C_{ij}$ are the cofactor matrixes of $A^\T A$, we have:

\[
\inv{A^\T A} = \inv{ \Det{ A^\T A } } {[ C_{ij} ]}^\T
\]

Observe that the denominator here is exactly the determinant of equation \ref{eqn:AdagdotA}.  This illustrates
the motivation of Hestenes to lable the explicit alternating vector-vector dot product expansion of a 
blade-vector dot product the ``generalized Laplace expansion''.

%  This allows us to give additional meaning to the
%$\inv{A^\T A}A^\T$ factor of the general projection matrix.
%  Operationally this appears to 
%provide a way to compute the dot product of a
%blade with vector (just scaled by the determinant and by the sign of the reversion).

\subsection{ Projection onto orthonormal columns. }

When the 
columns of the matrix $A$ are orthonormal, the projection matrix is reduced to:

\[
\Proj_A(x) = \left( A A^\T \right) x.
\]

The corresponding GA entity is a projection onto a unit magnitude blade.  With that scaling the inverse term also drops out leaving:

\[
\Proj_A(x) = A \left(A^\dagger \cdot x\right)
\]

This helps point out the similarity between the matrix inverse $\inv{A^\T A}$ and the blade product inverse $\inv{{A^\dagger}A}$ is only on the surface,
since this blade product is only a scalar.






\section{ Projection with generalized dot product. }

\section{ Oblique projection and reciprocal frame vectors. }




















%\begin{figure}[htp]
%\centering
%\includegraphics[totalheight=0.4\textheight]{visualize_subspace_projection}
%\caption{Visualizing projection onto a subspace.}\label{fig:Projection_subspace}
%\end{figure}
%
%We can geometrically visualize the projection problem 
%
%as in figure \ref{fig:Projection_subspace}.  Here
%the subspace can be pictured
%as a plane containing a set of mutually perpendicular basis vectors, as if
%one has visually projected all the higher dimensional vectors onto a plane.
%
%For a vector $x$ that contains some part not in the space we want to find
%the component in the space $p$, or characterize the projection operation
%that produces this vector, and also find the space of vectors that lie
%perpendicular to the space.




%This is why we say that $n$ is in the null space of $U^\T$,
%$N(U^\T)$ not $N(U)$.  One perhaps could say this is in the null
%or perpendicular space of the set $\{u_i\}$, but our use of columns as
%vectors coupled with the Euclianian norm requires transposition to express
%the same null space concept for a matrix of column vectors.
%
%Since $U^\T n = 0$ for any $n \in N(U)$ we also have


















%\subsection{ Application of projection as left pseudoinverse (ie: linear fitting). }
%
%%We have shown that the left pseudoinverse product with the matrix can
%%be expressed as a projection matrix (sum of the projection matrices
%%associated with a set of orthonormal vectors)
%
%%\begin{equation}\label{eqn:pseudoprojmatsum}
%%A^{+} A =
%%\sum_{k=1}^r {u_k}u_k^\T
%%\end{equation}
%%
%
%%(note this is a different ``$V$'' than the $V$ in $A = U \Sigma V^\T$ since it only includes the first $r$ columns).
%%This allows us to write the matrix of equation \ref{eqn:pseudoprojmatsum} as
%
%%\begin{equation}
%%A^{+} A = V V^\T
%%\end{equation}
%
%
%Equation \ref{eqn:projectiongeneralmatrix} provides us a way to find best solutions to general equations of the form:
%
%
%\[
%A x = b
%\]
%
%Here $A$ is the matrix of a linear transformation, $A : \mathbb{R}^k \rightarrow \mathbb{R}^n$, for some $k<n$.
%By ``best solutions'' here, we give this the geometrical meaning, namely, the solution matching the projection of $b$ onto the space.
%
%If b is not completely in the column space $C(A)$ of $A$, this can have no solution.  However, writing
%
%\[
%b = \Proj_A(b) + b_\perp
%\]
%
%as the components of $b$ in $C(A)$ and not in $C(A)$ respectively we can at least solve the reduced equation for $\hat{x}$:
%
%
%\begin{equation}\label{eqn:reducedinverseproblem}
%A \hat{x} = \Proj_A(b)
%\end{equation}
%
%
%This will be possible even in circumstances that the original equation had no solution.  Specifically, the vector b when projected onto the plane can be expressed as some
%linear combination of the columns of $A$ (a basis for the subspace).
%
%Substuition of our projection result into equation \ref{eqn:reducedinverseproblem} yields:
%
%\begin{align*}
%A \hat{x} 
%&= \Proj_{A}\left(b\right) = A (A^\T A)^{-1} A^\T b
%\end{align*}
%
%The simplest case here is when $A$ is of full column rank since one can pre-multiply this complete equation by $A^\T$ without any possibility of nulling
%$A \hat{x}$.
%
%\begin{align*}
%A^\T A \hat{x} 
%&= A^\T A (A^\T A)^{-1} A^\T b \\
%&= A^\T b \\
%\end{align*}
%
%Thus our best fit vector is
%
%\begin{equation}
%\hat{x} 
%= (A^\T A)^{-1} A^\T b
%\end{equation}
%
%Another way to view this is for any vector $x$ that is not in the null space $N(A)$, then the matrix:
%
%\begin{equation}
%A^{+}= (A^\T A)^{-1} A^\T
%\end{equation}
%
%has the action of a left inverse for any full column rank matrix $A$.  Thus when there is a solution to:
%
%\begin{equation}
%A x = b.
%\end{equation}
%
%It can be obtained by pre-multiplication using this "left" inverse.
%
%\begin{equation}
%A^{+} A x = x = A^{+} b
%\end{equation}
%
%
%
%
%
%
%
%
%
%
%
%
%
%
%
%
%
%
%
%
%
%
%
%
%
%
%
%
%
%
%
%
%
%
%
%
%
%
%
%
%
%
%
%\section{ SVD connection. }
%
%
%SVT decomposition is an factoring of $A \in M^{m \times n}$ with orthonormal matrices $U \in M^{m \times m}$
%
%and $V \in M^{n \times n} $ producing the following form:
%
%\[
%A = U \Sigma V^\T
%\]
%
%Sigma has the form:
%
%\[
%\Sigma = 
%\begin{bmatrix}
%D_{r,r} & 0_{r,n-r} \\
%0_{m-r,r} & 0_{m-r,n-r} \\
%\end{bmatrix}
%\]
%
%where $r = \rank(A)$, and $D$ is a diagonal matrix with the root of the (positive) eigenvalues of $A^\T A$.
%
%This provides a generalized spectral decomposition and similarity that applies to both non-square matrices and matrices not otherwise diagonalizable
%(ie: square matrix with similarity to a Jordon form matrix).  Given this decomposition we can write:
%
%\[
%\Sigma = U^\T A V
%\]
%
%If one were to ask the question of what is the closest that one could get to inverting such a matrix.  It's pretty clear that the closest one could get to
%identity will be with multiplication of a $\Sigma^{+}$ of the following form:
%
%\[
%\Sigma^{+} \Sigma
%=
%\begin{bmatrix}
%(D_{r,r})^{-1} & 0_{r,m-r} \\
%0_{n-r,r} & 0_{n-r,m-r} \\
%\end{bmatrix}
%\begin{bmatrix}
%D_{r,r} & 0_{r,n-r} \\
%0_{m-r,r} & 0_{m-r,n-r} \\
%\end{bmatrix}
%=
%\begin{bmatrix}
%I_{r,r} & 0_{r,n-r} \\
%0_{n-r,r} & 0_{n-r,n-r} \\
%\end{bmatrix}
%\]
%
%For a right pseudoinverse we have a similar result:
%
%\[
%\Sigma
%\Sigma^{+}
%=
%\begin{bmatrix}
%D_{r,r} & 0_{r,n-r} \\
%0_{m-r,r} & 0_{m-r,n-r} \\
%\end{bmatrix}
%\begin{bmatrix}
%(D_{r,r})^{-1} & 0_{r,m-r} \\
%0_{n-r,r} & 0_{n-r,m-r} \\
%\end{bmatrix}
%=
%\begin{bmatrix}
%I_{r,r} & 0_{r,m-r} \\
%0_{m-r,r} & 0_{m-r,m-r} \\
%\end{bmatrix}
%\]
%
%With either of these one can define a corresponding pseudoinverse (left or right) as:
%
%\begin{equation}
%A^{+} = V \Sigma^{+} U^\T
%\end{equation}
%
%This is a logical definition, but how close is it to the projective
%left inverse we calculated above in the case where $A$ is not of full column 
%rank?
%
%Multiplication gives: 
%
%\begin{align*}
%A^{+} A 
%&= V \Sigma^{+} U^\T U \Sigma V^\T \\
%&= V \Sigma^{+} \Sigma V^\T \\
%&=
%\begin{bmatrix}
%v_1 & v_2 & \cdots & v_r & v_{r+1} & \cdots & v_n \\
%\end{bmatrix}
%\begin{bmatrix}
%(D_{r,r})^{-1} & 0_{r,m-r} \\
%0_{n-r,r} & 0_{n-r,m-r} \\
%\end{bmatrix}
%\begin{bmatrix}
%D_{r,r} & 0_{r,n-r} \\
%0_{m-r,r} & 0_{m-r,n-r} \\
%\end{bmatrix}
%\begin{bmatrix}
%v_1^\T \\ v_2^\T \\ \vdots \\ v_r^\T \\ {v_{r+1}}^\T \\ \vdots \\ v_n^\T \\
%\end{bmatrix}
%\end{align*}
%%Embeded in that is the same "as-close-to" identity as calculated above.
%
%Writing $D_{r,r} = [\delta_{ij}\sigma_i]_{ij}$, we have:
%
%\begin{equation}\label{eqn:VIrVt}
%V \Sigma^{+} \Sigma V^\T 
%=
%\begin{bmatrix}
%\frac{v_1}{\sigma_1} & \frac{v_2}{\sigma_2} & \cdots & \frac{v_r}{\sigma_2} & 0 & \cdots & 0 
%\end{bmatrix}
%\begin{bmatrix}
%v_1^\T \sigma_1 \\ v_2^\T \sigma_2 \\ \vdots \\ v_r^\T \sigma_r \\ 0 \\ \vdots \\ 0 \\
%\end{bmatrix}
%\end{equation}
%
%Considering this as the product of block matrices we have a product here of the form
%
%\[
%\begin{bmatrix}
%A_{n,r} & 0_{n,n-r}
%\end{bmatrix}
%\begin{bmatrix}
%B_{r,n} \\ 0_{n-r,n}
%\end{bmatrix}
%=
%\begin{bmatrix}
%A_{n,r} B_{r,n} + 0_{n,n-r} 0_{n-r,n}
%\end{bmatrix}
%=
%\begin{bmatrix}
%A_{n,r} B_{r,n} + 0_{n,n}
%\end{bmatrix}
%=
%\begin{bmatrix}
%A_{n,r} B_{r,n}
%\end{bmatrix}
%\]
%
%Thus we can strip the block zero matrices from equation \ref{eqn:VIrVt} and write
%
%\begin{equation}\label{eqn:pseudoinversetimesmatrix}
%A^{+} A =
%V \Sigma^{+} \Sigma V^\T 
%=
%\begin{bmatrix}
%\frac{v_1}{\sigma_1} & \frac{v_2}{\sigma_2} & \cdots & \frac{v_r}{\sigma_2} 
%\end{bmatrix}
%\begin{bmatrix}
%v_1^\T \sigma_1 \\ v_2^\T \sigma_2 \\ \vdots \\ v_r^\T \sigma_r 
%\end{bmatrix}
%\end{equation}
%
%Eliminating the $\sigma$ terms we have:
%
%\begin{equation}\label{eqn:pseudoinversetimesmatrixsum}
%A^{+} A =
%\begin{bmatrix}
%\sum_{k=1}^r {v_k}v_k^\T
%\end{bmatrix}
%=
%\begin{bmatrix}
%v_1 & v_2 & \cdots & v_r 
%\end{bmatrix}
%\begin{bmatrix}
%v_1^\T \\ v_2^\T \\ \vdots \\ v_r^\T 
%\end{bmatrix}
%\end{equation}
%
%We previously calculated a left inverse using the projection matrix associated with a full column rank matrix.  For this product to have the properties of a
%left acting inverse we also expect it to be a projection.
%Let's disgress
%slightly before looking at whether equation
%\ref{eqn:pseudoinversetimesmatrixsum} satisifies this expectation.
%
%
%
%\subsection{ Correlating the SVD derived projection matrix back to $A$. }
%
%
%We now have to show that this is also the projection matrix associated
%
%with the columns of the 
%original matrix that we have an SVD factorization for
%
%\[
%A = U \Sigma V^\T
%\]
%
%Once we show this, then we have also demonstrated that the first $r$ 
%(orthonormal) column vectors in the matrix $V$ of this decomposition
%are a basis for the column space of $A$ itself.  Note that we are
%switching back to the original definition of $V \in M^{n,n}$ here, and
%not the $V \in M^{n,r}$ of equation \ref{eqn:projOrthonormal}.
%
%






\section{ Proof of omitted details and auxiliary stuff. }

\subsection{ That we can remove parenthesis to form projection matrix in line projection equation. }

Remove the parenthesis in some of these expressions may not always be correct, so it is worth demonstrating that this is okay as
done to calculate the projection matrix $P$ in 
equation \ref{eqn:projectmatrixline}.
We only need to look at the numerator since the denominator is a scalar in this case.

\begin{align*}
(a a^\T) b
&= [ a_i a_j ]_{ij} [b_i]_i \\
&= 
{\begin{bmatrix}
\sum_k a_i a_k b_k
\end{bmatrix}
}_i \\
&= 
{\begin{bmatrix}
a_i \sum_k a_k b_k
\end{bmatrix}
}_i \\
&= [ a_i ]_i a^\T b \\
&= a (a^\T b) \\
\end{align*}



\subsection{ Any invertible scaling of column space basis vectors does not change the projection }


Suppose that one introduces an alternate basis for the column space


\[
v_i = \sum \alpha_{ik} u_k
\]

This can be expressed in matrix form as:

\[
V = U E
\]

or

\[
U E^{-1} = V
\]

We should expect that the projection onto the plane expressed with this alternate basis should be identical to the original.  Verification
is straightforward:

\begin{align*}
\Proj_V 
&= V \left(V^\T V\right)^{-1} V^\T \\
&= \left(U E^{-1}\right) \left({\left(U E^{-1}\right)}^\T \left(U E^{-1}\right)\right)^{-1} {\left(U E^{-1}\right)}^\T \\
&= \left(U E^{-1}\right) \left( {E^{-1}}^T U^\T U E^{-1}\right)^{-1} {\left(U E^{-1}\right)}^\T \\
&= \left(U E^{-1}\right) E \left( U^\T U\right)^{-1} E^\T {\left(U E^{-1}\right)}^\T \\
&= U \left( U^\T U\right)^{-1} E^\T {E^{-1}}^\T U^\T \\
&= U \left( U^\T U\right)^{-1} U^\T \\
&= \Proj_U
\end{align*}

\end{document}               % End of document.

\documentclass{article}      % Specifies the document class

\usepackage{amsmath}
\usepackage{mathpazo}

%
% shorthand for bold symbols, convenient for vectors and matrices
%
\newcommand{\Ba}[0]{\mathbf{a}}
\newcommand{\Bb}[0]{\mathbf{b}}
\newcommand{\Bc}[0]{\mathbf{c}}
\newcommand{\Bd}[0]{\mathbf{d}}
\newcommand{\Be}[0]{\mathbf{e}}
\newcommand{\Bf}[0]{\mathbf{f}}
\newcommand{\Bg}[0]{\mathbf{g}}
\newcommand{\Bh}[0]{\mathbf{h}}
\newcommand{\Bi}[0]{\mathbf{i}}
\newcommand{\Bj}[0]{\mathbf{j}}
\newcommand{\Bk}[0]{\mathbf{k}}
\newcommand{\Bl}[0]{\mathbf{l}}
\newcommand{\Bm}[0]{\mathbf{m}}
\newcommand{\Bn}[0]{\mathbf{n}}
\newcommand{\Bo}[0]{\mathbf{o}}
\newcommand{\Bp}[0]{\mathbf{p}}
\newcommand{\Bq}[0]{\mathbf{q}}
\newcommand{\Br}[0]{\mathbf{r}}
\newcommand{\Bs}[0]{\mathbf{s}}
\newcommand{\Bt}[0]{\mathbf{t}}
\newcommand{\Bu}[0]{\mathbf{u}}
\newcommand{\Bv}[0]{\mathbf{v}}
\newcommand{\Bw}[0]{\mathbf{w}}
\newcommand{\Bx}[0]{\mathbf{x}}
\newcommand{\By}[0]{\mathbf{y}}
\newcommand{\Bz}[0]{\mathbf{z}}
\newcommand{\BA}[0]{\mathbf{A}}
\newcommand{\BB}[0]{\mathbf{B}}
\newcommand{\BC}[0]{\mathbf{C}}
\newcommand{\BD}[0]{\mathbf{D}}
\newcommand{\BE}[0]{\mathbf{E}}
\newcommand{\BF}[0]{\mathbf{F}}
\newcommand{\BG}[0]{\mathbf{G}}
\newcommand{\BH}[0]{\mathbf{H}}
\newcommand{\BI}[0]{\mathbf{I}}
\newcommand{\BJ}[0]{\mathbf{J}}
\newcommand{\BK}[0]{\mathbf{K}}
\newcommand{\BL}[0]{\mathbf{L}}
\newcommand{\BM}[0]{\mathbf{M}}
\newcommand{\BN}[0]{\mathbf{N}}
\newcommand{\BO}[0]{\mathbf{O}}
\newcommand{\BP}[0]{\mathbf{P}}
\newcommand{\BQ}[0]{\mathbf{Q}}
\newcommand{\BR}[0]{\mathbf{R}}
\newcommand{\BS}[0]{\mathbf{S}}
\newcommand{\BT}[0]{\mathbf{T}}
\newcommand{\BU}[0]{\mathbf{U}}
\newcommand{\BV}[0]{\mathbf{V}}
\newcommand{\BW}[0]{\mathbf{W}}
\newcommand{\BX}[0]{\mathbf{X}}
\newcommand{\BY}[0]{\mathbf{Y}}
\newcommand{\BZ}[0]{\mathbf{Z}}

\newcommand{\Bzero}[0]{\mathbf{0}}
\newcommand{\Btheta}[0]{\boldsymbol{\theta}}
\newcommand{\Btau}[0]{\boldsymbol{\tau}}
\newcommand{\Bomega}[0]{\boldsymbol{\omega}}

%
% shorthand for unit vectors
%
\newcommand{\acap}[0]{\hat{\Ba}}
\newcommand{\bcap}[0]{\hat{\Bb}}
\newcommand{\ccap}[0]{\hat{\Bc}}
\newcommand{\dcap}[0]{\hat{\Bd}}
\newcommand{\ecap}[0]{\hat{\Be}}
\newcommand{\fcap}[0]{\hat{\Bf}}
\newcommand{\gcap}[0]{\hat{\Bg}}
\newcommand{\hcap}[0]{\hat{\Bh}}
\newcommand{\icap}[0]{\hat{\Bi}}
\newcommand{\jcap}[0]{\hat{\Bj}}
\newcommand{\kcap}[0]{\hat{\Bk}}
\newcommand{\lcap}[0]{\hat{\Bl}}
\newcommand{\mcap}[0]{\hat{\Bm}}
\newcommand{\ncap}[0]{\hat{\Bn}}
\newcommand{\ocap}[0]{\hat{\Bo}}
\newcommand{\pcap}[0]{\hat{\Bp}}
\newcommand{\qcap}[0]{\hat{\Bq}}
\newcommand{\rcap}[0]{\hat{\Br}}
\newcommand{\scap}[0]{\hat{\Bs}}
\newcommand{\tcap}[0]{\hat{\Bt}}
\newcommand{\ucap}[0]{\hat{\Bu}}
\newcommand{\vcap}[0]{\hat{\Bv}}
\newcommand{\wcap}[0]{\hat{\Bw}}
\newcommand{\xcap}[0]{\hat{\Bx}}
\newcommand{\ycap}[0]{\hat{\By}}
\newcommand{\zcap}[0]{\hat{\Bz}}
\newcommand{\thetacap}[0]{\hat{\Btheta}}

%
% to write R^n and C^n in a distinguishable fashion.  Perhaps change this
% to the double lined characters upon figuring out how to do so.
%
\newcommand{\C}[1]{$\mathbb{C}^{#1}$}
\newcommand{\R}[1]{$\mathbb{R}^{#1}$}

%
% various generally useful helpers
%

% derivative of #1 wrt. #2:
\newcommand{\D}[2] {\frac {d#2} {d#1}}

\newcommand{\inv}[1]{\frac{1}{#1}}
\newcommand{\cross}[0]{\times}

\newcommand{\abs}[1]{\lvert{#1}\rvert}
\newcommand{\norm}[1]{\lVert{#1}\rVert}
\newcommand{\innerprod}[2]{\langle{#1}, {#2}\rangle}
\newcommand{\dotprod}[2]{{#1} \cdot {#2}}
\newcommand{\bdotprod}[2]{\left({#1} \cdot {#2}\right)}
\newcommand{\crossprod}[2]{{#1} \cross {#2}}
\newcommand{\tripleprod}[3]{\dotprod{\left(\crossprod{#1}{#2}\right)}{#3}}

\DeclareMathOperator{\Proj}{Proj}
\DeclareMathOperator{\Span}{span}
\DeclareMathOperator{\Sgn}{sgn}
\DeclareMathOperator{\Area}{Area}
\DeclareMathOperator{\Volume}{Volume}

%
% A few miscellaneous things specific to this document
%
\newcommand{\crossop}[1]{\crossprod{#1}{}}

% R2 vector.
\newcommand{\VectorTwo}[2]{
\begin{bmatrix}
 {#1} \\
 {#2}
\end{bmatrix}
}

\newcommand{\VectorN}[1]{
\begin{bmatrix}
{#1}_1 \\
{#1}_2 \\
\vdots \\
{#1}_N \\
\end{bmatrix}
}

\newcommand{\DETuvij}[4]{
\begin{vmatrix}
 {#1}_{#3} & {#1}_{#4} \\
 {#2}_{#3} & {#2}_{#4}
\end{vmatrix}
}

\newcommand{\DETuvwijk}[6]{
\begin{vmatrix}
 {#1}_{#4} & {#1}_{#5} & {#1}_{#6} \\
 {#2}_{#4} & {#2}_{#5} & {#2}_{#6} \\
 {#3}_{#4} & {#3}_{#5} & {#3}_{#6}
\end{vmatrix}
}

\newcommand{\DETuvwxijkl}[8]{
\begin{vmatrix}
 {#1}_{#5} & {#1}_{#6} & {#1}_{#7} & {#1}_{#8} \\
 {#2}_{#5} & {#2}_{#6} & {#2}_{#7} & {#2}_{#8} \\
 {#3}_{#5} & {#3}_{#6} & {#3}_{#7} & {#3}_{#8} \\
 {#4}_{#5} & {#4}_{#6} & {#4}_{#7} & {#4}_{#8} \\
\end{vmatrix}
}

%\newcommand{\DETuvwxyijklm}[10]{
%\begin{vmatrix}
% {#1}_{#6} & {#1}_{#7} & {#1}_{#8} & {#1}_{#9} & {#1}_{#10} \\
% {#2}_{#6} & {#2}_{#7} & {#2}_{#8} & {#2}_{#9} & {#2}_{#10} \\
% {#3}_{#6} & {#3}_{#7} & {#3}_{#8} & {#3}_{#9} & {#3}_{#10} \\
% {#4}_{#6} & {#4}_{#7} & {#4}_{#8} & {#4}_{#9} & {#4}_{#10} \\
% {#5}_{#6} & {#5}_{#7} & {#5}_{#8} & {#5}_{#9} & {#5}_{#10}
%\end{vmatrix}
%}

% R3 vector.
\newcommand{\VectorThree}[3]{
\begin{bmatrix}
 {#1} \\
 {#2} \\
 {#3}
\end{bmatrix}
}


\newcommand{\T}[0]{\text{T}}
\newcommand{\Bbeta}[0]{\boldsymbol{\beta}}

%
% The real thing:
%

                             % The preamble begins here.
\title{ Oblique projection and reciprocal frame vectors. }
\author{Peeter Joot}         % Declares the author's name.
%\date{}        % Deleting this command produces today's date.

\begin{document}             % End of preamble and beginning of text.

\maketitle{}

\section{ Motivation. }

Followup on wikipedia projection article's description of an oblique
projection.  Calculate this myself.

\section{ Using GA.  Oblique projection onto a line. }

INSERT DIAGRAM.

Problem is to project a vector $\Bx$ onto a line with direction $\pcap$, along a direction vector $\dcap$.

Write:

\begin{equation}\label{eqn:solveForprojLineUnit}
\Bx + \alpha \dcap = \beta \pcap
\end{equation}

and solve for $\Bp = \beta \pcap$.  Wedging with $\dcap$ provides the solution:

\[
\Bx \wedge \dcap + \alpha \underbrace{\dcap \wedge \dcap}_{=0} = \beta \pcap \wedge \dcap
\]
\[
\implies
\beta = \frac{\Bx \wedge \dcap}{\pcap \wedge \dcap}
\]

So the ``oblique'' projection onto this line (using direction $\dcap$) is:

\begin{equation}
\Proj_{\dcap \rightarrow \pcap}(\Bx) =
\frac{\Bx \wedge \dcap}{\pcap \wedge \dcap} \pcap
\end{equation}

This also shows that we do not need unit vectors for this sort of projection
operation, since we can scale these two vectors by any quantity since they are
in both the numerator and denominator.

Let $\BD$, and $\BP$ be vectors in the directions of $\dcap$, and $\pcap$ respectively.  Then the projection can also be written:

\begin{equation}\label{eqn:obliqueGAproj}
\Proj_{\BD \rightarrow \BP}(\Bx) =
\frac{\Bx \wedge \BD}{\BP \wedge \BD} \BP
\end{equation}

It's interesting to see projection expressed here without any sort of dot
product when all our previous projection calculations had intrinsic
requirements for a metric.

Now, let's compare this to the matrix forms of projection that we have become familiar with.  For the matrix result we need a metric, but because
this result is intrisically non-metric, we can introduce one if convienent and express this result with that too.  Such an expansion is:

\begin{align*}
\frac{\Bx \wedge \BD}{\BP \wedge \BD} \BP
&=
\Bx \wedge \BD
\frac{\BD \wedge \BP}{\BD \wedge \BP}
\inv{\BP \wedge \BD}
\BP \\
&=
(\Bx \wedge \BD) \cdot (\BD \wedge \BP)
\inv{\abs{\BP \wedge \BD}^2}
\BP \\
&=
((\Bx \wedge \BD) \cdot \BD) \cdot \BP
\inv{\abs{\BP \wedge \BD}^2}
\BP \\
&=
(
\Bx \BD^2 - \Bx \cdot \BD \BD
) \cdot \BP
\inv{\abs{\BP \wedge \BD}^2}
\BP \\
&=
\frac{\Bx \cdot \BP \BD^2 - \Bx \cdot \BD \BD \cdot \BP}
{\BP^2 \BD^2 - (\BP \cdot \BD)^2
%p ^ d . d ^ p =http://antwrp.gsfc.nasa.gov/apod/ap080424.html p ^ d . d . p = (p d^2 - d.p d) . p = p^2 d^2 - (d.p)^2
}
\BP \\
\end{align*}

This gives us the projection explicitly:

\begin{equation}\label{eqn:obliqueProjNearMatrixForm}
\Proj_{\BD \rightarrow \BP}(\Bx)
=
\left(\Bx \cdot \frac{\BP \BD^2 - \BD \BD \cdot \BP}{\BP^2 \BD^2 - (\BP \cdot \BD)^2}\right)
\BP
\end{equation}

It sure doesn't simplify things to expand things out, but we now have things prepared to express in matrix form.

Assuming a euclidian metrix, and a bit of playing shows that the denominator can be written more simply as:

\[
\BP^2 \BD^2 - (\BP \cdot \BD)^2 =
\begin{vmatrix}
U^\T U
\end{vmatrix}
\]

where:

\[
U =
\begin{bmatrix}
\BP & \BD \\
\end{bmatrix}
\]

Similarily the numerator can be written:

\[
\Bx \cdot \BP \BD^2 - \Bx \cdot \BD \BD \cdot \BP =
D^\T U
\begin{bmatrix}
0 & -1 \\
1 & 0 \\
\end{bmatrix}
U^\T \Bx.
\]

Combining these yields a projection matrix:

\begin{equation}\label{eqn:matrixprojfromwedge}
\Proj_{\BD \rightarrow \BP}(\Bx) =
\left(
\BP
\inv{
\begin{vmatrix}
U^\T U
\end{vmatrix}
}
\BD^\T U
\begin{bmatrix}
0 & -1 \\
1 & 0 \\
\end{bmatrix}
U^\T\right) \Bx.
\end{equation}

The alternation above suggests that this is related to the matrix inverse of something.  Let's try to calculate this directly instead.

\section{ Oblique projection onto a line using matrices. }

Let's start at the same place as in equation \ref{eqn:solveForprojLineUnit}, except that we know we can discard the unit vectors and work with any vectors in the projection directions:

\begin{equation}\label{eqn:solveForprojLineNonUnit}
\Bx + \alpha \BD = \beta \BP
\end{equation}

Assuming an inner product, we have two sets of results:

\begin{align*}
\innerprod{\BP}{\Bx} + \alpha \innerprod{\BP}{\BD} &= \beta \innerprod{\BP}{\BP} \\
\innerprod{\BD}{\Bx} + \alpha \innerprod{\BD}{\BD} &= \beta \innerprod{\BD}{\BP} \\
\end{align*}

and can solve this for $\alpha$, and $\beta$.

\begin{equation}\label{eqn:matrixtosolve}
\begin{bmatrix}
\innerprod{\BP}{\BD} & \innerprod{\BP}{\BP} \\
\innerprod{\BD}{\BD} & \innerprod{\BD}{\BP} \\
\end{bmatrix}
\begin{bmatrix}
-\alpha \\
\beta \\
\end{bmatrix}
=
\begin{bmatrix}
\innerprod{\BP}{\Bx} \\
\innerprod{\BD}{\Bx} \\
\end{bmatrix}
\end{equation}

%This can be solved with matrix inversion, but we are only

If our inner product is defined by $\innerprod{\Bu}{\Bv} = \Bu^* A \Bv$, we have:

\begin{align*}
\begin{bmatrix}
\innerprod{\BP}{\BD} & \innerprod{\BP}{\BP} \\
\innerprod{\BD}{\BD} & \innerprod{\BD}{\BP} \\
\end{bmatrix}
&=
\begin{bmatrix}
{\BP}^* A {\BD} & {\BP}^* A {\BP} \\
{\BD}^* A {\BD} & {\BD}^* A {\BP} \\
\end{bmatrix} \\
&=
{
\begin{bmatrix}
\BP & \BD \\
\end{bmatrix}
}^*
A
\begin{bmatrix}
{\BD} & {\BP} \\
\end{bmatrix} \\
\end{align*}

Thus the solution to equation \ref{eqn:matrixtosolve} is

\begin{equation}
\begin{bmatrix}
-\alpha \\
\beta \\
\end{bmatrix}
=
\left(
\inv{
{
\begin{bmatrix}
\BP & \BD \\
\end{bmatrix}
}^*
A
\begin{bmatrix}
{\BD} & {\BP} \\
\end{bmatrix}
}
{
\begin{bmatrix}
\BP & \BD \\
\end{bmatrix}
}^*
A
\right)
\Bx
\end{equation}

Again writing $U = 
\begin{bmatrix}
\BP & \BD \\
\end{bmatrix}
$, this is:

\begin{align*}
\begin{bmatrix}
-\alpha \\
\beta \\
\end{bmatrix}
&=
\left(
\inv{U^* A U 
\begin{bmatrix}
0 & 1 \\
1 & 0 \\
\end{bmatrix}
}
U^*
A
\right)
\Bx \\
&=
\left(
\begin{bmatrix}
0 & 1 \\
1 & 0 \\
\end{bmatrix}
\inv{U^* A U 
}
U^*
A
\right)
\Bx \\
\end{align*}

Since we only care about solution for $\beta$ to find the projection, we have to discard half the inversion work, and just select
that part of the solution (suggests that a Cramer's rule method is more efficient than matrix inversion in this case) :

\[
\beta = 
\begin{bmatrix}
0 & 1 \\
\end{bmatrix}
\begin{bmatrix}
-\alpha \\
\beta \\
\end{bmatrix}
\]

Thus the solution of this oblique projection problem in terms of matrixes is:

\begin{align*}
\Proj_{\BD \rightarrow \BP}(\Bx) 
&= 
\left(
\BP
\begin{bmatrix}
0 & 1 \\
\end{bmatrix}
\begin{bmatrix}
0 & 1 \\
1 & 0 \\
\end{bmatrix}
\inv{U^* A U 
}
U^*
A
\right)
\Bx
\end{align*}

Which is:

\begin{equation}
\Proj_{\BD \rightarrow \BP}(\Bx) = 
\left(
\BP
\begin{bmatrix}
1 & 0 \\
\end{bmatrix}
\inv{U^* A U 
}
U^*
A
\right)
\Bx
\end{equation}

Explicit expansion can be done easily enough to show that this is identical to equation \ref{eqn:obliqueProjNearMatrixForm}, so
the question of what we were implicitly inverting in equation \ref{eqn:matrixprojfromwedge} is answered.

\section{ Oblique projection onto hyperplane. }

Now that we've got this directed projection problem solved for a line in both GA and matrix form, the next logical step is a $k$-dimensional hyperplane projection.  The equation to solve is now:

\begin{equation}\label{eqn:solveForprojPlane}
\Bx + \alpha \BD = \sum \beta_i \BP_i
\end{equation}

\subsection{ Non metric solution using wedge products. }

For $\Bx$ with some component not in the hyperplane, we can wedge with $P = \BP_1 \wedge \BP_2 \wedge \cdots \wedge \BP_k$

\begin{equation*}
\Bx \wedge P + \alpha \BD \wedge P = \sum_{i=1}^k \beta_i \underbrace{\BP_i \wedge P}_{=0}
\end{equation*}
%- \Bx \wedge P = \alpha \BD \wedge P 

Thus the projection onto the hyperplane spanned by $P$ is going from $\Bx$ along $\BD$ is $\Bx + \alpha \BD$:

\begin{equation}\label{eqn:GAhyperplaneProjection}
\Proj_{\BD \rightarrow P}(\Bx) = \Bx - \frac{\Bx \wedge P}{\BD \wedge P} \BD
\end{equation}

\subsubsection{ Q: reduction of this }

When P is a single vector we can reduce this to our previous result:

\begin{align*}
\Proj_{\BD \rightarrow \BP}(\Bx) 
&= \Bx - \frac{\Bx \wedge \BP}{\BD \wedge \BP} \BD \\
&= \inv{\BD \wedge \BP} \left((\BD \wedge \BP)\Bx - (\Bx \wedge \BP) \BD \right) \\
&= \inv{\BD \wedge \BP} \left( (\BD \wedge \BP) \cdot \Bx - (\Bx \wedge \BP)  \cdot \BD \right) \\
&= \inv{\BD \wedge \BP} \left( \BD \BP \cdot \Bx -\BP \BD \cdot \Bx -\Bx \BP \cdot \BD +\BP \Bx \cdot \BD \right) \\
&= \inv{\BD \wedge \BP} \left( \BD \BP \cdot \Bx -\Bx \BP \cdot \BD \right) \\
\end{align*}

Which is:
\begin{equation}
\Proj_{\BD \rightarrow \BP}(\Bx) 
= \inv{\BP \wedge \BD} \BP \cdot ( \BD \wedge \Bx ).
\end{equation}

A result that is equivalent to our original equation \ref{eqn:obliqueGAproj}.  Can we similarily reduce the general result to something of this form.  Initially I wrote:

\begin{align*}
\Proj_{\BD \rightarrow P}(\Bx) 
&= \Bx - \frac{\Bx \wedge P}{\BD \wedge P} \BD \\
&= \frac{\BD \wedge P}{\BD \wedge P} \Bx - \frac{\Bx \wedge P}{\BD \wedge P} \BD \\
&= \inv{\BD \wedge P} \left( (\BD \wedge P) \Bx - (\Bx \wedge P) \BD \right) \\
&= \inv{\BD \wedge P} \left( (\BD \wedge P) \cdot \Bx - (\Bx \wedge P) \cdot \BD \right) \\
&= \inv{\BD \wedge P} \left( \BD P \cdot \Bx -P \BD \cdot \Bx - \Bx P \cdot \BD + P \Bx \cdot \BD \right) \\
&= \inv{\BD \wedge P} \left( \BD P \cdot \Bx - \Bx P \cdot \BD \right) \\ 
&= -\inv{\BD \wedge P} P \cdot (\BD \wedge \Bx ) \\
\end{align*}

However, I'm not sure that about the manipulations done on the last few lines where P has grade greater than 1 (ie: the triple product expansion and recollection later).

\subsection{ hyperplane directed projection using matrixes. }

To solve equation \ref{eqn:solveForprojPlane} using matrixes, we can take a set of inner products:

\begin{align*}
\innerprod{\BD}{\Bx} + \alpha \innerprod{\BD}{\BD} &= \sum_{u=1}^k \beta_u \innerprod{\BD}{\BP_u} \\
\innerprod{\BP_i}{\Bx} + \alpha \innerprod{\BP_i}{\BD} &= \sum_{u=1}^k \beta_u \innerprod{\BP_i}{\BP_u}
\end{align*}

Write $\BD = \BP_{k+1}$, and $\alpha = -\beta_{k+1}$ for symmetry, which reduces this to:

\begin{align*}
\innerprod{\BP_{k+1}}{\Bx} &= \sum_{u=1}^k \beta_u \innerprod{\BP_{k+1}}{\BP_u} + \beta_{k+1} \innerprod{\BP_{k+1}}{\BP_{k+1}}  \\
\innerprod{\BP_i}{\Bx} &= \sum_{u=1}^k \beta_u \innerprod{\BP_i}{\BP_u} + \beta_{k+1} \innerprod{\BP_i}{\BP_{k+1}}
\end{align*}

That is the following set of equations:

\[
\innerprod{\BP_i}{\Bx} = \sum_{u=1}^{k+1} \beta_u \innerprod{\BP_i}{\BP_u}
\]

Which we can now express as a single matrix equation (for $i,j \in [1,k+1]$) :

\begin{equation}
{
\begin{bmatrix}
\innerprod{\BP_i}{\Bx}
\end{bmatrix}
}_i
=
{
\begin{bmatrix}
\innerprod{\BP_i}{\BP_j}
\end{bmatrix}
}_{ij}
{
\begin{bmatrix}
\beta_i
\end{bmatrix}
}_i
\end{equation}

Solving for $\Bbeta = 
{
\begin{bmatrix}
\beta_i
\end{bmatrix}
}_i
$, gives:

\[
\Bbeta = 
\inv{
\begin{bmatrix}
\innerprod{\BP_i}{\BP_j}
\end{bmatrix}
}_{ij}
{
\begin{bmatrix}
\innerprod{\BP_i}{\Bx}
\end{bmatrix}
}_i
\]

The projective components of interest are $\sum_{i=1}^k \beta_i \BP_i$.  In matrix form that is:

\begin{align*}
\begin{bmatrix}
\BP_1 & \BP_2 & \cdots & \BP_k
\end{bmatrix}
\begin{bmatrix}
\beta_1 \\
\beta_2 \\
\vdots \\
\beta_k
\end{bmatrix}
=
\begin{bmatrix}
\BP_1 & \BP_2 & \cdots & \BP_k
\end{bmatrix}
\begin{bmatrix}
I_{k,k} & 0_{k,1}
\end{bmatrix}
\Bbeta
\end{align*}

Therefore the directed projection is:

\begin{equation}
\Proj_{\BD \rightarrow P}(\Bx) 
=
\begin{bmatrix}
\BP_1 & \BP_2 & \cdots & \BP_k
\end{bmatrix}
\begin{bmatrix}
I_{k,k} & 0_{k,1}
\end{bmatrix}
\inv{
\begin{bmatrix}
\innerprod{\BP_i}{\BP_j}
\end{bmatrix}
}_{ij}
{
\begin{bmatrix}
\innerprod{\BP_i}{\Bx}
\end{bmatrix}
}_i
\end{equation}

As before writing $U = 
\begin{bmatrix}
\BP_1 & \BP_2 & \cdots & \BP_k & \BD
\end{bmatrix}
$, and write $\innerprod{\Bu}{\Bv} = \Bu^* A \Bv$.  The directed projection is now:

\begin{equation*}
\Proj_{\BD \rightarrow P}(\Bx) 
=
\left(
U 
\begin{bmatrix}
I_{k,k} \\
0_{1,k} \\
\end{bmatrix}
\begin{bmatrix}
I_{k,k} & 0_{k,1}
\end{bmatrix}
\inv{
U^* A U
}
U^* A 
\right)
\Bx
\end{equation*}
\begin{equation}
=
\left(
U 
\begin{bmatrix}
I_{k,k} & 0_{k,1} \\
0_{1,k} & 0_{1,1} \\
\end{bmatrix}
\inv{
U^* A U
}
U^* A 
\right)
\Bx
\end{equation}

\end{document}               % End of document.

%
% Copyright � 2012 Peeter Joot.  All Rights Reserved.
% Licenced as described in the file LICENSE under the root directory of this GIT repository.
%

%
%
\chapter{Projection and Moore-Penrose vector inverse}
\index{projection}
\index{Moore-Penrose inverse}
\label{chap:projectionAndMoorePenroseVectorInverse}
%\date{May 16, 2008.  projectionAndMoorePenroseVectorInverse.tex}

\section{Projection and Moore-Penrose vector inverse}

One can observe that the Moore Penrose left vector inverse \(\Bv^+\) shows up in the projection matrix for a projection onto a line with a direction vector \(\Bv\):

\begin{equation}
\Proj_\Bv(\Bx) = \Bv \mathLabelBox{\inv{ \Bv^\T \Bv} \Bv^\T}{\(\Bv^+\)} \Bx
\end{equation}

I do not know of any other ``application'' of this Moore-Penrose vector inverse in traditional matrix algebra.  As stated it is an interesting mathematical curiosity that yes one can define a vector inverse, however what would you do with it?

In geometric algebra we also have a vector inverse, but it plays a much more fundamental role, and does not have the restriction of only acting from the left and
producing a scalar result.  As an example consider the projection, and rejection decomposition of a vector:

\begin{equation}\label{eqn:projectionAndMoorePenroseVectorInverse:120}
\begin{aligned}
\Bx
&= \Bv \inv{\Bv} \Bx \\
&= \Bv \left(\inv{\Bv} \cdot \Bx\right) + \Bv \left(\inv{\Bv} \wedge \Bx\right) \\
&= \Bv
\left(
\frac{\Bv}{\Bv^2} \cdot
 \Bx\right)
 + \Bv \left(\frac{\Bv}{\Bv^2} \wedge \Bx\right) \\
\end{aligned}
\end{equation}

In the above, \(\frac{\Bv}{\Bv^2} \cdot = \frac{\Bv^\T}{\Bv^\T \Bv} = \Bv^+\).  We can therefore describe the Moore Penrose vector left inverse as the matrix of the GA linear transformation \(\inv{\Bv} \cdot\).

Unlike the GA vector inverse, whos associativity allowed for the projection/rejection derivation above, this Moore-Penrose vector inverse has only left action, so in the above, you can not further write:

\begin{equation}\label{eqn:projectionAndMoorePenroseVectorInverse:20}
\Bv \Bv^{+} = 1
\end{equation}

(ie: \(\Bv \Bv^{+}\) is a projection matrix not scalar or matrix unity).

\subsection{matrix of wedge project transformation?}

Q: What is the matrix of the linear transformation \(\inv{\Bv} \wedge\)?

In rigid body dynamics we see the matrix of the linear transformation \(T_\Bv(\Bx) = (\Bv \cross)(\Bx)\).  This is the completely antisymmetric matrix as follows:

\begin{equation}
\Bv \times \Bx =
\begin{bmatrix}
0 & -v_3 & v_2 \\
v_3 & 0 & -v_1 \\
-v_2 & v_1 & 0 \\
\end{bmatrix}
\begin{bmatrix}
x_1 \\
x_2 \\
x_3 \\
\end{bmatrix}
\end{equation}

In order to specify the matrix of a vector-vector wedge product linear transformation we must introduce bivector coordinate vectors.  For the matrix of the cross product linear transformation the standard vector basis was the obvious choice.

Let us pick the following orthonormal basis:

\begin{equation}\label{eqn:projectionAndMoorePenroseVectorInverse:40}
\sigma = \{ \sigma_{ij} = \Be_i \wedge \Be_j \}_{i<j}
\end{equation}

and construct the matrix of the wedge project \(T_\Bv : \mathbb{R}^N \rightarrow {\bigwedge}^2\)

\begin{equation}\label{eqn:projectionAndMoorePenroseVectorInverse:60}
T_\Bv(\Bx) = \Bv \wedge \Bx = \sum_{\mu = ij, i<j} \DETuvij{v}{x}{i}{j} \sigma_{\mu}
\end{equation}
\begin{equation}\label{eqn:projectionAndMoorePenroseVectorInverse:80}
\implies
T_\Bv(\Be_k) \cdot {\sigma_{ij}}^\dagger =
\sum_{k \in ij, i<j} \DETuvij{v}{x}{i}{j}
= %\sum_{k \in ij, i<j}
v_i \delta_{kj} - v_j \delta_{ki}
\end{equation}

Since \(k\) cannot be simultaneously equal to both \(i\), and \(j\), this is:

\begin{equation}\label{eqn:projectionAndMoorePenroseVectorInverse:100}
T_\Bv(\Be_k) \cdot {\sigma_{ij}}^\dagger =
\left\{
\begin{array}{rl}
v_i & k=j \\
-v_j & k=i \\
0 & k \ne i,j \\
\end{array}
\right\}
\end{equation}

Unlike the left Moore-Penrose vector inverse that we find as the matrix of the linear transformation \(v \cdot ( \cdot )\), except for \R{3} where we have the cross product, I do not recognize this as the matrix of any common linear transformation.

%
% Copyright � 2012 Peeter Joot.  All Rights Reserved.
% Licenced as described in the file LICENSE under the root directory of this GIT repository.
%

%
%
\chapter{Angle between geometric elements}
\index{vectors!angle between}
\label{chap:angleBetweenLineAndPlane}
%\date{Mar 17, 2008.  angleBetweenLineAndPlane.tex}

Have the calculation for the angle between bivectors done elsewhere

\begin{equation}\label{eqn:angleLinePlane:bivectorangle}
\cos\theta = - \frac{\BA \cdot \BB}{\abs{\BA} \abs{\BB} }
\end{equation}

For \(\theta \in [0,\pi]\).

The vector/vector result is well known and also works fine in \R{N}

\begin{equation}\label{eqn:angleLinePlane:vectorangle}
\cos\theta = \frac{\Bu \cdot \Bv}{\abs{\Bu} \abs{\Bv} }
\end{equation}

\section{Calculation for a line and a plane}

Given a line with unit direction vector \(\Bu\), and plane with unit direction bivector \(\BA\), the component of that
vector in the plane is:

\begin{equation}\label{eqn:angleBetweenLineAndPlane:20}
-\Bu \cdot \BA \BA.
\end{equation}

So the direction cosine is available immediately

\begin{equation}\label{eqn:angleBetweenLineAndPlane:40}
\cos\theta = \Bu \cdot \frac{-\Bu \cdot \BA \BA}{\abs{\Bu \cdot \BA \BA}}
\end{equation}

However, this can be reduced significantly.  Start with the denominator

\begin{equation}\label{eqn:angleBetweenLineAndPlane:60}
\begin{aligned}
\abs{\Bu \cdot \BA \BA}^2
&= (\Bu \cdot \BA \BA)(\BA \BA \cdot \Bu) \\
&= (\Bu \cdot \BA )^2. \\
\end{aligned}
\end{equation}

And in the numerator we have:

\begin{equation}\label{eqn:angleBetweenLineAndPlane:80}
\begin{aligned}
\Bu \cdot (\Bu \cdot \BA \BA)
&= \inv{2}(
  \Bu (\Bu \cdot \BA \BA)
+ (\Bu \cdot \BA \BA) \Bu
) \\
&= \inv{2}(
  (\Bu \Bu \cdot \BA) \BA
+ (\Bu \cdot \BA) \BA \Bu
) \\
&= \inv{2}(
  (\BA \cdot \Bu \Bu) \BA
- (\BA \cdot \Bu) \BA \Bu
) \\
&= (\BA \cdot \Bu) \inv{2}( \Bu \BA - \BA \Bu ) \\
&= -(\BA \cdot \Bu)^2.
\end{aligned}
\end{equation}

Putting things back together

\begin{equation*}
\cos\theta
= \frac{(\BA \cdot \Bu)^2}{\abs{\Bu \cdot \BA}} = \abs{\Bu \cdot \BA}
\end{equation*}

The strictly positive value here is consistent with the fact that theta as calculated is in the \([0,\pi/2]\) range.

Restated for consistency with equations \eqnref{eqn:angleLinePlane:vectorangle} and \eqnref{eqn:angleLinePlane:bivectorangle} in terms of not necessarily
unit vector and bivectors \(\Bu\) and \(\BA\), we have

\begin{equation}
\cos\theta =
\frac{\abs{\Bu \cdot \BA}}{ \abs{\Bu} \abs{\BA} }
\end{equation}

% 
% 
% 
% Copyright � 2012 Peeter Joot
% All Rights Reserved
% 
% This file may be reproduced and distributed in whole or in part, without fee, subject to the following conditions:
% 
% o The copyright notice above and this permission notice must be preserved complete on all complete or partial copies.
% 
% o Any translation or derived work must be approved by the author in writing before distribution.
% 
% o If you distribute this work in part, instructions for obtaining the complete version of this file must be included, and a means for obtaining a complete version provided.
% 
% 
% Exceptions to these rules may be granted for academic purposes: Write to the author and ask.
% 
% 
% 
\chapter{Orthogonal decomposition take II.} 
\label{chap:orthodecomp}
\date{April 1, 2008.  orthodecomp.tex}
\section{Lemma.  Orthogonal decomposition.}
To do so we first need to be able to express a vector $\Bx$ in terms
of components parallel and perpendicular to the blade $\BA \in \wedge^k$.

\begin{align*}
\Bx 
&= \Bx \BA \inv{\BA} \\
&= (\Bx \cdot \BA + \Bx \wedge \BA) \inv{\BA} \\
&= 
(\Bx \cdot \BA) \cdot \inv{\BA}
+ \sum_{i=3,5,\cdots,2k-1} \gpgrade{(\Bx \cdot \BA) \inv{\BA}}{i} \\
&+ 
(\Bx \wedge \BA) \cdot \inv{\BA}
+ \sum_{i=3,5,\cdots,2k-1} \gpgrade{(\Bx \wedge \BA) \inv{\BA}}{i} 
+ \underbrace{(\Bx \wedge \BA) \wedge \inv{\BA}}_{=0}
\end{align*}

Since the LHS and RHS must both be vectors all the non-grade one terms
are either zero or cancel out.  This can be observed directly since:

\begin{align*}
\gpgrade{\Bx \cdot \BA \inv{\BA}}{i}
&= \gpgrade{ \frac{\Bx \BA - (-1)^{k}\BA\Bx}{2}\inv{\BA} }{i}  \\
&= -\frac{(-1)^{k}}{2} \gpgrade{ \BA\Bx \inv{\BA} }{i}  \\
\end{align*}

and

\begin{align*}
\gpgrade{\Bx \wedge \BA \inv{\BA}}{i} 
&= \gpgrade{ \frac{\Bx \BA + (-1)^{k}\BA\Bx}{2}\inv{\BA} }{i}  \\
&= +\frac{(-1)^{k}}{2} \gpgrade{ \BA\Bx \inv{\BA} }{i}  \\
\end{align*}

Thus all of the grade $3, \cdots ,2k-1$ terms cancel each other out.  Some terms
like $(\Bx \cdot \BA) \wedge \inv{\BA}$ are also independently zero.

(This is a result I've got in other places, but I thought it's worth
 writing down since I thought the direct cancellation is elegant).

%
% Copyright � 2012 Peeter Joot.  All Rights Reserved.
% Licenced as described in the file LICENSE under the root directory of this GIT repository.
%

%
%
\chapter{Matrix of grade k multivector linear transformations}
\label{chap:matrixOfLinearTx}

%\date{May 16, 2008.  matrixOfLinearTx.tex}

\section{Motivation}

The following shows explicitly the calculation required to form the matrix of a linear transformation between two grade k multivector subspaces (is there a name for a multivector of fixed grade that is not neccessarily a blade?).
This is nothing fancy or original, but just helpful
to have written out showing naturally how one generates this matrix from a consideration of the two sets of basis vectors.  After so much exposure to linear transformations only in matrix
form it is good to write this out in a way so that it is clear exactly how the coordinate matrices come in to the picture when and if they are introduced.

\section{Content}

Given \(T\), a linear transformation between two grade k multivector subspaces,
let \(\sigma = \{\sigma_i\}_{i=1}^m\) be a basis for a grade k multivector subspace.
For \(T(x) \in span\{ \beta_i \}\) (ie: image of T contained in this span).  Let \(\beta = \{\beta_i\}_{i=1}^n\) be a basis for this (possibly different) grade k multivector subspace.

Additionally introduce a set of respective reciprocal frames \(\{\sigma^i\}\), and \(\{\beta^i\}\).
Define the reciprocal frame with respect to the dot product for the space.  For a linearly independent, but not necessary orthogonal (or orthonormal), set of vectors \(\{u_i\}\) this set
has as its defining property:

\begin{equation}\label{eqn:matrixOfLinearTx:20}
u_i \cdot u^j = \delta_{ij}
\end{equation}

I have chosen to use this covariant, contravariant coordinate notation since that works well for both vectors (not necessarily orthogonal or orthonormal), as well as higher grade vectors.  When the basis is orthonormal these reciprocal frame grade k multivectors can be computed with just reversion.  For example, suppose that \(\{\beta_i\}\) is an orthonormal bivector basis for the image of \(T\), then the reciprocal frame bivectors are just \(\beta^i = {\beta_i}^\dagger\).

With this we can decompose the linear transformation into components generated by each of the \(\sigma_i\) grade k multivectors:

\begin{equation}
T(x) = T(\sum x \cdot \sigma_j \sigma^j) = \sum x_j T(\sigma^j)
\end{equation}

This we can write as a matrix equation:

\begin{equation}
T(x) =
\begin{bmatrix}
T(\sigma^1) & T(\sigma^1) & \cdots & T(\sigma^n)
\end{bmatrix}
\begin{bmatrix}
x_1 \\
x_2 \\
\vdots \\
x_n \\
\end{bmatrix}
\end{equation}

Now, decompose the \(T(\sigma^j)\) in terms of the basis \(\beta\):

\begin{equation}
T(\sigma^j) = \sum T(\sigma^j) \cdot \beta^i \beta_i
\end{equation}

This we can also write as a matrix equation

\begin{equation}
\begin{aligned}
T(\sigma^j) &=
\begin{bmatrix}
\beta_1 & \beta_2 & \cdots & \beta_m
\end{bmatrix}
\begin{bmatrix}
T(\sigma^j) \cdot \beta^1 \\
T(\sigma^j) \cdot \beta^2 \\
\vdots \\
T(\sigma^j) \cdot \beta^m \\
\end{bmatrix} \\
&=
\begin{bmatrix}
\beta_1 & \beta_2 & \cdots & \beta_m
\end{bmatrix}
\begin{bmatrix}
\beta^1 \cdot T(\sigma^j) \\
\beta^2 \cdot T(\sigma^j) \\
\vdots \\
\beta^m \cdot T(\sigma^j) \\
\end{bmatrix}
\end{aligned}
\end{equation}

These two sets of matrix equations, can be combined into a single equation:

\begin{equation}\label{eqn:matOfLinTx:matrixexpansion}
T(x) =
{
\begin{bmatrix}
\beta_1 \\
\beta_2 \\
\vdots \\
\beta_m
\end{bmatrix}
}^{\text{T}}
\begin{bmatrix}
\beta^1 \cdot T(\sigma^1) & \beta^1 \cdot T(\sigma^2) & \cdots & \beta^1 \cdot T(\sigma^n) \\
\beta^2 \cdot T(\sigma^1) & \beta^2 \cdot T(\sigma^2) & \cdots & \beta^2 \cdot T(\sigma^n) \\
\vdots & \cdots & \ddots & \vdots \\
\beta^m \cdot T(\sigma^1) & \beta^m \cdot T(\sigma^2) & \cdots & \beta^m \cdot T(\sigma^n) \\
\end{bmatrix}
\begin{bmatrix}
x_1 \\
x_2 \\
\vdots \\
x_n \\
\end{bmatrix}
\end{equation}

Here the matrix \(( x_1, x_2, \cdots, x_n )\) is a coordinate vector with respect to basis \(\sigma\), but the vector
\(( \beta_1, \beta_2, \cdots, \beta_n )\) is matrix of the basis vectors \(\beta_i \in \beta\).  This makes sense since the end result has not been defined in terms of
a coordinate vector space, but the space of T itself.

This can all be written more compactly as

\begin{equation}
T(x)
=
{
\begin{bmatrix}
\beta_i \\
\end{bmatrix}
}^{\text{T}}
\begin{bmatrix}
\beta^i \cdot T(\sigma^j)
\end{bmatrix}
\begin{bmatrix}
x_i \\
\end{bmatrix}
\end{equation}

We can also recover the original result from this by direct expansion and then regrouping:

\begin{equation}\label{eqn:matrixOfLinearTx:40}
\begin{aligned}
{
\begin{bmatrix}
\beta_i \\
\end{bmatrix}
}^{\text{T}}
\begin{bmatrix}
\sum_j \beta^i \cdot T(\sigma^j) x_j
\end{bmatrix}
&=
\begin{bmatrix}
\sum_{kj} \beta_k \beta^k \cdot T(\sigma^j) x_j
\end{bmatrix} \\
&=
\sum_{kj} \beta_k \beta^k \cdot T(\sigma^j) x_j \\
&=
\sum_{k} \beta_k \beta^k \cdot T(\sum_j \sigma^j x_j) \\
&=
\sum_{k} \beta_k \beta^k \cdot T(x) \\
&=
T(x) \\
\end{aligned}
\end{equation}

Observe that this demonstrates that we can write the coordinate vector \([T]_\beta\) as the two left most matrices above

\begin{equation}\label{eqn:matrixOfLinearTx:60}
\begin{aligned}
[T(x)]_\beta
&=
{
\begin{bmatrix}
\beta^i \cdot T(x)
\end{bmatrix}
}_i \\
&=
{
\begin{bmatrix}
\sum_{j} \beta^i \cdot T(\sigma^j) x_j
\end{bmatrix}
}_i \\
&=
\begin{bmatrix}
\beta^i \cdot T(\sigma^j)
\end{bmatrix}
\begin{bmatrix}
x_i \\
\end{bmatrix}
\end{aligned}
\end{equation}

Looking at the above I found it interesting that \eqnref{eqn:matOfLinTx:matrixexpansion} which embeds the coordinate vector of \(T(x)\) has the structure of a bilinear form, so in a sense one can view the
matrix of a linear transformation:

\begin{equation}
[T]_\sigma^\beta =
\begin{bmatrix}
\beta^i \cdot T(\sigma^j)
\end{bmatrix}
\end{equation}

as a bilinear form that can act as a mapping from the generating basis to the image basis.

\part{Rotation.}
%
% Copyright � 2012 Peeter Joot.  All Rights Reserved.
% Licenced as described in the file LICENSE under the root directory of this GIT repository.
%

%
%
\chapter{Rotor Notes}\label{chap:rotor}
\index{rotor}
%\date{Feb 19, 2008.  rotor.tex}

\section{Rotations strictly in a plane}

For a plane rotation, a rotation does not have to
be expressed in terms of left and right half angle rotations, as is the case
with complex numbers.  Starting with this ``natural'' one sided rotation
we will see why the half angle double sided Rotor formula works.

\subsection{Identifying a plane with a bivector.  Justification}
Given a bivector \(\BB\), we can say this defines the orientation of a plane
(through the origin)
since for any vector in the plane we have \(\BB \wedge \Bx = 0\), or any vector
strictly normal to the plane \(\BB \cdot \Bx = 0\).

Note that this naturally compares
to the equation of a line (through the origin) expressed in terms of a
direction vector \(\Bb\),
where \(\Bb \wedge \Bx=0\) if \(\Bx\) lies on the line, and \(\Bb \cdot \Bx = 0\)
if \(\Bx\) is normal to the line.

Given this it is not unreasonable to identify the plane with its bivector.  This
will be done below, and it should be clear that
loose language such as ``the plane \(\BB\)'', should really be interpreted
as ``the plane with direction bivector \(\BB\)'', where the direction bivector
has the wedge and dot product properties noted above.

\subsection{Components of a vector in and out of a plane}

To calculate the components of a vector in and out of a plane, we can form
the product

\begin{equation}\label{eqn:rotor:20}
\Bx = \Bx \BB \inv{\BB} = \Bx \cdot \BB \inv{\BB} + \Bx \wedge \BB \inv{\BB}
\end{equation}

This is an orthogonal decomposition of the vector \(\Bx\) where the first
part is the projective term onto the plane \(\BB\), and the second is the rejective
term, the component not in the plane.  Let us verify this.

Write \(\Bx = \Bx_\parallel + \Bx_\perp\), where \(\Bx_\parallel\), and \(\Bx_\perp\) are the components of \(\Bx\) parallel and perpendicular to the plane.  Also write
\(\BB = \Bb_1 \wedge \Bb_2\), where \(\Bb_i\) are non-colinear vectors in the plane \(\BB\).

If \(\Bx = \Bx_\parallel\), a vector entirely in the plane \(\BB\), then one can
write

\begin{equation}\label{eqn:rotor:40}
\Bx = a_1\Bb_1 + a_2\Bb_2
\end{equation}

and the wedge product term is zero

\begin{equation}\label{eqn:rotor:740}
\begin{aligned}
\Bx \wedge \BB
&= \left( a_1\Bb_1 + a_2\Bb_2 \right) \wedge \Bb_1 \wedge \Bb_2 \\
&= a_1 ( \Bb_1 \wedge \Bb_1 ) \wedge \Bb_2
 - a_2 ( \Bb_2 \wedge \Bb_2 ) \wedge \Bb_1 \\
&= 0
\end{aligned}
\end{equation}

Thus the component parallel to the plane is composed strictly of the dot
product term

\begin{equation}
\Bx_\parallel = \Bx \cdot \BB \inv{\BB}
\end{equation}

Or for a general vector not necessarily in the plane the component
of that vector in the plane, its projection onto the plane is,

\begin{equation}\label{eqn:rotor:60}
\Proj_{\BB}(\Bx) = \Bx \cdot \BB \inv{\BB}
= \inv{\abs{\BB}^2}(\BB \cdot \Bx)\BB
= (\hat{\BB} \cdot \Bx)\hat{\BB}
\end{equation}

Now, for a vector that lies completely perpendicular to the plane \(\Bx = \Bx_\perp\), the dot product term with the plane is zero.  To verify this observe

\begin{equation}\label{eqn:rotor:760}
\begin{aligned}
\Bx_\perp \cdot \BB
&= \Bx_\perp \cdot (\Bb_1 \wedge \Bb_2) \\
&= (\Bx_\perp \cdot \Bb_1) \Bb_2 - (\Bx_\perp \cdot \Bb_2) \Bb_1 \\
\end{aligned}
\end{equation}

Each of these dot products are zero since \(\Bx\) has no components that lie
in the plane (those components if they existed could be expressed as linear
combinations of \(\Bb_i\)).

Thus only the component perpendicular to the plane is composed strictly of the
wedge product term

\begin{equation}
\Bx_\perp = \Bx \wedge \BB \inv{\BB}
\end{equation}

And again for a general vector the component that lies out
of the plane as, the rejection of the plane from the vector is

\begin{equation}\label{eqn:rotor:80}
\RejName_{\BB}(\Bx)
= \Bx \wedge \BB \inv{\BB}
= -\inv{\abs{\BB}^2} \Bx \wedge \BB {\BB}
= -\Bx \wedge \hat{\BB} \hat{\BB}
\end{equation}

\section{Rotation around normal to arbitrarily oriented plane through origin}

Having established the preliminaries, we can now express a rotation around
the normal to a plane (with the plane and that normal through the origin).

\imageFigure{../../figures/gabook/rotor}{Rotation of Vector}{fig:rotor}{0.4}

Such a rotation is illustrated in \cref{fig:rotor}
preserves all components of the vector that are perpendicular
to the plane, and operates only on the components parallel to the plane.

Expressed in terms of exponentials and the projective and rejective decompositions above, this is

\begin{equation}\label{eqn:rotor:780}
\begin{aligned}
R_\theta(\Bx)
&= \Bx \wedge \BB \inv{\BB} + \left(\Bx \cdot \BB \inv{\BB}\right)e^{\hat{\BB}\theta} \\
&= \Bx \wedge \BB \inv{\BB} + e^{-\hat{\BB}\theta}\left(\Bx \cdot \BB \inv{\BB}\right) \\
\end{aligned}
\end{equation}

Where we have made explicit note that a plane rotation does not commute with a vector in a plane (its reverse is required).

To demonstrate this write \(i = \Be_2 \Be_1\), a unit bivector in some plane with unit vectors \(\Be_i\) also in the plane.  If a vector
lies in that plane we can write the rotation

\begin{equation}\label{eqn:rotor:800}
\begin{aligned}
\Bx e^{i\theta}
&= \left(a_1\Be_1 + a_2\Be_2\right)\left(\cos\theta + i\sin\theta\right) \\
&= \cos\theta\left(a_1\Be_1 + a_2\Be_2\right) + \left(a_1\Be_1 + a_2\Be_2\right)\left(\Be_2 \Be_1\sin\theta\right) \\
&= \cos\theta\left(a_1\Be_1 + a_2\Be_2\right) + \sin\theta \left(-a_1\Be_2 + a_2\Be_1\right) \\
&= \cos\theta\left(a_1\Be_1 + a_2\Be_2\right) -\Be_2 \Be_1\sin\theta \left(a_1\Be_1 + a_2\Be_2\right) \\
&= e^{-i\theta}\Bx \\
\end{aligned}
\end{equation}

Similarly for a vector that lies outside of the plane we can write

\begin{equation}\label{eqn:rotor:820}
\begin{aligned}
\Bx e^{i\theta}
&= (\sum_{j \ne 1,2} a_j \Be_j)(\cos\theta + \Be_2 \Be_1\sin\theta) \\
&= (\cos\theta + \Be_2 \Be_1\sin\theta) (\sum_{j \ne 1,2} a_j \Be_j) \\
&= e^{i\theta}\Bx
\end{aligned}
\end{equation}

The multivector for a rotation in a plane perpendicular to a vector commutes with that vector.  The properties of the
exponential allow us to factor a rotation

\begin{equation}\label{eqn:rotor:100}
R(\theta) = R(\alpha\theta) R((1-\alpha)\theta)
\end{equation}

where \(\alpha <= 1\), and in particular we can set \(\alpha = 1/2\), and write

\begin{equation}\label{eqn:rotor:840}
\begin{aligned}
R_\theta(\Bx)
&= \Bx \wedge \BB \inv{\BB} + \left(\Bx \cdot \BB \inv{\BB}\right)e^{\hat{\BB}\theta} \\
&= \left(\Bx \wedge \BB \inv{\BB}\right) e^{-\hat{\BB}\theta/2} e^{\hat{\BB}\theta/2}
 + \left(\Bx \cdot \BB \inv{\BB} \right) e^{\hat{\BB}\theta/2} e^{\hat{\BB}\theta/2} \\
&= e^{-\hat{\BB}\theta/2} \left(\Bx \wedge \BB \inv{\BB}\right) e^{\hat{\BB}\theta/2}
+ e^{-\hat{\BB}\theta/2} \left(\Bx \cdot \BB \inv{\BB}\right)e^{\hat{\BB}\theta/2} \\
&= e^{-\hat{\BB}\theta/2} \left(\Bx \wedge \BB + \Bx \cdot \BB\right) \inv{\BB} e^{\hat{\BB}\theta/2} \\
&= e^{-\hat{\BB}\theta/2} \left(\Bx \BB \inv{\BB} \right) e^{\hat{\BB}\theta/2}
\end{aligned}
\end{equation}

This takes us full circle from dot and wedge products back to \(\Bx\), and allows us to express the rotated vector as:

\begin{equation}\label{eqn:rotor:rotor}
R_\theta(\Bx)
= e^{-\hat{\BB}\theta/2} \Bx e^{\hat{\BB}\theta/2}
\end{equation}

Only when the vector lies in the plane (\(\Bx = \Bx_\parallel\), or \(\Bx \wedge \BB = 0\)) can be written using the familiar left or right ``full angle'' rotation exponential that we are used to from complex arithmetic:

\begin{equation}\label{eqn:rotor:120}
R_\theta(\Bx) = e^{-\hat{\BB}\theta} \Bx = \Bx e^{\hat{\BB}\theta}
\end{equation}

\section{Rotor equation in terms of normal to plane}

The rotor equation above is valid for any number of dimensions.  For \R{3} we can alternatively parametrize the plane in terms of
a unit normal \(\Bn\):

\begin{equation}\label{eqn:rotor:140}
\BB = k i\Bn
\end{equation}

Here \(i\) is the \R{3} pseudoscalar \(\Be_1 \Be_2 \Be_3\).

Thus we can write

\begin{equation}\label{eqn:rotor:160}
\hat{\BB} = i\Bn
\end{equation}

and expressing \eqnref{eqn:rotor:rotor} in terms of the unit normal becomes trivial

\begin{equation}
R_\theta(\Bx)
= e^{- i {\Bn}\theta/2} \Bx e^{i{\Bn}\theta/2}
\end{equation}

Expressing this in terms of components and the unit normal is a bit harder

\begin{equation}\label{eqn:rotor:860}
\begin{aligned}
R_\theta(\Bx)
&= \Bx \wedge \BB \inv{\BB} + \left(\Bx \cdot \BB \inv{\BB}\right)e^{\hat{\BB}\theta} \\
&= \Bx \wedge (i\Bn) \inv{i\Bn} + \left(\Bx \cdot (i\Bn) \inv{i\Bn}\right)e^{{i\Bn}\theta} \\
\end{aligned}
\end{equation}

Now,

\begin{equation}\label{eqn:rotor:880}
\begin{aligned}
\Bx \wedge (i\Bn)
&= \inv{2}(\Bx i \Bn + i \Bn \Bx) \\
&= \frac{i}{2}(\Bx \Bn + \Bn \Bx) \\
&= (\Bx \cdot \Bn) i
\end{aligned}
\end{equation}

And

\begin{equation}\label{eqn:rotor:900}
\begin{aligned}
\inv{i\Bn}
&= \inv{i\Bn} \inv{\Bn i} \Bn i \\
&= - i \Bn \\
\end{aligned}
\end{equation}

So the rejective term becomes
\begin{equation}\label{eqn:rotor:920}
\begin{aligned}
\Bx \wedge \BB \inv{\BB}
&= \Bx \wedge (i\Bn) \inv{i\Bn} \\
&= \Bx \wedge (i\Bn) \inv{i\Bn} \\
&= (\Bx \cdot \Bn) i (-i) \Bn \\
&= (\Bx \cdot \Bn) \Bn \\
&= \Proj_{\Bn}(\Bx) \\
\end{aligned}
\end{equation}

Now, for the dot product with the plane term, we have

\begin{equation}\label{eqn:rotor:940}
\begin{aligned}
\Bx \cdot \BB
&= \Bx \cdot (i \Bn) \\
&= \inv{2}(\Bx i \Bn - i \Bn \Bx) \\
&= (\Bx \wedge \Bn)i \\
\end{aligned}
\end{equation}

Putting it all together we have

\begin{equation}\label{eqn:rotor:rotexp}
R_\theta(\Bx)
= (\Bx \cdot \Bn) \Bn + (\Bx \wedge \Bn)\Bn e^{{i\Bn}\theta}
\end{equation}

In terms of explicit sine and cosine terms this is (observe that \((i\Bn)^2 = -1\)),

\begin{equation}\label{eqn:rotor:960}
\begin{aligned}
R_\theta(\Bx)
&= \left(\Bx \cdot \Bn\right) \Bn + \left(\Bx \wedge \Bn\right)\Bn \left(\cos\theta + i\Bn \sin\theta\right) \\
\end{aligned}
\end{equation}

\begin{equation}\label{eqn:rotor:rotnorm}
R_\theta(\Bx) =
\left(\Bx \cdot \Bn\right) \Bn + \left(\Bx \wedge \Bn\right)\Bn \cos\theta + (\Bx \wedge \Bn) i \sin\theta
\end{equation}

\imageFigure{../../figures/gabook/normalRot}{Direction vectors associated with rotation}{fig:normalRot}{0.4}

This triplet of mutually orthogonal direction vectors,
\(\Bn\), \((\Bx \wedge \Bn)\Bn\), and \((\Bx \wedge \Bn) i\)
are illustrated in \cref{fig:normalRot}.  The component of the vector in the direction of the normal
\(\Proj_\Bn(\Bx) = \Bx \cdot \Bn \Bn\) is unaltered by the rotation.
The rotation is applied to the remaining component of \(\Bx\), \(\RejName_{\Bn}(\Bx) = (\Bx \wedge \Bn)\Bn\), and we rotate
in the direction \((\Bx \wedge \Bn) i\)

\subsection{Vector rotation in terms of dot and cross products only}

Expression of this rotation formula \eqnref{eqn:rotor:rotnorm} in terms of ``vector'' relations is also possible, by removing the wedge
products and the pseudoscalar references.

First the rejective term

\begin{equation}\label{eqn:rotor:980}
\begin{aligned}
(\Bx \wedge \Bn) \Bn
&= ((\Bx \cross \Bn) i) \Bn \\
&= ((\Bx \cross \Bn) i) \cdot \Bn \\
&= \inv{2} ( ((\Bx \cross \Bn) i) \Bn - \Bn ((\Bx \cross \Bn) i)) \\
&= \frac{i}{2} ( (\Bx \cross \Bn) \Bn - \Bn (\Bx \cross \Bn) ) \\
&= i ( (\Bx \cross \Bn) \wedge \Bn ) \\
&= i^2 ( (\Bx \cross \Bn) \cross \Bn ) \\
&= \Bn \cross (\Bx \cross \Bn) \\
\end{aligned}
\end{equation}

The next term expressed in terms of the cross product is

\begin{equation}\label{eqn:rotor:1000}
\begin{aligned}
(\Bx \wedge \Bn) i
&=
(\Bx \cross \Bn) i^2 \\
&= \Bn \cross \Bx \\
\end{aligned}
\end{equation}

And putting it all together we have

\begin{equation}\label{eqn:rotor:rotcross}
R_\theta(\Bx) =
\left(\Bx \cdot \Bn\right) \Bn
 + \left(\Bn \cross \Bx\right) \cross \Bn \cos\theta
 + \Bn \cross \Bx \sin\theta
\end{equation}

Compare \eqnref{eqn:rotor:rotcross} to \eqnref{eqn:rotor:rotnorm} and \eqnref{eqn:rotor:rotexp}, and then back to \eqnref{eqn:rotor:rotor}.

\section{Giving a meaning to the sign of the bivector}

For a rotation between two vectors in the plane containing those vectors, we can write the rotation
in terms of the exponential as either a left or right rotation operator:

\begin{equation}\label{eqn:rotor:180}
\Bb = \Ba e^{\Bi\theta} = e^{-\Bi\theta}\Ba
\end{equation}
\begin{equation}\label{eqn:rotor:200}
\Bb = e^{\Bj\theta}\Ba = \Ba e^{-\Bj\theta/2}
\end{equation}

Here both \(\Bi\) and \(\Bj=-\Bi\) are unit bivectors with the property \(\Bi^2 = \Bj^2 = -1\).
Thus in order to write a rotation in exponential form a meaning must be assigned to the sign of the unit bivector that describes the
plane and the orientation of the rotation.

Consider for example the case of a rotation by \(\pi/2\).  For this is the exponential is:

\begin{equation}\label{eqn:rotor:220}
e^{\Bi\pi/2} = \cos(\pi/2) + \Bi \sin(\pi/2) = \Bi
\end{equation}

Thus for perpendicular unit vectors \(\Bu\) and \(\Bv\), if we wish \(\Bi\) to act as a \(\pi/2\) rotation left acting operator on \(\Bu\)
towards \(\Bv\) its value must be:

\begin{equation}\label{eqn:rotor:240}
\Bi = \Bu \wedge \Bv
\end{equation}
\begin{equation}\label{eqn:rotor:260}
\Bu\Bi = \Bu \Bu \wedge \Bv = \Bu\Bu\Bv = \Bv
\end{equation}

For that same rotation if the bivector is employed as a right acting operator, the reverse is required:

\begin{equation}\label{eqn:rotor:280}
\Bj = \Bv \wedge \Bu
\end{equation}
\begin{equation}\label{eqn:rotor:300}
\Bj\Bu = \Bv \wedge \Bu = \Bv\Bu\Bu = \Bv
\end{equation}

\imageFigure{../../figures/gabook/imaginaryorientation}{Orientation of unit imaginary}{fig:imaginaryorientation}{0.4}

In general, for any two vectors, one can find an angle \(\theta\) in the range \(0 \le \theta \le \pi\) between those vectors.
If one lets that angle define the orientation of the rotation between the vectors, and implicitly
define a sort of ``imaginary axis'' for that plane, that imaginary axis will have direction

\begin{equation}\label{eqn:rotor:320}
\inv{\Ba} \Ba \wedge \Bb = \Bb \wedge \Ba \inv {\Ba}.
\end{equation}

This is illustrated in \cref{fig:imaginaryorientation}.

Thus the bivector

\begin{equation}\label{eqn:rotor:340}
\Bi = \frac{\Ba \wedge \Bb}{\abs{\Ba \wedge \Bb}}
\end{equation}

When acting as an operator to the left (\(\Ba \Bi\)) with a vector in the plane can be interpreted as acting as a rotation by \(\pi/2\) towards \(\Bb\).

Similarly the bivector

\begin{equation}\label{eqn:rotor:360}
\Bj = \Bi^\dagger = -\Bi = \frac{\Bb \wedge \Ba}{\abs{\Bb \wedge \Ba}}
\end{equation}

also applied to a vector in the plane produces the same rotation when
acting as an operator to the right.  Thus, in general we can write
a rotation by theta in the plane containing non-colinear vectors \(\Ba\) and \(\Bb\) in the direction of minimal angle
from \(\Ba\) towards \(\Bb\) in one of the three forms:

\begin{equation}\label{eqn:rotor:380}
R_{\theta : \Ba \rightarrow \Bb}(\Ba)
= \Ba e^{ \frac{\Ba \wedge \Bb}{\abs{\Ba \wedge \Bb}} \theta }
= e^{ \frac{\Bb \wedge \Ba}{\abs{\Bb \wedge \Ba}} \theta } \Ba
\end{equation}

Or,
\begin{equation}\label{eqn:rotor:400}
R_{\theta : \Ba \rightarrow \Bb}(\Bx)
= e^{ \frac{\Bb \wedge \Ba}{\abs{\Bb \wedge \Ba}} \theta/2 } \Bx e^{ \frac{\Ba \wedge \Bb}{\abs{\Ba \wedge \Bb}} \theta/2 }
\end{equation}

This last (writing \(\Bx\) instead of \(\Ba\) since it also applies to vectors that lie outside of the \(\Ba \wedge \Bb\) plane),
is our rotor formula \eqnref{eqn:rotor:rotor}, reexpressed in a way that removes the sign ambiguity of the bivector \(\Bi\) in that equation.

\section{Rotation between two unit vectors}

\imageFigure{../../figures/gabook/parallelogramvec}{Sum of unit vectors bisects angle between}{fig:parallelogramvec}{0.4}

As illustrated in \cref{fig:parallelogramvec}, when the angle between two vectors is less than \(\pi\)
the fact that the sum of two arbitrarily oriented unit vectors bisects those vectors provides a convenient
way to compute the half angle rotation exponential.

Thus we can write

\begin{equation*}
\frac{\Ba + \Bb}{\abs{\Ba + \Bb}} = \Ba e^{\Bi\theta/2} = e^{\Bj\theta/2} \Ba
\end{equation*}

Where \(\Bi = \Bj^\dagger\) are unit bivectors of appropriate sign.  Multiplication through by \(\Ba\) gives

\begin{equation*}
e^{\Bi\theta/2} =
\frac{1 + \Ba\Bb}{\abs{\Ba + \Bb}}
\end{equation*}

Or,
\begin{equation*}
e^{\Bj\theta/2} =
\frac{1 + \Bb\Ba}{\abs{\Ba + \Bb}}
\end{equation*}

Thus we can write the total rotation from \(\Ba\) to \(\Bb\) as

\begin{equation*}
\Bb
= e^{-\Bi\theta/2} \Ba e^{\Bi\theta/2}
= e^{\Bj\theta/2} \Ba e^{-\Bj\theta/2}
= \left(\frac{1 + \Bb\Ba}{\abs{\Ba + \Bb}}\right) \Ba \left(\frac{1 + \Ba\Bb}{\abs{\Ba + \Bb}}\right)
\end{equation*}

For the case where the rotation is through an angle \(\theta\) where \(\pi < \theta < 2\pi\), again employing a left acting
exponential operator we have

\begin{equation}\label{eqn:rotor:1020}
\begin{aligned}
\frac{\Ba + \Bb}{\abs{\Ba + \Bb}}
&= \Bb e^{\Bi(2\pi - \theta)/2} \\
&= \Bb e^{\Bi \pi} e^{- \Bi\theta/2} \\
&= -\Bb e^{- \Bi\theta/2} \\
\end{aligned}
\end{equation}

Or,
\begin{equation}\label{eqn:rotor:420}
e^{- \Bi\theta/2} = -\frac{\Bb\Ba + 1}{\abs{\Ba + \Bb}}
\end{equation}

Thus

\begin{equation}\label{eqn:rotor:rotunit}
\Bb = e^{- \Bi\theta/2} \Ba e^{ \Bi\theta/2} =
\left(-\frac{1 + \Bb\Ba}{\abs{\Ba + \Bb}}\right) \Ba \left(-\frac{1 + \Ba\Bb}{\abs{\Ba + \Bb}}\right)
\end{equation}

Note that the two negatives cancel, giving the same result as in the \(\theta < \pi\) case.  Thus \eqnref{eqn:rotor:rotunit} is valid for all vectors \(\Ba \ne -\Bb\) (this can be verified by direct multiplication.)

These
half angle exponentials are called rotors, writing the rotor as

\begin{equation}\label{eqn:rotor:440}
R = \frac{1 + \Ba\Bb}{\abs{\Ba + \Bb}}
\end{equation}

and the rotation in terms of rotors is:

\begin{equation}\label{eqn:rotor:460}
\Bb = R^\dagger \Ba R
\end{equation}

The angle associated with this rotor \(R\) is the minimal angle between the two vectors (\(0 < \theta < \pi\)), and is directed from \(\Ba\) to \(\Bb\).  Inverting the rotor will not change the net effect of the rotation, but has the geometric meaning that the rotation from \(\Ba\) to \(\Bb\)
rotates in the opposite direction through the larger angle (\(\pi < \theta < 2\pi\)) between the vectors.

\section{Eigenvalues, vectors and coordinate vector and matrix of the rotation linear transformation}

Given the plane containing two orthogonal vectors \(\Bu\) and \(\Bv\), we can form a unit bivector for the plane

\begin{equation}\label{eqn:rotor:480}
\BB = \Bu\Bv
\end{equation}

A normal to this plane is \(\Bn = \Bv\Bu I\).

The rotation operator for a rotation around \(\Bn\) in that plane (directed from \(\Bu\) towards \(\Bv\)) is

\begin{equation}\label{eqn:rotor:500}
R_\theta(\Bx) = e^{\Bv\Bu \theta/2} \Bx e^{\Bu\Bv \theta/2}
\end{equation}

To form the matrix of this linear transformation assume an orthonormal basis \(\sigma = \{ \Be_i \}\).

In terms of these basis vectors we can write

\begin{equation}\label{eqn:rotor:520}
R_\theta(\Be_j) =
e^{-\Bv\Bu \theta/2} \Be_j e^{\Bu\Bv \theta/2}
=
\sum_i \left(e^{-\Bv\Bu \theta/2} \Be_j e^{\Bu\Bv \theta/2}\right) \cdot \Be_i \Be_i
\end{equation}

Thus the coordinate vector for this basis is

\begin{equation}\label{eqn:rotor:540}
{
\begin{bmatrix}
R_\theta(\Be_j)
\end{bmatrix}
}_\sigma
=
\begin{bmatrix}
\left(e^{-\Bv\Bu \theta/2} \Be_j e^{\Bu\Bv \theta/2}\right) \cdot \Be_1 \\
\vdots \\
\left(e^{-\Bv\Bu \theta/2} \Be_j e^{\Bu\Bv \theta/2}\right) \cdot \Be_n \\
\end{bmatrix}
\end{equation}

We can use this to form the matrix for the linear operator that takes coordinate vectors from
the basis \(\sigma\) to \(\sigma\):

\begin{equation}\label{eqn:rotor:560}
{
\begin{bmatrix}
R_\theta(\Bx)
\end{bmatrix}
}_\sigma
=
{
\begin{bmatrix}
R_\theta
\end{bmatrix}
}_\sigma^\sigma
{
\begin{bmatrix}
\Bx
\end{bmatrix}
}_\sigma
\end{equation}

Where
\begin{equation}\label{eqn:rotor:rotcoords}
{
\begin{bmatrix}
R_\theta
\end{bmatrix}
}_\sigma^\sigma
=
\begin{bmatrix}
{
\begin{bmatrix}
R_\theta(\Be_1)
\end{bmatrix}
}_\sigma
\hdots
{
\begin{bmatrix}
R_\theta(\Be_n)
\end{bmatrix}
}_\sigma
\end{bmatrix}
=
{
\begin{bmatrix}
\left(e^{-\Bv\Bu \theta/2} \Be_j e^{\Bu\Bv \theta/2}\right) \cdot \Be_i \\
\end{bmatrix}
}_{ij}
\end{equation}

If one uses the plane and its normal to form an alternate orthonormal basis
\(\alpha = \{\Bu, \Bv, \Bn\}\).

The transformation matrix for coordinate vectors in this basis is

\begin{equation}\label{eqn:rotor:580}
{
\begin{bmatrix}
R_\theta
\end{bmatrix}
}_\alpha^\alpha
=
\begin{bmatrix}
\left(\Bu e^{\Bu\Bv \theta}\right) \cdot \Bu & \left(\Bv e^{\Bu\Bv \theta}\right) \cdot \Bu & 0 \\
\left(\Bu e^{\Bu\Bv \theta}\right) \cdot \Bv & \left(\Bv e^{\Bu\Bv \theta}\right) \cdot \Bv & 0 \\
0 & 0 & \Bn\cdot\Bn \\
\end{bmatrix}
=
\begin{bmatrix}
\cos\theta & -\sin\theta & 0 \\
\sin\theta & \cos\theta & 0 \\
0 & 0 & 1 \\
\end{bmatrix}
\end{equation}

This matrix has eigenvalues \(e^{i\theta}, e^{-i\theta}, 1\), with (coordinate) eigenvectors

\begin{equation}\label{eqn:rotor:600}
\inv{\sqrt{2}}
\begin{bmatrix}
1 \\
-i \\
0 \\
\end{bmatrix},
\inv{\sqrt{2}}
\begin{bmatrix}
1 \\
i \\
0 \\
\end{bmatrix},
\begin{bmatrix}
0 \\
0 \\
1 \\
\end{bmatrix}
\end{equation}

Its interesting to observe that without introducing coordinate vectors an eigensolution is possible directly from
the linear transformation itself.

The rotation linear operator has right and left eigenvalues \(e^{\Bu\Bv \theta}\), \(e^{\Bv\Bu \theta}\) (respectively), where the eigenvectors for these are any vectors in the plane.  There is also a scalar eigenvalue \(1\) (both left and right eigenvalue), for the eigenvector \(\Bn\):

\begin{equation}\label{eqn:rotor:1040}
\begin{aligned}
R_\theta(\Bu) &= e^{\Bv \Bu \theta} \Bx = \Bx e^{\Bu \Bv \theta} \\
R_\theta(\Bu) &= e^{\Bv \Bu \theta} \Bx = \Bx e^{\Bu \Bv \theta} \\
R_\theta(\Bn) &= \Bn (1) \\
\end{aligned}
\end{equation}

Observe that the eigenvalues here are not all scalars, which is likely related
to the fact that the coordinate matrix was not diagonalizable with real vectors.

the matrix of the linear transformation.
Given this, one can write:

\begin{equation}\label{eqn:rotor:1060}
\begin{aligned}
\begin{bmatrix}
R_\theta(\Bu) & R_\theta(\Bv) & R_\theta(\Bn) \\
\end{bmatrix}
&=
\begin{bmatrix}
\Bu & \Bv & \Bn \\
\end{bmatrix}
\begin{bmatrix}
e^{\Bu \Bv \theta} & 0 & 0 \\
0 & e^{\Bu \Bv \theta} & 0 \\
0 & 0 & 1 \\
\end{bmatrix} \\
&=
\begin{bmatrix}
e^{\Bv \Bu \theta} & 0 & 0 \\
0 & e^{\Bv \Bu \theta} & 0 \\
0 & 0 & 1 \\
\end{bmatrix}
\begin{bmatrix}
\Bu & \Bv & \Bn \\
\end{bmatrix}
\end{aligned}
\end{equation}

But neither of these can be used to diagonalize the matrix of the transformation.  To do that
we require dot products that span the matrix product to form the coordinate vector columns.

Observe that interestingly
enough the left and right eigenvalues of the operator in the plane are of complex exponential form (\(e^{\pm \Bn I \theta}\)) just as the eigenvalues for
coordinate vectors restricted to the plane are complex exponentials (\(e^{\pm i\theta}\)).
%This suggests that a basis for a quaternion
%like space (0-2 multivectors) will be required to diagonalize a rotation operator.

\section{matrix for rotation linear transformation}

Let us expand the terms in \eqnref{eqn:rotor:rotcoords} to calculate explicitly the rotation matrix for an arbitrary
rotation.  Also, as before, write \(\Bn = \Bv\Bu I\), and parametrize the Rotor as follows:

\begin{equation}\label{eqn:rotor:620}
R = e^{\Bn I \theta/2} = \cos{\theta/2} + \Bn I \sin{\theta/2} = \alpha + I\Bbeta
\end{equation}

Thus the \(ij\) terms in the matrix are:

\begin{equation}\label{eqn:rotor:1080}
\begin{aligned}
\Be_i \cdot \left(e^{-\Bn I \theta/2} \Be_j e^{\Bn I \theta/2}\right)
&= \langle{ \Be_i (\alpha -I\Bbeta) \Be_j (\alpha +I\Bbeta) } \rangle \\
&= \langle{ \Be_i (\Be_j \alpha -I\Bbeta\Be_j) (\alpha +I\Bbeta) } \rangle \\
&= \langle{ \Be_i \left( \Be_j \alpha^2 -I\alpha(\Bbeta\Be_j - \Be_j\Bbeta) + \Bbeta\Be_j\Bbeta \right) } \rangle \\
&= \delta_{ij}\alpha^2 + \langle{ \Be_i \left( -2I\alpha(\Bbeta \wedge \Be_j) + \Bbeta\Be_j\Bbeta \right) } \rangle \\
&= \delta_{ij}\alpha^2 + 2\alpha \Be_i \cdot (\Bbeta \cross \Be_j) + \langle{ \Be_i \Bbeta \Be_j \Bbeta } \rangle \\
\end{aligned}
\end{equation}

Lets expand the last term separately:
\begin{equation}\label{eqn:rotor:1100}
\begin{aligned}
\langle{ \Be_i \Bbeta \Be_j \Bbeta } \rangle
&= \langle{ ( \Be_i \cdot \Bbeta + \Be_i \wedge \Bbeta) ( \Be_j \cdot \Bbeta + \Be_j \wedge \Bbeta) } \rangle  \\
&= (\Be_i \cdot \Bbeta)(\Be_j \cdot \Bbeta) + \langle{ (\Be_i \wedge \Bbeta) ( \Be_j \wedge \Bbeta) } \rangle  \\
\end{aligned}
\end{equation}

And once more considering first the \(i=j\) case (writing \(s \ne i \ne t\)).

\begin{equation}\label{eqn:rotor:1120}
\begin{aligned}
\langle{ (\Be_i \wedge \Bbeta)^2 }\rangle
&= \lr{ \sum_{k \ne i}{ \Be_{ik} \beta_k} }^2 \\
&= ( \Be_{is} \beta_s + \Be_{it} \beta_t ) ( \Be_{is} \beta_s + \Be_{it} \beta_t ) \\
&= -\beta_s^2 -\beta_t^2 -  \Be_{st} \beta_s \beta_t + \Be_{ts} \beta_t \beta_s  \\
&= -\beta_s^2 -\beta_t^2 \\
&= -\Bbeta^2 + \beta_i^2 \\
\end{aligned}
\end{equation}

For the \(i \ne j\) term, writing \(i \ne j \ne k\)
\begin{equation}\label{eqn:rotor:1140}
\begin{aligned}
\langle{(\Be_i \wedge \Bbeta) (\Be_j \wedge \Bbeta)}\rangle
&= \langle{\sum_{s \ne i} \Be_{is} \beta_s\sum_{t \ne i} \Be_{it} \beta_t}\rangle \\
&= \langle{( \Be_{ij} \beta_j + \Be_{ik} \beta_k) ( \Be_{ji} \beta_i + \Be_{jk} \beta_k)}\rangle \\
&= \beta_i\beta_j + \langle{ \Be_{ji} \beta_k^2 +\Be_{ik} \beta_j \beta_k +\Be_{kj} \beta_k \beta_i }\rangle \\
&= \beta_i\beta_j \\
\end{aligned}
\end{equation}

Thus
\begin{equation}\label{eqn:rotor:640}
\langle{ (\Be_i \wedge \Bbeta) ( \Be_j \wedge \Bbeta) } \rangle
= \delta_{ij}(-\Bbeta^2 + \beta_i^2) + (1-\delta_{ij})\beta_i\beta_j
= \beta_i\beta_j -\delta_{ij}\Bbeta^2
\end{equation}

And putting it all back together
\begin{equation}\label{eqn:rotor:rotmgreek}
\Be_i \cdot \left(e^{-\Bn I \theta/2} \Be_j e^{\Bn I \theta/2}\right)
= \delta_{ij}(\alpha^2 -\Bbeta^2) + 2\alpha \Be_i \cdot (\Bbeta \cross \Be_j) + 2\beta_i\beta_j
\end{equation}


The \(\alpha\) and \(\beta\) terms can be expanded in terms of \(\theta\).
we see that The \(\delta_{ij}\) coefficient is

\begin{equation}\label{eqn:rotor:660}
\alpha^2 -\Bbeta^2 = 2{\cos}^2{\theta} -1 = \cos\theta.
\end{equation}

The triple product \(\Be_i \cdot (\Bbeta \cross \Be_j)\) is zero along the diagonal where \(i=j\) since an \(\Be_j=\Be_i\) cross has no \(\Be_i\) component, so
for \(k \ne i \ne j\), the triple product term is

\begin{equation}\label{eqn:rotor:1160}
\begin{aligned}
2\alpha \Be_i \cdot (\Bbeta \cross \Be_j)
&= 2\alpha \beta_k \Be_i \cdot (\Be_k \cross \Be_j) \\
&= 2\alpha \beta_k \Sgn{\pi_{ikj}} \\
&= 2 n_k \cos({\theta/2})\sin({\theta/2}) \Sgn{\pi_{ikj}} \\
&= n_k \sin{\theta} \Sgn{\pi_{ikj}} \\
\end{aligned}
\end{equation}

The last term is:
\begin{equation}\label{eqn:rotor:680}
2\beta_i\beta_j
= 2 n_i n_j {\sin}^2({\theta/2})
= n_i n_j (1-\cos\theta)
\end{equation}

Thus we can alternatively write \eqnref{eqn:rotor:rotmgreek}

\begin{equation}\label{eqn:rotor:rotmn}
\Be_i \cdot \left(e^{-\Bn I \theta/2} \Be_j e^{\Bn I \theta/2}\right)
= \delta_{ij}\cos\theta
+ n_k \sin{\theta} \epsilon_{ikj} + n_i n_j (1-\cos\theta)
\end{equation}

This is enough to easily and explicitly write out the complete rotation matrix for a rotation about unit vector \(\Bn = (n_1, n_2, n_3)\):
(with basis \(\sigma = \{\Be_i\}\)):

\begin{equation}\label{eqn:rotor:700}
[
R_\theta
]_\sigma^\sigma
=
\begin{bmatrix}
\cos\theta(1 -n_1^2) + n_1^2 & n_1 n_2 (1-\cos\theta) - n_3 \sin\theta & n_1 n_3 (1-\cos\theta) + n_2 \sin\theta \\
n_1 n_2 (1-\cos\theta) + n_3 \sin\theta & \cos\theta(1 -n_2^2) + n_2^2 & n_2 n_3 (1-\cos\theta) - n_1 \sin\theta \\
n_1 n_3 (1-\cos\theta) - n_2 \sin\theta & n_2 n_3 (1-\cos\theta) + n_1 \sin\theta & \cos\theta(1 -n_3^2) + n_3^2 \\
\end{bmatrix}
\end{equation}

Note also that the \(n_i\) terms are the direction cosines of the unit normal for the rotation, so all the terms above
are really strictly sums of sine and cosine products, so we have the rotation matrix completely described in terms of four
angles.  Also observe how much additional complexity we have to express a rotation in terms of the matrix.  This representation also
does not work for plane rotations, just vectors (whereas that is not the case for the rotor form).

It is actually somewhat simpler looking to leave things in terms of the \(\alpha\), and \(\beta\) parameters.  We can rewrite
\eqnref{eqn:rotor:rotmgreek} as:

\begin{equation}
\Be_i \cdot \left(e^{-\Bn I \theta/2} \Be_j e^{\Bn I \theta/2}\right)
= \delta_{ij}(2\alpha^2 -1)
+2\alpha \beta_k \epsilon_{ikj} + 2\beta_i\beta_j
\end{equation}

and the rotation matrix:

\begin{equation}\label{eqn:rotor:720}
[
R_\theta
]_\sigma^\sigma
=
2
\begin{bmatrix}
\alpha^2 -\frac{1}{2} + \beta_1^2 & \beta_1 \beta_2  - \beta_3 \alpha & \beta_1 \beta_3  + \beta_2 \alpha \\
\beta_1 \beta_2  + \beta_3 \alpha & \alpha^2 -\frac{1}{2} + \beta_2^2 & \beta_2 \beta_3  - \beta_1 \alpha \\
\beta_1 \beta_3  - \beta_2 \alpha & \beta_2 \beta_3  + \beta_1 \alpha & \alpha^2 -\frac{1}{2} + \beta_3^2 \\
\end{bmatrix}
\end{equation}

Not really that much simpler, but a bit.  The trade off is that the similarity to the standard \(2x2\) rotation matrix is not obvious.


\documentclass{article}      % Specifies the document class

\usepackage{amsmath}
\usepackage{mathpazo}

%
% shorthand for bold symbols, convenient for vectors and matrices
%
\newcommand{\Ba}[0]{\mathbf{a}}
\newcommand{\Bb}[0]{\mathbf{b}}
\newcommand{\Bc}[0]{\mathbf{c}}
\newcommand{\Bd}[0]{\mathbf{d}}
\newcommand{\Be}[0]{\mathbf{e}}
\newcommand{\Bf}[0]{\mathbf{f}}
\newcommand{\Bg}[0]{\mathbf{g}}
\newcommand{\Bh}[0]{\mathbf{h}}
\newcommand{\Bi}[0]{\mathbf{i}}
\newcommand{\Bj}[0]{\mathbf{j}}
\newcommand{\Bk}[0]{\mathbf{k}}
\newcommand{\Bl}[0]{\mathbf{l}}
\newcommand{\Bm}[0]{\mathbf{m}}
\newcommand{\Bn}[0]{\mathbf{n}}
\newcommand{\Bo}[0]{\mathbf{o}}
\newcommand{\Bp}[0]{\mathbf{p}}
\newcommand{\Bq}[0]{\mathbf{q}}
\newcommand{\Br}[0]{\mathbf{r}}
\newcommand{\Bs}[0]{\mathbf{s}}
\newcommand{\Bt}[0]{\mathbf{t}}
\newcommand{\Bu}[0]{\mathbf{u}}
\newcommand{\Bv}[0]{\mathbf{v}}
\newcommand{\Bw}[0]{\mathbf{w}}
\newcommand{\Bx}[0]{\mathbf{x}}
\newcommand{\By}[0]{\mathbf{y}}
\newcommand{\Bz}[0]{\mathbf{z}}
\newcommand{\BA}[0]{\mathbf{A}}
\newcommand{\BB}[0]{\mathbf{B}}
\newcommand{\BC}[0]{\mathbf{C}}
\newcommand{\BD}[0]{\mathbf{D}}
\newcommand{\BE}[0]{\mathbf{E}}
\newcommand{\BF}[0]{\mathbf{F}}
\newcommand{\BG}[0]{\mathbf{G}}
\newcommand{\BH}[0]{\mathbf{H}}
\newcommand{\BI}[0]{\mathbf{I}}
\newcommand{\BJ}[0]{\mathbf{J}}
\newcommand{\BK}[0]{\mathbf{K}}
\newcommand{\BL}[0]{\mathbf{L}}
\newcommand{\BM}[0]{\mathbf{M}}
\newcommand{\BN}[0]{\mathbf{N}}
\newcommand{\BO}[0]{\mathbf{O}}
\newcommand{\BP}[0]{\mathbf{P}}
\newcommand{\BQ}[0]{\mathbf{Q}}
\newcommand{\BR}[0]{\mathbf{R}}
\newcommand{\BS}[0]{\mathbf{S}}
\newcommand{\BT}[0]{\mathbf{T}}
\newcommand{\BU}[0]{\mathbf{U}}
\newcommand{\BV}[0]{\mathbf{V}}
\newcommand{\BW}[0]{\mathbf{W}}
\newcommand{\BX}[0]{\mathbf{X}}
\newcommand{\BY}[0]{\mathbf{Y}}
\newcommand{\BZ}[0]{\mathbf{Z}}

\newcommand{\Bzero}[0]{\mathbf{0}}
\newcommand{\Btheta}[0]{\boldsymbol{\theta}}
\newcommand{\Btau}[0]{\boldsymbol{\tau}}
\newcommand{\Bomega}[0]{\boldsymbol{\omega}}

%
% shorthand for unit vectors
%
\newcommand{\acap}[0]{\hat{\Ba}}
\newcommand{\bcap}[0]{\hat{\Bb}}
\newcommand{\ccap}[0]{\hat{\Bc}}
\newcommand{\dcap}[0]{\hat{\Bd}}
\newcommand{\ecap}[0]{\hat{\Be}}
\newcommand{\fcap}[0]{\hat{\Bf}}
\newcommand{\gcap}[0]{\hat{\Bg}}
\newcommand{\hcap}[0]{\hat{\Bh}}
\newcommand{\icap}[0]{\hat{\Bi}}
\newcommand{\jcap}[0]{\hat{\Bj}}
\newcommand{\kcap}[0]{\hat{\Bk}}
\newcommand{\lcap}[0]{\hat{\Bl}}
\newcommand{\mcap}[0]{\hat{\Bm}}
\newcommand{\ncap}[0]{\hat{\Bn}}
\newcommand{\ocap}[0]{\hat{\Bo}}
\newcommand{\pcap}[0]{\hat{\Bp}}
\newcommand{\qcap}[0]{\hat{\Bq}}
\newcommand{\rcap}[0]{\hat{\Br}}
\newcommand{\scap}[0]{\hat{\Bs}}
\newcommand{\tcap}[0]{\hat{\Bt}}
\newcommand{\ucap}[0]{\hat{\Bu}}
\newcommand{\vcap}[0]{\hat{\Bv}}
\newcommand{\wcap}[0]{\hat{\Bw}}
\newcommand{\xcap}[0]{\hat{\Bx}}
\newcommand{\ycap}[0]{\hat{\By}}
\newcommand{\zcap}[0]{\hat{\Bz}}
\newcommand{\thetacap}[0]{\hat{\Btheta}}

%
% to write R^n and C^n in a distinguishable fashion.  Perhaps change this
% to the double lined characters upon figuring out how to do so.
%
\newcommand{\C}[1]{$\mathbb{C}^{#1}$}
\newcommand{\R}[1]{$\mathbb{R}^{#1}$}

%
% various generally useful helpers
%

% derivative of #1 wrt. #2:
\newcommand{\D}[2] {\frac {d#2} {d#1}}

\newcommand{\inv}[1]{\frac{1}{#1}}
\newcommand{\cross}[0]{\times}

\newcommand{\abs}[1]{\lvert{#1}\rvert}
\newcommand{\norm}[1]{\lVert{#1}\rVert}
\newcommand{\innerprod}[2]{\langle{#1}, {#2}\rangle}
\newcommand{\dotprod}[2]{{#1} \cdot {#2}}
\newcommand{\bdotprod}[2]{\left({#1} \cdot {#2}\right)}
\newcommand{\crossprod}[2]{{#1} \cross {#2}}
\newcommand{\tripleprod}[3]{\dotprod{\left(\crossprod{#1}{#2}\right)}{#3}}

\DeclareMathOperator{\Proj}{Proj}
\DeclareMathOperator{\Span}{span}
\DeclareMathOperator{\Sgn}{sgn}
\DeclareMathOperator{\Area}{Area}
\DeclareMathOperator{\Volume}{Volume}

%
% A few miscellaneous things specific to this document
%
\newcommand{\crossop}[1]{\crossprod{#1}{}}

% R2 vector.
\newcommand{\VectorTwo}[2]{
\begin{bmatrix}
 {#1} \\
 {#2}
\end{bmatrix}
}

\newcommand{\VectorN}[1]{
\begin{bmatrix}
{#1}_1 \\
{#1}_2 \\
\vdots \\
{#1}_N \\
\end{bmatrix}
}

\newcommand{\DETuvij}[4]{
\begin{vmatrix}
 {#1}_{#3} & {#1}_{#4} \\
 {#2}_{#3} & {#2}_{#4}
\end{vmatrix}
}

\newcommand{\DETuvwijk}[6]{
\begin{vmatrix}
 {#1}_{#4} & {#1}_{#5} & {#1}_{#6} \\
 {#2}_{#4} & {#2}_{#5} & {#2}_{#6} \\
 {#3}_{#4} & {#3}_{#5} & {#3}_{#6}
\end{vmatrix}
}

\newcommand{\DETuvwxijkl}[8]{
\begin{vmatrix}
 {#1}_{#5} & {#1}_{#6} & {#1}_{#7} & {#1}_{#8} \\
 {#2}_{#5} & {#2}_{#6} & {#2}_{#7} & {#2}_{#8} \\
 {#3}_{#5} & {#3}_{#6} & {#3}_{#7} & {#3}_{#8} \\
 {#4}_{#5} & {#4}_{#6} & {#4}_{#7} & {#4}_{#8} \\
\end{vmatrix}
}

%\newcommand{\DETuvwxyijklm}[10]{
%\begin{vmatrix}
% {#1}_{#6} & {#1}_{#7} & {#1}_{#8} & {#1}_{#9} & {#1}_{#10} \\
% {#2}_{#6} & {#2}_{#7} & {#2}_{#8} & {#2}_{#9} & {#2}_{#10} \\
% {#3}_{#6} & {#3}_{#7} & {#3}_{#8} & {#3}_{#9} & {#3}_{#10} \\
% {#4}_{#6} & {#4}_{#7} & {#4}_{#8} & {#4}_{#9} & {#4}_{#10} \\
% {#5}_{#6} & {#5}_{#7} & {#5}_{#8} & {#5}_{#9} & {#5}_{#10}
%\end{vmatrix}
%}

% R3 vector.
\newcommand{\VectorThree}[3]{
\begin{bmatrix}
 {#1} \\
 {#2} \\
 {#3}
\end{bmatrix}
}


%<misc>
%
\newcommand{\Abs}[1]{{\left\lvert{#1}\right\rvert}}
\newcommand{\spacegrad}[0]{\boldsymbol{\nabla}}
\newcommand{\grad}[0]{\nabla}
\newcommand{\LL}[0]{\mathcal{L}}

% == \partial_{#1} {#2}
\newcommand{\PD}[2]{\frac{\partial {#2}}{\partial {#1}}}
% inline variant
\newcommand{\PDi}[2]{{\partial {#2}}/{\partial {#1}}}

\newcommand{\PDD}[3]{\frac{\partial^2 {#3}}{\partial {#1}\partial {#2}}}
%\newcommand{\PDd}[2]{\frac{\partial^2 {#2}}{{\partial{#1}}^2}}
\newcommand{\PDsq}[2]{\frac{\partial^2 {#2}}{(\partial {#1})^2}}

\newcommand{\Partial}[2]{\frac{\partial {#1}}{\partial {#2}}}
\DeclareMathOperator{\RejName}{Rej}
\newcommand{\Rej}[2]{\RejName_{#1}\left( {#2} \right)}
\newcommand{\Rm}[1]{\mathbb{R}^{#1}}
\newcommand{\Cm}[1]{\mathbb{C}^{#1}}
\newcommand{\conj}[0]{{*}}

%</misc>

% <grade selection>
%
\newcommand{\gpgrade}[2] {{\left\langle{{#1}}\right\rangle}_{#2}}

\newcommand{\gpgradezero}[1] {\gpgrade{#1}{}}
%\newcommand{\gpscalargrade}[1] {{\left\langle{{#1}}\right\rangle}}
%\newcommand{\gpgradezero}[1] {\gpgrade{#1}{0}}

%\newcommand{\gpgradeone}[1] {{\left\langle{{#1}}\right\rangle}_{1}}
\newcommand{\gpgradeone}[1] {\gpgrade{#1}{1}}

\newcommand{\gpgradetwo}[1] {\gpgrade{#1}{2}}
\newcommand{\gpgradethree}[1] {\gpgrade{#1}{3}}
\newcommand{\gpgradefour}[1] {\gpgrade{#1}{4}}
%
% </grade selection>



\newcommand{\adot}[0]{{\dot{a}}}
\newcommand{\bdot}[0]{{\dot{b}}}
% taken for centered dot:
%\newcommand{\cdot}[0]{{\dot{c}}}
%\newcommand{\ddot}[0]{{\dot{d}}}
\newcommand{\edot}[0]{{\dot{e}}}
\newcommand{\fdot}[0]{{\dot{f}}}
\newcommand{\gdot}[0]{{\dot{g}}}
\newcommand{\hdot}[0]{{\dot{h}}}
\newcommand{\idot}[0]{{\dot{i}}}
\newcommand{\jdot}[0]{{\dot{j}}}
\newcommand{\kdot}[0]{{\dot{k}}}
\newcommand{\ldot}[0]{{\dot{l}}}
\newcommand{\mdot}[0]{{\dot{m}}}
\newcommand{\ndot}[0]{{\dot{n}}}
%\newcommand{\odot}[0]{{\dot{o}}}
\newcommand{\pdot}[0]{{\dot{p}}}
\newcommand{\qdot}[0]{{\dot{q}}}
\newcommand{\rdot}[0]{{\dot{r}}}
\newcommand{\sdot}[0]{{\dot{s}}}
\newcommand{\tdot}[0]{{\dot{t}}}
\newcommand{\udot}[0]{{\dot{u}}}
\newcommand{\vdot}[0]{{\dot{v}}}
\newcommand{\wdot}[0]{{\dot{w}}}
\newcommand{\xdot}[0]{{\dot{x}}}
\newcommand{\ydot}[0]{{\dot{y}}}
\newcommand{\zdot}[0]{{\dot{z}}}
\newcommand{\addot}[0]{{\ddot{a}}}
\newcommand{\bddot}[0]{{\ddot{b}}}
\newcommand{\cddot}[0]{{\ddot{c}}}
%\newcommand{\dddot}[0]{{\ddot{d}}}
\newcommand{\eddot}[0]{{\ddot{e}}}
\newcommand{\fddot}[0]{{\ddot{f}}}
\newcommand{\gddot}[0]{{\ddot{g}}}
\newcommand{\hddot}[0]{{\ddot{h}}}
\newcommand{\iddot}[0]{{\ddot{i}}}
\newcommand{\jddot}[0]{{\ddot{j}}}
\newcommand{\kddot}[0]{{\ddot{k}}}
\newcommand{\lddot}[0]{{\ddot{l}}}
\newcommand{\mddot}[0]{{\ddot{m}}}
\newcommand{\nddot}[0]{{\ddot{n}}}
\newcommand{\oddot}[0]{{\ddot{o}}}
\newcommand{\pddot}[0]{{\ddot{p}}}
\newcommand{\qddot}[0]{{\ddot{q}}}
\newcommand{\rddot}[0]{{\ddot{r}}}
\newcommand{\sddot}[0]{{\ddot{s}}}
\newcommand{\tddot}[0]{{\ddot{t}}}
\newcommand{\uddot}[0]{{\ddot{u}}}
\newcommand{\vddot}[0]{{\ddot{v}}}
\newcommand{\wddot}[0]{{\ddot{w}}}
\newcommand{\xddot}[0]{{\ddot{x}}}
\newcommand{\yddot}[0]{{\ddot{y}}}
\newcommand{\zddot}[0]{{\ddot{z}}}

%<bold and dot greek symbols>
%

\newcommand{\Deltadot}[0]{{\dot{\Delta}}}
\newcommand{\Gammadot}[0]{{\dot{\Gamma}}}
\newcommand{\Lambdadot}[0]{{\dot{\Lambda}}}
\newcommand{\Omegadot}[0]{{\dot{\Omega}}}
\newcommand{\Phidot}[0]{{\dot{\Phi}}}
\newcommand{\Pidot}[0]{{\dot{\Pi}}}
\newcommand{\Psidot}[0]{{\dot{\Psi}}}
\newcommand{\Sigmadot}[0]{{\dot{\Sigma}}}
\newcommand{\Thetadot}[0]{{\dot{\Theta}}}
\newcommand{\Upsilondot}[0]{{\dot{\Upsilon}}}
\newcommand{\Xidot}[0]{{\dot{\Xi}}}
\newcommand{\alphadot}[0]{{\dot{\alpha}}}
\newcommand{\betadot}[0]{{\dot{\beta}}}
\newcommand{\chidot}[0]{{\dot{\chi}}}
\newcommand{\deltadot}[0]{{\dot{\delta}}}
\newcommand{\epsilondot}[0]{{\dot{\epsilon}}}
\newcommand{\etadot}[0]{{\dot{\eta}}}
\newcommand{\gammadot}[0]{{\dot{\gamma}}}
\newcommand{\kappadot}[0]{{\dot{\kappa}}}
\newcommand{\lambdadot}[0]{{\dot{\lambda}}}
\newcommand{\mudot}[0]{{\dot{\mu}}}
\newcommand{\nudot}[0]{{\dot{\nu}}}
\newcommand{\omegadot}[0]{{\dot{\omega}}}
\newcommand{\phidot}[0]{{\dot{\phi}}}
\newcommand{\pidot}[0]{{\dot{\pi}}}
\newcommand{\psidot}[0]{{\dot{\psi}}}
\newcommand{\rhodot}[0]{{\dot{\rho}}}
\newcommand{\sigmadot}[0]{{\dot{\sigma}}}
\newcommand{\taudot}[0]{{\dot{\tau}}}
\newcommand{\thetadot}[0]{{\dot{\theta}}}
\newcommand{\upsilondot}[0]{{\dot{\upsilon}}}
\newcommand{\varepsilondot}[0]{{\dot{\varepsilon}}}
\newcommand{\varphidot}[0]{{\dot{\varphi}}}
\newcommand{\varpidot}[0]{{\dot{\varpi}}}
\newcommand{\varrhodot}[0]{{\dot{\varrho}}}
\newcommand{\varsigmadot}[0]{{\dot{\varsigma}}}
\newcommand{\varthetadot}[0]{{\dot{\vartheta}}}
\newcommand{\xidot}[0]{{\dot{\xi}}}
\newcommand{\zetadot}[0]{{\dot{\zeta}}}

\newcommand{\Deltaddot}[0]{{\ddot{\Delta}}}
\newcommand{\Gammaddot}[0]{{\ddot{\Gamma}}}
\newcommand{\Lambdaddot}[0]{{\ddot{\Lambda}}}
\newcommand{\Omegaddot}[0]{{\ddot{\Omega}}}
\newcommand{\Phiddot}[0]{{\ddot{\Phi}}}
\newcommand{\Piddot}[0]{{\ddot{\Pi}}}
\newcommand{\Psiddot}[0]{{\ddot{\Psi}}}
\newcommand{\Sigmaddot}[0]{{\ddot{\Sigma}}}
\newcommand{\Thetaddot}[0]{{\ddot{\Theta}}}
\newcommand{\Upsilonddot}[0]{{\ddot{\Upsilon}}}
\newcommand{\Xiddot}[0]{{\ddot{\Xi}}}
\newcommand{\alphaddot}[0]{{\ddot{\alpha}}}
\newcommand{\betaddot}[0]{{\ddot{\beta}}}
\newcommand{\chiddot}[0]{{\ddot{\chi}}}
\newcommand{\deltaddot}[0]{{\ddot{\delta}}}
\newcommand{\epsilonddot}[0]{{\ddot{\epsilon}}}
\newcommand{\etaddot}[0]{{\ddot{\eta}}}
\newcommand{\gammaddot}[0]{{\ddot{\gamma}}}
\newcommand{\kappaddot}[0]{{\ddot{\kappa}}}
\newcommand{\lambdaddot}[0]{{\ddot{\lambda}}}
\newcommand{\muddot}[0]{{\ddot{\mu}}}
\newcommand{\nuddot}[0]{{\ddot{\nu}}}
\newcommand{\omegaddot}[0]{{\ddot{\omega}}}
\newcommand{\phiddot}[0]{{\ddot{\phi}}}
\newcommand{\piddot}[0]{{\ddot{\pi}}}
\newcommand{\psiddot}[0]{{\ddot{\psi}}}
\newcommand{\rhoddot}[0]{{\ddot{\rho}}}
\newcommand{\sigmaddot}[0]{{\ddot{\sigma}}}
\newcommand{\tauddot}[0]{{\ddot{\tau}}}
\newcommand{\thetaddot}[0]{{\ddot{\theta}}}
\newcommand{\upsilonddot}[0]{{\ddot{\upsilon}}}
\newcommand{\varepsilonddot}[0]{{\ddot{\varepsilon}}}
\newcommand{\varphiddot}[0]{{\ddot{\varphi}}}
\newcommand{\varpiddot}[0]{{\ddot{\varpi}}}
\newcommand{\varrhoddot}[0]{{\ddot{\varrho}}}
\newcommand{\varsigmaddot}[0]{{\ddot{\varsigma}}}
\newcommand{\varthetaddot}[0]{{\ddot{\vartheta}}}
\newcommand{\xiddot}[0]{{\ddot{\xi}}}
\newcommand{\zetaddot}[0]{{\ddot{\zeta}}}

\newcommand{\BDelta}[0]{\boldsymbol{\Delta}}
\newcommand{\BGamma}[0]{\boldsymbol{\Gamma}}
\newcommand{\BLambda}[0]{\boldsymbol{\Lambda}}
\newcommand{\BOmega}[0]{\boldsymbol{\Omega}}
\newcommand{\BPhi}[0]{\boldsymbol{\Phi}}
\newcommand{\BPi}[0]{\boldsymbol{\Pi}}
\newcommand{\BPsi}[0]{\boldsymbol{\Psi}}
\newcommand{\BSigma}[0]{\boldsymbol{\Sigma}}
\newcommand{\BTheta}[0]{\boldsymbol{\Theta}}
\newcommand{\BUpsilon}[0]{\boldsymbol{\Upsilon}}
\newcommand{\BXi}[0]{\boldsymbol{\Xi}}
\newcommand{\Balpha}[0]{\boldsymbol{\alpha}}
\newcommand{\Bbeta}[0]{\boldsymbol{\beta}}
\newcommand{\Bchi}[0]{\boldsymbol{\chi}}
\newcommand{\Bdelta}[0]{\boldsymbol{\delta}}
\newcommand{\Bepsilon}[0]{\boldsymbol{\epsilon}}
\newcommand{\Beta}[0]{\boldsymbol{\eta}}
\newcommand{\Bgamma}[0]{\boldsymbol{\gamma}}
\newcommand{\Bkappa}[0]{\boldsymbol{\kappa}}
\newcommand{\Blambda}[0]{\boldsymbol{\lambda}}
\newcommand{\Bmu}[0]{\boldsymbol{\mu}}
\newcommand{\Bnu}[0]{\boldsymbol{\nu}}
%\newcommand{\Bomega}[0]{\boldsymbol{\omega}}
\newcommand{\Bphi}[0]{\boldsymbol{\phi}}
\newcommand{\Bpi}[0]{\boldsymbol{\pi}}
\newcommand{\Bpsi}[0]{\boldsymbol{\psi}}
\newcommand{\Brho}[0]{\boldsymbol{\rho}}
\newcommand{\Bsigma}[0]{\boldsymbol{\sigma}}
%\newcommand{\Btau}[0]{\boldsymbol{\tau}}
%\newcommand{\Btheta}[0]{\boldsymbol{\theta}}
\newcommand{\Bupsilon}[0]{\boldsymbol{\upsilon}}
\newcommand{\Bvarepsilon}[0]{\boldsymbol{\varepsilon}}
\newcommand{\Bvarphi}[0]{\boldsymbol{\varphi}}
\newcommand{\Bvarpi}[0]{\boldsymbol{\varpi}}
\newcommand{\Bvarrho}[0]{\boldsymbol{\varrho}}
\newcommand{\Bvarsigma}[0]{\boldsymbol{\varsigma}}
\newcommand{\Bvartheta}[0]{\boldsymbol{\vartheta}}
\newcommand{\Bxi}[0]{\boldsymbol{\xi}}
\newcommand{\Bzeta}[0]{\boldsymbol{\zeta}}
%
%</bold and dot greek symbols>
%<infrequent>
%
%\newcommand{\AreaOp}[1]{\AName_{#1}}
%\newcommand{\Babs}[0]{\abs{\BB}}
%\newcommand{\Bcap}[0]{\hat{\BB}}
%\newcommand{\BrPrimeRej}[0]{\rcap(\rcap \wedge \Br')}
%\newcommand{\CA}[0]{\mathcal{A}}
%\newcommand{\Cos}[1]{\cos{\left({#1}\right)}}
%\newcommand{\Det}[1] {\abs{#1}}
%\newcommand{\Dsq}[2] {\frac {\partial^2 {#1}} {\partial {#2}^2}}
%\newcommand{\Exp}[1]{\exp{\left({#1}\right)}}
%\newcommand{\Norm}[1]{\left\lVert{#1}\right\rVert}
%\newcommand{\Sin}[1]{\sin{\left({#1}\right)}}
%\newcommand{\T}[0]{\text{T}}
%\newcommand{\VolumeOp}[1]{\VName_{#1}}
%\newcommand{\agrad}[0]{\Ba \cdot \nabla}
%\newcommand{\alphacap}[0]{\hat{\boldsymbol{\alpha}}}
%\newcommand{\Fcap}[0]{\hat{\BF}}
%\newcommand{\bithree}[0]{{\Bi}_3}
%\newcommand{\bxa}[0]{\Bx\Ba}
%\newcommand{\coordvec}[2]{
%\newcommand{\costheta}[0]{\acap \cdot \xcap}
%\newcommand{\ddt}[1]{\ddot{#1}}
%\newcommand{\ddu}[1] {\frac {d{#1}} {du}}
%\newcommand{\dsqxj}[2] {\frac {\partial^2 {#1}} {\partial {x_{#2}}^2}}
%\newcommand{\dtheta}[1]{\frac{d {#1}}{d \theta}}
%\newcommand{\dt}[1]{\dot{#1}}
%\newcommand{\dt}[1]{\frac{d {#1}}{dt}}
%\newcommand{\dxj}[2] {\frac {\partial {#1}} {\partial {x_{#2}}}}
%\newcommand{\halfPhi}[0]{\frac{\phi}{2}}
%\newcommand{\half}[0]{\inv{2}}
%\newcommand{\inv}[1]{\frac{1}{#1}}
%\newcommand{\laplacian}[0]{\nabla^2}
%\newcommand{\matrixoftx}[3]{
%\newcommand{\nrrp}[0]{\norm{\rcap \wedge \Br'}}
%\newcommand{\oiint}{\bigcirc \hspace{-1.4em} \int \hspace{-.8em} \int}
%\newcommand{\transpose}[1]{{#1}^{\text{T}}}
%\newcommand{\transpose}[1]{{{#1}^{\TextTranspose}}}
%\newcommand{\transpose}[1]{{{#1}^{\text{T}}}}
%\newcommand{\barA}[0]{\bar{A}}
%\newcommand{\qbar}[0]{\bar{q}}
%\newcommand{\qdotbar}[0]{\dot{\bar{q}}}
%
%</infrequent>





%
% The real thing:
%

\usepackage[bookmarks=true]{hyperref}

                             % The preamble begins here.
\title{Some notes on Euler Angles.} % Declares the document's title.
\author{Peeter Joot}         % Declares the author's name.
\date{ November 1, 2008. Last Revision: $Date: 2008/11/02 04:15:55 $ } % Deleting this command produces today's date.

\begin{document}             % End of preamble and beginning of text.

\maketitle{}
\tableofcontents

\section{ Removing the rotors from the exponentials. }

In \cite{doran2003gap} section 2.7.5 the euler angle formula is 
developed for $\{z,x',z''\}$ axis rotations by $\{\phi, \theta, \psi\}$
respectively.

Other than a few details the derivation is pretty straightforward.  Equation
2.153 would be clearer with a series expansion hint like

\begin{align*}
\exp(R \alpha i R^\dagger) 
&= \sum_k \inv{k!} (R \alpha i R^\dagger)^k \\
&= \sum_k \inv{k!} R (\alpha i)^k R^\dagger \\
&= R \exp(\alpha i) R^\dagger
\end{align*}

where $i$ is a bivector, and $R$ is a rotor.

\section{ In matrix form. }

The end result of the composite Euler rotations is that the rotation is

\begin{align*}
R(x) &= R x R^\dagger \\
R &= \exp(-e_{12}\phi/2) \exp(-e_{23}\theta/2) \exp(-e_{12}\psi/2)
\end{align*}

Then there are notes saying this is easier to visualize and work with than
the equivalent matrix formula.  Let's see what the equivalent matrix formula
is



\bibliographystyle{plainnat} % supposed to allow for \url use.
\bibliography{myrefs}      % expects file "myrefs.bib"

\end{document}               % End of document.

\documentclass{article}

\usepackage{amsmath}
\usepackage{mathpazo}

%
% shorthand for bold symbols, convenient for vectors and matrices
%
\newcommand{\Ba}[0]{\mathbf{a}}
\newcommand{\Bb}[0]{\mathbf{b}}
\newcommand{\Bc}[0]{\mathbf{c}}
\newcommand{\Bd}[0]{\mathbf{d}}
\newcommand{\Be}[0]{\mathbf{e}}
\newcommand{\Bf}[0]{\mathbf{f}}
\newcommand{\Bg}[0]{\mathbf{g}}
\newcommand{\Bh}[0]{\mathbf{h}}
\newcommand{\Bi}[0]{\mathbf{i}}
\newcommand{\Bj}[0]{\mathbf{j}}
\newcommand{\Bk}[0]{\mathbf{k}}
\newcommand{\Bl}[0]{\mathbf{l}}
\newcommand{\Bm}[0]{\mathbf{m}}
\newcommand{\Bn}[0]{\mathbf{n}}
\newcommand{\Bo}[0]{\mathbf{o}}
\newcommand{\Bp}[0]{\mathbf{p}}
\newcommand{\Bq}[0]{\mathbf{q}}
\newcommand{\Br}[0]{\mathbf{r}}
\newcommand{\Bs}[0]{\mathbf{s}}
\newcommand{\Bt}[0]{\mathbf{t}}
\newcommand{\Bu}[0]{\mathbf{u}}
\newcommand{\Bv}[0]{\mathbf{v}}
\newcommand{\Bw}[0]{\mathbf{w}}
\newcommand{\Bx}[0]{\mathbf{x}}
\newcommand{\By}[0]{\mathbf{y}}
\newcommand{\Bz}[0]{\mathbf{z}}
\newcommand{\BA}[0]{\mathbf{A}}
\newcommand{\BB}[0]{\mathbf{B}}
\newcommand{\BC}[0]{\mathbf{C}}
\newcommand{\BD}[0]{\mathbf{D}}
\newcommand{\BE}[0]{\mathbf{E}}
\newcommand{\BF}[0]{\mathbf{F}}
\newcommand{\BG}[0]{\mathbf{G}}
\newcommand{\BH}[0]{\mathbf{H}}
\newcommand{\BI}[0]{\mathbf{I}}
\newcommand{\BJ}[0]{\mathbf{J}}
\newcommand{\BK}[0]{\mathbf{K}}
\newcommand{\BL}[0]{\mathbf{L}}
\newcommand{\BM}[0]{\mathbf{M}}
\newcommand{\BN}[0]{\mathbf{N}}
\newcommand{\BO}[0]{\mathbf{O}}
\newcommand{\BP}[0]{\mathbf{P}}
\newcommand{\BQ}[0]{\mathbf{Q}}
\newcommand{\BR}[0]{\mathbf{R}}
\newcommand{\BS}[0]{\mathbf{S}}
\newcommand{\BT}[0]{\mathbf{T}}
\newcommand{\BU}[0]{\mathbf{U}}
\newcommand{\BV}[0]{\mathbf{V}}
\newcommand{\BW}[0]{\mathbf{W}}
\newcommand{\BX}[0]{\mathbf{X}}
\newcommand{\BY}[0]{\mathbf{Y}}
\newcommand{\BZ}[0]{\mathbf{Z}}

\newcommand{\Bzero}[0]{\mathbf{0}}
\newcommand{\Btheta}[0]{\boldsymbol{\theta}}
\newcommand{\Btau}[0]{\boldsymbol{\tau}}
\newcommand{\Bomega}[0]{\boldsymbol{\omega}}

%
% shorthand for unit vectors
%
\newcommand{\acap}[0]{\hat{\Ba}}
\newcommand{\bcap}[0]{\hat{\Bb}}
\newcommand{\ccap}[0]{\hat{\Bc}}
\newcommand{\dcap}[0]{\hat{\Bd}}
\newcommand{\ecap}[0]{\hat{\Be}}
\newcommand{\fcap}[0]{\hat{\Bf}}
\newcommand{\gcap}[0]{\hat{\Bg}}
\newcommand{\hcap}[0]{\hat{\Bh}}
\newcommand{\icap}[0]{\hat{\Bi}}
\newcommand{\jcap}[0]{\hat{\Bj}}
\newcommand{\kcap}[0]{\hat{\Bk}}
\newcommand{\lcap}[0]{\hat{\Bl}}
\newcommand{\mcap}[0]{\hat{\Bm}}
\newcommand{\ncap}[0]{\hat{\Bn}}
\newcommand{\ocap}[0]{\hat{\Bo}}
\newcommand{\pcap}[0]{\hat{\Bp}}
\newcommand{\qcap}[0]{\hat{\Bq}}
\newcommand{\rcap}[0]{\hat{\Br}}
\newcommand{\scap}[0]{\hat{\Bs}}
\newcommand{\tcap}[0]{\hat{\Bt}}
\newcommand{\ucap}[0]{\hat{\Bu}}
\newcommand{\vcap}[0]{\hat{\Bv}}
\newcommand{\wcap}[0]{\hat{\Bw}}
\newcommand{\xcap}[0]{\hat{\Bx}}
\newcommand{\ycap}[0]{\hat{\By}}
\newcommand{\zcap}[0]{\hat{\Bz}}
\newcommand{\thetacap}[0]{\hat{\Btheta}}

%
% to write R^n and C^n in a distinguishable fashion.  Perhaps change this
% to the double lined characters upon figuring out how to do so.
%
\newcommand{\C}[1]{$\mathbb{C}^{#1}$}
\newcommand{\R}[1]{$\mathbb{R}^{#1}$}

%
% various generally useful helpers
%

% derivative of #1 wrt. #2:
\newcommand{\D}[2] {\frac {d#2} {d#1}}

\newcommand{\inv}[1]{\frac{1}{#1}}
\newcommand{\cross}[0]{\times}

\newcommand{\abs}[1]{\lvert{#1}\rvert}
\newcommand{\norm}[1]{\lVert{#1}\rVert}
\newcommand{\innerprod}[2]{\langle{#1}, {#2}\rangle}
\newcommand{\dotprod}[2]{{#1} \cdot {#2}}
\newcommand{\bdotprod}[2]{\left({#1} \cdot {#2}\right)}
\newcommand{\crossprod}[2]{{#1} \cross {#2}}
\newcommand{\tripleprod}[3]{\dotprod{\left(\crossprod{#1}{#2}\right)}{#3}}

\DeclareMathOperator{\Proj}{Proj}
\DeclareMathOperator{\Span}{span}
\DeclareMathOperator{\Sgn}{sgn}
\DeclareMathOperator{\Area}{Area}
\DeclareMathOperator{\Volume}{Volume}

%
% A few miscellaneous things specific to this document
%
\newcommand{\crossop}[1]{\crossprod{#1}{}}

% R2 vector.
\newcommand{\VectorTwo}[2]{
\begin{bmatrix}
 {#1} \\
 {#2}
\end{bmatrix}
}

\newcommand{\VectorN}[1]{
\begin{bmatrix}
{#1}_1 \\
{#1}_2 \\
\vdots \\
{#1}_N \\
\end{bmatrix}
}

\newcommand{\DETuvij}[4]{
\begin{vmatrix}
 {#1}_{#3} & {#1}_{#4} \\
 {#2}_{#3} & {#2}_{#4}
\end{vmatrix}
}

\newcommand{\DETuvwijk}[6]{
\begin{vmatrix}
 {#1}_{#4} & {#1}_{#5} & {#1}_{#6} \\
 {#2}_{#4} & {#2}_{#5} & {#2}_{#6} \\
 {#3}_{#4} & {#3}_{#5} & {#3}_{#6}
\end{vmatrix}
}

\newcommand{\DETuvwxijkl}[8]{
\begin{vmatrix}
 {#1}_{#5} & {#1}_{#6} & {#1}_{#7} & {#1}_{#8} \\
 {#2}_{#5} & {#2}_{#6} & {#2}_{#7} & {#2}_{#8} \\
 {#3}_{#5} & {#3}_{#6} & {#3}_{#7} & {#3}_{#8} \\
 {#4}_{#5} & {#4}_{#6} & {#4}_{#7} & {#4}_{#8} \\
\end{vmatrix}
}

%\newcommand{\DETuvwxyijklm}[10]{
%\begin{vmatrix}
% {#1}_{#6} & {#1}_{#7} & {#1}_{#8} & {#1}_{#9} & {#1}_{#10} \\
% {#2}_{#6} & {#2}_{#7} & {#2}_{#8} & {#2}_{#9} & {#2}_{#10} \\
% {#3}_{#6} & {#3}_{#7} & {#3}_{#8} & {#3}_{#9} & {#3}_{#10} \\
% {#4}_{#6} & {#4}_{#7} & {#4}_{#8} & {#4}_{#9} & {#4}_{#10} \\
% {#5}_{#6} & {#5}_{#7} & {#5}_{#8} & {#5}_{#9} & {#5}_{#10}
%\end{vmatrix}
%}

% R3 vector.
\newcommand{\VectorThree}[3]{
\begin{bmatrix}
 {#1} \\
 {#2} \\
 {#3}
\end{bmatrix}
}


%<misc>
%
\newcommand{\Abs}[1]{{\left\lvert{#1}\right\rvert}}
\newcommand{\spacegrad}[0]{\boldsymbol{\nabla}}
\newcommand{\grad}[0]{\nabla}
\newcommand{\LL}[0]{\mathcal{L}}

% == \partial_{#1} {#2}
\newcommand{\PD}[2]{\frac{\partial {#2}}{\partial {#1}}}
% inline variant
\newcommand{\PDi}[2]{{\partial {#2}}/{\partial {#1}}}

\newcommand{\PDD}[3]{\frac{\partial^2 {#3}}{\partial {#1}\partial {#2}}}
%\newcommand{\PDd}[2]{\frac{\partial^2 {#2}}{{\partial{#1}}^2}}
\newcommand{\PDsq}[2]{\frac{\partial^2 {#2}}{(\partial {#1})^2}}

\newcommand{\Partial}[2]{\frac{\partial {#1}}{\partial {#2}}}
\DeclareMathOperator{\RejName}{Rej}
\newcommand{\Rej}[2]{\RejName_{#1}\left( {#2} \right)}
\newcommand{\Rm}[1]{\mathbb{R}^{#1}}
\newcommand{\Cm}[1]{\mathbb{C}^{#1}}
\newcommand{\conj}[0]{{*}}

%</misc>

% <grade selection>
%
\newcommand{\gpgrade}[2] {{\left\langle{{#1}}\right\rangle}_{#2}}

\newcommand{\gpgradezero}[1] {\gpgrade{#1}{}}
%\newcommand{\gpscalargrade}[1] {{\left\langle{{#1}}\right\rangle}}
%\newcommand{\gpgradezero}[1] {\gpgrade{#1}{0}}

%\newcommand{\gpgradeone}[1] {{\left\langle{{#1}}\right\rangle}_{1}}
\newcommand{\gpgradeone}[1] {\gpgrade{#1}{1}}

\newcommand{\gpgradetwo}[1] {\gpgrade{#1}{2}}
\newcommand{\gpgradethree}[1] {\gpgrade{#1}{3}}
\newcommand{\gpgradefour}[1] {\gpgrade{#1}{4}}
%
% </grade selection>



\newcommand{\adot}[0]{{\dot{a}}}
\newcommand{\bdot}[0]{{\dot{b}}}
% taken for centered dot:
%\newcommand{\cdot}[0]{{\dot{c}}}
%\newcommand{\ddot}[0]{{\dot{d}}}
\newcommand{\edot}[0]{{\dot{e}}}
\newcommand{\fdot}[0]{{\dot{f}}}
\newcommand{\gdot}[0]{{\dot{g}}}
\newcommand{\hdot}[0]{{\dot{h}}}
\newcommand{\idot}[0]{{\dot{i}}}
\newcommand{\jdot}[0]{{\dot{j}}}
\newcommand{\kdot}[0]{{\dot{k}}}
\newcommand{\ldot}[0]{{\dot{l}}}
\newcommand{\mdot}[0]{{\dot{m}}}
\newcommand{\ndot}[0]{{\dot{n}}}
%\newcommand{\odot}[0]{{\dot{o}}}
\newcommand{\pdot}[0]{{\dot{p}}}
\newcommand{\qdot}[0]{{\dot{q}}}
\newcommand{\rdot}[0]{{\dot{r}}}
\newcommand{\sdot}[0]{{\dot{s}}}
\newcommand{\tdot}[0]{{\dot{t}}}
\newcommand{\udot}[0]{{\dot{u}}}
\newcommand{\vdot}[0]{{\dot{v}}}
\newcommand{\wdot}[0]{{\dot{w}}}
\newcommand{\xdot}[0]{{\dot{x}}}
\newcommand{\ydot}[0]{{\dot{y}}}
\newcommand{\zdot}[0]{{\dot{z}}}
\newcommand{\addot}[0]{{\ddot{a}}}
\newcommand{\bddot}[0]{{\ddot{b}}}
\newcommand{\cddot}[0]{{\ddot{c}}}
%\newcommand{\dddot}[0]{{\ddot{d}}}
\newcommand{\eddot}[0]{{\ddot{e}}}
\newcommand{\fddot}[0]{{\ddot{f}}}
\newcommand{\gddot}[0]{{\ddot{g}}}
\newcommand{\hddot}[0]{{\ddot{h}}}
\newcommand{\iddot}[0]{{\ddot{i}}}
\newcommand{\jddot}[0]{{\ddot{j}}}
\newcommand{\kddot}[0]{{\ddot{k}}}
\newcommand{\lddot}[0]{{\ddot{l}}}
\newcommand{\mddot}[0]{{\ddot{m}}}
\newcommand{\nddot}[0]{{\ddot{n}}}
\newcommand{\oddot}[0]{{\ddot{o}}}
\newcommand{\pddot}[0]{{\ddot{p}}}
\newcommand{\qddot}[0]{{\ddot{q}}}
\newcommand{\rddot}[0]{{\ddot{r}}}
\newcommand{\sddot}[0]{{\ddot{s}}}
\newcommand{\tddot}[0]{{\ddot{t}}}
\newcommand{\uddot}[0]{{\ddot{u}}}
\newcommand{\vddot}[0]{{\ddot{v}}}
\newcommand{\wddot}[0]{{\ddot{w}}}
\newcommand{\xddot}[0]{{\ddot{x}}}
\newcommand{\yddot}[0]{{\ddot{y}}}
\newcommand{\zddot}[0]{{\ddot{z}}}

%<bold and dot greek symbols>
%

\newcommand{\Deltadot}[0]{{\dot{\Delta}}}
\newcommand{\Gammadot}[0]{{\dot{\Gamma}}}
\newcommand{\Lambdadot}[0]{{\dot{\Lambda}}}
\newcommand{\Omegadot}[0]{{\dot{\Omega}}}
\newcommand{\Phidot}[0]{{\dot{\Phi}}}
\newcommand{\Pidot}[0]{{\dot{\Pi}}}
\newcommand{\Psidot}[0]{{\dot{\Psi}}}
\newcommand{\Sigmadot}[0]{{\dot{\Sigma}}}
\newcommand{\Thetadot}[0]{{\dot{\Theta}}}
\newcommand{\Upsilondot}[0]{{\dot{\Upsilon}}}
\newcommand{\Xidot}[0]{{\dot{\Xi}}}
\newcommand{\alphadot}[0]{{\dot{\alpha}}}
\newcommand{\betadot}[0]{{\dot{\beta}}}
\newcommand{\chidot}[0]{{\dot{\chi}}}
\newcommand{\deltadot}[0]{{\dot{\delta}}}
\newcommand{\epsilondot}[0]{{\dot{\epsilon}}}
\newcommand{\etadot}[0]{{\dot{\eta}}}
\newcommand{\gammadot}[0]{{\dot{\gamma}}}
\newcommand{\kappadot}[0]{{\dot{\kappa}}}
\newcommand{\lambdadot}[0]{{\dot{\lambda}}}
\newcommand{\mudot}[0]{{\dot{\mu}}}
\newcommand{\nudot}[0]{{\dot{\nu}}}
\newcommand{\omegadot}[0]{{\dot{\omega}}}
\newcommand{\phidot}[0]{{\dot{\phi}}}
\newcommand{\pidot}[0]{{\dot{\pi}}}
\newcommand{\psidot}[0]{{\dot{\psi}}}
\newcommand{\rhodot}[0]{{\dot{\rho}}}
\newcommand{\sigmadot}[0]{{\dot{\sigma}}}
\newcommand{\taudot}[0]{{\dot{\tau}}}
\newcommand{\thetadot}[0]{{\dot{\theta}}}
\newcommand{\upsilondot}[0]{{\dot{\upsilon}}}
\newcommand{\varepsilondot}[0]{{\dot{\varepsilon}}}
\newcommand{\varphidot}[0]{{\dot{\varphi}}}
\newcommand{\varpidot}[0]{{\dot{\varpi}}}
\newcommand{\varrhodot}[0]{{\dot{\varrho}}}
\newcommand{\varsigmadot}[0]{{\dot{\varsigma}}}
\newcommand{\varthetadot}[0]{{\dot{\vartheta}}}
\newcommand{\xidot}[0]{{\dot{\xi}}}
\newcommand{\zetadot}[0]{{\dot{\zeta}}}

\newcommand{\Deltaddot}[0]{{\ddot{\Delta}}}
\newcommand{\Gammaddot}[0]{{\ddot{\Gamma}}}
\newcommand{\Lambdaddot}[0]{{\ddot{\Lambda}}}
\newcommand{\Omegaddot}[0]{{\ddot{\Omega}}}
\newcommand{\Phiddot}[0]{{\ddot{\Phi}}}
\newcommand{\Piddot}[0]{{\ddot{\Pi}}}
\newcommand{\Psiddot}[0]{{\ddot{\Psi}}}
\newcommand{\Sigmaddot}[0]{{\ddot{\Sigma}}}
\newcommand{\Thetaddot}[0]{{\ddot{\Theta}}}
\newcommand{\Upsilonddot}[0]{{\ddot{\Upsilon}}}
\newcommand{\Xiddot}[0]{{\ddot{\Xi}}}
\newcommand{\alphaddot}[0]{{\ddot{\alpha}}}
\newcommand{\betaddot}[0]{{\ddot{\beta}}}
\newcommand{\chiddot}[0]{{\ddot{\chi}}}
\newcommand{\deltaddot}[0]{{\ddot{\delta}}}
\newcommand{\epsilonddot}[0]{{\ddot{\epsilon}}}
\newcommand{\etaddot}[0]{{\ddot{\eta}}}
\newcommand{\gammaddot}[0]{{\ddot{\gamma}}}
\newcommand{\kappaddot}[0]{{\ddot{\kappa}}}
\newcommand{\lambdaddot}[0]{{\ddot{\lambda}}}
\newcommand{\muddot}[0]{{\ddot{\mu}}}
\newcommand{\nuddot}[0]{{\ddot{\nu}}}
\newcommand{\omegaddot}[0]{{\ddot{\omega}}}
\newcommand{\phiddot}[0]{{\ddot{\phi}}}
\newcommand{\piddot}[0]{{\ddot{\pi}}}
\newcommand{\psiddot}[0]{{\ddot{\psi}}}
\newcommand{\rhoddot}[0]{{\ddot{\rho}}}
\newcommand{\sigmaddot}[0]{{\ddot{\sigma}}}
\newcommand{\tauddot}[0]{{\ddot{\tau}}}
\newcommand{\thetaddot}[0]{{\ddot{\theta}}}
\newcommand{\upsilonddot}[0]{{\ddot{\upsilon}}}
\newcommand{\varepsilonddot}[0]{{\ddot{\varepsilon}}}
\newcommand{\varphiddot}[0]{{\ddot{\varphi}}}
\newcommand{\varpiddot}[0]{{\ddot{\varpi}}}
\newcommand{\varrhoddot}[0]{{\ddot{\varrho}}}
\newcommand{\varsigmaddot}[0]{{\ddot{\varsigma}}}
\newcommand{\varthetaddot}[0]{{\ddot{\vartheta}}}
\newcommand{\xiddot}[0]{{\ddot{\xi}}}
\newcommand{\zetaddot}[0]{{\ddot{\zeta}}}

\newcommand{\BDelta}[0]{\boldsymbol{\Delta}}
\newcommand{\BGamma}[0]{\boldsymbol{\Gamma}}
\newcommand{\BLambda}[0]{\boldsymbol{\Lambda}}
\newcommand{\BOmega}[0]{\boldsymbol{\Omega}}
\newcommand{\BPhi}[0]{\boldsymbol{\Phi}}
\newcommand{\BPi}[0]{\boldsymbol{\Pi}}
\newcommand{\BPsi}[0]{\boldsymbol{\Psi}}
\newcommand{\BSigma}[0]{\boldsymbol{\Sigma}}
\newcommand{\BTheta}[0]{\boldsymbol{\Theta}}
\newcommand{\BUpsilon}[0]{\boldsymbol{\Upsilon}}
\newcommand{\BXi}[0]{\boldsymbol{\Xi}}
\newcommand{\Balpha}[0]{\boldsymbol{\alpha}}
\newcommand{\Bbeta}[0]{\boldsymbol{\beta}}
\newcommand{\Bchi}[0]{\boldsymbol{\chi}}
\newcommand{\Bdelta}[0]{\boldsymbol{\delta}}
\newcommand{\Bepsilon}[0]{\boldsymbol{\epsilon}}
\newcommand{\Beta}[0]{\boldsymbol{\eta}}
\newcommand{\Bgamma}[0]{\boldsymbol{\gamma}}
\newcommand{\Bkappa}[0]{\boldsymbol{\kappa}}
\newcommand{\Blambda}[0]{\boldsymbol{\lambda}}
\newcommand{\Bmu}[0]{\boldsymbol{\mu}}
\newcommand{\Bnu}[0]{\boldsymbol{\nu}}
%\newcommand{\Bomega}[0]{\boldsymbol{\omega}}
\newcommand{\Bphi}[0]{\boldsymbol{\phi}}
\newcommand{\Bpi}[0]{\boldsymbol{\pi}}
\newcommand{\Bpsi}[0]{\boldsymbol{\psi}}
\newcommand{\Brho}[0]{\boldsymbol{\rho}}
\newcommand{\Bsigma}[0]{\boldsymbol{\sigma}}
%\newcommand{\Btau}[0]{\boldsymbol{\tau}}
%\newcommand{\Btheta}[0]{\boldsymbol{\theta}}
\newcommand{\Bupsilon}[0]{\boldsymbol{\upsilon}}
\newcommand{\Bvarepsilon}[0]{\boldsymbol{\varepsilon}}
\newcommand{\Bvarphi}[0]{\boldsymbol{\varphi}}
\newcommand{\Bvarpi}[0]{\boldsymbol{\varpi}}
\newcommand{\Bvarrho}[0]{\boldsymbol{\varrho}}
\newcommand{\Bvarsigma}[0]{\boldsymbol{\varsigma}}
\newcommand{\Bvartheta}[0]{\boldsymbol{\vartheta}}
\newcommand{\Bxi}[0]{\boldsymbol{\xi}}
\newcommand{\Bzeta}[0]{\boldsymbol{\zeta}}
%
%</bold and dot greek symbols>
%<infrequent>
%
%\newcommand{\AreaOp}[1]{\AName_{#1}}
%\newcommand{\Babs}[0]{\abs{\BB}}
%\newcommand{\Bcap}[0]{\hat{\BB}}
%\newcommand{\BrPrimeRej}[0]{\rcap(\rcap \wedge \Br')}
%\newcommand{\CA}[0]{\mathcal{A}}
%\newcommand{\Cos}[1]{\cos{\left({#1}\right)}}
%\newcommand{\Det}[1] {\abs{#1}}
%\newcommand{\Dsq}[2] {\frac {\partial^2 {#1}} {\partial {#2}^2}}
%\newcommand{\Exp}[1]{\exp{\left({#1}\right)}}
%\newcommand{\Norm}[1]{\left\lVert{#1}\right\rVert}
%\newcommand{\Sin}[1]{\sin{\left({#1}\right)}}
%\newcommand{\T}[0]{\text{T}}
%\newcommand{\VolumeOp}[1]{\VName_{#1}}
%\newcommand{\agrad}[0]{\Ba \cdot \nabla}
%\newcommand{\alphacap}[0]{\hat{\boldsymbol{\alpha}}}
%\newcommand{\Fcap}[0]{\hat{\BF}}
%\newcommand{\bithree}[0]{{\Bi}_3}
%\newcommand{\bxa}[0]{\Bx\Ba}
%\newcommand{\coordvec}[2]{
%\newcommand{\costheta}[0]{\acap \cdot \xcap}
%\newcommand{\ddt}[1]{\ddot{#1}}
%\newcommand{\ddu}[1] {\frac {d{#1}} {du}}
%\newcommand{\dsqxj}[2] {\frac {\partial^2 {#1}} {\partial {x_{#2}}^2}}
%\newcommand{\dtheta}[1]{\frac{d {#1}}{d \theta}}
%\newcommand{\dt}[1]{\dot{#1}}
%\newcommand{\dt}[1]{\frac{d {#1}}{dt}}
%\newcommand{\dxj}[2] {\frac {\partial {#1}} {\partial {x_{#2}}}}
%\newcommand{\halfPhi}[0]{\frac{\phi}{2}}
%\newcommand{\half}[0]{\inv{2}}
%\newcommand{\inv}[1]{\frac{1}{#1}}
%\newcommand{\laplacian}[0]{\nabla^2}
%\newcommand{\matrixoftx}[3]{
%\newcommand{\nrrp}[0]{\norm{\rcap \wedge \Br'}}
%\newcommand{\oiint}{\bigcirc \hspace{-1.4em} \int \hspace{-.8em} \int}
%\newcommand{\transpose}[1]{{#1}^{\text{T}}}
%\newcommand{\transpose}[1]{{{#1}^{\TextTranspose}}}
%\newcommand{\transpose}[1]{{{#1}^{\text{T}}}}
%\newcommand{\barA}[0]{\bar{A}}
%\newcommand{\qbar}[0]{\bar{q}}
%\newcommand{\qdotbar}[0]{\dot{\bar{q}}}
%
%</infrequent>





\newcommand{\phicap}[0]{\hat{\boldsymbol{\phi}}}
\newcommand{\Lor}[2]{{{\Lambda^{#1}}_{#2}}}
\newcommand{\ILor}[2]{{{ \{{\Lambda^{-1}\} }^{#1}}_{#2}}}

\usepackage{color,cite,graphicx}
   % use colour in the document, put your citations as [1-4]
   % rather than [1,2,3,4] (it looks nicer, and the extended LaTeX2e
   % graphics package. 
\usepackage{latexsym,amssymb,epsf} % don't remember if these are
   % needed, but their inclusion can't do any damage


\usepackage[bookmarks=true]{hyperref}


\title{Some notes on spherical polar coordinates}
\author{Peeter Joot}
\date{ Nov 13, 2008.  Last Revision: $Date: 2008/11/17 14:13:55 $ }

\begin{document}

\maketitle{}
\tableofcontents

\section{ Motivation. }

Reading the math intro of \cite{zeilik1998iaa}, I found the statement that the gradient in spherical polar form is:

\begin{align*}
\grad &= 
\rcap \PD{r}{}
+\thetacap \inv{r} \PD{\theta}{}
+\phicap \inv{r \sin\phi}\PD{\phi}{}
\end{align*}
FIXME: verify if I recalled this sine term above correctly by doing this derivation.

There was no picture or description showing the conventions for measurement of the angles or directions.
To clarify things and leave a margin note I decided to derive the coordinates and unit vector transformation relationships,
gradient, divergence and curl in spherical polar coordinates.

Although these results can be found in many texts, including the excellent review article \cite{fleischCoords}, 
the exersize of personally working out the details will be worthwhile as a learning exersize since I hadn't attempted this since
way back in my school days.

\subsection{ Conventions. }

\begin{figure}[htp]
\centering
\includegraphics[totalheight=0.4\textheight]{spherical_polar}
\caption{Angles and lengths in spherical polar coordinates}\label{fig:spherical_polar}
\end{figure}

Figure \ref{fig:spherical_polar} illustrates the conventions used in 
these notes.  By inspection, the coordinates can be read off the diagram.

\begin{align}\label{eqn:coordinates}
u &= r \cos\phi \\
x &= u \cos\theta = r \cos\phi \cos\theta \\
y &= u \sin\theta = r \cos\phi \sin\theta \\
z &= r \sin\phi
\end{align}

\subsection{ The unit vectors. }

To calculate the unit vectors $\rcap$, $\thetacap$, $\phicap$ in the spherical polar frame we need to apply two sets of rotations.  The first is a rotation 
in the $x,y$ plane, and the second in the $x', z$ plane.

For the intermediate frame after just the $x,y$ plane rotation we have

\begin{align*}
R_\theta &= \exp(-\Be_{12}\theta/2) \\
\Be_i' &= R_\theta \Be_i R_\theta^\dagger
\end{align*}

Now for the rotational plane for the $\phi$ rotation is

\begin{align*}
\Be_1' \wedge \Be_3 
&= (R_\theta \Be_1 R_\theta^\dagger) \wedge \Be_3 \\
&= \inv{2} ( R_\theta \Be_1 R_\theta^\dagger \Be_3 - \Be_3 R_\theta \Be_1 R_\theta^\dagger ) \\
\end{align*}

Noting that $R_\theta$, having scalar, and $\Be_{12}$ components commutes with $\Be_3$, so we have

\begin{align*}
\Be_1' \wedge \Be_3 
&= R_\theta \inv{2} ( \Be_1 \Be_3 - \Be_3 \Be_1 ) R_\theta^\dagger \\
&= R_\theta \Be_1 \wedge \Be_3 R_\theta^\dagger \\
\end{align*}

Therefore the rotor for the second stage rotation is

\begin{align*}
R_\phi 
&= \exp( - R_\theta \Be_1 \wedge \Be_3 R_\theta^\dagger \phi/2 ) \\
&= \sum \inv{k!} \left( - R_\theta \Be_1 \wedge \Be_3 R_\theta^\dagger \phi/2 \right)^k \\
&= R_\theta \sum \inv{k!} ( - \Be_1 \wedge \Be_3 \phi/2 )^k R_\theta^\dagger \\
&= R_\theta \exp( - \Be_{13} \phi/2 ) R_\theta^\dagger \\
\end{align*}

Composing both sets of rotations one has

\begin{align*}
R(\Bx) 
&= R_\theta \exp( - \Be_{13} \phi/2 ) R_\theta^\dagger R_\theta \Bx R_\theta^\dagger R_\theta \exp( \Be_{13} \phi/2 ) R_\theta^\dagger \\
&= \exp( - \Be_{12} \theta/2 ) \exp( - \Be_{13} \phi/2 ) \Bx \exp( \Be_{13} \phi/2 ) \exp( \Be_{12} \theta/2 ) \\
\end{align*}

Or, more compactly

\begin{align}
R(\Bx) &= R \Bx R^\dagger \\
R &= R_\theta R_\phi \\
R_\phi &= \exp(-\Be_{13}\phi/2) \\
R_\theta &= \exp(-\Be_{12}\theta/2)
\end{align}

Application of these to the $\{\Be_i\}$ basis produces the $\{\rcap, \thetacap, \phicap\}$ basis.  First application 
of $R_\phi$ yields the basis vectors for the intermediate rotation.

\begin{align*}
\begin{array}{l l l}
{R_\phi}\Be_1 {R_\phi}^\dagger &= \Be_1 (\cos\phi + \Be_{13} \sin\phi) &= \Be_1 \cos\phi + \Be_3 \sin\phi \\
{R_\phi}\Be_2 {R_\phi}^\dagger &= \Be_2 R_\phi {R_\phi}^\dagger &= \Be_2 \\
{R_\phi}\Be_3 {R_\phi}^\dagger &= \Be_3 (\cos\phi + \Be_{13} \sin\phi) &= \Be_3 \cos\phi - \Be_1 \sin\phi \\
\end{array}
\end{align*}

Applying the second rotation to $R_\phi(\Be_i)$ we have
\begin{align*}
\rcap 
&= {R_\theta}( \Be_1 \cos\phi + \Be_3 \sin\phi ) {R_\theta}^\dagger \\
&=
\Be_1 \cos\phi (\cos\theta + \Be_{12} \sin\theta)
+ \Be_3 \sin\phi \\
&=
\Be_1 \cos\phi \cos\theta 
+ \Be_2 \cos\phi \sin\theta
+ \Be_3 \sin\phi \\
\thetacap
&= {R_\theta} ( \Be_2 ) {R_\theta}^\dagger \\
&= \Be_2 (\cos\theta + \Be_{12} \sin\theta) \\
&= - \Be_1 \sin\theta + \Be_2 \cos\theta \\
\phicap
&= {R_\theta} ( \Be_3 \cos\phi - \Be_1 \sin\phi ) {R_\theta}^\dagger \\
&= \Be_3 \cos\phi - \Be_1 \sin\phi (\cos\theta + \Be_{12} \sin\theta) \\
&= 
- \Be_1 \sin\phi \cos\theta 
- \Be_2 \sin\phi \sin\theta 
+ \Be_3 \cos\phi
\\
\end{align*}

\subsection{ An alternate pictorical derivation of the unit vectors. }

Somewhat more directly, $\rcap$ can be calculated from the coordinate expression of equation \ref{eqn:coordinates}

\begin{align*}
\rcap 
&= \inv{r} (x, y, z),
\end{align*}

which was found by inspection of the diagram.

For $\thetacap$, again from the figure, observe that it lies in an
latitudinal plane (ie: $x,y$ plane), and is perpendicular to the outwards radial vector in that plane.  That is

\begin{align*}
\thetacap 
&= (\cos\theta \Be_1 + \sin\theta \Be_2) \Be_1 \Be_2 \\
\end{align*}

Lastly, $\phicap$ can be calculated from the dual of $\rcap \wedge \thetacap$

\begin{align*}
\phicap 
&= - \Be_1 \Be_2 \Be_3 (\rcap \wedge \thetacap) \\
\end{align*}

Completing the algebra for the expressions above we have
\begin{align}
\rcap 
&=
\cos\phi \cos\theta \Be_1
+ \cos\phi \sin\theta \Be_2
+ \sin\phi \Be_3 \\
\thetacap 
&= \cos\theta \Be_2 - \sin\theta \Be_1 \\
\rcap \wedge \thetacap 
%&=
%\sin\phi \sin\theta \Be_1 \Be_3 
%- \cos\theta \sin\phi \Be_2 \Be_3 
%+ ( \cos\phi \sin\theta^2 + \cos\phi \cos\theta^2 ) \Be_1 \Be_2  \\
&=
\sin\phi \sin\theta \Be_1 \Be_3 
+ \sin\phi \cos\theta \Be_3 \Be_2 
+ \cos\phi \Be_1 \Be_2 \\
\phicap 
&=
- \sin\phi \cos\theta \Be_1 
- \sin\phi \sin\theta \Be_2 
+ \cos\phi \Be_3 %\\
\end{align}

Sure enough this produces the same result as with the rotor logic.

The rotor approach was purely algebraically and doesn't have
the same reliance on pictures.  That may have
an 
additional advantage
since one can then 
study any frame transformations of the general form $\{\Be_i'\} = \{ R \Be_i R^\dagger \}$, and produce results 
that apply to 
not only spherical polar coordinate systems but others such as the cylindrical polar.

\subsection{ Tensor transformation. }

Considering a linear transformation providing a mapping from one basis to another of the following form

\begin{align*}
f_i = \LL(e_i) = L e_i L^{-1}
\end{align*}

The coordinate representation, or Fourier decomposition, of the vectors in each of these frames is

\begin{align*}
x = x^i e_i = y^j f_j.
\end{align*}

Utilizing a reciprocal frame (ie: not yet requiring an orthonormal frame here), such that $e^i \cdot e_j = {\delta^i}_j$, 
then dot product provide the coordinate transformations
\begin{align*}
x^k e_k \cdot e^k &= y^j f_j \cdot e^k \\
y^j f_j \cdot f^i &= x^k e_k \cdot f^i \\
\implies \\
x^i &= y^j f_j \cdot e^i \\
y^i &= x^j e_j \cdot f^i
\end{align*}

The transformed reciprocal frame vectors can be expressed directly in terms of the initial reciprocal frame $f^i = \LL(e^i)$.  Taking
dot products confirms this

\begin{align*}
(L e_i L^{-1}) \cdot (L e^j L^{-1}) 
&= \gpgradezero{ L e_i L^{-1} L e^j L^{-1} } \\
&= \gpgradezero{ L e_i e^j L^{-1} } \\
&= e_i \cdot e^j \gpgradezero{ L L^{-1} } \\
&= e_i \cdot e^j
\end{align*}

This implies that the forward and inverse coordinate transformations may be summarized as
\begin{align*}
y^i &= x^j e_j \cdot \LL(e^i) \\
x^i &= y^j \LL(e_j) \cdot e^i \\
\end{align*}

Or in matrix form
\begin{align}\label{eqn:coordinateTxTensors}
\Lor{i}{j} &= \LL(e^i) \cdot e_j \\
\ILor{i}{j} &= \LL(e_j) \cdot e^i \\
y^i &= \Lor{i}{j} x^j \\
x^i &= \ILor{i}{j} y^j
\end{align}

The use of inverse notation is justified by the following

\begin{align*}
x^i &= \ILor{i}{k} y^k \\
&= \ILor{i}{k} \Lor{k}{j} x^j \\
\implies \\
\ILor{i}{k} \Lor{k}{j} &= \delta^i_j
\end{align*}

Some references such as \cite{MinahanTensors} use $\Lor{i}{j}$ for both the forward and inverse transformations, with specific conventions
about which index is varied to distinguish the two matrices.  I've found that confusing and have instead used the explicit inverse notation
of \cite{SpenceTensors}.

\subsection{ Gradient after change of coordinates. }

With the transformation matrixes enumerated above we are now equipt to take the gradient expressed in initial frame
\begin{align*}
\grad = \sum e^i \PD{x^i}{},
\end{align*}

and express it in the transformed frame.  The chain rule is required for the derivatives in terms of the transformed coordinates

\begin{align*}
\PD{x^i}{} 
&= \PD{x^i}{y^j} \PD{y^j}{} \\
&= \Lor{j}{i} \PD{y^j}{} \\
&= \LL(e^j) \cdot e_i \PD{y^j}{} \\
&= f^j \cdot e_i \PD{y^j}{}
\end{align*}

Therefore the gradient is
\begin{align*}
\grad &= \sum e^i (f^j \cdot e_i) \PD{y^j}{} \\
      &= \sum f^j \PD{y^j}{} \\
\end{align*}

This gets us most of the way towards the desired result for the spherical polar gradient since all that remains is a calculation of the $\PDi{y^j}{}$
values for
each of the $\rcap$, $\thetacap$, and $\phicap$ directions.

It is also interesting to observe (as in \cite{DenkerMaxwell}) that the gradient can also be written as

\begin{align*}
\grad &= \inv{f_j} \PD{y^j}{} \\
\end{align*}

Observe the similarity to the Fourier component decomposition of the vector itself $x = f_i y^i$.  Thus, roughly speaking, the differential operator
parts of the gradient can be seen to be directional derivatives 
along the directions of each of the frame vectors.

This is sufficient to read the elements of distance in each of the directions
off the figure

\begin{align*}
\delta \Bx \cdot \rcap &= \delta r \\
\delta \Bx \cdot \thetacap &= r \cos\phi \delta \theta \\
\delta \Bx \cdot \phicap &= r \delta \theta \\
\end{align*}

Therefore the gradient is just
\begin{align}
\grad = 
\rcap \PD{r}{}
+\thetacap \inv{r \cos\phi} \PD{\theta}{} 
+\phicap \inv{r} \PD{\phi}{}
\end{align}

Although this last bit has been
derived graphically, and not analyitically, it does
clarify the original question of exactly angle and unit vector 
conventions were intended in the text (polar angle measured from the North pole, not equator, and $\theta$, and $\phi$ reversed).

This was the long way to that particular result, but this has been
an exploratory treatment of frame rotation concepts that I personally 
felt the need to clarity for myself.

There are still some additional details that I will explore before concluding
(including an analyitic treatment of the above).

% FIXME: wrong!
%%%\subsection{ Element of distance along the curves. }
%%%
%%%Using the figure, one can observe that the distances along in each of the spherical polar unit vector directions are obtained in
%%%this particular case by varying each coordinate in turn.
%%%
%%%\begin{itemize}
%%%\item Along $\rcap$, a vectorial element of distance is just
%%%
%%%\begin{align*}
%%%\delta \Bx_r 
%%%&= (r + \delta r)\rcap - r\rcap \\
%%%&= \delta r \rcap.
%%%\end{align*}
%%%
%%%\item Along the $\phicap$ direction?
%%%
%%%Here one wants the great circle path obtained by fixing $\theta$ and $r$.  A difference of such position vectors 
%%%(in the standard basis) is
%%%
%%%\begin{align*}
%%%r 
%%%\begin{bmatrix}
%%%(\cos(\phi + \delta \phi) - \cos(\phi)) \cos(\theta) \\
%%%(\cos(\phi + \delta \phi) - \cos(\phi)) \sin(\theta) \\
%%%\sin(\phi + \delta \phi) - \sin(\phi) \\
%%%\end{bmatrix}
%%%&\approx
%%%r 
%%%\begin{bmatrix}
%%%-\sin(\phi) \cos(\theta) \\
%%%-\sin(\phi) \sin(\theta) \\
%%%\cos(\phi) \\
%%%\end{bmatrix}
%%%\delta \phi
%%%\end{align*}
%%%
%%%This is an unsatisfactory way to express the directed distance, since it is in terms of $\Be_i$.  We have relationships
%%%for $\rcap$, $\thetacap$, $\phicap$ in terms of $\Be_i$ and could invert that and multiply it out, but that is going to
%%%make things even messier before things get simpler.
%%%
%%%Instead form the unit bivector for the north south oriented great circle plane through the point of interest
%%%
%%%\begin{align*}
%%%j = \rcap \wedge \phicap
%%%\end{align*}
%%%
%%%For a point $\Bx$ we want to consider an incremental change in position along the $\phicap$ direction.  Forming the 
%%%projection and rejection from the plane we have
%%%
%%%\begin{align*}
%%%\Bx = \Bx j \inv{j} = (\Bx \cdot j) \inv{j} + (\Bx \wedge j) \inv{j} = \Bx_\parallel + \Bx_\perp
%%%\end{align*}
%%%
%%%So rotating to $\Bx'$ and taking differences we have 
%%%
%%%\begin{align*}
%%%\delta \Bx_\phi 
%%%&= \Bx' - \Bx \\
%%%&= \Bx_\parallel' -\Bx_\parallel \\
%%%&= \Bx_\parallel (\exp(j\delta \phi) - 1) \\
%%%&\approx \Bx_\parallel j \delta \phi \\
%%%&= (\Bx \cdot j) \inv{j} j \delta \phi \\
%%%&= (\Bx \cdot j) \delta \phi
%%%\end{align*}
%%%
%%%\item How about along the $\thetacap$ direction?
%%%Intuitively, one expects the magnitude to be $r \delta \theta$.  The algebra will be exactly the same as with $\phicap$ direction
%%%but we have a different bivector for the plane.  Let $i = \Be_1\Be_2$ we have
%%%
%%%\begin{align*}
%%%\delta \Bx_\theta = (\Bx \cdot i) \delta \theta
%%%\end{align*}
%%%
%%%\end{itemize}
%%%
%%%Now all this is a bit inexact.  What exactly are these $\delta$ increments?  They have to be small enough that they can be considered to be just along the
%%%time unit vectors for the rotated frame.

\subsection{ Notes on transformation of frame vectors vs. coordinates. }

To avoid confusion it is worth noting how the frame vectors vs. the components themselves differ under
rotational transformation.  Consideration of the example of a pair of orthonormal unit vectors for the plane illustrates this

\begin{align*}
\Be_1' &= \Be_1 \exp(\Be_{12}\theta) = \Be_1 \cos\theta + \Be_2 \sin\theta \\
\Be_2' &= \Be_2 \exp(\Be_{12}\theta) = \Be_2 \cos\theta - \Be_1 \sin\theta \\
\end{align*}

Forming a matrix for the transformation of these unit vectors we have
\begin{align*}
\begin{bmatrix}
\Be_1' \\
\Be_2'
\end{bmatrix}
=
\begin{bmatrix}
\cos\theta & \sin\theta \\
- \sin\theta & \cos\theta \\
\end{bmatrix}
\begin{bmatrix}
\Be_1 \\
\Be_2
\end{bmatrix}
\end{align*}

Now compare this to the transformation of a vector in its entirety

\begin{align*}
y^1 e_1 + y^2 e_2 
&= ( x^1 \Be_1 + x^2 \Be_2 ) \exp(\Be_{12}\theta) \\
&= x^1(\Be_1 \cos\theta + \Be_2 \sin\theta) 
 + x^2(\Be_2 \cos\theta - \Be_1 \sin\theta) \\
\end{align*}

And in matrix form this is

\begin{align*}
\begin{bmatrix}
y_1 \\
y_2
\end{bmatrix}
=
\begin{bmatrix}
\cos\theta & -\sin\theta \\
\sin\theta & \cos\theta \\
\end{bmatrix}
\begin{bmatrix}
x_1 \\
x_2
\end{bmatrix}
\end{align*}

Note the inversion of the transformation matrix here compared to the frame vector transformation itself.

%%%\section{ FIXME or delete what comes after this. }
%%%
%%%\subsection{ Expressing spherical polar rotation in matrix form. }
%%%
%%%For the spherical polar case, both sets of frame vectors are orthonormal
%%%
%%%\begin{align*}
%%%\Lor{i}{j}
%%%&= \LL(\Be_i) \cdot \Be_j \\
%%%&= (R^\dagger \Be_i R) \cdot \Be_j \\
%%%\end{align*}
%%%
%%%\begin{align}\label{eqn:rotation}
%%%\Lor{i}{j}
%%%&=
%%%\begin{bmatrix}
%%%%rcap . {e1 e2 e3}
%%%\cos\phi \cos\theta & \cos\phi \sin\theta & \sin\phi \\
%%%%thetacap . {e1 e2 e3}
%%%- \sin\theta & \cos\theta & 0 \\
%%%%phicap . {e1 e2 e3}
%%%- \sin\phi \cos\theta & - \sin\phi \sin\theta & \cos\phi \\
%%%\end{bmatrix}
%%%\end{align}
%%%
%%%From equation \ref{eqn:coordinateTxTensors}, one can see that the inverse transformation is the transpose.  This isn't clear looking at the matrix above until one factors it into a pair of separate matrixes
%%%\begin{align*}
%%%\Lor{i}{j}
%%%&=
%%%\begin{bmatrix}
%%%\cos\phi & 0 & \sin\phi \\
%%%0 & 1 & 0 \\
%%%-\sin\phi & 0 & \cos\phi
%%%\end{bmatrix}
%%%\begin{bmatrix}
%%%\cos\theta & \sin\theta & 0 \\
%%%-\sin\theta & \cos\theta & 0 \\
%%%0 & 0 & 1 \\
%%%\end{bmatrix}
%%%\end{align*}
%%%
%%%FIXME: from my Euler angle notes, I would have expected this matrix factorization to be in inverted order.  There's an inconsistency
%%%or misunderstanding here or there somewhere, but I can't find anything wrong in either place.
%%%
%%%%Hmm, after doing this, which I thought would shed some light on how to transform the gradient (and divergence and curl), but it isn't clear to me
%%%%that this was completely helpful.
%%%
%%%\subsection{ Verification of rotation matrix, using only matrix notation. }
%%%
%%%As a verification of \ref{eqn:rotation} lets calculate that directly.  The initial rotation is in the $x,y$ plane around the $-\Be_2' = -\Be_2 \exp(\Be_{12}\theta) = -\Be_2 \cos\theta + \Be_1 \sin\theta$ axis.
%%%
%%%From \cite{PJRotor} we have the rotation matrix for a $\phi$ rotation about
%%%unit vector $\Bn = (n_1, n_2, n_3) = (-\cos\theta, \sin\theta, 0) = (-C_\theta, S_\theta, 0)$ is
%%%
%%%\begin{align*}
%%%R_\phi R_\theta
%%%=
%%%\begin{bmatrix}
%%%\cos\phi(1 +{C_\theta}^2) - {C_\theta}^2 & -{C_\theta} {S_\theta} (1-\cos\phi) & {S_\theta} \sin\phi \\
%%%-{C_\theta} {S_\theta} (1-\cos\phi) & \cos\phi(1 -{S_\theta}^2) + {S_\theta}^2 & {C_\theta} \sin\phi \\
%%%-{S_\theta} \sin\phi & {-C_\theta} \sin\phi & \cos\phi \\
%%%\end{bmatrix}
%%%\begin{bmatrix}
%%%C_\theta & -S_\theta & 0 \\
%%%S_\theta & C_\theta & 0 \\
%%%0 & 0 & 1 \\
%%%\end{bmatrix}
%%%\end{align*}
%%%

\bibliographystyle{plainnat}
\bibliography{myrefs}

\end{document}

% 
% 
% 
% Copyright � 2012 Peeter Joot
% All Rights Reserved
% 
% This file may be reproduced and distributed in whole or in part, without fee, subject to the following conditions:
% 
% o The copyright notice above and this permission notice must be preserved complete on all complete or partial copies.
% 
% o Any translation or derived work must be approved by the author in writing before distribution.
% 
% o If you distribute this work in part, instructions for obtaining the complete version of this file must be included, and a means for obtaining a complete version provided.
% 
% 
% Exceptions to these rules may be granted for academic purposes: Write to the author and ask.
% 
% 
% 
\chapter{Rotor interpolation calculation.}
\label{chap:slerp}
\date{Nov 30, 2008.  slerp.tex}

The aim is to compute the interpolating rotor $r$ that takes an object
from one position to another in $n$ steps.
Here the initial and final positions are given by two rotors $R_1$, and $R_2$
like so

\begin{align*}
X_1 &= R_1 X {R_1}^\dagger \\
X_2 &= R_2 X {R_2}^\dagger = r^n R_1 X {R_1}^\dagger {r^n}^\dagger
\end{align*}

So, writing 

\begin{align*}
%r^n R_1 = R_2 
a = r^n = R_2 \inv{R_1} = \frac{R_2 {R_1}^\dagger}{R_1 {{R_1}^\dagger}} = \cos\theta + I \sin\theta
\end{align*}

So, 

\begin{align*}
\frac{\gpgradetwo{a}}{\gpgradezero{a}} &= 
\frac{\gpgradetwo{a}}{\Abs{\gpgradetwo{a}}} \frac{\Abs{\gpgradetwo{a}}}{\gpgradezero{a}} \\
&= I \tan\theta
\end{align*}

Therefore the interpolating rotor is:
\begin{align*}
I &= \frac{\gpgradetwo{a}}{\Abs{\gpgradetwo{a}}} \\
\theta &= \atan2\left(\Abs{\gpgradetwo{a}}, \gpgradezero{a}\right) \\
r &= \cos(\theta/n) + I \sin(\theta/n)
\end{align*}

In \citep{dorst2007gac}, equation $10.15$, they've got something like this
for a fractional angle, but then say that they don't use that in software, 
instead using $r$ directly, with a comment about designing more sophisticated
algorithms (bivector splines).  That spline comment in particular sounds
interesting.  Sounds like the details on that are to be found in the journals
mentioned in Further Reading section of chapter 10.

%
% Copyright � 2012 Peeter Joot.  All Rights Reserved.
% Licenced as described in the file LICENSE under the root directory of this GIT repository.
%

%
%
\chapter{Exponential of a blade}
\index{blade!exponentiation}
\label{chap:kvectorExponential}
%\date{March 12, 2008.  kvectorExponential.tex}

\section{Motivation}

Exponentials of bivectors and complex numbers are useful as generators of rotations, and exponentials of
square matrices can be used in linear differential equation solution.

How about exponentials of vectors?

Because any power of a vector can be calculated it should be perfectly well defined to use the exponential infinite series with k-vector parameters.  An
exponential function of this form will be expanded explicitly and compared to the real number result.  The first
derivative will also be calculated to examine its form.

In addition for completeness, the bivector and quaternion exponential forms will be examined.

\section{Vector Exponential}

The infinite series representation of the exponential defines a function for any \(x\) that can be repeatedly multiplied
with it self.

\begin{equation}
e^x = \sum_{k=0}^{\infty} \frac{x^k}{k!}
\end{equation}

Depending on the type of the parameter \(x\) this may or may not have properties consistent with
the real number exponential function.
For a vector \(\Bx = \xcap\abs{\Bx}\), after splitting the sum into even and odd terms this infinite series takes the
following form:

\begin{equation*}
e^{\pm\Bx}
=
\sum_{k=0}^{\infty} \frac{\Bx^{2k}}{(2k)!}
\pm
\sum_{k=0}^{\infty} \frac{\abs{\Bx}^{2k}\abs{\Bx}\xcap}{(2k+1)!}
\end{equation*}

\begin{equation}
\implies
e^{\pm\Bx}
=
\cosh\abs{\Bx}
\pm
\xcap \sinh\abs{\Bx}
\end{equation}

One can also employ symmetric and antisymmetric sums to write the hyperbolic functions in terms of the
vector exponentials:

\begin{equation}\label{eqn:kvectorExponential:20}
\cosh\abs{\Bx} = \frac{e^\Bx + e^{-\Bx}}{2}
\end{equation}
\begin{equation}\label{eqn:kvectorExponential:40}
\sinh\abs{\Bx} = \frac{e^\Bx - e^{-\Bx}}{2\xcap}
\end{equation}

\subsection{Vector Exponential derivative}
One of the defining properties of the exponential is that its derivative is related to itself

\begin{equation}\label{eqn:kvectorExponential:60}
\ddu{e^{x}} = \ddu{x}e^{x} = e^{x} \ddu{x}
\end{equation}

For a vector parameter \(\Bx\) one should not generally expect that.  Let us expand this to see the form of this
derivative:

\begin{equation}\label{eqn:kvectorExponential:180}
\begin{aligned}
\ddu{e^{\Bx}}
&= \ddu{}( \cosh\abs{\Bx} + \xcap \sinh\abs{\Bx} ) \\
&= ( \sinh\abs{\Bx} + \xcap \cosh\abs{\Bx} ) \ddu{\abs{\Bx}} + \ddu{\xcap} \sinh\abs{\Bx} \\
\end{aligned}
\end{equation}

Can calculate \(\ddu{\abs{\Bx}}\) with the usual trick:

\begin{equation}\label{eqn:kvectorExponential:80}
\ddu{\abs{\Bx}^2} = 2\abs{\Bx}\ddu{\abs{\Bx}} = \ddu{\Bx}\Bx + \Bx\ddu{\Bx} = 2 \ddu{\Bx} \cdot \Bx
\end{equation}

\begin{equation}\label{eqn:kvectorExponential:100}
\implies
\ddu{\abs{\Bx}} = \ddu{\Bx} \cdot \xcap
\end{equation}

Calculation of \(\ddu{\xcap}\) uses this result:

\begin{equation}\label{eqn:kvectorExponential:200}
\begin{aligned}
\ddu{\xcap}
&= \ddu{}\frac{\Bx}{\abs\Bx}  \\
&= \ddu{\Bx}\inv{\abs\Bx} - \frac{\Bx}{\abs{\Bx}^2}\ddu{\abs{\Bx}} \\
&= \ddu{\Bx}\inv{\abs\Bx} - \frac{\Bx}{\abs{\Bx}^2} \ddu{\Bx} \cdot \xcap \\
&= \inv{\abs\Bx}\left( \ddu{\Bx} - \xcap \left(\ddu{\Bx} \cdot \xcap\right) \right) \\
&= \frac{\xcap}{\abs\Bx} \left(\xcap \wedge \ddu{\Bx} \right) \\
&= \inv{\abs\Bx} {\RejName_{\xcap}\left(\ddu{\Bx}\right)} \\
\end{aligned}
\end{equation}

Putting these together one write the derivative in a few ways:

\begin{equation}\label{eqn:kvectorExponential:220}
\begin{aligned}
\ddu{e^{\Bx}}
&=
\left(\ddu{\Bx} \cdot \xcap\right) \xcap
( \xcap \sinh\abs{\Bx} + \cosh\abs{\Bx} )
 + \frac{\xcap}{\abs\Bx} \left(\xcap \wedge \ddu{\Bx} \right) \sinh\abs{\Bx} \\
&=
\xcap \left(\ddu{\Bx} \cdot \xcap\right)e^{\Bx}
 + \frac{\xcap}{\abs\Bx} \left(\xcap \wedge \ddu{\Bx} \right) \sinh\abs{\Bx} \\
&=
\Proj_{\xcap}\left(\ddu{\Bx}\right) e^{\Bx}
 + \inv{\abs\Bx} \RejName_{\xcap}\left(\ddu{\Bx}\right) \sinh\abs{\Bx} \\
\end{aligned}
\end{equation}

This is considerably different from the real number case.  Only when the vector \(\Bx\) and all its variation
\(\ddu{\Bx}\) are colinear does \(\ddu{\Bx} = \Proj_{\xcap}\left(\ddu{\Bx}\right)\) for the real number like result:

\begin{equation}
\ddu{e^{\Bx}} = \ddu{\Bx} e^{\Bx} = e^{\Bx} \ddu{\Bx}
\end{equation}

Note that the \(\sinh\) term can be explicitly removed

\begin{equation}\label{eqn:kvectorExponential:240}
\begin{aligned}
\ddu{e^{\Bx}}
=
\left(\xcap \left(\ddu{\Bx} \cdot \xcap\right) - \inv{2\abs\Bx}\left(\xcap \wedge \ddu{\Bx} \right) \right) e^{\Bx}
 - \inv{2\abs\Bx}\left(\xcap \wedge \ddu{\Bx} \right) e^{-\Bx} \\
\end{aligned}
\end{equation}

, but without a \(\RejName_{\xcap}\left(\ddu{\Bx}\right) = 0\)
constraint, there will always be a term that is not proportional to \(e^{\Bx}\).

\section{Bivector Exponential}

The bivector exponential can be expanded utilizing its complex number equivalence:

\begin{equation}\label{eqn:kvectorExponential:260}
\begin{aligned}
e^{\BB}
&= e^{\Bcap\Babs} \\
&= \cos{\Babs} + \Bcap\sin{\Babs} \\
\end{aligned}
\end{equation}

So, taking the derivative we have

\begin{equation}\label{eqn:kvectorExponential:280}
\begin{aligned}
(e^{\BB})'
&= \left(-\sin\Babs + \Bcap\cos\Babs\right)\Babs' + \Bcap' \sin\Babs \\
&= \Bcap \left(\Bcap\sin\Babs + \cos\Babs\right)\Babs' + \Bcap' \sin\Babs \\
&= \Bcap e^{\BB} \Babs' + \Bcap' \sin\Babs \\
&= e^{\BB} \Bcap \Babs' + \Bcap' \sin\Babs \\
\end{aligned}
\end{equation}

\subsection{bivector magnitude derivative}

As with the vector case we have got a couple helper derivatives required.  Here is the first:

\begin{equation}\label{eqn:kvectorExponential:300}
\begin{aligned}
({\Babs^2})' &= 2\Babs\Babs' = -(\BB\BB' + \BB'\BB) \\
\implies \\
{\Babs}' &= -\frac{\Bcap\BB' + \BB'\Bcap}{2} \\
\end{aligned}
\end{equation}

Unlike the vector case this last expression is not a bivector dot product \(= -\Bcap\cdot \BB'\) since there could be a
\(\gpgradefour{}\) term that this symmetric sum would also include.
That wedge term would be zero for example if \(\BB = \Bx \wedge \Bk\) for a constant vector \(\Bk\).

\subsection{Unit bivector derivative}

Now calculate \(\Bcap'\):

\begin{equation}\label{eqn:kvectorExponential:320}
\begin{aligned}
\Bcap'
&= \frac{\BB'}{\Babs} - \frac{\BB}{\Babs^2}\Babs' \\
&= \inv{\Babs}\left( \BB' + {\Bcap}\frac{\Bcap\BB' + \BB'\Bcap}{2} \right) \\
&= \inv{2\Babs}\left( \BB' + \Bcap\BB'\Bcap \right) \\
&= \frac{\Bcap}{\Babs}\frac{-\Bcap\BB' + \BB'\Bcap}{2} \\
\end{aligned}
\end{equation}

Thus, the derivative is a scaled bivector rejection:
\begin{equation}
\Bcap' = \inv{\BB}\gpgradetwo{\Bcap\BB'}
\end{equation}

Although this appears different from a unit vector derivative, a slight adjustment highlights the
similarities:

\begin{equation}\label{eqn:kvectorExponential:340}
\begin{aligned}
\rcap'
&= \frac{\rcap}{\abs\Br}\rcap \wedge \Br' \\
&= \inv{\Br}\gpgradetwo{\rcap\Br'} \\
\end{aligned}
\end{equation}

Note however the sign inversion that is built into the bivector inversion.

\subsection{combining results}

Putting the individual results back together we have:

\begin{equation}
(e^{\BB})'
= \inv{\Bcap}\frac{\Bcap\BB' + \BB'\Bcap}{2} e^{\BB} + \inv{\BB}\gpgradetwo{\Bcap\BB'} \sin\Babs
\end{equation}

In general with bivectors we can have two sorts of perpendicularity.  The first is perpendicular but intersecting (generated by the grade 2 term of the product), and perpendicular with no common line (generated by the grade 4 term).
In \R{3} we have only the first sort.

With a restriction that the derivative only changes the bivector enough to introduce the first term, this
exponential derivative is reduced to:

\begin{equation}\label{eqn:kvectorExponential:360}
\begin{aligned}
(e^{\BB})'
&= \inv{\Bcap}\Bcap \cdot \BB' e^{\BB} + \inv{\BB}\gpgradetwo{\Bcap\BB'} \sin\Babs \\
&= \Proj_{\Bcap}(\BB') e^{\BB} + \inv{\Babs}{\RejName_{\Bcap}(\BB')} \sin\Babs \\
\end{aligned}
\end{equation}

Only if the bivector variation is in the same plane as the bivector itself can the \(\gpgradetwo{}\) term be dropped
in which case, since the derivative will equal its projection one has:

\begin{equation}
(e^{\BB})' = \BB' e^{\BB} = e^{\BB} \BB'
\end{equation}

\section{Quaternion exponential derivative}

Using the phrase somewhat loosely a quaternion, or complex number is a multivector of the form

\begin{equation}\label{eqn:kvectorExponential:120}
\alpha + \BB
\end{equation}

Where \(\alpha\) is a scalar, and \(\BB\) is a bivector.

Using the results above, the derivative of a quaternion exponential (ie: a rotation operator) will be

\begin{equation}\label{eqn:kvectorExponential:380}
\begin{aligned}
(e^{\alpha + \BB})'
&= (e^{\alpha})' e^{\BB} + e^{\alpha} (e^{\BB})' \\
&= \alpha' e^{\alpha + \BB} + e^{\alpha}
\inv{\Bcap}\frac{\Bcap\BB' + \BB'\Bcap}{2} e^{\BB} + \inv{\BB}\gpgradetwo{\Bcap\BB'} e^{\alpha} \sin\Babs \\
\end{aligned}
\end{equation}

For the total derivative:
\begin{equation}
(e^{\alpha + \BB})'
= \left(\alpha' + \inv{\Bcap}\frac{\Bcap\BB' + \BB'\Bcap}{2} \right) e^{\alpha + \BB}
+ \inv{\BB}\gpgradetwo{\Bcap\BB'} e^{\alpha} \sin\Babs
\end{equation}

As with the bivector case, the two restrictions
\(\gpgradetwo{\Bcap\BB'} = 0\), and \(\gpgradefour{\Bcap\BB'} = 0\)
are required to get a real number like exponential derivative:

\begin{equation}
(e^{\alpha + \BB})'
= \left(\alpha + \BB\right)' e^{\alpha + \BB}
= e^{\alpha + \BB} \left(\alpha + \BB\right)'
\end{equation}

Note that both of these are true for the important class of multivectors, the complex number.

\subsection{bivector with only one degree of freedom}

For a bivector that includes a constant vector such as \(\BB = \Bx \wedge \Bk\) there will be no
\(\gpgradefour{}\) term.

\begin{equation}\label{eqn:kvectorExponential:140}
\gpgradefour{\Bcap\BB'}
\propto \gpgradefour{\Bx \wedge \Bk \Bx' \wedge \Bk}
= \Bx \wedge \Bk \wedge \Bx' \wedge \Bk
= 0
\end{equation}

Suppose \(\alpha + \BB = \Bx\Bk = \Bx \cdot \Bk + \Bx \wedge \Bk\).  In this case
this quaternion exponential derivative reduces to

\begin{equation}\label{eqn:kvectorExponential:400}
\begin{aligned}
(e^{\Bx\Bk})'
&= \left(\Bx' \cdot \Bk + \inv{\Bx \wedge \Bk} (\Bx \wedge \Bk) \cdot (\Bx' \wedge \Bk) \right) e^{\Bx\Bk} \\
&+ \inv{\Bx \wedge \Bk}\gpgradetwo{\frac{\Bx \wedge \Bk}{\abs{\Bx \wedge \Bk}}\Bx' \wedge \Bk} e^{\Bx \cdot \Bk} \sin\abs{\Bx \wedge \Bk} \\
\end{aligned}
\end{equation}

It is only with the addition restriction that all the bivector variation lies in the plane \(\Bi = \frac{\Bx \wedge \Bk }{ \abs{\Bx \wedge \Bk} }\).  ie:

\begin{equation}\label{eqn:kvectorExponential:160}
\gpgradetwo{\Bx \wedge \Bk \Bx' \wedge \Bk} = 0
\end{equation}

does one have:

\begin{equation}\label{eqn:kvectorExponential:420}
\begin{aligned}
(e^{\Bx\Bk})'
&= \left(\Bx' \cdot \Bk + \Proj_{\Bi}(\Bx' \wedge \Bk) \right) e^{\Bx\Bk} \\
&= \left(\Bx' \cdot \Bk + \Bx' \wedge \Bk \right) e^{\Bx\Bk} \\
\end{aligned}
\end{equation}

Thus with these two restrictions to the variation of the bivector term we have:

\begin{equation}
(e^{\Bx\Bk})' = \Bx'\Bk e^{\Bx\Bk} = e^{\Bx\Bk} \Bx'\Bk
\end{equation}

\part{Calculus.}
%
% Copyright � 2012 Peeter Joot.  All Rights Reserved.
% Licenced as described in the file LICENSE under the root directory of this GIT repository.
%

%
%
\chapter{Developing some intuition for Multivariable and Multivector Taylor Series}\label{chap:PJmultiTaylors}
\index{Taylor series}
%\date{April 28, 2009.  multivectorTaylors.tex}

The book \citep{doran2003gap} uses Geometric Calculus heavily in its Lagrangian treatment.  In particular it is used in some incomprehensible seeming ways in the stress energy tensor treatment.

In the treatment of transformation of the dependent variables (not the field variables themselves) of field Lagrangians, there is one bit that appears to be the first order linear term from a multivariable Taylor series expansion.  Play with multivariable Taylor series here a bit to develop some intuition with it.

\section{Single variable case, and generalization of it}

For the single variable case, Taylor series takes the form

\begin{equation}\label{eqn:multivectorTaylors:20}
\begin{aligned}
f(x) = \sum \frac{x^k}{k!} \left. \frac{d^k f(x)}{dx^k} \right\vert_{x=0}
\end{aligned}
\end{equation}

or
\begin{equation}\label{eqn:multivectorTaylors:40}
\begin{aligned}
f(x_0 + \epsilon) = \sum \frac{\epsilon^k}{k!} \left. \frac{d^k f(x)}{dx^k} \right\vert_{x=x_0}
\end{aligned}
\end{equation}

As pointed out in \citep{byron1992mca}, this can (as they demonstrated for polynomials) be put into exponential
operator form

\begin{equation}\label{eqn:multivectorTaylors:60}
\begin{aligned}
f(x_0 + \epsilon) = \left. e^{\epsilon d/dx} f(x) \right\vert_{x=x_0}
\end{aligned}
\end{equation}

Without proof, the multivector generalization of this is

\begin{equation}\label{eqn:multivectorTaylors:taylorsExponential}
\begin{aligned}
f(x_0 + \epsilon)
&= \left. e^{\epsilon \cdot \grad} f(x) \right\vert_{x=x_0}
\end{aligned}
\end{equation}

Or in full,

\begin{equation}\label{eqn:multivectorTaylors:taylorMulti}
\begin{aligned}
f(x_0 + \epsilon)
&= \sum \inv{k!} \left. {(\epsilon \cdot \grad)^k} f(x) \right\vert_{x=x_0}
\end{aligned}
\end{equation}

Let us work with this, and develop some comfort with what it means, then revisit the proof.

\section{Directional Derivatives}

First a definition of directional derivative is required.

In
\href{http://tutorial.math.lamar.edu/Classes/CalcIII/DirectionalDeriv.aspx}{standard two variable vector calculus} the directional derivative is defined in one of the following ways
\begin{equation}\label{eqn:multivectorTaylors:80}
\begin{aligned}
\spacegrad_\Bu f(x,y) &= \lim_{h \rightarrow 0} \frac{f(x + a h, y + b h) - f(x,y)}{h} \\
\Bu &= (a,b)
\end{aligned}
\end{equation}

Or \href{http://en.wikipedia.org/wiki/Directional_derivative}{in a more general vector form} as

\begin{equation}\label{eqn:multivectorTaylors:100}
\begin{aligned}
\spacegrad_\Bu f(\Bx) &= \lim_{h \rightarrow 0} \frac{f(\Bx + h\Bu) - f(\Bx)}{h}
\end{aligned}
\end{equation}

Or \href{http://mathworld.wolfram.com/DirectionalDerivative.html}{in terms of the gradient} as
\begin{equation}\label{eqn:multivectorTaylors:120}
\begin{aligned}
\spacegrad_\Bu f(\Bx) &= \frac{\Bu}{\Abs{\Bu}} \cdot \spacegrad f
\end{aligned}
\end{equation}

Each of these was for a vector parametrized scalar function, although the wikipedia article does mention a vector valued form that is identical to that use by \citep{doran2003gap}.  Specifically, that is

\begin{equation}\label{eqn:multivectorTaylors:140}
\begin{aligned}
(\epsilon \cdot \grad) f(x)
&= \lim_{h \rightarrow 0} \frac{f(x + h \epsilon) - f(x)}{h} \\
&= \left. \PD{h}{f(x + h \epsilon)} \right\vert_{h=0}
\end{aligned}
\end{equation}

Observe that this definition as a limit avoids the requirement to define the gradient upfront.  That definition is not necessarily obvious especially for multivector valued functions.

%\section{Work some examples}

\makeexample{First order linear vector polynomial}{example:multivectorTaylors:141}{

Let

\begin{equation}\label{eqn:multivectorTaylors:160}
\begin{aligned}
f(x) = a + x
\end{aligned}
\end{equation}

For this simplest of vector valued vector parametrized functions we have

\begin{equation}\label{eqn:multivectorTaylors:180}
\begin{aligned}
\PD{h}{f(x + h \epsilon)}
&= \PD{h}{} (a + x + h \epsilon) \\
&= \epsilon \\
&= (\epsilon \cdot \grad) f
\end{aligned}
\end{equation}

with no requirement to evaluate at \(h=0\) to complete the directional derivative computation.

The Taylor series expansion about \(0\) is thus

\begin{equation}\label{eqn:multivectorTaylors:200}
\begin{aligned}
f(\epsilon)
&= \left. (\epsilon \cdot \grad)^0 f \right\vert_{x=0} + \left. (\epsilon \cdot \grad)^1 f  \right\vert_{x=0} \\
&= a + \epsilon \\
\end{aligned}
\end{equation}

Nothing else could be expected.
}

\makeexample{Second order vector parametrized multivector polynomial}{example:multivectorTaylors:201}{

Now, step up the complexity slightly, and introduce a multivector valued second degree polynomial, say,

\begin{equation}\label{eqn:multivectorTaylors:secondOrder}
\begin{aligned}
f(x) = \alpha + a + x y + w x + c x^2 + d x e + x g x
\end{aligned}
\end{equation}

Here \(\alpha\) is a scalar, and all the other variables are vectors, so we have grades \(\le 3\).

For the first order partial we have
\begin{equation}\label{eqn:multivectorTaylors:220}
\begin{aligned}
&\PD{h}{f(x + h \epsilon)} \\
&= \PD{h}{} ( \alpha + a + (x + h\epsilon) y + w (x + h\epsilon) + c (x + h\epsilon)^2 + d (x + h\epsilon) e + (x + h\epsilon) g (x + h\epsilon) ) \\
&=
\epsilon y
+ w \epsilon
+ c \epsilon (x + h\epsilon)
+ c (x + h\epsilon) \epsilon
+ c \epsilon
+ d \epsilon e
+ \epsilon g (x + h\epsilon)
+ (x + h\epsilon) g \epsilon \\
\end{aligned}
\end{equation}

Evaluation at \(h=0\) we have

\begin{equation}\label{eqn:multivectorTaylors:240}
\begin{aligned}
(\epsilon \cdot \grad) f
&=
\epsilon y
+ w \epsilon
+ c \epsilon x
+ c x \epsilon
+ c \epsilon
+ d \epsilon e
+ \epsilon g x
+ x g \epsilon \\
\end{aligned}
\end{equation}

By inspection we have

\begin{equation}\label{eqn:multivectorTaylors:260}
\begin{aligned}
(\epsilon \cdot \grad)^2 f
&=
+ 2 c \epsilon^2
+ 2 \epsilon g \epsilon \\
\end{aligned}
\end{equation}

Combining things forming the Taylor series expansion about the origin we should recover our function

\begin{equation}\label{eqn:multivectorTaylors:280}
\begin{aligned}
f(\epsilon)
&= \inv{0!} \left. (\epsilon \cdot \grad)^0 f \right\vert_{x=0}
+ \inv{1!} \left. (\epsilon \cdot \grad)^1 f \right\vert_{x=0}
+ \inv{2} \left. (\epsilon \cdot \grad)^2 f \right\vert_{x=0} \\
&= \inv{1} (\alpha + a) + \inv{1} (\epsilon y + w \epsilon + c \epsilon + d \epsilon e ) + \inv{2}(2 c \epsilon^2 + 2 \epsilon g \epsilon ) \\
&= \alpha + a + \epsilon y + w \epsilon + c \epsilon + d \epsilon e + c \epsilon^2 + \epsilon g \epsilon \\
% cf:
%&f(x) = \alpha + a + x y + w x + c x^2 + d x e + x g x
\end{aligned}
\end{equation}

This should match \eqnref{eqn:multivectorTaylors:secondOrder}, with an \(x = \epsilon\) substitution, and does.  With the vector factors in these functions commutativity assumptions could not be made.  These calculations help provide a small verification that this form of Taylor series does in fact work out fine with such non-commutative variables.

Observe as well that there was really no requirement in this example that \(x\) or any of the other factors to be vectors.  If they were all bivectors or trivectors or some mix the calculations would have had the same results.
}

\section{Proof of the multivector Taylor expansion}

A peek back into \citep{hestenes1999nfc} shows that \eqnref{eqn:multivectorTaylors:taylorMulti} was in fact proved, but it was done in a very sneaky and clever way.  Rather than try to prove treat the multivector parameters explicitly, the following scalar parametrized hybrid function was created

\begin{equation}\label{eqn:multivectorTaylors:300}
\begin{aligned}
G(\tau) &= F(\Bx_0 + \tau\Ba)
\end{aligned}
\end{equation}

The scalar parametrized function \(G(\tau)\) can be Taylor expanded about the origin, and then evaluated at \(1\) resulting in \eqnref{eqn:multivectorTaylors:taylorMulti} in terms of powers of \((\Ba \cdot \grad)\).  I will not reproduce or try to enhance that proof for myself here since it is actually quite clear in the text.  Obviously the trick is non-intuitive enough that when thinking about how to prove this myself it did not occur to me.

\section{Explicit expansion for a scalar function}

Now, despite the \(a \cdot \grad\) notation being unfamiliar seeming, the end result is not.  Explicit expansion of this for a vector to scalar mapping will show this.  In fact this will also account for the \href{http://en.wikipedia.org/wiki/Hessian_matrix}{Hessian matrix}, as in

\begin{equation}\label{eqn:multivectorTaylors:320}
\begin{aligned}
y = f(\mathbf{x}+\Delta\mathbf{x}) \approx f(\mathbf{x}) + J(\mathbf{x}) \Delta \mathbf{x} %+\frac{1}{2} \Delta {\mathbf{x}}^\txtT H(\mathbf{x}) \Delta \mathbf{x}
\end{aligned}
\end{equation}

providing not only the background on where this comes from, but also the so often omitted third order and higher generalizations (most often referred to as \(\cdots\)).  Poking around a bit I see that the \href{http://en.wikipedia.org/wiki/Taylor_expansion}{wikipedia Taylor Series} does explicitly define the higher order case, but if I had seen that before the connection to the Hessian was not obvious.

\makeexample{Two variable case}{example:multivectorTaylors:341}{

Rather than start with the general case, the expansion of the first few powers of \((\Ba \cdot \spacegrad) f\) for the two variable case is enough to show the pattern.  How to further generalize this scalar function case will be clear from inspection.

Starting with the first order term, writing \(\Ba = (a,b)\) we have

\begin{equation}\label{eqn:multivectorTaylors:340}
\begin{aligned}
(\Ba \cdot \spacegrad) f(x,y)
&= \left. \PD{\tau}{} f(x + a\tau, y + b\tau) \right\vert_{\tau=0} \\
&=
\left. \left( \PD{x + a\tau}{} f(x + a\tau, y + b\tau) \PD{\tau}{(x + a\tau)} \right) \right\vert_{\tau=0} \\
&\quad+\left. \left( \PD{y + b\tau}{} f(x + a\tau, y + b\tau) \PD{\tau}{(y + b\tau)} \right) \right\vert_{\tau=0} \\
&=
a \PD{x}{f} +b \PD{y}{f} \\
&=
\Ba \cdot (\spacegrad f)
\end{aligned}
\end{equation}

For the second derivative operation we have
\begin{equation}\label{eqn:multivectorTaylors:360}
\begin{aligned}
(\Ba \cdot \spacegrad)^2 f(x,y)
&=
(\Ba \cdot \spacegrad)
\left( (\Ba \cdot \spacegrad) f(x,y) \right) \\
&=
(\Ba \cdot \spacegrad) \left( a \PD{x}{f} +b \PD{y}{f} \right) \\
&= \left. \PD{\tau}{} \left( a \PD{x}{f}(x + a\tau, y + b\tau) + b \PD{y}{f}(x + a\tau, y + b\tau) \right) \right\vert_{\tau=0} \\
\end{aligned}
\end{equation}

Especially if one makes a temporary substitution of the partials for some other named variables, it is clear this follows as
before, and one gets

\begin{equation}\label{eqn:multivectorTaylors:380}
\begin{aligned}
(\Ba \cdot \spacegrad)^2 f(x,y)
&=
a^2 \PDSq{x}{f} + b a \PDD{y}{x}{f}
+a b \PDD{x}{y}{f} + b^2 \PDSq{y}{f} \\
\end{aligned}
\end{equation}

Similarly the third order derivative operator gives us

\begin{equation}\label{eqn:multivectorTaylors:400}
\begin{aligned}
(\Ba \cdot \spacegrad)^3 f(x,y)
&=
a a a \PD{x}{}\PD{x}{}\PD{x}{} f + a b a \PD{x}{}\PD{y}{}\PD{x}{}{f}  \\
\quad&+a a b \PD{x}{}\PD{y}{}\PD{x}{} f + a b b \PD{x}{}\PD{y}{}\PD{y}{}{f} \\
&\quad+b a a \PD{y}{}\PD{x}{}\PD{x}{} f + b b a \PD{y}{}\PD{y}{}\PD{x}{}{f} \\
&\quad+b a b \PD{y}{}\PD{y}{}\PD{x}{} f + b b b \PD{y}{}\PD{y}{}\PD{y}{}{f} \\
&=
a^3 \frac{\partial^3 f}{\partial x^3}
+ 3 a^2 b \PDSq{x}{}\PD{y}{f}
+ 3 a b^2 \PD{x}{}\PDSq{y}{f}
+b^3 \frac{\partial^3 f}{\partial y^3}
\end{aligned}
\end{equation}

We no longer have the notational nicety of being able to use the gradient notation as was done for the first derivative term.  For the
first and second order derivative operations, one has the
option of using the gradient and Hessian matrix notations

\begin{equation}\label{eqn:multivectorTaylors:420}
\begin{aligned}
(\Ba \cdot \spacegrad) f(x,y) &=
\transpose{\Ba}
\begin{bmatrix}
f_{x} \\
f_{y}
\end{bmatrix}
\\
(\Ba \cdot \spacegrad)^2 f(x,y)
&=
\transpose{\Ba}
\begin{bmatrix}
f_{xx} & f_{xy} \\
f_{yx} & f_{yy}
\end{bmatrix}
\Ba
\end{aligned}
\end{equation}

But this will not be helpful past the second derivative.

Additionally, if we continue to restrict oneself to the two variable case,
it is clear that we have

\begin{equation}\label{eqn:multivectorTaylors:440}
\begin{aligned}
(\Ba \cdot \spacegrad)^n f(x,y)
&=
\sum_{k=0}^{n} \binom{n}{k} a^{n-k} b^{k}
\left( \PD{x}{} \right)^{n-k}
\left( \PD{y}{} \right)^{k} f(x,y)
\end{aligned}
\end{equation}

But it is also clear that if we switch to more than two variables, a binomial
series expansion of derivative powers in this fashion will no longer work.  For
example for three (or more) variables, writing for example \(\Ba = (a_1, a_2, a_3)\),
we have

\begin{equation}\label{eqn:multivectorTaylors:460}
\begin{aligned}
(\Ba \cdot \spacegrad) f(\Bx)
&=
\sum_{i}
\left( a_i \PD{x_i}{} \right)
f(\Bx) \\
(\Ba \cdot \spacegrad)^2 f(\Bx)
&=
\sum_{ij}
\left( a_i \PD{x_i}{} \right)
\left( a_j \PD{x_j}{} \right)
f(\Bx) \\
(\Ba \cdot \spacegrad)^3 f(\Bx)
&=
\sum_{ijk}
\left( a_i \PD{x_i}{} \right)
\left( a_j \PD{x_j}{} \right)
\left( a_k \PD{x_k}{} \right)
f(\Bx)
%\\
%&\vdots
\end{aligned}
\end{equation}

If the partials are all collected into a single indexed object, one really has a tensor.  For the first and second orders we
can represent this tensor in matrix form (as the gradient and Hessian respectively)
}

\section{Gradient with non-Euclidean basis}

The directional derivative has been calculated above for a scalar function.  There is nothing intrinsic to that argument
that requires an orthonormal basis.

Suppose we have a basis \(\{\gamma_\mu\}\), and a reciprocal frame \(\{\gamma^\mu\}\).  Let

\begin{equation}\label{eqn:multivectorTaylors:480}
\begin{aligned}
x &= x^\mu \gamma_\mu = x_\mu \gamma^\mu \\
a &= a^\mu \gamma_\mu = a_\mu \gamma^\mu
\end{aligned}
\end{equation}

The first order directional derivative is then

\begin{equation}\label{eqn:multivectorTaylors:500}
\begin{aligned}
(a \cdot \grad) f(x)
&=
\left. \PD{\tau}{f}(x + \tau a) \right\vert_{\tau=0} \\
\end{aligned}
\end{equation}

This is
\begin{equation}\label{eqn:multivectorTaylors:gradDotF}
\begin{aligned}
(a \cdot \grad) f(x) &= \sum_\mu a^\mu \PD{x^\mu}{f}(x)
\end{aligned}
\end{equation}

Now, we are used to \(\grad\) as a standalone object, and want that operator defined such that we can also write \eqnref{eqn:multivectorTaylors:gradDotF}
as
\begin{equation}\label{eqn:multivectorTaylors:520}
\begin{aligned}
a \cdot (\grad f(x))
&=
\left(a^\mu \gamma_\mu \right) \cdot (\grad f(x))
\end{aligned}
\end{equation}

Comparing these we see that our partials in \eqnref{eqn:multivectorTaylors:gradDotF} do the job provided that we form the vector operator

\begin{equation}\label{eqn:multivectorTaylors:gradForNonOrtho}
\begin{aligned}
\grad &= \sum_\mu \gamma^\mu \PD{x^\mu}{}
\end{aligned}
\end{equation}

The text \citep{doran2003gap} defines \(\grad\) in this fashion, but has no logical motivation of this idea.  One sees quickly enough that this definition works, and is the required form, but building up to the construction in a way that builds on previously established ideas is still desirable.  We see here that this reciprocal frame definition of the gradient follows inevitably from the definition of the directional derivative.  Additionally this is a definition with how the directional derivative is defined in a standard Euclidean space with an orthonormal basis.

\section{Work out Gradient for a few specific multivector spaces}

The directional derivative result expressed in \eqnref{eqn:multivectorTaylors:gradDotF} holds for arbitrarily parametrized multivector spaces, and the image space can also be a generalized one.  However, the corresponding result \eqnref{eqn:multivectorTaylors:gradForNonOrtho} for the gradient itself is good only when the parameters are vectors.  These vector parameters may be non-orthonormal, and the function this is applied to does not have to be a scalar function.

If we switch to functions parametrized by multivector spaces the vector dot gradient notation also becomes misleading.  The natural generalization of the Taylor expansion for such a function, instead of \eqnref{eqn:multivectorTaylors:taylorsExponential}, or \eqnref{eqn:multivectorTaylors:taylorMulti} should instead be

\begin{equation}\label{eqn:multivectorTaylors:taylorsExponentialScalarProd}
\begin{aligned}
f(x_0 + \epsilon)
&= \left. e^{\gpgradezero{\epsilon \grad}} f(x) \right\vert_{x=x_0}
\end{aligned}
\end{equation}

Or in full,

\begin{equation}\label{eqn:multivectorTaylors:taylorMultiScalarProd}
\begin{aligned}
f(x_0 + \epsilon)
&= \sum \inv{k!} \left. {\gpgradezero{\epsilon \grad}^k} f(x) \right\vert_{x=x_0}
\end{aligned}
\end{equation}

One could alternately express this in a notationally less different form using the scalar product operator instead of grade selection, if one writes

\begin{equation}\label{eqn:multivectorTaylors:540}
\begin{aligned}
{\epsilon \stardot \grad} &\equiv \gpgradezero{\epsilon \grad}
\end{aligned}
\end{equation}

However, regardless of the notation used, the fundamental definition is still going to be the same (and the same as in the vector case), which operationally is

\begin{equation}\label{eqn:multivectorTaylors:560}
\begin{aligned}
{\epsilon \conj \grad} f(x) = \gpgradezero{\epsilon \grad} f(x)
= \left. \PD{h}{f(x + h \epsilon)} \right\vert_{h=0}
\end{aligned}
\end{equation}

\makeexample{Complex numbers}{example:multivectorTaylors:581}{
\index{complex numbers}

The simplest grade mixed multivector space is that of the complex numbers.  Let us write out the directional derivative and gradient in this space explicitly.  Writing
\begin{equation}\label{eqn:multivectorTaylors:580}
\begin{aligned}
z_0 &= u + i v \\
z &= x + i y \\
\end{aligned}
\end{equation}

So we have
\begin{equation}\label{eqn:multivectorTaylors:600}
\begin{aligned}
\gpgradezero{z_0 \grad} f(z)
&= u \PD{x}{f} + v \PD{y}{f} \\
&= u \PD{x}{f} + i v \inv{i} \PD{y}{f} \\
&= \gpgradezero{ z_0 \left( \PD{x}{} + \inv{i} \PD{y}{} \right) } f(z) \\
\end{aligned}
\end{equation}

and we can therefore identify the gradient operator as

\begin{equation}\label{eqn:multivectorTaylors:620}
\begin{aligned}
\grad_{0,2} &= \PD{x}{} + \inv{i} \PD{y}{}
\end{aligned}
\end{equation}

Observe the similarity here between the vector gradient for a 2D Euclidean space, where we can form complex numbers by (left) factoring out a unit vector, as in

\begin{equation}\label{eqn:multivectorTaylors:640}
\begin{aligned}
\Bx
&= e_1 x + e_2 y \\
&= e_1 ( x + e_1 e_2 y ) \\
&= e_1 ( x + i y ) \\
&= e_1 z
\end{aligned}
\end{equation}

It appears that we can form this complex gradient, by (right) factoring out of the same unit vector from the vector gradient

\begin{equation}\label{eqn:multivectorTaylors:660}
\begin{aligned}
e_1 \PD{x}{} + e_2 \PD{y}{}
&=
\left( \PD{x}{} + e_2 e_1 \PD{y}{} \right) e_1 \\
&=
\left( \PD{x}{} + \inv{i} \PD{y}{} \right) e_1 \\
&=
\grad_{0,2} e_1 \\
\end{aligned}
\end{equation}

So, if we write \(\spacegrad\) as the \R{2} vector gradient, with \(\Bx = e_1 x + e_2 y = e_1 z\) as above, we have

\begin{equation}\label{eqn:multivectorTaylors:680}
\begin{aligned}
\spacegrad \Bx
&= \grad_{0,2} e_1 e_1 z \\
&= \grad_{0,2} z \\
\end{aligned}
\end{equation}

This is a rather curious equivalence between 2D vectors and complex numbers.

\paragraph{Comparison of contour integral and directional derivative Taylor series}
\index{contour integral}
\index{directional derivative}

Having a complex gradient is not familiar from standard complex variable theory.  Then again, neither is a non-contour integral formulation of complex Taylor series.  The two of these ought to be equivalent, which seems to imply there is a contour integral representation of the gradient in a complex number space too (one of the Hestenes paper's mentioned this but I did not understand the notation).

Let us do an initial comparison of the two.  We need a reminder of the contour integral form of the complex derivative.  For a function \(f(z)\) and its derivatives regular in a neighborhood of a point \(z_0\), we can evaluate

\begin{equation}\label{eqn:multivectorTaylors:700}
\begin{aligned}
\ointctrclockwise \frac{f(z) dz}{(z - z_0)^k}
&=
-\inv{k-1} \ointctrclockwise {f(z) dz}\left( \inv{(z - z_0)^{k-1}} \right)' \\
&=
\inv{k-1} \ointctrclockwise {f'(z) dz}\left( \inv{(z - z_0)^{k-1}} \right) \\
&=
\inv{(k-1)(k-2)} \ointctrclockwise {f^2(z) dz}\left( \inv{(z - z_0)^{k-2}} \right) \\
&=
\inv{(k-1)(k-2)\cdots(k-n)} \ointctrclockwise {f^n(z) dz}\left( \inv{(z - z_0)^{k-n}} \right) \\
% k-n = 1
% n = k-1
&=
\inv{(k-1)(k-2)\cdots(1)} \ointctrclockwise \frac{f^{k-1}(z) dz}{z - z_0} \\
&= \frac{2 \pi i}{(k-1)!} f^{k-1}(z_0)
\end{aligned}
\end{equation}

So we have

\begin{equation}\label{eqn:multivectorTaylors:720}
\begin{aligned}
\left. \frac{d^k}{dz^k} f(z) \right\vert_{z_0}
&=
\frac{k!}{2 \pi i}\ointctrclockwise \frac{f(z) dz}{(z - z_0)^{k+1}}
\end{aligned}
\end{equation}

Given this we now have a few alternate forms of complex Taylor series

\begin{equation}\label{eqn:multivectorTaylors:740}
\begin{aligned}
f(z_0 + \epsilon)
&= \sum \inv{k!} \left. \gpgradezero{\epsilon \grad}^k f(z) \right\vert_{z=z_0} \\
&= \sum \inv{k!} \epsilon^k \left. \frac{d^k}{dz^k} f(z) \right\vert_{z_0} \\
&= \inv{2 \pi i} \sum \epsilon^k \ointctrclockwise \frac{f(z) dz}{(z - z_0)^{k+1}}
\end{aligned}
\end{equation}

Observe that the the \(0,2\) subscript for the gradient has been dropped above (ie: this is the complex gradient, not the vector
form).

\paragraph{Complex gradient compared to the derivative}

A gradient operator has been identified by factoring it out of the directional derivative.  Let us compare this to a plain old complex derivative.

\begin{equation}\label{eqn:multivectorTaylors:760}
\begin{aligned}
f'(z_0) &= \lim_{z \rightarrow z_0} \frac{ f(z) - f(z_0) }{ z - z_0}
\end{aligned}
\end{equation}

In particular, evaluating this limit for \(z = z_0 + h\), approaching \(z_0\) along the x-axis, we have

\begin{equation}\label{eqn:multivectorTaylors:780}
\begin{aligned}
f'(z_0)
&= \lim_{z \rightarrow z_0} \frac{ f(z) - f(z_0) }{ z - z_0} \\
&= \lim_{h \rightarrow 0} \frac{ f(z_0 + h) - f(z_0) }{ h } \\
&= \PD{x}{f}(z_0)
\end{aligned}
\end{equation}

Evaluating this limit for \(z = z_0 + i h\), approaching \(z_0\) along the y-axis, we have

\begin{equation}\label{eqn:multivectorTaylors:800}
\begin{aligned}
f'(z_0)
&= \lim_{h \rightarrow 0} \frac{ f(z_0 + i h) - f(z_0) }{ i h } \\
&= -i \PD{y}{f}(z_0)
\end{aligned}
\end{equation}

We have the Cauchy equations by equating these, and if the derivative exists (ie: independent of path) we require at least

\begin{equation}\label{eqn:multivectorTaylors:820}
\begin{aligned}
\PD{x}{f}(z_0) =
-i \PD{y}{f}(z_0)
\end{aligned}
\end{equation}

Or
\begin{equation}\label{eqn:multivectorTaylors:840}
\begin{aligned}
0
&=
\PD{x}{f}(z_0) + i \PD{y}{f}(z_0) \\
&=
\tilde{\grad} f(z_0)
\end{aligned}
\end{equation}

Premultiplying by \(\grad\) produces the harmonic equation

\begin{equation}\label{eqn:multivectorTaylors:860}
\begin{aligned}
\grad \tilde{\grad} f = \left( \PDSq{x}{} + \PDSq{y}{} \right) f
\end{aligned}
\end{equation}

\paragraph{First order expansion around a point}

The above, while interesting or curious,
does not provide a way to express the differential operator directly in terms of the gradient.

We can write
\begin{equation}\label{eqn:multivectorTaylors:880}
\begin{aligned}
\left. \gpgradezero{ \epsilon \grad } f(z) \right\vert_{z_0}
&=
\frac{\epsilon}{ 2 \pi i } \ointctrclockwise \frac{f(z) dz}{(z - z_0)^{2}} \\
&=
\epsilon f'(z_0)
\end{aligned}
\end{equation}

One can probably integrate this in some circumstances (perhaps when f(z) is regular along the straight path from \(z_0\) to \(z = z_0 + \epsilon\)).  If so, then we have

\begin{equation}\label{eqn:multivectorTaylors:900}
\begin{aligned}
\epsilon \int_{s=z_0}^{z} f'(s) ds &= \int_{s=z_0}^{z} \left. \gpgradezero{ \epsilon \grad } f(z) \right\vert_{z=s} ds
\end{aligned}
\end{equation}

Or
\begin{equation}\label{eqn:multivectorTaylors:920}
\begin{aligned}
f(z) &= f(z_0) + \int_{s=z_0}^{z} \left. \inv{\epsilon}\gpgradezero{ \epsilon \grad } f(z) \right\vert_{z=s} ds
\end{aligned}
\end{equation}

Is there any validity to doing this?  The idea here is to play with some circumstances where we could see where the multivector gradient may show up.  Much more play is required, some of which for discovery and the rest to do things more rigorously.
}

\makeexample{4D scalar plus bivector space}{example:multivectorTaylors:801}{

Suppose we form a scalar, bivector space by factoring out the unit time vector in a Dirac vector representation

\begin{equation}\label{eqn:multivectorTaylors:940}
\begin{aligned}
x
&= x^\mu \gamma_\mu \\
&= \left( x^0 + x^k \gamma_k \gamma_0 \right) \gamma_0 \\
&= \left( x^0 + x^k \sigma_k \right) \gamma_0 \\
&= q \gamma_0 \\
\end{aligned}
\end{equation}

This \(q\) has the structure of a quaternion-like object (scalar, plus bivector), but the bivectors all have positive square.  Our directional derivative, for multivector direction \(Q = Q^0 + Q^k \sigma_k\) is

\begin{equation}\label{eqn:multivectorTaylors:960}
\begin{aligned}
\gpgradezero{Q \grad} f(q)
&= Q^0 \PD{x^0}{f} + \sum_k Q^k \PD{x^k}{f} \\
\end{aligned}
\end{equation}

So, we can write

\begin{equation}\label{eqn:multivectorTaylors:980}
\begin{aligned}
\grad
&= \PD{x^0}{} + \sum_k \sigma_k \PD{x^k}{} \\
\end{aligned}
\end{equation}

We can do something similar for an Euclidean four vector space

\begin{equation}\label{eqn:multivectorTaylors:1000}
\begin{aligned}
x
&= x^\mu e_\mu \\
&= \left( x^0 + x^k e_k e_0 \right) e_0 \\
&= \left( x^0 + x^k i_k \right) e_0 \\
&= q e_0 \\
\end{aligned}
\end{equation}

Here each of the bivectors \(i_k\) have a negative square, much more quaternion-like (and could easily be defined in an isomorphic fashion).  This time we have

\begin{equation}\label{eqn:multivectorTaylors:1020}
\begin{aligned}
\grad
&= \PD{x^0}{} + \sum_k \inv{i_k} \PD{x^k}{} \\
\end{aligned}
\end{equation}
}

\chapter{Exterior derivative and chain rule components of the gradient}
\date{ March 31, 2008.  $RCSfile: gradientAndForms.tex,v $ Last $Revision: 1.7 $ $Date: 2009/06/11 16:45:58 $ }

\section{Gradient formulation in terms of reciprocal frames. }

We have seen how to calculate reciprocal frames as a method to find
components of a vector with respect to an arbitrary basis (doesn't have
to be orthogonal).

This can be applied to any vector:

\[
\Bx = \sum \Ba_i (\Ba^i \cdot \Bx) = \sum \Ba^i (\Ba_i \cdot \Bx)
\]

so why not the gradient operator too.

\begin{equation}
\grad = \sum \Ba_i (\Ba^i \cdot \grad) = \sum \Ba^i (\Ba_i \cdot \grad)
\end{equation}

The dot product part:

\begin{equation}
(\Ba \cdot \grad) f(\Bu) = \lim_{\tau \rightarrow 0}
\frac{ f(\Bu + \Ba \tau) - f(\Bu ) }{ \tau }
\end{equation}

we know how to calculate explicitly (from NFCM) and is the direction
derivative.

So this gives us an explicit factorization of the gradient into components
in some arbitrary set of directions, all weighted appropriately.

\section{Allowing the basis to vary according to a parametrization. }

Now, if one allows the vector basis ${\Ba_i}$ to vary along a curve, it
is interesting to observe the consequences of this to the gradient expressed
as a component along the curve and perpendicular components.

Suppose that one has a parametrization $\phi(u_1, u_2, \cdots, u_{n-1}) \in \mathbb{R}^n$, defining a generalized surface of degree one less than the space.

Provided these surface direction vectors are linearly independent and non
zero, we can writing $\Bphi_{u_i} = \frac{\partial \Bphi}{ \partial u_i}$,
and form a basis for the space by extension with a reciprocal frame vector:

\[
\{ \Bphi_{u_1}, \Bphi_{u_2}, \cdots, \Bphi_{u_{n-1}}, 
(\Bphi_{u_1} \wedge \Bphi_{u_2} \cdots \wedge \Bphi_{u_{n-1}}) \inv{\BI_n}
 \}
\]

\subsection{Try it with the simplest case. }

Let's calculate $\grad$ in this basis.  Intuition says this will
produce something like the exterior derivative from differential forms
for the component that is normal to the surface.

To make things easy, consider the absolutely simplest case, a curve
in \R{2}, with parametrization $\Br = \Bphi(t)$.  The basis associated
with this curve at some point is

\[
\{ \Ba_1, \Ba_2 \} = \{ \Bphi_t, \BI \Bphi_t \}
\]

with a reciprocal basis of:
\[
\{ \Ba^1, \Ba^2 \} = \{ \inv{\Bphi_t}, -\inv{\Bphi_t} \BI \}
\]

In terms of this components, the gradient along the curve at the specified
point is:

\begin{align*}
\grad f
&= \left(\Ba^1 \Ba_1 \cdot \grad +\Ba^2 \Ba_2 \cdot \grad \right) f \\
&= \left(\inv{\Bphi_t} \Bphi_t \cdot \grad + \left( \BI \inv{\Bphi_t} \right) \left(\BI \Bphi_t \right) \cdot \grad \right) f \\
&= \inv{\Bphi_t}\left(\Bphi_t \cdot \grad -\BI \left(\BI \cdot \Bphi_t \right) \cdot \grad \right) f \\
&= \inv{\Bphi_t}\left(\Bphi_t \cdot \grad -\BI \left(\BI \cdot \left(\Bphi_t \wedge \grad \right) \right) \right) f \\
&= \inv{\Bphi_t}\left(\Bphi_t \cdot \grad -\BI \left(\BI \left(\Bphi_t \wedge \grad \right) \right) \right) f \\
&= \inv{\Bphi_t}\left(\Bphi_t \cdot \grad + \Bphi_t \wedge \grad \right) f \\
\end{align*}

Lo and behold, we come full circle through a mass of identities back to the geometric product.
As with many things in math, knowing the answer we can be clever and start from the answer going backwards.  This would have allowed the standard factorization of the gradient vector into 
orthogonal components in the usual fashion:

\begin{equation}\label{eqn:gradAndForms:grad_dot_wedge}
\grad 
= \inv{\Bphi_t}\left( \Bphi_t \grad \right)
= \inv{\Bphi_t}\left( \Bphi_t \cdot \grad + \Bphi_t \wedge \grad \right)
\end{equation}

Let's continue writing $\Bphi(t) = x(t) \Be_1 + y(t) \Be_2$.  Then

\[
\Bphi' = x' \Be_1 + y' \Be_2
\]
\[
\Bphi' \cdot \grad = 
\frac{dx}{dt} \frac{\partial}{\partial x}
+\frac{dy}{dt} \frac{\partial}{\partial y}
\]

\[
\Bphi' \wedge \grad = 
\BI
\left(\frac{dx}{dt} \frac{\partial}{\partial y} -\frac{dy}{dt} \frac{\partial}{\partial x}\right)
\]

Combining these and inserting back into \ref{eqn:gradAndForms:grad_dot_wedge} we have

\begin{align*}
\grad 
&= \frac{\Bphi'}{\left(\Bphi'\right)^2}\left( \Bphi' \cdot \grad + \Bphi' \wedge \grad \right) \\
&=
\inv{(x')^2 + (y')^2}
(x',y')\left(
\frac{dx}{dt} \frac{\partial}{\partial x}
+\frac{dy}{dt} \frac{\partial}{\partial y}
\right)
+
(-y',x')
\left(\frac{dx}{dt} \frac{\partial}{\partial y} -\frac{dy}{dt} \frac{\partial}{\partial x}\right)
\end{align*}

Now, here it's worth pointing out that the choice of the parametrization can break some of the assumptions made.  In particular
the curve can be completely continuous, but the parametrization could allow it to be zero for some interval since $(x(t), y(t))$
can be picked to be constant for a ``time'' before continuing.

This problem is eliminated by picking an arc length parametrization.  Provided the curve isn't degenerate (ie: a point), then
we have at least one of $dx/ds \ne 0$, or $dy/ds \ne 0$.  Additionally, by parametrization using arc length we have
$(dx/ds)^2 + (dy/ds)^2 = (ds/ds)^2 = 1$.  This eliminates the denominator leaving the following decomposition of the \R{2} gradient

\begin{equation}
\grad = 
\underbrace{(x',y')}_{\text{Unit tangent vector}}
\underbrace{
\left(
\frac{dx}{ds} \frac{\partial}{\partial x}
+\frac{dy}{ds} \frac{\partial}{\partial y}
\right)
}_{\text{ Chain Rule in operator form. }}
+ 
\underbrace{(-y',x')}_{\text{Unit normal vector}}
\underbrace{
\left(
\frac{dx}{ds} \frac{\partial}{\partial y}
-\frac{dy}{ds} \frac{\partial}{\partial x}
\right)
}_{\text{ Exterior derivative operator. }}
\end{equation}

Thus, loosely speaking we have the chain rule as the scalar component of the unit tangent vector along a parametrized curve,
and we have the exterior derivative as the component of the gradient that lies colinear to a unit normal to the curve (believe this
is the unit normal that points inwards to the curvature if there is any).

\subsection{Extension to higher dimensional curves. }

In \R{3} or above we can perform the same calculation.  The result is similar and straightforward to derive:

\begin{equation}
\nabla
=
\underbrace{
\left(\frac{dx_1}{ds}, \frac{dx_2}{ds}, \cdots, \frac{dx_n}{ds}\right)
 }_{\text{Unit tangent.} }
\sum_{i=1}^n \frac{dx_i}{ds}\frac{\partial}{\partial x_i}
+ \sum_{1 \le i < j \le n} 
\underbrace{
\left(
\mathbf{e}_j \frac{dx_i}{ds}
-\mathbf{e}_i \frac{dx_j}{ds}
\right)
}_{\text{ Normal to unit tangent. }}
\left(
\frac{dx_i}{ds}\frac{\partial}{\partial x_j}
-\frac{dx_j}{ds}\frac{\partial}{\partial x_i}
\right)
\end{equation}

Here we have a set of normals to the unit tangent.  For \R{3}, we have $ij=\{12,13,23\}$.  One of these unit normals must
be linearly dependent on the other two (or zero).  The exterior scalar factors here loose some of their resemblance to the
exterior derivative here.  Perhaps a parametrized (hyper-)surface is required to get the familiar form for \R{3} or above.

\documentclass{article}

\usepackage{amsmath}
\usepackage{mathpazo}

%
% shorthand for bold symbols, convenient for vectors and matrices
%
\newcommand{\Ba}[0]{\mathbf{a}}
\newcommand{\Bb}[0]{\mathbf{b}}
\newcommand{\Bc}[0]{\mathbf{c}}
\newcommand{\Bd}[0]{\mathbf{d}}
\newcommand{\Be}[0]{\mathbf{e}}
\newcommand{\Bf}[0]{\mathbf{f}}
\newcommand{\Bg}[0]{\mathbf{g}}
\newcommand{\Bh}[0]{\mathbf{h}}
\newcommand{\Bi}[0]{\mathbf{i}}
\newcommand{\Bj}[0]{\mathbf{j}}
\newcommand{\Bk}[0]{\mathbf{k}}
\newcommand{\Bl}[0]{\mathbf{l}}
\newcommand{\Bm}[0]{\mathbf{m}}
\newcommand{\Bn}[0]{\mathbf{n}}
\newcommand{\Bo}[0]{\mathbf{o}}
\newcommand{\Bp}[0]{\mathbf{p}}
\newcommand{\Bq}[0]{\mathbf{q}}
\newcommand{\Br}[0]{\mathbf{r}}
\newcommand{\Bs}[0]{\mathbf{s}}
\newcommand{\Bt}[0]{\mathbf{t}}
\newcommand{\Bu}[0]{\mathbf{u}}
\newcommand{\Bv}[0]{\mathbf{v}}
\newcommand{\Bw}[0]{\mathbf{w}}
\newcommand{\Bx}[0]{\mathbf{x}}
\newcommand{\By}[0]{\mathbf{y}}
\newcommand{\Bz}[0]{\mathbf{z}}
\newcommand{\BA}[0]{\mathbf{A}}
\newcommand{\BB}[0]{\mathbf{B}}
\newcommand{\BC}[0]{\mathbf{C}}
\newcommand{\BD}[0]{\mathbf{D}}
\newcommand{\BE}[0]{\mathbf{E}}
\newcommand{\BF}[0]{\mathbf{F}}
\newcommand{\BG}[0]{\mathbf{G}}
\newcommand{\BH}[0]{\mathbf{H}}
\newcommand{\BI}[0]{\mathbf{I}}
\newcommand{\BJ}[0]{\mathbf{J}}
\newcommand{\BK}[0]{\mathbf{K}}
\newcommand{\BL}[0]{\mathbf{L}}
\newcommand{\BM}[0]{\mathbf{M}}
\newcommand{\BN}[0]{\mathbf{N}}
\newcommand{\BO}[0]{\mathbf{O}}
\newcommand{\BP}[0]{\mathbf{P}}
\newcommand{\BQ}[0]{\mathbf{Q}}
\newcommand{\BR}[0]{\mathbf{R}}
\newcommand{\BS}[0]{\mathbf{S}}
\newcommand{\BT}[0]{\mathbf{T}}
\newcommand{\BU}[0]{\mathbf{U}}
\newcommand{\BV}[0]{\mathbf{V}}
\newcommand{\BW}[0]{\mathbf{W}}
\newcommand{\BX}[0]{\mathbf{X}}
\newcommand{\BY}[0]{\mathbf{Y}}
\newcommand{\BZ}[0]{\mathbf{Z}}

\newcommand{\Bzero}[0]{\mathbf{0}}
\newcommand{\Btheta}[0]{\boldsymbol{\theta}}
\newcommand{\Btau}[0]{\boldsymbol{\tau}}
\newcommand{\Bomega}[0]{\boldsymbol{\omega}}

%
% shorthand for unit vectors
%
\newcommand{\acap}[0]{\hat{\Ba}}
\newcommand{\bcap}[0]{\hat{\Bb}}
\newcommand{\ccap}[0]{\hat{\Bc}}
\newcommand{\dcap}[0]{\hat{\Bd}}
\newcommand{\ecap}[0]{\hat{\Be}}
\newcommand{\fcap}[0]{\hat{\Bf}}
\newcommand{\gcap}[0]{\hat{\Bg}}
\newcommand{\hcap}[0]{\hat{\Bh}}
\newcommand{\icap}[0]{\hat{\Bi}}
\newcommand{\jcap}[0]{\hat{\Bj}}
\newcommand{\kcap}[0]{\hat{\Bk}}
\newcommand{\lcap}[0]{\hat{\Bl}}
\newcommand{\mcap}[0]{\hat{\Bm}}
\newcommand{\ncap}[0]{\hat{\Bn}}
\newcommand{\ocap}[0]{\hat{\Bo}}
\newcommand{\pcap}[0]{\hat{\Bp}}
\newcommand{\qcap}[0]{\hat{\Bq}}
\newcommand{\rcap}[0]{\hat{\Br}}
\newcommand{\scap}[0]{\hat{\Bs}}
\newcommand{\tcap}[0]{\hat{\Bt}}
\newcommand{\ucap}[0]{\hat{\Bu}}
\newcommand{\vcap}[0]{\hat{\Bv}}
\newcommand{\wcap}[0]{\hat{\Bw}}
\newcommand{\xcap}[0]{\hat{\Bx}}
\newcommand{\ycap}[0]{\hat{\By}}
\newcommand{\zcap}[0]{\hat{\Bz}}
\newcommand{\thetacap}[0]{\hat{\Btheta}}

%
% to write R^n and C^n in a distinguishable fashion.  Perhaps change this
% to the double lined characters upon figuring out how to do so.
%
\newcommand{\C}[1]{$\mathbb{C}^{#1}$}
\newcommand{\R}[1]{$\mathbb{R}^{#1}$}

%
% various generally useful helpers
%

% derivative of #1 wrt. #2:
\newcommand{\D}[2] {\frac {d#2} {d#1}}

\newcommand{\inv}[1]{\frac{1}{#1}}
\newcommand{\cross}[0]{\times}

\newcommand{\abs}[1]{\lvert{#1}\rvert}
\newcommand{\norm}[1]{\lVert{#1}\rVert}
\newcommand{\innerprod}[2]{\langle{#1}, {#2}\rangle}
\newcommand{\dotprod}[2]{{#1} \cdot {#2}}
\newcommand{\bdotprod}[2]{\left({#1} \cdot {#2}\right)}
\newcommand{\crossprod}[2]{{#1} \cross {#2}}
\newcommand{\tripleprod}[3]{\dotprod{\left(\crossprod{#1}{#2}\right)}{#3}}

\DeclareMathOperator{\Proj}{Proj}
\DeclareMathOperator{\Span}{span}
\DeclareMathOperator{\Sgn}{sgn}
\DeclareMathOperator{\Area}{Area}
\DeclareMathOperator{\Volume}{Volume}

%
% A few miscellaneous things specific to this document
%
\newcommand{\crossop}[1]{\crossprod{#1}{}}

% R2 vector.
\newcommand{\VectorTwo}[2]{
\begin{bmatrix}
 {#1} \\
 {#2}
\end{bmatrix}
}

\newcommand{\VectorN}[1]{
\begin{bmatrix}
{#1}_1 \\
{#1}_2 \\
\vdots \\
{#1}_N \\
\end{bmatrix}
}

\newcommand{\DETuvij}[4]{
\begin{vmatrix}
 {#1}_{#3} & {#1}_{#4} \\
 {#2}_{#3} & {#2}_{#4}
\end{vmatrix}
}

\newcommand{\DETuvwijk}[6]{
\begin{vmatrix}
 {#1}_{#4} & {#1}_{#5} & {#1}_{#6} \\
 {#2}_{#4} & {#2}_{#5} & {#2}_{#6} \\
 {#3}_{#4} & {#3}_{#5} & {#3}_{#6}
\end{vmatrix}
}

\newcommand{\DETuvwxijkl}[8]{
\begin{vmatrix}
 {#1}_{#5} & {#1}_{#6} & {#1}_{#7} & {#1}_{#8} \\
 {#2}_{#5} & {#2}_{#6} & {#2}_{#7} & {#2}_{#8} \\
 {#3}_{#5} & {#3}_{#6} & {#3}_{#7} & {#3}_{#8} \\
 {#4}_{#5} & {#4}_{#6} & {#4}_{#7} & {#4}_{#8} \\
\end{vmatrix}
}

%\newcommand{\DETuvwxyijklm}[10]{
%\begin{vmatrix}
% {#1}_{#6} & {#1}_{#7} & {#1}_{#8} & {#1}_{#9} & {#1}_{#10} \\
% {#2}_{#6} & {#2}_{#7} & {#2}_{#8} & {#2}_{#9} & {#2}_{#10} \\
% {#3}_{#6} & {#3}_{#7} & {#3}_{#8} & {#3}_{#9} & {#3}_{#10} \\
% {#4}_{#6} & {#4}_{#7} & {#4}_{#8} & {#4}_{#9} & {#4}_{#10} \\
% {#5}_{#6} & {#5}_{#7} & {#5}_{#8} & {#5}_{#9} & {#5}_{#10}
%\end{vmatrix}
%}

% R3 vector.
\newcommand{\VectorThree}[3]{
\begin{bmatrix}
 {#1} \\
 {#2} \\
 {#3}
\end{bmatrix}
}


\newcommand{\grad}[0]{\nabla}
\newcommand{\PD}[2]{\frac{\partial {#2}}{\partial {#1}}}
\newcommand{\Abs}[1]{\left\lvert{#1}\right\rvert}
\newcommand{\gpgrade}[2] {{\left\langle{{#1}}\right\rangle}_{#2}}

\usepackage{color,cite,graphicx}
   % use colour in the document, put your citations as [1-4]
   % rather than [1,2,3,4] (it looks nicer, and the extended LaTeX2e
   % graphics package. 
\usepackage{latexsym,amssymb,epsf} % don't remember if these are
   % needed, but their inclusion can't do any damage



\usepackage[bookmarks=true]{hyperref}

\title{ Reconciling vector integral relations. }
\author{Peeter Joot}
\date{ Sept. 18, 2008.  Last Revision: $Date: 2008/09/20 01:09:31 $ }

\begin{document}

\maketitle{}

\tableofcontents

\section{ A hodge podge of relations. }

%I was never satisfied with how vector integral relationships were presented
%to me in my Calculus classes.  Some of these were presented as equations to
%memorize instead of with proof.  

The aim of these notes is to work through proofs of the following 
integral equations

\begin{itemize}

\item Gradient line integral. 

\begin{equation}\label{eqn:lineintegral}
\int_C (\grad f) \cdot d\Br = f \vert_{\partial C}
\end{equation}

\item Jacobian area determinants. 

Change of variables for a double integral

\begin{equation}
dA = dx dy =
\begin{vmatrix}
\PD{u}{x} & \PD{u}{y} \\
\PD{v}{x} & \PD{v}{y} \\
\end{vmatrix}
du dv
= \Abs{ \PD{(u,v)}{(x,y)} } du dv
\end{equation}

In Salus and Hille this is proved using Green's theorem, despite it 
being seeming like the more basic operation.  The greater than two
dimensional cases are not proved at all.

\item Green's theorem. 

\begin{equation}\label{eqn:greens}
\int\int \left(\PD{y}{Q} - \PD{x}{P}\right) dx dy = \oint P dx + Q dy
\end{equation}

\item Divergence theorem. 

\begin{equation}\label{eqn:divergenceplane}
\int\int \grad \cdot \Bv\, dx dy = \oint \Bv \cdot \ncap\, ds
\end{equation}

\begin{equation}\label{eqn:divergencevolume}
\int\int\int_V \grad \cdot \Bv\, dx dy dz = \int\int_S \Bv \cdot \ncap\, dA
\end{equation}

\begin{equation}\label{eqn:divergencegrad}
\int\int\int_V \grad \phi\, dV \int\int_S \ncap \phi\, dA
\end{equation}

\begin{equation}\label{eqn:divergencegradcross}
\int\int\int_V \grad \cross \Bv\, dV \int\int_S \Bv \cross \ncap\, dA
\end{equation}

\item Stokes theorem. 

\begin{equation}\label{eqn:stokes}
\int\int (\grad \cross \Bv) \cdot \ncap\, dx dy = \oint \Bv \cdot d\Br
\end{equation}

\end{itemize}

In particular I'd like to relate these to the geometrical concepts
of Clifford algebra now that I know how to work with that in a 
differential and algebraic fashion for many sorts of problems.  I am hoping
that working through proofs of these basic identities 
will be enough that I can go on to the more general approaches in 
differential forms and the geometric calculus of Hestenes.

John Denker's 
\href{ http://www.av8n.com/physics/straight-wire.pdf }{ article on the
magnetic field of a straight wire }
gives a simple looking high level description of vector form of Stokes'
theorem in it's Clifford formulation

\begin{equation}\label{eqn:stokesGA}
\int_S \grad \wedge F = \int_{\partial S} F
\end{equation}

This is simple enough looking, but there are some important details left
out.  In particular the grades do not match, so there must be some sort of
implied projection or dot product operations too.

I'd say this suffers from some of the things that I had trouble with in
attempting to study differential forms.

The basic ideas of how to formulate
the curve, surface, volume, ... of integration is not specified.  How to do
that in greater than three dimensions is not trivial seeming to me since
none of the traditional methods of dotting with a normal will not work.

Knowing now about how subspaces can be expressed using blades is likely the
key.  The Clifford algebra ideas seem particularly suited to this as many
of these ideas can be formulated independent of the calculus applications.
One can learn the geometric and algebraic concepts first and then move on
to the Calculus.

\section{ Gradient line integral. }

This is the easiest of the identities to prove.  Introduction of a reciprocal frame $\gamma^{\mu} \cdot \gamma_{\nu} = {\delta^{\mu}}_{\nu}$
also means that we can do in full generalitity with a possibly
non-orthonormal basis of any dimension, and an arbitrary metric.

Write the gradient as normal

\begin{equation*}
\grad = \sum \gamma^{\mu} \PD{x^{\mu}}{} = \gamma^{\mu} \partial_{\mu}
\end{equation*}

Here summation convention with implied sum over mixed upper and lower indexes is employed.

Express the position vector along the curve as
a parameterized path $\Br = \Br(\lambda) = \gamma_{\mu} x^{\mu}$, and use
this to form the element of vector length along the path
%This can be used to form the line integral element that we also need
%the differential element of vector length.  

\begin{equation*}
d\Br = \gamma_{\mu} \frac{d x^{\mu}}{d\lambda} d\lambda
\end{equation*}

Dotting the gradient and the path element we have
\begin{align*}
\grad f \cdot d\Br 
&= \left(\gamma^{\mu} \partial_{\mu} f\right) \cdot \left(\gamma_{\nu} \frac{d x^{\nu}}{d\lambda} \right) d\lambda \\
&= {\delta^{\mu}}_{\nu} \PD{x^{\mu}}{f} \frac{d x^{\nu}}{d\lambda} d\lambda \\
&= \sum \PD{x^{\mu}}{f} \frac{d x^{\mu}}{d\lambda} d\lambda \\
&= \frac{d f}{d \lambda} d\lambda
\end{align*}

Equation \ref{eqn:lineintegral} follows immediately, which we see to be really not much more than the chain rule.

Additionally this can be put into correspondance with equation \ref{eqn:stokesGA}, with the observation that one can write the gradient of a scalar function as a wedge product by the fundamental definition of wedge in terms of grade selection.  For blades $A$ and $B$ with grades $a$ and $b$ respectively, the wedge is

\begin{align*}
A \wedge B = \gpgrade{AB}{a+b}
\end{align*}

Therefore for a scalar function $f$

\begin{align*}
\grad \wedge f = \gpgrade{\grad f}{1+0} = \grad f
\end{align*}

Putting this back together one has the desired result

\begin{equation}\label{eqn:lineintegralwedge}
\int_C (\grad \wedge f) \cdot d\Br = f \vert_{\partial C}
\end{equation}

\subsection{ Motivating the non-orthonormal form of the gradient. }

An additional note about the derivation of this line integral result.  Having done this with the gradient expressed for possibly non-orthonormal frames, 
shows that if played backwards, it provides a nice motivation for the general form of the gradient, in terms
of a such a non-orthonormal basis.  That's a lot more obvious a way to get at this result than my previous way of observing that the Euler-Lagrange
equations when summed in vector form imply that this is the required form of the gradient.

\section{ Jacobian area determinants. }

Next in ease of proof is the Jacobian determinant.  This actually comes largely for free since we can utilize the wedge product to
express areas.

\begin{figure}[htp]
\centering
\includegraphics[totalheight=0.4\textheight]{planeParameterization}
\caption{Plane parameterization}\label{fig:planeParameterization}
\end{figure}

Introduce a two vector parameterization of the area as in figure \ref{fig:planeParameterization}

\begin{equation*}
\Br = \gamma^{i} \phi_i(u,v)
\end{equation*}

Provided that the partials are not colinear at the point of interest, we can compute the area of the parallogram spanned by these

\begin{align*}
d\BA 
&= \left(\PD{u}{\Br} du\right) \wedge \left(\PD{v}{\Br} dv\right) \\
&= \left(\gamma^{i} \PD{u}{\phi_i}\right) \wedge \left(\gamma^{j} \PD{v}{\phi_j} \right) du dv \\
&= \gamma^{i} \wedge \gamma^{j} \PD{u}{\phi_i} \PD{v}{\phi_j} du dv \\
&= \sum_{i<j} \gamma^{i} \wedge \gamma^{j} \left( \PD{u}{\phi_i} \PD{v}{\phi_j} - \PD{u}{\phi_i} \PD{v}{\phi_i} \right) du dv \\
&= \sum_{i<j} \gamma^{i} \wedge \gamma^{j} \PD{(u,v)}{(\phi_i,\phi_j)} du dv \\
\end{align*}

Here $d\BA$ is a bivector area element, so in the purely two dimensional case, where this is constrained to a plane, the scalar area element
is recovered by dividing by the plane unit pseudoscalar having the same orientation as this bivector.

One can also see how the same idea will be of use later in the Stokes' generalization of Green's theorem (considering a surface element small enough to be considered planar).

For now, considering just the 2D case we have, to divide through by the plane unit pseudoscalar $i = \Be_1\Be_2$ produced by the product of two orthonormal vectors we want to calculate the product:

\begin{align*}
\inv{i} \gamma^{1} \wedge \gamma^{2} 
&= (\Be_2 \wedge \Be_1) \cdot (\gamma^{1} \wedge \gamma^{2}) \\
&= \Be_2 \cdot (\Be_1 \cdot (\gamma^{1} \wedge \gamma^{2})) \\
&= \Be_2 \cdot ( (\Be_1 \cdot \gamma^{1}) \gamma^{2} -(\Be_1 \cdot \gamma^{2}) \gamma^{1} ) \\
&= (\Be_1 \cdot \gamma^{1}) (\Be_2 \cdot \gamma^{2}) -(\Be_1 \cdot \gamma^{2}) (\Be_2 \cdot \gamma^{1}) \\
&=
\begin{vmatrix}
\Be_1 \cdot \gamma^{1} & \Be_1 \cdot \gamma^{2} \\
\Be_2 \cdot \gamma^{1} & \Be_2 \cdot \gamma^{2}
\end{vmatrix}
\end{align*}

Thus the (scalar) area element is

\begin{align}\label{eqn:jacobianframe}
dA =
\begin{vmatrix}
\Be_1 \cdot \gamma^{1} & \Be_1 \cdot \gamma^{2} \\
\Be_2 \cdot \gamma^{1} & \Be_2 \cdot \gamma^{2}
\end{vmatrix}
\PD{(u,v)}{(\phi_1,\phi_2)} du dv
\end{align}

This is a slightly more general form than we are used to seeing since the position vector parameterization was allowed to be expresed in terms
of an arbitrary (possibly non-orthonormal) basis.  Also observe that the coefficients in the determinant preceding the Jacobian are exactly those of the matrix of the linear transformation between the two sets of basis vectors.

\subsection{ Orthonormal parameterization. } 

For the special (and usual) case of an orthonormal parameterization

\begin{align*}
\Br = x(u,v) \Be_1 + y(u,v) \Be_2
\end{align*}

the product of determinants in \ref{eqn:jacobianframe} takes the usual form

\begin{align}\label{eqn:jacobianarea}
dx dy = \PD{(u,v)}{(x,y)} du dv.
\end{align}

Now the danger of an expression like \ref{eqn:jacobianarea} is that the differential notation for the determinant makes it seem almost
obvious.  Now, if you understand the wedge product origin you can state
that obviousness after a little bit of algebra.  However, in a book like Salus and Hille (used for Calculus I-III in UofT Engineering) they
can't even derive this two dimensional case til close to the end of the book, since they required Green's theorem to do so.  I'd say that in that case it is not really so obvious.  The geometrical background just isn't there.

Note that there are degrees of freedom to alter the sign given an arbitrary pseudoscalar.  This illustrates why the absolute value of the Jacobian determinant is used in some circumstances.  Less dodgy is to say the positive area element after change of variables in a specific region is produced by dividing out the pseudoscalar with the same orientation as the area element bivector.

It is also not too hard to see that this idea will also work for change of variables for volume and higher
dimensional volume elements, after wedging N partials.  We just have to divide by the spatial (or higher dimensional) pseudoscalar of the
same orientation associated with the parameterization.

\subsection{ Surface area in higher dimensions. }

As well as being able to use these ideas to express scalar area and volume, or higher dimensional generalizations, this can be used to calculate surface area in any number of dimensions.  For a two parameter vector parameterization of a surface $\Br(u,v)$ we can write

\begin{equation}\label{eqn:surfacearea}
A = \int\int \inv{I_2(u,v)} \left(\PD{u}{\Br} \wedge \PD{v}{\Br}\right) du dv
\end{equation}

Here $I_2(u,v)$ is the unit pseudoscalar for the tangent space of the surface at the point of interest with the orientation of the bivector 
$d\BA = \PD{u}{\Br} \wedge \PD{v}{\Br}$.

This is in fact equivalent to the familiar normal form in 3D expressed in terms
of a cross product

\begin{equation}
A = \int\int \left(\PD{u}{\Br} \cross \PD{v}{\Br}\right) \cdot \ncap(u,v) du dv
\end{equation}

but the expression of equation \ref{eqn:surfacearea}, holds for any number of dimensions $N \ge 2$.  As with the wedge product form, we have a requirement that the parameterization is not degenerate at any point, so the 3D de-generalization of our requirement that
$\PD{u}{\Br} \wedge \PD{v}{\Br} \ne 0$
on the region of the surface of interest means that for 3D we simply require
$\PD{u}{\Br} \cross \PD{v}{\Br} \ne 0$.

A consequence of non-degeneracy for the region of the surface area being integrated means that the sign of the bivector cannot change sign, so we have equivalance with the concept of outwards normal to the surface by picking the tangent space unit pseudoscalar to have the same orientation as the bivector area element.

\section{ Green's theorem. }

\subsection{ Attempt to arrive at a more natural vector form for Green's theorem. }

It is pretty clear glancing at equation \ref{eqn:greens}, that the left
hand side can likely be expressed as the curl of a vector.  By curl here
is meant the more natural bivector "curl", where we form the operator $\grad \wedge$.

Writing $\Bv = P \Be_1 + Q \Be_2$, we have

\begin{equation*}
\PD{y}{Q} - \PD{x}{P} = -(\grad \wedge \Bv) \inv{i}
\end{equation*}

Similarily we've seen that we can express the area element in vector form.  Writing $\Br = x \Be_1 + y \Be_2$, we have

\begin{equation*}
dx dy = \inv{i}\left( \PD{u}{\Br} \wedge \PD{v}{\Br} \right) du dv
\end{equation*}

Re-assembling these, we can form the left hand side of the Green's theorem
equation completely in vector form

\begin{equation*}
\int\int -(\grad \wedge \Bv) \inv{i} \inv{i}\left( \PD{x}{\Br} \wedge \PD{y}{\Br} \right)
= \int\int (\grad \wedge \Bv) \left( \PD{u}{\Br} \wedge \PD{v}{\Br} \right) du dv
\end{equation*}

Considering the total differential of the position vector, it makes sense to introduce vector differential elements to
express this

\begin{equation*}
d\Br = \PD{u}{\Br} du + \PD{v}{\Br} dv = d\Bx + d\By
\end{equation*}

We can then rewrite the LHS once more in a slightly cleaner form, independent of the specific parameterization

\begin{equation}\label{eqn:greenLHS}
\int\int (\grad \wedge \Bv) \cdot (d\Bx \wedge d\By) = \int\int (\grad \wedge \Bv) \cdot d\BA
\end{equation}

Here we see that it becomes natural to work with the oriented bivector area element $d\BA = d\Bx \wedge d\By$.

Having arrived at what is likely the most natural vector form \ref{eqn:greenLHS} for the LHS of Green's theorem, lets
now attempt to integrate this in it's most general form, dropping references to the original $x$, and $y$ coordinates.
If this is the correct form, we should end up with a vector line integral around a path after doing so, and thus prove
Green's theorm.

\subsection{ Calculating the area integral. }

\section{ Divergence theorem. }

\section{ Stokes theorem. }

\end{document}               % End of document.

\documentclass{article}

\usepackage{amsmath}
\usepackage{mathpazo}

%
% shorthand for bold symbols, convenient for vectors and matrices
%
\newcommand{\Ba}[0]{\mathbf{a}}
\newcommand{\Bb}[0]{\mathbf{b}}
\newcommand{\Bc}[0]{\mathbf{c}}
\newcommand{\Bd}[0]{\mathbf{d}}
\newcommand{\Be}[0]{\mathbf{e}}
\newcommand{\Bf}[0]{\mathbf{f}}
\newcommand{\Bg}[0]{\mathbf{g}}
\newcommand{\Bh}[0]{\mathbf{h}}
\newcommand{\Bi}[0]{\mathbf{i}}
\newcommand{\Bj}[0]{\mathbf{j}}
\newcommand{\Bk}[0]{\mathbf{k}}
\newcommand{\Bl}[0]{\mathbf{l}}
\newcommand{\Bm}[0]{\mathbf{m}}
\newcommand{\Bn}[0]{\mathbf{n}}
\newcommand{\Bo}[0]{\mathbf{o}}
\newcommand{\Bp}[0]{\mathbf{p}}
\newcommand{\Bq}[0]{\mathbf{q}}
\newcommand{\Br}[0]{\mathbf{r}}
\newcommand{\Bs}[0]{\mathbf{s}}
\newcommand{\Bt}[0]{\mathbf{t}}
\newcommand{\Bu}[0]{\mathbf{u}}
\newcommand{\Bv}[0]{\mathbf{v}}
\newcommand{\Bw}[0]{\mathbf{w}}
\newcommand{\Bx}[0]{\mathbf{x}}
\newcommand{\By}[0]{\mathbf{y}}
\newcommand{\Bz}[0]{\mathbf{z}}
\newcommand{\BA}[0]{\mathbf{A}}
\newcommand{\BB}[0]{\mathbf{B}}
\newcommand{\BC}[0]{\mathbf{C}}
\newcommand{\BD}[0]{\mathbf{D}}
\newcommand{\BE}[0]{\mathbf{E}}
\newcommand{\BF}[0]{\mathbf{F}}
\newcommand{\BG}[0]{\mathbf{G}}
\newcommand{\BH}[0]{\mathbf{H}}
\newcommand{\BI}[0]{\mathbf{I}}
\newcommand{\BJ}[0]{\mathbf{J}}
\newcommand{\BK}[0]{\mathbf{K}}
\newcommand{\BL}[0]{\mathbf{L}}
\newcommand{\BM}[0]{\mathbf{M}}
\newcommand{\BN}[0]{\mathbf{N}}
\newcommand{\BO}[0]{\mathbf{O}}
\newcommand{\BP}[0]{\mathbf{P}}
\newcommand{\BQ}[0]{\mathbf{Q}}
\newcommand{\BR}[0]{\mathbf{R}}
\newcommand{\BS}[0]{\mathbf{S}}
\newcommand{\BT}[0]{\mathbf{T}}
\newcommand{\BU}[0]{\mathbf{U}}
\newcommand{\BV}[0]{\mathbf{V}}
\newcommand{\BW}[0]{\mathbf{W}}
\newcommand{\BX}[0]{\mathbf{X}}
\newcommand{\BY}[0]{\mathbf{Y}}
\newcommand{\BZ}[0]{\mathbf{Z}}

\newcommand{\Bzero}[0]{\mathbf{0}}
\newcommand{\Btheta}[0]{\boldsymbol{\theta}}
\newcommand{\Btau}[0]{\boldsymbol{\tau}}
\newcommand{\Bomega}[0]{\boldsymbol{\omega}}

%
% shorthand for unit vectors
%
\newcommand{\acap}[0]{\hat{\Ba}}
\newcommand{\bcap}[0]{\hat{\Bb}}
\newcommand{\ccap}[0]{\hat{\Bc}}
\newcommand{\dcap}[0]{\hat{\Bd}}
\newcommand{\ecap}[0]{\hat{\Be}}
\newcommand{\fcap}[0]{\hat{\Bf}}
\newcommand{\gcap}[0]{\hat{\Bg}}
\newcommand{\hcap}[0]{\hat{\Bh}}
\newcommand{\icap}[0]{\hat{\Bi}}
\newcommand{\jcap}[0]{\hat{\Bj}}
\newcommand{\kcap}[0]{\hat{\Bk}}
\newcommand{\lcap}[0]{\hat{\Bl}}
\newcommand{\mcap}[0]{\hat{\Bm}}
\newcommand{\ncap}[0]{\hat{\Bn}}
\newcommand{\ocap}[0]{\hat{\Bo}}
\newcommand{\pcap}[0]{\hat{\Bp}}
\newcommand{\qcap}[0]{\hat{\Bq}}
\newcommand{\rcap}[0]{\hat{\Br}}
\newcommand{\scap}[0]{\hat{\Bs}}
\newcommand{\tcap}[0]{\hat{\Bt}}
\newcommand{\ucap}[0]{\hat{\Bu}}
\newcommand{\vcap}[0]{\hat{\Bv}}
\newcommand{\wcap}[0]{\hat{\Bw}}
\newcommand{\xcap}[0]{\hat{\Bx}}
\newcommand{\ycap}[0]{\hat{\By}}
\newcommand{\zcap}[0]{\hat{\Bz}}
\newcommand{\thetacap}[0]{\hat{\Btheta}}

%
% to write R^n and C^n in a distinguishable fashion.  Perhaps change this
% to the double lined characters upon figuring out how to do so.
%
\newcommand{\C}[1]{$\mathbb{C}^{#1}$}
\newcommand{\R}[1]{$\mathbb{R}^{#1}$}

%
% various generally useful helpers
%

% derivative of #1 wrt. #2:
\newcommand{\D}[2] {\frac {d#2} {d#1}}

\newcommand{\inv}[1]{\frac{1}{#1}}
\newcommand{\cross}[0]{\times}

\newcommand{\abs}[1]{\lvert{#1}\rvert}
\newcommand{\norm}[1]{\lVert{#1}\rVert}
\newcommand{\innerprod}[2]{\langle{#1}, {#2}\rangle}
\newcommand{\dotprod}[2]{{#1} \cdot {#2}}
\newcommand{\bdotprod}[2]{\left({#1} \cdot {#2}\right)}
\newcommand{\crossprod}[2]{{#1} \cross {#2}}
\newcommand{\tripleprod}[3]{\dotprod{\left(\crossprod{#1}{#2}\right)}{#3}}

\DeclareMathOperator{\Proj}{Proj}
\DeclareMathOperator{\Span}{span}
\DeclareMathOperator{\Sgn}{sgn}
\DeclareMathOperator{\Area}{Area}
\DeclareMathOperator{\Volume}{Volume}

%
% A few miscellaneous things specific to this document
%
\newcommand{\crossop}[1]{\crossprod{#1}{}}

% R2 vector.
\newcommand{\VectorTwo}[2]{
\begin{bmatrix}
 {#1} \\
 {#2}
\end{bmatrix}
}

\newcommand{\VectorN}[1]{
\begin{bmatrix}
{#1}_1 \\
{#1}_2 \\
\vdots \\
{#1}_N \\
\end{bmatrix}
}

\newcommand{\DETuvij}[4]{
\begin{vmatrix}
 {#1}_{#3} & {#1}_{#4} \\
 {#2}_{#3} & {#2}_{#4}
\end{vmatrix}
}

\newcommand{\DETuvwijk}[6]{
\begin{vmatrix}
 {#1}_{#4} & {#1}_{#5} & {#1}_{#6} \\
 {#2}_{#4} & {#2}_{#5} & {#2}_{#6} \\
 {#3}_{#4} & {#3}_{#5} & {#3}_{#6}
\end{vmatrix}
}

\newcommand{\DETuvwxijkl}[8]{
\begin{vmatrix}
 {#1}_{#5} & {#1}_{#6} & {#1}_{#7} & {#1}_{#8} \\
 {#2}_{#5} & {#2}_{#6} & {#2}_{#7} & {#2}_{#8} \\
 {#3}_{#5} & {#3}_{#6} & {#3}_{#7} & {#3}_{#8} \\
 {#4}_{#5} & {#4}_{#6} & {#4}_{#7} & {#4}_{#8} \\
\end{vmatrix}
}

%\newcommand{\DETuvwxyijklm}[10]{
%\begin{vmatrix}
% {#1}_{#6} & {#1}_{#7} & {#1}_{#8} & {#1}_{#9} & {#1}_{#10} \\
% {#2}_{#6} & {#2}_{#7} & {#2}_{#8} & {#2}_{#9} & {#2}_{#10} \\
% {#3}_{#6} & {#3}_{#7} & {#3}_{#8} & {#3}_{#9} & {#3}_{#10} \\
% {#4}_{#6} & {#4}_{#7} & {#4}_{#8} & {#4}_{#9} & {#4}_{#10} \\
% {#5}_{#6} & {#5}_{#7} & {#5}_{#8} & {#5}_{#9} & {#5}_{#10}
%\end{vmatrix}
%}

% R3 vector.
\newcommand{\VectorThree}[3]{
\begin{bmatrix}
 {#1} \\
 {#2} \\
 {#3}
\end{bmatrix}
}


\newcommand{\grad}[0]{\nabla}
\newcommand{\spacegrad}[0]{\boldsymbol{\nabla}}
\newcommand{\gpgrade}[2] {{\left\langle{{#1}}\right\rangle}_{#2}}
\newcommand{\gpgradezero}[1] {\gpgrade{#1}{0}}
\newcommand{\gpgradeone}[1] {\gpgrade{#1}{1}}
\newcommand{\gpgradetwo}[1] {\gpgrade{#1}{2}}
\newcommand{\gpgradethree}[1] {\gpgrade{#1}{3}}

\usepackage{color,cite,graphicx}
   % use colour in the document, put your citations as [1-4]
   % rather than [1,2,3,4] (it looks nicer, and the extended LaTeX2e
   % graphics package. 
\usepackage{latexsym,amssymb,epsf} % don't remember if these are
   % needed, but their inclusion can't do any damage


% ointclockwise, ointctrclockwise
\usepackage{txfonts}

\usepackage[bookmarks=true]{hyperref}

\title{ Revisit Stokes derivation. }
\author{Peeter Joot}
\date{ Sept 27, 2008.  Last Revision: $Date: 2008/09/27 19:43:26 $ }

\begin{document}

\maketitle{}

\tableofcontents

\section{ Algebraic description of oriented boundaries. }

Having used pictorial methods to enumerate the bounding loop and area elements 
\href{http://www.geocities.com/peeter_joot/geometric_algebra/vector_integral_relations.pdf}{
in the previous derivation} of the vector and bivector forms of Stokes's, makes the application
of these formulas harder.  Here this will be revisited, with the aim of remedying this, as well as
obtaining a proof for the general case, which was not possible because of a lack of exactly this
algebraic formulation.

\subsection{ Parallelogram parameterization. }

\begin{figure}[htp]
\centering
\includegraphics[totalheight=0.4\textheight]{parallelogram_parameterized}
\caption{Parallelogram}\label{fig:parallelogram}
\end{figure}

\subsection{ Parallelopiped parameterization. }

\begin{figure}[htp]
\centering
\includegraphics[totalheight=0.4\textheight]{parallelopiped_parameterized}
\caption{Parallelopiped}\label{fig:parallelopiped}
\end{figure}

\end{document}

\documentclass{article}

\usepackage{amsmath}
\usepackage{mathpazo}

%
% shorthand for bold symbols, convenient for vectors and matrices
%
\newcommand{\Ba}[0]{\mathbf{a}}
\newcommand{\Bb}[0]{\mathbf{b}}
\newcommand{\Bc}[0]{\mathbf{c}}
\newcommand{\Bd}[0]{\mathbf{d}}
\newcommand{\Be}[0]{\mathbf{e}}
\newcommand{\Bf}[0]{\mathbf{f}}
\newcommand{\Bg}[0]{\mathbf{g}}
\newcommand{\Bh}[0]{\mathbf{h}}
\newcommand{\Bi}[0]{\mathbf{i}}
\newcommand{\Bj}[0]{\mathbf{j}}
\newcommand{\Bk}[0]{\mathbf{k}}
\newcommand{\Bl}[0]{\mathbf{l}}
\newcommand{\Bm}[0]{\mathbf{m}}
\newcommand{\Bn}[0]{\mathbf{n}}
\newcommand{\Bo}[0]{\mathbf{o}}
\newcommand{\Bp}[0]{\mathbf{p}}
\newcommand{\Bq}[0]{\mathbf{q}}
\newcommand{\Br}[0]{\mathbf{r}}
\newcommand{\Bs}[0]{\mathbf{s}}
\newcommand{\Bt}[0]{\mathbf{t}}
\newcommand{\Bu}[0]{\mathbf{u}}
\newcommand{\Bv}[0]{\mathbf{v}}
\newcommand{\Bw}[0]{\mathbf{w}}
\newcommand{\Bx}[0]{\mathbf{x}}
\newcommand{\By}[0]{\mathbf{y}}
\newcommand{\Bz}[0]{\mathbf{z}}
\newcommand{\BA}[0]{\mathbf{A}}
\newcommand{\BB}[0]{\mathbf{B}}
\newcommand{\BC}[0]{\mathbf{C}}
\newcommand{\BD}[0]{\mathbf{D}}
\newcommand{\BE}[0]{\mathbf{E}}
\newcommand{\BF}[0]{\mathbf{F}}
\newcommand{\BG}[0]{\mathbf{G}}
\newcommand{\BH}[0]{\mathbf{H}}
\newcommand{\BI}[0]{\mathbf{I}}
\newcommand{\BJ}[0]{\mathbf{J}}
\newcommand{\BK}[0]{\mathbf{K}}
\newcommand{\BL}[0]{\mathbf{L}}
\newcommand{\BM}[0]{\mathbf{M}}
\newcommand{\BN}[0]{\mathbf{N}}
\newcommand{\BO}[0]{\mathbf{O}}
\newcommand{\BP}[0]{\mathbf{P}}
\newcommand{\BQ}[0]{\mathbf{Q}}
\newcommand{\BR}[0]{\mathbf{R}}
\newcommand{\BS}[0]{\mathbf{S}}
\newcommand{\BT}[0]{\mathbf{T}}
\newcommand{\BU}[0]{\mathbf{U}}
\newcommand{\BV}[0]{\mathbf{V}}
\newcommand{\BW}[0]{\mathbf{W}}
\newcommand{\BX}[0]{\mathbf{X}}
\newcommand{\BY}[0]{\mathbf{Y}}
\newcommand{\BZ}[0]{\mathbf{Z}}

\newcommand{\Bzero}[0]{\mathbf{0}}
\newcommand{\Btheta}[0]{\boldsymbol{\theta}}
\newcommand{\Btau}[0]{\boldsymbol{\tau}}
\newcommand{\Bomega}[0]{\boldsymbol{\omega}}

%
% shorthand for unit vectors
%
\newcommand{\acap}[0]{\hat{\Ba}}
\newcommand{\bcap}[0]{\hat{\Bb}}
\newcommand{\ccap}[0]{\hat{\Bc}}
\newcommand{\dcap}[0]{\hat{\Bd}}
\newcommand{\ecap}[0]{\hat{\Be}}
\newcommand{\fcap}[0]{\hat{\Bf}}
\newcommand{\gcap}[0]{\hat{\Bg}}
\newcommand{\hcap}[0]{\hat{\Bh}}
\newcommand{\icap}[0]{\hat{\Bi}}
\newcommand{\jcap}[0]{\hat{\Bj}}
\newcommand{\kcap}[0]{\hat{\Bk}}
\newcommand{\lcap}[0]{\hat{\Bl}}
\newcommand{\mcap}[0]{\hat{\Bm}}
\newcommand{\ncap}[0]{\hat{\Bn}}
\newcommand{\ocap}[0]{\hat{\Bo}}
\newcommand{\pcap}[0]{\hat{\Bp}}
\newcommand{\qcap}[0]{\hat{\Bq}}
\newcommand{\rcap}[0]{\hat{\Br}}
\newcommand{\scap}[0]{\hat{\Bs}}
\newcommand{\tcap}[0]{\hat{\Bt}}
\newcommand{\ucap}[0]{\hat{\Bu}}
\newcommand{\vcap}[0]{\hat{\Bv}}
\newcommand{\wcap}[0]{\hat{\Bw}}
\newcommand{\xcap}[0]{\hat{\Bx}}
\newcommand{\ycap}[0]{\hat{\By}}
\newcommand{\zcap}[0]{\hat{\Bz}}
\newcommand{\thetacap}[0]{\hat{\Btheta}}

%
% to write R^n and C^n in a distinguishable fashion.  Perhaps change this
% to the double lined characters upon figuring out how to do so.
%
\newcommand{\C}[1]{$\mathbb{C}^{#1}$}
\newcommand{\R}[1]{$\mathbb{R}^{#1}$}

%
% various generally useful helpers
%

% derivative of #1 wrt. #2:
\newcommand{\D}[2] {\frac {d#2} {d#1}}

\newcommand{\inv}[1]{\frac{1}{#1}}
\newcommand{\cross}[0]{\times}

\newcommand{\abs}[1]{\lvert{#1}\rvert}
\newcommand{\norm}[1]{\lVert{#1}\rVert}
\newcommand{\innerprod}[2]{\langle{#1}, {#2}\rangle}
\newcommand{\dotprod}[2]{{#1} \cdot {#2}}
\newcommand{\bdotprod}[2]{\left({#1} \cdot {#2}\right)}
\newcommand{\crossprod}[2]{{#1} \cross {#2}}
\newcommand{\tripleprod}[3]{\dotprod{\left(\crossprod{#1}{#2}\right)}{#3}}

\DeclareMathOperator{\Proj}{Proj}
\DeclareMathOperator{\Span}{span}
\DeclareMathOperator{\Sgn}{sgn}
\DeclareMathOperator{\Area}{Area}
\DeclareMathOperator{\Volume}{Volume}

%
% A few miscellaneous things specific to this document
%
\newcommand{\crossop}[1]{\crossprod{#1}{}}

% R2 vector.
\newcommand{\VectorTwo}[2]{
\begin{bmatrix}
 {#1} \\
 {#2}
\end{bmatrix}
}

\newcommand{\VectorN}[1]{
\begin{bmatrix}
{#1}_1 \\
{#1}_2 \\
\vdots \\
{#1}_N \\
\end{bmatrix}
}

\newcommand{\DETuvij}[4]{
\begin{vmatrix}
 {#1}_{#3} & {#1}_{#4} \\
 {#2}_{#3} & {#2}_{#4}
\end{vmatrix}
}

\newcommand{\DETuvwijk}[6]{
\begin{vmatrix}
 {#1}_{#4} & {#1}_{#5} & {#1}_{#6} \\
 {#2}_{#4} & {#2}_{#5} & {#2}_{#6} \\
 {#3}_{#4} & {#3}_{#5} & {#3}_{#6}
\end{vmatrix}
}

\newcommand{\DETuvwxijkl}[8]{
\begin{vmatrix}
 {#1}_{#5} & {#1}_{#6} & {#1}_{#7} & {#1}_{#8} \\
 {#2}_{#5} & {#2}_{#6} & {#2}_{#7} & {#2}_{#8} \\
 {#3}_{#5} & {#3}_{#6} & {#3}_{#7} & {#3}_{#8} \\
 {#4}_{#5} & {#4}_{#6} & {#4}_{#7} & {#4}_{#8} \\
\end{vmatrix}
}

%\newcommand{\DETuvwxyijklm}[10]{
%\begin{vmatrix}
% {#1}_{#6} & {#1}_{#7} & {#1}_{#8} & {#1}_{#9} & {#1}_{#10} \\
% {#2}_{#6} & {#2}_{#7} & {#2}_{#8} & {#2}_{#9} & {#2}_{#10} \\
% {#3}_{#6} & {#3}_{#7} & {#3}_{#8} & {#3}_{#9} & {#3}_{#10} \\
% {#4}_{#6} & {#4}_{#7} & {#4}_{#8} & {#4}_{#9} & {#4}_{#10} \\
% {#5}_{#6} & {#5}_{#7} & {#5}_{#8} & {#5}_{#9} & {#5}_{#10}
%\end{vmatrix}
%}

% R3 vector.
\newcommand{\VectorThree}[3]{
\begin{bmatrix}
 {#1} \\
 {#2} \\
 {#3}
\end{bmatrix}
}


%<misc>
%
\newcommand{\Abs}[1]{{\left\lvert{#1}\right\rvert}}
\newcommand{\spacegrad}[0]{\boldsymbol{\nabla}}
\newcommand{\grad}[0]{\nabla}
\newcommand{\LL}[0]{\mathcal{L}}

% == \partial_{#1} {#2}
\newcommand{\PD}[2]{\frac{\partial {#2}}{\partial {#1}}}
% inline variant
\newcommand{\PDi}[2]{{\partial {#2}}/{\partial {#1}}}

\newcommand{\PDD}[3]{\frac{\partial^2 {#3}}{\partial {#1}\partial {#2}}}
%\newcommand{\PDd}[2]{\frac{\partial^2 {#2}}{{\partial{#1}}^2}}
\newcommand{\PDsq}[2]{\frac{\partial^2 {#2}}{(\partial {#1})^2}}

\newcommand{\Partial}[2]{\frac{\partial {#1}}{\partial {#2}}}
\DeclareMathOperator{\RejName}{Rej}
\newcommand{\Rej}[2]{\RejName_{#1}\left( {#2} \right)}
\newcommand{\Rm}[1]{\mathbb{R}^{#1}}
\newcommand{\Cm}[1]{\mathbb{C}^{#1}}
\newcommand{\conj}[0]{{*}}

%</misc>

% <grade selection>
%
\newcommand{\gpgrade}[2] {{\left\langle{{#1}}\right\rangle}_{#2}}

\newcommand{\gpgradezero}[1] {\gpgrade{#1}{}}
%\newcommand{\gpscalargrade}[1] {{\left\langle{{#1}}\right\rangle}}
%\newcommand{\gpgradezero}[1] {\gpgrade{#1}{0}}

%\newcommand{\gpgradeone}[1] {{\left\langle{{#1}}\right\rangle}_{1}}
\newcommand{\gpgradeone}[1] {\gpgrade{#1}{1}}

\newcommand{\gpgradetwo}[1] {\gpgrade{#1}{2}}
\newcommand{\gpgradethree}[1] {\gpgrade{#1}{3}}
\newcommand{\gpgradefour}[1] {\gpgrade{#1}{4}}
%
% </grade selection>



\newcommand{\adot}[0]{{\dot{a}}}
\newcommand{\bdot}[0]{{\dot{b}}}
% taken for centered dot:
%\newcommand{\cdot}[0]{{\dot{c}}}
%\newcommand{\ddot}[0]{{\dot{d}}}
\newcommand{\edot}[0]{{\dot{e}}}
\newcommand{\fdot}[0]{{\dot{f}}}
\newcommand{\gdot}[0]{{\dot{g}}}
\newcommand{\hdot}[0]{{\dot{h}}}
\newcommand{\idot}[0]{{\dot{i}}}
\newcommand{\jdot}[0]{{\dot{j}}}
\newcommand{\kdot}[0]{{\dot{k}}}
\newcommand{\ldot}[0]{{\dot{l}}}
\newcommand{\mdot}[0]{{\dot{m}}}
\newcommand{\ndot}[0]{{\dot{n}}}
%\newcommand{\odot}[0]{{\dot{o}}}
\newcommand{\pdot}[0]{{\dot{p}}}
\newcommand{\qdot}[0]{{\dot{q}}}
\newcommand{\rdot}[0]{{\dot{r}}}
\newcommand{\sdot}[0]{{\dot{s}}}
\newcommand{\tdot}[0]{{\dot{t}}}
\newcommand{\udot}[0]{{\dot{u}}}
\newcommand{\vdot}[0]{{\dot{v}}}
\newcommand{\wdot}[0]{{\dot{w}}}
\newcommand{\xdot}[0]{{\dot{x}}}
\newcommand{\ydot}[0]{{\dot{y}}}
\newcommand{\zdot}[0]{{\dot{z}}}
\newcommand{\addot}[0]{{\ddot{a}}}
\newcommand{\bddot}[0]{{\ddot{b}}}
\newcommand{\cddot}[0]{{\ddot{c}}}
%\newcommand{\dddot}[0]{{\ddot{d}}}
\newcommand{\eddot}[0]{{\ddot{e}}}
\newcommand{\fddot}[0]{{\ddot{f}}}
\newcommand{\gddot}[0]{{\ddot{g}}}
\newcommand{\hddot}[0]{{\ddot{h}}}
\newcommand{\iddot}[0]{{\ddot{i}}}
\newcommand{\jddot}[0]{{\ddot{j}}}
\newcommand{\kddot}[0]{{\ddot{k}}}
\newcommand{\lddot}[0]{{\ddot{l}}}
\newcommand{\mddot}[0]{{\ddot{m}}}
\newcommand{\nddot}[0]{{\ddot{n}}}
\newcommand{\oddot}[0]{{\ddot{o}}}
\newcommand{\pddot}[0]{{\ddot{p}}}
\newcommand{\qddot}[0]{{\ddot{q}}}
\newcommand{\rddot}[0]{{\ddot{r}}}
\newcommand{\sddot}[0]{{\ddot{s}}}
\newcommand{\tddot}[0]{{\ddot{t}}}
\newcommand{\uddot}[0]{{\ddot{u}}}
\newcommand{\vddot}[0]{{\ddot{v}}}
\newcommand{\wddot}[0]{{\ddot{w}}}
\newcommand{\xddot}[0]{{\ddot{x}}}
\newcommand{\yddot}[0]{{\ddot{y}}}
\newcommand{\zddot}[0]{{\ddot{z}}}

%<bold and dot greek symbols>
%

\newcommand{\Deltadot}[0]{{\dot{\Delta}}}
\newcommand{\Gammadot}[0]{{\dot{\Gamma}}}
\newcommand{\Lambdadot}[0]{{\dot{\Lambda}}}
\newcommand{\Omegadot}[0]{{\dot{\Omega}}}
\newcommand{\Phidot}[0]{{\dot{\Phi}}}
\newcommand{\Pidot}[0]{{\dot{\Pi}}}
\newcommand{\Psidot}[0]{{\dot{\Psi}}}
\newcommand{\Sigmadot}[0]{{\dot{\Sigma}}}
\newcommand{\Thetadot}[0]{{\dot{\Theta}}}
\newcommand{\Upsilondot}[0]{{\dot{\Upsilon}}}
\newcommand{\Xidot}[0]{{\dot{\Xi}}}
\newcommand{\alphadot}[0]{{\dot{\alpha}}}
\newcommand{\betadot}[0]{{\dot{\beta}}}
\newcommand{\chidot}[0]{{\dot{\chi}}}
\newcommand{\deltadot}[0]{{\dot{\delta}}}
\newcommand{\epsilondot}[0]{{\dot{\epsilon}}}
\newcommand{\etadot}[0]{{\dot{\eta}}}
\newcommand{\gammadot}[0]{{\dot{\gamma}}}
\newcommand{\kappadot}[0]{{\dot{\kappa}}}
\newcommand{\lambdadot}[0]{{\dot{\lambda}}}
\newcommand{\mudot}[0]{{\dot{\mu}}}
\newcommand{\nudot}[0]{{\dot{\nu}}}
\newcommand{\omegadot}[0]{{\dot{\omega}}}
\newcommand{\phidot}[0]{{\dot{\phi}}}
\newcommand{\pidot}[0]{{\dot{\pi}}}
\newcommand{\psidot}[0]{{\dot{\psi}}}
\newcommand{\rhodot}[0]{{\dot{\rho}}}
\newcommand{\sigmadot}[0]{{\dot{\sigma}}}
\newcommand{\taudot}[0]{{\dot{\tau}}}
\newcommand{\thetadot}[0]{{\dot{\theta}}}
\newcommand{\upsilondot}[0]{{\dot{\upsilon}}}
\newcommand{\varepsilondot}[0]{{\dot{\varepsilon}}}
\newcommand{\varphidot}[0]{{\dot{\varphi}}}
\newcommand{\varpidot}[0]{{\dot{\varpi}}}
\newcommand{\varrhodot}[0]{{\dot{\varrho}}}
\newcommand{\varsigmadot}[0]{{\dot{\varsigma}}}
\newcommand{\varthetadot}[0]{{\dot{\vartheta}}}
\newcommand{\xidot}[0]{{\dot{\xi}}}
\newcommand{\zetadot}[0]{{\dot{\zeta}}}

\newcommand{\Deltaddot}[0]{{\ddot{\Delta}}}
\newcommand{\Gammaddot}[0]{{\ddot{\Gamma}}}
\newcommand{\Lambdaddot}[0]{{\ddot{\Lambda}}}
\newcommand{\Omegaddot}[0]{{\ddot{\Omega}}}
\newcommand{\Phiddot}[0]{{\ddot{\Phi}}}
\newcommand{\Piddot}[0]{{\ddot{\Pi}}}
\newcommand{\Psiddot}[0]{{\ddot{\Psi}}}
\newcommand{\Sigmaddot}[0]{{\ddot{\Sigma}}}
\newcommand{\Thetaddot}[0]{{\ddot{\Theta}}}
\newcommand{\Upsilonddot}[0]{{\ddot{\Upsilon}}}
\newcommand{\Xiddot}[0]{{\ddot{\Xi}}}
\newcommand{\alphaddot}[0]{{\ddot{\alpha}}}
\newcommand{\betaddot}[0]{{\ddot{\beta}}}
\newcommand{\chiddot}[0]{{\ddot{\chi}}}
\newcommand{\deltaddot}[0]{{\ddot{\delta}}}
\newcommand{\epsilonddot}[0]{{\ddot{\epsilon}}}
\newcommand{\etaddot}[0]{{\ddot{\eta}}}
\newcommand{\gammaddot}[0]{{\ddot{\gamma}}}
\newcommand{\kappaddot}[0]{{\ddot{\kappa}}}
\newcommand{\lambdaddot}[0]{{\ddot{\lambda}}}
\newcommand{\muddot}[0]{{\ddot{\mu}}}
\newcommand{\nuddot}[0]{{\ddot{\nu}}}
\newcommand{\omegaddot}[0]{{\ddot{\omega}}}
\newcommand{\phiddot}[0]{{\ddot{\phi}}}
\newcommand{\piddot}[0]{{\ddot{\pi}}}
\newcommand{\psiddot}[0]{{\ddot{\psi}}}
\newcommand{\rhoddot}[0]{{\ddot{\rho}}}
\newcommand{\sigmaddot}[0]{{\ddot{\sigma}}}
\newcommand{\tauddot}[0]{{\ddot{\tau}}}
\newcommand{\thetaddot}[0]{{\ddot{\theta}}}
\newcommand{\upsilonddot}[0]{{\ddot{\upsilon}}}
\newcommand{\varepsilonddot}[0]{{\ddot{\varepsilon}}}
\newcommand{\varphiddot}[0]{{\ddot{\varphi}}}
\newcommand{\varpiddot}[0]{{\ddot{\varpi}}}
\newcommand{\varrhoddot}[0]{{\ddot{\varrho}}}
\newcommand{\varsigmaddot}[0]{{\ddot{\varsigma}}}
\newcommand{\varthetaddot}[0]{{\ddot{\vartheta}}}
\newcommand{\xiddot}[0]{{\ddot{\xi}}}
\newcommand{\zetaddot}[0]{{\ddot{\zeta}}}

\newcommand{\BDelta}[0]{\boldsymbol{\Delta}}
\newcommand{\BGamma}[0]{\boldsymbol{\Gamma}}
\newcommand{\BLambda}[0]{\boldsymbol{\Lambda}}
\newcommand{\BOmega}[0]{\boldsymbol{\Omega}}
\newcommand{\BPhi}[0]{\boldsymbol{\Phi}}
\newcommand{\BPi}[0]{\boldsymbol{\Pi}}
\newcommand{\BPsi}[0]{\boldsymbol{\Psi}}
\newcommand{\BSigma}[0]{\boldsymbol{\Sigma}}
\newcommand{\BTheta}[0]{\boldsymbol{\Theta}}
\newcommand{\BUpsilon}[0]{\boldsymbol{\Upsilon}}
\newcommand{\BXi}[0]{\boldsymbol{\Xi}}
\newcommand{\Balpha}[0]{\boldsymbol{\alpha}}
\newcommand{\Bbeta}[0]{\boldsymbol{\beta}}
\newcommand{\Bchi}[0]{\boldsymbol{\chi}}
\newcommand{\Bdelta}[0]{\boldsymbol{\delta}}
\newcommand{\Bepsilon}[0]{\boldsymbol{\epsilon}}
\newcommand{\Beta}[0]{\boldsymbol{\eta}}
\newcommand{\Bgamma}[0]{\boldsymbol{\gamma}}
\newcommand{\Bkappa}[0]{\boldsymbol{\kappa}}
\newcommand{\Blambda}[0]{\boldsymbol{\lambda}}
\newcommand{\Bmu}[0]{\boldsymbol{\mu}}
\newcommand{\Bnu}[0]{\boldsymbol{\nu}}
%\newcommand{\Bomega}[0]{\boldsymbol{\omega}}
\newcommand{\Bphi}[0]{\boldsymbol{\phi}}
\newcommand{\Bpi}[0]{\boldsymbol{\pi}}
\newcommand{\Bpsi}[0]{\boldsymbol{\psi}}
\newcommand{\Brho}[0]{\boldsymbol{\rho}}
\newcommand{\Bsigma}[0]{\boldsymbol{\sigma}}
%\newcommand{\Btau}[0]{\boldsymbol{\tau}}
%\newcommand{\Btheta}[0]{\boldsymbol{\theta}}
\newcommand{\Bupsilon}[0]{\boldsymbol{\upsilon}}
\newcommand{\Bvarepsilon}[0]{\boldsymbol{\varepsilon}}
\newcommand{\Bvarphi}[0]{\boldsymbol{\varphi}}
\newcommand{\Bvarpi}[0]{\boldsymbol{\varpi}}
\newcommand{\Bvarrho}[0]{\boldsymbol{\varrho}}
\newcommand{\Bvarsigma}[0]{\boldsymbol{\varsigma}}
\newcommand{\Bvartheta}[0]{\boldsymbol{\vartheta}}
\newcommand{\Bxi}[0]{\boldsymbol{\xi}}
\newcommand{\Bzeta}[0]{\boldsymbol{\zeta}}
%
%</bold and dot greek symbols>
%<infrequent>
%
%\newcommand{\AreaOp}[1]{\AName_{#1}}
%\newcommand{\Babs}[0]{\abs{\BB}}
%\newcommand{\Bcap}[0]{\hat{\BB}}
%\newcommand{\BrPrimeRej}[0]{\rcap(\rcap \wedge \Br')}
%\newcommand{\CA}[0]{\mathcal{A}}
%\newcommand{\Cos}[1]{\cos{\left({#1}\right)}}
%\newcommand{\Det}[1] {\abs{#1}}
%\newcommand{\Dsq}[2] {\frac {\partial^2 {#1}} {\partial {#2}^2}}
%\newcommand{\Exp}[1]{\exp{\left({#1}\right)}}
%\newcommand{\Norm}[1]{\left\lVert{#1}\right\rVert}
%\newcommand{\Sin}[1]{\sin{\left({#1}\right)}}
%\newcommand{\T}[0]{\text{T}}
%\newcommand{\VolumeOp}[1]{\VName_{#1}}
%\newcommand{\agrad}[0]{\Ba \cdot \nabla}
%\newcommand{\alphacap}[0]{\hat{\boldsymbol{\alpha}}}
%\newcommand{\Fcap}[0]{\hat{\BF}}
%\newcommand{\bithree}[0]{{\Bi}_3}
%\newcommand{\bxa}[0]{\Bx\Ba}
%\newcommand{\coordvec}[2]{
%\newcommand{\costheta}[0]{\acap \cdot \xcap}
%\newcommand{\ddt}[1]{\ddot{#1}}
%\newcommand{\ddu}[1] {\frac {d{#1}} {du}}
%\newcommand{\dsqxj}[2] {\frac {\partial^2 {#1}} {\partial {x_{#2}}^2}}
%\newcommand{\dtheta}[1]{\frac{d {#1}}{d \theta}}
%\newcommand{\dt}[1]{\dot{#1}}
%\newcommand{\dt}[1]{\frac{d {#1}}{dt}}
%\newcommand{\dxj}[2] {\frac {\partial {#1}} {\partial {x_{#2}}}}
%\newcommand{\halfPhi}[0]{\frac{\phi}{2}}
%\newcommand{\half}[0]{\inv{2}}
%\newcommand{\inv}[1]{\frac{1}{#1}}
%\newcommand{\laplacian}[0]{\nabla^2}
%\newcommand{\matrixoftx}[3]{
%\newcommand{\nrrp}[0]{\norm{\rcap \wedge \Br'}}
%\newcommand{\oiint}{\bigcirc \hspace{-1.4em} \int \hspace{-.8em} \int}
%\newcommand{\transpose}[1]{{#1}^{\text{T}}}
%\newcommand{\transpose}[1]{{{#1}^{\TextTranspose}}}
%\newcommand{\transpose}[1]{{{#1}^{\text{T}}}}
%\newcommand{\barA}[0]{\bar{A}}
%\newcommand{\qbar}[0]{\bar{q}}
%\newcommand{\qdotbar}[0]{\dot{\bar{q}}}
%
%</infrequent>





\usepackage[bookmarks=true]{hyperref}

\usepackage{color,cite,graphicx}
   % use colour in the document, put your citations as [1-4]
   % rather than [1,2,3,4] (it looks nicer, and the extended LaTeX2e
   % graphics package. 
\usepackage{latexsym,amssymb,epsf} % don't remember if these are
   % needed, but their inclusion can't do any damage


\title{ n-sphere volume. }
\author{Peeter Joot \quad peeter.joot@gmail.com }
\date{ Feb 26, 2009.  Last Revision: $Date: 2009/02/26 23:27:03 $ }

\begin{document}

\maketitle{}
\tableofcontents

\section{ Motivation. }

In \cite{PJ4dFourier} a 4D fourier transform solution 
of Maxwell's equation yielded a Green's function of the form

\begin{align*}
G(x) = \iiiint \frac{e^{i k_\mu x^\mu}}{k_\nu k^\nu} dk_1 dk_2 dk_3 dk_4
\end{align*}

To attempt to ``evaluate'' this integral, as done in
\cite{PJpoisson}
to produce the retarded time potentials,
a hypervolume equivalent to spherical polar coordinate
parameterization is probably desirable.

Before attempting to tackle the problem of interest, the basic question
of how to do volume and weighted volume integrals over a hyperspherical volumes
must be considered.

\section{ Some hints from wikipedia. }

%\begin{figure}[htp]
%\centering
%\includegraphics[totalheight=0.4\textheight]{picturepath}
%\caption{My Caption}\label{fig:pictlabel}
%\end{figure}
%
%... see figure \ref{fig:picturepath} ...

\bibliographystyle{plainnat}
\bibliography{myrefs}

\end{document}

%
% Copyright � 2012 Peeter Joot.  All Rights Reserved.
% Licenced as described in the file LICENSE under the root directory of this GIT repository.
%

%
%
\chapter{Vector Differential Identities}
\index{vector identities}
\label{chap:vectorDifferentialIdentities}
%\date{Jan 05, 2009.  vectorDifferentialIdentities.tex}

\citep{feynman1963flp} electrodynamics \textchapref{II} lists a number of
differential vector identities.

\begin{enumerate}
\item \(\spacegrad \cdot (\spacegrad T) = \spacegrad^2 T = \mbox{a scalar field}\)
\item \(\spacegrad \cross (\spacegrad T) = 0\)
\item \(\spacegrad (\spacegrad \cdot \Bh) = \mbox{a vector field}\)
\item \(\spacegrad \cdot (\spacegrad \cross \Bh) = 0\)
\item \(\spacegrad \cross (\spacegrad \cross \Bh) = \spacegrad(\spacegrad \cdot \Bh) - \spacegrad^2 \Bh\)
\item \((\spacegrad \cdot \spacegrad) \Bh = \mbox{a vector field}\)
\end{enumerate}

Let us see how all these translate to GA form.

\section{Divergence of a gradient}
\index{gradient!divergence}

This one has the same form, but expanding it can be evaluated by grade
selection

\begin{equation}\label{eqn:vectorDifferentialIdentities:20}
\begin{aligned}
\spacegrad \cdot (\spacegrad T)
&= \gpgradezero{\spacegrad \spacegrad T} \\
&= (\spacegrad^2) T
\end{aligned}
\end{equation}

A less sneaky expansion would be by coordinates

\begin{equation}\label{eqn:vectorDifferentialIdentities:40}
\begin{aligned}
\spacegrad \cdot (\spacegrad T)
&= {\sum_{k,j} (\sigma_k \partial_k) \cdot (\sigma_j \partial_j T)} \\
&= \gpgradezero{\sum_{k,j} (\sigma_k \partial_k) (\sigma_j \partial_j T)} \\
&= \gpgradezero{\left(\sum_{k,j} \sigma_k \partial_k \sigma_j \partial_j\right) T} \\
%&= {\left(\sum_{k,j} \sigma_k \partial_k \sigma_j \partial_j\right) } T \\
&= \gpgradezero{\spacegrad^2 T} \\
&= \spacegrad^2 T \\
\end{aligned}
\end{equation}

\section{Curl of a gradient is zero}
\index{gradient!curl}

The duality analogue of this is
\begin{equation}\label{eqn:vectorDifferentialIdentities:60}
\begin{aligned}
\spacegrad \cross (\spacegrad T) = -i(\spacegrad \wedge (\spacegrad T))
\end{aligned}
\end{equation}

Let us verify that this bivector curl is zero.  This can also be done by grade selection

\begin{equation}\label{eqn:vectorDifferentialIdentities:80}
\begin{aligned}
\spacegrad \wedge (\spacegrad T)
&= \gpgradetwo{\spacegrad (\spacegrad T)} \\
&= \gpgradetwo{(\spacegrad \spacegrad) T} \\
&= (\spacegrad \wedge \spacegrad) T \\
&= 0
\end{aligned}
\end{equation}

Again, this is sneaky and side steps the continuity requirement for mixed partial equality.  Again by coordinates is better
\begin{equation}\label{eqn:vectorDifferentialIdentities:100}
\begin{aligned}
\spacegrad \wedge (\spacegrad T)
&= \gpgradetwo{\sum_{k,j}\sigma_k \partial_k (\sigma_j \partial_j T)} \\
&= \gpgradetwo{\sum_{k<j}\sigma_k \sigma_j (\partial_k \partial_j - \partial_j \partial_k) T} \\
&= \sum_{k<j} \sigma_k \wedge \sigma_j (\partial_k \partial_j - \partial_j \partial_k) T \\
\end{aligned}
\end{equation}

So provided the mixed partials are zero the curl of a gradient is zero.

\section{Gradient of a divergence}
\index{divergence!gradient}

Nothing more to say about this one.

\section{Divergence of curl}
\index{curl!divergence}

This one looks like it will have a dual form using bivectors.

\begin{equation}\label{eqn:vectorDifferentialIdentities:120}
\begin{aligned}
\spacegrad \cdot (\spacegrad \cross \Bh)
&= \spacegrad \cdot (-i (\spacegrad \wedge \Bh)) \\
&= \gpgradezero{\spacegrad (-i (\spacegrad \wedge \Bh))} \\
&= \gpgradezero{-i \spacegrad (\spacegrad \wedge \Bh)} \\
&= -(i \spacegrad) \cdot (\spacegrad \wedge \Bh) \\
\end{aligned}
\end{equation}

Is this any better than the cross product relationship?

I do not really think so.  They both say the same thing, and only possible value to this duality form is if more than three dimensions are required (in which case the sign of the pseudoscalar \(i\) has to be dealt with more carefully).  Geometrically one has the dual of the gradient (a plane normal to the vector itself) dotted with the plane formed by the gradient and the vector operated on.  The corresponding statement for the cross product form is that we have a dot product of a vector with a vector normal to it, so also intuitively expect a zero.  In either case, because we are talking about operators here
just saying this is zero because of geometrical arguments is not necessarily convincing.  Let us evaluate this explicitly in
coordinates to verify

\begin{equation}\label{eqn:vectorDifferentialIdentities:140}
\begin{aligned}
(i \spacegrad) \cdot (\spacegrad \wedge \Bh)
&= \gpgradezero{i \spacegrad (\spacegrad \wedge \Bh) } \\
&= \gpgradezero{i \sum_{k,j,l} \sigma_k \partial_k \left((\sigma_j \wedge \sigma_l) \partial_j h^l\right) } \\
&= -i \sum_{l} \sigma_l \wedge \left( \sum_{k<j} (\sigma_k \wedge \sigma_j) (\partial_k \partial_j -\partial_j \partial_k) h^l \right) \\
\end{aligned}
\end{equation}

This inner quantity is zero, again by equality of mixed partials.  While the dual form of this identity was not really any better than the cross
product form, there is nothing in this zero equality proof that was tied to the dimension of the vectors involved, so we do have a more general form
than can be expressed by the cross product, which could be of value in Minkowski space later.

\section{Curl of a curl}
\index{curl!curl}

This will also have a dual form.  That is

\begin{equation}\label{eqn:vectorDifferentialIdentities:160}
\begin{aligned}
\spacegrad \cross (\spacegrad \cross \Bh)
&= -i (\spacegrad \wedge (\spacegrad \cross \Bh)) \\
&= -i (\spacegrad \wedge (-i (\spacegrad \wedge \Bh))) \\
&= -i \gpgradetwo{\spacegrad (-i (\spacegrad \wedge \Bh))} \\
&= i \gpgradetwo{i \spacegrad (\spacegrad \wedge \Bh)} \\
&= i^2 \spacegrad \cdot (\spacegrad \wedge \Bh) \\
&= - \spacegrad \cdot (\spacegrad \wedge \Bh) \\
\end{aligned}
\end{equation}

Now, let us expand this quantity
\begin{equation}\label{eqn:vectorDifferentialIdentities:180}
\begin{aligned}
\spacegrad \cdot (\spacegrad \wedge \Bh)
\end{aligned}
\end{equation}

If the gradient could be treated as a plain old vector we could just do
\begin{equation}\label{eqn:vectorDifferentialIdentities:200}
\begin{aligned}
\Ba \cdot (\Ba \wedge \Bh) &= \Ba^2 \Bh - \Ba(\Ba \cdot \Bh) \\
\end{aligned}
\end{equation}

With the gradient substituted this is exactly the desired identity (with the expected sign difference)

\begin{equation}\label{eqn:vectorDifferentialIdentities:220}
\begin{aligned}
\spacegrad \cdot (\spacegrad \wedge \Bh) &= \spacegrad^2 \Bh - \spacegrad(\spacegrad \cdot \Bh) \\
\end{aligned}
\end{equation}

A coordinate expansion to truly verify that this is valid is logically still required, but having done the others above, it is clear how this
would work out.

\section{Laplacian of a vector}
\index{vector!Laplacian}

This one is not interesting seeming.

\section{Zero curl implies gradient solution}

This theorems is mentioned in the text without proof.

Theorem was

\begin{equation*}
\begin{array}{l l l}
\text{If} &          \spacegrad \cross \BA &= 0 \\
\text{there is a } &                  \psi &    \\
\text{such that  } & \BA &= \spacegrad \psi \\
\end{array}
\end{equation*}

This appears to be half of an if and only if theorem.  The unstated part is if one has a gradient then the curl is zero

\begin{equation}\label{eqn:vectorDifferentialIdentities:240}
\begin{aligned}
\BA = \spacegrad \psi \\
\implies \\
\spacegrad \cross \BA &= \spacegrad \cross \spacegrad \psi = 0
\end{aligned}
\end{equation}

This last was proven above, and follows from the assumed mixed partial equality.  Now, the real problem here is to find \(\psi\) given \(\BA\).
First note that we can remove the three dimensionality of the theorem by duality writing \(\spacegrad \cross \BA = -i (\spacegrad \wedge \BA)\).
In one sense changing the theorem to use the wedge instead of cross makes the problem harder since the wedge product is defined not just
for \R{3}.  However, this also allows starting with the simpler \R{2} case, so let us do that one first.

Write

\begin{equation}\label{eqn:vecDiffIdent:aDefined}
\begin{aligned}
\BA = \sigma^1 A_1 + \sigma^2 A_2 = \sigma^1 (\partial_1 \psi) + \sigma^2 (\partial_2 \psi)
\end{aligned}
\end{equation}

The gradient is
\begin{equation}\label{eqn:vectorDifferentialIdentities:260}
\begin{aligned}
\spacegrad = \sigma^1 \partial_1 + \sigma^2 \partial_2
\end{aligned}
\end{equation}

Our curl is then

\begin{equation}\label{eqn:vectorDifferentialIdentities:280}
\begin{aligned}
(\sigma^1 \partial_1 + \sigma^2 \partial_2) \wedge (\sigma^1 A_1 + \sigma^2 A_2)
&=
(\sigma^1 \wedge \sigma^2) (\partial_1 A_2 - \partial_2 A_1)
\end{aligned}
\end{equation}

So we have
\begin{equation}\label{eqn:vecDiffIdent:curlZero}
\begin{aligned}
\partial_1 A_2 = \partial_2 A_1
\end{aligned}
\end{equation}

Now from \eqnref{eqn:vecDiffIdent:aDefined} this means we must have

\begin{equation}\label{eqn:vectorDifferentialIdentities:300}
\begin{aligned}
\partial_1 \partial_2 \psi = \partial_2 \partial_1 \psi
\end{aligned}
\end{equation}

This is just a statement of mixed partial equality, and does not look particularly useful for solving for \(\psi\).  It really shows that the
is redundancy in the problem, and instead of substituting for both of \(A_1\) and \(A_2\)
in \eqnref{eqn:vecDiffIdent:curlZero}, we can use one or the other.

Doing so we have two equations, either of which we can solve for
\begin{equation}\label{eqn:vectorDifferentialIdentities:320}
\begin{aligned}
\partial_2 \partial_1 \psi &= \partial_1 A_2 \\
\partial_1 \partial_2 \psi &= \partial_2 A_1
\end{aligned}
\end{equation}

Integrating once gives
\begin{equation}\label{eqn:vectorDifferentialIdentities:340}
\begin{aligned}
\partial_1 \psi &= \int \partial_1 A_2 dy + B(x) \\
\partial_2 \psi &= \int \partial_2 A_1 dx + C(y)
\end{aligned}
\end{equation}

And a second time produces solutions for \(\psi\) in terms of the vector coordinates
\begin{equation}\label{eqn:vecDiffIdent:twoIntegrations}
\begin{aligned}
\psi &= \iint \PD{x}{A_2} dy dx + \int B(x) dx + D(y) \\
\psi &= \iint \PD{y}{A_1} dx dy + \int C(y) dy + E(x)
\end{aligned}
\end{equation}

Is there a natural way to merge these so that \(\psi\) can be expressed more symmetrically in terms of both coordinates?
Looking at \eqnref{eqn:vecDiffIdent:twoIntegrations} I am led to guess that its possible to
combine these into a single equation expressing \(\psi\) in terms of both \(A_1\) and \(A_2\).  One way to do so is perhaps just to average the
two as in

\begin{equation}\label{eqn:vectorDifferentialIdentities:360}
\begin{aligned}
\psi &= \alpha \iint \PD{x}{A_2} dy dx + (1-\alpha) \iint \PD{y}{A_1} dx dy + \int C(y) dy + E(x) + \int B(x) dx + D(y)
\end{aligned}
\end{equation}

But that seems pretty arbitrary.  Perhaps that is the point?

FIXME: work some examples.

%above it looks like these will combine naturally via Green's theorem which introduces a difference in
%sign intrinsically related to the curl.  That is been explored in \chapcite{PJStokes1}, and \chapcite{PJStokes2}, and intuitively one expects that
%a complexification of the area element will produce this result.

%Lets get rid of
%the integration constant functions to start with, pulling them into the integrals.  Starting over this way, the first integration gives us

%\begin{align*}
%\PD{x}{ \psi(x,y)} &= \int_{v = b_1(x)}^{b_2(x)} \PD{x}{A_2(x,v)} dv \\
%\PD{y}{ \psi(x,y)} &= \int_{u = c_1(y)}^{c_2(y)} \PD{y}{A_1(u,y)} du
%\end{align*}
%
%FIXME: second integration gives?
%\begin{align*}
%\psi(x,y) &= \int_{u=d_1(x,y)}^{d_2(x,y)} \left(\int_{v = b_1(u)}^{b_2(u)} \PD{u}{A_2(u,v)} dv \right) du \\
%\psi(x,y) &= \int_{v=e_1(x,y)}^{e_2(x,y)} \left(\int_{u = c_1(v)}^{c_2(v)} \PD{v}{A_1(u,v)} du \right) dv
%\end{align*}
%
%... this does not seem right.  Think it through better.
%

FIXME: look at more than the \R{2} case.

\section{Zero divergence implies curl solution}

This theorems is mentioned in the text without proof.

Theorem was

\begin{equation*}
\begin{array}{l l l}
\text{If} &                 \spacegrad \cdot \BD &= 0 \\
\text{there is a } &                         \BC &    \\
\text{such that  } & \BD &= \spacegrad \cross \BC \\
\end{array}
\end{equation*}

As above, if \(\BD = \spacegrad \cross \BC\), then we have

\begin{equation}\label{eqn:vectorDifferentialIdentities:380}
\begin{aligned}
\spacegrad \cdot \BD &= \spacegrad \cdot (\spacegrad \cross \BC)
\end{aligned}
\end{equation}

and this has already been shown to be zero.  So the problem becomes find \(\BC\) given \(\BD\).

Also, as before an equivalent generalized (or de-generalized) problem can be expressed.

That is
\begin{equation}\label{eqn:vectorDifferentialIdentities:400}
\begin{aligned}
\spacegrad \cdot \BD
&= \gpgradezero{\spacegrad \BD} \\
&= \gpgradezero{\spacegrad (\spacegrad \cross \BC)} \\
&= \gpgradezero{\spacegrad -i (\spacegrad \wedge \BC)} \\
&= -\gpgradezero{i\spacegrad \cdot (\spacegrad \wedge \BC)} -\gpgradezero{i\spacegrad \wedge (\spacegrad \wedge \BC)} \\
&= -\gpgradezero{i\spacegrad \cdot (\spacegrad \wedge \BC)} \\
\end{aligned}
\end{equation}

So if \(\spacegrad \cdot \BD\) it is also true that \(\spacegrad \cdot (\spacegrad \wedge \BC) = 0\)

Thus the (de)generalized theorem to prove is

\begin{equation*}
\begin{array}{l l l}
\text{If} &                 \spacegrad \cdot D &= 0 \\
\text{there is a } &                       C &    \\
\text{such that  } & D &= \spacegrad \wedge C \\
\end{array}
\end{equation*}

In the \R{3} case, to prove the original theorem we want a bivector \(D = -i\BD\), and seek a vector \(C\) such that
\(D = \spacegrad \wedge C\) (\(\BD = -i (\spacegrad \wedge C)\)).

\begin{equation}\label{eqn:vectorDifferentialIdentities:420}
\begin{aligned}
\spacegrad \cdot D
&= \spacegrad \cdot (\spacegrad \wedge C) \\
&= (\sigma^k \partial_k) \cdot (\sigma^i \wedge \sigma^j \partial_i C_j) \\
&= \sigma^k \cdot (\sigma^i \wedge \sigma^j) \partial_k \partial_i C_j \\
&= ( \sigma^j \delta^{ki} - \sigma^i \delta^{kj} ) \partial_k \partial_i C_j \\
&= \sigma^j \partial_i \partial_i C_j - \sigma^i \partial_j \partial_i C_j \\
&= \sigma^j \partial_i (\partial_i C_j -\partial_j C_i) \\
\end{aligned}
\end{equation}

If this is to equal zero we must have the following constraint on C
\begin{equation}\label{eqn:vecDiffIdent:blah}
\begin{aligned}
\partial_{ii} C_j = \partial_{ij} C_i
\end{aligned}
\end{equation}

If the following equality was also true
\begin{equation}\label{eqn:vectorDifferentialIdentities:440}
\begin{aligned}
\partial_{i} C_j = \partial_j C_i
\end{aligned}
\end{equation}

Then this would also work, but would also mean \(D\) equals zero so that is not an interesting solution.  So, we must go back to
\eqnref{eqn:vecDiffIdent:blah} and solve for \(C_k\) in terms of \(D\).

Suppose we have D explicitly in terms of coordinates

\begin{equation}\label{eqn:vectorDifferentialIdentities:460}
\begin{aligned}
D
&= D_{ij} \sigma^i \wedge \sigma^j \\
&= \sum_{i<j} (D_{ij} -D_{ji})\sigma^i \wedge \sigma^j
\end{aligned}
\end{equation}

compare this to \(\spacegrad \wedge C\)

\begin{equation}\label{eqn:vectorDifferentialIdentities:480}
\begin{aligned}
C &= (\partial_i C_j ) \sigma^i \wedge \sigma^j \\
  &= \sum_{i<j} (\partial_i C_j -\partial_j C_i) \sigma^i \wedge \sigma^j
\end{aligned}
\end{equation}

With the identity
\begin{equation}\label{eqn:vectorDifferentialIdentities:500}
\begin{aligned}
\partial_i C_j = D_ij
\end{aligned}
\end{equation}

\Eqnref{eqn:vecDiffIdent:blah} becomes
\begin{equation}\label{eqn:vectorDifferentialIdentities:520}
\begin{aligned}
\partial_{ij} C_i &= \partial_{i} D_{ij}  \\
\implies \\
\partial_{j} C_i &= D_{ij} + \alpha_{ij}(x^{k \ne i})
\end{aligned}
\end{equation}

Where \(\alpha_{ij}(x^{k \ne i})\) is some function of all the \(x^k \ne x^i\).

Integrating once more we have
\begin{equation}\label{eqn:vectorDifferentialIdentities:540}
\begin{aligned}
C_i &= \int \left(D_{ij} + \alpha_{ij}(x^{k \ne i}) \right) dx^j + \beta_{ij}(x^{k \ne j})
\end{aligned}
\end{equation}

%
% Copyright � 2012 Peeter Joot.  All Rights Reserved.
% Licenced as described in the file LICENSE under the root directory of this GIT repository.
%

% 
% 
%\documentclass[]{eliblog}

\usepackage{amsmath}
\usepackage{mathpazo}

%
% shorthand for bold symbols, convenient for vectors and matrices
%
\newcommand{\Ba}[0]{\mathbf{a}}
\newcommand{\Bb}[0]{\mathbf{b}}
\newcommand{\Bc}[0]{\mathbf{c}}
\newcommand{\Bd}[0]{\mathbf{d}}
\newcommand{\Be}[0]{\mathbf{e}}
\newcommand{\Bf}[0]{\mathbf{f}}
\newcommand{\Bg}[0]{\mathbf{g}}
\newcommand{\Bh}[0]{\mathbf{h}}
\newcommand{\Bi}[0]{\mathbf{i}}
\newcommand{\Bj}[0]{\mathbf{j}}
\newcommand{\Bk}[0]{\mathbf{k}}
\newcommand{\Bl}[0]{\mathbf{l}}
\newcommand{\Bm}[0]{\mathbf{m}}
\newcommand{\Bn}[0]{\mathbf{n}}
\newcommand{\Bo}[0]{\mathbf{o}}
\newcommand{\Bp}[0]{\mathbf{p}}
\newcommand{\Bq}[0]{\mathbf{q}}
\newcommand{\Br}[0]{\mathbf{r}}
\newcommand{\Bs}[0]{\mathbf{s}}
\newcommand{\Bt}[0]{\mathbf{t}}
\newcommand{\Bu}[0]{\mathbf{u}}
\newcommand{\Bv}[0]{\mathbf{v}}
\newcommand{\Bw}[0]{\mathbf{w}}
\newcommand{\Bx}[0]{\mathbf{x}}
\newcommand{\By}[0]{\mathbf{y}}
\newcommand{\Bz}[0]{\mathbf{z}}
\newcommand{\BA}[0]{\mathbf{A}}
\newcommand{\BB}[0]{\mathbf{B}}
\newcommand{\BC}[0]{\mathbf{C}}
\newcommand{\BD}[0]{\mathbf{D}}
\newcommand{\BE}[0]{\mathbf{E}}
\newcommand{\BF}[0]{\mathbf{F}}
\newcommand{\BG}[0]{\mathbf{G}}
\newcommand{\BH}[0]{\mathbf{H}}
\newcommand{\BI}[0]{\mathbf{I}}
\newcommand{\BJ}[0]{\mathbf{J}}
\newcommand{\BK}[0]{\mathbf{K}}
\newcommand{\BL}[0]{\mathbf{L}}
\newcommand{\BM}[0]{\mathbf{M}}
\newcommand{\BN}[0]{\mathbf{N}}
\newcommand{\BO}[0]{\mathbf{O}}
\newcommand{\BP}[0]{\mathbf{P}}
\newcommand{\BQ}[0]{\mathbf{Q}}
\newcommand{\BR}[0]{\mathbf{R}}
\newcommand{\BS}[0]{\mathbf{S}}
\newcommand{\BT}[0]{\mathbf{T}}
\newcommand{\BU}[0]{\mathbf{U}}
\newcommand{\BV}[0]{\mathbf{V}}
\newcommand{\BW}[0]{\mathbf{W}}
\newcommand{\BX}[0]{\mathbf{X}}
\newcommand{\BY}[0]{\mathbf{Y}}
\newcommand{\BZ}[0]{\mathbf{Z}}

\newcommand{\Bzero}[0]{\mathbf{0}}
\newcommand{\Btheta}[0]{\boldsymbol{\theta}}
\newcommand{\Btau}[0]{\boldsymbol{\tau}}
\newcommand{\Bomega}[0]{\boldsymbol{\omega}}

%
% shorthand for unit vectors
%
\newcommand{\acap}[0]{\hat{\Ba}}
\newcommand{\bcap}[0]{\hat{\Bb}}
\newcommand{\ccap}[0]{\hat{\Bc}}
\newcommand{\dcap}[0]{\hat{\Bd}}
\newcommand{\ecap}[0]{\hat{\Be}}
\newcommand{\fcap}[0]{\hat{\Bf}}
\newcommand{\gcap}[0]{\hat{\Bg}}
\newcommand{\hcap}[0]{\hat{\Bh}}
\newcommand{\icap}[0]{\hat{\Bi}}
\newcommand{\jcap}[0]{\hat{\Bj}}
\newcommand{\kcap}[0]{\hat{\Bk}}
\newcommand{\lcap}[0]{\hat{\Bl}}
\newcommand{\mcap}[0]{\hat{\Bm}}
\newcommand{\ncap}[0]{\hat{\Bn}}
\newcommand{\ocap}[0]{\hat{\Bo}}
\newcommand{\pcap}[0]{\hat{\Bp}}
\newcommand{\qcap}[0]{\hat{\Bq}}
\newcommand{\rcap}[0]{\hat{\Br}}
\newcommand{\scap}[0]{\hat{\Bs}}
\newcommand{\tcap}[0]{\hat{\Bt}}
\newcommand{\ucap}[0]{\hat{\Bu}}
\newcommand{\vcap}[0]{\hat{\Bv}}
\newcommand{\wcap}[0]{\hat{\Bw}}
\newcommand{\xcap}[0]{\hat{\Bx}}
\newcommand{\ycap}[0]{\hat{\By}}
\newcommand{\zcap}[0]{\hat{\Bz}}
\newcommand{\thetacap}[0]{\hat{\Btheta}}

%
% to write R^n and C^n in a distinguishable fashion.  Perhaps change this
% to the double lined characters upon figuring out how to do so.
%
\newcommand{\C}[1]{$\mathbb{C}^{#1}$}
\newcommand{\R}[1]{$\mathbb{R}^{#1}$}

%
% various generally useful helpers
%

% derivative of #1 wrt. #2:
\newcommand{\D}[2] {\frac {d#2} {d#1}}

\newcommand{\inv}[1]{\frac{1}{#1}}
\newcommand{\cross}[0]{\times}

\newcommand{\abs}[1]{\lvert{#1}\rvert}
\newcommand{\norm}[1]{\lVert{#1}\rVert}
\newcommand{\innerprod}[2]{\langle{#1}, {#2}\rangle}
\newcommand{\dotprod}[2]{{#1} \cdot {#2}}
\newcommand{\bdotprod}[2]{\left({#1} \cdot {#2}\right)}
\newcommand{\crossprod}[2]{{#1} \cross {#2}}
\newcommand{\tripleprod}[3]{\dotprod{\left(\crossprod{#1}{#2}\right)}{#3}}

\DeclareMathOperator{\Proj}{Proj}
\DeclareMathOperator{\Span}{span}
\DeclareMathOperator{\Sgn}{sgn}
\DeclareMathOperator{\Area}{Area}
\DeclareMathOperator{\Volume}{Volume}

%
% A few miscellaneous things specific to this document
%
\newcommand{\crossop}[1]{\crossprod{#1}{}}

% R2 vector.
\newcommand{\VectorTwo}[2]{
\begin{bmatrix}
 {#1} \\
 {#2}
\end{bmatrix}
}

\newcommand{\VectorN}[1]{
\begin{bmatrix}
{#1}_1 \\
{#1}_2 \\
\vdots \\
{#1}_N \\
\end{bmatrix}
}

\newcommand{\DETuvij}[4]{
\begin{vmatrix}
 {#1}_{#3} & {#1}_{#4} \\
 {#2}_{#3} & {#2}_{#4}
\end{vmatrix}
}

\newcommand{\DETuvwijk}[6]{
\begin{vmatrix}
 {#1}_{#4} & {#1}_{#5} & {#1}_{#6} \\
 {#2}_{#4} & {#2}_{#5} & {#2}_{#6} \\
 {#3}_{#4} & {#3}_{#5} & {#3}_{#6}
\end{vmatrix}
}

\newcommand{\DETuvwxijkl}[8]{
\begin{vmatrix}
 {#1}_{#5} & {#1}_{#6} & {#1}_{#7} & {#1}_{#8} \\
 {#2}_{#5} & {#2}_{#6} & {#2}_{#7} & {#2}_{#8} \\
 {#3}_{#5} & {#3}_{#6} & {#3}_{#7} & {#3}_{#8} \\
 {#4}_{#5} & {#4}_{#6} & {#4}_{#7} & {#4}_{#8} \\
\end{vmatrix}
}

%\newcommand{\DETuvwxyijklm}[10]{
%\begin{vmatrix}
% {#1}_{#6} & {#1}_{#7} & {#1}_{#8} & {#1}_{#9} & {#1}_{#10} \\
% {#2}_{#6} & {#2}_{#7} & {#2}_{#8} & {#2}_{#9} & {#2}_{#10} \\
% {#3}_{#6} & {#3}_{#7} & {#3}_{#8} & {#3}_{#9} & {#3}_{#10} \\
% {#4}_{#6} & {#4}_{#7} & {#4}_{#8} & {#4}_{#9} & {#4}_{#10} \\
% {#5}_{#6} & {#5}_{#7} & {#5}_{#8} & {#5}_{#9} & {#5}_{#10}
%\end{vmatrix}
%}

% R3 vector.
\newcommand{\VectorThree}[3]{
\begin{bmatrix}
 {#1} \\
 {#2} \\
 {#3}
\end{bmatrix}
}



\author{Peeter Joot}
\email{peeter.joot@gmail.com}

%\usepackage{txfonts}

\chapter{Stokes theorem applied to vector and bivector fields}
\label{chap:stokesGradeTwo}

%%\date{July 17, 2009}
%%\revisionInfo{$RCSfile: stokesGradeTwo.tex,v $ Last $Revision: 1.5 $ $Date: 2009/08/01 22:12:18 $}

%\blogpage{http://sites.google.com/site/peeterjoot/math2009/stokesGradeTwo.pdf}
%\date{July 17, 2009.  $RCSfile: stokesGradeTwo.tex,v $ Last $Revision: 1.5 $ $Date: 2009/08/01 22:12:18 $}

\beginArtWithToc

\section{Vector Stokes Theorem}

I found my self forgetting stokes theorem once again.  Redo this for the simplest case of a parallelogram area element.

What I recall is that we have on one side the curl dotted into the plane of the surface area element

\begin{align}
\int ( \grad \wedge A ) \cdot d^2 x
\end{align}

and on the other side a loop integral

\begin{align}
\ointctrclockwise A \cdot dx
\end{align}

Comparing the two we should end up with the same form and thus determine the form of the grade two Stokes equation (i.e. for curl of a vector).

\subsection{Bivector product part}

\begin{align*}
( \grad \wedge A ) \cdot d^2 x 
&=
( \grad \wedge A ) \cdot \left(\PD{\alpha}{x} \wedge \PD{\beta}{x}\right) 
d\alpha d\beta \\
&=
\partial_\mu A_\nu \PD{\alpha}{x^\sigma} \PD{\beta}{x^\epsilon} (\gamma^\mu \wedge \gamma^\nu) \cdot (\gamma_\sigma \wedge \gamma_\epsilon) 
d\alpha d\beta \\
&=
\partial_\mu A_\nu \PD{\alpha}{x^\sigma} \PD{\beta}{x^\epsilon} ( {\delta^\mu}_\epsilon {\delta^\nu}_\sigma - {\delta^\mu}_\sigma {\delta^\nu}_\epsilon ) 
d\alpha d\beta \\
&=
\partial_\mu A_\nu \left( \PD{\alpha}{x^\nu} \PD{\beta}{x^\mu} - \PD{\alpha}{x^\mu} \PD{\beta}{x^\nu} \right) 
d\alpha d\beta \\
\end{align*}

So we have
\begin{align}\label{eqn:stokesGradeTwo:withJac}
( \grad \wedge A ) \cdot d^2 x &= -\partial_\mu A_\nu \frac{\partial (x^\mu, x^\nu)}{\partial (\alpha, \beta)} d\alpha d\beta
\end{align}

\subsection{Loop integral part}

Integrating around a parallelogram spacetime area element with sides $d\alpha \partial x/\partial \alpha$ and $d\beta \partial x/\partial \beta$, as
depicted in figure (\ref{fig:surface_area_element}), we have

\imageFigure{figures/surface_area_element}{Surface area element}{fig:surface_area_element}{0.4}

\begin{align*}
\ointctrclockwise 
A \cdot dx
&=
\int
{\left. A \right\vert}_{\beta=\beta_0} \cdot \PD{\alpha}{x} d\alpha
+ {\left. A \right\vert}_{\alpha=\alpha_1} \cdot \PD{\beta}{x} d\beta
+ {\left. A \right\vert}_{\beta=\beta_1} \cdot \left( -\PD{\alpha}{x} d\alpha \right)
+ {\left. A \right\vert}_{\alpha=\alpha_0} \cdot \left( -\PD{\beta}{x} d\beta \right) 
\\
&=
\int
\left( {\left. A \right\vert}_{\alpha=\alpha_1} - {\left. A \right\vert}_{\alpha=\alpha_0} \right) \cdot \PD{\beta}{x} d\beta
-\left( {\left. A \right\vert}_{\beta=\beta_1} - {\left. A \right\vert}_{\beta=\beta_0} \right) \cdot \PD{\alpha}{x} d\alpha 
\\
&=
\int
\PD{\alpha}{A} \cdot \PD{\beta}{x} d\alpha d\beta
-\PD{\beta}{A} \cdot \PD{\alpha}{x} d\beta d\alpha
\end{align*}

Expanding the derivatives in terms of coordinates we have

\begin{align*}
\PD{\sigma}{A} 
&=
\PD{\sigma}{A_\mu} \gamma^\mu \\
&= 
\PD{x^\nu}{A_\mu}\PD{\sigma}{x^\nu} \gamma^\mu \\
&= 
\partial_\nu A_\mu \PD{\sigma}{x^\nu} \gamma^\mu \\
\end{align*}

and
\begin{align*}
\PD{\sigma}{x} &= \PD{\sigma}{x^\nu} \gamma_\nu
\end{align*}

Assembling we have
\begin{align*}
\ointctrclockwise 
A \cdot dx
&=
\int
\partial_\nu A_\mu \left( \PD{\alpha}{x^\nu} \PD{\beta}{x^\mu} - \PD{\beta}{x^\nu} \PD{\alpha}{x^\mu} \right) d\alpha d\beta
\end{align*}

In terms of the Jacobian used in (\ref{eqn:stokesGradeTwo:withJac}) we have

\begin{align*}
\ointctrclockwise 
A \cdot dx &= \int \partial_\mu A_\nu \frac{\partial (x^\mu, x^\nu)}{\partial (\alpha, \beta)} d\alpha d\beta
\end{align*}

Comparing the two we have only a sign difference so the conclusion is that Stokes for a vector field (considering only a flat parallelogram area element) is

\begin{align}
\int ( \grad \wedge A ) \cdot d^2 x &= \ointclockwise A \cdot dx
\end{align}

Observe that there's an implied orientation of the area element on the LHS, required to match up with the orientation of the RHS integral.

\section{Bivector Stokes Theorem}

A parallelepiped volume element is depicted in figure (\ref{fig:volume_element}).  Three parameters $\alpha$, $\beta$, $\sigma$ generate a set of differential vector displacements spanning the three dimensional subspace

\imageFigure{figures/volume_element}{Differential volume element}{fig:volume_element}{0.4}

Writing the displacements 

\begin{align*}
dx_\alpha &= \PD{\alpha}{x} d\alpha \\
dx_\beta &= \PD{\beta}{x} d\beta \\
dx_\sigma &= \PD{\sigma}{x} d\sigma
\end{align*}

We have for the front, right and top face area elements 

\begin{align*}
dA_F &= dx_\alpha \wedge dx_\beta \\
dA_R &= dx_\beta \wedge dx_\sigma \\
dA_T &= dx_\sigma \wedge dx_\alpha \\
\end{align*}

These are the surfaces of constant parametrization, respectively, $\sigma = \sigma_1$, $\alpha = \alpha_1$, and $\beta = \beta_1$.  For a bivector, the flux through the surface is therefore

\begin{align*}
\int B \cdot dA 
&=
(B_{\sigma_1} \cdot dA_F - B_{\sigma_0} \cdot dA_P )
+ (B_{\alpha_1} \cdot dA_R - B_{\alpha_0} \cdot dA_L)
+ (B_{\beta_1} \cdot dA_T - B_{\beta_0} \cdot dA_B) \\
&=
d \sigma \PD{\sigma}{B} \cdot (dx_\alpha \wedge dx_\beta )
+ d \alpha \PD{\alpha}{B} \cdot (dx_\beta \wedge dx_\sigma) 
+ d \beta \PD{\beta}{B} \cdot (dx_\sigma \wedge dx_\alpha ) \\
\end{align*}

Written out in full this is a bit of a mess
\begin{align}\label{eqn:stokesGradeTwo:mess}
\int B \cdot dA 
&=
d \alpha d\beta d\sigma 
\partial_\mu B \cdot
\left(
\left(
- \PD{\sigma}{x^\mu} \PD{\beta}{x^\nu} \PD{\alpha}{x^\epsilon} 
+ \PD{\alpha}{x^\mu} \PD{\beta}{x^\nu} \PD{\sigma}{x^\epsilon} 
+ \PD{\beta}{x^\mu} \PD{\sigma}{x^\nu} \PD{\alpha}{x^\epsilon} 
\right) 
(\gamma_\nu \wedge \gamma_\epsilon )
\right) 
\end{align}

It should equal, at least up to a sign, $\int (\grad \wedge B) \cdot d^3 x$.  Expanding the latter is probably easier than regrouping the mess, and doing so we have

\begin{align*}
(\grad \wedge B) \cdot d^3 x
&=
d\alpha d\beta d\sigma ( \gamma^\mu \wedge \partial_\mu B)  \cdot \left( \PD{\alpha}{x} \wedge \PD{\beta}{x} \wedge \PD{\sigma}{x} \right) \\
&= 
d\alpha d\beta d\sigma \inv{2} ( \gamma^\mu \partial_\mu B + \partial_\mu B \gamma^\mu )  \cdot \left( \PD{\alpha}{x} \wedge \PD{\beta}{x} \wedge \PD{\sigma}{x} \right) \\
&=
d\alpha d\beta d\sigma \inv{2} \gpgradezero{
( \gamma^\mu \partial_\mu B + \partial_\mu B \gamma^\mu )  \left( \PD{\alpha}{x} \wedge \PD{\beta}{x} \wedge \PD{\sigma}{x} \right) }
\\
&=
d\alpha d\beta d\sigma \inv{2} 
\partial_\mu B \cdot
\gpgradetwo{
\left( \PD{\alpha}{x} \wedge \PD{\beta}{x} \wedge \PD{\sigma}{x} \right) \gamma^\mu 
+ \gamma^\mu \left( \PD{\alpha}{x} \wedge \PD{\beta}{x} \wedge \PD{\sigma}{x} \right) }
\\
&=
d\alpha d\beta d\sigma 
\partial_\mu B \cdot
\left( \left( \PD{\alpha}{x} \wedge \PD{\beta}{x} \wedge \PD{\sigma}{x} \right) \cdot \gamma^\mu \right)
\\
\end{align*}

Expanding just that trivector-vector dot product

\begin{align*}
\left( \PD{\alpha}{x} \wedge \PD{\beta}{x} \wedge \PD{\sigma}{x} \right) \cdot \gamma^\mu 
&=
\PD{\alpha}{x^\lambda} \PD{\beta}{x^\nu} \PD{\sigma}{x^\epsilon} \left( \gamma_\lambda \wedge \gamma_\nu \wedge \gamma_\epsilon \right) \cdot \gamma^\mu  \\
&=
\PD{\alpha}{x^\lambda} \PD{\beta}{x^\nu} \PD{\sigma}{x^\epsilon} \left( 
\gamma_\lambda \wedge \gamma_\nu {\delta_\epsilon}^\mu
-\gamma_\lambda \wedge \gamma_\epsilon {\delta_\nu}^\mu
+\gamma_\nu \wedge \gamma_\epsilon {\delta_\lambda}^\mu
\right) 
\end{align*}

So we have
\begin{align*}
(\grad \wedge B) \cdot d^3 x
&=
d\alpha d\beta d\sigma \PD{\alpha}{x^\lambda} \PD{\beta}{x^\nu} \PD{\sigma}{x^\epsilon} \partial_\mu B \cdot \left( 
\gamma_\lambda \wedge \gamma_\nu {\delta_\epsilon}^\mu
-\gamma_\lambda \wedge \gamma_\epsilon {\delta_\nu}^\mu
+\gamma_\nu \wedge \gamma_\epsilon {\delta_\lambda}^\mu
\right) 
\\
&=
d\alpha d\beta d\sigma 
\partial_\mu B \cdot \left( 
  \PD{\alpha}{x^\lambda} \PD{\beta}{x^\nu} \PD{\sigma}{x^\mu} \gamma_\lambda \wedge \gamma_\nu 
%- \PD{\alpha}{x^\lambda} \PD{\beta}{x^\mu} \PD{\sigma}{x^\epsilon} \gamma_\lambda \wedge \gamma_\epsilon 
+ \PD{\alpha}{x^\lambda} \PD{\beta}{x^\mu} \PD{\sigma}{x^\epsilon} \gamma_\epsilon \wedge \gamma_\lambda 
+ \PD{\alpha}{x^\mu} \PD{\beta}{x^\nu} \PD{\sigma}{x^\epsilon} \gamma_\nu \wedge \gamma_\epsilon 
\right) 
\\
&=
d\alpha d\beta d\sigma 
\partial_\mu B \cdot 
\left( 
\left( 
  \PD{\alpha}{x^\nu} \PD{\beta}{x^\epsilon} \PD{\sigma}{x^\mu} 
+ \PD{\alpha}{x^\epsilon} \PD{\beta}{x^\mu} \PD{\sigma}{x^\nu} 
+ \PD{\alpha}{x^\mu} \PD{\beta}{x^\nu} \PD{\sigma}{x^\epsilon} 
\right) 
\gamma_\nu \wedge \gamma_\epsilon 
\right) 
\\
\end{align*}

Noting that an $\epsilon$, $\nu$ interchange in the first term inverts the sign, we have an exact match with (\ref{eqn:stokesGradeTwo:mess}), thus fixing the sign for the
bivector form of Stokes theorem for the orientation picked in this diagram

\begin{align}
\int (\grad \wedge B) \cdot d^3 x &= \int B \cdot d^2 x
\end{align}

Like the vector case, there is a requirement to be very specific about the meaning given to the oriented surfaces, and the corresponding oriented volume element (which could be a volume subspace of a greater than three dimensional space).

%\EndNoBibArticle

%
% Copyright � 2012 Peeter Joot.  All Rights Reserved.
% Licenced as described in the file LICENSE under the root directory of this GIT repository.
%

% 
% 
%%
% Copyright � 2015 Peeter Joot.  All Rights Reserved.
% Licenced as described in the file LICENSE under the root directory of this GIT repository.
%
\documentclass[]{eliblog}

\usepackage{amsmath}
\usepackage{mathpazo}

%
% shorthand for bold symbols, convenient for vectors and matrices
%
\newcommand{\Ba}[0]{\mathbf{a}}
\newcommand{\Bb}[0]{\mathbf{b}}
\newcommand{\Bc}[0]{\mathbf{c}}
\newcommand{\Bd}[0]{\mathbf{d}}
\newcommand{\Be}[0]{\mathbf{e}}
\newcommand{\Bf}[0]{\mathbf{f}}
\newcommand{\Bg}[0]{\mathbf{g}}
\newcommand{\Bh}[0]{\mathbf{h}}
\newcommand{\Bi}[0]{\mathbf{i}}
\newcommand{\Bj}[0]{\mathbf{j}}
\newcommand{\Bk}[0]{\mathbf{k}}
\newcommand{\Bl}[0]{\mathbf{l}}
\newcommand{\Bm}[0]{\mathbf{m}}
\newcommand{\Bn}[0]{\mathbf{n}}
\newcommand{\Bo}[0]{\mathbf{o}}
\newcommand{\Bp}[0]{\mathbf{p}}
\newcommand{\Bq}[0]{\mathbf{q}}
\newcommand{\Br}[0]{\mathbf{r}}
\newcommand{\Bs}[0]{\mathbf{s}}
\newcommand{\Bt}[0]{\mathbf{t}}
\newcommand{\Bu}[0]{\mathbf{u}}
\newcommand{\Bv}[0]{\mathbf{v}}
\newcommand{\Bw}[0]{\mathbf{w}}
\newcommand{\Bx}[0]{\mathbf{x}}
\newcommand{\By}[0]{\mathbf{y}}
\newcommand{\Bz}[0]{\mathbf{z}}
\newcommand{\BA}[0]{\mathbf{A}}
\newcommand{\BB}[0]{\mathbf{B}}
\newcommand{\BC}[0]{\mathbf{C}}
\newcommand{\BD}[0]{\mathbf{D}}
\newcommand{\BE}[0]{\mathbf{E}}
\newcommand{\BF}[0]{\mathbf{F}}
\newcommand{\BG}[0]{\mathbf{G}}
\newcommand{\BH}[0]{\mathbf{H}}
\newcommand{\BI}[0]{\mathbf{I}}
\newcommand{\BJ}[0]{\mathbf{J}}
\newcommand{\BK}[0]{\mathbf{K}}
\newcommand{\BL}[0]{\mathbf{L}}
\newcommand{\BM}[0]{\mathbf{M}}
\newcommand{\BN}[0]{\mathbf{N}}
\newcommand{\BO}[0]{\mathbf{O}}
\newcommand{\BP}[0]{\mathbf{P}}
\newcommand{\BQ}[0]{\mathbf{Q}}
\newcommand{\BR}[0]{\mathbf{R}}
\newcommand{\BS}[0]{\mathbf{S}}
\newcommand{\BT}[0]{\mathbf{T}}
\newcommand{\BU}[0]{\mathbf{U}}
\newcommand{\BV}[0]{\mathbf{V}}
\newcommand{\BW}[0]{\mathbf{W}}
\newcommand{\BX}[0]{\mathbf{X}}
\newcommand{\BY}[0]{\mathbf{Y}}
\newcommand{\BZ}[0]{\mathbf{Z}}

\newcommand{\Bzero}[0]{\mathbf{0}}
\newcommand{\Btheta}[0]{\boldsymbol{\theta}}
\newcommand{\Btau}[0]{\boldsymbol{\tau}}
\newcommand{\Bomega}[0]{\boldsymbol{\omega}}

%
% shorthand for unit vectors
%
\newcommand{\acap}[0]{\hat{\Ba}}
\newcommand{\bcap}[0]{\hat{\Bb}}
\newcommand{\ccap}[0]{\hat{\Bc}}
\newcommand{\dcap}[0]{\hat{\Bd}}
\newcommand{\ecap}[0]{\hat{\Be}}
\newcommand{\fcap}[0]{\hat{\Bf}}
\newcommand{\gcap}[0]{\hat{\Bg}}
\newcommand{\hcap}[0]{\hat{\Bh}}
\newcommand{\icap}[0]{\hat{\Bi}}
\newcommand{\jcap}[0]{\hat{\Bj}}
\newcommand{\kcap}[0]{\hat{\Bk}}
\newcommand{\lcap}[0]{\hat{\Bl}}
\newcommand{\mcap}[0]{\hat{\Bm}}
\newcommand{\ncap}[0]{\hat{\Bn}}
\newcommand{\ocap}[0]{\hat{\Bo}}
\newcommand{\pcap}[0]{\hat{\Bp}}
\newcommand{\qcap}[0]{\hat{\Bq}}
\newcommand{\rcap}[0]{\hat{\Br}}
\newcommand{\scap}[0]{\hat{\Bs}}
\newcommand{\tcap}[0]{\hat{\Bt}}
\newcommand{\ucap}[0]{\hat{\Bu}}
\newcommand{\vcap}[0]{\hat{\Bv}}
\newcommand{\wcap}[0]{\hat{\Bw}}
\newcommand{\xcap}[0]{\hat{\Bx}}
\newcommand{\ycap}[0]{\hat{\By}}
\newcommand{\zcap}[0]{\hat{\Bz}}
\newcommand{\thetacap}[0]{\hat{\Btheta}}

%
% to write R^n and C^n in a distinguishable fashion.  Perhaps change this
% to the double lined characters upon figuring out how to do so.
%
\newcommand{\C}[1]{$\mathbb{C}^{#1}$}
\newcommand{\R}[1]{$\mathbb{R}^{#1}$}

%
% various generally useful helpers
%

% derivative of #1 wrt. #2:
\newcommand{\D}[2] {\frac {d#2} {d#1}}

\newcommand{\inv}[1]{\frac{1}{#1}}
\newcommand{\cross}[0]{\times}

\newcommand{\abs}[1]{\lvert{#1}\rvert}
\newcommand{\norm}[1]{\lVert{#1}\rVert}
\newcommand{\innerprod}[2]{\langle{#1}, {#2}\rangle}
\newcommand{\dotprod}[2]{{#1} \cdot {#2}}
\newcommand{\bdotprod}[2]{\left({#1} \cdot {#2}\right)}
\newcommand{\crossprod}[2]{{#1} \cross {#2}}
\newcommand{\tripleprod}[3]{\dotprod{\left(\crossprod{#1}{#2}\right)}{#3}}

\DeclareMathOperator{\Proj}{Proj}
\DeclareMathOperator{\Span}{span}
\DeclareMathOperator{\Sgn}{sgn}
\DeclareMathOperator{\Area}{Area}
\DeclareMathOperator{\Volume}{Volume}

%
% A few miscellaneous things specific to this document
%
\newcommand{\crossop}[1]{\crossprod{#1}{}}

% R2 vector.
\newcommand{\VectorTwo}[2]{
\begin{bmatrix}
 {#1} \\
 {#2}
\end{bmatrix}
}

\newcommand{\VectorN}[1]{
\begin{bmatrix}
{#1}_1 \\
{#1}_2 \\
\vdots \\
{#1}_N \\
\end{bmatrix}
}

\newcommand{\DETuvij}[4]{
\begin{vmatrix}
 {#1}_{#3} & {#1}_{#4} \\
 {#2}_{#3} & {#2}_{#4}
\end{vmatrix}
}

\newcommand{\DETuvwijk}[6]{
\begin{vmatrix}
 {#1}_{#4} & {#1}_{#5} & {#1}_{#6} \\
 {#2}_{#4} & {#2}_{#5} & {#2}_{#6} \\
 {#3}_{#4} & {#3}_{#5} & {#3}_{#6}
\end{vmatrix}
}

\newcommand{\DETuvwxijkl}[8]{
\begin{vmatrix}
 {#1}_{#5} & {#1}_{#6} & {#1}_{#7} & {#1}_{#8} \\
 {#2}_{#5} & {#2}_{#6} & {#2}_{#7} & {#2}_{#8} \\
 {#3}_{#5} & {#3}_{#6} & {#3}_{#7} & {#3}_{#8} \\
 {#4}_{#5} & {#4}_{#6} & {#4}_{#7} & {#4}_{#8} \\
\end{vmatrix}
}

%\newcommand{\DETuvwxyijklm}[10]{
%\begin{vmatrix}
% {#1}_{#6} & {#1}_{#7} & {#1}_{#8} & {#1}_{#9} & {#1}_{#10} \\
% {#2}_{#6} & {#2}_{#7} & {#2}_{#8} & {#2}_{#9} & {#2}_{#10} \\
% {#3}_{#6} & {#3}_{#7} & {#3}_{#8} & {#3}_{#9} & {#3}_{#10} \\
% {#4}_{#6} & {#4}_{#7} & {#4}_{#8} & {#4}_{#9} & {#4}_{#10} \\
% {#5}_{#6} & {#5}_{#7} & {#5}_{#8} & {#5}_{#9} & {#5}_{#10}
%\end{vmatrix}
%}

% R3 vector.
\newcommand{\VectorThree}[3]{
\begin{bmatrix}
 {#1} \\
 {#2} \\
 {#3}
\end{bmatrix}
}



\author{Peeter Joot}
\email{peeter.joot@gmail.com}

%\usepackage{txfonts}

\chapter{Stokes theorem.  Revisited again}
%\chapter{Stokes theorem in Geometric Algebra formalism}
\label{chap:stokesNoTensor}


%\blogpage{http://sites.google.com/site/peeterjoot/math2009/stokesNoTensor.pdf}
%\revisionInfo{stokesNoTensor.tex}
%\date{July 21, 2009}

\beginArtWithToc

\section{Motivation}

Relying on pictorial means and a brute force ugly comparison of left and right hand sides, a verification of Stokes theorem for the vector and bivector cases was performed (\chapcite{stokesGradeTwo}).  This was more of a confirmation than a derivation, and the technique fails the transition to the trivector case.  The trivector case is of particular interest in electromagnetism since that and a duality transformation provides a four-vector divergence theorem.

The fact that the pictorial means of defining the boundary surface doesn't work well in four vector space is not the only unsatisfactory aspect of the previous treatment.  The fact that a coordinate expansion of the hypervolume element and hypersurface element was performed in the LHS and RHS comparisons was required is particularly ugly.  It is a lot of work and essentially has to be undone on the opposing side of the equation.  Comparing to previous attempts to come to terms with Stokes theorem in (\chapcite{PJStokes1}) and (\chapcite{PJStokes2}) this more recent attempt at least avoids the requirement for a tensor expansion of the vector or bivector.  It should be possible to build on this and minimize the amount of coordinate expansion required and go directly from the volume integral to the expression of the boundary surface.

\section{Do it}
\subsection{Notation and Setup}

The desire is to relate the curl hypervolume integral to a hypersurface integral on the boundary

\begin{align}\label{eqn:stokesNoTensor:stokes}
\int (\grad \wedge F) \cdot d^k x = \int F \cdot d^{k-1} x
\end{align}

In order to put meaning to this statement the volume and surface elements need to be properly defined.  In order that this be a scalar equation, the object $F$ in the integral is required to be of grade $k-1$, and $k \le n$ where $n$ is the dimension of the vector space that generates the object $F$.

\subsection{Reciprocal frames}

As evident in equation (\ref{eqn:stokesNoTensor:stokes}) a metric is required to define the dot product.  If an affine non-metric formulation
of Stokes theorem is possible it will not be attempted here.  A reciprocal basis pair will be utilized, defined by

\begin{align}\label{eqn:stokesNoTensor:140}
\gamma^\mu \cdot \gamma_\nu = {\delta^\mu}_\nu
\end{align}

Both of the sets $\{\gamma_\mu\}$ and $\{\gamma^\mu\}$ are taken to span the space, but are not required to be orthogonal.  The notation is consistent with the Dirac reciprocal basis, and there will not be anything in this treatment that prohibits the Minkowski metric signature required for such a relativistic space.

Vector decomposition in terms of coordinates follows by taking dot products.  We write

\begin{align}\label{eqn:stokesNoTensor:150}
x = x^\mu \gamma_\mu = x_\nu \gamma^\nu
\end{align}

\subsection{Gradient}

When working with a non-orthonormal basis, use of the reciprocal frame can be utilized to express the gradient.

\begin{align}\label{eqn:stokesNoTensor:160}
\grad \equiv \gamma^\mu \partial_\mu \equiv \sum_\mu \gamma^\mu \PD{x^\mu}{}
\end{align}

This contains what may perhaps seem like an odd seeming mix of upper and lower indexes in this definition.  This is how the gradient is defined in \citep{doran2003gap}.  Although it is possible to accept this definition and work with it, this form can be justified by require of the gradient consistency with the the definition of directional derivative.  A definition of the directional derivative that works for single and multivector functions, in \R{3} and other more general spaces is

\begin{align}\label{eqn:stokesNoTensor:170}
a \cdot \grad F \equiv \lim_{\lambda \rightarrow 0} \frac{F(x + a\lambda) - F(x)}{\lambda} = {\left.\PD{\lambda}{F(x + a\lambda)} \right\vert}_{\lambda=0}
\end{align}

Taylor expanding about $\lambda=0$ in terms of coordinates we have

\begin{align*}
{\left.\PD{\lambda}{F(x + a\lambda)} \right\vert}_{\lambda=0}
&= a^\mu \PD{x^\mu}{F} \\
%&= a^\mu \partial_\mu F \\
&= (a^\nu \gamma_\nu) \cdot (\gamma^\mu \partial_\mu) F \\
&= a \cdot \grad F \quad\quad\quad\square
\end{align*}

The lower index representation of the vector coordinates could also have been used, so using the directional derivative to imply a definition of the gradient, we have an additional alternate representation of the gradient

\begin{align}\label{eqn:stokesNoTensor:180}
\grad \equiv \gamma_\mu \partial^\mu \equiv \sum_\mu \gamma_\mu \PD{x_\mu}{}
\end{align}

\subsection{Volume element}
We define the hypervolume in terms of parametrized vector displacements $x = x(a_1, a_2, ... a_k)$.  For the vector x we can form a pseudoscalar for the subspace spanned by this parametrization by wedging the displacements in each of the directions defined by variation of the parameters.  For $m \in [1,k]$ let

\begin{align}\label{eqn:stokesNoTensor:oneForm}
dx_i = \PD{a_i}{x} da_i = \gamma_\mu \PD{a_i}{x^\mu} da_i,
\end{align}

%\begin{align*}
%dx_1 & = \PD{a_1}{x} da_1 = \gamma_\mu \PD{a_1}{x^\mu} da_1 \\
%dx_2 & = \PD{a_2}{x} da_2 = \gamma_\mu \PD{a_2}{x^\mu} da_2 \\
%\hdots \\
%dx_k & = \PD{a_k}{x} da_k = \gamma_\mu \PD{a_k}{x^\mu} da_k,
%\end{align*}

so the hypervolume element for the subspace in question is
\begin{align}\label{eqn:stokesNoTensor:volumeElement}
d^k x \equiv dx_1 \wedge dx_2 \cdots dx_k
\end{align}

This can be expanded explicitly in coordinates

\begin{align*}
d^k x 
&= da_1 da_2 \cdots da_k 
\left(
\PD{a_1}{x^{\mu_1}} 
\PD{a_2}{x^{\mu_2}} 
\cdots
\PD{a_k}{x^{\mu_k}} 
\right)
( \gamma_{\mu_1} \wedge \gamma_{\mu_2} \wedge \cdots \wedge \gamma_{\mu_k} ) \\
%&= 
%da_1 da_2 \cdots da_k 
%\sum_{\mu_1 < \mu_2 < \cdots < \mu_k}
%\PD{(a_1, a_2, \cdots, a_k)}{(x^{\mu_1}, x^{\mu_2}, \cdots, x^{\mu_k})}
%( \gamma_{\mu_1} \wedge \gamma_{\mu_2} \wedge \cdots \wedge \gamma_{\mu_k} ) \\
\end{align*}

Observe that when $k$ is also the dimension of the space, we can employ a pseudoscalar $I = \gamma_0 \gamma_1 \cdots \gamma_k$ and can specify our volume element in terms of the Jacobian determinant.

This is
\begin{align}\label{eqn:stokesNoTensor:jacobian}
d^k x =
%da_1 da_2 \cdots da_k
%\PD{(a_1, a_2, \cdots, a_k)}{(x^{\mu_1}, x^{\mu_2}, \cdots, x^{\mu_k})}
%\epsilon_{ \mu_1 \mu_2 \cdots \mu_k } I
%= 
I da_1 da_2 \cdots da_k \Abs{
\PD{(a_1, a_2, \cdots, a_k)}{(x^1, x^2, \cdots, x^k)}
}
\end{align}

However, we won't have a requirement to express the Stokes result in terms of such Jacobians.

\subsection{Expansion of the curl and volume element product}

We are now prepared to go on to the meat of the issue.  The first order of business is the expansion of the curl and volume element product

\begin{align*}
( \grad \wedge F ) \cdot d^k x
&=
( \gamma^\mu \wedge \partial_\mu F ) \cdot d^k x \\
&=
\gpgradezero{ ( \gamma^\mu \wedge \partial_\mu F ) d^k x } \\
\end{align*}

The wedge product within the scalar grade selection operator can be expanded in symmetric or antisymmetric sums, but this is a grade dependent operation.  For odd grade blades $A$ (vector, trivector, ...), and vector $a$ we have for the dot and wedge product respectively

\begin{align*}
a \wedge A = \inv{2} (a A - A a) \\
a \cdot A = \inv{2} (a A + A a)
\end{align*}

Similarly for even grade blades we have

\begin{align*}
a \wedge A = \inv{2} (a A + A a) \\
a \cdot A = \inv{2} (a A - A a)
\end{align*}

First treating the odd grade case for $F$ we have

\begin{align*}
( \grad \wedge F ) \cdot d^k x
&=
\inv{2} \gpgradezero{ \gamma^\mu \partial_\mu F d^k x } - \inv{2} \gpgradezero{ \partial_\mu F \gamma^\mu d^k x } \\
\end{align*}

Employing cyclic scalar reordering within the scalar product for the first term

\begin{align}\label{eqn:stokesNoTensor:100}
\gpgradezero{a b c} = \gpgradezero{b c a}
\end{align}

we have

\begin{align*}
( \grad \wedge F ) \cdot d^k x
&=
\inv{2} \gpgradezero{ \partial_\mu F (d^k x \gamma^\mu - \gamma^\mu d^k x)} \\
&=
\inv{2} \gpgradezero{ \partial_\mu F (d^k x \cdot \gamma^\mu - \gamma^\mu d^k x)} \\
&=
\gpgradezero{ \partial_\mu F (d^k x \cdot \gamma^\mu)} \\
\end{align*}

The end result is 

\begin{align}\label{eqn:stokesNoTensor:startingPoint}
( \grad \wedge F ) \cdot d^k x &= \partial_\mu F \cdot (d^k x \cdot \gamma^\mu) 
\end{align}

For even grade $F$ (and thus odd grade $d^k x$) it is straightforward to show that (\ref{eqn:stokesNoTensor:startingPoint}) also holds.

\subsection{Expanding the volume dot product}

We want to expand the volume integral dot product

\begin{align}\label{eqn:stokesNoTensor:110}
d^k x \cdot \gamma^\mu
\end{align}

Picking $k = 4$ will serve to illustrate the pattern, and the generalization (or degeneralization to lower grades) will be clear.  We have

\begin{align*}
d^4 x \cdot \gamma^\mu
&=
( dx_1 \wedge dx_2 \wedge dx_3 \wedge dx_4 ) \cdot \gamma^\mu \\
&= ( dx_1 \wedge dx_2 \wedge dx_3 ) dx_4 \cdot \gamma^\mu \\
&-( dx_1 \wedge dx_2 \wedge dx_4 ) dx_3 \cdot \gamma^\mu \\
&+( dx_1 \wedge dx_3 \wedge dx_4 ) dx_2 \cdot \gamma^\mu \\
&-( dx_2 \wedge dx_3 \wedge dx_4 ) dx_1 \cdot \gamma^\mu  \\
\end{align*}

This avoids the requirement to do the entire Jacobian expansion of (\ref{eqn:stokesNoTensor:jacobian}).  The dot product of the differential displacement $dx_m$ with $\gamma^\mu$ can now be made explicit without as much mess.

\begin{align*}
dx_m \cdot \gamma^\mu 
&=
da_m \PD{a_m}{x^\nu} \gamma_\nu \cdot \gamma^\mu \\
&=
da_m \PD{a_m}{x^\mu} \\
\end{align*}

We now have products of the form

\begin{align*}
\partial_\mu F da_m \PD{a_m}{x^\mu} 
&=
da_m \PD{a_m}{x^\mu} \PD{x^\mu}{F} \\
&=
da_m \PD{a_m}{F} \\
\end{align*}

Now we see that the differential form of (\ref{eqn:stokesNoTensor:startingPoint}) for this $k=4$ example is reduced to

\begin{align*}
( \grad \wedge F ) \cdot d^4 x 
&= da_4 \PD{a_4}{F} \cdot ( dx_1 \wedge dx_2 \wedge dx_3 ) \\
&- da_3 \PD{a_3}{F} \cdot ( dx_1 \wedge dx_2 \wedge dx_4 ) \\
&+ da_2 \PD{a_2}{F} \cdot ( dx_1 \wedge dx_3 \wedge dx_4 ) \\
&- da_1 \PD{a_1}{F} \cdot ( dx_2 \wedge dx_3 \wedge dx_4 ) \\
\end{align*}

While \ref{eqn:stokesNoTensor:startingPoint} was a statement of Stokes theorem in this Geometric Algebra formulation, it was really incomplete without this explicit expansion of $(\partial_\mu F) \cdot (d^k x \cdot \gamma^\mu)$.  This expansion for the $k=4$ case serves to illustrate that we would write Stokes theorem as

\begin{equation}\label{eqn:stokesNoTensor:190}
\boxed{
\int
( \grad \wedge F ) \cdot d^k x 
=
\inv{(k-1)!} \epsilon^{ r s \cdots t u } \int da_u \PD{a_{u}}{F} \cdot 
(dx_r \wedge dx_s \wedge \cdots \wedge dx_t)
}
\end{equation}
%\left( 
%\PD{a_r}{x} \wedge \PD{a_s}{x} \wedge \cdots \wedge \PD{a_t}{x} \right)
%da_1 da_2 \cdots da_k

Here the indexes have the range $\{r, s, \cdots, t, u\} \in \{1, 2, \cdots k\}$.  This with the definitions \ref{eqn:stokesNoTensor:oneForm}, and \ref{eqn:stokesNoTensor:volumeElement} is really Stokes theorem in its full glory.

Observe that in this Geometric algebra form, the one forms $dx_i = da_i \PDi{a_i}{x}, i \in [1,k]$ are nothing more abstract that plain old vector differential elements.  In the formalism of differential forms, this would be vectors, and $(\grad \wedge F) \cdot d^k x$ would be a $k$ form.  In a context where we are working with vectors, or blades already, the Geometric Algebra statement of the theorem avoids a requirement to translate to the language of forms.

With a statement of the general theorem complete, let's return to our $k=4$ case where we can now integrate over each of the $a_1, a_2, \cdots, a_k$ parameters.  That is

\begin{align*}
\int ( \grad \wedge F ) \cdot d^4 x 
&= \int (F(a_4(1)) - F(a_4(0))) \cdot ( dx_1 \wedge dx_2 \wedge dx_3 ) \\
&- \int (F(a_3(1)) - F(a_3(0))) \cdot ( dx_1 \wedge dx_2 \wedge dx_4 ) \\
&+ \int (F(a_2(1)) - F(a_2(0))) \cdot ( dx_1 \wedge dx_3 \wedge dx_4 ) \\
&- \int (F(a_1(1)) - F(a_1(0))) \cdot ( dx_2 \wedge dx_3 \wedge dx_4 ) \\
\end{align*}

This is precisely Stokes theorem for the trivector case and makes the enumeration of the boundary surfaces explicit.  As derived there was no requirement for an orthonormal basis, nor a Euclidean metric, nor a parametrization along the basis directions.  The only requirement of the parametrization is that the associated volume element is non-trivial (i.e. none of $dx_q \wedge dx_r = 0$).  %The issue of how to extend this from the hyper-parallelepiped volume element to a general has been skipped, and should perhaps be thought through more carefully.

For completeness, note that our boundary surface and associated Stokes statement for the bivector and vector cases is, by inspection respectively

\begin{align*}
\int ( \grad \wedge F ) \cdot d^3 x 
&= \int (F(a_3(1)) - F(a_3(0))) \cdot ( dx_1 \wedge dx_2 ) \\
&- \int (F(a_2(1)) - F(a_2(0))) \cdot ( dx_1 \wedge dx_3 ) \\
&+ \int (F(a_1(1)) - F(a_1(0))) \cdot ( dx_2 \wedge dx_3 ) \\
\end{align*}

and
\begin{align*}
\int ( \grad \wedge F ) \cdot d^2 x 
&= \int (F(a_2(1)) - F(a_2(0))) \cdot dx_1 \\
&- \int (F(a_1(1)) - F(a_1(0))) \cdot dx_2 \\
\end{align*}

These three expansions can be summarized by the original single statement of (\ref{eqn:stokesNoTensor:stokes}), which repeating for reference, is

\begin{align*}
\int ( \grad \wedge F ) \cdot d^k x = \int F \cdot d^{k-1} x 
\end{align*}

Where it is implied that the blade $F$ is evaluated on the boundaries and dotted with the associated hypersurface boundary element.  However, having expanded this we now have an explicit statement of exactly what that surface element is now for any desired parametrization.

\section{Duality relations and special cases}

Some special (and more recognizable) cases of (\ref{eqn:stokesNoTensor:stokes}) are possible considering specific grades of $F$, and in some cases employing duality relations.  

\subsection{curl surface integral}

One important case is the \R{3} vector result, which can be expressed in terms of the cross product.

Write $\ncap d^2 x = -i dA$.  Then we have

\begin{align*}
( \spacegrad \wedge \Bf ) \cdot d^2 x
&=
\gpgradezero{ i (\spacegrad \cross \Bf) (- \ncap i dA) } \\
&=
(\spacegrad \cross \Bf) \cdot \ncap dA
\end{align*}

This recovers the familiar cross product form of Stokes law.

\begin{align}\label{eqn:stokesNoTensor:120}
\int (\spacegrad \cross \Bf) \cdot \ncap dA = \ointclockwise \Bf \cdot d\Bx
\end{align}

\subsection{3D divergence theorem}

Duality applied to the bivector Stokes result provides the divergence theorem in \R{3}.  For bivector $B$, let $iB = \Bf$, $d^3 x = i dV$, and $d^2 x = i \ncap dA$.  We then have

\begin{align*}
( \spacegrad \wedge B ) \cdot d^3 x
&=
\gpgradezero{ ( \spacegrad \wedge B ) \cdot d^3 x } \\
&=
\inv{2} \gpgradezero{ ( \spacegrad B + B \spacegrad ) i dV } \\
&=
\spacegrad \cdot \Bf dV \\
\end{align*}

Similarly

\begin{align*}
B \cdot d^2 x
&=
\gpgradezero{ 
-i\Bf i \ncap dA
} \\
&=
(\Bf \cdot \ncap) dA
 \\
\end{align*}

This recovers the \R{3} divergence equation

\begin{align}\label{eqn:stokesNoTensor:130}
\int \spacegrad \cdot \Bf dV = \int (\Bf \cdot \ncap) dA
\end{align}

\subsection{4D divergence theorem}

How about the four dimensional spacetime divergence?  Write, express a trivector as a dual four-vector $T = if$, and the four volume element $d^4 x = i dQ$.  This gives

\begin{align*}
(\grad \wedge T) \cdot d^4 x
&=
\inv{2} \gpgradezero{ (\grad T - T \grad) i } dQ \\
&=
\inv{2} \gpgradezero{ (\grad i f - if \grad) i } dQ \\
&=
\frac{1}{2} \gpgradezero{ (\grad f + f \grad) } dQ \\
&=
(\grad \cdot f) dQ
\end{align*}

For the boundary volume integral write $d^3 x = n i dV$, for
\begin{align*}
T \cdot d^3 x 
&= 
\gpgradezero{ (if) ( n i ) } dV \\
&= 
\gpgradezero{ f n } dV \\
&= 
(f \cdot n) dV
\end{align*}

So we have

\begin{align*}
\int \partial_\mu f^\mu dQ = \int f^\nu n_\nu dV
\end{align*}

the orientation of the fourspace volume element and the boundary normal is defined in terms of the parametrization, the duality relations and our explicit expansion of the 4D stokes boundary integral above.

\subsection{4D divergence theorem, continued}

The basic idea of using duality to express the 4D divergence integral as a stokes boundary surface integral has been explored.  Lets consider this in more detail picking a specific parametrization, namely rectangular four vector coordinates.  For the volume element write

\begin{align*}
d^4 x 
&= ( \gamma_0 dx^0 ) \wedge ( \gamma_1 dx^1 ) \wedge ( \gamma_2 dx^2 ) \wedge ( \gamma_3 dx^3 ) \\
&= \gamma_0 \gamma_1 \gamma_2 \gamma_3 dx^0 dx^1 dx^2 dx^3 \\
&= i dx^0 dx^1 dx^2 dx^3 \\
\end{align*}

As seen previously (but not separately), the divergence can be expressed as the dual of the curl

\begin{align*}
\grad \cdot f
&=
\gpgradezero{ \grad f } \\
&=
-\gpgradezero{ \grad i (\underbrace{i f}_{\text{grade 3}}) } \\
&=
\gpgradezero{ i \grad (i f) } \\
%&=
%\gpgradezero{ i ( \grad \cdot (i f) + \grad \wedge (i f) ) } \\
&=
\gpgradezero{ i ( \underbrace{\grad \cdot (i f)}_{\text{grade 2}} + \underbrace{\grad \wedge (i f)}_{\text{grade 4}} ) } \\
&=
i (\grad \wedge (i f)) \\
\end{align*}

So we have $\grad \wedge (i f) = -i (\grad \cdot f)$.  Putting things together, and writing $i f = -f i$ we have 

\begin{align*}
\int (\grad \wedge (i f)) \cdot d^4 x
&= 
\int (\grad \cdot f) dx^0 dx^1 dx^2 dx^3 \\
&=
\int dx^0 \partial_0 (f i) \cdot \gamma_{123} dx^1 dx^2 dx^3 \\
&-\int dx^1 \partial_1 (f i) \cdot \gamma_{023} dx^0 dx^2 dx^3 \\
&+\int dx^2 \partial_2 (f i) \cdot \gamma_{013} dx^0 dx^1 dx^3 \\
&-\int dx^3 \partial_3 (f i) \cdot \gamma_{012} dx^0 dx^1 dx^2 \\
\end{align*}

It is straightforward to reduce each of these dot products.  For example

\begin{align*}
\partial_2 (f i) \cdot \gamma_{013}
&=
\gpgradezero{ \partial_2 f \gamma_{0123013} } \\
&=
-\gpgradezero{ \partial_2 f \gamma_{2} } \\
&=
- \gamma_2 \partial_2 \cdot f \\
&=
\gamma^2 \partial_2 \cdot f 
\end{align*}

The rest proceed the same and rather anticlimactically we end up coming full circle 

\begin{align*}
\int (\grad \cdot f) dx^0 dx^1 dx^2 dx^3 
&=
\int dx^0 \gamma^0 \partial_0 \cdot f dx^1 dx^2 dx^3 \\
&+\int dx^1 \gamma^1 \partial_1 \cdot f dx^0 dx^2 dx^3 \\
&+\int dx^2 \gamma^2 \partial_2 \cdot f dx^0 dx^1 dx^3 \\
&+\int dx^3 \gamma^3 \partial_3 \cdot f dx^0 dx^1 dx^2 \\
\end{align*}

This is however nothing more than the definition of the divergence itself and no need to resort to Stokes theorem is required.  However, if we are integrating over a rectangle and perform each of the four integrals, we have (with $c=1$) from the dual Stokes equation the perhaps less obvious result

\begin{align*}
\int \partial_\mu f^\mu dt dx dy dz
&=
\int (f^0(t_1) - f^0(t_0)) dx dy dz \\
&+\int (f^1(x_1) - f^1(x_0)) dt dy dz \\
&+\int (f^2(y_1) - f^2(y_0)) dt dx dz \\
&+\int (f^3(z_1) - f^3(z_0)) dt dx dy \\
\end{align*}

When stated this way one sees that this could have just as easily have followed directly from the left hand side.  What's the point then of the divergence theorem or Stokes theorem?  I think that the value must really be the fact that the Stokes formulation naturally builds the volume element in a fashion independent of any specific parametrization.  Here in rectangular coordinates the result seems obvious, but would the equivalent result seem obvious if non-rectangular spacetime coordinates were employed?  Probably not.

%\EndArticle

\part{General Physics.}
%
% Copyright � 2012 Peeter Joot.  All Rights Reserved.
% Licenced as described in the file LICENSE under the root directory of this GIT repository.
%

%
%
\chapter{Angular Velocity and Acceleration (Again)}\label{chap:PJAngAcc}
\index{angular velocity}
\index{angular acceleration}
%\date{June 10, 2008.  angularAcc.tex}

A more coherent derivation of angular velocity and acceleration than
my initial attempt while first learning geometric algebra.

\section{Angular velocity}

The goal is to take first and second derivatives of a vector expressed radially:

\begin{equation}
\Br = r \rcap.
\end{equation}

The velocity is the derivative of our position vector, which in terms of radial components is:

\begin{equation}\label{eqn:angular_acc:velocityasrcapprime}
\Bv = \Br' = r' \rcap + r \rcap'.
\end{equation}

We can also calculate the projection and rejection of the velocity by multiplication by \(1 = \rcap^2\), and expanding
this product in an alternate order taking advantage of the associativity of the geometric product:

\begin{equation}\label{eqn:angularAcc:380}
\begin{aligned}
\Bv &= \rcap \rcap \Bv \\
    &= \rcap \left ( \rcap \cdot \Bv + \rcap \wedge \Bv \right) \\
\end{aligned}
\end{equation}

Since \(\rcap \wedge (\rcap \wedge \Bv) = 0\), the total velocity in terms of radial components is:

\begin{equation}\label{eqn:angular_acc:velocityprojrej}
\Bv = \rcap \left(\rcap \cdot \Bv\right) + \rcap \cdot \left(\rcap \wedge \Bv\right).
\end{equation}

Here the first component above is the projection of the vector in the radial direction:

\begin{equation}\label{eqn:angularAcc:20}
\Proj_{\Br}(\Bv) = \rcap \left(\rcap \cdot \Bv\right)
\end{equation}

This projective term can also be rewritten in terms of magnitude:

\begin{equation}\label{eqn:angularAcc:40}
\left(r^2\right)' = 2 r r' = \left(\Br \cdot \Br\right)' = 2 \Br \cdot \Bv.
\end{equation}

So the magnitude variation can be expressed the radial coordinate of the velocity:

\begin{equation}\label{eqn:angular_acc:rprime}
r' = \rcap \cdot \Bv
\end{equation}

The remainder is the rejection of the radial component from the velocity, leaving just the part
portion perpendicular to the radial direction.

\begin{equation}\label{eqn:angularAcc:60}
\RejName_{\Br}(\Bv) = \rcap \cdot \left(\rcap \wedge \Bv\right)
\end{equation}

It is traditional to introduce an angular velocity vector normal to the plane of rotation
that describes this rejective component using a triple cross product.  With the formulation above,
one can see that it is more natural to directly use an angular velocity bivector instead:

\begin{equation}
\BOmega = \frac{\Br \wedge \Bv}{r^2}
\end{equation}

This bivector encodes the
angular velocity as a plane directly.  The
product of a vector with the bivector that contains it produces another vector
in the plane.  That product is a scaled and rotated by 90 degrees, much like the
multiplication by a unit complex imaginary.  That is no coincidence since
the square of a bivector is negative and directly encodes this complex structure
of an arbitrarily oriented plane.

Using this angular velocity bivector we have the following radial expression for velocity:

\begin{equation}\label{eqn:angular_acc:velocityomega}
\Bv = \rcap r' + \Br \cdot \BOmega.
\end{equation}

A little thought will show that \(\rcap'\) is also entirely perpendicular to \(\rcap\).  The \(\rcap\) vector describes
a path traced out on the unit sphere, and any variation of that vector must be tangential to the sphere.
It is thus not surprising that we can also express \(\rcap'\) using the rejective term of equation
\eqnref{eqn:angular_acc:velocityprojrej}.  Using the angular velocity bivector this is:

\begin{equation}\label{eqn:angular_acc:rcapprime}
\rcap' = \rcap \cdot \BOmega.
\end{equation}

This identity will be useful below for the calculation of angular acceleration.

\section{Angular acceleration}

Next we want the second derivatives of position

\begin{equation}\label{eqn:angularAcc:400}
\begin{aligned}
\Ba
&= \Br'' \\
&= r'' \rcap + 2r' \rcap' + r \rcap'' \\
&= r'' \rcap + \inv{r}\left( r^2 \rcap' \right)' \\
\end{aligned}
\end{equation}

This last step I found scribbled in a margin note in
my old mechanics book.  It is a trick that somebody clever once noticed and it simplifies this derivation to use it
since it avoids the generation of a number of terms that will just cancel out anyways after more tedious manipulation
(see examples section).

Expanding just this last derivative:

\begin{equation}\label{eqn:angularAcc:420}
\begin{aligned}
\left( r^2 \rcap' \right)'
&= \left( r^2 \rcap \cdot \BOmega \right)' \\
&= \left( \rcap \cdot \left(\Br \wedge \Bv\right) \right)' \\
&= \left( \rcap \cdot \left(\Br \wedge \Bv\right) \right)' \\
&= \rcap' \cdot \left(\Br \wedge \Bv\right) +\rcap \cdot (\mathLabelBox{\Bv \wedge \Bv}{\(=0\)}) + \rcap \cdot \left(\Br \wedge \Ba\right) \\
\end{aligned}
\end{equation}

Thus the acceleration is:
\begin{equation*}
\Ba = r'' \rcap + \left(\Br \cdot \BOmega\right) \cdot \BOmega + \rcap \cdot \left(\rcap \wedge \Ba\right)
\end{equation*}


Note that the action of taking two successive dot products with the plane bivector \(\BOmega\) just acts to rotate the
vector by 180 degrees (as well as scale it).

One can verify this explicitly using grade selection operators.  This allows the total acceleration to be expressed
in the final form:

\begin{equation*}
\Ba = r'' \rcap + \Br \BOmega^2 + \rcap \cdot \left(\rcap \wedge \Ba\right)
\end{equation*}

Note that the squared bivector \(\BOmega^2\) is a negative scalar, so the first two terms are radially directed.
The last term is perpendicular to the acceleration, in the plane formed by the vector and its second derivative.

Given the acceleration, the force on a particle is thus:

\begin{equation}\label{eqn:angularAcc:80}
\BF = m\Ba = m\rcap r'' + m \Br \BOmega^2 + \frac{\Br}{r^2} \left(\Br \wedge \Bp\right)'
\end{equation}

Writing the angular momentum as:

\begin{equation}\label{eqn:angularAcc:100}
\BL = \Br \wedge \Bp = m r^2 \BOmega
\end{equation}
%m \BOmega^2 = \BL^2/m r^4

the force is thus:

\begin{equation}\label{eqn:angularAcc:120}
\BF = m\Ba = m\rcap r'' + m \Br \BOmega^2 + \frac{1}{\Br} \cdot \frac{d\BL}{dt}
\end{equation}

The derivative of the angular momentum is called the torque \(\Btau\), also a bivector:

\begin{equation}\label{eqn:angularAcc:140}
\Btau = \frac{d\BL}{dt}
\end{equation}


When \(\Br\) is constant this has the radial arm times force form that we expect of torque:

\begin{equation}\label{eqn:angularAcc:160}
\Btau = \Br \wedge \frac{d \Bp}{dt} = \Br \wedge \BF
\end{equation}

%We can also write the equation of motion in terms of angular momentum and torque, in which case we have:
%
%\[
%\BF = m\rcap r'' + \inv{m\Br^3} \BL^2 + \frac{1}{\Br} \cdot \Btau
%\]
We can also write the equation of motion in terms of torque, in which case we have:

\begin{equation}\label{eqn:angularAcc:180}
\BF = m\rcap r'' + m \Br \Omega^2 + \frac{1}{\Br} \cdot \Btau
\end{equation}

As with all these plane quantities (angular velocity, momentum, acceleration), the torque as well is a bivector as it is natural to express this as a planar quantity.  This makes
more sense in many ways than a cross product, since all of these quantities should be perfectly well defined in a plane (or in spaces of degree greater than three), whereas the
cross product is a strictly three dimensional entity.

\section{Expressing these using traditional form (cross product)}

To compare with traditional results to see if I got things right, remove the geometric algebra constructs
(wedge products and bivector/vector products) in favor of cross products.  Do this by
using the duality relationships, multiplication by the three dimensional pseudoscalar
\(i = \Be_1\Be_2\Be_3\), to convert bivectors to vectors and wedge products to cross and dot products
(\(\Bu \wedge \Bv = \Bu \cross \Bv i\)).

First define some vector quantities in terms of the corresponding bivectors:

\begin{equation}\label{eqn:angularAcc:200}
\Bomega = \BOmega / i = \frac{\Br \wedge \Bv}{r^2 i} = \frac{\Br \cross \Bv}{r^2}
\end{equation}

\begin{equation}\label{eqn:angularAcc:220}
\Br \cdot \BOmega = \inv{2}(\Br \Bomega i - \Bomega i \Br ) = \Br \wedge \Bomega i = \Bomega \cross \Br
\end{equation}

Thus the velocity is:

\begin{equation}\label{eqn:angularAcc:240}
\Bv = \rcap r' + \Bomega \cross \Br.
\end{equation}

In the same way, write the angular momentum vector as the dual of the angular momentum bivector:

\begin{equation}\label{eqn:angularAcc:260}
\Bl = \BL /i = \Br \cross \Bp = m r^2 \Bomega
\end{equation}

And the torque vector \(\BN\) as the dual of the torque bivector \(\Btau\)

\begin{equation}\label{eqn:angularAcc:280}
\BN = \Btau /i = \frac{d\Bl}{dt} = \frac{d}{dt} \left(\Br \cross \Bp \right)
\end{equation}

The equation of motion for a single particle then becomes:

% r . t = r . N i = 1/2(r N i - N i r) = r ^ N i = r x N i^2 = N x r
\begin{equation}\label{eqn:angularAcc:300}
\BF = m\rcap r'' - m \Br \norm{\Bomega}^2 + \BN \cross \frac{\Br}{r^2}
\end{equation}

\section{Examples (perhaps future exercises?)}

\subsection{Unit vector derivative}
\index{derivative!unit vector}

Demonstrate by direct calculation the result of \eqnref{eqn:angular_acc:rcapprime}.

\begin{equation}\label{eqn:angularAcc:440}
\begin{aligned}
\rcap'
&= \left(\frac{\Br}{r}\right)' \\
&= \frac{\Br'}{r} - \frac{\Br r'}{r^2} \\
&= \inv{r} \left( \Bv - \rcap \left(\rcap \cdot \Bv \right) \right) \\
&= \frac{\rcap}{r} \left( \rcap \Bv - \rcap \cdot \Bv \right) \\
&= \frac{\rcap}{r} \left( \rcap \wedge \Bv \right) \\
\end{aligned}
\end{equation}

\subsection{Direct calculation of acceleration}
\index{acceleration}

It is more natural to calculate this acceleration directly by taking derivatives of \eqnref{eqn:angular_acc:velocityomega}, but as noted above this is messier.  Here is exactly that calculation for
comparison.

Taking second derivatives of the velocity we have:

\begin{equation}\label{eqn:angularAcc:320}
\Bv' = \Ba = \left(\rcap r' + \frac{\Br}{r^2} \left(\Br \wedge \Bv\right)\right)'
\end{equation}

%\rcap' = \inv{r^3} \Br \Br \wedge \Bv
\begin{equation}\label{eqn:angularAcc:460}
\begin{aligned}
\Ba
&= \rcap' r' + \rcap r'' + \frac{\Br}{r^2}
\mathLabelBox
[
   labelstyle={below of=m\themathLableNode, below of=m\themathLableNode}
]
{\left(\Bv \wedge \Bv\right)}{\(=0\)} + \frac{\Br}{r^2} \left(\Br \wedge \Ba\right) + \left(\frac{\rcap}{r}\right)' \left(\Br \wedge \Bv\right) \\
&=
\rcap r''
+\rcap'\left( r'  + \frac{1}{r} \Br \wedge \Bv \right)
- r' \frac{\rcap}{r^2} \left(\Br \wedge \Bv\right)
+ \rcap \left(\rcap \wedge \Ba\right)  \\
&=
\rcap r''
+\inv{r^3} \Br\left( \Br \wedge \Bv\right) \left( r'  + \frac{1}{r} \Br \wedge \Bv \right)
- r' \frac{\rcap}{r^2} \left(\Br \wedge \Bv\right)
+ \rcap \left(\rcap \wedge \Ba\right)  \\
\end{aligned}
\end{equation}

The \(r'\) terms cancel out, leaving just:

\begin{equation}\label{eqn:angularAcc:340}
\Ba = \rcap r'' + \Br \BOmega^2 +
\rcap \left(\rcap \wedge \Ba\right)
\end{equation}

\subsection{Expand the omega omega triple product}

\begin{equation}\label{eqn:angularAcc:480}
\begin{aligned}
\left(\Br \cdot \BOmega\right) \cdot \BOmega
&= \gpgradeone{ \left(\Br \cdot \BOmega\right) \BOmega } \\
&= \inv{2} \gpgradeone{ \Br \BOmega^2 - \BOmega \Br \BOmega } \\
&= \inv{2} \Br \BOmega^2 - \inv{2}\gpgradeone{ \BOmega \Br \BOmega } \\
&= \inv{2} \Br \BOmega^2 + \inv{2}\gpgradeone{ \Br \BOmega \BOmega } \\
&= \inv{2} \Br \BOmega^2 + \inv{2}\Br \BOmega^2 \\
&= \Br \BOmega^2 \\
\end{aligned}
\end{equation}

Also used above implicitly was the following:

\begin{equation}\label{eqn:angularAcc:360}
\Br \BOmega = \Br \cdot \BOmega + \mathLabelBox{\Br \wedge \BOmega}{\(=0\)} = -\BOmega \cdot \Br = -\BOmega \Br
\end{equation}

(ie: a vector anticommutes with a bivector describing a plane that contains it).

\documentclass{article}      % Specifies the document class

\usepackage{amsmath}
\usepackage{mathpazo}

%
% shorthand for bold symbols, convenient for vectors and matrices
%
\newcommand{\Ba}[0]{\mathbf{a}}
\newcommand{\Bb}[0]{\mathbf{b}}
\newcommand{\Bc}[0]{\mathbf{c}}
\newcommand{\Bd}[0]{\mathbf{d}}
\newcommand{\Be}[0]{\mathbf{e}}
\newcommand{\Bf}[0]{\mathbf{f}}
\newcommand{\Bg}[0]{\mathbf{g}}
\newcommand{\Bh}[0]{\mathbf{h}}
\newcommand{\Bi}[0]{\mathbf{i}}
\newcommand{\Bj}[0]{\mathbf{j}}
\newcommand{\Bk}[0]{\mathbf{k}}
\newcommand{\Bl}[0]{\mathbf{l}}
\newcommand{\Bm}[0]{\mathbf{m}}
\newcommand{\Bn}[0]{\mathbf{n}}
\newcommand{\Bo}[0]{\mathbf{o}}
\newcommand{\Bp}[0]{\mathbf{p}}
\newcommand{\Bq}[0]{\mathbf{q}}
\newcommand{\Br}[0]{\mathbf{r}}
\newcommand{\Bs}[0]{\mathbf{s}}
\newcommand{\Bt}[0]{\mathbf{t}}
\newcommand{\Bu}[0]{\mathbf{u}}
\newcommand{\Bv}[0]{\mathbf{v}}
\newcommand{\Bw}[0]{\mathbf{w}}
\newcommand{\Bx}[0]{\mathbf{x}}
\newcommand{\By}[0]{\mathbf{y}}
\newcommand{\Bz}[0]{\mathbf{z}}
\newcommand{\BA}[0]{\mathbf{A}}
\newcommand{\BB}[0]{\mathbf{B}}
\newcommand{\BC}[0]{\mathbf{C}}
\newcommand{\BD}[0]{\mathbf{D}}
\newcommand{\BE}[0]{\mathbf{E}}
\newcommand{\BF}[0]{\mathbf{F}}
\newcommand{\BG}[0]{\mathbf{G}}
\newcommand{\BH}[0]{\mathbf{H}}
\newcommand{\BI}[0]{\mathbf{I}}
\newcommand{\BJ}[0]{\mathbf{J}}
\newcommand{\BK}[0]{\mathbf{K}}
\newcommand{\BL}[0]{\mathbf{L}}
\newcommand{\BM}[0]{\mathbf{M}}
\newcommand{\BN}[0]{\mathbf{N}}
\newcommand{\BO}[0]{\mathbf{O}}
\newcommand{\BP}[0]{\mathbf{P}}
\newcommand{\BQ}[0]{\mathbf{Q}}
\newcommand{\BR}[0]{\mathbf{R}}
\newcommand{\BS}[0]{\mathbf{S}}
\newcommand{\BT}[0]{\mathbf{T}}
\newcommand{\BU}[0]{\mathbf{U}}
\newcommand{\BV}[0]{\mathbf{V}}
\newcommand{\BW}[0]{\mathbf{W}}
\newcommand{\BX}[0]{\mathbf{X}}
\newcommand{\BY}[0]{\mathbf{Y}}
\newcommand{\BZ}[0]{\mathbf{Z}}

\newcommand{\Bzero}[0]{\mathbf{0}}
\newcommand{\Btheta}[0]{\boldsymbol{\theta}}
\newcommand{\Btau}[0]{\boldsymbol{\tau}}
\newcommand{\Bomega}[0]{\boldsymbol{\omega}}

%
% shorthand for unit vectors
%
\newcommand{\acap}[0]{\hat{\Ba}}
\newcommand{\bcap}[0]{\hat{\Bb}}
\newcommand{\ccap}[0]{\hat{\Bc}}
\newcommand{\dcap}[0]{\hat{\Bd}}
\newcommand{\ecap}[0]{\hat{\Be}}
\newcommand{\fcap}[0]{\hat{\Bf}}
\newcommand{\gcap}[0]{\hat{\Bg}}
\newcommand{\hcap}[0]{\hat{\Bh}}
\newcommand{\icap}[0]{\hat{\Bi}}
\newcommand{\jcap}[0]{\hat{\Bj}}
\newcommand{\kcap}[0]{\hat{\Bk}}
\newcommand{\lcap}[0]{\hat{\Bl}}
\newcommand{\mcap}[0]{\hat{\Bm}}
\newcommand{\ncap}[0]{\hat{\Bn}}
\newcommand{\ocap}[0]{\hat{\Bo}}
\newcommand{\pcap}[0]{\hat{\Bp}}
\newcommand{\qcap}[0]{\hat{\Bq}}
\newcommand{\rcap}[0]{\hat{\Br}}
\newcommand{\scap}[0]{\hat{\Bs}}
\newcommand{\tcap}[0]{\hat{\Bt}}
\newcommand{\ucap}[0]{\hat{\Bu}}
\newcommand{\vcap}[0]{\hat{\Bv}}
\newcommand{\wcap}[0]{\hat{\Bw}}
\newcommand{\xcap}[0]{\hat{\Bx}}
\newcommand{\ycap}[0]{\hat{\By}}
\newcommand{\zcap}[0]{\hat{\Bz}}
\newcommand{\thetacap}[0]{\hat{\Btheta}}

%
% to write R^n and C^n in a distinguishable fashion.  Perhaps change this
% to the double lined characters upon figuring out how to do so.
%
\newcommand{\C}[1]{$\mathbb{C}^{#1}$}
\newcommand{\R}[1]{$\mathbb{R}^{#1}$}

%
% various generally useful helpers
%

% derivative of #1 wrt. #2:
\newcommand{\D}[2] {\frac {d#2} {d#1}}

\newcommand{\inv}[1]{\frac{1}{#1}}
\newcommand{\cross}[0]{\times}

\newcommand{\abs}[1]{\lvert{#1}\rvert}
\newcommand{\norm}[1]{\lVert{#1}\rVert}
\newcommand{\innerprod}[2]{\langle{#1}, {#2}\rangle}
\newcommand{\dotprod}[2]{{#1} \cdot {#2}}
\newcommand{\bdotprod}[2]{\left({#1} \cdot {#2}\right)}
\newcommand{\crossprod}[2]{{#1} \cross {#2}}
\newcommand{\tripleprod}[3]{\dotprod{\left(\crossprod{#1}{#2}\right)}{#3}}

\DeclareMathOperator{\Proj}{Proj}
\DeclareMathOperator{\Span}{span}
\DeclareMathOperator{\Sgn}{sgn}
\DeclareMathOperator{\Area}{Area}
\DeclareMathOperator{\Volume}{Volume}

%
% A few miscellaneous things specific to this document
%
\newcommand{\crossop}[1]{\crossprod{#1}{}}

% R2 vector.
\newcommand{\VectorTwo}[2]{
\begin{bmatrix}
 {#1} \\
 {#2}
\end{bmatrix}
}

\newcommand{\VectorN}[1]{
\begin{bmatrix}
{#1}_1 \\
{#1}_2 \\
\vdots \\
{#1}_N \\
\end{bmatrix}
}

\newcommand{\DETuvij}[4]{
\begin{vmatrix}
 {#1}_{#3} & {#1}_{#4} \\
 {#2}_{#3} & {#2}_{#4}
\end{vmatrix}
}

\newcommand{\DETuvwijk}[6]{
\begin{vmatrix}
 {#1}_{#4} & {#1}_{#5} & {#1}_{#6} \\
 {#2}_{#4} & {#2}_{#5} & {#2}_{#6} \\
 {#3}_{#4} & {#3}_{#5} & {#3}_{#6}
\end{vmatrix}
}

\newcommand{\DETuvwxijkl}[8]{
\begin{vmatrix}
 {#1}_{#5} & {#1}_{#6} & {#1}_{#7} & {#1}_{#8} \\
 {#2}_{#5} & {#2}_{#6} & {#2}_{#7} & {#2}_{#8} \\
 {#3}_{#5} & {#3}_{#6} & {#3}_{#7} & {#3}_{#8} \\
 {#4}_{#5} & {#4}_{#6} & {#4}_{#7} & {#4}_{#8} \\
\end{vmatrix}
}

%\newcommand{\DETuvwxyijklm}[10]{
%\begin{vmatrix}
% {#1}_{#6} & {#1}_{#7} & {#1}_{#8} & {#1}_{#9} & {#1}_{#10} \\
% {#2}_{#6} & {#2}_{#7} & {#2}_{#8} & {#2}_{#9} & {#2}_{#10} \\
% {#3}_{#6} & {#3}_{#7} & {#3}_{#8} & {#3}_{#9} & {#3}_{#10} \\
% {#4}_{#6} & {#4}_{#7} & {#4}_{#8} & {#4}_{#9} & {#4}_{#10} \\
% {#5}_{#6} & {#5}_{#7} & {#5}_{#8} & {#5}_{#9} & {#5}_{#10}
%\end{vmatrix}
%}

% R3 vector.
\newcommand{\VectorThree}[3]{
\begin{bmatrix}
 {#1} \\
 {#2} \\
 {#3}
\end{bmatrix}
}



%
% The real thing:
%

                             % The preamble begins here.
\title{ Radial decomposition of angular velocity and angular velocity.} % Declares the document's title.
\author{Peeter Joot}         % Declares the author's name.
%\date{}        % Deleting this command produces today's date.

\begin{document}             % End of preamble and beginning of text.

\maketitle{}

\section{}

Starting point is taking derivatives of:

\[
\mathbf{r} = r \rcap
\]

\[
\mathbf{v} = r' \rcap + r \rcap'
\]

It can be shown (see for example, Salus and Hille, "Calculus") that the unit vector derivative can be expressed using the cross product:

\[
\rcap' = \inv{r} (\rcap \cross \frac{d\mathbf{r}}{dt}) \cross \rcap..
\]

Now, one can express $r'$ in terms of $\mathbf{r}$ as well as follows:

\[
\left(\mathbf{r} \cdot \mathbf{r}\right)' = 2 \mathbf{v} \cdot \mathbf{r} = 2 r r'.
\]

Thus the derivative of the vector magnitude is part of a projective term:

\[
r' = \rcap \cdot \mathbf{v}.
\]

Putting this together one has velocity in terms of projective and rejective
components along a radial direction:

\[
\mathbf{v} = \left(\rcap \cdot \mathbf{v}\right) \rcap + \left(\rcap \cross \frac{d\mathbf{r}}{dt}\right) \cross \rcap.
\]

Now $\boldsymbol{\omega} = \frac{\mathbf{r} \cross \mathbf{v}}{r^2}$ term is what we call the angular velocity.  The magnitude of this
is the rate of change of the angle between the radial arm and the direction of rotation.  The direction of this
cross product is normal to the plane of rotation and encodes both the rotational plane and the direction of the
rotation.  Putting these together one has the total velocity expressed radially:

\[
\mathbf{v} = \left(\rcap \cdot \mathbf{v}\right) \rcap + \boldsymbol{\omega} \cross \mathbf{r}.
\]

\end{document}               % End of document.

\documentclass{article}      % Specifies the document class

\usepackage{amsmath}
\usepackage{mathpazo}

%
% shorthand for bold symbols, convenient for vectors and matrices
%
\newcommand{\Ba}[0]{\mathbf{a}}
\newcommand{\Bb}[0]{\mathbf{b}}
\newcommand{\Bc}[0]{\mathbf{c}}
\newcommand{\Bd}[0]{\mathbf{d}}
\newcommand{\Be}[0]{\mathbf{e}}
\newcommand{\Bf}[0]{\mathbf{f}}
\newcommand{\Bg}[0]{\mathbf{g}}
\newcommand{\Bh}[0]{\mathbf{h}}
\newcommand{\Bi}[0]{\mathbf{i}}
\newcommand{\Bj}[0]{\mathbf{j}}
\newcommand{\Bk}[0]{\mathbf{k}}
\newcommand{\Bl}[0]{\mathbf{l}}
\newcommand{\Bm}[0]{\mathbf{m}}
\newcommand{\Bn}[0]{\mathbf{n}}
\newcommand{\Bo}[0]{\mathbf{o}}
\newcommand{\Bp}[0]{\mathbf{p}}
\newcommand{\Bq}[0]{\mathbf{q}}
\newcommand{\Br}[0]{\mathbf{r}}
\newcommand{\Bs}[0]{\mathbf{s}}
\newcommand{\Bt}[0]{\mathbf{t}}
\newcommand{\Bu}[0]{\mathbf{u}}
\newcommand{\Bv}[0]{\mathbf{v}}
\newcommand{\Bw}[0]{\mathbf{w}}
\newcommand{\Bx}[0]{\mathbf{x}}
\newcommand{\By}[0]{\mathbf{y}}
\newcommand{\Bz}[0]{\mathbf{z}}
\newcommand{\BA}[0]{\mathbf{A}}
\newcommand{\BB}[0]{\mathbf{B}}
\newcommand{\BC}[0]{\mathbf{C}}
\newcommand{\BD}[0]{\mathbf{D}}
\newcommand{\BE}[0]{\mathbf{E}}
\newcommand{\BF}[0]{\mathbf{F}}
\newcommand{\BG}[0]{\mathbf{G}}
\newcommand{\BH}[0]{\mathbf{H}}
\newcommand{\BI}[0]{\mathbf{I}}
\newcommand{\BJ}[0]{\mathbf{J}}
\newcommand{\BK}[0]{\mathbf{K}}
\newcommand{\BL}[0]{\mathbf{L}}
\newcommand{\BM}[0]{\mathbf{M}}
\newcommand{\BN}[0]{\mathbf{N}}
\newcommand{\BO}[0]{\mathbf{O}}
\newcommand{\BP}[0]{\mathbf{P}}
\newcommand{\BQ}[0]{\mathbf{Q}}
\newcommand{\BR}[0]{\mathbf{R}}
\newcommand{\BS}[0]{\mathbf{S}}
\newcommand{\BT}[0]{\mathbf{T}}
\newcommand{\BU}[0]{\mathbf{U}}
\newcommand{\BV}[0]{\mathbf{V}}
\newcommand{\BW}[0]{\mathbf{W}}
\newcommand{\BX}[0]{\mathbf{X}}
\newcommand{\BY}[0]{\mathbf{Y}}
\newcommand{\BZ}[0]{\mathbf{Z}}

\newcommand{\Bzero}[0]{\mathbf{0}}
\newcommand{\Btheta}[0]{\boldsymbol{\theta}}
\newcommand{\Btau}[0]{\boldsymbol{\tau}}
\newcommand{\Bomega}[0]{\boldsymbol{\omega}}

%
% shorthand for unit vectors
%
\newcommand{\acap}[0]{\hat{\Ba}}
\newcommand{\bcap}[0]{\hat{\Bb}}
\newcommand{\ccap}[0]{\hat{\Bc}}
\newcommand{\dcap}[0]{\hat{\Bd}}
\newcommand{\ecap}[0]{\hat{\Be}}
\newcommand{\fcap}[0]{\hat{\Bf}}
\newcommand{\gcap}[0]{\hat{\Bg}}
\newcommand{\hcap}[0]{\hat{\Bh}}
\newcommand{\icap}[0]{\hat{\Bi}}
\newcommand{\jcap}[0]{\hat{\Bj}}
\newcommand{\kcap}[0]{\hat{\Bk}}
\newcommand{\lcap}[0]{\hat{\Bl}}
\newcommand{\mcap}[0]{\hat{\Bm}}
\newcommand{\ncap}[0]{\hat{\Bn}}
\newcommand{\ocap}[0]{\hat{\Bo}}
\newcommand{\pcap}[0]{\hat{\Bp}}
\newcommand{\qcap}[0]{\hat{\Bq}}
\newcommand{\rcap}[0]{\hat{\Br}}
\newcommand{\scap}[0]{\hat{\Bs}}
\newcommand{\tcap}[0]{\hat{\Bt}}
\newcommand{\ucap}[0]{\hat{\Bu}}
\newcommand{\vcap}[0]{\hat{\Bv}}
\newcommand{\wcap}[0]{\hat{\Bw}}
\newcommand{\xcap}[0]{\hat{\Bx}}
\newcommand{\ycap}[0]{\hat{\By}}
\newcommand{\zcap}[0]{\hat{\Bz}}
\newcommand{\thetacap}[0]{\hat{\Btheta}}

%
% to write R^n and C^n in a distinguishable fashion.  Perhaps change this
% to the double lined characters upon figuring out how to do so.
%
\newcommand{\C}[1]{$\mathbb{C}^{#1}$}
\newcommand{\R}[1]{$\mathbb{R}^{#1}$}

%
% various generally useful helpers
%

% derivative of #1 wrt. #2:
\newcommand{\D}[2] {\frac {d#2} {d#1}}

\newcommand{\inv}[1]{\frac{1}{#1}}
\newcommand{\cross}[0]{\times}

\newcommand{\abs}[1]{\lvert{#1}\rvert}
\newcommand{\norm}[1]{\lVert{#1}\rVert}
\newcommand{\innerprod}[2]{\langle{#1}, {#2}\rangle}
\newcommand{\dotprod}[2]{{#1} \cdot {#2}}
\newcommand{\bdotprod}[2]{\left({#1} \cdot {#2}\right)}
\newcommand{\crossprod}[2]{{#1} \cross {#2}}
\newcommand{\tripleprod}[3]{\dotprod{\left(\crossprod{#1}{#2}\right)}{#3}}

\DeclareMathOperator{\Proj}{Proj}
\DeclareMathOperator{\Span}{span}
\DeclareMathOperator{\Sgn}{sgn}
\DeclareMathOperator{\Area}{Area}
\DeclareMathOperator{\Volume}{Volume}

%
% A few miscellaneous things specific to this document
%
\newcommand{\crossop}[1]{\crossprod{#1}{}}

% R2 vector.
\newcommand{\VectorTwo}[2]{
\begin{bmatrix}
 {#1} \\
 {#2}
\end{bmatrix}
}

\newcommand{\VectorN}[1]{
\begin{bmatrix}
{#1}_1 \\
{#1}_2 \\
\vdots \\
{#1}_N \\
\end{bmatrix}
}

\newcommand{\DETuvij}[4]{
\begin{vmatrix}
 {#1}_{#3} & {#1}_{#4} \\
 {#2}_{#3} & {#2}_{#4}
\end{vmatrix}
}

\newcommand{\DETuvwijk}[6]{
\begin{vmatrix}
 {#1}_{#4} & {#1}_{#5} & {#1}_{#6} \\
 {#2}_{#4} & {#2}_{#5} & {#2}_{#6} \\
 {#3}_{#4} & {#3}_{#5} & {#3}_{#6}
\end{vmatrix}
}

\newcommand{\DETuvwxijkl}[8]{
\begin{vmatrix}
 {#1}_{#5} & {#1}_{#6} & {#1}_{#7} & {#1}_{#8} \\
 {#2}_{#5} & {#2}_{#6} & {#2}_{#7} & {#2}_{#8} \\
 {#3}_{#5} & {#3}_{#6} & {#3}_{#7} & {#3}_{#8} \\
 {#4}_{#5} & {#4}_{#6} & {#4}_{#7} & {#4}_{#8} \\
\end{vmatrix}
}

%\newcommand{\DETuvwxyijklm}[10]{
%\begin{vmatrix}
% {#1}_{#6} & {#1}_{#7} & {#1}_{#8} & {#1}_{#9} & {#1}_{#10} \\
% {#2}_{#6} & {#2}_{#7} & {#2}_{#8} & {#2}_{#9} & {#2}_{#10} \\
% {#3}_{#6} & {#3}_{#7} & {#3}_{#8} & {#3}_{#9} & {#3}_{#10} \\
% {#4}_{#6} & {#4}_{#7} & {#4}_{#8} & {#4}_{#9} & {#4}_{#10} \\
% {#5}_{#6} & {#5}_{#7} & {#5}_{#8} & {#5}_{#9} & {#5}_{#10}
%\end{vmatrix}
%}

% R3 vector.
\newcommand{\VectorThree}[3]{
\begin{bmatrix}
 {#1} \\
 {#2} \\
 {#3}
\end{bmatrix}
}



\newcommand{\dt}[1]{\dot{#1}}
\newcommand{\ddt}[1]{\ddot{#1}}
\newcommand{\transpose}[1]{{#1}^{\text{T}}}
\newcommand{\Balpha}[0]{\boldsymbol{\alpha}}

\newcommand{\gpgrade}[2] {{\left\langle{{#1}}\right\rangle}_{#2}}
\newcommand{\gpgradeone}[1] {{\left\langle{{#1}}\right\rangle}_{1}}
\newcommand{\gpscalargrade}[1] {{\left\langle{{#1}}\right\rangle}}
\newcommand{\BOmega}[0]{\boldsymbol{\Omega}}


%
% The real thing:
%

                             % The preamble begins here.
\title{ Kinetic Energy in rotational frame. } % Declares the document's title.
\author{Peeter Joot \quad peeter.joot@gmail.com}         % Declares the author's name.
\date{ April 30, 2008.  Last Revision: $Date: 2009/02/22 15:35:07 $ }

\begin{document}             % End of preamble and beginning of text.

\maketitle{}

\section{ Motivation. }

Fill in the missing details of the rotational KE derivation in Tong's classical
dynamics paper and contrast matrix and GA approach.

Generalize acceleration in terms
of rotating frame coordinates without unproved extrapolation that the z axis result
of Tong's paper is good unconditionally (his cross products are kind of pulled out of
a magic hat and this write up will show a couple ways to see where they come from).

Given coordinates for a point in a rotating frame $\Br'$, the coordinate vector for that point
in a rest frame is:

\begin{equation}\label{eqn:rotcoord}
\Br = R \Br'
\end{equation}

Where the rotating frame moves according to the following z-axis rotation matrix:

\[
R = 
\begin{bmatrix}
\cos \theta & -\sin \theta & 0 \\
\sin \theta & \cos \theta & 0 \\
0 & 0 & 1 \\
\end{bmatrix}
\]

To compute the Lagrangian we want to reexpress the 
kinetic energy of a particle:

\[
K = 
\inv{2} m \dt{\Br}^2
\]

in terms of the rotating frame coordinate system.

\section{ With matrix formulation. }

The Tong paper does this for a z axis rotation with $\theta = \omega t$.
Constant angular frequency is assumed.

First we calculate our position vector in terms of the rotational frame

\[
\Br = R\Br'
\]

%Where
%
%\[
%R_\theta^{-1} = R_{-\theta} =
%\begin{bmatrix}
%\cos \theta & \sin \theta & 0 \\
%-\sin \theta & \cos \theta & 0 \\
%0 & 0 & 1 \\
%\end{bmatrix}
%\]

The rest frame velocity is:

\[
\dt{\Br} = \dt{R}_{\theta} \Br' + R_{\theta} \dt{\Br'}.
\]

Taking the matrix time derivative we have:

\[
\dt{R}_{\theta} =
-\dt{\theta}
\begin{bmatrix}
\sin \theta & \cos \theta & 0 \\
-\cos \theta & \sin \theta & 0 \\
0 & 0 & 0 \\
\end{bmatrix}.
\]

Taking magnitudes of the velocity we have three terms

\begin{align*}
\dt{\Br}^2 
&= 
(\dt{R}_{\theta} \Br') \cdot (\dt{R}_{\theta} \Br')
+2 (\dt{R}_{\theta} \Br') \cdot (R_{\theta} \dt{\Br'})
+(R_{\theta} \dt{\Br'}) \cdot (R_{\theta} \dt{\Br'}) \\
&= 
\transpose{\Br'}\transpose{\dt{R}_{\theta}} \dt{R}_{\theta} \Br'
+2 \transpose{\Br'} \transpose{\dt{R}_{\theta}} R_{\theta} \dt{\Br'}
+\dt{\Br'}^2 \\
\end{align*}

We need to calculate all the intermediate matrix products.  The last was 
identity, and the first is:

\[
\transpose{\dt{R}_{\theta}} \dt{R}_{\theta}
=
{\dt{\theta}}^2
\begin{bmatrix}
\sin \theta & -\cos \theta & 0 \\
\cos \theta & \sin \theta & 0 \\
0 & 0 & 0 \\
\end{bmatrix}
\begin{bmatrix}
\sin \theta & \cos \theta & 0 \\
-\cos \theta & \sin \theta & 0 \\
0 & 0 & 0 \\
\end{bmatrix}
\]
\[
=
{\dt{\theta}}^2
\begin{bmatrix}
1 & 0 & 0 \\
0 & 1 & 0 \\
0 & 0 & 0 \\
\end{bmatrix}
\]

This leaves just the mixed term

\[
\transpose{\dt{R}_{\theta}} {R_{\theta}}
=
-{\dt{\theta}}
\begin{bmatrix}
\sin \theta & -\cos \theta & 0 \\
\cos \theta & \sin \theta & 0 \\
0 & 0 & 0 \\
\end{bmatrix}
\begin{bmatrix}
\cos \theta & -\sin \theta & 0 \\
\sin \theta & \cos \theta & 0 \\
0 & 0 & 1 \\
\end{bmatrix}
\]
\[
=
-{\dt{\theta}}
\begin{bmatrix}
0 & -1 & 0 \\
1 & 0 & 0 \\
0 & 0 & 0 \\
\end{bmatrix}
\]

With $\dt{\theta} = \omega$, the total magnitude of the velocity is thus

\[
\dt{\Br}^2 = 
\transpose{\Br'} 
\omega^2
\begin{bmatrix}
1 & 0 & 0 \\
0 & 1 & 0 \\
0 & 0 & 0 \\
\end{bmatrix}
\Br'
-2 \omega \transpose{{\Br'}} 
\begin{bmatrix}
0 & -1 & 0 \\
1 & 0 & 0 \\
0 & 0 & 0 \\
\end{bmatrix}
\dt{\Br'}
+ {\dt{\Br'}}^2
\]

Tong's paper presents this expanded out in terms of coordinates:

\[
\dt{\Br}^2 = 
\omega^2\left( {x'}^{2} + {y'}^{2} \right)
+ 2 \omega \left( x' \dt{y'} -y' \dt{x'} \right)
+ \left( \dt{x'}^{2} + \dt{y'}^{2} + \dt{z'}^{2} \right)
\]

Or,
\begin{equation}\label{eqn:vmagwithmatrix}
\dt{\Br}^2 = 
\left( -\omega y' + \dt{x'} \right)^2 
+\left( \omega x' + \dt{y'} \right)^2 
+ \dt{z'}^2 
\end{equation}

He also then goes on to
show that this can be written, with $\Bomega = \omega \zcap$, as 

\[
\dt{\Br}^2 = ( \dt{\Br'} + \Bomega \cross \Br')^2
\]

The implication here is that this is a valid result for any rotating
coordinate system.   How to prove this in the general rotation case, is shown much later
in his treatment of rigid bodies.

\section{ With rotor. }

The equivalent to equation \ref{eqn:rotcoord} using a rotor is:

\begin{equation}
\Br' = R^\dagger \Br R
\end{equation}

Where $R = \exp( i\theta/2 )$.

Unlike the 
matrix formulation above we are free to pick any constant unit bivector
for $i$ if we want to generalize this to any rotational axis, but if we
want an equivalent to the above rotation matrix we just have to take
$i = \Be_1 \wedge \Be_2$.

We need a double sided inversion to get our unprimed vector:

\[
\Br = R \Br' R^\dagger
\]

and can then take derivatives:

\[
\dt{\Br} = 
\dt{R} \Br' R^\dagger
+{R} {\Br'} \dt{R}^\dagger
+{R} \dt{\Br'} R^\dagger
\]
\[
= 
i\omega \inv{2} {R} \Br' R^\dagger
- {R} \Br' R^\dagger i\omega\inv{2}
+{R} \dt{\Br'} R^\dagger
\]
\begin{equation}\label{eqn:velocityfixedrotplane}
\implies
\dt{\Br} = \omega i \cdot ({R} \Br' R^\dagger) +  {R} \dt{\Br'} R^\dagger
\end{equation}

One can put this into the traditional cross product form by introducing
a normal vector for the rotational axis in the usual way:

\[
\BOmega = \omega i
\]
\[
\Bomega = \BOmega / \BI_3
\]

We can describe the angular velocity by a scaled normal vector $(\Bomega)$ to the rotational plane, or by a scaled bivector for the plane itself ($\BOmega$).

\begin{align*}
\BOmega \cdot ({R} \Br' R^\dagger)
&= \gpgradeone{ \BOmega {R} \Br' R^\dagger } \\
&= \gpgradeone{ {R} \BOmega \Br' R^\dagger } \\
&= {R} \BOmega \cdot \Br' R^\dagger \\
&= {R} (\Bomega \BI_3) \cdot \Br' R^\dagger \\
&= {R} (\Bomega \cross \Br') R^\dagger \\
\end{align*}

Note that here as before this is valid only when the rotational plane orientation is constant (ie: no wobble), since only then can we assume $i$, and thus $\BOmega$ will commute with the rotor $R$.

Summarizing, we can write our velocity using rotational frame components
as: 
\begin{equation}\label{eqn:vrotcross}
\dt{\Br} = {R} \left( \Bomega \cross \Br' + \dt{\Br'} \right) R^\dagger
\end{equation}
Or
\begin{equation}
\dt{\Br} = {R} \left( \BOmega \cdot \Br' + \dt{\Br'} \right) R^\dagger
\end{equation}

Using the result above from equation \ref{eqn:vrotcross}, we can calculate
the squared magnitude directly:

\begin{align*}
\dt{\Br} ^2 
&= \gpscalargrade{ 
{R} \left( \Bomega \cross \Br' + \dt{\Br'} \right) R^\dagger
{R} \left( \Bomega \cross \Br' + \dt{\Br'} \right) R^\dagger
} \\
&= \gpscalargrade{ 
{R} ( \Bomega \cross \Br' + \dt{\Br'} ) ^2 R^\dagger
} \\
&= ( \Bomega \cross \Br' + \dt{\Br'} ) ^2 \\
\end{align*}

We are able to go straight to the end result this way without the mess
of sine and cosine terms in the rotation matrix.  This is something that
we can expand by components if desired:

\begin{align*}
\Bomega \cross \Br' + \dt{\Br'}
&= 
\begin{vmatrix}
\Be_1 & \Be_2 & \Be_3 \\
0 & 0 & \omega \\
x' & y' & z' \\
\end{vmatrix}
+ \dt{\Br'} \\
&=
\begin{bmatrix}
-\omega y' + \dt{x'} \\
\omega x' + \dt{y'} \\
 \dt{z'} \\
\end{bmatrix}
\end{align*}

This verifies the second part of Tong's equation 2.19, also consistent with the
derivation of equation \ref{eqn:vmagwithmatrix}.

\section{ Acceleration in rotating coordinates. }

Having calculated velocity in terms of rotational frame coordinates, acceleration is the next
logical step.

The starting point is the velocity

\[
\dt{\Br} = R ( \BOmega \cdot \Br' + \dt{\Br}' ) R^\dagger
\]

Taking deriviatives we have
\[
\ddt{\Br} = i \omega /2 \dt{\Br} - \dt{\Br} i \omega /2 + R \left( \dot{\BOmega} \cdot \Br' + \BOmega \cdot \dt{\Br}' + \ddt{\Br}' \right) R^\dagger
\]

The first two terms are a bivector vector dot product and we can simplify this as follows

\begin{align*}
i \omega /2 \dt{\Br} - \dt{\Br} i \omega /2
&= \BOmega /2 \dt{\Br} - \dt{\Br} \BOmega \\
&= \BOmega \cdot \dt{\Br} \\
&= \gpgradeone{ \BOmega R ( \BOmega \cdot \Br' + \dt{\Br}' ) R^\dagger } \\
&= \gpgradeone{ R ( \BOmega (\BOmega \cdot \Br' + \dt{\Br}') ) R^\dagger } \\
&= R ( \BOmega \cdot (\BOmega \cdot \Br') + \BOmega \cdot \dt{\Br}') R^\dagger \\
\end{align*}

Thus the total acceleration is

\begin{equation}
\ddt{\Br} = R \left( \BOmega \cdot (\BOmega \cdot \Br') +\dot{\BOmega} \cdot \Br' + 2 \BOmega \cdot \dt{\Br}' + \ddt{\Br}' \right) R^\dagger
\end{equation}

Or, in terms of cross products, and angular velocity and acceleration vectors $\Bomega$, and $\Balpha$ respectively, this is

\begin{equation}\label{eqn:accelerationfixedrotplane}
\ddt{\Br} = R \left( \Bomega \cross (\Bomega \cross \Br') + \Balpha \cross \Br' + 2 \Bomega \cross \dt{\Br}' + \ddt{\Br}' \right) R^\dagger
\end{equation}

\section{ Allow for a wobble in rotational plane. }

A calculation similar to this can be found in GAFP, but for strictly rigid motion.  It doesn't take too much to combine the two for a generalized result that
expresses the total acceleration expressed in rotating frame coordinates, but also allowing for general rotation where the frame rotation and the angular velocity
bivector don't have to be coplanar (ie: commute as above).

Since the primes and dots are kind of cumbersome switch to the GAFP notation where the position of a particle is expressed in terns of a rotational component $\Bx$
and origin translation $\Bx_0$:

\[
\By = R \Bx R^\dagger + \Bx_0
\]

Taking derivatives for velocity

\begin{equation}\label{eqn:velocity}
\dt{\By} = \dt{R} \Bx R^\dagger +R \Bx \dt{R^\dagger} +R \dt{\Bx} R^\dagger + \dt{\Bx}_0
\end{equation}

Now use the same observation that the derivative of $R R^\dagger = 1$ is zero:

\begin{equation*}
\frac{d (R R^\dagger)}{dt} = \dt{R}R^\dagger + R \dt{R^\dagger} = 0
\end{equation*}
\begin{equation}\label{eqn:rotdtrot}
\implies
\dt{R}R^\dagger = 
%-R \dt{R^\dagger} =
 - R \dt{R}^\dagger = -{\left( \dt{R} R^\dagger \right)}^\dagger
\end{equation}

Since $R$ has only grade 0 and 2 terms, so does its derivative.  Thus the product of the two has grade 0, 2, and 4 terms, but 
equation \ref{eqn:rotdtrot} shows that the product $\dt{R} R^\dagger$ has a value that is the negative of its reverse, so it must have
only grade 2 terms (the reverse of the grade 0 and 4 terms would not change sign).

As in equation \ref{eqn:velocityfixedrotplane} we want to write $\dt{R}$ as a bivector/rotor product and equation \ref{eqn:rotdtrot} gives us a means to do so.
This would have been clearer in GAFP if they had done the simple example first with the orientation of the rotational plane fixed.

So, write:

\[
\dt{R}R^\dagger = \inv{2}\BOmega
\]
\[
\dt{R} = \inv{2}\BOmega R
\]
\[
\dt{R}^\dagger = - \inv{2} R^\dagger \BOmega
\]

(including the $1/2$ here is a bit of a cheat ... it's here because having done the calculation on paper first one sees that it's natural to do so).

With this we can substitute back into equation \ref{eqn:velocity}, writing $\By_0 = \By - \Bx_0$ :

\begin{align*}
\dt{\By}
&= \inv{2} \BOmega {R} \Bx R^\dagger - \inv{2} R \Bx {R^\dagger} \BOmega +R \dt{\Bx} R^\dagger + \dt{\Bx}_0 \\
&= \inv{2}\left(\BOmega \By_ - \By_0 \BOmega\right) +R \dt{\Bx} R^\dagger + \dt{\Bx}_0 \\
&= \BOmega \cdot \By_0 +R \dt{\Bx} R^\dagger + \dt{\Bx}_0 \\
\end{align*}

We also want to pull in this $\BOmega$ into the rotor as in the fixed orientation
case, but cannot use commutivity this time since the rotor and angular velocity bivector aren't neccessarily in the same plane.

This is where GAFP introduces their body angular velocity, which applies an inverse rotation to the angular velocity.

Let:
\[
\BOmega = R \BOmega_B R^\dagger
\]

Computing this bivector dot product with $\By$ we have

\begin{align*}
\BOmega \cdot \By_0
&= (R \BOmega_B R^\dagger) \cdot (R \Bx R^\dagger) \\
&= \gpgradeone{R \BOmega_B R^\dagger R \Bx R^\dagger} \\
&= \gpgradeone{R \BOmega_B \Bx R^\dagger} \\
&= \gpgradeone{R (\BOmega_B \cdot \Bx + \BOmega_B \wedge \Bx) R^\dagger} \\
&= {R \BOmega_B \cdot \Bx R^\dagger} \\
\end{align*}

Thus the total velocity is:

\begin{equation}
\dt{\By} = {R (\BOmega_B \cdot \Bx + \dt{\Bx} )R^\dagger} + \dt{\Bx}_0
\end{equation}

Thus given any vector $\Bx$ in the rotating frame coordinate system, we have the relationship for the inertial frame velocity.  We can apply this a second
time to compute the inertial (rest frame) acceleration in terms of rotating coordinates.  Write $\Bv = \BOmega_B \cdot \Bx + \dt{\Bx}$, 

\begin{equation*}
\dt{\By} = {R \Bv R^\dagger} + \dt{\Bx}_0
\end{equation*}
\begin{equation*}
\implies
\ddt{\By} = {R (\BOmega_B \cdot \Bv + \dt{\Bv} ) R^\dagger} + \ddt{\Bx}_0
\end{equation*}

\[
\dt{\Bv} = 
\dt{\BOmega}_B \cdot \Bx 
+\BOmega_B \cdot \dt{\Bx}
+ \ddt{\Bx}
\]

Combining these we have:
\begin{align*}
\ddt{\By} 
&= {R (\BOmega_B \cdot ( \BOmega_B \cdot \Bx + \dt{\Bx} ) + \dt{\BOmega}_B \cdot \Bx +\BOmega_B \cdot \dt{\Bx} + \ddt{\Bx}) R^\dagger} + \ddt{\Bx}_0 \\
\end{align*}

\begin{equation}
\implies
\ddt{\By} 
= {R (\BOmega_B \cdot ( \BOmega_B \cdot \Bx ) + \dt{\BOmega}_B \cdot \Bx + 2\BOmega_B \cdot \dt{\Bx} + \ddt{\Bx}) R^\dagger} + \ddt{\Bx}_0
\end{equation}

This generalizes equation \ref{eqn:accelerationfixedrotplane}, providing the rest frame acceleration in terms of rotational frame coordinates, with centrifugal acceleration, euler force acceleration, and corolis force acceleration terms that accompany the plain old acceleration term $\ddt{\Bx}$.  The only
requirement for the generality of allowing the orientation of the rotational plane to potentially vary is the use of the ``body angular velocity''
$\BOmega_B$, replacing the angular velocity as seen from the rest frame $\BOmega$.

\subsection{ Body angular acceleration in terms of rest frame. }

Since we know the relationship between the body angular velocity $\BOmega_B$ with the Rotor (rest frame) angular velocity bivector, for
completeness, lets compute the body angular acceleration bivector $\dt{\BOmega}_B$ in terms of the rest frame angular acceleration $\dt{\BOmega}$.

\[
\BOmega_B = R^\dagger \BOmega R
\]
\begin{align*}
\implies
\dt{\BOmega}_B 
&= \dt{R}^\dagger \BOmega R +R^\dagger \dt{\BOmega} R +R^\dagger \BOmega \dt{R} \\
&= -\inv{2}R^\dagger \BOmega^2 R +R^\dagger \dt{\BOmega} R +R^\dagger \BOmega^2 R \inv{2} \\
&= \inv{2} \left(R^\dagger \BOmega^2 R - R^\dagger \BOmega^2 R\right) +R^\dagger \dt{\BOmega} R \\
&= R^\dagger \dt{\BOmega} R \\
\end{align*}

This shows that the body angular acceleration is just an inverse rotation of the rest frame angular acceleration like the angular velocities are.

\section{ Revisit general rotation using matrixes. }

Having fully calculated velocity and acceleration in terms of rotating frame coordinates, lets
go back and revisit this with matrixes and see how one would do the same for a general rotation.

Following GAFP express the rest frame coordinates for a point $\By$ in terms of a rotation
applied to a rotating frame position $\Bx$ (this is easier than the mess of primes and dots
used in Tong's paper).  Also omit the origin translation (that can be added in later if desired
easily enough)

\[
\By = R \Bx
\]

Thus the deriviative is:

\[
\dt{\By} = \dt{R} \Bx + R \dt{\Bx}.
\]

As in the GA case we want to factor this so that we have a rotation applied to a something
that is completely specified in the rotating frame.  This
is quite easy with matrixes, as we just have to factor out a rotation matrix from $\dt{R}$:

\begin{align*}
\dt{\By} 
&= R \transpose{R}\dt{R} \Bx + R \dt{\Bx} \\
&= R \left(\transpose{R}\dt{R} \Bx + \dt{\Bx} \right) \\
\end{align*}

This new product $\transpose{R}\dt{R} \Bx$ we have seen above in the special case of z-axis
rotation as a cross product.  In the GA general rotation case, we've seen that this as a
bivector-vector dot product.  Both of these are fundamentally antisymmetric operations,
so we expect this of the matrix operator too.  Verification of this antisymmetry follows
in almost the same fashion as the GA case, by observing that the deriviative of an identity
matrix $I = \transpose{R}R$ is zero:

\[
\dt{I} = 0
\]
\[
\implies
\transpose{\dt{R}}R + \transpose{R}\dt{R} = 0
\]
\[
\implies
\transpose{R}\dt{R} = -\transpose{\dt{R}}R = -\transpose{\transpose{R}\dt{R}}
\]

Thus if one writes:

\begin{equation}\label{eqn:bodyangularvelocitymatrix}
\BOmega = \transpose{R}\dt{R}
\end{equation}

the antisymetric property of this matrix can be summarized as:

\[
\BOmega = -\transpose{\BOmega}.
\]

Let's write out the form of this matrix in the first few dimensions:

\begin{itemize}
\item \R{2}

\[
\BOmega = 
\begin{bmatrix}
0 & -a \\
a & 0  \\
\end{bmatrix}
\]

For some $a$.

\item \R{3}

\[
\BOmega = 
\begin{bmatrix}
0 & -a & -b \\
a &  0 & -c \\
b &  c &  0 \\
\end{bmatrix}
\]

For some $a, b, c$.

\item \R{4}

\[
\BOmega = 
\begin{bmatrix}
0 & -a & -b & -d \\
a & 0 & -c & -e \\
b & c & 0 & -f \\
d & e & f & 0 \\
\end{bmatrix}
\]

For some $a, b, c, d, e, f$.
\end{itemize}

For \R{N} we have $(N^2-N)/2$ degrees of freedom.  It's noteworthy to observe that this is exactly the number of basis elements of a bivector.  For example, in \R{4}, such a bivector basis is
$\Be_{12}, \Be_{13}, \Be_{14}, \Be_{23}, \Be_{24}, \Be_{34}$.

For \R{3} we have three degrees of freedom and because of the antisymmetry 
can express this matrix-vector product using the cross product.  Let

\[
(a,b,c) = (\omega_3, -\omega_2, \omega_1)
\]

One has:

\[
\BOmega \Bx = 
\begin{bmatrix}
0 & -\omega_3 & \omega_2 \\
\omega_3 &  0 & -\omega_1 \\
-\omega_2 & \omega_1 &  0 \\
\end{bmatrix}
\begin{bmatrix}
x_1 \\
x_2 \\
x_3 \\
\end{bmatrix}
=
\begin{bmatrix}
-\omega_3 x_2 +\omega_2 x_3 \\
+\omega_3 x_1 -\omega_1 x_3 \\
-\omega_2 x_1 +\omega_1 x_2 \\
\end{bmatrix}
= \Bomega \cross \Bx
\]

Summarizing the velocity result we have, using $\BOmega$ from equation \ref{eqn:bodyangularvelocitymatrix}:

\begin{equation}
\dt{\By} = R \left( \BOmega \Bx + \dt{\Bx} \right)
\end{equation}

Or, for \R{3}, we can define a body angular velocity vector

\begin{equation}
\Bomega = 
\begin{bmatrix}
\BOmega_{32} \\
\BOmega_{13} \\
\BOmega_{21} \\
\end{bmatrix}
\end{equation}

and thus write the velocity as:

\begin{equation}
\dt{\By} = R \left( \Bomega \cross \Bx + \dt{\Bx} \right)
\end{equation}

This, like the GA result is good for general rotations.  Then don't have to be constant
rotation rates, and it allows for arbitrarily
oriented as well as wobbly motion of the rotating frame.

As with the GA general velocity calculation, this general form also allows us to calculate
the squared velocity easily, since the rotation matrixes will
cancel after transposition:

\[
\dt{\By}^2 = 
\left(R \left( \Bomega \cross \Bx + \dt{\Bx} \right)\right) \cdot
\left(R \left( \Bomega \cross \Bx + \dt{\Bx} \right)\right)
=
\transpose{\left( \Bomega \cross \Bx + \dt{\Bx} \right)} \transpose{R}
R \left( \Bomega \cross \Bx + \dt{\Bx} \right)
\]
\[
\implies
\dt{\By}^2 = 
{\left( \Bomega \cross \Bx + \dt{\Bx} \right)}^2
\]

\section{ Equations of motion from Lagrange partials. }

TBD.  Do this using the Rotor formulation.  How?

\end{document}               % End of document.

\documentclass{article}

\usepackage{amsmath}
\usepackage{mathpazo}

%
% shorthand for bold symbols, convenient for vectors and matrices
%
\newcommand{\Ba}[0]{\mathbf{a}}
\newcommand{\Bb}[0]{\mathbf{b}}
\newcommand{\Bc}[0]{\mathbf{c}}
\newcommand{\Bd}[0]{\mathbf{d}}
\newcommand{\Be}[0]{\mathbf{e}}
\newcommand{\Bf}[0]{\mathbf{f}}
\newcommand{\Bg}[0]{\mathbf{g}}
\newcommand{\Bh}[0]{\mathbf{h}}
\newcommand{\Bi}[0]{\mathbf{i}}
\newcommand{\Bj}[0]{\mathbf{j}}
\newcommand{\Bk}[0]{\mathbf{k}}
\newcommand{\Bl}[0]{\mathbf{l}}
\newcommand{\Bm}[0]{\mathbf{m}}
\newcommand{\Bn}[0]{\mathbf{n}}
\newcommand{\Bo}[0]{\mathbf{o}}
\newcommand{\Bp}[0]{\mathbf{p}}
\newcommand{\Bq}[0]{\mathbf{q}}
\newcommand{\Br}[0]{\mathbf{r}}
\newcommand{\Bs}[0]{\mathbf{s}}
\newcommand{\Bt}[0]{\mathbf{t}}
\newcommand{\Bu}[0]{\mathbf{u}}
\newcommand{\Bv}[0]{\mathbf{v}}
\newcommand{\Bw}[0]{\mathbf{w}}
\newcommand{\Bx}[0]{\mathbf{x}}
\newcommand{\By}[0]{\mathbf{y}}
\newcommand{\Bz}[0]{\mathbf{z}}
\newcommand{\BA}[0]{\mathbf{A}}
\newcommand{\BB}[0]{\mathbf{B}}
\newcommand{\BC}[0]{\mathbf{C}}
\newcommand{\BD}[0]{\mathbf{D}}
\newcommand{\BE}[0]{\mathbf{E}}
\newcommand{\BF}[0]{\mathbf{F}}
\newcommand{\BG}[0]{\mathbf{G}}
\newcommand{\BH}[0]{\mathbf{H}}
\newcommand{\BI}[0]{\mathbf{I}}
\newcommand{\BJ}[0]{\mathbf{J}}
\newcommand{\BK}[0]{\mathbf{K}}
\newcommand{\BL}[0]{\mathbf{L}}
\newcommand{\BM}[0]{\mathbf{M}}
\newcommand{\BN}[0]{\mathbf{N}}
\newcommand{\BO}[0]{\mathbf{O}}
\newcommand{\BP}[0]{\mathbf{P}}
\newcommand{\BQ}[0]{\mathbf{Q}}
\newcommand{\BR}[0]{\mathbf{R}}
\newcommand{\BS}[0]{\mathbf{S}}
\newcommand{\BT}[0]{\mathbf{T}}
\newcommand{\BU}[0]{\mathbf{U}}
\newcommand{\BV}[0]{\mathbf{V}}
\newcommand{\BW}[0]{\mathbf{W}}
\newcommand{\BX}[0]{\mathbf{X}}
\newcommand{\BY}[0]{\mathbf{Y}}
\newcommand{\BZ}[0]{\mathbf{Z}}

\newcommand{\Bzero}[0]{\mathbf{0}}
\newcommand{\Btheta}[0]{\boldsymbol{\theta}}
\newcommand{\Btau}[0]{\boldsymbol{\tau}}
\newcommand{\Bomega}[0]{\boldsymbol{\omega}}

%
% shorthand for unit vectors
%
\newcommand{\acap}[0]{\hat{\Ba}}
\newcommand{\bcap}[0]{\hat{\Bb}}
\newcommand{\ccap}[0]{\hat{\Bc}}
\newcommand{\dcap}[0]{\hat{\Bd}}
\newcommand{\ecap}[0]{\hat{\Be}}
\newcommand{\fcap}[0]{\hat{\Bf}}
\newcommand{\gcap}[0]{\hat{\Bg}}
\newcommand{\hcap}[0]{\hat{\Bh}}
\newcommand{\icap}[0]{\hat{\Bi}}
\newcommand{\jcap}[0]{\hat{\Bj}}
\newcommand{\kcap}[0]{\hat{\Bk}}
\newcommand{\lcap}[0]{\hat{\Bl}}
\newcommand{\mcap}[0]{\hat{\Bm}}
\newcommand{\ncap}[0]{\hat{\Bn}}
\newcommand{\ocap}[0]{\hat{\Bo}}
\newcommand{\pcap}[0]{\hat{\Bp}}
\newcommand{\qcap}[0]{\hat{\Bq}}
\newcommand{\rcap}[0]{\hat{\Br}}
\newcommand{\scap}[0]{\hat{\Bs}}
\newcommand{\tcap}[0]{\hat{\Bt}}
\newcommand{\ucap}[0]{\hat{\Bu}}
\newcommand{\vcap}[0]{\hat{\Bv}}
\newcommand{\wcap}[0]{\hat{\Bw}}
\newcommand{\xcap}[0]{\hat{\Bx}}
\newcommand{\ycap}[0]{\hat{\By}}
\newcommand{\zcap}[0]{\hat{\Bz}}
\newcommand{\thetacap}[0]{\hat{\Btheta}}

%
% to write R^n and C^n in a distinguishable fashion.  Perhaps change this
% to the double lined characters upon figuring out how to do so.
%
\newcommand{\C}[1]{$\mathbb{C}^{#1}$}
\newcommand{\R}[1]{$\mathbb{R}^{#1}$}

%
% various generally useful helpers
%

% derivative of #1 wrt. #2:
\newcommand{\D}[2] {\frac {d#2} {d#1}}

\newcommand{\inv}[1]{\frac{1}{#1}}
\newcommand{\cross}[0]{\times}

\newcommand{\abs}[1]{\lvert{#1}\rvert}
\newcommand{\norm}[1]{\lVert{#1}\rVert}
\newcommand{\innerprod}[2]{\langle{#1}, {#2}\rangle}
\newcommand{\dotprod}[2]{{#1} \cdot {#2}}
\newcommand{\bdotprod}[2]{\left({#1} \cdot {#2}\right)}
\newcommand{\crossprod}[2]{{#1} \cross {#2}}
\newcommand{\tripleprod}[3]{\dotprod{\left(\crossprod{#1}{#2}\right)}{#3}}

\DeclareMathOperator{\Proj}{Proj}
\DeclareMathOperator{\Span}{span}
\DeclareMathOperator{\Sgn}{sgn}
\DeclareMathOperator{\Area}{Area}
\DeclareMathOperator{\Volume}{Volume}

%
% A few miscellaneous things specific to this document
%
\newcommand{\crossop}[1]{\crossprod{#1}{}}

% R2 vector.
\newcommand{\VectorTwo}[2]{
\begin{bmatrix}
 {#1} \\
 {#2}
\end{bmatrix}
}

\newcommand{\VectorN}[1]{
\begin{bmatrix}
{#1}_1 \\
{#1}_2 \\
\vdots \\
{#1}_N \\
\end{bmatrix}
}

\newcommand{\DETuvij}[4]{
\begin{vmatrix}
 {#1}_{#3} & {#1}_{#4} \\
 {#2}_{#3} & {#2}_{#4}
\end{vmatrix}
}

\newcommand{\DETuvwijk}[6]{
\begin{vmatrix}
 {#1}_{#4} & {#1}_{#5} & {#1}_{#6} \\
 {#2}_{#4} & {#2}_{#5} & {#2}_{#6} \\
 {#3}_{#4} & {#3}_{#5} & {#3}_{#6}
\end{vmatrix}
}

\newcommand{\DETuvwxijkl}[8]{
\begin{vmatrix}
 {#1}_{#5} & {#1}_{#6} & {#1}_{#7} & {#1}_{#8} \\
 {#2}_{#5} & {#2}_{#6} & {#2}_{#7} & {#2}_{#8} \\
 {#3}_{#5} & {#3}_{#6} & {#3}_{#7} & {#3}_{#8} \\
 {#4}_{#5} & {#4}_{#6} & {#4}_{#7} & {#4}_{#8} \\
\end{vmatrix}
}

%\newcommand{\DETuvwxyijklm}[10]{
%\begin{vmatrix}
% {#1}_{#6} & {#1}_{#7} & {#1}_{#8} & {#1}_{#9} & {#1}_{#10} \\
% {#2}_{#6} & {#2}_{#7} & {#2}_{#8} & {#2}_{#9} & {#2}_{#10} \\
% {#3}_{#6} & {#3}_{#7} & {#3}_{#8} & {#3}_{#9} & {#3}_{#10} \\
% {#4}_{#6} & {#4}_{#7} & {#4}_{#8} & {#4}_{#9} & {#4}_{#10} \\
% {#5}_{#6} & {#5}_{#7} & {#5}_{#8} & {#5}_{#9} & {#5}_{#10}
%\end{vmatrix}
%}

% R3 vector.
\newcommand{\VectorThree}[3]{
\begin{bmatrix}
 {#1} \\
 {#2} \\
 {#3}
\end{bmatrix}
}


%<misc>
%
\newcommand{\Abs}[1]{{\left\lvert{#1}\right\rvert}}
\newcommand{\spacegrad}[0]{\boldsymbol{\nabla}}
\newcommand{\grad}[0]{\nabla}
\newcommand{\LL}[0]{\mathcal{L}}

% == \partial_{#1} {#2}
\newcommand{\PD}[2]{\frac{\partial {#2}}{\partial {#1}}}
% inline variant
\newcommand{\PDi}[2]{{\partial {#2}}/{\partial {#1}}}

\newcommand{\PDD}[3]{\frac{\partial^2 {#3}}{\partial {#1}\partial {#2}}}
%\newcommand{\PDd}[2]{\frac{\partial^2 {#2}}{{\partial{#1}}^2}}
\newcommand{\PDsq}[2]{\frac{\partial^2 {#2}}{(\partial {#1})^2}}

\newcommand{\Partial}[2]{\frac{\partial {#1}}{\partial {#2}}}
\DeclareMathOperator{\RejName}{Rej}
\newcommand{\Rej}[2]{\RejName_{#1}\left( {#2} \right)}
\newcommand{\Rm}[1]{\mathbb{R}^{#1}}
\newcommand{\Cm}[1]{\mathbb{C}^{#1}}
\newcommand{\conj}[0]{{*}}

%</misc>

% <grade selection>
%
\newcommand{\gpgrade}[2] {{\left\langle{{#1}}\right\rangle}_{#2}}

\newcommand{\gpgradezero}[1] {\gpgrade{#1}{}}
%\newcommand{\gpscalargrade}[1] {{\left\langle{{#1}}\right\rangle}}
%\newcommand{\gpgradezero}[1] {\gpgrade{#1}{0}}

%\newcommand{\gpgradeone}[1] {{\left\langle{{#1}}\right\rangle}_{1}}
\newcommand{\gpgradeone}[1] {\gpgrade{#1}{1}}

\newcommand{\gpgradetwo}[1] {\gpgrade{#1}{2}}
\newcommand{\gpgradethree}[1] {\gpgrade{#1}{3}}
\newcommand{\gpgradefour}[1] {\gpgrade{#1}{4}}
%
% </grade selection>



\newcommand{\adot}[0]{{\dot{a}}}
\newcommand{\bdot}[0]{{\dot{b}}}
% taken for centered dot:
%\newcommand{\cdot}[0]{{\dot{c}}}
%\newcommand{\ddot}[0]{{\dot{d}}}
\newcommand{\edot}[0]{{\dot{e}}}
\newcommand{\fdot}[0]{{\dot{f}}}
\newcommand{\gdot}[0]{{\dot{g}}}
\newcommand{\hdot}[0]{{\dot{h}}}
\newcommand{\idot}[0]{{\dot{i}}}
\newcommand{\jdot}[0]{{\dot{j}}}
\newcommand{\kdot}[0]{{\dot{k}}}
\newcommand{\ldot}[0]{{\dot{l}}}
\newcommand{\mdot}[0]{{\dot{m}}}
\newcommand{\ndot}[0]{{\dot{n}}}
%\newcommand{\odot}[0]{{\dot{o}}}
\newcommand{\pdot}[0]{{\dot{p}}}
\newcommand{\qdot}[0]{{\dot{q}}}
\newcommand{\rdot}[0]{{\dot{r}}}
\newcommand{\sdot}[0]{{\dot{s}}}
\newcommand{\tdot}[0]{{\dot{t}}}
\newcommand{\udot}[0]{{\dot{u}}}
\newcommand{\vdot}[0]{{\dot{v}}}
\newcommand{\wdot}[0]{{\dot{w}}}
\newcommand{\xdot}[0]{{\dot{x}}}
\newcommand{\ydot}[0]{{\dot{y}}}
\newcommand{\zdot}[0]{{\dot{z}}}
\newcommand{\addot}[0]{{\ddot{a}}}
\newcommand{\bddot}[0]{{\ddot{b}}}
\newcommand{\cddot}[0]{{\ddot{c}}}
%\newcommand{\dddot}[0]{{\ddot{d}}}
\newcommand{\eddot}[0]{{\ddot{e}}}
\newcommand{\fddot}[0]{{\ddot{f}}}
\newcommand{\gddot}[0]{{\ddot{g}}}
\newcommand{\hddot}[0]{{\ddot{h}}}
\newcommand{\iddot}[0]{{\ddot{i}}}
\newcommand{\jddot}[0]{{\ddot{j}}}
\newcommand{\kddot}[0]{{\ddot{k}}}
\newcommand{\lddot}[0]{{\ddot{l}}}
\newcommand{\mddot}[0]{{\ddot{m}}}
\newcommand{\nddot}[0]{{\ddot{n}}}
\newcommand{\oddot}[0]{{\ddot{o}}}
\newcommand{\pddot}[0]{{\ddot{p}}}
\newcommand{\qddot}[0]{{\ddot{q}}}
\newcommand{\rddot}[0]{{\ddot{r}}}
\newcommand{\sddot}[0]{{\ddot{s}}}
\newcommand{\tddot}[0]{{\ddot{t}}}
\newcommand{\uddot}[0]{{\ddot{u}}}
\newcommand{\vddot}[0]{{\ddot{v}}}
\newcommand{\wddot}[0]{{\ddot{w}}}
\newcommand{\xddot}[0]{{\ddot{x}}}
\newcommand{\yddot}[0]{{\ddot{y}}}
\newcommand{\zddot}[0]{{\ddot{z}}}

%<bold and dot greek symbols>
%

\newcommand{\Deltadot}[0]{{\dot{\Delta}}}
\newcommand{\Gammadot}[0]{{\dot{\Gamma}}}
\newcommand{\Lambdadot}[0]{{\dot{\Lambda}}}
\newcommand{\Omegadot}[0]{{\dot{\Omega}}}
\newcommand{\Phidot}[0]{{\dot{\Phi}}}
\newcommand{\Pidot}[0]{{\dot{\Pi}}}
\newcommand{\Psidot}[0]{{\dot{\Psi}}}
\newcommand{\Sigmadot}[0]{{\dot{\Sigma}}}
\newcommand{\Thetadot}[0]{{\dot{\Theta}}}
\newcommand{\Upsilondot}[0]{{\dot{\Upsilon}}}
\newcommand{\Xidot}[0]{{\dot{\Xi}}}
\newcommand{\alphadot}[0]{{\dot{\alpha}}}
\newcommand{\betadot}[0]{{\dot{\beta}}}
\newcommand{\chidot}[0]{{\dot{\chi}}}
\newcommand{\deltadot}[0]{{\dot{\delta}}}
\newcommand{\epsilondot}[0]{{\dot{\epsilon}}}
\newcommand{\etadot}[0]{{\dot{\eta}}}
\newcommand{\gammadot}[0]{{\dot{\gamma}}}
\newcommand{\kappadot}[0]{{\dot{\kappa}}}
\newcommand{\lambdadot}[0]{{\dot{\lambda}}}
\newcommand{\mudot}[0]{{\dot{\mu}}}
\newcommand{\nudot}[0]{{\dot{\nu}}}
\newcommand{\omegadot}[0]{{\dot{\omega}}}
\newcommand{\phidot}[0]{{\dot{\phi}}}
\newcommand{\pidot}[0]{{\dot{\pi}}}
\newcommand{\psidot}[0]{{\dot{\psi}}}
\newcommand{\rhodot}[0]{{\dot{\rho}}}
\newcommand{\sigmadot}[0]{{\dot{\sigma}}}
\newcommand{\taudot}[0]{{\dot{\tau}}}
\newcommand{\thetadot}[0]{{\dot{\theta}}}
\newcommand{\upsilondot}[0]{{\dot{\upsilon}}}
\newcommand{\varepsilondot}[0]{{\dot{\varepsilon}}}
\newcommand{\varphidot}[0]{{\dot{\varphi}}}
\newcommand{\varpidot}[0]{{\dot{\varpi}}}
\newcommand{\varrhodot}[0]{{\dot{\varrho}}}
\newcommand{\varsigmadot}[0]{{\dot{\varsigma}}}
\newcommand{\varthetadot}[0]{{\dot{\vartheta}}}
\newcommand{\xidot}[0]{{\dot{\xi}}}
\newcommand{\zetadot}[0]{{\dot{\zeta}}}

\newcommand{\Deltaddot}[0]{{\ddot{\Delta}}}
\newcommand{\Gammaddot}[0]{{\ddot{\Gamma}}}
\newcommand{\Lambdaddot}[0]{{\ddot{\Lambda}}}
\newcommand{\Omegaddot}[0]{{\ddot{\Omega}}}
\newcommand{\Phiddot}[0]{{\ddot{\Phi}}}
\newcommand{\Piddot}[0]{{\ddot{\Pi}}}
\newcommand{\Psiddot}[0]{{\ddot{\Psi}}}
\newcommand{\Sigmaddot}[0]{{\ddot{\Sigma}}}
\newcommand{\Thetaddot}[0]{{\ddot{\Theta}}}
\newcommand{\Upsilonddot}[0]{{\ddot{\Upsilon}}}
\newcommand{\Xiddot}[0]{{\ddot{\Xi}}}
\newcommand{\alphaddot}[0]{{\ddot{\alpha}}}
\newcommand{\betaddot}[0]{{\ddot{\beta}}}
\newcommand{\chiddot}[0]{{\ddot{\chi}}}
\newcommand{\deltaddot}[0]{{\ddot{\delta}}}
\newcommand{\epsilonddot}[0]{{\ddot{\epsilon}}}
\newcommand{\etaddot}[0]{{\ddot{\eta}}}
\newcommand{\gammaddot}[0]{{\ddot{\gamma}}}
\newcommand{\kappaddot}[0]{{\ddot{\kappa}}}
\newcommand{\lambdaddot}[0]{{\ddot{\lambda}}}
\newcommand{\muddot}[0]{{\ddot{\mu}}}
\newcommand{\nuddot}[0]{{\ddot{\nu}}}
\newcommand{\omegaddot}[0]{{\ddot{\omega}}}
\newcommand{\phiddot}[0]{{\ddot{\phi}}}
\newcommand{\piddot}[0]{{\ddot{\pi}}}
\newcommand{\psiddot}[0]{{\ddot{\psi}}}
\newcommand{\rhoddot}[0]{{\ddot{\rho}}}
\newcommand{\sigmaddot}[0]{{\ddot{\sigma}}}
\newcommand{\tauddot}[0]{{\ddot{\tau}}}
\newcommand{\thetaddot}[0]{{\ddot{\theta}}}
\newcommand{\upsilonddot}[0]{{\ddot{\upsilon}}}
\newcommand{\varepsilonddot}[0]{{\ddot{\varepsilon}}}
\newcommand{\varphiddot}[0]{{\ddot{\varphi}}}
\newcommand{\varpiddot}[0]{{\ddot{\varpi}}}
\newcommand{\varrhoddot}[0]{{\ddot{\varrho}}}
\newcommand{\varsigmaddot}[0]{{\ddot{\varsigma}}}
\newcommand{\varthetaddot}[0]{{\ddot{\vartheta}}}
\newcommand{\xiddot}[0]{{\ddot{\xi}}}
\newcommand{\zetaddot}[0]{{\ddot{\zeta}}}

\newcommand{\BDelta}[0]{\boldsymbol{\Delta}}
\newcommand{\BGamma}[0]{\boldsymbol{\Gamma}}
\newcommand{\BLambda}[0]{\boldsymbol{\Lambda}}
\newcommand{\BOmega}[0]{\boldsymbol{\Omega}}
\newcommand{\BPhi}[0]{\boldsymbol{\Phi}}
\newcommand{\BPi}[0]{\boldsymbol{\Pi}}
\newcommand{\BPsi}[0]{\boldsymbol{\Psi}}
\newcommand{\BSigma}[0]{\boldsymbol{\Sigma}}
\newcommand{\BTheta}[0]{\boldsymbol{\Theta}}
\newcommand{\BUpsilon}[0]{\boldsymbol{\Upsilon}}
\newcommand{\BXi}[0]{\boldsymbol{\Xi}}
\newcommand{\Balpha}[0]{\boldsymbol{\alpha}}
\newcommand{\Bbeta}[0]{\boldsymbol{\beta}}
\newcommand{\Bchi}[0]{\boldsymbol{\chi}}
\newcommand{\Bdelta}[0]{\boldsymbol{\delta}}
\newcommand{\Bepsilon}[0]{\boldsymbol{\epsilon}}
\newcommand{\Beta}[0]{\boldsymbol{\eta}}
\newcommand{\Bgamma}[0]{\boldsymbol{\gamma}}
\newcommand{\Bkappa}[0]{\boldsymbol{\kappa}}
\newcommand{\Blambda}[0]{\boldsymbol{\lambda}}
\newcommand{\Bmu}[0]{\boldsymbol{\mu}}
\newcommand{\Bnu}[0]{\boldsymbol{\nu}}
%\newcommand{\Bomega}[0]{\boldsymbol{\omega}}
\newcommand{\Bphi}[0]{\boldsymbol{\phi}}
\newcommand{\Bpi}[0]{\boldsymbol{\pi}}
\newcommand{\Bpsi}[0]{\boldsymbol{\psi}}
\newcommand{\Brho}[0]{\boldsymbol{\rho}}
\newcommand{\Bsigma}[0]{\boldsymbol{\sigma}}
%\newcommand{\Btau}[0]{\boldsymbol{\tau}}
%\newcommand{\Btheta}[0]{\boldsymbol{\theta}}
\newcommand{\Bupsilon}[0]{\boldsymbol{\upsilon}}
\newcommand{\Bvarepsilon}[0]{\boldsymbol{\varepsilon}}
\newcommand{\Bvarphi}[0]{\boldsymbol{\varphi}}
\newcommand{\Bvarpi}[0]{\boldsymbol{\varpi}}
\newcommand{\Bvarrho}[0]{\boldsymbol{\varrho}}
\newcommand{\Bvarsigma}[0]{\boldsymbol{\varsigma}}
\newcommand{\Bvartheta}[0]{\boldsymbol{\vartheta}}
\newcommand{\Bxi}[0]{\boldsymbol{\xi}}
\newcommand{\Bzeta}[0]{\boldsymbol{\zeta}}
%
%</bold and dot greek symbols>
%<infrequent>
%
%\newcommand{\AreaOp}[1]{\AName_{#1}}
%\newcommand{\Babs}[0]{\abs{\BB}}
%\newcommand{\Bcap}[0]{\hat{\BB}}
%\newcommand{\BrPrimeRej}[0]{\rcap(\rcap \wedge \Br')}
%\newcommand{\CA}[0]{\mathcal{A}}
%\newcommand{\Cos}[1]{\cos{\left({#1}\right)}}
%\newcommand{\Det}[1] {\abs{#1}}
%\newcommand{\Dsq}[2] {\frac {\partial^2 {#1}} {\partial {#2}^2}}
%\newcommand{\Exp}[1]{\exp{\left({#1}\right)}}
%\newcommand{\Norm}[1]{\left\lVert{#1}\right\rVert}
%\newcommand{\Sin}[1]{\sin{\left({#1}\right)}}
%\newcommand{\T}[0]{\text{T}}
%\newcommand{\VolumeOp}[1]{\VName_{#1}}
%\newcommand{\agrad}[0]{\Ba \cdot \nabla}
%\newcommand{\alphacap}[0]{\hat{\boldsymbol{\alpha}}}
%\newcommand{\Fcap}[0]{\hat{\BF}}
%\newcommand{\bithree}[0]{{\Bi}_3}
%\newcommand{\bxa}[0]{\Bx\Ba}
%\newcommand{\coordvec}[2]{
%\newcommand{\costheta}[0]{\acap \cdot \xcap}
%\newcommand{\ddt}[1]{\ddot{#1}}
%\newcommand{\ddu}[1] {\frac {d{#1}} {du}}
%\newcommand{\dsqxj}[2] {\frac {\partial^2 {#1}} {\partial {x_{#2}}^2}}
%\newcommand{\dtheta}[1]{\frac{d {#1}}{d \theta}}
%\newcommand{\dt}[1]{\dot{#1}}
%\newcommand{\dt}[1]{\frac{d {#1}}{dt}}
%\newcommand{\dxj}[2] {\frac {\partial {#1}} {\partial {x_{#2}}}}
%\newcommand{\halfPhi}[0]{\frac{\phi}{2}}
%\newcommand{\half}[0]{\inv{2}}
%\newcommand{\inv}[1]{\frac{1}{#1}}
%\newcommand{\laplacian}[0]{\nabla^2}
%\newcommand{\matrixoftx}[3]{
%\newcommand{\nrrp}[0]{\norm{\rcap \wedge \Br'}}
%\newcommand{\oiint}{\bigcirc \hspace{-1.4em} \int \hspace{-.8em} \int}
%\newcommand{\transpose}[1]{{#1}^{\text{T}}}
%\newcommand{\transpose}[1]{{{#1}^{\TextTranspose}}}
%\newcommand{\transpose}[1]{{{#1}^{\text{T}}}}
%\newcommand{\barA}[0]{\bar{A}}
%\newcommand{\qbar}[0]{\bar{q}}
%\newcommand{\qdotbar}[0]{\dot{\bar{q}}}
%
%</infrequent>





\usepackage[bookmarks=true]{hyperref}

\usepackage{color,cite,graphicx}
   % use colour in the document, put your citations as [1-4]
   % rather than [1,2,3,4] (it looks nicer, and the extended LaTeX2e
   % graphics package. 
\usepackage{latexsym,amssymb,epsf} % don't remember if these are
   % needed, but their inclusion can't do any damage


\title{ polar velocity and accerlation. }
\author{Peeter Joot}
\date{ Jan 13, 2009.  Last Revision: $Date: 2009/01/13 23:25:03 $ }

\begin{document}

\maketitle{}
%\tableofcontents

\section{ Motivation. }

Have previously worked out the radial velocity and acceleration components a pile of different ways in
\cite{PJAngAcc}, 
\cite{PJAngAccCross}, 
\cite{PJAngVel}, 
\cite{PJKeRot}, 
\cite{PJRadialDer}, and
\cite{PJUnitDer}.

So, what's a couple more?

When the motion is strictly restricted to a plane we can get away with doing this either in complex numbers
(used in a number of the Tong Lagrangian solutions), or with a polar form \R{2} vector (a polar representation
I haven't seen since High School).

\section{ With complex numbers. }

Let
\begin{align*}
z = r e^{i\theta}
\end{align*}

So our velocity is

\begin{align*}
\zdot = \rdot e^{i\theta} + i r \thetadot e^{i\theta}
\end{align*}

and the acceleration is
\begin{align*}
\ddot{z}
&= \ddot{r} e^{i\theta} + i \dot{r} \thetadot e^{i\theta}
 + i \rdot \thetadot e^{i\theta}
 + i r \ddot{\theta} e^{i\theta}
 - r \thetadot^2 e^{i\theta} \\
&= (\ddot{r} - r \thetadot^2 ) e^{i\theta} + (2 \dot{r} \thetadot + r \ddot{\theta} ) i e^{i\theta}
\end{align*}

\section{ Plane vector representation. }

Also can do this with polar vector representation directly (without involving the complexity of rotation matrixes or anything fancy)

\begin{align*}
\Br 
&= r 
\begin{bmatrix}
\cos\theta \\
\sin\theta
\end{bmatrix}
\end{align*}

Velocity is then
\begin{align*}
\Bv 
&= 
\rdot 
\begin{bmatrix}
\cos\theta \\
\sin\theta
\end{bmatrix}
+r \thetadot
\begin{bmatrix}
-\sin\theta \\
\cos\theta
\end{bmatrix}
\end{align*}

and for acceleration we have

\begin{align*}
\Ba 
&= 
\ddot{r}
\begin{bmatrix}
\cos\theta \\
\sin\theta
\end{bmatrix}
+\rdot \thetadot
\begin{bmatrix}
-\sin\theta \\
\cos\theta
\end{bmatrix}
+\rdot \thetadot
\begin{bmatrix}
-\sin\theta \\
\cos\theta
\end{bmatrix}
+r \ddot{\theta}
\begin{bmatrix}
-\sin\theta \\
\cos\theta
\end{bmatrix}
-r \thetadot^2
\begin{bmatrix}
\cos\theta \\
\sin\theta 
\end{bmatrix} \\
&=
(\ddot{r} -r \thetadot^2)
\begin{bmatrix}
\cos\theta \\
\sin\theta
\end{bmatrix}
+(2\rdot \thetadot +r \ddot{\theta})
\begin{bmatrix}
-\sin\theta \\
\cos\theta
\end{bmatrix}
\end{align*}

\bibliographystyle{plainnat}
\bibliography{myrefs}

\end{document}

\documentclass{article}      % Specifies the document class

\usepackage{amsmath}

%
% shorthand for bold symbols, convenient for vectors and matrices
%
\newcommand{\Ba}[0]{\mathbf{a}}
\newcommand{\Bb}[0]{\mathbf{b}}
\newcommand{\Bc}[0]{\mathbf{c}}
\newcommand{\Bd}[0]{\mathbf{d}}
\newcommand{\Be}[0]{\mathbf{e}}
\newcommand{\Bf}[0]{\mathbf{f}}
\newcommand{\Bg}[0]{\mathbf{g}}
\newcommand{\Bh}[0]{\mathbf{h}}
\newcommand{\Bi}[0]{\mathbf{i}}
\newcommand{\Bj}[0]{\mathbf{j}}
\newcommand{\Bk}[0]{\mathbf{k}}
\newcommand{\Bl}[0]{\mathbf{l}}
\newcommand{\Bm}[0]{\mathbf{m}}
\newcommand{\Bn}[0]{\mathbf{n}}
\newcommand{\Bo}[0]{\mathbf{o}}
\newcommand{\Bp}[0]{\mathbf{p}}
\newcommand{\Bq}[0]{\mathbf{q}}
\newcommand{\Br}[0]{\mathbf{r}}
\newcommand{\Bs}[0]{\mathbf{s}}
\newcommand{\Bt}[0]{\mathbf{t}}
\newcommand{\Bu}[0]{\mathbf{u}}
\newcommand{\Bv}[0]{\mathbf{v}}
\newcommand{\Bw}[0]{\mathbf{w}}
\newcommand{\Bx}[0]{\mathbf{x}}
\newcommand{\By}[0]{\mathbf{y}}
\newcommand{\Bz}[0]{\mathbf{z}}
\newcommand{\BA}[0]{\mathbf{A}}
\newcommand{\BB}[0]{\mathbf{B}}
\newcommand{\BC}[0]{\mathbf{C}}
\newcommand{\BD}[0]{\mathbf{D}}
\newcommand{\BE}[0]{\mathbf{E}}
\newcommand{\BF}[0]{\mathbf{F}}
\newcommand{\BG}[0]{\mathbf{G}}
\newcommand{\BH}[0]{\mathbf{H}}
\newcommand{\BI}[0]{\mathbf{I}}
\newcommand{\BJ}[0]{\mathbf{J}}
\newcommand{\BK}[0]{\mathbf{K}}
\newcommand{\BL}[0]{\mathbf{L}}
\newcommand{\BM}[0]{\mathbf{M}}
\newcommand{\BN}[0]{\mathbf{N}}
\newcommand{\BO}[0]{\mathbf{O}}
\newcommand{\BP}[0]{\mathbf{P}}
\newcommand{\BQ}[0]{\mathbf{Q}}
\newcommand{\BR}[0]{\mathbf{R}}
\newcommand{\BS}[0]{\mathbf{S}}
\newcommand{\BT}[0]{\mathbf{T}}
\newcommand{\BU}[0]{\mathbf{U}}
\newcommand{\BV}[0]{\mathbf{V}}
\newcommand{\BW}[0]{\mathbf{W}}
\newcommand{\BX}[0]{\mathbf{X}}
\newcommand{\BY}[0]{\mathbf{Y}}
\newcommand{\BZ}[0]{\mathbf{Z}}

\newcommand{\Bzero}[0]{\mathbf{0}}
\newcommand{\Btheta}[0]{\boldsymbol{\theta}}
\newcommand{\Btau}[0]{\boldsymbol{\tau}}
\newcommand{\Bomega}[0]{\boldsymbol{\omega}}

%
% shorthand for unit vectors
%
\newcommand{\acap}[0]{\hat{\Ba}}
\newcommand{\bcap}[0]{\hat{\Bb}}
\newcommand{\ccap}[0]{\hat{\Bc}}
\newcommand{\dcap}[0]{\hat{\Bd}}
\newcommand{\ecap}[0]{\hat{\Be}}
\newcommand{\fcap}[0]{\hat{\Bf}}
\newcommand{\gcap}[0]{\hat{\Bg}}
\newcommand{\hcap}[0]{\hat{\Bh}}
\newcommand{\icap}[0]{\hat{\Bi}}
\newcommand{\jcap}[0]{\hat{\Bj}}
\newcommand{\kcap}[0]{\hat{\Bk}}
\newcommand{\lcap}[0]{\hat{\Bl}}
\newcommand{\mcap}[0]{\hat{\Bm}}
\newcommand{\ncap}[0]{\hat{\Bn}}
\newcommand{\ocap}[0]{\hat{\Bo}}
\newcommand{\pcap}[0]{\hat{\Bp}}
\newcommand{\qcap}[0]{\hat{\Bq}}
\newcommand{\rcap}[0]{\hat{\Br}}
\newcommand{\scap}[0]{\hat{\Bs}}
\newcommand{\tcap}[0]{\hat{\Bt}}
\newcommand{\ucap}[0]{\hat{\Bu}}
\newcommand{\vcap}[0]{\hat{\Bv}}
\newcommand{\wcap}[0]{\hat{\Bw}}
\newcommand{\xcap}[0]{\hat{\Bx}}
\newcommand{\ycap}[0]{\hat{\By}}
\newcommand{\zcap}[0]{\hat{\Bz}}
\newcommand{\thetacap}[0]{\hat{\Btheta}}

%
% to write R^n and C^n in a distinguishable fashion.  Perhaps change this
% to the double lined characters upon figuring out how to do so.
%
\newcommand{\C}[1]{${\BC}^{#1}$}
\newcommand{\R}[1]{${\BR}^{#1}$}

%
% various generally useful helpers
%

% derivative of #1 wrt. #2:
\newcommand{\D}[2] {\frac {d#2} {d#1}}

\newcommand{\inv}[1]{\frac{1}{#1}}
\newcommand{\cross}[0]{\times}

\newcommand{\abs}[1]{\lvert#1\rvert}
\newcommand{\norm}[1]{\lVert#1\rVert}
\newcommand{\innerprod}[2]{\langle{#1}, {#2}\rangle}
\newcommand{\dotprod}[2]{#1 \cdot #2}
\newcommand{\crossprod}[2]{#1 \cross #2}
\newcommand{\tripleprod}[3]{\dotprod{\crossprod{#1}{#2}}{#3}}

%
% A few miscellaneous things specific to this document
%
\newcommand{\crossop}[1]{\crossprod{#1}{}}

\newcommand{\PDP}[2]{\BP^{#1}\BD{\BP^{#2}}}
\newcommand{\PDPDP}[3]{\Bv^T\BP^{#1}\BD\BP^{#2}\BD\BP^{#3}\Bv}

\newcommand{\Mp}[0]{
\begin{bmatrix}
0 & 1 & 0 & 0 \\
0 & 0 & 1 & 0 \\
0 & 0 & 0 & 1 \\
1 & 0 & 0 & 0
\end{bmatrix}
}
\newcommand{\Mpp}[0]{
\begin{bmatrix}
0 & 0 & 1 & 0 \\
0 & 0 & 0 & 1 \\
1 & 0 & 0 & 0 \\
0 & 1 & 0 & 0
\end{bmatrix}
}
\newcommand{\Mppp}[0]{
\begin{bmatrix}
0 & 0 & 0 & 1 \\
1 & 0 & 0 & 0 \\
0 & 1 & 0 & 0 \\
0 & 0 & 1 & 0
\end{bmatrix}
}
\newcommand{\Mpu}[0]{
\begin{bmatrix}
u_1 & 0 & 0 & 0 \\
0 & u_2 & 0 & 0 \\
0 & 0 & u_3 & 0 \\
0 & 0 & 0 & u_4
\end{bmatrix}
}

%
% The real thing:
%

                             % The preamble begins here.
\title{ Covariant/vector derivative notes, plus notes on raised and lowered indexes.  }
\author{Peeter Joot}         % Declares the author's name.

%\date{}        % Deleting this command produces today's date.

\begin{document}             % End of preamble and beginning of text.

\maketitle{}

\section{Motivation.}

My notes on tensors, mostly from Geometric Algebra for Physicists.  Write up enough notes for myself that I can understand the topics (if I can't explain to myself I don't understand sufficiently).  Conclude with the
solution of problem 6.1 to demonstrate the frame independence of the
covariant derivative.

\subsection{ Raised and lowered indexes. Coordinates of vectors with non-orthonormal frames. }

Let $\{ e_i \}$ represent a frame of not necessarily orthonormal basis vectors for a metric space, and $\{ e^i \}$ represent the reciprocal frame.

The reciprocal frame vectors are defined by the relation:

\begin{equation}
e_i \cdot e^j = {\delta_i}^j.
\end{equation}

Lets compute the coordinates of a vector $x$ in terms of both frames:

\[
x = \sum \alpha_j e_j = \sum \beta_j e^j
\]

Forming $x \cdot e^i$, and $x \cdot e_i$ respectively solves for the $\alpha$, and $\beta$ coefficients

\[
x \cdot e^i = \sum \alpha_j e_j \cdot e^i = \sum \alpha_j {\delta_j}^i = \alpha_i
\]

\[
x \cdot e_i = \sum \beta_j e^j \cdot e_i = \sum \beta_j {\delta_i}^j = \beta_i
\]

Thus, the reciprocal frame vectors allow for simple determination of coordinates for an arbitrary frame. We can summarize this as follows:

\[
x = \sum ( x \cdot e^i ) e_i = \sum ( x \cdot e_i ) e^i
\]

Now, for orthonormal frames, where $e_i = e^i$ we are used to writing:

\[
x = \sum x_i e_i,
\]

however for non-orthonormal frames the convention is to mix raised and lowered indexes as follows:

\[
x = \sum x^i e_i = \sum x_i e^i.
\]

Where, as demonstrated above these generalized coordinates have the values, $x^i = x \cdot e^i$, and $x_i = x \cdot e_i$.  This is a strange seeming notation at
first especially since most of linear algebra is done with always lowered (or always upper for some authors) indexes.  However one quickly gets used to it, after seeing how powerful the reciprocal frame concept is for dealing with non-orthonormal frames (otherwise one has to drag along matrices and their inverses to express the same vector decompositions).

\subsection{ Metric tensor. }

It is customary in tensor formulations of physics to utilize a metric tensor to express the dot product.

Compute the dot product using the coordinate vectors

\[
x \cdot y = \left(\sum x^i e_i \right)\left(\sum y^j e_j \right) = \sum x^i y^j \left( e_i \cdot e_j \right)
\]

\[
x \cdot y = \left(\sum x_i e^i \right)\left(\sum y_j e^j \right) = \sum x_i y_j \left( e^i \cdot e^j \right)
\]

Introducing second rank (symmetric) tensors for the dot product pairs $ e_i \cdot e_j = g_{ij}$, and $ g^{ij} = e^i \cdot e^j $ we have

\[
x \cdot y = \sum x_i y_j g^{ij} = \sum x^i y^j g_{ij} = \sum x_i y^i = \sum x^i y_i
\]

We see that the metric tensor provides a way to specify the dot product in index notation, and removes the explicit references to the original frame vectors.  Mixed indexes also removes the references to the original frame vectors, but additionally eliminates the need for either of the metric tensors.

Note that it is also common to see Einstein summation convention employed, which omits the $\sum$:

\[
x \cdot y = x_i y_j g^{ij} = x^i y^j g_{ij} = x^i y_i = x_i y^i
\]

Here summation over all matched upper, lower index pairs is implied.

\subsection{ Metric tensor relations to coordinates. }

Given a coordinate expression of a vector, we dot that with the frame vectors to observe the relation between coordinates and the metric tensor:

\[
x \cdot e_i = \sum x^j e_j \cdot e_i = \sum x^j g_{ij}
\]

\[
x \cdot e^i = \sum x_j e^j \cdot e^i = \sum x_j g^{ij}
\]

The metric tensors can therefore be used be used to express the relations between the upper and lower index coordinates:

\begin{align}
x_i &= \sum g_{ij} x^j \label{eqn:metric_upper_to_lower} \\
x^i &= \sum g^{ij} x_j \label{eqn:metric_lower_to_upper}
\end{align}

It is therefore apparent that the matrix of the index lowered metric tensor $g_{ij}$ is the inverse of the matrix for the raised index metric tensor $g^{ij}$.

\subsection{ Metric tensor as a Jacobian }

The relations of equations \ref{eqn:metric_upper_to_lower}, and \ref{eqn:metric_lower_to_upper} show that the metric tensor can be expressed in terms of partial derivatives:

\begin{align}
\frac{\partial x_i }{\partial x^j } &= g_{ij} \\
\frac{\partial x^i }{\partial x_j } &= g^{ij}
\end{align}

Therefore the metric tensors can also be expressed as Jacobian matrices (not Jacobian determinants) :

\begin{align}
g_{ij} &= \frac{\partial (x_1, \cdots, x_n) }{\partial (x^1, \cdots, x^n) } \\
g^{ij} &= \frac{\partial (x^1, \cdots, x^n) }{\partial (x_1, \cdots, x_n) }
\end{align}

Will this be useful in any way?

\subsection{ Change of basis. }

To perform a change of basis from one non-orthonormal basis $\{e_i\}$ to a second $\{f_i\}$, relations between the sets of vectors
are required.  Using Greek indexes for the $f$ frame, and English for the $e$ frame, those are:

\begin{align*}
e_i 		&= \sum f^{\mu} e_i \cdot f_{\mu} 	= \sum f_{\mu} e_i \cdot f^{\mu} \\
f_{\alpha} 	&= \sum e^k f_{\alpha} \cdot e_k 	= \sum e_k f_{\alpha} \cdot e^k \\
e^i 		&= \sum f^{\mu} e^i \cdot f_{\mu} 	= \sum f_{\mu} e^i \cdot f^{\mu} \\
f^{\alpha} 	&= \sum e^k f^{\alpha} \cdot e_k 	= \sum e_k f^{\alpha} \cdot e^k 
\end{align*}

Following GAFP we can write the dot product terms as a second order tensors $f$ (ie: matrix relation) :

\begin{align*}
e_i 		&= \sum f^{\mu} f_{i\mu}  	= \sum f_{\mu} {f_i}^{\mu} \\
f_{\alpha} 	&= \sum e^k f_{k\alpha} 	= \sum e_k {f^k}_{\alpha} \\
e^i 		&= \sum f^{\mu} {f^i}_{\mu} 	= \sum f_{\mu} f^{i \mu} \\
f^{\alpha} 	&= \sum e^k {f_k}^{\alpha}  	= \sum e_k f^{k\alpha}
\end{align*}

Looking at these relations in pairs, such as

\begin{align*}
f_{\alpha} 	&= \sum e^k f_{k\alpha} \\ 
e^i 		&= \sum f_{\mu} f^{i \mu} 
\end{align*}

and 

\begin{align*}
e_i 		&= \sum f^{\mu} f_{i\mu} \\
f^{\alpha} 	&= \sum e_k f^{k\alpha}
\end{align*}

It is clear that $f_{i\alpha}$ is the inverse of $f^{i\alpha}$.  FIXME: write this out explicitly in index notation, to specify
more exactly the inverse relationship ... that will help clarify the covariant derivative stuff later.  There are also inverse relationships for the mixed index tensors above.  Can those be used in the covariant derivative calculation to simplify things?

Note that all these various tensors are related to each other using the metric tensors for $f$ and $e$.  FIXME: show example.  Also note that using this notation the metric tensors $g_{ij}$ and $g_{\alpha\beta}$ are two completely different linear functions, and careful use of the index conventions are required to keep these straight.

\subsection{ Covariant derivative. }

GAFP exercise 6.1.  Show that the vector derivative:

\begin{equation}
\nabla = \sum e^i \frac{\partial}{\partial x^i}
\end{equation}

is not frame dependent.

To show this we will need to utilize the chain rule to rewrite the partials in terms of the alternate frame:

\begin{align*}
\frac{\partial}{\partial x^i} &= \sum \frac{\partial x^\alpha}{\partial x^i} \frac{\partial}{\partial x^\alpha} 
\end{align*}

To evaluate the first partial here, we write the coordinates of a vector in terms of both, and take dot products:

\begin{align*}
\left(\sum x^{\gamma} f_{\gamma}\right) \cdot f^{\alpha} = \left(\sum x^i e_i\right) \cdot f^{\alpha} \\
\end{align*}
\begin{align*}
x^{\alpha} &= \sum x^i {f_i}^{\alpha} \\
\end{align*}
\begin{align*}
\frac{\partial x^{\alpha}}{\partial x^i} &= {f_i}^{\alpha}
\end{align*}

Similar expressions for the other change of basis tensors is also possible, but
not required for this problem.

With this result we have the partial reexpressed in terms of coordinates
in the new frame.

\begin{align*}
\frac{\partial}{\partial x^i} &= \sum {f_i}^{\alpha} \frac{\partial}{\partial x^\alpha} 
\end{align*}

Combine this with the alternate contravariant frame vector as calculated above:

\[
e^i = \sum f^{\mu} {f^i}_{\mu}
\]

and we have:

\begin{align*}
\sum_i e^i \frac{\partial}{\partial x^i}
&= \sum_i \left(\sum_{\mu} f^{\mu} {f^i}_{\mu} \right) \left( \sum_{\alpha} {f_i}^{\alpha} \frac{\partial}{\partial x^\alpha}\right) \\
&= \sum_{\mu \alpha} \left(f^{\mu} \frac{\partial}{\partial x^\alpha} \right) \sum_i {f^i}_{\mu} {f_i}^{\alpha} \\
&= \sum_{\mu \alpha} \left(f^{\mu} \frac{\partial}{\partial x^\alpha} \right) {\delta_{\mu}}^{\alpha} \\
&= \sum_{\alpha} f^{\alpha} \frac{\partial}{\partial x^\alpha} \\
\end{align*}

FIXME: Proper justification of the step $\sum_i {f^i}_{\mu} {f_i}^{\alpha} = {\delta_{\mu}}^{\alpha}$ is missing.  This is possible due
to the imprecisely noted inverse relationships pointed out above.

Note that my original paper derivation of the above used only the tensors $f_{i\alpha}$, and $f^{i\alpha}$ instead of the mixed index versions used here.  That worked but also required a pair of metric tensors, and one more step to sum over those tensors to get at the final result.

\end{document}               % End of document.

\documentclass{article}      % Specifies the document class

\usepackage{amsmath}
\usepackage{mathpazo}

%
% shorthand for bold symbols, convenient for vectors and matrices
%
\newcommand{\Ba}[0]{\mathbf{a}}
\newcommand{\Bb}[0]{\mathbf{b}}
\newcommand{\Bc}[0]{\mathbf{c}}
\newcommand{\Bd}[0]{\mathbf{d}}
\newcommand{\Be}[0]{\mathbf{e}}
\newcommand{\Bf}[0]{\mathbf{f}}
\newcommand{\Bg}[0]{\mathbf{g}}
\newcommand{\Bh}[0]{\mathbf{h}}
\newcommand{\Bi}[0]{\mathbf{i}}
\newcommand{\Bj}[0]{\mathbf{j}}
\newcommand{\Bk}[0]{\mathbf{k}}
\newcommand{\Bl}[0]{\mathbf{l}}
\newcommand{\Bm}[0]{\mathbf{m}}
\newcommand{\Bn}[0]{\mathbf{n}}
\newcommand{\Bo}[0]{\mathbf{o}}
\newcommand{\Bp}[0]{\mathbf{p}}
\newcommand{\Bq}[0]{\mathbf{q}}
\newcommand{\Br}[0]{\mathbf{r}}
\newcommand{\Bs}[0]{\mathbf{s}}
\newcommand{\Bt}[0]{\mathbf{t}}
\newcommand{\Bu}[0]{\mathbf{u}}
\newcommand{\Bv}[0]{\mathbf{v}}
\newcommand{\Bw}[0]{\mathbf{w}}
\newcommand{\Bx}[0]{\mathbf{x}}
\newcommand{\By}[0]{\mathbf{y}}
\newcommand{\Bz}[0]{\mathbf{z}}
\newcommand{\BA}[0]{\mathbf{A}}
\newcommand{\BB}[0]{\mathbf{B}}
\newcommand{\BC}[0]{\mathbf{C}}
\newcommand{\BD}[0]{\mathbf{D}}
\newcommand{\BE}[0]{\mathbf{E}}
\newcommand{\BF}[0]{\mathbf{F}}
\newcommand{\BG}[0]{\mathbf{G}}
\newcommand{\BH}[0]{\mathbf{H}}
\newcommand{\BI}[0]{\mathbf{I}}
\newcommand{\BJ}[0]{\mathbf{J}}
\newcommand{\BK}[0]{\mathbf{K}}
\newcommand{\BL}[0]{\mathbf{L}}
\newcommand{\BM}[0]{\mathbf{M}}
\newcommand{\BN}[0]{\mathbf{N}}
\newcommand{\BO}[0]{\mathbf{O}}
\newcommand{\BP}[0]{\mathbf{P}}
\newcommand{\BQ}[0]{\mathbf{Q}}
\newcommand{\BR}[0]{\mathbf{R}}
\newcommand{\BS}[0]{\mathbf{S}}
\newcommand{\BT}[0]{\mathbf{T}}
\newcommand{\BU}[0]{\mathbf{U}}
\newcommand{\BV}[0]{\mathbf{V}}
\newcommand{\BW}[0]{\mathbf{W}}
\newcommand{\BX}[0]{\mathbf{X}}
\newcommand{\BY}[0]{\mathbf{Y}}
\newcommand{\BZ}[0]{\mathbf{Z}}

\newcommand{\Bzero}[0]{\mathbf{0}}
\newcommand{\Btheta}[0]{\boldsymbol{\theta}}
\newcommand{\Btau}[0]{\boldsymbol{\tau}}
\newcommand{\Bomega}[0]{\boldsymbol{\omega}}

%
% shorthand for unit vectors
%
\newcommand{\acap}[0]{\hat{\Ba}}
\newcommand{\bcap}[0]{\hat{\Bb}}
\newcommand{\ccap}[0]{\hat{\Bc}}
\newcommand{\dcap}[0]{\hat{\Bd}}
\newcommand{\ecap}[0]{\hat{\Be}}
\newcommand{\fcap}[0]{\hat{\Bf}}
\newcommand{\gcap}[0]{\hat{\Bg}}
\newcommand{\hcap}[0]{\hat{\Bh}}
\newcommand{\icap}[0]{\hat{\Bi}}
\newcommand{\jcap}[0]{\hat{\Bj}}
\newcommand{\kcap}[0]{\hat{\Bk}}
\newcommand{\lcap}[0]{\hat{\Bl}}
\newcommand{\mcap}[0]{\hat{\Bm}}
\newcommand{\ncap}[0]{\hat{\Bn}}
\newcommand{\ocap}[0]{\hat{\Bo}}
\newcommand{\pcap}[0]{\hat{\Bp}}
\newcommand{\qcap}[0]{\hat{\Bq}}
\newcommand{\rcap}[0]{\hat{\Br}}
\newcommand{\scap}[0]{\hat{\Bs}}
\newcommand{\tcap}[0]{\hat{\Bt}}
\newcommand{\ucap}[0]{\hat{\Bu}}
\newcommand{\vcap}[0]{\hat{\Bv}}
\newcommand{\wcap}[0]{\hat{\Bw}}
\newcommand{\xcap}[0]{\hat{\Bx}}
\newcommand{\ycap}[0]{\hat{\By}}
\newcommand{\zcap}[0]{\hat{\Bz}}
\newcommand{\thetacap}[0]{\hat{\Btheta}}

%
% to write R^n and C^n in a distinguishable fashion.  Perhaps change this
% to the double lined characters upon figuring out how to do so.
%
\newcommand{\C}[1]{$\mathbb{C}^{#1}$}
\newcommand{\R}[1]{$\mathbb{R}^{#1}$}

%
% various generally useful helpers
%

% derivative of #1 wrt. #2:
\newcommand{\D}[2] {\frac {d#2} {d#1}}

\newcommand{\inv}[1]{\frac{1}{#1}}
\newcommand{\cross}[0]{\times}

\newcommand{\abs}[1]{\lvert{#1}\rvert}
\newcommand{\norm}[1]{\lVert{#1}\rVert}
\newcommand{\innerprod}[2]{\langle{#1}, {#2}\rangle}
\newcommand{\dotprod}[2]{{#1} \cdot {#2}}
\newcommand{\bdotprod}[2]{\left({#1} \cdot {#2}\right)}
\newcommand{\crossprod}[2]{{#1} \cross {#2}}
\newcommand{\tripleprod}[3]{\dotprod{\left(\crossprod{#1}{#2}\right)}{#3}}

\DeclareMathOperator{\Proj}{Proj}
\DeclareMathOperator{\Span}{span}
\DeclareMathOperator{\Sgn}{sgn}
\DeclareMathOperator{\Area}{Area}
\DeclareMathOperator{\Volume}{Volume}

%
% A few miscellaneous things specific to this document
%
\newcommand{\crossop}[1]{\crossprod{#1}{}}

% R2 vector.
\newcommand{\VectorTwo}[2]{
\begin{bmatrix}
 {#1} \\
 {#2}
\end{bmatrix}
}

\newcommand{\VectorN}[1]{
\begin{bmatrix}
{#1}_1 \\
{#1}_2 \\
\vdots \\
{#1}_N \\
\end{bmatrix}
}

\newcommand{\DETuvij}[4]{
\begin{vmatrix}
 {#1}_{#3} & {#1}_{#4} \\
 {#2}_{#3} & {#2}_{#4}
\end{vmatrix}
}

\newcommand{\DETuvwijk}[6]{
\begin{vmatrix}
 {#1}_{#4} & {#1}_{#5} & {#1}_{#6} \\
 {#2}_{#4} & {#2}_{#5} & {#2}_{#6} \\
 {#3}_{#4} & {#3}_{#5} & {#3}_{#6}
\end{vmatrix}
}

\newcommand{\DETuvwxijkl}[8]{
\begin{vmatrix}
 {#1}_{#5} & {#1}_{#6} & {#1}_{#7} & {#1}_{#8} \\
 {#2}_{#5} & {#2}_{#6} & {#2}_{#7} & {#2}_{#8} \\
 {#3}_{#5} & {#3}_{#6} & {#3}_{#7} & {#3}_{#8} \\
 {#4}_{#5} & {#4}_{#6} & {#4}_{#7} & {#4}_{#8} \\
\end{vmatrix}
}

%\newcommand{\DETuvwxyijklm}[10]{
%\begin{vmatrix}
% {#1}_{#6} & {#1}_{#7} & {#1}_{#8} & {#1}_{#9} & {#1}_{#10} \\
% {#2}_{#6} & {#2}_{#7} & {#2}_{#8} & {#2}_{#9} & {#2}_{#10} \\
% {#3}_{#6} & {#3}_{#7} & {#3}_{#8} & {#3}_{#9} & {#3}_{#10} \\
% {#4}_{#6} & {#4}_{#7} & {#4}_{#8} & {#4}_{#9} & {#4}_{#10} \\
% {#5}_{#6} & {#5}_{#7} & {#5}_{#8} & {#5}_{#9} & {#5}_{#10}
%\end{vmatrix}
%}

% R3 vector.
\newcommand{\VectorThree}[3]{
\begin{bmatrix}
 {#1} \\
 {#2} \\
 {#3}
\end{bmatrix}
}



%
% The real thing:
%

                             % The preamble begins here.
\title{Inertia Tensor} % Declares the document's title.
\author{Peeter Joot}         % Declares the author's name.
%\date{}        % Deleting this command produces today's date.

\begin{document}             % End of preamble and beginning of text.

\maketitle{}

\section{}

GAFP derives the angular momentum for rotational motion in the following form

\[
L = R \left( \int \Bx \wedge (\Bx \cdot \Omega_B) dm \right) R^\dagger
\]

and calls the integral part, the inertia tensor

\[
\emph{I}(B) = \int \Bx \wedge (\Bx \cdot \Omega_B) dm
\]

which is a linear mapping from bivectors to bivectors.  To understand the
form of this I found it helpful to expanding the wedge product
part of this explicitly for the \R{3} case.

Ignoring the sum in this expansion write

\[
f(B) = \Bx \wedge (\Bx \cdot B)
\]

And writing $\Be_{ij} = \Be_i \Be_j$ introduce a basis

\[
b = \{ \Be_1 I, \Be_2 I, \Be_3 I \} = \{ \Be_{23}, \Be_{31}, \Be_{12} \}
\]

for the \R{3} bivector product space.

Now calculate $f(B)$ for each of the basis vectors

\begin{align*}
f(\Be_1 I) 
&= \Bx \wedge (\Bx \cdot \Be_{23}) \\
&= ( x_1 \Be_1 + x_2 \Be_2 + x_3 \Be_3) \wedge (x_2 \Be_3 - x_3 \Be_2) \\
\end{align*}

Completing this calculation for each of the unit basic bivectors, we have
%%+ (           + x_2 \Be_2            ) \wedge (x_2 \Be_3            ) \\
%%+ (                       + x_3 \Be_3) \wedge (          - x_3 \Be_2) \\
%+ ( x_1 \Be_1                        ) \wedge (x_2 \Be_3            ) \\
%+ ( x_1 \Be_1                        ) \wedge (          - x_3 \Be_2) \\
%%+ (           + x_2 \Be_2            ) \wedge (          - x_3 \Be_2) \\ %% = 0
%%+ (                       + x_3 \Be_3) \wedge (x_2 \Be_3            ) \\ %% = 0
\begin{align*}
f(\Be_1 I) &= (x_2^2 + x_3^2) \Be_{23} - (x_1 x_2) \Be_{31} - (x_1 x_3) \Be_{12} \\
f(\Be_2 I) &= -(x_1 x_2) \Be_{23} + (x_1^2 + x_3^2) \Be_{31} - (x_2 x_3) \Be_{12} \\
f(\Be_3 I) &= -(x_1 x_3) \Be_{23} - (x_2 x_3) \Be_{31} - (x_1^2 x_2^2) \Be_{12} \\
\end{align*}

Observe that taking dot products with $(\Be_i I)^\dagger$ will select just the $\Be_i I$ term of the result, so one can
form the matrix of this linear transformation that maps bivectors in basis $b$ to image vectors also in basis $b$ as follows

\[
{\begin{bmatrix}
\emph{I}(B)
\end{bmatrix}}_{b}^{b}
=
{\begin{bmatrix}
\emph{I}(\Be_i I) \cdot (\Be_j I)^\dagger
\end{bmatrix}}_{ij}
=
\int {\begin{bmatrix}
(x_2^2 + x_3^2)  & - (x_1 x_2)  & - (x_1 x_3)  \\
-(x_1 x_2)  & + (x_1^2 + x_3^2)  & - (x_2 x_3)  \\
-(x_1 x_3)  & - (x_2 x_3)  & - (x_1^2 x_2^2)  \\
\end{bmatrix}} dm
\]

Observe that this can also be written in a more typical tensor notation

\[
I_{ij} 
= \emph{I}(\Be_i I) \cdot (\Be_j I)^\dagger 
= \int (\delta_{ij} \Bx^2 - x_i x_j) dm
\]

Where, as usual for tensors, the meaning of the indexes and whether summation is required is implied.  In this case
the coordinate transformation matrix for this linear transformation has components $I_{ij}$ (and no summation).

\subsection{ coordinate transformation matrix for a couple other linear transformations }

Seeing a function of a bivector for the first time is kind of intriging.  We can form the matrix of such a linear transformation
from a basis of the bivector space to the space spanned by function.  For fun, let's calculate that matrix for the basis $b$ above
for the following function:

\[
f(B) = \Be_1 \wedge (\Be_2 \cdot B)
\]

For this function operating on \R{3} bivectors we have:

\begin{align*}
f(\Be_{23}) &= \Be_1 \wedge (\Be_2 \cdot \Be_{23}) = -\Be_{31} \\
f(\Be_{31}) &= \Be_1 \wedge (\Be_2 \cdot \Be_{31}) = 0 \\
f(\Be_{12}) &= \Be_1 \wedge (\Be_2 \cdot \Be_{12}) = 0 \\
\end{align*}

So

\[
{\begin{bmatrix}
f
\end{bmatrix}}_b^b
= 
\begin{bmatrix}
-1 & 0 & 0
\end{bmatrix}
\]

For \R{4} one orthonormal basis is

\[
b = \{ 
\Be_{12},
\Be_{13},
\Be_{14},
\Be_{23},
\Be_{24},
\Be_{34}
\}
\]

A basis for the span of $f$ is $b' = \{
\Be_{13},
\Be_{14}
\}$.  Like any other coordinate transformation associated with a linear transformation we can write the matrix of the transformation that
takes a coordinate vector in one basis into a coordinate vector for the basis for the image:

\[
{\begin{bmatrix}
f(x)
\end{bmatrix}}_{b'}
=
{\begin{bmatrix}
f
\end{bmatrix}}_b^{b'}
{\begin{bmatrix}
x
\end{bmatrix}}_{b}
\] 

For this function $f$ and these pair of basis bivectors we have:

\[
{\begin{bmatrix}
f
\end{bmatrix}}_b^{b'}
= 
\begin{bmatrix}
0 & 0 & 0 & 1 & 0 & 0 \\
0 & 0 & 0 & 0 & 1 & 0 \\
\end{bmatrix}
\]

\subsection{ Equation 3.126 details. }

This statement from GAFP deserves expansion (or at least an exersize):

\[
A \cdot ( \Bx \wedge (\Bx \cdot B) )
= \langle{A \Bx (\Bx \cdot B)}\rangle
= \langle (A \cdot \Bx) \Bx B \rangle
= B \cdot ( \Bx \wedge (\Bx \cdot A) )
\]

Perhaps this is obvious to the author, but wasn't to me.  To clarify this observe the following product

\[
\Bx ( \Bx \cdot B ) = \Bx \cdot ( \Bx \cdot B ) + \Bx \wedge ( \Bx \cdot B ) 
\]

By writing $B = \Bb \Bc = \Bb \wedge \Bc$ we can show that the dot product part of this product is zero:

\begin{align*}
\Bx \cdot ( \Bx \cdot B ) 
&= \Bx \cdot ( (\Bx \cdot \Bb) \Bc - (\Bx \cdot \Bc) \Bb ) \\
&= (\Bx \cdot \Bc) (\Bx \cdot \Bb) - (\Bx \cdot \Bb) (\Bx \cdot \Bc)) \\
&= 0
\end{align*}

This provides the justification for the wedge product removal in the text, since
one can write

\begin{equation}
\Bx \wedge ( \Bx \cdot B ) = \Bx ( \Bx \cdot B )
\end{equation}\label{eqn:inertia_wedge_to_product}

Although it wasn't stated in the text (\ref{eqn:inertia_wedge_to_product}), can
be used to put this inertia product in a pure dot product form

\begin{align*}
A^\dagger \cdot (\Bx \wedge (\Bx \cdot B) )
&= -\langle {A \Bx (\Bx \cdot B)} \rangle \\
&= -\langle (A \cdot \Bx + A \wedge \Bx)(\Bx \cdot B) \rangle \\
\end{align*}

The trivector vector product has only vector and trivector components
\[
(A \wedge \Bx)(\Bx \cdot B) = \langle{ (A \wedge \Bx)(\Bx \cdot B)}\rangle_1 + \langle{(A \wedge \Bx)(\Bx \cdot B)}\rangle_3
\]

So $\langle{(A \wedge \Bx)(\Bx \cdot B)}\rangle_0 = 0$, and one can write

\begin{align*}
A^\dagger \cdot (\Bx \wedge (\Bx \cdot B) )
&= - (A \cdot \Bx ) \cdot (\Bx \cdot B) \\
&= (\Bx \cdot A) \cdot (\Bx \cdot B) \\
\end{align*}

As pointed out in the text this is symetric.  That can't be more clear than above.

\end{document}               % End of document.

%
% Copyright � 2012 Peeter Joot.  All Rights Reserved.
% Licenced as described in the file LICENSE under the root directory of this GIT repository.
%

%
%
\chapter{Satellite triangulation over sphere}
\index{triangulation}
\label{chap:locateSatellite}
%\date{April 13, 2008.  locateSatellite.tex}

\section{Motivation and preparation}

Was playing around with what is probably traditionally a spherical trig type problem using geometric algebra (locate satellite position using angle measurements from two well separated points).  Origin of the problem was just me looking at my Feynman Lectures introduction where there is a diagram illustrating how triangulation could be used to locate "Sputnik" and thought I had try such a calculation, but in a way that I thought was more realistic.

\imageFigure{../../figures/gabook/satellite}{Satellite location by measuring direction from two points}{fig:satellite}{0.4}

\Cref{fig:satellite} illustrates the problem I attempted to solve.  Pick two arbitrary points \(P_1\), and \(P_2\) on the globe, separated far enough that the curvature of the earth may be a factor.
For this problem it is assumed that the angles to the satellite will be measured concurrently.

Place a fixed reference frame at the center of the earth.  In the figure this is shown translated to the \((0,0)\) point (equator and prime meridian intersection).  I have picked \(\Be_1\) facing east, \(\Be_2\) facing north, and \(\Be_3\) facing outwards from the core.

Each point \(P_i\) can be located by a rotation along the equatorial plane by angle \(\lambda_i\) (measured with an east facing orientation (direction of \(\Be_1\)), and a rotation \(\psi_i\) towards the north (directed towards \(\Be_2\)).

To identify a point on the surface we translate our \((0,0)\) reference frame to that point using the
following rotor equation:

\begin{equation}\label{eqn:locSat:northrotor}
R_{\psi} = \exp(-\Be_{32}\psi/2) = \cos(\psi/2) - \Be_{32}\sin(\psi/2)
\end{equation}
\begin{equation}\label{eqn:locSat:eastrotor}
R_{\lambda} = \exp(-\Be_{31}\lambda/2) = \cos(\lambda/2) - \Be_{31}\sin(\lambda/2)
\end{equation}
\begin{equation}
R(x) = R_{\psi} R_{\lambda} x R_{\lambda}^\dagger R_{\psi}^\dagger
\end{equation}

To verify that I got the sign of these rotations right, I applied them to the unit vectors using a \(\pi/2\) rotation.  We want the following for the equatorial plane rotation:

\begin{equation*}
R_{\lambda}(\pi/2)
\begin{bmatrix}
\Be_1 \\
\Be_2 \\
\Be_3 \\
\end{bmatrix}
R_{\lambda}(\pi/2)^\dagger
=
\begin{bmatrix}
-\Be_3 \\
\Be_2 \\
\Be_1 \\
\end{bmatrix}
\end{equation*}

And for the northwards rotation:

\begin{equation*}
R_{\psi}(\pi/2)
\begin{bmatrix}
\Be_1 \\
\Be_2 \\
\Be_3 \\
\end{bmatrix}
R_{\psi}(\pi/2)^\dagger
=
\begin{bmatrix}
\Be_1 \\
\Be_2 \\
-\Be_3 \\
\end{bmatrix}
\end{equation*}

Verifying this is simple enough using the explicit sine and cosine expansion of the rotors in \eqnref{eqn:locSat:northrotor} and \eqnref{eqn:locSat:eastrotor}.

Once we have the ability to translate our reference frame to each point on the Earth, we can use the inverse rotation to translate our measured unit vector
to the satellite at that point back to the reference frame.

Suppose one calculates a local unit vector \(\alpha'\) towards the satellite by measuring direction cosines in our local reference frame (ie: angle from gravity opposing (up facing) direction, east, and north directions at that point).
Once that is done, that unit vector \(\alpha\) in our reference frame is obtained by inverse rotation:

\begin{equation}
\alpha = R_{\lambda_i}^\dagger R_{\psi_i}^\dagger \alpha' R_{\psi_i} R_{\lambda_i}
\end{equation}

The other place we need this rotation for is to calculate the points \(P_i\) in our reference from (treating this now as being at the core of the earth).  This is just:

\begin{equation}
P_i = R_{\psi} R_{\lambda} A_i \Be_3 R_{\lambda}^\dagger R_{\psi}^\dagger
\end{equation}

Where \(A_i\) is the altitude (relative to the center of the earth) at the point of interest.

\section{Solution}

Solving for the position of the satellite \(P_s\) we have:

\begin{equation}\label{eqn:locateSatellite:20}
P_s = a_1 \alpha_1 + P_1 = a_2 \alpha_2 + P_2
\end{equation}

Solution of this follows directly by taking wedge products.  Solve for \(a_1\) for example, we wedge with \(\alpha_2\) :

\begin{equation}\label{eqn:locateSatellite:40}
a_1 \alpha_1 \wedge \alpha_2 + P_1 \wedge \alpha_2 = a_2 \mathLabelBox{\alpha_2 \wedge \alpha_2}{\(=0\)} + P_2 \wedge \alpha_2
\end{equation}

Provided the points are far enough apart to get distinct \(\alpha_i\) measurements, then we have:
\begin{equation}\label{eqn:locateSatellite:60}
a_1 = \frac{(P_2-P_1) \wedge \alpha_2}{ \alpha_1 \wedge \alpha_2 }.
\end{equation}

Thus the position vector from the core of earth reference frame to the satellite is:

\begin{equation}
P_s = \left(\frac{(P_2-P_1) \wedge \alpha_2}{ \alpha_1 \wedge \alpha_2 }\right) \alpha_1 + P_1
\end{equation}

Notice how all the trigonometry is encoded directly in the rotor equations.  If one had to calculate all this using the spherical trigonometry generalized triangle relations I expect that you would have an ungodly mess of sine and cosines here.

This demonstrates two very distinct applications of the wedge product.  The first was to define an oriented plane, and was used as a generator of rotations (very much like the unit imaginary).  This second application, to solve linear equations takes advantage of \(a \wedge a = 0\) property of the wedge product.  It was convenient as it allowed simple simultaneous solution of the three equations (one for each component) and two unknowns problem in this particular case.

\section{matrix formulation}

Instead of solving with the wedge product one could formulate this as a matrix equation:

\begin{equation}\label{eqn:locSat:twopointsmatrix}
\begin{bmatrix}
\alpha_1 & -\alpha_2 \\
\end{bmatrix}
\begin{bmatrix}
a_1 \\
a_2 \\
\end{bmatrix}
=
\begin{bmatrix}
P_2 - P_1 \\
\end{bmatrix}
\end{equation}

This highlights the fact that the equations are over-specified, which is more obvious still when this is written out in component form:

\begin{equation}
\begin{bmatrix}
\alpha_{11} & -\alpha_{21} \\
\alpha_{12} & -\alpha_{22} \\
\alpha_{13} & -\alpha_{23} \\
\end{bmatrix}
\begin{bmatrix}
a_1 \\
a_2 \\
\end{bmatrix}
=
\begin{bmatrix}
P_{21} - P_{11} \\
P_{22} - P_{12} \\
P_{23} - P_{13} \\
\end{bmatrix}
\end{equation}

We have one more equation than we need to actually solve it, and cannot use matrix inversion directly (Gaussian elimination or a generalized inverse is required).

Recall the figure in the Feynman lectures when the observation points and the satellite are all in the same plane.  For that all that was needed was two angles, whereas we have measured six for each of the direction cosines used above, so the fact that our equations can include more info than required to solve the problem is not unexpected.

We could also generalize this, perhaps to remove measurement error, by utilizing more than two observation points.  This will compound the over-specification of the equations, and makes it clear that we likely want a least squares approach to solve it.
Here is an example of the matrix to solve for three points:

\begin{equation}\label{eqn:locSat:threepointsmatrix}
\begin{bmatrix}
\alpha_1 & -\alpha_2 & 0 \\
-\alpha_1 & 0 & \alpha_3 \\
0 & \alpha_2 & -\alpha_3 \\
\end{bmatrix}
\begin{bmatrix}
a_1 \\
a_2 \\
a_3 \\
\end{bmatrix}
=
\begin{bmatrix}
P_2 - P_1 \\
P_1 - P_3 \\
P_3 - P_2 \\
\end{bmatrix}
\end{equation}

Since the \(\alpha_i\) are vectors, this matrix of rotated direction cosines has dimensions 9 by 3 (just as \eqnref{eqn:locSat:twopointsmatrix} is a 3 by 2 matrix).

\section{Question.  Order of latitude and longitude rotors?}

Looking at a globe, it initially seemed clear to me that these "perpendicular" (abusing the word) rotations could be applied in either order, but their rotors definitely do not commute, so I assume that together the non-commutative bits of the rotors "cancel out".

Question, is it actually true that the end effect of applying these rotors in either order is the same?

\begin{equation}\label{eqn:locateSatellite:80}
x' = R_{\psi} R_{\lambda} x R_{\lambda}^\dagger R_{\psi}^\dagger = R_{\lambda} R_{\psi} x R_{\psi}^\dagger R_{\lambda}^\dagger
\end{equation}

Attempting to show this is true or false by direct brute force expansion is not productive (perhaps would be okay with a symbolic GA calculator).  However, such a direct expansion
of just the rotor products in either order allows for a comparison:

\begin{equation}\label{eqn:locateSatellite:140}
\begin{aligned}
R_{\psi} R_{\lambda}
&= ( \cos(\psi/2) - \Be_{32}\sin(\psi/2) )(\cos(\lambda/2) - \Be_{31}\sin(\lambda/2)) \\
&= \cos(\psi/2)\cos(\lambda/2) - \Be_{32}\sin(\psi/2)\cos(\lambda/2) - \Be_{31} \cos(\psi/2) \sin(\lambda/2) - \Be_{21} \sin(\psi/2) \sin(\lambda/2) \\
\end{aligned}
\end{equation}
\begin{equation}\label{eqn:locateSatellite:160}
\begin{aligned}
R_{\lambda} R_{\psi}
&= (\cos(\lambda/2) - \Be_{31}\sin(\lambda/2)) ( \cos(\psi/2) - \Be_{32}\sin(\psi/2) ) \\
&= \mathLabelBox{\cos(\psi/2)\cos(\lambda/2)}{\(a_0\)} \\
&\qquad + \mathLabelBox{-\Be_{32}\sin(\psi/2)\cos(\lambda/2) - \Be_{31} \cos(\psi/2) \sin(\lambda/2)}{\(A\)} \\
&\qquad + \mathLabelBox{\Be_{21} \sin(\psi/2) \sin(\lambda/2)}{\(B\)}
\end{aligned}
\end{equation}

Observe that these are identical except for an inversion of sign of the \(\Be_{21}\) term.  Using the shorthand above the respective rotations are:

\begin{equation}\label{eqn:locateSatellite:100}
R_{\lambda,\psi}(x) = R_{\psi} R_{\lambda} x R_{\lambda}^\dagger R_{\psi}^\dagger = (a_0 + A - B) x (a_0 -A +B)
\end{equation}

And
\begin{equation}\label{eqn:locateSatellite:120}
R_{\psi,\lambda}(x) = R_{\lambda} R_{\psi} x R_{\psi}^\dagger R_{\lambda}^\dagger = (a_0 + A + B) x (a_0 -A -B)
\end{equation}

And this can be used to disprove the general rotation commutativity.  We take the difference between these two rotation results, and see if it can be shown to equal zero.
Taking differences, also temporarily writing \(a = a_0 + A\), and exploiting a grade one filter since the final result must be a vector we have:

\begin{equation}\label{eqn:locateSatellite:180}
\begin{aligned}
R_{\lambda,\psi}(x) - R_{\psi,\lambda}(x)
&= \gpgradeone{R_{\lambda,\psi}(x) - R_{\psi,\lambda}(x)} \\
&= \gpgradeone{ (a - B) x (a^\dagger +B) -(a + B) x (a^\dagger -B) } \\
&= \gpgradeone{ ( a x a^\dagger -B x B -B x a^\dagger +a x B ) + ( - a x a^\dagger +B x B +a x B -B x a^\dagger ) } \\
&= \gpgradeone{ ( -B x a^\dagger +a x B ) + ( +a x B -B x a^\dagger ) } \\
&= 2\gpgradeone{ -B x a^\dagger +a x B } \\
&= 2\gpgradeone{ -B x (a_0 - A) + (a_0 + A) x B } \\
&= 2 a_0 ( -B x + x B ) + 2 \gpgradeone{ B x A + A x B } \\
&= 4 a_0 x \cdot B + 2 \gpgradeone{ B x A + A x B } \\
&= 4 a_0 x \cdot B + 2 \gpgradeone{ B \cdot x A - A B \cdot x } + 2 \gpgradeone{ B \wedge x A + A x \wedge B }  \\
&= 4 a_0 x \cdot B + 4 (B \cdot x) \cdot A + 2 (B \wedge x) \cdot A + 2 A \cdot (B \wedge x )  \\
&= 4 a_0 x \cdot B + 4 (B \cdot x) \cdot A + 2 (B \wedge x) \cdot A - 2 (B \wedge x ) \cdot A \\
&= 4 a_0 x \cdot B + 4 (B \cdot x) \cdot A \\
&= 4 (B \cdot x) \cdot (-a_0 + A) \\
&= -4 (B \cdot x) \cdot a^\dagger \\
\end{aligned}
\end{equation}

Evaluate this for \(x = \Be_1\) we do not have zero (a vector with \(\Be_2\) and \(\Be_3\) components), and for \(x = \Be_2\) this difference has \(\Be_1\), and \(\Be_3\) components.  However, for \(x = \Be_3\) this is zero.  Thus these
rotations only commute when applied to a vector that is completely normal to the sphere.  This is what messes up the intuition.  Rotating a point (represented by a vector) in either order works fine, but rotating a frame located at the surface back to a different point on the surface, and maintaining the east and north orientations we have to be careful which orientation to use.

So which order is right?  It has to be rotate first in the equatorial plane (\(\lambda)\), then the northwards rotation, where both are great circle rotations.

A numeric confirmation of this is likely prudent.

\documentclass{article}      % Specifies the document class

\usepackage{amsmath}
\usepackage{mathpazo}

%
% shorthand for bold symbols, convenient for vectors and matrices
%
\newcommand{\Ba}[0]{\mathbf{a}}
\newcommand{\Bb}[0]{\mathbf{b}}
\newcommand{\Bc}[0]{\mathbf{c}}
\newcommand{\Bd}[0]{\mathbf{d}}
\newcommand{\Be}[0]{\mathbf{e}}
\newcommand{\Bf}[0]{\mathbf{f}}
\newcommand{\Bg}[0]{\mathbf{g}}
\newcommand{\Bh}[0]{\mathbf{h}}
\newcommand{\Bi}[0]{\mathbf{i}}
\newcommand{\Bj}[0]{\mathbf{j}}
\newcommand{\Bk}[0]{\mathbf{k}}
\newcommand{\Bl}[0]{\mathbf{l}}
\newcommand{\Bm}[0]{\mathbf{m}}
\newcommand{\Bn}[0]{\mathbf{n}}
\newcommand{\Bo}[0]{\mathbf{o}}
\newcommand{\Bp}[0]{\mathbf{p}}
\newcommand{\Bq}[0]{\mathbf{q}}
\newcommand{\Br}[0]{\mathbf{r}}
\newcommand{\Bs}[0]{\mathbf{s}}
\newcommand{\Bt}[0]{\mathbf{t}}
\newcommand{\Bu}[0]{\mathbf{u}}
\newcommand{\Bv}[0]{\mathbf{v}}
\newcommand{\Bw}[0]{\mathbf{w}}
\newcommand{\Bx}[0]{\mathbf{x}}
\newcommand{\By}[0]{\mathbf{y}}
\newcommand{\Bz}[0]{\mathbf{z}}
\newcommand{\BA}[0]{\mathbf{A}}
\newcommand{\BB}[0]{\mathbf{B}}
\newcommand{\BC}[0]{\mathbf{C}}
\newcommand{\BD}[0]{\mathbf{D}}
\newcommand{\BE}[0]{\mathbf{E}}
\newcommand{\BF}[0]{\mathbf{F}}
\newcommand{\BG}[0]{\mathbf{G}}
\newcommand{\BH}[0]{\mathbf{H}}
\newcommand{\BI}[0]{\mathbf{I}}
\newcommand{\BJ}[0]{\mathbf{J}}
\newcommand{\BK}[0]{\mathbf{K}}
\newcommand{\BL}[0]{\mathbf{L}}
\newcommand{\BM}[0]{\mathbf{M}}
\newcommand{\BN}[0]{\mathbf{N}}
\newcommand{\BO}[0]{\mathbf{O}}
\newcommand{\BP}[0]{\mathbf{P}}
\newcommand{\BQ}[0]{\mathbf{Q}}
\newcommand{\BR}[0]{\mathbf{R}}
\newcommand{\BS}[0]{\mathbf{S}}
\newcommand{\BT}[0]{\mathbf{T}}
\newcommand{\BU}[0]{\mathbf{U}}
\newcommand{\BV}[0]{\mathbf{V}}
\newcommand{\BW}[0]{\mathbf{W}}
\newcommand{\BX}[0]{\mathbf{X}}
\newcommand{\BY}[0]{\mathbf{Y}}
\newcommand{\BZ}[0]{\mathbf{Z}}

\newcommand{\Bzero}[0]{\mathbf{0}}
\newcommand{\Btheta}[0]{\boldsymbol{\theta}}
\newcommand{\Btau}[0]{\boldsymbol{\tau}}
\newcommand{\Bomega}[0]{\boldsymbol{\omega}}

%
% shorthand for unit vectors
%
\newcommand{\acap}[0]{\hat{\Ba}}
\newcommand{\bcap}[0]{\hat{\Bb}}
\newcommand{\ccap}[0]{\hat{\Bc}}
\newcommand{\dcap}[0]{\hat{\Bd}}
\newcommand{\ecap}[0]{\hat{\Be}}
\newcommand{\fcap}[0]{\hat{\Bf}}
\newcommand{\gcap}[0]{\hat{\Bg}}
\newcommand{\hcap}[0]{\hat{\Bh}}
\newcommand{\icap}[0]{\hat{\Bi}}
\newcommand{\jcap}[0]{\hat{\Bj}}
\newcommand{\kcap}[0]{\hat{\Bk}}
\newcommand{\lcap}[0]{\hat{\Bl}}
\newcommand{\mcap}[0]{\hat{\Bm}}
\newcommand{\ncap}[0]{\hat{\Bn}}
\newcommand{\ocap}[0]{\hat{\Bo}}
\newcommand{\pcap}[0]{\hat{\Bp}}
\newcommand{\qcap}[0]{\hat{\Bq}}
\newcommand{\rcap}[0]{\hat{\Br}}
\newcommand{\scap}[0]{\hat{\Bs}}
\newcommand{\tcap}[0]{\hat{\Bt}}
\newcommand{\ucap}[0]{\hat{\Bu}}
\newcommand{\vcap}[0]{\hat{\Bv}}
\newcommand{\wcap}[0]{\hat{\Bw}}
\newcommand{\xcap}[0]{\hat{\Bx}}
\newcommand{\ycap}[0]{\hat{\By}}
\newcommand{\zcap}[0]{\hat{\Bz}}
\newcommand{\thetacap}[0]{\hat{\Btheta}}

%
% to write R^n and C^n in a distinguishable fashion.  Perhaps change this
% to the double lined characters upon figuring out how to do so.
%
\newcommand{\C}[1]{$\mathbb{C}^{#1}$}
\newcommand{\R}[1]{$\mathbb{R}^{#1}$}

%
% various generally useful helpers
%

% derivative of #1 wrt. #2:
\newcommand{\D}[2] {\frac {d#2} {d#1}}

\newcommand{\inv}[1]{\frac{1}{#1}}
\newcommand{\cross}[0]{\times}

\newcommand{\abs}[1]{\lvert{#1}\rvert}
\newcommand{\norm}[1]{\lVert{#1}\rVert}
\newcommand{\innerprod}[2]{\langle{#1}, {#2}\rangle}
\newcommand{\dotprod}[2]{{#1} \cdot {#2}}
\newcommand{\bdotprod}[2]{\left({#1} \cdot {#2}\right)}
\newcommand{\crossprod}[2]{{#1} \cross {#2}}
\newcommand{\tripleprod}[3]{\dotprod{\left(\crossprod{#1}{#2}\right)}{#3}}

\DeclareMathOperator{\Proj}{Proj}
\DeclareMathOperator{\Span}{span}
\DeclareMathOperator{\Sgn}{sgn}
\DeclareMathOperator{\Area}{Area}
\DeclareMathOperator{\Volume}{Volume}

%
% A few miscellaneous things specific to this document
%
\newcommand{\crossop}[1]{\crossprod{#1}{}}

% R2 vector.
\newcommand{\VectorTwo}[2]{
\begin{bmatrix}
 {#1} \\
 {#2}
\end{bmatrix}
}

\newcommand{\VectorN}[1]{
\begin{bmatrix}
{#1}_1 \\
{#1}_2 \\
\vdots \\
{#1}_N \\
\end{bmatrix}
}

\newcommand{\DETuvij}[4]{
\begin{vmatrix}
 {#1}_{#3} & {#1}_{#4} \\
 {#2}_{#3} & {#2}_{#4}
\end{vmatrix}
}

\newcommand{\DETuvwijk}[6]{
\begin{vmatrix}
 {#1}_{#4} & {#1}_{#5} & {#1}_{#6} \\
 {#2}_{#4} & {#2}_{#5} & {#2}_{#6} \\
 {#3}_{#4} & {#3}_{#5} & {#3}_{#6}
\end{vmatrix}
}

\newcommand{\DETuvwxijkl}[8]{
\begin{vmatrix}
 {#1}_{#5} & {#1}_{#6} & {#1}_{#7} & {#1}_{#8} \\
 {#2}_{#5} & {#2}_{#6} & {#2}_{#7} & {#2}_{#8} \\
 {#3}_{#5} & {#3}_{#6} & {#3}_{#7} & {#3}_{#8} \\
 {#4}_{#5} & {#4}_{#6} & {#4}_{#7} & {#4}_{#8} \\
\end{vmatrix}
}

%\newcommand{\DETuvwxyijklm}[10]{
%\begin{vmatrix}
% {#1}_{#6} & {#1}_{#7} & {#1}_{#8} & {#1}_{#9} & {#1}_{#10} \\
% {#2}_{#6} & {#2}_{#7} & {#2}_{#8} & {#2}_{#9} & {#2}_{#10} \\
% {#3}_{#6} & {#3}_{#7} & {#3}_{#8} & {#3}_{#9} & {#3}_{#10} \\
% {#4}_{#6} & {#4}_{#7} & {#4}_{#8} & {#4}_{#9} & {#4}_{#10} \\
% {#5}_{#6} & {#5}_{#7} & {#5}_{#8} & {#5}_{#9} & {#5}_{#10}
%\end{vmatrix}
%}

% R3 vector.
\newcommand{\VectorThree}[3]{
\begin{bmatrix}
 {#1} \\
 {#2} \\
 {#3}
\end{bmatrix}
}



\newcommand{\laplacian}[0]{\nabla^2}
\newcommand{\Dsq}[2] {\frac {\partial^2 {#1}} {\partial {#2}^2}}
\newcommand{\dxj}[2] {\frac {\partial {#1}} {\partial {x_{#2}}}}
\newcommand{\dsqxj}[2] {\frac {\partial^2 {#1}} {\partial {x_{#2}}^2}}
\DeclareMathOperator{\Exp}{e}
\newcommand{\gpgrade}[2] {{\left\langle{{#1}}\right\rangle}_{#2}}

%
% The real thing:
%

                             % The preamble begins here.
\title{exponential solutions to laplace equation}
\author{Peeter Joot}         % Declares the author's name.
%\date{}        % Deleting this command produces today's date.

\begin{document}             % End of preamble and beginning of text.

\maketitle{}

\section{ The problem. }

Want solutions of

\[
\laplacian f = \sum_k \dsqxj{f}{k} = 0
\]

For real f.

\subsection{ One dimension. }

Here the problem is easy, we just integrate twice:

\[
f = cx + d
\]

\subsection{ Two dimensions. }

Solve:

\[
\dsqxj{f}{1} + \dsqxj{f}{2} = 0
\]

Can find solutions of the form $f = X(x_1)Y(x_2)$.  Differentiating we have:

\[
X''Y + XY'' = 0
\]

So, for $X \ne 0$, and $Y \ne 0$:
\[
\frac{X''}{X} = -\frac{Y''}{Y} = k^2
\]

\[
\implies
X = \Exp^{kx}
\]
\[
Y = \Exp^{k\Bi y}
\]

\[
\implies
f = XY = \Exp^{k(x + \Bi y)}
\]

Here $\Bi$ is anything that squares to -1.  Traditionally this is the
complex unit imaginary, but we are also free to use a geometric product unit bivector such as $\Bi = \Be_1 \wedge \Be_2 = \Be_1\Be_2 = \Be_{12}$, or $\Bi = \Be_{21}$.

With $\Bi = \Be_{12}$ for example we have:

\begin{align*}
f = XY = \Exp^{k(x + \Bi y)} 
&= \Exp^{k(x + \Be_{12} y)} \\
&= \Exp^{k(x\Be_{1}\Be_1 + \Be_{12} y)} \\
&= \Exp^{k\Be_1(x\Be_1 + \Be_2 y)} \\
\end{align*}

Writing $\Br = \sum x_i \Be_i$, all of the following are solutions
of the laplacian

\begin{align*}
\Exp^{k\Be_1\Br} \\
\Exp^{\Br k\Be_1} \\
\Exp^{k\Be_2\Br} \\
\Exp^{\Br k\Be_2} \\
\end{align*}

Now there isn't anything special about the use of the x and y axis so it is reasonable to expect that, given any constant vector $\Bv$, 
the the following are also solutions

\begin{align*}
\Exp^{\Br\Bv} &= \Exp^{\Br \cdot \Bv + \Br \wedge \Bv} \\
\Exp^{\Bv\Br} &= \Exp^{\Br \cdot \Bv - \Br \wedge \Bv} \\
\end{align*}

Provided $\Br$, and $\Bv$ aren't colinear, the wedge product component of the above can be written in terms of a unit bivector
$\Bi = \frac{\Br \wedge \Bv}{\abs{\Br \wedge \Bv}}$:

\begin{align*}
\Exp^{\Br\Bv} &= \Exp^{\Br \cdot \Bv + \Br \wedge \Bv} \\
&= \Exp^{\Br \cdot \Bv + \left( \frac{\Br \wedge \Bv}{\abs{\Br \wedge \Bv}} \right) {\abs{\Br \wedge \Bv}}} \\
&= \Exp^{\Br \cdot \Bv} \left( \cos{\abs{\Br \wedge \Bv}} + \left(\frac{\Br \wedge \Bv}{\abs{\Br \wedge \Bv}}\right) \sin{\abs{\Br \wedge \Bv}} \right) \\
&= \Exp^{\Br \cdot \Bv} \left( \cos{\abs{\Br \wedge \Bv}} + \Bi \sin{\abs{\Br \wedge \Bv}} \right) \\
\end{align*}

And, for the reverse:
\begin{align*}
(\Exp^{\Br\Bv})^\dagger = \Exp^{\Bv\Br}
&= \Exp^{\Br \cdot \Bv} \left( \cos{\abs{\Br \wedge \Bv}} - \Bi \sin{\abs{\Br \wedge \Bv}} \right) \\
\end{align*}

This exponential however has both scalar and bivector parts, and we are looking for real solutions, so we can use linear combinations of the
exponential and its reverse 
to write strictly real solutions for the $\Br \wedge \Bv \ne 0$ cases:

\begin{align*}
\inv{2}\left(\Exp^{\Br\Bv} + \Exp^{\Bv\Br}\right) = \Exp^{\Br\cdot\Bv}\cos{\abs{\Br \wedge \Bv}} \\
\inv{2\Bi}\left(\Exp^{\Br\Bv} - \Exp^{\Bv\Br}\right) = \Exp^{\Br\cdot\Bv}\sin{\abs{\Br \wedge \Bv}} \\
\end{align*}

Also note that further linear combinations (with positive and negative values for $\Bv$) can be taken, so we can write solutions to the laplacian in 
the symmetric, real numbered, coordinate free, form

\begin{align*}
\cosh(\Br\cdot\Bv)\cos{\abs{\Br \wedge \Bv}} \\
\sinh(\Br\cdot\Bv)\cos{\abs{\Br \wedge \Bv}} \\
\cosh(\Br\cdot\Bv)\sin{\abs{\Br \wedge \Bv}} \\
\sinh(\Br\cdot\Bv)\sin{\abs{\Br \wedge \Bv}} \\
\end{align*}

\subsection{ Verify the expontial solution. }

It is simple enough to verify that the vector product exponential solution above is valid in \R{2}, and it's tempting to assume that this
is also a solution for \R{N}.  That isn't the case, but to demonstrate this requires a few interesting geometric product manipulations (perhaps
obvious if tackled differently or to somebody with a better grasp of all the concepts).

Let's calculate one of the partials of the exponential solution

\begin{align*}
\dxj{}{j}\Exp^{\Br\Bv} 
&= \dxj{}{j}\Exp^{\Br \cdot \Bv + \Br \wedge \Bv} \\
&= \left(\dxj{}{j}\Exp^{\Br \cdot \Bv}\right) \Exp^{\Br \wedge \Bv} + \Exp^{\Br \cdot \Bv}\dxj{}{j}{\Exp^{\Br \wedge \Bv}} \\
&= \dxj{(\Br \cdot \Bv)}{j}\Exp^{\Br\Bv} + \Exp^{\Br \cdot \Bv}\dxj{}{j}{\Exp^{\Br \wedge \Bv}} \\
&= (\Be_j \cdot \Bv)\Exp^{\Br\Bv} + \Exp^{\Br \cdot \Bv}\dxj{}{j}{\Exp^{\Br \wedge \Bv}} \\
&= v_j\Exp^{\Br\Bv} + \Exp^{\Br \cdot \Bv}\dxj{}{j}{\Exp^{\Br \wedge \Bv}} \\
\end{align*}

Now, how do we differentiate this remaining bivector exponential?  Since bivectors do not generally commute, loose application of the 
chain rule $\frac{df(g)}{dx} = \frac{df}{dg}\frac{dg}{dx} = \frac{dg}{dx}\frac{df}{dg}$ may not be appropriate (what order would one use).  Instead
write this expontial in terms of scalar and bivector parts and differentiate that

\begin{align*}
\dxj{}{j}{\Exp^{\Br \wedge \Bv}} 
&= \dxj{}{j} \left( \cos{\abs{\Br \wedge \Bv}} + \Bi \sin{\abs{\Br \wedge \Bv}} \right) \\
&= \left(-\sin{\abs{\Br \wedge \Bv}} + \Bi \cos{\abs{\Br \wedge \Bv}} \right)\dxj{\abs{\Br \wedge \Bv}}{j} + \dxj{\Bi}{j} \sin{\abs{\Br \wedge \Bv}} \\
&= \Bi\left(\Bi\sin{\abs{\Br \wedge \Bv}} + \cos{\abs{\Br \wedge \Bv}} \right)\dxj{\abs{\Br \wedge \Bv}}{j} + \dxj{\Bi}{j} \sin{\abs{\Br \wedge \Bv}} \\
&= \Bi\Exp^{\Br \wedge \Bv} \dxj{\abs{\Br \wedge \Bv}}{j} + \dxj{\Bi}{j} \sin{\abs{\Br \wedge \Bv}} \\
&= \Exp^{\Br \wedge \Bv} \dxj{\abs{\Br \wedge \Bv}}{j} + \dxj{\Bi}{j} \sin{\abs{\Br \wedge \Bv}} \\
\end{align*}

%\subsubsection{ partial of the unit bivector term }
For \R{2}, it is fairly simple to confirm that 
\[
\dxj{\Bi}{j} = 0 
\]

This makes some intuitive sense (would the unit bivector for the plane be changed by varing one of the length of one of the components).  To prove this
in general takes a bit more work.

\begin{align*}
\dxj{\Bi}{j}
&= \dxj{}{j} \frac{\Br \wedge \Bv}{\abs{\Br \wedge \Bv}} \\
&= \dxj{(\Br \wedge \Bv)}{j} \inv{\abs{\Br \wedge \Bv}} + {(\Br \wedge \Bv)} \dxj{}{j} \inv{\abs{\Br \wedge \Bv}} \\
&= (\Be_j \wedge \Bv) \inv{\abs{\Br \wedge \Bv}} + (\Br \wedge \Bv) \dxj{}{j} {(\Br \wedge \Bv \Bv \wedge \Br)}^{-1/2} \\
&= (\Be_j \wedge \Bv) \inv{\abs{\Br \wedge \Bv}} -\inv{2 \abs{\Br \wedge \Bv}^3} {\Br \wedge \Bv} \dxj{}{j} ({\Br \wedge \Bv}{\Bv \wedge \Br}) \\
&= (\Be_j \wedge \Bv) \inv{\abs{\Br \wedge \Bv}} -\inv{2 \abs{\Br \wedge \Bv}^3} {\Br \wedge \Bv} (\Be_j \wedge \Bv\Bv \wedge \Br + \Br \wedge \Bv\Bv \wedge \Be_j) \\
&= (\Be_j \wedge \Bv) \inv{\abs{\Br \wedge \Bv}} +\inv{2 \abs{\Br \wedge \Bv}} \Bi ((\Be_j \wedge \Bv) \Bi + \Bi (\Be_j \wedge \Bv)) \\
&= \inv{\abs{\Br \wedge \Bv}} \left( \Be_j \wedge \Bv  +\inv{2}\left( \Bi(\Be_j \wedge \Bv)\Bi - \Be_j \wedge \Bv \right) \right) \\
&= \inv{2 \abs{\Br \wedge \Bv}} \left( \Be_j \wedge \Bv  + \Bi(\Be_j \wedge \Bv)\Bi \right) \\
&= \inv{2 \abs{\Br \wedge \Bv}} \left( \Be_j \wedge \Bv  - \Exp^{-\Bi \pi/2}(\Be_j \wedge \Bv)\Exp^{\Bi \pi/2} \right) \\
\end{align*}

This last term is a rotation by $\pi$ in the plane of $\Bi$.

If $\BB = \Be_j \wedge \Bv$ lies in the plane ($\BB = k\Bi$) we have:

\[
\BB + \Bi (k\Bi) \Bi = \BB - k\Bi = \BB - \BB = 0
\]

How about for other orientations of $\BB$?  In general this won't be zero.  As a linear operator, the rotation of $\BB$ will be the sum of the
rotations of the components that are in the plane of rotation and the components out of the plane, so taking the difference, all the components in the plane
will be zero.

Example, in \R{3}:

\[
(1,1,1)\wedge(-1,1,1) = 
\begin{vmatrix}
 1 & 1 \\
 -1 & 1 \\
\end{vmatrix}(\Be_{13} + \Be_{12}) = 2(\Be_{13} + \Be_{12})
\]

Rotation of the direction vectors for the plane by $\pi$ in the $\Be_{12}$ plane we have:

\[
-\Be_{12}(1,1,1)\Be_{12}= 
(\Be_{2}
-\Be_{1}
+\Be_{213})\Be_{12}
= (-\Be_{1} -\Be_{2} + \Be_{3})
\]

and

\[
-\Be_{12}(-1,1,1)\Be_{12}= 
(-\Be_{2}
-\Be_{1}
+\Be_{213})\Be_{12}
= (\Be_{1} -\Be_{2} + \Be_{3})
\]

So the bivector for the plane with the direction vectors rotated is:
\[
%(-1,-1,1)
%( 1,-1,1) = 
(-1,-1,1)\wedge(1,-1,1) = 
\begin{vmatrix}
 -1 & 1 \\
 -1 & 1 \\
\end{vmatrix}
\Be_{23}
+
\begin{vmatrix}
 -1 & 1 \\
 1 & 1 \\
\end{vmatrix}
\Be_{13}
+
\begin{vmatrix}
 -1 & -1 \\
 1 & -1 \\
\end{vmatrix}
\Be_{12}
=
-2\Be_{13} + 2\Be_{12}
\]

Taking the difference, the component in the plane of rotation vanishes as expected, and in this particular case we have:

\[
\BB + \Bi\BB\Bi = \BB - \Bi^\dagger\BB\Bi = 
2(\Be_{13} + \Be_{12}) -(-2\Be_{13} + 2\Be_{12}) = 4\Be_{13}
\]

In the general case to calculate the remainder, we have to know how to compute the projection of a bivector onto a plane.  The procedure for this is similar to a vector
projection onto a space, and we calculate

\begin{align*}
\BB\inv{\BA}\BA 
&= \left(\BB \cdot \inv{\BA} +{\langle{\BB \wedge \inv{\BA}}\rangle}_2 +\BB \wedge \inv{\BA}\right) \BA \\
&= 
\BB \cdot \inv{\BA} \BA \\
&+{\langle{ {\BB \inv{\BA}}\rangle}_2} \cdot \BA 
+{\langle{{\langle{ {\BB \inv{\BA}}\rangle}_2} \BA}\rangle}_2 
+{\langle{ {\BB \inv{\BA}}\rangle}_2} \wedge \BA  \\
&+(\BB \wedge \inv{\BA}) \cdot \BA 
+{\langle{\BB \wedge \inv{\BA} \BA}\rangle}_4 
+\BB \wedge \inv{\BA} \wedge \BA \\
\end{align*}

Since the LHS is a bivector, all the 0-grade, 4-grade and 6-grade terms must be zero (though the 6-grade term $\BB \wedge \inv{\BA} \wedge \BA$ is plainly zero anyhow).  That leaves:

\begin{equation}\label{eqn:bivectorprojbivector}
\BB
= 
\BB \cdot \inv{\BA} \BA \\
%+{\langle{{\langle{ {\BB \inv{\BA}}\rangle}_2} \BA}\rangle}_2 
+\gpgrade{\gpgrade{\BB\inv\BA}{2} \BA}{2}
+\left(\BB \wedge \inv{\BA}\right) \cdot \BA 
\end{equation}
\newcommand{\grade}[2] {{\left\langle{{#1}}\right\rangle}_{#2}}

%\begin{align*}
%\BB + \Bi\BB\Bi
%&= \left(\BB\cdot\inv{\Bi}\right)\Bi + \left(\BB\wedge\inv{\Bi}\right)\Bi - \Bi^\dagger\left( \left(\BB\cdot\inv{\Bi}\right)\Bi + \left(\BB\wedge\inv{\Bi}\right)\Bi \right) \Bi \\
%&= \left(\BB\cdot\inv{\Bi}\right)\Bi + \left(\BB\wedge\inv{\Bi}\right)\Bi - \Bi^\dagger \left( \left(\BB\cdot\inv{\Bi}\right)\Bi + \left(\BB\wedge\inv{\Bi}\right)\Bi\right) \Bi \\
%&= \left(\BB\wedge\inv{\Bi}\right)\Bi + \Bi \left(\BB\wedge\inv{\Bi}\right) \\
%&= -\left(\BB\wedge{\Bi}\right)\Bi - \Bi \left(\BB\wedge{\Bi}\right) \\
%\end{align*}
%
%Check:
%\begin{align*}
%\BB \wedge \Bi =
%2\left(\Be_{13} + \Be_{12}\right) \wedge \Be_{12}
%= 0
%\end{align*}
%
%Must be a mistake here somewhere, since these aren't consistent (pretty sure the numerical calculation is right).

\end{document}               % End of document.

\chapter{Hyper complex numbers and symplectic structure. }
\label{chap:complex}
\date{November 8, 2008.  complex.tex}

\section{On 4.2 Hermitian Norms and Unitary Groups. }

These are some rather rough notes filling in some details 
on the treatment of \citep{DoranHamiltonian}.

Expanding equation 4.17

\begin{align*}
J &= e_i \wedge f_i \\
a &= u_i e_i + v_i f_i \\
b &= x_i e_i + y_i f_i \\
B &= a \wedge b + (a \cdot J) \wedge (b \cdot J)  \\
\end{align*}

\begin{align*}
a \wedge b
&= (u_i e_i + v_i f_i) \wedge (x_j e_j + y_j f_j) \\
&= 
u_i x_j e_i \wedge e_j 
+ u_i y_j e_i \wedge f_j
+ v_i x_j f_i \wedge e_j
+ v_i y_j f_i \wedge f_j \\
\end{align*}

\begin{align*}
a \cdot J
&=
u_i e_i \cdot ( e_j \wedge f_j )
+ v_i f_i \cdot ( e_j \wedge f_j ) \\
&= u_j f_j - v_j e_j
\end{align*}

Search and replace for $b \cdot J$ gives

\begin{align*}
b \cdot J
&=
x_i e_i \cdot ( e_j \wedge f_j )
+ y_i f_i \cdot ( e_j \wedge f_j ) \\
&= x_j f_j - y_j e_j
\end{align*}

So we have

\begin{align*}
(a \cdot J) \wedge (b \cdot J) 
&= (u_i f_i - v_i e_i) \wedge (x_j f_j - y_j e_j) \\
&=
 u_i x_j f_i \wedge f_j 
-u_i y_j f_i \wedge e_j
- v_i x_j e_i \wedge f_j
+ v_i y_j e_i \wedge e_j
\end{align*}

For
\begin{align*}
a \wedge b + (a \cdot J) \wedge (b \cdot J) 
&= 
 ( u_i y_j - v_i x_j ) (e_i \wedge f_j - f_i \wedge e_j)
+ ( u_i x_j + v_i y_j ) (e_i \wedge e_j + f_i \wedge f_j)
\end{align*}

This shows why the elements were picked as a basis
\begin{align*}
e_i \wedge f_j - f_i \wedge e_j
\end{align*}
\begin{align*}
e_i \wedge e_j + f_i \wedge f_j
\end{align*}

The first of which is a multiple of $J_i = e_i \wedge f_i$ when $i=j$, and the second of which is zero if $i=j$.

\section{5.1 Conservation Theorems and Flows. } 

equation 5.10 is

\begin{align*}
\fdot = \xdot \cdot \grad f = (\grad f \wedge \grad H) \cdot J
\end{align*}

This one isn't obvious to me.  For $\fdot$ we have

\begin{align*}
\fdot = \PD{p_i}{f} \pdot_i +\PD{q_i}{f} \qdot_i + \underbrace{\PD{t}{f}}_{=0}
\end{align*}

compare to 

\begin{align*}
\xdot \cdot \grad f 
&= (\pdot_i e_i + \qdot_i f_i) \cdot (e_j \PD{p_j}{f} + f_j \PD{q_j}{f}) \\
&= \pdot_i \PD{p_i}{f} + \qdot_i \PD{q_i}{f}
\end{align*}

Okay, this part matches the first part of (5.10).  Writing this in terms of the Hamiltonian relation (5.9) $\xdot = \grad H \cdot J$ we have

\begin{align*}
\fdot
&= (\grad H \cdot J) \cdot \grad f \\
&= \grad f \cdot (\grad H \cdot J) \\
\end{align*}

The relation $a \cdot (b \cdot (c \wedge d)) = (a \wedge b) \cdot (c \wedge d)$, 
can be used here to factor out the $J$, we have
\begin{align*}
\fdot
&= \grad f \cdot (\grad H \cdot J) \\
&= (\grad f \wedge \grad H) \cdot J \\
\end{align*}

which completes (5.10).

Also with $f=H$ since H was also specified as having no explicit time dependence, one has

\begin{align*}
\dot{H} &= (\grad H \wedge \grad H) \cdot J = 0 \cdot J = 0
\end{align*}

%\bibliographystyle{plainnat}
%\bibliography{myrefs}

%\end{document}

\part{Relativity.}
\documentclass{article}      % Specifies the document class

\usepackage{amsmath}
\usepackage{mathpazo}

%
% shorthand for bold symbols, convenient for vectors and matrices
%
\newcommand{\Ba}[0]{\mathbf{a}}
\newcommand{\Bb}[0]{\mathbf{b}}
\newcommand{\Bc}[0]{\mathbf{c}}
\newcommand{\Bd}[0]{\mathbf{d}}
\newcommand{\Be}[0]{\mathbf{e}}
\newcommand{\Bf}[0]{\mathbf{f}}
\newcommand{\Bg}[0]{\mathbf{g}}
\newcommand{\Bh}[0]{\mathbf{h}}
\newcommand{\Bi}[0]{\mathbf{i}}
\newcommand{\Bj}[0]{\mathbf{j}}
\newcommand{\Bk}[0]{\mathbf{k}}
\newcommand{\Bl}[0]{\mathbf{l}}
\newcommand{\Bm}[0]{\mathbf{m}}
\newcommand{\Bn}[0]{\mathbf{n}}
\newcommand{\Bo}[0]{\mathbf{o}}
\newcommand{\Bp}[0]{\mathbf{p}}
\newcommand{\Bq}[0]{\mathbf{q}}
\newcommand{\Br}[0]{\mathbf{r}}
\newcommand{\Bs}[0]{\mathbf{s}}
\newcommand{\Bt}[0]{\mathbf{t}}
\newcommand{\Bu}[0]{\mathbf{u}}
\newcommand{\Bv}[0]{\mathbf{v}}
\newcommand{\Bw}[0]{\mathbf{w}}
\newcommand{\Bx}[0]{\mathbf{x}}
\newcommand{\By}[0]{\mathbf{y}}
\newcommand{\Bz}[0]{\mathbf{z}}
\newcommand{\BA}[0]{\mathbf{A}}
\newcommand{\BB}[0]{\mathbf{B}}
\newcommand{\BC}[0]{\mathbf{C}}
\newcommand{\BD}[0]{\mathbf{D}}
\newcommand{\BE}[0]{\mathbf{E}}
\newcommand{\BF}[0]{\mathbf{F}}
\newcommand{\BG}[0]{\mathbf{G}}
\newcommand{\BH}[0]{\mathbf{H}}
\newcommand{\BI}[0]{\mathbf{I}}
\newcommand{\BJ}[0]{\mathbf{J}}
\newcommand{\BK}[0]{\mathbf{K}}
\newcommand{\BL}[0]{\mathbf{L}}
\newcommand{\BM}[0]{\mathbf{M}}
\newcommand{\BN}[0]{\mathbf{N}}
\newcommand{\BO}[0]{\mathbf{O}}
\newcommand{\BP}[0]{\mathbf{P}}
\newcommand{\BQ}[0]{\mathbf{Q}}
\newcommand{\BR}[0]{\mathbf{R}}
\newcommand{\BS}[0]{\mathbf{S}}
\newcommand{\BT}[0]{\mathbf{T}}
\newcommand{\BU}[0]{\mathbf{U}}
\newcommand{\BV}[0]{\mathbf{V}}
\newcommand{\BW}[0]{\mathbf{W}}
\newcommand{\BX}[0]{\mathbf{X}}
\newcommand{\BY}[0]{\mathbf{Y}}
\newcommand{\BZ}[0]{\mathbf{Z}}

\newcommand{\Bzero}[0]{\mathbf{0}}
\newcommand{\Btheta}[0]{\boldsymbol{\theta}}
\newcommand{\Btau}[0]{\boldsymbol{\tau}}
\newcommand{\Bomega}[0]{\boldsymbol{\omega}}

%
% shorthand for unit vectors
%
\newcommand{\acap}[0]{\hat{\Ba}}
\newcommand{\bcap}[0]{\hat{\Bb}}
\newcommand{\ccap}[0]{\hat{\Bc}}
\newcommand{\dcap}[0]{\hat{\Bd}}
\newcommand{\ecap}[0]{\hat{\Be}}
\newcommand{\fcap}[0]{\hat{\Bf}}
\newcommand{\gcap}[0]{\hat{\Bg}}
\newcommand{\hcap}[0]{\hat{\Bh}}
\newcommand{\icap}[0]{\hat{\Bi}}
\newcommand{\jcap}[0]{\hat{\Bj}}
\newcommand{\kcap}[0]{\hat{\Bk}}
\newcommand{\lcap}[0]{\hat{\Bl}}
\newcommand{\mcap}[0]{\hat{\Bm}}
\newcommand{\ncap}[0]{\hat{\Bn}}
\newcommand{\ocap}[0]{\hat{\Bo}}
\newcommand{\pcap}[0]{\hat{\Bp}}
\newcommand{\qcap}[0]{\hat{\Bq}}
\newcommand{\rcap}[0]{\hat{\Br}}
\newcommand{\scap}[0]{\hat{\Bs}}
\newcommand{\tcap}[0]{\hat{\Bt}}
\newcommand{\ucap}[0]{\hat{\Bu}}
\newcommand{\vcap}[0]{\hat{\Bv}}
\newcommand{\wcap}[0]{\hat{\Bw}}
\newcommand{\xcap}[0]{\hat{\Bx}}
\newcommand{\ycap}[0]{\hat{\By}}
\newcommand{\zcap}[0]{\hat{\Bz}}
\newcommand{\thetacap}[0]{\hat{\Btheta}}

%
% to write R^n and C^n in a distinguishable fashion.  Perhaps change this
% to the double lined characters upon figuring out how to do so.
%
\newcommand{\C}[1]{$\mathbb{C}^{#1}$}
\newcommand{\R}[1]{$\mathbb{R}^{#1}$}

%
% various generally useful helpers
%

% derivative of #1 wrt. #2:
\newcommand{\D}[2] {\frac {d#2} {d#1}}

\newcommand{\inv}[1]{\frac{1}{#1}}
\newcommand{\cross}[0]{\times}

\newcommand{\abs}[1]{\lvert{#1}\rvert}
\newcommand{\norm}[1]{\lVert{#1}\rVert}
\newcommand{\innerprod}[2]{\langle{#1}, {#2}\rangle}
\newcommand{\dotprod}[2]{{#1} \cdot {#2}}
\newcommand{\bdotprod}[2]{\left({#1} \cdot {#2}\right)}
\newcommand{\crossprod}[2]{{#1} \cross {#2}}
\newcommand{\tripleprod}[3]{\dotprod{\left(\crossprod{#1}{#2}\right)}{#3}}

\DeclareMathOperator{\Proj}{Proj}
\DeclareMathOperator{\Span}{span}
\DeclareMathOperator{\Sgn}{sgn}
\DeclareMathOperator{\Area}{Area}
\DeclareMathOperator{\Volume}{Volume}

%
% A few miscellaneous things specific to this document
%
\newcommand{\crossop}[1]{\crossprod{#1}{}}

% R2 vector.
\newcommand{\VectorTwo}[2]{
\begin{bmatrix}
 {#1} \\
 {#2}
\end{bmatrix}
}

\newcommand{\VectorN}[1]{
\begin{bmatrix}
{#1}_1 \\
{#1}_2 \\
\vdots \\
{#1}_N \\
\end{bmatrix}
}

\newcommand{\DETuvij}[4]{
\begin{vmatrix}
 {#1}_{#3} & {#1}_{#4} \\
 {#2}_{#3} & {#2}_{#4}
\end{vmatrix}
}

\newcommand{\DETuvwijk}[6]{
\begin{vmatrix}
 {#1}_{#4} & {#1}_{#5} & {#1}_{#6} \\
 {#2}_{#4} & {#2}_{#5} & {#2}_{#6} \\
 {#3}_{#4} & {#3}_{#5} & {#3}_{#6}
\end{vmatrix}
}

\newcommand{\DETuvwxijkl}[8]{
\begin{vmatrix}
 {#1}_{#5} & {#1}_{#6} & {#1}_{#7} & {#1}_{#8} \\
 {#2}_{#5} & {#2}_{#6} & {#2}_{#7} & {#2}_{#8} \\
 {#3}_{#5} & {#3}_{#6} & {#3}_{#7} & {#3}_{#8} \\
 {#4}_{#5} & {#4}_{#6} & {#4}_{#7} & {#4}_{#8} \\
\end{vmatrix}
}

%\newcommand{\DETuvwxyijklm}[10]{
%\begin{vmatrix}
% {#1}_{#6} & {#1}_{#7} & {#1}_{#8} & {#1}_{#9} & {#1}_{#10} \\
% {#2}_{#6} & {#2}_{#7} & {#2}_{#8} & {#2}_{#9} & {#2}_{#10} \\
% {#3}_{#6} & {#3}_{#7} & {#3}_{#8} & {#3}_{#9} & {#3}_{#10} \\
% {#4}_{#6} & {#4}_{#7} & {#4}_{#8} & {#4}_{#9} & {#4}_{#10} \\
% {#5}_{#6} & {#5}_{#7} & {#5}_{#8} & {#5}_{#9} & {#5}_{#10}
%\end{vmatrix}
%}

% R3 vector.
\newcommand{\VectorThree}[3]{
\begin{bmatrix}
 {#1} \\
 {#2} \\
 {#3}
\end{bmatrix}
}



\newcommand{\laplacian}[0]{\nabla^2}

%
% The real thing:
%

                             % The preamble begins here.
\title{Derive Lorentz transformation from wave equation.} % Declares the document's title.
\author{Peeter Joot}         % Declares the author's name.
%\date{}        % Deleting this command produces today's date.

\begin{document}             % End of preamble and beginning of text.

\maketitle{}

\section{intro.}

Many introductory relativity texts mention how Lorentz observed that 
while Maxwells equations were not invarient with respect to Galelian
transformation, they were with his modified transformation.

I found it interesting to consider this statement with a bit of detail.

\section{}

From Maxwell's equations one can show that the electric field and magnetic field both satisfy the wave equation:

\begin{equation}
\laplacian - \inv{c^2}\frac{\partial^2}{\partial t^2} = 0 
\end{equation}

The wikipedia article Electromagnetic radiation (under Derivation)

%\htmladdnormallink{<URL>} { http://en.wikipedia.org/wiki/Electromagnetic_radiation#Derivation }

goes over this nicely.

Although this can be solved separately for either $\BE$ or $\BB$ the two are not independent.
This dependence is nicely expressed by writing the electromagnetic field as a complete
bivector $\BF = \BE + i c \BB$, and in that form the 
general solution to this equation for the combined electromagnetic
field is:

\begin{equation}
\BF = (\BE_0 + \kcap \wedge \BE_0) f( \kcap \cdot \Br \pm c t)
\end{equation}

Here f is any function, and represents the amplitude of the waveform.

\section{Verifing Lorentz invarience.}

The Lorentz transfrom for a moving (primed) frame where the motion is
along the x axis is:

\begin{equation*}
\begin{bmatrix}
x' \\
c t' \\
\end{bmatrix}
=
\gamma
\begin{bmatrix}
1 & -\beta \\
-\beta & 1 \\
\end{bmatrix}
\end{equation*}

Or,
\begin{equation*}
\begin{bmatrix}
x \\
c t \\
\end{bmatrix}
=
\gamma
\begin{bmatrix}
1 & \beta \\
\beta & 1 \\
\end{bmatrix}
\end{equation*}

Using this we can express the partials of the wave equation in the 
primed frame.  Starting with the first derivatives:

\begin{align*}
\frac{\partial}{\partial x} 
&= \frac{\partial x'}{\partial x} \frac{\partial}{\partial x'} + \frac{\partial c t'}{\partial x} \frac{\partial}{\partial c t'} \\
&= \gamma \frac{\partial}{\partial x'} - \gamma \beta \frac{\partial}{\partial c t'} \\
\end{align*}

And:

\begin{align*}
\frac{\partial}{\partial ct} 
&= \frac{\partial x'}{\partial ct} \frac{\partial}{\partial x'} + \frac{\partial c t'}{\partial ct} \frac{\partial}{\partial c t'} \\
&= -\beta \gamma \frac{\partial}{\partial x'} + \gamma \frac{\partial}{\partial c t'} \\
\end{align*}

...

\section{ Derive Lorentz Transformation requiring invarience of the wave equation. }

\end{document}               % End of document.

\documentclass{article}      % Specifies the document class

\usepackage{amsmath}
\usepackage{mathpazo}

%
% shorthand for bold symbols, convenient for vectors and matrices
%
\newcommand{\Ba}[0]{\mathbf{a}}
\newcommand{\Bb}[0]{\mathbf{b}}
\newcommand{\Bc}[0]{\mathbf{c}}
\newcommand{\Bd}[0]{\mathbf{d}}
\newcommand{\Be}[0]{\mathbf{e}}
\newcommand{\Bf}[0]{\mathbf{f}}
\newcommand{\Bg}[0]{\mathbf{g}}
\newcommand{\Bh}[0]{\mathbf{h}}
\newcommand{\Bi}[0]{\mathbf{i}}
\newcommand{\Bj}[0]{\mathbf{j}}
\newcommand{\Bk}[0]{\mathbf{k}}
\newcommand{\Bl}[0]{\mathbf{l}}
\newcommand{\Bm}[0]{\mathbf{m}}
\newcommand{\Bn}[0]{\mathbf{n}}
\newcommand{\Bo}[0]{\mathbf{o}}
\newcommand{\Bp}[0]{\mathbf{p}}
\newcommand{\Bq}[0]{\mathbf{q}}
\newcommand{\Br}[0]{\mathbf{r}}
\newcommand{\Bs}[0]{\mathbf{s}}
\newcommand{\Bt}[0]{\mathbf{t}}
\newcommand{\Bu}[0]{\mathbf{u}}
\newcommand{\Bv}[0]{\mathbf{v}}
\newcommand{\Bw}[0]{\mathbf{w}}
\newcommand{\Bx}[0]{\mathbf{x}}
\newcommand{\By}[0]{\mathbf{y}}
\newcommand{\Bz}[0]{\mathbf{z}}
\newcommand{\BA}[0]{\mathbf{A}}
\newcommand{\BB}[0]{\mathbf{B}}
\newcommand{\BC}[0]{\mathbf{C}}
\newcommand{\BD}[0]{\mathbf{D}}
\newcommand{\BE}[0]{\mathbf{E}}
\newcommand{\BF}[0]{\mathbf{F}}
\newcommand{\BG}[0]{\mathbf{G}}
\newcommand{\BH}[0]{\mathbf{H}}
\newcommand{\BI}[0]{\mathbf{I}}
\newcommand{\BJ}[0]{\mathbf{J}}
\newcommand{\BK}[0]{\mathbf{K}}
\newcommand{\BL}[0]{\mathbf{L}}
\newcommand{\BM}[0]{\mathbf{M}}
\newcommand{\BN}[0]{\mathbf{N}}
\newcommand{\BO}[0]{\mathbf{O}}
\newcommand{\BP}[0]{\mathbf{P}}
\newcommand{\BQ}[0]{\mathbf{Q}}
\newcommand{\BR}[0]{\mathbf{R}}
\newcommand{\BS}[0]{\mathbf{S}}
\newcommand{\BT}[0]{\mathbf{T}}
\newcommand{\BU}[0]{\mathbf{U}}
\newcommand{\BV}[0]{\mathbf{V}}
\newcommand{\BW}[0]{\mathbf{W}}
\newcommand{\BX}[0]{\mathbf{X}}
\newcommand{\BY}[0]{\mathbf{Y}}
\newcommand{\BZ}[0]{\mathbf{Z}}

\newcommand{\Bzero}[0]{\mathbf{0}}
\newcommand{\Btheta}[0]{\boldsymbol{\theta}}
\newcommand{\Btau}[0]{\boldsymbol{\tau}}
\newcommand{\Bomega}[0]{\boldsymbol{\omega}}

%
% shorthand for unit vectors
%
\newcommand{\acap}[0]{\hat{\Ba}}
\newcommand{\bcap}[0]{\hat{\Bb}}
\newcommand{\ccap}[0]{\hat{\Bc}}
\newcommand{\dcap}[0]{\hat{\Bd}}
\newcommand{\ecap}[0]{\hat{\Be}}
\newcommand{\fcap}[0]{\hat{\Bf}}
\newcommand{\gcap}[0]{\hat{\Bg}}
\newcommand{\hcap}[0]{\hat{\Bh}}
\newcommand{\icap}[0]{\hat{\Bi}}
\newcommand{\jcap}[0]{\hat{\Bj}}
\newcommand{\kcap}[0]{\hat{\Bk}}
\newcommand{\lcap}[0]{\hat{\Bl}}
\newcommand{\mcap}[0]{\hat{\Bm}}
\newcommand{\ncap}[0]{\hat{\Bn}}
\newcommand{\ocap}[0]{\hat{\Bo}}
\newcommand{\pcap}[0]{\hat{\Bp}}
\newcommand{\qcap}[0]{\hat{\Bq}}
\newcommand{\rcap}[0]{\hat{\Br}}
\newcommand{\scap}[0]{\hat{\Bs}}
\newcommand{\tcap}[0]{\hat{\Bt}}
\newcommand{\ucap}[0]{\hat{\Bu}}
\newcommand{\vcap}[0]{\hat{\Bv}}
\newcommand{\wcap}[0]{\hat{\Bw}}
\newcommand{\xcap}[0]{\hat{\Bx}}
\newcommand{\ycap}[0]{\hat{\By}}
\newcommand{\zcap}[0]{\hat{\Bz}}
\newcommand{\thetacap}[0]{\hat{\Btheta}}

%
% to write R^n and C^n in a distinguishable fashion.  Perhaps change this
% to the double lined characters upon figuring out how to do so.
%
\newcommand{\C}[1]{$\mathbb{C}^{#1}$}
\newcommand{\R}[1]{$\mathbb{R}^{#1}$}

%
% various generally useful helpers
%

% derivative of #1 wrt. #2:
\newcommand{\D}[2] {\frac {d#2} {d#1}}

\newcommand{\inv}[1]{\frac{1}{#1}}
\newcommand{\cross}[0]{\times}

\newcommand{\abs}[1]{\lvert{#1}\rvert}
\newcommand{\norm}[1]{\lVert{#1}\rVert}
\newcommand{\innerprod}[2]{\langle{#1}, {#2}\rangle}
\newcommand{\dotprod}[2]{{#1} \cdot {#2}}
\newcommand{\bdotprod}[2]{\left({#1} \cdot {#2}\right)}
\newcommand{\crossprod}[2]{{#1} \cross {#2}}
\newcommand{\tripleprod}[3]{\dotprod{\left(\crossprod{#1}{#2}\right)}{#3}}

\DeclareMathOperator{\Proj}{Proj}
\DeclareMathOperator{\Span}{span}
\DeclareMathOperator{\Sgn}{sgn}
\DeclareMathOperator{\Area}{Area}
\DeclareMathOperator{\Volume}{Volume}

%
% A few miscellaneous things specific to this document
%
\newcommand{\crossop}[1]{\crossprod{#1}{}}

% R2 vector.
\newcommand{\VectorTwo}[2]{
\begin{bmatrix}
 {#1} \\
 {#2}
\end{bmatrix}
}

\newcommand{\VectorN}[1]{
\begin{bmatrix}
{#1}_1 \\
{#1}_2 \\
\vdots \\
{#1}_N \\
\end{bmatrix}
}

\newcommand{\DETuvij}[4]{
\begin{vmatrix}
 {#1}_{#3} & {#1}_{#4} \\
 {#2}_{#3} & {#2}_{#4}
\end{vmatrix}
}

\newcommand{\DETuvwijk}[6]{
\begin{vmatrix}
 {#1}_{#4} & {#1}_{#5} & {#1}_{#6} \\
 {#2}_{#4} & {#2}_{#5} & {#2}_{#6} \\
 {#3}_{#4} & {#3}_{#5} & {#3}_{#6}
\end{vmatrix}
}

\newcommand{\DETuvwxijkl}[8]{
\begin{vmatrix}
 {#1}_{#5} & {#1}_{#6} & {#1}_{#7} & {#1}_{#8} \\
 {#2}_{#5} & {#2}_{#6} & {#2}_{#7} & {#2}_{#8} \\
 {#3}_{#5} & {#3}_{#6} & {#3}_{#7} & {#3}_{#8} \\
 {#4}_{#5} & {#4}_{#6} & {#4}_{#7} & {#4}_{#8} \\
\end{vmatrix}
}

%\newcommand{\DETuvwxyijklm}[10]{
%\begin{vmatrix}
% {#1}_{#6} & {#1}_{#7} & {#1}_{#8} & {#1}_{#9} & {#1}_{#10} \\
% {#2}_{#6} & {#2}_{#7} & {#2}_{#8} & {#2}_{#9} & {#2}_{#10} \\
% {#3}_{#6} & {#3}_{#7} & {#3}_{#8} & {#3}_{#9} & {#3}_{#10} \\
% {#4}_{#6} & {#4}_{#7} & {#4}_{#8} & {#4}_{#9} & {#4}_{#10} \\
% {#5}_{#6} & {#5}_{#7} & {#5}_{#8} & {#5}_{#9} & {#5}_{#10}
%\end{vmatrix}
%}

% R3 vector.
\newcommand{\VectorThree}[3]{
\begin{bmatrix}
 {#1} \\
 {#2} \\
 {#3}
\end{bmatrix}
}



%
% The real thing:
%

                             % The preamble begins here.
\title{} % Declares the document's title.
\author{Peeter Joot}         % Declares the author's name.
%\date{}        % Deleting this command produces today's date.

\begin{document}             % End of preamble and beginning of text.

\maketitle{}

\section{}

Hi Mentz, yes I get the same result:

\begin{equation*}
\ddot{x} = -\frac{m_x \dot{x}^2}{2 m}
\end{equation*}

However, I don't understand anything GR related that you did after that;)

Fwiw, I was able to do the same derivatives with proper time/velocity and I get:

\begin{equation*}
m\dot{v} + v \sum \dot{x}^{\beta} \frac{\partial m}{\partial x^{\beta}} = \frac{1}{2} v^2 \nabla m
\end{equation*}

where:
\begin{equation*}
v = \sum \gamma_{\mu} \dot{x}^{\mu}
\end{equation*}
\begin{equation*}
\nabla = \sum \gamma^{\mu} \frac{\partial}{\partial x^{\mu}}
\end{equation*}
\begin{equation*}
\gamma^{\mu} \cdot \gamma_{\nu} = {\delta^{\mu}}_{\nu}
\end{equation*}

I don't know if this is what you were also getting (and for GR I have the suspision that you let the frame vectors vary with position/time)?

\end{document}               % End of document.

%
% Copyright � 2012 Peeter Joot.  All Rights Reserved.
% Licenced as described in the file LICENSE under the root directory of this GIT repository.
%

% 
% 
\chapter{Understanding four velocity transform from rest frame}
\index{four velocity}
\index{boost}
\index{rest frame}
\label{chap:velocityTx}
%\date{August 13, 2008}

\section{}

\citep{doran2003gap} writes \(v = R \gamma_0 R^\dagger\), as a proper velocity expressed in terms of a rest frame velocity and a Lorentz boost.
This was not clear to me, and would probably be a lot more
obvious to me if I had fully read chapter 5, but in my defense it is a hard read without first getting
more familiarity with basic relativity.

Let us just expand this out
to see how this works.  First thing to note is that there is an omitted factor
of \(c\), and I will add that back in here, since I am not comfortable enough
without it explicitly for now.

With:

\begin{equation}\label{eqn:velocityTx:20}
\begin{aligned}
\Bv/c &= \tanh\left(\alpha\right)\vcap \\
R &= \exp\left(\alpha \vcap/2\right)
\end{aligned}
\end{equation}

We want to expansion this Lorentz boost exponential (see details section) and apply it to the rest frame basis vector.  Writing
\(C = \cosh\left(\alpha/2\right)\), and \(S = \sinh\left(\alpha/2\right)\), we have:

\begin{equation}\label{eqn:velocityTx:40}
\begin{aligned}
v
&= R \left(c \gamma_0\right) R^\dagger \\
&= c \left(C + \vcap S\right) \gamma_0 \left(C - \vcap S\right) \\
&= c \left(C \gamma_0 + S \vcap \gamma_0\right) \left(C - \vcap S\right) \\
&= c \left( C^2 \gamma_0 + SC \vcap \gamma_0 -CS \gamma_0\vcap - S^2 \vcap \gamma_0 \vcap \right) \\
\end{aligned}
\end{equation}

Now, here things can start to get confusing since \(\vcap\) is a spatial quantity with vector-like spacetime basis bivectors \(\sigma_i = \gamma_i \gamma_0\).  Factoring out the \(\gamma_0\) term, utilizing the fact that \(\gamma_0\) and \(\sigma_i\) anticommute (see below).

\begin{equation}\label{eqn:velocityTx:60}
\begin{aligned}
v
&= c \left( C^2 + S^2 + 2 SC \vcap \right) \gamma_0 \\
&= c \left( \cosh\left(\alpha\right) + \vcap \sinh\left(\alpha\right) \right) \gamma_0 \\
&= c \cosh\left(\alpha\right) \left( 1 + \vcap \tanh\left(\alpha\right) \right) \gamma_0 \\
&= c \cosh\left(\alpha\right) \left( 1 + \Bv/c \right) \gamma_0 \\
&= c \gamma \left( 1 + \Bv/c \right) \gamma_0 \\
&= \gamma \left( c \gamma_0 + \sum v^i \gamma_i\right) \\
&= \frac{dt}{d\tau}\left( c \gamma_0 + \sum v^i \gamma_i\right) \\
&= \frac{dt}{d\tau} \frac{d}{dt}\left( c t \gamma_0 + \sum x^i \gamma_i\right) \\
&= \frac{dt}{d\tau} \frac{d}{dt} \sum x^{\mu} \gamma_{\mu} \\
&= \frac{d}{d\tau} \sum x^{\mu} \gamma_{\mu} \\
&= \frac{dx}{d\tau}
\end{aligned}
\end{equation}

So, we get the end result that demonstrates that a Lorentz boost applied to the rest event vector \(x = x^0 \gamma_0 = c t \gamma_0\) directly produces the four velocity for the motion from the new viewpoint.  This makes some intuitive sense, but
I do not feel this is necessarily obvious without demonstration.

This also explains how the text is able to use the wedge and dot product ratios with the \(\gamma_0\) basis vector
to produce the relative spatial velocity.  If one introduces a rest frame proper velocity of
\(w = \frac{d}{dt}\left(ct \gamma_0\right) = c \gamma_0\), then one has:

\begin{equation}\label{eqn:velocityTx:80}
\begin{aligned}
v \cdot w 
&= \left(\sum \frac{d x^{\mu}}{d\tau} \gamma_{\mu}\right) \cdot \left(c\gamma_0\right) \\
&= c^2 \gamma
\end{aligned}
\end{equation}

\begin{equation}\label{eqn:velocityTx:100}
\begin{aligned}
v \wedge w 
&= \left(\sum \frac{d x^{\mu}}{d\tau} \gamma_{\mu}\right) \wedge \left(c\gamma_0\right) \\
&= \left(\sum \frac{d x^{i}}{d\tau} \gamma_{i}\right) \wedge \left(c\gamma_0\right) \\
&= c \sum \frac{d x^{i}}{d\tau} \sigma_{i} \\
&= c \frac{dt}{d\tau} \sum \frac{d x^{i}}{dt} \sigma_{i} \\
&= c \gamma \sum \frac{d x^{i}}{dt} \sigma_{i} \\
\end{aligned}
\end{equation}

Combining these one has the spatial observer dependent relative velocity:

\begin{equation}
\frac{v \wedge w}{v \cdot w} = \inv{c} \sum \frac{d x^{i}}{dt} \sigma_{i} = \frac{\Bv}{c}
\end{equation}

\subsection{Invariance of relative velocity?}

What is not clear to me is whether this can be used to determine the relative velocity between two particles in the general case, when one of them is not a rest frame velocity (time progression only at a fixed point in space.)
The text seems
to imply this is the case, so perhaps it is obvious to them only and not me;)

This can be verified relatively easily for the extreme case, where one boosts both the \(w\), and \(v\) velocities to measure \(v\) in its rest frame.

Expressed mathematically this is:

\begin{equation}\label{eqn:velocityTx:120}
\begin{aligned}
w &= c \gamma_0 \\
v &= R w R^\dagger \\
v' &= R^\dagger v R = R^\dagger R c \gamma_0 R^\dagger R = c \gamma_0 \\
w' &= R^\dagger w R \\
\end{aligned}
\end{equation}

Now, this last expression for \(w'\) can be expanded brute force as was done initially to calculate \(v\) (and I in
fact did that initially without thinking).  The end result matches what should have been the intuitive expectation, with the velocity components all negated in a conjugate like fashion:

\begin{equation*}
w' = \gamma\left( c\gamma_0 - \sum v^i \gamma_i \right)
\end{equation*}

With this result we have:

\begin{equation*}
v' \cdot w' = c \gamma_0 \cdot \gamma\left( c\gamma_0 - \sum v^i \gamma_i \right) = \gamma c^2
\end{equation*}

\begin{equation}\label{eqn:velocityTx:140}
\begin{aligned}
v' \wedge w' 
&= c \gamma_0 \wedge \gamma\left( c\gamma_0 - \sum v^i \gamma_i \right) \\
&= -c \gamma \sum v^i \gamma_0 \gamma_i \\
&= c \gamma \sum v^i \sigma_i \\
\end{aligned}
\end{equation}

Dividing the two we have the following relative velocity between the two proper velocities:

\begin{equation*}
\frac{v' \wedge w'}{v' \cdot w'} = \inv{c} \sum v^i \sigma_i = \Bv/c.
\end{equation*}

Lo and behold, this is the same as when the first event worldline was in its rest frame, so we have the same
relative velocity regardless of which of the two are observed at rest.  The remaining obvious question is 
how to show that this is a general condition, assuming that it is.

\subsection{General invariance?}

Intuitively, I would guess that this is fact the case because when only two particles are considered, the result should be the same independent of which of the
two is considered at rest.

Mathematically, I would express this statement by saying that if one has
a Lorentz boost that takes \(v' = T v T^\dagger\) to its rest frame, then application of this to both proper velocities leaves both the wedge and dot product 
parts of this ratio unchanged:

\begin{equation}\label{eqn:velocityTx:160}
\begin{aligned}
v \cdot w 
&= \left(T^\dagger v' T\right) \cdot \left(T^\dagger w' T\right) \\
&= \gpgradezero{\left(T^\dagger v' T\right) \left(T^\dagger w' T\right)} \\
&= \gpgradezero{T^\dagger v' w' T} \\
&= \gpgradezero{T^\dagger v' \cdot w' T} + 
\mathLabelBox
[
   labelstyle={xshift=2cm},
   linestyle={out=270,in=90, latex-}
]
{\gpgradezero{T^\dagger v' \wedge w' T}}{\(=0\)} \\
&= \left(v' \cdot w'\right)\gpgradezero{T^\dagger T} \\
&= v' \cdot w'
\end{aligned}
\end{equation}

\begin{equation}\label{eqn:velocityTx:180}
\begin{aligned}
v \wedge w 
&= \left(T^\dagger v' T\right) \wedge \left(T^\dagger w' T\right) \\
&= \gpgradetwo{\left(T^\dagger v' T\right) \left(T^\dagger w' T\right)} \\
&= \gpgradetwo{T^\dagger v' w' T} \\
&= 
\mathLabelBox
[
   labelstyle={xshift=2cm},
   linestyle={out=270,in=90, latex-}
]
{\gpgradetwo{T^\dagger v' \cdot w' T}}{\(=0\)} + \gpgradetwo{T^\dagger v' \wedge w' T} \\
&= T^\dagger \left(v' \wedge w'\right) T
\end{aligned}
\end{equation}

FIXME: can not those last \(T\) factors be removed somehow?

\section{Appendix. Omitted details from above}

\subsection{exponential of a vector}

Understanding the vector exponential is a prerequisite above.  This is defined
and interpreted by series expansion as with matrix exponentials.
Expanding
in series the exponential of a vector \(\Bx = x\xcap\), we have:

\begin{equation}\label{eqn:velocityTx:200}
\begin{aligned}
\exp\left(\Bx\right)
&= \sum \frac{\Bx^{2k}}{\left(2k\right)!} + \sum \frac{\Bx^{2k+1}}{\left(2k+1\right)!} \\
&= \sum \frac{x^{2k}}{\left(2k\right)!} + \xcap \sum \frac{x^{2k+1}}{\left(2k+1\right)!} \\
&= \cosh\left(x\right) + \xcap \sinh\left(x\right)
\end{aligned}
\end{equation}

Notationally this can also be written:

\begin{equation*}
\exp\left(\Bx\right) = \cosh\left(\Bx\right) + \sinh\left(\Bx\right)
\end{equation*}

But doing so will not really help.

\subsection{\texorpdfstring{\(\Bv\)}{v} anticommutes with \texorpdfstring{\(\gamma_0\)}{gamma 0}}

\begin{equation}\label{eqn:velocityTx:220}
\begin{aligned}
\Bv \gamma_0 
&= \sum v^i \sigma_i \gamma_0 \\
&= \sum v^i \gamma_i \gamma_0 \gamma_0 \\
&= -\sum v^i \gamma_0 \gamma_i \gamma_0 \\
&= - \gamma_0 \sum v^i \gamma_i \gamma_0 \\
&= - \gamma_0 \sum v^i \sigma_0 \\
&= - \gamma_0 \Bv
\end{aligned}
\end{equation}

%
% Copyright � 2012 Peeter Joot.  All Rights Reserved.
% Licenced as described in the file LICENSE under the root directory of this GIT repository.
%

%
%
\chapter{Four vector dot product invariance and Lorentz rotors}
\index{Lorentz invariance}
\label{chap:fourvecDotinvariance}
%\date{August 1, 2008}        % Deleting this command produces today's date.
\section{}

Prof. Ramamurti Shankar's
In the relativity lectures of
\citep{ShankarPhy200} Prof. Shankar
indicates that the
four vector dot product
is a Lorentz invariant.  This makes some logical sense, but lets demonstrate it explicitly.

Start with a Lorentz transform matrix between coordinates for two four vectors (omitting the components perpendicular  to the motion) :

\begin{equation*}
{
\begin{bmatrix}
x^1 \\
x^0 \\
\end{bmatrix}
}'
=
\gamma
\begin{bmatrix}
1 & -\beta \\
-\beta & 1
\end{bmatrix}
\begin{bmatrix}
x^1 \\
x^0 \\
\end{bmatrix}
\end{equation*}

\begin{equation*}
{
\begin{bmatrix}
y^1 \\
y^0 \\
\end{bmatrix}
}'
=
\gamma
\begin{bmatrix}
1 & -\beta \\
-\beta & 1
\end{bmatrix}
\begin{bmatrix}
y^1 \\
y^0 \\
\end{bmatrix}
\end{equation*}

Now write out the dot product between the two vectors given the perceived length and time measurements for the same events in the moving frame:

\begin{equation}\label{eqn:fourvecDotinvariance:20}
\begin{aligned}
X' \cdot Y'
&= \gamma^2 \left( (-\beta x^1 + x^0)(-\beta y^1 + y^0) -(x^1 -\beta x^0) (y^1 -\beta y^0) \right) \\
&= \gamma^2 \left( (\beta^2 x^1 y^1 + x^0 y^0) + x^0 y^1( -\beta + \beta ) + x^1 y^0( -\beta + \beta ) -(x^1 y^1 + \beta^2 x^0 y^0) \right) \\
&= \gamma^2 \left( x^0 y^0 (1-\beta^2) - (1-\beta^2) x^1 y^1 \right) \\
&= x^0 y^0 - x^1 y^1 \\
&= X \cdot Y
\end{aligned}
\end{equation}

This completes the proof of dot product Lorentz invariance.  An automatic consequence of this is invariance
of the Minkowski length.

\subsection{Invariance shown with hyperbolic trig functions}

Dot product or length invariance can also be shown with the hyperbolic representation of the Lorentz transformation:

\begin{equation}\label{eqn:fVecDotInv:hyperbolicmatrix}
{
\begin{bmatrix}
x^1 \\
x^0 \\
\end{bmatrix}
}'
=
\begin{bmatrix}
\cosh(\alpha) & -\sinh(\alpha) \\
-\sinh(\alpha) & \cosh(\alpha)
\end{bmatrix}
\begin{bmatrix}
x^1 \\
x^0 \\
\end{bmatrix}
\end{equation}

Writing \(S=\sinh(\alpha)\), and \(C=\cosh(\alpha)\) for short, this gives:

\begin{equation}\label{eqn:fourvecDotinvariance:40}
\begin{aligned}
X' \cdot Y'
&= \left( (-S x^1 + C x^0)(-S y^1 + C y^0) -(C x^1 -S x^0) (C y^1 -S y^0) \right) \\
&= \left( (S^2  x^1 y^1 + C^2  x^0 y^0) + x^0 y^1( -SC + SC ) + x^1 y^0( -SC + SC ) -(C^2  x^1 y^1 + S^2  x^0 y^0) \right) \\
&= \left( x^0 y^0 (C^2  -S^2 ) - (C^2 -S^2 ) x^1 y^1 \right) \\
&= x^0 y^0 - x^1 y^1 \\
&= X \cdot Y
\end{aligned}
\end{equation}

This is not really any less work.

\section{Geometric product formulation of Lorentz transform}

We can show the above invariance almost trivially when we write the Lorentz boost in exponential form.  However we first have to show
how to do so.

Writing the spacetime bivector \(\gamma_{10} = \gamma_1 \wedge \gamma_0\) for short, lets calculate the exponential of this spacetime bivector, as scaled with a rapidity
angle \(\alpha\) :

\begin{equation}\label{eqn:fVecDotInv:bivecexponential}
\exp(\gamma_{10}\alpha) = \sum \frac{(\gamma_{10}\alpha)^k}{k!}
\end{equation}

Now, the spacetime bivector has a unit square:

\begin{equation*}
{\gamma_{10}}^2 = \gamma_{1010} = -\gamma_{1001} = -\gamma_{11} = 1
\end{equation*}

so, we can split the sum of \eqnref{eqn:fVecDotInv:bivecexponential} into even and odd parts, and pull out the common bivector factor:

\begin{equation}\label{eqn:fVecDotInv:bivechyper}
\exp(\gamma_{10}\alpha)
= \sum \frac{\alpha^{2k}}{(2k)!} + \gamma_{10}\sum \frac{\alpha^{2k+1}}{(2k+1)!}
= \cosh(\alpha) + \gamma_{10} \sinh(\alpha)
\end{equation}

\subsection{Spatial rotation}
\index{rotation invariance}

So, this quite a similar form as bivector exponential with a Euclidean metric.  For such a space the bivector had a negative square, just like the complex unit imaginary,
which allowed for the normal trigonometric split of the exponential:

\begin{equation}
\exp(\Be_{12}\theta)
= \sum (-1)^k\frac{\theta^{2k}}{(2k)!} + \Be_{12}\sum (-1)^k\frac{\theta^{2k+1}}{(2k+1)!}
= \cos(\theta) + \Be_{12} \sin(\theta)
\end{equation}

Now, with the Minkowski metric having a negative square for purely spatial components, how does a purely spacial bivector behave when squared?  Let us try it with

\begin{equation*}
{\gamma_{12}}^2
= \gamma_{1212}
= -\gamma_{1221}
= \gamma_{11}
= -1
\end{equation*}

This also has a square that behaves like the unit imaginary, so we can do spacial rotations with rotors like we can with Euclidean space.  However, we have to invert the sign of the angle when using a Minkowski metric.  Take a specific example of a 90 degree rotation in the x-y plane, expressed in complex form:

\begin{equation}\label{eqn:fourvecDotinvariance:60}
\begin{aligned}
R_{\pi/2}(\gamma_1)
&= \gamma_1 \exp({ \gamma_{12} \pi/2 }) \\
&= \gamma_1 (0 + \gamma_{12}) \\
&= -\gamma_2 \\
\end{aligned}
\end{equation}

In general our Rotor equation with a Minkowski \((+,-,-,-)\) metric will be thus be:

\begin{equation}\label{eqn:fVecDotInv:spacerot}
R_{\theta}(x) = \exp( i\theta/2) x \exp( -i\theta/2)
\end{equation}

Here \(i\) is a spatial bivector (a bivector with negative square), such as \(\gamma_{1}\wedge\gamma_{2}\), and the rotation sense is with increasing angle from \(\gamma_1\) towards \(\gamma_2\).

\subsection{Validity of the double sided spatial rotor formula}

To demonstrate the validity of \eqnref{eqn:fVecDotInv:spacerot} one has to observe how the unit vectors \(\gamma_{\mu}\) behave with respect to commutation, and how that behavior results in either commutation or conjugate commutation with the exponential rotor.  Without any loss of generality one can restrict attention to a specific example, such as bivector \(\gamma_{12}\).  By inspection, \(\gamma_0\), and \(\gamma_3\) both commute since an even number of exchanges in position is required for either:

\begin{equation}\label{eqn:fourvecDotinvariance:80}
\begin{aligned}
\gamma_{0} \gamma_{12}
&= \gamma_{0} \wedge \gamma_{1} \wedge \gamma_{2} \\
&= \gamma_{1} \wedge \gamma_{2} \wedge \gamma_{0} \\
&= \gamma_{12} \gamma_0
\end{aligned}
\end{equation}

For this reason, application of the double sided rotation does not change any such (perpendicular) vector that commutes with the rotor:

\begin{equation}\label{eqn:fourvecDotinvariance:100}
\begin{aligned}
R_{\theta}(x_{\perp})
&= \exp( i\theta/2) x_{\perp} \exp( -i\theta/2) \\
&= x_{\perp} \exp( i\theta/2) \exp( -i\theta/2) \\
&= x_{\perp}
\end{aligned}
\end{equation}

Now for the basis vectors that lie in the plane of the spatial rotation we have anticommutation:

\begin{equation}\label{eqn:fourvecDotinvariance:120}
\begin{aligned}
\gamma_{1} \gamma_{12}
&= -\gamma_{1} \gamma_{21}  \\
&= -\gamma_{121} \\
&= -\gamma_{12} \gamma_{1}
\end{aligned}
\end{equation}

\begin{equation}\label{eqn:fourvecDotinvariance:140}
\begin{aligned}
\gamma_{2} \gamma_{12}
&= \gamma_{21}\gamma_{2} \\
&= -\gamma_{12}\gamma_{2}
\end{aligned}
\end{equation}

Given an understanding of how the unit vectors either commute or anticommute with the bivector for the plane of rotation, one can now see how these behave when multiplied by a rotor expressed exponentially:

\begin{equation}\label{eqn:fVecDotInv:spatialcommutationrule}
\gamma_{\mu}\exp(i\theta)
= \gamma_{\mu}\left( \cos(\theta) + i\sin(\theta) \right)
=
\left\{
\begin{array}{l l}
\left( \cos(\theta) + i\sin(\theta) \right) \gamma_{\mu} & \quad \mbox{if \(\gamma_{\mu} \cdot i = 0\)} \\
\left( \cos(\theta) - i\sin(\theta) \right) \gamma_{\mu} & \quad \mbox{if \(\gamma_{\mu} \cdot i \ne 0\)} \\
\end{array} \right.
\end{equation}

The condition \(\gamma_{\mu} \cdot i = 0\) corresponds to a spacelike vector perpendicular to the plane of rotation, or a timelike vector, or any combination of the two, whereas
\(\gamma_{\mu} \cdot i \ne 0\) is true for any spacelike vector that lies completely in the plane of rotation.

Putting this information all together, we now complete the verification that the double sided rotor formula leaves the perpendicular spacelike or the timelike components untouched.  For for purely spacelike vectors in the plane of rotation we recover the single sided complex form rotation as illustrated by the following x-y plane rotation:

\begin{equation}\label{eqn:fourvecDotinvariance:160}
\begin{aligned}
R_{\theta}(x_{\parallel})
&= \exp( \gamma_{12}\theta/2) x_{\parallel} \exp( -\gamma_{12}\theta/2) \\
&= x_{\parallel} \exp( -\gamma_{12}\theta/2) \exp( -\gamma_{12}\theta/2) \\
&= x_{\parallel} \exp( -\gamma_{12}\theta) \\
\end{aligned}
\end{equation}

\subsection{Back to time space rotation}

Now, like we can express a spatial rotation in exponential form, we can do the same for the hyperbolic ``rotation'' matrix of \eqnref{eqn:fVecDotInv:hyperbolicmatrix}.  Direct expansion
\footnote{
The paper ``Generalized relativistic velocity addition with spacetime algebra'', http://arxiv.org/pdf/physics/0511247.pdf derives the bivector form of this Lorentz boost directly in an interesting fashion.  Simple relativistic arguments are used that are quite similar to those of Einstein in his ``Relativity, the special and general theory'' appendix.  This paper is written in a form that requires you to work out many of the details yourself (likely for brevity).  However, once that extra work is done, I found the first half of that paper quite readable.
}
of the product is the easiest way to see that this is the case:

\begin{equation}\label{eqn:fourvecDotinvariance:180}
\begin{aligned}
\left(\gamma_{1} x^1 + \gamma_{0} x^0 \right)\exp(\gamma_{10}\alpha)
&= \left(\gamma_{1} x^1 + \gamma_{0} x^0 \right) \left( \cosh(\alpha) +\gamma_{10}\sinh(\alpha) \right) \\
\end{aligned}
\end{equation}

\begin{equation}\label{eqn:fVecDotInv:lorentz}
\begin{aligned}
&\left(\gamma_{1} x^1 + \gamma_{0} x^0 \right)\exp(\gamma_{10}\alpha) \\
&\qquad = \gamma_1\left( x^1 \cosh(\alpha) - x^0 \sinh(\alpha)\right)
 + \gamma_0\left( x^0 \cosh(\alpha) - x^1 \sinh(\alpha)\right)
\end{aligned}
\end{equation}

As with the spatial rotation, full characterization of this exponential rotation operator, in both single and double sided form requires that one looks at how the various unit vectors commute with the unit bivector.  Without loss of generality one can restrict attention to a specific case, as done with the \(\gamma_{10}\) above.

As in the spatial case, \(\gamma_{2}\), and \(\gamma_{3}\) both commute with \(\gamma_{10} = \gamma_1 \wedge \gamma_0\).  Example:

\begin{equation*}
\gamma_{2} \gamma_{10}
= \gamma_2 \wedge \gamma_1 \wedge \gamma_0
= \gamma_1 \wedge \gamma_0 \wedge \gamma_2
= \gamma_{10} \gamma_{2}
\end{equation*}

Now, consider each of the basis vectors in the spacetime plane.

\begin{equation*}
\gamma_{0} \gamma_{10}
= \gamma_{010}
= \gamma_{01} \gamma_{0}
= -\gamma_{10} \gamma_{0}
\end{equation*}

\begin{equation*}
\gamma_{1} \gamma_{10}
= \gamma_{110}
= -\gamma_{101}
= -\gamma_{10} \gamma_{1}
\end{equation*}

Both of the basis vectors in the spacetime plane anticommute with the bivector that describes the plane, and as a result we have a conjugate change in the exponential comparing left and right multiplication as with a spatial rotor.  Summarizing for the general case by introducing a spacetime rapidity plane described by a bivector

\(\Balpha = \alphacap \alpha\), we have:

\begin{equation}\label{eqn:fVecDotInv:spacetimecommutationrule}
\begin{aligned}
\gamma_{\mu}\exp(\Balpha)
&= \gamma_{\mu}\left( \cosh(\alpha) + \alphacap\sinh(\alpha) \right) \\
&=
\left\{
\begin{array}{l l}
\left( \cosh(\alpha) + \alphacap\sinh(\alpha) \right) \gamma_{\mu} & \quad \mbox{if \(\gamma_{\mu} \cdot \alphacap = 0\)} \\
\left( \cosh(\alpha) - \alphacap\sinh(\alpha) \right) \gamma_{\mu} & \quad \mbox{if \(\gamma_{\mu} \cdot \alphacap \ne 0\)} \\
\end{array} \right.
\end{aligned}
\end{equation}

Observe the similarity between \eqnref{eqn:fVecDotInv:spatialcommutationrule}, and \eqnref{eqn:fVecDotInv:spacetimecommutationrule} for spatial
and spacetime rotors.  Regardless of whether the plane is spacelike, or a spacetime plane we have the same rule:

\begin{equation}\label{eqn:fVecDotInv:generalrule}
\gamma_{\mu}\exp(\BB)
=
\left\{
\begin{array}{l l}
\exp(\BB) \gamma_{\mu} & \quad \mbox{if \(\gamma_{\mu} \cdot \Bcap = 0\)} \\
\exp(-\BB) \gamma_{\mu} & \quad \mbox{if \(\gamma_{\mu} \cdot \Bcap \ne 0\)}
\end{array} \right.
\end{equation}

Here, if \(\BB\) is a spacelike bivector (\(\BB^2 < 0\)) we get trigonometric functions generated by the exponentials, and if it represents
the spacetime plane \(\BB^2 > 0\) we get the hyperbolic functions.  As with the spatial rotor formulation, we have the same result for the
general signature bivector, and can write the generalized spacetime or spatial rotation as:

\begin{equation}
R_{\BB}(x) = \exp(-\BB/2) x \exp(\BB/2)
\end{equation}

Some care is required assigning meaning to the bivector angle \(\BB\).  We have seen that this is an negatively oriented spatial rotation in the $
\Bcap$ plane when spacelike.  How about for the spacetime case?
Lets go back and rewrite \eqnref{eqn:fVecDotInv:lorentz} in terms of vector
relations, with \(\Bv = v \vcap\)

\begin{equation}
\begin{aligned}
&\left( x^1 \vcap + x^0 \gamma_0 \right)
\left(
\frac{1}{\sqrt{1 -{\abs{(\Bv/c)}}^2}} + \frac{(\Bv/c) \gamma_0}{\sqrt{1 -{\abs{(\Bv/c)}}^2}}
\right) \\
&\qquad =
\vcap \gamma
\left( x^1 - x^0 v/c \right)
+
\gamma_0 \gamma
\left( x^0 - x^1 v/c \right)
\end{aligned}
\end{equation}

This allows for the following identification:

\begin{equation*}
\cosh(\alpha) + \vcap \gamma_0 \sinh(\alpha) = \exp( \vcap \gamma_0 \alpha)
=
\frac{1 + (\Bv/c) \gamma_0}{\sqrt{1 -{\abs{\Bv/c}}^2}}
\end{equation*}

which gives us the rapidity bivector (\(\BB\) above) in terms of the values we are familiar with:

\begin{equation*}
\vcap \gamma_0 \alpha = \log\left(
\frac{1 + (\Bv/c) \gamma_0}{\sqrt{1 -{\abs{\Bv/c}}^2}} \right)
\end{equation*}

Or,

\begin{equation*}
\BB = \vcap \gamma_0 \alpha = \tanh^{-1}(v/c) \vcap \gamma_0
\end{equation*}

Now since \(\abs{v/c} < 1\), the hyperbolic inverse tangent here can be expanded in (the slowly convergent) power series:

\begin{equation*}
\tanh^{-1}(x) = \sum_{k=0} \frac{x^{2k+1}}{2k+1}
\end{equation*}

Observe that this has only odd powers, and \(((\Bv/c) \gamma_0)^{2k+1} = \vcap\gamma_0 (v/c)^{2k+1}\).  This allows for the notational nicety of working with the spacetime bivector directly instead of only its magnitude:

\begin{equation}
\BB = \tanh^{-1}((\Bv/c) \gamma_0)
\end{equation}

\subsection{FIXME}

Revisit the equivalence of the two identities above.  How can one get from the log
expression to the hyperbolic inverse tangent directly?

\subsection{Apply to dot product invariance}

With composition of rotation and boost rotors we can form a generalized Lorentz transformation.  For example application of a rotation with rotor \(R\), to a boost with spacetime rotor \(L_0\), we get a combined more general transformation:

\begin{equation*}
L(x) = R ( L_0 x {L_0}^\dagger ) R^\dagger
\end{equation*}

In both cases, the rotor and its reverse when multiplied are identity:

\begin{equation*}
1 = R R^\dagger = L L^\dagger
\end{equation*}

It is not hard to see one can also compose an arbitrary set of rotations and boosts in the same fashion.  The new rotor will also satisfy \(L L^\dagger = 1\).

Application of such a rotor to a four vector we have:

\begin{equation*}
X' = L X L^\dagger
\end{equation*}

\begin{equation*}
Y' = L Y L^\dagger
\end{equation*}

\begin{equation}\label{eqn:fourvecDotinvariance:200}
\begin{aligned}
X' \cdot Y' &= (L X L^\dagger) \cdot (L Y L^\dagger) \\
&= \gpgradezero{ L X L^\dagger L Y L^\dagger } \\
&= \gpgradezero{ L X Y L^\dagger } \\
&= \gpgradezero{ L (X \cdot Y) L^\dagger } + \gpgradezero{ L (X \wedge Y) L^\dagger } \\
&= (X \cdot Y) \gpgradezero{ L L^\dagger } \\
&= X \cdot Y
\end{aligned}
\end{equation}

It is also clear that the four bivector \(X \wedge Y\) will also be Lorentz invariant.  This also implies that the geometric product of two four vectors \(X Y\) will also be Lorentz invariant.

UPDATE (Aug 14): I
do not
recall my reasons for thinking that the bivector invariance was clear initially.  It does not seem so clear now after the fact so I should have written it down.

\documentclass{article}      % Specifies the document class

\usepackage{amsmath}
\usepackage{mathpazo}

%
% shorthand for bold symbols, convenient for vectors and matrices
%
\newcommand{\Ba}[0]{\mathbf{a}}
\newcommand{\Bb}[0]{\mathbf{b}}
\newcommand{\Bc}[0]{\mathbf{c}}
\newcommand{\Bd}[0]{\mathbf{d}}
\newcommand{\Be}[0]{\mathbf{e}}
\newcommand{\Bf}[0]{\mathbf{f}}
\newcommand{\Bg}[0]{\mathbf{g}}
\newcommand{\Bh}[0]{\mathbf{h}}
\newcommand{\Bi}[0]{\mathbf{i}}
\newcommand{\Bj}[0]{\mathbf{j}}
\newcommand{\Bk}[0]{\mathbf{k}}
\newcommand{\Bl}[0]{\mathbf{l}}
\newcommand{\Bm}[0]{\mathbf{m}}
\newcommand{\Bn}[0]{\mathbf{n}}
\newcommand{\Bo}[0]{\mathbf{o}}
\newcommand{\Bp}[0]{\mathbf{p}}
\newcommand{\Bq}[0]{\mathbf{q}}
\newcommand{\Br}[0]{\mathbf{r}}
\newcommand{\Bs}[0]{\mathbf{s}}
\newcommand{\Bt}[0]{\mathbf{t}}
\newcommand{\Bu}[0]{\mathbf{u}}
\newcommand{\Bv}[0]{\mathbf{v}}
\newcommand{\Bw}[0]{\mathbf{w}}
\newcommand{\Bx}[0]{\mathbf{x}}
\newcommand{\By}[0]{\mathbf{y}}
\newcommand{\Bz}[0]{\mathbf{z}}
\newcommand{\BA}[0]{\mathbf{A}}
\newcommand{\BB}[0]{\mathbf{B}}
\newcommand{\BC}[0]{\mathbf{C}}
\newcommand{\BD}[0]{\mathbf{D}}
\newcommand{\BE}[0]{\mathbf{E}}
\newcommand{\BF}[0]{\mathbf{F}}
\newcommand{\BG}[0]{\mathbf{G}}
\newcommand{\BH}[0]{\mathbf{H}}
\newcommand{\BI}[0]{\mathbf{I}}
\newcommand{\BJ}[0]{\mathbf{J}}
\newcommand{\BK}[0]{\mathbf{K}}
\newcommand{\BL}[0]{\mathbf{L}}
\newcommand{\BM}[0]{\mathbf{M}}
\newcommand{\BN}[0]{\mathbf{N}}
\newcommand{\BO}[0]{\mathbf{O}}
\newcommand{\BP}[0]{\mathbf{P}}
\newcommand{\BQ}[0]{\mathbf{Q}}
\newcommand{\BR}[0]{\mathbf{R}}
\newcommand{\BS}[0]{\mathbf{S}}
\newcommand{\BT}[0]{\mathbf{T}}
\newcommand{\BU}[0]{\mathbf{U}}
\newcommand{\BV}[0]{\mathbf{V}}
\newcommand{\BW}[0]{\mathbf{W}}
\newcommand{\BX}[0]{\mathbf{X}}
\newcommand{\BY}[0]{\mathbf{Y}}
\newcommand{\BZ}[0]{\mathbf{Z}}

\newcommand{\Bzero}[0]{\mathbf{0}}
\newcommand{\Btheta}[0]{\boldsymbol{\theta}}
\newcommand{\Btau}[0]{\boldsymbol{\tau}}
\newcommand{\Bomega}[0]{\boldsymbol{\omega}}

%
% shorthand for unit vectors
%
\newcommand{\acap}[0]{\hat{\Ba}}
\newcommand{\bcap}[0]{\hat{\Bb}}
\newcommand{\ccap}[0]{\hat{\Bc}}
\newcommand{\dcap}[0]{\hat{\Bd}}
\newcommand{\ecap}[0]{\hat{\Be}}
\newcommand{\fcap}[0]{\hat{\Bf}}
\newcommand{\gcap}[0]{\hat{\Bg}}
\newcommand{\hcap}[0]{\hat{\Bh}}
\newcommand{\icap}[0]{\hat{\Bi}}
\newcommand{\jcap}[0]{\hat{\Bj}}
\newcommand{\kcap}[0]{\hat{\Bk}}
\newcommand{\lcap}[0]{\hat{\Bl}}
\newcommand{\mcap}[0]{\hat{\Bm}}
\newcommand{\ncap}[0]{\hat{\Bn}}
\newcommand{\ocap}[0]{\hat{\Bo}}
\newcommand{\pcap}[0]{\hat{\Bp}}
\newcommand{\qcap}[0]{\hat{\Bq}}
\newcommand{\rcap}[0]{\hat{\Br}}
\newcommand{\scap}[0]{\hat{\Bs}}
\newcommand{\tcap}[0]{\hat{\Bt}}
\newcommand{\ucap}[0]{\hat{\Bu}}
\newcommand{\vcap}[0]{\hat{\Bv}}
\newcommand{\wcap}[0]{\hat{\Bw}}
\newcommand{\xcap}[0]{\hat{\Bx}}
\newcommand{\ycap}[0]{\hat{\By}}
\newcommand{\zcap}[0]{\hat{\Bz}}
\newcommand{\thetacap}[0]{\hat{\Btheta}}

%
% to write R^n and C^n in a distinguishable fashion.  Perhaps change this
% to the double lined characters upon figuring out how to do so.
%
\newcommand{\C}[1]{$\mathbb{C}^{#1}$}
\newcommand{\R}[1]{$\mathbb{R}^{#1}$}

%
% various generally useful helpers
%

% derivative of #1 wrt. #2:
\newcommand{\D}[2] {\frac {d#2} {d#1}}

\newcommand{\inv}[1]{\frac{1}{#1}}
\newcommand{\cross}[0]{\times}

\newcommand{\abs}[1]{\lvert{#1}\rvert}
\newcommand{\norm}[1]{\lVert{#1}\rVert}
\newcommand{\innerprod}[2]{\langle{#1}, {#2}\rangle}
\newcommand{\dotprod}[2]{{#1} \cdot {#2}}
\newcommand{\bdotprod}[2]{\left({#1} \cdot {#2}\right)}
\newcommand{\crossprod}[2]{{#1} \cross {#2}}
\newcommand{\tripleprod}[3]{\dotprod{\left(\crossprod{#1}{#2}\right)}{#3}}

\DeclareMathOperator{\Proj}{Proj}
\DeclareMathOperator{\Span}{span}
\DeclareMathOperator{\Sgn}{sgn}
\DeclareMathOperator{\Area}{Area}
\DeclareMathOperator{\Volume}{Volume}

%
% A few miscellaneous things specific to this document
%
\newcommand{\crossop}[1]{\crossprod{#1}{}}

% R2 vector.
\newcommand{\VectorTwo}[2]{
\begin{bmatrix}
 {#1} \\
 {#2}
\end{bmatrix}
}

\newcommand{\VectorN}[1]{
\begin{bmatrix}
{#1}_1 \\
{#1}_2 \\
\vdots \\
{#1}_N \\
\end{bmatrix}
}

\newcommand{\DETuvij}[4]{
\begin{vmatrix}
 {#1}_{#3} & {#1}_{#4} \\
 {#2}_{#3} & {#2}_{#4}
\end{vmatrix}
}

\newcommand{\DETuvwijk}[6]{
\begin{vmatrix}
 {#1}_{#4} & {#1}_{#5} & {#1}_{#6} \\
 {#2}_{#4} & {#2}_{#5} & {#2}_{#6} \\
 {#3}_{#4} & {#3}_{#5} & {#3}_{#6}
\end{vmatrix}
}

\newcommand{\DETuvwxijkl}[8]{
\begin{vmatrix}
 {#1}_{#5} & {#1}_{#6} & {#1}_{#7} & {#1}_{#8} \\
 {#2}_{#5} & {#2}_{#6} & {#2}_{#7} & {#2}_{#8} \\
 {#3}_{#5} & {#3}_{#6} & {#3}_{#7} & {#3}_{#8} \\
 {#4}_{#5} & {#4}_{#6} & {#4}_{#7} & {#4}_{#8} \\
\end{vmatrix}
}

%\newcommand{\DETuvwxyijklm}[10]{
%\begin{vmatrix}
% {#1}_{#6} & {#1}_{#7} & {#1}_{#8} & {#1}_{#9} & {#1}_{#10} \\
% {#2}_{#6} & {#2}_{#7} & {#2}_{#8} & {#2}_{#9} & {#2}_{#10} \\
% {#3}_{#6} & {#3}_{#7} & {#3}_{#8} & {#3}_{#9} & {#3}_{#10} \\
% {#4}_{#6} & {#4}_{#7} & {#4}_{#8} & {#4}_{#9} & {#4}_{#10} \\
% {#5}_{#6} & {#5}_{#7} & {#5}_{#8} & {#5}_{#9} & {#5}_{#10}
%\end{vmatrix}
%}

% R3 vector.
\newcommand{\VectorThree}[3]{
\begin{bmatrix}
 {#1} \\
 {#2} \\
 {#3}
\end{bmatrix}
}


\newcommand{\spacegrad}[0]{\boldsymbol{\nabla}}
\newcommand{\grad}[0]{\nabla}

%
% The real thing:
%

                             % The preamble begins here.
\title{ Lorentz transformation of spacetime gradient }
\author{Peeter Joot}         % Declares the author's name.
%\date{}        % Deleting this command produces today's date.

\begin{document}             % End of preamble and beginning of text.

\maketitle{}

\section{ Motivation. }

We have observed that the wave equation is Lorentz invarient, and conversely that invarience of the form of the wave equation under linear transformation for light can be used to calculate the Lorentz transformation.  Specifically, this means that we require the equations of light (wave equation) retain its form after a change of variables that includes
a (possibly scaled) translation.  The wave equation should have no mixed partial terms, and retain the form:

\begin{equation*}
(\spacegrad^2 - \partial_{ct}^2) F = ({\spacegrad'}^2 - \partial_{ct'}^2) F = 0
\end{equation*}

Having expressed the spacetime gradient with a (STA) Minkowski basis, and knowing that the Maxwell equation written using the spacetime gradient is Lorentz invarient:

\begin{equation*}
\grad F = J,
\end{equation*}

we therefore expect that the square root of the wave equation (Laplacian) operator is also Lorentz invarient.  Here this idea is explored, and we look at how the spacetime
gradient behaves under Lorentz transformation.

Our spacetime gradient is

\begin{equation*}
\grad = \sum \gamma^{\mu} \frac{\partial}{\partial x^{\mu}}
\end{equation*}

Under Lorentz transformation we can transform the $x^1=x$, and $x^0 = ct$ coordinates:

\begin{equation*}
\begin{bmatrix}
x' \\
ct' \\
\end{bmatrix}
=
\gamma
\begin{bmatrix}
1 & -\beta \\
-\beta & 1 \\
\end{bmatrix}
\begin{bmatrix}
x \\
ct \\
\end{bmatrix}
\end{equation*}

%\begin{equation*}
%\end{equation*}
\end{document}               % End of document.

\documentclass{article}

\usepackage{amsmath}
\usepackage{mathpazo}

%
% shorthand for bold symbols, convenient for vectors and matrices
%
\newcommand{\Ba}[0]{\mathbf{a}}
\newcommand{\Bb}[0]{\mathbf{b}}
\newcommand{\Bc}[0]{\mathbf{c}}
\newcommand{\Bd}[0]{\mathbf{d}}
\newcommand{\Be}[0]{\mathbf{e}}
\newcommand{\Bf}[0]{\mathbf{f}}
\newcommand{\Bg}[0]{\mathbf{g}}
\newcommand{\Bh}[0]{\mathbf{h}}
\newcommand{\Bi}[0]{\mathbf{i}}
\newcommand{\Bj}[0]{\mathbf{j}}
\newcommand{\Bk}[0]{\mathbf{k}}
\newcommand{\Bl}[0]{\mathbf{l}}
\newcommand{\Bm}[0]{\mathbf{m}}
\newcommand{\Bn}[0]{\mathbf{n}}
\newcommand{\Bo}[0]{\mathbf{o}}
\newcommand{\Bp}[0]{\mathbf{p}}
\newcommand{\Bq}[0]{\mathbf{q}}
\newcommand{\Br}[0]{\mathbf{r}}
\newcommand{\Bs}[0]{\mathbf{s}}
\newcommand{\Bt}[0]{\mathbf{t}}
\newcommand{\Bu}[0]{\mathbf{u}}
\newcommand{\Bv}[0]{\mathbf{v}}
\newcommand{\Bw}[0]{\mathbf{w}}
\newcommand{\Bx}[0]{\mathbf{x}}
\newcommand{\By}[0]{\mathbf{y}}
\newcommand{\Bz}[0]{\mathbf{z}}
\newcommand{\BA}[0]{\mathbf{A}}
\newcommand{\BB}[0]{\mathbf{B}}
\newcommand{\BC}[0]{\mathbf{C}}
\newcommand{\BD}[0]{\mathbf{D}}
\newcommand{\BE}[0]{\mathbf{E}}
\newcommand{\BF}[0]{\mathbf{F}}
\newcommand{\BG}[0]{\mathbf{G}}
\newcommand{\BH}[0]{\mathbf{H}}
\newcommand{\BI}[0]{\mathbf{I}}
\newcommand{\BJ}[0]{\mathbf{J}}
\newcommand{\BK}[0]{\mathbf{K}}
\newcommand{\BL}[0]{\mathbf{L}}
\newcommand{\BM}[0]{\mathbf{M}}
\newcommand{\BN}[0]{\mathbf{N}}
\newcommand{\BO}[0]{\mathbf{O}}
\newcommand{\BP}[0]{\mathbf{P}}
\newcommand{\BQ}[0]{\mathbf{Q}}
\newcommand{\BR}[0]{\mathbf{R}}
\newcommand{\BS}[0]{\mathbf{S}}
\newcommand{\BT}[0]{\mathbf{T}}
\newcommand{\BU}[0]{\mathbf{U}}
\newcommand{\BV}[0]{\mathbf{V}}
\newcommand{\BW}[0]{\mathbf{W}}
\newcommand{\BX}[0]{\mathbf{X}}
\newcommand{\BY}[0]{\mathbf{Y}}
\newcommand{\BZ}[0]{\mathbf{Z}}

\newcommand{\Bzero}[0]{\mathbf{0}}
\newcommand{\Btheta}[0]{\boldsymbol{\theta}}
\newcommand{\Btau}[0]{\boldsymbol{\tau}}
\newcommand{\Bomega}[0]{\boldsymbol{\omega}}

%
% shorthand for unit vectors
%
\newcommand{\acap}[0]{\hat{\Ba}}
\newcommand{\bcap}[0]{\hat{\Bb}}
\newcommand{\ccap}[0]{\hat{\Bc}}
\newcommand{\dcap}[0]{\hat{\Bd}}
\newcommand{\ecap}[0]{\hat{\Be}}
\newcommand{\fcap}[0]{\hat{\Bf}}
\newcommand{\gcap}[0]{\hat{\Bg}}
\newcommand{\hcap}[0]{\hat{\Bh}}
\newcommand{\icap}[0]{\hat{\Bi}}
\newcommand{\jcap}[0]{\hat{\Bj}}
\newcommand{\kcap}[0]{\hat{\Bk}}
\newcommand{\lcap}[0]{\hat{\Bl}}
\newcommand{\mcap}[0]{\hat{\Bm}}
\newcommand{\ncap}[0]{\hat{\Bn}}
\newcommand{\ocap}[0]{\hat{\Bo}}
\newcommand{\pcap}[0]{\hat{\Bp}}
\newcommand{\qcap}[0]{\hat{\Bq}}
\newcommand{\rcap}[0]{\hat{\Br}}
\newcommand{\scap}[0]{\hat{\Bs}}
\newcommand{\tcap}[0]{\hat{\Bt}}
\newcommand{\ucap}[0]{\hat{\Bu}}
\newcommand{\vcap}[0]{\hat{\Bv}}
\newcommand{\wcap}[0]{\hat{\Bw}}
\newcommand{\xcap}[0]{\hat{\Bx}}
\newcommand{\ycap}[0]{\hat{\By}}
\newcommand{\zcap}[0]{\hat{\Bz}}
\newcommand{\thetacap}[0]{\hat{\Btheta}}

%
% to write R^n and C^n in a distinguishable fashion.  Perhaps change this
% to the double lined characters upon figuring out how to do so.
%
\newcommand{\C}[1]{$\mathbb{C}^{#1}$}
\newcommand{\R}[1]{$\mathbb{R}^{#1}$}

%
% various generally useful helpers
%

% derivative of #1 wrt. #2:
\newcommand{\D}[2] {\frac {d#2} {d#1}}

\newcommand{\inv}[1]{\frac{1}{#1}}
\newcommand{\cross}[0]{\times}

\newcommand{\abs}[1]{\lvert{#1}\rvert}
\newcommand{\norm}[1]{\lVert{#1}\rVert}
\newcommand{\innerprod}[2]{\langle{#1}, {#2}\rangle}
\newcommand{\dotprod}[2]{{#1} \cdot {#2}}
\newcommand{\bdotprod}[2]{\left({#1} \cdot {#2}\right)}
\newcommand{\crossprod}[2]{{#1} \cross {#2}}
\newcommand{\tripleprod}[3]{\dotprod{\left(\crossprod{#1}{#2}\right)}{#3}}

\DeclareMathOperator{\Proj}{Proj}
\DeclareMathOperator{\Span}{span}
\DeclareMathOperator{\Sgn}{sgn}
\DeclareMathOperator{\Area}{Area}
\DeclareMathOperator{\Volume}{Volume}

%
% A few miscellaneous things specific to this document
%
\newcommand{\crossop}[1]{\crossprod{#1}{}}

% R2 vector.
\newcommand{\VectorTwo}[2]{
\begin{bmatrix}
 {#1} \\
 {#2}
\end{bmatrix}
}

\newcommand{\VectorN}[1]{
\begin{bmatrix}
{#1}_1 \\
{#1}_2 \\
\vdots \\
{#1}_N \\
\end{bmatrix}
}

\newcommand{\DETuvij}[4]{
\begin{vmatrix}
 {#1}_{#3} & {#1}_{#4} \\
 {#2}_{#3} & {#2}_{#4}
\end{vmatrix}
}

\newcommand{\DETuvwijk}[6]{
\begin{vmatrix}
 {#1}_{#4} & {#1}_{#5} & {#1}_{#6} \\
 {#2}_{#4} & {#2}_{#5} & {#2}_{#6} \\
 {#3}_{#4} & {#3}_{#5} & {#3}_{#6}
\end{vmatrix}
}

\newcommand{\DETuvwxijkl}[8]{
\begin{vmatrix}
 {#1}_{#5} & {#1}_{#6} & {#1}_{#7} & {#1}_{#8} \\
 {#2}_{#5} & {#2}_{#6} & {#2}_{#7} & {#2}_{#8} \\
 {#3}_{#5} & {#3}_{#6} & {#3}_{#7} & {#3}_{#8} \\
 {#4}_{#5} & {#4}_{#6} & {#4}_{#7} & {#4}_{#8} \\
\end{vmatrix}
}

%\newcommand{\DETuvwxyijklm}[10]{
%\begin{vmatrix}
% {#1}_{#6} & {#1}_{#7} & {#1}_{#8} & {#1}_{#9} & {#1}_{#10} \\
% {#2}_{#6} & {#2}_{#7} & {#2}_{#8} & {#2}_{#9} & {#2}_{#10} \\
% {#3}_{#6} & {#3}_{#7} & {#3}_{#8} & {#3}_{#9} & {#3}_{#10} \\
% {#4}_{#6} & {#4}_{#7} & {#4}_{#8} & {#4}_{#9} & {#4}_{#10} \\
% {#5}_{#6} & {#5}_{#7} & {#5}_{#8} & {#5}_{#9} & {#5}_{#10}
%\end{vmatrix}
%}

% R3 vector.
\newcommand{\VectorThree}[3]{
\begin{bmatrix}
 {#1} \\
 {#2} \\
 {#3}
\end{bmatrix}
}


%<misc>
%
\newcommand{\Abs}[1]{{\left\lvert{#1}\right\rvert}}
\newcommand{\spacegrad}[0]{\boldsymbol{\nabla}}
\newcommand{\grad}[0]{\nabla}
\newcommand{\LL}[0]{\mathcal{L}}

% == \partial_{#1} {#2}
\newcommand{\PD}[2]{\frac{\partial {#2}}{\partial {#1}}}
% inline variant
\newcommand{\PDi}[2]{{\partial {#2}}/{\partial {#1}}}

\newcommand{\PDD}[3]{\frac{\partial^2 {#3}}{\partial {#1}\partial {#2}}}
%\newcommand{\PDd}[2]{\frac{\partial^2 {#2}}{{\partial{#1}}^2}}
\newcommand{\PDsq}[2]{\frac{\partial^2 {#2}}{(\partial {#1})^2}}

\newcommand{\Partial}[2]{\frac{\partial {#1}}{\partial {#2}}}
\DeclareMathOperator{\RejName}{Rej}
\newcommand{\Rej}[2]{\RejName_{#1}\left( {#2} \right)}
\newcommand{\Rm}[1]{\mathbb{R}^{#1}}
\newcommand{\Cm}[1]{\mathbb{C}^{#1}}
\newcommand{\conj}[0]{{*}}

%</misc>

% <grade selection>
%
\newcommand{\gpgrade}[2] {{\left\langle{{#1}}\right\rangle}_{#2}}

\newcommand{\gpgradezero}[1] {\gpgrade{#1}{}}
%\newcommand{\gpscalargrade}[1] {{\left\langle{{#1}}\right\rangle}}
%\newcommand{\gpgradezero}[1] {\gpgrade{#1}{0}}

%\newcommand{\gpgradeone}[1] {{\left\langle{{#1}}\right\rangle}_{1}}
\newcommand{\gpgradeone}[1] {\gpgrade{#1}{1}}

\newcommand{\gpgradetwo}[1] {\gpgrade{#1}{2}}
\newcommand{\gpgradethree}[1] {\gpgrade{#1}{3}}
\newcommand{\gpgradefour}[1] {\gpgrade{#1}{4}}
%
% </grade selection>



\newcommand{\adot}[0]{{\dot{a}}}
\newcommand{\bdot}[0]{{\dot{b}}}
% taken for centered dot:
%\newcommand{\cdot}[0]{{\dot{c}}}
%\newcommand{\ddot}[0]{{\dot{d}}}
\newcommand{\edot}[0]{{\dot{e}}}
\newcommand{\fdot}[0]{{\dot{f}}}
\newcommand{\gdot}[0]{{\dot{g}}}
\newcommand{\hdot}[0]{{\dot{h}}}
\newcommand{\idot}[0]{{\dot{i}}}
\newcommand{\jdot}[0]{{\dot{j}}}
\newcommand{\kdot}[0]{{\dot{k}}}
\newcommand{\ldot}[0]{{\dot{l}}}
\newcommand{\mdot}[0]{{\dot{m}}}
\newcommand{\ndot}[0]{{\dot{n}}}
%\newcommand{\odot}[0]{{\dot{o}}}
\newcommand{\pdot}[0]{{\dot{p}}}
\newcommand{\qdot}[0]{{\dot{q}}}
\newcommand{\rdot}[0]{{\dot{r}}}
\newcommand{\sdot}[0]{{\dot{s}}}
\newcommand{\tdot}[0]{{\dot{t}}}
\newcommand{\udot}[0]{{\dot{u}}}
\newcommand{\vdot}[0]{{\dot{v}}}
\newcommand{\wdot}[0]{{\dot{w}}}
\newcommand{\xdot}[0]{{\dot{x}}}
\newcommand{\ydot}[0]{{\dot{y}}}
\newcommand{\zdot}[0]{{\dot{z}}}
\newcommand{\addot}[0]{{\ddot{a}}}
\newcommand{\bddot}[0]{{\ddot{b}}}
\newcommand{\cddot}[0]{{\ddot{c}}}
%\newcommand{\dddot}[0]{{\ddot{d}}}
\newcommand{\eddot}[0]{{\ddot{e}}}
\newcommand{\fddot}[0]{{\ddot{f}}}
\newcommand{\gddot}[0]{{\ddot{g}}}
\newcommand{\hddot}[0]{{\ddot{h}}}
\newcommand{\iddot}[0]{{\ddot{i}}}
\newcommand{\jddot}[0]{{\ddot{j}}}
\newcommand{\kddot}[0]{{\ddot{k}}}
\newcommand{\lddot}[0]{{\ddot{l}}}
\newcommand{\mddot}[0]{{\ddot{m}}}
\newcommand{\nddot}[0]{{\ddot{n}}}
\newcommand{\oddot}[0]{{\ddot{o}}}
\newcommand{\pddot}[0]{{\ddot{p}}}
\newcommand{\qddot}[0]{{\ddot{q}}}
\newcommand{\rddot}[0]{{\ddot{r}}}
\newcommand{\sddot}[0]{{\ddot{s}}}
\newcommand{\tddot}[0]{{\ddot{t}}}
\newcommand{\uddot}[0]{{\ddot{u}}}
\newcommand{\vddot}[0]{{\ddot{v}}}
\newcommand{\wddot}[0]{{\ddot{w}}}
\newcommand{\xddot}[0]{{\ddot{x}}}
\newcommand{\yddot}[0]{{\ddot{y}}}
\newcommand{\zddot}[0]{{\ddot{z}}}

%<bold and dot greek symbols>
%

\newcommand{\Deltadot}[0]{{\dot{\Delta}}}
\newcommand{\Gammadot}[0]{{\dot{\Gamma}}}
\newcommand{\Lambdadot}[0]{{\dot{\Lambda}}}
\newcommand{\Omegadot}[0]{{\dot{\Omega}}}
\newcommand{\Phidot}[0]{{\dot{\Phi}}}
\newcommand{\Pidot}[0]{{\dot{\Pi}}}
\newcommand{\Psidot}[0]{{\dot{\Psi}}}
\newcommand{\Sigmadot}[0]{{\dot{\Sigma}}}
\newcommand{\Thetadot}[0]{{\dot{\Theta}}}
\newcommand{\Upsilondot}[0]{{\dot{\Upsilon}}}
\newcommand{\Xidot}[0]{{\dot{\Xi}}}
\newcommand{\alphadot}[0]{{\dot{\alpha}}}
\newcommand{\betadot}[0]{{\dot{\beta}}}
\newcommand{\chidot}[0]{{\dot{\chi}}}
\newcommand{\deltadot}[0]{{\dot{\delta}}}
\newcommand{\epsilondot}[0]{{\dot{\epsilon}}}
\newcommand{\etadot}[0]{{\dot{\eta}}}
\newcommand{\gammadot}[0]{{\dot{\gamma}}}
\newcommand{\kappadot}[0]{{\dot{\kappa}}}
\newcommand{\lambdadot}[0]{{\dot{\lambda}}}
\newcommand{\mudot}[0]{{\dot{\mu}}}
\newcommand{\nudot}[0]{{\dot{\nu}}}
\newcommand{\omegadot}[0]{{\dot{\omega}}}
\newcommand{\phidot}[0]{{\dot{\phi}}}
\newcommand{\pidot}[0]{{\dot{\pi}}}
\newcommand{\psidot}[0]{{\dot{\psi}}}
\newcommand{\rhodot}[0]{{\dot{\rho}}}
\newcommand{\sigmadot}[0]{{\dot{\sigma}}}
\newcommand{\taudot}[0]{{\dot{\tau}}}
\newcommand{\thetadot}[0]{{\dot{\theta}}}
\newcommand{\upsilondot}[0]{{\dot{\upsilon}}}
\newcommand{\varepsilondot}[0]{{\dot{\varepsilon}}}
\newcommand{\varphidot}[0]{{\dot{\varphi}}}
\newcommand{\varpidot}[0]{{\dot{\varpi}}}
\newcommand{\varrhodot}[0]{{\dot{\varrho}}}
\newcommand{\varsigmadot}[0]{{\dot{\varsigma}}}
\newcommand{\varthetadot}[0]{{\dot{\vartheta}}}
\newcommand{\xidot}[0]{{\dot{\xi}}}
\newcommand{\zetadot}[0]{{\dot{\zeta}}}

\newcommand{\Deltaddot}[0]{{\ddot{\Delta}}}
\newcommand{\Gammaddot}[0]{{\ddot{\Gamma}}}
\newcommand{\Lambdaddot}[0]{{\ddot{\Lambda}}}
\newcommand{\Omegaddot}[0]{{\ddot{\Omega}}}
\newcommand{\Phiddot}[0]{{\ddot{\Phi}}}
\newcommand{\Piddot}[0]{{\ddot{\Pi}}}
\newcommand{\Psiddot}[0]{{\ddot{\Psi}}}
\newcommand{\Sigmaddot}[0]{{\ddot{\Sigma}}}
\newcommand{\Thetaddot}[0]{{\ddot{\Theta}}}
\newcommand{\Upsilonddot}[0]{{\ddot{\Upsilon}}}
\newcommand{\Xiddot}[0]{{\ddot{\Xi}}}
\newcommand{\alphaddot}[0]{{\ddot{\alpha}}}
\newcommand{\betaddot}[0]{{\ddot{\beta}}}
\newcommand{\chiddot}[0]{{\ddot{\chi}}}
\newcommand{\deltaddot}[0]{{\ddot{\delta}}}
\newcommand{\epsilonddot}[0]{{\ddot{\epsilon}}}
\newcommand{\etaddot}[0]{{\ddot{\eta}}}
\newcommand{\gammaddot}[0]{{\ddot{\gamma}}}
\newcommand{\kappaddot}[0]{{\ddot{\kappa}}}
\newcommand{\lambdaddot}[0]{{\ddot{\lambda}}}
\newcommand{\muddot}[0]{{\ddot{\mu}}}
\newcommand{\nuddot}[0]{{\ddot{\nu}}}
\newcommand{\omegaddot}[0]{{\ddot{\omega}}}
\newcommand{\phiddot}[0]{{\ddot{\phi}}}
\newcommand{\piddot}[0]{{\ddot{\pi}}}
\newcommand{\psiddot}[0]{{\ddot{\psi}}}
\newcommand{\rhoddot}[0]{{\ddot{\rho}}}
\newcommand{\sigmaddot}[0]{{\ddot{\sigma}}}
\newcommand{\tauddot}[0]{{\ddot{\tau}}}
\newcommand{\thetaddot}[0]{{\ddot{\theta}}}
\newcommand{\upsilonddot}[0]{{\ddot{\upsilon}}}
\newcommand{\varepsilonddot}[0]{{\ddot{\varepsilon}}}
\newcommand{\varphiddot}[0]{{\ddot{\varphi}}}
\newcommand{\varpiddot}[0]{{\ddot{\varpi}}}
\newcommand{\varrhoddot}[0]{{\ddot{\varrho}}}
\newcommand{\varsigmaddot}[0]{{\ddot{\varsigma}}}
\newcommand{\varthetaddot}[0]{{\ddot{\vartheta}}}
\newcommand{\xiddot}[0]{{\ddot{\xi}}}
\newcommand{\zetaddot}[0]{{\ddot{\zeta}}}

\newcommand{\BDelta}[0]{\boldsymbol{\Delta}}
\newcommand{\BGamma}[0]{\boldsymbol{\Gamma}}
\newcommand{\BLambda}[0]{\boldsymbol{\Lambda}}
\newcommand{\BOmega}[0]{\boldsymbol{\Omega}}
\newcommand{\BPhi}[0]{\boldsymbol{\Phi}}
\newcommand{\BPi}[0]{\boldsymbol{\Pi}}
\newcommand{\BPsi}[0]{\boldsymbol{\Psi}}
\newcommand{\BSigma}[0]{\boldsymbol{\Sigma}}
\newcommand{\BTheta}[0]{\boldsymbol{\Theta}}
\newcommand{\BUpsilon}[0]{\boldsymbol{\Upsilon}}
\newcommand{\BXi}[0]{\boldsymbol{\Xi}}
\newcommand{\Balpha}[0]{\boldsymbol{\alpha}}
\newcommand{\Bbeta}[0]{\boldsymbol{\beta}}
\newcommand{\Bchi}[0]{\boldsymbol{\chi}}
\newcommand{\Bdelta}[0]{\boldsymbol{\delta}}
\newcommand{\Bepsilon}[0]{\boldsymbol{\epsilon}}
\newcommand{\Beta}[0]{\boldsymbol{\eta}}
\newcommand{\Bgamma}[0]{\boldsymbol{\gamma}}
\newcommand{\Bkappa}[0]{\boldsymbol{\kappa}}
\newcommand{\Blambda}[0]{\boldsymbol{\lambda}}
\newcommand{\Bmu}[0]{\boldsymbol{\mu}}
\newcommand{\Bnu}[0]{\boldsymbol{\nu}}
%\newcommand{\Bomega}[0]{\boldsymbol{\omega}}
\newcommand{\Bphi}[0]{\boldsymbol{\phi}}
\newcommand{\Bpi}[0]{\boldsymbol{\pi}}
\newcommand{\Bpsi}[0]{\boldsymbol{\psi}}
\newcommand{\Brho}[0]{\boldsymbol{\rho}}
\newcommand{\Bsigma}[0]{\boldsymbol{\sigma}}
%\newcommand{\Btau}[0]{\boldsymbol{\tau}}
%\newcommand{\Btheta}[0]{\boldsymbol{\theta}}
\newcommand{\Bupsilon}[0]{\boldsymbol{\upsilon}}
\newcommand{\Bvarepsilon}[0]{\boldsymbol{\varepsilon}}
\newcommand{\Bvarphi}[0]{\boldsymbol{\varphi}}
\newcommand{\Bvarpi}[0]{\boldsymbol{\varpi}}
\newcommand{\Bvarrho}[0]{\boldsymbol{\varrho}}
\newcommand{\Bvarsigma}[0]{\boldsymbol{\varsigma}}
\newcommand{\Bvartheta}[0]{\boldsymbol{\vartheta}}
\newcommand{\Bxi}[0]{\boldsymbol{\xi}}
\newcommand{\Bzeta}[0]{\boldsymbol{\zeta}}
%
%</bold and dot greek symbols>
%<infrequent>
%
%\newcommand{\AreaOp}[1]{\AName_{#1}}
%\newcommand{\Babs}[0]{\abs{\BB}}
%\newcommand{\Bcap}[0]{\hat{\BB}}
%\newcommand{\BrPrimeRej}[0]{\rcap(\rcap \wedge \Br')}
%\newcommand{\CA}[0]{\mathcal{A}}
%\newcommand{\Cos}[1]{\cos{\left({#1}\right)}}
%\newcommand{\Det}[1] {\abs{#1}}
%\newcommand{\Dsq}[2] {\frac {\partial^2 {#1}} {\partial {#2}^2}}
%\newcommand{\Exp}[1]{\exp{\left({#1}\right)}}
%\newcommand{\Norm}[1]{\left\lVert{#1}\right\rVert}
%\newcommand{\Sin}[1]{\sin{\left({#1}\right)}}
%\newcommand{\T}[0]{\text{T}}
%\newcommand{\VolumeOp}[1]{\VName_{#1}}
%\newcommand{\agrad}[0]{\Ba \cdot \nabla}
%\newcommand{\alphacap}[0]{\hat{\boldsymbol{\alpha}}}
%\newcommand{\Fcap}[0]{\hat{\BF}}
%\newcommand{\bithree}[0]{{\Bi}_3}
%\newcommand{\bxa}[0]{\Bx\Ba}
%\newcommand{\coordvec}[2]{
%\newcommand{\costheta}[0]{\acap \cdot \xcap}
%\newcommand{\ddt}[1]{\ddot{#1}}
%\newcommand{\ddu}[1] {\frac {d{#1}} {du}}
%\newcommand{\dsqxj}[2] {\frac {\partial^2 {#1}} {\partial {x_{#2}}^2}}
%\newcommand{\dtheta}[1]{\frac{d {#1}}{d \theta}}
%\newcommand{\dt}[1]{\dot{#1}}
%\newcommand{\dt}[1]{\frac{d {#1}}{dt}}
%\newcommand{\dxj}[2] {\frac {\partial {#1}} {\partial {x_{#2}}}}
%\newcommand{\halfPhi}[0]{\frac{\phi}{2}}
%\newcommand{\half}[0]{\inv{2}}
%\newcommand{\inv}[1]{\frac{1}{#1}}
%\newcommand{\laplacian}[0]{\nabla^2}
%\newcommand{\matrixoftx}[3]{
%\newcommand{\nrrp}[0]{\norm{\rcap \wedge \Br'}}
%\newcommand{\oiint}{\bigcirc \hspace{-1.4em} \int \hspace{-.8em} \int}
%\newcommand{\transpose}[1]{{#1}^{\text{T}}}
%\newcommand{\transpose}[1]{{{#1}^{\TextTranspose}}}
%\newcommand{\transpose}[1]{{{#1}^{\text{T}}}}
%\newcommand{\barA}[0]{\bar{A}}
%\newcommand{\qbar}[0]{\bar{q}}
%\newcommand{\qdotbar}[0]{\dot{\bar{q}}}
%
%</infrequent>




\newcommand{\barh}[0]{\bar{h}}

\usepackage[bookmarks=true]{hyperref}

\title{ Some rough notes on GravitoElectroMagnetism. }
\author{Peeter Joot}
\date{ October 26, 2008.  Last Revision: $Date: 2008/10/29 04:01:33 $ }

\begin{document}

\maketitle{}
\tableofcontents

\section{ Motivation. }

I found the GEM equations interesting, and explored the surface of them slightly.  Here are some notes, mostly as a reference for myself ... looking at the
GEM equations mostly generates questions, especially since I don't have the GR
background to understand where the potentials (ie: what is that stress energy
tensor $T_{\mu\nu}$) nor the specifics of where the metric tensor 
(pertubation of the Minkowski metric) came from.

\section{ Definitions. }

The article \cite{mashhoon2003gbr} outlines the GEM equations, which in short
are

Scalar and potential fields

\begin{align}
\Phi \approx \frac{GM}{r}, \quad \BA \approx \frac{G}{c} \frac{\BJ \cross \Bx}{r^3}
\end{align}

Guage condition

\begin{align}
\inv{c}\PD{t}{\Phi} + \spacegrad \cdot \left( \inv{2} \BA \right) = 0.
\end{align}

GEM fields
\begin{align}
\BE = - \spacegrad \Phi -\inv{c} \PD{t}{}\left( \inv{2} \BB \right), \quad \BB = \spacegrad \cross \BA
\end{align}

and finally the Maxwell-like equations are

\begin{align}
\spacegrad \cross \BE &= -\inv{c} \PD{t}{}\left(\inv{2}\BB\right) \\
\spacegrad \cdot \left( \inv{2} \BB \right) &= 0 \\
\spacegrad \cdot \BE &= 4 \pi G \rho \\
\spacegrad \cross \left( \inv{2} \BB \right) &= \inv{c} \PD{t}{\BE} + \frac{4\pi G}{c}\BJ
\end{align}

\section{ STA form. }

As with Maxwell's equations a clifford algebra representation should be possible to put this into a more symmetric form.  Combining the spatial div and grads, following conventions from \cite{doran2003gap} we have

\begin{align}
\spacegrad \BE &= 4 \pi G \rho + \inv{c} \PD{t}{}\left(\inv{2}I \BB\right) \\
\spacegrad \left( \inv{2} I \BB \right) &= \inv{c} \PD{t}{\BE} + \frac{4\pi G}{c}\BJ
\end{align}

Or
\begin{align}
\left( \spacegrad -\inv{c} \PD{t}{}\right) \left( \BE + \inv{2} I \BB \right) &= \frac{4\pi G}{c} \left( c \rho + \BJ \right)
\end{align}

Left multiplication with $\gamma_0$, using a time positive metric signature ($(\gamma_0)^2=1$), 
\begin{align}
\left( \spacegrad -\inv{c} \PD{t}{}\right) \gamma_0 \left( -\BE + \inv{2} I \BB \right) &= \frac{4\pi G}{c} \left( c \rho \gamma_0 + J^i \gamma_i \right)
\end{align}

But $\left( \spacegrad -\inv{c} \PD{t}{}\right) \gamma_0 = \gamma_i \partial_i - \gamma_0 \partial_0 = -\gamma^\mu \partial_\mu = -\grad$.  Introduction of a four vector mass density $J = c\rho \gamma_0 + J^i \gamma_i = J^\mu \gamma_\mu$, and a bivector field $F = \BE -\inv{2} I \BB$ this is

\begin{align}
\grad F = -\frac{4\pi G}{c} J
\end{align}

The guage condition suggests a four potential $V = \Phi \gamma_0 + \BA \gamma_0 = V^\mu \gamma_\mu$, where $V^0 = \Phi$, and $V^i = A^i/2$.  This merges the
space and time parts of the guage condition

\begin{align*}
\grad \cdot V = \gamma^\mu \partial_\mu \cdot \gamma_\nu V^\nu = \partial_\mu V^\mu = \inv{c}\PD{t}{\Phi} + \inv{2}\partial_i A^i.
\end{align*}

It is reasonable to assume that $F = \grad \wedge V$ as in electromagnetism.  Let's see if this is the case

\begin{align*}
\BE - I\BB/2 
&= - \spacegrad \Phi -\inv{c} \PD{t}{}\left( \inv{2} \BB \right) - I\spacegrad \cross \BA/2 \\
&= - \gamma_i \partial_i \gamma_0 V^0 - \inv{2} \partial_0 A^i \gamma_i \gamma_0 + \spacegrad \wedge \BA/2 \\
&= \gamma^i \partial_i \gamma_0 V^0 + \gamma^0 \partial_0 \gamma_i A^i/2 - \gamma_i \partial_i \wedge \gamma_j V^j \\
&= \gamma^i \partial_i \gamma_0 V^0 + \gamma^0 \partial_0 \gamma_i V^i + \gamma^i \partial_i \wedge \gamma_j V^j \\
&= \gamma^\mu \partial_\mu \wedge \gamma_\nu V^\nu \\
&= \grad \wedge V
\end{align*}

Okay, so in terms of potential we have the form as Maxwell's equation

\begin{align}\label{eqn:field}
\grad (\grad \wedge V) &= -\frac{4\pi G}{c} J.
\end{align}

With the guage condition $\grad \cdot V = 0$, this produces the wave equation

\begin{align}
\grad^2 V &= -\frac{4\pi G}{c} J.
\end{align}

In terms of the author's original equation 1.2 it appears that roughly 
$V^\mu = \barh_{0\mu}$, and $J^\mu \propto T_{0\mu}$.

This is logically how he is able to go from that equation to the maxwell
form since both have the same four-vector wave equation form (when $T_{ij} \approx 0$).  To give the potentials specific values in terms of mass and current
distribution appears to be where the retarded integrals are used.

The author expresses $T^{\mu\nu}$ in terms of $\rho$, and mass current $j$, but
the field equations are in terms of $T_{\mu\nu}$.  What metric tensor is
used to translate from upper to lower indexes in this case.  ie: is it $g_{\mu\nu}$, or $\eta_{\mu\nu}$ ?

\section{ Lagrangians. }

\subsection{ Field Lagrangian. }

Since the electrodynamic equation and corresponding field Lagrangian is
\begin{align*}
\grad (\grad \wedge A) &= \frac{J}{\epsilon_0 c} \\
\LL &= -\frac{\epsilon_0 c}{2} (\grad \wedge A)^2 + A \cdot J
\end{align*}

Then, from \ref{eqn:field}, the GEM field Lagrangian in covariant form is

\begin{align*}
\LL &= \frac{c}{8 \pi G} (\grad \wedge V)^2 + V \cdot J \\
\end{align*}

Writing $F^{\mu\nu} = \partial^\mu V^\nu - \partial^\nu V^\mu$, the scalar part of this Lagrangian is:

\begin{align*}
\LL &= -\frac{c}{16 \pi G} F^{\mu\nu} F_{\mu\nu} + V^\sigma J_\sigma \\
\end{align*}

Is this expression hiding in the Einstein field equations?

What is the Lagrangian for newtonian gravity, and how do they compare?

\subsection{ Interaction Lagrangian. }

The metric (equation 1.4) in the article is given to be

\begin{align*}
ds^2 &= 
-c^2\left(1 - 2 \frac{\Phi}{c^2}\right) dt^2
+\frac{4}{c}\left(\BA \cdot d\Bx \right) dt 
+\left(1 + 2 \frac{\Phi}{c^2}\right) \delta_{ij}dx^i dx^j \\
\implies
\Abs{ds^2} = c^2 (d\tau)^2 &= (dx^0)^2 - \sum_i (dx^i)^2
-2 \frac{V_0}{c^2} (dx^0)^2
-\frac{8}{c^2} V_i dx^i dx^0
- 2 \frac{V_0}{c^2} \delta_{ij}dx^i dx^j
\end{align*}

With $v = \gamma_\mu dx^\mu/d\tau$, the Lagrangian for interaction is

\begin{align*}
\LL 
&= \inv{2} m \Abs{\frac{ds}{d\tau}}^2  \\
&= \inv{2} m c^2 \\
&= \inv{2} m v^2 -2 \frac{m V_0}{c^2} \sum_\mu (\xdot^\mu)^2 -\frac{8 m}{c^2} V_i \xdot^0 \xdot^i  \\
\end{align*}

\begin{align}\label{eqn:interactionlagrangian}
\LL &= \inv{2} m v^2 - 2m \left( V_0 \sum_\mu (\xdot^\mu / c)^2 + 4 V_i (\xdot^0/c) (\xdot^i/c) \right)
\end{align}

Now, unlike the Lorentz force Lagrangian
\begin{align*}
\LL &= \inv{2} m v^2 + q A \cdot v/c,
\end{align*}

the Lagragian of \ref{eqn:interactionlagrangian} is quadradic in powers of $\xdot^\mu$.  
There are remarks in the article saying that the non-covariant Lagrangian used to arrive at the Lorentz force equivalent was a first order approximation.
Evaluation of this interaction Lagrangian does not produce anything like the 
$\pdot_\mu = \kappa F_{\mu\nu}\xdot^\nu$ that we see in electrodynamics.

The calculation isn't interesting but the end result for reference is

\begin{align*}
\pdot
%&= \frac{4m}{c^2} \frac{d}{d\tau}\left( V_0 \gamma^\mu v^\mu + 2V_i (v^i \gamma^0 + v^0 \gamma^i) \right) \\
%&- \frac{2m}{c^2} \left( \sum_\mu (v^\mu)^2 \grad V_0 + 4 v^0 v^i \grad V_i \right) \\
&= \frac{4m}{c^2} \left( (v \cdot \grad V_0) \gamma^\mu v^\mu + 2 (v \cdot \grad V_i) (v^i \gamma^0 + v^0 \gamma^i) \right) \\
&+ \frac{4m}{c^2} \left( V_0 \gamma^\mu a^\mu + 2V_i (a^i \gamma^0 + a^0 \gamma^i) \right) \\
&- \frac{2m}{c^2} \left( \sum_\mu (v^\mu)^2 \grad V_0 + 4 v^0 v^i \grad V_i \right)
\end{align*}

This can be simplified somewhat, but no matter what it will be quadratic in the velocity coordinates.

The article also says that the line element is approximate.
Has some of what
is required for a more symmetric covariant interaction proper force been
discarded?

\section{ Conclusion. }

The ideas here are interesting.  At a high level, roughly, as I see it, the equation

\begin{align*}
\grad^2 h_{0\mu} = T_{0\mu}
\end{align*}

has exactly the same form as Maxwell's equations in covariant form, so you can define an antisymmetric field tensor equation in the same way, treating these elements of h, and the corresponding elements of T as a four vector potential and mass current.

That said, I don't have the GR background to know understand the introduction.  For example, how to actually arrive at 1.2
or how to calculated your metric tensor in equation 1.4.  I would have expected 1.4 to have a more symmetric form like the covariant Lorentz force Lagrangian ($v^2 + kA.v$), since you can get a Lorentz force like equation out of it.  Because of the quadratic velocity terms, no matter how one varies that metric with respect to s as a parameter, one cannot get anything at all close to the electrodynamics Lorentz force equation $m\ddot{x}^\mu = q F_\mu\nu \dot{x}_\nu$, so the coorrespondance between electromagnetism and GR breaks down once one considers the interaction.

\bibliographystyle{plainnat}
\bibliography{myrefs}

\end{document}

%
% Copyright � 2012 Peeter Joot.  All Rights Reserved.
% Licenced as described in the file LICENSE under the root directory of this GIT repository.
%

% 
% 
%\documentclass[]{eliblog}

\usepackage{amsmath}
\usepackage{mathpazo}

%
% shorthand for bold symbols, convenient for vectors and matrices
%
\newcommand{\Ba}[0]{\mathbf{a}}
\newcommand{\Bb}[0]{\mathbf{b}}
\newcommand{\Bc}[0]{\mathbf{c}}
\newcommand{\Bd}[0]{\mathbf{d}}
\newcommand{\Be}[0]{\mathbf{e}}
\newcommand{\Bf}[0]{\mathbf{f}}
\newcommand{\Bg}[0]{\mathbf{g}}
\newcommand{\Bh}[0]{\mathbf{h}}
\newcommand{\Bi}[0]{\mathbf{i}}
\newcommand{\Bj}[0]{\mathbf{j}}
\newcommand{\Bk}[0]{\mathbf{k}}
\newcommand{\Bl}[0]{\mathbf{l}}
\newcommand{\Bm}[0]{\mathbf{m}}
\newcommand{\Bn}[0]{\mathbf{n}}
\newcommand{\Bo}[0]{\mathbf{o}}
\newcommand{\Bp}[0]{\mathbf{p}}
\newcommand{\Bq}[0]{\mathbf{q}}
\newcommand{\Br}[0]{\mathbf{r}}
\newcommand{\Bs}[0]{\mathbf{s}}
\newcommand{\Bt}[0]{\mathbf{t}}
\newcommand{\Bu}[0]{\mathbf{u}}
\newcommand{\Bv}[0]{\mathbf{v}}
\newcommand{\Bw}[0]{\mathbf{w}}
\newcommand{\Bx}[0]{\mathbf{x}}
\newcommand{\By}[0]{\mathbf{y}}
\newcommand{\Bz}[0]{\mathbf{z}}
\newcommand{\BA}[0]{\mathbf{A}}
\newcommand{\BB}[0]{\mathbf{B}}
\newcommand{\BC}[0]{\mathbf{C}}
\newcommand{\BD}[0]{\mathbf{D}}
\newcommand{\BE}[0]{\mathbf{E}}
\newcommand{\BF}[0]{\mathbf{F}}
\newcommand{\BG}[0]{\mathbf{G}}
\newcommand{\BH}[0]{\mathbf{H}}
\newcommand{\BI}[0]{\mathbf{I}}
\newcommand{\BJ}[0]{\mathbf{J}}
\newcommand{\BK}[0]{\mathbf{K}}
\newcommand{\BL}[0]{\mathbf{L}}
\newcommand{\BM}[0]{\mathbf{M}}
\newcommand{\BN}[0]{\mathbf{N}}
\newcommand{\BO}[0]{\mathbf{O}}
\newcommand{\BP}[0]{\mathbf{P}}
\newcommand{\BQ}[0]{\mathbf{Q}}
\newcommand{\BR}[0]{\mathbf{R}}
\newcommand{\BS}[0]{\mathbf{S}}
\newcommand{\BT}[0]{\mathbf{T}}
\newcommand{\BU}[0]{\mathbf{U}}
\newcommand{\BV}[0]{\mathbf{V}}
\newcommand{\BW}[0]{\mathbf{W}}
\newcommand{\BX}[0]{\mathbf{X}}
\newcommand{\BY}[0]{\mathbf{Y}}
\newcommand{\BZ}[0]{\mathbf{Z}}

\newcommand{\Bzero}[0]{\mathbf{0}}
\newcommand{\Btheta}[0]{\boldsymbol{\theta}}
\newcommand{\Btau}[0]{\boldsymbol{\tau}}
\newcommand{\Bomega}[0]{\boldsymbol{\omega}}

%
% shorthand for unit vectors
%
\newcommand{\acap}[0]{\hat{\Ba}}
\newcommand{\bcap}[0]{\hat{\Bb}}
\newcommand{\ccap}[0]{\hat{\Bc}}
\newcommand{\dcap}[0]{\hat{\Bd}}
\newcommand{\ecap}[0]{\hat{\Be}}
\newcommand{\fcap}[0]{\hat{\Bf}}
\newcommand{\gcap}[0]{\hat{\Bg}}
\newcommand{\hcap}[0]{\hat{\Bh}}
\newcommand{\icap}[0]{\hat{\Bi}}
\newcommand{\jcap}[0]{\hat{\Bj}}
\newcommand{\kcap}[0]{\hat{\Bk}}
\newcommand{\lcap}[0]{\hat{\Bl}}
\newcommand{\mcap}[0]{\hat{\Bm}}
\newcommand{\ncap}[0]{\hat{\Bn}}
\newcommand{\ocap}[0]{\hat{\Bo}}
\newcommand{\pcap}[0]{\hat{\Bp}}
\newcommand{\qcap}[0]{\hat{\Bq}}
\newcommand{\rcap}[0]{\hat{\Br}}
\newcommand{\scap}[0]{\hat{\Bs}}
\newcommand{\tcap}[0]{\hat{\Bt}}
\newcommand{\ucap}[0]{\hat{\Bu}}
\newcommand{\vcap}[0]{\hat{\Bv}}
\newcommand{\wcap}[0]{\hat{\Bw}}
\newcommand{\xcap}[0]{\hat{\Bx}}
\newcommand{\ycap}[0]{\hat{\By}}
\newcommand{\zcap}[0]{\hat{\Bz}}
\newcommand{\thetacap}[0]{\hat{\Btheta}}

%
% to write R^n and C^n in a distinguishable fashion.  Perhaps change this
% to the double lined characters upon figuring out how to do so.
%
\newcommand{\C}[1]{$\mathbb{C}^{#1}$}
\newcommand{\R}[1]{$\mathbb{R}^{#1}$}

%
% various generally useful helpers
%

% derivative of #1 wrt. #2:
\newcommand{\D}[2] {\frac {d#2} {d#1}}

\newcommand{\inv}[1]{\frac{1}{#1}}
\newcommand{\cross}[0]{\times}

\newcommand{\abs}[1]{\lvert{#1}\rvert}
\newcommand{\norm}[1]{\lVert{#1}\rVert}
\newcommand{\innerprod}[2]{\langle{#1}, {#2}\rangle}
\newcommand{\dotprod}[2]{{#1} \cdot {#2}}
\newcommand{\bdotprod}[2]{\left({#1} \cdot {#2}\right)}
\newcommand{\crossprod}[2]{{#1} \cross {#2}}
\newcommand{\tripleprod}[3]{\dotprod{\left(\crossprod{#1}{#2}\right)}{#3}}

\DeclareMathOperator{\Proj}{Proj}
\DeclareMathOperator{\Span}{span}
\DeclareMathOperator{\Sgn}{sgn}
\DeclareMathOperator{\Area}{Area}
\DeclareMathOperator{\Volume}{Volume}

%
% A few miscellaneous things specific to this document
%
\newcommand{\crossop}[1]{\crossprod{#1}{}}

% R2 vector.
\newcommand{\VectorTwo}[2]{
\begin{bmatrix}
 {#1} \\
 {#2}
\end{bmatrix}
}

\newcommand{\VectorN}[1]{
\begin{bmatrix}
{#1}_1 \\
{#1}_2 \\
\vdots \\
{#1}_N \\
\end{bmatrix}
}

\newcommand{\DETuvij}[4]{
\begin{vmatrix}
 {#1}_{#3} & {#1}_{#4} \\
 {#2}_{#3} & {#2}_{#4}
\end{vmatrix}
}

\newcommand{\DETuvwijk}[6]{
\begin{vmatrix}
 {#1}_{#4} & {#1}_{#5} & {#1}_{#6} \\
 {#2}_{#4} & {#2}_{#5} & {#2}_{#6} \\
 {#3}_{#4} & {#3}_{#5} & {#3}_{#6}
\end{vmatrix}
}

\newcommand{\DETuvwxijkl}[8]{
\begin{vmatrix}
 {#1}_{#5} & {#1}_{#6} & {#1}_{#7} & {#1}_{#8} \\
 {#2}_{#5} & {#2}_{#6} & {#2}_{#7} & {#2}_{#8} \\
 {#3}_{#5} & {#3}_{#6} & {#3}_{#7} & {#3}_{#8} \\
 {#4}_{#5} & {#4}_{#6} & {#4}_{#7} & {#4}_{#8} \\
\end{vmatrix}
}

%\newcommand{\DETuvwxyijklm}[10]{
%\begin{vmatrix}
% {#1}_{#6} & {#1}_{#7} & {#1}_{#8} & {#1}_{#9} & {#1}_{#10} \\
% {#2}_{#6} & {#2}_{#7} & {#2}_{#8} & {#2}_{#9} & {#2}_{#10} \\
% {#3}_{#6} & {#3}_{#7} & {#3}_{#8} & {#3}_{#9} & {#3}_{#10} \\
% {#4}_{#6} & {#4}_{#7} & {#4}_{#8} & {#4}_{#9} & {#4}_{#10} \\
% {#5}_{#6} & {#5}_{#7} & {#5}_{#8} & {#5}_{#9} & {#5}_{#10}
%\end{vmatrix}
%}

% R3 vector.
\newcommand{\VectorThree}[3]{
\begin{bmatrix}
 {#1} \\
 {#2} \\
 {#3}
\end{bmatrix}
}



\author{Peeter Joot}
\email{peeter.joot@gmail.com}


\chapter{Relativistic Doppler formula}
\index{Doppler equation!relativistic}
\label{chap:frequencyTx}
%\blogpage{http://sites.google.com/site/peeterjoot/math2009/frequencyTx.pdf}
%%\date{June 27, 2009}
%%\revisionInfo{\(RCSfile: frequencyTx.tex,v \) Last \(Revision: 1.6 \) \(Date: 2009/10/22 02:07:20 \)}

%\date{June 27, 2009 \(RCSfile: frequencyTx.tex,v \) Last \(Revision: 1.6 \) \(Date: 2009/10/22 02:07:20 \)}

\beginArtWithToc

\section{Transform of angular velocity four vector}

It was possible to derive the Lorentz boost matrix by requiring that the wave equation operator

\begin{equation}\label{eqn:frequencyTx:20}
\begin{aligned}
\grad^2 = \inv{c^2}\frac{\partial^2}{\partial t^2} - \spacegrad^2
\end{aligned}
\end{equation}

retain its form under linear transformation (\chapcite{PJLorentzWave}).  Applying spatial Fourier transforms (\chapcite{PJwaveFourier}), one finds that solutions to the wave equation 

\begin{equation}\label{eqn:frequencyTx:40}
\begin{aligned}
\grad^2 \psi(t,\Bx) = 0
\end{aligned}
\end{equation}

Have the form

\begin{equation}\label{eqn:frequencyTx:60}
\begin{aligned}
\psi(t, \Bx) = \int A(\Bk) e^{i(\Bk \cdot \Bx - \omega t)} d^3 k
\end{aligned}
\end{equation}

Provided that \(\omega = \pm c \Abs{\Bk}\).  Wave equation solutions can therefore be thought of as continuously weighted superpositions of constrained fundamental solutions

\begin{equation}\label{eqn:frequencyTx:80}
\begin{aligned}
\psi &= e^{i(\Bk \cdot \Bx - \omega t)} \\
c^2 \Bk^2 &= \omega^2
\end{aligned}
\end{equation}

The constraint on frequency and wave number has the look of a Lorentz square

\begin{equation}\label{eqn:frequencyTx:100}
\begin{aligned}
\omega^2 - c^2 \Bk^2 = 0
\end{aligned}
\end{equation}

Which suggests that in additional to the spacetime vector

\begin{equation}\label{eqn:frequencyTx:120}
\begin{aligned}
X = (ct, \Bx) = x^\mu \gamma_\mu
\end{aligned}
\end{equation}

evident in the wave equation fundamental solution, we also have a frequency-wavenumber four vector

\begin{equation}\label{eqn:frequencyTx:140}
\begin{aligned}
K = (\omega/c, \Bk) = k^\mu \gamma_\mu
\end{aligned}
\end{equation}

The pair of four vectors above allow the fundamental solutions to be put explicitly into covariant form

\begin{equation}\label{eqn:frequencyTx:160}
\begin{aligned}
K \cdot X = \omega t - \Bk \cdot \Bx = k_\mu x^\mu
\end{aligned}
\end{equation}

\begin{equation}\label{eqn:frequencyTx:180}
\begin{aligned}
\psi = e^{-i K \cdot X}
\end{aligned}
\end{equation}

Let us also examine the transformation properties of this fundamental solution, and see as a side effect that \(K\)
has transforms appropriately as a four vector.

\begin{equation}\label{eqn:frequencyTx:200}
\begin{aligned}
0 &= \grad^2 \psi(t,\Bx) \\
&= {\grad'}^2 \psi(t',\Bx') \\
&= {\grad'}^2 e^{i(\Bx' \cdot \Bk' - \omega' t')} \\
&= -\left(\frac{{\omega'}^2}{c^2} - {\Bk'}^2 \right) e^{i(\Bx' \cdot \Bk' - \omega' t')} \\
\end{aligned}
\end{equation}

We therefore have the same form of frequency wave number constraint in the transformed frame (if we require that
the wave function for light is unchanged under transformation)

\begin{equation}\label{eqn:frequencyTx:220}
\begin{aligned}
{\omega'}^2 = c^2 {\Bk'}^2 
\end{aligned}
\end{equation}

Writing this as

\begin{equation}\label{eqn:frequencyTx:240}
\begin{aligned}
0 = {\omega}^2 - c^2 {\Bk}^2 = {\omega'}^2 - c^2 {\Bk'}^2 
\end{aligned}
\end{equation}

singles out the Lorentz invariant nature of the \((\omega, \Bk)\) pairing, and we conclude that this pairing 
does indeed transform as a four vector.

\section{Application of one dimensional boost}

Having attempted to justify the four vector nature of the wave number vector \(K\), now move on to application of a boost along the x-axis to this vector.

\begin{equation}\label{eqn:frequencyTx:260}
\begin{aligned}
\begin{bmatrix}
\omega' \\
c k' \\
\end{bmatrix}
&=
\gamma
\begin{bmatrix}
1 & -\beta \\
-\beta& 1 \\
\end{bmatrix}
\begin{bmatrix}
\omega \\
c k \\
\end{bmatrix} 
\\
&=
\begin{bmatrix}
\omega - v k \\
c k - \beta \omega
\end{bmatrix} 
\end{aligned}
\end{equation}

We can take ratios of the frequencies if we make use of the dependency between \(\omega\) and \(k\).  Namely, \(\omega = \pm c k\).  We then have

\begin{equation}\label{eqn:frequencyTx:280}
\begin{aligned}
\frac{\omega'}{\omega}
%&= \omega - (v/c) (\pm \omega)
&= \gamma(1 \mp \beta) \\
&= \frac{1 \mp \beta}{\sqrt{1 - \beta^2}} \\
&= \frac{1 \mp \beta}{\sqrt{1 - \beta}\sqrt{1 + \beta}} \\
\end{aligned}
\end{equation}

For the positive angular frequency this is

\begin{equation}\label{eqn:frequencyTx:300}
\begin{aligned}
\frac{\omega'}{\omega}
&= \frac{\sqrt{1 - \beta}}{\sqrt{1 + \beta}} 
\\
\end{aligned}
\end{equation}

and for the negative frequency the reciprocal.

Deriving this with a Lorentz boost is much simpler than the time dilation argument in wikipedia doppler article \citep{wiki:relDoppler}.  EDIT: Later found exactly the above boost argument in the wiki k-vector article \citep{wiki:kvector}.

What is missing here is putting this in a physical context properly with source and reciever frequencies spelled out.  That would make this more than just math.

%\EndArticle

\documentclass{article}

\usepackage{amsmath}
\usepackage{mathpazo}

%
% shorthand for bold symbols, convenient for vectors and matrices
%
\newcommand{\Ba}[0]{\mathbf{a}}
\newcommand{\Bb}[0]{\mathbf{b}}
\newcommand{\Bc}[0]{\mathbf{c}}
\newcommand{\Bd}[0]{\mathbf{d}}
\newcommand{\Be}[0]{\mathbf{e}}
\newcommand{\Bf}[0]{\mathbf{f}}
\newcommand{\Bg}[0]{\mathbf{g}}
\newcommand{\Bh}[0]{\mathbf{h}}
\newcommand{\Bi}[0]{\mathbf{i}}
\newcommand{\Bj}[0]{\mathbf{j}}
\newcommand{\Bk}[0]{\mathbf{k}}
\newcommand{\Bl}[0]{\mathbf{l}}
\newcommand{\Bm}[0]{\mathbf{m}}
\newcommand{\Bn}[0]{\mathbf{n}}
\newcommand{\Bo}[0]{\mathbf{o}}
\newcommand{\Bp}[0]{\mathbf{p}}
\newcommand{\Bq}[0]{\mathbf{q}}
\newcommand{\Br}[0]{\mathbf{r}}
\newcommand{\Bs}[0]{\mathbf{s}}
\newcommand{\Bt}[0]{\mathbf{t}}
\newcommand{\Bu}[0]{\mathbf{u}}
\newcommand{\Bv}[0]{\mathbf{v}}
\newcommand{\Bw}[0]{\mathbf{w}}
\newcommand{\Bx}[0]{\mathbf{x}}
\newcommand{\By}[0]{\mathbf{y}}
\newcommand{\Bz}[0]{\mathbf{z}}
\newcommand{\BA}[0]{\mathbf{A}}
\newcommand{\BB}[0]{\mathbf{B}}
\newcommand{\BC}[0]{\mathbf{C}}
\newcommand{\BD}[0]{\mathbf{D}}
\newcommand{\BE}[0]{\mathbf{E}}
\newcommand{\BF}[0]{\mathbf{F}}
\newcommand{\BG}[0]{\mathbf{G}}
\newcommand{\BH}[0]{\mathbf{H}}
\newcommand{\BI}[0]{\mathbf{I}}
\newcommand{\BJ}[0]{\mathbf{J}}
\newcommand{\BK}[0]{\mathbf{K}}
\newcommand{\BL}[0]{\mathbf{L}}
\newcommand{\BM}[0]{\mathbf{M}}
\newcommand{\BN}[0]{\mathbf{N}}
\newcommand{\BO}[0]{\mathbf{O}}
\newcommand{\BP}[0]{\mathbf{P}}
\newcommand{\BQ}[0]{\mathbf{Q}}
\newcommand{\BR}[0]{\mathbf{R}}
\newcommand{\BS}[0]{\mathbf{S}}
\newcommand{\BT}[0]{\mathbf{T}}
\newcommand{\BU}[0]{\mathbf{U}}
\newcommand{\BV}[0]{\mathbf{V}}
\newcommand{\BW}[0]{\mathbf{W}}
\newcommand{\BX}[0]{\mathbf{X}}
\newcommand{\BY}[0]{\mathbf{Y}}
\newcommand{\BZ}[0]{\mathbf{Z}}

\newcommand{\Bzero}[0]{\mathbf{0}}
\newcommand{\Btheta}[0]{\boldsymbol{\theta}}
\newcommand{\Btau}[0]{\boldsymbol{\tau}}
\newcommand{\Bomega}[0]{\boldsymbol{\omega}}

%
% shorthand for unit vectors
%
\newcommand{\acap}[0]{\hat{\Ba}}
\newcommand{\bcap}[0]{\hat{\Bb}}
\newcommand{\ccap}[0]{\hat{\Bc}}
\newcommand{\dcap}[0]{\hat{\Bd}}
\newcommand{\ecap}[0]{\hat{\Be}}
\newcommand{\fcap}[0]{\hat{\Bf}}
\newcommand{\gcap}[0]{\hat{\Bg}}
\newcommand{\hcap}[0]{\hat{\Bh}}
\newcommand{\icap}[0]{\hat{\Bi}}
\newcommand{\jcap}[0]{\hat{\Bj}}
\newcommand{\kcap}[0]{\hat{\Bk}}
\newcommand{\lcap}[0]{\hat{\Bl}}
\newcommand{\mcap}[0]{\hat{\Bm}}
\newcommand{\ncap}[0]{\hat{\Bn}}
\newcommand{\ocap}[0]{\hat{\Bo}}
\newcommand{\pcap}[0]{\hat{\Bp}}
\newcommand{\qcap}[0]{\hat{\Bq}}
\newcommand{\rcap}[0]{\hat{\Br}}
\newcommand{\scap}[0]{\hat{\Bs}}
\newcommand{\tcap}[0]{\hat{\Bt}}
\newcommand{\ucap}[0]{\hat{\Bu}}
\newcommand{\vcap}[0]{\hat{\Bv}}
\newcommand{\wcap}[0]{\hat{\Bw}}
\newcommand{\xcap}[0]{\hat{\Bx}}
\newcommand{\ycap}[0]{\hat{\By}}
\newcommand{\zcap}[0]{\hat{\Bz}}
\newcommand{\thetacap}[0]{\hat{\Btheta}}

%
% to write R^n and C^n in a distinguishable fashion.  Perhaps change this
% to the double lined characters upon figuring out how to do so.
%
\newcommand{\C}[1]{$\mathbb{C}^{#1}$}
\newcommand{\R}[1]{$\mathbb{R}^{#1}$}

%
% various generally useful helpers
%

% derivative of #1 wrt. #2:
\newcommand{\D}[2] {\frac {d#2} {d#1}}

\newcommand{\inv}[1]{\frac{1}{#1}}
\newcommand{\cross}[0]{\times}

\newcommand{\abs}[1]{\lvert{#1}\rvert}
\newcommand{\norm}[1]{\lVert{#1}\rVert}
\newcommand{\innerprod}[2]{\langle{#1}, {#2}\rangle}
\newcommand{\dotprod}[2]{{#1} \cdot {#2}}
\newcommand{\bdotprod}[2]{\left({#1} \cdot {#2}\right)}
\newcommand{\crossprod}[2]{{#1} \cross {#2}}
\newcommand{\tripleprod}[3]{\dotprod{\left(\crossprod{#1}{#2}\right)}{#3}}

\DeclareMathOperator{\Proj}{Proj}
\DeclareMathOperator{\Span}{span}
\DeclareMathOperator{\Sgn}{sgn}
\DeclareMathOperator{\Area}{Area}
\DeclareMathOperator{\Volume}{Volume}

%
% A few miscellaneous things specific to this document
%
\newcommand{\crossop}[1]{\crossprod{#1}{}}

% R2 vector.
\newcommand{\VectorTwo}[2]{
\begin{bmatrix}
 {#1} \\
 {#2}
\end{bmatrix}
}

\newcommand{\VectorN}[1]{
\begin{bmatrix}
{#1}_1 \\
{#1}_2 \\
\vdots \\
{#1}_N \\
\end{bmatrix}
}

\newcommand{\DETuvij}[4]{
\begin{vmatrix}
 {#1}_{#3} & {#1}_{#4} \\
 {#2}_{#3} & {#2}_{#4}
\end{vmatrix}
}

\newcommand{\DETuvwijk}[6]{
\begin{vmatrix}
 {#1}_{#4} & {#1}_{#5} & {#1}_{#6} \\
 {#2}_{#4} & {#2}_{#5} & {#2}_{#6} \\
 {#3}_{#4} & {#3}_{#5} & {#3}_{#6}
\end{vmatrix}
}

\newcommand{\DETuvwxijkl}[8]{
\begin{vmatrix}
 {#1}_{#5} & {#1}_{#6} & {#1}_{#7} & {#1}_{#8} \\
 {#2}_{#5} & {#2}_{#6} & {#2}_{#7} & {#2}_{#8} \\
 {#3}_{#5} & {#3}_{#6} & {#3}_{#7} & {#3}_{#8} \\
 {#4}_{#5} & {#4}_{#6} & {#4}_{#7} & {#4}_{#8} \\
\end{vmatrix}
}

%\newcommand{\DETuvwxyijklm}[10]{
%\begin{vmatrix}
% {#1}_{#6} & {#1}_{#7} & {#1}_{#8} & {#1}_{#9} & {#1}_{#10} \\
% {#2}_{#6} & {#2}_{#7} & {#2}_{#8} & {#2}_{#9} & {#2}_{#10} \\
% {#3}_{#6} & {#3}_{#7} & {#3}_{#8} & {#3}_{#9} & {#3}_{#10} \\
% {#4}_{#6} & {#4}_{#7} & {#4}_{#8} & {#4}_{#9} & {#4}_{#10} \\
% {#5}_{#6} & {#5}_{#7} & {#5}_{#8} & {#5}_{#9} & {#5}_{#10}
%\end{vmatrix}
%}

% R3 vector.
\newcommand{\VectorThree}[3]{
\begin{bmatrix}
 {#1} \\
 {#2} \\
 {#3}
\end{bmatrix}
}


%<misc>
%
\newcommand{\Abs}[1]{{\left\lvert{#1}\right\rvert}}
\newcommand{\spacegrad}[0]{\boldsymbol{\nabla}}
\newcommand{\grad}[0]{\nabla}
\newcommand{\LL}[0]{\mathcal{L}}

% == \partial_{#1} {#2}
\newcommand{\PD}[2]{\frac{\partial {#2}}{\partial {#1}}}
% inline variant
\newcommand{\PDi}[2]{{\partial {#2}}/{\partial {#1}}}

\newcommand{\PDD}[3]{\frac{\partial^2 {#3}}{\partial {#1}\partial {#2}}}
%\newcommand{\PDd}[2]{\frac{\partial^2 {#2}}{{\partial{#1}}^2}}
\newcommand{\PDsq}[2]{\frac{\partial^2 {#2}}{(\partial {#1})^2}}

\newcommand{\Partial}[2]{\frac{\partial {#1}}{\partial {#2}}}
\DeclareMathOperator{\RejName}{Rej}
\newcommand{\Rej}[2]{\RejName_{#1}\left( {#2} \right)}
\newcommand{\Rm}[1]{\mathbb{R}^{#1}}
\newcommand{\Cm}[1]{\mathbb{C}^{#1}}
\newcommand{\conj}[0]{{*}}

%</misc>

% <grade selection>
%
\newcommand{\gpgrade}[2] {{\left\langle{{#1}}\right\rangle}_{#2}}

\newcommand{\gpgradezero}[1] {\gpgrade{#1}{}}
%\newcommand{\gpscalargrade}[1] {{\left\langle{{#1}}\right\rangle}}
%\newcommand{\gpgradezero}[1] {\gpgrade{#1}{0}}

%\newcommand{\gpgradeone}[1] {{\left\langle{{#1}}\right\rangle}_{1}}
\newcommand{\gpgradeone}[1] {\gpgrade{#1}{1}}

\newcommand{\gpgradetwo}[1] {\gpgrade{#1}{2}}
\newcommand{\gpgradethree}[1] {\gpgrade{#1}{3}}
\newcommand{\gpgradefour}[1] {\gpgrade{#1}{4}}
%
% </grade selection>



\newcommand{\adot}[0]{{\dot{a}}}
\newcommand{\bdot}[0]{{\dot{b}}}
% taken for centered dot:
%\newcommand{\cdot}[0]{{\dot{c}}}
%\newcommand{\ddot}[0]{{\dot{d}}}
\newcommand{\edot}[0]{{\dot{e}}}
\newcommand{\fdot}[0]{{\dot{f}}}
\newcommand{\gdot}[0]{{\dot{g}}}
\newcommand{\hdot}[0]{{\dot{h}}}
\newcommand{\idot}[0]{{\dot{i}}}
\newcommand{\jdot}[0]{{\dot{j}}}
\newcommand{\kdot}[0]{{\dot{k}}}
\newcommand{\ldot}[0]{{\dot{l}}}
\newcommand{\mdot}[0]{{\dot{m}}}
\newcommand{\ndot}[0]{{\dot{n}}}
%\newcommand{\odot}[0]{{\dot{o}}}
\newcommand{\pdot}[0]{{\dot{p}}}
\newcommand{\qdot}[0]{{\dot{q}}}
\newcommand{\rdot}[0]{{\dot{r}}}
\newcommand{\sdot}[0]{{\dot{s}}}
\newcommand{\tdot}[0]{{\dot{t}}}
\newcommand{\udot}[0]{{\dot{u}}}
\newcommand{\vdot}[0]{{\dot{v}}}
\newcommand{\wdot}[0]{{\dot{w}}}
\newcommand{\xdot}[0]{{\dot{x}}}
\newcommand{\ydot}[0]{{\dot{y}}}
\newcommand{\zdot}[0]{{\dot{z}}}
\newcommand{\addot}[0]{{\ddot{a}}}
\newcommand{\bddot}[0]{{\ddot{b}}}
\newcommand{\cddot}[0]{{\ddot{c}}}
%\newcommand{\dddot}[0]{{\ddot{d}}}
\newcommand{\eddot}[0]{{\ddot{e}}}
\newcommand{\fddot}[0]{{\ddot{f}}}
\newcommand{\gddot}[0]{{\ddot{g}}}
\newcommand{\hddot}[0]{{\ddot{h}}}
\newcommand{\iddot}[0]{{\ddot{i}}}
\newcommand{\jddot}[0]{{\ddot{j}}}
\newcommand{\kddot}[0]{{\ddot{k}}}
\newcommand{\lddot}[0]{{\ddot{l}}}
\newcommand{\mddot}[0]{{\ddot{m}}}
\newcommand{\nddot}[0]{{\ddot{n}}}
\newcommand{\oddot}[0]{{\ddot{o}}}
\newcommand{\pddot}[0]{{\ddot{p}}}
\newcommand{\qddot}[0]{{\ddot{q}}}
\newcommand{\rddot}[0]{{\ddot{r}}}
\newcommand{\sddot}[0]{{\ddot{s}}}
\newcommand{\tddot}[0]{{\ddot{t}}}
\newcommand{\uddot}[0]{{\ddot{u}}}
\newcommand{\vddot}[0]{{\ddot{v}}}
\newcommand{\wddot}[0]{{\ddot{w}}}
\newcommand{\xddot}[0]{{\ddot{x}}}
\newcommand{\yddot}[0]{{\ddot{y}}}
\newcommand{\zddot}[0]{{\ddot{z}}}

%<bold and dot greek symbols>
%

\newcommand{\Deltadot}[0]{{\dot{\Delta}}}
\newcommand{\Gammadot}[0]{{\dot{\Gamma}}}
\newcommand{\Lambdadot}[0]{{\dot{\Lambda}}}
\newcommand{\Omegadot}[0]{{\dot{\Omega}}}
\newcommand{\Phidot}[0]{{\dot{\Phi}}}
\newcommand{\Pidot}[0]{{\dot{\Pi}}}
\newcommand{\Psidot}[0]{{\dot{\Psi}}}
\newcommand{\Sigmadot}[0]{{\dot{\Sigma}}}
\newcommand{\Thetadot}[0]{{\dot{\Theta}}}
\newcommand{\Upsilondot}[0]{{\dot{\Upsilon}}}
\newcommand{\Xidot}[0]{{\dot{\Xi}}}
\newcommand{\alphadot}[0]{{\dot{\alpha}}}
\newcommand{\betadot}[0]{{\dot{\beta}}}
\newcommand{\chidot}[0]{{\dot{\chi}}}
\newcommand{\deltadot}[0]{{\dot{\delta}}}
\newcommand{\epsilondot}[0]{{\dot{\epsilon}}}
\newcommand{\etadot}[0]{{\dot{\eta}}}
\newcommand{\gammadot}[0]{{\dot{\gamma}}}
\newcommand{\kappadot}[0]{{\dot{\kappa}}}
\newcommand{\lambdadot}[0]{{\dot{\lambda}}}
\newcommand{\mudot}[0]{{\dot{\mu}}}
\newcommand{\nudot}[0]{{\dot{\nu}}}
\newcommand{\omegadot}[0]{{\dot{\omega}}}
\newcommand{\phidot}[0]{{\dot{\phi}}}
\newcommand{\pidot}[0]{{\dot{\pi}}}
\newcommand{\psidot}[0]{{\dot{\psi}}}
\newcommand{\rhodot}[0]{{\dot{\rho}}}
\newcommand{\sigmadot}[0]{{\dot{\sigma}}}
\newcommand{\taudot}[0]{{\dot{\tau}}}
\newcommand{\thetadot}[0]{{\dot{\theta}}}
\newcommand{\upsilondot}[0]{{\dot{\upsilon}}}
\newcommand{\varepsilondot}[0]{{\dot{\varepsilon}}}
\newcommand{\varphidot}[0]{{\dot{\varphi}}}
\newcommand{\varpidot}[0]{{\dot{\varpi}}}
\newcommand{\varrhodot}[0]{{\dot{\varrho}}}
\newcommand{\varsigmadot}[0]{{\dot{\varsigma}}}
\newcommand{\varthetadot}[0]{{\dot{\vartheta}}}
\newcommand{\xidot}[0]{{\dot{\xi}}}
\newcommand{\zetadot}[0]{{\dot{\zeta}}}

\newcommand{\Deltaddot}[0]{{\ddot{\Delta}}}
\newcommand{\Gammaddot}[0]{{\ddot{\Gamma}}}
\newcommand{\Lambdaddot}[0]{{\ddot{\Lambda}}}
\newcommand{\Omegaddot}[0]{{\ddot{\Omega}}}
\newcommand{\Phiddot}[0]{{\ddot{\Phi}}}
\newcommand{\Piddot}[0]{{\ddot{\Pi}}}
\newcommand{\Psiddot}[0]{{\ddot{\Psi}}}
\newcommand{\Sigmaddot}[0]{{\ddot{\Sigma}}}
\newcommand{\Thetaddot}[0]{{\ddot{\Theta}}}
\newcommand{\Upsilonddot}[0]{{\ddot{\Upsilon}}}
\newcommand{\Xiddot}[0]{{\ddot{\Xi}}}
\newcommand{\alphaddot}[0]{{\ddot{\alpha}}}
\newcommand{\betaddot}[0]{{\ddot{\beta}}}
\newcommand{\chiddot}[0]{{\ddot{\chi}}}
\newcommand{\deltaddot}[0]{{\ddot{\delta}}}
\newcommand{\epsilonddot}[0]{{\ddot{\epsilon}}}
\newcommand{\etaddot}[0]{{\ddot{\eta}}}
\newcommand{\gammaddot}[0]{{\ddot{\gamma}}}
\newcommand{\kappaddot}[0]{{\ddot{\kappa}}}
\newcommand{\lambdaddot}[0]{{\ddot{\lambda}}}
\newcommand{\muddot}[0]{{\ddot{\mu}}}
\newcommand{\nuddot}[0]{{\ddot{\nu}}}
\newcommand{\omegaddot}[0]{{\ddot{\omega}}}
\newcommand{\phiddot}[0]{{\ddot{\phi}}}
\newcommand{\piddot}[0]{{\ddot{\pi}}}
\newcommand{\psiddot}[0]{{\ddot{\psi}}}
\newcommand{\rhoddot}[0]{{\ddot{\rho}}}
\newcommand{\sigmaddot}[0]{{\ddot{\sigma}}}
\newcommand{\tauddot}[0]{{\ddot{\tau}}}
\newcommand{\thetaddot}[0]{{\ddot{\theta}}}
\newcommand{\upsilonddot}[0]{{\ddot{\upsilon}}}
\newcommand{\varepsilonddot}[0]{{\ddot{\varepsilon}}}
\newcommand{\varphiddot}[0]{{\ddot{\varphi}}}
\newcommand{\varpiddot}[0]{{\ddot{\varpi}}}
\newcommand{\varrhoddot}[0]{{\ddot{\varrho}}}
\newcommand{\varsigmaddot}[0]{{\ddot{\varsigma}}}
\newcommand{\varthetaddot}[0]{{\ddot{\vartheta}}}
\newcommand{\xiddot}[0]{{\ddot{\xi}}}
\newcommand{\zetaddot}[0]{{\ddot{\zeta}}}

\newcommand{\BDelta}[0]{\boldsymbol{\Delta}}
\newcommand{\BGamma}[0]{\boldsymbol{\Gamma}}
\newcommand{\BLambda}[0]{\boldsymbol{\Lambda}}
\newcommand{\BOmega}[0]{\boldsymbol{\Omega}}
\newcommand{\BPhi}[0]{\boldsymbol{\Phi}}
\newcommand{\BPi}[0]{\boldsymbol{\Pi}}
\newcommand{\BPsi}[0]{\boldsymbol{\Psi}}
\newcommand{\BSigma}[0]{\boldsymbol{\Sigma}}
\newcommand{\BTheta}[0]{\boldsymbol{\Theta}}
\newcommand{\BUpsilon}[0]{\boldsymbol{\Upsilon}}
\newcommand{\BXi}[0]{\boldsymbol{\Xi}}
\newcommand{\Balpha}[0]{\boldsymbol{\alpha}}
\newcommand{\Bbeta}[0]{\boldsymbol{\beta}}
\newcommand{\Bchi}[0]{\boldsymbol{\chi}}
\newcommand{\Bdelta}[0]{\boldsymbol{\delta}}
\newcommand{\Bepsilon}[0]{\boldsymbol{\epsilon}}
\newcommand{\Beta}[0]{\boldsymbol{\eta}}
\newcommand{\Bgamma}[0]{\boldsymbol{\gamma}}
\newcommand{\Bkappa}[0]{\boldsymbol{\kappa}}
\newcommand{\Blambda}[0]{\boldsymbol{\lambda}}
\newcommand{\Bmu}[0]{\boldsymbol{\mu}}
\newcommand{\Bnu}[0]{\boldsymbol{\nu}}
%\newcommand{\Bomega}[0]{\boldsymbol{\omega}}
\newcommand{\Bphi}[0]{\boldsymbol{\phi}}
\newcommand{\Bpi}[0]{\boldsymbol{\pi}}
\newcommand{\Bpsi}[0]{\boldsymbol{\psi}}
\newcommand{\Brho}[0]{\boldsymbol{\rho}}
\newcommand{\Bsigma}[0]{\boldsymbol{\sigma}}
%\newcommand{\Btau}[0]{\boldsymbol{\tau}}
%\newcommand{\Btheta}[0]{\boldsymbol{\theta}}
\newcommand{\Bupsilon}[0]{\boldsymbol{\upsilon}}
\newcommand{\Bvarepsilon}[0]{\boldsymbol{\varepsilon}}
\newcommand{\Bvarphi}[0]{\boldsymbol{\varphi}}
\newcommand{\Bvarpi}[0]{\boldsymbol{\varpi}}
\newcommand{\Bvarrho}[0]{\boldsymbol{\varrho}}
\newcommand{\Bvarsigma}[0]{\boldsymbol{\varsigma}}
\newcommand{\Bvartheta}[0]{\boldsymbol{\vartheta}}
\newcommand{\Bxi}[0]{\boldsymbol{\xi}}
\newcommand{\Bzeta}[0]{\boldsymbol{\zeta}}
%
%</bold and dot greek symbols>
%<infrequent>
%
%\newcommand{\AreaOp}[1]{\AName_{#1}}
%\newcommand{\Babs}[0]{\abs{\BB}}
%\newcommand{\Bcap}[0]{\hat{\BB}}
%\newcommand{\BrPrimeRej}[0]{\rcap(\rcap \wedge \Br')}
%\newcommand{\CA}[0]{\mathcal{A}}
%\newcommand{\Cos}[1]{\cos{\left({#1}\right)}}
%\newcommand{\Det}[1] {\abs{#1}}
%\newcommand{\Dsq}[2] {\frac {\partial^2 {#1}} {\partial {#2}^2}}
%\newcommand{\Exp}[1]{\exp{\left({#1}\right)}}
%\newcommand{\Norm}[1]{\left\lVert{#1}\right\rVert}
%\newcommand{\Sin}[1]{\sin{\left({#1}\right)}}
%\newcommand{\T}[0]{\text{T}}
%\newcommand{\VolumeOp}[1]{\VName_{#1}}
%\newcommand{\agrad}[0]{\Ba \cdot \nabla}
%\newcommand{\alphacap}[0]{\hat{\boldsymbol{\alpha}}}
%\newcommand{\Fcap}[0]{\hat{\BF}}
%\newcommand{\bithree}[0]{{\Bi}_3}
%\newcommand{\bxa}[0]{\Bx\Ba}
%\newcommand{\coordvec}[2]{
%\newcommand{\costheta}[0]{\acap \cdot \xcap}
%\newcommand{\ddt}[1]{\ddot{#1}}
%\newcommand{\ddu}[1] {\frac {d{#1}} {du}}
%\newcommand{\dsqxj}[2] {\frac {\partial^2 {#1}} {\partial {x_{#2}}^2}}
%\newcommand{\dtheta}[1]{\frac{d {#1}}{d \theta}}
%\newcommand{\dt}[1]{\dot{#1}}
%\newcommand{\dt}[1]{\frac{d {#1}}{dt}}
%\newcommand{\dxj}[2] {\frac {\partial {#1}} {\partial {x_{#2}}}}
%\newcommand{\halfPhi}[0]{\frac{\phi}{2}}
%\newcommand{\half}[0]{\inv{2}}
%\newcommand{\inv}[1]{\frac{1}{#1}}
%\newcommand{\laplacian}[0]{\nabla^2}
%\newcommand{\matrixoftx}[3]{
%\newcommand{\nrrp}[0]{\norm{\rcap \wedge \Br'}}
%\newcommand{\oiint}{\bigcirc \hspace{-1.4em} \int \hspace{-.8em} \int}
%\newcommand{\transpose}[1]{{#1}^{\text{T}}}
%\newcommand{\transpose}[1]{{{#1}^{\TextTranspose}}}
%\newcommand{\transpose}[1]{{{#1}^{\text{T}}}}
%\newcommand{\barA}[0]{\bar{A}}
%\newcommand{\qbar}[0]{\bar{q}}
%\newcommand{\qdotbar}[0]{\dot{\bar{q}}}
%
%</infrequent>




\newcommand{\symmetric}[2]{{\left\{{#1},{#2}\right\}}}
\newcommand{\antisymmetric}[2]{\left[{#1},{#2}\right]}
\DeclareMathOperator{\sgn}{sgn}
\DeclareMathOperator{\something}{something}

\newcommand{\uDETuvij}[4]{
\begin{vmatrix}
 {#1}^{#3} & {#1}^{#4} \\
 {#2}^{#3} & {#2}^{#4}
\end{vmatrix}
}

\newcommand{\PDSq}[2]{\frac{\partial^2 {#2}}{\partial {#1}^2}}
\newcommand{\transpose}[1]{{#1}^{\mathrm{T}}}
\newcommand{\stardot}[0]{{*}}

% bivector.tex:
\newcommand{\laplacian}[0]{\nabla^2}
\newcommand{\Dsq}[2] {\frac {\partial^2 {#1}} {\partial {#2}^2}}
\newcommand{\dxj}[2] {\frac {\partial {#1}} {\partial {x_{#2}}}}
\newcommand{\dsqxj}[2] {\frac {\partial^2 {#1}} {\partial {x_{#2}}^2}}
\DeclareMathOperator{\ExpName}{e}
%\DeclareMathOperator{\Exp}{e}
%\newcommand{\Exp}[1]{\exp{\left({#1}\right)}}
%\DeclareMathOperator{\Rej}{Rej}
\DeclareMathOperator{\Rot}{R}
%\newcommand{\gpgrade}[2] {{\left\langle{{#1}}\right\rangle}_{#2}}
%\newcommand{\gpgradezero}[1] {\gpgrade{#1}{0}}
%\newcommand{\gpgradetwo}[1] {\gpgrade{#1}{2}}
%\newcommand{\gpgradefour}[1] {\gpgrade{#1}{4}}

% ga_wiki_torque.tex:
\newcommand{\Fcap}[0]{\hat{\BF}}
\newcommand{\bithree}[0]{{\Bi}_3}
\newcommand{\nrrp}[0]{\norm{\rcap \wedge \Br'}}
\newcommand{\dtheta}[1]{\frac{d {#1}}{d \theta}}

% ga_wiki_unit_derivative.tex
\newcommand{\dt}[1]{\frac{d {#1}}{dt}}
\newcommand{\BrPrimeRej}[0]{\rcap(\rcap \wedge \Br')}

% radial_vector_derivatives.tex:
%\newcommand{\BrPrimeRej}[0]{\rcap(\rcap \wedge \Br')}

% angular_velocity.tex

%\newcommand{\dt}[1]{\frac{d {#1}}{dt}}
%\newcommand{\Norm}[1]{\left\lVert{#1}\right\rVert}
%\newcommand{\dtheta}[1]{\frac{d {#1}}{d \theta}}

% reciprocal_frame.tex
\DeclareMathOperator{\AbsName}{abs}

%\DeclareMathOperator{\RejName}{Rej}
%\newcommand{\Rej}[2]{\RejName_{#1}\left( {#2} \right)}

\DeclareMathOperator{\AName}{A}
\newcommand{\AreaOp}[1]{\AName_{#1}}

\DeclareMathOperator{\VName}{V}
\newcommand{\VolumeOp}[1]{\VName_{#1}}

%\newcommand{\gpgrade}[2] {{\left\langle{{#1}}\right\rangle}_{#2}}
%\newcommand{\gpgradeone}[1] {{\left\langle{{#1}}\right\rangle}_{1}}


% projection_with_matrix_comparison.tex
%\DeclareMathOperator{\Transpose}{T}
\DeclareMathOperator{\rank}{rank}
%\newcommand{\transpose}[1]{{{#1}^{\TextTranspose}}}
%\newcommand{\transpose}[1]{{{#1}^{\text{T}}}}
\newcommand{\T}[0]{{\text{T}}}
%\newcommand{\BOmega}[0]{\boldsymbol{\Omega}}

%\newcommand{\Det}[1] {\abs{#1}}

% oblique_proj.tex
%\newcommand{\T}[0]{\text{T}}
%\newcommand{\Bbeta}[0]{\boldsymbol{\beta}}

% spherical_polar.tex
\newcommand{\phicap}[0]{\hat{\boldsymbol{\phi}}}
\newcommand{\Lor}[2]{{{\Lambda^{#1}}_{#2}}}
\newcommand{\ILor}[2]{{{ \{{\Lambda^{-1}\} }^{#1}}_{#2}}}

% slerp.tex
\DeclareMathOperator{\atan2}{atan2}

% kvector_exponential.tex
%\DeclareMathOperator{\Exp}{e}
%\DeclareMathOperator{\Rej}{Rej}
\newcommand{\Bcap}[0]{\hat{\BB}}
\newcommand{\Babs}[0]{\abs{\BB}}
%\newcommand{\gpgrade}[2] {{\left\langle{{#1}}\right\rangle}_{#2}}
%\newcommand{\gpgradezero}[1] {\gpgrade{#1}{0}}
%\newcommand{\gpgradetwo}[1] {\gpgrade{#1}{2}}
%\newcommand{\gpgradefour}[1] {\gpgrade{#1}{4}}

\newcommand{\ddu}[1] {\frac {d{#1}} {du}}

% vector_integral_relations.tex
%\newcommand{\Oiint}{\bigcirc \hspace{-1.4em} \int \hspace{-.8em} \int}

% legendre.tex
\newcommand{\agrad}[0]{\Ba \cdot \nabla}
\newcommand{\bxa}[0]{\Bx\Ba}
\newcommand{\costheta}[0]{\acap \cdot \xcap}
%\newcommand{\inv}[1]{\frac{1}{#1}}
\newcommand{\half}[0]{\inv{2}}

% ke_rotation.tex
\newcommand{\DotT}[1]{\dot{#1}}
\newcommand{\DDotT}[1]{\ddot{#1}}
%\newcommand{\transpose}[1]{{#1}^{\text{T}}}
%\newcommand{\Balpha}[0]{\boldsymbol{\alpha}}

%\newcommand{\gpgrade}[2] {{\left\langle{{#1}}\right\rangle}_{#2}}
%\newcommand{\gpgradeone}[1] {{\left\langle{{#1}}\right\rangle}_{1}}
\newcommand{\gpscalargrade}[1] {{\left\langle{{#1}}\right\rangle}}
%\newcommand{\BOmega}[0]{\boldsymbol{\Omega}}

% gaussian_surface.tex
%\newcommand{\phicap}[0]{\hat{\Bphi}}

% newtonian_lagrangian_and_gradient.tex
% PD macro that is backwards from current in macros2:
\newcommand{\PDb}[2]{ \frac{\partial{#1}}{\partial {#2}} }

% inertial_tensor.tex
\newcommand{\matrixoftx}[3]{
{
\begin{bmatrix}
{#1}
\end{bmatrix}
}_{#2}^{#3}
}

\newcommand{\coordvec}[2]{
{
\begin{bmatrix}
{#1}
\end{bmatrix}
}_{#2}
}

% bohr.tex
\newcommand{\K}[0]{\inv{4 \pi \epsilon_0}}

% euler_lagrange.tex
\newcommand{\qbar}[0]{\bar{q}}
\newcommand{\qdotbar}[0]{\dot{\bar{q}}}
\newcommand{\DD}[2]{\frac{d{#2}}{d{#1}}}
\newcommand{\Xdot}[0]{\dot{X}}

% rayleigh_jeans.tex
\newcommand{\EE}[0]{\boldsymbol{\mathcal{E}}}
\newcommand{\HH}[0]{\boldsymbol{\mathcal{H}}}

% 4d_fourier.tex

%\newcommand{\PDSq}[2]{\frac{\partial^2 {#2}}{\partial {#1}^2}}
\DeclareMathOperator{\sinc}{sinc}
\DeclareMathOperator{\PV}{PV}
\newcommand{\FF}[0]{\mathcal{F}}
\newcommand{\IIinf}[0]{ \int_{-\infty}^\infty }

% poisson.tex
%\newcommand{\PDSq}[2]{\frac{\partial^2 {#2}}{\partial {#1}^2}}
%\DeclareMathOperator{\sinc}{sinc}
%\DeclareMathOperator{\PV}{PV}
%\newcommand{\FF}[0]{\mathcal{F}}
%\newcommand{\IIinf}[0]{ \int_{-\infty}^\infty }

% fourier_maxwell.tex
%\newcommand{\PDSq}[2]{\frac{\partial^2 {#2}}{\partial {#1}^2}}
%\DeclareMathOperator{\sinc}{sinc}
%\DeclareMathOperator{\sgn}{sgn}
%\DeclareMathOperator{\PV}{PV}
%\newcommand{\FF}[0]{\mathcal{F}}
%\newcommand{\IIinf}[0]{ \int_{-\infty}^\infty }

% firstorder_fourier_maxwell.tex
%\newcommand{\PDSq}[2]{\frac{\partial^2 {#2}}{\partial {#1}^2}}
%\DeclareMathOperator{\sinc}{sinc}
%\DeclareMathOperator{\PV}{PV}
%\newcommand{\FF}[0]{\mathcal{F}}
%\newcommand{\IIinf}[0]{ \int_{-\infty}^\infty }

% wave_fourier.tex
%\newcommand{\PDSq}[2]{\frac{\partial^2 {#2}}{\partial {#1}^2}}
%\DeclareMathOperator{\sinc}{sinc}
%\DeclareMathOperator{\PV}{PV}
%\newcommand{\FF}[0]{\mathcal{F}}
%\newcommand{\IIinf}[0]{ \int_{-\infty}^\infty }

% heat_fourier.tex
%\newcommand{\PDSq}[2]{\frac{\partial^2 {#2}}{\partial {#1}^2}}
%\DeclareMathOperator{\sinc}{sinc}
%\newcommand{\FF}[0]{\mathcal{F}}
%\newcommand{\IIinf}[0]{ \int_{-\infty}^\infty }

% proj_generalized_dot_prod.tex
%\newcommand{\T}[0]{\text{T}}

% fourier_tx.tex
%\newcommand{\FF}[0]{\mathcal{F}}
\newcommand{\FM}[0]{\inv{\sqrt{2\pi\hbar}}}
\newcommand{\Iinf}[1]{ \int_{-\infty}^\infty {#1}}
%\DeclareMathOperator{\PV}{PV}

% fourier_notation.tex
%\newcommand{\FF}[0]{\mathcal{F}}
%\newcommand{\IIinf}[0]{ \int_{-\infty}^\infty }
%\DeclareMathOperator{\PV}{PV}
%\DeclareMathOperator{\sinc}{sinc}

% planewave.tex
%\newcommand{\EE}[0]{\boldsymbol{\mathcal{E}}}
%\newcommand{\HH}[0]{\boldsymbol{\mathcal{H}}}
%\newcommand{\IIinf}[0]{ \int_{-\infty}^\infty }

% dirac_lagrangian.tex
\newcommand{\Dslash}[0]{ \not\!D }

% pauli_matrix.tex
\newcommand{\Clifford}[2]{\mathcal{C}_{\{{#1},{#2}\}}}
\DeclareMathOperator{\tr}{Tr}
%\DeclareMathOperator{\Scalar}{Scalar}
\DeclareMathOperator{\Real}{Re}
\DeclareMathOperator{\Imag}{Im}
\newcommand{\trace}[1]{\tr{#1}}
\newcommand{\scalarProduct}[2]{{#1} \bullet {#2}}
\newcommand{\traceB}[1]{\tr\left({#1}\right)}
%\newcommand{\symmetric}[2]{{\left\{{#1},{#2}\right\}}}
%\newcommand{\antisymmetric}[2]{\left[{#1},{#2}\right]}
%\newcommand{\Bcap}[0]{\hat{\BB}}

\newcommand{\xhat}[0]{\hat{x}}

\newcommand{\PauliI}[0]{
\begin{bmatrix}
1 & 0 \\
0 & 1 \\
\end{bmatrix}
}

\newcommand{\PauliX}[0]{
\begin{bmatrix}
0 & 1 \\
1 & 0 \\
\end{bmatrix}
}

\newcommand{\PauliY}[0]{
\begin{bmatrix}
0 & -i \\
i & 0 \\
\end{bmatrix}
}

\newcommand{\PauliYNoI}[0]{
\begin{bmatrix}
0 & -1 \\
1 & 0 \\
\end{bmatrix}
}

\newcommand{\PauliZ}[0]{
\begin{bmatrix}
1 & 0 \\
0 & -1 \\
\end{bmatrix}
}

% gamma.tex
%\newcommand{\scalarProduct}[2]{{#1} \bullet {#2}}
%\newcommand{\symmetric}[2]{{\left\{{#1},{#2}\right\}}}
%\newcommand{\antisymmetric}[2]{\left[{#1},{#2}\right]}

%\newcommand{\PauliX}[0]{
%\begin{bmatrix}
%0 & 1 \\
%1 & 0 \\
%\end{bmatrix}
%}

%\newcommand{\PauliY}[0]{
%\begin{bmatrix}
%0 & -i \\
%i & 0 \\
%\end{bmatrix}
%}

%\newcommand{\PauliYNoI}[0]{
%\begin{bmatrix}
%0 & -1 \\
%1 & 0 \\
%\end{bmatrix}
%}

%\newcommand{\PauliZ}[0]{
%\begin{bmatrix}
%1 & 0 \\
%0 & -1 \\
%\end{bmatrix}
%}

% em_bivector_metric_dependencies.tex

%\newcommand{\LL}[0]{\mathcal{L}}
%\newcommand{\gpgrade}[2] {{\left\langle{{#1}}\right\rangle}_{#2}}
%\newcommand{\gpgradezero}[1] {\gpgrade{#1}{0}}
%\newcommand{\gpgradetwo}[1] {\gpgrade{#1}{2}}
%\newcommand{\gpgradeone}[1] {\gpgrade{#1}{1}}
%\newcommand{\gpgradefour}[1] {\gpgrade{#1}{4}}
%\newcommand{\grad}[0]{\nabla}
%\newcommand{\spacegrad}[0]{\boldsymbol{\nabla}}
% == \partial_{#1} {#2}
%\newcommand{\PD}[2]{\frac{\partial {#2}}{\partial {#1}}}
%\newcommand{\PDD}[3]{\frac{\partial^2 {#3}}{\partial {#1}\partial {#2}}}
\newcommand{\PDsQ}[2]{\frac{\partial^2 {#2}}{\partial^2 {#1}}}

% gem.tex
\newcommand{\barh}[0]{\bar{h}}

% mass_vary_lagrangian.tex
%\newcommand{\LL}[0]{\mathcal{L}}
%\newcommand{\grad}[0]{\nabla}
%\newcommand{\PD}[2]{\frac{\partial {#2}}{\partial {#1}}}
%\newcommand{\xdot}[0]{\dot{x}}
%\newcommand{\vdot}[0]{\dot{v}}
%\newcommand{\mdot}[0]{\dot{m}}
%\newcommand{\xddot}[0]{\ddot{x}}
%\newcommand{\spacegrad}[0]{\boldsymbol{\nabla}}

% fourvec_dotinvariance.tex
%\newcommand{\Balpha}[0]{\boldsymbol{\alpha}}
\newcommand{\alphacap}[0]{\hat{\boldsymbol{\alpha}}}
%\newcommand{\Bcap}[0]{\hat{\BB}}
%\newcommand{\gpgrade}[2] {{\left\langle{{#1}}\right\rangle}_{#2}}
%\newcommand{\gpgradezero}[1] {\gpgrade{#1}{0}}

% lorentz.tex
%\newcommand{\laplacian}[0]{\nabla^2}

% field_lagrangian.tex
%\newcommand{\LL}[0]{\mathcal{L}}
%\newcommand{\PD}[2]{\frac{\partial {#2}}{\partial {#1}}}
\newcommand{\barA}[0]{\bar{A}}
%\newcommand{\grad}[0]{\nabla}
%\newcommand{\conj}[0]{{*}}

%\newcommand{\spacegrad}[0]{\boldsymbol{\nabla}}

%\newcommand{\gpgrade}[2] {{\left\langle{{#1}}\right\rangle}_{#2}}
%\newcommand{\gpgradezero}[1] {\gpgrade{#1}{0}}
%\newcommand{\gpgradetwo}[1] {\gpgrade{#1}{2}}
%\newcommand{\gpgradefour}[1] {\gpgrade{#1}{4}}

% lagrangian_field_density.tex
%\newcommand{\LL}[0]{\mathcal{L}}
%\newcommand{\gpgrade}[2] {{\left\langle{{#1}}\right\rangle}_{#2}}
%\newcommand{\gpgradezero}[1] {\gpgrade{#1}{0}}
%\newcommand{\gpgradetwo}[1] {\gpgrade{#1}{2}}
%\newcommand{\gpgradefour}[1] {\gpgrade{#1}{4}}
%\newcommand{\grad}[0]{\nabla}
%\newcommand{\spacegrad}[0]{\boldsymbol{\nabla}}
%\newcommand{\PD}[2]{\frac{\partial {#2}}{\partial {#1}}}
\newcommand{\PDd}[2]{\frac{\partial^2 {#2}}{{\partial{#1}}^2}}
%\newcommand{\PDD}[3]{\frac{\partial^2 {#3}}{\partial {#1}\partial {#2}}}

%\newcommand{\barA}[0]{\bar{A}}

% lorentz_force.tex
%\newcommand{\grad}[0]{\nabla}
%\newcommand{\spacegrad}[0]{\boldsymbol{\nabla}}
%\newcommand{\LL}[0]{\mathcal{L}}
%\newcommand{\xdot}[0]{\dot{x}}
%\newcommand{\xddot}[0]{\ddot{x}}
%\newcommand{\pdot}[0]{\dot{p}}
%\newcommand{\pddot}[0]{\ddot{p}}
%\newcommand{\fdot}[0]{\dot{f}}
%\newcommand{\fddot}[0]{\ddot{f}}

%\newcommand{\gpgrade}[2] {{\left\langle{{#1}}\right\rangle}_{#2}}
%\newcommand{\gpgradeone}[1] {\gpgrade{#1}{1}}
%\newcommand{\gpgradezero}[1] {\gpgrade{#1}{}}
%\newcommand{\grad}[0] {\nabla}
%\newcommand{\spacegrad}[0]{\boldsymbol{\nabla}}

%\newcommand{\pdot}[0]{\dot{p}}
%\newcommand{\pddot}[0]{\ddot{p}}

%\newcommand{\xdot}[0]{\dot{x}}
%\newcommand{\xddot}[0]{\ddot{x}}
%\newcommand{\PD}[2]{\frac{\partial {#2}}{\partial {#1}}}

% stokes_maxwell_application.tex
%\newcommand{\grad}[0]{\nabla}
%\newcommand{\PD}[2]{\frac{\partial {#2}}{\partial {#1}}}
%\newcommand{\spacegrad}[0]{\boldsymbol{\nabla}}
%\newcommand{\gpgrade}[2] {{\left\langle{{#1}}\right\rangle}_{#2}}
%\newcommand{\gpgradezero}[1] {\gpgrade{#1}{0}}
%\newcommand{\gpgradeone}[1] {\gpgrade{#1}{1}}
%\newcommand{\gpgradetwo}[1] {\gpgrade{#1}{2}}
%\newcommand{\gpgradethree}[1] {\gpgrade{#1}{3}}

% lorentz_rotation.tex
%\DeclareMathOperator{\Transpose}{T}
%\newcommand{\T}[0]{\text{T}}

% electron_rotor.tex
\newcommand{\reverse}[1]{\tilde{{#1}}}
%\newcommand{\ILambda}[0]{{(\Lambda^{-1})}}
\newcommand{\ILambda}[0]{\Pi}

% em_potential.tex
%\newcommand{\spacegrad}[0]{\boldsymbol{\nabla}}
%\newcommand{\grad}[0]{\nabla}
\newcommand{\CA}[0]{\mathcal{A}}
 
% maxwell_to_tensor.tex
%\newcommand{\LL}[0]{\mathcal{L}}
%\newcommand{\gpgrade}[2] {{\left\langle{{#1}}\right\rangle}_{#2}}
%\newcommand{\gpgradezero}[1] {\gpgrade{#1}{0}}
%\newcommand{\gpgradetwo}[1] {\gpgrade{#1}{2}}
%\newcommand{\gpgradeone}[1] {\gpgrade{#1}{1}}
%\newcommand{\gpgradefour}[1] {\gpgrade{#1}{4}}
%\newcommand{\grad}[0]{\nabla}
%\newcommand{\spacegrad}[0]{\boldsymbol{\nabla}}
% == \partial_{#1} {#2}
%\newcommand{\PD}[2]{\frac{\partial {#2}}{\partial {#1}}}
%\newcommand{\PDD}[3]{\frac{\partial^2 {#3}}{\partial {#1}\partial {#2}}}
%\newcommand{\PDsQ}[2]{\frac{\partial^2 {#2}}{\partial^2 {#1}}}

%\newcommand{\EE}[0]{\boldsymbol{\mathcal{E}}}
%\newcommand{\HH}[0]{\boldsymbol{\mathcal{H}}}
\newcommand{\Vcap}[0]{\hat{\BV}}




%\usepackage{listings}
%\usepackage{txfonts} % for ointctr... (also appears to make "prettier" \int and \sum's)
% makes \grad look funny though (almost like spacegrad, but narrower)
\usepackage[bookmarks=true]{hyperref}

\usepackage{color,cite,graphicx}
   % use colour in the document, put your citations as [1-4]
   % rather than [1,2,3,4] (it looks nicer, and the extended LaTeX2e
   % graphics package. 
\usepackage{latexsym,amssymb,epsf} % don't remember if these are
   % needed, but their inclusion can't do any damage

\title{Poincare transformation symmetries}
\author{Peeter Joot \quad peeter.joot@gmail.com }
\date{ June 1, 2009.  $RCSfile: poincareTx.tex,v $ Last $Revision: 1.1 $ $Date: 2009/07/01 14:33:59 $ }

\begin{document}

\maketitle{}
\tableofcontents
\section{Motivation}

In \cite{montesinos2006sem} a Poincare transformation is used to 
develop the symmetric stress energy tensor directly, in contrast to the
non-symmetric canonical stress energy tensor that results from 
spacetime translation.

Attempt to decode some bits of this article.

\section{Guts}

Equation (11) in the article, is labelled an infinitesimal Poincare
transformation

\begin{align}\label{eqn:txComponents}
{x'}^\mu
&=
{x'}^\mu
+ {{\epsilon}^\mu}_\nu x^\nu
+ {\epsilon}^\mu
\end{align}

It is stated that an antisymmetrization conditon $\epsilon_{\mu\nu} = \epsilon_{\nu\mu}$.  This is somewhat confusing 
since the infinitesimal transformation is given by a mixed upper and lower index tensor.   Due to the antisymmetry
perhaps this all a coordinate statement of the following vector to vector linear transformation

\begin{align}\label{eqn:blah}
x' = x + \epsilon + A \cdot x
\end{align}

This transformation is less restricted than a plain old spacetime transformation, as it also contains a projective term, where $x$ is projected onto the spacetime (or spatial) plane $A$ (a bivector), plus a rotation in that plane.

Writing as usual

\begin{align*}
x = \gamma_\mu x^\mu
\end{align*}

So that components are recovered by taking dot products, as in
\begin{align*}
x^\mu = x \cdot \gamma^\mu
\end{align*}

For the bivector term, write

\begin{align*}
A = c \wedge d = c^\alpha d^\beta (\gamma_\alpha \wedge \gamma_\beta)
\end{align*}

For
\begin{align*}
(A \cdot x ) \cdot \gamma^\mu 
&=
c^\alpha d^\beta x_\sigma ((\gamma_\alpha \wedge \gamma_\beta) \cdot \gamma^\sigma) \cdot \gamma^\mu \\
&=
c^\alpha d^\beta x_\sigma ( {\delta_\alpha}^\mu {\delta_\beta}^\sigma -{\delta_\beta}^\mu {\delta_\alpha}^\sigma ) \\
&=
(c^\mu d^\sigma -c^\sigma d^\mu ) x_\sigma 
\end{align*}

This allows for an identification $\epsilon^\mu\sigma = c^\mu d^\sigma -c^\sigma d^\mu$ which is antisymmetric as required.
With that identification we can write (\ref{eqn:txComponents}) via the equivalent vector relation (\ref{eqn:blah}) if
we write

\begin{align*}
{\epsilon^\mu}_\sigma x^\sigma = (c^\mu d_\sigma -c_\sigma d^\mu ) x^\sigma 
\end{align*}

Where ${\epsilon^\mu}_\sigma$ is defined implicitly in terms of components of the bivector $A = c \wedge d$.

Is this what a Poincare transformation is?  The \href{http://mathworld.wolfram.com/PoincareTransformation.html}{Poincare Transformation} article suggests not.  This article suggests that the Poincare transformation is a spacetime translation plus
a Lorentz tranformation (composition of boosts and rotations).  That Lorentz transformation will not be antisymmetric
however, so how can these be reconciled?  The key is probably the fact that this was an infinistesimal Poincare transformation
so lets consider a Taylor expansion of the Lorentz boost or rotation rotor.

\bibliographystyle{plainnat}
\bibliography{myrefs}

\end{document}

\part{Electrodynamics.}
%
% Copyright � 2012 Peeter Joot.  All Rights Reserved.
% Licenced as described in the file LICENSE under the root directory of this GIT repository.
%

%
%
\chapter{Maxwell's equations expressed with Geometric Algebra}
\index{Maxwell's equations}
\label{chap:maxwellsGa}
%\date{January 29, 2008.  maxwellsGa.tex}

\section{On different ways of expressing Maxwell's equations}

One of the most striking applications of the geometric product is the ability to formulate the eight Maxwell's equations in a coherent fashion as a single equation.

This is not a new idea, and this has been done historically using formulations based on quaternions (~1910.  dig up citation).  A formulation in terms of antisymmetric second rank tensors \(F_{\mu \nu}\) and \(G_{\mu \nu}\) (See: wiki:Formulation of Maxwell's equations in special relativity) reduces the eight equations to two, but also introduces complexity and obfuscates the connection to the physically measurable quantities.

A formulation in terms of differential forms (See: wiki:Maxwell's equations) is also possible.  This does not have the complexity of the tensor formulation, but requires the electromagnetic field to be expressed as a differential form.  This is arguably strange given a traditional vector calculus education.  One also does not have to integrate a field in any fashion, so what meaning should be given to a electrodynamic field as a differential form?

\subsection{Introduction of complex vector electromagnetic field}

To explore the ideas, the starting point is the traditional set of Maxwell's equations

\begin{dmath}\label{eqn:maxwellsGa:20}
\nabla \cdot \BE  = \frac {\rho} {\epsilon_0}
\end{dmath}
\begin{dmath}\label{eqn:maxwellsGa:40}
\nabla \cdot \BB  = 0
\end{dmath}
\begin{dmath}\label{eqn:maxwellsGa:60}
\nabla \times \BE  +\frac{\partial \BB } {\partial t} = 0
\end{dmath}
\begin{dmath}\label{eqn:maxwellsGa:80}
c^2 \nabla \times \BB  - \frac{\partial \BE } {\partial t}
= \frac{\BJ}{\epsilon_0}
\end{dmath}

It is customary in relativistic treatments of electrodynamics to introduce a four vector \((x, y, z, ict)\).  Using this as a hint, one can write the time partials in terms of \(ict\) and regrouping slightly

\begin{dmath}\label{eqn:maxwellsGa:100}
\nabla \cdot \BE  = \frac {\rho} {\epsilon_0}
\end{dmath}
\begin{dmath}\label{eqn:maxwellsGa:120}
\nabla \cdot (ic\BB ) = 0
\end{dmath}
\begin{dmath}\label{eqn:maxwellsGa:140}
\nabla \times \BE  +\frac{\partial (ic\BB )} {\partial (ict)} = 0
\end{dmath}
\begin{dmath}\label{eqn:maxwellsGa:160}
\nabla \times (ic\BB ) + \frac{\partial \BE } {\partial (ict)}
= i\frac{\BJ}{\epsilon_0 c}
\end{dmath}

There is no use of geometric or wedge products here, but the opposing signs in the two sets of curl and time partial equations is removed.  The pairs of equations can be added together without loss of information since the original equations can be recovered by taking real and imaginary parts.

\begin{dmath}\label{eqn:maxwellsGa:180}
\nabla \cdot (\BE + ic \BB) = \frac {\rho} {\epsilon_0}
\end{dmath}
\begin{dmath}\label{eqn:maxwellsGa:200}
\nabla \times (\BE + ic \BB) + \frac{\partial (\BE + ic \BB)} {\partial (ict)}
= i\frac{\BJ}{\epsilon_0 c}
\end{dmath}

It is thus natural to define a combined electrodynamic field as a complex vector, expressing the natural orthogonality of the electric and magnetic fields

\begin{dmath}\label{eqn:maxwellsGa:220}
\BF = \BE + ic \BB
\end{dmath}

The electric and magnetic fields can be recovered from this composite field by taking real and imaginary parts respectively, and we can now write write Maxwell's equations in terms of this single electrodynamic field

\begin{dmath}\label{eqn:maxwellsGa:240}
\nabla \cdot \BF = \frac {\rho} {\epsilon_0}
\end{dmath}
\begin{dmath}\label{eqn:maxwellsGa:260}
\nabla \times \BF + \frac{\partial \BF} {\partial (ict)}
= i\frac{\BJ}{\epsilon_0 c}
\end{dmath}

\subsection{Converting the curls in the pair of Maxwell's equations for the electrodynamic field to wedge and geometric products}

The above manipulations didn't make any assumptions about the structure of the ``imaginary'' denoted \(i\) above.  What was implied was a requirement that \(i^2 = -1\), and that \(i\) commutes with vectors.  Both of these conditions are met by the use of the pseudoscalar for 3D Euclidean space \(\Be_1 \Be_2 \Be_3\).  This is usually denoted \(I\) and we'll now switch notations for clarity.
XX
With multiplication of the second by a \(I\) factor to convert to a wedge product representation the remaining pair of equations can be written

\begin{dmath}\label{eqn:maxwellsGa:280}
\nabla \cdot \BF = \frac {\rho} {\epsilon_0}
\end{dmath}
\begin{dmath}\label{eqn:maxwellsGa:300}
I\nabla \times \BF + \frac{1}{c} \frac{\partial \BF}{\partial t}
= -\frac{\BJ}{\epsilon_0 c}
\end{dmath}

This last, in terms of the geometric product is,
\begin{dmath}\label{eqn:maxwellsGa:320}
\nabla \wedge \BF + \frac{1}{c} \frac{\partial \BF}{\partial t}
= -\frac{\BJ}{\epsilon_0 c}
\end{dmath}

These equations can be added without loss

\begin{dmath}\label{eqn:maxwellsGa:340}
\nabla \cdot \BF + \nabla \wedge \BF + \frac{1}{c} \frac{\partial \BF}{\partial t} = \frac {\rho} {\epsilon_0} - \frac{\BJ}{\epsilon_0 c}
\end{dmath}

Leading to the end result

\begin{dmath}\label{eqn:maxwellsGa:5}
\left(\frac{1}{c} \frac{\partial}{\partial t} + \nabla\right)\BF = \frac {1} {\epsilon_0}\left(\rho - \frac{\BJ}{c}\right)
\end{dmath}

Here we have all of Maxwell's equations as a single differential equation.
This gives a hint why it is hard to separately solve these equations for the electric or magnetic field components (the partials of which are scattered across the original eight different equations.)  Logically the electric and magnetic field components have to be kept together.

Solution of this equation will require some new tools.  Minimally, some relearning of existing vector calculus tools is required.

\subsection{Components of the geometric product Maxwell equation}

Explicit expansion of this equation, again using \(I={\Be}_1{\Be}_2{\Be}_3\), will yield a scalar, vector, bivector, and pseudoscalar components, and is an interesting exercise to verify the simpler field equation really describes the same thing.

\begin{dmath}\label{eqn:maxwellsGa:360}
\left(\frac{1}{c} \frac{\partial}{\partial t} + \nabla\right)\BF
= \frac{1}{c} \frac{\partial \BE}{\partial t} + I\frac{1}{c} \frac{\partial \BB}{\partial t}
+ \nabla \cdot \BE + \nabla \wedge \BE + \nabla \cdot I \BB + \nabla \wedge I \BB
\end{dmath}

The imaginary part of the field can be multiplied out as bivector components explicitly

\begin{equation}\label{eqn:maxwellsGa:620}
\begin{aligned}
I \BB &= \Be _1 \Be _2 \Be _3 ( \Be _1 B_1 + \Be _2 B_2 + \Be _3 B_3 ) \\
&= \Be _2 \Be _3 B_1 + \Be _3 \Be _1 B_2 + \Be _1 \Be _2 B_3
\end{aligned}
\end{equation}

which allows for direct calculation of the following

\begin{dmath}\label{eqn:maxwellsGa:380}
\nabla \wedge I\BB = I\nabla \cdot \BB
\end{dmath}
\begin{dmath}\label{eqn:maxwellsGa:400}
\nabla \cdot I\BB = -\nabla \times \BB.
\end{dmath}

These can be demonstrated by reducing \( \gpgradethree{ \nabla I \BB } \), and \( \gpgradeone{ \nabla I \BB } \) respectively.  Using these identities and writing the electric field curl term in terms of the cross product

\begin{dmath}\label{eqn:maxwellsGa:420}
\nabla \wedge \BE = I \nabla \times \BE,
\end{dmath}

allows for grouping of real and imaginary scalar and real and imaginary vector (bivector) components

\begin{dmath}\label{eqn:maxwellsGa:440}
   \left(\nabla \cdot \BE\right) + I\left(\nabla \cdot \BB\right)
+
   \left(\frac{1}{c} \frac{\partial \BE}{\partial t} - \nabla \times \BB\right)
+ I\left(\frac{1}{c} \frac{\partial \BB}{\partial t} + \nabla \times \BE\right)
\end{dmath}
\begin{dmath}\label{eqn:maxwellsGa:460}
= \frac{\rho}{\epsilon_0} + I\left(0\right) + \left(-\frac{\BJ}{\epsilon_0 c}\right) + I \Bzero.
\end{dmath}

Comparing each of the left and right side components recovers the original set of four (or eight depending on your point of view) Maxwell's equations.

%\section{Future: comparison to gravitation?}
%
%% Wed 03/05/2008
%
%The high school electrostatics equation, where \(\rho\) is either a continuous distribution or a spatial delta function for point masses:
%
%\begin{dmath}\label{eqn:maxwellsGa:480}
%\BE(\Br) = \inv{4 \pi \epsilon_0}\int{\rho(\Br') \frac{(\Br -\Br')}{(\Br -\Br')^2}}dV'
%\end{dmath}
%
%As a field equation this is written:
%\begin{dmath}\label{eqn:maxwellsGa:500}
%\nabla \cdot \BE(\Br) = \frac{\rho(\Br)}{\epsilon_0}
%\end{dmath}
%
%but this is both not relativistically correct nor does is
%include the propagation effects for ``electrostatics'' interactions
%which occur at the speed of light.
%
%We need the other three components of the Maxwell's
%equation \eqnref{eqn:maxwellsGa:5}, to get the propagation and relativistic
%corrections.
%
%Compare this to newton's gravitational field equation:
%
%\begin{dmath}\label{eqn:maxwellsGa:520}
%\BG(\Br) = -G\int{\rho(\Br') \frac{(\Br -\Br')}{(\Br -\Br')^2}}dV'
%\end{dmath}
%
%% G = 1/4 pi e
%% 4piG = 1/e
%
%which can be written as a field equation as:
%\begin{dmath}\label{eqn:maxwellsGa:540}
%\nabla \cdot \BG(\Br) = 4\pi G \rho(\Br).
%\end{dmath}
%
%If one assumes that electrodynamics and gravitation
%have the same form then is the corrected form of the gravitational field
%equation with respect to relativity and propagation at the speed of light
%as follows:
%
%\begin{equation}\label{eqn:maxwellsGa:600}
%\left(\frac{1}{c} \frac{\partial}{\partial t} + \nabla\right)\BG(\Br) =
%4 \pi G \rho(\Br)
%\end{equation}
%
%Is this correct in any sense?  Perhaps it matches the special relativity results
%but not the general relativity ones?

\documentclass{article}      % Specifies the document class

\usepackage{amsmath}
\usepackage{mathpazo}

%
% shorthand for bold symbols, convenient for vectors and matrices
%
\newcommand{\Ba}[0]{\mathbf{a}}
\newcommand{\Bb}[0]{\mathbf{b}}
\newcommand{\Bc}[0]{\mathbf{c}}
\newcommand{\Bd}[0]{\mathbf{d}}
\newcommand{\Be}[0]{\mathbf{e}}
\newcommand{\Bf}[0]{\mathbf{f}}
\newcommand{\Bg}[0]{\mathbf{g}}
\newcommand{\Bh}[0]{\mathbf{h}}
\newcommand{\Bi}[0]{\mathbf{i}}
\newcommand{\Bj}[0]{\mathbf{j}}
\newcommand{\Bk}[0]{\mathbf{k}}
\newcommand{\Bl}[0]{\mathbf{l}}
\newcommand{\Bm}[0]{\mathbf{m}}
\newcommand{\Bn}[0]{\mathbf{n}}
\newcommand{\Bo}[0]{\mathbf{o}}
\newcommand{\Bp}[0]{\mathbf{p}}
\newcommand{\Bq}[0]{\mathbf{q}}
\newcommand{\Br}[0]{\mathbf{r}}
\newcommand{\Bs}[0]{\mathbf{s}}
\newcommand{\Bt}[0]{\mathbf{t}}
\newcommand{\Bu}[0]{\mathbf{u}}
\newcommand{\Bv}[0]{\mathbf{v}}
\newcommand{\Bw}[0]{\mathbf{w}}
\newcommand{\Bx}[0]{\mathbf{x}}
\newcommand{\By}[0]{\mathbf{y}}
\newcommand{\Bz}[0]{\mathbf{z}}
\newcommand{\BA}[0]{\mathbf{A}}
\newcommand{\BB}[0]{\mathbf{B}}
\newcommand{\BC}[0]{\mathbf{C}}
\newcommand{\BD}[0]{\mathbf{D}}
\newcommand{\BE}[0]{\mathbf{E}}
\newcommand{\BF}[0]{\mathbf{F}}
\newcommand{\BG}[0]{\mathbf{G}}
\newcommand{\BH}[0]{\mathbf{H}}
\newcommand{\BI}[0]{\mathbf{I}}
\newcommand{\BJ}[0]{\mathbf{J}}
\newcommand{\BK}[0]{\mathbf{K}}
\newcommand{\BL}[0]{\mathbf{L}}
\newcommand{\BM}[0]{\mathbf{M}}
\newcommand{\BN}[0]{\mathbf{N}}
\newcommand{\BO}[0]{\mathbf{O}}
\newcommand{\BP}[0]{\mathbf{P}}
\newcommand{\BQ}[0]{\mathbf{Q}}
\newcommand{\BR}[0]{\mathbf{R}}
\newcommand{\BS}[0]{\mathbf{S}}
\newcommand{\BT}[0]{\mathbf{T}}
\newcommand{\BU}[0]{\mathbf{U}}
\newcommand{\BV}[0]{\mathbf{V}}
\newcommand{\BW}[0]{\mathbf{W}}
\newcommand{\BX}[0]{\mathbf{X}}
\newcommand{\BY}[0]{\mathbf{Y}}
\newcommand{\BZ}[0]{\mathbf{Z}}

\newcommand{\Bzero}[0]{\mathbf{0}}
\newcommand{\Btheta}[0]{\boldsymbol{\theta}}
\newcommand{\Btau}[0]{\boldsymbol{\tau}}
\newcommand{\Bomega}[0]{\boldsymbol{\omega}}

%
% shorthand for unit vectors
%
\newcommand{\acap}[0]{\hat{\Ba}}
\newcommand{\bcap}[0]{\hat{\Bb}}
\newcommand{\ccap}[0]{\hat{\Bc}}
\newcommand{\dcap}[0]{\hat{\Bd}}
\newcommand{\ecap}[0]{\hat{\Be}}
\newcommand{\fcap}[0]{\hat{\Bf}}
\newcommand{\gcap}[0]{\hat{\Bg}}
\newcommand{\hcap}[0]{\hat{\Bh}}
\newcommand{\icap}[0]{\hat{\Bi}}
\newcommand{\jcap}[0]{\hat{\Bj}}
\newcommand{\kcap}[0]{\hat{\Bk}}
\newcommand{\lcap}[0]{\hat{\Bl}}
\newcommand{\mcap}[0]{\hat{\Bm}}
\newcommand{\ncap}[0]{\hat{\Bn}}
\newcommand{\ocap}[0]{\hat{\Bo}}
\newcommand{\pcap}[0]{\hat{\Bp}}
\newcommand{\qcap}[0]{\hat{\Bq}}
\newcommand{\rcap}[0]{\hat{\Br}}
\newcommand{\scap}[0]{\hat{\Bs}}
\newcommand{\tcap}[0]{\hat{\Bt}}
\newcommand{\ucap}[0]{\hat{\Bu}}
\newcommand{\vcap}[0]{\hat{\Bv}}
\newcommand{\wcap}[0]{\hat{\Bw}}
\newcommand{\xcap}[0]{\hat{\Bx}}
\newcommand{\ycap}[0]{\hat{\By}}
\newcommand{\zcap}[0]{\hat{\Bz}}
\newcommand{\thetacap}[0]{\hat{\Btheta}}

%
% to write R^n and C^n in a distinguishable fashion.  Perhaps change this
% to the double lined characters upon figuring out how to do so.
%
\newcommand{\C}[1]{$\mathbb{C}^{#1}$}
\newcommand{\R}[1]{$\mathbb{R}^{#1}$}

%
% various generally useful helpers
%

% derivative of #1 wrt. #2:
\newcommand{\D}[2] {\frac {d#2} {d#1}}

\newcommand{\inv}[1]{\frac{1}{#1}}
\newcommand{\cross}[0]{\times}

\newcommand{\abs}[1]{\lvert{#1}\rvert}
\newcommand{\norm}[1]{\lVert{#1}\rVert}
\newcommand{\innerprod}[2]{\langle{#1}, {#2}\rangle}
\newcommand{\dotprod}[2]{{#1} \cdot {#2}}
\newcommand{\bdotprod}[2]{\left({#1} \cdot {#2}\right)}
\newcommand{\crossprod}[2]{{#1} \cross {#2}}
\newcommand{\tripleprod}[3]{\dotprod{\left(\crossprod{#1}{#2}\right)}{#3}}

\DeclareMathOperator{\Proj}{Proj}
\DeclareMathOperator{\Span}{span}
\DeclareMathOperator{\Sgn}{sgn}
\DeclareMathOperator{\Area}{Area}
\DeclareMathOperator{\Volume}{Volume}

%
% A few miscellaneous things specific to this document
%
\newcommand{\crossop}[1]{\crossprod{#1}{}}

% R2 vector.
\newcommand{\VectorTwo}[2]{
\begin{bmatrix}
 {#1} \\
 {#2}
\end{bmatrix}
}

\newcommand{\VectorN}[1]{
\begin{bmatrix}
{#1}_1 \\
{#1}_2 \\
\vdots \\
{#1}_N \\
\end{bmatrix}
}

\newcommand{\DETuvij}[4]{
\begin{vmatrix}
 {#1}_{#3} & {#1}_{#4} \\
 {#2}_{#3} & {#2}_{#4}
\end{vmatrix}
}

\newcommand{\DETuvwijk}[6]{
\begin{vmatrix}
 {#1}_{#4} & {#1}_{#5} & {#1}_{#6} \\
 {#2}_{#4} & {#2}_{#5} & {#2}_{#6} \\
 {#3}_{#4} & {#3}_{#5} & {#3}_{#6}
\end{vmatrix}
}

\newcommand{\DETuvwxijkl}[8]{
\begin{vmatrix}
 {#1}_{#5} & {#1}_{#6} & {#1}_{#7} & {#1}_{#8} \\
 {#2}_{#5} & {#2}_{#6} & {#2}_{#7} & {#2}_{#8} \\
 {#3}_{#5} & {#3}_{#6} & {#3}_{#7} & {#3}_{#8} \\
 {#4}_{#5} & {#4}_{#6} & {#4}_{#7} & {#4}_{#8} \\
\end{vmatrix}
}

%\newcommand{\DETuvwxyijklm}[10]{
%\begin{vmatrix}
% {#1}_{#6} & {#1}_{#7} & {#1}_{#8} & {#1}_{#9} & {#1}_{#10} \\
% {#2}_{#6} & {#2}_{#7} & {#2}_{#8} & {#2}_{#9} & {#2}_{#10} \\
% {#3}_{#6} & {#3}_{#7} & {#3}_{#8} & {#3}_{#9} & {#3}_{#10} \\
% {#4}_{#6} & {#4}_{#7} & {#4}_{#8} & {#4}_{#9} & {#4}_{#10} \\
% {#5}_{#6} & {#5}_{#7} & {#5}_{#8} & {#5}_{#9} & {#5}_{#10}
%\end{vmatrix}
%}

% R3 vector.
\newcommand{\VectorThree}[3]{
\begin{bmatrix}
 {#1} \\
 {#2} \\
 {#3}
\end{bmatrix}
}


\newcommand{\grad}[0]{\nabla}

%
% The real thing:
%

                             % The preamble begins here.
\title{} % Declares the document's title.
\author{Peeter Joot}         % Declares the author's name.
\date{ July 12, 2008 }        % Deleting this command produces today's date.

\begin{document}             % End of preamble and beginning of text.

\maketitle{}

\section{ Back to Maxwell's equations }

Having observed and demonstrated that the Lorentz transformation is a natural consequence of requiring the electromagnetic wave equation retains the
form of the wave equation under change of space and time variables that includes a velocity change in one spacial direction.

Lets step back and look at Maxwell's equations to see where the wave equation comes from.  We start with the equations in SI units:

\begin{align*}
\int_{S(\text{closed boundary of V})} \BE \cdot \ncap dA &= \inv{\epsilon_0} \int_V \rho dV \\
\int_{S(\text{any closed surface})} \BB \cdot \ncap dA &= 0 \\
\int_{C(\text{boundary of S})} \BE \cdot d\Bx &= - \int_{S} \frac{\partial \BB}{\partial t} \cdot \ncap dA \\
\int_{C(\text{boundary of S})} \BB \cdot d\Bx &= \mu_0\left(I + \epsilon_0\int_{S} \frac{\partial \BE}{\partial t} \cdot \ncap dA\right) \\
\end{align*}

As the surfaces and corresponding loops or volumes are made infinitely small, these equations (FIXME: demonstrate), can be written in differential form:

\begin{align*}
\grad \cdot \BE &= \frac{\rho}{\epsilon_0} \\
\grad \cdot \BB &= 0 \\
\grad \cross \BE &= - \frac{\partial \BB}{\partial t} \\
\grad \cross \BB &= \mu_0\left(\BJ + \epsilon_0 \frac{\partial \BE}{\partial t}\right) \\
\end{align*}

These are respectively, Gauss's Law for E, Gauss's Law for B, Faraday's Law, and the Ampere/Maxwell's Law.

This differential form can be manipulated to derive the wave equation for free space, or the wave equation with charge and current forcing terms in other space.

\subsection{ Regrouping terms for dimensional consistency. }

Derivation of the wave equation can be done nicely using geometric algebra, but first is it helpful to put these equations in a more dimensionally pleasant form.
Lets relate the dimensions of the electric and magnetic fields and the constants $\mu_0, \epsilon_0$.

From Faraday's equation we can relate the dimensions of
$\BB$, and $\BE$:

\begin{equation}
\frac{[\BE]}{[d]} = \frac{[\BB]}{[t]}
\end{equation}

We therefore see that $\BB$, and $\BE$ are related dimensionally by a velocity factor.

Looking at the dimensions of the displacement current density in the Ampere/Maxwell equation we see:

\begin{equation}
\frac{[\BB]}{[d]} = [\mu_0\epsilon_0] \frac{[\BE]}{[t]}
\end{equation}

From the two of these the dimensions of the $\mu_0\epsilon_0$ product can be seen to be:

\begin{equation}
[\mu_0\epsilon_0] = \frac{{[t]}^2}{{[d]}^2}
\end{equation}

So, we see that we have a velocity factor relating $\BE$, and $\BB$, and we also see that we have a squared velocity coefficient in Ampere/Maxwell's law.  Let's factor this out explicitly so that $\BE$ and $\BB$ take dimensionally consistent form:

\begin{align}
\tau &= \frac{t}{\sqrt{\mu_0\epsilon_0}}  \\
\grad \cdot \BE &= \frac{\rho}{\epsilon_0} \\
\grad \cdot \frac{\BB}{\sqrt{\mu_0\epsilon_0}} &= 0 \\
\grad \cross \BE &= - \frac{\partial}{\partial \tau} \frac{\BB}{\sqrt{\mu_0\epsilon_0}} \\
\grad \cross \frac{\BB}{\sqrt{\mu_0\epsilon_0}} &= \sqrt{\frac{\mu_0}{\epsilon_0}} \BJ + \frac{\partial \BE}{\partial \tau}
\end{align}

\subsection{ Refactoring the equations with the geometric product. }

Now that things are dimensionally consistent, we are ready to group these equations using the geometric product

\begin{equation}
\BA \BB = \BA \cdot \BB + \BA \wedge \BB = \BA \cdot \BB + i \BA \cross \BB
\end{equation}

where $i = \Be_1\Be_2\Be_3$ is the spatial pseudoscalar.  By grouping the divergence and curl terms for each of $\BB$, and $\BE$ we can write vector gradient equations
for each of the Electric and Magnetic fields:

\begin{align}
\grad \BE = \frac{\rho}{\epsilon_0} - i \frac{\partial}{\partial \tau} \frac{\BB}{\sqrt{\mu_0\epsilon_0}} \label{eqn:grad_e} \\
\grad \frac{\BB}{\sqrt{\mu_0\epsilon_0}} = i\sqrt{\frac{\mu_0}{\epsilon_0}} \BJ + i\frac{\partial \BE}{\partial \tau} \label{eqn:grad_b}
\end{align}

Multiplication of equation \ref{eqn:grad_b} with $i$, and adding to \ref{eqn:grad_e}, we have Maxwell's equations consolidated into:

\begin{equation}
\grad \left(\BE + i \frac{\BB}{\sqrt{\mu_0\epsilon_0}}\right) =
\left(\frac{\rho}{\epsilon_0} - \sqrt{\frac{\mu_0}{\epsilon_0}} \BJ\right)
- \frac{\partial}{\partial \tau} \left(\BE + \frac{i\BB}{\sqrt{\mu_0\epsilon_0}} \right)
\end{equation}

We see that we have a natural combined Electrodynamic field:

\begin{equation}
\BF = \BE + i \frac{\BB}{\sqrt{\mu_0\epsilon_0}},
\end{equation}

and can use this to write Maxwell's equations as:

\begin{equation}
\left(\grad + \sqrt{\mu_0\epsilon_0}\frac{\partial}{\partial t} \right) \BF = \frac{\rho}{\epsilon_0} - \sqrt{\frac{\mu_0}{\epsilon_0}} \BJ.  \label{eqn:maxwell}
\end{equation}

These are still four equations, and the originals can be recovered by taking scalar, vector, bivector and trivector parts.  However, in this
consolidated form, we are able to see the structure more easily.

\subsection{ Wave equation for light. }

In particular can do some operations, such as deriving the wave equation
with special ease.

Let's do just that, by taking the gradient of equation \ref{eqn:maxwell}:

\begin{equation*}
\grad^2 \BF + \sqrt{\mu_0\epsilon_0}\grad \frac{\partial}{\partial t} \BF =
\grad \left(\frac{\rho}{\epsilon_0} - \sqrt{\frac{\mu_0}{\epsilon_0}} \BJ\right)
\end{equation*}

%\grad \BF = - \sqrt{\mu_0\epsilon_0}\frac{\partial \BF}{\partial t} + \left(\frac{\rho}{\epsilon_0} - \sqrt{\frac{\mu_0}{\epsilon_0}} \BJ \right)

Assuming continuity suffienct for mixed partial equality, this is

\begin{align*}
\grad^2 \BF
&= - \sqrt{\mu_0\epsilon_0}\frac{\partial}{\partial t} \grad \BF + \grad \left(\frac{\rho}{\epsilon_0} - \sqrt{\frac{\mu_0}{\epsilon_0}} \BJ\right) \\
&= + \sqrt{\mu_0\epsilon_0}\frac{\partial}{\partial t}
\left(\sqrt{\mu_0\epsilon_0}\frac{\partial \BF}{\partial t} + \left(\frac{\rho}{\epsilon_0} - \sqrt{\frac{\mu_0}{\epsilon_0}} \BJ \right)\right)
 + \grad \left(\frac{\rho}{\epsilon_0} - \sqrt{\frac{\mu_0}{\epsilon_0}} \BJ\right) \\
\end{align*}

Or,

\begin{equation}
\left(\grad^2 - \mu_0\epsilon_0 \frac{\partial^2}{\partial t^2} \right) \BF =
\left(\grad + \sqrt{\mu_0\epsilon_0}\frac{\partial}{\partial t}\right)
\left(\frac{\rho}{\epsilon_0} - \sqrt{\frac{\mu_0}{\epsilon_0}} \BJ \right)
\end{equation}

Now there are a number of things that can be read out of this equation.  The first is that in a charge and current free region the electromagnetic field is described by the
wave equation:

\begin{equation}
\left(\grad^2 - \mu_0\epsilon_0 \frac{\partial^2}{\partial t^2} \right) \BF = 0.
\end{equation}

Dimensional analysis told us that $1/{\sqrt{\mu_0\epsilon_0}}$ had dimensions of velocity.  We now have a specific meaning for it, namely the
wave velocity $c$ for an electromagnetic wave in free space:

\begin{equation}
c = \inv{\sqrt{\mu_0\epsilon_0}}
\end{equation}

We can utilize this to tidy up many of the relations above, replacing $\mu_0$ with $\epsilon_0$ and $c$ since these three constants are dependent.

\begin{equation}
\BF = \BE + i c \BB
\end{equation}

\begin{equation}
\left(\grad + \partial_{ct}\right) \BF =
\inv{c \epsilon_0}\left( c \rho - \BJ \right)
\end{equation}

\begin{equation} \label{eqn:wave}
\left(\grad^2 - \partial_{ct, ct}\right) \BF =
\left(\grad + \partial_{c t}\right) \inv{c \epsilon_0}\left( c \rho - \BJ \right)
\end{equation}

Now, the left hand side of equation \ref{eqn:wave} has only vector and bivector parts.  This implies that the scalar components of the right hand side are zero.  Specifically:

\begin{equation*}
\partial_t \rho - \grad \cdot \BJ = 0
\end{equation*}

This is a statement of charge conservation, and is more easily interpretted in integral form:
%\partial_t \rho = \grad \cdot \BJ

\begin{equation}
\int_{S(\text{closed boundary of V})} \BJ \cdot \ncap dA = \frac{\partial}{\partial t} \int_V \rho dV = \frac{\partial Q_{enc}}{\partial t}
\end{equation}

The flux of the current density vector through a closed surface equals the time rate of change of the charge enclosed by that volume (ie: the current).  This could perhaps be viewed as the definition of the current density itself.  This fact would probably be more obvious if I did the math myself to demonstate exactly how to take Maxwells equations in integral form and convert those to their differential form.  In liew of having done that proof myself I can at least determine this as a side effect of a bit of math.

\end{document}               % End of document.

%
% Copyright � 2012 Peeter Joot.  All Rights Reserved.
% Licenced as described in the file LICENSE under the root directory of this GIT repository.
%

%
%
\chapter{Macroscopic Maxwell's equation}
\index{Maxwell's equation}
\label{chap:macroscopicMaxwell}
%\date{May 28, 2009.  macroscopicMaxwell.tex}

\section{Motivation}

In \citep{jackson1975cew} the macroscopic Maxwell's equations are given as

\begin{equation}\label{eqn:macroMax:MaxwellMixedFields}
\begin{aligned}
\spacegrad \cdot \BD &= 4 \pi \rho \\
\spacegrad \cross \BH - \inv{c} \PD{t}{\BD} &= \frac{4 \pi}{c} \BJ \\
\spacegrad \cross \BE + \inv{c} \PD{t}{\BB} &= 0 \\
\spacegrad \cdot \BB &= 0
\end{aligned}
\end{equation}

The \(\BH\) and \(\BD\) fields are then defined in terms of dipole, and quadrupole
fields

\begin{equation}\label{eqn:macroMax:dipoleCoordinates}
\begin{aligned}
D_\alpha &= E_\alpha + 4\pi \left( P_\alpha - \sum_\beta \PD{x_\beta}{Q'_{\alpha\beta}} + \cdots\right) \\
H_\alpha &= B_\alpha - 4\pi \left( M_\alpha + \cdots\right)
\end{aligned}
\end{equation}

Can this be put into the Geometric Algebra formulation that works so
nicely for microscopic Maxwell's equations, and if so what will it look like?

\section{Consolidation attempt}

Let us try this, writing
\begin{equation}\label{eqn:macroscopicMaxwell:20}
\begin{aligned}
\BP &= \sigma^\alpha \left( P_\alpha - \sum_\beta \PD{x_\beta}{Q'_{\alpha\beta}} + \cdots\right) \\
\BM &= \sigma^\alpha \left( M_\alpha + \cdots \right)
\end{aligned}
\end{equation}

We can then express the \(\BE\), \(\BB\) in terms of the derived fields

\begin{equation}\label{eqn:macroscopicMaxwell:40}
\begin{aligned}
\BE &= \BD - 4\pi \BP \\
\BB &= \BH + 4\pi \BM
\end{aligned}
\end{equation}

and in turn can write the macroscopic Maxwell equations \eqnref{eqn:macroMax:MaxwellMixedFields}
in terms of just the derived fields, the material properties, and the charges and currents

\begin{equation}\label{eqn:macroscopicMaxwell:60}
\begin{aligned}
\spacegrad \cdot \BD &= 4 \pi \rho \\
\spacegrad \cross \BH - \inv{c} \PD{t}{\BD} &= \frac{4 \pi}{c} \BJ \\
\spacegrad \cross \BD + \inv{c} \PD{t}{ \BH } &= 4 \pi \spacegrad \cross \BP + \frac{4\pi}{c} \PD{t}{ \BM }  \\
\spacegrad \cdot \BH &= - 4 \pi \spacegrad \cdot \BM \\
\end{aligned}
\end{equation}

Now, using \(\Ba \cross \Bb = -i (\Ba \wedge \Bb)\), we have

\begin{equation}\label{eqn:macroscopicMaxwell:80}
\begin{aligned}
\spacegrad \cdot \BD &= 4 \pi \rho \\
i \spacegrad \wedge \BH + \inv{c} \PD{t}{\BD} &= -\frac{4 \pi}{c} \BJ \\
\spacegrad \wedge \BD + \inv{c} \PD{t}{ i\BH } &= 4 \pi i \spacegrad \cross \BP + \frac{4\pi}{c} \PD{t}{ i \BM }  \\
i \spacegrad \cdot \BH &= - 4 \pi i \spacegrad \cdot \BM \\
\end{aligned}
\end{equation}

Summing these in pairs with \(\spacegrad \Ba = \spacegrad \cdot \Ba + \spacegrad \wedge \Ba\), and writing \(\PDi{(ct)}{} = \partial_0\) we have

\begin{equation}\label{eqn:macroscopicMaxwell:100}
\begin{aligned}
\spacegrad \BD + \partial_0 {i\BH } &= 4 \pi \rho + 4 \pi \spacegrad \wedge \BP + {4\pi} \partial_0 { i \BM }  \\
i \spacegrad \BH + \partial_0 {\BD} &= -\frac{4 \pi}{c} \BJ - 4 \pi i \spacegrad \cdot \BM \\
\end{aligned}
\end{equation}

Note that while had \(i\spacegrad \cdot \Ba \ne \spacegrad \cdot (i\Ba)\), and
\(i\spacegrad \wedge \Ba \ne \spacegrad \wedge (i\Ba)\)
(instead \(i\spacegrad \cdot \Ba = \spacegrad \wedge (i\Ba)\), and
\(i\spacegrad \wedge \Ba = \spacegrad \cdot (i\Ba)\)), but now that these are summed we can take advantage of the fact that the pseudoscalar \(i\)
commutes with all vectors (such as \(\spacegrad\)).  So, summing once again we have

\begin{equation}\label{eqn:macroscopicMaxwell:120}
\begin{aligned}
(\partial_0 + \spacegrad)(\BD + i\BH ) &=
\frac{4 \pi}{c} \left( c \rho - \BJ \right)
+ {4 \pi} \left( \spacegrad \wedge \BP + \partial_0 { i \BM }  - \spacegrad \wedge (i\BM) \right)
\\
\end{aligned}
\end{equation}

Finally, premultiplication by \(\gamma_0\), where \(\BJ = \sigma_k J^k = \gamma_k \gamma_0 J^k\), and \(\spacegrad = \sum_k \gamma_k \gamma_0 \partial_k\) we have

\begin{equation}\label{eqn:macroscopicMaxwell:140}
\begin{aligned}
\gamma^\mu \partial_\mu (\BD + i \BH)
&=
\frac{4 \pi}{c} \left( c \rho \gamma_0 + J^k \gamma_k \right)
+ {4 \pi \gamma_0} \left( \spacegrad \wedge \BP + \partial_0 { i \BM }  - \spacegrad \wedge (i\BM) \right)
\end{aligned}
\end{equation}

With
\begin{equation}\label{eqn:macroscopicMaxwell:160}
\begin{aligned}
J^0 &= c \rho \\
J &= \gamma_\mu J^\mu \\
\grad &= \gamma^\mu \partial_\mu \\
F &= \BD + i\BH
\end{aligned}
\end{equation}

For the remaining terms we have \(\spacegrad \wedge \BP, i\BM \in \Span\{\gamma_a \gamma_b\}\), and \(\gamma_0 \spacegrad \wedge (iM) \in \Span{ \gamma_1 \gamma_2 \gamma_3}\), so between the three of these we have a (Dirac) trivector, so it would be reasonable to write

\begin{equation}\label{eqn:macroscopicMaxwell:180}
\begin{aligned}
T &= {\gamma_0} \left( \spacegrad \wedge \BP + \partial_0 { i \BM }  - \spacegrad \wedge (i\BM) \right) \in \Span\{ \gamma_\mu \wedge \gamma_\nu \wedge \gamma_\sigma\}
\end{aligned}
\end{equation}

Putting things back together we have

\begin{equation}\label{eqn:macroscopicMaxwell:200}
\begin{aligned}
\grad F &= \frac{4\pi}{c} J + 4\pi T
\end{aligned}
\end{equation}

This has a nice symmetry, almost nicer than the original microscopic version of Maxwell's equation since we now have matched grades (vector plus trivector in the Dirac vector space) on both sides of the equation.

\subsection{Continuity equation}
\index{continuity equation}

Also observe that interestingly we still have the same continuity equation as in the microscopic case.  Application of another spacetime gradient and then selecting scalar grades we have

\begin{equation}\label{eqn:macroscopicMaxwell:220}
\begin{aligned}
\gpgradezero{ \grad \grad F } &= 4 \pi \gpgradezero{ \grad \left( \frac{J}{c} + T \right) }  \\
\grad^2 \gpgradezero{ F } &= \\
&= \frac{ 4 \pi }{c} \gpgradezero{ J } \\
&= \frac{ 4 \pi }{c} \partial_\mu J^\mu \\
\end{aligned}
\end{equation}

Since \(F\) is a Dirac bivector it has no scalar part, so this whole thing is zero by the grade selection on the LHS.  So, from the RHS we have

\begin{equation}\label{eqn:macroscopicMaxwell:240}
\begin{aligned}
0 &= \partial_\mu J^\mu \\
&= \inv{c} \PD{t}{c\rho} + \partial_k J^k \\
&= \PD{t}{\rho} + \spacegrad \cdot \BJ
\end{aligned}
\end{equation}

Despite the new trivector term in the equation due to the matter properties!

\documentclass{article}

\usepackage{amsmath}
\usepackage{mathpazo}

%
% shorthand for bold symbols, convenient for vectors and matrices
%
\newcommand{\Ba}[0]{\mathbf{a}}
\newcommand{\Bb}[0]{\mathbf{b}}
\newcommand{\Bc}[0]{\mathbf{c}}
\newcommand{\Bd}[0]{\mathbf{d}}
\newcommand{\Be}[0]{\mathbf{e}}
\newcommand{\Bf}[0]{\mathbf{f}}
\newcommand{\Bg}[0]{\mathbf{g}}
\newcommand{\Bh}[0]{\mathbf{h}}
\newcommand{\Bi}[0]{\mathbf{i}}
\newcommand{\Bj}[0]{\mathbf{j}}
\newcommand{\Bk}[0]{\mathbf{k}}
\newcommand{\Bl}[0]{\mathbf{l}}
\newcommand{\Bm}[0]{\mathbf{m}}
\newcommand{\Bn}[0]{\mathbf{n}}
\newcommand{\Bo}[0]{\mathbf{o}}
\newcommand{\Bp}[0]{\mathbf{p}}
\newcommand{\Bq}[0]{\mathbf{q}}
\newcommand{\Br}[0]{\mathbf{r}}
\newcommand{\Bs}[0]{\mathbf{s}}
\newcommand{\Bt}[0]{\mathbf{t}}
\newcommand{\Bu}[0]{\mathbf{u}}
\newcommand{\Bv}[0]{\mathbf{v}}
\newcommand{\Bw}[0]{\mathbf{w}}
\newcommand{\Bx}[0]{\mathbf{x}}
\newcommand{\By}[0]{\mathbf{y}}
\newcommand{\Bz}[0]{\mathbf{z}}
\newcommand{\BA}[0]{\mathbf{A}}
\newcommand{\BB}[0]{\mathbf{B}}
\newcommand{\BC}[0]{\mathbf{C}}
\newcommand{\BD}[0]{\mathbf{D}}
\newcommand{\BE}[0]{\mathbf{E}}
\newcommand{\BF}[0]{\mathbf{F}}
\newcommand{\BG}[0]{\mathbf{G}}
\newcommand{\BH}[0]{\mathbf{H}}
\newcommand{\BI}[0]{\mathbf{I}}
\newcommand{\BJ}[0]{\mathbf{J}}
\newcommand{\BK}[0]{\mathbf{K}}
\newcommand{\BL}[0]{\mathbf{L}}
\newcommand{\BM}[0]{\mathbf{M}}
\newcommand{\BN}[0]{\mathbf{N}}
\newcommand{\BO}[0]{\mathbf{O}}
\newcommand{\BP}[0]{\mathbf{P}}
\newcommand{\BQ}[0]{\mathbf{Q}}
\newcommand{\BR}[0]{\mathbf{R}}
\newcommand{\BS}[0]{\mathbf{S}}
\newcommand{\BT}[0]{\mathbf{T}}
\newcommand{\BU}[0]{\mathbf{U}}
\newcommand{\BV}[0]{\mathbf{V}}
\newcommand{\BW}[0]{\mathbf{W}}
\newcommand{\BX}[0]{\mathbf{X}}
\newcommand{\BY}[0]{\mathbf{Y}}
\newcommand{\BZ}[0]{\mathbf{Z}}

\newcommand{\Bzero}[0]{\mathbf{0}}
\newcommand{\Btheta}[0]{\boldsymbol{\theta}}
\newcommand{\Btau}[0]{\boldsymbol{\tau}}
\newcommand{\Bomega}[0]{\boldsymbol{\omega}}

%
% shorthand for unit vectors
%
\newcommand{\acap}[0]{\hat{\Ba}}
\newcommand{\bcap}[0]{\hat{\Bb}}
\newcommand{\ccap}[0]{\hat{\Bc}}
\newcommand{\dcap}[0]{\hat{\Bd}}
\newcommand{\ecap}[0]{\hat{\Be}}
\newcommand{\fcap}[0]{\hat{\Bf}}
\newcommand{\gcap}[0]{\hat{\Bg}}
\newcommand{\hcap}[0]{\hat{\Bh}}
\newcommand{\icap}[0]{\hat{\Bi}}
\newcommand{\jcap}[0]{\hat{\Bj}}
\newcommand{\kcap}[0]{\hat{\Bk}}
\newcommand{\lcap}[0]{\hat{\Bl}}
\newcommand{\mcap}[0]{\hat{\Bm}}
\newcommand{\ncap}[0]{\hat{\Bn}}
\newcommand{\ocap}[0]{\hat{\Bo}}
\newcommand{\pcap}[0]{\hat{\Bp}}
\newcommand{\qcap}[0]{\hat{\Bq}}
\newcommand{\rcap}[0]{\hat{\Br}}
\newcommand{\scap}[0]{\hat{\Bs}}
\newcommand{\tcap}[0]{\hat{\Bt}}
\newcommand{\ucap}[0]{\hat{\Bu}}
\newcommand{\vcap}[0]{\hat{\Bv}}
\newcommand{\wcap}[0]{\hat{\Bw}}
\newcommand{\xcap}[0]{\hat{\Bx}}
\newcommand{\ycap}[0]{\hat{\By}}
\newcommand{\zcap}[0]{\hat{\Bz}}
\newcommand{\thetacap}[0]{\hat{\Btheta}}

%
% to write R^n and C^n in a distinguishable fashion.  Perhaps change this
% to the double lined characters upon figuring out how to do so.
%
\newcommand{\C}[1]{$\mathbb{C}^{#1}$}
\newcommand{\R}[1]{$\mathbb{R}^{#1}$}

%
% various generally useful helpers
%

% derivative of #1 wrt. #2:
\newcommand{\D}[2] {\frac {d#2} {d#1}}

\newcommand{\inv}[1]{\frac{1}{#1}}
\newcommand{\cross}[0]{\times}

\newcommand{\abs}[1]{\lvert{#1}\rvert}
\newcommand{\norm}[1]{\lVert{#1}\rVert}
\newcommand{\innerprod}[2]{\langle{#1}, {#2}\rangle}
\newcommand{\dotprod}[2]{{#1} \cdot {#2}}
\newcommand{\bdotprod}[2]{\left({#1} \cdot {#2}\right)}
\newcommand{\crossprod}[2]{{#1} \cross {#2}}
\newcommand{\tripleprod}[3]{\dotprod{\left(\crossprod{#1}{#2}\right)}{#3}}

\DeclareMathOperator{\Proj}{Proj}
\DeclareMathOperator{\Span}{span}
\DeclareMathOperator{\Sgn}{sgn}
\DeclareMathOperator{\Area}{Area}
\DeclareMathOperator{\Volume}{Volume}

%
% A few miscellaneous things specific to this document
%
\newcommand{\crossop}[1]{\crossprod{#1}{}}

% R2 vector.
\newcommand{\VectorTwo}[2]{
\begin{bmatrix}
 {#1} \\
 {#2}
\end{bmatrix}
}

\newcommand{\VectorN}[1]{
\begin{bmatrix}
{#1}_1 \\
{#1}_2 \\
\vdots \\
{#1}_N \\
\end{bmatrix}
}

\newcommand{\DETuvij}[4]{
\begin{vmatrix}
 {#1}_{#3} & {#1}_{#4} \\
 {#2}_{#3} & {#2}_{#4}
\end{vmatrix}
}

\newcommand{\DETuvwijk}[6]{
\begin{vmatrix}
 {#1}_{#4} & {#1}_{#5} & {#1}_{#6} \\
 {#2}_{#4} & {#2}_{#5} & {#2}_{#6} \\
 {#3}_{#4} & {#3}_{#5} & {#3}_{#6}
\end{vmatrix}
}

\newcommand{\DETuvwxijkl}[8]{
\begin{vmatrix}
 {#1}_{#5} & {#1}_{#6} & {#1}_{#7} & {#1}_{#8} \\
 {#2}_{#5} & {#2}_{#6} & {#2}_{#7} & {#2}_{#8} \\
 {#3}_{#5} & {#3}_{#6} & {#3}_{#7} & {#3}_{#8} \\
 {#4}_{#5} & {#4}_{#6} & {#4}_{#7} & {#4}_{#8} \\
\end{vmatrix}
}

%\newcommand{\DETuvwxyijklm}[10]{
%\begin{vmatrix}
% {#1}_{#6} & {#1}_{#7} & {#1}_{#8} & {#1}_{#9} & {#1}_{#10} \\
% {#2}_{#6} & {#2}_{#7} & {#2}_{#8} & {#2}_{#9} & {#2}_{#10} \\
% {#3}_{#6} & {#3}_{#7} & {#3}_{#8} & {#3}_{#9} & {#3}_{#10} \\
% {#4}_{#6} & {#4}_{#7} & {#4}_{#8} & {#4}_{#9} & {#4}_{#10} \\
% {#5}_{#6} & {#5}_{#7} & {#5}_{#8} & {#5}_{#9} & {#5}_{#10}
%\end{vmatrix}
%}

% R3 vector.
\newcommand{\VectorThree}[3]{
\begin{bmatrix}
 {#1} \\
 {#2} \\
 {#3}
\end{bmatrix}
}


%<misc>
%
\newcommand{\Abs}[1]{{\left\lvert{#1}\right\rvert}}
\newcommand{\spacegrad}[0]{\boldsymbol{\nabla}}
\newcommand{\grad}[0]{\nabla}
\newcommand{\LL}[0]{\mathcal{L}}

% == \partial_{#1} {#2}
\newcommand{\PD}[2]{\frac{\partial {#2}}{\partial {#1}}}
% inline variant
\newcommand{\PDi}[2]{{\partial {#2}}/{\partial {#1}}}

\newcommand{\PDD}[3]{\frac{\partial^2 {#3}}{\partial {#1}\partial {#2}}}
%\newcommand{\PDd}[2]{\frac{\partial^2 {#2}}{{\partial{#1}}^2}}
\newcommand{\PDsq}[2]{\frac{\partial^2 {#2}}{(\partial {#1})^2}}

\newcommand{\Partial}[2]{\frac{\partial {#1}}{\partial {#2}}}
\DeclareMathOperator{\RejName}{Rej}
\newcommand{\Rej}[2]{\RejName_{#1}\left( {#2} \right)}
\newcommand{\Rm}[1]{\mathbb{R}^{#1}}
\newcommand{\Cm}[1]{\mathbb{C}^{#1}}
\newcommand{\conj}[0]{{*}}

%</misc>

% <grade selection>
%
\newcommand{\gpgrade}[2] {{\left\langle{{#1}}\right\rangle}_{#2}}

\newcommand{\gpgradezero}[1] {\gpgrade{#1}{}}
%\newcommand{\gpscalargrade}[1] {{\left\langle{{#1}}\right\rangle}}
%\newcommand{\gpgradezero}[1] {\gpgrade{#1}{0}}

%\newcommand{\gpgradeone}[1] {{\left\langle{{#1}}\right\rangle}_{1}}
\newcommand{\gpgradeone}[1] {\gpgrade{#1}{1}}

\newcommand{\gpgradetwo}[1] {\gpgrade{#1}{2}}
\newcommand{\gpgradethree}[1] {\gpgrade{#1}{3}}
\newcommand{\gpgradefour}[1] {\gpgrade{#1}{4}}
%
% </grade selection>



\newcommand{\adot}[0]{{\dot{a}}}
\newcommand{\bdot}[0]{{\dot{b}}}
% taken for centered dot:
%\newcommand{\cdot}[0]{{\dot{c}}}
%\newcommand{\ddot}[0]{{\dot{d}}}
\newcommand{\edot}[0]{{\dot{e}}}
\newcommand{\fdot}[0]{{\dot{f}}}
\newcommand{\gdot}[0]{{\dot{g}}}
\newcommand{\hdot}[0]{{\dot{h}}}
\newcommand{\idot}[0]{{\dot{i}}}
\newcommand{\jdot}[0]{{\dot{j}}}
\newcommand{\kdot}[0]{{\dot{k}}}
\newcommand{\ldot}[0]{{\dot{l}}}
\newcommand{\mdot}[0]{{\dot{m}}}
\newcommand{\ndot}[0]{{\dot{n}}}
%\newcommand{\odot}[0]{{\dot{o}}}
\newcommand{\pdot}[0]{{\dot{p}}}
\newcommand{\qdot}[0]{{\dot{q}}}
\newcommand{\rdot}[0]{{\dot{r}}}
\newcommand{\sdot}[0]{{\dot{s}}}
\newcommand{\tdot}[0]{{\dot{t}}}
\newcommand{\udot}[0]{{\dot{u}}}
\newcommand{\vdot}[0]{{\dot{v}}}
\newcommand{\wdot}[0]{{\dot{w}}}
\newcommand{\xdot}[0]{{\dot{x}}}
\newcommand{\ydot}[0]{{\dot{y}}}
\newcommand{\zdot}[0]{{\dot{z}}}
\newcommand{\addot}[0]{{\ddot{a}}}
\newcommand{\bddot}[0]{{\ddot{b}}}
\newcommand{\cddot}[0]{{\ddot{c}}}
%\newcommand{\dddot}[0]{{\ddot{d}}}
\newcommand{\eddot}[0]{{\ddot{e}}}
\newcommand{\fddot}[0]{{\ddot{f}}}
\newcommand{\gddot}[0]{{\ddot{g}}}
\newcommand{\hddot}[0]{{\ddot{h}}}
\newcommand{\iddot}[0]{{\ddot{i}}}
\newcommand{\jddot}[0]{{\ddot{j}}}
\newcommand{\kddot}[0]{{\ddot{k}}}
\newcommand{\lddot}[0]{{\ddot{l}}}
\newcommand{\mddot}[0]{{\ddot{m}}}
\newcommand{\nddot}[0]{{\ddot{n}}}
\newcommand{\oddot}[0]{{\ddot{o}}}
\newcommand{\pddot}[0]{{\ddot{p}}}
\newcommand{\qddot}[0]{{\ddot{q}}}
\newcommand{\rddot}[0]{{\ddot{r}}}
\newcommand{\sddot}[0]{{\ddot{s}}}
\newcommand{\tddot}[0]{{\ddot{t}}}
\newcommand{\uddot}[0]{{\ddot{u}}}
\newcommand{\vddot}[0]{{\ddot{v}}}
\newcommand{\wddot}[0]{{\ddot{w}}}
\newcommand{\xddot}[0]{{\ddot{x}}}
\newcommand{\yddot}[0]{{\ddot{y}}}
\newcommand{\zddot}[0]{{\ddot{z}}}

%<bold and dot greek symbols>
%

\newcommand{\Deltadot}[0]{{\dot{\Delta}}}
\newcommand{\Gammadot}[0]{{\dot{\Gamma}}}
\newcommand{\Lambdadot}[0]{{\dot{\Lambda}}}
\newcommand{\Omegadot}[0]{{\dot{\Omega}}}
\newcommand{\Phidot}[0]{{\dot{\Phi}}}
\newcommand{\Pidot}[0]{{\dot{\Pi}}}
\newcommand{\Psidot}[0]{{\dot{\Psi}}}
\newcommand{\Sigmadot}[0]{{\dot{\Sigma}}}
\newcommand{\Thetadot}[0]{{\dot{\Theta}}}
\newcommand{\Upsilondot}[0]{{\dot{\Upsilon}}}
\newcommand{\Xidot}[0]{{\dot{\Xi}}}
\newcommand{\alphadot}[0]{{\dot{\alpha}}}
\newcommand{\betadot}[0]{{\dot{\beta}}}
\newcommand{\chidot}[0]{{\dot{\chi}}}
\newcommand{\deltadot}[0]{{\dot{\delta}}}
\newcommand{\epsilondot}[0]{{\dot{\epsilon}}}
\newcommand{\etadot}[0]{{\dot{\eta}}}
\newcommand{\gammadot}[0]{{\dot{\gamma}}}
\newcommand{\kappadot}[0]{{\dot{\kappa}}}
\newcommand{\lambdadot}[0]{{\dot{\lambda}}}
\newcommand{\mudot}[0]{{\dot{\mu}}}
\newcommand{\nudot}[0]{{\dot{\nu}}}
\newcommand{\omegadot}[0]{{\dot{\omega}}}
\newcommand{\phidot}[0]{{\dot{\phi}}}
\newcommand{\pidot}[0]{{\dot{\pi}}}
\newcommand{\psidot}[0]{{\dot{\psi}}}
\newcommand{\rhodot}[0]{{\dot{\rho}}}
\newcommand{\sigmadot}[0]{{\dot{\sigma}}}
\newcommand{\taudot}[0]{{\dot{\tau}}}
\newcommand{\thetadot}[0]{{\dot{\theta}}}
\newcommand{\upsilondot}[0]{{\dot{\upsilon}}}
\newcommand{\varepsilondot}[0]{{\dot{\varepsilon}}}
\newcommand{\varphidot}[0]{{\dot{\varphi}}}
\newcommand{\varpidot}[0]{{\dot{\varpi}}}
\newcommand{\varrhodot}[0]{{\dot{\varrho}}}
\newcommand{\varsigmadot}[0]{{\dot{\varsigma}}}
\newcommand{\varthetadot}[0]{{\dot{\vartheta}}}
\newcommand{\xidot}[0]{{\dot{\xi}}}
\newcommand{\zetadot}[0]{{\dot{\zeta}}}

\newcommand{\Deltaddot}[0]{{\ddot{\Delta}}}
\newcommand{\Gammaddot}[0]{{\ddot{\Gamma}}}
\newcommand{\Lambdaddot}[0]{{\ddot{\Lambda}}}
\newcommand{\Omegaddot}[0]{{\ddot{\Omega}}}
\newcommand{\Phiddot}[0]{{\ddot{\Phi}}}
\newcommand{\Piddot}[0]{{\ddot{\Pi}}}
\newcommand{\Psiddot}[0]{{\ddot{\Psi}}}
\newcommand{\Sigmaddot}[0]{{\ddot{\Sigma}}}
\newcommand{\Thetaddot}[0]{{\ddot{\Theta}}}
\newcommand{\Upsilonddot}[0]{{\ddot{\Upsilon}}}
\newcommand{\Xiddot}[0]{{\ddot{\Xi}}}
\newcommand{\alphaddot}[0]{{\ddot{\alpha}}}
\newcommand{\betaddot}[0]{{\ddot{\beta}}}
\newcommand{\chiddot}[0]{{\ddot{\chi}}}
\newcommand{\deltaddot}[0]{{\ddot{\delta}}}
\newcommand{\epsilonddot}[0]{{\ddot{\epsilon}}}
\newcommand{\etaddot}[0]{{\ddot{\eta}}}
\newcommand{\gammaddot}[0]{{\ddot{\gamma}}}
\newcommand{\kappaddot}[0]{{\ddot{\kappa}}}
\newcommand{\lambdaddot}[0]{{\ddot{\lambda}}}
\newcommand{\muddot}[0]{{\ddot{\mu}}}
\newcommand{\nuddot}[0]{{\ddot{\nu}}}
\newcommand{\omegaddot}[0]{{\ddot{\omega}}}
\newcommand{\phiddot}[0]{{\ddot{\phi}}}
\newcommand{\piddot}[0]{{\ddot{\pi}}}
\newcommand{\psiddot}[0]{{\ddot{\psi}}}
\newcommand{\rhoddot}[0]{{\ddot{\rho}}}
\newcommand{\sigmaddot}[0]{{\ddot{\sigma}}}
\newcommand{\tauddot}[0]{{\ddot{\tau}}}
\newcommand{\thetaddot}[0]{{\ddot{\theta}}}
\newcommand{\upsilonddot}[0]{{\ddot{\upsilon}}}
\newcommand{\varepsilonddot}[0]{{\ddot{\varepsilon}}}
\newcommand{\varphiddot}[0]{{\ddot{\varphi}}}
\newcommand{\varpiddot}[0]{{\ddot{\varpi}}}
\newcommand{\varrhoddot}[0]{{\ddot{\varrho}}}
\newcommand{\varsigmaddot}[0]{{\ddot{\varsigma}}}
\newcommand{\varthetaddot}[0]{{\ddot{\vartheta}}}
\newcommand{\xiddot}[0]{{\ddot{\xi}}}
\newcommand{\zetaddot}[0]{{\ddot{\zeta}}}

\newcommand{\BDelta}[0]{\boldsymbol{\Delta}}
\newcommand{\BGamma}[0]{\boldsymbol{\Gamma}}
\newcommand{\BLambda}[0]{\boldsymbol{\Lambda}}
\newcommand{\BOmega}[0]{\boldsymbol{\Omega}}
\newcommand{\BPhi}[0]{\boldsymbol{\Phi}}
\newcommand{\BPi}[0]{\boldsymbol{\Pi}}
\newcommand{\BPsi}[0]{\boldsymbol{\Psi}}
\newcommand{\BSigma}[0]{\boldsymbol{\Sigma}}
\newcommand{\BTheta}[0]{\boldsymbol{\Theta}}
\newcommand{\BUpsilon}[0]{\boldsymbol{\Upsilon}}
\newcommand{\BXi}[0]{\boldsymbol{\Xi}}
\newcommand{\Balpha}[0]{\boldsymbol{\alpha}}
\newcommand{\Bbeta}[0]{\boldsymbol{\beta}}
\newcommand{\Bchi}[0]{\boldsymbol{\chi}}
\newcommand{\Bdelta}[0]{\boldsymbol{\delta}}
\newcommand{\Bepsilon}[0]{\boldsymbol{\epsilon}}
\newcommand{\Beta}[0]{\boldsymbol{\eta}}
\newcommand{\Bgamma}[0]{\boldsymbol{\gamma}}
\newcommand{\Bkappa}[0]{\boldsymbol{\kappa}}
\newcommand{\Blambda}[0]{\boldsymbol{\lambda}}
\newcommand{\Bmu}[0]{\boldsymbol{\mu}}
\newcommand{\Bnu}[0]{\boldsymbol{\nu}}
%\newcommand{\Bomega}[0]{\boldsymbol{\omega}}
\newcommand{\Bphi}[0]{\boldsymbol{\phi}}
\newcommand{\Bpi}[0]{\boldsymbol{\pi}}
\newcommand{\Bpsi}[0]{\boldsymbol{\psi}}
\newcommand{\Brho}[0]{\boldsymbol{\rho}}
\newcommand{\Bsigma}[0]{\boldsymbol{\sigma}}
%\newcommand{\Btau}[0]{\boldsymbol{\tau}}
%\newcommand{\Btheta}[0]{\boldsymbol{\theta}}
\newcommand{\Bupsilon}[0]{\boldsymbol{\upsilon}}
\newcommand{\Bvarepsilon}[0]{\boldsymbol{\varepsilon}}
\newcommand{\Bvarphi}[0]{\boldsymbol{\varphi}}
\newcommand{\Bvarpi}[0]{\boldsymbol{\varpi}}
\newcommand{\Bvarrho}[0]{\boldsymbol{\varrho}}
\newcommand{\Bvarsigma}[0]{\boldsymbol{\varsigma}}
\newcommand{\Bvartheta}[0]{\boldsymbol{\vartheta}}
\newcommand{\Bxi}[0]{\boldsymbol{\xi}}
\newcommand{\Bzeta}[0]{\boldsymbol{\zeta}}
%
%</bold and dot greek symbols>
%<infrequent>
%
%\newcommand{\AreaOp}[1]{\AName_{#1}}
%\newcommand{\Babs}[0]{\abs{\BB}}
%\newcommand{\Bcap}[0]{\hat{\BB}}
%\newcommand{\BrPrimeRej}[0]{\rcap(\rcap \wedge \Br')}
%\newcommand{\CA}[0]{\mathcal{A}}
%\newcommand{\Cos}[1]{\cos{\left({#1}\right)}}
%\newcommand{\Det}[1] {\abs{#1}}
%\newcommand{\Dsq}[2] {\frac {\partial^2 {#1}} {\partial {#2}^2}}
%\newcommand{\Exp}[1]{\exp{\left({#1}\right)}}
%\newcommand{\Norm}[1]{\left\lVert{#1}\right\rVert}
%\newcommand{\Sin}[1]{\sin{\left({#1}\right)}}
%\newcommand{\T}[0]{\text{T}}
%\newcommand{\VolumeOp}[1]{\VName_{#1}}
%\newcommand{\agrad}[0]{\Ba \cdot \nabla}
%\newcommand{\alphacap}[0]{\hat{\boldsymbol{\alpha}}}
%\newcommand{\Fcap}[0]{\hat{\BF}}
%\newcommand{\bithree}[0]{{\Bi}_3}
%\newcommand{\bxa}[0]{\Bx\Ba}
%\newcommand{\coordvec}[2]{
%\newcommand{\costheta}[0]{\acap \cdot \xcap}
%\newcommand{\ddt}[1]{\ddot{#1}}
%\newcommand{\ddu}[1] {\frac {d{#1}} {du}}
%\newcommand{\dsqxj}[2] {\frac {\partial^2 {#1}} {\partial {x_{#2}}^2}}
%\newcommand{\dtheta}[1]{\frac{d {#1}}{d \theta}}
%\newcommand{\dt}[1]{\dot{#1}}
%\newcommand{\dt}[1]{\frac{d {#1}}{dt}}
%\newcommand{\dxj}[2] {\frac {\partial {#1}} {\partial {x_{#2}}}}
%\newcommand{\halfPhi}[0]{\frac{\phi}{2}}
%\newcommand{\half}[0]{\inv{2}}
%\newcommand{\inv}[1]{\frac{1}{#1}}
%\newcommand{\laplacian}[0]{\nabla^2}
%\newcommand{\matrixoftx}[3]{
%\newcommand{\nrrp}[0]{\norm{\rcap \wedge \Br'}}
%\newcommand{\oiint}{\bigcirc \hspace{-1.4em} \int \hspace{-.8em} \int}
%\newcommand{\transpose}[1]{{#1}^{\text{T}}}
%\newcommand{\transpose}[1]{{{#1}^{\TextTranspose}}}
%\newcommand{\transpose}[1]{{{#1}^{\text{T}}}}
%\newcommand{\barA}[0]{\bar{A}}
%\newcommand{\qbar}[0]{\bar{q}}
%\newcommand{\qdotbar}[0]{\dot{\bar{q}}}
%
%</infrequent>





\usepackage[bookmarks=true]{hyperref}

\usepackage{color,cite,graphicx}
   % use colour in the document, put your citations as [1-4]
   % rather than [1,2,3,4] (it looks nicer, and the extended LaTeX2e
   % graphics package. 
\usepackage{latexsym,amssymb,epsf} % don't remember if these are
   % needed, but their inclusion can't do any damage


\title{ Expressing wave equation exponential solutions using four vectors. }
\author{Peeter Joot}
\date{ Nov 30, 2008.  Last Revision: $Date: 2008/12/01 05:08:35 $ }

\begin{document}

\maketitle{}

%\tableofcontents
%\section{}

For the unforced wave equation in 3D one wants solutions to

\begin{align}\label{eqn:waveEquation}
\left( \inv{\Bv^2} \partial_{tt} - \sum_{j=1}^3 \partial_{jj}\right) \phi = 0
\end{align}

For the single spatial variable case one can verify that 
$\phi = f( \Bx \pm \Abs{\Bv} t)$ is a solution for any function $f$.  In particular $\phi = \exp(i (\pm \Abs{\Bv} t + x))$ is a solution.  Similarily
$\phi = \exp(i (\pm \Abs{\Bv} t + \kcap \cdot \Bx))$ is a solution in the 3D case.

Can the relativistic four vector notation be used to put this in a more symmetric form with respect to time and position?  For the four
vector

\begin{align*}
x = x^\mu \gamma_\mu
\end{align*}

Lets try the following as a possible solution to \ref{eqn:waveEquation}

\begin{align*}
\phi = \exp(i k \cdot x)
\end{align*}

verifying that this can be a solution, and determining the constraints required on the four vector $k$.

Observe that

\begin{align*}
x \cdot k = x^\mu k_\mu
\end{align*}

so

\begin{align*}
\phi_\mu &= i k_\mu \\
\phi_{\mu\mu} &= (i k_\mu)^2 \phi = -(k_\mu)^2 \phi
\end{align*}

Since $\partial_t = c\partial_0$, we have $\phi_tt = c^2 \phi_{00}$, and

\begin{align*}
\left( \inv{\Bv^2} \partial_{tt} - \sum_{j=1}^3 \partial_{jj}\right) \phi &= 
\left( -\inv{\Bv^2} c^2 {k_0}^2 - \sum_{j=1}^3 -(k_j)^2\right) \phi \\ 
\end{align*}

For equality with zero, and $\Bbeta = \Bv/c$, we require 

\begin{align*}
\Bbeta^2 = \frac{(k_0)^2}{\sum_j (k_j)^2}
\end{align*}

Now want the components of $k = k_\mu \gamma^\mu$ in terms of $k$ directly.  First

\begin{align*}
k_0 = k \cdot \gamma_0
\end{align*}

The spacetime relative vector for $k$ is

\begin{align*}
\Bk &= k \wedge \gamma_0 = \sum k_\mu \gamma^\mu \wedge \gamma_0 = (\gamma_1)^2 \sum_j k_j \sigma_j \\
\Bk^2 &= (\pm 1)^2 \sum_j (k_j)^2 
\end{align*}

So the constraint on the four vector parameter $k$ is 
\begin{align*}
\Bbeta^2
&= \frac{(k_0)^2}{\sum_j (k_j)^2} \\
&= \frac{(k \cdot \gamma_0)^2}{(k \wedge \gamma_0)^2} \\
\end{align*}

It is interesting to compare this to the relative spacetime bivector for $x$

\begin{align*}
v &= \frac{dx}{d\tau} = c \frac{dt}{d\tau} \gamma_0 + \frac{dx^i}{d\tau} \gamma_i \\
v \cdot \gamma^0 &= \frac{dx}{d\tau} \cdot \gamma^0 = c \frac{dt}{d\tau} \\
v \wedge \gamma_0 &= \frac{dx}{d\tau} \wedge \gamma_0 \\
&= \frac{dx^i}{d\tau} \sigma_i \\
&= \frac{dx^i}{dt} \frac{dt}{d\tau} \sigma_i \\
\end{align*}

\begin{align*}
\Bv/c
&= \frac{d (x^i \sigma_i) }{dt} \\
&= \frac{v \wedge \gamma_0}{ v \cdot \gamma^0 }
\end{align*}

So, for $\phi = \exp(i k \cdot x)$ to be a solution to the wave equation for a wave travelling with velocity $\Abs{\Bv}$, the constraint on k
in terms of proper velocity $v$ is

\begin{align*}
\Abs{\frac{k \wedge \gamma_0}{ k \cdot \gamma^0 }}^{-1} &=
\Abs{\frac{v \wedge \gamma_0}{ v \cdot \gamma^0 }}
\end{align*}

So we see the relative spacetime vector of $k$ has an inverse relationship with the relative spacetime velocity vector of $x$.

%\bibliographystyle{plainnat}
%\bibliography{myrefs}

\end{document}

\chapter{Gaussian Surface invariance for radial field.}
\date{ November 22, 2008.  $RCSfile: gaussianSurface.tex,v $ Last $Revision: 1.8 $ $Date: 2009/06/11 16:45:58 $ }

\section{Flux independence of surface.}

\begin{figure}[htp]
\centering
\includegraphics[totalheight=0.4\textheight]{surface_flux_element}
\caption{fig:Flux through tilted spherical surface element.}
\label{fig:surface_flux_element}
\end{figure}

In \cite{purcell1963eam}, section $1.10$ is a demonstration that the flux
through any closed surface is the same as that through a sphere.

A similar demonstration of the same is possible using a spherical polar basis
$\{\rcap, \thetacap, \phicap\}$ with an element of surface area that is
tilted slightly as illustrated in figure \ref{fig:surface_flux_element}.

The tangential surface on the sphere at radius $r$ will have bivector

\begin{align}
d\BA_r = r^2 d\theta d\phi \thetacap\phicap
\end{align}

where $d\theta$, and $d\phi$ are the subtended angles (should have put them in the figure).

Now, as in the figure we want to compute the bivector for the tilted surface at radius $R$.  The vector $\Bu$ in the figure is required.
This is $\rcap R + R d\theta \thetacap - \rcap(R + dr)$, so the bivector for that area element is

\begin{align*}
\left(R \rcap + R d\theta \thetacap - (R + dr) \rcap \right) \wedge {R d\theta \phicap} 
&= \left(R d\theta \thetacap - dr \rcap \right) \wedge {R d\phi \phicap} \\
\end{align*}

For
\begin{align}
d\BA_R = R^2 d\theta d\phi \thetacap \phicap - R dr d\phi \rcap \phicap
\end{align}

Now normal area elements can be calculated by multiplication with a \R{3} pseudoscalar such as $I = \rcap \thetacap \phicap$.

\begin{align*}
\ncap_r \Abs{d\BA_r}
&= r^2 d\theta d\phi \rcap \thetacap \phicap \thetacap\phicap \\
&= -r^2 d\theta d\phi \rcap \\
\end{align*}

And

\begin{align*}
\ncap_R \Abs{d\BA_R}
&= \rcap \thetacap \phicap \left( R^2 d\theta d\phi \thetacap \phicap - R dr d\phi \rcap \phicap \right) \\
&= - R^2 d\theta d\phi \rcap - R dr d\phi \thetacap
\end{align*}

Calculating $\BE \cdot \ncap dA$ for the spherical surface element at radius $r$ we have

\begin{align*}
\BE(r) \cdot \ncap_r \Abs{d\BA_r}
&= \inv{4 \pi \epsilon_0 r^2} q \rcap \cdot (-r^2 d\theta d\phi \rcap) \\
&= \frac{-d\theta d\phi q}{4 \pi \epsilon_0}
\end{align*}

and for the tilted surface at $R$ 

\begin{align*}
\BE(R) \cdot \ncap_R \Abs{d\BA_R}
&= \frac{q}{4 \pi \epsilon_0 R^2} \rcap \cdot \left(- R^2 d\theta d\phi \rcap - R dr d\phi \thetacap \right) \\
&= \frac{-d\theta d\phi q}{4 \pi \epsilon_0}
\end{align*}

The $\thetacap$ component of the surface normal has no contribution to the flux since it is perpendicular to the outwards ($\rcap$ facing) field.  Here the particular normal to the surface happened to be inwards facing due to choice of the pseudoscalar, but because the normals chosen in each case had the same orientation this doesn't make a difference to the equivalence result.

\subsection{Suggests dual form of Gauss's law can be natural. }

The fact that the bivector area elements work well to describe the surface
can also be used to write Gauss's law in an alternate form.  Let $\ncap dA = -I d\BA$

\begin{align*}
\BE \cdot \ncap dA
&= -\BE \cdot (I d\BA) \\
&= \frac{-1}{2} ( \BE I d\BA + I d\BA \BE ) \\
&= \frac{-I}{2} ( \BE d\BA + d\BA \BE ) \\
&= -I ( \BE \wedge d\BA )
\end{align*}

So for

\begin{align*}
\int \BE \cdot \ncap dA
&= \int \frac{\rho}{\epsilon_0} dV
\end{align*}

with $d\BV = I dV$, we have Gauss's law in dual form:

\begin{align*}
\int \BE \wedge d\BA &= \int \frac{\rho}{\epsilon_0} d\BV
\end{align*}

Writing Gauss's law in this form it becomes almost obvious that we can 
deform the surface without changing the flux, since all the non-tangential
surface elements will have an $\rcap$ factor and thus produce a zero
once wedged with the radial field.


\documentclass{article}

\usepackage{amsmath}
\usepackage{mathpazo}

%
% shorthand for bold symbols, convenient for vectors and matrices
%
\newcommand{\Ba}[0]{\mathbf{a}}
\newcommand{\Bb}[0]{\mathbf{b}}
\newcommand{\Bc}[0]{\mathbf{c}}
\newcommand{\Bd}[0]{\mathbf{d}}
\newcommand{\Be}[0]{\mathbf{e}}
\newcommand{\Bf}[0]{\mathbf{f}}
\newcommand{\Bg}[0]{\mathbf{g}}
\newcommand{\Bh}[0]{\mathbf{h}}
\newcommand{\Bi}[0]{\mathbf{i}}
\newcommand{\Bj}[0]{\mathbf{j}}
\newcommand{\Bk}[0]{\mathbf{k}}
\newcommand{\Bl}[0]{\mathbf{l}}
\newcommand{\Bm}[0]{\mathbf{m}}
\newcommand{\Bn}[0]{\mathbf{n}}
\newcommand{\Bo}[0]{\mathbf{o}}
\newcommand{\Bp}[0]{\mathbf{p}}
\newcommand{\Bq}[0]{\mathbf{q}}
\newcommand{\Br}[0]{\mathbf{r}}
\newcommand{\Bs}[0]{\mathbf{s}}
\newcommand{\Bt}[0]{\mathbf{t}}
\newcommand{\Bu}[0]{\mathbf{u}}
\newcommand{\Bv}[0]{\mathbf{v}}
\newcommand{\Bw}[0]{\mathbf{w}}
\newcommand{\Bx}[0]{\mathbf{x}}
\newcommand{\By}[0]{\mathbf{y}}
\newcommand{\Bz}[0]{\mathbf{z}}
\newcommand{\BA}[0]{\mathbf{A}}
\newcommand{\BB}[0]{\mathbf{B}}
\newcommand{\BC}[0]{\mathbf{C}}
\newcommand{\BD}[0]{\mathbf{D}}
\newcommand{\BE}[0]{\mathbf{E}}
\newcommand{\BF}[0]{\mathbf{F}}
\newcommand{\BG}[0]{\mathbf{G}}
\newcommand{\BH}[0]{\mathbf{H}}
\newcommand{\BI}[0]{\mathbf{I}}
\newcommand{\BJ}[0]{\mathbf{J}}
\newcommand{\BK}[0]{\mathbf{K}}
\newcommand{\BL}[0]{\mathbf{L}}
\newcommand{\BM}[0]{\mathbf{M}}
\newcommand{\BN}[0]{\mathbf{N}}
\newcommand{\BO}[0]{\mathbf{O}}
\newcommand{\BP}[0]{\mathbf{P}}
\newcommand{\BQ}[0]{\mathbf{Q}}
\newcommand{\BR}[0]{\mathbf{R}}
\newcommand{\BS}[0]{\mathbf{S}}
\newcommand{\BT}[0]{\mathbf{T}}
\newcommand{\BU}[0]{\mathbf{U}}
\newcommand{\BV}[0]{\mathbf{V}}
\newcommand{\BW}[0]{\mathbf{W}}
\newcommand{\BX}[0]{\mathbf{X}}
\newcommand{\BY}[0]{\mathbf{Y}}
\newcommand{\BZ}[0]{\mathbf{Z}}

\newcommand{\Bzero}[0]{\mathbf{0}}
\newcommand{\Btheta}[0]{\boldsymbol{\theta}}
\newcommand{\Btau}[0]{\boldsymbol{\tau}}
\newcommand{\Bomega}[0]{\boldsymbol{\omega}}

%
% shorthand for unit vectors
%
\newcommand{\acap}[0]{\hat{\Ba}}
\newcommand{\bcap}[0]{\hat{\Bb}}
\newcommand{\ccap}[0]{\hat{\Bc}}
\newcommand{\dcap}[0]{\hat{\Bd}}
\newcommand{\ecap}[0]{\hat{\Be}}
\newcommand{\fcap}[0]{\hat{\Bf}}
\newcommand{\gcap}[0]{\hat{\Bg}}
\newcommand{\hcap}[0]{\hat{\Bh}}
\newcommand{\icap}[0]{\hat{\Bi}}
\newcommand{\jcap}[0]{\hat{\Bj}}
\newcommand{\kcap}[0]{\hat{\Bk}}
\newcommand{\lcap}[0]{\hat{\Bl}}
\newcommand{\mcap}[0]{\hat{\Bm}}
\newcommand{\ncap}[0]{\hat{\Bn}}
\newcommand{\ocap}[0]{\hat{\Bo}}
\newcommand{\pcap}[0]{\hat{\Bp}}
\newcommand{\qcap}[0]{\hat{\Bq}}
\newcommand{\rcap}[0]{\hat{\Br}}
\newcommand{\scap}[0]{\hat{\Bs}}
\newcommand{\tcap}[0]{\hat{\Bt}}
\newcommand{\ucap}[0]{\hat{\Bu}}
\newcommand{\vcap}[0]{\hat{\Bv}}
\newcommand{\wcap}[0]{\hat{\Bw}}
\newcommand{\xcap}[0]{\hat{\Bx}}
\newcommand{\ycap}[0]{\hat{\By}}
\newcommand{\zcap}[0]{\hat{\Bz}}
\newcommand{\thetacap}[0]{\hat{\Btheta}}

%
% to write R^n and C^n in a distinguishable fashion.  Perhaps change this
% to the double lined characters upon figuring out how to do so.
%
\newcommand{\C}[1]{$\mathbb{C}^{#1}$}
\newcommand{\R}[1]{$\mathbb{R}^{#1}$}

%
% various generally useful helpers
%

% derivative of #1 wrt. #2:
\newcommand{\D}[2] {\frac {d#2} {d#1}}

\newcommand{\inv}[1]{\frac{1}{#1}}
\newcommand{\cross}[0]{\times}

\newcommand{\abs}[1]{\lvert{#1}\rvert}
\newcommand{\norm}[1]{\lVert{#1}\rVert}
\newcommand{\innerprod}[2]{\langle{#1}, {#2}\rangle}
\newcommand{\dotprod}[2]{{#1} \cdot {#2}}
\newcommand{\bdotprod}[2]{\left({#1} \cdot {#2}\right)}
\newcommand{\crossprod}[2]{{#1} \cross {#2}}
\newcommand{\tripleprod}[3]{\dotprod{\left(\crossprod{#1}{#2}\right)}{#3}}

\DeclareMathOperator{\Proj}{Proj}
\DeclareMathOperator{\Span}{span}
\DeclareMathOperator{\Sgn}{sgn}
\DeclareMathOperator{\Area}{Area}
\DeclareMathOperator{\Volume}{Volume}

%
% A few miscellaneous things specific to this document
%
\newcommand{\crossop}[1]{\crossprod{#1}{}}

% R2 vector.
\newcommand{\VectorTwo}[2]{
\begin{bmatrix}
 {#1} \\
 {#2}
\end{bmatrix}
}

\newcommand{\VectorN}[1]{
\begin{bmatrix}
{#1}_1 \\
{#1}_2 \\
\vdots \\
{#1}_N \\
\end{bmatrix}
}

\newcommand{\DETuvij}[4]{
\begin{vmatrix}
 {#1}_{#3} & {#1}_{#4} \\
 {#2}_{#3} & {#2}_{#4}
\end{vmatrix}
}

\newcommand{\DETuvwijk}[6]{
\begin{vmatrix}
 {#1}_{#4} & {#1}_{#5} & {#1}_{#6} \\
 {#2}_{#4} & {#2}_{#5} & {#2}_{#6} \\
 {#3}_{#4} & {#3}_{#5} & {#3}_{#6}
\end{vmatrix}
}

\newcommand{\DETuvwxijkl}[8]{
\begin{vmatrix}
 {#1}_{#5} & {#1}_{#6} & {#1}_{#7} & {#1}_{#8} \\
 {#2}_{#5} & {#2}_{#6} & {#2}_{#7} & {#2}_{#8} \\
 {#3}_{#5} & {#3}_{#6} & {#3}_{#7} & {#3}_{#8} \\
 {#4}_{#5} & {#4}_{#6} & {#4}_{#7} & {#4}_{#8} \\
\end{vmatrix}
}

%\newcommand{\DETuvwxyijklm}[10]{
%\begin{vmatrix}
% {#1}_{#6} & {#1}_{#7} & {#1}_{#8} & {#1}_{#9} & {#1}_{#10} \\
% {#2}_{#6} & {#2}_{#7} & {#2}_{#8} & {#2}_{#9} & {#2}_{#10} \\
% {#3}_{#6} & {#3}_{#7} & {#3}_{#8} & {#3}_{#9} & {#3}_{#10} \\
% {#4}_{#6} & {#4}_{#7} & {#4}_{#8} & {#4}_{#9} & {#4}_{#10} \\
% {#5}_{#6} & {#5}_{#7} & {#5}_{#8} & {#5}_{#9} & {#5}_{#10}
%\end{vmatrix}
%}

% R3 vector.
\newcommand{\VectorThree}[3]{
\begin{bmatrix}
 {#1} \\
 {#2} \\
 {#3}
\end{bmatrix}
}


%<misc>
%
\newcommand{\Abs}[1]{{\left\lvert{#1}\right\rvert}}
\newcommand{\spacegrad}[0]{\boldsymbol{\nabla}}
\newcommand{\grad}[0]{\nabla}
\newcommand{\LL}[0]{\mathcal{L}}

% == \partial_{#1} {#2}
\newcommand{\PD}[2]{\frac{\partial {#2}}{\partial {#1}}}
% inline variant
\newcommand{\PDi}[2]{{\partial {#2}}/{\partial {#1}}}

\newcommand{\PDD}[3]{\frac{\partial^2 {#3}}{\partial {#1}\partial {#2}}}
%\newcommand{\PDd}[2]{\frac{\partial^2 {#2}}{{\partial{#1}}^2}}
\newcommand{\PDsq}[2]{\frac{\partial^2 {#2}}{(\partial {#1})^2}}

\newcommand{\Partial}[2]{\frac{\partial {#1}}{\partial {#2}}}
\DeclareMathOperator{\RejName}{Rej}
\newcommand{\Rej}[2]{\RejName_{#1}\left( {#2} \right)}
\newcommand{\Rm}[1]{\mathbb{R}^{#1}}
\newcommand{\Cm}[1]{\mathbb{C}^{#1}}
\newcommand{\conj}[0]{{*}}

%</misc>

% <grade selection>
%
\newcommand{\gpgrade}[2] {{\left\langle{{#1}}\right\rangle}_{#2}}

\newcommand{\gpgradezero}[1] {\gpgrade{#1}{}}
%\newcommand{\gpscalargrade}[1] {{\left\langle{{#1}}\right\rangle}}
%\newcommand{\gpgradezero}[1] {\gpgrade{#1}{0}}

%\newcommand{\gpgradeone}[1] {{\left\langle{{#1}}\right\rangle}_{1}}
\newcommand{\gpgradeone}[1] {\gpgrade{#1}{1}}

\newcommand{\gpgradetwo}[1] {\gpgrade{#1}{2}}
\newcommand{\gpgradethree}[1] {\gpgrade{#1}{3}}
\newcommand{\gpgradefour}[1] {\gpgrade{#1}{4}}
%
% </grade selection>



\newcommand{\adot}[0]{{\dot{a}}}
\newcommand{\bdot}[0]{{\dot{b}}}
% taken for centered dot:
%\newcommand{\cdot}[0]{{\dot{c}}}
%\newcommand{\ddot}[0]{{\dot{d}}}
\newcommand{\edot}[0]{{\dot{e}}}
\newcommand{\fdot}[0]{{\dot{f}}}
\newcommand{\gdot}[0]{{\dot{g}}}
\newcommand{\hdot}[0]{{\dot{h}}}
\newcommand{\idot}[0]{{\dot{i}}}
\newcommand{\jdot}[0]{{\dot{j}}}
\newcommand{\kdot}[0]{{\dot{k}}}
\newcommand{\ldot}[0]{{\dot{l}}}
\newcommand{\mdot}[0]{{\dot{m}}}
\newcommand{\ndot}[0]{{\dot{n}}}
%\newcommand{\odot}[0]{{\dot{o}}}
\newcommand{\pdot}[0]{{\dot{p}}}
\newcommand{\qdot}[0]{{\dot{q}}}
\newcommand{\rdot}[0]{{\dot{r}}}
\newcommand{\sdot}[0]{{\dot{s}}}
\newcommand{\tdot}[0]{{\dot{t}}}
\newcommand{\udot}[0]{{\dot{u}}}
\newcommand{\vdot}[0]{{\dot{v}}}
\newcommand{\wdot}[0]{{\dot{w}}}
\newcommand{\xdot}[0]{{\dot{x}}}
\newcommand{\ydot}[0]{{\dot{y}}}
\newcommand{\zdot}[0]{{\dot{z}}}
\newcommand{\addot}[0]{{\ddot{a}}}
\newcommand{\bddot}[0]{{\ddot{b}}}
\newcommand{\cddot}[0]{{\ddot{c}}}
%\newcommand{\dddot}[0]{{\ddot{d}}}
\newcommand{\eddot}[0]{{\ddot{e}}}
\newcommand{\fddot}[0]{{\ddot{f}}}
\newcommand{\gddot}[0]{{\ddot{g}}}
\newcommand{\hddot}[0]{{\ddot{h}}}
\newcommand{\iddot}[0]{{\ddot{i}}}
\newcommand{\jddot}[0]{{\ddot{j}}}
\newcommand{\kddot}[0]{{\ddot{k}}}
\newcommand{\lddot}[0]{{\ddot{l}}}
\newcommand{\mddot}[0]{{\ddot{m}}}
\newcommand{\nddot}[0]{{\ddot{n}}}
\newcommand{\oddot}[0]{{\ddot{o}}}
\newcommand{\pddot}[0]{{\ddot{p}}}
\newcommand{\qddot}[0]{{\ddot{q}}}
\newcommand{\rddot}[0]{{\ddot{r}}}
\newcommand{\sddot}[0]{{\ddot{s}}}
\newcommand{\tddot}[0]{{\ddot{t}}}
\newcommand{\uddot}[0]{{\ddot{u}}}
\newcommand{\vddot}[0]{{\ddot{v}}}
\newcommand{\wddot}[0]{{\ddot{w}}}
\newcommand{\xddot}[0]{{\ddot{x}}}
\newcommand{\yddot}[0]{{\ddot{y}}}
\newcommand{\zddot}[0]{{\ddot{z}}}

%<bold and dot greek symbols>
%

\newcommand{\Deltadot}[0]{{\dot{\Delta}}}
\newcommand{\Gammadot}[0]{{\dot{\Gamma}}}
\newcommand{\Lambdadot}[0]{{\dot{\Lambda}}}
\newcommand{\Omegadot}[0]{{\dot{\Omega}}}
\newcommand{\Phidot}[0]{{\dot{\Phi}}}
\newcommand{\Pidot}[0]{{\dot{\Pi}}}
\newcommand{\Psidot}[0]{{\dot{\Psi}}}
\newcommand{\Sigmadot}[0]{{\dot{\Sigma}}}
\newcommand{\Thetadot}[0]{{\dot{\Theta}}}
\newcommand{\Upsilondot}[0]{{\dot{\Upsilon}}}
\newcommand{\Xidot}[0]{{\dot{\Xi}}}
\newcommand{\alphadot}[0]{{\dot{\alpha}}}
\newcommand{\betadot}[0]{{\dot{\beta}}}
\newcommand{\chidot}[0]{{\dot{\chi}}}
\newcommand{\deltadot}[0]{{\dot{\delta}}}
\newcommand{\epsilondot}[0]{{\dot{\epsilon}}}
\newcommand{\etadot}[0]{{\dot{\eta}}}
\newcommand{\gammadot}[0]{{\dot{\gamma}}}
\newcommand{\kappadot}[0]{{\dot{\kappa}}}
\newcommand{\lambdadot}[0]{{\dot{\lambda}}}
\newcommand{\mudot}[0]{{\dot{\mu}}}
\newcommand{\nudot}[0]{{\dot{\nu}}}
\newcommand{\omegadot}[0]{{\dot{\omega}}}
\newcommand{\phidot}[0]{{\dot{\phi}}}
\newcommand{\pidot}[0]{{\dot{\pi}}}
\newcommand{\psidot}[0]{{\dot{\psi}}}
\newcommand{\rhodot}[0]{{\dot{\rho}}}
\newcommand{\sigmadot}[0]{{\dot{\sigma}}}
\newcommand{\taudot}[0]{{\dot{\tau}}}
\newcommand{\thetadot}[0]{{\dot{\theta}}}
\newcommand{\upsilondot}[0]{{\dot{\upsilon}}}
\newcommand{\varepsilondot}[0]{{\dot{\varepsilon}}}
\newcommand{\varphidot}[0]{{\dot{\varphi}}}
\newcommand{\varpidot}[0]{{\dot{\varpi}}}
\newcommand{\varrhodot}[0]{{\dot{\varrho}}}
\newcommand{\varsigmadot}[0]{{\dot{\varsigma}}}
\newcommand{\varthetadot}[0]{{\dot{\vartheta}}}
\newcommand{\xidot}[0]{{\dot{\xi}}}
\newcommand{\zetadot}[0]{{\dot{\zeta}}}

\newcommand{\Deltaddot}[0]{{\ddot{\Delta}}}
\newcommand{\Gammaddot}[0]{{\ddot{\Gamma}}}
\newcommand{\Lambdaddot}[0]{{\ddot{\Lambda}}}
\newcommand{\Omegaddot}[0]{{\ddot{\Omega}}}
\newcommand{\Phiddot}[0]{{\ddot{\Phi}}}
\newcommand{\Piddot}[0]{{\ddot{\Pi}}}
\newcommand{\Psiddot}[0]{{\ddot{\Psi}}}
\newcommand{\Sigmaddot}[0]{{\ddot{\Sigma}}}
\newcommand{\Thetaddot}[0]{{\ddot{\Theta}}}
\newcommand{\Upsilonddot}[0]{{\ddot{\Upsilon}}}
\newcommand{\Xiddot}[0]{{\ddot{\Xi}}}
\newcommand{\alphaddot}[0]{{\ddot{\alpha}}}
\newcommand{\betaddot}[0]{{\ddot{\beta}}}
\newcommand{\chiddot}[0]{{\ddot{\chi}}}
\newcommand{\deltaddot}[0]{{\ddot{\delta}}}
\newcommand{\epsilonddot}[0]{{\ddot{\epsilon}}}
\newcommand{\etaddot}[0]{{\ddot{\eta}}}
\newcommand{\gammaddot}[0]{{\ddot{\gamma}}}
\newcommand{\kappaddot}[0]{{\ddot{\kappa}}}
\newcommand{\lambdaddot}[0]{{\ddot{\lambda}}}
\newcommand{\muddot}[0]{{\ddot{\mu}}}
\newcommand{\nuddot}[0]{{\ddot{\nu}}}
\newcommand{\omegaddot}[0]{{\ddot{\omega}}}
\newcommand{\phiddot}[0]{{\ddot{\phi}}}
\newcommand{\piddot}[0]{{\ddot{\pi}}}
\newcommand{\psiddot}[0]{{\ddot{\psi}}}
\newcommand{\rhoddot}[0]{{\ddot{\rho}}}
\newcommand{\sigmaddot}[0]{{\ddot{\sigma}}}
\newcommand{\tauddot}[0]{{\ddot{\tau}}}
\newcommand{\thetaddot}[0]{{\ddot{\theta}}}
\newcommand{\upsilonddot}[0]{{\ddot{\upsilon}}}
\newcommand{\varepsilonddot}[0]{{\ddot{\varepsilon}}}
\newcommand{\varphiddot}[0]{{\ddot{\varphi}}}
\newcommand{\varpiddot}[0]{{\ddot{\varpi}}}
\newcommand{\varrhoddot}[0]{{\ddot{\varrho}}}
\newcommand{\varsigmaddot}[0]{{\ddot{\varsigma}}}
\newcommand{\varthetaddot}[0]{{\ddot{\vartheta}}}
\newcommand{\xiddot}[0]{{\ddot{\xi}}}
\newcommand{\zetaddot}[0]{{\ddot{\zeta}}}

\newcommand{\BDelta}[0]{\boldsymbol{\Delta}}
\newcommand{\BGamma}[0]{\boldsymbol{\Gamma}}
\newcommand{\BLambda}[0]{\boldsymbol{\Lambda}}
\newcommand{\BOmega}[0]{\boldsymbol{\Omega}}
\newcommand{\BPhi}[0]{\boldsymbol{\Phi}}
\newcommand{\BPi}[0]{\boldsymbol{\Pi}}
\newcommand{\BPsi}[0]{\boldsymbol{\Psi}}
\newcommand{\BSigma}[0]{\boldsymbol{\Sigma}}
\newcommand{\BTheta}[0]{\boldsymbol{\Theta}}
\newcommand{\BUpsilon}[0]{\boldsymbol{\Upsilon}}
\newcommand{\BXi}[0]{\boldsymbol{\Xi}}
\newcommand{\Balpha}[0]{\boldsymbol{\alpha}}
\newcommand{\Bbeta}[0]{\boldsymbol{\beta}}
\newcommand{\Bchi}[0]{\boldsymbol{\chi}}
\newcommand{\Bdelta}[0]{\boldsymbol{\delta}}
\newcommand{\Bepsilon}[0]{\boldsymbol{\epsilon}}
\newcommand{\Beta}[0]{\boldsymbol{\eta}}
\newcommand{\Bgamma}[0]{\boldsymbol{\gamma}}
\newcommand{\Bkappa}[0]{\boldsymbol{\kappa}}
\newcommand{\Blambda}[0]{\boldsymbol{\lambda}}
\newcommand{\Bmu}[0]{\boldsymbol{\mu}}
\newcommand{\Bnu}[0]{\boldsymbol{\nu}}
%\newcommand{\Bomega}[0]{\boldsymbol{\omega}}
\newcommand{\Bphi}[0]{\boldsymbol{\phi}}
\newcommand{\Bpi}[0]{\boldsymbol{\pi}}
\newcommand{\Bpsi}[0]{\boldsymbol{\psi}}
\newcommand{\Brho}[0]{\boldsymbol{\rho}}
\newcommand{\Bsigma}[0]{\boldsymbol{\sigma}}
%\newcommand{\Btau}[0]{\boldsymbol{\tau}}
%\newcommand{\Btheta}[0]{\boldsymbol{\theta}}
\newcommand{\Bupsilon}[0]{\boldsymbol{\upsilon}}
\newcommand{\Bvarepsilon}[0]{\boldsymbol{\varepsilon}}
\newcommand{\Bvarphi}[0]{\boldsymbol{\varphi}}
\newcommand{\Bvarpi}[0]{\boldsymbol{\varpi}}
\newcommand{\Bvarrho}[0]{\boldsymbol{\varrho}}
\newcommand{\Bvarsigma}[0]{\boldsymbol{\varsigma}}
\newcommand{\Bvartheta}[0]{\boldsymbol{\vartheta}}
\newcommand{\Bxi}[0]{\boldsymbol{\xi}}
\newcommand{\Bzeta}[0]{\boldsymbol{\zeta}}
%
%</bold and dot greek symbols>
%<infrequent>
%
%\newcommand{\AreaOp}[1]{\AName_{#1}}
%\newcommand{\Babs}[0]{\abs{\BB}}
%\newcommand{\Bcap}[0]{\hat{\BB}}
%\newcommand{\BrPrimeRej}[0]{\rcap(\rcap \wedge \Br')}
%\newcommand{\CA}[0]{\mathcal{A}}
%\newcommand{\Cos}[1]{\cos{\left({#1}\right)}}
%\newcommand{\Det}[1] {\abs{#1}}
%\newcommand{\Dsq}[2] {\frac {\partial^2 {#1}} {\partial {#2}^2}}
%\newcommand{\Exp}[1]{\exp{\left({#1}\right)}}
%\newcommand{\Norm}[1]{\left\lVert{#1}\right\rVert}
%\newcommand{\Sin}[1]{\sin{\left({#1}\right)}}
%\newcommand{\T}[0]{\text{T}}
%\newcommand{\VolumeOp}[1]{\VName_{#1}}
%\newcommand{\agrad}[0]{\Ba \cdot \nabla}
%\newcommand{\alphacap}[0]{\hat{\boldsymbol{\alpha}}}
%\newcommand{\Fcap}[0]{\hat{\BF}}
%\newcommand{\bithree}[0]{{\Bi}_3}
%\newcommand{\bxa}[0]{\Bx\Ba}
%\newcommand{\coordvec}[2]{
%\newcommand{\costheta}[0]{\acap \cdot \xcap}
%\newcommand{\ddt}[1]{\ddot{#1}}
%\newcommand{\ddu}[1] {\frac {d{#1}} {du}}
%\newcommand{\dsqxj}[2] {\frac {\partial^2 {#1}} {\partial {x_{#2}}^2}}
%\newcommand{\dtheta}[1]{\frac{d {#1}}{d \theta}}
%\newcommand{\dt}[1]{\dot{#1}}
%\newcommand{\dt}[1]{\frac{d {#1}}{dt}}
%\newcommand{\dxj}[2] {\frac {\partial {#1}} {\partial {x_{#2}}}}
%\newcommand{\halfPhi}[0]{\frac{\phi}{2}}
%\newcommand{\half}[0]{\inv{2}}
%\newcommand{\inv}[1]{\frac{1}{#1}}
%\newcommand{\laplacian}[0]{\nabla^2}
%\newcommand{\matrixoftx}[3]{
%\newcommand{\nrrp}[0]{\norm{\rcap \wedge \Br'}}
%\newcommand{\oiint}{\bigcirc \hspace{-1.4em} \int \hspace{-.8em} \int}
%\newcommand{\transpose}[1]{{#1}^{\text{T}}}
%\newcommand{\transpose}[1]{{{#1}^{\TextTranspose}}}
%\newcommand{\transpose}[1]{{{#1}^{\text{T}}}}
%\newcommand{\barA}[0]{\bar{A}}
%\newcommand{\qbar}[0]{\bar{q}}
%\newcommand{\qdotbar}[0]{\dot{\bar{q}}}
%
%</infrequent>





\usepackage[bookmarks=true]{hyperref}

\usepackage{color,cite,graphicx}
   % use colour in the document, put your citations as [1-4]
   % rather than [1,2,3,4] (it looks nicer, and the extended LaTeX2e
   % graphics package. 
\usepackage{latexsym,amssymb,epsf} % don't remember if these are
   % needed, but their inclusion can't do any damage

\title{ Electrodynamic wave equation solutions. }
\author{Peeter Joot}
\date{ Jan 25, 2009.  Last Revision: $Date: 2009/01/25 22:28:57 $ }

\begin{document}

\maketitle{}
\tableofcontents

\section{ Motivation. }

In \cite{PJwaveFourVector} four vector solutions to the mechanical wave
equations were explored.  What was obviously missing from that 
was consideration of the special case for $\Bv^2 = c^2$.

Here solutions to the electrodynamic wave equation will be examined.
Consideration of such solutions in more detail will is expected
to be helpful 
as background for the more complex study of quantum (matter) wave equations.

\section{ Electromagnetic wave equation solutions. }

For electrodynamics our equation to solve is

\begin{align*}
\grad F = J/\epsilon_0 c
\end{align*}

For the unforced (vacuum) solutions, with 
$F = \grad \wedge A$, and the Coulomb gauge $\grad \cdot A = 0$ this 
reduces to

\begin{align*}
0 
&= \left((\gamma^\mu)^2 \partial_{\mu\mu}\right) A  \\
&= \left( \inv{c^2}\partial_{tt} -\partial_{jj} \right) A
\end{align*}

These equations have the same form as the mechanical wave equation
where the wave velocity $\Bv^2 = c^2$ is the speed of light

\begin{align}\label{eqn:waveEquation}
\left( \inv{\Bv^2} \partial_{tt} - \sum_{j=1}^3 \partial_{jj}\right) \psi = 0
\end{align}

\subsection{ Separation of variables solution of potential equations. }

Let's solve this using separation of variables, and write $A^\nu = X Y Z T = \Pi_{\mu} X^{\mu}$

From this we have

\begin{align*}
\sum_\mu (\gamma^\mu)^2 \frac{(X^\mu)''}{X^\mu} = 0
\end{align*}

and can procede with the normal procedure of assuming that a solution can be
found by separately equating each term to a constant.  Writing those
constants explicitly as $(m_\mu)^2$, which we allow to be potentially complex
we have (no sum) 

\begin{align*}
X^\mu = \exp\left( \pm \sqrt{(\gamma^\mu)^2} m_\mu x^\mu \right)
\end{align*}

Now, let $k_\mu = \pm \sqrt{(\gamma^\mu)^2} m_\mu$, folding any sign variation
and complex factors into these constants.  Our complete solution
is thus

\begin{align*}
\Pi_\mu X^\mu = \exp\left( \sum k_\mu x^\mu \right)
\end{align*}

However, for this to be a solution, the wave equation imposes the constraint

\begin{align*}
\sum_\mu (\gamma^\mu)^2 (k_\mu)^2 = 0
\end{align*}

Or
\begin{align*}
(k_0)^2 - \sum_j (k_j)^2 = 0
\end{align*}

Summarizing each potential term has a solution expressable in terms of 
null "wave-number" vectors $K_\nu$

\begin{align}\label{eqn:potentialSolution}
A_\nu &= \exp\left( K_\nu \cdot x \right)  \\
\Abs{K_\nu} &= 0
\end{align}

\subsection{ Faraday bivector and tensor from the potential solutions. }

From the components of the potentials 
\ref{eqn:potentialSolution}
we can compute the curl for the complete
field.  That is

\begin{align*}
F &= \grad \wedge A \\
A &= \gamma^\nu \exp\left( K_\nu \cdot x \right)  \\
\end{align*}

This is

\begin{align*}
F 
&= \left(\gamma^\mu \wedge \gamma^\nu\right) \partial_\mu \exp\left( K_\nu \cdot x \right) \\
&= \left(\gamma^\mu \wedge \gamma^\nu\right) \partial_\mu \exp\left( \gamma^\alpha K_{\nu\alpha} \cdot \gamma_\sigma x^\sigma \right) \\
&= \left(\gamma^\mu \wedge \gamma^\nu\right) \partial_\mu \exp\left( K_{\nu\sigma} x^\sigma \right) \\
%&= \left(\gamma^\mu \wedge \gamma^\nu\right) K_{\nu\sigma} \delta_\mu\sigma \exp\left( K_{\nu\sigma} x^\sigma \right) \\
&= \left(\gamma^\mu \wedge \gamma^\nu\right) K_{\nu\mu} \exp\left( K_{\nu\sigma} x^\sigma \right) \\
&= \left(\gamma^\mu \wedge \gamma^\nu\right) K_{\nu\mu} \exp\left( K_{\nu} \cdot x \right) \\
&= \left(\gamma^\mu \wedge \gamma^\nu\right) 
\inv{2} \left( K_{\nu\mu} \exp\left( K_{\nu} \cdot x \right) - K_{\mu\nu} \exp\left( K_{\mu} \cdot x \right) \right) \\
\end{align*}

Writing our field in explicit tensor form

\begin{align*}
F = F_{\mu\nu} \gamma^\mu \wedge \gamma^\nu
\end{align*}

our vacuum solution is therefore

\begin{align}
F_{\mu\nu} &= \inv{2} \left( K_{\nu\mu} \exp\left( K_{\nu} \cdot x \right) - K_{\mu\nu} \exp\left( K_{\mu} \cdot x \right) \right) 
\end{align}

but subject to the null wave number and Lorentz guage constraints

\begin{align}
\Abs{K_\mu} &= 0 \\
\grad \cdot \left(\gamma^\mu \exp\left( K_\mu \cdot x \right)\right) &= 0
\end{align}

\subsection{ Examine the Lorentz gauge constraint. }

That Lorentz gauge constraint on the potential is a curious looking beastie.  Let's expand that out in full to examine it closer

\begin{align*}
\grad \cdot \left(\gamma^\mu \exp\left( K_\mu \cdot x \right)\right) 
&= \gamma^\alpha \partial_\alpha \cdot \left(\gamma^\mu \exp\left( K_\mu \cdot x \right)\right)  \\
&= \sum_\mu (\gamma^\mu)^2 \partial_\mu \exp\left( K_\mu \cdot x \right) \\
&= \sum_\mu (\gamma^\mu)^2 \partial_\mu \exp\left( \sum \gamma^\nu K_{\mu\nu} \cdot \gamma_\alpha x^\alpha \right) \\
&= \sum_\mu (\gamma^\mu)^2 \partial_\mu \exp\left( \sum K_{\mu\alpha} x^\alpha \right) \\
&= \sum_\mu (\gamma^\mu)^2 K_{\mu\mu} \exp\left( K_\mu \cdot x \right)
\end{align*}

If this must be zero for any $x$ it must also be zero for $x =0$, so the Lorentz gauge imposes an additional restriction on the
wave number four vectors $K_\mu$

\begin{align*}
\sum_\mu (\gamma^\mu)^2 K_{\mu\mu} = 0 
\end{align*}

Expanding in time and spatial coordinates this is

\begin{align*}
K_{00} - \sum_j K_{jj} = 0 
\end{align*}

One obvious way to satisfy this is to require that the tensor $K_{\mu\nu}$ be diagonal, but since we also have the null vector requirement
on each of the $K_\mu$ four vectors it isn't clear that this is an acceptable choice.

\subsection{ Summarizing so far. }

We have found that our field solution has the form

\begin{align*}
F_{\mu\nu} &= \inv{2} \left( K_{\nu\mu} \exp\left( K_{\nu} \cdot x \right) - K_{\mu\nu} \exp\left( K_{\mu} \cdot x \right) \right) \\
\end{align*}

Where
\begin{align*}
K_\mu &= \gamma^\nu K_{\mu\nu} \\
x &= \gamma_\mu x^\mu
\end{align*}

with constraints
\begin{align*}
0 &= \sum_\mu (\gamma^\mu)^2 K_{\mu\mu} \\
0 &= \sum_\mu (\gamma^\mu)^2 (K_{\nu\mu})^2
\end{align*}

Alternately, calling out the explict space time split of the constraint, we can
remove the explicit $\gamma^\mu$ factors 

\begin{align*}
0 = K_{00} - \sum_j K_{jj} = (K_{00})^2 - \sum_j (K_{jj} )^2
\end{align*}

%If each of $K_{\mu0} \ne 0$ we could alternately remove the explicit $\gamma^\mu$ factors and write
%
%\begin{align*}
%1 = \sum_j \frac{K_{jj}}{K_{00}} = \sum_j \left(\frac{K_{jj}}{K_{00}}\right)^2 
%\end{align*}
%
%Is this any better?

\section{ Looking for more general solutions. }

\subsection{ Using mechanical wave solutions as a guide. }

In the mechanical wave equation, we had exponential solutions of the form

\begin{align*}
f(\Bx,t) = \exp\left( \Bk \cdot \Bx + \omega t \right)
\end{align*}

which were solutions to \ref{eqn:waveEquation} provided that

\begin{align*}
\inv{\Bv^2} \omega^2 - \Bk^2 = 0.
\end{align*}

This meant that 
\begin{align*}
\omega = \pm \Abs{\Bv} \Abs{\Bk}
\end{align*}

and our function takes the (hyperbolic) form, or (sinusoidal) form respectively

\begin{align*}
f(\Bx,t) &= \exp\left( \Abs{\Bk}\left( \kcap \cdot \Bx \pm \Abs{\Bv} t \right) \right) \\
f(\Bx,t) &= \exp\left( i \Abs{\Bk}\left( \kcap \cdot \Bx \pm \Abs{\Bv} t \right) \right)
\end{align*}

Fourier series superposition of the latter solutions can be used to express any spatially periodic function, while fourier transforms 
can be used to express the non-periodic cases.

These superpositions, subject to boundary value conditions, allow for writing solutions to the wave equation in the form

\begin{align}\label{eqn:generalWaveSolution}
f(\Bx,t) &= g\left( \kcap \cdot \Bx \pm \Abs{\Bv} t \right)
\end{align}

Showing this logically follows from the original separation of variables approach has not been done.   However, despite this,
it is
simple enough to confirm that,
this more general function does satisify the unforced wave equation \ref{eqn:waveEquation}.

TODO: as followup here would like to go through the exersize of showing 
that the solution of \ref{eqn:generalWaveSolution} follows from a Fourier transform superposition.  Intuition says this is
possible, and I've said so without backing up the statement.

\subsection{ Back to the electrodynamic case. }

Using the above generalization argument as a guide we should be able to do something similar for the electrodynamic wave solution.

We want to solve for $F$ the following gradient equation for the field in free space

\begin{align}\label{eqn:maxwell}
\grad F = 0
\end{align}

Let's suppose that the following is a solution and find the required constraints

\begin{align}\label{eqn:testSol}
F = \gamma^\mu \wedge \gamma^\nu \left( K_{\mu\nu} f( x \cdot K_\mu ) -K_{\nu\mu} f( x \cdot K_\nu ) \right)
\end{align}

We have two different grade equations built into Maxwell's equation \ref{eqn:maxwell}, one of which is the vector equation, and the other
trivector.  Those are respectively

\begin{align*}
\grad \cdot F &= 0 \\
\grad \wedge F &= 0
\end{align*}

\subsubsection{ zero divergence }

So, let's substitute equation \ref{eqn:testSol} and see what comes out, starting with the zero divergence equation

\begin{align*}
\grad \cdot F 
&= \gamma^\alpha \cdot
\left(\gamma^\mu \wedge \gamma^\nu\right) \partial_\alpha \left( K_{\mu\nu} f( x \cdot K_\mu ) -K_{\nu\mu} f( x \cdot K_\nu ) \right) \\
&= 
(\gamma_\alpha)^2 
%\gamma_\alpha \cdot \left(\gamma^\mu \wedge \gamma^\nu\right) 
\left( \gamma^\nu {\delta_\alpha}^\mu -\gamma^\mu {\delta_\alpha}^\nu \right)
\partial_\alpha \left( K_{\mu\nu} f( x \cdot K_\mu ) -K_{\nu\mu} f( x \cdot K_\nu ) \right) \\
&= 
\left(
(\gamma_\mu)^2 \gamma^\nu \partial_\mu 
-(\gamma_\nu)^2 \gamma^\mu \partial_\nu 
\right)
\left( K_{\mu\nu} f( x \cdot K_\mu ) -K_{\nu\mu} f( x \cdot K_\nu ) \right) \\
&= 
(\gamma_\mu)^2 \gamma^\nu \partial_\mu \left( K_{\mu\nu} f( x \cdot K_\mu ) -K_{\nu\mu} f( x \cdot K_\nu ) \right) 
-(\gamma_\mu)^2 \gamma^\nu \partial_\mu \left( K_{\nu\mu} f( x \cdot K_\nu ) -K_{\mu\nu} f( x \cdot K_\mu ) \right) \\
&= 2 (\gamma_\mu)^2 \gamma^\nu \partial_\mu \left( K_{\mu\nu} f( x \cdot K_\mu ) -K_{\nu\mu} f( x \cdot K_\nu ) \right) \\
\end{align*}

Now, for the chain rule calculation for partial we want

\begin{align*}
\partial_\mu ( x \cdot K_\beta ) 
&= \partial_\mu ( x^\nu \gamma_\nu \cdot K_{\beta\sigma} \gamma^\sigma ) \\
&= \partial_\mu ( x^\sigma K_{\beta\sigma} \\
&= K_{\beta\mu}
\end{align*}

\begin{align*}
\grad \cdot F 
&= 2 (\gamma_\mu)^2 \gamma^\nu \left( K_{\mu\nu} K_{\mu\mu} f'( x \cdot K_\mu ) -K_{\nu\mu} K_{\nu\mu} f'( x \cdot K_\nu ) \right) \\
\end{align*}

For $\mu = \nu$ this is zero, which is expected since that should follow from the wedge product itself, but for the $\mu \ne \nu$
case it isn't clear cut.  Damn (on paper I missed some terms and it all cancelled).

\subsubsection{ zero wedge }

For the grade three term we have

\begin{align*}
\grad \wedge F 
&= 
\left( \gamma^\alpha \wedge \gamma^\mu \wedge \gamma^\nu\right)
\partial_\alpha \left( K_{\mu\nu} f( x \cdot K_\mu ) -K_{\nu\mu} f( x \cdot K_\nu ) \right) \\
&= 
%\partial_\alpha ( x \cdot K_\beta ) &= K_{\beta\alpha}
\left( \gamma^\alpha \wedge \gamma^\mu \wedge \gamma^\nu\right)
\left( K_{\mu\nu} K_{\mu\alpha} f'( x \cdot K_\mu ) -K_{\nu\mu} K_{\nu\alpha} f'( x \cdot K_\nu ) \right) \\
&= 
2\left( \gamma^\alpha \wedge \gamma^\mu \wedge \gamma^\nu\right) K_{\mu\nu} K_{\mu\alpha} f'( x \cdot K_\mu ) 
\\
\end{align*}

So, for this to be zero uniformly for all $f$, we require

\begin{align*}
K_{\mu\nu} K_{\mu\alpha} = 0 
\end{align*}

\bibliographystyle{plainnat}
\bibliography{myrefs}

\end{document}

%
% Copyright � 2012 Peeter Joot.  All Rights Reserved.
% Licenced as described in the file LICENSE under the root directory of this GIT repository.
%

% 
% 
\chapter{Magnetic field between two parallel wires}        
\label{chap:sgMx41}
\date{July 20, 2008}

\section{Student's guide to Maxwell's' equations.  problem 4.1}        

The 
\href{http://www4.wittenberg.edu/maxwell/chapter4/problem1/}{problem is}:

Two parallel wires carry currents I1 and 2I1 in opposite directions.  Use Ampere�s law to find the magnetic field at a point midway between the wires.

Do this instead (visualizing the cross section through the wires) for N wires
located at points $P_k$, with currents $I_k$.

\begin{figure}[htp]
\centering
\includegraphics[totalheight=0.4\textheight]{p41}
\caption{Currents through parallel wires}\label{fig:2wires}
\end{figure}

This is illustrated for two wires in figure \ref{fig:2wires}.

\subsection{}

Consider first just the magnetic field for one wire, temporarily putting
the origin at the point of the current.

\begin{equation*}
\int \BB \cdot d\Bl = \mu_0 I
\end{equation*}

At a point $\Br$ from the local origin the tangent vector is obtained by 
rotation of the unit vector:

\begin{equation*}
\ycap \exp{\left(\xcap\ycap \log{\left(\frac{\Br}{\norm{\Br}}\right)}\right)}
= \ycap {\left(\frac{\Br}{\norm{\Br}}\right)}^{\xcap\ycap}
\end{equation*}

Thus the magnetic field at the point $\Br$ due to this particular current is:

\begin{equation*}
\BB(\Br) 
= \frac{\mu_0 I \ycap}{2\pi \norm{\Br}} {\left(\frac{\Br}{\norm{\Br}}\right)}^{\xcap\ycap}
\end{equation*}

Considering additional currents with the wire centers at points $P_k$, and measurement of the field at point $\BR$ we have for each of those:

\begin{equation*}
\Br = \BR - \BP
\end{equation*}

Thus the total field at point $\BR$ is:

\begin{equation}
\BB(\BR) = \frac{\mu_0 \ycap}{2\pi} \sum_k \frac{I_k}{\norm{\BR - \BP_k}} {\left(\frac{\BR - \BP_k}{\norm{\BR - \BP_k}}\right)}^{\xcap\ycap}
\end{equation}

\subsection{Original problem}

For the problem as stated, put the origin between the two points with those two points on the x-axis.

\begin{align*}
\BP_1 &= - \xcap d/2 \\
\BP_2 &= \xcap d/2 
\end{align*}

Here $\BR$ = 0, so $\Br_1 = \BR - \BP_1 = \xcap d/2 $ and $\Br_2 = - \xcap d/2$.  With $\xcap\ycap = i$, this is:

\begin{align*}
\BB(0)
&= \frac{\mu_0 \ycap}{\pi d} \left( I_1 {(-\xcap)}^i + I_2 {\xcap^i} \right) \\
&= \frac{\mu_0 \ycap}{\pi d} \left( -I -2 I\right) \\
&= \frac{-3 I \mu_0 \ycap}{\pi d}
\end{align*}

Here unit vectors exponentials were evaluated with the equivalent complex number manipulations:

\begin{align*}
(-1)^i &= x \\
i \log{(-1)} &= \log{x} \\
i \pi &= \log{x} \\
\exp{(i \pi)} &= \log{x} \\
x &= -1
\end{align*}

\begin{align*}
(1)^i &= x \\
i \log{(1)} &= \log{x} \\
0 &= \log{x} \\
x &= 1 
\end{align*}

\documentclass{article}

\usepackage{amsmath}
\usepackage{mathpazo}

%
% shorthand for bold symbols, convenient for vectors and matrices
%
\newcommand{\Ba}[0]{\mathbf{a}}
\newcommand{\Bb}[0]{\mathbf{b}}
\newcommand{\Bc}[0]{\mathbf{c}}
\newcommand{\Bd}[0]{\mathbf{d}}
\newcommand{\Be}[0]{\mathbf{e}}
\newcommand{\Bf}[0]{\mathbf{f}}
\newcommand{\Bg}[0]{\mathbf{g}}
\newcommand{\Bh}[0]{\mathbf{h}}
\newcommand{\Bi}[0]{\mathbf{i}}
\newcommand{\Bj}[0]{\mathbf{j}}
\newcommand{\Bk}[0]{\mathbf{k}}
\newcommand{\Bl}[0]{\mathbf{l}}
\newcommand{\Bm}[0]{\mathbf{m}}
\newcommand{\Bn}[0]{\mathbf{n}}
\newcommand{\Bo}[0]{\mathbf{o}}
\newcommand{\Bp}[0]{\mathbf{p}}
\newcommand{\Bq}[0]{\mathbf{q}}
\newcommand{\Br}[0]{\mathbf{r}}
\newcommand{\Bs}[0]{\mathbf{s}}
\newcommand{\Bt}[0]{\mathbf{t}}
\newcommand{\Bu}[0]{\mathbf{u}}
\newcommand{\Bv}[0]{\mathbf{v}}
\newcommand{\Bw}[0]{\mathbf{w}}
\newcommand{\Bx}[0]{\mathbf{x}}
\newcommand{\By}[0]{\mathbf{y}}
\newcommand{\Bz}[0]{\mathbf{z}}
\newcommand{\BA}[0]{\mathbf{A}}
\newcommand{\BB}[0]{\mathbf{B}}
\newcommand{\BC}[0]{\mathbf{C}}
\newcommand{\BD}[0]{\mathbf{D}}
\newcommand{\BE}[0]{\mathbf{E}}
\newcommand{\BF}[0]{\mathbf{F}}
\newcommand{\BG}[0]{\mathbf{G}}
\newcommand{\BH}[0]{\mathbf{H}}
\newcommand{\BI}[0]{\mathbf{I}}
\newcommand{\BJ}[0]{\mathbf{J}}
\newcommand{\BK}[0]{\mathbf{K}}
\newcommand{\BL}[0]{\mathbf{L}}
\newcommand{\BM}[0]{\mathbf{M}}
\newcommand{\BN}[0]{\mathbf{N}}
\newcommand{\BO}[0]{\mathbf{O}}
\newcommand{\BP}[0]{\mathbf{P}}
\newcommand{\BQ}[0]{\mathbf{Q}}
\newcommand{\BR}[0]{\mathbf{R}}
\newcommand{\BS}[0]{\mathbf{S}}
\newcommand{\BT}[0]{\mathbf{T}}
\newcommand{\BU}[0]{\mathbf{U}}
\newcommand{\BV}[0]{\mathbf{V}}
\newcommand{\BW}[0]{\mathbf{W}}
\newcommand{\BX}[0]{\mathbf{X}}
\newcommand{\BY}[0]{\mathbf{Y}}
\newcommand{\BZ}[0]{\mathbf{Z}}

\newcommand{\Bzero}[0]{\mathbf{0}}
\newcommand{\Btheta}[0]{\boldsymbol{\theta}}
\newcommand{\Btau}[0]{\boldsymbol{\tau}}
\newcommand{\Bomega}[0]{\boldsymbol{\omega}}

%
% shorthand for unit vectors
%
\newcommand{\acap}[0]{\hat{\Ba}}
\newcommand{\bcap}[0]{\hat{\Bb}}
\newcommand{\ccap}[0]{\hat{\Bc}}
\newcommand{\dcap}[0]{\hat{\Bd}}
\newcommand{\ecap}[0]{\hat{\Be}}
\newcommand{\fcap}[0]{\hat{\Bf}}
\newcommand{\gcap}[0]{\hat{\Bg}}
\newcommand{\hcap}[0]{\hat{\Bh}}
\newcommand{\icap}[0]{\hat{\Bi}}
\newcommand{\jcap}[0]{\hat{\Bj}}
\newcommand{\kcap}[0]{\hat{\Bk}}
\newcommand{\lcap}[0]{\hat{\Bl}}
\newcommand{\mcap}[0]{\hat{\Bm}}
\newcommand{\ncap}[0]{\hat{\Bn}}
\newcommand{\ocap}[0]{\hat{\Bo}}
\newcommand{\pcap}[0]{\hat{\Bp}}
\newcommand{\qcap}[0]{\hat{\Bq}}
\newcommand{\rcap}[0]{\hat{\Br}}
\newcommand{\scap}[0]{\hat{\Bs}}
\newcommand{\tcap}[0]{\hat{\Bt}}
\newcommand{\ucap}[0]{\hat{\Bu}}
\newcommand{\vcap}[0]{\hat{\Bv}}
\newcommand{\wcap}[0]{\hat{\Bw}}
\newcommand{\xcap}[0]{\hat{\Bx}}
\newcommand{\ycap}[0]{\hat{\By}}
\newcommand{\zcap}[0]{\hat{\Bz}}
\newcommand{\thetacap}[0]{\hat{\Btheta}}

%
% to write R^n and C^n in a distinguishable fashion.  Perhaps change this
% to the double lined characters upon figuring out how to do so.
%
\newcommand{\C}[1]{$\mathbb{C}^{#1}$}
\newcommand{\R}[1]{$\mathbb{R}^{#1}$}

%
% various generally useful helpers
%

% derivative of #1 wrt. #2:
\newcommand{\D}[2] {\frac {d#2} {d#1}}

\newcommand{\inv}[1]{\frac{1}{#1}}
\newcommand{\cross}[0]{\times}

\newcommand{\abs}[1]{\lvert{#1}\rvert}
\newcommand{\norm}[1]{\lVert{#1}\rVert}
\newcommand{\innerprod}[2]{\langle{#1}, {#2}\rangle}
\newcommand{\dotprod}[2]{{#1} \cdot {#2}}
\newcommand{\bdotprod}[2]{\left({#1} \cdot {#2}\right)}
\newcommand{\crossprod}[2]{{#1} \cross {#2}}
\newcommand{\tripleprod}[3]{\dotprod{\left(\crossprod{#1}{#2}\right)}{#3}}

\DeclareMathOperator{\Proj}{Proj}
\DeclareMathOperator{\Span}{span}
\DeclareMathOperator{\Sgn}{sgn}
\DeclareMathOperator{\Area}{Area}
\DeclareMathOperator{\Volume}{Volume}

%
% A few miscellaneous things specific to this document
%
\newcommand{\crossop}[1]{\crossprod{#1}{}}

% R2 vector.
\newcommand{\VectorTwo}[2]{
\begin{bmatrix}
 {#1} \\
 {#2}
\end{bmatrix}
}

\newcommand{\VectorN}[1]{
\begin{bmatrix}
{#1}_1 \\
{#1}_2 \\
\vdots \\
{#1}_N \\
\end{bmatrix}
}

\newcommand{\DETuvij}[4]{
\begin{vmatrix}
 {#1}_{#3} & {#1}_{#4} \\
 {#2}_{#3} & {#2}_{#4}
\end{vmatrix}
}

\newcommand{\DETuvwijk}[6]{
\begin{vmatrix}
 {#1}_{#4} & {#1}_{#5} & {#1}_{#6} \\
 {#2}_{#4} & {#2}_{#5} & {#2}_{#6} \\
 {#3}_{#4} & {#3}_{#5} & {#3}_{#6}
\end{vmatrix}
}

\newcommand{\DETuvwxijkl}[8]{
\begin{vmatrix}
 {#1}_{#5} & {#1}_{#6} & {#1}_{#7} & {#1}_{#8} \\
 {#2}_{#5} & {#2}_{#6} & {#2}_{#7} & {#2}_{#8} \\
 {#3}_{#5} & {#3}_{#6} & {#3}_{#7} & {#3}_{#8} \\
 {#4}_{#5} & {#4}_{#6} & {#4}_{#7} & {#4}_{#8} \\
\end{vmatrix}
}

%\newcommand{\DETuvwxyijklm}[10]{
%\begin{vmatrix}
% {#1}_{#6} & {#1}_{#7} & {#1}_{#8} & {#1}_{#9} & {#1}_{#10} \\
% {#2}_{#6} & {#2}_{#7} & {#2}_{#8} & {#2}_{#9} & {#2}_{#10} \\
% {#3}_{#6} & {#3}_{#7} & {#3}_{#8} & {#3}_{#9} & {#3}_{#10} \\
% {#4}_{#6} & {#4}_{#7} & {#4}_{#8} & {#4}_{#9} & {#4}_{#10} \\
% {#5}_{#6} & {#5}_{#7} & {#5}_{#8} & {#5}_{#9} & {#5}_{#10}
%\end{vmatrix}
%}

% R3 vector.
\newcommand{\VectorThree}[3]{
\begin{bmatrix}
 {#1} \\
 {#2} \\
 {#3}
\end{bmatrix}
}


%<misc>
%
\newcommand{\Abs}[1]{{\left\lvert{#1}\right\rvert}}
\newcommand{\spacegrad}[0]{\boldsymbol{\nabla}}
\newcommand{\grad}[0]{\nabla}
\newcommand{\LL}[0]{\mathcal{L}}

% == \partial_{#1} {#2}
\newcommand{\PD}[2]{\frac{\partial {#2}}{\partial {#1}}}
% inline variant
\newcommand{\PDi}[2]{{\partial {#2}}/{\partial {#1}}}

\newcommand{\PDD}[3]{\frac{\partial^2 {#3}}{\partial {#1}\partial {#2}}}
%\newcommand{\PDd}[2]{\frac{\partial^2 {#2}}{{\partial{#1}}^2}}
\newcommand{\PDsq}[2]{\frac{\partial^2 {#2}}{(\partial {#1})^2}}

\newcommand{\Partial}[2]{\frac{\partial {#1}}{\partial {#2}}}
\DeclareMathOperator{\RejName}{Rej}
\newcommand{\Rej}[2]{\RejName_{#1}\left( {#2} \right)}
\newcommand{\Rm}[1]{\mathbb{R}^{#1}}
\newcommand{\Cm}[1]{\mathbb{C}^{#1}}
\newcommand{\conj}[0]{{*}}

%</misc>

% <grade selection>
%
\newcommand{\gpgrade}[2] {{\left\langle{{#1}}\right\rangle}_{#2}}

\newcommand{\gpgradezero}[1] {\gpgrade{#1}{}}
%\newcommand{\gpscalargrade}[1] {{\left\langle{{#1}}\right\rangle}}
%\newcommand{\gpgradezero}[1] {\gpgrade{#1}{0}}

%\newcommand{\gpgradeone}[1] {{\left\langle{{#1}}\right\rangle}_{1}}
\newcommand{\gpgradeone}[1] {\gpgrade{#1}{1}}

\newcommand{\gpgradetwo}[1] {\gpgrade{#1}{2}}
\newcommand{\gpgradethree}[1] {\gpgrade{#1}{3}}
\newcommand{\gpgradefour}[1] {\gpgrade{#1}{4}}
%
% </grade selection>



\newcommand{\adot}[0]{{\dot{a}}}
\newcommand{\bdot}[0]{{\dot{b}}}
% taken for centered dot:
%\newcommand{\cdot}[0]{{\dot{c}}}
%\newcommand{\ddot}[0]{{\dot{d}}}
\newcommand{\edot}[0]{{\dot{e}}}
\newcommand{\fdot}[0]{{\dot{f}}}
\newcommand{\gdot}[0]{{\dot{g}}}
\newcommand{\hdot}[0]{{\dot{h}}}
\newcommand{\idot}[0]{{\dot{i}}}
\newcommand{\jdot}[0]{{\dot{j}}}
\newcommand{\kdot}[0]{{\dot{k}}}
\newcommand{\ldot}[0]{{\dot{l}}}
\newcommand{\mdot}[0]{{\dot{m}}}
\newcommand{\ndot}[0]{{\dot{n}}}
%\newcommand{\odot}[0]{{\dot{o}}}
\newcommand{\pdot}[0]{{\dot{p}}}
\newcommand{\qdot}[0]{{\dot{q}}}
\newcommand{\rdot}[0]{{\dot{r}}}
\newcommand{\sdot}[0]{{\dot{s}}}
\newcommand{\tdot}[0]{{\dot{t}}}
\newcommand{\udot}[0]{{\dot{u}}}
\newcommand{\vdot}[0]{{\dot{v}}}
\newcommand{\wdot}[0]{{\dot{w}}}
\newcommand{\xdot}[0]{{\dot{x}}}
\newcommand{\ydot}[0]{{\dot{y}}}
\newcommand{\zdot}[0]{{\dot{z}}}
\newcommand{\addot}[0]{{\ddot{a}}}
\newcommand{\bddot}[0]{{\ddot{b}}}
\newcommand{\cddot}[0]{{\ddot{c}}}
%\newcommand{\dddot}[0]{{\ddot{d}}}
\newcommand{\eddot}[0]{{\ddot{e}}}
\newcommand{\fddot}[0]{{\ddot{f}}}
\newcommand{\gddot}[0]{{\ddot{g}}}
\newcommand{\hddot}[0]{{\ddot{h}}}
\newcommand{\iddot}[0]{{\ddot{i}}}
\newcommand{\jddot}[0]{{\ddot{j}}}
\newcommand{\kddot}[0]{{\ddot{k}}}
\newcommand{\lddot}[0]{{\ddot{l}}}
\newcommand{\mddot}[0]{{\ddot{m}}}
\newcommand{\nddot}[0]{{\ddot{n}}}
\newcommand{\oddot}[0]{{\ddot{o}}}
\newcommand{\pddot}[0]{{\ddot{p}}}
\newcommand{\qddot}[0]{{\ddot{q}}}
\newcommand{\rddot}[0]{{\ddot{r}}}
\newcommand{\sddot}[0]{{\ddot{s}}}
\newcommand{\tddot}[0]{{\ddot{t}}}
\newcommand{\uddot}[0]{{\ddot{u}}}
\newcommand{\vddot}[0]{{\ddot{v}}}
\newcommand{\wddot}[0]{{\ddot{w}}}
\newcommand{\xddot}[0]{{\ddot{x}}}
\newcommand{\yddot}[0]{{\ddot{y}}}
\newcommand{\zddot}[0]{{\ddot{z}}}

%<bold and dot greek symbols>
%

\newcommand{\Deltadot}[0]{{\dot{\Delta}}}
\newcommand{\Gammadot}[0]{{\dot{\Gamma}}}
\newcommand{\Lambdadot}[0]{{\dot{\Lambda}}}
\newcommand{\Omegadot}[0]{{\dot{\Omega}}}
\newcommand{\Phidot}[0]{{\dot{\Phi}}}
\newcommand{\Pidot}[0]{{\dot{\Pi}}}
\newcommand{\Psidot}[0]{{\dot{\Psi}}}
\newcommand{\Sigmadot}[0]{{\dot{\Sigma}}}
\newcommand{\Thetadot}[0]{{\dot{\Theta}}}
\newcommand{\Upsilondot}[0]{{\dot{\Upsilon}}}
\newcommand{\Xidot}[0]{{\dot{\Xi}}}
\newcommand{\alphadot}[0]{{\dot{\alpha}}}
\newcommand{\betadot}[0]{{\dot{\beta}}}
\newcommand{\chidot}[0]{{\dot{\chi}}}
\newcommand{\deltadot}[0]{{\dot{\delta}}}
\newcommand{\epsilondot}[0]{{\dot{\epsilon}}}
\newcommand{\etadot}[0]{{\dot{\eta}}}
\newcommand{\gammadot}[0]{{\dot{\gamma}}}
\newcommand{\kappadot}[0]{{\dot{\kappa}}}
\newcommand{\lambdadot}[0]{{\dot{\lambda}}}
\newcommand{\mudot}[0]{{\dot{\mu}}}
\newcommand{\nudot}[0]{{\dot{\nu}}}
\newcommand{\omegadot}[0]{{\dot{\omega}}}
\newcommand{\phidot}[0]{{\dot{\phi}}}
\newcommand{\pidot}[0]{{\dot{\pi}}}
\newcommand{\psidot}[0]{{\dot{\psi}}}
\newcommand{\rhodot}[0]{{\dot{\rho}}}
\newcommand{\sigmadot}[0]{{\dot{\sigma}}}
\newcommand{\taudot}[0]{{\dot{\tau}}}
\newcommand{\thetadot}[0]{{\dot{\theta}}}
\newcommand{\upsilondot}[0]{{\dot{\upsilon}}}
\newcommand{\varepsilondot}[0]{{\dot{\varepsilon}}}
\newcommand{\varphidot}[0]{{\dot{\varphi}}}
\newcommand{\varpidot}[0]{{\dot{\varpi}}}
\newcommand{\varrhodot}[0]{{\dot{\varrho}}}
\newcommand{\varsigmadot}[0]{{\dot{\varsigma}}}
\newcommand{\varthetadot}[0]{{\dot{\vartheta}}}
\newcommand{\xidot}[0]{{\dot{\xi}}}
\newcommand{\zetadot}[0]{{\dot{\zeta}}}

\newcommand{\Deltaddot}[0]{{\ddot{\Delta}}}
\newcommand{\Gammaddot}[0]{{\ddot{\Gamma}}}
\newcommand{\Lambdaddot}[0]{{\ddot{\Lambda}}}
\newcommand{\Omegaddot}[0]{{\ddot{\Omega}}}
\newcommand{\Phiddot}[0]{{\ddot{\Phi}}}
\newcommand{\Piddot}[0]{{\ddot{\Pi}}}
\newcommand{\Psiddot}[0]{{\ddot{\Psi}}}
\newcommand{\Sigmaddot}[0]{{\ddot{\Sigma}}}
\newcommand{\Thetaddot}[0]{{\ddot{\Theta}}}
\newcommand{\Upsilonddot}[0]{{\ddot{\Upsilon}}}
\newcommand{\Xiddot}[0]{{\ddot{\Xi}}}
\newcommand{\alphaddot}[0]{{\ddot{\alpha}}}
\newcommand{\betaddot}[0]{{\ddot{\beta}}}
\newcommand{\chiddot}[0]{{\ddot{\chi}}}
\newcommand{\deltaddot}[0]{{\ddot{\delta}}}
\newcommand{\epsilonddot}[0]{{\ddot{\epsilon}}}
\newcommand{\etaddot}[0]{{\ddot{\eta}}}
\newcommand{\gammaddot}[0]{{\ddot{\gamma}}}
\newcommand{\kappaddot}[0]{{\ddot{\kappa}}}
\newcommand{\lambdaddot}[0]{{\ddot{\lambda}}}
\newcommand{\muddot}[0]{{\ddot{\mu}}}
\newcommand{\nuddot}[0]{{\ddot{\nu}}}
\newcommand{\omegaddot}[0]{{\ddot{\omega}}}
\newcommand{\phiddot}[0]{{\ddot{\phi}}}
\newcommand{\piddot}[0]{{\ddot{\pi}}}
\newcommand{\psiddot}[0]{{\ddot{\psi}}}
\newcommand{\rhoddot}[0]{{\ddot{\rho}}}
\newcommand{\sigmaddot}[0]{{\ddot{\sigma}}}
\newcommand{\tauddot}[0]{{\ddot{\tau}}}
\newcommand{\thetaddot}[0]{{\ddot{\theta}}}
\newcommand{\upsilonddot}[0]{{\ddot{\upsilon}}}
\newcommand{\varepsilonddot}[0]{{\ddot{\varepsilon}}}
\newcommand{\varphiddot}[0]{{\ddot{\varphi}}}
\newcommand{\varpiddot}[0]{{\ddot{\varpi}}}
\newcommand{\varrhoddot}[0]{{\ddot{\varrho}}}
\newcommand{\varsigmaddot}[0]{{\ddot{\varsigma}}}
\newcommand{\varthetaddot}[0]{{\ddot{\vartheta}}}
\newcommand{\xiddot}[0]{{\ddot{\xi}}}
\newcommand{\zetaddot}[0]{{\ddot{\zeta}}}

\newcommand{\BDelta}[0]{\boldsymbol{\Delta}}
\newcommand{\BGamma}[0]{\boldsymbol{\Gamma}}
\newcommand{\BLambda}[0]{\boldsymbol{\Lambda}}
\newcommand{\BOmega}[0]{\boldsymbol{\Omega}}
\newcommand{\BPhi}[0]{\boldsymbol{\Phi}}
\newcommand{\BPi}[0]{\boldsymbol{\Pi}}
\newcommand{\BPsi}[0]{\boldsymbol{\Psi}}
\newcommand{\BSigma}[0]{\boldsymbol{\Sigma}}
\newcommand{\BTheta}[0]{\boldsymbol{\Theta}}
\newcommand{\BUpsilon}[0]{\boldsymbol{\Upsilon}}
\newcommand{\BXi}[0]{\boldsymbol{\Xi}}
\newcommand{\Balpha}[0]{\boldsymbol{\alpha}}
\newcommand{\Bbeta}[0]{\boldsymbol{\beta}}
\newcommand{\Bchi}[0]{\boldsymbol{\chi}}
\newcommand{\Bdelta}[0]{\boldsymbol{\delta}}
\newcommand{\Bepsilon}[0]{\boldsymbol{\epsilon}}
\newcommand{\Beta}[0]{\boldsymbol{\eta}}
\newcommand{\Bgamma}[0]{\boldsymbol{\gamma}}
\newcommand{\Bkappa}[0]{\boldsymbol{\kappa}}
\newcommand{\Blambda}[0]{\boldsymbol{\lambda}}
\newcommand{\Bmu}[0]{\boldsymbol{\mu}}
\newcommand{\Bnu}[0]{\boldsymbol{\nu}}
%\newcommand{\Bomega}[0]{\boldsymbol{\omega}}
\newcommand{\Bphi}[0]{\boldsymbol{\phi}}
\newcommand{\Bpi}[0]{\boldsymbol{\pi}}
\newcommand{\Bpsi}[0]{\boldsymbol{\psi}}
\newcommand{\Brho}[0]{\boldsymbol{\rho}}
\newcommand{\Bsigma}[0]{\boldsymbol{\sigma}}
%\newcommand{\Btau}[0]{\boldsymbol{\tau}}
%\newcommand{\Btheta}[0]{\boldsymbol{\theta}}
\newcommand{\Bupsilon}[0]{\boldsymbol{\upsilon}}
\newcommand{\Bvarepsilon}[0]{\boldsymbol{\varepsilon}}
\newcommand{\Bvarphi}[0]{\boldsymbol{\varphi}}
\newcommand{\Bvarpi}[0]{\boldsymbol{\varpi}}
\newcommand{\Bvarrho}[0]{\boldsymbol{\varrho}}
\newcommand{\Bvarsigma}[0]{\boldsymbol{\varsigma}}
\newcommand{\Bvartheta}[0]{\boldsymbol{\vartheta}}
\newcommand{\Bxi}[0]{\boldsymbol{\xi}}
\newcommand{\Bzeta}[0]{\boldsymbol{\zeta}}
%
%</bold and dot greek symbols>
%<infrequent>
%
%\newcommand{\AreaOp}[1]{\AName_{#1}}
%\newcommand{\Babs}[0]{\abs{\BB}}
%\newcommand{\Bcap}[0]{\hat{\BB}}
%\newcommand{\BrPrimeRej}[0]{\rcap(\rcap \wedge \Br')}
%\newcommand{\CA}[0]{\mathcal{A}}
%\newcommand{\Cos}[1]{\cos{\left({#1}\right)}}
%\newcommand{\Det}[1] {\abs{#1}}
%\newcommand{\Dsq}[2] {\frac {\partial^2 {#1}} {\partial {#2}^2}}
%\newcommand{\Exp}[1]{\exp{\left({#1}\right)}}
%\newcommand{\Norm}[1]{\left\lVert{#1}\right\rVert}
%\newcommand{\Sin}[1]{\sin{\left({#1}\right)}}
%\newcommand{\T}[0]{\text{T}}
%\newcommand{\VolumeOp}[1]{\VName_{#1}}
%\newcommand{\agrad}[0]{\Ba \cdot \nabla}
%\newcommand{\alphacap}[0]{\hat{\boldsymbol{\alpha}}}
%\newcommand{\Fcap}[0]{\hat{\BF}}
%\newcommand{\bithree}[0]{{\Bi}_3}
%\newcommand{\bxa}[0]{\Bx\Ba}
%\newcommand{\coordvec}[2]{
%\newcommand{\costheta}[0]{\acap \cdot \xcap}
%\newcommand{\ddt}[1]{\ddot{#1}}
%\newcommand{\ddu}[1] {\frac {d{#1}} {du}}
%\newcommand{\dsqxj}[2] {\frac {\partial^2 {#1}} {\partial {x_{#2}}^2}}
%\newcommand{\dtheta}[1]{\frac{d {#1}}{d \theta}}
%\newcommand{\dt}[1]{\dot{#1}}
%\newcommand{\dt}[1]{\frac{d {#1}}{dt}}
%\newcommand{\dxj}[2] {\frac {\partial {#1}} {\partial {x_{#2}}}}
%\newcommand{\halfPhi}[0]{\frac{\phi}{2}}
%\newcommand{\half}[0]{\inv{2}}
%\newcommand{\inv}[1]{\frac{1}{#1}}
%\newcommand{\laplacian}[0]{\nabla^2}
%\newcommand{\matrixoftx}[3]{
%\newcommand{\nrrp}[0]{\norm{\rcap \wedge \Br'}}
%\newcommand{\oiint}{\bigcirc \hspace{-1.4em} \int \hspace{-.8em} \int}
%\newcommand{\transpose}[1]{{#1}^{\text{T}}}
%\newcommand{\transpose}[1]{{{#1}^{\TextTranspose}}}
%\newcommand{\transpose}[1]{{{#1}^{\text{T}}}}
%\newcommand{\barA}[0]{\bar{A}}
%\newcommand{\qbar}[0]{\bar{q}}
%\newcommand{\qdotbar}[0]{\dot{\bar{q}}}
%
%</infrequent>





\usepackage[bookmarks=true]{hyperref}

\usepackage{color,cite,graphicx}
   % use colour in the document, put your citations as [1-4]
   % rather than [1,2,3,4] (it looks nicer, and the extended LaTeX2e
   % graphics package. 
\usepackage{latexsym,amssymb,epsf} % don't remember if these are
   % needed, but their inclusion can't do any damage


\title{ Field due to line charge in arc.}
\author{Peeter Joot}
\date{ Nov 23, 2008.  Last Revision: $Date: 2008/11/24 14:52:43 $ }

\begin{document}

\maketitle{}

%\tableofcontents

\section{ Motivation. }

Problem $1.5$ from \cite{purcell1963eam}, is to calculate the field
at the center of a half circular arc of line charge.  Do this calculation
and setup for the calculation at other points.

\section{ Calculation. }

%\begin{figure}[htp]
%\centering
%\includegraphics[totalheight=0.4\textheight]{picturepath}
%\caption{My Caption}\label{fig:pictlabel}
%\end{figure}
%
%... see figure \ref{fig:picturepath} ...

\bibliographystyle{plainnat}
\bibliography{myrefs}

\end{document}

\documentclass{article}

\usepackage{amsmath}
\usepackage{mathpazo}

%
% shorthand for bold symbols, convenient for vectors and matrices
%
\newcommand{\Ba}[0]{\mathbf{a}}
\newcommand{\Bb}[0]{\mathbf{b}}
\newcommand{\Bc}[0]{\mathbf{c}}
\newcommand{\Bd}[0]{\mathbf{d}}
\newcommand{\Be}[0]{\mathbf{e}}
\newcommand{\Bf}[0]{\mathbf{f}}
\newcommand{\Bg}[0]{\mathbf{g}}
\newcommand{\Bh}[0]{\mathbf{h}}
\newcommand{\Bi}[0]{\mathbf{i}}
\newcommand{\Bj}[0]{\mathbf{j}}
\newcommand{\Bk}[0]{\mathbf{k}}
\newcommand{\Bl}[0]{\mathbf{l}}
\newcommand{\Bm}[0]{\mathbf{m}}
\newcommand{\Bn}[0]{\mathbf{n}}
\newcommand{\Bo}[0]{\mathbf{o}}
\newcommand{\Bp}[0]{\mathbf{p}}
\newcommand{\Bq}[0]{\mathbf{q}}
\newcommand{\Br}[0]{\mathbf{r}}
\newcommand{\Bs}[0]{\mathbf{s}}
\newcommand{\Bt}[0]{\mathbf{t}}
\newcommand{\Bu}[0]{\mathbf{u}}
\newcommand{\Bv}[0]{\mathbf{v}}
\newcommand{\Bw}[0]{\mathbf{w}}
\newcommand{\Bx}[0]{\mathbf{x}}
\newcommand{\By}[0]{\mathbf{y}}
\newcommand{\Bz}[0]{\mathbf{z}}
\newcommand{\BA}[0]{\mathbf{A}}
\newcommand{\BB}[0]{\mathbf{B}}
\newcommand{\BC}[0]{\mathbf{C}}
\newcommand{\BD}[0]{\mathbf{D}}
\newcommand{\BE}[0]{\mathbf{E}}
\newcommand{\BF}[0]{\mathbf{F}}
\newcommand{\BG}[0]{\mathbf{G}}
\newcommand{\BH}[0]{\mathbf{H}}
\newcommand{\BI}[0]{\mathbf{I}}
\newcommand{\BJ}[0]{\mathbf{J}}
\newcommand{\BK}[0]{\mathbf{K}}
\newcommand{\BL}[0]{\mathbf{L}}
\newcommand{\BM}[0]{\mathbf{M}}
\newcommand{\BN}[0]{\mathbf{N}}
\newcommand{\BO}[0]{\mathbf{O}}
\newcommand{\BP}[0]{\mathbf{P}}
\newcommand{\BQ}[0]{\mathbf{Q}}
\newcommand{\BR}[0]{\mathbf{R}}
\newcommand{\BS}[0]{\mathbf{S}}
\newcommand{\BT}[0]{\mathbf{T}}
\newcommand{\BU}[0]{\mathbf{U}}
\newcommand{\BV}[0]{\mathbf{V}}
\newcommand{\BW}[0]{\mathbf{W}}
\newcommand{\BX}[0]{\mathbf{X}}
\newcommand{\BY}[0]{\mathbf{Y}}
\newcommand{\BZ}[0]{\mathbf{Z}}

\newcommand{\Bzero}[0]{\mathbf{0}}
\newcommand{\Btheta}[0]{\boldsymbol{\theta}}
\newcommand{\Btau}[0]{\boldsymbol{\tau}}
\newcommand{\Bomega}[0]{\boldsymbol{\omega}}

%
% shorthand for unit vectors
%
\newcommand{\acap}[0]{\hat{\Ba}}
\newcommand{\bcap}[0]{\hat{\Bb}}
\newcommand{\ccap}[0]{\hat{\Bc}}
\newcommand{\dcap}[0]{\hat{\Bd}}
\newcommand{\ecap}[0]{\hat{\Be}}
\newcommand{\fcap}[0]{\hat{\Bf}}
\newcommand{\gcap}[0]{\hat{\Bg}}
\newcommand{\hcap}[0]{\hat{\Bh}}
\newcommand{\icap}[0]{\hat{\Bi}}
\newcommand{\jcap}[0]{\hat{\Bj}}
\newcommand{\kcap}[0]{\hat{\Bk}}
\newcommand{\lcap}[0]{\hat{\Bl}}
\newcommand{\mcap}[0]{\hat{\Bm}}
\newcommand{\ncap}[0]{\hat{\Bn}}
\newcommand{\ocap}[0]{\hat{\Bo}}
\newcommand{\pcap}[0]{\hat{\Bp}}
\newcommand{\qcap}[0]{\hat{\Bq}}
\newcommand{\rcap}[0]{\hat{\Br}}
\newcommand{\scap}[0]{\hat{\Bs}}
\newcommand{\tcap}[0]{\hat{\Bt}}
\newcommand{\ucap}[0]{\hat{\Bu}}
\newcommand{\vcap}[0]{\hat{\Bv}}
\newcommand{\wcap}[0]{\hat{\Bw}}
\newcommand{\xcap}[0]{\hat{\Bx}}
\newcommand{\ycap}[0]{\hat{\By}}
\newcommand{\zcap}[0]{\hat{\Bz}}
\newcommand{\thetacap}[0]{\hat{\Btheta}}

%
% to write R^n and C^n in a distinguishable fashion.  Perhaps change this
% to the double lined characters upon figuring out how to do so.
%
\newcommand{\C}[1]{$\mathbb{C}^{#1}$}
\newcommand{\R}[1]{$\mathbb{R}^{#1}$}

%
% various generally useful helpers
%

% derivative of #1 wrt. #2:
\newcommand{\D}[2] {\frac {d#2} {d#1}}

\newcommand{\inv}[1]{\frac{1}{#1}}
\newcommand{\cross}[0]{\times}

\newcommand{\abs}[1]{\lvert{#1}\rvert}
\newcommand{\norm}[1]{\lVert{#1}\rVert}
\newcommand{\innerprod}[2]{\langle{#1}, {#2}\rangle}
\newcommand{\dotprod}[2]{{#1} \cdot {#2}}
\newcommand{\bdotprod}[2]{\left({#1} \cdot {#2}\right)}
\newcommand{\crossprod}[2]{{#1} \cross {#2}}
\newcommand{\tripleprod}[3]{\dotprod{\left(\crossprod{#1}{#2}\right)}{#3}}

\DeclareMathOperator{\Proj}{Proj}
\DeclareMathOperator{\Span}{span}
\DeclareMathOperator{\Sgn}{sgn}
\DeclareMathOperator{\Area}{Area}
\DeclareMathOperator{\Volume}{Volume}

%
% A few miscellaneous things specific to this document
%
\newcommand{\crossop}[1]{\crossprod{#1}{}}

% R2 vector.
\newcommand{\VectorTwo}[2]{
\begin{bmatrix}
 {#1} \\
 {#2}
\end{bmatrix}
}

\newcommand{\VectorN}[1]{
\begin{bmatrix}
{#1}_1 \\
{#1}_2 \\
\vdots \\
{#1}_N \\
\end{bmatrix}
}

\newcommand{\DETuvij}[4]{
\begin{vmatrix}
 {#1}_{#3} & {#1}_{#4} \\
 {#2}_{#3} & {#2}_{#4}
\end{vmatrix}
}

\newcommand{\DETuvwijk}[6]{
\begin{vmatrix}
 {#1}_{#4} & {#1}_{#5} & {#1}_{#6} \\
 {#2}_{#4} & {#2}_{#5} & {#2}_{#6} \\
 {#3}_{#4} & {#3}_{#5} & {#3}_{#6}
\end{vmatrix}
}

\newcommand{\DETuvwxijkl}[8]{
\begin{vmatrix}
 {#1}_{#5} & {#1}_{#6} & {#1}_{#7} & {#1}_{#8} \\
 {#2}_{#5} & {#2}_{#6} & {#2}_{#7} & {#2}_{#8} \\
 {#3}_{#5} & {#3}_{#6} & {#3}_{#7} & {#3}_{#8} \\
 {#4}_{#5} & {#4}_{#6} & {#4}_{#7} & {#4}_{#8} \\
\end{vmatrix}
}

%\newcommand{\DETuvwxyijklm}[10]{
%\begin{vmatrix}
% {#1}_{#6} & {#1}_{#7} & {#1}_{#8} & {#1}_{#9} & {#1}_{#10} \\
% {#2}_{#6} & {#2}_{#7} & {#2}_{#8} & {#2}_{#9} & {#2}_{#10} \\
% {#3}_{#6} & {#3}_{#7} & {#3}_{#8} & {#3}_{#9} & {#3}_{#10} \\
% {#4}_{#6} & {#4}_{#7} & {#4}_{#8} & {#4}_{#9} & {#4}_{#10} \\
% {#5}_{#6} & {#5}_{#7} & {#5}_{#8} & {#5}_{#9} & {#5}_{#10}
%\end{vmatrix}
%}

% R3 vector.
\newcommand{\VectorThree}[3]{
\begin{bmatrix}
 {#1} \\
 {#2} \\
 {#3}
\end{bmatrix}
}


%<misc>
%
\newcommand{\Abs}[1]{{\left\lvert{#1}\right\rvert}}
\newcommand{\spacegrad}[0]{\boldsymbol{\nabla}}
\newcommand{\grad}[0]{\nabla}
\newcommand{\LL}[0]{\mathcal{L}}

% == \partial_{#1} {#2}
\newcommand{\PD}[2]{\frac{\partial {#2}}{\partial {#1}}}
% inline variant
\newcommand{\PDi}[2]{{\partial {#2}}/{\partial {#1}}}

\newcommand{\PDD}[3]{\frac{\partial^2 {#3}}{\partial {#1}\partial {#2}}}
%\newcommand{\PDd}[2]{\frac{\partial^2 {#2}}{{\partial{#1}}^2}}
\newcommand{\PDsq}[2]{\frac{\partial^2 {#2}}{(\partial {#1})^2}}

\newcommand{\Partial}[2]{\frac{\partial {#1}}{\partial {#2}}}
\DeclareMathOperator{\RejName}{Rej}
\newcommand{\Rej}[2]{\RejName_{#1}\left( {#2} \right)}
\newcommand{\Rm}[1]{\mathbb{R}^{#1}}
\newcommand{\Cm}[1]{\mathbb{C}^{#1}}
\newcommand{\conj}[0]{{*}}

%</misc>

% <grade selection>
%
\newcommand{\gpgrade}[2] {{\left\langle{{#1}}\right\rangle}_{#2}}

\newcommand{\gpgradezero}[1] {\gpgrade{#1}{}}
%\newcommand{\gpscalargrade}[1] {{\left\langle{{#1}}\right\rangle}}
%\newcommand{\gpgradezero}[1] {\gpgrade{#1}{0}}

%\newcommand{\gpgradeone}[1] {{\left\langle{{#1}}\right\rangle}_{1}}
\newcommand{\gpgradeone}[1] {\gpgrade{#1}{1}}

\newcommand{\gpgradetwo}[1] {\gpgrade{#1}{2}}
\newcommand{\gpgradethree}[1] {\gpgrade{#1}{3}}
\newcommand{\gpgradefour}[1] {\gpgrade{#1}{4}}
%
% </grade selection>



\newcommand{\adot}[0]{{\dot{a}}}
\newcommand{\bdot}[0]{{\dot{b}}}
% taken for centered dot:
%\newcommand{\cdot}[0]{{\dot{c}}}
%\newcommand{\ddot}[0]{{\dot{d}}}
\newcommand{\edot}[0]{{\dot{e}}}
\newcommand{\fdot}[0]{{\dot{f}}}
\newcommand{\gdot}[0]{{\dot{g}}}
\newcommand{\hdot}[0]{{\dot{h}}}
\newcommand{\idot}[0]{{\dot{i}}}
\newcommand{\jdot}[0]{{\dot{j}}}
\newcommand{\kdot}[0]{{\dot{k}}}
\newcommand{\ldot}[0]{{\dot{l}}}
\newcommand{\mdot}[0]{{\dot{m}}}
\newcommand{\ndot}[0]{{\dot{n}}}
%\newcommand{\odot}[0]{{\dot{o}}}
\newcommand{\pdot}[0]{{\dot{p}}}
\newcommand{\qdot}[0]{{\dot{q}}}
\newcommand{\rdot}[0]{{\dot{r}}}
\newcommand{\sdot}[0]{{\dot{s}}}
\newcommand{\tdot}[0]{{\dot{t}}}
\newcommand{\udot}[0]{{\dot{u}}}
\newcommand{\vdot}[0]{{\dot{v}}}
\newcommand{\wdot}[0]{{\dot{w}}}
\newcommand{\xdot}[0]{{\dot{x}}}
\newcommand{\ydot}[0]{{\dot{y}}}
\newcommand{\zdot}[0]{{\dot{z}}}
\newcommand{\addot}[0]{{\ddot{a}}}
\newcommand{\bddot}[0]{{\ddot{b}}}
\newcommand{\cddot}[0]{{\ddot{c}}}
%\newcommand{\dddot}[0]{{\ddot{d}}}
\newcommand{\eddot}[0]{{\ddot{e}}}
\newcommand{\fddot}[0]{{\ddot{f}}}
\newcommand{\gddot}[0]{{\ddot{g}}}
\newcommand{\hddot}[0]{{\ddot{h}}}
\newcommand{\iddot}[0]{{\ddot{i}}}
\newcommand{\jddot}[0]{{\ddot{j}}}
\newcommand{\kddot}[0]{{\ddot{k}}}
\newcommand{\lddot}[0]{{\ddot{l}}}
\newcommand{\mddot}[0]{{\ddot{m}}}
\newcommand{\nddot}[0]{{\ddot{n}}}
\newcommand{\oddot}[0]{{\ddot{o}}}
\newcommand{\pddot}[0]{{\ddot{p}}}
\newcommand{\qddot}[0]{{\ddot{q}}}
\newcommand{\rddot}[0]{{\ddot{r}}}
\newcommand{\sddot}[0]{{\ddot{s}}}
\newcommand{\tddot}[0]{{\ddot{t}}}
\newcommand{\uddot}[0]{{\ddot{u}}}
\newcommand{\vddot}[0]{{\ddot{v}}}
\newcommand{\wddot}[0]{{\ddot{w}}}
\newcommand{\xddot}[0]{{\ddot{x}}}
\newcommand{\yddot}[0]{{\ddot{y}}}
\newcommand{\zddot}[0]{{\ddot{z}}}

%<bold and dot greek symbols>
%

\newcommand{\Deltadot}[0]{{\dot{\Delta}}}
\newcommand{\Gammadot}[0]{{\dot{\Gamma}}}
\newcommand{\Lambdadot}[0]{{\dot{\Lambda}}}
\newcommand{\Omegadot}[0]{{\dot{\Omega}}}
\newcommand{\Phidot}[0]{{\dot{\Phi}}}
\newcommand{\Pidot}[0]{{\dot{\Pi}}}
\newcommand{\Psidot}[0]{{\dot{\Psi}}}
\newcommand{\Sigmadot}[0]{{\dot{\Sigma}}}
\newcommand{\Thetadot}[0]{{\dot{\Theta}}}
\newcommand{\Upsilondot}[0]{{\dot{\Upsilon}}}
\newcommand{\Xidot}[0]{{\dot{\Xi}}}
\newcommand{\alphadot}[0]{{\dot{\alpha}}}
\newcommand{\betadot}[0]{{\dot{\beta}}}
\newcommand{\chidot}[0]{{\dot{\chi}}}
\newcommand{\deltadot}[0]{{\dot{\delta}}}
\newcommand{\epsilondot}[0]{{\dot{\epsilon}}}
\newcommand{\etadot}[0]{{\dot{\eta}}}
\newcommand{\gammadot}[0]{{\dot{\gamma}}}
\newcommand{\kappadot}[0]{{\dot{\kappa}}}
\newcommand{\lambdadot}[0]{{\dot{\lambda}}}
\newcommand{\mudot}[0]{{\dot{\mu}}}
\newcommand{\nudot}[0]{{\dot{\nu}}}
\newcommand{\omegadot}[0]{{\dot{\omega}}}
\newcommand{\phidot}[0]{{\dot{\phi}}}
\newcommand{\pidot}[0]{{\dot{\pi}}}
\newcommand{\psidot}[0]{{\dot{\psi}}}
\newcommand{\rhodot}[0]{{\dot{\rho}}}
\newcommand{\sigmadot}[0]{{\dot{\sigma}}}
\newcommand{\taudot}[0]{{\dot{\tau}}}
\newcommand{\thetadot}[0]{{\dot{\theta}}}
\newcommand{\upsilondot}[0]{{\dot{\upsilon}}}
\newcommand{\varepsilondot}[0]{{\dot{\varepsilon}}}
\newcommand{\varphidot}[0]{{\dot{\varphi}}}
\newcommand{\varpidot}[0]{{\dot{\varpi}}}
\newcommand{\varrhodot}[0]{{\dot{\varrho}}}
\newcommand{\varsigmadot}[0]{{\dot{\varsigma}}}
\newcommand{\varthetadot}[0]{{\dot{\vartheta}}}
\newcommand{\xidot}[0]{{\dot{\xi}}}
\newcommand{\zetadot}[0]{{\dot{\zeta}}}

\newcommand{\Deltaddot}[0]{{\ddot{\Delta}}}
\newcommand{\Gammaddot}[0]{{\ddot{\Gamma}}}
\newcommand{\Lambdaddot}[0]{{\ddot{\Lambda}}}
\newcommand{\Omegaddot}[0]{{\ddot{\Omega}}}
\newcommand{\Phiddot}[0]{{\ddot{\Phi}}}
\newcommand{\Piddot}[0]{{\ddot{\Pi}}}
\newcommand{\Psiddot}[0]{{\ddot{\Psi}}}
\newcommand{\Sigmaddot}[0]{{\ddot{\Sigma}}}
\newcommand{\Thetaddot}[0]{{\ddot{\Theta}}}
\newcommand{\Upsilonddot}[0]{{\ddot{\Upsilon}}}
\newcommand{\Xiddot}[0]{{\ddot{\Xi}}}
\newcommand{\alphaddot}[0]{{\ddot{\alpha}}}
\newcommand{\betaddot}[0]{{\ddot{\beta}}}
\newcommand{\chiddot}[0]{{\ddot{\chi}}}
\newcommand{\deltaddot}[0]{{\ddot{\delta}}}
\newcommand{\epsilonddot}[0]{{\ddot{\epsilon}}}
\newcommand{\etaddot}[0]{{\ddot{\eta}}}
\newcommand{\gammaddot}[0]{{\ddot{\gamma}}}
\newcommand{\kappaddot}[0]{{\ddot{\kappa}}}
\newcommand{\lambdaddot}[0]{{\ddot{\lambda}}}
\newcommand{\muddot}[0]{{\ddot{\mu}}}
\newcommand{\nuddot}[0]{{\ddot{\nu}}}
\newcommand{\omegaddot}[0]{{\ddot{\omega}}}
\newcommand{\phiddot}[0]{{\ddot{\phi}}}
\newcommand{\piddot}[0]{{\ddot{\pi}}}
\newcommand{\psiddot}[0]{{\ddot{\psi}}}
\newcommand{\rhoddot}[0]{{\ddot{\rho}}}
\newcommand{\sigmaddot}[0]{{\ddot{\sigma}}}
\newcommand{\tauddot}[0]{{\ddot{\tau}}}
\newcommand{\thetaddot}[0]{{\ddot{\theta}}}
\newcommand{\upsilonddot}[0]{{\ddot{\upsilon}}}
\newcommand{\varepsilonddot}[0]{{\ddot{\varepsilon}}}
\newcommand{\varphiddot}[0]{{\ddot{\varphi}}}
\newcommand{\varpiddot}[0]{{\ddot{\varpi}}}
\newcommand{\varrhoddot}[0]{{\ddot{\varrho}}}
\newcommand{\varsigmaddot}[0]{{\ddot{\varsigma}}}
\newcommand{\varthetaddot}[0]{{\ddot{\vartheta}}}
\newcommand{\xiddot}[0]{{\ddot{\xi}}}
\newcommand{\zetaddot}[0]{{\ddot{\zeta}}}

\newcommand{\BDelta}[0]{\boldsymbol{\Delta}}
\newcommand{\BGamma}[0]{\boldsymbol{\Gamma}}
\newcommand{\BLambda}[0]{\boldsymbol{\Lambda}}
\newcommand{\BOmega}[0]{\boldsymbol{\Omega}}
\newcommand{\BPhi}[0]{\boldsymbol{\Phi}}
\newcommand{\BPi}[0]{\boldsymbol{\Pi}}
\newcommand{\BPsi}[0]{\boldsymbol{\Psi}}
\newcommand{\BSigma}[0]{\boldsymbol{\Sigma}}
\newcommand{\BTheta}[0]{\boldsymbol{\Theta}}
\newcommand{\BUpsilon}[0]{\boldsymbol{\Upsilon}}
\newcommand{\BXi}[0]{\boldsymbol{\Xi}}
\newcommand{\Balpha}[0]{\boldsymbol{\alpha}}
\newcommand{\Bbeta}[0]{\boldsymbol{\beta}}
\newcommand{\Bchi}[0]{\boldsymbol{\chi}}
\newcommand{\Bdelta}[0]{\boldsymbol{\delta}}
\newcommand{\Bepsilon}[0]{\boldsymbol{\epsilon}}
\newcommand{\Beta}[0]{\boldsymbol{\eta}}
\newcommand{\Bgamma}[0]{\boldsymbol{\gamma}}
\newcommand{\Bkappa}[0]{\boldsymbol{\kappa}}
\newcommand{\Blambda}[0]{\boldsymbol{\lambda}}
\newcommand{\Bmu}[0]{\boldsymbol{\mu}}
\newcommand{\Bnu}[0]{\boldsymbol{\nu}}
%\newcommand{\Bomega}[0]{\boldsymbol{\omega}}
\newcommand{\Bphi}[0]{\boldsymbol{\phi}}
\newcommand{\Bpi}[0]{\boldsymbol{\pi}}
\newcommand{\Bpsi}[0]{\boldsymbol{\psi}}
\newcommand{\Brho}[0]{\boldsymbol{\rho}}
\newcommand{\Bsigma}[0]{\boldsymbol{\sigma}}
%\newcommand{\Btau}[0]{\boldsymbol{\tau}}
%\newcommand{\Btheta}[0]{\boldsymbol{\theta}}
\newcommand{\Bupsilon}[0]{\boldsymbol{\upsilon}}
\newcommand{\Bvarepsilon}[0]{\boldsymbol{\varepsilon}}
\newcommand{\Bvarphi}[0]{\boldsymbol{\varphi}}
\newcommand{\Bvarpi}[0]{\boldsymbol{\varpi}}
\newcommand{\Bvarrho}[0]{\boldsymbol{\varrho}}
\newcommand{\Bvarsigma}[0]{\boldsymbol{\varsigma}}
\newcommand{\Bvartheta}[0]{\boldsymbol{\vartheta}}
\newcommand{\Bxi}[0]{\boldsymbol{\xi}}
\newcommand{\Bzeta}[0]{\boldsymbol{\zeta}}
%
%</bold and dot greek symbols>
%<infrequent>
%
%\newcommand{\AreaOp}[1]{\AName_{#1}}
%\newcommand{\Babs}[0]{\abs{\BB}}
%\newcommand{\Bcap}[0]{\hat{\BB}}
%\newcommand{\BrPrimeRej}[0]{\rcap(\rcap \wedge \Br')}
%\newcommand{\CA}[0]{\mathcal{A}}
%\newcommand{\Cos}[1]{\cos{\left({#1}\right)}}
%\newcommand{\Det}[1] {\abs{#1}}
%\newcommand{\Dsq}[2] {\frac {\partial^2 {#1}} {\partial {#2}^2}}
%\newcommand{\Exp}[1]{\exp{\left({#1}\right)}}
%\newcommand{\Norm}[1]{\left\lVert{#1}\right\rVert}
%\newcommand{\Sin}[1]{\sin{\left({#1}\right)}}
%\newcommand{\T}[0]{\text{T}}
%\newcommand{\VolumeOp}[1]{\VName_{#1}}
%\newcommand{\agrad}[0]{\Ba \cdot \nabla}
%\newcommand{\alphacap}[0]{\hat{\boldsymbol{\alpha}}}
%\newcommand{\Fcap}[0]{\hat{\BF}}
%\newcommand{\bithree}[0]{{\Bi}_3}
%\newcommand{\bxa}[0]{\Bx\Ba}
%\newcommand{\coordvec}[2]{
%\newcommand{\costheta}[0]{\acap \cdot \xcap}
%\newcommand{\ddt}[1]{\ddot{#1}}
%\newcommand{\ddu}[1] {\frac {d{#1}} {du}}
%\newcommand{\dsqxj}[2] {\frac {\partial^2 {#1}} {\partial {x_{#2}}^2}}
%\newcommand{\dtheta}[1]{\frac{d {#1}}{d \theta}}
%\newcommand{\dt}[1]{\dot{#1}}
%\newcommand{\dt}[1]{\frac{d {#1}}{dt}}
%\newcommand{\dxj}[2] {\frac {\partial {#1}} {\partial {x_{#2}}}}
%\newcommand{\halfPhi}[0]{\frac{\phi}{2}}
%\newcommand{\half}[0]{\inv{2}}
%\newcommand{\inv}[1]{\frac{1}{#1}}
%\newcommand{\laplacian}[0]{\nabla^2}
%\newcommand{\matrixoftx}[3]{
%\newcommand{\nrrp}[0]{\norm{\rcap \wedge \Br'}}
%\newcommand{\oiint}{\bigcirc \hspace{-1.4em} \int \hspace{-.8em} \int}
%\newcommand{\transpose}[1]{{#1}^{\text{T}}}
%\newcommand{\transpose}[1]{{{#1}^{\TextTranspose}}}
%\newcommand{\transpose}[1]{{{#1}^{\text{T}}}}
%\newcommand{\barA}[0]{\bar{A}}
%\newcommand{\qbar}[0]{\bar{q}}
%\newcommand{\qdotbar}[0]{\dot{\bar{q}}}
%
%</infrequent>





\usepackage[bookmarks=true]{hyperref}

\usepackage{color,cite,graphicx}
   % use colour in the document, put your citations as [1-4]
   % rather than [1,2,3,4] (it looks nicer, and the extended LaTeX2e
   % graphics package. 
\usepackage{latexsym,amssymb,epsf} % don't remember if these are
   % needed, but their inclusion can't do any damage


\title{ Charge line element. }
\author{Peeter Joot \quad peeter.joot@gmail.com}
\date{ Nov 23, 2008.  Last Revision: $Date: 2009/02/22 15:11:52 $ }

\begin{document}

\maketitle{}

\section{Motivation}

In \cite{purcell1963eam} the electric field for an infinite length charged line element is derived in two ways.  First using summation directly, then with Guass's law.  Associated with the first was the statement that the field must be radial by symmetry.  This was not obvious to me when initially taking my E\&M course, so I thought it was worth revisiting.

\section{ Calculation of electric field for non-infinite length line element. }

\begin{figure}[htp]
\centering
\includegraphics[totalheight=0.4\textheight]{charge_line_element_figure}
\caption{Charge on wire.}\label{fig:chargeLineElement}
\end{figure}

This calculation will be done with a thickness neglected wire running up and down along the $y$ axis as illustrated 
in figure \ref{fig:chargeLineElement}, where the field is being measured at $P = r \Be_1$, and the field contributions
due to all charge elements $dq = \lambda dy$ are to be summed.

We want to sum each of the field contributions along the line, so with

\begin{align*}
d\BE &= \frac{dq \ucap(\theta)}{4 \pi \epsilon_0 R^2} \\
r/R &= \cos\theta \\
dy &= r d(\tan\theta) = r \sec^2 \theta \\
\ucap(\theta) &= \Be_1 e^{i \theta} \\
i &= \Be_1 \Be_2 
\end{align*}

% check:
% 3 pi/2 : e1 e^i\theta = e1 \cos 3\pi/2 + e2\sin 3\pi/2 = -e2 
%   pi/2 : e1 e^i\theta = e1 \cos \pi/2 + e2\sin \pi/2   = e2
%      0 : e1 e^i\theta = e1 \cos 0 + e2\sin 0           = e1
%
% 3 pi/2 : e^i\theta = \cos 3\pi/2 + e1e2\sin 3\pi/2     = -e1e2 
% pi/2   : e^i\theta = \cos \pi/2 + e1e2\sin \pi/2       = e1e2
% 0      : e^i\theta = \cos 0 + e1e2\sin 0               = 1

Putting things together we have

\begin{align*}
d\BE
&= \frac{\lambda r \sec^2 \theta \Be_1 e^{i\theta} d\theta}{4 \pi \epsilon_0 r^2 \sec^2 \theta} \\
&= \frac{\lambda \Be_1 e^{i\theta} d\theta}{4 \pi \epsilon_0 r} \\
&= -\frac{\lambda \Be_1 i d(e^{i\theta})}{4 \pi \epsilon_0 r} \\
\end{align*}

Thus the total field is
\begin{align*}
\BE
&= \int d\BE \\
&= -\frac{\lambda \Be_2}{4 \pi \epsilon_0 r} \int d(e^{i\theta}) \\
\end{align*}

We see that the integration, which has the value

\begin{align}
\BE = -\frac{\lambda} {4 \pi \epsilon_0 r} \Be_2 e^{i\delta\theta}
\end{align}

The integration range for the infinite wire is $\theta \in [3\pi/2, \pi/2]$
so the field for the infinite wire is

\begin{align*}
\BE
&= -\frac{\lambda} {4 \pi \epsilon_0 r} \Be_2 \left. e^{i\theta} \right\vert^{\theta = \pi/2}_{\theta = 3\pi/2} \\
&= -\frac{\lambda} {4 \pi \epsilon_0 r} \Be_2 (e^{i\pi/2} - e^{3i\pi/2}) \\
&= -\frac{\lambda} {4 \pi \epsilon_0 r} \Be_2 (\Be_1 \Be_2 - (-\Be_1 \Be_2)) \\
&= \frac{\lambda} {2 \pi \epsilon_0 r} \Be_1 \\
\end{align*}
% 3 pi/2 : e^i\theta = \cos 3\pi/2 + e1e2\sin 3\pi/2     = -e1e2 
% pi/2   : e^i\theta = \cos \pi/2 + e1e2\sin \pi/2       = e1e2

%and $e^{-i\pi}$
%\ucap(\theta) &= \Be_1 e^{i \theta} \quad \theta \in [3\pi/2, \pi/2] \\

Invoking symmetry was done in order to work with coordinates, but working with the vector quantities directly
avoids this requirement and gives the general result for any subset of angles.

For a finite length wire all that is required is an angle parameterization of that wire's length

\begin{align*}
%[y_1, y_2] = r[\tan\theta_1, \tan\theta_2].
[\theta_1, \theta_2] = [\tan^{-1}(y_1/r), \tan^{-1}(y_2/r)]
\end{align*}

For such a range the exponential difference for the integral is

\begin{align*}
\left. e^{i\theta} \right \vert_{\theta_1}^{\theta_2} 
&= e^{i\theta_2} - e^{i\theta_1} \\
&= e^{i(\theta_1 + \theta_2)/2} \left( e^{i(\theta_2 - \theta_1)/2} -e^{i(\theta_2 - \theta_1)/2} \right) \\
&= 2 i e^{i(\theta_1 + \theta_2)/2} \sin((\theta_2 - \theta_1)/2) \\
\end{align*}

thus the associated field is

\begin{align*}
\BE 
&= -\frac{\lambda} {2 \pi \epsilon_0 r} \Be_2 i e^{i(\theta_1 + \theta_2)/2} \sin((\theta_2 - \theta_1)/2) \\
&= \frac{\lambda} {2 \pi \epsilon_0 r} \Be_1 e^{i(\theta_1 + \theta_2)/2} \sin((\theta_2 - \theta_1)/2) \\
\end{align*}

\bibliographystyle{plainnat}
\bibliography{myrefs}

\end{document}

%
% Copyright � 2012 Peeter Joot.  All Rights Reserved.
% Licenced as described in the file LICENSE under the root directory of this GIT repository.
%

%
%
\chapter{Biot Savart Derivation}\label{chap:biotSavart}
\index{Biot Savart}
%\date{April 18, 2009.  biotSavart.tex}

\section{Motivation}

Looked at my Biot-Savart derivation in \chapcite{PJelectricFieldEnergy}.  There I was playing with doing this without first dropping down to the
familiar vector relations, and end up with an expression of the Biot Savart law in terms of the complete Faraday bivector.  This is
an excessive approach, albeit interesting (to me).  Let us try this again in terms of just the magnetic field.

\section{Do it}

\subsection{Setup. Ampere-Maxwell equation for steady state}
\index{Ampere-Maxwell equation}

The starting point can still be Maxwell's equation

\begin{equation}\label{eqn:biotSavart:20}
\begin{aligned}
\grad F = J/\epsilon_0 c
\end{aligned}
\end{equation}

and the approach taken will be the more usual consideration of a loop of steady-state (no-time variation) current.

In the steady state we have

\begin{equation}\label{eqn:biotSavart:40}
\begin{aligned}
\grad = \gamma^0 \inv{c} \partial_t + \gamma^k \partial_k = \gamma^k \partial_k
\end{aligned}
\end{equation}

and in particular

\begin{equation}\label{eqn:biotSavart:60}
\begin{aligned}
\gamma_0 \grad F
&= \gamma_0 \gamma^k \partial_k F \\
&= \gamma_k \gamma_0 \partial_k F \\
&= \sigma_k \partial_k F \\
&= \spacegrad (\BE + I c \BB) \\
\end{aligned}
\end{equation}

and for the RHS,

\begin{equation}\label{eqn:biotSavart:80}
\begin{aligned}
\gamma_0 J/\epsilon_0 c
&=
\gamma_0 (c \rho \gamma_0 + J^k \gamma_k)/\epsilon_0 c  \\
&=
(c \rho - J^k \sigma_k)/\epsilon_0 c  \\
&=
(c \rho - \Bj)/\epsilon_0 c  \\
\end{aligned}
\end{equation}

So we have

\begin{equation}\label{eqn:biotSavart:spacetimeSplitSteadyState}
\begin{aligned}
\spacegrad (\BE + I c \BB)
&=
\inv{\epsilon_0}\rho - \frac{\Bj}{\epsilon_0 c}
\end{aligned}
\end{equation}

Selection of the (spatial) vector grades gives

\begin{equation}\label{eqn:biotSavart:100}
\begin{aligned}
I c (\spacegrad \wedge \BB) &= - \frac{\Bj}{\epsilon_0 c}
\end{aligned}
\end{equation}

or with \(\Ba \wedge \Bb = I (\Ba \cross \Bb)\), and \(\epsilon_0 \mu_0 c^2 = 1\), this is the familiar Ampere-Maxwell equation when \(\PDi{t}{\BE} = 0\).

\begin{equation}\label{eqn:biotSavart:AmpereMaxwellSteady}
\begin{aligned}
\spacegrad \cross \BB &= \mu_0 \Bj
\end{aligned}
\end{equation}

\subsection{Three vector potential solution}

With \(\spacegrad \cdot \BB = 0\) (the trivector part of \eqnref{eqn:biotSavart:spacetimeSplitSteadyState}), we can write

\begin{equation}\label{eqn:biotSavart:120}
\begin{aligned}
\BB = \spacegrad \cross \BA
\end{aligned}
\end{equation}

For some vector potential \(\BA\).  In particular, we have in \eqnref{eqn:biotSavart:AmpereMaxwellSteady},

\begin{equation}\label{eqn:biotSavart:140}
\begin{aligned}
\spacegrad \cross \BB
&=
\spacegrad \cross (\spacegrad \cross \BA) \\
&=
-I (\spacegrad \wedge (\spacegrad \cross \BA) ) \\
&=
-\frac{I}{2} (
\spacegrad (\spacegrad \cross \BA)
- (\spacegrad \cross \BA) \spacegrad
) \\
&=
\frac{I^2}{2} (
\spacegrad (\spacegrad \wedge \BA)
- (\spacegrad \wedge \BA) \spacegrad
) \\
&=
- \spacegrad \cdot (\spacegrad \wedge \BA)
\\
\end{aligned}
\end{equation}

Therefore the three vector potential equation for the magnetic field is

\begin{equation}\label{eqn:biotSavart:vecAwithJ}
\begin{aligned}
\spacegrad (\spacegrad \cdot \BA) - \spacegrad^2 \BA &= \mu_0 \Bj
\end{aligned}
\end{equation}

\subsection{Gauge freedom}
\index{gauge freedom}

We have the freedom to set \(\spacegrad \cdot \BA = 0\), in \eqnref{eqn:biotSavart:vecAwithJ}.  To see this suppose that the vector potential is
expressed in terms of some other potential \(\BA'\) that does have zero divergence (\(\spacegrad \cdot \BA' = 0\)) plus a (spatial) gradient

\begin{equation}\label{eqn:biotSavart:160}
\begin{aligned}
\BA = \BA' + \spacegrad \phi
\end{aligned}
\end{equation}

Provided such a construction is possible, then we have

\begin{equation}\label{eqn:biotSavart:180}
\begin{aligned}
\spacegrad (\spacegrad \cdot \BA) - \spacegrad^2 \BA
&=
\spacegrad (\spacegrad \cdot (\BA' + \spacegrad \phi)) - \spacegrad^2 (\BA' + \spacegrad \phi) \\
&=
- \spacegrad^2 \BA'
\end{aligned}
\end{equation}

and can instead solve the simpler equivalent problem

\begin{equation}\label{eqn:biotSavart:blah}
\begin{aligned}
\spacegrad^2 \BA'  &= -\mu_0 \Bj
\end{aligned}
\end{equation}

Addition of the gradient \(\spacegrad \phi\) to \(A'\) will not change the magnetic field \(\BB\) since \(\spacegrad \cross (\spacegrad \phi) = 0\).

FIXME: what was not shown here is that it is possible to express any vector potential \(\BA\) in terms of a divergence free potential and a
gradient.  How would one show this?

\subsection{Solution to the vector Poisson equation}
\index{Poisson equation}

The solution (dropping primes) to the Poisson \eqnref{eqn:biotSavart:blah} is

\begin{equation}\label{eqn:biotSavart:200}
\begin{aligned}
\BA = \frac{\mu_0}{4 \pi} \int \frac{\Bj}{r} dV
\end{aligned}
\end{equation}

(See \citep{schwartz1987pe} for example.)

The magnetic field follows by taking the spatial curl

\begin{equation}\label{eqn:biotSavart:220}
\begin{aligned}
\BB
&= \spacegrad \cross \BA \\
&= \frac{\mu_0}{4 \pi} \spacegrad \cross \int \frac{\Bj'}{\Abs{\Br-\Br'}} dV'
\end{aligned}
\end{equation}

Pulling the curl into the integral and writing the gradient in terms of radial components

\begin{equation}\label{eqn:biotSavart:240}
\begin{aligned}
\spacegrad &= \frac{\Br - \Br'}{\Abs{\Br - \Br'}} \PD{\Abs{\Br - \Br'}}{}
\end{aligned}
\end{equation}

we have
\begin{equation}\label{eqn:biotSavart:260}
\begin{aligned}
\BB
&= \frac{\mu_0}{4 \pi} \int
\frac{\Br - \Br'}{\Abs{\Br - \Br'}} \cross \Bj' \PD{\Abs{\Br - \Br'}}{}{\inv{ \Abs{\Br-\Br'} }} dV' \\
&= -\frac{\mu_0}{4 \pi} \int
\frac{\Br - \Br'}{\Abs{\Br - \Br'}^3} \cross \Bj' dV' \\
\end{aligned}
\end{equation}

Finally with \(\Bj' dV' = I \jCap' dl'\), we have

\begin{equation}\label{eqn:biotSavart:280}
\begin{aligned}
\BB(\Br) &= \frac{\mu_0}{4 \pi} \int dl' \jCap' \cross \frac{\Br - \Br'}{\Abs{\Br - \Br'}^3}
\end{aligned}
\end{equation}

%
% Copyright � 2012 Peeter Joot.  All Rights Reserved.
% Licenced as described in the file LICENSE under the root directory of this GIT repository.
%

% 
% 
\chapter{Vector forms of Maxwell's equations as projection and rejection operations}\label{chap:PJMaxwellProj}
\index{Maxwell's equations!projection}
\index{Maxwell's equations!rejection}
%\date{Sept 9, 2008.  vectorMaxwellsProjection.tex}

\section{Vector form of Maxwell's equations}

%FIXME: used \(i\) here as a pseudoscalar and an index depending on context.  Switched to \(I\) instead for the pseudoscalar for clarity ... hope I got them all.

We saw how to extract the tensor formulation of Maxwell's equations
from \(\grad F = J\).  A little bit of play shows how to pick off the divergence
equations we are used to as well.

The end result is that we can pick off two of the eight coordinate equations
with specific product operations.

It is helpful in the following to write \(\grad F\) in index notation
% See: em_bivector_metric_dependencies.pdf
%\eqnref{eqn:vecMaxProj:Fcomp}

\begin{equation}
\grad F = \PD{x^\mu}{E^i} {\gamma^{\mu}}_{i 0} - \epsilon_{i j k} c \PD{x^\mu}{B^i} {\gamma^{\mu}}_{j k}
\end{equation}

In particular, look at the span of the vector, or trivector multiplicands of
the partials of the electric and magnetic field coordinates

\begin{equation}\label{eqn:vecMaxProj:spanpartialE}
{\gamma^{\mu}}_{i 0} \in \Span \{ \gamma_{\mu}, \gamma_{0 i j} \}
\end{equation}

\begin{equation}\label{eqn:vecMaxProj:spanpartialB}
{\gamma^{\mu}}_{j k} \in \Span \{ \gamma_{i j \mu}, \gamma_i \}
\end{equation}

\subsection{Gauss's law for electrostatics}
\index{Gauss's law}
\index{electrostatics}

For extract Gauss's law for electric fields that operation is to take the scalar
parts of the product with \(\gamma^0\).

Dotting with \(\gamma^0\) will pick off the \(\rho\) term from
\(J\)

\begin{equation*}
\frac{J}{\epsilon_0 c} \cdot \gamma^0 = \rho/\epsilon_0,
\end{equation*}

We see that dotting
with \(\gamma_0\) will leave bivector parts contributed by the trivectors in
the span of \eqnref{eqn:vecMaxProj:spanpartialE}.  Similarly the magnetic partials
will contribute bivectors and scalars with this product.  Therefore to
get an equation with strictly scalar parts equal to \(\rho/\epsilon_0\) we need
to compute

\begin{equation}\label{eqn:vectorMaxwellsProjection:20}
\begin{aligned}
\gpgradezero{\left(\grad F - J/\epsilon_0 c\right) \gamma^0} 
&= \gpgradezero{\grad \BE \gamma^0} - \rho/\epsilon_0 \\
&= \gpgradezero{\grad E^k {\gamma_{k0}}^0} - \rho/\epsilon_0 \\
&= \gpgradezero{ \gamma^j \partial_{j} E^k \gamma_{k} } - \rho/\epsilon_0 \\
&= {\delta^j}_k \partial_{j} E^k - \rho/\epsilon_0 \\
&= \partial_{k} E^k - \rho/\epsilon_0 \\
\end{aligned}
\end{equation}

This is Gauss's law for electrostatics:
\begin{equation}\label{eqn:vecMaxProj:gausselectro}
\gpgradezero{\left(\grad F - J/\epsilon_0 c\right) \gamma^0} = \spacegrad \cdot \BE - \rho/\epsilon_0 = 0
\end{equation}

\subsection{Gauss's law for magnetostatics}
\index{Gauss's law}
\index{magnetostatics}

Here we are interested in just the trivector terms that are equal to zero that we saw before in \(\grad \wedge \grad \wedge A = 0\).

The divergence like equation of these four can be obtained by dotting with \(\gamma_{123} = \gamma^0 I\).  From the span enumerated
in \eqnref{eqn:vecMaxProj:spanpartialB}, we see that only the \(\BB\) field contributes such a trivector.  An addition scalar part selection is used
to eliminate the bivector that \(J\) contributes.

\begin{equation}\label{eqn:vectorMaxwellsProjection:40}
\begin{aligned}
\gpgradezero{\left(\grad F - J/\epsilon_0 c\right) \cdot \left(\gamma^0 I\right)}
&= (\grad I c \BB) \cdot \left(\gamma^0 I\right) \\
&= \gpgradezero{ \grad I c \BB \gamma^0 I } \\
&= \gpgradezero{ I \grad I c \BB \gamma^0 } \\
&= - c \gpgradezero{ I^2 \grad \BB \gamma^0 } \\
&= c \gpgradezero{ \grad \BB \gamma^0 } \\
&= c \gpgradezero{ \gamma^{\mu} \partial_{\mu} B^k \gamma_k } \\
&= c {\delta^{\mu}}_k \partial_{\mu} B^k \\
&= c \partial_{k} B^k \\
&= 0
\end{aligned}
\end{equation}

This is just the divergence, and therefore yields Gauss's law for magnetostatics:

\begin{equation}\label{eqn:vecMaxProj:gaussmagnet}
\left(\grad F - J/\epsilon_0 c\right) \cdot \left(\gamma^0 I / c \right) = \spacegrad \cdot \BB = 0
\end{equation}

\subsection{Faraday's Law}
\index{Faraday's law}

We have three more trivector equal zero terms to extract from our field equation.

Taking dot products for those remaining three trivectors we have

\begin{equation}\label{eqn:vectorMaxwellsProjection:60}
\begin{aligned}
( \grad F - J/\epsilon_0 c ) \cdot (\gamma^j I)
\end{aligned}
\end{equation}

This will leave a contribution from \(J\), so to exclude that we want to calculate

\begin{equation}\label{eqn:vectorMaxwellsProjection:80}
\begin{aligned}
\gpgradezero{( \grad F - J/\epsilon_0 c ) \cdot (\gamma^j I)}
\end{aligned}
\end{equation}

The electric field contribution gives us
\begin{equation}\label{eqn:vectorMaxwellsProjection:100}
\begin{aligned}
\partial_{\mu} E^k \gpgradezero{ \gamma^{\mu} \gamma_{k0} {\gamma^j}_{0123} } 
&=
-\partial_{\mu} E^k (\gamma_0)^2 \gpgradezero{ \gamma^{\mu} \gamma_{k} {\gamma^j}_{123} }
\end{aligned}
\end{equation}

the terms \(\mu = 0\) will not produce a scalar, so this leaves
\begin{equation}\label{eqn:vectorMaxwellsProjection:120}
\begin{aligned}
-\partial_{i} E^k (\gamma_0)^2 \gpgradezero{ \gamma^{i} \gamma_{k} {\gamma^j}_{123} }
&= -\partial_{i} E^k (\gamma_0)^2 (\gamma_k)^2 \epsilon_{jki} \\
&= \partial_{i} E^k \epsilon_{j k i} \\
&= -\partial_{i} E^k \epsilon_{jik} \\
\end{aligned}
\end{equation}

Now, for the magnetic field contribution we have
\begin{equation}\label{eqn:vectorMaxwellsProjection:140}
\begin{aligned}
c \partial_{\mu} B^k \gpgradezero{ \gamma^{\mu} I \gamma_{k0} {\gamma^j} I } 
&= - c \partial_{\mu} B^k \gpgradezero{ I \gamma^{\mu} \gamma_{k0} {\gamma^j} I } \\
&= - c \partial_{\mu} B^k \gpgradezero{ I^2 \gamma^{\mu} \gamma_{k0} {\gamma^j} } \\
&= c \partial_{\mu} B^k \gpgradezero{ \gamma^{\mu} \gamma_{k0} {\gamma^j} } \\
\end{aligned}
\end{equation}

For a scalar part we need \(\mu = 0\) leaving
\begin{equation}\label{eqn:vectorMaxwellsProjection:160}
\begin{aligned}
c \partial_{0} B^k \gpgradezero{ \gamma^{0} \gamma_{k0} {\gamma^j} } 
&= -\partial_{t} B^k \gpgradezero{ \gamma_{k} {\gamma^j} } \\
&= -\partial_{t} B^k {\delta_{k}}^j \\
&= -\partial_{t} B^j
\end{aligned}
\end{equation}

Combining the results and summing as a vector we have:
\begin{equation}\label{eqn:vectorMaxwellsProjection:180}
\begin{aligned}
\sum \sigma_j \gpgradezero{( \grad F - J/\epsilon_0 c ) \cdot (\gamma^j I)}
&= -\partial_{i} E^k \epsilon_{jik} \sigma_j -\partial_{t} B^j \sigma_j \\
&= -\partial_{j} E^k \epsilon_{i j k} \sigma_i -\partial_{t} B^i \sigma_i \\
&= -\spacegrad \cross \BE - \PD{t}{\BB} \\
&= 0
\end{aligned}
\end{equation}

Moving one term to the opposite side of the equation yields the familiar vector form for Faraday's law

\begin{equation}
\spacegrad \cross \BE = -\PD{t}{\BB}
\end{equation}

\subsection{Ampere Maxwell law}
\index{Ampere Maxwell law}

For the last law, we want the current density, so to extract the Ampere Maxwell law we must have to wedge with \(\gamma^0\).  Such a wedge will eliminate all the trivectors from the span of \eqnref{eqn:vecMaxProj:spanpartialE}, but can contribute pseudoscalar components from the trivectors in \eqnref{eqn:vecMaxProj:spanpartialB}.  Therefore the desired calculation is

\begin{equation}\label{eqn:vectorMaxwellsProjection:200}
\begin{aligned}
\gpgradetwo{\left(\grad F - J/\epsilon_0 c\right) \wedge \gamma^0}
&= \gpgradetwo{ (({\gamma^{\mu}}_{j0}) \wedge \gamma^0 \partial_{\mu} E^j + (\grad I c B) \wedge \gamma^0 } - (\gamma_0)^2 \BJ/\epsilon_0 c \\
&= \gpgradetwo{ -(({\gamma^{0}}_{0j}) \wedge \gamma^0 \partial_{0} E^j + (\grad I c B) \wedge \gamma^0 } - (\gamma_0)^2 \BJ/\epsilon_0 c \\
&= -{\gamma_{j}}^0 \inv{c} \partial_t E^j + \gpgradetwo{ (\grad I c B) \wedge \gamma^0 } - (\gamma_0)^2 \BJ/\epsilon_0 c \\
&= - \frac{(\gamma_0)^2}{c} \PD{t}{\BE} + c \gpgradeone{ \grad I B} \wedge \gamma^0 - (\gamma_0)^2 \BJ/\epsilon_0 c \\
\end{aligned}
\end{equation}

Let us take just that middle term

\begin{equation}\label{eqn:vectorMaxwellsProjection:220}
\begin{aligned}
\gpgradeone{ \grad I B } \wedge \gamma^0
&= -\gpgradeone{ I \gamma^{\mu} \partial_{\mu} B^k \gamma_{k0} } \wedge \gamma^0 \\
&= - \partial_{\mu} B^k \gpgradeone{ \gamma_{0123} \gamma^{\mu} \gamma_{k0} } \wedge \gamma^0 \\
&= \partial_{\mu} B^k \left(\gpgradetwo{ \gamma_{0123} \gamma^{\mu} \gamma_{0} } \cdot \gamma_k\right) \wedge \gamma^0
\end{aligned}
\end{equation}

Here \(\mu \ne 0\) since that leaves just a pseudoscalar in the grade two selection.
\begin{equation}\label{eqn:vectorMaxwellsProjection:240}
\begin{aligned}
\gpgradeone{ \grad I B } \wedge \gamma^0
&= \partial_{j} B^k \left(\gpgradetwo{ \gamma_{0123} \gamma^{j} \gamma_{0} } \cdot \gamma_k\right) \wedge \gamma^0 \\
&= (\gamma_0)^2 \partial_{j} B^k \left(\gpgradetwo{ \gamma_{123} \gamma^{j} } \cdot \gamma_k\right) \wedge \gamma^0 \\
&= (\gamma_0)^2 \partial_{j} B^k \left(\gpgradetwo{ \epsilon^{hkj}\gamma_{hkj} \gamma^{j} } \cdot \gamma_k\right) \wedge \gamma^0 \\
&= \partial_{j} B^k \epsilon^{hkj} (\gamma_0)^2 (\gamma_k)^2 {\gamma_{h}}^0 \\
&= - (\gamma_0)^2 \partial_{j} B^k \epsilon^{hkj} \sigma_h \\
&= (\gamma_0)^2 \spacegrad \cross \BB
\end{aligned}
\end{equation}

%\partial_{j} B^k \epsilon^{hkj} \sigma_h = - \spacegrad \cross \BB
Putting things back together and factoring out the common metric dependent \((\gamma_0)^2\) term we have

\begin{equation}\label{eqn:vectorMaxwellsProjection:260}
\begin{aligned}
- \inv{c} \PD{t}{\BE} + c \spacegrad \cross \BB - \BJ/\epsilon_0 c &= 0 \\
\implies \\
- \inv{c^2} \PD{t}{\BE} + \spacegrad \cross \BB - \BJ/\epsilon_0 c^2 &= 0
\end{aligned}
\end{equation}

With \(\inv{c^2} = \mu_0 \epsilon_0\) this is the Ampere Maxwell law

\begin{equation}
\spacegrad \cross \BB = \mu_0 \left(\BJ + \epsilon_0 \PD{t}{\BE} \right)
\end{equation}

which we can put in the projection form of \eqnref{eqn:vecMaxProj:gausselectro} and \eqnref{eqn:vecMaxProj:gaussmagnet} as:

\begin{equation}\label{eqn:vecMaxProj:amperemaxwell}
\gpgradetwo{\left(\grad F - J/\epsilon_0 c\right) \wedge (\gamma_0/c)} =
\spacegrad \cross \BB - \mu_0 \left(\BJ + \epsilon_0 \PD{t}{\BE} \right) = 0
\end{equation}

\section{Summary of traditional Maxwell's equations as projective operations on Maxwell Equation}

\begin{equation}\label{eqn:vectorMaxwellsProjection:280}
\begin{aligned}
\gpgradezero{\left(\grad F - J/\epsilon_0 c\right) \gamma^0} &= \spacegrad \cdot \BE - \rho/\epsilon_0 = 0 \\
\gpgradezero{\left(\grad F - J/\epsilon_0 c\right) \cdot \left(\gamma^0 I / c \right)} &= \spacegrad \cdot \BB = 0 \\
\sum \sigma_j \gpgradezero{( \grad F - J/\epsilon_0 c ) \cdot (\gamma^j I)} &= -\spacegrad \cross \BE - \PD{t}{\BB} = 0 \\
\gpgradetwo{\left(\grad F - J/\epsilon_0 c\right) \wedge (\gamma_0/c)} &= \spacegrad \cross \BB - \mu_0 \left(\BJ + \epsilon_0 \PD{t}{\BE} \right) = 0
\end{aligned}
\end{equation}

Faraday's law requiring a sum suggests that this can likely be written instead using a rejective operation.  Will leave that as a possible future followup.

%
% Copyright � 2012 Peeter Joot.  All Rights Reserved.
% Licenced as described in the file LICENSE under the root directory of this GIT repository.
%

%
%
% ointclockwise, ointctrclockwise
%\usepackage{txfonts}

\chapter{Application of Stokes Integrals to Maxwell's Equation}
\index{Stokes theorem!Maxwell's equation}
\label{chap:stokesMaxwellApplication}
%\date{Sept 26, 2008.  stokesMaxwellApplication.tex}

\section{Putting Maxwell's equation in curl form}

These notes contain an application of the bivector Stokes equations
detailed in
\chapcite{PJStokes1}
.  Background of interest can also be found
in \citep{DenkerWire}, which contained the core statement of
the multivector form of Stokes equation and Biot-Savart like application
of it.  Also informative as background is the following excellent
\citep{DenkerMaxwell}.
introduction to the STA form of Maxwell's equation.

Stokes equation applied to a bivector takes the following form

\begin{equation}\label{eqn:stokesMax:summaryStokesVolume}
\iiint (\grad \wedge F) \cdot d^3\Bx = \oiintclockwise F \cdot d^2\Bx,
\end{equation}

where we will write \(F\) as the electromagnetic field bivector, and apply
it to Maxwell's equation

\begin{equation}\label{eqn:stokesMax:maxwell}
\grad F = J/\epsilon_0 c.
\end{equation}

Taking vector and trivector parts we have two equations
\begin{equation}\label{eqn:stokesMax:maxwellv}
\grad \cdot F = J/\epsilon_0 c,
\end{equation}

and
\begin{equation}\label{eqn:stokesMax:maxwellt}
\grad \wedge F = 0.
\end{equation}

\subsection{Trivector equation part}

The second of these, \eqnref{eqn:stokesMax:maxwellt}, we can apply Stokes to
directly:

\begin{equation}
\iiint (\grad \wedge F) \cdot d^3 \Bx = \oiintclockwise F \cdot d^2\Bx = 0.
\end{equation}

This area integral is a flux like quantity.  Suppose we call this the field flux, then this says
says
the flux of the combined electromagnetic field through any surface is zero
independent of the charge or current densities.
Note that here \(d^3\Bx\)
can be a regular spatial volume trivector element, but one can also pick
a spacetime (area times time) ``volume'' to integrate over, in which case
\(d^2\Bx\) are the oriented ``surfaces'' of such a spacetime volume.

This does not seem like a result that I am familiar with based on the traditional
vector forms of Maxwell's equation.  Perhaps it is recognizable in terms of
\(\BE\) and \(\BB\) explicitly:

\begin{equation}\label{eqn:stokesMax:gaussmagnetostatics}
\oiintclockwise \BE \cdot d^2\Bx = - c \oiintclockwise \BB \cdot (d^2\Bx I)
\end{equation}

On the surface this does not look like a familiar identity.  It is in fact Gauss's law for magneto-statics, which will be shown
later.

Note also the subtle difference from traditional vector treatments where
\(\BE\) and \(\BB\) were spatial vectors.  Here they are written as spacetime
bivectors,
\(\BE = E^i \sigma_i = E^i \gamma_i \wedge \gamma_0\),
\(\BB = B^i \sigma_i = B^i \gamma_i \wedge \gamma_0\).


\subsection{Vector part}

Moving on to the charge and current dependent vector terms of Maxwell's equation, we want express \eqnref{eqn:stokesMax:maxwellv} as a spacetime curl so that we can apply stokes to it.

We can do this by temporarily writing our field in terms of a potential as well its dual bivector.

\begin{equation}\label{eqn:stokesMaxwellApplication:20}
\begin{aligned}
F = \grad \wedge A = I D
\end{aligned}
\end{equation}
\begin{equation}\label{eqn:stokesMaxwellApplication:40}
\begin{aligned}
\grad F
&= \grad (\grad \wedge A) \\
&= \grad \cdot (\grad \wedge A) + \grad \wedge (\grad \wedge A) \\
&= \grad \cdot (I D) \\
&= \gpgradeone{ \grad I D } \\
&= -\gpgradeone{ I (\mathLabelBox{\grad \cdot D}{1 vector} +
\mathLabelBox
[
   labelstyle={below of=m\themathLableNode, below of=m\themathLableNode}
]
{\grad \wedge D}{3 vector}) } \\
&= - I (\grad \wedge D) \\
\end{aligned}
\end{equation}

or
\begin{equation}\label{eqn:stokesMaxwellApplication:60}
\begin{aligned}
I \grad F = \grad \wedge D.
\end{aligned}
\end{equation}

Applying stokes we have
\begin{equation}\label{eqn:stokesMaxwellApplication:80}
\begin{aligned}
\int (\grad \wedge D) \cdot d^3\Bx &= \oiintclockwise D \cdot d^2\Bx \\
\int (I \grad F) \cdot d^3\Bx
&= \oiintclockwise (-I F) \cdot d^2\Bx \\
&= \oiintclockwise \gpgradezero{ - F d^2\Bx I } \\
&= -\oiintclockwise F \cdot (d^2\Bx I) \\
\inv{\epsilon_0 c} \int (I J) \cdot d^3\Bx &= \\
\inv{\epsilon_0 c} \int \gpgradezero{ I J d^3\Bx } &= \\
\inv{\epsilon_0 c} \int \gpgradezero{ J d^3\Bx I } &= \\
\inv{\epsilon_0 c} \int J \cdot (d^3\Bx I) &= \\
\end{aligned}
\end{equation}

Or
\begin{equation}\label{eqn:stokesMax:maxwellint}
\oiintctrclockwise F \cdot (d^2\Bx I) = \int \frac{J}{\epsilon_0 c} \cdot (d^3\Bx I)
\end{equation}

This is the integral form of the vector part of Maxwell's equation \eqnref{eqn:stokesMax:maxwell}.
This does not look terribly familiar, but we are not used to
seeing Maxwell's equations in a non-disassembled form.
Hiding in there should be a subset of the
traditional eight Maxwell's equations in integral form.  It will be
possible to extract these by considering variations of current and charge density and different
volume and surface integration regions.

\section{Extracting the vector integral forms of Maxwell's equations}

One can extract the integral forms of Maxwell's equations
from \eqnref{eqn:stokesMax:maxwell}, by first extracting the differential vector
equations, and then using the spatial
divergence and stokes equations.
However, having formulated Stokes equation in its bivector form
we can go directly to those equations by appropriate selection of spatial
or spacetime volumes.
Of course we also now have new tools to work with the field in its entirety,
but lets use this as an exercise to verify that all the previous computation
that led to Stokes equation gives us the expected results.  In particular
this should be a good way to verify that
sign mistakes or other similar small errors (which would not be too hard)
have not been made.

\subsection{Zero current density. Gauss's law for Electrostatics}

With \(J = c \rho \gamma_0\), the integral form of Maxwell's equation becomes

\begin{equation}\label{eqn:stokesMaxwellApplication:100}
\begin{aligned}
\oiintctrclockwise F \cdot (d^2\Bx I)
&= \int \frac{\rho}{\epsilon_0} \gpgradezero{\gamma_0 d^3\Bx I} \\
&= \int \frac{\rho}{\epsilon_0} \gpgradezero{\gamma_{0123} \gamma_0 d^3\Bx} \\
&= -\inv{\epsilon_0} {\gamma_0}^2 \int \frac{\rho}{\epsilon_0} \gpgradezero{\gamma_{123} d^3\Bx} \\
\end{aligned}
\end{equation}

From this we see that, in the absence of currents the LHS integral must be zero unless the volume is purely spatial.  Denoting the boundary of a spacetime volume as \(\partial A c t\), this is

\begin{equation}\label{eqn:stokesMaxwellApplication:120}
\begin{aligned}
\oiintctrclockwise_{\partial {A c t}} F \cdot (d^2\Bx I) &= 0.
\end{aligned}
\end{equation}

For a purely spatial volume the dual surfaces \(d^2\Bx I\) always includes a spacetime bivector, therefore the magnetic field contributes nothing

\begin{equation*}
\oiintctrclockwise_{\partial V} I c B \cdot (d^2\Bx I) =
-c \oiintctrclockwise_{\partial V} B \cdot d^2\Bx = 0
\end{equation*}

Although this looks similar to the integral equivalent of \(\spacegrad \cdot B = 0\), we should look elsewhere for that since
that is true for the non-zero current density case too.

That leaves

\begin{equation}\label{eqn:stokesMaxwellApplication:140}
\begin{aligned}
\oiintctrclockwise E \cdot (d^2\Bx I) &= -\inv{\epsilon_0} {\gamma_0}^2 \int_{V} \rho \gpgradezero{\gamma_{123} d^3\Bx} \\
\end{aligned}
\end{equation}

Letting \(d^3 \Bx = dx^1 dx^2 dx^3 \gamma_{123}\).  Within the charge integral becomes

\begin{equation}\label{eqn:stokesMaxwellApplication:160}
\begin{aligned}
-\inv{\epsilon_0} {\gamma_0}^2 \int_{V} \rho \gpgradezero{\gamma_{123} d^3\Bx}
&=
\inv{\epsilon_0}
\mathLabelBox{{\gamma_0}^2 {\gamma_1}^2}{\(=-1\)}
\mathLabelBox
[
   labelstyle={below of=m\themathLableNode, below of=m\themathLableNode}
]
{{\gamma_2}^2 {\gamma_3}^2}{\(=(\pm 1)^2\)}
 \int_{V} \rho dx^1 dx^2 dx^3
&= -\inv{\epsilon_0} \int_{V} \rho dx^1 dx^2 dx^3
\end{aligned}
\end{equation}

To put this in correspondence with the forms we are used to consider the surfaces separately.  For the dual to the
front surface (see: \chapcite{PJStokes1})
we have

\begin{equation}\label{eqn:stokesMaxwellApplication:180}
\begin{aligned}
d^2 \Bx I
&= dx^1 dx^2 \gamma_{12} I \\
&= dx^1 dx^2 \gamma_{120123} \\
&= dx^1 dx^2 \gamma_{112023} \\
&= -dx^1 dx^2 \gamma_{112203} \\
&= -(\pm 1)^2 dx^1 dx^2 \gamma_{03} \\
&= dx^1 dx^2 \sigma_3
\end{aligned}
\end{equation}

For the left surface
\begin{equation}\label{eqn:stokesMaxwellApplication:200}
\begin{aligned}
d^2 \Bx I
&= dx^3 dx^2 \gamma_{32} I \\
&= dx^3 dx^2 \gamma_{320123} \\
&= dx^3 dx^2 \gamma_{332012} \\
&= dx^3 dx^2 \gamma_{332201} \\
&= dx^3 dx^2 (\pm 1)^2 \gamma_{01} \\
&= -dx^3 dx^2 \sigma_1 \\
\end{aligned}
\end{equation}

and for the top
\begin{equation}\label{eqn:stokesMaxwellApplication:220}
\begin{aligned}
d^2 \Bx I
&= dx^1 dx^3 \gamma_{13} I \\
&= dx^1 dx^3 \gamma_{130123} \\
&= dx^1 dx^3 \gamma_{113023} \\
&= dx^1 dx^3 \gamma_{113302} \\
&= -dx^1 dx^3 \sigma_2 \\
\end{aligned}
\end{equation}

Assembling results, writing \((x^1, x^2, x^3) = (x,y,z)\) we have
%\begin{align*}
%\iint
%(E_x(x, y, z_0) - E_x(x, y, z_1)) dx dy \\
%-\iint
%(E_y(x_1, y, z) - E_y(x_0, y, z)) dy dz \\
%-\iint
%(E_z(x, y_1, z) - E_z(x, y_0, z)) dx dz \\
%&= -\inv{\epsilon_0} \int_{V} \rho dx dy dz
%\end{align*}
\begin{equation}\label{eqn:stokesMaxwellApplication:240}
\begin{aligned}
\inv{\epsilon_0} \int_{V} \rho dx dy dz
&=
\iint
(E_x(x, y, z_1) - E_x(x, y, z_0)) dx dy \\
&+\iint
(E_y(x_1, y, z) - E_y(x_0, y, z)) dy dz \\
&+\iint
(E_z(x, y_1, z) - E_z(x, y_0, z)) dx dz \\
\end{aligned}
\end{equation}

This is Gauss's law for electrostatics in integral form

\begin{equation}
\iint \BE \cdot \ncap dA = \iiint \frac{\rho}{\epsilon_0} dV
\end{equation}

Although this extraction method is easy to understand, it is apparent that having only a pictorial way of enumerating the
oriented bivector
area elements is not efficient for high level computation.  Revisiting the stokes derivation with a more algebraic enumeration
of the surfaces should be done!

\subsection{Gauss's law for magneto-statics}
\index{Gauss's law}
\index{magnetostatics}

Return now to \eqnref{eqn:stokesMax:gaussmagnetostatics}, which resulted from considering the trivector part of Maxwell's equation

\begin{equation}
\oiintclockwise \BE \cdot d^2\Bx = - c \oiintclockwise \BB \cdot (d^2\Bx I).
\end{equation}

To start some observations can be made.

Only the spacetime surfaces of the volume
contribute to the LHS integral since \(\sigma_i \cdot (\gamma_j \wedge \gamma_k) = 0\).

For the RHS, only the purely spatial surfaces contribute to that \(\BB\) integral, since the dual surface \(d^2\Bx I\) must have a spacetime component for that dot product to be non-zero.  We have also just enumerated these dual surface area elements \(d^2 \Bx I\) for a purely
spatial surface, therefore with a \(E,B\) substitution we must have

\begin{equation}\label{eqn:stokesMaxwellApplication:260}
\begin{aligned}
0 &=
\iint
(B_x(x, y, z_1) - B_x(x, y, z_0)) dx dy  \\
&+\iint
(B_y(x_1, y, z) - B_y(x_0, y, z)) dy dz  \\
&+\iint
(B_z(x, y_1, z) - B_z(x, y_0, z)) dx dz
\end{aligned}
\end{equation}

or, more compactly

\begin{equation}
\iint \BB \cdot \ncap dA = 0
\end{equation}

For any current or charge distribution.  We have therefore obtained two of the eight Maxwell's equations.

\subsection{Zero charge.  Current density in single direction}
\index{current density}

Next to consider is \(J = j^i \gamma_i\).  For simplicity, consider current in only one direction,
taking \(J = j^1 \gamma_1\).  The exercise will be to compute the integrals of \eqnref{eqn:stokesMax:maxwellint}.

\begin{equation}\label{eqn:stokesMaxwellApplication:280}
\begin{aligned}
\oiintclockwise F \cdot (I d^2\Bx)
&= \int \frac{J}{\epsilon_0 c} \cdot (I d^3\Bx) \\
&= \int \frac{j^1}{\epsilon_0 c} \gamma_1 \cdot (I d^3\Bx) \\
\end{aligned}
\end{equation}

Unlike the calculations for the Gauss's law equations above, this one
will be done using the area orientation
methods from \chapcite{PJStokes2} since algebraically enumerating the surfaces
should make life easier.  The two Gauss's law results above were done without this, which was not too bad for a purely spatial volume, but with spacetime
volumes this is probably confusing in addition to being harder.

Starting with the volume element, one can observe that the current density
will not contribute to the boundary integral unless \(d^3\Bx\) has no \(\gamma_1\)
component, thus for a rectangular prism integration spacetime volume let
\(d^3 \Bx = \gamma_{023} dx^0 dx^2 dx^3\)

\begin{equation}\label{eqn:stokesMaxwellApplication:300}
\begin{aligned}
\gamma_1 \cdot (I d^3\Bx)
&= \gamma_1 \cdot \gamma_{0123023} dx^0 dx^2 dx^3 \\
&= \gamma_1 \cdot \gamma_{0012233} dx^0 dx^2 dx^3 \\
&= \gamma_1 \cdot \gamma_{111} dx^0 dx^2 dx^3 \\
&= -\gamma_1 \cdot \gamma^{1} dx^0 dx^2 dx^3 \\
&= - dx^0 dx^2 dx^3 \\
\end{aligned}
\end{equation}

Now for all the surfaces we want to calculate
\(I d^2\Bx\) for each of the surfaces.
For each of \(\mu \in \{0, 2, 3\}\), calculation of \(I (d^2\Bx)_\mu\) is required where

\begin{equation}\label{eqn:stokesMaxwellApplication:320}
\begin{aligned}
(d^2 \Bx)_\mu &= d^3 \Bx \cdot \Br^{\mu} \\
\Br &= x^i \gamma_i \\
\Br_\mu
&= \PD{x^{\mu}}{\Br} \\
&= \gamma_\mu \\
\Br^\mu &= \gamma^\mu
\end{aligned}
\end{equation}

Calculating the surfaces
\begin{equation}\label{eqn:stokesMaxwellApplication:340}
\begin{aligned}
I (d^2 \Bx)_\mu \frac{dx^{\mu}}{dx^0 dx^2 dx^3}
&= \gpgradetwo{ \gamma_{0123} (\gamma_{023} \cdot \gamma^{\mu}) } \\
&= \inv{2} \gpgradetwo{ \gamma_{0123} ( \gamma_{023} \gamma^{\mu} + \gamma^{\mu} \gamma_{023} ) } \\
&= \inv{2} \gpgradetwo{ \gamma_{0123} ( \gamma_{023} \gamma^{\mu} + \gamma_{023} \gamma^{\mu} ) } \\
&= \gpgradetwo{ \gamma_{0012233} \gamma^{\mu} } \\
&= -\gpgradetwo{ \gamma_{133} \gamma^{\mu} } \\
&= -\gpgradetwo{ \gamma^{1} \gamma^{\mu} } \\
&= \gamma^{\mu} \wedge \gamma^{1} \\
\end{aligned}
\end{equation}

Putting things back together we have

\begin{equation}\label{eqn:stokesMaxwellApplication:360}
\begin{aligned}
-\int j^1 dx^0 dx^2 dx^3 = \int \sum_{\mu = 0,2,3}
\left. F \cdot \left(\gamma^{\mu} \wedge \gamma^{1}\right) \right\vert_{\partial x^{\mu}}
\frac{dx^0 dx^2 dx^3}{dx^{\mu}}
\end{aligned}
\end{equation}

Now, for \(\mu=0\) we pick up the electric field component of the field

\begin{equation}\label{eqn:stokesMaxwellApplication:380}
\begin{aligned}
F \cdot \gamma^{01}
&= \left( E^i \gamma_{i0} -\epsilon_{ijk} c B^k \gamma_{ij} \right) \cdot \gamma^{01} \\
&= E^i,
\end{aligned}
\end{equation}

and for \(\mu=2,3\) we pick up magnetic field components
\begin{equation}\label{eqn:stokesMaxwellApplication:400}
\begin{aligned}
F \cdot \gamma^{\mu1}
&= \left( E^i \gamma_{i0} -\epsilon_{ijk} c B^k \gamma_{ij} \right) \cdot \gamma^{\mu1} \\
&= -\epsilon_{1 \mu k} c B^k \gamma_{1\mu} \cdot \gamma^{\mu1}.
\end{aligned}
\end{equation}

For \(\mu=2\) this is \(- c B^3\), and for \(\mu=3\), \(-\epsilon_{132} c B^2 = c B^2\), so we have

\begin{equation}\label{eqn:stokesMaxwellApplication:420}
\begin{aligned}
0
&= \int \frac{j^1}{c\epsilon_0} dx^0 dx^2 dx^3
+ \int \left. E^1 dx^2 dx^3 \right\vert_{\partial x^{0}}
+c\int \left. B^2 dx^0 dx^2 \right\vert_{\partial x^{3}}
-c\int \left. B^3 dx^0 dx^3 \right\vert_{\partial x^{2}} \\
&= \int \frac{j^1}{c\epsilon_0} dx^0 dx^2 dx^3
+ \int \PD{x^0}{E^1} dx^0 dx^2 dx^3
+c\int \PD{x^3}{B^2} dx^3 dx^2 dx^0
-c\int \PD{x^2}{B^3} dx^2 dx^0 dx^3  \\
&= \int dx^0 \int dx^2 dx^3 \left(\frac{j^1}{c\epsilon_0} + \inv{c}\PD{t}{E^1} +c\PD{x^3}{B^2} -c\PD{x^2}{B^3} \right) \\
\end{aligned}
\end{equation}

If this is zero for all time intervals, then the inner integral is also zero.  Utilizing \(c^2\mu_0\epsilon_0 = 1\) this is

\begin{equation}\label{eqn:stokesMaxwellApplication:440}
\begin{aligned}
0
&= \int dx^2 dx^3 \left( \mu_0 \left( j^1 + \epsilon_0 \PD{t}{E^1} \right) + \left( \PD{x^3}{B^2} -\PD{x^2}{B^3} \right) \right).
\end{aligned}
\end{equation}

Writing \(d\BA = \sigma_1 dx^2 dx^3\), \(\Bj = j^1 \sigma_1\), \(\BE = E^1 \sigma_1\), and \(\BB = B^i \sigma_i\) we can pick off
the differential form of the Maxwell-Ampere equation

\begin{equation}
\spacegrad \cross \BB = \mu_0 \left( \Bj + \epsilon_0 \PD{t}{\BE} \right),
\end{equation}

as well as the integral form
\begin{equation}
\int (\spacegrad \cross \BB) \cdot d\BA
= \mu_0 \left(\int \Bj \cdot d\BA + \epsilon_0 \int \PD{t}{\BE} \cdot d\BA \right)
\end{equation}

Both of these forms come straight from the application of the generalized Stokes equation integrating an appropriate
spacetime volume.

Now it is normal to have the spatial curl of \(\BB\) written as a closed loop integral.  Stokes can be employed
again to get exactly that form.  This really just undoes the fact that the partials to used as a convenience enumerate
exactly those loop boundaries (although they were originally oriented area boundaries).

\begin{equation}\label{eqn:stokesMaxwellApplication:460}
\begin{aligned}
\int \PD{x^3}{B^2} dx^3 &= B^2(t, x, y, z_1) - B^2(t, x, y, {z}_0) \\
\int \PD{x^2}{B^3} dx^2 &= B^3(t, x, {y}_1, z) - B^3(t, x, {y}_0, z)
\end{aligned}
\end{equation}

Also observe that this whole treatment was
done with \(J = j^1 \gamma_1\) only.  It is not hard to see that doing the same with \(j^i\) and summing over \(\sigma_i\)
will produce the same result.  Of course more care is required to handle the more abstract symbolic indices since a nice
hard-coded number is easier.
On the other hand the usual dodge, employing freedom to orient the coordinate system along the
\(\gamma_1\) direction makes the more general algebraic approach a less interesting exercise.

\subsection{Faraday's law}
\index{Faraday's law}

We have five of the eight Maxwell's equations.  Gauss's law for electrostatics
from the vector part of \eqnref{eqn:stokesMax:maxwellv}, integrating over a spatial
volume, and the Maxwell-Ampere equation from the same, integrating over
a spacetime volume.  Gauss's law for magneto-statics from the trivector part
of \eqnref{eqn:stokesMax:maxwellv}, integrating over a spatial volume.  This suggests
that our remaining three (one three-vector) equation will come from
integrating the trivector parts over a spacetime volume.

Stokes' gives us

\begin{equation*}
\int_V (\grad \wedge F) \cdot d^3 \Bx = \int_{\partial V} F \cdot (d^2\Bx)
\end{equation*}

Picking a spacetime volume element, and corresponding area elements

\begin{equation}\label{eqn:stokesMaxwellApplication:480}
\begin{aligned}
d^3 \Bx &= \gamma_{0ij} dx^0 dx^i dx^j \\
(d^2 \Bx)_\mu &= (\gamma_{0ij} \cdot \gamma^\mu) \frac{dx^0 dx^i dx^j}{dx^\mu}
\end{aligned}
\end{equation}

Our area integral (expanding boundaries as one more integral of partials) is
\begin{equation*}
\int \sum_{\mu = 0,i,j} dx^0 dx^i dx^j \left( \PD{x^\mu}{F} \cdot (\gamma_{0ij}\cdot\gamma^\mu) \right).
\end{equation*}

For the dot products of the area elements we have

\begin{equation*}
\left\{
\begin{array}{l l}
\gamma_{ij} & \quad \mbox{if \(\mu = 0\)} \\
\gamma_{0i} = -\sigma_i & \quad \mbox{if \(\mu = j\)} \\
-\gamma_{0j} = \sigma_j & \quad \mbox{if \(\mu = i\)} \\
\end{array} \right.
\end{equation*}

Our field derivatives in coordinates are

\begin{equation*}
\PD{x^\mu}{F}
= \PD{x^\mu}{E^m} \sigma_m - \epsilon_{klm} c \PD{x^\mu}{B^m} \gamma_{kl}
\end{equation*}

Observe that \(\mu\ne0\) selects only the electric field components, and \(\mu=0\) only the magnetic field components are selected.  Specifically

\begin{equation*}
\PD{x^\mu}{F} =
\left\{
\begin{array}{l l l}
-\epsilon_{jim} c \PD{x^0}{B^m} (\gamma_i)^2(\gamma_j)^2 &= \epsilon_{ijk}\PD{t}{B^k} & \quad \mbox{if \(\mu = 0\)} \\
\PD{x^j}{E^m}\sigma_m \cdot (-\sigma_i) &= -\PD{x^j}{E^i} & \quad \mbox{if \(\mu = j\)} \\
\PD{x^i}{E^m}\sigma_m \cdot (\sigma_j) &= \PD{x^i}{E^j} & \quad \mbox{if \(\mu = i\)} \\
\end{array} \right.
\end{equation*}

Reassembling the integral we have

\begin{equation}\label{eqn:stokesMaxwellApplication:500}
\begin{aligned}
0
&= \int dx^0 dx^i dx^j \left( \PD{x^i}{E^j} -\PD{x^j}{E^i} + \epsilon_{ijk} \PD{t}{B^k} \right) \\
&= \int dx^0 \epsilon_{ijk} \int dx^i dx^j \sigma_k \left( \sigma_k \epsilon_{ijk} \left(\PD{x^i}{E^j} -\PD{x^j}{E^i}\right) + \sigma_k \PD{t}{B^k} \right)
\end{aligned}
\end{equation}

Summing over \(k\), we can pick out the differential form of Faraday's law

\begin{equation}
0 = \PD{t}{\BB} + \spacegrad \cross \BE
\end{equation}

as well as the integral form

\begin{equation}\label{eqn:stokesMaxwellApplication:520}
\begin{aligned}
0
&=
\sum_k
\int dx^i dx^j \sigma_k \left( \sigma_k \epsilon_{ijk} \left(\PD{x^i}{E^j} -\PD{x^j}{E^i}\right) + \sigma_k \PD{t}{B^k} \right) \\
&=
\sum_k \epsilon_{ijk} \int dx^j \left.E^j\right\vert_{\partial x^i}
-\sum_k \epsilon_{ijk} \int dx^i \left.E^i\right\vert_{\partial x^j}
+ \int \PD{t}{\BB} \cdot \ncap d\BA
\end{aligned}
\end{equation}

which is
\begin{equation}
0 = \ointctrclockwise \BE \cdot d\Br + \int \PD{t}{\BB} \cdot d\BA.
\end{equation}

\section{Conclusion}

In the treatment of these notes, the
traditional integral form of Maxwell's equations are
obtained directly from the STA Maxwell's equation
using the bivector Stokes equation, and various spacetime integration volumes.

\subsection{Summary of results}

We started with the bivector form of Stokes law

\begin{equation}
\iiint (\grad \wedge F) \cdot d^3\Bx = \oiintclockwise F \cdot d^2\Bx,
\end{equation}

and the multivector Maxwell equation

\begin{equation}\label{eqn:stokesMax:summaryMaxwell}
\grad F = J/\epsilon_0 c.
\end{equation}

The trivector parts of this can be integrated directly.  This integral is always zero for all spacetime or spatial surfaces

\begin{equation}\label{eqn:stokesMaxwellApplication:540}
\begin{aligned}
\int \left(\grad \wedge F\right) \cdot d^3 \Bx &= 0
\end{aligned}
\end{equation}

Duality relations were used to put the vector parts of \eqnref{eqn:stokesMax:summaryMaxwell} into a form that Stokes can be applied to.  This gives us

\begin{equation}\label{eqn:stokesMaxwellApplication:560}
\begin{aligned}
\oiintctrclockwise F \cdot (d^2\Bx I) &= \int \frac{J}{\epsilon_0 c} \cdot (d^3\Bx I).
\end{aligned}
\end{equation}

Integration of the trivector parts

\begin{equation}
\iiint (\grad \wedge F) \cdot d^3 \Bx = \oiintclockwise F \cdot d^2\Bx = 0,
\end{equation}

produces a combined electric and magnetic field form of a Faraday's law and Gauss' magneto-statics law that does not look terribly familiar

\begin{equation}\label{eqn:stokesMax:summaryGaussMandFaraday}
\oiintclockwise \BE \cdot d^2\Bx = - c \oiintclockwise \BB \cdot (d^2\Bx I),
\end{equation}

but integration of this using a spatial volume produces the familiar Gauss's magneto-static law

\begin{equation}\label{eqn:stokesMaxwellApplication:580}
\begin{aligned}
\iint \BB \cdot d\BA &= 0 \\
\spacegrad \cdot \BB  &= 0.
\end{aligned}
\end{equation}

Integration and summation of the same trivector parts in \eqnref{eqn:stokesMax:summaryGaussMandFaraday} over each of the possible three spacetime volumes gives us Faraday's law
in its familiar forms
\begin{equation}\label{eqn:stokesMaxwellApplication:600}
\begin{aligned}
\PD{t}{\BB} + \spacegrad \cross \BE &= 0 \\
\ointctrclockwise \BE \cdot d\Br + \int \PD{t}{\BB} \cdot d\BA &= 0.
\end{aligned}
\end{equation}

Now, the vector parts of Maxwell's multivector equation integrated over a spatial volume produces Gauss's law for electrostatics

\begin{equation}\label{eqn:stokesMaxwellApplication:620}
\begin{aligned}
\iint \BE \cdot d\BA &= \int \frac{\rho}{\epsilon_0} dV \\
\spacegrad \cdot \BE &= \frac{\rho}{\epsilon_0}.
\end{aligned}
\end{equation}

Finally, integration of the same with summation over all spacetime volumes gives us the famous Maxwell-Ampere equation
\begin{equation}\label{eqn:stokesMaxwellApplication:640}
\begin{aligned}
\spacegrad \cross \BB &= \mu_0 \left( \Bj + \epsilon_0 \PD{t}{\BE} \right) \\
\ointctrclockwise \BB \cdot d\Br &= \mu_0 \left(\int \Bj \cdot d\BA + \epsilon_0 \int \PD{t}{\BE} \cdot d\BA \right).
\end{aligned}
\end{equation}

In the process of arriving at these results it appears that some of the use of Stokes equation was actually superfluous.  One of the first things
that was done once the area elements were established was to undo the boundary integral writing things once more in terms of the partials over those boundaries.
Doing all this with just the volume integrals would possibly have been simpler.  That said, as an exercise to validate the generalized Stokes equation
formulation it worked well!

Conceptually the idea that integration of Maxwell's equation over various volumes produces all the traditional vector differential and integral forms
that we are used to is quite nice.  It seems less arbitrary than trying to figure out the exactly what specific projection like operations, as done in
\chapcite{PJMaxwellProj}, will produce the various traditional vector differential equations.  Of course those can be used once found to develop the integral relations,
but here we get them all in one shot.
%The old 1966 Britannica article uses the integral forms

\subsection{Getting a glimpse of how the pieces fit together?}

I think I am starting to see a bit of the big picture for electrodynamics.
In \chapcite{PJMaxwell2}, an earlier treatment of Maxwell's equations in a GA context, I used
dimensional analysis to group electric and magnetic fields in a logical way, and employs the spatial pseudoscalar to combine divergence and curl terms.  This
I thought was a good motivation for the STA form of the equation, using
ideas familiar from school.  Similar treatments can be found elsewhere
such as in \citep{doran2003gap}
but understanding that takes a lot more work.

Once the STA form is taken as more fundamental, one can take that and show
the types of spacetime projection operations, as in
\chapcite{PJMaxwellProj}, and produce the various traditional vector
differential forms of Maxwell's equations.  Alternatively, as in
\chapcite{PJMaxwellTensor}, we can extract the traditional tensor
form of the equations.

From an even higher level point of view we can relate the STA Maxwell's
equations to the least action principles, as done in
\citep{classicalmechanics:PJSrLagrangian}, to
find the Lorentz force law in STA form using the Euler-Lagrange equations,
and finally in \citep{classicalmechanics:PJMaxwellLagrangian} where the STA
form of Maxwell's equation is obtained directly from a complex valued
field Lagrangian.

Goldstein
\citep{goldstein1951cm}
has an interesting treatment of a combined Lagrangian for both
the Lorentz force law and the field equations (using spatial delta functions).  Minimization of the action for that Lagrangian with respect to the potential
produces the field equations, and with respect to coordinates produces the
Lorentz force law.  Have to work through that in a covariant form to see
how this relates to my previous treatments.

\subsection{Followup}

It would be interesting to see if any of the problems in a Maxwell's equation
text like
\citep{fleisch2007ssg}
would be any easier with a combined field as
is possible in the STA formulation (ie: the ones based on just current
or charge distributions).

There is also some interesting looking treatments of complex number residue like integrals for the field equation
in references such as
\citep{HestenesFormsGA}.
I re-encountered that paper after writing up these notes.  I had seen it before but
those parts that cover (tersely) the same material as above did not make much sense until I had independently worked it all
out in detail myself.
Perhaps I am dense, but I find that many academic papers are ironically not very good at all for learning from!

I believe these residue/green's function ideas both relate to the
Biot-Savart law, as mentioned in
\citep{HestenesFormsGA}, \citep{doran2003gap}, and
\citep{DenkerWire}.  All of those are either too terse or have details missing
that indicate I need to study the ideas in more depth to understand.

%\bibliographystyle{plain}

%
% Copyright � 2012 Peeter Joot.  All Rights Reserved.
% Licenced as described in the file LICENSE under the root directory of this GIT repository.
%

%
%
\chapter{Tensor relations from bivector field equation}\label{chap:PJMaxwellTensor}
\index{Maxwell's equations!tensor}
%\date{Sept 7, 2008.  maxwellToTensor.tex}

\section{Motivation}

This contains a somewhat unstructured collection of notes translating between tensor and bivector forms of Maxwell's equation(s).

\section{Electrodynamic tensor}

John Denker's paper \citep{DenkerMaxwell} writes:

\begin{equation}
F = (\BE + ic\BB) \gamma_0,
\end{equation}

with
\begin{equation}\label{eqn:maxwellToTensor:20}
\begin{aligned}
\BE &= E^i \gamma_i \\
\BB &= B^i \gamma_i.
\end{aligned}
\end{equation}

Since he uses the positive end of the metric for spatial indices this works fine.  Contrast to \citep{doran2003gap} who write:

\begin{equation}
F = \BE + ic\BB,
\end{equation}

with the following implied spatial bivector representation:
\begin{equation}\label{eqn:maxwellToTensor:40}
\begin{aligned}
\BE &= E^i \sigma_i = E^i \gamma_{i0} \\
\BB &= B^i \sigma_i = B^i \gamma_{i0}.
\end{aligned}
\end{equation}

That implied representation was not obvious to me, but I eventually figured out what they meant.  They also use \(c=1\), so I have added it back in here for clarity.

The end result in both cases is a pure bivector representation for the complete field:

\begin{equation*}
F = E^j \gamma_{j0} + ic B^j \gamma_{j0}.
\end{equation*}

%ASIDE: Is bivector the term used for a completely grade two multivector, but not neccessarily a wedge product?
%This field multivector is not definitely not a blade a term reserved for something that can be created by wedging (simple element in grassman algebra terms).
%Here the field multivector cannot be expressed as the wedge product of two vectors unless one of the electric or magnetic fields is entirely zero (essentially
%reducing the dimension of the spanning basis to a 3D bivector).

Let us look at the \(B^j\) basis bivectors a bit more closely:

\begin{equation*}
i\gamma_{j0}
= \gamma_{0123j0}
= -\gamma_{01230j}
= +\gamma_{00123j}
= (\gamma_0)^2 \gamma_{123j}.
\end{equation*}

Where,
\begin{equation*}
\gamma_{123j} =
\left\{
\begin{array}{l l}
(\gamma_{j})^2 \gamma_{23} & \quad \mbox{if \(j = 1\)} \\
(\gamma_{j})^2 \gamma_{31} & \quad \mbox{if \(j = 2\)} \\
(\gamma_{j})^2 \gamma_{12} & \quad \mbox{if \(j = 3\)}.
\end{array} \right.
\end{equation*}

Combining these results we have a \((\gamma_0)^2 (\gamma_{j})^2 = -1\) coefficient that is metric invariant, and can write:

\begin{equation*}
i \sigma_{j} =
i \gamma_{j0} =
\left\{
\begin{array}{l l}
\gamma_{32} & \quad \mbox{if \(j = 1\)} \\
\gamma_{13} & \quad \mbox{if \(j = 2\)} \\
\gamma_{21} & \quad \mbox{if \(j = 3\)}.
\end{array} \right.
\end{equation*}

% -1:23
% -2:31
% -3:12

Or, more compactly:

\begin{equation*}
i \sigma_{a} =
i \gamma_{a0} =
-\epsilon_{abc} \gamma_{bc}.
\end{equation*}

Putting things back together, our bivector field in index notation is:

\begin{equation}\label{eqn:maxToTensor:Fcomp}
F = E^i \gamma_{i 0} - \epsilon_{i j k} c B^i \gamma_{j k}.
\end{equation}

\subsection{Tensor components}

Now, given a grade two multivector such as our field, how can we in general compute the components of that field given any arbitrary basis.  This can be done using the reciprocal bivector frame:

\begin{equation*}
F = \sum a_{{\mu} {\nu}} (e_{\mu} \wedge e_{\nu}).
\end{equation*}

To calculate the coordinates \(a_{{\mu} {\nu}}\) we can dot with
\(e^{\nu} \wedge e^{\mu}\):

\begin{equation}\label{eqn:maxwellToTensor:60}
\begin{aligned}
F \cdot (e^{\nu} \wedge e^{\mu})
&= \sum a_{{\alpha} {\beta}} (e_{\alpha} \wedge e_{\beta}) \cdot (e^{\nu} \wedge e^{\mu}) \\
&= ( a_{{\mu} {\nu}} (e_{\mu} \wedge e_{\nu}) + a_{{\nu} {\mu}} (e_{\nu} \wedge e_{\mu}) ) \cdot (e^{\nu} \wedge e^{\mu}) \\
&= a_{{\mu} {\nu}} - a_{{\nu} {\mu}} \\
&= 2 a_{{\mu} {\nu}}.
\end{aligned}
\end{equation}

Therefore
\begin{equation*}
F = \inv{2} \sum (F \cdot (e^{\nu} \wedge e^{\mu})) (e_{\mu} \wedge e_{\nu}) = \sum_{{\mu}<{\nu}} (F \cdot (e^{\nu} \wedge e^{\mu})) (e_{\mu} \wedge e_{\nu}).
\end{equation*}

With \(F^{{\mu} {\nu}} = F \cdot (e^{\nu} \wedge e^{\mu})\) and summation convention:

\begin{equation}
F = \inv{2} F^{{\mu} {\nu}} (e_{\mu} \wedge e_{\nu}).
\end{equation}

It is not hard to see that the representation with respect to the reciprocal frame, with
\(F_{{\mu} {\nu}} = F \cdot (e_{\nu} \wedge e_{\mu})\) must be:

\begin{equation}
F = \inv{2} F_{{\mu} {\nu}} (e^{\mu} \wedge e^{\nu}).
\end{equation}

Writing \(F^{\mu\nu}\) or \(F_{\mu\nu}\) leaves a lot unspecified.  You will get a different tensor for each choice of basis.  Using this form amounts to the equivalent of using the matrix of a linear transformation with respect to a specified basis.

\subsection{Electromagnetic tensor components}

Next, let us calculate these
\(F_{{\mu} {\nu}}\), and \(F^{{\mu} {\nu}}\) values and relate them to our electric and magnetic fields so we can work in or translate to and from all of the traditional vector, the tensor, and the Clifford/geometric languages.

\begin{equation*}
F^{{\mu} {\nu}} = \left( E^i \gamma_{i 0} - \epsilon_{i j k} c B^i \gamma_{j k} \right) \cdot \gamma^{\nu\mu}.
\end{equation*}

By inspection our electric field components we have:

\begin{equation*}
F^{i0} = E^i,
\end{equation*}

and for the magnetic field:

\begin{equation*}
F^{{i} {j}} = - \epsilon_{k i j} c B^k = - \epsilon_{i j k} c B^k.
\end{equation*}

Putting in sample numbers this is:

\begin{equation}\label{eqn:maxwellToTensor:80}
\begin{aligned}
F^{{3} {2}} &= - \epsilon_{3 2 1} c B^1 = c B^1 \\
F^{{1} {3}} &= - \epsilon_{1 3 2} c B^2 = c B^2 \\
F^{{2} {1}} &= - \epsilon_{2 1 3} c B^3 = c B^3.
\end{aligned}
\end{equation}

This can be summarized in matrix form:

\begin{equation}\label{eqn:maxToTensor:matrixtensor}
F^{\mu\nu} =
\begin{bmatrix}
0   & -E^1 & -E^2 & -E^3 \\
E^1 &   0  & -c B^3 &  c B^2 \\
E^2 &  c B^3 &   0  & -c B^1 \\
E^3 & -c B^2 &  c B^1 &   0
\end{bmatrix}
.
\end{equation}

Observe that no specific reference to a metric was required to evaluate these components.

\subsection{reciprocal tensor (name?)}

The reciprocal frame representation of \eqnref{eqn:maxToTensor:Fcomp} is %metric dependent when expressed

\begin{equation}\label{eqn:maxwellToTensor:100}
\begin{aligned}
F
&= E^i \gamma_{i 0} - \epsilon_{i j k} c B^i \gamma_{j k} \\
&= -E^i \gamma^{i 0} - \epsilon_{i j k} c B^i \gamma^{j k}.
\end{aligned}
\end{equation}

%FIXME: WHAT AM I TALKING ABOUT HERE "is metric dependent"?

Calculation of the reciprocal representation of the field tensor \(F_{{\mu} {\nu}} = F \cdot \gamma_{\nu\mu}\) is now possible, and by inspection
%, regardless of the metric:

\begin{equation}\label{eqn:maxwellToTensor:120}
\begin{aligned}
F_{i0} &= -E^i = -F^{i0} \\
F_{ij} &= - \epsilon_{i j k} c B^k = F^{ij}.
\end{aligned}
\end{equation}

So, all the electric field components in the tensor have inverted sign:
\begin{equation*}
F_{\mu\nu} =
\begin{bmatrix}
0   & E^1 & E^2 & E^3 \\
-E^1 &   0  & -c B^3 &  c B^2 \\
-E^2 &  c B^3 &   0  & -c B^1 \\
-E^3 & -c B^2 &  c B^1 &   0  \\
\end{bmatrix}
.
\end{equation*}

This is metric independent with this bivector based definition of \(F_{\mu\nu}\), and \(F^{\mu\nu}\).  Surprising, since I thought I had read otherwise.

\subsection{Lagrangian density}

\citep{doran2003gap}
%Doran/Lasenby
write the Lagrangian density in terms of \(\gpgradezero{F^2}\), whereas Denker writes it in terms of \(\gpgradezero{F \tilde{F}}\).  Is their
alternate choice in metric responsible for this difference.

Reversing the field since it is a bivector, just inverts the sign:

\begin{equation}\label{eqn:maxwellToTensor:140}
\begin{aligned}
F &= E^i \gamma_{i 0} - \epsilon_{i j k} c B^i \gamma_{j k} \\
\tilde{F} &= E^i \gamma_{0 i} - \epsilon_{i j k} c B^i \gamma_{k j} = -F.
\end{aligned}
\end{equation}

So the choice of \(\gpgradezero{F^2}\) vs. \(\gpgradezero{F \tilde{F}}\) is just a sign choice, and does not have anything to do with the metric.

Let us evaluate one of these:

\begin{equation}\label{eqn:maxwellToTensor:160}
\begin{aligned}
F^2
&=
(E^i \gamma_{i 0} - \epsilon_{i j k} c B^i \gamma_{j k}) (E^u \gamma_{u 0} - \epsilon_{u v w} c B^u \gamma_{v w})  \\
&=
E^i E^u \gamma_{i 0} \gamma_{u 0}
- \epsilon_{u v w} E^i c B^u \gamma_{v w} \gamma_{i 0}
- \epsilon_{i j k} E^u c B^i \gamma_{j k} \gamma_{u 0}
+ \epsilon_{i j k} \epsilon_{u v w} c^2 B^i B^u \gamma_{v w} \gamma_{j k}.
\end{aligned}
\end{equation}

That first term is:

\begin{equation}\label{eqn:maxwellToTensor:180}
\begin{aligned}
E^i E^u \gamma_{i 0} \gamma_{u 0}
&= \BE^2 + \sum_{i \ne j} E^i E^j ( \sigma_i \sigma_j + \sigma_j \sigma_i ) \\
&= \BE^2 + \sum_{i \ne j} 2 E^i E^j \sigma_i \cdot \sigma_j \\
&= \BE^2.
\end{aligned}
\end{equation}

Hmm.  This is messy.  Let us try with \(F = \BE + i c \BB\) directly (with the Doran/Lasenby convention: \(\BE = E^k \sigma_k\)) :

\begin{equation}\label{eqn:maxwellToTensor:200}
\begin{aligned}
F^2
&= (\BE + i c \BB) (\BE + i c \BB) \\
&= \BE^2 + c^2 (i \BB) (i \BB) + c (i \BB \BE + \BE i \BB) \\
&= \BE^2 + c^2 (\BB i) (i \BB) + i c (\BB \BE + \BE \BB) \\
&= \BE^2 - c^2 \BB^2 + 2 i c (\BB \cdot \BE).
\end{aligned}
\end{equation}

\subsubsection{Compared to tensor form}

Now lets compare to the tensor form, where the Lagrangian density is written in terms of the product of upper and lower index tensors:

\begin{equation}\label{eqn:maxwellToTensor:220}
\begin{aligned}
F_{\mu\nu}F^{\mu\nu}
&= F_{i 0}F^{i 0} +F_{0 i}F^{0 i} +\sum_{i<j} F_{i j}F^{i j} +\sum_{j<i} F_{i j}F^{i j} \\
&= 2 F_{i 0}F^{i 0} + 2 \sum_{i<j} F_{i j}F^{i j} \\
&= 2 (-E^i)(E^i) + 2 \sum_{i<j} (F^{i j})^2 \\
&= -2 \BE^2 + 2 \sum_{i<j} ( -\epsilon_{ijk} c B^k )^2 \\
&= -2 ( \BE^2 - c^2 \BB^2 ).
\end{aligned}
\end{equation}

Summarizing with a comparison of the bivector and tensor forms we have:

\begin{equation}
\inv{2} F_{\mu\nu}F^{\mu\nu} = c^2 \BB^2 - \BE^2 = - \gpgradezero{F^2} = \gpgradezero{ F \tilde{F} }.
\end{equation}

But to put this in context we need to figure out how to apply this in the Lagrangian.  That appears to require a potential formulation of the field equations, so that is the next step.

\subsubsection{Potential and relation to electromagnetic tensor}

Since the field is a bivector is it reasonable to assume that it may be possible to express as the curl of a vector

\begin{equation*}
F = \grad \wedge A.
\end{equation*}

Inserting this into the field equation we have:
\begin{equation}\label{eqn:maxwellToTensor:240}
\begin{aligned}
\grad (\grad \wedge A)
&= \grad \cdot (\grad \wedge A) + \mathLabelBox{\grad \wedge \grad}{\(=0\)} \wedge A \\
&= \grad^2 A - \grad ( \grad \cdot A ) \\
&= \inv{\epsilon_0 c} J.
\end{aligned}
\end{equation}

With application of the gauge condition \(\grad \cdot A = 0\), one is left with the four scalar equations:

\begin{equation}\label{eqn:maxToTensor:potential}
\grad^2 A = \inv{\epsilon_0 c} J.
\end{equation}

This can also be seen more directly since the gauge condition implies:

\begin{equation*}
\grad \wedge A = \grad \wedge A + \grad \cdot A = \grad A,
\end{equation*}

from which \eqnref{eqn:maxToTensor:potential} follows directly.  Observe that although the field equation was not metric
dependent, the equivalent potential equation is since it has a squared Laplacian.

\subsubsection{Index raising or lowering}
Any raising or lowering of indices, whether it be in the partials or the basis vectors corresponds to a multiplication by a \((\gamma_{\alpha})^2 = \pm 1\) value, so doing this twice cancels out \((\pm 1)^2 = 1\).

Vector coordinates in the reciprocal basis is translated by such a squared
factor when we are using an orthonormal basis:

\begin{equation}\label{eqn:maxwellToTensor:260}
\begin{aligned}
x
&= \sum \gamma^{\mu} ( \gamma_{\mu} \cdot x ) \\
&= \sum \gamma^{\mu} x_{\mu} \\
&= \sum \gamma^{\mu} (\gamma^{\mu} \gamma_{\mu}) x_{\mu} \\
&= \sum (\gamma^{\mu})^2 \gamma_{\mu} x_{\mu},
\end{aligned}
\end{equation}

therefore

\begin{equation*}
x^{\mu} = x \cdot \gamma^{\mu} = (\gamma^{\mu})^2 x_{\mu}.
\end{equation*}

Similarly our partial derivatives can be raised or lowered since they are just derivatives in terms of one of the choices of coordinates

\begin{equation*}
\partial_{\mu} = \PD{x^{\mu}}{} = \PD{(\gamma_{\mu})^2 x_{\mu}}{} = (\gamma_{\mu})^2 \PD{x_{\mu}}{} = (\gamma_{\mu})^2 \partial^{\mu},
\end{equation*}

when written as a gradient, we have two pairs of \((\gamma_{\mu})^2\) factors that cancel if we switch both indices:

\begin{equation*}
\grad = \gamma^{\mu} \PD{x^{\mu}}{} = (\gamma_{\mu})^2 (\gamma_{\mu})^2 \gamma_{\mu} \PD{x_{\mu}}{} = (\pm 1)^2 \gamma_{\mu} \PD{x_{\mu}}{}.
\end{equation*}

Or in short with the notation above

\begin{equation*}
\grad = \gamma^{\mu} \partial_{\mu} = \gamma_{\mu} \partial^{\mu}.
\end{equation*}

\subsubsection{Back to tensor in terms of potential}
Utilizing matched raising and lowering of indices, our field can be written in any of the following ways
%This metric dependency also shows up if one calculates the em tensor in terms of potential.

\begin{equation}\label{eqn:maxwellToTensor:280}
\begin{aligned}
\grad \wedge A
&= {\gamma_{\mu}} \wedge \gamma_{\nu} \partial^{\mu} A^{\nu} = \sum_{\mu<\nu} {\gamma_{\mu}} \wedge \gamma_{\nu} \left( \partial^{\mu} A^{\nu} - \partial^{\nu} A^{\mu} \right) \\
&= {\gamma^{\mu}} \wedge \gamma^{\nu} \partial_{\mu} A_{\nu} = \sum_{\mu<\nu} {\gamma^{\mu}} \wedge \gamma^{\nu} \left( \partial_{\mu} A_{\nu} - \partial_{\nu} A_{\mu} \right) \\
&= {\gamma_{\mu}} \wedge \gamma^{\nu} \partial^{\mu} A_{\nu} = \sum_{\mu<\nu} {\gamma_{\mu}} \wedge \gamma^{\nu} \left( \partial^{\mu} A_{\nu} - \partial_{\nu} A^{\mu} \right) \\
&= {\gamma^{\mu}} \wedge \gamma_{\nu} \partial_{\mu} A^{\nu} = \sum_{\mu<\nu} {\gamma^{\mu}} \wedge \gamma_{\nu} \left( \partial_{\mu} A^{\nu} - \partial^{\nu} A_{\mu} \right).
\end{aligned}
\end{equation}

These implicitly define the tensor in terms of potential, so we can write:
Calculating the tensor in terms of the bivector we have:

\begin{equation}\label{eqn:maxToTensor:tensorpot}
\begin{aligned}
F^{\mu\nu} &= F \cdot (\gamma^{\nu} \wedge \gamma^{\mu}) = \partial^{\mu} A^{\nu} - \partial^{\nu} A^{\mu} \\
F_{\mu\nu} &= F \cdot (\gamma_{\nu} \wedge \gamma_{\mu}) = \partial_{\mu} A_{\nu} - \partial_{\nu} A_{\mu} \\
{F^{\mu}}_{\nu} &= F \cdot (\gamma^{\nu} \wedge \gamma_{\mu}) = \partial^{\mu} A_{\nu} - \partial_{\nu} A^{\mu} \\
{F_{\mu}}^{\nu} &= F \cdot (\gamma_{\nu} \wedge \gamma^{\mu}) = \partial_{\mu} A^{\nu} - \partial^{\nu} A_{\mu}.
\end{aligned}
\end{equation}

These potential based equations of \eqnref{eqn:maxToTensor:tensorpot}, are consistent with the definition of the field tensor in terms of potential in the
\href{http://en.wikipedia.org/wiki/Covariant\_formulation\_of\_classical\_electromagnetism}{ wikipedia Covariant electromagnetism } article.
That article's definition of the field tensor is also consistent with the field tensor in matrix form of \eqnref{eqn:maxToTensor:matrixtensor}.

However, the \href{http://en.wikipedia.org/wiki/Electromagnetic_tensor}{wikipedia Electromagnetic Tensor}
uses different conventions (at the time of this writing), but both claim a \(-+++\) metric, so I think one is wrong.  I had naturally favor the
covariant article since it agrees with my results.

\subsection{Field equations in tensor form}

\begin{equation}\label{eqn:maxwellToTensor:300}
\begin{aligned}
J/c \epsilon_0
&= \grad (\grad \wedge A) \\
&= \grad \cdot ( \grad \wedge A ) + \grad \wedge \grad \wedge A.
\end{aligned}
\end{equation}

This produces two equations
\begin{equation*}
\grad \cdot ( \grad \wedge A ) = J/c \epsilon_0
\end{equation*}
\begin{equation*}
\grad \wedge \grad \wedge A = 0.
\end{equation*}

\subsubsection{Vector equation part}

Expanding the first in coordinates we have
\begin{equation}\label{eqn:maxwellToTensor:320}
\begin{aligned}
J/c \epsilon_0
&= \gamma^{\alpha} \partial_{\alpha} \cdot ( \gamma^{\mu} \wedge \gamma_{\nu} \partial_{\mu} A^{\nu} ) \\
&= (\gamma^{\alpha} \cdot \gamma_{\mu\nu}) \partial_{\alpha} \partial^{\mu} A^{\nu} \\
&= (
%%\gamma^{\alpha} \cdot \gamma_{\mu}_{\nu}
\delta^{\alpha}_{\mu} \gamma_{\nu}
-\delta^{\alpha}_{\nu} \gamma_{\mu}
) \partial_{\alpha} \partial^{\mu} A^{\nu} \\
&= ( \gamma_{\nu} \partial_{\mu} - \gamma_{\mu} \partial_{\nu} ) \partial^{\mu} A^{\nu} \\
&= \gamma_{\nu} \partial_{\mu} (\partial^{\mu} A^{\nu} -\partial^{\nu} A^{\mu} ) \\
&= \gamma_{\nu} \partial_{\mu} F^{\mu\nu}.
\end{aligned}
\end{equation}

Dotting the LHS with \(\gamma^{\alpha}\) we have
\begin{equation}\label{eqn:maxwellToTensor:340}
\begin{aligned}
\gamma^{\alpha} \cdot J/c \epsilon_0
&= \gamma^{\alpha} \cdot \gamma_{\beta }J^{\beta}/c \epsilon_0 \\
&= \delta^{\alpha}_{\beta }J^{\beta}/c \epsilon_0 \\
&= J^{\alpha}/c \epsilon_0.
\end{aligned}
\end{equation}

and for the RHS
\begin{equation}\label{eqn:maxwellToTensor:360}
\begin{aligned}
\gamma^{\alpha} \cdot \gamma_{\nu} \partial_{\mu} F^{\mu\nu}
&= \partial_{\mu} F^{\mu\alpha}.
\end{aligned}
\end{equation}

Or,
\begin{equation}
\partial_{\mu} F^{\mu\alpha} = J^{\alpha}/c \epsilon_0.
\end{equation}

This is exactly (with index switch) the tensor equation in
\href{http://en.wikipedia.org/wiki/Covariant\_formulation\_of\_classical\_electromagnetism}{ wikipedia Covariant electromagnetism } article.
It however, differs from the
\href{http://en.wikipedia.org/wiki/Electromagnetic_tensor}{wikipedia Electromagnetic Tensor} article.

\subsubsection{Trivector part}

Now, the trivector part of this equation does not seem like it is worth much consideration

\begin{equation}
\grad \wedge \grad \wedge A = 0.
\end{equation}

But this is four of the eight traditional Maxwell's equations when written out in terms of coordinates.  Let us write this out in tensor form
and see how this follows.

\begin{equation}\label{eqn:maxwellToTensor:380}
\begin{aligned}
\grad \wedge \grad \wedge A
&= (\gamma^{\alpha} \partial_{\alpha}) \wedge (\gamma^{\beta} \partial_{\beta}) \wedge (\gamma^{\sigma} A_{\sigma}) \\
&= (\gamma^{\alpha} \wedge \gamma^{\beta} \wedge \gamma^{\sigma}) \partial_{\alpha} \partial_{\beta} A_{\sigma} \\
&=
\inv{2}
(\gamma^{\alpha} \wedge \gamma^{\beta} \wedge \gamma^{\sigma}) \partial_{\alpha} \partial_{\beta} A_{\sigma}
+
\inv{2}
(\gamma^{\alpha} \wedge \gamma^{\sigma} \wedge \gamma^{\beta}) \partial_{\alpha} \partial_{\sigma} A_{\beta}
\\
&=
\inv{2}(\gamma^{\alpha} \wedge \gamma^{\beta} \wedge \gamma^{\sigma}) \partial_{\alpha} (\partial_{\beta} A_{\sigma} - \partial_{\sigma} A_{\beta}) \\
&= \inv{2}(\gamma^{\alpha} \wedge \gamma^{\beta} \wedge \gamma^{\sigma}) \partial_{\alpha} F_{\beta\sigma}.
\end{aligned}
\end{equation}

For each of the four trivectors that span the trivector space the coefficients of those trivectors must all therefore equal zero.  The duality set

\begin{equation*}
\{ i \gamma^{\mu} \}
\end{equation*}

can be used to enumerate these four equations, so to separate these from the wedge products we have to perform the dot products.  Here \(i\) can be any pseudoscalar
associated with the four vector space, and it will be convenient to use an "index-up" pseudoscalar \(i=\gamma^{0123}\).   This will still anticommute with any of the \(\gamma^{\mu}\) vectors.

\begin{equation}\label{eqn:maxwellToTensor:400}
\begin{aligned}
(\gamma^{\alpha} \wedge \gamma^{\beta} \wedge \gamma^{\sigma}) \cdot ( i \gamma^{\mu} )
&= \gpgradezero{ (\gamma^{\alpha} \wedge \gamma^{\beta} \wedge \gamma^{\sigma}) ( i \gamma^{\mu} ) } \\
&= -\gpgradezero{ \gamma^{\alpha} \gamma^{\beta} \gamma^{\sigma} \gamma^{\mu 0123} } \\
&= -\gpgradezero{ \gamma^{\alpha \beta \sigma \mu 0123} } \\
&= \epsilon^{ \alpha \beta \sigma \mu }.
\end{aligned}
\end{equation}

The last line follows with the observation that the scalar part will be zero unless \(\alpha\), \(\beta\), \(\sigma\), and \(\mu\) are all unique.  When they are \(0,1,2,3\) for example then we have \(i^2 = -1\), and any odd permutation will change the sign.

% i^2 : metric independent
%i^2
%=01230123
%=-00123123
%=-00112323
%=00112233
%=00 (\pm 1)^2 33
%=00 33
%=-1

Application of this to our curl of curl expression we have

\begin{equation*}
(\grad \wedge \grad \wedge A) \cdot (i \gamma^{\mu} ) = \inv{2}\epsilon^{ \alpha \beta \sigma \mu } \partial_{\alpha} F_{\beta\sigma}.
\end{equation*}

Because there are no magnetic sources, the one-half scale factor can be dropped, which leaves the remaining four equations of Maxwell's equation in standard tensor form

\begin{equation}
\epsilon^{ \alpha \beta \sigma \mu } \partial_{\alpha} F_{\beta\sigma} = 0.
\end{equation}

One of these will be Gauss's law \(\spacegrad \cdot \BB = 0\), and the other three can be summed in vector form for Faraday's law \(\spacegrad \cross \BE + \PD{t}{\BB} = 0\).

\subsection{Lagrangian density in terms of potential}

We have seen that we can write the core of the Lagrangian density in two forms:

\begin{equation*}
\inv{2} F_{\mu\nu}F^{\mu\nu} = -\gpgradezero{F^2} = c^2 \BB^2 -\BE^2,
\end{equation*}

where summarizing the associated relations we have:

\begin{equation*}
F = \BE + i c \BB = \inv{2} F^{\mu\nu} \gamma_{\mu\nu} = \grad \wedge A = E^i \gamma_{i 0} - \epsilon_{i j k} c B^i \gamma_{j k}
\end{equation*}
\begin{equation}\label{eqn:maxwellToTensor:420}
\begin{aligned}
F^{\mu\nu} &= \partial^{\mu} A^{\nu} - \partial^{\nu} A^{\mu} \\
F_{\mu\nu} &= \partial_{\mu} A_{\nu} - \partial_{\nu} A_{\mu} \\
F^{i0} &= E^i = -F_{i0} \\
F^{ij} &= -\epsilon_{i j k} c B^k = F_{ij}.
\end{aligned}
\end{equation}

Now, if we want the density in terms of potential, by inspection we can form this from the tensor as:

\begin{equation*}
\inv{2} F_{\mu\nu}F^{\mu\nu} = \inv{2} (\partial_{\mu} A_{\nu} - \partial_{\nu} A_{\mu} ) (\partial^{\mu} A^{\nu} - \partial^{\nu} A^{\mu} ).
\end{equation*}

We should also be able to calculate this directly from the bivector square.  Lets verify this:

\begin{equation}\label{eqn:maxwellToTensor:440}
\begin{aligned}
\gpgradezero{ F^2 }
&= \gpgradezero{ (\grad \wedge A)(\grad \wedge A) } \\
&= \gpgradezero{ (\gamma^{\mu} \wedge \gamma_{\nu} \partial_{\mu} A^{\nu}) (\gamma^{\alpha} \wedge \gamma_{\beta} \partial_{\alpha} A^{\beta}) } \\
&= (\gamma^{\mu} \wedge \gamma^{\nu} \partial_{\mu} A_{\nu}) \cdot (\gamma_{\alpha} \wedge \gamma_{\beta} \partial^{\alpha} A^{\beta}) \\
&= ( ( (\gamma^{\mu} \wedge \gamma^{\nu}) \cdot \gamma_{\alpha} ) \cdot \gamma_{\beta} ) \partial_{\mu} A_{\nu} \partial^{\alpha} A^{\beta} \\
&= \left( \delta^{\mu}_{\beta} \delta^{\nu}_{\alpha} - \delta^{\mu}_{\alpha} \delta^{\nu}_{\beta} \right) \partial_{\mu} A_{\nu} \partial^{\alpha} A^{\beta} \\
&= \partial_{\mu} A_{\nu} \partial^{\nu} A^{\mu} - \partial_{\mu} A_{\nu} \partial^{\mu} A^{\nu} \\
&= \partial_{\mu} A_{\nu} \left( \partial^{\nu} A^{\mu} - \partial^{\mu} A^{\nu} \right) \\
&=
\inv{2} \left( \partial_{\mu} A_{\nu} \left( \partial^{\nu} A^{\mu} - \partial^{\mu} A^{\nu} \right)
+\partial_{\nu} A_{\mu} \left( \partial^{\mu} A^{\nu} - \partial^{\nu} A^{\mu} \right) \right)
\\
&= \inv{2} \left( \partial_{\nu} A_{\mu} - \partial_{\mu} A_{\nu} \right) \left( \partial^{\nu} A^{\mu} - \partial^{\mu} A^{\nu} \right) \\
&= -\inv{2} F_{\mu\nu}F^{\mu\nu},
\end{aligned}
\end{equation}

as expected.

The factor of \(1/2\) appearance is a \(x = (1/2)(x + x)\) operation, plus a switch of dummy indices in one half of the sum.

With the density expanded completely in terms of potentials things are in a form for an attempt to
evaluate the Lagrangian equations or do the
variational exercise (as in Feynman
\citep{feynman1963flp}
with the electrostatic case) and see that this recovers the field equations (covered in a subsequent set of notes in both fashions).

%
% Copyright � 2012 Peeter Joot.  All Rights Reserved.
% Licenced as described in the file LICENSE under the root directory of this GIT repository.
%

%
%
\chapter{Four vector potential}
\index{vector potential}
\index{four potential}
\label{chap:emPotential}
%\date{August 15, 2008}

\section{}

Goldstein's classical mechanics, and many other texts, will introduce the four potential starting with
Maxwell's equation in scalar, vector, bivector, trivector expanded form:

\begin{equation}\label{eqn:emPotential:20}
\begin{aligned}
\spacegrad \cdot \BE &= \frac{\rho}{\epsilon_0} \\
\spacegrad \cdot \BB &= 0 \\
\spacegrad \cross \BE &= - \frac{\partial \BB}{\partial t} \\
\spacegrad \cross \BB &= \mu_0\left(\BJ + \epsilon_0 \frac{\partial \BE}{\partial t}\right) \\
\end{aligned}
\end{equation}

ie: E can not be a gradient, since it has a curl, but B can be the curl of something since it has zero
divergence, so we have \(\BB = \spacegrad \cross \BA\).  Faraday's law above gives:

\begin{equation}\label{eqn:emPotential:40}
\begin{aligned}
0 &= \spacegrad \cross \BE + \frac{\partial \spacegrad \cross \BA}{\partial t} \\
&= \spacegrad \cross \left(\BE + \frac{\partial \BA}{\partial t}\right) \\
\end{aligned}
\end{equation}

Because this curl is zero, one can write it as a gradient, say \(-\spacegrad \phi\).

The end result are the equations:

\begin{align}
\BE &= - \left( \spacegrad \phi + \partial_t \BA \right) \label{eqn:fourPot:BE} \\
\BB &= \spacegrad \cross \BA \label{eqn:fourPot:BB}
\end{align}

Looking at what Goldstein does with this (which I re-derived above to put in the SI form I am used to), my
immediate question is how would the combined bivector field look when expressed using an STA basis, and
then once that is resolved, how would his Lagrangian for a charged point particle look in explicit four
vector form?

Intuition says that this is all going to work out to be a spacetime gradient of a four vector, but
I am not sure how the Lorentz gauge freedom will turn out.  Here is an exploration of this.

\subsection{}

Forming as usual

\begin{equation}\label{eqn:fourPot:BF}
\BF = \BE + i c \BB
\end{equation}

We can combine the equations \eqnref{eqn:fourPot:BE} and \eqnref{eqn:fourPot:BB} into bivector form

\begin{equation}\label{eqn:fourPot:BFA}
\BF = - \left( \spacegrad \phi + \partial_t \BA \right) + c \spacegrad \wedge \BA
\end{equation}

\subsection{Dimensions}

Let us do a dimensional check before continuing:

\Eqnref{eqn:fourPot:BF} gives:

\begin{equation*}
[\BE] = \frac{[m][d]}{[q][t]^2}
\end{equation*}

That and \eqnref{eqn:fourPot:BFA} gives
\begin{equation*}
[\phi] = \frac{[m][d]^2}{[q][t]^2}
\end{equation*}

And the two \(\BA\) terms of \eqnref{eqn:fourPot:BFA} both give:
\begin{equation*}
[\BA] = \frac{[m][d]}{[q][t]}.
\end{equation*}

Therefore if we create a four vector out of \(\phi\), and \(\BA\) in SI units we will need that factor \(c\) with \(\BA\) with velocity dimensions to fix things up.

\subsection{Expansion in terms of components.  STA split}

\begin{equation}\label{eqn:emPotential:80}
\begin{aligned}
\BF
&= - \left( \spacegrad \phi + \partial_t \BA \right) + c \spacegrad \wedge \BA \\
&= - \sum \gamma_i \gamma_0 \partial_{x^i}\phi - \sum \gamma_i \gamma_0 \partial_t A^i
+ c \left(\sum \sigma_i \partial_{x^i}\right) \wedge \left(\sum \sigma_j A^j\right) \\
&= \sum \gamma^i \partial_{x^i} (\gamma_0 \phi) + \sum \gamma_0 \partial_{ct} c \gamma_i A^i
- \left(\sum \gamma_i \partial_{x^i}\right) \wedge \left(\sum \gamma_j c A^j\right) \\
&= \sum \gamma^i \wedge \gamma_0 \partial_{x^i} \phi + \sum \gamma^0 \wedge \gamma_i \partial_{x^0} c A^i
+ \sum \gamma^i \wedge \gamma_j \partial_{x^i} c A^j \\
&= \left(\sum \gamma^i \partial_{x^i}\right) \wedge \left(\gamma_0 \phi + \gamma_i c A^i \right) \\
&= \grad \wedge \left(\gamma_0 \phi + \sum \gamma_i c A^i \right)
\end{aligned}
\end{equation}

Once the electric and magnetic fields are treated as one entity, the separate equations of
\eqnref{eqn:fourPot:BE} and \eqnref{eqn:fourPot:BB} become nothing more than a statement that the bivector field \(\BF\) is the spacetime curl
of a four vector potential \(A = \gamma_0 \phi + \sum \gamma_i c A^i\).

This original choice of components \(A^i\), defined such that \(\BB = \spacegrad \cross \BA\) is a bit unfortunate in SI
units.  Setting \(\calA^i = cA^i\), and \(\calA^0 = \phi\), one then has the more symmetric form.

\begin{equation*}
A = \sum \gamma_{\mu} \calA^{\mu}.
\end{equation*}

Of course the same thing could be achieved with \(c=1\) ;)

Anyways, substitution of this back into Maxwell's equation gives:

\begin{equation*}
\grad (\grad \wedge A) = \grad \cdot (\grad \wedge A) + \mathLabelBox{\grad \wedge \grad \wedge A}{\(=0\)} = J
\end{equation*}

One can see an immediate simplification possible if one requires:

\begin{equation*}
\grad \cdot A = 0.
\end{equation*}

Then we are left with a forced wave equation to solve for the four potential:

\begin{equation*}
\grad^2 A = -\left(\sum \partial_{x^i x^i} - \inv{c^2}\partial_{tt}\right) A = J.
\end{equation*}

Now, without all this mess of algebra, I could have gone straight to this end result (and had done so previously).  I just
wanted to see where I would get applying the STA basis to the classical vector+scalar four vector ideas.

\subsection{Lorentz gauge}
\index{Lorentz gauge}

Looking at \(\grad \cdot A = 0\), I was guessing that this was what I recalled being called the Lorentz gauge, but in a
slightly different form.

If one expands this you get:

\begin{equation}\label{eqn:emPotential:100}
\begin{aligned}
0
&= \grad \cdot A \\
&= \sum \gamma^{\mu} \partial_{\mu} \cdot \left(\gamma_0 \phi + c \sum \gamma_j A^j \right) \\
&= \partial_{ct} \phi + c \sum \partial_{x^i}A^i \\
&= \partial_{ct} \phi + c \spacegrad \cdot \BA
\end{aligned}
\end{equation}

Or,

\begin{equation}
\spacegrad \cdot \BA = -\inv{c^2}\partial_{t} \phi
\end{equation}

Checked my Feynman book.  Yes, this is the Lorentz Gauge.

Another note.  Again the SI units make things ugly.  With the above modification of components that hide this, where one sets \(A = \sum \gamma_i \calA^i\), this gauge equation also takes a simpler form:

\begin{equation*}
0 = \grad \cdot A = \left(\sum \gamma^{\mu} \partial_{x^{\mu}}\right) \cdot \left(\sum \gamma_{\nu} \calA^{\nu}\right) = \sum \partial_{x^{\mu}} \calA^{\mu}.
\end{equation*}

\section{Appendix}

\subsection{wedge of spacetime bivector basis elements}

For \(i \ne j\):

\begin{equation}\label{eqn:emPotential:120}
\begin{aligned}
\sigma_i \wedge \sigma_j
&= \inv{2}\left( \sigma_i \sigma_j - \sigma_j \sigma_i \right) \\
&= \inv{2}\left( \gamma_{i0j0} - \gamma_{j0i0} \right) \\
&= \inv{2}\left( -\gamma_{ij} +\gamma_{ji} \right) \\
&= \gamma_{ji}
\end{aligned}
\end{equation}

\documentclass{article}

\usepackage{amsmath}
\usepackage{mathpazo}

%
% shorthand for bold symbols, convenient for vectors and matrices
%
\newcommand{\Ba}[0]{\mathbf{a}}
\newcommand{\Bb}[0]{\mathbf{b}}
\newcommand{\Bc}[0]{\mathbf{c}}
\newcommand{\Bd}[0]{\mathbf{d}}
\newcommand{\Be}[0]{\mathbf{e}}
\newcommand{\Bf}[0]{\mathbf{f}}
\newcommand{\Bg}[0]{\mathbf{g}}
\newcommand{\Bh}[0]{\mathbf{h}}
\newcommand{\Bi}[0]{\mathbf{i}}
\newcommand{\Bj}[0]{\mathbf{j}}
\newcommand{\Bk}[0]{\mathbf{k}}
\newcommand{\Bl}[0]{\mathbf{l}}
\newcommand{\Bm}[0]{\mathbf{m}}
\newcommand{\Bn}[0]{\mathbf{n}}
\newcommand{\Bo}[0]{\mathbf{o}}
\newcommand{\Bp}[0]{\mathbf{p}}
\newcommand{\Bq}[0]{\mathbf{q}}
\newcommand{\Br}[0]{\mathbf{r}}
\newcommand{\Bs}[0]{\mathbf{s}}
\newcommand{\Bt}[0]{\mathbf{t}}
\newcommand{\Bu}[0]{\mathbf{u}}
\newcommand{\Bv}[0]{\mathbf{v}}
\newcommand{\Bw}[0]{\mathbf{w}}
\newcommand{\Bx}[0]{\mathbf{x}}
\newcommand{\By}[0]{\mathbf{y}}
\newcommand{\Bz}[0]{\mathbf{z}}
\newcommand{\BA}[0]{\mathbf{A}}
\newcommand{\BB}[0]{\mathbf{B}}
\newcommand{\BC}[0]{\mathbf{C}}
\newcommand{\BD}[0]{\mathbf{D}}
\newcommand{\BE}[0]{\mathbf{E}}
\newcommand{\BF}[0]{\mathbf{F}}
\newcommand{\BG}[0]{\mathbf{G}}
\newcommand{\BH}[0]{\mathbf{H}}
\newcommand{\BI}[0]{\mathbf{I}}
\newcommand{\BJ}[0]{\mathbf{J}}
\newcommand{\BK}[0]{\mathbf{K}}
\newcommand{\BL}[0]{\mathbf{L}}
\newcommand{\BM}[0]{\mathbf{M}}
\newcommand{\BN}[0]{\mathbf{N}}
\newcommand{\BO}[0]{\mathbf{O}}
\newcommand{\BP}[0]{\mathbf{P}}
\newcommand{\BQ}[0]{\mathbf{Q}}
\newcommand{\BR}[0]{\mathbf{R}}
\newcommand{\BS}[0]{\mathbf{S}}
\newcommand{\BT}[0]{\mathbf{T}}
\newcommand{\BU}[0]{\mathbf{U}}
\newcommand{\BV}[0]{\mathbf{V}}
\newcommand{\BW}[0]{\mathbf{W}}
\newcommand{\BX}[0]{\mathbf{X}}
\newcommand{\BY}[0]{\mathbf{Y}}
\newcommand{\BZ}[0]{\mathbf{Z}}

\newcommand{\Bzero}[0]{\mathbf{0}}
\newcommand{\Btheta}[0]{\boldsymbol{\theta}}
\newcommand{\Btau}[0]{\boldsymbol{\tau}}
\newcommand{\Bomega}[0]{\boldsymbol{\omega}}

%
% shorthand for unit vectors
%
\newcommand{\acap}[0]{\hat{\Ba}}
\newcommand{\bcap}[0]{\hat{\Bb}}
\newcommand{\ccap}[0]{\hat{\Bc}}
\newcommand{\dcap}[0]{\hat{\Bd}}
\newcommand{\ecap}[0]{\hat{\Be}}
\newcommand{\fcap}[0]{\hat{\Bf}}
\newcommand{\gcap}[0]{\hat{\Bg}}
\newcommand{\hcap}[0]{\hat{\Bh}}
\newcommand{\icap}[0]{\hat{\Bi}}
\newcommand{\jcap}[0]{\hat{\Bj}}
\newcommand{\kcap}[0]{\hat{\Bk}}
\newcommand{\lcap}[0]{\hat{\Bl}}
\newcommand{\mcap}[0]{\hat{\Bm}}
\newcommand{\ncap}[0]{\hat{\Bn}}
\newcommand{\ocap}[0]{\hat{\Bo}}
\newcommand{\pcap}[0]{\hat{\Bp}}
\newcommand{\qcap}[0]{\hat{\Bq}}
\newcommand{\rcap}[0]{\hat{\Br}}
\newcommand{\scap}[0]{\hat{\Bs}}
\newcommand{\tcap}[0]{\hat{\Bt}}
\newcommand{\ucap}[0]{\hat{\Bu}}
\newcommand{\vcap}[0]{\hat{\Bv}}
\newcommand{\wcap}[0]{\hat{\Bw}}
\newcommand{\xcap}[0]{\hat{\Bx}}
\newcommand{\ycap}[0]{\hat{\By}}
\newcommand{\zcap}[0]{\hat{\Bz}}
\newcommand{\thetacap}[0]{\hat{\Btheta}}

%
% to write R^n and C^n in a distinguishable fashion.  Perhaps change this
% to the double lined characters upon figuring out how to do so.
%
\newcommand{\C}[1]{$\mathbb{C}^{#1}$}
\newcommand{\R}[1]{$\mathbb{R}^{#1}$}

%
% various generally useful helpers
%

% derivative of #1 wrt. #2:
\newcommand{\D}[2] {\frac {d#2} {d#1}}

\newcommand{\inv}[1]{\frac{1}{#1}}
\newcommand{\cross}[0]{\times}

\newcommand{\abs}[1]{\lvert{#1}\rvert}
\newcommand{\norm}[1]{\lVert{#1}\rVert}
\newcommand{\innerprod}[2]{\langle{#1}, {#2}\rangle}
\newcommand{\dotprod}[2]{{#1} \cdot {#2}}
\newcommand{\bdotprod}[2]{\left({#1} \cdot {#2}\right)}
\newcommand{\crossprod}[2]{{#1} \cross {#2}}
\newcommand{\tripleprod}[3]{\dotprod{\left(\crossprod{#1}{#2}\right)}{#3}}

\DeclareMathOperator{\Proj}{Proj}
\DeclareMathOperator{\Span}{span}
\DeclareMathOperator{\Sgn}{sgn}
\DeclareMathOperator{\Area}{Area}
\DeclareMathOperator{\Volume}{Volume}

%
% A few miscellaneous things specific to this document
%
\newcommand{\crossop}[1]{\crossprod{#1}{}}

% R2 vector.
\newcommand{\VectorTwo}[2]{
\begin{bmatrix}
 {#1} \\
 {#2}
\end{bmatrix}
}

\newcommand{\VectorN}[1]{
\begin{bmatrix}
{#1}_1 \\
{#1}_2 \\
\vdots \\
{#1}_N \\
\end{bmatrix}
}

\newcommand{\DETuvij}[4]{
\begin{vmatrix}
 {#1}_{#3} & {#1}_{#4} \\
 {#2}_{#3} & {#2}_{#4}
\end{vmatrix}
}

\newcommand{\DETuvwijk}[6]{
\begin{vmatrix}
 {#1}_{#4} & {#1}_{#5} & {#1}_{#6} \\
 {#2}_{#4} & {#2}_{#5} & {#2}_{#6} \\
 {#3}_{#4} & {#3}_{#5} & {#3}_{#6}
\end{vmatrix}
}

\newcommand{\DETuvwxijkl}[8]{
\begin{vmatrix}
 {#1}_{#5} & {#1}_{#6} & {#1}_{#7} & {#1}_{#8} \\
 {#2}_{#5} & {#2}_{#6} & {#2}_{#7} & {#2}_{#8} \\
 {#3}_{#5} & {#3}_{#6} & {#3}_{#7} & {#3}_{#8} \\
 {#4}_{#5} & {#4}_{#6} & {#4}_{#7} & {#4}_{#8} \\
\end{vmatrix}
}

%\newcommand{\DETuvwxyijklm}[10]{
%\begin{vmatrix}
% {#1}_{#6} & {#1}_{#7} & {#1}_{#8} & {#1}_{#9} & {#1}_{#10} \\
% {#2}_{#6} & {#2}_{#7} & {#2}_{#8} & {#2}_{#9} & {#2}_{#10} \\
% {#3}_{#6} & {#3}_{#7} & {#3}_{#8} & {#3}_{#9} & {#3}_{#10} \\
% {#4}_{#6} & {#4}_{#7} & {#4}_{#8} & {#4}_{#9} & {#4}_{#10} \\
% {#5}_{#6} & {#5}_{#7} & {#5}_{#8} & {#5}_{#9} & {#5}_{#10}
%\end{vmatrix}
%}

% R3 vector.
\newcommand{\VectorThree}[3]{
\begin{bmatrix}
 {#1} \\
 {#2} \\
 {#3}
\end{bmatrix}
}


\newcommand{\gpgrade}[2] {{\left\langle{{#1}}\right\rangle}_{#2}}
\newcommand{\gpgradezero}[1] {\gpgrade{#1}{0}}
\newcommand{\gpgradetwo}[1] {\gpgrade{#1}{2}}
\newcommand{\gpgradefour}[1] {\gpgrade{#1}{4}}
\newcommand{\grad}[0]{\nabla}
\newcommand{\spacegrad}[0]{\boldsymbol{\nabla}}
% == \partial_{#1} {#2}
\newcommand{\PD}[2]{\frac{\partial {#2}}{\partial {#1}}}
\newcommand{\PDD}[3]{\frac{\partial^2 {#3}}{\partial {#1}\partial {#2}}}
\newcommand{\PDsq}[2]{\frac{\partial^2 {#2}}{\partial^2 {#1}}}

\title{ Metric signature dependencies for electromagnetic equations. }
\author{Peeter Joot}
\date{ Sept 5, 2008.  Last Revision: $Date: 2008/09/07 04:55:12 $ }

\begin{document}

\maketitle{}

\section{ Motivation. }

Doran/Lasenby use a $+,-,-,-$ signature, and I had gotten used to that.  On first seeing the alternate signature used by John Denker's excellent
explainatory paper:

http://www.av8n.com/physics/maxwell-ga.pdf

I found myself disoriented.  How many of the identities that I was used to were metric dependent?   Here are some notes that explore some of the
metric dependencies of STA, in particular observing which identities are metric dependent and which aren't.

\section{}

\subsection{ Spatial basis. }

Our spatial (bivector) basis:

\begin{equation*}
\sigma_i = \gamma_i \wedge \gamma_0 = \gamma_{i0},
\end{equation*}

that behaves like Euclidean vectors (positive square) still behave as desired, regardless of the signature:

\begin{align*}
\sigma_i \cdot \sigma_j
&= \gpgradezero{\gamma_{i0j0}}  \\
&= - \gpgradezero{\gamma_{ij}} (\gamma_{0})^2  \\
&= -\delta_{ij} (\gamma_i)^2 (\gamma_{0})^2
\end{align*}

Regardless of the signature the pair of products $(\gamma_i)^2 (\gamma_{0})^2 = -1$, so our spatial bivectors are metric invariant.

\subsection{ How about commutation? }

Commutation with
\begin{equation*}
i \gamma_{\mu} = \gamma_{0123\mu} = \gamma_{\mu0123}
\end{equation*}

$\mu$ has to "pass" three indexes regardless of metric, so anticommutes for any $\mu$.

\begin{equation*}
\sigma_k \gamma_{\mu} = \gamma_{k0\mu}
\end{equation*}

If $k = \mu$, or $0 = \mu$, then we get a sign inversion, and otherwise commute (pass two indexes).  This is also metric invariant.

\subsection{ Spatial and time component selection. }

With a postive time metric (Doran/Lasenby) selection of the $x^0$ component of a vector $x$ requires a dot product:

\begin{equation*}
x = x^0 \gamma_0 + x^i \gamma_i
\end{equation*}

\begin{equation*}
x \cdot \gamma_0 = x^0 (\gamma_0)^2
\end{equation*}

Obviously this is a metric dependent operation.  To generalize it appropriately, we need to dot with $\gamma^0$ instead:

\begin{equation*}
x \cdot \gamma^0 = x^0
\end{equation*}

Now, what do we get when wedging with this upper index quantity instead.

\begin{align*}
x \wedge \gamma^0 
&= \left(x^0 \gamma_0 + x^i \gamma_i\right) \wedge \gamma^0 \\
&= x^i \gamma_i \wedge \gamma^0 \\
&= x^i \gamma_{i0} (\gamma^0)^2 \\
&= x^i \sigma_i (\gamma^0)^2 \\
&= \Bx \left(\gamma^0\right)^2
\end{align*}

Not quite the usual expression we are used to, but it still behaves as a euclidan vector (positive square), regardless of the metric:

\begin{equation*}
(x \wedge \gamma^0)^2 = \left(\pm \Bx\right)^2 = \Bx^2
\end{equation*}

This suggests that we should define our spatial projection vector as $x \wedge \gamma^0$ instead of $x \wedge \gamma_0$ as done in 
Doran/Lasenby (where a positive time metric is used).

\subsubsection{ Velocity. }

Variation of a event path with some parameter we have:

\begin{align*}
\frac{ d x }{ d \lambda } 
&= \frac{ d x^{\mu} }{ d \lambda } \gamma_{\mu} = c \frac{dt}{d\lambda} \gamma_0 + \frac{d x^i }{d\lambda} \gamma_i \\
&= \frac{d t}{d \lambda} \left( c \gamma_0 + \frac{d x^i }{dt} \gamma_i \right)
\end{align*}

The square of this is:
%becomes metric dependent:
\begin{align*}
\inv{c^2} \left(\frac{ d x }{ d \lambda } \right)^2
&= \left(\frac{dt }{d\lambda}\right)^2 (\gamma_0)^2 \left( 1 + \inv{c^2}\left(\frac{d x^i }{dt}\right)^2 (\gamma_i)^2 (\gamma_0)^2 \right) \\
&= \left(\frac{d t}{d \lambda}\right)^2 (\gamma_0)^2 \left( 1 - (\Bv/c)^2 \right) \\
\frac{ (\gamma_0)^2 }{c^2} \left(\frac{ d x }{ d \lambda } \right)^2 &= \left(\frac{d t}{d \lambda}\right)^2 \left( 1 - (\Bv/c)^2 \right) \\
\end{align*}

We define the proper time $\tau$ as that particular parameterization $c \tau = \lambda$ such that the LHS equals 1.  This is implicitly defined
via the integral

\begin{equation*}
\tau = \int \sqrt{ 1 - (\Bv/c)^2 } dt = \int \sqrt{ 1 - \left(\inv{c} \frac{dx^i }{d \alpha} \right)^2 } d\alpha
\end{equation*}

Regardless of this parameterization $\alpha = \alpha(t)$, this velocity scaled 4D arc length is the same.  This is a bit of a digression from the
ideas of metric dependence investigation.  There is however a metric dependence in the first steps arriving at this result.

with proper velocity defined in terms of proper time $v = dx/d\tau$, we also have:

\begin{equation}\label{eqn:gamma}
\gamma = \frac{dt}{d\tau} = \inv{ \sqrt{ 1 - (\Bv/c)^2 } }
\end{equation}
\begin{equation}
v = \gamma \left(c \gamma_0 + \frac{d x^i }{d t} \gamma_i \right)
\end{equation}

Therefore we can select this quantity $\gamma$, and our spatial velocity components, from our proper velocity:

\begin{equation*}
c \gamma = v \cdot \gamma^0
\end{equation*}

In equation \ref{eqn:gamma} we didn't define $\Bv$, only implictly requiring that it's square was $\sum (dx^i/dt)^2$, as we require for correspondence with Euclidean meaning.  This can be made more exact by
taking wedge products to weed out the time component:

\begin{equation*}
v \wedge \gamma^0 = \gamma \frac{d x^i }{d t} \gamma_i \wedge \gamma^0 
\end{equation*}

With a definition of $\Bv = \frac{d x^i }{d t} \gamma_i \wedge \gamma^0$ (which has the desired positive square), we therefore have:

\begin{align*}
\Bv
&= \frac{v \wedge \gamma^0 }{\gamma} \\
&= \frac{v \wedge \gamma^0 }{ v/c \cdot \gamma^0 } \\
\end{align*}

Or,
\begin{equation}
\Bv/c = \frac{v/c \wedge \gamma^0 }{ v/c \cdot \gamma^0 }
\end{equation}

All the lead up to this allows for expression of the spatial component of the proper velocity in a metric independent fashion.

\subsection{ Reciprocal Vectors. }

By reciprocal frame we mean the set of vectors $\{u^{\alpha}\}$ associated with a basis
for some linear subspace $\{u_{\alpha}\}$ such that:

\begin{equation*}
u_{\alpha} \cdot u^{\beta} = \delta_{\alpha}^\beta
\end{equation*}

In the special case of orthonormal vectors $u_{\alpha} \cdot u_{\beta} = \pm \delta_{\alpha\beta}$ the reciprocal frame vectors
are just the inverses (literally reciprocals), which can be verified by taking dot products:

\begin{align*}
\inv{u_{\alpha}} \cdot {u_{\alpha}}
&= \gpgradezero{ \inv{u_{\alpha}} {u_{\alpha}} } \\
&= \gpgradezero{ \inv{u_{\alpha}} \frac{u_{\alpha}}{u_{\alpha}} {u_{\alpha}} } \\
&= \gpgradezero{ \frac{(u_{\alpha})^2}{(u_{\alpha})^2} } \\
&= 1
\end{align*}

Written out explicitly for our positive "orthonormal" time metric:

\begin{align*}
(\gamma_0)^2 &= 1 \\
(\gamma_i)^2 &= -1,
\end{align*}

we have the reciprocal vectors:
\begin{align*}
\gamma_0 &= \gamma^0 \\
\gamma_i &= -\gamma^i \\
\end{align*}

Note that this last statement is consistent with $(\gamma_i)^2 = -1$, since $(\gamma_i)^2 = \gamma_i (-\gamma^i) = -\delta_i^i = -1$

Contrast this with a positive spatial metric:

\begin{align*}
(\gamma_0)^2 &= -1 \\
(\gamma_i)^2 &= 1,
\end{align*}

with reciprocal vectors:
\begin{align*}
\gamma_0 &= -\gamma^0 \\
\gamma_i &= \gamma^i \\
\end{align*}

where we have the opposite.

\subsection{ Reciprocal Bivectors. }

Now, let's examine the bivector reciprocals.  Given our orthonormal vector basis, let's invert the bivector and verify that is what we want:

\begin{align*}
\inv{\gamma_{\mu\nu}}
&= \inv{\gamma_{\mu\nu}} \frac{ \gamma_{\nu\mu} }{ \gamma_{\nu\mu} } \\
&= \inv{\gamma_{\mu\nu}} \inv{ \gamma_{\nu\mu} }{ \gamma_{\nu\mu} } \\
&= \inv{\gamma_{\mu\nu\nu\mu} }{ \gamma_{\nu\mu}} \\
&= \inv{ (\gamma_{\mu})^2 (\gamma_{\nu})^2 } { \gamma_{\nu\mu}} \\
\end{align*}

Multiplication with our vector we will get 1 if this has the required reciprocal relationship:
\begin{align*}
\inv{\gamma_{\mu\nu}} \gamma_{\mu\nu}
&= \inv{ (\gamma_{\mu})^2 (\gamma_{\nu})^2 } { \gamma_{\nu\mu}} \gamma_{\mu\nu} \\
&= \frac{ (\gamma_{\mu})^2 (\gamma_{\nu})^2 }{ (\gamma_{\mu})^2 (\gamma_{\nu})^2 } \\
&= 1
\end{align*}

Observe that unlike our basis vectors the bivector reciprocals are metric independant.  Let's verify this explicitly:

\begin{align*}
\inv{\gamma_{i0}} &= \inv{ (\gamma_{i})^2 (\gamma_{0})^2 } { \gamma_{0i}} \\
\inv{\gamma_{ij}} &= \inv{ (\gamma_{i})^2 (\gamma_{j})^2 } { \gamma_{ji}} \\
\inv{\gamma_{0i}} &= \inv{ (\gamma_{0})^2 (\gamma_{i})^2 } { \gamma_{i0}} \\
\end{align*}

With a spacetime mix of indexes we have a $-1$ denominator for either metric.  With a spatial only mix ($B$ components) we have $1$ in the denominator $1^2 = (-1)^2$ for either metric.

Now, perhaps counter to intuition the reciprocal $\inv{\gamma_{\mu\nu}}$ of $\gamma_{\mu\nu}$ is not $\gamma^{\mu\nu}$, but instead $\gamma^{\nu\mu}$.  Here the shorthand can be deceptive and it is worth verifying this statement explicitly:

\begin{align*}
\gamma_{\mu\nu} \cdot \gamma^{\alpha\beta}
&= (\gamma_{\mu} \wedge \gamma_{\nu}) \cdot (\gamma^{\alpha} \wedge \gamma^{\beta}) \\
&= ((\gamma_{\mu} \wedge \gamma_{\nu}) \cdot \gamma^{\alpha}) \cdot \gamma^{\beta}) \\
&= ( \gamma_{\mu} (\gamma_{\nu} \cdot \gamma^{\alpha}) - \gamma_{\nu} (\gamma_{\mu} \cdot \gamma^{\alpha}) ) \cdot \gamma^{\beta}) \\
&= ( \gamma_{\mu} {\delta_{\nu}}^{\alpha} - \gamma_{\nu} {\delta_{\mu}}^{\alpha} ) \cdot \gamma^{\beta} \\
\end{align*}

Or,
\begin{equation}
\gamma_{\mu\nu} \cdot \gamma^{\alpha\beta} = {\delta_{\mu}}^{\beta} {\delta_{\nu}}^{\alpha} - {\delta_{\nu}}^{\beta} {\delta_{\mu}}^{\alpha}
\end{equation}

In particular for matched pairs of indexes we have:
\begin{equation*}
\gamma_{\mu\nu} \cdot \gamma^{\nu\mu} = {\delta_{\mu}}^{\mu} {\delta_{\nu}}^{\nu} - {\delta_{\nu}}^{\mu} {\delta_{\mu}}^{\nu} = 1
\end{equation*}

\subsection{ Pseudoscalar expresed with reciprocal frame vectors. }

With a positive time metric

\begin{equation*}
\gamma_{0123} = -\gamma^{0123}
\end{equation*}

(three inversions for each of the spatial quantities).  This is metric invariant too since it will match the single negation for the same operation
using a positive spatial metric.

\subsection{ Spatial bivector basis commutation with pseudoscalar. }

I have been used to writing:
\begin{equation*}
\sigma_j = \gamma_{j0}
\end{equation*}

as a spatial basis, and having this equivalent to the four-pseudoscalar, but this only works with a time positive metric:
\begin{equation*}
i_3 = \sigma_{123} = \gamma_{102030} = \gamma_{0123} (\gamma_0)^2
\end{equation*}

With the spatial positive spacetime metric we therefore have:

\begin{equation*}
i_3 = \sigma_{123} = \gamma_{102030} = -i_4
\end{equation*}

instead of $i_3 = i_4$ as is the case with a time positive spacetime metric.  We see that the metric choice can also be interpretted as a choice of handedness.

That choice allowed Doran/Lasenby to initially write the field as a vector plus trivector where $i$ is the spatial pseudoscalar:

\begin{equation}\label{eqn:field}
F = \BE + i c \BB,
\end{equation}

and then later switch the interpretation of $i$ to the four space pseudoscalar.  The freedom to do so is metric dependent freedom, but
equation \ref{eqn:field} works regardless of metric when $i$ is uniformly interpretted as the spacetime pseudoscalar.

Regardless of the metric the spacetime pseudoscalar commutes with $\sigma_j = \gamma_{j0}$, since it anticommutes twice to cross:

\begin{equation*}
\sigma_j i = \gamma_{j00123} = \gamma_{00123j} = \gamma_{0123j0} = i \sigma_j
\end{equation*}

\subsection{ Gradient and Laplacian. }

As seen by the Lagrangian based derivation of the (spacetime or spatial) gradient, the form is metric independant and valid even for non-orthonormal frames:

\begin{equation*}
\grad = \gamma^{\mu} \PD{x^{\mu}}{}
\end{equation*}

\subsubsection{ Vector derivative. }

A cute aside, as pointed out in John Denker's paper, for orthonormal frames, this can also be written as:

\begin{equation}\label{eqn:gradient}
\grad = \inv{\gamma_{\mu}} \PD{x^{\mu}}{}
\end{equation}

as a mnemonic for remembering where the signs go, since in that form the upper and lower indexes are nicely matched in summation convention fashion.

Now, $\gamma_{\mu}$ is a constant when we are not working in curvalinear coordinates, and for constants we are used to the freedom to pull them into our
derivatives as in:

\begin{equation*}
\inv{c} \PD{t}{} = \PD{(ct)}{}
\end{equation*}

Supposing that one had an orthogonal vector decomposition:

\begin{equation*}
\Bx = \sum \gamma_i x^i = \sum \Bx_i
\end{equation*}

then, we can abuse notation and do the same thing with our unit vectors, rewriting the gradient equation \ref{eqn:gradient} as:

\begin{equation}\label{eqn:gradvec}
\grad = \PD{(\gamma_{\mu} x^{\mu})}{} = \sum \PD{\Bx_i}{}
\end{equation}

Is there anything to this that isn't just abuse of notation?  I think so, and I'm guessing the notational freedom to do this is closely related to
what Hestenes calls geometric calculus.

Expanding out the gradient in the form of equation \ref{eqn:gradvec} as a limit statement this becomes, rather loosely:

\begin{equation*}
\grad = \sum_i \lim_{d\Bx_i \to 0} \inv{ d \Bx_i } \left(f( \Bx + d\Bx_i ) - f( \Bx )\right)
\end{equation*}

If nothing else this justifies the notation for the polar form gradient of a function that is only radially dependent, where the quantity:

\begin{equation*}
\spacegrad = \rcap\PD{r}{} = \inv{\rcap}\PD{r}{}
\end{equation*}

is sometimes written:

\begin{equation*}
\spacegrad = \PD{\Br}{}
\end{equation*}

Tong does this for example in his online dynamics paper, although there it appears to be not much more than a fancy shorthand for gradient.

\subsection{ Four-Laplacian. }

Now, although our gradient is metric invarient, it's square the four-Laplacian is not.  There we have:

\begin{align*}
\grad^2
&= \sum (\gamma^{\mu})^2 \PDsq{x^{\mu}}{} \\
&= (\gamma^0)^2 \left( \PDsq{x^0}{} + (\gamma^0)^2 (\gamma^i)^2 \PDsq{x^i}{} \right) \\
&= (\gamma^0)^2 \left( \PDsq{x^0}{} - \PDsq{x^i}{} \right)
\end{align*}

This makes the metric dependency explicit so that we have:

\begin{equation*}
\grad^2 = \inv{c^2} \PDsq{t}{} - \PDsq{x^i}{} \quad \mbox{if $(\gamma^0)^2 = 1$}
\end{equation*}
\begin{equation*}
\grad^2 = \PDsq{x^i}{} - \inv{c^2} \PDsq{t}{} \quad \mbox{if $(\gamma^0)^2 = -1$}
\end{equation*}


\subsection{ Electrodynamic tensor. }

John Denker's paper writes:

\begin{equation}
F = (\BE + ic\BB) \gamma_0
\end{equation}

with
\begin{align*}
\BE &= E^i \gamma_i \\
\BB &= B^i \gamma_i
\end{align*}

Since he uses the postive end of the metric for spatial indexes this works fine.  Contrast to Doran/Lasenby who write:

\begin{equation}
F = \BE + ic\BB
\end{equation}

with the following implied spatial bivector representation:
\begin{align*}
\BE &= E^i \sigma_i = E^i \gamma_{i0} \\
\BB &= B^i \sigma_i = B^i \gamma_{i0}.
\end{align*}

That implied representation wasn't obvious to me, but I eventually figured out what they meant.  They also use $c=1$, so I've added it back in here for clarity.

The end result in both cases is a pure "bivector" representation for the complete field:

\begin{equation*}
F = E^j \gamma_{j0} + ic B^j \gamma_{j0}
\end{equation*}

ASIDE: Is bivector the term used for a completely grade two multivector, but not neccessarily a wedge product?
This field multivector is not definitely not a blade a term reserved for something that can be created by wedging (simple element in grassman algebra terms).
Here the field multivector cannot be expressed as the wedge product of two vectors unless one of the electric or magnetic fields is entirely zero (essentially
reducing the dimension of the spanning basis to a 3D bivector).

Let's look at the $B^j$ basis bivectors a bit more closely:

\begin{equation*}
i\gamma_{j0}
= \gamma_{0123j0}
= -\gamma_{01230j}
= +\gamma_{00123j}
= (\gamma_0)^2 \gamma_{123j}
\end{equation*}

Where,
\begin{equation*}
\gamma_{123j} =
\left\{
\begin{array}{l l}
(\gamma_{j})^2 \gamma_{23} & \quad \mbox{if $j = 1$} \\
(\gamma_{j})^2 \gamma_{31} & \quad \mbox{if $j = 2$} \\
(\gamma_{j})^2 \gamma_{12} & \quad \mbox{if $j = 3$} \\
\end{array} \right.
\end{equation*}

Combining these results we have a $(\gamma_0)^2 (\gamma_{j})^2 = -1$ coefficient that is metric invariant, and can write:

\begin{equation*}
i \sigma_{j} =
i \gamma_{j0} =
\left\{
\begin{array}{l l}
\gamma_{32} & \quad \mbox{if $j = 1$} \\
\gamma_{13} & \quad \mbox{if $j = 2$} \\
\gamma_{21} & \quad \mbox{if $j = 3$} \\
\end{array} \right.
\end{equation*}

% -1:23
% -2:31
% -3:12

Or, more compactly:

\begin{equation*}
i \sigma_{a} =
i \gamma_{a0} =
-\epsilon_{abc} \gamma_{bc}
\end{equation*}

Putting things back together, our bivector field in index notation is:

\begin{equation}\label{eqn:Fcomp}
F = E^i \gamma_{i 0} - \epsilon_{i j k} c B^i \gamma_{j k}
\end{equation}

\subsection{ Tensor components }

Now, given a grade two multivector such as our field, how can we in general compute the components of that field given any arbitrary basis.  This can be done using the reciprocal bivector frame:

\begin{equation*}
F = \sum a_{{\mu} {\nu}} (e_{\mu} \wedge e_{\nu})
\end{equation*}

To calculate the coordinates $a_{{\mu} {\nu}}$ we can dot with
$e^{\nu} \wedge e^{\mu}$:

\begin{align*}
F \cdot (e^{\nu} \wedge e^{\mu})
&= \sum a_{{\alpha} {\beta}} (e_{\alpha} \wedge e_{\beta}) \cdot (e^{\nu} \wedge e^{\mu}) \\
&= ( a_{{\mu} {\nu}} (e_{\mu} \wedge e_{\nu}) + a_{{\nu} {\mu}} (e_{\nu} \wedge e_{\mu}) ) \cdot (e^{\nu} \wedge e^{\mu}) \\
&= a_{{\mu} {\nu}} - a_{{\nu} {\mu}} \\
&= 2 a_{{\mu} {\nu}}
\end{align*}

Therefore
\begin{equation*}
F = \inv{2} \sum (F \cdot (e^{\nu} \wedge e^{\mu})) (e_{\mu} \wedge e_{\nu}) = \sum_{{\mu}<{\nu}} (F \cdot (e^{\nu} \wedge e^{\mu})) (e_{\mu} \wedge e_{\nu})
\end{equation*}

Or, with $F^{{\mu} {\nu}} = F \cdot (e^{\nu} \wedge e^{\mu})$ and summation convention:

\begin{equation}
F = \inv{2} F^{{\mu} {\nu}} (e_{\mu} \wedge e_{\nu})
\end{equation}

It is not hard to see that the representation with respect to the reciprocal frame, with
$F_{{\mu} {\nu}} = F \cdot (e_{\nu} \wedge e_{\mu})$ must be:

\begin{equation}
F = \inv{2} F_{{\mu} {\nu}} (e^{\mu} \wedge e^{\nu})
\end{equation}

Writing $F^{\mu\nu}$ or $F_{\mu\nu}$ leaves a lot unspecified.  You will get a different tensor for each choice of basis.  Using this form amounts to the equivalent of using the matrix of a linear transformation with respect to a specified basis.

\subsection{ Electromagnetic tensor components. }

Next, let's calculate these
$F_{{\mu} {\nu}}$, and $F^{{\mu} {\nu}}$ values and relate them to our electric and magnetic fields so we can work in or translate to and from all of the traditional vector, the tensor, and the clifford/geometric languages.

\begin{equation*}
F^{{\mu} {\nu}} = \left( E^i \gamma_{i 0} - \epsilon_{i j k} c B^i \gamma_{j k} \right) \cdot \gamma^{\nu\mu}
\end{equation*}

By inspection our electric field components we have:

\begin{equation*}
F^{i0} = E^i,
\end{equation*}

and for the magnetic field:

\begin{equation*}
F^{{i} {j}} = - \epsilon_{k i j} c B^k = - \epsilon_{i j k} c B^k.
\end{equation*}

Putting in sample numbers this is:

\begin{align*}
F^{{3} {2}} &= - \epsilon_{3 2 1} c B^1 = c B^1 \\
F^{{1} {3}} &= - \epsilon_{1 3 2} c B^2 = c B^2 \\
F^{{2} {1}} &= - \epsilon_{2 1 3} c B^3 = c B^3 \\
\end{align*}

This can be summarized in matrix form:

\begin{equation*}
F^{\mu\nu} =
\begin{bmatrix}
0   & -E^1 & -E^2 & -E^3 \\
E^1 &   0  & -c B^3 &  c B^2 \\
E^2 &  c B^3 &   0  & -c B^1 \\
E^3 & -c B^2 &  c B^1 &   0  \\
\end{bmatrix}
\end{equation*}

Observe that no specific reference to a metric was required to evaluate these components.

\subsection{ reciprocal tensor (name?) }

The reciprocal frame representation of equation \ref{eqn:Fcomp} is metric dependent when expressed

\begin{align*}
F
&= E^i \gamma_{i 0} - \epsilon_{i j k} c B^i \gamma_{j k} \\
&= -E^i \gamma^{i 0} - \epsilon_{i j k} c B^i \gamma^{j k}
\end{align*}

Calculation of the reciprocal representation of the field tensor $F_{{\mu} {\nu}} = F \cdot \gamma_{\nu\mu}$ is now possible, and by inspection, regardless
of the metric:

\begin{align*}
F_{i0} &= -E^a = -F^{i0} \\
F_{ij} &= - \epsilon_{i j k} c B^k \gamma^{i j} = F^{ij}
\end{align*}

So, all the electric field components in the tensor invert:
\begin{equation*}
F_{\mu\nu} =
\begin{bmatrix}
0   & E^1 & E^2 & E^3 \\
-E^1 &   0  & -c B^3 &  c B^2 \\
-E^2 &  c B^3 &   0  & -c B^1 \\
-E^3 & -c B^2 &  c B^1 &   0  \\
\end{bmatrix}
\end{equation*}

Again, this is metric independent with this bivector based definition of $F_{\mu\nu}$, and $F^{\mu\nu}$.  Suprising, since I thought I had read otherwise.

\subsection{ Lagrangian density. }

Doran/Lasenby write the Lagrangian density in terms of $\gpgradezero{F^2}$, whereas Denker writes it in terms of $\gpgradezero{F \tilde{F}}$.  Is their
alternate choice in metric responsible for this difference.

Reversing the field since it is a bivector, just inverts the sign:

\begin{align*}
F &= E^i \gamma_{i 0} - \epsilon_{i j k} c B^i \gamma_{j k} \\
\tilde{F} &= E^i \gamma_{0 i} - \epsilon_{i j k} c B^i \gamma_{k j} = -F
\end{align*}

So the choice of $\gpgradezero{F^2}$ vs. $\gpgradezero{F \tilde{F}}$ is just a sign choice, and does not have anything to do with the metric.

Let's evaluate one of these:

\begin{align*}
F^2
&=
(E^i \gamma_{i 0} - \epsilon_{i j k} c B^i \gamma_{j k}) (E^u \gamma_{u 0} - \epsilon_{u v w} c B^u \gamma_{v w})  \\
&=
E^i E^u \gamma_{i 0} \gamma_{u 0}
- \epsilon_{u v w} E^i c B^u \gamma_{v w} \gamma_{i 0}
- \epsilon_{i j k} E^u c B^i \gamma_{j k} \gamma_{u 0}
+ \epsilon_{i j k} \epsilon_{u v w} c^2 B^i B^u \gamma_{v w} \gamma_{j k}
\\
\end{align*}

That first term is:

\begin{align*}
E^i E^u \gamma_{i 0} \gamma_{u 0}
&= \BE^2 + \sum_{i \ne j} E^i E^j ( \sigma_i \sigma_j + \sigma_j \sigma_i ) \\
&= \BE^2 + \sum_{i \ne j} 2 E^i E^j \sigma_i \cdot \sigma_j \\
&= \BE^2
\end{align*}

Hmm.  This is messy.  Let's try with $F = \BE + i c \BB$ directly (with the Doran/Lasenby convention: $\BE = E^k \sigma_k$) :

\begin{align*}
F^2
&= (\BE + i c \BB) (\BE + i c \BB) \\
&= \BE^2 + c^2 (i \BB) (i \BB) + c (i \BB \BE + \BE i \BB) \\
&= \BE^2 + c^2 (\BB i) (i \BB) + i c (\BB \BE + \BE \BB) \\
&= \BE^2 - c^2 \BB^2 + 2 i c (\BB \cdot \BE) \\
\end{align*}

\subsubsection{ Compared to tensor form. }

Now lets compare to the tensor form, where the Lagrangian density is written in terms of the product of upper and lower index tensors:

\begin{align*}
F_{\mu\nu}F^{\mu\nu}
&= F_{i 0}F^{i 0} +F_{0 i}F^{0 i} +\sum_{i<j} F_{i j}F^{i j} +\sum_{j<i} F_{i j}F^{i j} \\
&= 2 F_{i 0}F^{i 0} + 2 \sum_{i<j} F_{i j}F^{i j} \\
&= 2 (-E^i)(E^i) + 2 \sum_{i<j} (F^{i j})^2 \\
&= -2 \BE^2 + 2 \sum_{i<j} ( -\epsilon_{ijk} c B^k )^2 \\
&= -( \BE^2 + c^2 \BB^2 )
\end{align*}

Summarizing with a comparision of the bivector and tensor forms we have:

\begin{equation}
\inv{2} F_{\mu\nu}F^{\mu\nu} = c^2 \BB^2 - \BE^2 = - \gpgradezero{F^2} = \gpgradezero{ F \tilde{F} }
\end{equation}

But to put this in context we need to figure out how to apply this in the Lagrangian.  That appears to require a potential formulation of the field equations, so that is the next step.

\subsubsection{ Potential and relation to electromagnetic tensor. }

Since the field is a bivector is it reasonable to assume that it may be possible to express as the curl of a vector

\begin{equation*}
F = \grad \wedge A.
\end{equation*}

Inserting this into the field equation we have:
\begin{align*}
\grad (\grad \wedge A)
&= \grad \cdot (\grad \wedge A) + \underbrace{\grad \wedge \grad}_{=0} \wedge A \\
&= \grad^2 A - \grad ( \grad \cdot A ) \\
&= \inv{\epsilon_0 c} J
\end{align*}

With application of the guage condition $\grad \cdot A = 0$, one is left with the four scalar equations:

\begin{equation}\label{eqn:potential}
\grad^2 A = \inv{\epsilon_0 c} J
\end{equation}

This can also be seen more directly since the guage condition implies:

\begin{equation*}
\grad \wedge A = \grad \wedge A + \grad \cdot A = \grad A
\end{equation*}

from which equation \ref{eqn:potential} follows directly.  Observe that although the field equation was not metric
dependent, the equivalent potential equation is.

This metric dependency also shows up if one calculates the em tensor in terms of potential.

\begin{align*}
\grad \wedge A
&= \gamma^{\mu} \wedge \underbrace{\gamma_{\nu}}_{\gamma^{\nu} \gamma_{\nu} \gamma_{\nu}} \partial_{\mu} A^{\nu} \\
&= (\gamma_{\nu})^2 \gamma^{\mu\nu} \partial_{\mu} A^{\nu} \\
\end{align*}

Calcualating the tensor in terms of the bivector we have:

\begin{align*}
F^{\mu\nu}
&= F \cdot \gamma_{\nu\mu} \\
&= (\gamma_{\beta})^2 \partial_{\alpha} A^{\beta} \gamma^{\alpha\beta} \cdot \gamma_{\nu\mu} \\
&= (\gamma_{\beta})^2 \partial_{\alpha} A^{\beta} ((\gamma^{\alpha} \wedge \gamma^{\beta}) \cdot \gamma_{\nu}) \cdot \gamma_{\mu} \\
&= (\gamma_{\beta})^2 \partial_{\alpha} A^{\beta} ( \gamma^{\alpha} \delta^{\beta}_{\nu} -\gamma^{\beta} \delta^{\alpha}_{\nu} ) \cdot \gamma_{\mu} \\
&= (\gamma_{\beta})^2 \partial_{\alpha} A^{\beta} ( \delta^{\alpha}_{\mu} \delta^{\beta}_{\nu} -\delta^{\beta}_{\mu} \delta^{\alpha}_{\nu} ) \\
&= \partial_{\mu} A^{\nu} (\gamma_{\nu})^2 - \partial_{\nu} A^{\mu} (\gamma_{\mu})^2 \\
\end{align*}

This also explains the inconsistency in comparision to the wikipedia em tensor page that says this tensor is metric dependent.  There the tensor is defined in terms of a different field quantity, say $B$ directly in terms of the partials of that field.  Writing $B^{\mu} = A^{\mu} (\gamma_{\mu})^2$, that definition is:

\begin{equation*}
F^{\mu\nu} = \partial_{\mu} B^{\nu} - \partial_{\nu} B^{\mu}.
\end{equation*}

In our bivector based definition of the em tensor we have these magic $(\gamma_{\mu})^2$ accompanying each $A^{\mu}$ terms, ``cancelling'' the
metric dependence on the field itself in the resulting tensor field equation.

\subsection{ Lagrangian density in terms of potential. }

\end{document}

\documentclass{article}

\usepackage{amsmath}
\usepackage{mathpazo}

%
% shorthand for bold symbols, convenient for vectors and matrices
%
\newcommand{\Ba}[0]{\mathbf{a}}
\newcommand{\Bb}[0]{\mathbf{b}}
\newcommand{\Bc}[0]{\mathbf{c}}
\newcommand{\Bd}[0]{\mathbf{d}}
\newcommand{\Be}[0]{\mathbf{e}}
\newcommand{\Bf}[0]{\mathbf{f}}
\newcommand{\Bg}[0]{\mathbf{g}}
\newcommand{\Bh}[0]{\mathbf{h}}
\newcommand{\Bi}[0]{\mathbf{i}}
\newcommand{\Bj}[0]{\mathbf{j}}
\newcommand{\Bk}[0]{\mathbf{k}}
\newcommand{\Bl}[0]{\mathbf{l}}
\newcommand{\Bm}[0]{\mathbf{m}}
\newcommand{\Bn}[0]{\mathbf{n}}
\newcommand{\Bo}[0]{\mathbf{o}}
\newcommand{\Bp}[0]{\mathbf{p}}
\newcommand{\Bq}[0]{\mathbf{q}}
\newcommand{\Br}[0]{\mathbf{r}}
\newcommand{\Bs}[0]{\mathbf{s}}
\newcommand{\Bt}[0]{\mathbf{t}}
\newcommand{\Bu}[0]{\mathbf{u}}
\newcommand{\Bv}[0]{\mathbf{v}}
\newcommand{\Bw}[0]{\mathbf{w}}
\newcommand{\Bx}[0]{\mathbf{x}}
\newcommand{\By}[0]{\mathbf{y}}
\newcommand{\Bz}[0]{\mathbf{z}}
\newcommand{\BA}[0]{\mathbf{A}}
\newcommand{\BB}[0]{\mathbf{B}}
\newcommand{\BC}[0]{\mathbf{C}}
\newcommand{\BD}[0]{\mathbf{D}}
\newcommand{\BE}[0]{\mathbf{E}}
\newcommand{\BF}[0]{\mathbf{F}}
\newcommand{\BG}[0]{\mathbf{G}}
\newcommand{\BH}[0]{\mathbf{H}}
\newcommand{\BI}[0]{\mathbf{I}}
\newcommand{\BJ}[0]{\mathbf{J}}
\newcommand{\BK}[0]{\mathbf{K}}
\newcommand{\BL}[0]{\mathbf{L}}
\newcommand{\BM}[0]{\mathbf{M}}
\newcommand{\BN}[0]{\mathbf{N}}
\newcommand{\BO}[0]{\mathbf{O}}
\newcommand{\BP}[0]{\mathbf{P}}
\newcommand{\BQ}[0]{\mathbf{Q}}
\newcommand{\BR}[0]{\mathbf{R}}
\newcommand{\BS}[0]{\mathbf{S}}
\newcommand{\BT}[0]{\mathbf{T}}
\newcommand{\BU}[0]{\mathbf{U}}
\newcommand{\BV}[0]{\mathbf{V}}
\newcommand{\BW}[0]{\mathbf{W}}
\newcommand{\BX}[0]{\mathbf{X}}
\newcommand{\BY}[0]{\mathbf{Y}}
\newcommand{\BZ}[0]{\mathbf{Z}}

\newcommand{\Bzero}[0]{\mathbf{0}}
\newcommand{\Btheta}[0]{\boldsymbol{\theta}}
\newcommand{\Btau}[0]{\boldsymbol{\tau}}
\newcommand{\Bomega}[0]{\boldsymbol{\omega}}

%
% shorthand for unit vectors
%
\newcommand{\acap}[0]{\hat{\Ba}}
\newcommand{\bcap}[0]{\hat{\Bb}}
\newcommand{\ccap}[0]{\hat{\Bc}}
\newcommand{\dcap}[0]{\hat{\Bd}}
\newcommand{\ecap}[0]{\hat{\Be}}
\newcommand{\fcap}[0]{\hat{\Bf}}
\newcommand{\gcap}[0]{\hat{\Bg}}
\newcommand{\hcap}[0]{\hat{\Bh}}
\newcommand{\icap}[0]{\hat{\Bi}}
\newcommand{\jcap}[0]{\hat{\Bj}}
\newcommand{\kcap}[0]{\hat{\Bk}}
\newcommand{\lcap}[0]{\hat{\Bl}}
\newcommand{\mcap}[0]{\hat{\Bm}}
\newcommand{\ncap}[0]{\hat{\Bn}}
\newcommand{\ocap}[0]{\hat{\Bo}}
\newcommand{\pcap}[0]{\hat{\Bp}}
\newcommand{\qcap}[0]{\hat{\Bq}}
\newcommand{\rcap}[0]{\hat{\Br}}
\newcommand{\scap}[0]{\hat{\Bs}}
\newcommand{\tcap}[0]{\hat{\Bt}}
\newcommand{\ucap}[0]{\hat{\Bu}}
\newcommand{\vcap}[0]{\hat{\Bv}}
\newcommand{\wcap}[0]{\hat{\Bw}}
\newcommand{\xcap}[0]{\hat{\Bx}}
\newcommand{\ycap}[0]{\hat{\By}}
\newcommand{\zcap}[0]{\hat{\Bz}}
\newcommand{\thetacap}[0]{\hat{\Btheta}}

%
% to write R^n and C^n in a distinguishable fashion.  Perhaps change this
% to the double lined characters upon figuring out how to do so.
%
\newcommand{\C}[1]{$\mathbb{C}^{#1}$}
\newcommand{\R}[1]{$\mathbb{R}^{#1}$}

%
% various generally useful helpers
%

% derivative of #1 wrt. #2:
\newcommand{\D}[2] {\frac {d#2} {d#1}}

\newcommand{\inv}[1]{\frac{1}{#1}}
\newcommand{\cross}[0]{\times}

\newcommand{\abs}[1]{\lvert{#1}\rvert}
\newcommand{\norm}[1]{\lVert{#1}\rVert}
\newcommand{\innerprod}[2]{\langle{#1}, {#2}\rangle}
\newcommand{\dotprod}[2]{{#1} \cdot {#2}}
\newcommand{\bdotprod}[2]{\left({#1} \cdot {#2}\right)}
\newcommand{\crossprod}[2]{{#1} \cross {#2}}
\newcommand{\tripleprod}[3]{\dotprod{\left(\crossprod{#1}{#2}\right)}{#3}}

\DeclareMathOperator{\Proj}{Proj}
\DeclareMathOperator{\Span}{span}
\DeclareMathOperator{\Sgn}{sgn}
\DeclareMathOperator{\Area}{Area}
\DeclareMathOperator{\Volume}{Volume}

%
% A few miscellaneous things specific to this document
%
\newcommand{\crossop}[1]{\crossprod{#1}{}}

% R2 vector.
\newcommand{\VectorTwo}[2]{
\begin{bmatrix}
 {#1} \\
 {#2}
\end{bmatrix}
}

\newcommand{\VectorN}[1]{
\begin{bmatrix}
{#1}_1 \\
{#1}_2 \\
\vdots \\
{#1}_N \\
\end{bmatrix}
}

\newcommand{\DETuvij}[4]{
\begin{vmatrix}
 {#1}_{#3} & {#1}_{#4} \\
 {#2}_{#3} & {#2}_{#4}
\end{vmatrix}
}

\newcommand{\DETuvwijk}[6]{
\begin{vmatrix}
 {#1}_{#4} & {#1}_{#5} & {#1}_{#6} \\
 {#2}_{#4} & {#2}_{#5} & {#2}_{#6} \\
 {#3}_{#4} & {#3}_{#5} & {#3}_{#6}
\end{vmatrix}
}

\newcommand{\DETuvwxijkl}[8]{
\begin{vmatrix}
 {#1}_{#5} & {#1}_{#6} & {#1}_{#7} & {#1}_{#8} \\
 {#2}_{#5} & {#2}_{#6} & {#2}_{#7} & {#2}_{#8} \\
 {#3}_{#5} & {#3}_{#6} & {#3}_{#7} & {#3}_{#8} \\
 {#4}_{#5} & {#4}_{#6} & {#4}_{#7} & {#4}_{#8} \\
\end{vmatrix}
}

%\newcommand{\DETuvwxyijklm}[10]{
%\begin{vmatrix}
% {#1}_{#6} & {#1}_{#7} & {#1}_{#8} & {#1}_{#9} & {#1}_{#10} \\
% {#2}_{#6} & {#2}_{#7} & {#2}_{#8} & {#2}_{#9} & {#2}_{#10} \\
% {#3}_{#6} & {#3}_{#7} & {#3}_{#8} & {#3}_{#9} & {#3}_{#10} \\
% {#4}_{#6} & {#4}_{#7} & {#4}_{#8} & {#4}_{#9} & {#4}_{#10} \\
% {#5}_{#6} & {#5}_{#7} & {#5}_{#8} & {#5}_{#9} & {#5}_{#10}
%\end{vmatrix}
%}

% R3 vector.
\newcommand{\VectorThree}[3]{
\begin{bmatrix}
 {#1} \\
 {#2} \\
 {#3}
\end{bmatrix}
}


%<misc>
%
\newcommand{\Abs}[1]{{\left\lvert{#1}\right\rvert}}
\newcommand{\spacegrad}[0]{\boldsymbol{\nabla}}
\newcommand{\grad}[0]{\nabla}
\newcommand{\LL}[0]{\mathcal{L}}

% == \partial_{#1} {#2}
\newcommand{\PD}[2]{\frac{\partial {#2}}{\partial {#1}}}
% inline variant
\newcommand{\PDi}[2]{{\partial {#2}}/{\partial {#1}}}

\newcommand{\PDD}[3]{\frac{\partial^2 {#3}}{\partial {#1}\partial {#2}}}
%\newcommand{\PDd}[2]{\frac{\partial^2 {#2}}{{\partial{#1}}^2}}
\newcommand{\PDsq}[2]{\frac{\partial^2 {#2}}{(\partial {#1})^2}}

\newcommand{\Partial}[2]{\frac{\partial {#1}}{\partial {#2}}}
\DeclareMathOperator{\RejName}{Rej}
\newcommand{\Rej}[2]{\RejName_{#1}\left( {#2} \right)}
\newcommand{\Rm}[1]{\mathbb{R}^{#1}}
\newcommand{\Cm}[1]{\mathbb{C}^{#1}}
\newcommand{\conj}[0]{{*}}

%</misc>

% <grade selection>
%
\newcommand{\gpgrade}[2] {{\left\langle{{#1}}\right\rangle}_{#2}}

\newcommand{\gpgradezero}[1] {\gpgrade{#1}{}}
%\newcommand{\gpscalargrade}[1] {{\left\langle{{#1}}\right\rangle}}
%\newcommand{\gpgradezero}[1] {\gpgrade{#1}{0}}

%\newcommand{\gpgradeone}[1] {{\left\langle{{#1}}\right\rangle}_{1}}
\newcommand{\gpgradeone}[1] {\gpgrade{#1}{1}}

\newcommand{\gpgradetwo}[1] {\gpgrade{#1}{2}}
\newcommand{\gpgradethree}[1] {\gpgrade{#1}{3}}
\newcommand{\gpgradefour}[1] {\gpgrade{#1}{4}}
%
% </grade selection>



\newcommand{\adot}[0]{{\dot{a}}}
\newcommand{\bdot}[0]{{\dot{b}}}
% taken for centered dot:
%\newcommand{\cdot}[0]{{\dot{c}}}
%\newcommand{\ddot}[0]{{\dot{d}}}
\newcommand{\edot}[0]{{\dot{e}}}
\newcommand{\fdot}[0]{{\dot{f}}}
\newcommand{\gdot}[0]{{\dot{g}}}
\newcommand{\hdot}[0]{{\dot{h}}}
\newcommand{\idot}[0]{{\dot{i}}}
\newcommand{\jdot}[0]{{\dot{j}}}
\newcommand{\kdot}[0]{{\dot{k}}}
\newcommand{\ldot}[0]{{\dot{l}}}
\newcommand{\mdot}[0]{{\dot{m}}}
\newcommand{\ndot}[0]{{\dot{n}}}
%\newcommand{\odot}[0]{{\dot{o}}}
\newcommand{\pdot}[0]{{\dot{p}}}
\newcommand{\qdot}[0]{{\dot{q}}}
\newcommand{\rdot}[0]{{\dot{r}}}
\newcommand{\sdot}[0]{{\dot{s}}}
\newcommand{\tdot}[0]{{\dot{t}}}
\newcommand{\udot}[0]{{\dot{u}}}
\newcommand{\vdot}[0]{{\dot{v}}}
\newcommand{\wdot}[0]{{\dot{w}}}
\newcommand{\xdot}[0]{{\dot{x}}}
\newcommand{\ydot}[0]{{\dot{y}}}
\newcommand{\zdot}[0]{{\dot{z}}}
\newcommand{\addot}[0]{{\ddot{a}}}
\newcommand{\bddot}[0]{{\ddot{b}}}
\newcommand{\cddot}[0]{{\ddot{c}}}
%\newcommand{\dddot}[0]{{\ddot{d}}}
\newcommand{\eddot}[0]{{\ddot{e}}}
\newcommand{\fddot}[0]{{\ddot{f}}}
\newcommand{\gddot}[0]{{\ddot{g}}}
\newcommand{\hddot}[0]{{\ddot{h}}}
\newcommand{\iddot}[0]{{\ddot{i}}}
\newcommand{\jddot}[0]{{\ddot{j}}}
\newcommand{\kddot}[0]{{\ddot{k}}}
\newcommand{\lddot}[0]{{\ddot{l}}}
\newcommand{\mddot}[0]{{\ddot{m}}}
\newcommand{\nddot}[0]{{\ddot{n}}}
\newcommand{\oddot}[0]{{\ddot{o}}}
\newcommand{\pddot}[0]{{\ddot{p}}}
\newcommand{\qddot}[0]{{\ddot{q}}}
\newcommand{\rddot}[0]{{\ddot{r}}}
\newcommand{\sddot}[0]{{\ddot{s}}}
\newcommand{\tddot}[0]{{\ddot{t}}}
\newcommand{\uddot}[0]{{\ddot{u}}}
\newcommand{\vddot}[0]{{\ddot{v}}}
\newcommand{\wddot}[0]{{\ddot{w}}}
\newcommand{\xddot}[0]{{\ddot{x}}}
\newcommand{\yddot}[0]{{\ddot{y}}}
\newcommand{\zddot}[0]{{\ddot{z}}}

%<bold and dot greek symbols>
%

\newcommand{\Deltadot}[0]{{\dot{\Delta}}}
\newcommand{\Gammadot}[0]{{\dot{\Gamma}}}
\newcommand{\Lambdadot}[0]{{\dot{\Lambda}}}
\newcommand{\Omegadot}[0]{{\dot{\Omega}}}
\newcommand{\Phidot}[0]{{\dot{\Phi}}}
\newcommand{\Pidot}[0]{{\dot{\Pi}}}
\newcommand{\Psidot}[0]{{\dot{\Psi}}}
\newcommand{\Sigmadot}[0]{{\dot{\Sigma}}}
\newcommand{\Thetadot}[0]{{\dot{\Theta}}}
\newcommand{\Upsilondot}[0]{{\dot{\Upsilon}}}
\newcommand{\Xidot}[0]{{\dot{\Xi}}}
\newcommand{\alphadot}[0]{{\dot{\alpha}}}
\newcommand{\betadot}[0]{{\dot{\beta}}}
\newcommand{\chidot}[0]{{\dot{\chi}}}
\newcommand{\deltadot}[0]{{\dot{\delta}}}
\newcommand{\epsilondot}[0]{{\dot{\epsilon}}}
\newcommand{\etadot}[0]{{\dot{\eta}}}
\newcommand{\gammadot}[0]{{\dot{\gamma}}}
\newcommand{\kappadot}[0]{{\dot{\kappa}}}
\newcommand{\lambdadot}[0]{{\dot{\lambda}}}
\newcommand{\mudot}[0]{{\dot{\mu}}}
\newcommand{\nudot}[0]{{\dot{\nu}}}
\newcommand{\omegadot}[0]{{\dot{\omega}}}
\newcommand{\phidot}[0]{{\dot{\phi}}}
\newcommand{\pidot}[0]{{\dot{\pi}}}
\newcommand{\psidot}[0]{{\dot{\psi}}}
\newcommand{\rhodot}[0]{{\dot{\rho}}}
\newcommand{\sigmadot}[0]{{\dot{\sigma}}}
\newcommand{\taudot}[0]{{\dot{\tau}}}
\newcommand{\thetadot}[0]{{\dot{\theta}}}
\newcommand{\upsilondot}[0]{{\dot{\upsilon}}}
\newcommand{\varepsilondot}[0]{{\dot{\varepsilon}}}
\newcommand{\varphidot}[0]{{\dot{\varphi}}}
\newcommand{\varpidot}[0]{{\dot{\varpi}}}
\newcommand{\varrhodot}[0]{{\dot{\varrho}}}
\newcommand{\varsigmadot}[0]{{\dot{\varsigma}}}
\newcommand{\varthetadot}[0]{{\dot{\vartheta}}}
\newcommand{\xidot}[0]{{\dot{\xi}}}
\newcommand{\zetadot}[0]{{\dot{\zeta}}}

\newcommand{\Deltaddot}[0]{{\ddot{\Delta}}}
\newcommand{\Gammaddot}[0]{{\ddot{\Gamma}}}
\newcommand{\Lambdaddot}[0]{{\ddot{\Lambda}}}
\newcommand{\Omegaddot}[0]{{\ddot{\Omega}}}
\newcommand{\Phiddot}[0]{{\ddot{\Phi}}}
\newcommand{\Piddot}[0]{{\ddot{\Pi}}}
\newcommand{\Psiddot}[0]{{\ddot{\Psi}}}
\newcommand{\Sigmaddot}[0]{{\ddot{\Sigma}}}
\newcommand{\Thetaddot}[0]{{\ddot{\Theta}}}
\newcommand{\Upsilonddot}[0]{{\ddot{\Upsilon}}}
\newcommand{\Xiddot}[0]{{\ddot{\Xi}}}
\newcommand{\alphaddot}[0]{{\ddot{\alpha}}}
\newcommand{\betaddot}[0]{{\ddot{\beta}}}
\newcommand{\chiddot}[0]{{\ddot{\chi}}}
\newcommand{\deltaddot}[0]{{\ddot{\delta}}}
\newcommand{\epsilonddot}[0]{{\ddot{\epsilon}}}
\newcommand{\etaddot}[0]{{\ddot{\eta}}}
\newcommand{\gammaddot}[0]{{\ddot{\gamma}}}
\newcommand{\kappaddot}[0]{{\ddot{\kappa}}}
\newcommand{\lambdaddot}[0]{{\ddot{\lambda}}}
\newcommand{\muddot}[0]{{\ddot{\mu}}}
\newcommand{\nuddot}[0]{{\ddot{\nu}}}
\newcommand{\omegaddot}[0]{{\ddot{\omega}}}
\newcommand{\phiddot}[0]{{\ddot{\phi}}}
\newcommand{\piddot}[0]{{\ddot{\pi}}}
\newcommand{\psiddot}[0]{{\ddot{\psi}}}
\newcommand{\rhoddot}[0]{{\ddot{\rho}}}
\newcommand{\sigmaddot}[0]{{\ddot{\sigma}}}
\newcommand{\tauddot}[0]{{\ddot{\tau}}}
\newcommand{\thetaddot}[0]{{\ddot{\theta}}}
\newcommand{\upsilonddot}[0]{{\ddot{\upsilon}}}
\newcommand{\varepsilonddot}[0]{{\ddot{\varepsilon}}}
\newcommand{\varphiddot}[0]{{\ddot{\varphi}}}
\newcommand{\varpiddot}[0]{{\ddot{\varpi}}}
\newcommand{\varrhoddot}[0]{{\ddot{\varrho}}}
\newcommand{\varsigmaddot}[0]{{\ddot{\varsigma}}}
\newcommand{\varthetaddot}[0]{{\ddot{\vartheta}}}
\newcommand{\xiddot}[0]{{\ddot{\xi}}}
\newcommand{\zetaddot}[0]{{\ddot{\zeta}}}

\newcommand{\BDelta}[0]{\boldsymbol{\Delta}}
\newcommand{\BGamma}[0]{\boldsymbol{\Gamma}}
\newcommand{\BLambda}[0]{\boldsymbol{\Lambda}}
\newcommand{\BOmega}[0]{\boldsymbol{\Omega}}
\newcommand{\BPhi}[0]{\boldsymbol{\Phi}}
\newcommand{\BPi}[0]{\boldsymbol{\Pi}}
\newcommand{\BPsi}[0]{\boldsymbol{\Psi}}
\newcommand{\BSigma}[0]{\boldsymbol{\Sigma}}
\newcommand{\BTheta}[0]{\boldsymbol{\Theta}}
\newcommand{\BUpsilon}[0]{\boldsymbol{\Upsilon}}
\newcommand{\BXi}[0]{\boldsymbol{\Xi}}
\newcommand{\Balpha}[0]{\boldsymbol{\alpha}}
\newcommand{\Bbeta}[0]{\boldsymbol{\beta}}
\newcommand{\Bchi}[0]{\boldsymbol{\chi}}
\newcommand{\Bdelta}[0]{\boldsymbol{\delta}}
\newcommand{\Bepsilon}[0]{\boldsymbol{\epsilon}}
\newcommand{\Beta}[0]{\boldsymbol{\eta}}
\newcommand{\Bgamma}[0]{\boldsymbol{\gamma}}
\newcommand{\Bkappa}[0]{\boldsymbol{\kappa}}
\newcommand{\Blambda}[0]{\boldsymbol{\lambda}}
\newcommand{\Bmu}[0]{\boldsymbol{\mu}}
\newcommand{\Bnu}[0]{\boldsymbol{\nu}}
%\newcommand{\Bomega}[0]{\boldsymbol{\omega}}
\newcommand{\Bphi}[0]{\boldsymbol{\phi}}
\newcommand{\Bpi}[0]{\boldsymbol{\pi}}
\newcommand{\Bpsi}[0]{\boldsymbol{\psi}}
\newcommand{\Brho}[0]{\boldsymbol{\rho}}
\newcommand{\Bsigma}[0]{\boldsymbol{\sigma}}
%\newcommand{\Btau}[0]{\boldsymbol{\tau}}
%\newcommand{\Btheta}[0]{\boldsymbol{\theta}}
\newcommand{\Bupsilon}[0]{\boldsymbol{\upsilon}}
\newcommand{\Bvarepsilon}[0]{\boldsymbol{\varepsilon}}
\newcommand{\Bvarphi}[0]{\boldsymbol{\varphi}}
\newcommand{\Bvarpi}[0]{\boldsymbol{\varpi}}
\newcommand{\Bvarrho}[0]{\boldsymbol{\varrho}}
\newcommand{\Bvarsigma}[0]{\boldsymbol{\varsigma}}
\newcommand{\Bvartheta}[0]{\boldsymbol{\vartheta}}
\newcommand{\Bxi}[0]{\boldsymbol{\xi}}
\newcommand{\Bzeta}[0]{\boldsymbol{\zeta}}
%
%</bold and dot greek symbols>
%<infrequent>
%
%\newcommand{\AreaOp}[1]{\AName_{#1}}
%\newcommand{\Babs}[0]{\abs{\BB}}
%\newcommand{\Bcap}[0]{\hat{\BB}}
%\newcommand{\BrPrimeRej}[0]{\rcap(\rcap \wedge \Br')}
%\newcommand{\CA}[0]{\mathcal{A}}
%\newcommand{\Cos}[1]{\cos{\left({#1}\right)}}
%\newcommand{\Det}[1] {\abs{#1}}
%\newcommand{\Dsq}[2] {\frac {\partial^2 {#1}} {\partial {#2}^2}}
%\newcommand{\Exp}[1]{\exp{\left({#1}\right)}}
%\newcommand{\Norm}[1]{\left\lVert{#1}\right\rVert}
%\newcommand{\Sin}[1]{\sin{\left({#1}\right)}}
%\newcommand{\T}[0]{\text{T}}
%\newcommand{\VolumeOp}[1]{\VName_{#1}}
%\newcommand{\agrad}[0]{\Ba \cdot \nabla}
%\newcommand{\alphacap}[0]{\hat{\boldsymbol{\alpha}}}
%\newcommand{\Fcap}[0]{\hat{\BF}}
%\newcommand{\bithree}[0]{{\Bi}_3}
%\newcommand{\bxa}[0]{\Bx\Ba}
%\newcommand{\coordvec}[2]{
%\newcommand{\costheta}[0]{\acap \cdot \xcap}
%\newcommand{\ddt}[1]{\ddot{#1}}
%\newcommand{\ddu}[1] {\frac {d{#1}} {du}}
%\newcommand{\dsqxj}[2] {\frac {\partial^2 {#1}} {\partial {x_{#2}}^2}}
%\newcommand{\dtheta}[1]{\frac{d {#1}}{d \theta}}
%\newcommand{\dt}[1]{\dot{#1}}
%\newcommand{\dt}[1]{\frac{d {#1}}{dt}}
%\newcommand{\dxj}[2] {\frac {\partial {#1}} {\partial {x_{#2}}}}
%\newcommand{\halfPhi}[0]{\frac{\phi}{2}}
%\newcommand{\half}[0]{\inv{2}}
%\newcommand{\inv}[1]{\frac{1}{#1}}
%\newcommand{\laplacian}[0]{\nabla^2}
%\newcommand{\matrixoftx}[3]{
%\newcommand{\nrrp}[0]{\norm{\rcap \wedge \Br'}}
%\newcommand{\oiint}{\bigcirc \hspace{-1.4em} \int \hspace{-.8em} \int}
%\newcommand{\transpose}[1]{{#1}^{\text{T}}}
%\newcommand{\transpose}[1]{{{#1}^{\TextTranspose}}}
%\newcommand{\transpose}[1]{{{#1}^{\text{T}}}}
%\newcommand{\barA}[0]{\bar{A}}
%\newcommand{\qbar}[0]{\bar{q}}
%\newcommand{\qdotbar}[0]{\dot{\bar{q}}}
%
%</infrequent>




\newcommand{\symmetric}[2]{{\left\{{#1},{#2}\right\}}}
\newcommand{\antisymmetric}[2]{\left[{#1},{#2}\right]}
\DeclareMathOperator{\sgn}{sgn}
\DeclareMathOperator{\something}{something}

\newcommand{\uDETuvij}[4]{
\begin{vmatrix}
 {#1}^{#3} & {#1}^{#4} \\
 {#2}^{#3} & {#2}^{#4}
\end{vmatrix}
}

\newcommand{\PDSq}[2]{\frac{\partial^2 {#2}}{\partial {#1}^2}}
\newcommand{\transpose}[1]{{#1}^{\mathrm{T}}}
\newcommand{\stardot}[0]{{*}}

% bivector.tex:
\newcommand{\laplacian}[0]{\nabla^2}
\newcommand{\Dsq}[2] {\frac {\partial^2 {#1}} {\partial {#2}^2}}
\newcommand{\dxj}[2] {\frac {\partial {#1}} {\partial {x_{#2}}}}
\newcommand{\dsqxj}[2] {\frac {\partial^2 {#1}} {\partial {x_{#2}}^2}}
\DeclareMathOperator{\ExpName}{e}
%\DeclareMathOperator{\Exp}{e}
%\newcommand{\Exp}[1]{\exp{\left({#1}\right)}}
%\DeclareMathOperator{\Rej}{Rej}
\DeclareMathOperator{\Rot}{R}
%\newcommand{\gpgrade}[2] {{\left\langle{{#1}}\right\rangle}_{#2}}
%\newcommand{\gpgradezero}[1] {\gpgrade{#1}{0}}
%\newcommand{\gpgradetwo}[1] {\gpgrade{#1}{2}}
%\newcommand{\gpgradefour}[1] {\gpgrade{#1}{4}}

% ga_wiki_torque.tex:
\newcommand{\Fcap}[0]{\hat{\BF}}
\newcommand{\bithree}[0]{{\Bi}_3}
\newcommand{\nrrp}[0]{\norm{\rcap \wedge \Br'}}
\newcommand{\dtheta}[1]{\frac{d {#1}}{d \theta}}

% ga_wiki_unit_derivative.tex
\newcommand{\dt}[1]{\frac{d {#1}}{dt}}
\newcommand{\BrPrimeRej}[0]{\rcap(\rcap \wedge \Br')}

% radial_vector_derivatives.tex:
%\newcommand{\BrPrimeRej}[0]{\rcap(\rcap \wedge \Br')}

% angular_velocity.tex

%\newcommand{\dt}[1]{\frac{d {#1}}{dt}}
%\newcommand{\Norm}[1]{\left\lVert{#1}\right\rVert}
%\newcommand{\dtheta}[1]{\frac{d {#1}}{d \theta}}

% reciprocal_frame.tex
\DeclareMathOperator{\AbsName}{abs}

%\DeclareMathOperator{\RejName}{Rej}
%\newcommand{\Rej}[2]{\RejName_{#1}\left( {#2} \right)}

\DeclareMathOperator{\AName}{A}
\newcommand{\AreaOp}[1]{\AName_{#1}}

\DeclareMathOperator{\VName}{V}
\newcommand{\VolumeOp}[1]{\VName_{#1}}

%\newcommand{\gpgrade}[2] {{\left\langle{{#1}}\right\rangle}_{#2}}
%\newcommand{\gpgradeone}[1] {{\left\langle{{#1}}\right\rangle}_{1}}


% projection_with_matrix_comparison.tex
%\DeclareMathOperator{\Transpose}{T}
\DeclareMathOperator{\rank}{rank}
%\newcommand{\transpose}[1]{{{#1}^{\TextTranspose}}}
%\newcommand{\transpose}[1]{{{#1}^{\text{T}}}}
\newcommand{\T}[0]{{\text{T}}}
%\newcommand{\BOmega}[0]{\boldsymbol{\Omega}}

%\newcommand{\Det}[1] {\abs{#1}}

% oblique_proj.tex
%\newcommand{\T}[0]{\text{T}}
%\newcommand{\Bbeta}[0]{\boldsymbol{\beta}}

% spherical_polar.tex
\newcommand{\phicap}[0]{\hat{\boldsymbol{\phi}}}
\newcommand{\Lor}[2]{{{\Lambda^{#1}}_{#2}}}
\newcommand{\ILor}[2]{{{ \{{\Lambda^{-1}\} }^{#1}}_{#2}}}

% slerp.tex
\DeclareMathOperator{\atan2}{atan2}

% kvector_exponential.tex
%\DeclareMathOperator{\Exp}{e}
%\DeclareMathOperator{\Rej}{Rej}
\newcommand{\Bcap}[0]{\hat{\BB}}
\newcommand{\Babs}[0]{\abs{\BB}}
%\newcommand{\gpgrade}[2] {{\left\langle{{#1}}\right\rangle}_{#2}}
%\newcommand{\gpgradezero}[1] {\gpgrade{#1}{0}}
%\newcommand{\gpgradetwo}[1] {\gpgrade{#1}{2}}
%\newcommand{\gpgradefour}[1] {\gpgrade{#1}{4}}

\newcommand{\ddu}[1] {\frac {d{#1}} {du}}

% vector_integral_relations.tex
%\newcommand{\Oiint}{\bigcirc \hspace{-1.4em} \int \hspace{-.8em} \int}

% legendre.tex
\newcommand{\agrad}[0]{\Ba \cdot \nabla}
\newcommand{\bxa}[0]{\Bx\Ba}
\newcommand{\costheta}[0]{\acap \cdot \xcap}
%\newcommand{\inv}[1]{\frac{1}{#1}}
\newcommand{\half}[0]{\inv{2}}

% ke_rotation.tex
\newcommand{\DotT}[1]{\dot{#1}}
\newcommand{\DDotT}[1]{\ddot{#1}}
%\newcommand{\transpose}[1]{{#1}^{\text{T}}}
%\newcommand{\Balpha}[0]{\boldsymbol{\alpha}}

%\newcommand{\gpgrade}[2] {{\left\langle{{#1}}\right\rangle}_{#2}}
%\newcommand{\gpgradeone}[1] {{\left\langle{{#1}}\right\rangle}_{1}}
\newcommand{\gpscalargrade}[1] {{\left\langle{{#1}}\right\rangle}}
%\newcommand{\BOmega}[0]{\boldsymbol{\Omega}}

% gaussian_surface.tex
%\newcommand{\phicap}[0]{\hat{\Bphi}}

% newtonian_lagrangian_and_gradient.tex
% PD macro that is backwards from current in macros2:
\newcommand{\PDb}[2]{ \frac{\partial{#1}}{\partial {#2}} }

% inertial_tensor.tex
\newcommand{\matrixoftx}[3]{
{
\begin{bmatrix}
{#1}
\end{bmatrix}
}_{#2}^{#3}
}

\newcommand{\coordvec}[2]{
{
\begin{bmatrix}
{#1}
\end{bmatrix}
}_{#2}
}

% bohr.tex
\newcommand{\K}[0]{\inv{4 \pi \epsilon_0}}

% euler_lagrange.tex
\newcommand{\qbar}[0]{\bar{q}}
\newcommand{\qdotbar}[0]{\dot{\bar{q}}}
\newcommand{\DD}[2]{\frac{d{#2}}{d{#1}}}
\newcommand{\Xdot}[0]{\dot{X}}

% rayleigh_jeans.tex
\newcommand{\EE}[0]{\boldsymbol{\mathcal{E}}}
\newcommand{\HH}[0]{\boldsymbol{\mathcal{H}}}

% 4d_fourier.tex

%\newcommand{\PDSq}[2]{\frac{\partial^2 {#2}}{\partial {#1}^2}}
\DeclareMathOperator{\sinc}{sinc}
\DeclareMathOperator{\PV}{PV}
\newcommand{\FF}[0]{\mathcal{F}}
\newcommand{\IIinf}[0]{ \int_{-\infty}^\infty }

% poisson.tex
%\newcommand{\PDSq}[2]{\frac{\partial^2 {#2}}{\partial {#1}^2}}
%\DeclareMathOperator{\sinc}{sinc}
%\DeclareMathOperator{\PV}{PV}
%\newcommand{\FF}[0]{\mathcal{F}}
%\newcommand{\IIinf}[0]{ \int_{-\infty}^\infty }

% fourier_maxwell.tex
%\newcommand{\PDSq}[2]{\frac{\partial^2 {#2}}{\partial {#1}^2}}
%\DeclareMathOperator{\sinc}{sinc}
%\DeclareMathOperator{\sgn}{sgn}
%\DeclareMathOperator{\PV}{PV}
%\newcommand{\FF}[0]{\mathcal{F}}
%\newcommand{\IIinf}[0]{ \int_{-\infty}^\infty }

% firstorder_fourier_maxwell.tex
%\newcommand{\PDSq}[2]{\frac{\partial^2 {#2}}{\partial {#1}^2}}
%\DeclareMathOperator{\sinc}{sinc}
%\DeclareMathOperator{\PV}{PV}
%\newcommand{\FF}[0]{\mathcal{F}}
%\newcommand{\IIinf}[0]{ \int_{-\infty}^\infty }

% wave_fourier.tex
%\newcommand{\PDSq}[2]{\frac{\partial^2 {#2}}{\partial {#1}^2}}
%\DeclareMathOperator{\sinc}{sinc}
%\DeclareMathOperator{\PV}{PV}
%\newcommand{\FF}[0]{\mathcal{F}}
%\newcommand{\IIinf}[0]{ \int_{-\infty}^\infty }

% heat_fourier.tex
%\newcommand{\PDSq}[2]{\frac{\partial^2 {#2}}{\partial {#1}^2}}
%\DeclareMathOperator{\sinc}{sinc}
%\newcommand{\FF}[0]{\mathcal{F}}
%\newcommand{\IIinf}[0]{ \int_{-\infty}^\infty }

% proj_generalized_dot_prod.tex
%\newcommand{\T}[0]{\text{T}}

% fourier_tx.tex
%\newcommand{\FF}[0]{\mathcal{F}}
\newcommand{\FM}[0]{\inv{\sqrt{2\pi\hbar}}}
\newcommand{\Iinf}[1]{ \int_{-\infty}^\infty {#1}}
%\DeclareMathOperator{\PV}{PV}

% fourier_notation.tex
%\newcommand{\FF}[0]{\mathcal{F}}
%\newcommand{\IIinf}[0]{ \int_{-\infty}^\infty }
%\DeclareMathOperator{\PV}{PV}
%\DeclareMathOperator{\sinc}{sinc}

% planewave.tex
%\newcommand{\EE}[0]{\boldsymbol{\mathcal{E}}}
%\newcommand{\HH}[0]{\boldsymbol{\mathcal{H}}}
%\newcommand{\IIinf}[0]{ \int_{-\infty}^\infty }

% dirac_lagrangian.tex
\newcommand{\Dslash}[0]{ \not\!D }

% pauli_matrix.tex
\newcommand{\Clifford}[2]{\mathcal{C}_{\{{#1},{#2}\}}}
\DeclareMathOperator{\tr}{Tr}
%\DeclareMathOperator{\Scalar}{Scalar}
\DeclareMathOperator{\Real}{Re}
\DeclareMathOperator{\Imag}{Im}
\newcommand{\trace}[1]{\tr{#1}}
\newcommand{\scalarProduct}[2]{{#1} \bullet {#2}}
\newcommand{\traceB}[1]{\tr\left({#1}\right)}
%\newcommand{\symmetric}[2]{{\left\{{#1},{#2}\right\}}}
%\newcommand{\antisymmetric}[2]{\left[{#1},{#2}\right]}
%\newcommand{\Bcap}[0]{\hat{\BB}}

\newcommand{\xhat}[0]{\hat{x}}

\newcommand{\PauliI}[0]{
\begin{bmatrix}
1 & 0 \\
0 & 1 \\
\end{bmatrix}
}

\newcommand{\PauliX}[0]{
\begin{bmatrix}
0 & 1 \\
1 & 0 \\
\end{bmatrix}
}

\newcommand{\PauliY}[0]{
\begin{bmatrix}
0 & -i \\
i & 0 \\
\end{bmatrix}
}

\newcommand{\PauliYNoI}[0]{
\begin{bmatrix}
0 & -1 \\
1 & 0 \\
\end{bmatrix}
}

\newcommand{\PauliZ}[0]{
\begin{bmatrix}
1 & 0 \\
0 & -1 \\
\end{bmatrix}
}

% gamma.tex
%\newcommand{\scalarProduct}[2]{{#1} \bullet {#2}}
%\newcommand{\symmetric}[2]{{\left\{{#1},{#2}\right\}}}
%\newcommand{\antisymmetric}[2]{\left[{#1},{#2}\right]}

%\newcommand{\PauliX}[0]{
%\begin{bmatrix}
%0 & 1 \\
%1 & 0 \\
%\end{bmatrix}
%}

%\newcommand{\PauliY}[0]{
%\begin{bmatrix}
%0 & -i \\
%i & 0 \\
%\end{bmatrix}
%}

%\newcommand{\PauliYNoI}[0]{
%\begin{bmatrix}
%0 & -1 \\
%1 & 0 \\
%\end{bmatrix}
%}

%\newcommand{\PauliZ}[0]{
%\begin{bmatrix}
%1 & 0 \\
%0 & -1 \\
%\end{bmatrix}
%}

% em_bivector_metric_dependencies.tex

%\newcommand{\LL}[0]{\mathcal{L}}
%\newcommand{\gpgrade}[2] {{\left\langle{{#1}}\right\rangle}_{#2}}
%\newcommand{\gpgradezero}[1] {\gpgrade{#1}{0}}
%\newcommand{\gpgradetwo}[1] {\gpgrade{#1}{2}}
%\newcommand{\gpgradeone}[1] {\gpgrade{#1}{1}}
%\newcommand{\gpgradefour}[1] {\gpgrade{#1}{4}}
%\newcommand{\grad}[0]{\nabla}
%\newcommand{\spacegrad}[0]{\boldsymbol{\nabla}}
% == \partial_{#1} {#2}
%\newcommand{\PD}[2]{\frac{\partial {#2}}{\partial {#1}}}
%\newcommand{\PDD}[3]{\frac{\partial^2 {#3}}{\partial {#1}\partial {#2}}}
\newcommand{\PDsQ}[2]{\frac{\partial^2 {#2}}{\partial^2 {#1}}}

% gem.tex
\newcommand{\barh}[0]{\bar{h}}

% mass_vary_lagrangian.tex
%\newcommand{\LL}[0]{\mathcal{L}}
%\newcommand{\grad}[0]{\nabla}
%\newcommand{\PD}[2]{\frac{\partial {#2}}{\partial {#1}}}
%\newcommand{\xdot}[0]{\dot{x}}
%\newcommand{\vdot}[0]{\dot{v}}
%\newcommand{\mdot}[0]{\dot{m}}
%\newcommand{\xddot}[0]{\ddot{x}}
%\newcommand{\spacegrad}[0]{\boldsymbol{\nabla}}

% fourvec_dotinvariance.tex
%\newcommand{\Balpha}[0]{\boldsymbol{\alpha}}
\newcommand{\alphacap}[0]{\hat{\boldsymbol{\alpha}}}
%\newcommand{\Bcap}[0]{\hat{\BB}}
%\newcommand{\gpgrade}[2] {{\left\langle{{#1}}\right\rangle}_{#2}}
%\newcommand{\gpgradezero}[1] {\gpgrade{#1}{0}}

% lorentz.tex
%\newcommand{\laplacian}[0]{\nabla^2}

% field_lagrangian.tex
%\newcommand{\LL}[0]{\mathcal{L}}
%\newcommand{\PD}[2]{\frac{\partial {#2}}{\partial {#1}}}
\newcommand{\barA}[0]{\bar{A}}
%\newcommand{\grad}[0]{\nabla}
%\newcommand{\conj}[0]{{*}}

%\newcommand{\spacegrad}[0]{\boldsymbol{\nabla}}

%\newcommand{\gpgrade}[2] {{\left\langle{{#1}}\right\rangle}_{#2}}
%\newcommand{\gpgradezero}[1] {\gpgrade{#1}{0}}
%\newcommand{\gpgradetwo}[1] {\gpgrade{#1}{2}}
%\newcommand{\gpgradefour}[1] {\gpgrade{#1}{4}}

% lagrangian_field_density.tex
%\newcommand{\LL}[0]{\mathcal{L}}
%\newcommand{\gpgrade}[2] {{\left\langle{{#1}}\right\rangle}_{#2}}
%\newcommand{\gpgradezero}[1] {\gpgrade{#1}{0}}
%\newcommand{\gpgradetwo}[1] {\gpgrade{#1}{2}}
%\newcommand{\gpgradefour}[1] {\gpgrade{#1}{4}}
%\newcommand{\grad}[0]{\nabla}
%\newcommand{\spacegrad}[0]{\boldsymbol{\nabla}}
%\newcommand{\PD}[2]{\frac{\partial {#2}}{\partial {#1}}}
\newcommand{\PDd}[2]{\frac{\partial^2 {#2}}{{\partial{#1}}^2}}
%\newcommand{\PDD}[3]{\frac{\partial^2 {#3}}{\partial {#1}\partial {#2}}}

%\newcommand{\barA}[0]{\bar{A}}

% lorentz_force.tex
%\newcommand{\grad}[0]{\nabla}
%\newcommand{\spacegrad}[0]{\boldsymbol{\nabla}}
%\newcommand{\LL}[0]{\mathcal{L}}
%\newcommand{\xdot}[0]{\dot{x}}
%\newcommand{\xddot}[0]{\ddot{x}}
%\newcommand{\pdot}[0]{\dot{p}}
%\newcommand{\pddot}[0]{\ddot{p}}
%\newcommand{\fdot}[0]{\dot{f}}
%\newcommand{\fddot}[0]{\ddot{f}}

%\newcommand{\gpgrade}[2] {{\left\langle{{#1}}\right\rangle}_{#2}}
%\newcommand{\gpgradeone}[1] {\gpgrade{#1}{1}}
%\newcommand{\gpgradezero}[1] {\gpgrade{#1}{}}
%\newcommand{\grad}[0] {\nabla}
%\newcommand{\spacegrad}[0]{\boldsymbol{\nabla}}

%\newcommand{\pdot}[0]{\dot{p}}
%\newcommand{\pddot}[0]{\ddot{p}}

%\newcommand{\xdot}[0]{\dot{x}}
%\newcommand{\xddot}[0]{\ddot{x}}
%\newcommand{\PD}[2]{\frac{\partial {#2}}{\partial {#1}}}

% stokes_maxwell_application.tex
%\newcommand{\grad}[0]{\nabla}
%\newcommand{\PD}[2]{\frac{\partial {#2}}{\partial {#1}}}
%\newcommand{\spacegrad}[0]{\boldsymbol{\nabla}}
%\newcommand{\gpgrade}[2] {{\left\langle{{#1}}\right\rangle}_{#2}}
%\newcommand{\gpgradezero}[1] {\gpgrade{#1}{0}}
%\newcommand{\gpgradeone}[1] {\gpgrade{#1}{1}}
%\newcommand{\gpgradetwo}[1] {\gpgrade{#1}{2}}
%\newcommand{\gpgradethree}[1] {\gpgrade{#1}{3}}

% lorentz_rotation.tex
%\DeclareMathOperator{\Transpose}{T}
%\newcommand{\T}[0]{\text{T}}

% electron_rotor.tex
\newcommand{\reverse}[1]{\tilde{{#1}}}
%\newcommand{\ILambda}[0]{{(\Lambda^{-1})}}
\newcommand{\ILambda}[0]{\Pi}

% em_potential.tex
%\newcommand{\spacegrad}[0]{\boldsymbol{\nabla}}
%\newcommand{\grad}[0]{\nabla}
\newcommand{\CA}[0]{\mathcal{A}}
 
% maxwell_to_tensor.tex
%\newcommand{\LL}[0]{\mathcal{L}}
%\newcommand{\gpgrade}[2] {{\left\langle{{#1}}\right\rangle}_{#2}}
%\newcommand{\gpgradezero}[1] {\gpgrade{#1}{0}}
%\newcommand{\gpgradetwo}[1] {\gpgrade{#1}{2}}
%\newcommand{\gpgradeone}[1] {\gpgrade{#1}{1}}
%\newcommand{\gpgradefour}[1] {\gpgrade{#1}{4}}
%\newcommand{\grad}[0]{\nabla}
%\newcommand{\spacegrad}[0]{\boldsymbol{\nabla}}
% == \partial_{#1} {#2}
%\newcommand{\PD}[2]{\frac{\partial {#2}}{\partial {#1}}}
%\newcommand{\PDD}[3]{\frac{\partial^2 {#3}}{\partial {#1}\partial {#2}}}
%\newcommand{\PDsQ}[2]{\frac{\partial^2 {#2}}{\partial^2 {#1}}}

%\newcommand{\EE}[0]{\boldsymbol{\mathcal{E}}}
%\newcommand{\HH}[0]{\boldsymbol{\mathcal{H}}}
\newcommand{\Vcap}[0]{\hat{\BV}}



%\usepackage{listings}
%\usepackage{txfonts} % for ointctr... (also appears to make "prettier" \int and \sum's)
% makes \grad look funny though (almost like spacegrad, but narrower)
\usepackage[bookmarks=true]{hyperref}

\usepackage{color,cite,graphicx}
   % use colour in the document, put your citations as [1-4]
   % rather than [1,2,3,4] (it looks nicer, and the extended LaTeX2e
   % graphics package. 
\usepackage{latexsym,amssymb,epsf} % don't remember if these are
   % needed, but their inclusion can't do any damage


\title{Wave equation form of Maxwell's equations}
\author{Peeter Joot \quad peeter.joot@gmail.com }
\date{ June 21, 2009.  $RCSfile: emVacWave.tex,v $ Last $Revision: 1.1 $ $Date: 2009/06/21 23:12:53 $ }

\begin{document}

\maketitle{}
\tableofcontents

\section{Motivation.}

In \cite{jackson1975cew}, on plane waves, he writes "we find easily..." to show that the wave equation for each of the components
of $\BE$, and $\BB$ in the abscence of current and charge satisfy the wave equation.  Do this calculation.

\section{Vacuum case.}

Avoiding the non-vacuum medium temporarily, Maxwell's vacuum equations (in SI units) are

\begin{align}\label{eqn:divE}
\spacegrad \cdot \BE = 0
\end{align}
\begin{align}\label{eqn:divB}
\spacegrad \cdot \BB = 0
\end{align}
\begin{align}\label{eqn:curlB}
\spacegrad \cross \BB = \inv{c^2} \frac{\partial \BE}{\partial t}
\end{align}
\begin{align}\label{eqn:curlE}
\spacegrad \cross \BE = -\frac{\partial \BB}{\partial t}
\end{align}

The last two curl equations can be decoupled by once more calculating the curl.
Illustrating by example

\begin{align}\label{eqn:curlCurlE}
\spacegrad \cross (\spacegrad \cross \BE) = -\frac{\partial }{\partial t} \spacegrad \cross \BB = -\inv{c^2} \frac{\partial^2 \BE}{\partial t^2}
\end{align}

Digging out vector identities and utilizing the zero divergence we have

\begin{align}\label{eqn:identForcurlCurlE}
\spacegrad \cross (\spacegrad \cross \BE) = \spacegrad (\spacegrad \cdot \BE) - \spacegrad^2 \BE = -\spacegrad^2 \BE
\end{align}

Putting \ref{eqn:curlCurlE}, and \ref{eqn:identForcurlCurlE} together provides a wave equation for the electric field vector

\begin{align}\label{eqn:waveE}
\inv{c^2} \frac{\partial^2 \BE}{\partial t^2} - \spacegrad^2 \BE = 0
\end{align}

Operating with curl on the remainding Maxwell equation similarily produces a wave equation for the magnetic field vector

\begin{align}\label{eqn:waveB}
\inv{c^2} \frac{\partial^2 \BB}{\partial t^2} - \spacegrad^2 \BB = 0
\end{align}

This is really six wave equations, one for each of the field coordinates.

\section{With Geometric Algebra.}

Arriving at (\ref{eqn:waveE}), and (\ref{eqn:waveB}) is much easier using the GA formalism of (\cite{doran2003gap}).

Pre or post multiplication of the gradient with the observer frame time basis unit vector $\gamma_0$ has a conjugate like
action

\begin{align*}
\grad \gamma_0
&=
\gamma^0 \gamma_0 \partial_0 + \gamma^k \gamma_0 \partial_k \\
&=
\partial_0 - \spacegrad \\
\end{align*}

(where as usual our spatial basis is $\sigma_k = \gamma_k \gamma_0$).

Similarily
\begin{align*}
\gamma_0 \grad 
&=
\partial_0 + \spacegrad \\
\end{align*}

For the vacuum Maxwell's equation is just
\begin{align*}
\grad F = \grad (\BE + I c \BB) = 0
\end{align*}

With nothing more than an algebraic operation we have

\begin{align*}
0 
&= \grad \gamma_0 \gamma_0 \grad F \\
&=
( \partial_0 - \spacegrad ) ( \partial_0 + \spacegrad ) (\BE + I c \BB) \\
&=
\left( \inv{c^2} \frac{\partial^2}{\partial t^2} - \spacegrad^2 \right) (\BE + I c \BB) \\
\end{align*}

This equality is true independently for each of the components of $\BE$ and $\BB$, so we have as before

These wave equations are still subject to the constraints of the original Maxwell equations.  

\begin{align*}
0 &= \gamma_0 \grad F \\
&= (\partial_0 + \spacegrad) (\BE + I c \BB) \\
&= 
  \spacegrad \cdot \BE 
+ (\partial_0 \BE - c \spacegrad \cross \BB)
+ I ( c \partial_0 \BB + \spacegrad \cross \BE )
+ I c \spacegrad \cdot \BB 
\\
\end{align*}

\section{Tensor approach?}

In both the traditional vector and the GA form one can derive the wave equation relations 
of (\ref{eqn:waveE}), \ref{eqn:waveB}).  One can obviously summarize these in tensor form as

\begin{align}\label{eqn:waveFaraday}
\partial_\mu\partial^\mu F^{\alpha\beta} = 0
\end{align}

working backwards from the vector or GA result.  In this notation, the coupling constraint would be that the field variables
$F^{\alpha\beta}$ are subject to the Maxwell divergence equation (name?)

\begin{align}\label{eqn:divergenceFaraday}
\partial_\mu F^{\mu\nu} = 0
\end{align}

and also the dual tensor relation

\begin{align}\label{eqn:dualFaraday}
\epsilon^{\sigma\mu\alpha\beta} \partial_\mu F_{\alpha\beta} = 0
\end{align}

I cannot seem to figure out how to derive (\ref{eqn:waveFaraday}) starting from these tensor relations?

This probably has something to do with the fact that we require both the divergence and the dual relations 
(
\ref{eqn:divergenceFaraday}
,
\ref{eqn:dualFaraday}
)
expressed together to do this.

\section{Electromagnetic waves in media.}

Jackson lists the Macroscopic Maxwell equations in (6.70) as 

\begin{align*}
\spacegrad \cdot \BB &= 0 \\
\spacegrad \cdot \BD &= 4 \pi \rho \\
\spacegrad \cross \BE + \inv{c}\PD{t}{\BB} &= 0 \\
\spacegrad \cross \BH - \inv{c}\PD{t}{\BD} &= \frac{4 \pi}{c} \BJ  \\
\end{align*}

For vacuum, and $\BB = \mu \BH$, and $\BD = \epsilon \BE$, we can assemble these into his (7.1) equations

\begin{align*}
\spacegrad \cdot \BB &= 0 \\
\spacegrad \cdot \BE &= 0 \\
\spacegrad \cross \BE + \inv{c}\PD{t}{\BB} &= 0 \\
\spacegrad \cross \BB - \frac{\epsilon \mu}{c}\PD{t}{\BE} &= 0  \\
\end{align*}

In this macroscopic form, we cannot assemble the equations into a nice tidy GA form.  The best we can do is

\begin{align}
\spacegrad \BE + \partial_0 (I\BB) &= 0 \\
\spacegrad (I \BB) + \epsilon \mu \partial_0 \BE &= 0 
\end{align}

Although not as pretty, we can at least derive the wave equations from these.  For example for $\BE$, we apply one additional
spatial gradient

\begin{align*}
0 
&= \spacegrad^2 \BE + \partial_0 (\spacegrad I \BB) \\
&= \spacegrad^2 \BE + \partial_0 ( -\epsilon \mu \partial_0 \BE ) \\
\end{align*}

For $\BB$ we get the same, and have two wave equations

\begin{align}
\frac{\mu \epsilon}{c^2} \frac{\partial^2 \BE}{\partial t^2} - \spacegrad^2 \BE &= 0 \\
\frac{\mu \epsilon}{c^2} \frac{\partial^2 \BB}{\partial t^2} - \spacegrad^2 \BB &= 0
\end{align}

The wave velocity is thus not $c$, but instead the reduced speed of $c/{\sqrt{\mu\epsilon}}$.

\bibliographystyle{plainnat}
\bibliography{myrefs}

\end{document}

%
% Copyright � 2012 Peeter Joot.  All Rights Reserved.
% Licenced as described in the file LICENSE under the root directory of this GIT repository.
%

% 
% 
%\documentclass[]{eliblog}

\usepackage{amsmath}
\usepackage{mathpazo}

%
% shorthand for bold symbols, convenient for vectors and matrices
%
\newcommand{\Ba}[0]{\mathbf{a}}
\newcommand{\Bb}[0]{\mathbf{b}}
\newcommand{\Bc}[0]{\mathbf{c}}
\newcommand{\Bd}[0]{\mathbf{d}}
\newcommand{\Be}[0]{\mathbf{e}}
\newcommand{\Bf}[0]{\mathbf{f}}
\newcommand{\Bg}[0]{\mathbf{g}}
\newcommand{\Bh}[0]{\mathbf{h}}
\newcommand{\Bi}[0]{\mathbf{i}}
\newcommand{\Bj}[0]{\mathbf{j}}
\newcommand{\Bk}[0]{\mathbf{k}}
\newcommand{\Bl}[0]{\mathbf{l}}
\newcommand{\Bm}[0]{\mathbf{m}}
\newcommand{\Bn}[0]{\mathbf{n}}
\newcommand{\Bo}[0]{\mathbf{o}}
\newcommand{\Bp}[0]{\mathbf{p}}
\newcommand{\Bq}[0]{\mathbf{q}}
\newcommand{\Br}[0]{\mathbf{r}}
\newcommand{\Bs}[0]{\mathbf{s}}
\newcommand{\Bt}[0]{\mathbf{t}}
\newcommand{\Bu}[0]{\mathbf{u}}
\newcommand{\Bv}[0]{\mathbf{v}}
\newcommand{\Bw}[0]{\mathbf{w}}
\newcommand{\Bx}[0]{\mathbf{x}}
\newcommand{\By}[0]{\mathbf{y}}
\newcommand{\Bz}[0]{\mathbf{z}}
\newcommand{\BA}[0]{\mathbf{A}}
\newcommand{\BB}[0]{\mathbf{B}}
\newcommand{\BC}[0]{\mathbf{C}}
\newcommand{\BD}[0]{\mathbf{D}}
\newcommand{\BE}[0]{\mathbf{E}}
\newcommand{\BF}[0]{\mathbf{F}}
\newcommand{\BG}[0]{\mathbf{G}}
\newcommand{\BH}[0]{\mathbf{H}}
\newcommand{\BI}[0]{\mathbf{I}}
\newcommand{\BJ}[0]{\mathbf{J}}
\newcommand{\BK}[0]{\mathbf{K}}
\newcommand{\BL}[0]{\mathbf{L}}
\newcommand{\BM}[0]{\mathbf{M}}
\newcommand{\BN}[0]{\mathbf{N}}
\newcommand{\BO}[0]{\mathbf{O}}
\newcommand{\BP}[0]{\mathbf{P}}
\newcommand{\BQ}[0]{\mathbf{Q}}
\newcommand{\BR}[0]{\mathbf{R}}
\newcommand{\BS}[0]{\mathbf{S}}
\newcommand{\BT}[0]{\mathbf{T}}
\newcommand{\BU}[0]{\mathbf{U}}
\newcommand{\BV}[0]{\mathbf{V}}
\newcommand{\BW}[0]{\mathbf{W}}
\newcommand{\BX}[0]{\mathbf{X}}
\newcommand{\BY}[0]{\mathbf{Y}}
\newcommand{\BZ}[0]{\mathbf{Z}}

\newcommand{\Bzero}[0]{\mathbf{0}}
\newcommand{\Btheta}[0]{\boldsymbol{\theta}}
\newcommand{\Btau}[0]{\boldsymbol{\tau}}
\newcommand{\Bomega}[0]{\boldsymbol{\omega}}

%
% shorthand for unit vectors
%
\newcommand{\acap}[0]{\hat{\Ba}}
\newcommand{\bcap}[0]{\hat{\Bb}}
\newcommand{\ccap}[0]{\hat{\Bc}}
\newcommand{\dcap}[0]{\hat{\Bd}}
\newcommand{\ecap}[0]{\hat{\Be}}
\newcommand{\fcap}[0]{\hat{\Bf}}
\newcommand{\gcap}[0]{\hat{\Bg}}
\newcommand{\hcap}[0]{\hat{\Bh}}
\newcommand{\icap}[0]{\hat{\Bi}}
\newcommand{\jcap}[0]{\hat{\Bj}}
\newcommand{\kcap}[0]{\hat{\Bk}}
\newcommand{\lcap}[0]{\hat{\Bl}}
\newcommand{\mcap}[0]{\hat{\Bm}}
\newcommand{\ncap}[0]{\hat{\Bn}}
\newcommand{\ocap}[0]{\hat{\Bo}}
\newcommand{\pcap}[0]{\hat{\Bp}}
\newcommand{\qcap}[0]{\hat{\Bq}}
\newcommand{\rcap}[0]{\hat{\Br}}
\newcommand{\scap}[0]{\hat{\Bs}}
\newcommand{\tcap}[0]{\hat{\Bt}}
\newcommand{\ucap}[0]{\hat{\Bu}}
\newcommand{\vcap}[0]{\hat{\Bv}}
\newcommand{\wcap}[0]{\hat{\Bw}}
\newcommand{\xcap}[0]{\hat{\Bx}}
\newcommand{\ycap}[0]{\hat{\By}}
\newcommand{\zcap}[0]{\hat{\Bz}}
\newcommand{\thetacap}[0]{\hat{\Btheta}}

%
% to write R^n and C^n in a distinguishable fashion.  Perhaps change this
% to the double lined characters upon figuring out how to do so.
%
\newcommand{\C}[1]{$\mathbb{C}^{#1}$}
\newcommand{\R}[1]{$\mathbb{R}^{#1}$}

%
% various generally useful helpers
%

% derivative of #1 wrt. #2:
\newcommand{\D}[2] {\frac {d#2} {d#1}}

\newcommand{\inv}[1]{\frac{1}{#1}}
\newcommand{\cross}[0]{\times}

\newcommand{\abs}[1]{\lvert{#1}\rvert}
\newcommand{\norm}[1]{\lVert{#1}\rVert}
\newcommand{\innerprod}[2]{\langle{#1}, {#2}\rangle}
\newcommand{\dotprod}[2]{{#1} \cdot {#2}}
\newcommand{\bdotprod}[2]{\left({#1} \cdot {#2}\right)}
\newcommand{\crossprod}[2]{{#1} \cross {#2}}
\newcommand{\tripleprod}[3]{\dotprod{\left(\crossprod{#1}{#2}\right)}{#3}}

\DeclareMathOperator{\Proj}{Proj}
\DeclareMathOperator{\Span}{span}
\DeclareMathOperator{\Sgn}{sgn}
\DeclareMathOperator{\Area}{Area}
\DeclareMathOperator{\Volume}{Volume}

%
% A few miscellaneous things specific to this document
%
\newcommand{\crossop}[1]{\crossprod{#1}{}}

% R2 vector.
\newcommand{\VectorTwo}[2]{
\begin{bmatrix}
 {#1} \\
 {#2}
\end{bmatrix}
}

\newcommand{\VectorN}[1]{
\begin{bmatrix}
{#1}_1 \\
{#1}_2 \\
\vdots \\
{#1}_N \\
\end{bmatrix}
}

\newcommand{\DETuvij}[4]{
\begin{vmatrix}
 {#1}_{#3} & {#1}_{#4} \\
 {#2}_{#3} & {#2}_{#4}
\end{vmatrix}
}

\newcommand{\DETuvwijk}[6]{
\begin{vmatrix}
 {#1}_{#4} & {#1}_{#5} & {#1}_{#6} \\
 {#2}_{#4} & {#2}_{#5} & {#2}_{#6} \\
 {#3}_{#4} & {#3}_{#5} & {#3}_{#6}
\end{vmatrix}
}

\newcommand{\DETuvwxijkl}[8]{
\begin{vmatrix}
 {#1}_{#5} & {#1}_{#6} & {#1}_{#7} & {#1}_{#8} \\
 {#2}_{#5} & {#2}_{#6} & {#2}_{#7} & {#2}_{#8} \\
 {#3}_{#5} & {#3}_{#6} & {#3}_{#7} & {#3}_{#8} \\
 {#4}_{#5} & {#4}_{#6} & {#4}_{#7} & {#4}_{#8} \\
\end{vmatrix}
}

%\newcommand{\DETuvwxyijklm}[10]{
%\begin{vmatrix}
% {#1}_{#6} & {#1}_{#7} & {#1}_{#8} & {#1}_{#9} & {#1}_{#10} \\
% {#2}_{#6} & {#2}_{#7} & {#2}_{#8} & {#2}_{#9} & {#2}_{#10} \\
% {#3}_{#6} & {#3}_{#7} & {#3}_{#8} & {#3}_{#9} & {#3}_{#10} \\
% {#4}_{#6} & {#4}_{#7} & {#4}_{#8} & {#4}_{#9} & {#4}_{#10} \\
% {#5}_{#6} & {#5}_{#7} & {#5}_{#8} & {#5}_{#9} & {#5}_{#10}
%\end{vmatrix}
%}

% R3 vector.
\newcommand{\VectorThree}[3]{
\begin{bmatrix}
 {#1} \\
 {#2} \\
 {#3}
\end{bmatrix}
}



\author{Peeter Joot}
\email{peeter.joot@gmail.com}


\chapter{Space time algebra solutions of the Maxwell equation for discrete frequencies}
\label{chap:maxwellVacuum}
%\date{July 2, 2009 $RCSfile: maxwellVacuum.tex,v $ Last $Revision: 1.8 $ $Date: 2009/08/06 09:35:17 $}
%%\date{July 2, 2009}
%%\revisionInfo{$RCSfile: maxwellVacuum.tex,v $ Last $Revision: 1.8 $ $Date: 2009/08/06 09:35:17 $}
%\blogpage{http://sites.google.com/site/peeterjoot/math2009/maxwellVacuum.pdf}

\beginArtWithToc

\section{Motivation}

How to obtain solutions to Maxwell's equations in vacuum is well known.  The aim here is to explore the same problem starting with the Geometric Algebra (GA) formalism (\citep{doran2003gap}) of the Maxwell equation.

\begin{align}\label{eqn:maxwellVacuum:maxwell}
\grad F &= J/\epsilon_0 c \\
F &= \grad \wedge A = \BE + i c \BB
\end{align}

A Fourier transformation attack on the equation should be possible, so let us see what falls out doing so.

\subsection{Fourier problem}

Picking an observer bias for the gradient by premultiplying with $\gamma_0$ the vacuum equation for light can therefore also be written as

\begin{align*}
0
&= \gamma_0 \grad F \\
&= \gamma_0 (\gamma^0 \partial_0 + \gamma^k \partial_k) F \\
&= (\partial_0 - \gamma^k \gamma_0 \partial_k) F \\
&= (\partial_0 + \sigma^k \partial_k) F \\
&= \left(\inv{c}\partial_t + \spacegrad \right) F \\
\end{align*}

A Fourier transformation of this equation produces

\begin{align*}
0 &= \inv{c} \frac{\partial F}{\partial t}(\Bk,t) + \inv{(\sqrt{2\pi})^3} \int \sigma^m \partial_m F(\Bx,t) e^{-i \Bk \cdot \Bx} d^3 x
\end{align*}

and with a single integration by parts one has

\begin{align*}
0
&= \inv{c} \frac{\partial F}{\partial t}(\Bk,t) - \inv{(\sqrt{2\pi})^3} \int \sigma^m F(\Bx,t) (-i k_m) e^{-i \Bk \cdot \Bx} d^3 x \\
&= \inv{c} \frac{\partial F}{\partial t}(\Bk,t) + \inv{(\sqrt{2\pi})^3} \int \Bk F(\Bx,t) i e^{-i \Bk \cdot \Bx} d^3 x \\
&= \inv{c} \frac{\partial F}{\partial t}(\Bk,t) + i \Bk \hat{F}(\Bk,t)
\end{align*}

The flexibility to employ the pseudoscalar as the imaginary $i = \gamma_0 \gamma_1 \gamma_2 \gamma_3$ has been employed above, so it should be noted that pseudoscalar commutation with Dirac bivectors was implied above, but also that we do not have the flexibility to commute $\Bk$ with $F$.

Having done this, the problem to solve is now Maxwell's vacuum equation in the frequency domain

\begin{align*}
\frac{\partial F}{\partial t}(\Bk,t) = -i c \Bk \hat{F}(\Bk,t)
\end{align*}

Introducing an angular frequency (spatial) bivector, and its vector dual

\begin{align}
\Omega &= -i c \Bk \\
\Bomega &= c \Bk
\end{align}

This becomes

\begin{align}\label{eqn:maxwellVacuum:MaxwellFreq}
\hat{F}' = \Omega F
\end{align}

With solution

\begin{align}
\hat{F} = e^{\Omega t} \hat{F}(\Bk,0)
\end{align}

Differentiation with respect to time verifies that the ordering of the terms is correct and this does in fact solve (\ref{eqn:maxwellVacuum:MaxwellFreq}).  This is something we have to be careful of due to the possibility of non-commuting variables.

Back substitution into the inverse transform now supplies the time evolution of the field given the initial time specification

\begin{align*}
F(\Bx,t)
&= \inv{(\sqrt{2\pi})^3} \int e^{\Omega t} \hat{F}(\Bk,0) e^{i \Bk \cdot \Bx} d^3 k \\
&= \inv{(2\pi)^3} \int e^{\Omega t} \left( \int {F}(\Bx',0) e^{-i \Bk \cdot \Bx'} d^3 x' \right) e^{i \Bk \cdot \Bx} d^3 k
\end{align*}

Observe that Pseudoscalar exponentials commute with the field because $i$ commutes with spatial vectors and itself

\begin{align*}
F e^{i\theta}
&= (\BE + i c \BB) (C + iS) \\
&=
C (\BE + i c \BB)
+ S (\BE + i c \BB) i  \\
&=
C (\BE + i c \BB)
+ S i (\BE + i c \BB) \\
&=
e^{i\theta} F
\end{align*}

This allows the specifics of the initial time conditions to be suppressed

\begin{align}
F(\Bx,t) &= \int d^3 k e^{\Omega t} e^{i \Bk \cdot \Bx} \int \inv{(2\pi)^3} {F}(\Bx',0) e^{-i \Bk \cdot \Bx'}  d^3 x'
\end{align}

The interior integral has the job of a weighting function over plane wave solutions, and this can be made explicit writing

\begin{align}
D(\Bk) &= \inv{(2\pi)^3} \int {F}(\Bx',0) e^{-i \Bk \cdot \Bx'}  d^3 x' \\
F(\Bx,t) &= \int e^{\Omega t} e^{i \Bk \cdot \Bx} D(\Bk) d^3 k
\end{align}

Many assumptions have been made here, not the least of which was a requirement for the Fourier transform of a bivector valued function to be meaningful, and have an inverse.  It is therefore reasonable to verify that this weighted plane wave result is in fact a solution to the original Maxwell vacuum equation.  Differentiation verifies that things are okay so far

\begin{align*}
\gamma_0 \grad F(\Bx,t)
&=
\left(\inv{c}\partial_t + \spacegrad \right)\int e^{\Omega t} e^{i \Bk \cdot \Bx} D(\Bk) d^3 k \\
&=
\int \left(\inv{c}\Omega e^{\Omega t} + \sigma^m e^{\Omega t} i k_m \right) e^{i \Bk \cdot \Bx} D(\Bk) d^3 k \\
&=
\int \left(\inv{c}(-i \Bk c) + i \Bk \right) e^{\Omega t} e^{i \Bk \cdot \Bx} D(\Bk) d^3 k \\
&= 0 \quad\quad\quad\square
\end{align*}

\subsection{Discretizing and grade restrictions}

The fact that it the integral has zero gradient does not mean that it is a bivector, so there must also be at least also be restrictions on the grades of $D(\Bk)$.

To simplify discussion, let us discretize the integral writing

\begin{align*}
D(\Bk') = D_\Bk \delta^3 (\Bk - \Bk')
\end{align*}

So we have

\begin{align*}
F(\Bx,t)
&= \int e^{\Omega t} e^{i \Bk' \cdot \Bx} D(\Bk') d^3 k' \\
&= \int e^{\Omega t} e^{i \Bk' \cdot \Bx} D_\Bk \delta^3(\Bk - \Bk') d^3 k' \\
\end{align*}

This produces something planewave-ish

\begin{align}\label{eqn:maxwellVacuum:planewaveish}
F(\Bx,t) &= e^{\Omega t} e^{i \Bk \cdot \Bx} D_\Bk
\end{align}

Observe that at $t=0$ we have

\begin{align*}
F(\Bx,0)
&= e^{i \Bk \cdot \Bx} D_\Bk  \\
&= (\cos (\Bk \cdot \Bx) + i \sin(\Bk \cdot \Bx)) D_\Bk  \\
\end{align*}

There is therefore a requirement for $D_\Bk$ to be either a spatial vector or its dual, a spatial bivector.  For example taking $D_k$ to be a spatial vector we can then identify the electric and magnetic components of the field

\begin{align*}
\BE(\Bx,0) &= \cos (\Bk \cdot \Bx) D_\Bk \\
c \BB(\Bx,0) &= \sin (\Bk \cdot \Bx) D_\Bk
\end{align*}

and if $D_k$ is taken to be a spatial bivector, this pair of identifications would be inverted.

Considering (\ref{eqn:maxwellVacuum:planewaveish}) at $\Bx=0$, we have

\begin{align*}
F(0, t)
&= e^{\Omega t} D_\Bk \\
&= (\cos(\Abs{\Omega} t) + \hat{\Omega} \sin(\Abs{\Omega} t)) D_\Bk \\
&= (\cos(\Abs{\Omega} t) - i \hat{\Bk} \sin(\Abs{\Omega} t)) D_\Bk \\
\end{align*}

If $D_\Bk$ is first assumed to be a spatial vector, then $F$ would have a pseudoscalar component if $D_\Bk$ has any component parallel to $\hat{\Bk}$.

\begin{align}\label{eqn:maxwellVacuum:commutationRequirementVector}
D_\Bk \in \span\{\sigma^m\} \implies D_\Bk \cdot \hat{\Bk} = 0
\end{align}
\begin{align}\label{eqn:maxwellVacuum:commutationRequirementBiVector}
D_\Bk \in \span\{\sigma^a \wedge \sigma^b\} \implies D_\Bk \cdot (i\hat{\Bk}) = 0
\end{align}

Since we can convert between the spatial vector and bivector cases using a duality transformation, there may not appear to be any loss of generality imposing a spatial vector restriction on $D_\Bk$, at least in this current free case.  However, an attempt to do so leads to trouble.  In particular, this leads to collinear electric and magnetic fields, and thus the odd seeming condition where the field energy density is non-zero but the field momentum density (Poynting vector $\BP \propto \BE \cross \BB$) is zero.  In retrospect being forced down the path of including both grades is not unreasonable, especially since this gives $D_\Bk$ precisely the form of the field itself $F = \BE + i c \BB$.

\section{Electric and Magnetic field split}

With the basic form of the Maxwell vacuum solution determined, we are now ready to start extracting information from the solution and making comparisons with the more familiar vector form.  To start doing the phasor form of the fundamental solution can be expanded explicitly in terms of two arbitrary spatial parametrization vectors $\BE_\Bk$ and $\BB_\Bk$.

\begin{align}\label{eqn:maxwellVacuum:phasor}
F &= e^{-i\Bomega t} e^{i \Bk \cdot \Bx} (\BE_\Bk + i c \BB_\Bk)
\end{align}

Whether these parametrization vectors have any relation to electric and magnetic fields respectively will have to be determined, but making that assumption for now to label these uniquely does not seem unreasonable.

From (\ref{eqn:maxwellVacuum:phasor}) we can compute the electric and magnetic fields by the conjugate relations (\ref{eqn:maxwellVacuum:conjuagateSplit}).  Our conjugate is

\begin{align*}
F^\dagger
&= (\BE_\Bk - i c \BB_\Bk) e^{-i \Bk \cdot \Bx} e^{i\Bomega t} \\
&=
e^{-i\Bomega t}
e^{-i \Bk \cdot \Bx}
(\BE_\Bk - i c \BB_\Bk)
\end{align*}

Thus for the electric field

\begin{align*}
F + F^\dagger
&=
e^{-i\Bomega t} \left(
 e^{i \Bk \cdot \Bx} (\BE_\Bk + i c \BB_\Bk)
+e^{-i \Bk \cdot \Bx} (\BE_\Bk - i c \BB_\Bk)
\right) \\
&=
e^{-i\Bomega t} \left(
 2 \cos(\Bk \cdot \Bx) \BE_\Bk
+ i c (2 i) \sin(\Bk \cdot \Bx) \BB_\Bk
\right) \\
&=
2 \cos(\omega t) \left(
 \cos(\Bk \cdot \Bx) \BE_\Bk
- c \sin(\Bk \cdot \Bx) \BB_\Bk
\right) \\
&+ 2
\sin(\omega t)
\kcap \cross
\left(
 \cos(\Bk \cdot \Bx) \BE_\Bk
- c \sin(\Bk \cdot \Bx) \BB_\Bk
\right) \\
\end{align*}

So for the electric field $\BE = \inv{2}(F + F^\dagger)$ we have

\begin{align}\label{eqn:maxwellVacuum:electricSplit}
\BE &=
\left( \cos(\omega t) + \sin(\omega t) \kcap \cross \right)
\left(
 \cos(\Bk \cdot \Bx) \BE_\Bk
- c \sin(\Bk \cdot \Bx) \BB_\Bk
\right)
\end{align}

Similarly for the magnetic field we have
\begin{align*}
F - F^\dagger
&=
e^{-i\Bomega t} \left(
 e^{i \Bk \cdot \Bx} (\BE_\Bk + i c \BB_\Bk)
-e^{-i \Bk \cdot \Bx} (\BE_\Bk - i c \BB_\Bk)
\right) \\
&=
e^{-i\Bomega t} \left(
 2 i \sin(\Bk \cdot \Bx) \BE_\Bk
+ 2 i c \cos(\Bk \cdot \Bx) \BB_\Bk
\right) \\
\end{align*}

This gives $c \BB = \inv{2i}(F - F^\dagger)$ we have

\begin{align}\label{eqn:maxwellVacuum:magneticSplit}
c \BB &=
\left( \cos(\omega t) + \sin(\omega t) \kcap \cross \right)
\left(
 \sin(\Bk \cdot \Bx) \BE_\Bk
+ c \cos(\Bk \cdot \Bx) \BB_\Bk
\right)
\end{align}

Observe that the action of the time dependent phasor has been expressed, somewhat abusively and sneakily, in a scalar plus cross product operator form.  The end result, when applied to a vector perpendicular to $\kcap$, is still a vector

\begin{align*}
e^{-i\Bomega t} \Ba
&=
\left( \cos(\omega t) + \sin(\omega t) \kcap \cross \right) \Ba
\end{align*}

Also observe that the Hermitian conjugate split of the total field bivector $F$ produces vectors $\BE$ and $\BB$, not phasors.  There is no further need to take real or imaginary parts nor treat the phasor (\ref{eqn:maxwellVacuum:phasor}) as an artificial mathematical construct used for convenience only.

With $\BE \cdot \kcap = \BB \cdot \kcap = 0$, we have here what Jackson (\citep{jackson1975cew}, ch7), calls a transverse wave.

\subsection{Polar Form}

Suppose an explicit polar form is introduced for the plane vectors $\BE_\Bk$, and $\BB_\Bk$.  Let

\begin{align*}
\BE_\Bk &= E {\hat{\BE}_k} \\
\BB_\Bk &= B {\hat{\BE}_k} e^{i\kcap \theta}
\end{align*}

Then for the field we have

\begin{align}\label{eqn:maxwellVacuum:phasorPolar}
F &= e^{-i\Bomega t} e^{i \Bk \cdot \Bx} (E + i c B e^{-i\kcap \theta}) \hat{\BE}_k
\end{align}

For the conjugate
\begin{align*}
F^\dagger
&=
\hat{\BE}_k
(E - i c B e^{i\kcap \theta})
e^{-i \Bk \cdot \Bx}
e^{i\Bomega t} \\
&=
e^{-i\Bomega t} e^{-i \Bk \cdot \Bx} (E - i c B e^{-i\kcap \theta}) \hat{\BE}_k
\end{align*}

So, in the polar form we have for the electric, and magnetic fields

\begin{align}\label{eqn:maxwellVacuum:fieldsPolar}
\BE &= e^{-i\Bomega t} (E \cos(\Bk \cdot \Bx) - c B \sin(\Bk \cdot \Bx) e^{-i \kcap\theta}) \hat{\BE}_k \\
c \BB &= e^{-i\Bomega t} (E \sin(\Bk \cdot \Bx) + c B \cos(\Bk \cdot \Bx) e^{-i \kcap\theta}) \hat{\BE}_k
\end{align}

Observe when $\theta$ is an integer multiple of $\pi$, $\BE$ and $\BB$ are colinear, having the zero Poynting vector mentioned previously.
Now, for arbitrary $\theta$ it does not appear that there is any inherent perpendicularity between the electric and magnetic fields.  It is common
to read of light being the propagation of perpendicular fields, both perpendicular to the propagation direction.  We have perpendicularity to the
propagation direction by virtue of requiring that the field be a (Dirac) bivector, but it does not look like the solution requires any inherent perpendicularity for the field components.  It appears that a normal triplet of field vectors and propagation directions must actually be a special case.
Intuition says that this freedom to pick different magnitude or angle between $\BE_\Bk$ and $\BB_\Bk$ in the plane perpendicular to the transmission direction may correspond to different mixes of linear, circular, and elliptic polarization, but this has to be confirmed.

Working towards confirming (or disproving) this intuition, lets find the constraints on the fields that lead to normal electric and magnetic fields.  This should follow by taking dot products

\begin{align*}
\BE \cdot \BB c
&=
%\gpgradezero{
\left\langle{
e^{-i\Bomega t} (E \cos(\Bk \cdot \Bx) - c B \sin(\Bk \cdot \Bx) e^{-i \kcap\theta}) \hat{\BE}_k
\hat{\BE}_k
e^{i\Bomega t} (E \sin(\Bk \cdot \Bx) + c B \cos(\Bk \cdot \Bx) e^{i \kcap\theta})
%} \\
}\right\rangle \\
&=
%\gpgradezero{
\left\langle{
(E \cos(\Bk \cdot \Bx) - c B \sin(\Bk \cdot \Bx) e^{-i \kcap\theta})
(E \sin(\Bk \cdot \Bx) + c B \cos(\Bk \cdot \Bx) e^{i \kcap\theta})
%} \\
}\right\rangle \\
&=
(E^2 - c^2 B^2) \cos(\Bk \cdot \Bx) \sin(\Bk \cdot \Bx)
+ c E B
%\gpgradezero{
\left\langle{
\cos^2(\Bk \cdot \Bx) e^{i \kcap \theta}
-\sin^2(\Bk \cdot \Bx) e^{-i \kcap \theta}
%} \\
}\right\rangle \\
&=
(E^2 - c^2 B^2) \cos(\Bk \cdot \Bx) \sin(\Bk \cdot \Bx)
+ c E B \cos(\theta) ( \cos^2(\Bk \cdot \Bx) -\sin^2(\Bk \cdot \Bx) ) \\
&=
(E^2 - c^2 B^2) \cos(\Bk \cdot \Bx) \sin(\Bk \cdot \Bx)
+ c E B \cos(\theta) ( \cos^2(\Bk \cdot \Bx) -\sin^2(\Bk \cdot \Bx) ) \\
&=
\inv{2} (E^2 - c^2 B^2) \sin(2 \Bk \cdot \Bx)
+ c E B \cos(\theta) \cos(2 \Bk \cdot \Bx) \\
\end{align*}

The only way this can be zero for any $\Bx$ is if the left and right terms are separately zero, which means

\begin{align*}
\Abs{\BE_k} &= c \Abs{\BB_k} \\
\theta &= \frac{\pi}{2} + n \pi
\end{align*}

This simplifies the phasor considerably, leaving

\begin{align*}
E + i c B e^{-i\kcap \theta}
&=
E(1 + i (\mp i\kcap )) \\
&=
E(1 \pm \kcap)
\end{align*}

So the field is just

\begin{align}
F = e^{-i \Bomega t} e^{i \Bk \cdot \Bx} (1 \pm \kcap) \BE_\Bk
\end{align}

Using this, and some regrouping, a calculation of the field components yields

\begin{align}
\BE &= e^{i \kcap( \pm \Bk \cdot \Bx -\omega t )} \BE_\Bk \\
c \BB &= \pm e^{i \kcap( \pm \Bk \cdot \Bx -\omega t )} i \Bk \BE_\Bk
\end{align}

Observe that $i\Bk$ rotates any vector in the plane perpendicular to $\kcap$ by 90 degrees, so we have here $c \BB = \pm \kcap \cross \BE$.  This is consistent with the transverse wave restriction (7.11) of Jackson (\citep{jackson1975cew}), where he says, the ``curl equations provide a further restriction, namely'', and 

\begin{align}\label{eqn:fooX}
\mathcal{B} = \sqrt{\mu\epsilon} \Bn \cross \mathcal{E}
\end{align}

He works in explicit complex phasor form and CGS units.  He also allows $\Bn$ to be complex.  With real $\Bk$, and no $\BE \cdot \BB = 0$ constraint, it appears that we cannot have such a simple coupling between the field components?  Is it possible that allowing $\Bk$ to be complex allows this cross product coupling constraint on the fields without the explicit 90 degree phase difference between the electric and magnetic fields?

\section{Energy and momentum for the phasor}

To calculate the field energy density we can work with the two fields of equations (\ref{eqn:maxwellVacuum:fieldsPolar}), or work with the phasor (\ref{eqn:maxwellVacuum:phasor}) directly.  From the phasor and the energy-momentum four vector (\ref{eqn:maxwellVacuum:emFourVect}) we have for the energy density 

\begin{align*}
U &= T(\gamma_0) \cdot \gamma_0 \\
&= \frac{-\epsilon_0}{2}\gpgradezero{ F \gamma_0 F \gamma_0 } \\
&= \frac{-\epsilon_0}{2}
%\gpgradezero{ 
\left\langle{
e^{-i\Bomega t} e^{i \Bk \cdot \Bx} (\BE_\Bk + i c \BB_\Bk) \gamma_0 e^{-i\Bomega t} e^{i \Bk \cdot \Bx} (\BE_\Bk + i c \BB_\Bk) \gamma_0 
%} \\
}\right\rangle \\
&= \frac{-\epsilon_0}{2}
%\gpgradezero{ 
\left\langle{
e^{-i\Bomega t} e^{i \Bk \cdot \Bx} (\BE_\Bk + i c \BB_\Bk) (\gamma_0)^2 e^{-i\Bomega t} e^{-i \Bk \cdot \Bx} (-\BE_\Bk + i c \BB_\Bk) 
%} \\
}\right\rangle \\
&= \frac{-\epsilon_0}{2}
%\gpgradezero{ 
\left\langle{
e^{-i\Bomega t} (\BE_\Bk + i c \BB_\Bk) e^{-i\Bomega t} (-\BE_\Bk + i c \BB_\Bk) 
%} \\
}\right\rangle \\
&= \frac{\epsilon_0}{2}\gpgradezero{ (\BE_\Bk + i c \BB_\Bk) (\BE_\Bk - i c \BB_\Bk) } \\
&= 
\frac{\epsilon_0}{2} \left( (\BE_k)^2 + c^2 (\BB_\Bk)^2\right) + {c \epsilon_0} \gpgradezero{ i \BE_\Bk \wedge \BB_\Bk } \\
&= 
\frac{\epsilon_0}{2} \left( (\BE_k)^2 + c^2 (\BB_\Bk)^2\right) + {c \epsilon_0} \gpgradezero{ \BB_\Bk \cross \BE_\Bk } \\
\end{align*}

Quite anticlimactically we have for the energy the sum of the energies associated with the parametrization constants, lending some justification for the initial choice to label these as electric and magnetic fields

\begin{align}
U = \frac{\epsilon_0}{2} \left( (\BE_k)^2 + c^2 (\BB_\Bk)^2\right)
\end{align}

For the momentum, we want the difference of $F F^\dagger$, and $F^\dagger F$

\begin{align*}
F F^\dagger 
&= e^{-i\Bomega t} e^{i \Bk \cdot \Bx} (\BE_\Bk + i c \BB_\Bk) (\BE_\Bk - i c \BB_\Bk) e^{-i \Bk \cdot \Bx} e^{i\Bomega t}  \\
&= (\BE_\Bk + i c \BB_\Bk) (\BE_\Bk - i c \BB_\Bk) \\
&= (\BE_\Bk)^2 + c^2 (\BB_\Bk)^2 - 2 c \BB_\Bk \cross \BE_\Bk
\end{align*}

\begin{align*}
F F^\dagger 
&= (\BE_\Bk - i c \BB_\Bk) e^{-i \Bk \cdot \Bx} e^{i\Bomega t}  e^{-i\Bomega t} e^{i \Bk \cdot \Bx} (\BE_\Bk + i c \BB_\Bk)  \\
&= (\BE_\Bk - i c \BB_\Bk) (\BE_\Bk + i c \BB_\Bk) \\
&= (\BE_\Bk)^2 + c^2 (\BB_\Bk)^2 + 2 c \BB_\Bk \cross \BE_\Bk
\end{align*}

So we have for the momentum, also anticlimactically

\begin{align}
\BP = \inv{c} T(\gamma_0) \wedge \gamma_0 = \epsilon_0 \BE_\Bk \cross \BB_\Bk 
\end{align}

\section{Followup}

Well, that is enough for one day.  Understanding how to express circular and elliptic polarization is one of the logical next steps.  I seem to recall from Susskind's QM lectures that these can be considered superpositions of linearly polarized waves, so examining a sum of two co-directionally propagating fields would seem to be in order.  Also there ought to be a more natural way to express the perpendicularity requirement for the field and the propagation direction.  The fact that the field components and propagation direction when all multiplied is proportional to the spatial pseudoscalar can probably be utilized to tidy this up and also produce a form that allows for simpler summation of fields in different propagation directions.  It also seems reasonable to consider a planar Fourier decomposition of the field components, perhaps framing the superposition of multiple fields in that context.

Reconsilation of the Jackson's (7.11) restriction for perpendicularity of the fields noted above has not been done.  If such a restriction is required with an explicit dot and cross product split of Maxwell's equation, it would make sense to also have this required of a GA based solution.  Is this just a conquense of the differences between his explicit phasor representation, and this geometric approach where the phasor has an explicit representation in terms of the transverse plane?

\section{Appendix.  Background details}

\subsection{Conjugate split}

The Hermitian conjugate is defined as

\begin{align}
A^\dagger = \gamma_0 \tilde{A} \gamma_0
\end{align}

The conjugate action on a multivector product is straightforward to calculate

\begin{align*}
(A B)^\dagger
&= \gamma_0 (A B)^{\tilde{}} \gamma_0 \\
&= \gamma_0 \tilde{B} \tilde{A} \gamma_0 \\
&= \gamma_0 \tilde{B} {\gamma_0}^2 \tilde{A} \gamma_0 \\
&= B^\dagger A^\dagger
\end{align*}

For a spatial vector Hermitian conjugation leaves the vector unaltered

\begin{align*}
\Ba
&= \gamma_0 (\gamma_k \gamma_0)^{\tilde{}} a^k \gamma_0 \\
&= \gamma_0 (\gamma_0 \gamma_k) a^k \gamma_0 \\
&= \gamma_k a^k \gamma_0 \\
&= \Ba
\end{align*}

But the pseudoscalar is negated

\begin{align*}
i^\dagger
&=
\gamma_0 \tilde{i} \gamma_0 \\
&=
\gamma_0 i \gamma_0 \\
&=
-\gamma_0 \gamma_0 i \\
&=
- i \\
\end{align*}

This allows for a split by conjugation of the field into its electric and magnetic field components.

\begin{align*}
F^\dagger
&= -\gamma_0 ( \BE + i c \BB) \gamma_0 \\
&= -\gamma_0^2 ( -\BE + i c \BB) \\
&= \BE - i c\BB \\
\end{align*}

So we have

\begin{align}\label{eqn:maxwellVacuum:conjuagateSplit}
\BE &= \inv{2}(F + F^\dagger) \\
c \BB &= \inv{2i}(F - F^\dagger)
\end{align}

\subsection{Field Energy Momentum density four vector}

In the GA formalism the energy momentum tensor is

\begin{align}
T(a) = \frac{\epsilon_0}{2} F a \tilde{F}
\end{align}

It is not necessarily obvious this bivector-vector-bivector product construction is even a vector quantity.  Expansion of $T(\gamma_0)$ in terms of the electric and magnetic fields demonstrates this vectorial nature.

\begin{align*}
F \gamma_0 \tilde{F}
&=
-(\BE + i c \BB) \gamma_0 (\BE + i c \BB) \\
&=
-\gamma_0 (-\BE + i c \BB) (\BE + i c \BB) \\
&=
-\gamma_0 (-\BE^2 - c^2 \BB^2 + i c (\BB \BE - \BE \BB) ) \\
&=
\gamma_0 (\BE^2 + c^2 \BB^2) - 2 \gamma_0 i c (\BB \wedge \BE) ) \\
&=
\gamma_0 (\BE^2 + c^2 \BB^2) + 2 \gamma_0 c (\BB \cross \BE) \\
&=
\gamma_0 (\BE^2 + c^2 \BB^2) + 2 \gamma_0 c \gamma_k \gamma_0 (\BB \cross \BE)^k \\
&=
\gamma_0 (\BE^2 + c^2 \BB^2) + 2 \gamma_k (\BE \cross (c \BB))^k \\
\end{align*}

Therefore, $T(\gamma_0)$, the energy momentum tensor biased towards a particular observer frame $\gamma_0$
is

\begin{align}\label{eqn:maxwellVacuum:emFourVect}
T(\gamma_0)
&=
\gamma_0 \frac{\epsilon_0}{2} (\BE^2 + c^2 \BB^2) + \gamma_k \epsilon_0 (\BE \cross (c \BB))^k
\end{align}

Recognizable here in the components $T(\gamma_0)$ are the field energy density and momentum density.  In particular the energy density can be obtained by dotting with $\gamma_0$, whereas the (spatial vector) momentum by wedging with $\gamma_0$.

These are

\begin{align}
U \equiv T(\gamma_0) \cdot \gamma_0 &= \frac{1}{2} \left( \epsilon_0 \BE^2 + \inv{\mu_0} \BB^2 \right) \\
c \BP \equiv T(\gamma_0) \wedge \gamma_0 &= \inv{\mu_0} \BE \cross \BB
\end{align}

In terms of the combined field these are

\begin{align}
U &= \frac{-\epsilon_0}{2}( F \gamma_0 F \gamma_0 + \gamma_0 F \gamma_0 F) \\
c \BP &= \frac{-\epsilon_0}{2}( F \gamma_0 F \gamma_0 - \gamma_0 F \gamma_0 F)
\end{align}

Summarizing with the Hermitian conjugate

\begin{align}
U &= \frac{\epsilon_0}{2}( F F^\dagger + F^\dagger F) \\
c \BP &= \frac{\epsilon_0}{2}( F F^\dagger - F^\dagger F)
\end{align}

\subsubsection{Divergence}

Calculation of the divergence produces the components of the Lorentz force densities

\begin{align*}
\grad \cdot T(a)
&= \frac{\epsilon_0}{2} \gpgradezero{ \grad (F a F) } \\
&= \frac{\epsilon_0}{2} \gpgradezero{ (\grad F) a F + (F \grad) F a } \\
\end{align*}

Here the gradient is used implicitly in bidirectional form, where the direction is implied by context.  From Maxwell's equation we have

\begin{align*}
J/\epsilon_0 c
&= (\grad F)^{\tilde{}} \\
&= (\tilde{F} \tilde{\grad}) \\
&= -(F \grad)
\end{align*}

and continuing the expansion

\begin{align*}
\grad \cdot T(a)
&= \frac{1}{2c} \gpgradezero{ J a F - J F a } \\
&= \frac{1}{2c} \gpgradezero{ F J a - J F a } \\
&= \frac{1}{2c} \gpgradezero{ (F J - J F) a } \\
\end{align*}

Wrapping up, the divergence and the adjoint of the energy momentum tensor are

\begin{align}
\grad \cdot T(a) &= \frac{1}{c} (F \cdot J) \cdot a \\
\overbar{T}(\grad) &= F \cdot J/c
\end{align}

When integrated over a volume, the quantities $F \cdot J/c$ are the components of the RHS of the Lorentz force equation $\dot{p} = q F \cdot v/c$.

%\EndArticle

%
% Copyright � 2012 Peeter Joot.  All Rights Reserved.
% Licenced as described in the file LICENSE under the root directory of this GIT repository.
%

% 
% 
%\documentclass[]{eliblog}

\usepackage{amsmath}
\usepackage{mathpazo}

%
% shorthand for bold symbols, convenient for vectors and matrices
%
\newcommand{\Ba}[0]{\mathbf{a}}
\newcommand{\Bb}[0]{\mathbf{b}}
\newcommand{\Bc}[0]{\mathbf{c}}
\newcommand{\Bd}[0]{\mathbf{d}}
\newcommand{\Be}[0]{\mathbf{e}}
\newcommand{\Bf}[0]{\mathbf{f}}
\newcommand{\Bg}[0]{\mathbf{g}}
\newcommand{\Bh}[0]{\mathbf{h}}
\newcommand{\Bi}[0]{\mathbf{i}}
\newcommand{\Bj}[0]{\mathbf{j}}
\newcommand{\Bk}[0]{\mathbf{k}}
\newcommand{\Bl}[0]{\mathbf{l}}
\newcommand{\Bm}[0]{\mathbf{m}}
\newcommand{\Bn}[0]{\mathbf{n}}
\newcommand{\Bo}[0]{\mathbf{o}}
\newcommand{\Bp}[0]{\mathbf{p}}
\newcommand{\Bq}[0]{\mathbf{q}}
\newcommand{\Br}[0]{\mathbf{r}}
\newcommand{\Bs}[0]{\mathbf{s}}
\newcommand{\Bt}[0]{\mathbf{t}}
\newcommand{\Bu}[0]{\mathbf{u}}
\newcommand{\Bv}[0]{\mathbf{v}}
\newcommand{\Bw}[0]{\mathbf{w}}
\newcommand{\Bx}[0]{\mathbf{x}}
\newcommand{\By}[0]{\mathbf{y}}
\newcommand{\Bz}[0]{\mathbf{z}}
\newcommand{\BA}[0]{\mathbf{A}}
\newcommand{\BB}[0]{\mathbf{B}}
\newcommand{\BC}[0]{\mathbf{C}}
\newcommand{\BD}[0]{\mathbf{D}}
\newcommand{\BE}[0]{\mathbf{E}}
\newcommand{\BF}[0]{\mathbf{F}}
\newcommand{\BG}[0]{\mathbf{G}}
\newcommand{\BH}[0]{\mathbf{H}}
\newcommand{\BI}[0]{\mathbf{I}}
\newcommand{\BJ}[0]{\mathbf{J}}
\newcommand{\BK}[0]{\mathbf{K}}
\newcommand{\BL}[0]{\mathbf{L}}
\newcommand{\BM}[0]{\mathbf{M}}
\newcommand{\BN}[0]{\mathbf{N}}
\newcommand{\BO}[0]{\mathbf{O}}
\newcommand{\BP}[0]{\mathbf{P}}
\newcommand{\BQ}[0]{\mathbf{Q}}
\newcommand{\BR}[0]{\mathbf{R}}
\newcommand{\BS}[0]{\mathbf{S}}
\newcommand{\BT}[0]{\mathbf{T}}
\newcommand{\BU}[0]{\mathbf{U}}
\newcommand{\BV}[0]{\mathbf{V}}
\newcommand{\BW}[0]{\mathbf{W}}
\newcommand{\BX}[0]{\mathbf{X}}
\newcommand{\BY}[0]{\mathbf{Y}}
\newcommand{\BZ}[0]{\mathbf{Z}}

\newcommand{\Bzero}[0]{\mathbf{0}}
\newcommand{\Btheta}[0]{\boldsymbol{\theta}}
\newcommand{\Btau}[0]{\boldsymbol{\tau}}
\newcommand{\Bomega}[0]{\boldsymbol{\omega}}

%
% shorthand for unit vectors
%
\newcommand{\acap}[0]{\hat{\Ba}}
\newcommand{\bcap}[0]{\hat{\Bb}}
\newcommand{\ccap}[0]{\hat{\Bc}}
\newcommand{\dcap}[0]{\hat{\Bd}}
\newcommand{\ecap}[0]{\hat{\Be}}
\newcommand{\fcap}[0]{\hat{\Bf}}
\newcommand{\gcap}[0]{\hat{\Bg}}
\newcommand{\hcap}[0]{\hat{\Bh}}
\newcommand{\icap}[0]{\hat{\Bi}}
\newcommand{\jcap}[0]{\hat{\Bj}}
\newcommand{\kcap}[0]{\hat{\Bk}}
\newcommand{\lcap}[0]{\hat{\Bl}}
\newcommand{\mcap}[0]{\hat{\Bm}}
\newcommand{\ncap}[0]{\hat{\Bn}}
\newcommand{\ocap}[0]{\hat{\Bo}}
\newcommand{\pcap}[0]{\hat{\Bp}}
\newcommand{\qcap}[0]{\hat{\Bq}}
\newcommand{\rcap}[0]{\hat{\Br}}
\newcommand{\scap}[0]{\hat{\Bs}}
\newcommand{\tcap}[0]{\hat{\Bt}}
\newcommand{\ucap}[0]{\hat{\Bu}}
\newcommand{\vcap}[0]{\hat{\Bv}}
\newcommand{\wcap}[0]{\hat{\Bw}}
\newcommand{\xcap}[0]{\hat{\Bx}}
\newcommand{\ycap}[0]{\hat{\By}}
\newcommand{\zcap}[0]{\hat{\Bz}}
\newcommand{\thetacap}[0]{\hat{\Btheta}}

%
% to write R^n and C^n in a distinguishable fashion.  Perhaps change this
% to the double lined characters upon figuring out how to do so.
%
\newcommand{\C}[1]{$\mathbb{C}^{#1}$}
\newcommand{\R}[1]{$\mathbb{R}^{#1}$}

%
% various generally useful helpers
%

% derivative of #1 wrt. #2:
\newcommand{\D}[2] {\frac {d#2} {d#1}}

\newcommand{\inv}[1]{\frac{1}{#1}}
\newcommand{\cross}[0]{\times}

\newcommand{\abs}[1]{\lvert{#1}\rvert}
\newcommand{\norm}[1]{\lVert{#1}\rVert}
\newcommand{\innerprod}[2]{\langle{#1}, {#2}\rangle}
\newcommand{\dotprod}[2]{{#1} \cdot {#2}}
\newcommand{\bdotprod}[2]{\left({#1} \cdot {#2}\right)}
\newcommand{\crossprod}[2]{{#1} \cross {#2}}
\newcommand{\tripleprod}[3]{\dotprod{\left(\crossprod{#1}{#2}\right)}{#3}}

\DeclareMathOperator{\Proj}{Proj}
\DeclareMathOperator{\Span}{span}
\DeclareMathOperator{\Sgn}{sgn}
\DeclareMathOperator{\Area}{Area}
\DeclareMathOperator{\Volume}{Volume}

%
% A few miscellaneous things specific to this document
%
\newcommand{\crossop}[1]{\crossprod{#1}{}}

% R2 vector.
\newcommand{\VectorTwo}[2]{
\begin{bmatrix}
 {#1} \\
 {#2}
\end{bmatrix}
}

\newcommand{\VectorN}[1]{
\begin{bmatrix}
{#1}_1 \\
{#1}_2 \\
\vdots \\
{#1}_N \\
\end{bmatrix}
}

\newcommand{\DETuvij}[4]{
\begin{vmatrix}
 {#1}_{#3} & {#1}_{#4} \\
 {#2}_{#3} & {#2}_{#4}
\end{vmatrix}
}

\newcommand{\DETuvwijk}[6]{
\begin{vmatrix}
 {#1}_{#4} & {#1}_{#5} & {#1}_{#6} \\
 {#2}_{#4} & {#2}_{#5} & {#2}_{#6} \\
 {#3}_{#4} & {#3}_{#5} & {#3}_{#6}
\end{vmatrix}
}

\newcommand{\DETuvwxijkl}[8]{
\begin{vmatrix}
 {#1}_{#5} & {#1}_{#6} & {#1}_{#7} & {#1}_{#8} \\
 {#2}_{#5} & {#2}_{#6} & {#2}_{#7} & {#2}_{#8} \\
 {#3}_{#5} & {#3}_{#6} & {#3}_{#7} & {#3}_{#8} \\
 {#4}_{#5} & {#4}_{#6} & {#4}_{#7} & {#4}_{#8} \\
\end{vmatrix}
}

%\newcommand{\DETuvwxyijklm}[10]{
%\begin{vmatrix}
% {#1}_{#6} & {#1}_{#7} & {#1}_{#8} & {#1}_{#9} & {#1}_{#10} \\
% {#2}_{#6} & {#2}_{#7} & {#2}_{#8} & {#2}_{#9} & {#2}_{#10} \\
% {#3}_{#6} & {#3}_{#7} & {#3}_{#8} & {#3}_{#9} & {#3}_{#10} \\
% {#4}_{#6} & {#4}_{#7} & {#4}_{#8} & {#4}_{#9} & {#4}_{#10} \\
% {#5}_{#6} & {#5}_{#7} & {#5}_{#8} & {#5}_{#9} & {#5}_{#10}
%\end{vmatrix}
%}

% R3 vector.
\newcommand{\VectorThree}[3]{
\begin{bmatrix}
 {#1} \\
 {#2} \\
 {#3}
\end{bmatrix}
}



\author{Peeter Joot}
\email{peeter.joot@gmail.com}


\chapter{Transverse electric and magnetic fields}
\index{electric field!transverse}
\index{magnetic field!transverse}
\label{chap:transverseField}

%\blogpage{http://sites.google.com/site/peeterjoot/math2009/transverseField.pdf}

%%\date{July 30, 2009}
%%\revisionInfo{\(RCSfile: transverseField.tex,v \) Last \(Revision: 1.12 \) \(Date: 2009/08/06 09:35:17 \)}

%\date{July 30, 2009.  \(RCSfile: transverseField.tex,v \) Last \(Revision: 1.12 \) \(Date: 2009/08/06 09:35:17 \)}

\beginArtWithToc

\section{Motivation}

In Eli's \href{http://behindtheguesses.blogspot.com/2009/07/transverse-electric-and-magnetic-fields.html}{Transverse Electric and Magnetic Fields in a Conducting Waveguide} blog entry he works through the algebra calculating the transverse components, the perpendicular to the propagation direction components.

This should be possible using Geometric Algebra too, and trying this made for a good exercise.

\section{Setup}

The starting point can be the same, the source free Maxwell's equations.  Writing \(\partial_0 = (1/c) \partial/{\partial t}\), we have

\begin{equation}\label{eqn:transverseField:blah1}
\begin{aligned}
\spacegrad \cdot \BE &= 0 \\
\spacegrad \cdot \BB &= 0 \\
\spacegrad \cross \BE &= - \partial_0 \BB \\
\spacegrad \cross \BB &= \mu \epsilon \partial_0 \BE
\end{aligned}
\end{equation}

Multiplication of the last two equations by the spatial pseudoscalar \(I\), and using \(I \Ba \cross \Bb = \Ba \wedge \Bb\), the curl equations can be written in their dual bivector form

\begin{equation}\label{eqn:transverseField:blah2}
\begin{aligned}
\spacegrad \wedge \BE &= - \partial_0 I \BB \\
\spacegrad \wedge \BB &= \mu \epsilon \partial_0 I \BE
\end{aligned}
\end{equation}

Now adding the dot and curl equations using \(\Ba \Bb = \Ba \cdot \Bb + \Ba \wedge \Bb\) eliminates the cross products

\begin{equation}\label{eqn:transverseField:twoEquations}
\begin{aligned}
\spacegrad \BE &= - \partial_0 I \BB \\
\spacegrad \BB &= \mu \epsilon \partial_0 I \BE
\end{aligned}
\end{equation}

These can be further merged without any loss, into the GA first order equation for Maxwell's equation in \textAndIndex{cgs} units

\begin{equation}\label{eqn:transverseField:maxwell}
\begin{aligned}
\left(\spacegrad + \frac{\sqrt{\mu\epsilon}}{c}\partial_t\right) \left(\BE + \frac{I\BB}{\sqrt{\mu\epsilon}} \right) = 0.
\end{aligned}
\end{equation}

We are really after solutions to the total multivector field \(F = \BE + I \BB/\sqrt{\mu\epsilon}\).  For this problem where separate electric and magnetic field components are desired, working from \eqnref{eqn:transverseField:twoEquations} is perhaps what we want?

Following Eli and Jackson, write \(\spacegrad = \spacegrad_t + \zcap \partial_z\), and 

\begin{equation}\label{eqn:transverseField:blah3}
\begin{aligned}
\BE(x,y,z,t) &= \BE(x,y) e^{\pm i k z - i \omega t} \\
\BB(x,y,z,t) &= \BB(x,y) e^{\pm i k z - i \omega t}
\end{aligned}
\end{equation}

Evaluating the \(z\) and \(t\) partials we have
\begin{equation}\label{eqn:transverseField:blah4}
\begin{aligned}
(\spacegrad_t \pm i k \zcap) \BE(x,y) &= \frac{i\omega}{c} I \BB(x,y) \\
(\spacegrad_t \pm i k \zcap) \BB(x,y) &= -\mu \epsilon \frac{i\omega}{c} I \BE(x,y)
\end{aligned}
\end{equation}

For the remainder of these notes, the explicit \((x,y)\) dependence will be assumed for \(\BE\) and \(\BB\).

An obvious thing to try with these equations is just substitute one into the other.  If that is done we get the pair of second order harmonic equations

\begin{equation}\label{eqn:transverseField:blah5}
\begin{aligned}
{\spacegrad_t}^2
\begin{pmatrix}\BE \\ \BB \end{pmatrix}
= \left( k^2 - \mu \epsilon \frac{\omega^2}{c^2} \right)
\begin{pmatrix}\BE \\ \BB \end{pmatrix}
\end{aligned}
\end{equation}

One could consider the problem solved here.  Separately equating both sides of this equation to zero, we have the \(k^2 = \mu\epsilon \omega^2/c^2\) constraint on the wave number and angular velocity, and the second order Laplacian on the left hand side is solved by the real or imaginary parts of any analytic function.  Especially when one considers that we are after a multivector field that of intrinsic complex nature.

However, that is not really what we want as a solution.  Doing the same on the unified Maxwell equation \eqnref{eqn:transverseField:maxwell}, we have

\begin{equation}\label{eqn:transverseField:max2}
\begin{aligned}
\left(\spacegrad_t \pm i k \zcap - \sqrt{\mu\epsilon}\frac{i\omega}{c}\right) \left(\BE + \frac{I\BB}{\sqrt{\mu\epsilon}} \right) = 0
\end{aligned}
\end{equation}

Selecting scalar, vector, bivector and trivector grades of this equation produces the following respective relations between the various components

\begin{equation}\label{eqn:transverseField:blah6}
\begin{aligned}
0 = \gpgradezero{\cdots} &= \spacegrad_t \cdot \BE \pm i k \zcap \cdot \BE \\
0 = \gpgradeone{\cdots} &= I \spacegrad_t \wedge \BB/\sqrt{\mu\epsilon} \pm i I k \zcap \wedge \BB/\sqrt{\mu\epsilon} - i \sqrt{\mu\epsilon}\frac{\omega}{c} \BE \\
0 = \gpgradetwo{\cdots} &= \spacegrad_t \wedge \BE \pm i k \zcap \wedge \BE - i \frac{\omega}{c} I \BB \\
0 = \gpgradethree{\cdots} &= I \spacegrad_t \cdot \BB/\sqrt{\mu\epsilon} \pm i I k \zcap \cdot \BB/\sqrt{\mu\epsilon}
\end{aligned}
\end{equation}

From the scalar and pseudoscalar grades we have the propagation components in terms of the transverse ones

\begin{equation}\label{eqn:transverseField:blah7}
\begin{aligned}
E_z &= \frac{\pm i}{k} \spacegrad_t \cdot \BE_t \\
B_z &= \frac{\pm i}{k} \spacegrad_t \cdot \BB_t 
\end{aligned}
\end{equation}

But this is the opposite of the relations that we are after.  On the other hand from the vector and bivector grades we have

\begin{equation}\label{eqn:transverseField:messy}
\begin{aligned}
i \frac{\omega}{c} \BE &= -\inv{\mu\epsilon}\left(\spacegrad_t \cross \BB_z \pm i k \zcap \cross \BB_t\right) \\
i \frac{\omega}{c} \BB &= \spacegrad_t \cross \BE_z \pm i k \zcap \cross \BE_t
\end{aligned}
\end{equation}

\section{A clue from the final result}

From \eqnref{eqn:transverseField:messy} and a lot of messy algebra we should be able to get the transverse equations.  Is there a slicker way?  The end result that Eli obtained suggests a path.  That result was

\begin{equation}\label{eqn:transverseField:blah8}
\begin{aligned}
\BE_t = \frac{i}{\mu\epsilon \frac{\omega^2}{c^2} - k^2} \left( \pm k \spacegrad_t E_z - \frac{\omega}{c} \zcap \cross \spacegrad_t B_z \right)
\end{aligned}
\end{equation}

The numerator looks like it can be factored, and after a bit of playing around a suitable factorization can be obtained:

\begin{equation}\label{eqn:transverseField:30}
\begin{aligned}
\gpgradeone{ \left( \pm k + \frac{\omega}{c} \zcap \right) \spacegrad_t \zcap \left( \BE_z + I \BB_z \right) }
&=
\gpgradeone{ \left( \pm k + \frac{\omega}{c} \zcap \right) \spacegrad_t \left( E_z + I B_z \right) } \\
&=
\pm k \spacegrad E_z + \frac{\omega}{c} \gpgradeone{ I \zcap \spacegrad_t B_z } \\
&=
\pm k \spacegrad E_z + \frac{\omega}{c} I \zcap \wedge \spacegrad_t B_z \\
&=
\pm k \spacegrad E_z - \frac{\omega}{c} \zcap \cross \spacegrad_t B_z \\
\end{aligned}
\end{equation}

Observe that the propagation components of the field \(\BE_z + I\BE_z\) can be written in terms of the symmetric product

\begin{equation}\label{eqn:transverseField:50}
\begin{aligned}
\inv{2} \left( \zcap (\BE + I\BB) + (\BE + I\BB) \zcap \right)
&=
\inv{2} \left( \zcap \BE + \BE \zcap \right) + \frac{I}{2} \left( \zcap \BB + \BB \zcap + I \right) \\
&=
\zcap \cdot \BE + I \zcap \cdot \BB
\end{aligned}
\end{equation}

Now the total field in CGS units was actually \(F = \BE + I \BB/\sqrt{\mu\epsilon}\), not \(F = \BE + I \BB\), so the factorization above is not exactly what we want.   It does however, provide the required clue.  We probably get the result we want by forming the symmetric product (a hybrid dot product selecting both the vector and bivector terms).

\section{Symmetric product of the field with the direction vector}

Rearranging Maxwell's equation \eqnref{eqn:transverseField:max2} in terms of the transverse gradient and the total field \(F\) we have

\begin{equation}\label{eqn:transverseField:blah9}
\begin{aligned}
\spacegrad_t F = \left( \mp i k \zcap + \sqrt{\mu\epsilon}\frac{i\omega}{c}\right) F
\end{aligned}
\end{equation}

With this our symmetric product is

\begin{equation}\label{eqn:transverseField:70}
\begin{aligned}
\spacegrad_t ( F \zcap + \zcap F) 
&= (\spacegrad_t F) \zcap - \zcap (\spacegrad_t F) \\
&=
\left( \mp i k \zcap + \sqrt{\mu\epsilon}\frac{i\omega}{c}\right) F \zcap
- \zcap \left( \mp i k \zcap + \sqrt{\mu\epsilon}\frac{i\omega}{c}\right) F \\
&=
i \left( \mp k \zcap + \sqrt{\mu\epsilon}\frac{\omega}{c}\right) (F \zcap - \zcap F) \\
\end{aligned}
\end{equation}

The antisymmetric product on the right hand side should contain the desired transverse field components.  To verify multiply it out

\begin{equation}\label{eqn:transverseField:90}
\begin{aligned}
\inv{2}(F \zcap - \zcap F)  
&=
\inv{2}\left( \left(\BE + I \BB/\sqrt{\mu\epsilon}\right) \zcap - \zcap \left(\BE + I \BB/\sqrt{\mu\epsilon}\right) \right)  \\
&=
\BE \wedge \zcap + I \BB/\sqrt{\mu\epsilon} \wedge \zcap \\
&=
(\BE_t + I \BB_t/\sqrt{\mu\epsilon}) \zcap \\
\end{aligned}
\end{equation}

Now, with multiplication by the conjugate quantity \(-i(\pm k \zcap + \sqrt{\mu\epsilon}\omega/c)\), we can extract these transverse components.

\begin{equation}\label{eqn:transverseField:110}
\begin{aligned}
\left( \pm k \zcap + \sqrt{\mu\epsilon}\frac{\omega}{c}\right) \left( \mp k \zcap + \sqrt{\mu\epsilon}\frac{\omega}{c}\right) (F \zcap - \zcap F) &=
\left( -k^2 + {\mu\epsilon}\frac{\omega^2}{c^2}\right) (F \zcap - \zcap F) 
\end{aligned}
\end{equation}

Rearranging, we have the transverse components of the field

\begin{equation}\label{eqn:transverseField:blah10}
\begin{aligned}
(\BE_t + I \BB_t/\sqrt{\mu\epsilon}) \zcap &=
\frac{i}{k^2 - \mu\epsilon\frac{\omega^2}{c^2}} \left( \pm k \zcap + \sqrt{\mu\epsilon}\frac{\omega}{c}\right) \spacegrad_t \inv{2}( F \zcap + \zcap F) 
\end{aligned}
\end{equation}

With left multiplication by \(\zcap\), and writing \(F = F_t + F_z\) we have

\begin{equation}\label{eqn:transverseField:transverseBoth}
\begin{aligned}
F_t &= \frac{i}{k^2 - \mu\epsilon\frac{\omega^2}{c^2}} \left( \pm k \zcap + \sqrt{\mu\epsilon}\frac{\omega}{c}\right) \spacegrad_t F_z
\end{aligned}
\end{equation}

While this is a complete solution, we can additionally extract the electric and magnetic fields to compare results with Eli's calculation.  We take 
vector grades to do so with \(\BE_t = \gpgradeone{F_t}\), and \(\BB_t/\sqrt{\mu\epsilon} = \gpgradeone{-I F_t}\).   For the transverse electric field

\begin{equation}\label{eqn:transverseField:130}
\begin{aligned}
\gpgradeone{ \left( \pm k \zcap + \sqrt{\mu\epsilon}\frac{\omega}{c}\right) \spacegrad_t (\BE_z + I \BB_z/\sqrt{/\mu\epsilon}) } 
&=
\pm k \zcap (-\zcap) \spacegrad_t E_z + \frac{\omega}{c} \mathLabelBox{\gpgradeone{I \spacegrad_t \zcap}}{\(-I^2 \zcap \cross \spacegrad_t\)} B_z \\
&=
\mp k \spacegrad_t E_z + \frac{\omega}{c} \zcap \cross \spacegrad_t B_z \\
\end{aligned}
\end{equation}

and for the transverse magnetic field

\begin{equation}\label{eqn:transverseField:150}
\begin{aligned}
&\gpgradeone{ -I \left( \pm k \zcap + \sqrt{\mu\epsilon}\frac{\omega}{c}\right) \spacegrad_t (\BE_z + I \BB_z/\sqrt{\mu\epsilon}) }  \\
&=
-I \sqrt{\mu\epsilon}\frac{\omega}{c} \spacegrad_t \BE_z
+\gpgradeone{ \left( \pm k \zcap + \sqrt{\mu\epsilon}\frac{\omega}{c}\right) \spacegrad_t \BB_z/\sqrt{\mu\epsilon} }  \\
&=
- \sqrt{\mu\epsilon}\frac{\omega}{c} \zcap \cross \spacegrad_t E_z
\mp k \spacegrad_t B_z/\sqrt{\mu\epsilon} \\
\end{aligned}
\end{equation}

Thus the split of transverse field into the electric and magnetic components yields

\begin{equation}\label{eqn:transverseField:transversePair}
\begin{aligned}
\BE_t &= \frac{i}{k^2 - \mu\epsilon\frac{\omega^2}{c^2}} \left( \mp k \spacegrad_t E_z + \frac{\omega}{c} \zcap \cross \spacegrad_t B_z \right) \\
\BB_t &= \frac{i}{k^2 - \mu\epsilon\frac{\omega^2}{c^2}} \left( - {\mu\epsilon}\frac{\omega}{c} \zcap \cross \spacegrad_t E_z \mp k \spacegrad_t B_z \right) 
\end{aligned}
\end{equation}

Compared to Eli's method using messy traditional vector algebra, this method also has a fair amount of messy tricky algebra, but of a different sort.

\section{Summary}

There is potentially a lot of new ideas above (some for me even with previous exposure to the Geometric Algebra formalism).  There was no real attempt to teach GA here, but for completeness the GA form of Maxwell's equation was developed from the traditional divergence and curl formulation of Maxwell's equations.  That was mainly due to use of CGS units which differ since this makes Maxwell's equation take a different form from the usual (see \citep{doran2003gap}).

Here a less exploratory summary of the previous results above is assembled.

In these CGS units our field \(F\), and Maxwell's equation (in absence of charge and current), take the form

\begin{equation}\label{eqn:transverseField:foo2}
\begin{aligned}
F &= \BE + \frac{I\BB}{\sqrt{\mu\epsilon}} \\
0 &= \left(\spacegrad + \frac{\sqrt{\mu\epsilon}}{c}\partial_t\right) F 
\end{aligned}
\end{equation}

The electric and magnetic fields can be picked off by selecting the grade one (vector) components

\begin{equation}\label{eqn:transverseField:foo8}
\begin{aligned}
\BE &= \gpgradeone{F} \\
\BB &= \sqrt{\mu\epsilon} \gpgradeone{-I F}
\end{aligned}
\end{equation}

With an explicit sinusoidal and \(z\)-axis time dependence for the field

\begin{equation}\label{eqn:transverseField:foo3}
\begin{aligned}
F(x,y,z,t) &= F(x,y) e^{\pm i k z - i \omega t} 
\end{aligned}
\end{equation}

and a split of the gradient into transverse and \(z\)-axis components \(\spacegrad = \spacegrad_t + \zcap \partial_z\), Maxwell's equation takes the form

\begin{equation}\label{eqn:transverseField:summaryMax2}
\begin{aligned}
\left(\spacegrad_t \pm i k \zcap - \sqrt{\mu\epsilon}\frac{i\omega}{c}\right) F(x,y) = 0
\end{aligned}
\end{equation}

Writing for short \(F = F(x,y)\), we can split the field into transverse and \(z\)-axis components with the commutator and anticommutator products respectively.  For the \(z\)-axis components we have

\begin{equation}\label{eqn:transverseField:foo4}
\begin{aligned}
F_z \zcap \equiv E_z + I B_z = \inv{2} (F \zcap + \zcap F) 
\end{aligned}
\end{equation}

The projections onto the \(z\)-axis and and transverse directions are respectively 

\begin{equation}\label{eqn:transverseField:foo5}
\begin{aligned}
F_z &= \BE_z + I \BB_z = \inv{2} (F + \zcap F \zcap) \\
F_t &= \BE_t + I \BB_t = \inv{2} (F - \zcap F \zcap)
\end{aligned}
\end{equation}

With an application of the transverse gradient to the \(z\)-axis field we easily found the relation between the two field components

\begin{equation}\label{eqn:transverseField:foo6}
\begin{aligned}
\spacegrad_t F_z &= i \left( \pm k \zcap - \sqrt{\mu\epsilon}\frac{\omega}{c}\right) F_t
\end{aligned}
\end{equation}

A left division by the multivector factor gives the total transverse field

\begin{equation}\label{eqn:transverseField:foo7}
\begin{aligned}
F_t &= \inv{i \left( \pm k \zcap - \sqrt{\mu\epsilon}\frac{\omega}{c}\right) } \spacegrad_t F_z 
\end{aligned}
\end{equation}

Multiplication of both the numerator and denominator by the conjugate normalizes this

\begin{equation}\label{eqn:transverseField:summaryTransverseBoth}
\begin{aligned}
F_t &= \frac{i}{k^2 - \mu\epsilon\frac{\omega^2}{c^2}} \left( \pm k \zcap + \sqrt{\mu\epsilon}\frac{\omega}{c}\right) \spacegrad_t F_z
\end{aligned}
\end{equation}

From this the transverse electric and magnetic fields may be picked off using the projective grade selection operations of \eqnref{eqn:transverseField:foo8}, and are

\begin{equation}\label{eqn:transverseField:SummaryTransversePair}
\begin{aligned}
\BE_t &= \frac{i}{\mu\epsilon\frac{\omega^2}{c^2} -k^2} \left( \pm k \spacegrad_t E_z - \frac{\omega}{c} \zcap \cross \spacegrad_t B_z \right) \\
\BB_t &= \frac{i}{\mu\epsilon\frac{\omega^2}{c^2} -k^2} \left( {\mu\epsilon}\frac{\omega}{c} \zcap \cross \spacegrad_t E_z \pm k \spacegrad_t B_z \right) 
\end{aligned}
\end{equation}

%\EndArticle

\part{Lorentz Force.}
\documentclass{article}

\usepackage{amsmath}
\usepackage{mathpazo}

%
% shorthand for bold symbols, convenient for vectors and matrices
%
\newcommand{\Ba}[0]{\mathbf{a}}
\newcommand{\Bb}[0]{\mathbf{b}}
\newcommand{\Bc}[0]{\mathbf{c}}
\newcommand{\Bd}[0]{\mathbf{d}}
\newcommand{\Be}[0]{\mathbf{e}}
\newcommand{\Bf}[0]{\mathbf{f}}
\newcommand{\Bg}[0]{\mathbf{g}}
\newcommand{\Bh}[0]{\mathbf{h}}
\newcommand{\Bi}[0]{\mathbf{i}}
\newcommand{\Bj}[0]{\mathbf{j}}
\newcommand{\Bk}[0]{\mathbf{k}}
\newcommand{\Bl}[0]{\mathbf{l}}
\newcommand{\Bm}[0]{\mathbf{m}}
\newcommand{\Bn}[0]{\mathbf{n}}
\newcommand{\Bo}[0]{\mathbf{o}}
\newcommand{\Bp}[0]{\mathbf{p}}
\newcommand{\Bq}[0]{\mathbf{q}}
\newcommand{\Br}[0]{\mathbf{r}}
\newcommand{\Bs}[0]{\mathbf{s}}
\newcommand{\Bt}[0]{\mathbf{t}}
\newcommand{\Bu}[0]{\mathbf{u}}
\newcommand{\Bv}[0]{\mathbf{v}}
\newcommand{\Bw}[0]{\mathbf{w}}
\newcommand{\Bx}[0]{\mathbf{x}}
\newcommand{\By}[0]{\mathbf{y}}
\newcommand{\Bz}[0]{\mathbf{z}}
\newcommand{\BA}[0]{\mathbf{A}}
\newcommand{\BB}[0]{\mathbf{B}}
\newcommand{\BC}[0]{\mathbf{C}}
\newcommand{\BD}[0]{\mathbf{D}}
\newcommand{\BE}[0]{\mathbf{E}}
\newcommand{\BF}[0]{\mathbf{F}}
\newcommand{\BG}[0]{\mathbf{G}}
\newcommand{\BH}[0]{\mathbf{H}}
\newcommand{\BI}[0]{\mathbf{I}}
\newcommand{\BJ}[0]{\mathbf{J}}
\newcommand{\BK}[0]{\mathbf{K}}
\newcommand{\BL}[0]{\mathbf{L}}
\newcommand{\BM}[0]{\mathbf{M}}
\newcommand{\BN}[0]{\mathbf{N}}
\newcommand{\BO}[0]{\mathbf{O}}
\newcommand{\BP}[0]{\mathbf{P}}
\newcommand{\BQ}[0]{\mathbf{Q}}
\newcommand{\BR}[0]{\mathbf{R}}
\newcommand{\BS}[0]{\mathbf{S}}
\newcommand{\BT}[0]{\mathbf{T}}
\newcommand{\BU}[0]{\mathbf{U}}
\newcommand{\BV}[0]{\mathbf{V}}
\newcommand{\BW}[0]{\mathbf{W}}
\newcommand{\BX}[0]{\mathbf{X}}
\newcommand{\BY}[0]{\mathbf{Y}}
\newcommand{\BZ}[0]{\mathbf{Z}}

\newcommand{\Bzero}[0]{\mathbf{0}}
\newcommand{\Btheta}[0]{\boldsymbol{\theta}}
\newcommand{\Btau}[0]{\boldsymbol{\tau}}
\newcommand{\Bomega}[0]{\boldsymbol{\omega}}

%
% shorthand for unit vectors
%
\newcommand{\acap}[0]{\hat{\Ba}}
\newcommand{\bcap}[0]{\hat{\Bb}}
\newcommand{\ccap}[0]{\hat{\Bc}}
\newcommand{\dcap}[0]{\hat{\Bd}}
\newcommand{\ecap}[0]{\hat{\Be}}
\newcommand{\fcap}[0]{\hat{\Bf}}
\newcommand{\gcap}[0]{\hat{\Bg}}
\newcommand{\hcap}[0]{\hat{\Bh}}
\newcommand{\icap}[0]{\hat{\Bi}}
\newcommand{\jcap}[0]{\hat{\Bj}}
\newcommand{\kcap}[0]{\hat{\Bk}}
\newcommand{\lcap}[0]{\hat{\Bl}}
\newcommand{\mcap}[0]{\hat{\Bm}}
\newcommand{\ncap}[0]{\hat{\Bn}}
\newcommand{\ocap}[0]{\hat{\Bo}}
\newcommand{\pcap}[0]{\hat{\Bp}}
\newcommand{\qcap}[0]{\hat{\Bq}}
\newcommand{\rcap}[0]{\hat{\Br}}
\newcommand{\scap}[0]{\hat{\Bs}}
\newcommand{\tcap}[0]{\hat{\Bt}}
\newcommand{\ucap}[0]{\hat{\Bu}}
\newcommand{\vcap}[0]{\hat{\Bv}}
\newcommand{\wcap}[0]{\hat{\Bw}}
\newcommand{\xcap}[0]{\hat{\Bx}}
\newcommand{\ycap}[0]{\hat{\By}}
\newcommand{\zcap}[0]{\hat{\Bz}}
\newcommand{\thetacap}[0]{\hat{\Btheta}}

%
% to write R^n and C^n in a distinguishable fashion.  Perhaps change this
% to the double lined characters upon figuring out how to do so.
%
\newcommand{\C}[1]{$\mathbb{C}^{#1}$}
\newcommand{\R}[1]{$\mathbb{R}^{#1}$}

%
% various generally useful helpers
%

% derivative of #1 wrt. #2:
\newcommand{\D}[2] {\frac {d#2} {d#1}}

\newcommand{\inv}[1]{\frac{1}{#1}}
\newcommand{\cross}[0]{\times}

\newcommand{\abs}[1]{\lvert{#1}\rvert}
\newcommand{\norm}[1]{\lVert{#1}\rVert}
\newcommand{\innerprod}[2]{\langle{#1}, {#2}\rangle}
\newcommand{\dotprod}[2]{{#1} \cdot {#2}}
\newcommand{\bdotprod}[2]{\left({#1} \cdot {#2}\right)}
\newcommand{\crossprod}[2]{{#1} \cross {#2}}
\newcommand{\tripleprod}[3]{\dotprod{\left(\crossprod{#1}{#2}\right)}{#3}}

\DeclareMathOperator{\Proj}{Proj}
\DeclareMathOperator{\Span}{span}
\DeclareMathOperator{\Sgn}{sgn}
\DeclareMathOperator{\Area}{Area}
\DeclareMathOperator{\Volume}{Volume}

%
% A few miscellaneous things specific to this document
%
\newcommand{\crossop}[1]{\crossprod{#1}{}}

% R2 vector.
\newcommand{\VectorTwo}[2]{
\begin{bmatrix}
 {#1} \\
 {#2}
\end{bmatrix}
}

\newcommand{\VectorN}[1]{
\begin{bmatrix}
{#1}_1 \\
{#1}_2 \\
\vdots \\
{#1}_N \\
\end{bmatrix}
}

\newcommand{\DETuvij}[4]{
\begin{vmatrix}
 {#1}_{#3} & {#1}_{#4} \\
 {#2}_{#3} & {#2}_{#4}
\end{vmatrix}
}

\newcommand{\DETuvwijk}[6]{
\begin{vmatrix}
 {#1}_{#4} & {#1}_{#5} & {#1}_{#6} \\
 {#2}_{#4} & {#2}_{#5} & {#2}_{#6} \\
 {#3}_{#4} & {#3}_{#5} & {#3}_{#6}
\end{vmatrix}
}

\newcommand{\DETuvwxijkl}[8]{
\begin{vmatrix}
 {#1}_{#5} & {#1}_{#6} & {#1}_{#7} & {#1}_{#8} \\
 {#2}_{#5} & {#2}_{#6} & {#2}_{#7} & {#2}_{#8} \\
 {#3}_{#5} & {#3}_{#6} & {#3}_{#7} & {#3}_{#8} \\
 {#4}_{#5} & {#4}_{#6} & {#4}_{#7} & {#4}_{#8} \\
\end{vmatrix}
}

%\newcommand{\DETuvwxyijklm}[10]{
%\begin{vmatrix}
% {#1}_{#6} & {#1}_{#7} & {#1}_{#8} & {#1}_{#9} & {#1}_{#10} \\
% {#2}_{#6} & {#2}_{#7} & {#2}_{#8} & {#2}_{#9} & {#2}_{#10} \\
% {#3}_{#6} & {#3}_{#7} & {#3}_{#8} & {#3}_{#9} & {#3}_{#10} \\
% {#4}_{#6} & {#4}_{#7} & {#4}_{#8} & {#4}_{#9} & {#4}_{#10} \\
% {#5}_{#6} & {#5}_{#7} & {#5}_{#8} & {#5}_{#9} & {#5}_{#10}
%\end{vmatrix}
%}

% R3 vector.
\newcommand{\VectorThree}[3]{
\begin{bmatrix}
 {#1} \\
 {#2} \\
 {#3}
\end{bmatrix}
}


%<misc>
%
\newcommand{\Abs}[1]{{\left\lvert{#1}\right\rvert}}
\newcommand{\spacegrad}[0]{\boldsymbol{\nabla}}
\newcommand{\grad}[0]{\nabla}
\newcommand{\LL}[0]{\mathcal{L}}

% == \partial_{#1} {#2}
\newcommand{\PD}[2]{\frac{\partial {#2}}{\partial {#1}}}
% inline variant
\newcommand{\PDi}[2]{{\partial {#2}}/{\partial {#1}}}

\newcommand{\PDD}[3]{\frac{\partial^2 {#3}}{\partial {#1}\partial {#2}}}
%\newcommand{\PDd}[2]{\frac{\partial^2 {#2}}{{\partial{#1}}^2}}
\newcommand{\PDsq}[2]{\frac{\partial^2 {#2}}{(\partial {#1})^2}}

\newcommand{\Partial}[2]{\frac{\partial {#1}}{\partial {#2}}}
\DeclareMathOperator{\RejName}{Rej}
\newcommand{\Rej}[2]{\RejName_{#1}\left( {#2} \right)}
\newcommand{\Rm}[1]{\mathbb{R}^{#1}}
\newcommand{\Cm}[1]{\mathbb{C}^{#1}}
\newcommand{\conj}[0]{{*}}

%</misc>

% <grade selection>
%
\newcommand{\gpgrade}[2] {{\left\langle{{#1}}\right\rangle}_{#2}}

\newcommand{\gpgradezero}[1] {\gpgrade{#1}{}}
%\newcommand{\gpscalargrade}[1] {{\left\langle{{#1}}\right\rangle}}
%\newcommand{\gpgradezero}[1] {\gpgrade{#1}{0}}

%\newcommand{\gpgradeone}[1] {{\left\langle{{#1}}\right\rangle}_{1}}
\newcommand{\gpgradeone}[1] {\gpgrade{#1}{1}}

\newcommand{\gpgradetwo}[1] {\gpgrade{#1}{2}}
\newcommand{\gpgradethree}[1] {\gpgrade{#1}{3}}
\newcommand{\gpgradefour}[1] {\gpgrade{#1}{4}}
%
% </grade selection>



\newcommand{\adot}[0]{{\dot{a}}}
\newcommand{\bdot}[0]{{\dot{b}}}
% taken for centered dot:
%\newcommand{\cdot}[0]{{\dot{c}}}
%\newcommand{\ddot}[0]{{\dot{d}}}
\newcommand{\edot}[0]{{\dot{e}}}
\newcommand{\fdot}[0]{{\dot{f}}}
\newcommand{\gdot}[0]{{\dot{g}}}
\newcommand{\hdot}[0]{{\dot{h}}}
\newcommand{\idot}[0]{{\dot{i}}}
\newcommand{\jdot}[0]{{\dot{j}}}
\newcommand{\kdot}[0]{{\dot{k}}}
\newcommand{\ldot}[0]{{\dot{l}}}
\newcommand{\mdot}[0]{{\dot{m}}}
\newcommand{\ndot}[0]{{\dot{n}}}
%\newcommand{\odot}[0]{{\dot{o}}}
\newcommand{\pdot}[0]{{\dot{p}}}
\newcommand{\qdot}[0]{{\dot{q}}}
\newcommand{\rdot}[0]{{\dot{r}}}
\newcommand{\sdot}[0]{{\dot{s}}}
\newcommand{\tdot}[0]{{\dot{t}}}
\newcommand{\udot}[0]{{\dot{u}}}
\newcommand{\vdot}[0]{{\dot{v}}}
\newcommand{\wdot}[0]{{\dot{w}}}
\newcommand{\xdot}[0]{{\dot{x}}}
\newcommand{\ydot}[0]{{\dot{y}}}
\newcommand{\zdot}[0]{{\dot{z}}}
\newcommand{\addot}[0]{{\ddot{a}}}
\newcommand{\bddot}[0]{{\ddot{b}}}
\newcommand{\cddot}[0]{{\ddot{c}}}
%\newcommand{\dddot}[0]{{\ddot{d}}}
\newcommand{\eddot}[0]{{\ddot{e}}}
\newcommand{\fddot}[0]{{\ddot{f}}}
\newcommand{\gddot}[0]{{\ddot{g}}}
\newcommand{\hddot}[0]{{\ddot{h}}}
\newcommand{\iddot}[0]{{\ddot{i}}}
\newcommand{\jddot}[0]{{\ddot{j}}}
\newcommand{\kddot}[0]{{\ddot{k}}}
\newcommand{\lddot}[0]{{\ddot{l}}}
\newcommand{\mddot}[0]{{\ddot{m}}}
\newcommand{\nddot}[0]{{\ddot{n}}}
\newcommand{\oddot}[0]{{\ddot{o}}}
\newcommand{\pddot}[0]{{\ddot{p}}}
\newcommand{\qddot}[0]{{\ddot{q}}}
\newcommand{\rddot}[0]{{\ddot{r}}}
\newcommand{\sddot}[0]{{\ddot{s}}}
\newcommand{\tddot}[0]{{\ddot{t}}}
\newcommand{\uddot}[0]{{\ddot{u}}}
\newcommand{\vddot}[0]{{\ddot{v}}}
\newcommand{\wddot}[0]{{\ddot{w}}}
\newcommand{\xddot}[0]{{\ddot{x}}}
\newcommand{\yddot}[0]{{\ddot{y}}}
\newcommand{\zddot}[0]{{\ddot{z}}}

%<bold and dot greek symbols>
%

\newcommand{\Deltadot}[0]{{\dot{\Delta}}}
\newcommand{\Gammadot}[0]{{\dot{\Gamma}}}
\newcommand{\Lambdadot}[0]{{\dot{\Lambda}}}
\newcommand{\Omegadot}[0]{{\dot{\Omega}}}
\newcommand{\Phidot}[0]{{\dot{\Phi}}}
\newcommand{\Pidot}[0]{{\dot{\Pi}}}
\newcommand{\Psidot}[0]{{\dot{\Psi}}}
\newcommand{\Sigmadot}[0]{{\dot{\Sigma}}}
\newcommand{\Thetadot}[0]{{\dot{\Theta}}}
\newcommand{\Upsilondot}[0]{{\dot{\Upsilon}}}
\newcommand{\Xidot}[0]{{\dot{\Xi}}}
\newcommand{\alphadot}[0]{{\dot{\alpha}}}
\newcommand{\betadot}[0]{{\dot{\beta}}}
\newcommand{\chidot}[0]{{\dot{\chi}}}
\newcommand{\deltadot}[0]{{\dot{\delta}}}
\newcommand{\epsilondot}[0]{{\dot{\epsilon}}}
\newcommand{\etadot}[0]{{\dot{\eta}}}
\newcommand{\gammadot}[0]{{\dot{\gamma}}}
\newcommand{\kappadot}[0]{{\dot{\kappa}}}
\newcommand{\lambdadot}[0]{{\dot{\lambda}}}
\newcommand{\mudot}[0]{{\dot{\mu}}}
\newcommand{\nudot}[0]{{\dot{\nu}}}
\newcommand{\omegadot}[0]{{\dot{\omega}}}
\newcommand{\phidot}[0]{{\dot{\phi}}}
\newcommand{\pidot}[0]{{\dot{\pi}}}
\newcommand{\psidot}[0]{{\dot{\psi}}}
\newcommand{\rhodot}[0]{{\dot{\rho}}}
\newcommand{\sigmadot}[0]{{\dot{\sigma}}}
\newcommand{\taudot}[0]{{\dot{\tau}}}
\newcommand{\thetadot}[0]{{\dot{\theta}}}
\newcommand{\upsilondot}[0]{{\dot{\upsilon}}}
\newcommand{\varepsilondot}[0]{{\dot{\varepsilon}}}
\newcommand{\varphidot}[0]{{\dot{\varphi}}}
\newcommand{\varpidot}[0]{{\dot{\varpi}}}
\newcommand{\varrhodot}[0]{{\dot{\varrho}}}
\newcommand{\varsigmadot}[0]{{\dot{\varsigma}}}
\newcommand{\varthetadot}[0]{{\dot{\vartheta}}}
\newcommand{\xidot}[0]{{\dot{\xi}}}
\newcommand{\zetadot}[0]{{\dot{\zeta}}}

\newcommand{\Deltaddot}[0]{{\ddot{\Delta}}}
\newcommand{\Gammaddot}[0]{{\ddot{\Gamma}}}
\newcommand{\Lambdaddot}[0]{{\ddot{\Lambda}}}
\newcommand{\Omegaddot}[0]{{\ddot{\Omega}}}
\newcommand{\Phiddot}[0]{{\ddot{\Phi}}}
\newcommand{\Piddot}[0]{{\ddot{\Pi}}}
\newcommand{\Psiddot}[0]{{\ddot{\Psi}}}
\newcommand{\Sigmaddot}[0]{{\ddot{\Sigma}}}
\newcommand{\Thetaddot}[0]{{\ddot{\Theta}}}
\newcommand{\Upsilonddot}[0]{{\ddot{\Upsilon}}}
\newcommand{\Xiddot}[0]{{\ddot{\Xi}}}
\newcommand{\alphaddot}[0]{{\ddot{\alpha}}}
\newcommand{\betaddot}[0]{{\ddot{\beta}}}
\newcommand{\chiddot}[0]{{\ddot{\chi}}}
\newcommand{\deltaddot}[0]{{\ddot{\delta}}}
\newcommand{\epsilonddot}[0]{{\ddot{\epsilon}}}
\newcommand{\etaddot}[0]{{\ddot{\eta}}}
\newcommand{\gammaddot}[0]{{\ddot{\gamma}}}
\newcommand{\kappaddot}[0]{{\ddot{\kappa}}}
\newcommand{\lambdaddot}[0]{{\ddot{\lambda}}}
\newcommand{\muddot}[0]{{\ddot{\mu}}}
\newcommand{\nuddot}[0]{{\ddot{\nu}}}
\newcommand{\omegaddot}[0]{{\ddot{\omega}}}
\newcommand{\phiddot}[0]{{\ddot{\phi}}}
\newcommand{\piddot}[0]{{\ddot{\pi}}}
\newcommand{\psiddot}[0]{{\ddot{\psi}}}
\newcommand{\rhoddot}[0]{{\ddot{\rho}}}
\newcommand{\sigmaddot}[0]{{\ddot{\sigma}}}
\newcommand{\tauddot}[0]{{\ddot{\tau}}}
\newcommand{\thetaddot}[0]{{\ddot{\theta}}}
\newcommand{\upsilonddot}[0]{{\ddot{\upsilon}}}
\newcommand{\varepsilonddot}[0]{{\ddot{\varepsilon}}}
\newcommand{\varphiddot}[0]{{\ddot{\varphi}}}
\newcommand{\varpiddot}[0]{{\ddot{\varpi}}}
\newcommand{\varrhoddot}[0]{{\ddot{\varrho}}}
\newcommand{\varsigmaddot}[0]{{\ddot{\varsigma}}}
\newcommand{\varthetaddot}[0]{{\ddot{\vartheta}}}
\newcommand{\xiddot}[0]{{\ddot{\xi}}}
\newcommand{\zetaddot}[0]{{\ddot{\zeta}}}

\newcommand{\BDelta}[0]{\boldsymbol{\Delta}}
\newcommand{\BGamma}[0]{\boldsymbol{\Gamma}}
\newcommand{\BLambda}[0]{\boldsymbol{\Lambda}}
\newcommand{\BOmega}[0]{\boldsymbol{\Omega}}
\newcommand{\BPhi}[0]{\boldsymbol{\Phi}}
\newcommand{\BPi}[0]{\boldsymbol{\Pi}}
\newcommand{\BPsi}[0]{\boldsymbol{\Psi}}
\newcommand{\BSigma}[0]{\boldsymbol{\Sigma}}
\newcommand{\BTheta}[0]{\boldsymbol{\Theta}}
\newcommand{\BUpsilon}[0]{\boldsymbol{\Upsilon}}
\newcommand{\BXi}[0]{\boldsymbol{\Xi}}
\newcommand{\Balpha}[0]{\boldsymbol{\alpha}}
\newcommand{\Bbeta}[0]{\boldsymbol{\beta}}
\newcommand{\Bchi}[0]{\boldsymbol{\chi}}
\newcommand{\Bdelta}[0]{\boldsymbol{\delta}}
\newcommand{\Bepsilon}[0]{\boldsymbol{\epsilon}}
\newcommand{\Beta}[0]{\boldsymbol{\eta}}
\newcommand{\Bgamma}[0]{\boldsymbol{\gamma}}
\newcommand{\Bkappa}[0]{\boldsymbol{\kappa}}
\newcommand{\Blambda}[0]{\boldsymbol{\lambda}}
\newcommand{\Bmu}[0]{\boldsymbol{\mu}}
\newcommand{\Bnu}[0]{\boldsymbol{\nu}}
%\newcommand{\Bomega}[0]{\boldsymbol{\omega}}
\newcommand{\Bphi}[0]{\boldsymbol{\phi}}
\newcommand{\Bpi}[0]{\boldsymbol{\pi}}
\newcommand{\Bpsi}[0]{\boldsymbol{\psi}}
\newcommand{\Brho}[0]{\boldsymbol{\rho}}
\newcommand{\Bsigma}[0]{\boldsymbol{\sigma}}
%\newcommand{\Btau}[0]{\boldsymbol{\tau}}
%\newcommand{\Btheta}[0]{\boldsymbol{\theta}}
\newcommand{\Bupsilon}[0]{\boldsymbol{\upsilon}}
\newcommand{\Bvarepsilon}[0]{\boldsymbol{\varepsilon}}
\newcommand{\Bvarphi}[0]{\boldsymbol{\varphi}}
\newcommand{\Bvarpi}[0]{\boldsymbol{\varpi}}
\newcommand{\Bvarrho}[0]{\boldsymbol{\varrho}}
\newcommand{\Bvarsigma}[0]{\boldsymbol{\varsigma}}
\newcommand{\Bvartheta}[0]{\boldsymbol{\vartheta}}
\newcommand{\Bxi}[0]{\boldsymbol{\xi}}
\newcommand{\Bzeta}[0]{\boldsymbol{\zeta}}
%
%</bold and dot greek symbols>
%<infrequent>
%
%\newcommand{\AreaOp}[1]{\AName_{#1}}
%\newcommand{\Babs}[0]{\abs{\BB}}
%\newcommand{\Bcap}[0]{\hat{\BB}}
%\newcommand{\BrPrimeRej}[0]{\rcap(\rcap \wedge \Br')}
%\newcommand{\CA}[0]{\mathcal{A}}
%\newcommand{\Cos}[1]{\cos{\left({#1}\right)}}
%\newcommand{\Det}[1] {\abs{#1}}
%\newcommand{\Dsq}[2] {\frac {\partial^2 {#1}} {\partial {#2}^2}}
%\newcommand{\Exp}[1]{\exp{\left({#1}\right)}}
%\newcommand{\Norm}[1]{\left\lVert{#1}\right\rVert}
%\newcommand{\Sin}[1]{\sin{\left({#1}\right)}}
%\newcommand{\T}[0]{\text{T}}
%\newcommand{\VolumeOp}[1]{\VName_{#1}}
%\newcommand{\agrad}[0]{\Ba \cdot \nabla}
%\newcommand{\alphacap}[0]{\hat{\boldsymbol{\alpha}}}
%\newcommand{\Fcap}[0]{\hat{\BF}}
%\newcommand{\bithree}[0]{{\Bi}_3}
%\newcommand{\bxa}[0]{\Bx\Ba}
%\newcommand{\coordvec}[2]{
%\newcommand{\costheta}[0]{\acap \cdot \xcap}
%\newcommand{\ddt}[1]{\ddot{#1}}
%\newcommand{\ddu}[1] {\frac {d{#1}} {du}}
%\newcommand{\dsqxj}[2] {\frac {\partial^2 {#1}} {\partial {x_{#2}}^2}}
%\newcommand{\dtheta}[1]{\frac{d {#1}}{d \theta}}
%\newcommand{\dt}[1]{\dot{#1}}
%\newcommand{\dt}[1]{\frac{d {#1}}{dt}}
%\newcommand{\dxj}[2] {\frac {\partial {#1}} {\partial {x_{#2}}}}
%\newcommand{\halfPhi}[0]{\frac{\phi}{2}}
%\newcommand{\half}[0]{\inv{2}}
%\newcommand{\inv}[1]{\frac{1}{#1}}
%\newcommand{\laplacian}[0]{\nabla^2}
%\newcommand{\matrixoftx}[3]{
%\newcommand{\nrrp}[0]{\norm{\rcap \wedge \Br'}}
%\newcommand{\oiint}{\bigcirc \hspace{-1.4em} \int \hspace{-.8em} \int}
%\newcommand{\transpose}[1]{{#1}^{\text{T}}}
%\newcommand{\transpose}[1]{{{#1}^{\TextTranspose}}}
%\newcommand{\transpose}[1]{{{#1}^{\text{T}}}}
%\newcommand{\barA}[0]{\bar{A}}
%\newcommand{\qbar}[0]{\bar{q}}
%\newcommand{\qdotbar}[0]{\dot{\bar{q}}}
%
%</infrequent>





%\usepackage{listings}
%\usepackage{txfonts} % for ointctr... (also appears to make "prettier" \int and \sum's)
% makes \grad look funny though (almost like spacegrad, but narrower)
\usepackage[bookmarks=true]{hyperref}

\usepackage{color,cite,graphicx}
   % use colour in the document, put your citations as [1-4]
   % rather than [1,2,3,4] (it looks nicer, and the extended LaTeX2e
   % graphics package. 
\usepackage{latexsym,amssymb,epsf} % don't remember if these are
   % needed, but their inclusion can't do any damage


\title{ Lorentz boost of Lorentz force equations. }
\author{Peeter Joot \quad peeter.joot@gmail.com }
\date{ May 22, 2009.  Last Revision: $Date: 2009/05/23 20:56:55 $ }

\begin{document}

\maketitle{}
\tableofcontents
\section{ Motivation. }

Reading of \cite{bohm1996str} is a treatment of the Lorentz transform
properties of the Lorentz force equation.  This isn't clear to me
without working through it myself, but in doing so I have the urge to
try it with the GA formulation of the Lorentz transformation.

\section{ GA Lorentz tx. }

I've got the coordinate and Geometric Algebra versions of the Lorentz
transform all mixed up in my head.  Let's work through the relations 
between these once more.

Write the Lorentz boost of a four vector $x = x^\mu \gamma_\mu = ct \gamma_0 + x^k \gamma_k$ as

\begin{align}\label{eqn:LorentzBoost}
L(x) &= 
e^{-\alpha \vcap/2}
x
e^{\alpha \vcap/2}
\end{align}

\subsection{ Invariance property. }

A Lorentz transformation (boost or rotation) can be defined as those transformation that leave the four vector square unchanged.

Following \cite{doran2003gap}, work with a $+---$ metric signature ($1 = \gamma_0^2 = -\gamma_k^2$), and $\sigma_k = \gamma_k \gamma_0$.  Our four vector square in this representation has the familiar invariant form

\begin{align*}
x^2 
&= (ct \gamma_0 + x^m \gamma_m) (ct \gamma_0 + x^k \gamma_k) \\
&= (ct \gamma_0 + x^m \gamma_m) \gamma_0^2 (ct \gamma_0 + x^k \gamma_k) \\
&= (ct + x^m \sigma_m) (ct - x^k \sigma_k) \\
&= (ct + \Bx) (ct - \Bx) \\
&= (ct)^2 - \Bx^2
\end{align*}

and we expect this of the Lorentz boost of equation \ref{eqn:LorentzBoost}.  To verify we have

\begin{align*}
L(x)^2 
&=
e^{-\alpha \vcap/2}
x
e^{\alpha \vcap/2}
e^{-\alpha \vcap/2}
x
e^{\alpha \vcap/2} \\
&=
e^{-\alpha \vcap/2}
x
x
e^{\alpha \vcap/2} \\
&=
x^2
e^{-\alpha \vcap/2}
e^{\alpha \vcap/2} \\
&=
x^2
\end{align*}

\subsection{ Sign of the rapidity angle. }

The factor $\alpha$ will be the rapidity angle, but what sign do we want for a boost along the positive $\vcap$ direction?

Dropping to coordinates is an easy way to determine the sign convention in effect.  Write $\vcap = \sigma_1$

\begin{align*}
L(x) &= 
e^{-\alpha \vcap/2}
x
e^{\alpha \vcap/2} \\
&=
(\cosh(\alpha/2) - \sigma_1 \sinh(\alpha/2))
(
x^0 \gamma_0
+x^1 \gamma_1
+x^2 \gamma_2
+x^3 \gamma_3
)
(\cosh(\alpha/2) + \sigma_1 \sinh(\alpha/2))
\end{align*}

$\sigma_1$ commutes with $\gamma_2$ and $\gamma_3$ and anticommutes otherwise, so we have

\begin{align*}
L(x) &= 
(
x^2 \gamma_2
+x^3 \gamma_3
) 
e^{-\alpha \vcap/2}
e^{\alpha \vcap/2}
+
(
x^0 \gamma_0
+x^1 \gamma_1
)
e^{\alpha \vcap} \\
&=
x^2 \gamma_2
+x^3 \gamma_3
+(
x^0 \gamma_0
+x^1 \gamma_1
)
e^{\alpha \vcap} \\
&=
x^2 \gamma_2
+x^3 \gamma_3
+(
x^0 \gamma_0
+x^1 \gamma_1
)
(\cosh(\alpha) + \sigma_1 \sinh(\alpha))
\end{align*}

Expanding out just the $0,1$ terms changed by the transformation we have

\begin{align*}
(
&x^0 \gamma_0
+x^1 \gamma_1
)
(\cosh(\alpha) + \sigma_1 \sinh(\alpha)) \\
&=
x^0 \gamma_0 \cosh(\alpha) 
+x^1 \gamma_1 \cosh(\alpha) 
+x^0 \gamma_0 \sigma_1 \sinh(\alpha)
+x^1 \gamma_1 \sigma_1 \sinh(\alpha) \\
&=
x^0 \gamma_0 \cosh(\alpha) 
+x^1 \gamma_1 \cosh(\alpha) 
+x^0 \gamma_0 \gamma_1 \gamma_0 \sinh(\alpha)
+x^1 \gamma_1 \gamma_1 \gamma_0 \sinh(\alpha) \\
&=
x^0 \gamma_0 \cosh(\alpha) 
+x^1 \gamma_1 \cosh(\alpha) 
-x^0 \gamma_1 \sinh(\alpha)
-x^1 \gamma_0 \sinh(\alpha) \\
&=
\gamma_0 (x^0 \cosh(\alpha) -x^1 \sinh(\alpha) )
+\gamma_1 (x^1 \cosh(\alpha) -x^0 \sinh(\alpha) )
\\
\end{align*}

Writing ${x^\mu}' = L(x) \cdot \gamma^\mu$, and $x^\mu = x \cdot \gamma^\mu$,
and a substitution of $\cosh(\alpha) = 1/\sqrt{1 - \Bv^2/c^2}$, and $\alpha \vcap = \tanh^{-1}(\Bv/c)$,
we have the traditional coordinate
expression for the one directional Lorentz boost

\begin{align}
\begin{bmatrix}
{x^0}' \\
{x^1}' \\
{x^2}' \\
{x^3}'
\end{bmatrix}
&=
\begin{bmatrix}
\cosh\alpha & -\sinh\alpha & 0 & 0 \\
-\sinh\alpha & \cosh\alpha & 0 & 0 \\
0 & 0 & 1 & 0 \\
0 & 0 & 0 & 1 \\
\end{bmatrix}
\begin{bmatrix}
x^0 \\
x^1 \\
x^2 \\
x^3
\end{bmatrix}
\end{align}

Performing this expansion showed initially showed that I had the wrong sign for $\alpha$ in the exponentials and I went back and
adjusted it all accordingly.

\subsection{ Expanding out the Lorentz boost for projective and rejective directions. }

Our position four vector can be decomposed into components that commute and anticommute with the spatial boost direction
$\vcap$.  Write

\begin{align*}
x &= x_\parallel + x_\perp \\
x_\perp \vcap &= \vcap x_\perp \\
x_\parallel \vcap &= -\vcap x_\parallel
\end{align*}

This then allows for a similar, but more general expansion of the coordinate expansion above.  Following the same arguments
as above we have

\begin{align*}
L(x) 
&= x_\perp + x_\parallel e^{\alpha \vcap} \\
&= x_\perp + x_\parallel (\cosh\alpha + \vcap\sinh\alpha) \\
\end{align*}

To procede further a decomposition of $x_\parallel$ is required.  Introducing a four vector with only spatial components for
convienence

\begin{align}
u = \vcap \gamma_0 = - \gamma_0 \vcap
\end{align}

we can compute projective and rejective parts of $x_\parallel$.  This is

\begin{align*}
x_\parallel 
&= x_\parallel u \inv{u} \\
&= (x_\parallel u ) \inv{u} \\
&= (x_\parallel \cdot u + x_\parallel \wedge u) \inv{u} \\
\end{align*}

This is
\begin{align}
x_\parallel &= \underbrace{(x_\parallel \cdot u) \inv{u}}_{\text{spatial four vector components}} + \underbrace{(x_\parallel \wedge u) \cdot \inv{u} }_{\text{timelike component only.}}
\end{align}

As no time components contribute to the first term we can identify $(x_\parallel \cdot u)/u$ with the spatial component of $x_\parallel$ in the direction of $\vcap \gamma_0$.  Similarily $(x_\parallel\wedge u)/u$ should have only the timelike components of $x_\parallel$ with respect to $\gamma_0$.

Some mathematical followup to justify the intuitive description above will be required.  In addition we will want to show that $x_\perp$ has no timelike component ($x_\perp \cdot \gamma_0 = 0$), and that $x_\perp$ also has no components in the direction $\vcap \gamma_0$ ($x_\perp \cdot (\vcap\gamma_0) = 0$).
Let's come back to that afterwards.  For now we can do a preliminary expansion of the boosted vector in terms of this 
projective and rejective split

\begin{align*}
L(x)
&= x_\perp + \left(
\frac{x_\parallel \cdot u}{u} + \left(x_\parallel \wedge u\right) \inv{u} 
\right)\left(\cosh\alpha + u \gamma_0 \sinh\alpha\right) \\
&= x_\perp \\
&+ \left(x_\parallel \wedge u\right) \inv{u} \cosh\alpha 
+ \left(x_\parallel \cdot u\right) \gamma_0 \sinh\alpha \\
&+ \frac{x_\parallel \cdot u}{u} \cosh\alpha 
+ \left(x_\parallel \wedge u\right) \gamma_0 \sinh\alpha
\end{align*}

Here a grouping into pairs of timelike and spacelike components has been used.

Next it is desirable to express the $x_\parallel \cdot u$, and $x_\parallel \wedge u$ products in terms of $\vcap$, and $\gamma_0$ explicitly.

\bibliographystyle{plainnat}
\bibliography{myrefs}

\end{document}

\documentclass{article}      

\usepackage{amsmath}
\usepackage{mathpazo}

%
% shorthand for bold symbols, convenient for vectors and matrices
%
\newcommand{\Ba}[0]{\mathbf{a}}
\newcommand{\Bb}[0]{\mathbf{b}}
\newcommand{\Bc}[0]{\mathbf{c}}
\newcommand{\Bd}[0]{\mathbf{d}}
\newcommand{\Be}[0]{\mathbf{e}}
\newcommand{\Bf}[0]{\mathbf{f}}
\newcommand{\Bg}[0]{\mathbf{g}}
\newcommand{\Bh}[0]{\mathbf{h}}
\newcommand{\Bi}[0]{\mathbf{i}}
\newcommand{\Bj}[0]{\mathbf{j}}
\newcommand{\Bk}[0]{\mathbf{k}}
\newcommand{\Bl}[0]{\mathbf{l}}
\newcommand{\Bm}[0]{\mathbf{m}}
\newcommand{\Bn}[0]{\mathbf{n}}
\newcommand{\Bo}[0]{\mathbf{o}}
\newcommand{\Bp}[0]{\mathbf{p}}
\newcommand{\Bq}[0]{\mathbf{q}}
\newcommand{\Br}[0]{\mathbf{r}}
\newcommand{\Bs}[0]{\mathbf{s}}
\newcommand{\Bt}[0]{\mathbf{t}}
\newcommand{\Bu}[0]{\mathbf{u}}
\newcommand{\Bv}[0]{\mathbf{v}}
\newcommand{\Bw}[0]{\mathbf{w}}
\newcommand{\Bx}[0]{\mathbf{x}}
\newcommand{\By}[0]{\mathbf{y}}
\newcommand{\Bz}[0]{\mathbf{z}}
\newcommand{\BA}[0]{\mathbf{A}}
\newcommand{\BB}[0]{\mathbf{B}}
\newcommand{\BC}[0]{\mathbf{C}}
\newcommand{\BD}[0]{\mathbf{D}}
\newcommand{\BE}[0]{\mathbf{E}}
\newcommand{\BF}[0]{\mathbf{F}}
\newcommand{\BG}[0]{\mathbf{G}}
\newcommand{\BH}[0]{\mathbf{H}}
\newcommand{\BI}[0]{\mathbf{I}}
\newcommand{\BJ}[0]{\mathbf{J}}
\newcommand{\BK}[0]{\mathbf{K}}
\newcommand{\BL}[0]{\mathbf{L}}
\newcommand{\BM}[0]{\mathbf{M}}
\newcommand{\BN}[0]{\mathbf{N}}
\newcommand{\BO}[0]{\mathbf{O}}
\newcommand{\BP}[0]{\mathbf{P}}
\newcommand{\BQ}[0]{\mathbf{Q}}
\newcommand{\BR}[0]{\mathbf{R}}
\newcommand{\BS}[0]{\mathbf{S}}
\newcommand{\BT}[0]{\mathbf{T}}
\newcommand{\BU}[0]{\mathbf{U}}
\newcommand{\BV}[0]{\mathbf{V}}
\newcommand{\BW}[0]{\mathbf{W}}
\newcommand{\BX}[0]{\mathbf{X}}
\newcommand{\BY}[0]{\mathbf{Y}}
\newcommand{\BZ}[0]{\mathbf{Z}}

\newcommand{\Bzero}[0]{\mathbf{0}}
\newcommand{\Btheta}[0]{\boldsymbol{\theta}}
\newcommand{\Btau}[0]{\boldsymbol{\tau}}
\newcommand{\Bomega}[0]{\boldsymbol{\omega}}

%
% shorthand for unit vectors
%
\newcommand{\acap}[0]{\hat{\Ba}}
\newcommand{\bcap}[0]{\hat{\Bb}}
\newcommand{\ccap}[0]{\hat{\Bc}}
\newcommand{\dcap}[0]{\hat{\Bd}}
\newcommand{\ecap}[0]{\hat{\Be}}
\newcommand{\fcap}[0]{\hat{\Bf}}
\newcommand{\gcap}[0]{\hat{\Bg}}
\newcommand{\hcap}[0]{\hat{\Bh}}
\newcommand{\icap}[0]{\hat{\Bi}}
\newcommand{\jcap}[0]{\hat{\Bj}}
\newcommand{\kcap}[0]{\hat{\Bk}}
\newcommand{\lcap}[0]{\hat{\Bl}}
\newcommand{\mcap}[0]{\hat{\Bm}}
\newcommand{\ncap}[0]{\hat{\Bn}}
\newcommand{\ocap}[0]{\hat{\Bo}}
\newcommand{\pcap}[0]{\hat{\Bp}}
\newcommand{\qcap}[0]{\hat{\Bq}}
\newcommand{\rcap}[0]{\hat{\Br}}
\newcommand{\scap}[0]{\hat{\Bs}}
\newcommand{\tcap}[0]{\hat{\Bt}}
\newcommand{\ucap}[0]{\hat{\Bu}}
\newcommand{\vcap}[0]{\hat{\Bv}}
\newcommand{\wcap}[0]{\hat{\Bw}}
\newcommand{\xcap}[0]{\hat{\Bx}}
\newcommand{\ycap}[0]{\hat{\By}}
\newcommand{\zcap}[0]{\hat{\Bz}}
\newcommand{\thetacap}[0]{\hat{\Btheta}}

%
% to write R^n and C^n in a distinguishable fashion.  Perhaps change this
% to the double lined characters upon figuring out how to do so.
%
\newcommand{\C}[1]{$\mathbb{C}^{#1}$}
\newcommand{\R}[1]{$\mathbb{R}^{#1}$}

%
% various generally useful helpers
%

% derivative of #1 wrt. #2:
\newcommand{\D}[2] {\frac {d#2} {d#1}}

\newcommand{\inv}[1]{\frac{1}{#1}}
\newcommand{\cross}[0]{\times}

\newcommand{\abs}[1]{\lvert{#1}\rvert}
\newcommand{\norm}[1]{\lVert{#1}\rVert}
\newcommand{\innerprod}[2]{\langle{#1}, {#2}\rangle}
\newcommand{\dotprod}[2]{{#1} \cdot {#2}}
\newcommand{\bdotprod}[2]{\left({#1} \cdot {#2}\right)}
\newcommand{\crossprod}[2]{{#1} \cross {#2}}
\newcommand{\tripleprod}[3]{\dotprod{\left(\crossprod{#1}{#2}\right)}{#3}}

\DeclareMathOperator{\Proj}{Proj}
\DeclareMathOperator{\Span}{span}
\DeclareMathOperator{\Sgn}{sgn}
\DeclareMathOperator{\Area}{Area}
\DeclareMathOperator{\Volume}{Volume}

%
% A few miscellaneous things specific to this document
%
\newcommand{\crossop}[1]{\crossprod{#1}{}}

% R2 vector.
\newcommand{\VectorTwo}[2]{
\begin{bmatrix}
 {#1} \\
 {#2}
\end{bmatrix}
}

\newcommand{\VectorN}[1]{
\begin{bmatrix}
{#1}_1 \\
{#1}_2 \\
\vdots \\
{#1}_N \\
\end{bmatrix}
}

\newcommand{\DETuvij}[4]{
\begin{vmatrix}
 {#1}_{#3} & {#1}_{#4} \\
 {#2}_{#3} & {#2}_{#4}
\end{vmatrix}
}

\newcommand{\DETuvwijk}[6]{
\begin{vmatrix}
 {#1}_{#4} & {#1}_{#5} & {#1}_{#6} \\
 {#2}_{#4} & {#2}_{#5} & {#2}_{#6} \\
 {#3}_{#4} & {#3}_{#5} & {#3}_{#6}
\end{vmatrix}
}

\newcommand{\DETuvwxijkl}[8]{
\begin{vmatrix}
 {#1}_{#5} & {#1}_{#6} & {#1}_{#7} & {#1}_{#8} \\
 {#2}_{#5} & {#2}_{#6} & {#2}_{#7} & {#2}_{#8} \\
 {#3}_{#5} & {#3}_{#6} & {#3}_{#7} & {#3}_{#8} \\
 {#4}_{#5} & {#4}_{#6} & {#4}_{#7} & {#4}_{#8} \\
\end{vmatrix}
}

%\newcommand{\DETuvwxyijklm}[10]{
%\begin{vmatrix}
% {#1}_{#6} & {#1}_{#7} & {#1}_{#8} & {#1}_{#9} & {#1}_{#10} \\
% {#2}_{#6} & {#2}_{#7} & {#2}_{#8} & {#2}_{#9} & {#2}_{#10} \\
% {#3}_{#6} & {#3}_{#7} & {#3}_{#8} & {#3}_{#9} & {#3}_{#10} \\
% {#4}_{#6} & {#4}_{#7} & {#4}_{#8} & {#4}_{#9} & {#4}_{#10} \\
% {#5}_{#6} & {#5}_{#7} & {#5}_{#8} & {#5}_{#9} & {#5}_{#10}
%\end{vmatrix}
%}

% R3 vector.
\newcommand{\VectorThree}[3]{
\begin{bmatrix}
 {#1} \\
 {#2} \\
 {#3}
\end{bmatrix}
}


\newcommand{\gpgrade}[2] {{\left\langle{{#1}}\right\rangle}_{#2}}
\newcommand{\gpgradeone}[1] {\gpgrade{#1}{1}}

\title{ Some notes on GAFP 5.5.3 The Lorentz force Law.} 
\author{Peeter Joot}         
\date{August 16, 2008}        

\begin{document}             

\maketitle{}

\section{}

The idea behind this derivation, is to express the vector part of the proper force in covarient form, and then
do the same for the energy change part of the proper momentum.  That first part is:

\begin{align*}
\frac{dp}{d\tau} \wedge \gamma_0 
&= \frac{d \Bp}{d\tau} \\
&= \frac{d \Bp}{dt} \frac{dt}{d\tau} \\
&= \frac{dt}{d\tau} q \left( \BE + \Bv \cross \BB \right)
\end{align*}

Now, the spacetime split of velocity is done in the normal fashion:

\begin{align*}
x &= c t \gamma_0 + \sum x^i \gamma_i \\
v &= \frac{dx}{d\tau} = c \frac{dt}{d\tau} \gamma_0 + \sum \frac{dx^i}{d\tau} \gamma_i \\
v \cdot \gamma_0 &= c \frac{dt}{d\tau} = c \gamma \\
v \wedge \gamma_0
&= \sum \frac{dx^i}{dt} \frac{dt}{d\tau} \gamma_i \gamma_0 \\
&= (v \cdot \gamma_0)/c \sum v^i \sigma_i \\
&= (v \cdot \gamma_0) \Bv/c.
\end{align*}

Writing $\dot{p} = dp/d\tau$, substituite the gamma factor into the force equation:

\begin{equation*}
\dot{p} \wedge \gamma_0 = ( v/c \cdot \gamma_0 ) q \left( \BE + \Bv \cross \BB \right)
\end{equation*}

Now, GAFP goes on to show that the $\gamma \BE$ term can be reduced to the form $(\BE \cdot v) \wedge \gamma_0$.  Their
method isn't exactly obvious, for example writing $\BE = (1/2)(\BE + \BE)$ to start.  Let's just do this backwards 
instead, expanding $\BE \cdot v$ to see the form of that term:

\begin{align*}
\BE \cdot v
&= \left(\sum E^i \gamma_{i0}\right) \cdot \left( \sum v^{\mu} \gamma_{\mu}\right) \\
&= \sum E^i v^{\mu} \gpgradeone{ \gamma_{i0\mu}} \\
&= v^0 \sum E^i \gamma_{i} + \sum E^i v^{j} \underbrace{\gpgradeone{ \gamma_{i0j}}}_{-\delta_{ij} \gamma_0} \\
&= v^0 \sum E^i \gamma_i - \sum E^i v^i \gamma_0.
\end{align*}

Wedging with $\gamma_0$ we have the desired result:

\begin{equation*}
(\BE \cdot v) \wedge \gamma_0 = v^0 \sum E^i \gamma_{i0} = (v \cdot \gamma_0) \BE = c \gamma \BE
\end{equation*}

Now, for equation 5.164 there aren't any suprising steps, but lets try this backwards too:

\begin{align*}
(I \BB) \cdot v
&= \left(\sum B^i \underbrace{\gamma_{102030i0}}_{\gamma_{123i}} \right) \cdot \left( \sum v^{\mu} \gamma_{\mu} \right) \\
&= \sum B^i v^{\mu} \gpgradeone{\gamma_{123i\mu}}
\end{align*}

That vector selection does yield the cross product as expected:

\begin{equation*}
\gpgradeone{\gamma_{123i\mu}} =
\left\{ 
\begin{array}{l l}
0 & \quad \mu = 0 \\
0 & \quad i = \mu \\
\gamma_1 & \quad i\mu = 32 \\
-\gamma_2 & \quad i\mu = 31 \\
\gamma_3 & \quad i\mu = 21 \\
\end{array} \right.
\end{equation*}

(with alternation for the missing set of index pairs).

This gives:
\begin{align*}
(I \BB) \cdot v
= (B^3 v^2 - B^2 v^3) \gamma_1
+ (B^1 v^3 - B^3 v^1) \gamma_2
+ (B^2 v^1 - B^1 v^2) \gamma_3,
\end{align*}

thus, since $v^i = \gamma d{x^i}/dt$, this yields the desired result

\begin{equation*}
((I\BB) \cdot v) \wedge \gamma_0 = \gamma \Bv \cross \BB
\end{equation*}

In retrospect, the GAFP way is clearer and easier, than to try to do it the dumb way.

Combining the results we have:

\begin{align*}
\dot{p} \wedge \gamma_0 
&= q \gamma ( \BE + \Bv \cross \BB ) \\
&= q (( \BE + c I \BB ) \cdot (v/c)) \wedge \gamma_0 \\
\end{align*}

Or with $F = \BE + c I \BB$, we have:

\begin{equation}
\dot{p} \wedge \gamma_0 = q ( F \cdot v/c ) \wedge \gamma_0
\end{equation}

It is tempting here to attempt to cancel the $\wedge \gamma_0$ parts of this equation, but that cannot be done
until one also shows:

\begin{equation*}
\dot{p} \cdot \gamma_0 = q ( F \cdot v/c ) \cdot \gamma_0
\end{equation*}

\end{document}               

\chapter{Lorentz Force Trajectory.}
\label{chap:lorentzRotation}
\date{May 7, 2008.  lorentzRotation.tex}

\section{How to solve without GA? }

%% http://www.physicsforums.com/showthread.php?t=231213

\cite{doran2003gap} treats the solution of the Lorentz force equation in covariant form.

While some simplified variants of the equation can be solved by picking an appropriate axis of symmetry, is a solution for the trajectory for a more general field pair possible without the softistication of the GA methods?

Here's a bash at starting it.  My approach essentially replays a rigid body rotation derivation backwards \href{http://www.damtp.cam.ac.uk/user/tong/dynamics/three.pdf}{such as the one found in Tong}.

In the Lorentz Force Law, we essentially have an equation like the rigid body rotation equation $y' = \omega \times y + x_0'$, that resulted from $y = R x + x_0$.  That's a good hint about the anti-derivative required for this Lorentz problem, so we have to go backwards from the cross product, and solve for the rotation:
 
\begin{equation}\label{eqn:lorentzRotation:n}
\frac{d(\Bp}{dt} = q ( \BE + \Bv \times \BB ).
\end{equation}

While $\Bp = \gamma m \Bv$, for $v \ll c$, we can write this as a matrix equation

\begin{equation}\label{eqn:lorentzRotation:n}
\Bv = \frac{q}{m} \BE + \Omega \Bv,
\end{equation}

where $\Omega$ is

\begin{equation}\label{eqn:lorentzRotation:n}
\frac{q}{m}
\begin{bmatrix}
0 & B_3 & -B_2 \\
-B_3 & 0 & B_1 \\
B_2 & -B_1 & 0 \\
\end{bmatrix}.
\end{equation}

Following the rigid body treatment (in Tong above) this antisymmetric matrix can be expressed in terms of a rotation matrix (ie: essentially it is a rotation matrix derivative with a rotation factored out of it).  So, let

\begin{equation}\label{eqn:lorentzRotation:n}
\Omega = R' R^\T,
\end{equation}

and use one more trick from the rigid body analysis:

\begin{equation}\label{eqn:lorentzRotation:n}
(RR^\T)' = R' R^\T + R {R'}^\T = I' = 0
\end{equation}

\begin{align*}
\Bv' - R' R^\T \Bv 
&= \Bv' + R {R'}^\T \Bv  \\
&= R (R^\T \Bv' + {R'}^\T \Bv)  \\
&\implies \\
R (R^\T \Bv)' &= \frac{q}{m} \BE
\end{align*}

Thus the solution can be written as two equations, one explicit for $\Bv$, and one matrix differential equation to solve for R:

\begin{equation}\label{eqn:lorentzRotation:n}
\Bv = \frac{q}{m} \BE t + R^\T \Bc
\end{equation}

\begin{equation}\label{eqn:lorentzRotation:n}
R' = \frac{q}{m}
\begin{bmatrix}
0 & B_3 & -B_2 \\
-B_3 & 0 & B_1 \\
B_2 & -B_1 & 0 \\
\end{bmatrix}
R
\end{equation}

\section{Solving for the rotation}

I haven't tried solving that last equation numerically (or analytically), but intuition says that diagonalization would do the trick.  ie: with a solution having a term of the form:

\begin{equation}\label{eqn:lorentzRotation:n}
e^{Dt}
\end{equation}

I should try this with an example to see if it all holds together (at the bare minimum, if I did things right, it should work for $\BB$ along the $\zcap$ axis;)

Now as mentioned in \cite{doran2003gap} there's a treatment of this problem in chapter 5 on spacetime algebra.  So far reading that book, I'd temporarily skipped that chapter for some easier stuff in chapter 6 (vector calculus chapter).  Going back and reading just this fragment, I can't say I fully understand their treatment.  It's interesting looking though;)  They end up reformulating the equation as:

\begin{equation}\label{eqn:lorentzRotation:n}
m v' = q F \cdot v,
\end{equation}

where $F$ is a combined electrodynamic field:

\begin{equation}\label{eqn:lorentzRotation:n}
F = \BE + I \BB,
\end{equation}

and $v = dx/d\tau$ is the four vector proper velocity.

Then they introduce a rotor parametrization of the velocity, and an equation to solve for the rotor that's similar to the rotation matrix equation I had (factor of two difference because the rotor is a double sided half angle operator unlike the single sided rotation matrix) :

\begin{equation}\label{eqn:lorentzRotation:n}
R' = \frac{q}{2m} F R
\end{equation}

There's a lot of similarities to what I hacked up, but it will probably take me a while before I can get to the point to digest and compare the two.   Their rotor equation ends up with terms for both electric field and magnetic field whereas mine is magnetic only.  Something that involves both makes more sense.

%
% Copyright � 2012 Peeter Joot.  All Rights Reserved.
% Licenced as described in the file LICENSE under the root directory of this GIT repository.
%

% 
% 
\chapter{Lorentz force rotor formulation}
\index{Lorentz force!rotor}
\label{chap:electronRotor}
%\date{March 18, 2009.  electronRotor.tex}

\section{Motivation}

Both \citep{baylis-2007} and \citep{doran2003gap} cover rotor formulations
of the Lorentz force equation.  Work through some of this on my own to 
better understand it.

\section{In terms of GA}

An active Lorentz transformation can be used to translate from the rest frame of a particle with worldline \(x\) to 
an observer frame, as in

\begin{equation}\label{eqn:eRotor:LorentzTx}
\begin{aligned}
y &= \Lambda x \reverse{\Lambda}
\end{aligned}
\end{equation}

Here Lorentz transformation is used in the general sense, and can include both spatial rotation and boost effects, but satisfies \(\Lambda\reverse{\Lambda} = 1\).  Taking proper time derivatives we have

\begin{equation}\label{eqn:electronRotor:20}
\begin{aligned}
\ydot 
&= \Lambdadot x \reverse{\Lambda} + \Lambda x \reverse{\Lambdadot} \\
&= \Lambda \left(\reverse{\Lambda} \Lambdadot\right) x \reverse{\Lambda} + \Lambda x \left(\reverse{\Lambdadot} \Lambda \right) \reverse{\Lambda} \\
\end{aligned}
\end{equation}

Since \(\reverse{\Lambda}\Lambda = \Lambda\reverse{\Lambda} = 1\) we also have

\begin{equation}\label{eqn:electronRotor:40}
\begin{aligned}
0 &= \Lambdadot\reverse{\Lambda} + \Lambda\reverse{\Lambdadot}  \\
0 &= \reverse{\Lambda}\Lambdadot + \reverse{\Lambdadot}\Lambda
\end{aligned}
\end{equation}

Here is where a bivector variable 

\begin{equation}\label{eqn:electronRotor:60}
\begin{aligned}
\Omega/2 = \reverse{\Lambda} \Lambdadot
\end{aligned}
\end{equation}

is introduced, from which we have \(\reverse{\Lambdadot} \Lambda = -\Omega/2\), and

\begin{equation}\label{eqn:electronRotor:80}
\begin{aligned}
\ydot &= \inv{2} \left( \Lambda \Omega x \reverse{\Lambda} - \Lambda x \Omega \reverse{\Lambda} \right) \\
\end{aligned}
\end{equation}

Or
\begin{equation}\label{eqn:electronRotor:100}
\begin{aligned}
\reverse{\Lambda} \ydot \Lambda &= \inv{2} \left( \Omega x - x \Omega \right) \\
\end{aligned}
\end{equation}

The inclusion of the factor of two in the definition of \(\Omega\) was cheating, so that we get the bivector vector dot product above.  Presuming \(\Omega\) is really a bivector (return to this in a bit), we then have

\begin{equation}\label{eqn:electronRotor:120}
\begin{aligned}
\reverse{\Lambda} \ydot \Lambda &= \Omega \cdot x 
\end{aligned}
\end{equation}

We can express the time evolution of \(y\) using this as a stepping stone, since we have

\begin{equation}\label{eqn:electronRotor:140}
\begin{aligned}
\reverse{\Lambda} y \Lambda &= x 
\end{aligned}
\end{equation}

Using this we have
\begin{equation}\label{eqn:electronRotor:160}
\begin{aligned}
0 
&= \gpgradeone{ \reverse{\Lambda} \ydot \Lambda - \Omega \cdot x } \\
&= \gpgradeone{ \reverse{\Lambda} \ydot \Lambda - \Omega x } \\
&= \gpgradeone{ \reverse{\Lambda} \ydot \Lambda - \Omega \reverse{\Lambda} y \Lambda } \\
&= \gpgradeone{ \left( \reverse{\Lambda} \ydot - \reverse{\Lambda} \Lambda \Omega \reverse{\Lambda} y \right) \Lambda } \\
&= \gpgradeone{ \reverse{\Lambda} \left( \ydot - \Lambda \Omega \reverse{\Lambda} y \right) \Lambda } \\
\end{aligned}
\end{equation}

So we have the complete time evolution of our observer frame worldline for the particle, as a sort of an eigenvalue 
equation for the proper time differential operator

\begin{equation}\label{eqn:electronRotor:180}
\begin{aligned}
\ydot 
&= \left( \Lambda \Omega \reverse{\Lambda} \right) \cdot y = \left( 2 \Lambdadot \reverse{\Lambda} \right) \cdot y 
\end{aligned}
\end{equation}

Now, what 
\href{http://www.ime.unicamp.br/%7Eicca8/videos/baylis.avi}{Baylis did in his lecture}, and what Doran/Lasenby did as
well in the text (but I did not understand it then when I read it the first time) was to identify this time evolution
in terms of Lorentz transform change with the Lorentz force.

Recall that the Lorentz force equation is

\begin{equation}\label{eqn:eRotor:LorentzForce}
\begin{aligned}
\vdot = \frac{e}{m c} F \cdot v
\end{aligned}
\end{equation}

where \(F = \BE + i c \BB\), like \(\Lambdadot\reverse{\Lambda}\) is also a bivector.  If we write the velocity worldline
of the particle in the lab frame in terms of the rest frame particle worldline as

\begin{equation}\label{eqn:electronRotor:200}
\begin{aligned}
v = \Lambda c t \gamma_0 \reverse{\Lambda}
\end{aligned}
\end{equation}

Then for the field \(F\) observed in the lab frame we are left with a differential equation 
\(2 \Lambdadot \reverse{\Lambda} = e F / mc\)
for the Lorentz transformation
that produces the observed motion of the particle given the field that acts on it

\begin{equation}\label{eqn:eRotor:LorentzTxEvolution}
\begin{aligned}
\Lambdadot = \frac{e}{2 m c} F \Lambda
\end{aligned}
\end{equation}

Okay, good.  I understand now well enough what they have done to reproduce the end result (with the exception of my 
result including a factor of \(c\) since they have worked with \(c=1\)).

\subsection{Omega bivector}

It has been assumed above that \(\Omega = 2 \reverse{\Lambda} \Lambdadot\) is a bivector.  One way to confirm this is by examining the grades of this product.  Two bivectors, not necessarily related can only have grades 0, 2, and 4.  Because \(\Omega = -\reverse{\Omega}\), as seen above, it can have no grade 0 or grade 4 parts.

While this is a powerful way to verify the bivector nature of this object it is fairly abstract.  To get a better feel for this, let us 
consider this object in detail for a purely spatial rotation, such as

\begin{equation}\label{eqn:electronRotor:220}
\begin{aligned}
R_\theta(x) &= \Lambda x \reverse{\Lambda} \\
\Lambda &= \exp( -i n \theta/ 2 ) = \cos( \theta/ 2 ) - i n \sin( \theta/ 2 )
\end{aligned}
\end{equation}

where \(n\) is a spatial unit bivector, \(n^2 = 1\), in the span of \(\{\sigma_k = \gamma_k \gamma_0\}\).

\subsubsection{Verify rotation form}

To verify that this has the appropriate action, by linearity two two cases must be considered.  
First is the action on \(n\) or the components of any vector in this direction.

\begin{equation}\label{eqn:electronRotor:240}
\begin{aligned}
R_\theta(n) 
&= \Lambda n \reverse{\Lambda} \\
&= \left( \cos( \theta/ 2 ) - i n \sin( \theta/ 2 ) \right) n \reverse{\Lambda} \\
&= n \left( \cos( \theta/ 2 ) - i n \sin( \theta/ 2 ) \right) \reverse{\Lambda} \\
&= n \Lambda \reverse{\Lambda} \\
&= n \\
\end{aligned}
\end{equation}

The rotation operator does not change any vector colinear with the axis of rotation (the normal).  For a 
vector \(m\) that is perpendicular to axis of rotation \(n\) (ie: \(2 ( m \cdot n ) = mn + nm = 0 \)), we have

\begin{equation}\label{eqn:electronRotor:260}
\begin{aligned}
R_\theta(m) 
&= \Lambda m \reverse{\Lambda} \\
&= \left( \cos( \theta/ 2 ) - i n \sin( \theta/ 2 ) \right) m \reverse{\Lambda} \\
&= \left( m \cos( \theta/ 2 ) - i (n m) \sin( \theta/ 2 ) \right) \reverse{\Lambda} \\
&= \left( m \cos( \theta/ 2 ) + i (m n) \sin( \theta/ 2 ) \right) \reverse{\Lambda} \\
&= m (\reverse{\Lambda})^2 \\
&= m \exp( i n \theta )
\end{aligned}
\end{equation}

This is a rotation of the vector \(m\) that lies in the \(i n\) plane by \(\theta\) as desired.

\subsubsection{The rotation bivector}

We want derivatives of the \(\Lambda\) object.

\begin{equation}\label{eqn:electronRotor:280}
\begin{aligned}
\Lambdadot 
&= \frac{\thetadot}{2} \left( -\sin( \theta/ 2 ) - i n \cos( \theta/ 2 ) \right) - i \ndot \cos(\theta/2) \\
&= \frac{i n \thetadot}{2} \left( i n \sin( \theta/ 2 ) - \cos( \theta/ 2 ) \right) - i \ndot \cos(\theta/2) \\
&= -\inv{2} \exp( -i n \theta/2 ) {i n \thetadot} - i \ndot \cos(\theta/2) \\
\end{aligned}
\end{equation}

So we have

\begin{equation}\label{eqn:electronRotor:300}
\begin{aligned}
\Omega 
&= 2 \reverse{\Lambda} \Lambdadot \\
&= -{i n \thetadot} - 2 \exp(i n \theta/2) i \ndot \cos(\theta/2) \\
&= -{i n \thetadot} - 2 \cos(\theta/2) \left( \cos(\theta/2) - i n \sin(\theta/2)  \right) i \ndot \\
&= -{i n \thetadot} - 2 \cos(\theta/2) \left( \cos(\theta/2) i \ndot + n \ndot \sin(\theta/2)  \right) \\
\end{aligned}
\end{equation}

Since \(n \cdot \ndot = 0\), we have \(n \ndot = n \wedge \ndot\), and sure enough all the terms are bivectors.  Specifically
we have

\begin{equation}\label{eqn:electronRotor:320}
\begin{aligned}
\Omega 
&= -\thetadot(i n) - (1 + \cos\theta ) (i \ndot) - \sin\theta (n \wedge \ndot)
\end{aligned}
\end{equation}

\subsection{Omega bivector for boost}

TODO.

\section{Tensor variation of the Rotor Lorentz force result}

There is not anything in the initial Lorentz force rotor result that intrinsically requires geometric algebra.  At least until
one actually
wants to express the Lorentz transformation concisely in terms of half angle or boost rapidity exponentials.

In fact
the logic above is not much different than the approach used in \citep{TongDynamics} for rigid body motion.  Let us try this in matrix or tensor
form and see how it looks.

\subsection{Tensor setup}
\index{Lorentz force!tensor}

Before anything else some notation for the tensor work must be established.  Similar to \eqnref{eqn:eRotor:LorentzTx} write a Lorentz transformed vector as a 
linear transformation.  Since we want only the matrix of this linear transformation with respect to a specific observer frame, the details
of the transformation can be omitted for now.  Write

\begin{equation}\label{eqn:electronRotor:340}
\begin{aligned}
y = \LL(x)
\end{aligned}
\end{equation}

and introduce an orthonormal frame \(\{\gamma_\mu\}\), and the corresponding reciprocal frame
\(\{\gamma^\mu\}\), where \(\gamma_\mu \cdot \gamma^\nu = {\delta_\mu}^\nu\).
In this basis, the relationship between the vectors becomes

\begin{equation}\label{eqn:electronRotor:360}
\begin{aligned}
y^\mu \gamma_\mu 
&= \LL(x^\nu \gamma_\nu) \\
&= x^\nu \LL(\gamma_\nu) \\
\end{aligned}
\end{equation}

Or
\begin{equation}\label{eqn:electronRotor:380}
\begin{aligned}
y^\mu &= x^\nu \LL(\gamma_\nu) \cdot \gamma^\mu \\
\end{aligned}
\end{equation}

The matrix of the linear transformation can now be written as

\begin{equation}\label{eqn:electronRotor:400}
\begin{aligned}
{\Lambda_\nu}^\mu &= \LL(\gamma_\nu) \cdot \gamma^\mu
\end{aligned}
\end{equation}

and this can now be used to express the coordinate transformation in abstract index notation

\begin{equation}\label{eqn:electronRotor:420}
\begin{aligned}
y^\mu &= x^\nu {\Lambda_\nu}^\mu 
\end{aligned}
\end{equation}

Similarly, for the inverse transformation, we can write

\begin{equation}\label{eqn:electronRotor:440}
\begin{aligned}
x &= \LL^{-1}(y) \\
{\ILambda_\nu}^\mu &= \LL^{-1}(\gamma_\nu) \cdot \gamma^\mu \\
x^\mu &= y^\nu {\ILambda_\nu}^\mu 
\end{aligned}
\end{equation}

I have seen this expressed using primed indices and the same symbol \(\Lambda\) used for both the forward and inverse
transformation ... lacking skill in tricky index manipulation I have avoided such a notation because I will probably get it
wrong.  Instead different symbols for the two different matrices will be used here and \(\Pi\) was picked for the inverse
rather arbitrarily.

With substitution

\begin{equation}\label{eqn:electronRotor:460}
\begin{aligned}
y^\mu &= x^\nu {\Lambda_\nu}^\mu = (y^\alpha {\ILambda_\alpha}^\nu) {\Lambda_\nu}^\mu  \\
x^\mu &= y^\nu {\ILambda_\nu}^\mu = (x^\alpha {\Lambda_\alpha}^\nu) {\ILambda_\nu}^\mu 
\end{aligned}
\end{equation}

the pair of explicit inverse relationships between the two matrices can be read off as

\begin{equation}\label{eqn:electronRotor:480}
\begin{aligned}
{\delta_\alpha}^\mu &= {\ILambda_\alpha}^\nu {\Lambda_\nu}^\mu = {\Lambda_\alpha}^\nu {\ILambda_\nu}^\mu 
\end{aligned}
\end{equation}

\subsection{Lab frame velocity of particle in tensor form}

In tensor form we want to express the worldline of the particle in the lab frame coordinates.  That is

\begin{equation}\label{eqn:electronRotor:500}
\begin{aligned}
v 
&= \LL(c t \gamma_0) \\
&= \LL(x^0 \gamma_0) \\
&= x^0 \LL(\gamma_0) \\
\end{aligned}
\end{equation}

Or
\begin{equation}\label{eqn:electronRotor:520}
\begin{aligned}
v^\mu 
&= x^0 \LL(\gamma_0) \cdot \gamma^\mu \\
&= x^0 {\Lambda_0}^\mu
\end{aligned}
\end{equation}

\subsection{Lorentz force in tensor form}

The Lorentz force equation \eqnref{eqn:eRotor:LorentzForce} in tensor form will also be needed.  The bivector \(F\) is

\begin{equation}\label{eqn:electronRotor:540}
\begin{aligned}
F = \inv{2} F_{\mu\nu} \gamma^\mu \wedge \gamma^\nu
\end{aligned}
\end{equation}

So we can write

\begin{equation}\label{eqn:electronRotor:560}
\begin{aligned}
F \cdot v 
&= \inv{2} F_{\mu\nu} (\gamma^\mu \wedge \gamma^\nu) \cdot \gamma_\alpha v^\alpha \\
&= \inv{2} F_{\mu\nu} (\gamma^\mu {\delta^\nu}_\alpha - \gamma^\nu {\delta^\mu}_\alpha) v^\alpha \\
&= \inv{2} (v^\alpha F_{\mu\alpha} \gamma^\mu -v^\alpha F_{\alpha\nu} \gamma^\nu )
\end{aligned}
\end{equation}

And
\begin{equation}\label{eqn:electronRotor:580}
\begin{aligned}
\vdot_\sigma 
&= \frac{e}{m c} ( F \cdot v ) \cdot \gamma_\sigma \\
&= \frac{e}{2 m c} (v^\alpha F_{\mu\alpha} \gamma^\mu -v^\alpha F_{\alpha\nu} \gamma^\nu ) \cdot \gamma_\sigma \\
&= \frac{e}{2 m c} v^\alpha ( F_{\sigma\alpha} - F_{\alpha\sigma} ) \\
&= \frac{e}{m c} v^\alpha F_{\sigma\alpha} \\
\end{aligned}
\end{equation}

Or

\begin{equation}\label{eqn:electronRotor:600}
\begin{aligned}
\vdot^\sigma &= \frac{e}{m c} v^\alpha {F^\sigma}_\alpha 
\end{aligned}
\end{equation}

\subsection{Evolution of Lab frame vector}

Given a lab frame vector with all the (proper) time evolution expressed via the Lorentz transformation

\begin{equation}\label{eqn:electronRotor:620}
\begin{aligned}
y^\mu 
&= x^\nu {\Lambda_\nu}^\mu \\
\end{aligned}
\end{equation}

we want to calculate the derivatives as in the GA procedure

\begin{equation}\label{eqn:electronRotor:640}
\begin{aligned}
\ydot^\mu 
&= x^\nu {\Lambdadot_\nu}^\mu \\
&= x^\alpha {\delta_\alpha}^\nu {\Lambdadot_\nu}^\mu \\
&= x^\alpha {\Lambda_\alpha}^\beta {\ILambda_\beta}^\nu {\Lambdadot_\nu}^\mu \\
\end{aligned}
\end{equation}

With \(y = v\), this is

\begin{equation}\label{eqn:electronRotor:660}
\begin{aligned}
\vdot^\sigma 
&= v^\alpha {\ILambda_\alpha}^\nu {\Lambdadot_\nu}^\sigma \\
&= v^\alpha \frac{e}{m c} {F^\sigma}_\alpha 
\end{aligned}
\end{equation}

So we can make the identification of the bivector field with the Lorentz transformation matrix

\begin{equation}\label{eqn:electronRotor:680}
\begin{aligned}
{\ILambda_\alpha}^\nu {\Lambdadot_\nu}^\sigma &= \frac{e}{m c} {F^\sigma}_\alpha 
\end{aligned}
\end{equation}

With an additional summation to invert we have
\begin{equation}\label{eqn:electronRotor:700}
\begin{aligned}
{\Lambda_\beta}^\alpha {\ILambda_\alpha}^\nu {\Lambdadot_\nu}^\sigma &= {\Lambda_\beta}^\alpha \frac{e}{m c} {F^\sigma}_\alpha 
\end{aligned}
\end{equation}
%{\delta_\beta}^\nu &= {\ILambda_\beta}^\alpha {\Lambda_\alpha}^\nu = {\Lambda_\beta}^\alpha {\ILambda_\alpha}^\nu 

This leaves a tensor differential equation that will provide the complete time evolution of the lab frame worldline for the particle in the field

\begin{equation}\label{eqn:electronRotor:720}
\begin{aligned}
{{\Lambdadot}_\mu}^\nu &= \frac{e}{m c} {\Lambda_\mu}^\alpha {F^\nu}_\alpha 
\end{aligned}
\end{equation}

This is the equivalent of the GA equation \eqnref{eqn:eRotor:LorentzTxEvolution}.  However, while the GA equation is directly integrable for constant \(F\), how to do this in the equivalent tensor formulation is not so clear.

Want to revisit this, and try to perform this integral in both forms, ideally
for both the simpler constant field case, as well as for a more general field.
Even better would be to be able to express \(F\) in terms of the current 
density vector, and then treat the proper interaction of two charged particles.

\section{Gauge transformation for spin}

In the Baylis article \eqnref{eqn:eRotor:LorentzTxEvolution} is transformed as
\(\Lambda \rightarrow \Lambda_{\omega_0} \exp( -i \Be_3 \omega_0 \tau)\).

Using this we have

\begin{equation}\label{eqn:electronRotor:740}
\begin{aligned}
\Lambdadot 
&\rightarrow \frac{d}{d\tau}\left(\Lambda_{\omega_0} \exp( -i \Be_3 \omega_0 \tau) \right) \\
&= \Lambdadot_{\omega_0} \exp( -i \Be_3 \omega_0 \tau) 
- \Lambda_{\omega_0} ( i \Be_3 \omega_0 ) \exp( -i \Be_3 \omega_0 \tau) 
\end{aligned}
\end{equation}

For the transformed \eqnref{eqn:eRotor:LorentzTxEvolution} this gives

\begin{equation}\label{eqn:electronRotor:760}
\begin{aligned}
\Lambdadot_{\omega_0} \exp( -i \Be_3 \omega_0 \tau) 
- \Lambda_{\omega_0} ( i \Be_3 \omega_0 ) \exp( -i \Be_3 \omega_0 \tau) 
&= \frac{e}{2 m c} F \Lambda_{\omega_0} \exp( -i \Be_3 \omega_0 \tau)
\end{aligned}
\end{equation}

Canceling the exponentials, and shuffling

\begin{equation}\label{eqn:eRotor:firstTry}
\begin{aligned}
\Lambdadot_{\omega_0} &= \frac{e}{2 m c} F \Lambda_{\omega_0} + \Lambda_{\omega_0} ( i \Be_3 \omega_0 ) 
\end{aligned}
\end{equation}

How does he commute the \(i\Be_3\) term with the Lorentz transform?  How about instead 
transforming as
\(\Lambda \rightarrow \exp( -i \Be_3 \omega_0 \tau) \Lambda_{\omega_0}\).

Using this we have

\begin{equation}\label{eqn:electronRotor:780}
\begin{aligned}
\Lambdadot 
&\rightarrow \frac{d}{d\tau}\left(
\exp( -i \Be_3 \omega_0 \tau) 
\Lambda_{\omega_0} 
\right) \\
&= 
\exp( -i \Be_3 \omega_0 \tau) 
\Lambdadot_{\omega_0} 
- 
( i \Be_3 \omega_0 ) \exp( -i \Be_3 \omega_0 \tau) 
\Lambda_{\omega_0} 
\end{aligned}
\end{equation}

then, the transformed \eqnref{eqn:eRotor:LorentzTxEvolution} gives

\begin{equation}\label{eqn:electronRotor:800}
\begin{aligned}
\exp( -i \Be_3 \omega_0 \tau) 
\Lambdadot_{\omega_0} 
- 
( i \Be_3 \omega_0 ) \exp( -i \Be_3 \omega_0 \tau) 
\Lambda_{\omega_0} 
&= \frac{e}{2 m c} F 
\exp( -i \Be_3 \omega_0 \tau)
\Lambda_{\omega_0} 
\end{aligned}
\end{equation}

Multiplying by the inverse exponential, and shuffling, noting that \(\exp(i\Be_3\alpha)\) commutes with \(i\Be_3\), we have

\begin{equation}\label{eqn:electronRotor:820}
\begin{aligned}
\Lambdadot_{\omega_0} 
&= 
( i \Be_3 \omega_0 ) \Lambda_{\omega_0} 
+ \frac{e}{2 m c} 
\exp( i \Be_3 \omega_0 \tau) 
F 
\exp( -i \Be_3 \omega_0 \tau)
\Lambda_{\omega_0}  \\
&=
\frac{e}{2 m c} \left(
\frac{2 m c}{e} ( i \Be_3 \omega_0 ) 
+ 
\exp( i \Be_3 \omega_0 \tau) 
F 
\exp( -i \Be_3 \omega_0 \tau)
\right)
\Lambda_{\omega_0} 
\end{aligned}
\end{equation}

So, if one writes \(F_{\omega_0} = \exp( i \Be_3 \omega_0 \tau) F \exp( -i \Be_3 \omega_0 \tau)\), then
the transformed differential equation for the Lorentz transformation takes the form 

\begin{equation}\label{eqn:electronRotor:840}
\begin{aligned}
\Lambdadot_{\omega_0}
&=
\frac{e}{2 m c} \left(
\frac{2 m c}{e} ( i \Be_3 \omega_0 ) 
+ 
F_{\omega_0}
\right)
\Lambda_{\omega_0} 
\end{aligned}
\end{equation}

This is closer to Baylis's equation 31.
Dropping \(\omega_0\) subscripts this is

\begin{equation}\label{eqn:electronRotor:860}
\begin{aligned}
\Lambdadot
&=
\frac{e}{2 m c} \left(
\frac{2 m c}{e} ( i \Be_3 \omega_0 ) 
+ 
F
\right)
\Lambda
\end{aligned}
\end{equation}

A phase change in the Lorentz transformation rotor has introduced an additional term, one that 
Baylis appears to identify with the spin vector \(\BS\).  My way of getting there seems fishy, so I think that 
I am missing something.

Ah, I see.  If we go back to \eqnref{eqn:eRotor:firstTry}, then with 
\(\BS = \Lambda_{\omega_0} ( i \Be_3 ) \reverse{\Lambda}_{\omega_0}\) (an application of a Lorentz transform to the unit bivector for the \(\Be_2 \Be_3\) plane), one has

\begin{equation}\label{eqn:electronRotor:880}
\begin{aligned}
\Lambdadot_{\omega_0} 
&= \inv{2} \left( \frac{e}{m c} F + 2 \omega_0 \BS \right) \Lambda_{\omega_0} 
\end{aligned}
\end{equation}

\part{Electrodynamics Stress Energy.}
\documentclass{article}

\usepackage{amsmath}
\usepackage{mathpazo}

%
% shorthand for bold symbols, convenient for vectors and matrices
%
\newcommand{\Ba}[0]{\mathbf{a}}
\newcommand{\Bb}[0]{\mathbf{b}}
\newcommand{\Bc}[0]{\mathbf{c}}
\newcommand{\Bd}[0]{\mathbf{d}}
\newcommand{\Be}[0]{\mathbf{e}}
\newcommand{\Bf}[0]{\mathbf{f}}
\newcommand{\Bg}[0]{\mathbf{g}}
\newcommand{\Bh}[0]{\mathbf{h}}
\newcommand{\Bi}[0]{\mathbf{i}}
\newcommand{\Bj}[0]{\mathbf{j}}
\newcommand{\Bk}[0]{\mathbf{k}}
\newcommand{\Bl}[0]{\mathbf{l}}
\newcommand{\Bm}[0]{\mathbf{m}}
\newcommand{\Bn}[0]{\mathbf{n}}
\newcommand{\Bo}[0]{\mathbf{o}}
\newcommand{\Bp}[0]{\mathbf{p}}
\newcommand{\Bq}[0]{\mathbf{q}}
\newcommand{\Br}[0]{\mathbf{r}}
\newcommand{\Bs}[0]{\mathbf{s}}
\newcommand{\Bt}[0]{\mathbf{t}}
\newcommand{\Bu}[0]{\mathbf{u}}
\newcommand{\Bv}[0]{\mathbf{v}}
\newcommand{\Bw}[0]{\mathbf{w}}
\newcommand{\Bx}[0]{\mathbf{x}}
\newcommand{\By}[0]{\mathbf{y}}
\newcommand{\Bz}[0]{\mathbf{z}}
\newcommand{\BA}[0]{\mathbf{A}}
\newcommand{\BB}[0]{\mathbf{B}}
\newcommand{\BC}[0]{\mathbf{C}}
\newcommand{\BD}[0]{\mathbf{D}}
\newcommand{\BE}[0]{\mathbf{E}}
\newcommand{\BF}[0]{\mathbf{F}}
\newcommand{\BG}[0]{\mathbf{G}}
\newcommand{\BH}[0]{\mathbf{H}}
\newcommand{\BI}[0]{\mathbf{I}}
\newcommand{\BJ}[0]{\mathbf{J}}
\newcommand{\BK}[0]{\mathbf{K}}
\newcommand{\BL}[0]{\mathbf{L}}
\newcommand{\BM}[0]{\mathbf{M}}
\newcommand{\BN}[0]{\mathbf{N}}
\newcommand{\BO}[0]{\mathbf{O}}
\newcommand{\BP}[0]{\mathbf{P}}
\newcommand{\BQ}[0]{\mathbf{Q}}
\newcommand{\BR}[0]{\mathbf{R}}
\newcommand{\BS}[0]{\mathbf{S}}
\newcommand{\BT}[0]{\mathbf{T}}
\newcommand{\BU}[0]{\mathbf{U}}
\newcommand{\BV}[0]{\mathbf{V}}
\newcommand{\BW}[0]{\mathbf{W}}
\newcommand{\BX}[0]{\mathbf{X}}
\newcommand{\BY}[0]{\mathbf{Y}}
\newcommand{\BZ}[0]{\mathbf{Z}}

\newcommand{\Bzero}[0]{\mathbf{0}}
\newcommand{\Btheta}[0]{\boldsymbol{\theta}}
\newcommand{\Btau}[0]{\boldsymbol{\tau}}
\newcommand{\Bomega}[0]{\boldsymbol{\omega}}

%
% shorthand for unit vectors
%
\newcommand{\acap}[0]{\hat{\Ba}}
\newcommand{\bcap}[0]{\hat{\Bb}}
\newcommand{\ccap}[0]{\hat{\Bc}}
\newcommand{\dcap}[0]{\hat{\Bd}}
\newcommand{\ecap}[0]{\hat{\Be}}
\newcommand{\fcap}[0]{\hat{\Bf}}
\newcommand{\gcap}[0]{\hat{\Bg}}
\newcommand{\hcap}[0]{\hat{\Bh}}
\newcommand{\icap}[0]{\hat{\Bi}}
\newcommand{\jcap}[0]{\hat{\Bj}}
\newcommand{\kcap}[0]{\hat{\Bk}}
\newcommand{\lcap}[0]{\hat{\Bl}}
\newcommand{\mcap}[0]{\hat{\Bm}}
\newcommand{\ncap}[0]{\hat{\Bn}}
\newcommand{\ocap}[0]{\hat{\Bo}}
\newcommand{\pcap}[0]{\hat{\Bp}}
\newcommand{\qcap}[0]{\hat{\Bq}}
\newcommand{\rcap}[0]{\hat{\Br}}
\newcommand{\scap}[0]{\hat{\Bs}}
\newcommand{\tcap}[0]{\hat{\Bt}}
\newcommand{\ucap}[0]{\hat{\Bu}}
\newcommand{\vcap}[0]{\hat{\Bv}}
\newcommand{\wcap}[0]{\hat{\Bw}}
\newcommand{\xcap}[0]{\hat{\Bx}}
\newcommand{\ycap}[0]{\hat{\By}}
\newcommand{\zcap}[0]{\hat{\Bz}}
\newcommand{\thetacap}[0]{\hat{\Btheta}}

%
% to write R^n and C^n in a distinguishable fashion.  Perhaps change this
% to the double lined characters upon figuring out how to do so.
%
\newcommand{\C}[1]{$\mathbb{C}^{#1}$}
\newcommand{\R}[1]{$\mathbb{R}^{#1}$}

%
% various generally useful helpers
%

% derivative of #1 wrt. #2:
\newcommand{\D}[2] {\frac {d#2} {d#1}}

\newcommand{\inv}[1]{\frac{1}{#1}}
\newcommand{\cross}[0]{\times}

\newcommand{\abs}[1]{\lvert{#1}\rvert}
\newcommand{\norm}[1]{\lVert{#1}\rVert}
\newcommand{\innerprod}[2]{\langle{#1}, {#2}\rangle}
\newcommand{\dotprod}[2]{{#1} \cdot {#2}}
\newcommand{\bdotprod}[2]{\left({#1} \cdot {#2}\right)}
\newcommand{\crossprod}[2]{{#1} \cross {#2}}
\newcommand{\tripleprod}[3]{\dotprod{\left(\crossprod{#1}{#2}\right)}{#3}}

\DeclareMathOperator{\Proj}{Proj}
\DeclareMathOperator{\Span}{span}
\DeclareMathOperator{\Sgn}{sgn}
\DeclareMathOperator{\Area}{Area}
\DeclareMathOperator{\Volume}{Volume}

%
% A few miscellaneous things specific to this document
%
\newcommand{\crossop}[1]{\crossprod{#1}{}}

% R2 vector.
\newcommand{\VectorTwo}[2]{
\begin{bmatrix}
 {#1} \\
 {#2}
\end{bmatrix}
}

\newcommand{\VectorN}[1]{
\begin{bmatrix}
{#1}_1 \\
{#1}_2 \\
\vdots \\
{#1}_N \\
\end{bmatrix}
}

\newcommand{\DETuvij}[4]{
\begin{vmatrix}
 {#1}_{#3} & {#1}_{#4} \\
 {#2}_{#3} & {#2}_{#4}
\end{vmatrix}
}

\newcommand{\DETuvwijk}[6]{
\begin{vmatrix}
 {#1}_{#4} & {#1}_{#5} & {#1}_{#6} \\
 {#2}_{#4} & {#2}_{#5} & {#2}_{#6} \\
 {#3}_{#4} & {#3}_{#5} & {#3}_{#6}
\end{vmatrix}
}

\newcommand{\DETuvwxijkl}[8]{
\begin{vmatrix}
 {#1}_{#5} & {#1}_{#6} & {#1}_{#7} & {#1}_{#8} \\
 {#2}_{#5} & {#2}_{#6} & {#2}_{#7} & {#2}_{#8} \\
 {#3}_{#5} & {#3}_{#6} & {#3}_{#7} & {#3}_{#8} \\
 {#4}_{#5} & {#4}_{#6} & {#4}_{#7} & {#4}_{#8} \\
\end{vmatrix}
}

%\newcommand{\DETuvwxyijklm}[10]{
%\begin{vmatrix}
% {#1}_{#6} & {#1}_{#7} & {#1}_{#8} & {#1}_{#9} & {#1}_{#10} \\
% {#2}_{#6} & {#2}_{#7} & {#2}_{#8} & {#2}_{#9} & {#2}_{#10} \\
% {#3}_{#6} & {#3}_{#7} & {#3}_{#8} & {#3}_{#9} & {#3}_{#10} \\
% {#4}_{#6} & {#4}_{#7} & {#4}_{#8} & {#4}_{#9} & {#4}_{#10} \\
% {#5}_{#6} & {#5}_{#7} & {#5}_{#8} & {#5}_{#9} & {#5}_{#10}
%\end{vmatrix}
%}

% R3 vector.
\newcommand{\VectorThree}[3]{
\begin{bmatrix}
 {#1} \\
 {#2} \\
 {#3}
\end{bmatrix}
}


%<misc>
%
\newcommand{\Abs}[1]{{\left\lvert{#1}\right\rvert}}
\newcommand{\spacegrad}[0]{\boldsymbol{\nabla}}
\newcommand{\grad}[0]{\nabla}
\newcommand{\LL}[0]{\mathcal{L}}

% == \partial_{#1} {#2}
\newcommand{\PD}[2]{\frac{\partial {#2}}{\partial {#1}}}
% inline variant
\newcommand{\PDi}[2]{{\partial {#2}}/{\partial {#1}}}

\newcommand{\PDD}[3]{\frac{\partial^2 {#3}}{\partial {#1}\partial {#2}}}
%\newcommand{\PDd}[2]{\frac{\partial^2 {#2}}{{\partial{#1}}^2}}
\newcommand{\PDsq}[2]{\frac{\partial^2 {#2}}{(\partial {#1})^2}}

\newcommand{\Partial}[2]{\frac{\partial {#1}}{\partial {#2}}}
\DeclareMathOperator{\RejName}{Rej}
\newcommand{\Rej}[2]{\RejName_{#1}\left( {#2} \right)}
\newcommand{\Rm}[1]{\mathbb{R}^{#1}}
\newcommand{\Cm}[1]{\mathbb{C}^{#1}}
\newcommand{\conj}[0]{{*}}

%</misc>

% <grade selection>
%
\newcommand{\gpgrade}[2] {{\left\langle{{#1}}\right\rangle}_{#2}}

\newcommand{\gpgradezero}[1] {\gpgrade{#1}{}}
%\newcommand{\gpscalargrade}[1] {{\left\langle{{#1}}\right\rangle}}
%\newcommand{\gpgradezero}[1] {\gpgrade{#1}{0}}

%\newcommand{\gpgradeone}[1] {{\left\langle{{#1}}\right\rangle}_{1}}
\newcommand{\gpgradeone}[1] {\gpgrade{#1}{1}}

\newcommand{\gpgradetwo}[1] {\gpgrade{#1}{2}}
\newcommand{\gpgradethree}[1] {\gpgrade{#1}{3}}
\newcommand{\gpgradefour}[1] {\gpgrade{#1}{4}}
%
% </grade selection>



\newcommand{\adot}[0]{{\dot{a}}}
\newcommand{\bdot}[0]{{\dot{b}}}
% taken for centered dot:
%\newcommand{\cdot}[0]{{\dot{c}}}
%\newcommand{\ddot}[0]{{\dot{d}}}
\newcommand{\edot}[0]{{\dot{e}}}
\newcommand{\fdot}[0]{{\dot{f}}}
\newcommand{\gdot}[0]{{\dot{g}}}
\newcommand{\hdot}[0]{{\dot{h}}}
\newcommand{\idot}[0]{{\dot{i}}}
\newcommand{\jdot}[0]{{\dot{j}}}
\newcommand{\kdot}[0]{{\dot{k}}}
\newcommand{\ldot}[0]{{\dot{l}}}
\newcommand{\mdot}[0]{{\dot{m}}}
\newcommand{\ndot}[0]{{\dot{n}}}
%\newcommand{\odot}[0]{{\dot{o}}}
\newcommand{\pdot}[0]{{\dot{p}}}
\newcommand{\qdot}[0]{{\dot{q}}}
\newcommand{\rdot}[0]{{\dot{r}}}
\newcommand{\sdot}[0]{{\dot{s}}}
\newcommand{\tdot}[0]{{\dot{t}}}
\newcommand{\udot}[0]{{\dot{u}}}
\newcommand{\vdot}[0]{{\dot{v}}}
\newcommand{\wdot}[0]{{\dot{w}}}
\newcommand{\xdot}[0]{{\dot{x}}}
\newcommand{\ydot}[0]{{\dot{y}}}
\newcommand{\zdot}[0]{{\dot{z}}}
\newcommand{\addot}[0]{{\ddot{a}}}
\newcommand{\bddot}[0]{{\ddot{b}}}
\newcommand{\cddot}[0]{{\ddot{c}}}
%\newcommand{\dddot}[0]{{\ddot{d}}}
\newcommand{\eddot}[0]{{\ddot{e}}}
\newcommand{\fddot}[0]{{\ddot{f}}}
\newcommand{\gddot}[0]{{\ddot{g}}}
\newcommand{\hddot}[0]{{\ddot{h}}}
\newcommand{\iddot}[0]{{\ddot{i}}}
\newcommand{\jddot}[0]{{\ddot{j}}}
\newcommand{\kddot}[0]{{\ddot{k}}}
\newcommand{\lddot}[0]{{\ddot{l}}}
\newcommand{\mddot}[0]{{\ddot{m}}}
\newcommand{\nddot}[0]{{\ddot{n}}}
\newcommand{\oddot}[0]{{\ddot{o}}}
\newcommand{\pddot}[0]{{\ddot{p}}}
\newcommand{\qddot}[0]{{\ddot{q}}}
\newcommand{\rddot}[0]{{\ddot{r}}}
\newcommand{\sddot}[0]{{\ddot{s}}}
\newcommand{\tddot}[0]{{\ddot{t}}}
\newcommand{\uddot}[0]{{\ddot{u}}}
\newcommand{\vddot}[0]{{\ddot{v}}}
\newcommand{\wddot}[0]{{\ddot{w}}}
\newcommand{\xddot}[0]{{\ddot{x}}}
\newcommand{\yddot}[0]{{\ddot{y}}}
\newcommand{\zddot}[0]{{\ddot{z}}}

%<bold and dot greek symbols>
%

\newcommand{\Deltadot}[0]{{\dot{\Delta}}}
\newcommand{\Gammadot}[0]{{\dot{\Gamma}}}
\newcommand{\Lambdadot}[0]{{\dot{\Lambda}}}
\newcommand{\Omegadot}[0]{{\dot{\Omega}}}
\newcommand{\Phidot}[0]{{\dot{\Phi}}}
\newcommand{\Pidot}[0]{{\dot{\Pi}}}
\newcommand{\Psidot}[0]{{\dot{\Psi}}}
\newcommand{\Sigmadot}[0]{{\dot{\Sigma}}}
\newcommand{\Thetadot}[0]{{\dot{\Theta}}}
\newcommand{\Upsilondot}[0]{{\dot{\Upsilon}}}
\newcommand{\Xidot}[0]{{\dot{\Xi}}}
\newcommand{\alphadot}[0]{{\dot{\alpha}}}
\newcommand{\betadot}[0]{{\dot{\beta}}}
\newcommand{\chidot}[0]{{\dot{\chi}}}
\newcommand{\deltadot}[0]{{\dot{\delta}}}
\newcommand{\epsilondot}[0]{{\dot{\epsilon}}}
\newcommand{\etadot}[0]{{\dot{\eta}}}
\newcommand{\gammadot}[0]{{\dot{\gamma}}}
\newcommand{\kappadot}[0]{{\dot{\kappa}}}
\newcommand{\lambdadot}[0]{{\dot{\lambda}}}
\newcommand{\mudot}[0]{{\dot{\mu}}}
\newcommand{\nudot}[0]{{\dot{\nu}}}
\newcommand{\omegadot}[0]{{\dot{\omega}}}
\newcommand{\phidot}[0]{{\dot{\phi}}}
\newcommand{\pidot}[0]{{\dot{\pi}}}
\newcommand{\psidot}[0]{{\dot{\psi}}}
\newcommand{\rhodot}[0]{{\dot{\rho}}}
\newcommand{\sigmadot}[0]{{\dot{\sigma}}}
\newcommand{\taudot}[0]{{\dot{\tau}}}
\newcommand{\thetadot}[0]{{\dot{\theta}}}
\newcommand{\upsilondot}[0]{{\dot{\upsilon}}}
\newcommand{\varepsilondot}[0]{{\dot{\varepsilon}}}
\newcommand{\varphidot}[0]{{\dot{\varphi}}}
\newcommand{\varpidot}[0]{{\dot{\varpi}}}
\newcommand{\varrhodot}[0]{{\dot{\varrho}}}
\newcommand{\varsigmadot}[0]{{\dot{\varsigma}}}
\newcommand{\varthetadot}[0]{{\dot{\vartheta}}}
\newcommand{\xidot}[0]{{\dot{\xi}}}
\newcommand{\zetadot}[0]{{\dot{\zeta}}}

\newcommand{\Deltaddot}[0]{{\ddot{\Delta}}}
\newcommand{\Gammaddot}[0]{{\ddot{\Gamma}}}
\newcommand{\Lambdaddot}[0]{{\ddot{\Lambda}}}
\newcommand{\Omegaddot}[0]{{\ddot{\Omega}}}
\newcommand{\Phiddot}[0]{{\ddot{\Phi}}}
\newcommand{\Piddot}[0]{{\ddot{\Pi}}}
\newcommand{\Psiddot}[0]{{\ddot{\Psi}}}
\newcommand{\Sigmaddot}[0]{{\ddot{\Sigma}}}
\newcommand{\Thetaddot}[0]{{\ddot{\Theta}}}
\newcommand{\Upsilonddot}[0]{{\ddot{\Upsilon}}}
\newcommand{\Xiddot}[0]{{\ddot{\Xi}}}
\newcommand{\alphaddot}[0]{{\ddot{\alpha}}}
\newcommand{\betaddot}[0]{{\ddot{\beta}}}
\newcommand{\chiddot}[0]{{\ddot{\chi}}}
\newcommand{\deltaddot}[0]{{\ddot{\delta}}}
\newcommand{\epsilonddot}[0]{{\ddot{\epsilon}}}
\newcommand{\etaddot}[0]{{\ddot{\eta}}}
\newcommand{\gammaddot}[0]{{\ddot{\gamma}}}
\newcommand{\kappaddot}[0]{{\ddot{\kappa}}}
\newcommand{\lambdaddot}[0]{{\ddot{\lambda}}}
\newcommand{\muddot}[0]{{\ddot{\mu}}}
\newcommand{\nuddot}[0]{{\ddot{\nu}}}
\newcommand{\omegaddot}[0]{{\ddot{\omega}}}
\newcommand{\phiddot}[0]{{\ddot{\phi}}}
\newcommand{\piddot}[0]{{\ddot{\pi}}}
\newcommand{\psiddot}[0]{{\ddot{\psi}}}
\newcommand{\rhoddot}[0]{{\ddot{\rho}}}
\newcommand{\sigmaddot}[0]{{\ddot{\sigma}}}
\newcommand{\tauddot}[0]{{\ddot{\tau}}}
\newcommand{\thetaddot}[0]{{\ddot{\theta}}}
\newcommand{\upsilonddot}[0]{{\ddot{\upsilon}}}
\newcommand{\varepsilonddot}[0]{{\ddot{\varepsilon}}}
\newcommand{\varphiddot}[0]{{\ddot{\varphi}}}
\newcommand{\varpiddot}[0]{{\ddot{\varpi}}}
\newcommand{\varrhoddot}[0]{{\ddot{\varrho}}}
\newcommand{\varsigmaddot}[0]{{\ddot{\varsigma}}}
\newcommand{\varthetaddot}[0]{{\ddot{\vartheta}}}
\newcommand{\xiddot}[0]{{\ddot{\xi}}}
\newcommand{\zetaddot}[0]{{\ddot{\zeta}}}

\newcommand{\BDelta}[0]{\boldsymbol{\Delta}}
\newcommand{\BGamma}[0]{\boldsymbol{\Gamma}}
\newcommand{\BLambda}[0]{\boldsymbol{\Lambda}}
\newcommand{\BOmega}[0]{\boldsymbol{\Omega}}
\newcommand{\BPhi}[0]{\boldsymbol{\Phi}}
\newcommand{\BPi}[0]{\boldsymbol{\Pi}}
\newcommand{\BPsi}[0]{\boldsymbol{\Psi}}
\newcommand{\BSigma}[0]{\boldsymbol{\Sigma}}
\newcommand{\BTheta}[0]{\boldsymbol{\Theta}}
\newcommand{\BUpsilon}[0]{\boldsymbol{\Upsilon}}
\newcommand{\BXi}[0]{\boldsymbol{\Xi}}
\newcommand{\Balpha}[0]{\boldsymbol{\alpha}}
\newcommand{\Bbeta}[0]{\boldsymbol{\beta}}
\newcommand{\Bchi}[0]{\boldsymbol{\chi}}
\newcommand{\Bdelta}[0]{\boldsymbol{\delta}}
\newcommand{\Bepsilon}[0]{\boldsymbol{\epsilon}}
\newcommand{\Beta}[0]{\boldsymbol{\eta}}
\newcommand{\Bgamma}[0]{\boldsymbol{\gamma}}
\newcommand{\Bkappa}[0]{\boldsymbol{\kappa}}
\newcommand{\Blambda}[0]{\boldsymbol{\lambda}}
\newcommand{\Bmu}[0]{\boldsymbol{\mu}}
\newcommand{\Bnu}[0]{\boldsymbol{\nu}}
%\newcommand{\Bomega}[0]{\boldsymbol{\omega}}
\newcommand{\Bphi}[0]{\boldsymbol{\phi}}
\newcommand{\Bpi}[0]{\boldsymbol{\pi}}
\newcommand{\Bpsi}[0]{\boldsymbol{\psi}}
\newcommand{\Brho}[0]{\boldsymbol{\rho}}
\newcommand{\Bsigma}[0]{\boldsymbol{\sigma}}
%\newcommand{\Btau}[0]{\boldsymbol{\tau}}
%\newcommand{\Btheta}[0]{\boldsymbol{\theta}}
\newcommand{\Bupsilon}[0]{\boldsymbol{\upsilon}}
\newcommand{\Bvarepsilon}[0]{\boldsymbol{\varepsilon}}
\newcommand{\Bvarphi}[0]{\boldsymbol{\varphi}}
\newcommand{\Bvarpi}[0]{\boldsymbol{\varpi}}
\newcommand{\Bvarrho}[0]{\boldsymbol{\varrho}}
\newcommand{\Bvarsigma}[0]{\boldsymbol{\varsigma}}
\newcommand{\Bvartheta}[0]{\boldsymbol{\vartheta}}
\newcommand{\Bxi}[0]{\boldsymbol{\xi}}
\newcommand{\Bzeta}[0]{\boldsymbol{\zeta}}
%
%</bold and dot greek symbols>
%<infrequent>
%
%\newcommand{\AreaOp}[1]{\AName_{#1}}
%\newcommand{\Babs}[0]{\abs{\BB}}
%\newcommand{\Bcap}[0]{\hat{\BB}}
%\newcommand{\BrPrimeRej}[0]{\rcap(\rcap \wedge \Br')}
%\newcommand{\CA}[0]{\mathcal{A}}
%\newcommand{\Cos}[1]{\cos{\left({#1}\right)}}
%\newcommand{\Det}[1] {\abs{#1}}
%\newcommand{\Dsq}[2] {\frac {\partial^2 {#1}} {\partial {#2}^2}}
%\newcommand{\Exp}[1]{\exp{\left({#1}\right)}}
%\newcommand{\Norm}[1]{\left\lVert{#1}\right\rVert}
%\newcommand{\Sin}[1]{\sin{\left({#1}\right)}}
%\newcommand{\T}[0]{\text{T}}
%\newcommand{\VolumeOp}[1]{\VName_{#1}}
%\newcommand{\agrad}[0]{\Ba \cdot \nabla}
%\newcommand{\alphacap}[0]{\hat{\boldsymbol{\alpha}}}
%\newcommand{\Fcap}[0]{\hat{\BF}}
%\newcommand{\bithree}[0]{{\Bi}_3}
%\newcommand{\bxa}[0]{\Bx\Ba}
%\newcommand{\coordvec}[2]{
%\newcommand{\costheta}[0]{\acap \cdot \xcap}
%\newcommand{\ddt}[1]{\ddot{#1}}
%\newcommand{\ddu}[1] {\frac {d{#1}} {du}}
%\newcommand{\dsqxj}[2] {\frac {\partial^2 {#1}} {\partial {x_{#2}}^2}}
%\newcommand{\dtheta}[1]{\frac{d {#1}}{d \theta}}
%\newcommand{\dt}[1]{\dot{#1}}
%\newcommand{\dt}[1]{\frac{d {#1}}{dt}}
%\newcommand{\dxj}[2] {\frac {\partial {#1}} {\partial {x_{#2}}}}
%\newcommand{\halfPhi}[0]{\frac{\phi}{2}}
%\newcommand{\half}[0]{\inv{2}}
%\newcommand{\inv}[1]{\frac{1}{#1}}
%\newcommand{\laplacian}[0]{\nabla^2}
%\newcommand{\matrixoftx}[3]{
%\newcommand{\nrrp}[0]{\norm{\rcap \wedge \Br'}}
%\newcommand{\oiint}{\bigcirc \hspace{-1.4em} \int \hspace{-.8em} \int}
%\newcommand{\transpose}[1]{{#1}^{\text{T}}}
%\newcommand{\transpose}[1]{{{#1}^{\TextTranspose}}}
%\newcommand{\transpose}[1]{{{#1}^{\text{T}}}}
%\newcommand{\barA}[0]{\bar{A}}
%\newcommand{\qbar}[0]{\bar{q}}
%\newcommand{\qdotbar}[0]{\dot{\bar{q}}}
%
%</infrequent>





\usepackage[bookmarks=true]{hyperref}

\usepackage{color,cite,graphicx}
   % use colour in the document, put your citations as [1-4]
   % rather than [1,2,3,4] (it looks nicer, and the extended LaTeX2e
   % graphics package. 
\usepackage{latexsym,amssymb,epsf} % don't remember if these are
   % needed, but their inclusion can't do any damage


\title{ Poynting vector and Electromagnetic Energy conservation. }
\author{Peeter Joot}
\date{ Dec 29, 2008.  Last Revision: $Date: 2008/12/31 01:06:10 $ }

\begin{document}

\maketitle{}

\tableofcontents

\section{ Motivation. }

Clarify Poynting discussion from \cite{doran2003gap}.

Equation 7.59 and 7.60 derives a $\BE \cross \BB$ quantity, the Poynting vector, as a sort of energy flux through the surface of the containing volume.

There are a couple of magic steps here that were not at all obvious to me.  Go through this in enough detail that it makes sense to me.

\section{ Charge free case. }

In SI units the Energy density is given as

\begin{align*}
U = \frac{\epsilon_0}{2}\left( \BE^2 + c^2 \BB^2 \right)
\end{align*}

FIXME: Don't truely understand where this part comes from.  The article \href{http://farside.ph.utexas.edu/teaching/em/lectures/node89.html}{Energy Conservation} looks promising to study this.

Given this energy density the rate of change of energy in a volume is then

\begin{align*}
\frac{dU}{dt} 
&= 
\frac{d}{dt} 
\frac{\epsilon_0}{2} \int dV \left( \BE^2 + c^2 \BB^2 \right) \\
&= 
\epsilon_0 \int dV \left( \BE \cdot \PD{t}{\BE} + c^2 \BB \cdot \PD{t}{\BB} \right) \\
\end{align*}

The next (omitted in the text) step is to utilize Maxwell's equation to eliminate the time derivatives.  Since this is the
charge and current free case, we can write Maxwell's as

\begin{align*}
0
&= \gamma_0 \grad F \\
&= \gamma_0 (\gamma^0 \partial_0 + \gamma^k \partial_k) F \\
&= (\partial_0 + \gamma_k\gamma_0 \partial_k) F \\
&= (\partial_0 + \sigma_k \partial_k) F \\
&= (\partial_0 + \spacegrad)F \\
&= (\partial_0 + \spacegrad)(\BE + ic \BB) \\
&= \partial_0 \BE + ic \partial_0 \BB + \spacegrad \BE + ic \spacegrad \BB \\
\end{align*}

In the spatial ($\sigma$) basis we can separate this into even and odd grades, which are separately equal to zero

\begin{align*}
0 &= \partial_0 \BE + ic \spacegrad \BB \\
%   1                  3,1 
0 &= ic \partial_0 \BB + \spacegrad \BE 
%  2                    0,2
\end{align*}

A selection of just the vector parts is

\begin{align*}
\partial_t \BE &= - ic^2 \spacegrad \wedge \BB \\
\partial_t \BB &= i\spacegrad \wedge \BE 
\end{align*}

Which can be back substituited into the energy flux
\begin{align*}
\frac{dU}{dt} 
&= \epsilon_0 \int dV \left( \BE \cdot (-i c^2 \spacegrad \wedge \BB) + c^2 \BB \cdot (i \spacegrad \wedge \BE) \right) \\
&= \epsilon_0 c^2 \int dV \gpgradezero{ \BB i \spacegrad \wedge \BE -\BE i \spacegrad \wedge \BB } \\
\end{align*}

Since the two divergence terms are zero we can drop the wedges here for

\begin{align*}
\frac{dU}{dt} 
&= \epsilon_0 c^2 \int dV \gpgradezero{ \BB i \spacegrad \BE -\BE i \spacegrad \BB } \\
&= \epsilon_0 c^2 \int dV \gpgradezero{ (i \BB) \spacegrad \BE -\BE \spacegrad (i\BB) } \\
&= \epsilon_0 c^2 \int dV \spacegrad \cdot ( (i \BB) \cdot \BE ) \\
\end{align*}

Justification for this last step can be found below in the derivation of equation \ref{eqn:poyntingDivergence}.

We can now use Stokes theorem to change this into a surface integral for a final energy flux 

\begin{align*}
\frac{dU}{dt} 
&= \epsilon_0 c^2 \int d\BA \cdot ( (i \BB) \cdot \BE ) \\
\end{align*}

This last bivector/vector dot product is the Poynting vector

\begin{align*}
(i \BB) \cdot \BE 
&= \gpgradeone{ (i \BB) \cdot \BE } \\
&= \gpgradeone{ i \BB \BE } \\
&= \gpgradeone{ i (\BB \wedge \BE) } \\
&= i (\BB \wedge \BE) \\
&= i^2(\BB \cross \BE) \\
&= \BE \cross \BB \\
\end{align*}

So, we can identity the quantity 

\begin{align}\label{eqn:poynting}
\epsilon_0 c^2 \BE \cross \BB = \epsilon_0 c (i c \BB) \cdot \BE 
\end{align}

As a directed energy density flux through the surface of a containing volume.


\section{ With charges and currents }
 
To calculate time derivatives we want to take Maxwell's equation and put into a form with explicit time derivatives, as was done before, but this time be more careful with the handling of the four vector current term.  Starting with left factoring out of a $\gamma_0$ from the spacetime gradient. 
 
\begin{align*}
\grad &= \gamma^0 \partial_0 + \gamma^k \partial_k \\
&= \gamma^0 (\partial_0 - \gamma^k \gamma_0 \partial_k) \\
&= \gamma^0 (\partial_0 + \sigma_k \partial_k) \\
\end{align*}

Similarily, the $\gamma_0$ can be factored from the current density

\begin{align*}
J 
&= \gamma_0 c \rho + \gamma_k J^k \\
&= \gamma_0 (c \rho - \gamma_k \gamma_0 J^k) \\
&= \gamma_0 (c \rho - \sigma_k J^k) \\
&= \gamma_0 (c \rho - \Bj )
\end{align*}

With this Maxwell's equation becomes
 
\begin{align*}
\gamma_0 \grad F &= \gamma_0 J / \epsilon_0 c \\
(\partial_0 + \spacegrad) ( \BE + i c \BB ) &= \rho/\epsilon_0 - \Bj/\epsilon_0 c \\
\end{align*}
 
A split into even and odd grades including current and charge density is thus
 
\begin{align*}
\spacegrad \BE + \partial_t (i \BB) &= \rho/\epsilon_0 \\
\spacegrad (i \BB) c^2 + \partial_t \BE &= -\Bj/\epsilon_0
\end{align*}
 
Now, taking time derivatives of the energy density gives

\begin{align*}
\PD{t}{U} 
&= \PD{t}{}\inv{2} \epsilon_0 \left( \BE^2 - (ic \BB)^2 \right) \\
&= \epsilon_0 \left( \BE \cdot \partial_t \BE - c^2 (i\BB) \cdot \partial_t (i\BB) \right) \\
&= \epsilon_0 \gpgradezero{ \BE ( -\Bj/\epsilon_0 -\spacegrad (i \BB) c^2 ) - c^2 (i\BB) ( -\spacegrad \BE + \rho/\epsilon_0 ) } \\
&= -\BE \cdot \Bj + c^2 \epsilon_0 \gpgradezero{ i\BB \spacegrad \BE -\BE \spacegrad (i \BB) } \\
&= -\BE \cdot \Bj + c^2 \epsilon_0 \left( (i\BB) \cdot (\spacegrad \wedge \BE) - \BE \cdot (\spacegrad \cdot (i \BB)) \right) \\
\end{align*}

Using equation \ref{eqn:poyntingDivergence}, we now have the rate of change of
field energy for the general case including currents.  That is

\begin{align}
\PD{t}{U} &= -\BE \cdot \Bj + c^2 \epsilon_0 \spacegrad \cdot (\BE \cdot (i\BB)) 
\end{align}

Written out in full, and in terms of the Poynting vector this is

\begin{align}
\PD{t}{}\frac{\epsilon_0}{2} \left(\BE^2 + c^2 \BB^2\right) + c^2 \epsilon_0 \spacegrad \cdot (\BE \cross \BB) &= -\BE \cdot \Bj 
\end{align}

\section{ Poynting vector in terms of complete field. }

In equation \ref{eqn:poynting} the individual parts of the complete Faraday
bivector $F = \BE + i c \BB$ stand out.  How would the Poynting vector be
expressed in terms of $F$ or in tensor form?

Since
\begin{align*}
F \gamma_0 = - \gamma_0(\BE - i c \BB)
\end{align*}

we have
\begin{align*}
\gamma^0 F \gamma_0 = - (\BE - i c \BB)
\end{align*}

and
\begin{align*}
i c \BB &= \inv{2}(F + \gamma^0 F \gamma_0) \\
\BE &= \inv{2}(F - \gamma^0 F \gamma_0) \\
\end{align*}

FIXME: tried using these but messed up.

%Without justifying all the steps I think that the following is valid
%
%\begin{align*}
%(i c \BB) \cdot \BE 
%&= \gpgradeone{(i c \BB) \cdot \BE } \\
%&= \gpgradeone{i c \BB \BE } \\
%&= \inv{4} (F + \gamma_0 F \gamma_0) \cdot (F - \gamma_0 F \gamma_0) \\
%&= \inv{4} (F^2 - \gamma_0 F \gamma_0 \gamma_0 F \gamma_0 + (\gamma_0 F \gamma_0) \cdot F - F \cdot (\gamma_0 F \gamma_0) ) \\
%&= \inv{4} ( (\gamma_0 F \gamma_0) \cdot F - F \cdot (\gamma_0 F \gamma_0) ) \\
%&= \inv{2} (\gamma_0 F \gamma_0) \cdot F 
%\end{align*}
%
%  The above is wrong.  This is - <F^\dagger F>/2, which is c^2 B^2 - E^2, which isn't even vector.

\section{ Energy Density from Lagrangian. }

I didn't get too far trying to calculate the electrodynamic Hamiltonian density for the general case, so I tried it for a very 
simple special case, with just an electric field component in one direction:

\begin{align*}
\mathcal{L}
&= \frac{1}{2}(E_x)^2 \\
&= \frac{1}{2}(F_{01})^2 \\
&= \frac{1}{2}(\partial_0 A_1 - \partial_1 A_0)^2 \\
\end{align*}

Goldstein gives the Hamiltonian density as

\begin{align*}
\pi &= \frac{\partial \mathcal{L}}{\partial \dot{n}} \\
\mathcal{H} &= \dot{n} \pi - \mathcal{L}
\end{align*}

If I try calculating this I get

\begin{align*}
\pi 
&= \frac{\partial}{\partial (\partial_0 A_1)} \left( \frac{1}{2}(\partial_0 A_1 - \partial_1 A_0)^2 \right) \\
&= \partial_0 A_1 - \partial_1 A_0 \\
&= F_{01} \\
\end{align*}

So this gives a Hamiltonian of
\begin{align*}
\mathcal{H}
&= \partial_0 A_1 F_{01} - \frac{1}{2}(\partial_0 A_1 - \partial_1 A_0)F_{01} \\
&= \frac{1}{2} (\partial_0 A_1 + \partial_1 A_0 )F_{01} 
&= \frac{1}{2} ((\partial_0 A_1)^2 - (\partial_1 A_0)^2 )
\end{align*}

For a Lagrangian density of $E^2 - B^2$ we have an energy density of $E^2 + B^2$, so I'd have expected the Hamiltonian density here to stay equal to $E_x^2/2$, but it 
doesn't look like that's what I get (what I calculated isn't at all familiar seeming).

If I haven't made a mistake here, perhaps I'm incorrect in assuming that the Hamiltonian density of the electrodynamic Lagrangian should be the energy density?

\section{ Appendix.  Messy details. }

For both the charge and the charge free case, we need a proof of 

\begin{align*}
(i\BB) \cdot (\spacegrad \wedge \BE) - \BE \cdot (\spacegrad \cdot (i \BB)) 
&= \spacegrad \cdot (\BE \cdot (i\BB)) 
\end{align*}

This is relativity straightforward, albeit tedious, to do backwards.

\begin{align*}
\spacegrad \cdot ((i \BB) \cdot \BE)
&= \gpgradezero{ \spacegrad ((i \BB) \cdot \BE)} \\
&= \inv{2} \gpgradezero{ \spacegrad ( i \BB \BE - \BE i \BB ) } \\
&= \inv{2} \gpgradezero{ 
  \dot{\spacegrad} i \dot{\BB} \BE 
+ \dot{\spacegrad} i \BB \dot{\BE}
- \dot{\spacegrad} \dot{\BE} i \BB 
- \dot{\spacegrad} \BE i \dot{\BB}
} \\
&= \inv{2} \gpgradezero{ 
  \BE \spacegrad (i \BB) - (i\dot{\BB}) \dot{\spacegrad} \BE
+ \dot{\BE} \dot{\spacegrad} i \BB - i \BB \spacegrad \BE
} \\
&= \inv{2} \left(
  \BE \cdot (\spacegrad \cdot (i \BB)) - ((i\dot{\BB}) \cdot \dot{\spacegrad}) \cdot \BE
+ (\dot{\BE} \wedge \dot{\spacegrad}) \cdot (i \BB) - (i \BB) \cdot (\spacegrad \wedge \BE) 
\right)
\\
\end{align*}

Grouping the two sets of repeated terms after reordering and the associated sign adjustments we have

\begin{align}\label{eqn:poyntingDivergence}
\spacegrad \cdot ((i \BB) \cdot \BE) &= \BE \cdot (\spacegrad \cdot (i \BB)) - (i \BB) \cdot (\spacegrad \wedge \BE)
\end{align}

which is the desired identity (in negated form) that was to be proved.

There is likely some theorem that could be used to avoid some of this algebra.

\bibliographystyle{plainnat}
\bibliography{myrefs}

\end{document}

\documentclass{article}

\usepackage{amsmath}
\usepackage{mathpazo}

%
% shorthand for bold symbols, convenient for vectors and matrices
%
\newcommand{\Ba}[0]{\mathbf{a}}
\newcommand{\Bb}[0]{\mathbf{b}}
\newcommand{\Bc}[0]{\mathbf{c}}
\newcommand{\Bd}[0]{\mathbf{d}}
\newcommand{\Be}[0]{\mathbf{e}}
\newcommand{\Bf}[0]{\mathbf{f}}
\newcommand{\Bg}[0]{\mathbf{g}}
\newcommand{\Bh}[0]{\mathbf{h}}
\newcommand{\Bi}[0]{\mathbf{i}}
\newcommand{\Bj}[0]{\mathbf{j}}
\newcommand{\Bk}[0]{\mathbf{k}}
\newcommand{\Bl}[0]{\mathbf{l}}
\newcommand{\Bm}[0]{\mathbf{m}}
\newcommand{\Bn}[0]{\mathbf{n}}
\newcommand{\Bo}[0]{\mathbf{o}}
\newcommand{\Bp}[0]{\mathbf{p}}
\newcommand{\Bq}[0]{\mathbf{q}}
\newcommand{\Br}[0]{\mathbf{r}}
\newcommand{\Bs}[0]{\mathbf{s}}
\newcommand{\Bt}[0]{\mathbf{t}}
\newcommand{\Bu}[0]{\mathbf{u}}
\newcommand{\Bv}[0]{\mathbf{v}}
\newcommand{\Bw}[0]{\mathbf{w}}
\newcommand{\Bx}[0]{\mathbf{x}}
\newcommand{\By}[0]{\mathbf{y}}
\newcommand{\Bz}[0]{\mathbf{z}}
\newcommand{\BA}[0]{\mathbf{A}}
\newcommand{\BB}[0]{\mathbf{B}}
\newcommand{\BC}[0]{\mathbf{C}}
\newcommand{\BD}[0]{\mathbf{D}}
\newcommand{\BE}[0]{\mathbf{E}}
\newcommand{\BF}[0]{\mathbf{F}}
\newcommand{\BG}[0]{\mathbf{G}}
\newcommand{\BH}[0]{\mathbf{H}}
\newcommand{\BI}[0]{\mathbf{I}}
\newcommand{\BJ}[0]{\mathbf{J}}
\newcommand{\BK}[0]{\mathbf{K}}
\newcommand{\BL}[0]{\mathbf{L}}
\newcommand{\BM}[0]{\mathbf{M}}
\newcommand{\BN}[0]{\mathbf{N}}
\newcommand{\BO}[0]{\mathbf{O}}
\newcommand{\BP}[0]{\mathbf{P}}
\newcommand{\BQ}[0]{\mathbf{Q}}
\newcommand{\BR}[0]{\mathbf{R}}
\newcommand{\BS}[0]{\mathbf{S}}
\newcommand{\BT}[0]{\mathbf{T}}
\newcommand{\BU}[0]{\mathbf{U}}
\newcommand{\BV}[0]{\mathbf{V}}
\newcommand{\BW}[0]{\mathbf{W}}
\newcommand{\BX}[0]{\mathbf{X}}
\newcommand{\BY}[0]{\mathbf{Y}}
\newcommand{\BZ}[0]{\mathbf{Z}}

\newcommand{\Bzero}[0]{\mathbf{0}}
\newcommand{\Btheta}[0]{\boldsymbol{\theta}}
\newcommand{\Btau}[0]{\boldsymbol{\tau}}
\newcommand{\Bomega}[0]{\boldsymbol{\omega}}

%
% shorthand for unit vectors
%
\newcommand{\acap}[0]{\hat{\Ba}}
\newcommand{\bcap}[0]{\hat{\Bb}}
\newcommand{\ccap}[0]{\hat{\Bc}}
\newcommand{\dcap}[0]{\hat{\Bd}}
\newcommand{\ecap}[0]{\hat{\Be}}
\newcommand{\fcap}[0]{\hat{\Bf}}
\newcommand{\gcap}[0]{\hat{\Bg}}
\newcommand{\hcap}[0]{\hat{\Bh}}
\newcommand{\icap}[0]{\hat{\Bi}}
\newcommand{\jcap}[0]{\hat{\Bj}}
\newcommand{\kcap}[0]{\hat{\Bk}}
\newcommand{\lcap}[0]{\hat{\Bl}}
\newcommand{\mcap}[0]{\hat{\Bm}}
\newcommand{\ncap}[0]{\hat{\Bn}}
\newcommand{\ocap}[0]{\hat{\Bo}}
\newcommand{\pcap}[0]{\hat{\Bp}}
\newcommand{\qcap}[0]{\hat{\Bq}}
\newcommand{\rcap}[0]{\hat{\Br}}
\newcommand{\scap}[0]{\hat{\Bs}}
\newcommand{\tcap}[0]{\hat{\Bt}}
\newcommand{\ucap}[0]{\hat{\Bu}}
\newcommand{\vcap}[0]{\hat{\Bv}}
\newcommand{\wcap}[0]{\hat{\Bw}}
\newcommand{\xcap}[0]{\hat{\Bx}}
\newcommand{\ycap}[0]{\hat{\By}}
\newcommand{\zcap}[0]{\hat{\Bz}}
\newcommand{\thetacap}[0]{\hat{\Btheta}}

%
% to write R^n and C^n in a distinguishable fashion.  Perhaps change this
% to the double lined characters upon figuring out how to do so.
%
\newcommand{\C}[1]{$\mathbb{C}^{#1}$}
\newcommand{\R}[1]{$\mathbb{R}^{#1}$}

%
% various generally useful helpers
%

% derivative of #1 wrt. #2:
\newcommand{\D}[2] {\frac {d#2} {d#1}}

\newcommand{\inv}[1]{\frac{1}{#1}}
\newcommand{\cross}[0]{\times}

\newcommand{\abs}[1]{\lvert{#1}\rvert}
\newcommand{\norm}[1]{\lVert{#1}\rVert}
\newcommand{\innerprod}[2]{\langle{#1}, {#2}\rangle}
\newcommand{\dotprod}[2]{{#1} \cdot {#2}}
\newcommand{\bdotprod}[2]{\left({#1} \cdot {#2}\right)}
\newcommand{\crossprod}[2]{{#1} \cross {#2}}
\newcommand{\tripleprod}[3]{\dotprod{\left(\crossprod{#1}{#2}\right)}{#3}}

\DeclareMathOperator{\Proj}{Proj}
\DeclareMathOperator{\Span}{span}
\DeclareMathOperator{\Sgn}{sgn}
\DeclareMathOperator{\Area}{Area}
\DeclareMathOperator{\Volume}{Volume}

%
% A few miscellaneous things specific to this document
%
\newcommand{\crossop}[1]{\crossprod{#1}{}}

% R2 vector.
\newcommand{\VectorTwo}[2]{
\begin{bmatrix}
 {#1} \\
 {#2}
\end{bmatrix}
}

\newcommand{\VectorN}[1]{
\begin{bmatrix}
{#1}_1 \\
{#1}_2 \\
\vdots \\
{#1}_N \\
\end{bmatrix}
}

\newcommand{\DETuvij}[4]{
\begin{vmatrix}
 {#1}_{#3} & {#1}_{#4} \\
 {#2}_{#3} & {#2}_{#4}
\end{vmatrix}
}

\newcommand{\DETuvwijk}[6]{
\begin{vmatrix}
 {#1}_{#4} & {#1}_{#5} & {#1}_{#6} \\
 {#2}_{#4} & {#2}_{#5} & {#2}_{#6} \\
 {#3}_{#4} & {#3}_{#5} & {#3}_{#6}
\end{vmatrix}
}

\newcommand{\DETuvwxijkl}[8]{
\begin{vmatrix}
 {#1}_{#5} & {#1}_{#6} & {#1}_{#7} & {#1}_{#8} \\
 {#2}_{#5} & {#2}_{#6} & {#2}_{#7} & {#2}_{#8} \\
 {#3}_{#5} & {#3}_{#6} & {#3}_{#7} & {#3}_{#8} \\
 {#4}_{#5} & {#4}_{#6} & {#4}_{#7} & {#4}_{#8} \\
\end{vmatrix}
}

%\newcommand{\DETuvwxyijklm}[10]{
%\begin{vmatrix}
% {#1}_{#6} & {#1}_{#7} & {#1}_{#8} & {#1}_{#9} & {#1}_{#10} \\
% {#2}_{#6} & {#2}_{#7} & {#2}_{#8} & {#2}_{#9} & {#2}_{#10} \\
% {#3}_{#6} & {#3}_{#7} & {#3}_{#8} & {#3}_{#9} & {#3}_{#10} \\
% {#4}_{#6} & {#4}_{#7} & {#4}_{#8} & {#4}_{#9} & {#4}_{#10} \\
% {#5}_{#6} & {#5}_{#7} & {#5}_{#8} & {#5}_{#9} & {#5}_{#10}
%\end{vmatrix}
%}

% R3 vector.
\newcommand{\VectorThree}[3]{
\begin{bmatrix}
 {#1} \\
 {#2} \\
 {#3}
\end{bmatrix}
}


%<misc>
%
\newcommand{\Abs}[1]{{\left\lvert{#1}\right\rvert}}
\newcommand{\spacegrad}[0]{\boldsymbol{\nabla}}
\newcommand{\grad}[0]{\nabla}
\newcommand{\LL}[0]{\mathcal{L}}

% == \partial_{#1} {#2}
\newcommand{\PD}[2]{\frac{\partial {#2}}{\partial {#1}}}
% inline variant
\newcommand{\PDi}[2]{{\partial {#2}}/{\partial {#1}}}

\newcommand{\PDD}[3]{\frac{\partial^2 {#3}}{\partial {#1}\partial {#2}}}
%\newcommand{\PDd}[2]{\frac{\partial^2 {#2}}{{\partial{#1}}^2}}
\newcommand{\PDsq}[2]{\frac{\partial^2 {#2}}{(\partial {#1})^2}}

\newcommand{\Partial}[2]{\frac{\partial {#1}}{\partial {#2}}}
\DeclareMathOperator{\RejName}{Rej}
\newcommand{\Rej}[2]{\RejName_{#1}\left( {#2} \right)}
\newcommand{\Rm}[1]{\mathbb{R}^{#1}}
\newcommand{\Cm}[1]{\mathbb{C}^{#1}}
\newcommand{\conj}[0]{{*}}

%</misc>

% <grade selection>
%
\newcommand{\gpgrade}[2] {{\left\langle{{#1}}\right\rangle}_{#2}}

\newcommand{\gpgradezero}[1] {\gpgrade{#1}{}}
%\newcommand{\gpscalargrade}[1] {{\left\langle{{#1}}\right\rangle}}
%\newcommand{\gpgradezero}[1] {\gpgrade{#1}{0}}

%\newcommand{\gpgradeone}[1] {{\left\langle{{#1}}\right\rangle}_{1}}
\newcommand{\gpgradeone}[1] {\gpgrade{#1}{1}}

\newcommand{\gpgradetwo}[1] {\gpgrade{#1}{2}}
\newcommand{\gpgradethree}[1] {\gpgrade{#1}{3}}
\newcommand{\gpgradefour}[1] {\gpgrade{#1}{4}}
%
% </grade selection>



\newcommand{\adot}[0]{{\dot{a}}}
\newcommand{\bdot}[0]{{\dot{b}}}
% taken for centered dot:
%\newcommand{\cdot}[0]{{\dot{c}}}
%\newcommand{\ddot}[0]{{\dot{d}}}
\newcommand{\edot}[0]{{\dot{e}}}
\newcommand{\fdot}[0]{{\dot{f}}}
\newcommand{\gdot}[0]{{\dot{g}}}
\newcommand{\hdot}[0]{{\dot{h}}}
\newcommand{\idot}[0]{{\dot{i}}}
\newcommand{\jdot}[0]{{\dot{j}}}
\newcommand{\kdot}[0]{{\dot{k}}}
\newcommand{\ldot}[0]{{\dot{l}}}
\newcommand{\mdot}[0]{{\dot{m}}}
\newcommand{\ndot}[0]{{\dot{n}}}
%\newcommand{\odot}[0]{{\dot{o}}}
\newcommand{\pdot}[0]{{\dot{p}}}
\newcommand{\qdot}[0]{{\dot{q}}}
\newcommand{\rdot}[0]{{\dot{r}}}
\newcommand{\sdot}[0]{{\dot{s}}}
\newcommand{\tdot}[0]{{\dot{t}}}
\newcommand{\udot}[0]{{\dot{u}}}
\newcommand{\vdot}[0]{{\dot{v}}}
\newcommand{\wdot}[0]{{\dot{w}}}
\newcommand{\xdot}[0]{{\dot{x}}}
\newcommand{\ydot}[0]{{\dot{y}}}
\newcommand{\zdot}[0]{{\dot{z}}}
\newcommand{\addot}[0]{{\ddot{a}}}
\newcommand{\bddot}[0]{{\ddot{b}}}
\newcommand{\cddot}[0]{{\ddot{c}}}
%\newcommand{\dddot}[0]{{\ddot{d}}}
\newcommand{\eddot}[0]{{\ddot{e}}}
\newcommand{\fddot}[0]{{\ddot{f}}}
\newcommand{\gddot}[0]{{\ddot{g}}}
\newcommand{\hddot}[0]{{\ddot{h}}}
\newcommand{\iddot}[0]{{\ddot{i}}}
\newcommand{\jddot}[0]{{\ddot{j}}}
\newcommand{\kddot}[0]{{\ddot{k}}}
\newcommand{\lddot}[0]{{\ddot{l}}}
\newcommand{\mddot}[0]{{\ddot{m}}}
\newcommand{\nddot}[0]{{\ddot{n}}}
\newcommand{\oddot}[0]{{\ddot{o}}}
\newcommand{\pddot}[0]{{\ddot{p}}}
\newcommand{\qddot}[0]{{\ddot{q}}}
\newcommand{\rddot}[0]{{\ddot{r}}}
\newcommand{\sddot}[0]{{\ddot{s}}}
\newcommand{\tddot}[0]{{\ddot{t}}}
\newcommand{\uddot}[0]{{\ddot{u}}}
\newcommand{\vddot}[0]{{\ddot{v}}}
\newcommand{\wddot}[0]{{\ddot{w}}}
\newcommand{\xddot}[0]{{\ddot{x}}}
\newcommand{\yddot}[0]{{\ddot{y}}}
\newcommand{\zddot}[0]{{\ddot{z}}}

%<bold and dot greek symbols>
%

\newcommand{\Deltadot}[0]{{\dot{\Delta}}}
\newcommand{\Gammadot}[0]{{\dot{\Gamma}}}
\newcommand{\Lambdadot}[0]{{\dot{\Lambda}}}
\newcommand{\Omegadot}[0]{{\dot{\Omega}}}
\newcommand{\Phidot}[0]{{\dot{\Phi}}}
\newcommand{\Pidot}[0]{{\dot{\Pi}}}
\newcommand{\Psidot}[0]{{\dot{\Psi}}}
\newcommand{\Sigmadot}[0]{{\dot{\Sigma}}}
\newcommand{\Thetadot}[0]{{\dot{\Theta}}}
\newcommand{\Upsilondot}[0]{{\dot{\Upsilon}}}
\newcommand{\Xidot}[0]{{\dot{\Xi}}}
\newcommand{\alphadot}[0]{{\dot{\alpha}}}
\newcommand{\betadot}[0]{{\dot{\beta}}}
\newcommand{\chidot}[0]{{\dot{\chi}}}
\newcommand{\deltadot}[0]{{\dot{\delta}}}
\newcommand{\epsilondot}[0]{{\dot{\epsilon}}}
\newcommand{\etadot}[0]{{\dot{\eta}}}
\newcommand{\gammadot}[0]{{\dot{\gamma}}}
\newcommand{\kappadot}[0]{{\dot{\kappa}}}
\newcommand{\lambdadot}[0]{{\dot{\lambda}}}
\newcommand{\mudot}[0]{{\dot{\mu}}}
\newcommand{\nudot}[0]{{\dot{\nu}}}
\newcommand{\omegadot}[0]{{\dot{\omega}}}
\newcommand{\phidot}[0]{{\dot{\phi}}}
\newcommand{\pidot}[0]{{\dot{\pi}}}
\newcommand{\psidot}[0]{{\dot{\psi}}}
\newcommand{\rhodot}[0]{{\dot{\rho}}}
\newcommand{\sigmadot}[0]{{\dot{\sigma}}}
\newcommand{\taudot}[0]{{\dot{\tau}}}
\newcommand{\thetadot}[0]{{\dot{\theta}}}
\newcommand{\upsilondot}[0]{{\dot{\upsilon}}}
\newcommand{\varepsilondot}[0]{{\dot{\varepsilon}}}
\newcommand{\varphidot}[0]{{\dot{\varphi}}}
\newcommand{\varpidot}[0]{{\dot{\varpi}}}
\newcommand{\varrhodot}[0]{{\dot{\varrho}}}
\newcommand{\varsigmadot}[0]{{\dot{\varsigma}}}
\newcommand{\varthetadot}[0]{{\dot{\vartheta}}}
\newcommand{\xidot}[0]{{\dot{\xi}}}
\newcommand{\zetadot}[0]{{\dot{\zeta}}}

\newcommand{\Deltaddot}[0]{{\ddot{\Delta}}}
\newcommand{\Gammaddot}[0]{{\ddot{\Gamma}}}
\newcommand{\Lambdaddot}[0]{{\ddot{\Lambda}}}
\newcommand{\Omegaddot}[0]{{\ddot{\Omega}}}
\newcommand{\Phiddot}[0]{{\ddot{\Phi}}}
\newcommand{\Piddot}[0]{{\ddot{\Pi}}}
\newcommand{\Psiddot}[0]{{\ddot{\Psi}}}
\newcommand{\Sigmaddot}[0]{{\ddot{\Sigma}}}
\newcommand{\Thetaddot}[0]{{\ddot{\Theta}}}
\newcommand{\Upsilonddot}[0]{{\ddot{\Upsilon}}}
\newcommand{\Xiddot}[0]{{\ddot{\Xi}}}
\newcommand{\alphaddot}[0]{{\ddot{\alpha}}}
\newcommand{\betaddot}[0]{{\ddot{\beta}}}
\newcommand{\chiddot}[0]{{\ddot{\chi}}}
\newcommand{\deltaddot}[0]{{\ddot{\delta}}}
\newcommand{\epsilonddot}[0]{{\ddot{\epsilon}}}
\newcommand{\etaddot}[0]{{\ddot{\eta}}}
\newcommand{\gammaddot}[0]{{\ddot{\gamma}}}
\newcommand{\kappaddot}[0]{{\ddot{\kappa}}}
\newcommand{\lambdaddot}[0]{{\ddot{\lambda}}}
\newcommand{\muddot}[0]{{\ddot{\mu}}}
\newcommand{\nuddot}[0]{{\ddot{\nu}}}
\newcommand{\omegaddot}[0]{{\ddot{\omega}}}
\newcommand{\phiddot}[0]{{\ddot{\phi}}}
\newcommand{\piddot}[0]{{\ddot{\pi}}}
\newcommand{\psiddot}[0]{{\ddot{\psi}}}
\newcommand{\rhoddot}[0]{{\ddot{\rho}}}
\newcommand{\sigmaddot}[0]{{\ddot{\sigma}}}
\newcommand{\tauddot}[0]{{\ddot{\tau}}}
\newcommand{\thetaddot}[0]{{\ddot{\theta}}}
\newcommand{\upsilonddot}[0]{{\ddot{\upsilon}}}
\newcommand{\varepsilonddot}[0]{{\ddot{\varepsilon}}}
\newcommand{\varphiddot}[0]{{\ddot{\varphi}}}
\newcommand{\varpiddot}[0]{{\ddot{\varpi}}}
\newcommand{\varrhoddot}[0]{{\ddot{\varrho}}}
\newcommand{\varsigmaddot}[0]{{\ddot{\varsigma}}}
\newcommand{\varthetaddot}[0]{{\ddot{\vartheta}}}
\newcommand{\xiddot}[0]{{\ddot{\xi}}}
\newcommand{\zetaddot}[0]{{\ddot{\zeta}}}

\newcommand{\BDelta}[0]{\boldsymbol{\Delta}}
\newcommand{\BGamma}[0]{\boldsymbol{\Gamma}}
\newcommand{\BLambda}[0]{\boldsymbol{\Lambda}}
\newcommand{\BOmega}[0]{\boldsymbol{\Omega}}
\newcommand{\BPhi}[0]{\boldsymbol{\Phi}}
\newcommand{\BPi}[0]{\boldsymbol{\Pi}}
\newcommand{\BPsi}[0]{\boldsymbol{\Psi}}
\newcommand{\BSigma}[0]{\boldsymbol{\Sigma}}
\newcommand{\BTheta}[0]{\boldsymbol{\Theta}}
\newcommand{\BUpsilon}[0]{\boldsymbol{\Upsilon}}
\newcommand{\BXi}[0]{\boldsymbol{\Xi}}
\newcommand{\Balpha}[0]{\boldsymbol{\alpha}}
\newcommand{\Bbeta}[0]{\boldsymbol{\beta}}
\newcommand{\Bchi}[0]{\boldsymbol{\chi}}
\newcommand{\Bdelta}[0]{\boldsymbol{\delta}}
\newcommand{\Bepsilon}[0]{\boldsymbol{\epsilon}}
\newcommand{\Beta}[0]{\boldsymbol{\eta}}
\newcommand{\Bgamma}[0]{\boldsymbol{\gamma}}
\newcommand{\Bkappa}[0]{\boldsymbol{\kappa}}
\newcommand{\Blambda}[0]{\boldsymbol{\lambda}}
\newcommand{\Bmu}[0]{\boldsymbol{\mu}}
\newcommand{\Bnu}[0]{\boldsymbol{\nu}}
%\newcommand{\Bomega}[0]{\boldsymbol{\omega}}
\newcommand{\Bphi}[0]{\boldsymbol{\phi}}
\newcommand{\Bpi}[0]{\boldsymbol{\pi}}
\newcommand{\Bpsi}[0]{\boldsymbol{\psi}}
\newcommand{\Brho}[0]{\boldsymbol{\rho}}
\newcommand{\Bsigma}[0]{\boldsymbol{\sigma}}
%\newcommand{\Btau}[0]{\boldsymbol{\tau}}
%\newcommand{\Btheta}[0]{\boldsymbol{\theta}}
\newcommand{\Bupsilon}[0]{\boldsymbol{\upsilon}}
\newcommand{\Bvarepsilon}[0]{\boldsymbol{\varepsilon}}
\newcommand{\Bvarphi}[0]{\boldsymbol{\varphi}}
\newcommand{\Bvarpi}[0]{\boldsymbol{\varpi}}
\newcommand{\Bvarrho}[0]{\boldsymbol{\varrho}}
\newcommand{\Bvarsigma}[0]{\boldsymbol{\varsigma}}
\newcommand{\Bvartheta}[0]{\boldsymbol{\vartheta}}
\newcommand{\Bxi}[0]{\boldsymbol{\xi}}
\newcommand{\Bzeta}[0]{\boldsymbol{\zeta}}
%
%</bold and dot greek symbols>
%<infrequent>
%
%\newcommand{\AreaOp}[1]{\AName_{#1}}
%\newcommand{\Babs}[0]{\abs{\BB}}
%\newcommand{\Bcap}[0]{\hat{\BB}}
%\newcommand{\BrPrimeRej}[0]{\rcap(\rcap \wedge \Br')}
%\newcommand{\CA}[0]{\mathcal{A}}
%\newcommand{\Cos}[1]{\cos{\left({#1}\right)}}
%\newcommand{\Det}[1] {\abs{#1}}
%\newcommand{\Dsq}[2] {\frac {\partial^2 {#1}} {\partial {#2}^2}}
%\newcommand{\Exp}[1]{\exp{\left({#1}\right)}}
%\newcommand{\Norm}[1]{\left\lVert{#1}\right\rVert}
%\newcommand{\Sin}[1]{\sin{\left({#1}\right)}}
%\newcommand{\T}[0]{\text{T}}
%\newcommand{\VolumeOp}[1]{\VName_{#1}}
%\newcommand{\agrad}[0]{\Ba \cdot \nabla}
%\newcommand{\alphacap}[0]{\hat{\boldsymbol{\alpha}}}
%\newcommand{\Fcap}[0]{\hat{\BF}}
%\newcommand{\bithree}[0]{{\Bi}_3}
%\newcommand{\bxa}[0]{\Bx\Ba}
%\newcommand{\coordvec}[2]{
%\newcommand{\costheta}[0]{\acap \cdot \xcap}
%\newcommand{\ddt}[1]{\ddot{#1}}
%\newcommand{\ddu}[1] {\frac {d{#1}} {du}}
%\newcommand{\dsqxj}[2] {\frac {\partial^2 {#1}} {\partial {x_{#2}}^2}}
%\newcommand{\dtheta}[1]{\frac{d {#1}}{d \theta}}
%\newcommand{\dt}[1]{\dot{#1}}
%\newcommand{\dt}[1]{\frac{d {#1}}{dt}}
%\newcommand{\dxj}[2] {\frac {\partial {#1}} {\partial {x_{#2}}}}
%\newcommand{\halfPhi}[0]{\frac{\phi}{2}}
%\newcommand{\half}[0]{\inv{2}}
%\newcommand{\inv}[1]{\frac{1}{#1}}
%\newcommand{\laplacian}[0]{\nabla^2}
%\newcommand{\matrixoftx}[3]{
%\newcommand{\nrrp}[0]{\norm{\rcap \wedge \Br'}}
%\newcommand{\oiint}{\bigcirc \hspace{-1.4em} \int \hspace{-.8em} \int}
%\newcommand{\transpose}[1]{{#1}^{\text{T}}}
%\newcommand{\transpose}[1]{{{#1}^{\TextTranspose}}}
%\newcommand{\transpose}[1]{{{#1}^{\text{T}}}}
%\newcommand{\barA}[0]{\bar{A}}
%\newcommand{\qbar}[0]{\bar{q}}
%\newcommand{\qdotbar}[0]{\dot{\bar{q}}}
%
%</infrequent>





\usepackage[bookmarks=true]{hyperref}

\usepackage{color,cite,graphicx}
   % use colour in the document, put your citations as [1-4]
   % rather than [1,2,3,4] (it looks nicer, and the extended LaTeX2e
   % graphics package. 
\usepackage{latexsym,amssymb,epsf} % don't remember if these are
   % needed, but their inclusion can't do any damage


\title{}
\author{Peeter Joot}
\date{ Mmm dd, 2009.  Last Revision: $Date: 2009/01/18 00:44:16 $ }

\begin{document}

\maketitle{}

\tableofcontents

\section{ Motivation }

Derive the conservation laws for the time rate of change of the Poynting vector, which appears to be a momentum density like quantity.

The Poynting conservation relationship has been derived previously.  Additionally a starting
exploration 
\cite{PJemstresstensor}
of the related four vector quantity has been related to a subset of the energy momentum stress tensor.
This was incomplete since the meaning of the $T_{kj}$ terms of the tensor were unknown and the expected
Lorentz transform relationships had not been determined.  The aim here is to try to figure out this remainder.

\section{ Calculation }

Repeating again from \cite{PJpoynting}, the electrodynamic energy density $U$ and momentum flux density vectors are related as follows

\begin{align}\label{eqn:fromPoyntingNotes}
U &= \frac{\epsilon_0}{2}\left( \BE^2 + c^2 \BB^2 \right) \\
\BP &= \inv{\mu_0}\BE \cross \BB = \inv{\mu_0} (i \BB) \cdot \BE \\
0 &= \PD{t}{U} + \spacegrad \cdot \BP + \BE \cdot \Bj
\end{align}

We want to now calculate the time rate of change of this Poynting (field momentum density) vector.

\begin{align*}
\PD{t}{\BP}
&= \PD{t}{} \left( \inv{\mu_0} \BE \cross \BB \right) \\
&= \PD{t}{} \left( \inv{\mu_0}(i\BB) \cdot \BE \right) \\
&= \partial_0 \left( \inv{\mu_0}(i c\BB) \cdot \BE \right) \\
&= \inv{\mu_0} \left( \partial_0 (i c\BB) \cdot \BE  + (i c\BB) \cdot \partial_0 \BE  \right)
\end{align*}

%Let's ignore the $\mu_0$ factor for now, and focus just on the field dot products.  
We will want to express these time derivatives in
terms of the current and spatial derivatives to determine the conservation identity.  To do this let's go back to Maxwell's equation
once more, with a premultiplication by $\gamma_0$ to provide us with an observer dependent spacetime split

\begin{align*}
\gamma_0 \grad F &= \gamma_0 J / \epsilon_0 c \\
(\partial_0 + \spacegrad) ( \BE + i c \BB ) &= \rho/\epsilon_0 - \Bj/\epsilon_0 c \\
\end{align*}

We want the grade one and grade two components for the time derivative terms.  For grade one we have

\begin{align*}
- \Bj/\epsilon_0 c
&= \gpgradeone{(\partial_0 + \spacegrad) ( \BE + i c \BB )} \\
&= \partial_0 \BE + \spacegrad \cdot (ic \BB)
\end{align*}

and for grade two
\begin{align*}
0 
&= \gpgradetwo{(\partial_0 + \spacegrad) ( \BE + i c \BB )} \\
&= \partial_0 (i c \BB) + \spacegrad \wedge \BE
\end{align*}

Using these we can express the time derivatives for back substitution
\begin{align*}
\partial_0 \BE &= - \Bj/\epsilon_0 c - \spacegrad \cdot (ic \BB) \\
\partial_0 (i c \BB) &= -\spacegrad \wedge \BE
\end{align*}

yielding
\begin{align*}
\mu_0 \PD{t}{\BP}
&= \partial_0 (i c\BB) \cdot \BE  + (i c\BB) \cdot \partial_0 \BE \\
&= -(\spacegrad \wedge \BE) \cdot \BE  - (i c\BB) \cdot \left( \Bj/\epsilon_0 c + \spacegrad \cdot (ic \BB) \right) \\
\end{align*}

Or
\begin{align*}
\PD{t}{(i\BB) \cdot \BE} +(\spacegrad \wedge \BE) \cdot \BE + (i c\BB) \cdot (\spacegrad \cdot (ic \BB)) &= - (i c\BB) \cdot \Bj/\epsilon_0 c \\
\end{align*}

Now, while this contains the essence of the momentum conservation it is a bit sloppy.  We expect something like the four vector
divergence as we had in the Poynting energy/momentum density equations.  Let's see if we can put the second two terms into
the form of a spatial divergence.  The expectation here since we have vector quantities all around is that this will be the spatial 
divergence of a bivector quantity of some sort.  Also note that the Poynting vector is also really the dual of a bivector,
and that we have an implicit cross product in dual form with the current density vector on the right hand side as well.

% WANT:
%This equation is equivalent to the following 3D conservation laws

%$\frac{\partial u_{em}}{\partial t} + \vec{\nabla} \cdot \vec{S} + \vec{J} \cdot \vec{E} = 0 \,$
%
%$\frac{\partial \vec{p}_{em}}{\partial t} - \vec{\nabla}\cdot \sigma + \rho \vec{E} + \vec{J} \times \vec{B} = 0 \,$

%where
%Electromagnetic energy density (joules/meter^3) is $u_{em} = \frac{\epsilon_0}{2}E^2 + \frac{1}{2\mu_0}B^2 \,$
%
%Poynting vector (watts/meter^2) is $\vec{S} = \frac{1}{\mu_0} \vec{E} \times \vec{B} \,$
%
%Density of electric current (amperes/meter^2) is $\vec{J} \,$
%
%Electromagnetic momentum density (newton&middot;seconds/meter${}^3$) is $\vec{p}_{em} = {\vec{S} \over c^2} \,$
%
%Maxwell stress tensor (newtons/meter^2) is $\sigma_{ij} = \epsilon_0 E_i E_j   + \frac{1}{{\mu _0 }}B_i B_j - \frac{1}{2}\left( {\epsilon_0 E^2  + \frac{1}{{\mu _0 }}B^2 } \right)\delta _{ij} \,$

\bibliographystyle{plainnat}
\bibliography{myrefs}

\end{document}


\documentclass{article}

\usepackage{amsmath}
\usepackage{mathpazo}

%
% shorthand for bold symbols, convenient for vectors and matrices
%
\newcommand{\Ba}[0]{\mathbf{a}}
\newcommand{\Bb}[0]{\mathbf{b}}
\newcommand{\Bc}[0]{\mathbf{c}}
\newcommand{\Bd}[0]{\mathbf{d}}
\newcommand{\Be}[0]{\mathbf{e}}
\newcommand{\Bf}[0]{\mathbf{f}}
\newcommand{\Bg}[0]{\mathbf{g}}
\newcommand{\Bh}[0]{\mathbf{h}}
\newcommand{\Bi}[0]{\mathbf{i}}
\newcommand{\Bj}[0]{\mathbf{j}}
\newcommand{\Bk}[0]{\mathbf{k}}
\newcommand{\Bl}[0]{\mathbf{l}}
\newcommand{\Bm}[0]{\mathbf{m}}
\newcommand{\Bn}[0]{\mathbf{n}}
\newcommand{\Bo}[0]{\mathbf{o}}
\newcommand{\Bp}[0]{\mathbf{p}}
\newcommand{\Bq}[0]{\mathbf{q}}
\newcommand{\Br}[0]{\mathbf{r}}
\newcommand{\Bs}[0]{\mathbf{s}}
\newcommand{\Bt}[0]{\mathbf{t}}
\newcommand{\Bu}[0]{\mathbf{u}}
\newcommand{\Bv}[0]{\mathbf{v}}
\newcommand{\Bw}[0]{\mathbf{w}}
\newcommand{\Bx}[0]{\mathbf{x}}
\newcommand{\By}[0]{\mathbf{y}}
\newcommand{\Bz}[0]{\mathbf{z}}
\newcommand{\BA}[0]{\mathbf{A}}
\newcommand{\BB}[0]{\mathbf{B}}
\newcommand{\BC}[0]{\mathbf{C}}
\newcommand{\BD}[0]{\mathbf{D}}
\newcommand{\BE}[0]{\mathbf{E}}
\newcommand{\BF}[0]{\mathbf{F}}
\newcommand{\BG}[0]{\mathbf{G}}
\newcommand{\BH}[0]{\mathbf{H}}
\newcommand{\BI}[0]{\mathbf{I}}
\newcommand{\BJ}[0]{\mathbf{J}}
\newcommand{\BK}[0]{\mathbf{K}}
\newcommand{\BL}[0]{\mathbf{L}}
\newcommand{\BM}[0]{\mathbf{M}}
\newcommand{\BN}[0]{\mathbf{N}}
\newcommand{\BO}[0]{\mathbf{O}}
\newcommand{\BP}[0]{\mathbf{P}}
\newcommand{\BQ}[0]{\mathbf{Q}}
\newcommand{\BR}[0]{\mathbf{R}}
\newcommand{\BS}[0]{\mathbf{S}}
\newcommand{\BT}[0]{\mathbf{T}}
\newcommand{\BU}[0]{\mathbf{U}}
\newcommand{\BV}[0]{\mathbf{V}}
\newcommand{\BW}[0]{\mathbf{W}}
\newcommand{\BX}[0]{\mathbf{X}}
\newcommand{\BY}[0]{\mathbf{Y}}
\newcommand{\BZ}[0]{\mathbf{Z}}

\newcommand{\Bzero}[0]{\mathbf{0}}
\newcommand{\Btheta}[0]{\boldsymbol{\theta}}
\newcommand{\Btau}[0]{\boldsymbol{\tau}}
\newcommand{\Bomega}[0]{\boldsymbol{\omega}}

%
% shorthand for unit vectors
%
\newcommand{\acap}[0]{\hat{\Ba}}
\newcommand{\bcap}[0]{\hat{\Bb}}
\newcommand{\ccap}[0]{\hat{\Bc}}
\newcommand{\dcap}[0]{\hat{\Bd}}
\newcommand{\ecap}[0]{\hat{\Be}}
\newcommand{\fcap}[0]{\hat{\Bf}}
\newcommand{\gcap}[0]{\hat{\Bg}}
\newcommand{\hcap}[0]{\hat{\Bh}}
\newcommand{\icap}[0]{\hat{\Bi}}
\newcommand{\jcap}[0]{\hat{\Bj}}
\newcommand{\kcap}[0]{\hat{\Bk}}
\newcommand{\lcap}[0]{\hat{\Bl}}
\newcommand{\mcap}[0]{\hat{\Bm}}
\newcommand{\ncap}[0]{\hat{\Bn}}
\newcommand{\ocap}[0]{\hat{\Bo}}
\newcommand{\pcap}[0]{\hat{\Bp}}
\newcommand{\qcap}[0]{\hat{\Bq}}
\newcommand{\rcap}[0]{\hat{\Br}}
\newcommand{\scap}[0]{\hat{\Bs}}
\newcommand{\tcap}[0]{\hat{\Bt}}
\newcommand{\ucap}[0]{\hat{\Bu}}
\newcommand{\vcap}[0]{\hat{\Bv}}
\newcommand{\wcap}[0]{\hat{\Bw}}
\newcommand{\xcap}[0]{\hat{\Bx}}
\newcommand{\ycap}[0]{\hat{\By}}
\newcommand{\zcap}[0]{\hat{\Bz}}
\newcommand{\thetacap}[0]{\hat{\Btheta}}

%
% to write R^n and C^n in a distinguishable fashion.  Perhaps change this
% to the double lined characters upon figuring out how to do so.
%
\newcommand{\C}[1]{$\mathbb{C}^{#1}$}
\newcommand{\R}[1]{$\mathbb{R}^{#1}$}

%
% various generally useful helpers
%

% derivative of #1 wrt. #2:
\newcommand{\D}[2] {\frac {d#2} {d#1}}

\newcommand{\inv}[1]{\frac{1}{#1}}
\newcommand{\cross}[0]{\times}

\newcommand{\abs}[1]{\lvert{#1}\rvert}
\newcommand{\norm}[1]{\lVert{#1}\rVert}
\newcommand{\innerprod}[2]{\langle{#1}, {#2}\rangle}
\newcommand{\dotprod}[2]{{#1} \cdot {#2}}
\newcommand{\bdotprod}[2]{\left({#1} \cdot {#2}\right)}
\newcommand{\crossprod}[2]{{#1} \cross {#2}}
\newcommand{\tripleprod}[3]{\dotprod{\left(\crossprod{#1}{#2}\right)}{#3}}

\DeclareMathOperator{\Proj}{Proj}
\DeclareMathOperator{\Span}{span}
\DeclareMathOperator{\Sgn}{sgn}
\DeclareMathOperator{\Area}{Area}
\DeclareMathOperator{\Volume}{Volume}

%
% A few miscellaneous things specific to this document
%
\newcommand{\crossop}[1]{\crossprod{#1}{}}

% R2 vector.
\newcommand{\VectorTwo}[2]{
\begin{bmatrix}
 {#1} \\
 {#2}
\end{bmatrix}
}

\newcommand{\VectorN}[1]{
\begin{bmatrix}
{#1}_1 \\
{#1}_2 \\
\vdots \\
{#1}_N \\
\end{bmatrix}
}

\newcommand{\DETuvij}[4]{
\begin{vmatrix}
 {#1}_{#3} & {#1}_{#4} \\
 {#2}_{#3} & {#2}_{#4}
\end{vmatrix}
}

\newcommand{\DETuvwijk}[6]{
\begin{vmatrix}
 {#1}_{#4} & {#1}_{#5} & {#1}_{#6} \\
 {#2}_{#4} & {#2}_{#5} & {#2}_{#6} \\
 {#3}_{#4} & {#3}_{#5} & {#3}_{#6}
\end{vmatrix}
}

\newcommand{\DETuvwxijkl}[8]{
\begin{vmatrix}
 {#1}_{#5} & {#1}_{#6} & {#1}_{#7} & {#1}_{#8} \\
 {#2}_{#5} & {#2}_{#6} & {#2}_{#7} & {#2}_{#8} \\
 {#3}_{#5} & {#3}_{#6} & {#3}_{#7} & {#3}_{#8} \\
 {#4}_{#5} & {#4}_{#6} & {#4}_{#7} & {#4}_{#8} \\
\end{vmatrix}
}

%\newcommand{\DETuvwxyijklm}[10]{
%\begin{vmatrix}
% {#1}_{#6} & {#1}_{#7} & {#1}_{#8} & {#1}_{#9} & {#1}_{#10} \\
% {#2}_{#6} & {#2}_{#7} & {#2}_{#8} & {#2}_{#9} & {#2}_{#10} \\
% {#3}_{#6} & {#3}_{#7} & {#3}_{#8} & {#3}_{#9} & {#3}_{#10} \\
% {#4}_{#6} & {#4}_{#7} & {#4}_{#8} & {#4}_{#9} & {#4}_{#10} \\
% {#5}_{#6} & {#5}_{#7} & {#5}_{#8} & {#5}_{#9} & {#5}_{#10}
%\end{vmatrix}
%}

% R3 vector.
\newcommand{\VectorThree}[3]{
\begin{bmatrix}
 {#1} \\
 {#2} \\
 {#3}
\end{bmatrix}
}


%<misc>
%
\newcommand{\Abs}[1]{{\left\lvert{#1}\right\rvert}}
\newcommand{\spacegrad}[0]{\boldsymbol{\nabla}}
\newcommand{\grad}[0]{\nabla}
\newcommand{\LL}[0]{\mathcal{L}}

% == \partial_{#1} {#2}
\newcommand{\PD}[2]{\frac{\partial {#2}}{\partial {#1}}}
% inline variant
\newcommand{\PDi}[2]{{\partial {#2}}/{\partial {#1}}}

\newcommand{\PDD}[3]{\frac{\partial^2 {#3}}{\partial {#1}\partial {#2}}}
%\newcommand{\PDd}[2]{\frac{\partial^2 {#2}}{{\partial{#1}}^2}}
\newcommand{\PDsq}[2]{\frac{\partial^2 {#2}}{(\partial {#1})^2}}

\newcommand{\Partial}[2]{\frac{\partial {#1}}{\partial {#2}}}
\DeclareMathOperator{\RejName}{Rej}
\newcommand{\Rej}[2]{\RejName_{#1}\left( {#2} \right)}
\newcommand{\Rm}[1]{\mathbb{R}^{#1}}
\newcommand{\Cm}[1]{\mathbb{C}^{#1}}
\newcommand{\conj}[0]{{*}}

%</misc>

% <grade selection>
%
\newcommand{\gpgrade}[2] {{\left\langle{{#1}}\right\rangle}_{#2}}

\newcommand{\gpgradezero}[1] {\gpgrade{#1}{}}
%\newcommand{\gpscalargrade}[1] {{\left\langle{{#1}}\right\rangle}}
%\newcommand{\gpgradezero}[1] {\gpgrade{#1}{0}}

%\newcommand{\gpgradeone}[1] {{\left\langle{{#1}}\right\rangle}_{1}}
\newcommand{\gpgradeone}[1] {\gpgrade{#1}{1}}

\newcommand{\gpgradetwo}[1] {\gpgrade{#1}{2}}
\newcommand{\gpgradethree}[1] {\gpgrade{#1}{3}}
\newcommand{\gpgradefour}[1] {\gpgrade{#1}{4}}
%
% </grade selection>



\newcommand{\adot}[0]{{\dot{a}}}
\newcommand{\bdot}[0]{{\dot{b}}}
% taken for centered dot:
%\newcommand{\cdot}[0]{{\dot{c}}}
%\newcommand{\ddot}[0]{{\dot{d}}}
\newcommand{\edot}[0]{{\dot{e}}}
\newcommand{\fdot}[0]{{\dot{f}}}
\newcommand{\gdot}[0]{{\dot{g}}}
\newcommand{\hdot}[0]{{\dot{h}}}
\newcommand{\idot}[0]{{\dot{i}}}
\newcommand{\jdot}[0]{{\dot{j}}}
\newcommand{\kdot}[0]{{\dot{k}}}
\newcommand{\ldot}[0]{{\dot{l}}}
\newcommand{\mdot}[0]{{\dot{m}}}
\newcommand{\ndot}[0]{{\dot{n}}}
%\newcommand{\odot}[0]{{\dot{o}}}
\newcommand{\pdot}[0]{{\dot{p}}}
\newcommand{\qdot}[0]{{\dot{q}}}
\newcommand{\rdot}[0]{{\dot{r}}}
\newcommand{\sdot}[0]{{\dot{s}}}
\newcommand{\tdot}[0]{{\dot{t}}}
\newcommand{\udot}[0]{{\dot{u}}}
\newcommand{\vdot}[0]{{\dot{v}}}
\newcommand{\wdot}[0]{{\dot{w}}}
\newcommand{\xdot}[0]{{\dot{x}}}
\newcommand{\ydot}[0]{{\dot{y}}}
\newcommand{\zdot}[0]{{\dot{z}}}
\newcommand{\addot}[0]{{\ddot{a}}}
\newcommand{\bddot}[0]{{\ddot{b}}}
\newcommand{\cddot}[0]{{\ddot{c}}}
%\newcommand{\dddot}[0]{{\ddot{d}}}
\newcommand{\eddot}[0]{{\ddot{e}}}
\newcommand{\fddot}[0]{{\ddot{f}}}
\newcommand{\gddot}[0]{{\ddot{g}}}
\newcommand{\hddot}[0]{{\ddot{h}}}
\newcommand{\iddot}[0]{{\ddot{i}}}
\newcommand{\jddot}[0]{{\ddot{j}}}
\newcommand{\kddot}[0]{{\ddot{k}}}
\newcommand{\lddot}[0]{{\ddot{l}}}
\newcommand{\mddot}[0]{{\ddot{m}}}
\newcommand{\nddot}[0]{{\ddot{n}}}
\newcommand{\oddot}[0]{{\ddot{o}}}
\newcommand{\pddot}[0]{{\ddot{p}}}
\newcommand{\qddot}[0]{{\ddot{q}}}
\newcommand{\rddot}[0]{{\ddot{r}}}
\newcommand{\sddot}[0]{{\ddot{s}}}
\newcommand{\tddot}[0]{{\ddot{t}}}
\newcommand{\uddot}[0]{{\ddot{u}}}
\newcommand{\vddot}[0]{{\ddot{v}}}
\newcommand{\wddot}[0]{{\ddot{w}}}
\newcommand{\xddot}[0]{{\ddot{x}}}
\newcommand{\yddot}[0]{{\ddot{y}}}
\newcommand{\zddot}[0]{{\ddot{z}}}

%<bold and dot greek symbols>
%

\newcommand{\Deltadot}[0]{{\dot{\Delta}}}
\newcommand{\Gammadot}[0]{{\dot{\Gamma}}}
\newcommand{\Lambdadot}[0]{{\dot{\Lambda}}}
\newcommand{\Omegadot}[0]{{\dot{\Omega}}}
\newcommand{\Phidot}[0]{{\dot{\Phi}}}
\newcommand{\Pidot}[0]{{\dot{\Pi}}}
\newcommand{\Psidot}[0]{{\dot{\Psi}}}
\newcommand{\Sigmadot}[0]{{\dot{\Sigma}}}
\newcommand{\Thetadot}[0]{{\dot{\Theta}}}
\newcommand{\Upsilondot}[0]{{\dot{\Upsilon}}}
\newcommand{\Xidot}[0]{{\dot{\Xi}}}
\newcommand{\alphadot}[0]{{\dot{\alpha}}}
\newcommand{\betadot}[0]{{\dot{\beta}}}
\newcommand{\chidot}[0]{{\dot{\chi}}}
\newcommand{\deltadot}[0]{{\dot{\delta}}}
\newcommand{\epsilondot}[0]{{\dot{\epsilon}}}
\newcommand{\etadot}[0]{{\dot{\eta}}}
\newcommand{\gammadot}[0]{{\dot{\gamma}}}
\newcommand{\kappadot}[0]{{\dot{\kappa}}}
\newcommand{\lambdadot}[0]{{\dot{\lambda}}}
\newcommand{\mudot}[0]{{\dot{\mu}}}
\newcommand{\nudot}[0]{{\dot{\nu}}}
\newcommand{\omegadot}[0]{{\dot{\omega}}}
\newcommand{\phidot}[0]{{\dot{\phi}}}
\newcommand{\pidot}[0]{{\dot{\pi}}}
\newcommand{\psidot}[0]{{\dot{\psi}}}
\newcommand{\rhodot}[0]{{\dot{\rho}}}
\newcommand{\sigmadot}[0]{{\dot{\sigma}}}
\newcommand{\taudot}[0]{{\dot{\tau}}}
\newcommand{\thetadot}[0]{{\dot{\theta}}}
\newcommand{\upsilondot}[0]{{\dot{\upsilon}}}
\newcommand{\varepsilondot}[0]{{\dot{\varepsilon}}}
\newcommand{\varphidot}[0]{{\dot{\varphi}}}
\newcommand{\varpidot}[0]{{\dot{\varpi}}}
\newcommand{\varrhodot}[0]{{\dot{\varrho}}}
\newcommand{\varsigmadot}[0]{{\dot{\varsigma}}}
\newcommand{\varthetadot}[0]{{\dot{\vartheta}}}
\newcommand{\xidot}[0]{{\dot{\xi}}}
\newcommand{\zetadot}[0]{{\dot{\zeta}}}

\newcommand{\Deltaddot}[0]{{\ddot{\Delta}}}
\newcommand{\Gammaddot}[0]{{\ddot{\Gamma}}}
\newcommand{\Lambdaddot}[0]{{\ddot{\Lambda}}}
\newcommand{\Omegaddot}[0]{{\ddot{\Omega}}}
\newcommand{\Phiddot}[0]{{\ddot{\Phi}}}
\newcommand{\Piddot}[0]{{\ddot{\Pi}}}
\newcommand{\Psiddot}[0]{{\ddot{\Psi}}}
\newcommand{\Sigmaddot}[0]{{\ddot{\Sigma}}}
\newcommand{\Thetaddot}[0]{{\ddot{\Theta}}}
\newcommand{\Upsilonddot}[0]{{\ddot{\Upsilon}}}
\newcommand{\Xiddot}[0]{{\ddot{\Xi}}}
\newcommand{\alphaddot}[0]{{\ddot{\alpha}}}
\newcommand{\betaddot}[0]{{\ddot{\beta}}}
\newcommand{\chiddot}[0]{{\ddot{\chi}}}
\newcommand{\deltaddot}[0]{{\ddot{\delta}}}
\newcommand{\epsilonddot}[0]{{\ddot{\epsilon}}}
\newcommand{\etaddot}[0]{{\ddot{\eta}}}
\newcommand{\gammaddot}[0]{{\ddot{\gamma}}}
\newcommand{\kappaddot}[0]{{\ddot{\kappa}}}
\newcommand{\lambdaddot}[0]{{\ddot{\lambda}}}
\newcommand{\muddot}[0]{{\ddot{\mu}}}
\newcommand{\nuddot}[0]{{\ddot{\nu}}}
\newcommand{\omegaddot}[0]{{\ddot{\omega}}}
\newcommand{\phiddot}[0]{{\ddot{\phi}}}
\newcommand{\piddot}[0]{{\ddot{\pi}}}
\newcommand{\psiddot}[0]{{\ddot{\psi}}}
\newcommand{\rhoddot}[0]{{\ddot{\rho}}}
\newcommand{\sigmaddot}[0]{{\ddot{\sigma}}}
\newcommand{\tauddot}[0]{{\ddot{\tau}}}
\newcommand{\thetaddot}[0]{{\ddot{\theta}}}
\newcommand{\upsilonddot}[0]{{\ddot{\upsilon}}}
\newcommand{\varepsilonddot}[0]{{\ddot{\varepsilon}}}
\newcommand{\varphiddot}[0]{{\ddot{\varphi}}}
\newcommand{\varpiddot}[0]{{\ddot{\varpi}}}
\newcommand{\varrhoddot}[0]{{\ddot{\varrho}}}
\newcommand{\varsigmaddot}[0]{{\ddot{\varsigma}}}
\newcommand{\varthetaddot}[0]{{\ddot{\vartheta}}}
\newcommand{\xiddot}[0]{{\ddot{\xi}}}
\newcommand{\zetaddot}[0]{{\ddot{\zeta}}}

\newcommand{\BDelta}[0]{\boldsymbol{\Delta}}
\newcommand{\BGamma}[0]{\boldsymbol{\Gamma}}
\newcommand{\BLambda}[0]{\boldsymbol{\Lambda}}
\newcommand{\BOmega}[0]{\boldsymbol{\Omega}}
\newcommand{\BPhi}[0]{\boldsymbol{\Phi}}
\newcommand{\BPi}[0]{\boldsymbol{\Pi}}
\newcommand{\BPsi}[0]{\boldsymbol{\Psi}}
\newcommand{\BSigma}[0]{\boldsymbol{\Sigma}}
\newcommand{\BTheta}[0]{\boldsymbol{\Theta}}
\newcommand{\BUpsilon}[0]{\boldsymbol{\Upsilon}}
\newcommand{\BXi}[0]{\boldsymbol{\Xi}}
\newcommand{\Balpha}[0]{\boldsymbol{\alpha}}
\newcommand{\Bbeta}[0]{\boldsymbol{\beta}}
\newcommand{\Bchi}[0]{\boldsymbol{\chi}}
\newcommand{\Bdelta}[0]{\boldsymbol{\delta}}
\newcommand{\Bepsilon}[0]{\boldsymbol{\epsilon}}
\newcommand{\Beta}[0]{\boldsymbol{\eta}}
\newcommand{\Bgamma}[0]{\boldsymbol{\gamma}}
\newcommand{\Bkappa}[0]{\boldsymbol{\kappa}}
\newcommand{\Blambda}[0]{\boldsymbol{\lambda}}
\newcommand{\Bmu}[0]{\boldsymbol{\mu}}
\newcommand{\Bnu}[0]{\boldsymbol{\nu}}
%\newcommand{\Bomega}[0]{\boldsymbol{\omega}}
\newcommand{\Bphi}[0]{\boldsymbol{\phi}}
\newcommand{\Bpi}[0]{\boldsymbol{\pi}}
\newcommand{\Bpsi}[0]{\boldsymbol{\psi}}
\newcommand{\Brho}[0]{\boldsymbol{\rho}}
\newcommand{\Bsigma}[0]{\boldsymbol{\sigma}}
%\newcommand{\Btau}[0]{\boldsymbol{\tau}}
%\newcommand{\Btheta}[0]{\boldsymbol{\theta}}
\newcommand{\Bupsilon}[0]{\boldsymbol{\upsilon}}
\newcommand{\Bvarepsilon}[0]{\boldsymbol{\varepsilon}}
\newcommand{\Bvarphi}[0]{\boldsymbol{\varphi}}
\newcommand{\Bvarpi}[0]{\boldsymbol{\varpi}}
\newcommand{\Bvarrho}[0]{\boldsymbol{\varrho}}
\newcommand{\Bvarsigma}[0]{\boldsymbol{\varsigma}}
\newcommand{\Bvartheta}[0]{\boldsymbol{\vartheta}}
\newcommand{\Bxi}[0]{\boldsymbol{\xi}}
\newcommand{\Bzeta}[0]{\boldsymbol{\zeta}}
%
%</bold and dot greek symbols>
%<infrequent>
%
%\newcommand{\AreaOp}[1]{\AName_{#1}}
%\newcommand{\Babs}[0]{\abs{\BB}}
%\newcommand{\Bcap}[0]{\hat{\BB}}
%\newcommand{\BrPrimeRej}[0]{\rcap(\rcap \wedge \Br')}
%\newcommand{\CA}[0]{\mathcal{A}}
%\newcommand{\Cos}[1]{\cos{\left({#1}\right)}}
%\newcommand{\Det}[1] {\abs{#1}}
%\newcommand{\Dsq}[2] {\frac {\partial^2 {#1}} {\partial {#2}^2}}
%\newcommand{\Exp}[1]{\exp{\left({#1}\right)}}
%\newcommand{\Norm}[1]{\left\lVert{#1}\right\rVert}
%\newcommand{\Sin}[1]{\sin{\left({#1}\right)}}
%\newcommand{\T}[0]{\text{T}}
%\newcommand{\VolumeOp}[1]{\VName_{#1}}
%\newcommand{\agrad}[0]{\Ba \cdot \nabla}
%\newcommand{\alphacap}[0]{\hat{\boldsymbol{\alpha}}}
%\newcommand{\Fcap}[0]{\hat{\BF}}
%\newcommand{\bithree}[0]{{\Bi}_3}
%\newcommand{\bxa}[0]{\Bx\Ba}
%\newcommand{\coordvec}[2]{
%\newcommand{\costheta}[0]{\acap \cdot \xcap}
%\newcommand{\ddt}[1]{\ddot{#1}}
%\newcommand{\ddu}[1] {\frac {d{#1}} {du}}
%\newcommand{\dsqxj}[2] {\frac {\partial^2 {#1}} {\partial {x_{#2}}^2}}
%\newcommand{\dtheta}[1]{\frac{d {#1}}{d \theta}}
%\newcommand{\dt}[1]{\dot{#1}}
%\newcommand{\dt}[1]{\frac{d {#1}}{dt}}
%\newcommand{\dxj}[2] {\frac {\partial {#1}} {\partial {x_{#2}}}}
%\newcommand{\halfPhi}[0]{\frac{\phi}{2}}
%\newcommand{\half}[0]{\inv{2}}
%\newcommand{\inv}[1]{\frac{1}{#1}}
%\newcommand{\laplacian}[0]{\nabla^2}
%\newcommand{\matrixoftx}[3]{
%\newcommand{\nrrp}[0]{\norm{\rcap \wedge \Br'}}
%\newcommand{\oiint}{\bigcirc \hspace{-1.4em} \int \hspace{-.8em} \int}
%\newcommand{\transpose}[1]{{#1}^{\text{T}}}
%\newcommand{\transpose}[1]{{{#1}^{\TextTranspose}}}
%\newcommand{\transpose}[1]{{{#1}^{\text{T}}}}
%\newcommand{\barA}[0]{\bar{A}}
%\newcommand{\qbar}[0]{\bar{q}}
%\newcommand{\qdotbar}[0]{\dot{\bar{q}}}
%
%</infrequent>





\usepackage[bookmarks=true]{hyperref}

\usepackage{color,cite,graphicx}
   % use colour in the document, put your citations as [1-4]
   % rather than [1,2,3,4] (it looks nicer, and the extended LaTeX2e
   % graphics package. 
\usepackage{latexsym,amssymb,epsf} % don't remember if these are
   % needed, but their inclusion can't do any damage

\title{ Field and wave energy and momentum. }
\author{Peeter Joot}
\date{ Jan 03, 2009.  Last Revision: $Date: 2009/01/04 16:59:26 $ }

\begin{document}

\maketitle{}

%\tableofcontents
\section{ Motivation. }

The concept of energy in the electric and magnetic fields I am getting closer to understanding, but there�s a few ways that I would like to approach it.

I've now explored the Poynting vector energy conservation relationships in 
\cite{PJpoynting}, and
\cite{PJemstresstensor}
, but hadn't understood fully where the energy expressions in the electro and magneto statics cases came from
separately.  I also don't yet know where the $F \gamma_k F$ terms of the stress tensor fit in the big picture?  I suspect that they can be obtained by Lorentz transforming the rest frame expression $F \gamma_0 F$ (the energy density, Poynting momentum density four vector).

It also ought to be possible to relate the field energies to a Lagrangian and Hamiltonian, but I haven't had success doing so.

The last thing that I'd like to understand is how the energy and momentum of a wave can be expressed, both in terms of the abstract conguate field momentum concept and with a concrete example such as the one dimensional oscillating rod that can be treated in a limiting coupled oscillator approach as in 
\cite{goldstein1951cm}.

Once I've got all these down I think I'll be ready to revisit Bohm's Rayleigh-Jeans law treatment in \cite{bohm1989qt}.  Unfortunately, each time I try persuing some 
interesting aspect of QM I find that I end up back studying electrodynamics, and suspect that needs to be my focus for the forseeable future (perhaps working throughly through Feynman's
volume II).

\section{ Electrostatic energy in a field. } 

Feynman�s treatment in 
\cite{feynman1963flp}
of the energy $\frac{\epsilon_0}{2}\BE^2$ associated with the electrostatic $\BE$ field is very easy to understand.  Here is a write up of this myself without looking at the book to see if I really understood the ideas.

The first step is consideration of the force times distance for two charges gives you the energy required (or gained) by moving one of those charges from infinity to some given separation

\begin{align*}
W &= \frac{1}{4\pi\epsilon_0} \int_{\infty}^{r} \frac{q_1 q_2}{x^2} \Be_1 \cdot (-\Be_1 dx) \\
&= \frac{q_1 q_2}{4 \pi \epsilon_0 r}
\end{align*}

This provides a quantization for an energy in a field concept.  A distribution of charge requires energy to assemble and it is possible to enumerate that energy separately by considering all the charges, or alternatively, by not looking at the final charge distribution, but only considering the net field associated with this charge distribution.  This is a pretty powerful, but somewhat abstract seeming idea.

The generalization to continuous charge distribution from there was pretty straightforward, requiring a double integration over all space

\begin{align*}
W &= \frac{1}{2} \int \frac{1}{4\pi\epsilon_0} \frac{\rho_1 dV_1 \rho_2 dV_2}{r_{12}}  \\
&= \frac{1}{2} \int \rho_1 \phi_2 dV_1
\end{align*}

The $1/2$ factor was due to double counting all "pairs" of charge elements.  The next step was to rewrite the charge density by using Maxwell's equations.  In terms of the four vector potential Maxwell's equation (with the $\grad \cdot A = 0$ gauge) is

\begin{align*}
\grad^2 A = \inv{\epsilon_0 c}( c \rho \gamma_0 + J^k \gamma_k)
\end{align*}

So, to write $\rho$ in terms of potential $A^0 = \phi$, we have

\begin{align*}
\left(\inv{c^2}\PDsq{t}{} - \spacegrad^2\right) \phi = \inv{\epsilon_0} \rho 
\end{align*}

In the statics case, where $\PD{t}{\phi} = 0$, we can thus write the charge density in terms of the potential

\begin{align*}
\rho = -\epsilon_0 \spacegrad^2 \phi
\end{align*}

and substitute back into the energy summation

\begin{align*}
W 
&= \frac{1}{2} \int \rho \phi dV \\
&= \frac{-\epsilon_0}{2} \int \phi \spacegrad^2 \phi dV \\
\end{align*}

Now, Feynman's last step was a bit sneaky, which was to convert the $\phi \spacegrad^2 \phi$ term into a divergence integral.  Working backwards to derive the identity that he used

\begin{align*}
\spacegrad \cdot (\phi \spacegrad \phi) 
&= \gpgradezero{ \spacegrad (\phi \spacegrad \phi) } \\
&= \gpgradezero{ (\spacegrad \phi) \spacegrad \phi + \phi \spacegrad (\spacegrad \phi) } \\
&= (\spacegrad \phi)^2 + \phi \spacegrad^2 \phi \\
\end{align*}

This can then be used with Stokes theorem in its dual form to convert our $\phi \spacegrad^2 \phi$ the into plain volume and surface integral
\begin{align*}
W 
&= \frac{\epsilon_0}{2} \int \left( (\spacegrad \phi)^2 -\spacegrad \cdot (\phi \spacegrad \phi) \right) dV \\
&= \frac{\epsilon_0}{2} \int (\spacegrad \phi)^2 dV - \frac{\epsilon_0}{2} \int_{\partial V} (\phi \spacegrad \phi) \cdot \ncap dA \\
\end{align*}

Letting the surface go to infinity and employing a limiting argument on the magnitudes of the $\phi$ and $\spacegrad \phi$ terms was enough to produce the final electrostatics 
energy result

\begin{align*}
W 
&= \frac{\epsilon_0}{2} \int (\spacegrad \phi)^2 dV \\
&= \frac{\epsilon_0}{2} \int \BE^2 dV 
\end{align*}

\section{ Magnetostatic field energy. } 

Feynman's energy discussion of the magnetic field for a constant current loop (magnetostatics), is not so easy to follow.  He considers the dipole moment of a small loop, obtained by comparision to previous electrostatic results (that I'd have to go back and read or re-derive) and some
subtle seeming arguments about the mechanical vs. total energy of the system.

As an attempt to understand all this, let's break it up into pieces.  First, is calculation of the field for a current loop.
Going back to
Maxwell's equation (with the $\grad \cdot A$ gauge again) will show us how this Biot-Savart calculation is expressed in the STA bivector form

\begin{align*}
\grad^2 A^\mu = J^\mu/\epsilon_0 c
\end{align*}

For a constant current situation, we have $\PDi{t}{A^\mu} = 0$, and our vector potential equations are

\begin{align*}
\spacegrad^2 A^k = -J^k/\epsilon_0 c
\end{align*}

Recall that the solution of $A^k$ can be expressed as the impulse response of a function of the following form

\begin{align*}
A^k = C\inv{r}
\end{align*}

and that $\spacegrad \cdot (\spacegrad (1/r))$ is zero for all $r \ne 0$.  Performing a volume integral of the expected Laplacian we
can integrate over an infinitesimal spherical volume of radius $R$

\begin{align*}
\int \spacegrad^2 A^k dV 
&= C \int \spacegrad \cdot \spacegrad \inv{r} dV \\
&= C \int \spacegrad \cdot \left( -\rcap \inv{r^2} \right) dV \\
&= -C \int_{\partial_V} \rcap \inv{r^2} \cdot \rcap dA \\
&= -C \inv{R^2} 4 \pi R^2 \\
&= - 4 \pi C \\
\end{align*}

Equating we can solve for $C$

\begin{align*}
- 4 \pi C &= -J^k/\epsilon_0 c \\
C &= \inv{4 \pi \epsilon_0 c} J^k
\end{align*}

Note that this is cheating slightly since C was kind of treated as a constant, whereas this equality makes it a function.  It works because
the infinitesimal volume can be made small enough that $J^k$ can be treated as a constant.  This therefore provides our 
potential function in terms of this impulse response

\begin{align*}
A^k &= \int dV \inv{4 \pi \epsilon_0 c r} J^k
\end{align*}

Now, this could have all been done with a comparison to the electrostatic result.  Regardless, it now leaves us in the position to
calcuate the field bivector

\begin{align*}
F 
&= \grad \wedge A \\
&= (\gamma^\mu \wedge \gamma_k) \partial_\mu A^k \\
&= -(\gamma_m \wedge \gamma_k) \partial_m A^k \\
&= (\sigma_m \wedge \sigma_k) \partial_m A^k \\
\end{align*}

Now, 
\begin{align*}
\partial_m A^k
&= \inv{4 \pi \epsilon_0 c} \int dV \partial_m \frac{J^k}{r} \\
&= \inv{4 \pi \epsilon_0 c} \int dV \left( J^k \partial_m \inv{r} + \inv{r} \partial_m {J^k} \right) \\
&= \inv{4 \pi \epsilon_0 c} \int dV \left( J^k \partial_m \left(\sum_j ((x^j)^2)^{-1/2}\right) + \inv{r} \partial_m {J^k} \right) \\
&= \inv{4 \pi \epsilon_0 c} \int dV \left( J^k \left(-\inv{2}\right) 2 x^m \inv{r^3} + \inv{r} \partial_m {J^k} \right) \\
&= \inv{4 \pi \epsilon_0 c} \int \inv{r^3} dV \left( -x^m J^k + r^2 \partial_m {J^k} \right) \\
\end{align*}

So with $\Bj = J^k \sigma_k$ we have

\begin{align*}
F 
&= \inv{4 \pi \epsilon_0 c} \int \inv{r^3} dV \left( -\Br \wedge \Bj + r^2 (\spacegrad \wedge \Bj) \right) \\
&= \inv{4 \pi \epsilon_0 c} \int dV \left( \frac{\Bj \wedge \rcap}{r^2} + \inv{r}(\spacegrad \wedge \Bj) \right) \\
\end{align*}

The first term here is essentially the Biot Savart law once the current density is converted to current $\int \Bj dV = I \int \jcap dl$, so we expect the second term to be zero.  How to show this?

\section{ Complete field energy. } 

TODO: Can a integral of the Lorentz force coupled with Maxwell's equations in their entirety produce the energy expression $\frac{\epsilon_0}{2}\left(\BE^2 + c^2\BB^2\right)$?  It seems like cheating to add these arbitrarily and then follow the Poynting derivation by taking derivatives.  That shows this quantity is a conserved quantity, but does it really show that it is the
energy?  One could imagine that there could be other terms in a total energy expression such as $\BE \cdot \BB$.

\bibliographystyle{plainnat}
\bibliography{myrefs}

\end{document}

%
% Copyright � 2012 Peeter Joot.  All Rights Reserved.
% Licenced as described in the file LICENSE under the root directory of this GIT repository.
%

%
%
\chapter{Energy momentum tensor}\label{chap:PJemstresstensor}
%\date{Jan 01, 2009.  energyMomentumTensor.tex}

\section{Expanding out the stress energy vector in tensor form}

\citep{doran2003gap} defines (with \(\epsilon_0\) omitted),
the energy momentum stress tensor as a vector to
vector mapping of the following form:

\begin{equation}\label{eqn:energyMomentumTensor:20}
\begin{aligned}
T(a)
&= \frac{\epsilon_0}{2} F a \tilde{F}
= - \frac{\epsilon_0}{2} F a F
\end{aligned}
\end{equation}

This quantity can only have vector, trivector, and five vector grades.  The
grade five term must be zero

\begin{equation}\label{eqn:energyMomentumTensor:40}
\begin{aligned}
\gpgrade{T(a)}{5}
&= \frac{\epsilon_0}{2} F \wedge a \wedge \tilde{F} \\
&= \frac{\epsilon_0}{2} a \wedge (F \wedge \tilde{F}) \\
&= 0
\end{aligned}
\end{equation}

Since \((T(a))^{\tilde{}} = T(a)\), the grade three term is also zero (trivectors invert on reversion), so this must therefore be a vector.

As a vector this can be expanded in coordinates

\begin{equation}\label{eqn:energyMomentumTensor:60}
\begin{aligned}
T(a)
&= \left(T(a) \cdot \gamma^\nu \right) \gamma_\nu \\
&= \left(T(a^\mu \gamma_\mu) \cdot \gamma^\nu \right) \gamma_\nu \\
&= a^\mu \gamma_\nu \left(T(\gamma_\mu) \cdot \gamma^\nu \right) \\
\end{aligned}
\end{equation}

It is this last bit that has the form of a traditional tensor, so we can write

\begin{equation}\label{eqn:energyMomentumTensor:80}
\begin{aligned}
T(a) &= a^\mu \gamma_\nu {T_\mu}^{\nu} \\
{T_\mu}^{\nu} &= T(\gamma_\mu) \cdot \gamma^\nu
\end{aligned}
\end{equation}

Let us expand this tensor \({T_\mu}^{\nu}\) explicitly to verify its form.

We want to expand, and dot with \(\gamma^\nu\), the following

\begin{equation}\label{eqn:energyMomentumTensor:100}
\begin{aligned}
-2 \inv{\epsilon_0} \left(T(\gamma_\mu) \cdot \gamma^\nu \right) \gamma_\nu
&= \gpgradeone{(\grad \wedge A) \gamma_\mu (\grad \wedge A)} \\
&= \gpgradeone{
(\grad \wedge A) \cdot \gamma_\mu (\grad \wedge A)
+ (\grad \wedge A) \wedge \gamma^\mu (\grad \wedge A)
} \\
&=
((\grad \wedge A) \cdot \gamma_\mu) \cdot (\grad \wedge A)
+ ((\grad \wedge A) \wedge \gamma_\mu) \cdot (\grad \wedge A)
\\
\end{aligned}
\end{equation}

Both of these will get temporarily messy, so let us do them in parts.  Starting
with

\begin{equation}\label{eqn:energyMomentumTensor:120}
\begin{aligned}
(\grad \wedge A) \cdot \gamma_\mu
&= (\gamma^{\alpha} \wedge \gamma^{\beta}) \cdot \gamma_{\mu} \partial_{\alpha} A_{\beta} \\
&= (\gamma^{\alpha} {\delta^{\beta}}_{\mu} -\gamma^{\beta} {\delta^{\alpha}}_{\mu} ) \partial_{\alpha} A_{\beta} \\
&=
\gamma^{\alpha} \partial_{\alpha} A_{\mu}
-\gamma^{\beta} \partial_{\mu} A_{\beta} \\
&= \gamma^{\alpha} (\partial_{\alpha} A_{\mu} -\partial_{\mu} A_{\alpha} ) \\
&= \gamma^{\alpha} F_{\alpha \mu} \\
\end{aligned}
\end{equation}


\begin{equation}\label{eqn:energyMomentumTensor:140}
\begin{aligned}
((\grad \wedge A) \cdot \gamma_\mu) \cdot (\grad \wedge A)
&= (\gamma^{\alpha} F_{\alpha \mu}) \cdot (\gamma_{\beta} \wedge \gamma_{\lambda}) \partial^{\beta} A^{\lambda} \\
%&=
%\partial^{\beta} A^{\lambda} F_{\alpha \mu}
%\gamma^{\alpha} \cdot (\gamma_{\beta} \wedge \gamma_{\lambda})
%\\
&=
\partial^{\beta} A^{\lambda} F_{\alpha \mu}
(
{\delta^{\alpha}}_{\beta} \gamma_{\lambda}
-{\delta^{\alpha}}_{\lambda} \gamma_{\beta}
)
\\
&=
(\partial^{\alpha} A^{\beta} F_{\alpha \mu} -\partial^{\beta} A^{\alpha} F_{\alpha \mu} )\gamma_{\beta}
\\
&= F^{\alpha \beta} F_{\alpha \mu} \gamma_{\beta} \\
\end{aligned}
\end{equation}

So, by dotting with \(\gamma^\nu\) we have

\begin{equation}\label{eqn:energy_momentum_tensor:firstPartDone}
\begin{aligned}
((\grad \wedge A) \cdot \gamma_\mu) \cdot (\grad \wedge A) \cdot \gamma^{\nu}
&= F^{\alpha \nu} F_{\alpha \mu}
\end{aligned}
\end{equation}

Moving on to the next bit,
\((((\grad \wedge A) \wedge \gamma^\mu) \cdot (\grad \wedge A)) \cdot \gamma^\nu\).� By inspection the first part of this is

\begin{equation}\label{eqn:energyMomentumTensor:160}
\begin{aligned}
(\grad \wedge A) \wedge \gamma_\mu
&= (\gamma_\mu)^2 (\gamma^{\alpha} \wedge \gamma^{\beta}) \wedge \gamma^{\mu} \partial_{\alpha} A_{\beta} \\
\end{aligned}
\end{equation}

so dotting with \(\grad \wedge A\), we have

\begin{equation}\label{eqn:energyMomentumTensor:180}
\begin{aligned}
((\grad \wedge A) \wedge \gamma_\mu) \cdot (\grad \wedge A)
&=
(\gamma_\mu)^2
\partial_{\alpha} A_{\beta}
\partial^{\lambda} A^{\delta}
(\gamma^{\alpha} \wedge \gamma^{\beta} \wedge \gamma^{\mu}) \cdot
(\gamma_{\lambda} \wedge \gamma_{\delta})
\\
&=
(\gamma_\mu)^2
\partial_{\alpha} A_{\beta}
\partial^{\lambda} A^{\delta}
((\gamma^{\alpha} \wedge \gamma^{\beta} \wedge \gamma^{\mu}) \cdot \gamma_{\lambda} ) \cdot \gamma_{\delta}
\\
\end{aligned}
\end{equation}

Expanding just the dot product parts of this we have
\begin{equation}\label{eqn:energyMomentumTensor:200}
\begin{aligned}
&(((\gamma^{\alpha} \wedge \gamma^{\beta}) \wedge \gamma^{\mu}) \cdot \gamma_{\lambda} ) \cdot \gamma_{\delta} \\
&=
(\gamma^{\alpha} \wedge \gamma^{\beta}) {\delta^{\mu}}_{\lambda}
-(\gamma^{\alpha} \wedge \gamma^{\mu}) {\delta^{\beta}}_{\lambda}
+(\gamma^{\beta} \wedge \gamma^{\mu}) {\delta^{\alpha}}_{\lambda}
) \cdot \gamma_{\delta}
\\
%&=
%(
%  \gamma^{\alpha} {\delta^{\beta}}_{\delta} {\delta^{\mu}}_{\lambda}
%- \gamma^{\alpha} {\delta^{\mu}}_{\delta} {\delta^{\beta}}_{\lambda}
%+ \gamma^{\beta} {\delta^{\mu}}_{\delta} {\delta^{\alpha}}_{\lambda}
%- \gamma^{\beta} {\delta^{\alpha}}_{\delta} {\delta^{\mu}}_{\lambda}
%+ \gamma^{\mu} {\delta^{\alpha}}_{\delta} {\delta^{\beta}}_{\lambda}
%- \gamma^{\mu} {\delta^{\beta}}_{\delta} {\delta^{\alpha}}_{\lambda}
%)
%\\
&=
  \gamma^{\alpha} ({\delta^{\beta}}_{\delta} {\delta^{\mu}}_{\lambda}
-            {\delta^{\mu}}_{\delta} {\delta^{\beta}}_{\lambda})
+ \gamma^{\beta} ({\delta^{\mu}}_{\delta} {\delta^{\alpha}}_{\lambda}
-            {\delta^{\alpha}}_{\delta} {\delta^{\mu}}_{\lambda})
+ \gamma^{\mu} ({\delta^{\alpha}}_{\delta} {\delta^{\beta}}_{\lambda}
-                 {\delta^{\beta}}_{\delta} {\delta^{\alpha}}_{\lambda})
\\
\end{aligned}
\end{equation}

This can now be applied to \(\partial^{\lambda} A^{\delta}\)

\begin{equation}\label{eqn:energyMomentumTensor:220}
\begin{aligned}
\partial^{\lambda} A^{\delta} &(((\gamma^{\alpha} \wedge \gamma^{\beta}) \wedge \gamma^{\mu}) \cdot \gamma_{\lambda} ) \cdot \gamma_{\delta} \\
&=
  \partial^{\mu} A^{\beta} \gamma^{\alpha}
- \partial^{\beta} A^{\mu} \gamma^{\alpha}
+ \partial^{\alpha} A^{\mu} \gamma^{\beta}
- \partial^{\mu} A^{\alpha} \gamma^{\beta}
+ \partial^{\beta} A^{\alpha} \gamma^{\mu}
- \partial^{\alpha} A^{\beta} \gamma^{\mu}
\\
&=
(  \partial^{\mu} A^{\beta}
- \partial^{\beta} A^{\mu} ) \gamma^{\alpha}
+( \partial^{\alpha} A^{\mu}
- \partial^{\mu} A^{\alpha} ) \gamma^{\beta}
+( \partial^{\beta} A^{\alpha}
- \partial^{\alpha} A^{\beta} ) \gamma^{\mu}
\\
&=
%(  \partial^{\mu} A^{\beta} - \partial^{\beta} A^{\mu} )
F^{\mu \beta}
\gamma^{\alpha}
+
%( \partial^{\alpha} A^{\mu} - \partial^{\mu} A^{\alpha} )
F^{\alpha \mu}
\gamma^{\beta}
+
%( \partial^{\beta} A^{\alpha} - \partial^{\alpha} A^{\beta} )
F^{\beta \alpha}
\gamma^{\mu}
\\
\end{aligned}
\end{equation}

This is getting closer, and we can now write
\begin{equation}\label{eqn:energyMomentumTensor:240}
\begin{aligned}
((\grad \wedge A) \wedge \gamma_\mu) \cdot (\grad \wedge A) &=
(\gamma_\mu)^2 \partial_{\alpha} A_{\beta}
(
  F^{\mu \beta} \gamma^{\alpha}
+ F^{\alpha \mu} \gamma^{\beta}
+ F^{\beta \alpha} \gamma^{\mu}
) \\
&=
  (\gamma_\mu)^2 \partial_{\beta} A_{\alpha} F^{\mu \alpha} \gamma^{\beta}
+ (\gamma_\mu)^2 \partial_{\alpha} A_{\beta} F^{\alpha \mu} \gamma^{\beta}
+ (\gamma_\mu)^2 \partial_{\alpha} A_{\beta} F^{\beta \alpha} \gamma^{\mu}
\\
&=
  F^{\beta \alpha} F_{\mu \alpha} \gamma_{\beta}
+ \partial_{\alpha} A_{\beta} F^{\beta \alpha} \gamma_{\mu}
\\
\end{aligned}
\end{equation}

This can now be dotted with \(\gamma^\nu\),

\begin{equation}\label{eqn:energyMomentumTensor:260}
\begin{aligned}
((\grad \wedge A) \wedge \gamma_\mu) \cdot (\grad \wedge A) \cdot \gamma^\nu
&=
 F^{\beta \alpha} F_{\mu \alpha} {\delta_{\beta}}^\nu
+ \partial_{\alpha} A_{\beta} F^{\beta \alpha} {\delta_{\mu}}^\nu
\\
%&= F^{\nu \alpha} F_{\mu \alpha} + \inv{2} F_{\alpha \beta} F^{\beta \alpha} {\delta_{\mu}}^\nu \\
\end{aligned}
\end{equation}

which is
\begin{equation}\label{eqn:energy_momentum_tensor:secondPart}
\begin{aligned}
((\grad \wedge A) \wedge \gamma_\mu) \cdot (\grad \wedge A) \cdot \gamma^\nu
&= F^{\nu \alpha} F_{\mu \alpha} + \inv{2} F_{\alpha \beta} F^{\beta \alpha} {\delta_{\mu}}^\nu
%F^{\nu \beta} F_{\mu \beta} +\inv{2} F_{\alpha \beta} F^{\beta \alpha} {\delta_{\mu}}^\nu
\end{aligned}
\end{equation}

The final combination of results \eqnref{eqn:energy_momentum_tensor:firstPartDone}, and
\eqnref{eqn:energy_momentum_tensor:secondPart} gives

\begin{equation}\label{eqn:energyMomentumTensor:280}
\begin{aligned}
(F \gamma_\mu F ) \cdot \gamma^\nu
&=
2 F^{\alpha \nu} F_{\alpha \mu}
+\inv{2} F_{\alpha \beta} F^{\beta \alpha} {\delta_{\mu}}^\nu
\end{aligned}
\end{equation}

Yielding the tensor

\begin{equation}\label{eqn:energy_momentum_tensor:messyTensor}
\begin{aligned}
{T_\mu}^{\nu}
&=
\epsilon_0 \left(
\inv{4} F_{\alpha \beta} F^{\alpha \beta} {\delta_{\mu}}^\nu
-
F_{\alpha \mu}
F^{\alpha \nu}
\right)
\end{aligned}
\end{equation}

\section{Validate against previously calculated Poynting result}

In \chapcite{PJpoynting}, the electrodynamic energy density \(U\) and momentum flux density vectors were related as follows

\begin{equation}\label{eqn:energy_momentum_tensor:fromPoyntingNotes}
\begin{aligned}
U &= \frac{\epsilon_0}{2}\left( \BE^2 + c^2 \BB^2 \right) \\
\BP &= \epsilon_0 c^2 \BE \cross \BB = \epsilon_0 c (i c \BB) \cdot \BE \\
0 &= \PD{t}{}\frac{\epsilon_0}{2} \left(\BE^2 + c^2 \BB^2\right) + c^2 \epsilon_0 \spacegrad \cdot (\BE \cross \BB) + \BE \cdot \Bj
\end{aligned}
\end{equation}

Additionally the energy and momentum flux densities are components of this stress tensor four vector

\begin{equation}\label{eqn:energyMomentumTensor:300}
\begin{aligned}
T(\gamma_0) &= U \gamma_0 + \inv{c} \BP \gamma_0 \\
\end{aligned}
\end{equation}

From this we can read the first row of the tensor elements

\begin{equation}\label{eqn:energyMomentumTensor:320}
\begin{aligned}
{T_0}^0 &= U
= \frac{\epsilon_0}{2}\left( \BE^2 + c^2 \BB^2 \right) \\
{T_0}^k &= \inv{c} (\BP \gamma_0) \cdot \gamma^k = \epsilon_0 c E^a B^b \epsilon_{k a b}
\end{aligned}
\end{equation}

Let us compare these to \eqnref{eqn:energy_momentum_tensor:messyTensor}, which gives
\begin{equation}\label{eqn:energyMomentumTensor:340}
\begin{aligned}
{T_0}^{0}
&= \epsilon_0 \left( \inv{4} F_{\alpha \beta} F^{\alpha \beta} - F_{\alpha 0} F^{\alpha 0} \right) \\
&= \frac{\epsilon_0}{4} \left( F_{\alpha j} F^{\alpha j} - {3} F_{j 0} F^{j 0} \right) \\
&= \frac{\epsilon_0}{4} \left( F_{m j} F^{m j} +F_{0 j} F^{0 j} - {3} F_{j 0} F^{j 0} \right) \\
&= \frac{\epsilon_0}{4} \left( F_{m j} F^{m j} - {2} F_{j 0} F^{j 0} \right) \\
{T_0}^{k}
&= -\epsilon_0 F_{\alpha 0} F^{\alpha k} \\
&= -\epsilon_0 F_{j 0} F^{j k} \\
\end{aligned}
\end{equation}

Now, our field in terms of electric and magnetic coordinates is

\begin{equation}\label{eqn:energyMomentumTensor:360}
\begin{aligned}
F &= \BE + i c \BB \\
  &= E^k \gamma_k \gamma_0 + i c B^k \gamma_k \gamma_0 \\
  &= E^k \gamma_k \gamma_0 - c \epsilon_{a b k} B^k \gamma_a \gamma_b
\end{aligned}
\end{equation}

so the electric field tensor components are

\begin{equation}\label{eqn:energyMomentumTensor:380}
\begin{aligned}
F^{j 0}
&= (F \cdot \gamma^0) \cdot \gamma^j \\
&= E^k {\delta_k}^j \\
&= E^j
\end{aligned}
\end{equation}

and
\begin{equation}\label{eqn:energyMomentumTensor:400}
\begin{aligned}
F_{j 0} &= (\gamma_j)^2 (\gamma_0)^2 F^{j 0} \\
&= -E^j
\end{aligned}
\end{equation}

and the magnetic tensor components are

\begin{equation}\label{eqn:energyMomentumTensor:420}
\begin{aligned}
F^{m j} &= F_{m j} \\
&= - c \epsilon_{a b k} B^k ((\gamma_a \gamma_b) \cdot \gamma_{j}) \cdot \gamma_m \\
&= - c \epsilon_{m j k} B^k
\end{aligned}
\end{equation}

This gives
\begin{equation}\label{eqn:energyMomentumTensor:440}
\begin{aligned}
{T_0}^{0}
&= \frac{\epsilon_0}{4} \left( 2 c^2 B^k B^k + {2} E^j E^{j} \right) \\
&= \frac{\epsilon_0}{2} \left( c^2 \BB^2 + \BE^2 \right) \\
{T_0}^{k}
&= \epsilon_0 E^{j} F^{j k} \\
&= \epsilon_0 c \epsilon_{k e f} E^e B^f \\
&= \epsilon_0 (c \BE \cross \BB)_k \\
&= \inv{c} (\BP \cdot \sigma_k)
\end{aligned}
\end{equation}

Okay, good.  This checks 4 of the elements of \eqnref{eqn:energy_momentum_tensor:messyTensor} against the explicit \(\BE\) and \(\BB\) based representation of \(T(\gamma_0)\) in \eqnref{eqn:energy_momentum_tensor:fromPoyntingNotes}, leaving only 6 unique elements in the remaining parts of the (symmetric) tensor to verify.

\section{Four vector form of energy momentum conservation relationship}

One can observe that there is a spacetime divergence hiding there directly
in the energy conservation equation of
\eqnref{eqn:energy_momentum_tensor:fromPoyntingNotes}.  In particular, writing the last of those as

\begin{equation}\label{eqn:energyMomentumTensor:460}
\begin{aligned}
0 &= \partial_{0}{}\frac{\epsilon_0}{2} \left(\BE^2 + c^2 \BB^2\right) + \spacegrad \cdot \BP/c + \BE \cdot \Bj/c
\end{aligned}
\end{equation}

We can then write the energy-momentum parts as a four vector divergence
\begin{equation}\label{eqn:energyMomentumTensor:480}
\begin{aligned}
\grad \cdot \left(
\frac{\epsilon_0 \gamma_0}{2} \left(\BE^2 + c^2 \BB^2\right)
+ \inv{c} P^k \gamma_k
\right) &= - \BE \cdot \Bj/c
\end{aligned}
\end{equation}

Since we have a divergence relationship, it should also be possible to convert a spacetime hypervolume
integration of this quantity into a time-surface integral or a pure volume integral.  Pursing this
will probably clarify how the tensor is related to the
hypersurface flux as mentioned in the text here, but
making this concrete will take a bit more thought.

Having seen that we have a divergence relationship for the energy momentum tensor in the rest frame, it is clear that the
Poynting energy momentum flux relationship should follow much more directly if we play it backwards in a relativistic setting.

This is a very sneaky way to do it since we have to have seen the answer to get there, but it should avoid the complexity of
trying to factor out the spacial gradients and recover the divergence relationship that provides the Poynting vector.  Our sneaky
starting point is to compute

\begin{equation}\label{eqn:energyMomentumTensor:500}
\begin{aligned}
\grad \cdot ( F \gamma_0 \tilde{F} )
&= \gpgradezero{ \grad (F \gamma_0 \tilde{F}) } \\
&= \gpgradezero{
(\grad F) \gamma_0 \tilde{F}
+ \dot{\grad} F \gamma_0 \dot{\tilde{F}}
} \\
&= \gpgradezero{
(\grad F) \gamma_0 \tilde{F}
+ \dot{\tilde{F}} \dot{\grad} F \gamma_0
} \\
\end{aligned}
\end{equation}

Since this is a scalar quantity, it is equal to its own reverse and we can reverse all factors in this second term to convert the left acting
gradient to a more regular right acting form.  This is

\begin{equation}\label{eqn:energyMomentumTensor:520}
\begin{aligned}
\grad \cdot ( F \gamma_0 \tilde{F} )
&= \gpgradezero{
(\grad F) \gamma_0 \tilde{F}
+ \gamma_0 \tilde{F} (\grad F)
} \\
\end{aligned}
\end{equation}

Now using Maxwell's equation \(\grad F = J/\epsilon_0 c\), we have

\begin{equation}\label{eqn:energyMomentumTensor:540}
\begin{aligned}
\grad \cdot ( F \gamma_0 \tilde{F} )
&=
\inv{\epsilon_0 c}
\gpgradezero{
J \gamma_0 \tilde{F}
+ \gamma_0 \tilde{F} J
} \\
&=
\frac{2}{\epsilon_0 c}
\gpgradezero{
J \gamma_0 \tilde{F}
} \\
&= \frac{2}{\epsilon_0 c} (J \wedge \gamma_0) \cdot \tilde{F} \\
\end{aligned}
\end{equation}

Now, \(J = \gamma_0 c \rho + \gamma_k J^k\), so \(J \wedge \gamma_0 = J^k \gamma_k \gamma_0 = J^k \sigma_k = \Bj\), and dotting this with \(\tilde{F} = -\BE - i c \BB\) will pick up only the (negated) electric field components, so we have

\begin{equation}\label{eqn:energyMomentumTensor:560}
\begin{aligned}
(J \wedge \gamma_0) \cdot \tilde{F} &= \Bj \cdot (-\BE)
\end{aligned}
\end{equation}

Although done in \chapcite{PJpoynting}, for
completeness let us re-expand \(F \gamma_0 \tilde{F}\) in terms of the electric and magnetic field vectors.

\begin{equation}\label{eqn:energyMomentumTensor:580}
\begin{aligned}
F \gamma_0 \tilde{F}
&= -(\BE + i c \BB) \gamma_0 (\BE + i c \BB) \\
%&= - \gamma_0 (-\BE + i c \BB) (\BE + i c \BB) \\
&= \gamma_0 (\BE - i c \BB) (\BE + i c \BB) \\
&= \gamma_0 (\BE^2 + c^2 \BB^2 + i c (\BE \BB -\BB \BE) ) \\
&= \gamma_0 (\BE^2 + c^2 \BB^2 + 2 i c (\BE \wedge \BB) ) \\
&= \gamma_0 (\BE^2 + c^2 \BB^2 - 2 c (\BE \cross \BB) ) \\
\end{aligned}
\end{equation}

Next, we want an explicit spacetime split of the gradient

\begin{equation}\label{eqn:energyMomentumTensor:600}
\begin{aligned}
\grad \gamma_0
&= (\gamma^0 \partial_0 + \gamma^k \partial_k) \gamma_0 \\
&= \partial_0 - \gamma_k \gamma_0 \partial_k \\
&= \partial_0 - \sigma_k \partial_k \\
&= \partial_0 - \spacegrad \\
\end{aligned}
\end{equation}

We are now in shape to assemble all the intermediate results for the left hand side

\begin{equation}\label{eqn:energyMomentumTensor:620}
\begin{aligned}
\grad \cdot (F \gamma_0 \tilde{F})
&= \gpgradezero{ \grad (F \gamma_0 \tilde{F}) } \\
&= \gpgradezero{ (\partial_0 - \spacegrad) (\BE^2 + c^2 \BB^2 - 2 c (\BE \cross \BB) ) } \\
&= \partial_0 (\BE^2 + c^2 \BB^2) + 2 c \spacegrad \cdot (\BE \cross \BB)
\end{aligned}
\end{equation}

With a final reassembly of the left and right hand sides of
\(\grad \cdot T(\gamma_0)\),
the spacetime divergence of the rest frame stress vector we have
\begin{equation}\label{eqn:energyMomentumTensor:640}
\begin{aligned}
\inv{c} \partial_t (\BE^2 + c^2 \BB^2) + 2 c \spacegrad \cdot (\BE \cross \BB) &= -\frac{2}{c \epsilon_0}\Bj \cdot \BE
\end{aligned}
\end{equation}

Multiplying through by \(\epsilon_0 c/2\) we have the classical Poynting vector energy conservation relationship.

\begin{equation}\label{eqn:energy_momentum_tensor:conservation}
\begin{aligned}
\PD{t}{} \frac{\epsilon_0}{2}(\BE^2 + c^2 \BB^2) + \spacegrad \cdot \inv{\mu_0}(\BE \cross \BB) &= -\Bj \cdot \BE
\end{aligned}
\end{equation}

Observe that the momentum flux density, the Poynting vector \(\BP = (\BE \cross \BB)/\mu_0\),
is zero in the rest frame, which makes sense since there is no magnetic field
for a static charge distribution.  So with no currents and therefore no magnetic fields the field energy is a constant.

\subsection{Transformation properties}

\Eqnref{eqn:energy_momentum_tensor:conservation} is the explicit spacetime expansion of the equivalent relativistic equation

\begin{equation}\label{eqn:energyMomentumTensor:660}
\begin{aligned}
\grad \cdot \left( c T(\gamma_0) \right) &=
\grad \cdot \left(\frac{c \epsilon_0}{2} F \gamma_0 \tilde{F}\right) = \gpgradezero{ J \gamma_0 \tilde{F} }
\end{aligned}
\end{equation}

This has all the same content, but in relativistic form seems almost trivial.  While the stress vector \(T(\gamma_0)\) is not itself
a relativistic invariant, this divergence equation is.

Suppose we form a Lorentz transformation \(\LL(x) = R x \tilde{R}\), applied to this equation we have

\begin{equation}\label{eqn:energyMomentumTensor:680}
\begin{aligned}
F'
&= (R\grad \tilde{R}) \wedge (R A \tilde{R}) \\
&= \gpgradetwo{ R \grad \tilde{R} R A \tilde{R} } \\
&= \gpgradetwo{ R \grad A \tilde{R} } \\
&= R (\grad \wedge A) \tilde{R} \\
&= R F \tilde{R} \\
\end{aligned}
\end{equation}

Transforming all the objects in the equation we have
\begin{equation}\label{eqn:energyMomentumTensor:700}
\begin{aligned}
\grad' \cdot \left(\frac{c \epsilon_0}{2} F' \gamma_0' \tilde{F'} \right) &= \gpgradezero{ J' \gamma_0' \tilde{F'} } \\
(R \grad \tilde{R}) \cdot \left(\frac{c \epsilon_0}{2} R F \tilde{R} R \gamma_0 R \tilde{R} (R F\tilde{R})^{\tilde{}} \right)
&= \gpgradezero{ R J \tilde{R} R \gamma_0 \tilde{R} (R F \tilde{R})^{\tilde{}} } \\
\end{aligned}
\end{equation}

This is nothing more than the original untransformed quantity

\begin{equation}\label{eqn:energyMomentumTensor:720}
\begin{aligned}
\grad \cdot \left(\frac{c \epsilon_0}{2} F \gamma_0 \tilde{F} \right) &= \gpgradezero{ J \gamma_0 \tilde{F} } \\
\end{aligned}
\end{equation}

\section{Validate with relativistic transformation}

As a relativistic quantity we should be able to verify the messy tensor relationship
by Lorentz transforming the energy density from a rest frame to a
moving frame.

Now let us try the Lorentz transformation of the energy density.

FIXME: TODO.

%
% Copyright � 2012 Peeter Joot.  All Rights Reserved.
% Licenced as described in the file LICENSE under the root directory of this GIT repository.
%

% 
% 
\chapter{Lorentz force relation to the energy momentum tensor}\label{chap:PJstressEnergyLorentz}
\index{Lorentz force!energy momentum tensor}
%\date{Feb 13, 2009.  stressEnergyLorentz.tex}

\section{Motivation}

Have now made a few excursions related to the concepts of electrodynamic
field energy and momentum.

In \chapcite{PJpoynting} the energy density rate and Poynting divergence 
relationship was demonstrated using Maxwell's equation.  That was:

\begin{equation}\label{eqn:stressEnergyLorentz:20}
\begin{aligned}
\PD{t}{}\frac{\epsilon_0}{2} \left(\BE^2 + c^2 \BB^2\right) + \spacegrad \cdot \inv{\mu_0}(\BE \cross \BB) &= -\BE \cdot \Bj 
\end{aligned}
\end{equation}

In terms of the field energy density \(U\), and Poynting vector \(\BP\), this is
\begin{equation}\label{eqn:seLorentz:energyDensityPoyntingDefined}
\begin{aligned}
U &= \frac{\epsilon_0}{2} \left(\BE^2 + c^2 \BB^2\right) \\
\BP &= \inv{\mu_0}(\BE \cross \BB) \\
\PD{t}{U} + \spacegrad \cdot \BP &= -\BE \cdot \Bj 
\end{aligned}
\end{equation}

In \chapcite{PJemstresstensor} this was related to the 
energy momentum four vectors

\begin{equation}\label{eqn:seLorentz:lorentzForceT}
\begin{aligned}
T(a) &= \frac{\epsilon_0}{2} F a \tilde{F}
\end{aligned}
\end{equation}

as defined
in \citep{doran2003gap}, but the big picture view 
of things was missing.

Later in \chapcite{PJpoyntingRate} the rate of change of Poynting vector
was calculated, with an additional attempt to relate this to \(T(\gamma_\mu)\).

These relationships, and the operations required to factoring out the divergence were considerably messier.

Finally, in \chapcite{PJelectricFieldEnergy} the four vector \(T(\gamma_\mu)\)
was related to the Lorentz force and the work done moving a charge against
a field.  This provides the natural context for the energy momentum tensor, 
since it appears that the spacetime divergence of each of the
\(T(\gamma_\mu)\) four vectors appears to be a component of the
four vector Lorentz force (density).  

In these notes the divergences will be calculated to confirm the
connection between the Lorentz force and energy momentum tensor directly.
This is actually expected to be simpler than the previous calculations.

It is also 
potentially of interest, as shown in \chapcite{PJFourierVacuum}, and
\chapcite{PJplaneWave}
that the energy density and Poynting vectors, and energy momentum four vector,
were seen to be naturally expressible as Hermitian conjugate operations

\begin{equation}\label{eqn:stressEnergyLorentz:40}
\begin{aligned}
F^\dagger &= \gamma_0 \tilde{F} \gamma_0
\end{aligned}
\end{equation}
\begin{equation}\label{eqn:stressEnergyLorentz:60}
\begin{aligned}
T(\gamma_0) &= \frac{\epsilon_0}{2} F F^\dagger \gamma_0
\end{aligned}
\end{equation}
\begin{equation}\label{eqn:stressEnergyLorentz:80}
\begin{aligned}
U &= T(\gamma_0) \cdot \gamma_0 = \frac{\epsilon_0}{4} \left(F F^\dagger + F^\dagger F \right) \\
\BP/c &= T(\gamma_0) \wedge \gamma_0 = \frac{\epsilon_0}{4} \left(F F^\dagger - F^\dagger F \right)
\end{aligned}
\end{equation}

It is conceivable that a generalization of Hermitian conjugation, where the spatial basis vectors are used instead of \(\gamma_0\), will 
provide a mapping and driving structure from the Four vector quantities and the somewhat scrambled seeming set
of relationships observed in the split spatial and time domain.  That will also be explored here.

\section{Spacetime divergence of the energy momentum four vectors}

The spacetime divergence of the field energy momentum four vector \(T(\gamma_0)\) has been calculated previously.  Let us redo this 
calculation for the other components.

\begin{equation}\label{eqn:stressEnergyLorentz:100}
\begin{aligned}
\grad \cdot T(\gamma_\mu) 
&= \frac{\epsilon_0}{2} \gpgradezero{ \grad (F \gamma_\mu \tilde{F}) } \\
&= \frac{\epsilon_0}{2} \gpgradezero{ (\grad F) \gamma_\mu \tilde{F} + (\tilde{F} \grad) F \gamma_\mu } \\
&= \frac{\epsilon_0}{2} \gpgradezero{ (\grad F) \gamma_\mu \tilde{F} + \gamma_\mu \tilde{F} (\grad F) } \\
&= {\epsilon_0} \gpgradezero{ (\grad F) \gamma_\mu \tilde{F} } \\
&= \inv{c} \gpgradezero{ J \gamma_\mu \tilde{F} } \\
\end{aligned}
\end{equation}

The ability to perform cyclic reordering of terms in a scalar product has been used above.  Application of one more
reverse operation (which does not change a scalar), gives us

\begin{equation}\label{eqn:seLorentz:lorentzForceTdivergence}
\begin{aligned}
\grad \cdot T(\gamma_\mu) &= \inv{c} \gpgradezero{ F \gamma_\mu J } 
\end{aligned}
\end{equation}

Let us expand the right hand size first.

\begin{equation}\label{eqn:stressEnergyLorentz:120}
\begin{aligned}
\inv{c} \gpgradezero{ F \gamma_\mu J } &= \inv{c} \gpgradezero{ (\BE + i c \BB) \gamma_\mu (c \rho \gamma_0 + \Bj \gamma_0) } 
\end{aligned}
\end{equation}

The \(\mu = 0\) term looks the easiest, and for that one we have

\begin{equation}\label{eqn:stressEnergyLorentz:140}
\begin{aligned}
\inv{c} \gpgradezero{ (\BE + i c \BB) (c \rho - \Bj) }  = -\Bj \cdot \BE
\end{aligned}
\end{equation}

Now, for the other terms, say \(\mu = k\), we have

\begin{equation}\label{eqn:stressEnergyLorentz:160}
\begin{aligned}
\inv{c} \gpgradezero{ (\BE + i c \BB) (c \rho \sigma_k - \sigma_k \Bj ) } 
&= E^k \rho - \gpgradezero{ i \BB \sigma_k \Bj }  \\
&= E^k \rho - J^a B^b \gpgradezero{ \sigma_1 \sigma_2 \sigma_3 \sigma_b \sigma_k \sigma_a }  \\
&= E^k \rho - J^a B^b \epsilon_{a k b} \\
&= E^k \rho + J^a B^b \epsilon_{k a b} \\
&= (\rho \BE + \Bj \cross \BB) \cdot \sigma_k
\end{aligned}
\end{equation}

Summarizing the two results we have

\begin{equation}\label{eqn:seLorentz:lorentzForcePair}
\begin{aligned}
\inv{c} \gpgradezero{ F \gamma_0 J } &= -\Bj \cdot \BE \\
\inv{c} \gpgradezero{ F \gamma_k J } &= (\rho \BE + \Bj \cross \BB) \cdot \sigma_k
\end{aligned}
\end{equation}

The second of these is easily recognizable as components of the Lorentz force for an element of charge (density).  The first
of these is actually the energy component of the four vector Lorentz force, so expanding that in terms of spacetime quantities
is the next order of business.

\section{Four vector Lorentz Force}

The Lorentz force in covariant form is

%\begin{align}
%m \ddot{x}_\mu &= q F_{} \cdot \frac{dx}{d\tau}
%\end{align}
%
%Or in vector/bivector form

\begin{equation}\label{eqn:seLorentz:lorentzForceGA}
\begin{aligned}
m \ddot{x} &= q F \cdot \dot{x}/c
\end{aligned}
\end{equation}

Two verifications of this are in order.  One is that we get the traditional vector form of the Lorentz force equation from
this and the other is that we can get the traditional tensor form from this equation.

\subsection{Lorentz force in tensor form}
\index{Lorentz force!tensor}

Recovering the tensor form is probably the easier of the two operations.  We have

\begin{equation}\label{eqn:stressEnergyLorentz:180}
\begin{aligned}
m \ddot{x}_\mu \gamma^\mu
&= \frac{q}{2} F_{\alpha\beta} \dot{x}_\sigma (\gamma^{\alpha} \wedge \gamma^\beta) \cdot \gamma^\sigma \\
&= \frac{q}{2} F_{\alpha\beta} \dot{x}^\sigma (\gamma^{\alpha} {\delta^\beta}_\sigma -\gamma^{\beta} {\delta^\alpha}_\sigma ) \\
&= \frac{q}{2} F_{\alpha\beta} \dot{x}^\beta \gamma^{\alpha} - \frac{q}{2} F_{\alpha\beta} \dot{x}^\alpha \gamma^{\beta} \\
\end{aligned}
\end{equation}

Dotting with \(\gamma_\mu\) the right hand side is

\begin{equation}\label{eqn:stressEnergyLorentz:200}
\begin{aligned}
\frac{q}{2} F_{\mu\beta} \dot{x}^\beta - \frac{q}{2} F_{\alpha\mu} \dot{x}^\alpha 
&= {q} F_{\mu\alpha} \dot{x}^\alpha 
\end{aligned}
\end{equation}

Which recovers the tensor form of the equation
\begin{equation}\label{eqn:seLorentz:lorentzForceTensor}
\begin{aligned}
m \ddot{x}_\mu &= {q} F_{\mu\alpha} \dot{x}^\alpha 
\end{aligned}
\end{equation}

\subsection{Lorentz force components in vector form}
\index{Lorentz force!vector form}

\begin{equation}\label{eqn:stressEnergyLorentz:220}
\begin{aligned}
m \gamma \frac{d}{dt} \gamma \left(c + \sigma_k \frac{dx^k}{dt}\right) \gamma_0
&= \frac{q}{2c}(F v - v F) \\
&=
\frac{q \gamma}{2c} 
(\BE + i c\BB) 
\left(c + \sigma_k \frac{dx^k}{dt}\right) \gamma_0
\\
&\quad -
\frac{q \gamma}{2c} 
\left(c + \sigma_k \frac{dx^k}{dt}\right) \gamma_0
(\BE + i c\BB) \\
\end{aligned}
\end{equation}

Right multiplication by \(\gamma_0/\gamma\) we have
\begin{equation}\label{eqn:stressEnergyLorentz:240}
\begin{aligned}
m \frac{d}{dt} \gamma \left(c + \Bv \right) 
&= \frac{q }{2c} \left( (\BE + i c\BB) \left(c + \Bv \right) -\left(c + \Bv \right) (-\BE + i c\BB) \right) \\
&= \frac{q }{2c} \left( 
%(\BE + i c\BB) (c + \Bv ) 
%-(c + \Bv ) (-\BE + i c\BB) 
+ 2 \BE c 
+ \BE \Bv  + \Bv \BE 
+ i c (\BB \Bv  - \Bv \BB)
\right) \\
\end{aligned}
\end{equation}

After a last bit of reduction this is
\begin{equation}\label{eqn:stressEnergyLorentz:260}
\begin{aligned}
m \frac{d}{dt} \gamma \left(c + \Bv \right) &= q (\BE + \Bv \cross \BB) + q \BE \cdot \Bv/c
\end{aligned}
\end{equation}

In terms of four vector momentum this is
\begin{equation}\label{eqn:seLorentz:lorentzForceVec}
\begin{aligned}
\dot{p} = q ( \BE \cdot \Bv/c + \BE + \Bv \cross \BB ) \gamma_0
\end{aligned}
\end{equation}

\subsection{Relation to the energy momentum tensor}
\index{Lorentz force!energy momentum tensor}

It appears that to relate the energy momentum tensor to the Lorentz force we have 
to work with the upper index quantities rather than the lower index stress tensor vectors.  Doing so
our four vector force per unit volume is
 
\begin{equation}\label{eqn:stressEnergyLorentz:280}
\begin{aligned}
\PD{V}{\dot{p}}
&= (\Bj \cdot \BE + \rho \BE + \Bj \cross \BB) \gamma_0 \\
%&= - \inv{c} \left(\gpgradezero{ F \gamma^0 J } \gamma_0 + \gpgradezero{ F \gamma^k J } \gamma_k \right) \\
&= - \inv{c} \gpgradezero{ F \gamma^\mu J } \gamma_\mu \\
&= - (\grad \cdot T(\gamma^\mu)) \gamma_\mu
\end{aligned}
\end{equation}

The term \(\gpgradezero{ F \gamma^\mu J } \gamma_\mu \) appears to be expressed simply has \(F \cdot J\) in
\citep{doran2003gap}.  Understanding that simple statement is now possible now that an exploration
of some of the underlying ideas has been made.  In retrospect having seen the bivector product form of the Lorentz force equation, it should have been 
clear, but some of the associated trickiness in their treatment obscured this
fact ( Although their treatment is only two pages, I still only 
understand half of what they are doing!)


\section{Expansion of the energy momentum tensor}

While all the components of the divergence of the energy momentum tensor have been expanded explicitly, this has not been
done here for the tensor itself.  A mechanical expansion of the tensor in terms of field tensor components \(F^{\mu\nu}\) has been 
done previously and is not particularly enlightening.  Let us work it out here in terms of electric and magnetic field components.  In particular for the \(T^{0\mu}\) and \(T^{\mu0}\) components of the tensor in terms of energy density and the Poynting vector.

\subsection{In terms of electric and magnetic field components}

Here we want to expand 

\begin{equation}\label{eqn:stressEnergyLorentz:300}
\begin{aligned}
T(\gamma^\mu) = \frac{-\epsilon_0}{2} (\BE + i c \BB) \gamma^\mu (\BE + i c \BB)
\end{aligned}
\end{equation}

It will be convenient here to temporarily work with \(\epsilon_0 = c = 1\), and put them back in afterward.
%T(\gamma^\mu) = \frac{-1}{2} (\BE + i \BB) \gamma^\mu (\BE + i \BB)

\subsubsection{First row}

First expanding \(T(\gamma^0)\) we have
\begin{equation}\label{eqn:stressEnergyLorentz:320}
\begin{aligned}
T(\gamma^0) 
&= \frac{1}{2} (\BE + i \BB) (\BE - i \BB) \gamma^0 \\
&= \frac{1}{2} (\BE^2 + \BB^2 + i (\BB \BE - \BE \BB)) \gamma^0 \\
&= \frac{1}{2} (\BE^2 + \BB^2) \gamma^0 + i ( \BB \wedge \BE ) \gamma^0 \\
\end{aligned}
\end{equation}

Using the wedge product dual \(\Ba \wedge \Bb = i (\Ba \cross \Bb)\), and putting back in the units, we have our
first stress energy four vector, 

\begin{equation}\label{eqn:stressEnergyLorentz:340}
\begin{aligned}
T(\gamma^0) &= \left( \frac{\epsilon_0}{2} (\BE^2 + c^2 \BB^2) + \inv{\mu_0 c} (\BE \cross \BB) \right) \gamma^0 
\end{aligned}
\end{equation}

In particular the energy density and the components of the Poynting vector can be picked off by dotting with each of the \(\gamma^\mu\) vectors.  That is

\begin{equation}\label{eqn:stressEnergyLorentz:360}
\begin{aligned}
U                    &= T(\gamma^0) \cdot \gamma^0 \\
\BP/c \cdot \sigma_k &= T(\gamma^0) \cdot \gamma^k
\end{aligned}
\end{equation}

\subsubsection{First column}

We have Poynting vector terms in the \(T^{0k}\) elements of the matrix.  Let us quickly verify that we have them
in the \(T^{k0}\) positions too.

To do so, again with \(c = \epsilon_0 = 1\) temporarily this is a computation of

\begin{equation}\label{eqn:stressEnergyLorentz:380}
\begin{aligned}
T(\gamma^k) \cdot \gamma^0 
&= \inv{2} (T(\gamma^k) \gamma^0 + \gamma^0 T(\gamma^k)) \\
&= \frac{-1}{4} (F \gamma^k F \gamma^0 + \gamma^0 F \gamma^k F) \\
&= \frac{1}{4} (F \sigma_k \gamma_0 F \gamma^0 - \gamma^0 F \gamma_0 \sigma_k F) \\
&= \frac{1}{4} (F \sigma_k (-\BE + i \BB) - (-\BE + i\BB) \sigma_k F) \\
&= \frac{1}{4} \gpgradezero{\sigma_k (-\BE + i \BB)(\BE + i \BB) - \sigma_k (\BE + i \BB)(-\BE + i\BB) } \\
&= \frac{1}{4} \gpgradezero{\sigma_k (-\BE^2 -\BB^2 +2 (\BE \cross \BB)) - \sigma_k (-\BE^2 -\BB^2 - 2(\BE \cross \BB)) } \\
\end{aligned}
\end{equation}

Adding back in the units we have

\begin{equation}\label{eqn:stressEnergyLorentz:400}
\begin{aligned}
T(\gamma^k) \cdot \gamma^0 &= \epsilon_0 c (\BE \cross \BB) \cdot \sigma_k = \inv{c}\BP \cdot \sigma_k
\end{aligned}
\end{equation}

As expected, these are the components of the Poynting vector (scaled by \(1/c\) for units of energy density).

\subsubsection{Diagonal and remaining terms}

\begin{equation}\label{eqn:stressEnergyLorentz:420}
\begin{aligned}
T(\gamma^a) \cdot \gamma^b 
&= \inv{2} (T(\gamma^a) \gamma^b + \gamma^b T(\gamma^a)) \\
&= \frac{-1}{4} (F \gamma^a F \gamma^b + \gamma^a F \gamma^b F) \\
&= \frac{1}{4} (F \sigma_a \gamma_0 F \gamma^b - \gamma^a F \gamma_0 \sigma_b F) \\
&= \frac{1}{4} (F \sigma_a (-\BE + i \BB) \sigma_b + \sigma_a (-\BE + i \BB) \sigma_b F) \\
&= \frac{1}{2} \gpgradezero{\sigma_a (-\BE + i \BB) \sigma_b (\BE + i\BB) } \\
\end{aligned}
\end{equation}

From this point is there any particularly good or clever way to do the remaining reduction?  Doing it with 
coordinates looks like it would be easy, but also messy.  A decomposition of \(\BE\) and \(\BB\) that are parallel
and perpendicular to the spatial basis vectors also looks feasible.

Let us try the dumb way first

\begin{equation}\label{eqn:stressEnergyLorentz:440}
\begin{aligned}
T(\gamma^a) \cdot \gamma^b 
&= \frac{1}{2} \gpgradezero{\sigma_a (-E^k \sigma_k + i B^k \sigma_k) \sigma_b (E^m \sigma_m + i B^m \sigma_m) } \\
&= 
\inv{2} (B^k E^m - E^k B^m) \gpgradezero{ i \sigma_a \sigma_k \sigma_b \sigma_m } 
- \inv{2} (E^k E^m + B^k B^m) \gpgradezero{ \sigma_a \sigma_k \sigma_b \sigma_m } \\
\end{aligned}
\end{equation}

Reducing the scalar operations is going to be much different for the \(a = b\), and \(a \ne b\) cases.  For the diagonal case
we have 

\begin{equation}\label{eqn:stressEnergyLorentz:460}
\begin{aligned}
T(\gamma^a) \cdot \gamma^a 
&= 
\inv{2} (B^k E^m - E^k B^m) \gpgradezero{ i \sigma_a \sigma_k \sigma_a \sigma_m } 
- \inv{2} (E^k E^m + B^k B^m) \gpgradezero{ \sigma_a \sigma_k \sigma_a \sigma_m } \\
&= 
- \inv{2} \sum_{m, k \ne a} \inv{2} (B^k E^m - E^k B^m) \gpgradezero{ i \sigma_k \sigma_m } 
+ \inv{2} \sum_{m, k \ne a} (E^k E^m + B^k B^m) \gpgradezero{ \sigma_k \sigma_m } \\
&+ \inv{2} \sum_{m} (B^a E^m - E^a B^m) \gpgradezero{ i \sigma_a \sigma_m } 
- \inv{2} \sum_m (E^a E^m + B^a B^m) \gpgradezero{ \sigma_a \sigma_m } \\
\end{aligned}
\end{equation}

Inserting the units again we have
\begin{equation}\label{eqn:stressEnergyLorentz:480}
\begin{aligned}
T(\gamma^a) \cdot \gamma^a 
&= 
\frac{\epsilon_0}{2} \left( \sum_{k \ne a} \left( (E^k)^2 + c^2 (B^k)^2 \right) - \left( (E^a)^2  + c^2 (B^a)^2  \right) \right)
\end{aligned}
\end{equation}

Or, adding and subtracting, we have the diagonal in terms of energy density (minus a fudge) 

\begin{equation}\label{eqn:stressEnergyLorentz:500}
\begin{aligned}
T(\gamma^a) \cdot \gamma^a &= U - \epsilon_0 \left( (E^a)^2  + c^2 (B^a)^2  \right)
\end{aligned}
\end{equation}

Now, for the off diagonal terms.  For \(a \ne b\) this is
\begin{equation}\label{eqn:stressEnergyLorentz:520}
\begin{aligned}
T(\gamma^a) \cdot \gamma^b 
&= 
\inv{2} \sum_m (B^a E^m - E^a B^m) \gpgradezero{ i \sigma_b \sigma_m } 
+\inv{2} \sum_{m}(B^b E^m - E^b B^m) \gpgradezero{ i \sigma_a \sigma_m } \\
&- \inv{2} \sum_m (E^a E^m + B^a B^m) \gpgradezero{ \sigma_b \sigma_m } 
- \inv{2} \sum_{m}(E^b E^m + B^b B^m) \gpgradezero{ \sigma_a \sigma_m } \\
&+\inv{2} \sum_{m, k \ne a,b}(B^k E^m - E^k B^m) \gpgradezero{ i \sigma_a \sigma_k \sigma_b \sigma_m } 
- \inv{2} \sum_{m, k \ne a,b}(E^k E^m + B^k B^m) \gpgradezero{ \sigma_a \sigma_k \sigma_b \sigma_m } \\
\end{aligned}
\end{equation}

The first two scalar filters that include \(i\) will be zero, and we have deltas
\(\gpgradezero{ \sigma_b \sigma_m } = \delta_{bm}\) in the next two.
The remaining two terms have only vector and bivector terms, so we have zero scalar parts.
That leaves (restoring units)

\begin{equation}\label{eqn:stressEnergyLorentz:540}
\begin{aligned}
T(\gamma^a) \cdot \gamma^b 
&= - \frac{\epsilon_0}{2} \left( E^a E^b + E^b E^a + c^2 (B^a B^b + B^b B^a) \right)
\end{aligned}
\end{equation}

\subsection{Summarizing}

Collecting all the results, with \(T^{\mu\nu} = T(\gamma^\mu) \cdot \gamma^\nu\), we have

\begin{equation}\label{eqn:stressEnergyLorentz:560}
\begin{aligned}
T^{00} &= \frac{\epsilon_0}{2} \left(\BE^2 + c^2 \BB^2\right) \\
T^{aa} &= \frac{\epsilon_0}{2} \left(\BE^2 + c^2 \BB^2\right) - \epsilon_0 \left( (E^a)^2  + c^2 (B^a)^2  \right) \\
T^{k0} = T^{0k} &= \inv{c} \left( \inv{\mu_0}(\BE \cross \BB) \right) \cdot \sigma_k \\
T^{ab} = T^{ba} &= - \frac{\epsilon_0}{2} \left( E^a E^b + E^b E^a + c^2 (B^a B^b + B^b B^a) \right)
\end{aligned}
\end{equation}

\subsection{Assembling a four vector}

Let us see what one of the \(T^{a\mu} \gamma_\mu\) rows of the tensor looks like in four vector form.  Let \(f \ne g \ne h\) 
represent an even permutation of the integers \(1,2,3\).  Then we have

\begin{equation}\label{eqn:stressEnergyLorentz:580}
\begin{aligned}
T^f 
&= T^{f\mu} \gamma_\mu \\
&= 
\frac{\epsilon_0}{2} c (E^g B^h - E^h B^g) \gamma_0 \\
&+\frac{\epsilon_0}{2} \left( -(E^f)^2 +(E^g)^2 +(E^h)^2 + c^2 ( -(B^f)^2 +(B^g)^2 +(B^h)^2 ) \right) \gamma_f \\
&-\frac{\epsilon_0}{2} \left( E^f E^g + E^g E^f + c^2 (B^f B^g + B^g B^f) \right) \gamma_g \\
&-\frac{\epsilon_0}{2} \left( E^f E^h + E^h E^f + c^2 (B^f B^h + B^h B^f) \right) \gamma_h \\
\end{aligned}
\end{equation}

It is pretty amazing that the divergence of this produces the \(f\) component of the Lorentz force (density)

\begin{equation}\label{eqn:stressEnergyLorentz:600}
\begin{aligned}
\partial_\mu T^{f\mu} = (\rho \BE + \Bj \cross \BB) \cdot \sigma_f
\end{aligned}
\end{equation}

Demonstrating this directly without having STA as an available tool would be quite tedious, and looking at this
expression inspires no particular attempt to try!

%where \(U\), and \(\BP\) are as defined in \eqnref{eqn:seLorentz:energyDensityPoyntingDefined}.

% other way:
%
%To reduce this, let us write the fields in terms of projections and rejections onto the \(\sigma_k\) direction, as in \(\BE_\parallel = (\BE \cdot \sigma_k) \sigma_k\), and \(\BE_\perp = (\BE \wedge \sigma_k) \sigma_k\).  Then we have
%

\section{Conjugation?}
% from planewave.ltx
\subsection{Followup: energy momentum tensor}

This also suggests a relativistic generalization of conjugation, since the time basis vector should perhaps not have
a distinguishing role.  Something like this:

\begin{equation}\label{eqn:stressEnergyLorentz:620}
\begin{aligned}
F^{\dagger_\mu} &= \gamma_\mu \tilde{F} \gamma_\mu
\end{aligned}
\end{equation}

Or perhaps:
\begin{equation}\label{eqn:stressEnergyLorentz:640}
\begin{aligned}
F^{\dagger_\mu} &= \gamma_\mu \tilde{F} \gamma^\mu
\end{aligned}
\end{equation}

may make sense for consideration of the other components of the general energy momentum tensor, which had roughly the form:

\begin{equation}\label{eqn:stressEnergyLorentz:660}
\begin{aligned}
T^{\mu\nu} \propto T(\gamma_\mu) \cdot \gamma^\nu
\end{aligned}
\end{equation}

(with some probable adjustments to index positions).  Think this through later.

%
% Copyright � 2012 Peeter Joot.  All Rights Reserved.
% Licenced as described in the file LICENSE under the root directory of this GIT repository.
%

%
%
\chapter{Energy momentum tensor relation to Lorentz force}\label{chap:PJenMtensor}
%\date{Feb 17, 2009.  enMTensor.tex}

\section{Motivation}

In \chapcite{PJstressEnergyLorentz} the energy momentum tensor was related
to the Lorentz force in STA form.  Work the same calculation strictly in
tensor form, to
%compare the difficulty of the algebra and
develop more comfort with tensor manipulation.  This should also serve
as a translation aid to compare signs due to metric tensor differences
in other reading.

\subsection{Definitions}

The energy momentum ``tensor'', really a four vector, is defined
in \citep{doran2003gap}
as

\begin{equation}\label{eqn:enMTensor:20}
\begin{aligned}
T(a) &=
\frac{\epsilon_0}{2} F a \tilde{F} = -\frac{\epsilon_0}{2} F a {F}
\end{aligned}
\end{equation}

We have seen that the divergence of the \(T(\gamma^\mu)\) vectors generate
the Lorentz force relations.

Let us expand this with respect to index lower basis vectors for use in the
divergence calculation.

\begin{equation}\label{eqn:enMTensor:40}
\begin{aligned}
T(\gamma^\mu) &=
\lr{ T(\gamma^\mu) \cdot \gamma^\nu } \gamma_\nu
\end{aligned}
\end{equation}

So we define
\begin{equation}\label{eqn:enMTensor:60}
\begin{aligned}
T^{\mu \nu}
&= T(\gamma^\mu) \cdot \gamma^\nu
\end{aligned}
\end{equation}

and can write these four vectors in tensor form as
\begin{equation}\label{eqn:enMTensor:80}
\begin{aligned}
T(\gamma^\mu) &= T^{\mu \nu} \gamma_\nu
\end{aligned}
\end{equation}

\subsection{Expanding out the tensor}

An expansion of \(T^{\mu\nu}\) was done in \chapcite{PJemstresstensor}, but looking
back that seems a peculiar way, using the four vector potential.

Let us try again in terms of \(F^{\mu\nu}\) instead.  Our field is

\begin{equation}\label{eqn:enMTensor:100}
\begin{aligned}
F
&= \inv{2} F^{\mu\nu} \gamma_{\mu} \wedge \gamma_\nu
%\\
%&= \inv{2} F_{\mu\nu} \gamma^{\mu} \wedge \gamma^\nu
\end{aligned}
\end{equation}

So our tensor components are
\begin{equation}\label{eqn:enMTensor:120}
\begin{aligned}
T^{\mu\nu}
&= T(\gamma^\mu) \cdot \gamma^\nu \\
&= -\frac{\epsilon_0}{8} F^{\lambda\sigma} F^{\alpha\beta}
\gpgradezero{ (\gamma_\lambda \wedge \gamma_\sigma) \gamma^\mu (\gamma_\alpha \wedge \gamma_\beta) \gamma^\nu } \\
\end{aligned}
\end{equation}

Or
\begin{equation}\label{eqn:enMTensor:140}
\begin{aligned}
-8{\inv{\epsilon_0}} T^{\mu\nu}
&=
F^{\lambda\sigma} F^{\alpha\beta}
\gpgradezero{
(\gamma_\lambda {\delta_\sigma}^\mu
-\gamma_\sigma {\delta_\lambda}^\mu)
(\gamma_\alpha {\delta_\beta}^\nu
-\gamma_\beta {\delta_\alpha}^\nu)
} \\
&+
F^{\lambda\sigma} F^{\alpha\beta}
\gpgradezero{ (\gamma_\lambda \wedge \gamma_\sigma \wedge \gamma^\mu) (\gamma_\alpha \wedge \gamma_\beta \wedge \gamma^\nu) } \\
\end{aligned}
\end{equation}

Expanding only the first term to start with
\begin{equation}\label{eqn:enMTensor:160}
\begin{aligned}
&
F^{\lambda\sigma} F^{\alpha\beta} (\gamma_\lambda {\delta_\sigma}^\mu) \cdot  (\gamma_\alpha {\delta_\beta}^\nu)
+F^{\lambda\sigma} F^{\alpha\beta} (\gamma_\sigma {\delta_\lambda}^\mu) \cdot (\gamma_\beta {\delta_\alpha}^\nu)  \\
&-F^{\lambda\sigma} F^{\alpha\beta} (\gamma_\lambda {\delta_\sigma}^\mu) \cdot (\gamma_\beta {\delta_\alpha}^\nu)
-F^{\lambda\sigma} F^{\alpha\beta} (\gamma_\sigma {\delta_\lambda}^\mu) \cdot (\gamma_\alpha {\delta_\beta}^\nu)  \\
&=
F^{\lambda\mu} F^{\alpha\nu} \gamma_\lambda \cdot \gamma_\alpha
+F^{\mu\sigma} F^{\nu\beta} \gamma_\sigma \cdot \gamma_\beta
-F^{\lambda\mu} F^{\nu\beta} \gamma_\lambda \cdot \gamma_\beta
-F^{\mu\sigma} F^{\alpha\nu} \gamma_\sigma \cdot \gamma_\alpha \\
&=
\eta_{\alpha\beta}
(
F^{\lambda\mu} F^{\alpha\nu} \gamma_\lambda \cdot \gamma^\beta
+
%\eta_{\alpha\beta}
F^{\mu\sigma} F^{\nu\alpha} \gamma_\sigma \cdot \gamma^\beta
-
%\eta_{\alpha\beta}
F^{\lambda\mu} F^{\nu\alpha} \gamma_\lambda \cdot \gamma^\beta
-
%\eta_{\alpha\beta}
F^{\mu\sigma} F^{\alpha\nu} \gamma_\sigma \cdot \gamma^\beta )
\\
&=
\eta_{\alpha\lambda} F^{\lambda\mu} F^{\alpha\nu} + \eta_{\alpha\sigma} F^{\mu\sigma} F^{\nu\alpha}
- \eta_{\alpha\lambda} F^{\lambda\mu} F^{\nu\alpha} - \eta_{\alpha\sigma} F^{\mu\sigma} F^{\alpha\nu}  \\
&=
2( \eta_{\alpha\lambda} F^{\lambda\mu} F^{\alpha\nu} + \eta_{\alpha\sigma} F^{\mu\sigma} F^{\nu\alpha} ) \\
&=
2( \eta_{\alpha\beta} F^{\beta\mu} F^{\alpha\nu} +
\eta_{\alpha\beta} F^{\mu\beta} F^{\nu\alpha} ) \\
&=
4 \eta_{\alpha\beta} F^{\beta\mu} F^{\alpha\nu}  \\
&= 4 F^{\beta\mu} {F_{\beta}}^{\nu} \\
&= 4 F^{\alpha\mu} {F_{\alpha}}^{\nu} \\
\end{aligned}
\end{equation}

For the second term after a shuffle of indices we have
\begin{equation}\label{eqn:enMTensor:180}
\begin{aligned}
F^{\lambda\sigma} F_{\alpha\beta}
\eta^{\mu\mu'} \gpgradezero{ (\gamma_\lambda \wedge \gamma_\sigma \wedge \gamma_\mu) (\gamma^\alpha \wedge \gamma^\beta \wedge \gamma^\nu) } \\
\end{aligned}
\end{equation}

This dot product is reducible with the identity
\begin{equation}\label{eqn:enMTensor:200}
\begin{aligned}
(a \wedge b \wedge c) \cdot (d \wedge e \wedge f) &=
(((a \wedge b \wedge c) \cdot d) \cdot e) \cdot f
\end{aligned}
\end{equation}

leaving a completely antisymmetized sum

\begin{equation}\label{eqn:enMTensor:220}
\begin{aligned}
&
F^{\lambda\sigma} F_{\alpha\beta}
\eta^{\mu\mu'}
(
{\delta_\lambda}^\nu {\delta_\sigma}^\beta {\delta_{\mu'}}^\alpha
-{\delta_\lambda}^\nu {\delta_\sigma}^\alpha {\delta_{\mu'}}^\beta
-{\delta_\lambda}^\beta {\delta_\sigma}^\nu {\delta_{\mu'}}^\alpha
+{\delta_\lambda}^\alpha {\delta_\sigma}^\nu {\delta_{\mu'}}^\beta
+{\delta_\lambda}^\beta {\delta_\sigma}^\alpha {\delta_{\mu'}}^\nu
-{\delta_\lambda}^\alpha {\delta_\sigma}^\beta {\delta_{\mu'}}^\nu
) \\
&=
  F^{\nu\beta} F_{{\mu'}\beta} \eta^{\mu\mu'}
- F^{\nu\alpha} F_{\alpha{\mu'}} \eta^{\mu\mu'}
- F^{\beta\nu} F_{{\mu'}\beta} \eta^{\mu\mu'}
+ F^{\alpha\nu} F_{\alpha{\mu'}} \eta^{\mu\mu'}
+ F^{\beta\alpha} F_{\alpha\beta} \eta^{\mu\mu'} {\delta_{\mu'}}^\nu
- F^{\alpha\beta} F_{\alpha\beta} \eta^{\mu\mu'} {\delta_{\mu'}}^\nu
 \\
&=
4 F^{\nu\alpha} F_{{\mu'}\alpha} \eta^{\mu\mu'}
+ 2 F^{\beta\alpha} F_{\alpha\beta} \eta^{\mu\mu'} {\delta_{\mu'}}^\nu
 \\
&=
4 F^{\nu\alpha} {F^{\mu}}_{\alpha}
+ 2 F^{\beta\alpha} F_{\alpha\beta} \eta^{\mu\nu}
 \\
\end{aligned}
\end{equation}

Combining these we have
\begin{equation}\label{eqn:enMTensor:240}
\begin{aligned}
T^{\mu\nu}
&=
-\frac{\epsilon_0}{8} \left(
 4 F^{\alpha\mu} {F_{\alpha}}^{\nu}
+ 4 F^{\nu\alpha} {F^{\mu}}_{\alpha}
+ 2 F^{\beta\alpha} F_{\alpha\beta} \eta^{\mu\nu}
\right) \\
&=
\frac{\epsilon_0}{8} \left(
- 4 F^{\alpha\mu} {F_{\alpha}}^{\nu}
+ 4 F^{\alpha\mu} {F^{\nu}}_{\alpha}
+ 2 F^{\alpha\beta} F_{\alpha\beta} \eta^{\mu\nu}
\right) \\
\end{aligned}
\end{equation}


If by some miracle all the index manipulation worked out, we have
\begin{equation}\label{eqn:stressEnTen:miracle}
\begin{aligned}
T^{\mu\nu} &= {\epsilon_0} \left( F^{\alpha\mu} {F^{\nu}}_{\alpha} + \inv{4} F^{\alpha\beta} F_{\alpha\beta} \eta^{\mu\nu} \right)
\end{aligned}
\end{equation}

\subsubsection{Justifying some of the steps}

For justification of some of the
index manipulations of the \(F\) tensor components it is
helpful to think back to the definitions in terms of four vector potentials

\begin{equation}\label{eqn:enMTensor:260}
\begin{aligned}
F &= \grad \wedge A \\
&= \partial^\mu A^\nu \gamma_\mu \wedge \gamma_\nu \\
&= \partial_\mu A_\nu \gamma^\mu \wedge \gamma^\nu \\
&= \partial_\mu A^\nu \gamma^\mu \wedge \gamma_\nu \\
&= \partial^\mu A_\nu \gamma_\mu \wedge \gamma^\nu \\
&= \inv{2}(\partial^\mu A^\nu -\partial^\nu A^\mu ) \gamma_\mu \wedge \gamma_\nu \\
&= \inv{2}(\partial_\mu A_\nu -\partial_\nu A_\mu ) \gamma^\mu \wedge \gamma^\nu \\
&= \inv{2}(\partial_\mu A^\nu -\partial^\nu A_\mu ) \gamma^\mu \wedge \gamma_\nu \\
&= \inv{2}(\partial^\mu A_\nu -\partial_\nu A^\mu ) \gamma_\mu \wedge \gamma^\nu
\end{aligned}
\end{equation}

So with the shorthand
\begin{equation}\label{eqn:enMTensor:280}
\begin{aligned}
F^{\mu\nu} &= \partial^\mu A^\nu -\partial^\nu A^\mu \\
F_{\mu\nu} &= \partial_\mu A_\nu -\partial_\nu A_\mu \\
{F_{\mu}}^{\nu} &= \partial_\mu A^\nu -\partial^\nu A_\mu \\
{F^{\mu}}_{\nu} &= \partial^\mu A_\nu -\partial_\nu A^\mu
\end{aligned}
\end{equation}

We have
\begin{equation}\label{eqn:enMTensor:300}
\begin{aligned}
F
&= \inv{2}F^{\mu\nu} \gamma_\mu \wedge \gamma_\nu \\
&= \inv{2}F_{\mu\nu} \gamma^\mu \wedge \gamma^\nu \\
&= \inv{2}{F_\mu}^\nu \gamma^\mu \wedge \gamma_\nu \\
&= \inv{2}{F^\mu}_\nu \gamma_\mu \wedge \gamma^\nu
\end{aligned}
\end{equation}

In particular, and perhaps not obvious without the definitions handy, the following was used above

\begin{equation}\label{eqn:enMTensor:320}
\begin{aligned}
{F^{\mu}}_{\nu} &= -{F_{\nu}}^{\mu}
\end{aligned}
\end{equation}

\subsection{The divergence}

What is our divergence in tensor form?  This would be

\begin{equation}\label{eqn:enMTensor:340}
\begin{aligned}
\grad \cdot T(\gamma^\mu)
&= (\gamma^\alpha \partial_\alpha ) \cdot (T^{\mu\nu} \gamma_\nu) \\
\end{aligned}
\end{equation}

So we have
\begin{equation}\label{eqn:enMTensor:360}
\begin{aligned}
\grad \cdot T(\gamma^\mu)
&= \partial_\nu T^{\mu\nu}
\end{aligned}
\end{equation}

Ignoring the \(\epsilon_0\) factor for now, chain rule gives us

\begin{equation}\label{eqn:enMTensor:380}
\begin{aligned}
(\partial_\nu &F^{\alpha\mu}) {F^{\nu}}_{\alpha} +
F^{\alpha\mu} (\partial_\nu {F^{\nu}}_{\alpha}) +
\inv{2} (\partial_\nu F^{\alpha\beta}) F_{\alpha\beta} \eta^{\mu\nu} \\
&=
(\partial_\nu F^{\alpha\mu}) {F^{\nu}}_{\alpha} +
{F_{\alpha}}^{\mu}
(\partial_\nu F^{\nu\alpha}) +
\inv{2} (\partial_\nu F^{\alpha\beta}) F_{\alpha\beta} \eta^{\mu\nu}
\end{aligned}
\end{equation}

Only this center term is recognizable in terms of current since we
have
\begin{equation}\label{eqn:enMTensor:400}
\begin{aligned}
\grad \cdot F &= J/\epsilon_0 c
\end{aligned}
\end{equation}

Where the LHS is
\begin{equation}\label{eqn:enMTensor:420}
\begin{aligned}
\grad \cdot F
&= \gamma^\alpha \partial_\alpha \cdot \left( \inv{2} F^{\mu\nu} \gamma_\mu \wedge \gamma_\nu \right) \\
&= \inv{2} \partial_\alpha F^{\mu\nu} ( {\delta^\alpha}_\mu \gamma_\nu -{\delta^\alpha}_\nu \gamma_\mu ) \\
&= \partial_\mu F^{\mu\nu} \gamma_\nu
\end{aligned}
\end{equation}

So we have

\begin{equation}\label{eqn:enMTensor:440}
\begin{aligned}
\partial_\mu F^{\mu\nu}
&= (J \cdot \gamma^\nu)/\epsilon_0 c \\
&= ((J^\alpha \gamma_\alpha) \cdot \gamma^\nu)/\epsilon_0 c \\
&= J^\nu/\epsilon_0 c
\end{aligned}
\end{equation}

Or
\begin{equation}\label{eqn:enMTensor:460}
\begin{aligned}
\partial_\mu F^{\mu\nu} &= J^\nu/\epsilon_0 c
\end{aligned}
\end{equation}
%\partial_\nu F^{\nu\alpha} &= J^\alpha/\epsilon_0 c

So we have

\begin{equation}\label{eqn:enMTensor:480}
\begin{aligned}
\grad \cdot T(\gamma^\mu)
&= \epsilon_0\left(
(\partial_\nu F^{\alpha\mu}) {F^{\nu}}_{\alpha} +
\inv{2} (\partial_\nu F^{\alpha\beta}) F_{\alpha\beta} \eta^{\mu\nu}
\right)
+
{F_{\alpha}}^{\mu} J^\alpha/c
\end{aligned}
\end{equation}

So, to get the expected result the remaining two derivative terms must somehow cancel.  How to reduce these?  Let us look at twice that

\begin{equation}\label{eqn:enMTensor:500}
\begin{aligned}
2 (\partial_\nu &F^{\alpha\mu}) {F^{\nu}}_{\alpha} + (\partial_\nu F^{\alpha\beta}) F_{\alpha\beta} \eta^{\mu\nu} \\
&= 2 (\partial^\nu F^{\alpha\mu}) F_{\nu\alpha} + (\partial^\mu F^{\alpha\beta}) F_{\alpha\beta} \\
&= (\partial^\nu F^{\alpha\mu}) (F_{\nu\alpha} -F_{\alpha\nu}) + (\partial^\mu F^{\alpha\beta}) F_{\alpha\beta} \\
&=
(\partial^\alpha F^{\beta\mu}) F_{\alpha\beta}
+(\partial^\beta F^{\mu\alpha}) F_{\alpha\beta}
+ (\partial^\mu F^{\alpha\beta}) F_{\alpha\beta} \\
&=
(\partial^\alpha F^{\beta\mu} +\partial^\beta F^{\mu\alpha} + \partial^\mu F^{\alpha\beta}) F_{\alpha\beta} \\
\end{aligned}
\end{equation}

Ah, there is the trivector term of Maxwell's equation hiding in there.

\begin{equation}\label{eqn:enMTensor:520}
\begin{aligned}
0
&= \grad \wedge F \\
&= \gamma_\mu \partial^\mu \wedge \left(\inv{2} F^{\alpha\beta} (\gamma_\alpha \wedge \gamma_\beta) \right) \\
&= \inv{2} (\partial^\mu F^{\alpha\beta}) (\gamma_\mu \wedge \gamma_\alpha \wedge \gamma_\beta) \\
&= \inv{3!}
\left(
\partial^\mu F^{\alpha\beta}
+\partial^\alpha F^{\beta\mu}
+\partial^\beta F^{\mu\alpha}
\right)
(\gamma_\mu \wedge \gamma_\alpha \wedge \gamma_\beta)
\end{aligned}
\end{equation}

Since this is zero, each component of this trivector must separately equal zero, and we have

\begin{equation}\label{eqn:enMTensor:540}
\begin{aligned}
\partial^\mu F^{\alpha\beta} +\partial^\alpha F^{\beta\mu} +\partial^\beta F^{\mu\alpha} = 0
\end{aligned}
\end{equation}

So, where \(T^{\mu\nu}\) is defined by \eqnref{eqn:stressEnTen:miracle}, the final result is

\begin{equation}\label{eqn:stressEnTen:covariantTensor}
\begin{aligned}
\partial_\nu T^{\mu\nu} &= F^{\alpha\mu} J_\alpha/c
\end{aligned}
\end{equation}

%
% Copyright � 2012 Peeter Joot.  All Rights Reserved.
% Licenced as described in the file LICENSE under the root directory of this GIT repository.
%

%
%
\chapter{DC Power consumption formula for resistive load}
\label{chap:dcPower}
%\date{Jan 06, 2009.  dcPower.tex}

\section{Motivation}

Despite a lot of recent study of electrodynamics, faced with a simple electrical problem:

``What capacity generator would be required for an arc welder on a 30 Amp breaker using a 220 volt circuit''.

I could not think of how to answer this off the top of my head.  Back in school without hesitation I would have
been able to plug into \(P = I V\) to get a capacity estimation for the generator.

Having forgotten the formula, let us see how we get that \(P = I V\) relationship from Maxwell's equations.

\section{}

Having just derived the Poynting energy momentum density relationship from Maxwell's equations, let that be the starting
point

\begin{equation}\label{eqn:dcPower:20}
\begin{aligned}
\frac{d}{dt}\left(\frac{\epsilon_0}{2}\left(\BE^2 + c^2\BB^2 \right) \right) = - \inv{\mu_0} \left(\BE \cross \BB \right) - \BE \cdot \Bj
\end{aligned}
\end{equation}

The left hand side is the energy density time variation, which is power per unit volume, so we can integrate this
over a volume to determine the power associated with a change in the field.

\begin{equation}\label{eqn:dcPower:40}
\begin{aligned}
P = -\int dV \left( \inv{\mu_0} \left(\BE \cross \BB \right) + \BE \cdot \Bj \right)
\end{aligned}
\end{equation}

As a reminder, let us write the magnetic and electric fields in terms of potentials.

In terms of the ``native'' four potential our field is

\begin{equation}\label{eqn:dcPower:60}
\begin{aligned}
F
&= \BE + ic \BB \\
&= \grad \wedge A \\
&= \gamma^0 \gamma_k \partial_0 A^k + \gamma^j \gamma_0 \partial_j A^0 + \gamma^m \wedge \gamma_n \partial_m A^n \\
\end{aligned}
\end{equation}

The electric field is

\begin{equation}\label{eqn:dcPower:80}
\begin{aligned}
\BE &= \sum_k (\grad \wedge A) \cdot (\gamma^0 \gamma^k) \gamma_k \gamma_0 \\
\end{aligned}
\end{equation}

From this, with \(\phi = A^0\), and \(\BA = \sigma_k A^k\) we have
\begin{equation}\label{eqn:dcPower:100}
\begin{aligned}
\BE &= -\inv{c} \PD{t}{\BA} - \grad \phi \\
i\BB &= \spacegrad \wedge \BA
\end{aligned}
\end{equation}

Now, the arc welder is (I think) a DC device, and to
get a rough idea of what it requires lets just assume that its a rectifier that outputs RMS DC.
So if we make this simplification, and assume that we have a
purely resistive load (ie: no inductance and therefore no magnetic fields) and a DC supply and constant current, then
we eliminate the vector potential terms.

This wipes out the \(\BB\) and the Poynting vector, and leaves our electric field specified in terms
of the potential difference across the load \(\BE = -\spacegrad \phi\).

That is
\begin{equation}\label{eqn:dcPower:120}
\begin{aligned}
P &= \int dV (\spacegrad \phi) \cdot \Bj
\end{aligned}
\end{equation}

Suppose we are integrating over the length of a uniformly resistive load with some fixed cross sectional area.  \(\Bj dV\) is then the magnitude of the current directed along the wire for its length.  This basically leaves us with a line integral over the length of the wire that we are measuring our potential drop over so we are left with just

\begin{equation}\label{eqn:dcPower:140}
\begin{aligned}
P &= (\delta \phi) I
\end{aligned}
\end{equation}

This \(\delta \phi\) is just our voltage drop \(V\), and this gives us the desired \(P = I V\) equation.
Now, I also recall from school
now that I think about it that \(P = I V\) also applied to inductive loads, but it required that \(I\) and \(V\) be phasors that
represented the sinusoidal currents and sources.  A good followup exercise would be to show from Maxwell's equations
that this is in fact valid.  Eventually I will know the origin of all the formulas from my old engineering courses.

%
% Copyright � 2012 Peeter Joot.  All Rights Reserved.
% Licenced as described in the file LICENSE under the root directory of this GIT repository.
%

% 
% 
\chapter{Rayleigh-Jeans Law Notes}\label{chap:PJrayleighJeans}
\index{Rayleigh-Jeans law}
%\date{Dec 27, 2008.  rayleighJeans.tex}

\section{Motivation}

Fill in the gaps for a reading of the
initial parts of the Rayleigh-Jeans discussion of \citep{bohm1989qt}.

\section{2. Electromagnetic energy}

Energy of the field given to be:

\begin{equation}\label{eqn:rayleighJeans:20}
\begin{aligned}
E = \inv{8\pi} \int (\EE^2 + \HH^2)
\end{aligned}
\end{equation}

I still do not really know where this comes from.
Could perhaps justify this with a Hamiltonian of a field (although this is
uncomfortably abstract).

With the particle Hamiltonian we have

\begin{equation}\label{eqn:rayleighJeans:40}
\begin{aligned}
H = \qdot_i p_i -\LL
\end{aligned}
\end{equation}

What is the field equivalent of this?  Try to get the feel for this with some simple fields (such as the one dimensional vibrating string), and the Coulomb field.  For the physical case, do this with both the Hamiltonian approach and a physical limiting argument.

\section{3. Electromagnetic Potentials}

Bohm writes Maxwell's equations in non-SI units, and also, naturally, not in STA form which would be somewhat more natural for a gauge
discussion.

\begin{equation}\label{eqn:rayleighJeans:60}
\begin{aligned}
\spacegrad \cross \EE &= -\inv{c} \partial_t \HH \\
\spacegrad \cdot \EE &= 4 \pi \rho \\
\spacegrad \cross \HH &= \inv{c} \partial_t \EE + 4 \pi \Bj \\
\spacegrad \cdot \HH &= 0
\end{aligned}
\end{equation}

In STA form this is

\begin{equation}\label{eqn:rayleighJeans:80}
\begin{aligned}
\spacegrad \EE &= - \partial_0 i\HH + 4 \pi \rho \\
\spacegrad i\HH &= -\partial_0 \EE - 4 \pi \Bj \\
\end{aligned}
\end{equation}

Or
\begin{equation}\label{eqn:rayleigh_jeans:maxwellNotQuiteCovariant}
\begin{aligned}
\spacegrad (\EE + i\HH) + \partial_0 (\EE + i\HH) &= 4 \pi (\rho - \Bj)
\end{aligned}
\end{equation}

Left multiplying by \(\gamma_0\) gives

\begin{equation}\label{eqn:rayleighJeans:100}
\begin{aligned}
\gamma_0 \spacegrad 
&= \gamma_0 \sum_k \sigma_k \partial_k \\
&= \gamma_0 \sum_k \gamma_k \gamma_0 \partial_k \\
&= -\sum_k \gamma_k \partial_k \\
&= \gamma^k \partial_k \\
\end{aligned}
\end{equation}

and

\begin{equation}\label{eqn:rayleighJeans:120}
\begin{aligned}
\gamma_0 \Bj 
&= \sum_k \gamma_0 \sigma_k j^k \\
&= -\sum_k \gamma_k j^k,
\end{aligned}
\end{equation}

so with \(J^0 = \rho\), \(J^k = j^k\) and \(J = \gamma_\mu J^\mu\), we have

\begin{equation}\label{eqn:rayleighJeans:140}
\begin{aligned}
\gamma^\mu \partial_\mu (\EE + i\HH) &= 4 \pi J
\end{aligned}
\end{equation}

and finally with \(F = \EE + i\HH\), we have Maxwell's equation in covariant form

\begin{equation}\label{eqn:rayleighJeans:160}
\begin{aligned}
\grad F &= 4 \pi J.
\end{aligned}
\end{equation}

Next it is stated that general solutions can be expressed as

\begin{equation}\label{eqn:rayleigh_jeans:fieldsFromPotentials}
\begin{aligned}
\HH &= \spacegrad \cross \Ba \\
\EE &= - \inv{c} \PD{t}{\Ba} - \spacegrad \phi
\end{aligned}
\end{equation}

Let us double check that this jives with the bivector potential solution \(F = \grad \wedge A = \EE + i\HH\).  Let us split our bivector
into spacetime and spatial components by the conjugate operation

\begin{equation}\label{eqn:rayleighJeans:180}
\begin{aligned}
F^\conj &=\gamma_0 F \gamma_0 \\
&= \gamma_0 \gamma^\mu \wedge \gamma^\nu \partial_\mu A_\mu \gamma_0 \\
&=
\left\{
\begin{array}{l l}
0 & \quad \mbox{if \(\mu = \nu\)} \\
\gamma^\mu \gamma^\nu \partial_\mu A_\nu & \quad \mbox{if \(\mu \in \{1,2,3\}\), and \(\nu \in \{1,2,3\}\)} \\
-\gamma^\mu \gamma^\nu \partial_\mu A_\nu & \quad \mbox{one of \(\mu = 0\) or \(\nu = 0\) } \\
\end{array} \right.
\end{aligned}
\end{equation}

\begin{equation}\label{eqn:rayleighJeans:200}
\begin{aligned}
F 
&= \EE + i\HH \\
&= \inv{2}(F - F^\conj) + \inv{2}(F + F^\conj) \\
&= \left(\gamma^k \wedge \gamma^0 \partial_k A_0 +\gamma^0 \wedge \gamma^k \partial_0 A_k\right) + \left(\gamma^a \wedge \gamma^b \partial_a A_b\right) \\
&= -\left(\sum_k \sigma_k \partial_k A^0 + \partial_0 \sigma_k A^k\right) + i\left(\epsilon_{abc}\sigma_a \partial_b A^c\right) \\
\end{aligned}
\end{equation}

So, with \(\Ba = \sigma_k A^k\), and \(\phi = A^0\), we do have equations \eqnref{eqn:rayleigh_jeans:fieldsFromPotentials} as identical to \(F = \grad \wedge A\).

Now how about the gauge variations of the fields?  Bohm writes that we can alter the potentials by

\begin{equation}\label{eqn:rayleigh_jeans:gauge}
\begin{aligned}
\Ba' &= \Ba - \spacegrad \psi \\
\phi' &= \phi + \inv{c}\PD{t}{\psi}
\end{aligned}
\end{equation}

How does this translate to an alteration of the four potential?  For the vector potential we have

\begin{equation}\label{eqn:rayleighJeans:220}
\begin{aligned}
\sigma_k {A^k}' &= \sigma_k A^k - \sigma_k \partial \psi \\
\gamma_k \gamma_0 {A^k}' &= \gamma_k \gamma_0 A^k - \gamma_k \gamma_0 \partial_k \psi \\
-\gamma_0 \gamma_k {A^k}' &= -\gamma_0 \gamma_k A^k - \gamma_0 \gamma^k \partial_k \psi \\
\gamma_k {A^k}' &= \gamma_k A^k + \gamma^k \partial_k \psi \\
\end{aligned}
\end{equation}

with \(\phi = A^0\), add in the \(\phi\) term

\begin{equation}\label{eqn:rayleighJeans:240}
\begin{aligned}
\gamma_0 \phi' &= \gamma_0 \phi + \gamma_0 \PD{x^0}{\psi} \\
\gamma_0 \phi' &= \gamma_0 \phi + \gamma^0 \PD{x^0}{\psi}
\end{aligned}
\end{equation}

For
\begin{equation}\label{eqn:rayleighJeans:260}
\begin{aligned}
\gamma_\mu {A^\mu}' &= \gamma_\mu A^\mu + \gamma^\mu \partial_\mu \psi \\
\end{aligned}
\end{equation}

Which is just a statement that we can add a spacetime gradient to our vector potential without altering the field equation:

\begin{equation}\label{eqn:rayleighJeans:280}
\begin{aligned}
A' &= A + \grad \psi
\end{aligned}
\end{equation}

Let us verify that this does in fact not alter Maxwell's equation.

\begin{equation}\label{eqn:rayleighJeans:300}
\begin{aligned}
\grad (\grad \wedge (A + \grad \psi) &= 4 \pi J
\grad (\grad \wedge A) + \grad (\grad \wedge \grad \psi) &= 
\end{aligned}
\end{equation}

Since \(\grad \wedge \grad = 0\) we have

\begin{equation}\label{eqn:rayleighJeans:320}
\begin{aligned}
\grad (\grad \wedge A') = \grad (\grad \wedge A)
\end{aligned}
\end{equation}

Now the statement that \(\grad \wedge \grad\) as an operator equals zero, just by virtue of \(\grad\) being a vector is worth explicit
confirmation.  Let us expand that to verify

\begin{equation}\label{eqn:rayleighJeans:340}
\begin{aligned}
\grad \wedge \grad \psi 
&= \gamma^\mu \wedge \gamma^\nu \partial_\mu \partial_\nu \psi \\
&= \left(\sum_{\mu < \nu} + \sum_{\nu < \mu}\right) \gamma^\mu \wedge \gamma^\nu \partial_\mu \partial_\nu \psi \\
&= \sum_{\mu < \nu} \gamma^\mu \wedge \gamma^\nu (\partial_\mu \partial_\nu \psi - \partial_\nu \partial_\mu \psi) \\
\end{aligned}
\end{equation}

So, we see that we additionally need the field variable \(\psi\) to be sufficiently continuous for mixed partial equality for the
statement that \(\grad \wedge \grad = 0\) to be valid.  Assuming that continuity is taken as a given the confirmation of the invariance under this transformation is thus complete.

Now, Bohm says it is possible to pick \(\spacegrad \cdot \Ba' = 0\).  From \eqnref{eqn:rayleigh_jeans:gauge} that implies

\begin{equation}\label{eqn:rayleighJeans:360}
\begin{aligned}
\spacegrad \cdot \Ba'
&= \spacegrad \cdot \Ba - \spacegrad \cdot \spacegrad \psi \\
&= \spacegrad \cdot \Ba - \spacegrad^2 \psi = 0 \\
\end{aligned}
\end{equation}

So, provided we can find a solution to the Poisson equation

\begin{equation}\label{eqn:rayleigh_jeans:psiForDivAPrimeEqZero}
\begin{aligned}
\spacegrad^2 \psi = \spacegrad \cdot \Ba
\end{aligned}
\end{equation}

one can find a \(\psi, \Ba\) gauge transformation that has the particular quality that \(\spacegrad \cdot \Ba' = 0\).

That solution, from \eqnref{eqn:rayleigh_jeans:laplacianOfPoisson} is

\begin{equation}\label{eqn:rayleighJeans:380}
\begin{aligned}
\psi(\Br) = -\inv{4\pi}\int (\spacegrad' \cdot \Ba(\Br')) dV' \inv{\Abs{\Br-\Br'}}
\end{aligned}
\end{equation}

The corollary to this
is that one may similarly impose a requirement that \(\spacegrad \cdot \Ba = 0\), since if that is not the case, some \(\Ba'\) can be added to the vector potential to make that the case.

FIXME: handwaving description here.  Show with a math statement with \(\Ba \rightarrow \Ba'\).

\subsection{Free space solutions}

From \eqnref{eqn:rayleigh_jeans:maxwellNotQuiteCovariant} and \eqnref{eqn:rayleigh_jeans:fieldsFromPotentials}
the free space solution to Maxwell's
equation must satisfy 

\begin{equation}\label{eqn:rayleighJeans:400}
\begin{aligned}
0 
&= \left(\spacegrad + \partial_0\right) (\EE + i\HH) \\
&= \left(\spacegrad + \partial_0\right) \left(- \partial_0{\Ba} - \spacegrad \phi + \spacegrad \wedge \Ba \right) \\
&= - \spacegrad \partial_0{\Ba} - \spacegrad^2 \phi + \spacegrad (\spacegrad \wedge \Ba) 
 - \partial_{00}{\Ba} - \partial_0 \spacegrad \phi + \partial_0 (\spacegrad \wedge \Ba) \\
&= - \spacegrad \cdot \partial_0{\Ba} - \spacegrad^2 \phi + \spacegrad \cdot (\spacegrad \wedge \Ba) 
 - \partial_{00}{\Ba} - \partial_0 \spacegrad \phi  \\
\end{aligned}
\end{equation}

Since the scalar and vector parts of this equation must separately equal zero we have

\begin{equation}\label{eqn:rayleighJeans:420}
\begin{aligned}
0 &= - \partial_0 \spacegrad \cdot {\Ba} - \spacegrad^2 \phi \\
0 &= \spacegrad \cdot (\spacegrad \wedge \Ba) - \partial_{00}{\Ba} - \partial_0 \spacegrad \phi  \\
\end{aligned}
\end{equation}

If one picks a gauge transformation such that \(\spacegrad \cdot \Ba = 0\) we then have
\begin{equation}\label{eqn:rayleighJeans:440}
\begin{aligned}
0 &= \spacegrad^2 \phi \\
0 &= \spacegrad^2 \Ba - \partial_{00}{\Ba} - \partial_0 \spacegrad \phi  \\
\end{aligned}
\end{equation}

For the first Bohm argues that ``It is well known that the only solution of this equation that is regular over all space is \(\phi = 0\)'', and anything else implies charge in the region.  What does regular mean here?  I suppose this seems like a reasonable enough statement, but I think the proper way to think about this is really that one has picked the covariant gauge \(\grad \cdot A = 0\) (that is simpler anyhow).  With an acceptance of the \(\phi =0\) argument one is left with the vector potential wave equation which was the desired goal of that section.

Note: The following \href{http://www.physicsforums.com/showthread.php?t=281874}{physicsforums thread} discusses some of the confusion I had in this bit of text.

\subsection{Doing this all directly}

Now, the whole point of the gauge transformation appears to be to show that one can find the four wave equation solutions for 
Maxwell's equation by picking a specific gauge.  This is actually trivial to do from the STA Maxwell equation:

\begin{equation}\label{eqn:rayleighJeans:460}
\begin{aligned}
\grad (\grad \wedge A) = \grad( \grad A - \grad \cdot A ) = \grad^2 A - \grad (\grad \cdot A) = 4 \pi J
\end{aligned}
\end{equation}

So, if one picks a gauge transformation with \(\grad \cdot A = 0\), one has

\begin{equation}\label{eqn:rayleighJeans:480}
\begin{aligned}
\grad^2 A = 4 \pi J
\end{aligned}
\end{equation}

This is precisely the four wave equations desired
\begin{equation}\label{eqn:rayleighJeans:500}
\begin{aligned}
\partial_\nu\partial^\nu A^\mu = 4 \pi J^\mu
\end{aligned}
\end{equation}

FIXME: show the precise gauge transformation \(A \rightarrow A'\) that leads to \(\grad \cdot A = 0\).

\section{Energy density.  Get the units right with these CGS equations}

We will want to calculate the equivalent of 

\begin{equation}\label{eqn:rayleighJeans:520}
\begin{aligned}
U = \frac{\epsilon_0}{2} (\BE^2 + c^2 \BB^2)
\end{aligned}
\end{equation}

but are faced with the alternate units of Bohm's text.  Let us repeat the
derivation of the electric field energy from \chapcite{PJelectricFieldEnergy}
in the CGS units directly from Maxwell's equation

\begin{equation}\label{eqn:rayleigh_jeans:maxwell}
\begin{aligned}
F &= \EE + i\HH \\
J &= (\rho + \Bj) \gamma_0 \\
\grad F &= 4 \pi J
\end{aligned}
\end{equation}

to ensure we get it right.

To start with we our spacetime split of \eqnref{eqn:rayleigh_jeans:maxwell} is

\begin{equation}\label{eqn:rayleighJeans:540}
\begin{aligned}
( \partial_0 + \spacegrad ) (\EE + \HH) = 4 \pi (\rho - \Bj)
\end{aligned}
\end{equation}

The scalar part gives us Coulomb's law

\begin{equation}\label{eqn:rayleighJeans:560}
\begin{aligned}
\spacegrad \cdot \EE = 4 \pi \rho 
\end{aligned}
\end{equation}

Gauss's theorem applied to a spherical constant density charge distribution
gives us
\begin{equation}\label{eqn:rayleighJeans:580}
\begin{aligned}
\int \spacegrad \cdot \EE dV &= 4 \pi \int \rho dV \\
\implies \\
\int {\EE} \cdot \ncap dA &= 4 \pi Q \\
\implies \\
\Abs{\EE} 4 \pi r^2 &= 4 \pi Q \\
\end{aligned}
\end{equation}

so we have the expected ``unitless'' Coulomb law force equation

\begin{equation}\label{eqn:rayleighJeans:600}
\begin{aligned}
{\BF} = q\EE = \frac{q Q }{r^2} \rcap
\end{aligned}
\end{equation}

So far so good.  Next introduction of a potential.  For statics we do not care
about the four vectors and stick with the old fashion definition of the potential \(\phi\) indirectly in terms of \(\EE\).  That is

\begin{equation}\label{eqn:rayleighJeans:620}
\begin{aligned}
\EE = -\spacegrad \phi
\end{aligned}
\end{equation}

A line integral of this gives us \(\phi\) in terms of \(\EE\)
\begin{equation}\label{eqn:rayleighJeans:640}
\begin{aligned}
-\int \EE \cdot \Br 
&= \int \spacegrad \phi \cdot d\Br \\
&= \phi - \phi_0 \\
\end{aligned}
\end{equation}

With \(\phi(\infty) = 0\) this is

\begin{equation}\label{eqn:rayleighJeans:660}
\begin{aligned}
\phi(d)
&= -\int_{r=\infty}^d \EE \cdot d\Br  \\
&= -\int_{r=\infty}^d \frac{Q}{r^2} \rcap \cdot d\Br  \\
&= -\int_{r=\infty}^d \frac{Q}{r^2} dr  \\
&= \frac{Q}{d} \\
%&= -\int \frac{\rho dV}{r^2} dr 
\end{aligned}
\end{equation}

Okay.  Now onto the electrostatic energy.  The work done to move one charge from infinite to some separation \(d\) of another like sign charge 
is

\begin{equation}\label{eqn:rayleighJeans:680}
\begin{aligned}
\int_{r=\infty}^{d} F \cdot d\Br 
&= \int_{r= \infty}^d \frac{q Q}{r^2} \rcap \cdot (-d\Br)  \\
&= -\int_{r= \infty}^d \frac{qQ}{r^2} dr  \\
&= \frac{qQ}{d} \\
&= q_1 \phi_2(d) \\
\end{aligned}
\end{equation}


For a distribution of discrete charges we have to sum over all pairs

\begin{equation}\label{eqn:rayleighJeans:700}
\begin{aligned}
W 
&= \sum_{i \ne j} \frac{q_i q_j}{d_{ij}} \\
&= \sum_{i,j} \inv{2} \frac{q_i q_j}{d_{ij}} \\
\end{aligned}
\end{equation}

In a similar fashion we can do a continuous variation, employing a double summation over all space.  Note first
that we can also write
one of the charge densities in terms of the potential 

\begin{equation}\label{eqn:rayleighJeans:720}
\begin{aligned}
\EE &= - \spacegrad \phi \\
\implies \\
\spacegrad \cdot \EE 
&= - \spacegrad \cdot \spacegrad \phi \\
&= - \spacegrad^2 \phi \\
&= 4 \pi \rho
\end{aligned}
\end{equation}

\begin{equation}\label{eqn:rayleighJeans:740}
\begin{aligned}
W 
&= \inv{2} \int \rho \phi(r) dV \\
&= -\inv{8\pi} \int \phi \spacegrad^2 \phi dV \\
&= \inv{8\pi} \int ( (\spacegrad \phi)^2 - \spacegrad \cdot (\phi \spacegrad \phi)) dV \\
&= \inv{8\pi} \int (-\EE)^2 - \inv{8\pi} \int (\phi \spacegrad \phi) \cdot \ncap dA
\end{aligned}
\end{equation}

Here the one and two subscripts could be dropped with a switch to the total charge density and the potential from this complete charge superposition.
For our final result we have an energy density of

\begin{equation}\label{eqn:rayleighJeans:760}
\begin{aligned}
\frac{dW}{dV} &= \inv{8\pi} {\EE}^2 
\end{aligned}
\end{equation}

\section{Auxiliary details}

\subsection{Confirm Poisson solution to Laplacian}
\index{Laplacian!Poisson solution}

Bohm lists the solution for \eqnref{eqn:rayleigh_jeans:psiForDivAPrimeEqZero} (a Poisson integral), but I forget how one shows this.  I can not figure out how to integrate this Laplacian, but it is simple enough to confirm this by back substitution.

Suppose one has

\begin{equation}\label{eqn:rayleighJeans:780}
\begin{aligned}
\psi = \int \frac{\rho(\Br')}{\Abs{\Br - \Br'}} dV'
\end{aligned}
\end{equation}

We can take the Laplacian by direct differentiation under the integration sign

\begin{equation}\label{eqn:rayleighJeans:800}
\begin{aligned}
\spacegrad^2 \psi = \int {\rho(\Br')} dV' \spacegrad^2 \inv{\Abs{\Br - \Br'}}
\end{aligned}
\end{equation}

To evaluate the Laplacian we need

\begin{equation}\label{eqn:rayleighJeans:820}
\begin{aligned}
\PD{x_i}{\Abs{\Br - \Br'}^k} 
&= \PD{x_i}{} \left(\sum_j (x_j - x_j')^2 \right)^{k/2} \\
&= k { 2\Abs{\Br - \Br'}^{k-2}} \PD{x_i}{} \left(\sum_j (x_j - x_j')^2 \right) \\
%&= k { 2\Abs{\Br - \Br'}^{k-2}} 2 (x_i - x_i')
&= k { \Abs{\Br - \Br'}^{k-2}} (x_i - x_i')
\end{aligned}
\end{equation}

So we have
\begin{equation}\label{eqn:rayleighJeans:840}
\begin{aligned}
\PD{x_i}{} \PD{x_i}{} {\Abs{\Br - \Br'}^{-1}} 
&= 
- (x_i - x_i') \PD{x_i}{}{ \inv{ \Abs{\Br - \Br'}^3} } 
- \inv{ \Abs{\Br - \Br'}^3} \PD{x_i}{ (x_i - x_i') } \\
&= 
3 (x_i - x_i')^2 { \inv{ \Abs{\Br - \Br'}^5} } 
- \inv{ \Abs{\Br - \Br'}^3} \\
\end{aligned}
\end{equation}

So, provided \(\Br \ne \Br'\) we have

\begin{equation}\label{eqn:rayleighJeans:860}
\begin{aligned}
\spacegrad^2 \psi &= 
3 (\Br - \Br')^2 \inv{ \Abs{\Br - \Br'}^5} 
- 3 \inv{ \Abs{\Br - \Br'}^3} \\
&= 0 
\end{aligned}
\end{equation}

Observe that this is true only for \R{3}.  Now, one is left with only an integral around a neighborhood around the point \(\Br\) which can be made small enough that \(\rho(\Br') = \rho(\Br)\) in that volume can be taken as constant.

\begin{equation}\label{eqn:rayleighJeans:880}
\begin{aligned}
\spacegrad^2 \psi 
&= \rho(\Br) \int dV' \spacegrad^2 \inv{\Abs{\Br - \Br'}} \\
&= \rho(\Br) \int dV' \spacegrad \cdot \spacegrad \inv{\Abs{\Br - \Br'}} \\
&= -\rho(\Br) \int dV' \spacegrad \cdot \frac{(\Br -\Br')}{\Abs{\Br - \Br'}^3} \\
\end{aligned}
\end{equation}

Now, if the divergence in this integral was with respect to the primed variable that ranges over the infinitesimal volume, then this could be converted to a surface integral.  
Observe that a radial expansion of this divergence allows for convenient change of variables to the primed \(x_i'\) coordinates

\begin{equation}\label{eqn:rayleighJeans:900}
\begin{aligned}
\spacegrad \cdot \frac{(\Br -\Br')}{\Abs{\Br - \Br'}^3}
&= 
\left(\frac{\Br - \Br'}{\Abs{\Br-\Br'}} \PD{\Abs{\Br-\Br'}}{}\right) \cdot
\left(\frac{\Br - \Br'}{\Abs{\Br-\Br'}} \inv{\Abs{\Br-\Br'}^2}\right) \\
&= 
\PD{\Abs{\Br'-\Br}}{} {\Abs{\Br'-\Br}^{-2}} \\
&= 
\left(\frac{\Br' - \Br}{\Abs{\Br'-\Br}} \PD{\Abs{\Br'-\Br}}{}\right) \cdot
\left(\frac{\Br' - \Br}{\Abs{\Br'-\Br}} \inv{\Abs{\Br'-\Br}^2}\right) \\
&= \spacegrad' \cdot \frac{(\Br'-\Br)}{\Abs{\Br' - \Br}^3}
\end{aligned}
\end{equation}

Now, since \(\Br'-\Br\) is in the direction of the outwards normal the divergence theorem can be used

\begin{equation}\label{eqn:rayleighJeans:920}
\begin{aligned}
\spacegrad^2 \psi 
&= -\rho(\Br) \int dV' \spacegrad' \cdot \frac{(\Br' -\Br)}{\Abs{\Br' - \Br}^3} \\
&= -\rho(\Br) \int_{\partial V'} dA' \inv{\Abs{\Br' - \Br}^2} \\
\end{aligned}
\end{equation}

Picking a spherical integration volume, for which the radius is constant \(R = \Abs{\Br'-\Br}\), we have

\begin{equation}\label{eqn:rayleighJeans:940}
\begin{aligned}
\spacegrad^2 \psi 
&= -\rho(\Br) 4 \pi R^2 \inv{R^2} \\
\end{aligned}
\end{equation}

In summary this is

\begin{equation}\label{eqn:rayleigh_jeans:laplacianOfPoisson}
\begin{aligned}
\psi &= \int \frac{\rho(\Br')}{\Abs{\Br - \Br'}} dV' \\
\spacegrad^2 \psi &= - 4 \pi \rho(\Br)
\end{aligned}
\end{equation}

Having written this out I recall that the same approach was used in
\citep{schwartz1987pe} (there it was to calculate \(\spacegrad \cdot \BE\) in terms of the charge density, but the ideas are all the same.)

\part{Lagrangian Topics}
\documentclass{article}

\usepackage{amsmath}
\usepackage{mathpazo}

%
% shorthand for bold symbols, convenient for vectors and matrices
%
\newcommand{\Ba}[0]{\mathbf{a}}
\newcommand{\Bb}[0]{\mathbf{b}}
\newcommand{\Bc}[0]{\mathbf{c}}
\newcommand{\Bd}[0]{\mathbf{d}}
\newcommand{\Be}[0]{\mathbf{e}}
\newcommand{\Bf}[0]{\mathbf{f}}
\newcommand{\Bg}[0]{\mathbf{g}}
\newcommand{\Bh}[0]{\mathbf{h}}
\newcommand{\Bi}[0]{\mathbf{i}}
\newcommand{\Bj}[0]{\mathbf{j}}
\newcommand{\Bk}[0]{\mathbf{k}}
\newcommand{\Bl}[0]{\mathbf{l}}
\newcommand{\Bm}[0]{\mathbf{m}}
\newcommand{\Bn}[0]{\mathbf{n}}
\newcommand{\Bo}[0]{\mathbf{o}}
\newcommand{\Bp}[0]{\mathbf{p}}
\newcommand{\Bq}[0]{\mathbf{q}}
\newcommand{\Br}[0]{\mathbf{r}}
\newcommand{\Bs}[0]{\mathbf{s}}
\newcommand{\Bt}[0]{\mathbf{t}}
\newcommand{\Bu}[0]{\mathbf{u}}
\newcommand{\Bv}[0]{\mathbf{v}}
\newcommand{\Bw}[0]{\mathbf{w}}
\newcommand{\Bx}[0]{\mathbf{x}}
\newcommand{\By}[0]{\mathbf{y}}
\newcommand{\Bz}[0]{\mathbf{z}}
\newcommand{\BA}[0]{\mathbf{A}}
\newcommand{\BB}[0]{\mathbf{B}}
\newcommand{\BC}[0]{\mathbf{C}}
\newcommand{\BD}[0]{\mathbf{D}}
\newcommand{\BE}[0]{\mathbf{E}}
\newcommand{\BF}[0]{\mathbf{F}}
\newcommand{\BG}[0]{\mathbf{G}}
\newcommand{\BH}[0]{\mathbf{H}}
\newcommand{\BI}[0]{\mathbf{I}}
\newcommand{\BJ}[0]{\mathbf{J}}
\newcommand{\BK}[0]{\mathbf{K}}
\newcommand{\BL}[0]{\mathbf{L}}
\newcommand{\BM}[0]{\mathbf{M}}
\newcommand{\BN}[0]{\mathbf{N}}
\newcommand{\BO}[0]{\mathbf{O}}
\newcommand{\BP}[0]{\mathbf{P}}
\newcommand{\BQ}[0]{\mathbf{Q}}
\newcommand{\BR}[0]{\mathbf{R}}
\newcommand{\BS}[0]{\mathbf{S}}
\newcommand{\BT}[0]{\mathbf{T}}
\newcommand{\BU}[0]{\mathbf{U}}
\newcommand{\BV}[0]{\mathbf{V}}
\newcommand{\BW}[0]{\mathbf{W}}
\newcommand{\BX}[0]{\mathbf{X}}
\newcommand{\BY}[0]{\mathbf{Y}}
\newcommand{\BZ}[0]{\mathbf{Z}}

\newcommand{\Bzero}[0]{\mathbf{0}}
\newcommand{\Btheta}[0]{\boldsymbol{\theta}}
\newcommand{\Btau}[0]{\boldsymbol{\tau}}
\newcommand{\Bomega}[0]{\boldsymbol{\omega}}

%
% shorthand for unit vectors
%
\newcommand{\acap}[0]{\hat{\Ba}}
\newcommand{\bcap}[0]{\hat{\Bb}}
\newcommand{\ccap}[0]{\hat{\Bc}}
\newcommand{\dcap}[0]{\hat{\Bd}}
\newcommand{\ecap}[0]{\hat{\Be}}
\newcommand{\fcap}[0]{\hat{\Bf}}
\newcommand{\gcap}[0]{\hat{\Bg}}
\newcommand{\hcap}[0]{\hat{\Bh}}
\newcommand{\icap}[0]{\hat{\Bi}}
\newcommand{\jcap}[0]{\hat{\Bj}}
\newcommand{\kcap}[0]{\hat{\Bk}}
\newcommand{\lcap}[0]{\hat{\Bl}}
\newcommand{\mcap}[0]{\hat{\Bm}}
\newcommand{\ncap}[0]{\hat{\Bn}}
\newcommand{\ocap}[0]{\hat{\Bo}}
\newcommand{\pcap}[0]{\hat{\Bp}}
\newcommand{\qcap}[0]{\hat{\Bq}}
\newcommand{\rcap}[0]{\hat{\Br}}
\newcommand{\scap}[0]{\hat{\Bs}}
\newcommand{\tcap}[0]{\hat{\Bt}}
\newcommand{\ucap}[0]{\hat{\Bu}}
\newcommand{\vcap}[0]{\hat{\Bv}}
\newcommand{\wcap}[0]{\hat{\Bw}}
\newcommand{\xcap}[0]{\hat{\Bx}}
\newcommand{\ycap}[0]{\hat{\By}}
\newcommand{\zcap}[0]{\hat{\Bz}}
\newcommand{\thetacap}[0]{\hat{\Btheta}}

%
% to write R^n and C^n in a distinguishable fashion.  Perhaps change this
% to the double lined characters upon figuring out how to do so.
%
\newcommand{\C}[1]{$\mathbb{C}^{#1}$}
\newcommand{\R}[1]{$\mathbb{R}^{#1}$}

%
% various generally useful helpers
%

% derivative of #1 wrt. #2:
\newcommand{\D}[2] {\frac {d#2} {d#1}}

\newcommand{\inv}[1]{\frac{1}{#1}}
\newcommand{\cross}[0]{\times}

\newcommand{\abs}[1]{\lvert{#1}\rvert}
\newcommand{\norm}[1]{\lVert{#1}\rVert}
\newcommand{\innerprod}[2]{\langle{#1}, {#2}\rangle}
\newcommand{\dotprod}[2]{{#1} \cdot {#2}}
\newcommand{\bdotprod}[2]{\left({#1} \cdot {#2}\right)}
\newcommand{\crossprod}[2]{{#1} \cross {#2}}
\newcommand{\tripleprod}[3]{\dotprod{\left(\crossprod{#1}{#2}\right)}{#3}}

\DeclareMathOperator{\Proj}{Proj}
\DeclareMathOperator{\Span}{span}
\DeclareMathOperator{\Sgn}{sgn}
\DeclareMathOperator{\Area}{Area}
\DeclareMathOperator{\Volume}{Volume}

%
% A few miscellaneous things specific to this document
%
\newcommand{\crossop}[1]{\crossprod{#1}{}}

% R2 vector.
\newcommand{\VectorTwo}[2]{
\begin{bmatrix}
 {#1} \\
 {#2}
\end{bmatrix}
}

\newcommand{\VectorN}[1]{
\begin{bmatrix}
{#1}_1 \\
{#1}_2 \\
\vdots \\
{#1}_N \\
\end{bmatrix}
}

\newcommand{\DETuvij}[4]{
\begin{vmatrix}
 {#1}_{#3} & {#1}_{#4} \\
 {#2}_{#3} & {#2}_{#4}
\end{vmatrix}
}

\newcommand{\DETuvwijk}[6]{
\begin{vmatrix}
 {#1}_{#4} & {#1}_{#5} & {#1}_{#6} \\
 {#2}_{#4} & {#2}_{#5} & {#2}_{#6} \\
 {#3}_{#4} & {#3}_{#5} & {#3}_{#6}
\end{vmatrix}
}

\newcommand{\DETuvwxijkl}[8]{
\begin{vmatrix}
 {#1}_{#5} & {#1}_{#6} & {#1}_{#7} & {#1}_{#8} \\
 {#2}_{#5} & {#2}_{#6} & {#2}_{#7} & {#2}_{#8} \\
 {#3}_{#5} & {#3}_{#6} & {#3}_{#7} & {#3}_{#8} \\
 {#4}_{#5} & {#4}_{#6} & {#4}_{#7} & {#4}_{#8} \\
\end{vmatrix}
}

%\newcommand{\DETuvwxyijklm}[10]{
%\begin{vmatrix}
% {#1}_{#6} & {#1}_{#7} & {#1}_{#8} & {#1}_{#9} & {#1}_{#10} \\
% {#2}_{#6} & {#2}_{#7} & {#2}_{#8} & {#2}_{#9} & {#2}_{#10} \\
% {#3}_{#6} & {#3}_{#7} & {#3}_{#8} & {#3}_{#9} & {#3}_{#10} \\
% {#4}_{#6} & {#4}_{#7} & {#4}_{#8} & {#4}_{#9} & {#4}_{#10} \\
% {#5}_{#6} & {#5}_{#7} & {#5}_{#8} & {#5}_{#9} & {#5}_{#10}
%\end{vmatrix}
%}

% R3 vector.
\newcommand{\VectorThree}[3]{
\begin{bmatrix}
 {#1} \\
 {#2} \\
 {#3}
\end{bmatrix}
}


\newcommand{\grad}[0]{\nabla}
\newcommand{\PD}[2]{ \frac{\partial{#1}}{\partial {#2}} }

\title{ Derivation of Newton's Law from Lagrangian and general gradient.}
\author{Peeter Joot \quad peeter.joot@gmail.com}
\date{August 9, 2008}

\begin{document}

\maketitle{}

\section{}

In the classical limit the Lagrangian action for a point particle in a general
position dependent field is:

\begin{equation}
S = \inv{2} m\Bv^2 - \varphi
\end{equation}

Given the Lagrange equations that minimize the action, it is fairly simple
to derive the Newtonian force law.

\begin{align*}
0
&= \PD{S}{x^i} - \frac{d}{dt}\PD{S}{\dot{x}^i} \\
&= -\PD{\varphi}{x^i} - \frac{d}{dt}\left(m \dot{x}^i\right)
\end{align*}

Multiplication of this result with the unit vector $\Be_i$, and summing over
all unit vectors we have:

\begin{equation*}
\sum \Be_i \frac{d}{dt}\left(m \dot{x}^i\right) = - \sum \Be_i \PD{\varphi}{x^i}
\end{equation*}

Or, using the gradient operator, and writing $\Bv = \sum \Be_i \dot{x}^i$, we have:

\begin{equation}
\BF = \frac{d (m \Bv)}{dt} = - \grad \varphi
\end{equation}

\subsection{ The mistake hiding above. }

Now, despite the use of upper and lower pairs of indexes for the basis vectors and coordinates, this
result is not valid for a general set of basis vectors.  This initially confused the author, since the RHS
sum $\Bv = \sum \Be_i v^i$ is valid for any set of basis vectors independent of the orthonormality of that
set of basis vectors.  This is assuming that these coordinate pairs follow the usual reciprocal relationships:

\begin{equation*}
\Bx = \sum \Be_i x^i
\end{equation*}
\begin{equation*}
x^i = \Bx \cdot \Be^i
\end{equation*}
\begin{equation*}
\Be^i \cdot \Be_j = {\delta^i}_j
\end{equation*}

However, the LHS that implicitly defines the gradient as:
\begin{equation*}
\grad = \sum \Be_i \PD{}{x^i}
\end{equation*}

is a result that is only valid when the set of basis vectors $\Be_i$ is orthonormal.  The general result is
expected instead to be:

\begin{equation*}
\grad = \sum \Be^i \PD{}{x^i}
\end{equation*}

This is how the gradient is defined (without motivation) in Doran/Lasenby.  One can however demonstrate that this definition, and not $\grad = \sum \Be_i \PD{}{x^i}$, is required by doing a computation of something like $\grad \norm{\Bx}^\alpha$ with $\Bx = \sum x^i \Be_i$ for a general basis $\Be_i$ to demonstrate this.  An example of this can be found in the appendix below.

So where did things go wrong?  It was in one of the ``obvious'' skipped steps: $\Bv = \sum \dot{x^i} \dot{x^i}$.  It is in that
spot where there is a hidden orthonormal frame vector requirement since a general basis will have mixed product terms too
(ie: non-diagonal metric tensor).

Expressed in full for general frame vectors the action to minimize is the following:

\begin{equation}
S = \inv{2} m \sum \dot{x}^i \dot{x}^j \Be_i \cdot \Be_j -\varphi
\end{equation}

Or, expressed using a metric tensor $g_{ij} = \Be_i \cdot \Be_j$, this is:

\begin{equation}
S = \inv{2} m \sum \dot{x}^i \dot{x}^j g_{ij} -\varphi
\end{equation}

\subsection{ Equations of motion for vectors in a general frame. }

Now we are in shape to properly calculate the equations of motion from the Lagrangian action minimization equations.

\begin{align*}
0
&= \PD{S}{x^k} - \frac{d}{dt}\PD{S}{\dot{x}^k} \\
&= -\PD{\varphi}{x^k} - \frac{d}{dt} \left( \inv{2} m \sum g_{ij} \PD{\dot{x}^i}{\dot{x}^k} \dot{x}^j + \dot{x}^i \PD{\dot{x}^j}{\dot{x}^k} \right) \\
&= -\PD{\varphi}{x^k} - \frac{d}{dt} \left( \inv{2} m \sum g_{ij} ({\delta^i}_k \dot{x}^j + \dot{x}^i {\delta^j}_k) \right) \\
&= -\PD{\varphi}{x^k} - \frac{d}{dt} \left( \inv{2} m \sum (g_{kj} \dot{x}^j + g_{ik}\dot{x}^i) \right) \\
&= -\PD{\varphi}{x^k} - \frac{d}{dt} \left( m \sum g_{kj} \dot{x}^j \right) \\
\frac{d}{dt} \left( m \sum g_{kj} \dot{x}^j \right) &= -\PD{\varphi}{x^k} \\
\sum \Be^k \frac{d}{dt} \left( m \sum g_{kj} \dot{x}^j \right) &= - \sum \Be^k \PD{\varphi}{x^k} \\
\frac{d}{dt} \left( m \sum_j \dot{x}^j \underbrace{\sum_k \Be^k \Be_k \cdot \Be_j}_{=\Be_j} \right) &= \\
\frac{d}{dt} \left( m \sum_j \dot{x}^j \Be_j \right) &= \\
\end{align*}

The requirement for reciprocal pairs of coordinates and basis frame vectors is due to the
summation $\Bv = \sum \Be_i \dot{x}^i$, and it allows us to write all of the Lagrangian equations
in vector form for an arbitrary frame basis as:

\begin{equation}
\BF = \frac{d (m \Bv)}{dt} = - \sum \Be^k \PD{\varphi}{x^k}
\end{equation}

If we are calling this RHS a gradient relationship in an orthonormal frame, we therefore must define
the following as the gradient for the general frame:

\begin{equation}\label{eqn:gradient}
\grad = \sum \Be^k \PD{}{x^k}
\end{equation}

The Lagrange equations that minimize the action still generate equations of
motion that hold when the coordinate and basis vectors cannot be summed in this fashion.
In such a case, however, the ability to merge the generalized coordinate equations of motion into a single
vector relationship will not be possible.

\section{ Appendix.  Scratch calculations. }

\subsection{ frame vector in terms of metric tensor, and reciprocal pairs. }

\begin{align*}
e_j &= \sum a_k e^k \\
e_j \cdot e_k &= \sum a_i e^i \cdot e_k \\
e_j \cdot e_k &= a_k \\
\implies & \\
e_j &= \sum e_j \cdot e_k e^k \\
e_j &= \sum g_{jk} e^k
\end{align*}

\subsection{ Gradient calculation for an absolute vector magnitude function. }

As a verification that the gradient as defined in equation \ref{eqn:gradient} works as expected, lets do a calculation that we know the answer as computed with an
orthonormal basis.

\begin{equation*}
f(\Br) = \norm{\Br}^\alpha
\end{equation*}

\begin{align*}
\grad f(\Br) 
&= \grad \norm{\Br}^\alpha \\
&= \sum \Be^k \PD{} {x^k} {{\left(\sum x^i x^j g_{ij}\right)}^{\alpha/2}} \\
&= \frac{\alpha}{2} \sum \Be^k {{\left(\sum x^i x^j g_{ij}\right)}^{\alpha/2 -1}} \quad \PD{}{x^k} {\left(\sum x^i x^j g_{ij}\right)} \\
&= \alpha \norm{\Br}^{\alpha-2} \sum \Be^k x^i g_{ki} \\
&= \alpha \norm{\Br}^{\alpha-2} \sum_i x^i \underbrace{\sum_k \Be^k \Be_k \cdot \Be_i}_{=\Be_i} \\
&= \alpha \norm{\Br}^{\alpha-2} \Br
\end{align*}

\end{document}
  % Aug 9/08
%
% Copyright � 2012 Peeter Joot.  All Rights Reserved.
% Licenced as described in the file LICENSE under the root directory of this GIT repository.
%

%
%
\chapter{Covariant Lagrangian, and electrodynamic potential}
\index{electrodynamic potential!covariant}
\label{chap:PJSrLagrangian}
\label{chap:srLagrangian}
%\date{August 21, 2008.  srLagrangian.tex}

\section{Motivation}

In \citep{gabookII:PJSrGAFPLorentzForce}, it was observed that insertion of \(F = \grad \wedge A\) into
the covariant form of the Lorentz force:

\begin{equation}\label{eqn:srLag:lorentz}
\pdot = q (F \cdot v/c)
\end{equation}

allowed this law to be expressed as a gradient equation:

\begin{equation}
\pdot = q \grad (A \cdot v/c).
\end{equation}

Now, this suggests the possibility of a covariant potential that could be
used in a Lagrangian to produce \eqnref{eqn:srLag:lorentz} directly.  An
initial incorrect guess at what this Lagrangian would be was done, and
here some better guesses are made as well as a bit of raw algebra to verify
that it works out.

\section{Guess at the Lagrange equations for relativistic correctness}

Now, the author does not at the moment know any variational calculus worth
speaking of, but can guess at what the Lagrangian equations that would
solve the relativistic minimization problem.  Specifically, use proper
time in place of any local time derivatives:

\begin{equation}\label{eqn:srLag:properLagrange}
\frac{\partial \calL}{\partial x^{\mu}} =
\frac{d}{d\tau} \frac{\partial \calL}{\partial \xdot^{\mu}}.
\end{equation}

Note that in this equation \(\xdot^{\mu} = \frac{d x^{\mu}}{d\tau}\).

\subsection{Try it with a non-velocity dependent potential}

Lets see if this works as expected, by applying it to the simplest general
kinetic and potential Lagrangian.

\begin{equation}
\calL = \inv{2} m v^2 + \phi
\end{equation}

Calculate the Lagrangian equations:

\begin{equation}\label{eqn:srLagrangian:20}
\begin{aligned}
\frac{\partial \calL}{\partial x^{\mu}} &= \frac{d}{d\tau} \frac{\partial \calL}{\partial \xdot^{\mu}} \\
+ \frac{\partial \phi}{\partial x^{\mu}}
&= \inv{2} m \frac{d}{d\tau} \frac{\partial}{\partial \xdot^{\mu}} \sum \gamma_{\alpha} \cdot \gamma_{\beta} \xdot^{\alpha} \xdot^{\beta} \\
&= \inv{2} m \sum \gamma_{\alpha} \cdot \gamma_{\beta} \frac{d}{d\tau} \left({\delta^{\alpha}}_{\mu} \xdot^{\beta} + \xdot^{\alpha} {\delta^{\beta}}_{\mu}\right) \\
&= \inv{2} m \sum \frac{d}{d\tau}
\left(\gamma_{\mu} \cdot \gamma_{\beta} \xdot^{\beta} + \gamma_{\alpha} \cdot \gamma_{\mu} \xdot^{\alpha} \right) \\
&= m \sum \frac{d}{d\tau} \gamma_{\mu} \cdot \gamma_{\alpha} \xdot^{\alpha} \\
&= m \sum \gamma_{\mu} \cdot \gamma_{\alpha} \ddot{x}^{\alpha} \\
\end{aligned}
\end{equation}

Now, as in the Newtonian case, where we could show the correct form of the gradient for non-orthonormal frames could be derived from the Lagrangian equations using appropriate reciprocal vector pairs, we do the same thing here, summing the product of this last result with the reciprocal frame vectors:

\begin{equation}\label{eqn:srLagrangian:40}
\begin{aligned}
\sum \gamma^{\mu} \left(\frac{\partial \phi}{\partial x^{\mu}}\right) &= \sum \gamma^{\mu} \left(m \gamma_{\mu} \cdot \gamma_{\alpha} \ddot{x}^{\alpha}\right) \\
\left(\sum \gamma^{\mu} \frac{\partial}{\partial x^{\mu}}\right) \phi
&= m \sum \gamma_{\alpha} \ddot{x}^{\alpha} \\
&= m \ddot{x}
\end{aligned}
\end{equation}

Now, this left hand operator quantity is exactly our spacetime gradient:

\begin{equation}
\grad = \sum \gamma^{\mu} \frac{\partial}{\partial x^{\mu}},
\end{equation}

and the right hand side is our proper momentum.  Therefore the result of following through with the assumed Lagrangian equations yield the expected result:

\begin{equation}
\pdot = \grad \phi.
\end{equation}

Additionally, this demonstrates that the spacetime gradient used in GAFP is appropriate for any spacetime basis, regardless of whether the chosen basis vectors are orthonormal.

There are two features that are of interest here, one is that this result is independent of dimension, and the other is that there is also no requirement for any particular metric signature, Minkowski, Euclidean, or other.  That has to come another source.

\subsection{Velocity dependent potential}

The simplest scalar potential that is dependent on velocity is a potential that is composed of the dot product of a vector with that velocity.  Lets calculate the Lagrangian
equation for such an abstract potential, \(\phi(x,v) = A \cdot v\) (where any required unit adjustment to make this physically meaningful can be thought of as temporarily incorporated into \(A\)).

\begin{equation}
\calL = \inv{2}m v^2 + A \cdot v
\end{equation}

\begin{equation}\label{eqn:srLagrangian:60}
\begin{aligned}
\frac{\partial \calL}{\partial x^{\mu}} &= \frac{d}{d\tau} \frac{\partial \calL}{\partial \xdot^{\mu}} \\
\frac{\partial A}{\partial x^{\mu}} \cdot v &= m \gamma_{\mu} \cdot \gamma_{\alpha} \ddot{x}^{\alpha} +\frac{d}{d\tau} \frac{\partial A \cdot v}{\partial \xdot^{\mu}} \\
\end{aligned}
\end{equation}

To simplify matters this last term can be treated separately.  First observe that the coordinate representation of the proper velocity \(v\) follows from the worldline
position vector as follows

\begin{equation}\label{eqn:srLagrangian:80}
\begin{aligned}
x &= x^\mu \gamma_\mu \\
v &= \frac{dx}{d\tau} = \xdot^\mu \gamma_\mu.
\end{aligned}
\end{equation}

This gives

\begin{equation}\label{eqn:srLagrangian:100}
\begin{aligned}
\PD{\xdot^\mu}{A \cdot v}
&= \PD{\xdot^\mu}{} A \cdot \xdot^\nu \gamma_\nu \\
&= \left( A \PD{\xdot^\mu}{\xdot^\nu} \right) \cdot \gamma_\nu \\
&= {\delta^\mu}_\nu A \cdot \gamma_\nu \\
&= A \cdot \gamma_\mu.
\end{aligned}
\end{equation}

The taking the derivative of this conjugate momentum term we have

\begin{equation}\label{eqn:srLagrangian:120}
\begin{aligned}
\frac{d}{d\tau} \PD{\xdot^\mu}{A \cdot v}
&= \frac{d}{d\tau} A \cdot \gamma_\mu \\
&= \frac{d}{d\tau} (A^\nu \gamma_\nu) \cdot \gamma_\mu \\
&= \frac{dA^\nu}{d\tau} \gamma_\nu \cdot \gamma_\mu \\
\end{aligned}
\end{equation}

Reassembling things this is
\begin{equation}\label{eqn:srLagrangian:140}
\begin{aligned}
m \gamma_{\mu} \cdot \gamma_{\alpha} \ddot{x}^{\alpha} &=
\frac{\partial A}{\partial x^{\mu}} \cdot v -\frac{d}{d\tau} \frac{\partial A \cdot v}{\partial \xdot^{\mu}} \\
&= \frac{\partial A}{\partial x^{\mu}} \cdot v - \gamma_{\mu} \cdot \gamma_{\alpha} \xdot^{\beta} \frac{\partial A^{\alpha}}{\partial x^{\beta}} \\
&= - v^{\beta} \gamma_{\mu} \cdot \gamma_{\alpha} \frac{\partial A^{\alpha}}{\partial x^{\beta}} + \frac{\partial A^{\alpha}}{\partial x^{\mu}} {v}^{\beta} \gamma_{\alpha} \cdot \gamma_{\beta} \\
&= v^{\beta} \gamma_{\alpha} \cdot \left( -\gamma_{\mu} \frac{\partial}{\partial x^{\beta}} + \frac{\partial}{\partial x^{\mu}} \gamma_{\beta} \right) A^{\alpha} \\
\implies \\
\pdot &= v^{\beta} \gamma^{\mu} \gamma_{\alpha} \cdot \left( -\gamma_{\mu} \frac{\partial}{\partial x^{\beta}} + \gamma_{\beta} \frac{\partial}{\partial x^{\mu}} \right) A^{\alpha} \\
\end{aligned}
\end{equation}

Now, this last result has an alternation that suggests the wedge product is somehow involved, but is something slightly different.  Working (guessing) backwards, lets
see if this matches the following:

\begin{equation}\label{eqn:srLagrangian:160}
\begin{aligned}
(\grad \wedge A) \cdot v
&= \sum ( \gamma^{\mu} \wedge \gamma_{\alpha} ) \cdot \gamma_{\beta} \partial_{\mu} A^{\alpha} v^{\beta} \\
&= \sum (\gamma^{\mu} \gamma_{\alpha} \cdot \gamma_{\beta} - \gamma_{\alpha} \gamma^{\mu} \cdot \gamma_{\beta}) \partial_{\mu} A^{\alpha} v^{\beta} \\
&= \sum (\gamma^{\mu} \gamma_{\alpha} \cdot \gamma_{\beta} - \gamma_{\alpha} {\delta^{\mu}}_{\beta}) \partial_{\mu} A^{\alpha} v^{\beta} \\
&= \sum
 \gamma^{\mu} \gamma_{\alpha} \cdot \gamma_{\beta} v^{\beta} \partial_{\mu} A^{\alpha}
-\gamma_{\alpha} v^{\mu} \partial_{\mu} A^{\alpha} \\
&= \sum
 \gamma^{\mu} \gamma_{\alpha} \cdot \gamma_{\beta} v^{\beta} \partial_{\mu} A^{\alpha}
-
%\gamma_{\alpha} = \gamma^{\beta} \gamma_{\beta} \cdot \gamma_{\alpha}
\gamma^{\beta} \gamma_{\beta} \cdot \gamma_{\alpha}
 v^{\mu} \partial_{\mu} A^{\alpha} \\
&= \sum
 \gamma^{\mu} \gamma_{\alpha} \cdot \gamma_{\beta} v^{\beta} \partial_{\mu} A^{\alpha}
-
\gamma^{\mu} \gamma_{\mu} \cdot \gamma_{\alpha} v^{\beta} \partial_{\beta} A^{\alpha} \\
&= \sum v^{\beta} \gamma^{\mu} \gamma_{\alpha} \cdot ( \gamma_{\beta} \partial_{\mu} - \gamma_{\mu} \partial_{\beta} ) A^{\alpha} \\
\end{aligned}
\end{equation}

From this we can conclude that the covariant Lagrangian for the Lorentz force law has the form:

\begin{equation}\label{eqn:srLag:lorentzLagrange}
\calL = \inv{2}m v^2 + q (A \cdot v/c)
\end{equation}

where application of the proper time variant of Lagrange's equation \eqnref{eqn:srLag:properLagrange} results in the equation:

\begin{equation}
\pdot = q (\grad \wedge A) \cdot v/c = q F \cdot v/c
\end{equation}

Adding in Maxwell's equation:

\begin{equation}\label{eqn:srLag:maxwell}
\grad F = \grad \wedge A = J,
\end{equation}

we have a complete statement of pre-quantum electrodynamics and relativistic dynamics all buried in three small equations \eqnref{eqn:srLag:maxwell}, \eqnref{eqn:srLag:properLagrange}, and \eqnref{eqn:srLag:lorentzLagrange}.

Wow!  Very cool.  Now, I have also seen that Maxwell's equations can be expressed in Lagrangian form (have seen a tensor something like \(F^{\mu\nu} F_{\mu\nu}\) used to express this).  Next step is probably to work out the details of how that would fit.

Also worth noting here is the fact that no gauge invariance condition was required.  Adding that in yields the ability to solve for \(A\) directly from the wave equation \(\grad^2 A = J\).
                      % Aug 21/08
%
% Copyright � 2012 Peeter Joot.  All Rights Reserved.
% Licenced as described in the file LICENSE under the root directory of this GIT repository.
%

%
%
\chapter{Vector canonical momentum}
\index{canonical momentum}
\label{chap:PJCanMomentum}
\label{chap:canonicalMomentum}
%\date{Sept. 1, 2008.  canonicalMomentum.tex}

\citep{goldstein1951cm} defines the canonical momentum as:

\begin{equation*}
\PD{\xdot^{\mu}}{\LL}
\end{equation*}

and gives an example (Lorentz force) about how this can generalize the
concept of momentum to include contributions from velocity dependent
potentials.

Lets look at his example, but put into the more natural covariant form
with the Lorentz Lagrangian (using summation convention here)

\begin{equation*}
\LL = \inv{2} m v^2 + q A \cdot v /c =
\inv{2} m \gamma_{\alpha} \cdot \gamma_{\beta} \xdot^{\alpha} \xdot^{\beta}
+ \frac{q}{c} \gamma_{\alpha} \cdot \gamma_{\beta} A^{\alpha} \xdot^{\beta}
\end{equation*}

Calculation of the canonical momentum gives:

\begin{equation}\label{eqn:canonicalMomentum:20}
\begin{aligned}
\PD{\xdot^{\mu}}{\LL}
&=
m \gamma_{\alpha} \cdot \gamma_{\beta} {\delta^{\alpha}}_{\mu} \xdot^{\beta}
+ \frac{q}{c} \gamma_{\alpha} \cdot \gamma_{\beta} A^{\alpha} {\delta^{\beta}}_{\mu} \\
&=
m \gamma_{\mu} \cdot \gamma_{\alpha} \xdot^{\alpha}
+ \frac{q}{c} \gamma_{\alpha} \cdot \gamma_{\mu} A^{\alpha} \\
&=
\gamma_{\mu} \cdot \left(
m \gamma_{\alpha} \xdot^{\alpha}
+ \frac{q}{c} \gamma_{\alpha} A^{\alpha}
\right) \\
&= \gamma_{\mu} \cdot \left( m v + \frac{q}{c} A \right) \\
\end{aligned}
\end{equation}

So, if we are to call this modified quantity \(p = m v + q A / c\) the total general momentum for the system, then the canonical momentum conjugate to \(x^{\mu}\) is:

\begin{equation*}
\PD{\xdot^{\mu}}{\LL} = \gamma_{\mu} \cdot p.
\end{equation*}

In terms of our reciprocal frame vectors, the components of \(p\) are:

\begin{equation*}
p = \gamma_{\mu} \gamma^{\mu} \cdot p = \gamma_{\mu} p^{\mu}
\end{equation*}
\begin{equation*}
p = \gamma^{\mu} \gamma_{\mu} \cdot p = \gamma^{\mu} p_{\mu}
\end{equation*}

From this we see that the conjugate momentum gives us our vector momentum
component with respect to the reciprocal frame.  We can therefore recover
our total momentum by summing over the reciprocal frame vectors.

\begin{equation}\label{eqn:canonicalMomentum:40}
\begin{aligned}
\PD{x^{\mu}}{\LL}
&= \frac{d}{d\tau}\PD{\xdot^{\mu}}{\LL} \\
&= \frac{d}{d\tau} p_{\mu} \\
\implies \\
\sum \gamma^{\mu} \PD{x^{\mu}}{\LL} &= \sum \frac{d}{d\tau} \gamma^{\mu} p_{\mu} \\
\end{aligned}
\end{equation}

Observe that we have nothing more than our spacetime gradient on the left hand side, and a velocity
specific spacetime gradient on the right hand side.  Summarizing, this allows for writing the Euler-Lagrange equations in vector form as follows:

\begin{equation}\label{eqn:canonicalMomentum:60}
\begin{aligned}
\frac{d p}{d\tau} &= \grad \LL \\
p &= \grad_v \LL \\
\grad &= \gamma^{\mu} \PD{x^{\mu}}{} \\
\grad_v &= \gamma^{\mu} \PD{\xdot^{\mu}}{}
\end{aligned}
\end{equation}

Now, perhaps this is a step backwards, since the Lagrangian formulation allows for not having to use vector representations explicitly, nor to be constrained to specific parameterizations such as this constant frame vector representation.  However, it is nice to see things in a form that is closer to what one is used to, and this is not too different seeming than the familiar spatial Newtonian formulation:

\begin{equation*}
\frac{d\Bp}{dt} = -\spacegrad \phi
\end{equation*}

%(as seen previously, the negation is due to the Minkowski metric, and selection of the spatial components of the gradient).
                 % Sep 1/08
\documentclass{article}

\usepackage{amsmath}
\usepackage{mathpazo}

%
% shorthand for bold symbols, convenient for vectors and matrices
%
\newcommand{\Ba}[0]{\mathbf{a}}
\newcommand{\Bb}[0]{\mathbf{b}}
\newcommand{\Bc}[0]{\mathbf{c}}
\newcommand{\Bd}[0]{\mathbf{d}}
\newcommand{\Be}[0]{\mathbf{e}}
\newcommand{\Bf}[0]{\mathbf{f}}
\newcommand{\Bg}[0]{\mathbf{g}}
\newcommand{\Bh}[0]{\mathbf{h}}
\newcommand{\Bi}[0]{\mathbf{i}}
\newcommand{\Bj}[0]{\mathbf{j}}
\newcommand{\Bk}[0]{\mathbf{k}}
\newcommand{\Bl}[0]{\mathbf{l}}
\newcommand{\Bm}[0]{\mathbf{m}}
\newcommand{\Bn}[0]{\mathbf{n}}
\newcommand{\Bo}[0]{\mathbf{o}}
\newcommand{\Bp}[0]{\mathbf{p}}
\newcommand{\Bq}[0]{\mathbf{q}}
\newcommand{\Br}[0]{\mathbf{r}}
\newcommand{\Bs}[0]{\mathbf{s}}
\newcommand{\Bt}[0]{\mathbf{t}}
\newcommand{\Bu}[0]{\mathbf{u}}
\newcommand{\Bv}[0]{\mathbf{v}}
\newcommand{\Bw}[0]{\mathbf{w}}
\newcommand{\Bx}[0]{\mathbf{x}}
\newcommand{\By}[0]{\mathbf{y}}
\newcommand{\Bz}[0]{\mathbf{z}}
\newcommand{\BA}[0]{\mathbf{A}}
\newcommand{\BB}[0]{\mathbf{B}}
\newcommand{\BC}[0]{\mathbf{C}}
\newcommand{\BD}[0]{\mathbf{D}}
\newcommand{\BE}[0]{\mathbf{E}}
\newcommand{\BF}[0]{\mathbf{F}}
\newcommand{\BG}[0]{\mathbf{G}}
\newcommand{\BH}[0]{\mathbf{H}}
\newcommand{\BI}[0]{\mathbf{I}}
\newcommand{\BJ}[0]{\mathbf{J}}
\newcommand{\BK}[0]{\mathbf{K}}
\newcommand{\BL}[0]{\mathbf{L}}
\newcommand{\BM}[0]{\mathbf{M}}
\newcommand{\BN}[0]{\mathbf{N}}
\newcommand{\BO}[0]{\mathbf{O}}
\newcommand{\BP}[0]{\mathbf{P}}
\newcommand{\BQ}[0]{\mathbf{Q}}
\newcommand{\BR}[0]{\mathbf{R}}
\newcommand{\BS}[0]{\mathbf{S}}
\newcommand{\BT}[0]{\mathbf{T}}
\newcommand{\BU}[0]{\mathbf{U}}
\newcommand{\BV}[0]{\mathbf{V}}
\newcommand{\BW}[0]{\mathbf{W}}
\newcommand{\BX}[0]{\mathbf{X}}
\newcommand{\BY}[0]{\mathbf{Y}}
\newcommand{\BZ}[0]{\mathbf{Z}}

\newcommand{\Bzero}[0]{\mathbf{0}}
\newcommand{\Btheta}[0]{\boldsymbol{\theta}}
\newcommand{\Btau}[0]{\boldsymbol{\tau}}
\newcommand{\Bomega}[0]{\boldsymbol{\omega}}

%
% shorthand for unit vectors
%
\newcommand{\acap}[0]{\hat{\Ba}}
\newcommand{\bcap}[0]{\hat{\Bb}}
\newcommand{\ccap}[0]{\hat{\Bc}}
\newcommand{\dcap}[0]{\hat{\Bd}}
\newcommand{\ecap}[0]{\hat{\Be}}
\newcommand{\fcap}[0]{\hat{\Bf}}
\newcommand{\gcap}[0]{\hat{\Bg}}
\newcommand{\hcap}[0]{\hat{\Bh}}
\newcommand{\icap}[0]{\hat{\Bi}}
\newcommand{\jcap}[0]{\hat{\Bj}}
\newcommand{\kcap}[0]{\hat{\Bk}}
\newcommand{\lcap}[0]{\hat{\Bl}}
\newcommand{\mcap}[0]{\hat{\Bm}}
\newcommand{\ncap}[0]{\hat{\Bn}}
\newcommand{\ocap}[0]{\hat{\Bo}}
\newcommand{\pcap}[0]{\hat{\Bp}}
\newcommand{\qcap}[0]{\hat{\Bq}}
\newcommand{\rcap}[0]{\hat{\Br}}
\newcommand{\scap}[0]{\hat{\Bs}}
\newcommand{\tcap}[0]{\hat{\Bt}}
\newcommand{\ucap}[0]{\hat{\Bu}}
\newcommand{\vcap}[0]{\hat{\Bv}}
\newcommand{\wcap}[0]{\hat{\Bw}}
\newcommand{\xcap}[0]{\hat{\Bx}}
\newcommand{\ycap}[0]{\hat{\By}}
\newcommand{\zcap}[0]{\hat{\Bz}}
\newcommand{\thetacap}[0]{\hat{\Btheta}}

%
% to write R^n and C^n in a distinguishable fashion.  Perhaps change this
% to the double lined characters upon figuring out how to do so.
%
\newcommand{\C}[1]{$\mathbb{C}^{#1}$}
\newcommand{\R}[1]{$\mathbb{R}^{#1}$}

%
% various generally useful helpers
%

% derivative of #1 wrt. #2:
\newcommand{\D}[2] {\frac {d#2} {d#1}}

\newcommand{\inv}[1]{\frac{1}{#1}}
\newcommand{\cross}[0]{\times}

\newcommand{\abs}[1]{\lvert{#1}\rvert}
\newcommand{\norm}[1]{\lVert{#1}\rVert}
\newcommand{\innerprod}[2]{\langle{#1}, {#2}\rangle}
\newcommand{\dotprod}[2]{{#1} \cdot {#2}}
\newcommand{\bdotprod}[2]{\left({#1} \cdot {#2}\right)}
\newcommand{\crossprod}[2]{{#1} \cross {#2}}
\newcommand{\tripleprod}[3]{\dotprod{\left(\crossprod{#1}{#2}\right)}{#3}}

\DeclareMathOperator{\Proj}{Proj}
\DeclareMathOperator{\Span}{span}
\DeclareMathOperator{\Sgn}{sgn}
\DeclareMathOperator{\Area}{Area}
\DeclareMathOperator{\Volume}{Volume}

%
% A few miscellaneous things specific to this document
%
\newcommand{\crossop}[1]{\crossprod{#1}{}}

% R2 vector.
\newcommand{\VectorTwo}[2]{
\begin{bmatrix}
 {#1} \\
 {#2}
\end{bmatrix}
}

\newcommand{\VectorN}[1]{
\begin{bmatrix}
{#1}_1 \\
{#1}_2 \\
\vdots \\
{#1}_N \\
\end{bmatrix}
}

\newcommand{\DETuvij}[4]{
\begin{vmatrix}
 {#1}_{#3} & {#1}_{#4} \\
 {#2}_{#3} & {#2}_{#4}
\end{vmatrix}
}

\newcommand{\DETuvwijk}[6]{
\begin{vmatrix}
 {#1}_{#4} & {#1}_{#5} & {#1}_{#6} \\
 {#2}_{#4} & {#2}_{#5} & {#2}_{#6} \\
 {#3}_{#4} & {#3}_{#5} & {#3}_{#6}
\end{vmatrix}
}

\newcommand{\DETuvwxijkl}[8]{
\begin{vmatrix}
 {#1}_{#5} & {#1}_{#6} & {#1}_{#7} & {#1}_{#8} \\
 {#2}_{#5} & {#2}_{#6} & {#2}_{#7} & {#2}_{#8} \\
 {#3}_{#5} & {#3}_{#6} & {#3}_{#7} & {#3}_{#8} \\
 {#4}_{#5} & {#4}_{#6} & {#4}_{#7} & {#4}_{#8} \\
\end{vmatrix}
}

%\newcommand{\DETuvwxyijklm}[10]{
%\begin{vmatrix}
% {#1}_{#6} & {#1}_{#7} & {#1}_{#8} & {#1}_{#9} & {#1}_{#10} \\
% {#2}_{#6} & {#2}_{#7} & {#2}_{#8} & {#2}_{#9} & {#2}_{#10} \\
% {#3}_{#6} & {#3}_{#7} & {#3}_{#8} & {#3}_{#9} & {#3}_{#10} \\
% {#4}_{#6} & {#4}_{#7} & {#4}_{#8} & {#4}_{#9} & {#4}_{#10} \\
% {#5}_{#6} & {#5}_{#7} & {#5}_{#8} & {#5}_{#9} & {#5}_{#10}
%\end{vmatrix}
%}

% R3 vector.
\newcommand{\VectorThree}[3]{
\begin{bmatrix}
 {#1} \\
 {#2} \\
 {#3}
\end{bmatrix}
}


\newcommand{\LL}[0]{\mathcal{L}}
\newcommand{\gpgrade}[2] {{\left\langle{{#1}}\right\rangle}_{#2}}
\newcommand{\gpgradezero}[1] {\gpgrade{#1}{0}}
\newcommand{\gpgradetwo}[1] {\gpgrade{#1}{2}}
\newcommand{\gpgradefour}[1] {\gpgrade{#1}{4}}
\newcommand{\grad}[0]{\nabla}
\newcommand{\spacegrad}[0]{\boldsymbol{\nabla}}
\newcommand{\PD}[2]{\frac{\partial {#2}}{\partial {#1}}}
\newcommand{\PDd}[2]{\frac{\partial^2 {#2}}{{\partial{#1}}^2}}
%\newcommand{\PDD}[3]{\frac{\partial^2 {#3}}{\partial {#1}\partial {#2}}}

\newcommand{\barA}[0]{\bar{A}}

\usepackage[
bookmarks=true
%,pdffitwindow
%,pdfcenterwindow
]{hyperref}

\title{ A straightforward variational approach to derive the Maxwell field equation from the electrodynamic Lagrangian density. }
\author{Peeter Joot}
\date{ Sept 8, 2008.  Last Revision: $Date: 2008/09/14 16:44:15 $ }

\begin{document}

\tableofcontents

\maketitle{}

\section{ Motivation, definitions and setup. }

This document will attempt to calculate Maxwells equation, which in multivector form is

\begin{equation}\label{eqn:maxwell}
\grad F = J/\epsilon_0 c
\end{equation}

using a Lagrangian energy density variational approach.

\subsection{ Notation and definitions. }

For standalone purposes, here is a summary of the notation and definitions that will be used.  Greek letters range over all indexes and
english indexes range over $1,2,3$.  Bold vectors are spatial enties and non-bold is used for four vectors.

\begin{equation*}
\begin{array}{l l l}
\gamma_{\mu} & & \quad \mbox{Four vector basis vector)} \\
& & \quad \mbox{($\gamma_{\mu} \cdot \gamma_{\nu} = \pm {\delta^{\mu}}_{\nu}$)} \\
{(\gamma_0)}^2 {(\gamma_i)}^2 &= -1 & \quad \mbox{Minkowski metric} \\
\sigma_i = \sigma^i &= \gamma_{i} \wedge \gamma_0 & \quad \mbox{Spatial basis bivector. ($\sigma_i \cdot \sigma_j = \delta_{ij}$)} \\
                    &= \gamma_{i0} \\
I &= \gamma_{0} \wedge \gamma_1 \wedge \gamma_{2} \wedge \gamma_3 & \quad \mbox{Four-vector pseudoscalar} \\
  &= \gamma_{0123} \\
\gamma^{\mu} \cdot \gamma_{\nu} &= {\delta^{\mu}}_{\nu} & \quad \mbox{Reciprocal basis vectors} \\
x^{\mu} &= x \cdot \gamma^{\mu} & \quad \mbox{Vector coordinate} \\
x_{\mu} &= x \cdot \gamma_{\mu} & \quad \mbox{Coordinate for reciprocal basis} \\
x &= \sum \gamma_{\mu} x^{\mu} & \quad \mbox{Four vector in terms of coordinates} \\
  &= \sum \gamma^{\mu} x_{\mu} \\
\BE &= \sum E^i \sigma_i & \quad \mbox{Electric field spatial vector} \\
\BB &= \sum B^i \sigma_i & \quad \mbox{Magnetic field spatial vector} \\
J &= \sum \gamma_{\mu} J^{\mu} & \quad \mbox{Current density four vector.} \\
  &= \sum \gamma^{\mu} J_{\mu} \\
F &= \BE + I c \BB & \quad \mbox{Electromagnetic bivector} \\
x^{0} &= x \cdot \gamma^0 & \quad \mbox{Time coordinate (length dim.)} \\
      &= c t \\
\Bx &= x \wedge \gamma_0 & \quad \mbox{Spatial vector} \\
    &= x^i \sigma_i \\
J^{0} &= J \cdot \gamma^0 & \quad \mbox{Charge density.} \\
      &= c \rho & \quad \mbox{(current density dimensions.)} \\
\BJ &= J \wedge \gamma_0 & \quad \mbox{Current density spatial vector} \\
    &= \sum J^i \sigma_i \\
\grad &= \sum \gamma^{\mu} \partial/\partial {x^{\mu}} & \quad \mbox{Spacetime gradient} \\
      &= \sum \gamma^{\mu}\partial_{\mu} \\
      &= \sum \gamma_{\mu} \partial/\partial {x_{\mu}} \\
      &= \sum \gamma_{\mu}\partial^{\mu} \\
\spacegrad &= \sum \sigma^{i} \partial/\partial{x^{i}} & \quad \mbox{Spatial gradient} \\
           &= \sum \sigma^{i}\partial_{i} \\
\end{array}
\end{equation*}

Summation convention, where summation over all sets of matched upper and lower indexes is implied, will be in effect from this point on.

\subsection{ Tensor form of the field. }

Explicit expansion of the field bivector in terms of coordinates one has

\begin{align*}
F
&= \BE + I c \BB \\
&= E^k \gamma_{k0} + \gamma_{0123k0} c B^k \\
&= E^k \gamma_{k0} + {(\gamma_{0})}^2 {(\gamma_{k})}^2 {\epsilon^{ij}}_k c \gamma_{ij} B^k \\
\end{align*}

Or,
\begin{equation}
F = E^k \gamma_{k0} - c {\epsilon^{ij}}_k B^k \gamma_{ij}
\end{equation}

When this bivector is expressed in terms of basis bivectors $\gamma_{\mu\nu}$ we have

\begin{align*}
F
= \sum_{\mu<\nu} (F \cdot \gamma^{\nu\mu}) \gamma_{\mu\nu}
= \inv{2} (F \cdot \gamma^{\nu\mu}) \gamma_{\mu\nu}
\end{align*}

As shorthand for the coordinates the field can be expressed with respect to various bivector basis sets in tensor form

\begin{equation*}
\begin{array}{l l l}
F^{\mu\nu} &= F \cdot \gamma^{\nu\mu} & \quad F = (1/2) F^{\mu\nu} \gamma_{\mu\nu} \\
F_{\mu\nu} &= F \cdot \gamma_{\nu\mu} & \quad F = (1/2) F_{\mu\nu} \gamma^{\mu\nu} \\
{F_{\mu}}^\nu &= F \cdot {\gamma_{\nu}}^{\mu} & \quad F = (1/2) {F_{\mu}}^{\nu} {\gamma^{\mu}}_{\nu} \\
{F^{\mu}}_\nu &= F \cdot {\gamma^{\nu}}_{\mu} & \quad F = (1/2) {F^{\mu}}_{\nu} {\gamma_{\mu}}^{\nu}
\end{array}
\end{equation*}

In particular, we can extract the electric field components by dotting with a spacetime mix of indexes

\begin{equation*}
F^{i0} = E^k \gamma_{k0} \cdot \gamma^{0i} = E^i = -F_{i0}
\end{equation*}

and the magnetic field components by dotting with the bivectors having a pure spatial mix of indexes

\begin{equation*}
F^{ij} = - c {\epsilon^{a b}}_k B^k \gamma_{a b} \cdot \gamma^{ji} = - c {\epsilon^{i j}}_k B^k = F_{ij}
\end{equation*}

It is customary to summarize these tensors in matrix form
\begin{equation}\label{eqn:matrixtensor}
F^{\mu\nu} =
\begin{bmatrix}
0   & -E^1 & -E^2 & -E^3 \\
E^1 &   0  & -c B^3 &  c B^2 \\
E^2 &  c B^3 &   0  & -c B^1 \\
E^3 & -c B^2 &  c B^1 &   0  \\
\end{bmatrix}
\end{equation}

\begin{equation}
F_{\mu\nu} =
\begin{bmatrix}
0   & E^1 & E^2 & E^3 \\
-E^1 &   0  & -c B^3 &  c B^2 \\
-E^2 &  c B^3 &   0  & -c B^1 \\
-E^3 & -c B^2 &  c B^1 &   0  \\
\end{bmatrix}.
\end{equation}

Neither of these matrixes will be needed explictly, but are included for comparison since there is some variation in the sign conventions and units used
for the field tensor.

\subsection{ Maxwells equation in tensor form. }

Taking vector and trivector parts of Maxwells equation \ref{eqn:maxwell}, and writing in terms of coordinates produces two equations respectively

\begin{equation}
\partial_{\mu} F^{\mu\alpha} = J^{\alpha}/c \epsilon_0
\end{equation}

\begin{equation}\label{eqn:dualpartofMaxwells}
\epsilon^{ \alpha \beta \sigma \mu } \partial_{\alpha} F_{\beta\sigma} = 0.
\end{equation}

The aim here to show that these can be derived from an appropriate Lagrangian density.

\subsubsection{ Potential form. }

With the assumption that the field can be expressed in terms of the curl of a potential vector

\begin{equation}\label{eqn:potentialdef}
F = \grad \wedge A
\end{equation}

the tensor expression of the field becomes

\begin{align*}\label{eqn:tensorpot}
F^{\mu\nu} &= F \cdot (\gamma^{\nu} \wedge \gamma^{\mu}) = \partial^{\mu} A^{\nu} - \partial^{\nu} A^{\mu} \\
F_{\mu\nu} &= F \cdot (\gamma_{\nu} \wedge \gamma_{\mu}) = \partial_{\mu} A_{\nu} - \partial_{\nu} A_{\mu} \\
{F^{\mu}}_{\nu} &= F \cdot (\gamma^{\nu} \wedge \gamma_{\mu}) = \partial^{\mu} A_{\nu} - \partial_{\nu} A^{\mu} \\
{F_{\mu}}^{\nu} &= F \cdot (\gamma_{\nu} \wedge \gamma^{\mu}) = \partial_{\mu} A^{\nu} - \partial^{\nu} A_{\mu}
\end{align*}

These field bivector coordinates will be used in the Lagrangian calculations.

\subsection{ Field square. }

Our Lagrangian will be formed from the scalar part (will the pseudoscalar part of the field also play a part?) of the squared bivector

\begin{align*}
F^2
&= (\BE + I c \BB) (\BE + I c \BB) \\
&= \BE^2 - c^2 \BB^2 + c \left( I \BB \BE + \BE I \BB \right) \\
&= \BE^2 - c^2 \BB^2 + c I \left( \BB \BE + \BE \BB \right) \\
&= \BE^2 - c^2 \BB^2 + 2 c I \BE \cdot \BB
\end{align*}

\subsubsection{ Scalar part. }

One can also show that the following are all identical representations.

\begin{equation}
\inv{2} F_{\mu\nu}F^{\mu\nu} = -\gpgradezero{F^2} = c^2 \BB^2 -\BE^2
\end{equation}

In particular, we will use the tensor form with the field defined in terms of the vector potential of equation \ref{eqn:potentialdef}.

Expanding in coordinates this squared curl we have

\begin{align*}
(\grad \wedge A) (\grad \wedge A)
&= (\gamma^{\mu\nu} \partial_{\mu} A_{\nu})(\gamma^{\alpha\beta} \partial_{\alpha} A_{\beta}) \\
\end{align*}

Implied here is that $\mu \ne \nu$ and $\alpha \ne \beta$.  Given that expansion of the scalar and pseudoscalar parts of this quantity we have

\begin{align*}
\gpgradezero{ (\grad \wedge A)^2 }
&= (\gamma^{\mu\nu}) \cdot (\gamma_{\alpha\beta}) \partial_{\mu} A_{\nu} \partial^{\alpha} A^{\beta} \\
&= \left(\delta^{\mu}_{\beta} \delta^{\nu}_{\alpha} - \delta^{\nu}_{\beta} \delta^{\mu}_{\alpha}\right) \partial_{\mu} A_{\nu} \partial^{\alpha} A^{\beta} \\
&= \partial_{\mu} A_{\nu} \partial^{\nu} A^{\mu} -\partial_{\mu} A_{\nu} \partial^{\mu} A^{\nu} \\
&= -\partial_{\mu} A_{\nu} \left( \partial^{\mu} A^{\nu} - \partial^{\nu} A^{\mu} \right) \\
\end{align*}

That is
\begin{equation}
\gpgradezero{ (\grad \wedge A)^2 } = -\partial_{\mu} A_{\nu} F^{\mu\nu} = -\inv{2} F_{\mu\nu}F^{\mu\nu}.
\end{equation}

For the pseudoscalar parts of the product we have
\begin{align*}
\gpgrade{ (\grad \wedge A)^2 }{4} &= (\grad \wedge A) \wedge (\grad \wedge A) \\
&= (\gamma_{\mu\nu}) \wedge (\gamma_{\alpha\beta}) \partial^{\mu} A^{\nu} \partial^{\alpha} A^{\beta} \\
\end{align*}

That is

\begin{equation}\label{eqn:pseudoscalarFieldSquare}
\gpgrade{ (\grad \wedge A)^2 }{4} = \epsilon_{\mu\nu\alpha\beta} I \partial^{\mu} A^{\nu} \partial^{\alpha} A^{\beta}
\end{equation}

FIXME: how about the commutator part of the product.  Does that also yield something when considered as a Lagrangian.

We will work first with the Lagrangian field density in the following form
\begin{align}\label{eqn:density}
\LL &= -\frac{\kappa}{4} \gpgradezero{F^2} + J \cdot A \\
%&= \frac{\kappa}{4} F_{\mu\nu} F^{\mu\nu} + J_{\alpha} A^{\alpha} \\
%&= \frac{\kappa}{4} ( \partial_{\mu} A_{\nu} - \partial_{\nu} A_{\mu} ) ( \partial^{\mu} A^{\nu} - \partial^{\nu} A^{\mu} ) + J_{\alpha} A^{\alpha}
&= \frac{\kappa}{2} \partial_{\mu} A_{\nu} ( \partial^{\mu} A^{\nu} - \partial^{\nu} A^{\mu} ) + J_{\alpha} A^{\alpha}
\end{align}

%It will be convient to write the density term in a slightly different fashion
%
%\begin{align*}
%F^{\mu\nu} F_{\mu\nu}
%&=
%\partial_{\mu} A_{\nu} \partial^{\mu} A^{\nu}
%-\partial_{\mu} A_{\nu} \partial^{\nu} A^{\mu}
%-\partial_{\nu} A_{\mu} \partial^{\mu} A^{\nu}
%+\partial_{\nu} A_{\mu} \partial^{\nu} A^{\mu} \\
%&= 2 \left( \partial_{\mu} A_{\nu} \partial^{\mu} A^{\nu} -\partial_{\mu} A_{\nu} \partial^{\nu} A^{\mu} \right) \\
%&= 2 \partial_{\mu} A_{\nu} \left( \partial^{\mu} A^{\nu} -\partial^{\nu} A^{\mu} \right) \\
%\end{align*}

%So we want to evaluate the equations for:
%\begin{equation}
%\end{equation}

\subsubsection{ Pseudoscalar part. }

FIXME: write the $\BE \cdot \BB$ term in tensor notation and see if the second half of Maxwells equation follows from that.

\section{ Variational background. }

Trying to blindly plug into the proper time variation of the Euler-Lagrange equations that can be used to derive the Lorentz force law from a $A \cdot v$ based Lagrangian

\begin{equation}\label{eqn:eulerlag}
\PD{x^\mu}{\LL} = \frac{d}{d\tau} \PD{\dot{x}^\mu}{\LL}
\end{equation}

did not really come close to producing Maxwell's equations from the Lagrangian in equation \ref{eqn:density}.  Whatever the equivalent of the Euler-Lagrange equations is for an energy density Lagrangian they aren't what is in equation \ref{eqn:eulerlag}.

However, what did work was Feynman's way from the second volume of the Lectures (the ``entertainment'' chapter on Principle of Least Action).
which uses some slightly ad-hoc seeming variational techniques directly.  To demonstrate the technique some simple examples
will be calculated to get the feel for the method.  After this we move on to the more complex case of trying with the electrodynamic Lagrange
density of equation \ref{eqn:density}.

\subsection{ One dimensional purely kinentic Lagrangian. }

Here is pretty much the simplest case, and illustrates the technique well.

Suppose we have an action associated with a kinetic Lagrangian density $(1/2) m v^2$

\begin{equation}\label{eqn:oneDimKinetic}
S = \int_a^b \frac{m}{2} { \left(\frac{dx}{dt}\right) }^2 dt
\end{equation}

where $x = x(t)$ is the undetermined function to solve for.  Feynman's technique is similar to Goldstein's way of deriving the Euler Lagrange equations, but instead of writing

\begin{equation*}
x(t, \epsilon) = x(t, 0) + \epsilon n(t)
\end{equation*}

and taking derivatives under the integral sign with respect to $\epsilon$, instead he just writes

\begin{equation}\label{eqn:xbarplusn}
x = \bar{x} + n
\end{equation}

In either case, the function $n = n(t)$ is zero at the boundaries of the integration region, and is allowed to take any value in between.

Substitution of \ref{eqn:xbarplusn} into \ref{eqn:oneDimKinetic} we have

\begin{equation}
S =
\int_a^b \frac{m}{2} { \left(\frac{d\bar{x}}{dt}\right) }^2 dt
+ 2 \int_a^b \frac{m}{2} \frac{d\bar{x}}{dt} \frac{d n}{dt} dt
+ \int_a^b \frac{m}{2} { \left(\frac{d n}{dt}\right) }^2 dt
\end{equation}

The last term being quadratic and presumed small is just dropped.  The first term is strictly positive and doesn't vary with $n$ in any way.  The middle term, just as in
Goldstein is integrated by parts

\begin{align*}
\int_a^b \underbrace{ \frac{d\bar{x}}{dt} \frac{d n}{dt} }_{fg'} dt
&= \underbrace{ \left. \frac{d\bar{x}}{dt} n \right\vert_a^b }_{fg} - \int_a^b \underbrace{\frac{d^2\bar{x}}{dt^2} n}_{f'g} dt.
\end{align*}

Since $n(a) = n(b) = 0$ the first term is zero.  For the remainder to be independent of path (ie: independent of $n$) the $\bar{x}''$ term is set to zero.  That is

\begin{equation*}
\frac{d^2\bar{x}}{dt^2} = 0.
\end{equation*}

As the solution to the extreme value problem.  This is nothing but the equation for a straight line, which is what we expect if there are no external forces

\begin{equation*}
\bar{x} - x_0 = v(t - t_0).
\end{equation*}

\subsection{ Electrostatic potential Lagrangian. }

Next is to apply the same idea to the field Lagrangian for electrostatics.  The Lagrangian is assumed to be of the following form

\begin{equation*}
\LL = \kappa (\spacegrad \phi)^2 + \rho \phi
\end{equation*}

Let's see if we can recover the electrostatics equation from this with an action of

\begin{equation}\label{eqn:densityaction}
S = \int_{\Omega} \LL dx dy
\end{equation}

Doing this for the simpler case of one dimension wouldn't be too much different from the previous kinetic calculation, and doing this in two dimensions is enough
to see how to apply this to the four dimensional case for the general electrodynamic case.

As above we assume that the general varied potential be written in terms of unknown function for which the action takes its extreme value, plus any other unspecified
function

\begin{equation}\label{eqn:barphi}
\phi = \bar{\phi} + n
\end{equation}

The function $n$ is required to be zero on the boundary of the area $\Omega$.  One can likely assume any sort of area, but for this calculation
the area will be assumed to be both type I and type II (in the lingo of Salus and Hille).

FIXME: picture here to explain.  Want to describe an open area like a ellipse, or rectangle where bounding functions on the top/bottom, or left/right and a fixed interval
in the other direction.

% for the area here I think we want both type I and type II (elipse, square, ...).  Can probably generalize to
% other topological forms like donuts but will require that the variational function vanish on _any_ boundary.

Substituting the assumed form of the solution from equation \ref{eqn:barphi} into the action integral \ref{eqn:densityaction} one has

\begin{equation*}
S =
    \int_{\Omega} dA \left( \kappa {(\spacegrad \bar{\phi})}^2 + \rho \bar{\phi} \right)
+   \int_{\Omega} dA \left( 2 \kappa (\spacegrad \bar{\phi}) \cdot \spacegrad n + \rho n \right)
+   \kappa \int_{\Omega} dA {(\spacegrad n)}^2
\end{equation*}

Again the idea here is to neglect the last integral, ignore the first integral which is fixed, and use integration by parts to eliminate derivatives of $n$
in the middle integral.  The portion of that integral to focus on is

\begin{equation*}
2 \kappa \sum \int_{\Omega} dx_1 dx_2 \PD{x_i}{\bar{\phi}} \PD{x_i}{n},
\end{equation*}

but how do we do integration by parts on such a beast?  We have partial derivatives and multiple integration to deal with.  Consider just one part of this sum, also ignoring the scale factor, and write it as a definite integral

\begin{align*}
\int_{\Omega} dx_1 dx_2 \PD{x_1}{\bar{\phi}} \PD{x_1}{n}
&= \int_{x_1=a}^{x_1=b} dx_1 \int_{x_2 = \theta_1(x_1)}^{x_2 = \theta_2(x_1)} dx_2 \PD{x_2}{\bar{\phi}(x_1, x_2)} \PD{x_2}{n(x_1, x_2)} \\
\end{align*}

In the inner integral $x_1$ can be considered constant, and one can consider $n(x_1, x_2)$ to be a set function of just $x_2$, say

\begin{equation*}
m_{x_1}(x_2) = n(x_1, x_2)
\end{equation*}

Then $dm/dx_2$ is our partial of $n$

\begin{equation*}
\frac{d m_{x_1}(x_2)}{d x_2} = \PD{x_2}{n(x_1, x_2)}
\end{equation*}

and we can apply integration by parts
\begin{align*}
\int_{\Omega} dx_1 dx_2 \PD{x_1}{\bar{\phi}} \PD{x_1}{n}
&= \int_{x_1=a}^{x_1=b} dx_1
%\int_{x_2 = \theta_1(x_1)}^{x_2 = \theta_2(x_1)} dx_2 \PD{x_2}{\bar{\phi}(x_1, x_2)} \PD{x_2}{n(x_1, x_2)}
% f
%\PD{x_2}{\bar{\phi}(x_1, x_2)}
% g'
%\PD{x_2}{n(x_1, x_2)}
% fg:
\left. \PD{x_2}{\bar{\phi}(x_1, x_2)} n(x_1, x_2) \right\vert_{x_2 = \theta_1(x_1)}^{x_2 = \theta_2(x_1)} \\
&- \int_{x_1=a}^{x_1=b} dx_1 \int_{x_2 = \theta_1(x_1)}^{x_2 = \theta_2(x_1)} dx_2 \frac{d}{dx_2} \PD{x_2}{\bar{\phi}(x_1, x_2)} n(x_1, x_2) \\
\end{align*}

In the remaining single integral we have
$n(x_1, \theta_1(x_1))$, and $n(x_1, \theta_2(x_1))$ but these are both points on the boundary, so by the definition of $n$ these are zero (Feynman takes the
region as all space and has $n=0$ at infinity).

In the remaining term, the derivative $\frac{d}{d x_2} \PD{x_2}{\bar{\phi}(x_1, x_2)}$ is taken with $x_1$ fixed so is just a second partial.
Doing in the same integration by parts for the other part of the sum and reassembling results we have

\begin{align*}
S = \int_{\Omega} dA \left( \kappa {(\spacegrad \bar{\phi})}^2 + \rho \bar{\phi} \right)
  + \int_{\Omega} dA \left( -2 \kappa \PDd{x_1}{\bar{\phi}} -2 \kappa \PDd{x_2}{\bar{\phi}} + \rho \right) n
  + \kappa \int_{\Omega} dA {(\spacegrad n)}^2
\end{align*}

As before we set this inner term to zero so that it holds for any $n$, and recover the field equation as

\begin{equation*}
\spacegrad^2 \bar{\phi} = \rho/2 \kappa.
\end{equation*}

provided we set the constant $\kappa = -\epsilon_0/2$.  This also fixes the unknown constant in the associated Lagrangian density and action

\begin{equation}
S = \int_{\Omega} \left(- \frac{\epsilon_0}{2} (\spacegrad \phi)^2 + \rho \phi \right) d\Omega.
\end{equation}

It is also clear that the arguments above would also hold for the three dimensional case $\phi = \phi(x, y, z)$.

\section{ General electrodynamic Lagrangian. }

We want to do the same for the general electrodynamic Lagrangian density (where $\kappa$ is still undetermined)

\begin{equation}
\LL = \frac{\kappa}{2} \partial_{\mu} A_{\nu} ( \partial^{\mu} A^{\nu} - \partial^{\nu} A^{\mu} ) + J_{\alpha} A^{\alpha}
\end{equation}


Using the same trick we introduce the desired solution and an variational function for each $A^{\mu}$

\begin{equation*}
A^{\mu} = \barA^{\mu} + n^{\mu}
\end{equation*}

\begin{align*}
\LL 
&= \frac{\kappa}{2} \partial_{\mu} (\barA_{\nu} + n_{\nu}) ( \partial^{\mu} (\barA^{\nu} + n^{\nu}) - \partial^{\nu} (\barA^{\mu} + n^{\mu}) ) + J_{\alpha} (\barA^{\alpha} + n^{\alpha}) \\
&= \frac{\kappa}{2} \partial_{\mu} \barA_{\nu} ( \partial^{\mu} \barA^{\nu} - \partial^{\nu} \barA^{\mu} ) + \frac{\kappa}{2} \partial_{\mu} n_{\nu} ( \partial^{\mu} n^{\nu} - \partial^{\nu} n^{\mu} )  \\
&+ \frac{\kappa}{2} \partial_{\mu} \barA_{\nu} ( \partial^{\mu} n^{\nu} - \partial^{\nu} n^{\mu} ) + \frac{\kappa}{2} \partial_{\mu} n_{\nu} ( \partial^{\mu} \barA^{\nu} - \partial^{\nu} \barA^{\mu} )  \\
&+ J_{\alpha} (\barA^{\alpha} + n^{\alpha}) \\
\end{align*}

The idea again is the same.  Treat the first term as fixed (it's the solution that takes the extreme value), neglect the quadratic term that follows, and use integration by parts
to remove any remaining $n^{\mu}$ derivatives.  Those derivative terms multiplied out are

\begin{align*}
& \frac{\kappa}{2} \partial_{\mu} \barA_{\nu} \partial^{\mu} n^{\nu}
- \frac{\kappa}{2} \partial_{\mu} \barA_{\nu} \partial^{\nu} n^{\mu}
+ \frac{\kappa}{2} \partial_{\mu} n_{\nu} \partial^{\mu} \barA^{\nu}
- \frac{\kappa}{2} \partial_{\mu} n_{\nu} \partial^{\nu} \barA^{\mu} \\
&= \frac{\kappa}{2} \partial^{\mu} \barA_{\nu} \partial_{\mu} n^{\nu}
- \frac{\kappa}{2} \partial_{\mu} \barA^{\nu} \partial_{\nu} n^{\mu}
+ \frac{\kappa}{2} \partial_{\mu} n^{\nu} \partial^{\mu} \barA_{\nu}
- \frac{\kappa}{2} \partial_{\mu} n^{\nu} \partial^{\nu} \barA_{\mu} \\
&= \kappa \partial^{\mu} \barA_{\nu} \partial_{\mu} n^{\nu}
- \frac{\kappa}{2} \partial_{\mu} \barA^{\nu} \partial_{\nu} n^{\mu}
- \frac{\kappa}{2} \partial_{\nu} n^{\mu} \partial^{\mu} \barA_{\nu} \\
&= \kappa \partial^{\mu} \barA_{\nu} \partial_{\mu} n^{\nu}
- \kappa \partial_{\nu} \barA^{\mu} \partial_{\mu} n^{\nu} \\
&= \kappa \left( \partial^{\mu} \barA_{\nu} -\partial_{\nu} \barA^{\mu} \right) \partial_{\mu} n^{\nu} \\
&= \kappa \left( \partial^{\mu} \barA^{\nu} -\partial^{\nu} \barA^{\mu} \right) \partial_{\mu} n_{\nu} \\
&= \kappa {F_{\barA}}^{\mu\nu} \partial_{\mu} n_{\nu} \\
\end{align*}

Collecting all the non-fixed and non-quadratic $n^{\mu}$ terms of the action we have

\begin{align*}
\delta S 
&= \int d^4 x \kappa {F_{\barA}}^{\mu\nu} \partial_{\mu} n_{\nu} + J^{\alpha} n_{\alpha} \\
&= \int \left. d^3 \widehat{x^{\mu}} \kappa {F_{\barA}}^{\mu\nu} n_{\nu} \right\vert_{\partial \mu}
 + \int d^4 x \left( 
- \kappa \partial_{\mu} {F_{\barA}}^{\mu\nu} + J^{\nu}
\right) n_{\nu} \\
\end{align*}

Where $d^3 \widehat{x^{\mu}} = dx^\alpha dx^\beta dx^\gamma, \text{for} \{\alpha, \beta, \gamma\} \ne \mu$ denotes the remaining three volume differential element remaining after integration by $dx^{\mu}$.  The expression $\partial \mu$ denotes the boundary of this first integration, and since we have $n=0$ on this boundary this first integral equals zero.

Finally, setting the interior term equal to zero for an extreme value independent of $n^{\nu}$ we have

\begin{equation*}
\partial_{\mu} {F_{\barA}}^{\mu\nu} = \inv{\kappa} J^{\nu}
\end{equation*}

This fixes $\kappa = c \epsilon_0$, and completes half of the recovery of Maxwells equation from a Lagrangian

\begin{equation}
\LL = -\frac{c \epsilon_0}{4} \gpgradezero{(\grad \wedge A)^2} + J \cdot A
\end{equation}

When one adds the Lagrangian that gives us the Lorentz force law

\begin{equation*}
\LL = \inv{2}m v^2 + A \cdot v
\end{equation*}

(positive metric for time implied in this formula), we have a good chunk of non-quantum electrodynamics described in a couple energy minimization relationships.

\subsection{ Determination of scale factor by comparision to electrostatics case. }

FIXME: write this up.  A first attempt to do it this way (ie: before doing the variational work first) ended up with a $1/c$ factor error.   Correct the latex.

%% FIXME:
%%Having done the derivation of electrostatics equation, we can also fix the constant $\kappa$ in the general Lagrangian by
%%comparison by setting $A_\mu = A_\nu = 0$
%%
%%\begin{align*}
%%\LL
%%&=
%%\frac{\kappa}{4}
%%\left(
%%%( \partial_{\mu} A_{\nu} - \partial_{\nu} A_{\mu} ) ( \partial^{\mu} A^{\nu} - \partial^{\nu} A^{\mu} )
%%\partial_{\mu} A_{\nu} \partial^{\mu} A^{\nu}
%%-\partial_{\mu} A_{\nu} \partial^{\nu} A^{\mu}
%%-\partial_{\nu} A_{\mu} \partial^{\mu} A^{\nu}
%%+\partial_{\nu} A_{\mu} \partial^{\nu} A^{\mu}
%%\right)
%%+ J_{\alpha} A^{\alpha} \\
%%&=
%%\frac{\kappa}{4}
%%\left(
%%\partial_{\mu} A_{0} \partial^{\mu} A^{0}
%%-\partial_{0} A_{0} \partial^{0} A^{0}
%%-\partial_{0} A_{0} \partial^{0} A^{0}
%%+\partial_{\nu} A_{0} \partial^{\nu} A^{0}
%%\right)
%%+ J_{0} A^{0} \\
%%&= \frac{\kappa}{2} \partial_{i} A_{0} \partial^{i} A^{0} + J_{0} A^{0} \\
%%&= -\frac{\kappa}{2} (\partial_{i} A^{0})^2 + J_{0} A^{0} \\
%%&= -\frac{\kappa}{2} (\partial_{i} \phi/c)^2 + c \rho \phi/c \\
%%&= -\frac{\kappa}{2 c^2} (\partial_{i} \phi)^2 + \rho \phi \\
%%&= -\frac{\epsilon_0}{2} (\spacegrad \phi)^2 + \rho \phi \\
%%\end{align*}
%%
%%% e mu = 1/c^2
%%% e c^2 mu = 1
%%% e c^2 = 1/mu
%%This dimensional analysis and comparision supplies $\kappa = \epsilon_0 c^2$.  With
%%$\epsilon_0 \mu_0 = 1/c^2$, we have our scale factor before actually solving the extremal problem and equation \ref{eqn:density} takes the form
%%
%%\begin{equation}
%%\LL = \inv{2 \mu_0} \partial_{\mu} A_{\nu} ( \partial^{\mu} A^{\nu} - \partial^{\nu} A^{\mu} ) + J_{\alpha} A^{\alpha}
%%\end{equation}


%%% mu = 1/e c^2
%%
%%Hmm.  Off by a factor of $1/c$ so something went wrong with the determination of $\kappa$.  Other than that this is exactly the tensor form of the vector part of Maxwells equation.

\section{ Trivector components of Maxwells equation from a Lagrangian? }

Next it will be shown that the remaining half \ref{eqn:dualpartofMaxwells}
of Maxwells equation in tensor form can be calculated from the pseudoscalar part of the field square 
\ref{eqn:pseudoscalarFieldSquare} when used as a Lagrangian density.

% F^2 = 2cI E.B
\begin{align*}
\LL 
&= -\frac{\kappa I}{4} \gpgrade{F^2}{4} \\
&= -\frac{\kappa c I}{2} \BE \cdot \BB \\
&= -\frac{\kappa I}{4} \epsilon_{\mu\nu\alpha\beta} \partial^{\mu} A^{\nu} \partial^{\alpha} A^{\beta}
\end{align*}

I think that it will be natural as a follow on to retain the complex nature of the field square later and work with an entirely complex Lagrangian, but for now 
the $-\kappa I/4$ cooeffienent can be dropped since it will not change the overall result.  Where it is useful is that the factor of $I$ implicitly balances the upper and lower indexes used, so a complete index lowering will also not change the result (ie: the pseudoscalar is then correspondingly expressed in terms of index upper basis vectors).

\end{document}               % End of document.
           % Sep 8/08
\documentclass{article}

\usepackage{amsmath}
\usepackage{mathpazo}

%
% shorthand for bold symbols, convenient for vectors and matrices
%
\newcommand{\Ba}[0]{\mathbf{a}}
\newcommand{\Bb}[0]{\mathbf{b}}
\newcommand{\Bc}[0]{\mathbf{c}}
\newcommand{\Bd}[0]{\mathbf{d}}
\newcommand{\Be}[0]{\mathbf{e}}
\newcommand{\Bf}[0]{\mathbf{f}}
\newcommand{\Bg}[0]{\mathbf{g}}
\newcommand{\Bh}[0]{\mathbf{h}}
\newcommand{\Bi}[0]{\mathbf{i}}
\newcommand{\Bj}[0]{\mathbf{j}}
\newcommand{\Bk}[0]{\mathbf{k}}
\newcommand{\Bl}[0]{\mathbf{l}}
\newcommand{\Bm}[0]{\mathbf{m}}
\newcommand{\Bn}[0]{\mathbf{n}}
\newcommand{\Bo}[0]{\mathbf{o}}
\newcommand{\Bp}[0]{\mathbf{p}}
\newcommand{\Bq}[0]{\mathbf{q}}
\newcommand{\Br}[0]{\mathbf{r}}
\newcommand{\Bs}[0]{\mathbf{s}}
\newcommand{\Bt}[0]{\mathbf{t}}
\newcommand{\Bu}[0]{\mathbf{u}}
\newcommand{\Bv}[0]{\mathbf{v}}
\newcommand{\Bw}[0]{\mathbf{w}}
\newcommand{\Bx}[0]{\mathbf{x}}
\newcommand{\By}[0]{\mathbf{y}}
\newcommand{\Bz}[0]{\mathbf{z}}
\newcommand{\BA}[0]{\mathbf{A}}
\newcommand{\BB}[0]{\mathbf{B}}
\newcommand{\BC}[0]{\mathbf{C}}
\newcommand{\BD}[0]{\mathbf{D}}
\newcommand{\BE}[0]{\mathbf{E}}
\newcommand{\BF}[0]{\mathbf{F}}
\newcommand{\BG}[0]{\mathbf{G}}
\newcommand{\BH}[0]{\mathbf{H}}
\newcommand{\BI}[0]{\mathbf{I}}
\newcommand{\BJ}[0]{\mathbf{J}}
\newcommand{\BK}[0]{\mathbf{K}}
\newcommand{\BL}[0]{\mathbf{L}}
\newcommand{\BM}[0]{\mathbf{M}}
\newcommand{\BN}[0]{\mathbf{N}}
\newcommand{\BO}[0]{\mathbf{O}}
\newcommand{\BP}[0]{\mathbf{P}}
\newcommand{\BQ}[0]{\mathbf{Q}}
\newcommand{\BR}[0]{\mathbf{R}}
\newcommand{\BS}[0]{\mathbf{S}}
\newcommand{\BT}[0]{\mathbf{T}}
\newcommand{\BU}[0]{\mathbf{U}}
\newcommand{\BV}[0]{\mathbf{V}}
\newcommand{\BW}[0]{\mathbf{W}}
\newcommand{\BX}[0]{\mathbf{X}}
\newcommand{\BY}[0]{\mathbf{Y}}
\newcommand{\BZ}[0]{\mathbf{Z}}

\newcommand{\Bzero}[0]{\mathbf{0}}
\newcommand{\Btheta}[0]{\boldsymbol{\theta}}
\newcommand{\Btau}[0]{\boldsymbol{\tau}}
\newcommand{\Bomega}[0]{\boldsymbol{\omega}}

%
% shorthand for unit vectors
%
\newcommand{\acap}[0]{\hat{\Ba}}
\newcommand{\bcap}[0]{\hat{\Bb}}
\newcommand{\ccap}[0]{\hat{\Bc}}
\newcommand{\dcap}[0]{\hat{\Bd}}
\newcommand{\ecap}[0]{\hat{\Be}}
\newcommand{\fcap}[0]{\hat{\Bf}}
\newcommand{\gcap}[0]{\hat{\Bg}}
\newcommand{\hcap}[0]{\hat{\Bh}}
\newcommand{\icap}[0]{\hat{\Bi}}
\newcommand{\jcap}[0]{\hat{\Bj}}
\newcommand{\kcap}[0]{\hat{\Bk}}
\newcommand{\lcap}[0]{\hat{\Bl}}
\newcommand{\mcap}[0]{\hat{\Bm}}
\newcommand{\ncap}[0]{\hat{\Bn}}
\newcommand{\ocap}[0]{\hat{\Bo}}
\newcommand{\pcap}[0]{\hat{\Bp}}
\newcommand{\qcap}[0]{\hat{\Bq}}
\newcommand{\rcap}[0]{\hat{\Br}}
\newcommand{\scap}[0]{\hat{\Bs}}
\newcommand{\tcap}[0]{\hat{\Bt}}
\newcommand{\ucap}[0]{\hat{\Bu}}
\newcommand{\vcap}[0]{\hat{\Bv}}
\newcommand{\wcap}[0]{\hat{\Bw}}
\newcommand{\xcap}[0]{\hat{\Bx}}
\newcommand{\ycap}[0]{\hat{\By}}
\newcommand{\zcap}[0]{\hat{\Bz}}
\newcommand{\thetacap}[0]{\hat{\Btheta}}

%
% to write R^n and C^n in a distinguishable fashion.  Perhaps change this
% to the double lined characters upon figuring out how to do so.
%
\newcommand{\C}[1]{$\mathbb{C}^{#1}$}
\newcommand{\R}[1]{$\mathbb{R}^{#1}$}

%
% various generally useful helpers
%

% derivative of #1 wrt. #2:
\newcommand{\D}[2] {\frac {d#2} {d#1}}

\newcommand{\inv}[1]{\frac{1}{#1}}
\newcommand{\cross}[0]{\times}

\newcommand{\abs}[1]{\lvert{#1}\rvert}
\newcommand{\norm}[1]{\lVert{#1}\rVert}
\newcommand{\innerprod}[2]{\langle{#1}, {#2}\rangle}
\newcommand{\dotprod}[2]{{#1} \cdot {#2}}
\newcommand{\bdotprod}[2]{\left({#1} \cdot {#2}\right)}
\newcommand{\crossprod}[2]{{#1} \cross {#2}}
\newcommand{\tripleprod}[3]{\dotprod{\left(\crossprod{#1}{#2}\right)}{#3}}

\DeclareMathOperator{\Proj}{Proj}
\DeclareMathOperator{\Span}{span}
\DeclareMathOperator{\Sgn}{sgn}
\DeclareMathOperator{\Area}{Area}
\DeclareMathOperator{\Volume}{Volume}

%
% A few miscellaneous things specific to this document
%
\newcommand{\crossop}[1]{\crossprod{#1}{}}

% R2 vector.
\newcommand{\VectorTwo}[2]{
\begin{bmatrix}
 {#1} \\
 {#2}
\end{bmatrix}
}

\newcommand{\VectorN}[1]{
\begin{bmatrix}
{#1}_1 \\
{#1}_2 \\
\vdots \\
{#1}_N \\
\end{bmatrix}
}

\newcommand{\DETuvij}[4]{
\begin{vmatrix}
 {#1}_{#3} & {#1}_{#4} \\
 {#2}_{#3} & {#2}_{#4}
\end{vmatrix}
}

\newcommand{\DETuvwijk}[6]{
\begin{vmatrix}
 {#1}_{#4} & {#1}_{#5} & {#1}_{#6} \\
 {#2}_{#4} & {#2}_{#5} & {#2}_{#6} \\
 {#3}_{#4} & {#3}_{#5} & {#3}_{#6}
\end{vmatrix}
}

\newcommand{\DETuvwxijkl}[8]{
\begin{vmatrix}
 {#1}_{#5} & {#1}_{#6} & {#1}_{#7} & {#1}_{#8} \\
 {#2}_{#5} & {#2}_{#6} & {#2}_{#7} & {#2}_{#8} \\
 {#3}_{#5} & {#3}_{#6} & {#3}_{#7} & {#3}_{#8} \\
 {#4}_{#5} & {#4}_{#6} & {#4}_{#7} & {#4}_{#8} \\
\end{vmatrix}
}

%\newcommand{\DETuvwxyijklm}[10]{
%\begin{vmatrix}
% {#1}_{#6} & {#1}_{#7} & {#1}_{#8} & {#1}_{#9} & {#1}_{#10} \\
% {#2}_{#6} & {#2}_{#7} & {#2}_{#8} & {#2}_{#9} & {#2}_{#10} \\
% {#3}_{#6} & {#3}_{#7} & {#3}_{#8} & {#3}_{#9} & {#3}_{#10} \\
% {#4}_{#6} & {#4}_{#7} & {#4}_{#8} & {#4}_{#9} & {#4}_{#10} \\
% {#5}_{#6} & {#5}_{#7} & {#5}_{#8} & {#5}_{#9} & {#5}_{#10}
%\end{vmatrix}
%}

% R3 vector.
\newcommand{\VectorThree}[3]{
\begin{bmatrix}
 {#1} \\
 {#2} \\
 {#3}
\end{bmatrix}
}


\newcommand{\grad}[0]{\nabla}
\newcommand{\spacegrad}[0]{\boldsymbol{\nabla}}
\newcommand{\LL}[0]{\mathcal{L}}
\newcommand{\xdot}[0]{\dot{x}}
\newcommand{\xddot}[0]{\ddot{x}}
\newcommand{\pdot}[0]{\dot{p}}
\newcommand{\pddot}[0]{\ddot{p}}

\usepackage[bookmarks=true]{hyperref}

\title{ Revisit Lorentz force from Lagrangian. }
\author{Peeter Joot}
\date{ October 8, 2008.  Last Revision: $Date: 2008/10/09 04:24:58 $ }

\begin{document}

\maketitle{}
%\tableofcontents
%\section{ Motivation }

In \cite{PJSrLagrangian} a derivation of the Lorentz force in covariant
form was performed.  Intuition says that result, because of the squared
proper velocity, was dependent on the 
positive time Minkowski signature.  This signature is the common convention
in \cite{doran2003gap}, using Hestenes' STA relativistic formulation.  With many
GR references using the opposite signature, it seems worthwhile to understand
what results are signature dependent and put them in a signature invariant form.

Here the result will be rederived without assuming this signature.

Assume a Lagrangian of the following form

\begin{align}
\LL = \inv{2}m v^2 + \kappa A \cdot v
\end{align}

where $v$ is the proper velocity.  Here $A(x^\mu,\xdot^\nu) = A(x^\mu)$ is a position but not velocity dependent four vector potential.  The constant $\kappa$ includes the charge of the test mass, and will be determined exactly in due course.

As observed in \cite{PJCanMomentum} the equations of motion can be written

\begin{align}
\grad \LL = \frac{d}{d\tau}(\grad_v \LL)
\end{align}

We have

\begin{align*}
\grad v^2 &= 0 \\
\grad (A \cdot v)
&= \grad A_\mu \xdot^\mu \\
&= \gamma^\nu \xdot^\mu \partial_\nu A_\mu \\
\inv{2} \grad_v v^2 
&= \inv{2} \grad_v (\gamma_\mu)^2 (\xdot^\mu)^2 \\
&= \gamma^\nu (\gamma_\mu)^2 \partial_{\xdot^\nu} \xdot^\mu \\
&= \gamma^\mu (\gamma_\mu)^2 \xdot^\mu \\
&= \gamma_\mu \xdot^\mu \\
&= v \\
\grad_v (A \cdot v)
&= \gamma^\nu \partial_{\xdot^\nu} A_\mu \xdot^\mu \\
&= \gamma^\nu A_\mu {\delta^\mu}_\nu \\
&= \gamma^\mu A_\mu \\
&= A \\
\frac{d}{d\tau} &= \xdot^\mu \partial_\mu
\end{align*}

Putting all this back together 
\begin{align*}
\grad \LL &= \frac{d}{d\tau}(\grad_v \LL) \\
\kappa \gamma^\nu \xdot^\mu \partial_\nu A_\mu &= \frac{d}{d\tau}\left( m v + \kappa A \right) \\
\implies \\
\pdot 
&= \kappa \left( \gamma^\nu \xdot^\mu \partial_\nu A_\mu - \xdot^\nu \partial_\nu \gamma^\mu A_\mu \right) \\
&= \kappa \partial_\nu A_\mu \left( \gamma^\nu \xdot^\mu - \xdot^\nu \gamma^\mu \right) \\
\end{align*}

We know this will be related to $F \cdot v$, where $F = \grad \wedge A$.  Expanding that for comparision

\begin{align*}
F \cdot v
&= (\grad \wedge A) \cdot v \\
&= (\gamma^\mu \wedge \gamma^\nu) \cdot \gamma_\alpha \xdot^\alpha \partial_\mu A_\nu \\
&= \left( \gamma^\mu {\delta^\nu}_\alpha -\gamma^\nu {\delta^\mu}_\alpha \right) \xdot^\alpha \partial_\mu A_\nu \\
&= 
\gamma^\mu \xdot^\nu \partial_\mu A_\nu 
-\gamma^\nu \xdot^\mu \partial_\mu A_\nu \\
&= \partial_\nu A_\mu \left( \gamma^\nu \xdot^\mu -\gamma^\mu \xdot^\nu \right) \\
\end{align*}

Comparison shows that we therefore have

\begin{align}\label{eqn:lorentzUndeterminedConst}
\pdot = \kappa F \cdot v
\end{align}

A reasonable approach to fix the constant $\kappa$ is to put this into correspondance with the classical
vector form of the Lorentz force equation.

Introduce a rest observer, with worldline $x = ct e_0$.  Computation of the spatial parts of the four vector force equation \ref{eqn:lorentzUndeterminedConst} for this rest observer requires taking the wedge product
with the observer velocity $v = c \gamma e_0$.  This will discard the timelike components of the force equation with
respect to this observer rest frame, and leave only the 
purely spatial (using Euclidian vector-like spatial bivector basis $\{\sigma_i = e_i \wedge e_0\}$) components
for that observer.  For clarity, for the observer frame we use a different set of basis vectors $\{e_\mu\}$, to point
out that $\gamma_0$ of the derivation above does not have to equal $e_0$.  Since the end result of the Lagrangian calculation
ended up being coordinate and signature free, this is perhaps superfluous.

Omitting the scale factor $\gamma = dt/d\tau$ for now, application of a wedge with $e_0$ operation to both sides 
will suffice to determine this observer dependent expression of the force

\bibliographystyle{plainnat} % supposed to allow for \url use.
\bibliography{myrefs}      % expects file "myrefs.bib"

\end{document}               % End of document.
                      % Oct 8/08
\documentclass{article}

\usepackage{amsmath}
\usepackage{mathpazo}

%
% shorthand for bold symbols, convenient for vectors and matrices
%
\newcommand{\Ba}[0]{\mathbf{a}}
\newcommand{\Bb}[0]{\mathbf{b}}
\newcommand{\Bc}[0]{\mathbf{c}}
\newcommand{\Bd}[0]{\mathbf{d}}
\newcommand{\Be}[0]{\mathbf{e}}
\newcommand{\Bf}[0]{\mathbf{f}}
\newcommand{\Bg}[0]{\mathbf{g}}
\newcommand{\Bh}[0]{\mathbf{h}}
\newcommand{\Bi}[0]{\mathbf{i}}
\newcommand{\Bj}[0]{\mathbf{j}}
\newcommand{\Bk}[0]{\mathbf{k}}
\newcommand{\Bl}[0]{\mathbf{l}}
\newcommand{\Bm}[0]{\mathbf{m}}
\newcommand{\Bn}[0]{\mathbf{n}}
\newcommand{\Bo}[0]{\mathbf{o}}
\newcommand{\Bp}[0]{\mathbf{p}}
\newcommand{\Bq}[0]{\mathbf{q}}
\newcommand{\Br}[0]{\mathbf{r}}
\newcommand{\Bs}[0]{\mathbf{s}}
\newcommand{\Bt}[0]{\mathbf{t}}
\newcommand{\Bu}[0]{\mathbf{u}}
\newcommand{\Bv}[0]{\mathbf{v}}
\newcommand{\Bw}[0]{\mathbf{w}}
\newcommand{\Bx}[0]{\mathbf{x}}
\newcommand{\By}[0]{\mathbf{y}}
\newcommand{\Bz}[0]{\mathbf{z}}
\newcommand{\BA}[0]{\mathbf{A}}
\newcommand{\BB}[0]{\mathbf{B}}
\newcommand{\BC}[0]{\mathbf{C}}
\newcommand{\BD}[0]{\mathbf{D}}
\newcommand{\BE}[0]{\mathbf{E}}
\newcommand{\BF}[0]{\mathbf{F}}
\newcommand{\BG}[0]{\mathbf{G}}
\newcommand{\BH}[0]{\mathbf{H}}
\newcommand{\BI}[0]{\mathbf{I}}
\newcommand{\BJ}[0]{\mathbf{J}}
\newcommand{\BK}[0]{\mathbf{K}}
\newcommand{\BL}[0]{\mathbf{L}}
\newcommand{\BM}[0]{\mathbf{M}}
\newcommand{\BN}[0]{\mathbf{N}}
\newcommand{\BO}[0]{\mathbf{O}}
\newcommand{\BP}[0]{\mathbf{P}}
\newcommand{\BQ}[0]{\mathbf{Q}}
\newcommand{\BR}[0]{\mathbf{R}}
\newcommand{\BS}[0]{\mathbf{S}}
\newcommand{\BT}[0]{\mathbf{T}}
\newcommand{\BU}[0]{\mathbf{U}}
\newcommand{\BV}[0]{\mathbf{V}}
\newcommand{\BW}[0]{\mathbf{W}}
\newcommand{\BX}[0]{\mathbf{X}}
\newcommand{\BY}[0]{\mathbf{Y}}
\newcommand{\BZ}[0]{\mathbf{Z}}

\newcommand{\Bzero}[0]{\mathbf{0}}
\newcommand{\Btheta}[0]{\boldsymbol{\theta}}
\newcommand{\Btau}[0]{\boldsymbol{\tau}}
\newcommand{\Bomega}[0]{\boldsymbol{\omega}}

%
% shorthand for unit vectors
%
\newcommand{\acap}[0]{\hat{\Ba}}
\newcommand{\bcap}[0]{\hat{\Bb}}
\newcommand{\ccap}[0]{\hat{\Bc}}
\newcommand{\dcap}[0]{\hat{\Bd}}
\newcommand{\ecap}[0]{\hat{\Be}}
\newcommand{\fcap}[0]{\hat{\Bf}}
\newcommand{\gcap}[0]{\hat{\Bg}}
\newcommand{\hcap}[0]{\hat{\Bh}}
\newcommand{\icap}[0]{\hat{\Bi}}
\newcommand{\jcap}[0]{\hat{\Bj}}
\newcommand{\kcap}[0]{\hat{\Bk}}
\newcommand{\lcap}[0]{\hat{\Bl}}
\newcommand{\mcap}[0]{\hat{\Bm}}
\newcommand{\ncap}[0]{\hat{\Bn}}
\newcommand{\ocap}[0]{\hat{\Bo}}
\newcommand{\pcap}[0]{\hat{\Bp}}
\newcommand{\qcap}[0]{\hat{\Bq}}
\newcommand{\rcap}[0]{\hat{\Br}}
\newcommand{\scap}[0]{\hat{\Bs}}
\newcommand{\tcap}[0]{\hat{\Bt}}
\newcommand{\ucap}[0]{\hat{\Bu}}
\newcommand{\vcap}[0]{\hat{\Bv}}
\newcommand{\wcap}[0]{\hat{\Bw}}
\newcommand{\xcap}[0]{\hat{\Bx}}
\newcommand{\ycap}[0]{\hat{\By}}
\newcommand{\zcap}[0]{\hat{\Bz}}
\newcommand{\thetacap}[0]{\hat{\Btheta}}

%
% to write R^n and C^n in a distinguishable fashion.  Perhaps change this
% to the double lined characters upon figuring out how to do so.
%
\newcommand{\C}[1]{$\mathbb{C}^{#1}$}
\newcommand{\R}[1]{$\mathbb{R}^{#1}$}

%
% various generally useful helpers
%

% derivative of #1 wrt. #2:
\newcommand{\D}[2] {\frac {d#2} {d#1}}

\newcommand{\inv}[1]{\frac{1}{#1}}
\newcommand{\cross}[0]{\times}

\newcommand{\abs}[1]{\lvert{#1}\rvert}
\newcommand{\norm}[1]{\lVert{#1}\rVert}
\newcommand{\innerprod}[2]{\langle{#1}, {#2}\rangle}
\newcommand{\dotprod}[2]{{#1} \cdot {#2}}
\newcommand{\bdotprod}[2]{\left({#1} \cdot {#2}\right)}
\newcommand{\crossprod}[2]{{#1} \cross {#2}}
\newcommand{\tripleprod}[3]{\dotprod{\left(\crossprod{#1}{#2}\right)}{#3}}

\DeclareMathOperator{\Proj}{Proj}
\DeclareMathOperator{\Span}{span}
\DeclareMathOperator{\Sgn}{sgn}
\DeclareMathOperator{\Area}{Area}
\DeclareMathOperator{\Volume}{Volume}

%
% A few miscellaneous things specific to this document
%
\newcommand{\crossop}[1]{\crossprod{#1}{}}

% R2 vector.
\newcommand{\VectorTwo}[2]{
\begin{bmatrix}
 {#1} \\
 {#2}
\end{bmatrix}
}

\newcommand{\VectorN}[1]{
\begin{bmatrix}
{#1}_1 \\
{#1}_2 \\
\vdots \\
{#1}_N \\
\end{bmatrix}
}

\newcommand{\DETuvij}[4]{
\begin{vmatrix}
 {#1}_{#3} & {#1}_{#4} \\
 {#2}_{#3} & {#2}_{#4}
\end{vmatrix}
}

\newcommand{\DETuvwijk}[6]{
\begin{vmatrix}
 {#1}_{#4} & {#1}_{#5} & {#1}_{#6} \\
 {#2}_{#4} & {#2}_{#5} & {#2}_{#6} \\
 {#3}_{#4} & {#3}_{#5} & {#3}_{#6}
\end{vmatrix}
}

\newcommand{\DETuvwxijkl}[8]{
\begin{vmatrix}
 {#1}_{#5} & {#1}_{#6} & {#1}_{#7} & {#1}_{#8} \\
 {#2}_{#5} & {#2}_{#6} & {#2}_{#7} & {#2}_{#8} \\
 {#3}_{#5} & {#3}_{#6} & {#3}_{#7} & {#3}_{#8} \\
 {#4}_{#5} & {#4}_{#6} & {#4}_{#7} & {#4}_{#8} \\
\end{vmatrix}
}

%\newcommand{\DETuvwxyijklm}[10]{
%\begin{vmatrix}
% {#1}_{#6} & {#1}_{#7} & {#1}_{#8} & {#1}_{#9} & {#1}_{#10} \\
% {#2}_{#6} & {#2}_{#7} & {#2}_{#8} & {#2}_{#9} & {#2}_{#10} \\
% {#3}_{#6} & {#3}_{#7} & {#3}_{#8} & {#3}_{#9} & {#3}_{#10} \\
% {#4}_{#6} & {#4}_{#7} & {#4}_{#8} & {#4}_{#9} & {#4}_{#10} \\
% {#5}_{#6} & {#5}_{#7} & {#5}_{#8} & {#5}_{#9} & {#5}_{#10}
%\end{vmatrix}
%}

% R3 vector.
\newcommand{\VectorThree}[3]{
\begin{bmatrix}
 {#1} \\
 {#2} \\
 {#3}
\end{bmatrix}
}


\newcommand{\LL}[0]{\mathcal{L}}
\newcommand{\PD}[2]{\frac{\partial {#2}}{\partial {#1}}}
\newcommand{\barA}[0]{\bar{A}}

\usepackage[bookmarks=true]{hyperref}

\title{ Derivation of Euler-Lagrange field Lagrangian equations.}
\author{Peeter Joot}
\date{ October 10, 2008.  Last Revision: $Date: 2008/10/11 04:49:09 $ }

\begin{document}

\maketitle{}

\tableofcontents

\section{ Motivation. }

In \cite{PJMaxwellLagrangian} Maxwell's equations were derived from
a Lagrangian action in tensor and STA forms.  This was done with 
Feynman's \cite{feynman1963flp} simple, but somewhat non-rigorous, direct variational technique.

An alternate approach is to use a field form of the Euler-Lagrange
equations as done in the wikipedia article \cite{wikiemtensor}.  I had
trouble understanding that derivation, probably because
I didn't understand the notation, nor what the source of that equation.

Here Feynman's approach will be used to derive the field versions of the Euler-Lagrange
equations, which clarifies the notation.  As a verification of the correctness these
will be applied to derive Maxwell's equation.

\section{ Deriving the field Lagrangian equations }

That essence of Feynman's method from his 
``Principle of Least Action'' entertainment chapter of the Lectures is
to do a first order linear expansion of the function, ignore all the higher order terms,
then do the integration by parts for the remainder.

Looking at the Maxwell field Lagrangian and action for motivation,

\begin{align*}
\LL &= (\partial^\mu A^\nu - \partial^\nu A^\mu) (\partial_\mu A_\nu - \partial_\nu A_\mu) + \kappa J^\sigma A_\sigma \\
S &= \int d^4 x \LL
\end{align*}

where the potential functions $A^\mu$ (or their index lowered variants)
are to be determined by extreme values of the action variation.  Note the use of the shorthand
$\partial_\mu \equiv \PD{x^\mu}{}$.

We want to consider general Lagrangians of this form.  Write

\begin{align}
\LL = \LL( A^\mu, \partial_\nu A^\sigma) = \LL(A^0, A^1, \cdots, \partial_0 A^0, \partial_1 A^0, \cdots )
\end{align}

\subsection{ First order Taylor expansion of a multi variable function. }

Given an abstractly specified function like this, with indexes and partials flying around, how to do a first order Taylor series expansion may not be obvious, especially since the variables are all undetermined functions!

Consideration of a simple case guides the way.  Assume that a two variable function can be expressed as a polynomial of some order

\begin{align*}
f(x,y) = a_{i j} x^i y^j
\end{align*}

Evaluation of this function or its partials at $(x,y) = (0,0)$ supply the constants $a_{i j}$.  Simplest is the lowest order constant

\begin{align*}
f(0,0) &= a_{0 0} \\
\end{align*}

\begin{align*}
\partial_x f &= i a_{i j} x^{i-1} y^j \\
\partial_y f &= j a_{i j} x^{i} y^{j-1} \\
\partial_{xx} f &= i(i-1) a_{i j} x^{i-2} y^j \\
\partial_{yy} f &= j(j-1) a_{i j} x^{i} y^{j-2} \\
\partial_{xy} f &= i j a_{i j} x^{i-1} y^{j-1} \\
\hdots \\
\implies \\
a_{1 0} &= (\partial_x f) \vert_0 \\
a_{0 1} &= (\partial_y f) \vert_0 \\
a_{2 0} &= \inv{2!} (\partial_{xx} f) \vert_0 \\
a_{0 2} &= \inv{2!} (\partial_{yy} f) \vert_0 \\
a_{1 1} &= (\partial_{xy} f) \vert_0 = (\partial_{yx} f) \vert_0 \\
\hdots
\end{align*}

Or

\begin{align*}
f(x,y) &= f \vert_0 + x (\partial_x f) \vert_0 + y (\partial_y f) \vert_0 \\
&+ \inv{2} \left( x^2 (\partial_{x x} f) \vert_0 + x y (\partial_{x y} f) \vert_0  + y x (\partial_{y x} f) \vert_0  + y^2(\partial_{y y} f) \vert_0 \right) \\
&+ \sum_{(i+j) > 2} a_{i j} x^i y^j
\end{align*}

\subsection{ First order expansion of the Lagrangian function. }

It isn't hard to see that the same thing can be done for higher degree functions too, although enumerating the 
higher order terms will get messier, however for the purposes of this variational exercise the assumption is that only
the first order differential terms are significant.

How to do the first order Taylor expansion of a multivariable function has been established.  Next write $A^\mu = \barA^\mu + n^\mu$, where the $\barA^\mu$ functions are the desired solutions and each of $n^\mu$ vanishes on the boundaries of the integration region.  Expansion of $\LL$ around the desired solutions one has

\begin{align*}
\LL(\barA^\mu + n^\mu, \partial_\nu( \barA^\sigma + n^\sigma) )
&=
\LL(\barA^\mu, \partial_\nu \barA^\sigma ) \\
&+ (\barA^\mu + n^\mu) \left( \left. \PD{A^\mu}{\LL} \right) \right\vert_{A^\mu = \barA^\mu} \\
&+ (\partial_\nu \barA^\sigma + \partial_\nu n^\sigma) \left( \left. \PD{(\partial_\nu A^\sigma)}{\LL} \right) \right\vert_{\partial_\nu A^\sigma = \partial_\nu \barA^\sigma} \\
&+ \sum_{i+j>2} (\barA^\mu + n^\mu)^i (\partial_\nu \barA^\sigma + \partial_\nu n^\sigma)^j \underbrace{\left(\cdots\right)}_{\text{higher order derivatives}}
\end{align*}

\subsection{ Example for clarification. }
Here we see the first use of the peculiar looking partials from the wikipedia article

\begin{align*}
\PD{(\partial_\nu A^\sigma)}{\LL}.
\end{align*}

Initially looking at that I couldn't fathom what it meant, but it is just what it says, 
differentiation with respect to a variable $\partial_\nu A^\sigma$.  As an example, for

\begin{align*}
\LL 
&= u A^0 + v A^1 + a \partial_1 A^0 + b \partial_0 A^1 \\
&= u A^0 + v A^1 + a \PD{x^1}{A^0} + b \PD{x^0}{A^1}
\end{align*}

where $u$,$v$,$a$, and $b$ are constants.  Then an corresponding example of such a partial term is

\begin{align*}
\PD{(\partial_1 A^0)}{\LL} = a.
\end{align*}

\subsection{ Calculationg of the action for the general field Lagrangian. }

\begin{align*}
S &= \int d^4 x \LL \\
&= \int d^4 x \LL(\barA^\mu, \partial_\nu \barA^\sigma ) \\
&+ \int d^4 x (\barA^\mu + n^\mu) \left( \left. \PD{A^\mu}{\LL} \right) \right\vert_{A^\mu = \barA^\mu} \\
&+ \int d^4 x (\partial_\nu \barA^\sigma + \partial_\nu n^\sigma) \left( \left. \PD{(\partial_\nu A^\sigma)}{\LL} \right) \right\vert_{\partial_\nu A^\sigma = \partial_\nu \barA^\sigma} \\
&+ \int d^4 x (\cdots \text{neglected higher order terms} \cdots )
\end{align*}

Grouping this into parts associated with the assumed variational solution, and the varied parts we have

\begin{align*}
S &= \int d^4 x 
\left(
\LL(\barA^\mu, \partial_\nu \barA^\sigma ) + \barA^\mu \left. \left( \PD{A^\mu}{\LL} \right) \right\vert_{A^\mu = \barA^\mu} 
+ \partial_\nu \barA^\sigma \left. \left( \PD{(\partial_\nu A^\sigma)}{\LL} \right) \right\vert_{\partial_\nu A^\sigma = \partial_\nu \barA^\sigma} 
\right) \\
&+ \int d^4 x 
\left(
n^\mu \left. \left( \PD{A^\mu}{\LL} \right) \right\vert_{A^\mu = \barA^\mu} 
+\partial_\nu n^\sigma \left. \left( \PD{(\partial_\nu A^\sigma)}{\LL} \right) \right\vert_{\partial_\nu A^\sigma = \partial_\nu \barA^\sigma}
\right) \\
&+ \cdots
\end{align*}

None of the terms in the first integral are of interest since they are fixed.  The second term of the remaining integral is the one to integrate by
parts.  For short, let

\begin{align*}
u &= \left. \left( \PD{(\partial_\nu A^\sigma)}{\LL} \right) \right\vert_{\partial_\nu A^\mu = \partial_\nu \barA^\mu}
\end{align*}

then this integral is
%(u v)' = u' v + u v'
%\int u' v = \int (u v)' - \int u v' 
%\int u' v = u v - \int u v' 
\begin{align*}
\int d^3 x d x^\nu \PD{x^\nu}{n^\sigma} u 
&= \int d^3 x \left. \left( {n^\sigma} u \right) \right\vert_{\partial x^\nu} - \int d^4 x n^\sigma \PD{x^\nu}{} u
\end{align*}

Here $\partial x^\nu$ denotes the boundary of the integration.  Because $n^\sigma$ was by definition zero on all boundaries of the
integral region this first integral is zero.  Denoting the non-variational parts of the action integral by $\delta S$, we have

\begin{align*}
\delta S
&= \int d^4 x \left(
n^\mu \left. \left( \PD{A^\mu}{\LL} \right) \right\vert_{A^\mu = \barA^\mu} 
- n^\sigma \PD{x^\nu}{} \left. \left( \PD{(\partial_\nu A^\sigma)}{\LL} \right) \right\vert_{\partial_\nu A^\sigma = \partial_\nu \barA^\sigma}
\right) \\
&= \int d^4 x
n^\sigma
\left(
\left. \left( \PD{A^\sigma}{\LL} \right) \right\vert_{A^\sigma = \barA^\sigma} 
- \PD{x^\nu}{} \left. \left( \PD{(\partial_\nu A^\sigma)}{\LL} \right) \right\vert_{\partial_\nu A^\sigma = \partial_\nu \barA^\sigma}
\right)
\end{align*}

Now, for $\delta S = 0$ for all possible variations $n^\sigma$ from the optimal solution $\barA^\sigma$, then the inner expression must also be zero
for all $\sigma$.  Specifically

\begin{align}\label{eqn:eulerLagrangeField}
\PD{A^\sigma}{\LL} = \PD{x^\nu}{} \PD{(\partial_\nu A^\sigma)}{\LL}.
\end{align}

Feynman's direct approach does not require too much to understand, and one can intuit through it fairly easily.  Contrast to 
\cite{goldstein1951cm}
where the same result appears to be derived in Chapter 13.  That approach requires the use and familiarity with a functional derivative

\begin{align*}
\frac{\delta \LL}{\delta A^\sigma} &= \PD{A^\sigma}{\LL} - \frac{d}{dx^\nu} \PD{(\partial_\nu A^\sigma)}{\LL}
\end{align*}

which must be defined and explained somewhere earlier in the book in one of the chapters that I skimmed over to get to the interesting ``Continuous Systems and Fields'' content
at the end of the book.

FIXME: I am also not clear why Goldstein would have a complete derivative $d/dx^\nu$ here instead of $\partial_\nu = \partial/{\partial x^\nu}$.  A more thoroughly worked simple example
of the integration by parts in two variables can be found in the plane solution of an electrostatics Lagrangian in \cite{PJMaxwellLagrangian}.  Based on the arguments there I think that it has to be a partial derivative.   The partial also happens to be consistent with both the wikipedia article \cite{wikiemtensor}, and with my-on-paper derivation of maxwell's equations using the result \ref{eqn:eulerLagrangeField} (to-be-typed-up).

\section{ Verifying the equations.  Maxwell's equation derivation from action. }

\bibliographystyle{plainnat} % supposed to allow for \url use.
\bibliography{myrefs}      % expects file "myrefs.bib"

\end{document}               % End of document.
                   % Oct 10/08
\documentclass{article}

\usepackage{amsmath}
\usepackage{mathpazo}

%
% shorthand for bold symbols, convenient for vectors and matrices
%
\newcommand{\Ba}[0]{\mathbf{a}}
\newcommand{\Bb}[0]{\mathbf{b}}
\newcommand{\Bc}[0]{\mathbf{c}}
\newcommand{\Bd}[0]{\mathbf{d}}
\newcommand{\Be}[0]{\mathbf{e}}
\newcommand{\Bf}[0]{\mathbf{f}}
\newcommand{\Bg}[0]{\mathbf{g}}
\newcommand{\Bh}[0]{\mathbf{h}}
\newcommand{\Bi}[0]{\mathbf{i}}
\newcommand{\Bj}[0]{\mathbf{j}}
\newcommand{\Bk}[0]{\mathbf{k}}
\newcommand{\Bl}[0]{\mathbf{l}}
\newcommand{\Bm}[0]{\mathbf{m}}
\newcommand{\Bn}[0]{\mathbf{n}}
\newcommand{\Bo}[0]{\mathbf{o}}
\newcommand{\Bp}[0]{\mathbf{p}}
\newcommand{\Bq}[0]{\mathbf{q}}
\newcommand{\Br}[0]{\mathbf{r}}
\newcommand{\Bs}[0]{\mathbf{s}}
\newcommand{\Bt}[0]{\mathbf{t}}
\newcommand{\Bu}[0]{\mathbf{u}}
\newcommand{\Bv}[0]{\mathbf{v}}
\newcommand{\Bw}[0]{\mathbf{w}}
\newcommand{\Bx}[0]{\mathbf{x}}
\newcommand{\By}[0]{\mathbf{y}}
\newcommand{\Bz}[0]{\mathbf{z}}
\newcommand{\BA}[0]{\mathbf{A}}
\newcommand{\BB}[0]{\mathbf{B}}
\newcommand{\BC}[0]{\mathbf{C}}
\newcommand{\BD}[0]{\mathbf{D}}
\newcommand{\BE}[0]{\mathbf{E}}
\newcommand{\BF}[0]{\mathbf{F}}
\newcommand{\BG}[0]{\mathbf{G}}
\newcommand{\BH}[0]{\mathbf{H}}
\newcommand{\BI}[0]{\mathbf{I}}
\newcommand{\BJ}[0]{\mathbf{J}}
\newcommand{\BK}[0]{\mathbf{K}}
\newcommand{\BL}[0]{\mathbf{L}}
\newcommand{\BM}[0]{\mathbf{M}}
\newcommand{\BN}[0]{\mathbf{N}}
\newcommand{\BO}[0]{\mathbf{O}}
\newcommand{\BP}[0]{\mathbf{P}}
\newcommand{\BQ}[0]{\mathbf{Q}}
\newcommand{\BR}[0]{\mathbf{R}}
\newcommand{\BS}[0]{\mathbf{S}}
\newcommand{\BT}[0]{\mathbf{T}}
\newcommand{\BU}[0]{\mathbf{U}}
\newcommand{\BV}[0]{\mathbf{V}}
\newcommand{\BW}[0]{\mathbf{W}}
\newcommand{\BX}[0]{\mathbf{X}}
\newcommand{\BY}[0]{\mathbf{Y}}
\newcommand{\BZ}[0]{\mathbf{Z}}

\newcommand{\Bzero}[0]{\mathbf{0}}
\newcommand{\Btheta}[0]{\boldsymbol{\theta}}
\newcommand{\Btau}[0]{\boldsymbol{\tau}}
\newcommand{\Bomega}[0]{\boldsymbol{\omega}}

%
% shorthand for unit vectors
%
\newcommand{\acap}[0]{\hat{\Ba}}
\newcommand{\bcap}[0]{\hat{\Bb}}
\newcommand{\ccap}[0]{\hat{\Bc}}
\newcommand{\dcap}[0]{\hat{\Bd}}
\newcommand{\ecap}[0]{\hat{\Be}}
\newcommand{\fcap}[0]{\hat{\Bf}}
\newcommand{\gcap}[0]{\hat{\Bg}}
\newcommand{\hcap}[0]{\hat{\Bh}}
\newcommand{\icap}[0]{\hat{\Bi}}
\newcommand{\jcap}[0]{\hat{\Bj}}
\newcommand{\kcap}[0]{\hat{\Bk}}
\newcommand{\lcap}[0]{\hat{\Bl}}
\newcommand{\mcap}[0]{\hat{\Bm}}
\newcommand{\ncap}[0]{\hat{\Bn}}
\newcommand{\ocap}[0]{\hat{\Bo}}
\newcommand{\pcap}[0]{\hat{\Bp}}
\newcommand{\qcap}[0]{\hat{\Bq}}
\newcommand{\rcap}[0]{\hat{\Br}}
\newcommand{\scap}[0]{\hat{\Bs}}
\newcommand{\tcap}[0]{\hat{\Bt}}
\newcommand{\ucap}[0]{\hat{\Bu}}
\newcommand{\vcap}[0]{\hat{\Bv}}
\newcommand{\wcap}[0]{\hat{\Bw}}
\newcommand{\xcap}[0]{\hat{\Bx}}
\newcommand{\ycap}[0]{\hat{\By}}
\newcommand{\zcap}[0]{\hat{\Bz}}
\newcommand{\thetacap}[0]{\hat{\Btheta}}

%
% to write R^n and C^n in a distinguishable fashion.  Perhaps change this
% to the double lined characters upon figuring out how to do so.
%
\newcommand{\C}[1]{$\mathbb{C}^{#1}$}
\newcommand{\R}[1]{$\mathbb{R}^{#1}$}

%
% various generally useful helpers
%

% derivative of #1 wrt. #2:
\newcommand{\D}[2] {\frac {d#2} {d#1}}

\newcommand{\inv}[1]{\frac{1}{#1}}
\newcommand{\cross}[0]{\times}

\newcommand{\abs}[1]{\lvert{#1}\rvert}
\newcommand{\norm}[1]{\lVert{#1}\rVert}
\newcommand{\innerprod}[2]{\langle{#1}, {#2}\rangle}
\newcommand{\dotprod}[2]{{#1} \cdot {#2}}
\newcommand{\bdotprod}[2]{\left({#1} \cdot {#2}\right)}
\newcommand{\crossprod}[2]{{#1} \cross {#2}}
\newcommand{\tripleprod}[3]{\dotprod{\left(\crossprod{#1}{#2}\right)}{#3}}

\DeclareMathOperator{\Proj}{Proj}
\DeclareMathOperator{\Span}{span}
\DeclareMathOperator{\Sgn}{sgn}
\DeclareMathOperator{\Area}{Area}
\DeclareMathOperator{\Volume}{Volume}

%
% A few miscellaneous things specific to this document
%
\newcommand{\crossop}[1]{\crossprod{#1}{}}

% R2 vector.
\newcommand{\VectorTwo}[2]{
\begin{bmatrix}
 {#1} \\
 {#2}
\end{bmatrix}
}

\newcommand{\VectorN}[1]{
\begin{bmatrix}
{#1}_1 \\
{#1}_2 \\
\vdots \\
{#1}_N \\
\end{bmatrix}
}

\newcommand{\DETuvij}[4]{
\begin{vmatrix}
 {#1}_{#3} & {#1}_{#4} \\
 {#2}_{#3} & {#2}_{#4}
\end{vmatrix}
}

\newcommand{\DETuvwijk}[6]{
\begin{vmatrix}
 {#1}_{#4} & {#1}_{#5} & {#1}_{#6} \\
 {#2}_{#4} & {#2}_{#5} & {#2}_{#6} \\
 {#3}_{#4} & {#3}_{#5} & {#3}_{#6}
\end{vmatrix}
}

\newcommand{\DETuvwxijkl}[8]{
\begin{vmatrix}
 {#1}_{#5} & {#1}_{#6} & {#1}_{#7} & {#1}_{#8} \\
 {#2}_{#5} & {#2}_{#6} & {#2}_{#7} & {#2}_{#8} \\
 {#3}_{#5} & {#3}_{#6} & {#3}_{#7} & {#3}_{#8} \\
 {#4}_{#5} & {#4}_{#6} & {#4}_{#7} & {#4}_{#8} \\
\end{vmatrix}
}

%\newcommand{\DETuvwxyijklm}[10]{
%\begin{vmatrix}
% {#1}_{#6} & {#1}_{#7} & {#1}_{#8} & {#1}_{#9} & {#1}_{#10} \\
% {#2}_{#6} & {#2}_{#7} & {#2}_{#8} & {#2}_{#9} & {#2}_{#10} \\
% {#3}_{#6} & {#3}_{#7} & {#3}_{#8} & {#3}_{#9} & {#3}_{#10} \\
% {#4}_{#6} & {#4}_{#7} & {#4}_{#8} & {#4}_{#9} & {#4}_{#10} \\
% {#5}_{#6} & {#5}_{#7} & {#5}_{#8} & {#5}_{#9} & {#5}_{#10}
%\end{vmatrix}
%}

% R3 vector.
\newcommand{\VectorThree}[3]{
\begin{bmatrix}
 {#1} \\
 {#2} \\
 {#3}
\end{bmatrix}
}



\newcommand{\LL}[0]{\mathcal{L}}
\newcommand{\xdot}[0]{\dot{x}}
\newcommand{\xddot}[0]{\ddot{x}}
\newcommand{\pdot}[0]{\dot{p}}
\newcommand{\pddot}[0]{\ddot{p}}
\newcommand{\PD}[2]{\frac{\partial {#2}}{\partial {#1}}}

\usepackage[bookmarks=true]{hyperref}

\title{Tensor Derivation of Covariant Lorentz Force from Lagrangian}
\author{Peeter Joot}
\date{ October 12, 2008.  Last Revision: $Date: 2008/10/25 21:31:36 $ }

\begin{document}

\maketitle{}

\tableofcontents

\section{ Motivation. }

In \cite{PJSrLorentzForce}, and before that in \cite{PJSrLagrangian} clifford
algebra derivations of the STA form of the covariant Lorentz force equation
were derived.  As an exersize in tensor manipulation try the equivalent
calculation using only tensor manipulation.

\section{ Calculation. }

The starting point will be an assumed Lagrangian of the following form

\begin{align*}
\LL &= \inv{2} v^2 + \kappa A \cdot v \\
&= \inv{2} \xdot_\alpha \xdot^\alpha + \kappa A_\beta \xdot^\beta
\end{align*}

Here $v$ is the proper (four)velocity, and $A$ is the four potential.

\begin{align*}
\PD{x^\mu}{\LL} = \kappa \PD{x^\mu}{A_\beta} \xdot^\beta
\end{align*}

\begin{align*}
\PD{\xdot^\mu}{\LL}
&= \PD{\xdot^\mu}{} \left( \inv{2} g_{\alpha\beta} \xdot^\beta \xdot^\alpha \right) + \kappa \PD{\xdot^\mu}{(A_\alpha \xdot^\alpha)} \\
&= \inv{2} \left( g_{\alpha\mu} \xdot^\alpha +g_{\mu\beta} \xdot^\beta \right) + \kappa A_\mu \\
&= \xdot_\mu + \kappa A_\mu \\
\end{align*}

\begin{align*}
\PD{x^\mu}{\LL} &= \frac{d}{d\tau} \PD{\xdot^\mu}{\LL} \\
\kappa \PD{x^\mu}{A_\beta} \xdot^\beta &= \xddot_\mu + \kappa \xdot^\beta \PD{x^\beta}{A_\mu} \\
\implies \\
\xddot_\mu
&= \kappa \xdot^\beta \left( \PD{x^\mu}{A_\beta} - \PD{x^\beta}{A_\mu} \right) \\
&= \kappa \xdot^\beta \left( \partial_\mu {A_\beta} - \partial_{\beta}{A_\mu} \right) \\
\end{align*}

Setting $\kappa = q/mc$ this is the desired result

\begin{align*}
m \xddot_\mu &= (q/c) \xdot^\beta F_{\mu\beta} \\
\end{align*}

The wikipedia article \cite{wikiLorentzForce} writes this in the equivalent indexes toggled form
\begin{align*}
m \xddot^\mu &= (q/c) \xdot_\beta F^{\mu\beta}
\end{align*}

\section{ Compare for reference to GA form. }

To verify that this form is identical to familiar STA Lorentz Force equation,
reduce that equation to coordinates

\begin{align*}
\pdot \cdot \gamma_\mu &= q (F \cdot v/c) \cdot \gamma_\mu \\
m \xddot_\alpha \gamma^\alpha \cdot \gamma_\mu &= (q/2c) F_{\alpha\beta} \xdot^\nu
((\gamma^{\alpha} \wedge \gamma^{\beta}) \cdot \gamma_\nu) \cdot \gamma_\mu \\
m \xddot_\mu &=
(q/2c) F_{\alpha\beta} \xdot^\nu \left( \gamma^{\alpha} (\gamma^{\beta} \cdot \gamma_\nu) -\gamma^{\beta} (\gamma^{\alpha} \cdot \gamma_\nu) \right) \cdot \gamma_\mu \\
m \xddot_\mu &= 
(q/2c) \left( F_{\alpha\nu} \xdot^\alpha \gamma^{\alpha} - F_{\nu\beta} \xdot^\beta \gamma^{\beta} \right) \cdot \gamma_\mu \\
&= (q/c) F_{\alpha\nu} \xdot^\alpha \gamma^{\alpha} \cdot \gamma_\mu \\
\end{align*}

Which is close to the desired tensor form (off by factor of -1):
\begin{align*}
m \xddot_\mu &= (q/c) F_{\mu\nu} \xdot^\mu \\
\end{align*}

Note that \cite{doran2003gap} employs a positive time signature for the metric tensor which
should account for the difference.

\bibliographystyle{plainnat} % supposed to allow for \url use.
\bibliography{myrefs}      % expects file "myrefs.bib"

\end{document}               % End of document.
          % Oct 12/08
%
% Copyright � 2012 Peeter Joot.  All Rights Reserved.
% Licenced as described in the file LICENSE under the root directory of this GIT repository.
%

% 
% 
\chapter{Euler Lagrange Equations}
\label{chap:PJEulerLagrange}
\label{chap:eulerLagrange}
%\date{October 13, 2008.  eulerLagrange.tex}

\section{Scalar form of Euler-Lagrange equations}

\citep{lasenby1993mda} presents a multivector Lagrangian treatment.  To
preparation for understanding that I have gone 
back and derived the scalar
case myself.  As in my recent field Lagrangian derivations Feynman's
\citep{feynman1963flp} simple action procedure will be used.

Write 

\begin{align*}
\LL &= \LL(q^i, \qdot^i, \lambda) \\
q^i &= \barq^i + n^i \\
S &= \int_{\partial \lambda} \LL d\lambda
\end{align*}

Here $\barq^i$ are the desired optimal solutions, and the functions $n^i$
are all zero at the end points of the integration range $\partial \lambda$.

A first order Taylor expansion of a multivariable function
$f(a^i) = f(a^1, a^2, \cdots, a^n)$
takes the form

\begin{align*}
f(a^i + x^i) \approx f(a^i) + \sum_i (a^i + x^i) \left. \PD{x^i}{f} \right\vert_{x^i = a^i}
\end{align*}

In this case the $x^i$ take the values $q^i$, and $\qdot^i$, so the first
order Lagrangian approximation requires summation over differential contributions for both sets of terms

\begin{align}\label{eqn:euler_lagrange:linearizedLagrangian}
\LL(q^i, \qdot^i, \lambda) 
&\approx \LL(\barq^i, \barqdot^i, \lambda) 
+ \sum_i (\barq^i + n^i) \left. \PD{q^i}{\LL} \right\vert_{q^i = \barq^i}
+ \sum_i (\barqdot^i + \ndot^i) \left. \PD{\qdot^i}{\LL} \right\vert_{q^i = \barq^i}
\end{align}

%Here subscript $0$ represents evaluation of the integrals at $q_j = \barq_j$, 
%or $\qdot_j = \barqdot_j$.
%
Now form the action, and group the terms in fixed and variable sets

\begin{align*}
S &= \int \LL d\lambda \\
&\approx
\int d\lambda 
\left(
\LL(\barq^i, \barqdot^i, \lambda)
+ \sum_i \barq^i \left. \PD{q^i}{\LL} \right\vert_{q^i = \barq^i}
+ \sum_i \barqdot^i \left. \PD{\qdot^i}{\LL} \right\vert_{q^i = \barq^i}
\right) \\
&+
\mathLabelBox{
\sum_i \int d\lambda
\left(
n^i \left. \PD{q^i}{\LL} \right\vert_{q^i = \barq^i}
+\ndot^i \left. \PD{\qdot^i}{\LL} \right\vert_{q^i = \barq^i}
\right)
}{$\delta S$}
\end{align*}

For the optimal solution we want $\delta S = 0$ for all possible paths $n^i$.  Now do the integration by parts writing
$u' = \ndot^i$, and $v = \partial \LL/{\partial \qdot^i}$ 

\begin{align*}
\int u' v = u v - \int u v'
\end{align*}

The action variation is then

\begin{align*}
\delta S =
+ \sum_i \left. \left( n^i \PD{\qdot^i}{\LL} \right) \right\vert_{\partial \lambda}
+ \sum_i \int d\lambda n^i
\left(
\left. \PD{q^i}{\LL} \right\vert_{q^i = \barq^i}
-\frac{d}{d\lambda} \left. \PD{\qdot^i}{\LL} \right\vert_{q^i = \barq^i}
\right)
\end{align*}

The non-integral term is zero since by definition $n^i = 0$ on the boundary of the desired integration region, so for the
total variation to equal zero for all possible paths $n^i$ one must have

\begin{align}\label{eqn:euler_lagrange:eulerlag}
\PD{q^i}{\LL} -\frac{d}{d\lambda} \PD{\qdot^i}{\LL} = 0.
\end{align}

Evaluation of these derivatives at the optimal desired paths has been suppressed since these equations now define that path.

\subsection{Some comparison to the Goldstein approach}

\citep{goldstein1951cm} calls the quantity \eqnref{eqn:euler_lagrange:eulerlag} the functional derivative

\begin{align*}
\frac{\delta S}{\delta q^i} = \PD{q^i}{\LL} -\frac{d}{d\lambda} \PD{\qdot^i}{\LL}
\end{align*}

(with higher order derivatives if the Lagrangian has dependencies on more than generalized position and velocity terms).  Goldstein's 
approach is also harder to follow than Feynman's (Goldstein introduces a parameter $\epsilon$, writing

\begin{align}\label{eqn:euler_lagrange:epsilonvariation}
q^i = \barq^i + \epsilon n^i
\end{align}

He then takes derivatives under the integral sign for the end result.

While his approach is a bit harder to follow initially, that additional $\epsilon$ parametrization of the variation path also fits nicely with this
linearization procedure.
After the integration by parts and subsequent differentiation under integral sign nicely does the job of
discarding all the ``fixed'' $\barq^i$ contributions to the action leaving:

\begin{align*}
\frac{dS}{d\epsilon} = \int d\lambda \sum_i n^i \left. \frac{\delta S}{\delta q^i} \right\vert_{q^i = \barq^i}
\end{align*}

Introducing this idea does firm things up, eliminating some handwaving.  To obtain the extremal solution it does
make sense to set the derivative of the action equal to zero, and introducing an additional scalar variational control
in the paths from the optimal solution provides that something to take derivatives with respect to.

Goldstein also writes that this action derivative is then evaluated at $\epsilon = 0$.  This really says the same
thing as Feynman... toss all the higher order terms, since factors of epsilon will be left associated with of these.
With my initial read of Goldstein this was not the least bit clear... it was really yet another example of the classic
physics approach of solving something with a first order linear approximation.

\subsection{Noether's theorem}

Also covered in 
\citep{doran2003gap} is Noether's theorem in multivector form.  This is used
to calculate the conserved quantity the Hamiltonian for Lagrangian's with no
time dependence.  Lets try something similar for the scalar variable case,
after which the multivector case may make more sense.

At its heart Noether's theorem appears to describe change of variables in
Lagrangians.

Given a Lagrangian dependent on generalized coordinates $q^i$, and their
first order derivatives, as well as the path parameter $\lambda$.

\begin{align*}
\LL &= \LL(q^i, \qdot^i, \lambda) \\
q^i &= q^i(r^i(\lambda), \alpha)
\end{align*}

One 
example of such a change of variables would be the Galilean transformation $q^i = x^i(t) + v t$, with $\lambda = t$.

Application of the chain rule shows how to calculate the first order change
of the Lagrangian with respect to the new parameter $\alpha$.

\begin{align}\label{eqn:euler_lagrange:nothFirstChain}
\frac{d\LL}{d\alpha}
&=
\PD{q^i}{\LL} \PD{\alpha}{q^i}
+\PD{\qdot^i}{\LL} \PD{\alpha}{\qdot^i}
\end{align}

If $q^i$, and $\qdot^i$ satisfy the Euler-Lagrange equations \eqnref{eqn:euler_lagrange:eulerlag}, then this can be written

\begin{align}\label{eqn:euler_lagrange:noethchain}
\frac{d\LL}{d\alpha}
&=
\left(\frac{d}{d\lambda} \PD{\qdot^i}{\LL}\right)
\PD{\alpha}{q^i}
+\PD{\qdot^i}{\LL} \PD{\alpha}{\qdot^i}
\end{align}

If one additionally has
\begin{align*}
\PDsq{\alpha}{q^i} =\PDsq{\alpha}{\qdot^i} = 0,
\end{align*}

so that $\PDi{\alpha}{q^i}$, and $\PDi{\alpha}{\qdot^i}$ are dependent only
on $\lambda$, then \eqnref{eqn:euler_lagrange:noethchain} can be written as a total derivative

\begin{align}\label{eqn:euler_lagrange:noethTotalDer}
\frac{d\LL}{d\alpha} = \frac{d}{d\lambda} \left( \PD{\qdot^i}{\LL} \PD{\alpha}{q^i} \right)
\end{align}

If there is an $\alpha$ dependence in these derivatives a weaker total derivative statement is still possible, by evaluating the 
Lagrangian derivative and $\PDi{\alpha}{q^i}$ at some specific constant value of $\alpha$.  This is

\begin{align*}
\left.\frac{d\LL}{d\alpha}\right\vert_{\alpha = \alpha_0} = \frac{d}{d\lambda} \left( \PD{\qdot^i}{\LL} \left.\PD{\alpha}{q^i}\right\vert_{\alpha = \alpha_0} \right)
\end{align*}


\subsubsection{Hamiltonian}

Hmm, the above equations do not much like the Noether's equation in \citep{doran2003gap}.  However, in this form, we can get at the Hamiltonian statement
without any trouble.  Let us do that first, then return to Noether's

Of particular interest is when the change of variables for the generalized coordinates is dependent on the parameter $\alpha = \lambda$.
Given this type of transformation we can write
\eqnref{eqn:euler_lagrange:noethTotalDer} as

\begin{align}\label{eqn:euler_lagrange:ham1}
\frac{d\LL}{d\lambda} = \frac{d}{d\lambda} \left( \PD{\qdot^i}{\LL} \PD{\lambda}{q^i} \right)
\end{align}

For this to be valid in this $\alpha = \lambda$ case, note that the Lagrangian
itself may not be explicitly dependent on the parameter $\lambda$.  Such a dependence would mean that \eqnref{eqn:euler_lagrange:nothFirstChain}
would require an additional $\PDi{\lambda}{\LL}$ term.

The difference of the \eqnref{eqn:euler_lagrange:ham1} terms is called the Hamiltonian $H$

\begin{align*}
\frac{dH}{d\lambda} &= \frac{d}{d\lambda} \left( \PD{\qdot^i}{\LL} \qdot^i - \frac{d\LL}{d\lambda} \right) = 0 \\
\end{align*}

Or,
\begin{align*}
H &= \PD{\qdot^i}{\LL} \qdot^i - \frac{d\LL}{d\lambda},
\end{align*}

which is a conserved quantity 
when the Lagrangian has no explicit $\lambda$ dependence.

\subsubsection{Noether's take II}

Noether's theorem is about conserved quantities under symmetry transformations.  Let us revisit the attempt at derivation once more cutting down the complexity
even further, considering a transformation of a single generalized coordinate and the corresponding change to the Lagrangian under such a transformation.

Write

\begin{align*}
q &\rightarrow q' = f(q, \alpha) \\
\LL(q, \qdot, \lambda) &\rightarrow \LL' = \LL(q', \qdot', \lambda) = \LL(f, \fdot, \lambda)
\end{align*}

Now as before consider the derivative 

\begin{align}\label{eqn:euler_lagrange:noethtmp1}
\DD{\alpha}{\LL'} &= \PD{f}{\LL} \PD{\alpha}{f} + \PD{\fdot}{\LL} \PD{\alpha}{\fdot}
\end{align}

In terms of the transformed coordinates the Euler-Lagrange equations require

\begin{align*}
\PD{f}{\LL} = \frac{d}{d\lambda} \PD{\fdot}{\LL}
\end{align*}

and back-substitution into \eqnref{eqn:euler_lagrange:noethtmp1} gives
\begin{align}\label{eqn:euler_lagrange:noethtmp2}
\DD{\alpha}{\LL'} &= \frac{d}{d\lambda} \left( \PD{\fdot}{\LL} \right) \PD{\alpha}{f} + \PD{\fdot}{\LL} \PD{\alpha}{\fdot}
\end{align}

This can be written as a total derivative if 
\begin{align*}
\PD{\alpha}{\fdot} &= \frac{d}{d\lambda} \PD{\alpha}{f} \\
\PD{\alpha}{} \frac{df}{d\lambda} &= \PDD{q}{\alpha}{f} \qdot + \PDsq{\alpha}{f} \alphadot \\
\PD{\alpha}{} \left( \PD{q}{f} \qdot + \PD{\alpha}{f} \alphadot \right) &= \\
\PDD{\alpha}{q}{f} \qdot + \PDsq{\alpha}{f} \alphadot + \PD{\alpha}{f} \PD{\alpha}{\alphadot} &= \\
\end{align*}

Thus given a constraint of sufficient continuity
\begin{align*}
\PDD{\alpha}{q}{f} &= \PDD{q}{\alpha}{f} \\
\end{align*}

and also that $\alphadot$ is not a function of $\alpha$

\begin{align*}
\PD{\alpha}{\alphadot} &= 0
\end{align*}

we have from \eqnref{eqn:euler_lagrange:noethtmp2}
\begin{align*}
\DD{\alpha}{\LL'}
&= \frac{d}{d\lambda} \left( \PD{\fdot}{\LL} \PD{\alpha}{f} \right) \\
\end{align*}

Or
\begin{align}
\DD{\alpha}{\LL'}
&= \frac{d}{d\lambda} \left( \PD{\qdot'}{\LL} \PD{\alpha}{q'} \right)
\end{align}

The details of generalizing this to multiple variables are almost the same, but does not really add anything to the understanding.  This generalization is included as an appendix below for completeness, but the end result is

\begin{align}\label{eqn:euler_lagrange:noethersgeneral}
\DD{\alpha}{\LL'}
&= \frac{d}{d\lambda} \left( \sum_i 
\PD{ {{\dot{q'}}^i} }{\LL} 
%\PD{ {{\qdot}^i}' }{\LL} 
\PD{\alpha}{{q^i}'} \right)
\end{align}

In words, when the transformed Lagrangian is symmetric (not a function of $\alpha$) under coordinate transformation then this
inner quantity, a generalized momentum velocity product, is constant (conserved)

\begin{align*}
\sum_i \PD{{\dot{q'}^i}}{\LL} \PD{\alpha}{{q^i}'} = \text{constant}
\end{align*}

Transformations that leave the Lagrangian unchanged have this associated conserved quantity, which dimensionally, assuming a time parametrization, has units of
energy ($mv^2$).

FIXME: The $\PDi{\alpha}{\alphadot} = 0$ requirement is what is removed by evaluation at $\alpha = \alpha_0$.  This statement seems somewhat handwaving like.  Firm it up with an example and concrete justification.

Note that it still does not quite match the multivector result from
\citep{doran2003gap}, equation $12.10$

\begin{align*}
\left.\DD{\alpha}{\LL'}\right\vert_{\alpha=0}
&= \frac{d}{dt} \sum_{i=1}^n \left( 
\PD{\alpha}{X_i'} \conj \partial_{\Xdot_i}{\LL}
\right) \\
\end{align*}

I believe there is a missing prime there, and it should read

\begin{align*}
\left.\DD{\alpha}{\LL'}\right\vert_{\alpha=0}
&= \frac{d}{dt} \sum_{i=1}^n \left( 
\PD{\alpha}{X_i'} \conj \partial_{\Xdot_i'}{\LL}
\right) \\
\end{align*}

\section{Vector formulation of Euler-Lagrange equations}

\subsection{Simple case.  Unforced purely kinetic Lagrangian}

Before considering multivector Lagrangians, a step back to the simplest vector Lagrangian is in order

\begin{align}\label{eqn:euler_lagrange:simpleKineticLagrangian}
\LL = \inv{2}m \dot{\Bx} \cdot \dot{\Bx}
\end{align}

Writing $\Bx(\lambda) = \overbar{\Bx} + \epsilon \Bn$, and using the variational technique directly the equation of motion for this unforced path should follow directly
in vector form

\begin{align*}
S = \int d\lambda \inv{2} m \dot{\overbar{\Bx}}^2 + \int m d\lambda \epsilon \dot{\overbar{\Bx}} \cdot \dot{\Bn} + \int d\lambda \inv{2} m \epsilon^2 \dot{\overbar{\Bn}}^2
\end{align*}

Integration by parts operating directly on the vector function we have

\begin{align*}
\left.\frac{d S}{d\epsilon}\right\vert_{\epsilon=0} 
&= \left.{m \dot{\overbar{\Bx}} \cdot {\Bn}}\right\vert_{\partial \lambda} - \int m d\lambda \ddot{\overbar{\Bx}} \cdot {\Bn} \\
&= - \int m d\lambda \ddot{\overbar{\Bx}} \cdot {\Bn} \\
\end{align*}

Introducing shorthand $\delta S/\delta \Bx$, for a vector functional derivative, we have
\begin{align*}
\left.\frac{d S}{d\epsilon}\right\vert_{\epsilon=0} &= \int d\lambda \Bn \cdot \frac{\delta S}{\delta \Bx},
\end{align*}

where the extremal condition is
\begin{align*}
\frac{\delta S}{\delta \Bx} = - m \ddot{\overbar{\Bx}} = 0.
\end{align*}

Here the expected and desired Euler Lagrange equation for the Lagrangian (constant velocity in some direction dependent on initial conditions) is arrived at directly in vector form without dropping down to coordinates and reassembling them to get back the vector expression.

\subsection{Position and velocity gradients in the configuration space}

Having tackled the simplest case, to generalize this we need a construct to do first order Taylor series expansion in the neighborhood of a vector
position.  The (multivector) gradient is the obvious candidate operator to do the job.
Before going down that road consider the scalar Lagrangian case once more, where we will see that it is natural to define position and velocity gradients
in the configuration space.  It will also be observed that the chain rule essentially motivates the initially somewhat odd seeming reciprocal basis
used to express the gradient when operating in a non-orthonormal frame.

In \eqnref{eqn:euler_lagrange:linearizedLagrangian}, the linear differential increment in the neighborhood of the optimal solution had the form

\begin{align}\label{eqn:euler_lagrange:gradientMotivator}
\Delta \LL &=
+ \sum_i (\barq^i + n^i) \left. \PD{q^i}{\LL} \right\vert_{q^i = \barq^i}
+ \sum_i (\barqdot^i + \ndot^i) \left. \PD{\qdot^i}{\LL} \right\vert_{q^i = \barq^i}
\end{align}

If one defines a configuration space position and velocity gradients respectively as

\begin{align*}
\grad_{\Bq} &= \left(\PD{q^1}{}, \PD{q^2}{}, \cdots, \PD{q^n}{}\right) = f_k \PD{q^k}{} \\
\grad_{\dot{\Bq}} &= \left(\PD{\qdot^1}{}, \PD{\qdot^2}{}, \cdots, \PD{\qdot^n}{}\right) = f_k \PD{\qdot^k}{}
\end{align*}

and forms a configuration space vector with respect to some linearly independent, but not necessarily orthonormal, basis

\begin{align*}
\Bq = q^i e_i
\end{align*}

then the chain rule dictates the relationship between the configuration vector basis and the basis with which the gradient must be expressed.  In
particular, if we wish to write \eqnref{eqn:euler_lagrange:gradientMotivator} in terms of the configuration space gradients

\begin{align*}
\Delta \LL &=
(\overbar{\Bq} + \Bn) \cdot \left.\grad_\Bq \LL \right\vert_{\Bq = \overbar{\Bq}}
+ (\dot{\overbar{\Bq}} + \dot{\Bn}) \cdot \left.\grad_{\dot{\Bq}} \LL \right\vert_{\dot{\Bq} = \dot{\overbar{\Bq}}}
\end{align*}

Then we must have a reciprocal relationship between the basis vector for the configuration space vectors $e_i$, and the corresponding vectors
from which the gradient was formed

\begin{align*}
e_i \cdot f_j &= \delta_{i j} \\
\implies \\
f_j = e^j
\end{align*}

This gives us the position and velocity gradients in the configuration space

\begin{align}
\grad_{\Bq} &= e^k \PD{q^k}{} \\
\grad_{\dot{\Bq}} &= e^k \PD{\qdot^k}{}.
\end{align}

Note also that the size of this configuration space does not have to be the same space as the problem.  With this definitions completion of the integration
by parts yields the Euler-Lagrange equations in a hybrid configuration space vector form

\begin{align}
\grad_{\Bq} \LL = \frac{d}{d\lambda} \grad_{\dot{\Bq}} \LL
\end{align}

When the configuration space equals the geometrical space being operated in (ie: generalized coordinates are regular old coordinates), this 
provides a nice explanation for why we must have the funny pairing of upper index coordinates in the partials of the gradient and reciprocal frame
vectors multiplying all these partials.  Contrast to a text like \citep{doran2003gap} where the gradient (and spacetime gradient) are defined in this
fashion instead, and one gradually sees that this does in fact work out.

That said, the negative side of this vector notation is that 
it obscures somewhat the Euler-Lagrange equations, which are not terribly complicated to begin with.  However, since this appears to be the form
of the multivector form of the Euler-Lagrange equations it is likely worthwhile to see how this also expresses the simpler familiar scalar case too.

\section{Example applications of Noether's theorem}

Linear translation and rotational translation appear to be the usual first example
applications.  \citep{TongDynamics} does this, as does the wikipedia article.
Reading about those without actually working through it 
myself never made complete sense (esp. want to do the angular momentum
example).

Noether's theorem is not really required to see that in the case of unforced motion 
\eqnref{eqn:euler_lagrange:simpleKineticLagrangian}, translation of coordinates $\Bx \rightarrow \Bx + \Ba$
will not change the equation of motion.  This is the conservation of linear momentum result
so familiar from high school physics.

\subsection{Angular momentum in a radial potential}

The conservation of angular momentum case is more interesting.

Suppose that one has a radial potential applied to a point particle

\begin{align*}
\LL = \inv{2}m \dot{\Bx}^2 - \phi(\Abs{\Bx}^k)
\end{align*}

and apply a rotational transformation to the coordinates $\Bx \rightarrow \exp(i\theta/2) \Bx \exp(-i\theta/2)$.

Provided that this is a fixed rotation with $i$, and $\theta$ constant (not functions of time), the transformed squared velocity is:

\begin{align*}
\dot{\Bx}' \cdot \dot{\Bx}' 
&= \gpgradezero{ \exp(i\theta/2) \dot{\Bx} \exp(-i\theta/2) \exp(i\theta/2) \dot{\Bx} \exp(-i\theta/2) } \\
&= \gpgradezero{ \exp(i\theta/2) \dot{\Bx} \dot{\Bx} \exp(-i\theta/2) } \\
&= \dot{\Bx}^2 \gpgradezero{ \exp(i\theta/2) \exp(-i\theta/2) } \\
&= \dot{\Bx}^2 \\
\end{align*}

Since $\Abs{\Bx'} = \Abs{\Bx}$ the transformed Lagrangian is unchanged by any rotation of coordinates.

Noether's equation \eqnref{eqn:euler_lagrange:noethersgeneral} takes the form

\begin{align*}
\PD{\theta}{\LL'} 
&= \frac{d}{dt} \left( \PD{\theta}{\Bx'} \cdot \spacegrad_{\Bv'}{\LL} \right)
\end{align*}

Here the configuration space gradient is used to express the chain rule terms, picking the \R{3} standard basis vectors
to express that gradient.

The velocity term can be expanded as

\begin{align*}
\PD{\theta}{\Bx'} 
&= \PD{\theta}{} \left( \exp(i\theta/2) \Bx \exp(-i\theta/2) \right) \\
&= \inv{2}(i \Bx' - \Bx'i) \\
&= i \cdot \Bx' \\
\end{align*}

The transformed conjugate momentum is 

\begin{align*}
\spacegrad_{\Bv'} \inv{2} m {\Bv'}^2 = m \Bv' = \Bp'
\end{align*}

so the conserved quantity is

\begin{align*}
(i \cdot \Bx') \cdot \Bp' = \text{constant}
\end{align*}

Temporarily expressing the bivector for the rotational plane in terms of a dual relationship, $i = I \Bn$, where $\Bn$ is a unit normal to the plane we have

\begin{align*}
(i \cdot \Bx') \cdot \Bp' 
&= ((I \Bn) \cdot \Bx') \cdot \Bp' \\
&= \inv{2} (I \Bn \Bx' - \Bx' I \Bn) \cdot \Bp' \\
&= \inv{2} \gpgradezero{I (\Bn \Bx' - \Bx'\Bn) \Bp'} \\
&= \inv{2} \gpgradezero{I \Bn \Bx'\Bp} - \gpgradezero{ I \Bn \Bp' \Bx' } \\
&= \inv{2} \left( \gpgradezero{i (\Bx' \wedge \Bp')} - \gpgradezero{ i (\Bp' \wedge \Bx') }\right) \\
&= i \cdot (\Bx' \wedge \Bp') \\
\end{align*}

Since $i$ is a constant bivector we have angular momentum (dropping primes), by virtue of Lagrangian transformational symmetry and Noether's theorem the angular momentum

\begin{align}\label{eqn:euler_lagrange:angularmomentum}
\Bx \wedge \Bp = \text{constant},
\end{align}

is a constant of motion for a point particle Lagrangian in a radial potential field.

This is typically expressed in terms of the dual relationship using cross products

\begin{align*}
\Bx \cross \Bp = \text{constant}.
\end{align*}

Also observe the time derivative of the angular momentum in \eqnref{eqn:euler_lagrange:angularmomentum}

\begin{align*}
\frac{d}{dt} (\Bx \wedge \Bp) 
&= \Bp/m \wedge \Bp + \Bx \wedge \dot{\Bp} \\
&= \Bx \wedge \dot{\Bp} \\
&= 0
\end{align*}

Which says that the torque on a particle in a radial potential is zero.  This finally supplies the rational for texts like
\citep{lewis1965mbp}, which while implicitly talking about motion in a (radial) gravitational potential, says something to the effect 
of ``in the absence of external torques the angular momentum is conserved''!

What other more general non-radial potentials, if any, allow for this conservation statement?
I had guess that something like
the Lorentz force with velocity dependence in the potential will explicitly not have this conservation of angular momentum.
\citep{TongDynamics} and
\citep{goldstein1951cm} both cover Lagrangian transformation, and specifically cover this angular momentum issue, but 
blundering through it myself as done here was required to really see where it was coming from and to apply the idea.
%with my initial read of both I hhad not built up enough
%comfort with the ideas to be able to understand nor apply the ideas myself.

\subsection{Hamiltonian}

Consider a general kinetic form and a possibly velocity dependent potential

\begin{align*}
\LL &= K - \phi = \inv{2} \sum_{ij} g_{ij} \qdot^i \qdot^j - \phi
\end{align*}

and form the Hamiltonian.  First calculate

\begin{align*}
\PD{\qdot^i}{\LL} &= m \sum_{j} g_{ij} \qdot^j - \PD{\qdot^i}{\phi} \\
\implies \\
\sum_i \qdot^i \PD{\qdot^i}{\LL} 
&= m \sum_{ij} g_{ij} \qdot^i \qdot^j - \sum_i \qdot^i \PD{\qdot^i}{\phi} \\
&= 2 K - \sum_i \qdot^i \PD{\qdot^i}{\phi} \\
%&= 2 K - \Bv \cdot \spacegrad_{\Bv} {\phi} \\
\end{align*}

So, the Hamiltonian is
\begin{align*}
H &= K - \sum_i \qdot^i \PD{\qdot^i}{\phi} + \phi \\
\end{align*}

For the less general case where $\Bv^2 = g_{ij} \qdot^i \qdot^j$, this is
\begin{align*}
H &= K - \Bv \cdot \spacegrad_{\Bv} {\phi} + \phi \\
\end{align*}

a conserved quantity with respect to the time derivative.

Similarly, for squared proper velocity $v^2 = g_{ij} \qdot^i \qdot^j$, and derivatives with respect to proper time

\begin{align*}
H &= K - v \cdot \grad_{v} {\phi} + \phi \\
\end{align*}

is conserved with respect to proper time.

As an example, consider the Lorentz force Lagrangian.  For 
proper velocity $v$, four potential $A$, and positive time metric signature $(\gamma_0)^2 = 1$, the Lorentz force Lagrangian is

\begin{align}\label{eqn:euler_lagrange:lorentzforce}
\LL = \inv{2}m v \cdot v + q A \cdot v/c
\end{align}

We therefore have

\begin{align*}
0 
&= \frac{d}{d\tau}
\left(
\inv{2} m v^2 + v \cdot \grad_{v} (q A \cdot v/c) - q A \cdot v/c 
\right) \\
\end{align*}

Or

\begin{align*}
\inv{2} m v^2 + v \cdot \grad_{v} (q A \cdot v/c) - q A \cdot v/c = \kappa
\end{align*}

Where $\kappa$ is some constant.  Since $\grad_v A^{\mu} = 0$, we have $\grad_v A\cdot v = A$, and

\begin{align*}
\kappa 
&= \inv{2} m v^2 + v \cdot (q A/c) - q A \cdot v/c \\
&= \inv{2} m v^2 \\
\end{align*}

At a glance this does not look terribly interesting, since 
by definition of proper time we already know that $\inv{2} mv^2 = \inv{2}mc^2$ is a constant.

However, suppose that one did not assume proper time to start with, and instead considered
an arbitrarily parametrized coordinate worldline and their corresponding solutions 

\begin{align*}
x &= x(\lambda) \\
\LL &= \inv{2}m \frac{dx}{d\lambda} \cdot \frac{dx}{d\lambda} + q A \cdot \frac{dx}{d\lambda}/c \\
\PD{\lambda}{\LL} &= \frac{d}{d\lambda} \PD{\lambda}{\LL}
\end{align*}

The Hamiltonian conservation with respect to this parametrization then implies

\begin{align*}
\frac{d}{d\lambda} \left( \inv{2} m \frac{dx}{d\lambda} \cdot \frac{dx}{d\lambda} \right) &= 0
\end{align*}

So that, independent of the parametrization, the quantity $\inv{2} m \frac{dx}{d\lambda} \cdot \frac{dx}{d\lambda}$ is a constant.  This then follows as a consequence of Noether's theorem instead of by definition.
Proper time then becomes that particular worldline parametrization $\lambda = \tau$ such that 
$\inv{2} m \frac{dx}{d\tau} \cdot \frac{dx}{d\tau} = \inv{2} m c^2$.

\subsection{Covariant Lorentz force Lagrangian}

The Hamiltonian was used above to extract $v^2$ invariance from the Lorentz force Lagrangian under changes of proper time.  The next obvious
Noether's application is for a Lorentz transformation of the interaction Lagrangian.  This was interesting enough seeming in its own right to 
treat separately and has been moved to \chapcite{PJLorentzTxInteraction}.

\subsection{Vector Lorentz force Lagrangian}

FIXME: Try this with $\BA \cdot \Bv$ form of the Lagrangian and rotation... cross product terms should result.

\subsection{An example where the transformation has to be evaluated at fixed point}

FIXME: find an example of this and calculate with it.

\subsection{Comparison to cyclic coordinates}

FIXME: Also calculate with some examples where cyclic
coordinates are discovered by actually computing the Euler-Lagrange equations
... see how to observed this directly from the Lagrangian itself under transformation without actually evaluating the equations (despite the fact that
this is simple for the cyclic case).

\section{Appendix}

\subsection{Noether's equation derivation, multivariable case}

Employing a couple judicious regular expressions starting from the text for the single variable treatment, plus some minor summation sign addition does the job.

\begin{align*}
q^i &\rightarrow {q^i}' = f^i(q^i, \alpha) \\
\LL(q^i, \qdot^i, \lambda) &\rightarrow \LL' = \LL({q^i}', {\dot{q'}^i}, \lambda) = \LL(f^i, \fdot^i, \lambda)
\end{align*}

Now as before consider the derivative 

\begin{align}\label{eqn:euler_lagrange:gnoethtmp1}
\DD{\alpha}{\LL'} &= \sum_i \PD{f^i}{\LL} \PD{\alpha}{f^i} + \PD{\fdot^i}{\LL} \PD{\alpha}{\fdot^i}
\end{align}

In terms of the transformed coordinates the Euler-Lagrange equations require

\begin{align*}
\PD{f^i}{\LL} = \frac{d}{d\lambda} \PD{\fdot^i}{\LL}
\end{align*}

and backsubstitution into \eqnref{eqn:euler_lagrange:gnoethtmp1} gives
\begin{align}\label{eqn:euler_lagrange:gnoethtmp2}
\DD{\alpha}{\LL'} &= \sum_i \frac{d}{d\lambda} \left( \PD{\fdot^i}{\LL} \right) \PD{\alpha}{f^i} + \PD{\fdot^i}{\LL} \PD{\alpha}{\fdot^i}
\end{align}

This can be written as a total derivative if 
\begin{align*}
\PD{\alpha}{\fdot^i} &= \frac{d}{d\lambda} \PD{\alpha}{f^i} \\
\PD{\alpha}{} \frac{df}{d\lambda} &= \sum_j \PDD{q^j}{\alpha}{f^i} \qdot^j + \PDsq{\alpha}{f^i} \alphadot \\
\PD{\alpha}{} \left( \sum_j \PD{q^j}{f^i} \qdot^j + \PD{\alpha}{f^i} \alphadot \right) &= \\
\sum_j \PDD{\alpha}{q^j}{f^i} \qdot^j + \PDsq{\alpha}{f^i} \alphadot + \PD{\alpha}{f^i} \PD{\alpha}{\alphadot} &= \\
\end{align*}

Thus given constraints of sufficient continuity
\begin{align*}
\PDD{\alpha}{q^j}{f^i} &= \PDD{q^j}{\alpha}{f^i} \\
\end{align*}

and also that $\alphadot$ is not a function of $\alpha$

\begin{align*}
\PD{\alpha}{\alphadot} &= 0
\end{align*}

we have from \eqnref{eqn:euler_lagrange:gnoethtmp2}
\begin{align*}
\DD{\alpha}{\LL'}
&= \frac{d}{d\lambda} \left( \sum_i \PD{\fdot^i}{\LL} \PD{\alpha}{f^i} \right) \\
\end{align*}

QED.
                     % Oct 13/08
%
% Copyright � 2012 Peeter Joot.  All Rights Reserved.
% Licenced as described in the file LICENSE under the root directory of this GIT repository.
%

% 
% 
\chapter{Lorentz Invariance of Maxwell Lagrangian}
\label{chap:PJBoostMaxwell}
\label{chap:boostMaxwellLagrangian}
%\date{October 19, 2008.  boostMaxwellLagrangian.tex}

\section{Working in multivector form}
\subsection{Application of Lorentz boost to the field Lagrangian}

The multivector form of the field Lagrangian is

\begin{align}\label{eqn:boostMLag:lagrangian}
\LL &= \kappa (\grad \wedge A)^2 + A \cdot J \\
\kappa &= -\frac{\epsilon_0 c}{2}
\end{align}

Write the boosting transformation on a four vector in exponential form

\begin{align*}
L(X) = \exp( \alpha \acap/2 ) X \exp( -\alpha \acap/2 ) = \Lambda X \Lambda^\dagger
\end{align*}

where $\acap = a^i \gamma_i \wedge \gamma_0$ is any unit spacetime bivector, and $\alpha$ represents the rapidity angle.

Consider first the transformation of the interaction term with $A' = L(A)$, and $J' = L(J)$

\begin{align*}
A' \cdot J'
&= \gpgradezero{L(A) L(J)} \\
&= \gpgradezero{ \Lambda A \Lambda^\dagger \Lambda J \Lambda^\dagger } \\
&= \gpgradezero{ \Lambda A J \Lambda^\dagger } \\
&= \gpgradezero{ \Lambda^\dagger \Lambda A J } \\
&= \gpgradezero{ A J } \\
&= A \cdot J \\
\end{align*}

Now consider the boost applied to the field bivector $F = \BE + Ic\BB = \grad \wedge A$, by boosting both the gradient and the potential

\begin{align*}
\grad' \wedge A'
&= L(\grad) \wedge L(A) \\
&= \Lambda \grad) \wedge L(A) \\
&= (\Lambda \grad \Lambda^\dagger ) \wedge (\Lambda A \Lambda^\dagger) \\
&= \inv{2}\left( (\Lambda \grad \Lambda^\dagger )  (\Lambda A \Lambda^\dagger) - (\Lambda A \Lambda^\dagger )  (\Lambda \grad \Lambda^\dagger) \right) \\
&= \inv{2}\left( \Lambda \grad A \Lambda^\dagger - \Lambda A \grad \Lambda^\dagger \right) \\
&= \Lambda (\grad \wedge A) \Lambda^\dagger \\
\end{align*}

The boosted squared field bivector in the Lagrangian is thus

\begin{align*}
(\grad' \wedge A' )^2
&= \Lambda (\grad \wedge A)^2 \Lambda^\dagger \\
&= \Lambda (\BE + Ic\BB)^2 \Lambda^\dagger \\
&= \Lambda ( (\BE^2 - c^2\BB^2) + 2 I c \BE \cdot \BB ) \Lambda^\dagger \\
&= ( (\BE^2 - c^2\BB^2) \Lambda \Lambda^\dagger + 2 (\Lambda I \Lambda^\dagger) c \BE \cdot \BB ) \\
&= ( (\BE^2 - c^2\BB^2) + 2 I \Lambda \Lambda^\dagger c \BE \cdot \BB ) \\
&= ( (\BE^2 - c^2\BB^2) + 2 I c \BE \cdot \BB ) \\
&= (\BE + Ic\BB)^2 \\
&= (\grad \wedge A )^2
\end{align*}

The commutation of the pseudoscalar $I$ with the boost exponential $\Lambda = \exp( \alpha \acap/2 ) = \cosh(\alpha/2) + \acap\sinh(\alpha/2)$ is possible
since $I$
anticommutes with all four vectors and thus commutes with bivectors, such as $\acap$.  $I$ also necessarily commutes with the scalar
components of this exponential, and thus commutes with any even grade multivector.

Putting all the pieces together this shows that the Lagrangian in its entirety is a Lorentz invariant

\begin{align*}
\LL' &= \kappa (\grad' \wedge A')^2 + A' \cdot J' = \kappa (\grad \wedge A)^2 + A \cdot J = \LL
\end{align*}

FIXME: what is the conserved quantity associated with this?  There should be one according to Noether's theorem?  Is it the gauge condition $\grad \cdot A = 0$?

\subsubsection{Maxwell equation invariance}

Somewhat related, having calculated the Lorentz transform of $F = \grad \wedge A$, is an aside showing that the Maxwell equation
is unsurprisingly also is a Lorentz invariant.

\begin{align*}
\grad' (\grad' \wedge A') &= J' \\
\Lambda \grad \Lambda^\dagger \Lambda (\grad \wedge A) \Lambda^\dagger &= \Lambda J \Lambda^\dagger \\
\Lambda \grad (\grad \wedge A) \Lambda^\dagger &= \Lambda J \Lambda^\dagger \\
\end{align*}

Pre and post multiplying with $\Lambda^\dagger$, and $\Lambda$ respectively returns the unboosted equation

\begin{align*}
\grad (\grad \wedge A) &= J
\end{align*}

\subsection{Lorentz boost applied to the Lorentz force Lagrangian}

Next interesting case is the Lorentz force, which for a time positive metric 
signature is:

\begin{align*}
\LL &= q A \cdot v/c + \inv{2} m v \cdot v
\end{align*}

The boost invariance of the $A \cdot J$ dot product demonstrated above demonstrates the general invariance property for any four vector dot product, and this
Lagrangian has nothing but dot products in it.  It thus follows directly that the Lorentz force Lagrangian is also a Lorentz invariant.

\section{Repeat in tensor form}

Now, I can follow the above, but presented with the same sort of calculation in tensor form I am hopeless to understand it.  To attempt translating this
into tensor form, it appears the first step is putting the Lorentz transform itself into tensor or matrix form.

\subsection{Translating versors to matrix form}

To get the feeling for how this will work, assume $\acap = \sigma_1$, so that the boost is along the x-axis.  In that case we have

\begin{align*}
L(X)
%&= (\cosh( \alpha/2 ) + \gamma_{10} \sinh( \alpha/2 )) ( x^0 \gamma_0 + x^1 \gamma_1 + x^2 \gamma_2 + x^3 \gamma_3 ) (\cosh( \alpha/2 ) + \gamma_{01} \sinh( \alpha/2 )) \\
&= (\cosh( \alpha/2 ) + \gamma_{10} \sinh( \alpha/2 )) x^\mu \gamma_\mu (\cosh( \alpha/2 ) + \gamma_{01} \sinh( \alpha/2 )) \\
\end{align*}

Writing $C = \cosh(\alpha/2)$, and $S = \sinh(\alpha/2)$, and observing that the exponentials commute with the $\gamma_2$, and $\gamma_3$ directions so the exponential
action on those directions cancel.

\begin{align*}
L(X)
&= x^2 \gamma_2 + x^3 \gamma_3 + (C + \gamma_{10} S) ( x^0 \gamma_0 + x^1 \gamma_1 ) (C + \gamma_{01} S) \\
\end{align*}

Expanding just the non-perpendicular parts of the above
\begin{align*}
&(C + \gamma_{10} S) ( x^0 \gamma_0 + x^1 \gamma_1 ) (C + \gamma_{01} S) \\
&=
x^0 (C^2 \gamma_0 + \gamma_{10001} S^2) + x^0 S C (\gamma_{001} + \gamma_{100})
+x^1 (C^2 \gamma_1 + \gamma_{10101} S^2) + x^1 S C (\gamma_{101} + \gamma_{101}) \\
&=
x^0 (C^2 \gamma_0 - \gamma_{01100} S^2) + 2 x^0 S C \gamma_{001} 
+x^1 (C^2 \gamma_1 - \gamma_{11001} S^2) - 2 x^1 S C \gamma_{011} \\
&= (x^0 \gamma_0 + x^1 \gamma_1) (C^2 + S^2) + 2 (\gamma_0)^2 S C (x^0 \gamma_{1} + x^1 \gamma_{0}) \\
&= (x^0 \gamma_0 + x^1 \gamma_1) \cosh(\alpha) + (\gamma_0)^2 \sinh(\alpha) (x^0 \gamma_{1} + x^1 \gamma_{0}) \\
&= 
\gamma_0 ( x^0 \cosh(\alpha) + x^1 \sinh((\gamma_0)^2 \alpha) )
+\gamma_1 ( x^1 \cosh(\alpha) + x^0 \sinh((\gamma_0)^2 \alpha) ) \\
\end{align*}


In matrix form the complete transformation is thus

\begin{align*}
{\begin{bmatrix}
x^0 \\
x^1 \\
x^2 \\
x^3 \\
\end{bmatrix}}'
&=
\begin{bmatrix}
\cosh(\alpha(\gamma_0)^2) & \sinh(\alpha(\gamma_0)^2) & 0 & 0 \\
\sinh(\alpha(\gamma_0)^2) & \cosh(\alpha(\gamma_0)^2) & 0 & 0 \\
0 & 0 & 1 & 0 \\
0 & 0 & 0 & 1 \\
\end{bmatrix}
\begin{bmatrix}
x^0 \\
x^1 \\
x^2 \\
x^3 \\
\end{bmatrix} \\
&=
\cosh(\alpha(\gamma_0)^2) 
\begin{bmatrix}
1                         & \tanh(\alpha(\gamma_0)^2) & 0 & 0 \\
\tanh(\alpha(\gamma_0)^2) & 1                         & 0 & 0 \\
0 & 0 & 1 & 0 \\
0 & 0 & 0 & 1 \\
\end{bmatrix}
\begin{bmatrix}
x^0 \\
x^1 \\
x^2 \\
x^3 \\
\end{bmatrix}
\end{align*}

This supplies the specific meaning for the $\alpha$ factor in the exponential form, namely:

\begin{align*}
\alpha
&= -\tanh^{-1}(\beta (\gamma_0)^2) \\
&= -\tanh^{-1}(\Abs{\Bv}/c (\gamma_0)^2)
\end{align*}

Or
\begin{align*}
\alpha \acap
&= -\tanh^{-1}(\acap \Abs{\Bv}/c (\gamma_0)^2) \\
&= -\tanh^{-1}(\Bv/c (\gamma_0)^2) \\
\end{align*}

Putting this back into the original Lorentz boost equation to tidy it up, and 
writing $\tanh(\BA) = \Bv/c$, the Lorentz boost is 

\begin{align*}
L(X) &= 
\left\{
\begin{array}{l l}
\exp(-\BA/2) X \exp(\BA/2) & \quad \mbox{for $(\gamma_0)^2 = 1$} \\
\exp(\BA/2) X \exp(-\BA/2) & \quad \mbox{for $(\gamma_0)^2 = -1$} \\
\end{array} \right. \\
\end{align*}

Both of the metric signature options are indicated here for future reference and comparison with results using the alternate signature.

\subsubsection{Revisit the expansion to matrix form above}

Looking back, multiplying out all the half angle terms as done above is this is the long dumb hard way to do it.
A more sensible way would be to note that $\exp(\alpha\acap/2)$ anticommutes with both $\gamma_0$ and $\gamma_1$ thus

\begin{align*}
\exp( \alpha \acap/2 ) (x^0\gamma_0 + x^1\gamma_1) \exp( -\alpha \acap/2 )
&= \exp( \alpha \acap ) (x^0\gamma_0 + x^1\gamma_1) \\
&= (\cosh( \alpha ) + \acap \sinh(\alpha)) (x^0\gamma_0 + x^1\gamma_1) \\
\end{align*}

The matrix form thus follows directly.

\section{Translating versors tensor form}

After this temporary digression back to the multivector form of the Lorentz transformation lets dispose of the specifics of the boost direction and magnitude, and also the metric
signature.  Instead encode all of these in a single versor variable $\Lambda$, again writing

\begin{align}
L(X) &= \Lambda X \Lambda^\dagger
\end{align}

\subsection{Expressing vector Lorentz transform in tensor form}

What is the general way to encode this linear transformation in tensor/matrix form?  The transformed vector is just that a vector, and thus can be written in terms of 
coordinates for some basis

\begin{align*}
L(X) 
&= (L(X) \cdot e^\mu) e_\mu \\
&= ((\Lambda (x^\nu \gamma_\nu) \Lambda^\dagger) \cdot e^\mu) e_\mu \\
&= x^\nu ((\Lambda \gamma_\nu \Lambda^\dagger) \cdot e^\mu) e_\mu \\
\end{align*}

The inner term is just the tensor that we want.  Write

\begin{align*}
{\Lambda_{\nu}}^{\mu} &= (\Lambda \gamma_\nu \Lambda^\dagger) \cdot e^\mu \\
{\Lambda^{\nu}}_{\mu} &= (\Lambda \gamma^\nu \Lambda^\dagger) \cdot e_\mu
\end{align*}

for 
\begin{align*}
L(X) &= x^\nu {\Lambda_{\nu}}^{\mu} e_\mu \\
     &= x_\nu {\Lambda^{\nu}}_{\mu} e^\mu \\
\end{align*}

Completely eliminating the basis, working in just the coordinates $X = {x'}^\mu e_\mu = {x'}_\mu e^\mu$ this is

\begin{align}
{x'}^\mu &= x^\nu {\Lambda_{\nu}}^{\mu} \\
{x'}_\mu &= x_\nu {\Lambda^{\nu}}_{\mu}
\end{align}

Now, in particular, having observed that the dot product is a Lorentz invariant this should supply the index manipulation rule for operating
with the Lorentz boost tensor in a dot product context.

Write
\begin{align*}
L(X) \cdot L(Y)
&= (x^\nu {\Lambda_{\nu}}^{\mu} e_\mu) \cdot (y_\alpha {\Lambda^{\alpha}}_{\beta} e^\beta) \\
&= x^\nu y_\alpha {\Lambda_{\nu}}^{\mu} {\Lambda^{\alpha}}_{\beta} e_\mu \cdot e^\beta \\
&= x^\nu y_\alpha {\Lambda_{\nu}}^{\mu} {\Lambda^{\alpha}}_{\mu} \\
\end{align*}

Since this equals $x^\nu y_\nu$, the tensor rule must therefore be
\begin{align}\label{eqn:boostMLag:boostinverse}
{\Lambda_{\mu}}^{\sigma} {\Lambda^{\nu}}_{\sigma} = {\delta_\mu}^\nu
\end{align}

After a somewhat long path, the core idea behind the Lorentz boost tensor
is that it is the ``matrix'' of a linear transformation that leaves
the four vector dot product unchanged.  There is no need to consider
any Clifford algebra formulations to express just that idea.

\subsection{Misc notes}

FIXME: To complete
the expression of this in tensor form enumerating exactly how to express
the dot product in tensor form would also be reasonable.  ie: how to compute
the reciprocal coordinates without describing the basis.  Doing this
will introduce the metric tensor into the mix.

Looks like the result \eqnref{eqn:boostMLag:boostinverse}
is consistent with \citep{MinahanTensors} and
that doc starts making a bit more sense now.  I do see that
he uses primes to distinguish the boost tensor from its inverse (using
the inverse tensor (primed index down) to transform the covariant (down)
coordinates).  Is there a convention for keeping free vs. varied indices
close to the body of the operator?  For the boost tensor he puts the free
index closer to $\Lambda$, but for the inverse tensor for a covariant
coordinate transformation puts the free index further out?

This also appears to be notational consistent with \citep{SpenceTensors}.

\subsection{Expressing bivector Lorentz transform in tensor form}

Having translated a vector Lorentz transform into tensor form, the next step is to do the same for
a bivector.  In particular for the field bivector $F = \grad \wedge A$.

Write

\begin{align*}
\grad' &= \Lambda \gamma_\mu \partial^\mu \Lambda^\dagger \\
A' &= \Lambda A^\nu \gamma_\nu \Lambda^\dagger \\
\end{align*}
\begin{align*}
\grad' \cdot e^\beta &= (\Lambda \gamma_\mu \Lambda^\dagger) \cdot e^\beta \partial^\mu = {\Lambda_\mu}^\beta \partial^\mu \\
A' \cdot e^\beta &= (\Lambda \gamma_\nu \Lambda^\dagger) \cdot e^\beta A^\nu = {\Lambda_\nu}^\beta A^\nu \\
\end{align*}

Then the transformed bivector is
\begin{align*}
F' = \grad' \wedge A'
&= ((\grad' \cdot e^\alpha) e_\alpha) \wedge ((A' \cdot e^\beta) e_\beta) \\
&= (e_\alpha \wedge e_\beta) {\Lambda_\mu}^\alpha {\Lambda_\nu}^\beta \partial^\mu A^\nu \\
\end{align*}

and finally the transformed tensor is thus

\begin{align*}
{F^{ab}}'
&= F' \cdot (e^b \wedge e^a) \\
&= (e_\alpha \wedge e_\beta) \cdot (e^b \wedge e^a) {\Lambda_\mu}^\alpha {\Lambda_\nu}^\beta \partial^\mu A^\nu \\
&= ( {\delta_\alpha}^a {\delta_\beta}^b - {\delta_\beta}^a {\delta_\alpha}^b ) {\Lambda_\mu}^\alpha {\Lambda_\nu}^\beta \partial^\mu A^\nu \\
&= {\Lambda_\mu}^a {\Lambda_\nu}^b \partial^\mu A^\nu
-{\Lambda_\mu}^b {\Lambda_\nu}^a \partial^\mu A^\nu \\
&= {\Lambda_\mu}^a {\Lambda_\nu}^b ( \partial^\mu A^\nu -\partial^\nu A^\mu ) \\
\end{align*}

Which gives the final transformation rule for the field bivector in tensor form

\begin{align}
{F^{ab}}' = {\Lambda_\mu}^a {\Lambda_\nu}^b F^{\mu\nu}
\end{align}

Returning to the original problem of field Lagrangian invariance, we want to examine how ${F^{ab}}' {F_{ab}}'$ transforms.  That is
\begin{dmath}
{F^{ab}}' {F_{ab}}'
= {\Lambda_\mu}^a {\Lambda_\nu}^b F^{\mu\nu} {\Lambda^\alpha}_a {\Lambda^\beta}_b F_{\alpha\beta} 
= ({\Lambda_\mu}^a {\Lambda^\alpha}_a) ({\Lambda_\nu}^b {\Lambda^\beta}_b) F^{\mu\nu} F_{\alpha\beta} 
= {\delta_\mu}^\alpha {\delta_\nu}^\beta F^{\mu\nu} F_{\alpha\beta} 
= F^{\mu\nu} F_{\mu\nu}
\end{dmath}

which is the desired result.  Since the dot product remainder of the Lagrangian \eqnref{eqn:boostMLag:lagrangian}
has already been shown to be Lorentz invariant this is sufficient to prove the Lagrangian boost or rotational invariance using tensor algebra.

Working this way is fairly compact and efficient, and required a few less steps than the multivector equivalent.  To compare apples to applies,
for the algebraic tools, it should be noted that if only the 
scalar part of $(\grad \wedge A)^2$ was considered as implicitly done in the tensor argument above, the multivector approach would likely have been as 
compact as well.
           % Oct 19/08
\documentclass{article}      % Specifies the document class

\usepackage{amsmath}
\usepackage{mathpazo}

%
% shorthand for bold symbols, convenient for vectors and matrices
%
\newcommand{\Ba}[0]{\mathbf{a}}
\newcommand{\Bb}[0]{\mathbf{b}}
\newcommand{\Bc}[0]{\mathbf{c}}
\newcommand{\Bd}[0]{\mathbf{d}}
\newcommand{\Be}[0]{\mathbf{e}}
\newcommand{\Bf}[0]{\mathbf{f}}
\newcommand{\Bg}[0]{\mathbf{g}}
\newcommand{\Bh}[0]{\mathbf{h}}
\newcommand{\Bi}[0]{\mathbf{i}}
\newcommand{\Bj}[0]{\mathbf{j}}
\newcommand{\Bk}[0]{\mathbf{k}}
\newcommand{\Bl}[0]{\mathbf{l}}
\newcommand{\Bm}[0]{\mathbf{m}}
\newcommand{\Bn}[0]{\mathbf{n}}
\newcommand{\Bo}[0]{\mathbf{o}}
\newcommand{\Bp}[0]{\mathbf{p}}
\newcommand{\Bq}[0]{\mathbf{q}}
\newcommand{\Br}[0]{\mathbf{r}}
\newcommand{\Bs}[0]{\mathbf{s}}
\newcommand{\Bt}[0]{\mathbf{t}}
\newcommand{\Bu}[0]{\mathbf{u}}
\newcommand{\Bv}[0]{\mathbf{v}}
\newcommand{\Bw}[0]{\mathbf{w}}
\newcommand{\Bx}[0]{\mathbf{x}}
\newcommand{\By}[0]{\mathbf{y}}
\newcommand{\Bz}[0]{\mathbf{z}}
\newcommand{\BA}[0]{\mathbf{A}}
\newcommand{\BB}[0]{\mathbf{B}}
\newcommand{\BC}[0]{\mathbf{C}}
\newcommand{\BD}[0]{\mathbf{D}}
\newcommand{\BE}[0]{\mathbf{E}}
\newcommand{\BF}[0]{\mathbf{F}}
\newcommand{\BG}[0]{\mathbf{G}}
\newcommand{\BH}[0]{\mathbf{H}}
\newcommand{\BI}[0]{\mathbf{I}}
\newcommand{\BJ}[0]{\mathbf{J}}
\newcommand{\BK}[0]{\mathbf{K}}
\newcommand{\BL}[0]{\mathbf{L}}
\newcommand{\BM}[0]{\mathbf{M}}
\newcommand{\BN}[0]{\mathbf{N}}
\newcommand{\BO}[0]{\mathbf{O}}
\newcommand{\BP}[0]{\mathbf{P}}
\newcommand{\BQ}[0]{\mathbf{Q}}
\newcommand{\BR}[0]{\mathbf{R}}
\newcommand{\BS}[0]{\mathbf{S}}
\newcommand{\BT}[0]{\mathbf{T}}
\newcommand{\BU}[0]{\mathbf{U}}
\newcommand{\BV}[0]{\mathbf{V}}
\newcommand{\BW}[0]{\mathbf{W}}
\newcommand{\BX}[0]{\mathbf{X}}
\newcommand{\BY}[0]{\mathbf{Y}}
\newcommand{\BZ}[0]{\mathbf{Z}}

\newcommand{\Bzero}[0]{\mathbf{0}}
\newcommand{\Btheta}[0]{\boldsymbol{\theta}}
\newcommand{\Btau}[0]{\boldsymbol{\tau}}
\newcommand{\Bomega}[0]{\boldsymbol{\omega}}

%
% shorthand for unit vectors
%
\newcommand{\acap}[0]{\hat{\Ba}}
\newcommand{\bcap}[0]{\hat{\Bb}}
\newcommand{\ccap}[0]{\hat{\Bc}}
\newcommand{\dcap}[0]{\hat{\Bd}}
\newcommand{\ecap}[0]{\hat{\Be}}
\newcommand{\fcap}[0]{\hat{\Bf}}
\newcommand{\gcap}[0]{\hat{\Bg}}
\newcommand{\hcap}[0]{\hat{\Bh}}
\newcommand{\icap}[0]{\hat{\Bi}}
\newcommand{\jcap}[0]{\hat{\Bj}}
\newcommand{\kcap}[0]{\hat{\Bk}}
\newcommand{\lcap}[0]{\hat{\Bl}}
\newcommand{\mcap}[0]{\hat{\Bm}}
\newcommand{\ncap}[0]{\hat{\Bn}}
\newcommand{\ocap}[0]{\hat{\Bo}}
\newcommand{\pcap}[0]{\hat{\Bp}}
\newcommand{\qcap}[0]{\hat{\Bq}}
\newcommand{\rcap}[0]{\hat{\Br}}
\newcommand{\scap}[0]{\hat{\Bs}}
\newcommand{\tcap}[0]{\hat{\Bt}}
\newcommand{\ucap}[0]{\hat{\Bu}}
\newcommand{\vcap}[0]{\hat{\Bv}}
\newcommand{\wcap}[0]{\hat{\Bw}}
\newcommand{\xcap}[0]{\hat{\Bx}}
\newcommand{\ycap}[0]{\hat{\By}}
\newcommand{\zcap}[0]{\hat{\Bz}}
\newcommand{\thetacap}[0]{\hat{\Btheta}}

%
% to write R^n and C^n in a distinguishable fashion.  Perhaps change this
% to the double lined characters upon figuring out how to do so.
%
\newcommand{\C}[1]{$\mathbb{C}^{#1}$}
\newcommand{\R}[1]{$\mathbb{R}^{#1}$}

%
% various generally useful helpers
%

% derivative of #1 wrt. #2:
\newcommand{\D}[2] {\frac {d#2} {d#1}}

\newcommand{\inv}[1]{\frac{1}{#1}}
\newcommand{\cross}[0]{\times}

\newcommand{\abs}[1]{\lvert{#1}\rvert}
\newcommand{\norm}[1]{\lVert{#1}\rVert}
\newcommand{\innerprod}[2]{\langle{#1}, {#2}\rangle}
\newcommand{\dotprod}[2]{{#1} \cdot {#2}}
\newcommand{\bdotprod}[2]{\left({#1} \cdot {#2}\right)}
\newcommand{\crossprod}[2]{{#1} \cross {#2}}
\newcommand{\tripleprod}[3]{\dotprod{\left(\crossprod{#1}{#2}\right)}{#3}}

\DeclareMathOperator{\Proj}{Proj}
\DeclareMathOperator{\Span}{span}
\DeclareMathOperator{\Sgn}{sgn}
\DeclareMathOperator{\Area}{Area}
\DeclareMathOperator{\Volume}{Volume}

%
% A few miscellaneous things specific to this document
%
\newcommand{\crossop}[1]{\crossprod{#1}{}}

% R2 vector.
\newcommand{\VectorTwo}[2]{
\begin{bmatrix}
 {#1} \\
 {#2}
\end{bmatrix}
}

\newcommand{\VectorN}[1]{
\begin{bmatrix}
{#1}_1 \\
{#1}_2 \\
\vdots \\
{#1}_N \\
\end{bmatrix}
}

\newcommand{\DETuvij}[4]{
\begin{vmatrix}
 {#1}_{#3} & {#1}_{#4} \\
 {#2}_{#3} & {#2}_{#4}
\end{vmatrix}
}

\newcommand{\DETuvwijk}[6]{
\begin{vmatrix}
 {#1}_{#4} & {#1}_{#5} & {#1}_{#6} \\
 {#2}_{#4} & {#2}_{#5} & {#2}_{#6} \\
 {#3}_{#4} & {#3}_{#5} & {#3}_{#6}
\end{vmatrix}
}

\newcommand{\DETuvwxijkl}[8]{
\begin{vmatrix}
 {#1}_{#5} & {#1}_{#6} & {#1}_{#7} & {#1}_{#8} \\
 {#2}_{#5} & {#2}_{#6} & {#2}_{#7} & {#2}_{#8} \\
 {#3}_{#5} & {#3}_{#6} & {#3}_{#7} & {#3}_{#8} \\
 {#4}_{#5} & {#4}_{#6} & {#4}_{#7} & {#4}_{#8} \\
\end{vmatrix}
}

%\newcommand{\DETuvwxyijklm}[10]{
%\begin{vmatrix}
% {#1}_{#6} & {#1}_{#7} & {#1}_{#8} & {#1}_{#9} & {#1}_{#10} \\
% {#2}_{#6} & {#2}_{#7} & {#2}_{#8} & {#2}_{#9} & {#2}_{#10} \\
% {#3}_{#6} & {#3}_{#7} & {#3}_{#8} & {#3}_{#9} & {#3}_{#10} \\
% {#4}_{#6} & {#4}_{#7} & {#4}_{#8} & {#4}_{#9} & {#4}_{#10} \\
% {#5}_{#6} & {#5}_{#7} & {#5}_{#8} & {#5}_{#9} & {#5}_{#10}
%\end{vmatrix}
%}

% R3 vector.
\newcommand{\VectorThree}[3]{
\begin{bmatrix}
 {#1} \\
 {#2} \\
 {#3}
\end{bmatrix}
}


%<misc>
%
\newcommand{\Abs}[1]{{\left\lvert{#1}\right\rvert}}
\newcommand{\spacegrad}[0]{\boldsymbol{\nabla}}
\newcommand{\grad}[0]{\nabla}
\newcommand{\LL}[0]{\mathcal{L}}

% == \partial_{#1} {#2}
\newcommand{\PD}[2]{\frac{\partial {#2}}{\partial {#1}}}
% inline variant
\newcommand{\PDi}[2]{{\partial {#2}}/{\partial {#1}}}

\newcommand{\PDD}[3]{\frac{\partial^2 {#3}}{\partial {#1}\partial {#2}}}
%\newcommand{\PDd}[2]{\frac{\partial^2 {#2}}{{\partial{#1}}^2}}
\newcommand{\PDsq}[2]{\frac{\partial^2 {#2}}{(\partial {#1})^2}}

\newcommand{\Partial}[2]{\frac{\partial {#1}}{\partial {#2}}}
\DeclareMathOperator{\RejName}{Rej}
\newcommand{\Rej}[2]{\RejName_{#1}\left( {#2} \right)}
\newcommand{\Rm}[1]{\mathbb{R}^{#1}}
\newcommand{\Cm}[1]{\mathbb{C}^{#1}}
\newcommand{\conj}[0]{{*}}

%</misc>

% <grade selection>
%
\newcommand{\gpgrade}[2] {{\left\langle{{#1}}\right\rangle}_{#2}}

\newcommand{\gpgradezero}[1] {\gpgrade{#1}{}}
%\newcommand{\gpscalargrade}[1] {{\left\langle{{#1}}\right\rangle}}
%\newcommand{\gpgradezero}[1] {\gpgrade{#1}{0}}

%\newcommand{\gpgradeone}[1] {{\left\langle{{#1}}\right\rangle}_{1}}
\newcommand{\gpgradeone}[1] {\gpgrade{#1}{1}}

\newcommand{\gpgradetwo}[1] {\gpgrade{#1}{2}}
\newcommand{\gpgradethree}[1] {\gpgrade{#1}{3}}
\newcommand{\gpgradefour}[1] {\gpgrade{#1}{4}}
%
% </grade selection>



\newcommand{\adot}[0]{{\dot{a}}}
\newcommand{\bdot}[0]{{\dot{b}}}
% taken for centered dot:
%\newcommand{\cdot}[0]{{\dot{c}}}
%\newcommand{\ddot}[0]{{\dot{d}}}
\newcommand{\edot}[0]{{\dot{e}}}
\newcommand{\fdot}[0]{{\dot{f}}}
\newcommand{\gdot}[0]{{\dot{g}}}
\newcommand{\hdot}[0]{{\dot{h}}}
\newcommand{\idot}[0]{{\dot{i}}}
\newcommand{\jdot}[0]{{\dot{j}}}
\newcommand{\kdot}[0]{{\dot{k}}}
\newcommand{\ldot}[0]{{\dot{l}}}
\newcommand{\mdot}[0]{{\dot{m}}}
\newcommand{\ndot}[0]{{\dot{n}}}
%\newcommand{\odot}[0]{{\dot{o}}}
\newcommand{\pdot}[0]{{\dot{p}}}
\newcommand{\qdot}[0]{{\dot{q}}}
\newcommand{\rdot}[0]{{\dot{r}}}
\newcommand{\sdot}[0]{{\dot{s}}}
\newcommand{\tdot}[0]{{\dot{t}}}
\newcommand{\udot}[0]{{\dot{u}}}
\newcommand{\vdot}[0]{{\dot{v}}}
\newcommand{\wdot}[0]{{\dot{w}}}
\newcommand{\xdot}[0]{{\dot{x}}}
\newcommand{\ydot}[0]{{\dot{y}}}
\newcommand{\zdot}[0]{{\dot{z}}}
\newcommand{\addot}[0]{{\ddot{a}}}
\newcommand{\bddot}[0]{{\ddot{b}}}
\newcommand{\cddot}[0]{{\ddot{c}}}
%\newcommand{\dddot}[0]{{\ddot{d}}}
\newcommand{\eddot}[0]{{\ddot{e}}}
\newcommand{\fddot}[0]{{\ddot{f}}}
\newcommand{\gddot}[0]{{\ddot{g}}}
\newcommand{\hddot}[0]{{\ddot{h}}}
\newcommand{\iddot}[0]{{\ddot{i}}}
\newcommand{\jddot}[0]{{\ddot{j}}}
\newcommand{\kddot}[0]{{\ddot{k}}}
\newcommand{\lddot}[0]{{\ddot{l}}}
\newcommand{\mddot}[0]{{\ddot{m}}}
\newcommand{\nddot}[0]{{\ddot{n}}}
\newcommand{\oddot}[0]{{\ddot{o}}}
\newcommand{\pddot}[0]{{\ddot{p}}}
\newcommand{\qddot}[0]{{\ddot{q}}}
\newcommand{\rddot}[0]{{\ddot{r}}}
\newcommand{\sddot}[0]{{\ddot{s}}}
\newcommand{\tddot}[0]{{\ddot{t}}}
\newcommand{\uddot}[0]{{\ddot{u}}}
\newcommand{\vddot}[0]{{\ddot{v}}}
\newcommand{\wddot}[0]{{\ddot{w}}}
\newcommand{\xddot}[0]{{\ddot{x}}}
\newcommand{\yddot}[0]{{\ddot{y}}}
\newcommand{\zddot}[0]{{\ddot{z}}}

%<bold and dot greek symbols>
%

\newcommand{\Deltadot}[0]{{\dot{\Delta}}}
\newcommand{\Gammadot}[0]{{\dot{\Gamma}}}
\newcommand{\Lambdadot}[0]{{\dot{\Lambda}}}
\newcommand{\Omegadot}[0]{{\dot{\Omega}}}
\newcommand{\Phidot}[0]{{\dot{\Phi}}}
\newcommand{\Pidot}[0]{{\dot{\Pi}}}
\newcommand{\Psidot}[0]{{\dot{\Psi}}}
\newcommand{\Sigmadot}[0]{{\dot{\Sigma}}}
\newcommand{\Thetadot}[0]{{\dot{\Theta}}}
\newcommand{\Upsilondot}[0]{{\dot{\Upsilon}}}
\newcommand{\Xidot}[0]{{\dot{\Xi}}}
\newcommand{\alphadot}[0]{{\dot{\alpha}}}
\newcommand{\betadot}[0]{{\dot{\beta}}}
\newcommand{\chidot}[0]{{\dot{\chi}}}
\newcommand{\deltadot}[0]{{\dot{\delta}}}
\newcommand{\epsilondot}[0]{{\dot{\epsilon}}}
\newcommand{\etadot}[0]{{\dot{\eta}}}
\newcommand{\gammadot}[0]{{\dot{\gamma}}}
\newcommand{\kappadot}[0]{{\dot{\kappa}}}
\newcommand{\lambdadot}[0]{{\dot{\lambda}}}
\newcommand{\mudot}[0]{{\dot{\mu}}}
\newcommand{\nudot}[0]{{\dot{\nu}}}
\newcommand{\omegadot}[0]{{\dot{\omega}}}
\newcommand{\phidot}[0]{{\dot{\phi}}}
\newcommand{\pidot}[0]{{\dot{\pi}}}
\newcommand{\psidot}[0]{{\dot{\psi}}}
\newcommand{\rhodot}[0]{{\dot{\rho}}}
\newcommand{\sigmadot}[0]{{\dot{\sigma}}}
\newcommand{\taudot}[0]{{\dot{\tau}}}
\newcommand{\thetadot}[0]{{\dot{\theta}}}
\newcommand{\upsilondot}[0]{{\dot{\upsilon}}}
\newcommand{\varepsilondot}[0]{{\dot{\varepsilon}}}
\newcommand{\varphidot}[0]{{\dot{\varphi}}}
\newcommand{\varpidot}[0]{{\dot{\varpi}}}
\newcommand{\varrhodot}[0]{{\dot{\varrho}}}
\newcommand{\varsigmadot}[0]{{\dot{\varsigma}}}
\newcommand{\varthetadot}[0]{{\dot{\vartheta}}}
\newcommand{\xidot}[0]{{\dot{\xi}}}
\newcommand{\zetadot}[0]{{\dot{\zeta}}}

\newcommand{\Deltaddot}[0]{{\ddot{\Delta}}}
\newcommand{\Gammaddot}[0]{{\ddot{\Gamma}}}
\newcommand{\Lambdaddot}[0]{{\ddot{\Lambda}}}
\newcommand{\Omegaddot}[0]{{\ddot{\Omega}}}
\newcommand{\Phiddot}[0]{{\ddot{\Phi}}}
\newcommand{\Piddot}[0]{{\ddot{\Pi}}}
\newcommand{\Psiddot}[0]{{\ddot{\Psi}}}
\newcommand{\Sigmaddot}[0]{{\ddot{\Sigma}}}
\newcommand{\Thetaddot}[0]{{\ddot{\Theta}}}
\newcommand{\Upsilonddot}[0]{{\ddot{\Upsilon}}}
\newcommand{\Xiddot}[0]{{\ddot{\Xi}}}
\newcommand{\alphaddot}[0]{{\ddot{\alpha}}}
\newcommand{\betaddot}[0]{{\ddot{\beta}}}
\newcommand{\chiddot}[0]{{\ddot{\chi}}}
\newcommand{\deltaddot}[0]{{\ddot{\delta}}}
\newcommand{\epsilonddot}[0]{{\ddot{\epsilon}}}
\newcommand{\etaddot}[0]{{\ddot{\eta}}}
\newcommand{\gammaddot}[0]{{\ddot{\gamma}}}
\newcommand{\kappaddot}[0]{{\ddot{\kappa}}}
\newcommand{\lambdaddot}[0]{{\ddot{\lambda}}}
\newcommand{\muddot}[0]{{\ddot{\mu}}}
\newcommand{\nuddot}[0]{{\ddot{\nu}}}
\newcommand{\omegaddot}[0]{{\ddot{\omega}}}
\newcommand{\phiddot}[0]{{\ddot{\phi}}}
\newcommand{\piddot}[0]{{\ddot{\pi}}}
\newcommand{\psiddot}[0]{{\ddot{\psi}}}
\newcommand{\rhoddot}[0]{{\ddot{\rho}}}
\newcommand{\sigmaddot}[0]{{\ddot{\sigma}}}
\newcommand{\tauddot}[0]{{\ddot{\tau}}}
\newcommand{\thetaddot}[0]{{\ddot{\theta}}}
\newcommand{\upsilonddot}[0]{{\ddot{\upsilon}}}
\newcommand{\varepsilonddot}[0]{{\ddot{\varepsilon}}}
\newcommand{\varphiddot}[0]{{\ddot{\varphi}}}
\newcommand{\varpiddot}[0]{{\ddot{\varpi}}}
\newcommand{\varrhoddot}[0]{{\ddot{\varrho}}}
\newcommand{\varsigmaddot}[0]{{\ddot{\varsigma}}}
\newcommand{\varthetaddot}[0]{{\ddot{\vartheta}}}
\newcommand{\xiddot}[0]{{\ddot{\xi}}}
\newcommand{\zetaddot}[0]{{\ddot{\zeta}}}

\newcommand{\BDelta}[0]{\boldsymbol{\Delta}}
\newcommand{\BGamma}[0]{\boldsymbol{\Gamma}}
\newcommand{\BLambda}[0]{\boldsymbol{\Lambda}}
\newcommand{\BOmega}[0]{\boldsymbol{\Omega}}
\newcommand{\BPhi}[0]{\boldsymbol{\Phi}}
\newcommand{\BPi}[0]{\boldsymbol{\Pi}}
\newcommand{\BPsi}[0]{\boldsymbol{\Psi}}
\newcommand{\BSigma}[0]{\boldsymbol{\Sigma}}
\newcommand{\BTheta}[0]{\boldsymbol{\Theta}}
\newcommand{\BUpsilon}[0]{\boldsymbol{\Upsilon}}
\newcommand{\BXi}[0]{\boldsymbol{\Xi}}
\newcommand{\Balpha}[0]{\boldsymbol{\alpha}}
\newcommand{\Bbeta}[0]{\boldsymbol{\beta}}
\newcommand{\Bchi}[0]{\boldsymbol{\chi}}
\newcommand{\Bdelta}[0]{\boldsymbol{\delta}}
\newcommand{\Bepsilon}[0]{\boldsymbol{\epsilon}}
\newcommand{\Beta}[0]{\boldsymbol{\eta}}
\newcommand{\Bgamma}[0]{\boldsymbol{\gamma}}
\newcommand{\Bkappa}[0]{\boldsymbol{\kappa}}
\newcommand{\Blambda}[0]{\boldsymbol{\lambda}}
\newcommand{\Bmu}[0]{\boldsymbol{\mu}}
\newcommand{\Bnu}[0]{\boldsymbol{\nu}}
%\newcommand{\Bomega}[0]{\boldsymbol{\omega}}
\newcommand{\Bphi}[0]{\boldsymbol{\phi}}
\newcommand{\Bpi}[0]{\boldsymbol{\pi}}
\newcommand{\Bpsi}[0]{\boldsymbol{\psi}}
\newcommand{\Brho}[0]{\boldsymbol{\rho}}
\newcommand{\Bsigma}[0]{\boldsymbol{\sigma}}
%\newcommand{\Btau}[0]{\boldsymbol{\tau}}
%\newcommand{\Btheta}[0]{\boldsymbol{\theta}}
\newcommand{\Bupsilon}[0]{\boldsymbol{\upsilon}}
\newcommand{\Bvarepsilon}[0]{\boldsymbol{\varepsilon}}
\newcommand{\Bvarphi}[0]{\boldsymbol{\varphi}}
\newcommand{\Bvarpi}[0]{\boldsymbol{\varpi}}
\newcommand{\Bvarrho}[0]{\boldsymbol{\varrho}}
\newcommand{\Bvarsigma}[0]{\boldsymbol{\varsigma}}
\newcommand{\Bvartheta}[0]{\boldsymbol{\vartheta}}
\newcommand{\Bxi}[0]{\boldsymbol{\xi}}
\newcommand{\Bzeta}[0]{\boldsymbol{\zeta}}
%
%</bold and dot greek symbols>
%<infrequent>
%
%\newcommand{\AreaOp}[1]{\AName_{#1}}
%\newcommand{\Babs}[0]{\abs{\BB}}
%\newcommand{\Bcap}[0]{\hat{\BB}}
%\newcommand{\BrPrimeRej}[0]{\rcap(\rcap \wedge \Br')}
%\newcommand{\CA}[0]{\mathcal{A}}
%\newcommand{\Cos}[1]{\cos{\left({#1}\right)}}
%\newcommand{\Det}[1] {\abs{#1}}
%\newcommand{\Dsq}[2] {\frac {\partial^2 {#1}} {\partial {#2}^2}}
%\newcommand{\Exp}[1]{\exp{\left({#1}\right)}}
%\newcommand{\Norm}[1]{\left\lVert{#1}\right\rVert}
%\newcommand{\Sin}[1]{\sin{\left({#1}\right)}}
%\newcommand{\T}[0]{\text{T}}
%\newcommand{\VolumeOp}[1]{\VName_{#1}}
%\newcommand{\agrad}[0]{\Ba \cdot \nabla}
%\newcommand{\alphacap}[0]{\hat{\boldsymbol{\alpha}}}
%\newcommand{\Fcap}[0]{\hat{\BF}}
%\newcommand{\bithree}[0]{{\Bi}_3}
%\newcommand{\bxa}[0]{\Bx\Ba}
%\newcommand{\coordvec}[2]{
%\newcommand{\costheta}[0]{\acap \cdot \xcap}
%\newcommand{\ddt}[1]{\ddot{#1}}
%\newcommand{\ddu}[1] {\frac {d{#1}} {du}}
%\newcommand{\dsqxj}[2] {\frac {\partial^2 {#1}} {\partial {x_{#2}}^2}}
%\newcommand{\dtheta}[1]{\frac{d {#1}}{d \theta}}
%\newcommand{\dt}[1]{\dot{#1}}
%\newcommand{\dt}[1]{\frac{d {#1}}{dt}}
%\newcommand{\dxj}[2] {\frac {\partial {#1}} {\partial {x_{#2}}}}
%\newcommand{\halfPhi}[0]{\frac{\phi}{2}}
%\newcommand{\half}[0]{\inv{2}}
%\newcommand{\inv}[1]{\frac{1}{#1}}
%\newcommand{\laplacian}[0]{\nabla^2}
%\newcommand{\matrixoftx}[3]{
%\newcommand{\nrrp}[0]{\norm{\rcap \wedge \Br'}}
%\newcommand{\oiint}{\bigcirc \hspace{-1.4em} \int \hspace{-.8em} \int}
%\newcommand{\transpose}[1]{{#1}^{\text{T}}}
%\newcommand{\transpose}[1]{{{#1}^{\TextTranspose}}}
%\newcommand{\transpose}[1]{{{#1}^{\text{T}}}}
%\newcommand{\barA}[0]{\bar{A}}
%\newcommand{\qbar}[0]{\bar{q}}
%\newcommand{\qdotbar}[0]{\dot{\bar{q}}}
%
%</infrequent>





%
% The real thing:
%

\usepackage[bookmarks=true]{hyperref}

                             % The preamble begins here.
\title{ Application of Noether's to Lorentz transformed interaction Lagrangian } % Declares the document's title.
\author{Peeter Joot}         % Declares the author's name.
\date{ October 22, 2008.  Last Revision: $Date: 2008/10/23 13:08:16 $ } % Deleting this command produces today's date.

\begin{document}             % End of preamble and beginning of text.

\maketitle{}

\tableofcontents

\section{ Motivation }

Here we consider Noether's theorem applied to the covariant form of the Lorentz force Lagrangian.  Boost under rotation or boost or a combination of the two will be considered.

\subsection{ Covariant result. } 

For proper velocity $v$, four potential $A$, and positive time metric signature $(\gamma_0)^2 = 1$, the Lorentz for
Lagrangian is

\begin{align}\label{eqn:lorentzforce}
\LL = \inv{2}m v \cdot v + q A \cdot v/c
\end{align}

Let's see if Noether's can be used to extract an invariant from 
the Lorentz force Lagrangian \ref{eqn:lorentzforce} under a
Lorentz boost or a spatial rotational transformation.

Four vector dot products are Lorentz invariants.  This can be thought of as the definition of a Lorentz transform (ie: the transformations
that leave the four vector dot products unchanged).  Alternatively, this can be shown using the exponential form of the boost

\begin{align*}
L(x) = \exp(-\alpha \acap/2) x \exp(\alpha \acap/2)
\end{align*}

\begin{align*}
L(x) \cdot L(y)
&= \gpgradezero{ \exp(-\alpha \acap/2) x \exp(\alpha \acap/2) \exp(-\alpha \acap/2) y \exp(\alpha \acap/2) } \\
&= \gpgradezero{ \exp(-\alpha \acap/2) x y \exp(\alpha \acap/2) } \\
&= x \cdot y \gpgradezero{ \exp(-\alpha \acap/2) \exp(\alpha \acap/2) } \\
&= x \cdot y \\
\end{align*}

Using the exponential form of the boost operation, boosting $v$, $A$ leaves the Lagrangian unchanged.
Therefore there is a conserved quantity according to Noether's, but what is it?

Also observe that the spacetime nature of the bivector $\acap$ has not actually been specified, which means that all the
subsequent results apply to spatial rotation as well.  Due to the negative spatial signature ($(\gamma_i)^2=-1$) used here, for a spatial rotation $\alpha$ will represent a rotation in the negative sense in the oriented plane specified by the unit bivector $\acap$.

Consider change with respect to the rapidity factor (or rotational angle) $\alpha$

\begin{align}\label{eqn:lorentzboosted}
\PD{\alpha}{\LL'} = \frac{d}{d\tau} \left( \PD{\alpha}{x'} \cdot \grad_{v'} \LL \right)
\end{align}

The boost spacetime plane (or rotational plane) $\acap$ could also be considered a parameter in the transformation, but to use that or the combination of the
two we need the multivector form of Noethers.  These notes were in fact originally part of an attempt \cite{PJEulerLagrange}
to get a feeling for the scalar case as lead up to that so this is an exersize for later.

As for the derivatives in \ref{eqn:lorentzboosted} we have

\begin{align*}
\PD{\alpha}{x'} 
&= \PD{\alpha}{} \exp(-\alpha \acap/2) x \exp(\alpha \acap/2) \\
&= -\inv{2} \left(\acap x' - x'\acap\right) \\
&= - \acap \cdot x' \\
\end{align*}

\begin{align*}
\grad_{v'} \LL &= p' + qA'/c
\end{align*}

So the conserved quantity is
\begin{align*}
- (\acap \cdot x') \cdot \left( p' + qA'/c \right)
&= - \acap \cdot (x' \wedge (p' + qA'/c ) ) \\
&= -\acap \cdot \kappa
\end{align*}

So we have a conserved quantity

\begin{align*}
x \wedge (p + qA/c ) = \kappa
\end{align*}

This has the looks of the three dimensional angular momentum conservation expression (with an added term due to non-radial potential),
but doesn't look like any quantity from relativistic texts that I have seen (not that I've really seen too much).

As an example to get a feeling for this take $x$ to be a rest frame worldline.  Then we have

\begin{align}
c t \gamma_0 \wedge ( m \tdot \gamma_0 + qA/c ) = -q t \BA = \kappa
\end{align}

Which indicates that the product of observer time and the observers' three vector potential is a constant of motion.  Curious.  Not a familar result.

Assuming these calculations are correct, then if this holds for all time for then $\kappa =0$ due to the origin time of $x$.
I would interpret this to mean that for the charged mass to be at rest, the vector potential must also be zero.  So while $x = ct\gamma_0$ 
is simple for calculations, it does not appear to be a terribly interesting case.

FIXME: try plugging in specific solutions to the Lorentz force equation here to validate or invalidate this calculation.

One further thing that can be observed about this is that if we take derivatives of

\begin{align*}
x \wedge (p + qA/c ) = \kappa
\end{align*}

we have
\begin{align*}
v \wedge (p + qA/c ) + x \wedge (\pdot + q\dot{A}/c ) = 0
\end{align*}

Or
\begin{align}\label{eqn:noethersLxTx}
x \wedge \pdot &= \frac{d}{d\tau} \left( q/c A \wedge x \right) \\
&= qA \wedge v/c + q/c \dot{A} \wedge x 
\end{align}

So we have a relativistic torque expressed in terms of the potential, proper velocity and the variation of the potential.

\subsection{ Expansion in observer frame. } 

This still isn't familiar looking, but lets expand this in terms of a particular observable, and see what falls out.  First the LHS, with $dt/d\tau = \gamma$

\begin{align*}
x \wedge \pdot &= (ct \gamma_0 + x^i \gamma_i) \wedge 
\left( \gamma \frac{d}{dt}\left( m \gamma (c \gamma_0 + \frac{dx^j}{dt} \gamma_j ) \right) \right) \\
\end{align*}

So
\begin{align*}
\inv{\gamma} (x \wedge \pdot)
&= 
- ct \frac{d (\gamma \Bp)}{dt}
+ \Bx \frac{d (m c \gamma)}{dt} 
+ x^i \gamma_i \wedge 
\frac{d}{dt}\left( m \gamma \frac{dx^j}{dt} \gamma_j \right) \\
\end{align*}

But
\begin{align*}
\sigma_i \wedge \sigma_j
&= \inv{2}( \gamma_i \gamma_0 \gamma_j \gamma_0 - \gamma_j \gamma_0 \gamma_i \gamma_0) \\
&= -\frac{(\gamma_0)^2}{2}( \gamma_i \gamma_j - \gamma_j \gamma_i ) \\
&= -\gamma_i \wedge \gamma_j \\
\end{align*}

for
\begin{align*}
\inv{\gamma} (x \wedge \pdot)
&= 
- ct \frac{d (\gamma \Bp)}{dt}
+ \Bx \frac{d (m c \gamma)}{dt} 
- \Bx \wedge \frac{d( \gamma \Bp )}{dt} \\
\end{align*}

Now, for the RHS of \ref{eqn:noethersLxTx}, with $A^0 = \phi$

\begin{align*}
\frac{q}{c} \gamma \frac{d(x \wedge A)}{dt}
&= \frac{q}{c} \gamma \frac{d}{dt} (ct \gamma_0 + x^i \gamma_i) \wedge (\phi \gamma_0 + A^j \gamma_j) \\
&= \frac{q}{c} \gamma \frac{d}{dt} \left( -ct \BA + \phi \Bx - \Bx \wedge \BA \right)
\end{align*}

Equating the vector and bivector parts, and employing a duality transformation for the bivector parts leaves two vector relationships
\begin{align}\label{eqn:boostresultspacetime}
ct \frac{d (\gamma \Bp)}{dt} - \Bx \frac{d (m c \gamma)}{dt} &= \frac{q}{c} \frac{d \left( ct \BA - \phi \Bx \right) }{dt}
\end{align}
\begin{align}\label{eqn:rotation}
\Bx \cross \frac{d( \gamma \Bp )}{dt} &= \frac{q}{c} \frac{d}{dt} \left( \Bx \cross \BA \right)
\end{align}

FIXME: the first equation looks like it could also be expressed in some sort more symmetric form.  Perhaps a grade two (commutator) product between the multivectors $(mc\gamma, \Bp) = p \gamma_0$, and $(\phi, \BA) = A \gamma_0$?

\subsection{ In tensor form. }

As can be seen above, the four vector form of equation \ref{eqn:noethersLxTx} is much more symmetric.  What does it look like in 
tensor form?  After first re-consolidating the proper time derivatives we can read the coordinate form off by inspection

\begin{align*}
x \wedge \pdot &= \frac{d}{d\tau} \left( q/c A \wedge x \right) \\
\end{align*}

\begin{align*}
\gamma_\mu \wedge \gamma_\nu x^\mu m v^\nu &= \frac{d}{d\tau} \left( q/c A^\alpha x^\beta \right) \gamma_\alpha \wedge \gamma_\beta \\
\end{align*}

Which gives the tensor expression

\begin{align}
\epsilon_{\mu\nu} \left(x^\mu v^\nu - \frac{d}{d\tau} \left( \frac{q}{mc} A^\mu x^\nu \right) \right) = 0
\end{align}

This in turn implies the following six equations in $\mu$, and $\nu$

\begin{align}\label{eqn:coords}
x^\mu v^\nu - x^\nu v^\mu = \frac{q}{mc} \frac{d}{d\tau} \left( A^\mu x^\nu - A^\nu x^\mu \right)
\end{align}

Looking to see if I got the right result, I asked on PF, and was pointed to 
\cite{BaezBoosts}.
That ascii thread is hard to read but at least my result is similar.  I'll have to massage things to match them up more closely.

What I didn't realize until I read that is that my rotation wasn't fixed as either hyperbolic or euclidean since I didn't actually specify the specific nature of the bivector for the rotational plane.  So I ended up with results for both the spatial invariance and the boost invariance at the same time.  Have adjusted things above, but that is why the spatial rotation references all appear as afterthoughts.

Of the six equations in \ref{eqn:coords}, taking space time indexes 
yields the vector equation \ref{eqn:boostresultspacetime} as the conserved quantity for a boost.  Similarily
the second vector result in \ref{eqn:rotation} for purely spatial indexes is the conserved quantity for spatial rotation.
That makes my result seem more reasonable since I didn't expect to get so much only considering boost.

\bibliographystyle{plainnat} % supposed to allow for \url use.
\bibliography{myrefs}      % expects file "myrefs.bib"

\end{document}               % End of document.
             % Oct 22/08
\documentclass{article}

\usepackage{amsmath}
\usepackage{mathpazo}

%
% shorthand for bold symbols, convenient for vectors and matrices
%
\newcommand{\Ba}[0]{\mathbf{a}}
\newcommand{\Bb}[0]{\mathbf{b}}
\newcommand{\Bc}[0]{\mathbf{c}}
\newcommand{\Bd}[0]{\mathbf{d}}
\newcommand{\Be}[0]{\mathbf{e}}
\newcommand{\Bf}[0]{\mathbf{f}}
\newcommand{\Bg}[0]{\mathbf{g}}
\newcommand{\Bh}[0]{\mathbf{h}}
\newcommand{\Bi}[0]{\mathbf{i}}
\newcommand{\Bj}[0]{\mathbf{j}}
\newcommand{\Bk}[0]{\mathbf{k}}
\newcommand{\Bl}[0]{\mathbf{l}}
\newcommand{\Bm}[0]{\mathbf{m}}
\newcommand{\Bn}[0]{\mathbf{n}}
\newcommand{\Bo}[0]{\mathbf{o}}
\newcommand{\Bp}[0]{\mathbf{p}}
\newcommand{\Bq}[0]{\mathbf{q}}
\newcommand{\Br}[0]{\mathbf{r}}
\newcommand{\Bs}[0]{\mathbf{s}}
\newcommand{\Bt}[0]{\mathbf{t}}
\newcommand{\Bu}[0]{\mathbf{u}}
\newcommand{\Bv}[0]{\mathbf{v}}
\newcommand{\Bw}[0]{\mathbf{w}}
\newcommand{\Bx}[0]{\mathbf{x}}
\newcommand{\By}[0]{\mathbf{y}}
\newcommand{\Bz}[0]{\mathbf{z}}
\newcommand{\BA}[0]{\mathbf{A}}
\newcommand{\BB}[0]{\mathbf{B}}
\newcommand{\BC}[0]{\mathbf{C}}
\newcommand{\BD}[0]{\mathbf{D}}
\newcommand{\BE}[0]{\mathbf{E}}
\newcommand{\BF}[0]{\mathbf{F}}
\newcommand{\BG}[0]{\mathbf{G}}
\newcommand{\BH}[0]{\mathbf{H}}
\newcommand{\BI}[0]{\mathbf{I}}
\newcommand{\BJ}[0]{\mathbf{J}}
\newcommand{\BK}[0]{\mathbf{K}}
\newcommand{\BL}[0]{\mathbf{L}}
\newcommand{\BM}[0]{\mathbf{M}}
\newcommand{\BN}[0]{\mathbf{N}}
\newcommand{\BO}[0]{\mathbf{O}}
\newcommand{\BP}[0]{\mathbf{P}}
\newcommand{\BQ}[0]{\mathbf{Q}}
\newcommand{\BR}[0]{\mathbf{R}}
\newcommand{\BS}[0]{\mathbf{S}}
\newcommand{\BT}[0]{\mathbf{T}}
\newcommand{\BU}[0]{\mathbf{U}}
\newcommand{\BV}[0]{\mathbf{V}}
\newcommand{\BW}[0]{\mathbf{W}}
\newcommand{\BX}[0]{\mathbf{X}}
\newcommand{\BY}[0]{\mathbf{Y}}
\newcommand{\BZ}[0]{\mathbf{Z}}

\newcommand{\Bzero}[0]{\mathbf{0}}
\newcommand{\Btheta}[0]{\boldsymbol{\theta}}
\newcommand{\Btau}[0]{\boldsymbol{\tau}}
\newcommand{\Bomega}[0]{\boldsymbol{\omega}}

%
% shorthand for unit vectors
%
\newcommand{\acap}[0]{\hat{\Ba}}
\newcommand{\bcap}[0]{\hat{\Bb}}
\newcommand{\ccap}[0]{\hat{\Bc}}
\newcommand{\dcap}[0]{\hat{\Bd}}
\newcommand{\ecap}[0]{\hat{\Be}}
\newcommand{\fcap}[0]{\hat{\Bf}}
\newcommand{\gcap}[0]{\hat{\Bg}}
\newcommand{\hcap}[0]{\hat{\Bh}}
\newcommand{\icap}[0]{\hat{\Bi}}
\newcommand{\jcap}[0]{\hat{\Bj}}
\newcommand{\kcap}[0]{\hat{\Bk}}
\newcommand{\lcap}[0]{\hat{\Bl}}
\newcommand{\mcap}[0]{\hat{\Bm}}
\newcommand{\ncap}[0]{\hat{\Bn}}
\newcommand{\ocap}[0]{\hat{\Bo}}
\newcommand{\pcap}[0]{\hat{\Bp}}
\newcommand{\qcap}[0]{\hat{\Bq}}
\newcommand{\rcap}[0]{\hat{\Br}}
\newcommand{\scap}[0]{\hat{\Bs}}
\newcommand{\tcap}[0]{\hat{\Bt}}
\newcommand{\ucap}[0]{\hat{\Bu}}
\newcommand{\vcap}[0]{\hat{\Bv}}
\newcommand{\wcap}[0]{\hat{\Bw}}
\newcommand{\xcap}[0]{\hat{\Bx}}
\newcommand{\ycap}[0]{\hat{\By}}
\newcommand{\zcap}[0]{\hat{\Bz}}
\newcommand{\thetacap}[0]{\hat{\Btheta}}

%
% to write R^n and C^n in a distinguishable fashion.  Perhaps change this
% to the double lined characters upon figuring out how to do so.
%
\newcommand{\C}[1]{$\mathbb{C}^{#1}$}
\newcommand{\R}[1]{$\mathbb{R}^{#1}$}

%
% various generally useful helpers
%

% derivative of #1 wrt. #2:
\newcommand{\D}[2] {\frac {d#2} {d#1}}

\newcommand{\inv}[1]{\frac{1}{#1}}
\newcommand{\cross}[0]{\times}

\newcommand{\abs}[1]{\lvert{#1}\rvert}
\newcommand{\norm}[1]{\lVert{#1}\rVert}
\newcommand{\innerprod}[2]{\langle{#1}, {#2}\rangle}
\newcommand{\dotprod}[2]{{#1} \cdot {#2}}
\newcommand{\bdotprod}[2]{\left({#1} \cdot {#2}\right)}
\newcommand{\crossprod}[2]{{#1} \cross {#2}}
\newcommand{\tripleprod}[3]{\dotprod{\left(\crossprod{#1}{#2}\right)}{#3}}

\DeclareMathOperator{\Proj}{Proj}
\DeclareMathOperator{\Span}{span}
\DeclareMathOperator{\Sgn}{sgn}
\DeclareMathOperator{\Area}{Area}
\DeclareMathOperator{\Volume}{Volume}

%
% A few miscellaneous things specific to this document
%
\newcommand{\crossop}[1]{\crossprod{#1}{}}

% R2 vector.
\newcommand{\VectorTwo}[2]{
\begin{bmatrix}
 {#1} \\
 {#2}
\end{bmatrix}
}

\newcommand{\VectorN}[1]{
\begin{bmatrix}
{#1}_1 \\
{#1}_2 \\
\vdots \\
{#1}_N \\
\end{bmatrix}
}

\newcommand{\DETuvij}[4]{
\begin{vmatrix}
 {#1}_{#3} & {#1}_{#4} \\
 {#2}_{#3} & {#2}_{#4}
\end{vmatrix}
}

\newcommand{\DETuvwijk}[6]{
\begin{vmatrix}
 {#1}_{#4} & {#1}_{#5} & {#1}_{#6} \\
 {#2}_{#4} & {#2}_{#5} & {#2}_{#6} \\
 {#3}_{#4} & {#3}_{#5} & {#3}_{#6}
\end{vmatrix}
}

\newcommand{\DETuvwxijkl}[8]{
\begin{vmatrix}
 {#1}_{#5} & {#1}_{#6} & {#1}_{#7} & {#1}_{#8} \\
 {#2}_{#5} & {#2}_{#6} & {#2}_{#7} & {#2}_{#8} \\
 {#3}_{#5} & {#3}_{#6} & {#3}_{#7} & {#3}_{#8} \\
 {#4}_{#5} & {#4}_{#6} & {#4}_{#7} & {#4}_{#8} \\
\end{vmatrix}
}

%\newcommand{\DETuvwxyijklm}[10]{
%\begin{vmatrix}
% {#1}_{#6} & {#1}_{#7} & {#1}_{#8} & {#1}_{#9} & {#1}_{#10} \\
% {#2}_{#6} & {#2}_{#7} & {#2}_{#8} & {#2}_{#9} & {#2}_{#10} \\
% {#3}_{#6} & {#3}_{#7} & {#3}_{#8} & {#3}_{#9} & {#3}_{#10} \\
% {#4}_{#6} & {#4}_{#7} & {#4}_{#8} & {#4}_{#9} & {#4}_{#10} \\
% {#5}_{#6} & {#5}_{#7} & {#5}_{#8} & {#5}_{#9} & {#5}_{#10}
%\end{vmatrix}
%}

% R3 vector.
\newcommand{\VectorThree}[3]{
\begin{bmatrix}
 {#1} \\
 {#2} \\
 {#3}
\end{bmatrix}
}


%<misc>
%
\newcommand{\Abs}[1]{{\left\lvert{#1}\right\rvert}}
\newcommand{\spacegrad}[0]{\boldsymbol{\nabla}}
\newcommand{\grad}[0]{\nabla}
\newcommand{\LL}[0]{\mathcal{L}}

% == \partial_{#1} {#2}
\newcommand{\PD}[2]{\frac{\partial {#2}}{\partial {#1}}}
% inline variant
\newcommand{\PDi}[2]{{\partial {#2}}/{\partial {#1}}}

\newcommand{\PDD}[3]{\frac{\partial^2 {#3}}{\partial {#1}\partial {#2}}}
%\newcommand{\PDd}[2]{\frac{\partial^2 {#2}}{{\partial{#1}}^2}}
\newcommand{\PDsq}[2]{\frac{\partial^2 {#2}}{(\partial {#1})^2}}

\newcommand{\Partial}[2]{\frac{\partial {#1}}{\partial {#2}}}
\DeclareMathOperator{\RejName}{Rej}
\newcommand{\Rej}[2]{\RejName_{#1}\left( {#2} \right)}
\newcommand{\Rm}[1]{\mathbb{R}^{#1}}
\newcommand{\Cm}[1]{\mathbb{C}^{#1}}
\newcommand{\conj}[0]{{*}}

%</misc>

% <grade selection>
%
\newcommand{\gpgrade}[2] {{\left\langle{{#1}}\right\rangle}_{#2}}

\newcommand{\gpgradezero}[1] {\gpgrade{#1}{}}
%\newcommand{\gpscalargrade}[1] {{\left\langle{{#1}}\right\rangle}}
%\newcommand{\gpgradezero}[1] {\gpgrade{#1}{0}}

%\newcommand{\gpgradeone}[1] {{\left\langle{{#1}}\right\rangle}_{1}}
\newcommand{\gpgradeone}[1] {\gpgrade{#1}{1}}

\newcommand{\gpgradetwo}[1] {\gpgrade{#1}{2}}
\newcommand{\gpgradethree}[1] {\gpgrade{#1}{3}}
\newcommand{\gpgradefour}[1] {\gpgrade{#1}{4}}
%
% </grade selection>



\newcommand{\adot}[0]{{\dot{a}}}
\newcommand{\bdot}[0]{{\dot{b}}}
% taken for centered dot:
%\newcommand{\cdot}[0]{{\dot{c}}}
%\newcommand{\ddot}[0]{{\dot{d}}}
\newcommand{\edot}[0]{{\dot{e}}}
\newcommand{\fdot}[0]{{\dot{f}}}
\newcommand{\gdot}[0]{{\dot{g}}}
\newcommand{\hdot}[0]{{\dot{h}}}
\newcommand{\idot}[0]{{\dot{i}}}
\newcommand{\jdot}[0]{{\dot{j}}}
\newcommand{\kdot}[0]{{\dot{k}}}
\newcommand{\ldot}[0]{{\dot{l}}}
\newcommand{\mdot}[0]{{\dot{m}}}
\newcommand{\ndot}[0]{{\dot{n}}}
%\newcommand{\odot}[0]{{\dot{o}}}
\newcommand{\pdot}[0]{{\dot{p}}}
\newcommand{\qdot}[0]{{\dot{q}}}
\newcommand{\rdot}[0]{{\dot{r}}}
\newcommand{\sdot}[0]{{\dot{s}}}
\newcommand{\tdot}[0]{{\dot{t}}}
\newcommand{\udot}[0]{{\dot{u}}}
\newcommand{\vdot}[0]{{\dot{v}}}
\newcommand{\wdot}[0]{{\dot{w}}}
\newcommand{\xdot}[0]{{\dot{x}}}
\newcommand{\ydot}[0]{{\dot{y}}}
\newcommand{\zdot}[0]{{\dot{z}}}
\newcommand{\addot}[0]{{\ddot{a}}}
\newcommand{\bddot}[0]{{\ddot{b}}}
\newcommand{\cddot}[0]{{\ddot{c}}}
%\newcommand{\dddot}[0]{{\ddot{d}}}
\newcommand{\eddot}[0]{{\ddot{e}}}
\newcommand{\fddot}[0]{{\ddot{f}}}
\newcommand{\gddot}[0]{{\ddot{g}}}
\newcommand{\hddot}[0]{{\ddot{h}}}
\newcommand{\iddot}[0]{{\ddot{i}}}
\newcommand{\jddot}[0]{{\ddot{j}}}
\newcommand{\kddot}[0]{{\ddot{k}}}
\newcommand{\lddot}[0]{{\ddot{l}}}
\newcommand{\mddot}[0]{{\ddot{m}}}
\newcommand{\nddot}[0]{{\ddot{n}}}
\newcommand{\oddot}[0]{{\ddot{o}}}
\newcommand{\pddot}[0]{{\ddot{p}}}
\newcommand{\qddot}[0]{{\ddot{q}}}
\newcommand{\rddot}[0]{{\ddot{r}}}
\newcommand{\sddot}[0]{{\ddot{s}}}
\newcommand{\tddot}[0]{{\ddot{t}}}
\newcommand{\uddot}[0]{{\ddot{u}}}
\newcommand{\vddot}[0]{{\ddot{v}}}
\newcommand{\wddot}[0]{{\ddot{w}}}
\newcommand{\xddot}[0]{{\ddot{x}}}
\newcommand{\yddot}[0]{{\ddot{y}}}
\newcommand{\zddot}[0]{{\ddot{z}}}

%<bold and dot greek symbols>
%

\newcommand{\Deltadot}[0]{{\dot{\Delta}}}
\newcommand{\Gammadot}[0]{{\dot{\Gamma}}}
\newcommand{\Lambdadot}[0]{{\dot{\Lambda}}}
\newcommand{\Omegadot}[0]{{\dot{\Omega}}}
\newcommand{\Phidot}[0]{{\dot{\Phi}}}
\newcommand{\Pidot}[0]{{\dot{\Pi}}}
\newcommand{\Psidot}[0]{{\dot{\Psi}}}
\newcommand{\Sigmadot}[0]{{\dot{\Sigma}}}
\newcommand{\Thetadot}[0]{{\dot{\Theta}}}
\newcommand{\Upsilondot}[0]{{\dot{\Upsilon}}}
\newcommand{\Xidot}[0]{{\dot{\Xi}}}
\newcommand{\alphadot}[0]{{\dot{\alpha}}}
\newcommand{\betadot}[0]{{\dot{\beta}}}
\newcommand{\chidot}[0]{{\dot{\chi}}}
\newcommand{\deltadot}[0]{{\dot{\delta}}}
\newcommand{\epsilondot}[0]{{\dot{\epsilon}}}
\newcommand{\etadot}[0]{{\dot{\eta}}}
\newcommand{\gammadot}[0]{{\dot{\gamma}}}
\newcommand{\kappadot}[0]{{\dot{\kappa}}}
\newcommand{\lambdadot}[0]{{\dot{\lambda}}}
\newcommand{\mudot}[0]{{\dot{\mu}}}
\newcommand{\nudot}[0]{{\dot{\nu}}}
\newcommand{\omegadot}[0]{{\dot{\omega}}}
\newcommand{\phidot}[0]{{\dot{\phi}}}
\newcommand{\pidot}[0]{{\dot{\pi}}}
\newcommand{\psidot}[0]{{\dot{\psi}}}
\newcommand{\rhodot}[0]{{\dot{\rho}}}
\newcommand{\sigmadot}[0]{{\dot{\sigma}}}
\newcommand{\taudot}[0]{{\dot{\tau}}}
\newcommand{\thetadot}[0]{{\dot{\theta}}}
\newcommand{\upsilondot}[0]{{\dot{\upsilon}}}
\newcommand{\varepsilondot}[0]{{\dot{\varepsilon}}}
\newcommand{\varphidot}[0]{{\dot{\varphi}}}
\newcommand{\varpidot}[0]{{\dot{\varpi}}}
\newcommand{\varrhodot}[0]{{\dot{\varrho}}}
\newcommand{\varsigmadot}[0]{{\dot{\varsigma}}}
\newcommand{\varthetadot}[0]{{\dot{\vartheta}}}
\newcommand{\xidot}[0]{{\dot{\xi}}}
\newcommand{\zetadot}[0]{{\dot{\zeta}}}

\newcommand{\Deltaddot}[0]{{\ddot{\Delta}}}
\newcommand{\Gammaddot}[0]{{\ddot{\Gamma}}}
\newcommand{\Lambdaddot}[0]{{\ddot{\Lambda}}}
\newcommand{\Omegaddot}[0]{{\ddot{\Omega}}}
\newcommand{\Phiddot}[0]{{\ddot{\Phi}}}
\newcommand{\Piddot}[0]{{\ddot{\Pi}}}
\newcommand{\Psiddot}[0]{{\ddot{\Psi}}}
\newcommand{\Sigmaddot}[0]{{\ddot{\Sigma}}}
\newcommand{\Thetaddot}[0]{{\ddot{\Theta}}}
\newcommand{\Upsilonddot}[0]{{\ddot{\Upsilon}}}
\newcommand{\Xiddot}[0]{{\ddot{\Xi}}}
\newcommand{\alphaddot}[0]{{\ddot{\alpha}}}
\newcommand{\betaddot}[0]{{\ddot{\beta}}}
\newcommand{\chiddot}[0]{{\ddot{\chi}}}
\newcommand{\deltaddot}[0]{{\ddot{\delta}}}
\newcommand{\epsilonddot}[0]{{\ddot{\epsilon}}}
\newcommand{\etaddot}[0]{{\ddot{\eta}}}
\newcommand{\gammaddot}[0]{{\ddot{\gamma}}}
\newcommand{\kappaddot}[0]{{\ddot{\kappa}}}
\newcommand{\lambdaddot}[0]{{\ddot{\lambda}}}
\newcommand{\muddot}[0]{{\ddot{\mu}}}
\newcommand{\nuddot}[0]{{\ddot{\nu}}}
\newcommand{\omegaddot}[0]{{\ddot{\omega}}}
\newcommand{\phiddot}[0]{{\ddot{\phi}}}
\newcommand{\piddot}[0]{{\ddot{\pi}}}
\newcommand{\psiddot}[0]{{\ddot{\psi}}}
\newcommand{\rhoddot}[0]{{\ddot{\rho}}}
\newcommand{\sigmaddot}[0]{{\ddot{\sigma}}}
\newcommand{\tauddot}[0]{{\ddot{\tau}}}
\newcommand{\thetaddot}[0]{{\ddot{\theta}}}
\newcommand{\upsilonddot}[0]{{\ddot{\upsilon}}}
\newcommand{\varepsilonddot}[0]{{\ddot{\varepsilon}}}
\newcommand{\varphiddot}[0]{{\ddot{\varphi}}}
\newcommand{\varpiddot}[0]{{\ddot{\varpi}}}
\newcommand{\varrhoddot}[0]{{\ddot{\varrho}}}
\newcommand{\varsigmaddot}[0]{{\ddot{\varsigma}}}
\newcommand{\varthetaddot}[0]{{\ddot{\vartheta}}}
\newcommand{\xiddot}[0]{{\ddot{\xi}}}
\newcommand{\zetaddot}[0]{{\ddot{\zeta}}}

\newcommand{\BDelta}[0]{\boldsymbol{\Delta}}
\newcommand{\BGamma}[0]{\boldsymbol{\Gamma}}
\newcommand{\BLambda}[0]{\boldsymbol{\Lambda}}
\newcommand{\BOmega}[0]{\boldsymbol{\Omega}}
\newcommand{\BPhi}[0]{\boldsymbol{\Phi}}
\newcommand{\BPi}[0]{\boldsymbol{\Pi}}
\newcommand{\BPsi}[0]{\boldsymbol{\Psi}}
\newcommand{\BSigma}[0]{\boldsymbol{\Sigma}}
\newcommand{\BTheta}[0]{\boldsymbol{\Theta}}
\newcommand{\BUpsilon}[0]{\boldsymbol{\Upsilon}}
\newcommand{\BXi}[0]{\boldsymbol{\Xi}}
\newcommand{\Balpha}[0]{\boldsymbol{\alpha}}
\newcommand{\Bbeta}[0]{\boldsymbol{\beta}}
\newcommand{\Bchi}[0]{\boldsymbol{\chi}}
\newcommand{\Bdelta}[0]{\boldsymbol{\delta}}
\newcommand{\Bepsilon}[0]{\boldsymbol{\epsilon}}
\newcommand{\Beta}[0]{\boldsymbol{\eta}}
\newcommand{\Bgamma}[0]{\boldsymbol{\gamma}}
\newcommand{\Bkappa}[0]{\boldsymbol{\kappa}}
\newcommand{\Blambda}[0]{\boldsymbol{\lambda}}
\newcommand{\Bmu}[0]{\boldsymbol{\mu}}
\newcommand{\Bnu}[0]{\boldsymbol{\nu}}
%\newcommand{\Bomega}[0]{\boldsymbol{\omega}}
\newcommand{\Bphi}[0]{\boldsymbol{\phi}}
\newcommand{\Bpi}[0]{\boldsymbol{\pi}}
\newcommand{\Bpsi}[0]{\boldsymbol{\psi}}
\newcommand{\Brho}[0]{\boldsymbol{\rho}}
\newcommand{\Bsigma}[0]{\boldsymbol{\sigma}}
%\newcommand{\Btau}[0]{\boldsymbol{\tau}}
%\newcommand{\Btheta}[0]{\boldsymbol{\theta}}
\newcommand{\Bupsilon}[0]{\boldsymbol{\upsilon}}
\newcommand{\Bvarepsilon}[0]{\boldsymbol{\varepsilon}}
\newcommand{\Bvarphi}[0]{\boldsymbol{\varphi}}
\newcommand{\Bvarpi}[0]{\boldsymbol{\varpi}}
\newcommand{\Bvarrho}[0]{\boldsymbol{\varrho}}
\newcommand{\Bvarsigma}[0]{\boldsymbol{\varsigma}}
\newcommand{\Bvartheta}[0]{\boldsymbol{\vartheta}}
\newcommand{\Bxi}[0]{\boldsymbol{\xi}}
\newcommand{\Bzeta}[0]{\boldsymbol{\zeta}}
%
%</bold and dot greek symbols>
%<infrequent>
%
%\newcommand{\AreaOp}[1]{\AName_{#1}}
%\newcommand{\Babs}[0]{\abs{\BB}}
%\newcommand{\Bcap}[0]{\hat{\BB}}
%\newcommand{\BrPrimeRej}[0]{\rcap(\rcap \wedge \Br')}
%\newcommand{\CA}[0]{\mathcal{A}}
%\newcommand{\Cos}[1]{\cos{\left({#1}\right)}}
%\newcommand{\Det}[1] {\abs{#1}}
%\newcommand{\Dsq}[2] {\frac {\partial^2 {#1}} {\partial {#2}^2}}
%\newcommand{\Exp}[1]{\exp{\left({#1}\right)}}
%\newcommand{\Norm}[1]{\left\lVert{#1}\right\rVert}
%\newcommand{\Sin}[1]{\sin{\left({#1}\right)}}
%\newcommand{\T}[0]{\text{T}}
%\newcommand{\VolumeOp}[1]{\VName_{#1}}
%\newcommand{\agrad}[0]{\Ba \cdot \nabla}
%\newcommand{\alphacap}[0]{\hat{\boldsymbol{\alpha}}}
%\newcommand{\Fcap}[0]{\hat{\BF}}
%\newcommand{\bithree}[0]{{\Bi}_3}
%\newcommand{\bxa}[0]{\Bx\Ba}
%\newcommand{\coordvec}[2]{
%\newcommand{\costheta}[0]{\acap \cdot \xcap}
%\newcommand{\ddt}[1]{\ddot{#1}}
%\newcommand{\ddu}[1] {\frac {d{#1}} {du}}
%\newcommand{\dsqxj}[2] {\frac {\partial^2 {#1}} {\partial {x_{#2}}^2}}
%\newcommand{\dtheta}[1]{\frac{d {#1}}{d \theta}}
%\newcommand{\dt}[1]{\dot{#1}}
%\newcommand{\dt}[1]{\frac{d {#1}}{dt}}
%\newcommand{\dxj}[2] {\frac {\partial {#1}} {\partial {x_{#2}}}}
%\newcommand{\halfPhi}[0]{\frac{\phi}{2}}
%\newcommand{\half}[0]{\inv{2}}
%\newcommand{\inv}[1]{\frac{1}{#1}}
%\newcommand{\laplacian}[0]{\nabla^2}
%\newcommand{\matrixoftx}[3]{
%\newcommand{\nrrp}[0]{\norm{\rcap \wedge \Br'}}
%\newcommand{\oiint}{\bigcirc \hspace{-1.4em} \int \hspace{-.8em} \int}
%\newcommand{\transpose}[1]{{#1}^{\text{T}}}
%\newcommand{\transpose}[1]{{{#1}^{\TextTranspose}}}
%\newcommand{\transpose}[1]{{{#1}^{\text{T}}}}
%\newcommand{\barA}[0]{\bar{A}}
%\newcommand{\qbar}[0]{\bar{q}}
%\newcommand{\qdotbar}[0]{\dot{\bar{q}}}
%
%</infrequent>






\usepackage[bookmarks=true]{hyperref}


\title{Field form of Noether's Law.}
\author{Peeter Joot}
\date{ October 29, 2008.  Last Revision: $Date: 2008/10/30 01:36:13 $ }

\begin{document}

\maketitle{}
%\tableofcontents

\section{ Derivation. }

Noether's law for a line integral action was developed in 
%euler_lagrange.ltx
\cite{PJEulerLagrange} along with the Euler-Lagrange equations
themselves.  Required for this is
the field form of the Euler-Lagrange equations as described
in
%field_lagragian.ltx
\cite{PJFieldLagrangian}.

\section{ Examples. }

\subsection{ Schrodenger invarience under phase change. }

The relativistic Schrodenger Lagrangian

\begin{align*}
\LL = \eta^{\mu\nu}\partial_\mu \psi \partial_\nu \psi^\conj + m^2 \psi \psi^\conj,
\end{align*}

gives a simple example application of the field form of Noether's equation, for a 
transformation that involves a phase change

\begin{align*}
\psi &\rightarrow \psi' = e^{i\theta}\psi \\
\psi^\conj &\rightarrow {\psi^\conj}' = e^{-i\theta}\psi^\conj.
\end{align*}

This transformation leaves the Lagrangian unchanged, so there is an associated conserved
quantity.

\begin{align*}
%\PD{\psi'}{\LL} &= m^2 {\psi'}^\conj \\
\PD{\theta}{\psi'} &= i \psi'
\PD{\partial_\mu \psi'}{\LL} &= \eta^{\mu\nu}\partial_\nu {\psi'}^\conj = \partial^\mu {\psi'}^\conj 
\end{align*}

Summing all the field partials, treating $\psi$, and $\psi^\conj$ as separate
field variables the divergence conservation statement is
\begin{align*}
\partial_\mu \left(
\underbrace{
\partial^\mu {\psi'}^\conj i\psi'
-\partial^\mu {\psi'} i{\psi'}^\conj
}_{{J'}^\mu}
\right) = 0
\end{align*}

Dropping primes and writing $J = \gamma_\mu J^\mu$, this is

\begin{align*}
J &= i (\psi \grad {\psi}^\conj - \psi^\conj \grad {\psi} ) \\
\grad \cdot J &= 0
\end{align*}

Apparently with charge added this quantity actually represents electric current density.  It will be interesting to
learn some quantum mechanics and see how this works.

\subsection{ Lorentz boost and rotation invariance of Maxwell Lagrangian. }

% followup for:
% 
%boost_maxwell_lagrangian.ltx
%\cite{PJBoostMaxwell}
%
%noethers_lorentz_force.ltx
%\cite{PJLorentzTxInteraction}
%

\bibliographystyle{plainnat} % supposed to allow for \url use.
\bibliography{myrefs}      % expects file "myrefs.bib"

\end{document}               % End of document.
                     % Oct 29/08
%
% Copyright � 2012 Peeter Joot.  All Rights Reserved.
% Licenced as described in the file LICENSE under the root directory of this GIT repository.
%

%
%
\chapter{Lorentz force Lagrangian with conjugate momentum}
\index{conjugate momentum}
\label{chap:lorentzForcePQA}
%\date{April 15, 2009.  lorentzForcePQA.tex}

\section{Motivation}

The covariant Lorentz force Lagrangian (for metric \(+---\))

\begin{equation}\label{eqn:lorentzForcePQA:20}
\begin{aligned}
\LL &= \inv{2} m v^2 + q A \cdot (v/c)
\end{aligned}
\end{equation}

Can be used to find the Lorentz force equation (here in four vector form)
\index{Lorentz force equation}

\begin{equation}\label{eqn:lorentzForcePQA:40}
\begin{aligned}
m\vdot &= q F \cdot (v/c)
\end{aligned}
\end{equation}

A derivation of this can be found in \chapcite{PJSrLorentzForce}.

However, in
\citep{pauli2000wm} the Lorentz force equation (in non-covariant form) is
derived as a limiting classical case via calculation of the expectation
value of the Hamiltonian

\begin{equation}\label{eqn:lorForcePqA:interactionConjMomHamiltonian}
\begin{aligned}
H = \inv{2m}\sum_{k=1}^3 \left(p_k - \frac{e}{c} A_k \right)^2 + V
\end{aligned}
\end{equation}

This has a much different looking structure than \(\LL\) above, so reconciliation of the two
is justifiable.

\section{Lorentz force Lagrangian with conjugate momentum}

Is there a relativistic form for the interaction Lagrangian with a structure similar to \eqnref{eqn:lorForcePqA:interactionConjMomHamiltonian}?

Let us try

\begin{equation}\label{eqn:lorentzForcePQA:60}
\begin{aligned}
\LL
&= \inv{2 m}\left( m v - \kappa A \right)^2 \\
&= \inv{2 m}\left( m^2 v^2 - 2 m \kappa A \cdot v + \kappa^2 A^2 \right)^2 \\
&= \inv{2 m} \left(
m^2 \xdot^\alpha \xdot_\alpha
- 2 m \kappa A_\alpha \xdot^\alpha
+ \kappa^2 A_\alpha A^\alpha
\right) \\
\end{aligned}
\end{equation}

where \(\kappa\) is to be determined.

For this Lagrangian the Euler-Lagrange calculation for variation of \(S = \int d^4 x \LL\) is

\begin{equation}\label{eqn:lorentzForcePQA:80}
\begin{aligned}
\PD{x^\mu}{\LL}
&= - \kappa (\partial_\mu A_\alpha) \xdot^\alpha + \inv{m} \kappa^2 (\partial_\mu A_\alpha) A^\alpha \\
\frac{d}{d\tau} \PD{\xdot^\mu}{\LL}
&= \frac{d}{d\tau} \left(m \xdot_\mu - \kappa A_\mu \right)
\end{aligned}
\end{equation}

Assembling and shuffling we have
\begin{equation}\label{eqn:lorentzForcePQA:100}
\begin{aligned}
m \xddot_\mu
&= \kappa (\partial_\alpha A_\mu) \xdot^\alpha
- \kappa (\partial_\mu A_\alpha) \xdot^\alpha + \inv{m} \kappa^2 (\partial_\mu A_\alpha) A^\alpha \\
&= \kappa (\partial_\alpha A_\mu - \partial_\mu A_\alpha) \xdot^\alpha + \inv{m} \kappa^2 (\partial_\mu A_\alpha) A^\alpha \\
&= \kappa F_{\alpha\mu} \xdot^\alpha + \inv{m} \kappa^2 (\partial_\mu A_\alpha) A^\alpha \\
\end{aligned}
\end{equation}

Comparing to the Lorentz force equation (again for a \(+---\) metric)

\begin{equation}\label{eqn:lorentzForcePQA:120}
\begin{aligned}
m v_\mu = \frac{q}{c} F_{\mu\nu} v^\nu
\end{aligned}
\end{equation}

We see that we need \(\kappa = -q/c\), but we have an extra factor that does not look familiar.  In vector form this Lagrangian would give us the
equation of motion

\begin{equation}\label{eqn:lorentzForcePQA:140}
\begin{aligned}
m v = \frac{q}{c} F \cdot v + \frac{q^2}{m c^2} (\grad A_\mu) A^\mu
\end{aligned}
\end{equation}

Assuming that this extra term has no place in the Lorentz force equation we need to adjust the original Lagrangian as follows

\begin{equation}\label{eqn:lorForcePqA:interactionLagPsq}
\begin{aligned}
\LL
&= \inv{2 m}\left( m v + \frac{q}{c} A \right)^2 - \frac{q^2}{ 2 m c^2} A^2
\end{aligned}
\end{equation}

Expressing this energy density in terms of the canonical momentum is
somewhat interesting.  It provides some extra structure, allowing for a loose identification of the two terms as

\begin{equation}\label{eqn:lorentzForcePQA:160}
\begin{aligned}
\LL &= K - V
\end{aligned}
\end{equation}

(ie: \(K = p^2/2m\), where \(p\) is the sum of the (proper) mechanical momentum and electromagnetic momentum).

However, that said, observe that expanding the square gives

\begin{equation}\label{eqn:lorentzForcePQA:180}
\begin{aligned}
\LL &= \inv{2} m v^2 + \frac{q}{c} A \cdot v
\end{aligned}
\end{equation}

which is exactly the original Lorentz force Lagrangian, so in the end this works out to only differ from the original cosmetically.

\section{On terminology.  The use of the term conjugate momentum}

\citep{goldstein1951cm} uses the term conjugate momentum in reference to a specific coordinate.  For example in

\begin{equation}\label{eqn:lorentzForcePQA:200}
\begin{aligned}
\LL(\rho, \rhodot) = f(\rho, \rhodot)
\end{aligned}
\end{equation}

the value
\begin{equation}\label{eqn:lorentzForcePQA:220}
\begin{aligned}
\PD{\rhodot}{f}
\end{aligned}
\end{equation}

is the momentum canonically conjugate to \(\rho\).  Above I have called the vector quantity \(m v + q A/c = (m v^\mu + q A^\mu/c) \gamma_\mu\) the canonical momentum.  My justification for doing so comes from a vectorization of the Euler-Lagrange equations.

Equating all the variational derivatives to zero separately

\begin{equation}\label{eqn:lorentzForcePQA:240}
\begin{aligned}
\frac{\delta \LL}{\delta x^\mu} = \PD{x^\mu}{\LL} - \frac{d}{d\tau} \PD{\xdot^\mu}{\LL} = 0
\end{aligned}
\end{equation}

can be replaced by an equivalent vector equation (note that summation is now implied)

\begin{equation}\label{eqn:lorentzForcePQA:260}
\begin{aligned}
\gamma^\mu \frac{\delta \LL}{\delta x^\mu} = \gamma^\mu \PD{x^\mu}{\LL} - \frac{d}{d\tau} \gamma^\mu \PD{\xdot^\mu}{\LL} = 0
\end{aligned}
\end{equation}

This has two distinct vector operations, a spacetime gradient, and a spacetime ``velocity gradient'', and it is not terribly abusive
of notation to write

\begin{equation}\label{eqn:lorentzForcePQA:280}
\begin{aligned}
\grad &= \gamma^\mu \PD{x^\mu}{} \\
\grad_v &= \gamma^\mu \PD{\xdot^\mu}{}
\end{aligned}
\end{equation}

with which all the Euler Lagrange equations can be summarized as

\begin{equation}\label{eqn:lorentzForcePQA:300}
\begin{aligned}
\grad \LL = \frac{d}{d\tau} \grad_v \LL
\end{aligned}
\end{equation}

It is thus natural, in a vector context, to name the quantity \(\grad_v \LL\), the canonical momentum.  It is a vectorized representation of all the individual momenta that are canonically conjugate to the respective coordinates.

This vectorization is really only valid when the basis vectors are fixed (they do not have to be orthonormal as the use of the reciprocal basis here highlights).  In a curvilinear system where the vectors vary with position, one cannot necessarily pull the \(\gamma^\mu\) into the \(d/d\tau\) derivative.

%\subsection{Performing the Euler-Lagrange calculation in the vector representation}
                 % Apr 15/09
\documentclass{article}

\usepackage{amsmath}
\usepackage{mathpazo}

%
% shorthand for bold symbols, convenient for vectors and matrices
%
\newcommand{\Ba}[0]{\mathbf{a}}
\newcommand{\Bb}[0]{\mathbf{b}}
\newcommand{\Bc}[0]{\mathbf{c}}
\newcommand{\Bd}[0]{\mathbf{d}}
\newcommand{\Be}[0]{\mathbf{e}}
\newcommand{\Bf}[0]{\mathbf{f}}
\newcommand{\Bg}[0]{\mathbf{g}}
\newcommand{\Bh}[0]{\mathbf{h}}
\newcommand{\Bi}[0]{\mathbf{i}}
\newcommand{\Bj}[0]{\mathbf{j}}
\newcommand{\Bk}[0]{\mathbf{k}}
\newcommand{\Bl}[0]{\mathbf{l}}
\newcommand{\Bm}[0]{\mathbf{m}}
\newcommand{\Bn}[0]{\mathbf{n}}
\newcommand{\Bo}[0]{\mathbf{o}}
\newcommand{\Bp}[0]{\mathbf{p}}
\newcommand{\Bq}[0]{\mathbf{q}}
\newcommand{\Br}[0]{\mathbf{r}}
\newcommand{\Bs}[0]{\mathbf{s}}
\newcommand{\Bt}[0]{\mathbf{t}}
\newcommand{\Bu}[0]{\mathbf{u}}
\newcommand{\Bv}[0]{\mathbf{v}}
\newcommand{\Bw}[0]{\mathbf{w}}
\newcommand{\Bx}[0]{\mathbf{x}}
\newcommand{\By}[0]{\mathbf{y}}
\newcommand{\Bz}[0]{\mathbf{z}}
\newcommand{\BA}[0]{\mathbf{A}}
\newcommand{\BB}[0]{\mathbf{B}}
\newcommand{\BC}[0]{\mathbf{C}}
\newcommand{\BD}[0]{\mathbf{D}}
\newcommand{\BE}[0]{\mathbf{E}}
\newcommand{\BF}[0]{\mathbf{F}}
\newcommand{\BG}[0]{\mathbf{G}}
\newcommand{\BH}[0]{\mathbf{H}}
\newcommand{\BI}[0]{\mathbf{I}}
\newcommand{\BJ}[0]{\mathbf{J}}
\newcommand{\BK}[0]{\mathbf{K}}
\newcommand{\BL}[0]{\mathbf{L}}
\newcommand{\BM}[0]{\mathbf{M}}
\newcommand{\BN}[0]{\mathbf{N}}
\newcommand{\BO}[0]{\mathbf{O}}
\newcommand{\BP}[0]{\mathbf{P}}
\newcommand{\BQ}[0]{\mathbf{Q}}
\newcommand{\BR}[0]{\mathbf{R}}
\newcommand{\BS}[0]{\mathbf{S}}
\newcommand{\BT}[0]{\mathbf{T}}
\newcommand{\BU}[0]{\mathbf{U}}
\newcommand{\BV}[0]{\mathbf{V}}
\newcommand{\BW}[0]{\mathbf{W}}
\newcommand{\BX}[0]{\mathbf{X}}
\newcommand{\BY}[0]{\mathbf{Y}}
\newcommand{\BZ}[0]{\mathbf{Z}}

\newcommand{\Bzero}[0]{\mathbf{0}}
\newcommand{\Btheta}[0]{\boldsymbol{\theta}}
\newcommand{\Btau}[0]{\boldsymbol{\tau}}
\newcommand{\Bomega}[0]{\boldsymbol{\omega}}

%
% shorthand for unit vectors
%
\newcommand{\acap}[0]{\hat{\Ba}}
\newcommand{\bcap}[0]{\hat{\Bb}}
\newcommand{\ccap}[0]{\hat{\Bc}}
\newcommand{\dcap}[0]{\hat{\Bd}}
\newcommand{\ecap}[0]{\hat{\Be}}
\newcommand{\fcap}[0]{\hat{\Bf}}
\newcommand{\gcap}[0]{\hat{\Bg}}
\newcommand{\hcap}[0]{\hat{\Bh}}
\newcommand{\icap}[0]{\hat{\Bi}}
\newcommand{\jcap}[0]{\hat{\Bj}}
\newcommand{\kcap}[0]{\hat{\Bk}}
\newcommand{\lcap}[0]{\hat{\Bl}}
\newcommand{\mcap}[0]{\hat{\Bm}}
\newcommand{\ncap}[0]{\hat{\Bn}}
\newcommand{\ocap}[0]{\hat{\Bo}}
\newcommand{\pcap}[0]{\hat{\Bp}}
\newcommand{\qcap}[0]{\hat{\Bq}}
\newcommand{\rcap}[0]{\hat{\Br}}
\newcommand{\scap}[0]{\hat{\Bs}}
\newcommand{\tcap}[0]{\hat{\Bt}}
\newcommand{\ucap}[0]{\hat{\Bu}}
\newcommand{\vcap}[0]{\hat{\Bv}}
\newcommand{\wcap}[0]{\hat{\Bw}}
\newcommand{\xcap}[0]{\hat{\Bx}}
\newcommand{\ycap}[0]{\hat{\By}}
\newcommand{\zcap}[0]{\hat{\Bz}}
\newcommand{\thetacap}[0]{\hat{\Btheta}}

%
% to write R^n and C^n in a distinguishable fashion.  Perhaps change this
% to the double lined characters upon figuring out how to do so.
%
\newcommand{\C}[1]{$\mathbb{C}^{#1}$}
\newcommand{\R}[1]{$\mathbb{R}^{#1}$}

%
% various generally useful helpers
%

% derivative of #1 wrt. #2:
\newcommand{\D}[2] {\frac {d#2} {d#1}}

\newcommand{\inv}[1]{\frac{1}{#1}}
\newcommand{\cross}[0]{\times}

\newcommand{\abs}[1]{\lvert{#1}\rvert}
\newcommand{\norm}[1]{\lVert{#1}\rVert}
\newcommand{\innerprod}[2]{\langle{#1}, {#2}\rangle}
\newcommand{\dotprod}[2]{{#1} \cdot {#2}}
\newcommand{\bdotprod}[2]{\left({#1} \cdot {#2}\right)}
\newcommand{\crossprod}[2]{{#1} \cross {#2}}
\newcommand{\tripleprod}[3]{\dotprod{\left(\crossprod{#1}{#2}\right)}{#3}}

\DeclareMathOperator{\Proj}{Proj}
\DeclareMathOperator{\Span}{span}
\DeclareMathOperator{\Sgn}{sgn}
\DeclareMathOperator{\Area}{Area}
\DeclareMathOperator{\Volume}{Volume}

%
% A few miscellaneous things specific to this document
%
\newcommand{\crossop}[1]{\crossprod{#1}{}}

% R2 vector.
\newcommand{\VectorTwo}[2]{
\begin{bmatrix}
 {#1} \\
 {#2}
\end{bmatrix}
}

\newcommand{\VectorN}[1]{
\begin{bmatrix}
{#1}_1 \\
{#1}_2 \\
\vdots \\
{#1}_N \\
\end{bmatrix}
}

\newcommand{\DETuvij}[4]{
\begin{vmatrix}
 {#1}_{#3} & {#1}_{#4} \\
 {#2}_{#3} & {#2}_{#4}
\end{vmatrix}
}

\newcommand{\DETuvwijk}[6]{
\begin{vmatrix}
 {#1}_{#4} & {#1}_{#5} & {#1}_{#6} \\
 {#2}_{#4} & {#2}_{#5} & {#2}_{#6} \\
 {#3}_{#4} & {#3}_{#5} & {#3}_{#6}
\end{vmatrix}
}

\newcommand{\DETuvwxijkl}[8]{
\begin{vmatrix}
 {#1}_{#5} & {#1}_{#6} & {#1}_{#7} & {#1}_{#8} \\
 {#2}_{#5} & {#2}_{#6} & {#2}_{#7} & {#2}_{#8} \\
 {#3}_{#5} & {#3}_{#6} & {#3}_{#7} & {#3}_{#8} \\
 {#4}_{#5} & {#4}_{#6} & {#4}_{#7} & {#4}_{#8} \\
\end{vmatrix}
}

%\newcommand{\DETuvwxyijklm}[10]{
%\begin{vmatrix}
% {#1}_{#6} & {#1}_{#7} & {#1}_{#8} & {#1}_{#9} & {#1}_{#10} \\
% {#2}_{#6} & {#2}_{#7} & {#2}_{#8} & {#2}_{#9} & {#2}_{#10} \\
% {#3}_{#6} & {#3}_{#7} & {#3}_{#8} & {#3}_{#9} & {#3}_{#10} \\
% {#4}_{#6} & {#4}_{#7} & {#4}_{#8} & {#4}_{#9} & {#4}_{#10} \\
% {#5}_{#6} & {#5}_{#7} & {#5}_{#8} & {#5}_{#9} & {#5}_{#10}
%\end{vmatrix}
%}

% R3 vector.
\newcommand{\VectorThree}[3]{
\begin{bmatrix}
 {#1} \\
 {#2} \\
 {#3}
\end{bmatrix}
}


%<misc>
%
\newcommand{\Abs}[1]{{\left\lvert{#1}\right\rvert}}
\newcommand{\spacegrad}[0]{\boldsymbol{\nabla}}
\newcommand{\grad}[0]{\nabla}
\newcommand{\LL}[0]{\mathcal{L}}

% == \partial_{#1} {#2}
\newcommand{\PD}[2]{\frac{\partial {#2}}{\partial {#1}}}
% inline variant
\newcommand{\PDi}[2]{{\partial {#2}}/{\partial {#1}}}

\newcommand{\PDD}[3]{\frac{\partial^2 {#3}}{\partial {#1}\partial {#2}}}
%\newcommand{\PDd}[2]{\frac{\partial^2 {#2}}{{\partial{#1}}^2}}
\newcommand{\PDsq}[2]{\frac{\partial^2 {#2}}{(\partial {#1})^2}}

\newcommand{\Partial}[2]{\frac{\partial {#1}}{\partial {#2}}}
\DeclareMathOperator{\RejName}{Rej}
\newcommand{\Rej}[2]{\RejName_{#1}\left( {#2} \right)}
\newcommand{\Rm}[1]{\mathbb{R}^{#1}}
\newcommand{\Cm}[1]{\mathbb{C}^{#1}}
\newcommand{\conj}[0]{{*}}

%</misc>

% <grade selection>
%
\newcommand{\gpgrade}[2] {{\left\langle{{#1}}\right\rangle}_{#2}}

\newcommand{\gpgradezero}[1] {\gpgrade{#1}{}}
%\newcommand{\gpscalargrade}[1] {{\left\langle{{#1}}\right\rangle}}
%\newcommand{\gpgradezero}[1] {\gpgrade{#1}{0}}

%\newcommand{\gpgradeone}[1] {{\left\langle{{#1}}\right\rangle}_{1}}
\newcommand{\gpgradeone}[1] {\gpgrade{#1}{1}}

\newcommand{\gpgradetwo}[1] {\gpgrade{#1}{2}}
\newcommand{\gpgradethree}[1] {\gpgrade{#1}{3}}
\newcommand{\gpgradefour}[1] {\gpgrade{#1}{4}}
%
% </grade selection>



\newcommand{\adot}[0]{{\dot{a}}}
\newcommand{\bdot}[0]{{\dot{b}}}
% taken for centered dot:
%\newcommand{\cdot}[0]{{\dot{c}}}
%\newcommand{\ddot}[0]{{\dot{d}}}
\newcommand{\edot}[0]{{\dot{e}}}
\newcommand{\fdot}[0]{{\dot{f}}}
\newcommand{\gdot}[0]{{\dot{g}}}
\newcommand{\hdot}[0]{{\dot{h}}}
\newcommand{\idot}[0]{{\dot{i}}}
\newcommand{\jdot}[0]{{\dot{j}}}
\newcommand{\kdot}[0]{{\dot{k}}}
\newcommand{\ldot}[0]{{\dot{l}}}
\newcommand{\mdot}[0]{{\dot{m}}}
\newcommand{\ndot}[0]{{\dot{n}}}
%\newcommand{\odot}[0]{{\dot{o}}}
\newcommand{\pdot}[0]{{\dot{p}}}
\newcommand{\qdot}[0]{{\dot{q}}}
\newcommand{\rdot}[0]{{\dot{r}}}
\newcommand{\sdot}[0]{{\dot{s}}}
\newcommand{\tdot}[0]{{\dot{t}}}
\newcommand{\udot}[0]{{\dot{u}}}
\newcommand{\vdot}[0]{{\dot{v}}}
\newcommand{\wdot}[0]{{\dot{w}}}
\newcommand{\xdot}[0]{{\dot{x}}}
\newcommand{\ydot}[0]{{\dot{y}}}
\newcommand{\zdot}[0]{{\dot{z}}}
\newcommand{\addot}[0]{{\ddot{a}}}
\newcommand{\bddot}[0]{{\ddot{b}}}
\newcommand{\cddot}[0]{{\ddot{c}}}
%\newcommand{\dddot}[0]{{\ddot{d}}}
\newcommand{\eddot}[0]{{\ddot{e}}}
\newcommand{\fddot}[0]{{\ddot{f}}}
\newcommand{\gddot}[0]{{\ddot{g}}}
\newcommand{\hddot}[0]{{\ddot{h}}}
\newcommand{\iddot}[0]{{\ddot{i}}}
\newcommand{\jddot}[0]{{\ddot{j}}}
\newcommand{\kddot}[0]{{\ddot{k}}}
\newcommand{\lddot}[0]{{\ddot{l}}}
\newcommand{\mddot}[0]{{\ddot{m}}}
\newcommand{\nddot}[0]{{\ddot{n}}}
\newcommand{\oddot}[0]{{\ddot{o}}}
\newcommand{\pddot}[0]{{\ddot{p}}}
\newcommand{\qddot}[0]{{\ddot{q}}}
\newcommand{\rddot}[0]{{\ddot{r}}}
\newcommand{\sddot}[0]{{\ddot{s}}}
\newcommand{\tddot}[0]{{\ddot{t}}}
\newcommand{\uddot}[0]{{\ddot{u}}}
\newcommand{\vddot}[0]{{\ddot{v}}}
\newcommand{\wddot}[0]{{\ddot{w}}}
\newcommand{\xddot}[0]{{\ddot{x}}}
\newcommand{\yddot}[0]{{\ddot{y}}}
\newcommand{\zddot}[0]{{\ddot{z}}}

%<bold and dot greek symbols>
%

\newcommand{\Deltadot}[0]{{\dot{\Delta}}}
\newcommand{\Gammadot}[0]{{\dot{\Gamma}}}
\newcommand{\Lambdadot}[0]{{\dot{\Lambda}}}
\newcommand{\Omegadot}[0]{{\dot{\Omega}}}
\newcommand{\Phidot}[0]{{\dot{\Phi}}}
\newcommand{\Pidot}[0]{{\dot{\Pi}}}
\newcommand{\Psidot}[0]{{\dot{\Psi}}}
\newcommand{\Sigmadot}[0]{{\dot{\Sigma}}}
\newcommand{\Thetadot}[0]{{\dot{\Theta}}}
\newcommand{\Upsilondot}[0]{{\dot{\Upsilon}}}
\newcommand{\Xidot}[0]{{\dot{\Xi}}}
\newcommand{\alphadot}[0]{{\dot{\alpha}}}
\newcommand{\betadot}[0]{{\dot{\beta}}}
\newcommand{\chidot}[0]{{\dot{\chi}}}
\newcommand{\deltadot}[0]{{\dot{\delta}}}
\newcommand{\epsilondot}[0]{{\dot{\epsilon}}}
\newcommand{\etadot}[0]{{\dot{\eta}}}
\newcommand{\gammadot}[0]{{\dot{\gamma}}}
\newcommand{\kappadot}[0]{{\dot{\kappa}}}
\newcommand{\lambdadot}[0]{{\dot{\lambda}}}
\newcommand{\mudot}[0]{{\dot{\mu}}}
\newcommand{\nudot}[0]{{\dot{\nu}}}
\newcommand{\omegadot}[0]{{\dot{\omega}}}
\newcommand{\phidot}[0]{{\dot{\phi}}}
\newcommand{\pidot}[0]{{\dot{\pi}}}
\newcommand{\psidot}[0]{{\dot{\psi}}}
\newcommand{\rhodot}[0]{{\dot{\rho}}}
\newcommand{\sigmadot}[0]{{\dot{\sigma}}}
\newcommand{\taudot}[0]{{\dot{\tau}}}
\newcommand{\thetadot}[0]{{\dot{\theta}}}
\newcommand{\upsilondot}[0]{{\dot{\upsilon}}}
\newcommand{\varepsilondot}[0]{{\dot{\varepsilon}}}
\newcommand{\varphidot}[0]{{\dot{\varphi}}}
\newcommand{\varpidot}[0]{{\dot{\varpi}}}
\newcommand{\varrhodot}[0]{{\dot{\varrho}}}
\newcommand{\varsigmadot}[0]{{\dot{\varsigma}}}
\newcommand{\varthetadot}[0]{{\dot{\vartheta}}}
\newcommand{\xidot}[0]{{\dot{\xi}}}
\newcommand{\zetadot}[0]{{\dot{\zeta}}}

\newcommand{\Deltaddot}[0]{{\ddot{\Delta}}}
\newcommand{\Gammaddot}[0]{{\ddot{\Gamma}}}
\newcommand{\Lambdaddot}[0]{{\ddot{\Lambda}}}
\newcommand{\Omegaddot}[0]{{\ddot{\Omega}}}
\newcommand{\Phiddot}[0]{{\ddot{\Phi}}}
\newcommand{\Piddot}[0]{{\ddot{\Pi}}}
\newcommand{\Psiddot}[0]{{\ddot{\Psi}}}
\newcommand{\Sigmaddot}[0]{{\ddot{\Sigma}}}
\newcommand{\Thetaddot}[0]{{\ddot{\Theta}}}
\newcommand{\Upsilonddot}[0]{{\ddot{\Upsilon}}}
\newcommand{\Xiddot}[0]{{\ddot{\Xi}}}
\newcommand{\alphaddot}[0]{{\ddot{\alpha}}}
\newcommand{\betaddot}[0]{{\ddot{\beta}}}
\newcommand{\chiddot}[0]{{\ddot{\chi}}}
\newcommand{\deltaddot}[0]{{\ddot{\delta}}}
\newcommand{\epsilonddot}[0]{{\ddot{\epsilon}}}
\newcommand{\etaddot}[0]{{\ddot{\eta}}}
\newcommand{\gammaddot}[0]{{\ddot{\gamma}}}
\newcommand{\kappaddot}[0]{{\ddot{\kappa}}}
\newcommand{\lambdaddot}[0]{{\ddot{\lambda}}}
\newcommand{\muddot}[0]{{\ddot{\mu}}}
\newcommand{\nuddot}[0]{{\ddot{\nu}}}
\newcommand{\omegaddot}[0]{{\ddot{\omega}}}
\newcommand{\phiddot}[0]{{\ddot{\phi}}}
\newcommand{\piddot}[0]{{\ddot{\pi}}}
\newcommand{\psiddot}[0]{{\ddot{\psi}}}
\newcommand{\rhoddot}[0]{{\ddot{\rho}}}
\newcommand{\sigmaddot}[0]{{\ddot{\sigma}}}
\newcommand{\tauddot}[0]{{\ddot{\tau}}}
\newcommand{\thetaddot}[0]{{\ddot{\theta}}}
\newcommand{\upsilonddot}[0]{{\ddot{\upsilon}}}
\newcommand{\varepsilonddot}[0]{{\ddot{\varepsilon}}}
\newcommand{\varphiddot}[0]{{\ddot{\varphi}}}
\newcommand{\varpiddot}[0]{{\ddot{\varpi}}}
\newcommand{\varrhoddot}[0]{{\ddot{\varrho}}}
\newcommand{\varsigmaddot}[0]{{\ddot{\varsigma}}}
\newcommand{\varthetaddot}[0]{{\ddot{\vartheta}}}
\newcommand{\xiddot}[0]{{\ddot{\xi}}}
\newcommand{\zetaddot}[0]{{\ddot{\zeta}}}

\newcommand{\BDelta}[0]{\boldsymbol{\Delta}}
\newcommand{\BGamma}[0]{\boldsymbol{\Gamma}}
\newcommand{\BLambda}[0]{\boldsymbol{\Lambda}}
\newcommand{\BOmega}[0]{\boldsymbol{\Omega}}
\newcommand{\BPhi}[0]{\boldsymbol{\Phi}}
\newcommand{\BPi}[0]{\boldsymbol{\Pi}}
\newcommand{\BPsi}[0]{\boldsymbol{\Psi}}
\newcommand{\BSigma}[0]{\boldsymbol{\Sigma}}
\newcommand{\BTheta}[0]{\boldsymbol{\Theta}}
\newcommand{\BUpsilon}[0]{\boldsymbol{\Upsilon}}
\newcommand{\BXi}[0]{\boldsymbol{\Xi}}
\newcommand{\Balpha}[0]{\boldsymbol{\alpha}}
\newcommand{\Bbeta}[0]{\boldsymbol{\beta}}
\newcommand{\Bchi}[0]{\boldsymbol{\chi}}
\newcommand{\Bdelta}[0]{\boldsymbol{\delta}}
\newcommand{\Bepsilon}[0]{\boldsymbol{\epsilon}}
\newcommand{\Beta}[0]{\boldsymbol{\eta}}
\newcommand{\Bgamma}[0]{\boldsymbol{\gamma}}
\newcommand{\Bkappa}[0]{\boldsymbol{\kappa}}
\newcommand{\Blambda}[0]{\boldsymbol{\lambda}}
\newcommand{\Bmu}[0]{\boldsymbol{\mu}}
\newcommand{\Bnu}[0]{\boldsymbol{\nu}}
%\newcommand{\Bomega}[0]{\boldsymbol{\omega}}
\newcommand{\Bphi}[0]{\boldsymbol{\phi}}
\newcommand{\Bpi}[0]{\boldsymbol{\pi}}
\newcommand{\Bpsi}[0]{\boldsymbol{\psi}}
\newcommand{\Brho}[0]{\boldsymbol{\rho}}
\newcommand{\Bsigma}[0]{\boldsymbol{\sigma}}
%\newcommand{\Btau}[0]{\boldsymbol{\tau}}
%\newcommand{\Btheta}[0]{\boldsymbol{\theta}}
\newcommand{\Bupsilon}[0]{\boldsymbol{\upsilon}}
\newcommand{\Bvarepsilon}[0]{\boldsymbol{\varepsilon}}
\newcommand{\Bvarphi}[0]{\boldsymbol{\varphi}}
\newcommand{\Bvarpi}[0]{\boldsymbol{\varpi}}
\newcommand{\Bvarrho}[0]{\boldsymbol{\varrho}}
\newcommand{\Bvarsigma}[0]{\boldsymbol{\varsigma}}
\newcommand{\Bvartheta}[0]{\boldsymbol{\vartheta}}
\newcommand{\Bxi}[0]{\boldsymbol{\xi}}
\newcommand{\Bzeta}[0]{\boldsymbol{\zeta}}
%
%</bold and dot greek symbols>
%<infrequent>
%
%\newcommand{\AreaOp}[1]{\AName_{#1}}
%\newcommand{\Babs}[0]{\abs{\BB}}
%\newcommand{\Bcap}[0]{\hat{\BB}}
%\newcommand{\BrPrimeRej}[0]{\rcap(\rcap \wedge \Br')}
%\newcommand{\CA}[0]{\mathcal{A}}
%\newcommand{\Cos}[1]{\cos{\left({#1}\right)}}
%\newcommand{\Det}[1] {\abs{#1}}
%\newcommand{\Dsq}[2] {\frac {\partial^2 {#1}} {\partial {#2}^2}}
%\newcommand{\Exp}[1]{\exp{\left({#1}\right)}}
%\newcommand{\Norm}[1]{\left\lVert{#1}\right\rVert}
%\newcommand{\Sin}[1]{\sin{\left({#1}\right)}}
%\newcommand{\T}[0]{\text{T}}
%\newcommand{\VolumeOp}[1]{\VName_{#1}}
%\newcommand{\agrad}[0]{\Ba \cdot \nabla}
%\newcommand{\alphacap}[0]{\hat{\boldsymbol{\alpha}}}
%\newcommand{\Fcap}[0]{\hat{\BF}}
%\newcommand{\bithree}[0]{{\Bi}_3}
%\newcommand{\bxa}[0]{\Bx\Ba}
%\newcommand{\coordvec}[2]{
%\newcommand{\costheta}[0]{\acap \cdot \xcap}
%\newcommand{\ddt}[1]{\ddot{#1}}
%\newcommand{\ddu}[1] {\frac {d{#1}} {du}}
%\newcommand{\dsqxj}[2] {\frac {\partial^2 {#1}} {\partial {x_{#2}}^2}}
%\newcommand{\dtheta}[1]{\frac{d {#1}}{d \theta}}
%\newcommand{\dt}[1]{\dot{#1}}
%\newcommand{\dt}[1]{\frac{d {#1}}{dt}}
%\newcommand{\dxj}[2] {\frac {\partial {#1}} {\partial {x_{#2}}}}
%\newcommand{\halfPhi}[0]{\frac{\phi}{2}}
%\newcommand{\half}[0]{\inv{2}}
%\newcommand{\inv}[1]{\frac{1}{#1}}
%\newcommand{\laplacian}[0]{\nabla^2}
%\newcommand{\matrixoftx}[3]{
%\newcommand{\nrrp}[0]{\norm{\rcap \wedge \Br'}}
%\newcommand{\oiint}{\bigcirc \hspace{-1.4em} \int \hspace{-.8em} \int}
%\newcommand{\transpose}[1]{{#1}^{\text{T}}}
%\newcommand{\transpose}[1]{{{#1}^{\TextTranspose}}}
%\newcommand{\transpose}[1]{{{#1}^{\text{T}}}}
%\newcommand{\barA}[0]{\bar{A}}
%\newcommand{\qbar}[0]{\bar{q}}
%\newcommand{\qdotbar}[0]{\dot{\bar{q}}}
%
%</infrequent>





%\usepackage{listings}
%\usepackage{txfonts} % for ointctr... (also appears to make "prettier" \int and \sum's)
\usepackage[bookmarks=true]{hyperref}

\usepackage{color,cite,graphicx}
   % use colour in the document, put your citations as [1-4]
   % rather than [1,2,3,4] (it looks nicer, and the extended LaTeX2e
   % graphics package. 
\usepackage{latexsym,amssymb,epsf} % don't remember if these are
   % needed, but their inclusion can't do any damage


\title{ Tensor derivation of Maxwell equation (non-dual part) from Lagrangian. }
\author{Peeter Joot \quad peeter.joot@gmail.com }
\date{ April 20, 2009.  Last Revision: $Date: 2009/04/21 03:37:35 $ }

\begin{document}

\maketitle{}
\tableofcontents
\section{ Motivation. }

Looking through my notes for a purely tensor derivation of Maxwell's equation, and not finding one.  Have done this on
paper a number of times, but writing it up once for reference to refer to for signs will be useful.

\section{ Lagrangian. }

Notes containing derivations of Maxwell's equation

\begin{align}
\grad F = J/\epsilon_0 c
\end{align}

From the Lagrangian

\begin{align}\label{eqn:maxlag}
\LL &= -\frac{\epsilon_0}{2} (\grad \wedge A)^2 + \frac{J}{c} \cdot A
\end{align}

can be found in \cite{PJFieldLagrangian}, and the earlier \cite{PJMaxwellLagrangian}.

We will work from the scalar part of this Lagrangian, expressed strictly in tensor form

\begin{align}
\LL &= \frac{\epsilon_0}{4} F_{\mu\nu}F^{\mu\nu} + \inv{c} J_\mu \cdot A^\mu
\end{align}

\section{ Calculation. }

In preparation, an expansion of the Faraday tensor in terms of potentials is desirable

\begin{align*}
F_{\mu\nu}F^{\mu\nu}
&=
(\partial_\mu A_\nu - \partial_\nu A_\mu) (\partial^\mu A^\nu - \partial^\nu A^\mu) \\
&=
\partial_\mu A_\nu \partial^\mu A^\nu 
-\partial_\mu A_\nu \partial^\nu A^\mu
- \partial_\nu A_\mu \partial^\mu A^\nu 
+ \partial_\nu A_\mu \partial^\nu A^\mu \\
&=
2 (\partial_\mu A_\nu \partial^\mu A^\nu - \partial_\mu A_\nu \partial^\nu A^\mu)
\\
\end{align*}

So we have
\begin{align*}
\LL &= \frac{\epsilon_0}{2} \partial_\mu A_\nu (\partial^\mu A^\nu - \partial^\nu A^\mu)
+ \inv{c} J_\mu \cdot A^\mu
\end{align*}

\bibliographystyle{plainnat}
\bibliography{myrefs}

\end{document}
     % Apr 20/09
%
% Copyright � 2012 Peeter Joot.  All Rights Reserved.
% Licenced as described in the file LICENSE under the root directory of this GIT repository.
%

% 
% 
\chapter{Canonical energy momentum tensor and Lagrangian translation}
\label{chap:stressEnergyNoethers}
%\date{June 5, 2009.  stressEnergyNoethers.tex}

\section{Motivation and direction}

In \citep{gabook:PJstressEnergyLorentz} we saw that it was possible to express the Lorentz force equation for the charge per unit volume in terms of the energy momentum tensor.

Repeating %from \eqnref{eqn:seLorentz:lorentzForceT}
% and
%\eqnref{eqn:seLorentz:lorentzForceT} ???

\begin{align}
\grad \cdot T(\gamma_\mu) &= \inv{c} \gpgradezero{ F \gamma_\mu J } \\
T(a) &= \frac{\epsilon_0}{2} F a \tilde{F}
\end{align}

While these may not appear too much like the Lorentz force equation as we are used to seeing it, with some manipulation we found %\eqnref{eqn:seLorentz:lorentzForcePair}

\begin{align}
\inv{c} \gpgradezero{ F \gamma_0 J } &= -\Bj \cdot \BE \\
\inv{c} \gpgradezero{ F \gamma_k J } &= (\rho \BE + \Bj \cross \BB) \cdot \sigma_k
\end{align}

where we now have an energy momentum pair of equations, the second
of which if integrated over a volume is the Lorentz force for the charge
in that volume.

We have also seen 
%in
%\eqnref{eqn:seLorentz:lorentzForceGA}
that we can express the Lorentz force equation in GA form
\begin{align}
m \ddot{x} &= q F \cdot \dot{x}/c
\end{align}

%In
%\eqnref{eqn:seLorentz:lorentzForceTensor}
This was expressed in tensor form, toggling indices that was
\begin{align}
m \ddot{x}_\mu &= {q} F_{\mu\alpha} \dot{x}^\alpha
\end{align}

We then saw in \citep{gabook:PJenMtensor} 
%in equations \eqnref{eqn:stressEnTen:miracle} and
%\eqnref{eqn:stressEnTen:covariantTensor}
that the
the covariant form of the energy momentum tensor relation was

\begin{align}
T^{\mu\nu} &= {\epsilon_0} \left( F^{\alpha\mu} {F^{\nu}}_{\alpha} + \inv{4} F^{\alpha\beta} F_{\alpha\beta} \eta^{\mu\nu} \right) \\
\partial_\nu T^{\mu\nu} &= F^{\alpha\mu} J_\alpha/c
\end{align}

this has identical structure (FIXME: sign error here?) to the
covariant Lorentz force equation.

Now the energy momentum conservation equations above did not require
the Lorentz force equations at all for their derivation, nor have we
used the Lorentz force interaction Lagrangian to arrive at them.
With Maxwell's equation and the Lorentz force equation together (
or the equivalent field and interaction Lagrangians) we have
the complete specification of classical electrodynamics.  Curiously
it
appears that we have most of the structure of the Lorentz force equation
(except for the association with mass) all in embedded in Maxwell's equation
or the Maxwell field Lagrangian.

Now, a proper treatment of the field and charged mass interaction
likely requires the Dirac Lagrangian, and hiding in there if one
could extract it, is probably everything that could be said on the
topic.  It will be a long journey to get to that point, but how
much can we do considering just the field Lagrangian?

For these reasons it seems desirable to
understand the background behind the energy momentum tensor much better.
In particular, it is natural to then expect that these conservation
relations may also be found as a
consequence of a symmetry and an associated Noether current (see \chapcite{PJNoethersField}).  What
is that symmetry?  That symmetry should leave the field equations as
calculated by the field Euler-Lagrange equations
Given that symmetry how would one go about
actually showing that this is the case?   These are the questions
to tackle here.

\section{On translation and divergence symmetries}

\subsection{Symmetry due to total derivative addition to the Lagrangian}

In \citep{doran2003gap} the energy momentum tensor is treated
by considering spacetime translation, but I have unfortunately
not understood much more than vague direction in that treatment.

In \citep{srednicki2007qft} it is also stated that the energy momentum tensor
is the result of a Lagrangian spacetime translation, but I did not find
details there.

There are examples
of the canonical energy momentum tensor (in the simpler non-GA tensor form)
and the symmetric energy momentum tensor in
\citep{jackson1975cew}.  However, that treatment relies on analogy with
mechanical form of Noether's theorem, and I had rather see it developed
explicitly.

Finally, in an unexpected place (since I am not studying QFT but was merely curious), the
clue required to understand the details of
how this spacetime translation results in the energy momentum tensor was found in
\citep{TongQFT}.
%  Also helpful was problem on exponential operator form of taylor series
% in byram & ...

In Tong's treatment it is pointed out there is a symmetry for the Lagrangian
if it is altered by a divergence.

\begin{align*}
\LL \rightarrow \LL + \partial_\mu F^\mu
\end{align*}

It took me a while to figure out how this was a symmetry, but after a
nice refreshing motorcycle ride, the answer suddenly surfaced.  One
can add a derivative to a mechanical Lagrangian and not change
the resulting equations of motion.  While tackling problem 5 of
Tong's mechanics in \chapcite{PJTongMf1}, such an invariance was considered in
detail in one of the problems for Tong's classical mechanics notes
\chapcite{PJTongMf1:addDerivative}
.

%FOLLOWUP: No Noether current was
%derived for this sort of mechanical symmetry.  From
%what I have read, such a Noether's current calculation is
%likely to provide a demonstration of linear momentum conservation.  Worth
%verifying.
%
If one has altered the Lagrangian by adding an arbitrary function $f$
to it.

\begin{align*}
\LL' = \LL + f
\end{align*}

Assuming to start a Lagrangian that is a function of a single field variable
$\LL = \LL(\phi, \partial_\mu \phi)$, then the
variation of the Lagrangian for the field equations yields

\begin{align*}
\frac{\delta \LL'}{\delta \phi}
&=
\frac{\partial \LL'}{\partial \phi}
-
\partial_\sigma \frac{\partial \LL'}{\partial (\partial_\sigma \phi)} \\
&=
\mathLabelBox
[
   labelstyle={below of=m\themathLableNode, below of=m\themathLableNode}
]
{
\frac{\partial \LL}{\partial \phi}
-
\partial_\sigma \frac{\partial \LL}{\partial (\partial_\sigma \phi)} 
}{$=0$}
+\frac{\partial f}{\partial \phi}
-
\partial_\sigma \frac{\partial f}{\partial (\partial_\sigma \phi)} \\
\end{align*}

So, if this transformed Lagrangian is a symmetry, it is sufficient to
find the conditions for the variation of additional part to be zero

\begin{align}\label{eqn:stressEnergyNoethers:requirement}
\frac{\delta f}{\delta \phi} = 0
\end{align}

\subsection{Some examples adding a divergence}

To validate the fact that we can add a divergence to the Lagrangian without
changing the field equations lets work out a few concrete examples
of \eqnref{eqn:stressEnergyNoethers:requirement} of for Lagrangian alterations by a divergence
$f = \partial_\mu F^\mu$.

Each of these examples will be for a single field variable Lagrangian
with generalized coordinates $x^1 = x$, and $x^1 = y$.

\subsubsection{Simplest case.  No partials}

Let

\begin{align*}
F^1 &= \phi \\
F^2 &= 0 \\
\end{align*}

With this the divergence is

\begin{align*}
f
&= \partial_x F^x + \partial_y F^y  \\
&=
\PD{x}{\phi}
\end{align*}

Now the variation is
\begin{align*}
\frac{\delta f}{\delta \phi}
&=
\left( \PD{\phi}{}
- \PD{x}{}\PD{(\PDi{x}{\phi})}{}
- \PD{y}{}\PD{(\PDi{y}{\phi})}{}
\right) \PD{x}{\phi} \\
&=
\PD{x}{}\PD{\phi}{\phi} - \PD{x}{1} \\
&= 0
\end{align*}

Okay, so far so good.

\subsubsection{One partial}

Now, let

\begin{align*}
F^1 &= \PD{x}{\phi} \\
F^2 &= 0 \\
\end{align*}

With this the divergence is

\begin{align*}
f
&= \partial_x F^x + \partial_y F^y  \\
&=
\PD{x}{}\PD{x}{\phi}
\end{align*}

And the variation is
\begin{align*}
\frac{\delta f}{\delta \phi}
&=
\left( \PD{\phi}{}
- \PD{x}{}\PD{(\PDi{x}{\phi})}{}
- \PD{y}{}\PD{(\PDi{y}{\phi})}{}
\right) \PD{x}{}\PD{x}{\phi} \\
&=
\PD{\phi}{} \PD{x}{}\PD{x}{\phi} \\
&=
\PD{x}{} \PD{x}{}\PD{\phi}{\phi} \\
&=
\PD{x}{} \PD{x}{1} \\
&= 0
\end{align*}

Again, assuming I am okay to switch the differentiation order, we have zero.

\subsubsection{Another partial}

For the last concrete example before going on to the general case, try

\begin{align*}
F^1 &= \PD{y}{\phi} \\
F^2 &= 0 \\
\end{align*}

The divergence is

\begin{align*}
f
&= \partial_x F^x + \partial_y F^y  \\
&=
\PD{x}{}\PD{y}{\phi}
\end{align*}

And the variation is
\begin{align*}
\frac{\delta f}{\delta \phi}
&=
\left( \PD{\phi}{}
- \PD{x}{}\PD{(\PDi{x}{\phi})}{}
- \PD{y}{}\PD{(\PDi{y}{\phi})}{}
\right) \PD{x}{}\PD{y}{\phi} \\
&=
- \PD{y}{}\PD{x}{1} \\
&= 0
\end{align*}

\subsubsection{The general case}

Because of linearity we have now seen that we can construct functions with
any linear combinations of first and second derivatives

\begin{align*}
F^\mu = a^\mu \phi + \sum_\sigma {b_\sigma}^\mu \PD{x^\sigma}{\phi}
\end{align*}

and for such a function we will have

\begin{align*}
\frac{\delta (\partial_\mu F^\mu)}{\delta \phi}  = 0
\end{align*}

How general can the function $F^\mu = F^\mu(\phi, \partial_\sigma \phi)$ be
made and still yield a zero variational derivative?

To answer this, let us compute the derivative for a general divergence
added to a single field variable Lagrangian.  This is

\begin{align*}
\frac{\delta (\partial_\mu F^\mu)}{\delta \phi}
&=
\sum_\mu \left( \PD{\phi}{}
- \sum_\sigma \PD{x^\sigma}{}\PD{(\PDi{x^\sigma}{\phi})}{}
\right) \PD{x^\mu}{F^\mu} \\
&=
\sum_\mu \PD{x^\mu}{} \PD{\phi}{F^\mu}
- \sum_{\mu,\sigma} \PD{x^\sigma}{}\PD{(\PDi{x^\sigma}{\phi})}{}
\left(
\PD{\phi}{F^\mu} \PD{x^\mu}{\phi}
+ \sum_\alpha \PD{(\PDi{x^\alpha}{\phi})}{F^\mu} \PD{x^\mu}{(\PDi{x^\alpha}{\phi})} \right) \\
&=
\partial_\mu \PD{\phi}{F^\mu}
- \partial_\sigma \PD{(\partial_\sigma \phi)}{}
\left(
\PD{\phi}{F^\mu} \partial_\mu {\phi}
+ \PD{(\partial_\alpha \phi)}{F^\mu} \partial_{\mu\alpha} \phi \right) \\
\end{align*}

For tractability in this last line the shorthand for the partials has been injected.
Sums over $\alpha$, $\mu$, and $\sigma$ are also now implied (this was made explicit prior to
this in all cases where upper and lower indices were matched).

Treating these two last derivatives separately, we have for the first

\begin{align*}
\partial_\sigma
\PD{(\partial_\sigma \phi)}{}
\PD{\phi}{F^\mu} \partial_\mu {\phi}
&=
\partial_\sigma
\left(\PD{(\partial_\sigma \phi)}{} \PD{\phi}{F^\mu} \right) \partial_\mu {\phi}
+
\partial_\sigma
\PD{\phi}{F^\mu} \PD{(\partial_\sigma \phi)}{} \partial_\mu {\phi} \\
&=
\partial_\sigma
\left(\PD{(\partial_\sigma \phi)}{} \PD{\phi}{F^\mu} \right) \partial_\mu {\phi}
+
\partial_\mu \PD{\phi}{F^\mu}
\end{align*}

So our $\PDi{\phi}{F^\mu}$'s cancel out, and we are left with

\begin{align*}
\frac{\delta (\partial_\mu F^\mu)}{\delta \phi}
&=
-\partial_\sigma
\left(
\left(\PD{(\partial_\sigma \phi)}{} \PD{\phi}{F^\mu} \right) \partial_\mu {\phi}
+
\PD{(\partial_\sigma \phi)}{}
\left(
\PD{(\partial_\alpha \phi)}{F^\mu} \partial_{\mu\alpha} \phi
\right)
\right)
\\
&=
-\partial_\sigma
\left(
\partial_\mu {\phi}
\left(\PD{(\partial_\sigma \phi)}{} \PD{\phi}{F^\mu} \right)
+
\partial_{\mu\alpha} \phi
\PD{(\partial_\sigma \phi)}{}
\left(
\PD{(\partial_\alpha \phi)}{F^\mu}
\right)
\right)
\\
&=
-\partial_\sigma
\left(
(\partial_\mu {\phi})
\PD{\phi}{}
\PD{(\partial_\sigma \phi)}{}
F^\mu
+
\left(\partial_{\mu}
\PD{x^\alpha}{\phi}\right)
\PD{(\partial_\alpha \phi)}{}
\PD{(\partial_\sigma \phi)}{}
{F^\mu}
\right)
\\
\end{align*}

Now there is a lot of indices and derivatives floating around.  Writing $g^\mu = \PDi{(\partial_\sigma \phi)}{F^\mu}$, we have something a bit easier to look at

\begin{align*}
\frac{\delta (\partial_\mu F^\mu)}{\delta \phi}
&=
-\partial_\sigma
\left(
(\partial_\mu {\phi})
\PD{\phi}{ g^\mu }
+
\left(\partial_{\mu}
\PD{x^\alpha}{\phi}\right)
\PD{(\partial_\alpha \phi)}{ g^\mu }
\right)
\\
\end{align*}

But this is a chain rule expansion of the
derivative $\partial_\mu g^\mu$

\begin{align*}
\PD{x^\mu}{g^\mu} &=
\PD{x^\mu}{\phi}\PD{\phi}{g^\mu}
+ \PD{x^\mu}{\partial_\beta \phi}\PD{\partial_\beta \phi}{g^\mu}
\end{align*}

So, we finally have

\begin{align*}
\frac{\delta (\partial_\mu F^\mu)}{\delta \phi}
&=
-\partial_{\sigma \mu} g^\mu \\
&=
\end{align*}

This is
\begin{align}\label{eqn:stressEnergyNoethers:requiredZeroForSymmetry}
\frac{\delta (\partial_\mu F^\mu)}{\delta \phi}
&=
-\partial_{\sigma \mu} \PD{(\partial_\sigma \phi)}{F^\mu}
\end{align}

I do not think we have any right asserting that this is zero for arbitrary $F^\mu$.  However
if the Taylor expansion of $F^\mu$ with respect to variables $\phi$, and $\partial_\sigma \phi$ has
no higher than first order terms in the field variables $\partial_\sigma \phi$, we will
certainly have a zero variational derivative and a corresponding symmetry.

\subsubsection{More examples to confirm the symmetry requirements}

As a confirmation that a zero in \eqnref{eqn:stressEnergyNoethers:requiredZeroForSymmetry} requires linear field derivatives,
lets lets try two more example calculations.

First with non-linear
powers of $\phi$ to show that we have more freedom to construct the function first powers.  Let

\begin{align*}
F^1 &= \phi^2 \\
F^2 &= 0
\end{align*}

We have

\begin{align*}
\frac{\delta (\partial_\mu F^\mu)}{\delta \phi}
&=
\left(\PD{\phi}{} - \partial_\sigma \PD{(\partial_\sigma \phi)}{}\right) 2 \phi \phi_x \\
&=
2 \phi_x - \partial_x (2 \phi) \\
&= 0
\end{align*}

Zero as expected.  Generalizing the function to include arbitrary polynomial powers is no harder.

Let
\begin{align*}
F^1 &= \phi^k \\
F^2 &= 0 \\
\partial_\mu F^\mu &= k \phi^{k-1} \phi_x \\
\end{align*}

So we have
\begin{align*}
\frac{\delta (\partial_\mu F^\mu)}{\delta \phi}
&=
k (k-1) \phi^{k-2} \phi_x - \partial_x (k \phi^{k-1})  \\
&= 0
\end{align*}

Okay, now moving on to the derivatives.  Picking a divergence that should not will not generate a
symmetry, something with a non-linear derivative should do the trick.  Let us Try

\begin{align*}
F^1 &= (\phi_x)^2 \\
F^2 &= 0
\end{align*}

\begin{align*}
\frac{\delta (\partial_\mu F^\mu)}{\delta \phi}
&=
\left(\PD{\phi}{} - \partial_\sigma \PD{(\partial_\sigma \phi)}{}\right) 2 \phi_x \phi_{xx} \\
&=
- 2 \partial_x \phi_{xx} \\
&=
- 2 \phi_{xxx} \\
\end{align*}

So, sure enough, unless additional conditions can be imposed on $\phi$, such a transformation
will not be a symmetry.

\subsection{Symmetry for Wave equation under spacetime translation}

The Lagrangian for a one dimensional wave equation is

\begin{align}\label{eqn:stressEnergyNoethers:oneDimWave}
\LL =
\inv{2 v^2} \left(\PD{t}{\phi}\right)^2 - \inv{2} \left(\PD{x}{\phi}\right)^2
\end{align}

Under a transformation of variables

\begin{align*}
x &\rightarrow x' = x + a \\
t &\rightarrow t' = t + \tau
\end{align*}

Employing a multivariable Taylor expansion
(see
\citep{gabook:PJmultiTaylors}
)
for our Lagrangian having no explicit dependence on $t$ and $x$, we have

\begin{align*}
\LL' &= \LL + 
\mathLabelBox{(a \partial_x + \tau \partial_t)\LL}{$\conj$}
+ \cdots
\end{align*}

That first order term of the Taylor expansion $\conj$,
can be written as a divergence $\partial_\mu F^\mu$, with $F^1 = a \LL$, and $F^2 = \tau\LL$, however
both of these are quadratic in $\phi_x$, and $\phi_t$, which is not linear.
That linearity in the derivatives was required for \eqnref{eqn:stressEnergyNoethers:requiredZeroForSymmetry} to be
definitively zero for the transformation to be a symmetry.  So after all that goofing around
with derivatives and algebra it is defeated by the
simplest field Lagrangian.


Now, if we continue we find that we do in fact still have a symmetry by introducing a linearized spacetime translation.
This follows from direct expansion

\begin{align*}
(*)
&= (a \partial_x + \tau \partial_t) \LL \\
&=
a
\left(
\inv{v^2} \phi_t \partial_x \phi_t
- \phi_x \partial_x \phi_x
\right)
+ \tau
\left(
\inv{v^2} \phi_t \partial_t \phi_t
- \phi_x \partial_t \phi_x
\right) \\
\end{align*}

Next, calculation of the variational derivative we have

\begin{align*}
\frac{\delta(*)}{\delta \phi}
&=
\left( \PD{\phi}{} - \partial_x \PD{\phi_x}{} - \partial_t \PD{\phi_t}{} \right) (*) \\
&=
-\partial_x \left( -a \partial_{xx} \phi - \tau \partial_{tx}\phi \right)
-\inv{v^2} \partial_t \left( a \partial_{xt} \phi + \tau \partial_{tt}\phi \right) \\
&=
a \left(
\partial_x \left(
\phi_{xx} - \inv{v^2} \phi_{tt}
\right)
\right)
+ \tau \left(
\partial_t \left(
\phi_{xx} - \inv{v^2} \phi_{tt}
\right)
\right) \\
\end{align*}

Since we have $\phi_{xx} = \inv{v^2} \phi_{tt}$ by variation of \eqnref{eqn:stressEnergyNoethers:oneDimWave}.  So we do in fact have a symmetry from the
linearized spacetime translation for any shift $(t,x) \rightarrow (t+\tau, x+a)$.

\subsection{Symmetry condition for arbitrary linearized spacetime translation}

If we want to be able to alter the Lagrangian with a linearized vector translation of the generalized coordinates by some arbitrary
shift, since we do not have the linear derivatives for many Lagrangians of interest (wave equations, Maxwell equation, ...)
then can we find a general condition that is responsible for the translation symmetry that we have observed must exist for
the simple wave equation.

For a general Lagrangian $\LL = \LL(\phi(x), \partial_\mu \phi(x))$ under shift by some vector $a$

\begin{align}\label{eqn:stressEnergyNoethers:shiftXbyA}
x \rightarrow x' = x + a
\end{align}

we have

\begin{align*}
\LL' = \left( e^{a \cdot \grad}\right) \LL = \LL + (a \cdot \grad)\LL + \inv{2!}(a \cdot \grad)^2 \LL + \cdots
\end{align*}

Now, if we have

\begin{align*}
\frac{\delta ((a \cdot \grad) \LL)}{\delta \phi}
&
\questionEquals
(a \cdot \grad) 
\mathLabelBox{\frac{\delta \LL}{\delta \phi}}{$=0$}
 \\
\end{align*}

then this would explain the fact that we have a symmetry under linearized translation for the wave equation Lagrangian.  Can this interchange
of differentiation order be justified?

Writing out this variational derivative in full we have
\begin{align*}
\frac{\delta ((a \cdot \grad) \LL)}{\delta \phi}
&=
\left( \PD{\phi}{} - \partial_\sigma \PD{\phi_\sigma}{} \right) a^\mu \partial_\mu \LL \\
&=
a^\mu \left(
\PD{\phi}{} \PD{x^\mu}{}
 - \PD{x^\sigma}{} \PD{\phi_\sigma}{} \PD{x^\mu}{}
\right)
\LL \\
\end{align*}

Now, one can impose continuity conditions on the
field variables and Lagrangian sufficient to allow the commutation
of the coordinate partials.  Namely

\begin{align*}
\PD{x^\mu}{}\PD{x^\nu}{} f(\phi, \partial_\sigma \phi)
&=
\PD{x^\nu}{}\PD{x^\mu}{} f(\phi, \partial_\sigma \phi)
\end{align*}

However, we have a dependence between the field variables and the coordinates

\begin{align*}
\PD{x^\mu}{} =
\PD{x^\mu}{\phi} \PD{\phi}{}
+\sum_\sigma \PD{x^\mu}{\phi_\sigma} \PD{\phi_\sigma}{}
\end{align*}

Given this, can we commute the field partials and the coordinate partials like so

\begin{align*}
\PD{\phi}{} \PD{x^\mu}{} &\questionEquals \PD{x^\mu}{} \PD{\phi}{} \\
\PD{\phi_\sigma}{} \PD{x^\mu}{}  &\questionEquals \PD{x^\mu}{} \PD{\phi_\sigma}{}
\end{align*}

This is not obvious to me due to the dependence between the two.

If
that is a reasonable thing to do, then the variational derivative of this directional derivative is zero

\begin{align*}
\frac{\delta ((a \cdot \grad) \LL)}{\delta \phi}
&=
a^\mu \PD{x^\mu}{}
\left(
\PD{\phi}{}
 - \PD{x^\sigma}{} \PD{\phi_\sigma}{}
\right)
\LL \\
&=
(a \cdot \grad) \frac{\delta \LL}{\delta \phi} \\
&= 0
\end{align*}

To make any progress below I had to assume that this is justifiable.  With this assumption or requirement
we therefore have a symmetry for any Lagrangian
altered by the addition of a directional derivative, as is required for the first order Taylor series
approximation associated with a spacetime (or spatial or timelike) translation.

\subsubsection{An error above to revisit}

In an email discussing what I initially thought was a typo in
\citep{TongQFT}, he says
%There is this subtlety with active vs. passive transformations which is explained after (1.26). If the x coordinate gets shifted one way, then the argument of the field goes the other way. After (1.26), I describe this for Lorentz transformations, but it is also true for translations which is the situation in (1.39).
%
%Yes - if you did it the other way, you would also get the right Noether current. And you are right -- you should just be expanding out the first term in the Taylor series. The point is that when the spacetime coordinate
%changes as
%
%x -> x-a
%
%you do not just put this into the argument \phi(x). Instead the field transforms as
%
%\phi(x) -> \phi(x+a)
%
%which you then expand. The reason for this strange change of sign is to do with the difference between an active and a passive transformation. It is a little subtle, and it is what I tried to explain in the words below (1.26)
%
%
that while it is correct to transform the Lagrangian
using a Taylor expansion in $\phi(x+a)$ as I have done, this actually results
from $x \rightarrow x - a$, as opposed to the positive shift given in
\eqnref{eqn:stressEnergyNoethers:shiftXbyA}.  There was discussion of this in the context of Lorentz transformations around (1.26) of his QFT course notes, also applicable to translations.
The subtlety is apparently due to differences between passive and active
transformations.
I am sure he is right, and I think this is actually consistent with the
treatment of
\citep{doran2003gap} where they include an inverse operation in the
transformed Lagrangian (that minus is surely associated with the inverse
of the translation transformation).
It will take further study for me to completely understand this point, but
provided the starting point is really considered the Taylor series expansion
based on $\phi(x) \rightarrow \phi(x+a)$ and not based on \eqnref{eqn:stressEnergyNoethers:shiftXbyA}
then nothing else I have done here is wrong.  Also note that in the end our
Noether current can be adjusted by an arbitrary multiplicative constant so
the direction of the translation will also not change the final result.

\section{Noether current}

\subsection{Vector parametrized Noether current}

In \chapcite{PJFieldLagrangian} the derivation of Noether's theorem given a single variable parametrized
alteration of the Lagrangian was seen to essentially be an exercise in the
application of the chain rule.

How to extend that argument to the multiple variable case is not immediately
obvious.  In GA we can divide by vectors but attempting to formulate
a derivative this way gives us left and right sided derivatives.  How do we
overcome this to examine change of the Lagrangian with respect to a
vector parametrization?
One possibility is a scalar parametrization of the magnitude
of the translation vector.  If the translation is along $a = \alpha u$,
where $u$ is a unit vector we can write

\begin{align*}
\LL' &= \LL + \delta \LL  \\
&= \LL + (a \cdot \grad) \LL \\
&= \LL + \alpha (u \cdot \grad) \LL \\
\end{align*}

So we have

\begin{align*}
\frac{d\LL'}{d\alpha}
&=
(u \cdot \grad) \LL
\end{align*}

Now our previous Noether's current was derived by considering just
the sort of derivative on the LHS above, but on the RHS we are back
to working with a directional derivative.  The key is finding a
logical starting point for the chain rule like expansion that we expect
to produce the conservation current.

\begin{align*}
\delta \LL
&=  (a \cdot \grad ) \LL \\
&=  a^\mu \partial_\mu \LL \\
&=  a^\mu \left(
\PD{x^\mu}{\phi} \PD{\phi}{\LL}
+\sum_\sigma
\PD{x^\mu}{\phi_\sigma} \PD{\phi_\sigma}{\LL}
\right) \\
&=
\PD{\phi}{\LL} (a \cdot \grad) \phi
+
\sum_\sigma
\PD{\phi_\sigma}{\LL}
(a \cdot \grad) \phi_\sigma
\\
&=
\left(
\sum_\sigma
\partial_\sigma
\PD{\phi_\sigma}{\LL}
\right)
(a \cdot \grad) \phi
+
\sum_\sigma
\PD{\phi_\sigma}{\LL}
(a \cdot \grad) \phi_\sigma
\\
&=
\left(
\sum_\sigma
\partial_\sigma
\PD{\phi_\sigma}{\LL}
\right)
(a \cdot \grad) \phi
+
\sum_\sigma
\PD{\phi_\sigma}{\LL}
\partial_\sigma
((a \cdot \grad) \phi)
\\
&=
\sum_\sigma
\partial_\sigma
\left(
\PD{\phi_\sigma}{\LL}
(a \cdot \grad) \phi
\right)
\\
\end{align*}

So far so good, but where to go from here?  The trick (again from Tong) is that
the difference with itself is zero.  With a switch of dummy indices $\sigma \rightarrow \mu$, we have

\begin{align*}
0 &=
\delta \LL - \delta \LL  \\
&=
\sum_\mu
\partial_\mu
\left(
\PD{\phi_\mu}{\LL}
(a \cdot \grad) \phi
\right) -
 a^\mu \partial_\mu \LL
\\
&=
\sum_\mu
\partial_\mu
\left(
\PD{\phi_\mu}{\LL}
(a \cdot \grad) \phi
- a^\mu \LL
\right)
\\
\end{align*}

Now we have a quantity that is zero for any vector $a$, and can say we have
a conserved current $T(a)$ with coordinates

\begin{align}\label{eqn:stressEnergyNoethers:JmuOfAoneVar}
T^\mu(a)
=
\PD{\phi_\mu}{\LL}
(a \cdot \grad) \phi
- a^\mu \LL
\end{align}

Finally, putting this back into vector form

\begin{align*}
T(a) &= \gamma_\mu T^\mu(a) \\
&=
\left( \gamma_\mu \PD{\phi_\mu}{\LL} \right)
(a \cdot \grad) \phi
- \gamma_\mu a^\mu \LL  \\
\end{align*}

So we have

\begin{align}\label{eqn:stressEnergyNoethers:vCurrent}
T(a) &=
\left( \left(\gamma_\mu \PD{\phi_\mu}{} \right) \LL \right)
(a \cdot \grad) \phi
- a \LL  \\
\grad \cdot T(a) &= 0
\end{align}

So after a long journey, I have in
\eqnref{eqn:stressEnergyNoethers:vCurrent}
a derivation of a conservation current associated
with
a linearized vector displacement of the generalized coordinates.  I
recalled that the treatment in
\citep{doran2003gap} somehow eliminated the $a$.  That argument is still tricky involving
their linear operator theory, but I have at least obtained their equation (13.15).
They treat a multivector displacement whereas I only looked at
vector displacement.  They also do it in three lines, whereas building up to this
(or even understanding it) based on what I know required 13 pages.

\subsection{Comment on the operator above}

We have something above that is gradient like in
\eqnref{eqn:stressEnergyNoethers:vCurrent}.  Our spacetime gradient operator is

\begin{align*}
\grad = \gamma^\mu \PD{x^\mu}{}
\end{align*}

Whereas this unknown field variable derivative operator

\begin{align*}
\something = \gamma_\mu \PD{\phi_\mu}{}
\end{align*}

is somewhat like a velocity gradient with respect to the field variable.  It would be reasonable to expect that this will have a role in the field canonical momentum.

\subsection{In tensor form}

The conserved current
of \eqnref{eqn:stressEnergyNoethers:vCurrent}
can be put into tensor form by considering the action on
each of the basis vectors.

\begin{align*}
T(\gamma_\nu) \cdot \gamma^\mu
&=
\left( \left(\PD{\phi_\mu}{} \right) \LL \right)
(\gamma_\nu \cdot (\gamma^\sigma \partial_\sigma)) \phi
- \gamma_\nu \cdot \gamma^\mu \LL  \\
\end{align*}

Thus writing ${T^\mu}_\nu = T(\gamma_\nu) \cdot \gamma^\mu$ we have

\begin{align}\label{eqn:stressEnergyNoethers:currentTensor}
{T^\mu}_\nu = \PD{\phi_\mu}{\LL} \partial_\nu \phi - {\delta_\nu}^\mu \LL
\end{align}

\subsection{Multiple field variables}

In order to deal with the Maxwell Lagrangian a generalization to multiple
field variables is required.  Suppose now that we have a Lagrangian
density $\LL = \LL(\phi^\alpha, \partial_\beta \phi^\alpha)$.  Proceeding
with the chain rule application again we have after some latex
search and replace
adding in indices in all the right places (proof by regular expressions)

\begin{align*}
\delta \LL
&=  (a \cdot \grad ) \LL \\
&=  a^\mu \partial_\mu \LL \\
&=  a^\mu \left(
\PD{x^\mu}{\phi^\alpha} \PD{\phi^\alpha}{\LL}
+
\PD{x^\mu}{\partial_\sigma \phi^\alpha} \PD{\partial_\sigma \phi^\alpha}{\LL}
\right) \\
&=
\PD{\phi^\alpha}{\LL} (a \cdot \grad) \phi^\alpha
+
\PD{\partial_\sigma \phi^\alpha}{\LL}
(a \cdot \grad) \partial_\sigma \phi^\alpha
\\
&=
\left(
\partial_\sigma
\PD{\partial_\sigma \phi^\alpha}{\LL}
\right)
(a \cdot \grad) \phi^\alpha
+
\PD{\partial_\sigma \phi^\alpha}{\LL}
(a \cdot \grad) \partial_\sigma \phi^\alpha
\\
&=
\left(
\partial_\sigma
\PD{\partial_\sigma \phi^\alpha}{\LL}
\right)
(a \cdot \grad) \phi^\alpha
+
\PD{\partial_\sigma \phi^\alpha}{\LL}
\partial_\sigma
((a \cdot \grad) \phi^\alpha)
\\
&=
\partial_\sigma
\left(
\PD{\partial_\sigma \phi^\alpha}{\LL}
(a \cdot \grad) \phi^\alpha
\right)
\\
\end{align*}

In the above manipulations (and those below), any repeated index, regardless of whether upper and lower indices are matched implies summation.

Using this we have a multiple field generalization of
\eqnref{eqn:stressEnergyNoethers:JmuOfAoneVar}.   The
Noether current and its conservation law in coordinate form is

\begin{align}\label{eqn:stressEnergyNoethers:JmuOfAmanyVar}
T^\mu(a)
&=
\PD{\partial_\mu \phi^\alpha}{\LL}
(a \cdot \grad) \phi^\alpha
- a^\mu \LL \\
\partial_\mu T^\mu(a) &= 0
\end{align}

Or in vector form, corresponding to \eqnref{eqn:stressEnergyNoethers:vCurrent}

\begin{align}\label{eqn:stressEnergyNoethers:vCurrentmanyField}
T(a) &=
\left( \left(\gamma_\mu \PD{\partial_\mu \phi^\alpha}{} \right) \LL \right)
(a \cdot \grad) \phi^\alpha
- a \LL  \\
\grad \cdot T(a) &= 0
\end{align}

And finally
in tensor form, as in \eqnref{eqn:stressEnergyNoethers:currentTensor}

\begin{align}\label{eqn:stressEnergyNoethers:currentTensormany}
{T^\mu}_\nu &= \PD{\partial_\mu \phi^\alpha}{\LL} \partial_\nu \phi^\alpha - {\delta_\nu}^\mu \LL \\
\partial_\mu {T^\mu}_\nu &= 0
\end{align}

\subsection{Spatial Noether current}

The conservation arguments above have been expressed with the assumption that the Lagrangian density is a function
of both spatial and time coordinates, and this was made explicit with the use of the Dirac basis to express the
Noether current.

It should be pointed out that for a purely spatial Lagrangian density, such as that of electrostatics

\begin{align*}
\LL &= -\frac{\epsilon_0}{2} (\spacegrad \phi)^2 + \rho \phi
\end{align*}

the same results apply.  In this case it would be reasonable to summarize the conservation under translation
using the Pauli basis and write

\begin{align}
T(\Ba) &= \sigma_k \PD{\partial_k \phi}{\LL} \Ba \cdot \spacegrad \phi - \Ba \LL \\
\spacegrad \cdot T(\Ba) &= 0
\end{align}

Without the time translation, calling the vector Noether current the energy momentum tensor is not likely appropriate.  Perhaps just
the canonical energy momentum tensor?  Working with such a spatial Lagrangian density later should help clarify how to label things.

\section{Field Hamiltonian}

A special case of \eqnref{eqn:stressEnergyNoethers:currentTensor} is for time translation of the
Lagrangian.

For that, our Noether current, writing $\mathcal{H}^\mu = {T^\mu}_0$ is

%\newcommand{\HH}[0]{\boldsymbol{\mathcal{H}}}
\begin{align}
\mathcal{H}^0 &= \PD{\dotphi}{\LL} \dotphi - \LL \\
\mathcal{H}^k &= \PD{\phi_k}{\LL} \dotphi
\end{align}

These are expected to have a role associated with field energy and
momentum respectively.

For the Maxwell Lagrangian we will need the multiple field current

\begin{align*}
\mathcal{H}^0 &= \PD{\partial_0 \phi^\alpha}{\LL} \partial_0 \phi^\alpha - \LL \\
\mathcal{H}^k &= \PD{\partial_k \phi^\alpha}{\LL} \partial_0 \phi^\alpha
\end{align*}


\section{Wave equation}

Having computed the general energy momentum tensor for field Lagrangians, this can now be applied
to some specific field equations.  The Lagrangian for the relativistic wave equation is an
obvious first candidate due to simplicity.

\subsection{Tensor components and energy term}

%  One of the simplest
%Lagrangian with a single field variable, and the next ones are for multiple field Lagrangians where
%we will need to think through a generalization of the Noether conservation current equation first.

\begin{align}
\LL = \inv{2} \partial_\mu \phi \partial^\mu \phi = \inv{2} \phi_\mu \phi^\mu = \inv{2} (\grad \phi)^2 = \inv{2}(\dotphi^2 - (\spacegrad \phi)^2)
\end{align}

In the explicit spacetime split above we have a split into terms that appear
to correspond to kinetic and potential terms

\begin{align*}
\LL = K - V
\end{align*}

To compute the tensor, we first need
$\PDi{\phi_\mu}{\LL} = \phi^\mu$, which gives us

\begin{align}\label{eqn:stressEnergyNoethers:tensorInIndexForm}
{T^\mu}_\nu
&= \phi^\mu \phi_\nu - {\delta_\nu}^\mu \LL
\end{align}

Writing this out in matrix form (with rows $\mu$, and columns $\nu$), we have

\begin{align}\label{eqn:stressEnergyNoethers:bigTensorMatrix}
\begin{bmatrix}
\inv{2}(\dotphi^2 + \phi_x^2 + \phi_y^2 + \phi_z^2) & \dotphi \phi_x & \dotphi \phi_y & \dotphi \phi_z \\
-\phi_x \dotphi & \inv{2}(-\dotphi^2 - \phi_x^2 + \phi_y^2 + \phi_z^2) & -\phi_x \phi_y & -\phi_x \phi_z \\
-\phi_y \dotphi & -\phi_y \phi_x & \inv{2}(-\dotphi^2 + \phi_x^2 - \phi_y^2 + \phi_z^2) & -\phi_y \phi_z \\
-\phi_z \dotphi & -\phi_z \phi_x & -\phi_z \phi_y & \inv{2}(-\dotphi^2 + \phi_x^2 + \phi_y^2 - \phi_z^2) \\
\end{bmatrix}
\end{align}

As mentioned by Jackson, the canonical energy momentum tensor is not necessarily symmetric, and we see that here.
Do do however, have what is expected for the wave energy in the $0,0$ element

\begin{align*}
{T^0}_0 &= K + V  \\
&= \inv{2} (\dotphi^2 + (\spacegrad \phi)^2)
\end{align*}

\subsection{Conservation equations}

How about the conservation equations when written in full.  The first is

\begin{align*}
0
&= \partial_\mu {T^\mu}_0 \\
&=
\inv{2} \partial_t (\dotphi^2 + \phi_x^2 + \phi_y^2 + \phi_z^2)
-\partial_x(\phi_x \dotphi )
-\partial_y(\phi_y \dotphi )
-\partial_z(\phi_z \dotphi ) \\
&=
\dotphi \ddotphi
+ \phi_x \phi_{xt}
+ \phi_y \phi_{yt}
+ \phi_z \phi_{zt}
-\phi_{xx} \dotphi
-\phi_{yy} \dotphi
-\phi_{zz} \dotphi
-\phi_x \phi_{tx}
-\phi_y \phi_{ty}
-\phi_z \phi_{tz}
\\
&=
\dotphi (\ddotphi -\phi_{xx} -\phi_{yy} -\phi_{zz} )
\end{align*}

So our first conservation equation is

\begin{align*}
0 = \dotphi (\grad^2 \phi)
\end{align*}

But $\grad^2 \phi = 0$ is just our wave equation, the result of the variation of the Lagrangian itself.
So curiously the divergence of energy-momentum four vector ${T^\mu}_0$
ends up as another method of supplying the wave equation!

How about one of the other conservation equations?  The pattern will all be the same, so calculating one is sufficient.

\begin{align*}
0
&= \partial_\mu {T^\mu}_1 \\
&=
\partial_t(\dotphi \phi_x )
+\inv{2} \partial_x (-\dotphi^2 - \phi_x^2 + \phi_y^2 + \phi_z^2)
-\partial_y(\phi_y \phi_x )
-\partial_z(\phi_z \phi_x ) \\
&=
\ddotphi \phi_x
+ \dotphi \phi_{xt}
-\dotphi \phi_{tx} - \phi_x \phi_{xx} + \phi_y \phi_{yx}+ \phi_z \phi_{zx}
- \phi_{yy} \phi_x - \phi_y \phi_{xy}
- \phi_{zz} \phi_x - \phi_z \phi_{xz}
\\
&=
\phi_x (\ddotphi -\phi_{xx} -\phi_{yy} -\phi_{zz} )
\end{align*}

It should probably not be surprising that we have such a symmetric relation between space and time for the wave equations
and we can summarize the spacetime translation conservation equations by

\begin{align*}
0
&= \partial_\mu {T^\mu}_\nu \\
&= \phi_\nu (\grad^2 \phi)
\end{align*}

\subsection{Invariant length}

It has been assumed that $T(\gamma_\mu)$ are four vectors.  If that is the
cast we ought to have an invariant length.

Let us calculate the vector square of $T(\gamma_0)$.  Picking off first column
of our tensor in \eqnref{eqn:stressEnergyNoethers:bigTensorMatrix}, we have

\begin{align*}
(T(\gamma_0))^2
&= (\gamma_\mu {T^\mu}_0) \cdot ( \gamma_\nu {T^\nu}_0 ) \\
&= ({T^0}_0)^2 -({T^1}_0)^2 -({T^2}_0)^2 -({T^3}_0)^2 \\
&=
 \inv{4}\left( \dotphi^2 + \phi_x^2 + \phi_y^2 + \phi_z^2 \right)^2
- \phi_x^2 \dotphi^2
- \phi_y^2 \dotphi^2
- \phi_z^2 \dotphi^2 \\
&=
 \inv{4}\left( \dotphi^4 + \phi_x^4 + \phi_y^4 + \phi_z^4 \right)
-\inv{2}\left(
\dotphi^2 \phi_x^2
+\dotphi^2 \phi_y^2
+\dotphi^2 \phi_z^2
\right)
+\inv{2}\left(
+\phi_x^2 \phi_y^2
+\phi_y^2 \phi_z^2
+\phi_z^2 \phi_x^2 \right)
%- \phi_x^2 \dotphi^2
%- \phi_y^2 \dotphi^2
%- \phi_z^2 \dotphi^2
\\
&=
 \inv{4}\left( \dotphi^2 - \phi_x^2 - \phi_y^2 - \phi_z^2 \right)^2 \\
\end{align*}

But this is just our (squared) Lagrangian density, and we therefore have

\begin{align}
(T(\gamma_0))^2 &= \LL^2
\end{align}

%\begin{align*}
%T(\gamma_\nu)^2
%&= (\gamma_\mu {T^\mu}_\nu)^2 \\
%&=
%\sum_\mu (\gamma_\mu)^2 (\phi^\mu \phi_\nu - {\delta_\nu}^\mu \LL)^2 \\
%&=
%\sum_\mu (\gamma_\mu)^2 (-\phi_\mu \phi_\nu - {\delta_\nu}^\mu \LL)^2 \\
%&=
%\sum_\mu (\gamma_\mu)^2 (
%\phi_\mu^2 \phi_\nu^2 + {\delta_\nu}^\mu \LL^2
%+2 \phi_\mu \phi_\nu {\delta_\nu}^\mu \LL
%) \\
%&=
%\sum_\mu (\gamma_\mu)^2
%\phi_\mu^2 \phi_\nu^2  + (\gamma_\nu)^2 (\LL^2 +2 \phi_\nu^2 \LL)
%\\
%\end{align*}

Doing the same calculation for the second column, which is representative of the other two by symmetry, we have
\begin{align}
(T(\gamma_k))^2 &= -\LL^2
\end{align}

Summarizing all four squares we have
\begin{align}\label{eqn:stressEnergyNoethers:invariantLength}
(T(\gamma_\mu))^2 &= (\gamma_\mu)^2 \LL^2
\end{align}

All of these conservation current four vectors have the same length up to a sign, where $T(\gamma_0)$ is timelike (positive square),
whereas $T(\gamma_k)$ is spacelike (negative square).

Now, is $\LL^2$ a Lorentz invariant?  If so we can justify calling $T(\gamma_\mu)$ four vectors.  Reflection shows that this is in fact the case, since $\LL$ is a Lorentz invariant.  The transformation properties of $\LL$ go with the gradient.  Writing $\grad' = R \grad \tilde{R}$, we have

\begin{align*}
\LL'
&= \inv{2} \grad' \phi \cdot \grad' \phi \\
&= \inv{2} \gpgradezero{ R \grad \tilde{R} \phi R \grad \tilde{R} \phi} \\
&= \inv{2} \gpgradezero{ R \grad \phi \grad \tilde{R} \phi} \\
&= \inv{2} \gpgradezero{ 
\mathLabelBox
[
   labelstyle={xshift=2cm},
   linestyle={out=270,in=90, latex-}
]
{\tilde{R} R}{$=1$}
\grad \phi \grad \phi} \\
&= \inv{2} \grad \phi \cdot \grad \phi \\
&= \LL
\end{align*}

\subsection{Diagonal terms of the tensor}

There is a conjugate structure evident in the diagonal terms of the matrix for
the tensor.  In particular, the ${T^0}_0$ can be expressed using the
Hermitian conjugate from QM.  For a multivector $F$, this was defined
as

\begin{align}\label{eqn:stressEnergyNoethers:HermitianConj}
F^\dagger = \gamma_0 \tilde{F} \gamma_0
\end{align}

We have for ${T^0}_0$

\begin{align*}
{T^0}_0
&= \inv{2} (\grad \phi)^\dagger \cdot (\grad \phi) \\
&= \inv{2} \gpgradezero{ \gamma_0 \grad \gamma_0 \phi \grad \phi } \\
&= \inv{2} \gpgradezero{ (\gamma_0 \grad \phi)^2 } \\
&= \inv{2} \gpgradezero{ (\gamma_0 (\gamma^0 \partial_0 + \gamma^k \partial_k ) \phi )^2 } \\
&= \inv{2} \gpgradezero{ ((\partial_0 - \gamma^k \gamma_0 \partial_k ) \phi)^2 } \\
&= \inv{2} \gpgradezero{ ((\partial_0 + \spacegrad) \phi)^2 } \\
&= \inv{2} \left( \dotphi^2 + (\spacegrad \phi)^2 \right) \\
\end{align*}

Now conjugation with respect to the time basis vector should not be special in any way, and should be equally justified defining a
conjugation operation along any of the spatial directions too.  Is there a symbol for this?  Let us write for now

\begin{align}\label{eqn:stressEnergyNoethers:HermitianConjMu}
F^{\dagger_\mu} \equiv \gamma_\mu \tilde{F} \gamma^\mu
\end{align}

There is a possibility that the sign picked here is not appropriate for all purposes.
It is hard to tell for now since we have a vector $F$ that equals its reverse, and in fact
after a computation with both $\mu$ indices down I have raised an index altering
an
initial choice of $F^{\dagger_\mu} = \gamma_\mu \tilde{F} \gamma_\mu$

Applying this, for $\mu \ne 0$ we have

\begin{align*}
(\grad \phi)^{\dagger_\mu} \cdot (\grad \phi)
&=
-\gpgradezero{ \gamma_\mu \grad \gamma_\mu \phi \grad \phi } \\
&=
-\gpgradezero{ ((\partial_\mu + \gamma_\mu \sum_{\nu \ne \mu} \gamma^\nu \partial_\nu) \phi )^2 } \\
&=
-((\partial_\mu \phi)^2 + \sum_{\nu \ne \mu} (\gamma_\mu \gamma^\nu)^2 (\partial_\nu \phi )^2) \\
&=
-((\partial_\mu \phi)^2 - \sum_{\nu \ne \mu} (\gamma_\mu)^2 (\gamma^\nu)^2 (\partial_\nu \phi )^2) \\
&=
-((\partial_\mu \phi)^2 + \sum_{\nu \ne \mu} (\gamma^\nu)^2 (\partial_\nu \phi )^2) \\
&=
-(\partial_\mu \phi)^2 - (\partial_0 \phi )^2 + \sum_{k \ne \mu, k \ne 0} (\partial_k \phi )^2 \\
\end{align*}

This recovers the diagonal terms, and allows us to write (no sum)

\begin{align}
{T^\mu}_\mu = \inv{2} (\grad \phi)^{\dagger_\mu} \cdot (\grad \phi)
\end{align}

\subsubsection{As a projection?}

As a vector (a projection of $T(\gamma_\mu)$ onto the $\gamma_\mu$ direction) this is (again no sum)

\begin{align*}
\gamma_\mu {T^\mu}_\mu
&= \inv{2} \gamma_\mu (\grad \phi)^{\dagger_\mu} \cdot (\grad \phi) \\
&= \inv{2} \gamma_\mu \gpgradezero{ \gamma^\mu \grad \phi \gamma_\mu \grad \phi } \\
&= \inv{4} \gamma_\mu ( \gamma^\mu \grad \phi \gamma_\mu \grad \phi + \grad \phi \gamma_\mu \grad \phi \gamma^\mu ) \\
&= \inv{4} ( (\grad \phi \gamma_\mu \grad \phi) + \gamma_\mu (\grad \phi \gamma_\mu \grad \phi) \gamma^\mu )
\end{align*}

Intuition says this may have a use when assembling a complete vector representation of $T(\gamma_\mu)$ in
terms of the gradient, but what that is now is not clear.
%Is the grade selection here required?  Does this reduce to $(\grad \phi \gamma_\mu \grad \phi)/2 $ ?

\subsection{Momentum}

Now, let us look at the four vector $T(\gamma_0) = \gamma_\mu{T^\mu}_0$ more carefully.  We have seen
the energy term of this, but have not looked at the spatial part (momentum).

We can calculate the spatial component by wedging with the observer unit velocity $\gamma_0$, and get

\begin{align*}
T(\gamma_0) \wedge \gamma_0
&= \gamma_k \gamma_0 {T^k}_0 \\
&= -\sigma_k \dotphi \phi_k \\
&= - \dotphi \spacegrad \phi \\
\end{align*}

Right away we have something interesting!  The wave momentum is related to the gradient operator, exactly as
we have in quantum physics, despite the fact that we are only looking at the classical wave equation (for
light or some other massless field effect).

\section{Wave equation.  GA form for the energy momentum tensor}

Some of the playing around above was attempting to find more structure for the terms of the energy momentum tensor.  For
the diagonal terms this was done successfully.  However, doing so for the remainder is harder when working backwards
from the tensor in coordinate form.

\subsection{Calculate GA form}

Let us step back to the defining relation \eqnref{eqn:stressEnergyNoethers:vCurrentmanyField}, from which we
see that we wish to calculate

\begin{align*}
\gamma_\mu \PD{\partial_\mu \phi^\alpha}{\LL}
&=
\gamma_\mu \partial^\mu \phi \\
&=
\gamma^\mu \partial_\mu \phi \\
&=
\grad \phi
\end{align*}

This completely removes the indices from the tensor, leaving us with

\begin{align*}
T(a)
&= (\grad \phi) a \cdot \grad \phi - \frac{a}{2} (\grad \phi)^2 \\
&= (\grad \phi) \left( \inv{2}(a \grad \phi + \grad \phi a) - \grad \phi \frac{a}{2} \right) \\
\end{align*}

Thus we have

\begin{align}\label{eqn:stressEnergyNoethers:GAwaveStressEnergy}
T(a) &= \inv{2} (\grad \phi) a (\grad \phi)
\end{align}

This meets the intuitive expectation that the energy momentum tensor for the wave equation could be expressed
completely in terms of the gradient.

\subsection{Verify against tensor expression}

There is in fact a surprising simplicity to the result of \eqnref{eqn:stressEnergyNoethers:GAwaveStressEnergy}.
It is somewhat hard to believe that it summarizes the messy matrix we have calculated above.
To verify this let us derive the tensor relation of \eqnref{eqn:stressEnergyNoethers:tensorInIndexForm}.

\begin{align*}
{T^\mu}_\nu
&= T(\gamma_\nu) \cdot \gamma^\mu \\
&= \inv{2} \gpgradezero{ (\grad \phi) \gamma_\nu (\grad \phi) \gamma^\mu } \\
&= \inv{2} \gpgradezero{ \gamma^\alpha \partial_\alpha \phi \gamma_\nu \gamma_\beta \partial^\beta \phi \gamma^\mu } \\
&= \inv{2}
\partial_\alpha \phi \partial^\beta \phi
\gpgradezero{ \gamma^\alpha \gamma_\nu \gamma_\beta \gamma^\mu } \\
&= \inv{2}
\partial_\alpha \phi \partial^\beta \phi
\left(
{\delta^\alpha}_\nu {\delta_\beta}^\mu
+
(\gamma^\alpha \wedge \gamma_\nu) \cdot (\gamma_\beta \wedge \gamma^\mu )
\right) \\
&= \inv{2} \left(
\partial_\nu \phi \partial^\mu \phi
+
\partial^\alpha \phi \partial_\beta \phi (\gamma_\alpha \wedge \gamma_\nu) \cdot (\gamma^\beta \wedge \gamma^\mu )
\right) \\
&= \inv{2} \left(
\partial_\nu \phi \partial^\mu \phi
+
(\partial^\alpha \phi \partial_\beta \phi) \gamma_\alpha \cdot (
\mathLabelBox{\gamma_\nu \cdot (\gamma^\beta \wedge \gamma^\mu )}{$={\delta_\nu}^\beta \gamma^\mu -{\delta_\nu}^\mu \gamma^\beta $}
)
\right) \\
&= \inv{2} \left(
\partial_\nu \phi \partial^\mu \phi
+
(\partial^\alpha \phi \partial_\beta \phi) ( {\delta_\nu}^\beta {\delta_\alpha}^\mu - {\delta_\nu}^\mu {\delta_\alpha}^\beta )
\right) \\
&= \inv{2} \left( \partial_\nu \phi \partial^\mu \phi + \partial^\mu \phi \partial_\nu \phi - {\delta_\nu}^\mu (\partial^\alpha \phi \partial_\alpha \phi) \right) \\
&= \partial_\nu \phi \partial^\mu \phi - {\delta_\nu}^\mu \LL \quad\quad\quad\square \\
\end{align*}

\subsection{Invariant length}

Putting the energy momentum tensor in GA form makes the demonstration of the invariant length almost trivial.  We have for any $a$

\begin{align*}
(T(a))^2
&= \inv{4} \grad \phi a \grad \phi \grad \phi a \grad \phi \\
&= \inv{4} (\grad \phi)^2 \grad \phi a^2 \grad \phi \\
&= \inv{4} (\grad \phi)^4 a^2 \\
&= \LL^2 a^2 \\
\end{align*}

This recovers \eqnref{eqn:stressEnergyNoethers:invariantLength}, which came at considerably higher cost in terms of guesswork.

\subsection{Energy and Momentum split (again)}

By wedging with $\gamma_0$ we can extract the momentum terms of $T(\gamma_0)$.  That is

\begin{align*}
T(\gamma_0) \wedge \gamma_0
&=
\left( (\gamma_0 \cdot \grad \phi) \grad \phi - \inv{2} (\grad \phi)^2 \gamma_0 \right) \wedge \gamma_0 \\
&=
(\gamma_0 \cdot \grad \phi) (\grad \phi \wedge \gamma_0) - \inv{2} (\grad \phi)^2 
\mathLabelBox{(\gamma_0 \wedge \gamma_0)}{$=0$}
\\
&=
\dotphi (\gamma^k \gamma_0 \partial_k \phi ) \\
&=
-\dotphi \spacegrad \phi
\end{align*}

For the energy term, dotting with $\gamma_0$ we have

\begin{align*}
T(\gamma_0) \cdot \gamma_0
&=
\left( (\gamma_0 \cdot \grad \phi) \grad \phi - \inv{2} (\grad \phi)^2 \gamma_0 \right) \cdot \gamma_0 \\
&=
(\gamma_0 \cdot \grad \phi)^2 - \inv{2} (\grad \phi)^2  \\
&=
\dotphi^2 - \inv{2}( \dotphi^2 - (\spacegrad \phi)^2 ) \\
&=
\inv{2} ( \dotphi^2 + (\spacegrad \phi)^2 ) \\
\end{align*}

Wedging with $\gamma_0$ itself does not provide us with a relative spatial vector.  For example, consider the proper time velocity four vector
(still working with $c=1$)

\begin{align*}
v
&= \frac{dt}{d\tau}\frac{d}{dt}\left( t \gamma_0 + \gamma_k x^k \right) \\
&= \frac{dt}{d\tau}\left( \gamma_0 + \gamma_k \frac{dx^k}{dt} \right) \\
\end{align*}

We have
\begin{align*}
v \cdot \gamma_0 &= \frac{dt}{d\tau} = \gamma
\end{align*}

and
\begin{align*}
v \wedge \gamma_0 &= \frac{dt}{d\tau} \sigma_k \frac{dx^k}{dt}
\end{align*}

Or
\begin{align*}
\Bv
&\equiv \sigma_k \frac{dx^k}{dt} \\
&= \frac{v \wedge \gamma_0}{v \cdot \gamma_0}
\end{align*}

This suggests that the form for the relative momentum (spatial) vector for the field should therefore be

\begin{align*}
\Bp
&\equiv \frac{T(\gamma_0) \wedge \gamma_0}{T(\gamma_0) \cdot \gamma_0} \\
&=
-\frac{\dotphi}{\inv{2}(\dotphi^2 + (\spacegrad \phi)^2)} \spacegrad \phi \\
&=
-\frac{2}{1 + \left(\frac{\spacegrad \phi}{\dotphi}\right)^2} \frac{\spacegrad \phi}{\dotphi} \\
&=
-\frac{2}{\frac{\dotphi}{\spacegrad \phi} + \frac{\spacegrad \phi}{\dotphi}} \\
\end{align*}

This has been written in a few different ways, looking for something familiar, and not really finding it.  It would be useful to
revisit this after considering in detail wave momentum in a mechanical sense, perhaps with a limiting argument as given
in \citep{goldstein1951cm} (ie: one dimensional Lagrangian density considering infinite sequence of springs in a line).

\section{Scalar Klein Gordon}

A number of details have been extracted considering the scalar wave equation.  Now lets move to a two field variable Lagrangian.

\begin{align}
\LL = \inv{2} \partial_\mu \psi \partial^\mu \psi - \frac{m^2 c^2}{2 \Hbar^2} \psi^2
\end{align}

This forced wave equation will have almost the same energy momentum tensor.  The exception will be the diagonal terms
for which we have an additional factor of $m^2 c^2 \psi^2/ 2\Hbar^2$.

This also means that the conservation equations will be altered slightly

\begin{align*}
0 &= \partial_\mu {T^\mu}_\nu \\
&= \phi_\nu \left( \grad^2 \phi + \frac{m^2 c^2}{\Hbar^2} \phi \right)
\end{align*}

Again the divergence of the individual canonical energy momentum tensor four vectors reproduces the field equations
that we also obtain from the variation.

\section{Complex Klein Gordon}

\subsection{Tensor in GA form}

\begin{align}
\LL = \partial_\mu \psi \partial^\mu \psi^\conj - \frac{m^2 c^2}{\Hbar^2} \psi \psi^\conj
\end{align}

We first want to calculate what perhaps could be called the field velocity gradient

\begin{align*}
\gamma_\mu \PD{(\partial_\mu \psi)}{\LL}
&=
\gamma_\mu \partial^\mu \psi \\
&=
\grad \psi \\
\end{align*}

Similarly
\begin{align*}
\gamma_\mu \PD{(\partial_\mu \psi^\conj)}{\LL}
&=
\gamma_\mu \partial^\mu \psi^\conj \\
&=
\grad \psi^\conj \\
\end{align*}

Assembling results into an application of \eqnref{eqn:stressEnergyNoethers:vCurrentmanyField}, we have

\begin{align*}
T(a)
&= \grad \psi (a \cdot \grad) \psi^\conj +\grad \psi^\conj (a \cdot \grad) \psi - a \LL  \\
&= \grad \psi (a \cdot \grad) \psi^\conj +\grad \psi^\conj (a \cdot \grad) \psi - a \inv{2}(\grad \psi \grad \psi^\conj + \grad \psi^\conj \grad \psi )
+ a \frac{m^2 c^2}{\Hbar^2} \psi \psi^\conj \\
&=
\grad \psi ( a \cdot \grad \psi^\conj - \inv{2} \grad \psi^\conj a)
+\grad \psi^\conj ( a \cdot \grad \psi - \inv{2} \grad \psi a)
+ a \frac{m^2 c^2}{\Hbar^2} \psi \psi^\conj \\
&=
\inv{2} \left(
(\grad \psi ) a (\grad \psi^\conj)
+(\grad \psi^\conj ) a (\grad \psi) \right)
+ a \frac{m^2 c^2}{\Hbar^2} \psi \psi^\conj \\
\end{align*}

Since vectors equal their own reverse this is just

\begin{align}
T(a) &= (\grad \psi ) a (\grad \psi^\conj) + a \frac{m^2 c^2}{\Hbar^2} \psi \psi^\conj
\end{align}

\subsection{Tensor in index form}

Expanding the energy momentum tensor in index notation we have

\begin{align*}
{T^\mu}_\nu &= T(\gamma_\nu) \cdot \gamma^\mu \\
&=
\partial_\alpha \psi \partial_\beta \psi^\conj
\gpgradezero{ \gamma^\alpha \gamma_\nu \gamma^\beta \gamma^\mu } + {\delta_\nu}^\mu \frac{m^2 c^2}{\Hbar^2} \psi \psi^\conj \\
&=
\partial_\nu \psi \partial^\mu \psi^\conj
+\partial^\mu \psi \partial_\nu \psi^\conj
-\partial^\alpha \psi \partial_\alpha \psi^\conj {\delta_\nu}^\mu
+ {\delta_\nu}^\mu \frac{m^2 c^2}{\Hbar^2} \psi \psi^\conj \\
\end{align*}

So we have
\begin{align}
{T^\mu}_\nu &= \partial^\mu \psi \partial_\nu \psi^\conj + \partial^\mu \psi^\conj \partial_\nu \psi - {\delta_\nu}^\mu \LL
\end{align}

This index representation also has a nice compact elegance.

\subsection{Invariant Length?}

Writing for short $b = \grad \psi$, and working in natural units $m^2 c^2 = \Hbar^2$, we have

\begin{align*}
(T(a))^2
&=
(b a b^\conj + a \psi \psi^\conj)^2 \\
&=
\gpgradezero{ a b^\conj b a b^\conj b } + a^2 \psi^2 (\psi^\conj)^2 + 2 a \cdot (b a b^\conj) \\
\end{align*}

Unlike the light wave equation this does not (obviously) appear to have a natural split into something times $a^2$.  Is there a way to do it?

\subsection{Divergence relation}

Borrowing notation from above to calculate the divergence we want

\begin{align*}
\grad \cdot (b a b^\conj)
&=
\gpgradezero{ \grad (b a b^\conj) } \\
&=
\gpgradezero{ (b^\conj \lrgrad b) a } \\
&=
a \cdot \gpgradeone{ b^\conj \lrgrad b } \\
\end{align*}

Here cyclic reordering of factors within the scalar product was used.  In order for that to be a meaningful operation the gradient must
be allowed to operate bidirectionally, so this is really just shorthand for

\begin{align}
b^\conj \lrgrad b
\equiv
\dot{b}^\conj \dot{\grad} b
+{b}^\conj \dot{\grad} \dot{b}
\end{align}

Where the more conventional overdot notation is used to indicate the scope of the operation.

In particular, for $b = \grad \psi$, we have

\begin{align*}
\gpgradeone{ b^\conj \lrgrad b }
&=
(\grad^2 \psi^\conj) (\grad \psi) + (\grad \psi^\conj) (\grad^2 \psi)
\end{align*}

Our tensor also has a vector scalar product that we need the divergence of.  That is

\begin{align*}
\grad \cdot (a \psi \psi^\conj)
&=
\gpgradezero{ \grad (a \psi \psi^\conj) } \\
&=
a \cdot \grad (\psi \psi^\conj) \\
\end{align*}

Putting things back together we have
\begin{align*}
\grad \cdot T(a)
&= a \cdot \left(
\gpgradeone{ (\grad \psi^\conj) \lrgrad (\grad \psi) } + \frac{m^2 c^2}{\Hbar^2} \grad (\psi \psi^\conj)
\right)
\end{align*}

This is
\begin{align}
0 = \grad \cdot T(a) = a \cdot \left(
(\grad^2 \psi^\conj) (\grad \psi) + (\grad \psi^\conj) (\grad^2 \psi) + \frac{m^2 c^2}{\Hbar^2} \grad (\psi \psi^\conj)
\right)
\end{align}

Again, we see that the divergence of the canonical energy momentum tensor produces the field equations that we get by direct variation!  Put explicitly
we have zero for all displacements $a$, so must also have

\begin{align}\label{eqn:stressEnergyNoethers:adjointEqualsZero}
0 =
(\grad \psi) \left(\grad^2 \psi^\conj + \frac{m^2 c^2}{\Hbar^2} \psi^\conj \right)
+ (\grad \psi^\conj) \left(\grad^2 \psi + \frac{m^2 c^2}{\Hbar^2} \psi \right)
\end{align}

Also noteworthy above is the adjoint relationship.  The adjoint $\overbar{F}$ of a an operator $F$ was defined via the dot product

\begin{align}
a \cdot F(b) \equiv b \cdot \overbar{F}(a)
\end{align}

So we have a concrete example of the adjoint applied to the gradient, and for this energy momentum tensor we have

\begin{align}
\overbar{T}(\grad) &= \gpgradeone{ (\grad \psi^\conj) \grad (\grad \psi) } + \frac{m^2 c^2}{\Hbar^2} \grad (\psi \psi^\conj)
\end{align}

Here the arrows notation has been dropped, where it is implied that this derivative acts on all neighboring vectors either unidirectionally or bidirectionally
as appropriate.

Now, this adjoint tensor is a curious beastie.  Intuition says this this one will have a Lorentz invariant length.  A moment of reflection
shows that this is in fact the case since the adjoint was fully expanded in \eqnref{eqn:stressEnergyNoethers:adjointEqualsZero}.  That vector is zero, and the
length is therefore also necessarily invariant.

\subsection{TODO}

How about the energy and momentum split in this adjoint form?  Could also write out adjoint in index
notation for comparison to non-adjoint tensor in index form.

\section{Electrostatics Poisson Equation}

\subsection{Lagrangian and spatial Noether current}

\begin{align}
\LL &= -\frac{\epsilon_0}{2} (\spacegrad \phi)^2 + \rho \phi
\end{align}

Evaluating this yields the desired $\spacegrad^2 \phi = -\rho/\epsilon_0$, or $\spacegrad \cdot \BE = \rho/\epsilon_0$.

\subsection{Energy momentum tensor}
In this particular case we then have

\begin{align*}
T(\Ba)
&= \sigma_k (-\epsilon_0 \partial_k \phi) \Ba \cdot \spacegrad \phi - \Ba \LL \\
&= -\epsilon_0 (\spacegrad \phi) \Ba \cdot \spacegrad \phi - \Ba (-\frac{\epsilon_0}{2} (\spacegrad \phi)^2 + \rho \phi) \\
&= -\epsilon_0 (\spacegrad \phi) \Ba \cdot \spacegrad \phi + (\spacegrad \phi)^2 \Ba \frac{\epsilon_0}{2} - \Ba \rho \phi \\
&= \frac{\epsilon_0}{2} (\spacegrad \phi) \left( -2 \Ba \cdot \spacegrad \phi + \spacegrad \phi \Ba \right) - \Ba \rho \phi \\
&= \frac{\epsilon_0}{2} (\spacegrad \phi) \left( -\Ba \spacegrad \phi - \spacegrad \phi \Ba + \spacegrad \phi \Ba \right) - \Ba \rho \phi \\
&= -\frac{\epsilon_0}{2} (\spacegrad \phi) \Ba \spacegrad \phi - \Ba \rho \phi \\
\end{align*}

It in terms of $\BE = -\spacegrad \phi$ this is

\begin{align}\label{eqn:stressEnergyNoethers:electrostaticTensor}
T(\Ba) &= -\frac{\epsilon_0}{2} \BE \Ba \BE - \Ba \rho \phi
\end{align}

This is not immediately recognizable (at least to me), and also does not appear to be easily separable into something times $\Ba$.

\subsection{Divergence and adjoint tensor}

What will we get with the divergence calculation?

\begin{align*}
\spacegrad \cdot (\BE \Ba \BE)
&=
\gpgradezero{ \spacegrad (\BE \Ba \BE) } \\
&=
\Ba \cdot \gpgradeone{ \BE \lrspacegrad \BE } \\
\end{align*}

Also want
\begin{align*}
\spacegrad \cdot (\Ba \rho \phi)
&=
\gpgradezero{ \spacegrad (\Ba \rho \phi)} \\
&=
\Ba \cdot \spacegrad (\rho\phi)
\end{align*}

Assembling these we have
\begin{align*}
\spacegrad \cdot T(\Ba) &=
-\Ba \cdot \left( \gpgradeone{ \frac{\epsilon_0}{2} \BE \lrspacegrad \BE } + \spacegrad (\rho\phi) \right)
\\
\end{align*}

From this we can pick off the adjoint
\begin{align*}
\overbar{T}(\spacegrad)
&=
-\frac{\epsilon_0}{2} \gpgradeone{ \BE \lrspacegrad \BE } - \spacegrad (\rho\phi)  \\
&=
-\frac{\epsilon_0}{2}\left(
(\dot{\BE} \cdot \dot{\spacegrad}) \BE
\BE (\spacegrad \cdot \BE)
\right)
- \spacegrad (\rho\phi)  \\
&=
-\epsilon_0 (\spacegrad^2 \phi) \spacegrad \phi - \spacegrad (\rho\phi)  \\
&=
-\epsilon_0 \spacegrad (\spacegrad \phi)^2 - \spacegrad (\rho\phi)  \\
&=
\spacegrad \left( -\epsilon_0 (\spacegrad \phi)^2 - \rho\phi \right)  \\
\end{align*}

If we write $\LL = K - V$, then we have in this case

\begin{align*}
\overbar{T}(\spacegrad) = \spacegrad (K + V) = 0
\end{align*}

Since the gradient of this quantity is zero everywhere it must be constant

\begin{align}\label{eqn:stressEnergyNoethers:electrostaticConst}
K + V = \text{constant}
\end{align}

We did not have any time dependence in the Lagrangian, and blindly
following the math to calculate the 
associated symmetry with the field translation, we end up with a
conservation statement that appears to be about energy.

TODO: am used to (as in \citep{feynman1963flp})
seeing electrostatic energy written

\begin{align*}
U = \inv{2} \epsilon_0 \int \BE^2 dV = \inv{2} \int \rho \phi dV
\end{align*}

Reconcile this with \eqnref{eqn:stressEnergyNoethers:electrostaticConst}.

%\section{Magnetostatics Equation}
%
%FIXME: What is the Lagrangian for this case?  work it through.

\section{Schr\"{o}dinger equation}

While not a Lorentz invariant Lagrangian, we do not have a dependence on that,
and can still calculate a Noether current on spatial translation.

\begin{align}
\LL = \frac{\Hbar^2}{2m}
(\spacegrad \psi) \cdot (\spacegrad \psi^\conj) + V \psi \psi^\conj + {i \Hbar} \left( \psi \partial_t \psi^\conj - \psi^\conj \partial_t \psi \right)
\end{align}

For this Lagrangian density it is worth noting that the action is in fact

\begin{align*}
S = \int d^3 x \LL
\end{align*}

... ie: $\partial_t \psi$ is not a field variable in the variation (this is why there is no factor of $1/2$ in the probability current term).

Calculating the Noether current for a vector translation $\Ba$ we have

\begin{align*}
T(\Ba)
&=
\frac{\Hbar^2}{2m} \spacegrad \psi \Ba \cdot \spacegrad \psi^\conj
+\frac{\Hbar^2}{2m} \spacegrad \psi^\conj \Ba \cdot \spacegrad \psi
- \Ba \LL
\end{align*}

Expanding the divergence is messy but straightforward

\begin{align*}
&\spacegrad \cdot T(\Ba) \\
&=
\frac{\Hbar^2}{2m}
\gpgradezero{
\spacegrad \left( \spacegrad \psi \Ba \cdot \spacegrad \psi^\conj + \spacegrad \psi^\conj \Ba \cdot \spacegrad \psi \right)
- \spacegrad(\spacegrad \psi \cdot \spacegrad \psi^\conj) \Ba
} \\
   &\quad- \Ba \cdot \spacegrad \left( V \psi \psi^\conj + i\Hbar (\psi \dotpsi^\conj - \psi^\conj \dotpsi) \right) \\
&=
\frac{\Hbar^2}{4m}
\gpgradezero{
\spacegrad \left( \spacegrad \psi (\Ba \spacegrad \psi^\conj + \spacegrad \psi^\conj \Ba) + \spacegrad \psi^\conj
(\Ba \spacegrad \psi
+\spacegrad \psi \Ba)
\right)
- 2 \spacegrad(\spacegrad \psi \cdot \spacegrad \psi^\conj) \Ba
} \\
   &\quad- \Ba \cdot \spacegrad \left( V \psi \psi^\conj + i\Hbar (\psi \dotpsi^\conj - \psi^\conj \dotpsi) \right) \\
&=
\frac{\Hbar^2}{4m}
\Ba \cdot
\gpgradeone{
  \spacegrad \psi^\conj \lrspacegrad \spacegrad \psi
+ \spacegrad \psi \lrspacegrad \spacegrad \psi^\conj
+ \spacegrad ( \spacegrad \psi \spacegrad \psi^\conj )
+ \spacegrad ( \spacegrad \psi^\conj \spacegrad \psi )
-2 \spacegrad(\spacegrad \psi \cdot \spacegrad \psi^\conj)
} \\
&\quad- \Ba \cdot \spacegrad \left( V \psi \psi^\conj + i\Hbar (\psi \dotpsi^\conj - \psi^\conj \dotpsi) \right) \\
&=
\frac{\Hbar^2}{4m}
\Ba \cdot
\gpgradeone{
  \spacegrad \psi^\conj \lrspacegrad \spacegrad \psi
+ \spacegrad \psi \lrspacegrad \spacegrad \psi^\conj
} \\
&\quad- \Ba \cdot \spacegrad \left( V \psi \psi^\conj + i\Hbar (\psi \dotpsi^\conj - \psi^\conj \dotpsi) \right) \\
&=
\frac{\Hbar^2}{4m}
\Ba \cdot 2 \left( \spacegrad \psi^\conj \spacegrad^2 \psi + \spacegrad \psi \spacegrad^2 \psi^\conj \right)
- \Ba \cdot \spacegrad \left( V \psi \psi^\conj + i\Hbar (\psi \dotpsi^\conj - \psi^\conj \dotpsi) \right) \\
&=
\frac{\Hbar^2}{2m}
\Ba \cdot \spacegrad \left( \spacegrad \psi^\conj \cdot \spacegrad \psi \right)
- \Ba \cdot \spacegrad \left( V \psi \psi^\conj + i\Hbar (\psi \dotpsi^\conj - \psi^\conj \dotpsi) \right) \\
\end{align*}

Which is, finally,

\begin{align}
\spacegrad \cdot T(\Ba)
=
\Ba \cdot \spacegrad
\left(
\frac{\Hbar^2}{2m}
\spacegrad \psi^\conj \cdot \spacegrad \psi
- V \psi \psi^\conj - i\Hbar (\psi \dotpsi^\conj - \psi^\conj \dotpsi)
\right)
\end{align}

Picking off the adjoint we have

\begin{align}
\overbar{T}(\spacegrad)
=
\frac{\Hbar^2}{2m}
\spacegrad \psi^\conj \cdot \spacegrad \psi
- V \psi \psi^\conj - i\Hbar (\psi \dotpsi^\conj - \psi^\conj \dotpsi)
\end{align}

Just like the electrostatics equation, it appears that we can make an association with Kinetic ($K$) and Potential ($\phi$) energies with the adjoint
stress tensor.

\begin{align*}
K &= \frac{\Hbar^2}{2m} \spacegrad \psi^\conj \cdot \spacegrad \psi \\
\phi &= V \psi \psi^\conj + i\Hbar (\psi \dotpsi^\conj - \psi^\conj \dotpsi) \\
\LL &= K - \phi \\
\overbar{T}(\spacegrad) &= K + \phi
\end{align*}

FIXME: Unlike the electrostatics case however, there is no
conserved scalar quantity that is obvious.
The association in this case with energy is by analogy, not
connected to anything reasonably physical seeming.  How to connect this
with actual physical concepts?
Can this be written as the
gradient of something?  Because of the time derivatives perhaps the space
time gradient would be required, however, because of the non-Lorentz 
invariant nature I had expect that terms may have to be added or subtracted
to make that possible.

\section{Maxwell equation}

Wanting to see some of the connections between the Maxwell equation and the Lorentz force was the
original reason for examining this canonical energy momentum tensor concept in detail.

\subsection{Lagrangian}

Recall that the Lagrangian for the vector grades of Maxwell's equation

\begin{align}\label{eqn:stressEnergyNoethers:maxwell}
\grad F = J/\epsilon_0 c
\end{align}

is of the form

\begin{align*}
\LL
&= \kappa (\grad \wedge A) \cdot (\grad \wedge A) + J \cdot A \\
&= \kappa (\gamma^\mu \wedge \gamma^\nu) \cdot (\gamma_\alpha \wedge \gamma_\beta) \partial_\mu A_\nu \partial^\alpha A^\beta + J^\sigma A_\sigma \\
\end{align*}

We can fix the constant $\kappa$ by taking variational derivatives and comparing with \eqnref{eqn:stressEnergyNoethers:maxwell}

\begin{align*}
0
&= \PD{A_\sigma}{\LL} - \partial_\mu \PD{(\partial_\mu A_\sigma)}{\LL} \\
&= J^\sigma - 2 \kappa (\gamma^\mu \wedge \gamma^\sigma) \cdot (\gamma_\alpha \wedge \gamma_\beta) \partial_\mu \partial^\alpha A^\beta  \\
\end{align*}

Taking $\gamma^\sigma$ dot products with \eqnref{eqn:stressEnergyNoethers:maxwell} we have

\begin{align*}
0
&= \gamma^\sigma \cdot (J - \epsilon_0 c \grad \cdot F ) \\
&= J^\sigma - \epsilon_0 c \gamma^\sigma \cdot (\gamma^\mu \cdot (\gamma_\alpha \wedge \gamma_\beta)) \partial_\mu \partial^\alpha A^\beta \\
\end{align*}

So we have $2\kappa = -\epsilon_0 c$, and can write our Lagrangian density as
\begin{align}
\LL
&= -\frac{\epsilon_0}{2} (\grad \wedge A) \cdot (\grad \wedge A) + \frac{J}{c} \cdot A \\
&= -\frac{\epsilon_0}{2} (\gamma^\mu \wedge \gamma^\nu) \cdot (\gamma_\alpha \wedge \gamma_\beta) \partial_\mu A_\nu \partial^\alpha A^\beta + \frac{J^\sigma}{c} A_\sigma
\end{align}

\subsection{Energy momentum tensor}

For the Lagrangian density we have

\begin{align*}
\gamma_\mu \PD{(\partial_\mu A_\nu)}{\LL}
&= -{\epsilon_0} \gamma_\mu (\gamma^\mu \wedge \gamma^\nu) \cdot (\gamma_\alpha \wedge \gamma_\beta) \partial^\alpha A^\beta \\
&= -{\epsilon_0} \gamma_\mu ( {\delta^\mu}_\beta {\delta^\nu}_\alpha -{\delta^\mu}_\alpha {\delta^\nu}_\beta ) \partial^\alpha A^\beta \\
&= 
-{\epsilon_0} \gamma_\mu (\partial^\nu A^\mu -\partial^\mu A^\nu) \\
&= {\epsilon_0} \gamma_\mu F^{\mu\nu} \\
\end{align*}

One can guess that the vector contraction of $F^{\mu\nu}$ above is an expression of a dot product with our bivector field.  This is in fact the case

\begin{align*}
F \cdot \gamma^\nu
&=
(\gamma_\alpha \wedge \gamma_\beta) \cdot \gamma^\nu \partial^\alpha A^\beta \\
&=
(\gamma_\alpha {\delta_\beta}^\nu -\gamma_\beta {\delta_\alpha}^\nu ) \partial^\alpha A^\beta 
\\
&=
\gamma_\mu (\partial^\mu A^\nu - \partial^\nu A^\mu )
\\
&=
\gamma_\mu F^{\mu\nu}
\\
\end{align*}

We therefore have

\begin{align}
T(a) 
&= {\epsilon_0} (F \cdot \gamma^\nu) a \cdot \grad A_\nu - a \LL \\
&= {\epsilon_0} \left( (F \cdot \gamma^\nu) a \cdot \grad A_\nu + \frac{a}{2} F \cdot F \right) - a \left( A \cdot J/c \right)
\end{align}

\subsection{Index form of tensor}

Before trying to factor out $a$, let us expand the tensor in abstract index form.  This is
\begin{align*}
{T_\nu}^\mu
&=
T(\gamma_\nu) \cdot \gamma^\mu \\
&= {\epsilon_0} \left( F^{\mu\beta} \partial_\nu A_\beta + \frac{{\delta_\nu}^\mu}{2} F \cdot F \right) - {\delta_\nu}^\mu A^\sigma J_\sigma/c \\
&= {\epsilon_0} \left( F^{\mu\beta} \partial_\nu A_\beta - \frac{{\delta_\nu}^\mu}{4} F^{\alpha\beta}F_{\alpha\beta} \right) - {\delta_\nu}^\mu A^\sigma J_\sigma/c \\
\end{align*}

In particular, note that this is not the familiar symmetric tensor from the Poynting relations.

\subsection{Expansion in terms of \texorpdfstring{$\BE$ and $\BB$}{E and B}}

TODO.

\subsection{Adjoint}

Now, we want to move on to a computation of the adjoint so that $a$ can essentially be factored out.
Doing so is resisting initial attempts.  As an aid, introduce a few vector valued helper variables

\begin{align*}
F^\mu &= F \cdot \gamma^\mu \\
G_\mu &= \grad A_\nu \\
\end{align*}

Then we have

\begin{align*}
\grad \cdot T(a) 
&= \frac{\epsilon_0}{2} \left( \gpgradezero{ \grad(F^\nu (a G_\nu + G_\nu a)} + {a} \cdot \gpgradeone{\grad(F^2)} \right) - a \cdot \grad \left( A \cdot J/c \right) \\
&= \frac{\epsilon_0}{2} a \cdot 
\gpgradeone{ 
G_\nu \lrgrad F^\nu  
+\grad(F^\nu G_\nu)
+ \grad(F^2)} - a \cdot \grad \left( A \cdot J/c \right) \\
\end{align*}

This provides the adjoint energy momentum tensor, albeit in a form that looks like it can be reduced further

\begin{align}
0 &= \overbar{T}(\grad) = \frac{\epsilon_0}{2} \gpgradeone{ G_\nu \lrgrad F^\nu  +\grad(F^\nu G_\nu) + \grad(F^2)} - \grad \left( A \cdot J/c \right) 
\end{align}

We want to write this as a gradient of something, to determine the conserved quantity.  Getting part way is not too hard.

\begin{align*}
\overbar{T}(\grad) 
&= \frac{\epsilon_0}{2} 
\mathLabelBox{
\left(
\gpgradeone{ G_\nu \lrgrad F^\nu } + \grad \cdot (F^\nu \wedge G_\nu) 
\right)
}{$\conj$}
+ \grad \left( \frac{\epsilon_0}{2} (F^\nu \cdot G_\nu + F \cdot F) - A \cdot J/c \right) 
\end{align*}

It would be nice if these first two terms $\conj$ cancel.  Can we be so lucky?

\begin{align*}
(*) &=
\gpgradeone{ G_\nu \lrgrad F^\nu } + \grad \cdot (F^\nu \wedge G_\nu) \\
&=
\gpgradeone{ 
(G_\nu \lgrad) F^\nu 
+G_\nu (\rgrad F^\nu)
} 
+ (\grad \cdot F^\nu) G_\nu
- F^\nu (\grad \cdot G_\nu)  \\
&=
(\grad \cdot G_\nu) F^\nu 
+F^\nu \cdot (\grad \wedge G_\nu ) 
+G_\nu (\grad \cdot F^\nu)
+G_\nu \cdot (\grad \wedge F^\nu)
+ (\grad \cdot F^\nu) G_\nu
- F^\nu (\grad \cdot G_\nu)  \\
&=
2 G_\nu (\grad \cdot F^\nu) +G_\nu \cdot (\grad \wedge F^\nu) \\
%&=
% G_\nu (\grad \cdot F^\nu) + \gpgradeone{G_\nu (\grad F^\nu)} \\
\end{align*}

This is not obviously zero.  How about $F^\nu \cdot G_\nu$?

\begin{align*}
F^\nu \cdot G_\nu 
&=
\gpgradezero{((\gamma_\alpha \wedge \gamma_\beta) \cdot \gamma^\nu) \gamma^\sigma } \partial^\alpha A^\beta \partial_\sigma A_\nu \\
&=
({\delta_\alpha}^\sigma {\delta_\beta}^\nu - {\delta_\beta}^\sigma {\delta_\alpha}^\nu) \partial^\alpha A^\beta \partial_\sigma A_\nu \\
&=
\partial^\alpha A^\beta (\partial_\alpha A_\beta - \partial_\beta A_\alpha) \\
&=
\partial^\alpha A^\beta F_{\alpha\beta} \\
&=
\inv{2} F_{\alpha\beta} F^{\alpha\beta} 
\end{align*}

Ah.  Up to a sign, this was $F \cdot F$.  What is the sign?

\begin{align*}
F \cdot F
&=
(\gamma_\alpha \wedge \gamma_\beta) \cdot (\gamma^\mu \wedge \gamma^\nu) \partial^\alpha A^\beta \partial_\mu A_\nu \\
&=
({\delta_\alpha}^\nu {\delta_\beta}^\mu - {\delta_\beta}^\nu {\delta_\alpha}^\mu) \partial^\alpha A^\beta \partial_\mu A_\nu \\
&=
\partial^\alpha A^\beta (\partial_\beta A_\alpha - \partial_\alpha A_\beta) \\
&=
\partial^\alpha A^\beta F_{\beta\alpha} \\
&=
\inv{2} 
F_{\beta\alpha} ( \partial^\alpha A^\beta -\partial^\beta A^\alpha ) \\
&=
\inv{2} F_{\beta\alpha} F^{\alpha\beta} \\
&= - F^\nu \cdot G_\nu
\end{align*}

Bad first guess.  It is the second two terms that cancel, not the first, leaving us with

\begin{align*}
\overbar{T}(\grad) 
&= \frac{\epsilon_0}{2} \left(
\gpgradeone{ G_\nu \lrgrad F^\nu }
+ \grad \cdot (F^\nu \wedge G_\nu) 
\right)
- \grad \left( A \cdot J/c \right) 
\end{align*}

Now, intuition tells me that it ought to be possible to simplify this further,
in particular, eliminating the $\nu$ indices.

Think I will take a break from this for a while, and come back to it later.

\section{Nomenclature.  Linearized spacetime translation}

Applying the translation $x^\mu \rightarrow x^\mu + e^\mu$, is what I thought
would be called ``spacetime translation''.  But to do so we need higher
order powers of the exponential vector translation operator (ie:
multivariable Taylor series operator)

\begin{align*}
\sum_k (1/k!) (e^\mu \partial_\mu)^k
\end{align*}

The transformation that appears to result in the canonical
energy momentum tensor has only the linear term of this operator, so I called
it ``linearized spacetime translation operator'', which seemed like a
better name (to me).  That is all.  My guess is that what is typically
referred to as the spacetime translation that generates the canonical
energy momentum tensor is really just the first order term of the
translation operation, and not truly a complete translation.  If
that is the case, then dropping the linearized adjective would probably
be reasonable.

It is somewhat odd that the derived conditions for a divergence added
to the Lagrangian are immediately busted by the wave equation.
I think the saving grace is the fact that
an arbitrary $\partial_\mu F^\mu$ is not necessarily a symmetry is the
fact the translation of the coordinates is not an arbitrary divergence.
 This directional derivative operator is applied to the Lagrangian
itself and not to an arbitrary function.  This builds in the required
symmetry (you could also add in or subtract out additional divergence
terms that meet the derived conditions and not change anything).

Now, if the first order term of the Taylor expansion is a symmetry
because we can commute the field partials and the coordinate partials
then the higher order terms should also be symmetries.  This would mean
that a true translation $\LL \rightarrow \exp(e^\mu \partial_\mu) \LL$
would also be a symmetry.  What conservation current would we get from
that?  Would it be the symmetric energy momentum tensor?

%Yes, I had expect that the first order term of the Taylor expansion of
%the translation would be related to an infinitesimal operation, but
%have not thought that through in detail.  Additionally both Tong and
%D&L include inverse operators when they treat more general
%transformations than translations, and I have not figured out where
%those come from.

\documentclass{article}

\usepackage{amsmath}
\usepackage{mathpazo}

%
% shorthand for bold symbols, convenient for vectors and matrices
%
\newcommand{\Ba}[0]{\mathbf{a}}
\newcommand{\Bb}[0]{\mathbf{b}}
\newcommand{\Bc}[0]{\mathbf{c}}
\newcommand{\Bd}[0]{\mathbf{d}}
\newcommand{\Be}[0]{\mathbf{e}}
\newcommand{\Bf}[0]{\mathbf{f}}
\newcommand{\Bg}[0]{\mathbf{g}}
\newcommand{\Bh}[0]{\mathbf{h}}
\newcommand{\Bi}[0]{\mathbf{i}}
\newcommand{\Bj}[0]{\mathbf{j}}
\newcommand{\Bk}[0]{\mathbf{k}}
\newcommand{\Bl}[0]{\mathbf{l}}
\newcommand{\Bm}[0]{\mathbf{m}}
\newcommand{\Bn}[0]{\mathbf{n}}
\newcommand{\Bo}[0]{\mathbf{o}}
\newcommand{\Bp}[0]{\mathbf{p}}
\newcommand{\Bq}[0]{\mathbf{q}}
\newcommand{\Br}[0]{\mathbf{r}}
\newcommand{\Bs}[0]{\mathbf{s}}
\newcommand{\Bt}[0]{\mathbf{t}}
\newcommand{\Bu}[0]{\mathbf{u}}
\newcommand{\Bv}[0]{\mathbf{v}}
\newcommand{\Bw}[0]{\mathbf{w}}
\newcommand{\Bx}[0]{\mathbf{x}}
\newcommand{\By}[0]{\mathbf{y}}
\newcommand{\Bz}[0]{\mathbf{z}}
\newcommand{\BA}[0]{\mathbf{A}}
\newcommand{\BB}[0]{\mathbf{B}}
\newcommand{\BC}[0]{\mathbf{C}}
\newcommand{\BD}[0]{\mathbf{D}}
\newcommand{\BE}[0]{\mathbf{E}}
\newcommand{\BF}[0]{\mathbf{F}}
\newcommand{\BG}[0]{\mathbf{G}}
\newcommand{\BH}[0]{\mathbf{H}}
\newcommand{\BI}[0]{\mathbf{I}}
\newcommand{\BJ}[0]{\mathbf{J}}
\newcommand{\BK}[0]{\mathbf{K}}
\newcommand{\BL}[0]{\mathbf{L}}
\newcommand{\BM}[0]{\mathbf{M}}
\newcommand{\BN}[0]{\mathbf{N}}
\newcommand{\BO}[0]{\mathbf{O}}
\newcommand{\BP}[0]{\mathbf{P}}
\newcommand{\BQ}[0]{\mathbf{Q}}
\newcommand{\BR}[0]{\mathbf{R}}
\newcommand{\BS}[0]{\mathbf{S}}
\newcommand{\BT}[0]{\mathbf{T}}
\newcommand{\BU}[0]{\mathbf{U}}
\newcommand{\BV}[0]{\mathbf{V}}
\newcommand{\BW}[0]{\mathbf{W}}
\newcommand{\BX}[0]{\mathbf{X}}
\newcommand{\BY}[0]{\mathbf{Y}}
\newcommand{\BZ}[0]{\mathbf{Z}}

\newcommand{\Bzero}[0]{\mathbf{0}}
\newcommand{\Btheta}[0]{\boldsymbol{\theta}}
\newcommand{\Btau}[0]{\boldsymbol{\tau}}
\newcommand{\Bomega}[0]{\boldsymbol{\omega}}

%
% shorthand for unit vectors
%
\newcommand{\acap}[0]{\hat{\Ba}}
\newcommand{\bcap}[0]{\hat{\Bb}}
\newcommand{\ccap}[0]{\hat{\Bc}}
\newcommand{\dcap}[0]{\hat{\Bd}}
\newcommand{\ecap}[0]{\hat{\Be}}
\newcommand{\fcap}[0]{\hat{\Bf}}
\newcommand{\gcap}[0]{\hat{\Bg}}
\newcommand{\hcap}[0]{\hat{\Bh}}
\newcommand{\icap}[0]{\hat{\Bi}}
\newcommand{\jcap}[0]{\hat{\Bj}}
\newcommand{\kcap}[0]{\hat{\Bk}}
\newcommand{\lcap}[0]{\hat{\Bl}}
\newcommand{\mcap}[0]{\hat{\Bm}}
\newcommand{\ncap}[0]{\hat{\Bn}}
\newcommand{\ocap}[0]{\hat{\Bo}}
\newcommand{\pcap}[0]{\hat{\Bp}}
\newcommand{\qcap}[0]{\hat{\Bq}}
\newcommand{\rcap}[0]{\hat{\Br}}
\newcommand{\scap}[0]{\hat{\Bs}}
\newcommand{\tcap}[0]{\hat{\Bt}}
\newcommand{\ucap}[0]{\hat{\Bu}}
\newcommand{\vcap}[0]{\hat{\Bv}}
\newcommand{\wcap}[0]{\hat{\Bw}}
\newcommand{\xcap}[0]{\hat{\Bx}}
\newcommand{\ycap}[0]{\hat{\By}}
\newcommand{\zcap}[0]{\hat{\Bz}}
\newcommand{\thetacap}[0]{\hat{\Btheta}}

%
% to write R^n and C^n in a distinguishable fashion.  Perhaps change this
% to the double lined characters upon figuring out how to do so.
%
\newcommand{\C}[1]{$\mathbb{C}^{#1}$}
\newcommand{\R}[1]{$\mathbb{R}^{#1}$}

%
% various generally useful helpers
%

% derivative of #1 wrt. #2:
\newcommand{\D}[2] {\frac {d#2} {d#1}}

\newcommand{\inv}[1]{\frac{1}{#1}}
\newcommand{\cross}[0]{\times}

\newcommand{\abs}[1]{\lvert{#1}\rvert}
\newcommand{\norm}[1]{\lVert{#1}\rVert}
\newcommand{\innerprod}[2]{\langle{#1}, {#2}\rangle}
\newcommand{\dotprod}[2]{{#1} \cdot {#2}}
\newcommand{\bdotprod}[2]{\left({#1} \cdot {#2}\right)}
\newcommand{\crossprod}[2]{{#1} \cross {#2}}
\newcommand{\tripleprod}[3]{\dotprod{\left(\crossprod{#1}{#2}\right)}{#3}}

\DeclareMathOperator{\Proj}{Proj}
\DeclareMathOperator{\Span}{span}
\DeclareMathOperator{\Sgn}{sgn}
\DeclareMathOperator{\Area}{Area}
\DeclareMathOperator{\Volume}{Volume}

%
% A few miscellaneous things specific to this document
%
\newcommand{\crossop}[1]{\crossprod{#1}{}}

% R2 vector.
\newcommand{\VectorTwo}[2]{
\begin{bmatrix}
 {#1} \\
 {#2}
\end{bmatrix}
}

\newcommand{\VectorN}[1]{
\begin{bmatrix}
{#1}_1 \\
{#1}_2 \\
\vdots \\
{#1}_N \\
\end{bmatrix}
}

\newcommand{\DETuvij}[4]{
\begin{vmatrix}
 {#1}_{#3} & {#1}_{#4} \\
 {#2}_{#3} & {#2}_{#4}
\end{vmatrix}
}

\newcommand{\DETuvwijk}[6]{
\begin{vmatrix}
 {#1}_{#4} & {#1}_{#5} & {#1}_{#6} \\
 {#2}_{#4} & {#2}_{#5} & {#2}_{#6} \\
 {#3}_{#4} & {#3}_{#5} & {#3}_{#6}
\end{vmatrix}
}

\newcommand{\DETuvwxijkl}[8]{
\begin{vmatrix}
 {#1}_{#5} & {#1}_{#6} & {#1}_{#7} & {#1}_{#8} \\
 {#2}_{#5} & {#2}_{#6} & {#2}_{#7} & {#2}_{#8} \\
 {#3}_{#5} & {#3}_{#6} & {#3}_{#7} & {#3}_{#8} \\
 {#4}_{#5} & {#4}_{#6} & {#4}_{#7} & {#4}_{#8} \\
\end{vmatrix}
}

%\newcommand{\DETuvwxyijklm}[10]{
%\begin{vmatrix}
% {#1}_{#6} & {#1}_{#7} & {#1}_{#8} & {#1}_{#9} & {#1}_{#10} \\
% {#2}_{#6} & {#2}_{#7} & {#2}_{#8} & {#2}_{#9} & {#2}_{#10} \\
% {#3}_{#6} & {#3}_{#7} & {#3}_{#8} & {#3}_{#9} & {#3}_{#10} \\
% {#4}_{#6} & {#4}_{#7} & {#4}_{#8} & {#4}_{#9} & {#4}_{#10} \\
% {#5}_{#6} & {#5}_{#7} & {#5}_{#8} & {#5}_{#9} & {#5}_{#10}
%\end{vmatrix}
%}

% R3 vector.
\newcommand{\VectorThree}[3]{
\begin{bmatrix}
 {#1} \\
 {#2} \\
 {#3}
\end{bmatrix}
}


%<misc>
%
\newcommand{\Abs}[1]{{\left\lvert{#1}\right\rvert}}
\newcommand{\spacegrad}[0]{\boldsymbol{\nabla}}
\newcommand{\grad}[0]{\nabla}
\newcommand{\LL}[0]{\mathcal{L}}

% == \partial_{#1} {#2}
\newcommand{\PD}[2]{\frac{\partial {#2}}{\partial {#1}}}
% inline variant
\newcommand{\PDi}[2]{{\partial {#2}}/{\partial {#1}}}

\newcommand{\PDD}[3]{\frac{\partial^2 {#3}}{\partial {#1}\partial {#2}}}
%\newcommand{\PDd}[2]{\frac{\partial^2 {#2}}{{\partial{#1}}^2}}
\newcommand{\PDsq}[2]{\frac{\partial^2 {#2}}{(\partial {#1})^2}}

\newcommand{\Partial}[2]{\frac{\partial {#1}}{\partial {#2}}}
\DeclareMathOperator{\RejName}{Rej}
\newcommand{\Rej}[2]{\RejName_{#1}\left( {#2} \right)}
\newcommand{\Rm}[1]{\mathbb{R}^{#1}}
\newcommand{\Cm}[1]{\mathbb{C}^{#1}}
\newcommand{\conj}[0]{{*}}

%</misc>

% <grade selection>
%
\newcommand{\gpgrade}[2] {{\left\langle{{#1}}\right\rangle}_{#2}}

\newcommand{\gpgradezero}[1] {\gpgrade{#1}{}}
%\newcommand{\gpscalargrade}[1] {{\left\langle{{#1}}\right\rangle}}
%\newcommand{\gpgradezero}[1] {\gpgrade{#1}{0}}

%\newcommand{\gpgradeone}[1] {{\left\langle{{#1}}\right\rangle}_{1}}
\newcommand{\gpgradeone}[1] {\gpgrade{#1}{1}}

\newcommand{\gpgradetwo}[1] {\gpgrade{#1}{2}}
\newcommand{\gpgradethree}[1] {\gpgrade{#1}{3}}
\newcommand{\gpgradefour}[1] {\gpgrade{#1}{4}}
%
% </grade selection>



\newcommand{\adot}[0]{{\dot{a}}}
\newcommand{\bdot}[0]{{\dot{b}}}
% taken for centered dot:
%\newcommand{\cdot}[0]{{\dot{c}}}
%\newcommand{\ddot}[0]{{\dot{d}}}
\newcommand{\edot}[0]{{\dot{e}}}
\newcommand{\fdot}[0]{{\dot{f}}}
\newcommand{\gdot}[0]{{\dot{g}}}
\newcommand{\hdot}[0]{{\dot{h}}}
\newcommand{\idot}[0]{{\dot{i}}}
\newcommand{\jdot}[0]{{\dot{j}}}
\newcommand{\kdot}[0]{{\dot{k}}}
\newcommand{\ldot}[0]{{\dot{l}}}
\newcommand{\mdot}[0]{{\dot{m}}}
\newcommand{\ndot}[0]{{\dot{n}}}
%\newcommand{\odot}[0]{{\dot{o}}}
\newcommand{\pdot}[0]{{\dot{p}}}
\newcommand{\qdot}[0]{{\dot{q}}}
\newcommand{\rdot}[0]{{\dot{r}}}
\newcommand{\sdot}[0]{{\dot{s}}}
\newcommand{\tdot}[0]{{\dot{t}}}
\newcommand{\udot}[0]{{\dot{u}}}
\newcommand{\vdot}[0]{{\dot{v}}}
\newcommand{\wdot}[0]{{\dot{w}}}
\newcommand{\xdot}[0]{{\dot{x}}}
\newcommand{\ydot}[0]{{\dot{y}}}
\newcommand{\zdot}[0]{{\dot{z}}}
\newcommand{\addot}[0]{{\ddot{a}}}
\newcommand{\bddot}[0]{{\ddot{b}}}
\newcommand{\cddot}[0]{{\ddot{c}}}
%\newcommand{\dddot}[0]{{\ddot{d}}}
\newcommand{\eddot}[0]{{\ddot{e}}}
\newcommand{\fddot}[0]{{\ddot{f}}}
\newcommand{\gddot}[0]{{\ddot{g}}}
\newcommand{\hddot}[0]{{\ddot{h}}}
\newcommand{\iddot}[0]{{\ddot{i}}}
\newcommand{\jddot}[0]{{\ddot{j}}}
\newcommand{\kddot}[0]{{\ddot{k}}}
\newcommand{\lddot}[0]{{\ddot{l}}}
\newcommand{\mddot}[0]{{\ddot{m}}}
\newcommand{\nddot}[0]{{\ddot{n}}}
\newcommand{\oddot}[0]{{\ddot{o}}}
\newcommand{\pddot}[0]{{\ddot{p}}}
\newcommand{\qddot}[0]{{\ddot{q}}}
\newcommand{\rddot}[0]{{\ddot{r}}}
\newcommand{\sddot}[0]{{\ddot{s}}}
\newcommand{\tddot}[0]{{\ddot{t}}}
\newcommand{\uddot}[0]{{\ddot{u}}}
\newcommand{\vddot}[0]{{\ddot{v}}}
\newcommand{\wddot}[0]{{\ddot{w}}}
\newcommand{\xddot}[0]{{\ddot{x}}}
\newcommand{\yddot}[0]{{\ddot{y}}}
\newcommand{\zddot}[0]{{\ddot{z}}}

%<bold and dot greek symbols>
%

\newcommand{\Deltadot}[0]{{\dot{\Delta}}}
\newcommand{\Gammadot}[0]{{\dot{\Gamma}}}
\newcommand{\Lambdadot}[0]{{\dot{\Lambda}}}
\newcommand{\Omegadot}[0]{{\dot{\Omega}}}
\newcommand{\Phidot}[0]{{\dot{\Phi}}}
\newcommand{\Pidot}[0]{{\dot{\Pi}}}
\newcommand{\Psidot}[0]{{\dot{\Psi}}}
\newcommand{\Sigmadot}[0]{{\dot{\Sigma}}}
\newcommand{\Thetadot}[0]{{\dot{\Theta}}}
\newcommand{\Upsilondot}[0]{{\dot{\Upsilon}}}
\newcommand{\Xidot}[0]{{\dot{\Xi}}}
\newcommand{\alphadot}[0]{{\dot{\alpha}}}
\newcommand{\betadot}[0]{{\dot{\beta}}}
\newcommand{\chidot}[0]{{\dot{\chi}}}
\newcommand{\deltadot}[0]{{\dot{\delta}}}
\newcommand{\epsilondot}[0]{{\dot{\epsilon}}}
\newcommand{\etadot}[0]{{\dot{\eta}}}
\newcommand{\gammadot}[0]{{\dot{\gamma}}}
\newcommand{\kappadot}[0]{{\dot{\kappa}}}
\newcommand{\lambdadot}[0]{{\dot{\lambda}}}
\newcommand{\mudot}[0]{{\dot{\mu}}}
\newcommand{\nudot}[0]{{\dot{\nu}}}
\newcommand{\omegadot}[0]{{\dot{\omega}}}
\newcommand{\phidot}[0]{{\dot{\phi}}}
\newcommand{\pidot}[0]{{\dot{\pi}}}
\newcommand{\psidot}[0]{{\dot{\psi}}}
\newcommand{\rhodot}[0]{{\dot{\rho}}}
\newcommand{\sigmadot}[0]{{\dot{\sigma}}}
\newcommand{\taudot}[0]{{\dot{\tau}}}
\newcommand{\thetadot}[0]{{\dot{\theta}}}
\newcommand{\upsilondot}[0]{{\dot{\upsilon}}}
\newcommand{\varepsilondot}[0]{{\dot{\varepsilon}}}
\newcommand{\varphidot}[0]{{\dot{\varphi}}}
\newcommand{\varpidot}[0]{{\dot{\varpi}}}
\newcommand{\varrhodot}[0]{{\dot{\varrho}}}
\newcommand{\varsigmadot}[0]{{\dot{\varsigma}}}
\newcommand{\varthetadot}[0]{{\dot{\vartheta}}}
\newcommand{\xidot}[0]{{\dot{\xi}}}
\newcommand{\zetadot}[0]{{\dot{\zeta}}}

\newcommand{\Deltaddot}[0]{{\ddot{\Delta}}}
\newcommand{\Gammaddot}[0]{{\ddot{\Gamma}}}
\newcommand{\Lambdaddot}[0]{{\ddot{\Lambda}}}
\newcommand{\Omegaddot}[0]{{\ddot{\Omega}}}
\newcommand{\Phiddot}[0]{{\ddot{\Phi}}}
\newcommand{\Piddot}[0]{{\ddot{\Pi}}}
\newcommand{\Psiddot}[0]{{\ddot{\Psi}}}
\newcommand{\Sigmaddot}[0]{{\ddot{\Sigma}}}
\newcommand{\Thetaddot}[0]{{\ddot{\Theta}}}
\newcommand{\Upsilonddot}[0]{{\ddot{\Upsilon}}}
\newcommand{\Xiddot}[0]{{\ddot{\Xi}}}
\newcommand{\alphaddot}[0]{{\ddot{\alpha}}}
\newcommand{\betaddot}[0]{{\ddot{\beta}}}
\newcommand{\chiddot}[0]{{\ddot{\chi}}}
\newcommand{\deltaddot}[0]{{\ddot{\delta}}}
\newcommand{\epsilonddot}[0]{{\ddot{\epsilon}}}
\newcommand{\etaddot}[0]{{\ddot{\eta}}}
\newcommand{\gammaddot}[0]{{\ddot{\gamma}}}
\newcommand{\kappaddot}[0]{{\ddot{\kappa}}}
\newcommand{\lambdaddot}[0]{{\ddot{\lambda}}}
\newcommand{\muddot}[0]{{\ddot{\mu}}}
\newcommand{\nuddot}[0]{{\ddot{\nu}}}
\newcommand{\omegaddot}[0]{{\ddot{\omega}}}
\newcommand{\phiddot}[0]{{\ddot{\phi}}}
\newcommand{\piddot}[0]{{\ddot{\pi}}}
\newcommand{\psiddot}[0]{{\ddot{\psi}}}
\newcommand{\rhoddot}[0]{{\ddot{\rho}}}
\newcommand{\sigmaddot}[0]{{\ddot{\sigma}}}
\newcommand{\tauddot}[0]{{\ddot{\tau}}}
\newcommand{\thetaddot}[0]{{\ddot{\theta}}}
\newcommand{\upsilonddot}[0]{{\ddot{\upsilon}}}
\newcommand{\varepsilonddot}[0]{{\ddot{\varepsilon}}}
\newcommand{\varphiddot}[0]{{\ddot{\varphi}}}
\newcommand{\varpiddot}[0]{{\ddot{\varpi}}}
\newcommand{\varrhoddot}[0]{{\ddot{\varrho}}}
\newcommand{\varsigmaddot}[0]{{\ddot{\varsigma}}}
\newcommand{\varthetaddot}[0]{{\ddot{\vartheta}}}
\newcommand{\xiddot}[0]{{\ddot{\xi}}}
\newcommand{\zetaddot}[0]{{\ddot{\zeta}}}

\newcommand{\BDelta}[0]{\boldsymbol{\Delta}}
\newcommand{\BGamma}[0]{\boldsymbol{\Gamma}}
\newcommand{\BLambda}[0]{\boldsymbol{\Lambda}}
\newcommand{\BOmega}[0]{\boldsymbol{\Omega}}
\newcommand{\BPhi}[0]{\boldsymbol{\Phi}}
\newcommand{\BPi}[0]{\boldsymbol{\Pi}}
\newcommand{\BPsi}[0]{\boldsymbol{\Psi}}
\newcommand{\BSigma}[0]{\boldsymbol{\Sigma}}
\newcommand{\BTheta}[0]{\boldsymbol{\Theta}}
\newcommand{\BUpsilon}[0]{\boldsymbol{\Upsilon}}
\newcommand{\BXi}[0]{\boldsymbol{\Xi}}
\newcommand{\Balpha}[0]{\boldsymbol{\alpha}}
\newcommand{\Bbeta}[0]{\boldsymbol{\beta}}
\newcommand{\Bchi}[0]{\boldsymbol{\chi}}
\newcommand{\Bdelta}[0]{\boldsymbol{\delta}}
\newcommand{\Bepsilon}[0]{\boldsymbol{\epsilon}}
\newcommand{\Beta}[0]{\boldsymbol{\eta}}
\newcommand{\Bgamma}[0]{\boldsymbol{\gamma}}
\newcommand{\Bkappa}[0]{\boldsymbol{\kappa}}
\newcommand{\Blambda}[0]{\boldsymbol{\lambda}}
\newcommand{\Bmu}[0]{\boldsymbol{\mu}}
\newcommand{\Bnu}[0]{\boldsymbol{\nu}}
%\newcommand{\Bomega}[0]{\boldsymbol{\omega}}
\newcommand{\Bphi}[0]{\boldsymbol{\phi}}
\newcommand{\Bpi}[0]{\boldsymbol{\pi}}
\newcommand{\Bpsi}[0]{\boldsymbol{\psi}}
\newcommand{\Brho}[0]{\boldsymbol{\rho}}
\newcommand{\Bsigma}[0]{\boldsymbol{\sigma}}
%\newcommand{\Btau}[0]{\boldsymbol{\tau}}
%\newcommand{\Btheta}[0]{\boldsymbol{\theta}}
\newcommand{\Bupsilon}[0]{\boldsymbol{\upsilon}}
\newcommand{\Bvarepsilon}[0]{\boldsymbol{\varepsilon}}
\newcommand{\Bvarphi}[0]{\boldsymbol{\varphi}}
\newcommand{\Bvarpi}[0]{\boldsymbol{\varpi}}
\newcommand{\Bvarrho}[0]{\boldsymbol{\varrho}}
\newcommand{\Bvarsigma}[0]{\boldsymbol{\varsigma}}
\newcommand{\Bvartheta}[0]{\boldsymbol{\vartheta}}
\newcommand{\Bxi}[0]{\boldsymbol{\xi}}
\newcommand{\Bzeta}[0]{\boldsymbol{\zeta}}
%
%</bold and dot greek symbols>
%<infrequent>
%
%\newcommand{\AreaOp}[1]{\AName_{#1}}
%\newcommand{\Babs}[0]{\abs{\BB}}
%\newcommand{\Bcap}[0]{\hat{\BB}}
%\newcommand{\BrPrimeRej}[0]{\rcap(\rcap \wedge \Br')}
%\newcommand{\CA}[0]{\mathcal{A}}
%\newcommand{\Cos}[1]{\cos{\left({#1}\right)}}
%\newcommand{\Det}[1] {\abs{#1}}
%\newcommand{\Dsq}[2] {\frac {\partial^2 {#1}} {\partial {#2}^2}}
%\newcommand{\Exp}[1]{\exp{\left({#1}\right)}}
%\newcommand{\Norm}[1]{\left\lVert{#1}\right\rVert}
%\newcommand{\Sin}[1]{\sin{\left({#1}\right)}}
%\newcommand{\T}[0]{\text{T}}
%\newcommand{\VolumeOp}[1]{\VName_{#1}}
%\newcommand{\agrad}[0]{\Ba \cdot \nabla}
%\newcommand{\alphacap}[0]{\hat{\boldsymbol{\alpha}}}
%\newcommand{\Fcap}[0]{\hat{\BF}}
%\newcommand{\bithree}[0]{{\Bi}_3}
%\newcommand{\bxa}[0]{\Bx\Ba}
%\newcommand{\coordvec}[2]{
%\newcommand{\costheta}[0]{\acap \cdot \xcap}
%\newcommand{\ddt}[1]{\ddot{#1}}
%\newcommand{\ddu}[1] {\frac {d{#1}} {du}}
%\newcommand{\dsqxj}[2] {\frac {\partial^2 {#1}} {\partial {x_{#2}}^2}}
%\newcommand{\dtheta}[1]{\frac{d {#1}}{d \theta}}
%\newcommand{\dt}[1]{\dot{#1}}
%\newcommand{\dt}[1]{\frac{d {#1}}{dt}}
%\newcommand{\dxj}[2] {\frac {\partial {#1}} {\partial {x_{#2}}}}
%\newcommand{\halfPhi}[0]{\frac{\phi}{2}}
%\newcommand{\half}[0]{\inv{2}}
%\newcommand{\inv}[1]{\frac{1}{#1}}
%\newcommand{\laplacian}[0]{\nabla^2}
%\newcommand{\matrixoftx}[3]{
%\newcommand{\nrrp}[0]{\norm{\rcap \wedge \Br'}}
%\newcommand{\oiint}{\bigcirc \hspace{-1.4em} \int \hspace{-.8em} \int}
%\newcommand{\transpose}[1]{{#1}^{\text{T}}}
%\newcommand{\transpose}[1]{{{#1}^{\TextTranspose}}}
%\newcommand{\transpose}[1]{{{#1}^{\text{T}}}}
%\newcommand{\barA}[0]{\bar{A}}
%\newcommand{\qbar}[0]{\bar{q}}
%\newcommand{\qdotbar}[0]{\dot{\bar{q}}}
%
%</infrequent>




\newcommand{\symmetric}[2]{{\left\{{#1},{#2}\right\}}}
\newcommand{\antisymmetric}[2]{\left[{#1},{#2}\right]}
\DeclareMathOperator{\sgn}{sgn}
\DeclareMathOperator{\something}{something}

\newcommand{\uDETuvij}[4]{
\begin{vmatrix}
 {#1}^{#3} & {#1}^{#4} \\
 {#2}^{#3} & {#2}^{#4}
\end{vmatrix}
}

\newcommand{\PDSq}[2]{\frac{\partial^2 {#2}}{\partial {#1}^2}}
\newcommand{\transpose}[1]{{#1}^{\mathrm{T}}}
\newcommand{\stardot}[0]{{*}}

% bivector.tex:
\newcommand{\laplacian}[0]{\nabla^2}
\newcommand{\Dsq}[2] {\frac {\partial^2 {#1}} {\partial {#2}^2}}
\newcommand{\dxj}[2] {\frac {\partial {#1}} {\partial {x_{#2}}}}
\newcommand{\dsqxj}[2] {\frac {\partial^2 {#1}} {\partial {x_{#2}}^2}}
\DeclareMathOperator{\ExpName}{e}
%\DeclareMathOperator{\Exp}{e}
%\newcommand{\Exp}[1]{\exp{\left({#1}\right)}}
%\DeclareMathOperator{\Rej}{Rej}
\DeclareMathOperator{\Rot}{R}
%\newcommand{\gpgrade}[2] {{\left\langle{{#1}}\right\rangle}_{#2}}
%\newcommand{\gpgradezero}[1] {\gpgrade{#1}{0}}
%\newcommand{\gpgradetwo}[1] {\gpgrade{#1}{2}}
%\newcommand{\gpgradefour}[1] {\gpgrade{#1}{4}}

% ga_wiki_torque.tex:
\newcommand{\Fcap}[0]{\hat{\BF}}
\newcommand{\bithree}[0]{{\Bi}_3}
\newcommand{\nrrp}[0]{\norm{\rcap \wedge \Br'}}
\newcommand{\dtheta}[1]{\frac{d {#1}}{d \theta}}

% ga_wiki_unit_derivative.tex
\newcommand{\dt}[1]{\frac{d {#1}}{dt}}
\newcommand{\BrPrimeRej}[0]{\rcap(\rcap \wedge \Br')}

% radial_vector_derivatives.tex:
%\newcommand{\BrPrimeRej}[0]{\rcap(\rcap \wedge \Br')}

% angular_velocity.tex

%\newcommand{\dt}[1]{\frac{d {#1}}{dt}}
%\newcommand{\Norm}[1]{\left\lVert{#1}\right\rVert}
%\newcommand{\dtheta}[1]{\frac{d {#1}}{d \theta}}

% reciprocal_frame.tex
\DeclareMathOperator{\AbsName}{abs}

%\DeclareMathOperator{\RejName}{Rej}
%\newcommand{\Rej}[2]{\RejName_{#1}\left( {#2} \right)}

\DeclareMathOperator{\AName}{A}
\newcommand{\AreaOp}[1]{\AName_{#1}}

\DeclareMathOperator{\VName}{V}
\newcommand{\VolumeOp}[1]{\VName_{#1}}

%\newcommand{\gpgrade}[2] {{\left\langle{{#1}}\right\rangle}_{#2}}
%\newcommand{\gpgradeone}[1] {{\left\langle{{#1}}\right\rangle}_{1}}


% projection_with_matrix_comparison.tex
%\DeclareMathOperator{\Transpose}{T}
\DeclareMathOperator{\rank}{rank}
%\newcommand{\transpose}[1]{{{#1}^{\TextTranspose}}}
%\newcommand{\transpose}[1]{{{#1}^{\text{T}}}}
\newcommand{\T}[0]{{\text{T}}}
%\newcommand{\BOmega}[0]{\boldsymbol{\Omega}}

%\newcommand{\Det}[1] {\abs{#1}}

% oblique_proj.tex
%\newcommand{\T}[0]{\text{T}}
%\newcommand{\Bbeta}[0]{\boldsymbol{\beta}}

% spherical_polar.tex
\newcommand{\phicap}[0]{\hat{\boldsymbol{\phi}}}
\newcommand{\Lor}[2]{{{\Lambda^{#1}}_{#2}}}
\newcommand{\ILor}[2]{{{ \{{\Lambda^{-1}\} }^{#1}}_{#2}}}

% slerp.tex
\DeclareMathOperator{\atan2}{atan2}

% kvector_exponential.tex
%\DeclareMathOperator{\Exp}{e}
%\DeclareMathOperator{\Rej}{Rej}
\newcommand{\Bcap}[0]{\hat{\BB}}
\newcommand{\Babs}[0]{\abs{\BB}}
%\newcommand{\gpgrade}[2] {{\left\langle{{#1}}\right\rangle}_{#2}}
%\newcommand{\gpgradezero}[1] {\gpgrade{#1}{0}}
%\newcommand{\gpgradetwo}[1] {\gpgrade{#1}{2}}
%\newcommand{\gpgradefour}[1] {\gpgrade{#1}{4}}

\newcommand{\ddu}[1] {\frac {d{#1}} {du}}

% vector_integral_relations.tex
%\newcommand{\Oiint}{\bigcirc \hspace{-1.4em} \int \hspace{-.8em} \int}

% legendre.tex
\newcommand{\agrad}[0]{\Ba \cdot \nabla}
\newcommand{\bxa}[0]{\Bx\Ba}
\newcommand{\costheta}[0]{\acap \cdot \xcap}
%\newcommand{\inv}[1]{\frac{1}{#1}}
\newcommand{\half}[0]{\inv{2}}

% ke_rotation.tex
\newcommand{\DotT}[1]{\dot{#1}}
\newcommand{\DDotT}[1]{\ddot{#1}}
%\newcommand{\transpose}[1]{{#1}^{\text{T}}}
%\newcommand{\Balpha}[0]{\boldsymbol{\alpha}}

%\newcommand{\gpgrade}[2] {{\left\langle{{#1}}\right\rangle}_{#2}}
%\newcommand{\gpgradeone}[1] {{\left\langle{{#1}}\right\rangle}_{1}}
\newcommand{\gpscalargrade}[1] {{\left\langle{{#1}}\right\rangle}}
%\newcommand{\BOmega}[0]{\boldsymbol{\Omega}}

% gaussian_surface.tex
%\newcommand{\phicap}[0]{\hat{\Bphi}}

% newtonian_lagrangian_and_gradient.tex
% PD macro that is backwards from current in macros2:
\newcommand{\PDb}[2]{ \frac{\partial{#1}}{\partial {#2}} }

% inertial_tensor.tex
\newcommand{\matrixoftx}[3]{
{
\begin{bmatrix}
{#1}
\end{bmatrix}
}_{#2}^{#3}
}

\newcommand{\coordvec}[2]{
{
\begin{bmatrix}
{#1}
\end{bmatrix}
}_{#2}
}

% bohr.tex
\newcommand{\K}[0]{\inv{4 \pi \epsilon_0}}

% euler_lagrange.tex
\newcommand{\qbar}[0]{\bar{q}}
\newcommand{\qdotbar}[0]{\dot{\bar{q}}}
\newcommand{\DD}[2]{\frac{d{#2}}{d{#1}}}
\newcommand{\Xdot}[0]{\dot{X}}

% rayleigh_jeans.tex
\newcommand{\EE}[0]{\boldsymbol{\mathcal{E}}}
\newcommand{\HH}[0]{\boldsymbol{\mathcal{H}}}

% 4d_fourier.tex

%\newcommand{\PDSq}[2]{\frac{\partial^2 {#2}}{\partial {#1}^2}}
\DeclareMathOperator{\sinc}{sinc}
\DeclareMathOperator{\PV}{PV}
\newcommand{\FF}[0]{\mathcal{F}}
\newcommand{\IIinf}[0]{ \int_{-\infty}^\infty }

% poisson.tex
%\newcommand{\PDSq}[2]{\frac{\partial^2 {#2}}{\partial {#1}^2}}
%\DeclareMathOperator{\sinc}{sinc}
%\DeclareMathOperator{\PV}{PV}
%\newcommand{\FF}[0]{\mathcal{F}}
%\newcommand{\IIinf}[0]{ \int_{-\infty}^\infty }

% fourier_maxwell.tex
%\newcommand{\PDSq}[2]{\frac{\partial^2 {#2}}{\partial {#1}^2}}
%\DeclareMathOperator{\sinc}{sinc}
%\DeclareMathOperator{\sgn}{sgn}
%\DeclareMathOperator{\PV}{PV}
%\newcommand{\FF}[0]{\mathcal{F}}
%\newcommand{\IIinf}[0]{ \int_{-\infty}^\infty }

% firstorder_fourier_maxwell.tex
%\newcommand{\PDSq}[2]{\frac{\partial^2 {#2}}{\partial {#1}^2}}
%\DeclareMathOperator{\sinc}{sinc}
%\DeclareMathOperator{\PV}{PV}
%\newcommand{\FF}[0]{\mathcal{F}}
%\newcommand{\IIinf}[0]{ \int_{-\infty}^\infty }

% wave_fourier.tex
%\newcommand{\PDSq}[2]{\frac{\partial^2 {#2}}{\partial {#1}^2}}
%\DeclareMathOperator{\sinc}{sinc}
%\DeclareMathOperator{\PV}{PV}
%\newcommand{\FF}[0]{\mathcal{F}}
%\newcommand{\IIinf}[0]{ \int_{-\infty}^\infty }

% heat_fourier.tex
%\newcommand{\PDSq}[2]{\frac{\partial^2 {#2}}{\partial {#1}^2}}
%\DeclareMathOperator{\sinc}{sinc}
%\newcommand{\FF}[0]{\mathcal{F}}
%\newcommand{\IIinf}[0]{ \int_{-\infty}^\infty }

% proj_generalized_dot_prod.tex
%\newcommand{\T}[0]{\text{T}}

% fourier_tx.tex
%\newcommand{\FF}[0]{\mathcal{F}}
\newcommand{\FM}[0]{\inv{\sqrt{2\pi\hbar}}}
\newcommand{\Iinf}[1]{ \int_{-\infty}^\infty {#1}}
%\DeclareMathOperator{\PV}{PV}

% fourier_notation.tex
%\newcommand{\FF}[0]{\mathcal{F}}
%\newcommand{\IIinf}[0]{ \int_{-\infty}^\infty }
%\DeclareMathOperator{\PV}{PV}
%\DeclareMathOperator{\sinc}{sinc}

% planewave.tex
%\newcommand{\EE}[0]{\boldsymbol{\mathcal{E}}}
%\newcommand{\HH}[0]{\boldsymbol{\mathcal{H}}}
%\newcommand{\IIinf}[0]{ \int_{-\infty}^\infty }

% dirac_lagrangian.tex
\newcommand{\Dslash}[0]{ \not\!D }

% pauli_matrix.tex
\newcommand{\Clifford}[2]{\mathcal{C}_{\{{#1},{#2}\}}}
\DeclareMathOperator{\tr}{Tr}
%\DeclareMathOperator{\Scalar}{Scalar}
\DeclareMathOperator{\Real}{Re}
\DeclareMathOperator{\Imag}{Im}
\newcommand{\trace}[1]{\tr{#1}}
\newcommand{\scalarProduct}[2]{{#1} \bullet {#2}}
\newcommand{\traceB}[1]{\tr\left({#1}\right)}
%\newcommand{\symmetric}[2]{{\left\{{#1},{#2}\right\}}}
%\newcommand{\antisymmetric}[2]{\left[{#1},{#2}\right]}
%\newcommand{\Bcap}[0]{\hat{\BB}}

\newcommand{\xhat}[0]{\hat{x}}

\newcommand{\PauliI}[0]{
\begin{bmatrix}
1 & 0 \\
0 & 1 \\
\end{bmatrix}
}

\newcommand{\PauliX}[0]{
\begin{bmatrix}
0 & 1 \\
1 & 0 \\
\end{bmatrix}
}

\newcommand{\PauliY}[0]{
\begin{bmatrix}
0 & -i \\
i & 0 \\
\end{bmatrix}
}

\newcommand{\PauliYNoI}[0]{
\begin{bmatrix}
0 & -1 \\
1 & 0 \\
\end{bmatrix}
}

\newcommand{\PauliZ}[0]{
\begin{bmatrix}
1 & 0 \\
0 & -1 \\
\end{bmatrix}
}

% gamma.tex
%\newcommand{\scalarProduct}[2]{{#1} \bullet {#2}}
%\newcommand{\symmetric}[2]{{\left\{{#1},{#2}\right\}}}
%\newcommand{\antisymmetric}[2]{\left[{#1},{#2}\right]}

%\newcommand{\PauliX}[0]{
%\begin{bmatrix}
%0 & 1 \\
%1 & 0 \\
%\end{bmatrix}
%}

%\newcommand{\PauliY}[0]{
%\begin{bmatrix}
%0 & -i \\
%i & 0 \\
%\end{bmatrix}
%}

%\newcommand{\PauliYNoI}[0]{
%\begin{bmatrix}
%0 & -1 \\
%1 & 0 \\
%\end{bmatrix}
%}

%\newcommand{\PauliZ}[0]{
%\begin{bmatrix}
%1 & 0 \\
%0 & -1 \\
%\end{bmatrix}
%}

% em_bivector_metric_dependencies.tex

%\newcommand{\LL}[0]{\mathcal{L}}
%\newcommand{\gpgrade}[2] {{\left\langle{{#1}}\right\rangle}_{#2}}
%\newcommand{\gpgradezero}[1] {\gpgrade{#1}{0}}
%\newcommand{\gpgradetwo}[1] {\gpgrade{#1}{2}}
%\newcommand{\gpgradeone}[1] {\gpgrade{#1}{1}}
%\newcommand{\gpgradefour}[1] {\gpgrade{#1}{4}}
%\newcommand{\grad}[0]{\nabla}
%\newcommand{\spacegrad}[0]{\boldsymbol{\nabla}}
% == \partial_{#1} {#2}
%\newcommand{\PD}[2]{\frac{\partial {#2}}{\partial {#1}}}
%\newcommand{\PDD}[3]{\frac{\partial^2 {#3}}{\partial {#1}\partial {#2}}}
\newcommand{\PDsQ}[2]{\frac{\partial^2 {#2}}{\partial^2 {#1}}}

% gem.tex
\newcommand{\barh}[0]{\bar{h}}

% mass_vary_lagrangian.tex
%\newcommand{\LL}[0]{\mathcal{L}}
%\newcommand{\grad}[0]{\nabla}
%\newcommand{\PD}[2]{\frac{\partial {#2}}{\partial {#1}}}
%\newcommand{\xdot}[0]{\dot{x}}
%\newcommand{\vdot}[0]{\dot{v}}
%\newcommand{\mdot}[0]{\dot{m}}
%\newcommand{\xddot}[0]{\ddot{x}}
%\newcommand{\spacegrad}[0]{\boldsymbol{\nabla}}

% fourvec_dotinvariance.tex
%\newcommand{\Balpha}[0]{\boldsymbol{\alpha}}
\newcommand{\alphacap}[0]{\hat{\boldsymbol{\alpha}}}
%\newcommand{\Bcap}[0]{\hat{\BB}}
%\newcommand{\gpgrade}[2] {{\left\langle{{#1}}\right\rangle}_{#2}}
%\newcommand{\gpgradezero}[1] {\gpgrade{#1}{0}}

% lorentz.tex
%\newcommand{\laplacian}[0]{\nabla^2}

% field_lagrangian.tex
%\newcommand{\LL}[0]{\mathcal{L}}
%\newcommand{\PD}[2]{\frac{\partial {#2}}{\partial {#1}}}
\newcommand{\barA}[0]{\bar{A}}
%\newcommand{\grad}[0]{\nabla}
%\newcommand{\conj}[0]{{*}}

%\newcommand{\spacegrad}[0]{\boldsymbol{\nabla}}

%\newcommand{\gpgrade}[2] {{\left\langle{{#1}}\right\rangle}_{#2}}
%\newcommand{\gpgradezero}[1] {\gpgrade{#1}{0}}
%\newcommand{\gpgradetwo}[1] {\gpgrade{#1}{2}}
%\newcommand{\gpgradefour}[1] {\gpgrade{#1}{4}}

% lagrangian_field_density.tex
%\newcommand{\LL}[0]{\mathcal{L}}
%\newcommand{\gpgrade}[2] {{\left\langle{{#1}}\right\rangle}_{#2}}
%\newcommand{\gpgradezero}[1] {\gpgrade{#1}{0}}
%\newcommand{\gpgradetwo}[1] {\gpgrade{#1}{2}}
%\newcommand{\gpgradefour}[1] {\gpgrade{#1}{4}}
%\newcommand{\grad}[0]{\nabla}
%\newcommand{\spacegrad}[0]{\boldsymbol{\nabla}}
%\newcommand{\PD}[2]{\frac{\partial {#2}}{\partial {#1}}}
\newcommand{\PDd}[2]{\frac{\partial^2 {#2}}{{\partial{#1}}^2}}
%\newcommand{\PDD}[3]{\frac{\partial^2 {#3}}{\partial {#1}\partial {#2}}}

%\newcommand{\barA}[0]{\bar{A}}

% lorentz_force.tex
%\newcommand{\grad}[0]{\nabla}
%\newcommand{\spacegrad}[0]{\boldsymbol{\nabla}}
%\newcommand{\LL}[0]{\mathcal{L}}
%\newcommand{\xdot}[0]{\dot{x}}
%\newcommand{\xddot}[0]{\ddot{x}}
%\newcommand{\pdot}[0]{\dot{p}}
%\newcommand{\pddot}[0]{\ddot{p}}
%\newcommand{\fdot}[0]{\dot{f}}
%\newcommand{\fddot}[0]{\ddot{f}}

%\newcommand{\gpgrade}[2] {{\left\langle{{#1}}\right\rangle}_{#2}}
%\newcommand{\gpgradeone}[1] {\gpgrade{#1}{1}}
%\newcommand{\gpgradezero}[1] {\gpgrade{#1}{}}
%\newcommand{\grad}[0] {\nabla}
%\newcommand{\spacegrad}[0]{\boldsymbol{\nabla}}

%\newcommand{\pdot}[0]{\dot{p}}
%\newcommand{\pddot}[0]{\ddot{p}}

%\newcommand{\xdot}[0]{\dot{x}}
%\newcommand{\xddot}[0]{\ddot{x}}
%\newcommand{\PD}[2]{\frac{\partial {#2}}{\partial {#1}}}

% stokes_maxwell_application.tex
%\newcommand{\grad}[0]{\nabla}
%\newcommand{\PD}[2]{\frac{\partial {#2}}{\partial {#1}}}
%\newcommand{\spacegrad}[0]{\boldsymbol{\nabla}}
%\newcommand{\gpgrade}[2] {{\left\langle{{#1}}\right\rangle}_{#2}}
%\newcommand{\gpgradezero}[1] {\gpgrade{#1}{0}}
%\newcommand{\gpgradeone}[1] {\gpgrade{#1}{1}}
%\newcommand{\gpgradetwo}[1] {\gpgrade{#1}{2}}
%\newcommand{\gpgradethree}[1] {\gpgrade{#1}{3}}

% lorentz_rotation.tex
%\DeclareMathOperator{\Transpose}{T}
%\newcommand{\T}[0]{\text{T}}

% electron_rotor.tex
\newcommand{\reverse}[1]{\tilde{{#1}}}
%\newcommand{\ILambda}[0]{{(\Lambda^{-1})}}
\newcommand{\ILambda}[0]{\Pi}

% em_potential.tex
%\newcommand{\spacegrad}[0]{\boldsymbol{\nabla}}
%\newcommand{\grad}[0]{\nabla}
\newcommand{\CA}[0]{\mathcal{A}}
 
% maxwell_to_tensor.tex
%\newcommand{\LL}[0]{\mathcal{L}}
%\newcommand{\gpgrade}[2] {{\left\langle{{#1}}\right\rangle}_{#2}}
%\newcommand{\gpgradezero}[1] {\gpgrade{#1}{0}}
%\newcommand{\gpgradetwo}[1] {\gpgrade{#1}{2}}
%\newcommand{\gpgradeone}[1] {\gpgrade{#1}{1}}
%\newcommand{\gpgradefour}[1] {\gpgrade{#1}{4}}
%\newcommand{\grad}[0]{\nabla}
%\newcommand{\spacegrad}[0]{\boldsymbol{\nabla}}
% == \partial_{#1} {#2}
%\newcommand{\PD}[2]{\frac{\partial {#2}}{\partial {#1}}}
%\newcommand{\PDD}[3]{\frac{\partial^2 {#3}}{\partial {#1}\partial {#2}}}
%\newcommand{\PDsQ}[2]{\frac{\partial^2 {#2}}{\partial^2 {#1}}}

%\newcommand{\EE}[0]{\boldsymbol{\mathcal{E}}}
%\newcommand{\HH}[0]{\boldsymbol{\mathcal{H}}}
\newcommand{\Vcap}[0]{\hat{\BV}}



%\usepackage{listings}
%\usepackage{txfonts} % for ointctr... (also appears to make "prettier" \int and \sum's)
% makes \grad look funny though (almost like spacegrad, but narrower)
\usepackage[bookmarks=true]{hyperref}

\usepackage{color,cite,graphicx}
   % use colour in the document, put your citations as [1-4]
   % rather than [1,2,3,4] (it looks nicer, and the extended LaTeX2e
   % graphics package. 
\usepackage{latexsym,amssymb,epsf} % don't remember if these are
   % needed, but their inclusion can't do any damage


\title{Comparison of two covariant Lorentz force Lagrangians}
\author{Peeter Joot \quad peeter.joot@gmail.com }
\date{ June 17, 2009.  $RCSfile: lForceLag2.tex,v $ Last $Revision: 1.1 $ $Date: 2009/06/17 21:27:27 $ }

\begin{document}

\maketitle{}
\tableofcontents
\section{Motivation}

In \cite{poisson1999ild}, 
the covariant Lorentz force Lagrangian is given by

\begin{align}
\LL = \int A_\alpha j^\alpha d^4 x - m \int d\tau
\end{align}

which is not quadratic in proper time as seen previously in
\cite{PJSrLorentzForce}
%\ref{eqn:lorForce:summarize}
, and
\cite{lorentzForcePQA} 
%\ref{eqn:lorForcePqA:interactionLagPsq}

\begin{align}
\LL
%&= \inv{2} m v^2 + q (\gamma_0)^2 A \cdot (v/c) \\
&= \inv{2} m v^2 + q A \cdot (v/c) \\
&= \inv{2 m}\left( m v + \frac{q}{c} A \right)^2 - \frac{q^2}{ 2 m c^2} A^2
\end{align}

These two forms are identical, but the second is expressed explicitly
in terms of the conjugate momentum, and calls out the explicit kinetic
vs potential terms in the Lagrangian nicely.

\section{Quadradic Lagrangian}

For review purposes lets once again compute the equations of motion 
with an evaluation of the Euler-Lagrange equations.  With hindsight
this can also be done more compactly than in previous notes.

\bibliographystyle{plainnat}
\bibliography{myrefs}

\end{document}

\part{Quantum Mechanics.}
%
% Copyright � 2012 Peeter Joot.  All Rights Reserved.
% Licenced as described in the file LICENSE under the root directory of this GIT repository.
%

%
%
\chapter{Bohr Model}
\index{Bohr model}
\label{chap:bohr}
%\date{Dec 11, 2008.  bohr.tex}

\section{Motivation}

The Bohr model is taught as early as high school chemistry when the various orbitals are
discussed (or maybe it was high school physics).  I recall
that the first time I saw this I did not see where all the ideas came from.
With a bit more math under my belt now, reexamine these ideas as a lead up to
the proper wave mechanics.

\section{Calculations}

\subsection{Equations of motion}

A prerequisite to discussing electron orbits is first setting up the equations of motion
for the two charged particles (ie: the proton and electron).

With the proton position at \(\Br_p\), and the electron at \(\Br_e\), we have two equations, one
for the force on the proton from the electron and the other for the force on the proton from
the electron.  These are respectively

\begin{equation}\label{eqn:bohr:chargeEquations}
\begin{aligned}
  \K e^2 \frac { \Br_e - \Br_p } { \Abs{\Br_e - \Br_p}^3 } &= m_p \frac{d^2 \Br_p }{dt^2} \\
- \K e^2 \frac { \Br_e - \Br_p } { \Abs{\Br_e - \Br_p}^3 } &= m_e \frac{d^2 \Br_e }{dt^2}
\end{aligned}
\end{equation}

In lieu of a picture, setting \(\Br_p = 0\) works to check signs, leaving an inwards force on the electron as desired.

% FIXME: Add one.
%\begin{figure}[htp]
%\centering
%\includegraphics[totalheight=0.4\textheight]{picturepath}
%\caption{My Caption}\label{fig:pictlabel}
%\end{figure}
%
%... see \cref{fig:picturepath} ...

As usual for a two body problem, use of the difference vector and center of mass vector is desirable.  That is

\begin{equation}\label{eqn:bohr:20}
\begin{aligned}
\Bx &= \Br_e - \Br_p \\
M &= m_e + m_p \\
\BR &= \inv{M}(m_e \Br_e + m_p \Br_p)
\end{aligned}
\end{equation}

Solving for \(\Br_p\) and \(\Br_e\) in terms of \(\BR\) and \(\Bx\) we have

\begin{equation}\label{eqn:bohr:40}
\begin{aligned}
\Br_e &= \frac{m_p}{M} \Bx + \BR \\
\Br_p &= \frac{-m_e}{M} \Bx + \BR \\
\end{aligned}
\end{equation}

% check:
%r_e - r_p = M/M x
%m_e \Br_e + m_p \Br_p &= \frac{-m_e m_p}{M} \Bx + m_e \BR + \frac{m_p m_e}{M} \Bx + m_p \BR \\

Substitution back into \eqnref{eqn:bohr:chargeEquations} we have

\begin{equation}\label{eqn:bohr:60}
\begin{aligned}
  \K e^2 \frac {\Bx} { \Abs{\Bx}^3 } &= m_p \frac{d^2}{dt^2}\left( \frac{-m_e}{M} \Bx + \BR \right) \\
 -\K e^2 \frac {\Bx} { \Abs{\Bx}^3 } &= m_e \frac{d^2}{dt^2}\left( \frac{m_p}{M} \Bx + \BR \right),
\end{aligned}
\end{equation}

and sums and (scaled) differences of that give us our reduced mass equation and constant center-of-mass velocity equation
\begin{equation}\label{eqn:bohr:80}
\begin{aligned}
\frac{d^2 \Bx}{dt^2} &= -\K e^2 \frac {\Bx} { \Abs{\Bx}^3 } \left( \inv{m_e} + \inv{m_p} \right) \\
\frac{d^2 \BR}{dt^2} &= 0
\end{aligned}
\end{equation}

writing \(1/\mu = 1/m_e + 1/m_p\), and \(k = e^2/4 \pi \epsilon_0\), our difference vector equation is thus

\begin{equation}\label{eqn:bohr:reduceEOM}
\begin{aligned}
\mu \frac{d^2 \Bx}{dt^2} &= -k \frac {\Bx} { \Abs{\Bx}^3 }
\end{aligned}
\end{equation}

\subsection{Circular solution}

The Bohr model postulates that electron orbits are circular.  It is easy enough to verify that a circular orbit in the center of mass frame is a solution to equation
\eqnref{eqn:bohr:reduceEOM}.   Write the path in terms of the unit bivector for the plane of rotation \(i\) and an initial vector position \(\Bx_0\)

\begin{equation}\label{eqn:bohr:circular}
\begin{aligned}
\Bx = \Bx_0 e^{i \omega t}
\end{aligned}
\end{equation}

For constant \(i\) and \(\omega\), we have

\begin{equation}\label{eqn:bohr:100}
\begin{aligned}
\mu \Bx_0 (i\omega)^2 e^{i\omega t} = - k \frac{\Bx_0}{\Abs{\Bx_0}^3} e^{i\omega t}
\end{aligned}
\end{equation}

This provides the
angular velocity in terms of the reduced mass of the system and the charge constants

\begin{equation}\label{eqn:bohr:omegaSquared}
\begin{aligned}
\omega^2 = \frac{k}{\mu \Abs{\Bx_0}^3} = \frac{e^2}{4 \pi \epsilon_0 \mu \Abs{\Bx_0}^3}.
\end{aligned}
\end{equation}

Although not relevant to the quantum theme, it is hard not to call out the observation that this is
a Kepler's law like relation for the period of the circular orbit given the radial distance from the center of mass

\begin{equation}\label{eqn:bohr:120}
\begin{aligned}
T^2 = \frac{16 \pi^3 \epsilon_0 \mu}{e^2} \Abs{\Bx_0}^3
\end{aligned}
\end{equation}

Kepler's law also holds for elliptical orbits, but this takes more work to show.

\subsection{Angular momentum conservation}
\index{angular momentum conservation}

Now, the next step in the Bohr argument was that the angular momentum, a conserved quantity is also quantized.  To give real
meaning to the conservation statement we need the equivalent Lagrangian formulation of \eqnref{eqn:bohr:reduceEOM}.  Anti-differentiation
gives

\begin{equation}\label{eqn:bohr:140}
\begin{aligned}
\grad_\Bv \left( \inv{2} \mu \Bv^2 \right)
&= k \xcap \partial_x \inv{x} \\
&= - \grad_\Bx
\mathLabelBox
[
   labelstyle={below of=m\themathLableNode, below of=m\themathLableNode}
]
{\left(- k\inv{\Abs{\Bx}}\right)}{\(=\phi\)}
\end{aligned}
\end{equation}

So, our Lagrangian is
\begin{equation}\label{eqn:bohr:160}
\begin{aligned}
\LL = K - \phi = \inv{2} \mu \Bv^2 + k \inv{\Abs{\Bx}}
\end{aligned}
\end{equation}

The essence of the conservation argument, an application of
Noether's theorem,
is that a rotational transformation of the Lagrangian leaves this energy relationship unchanged.  Repeating
the angular momentum example from \citep{classicalmechanics:PJEulerLagrange} (which was done for the more general case of any radial potential), we
write \(\hat{B}\) for the unit bivector associated with a rotational plane.  The position vector is transformed by rotation in this plane as follows

\begin{equation}\label{eqn:bohr:180}
\begin{aligned}
\Bx &\rightarrow \Bx' \\
\Bx' &= R \Bx R^\dagger \\
R &= \exp{\hat{B}\theta/2}
\end{aligned}
\end{equation}

The magnitude of the position vector is rotation invariant

\begin{equation}\label{eqn:bohr:200}
\begin{aligned}
(\Bx')^2 &= R \Bx R^\dagger R \Bx R^\dagger = \Bx^2,
\end{aligned}
\end{equation}

as is our the square of the transformed velocity.  The transformed velocity is

\begin{equation}\label{eqn:bohr:220}
\begin{aligned}
\frac{d\Bx'}{dt} &= \dot{R} \Bx R + R \dot{\Bx} R^\dagger + R \Bx \dot{R}^\dagger
\end{aligned}
\end{equation}

but with \(\dot{\theta} = 0\), \(\dot{R} = 0\) its square is just

\begin{equation}\label{eqn:bohr:240}
\begin{aligned}
(\Bv')^2 &= R {\Bv} R^\dagger R \dot{\Bv} R^\dagger = \Bv^2.
\end{aligned}
\end{equation}

We therefore have a Lagrangian that is invariant under this rotational transformation

\begin{equation}\label{eqn:bohr:260}
\begin{aligned}
\LL \rightarrow \LL' = \LL,
\end{aligned}
\end{equation}

and by Noether's theorem (essentially application of the chain rule), we have

\begin{equation}\label{eqn:bohr:280}
\begin{aligned}
\frac{d\LL'}{d\theta}
&= \frac{d}{dt} \left(\frac{d\Bx'}{d\theta} \cdot \grad_{\Bv'} \LL \right) \\
&= \frac{d}{dt} \left( (\hat{B} \cdot \Bx') \cdot \mu \Bv' \right).
\end{aligned}
\end{equation}

But \(d\LL'/d\theta = 0\), so we have for any \(\hat{B}\)

\begin{equation}\label{eqn:bohr:300}
\begin{aligned}
(\hat{B} \cdot \Bx') \cdot (\mu \Bv') &= \hat{B} \cdot (\Bx' \wedge (\mu \Bv')) = \text{constant}
\end{aligned}
\end{equation}

Dropping primes this is

\begin{equation}\label{eqn:bohr:320}
\begin{aligned}
L = \Bx \wedge (\mu \Bv) = \text{constant},
\end{aligned}
\end{equation}

a constant bivector for the conserved center of mass (reduced-mass) angular momentum associated with the Lagrangian of this system.

\subsection{Quantized angular momentum for circular solution}

In terms of the circular solution of \eqnref{eqn:bohr:circular} the angular momentum bivector is

\begin{equation}\label{eqn:bohr:340}
\begin{aligned}
L = \Bx \wedge (\mu \Bv)
&= \gpgradetwo{ \Bx_0 e^{i \omega t} \mu \Bx_0 i \omega e^{i \omega t} } \\
&= \gpgradetwo{ e^{-i \omega t} \Bx_0 \mu \Bx_0 \omega e^{i \omega t} i } \\
&= (\Bx_0)^2 \mu \omega i \\
%&= i \frac{e \mu}{2} \sqrt{\frac{\Abs{\Bx_0}}{\pi \epsilon_0 \mu}} \\
&= i e \sqrt{\frac{\mu \Abs{\Bx_0}}{4 \pi \epsilon_0}}
\end{aligned}
\end{equation}

%\begin{align}\label{eqn:bohr:omegaSquared}
%\omega = \frac{e}{2 \sqrt{\pi \epsilon_0}} \Abs{\Bx_0}^{-3/2}

Now if this angular momentum is quantized with quantum magnitude \(l\) we have we have for the bivector angular momentum the values

\begin{equation}\label{eqn:bohr:360}
\begin{aligned}
L = i n l = i e \sqrt{\frac{\mu \Abs{\Bx_0}}{4 \pi \epsilon_0}}
\end{aligned}
\end{equation}

Which with \(l = \Hbar\) (where experiment in the form of the spectral hydrogen line values is required to fix this constant and relate it to Plank's black body constant)
is the momentum equation in terms of
the Bohr radius \(\Bx_0\) at each energy level.  Writing that radius \(r_n = \Abs{\Bx_0}\) explicitly as a function of n, we have

\begin{equation}\label{eqn:bohr:380}
\begin{aligned}
r_n = \frac{4 \pi \epsilon_0}{\mu} \left(\frac{n \Hbar}{e}\right)^2
\end{aligned}
\end{equation}

\subsubsection{Velocity}

One of the assumptions of this treatment is a \(\Abs{\Bv_e} << c\) requirement so that Coulombs law is valid (ie: slow enough that all the other Maxwell's equations can be neglected).
Let us evaluate the velocity numerically at the some of the quantization levels and see how this compares to the speed of light.

First we need an expression for the velocity itself.  This is

\begin{equation}\label{eqn:bohr:400}
\begin{aligned}
\Bv^2
&= ( \Bx_0 i \omega e^{i \omega t} )^2 \\
&= \frac{e^2}{4 \pi \epsilon_0 \mu r_n} \\
&= \frac{e^4}{(4 \pi \epsilon_0)^2 (n \Hbar)^2}.
\end{aligned}
\end{equation}

For
\begin{equation}\label{eqn:bohr:420}
\begin{aligned}
v_n
&= \frac{e^2}{4 \pi \epsilon_0 n \Hbar} \\
&= 2.1 \times 10^6 m/s
\end{aligned}
\end{equation}

This is the \(1/137\) of the speed of light value that one sees googling electron speed in hydrogen, and only decreases with quantum number so the non-relativistic speed approximation holds
(\(\gamma = 1.00002663\)).  This speed is still pretty zippy, even if it is not relativistic, so it is not unreasonable to attempt to repeat this treatment trying to incorporate the remainder
of Maxwell's equations.

Interestingly the velocity is not a function of the reduced mass at all, but just the charge and quantum numbers.  One also gets a good hint at why the Bohr theory breaks down
for larger atoms.  An electron in circular orbit around an ion of Gold would have a velocity of \(79/137\) the speed of light!

% google calculator:
%1/sqrt(1- ((elementary charge)^2 / 4 / pi / epsilon_0 /hbar/c)^2)

% - discuss connection to Sch. results?
% - try: proper maxwell's/Lorentz equations instead of just the Coulomb force.

%
% Copyright � 2012 Peeter Joot.  All Rights Reserved.
% Licenced as described in the file LICENSE under the root directory of this GIT repository.
%

% 
% 
\chapter{Schr\"{o}dinger equation probability conservation}
\label{chap:schCurrent}
\date{Jan 11, 2009.  schCurrent.tex}
\section{Motivation}

In \citep{mcmahon2005qmd} is a one dimensional probability conservation
derivation from
Schr\"{o}dinger's equation.  Do this for the three dimensional case.

\section{}

Consider the time rate of change of the probability as expressed 
in terms of the wave function

\begin{align*}
\PD{t}{\rho} 
&= \PD{t}{\psi^\conj \psi} \\
&= \PD{t}{\psi^\conj} \psi + \psi^\conj \PD{t}{\psi} \\
\end{align*}

This can be calculated from Schr\"{o}dinger's equation and its complex
conjugate

\begin{align*}
\partial_t \psi &= \left(-\frac{\hbar}{2mi}\spacegrad^2 + \inv{i\hbar} V \right) \psi \\
\partial_t \psi^\conj &= \left(\frac{\hbar}{2mi}\spacegrad^2 - \inv{i\hbar} V \right) \psi^\conj \\
\end{align*}

Multiplying by the conjugate wave functions and adding we have
\begin{align*}
\PD{t}{\rho}
&=
\psi^\conj \left(-\frac{\hbar}{2mi}\spacegrad^2 + \inv{i\hbar} V \right) \psi +
\psi \left(\frac{\hbar}{2mi}\spacegrad^2 - \inv{i\hbar} V \right) \psi^\conj \\
&=
\frac{\hbar}{2mi} \left(
-\psi^\conj \spacegrad^2 \psi + \psi \spacegrad^2 \psi^\conj \right) \\
\end{align*}

So we have the following conservation law
\begin{align}\label{eqn:sch_current:intermediate}
\PD{t}{\rho} + \frac{\hbar}{2mi} \left( \psi^\conj \spacegrad^2 \psi - \psi \spacegrad^2 \psi^\conj \right) = 0
\end{align}

The text indicates that the second order terms here can be written as a divergence.  Somewhat loosely, by treating $\psi$ as a scalar field one can show that this is the case

\begin{align*}
\spacegrad \cdot \left( \psi^\conj \spacegrad \psi - \psi \spacegrad \psi^\conj \right) 
&=
\gpgradezero{
\spacegrad \left( \psi^\conj \spacegrad \psi - \psi \spacegrad \psi^\conj \right) 
} \\
&=
\gpgradezero{
(\spacegrad \psi^\conj) (\spacegrad \psi) - (\spacegrad \psi) (\spacegrad \psi^\conj)
+\psi^\conj \spacegrad^2 \psi - \psi \spacegrad^2 \psi^\conj
} \\
&=
\gpgradezero{
2 (\spacegrad \psi^\conj) \wedge (\spacegrad \psi) 
+\psi^\conj \spacegrad^2 \psi - \psi \spacegrad^2 \psi^\conj
} \\
&=
\psi^\conj \spacegrad^2 \psi - \psi \spacegrad^2 \psi^\conj
 \\
\end{align*}

Assuming that this procedure is justified.
equation \ref{eqn:sch_current:intermediate} therefore can be written
in terms of a probability current very reminiscent of the current density vector of electrodynamics

\begin{align}\label{eqn:sch_current:pcons}
\BJ &= \frac{\hbar}{2mi} \left( \psi^\conj \spacegrad \psi - \psi \spacegrad \psi^\conj \right) \\
0 &= \PD{t}{\rho} + \spacegrad \cdot \BJ 
\end{align}

Regarding justification, this should be revisited.
It appears to give the right answer, despite the fact that $\psi$ is a complex (mixed grade) object, which
likely has some additional significance.

\section{}

Now, having calculated the probability conservation equation \ref{eqn:sch_current:pcons}, it is interesting to
note the similarity to the relativistic spacetime divergence from Maxwell's equation.

We can write
\begin{align*}
0 = \PD{t}{\rho} + \spacegrad \cdot \BJ &= \grad \cdot \left( c\rho \gamma_0 + \BJ \gamma_0 \right)
\end{align*}

and form something that has the appearance of a relativistic four vector, re-writing the conservation equation as

\begin{align*}
J &= c\rho \gamma_0 + \BJ \gamma_0 \\
0 &= \grad \cdot J
\end{align*}

Expanding this four component vector shows an interesting form:

\begin{align*}
J &= c \rho \gamma_0 + 
\frac{\hbar}{2mi} \left( \psi^\conj \spacegrad \psi - \psi \spacegrad \psi^\conj \right) \gamma_0 \\
\end{align*}

Now, if one assumes the wave function can be represented as a even grade object with the following complex
structure
\begin{align*}
\psi &= \alpha + \gamma^m \wedge \gamma^n \beta_{mn}
\end{align*}

then $\gamma_0$ will commute with $\psi$.  Noting that $\spacegrad \gamma_0 = \sum_k \gamma_k \partial_k = -\gamma^k \partial_k$, we have

\begin{align*}
m J &= m c \psi^\conj \psi \gamma_0 + 
\frac{i\hbar}{2} \left( \psi^\conj \gamma^k \partial_k \psi - \psi \gamma^k \partial_k \psi^\conj \right) 
\end{align*}

Now, this is an interesting form.  In particular compare this to the Dirac Lagrangian, as given in 
the \href{http://en.wikipedia.org/wiki/Dirac_equation#Adjoint_equation_and_Dirac_current}{wikipedia Dirac equation} article.

\begin{align*}
L = mc \bar{\psi}\psi - \frac{i\hbar}{2}(\bar{\psi}\gamma^\mu (\partial_\mu\psi) - (\partial_\mu\bar{\psi})\gamma^\mu \psi)
\end{align*}

Although the Schr\"{o}dinger equation is a non-relativistic equation, it appears that the probability current, 
when we add the $\gamma^0 \partial_0$ term required to put this into a covariant form, is in fact the Lagrangian density
for the Dirac equation (when scaled by mass).

I don't know enough yet about QM to see what exactly the implications of this are, but I suspect that there is something
of some interesting significance to this particular observation.

\section{On the grades of the QM complex numbers}

To get to equation \ref{eqn:sch_current:intermediate}, no assumptions about the representation of the field variable $\psi$ were
required.  However, to make the identification

\begin{align*}
\psi^\conj \spacegrad^2 \psi - \psi \spacegrad^2 \psi^\conj 
&= \spacegrad \cdot \left( \psi^\conj \spacegrad^2 \psi - \psi \spacegrad^2 \psi^\conj \right)
\end{align*}

we need some knowledge or assumptions about the representation.  The assumption made initially was that we could treat
$\psi$ as a scalar, but then we later see there is value trying to switch to the Dirac representation (which appears
to be the logical way to relativistically extend the probability current).

For example, with a geometric algebra multivector representation we have many ways to construct complex quantities.  Assuming a
Euclidean basis we can construct a complex number we can factor out one of the basis vectors

\begin{align*}
\sigma_1 x_1 + \sigma_2 x_2 = \sigma_1 ( x_1 + \sigma_1 \sigma_2 x_2 )
\end{align*}

However, this isn't going to commute with vectors (ie: such as the gradient), unless that vector is perpendicular to the
plane spanned by this vector.  As an example

\begin{align*}
i = \sigma_1 \sigma_2
\end{align*}

\begin{align*}
i \sigma_1 &= -\sigma_1 i \\
i \sigma_2 &= -\sigma_2 i \\
i \sigma_3 &=  \sigma_3 i
\end{align*}

What would work is a complex representation using the \R{3} pseudoscalar (aka the Dirac pseudoscalar).

\begin{align*}
\psi = \alpha + \sigma_1 \sigma_2 \sigma_3 \beta = \alpha + \gamma_0 \gamma_1 \gamma_2 \gamma_3 \beta 
\end{align*}

%
% Copyright � 2012 Peeter Joot.  All Rights Reserved.
% Licenced as described in the file LICENSE under the root directory of this GIT repository.
%

%
%
\chapter{Dirac Lagrangian}
\index{Dirac Lagrangian}
\label{chap:diracLagrangian}
%\date{Dec 21, 2008.  diracLagrangian.tex}

\section{Dirac Lagrangian with Feynman slash notation}

Wikipedia's \href{http://en.wikipedia.org/wiki/Lagrangian#Dirac_Lagrangian}{Dirac Lagrangian} entry lists the Lagrangian as

\begin{equation}\label{eqn:diracLagrangian:24}
\begin{aligned}
\LL = \overbar{\psi} (i \Hbar c \Dslash - mc^2) \psi
\end{aligned}
\end{equation}

"where \(\psi\!\) is a Dirac spinor, \(\overbar{\psi} = \psi^\dagger \gamma^0\) is its Dirac adjoint, \(D\!\) is the gauge covariant derivative, and \(\Dslash\) is Feynman slash notation|Feynman notation for \(\gamma^\sigma D_\sigma\!\)."

Let us decode this.  First, what is \(D_\sigma\)?

From \href{http://en.wikipedia.org/wiki/Gauge_covariant_derivative}{Gauge theory}

\begin{equation}\label{eqn:diracLagrangian:44}
\begin{aligned}
D_\mu := \partial_\mu - i e A_\mu
\end{aligned}
\end{equation}

where \(A_\mu\) is the electromagnetic vector potential.

So, in four-vector notation we have

\begin{equation}\label{eqn:diracLagrangian:64}
\begin{aligned}
\Dslash
&= \gamma^\mu \partial_\mu - i e \gamma^\mu A_\mu \\
&= \grad - i e A \\
\end{aligned}
\end{equation}

So our Lagrangian written out in full is left as

\begin{equation}\label{eqn:diracLag:lag1}
\begin{aligned}
\LL = \psi^\dagger \gamma^0 ( i \Hbar c \grad + \Hbar c e A - mc^2) \psi
\end{aligned}
\end{equation}

How about this \(\gamma^0 i \grad\) term?  If we assume that \(i = \gamma_0 \gamma_1 \gamma_2 \gamma_3\) is the four space pseudoscalar, then this is

\begin{equation}\label{eqn:diracLagrangian:84}
\begin{aligned}
\gamma^0 i \grad
&= - i \gamma^0 (\gamma^0 \partial_0 + \gamma^i \partial_i) \\
&= - i (\partial_0 + \sigma_i \partial_i) \\
\end{aligned}
\end{equation}

So, operationally, we have the dual of a quaternion like gradient operator.  If \(\psi\) is an even grade object, as I had guess can be implied by
the term spinor, then there is some sense to requiring a gradient operation that has scalar and spacetime bivector components.

Let us write this

\begin{equation}\label{eqn:diracLagrangian:104}
\begin{aligned}
\gamma^0 \grad &= \partial_0 + \sigma_i \partial_i = {\grad}_{0,2}
\end{aligned}
\end{equation}

Now, how about the meaning of \(\overbar{\psi} = \psi^\dagger \gamma^0\)?  I initially assumed that \(\psi^\dagger\) was the reverse operation.
However, looking in the quantum treatment of \citep{doran2003gap} and their earlier relativity content, I see that they explicitly avoid dagger as a reverse in a relativistic context since it is used for ``something-else'' in a quantum context.  It appears that their mapping from matrix algebra to Clifford
algebra is

\begin{equation}\label{eqn:diracLagrangian:124}
\begin{aligned}
\psi^\dagger \equiv \gamma_0 \tilde{\psi} \gamma_0,
\end{aligned}
\end{equation}

where tilde is used for the reverse operation.

This then implies that

\begin{equation}\label{eqn:diracLagrangian:144}
\begin{aligned}
\overbar{\psi} = \psi^\dagger \gamma^0 = \gamma_0 \tilde{\psi}
\end{aligned}
\end{equation}

We now have an expression of the Lagrangian in full in terms of geometric objects

\begin{equation}\label{eqn:diracLag:lag2}
\begin{aligned}
\LL = \gamma_0 \tilde{\psi} ( i \Hbar c \grad + \Hbar c e A - mc^2) \psi.
\end{aligned}
\end{equation}

Assuming that this is now the correct geometric interpretation of the Lagrangian, why bother having that first \(\gamma_0\) factor.  It should not change the field equations (just as a constant factor should not make a difference).  It seems more natural to instead write the Lagrangian as just

\begin{equation}\label{eqn:diracLag:lag3}
\begin{aligned}
\LL = \tilde{\psi} \left( i \grad + e A - \frac{mc}{\Hbar} \right) \psi,
\end{aligned}
\end{equation}

where both the constant vector factor \(\gamma_0\), the redundant common factor of \(c\) have been removed, and we divide throughout by \(\Hbar\) to tidy up a bit.  Perhaps this tidy up should be omitted since it sacrifices the
energy dimensionality of the original.

\subsection{Dirac adjoint field}

%Now, it is somewhat interesting to note that treating \(\psi^\dagger\) as a reverse operation resulted, after calculation of the field equations, in the quantity \(\gamma_0 \psi^\dagger \gamma_0\) showing up as a significant quanity.  It was almost like the math was auto-correcting itself attempting to show that the this was the real quantity of interest.

The reverse sandwich operation of \(\gamma_0 \tilde{\psi} \gamma_0\) to produce the Dirac adjoint field from \(\psi\)
can be recognized
as very similar to the mechanism used to split the Faraday bivector for the electromagnetic field into electric and magnetic terms.  There addition and subtraction of the sandwich'ed fields with the original acted as a spacetime split operation, producing separate electric field spacetime (Pauli) bivectors and pure spatial bivectors (magnetic components) from the total field.  Here we have a
quaternion like field variable with scalar and bivector terms.  Is there a physical (observables) significance
only for a subset of the six possible bivectors that make up the spinor field?
%only in the
%spacetime bivector parts (ie: Pauli matrix components) of the six possible bivectors,
If so, then this adjoint operation can be used as a filter to select only the desired components.

%Observe that we have an almost recognizable term in the effective field variable \(\gamma_0 \psi \gamma_0\) of
%\eqnref{eqn:diracLag:antimatter}.  In particular recall that the electric field and magnetic field components of the faraday bivector can be obtained from a
%\(\gamma^0 F \gamma_0\) spacetime split.
Recall that the Faraday bivector is

\begin{equation}\label{eqn:diracLagrangian:164}
\begin{aligned}
F
&= \BE + i c \BB \\
&= E^j \sigma_j + i c B^j \sigma_j \\
&= E^j \gamma_j \gamma_0 + i c B^j \gamma_j \gamma_0 \\
\end{aligned}
\end{equation}

So we have

\begin{equation}\label{eqn:diracLagrangian:184}
\begin{aligned}
\gamma_0 F \gamma_0
&= E^j \gamma_0 \gamma_j + \gamma_0 i c B^j \gamma_j \\
&= -E^j \sigma_j + i c B^j \sigma_j \\
&= -\BE + i c \BB
\end{aligned}
\end{equation}

So we have

\begin{equation}\label{eqn:diracLagrangian:204}
\begin{aligned}
\inv{2} \left(F - \gamma_0 F \gamma_0 \right) &= \BE \\
\inv{2i} \left(F + \gamma_0 F \gamma_0 \right) &= c \BB
\end{aligned}
\end{equation}

How does this sandwich operation act on other grade objects?

\begin{itemize}
\item scalar

\begin{equation}\label{eqn:diracLagrangian:224}
\begin{aligned}
\gamma_0 \alpha \gamma_0 = \alpha
\end{aligned}
\end{equation}

\item vector

\begin{equation}\label{eqn:diracLagrangian:244}
\begin{aligned}
\gamma_0 \gamma_\mu \gamma_0
&= \left(2 \gamma_0 \cdot \gamma_\mu - \gamma_\mu \gamma_0\right) \gamma_0 \\
&= 2 (\gamma_0 \cdot \gamma_\mu) \gamma_0 - \gamma_\mu \\
&=
\left\{
\begin{array}{l l}
\gamma_0 & \quad \mbox{if \(\mu = 0\)} \\
-\gamma_i & \quad \mbox{if \(\mu = i \ne 0\)} \\
\end{array} \right.
\end{aligned}
\end{equation}

\item trivector

For the duals of the vectors we have the opposite split, where for the dual of \(\gamma_0\) we have a sign
toggle

\begin{equation}\label{eqn:diracLagrangian:264}
\begin{aligned}
\gamma_0 \gamma_i \gamma_j \gamma_k \gamma_0 = -\gamma_i \gamma_j \gamma_k
\end{aligned}
\end{equation}

whereas for the duals of \(\gamma_k\) we have invariant sign under sandwich
\begin{equation}\label{eqn:diracLagrangian:284}
\begin{aligned}
\gamma_0 \gamma_i \gamma_j \gamma_0 \gamma_0 = \gamma_i \gamma_j \gamma_0
\end{aligned}
\end{equation}

\item pseudoscalar

\begin{equation}\label{eqn:diracLagrangian:304}
\begin{aligned}
\gamma_0 i \gamma_0
&= \gamma_0 \gamma_0 \gamma_1 \gamma_2 \gamma_3 \gamma_0 \\
&= -i
\end{aligned}
\end{equation}
\end{itemize}

Ah ha!  Recalling the conjugation results from \chapcite{PJDiracGamma}, one sees that this sandwich operation is in fact just the equivalent of the conjugate operation on Dirac matrix algebra elements.  So we can write

\begin{equation}\label{eqn:diracLagrangian:324}
\begin{aligned}
\psi^\conj \equiv \gamma_0 \psi \gamma_0
\end{aligned}
\end{equation}

and can thus identify \(\gamma_0 \tilde{\psi} \gamma_0 = \psi^\dagger\) as the reverse of that conjugate quantity.  That is

\begin{equation}\label{eqn:diracLagrangian:344}
\begin{aligned}
\psi^\dagger = (\psi^\conj)^{\tilde{}}
\end{aligned}
\end{equation}

This does not really help identify the significance of this term but this identification may prove useful later.

\subsection{Field equations}

Now, how to recover the field equation from \eqnref{eqn:diracLag:lag3}
%or \eqnref{eqn:diracLag:lag2}
?  If one assumes that the Euler-Lagrange field equations

\begin{equation}\label{eqn:diracLagrangian:364}
\begin{aligned}
\PD{\eta}{\LL} - \partial_\mu \PD{(\partial_\mu \eta)}{\LL} = 0
\end{aligned}
\end{equation}

hold for these even grade field variables \(\psi\), then treating \(\psi\) and \(\overbar{\psi}\) as separate field variables one has for the reversed field variable

\begin{equation}\label{eqn:diracLagrangian:384}
\begin{aligned}
\PD{\tilde{\psi}}{\LL} - \partial_\mu \PD{(\partial_\mu \tilde{\psi})}{\LL} &= 0 \\
\left( i \grad + e A - \frac{mc}{\Hbar}\right) \psi - (0) &= 0
\end{aligned}
\end{equation}

Or
\begin{equation}\label{eqn:diracLagrangian:404}
\begin{aligned}
\Hbar (i \grad + e A) \psi = mc \psi
\end{aligned}
\end{equation}

Except for the additional \(e A\) term here, this is the Dirac equation that we get from taking square roots of the Klein-Gordon equation.  Should \(A\) be considered a field variable?  More likely is that this is a supplied potential as in the \(V\) of the non-relativistic
Schr\"{o}dinger's equation.

Being so loose with the math here (ie: taking partials with respect to non-scalar variables) is somewhat disturbing but developing some intuition is worthwhile before getting the details down.

\subsection{Conjugate field equation}

Our Lagrangian is not at all symmetric looking, having derivatives of \(\psi\), but not \(\overbar{\psi}\).  Compare this to the Lagrangians for the
Schr\"{o}dinger's equation, and Klein-Gordon equation respectively, which are

\begin{equation}\label{eqn:diracLagrangian:424}
\begin{aligned}
\LL &= \frac{\Hbar^2}{2m}
(\spacegrad \psi) \cdot (\spacegrad \psi^\conj) + V \psi \psi^\conj + {i \Hbar} \left( \psi \partial_t \psi^\conj - \psi^\conj \partial_t \psi \right) \\
\LL &= -\partial^\nu \psi \partial_\nu \psi^\conj + \frac{m^2 c^2}{\Hbar^2} \psi \psi^\conj.
\end{aligned}
\end{equation}

With these Lagrangians one gets the field equation for \(\psi\), differentiating with respect to the conjugate field \(\psi^\conj\), and the conjugate equation with differentiation with respect to \(\psi\) (where \(\psi\) and \(\psi^\conj\) are treated as independent field variables).

It is not obvious that evaluating the Euler-Lagrange equations will produce a similarly regular result, so
%we would get a similarly nice symmetric reversal result with respect to \(\psi\) and \(\tilde{\psi}\).
let us compute the derivatives with respect to the \(\psi\) field variables to compute the equations for \(\overbar{\psi}\) or \(\tilde{\psi}\) to see what results.  Written out in coordinates so that we can apply the Euler-Lagrange equations, our Lagrangian (with \(A\) terms omitted) is

\begin{equation}\label{eqn:diracLag:lag4}
\begin{aligned}
\LL = \tilde{\psi}\left( i \gamma^\mu \partial_\mu + e A - \frac{m c}{\Hbar} \right) \psi
\end{aligned}
\end{equation}

Again abusing the Euler Lagrange equations, ignoring the possible issues with commuting partials taken
with respect to spinors (not scalar variables), blinding plugging into the formulas we have

\begin{equation}\label{eqn:diracLagrangian:444}
\begin{aligned}
\PD{\psi}{\LL} &= \partial_\mu \PD{\partial_\mu \psi}{\LL} \\
\tilde{\psi}\left( e A - \frac{m c}{\Hbar} \right) &= \partial_\mu \left(\tilde{\psi} i \gamma^\mu \right)
\end{aligned}
\end{equation}

reversing this entire equation we have

\begin{equation}\label{eqn:diracLagrangian:464}
\begin{aligned}
\left( e A - \frac{m c}{\Hbar} \right) \psi &= \gamma^\mu i \partial_\mu \psi = - i \grad \psi
\end{aligned}
\end{equation}

Or
\begin{equation}\label{eqn:diracLagrangian:484}
\begin{aligned}
\Hbar \left( i \grad + e A \right) \psi = m c \psi
\end{aligned}
\end{equation}

So we do in fact get the same field equation regardless of which of the two field variables one differentiates with.  That is not obvious looking at the Lagrangian.

\section{Alternate Dirac Lagrangian with antisymmetric terms}

Now, the wikipedia article
\href{http://en.wikipedia.org/wiki/Dirac_equation#Adjoint_equation_and_Dirac_current}{Adjoint equation and Dirac current} lists the Lagrangian as

\begin{equation}\label{eqn:diracLagrangian:504}
\begin{aligned}
\LL = mc \overbar{\psi}\psi - {\inv{2}i\Hbar}(\overbar{\psi}\gamma^\mu (\partial_\mu\psi) - (\partial_\mu\overbar{\psi})\gamma^\mu \psi)
\end{aligned}
\end{equation}

Computing the Euler Lagrange equations for this potential free Lagrangian we have

\begin{equation}\label{eqn:diracLagrangian:524}
\begin{aligned}
m c \psi - \inv{2} i \Hbar \gamma^\mu \partial_\mu \psi = \partial_\mu \left( \inv{2} i \Hbar \gamma^\mu \psi \right)
\end{aligned}
\end{equation}

Or,
\begin{equation}\label{eqn:diracLagrangian:544}
\begin{aligned}
m c \psi = i \Hbar \grad \psi
\end{aligned}
\end{equation}

And the same computation, treating \(\overbar{\psi}\) as the independent field variable of interest we have

\begin{equation}\label{eqn:diracLagrangian:564}
\begin{aligned}
m c \psi + \inv{2} i \Hbar \partial_\mu \overbar{\psi} \gamma^\mu = -\inv{2} i \Hbar \partial_\mu \overbar{\psi} \gamma^\mu
\end{aligned}
\end{equation}

which is

\begin{equation}\label{eqn:diracLagrangian:584}
\begin{aligned}
m c \overbar{\psi} &= - i \Hbar \partial_\mu \overbar{\psi} \gamma^\mu \\
m c \gamma_0 \tilde{\psi} &= - i \Hbar \partial_\mu \gamma_0 \tilde{\psi} \gamma^\mu \\
m c \tilde{\psi} &= i \Hbar \partial_\mu \tilde{\psi} \gamma^\mu \\
m c \psi &= \Hbar \grad \psi i \\
\end{aligned}
\end{equation}

Or,
\begin{equation}\label{eqn:diracLagrangian:604}
\begin{aligned}
i \Hbar \grad \psi &= -m c \psi
\end{aligned}
\end{equation}

FIXME: This differs in sign from the same calculation with the Lagrangian of \eqnref{eqn:diracLag:lag4}.  Based on
the possibility of both roots in the Klein-Gordon equation, I suspect I have made a sign error in the first
calculation.

\section{Appendix}

\subsection{Pseudoscalar reversal}
\index{pseudoscalar}

The pseudoscalar reverses to itself

\begin{equation}\label{eqn:diracLagrangian:624}
\begin{aligned}
\tilde{i}
&= \gamma_{3210} \\
&= -\gamma_{2103} \\
&= -\gamma_{1023} \\
&= \gamma_{0123} \\
&= i,
\end{aligned}
\end{equation}

\subsection{Form of the spinor}

The specific structure of the spinor has not been defined here.  It has been assumed to be quaternion like,
and contain only even grades, but in the Dirac/Minkowski algebra that gives us two possibilities

\begin{equation}\label{eqn:diracLagrangian:644}
\begin{aligned}
\psi
&= \alpha + P^a \gamma_a \gamma_0 \\
&= \alpha + P^a \sigma_a
\end{aligned}
\end{equation}

Or
\begin{equation}\label{eqn:diracLagrangian:664}
\begin{aligned}
\psi
&= \alpha + P^c \gamma_a \wedge \gamma_b \\
&= \alpha - P^c \sigma_a \wedge \sigma_b \\
&= \alpha - i \epsilon_{a b c} P^c \sigma_c \\
\end{aligned}
\end{equation}

Spinors in Doran/Lasenby appear to use the latter form of dual Pauli vectors (wedge products of the Pauli spatial basis elements).  This actually makes sense since one wants a spatial bivector for rotation (ie: ``spin''), and not the spacetime bivectors, which provide a Lorentz boost action.

%
% Copyright � 2012 Peeter Joot.  All Rights Reserved.
% Licenced as described in the file LICENSE under the root directory of this GIT repository.
%

%
%
%\documentclass[]{eliblog}

\usepackage{amsmath}
\usepackage{mathpazo}

%
% shorthand for bold symbols, convenient for vectors and matrices
%
\newcommand{\Ba}[0]{\mathbf{a}}
\newcommand{\Bb}[0]{\mathbf{b}}
\newcommand{\Bc}[0]{\mathbf{c}}
\newcommand{\Bd}[0]{\mathbf{d}}
\newcommand{\Be}[0]{\mathbf{e}}
\newcommand{\Bf}[0]{\mathbf{f}}
\newcommand{\Bg}[0]{\mathbf{g}}
\newcommand{\Bh}[0]{\mathbf{h}}
\newcommand{\Bi}[0]{\mathbf{i}}
\newcommand{\Bj}[0]{\mathbf{j}}
\newcommand{\Bk}[0]{\mathbf{k}}
\newcommand{\Bl}[0]{\mathbf{l}}
\newcommand{\Bm}[0]{\mathbf{m}}
\newcommand{\Bn}[0]{\mathbf{n}}
\newcommand{\Bo}[0]{\mathbf{o}}
\newcommand{\Bp}[0]{\mathbf{p}}
\newcommand{\Bq}[0]{\mathbf{q}}
\newcommand{\Br}[0]{\mathbf{r}}
\newcommand{\Bs}[0]{\mathbf{s}}
\newcommand{\Bt}[0]{\mathbf{t}}
\newcommand{\Bu}[0]{\mathbf{u}}
\newcommand{\Bv}[0]{\mathbf{v}}
\newcommand{\Bw}[0]{\mathbf{w}}
\newcommand{\Bx}[0]{\mathbf{x}}
\newcommand{\By}[0]{\mathbf{y}}
\newcommand{\Bz}[0]{\mathbf{z}}
\newcommand{\BA}[0]{\mathbf{A}}
\newcommand{\BB}[0]{\mathbf{B}}
\newcommand{\BC}[0]{\mathbf{C}}
\newcommand{\BD}[0]{\mathbf{D}}
\newcommand{\BE}[0]{\mathbf{E}}
\newcommand{\BF}[0]{\mathbf{F}}
\newcommand{\BG}[0]{\mathbf{G}}
\newcommand{\BH}[0]{\mathbf{H}}
\newcommand{\BI}[0]{\mathbf{I}}
\newcommand{\BJ}[0]{\mathbf{J}}
\newcommand{\BK}[0]{\mathbf{K}}
\newcommand{\BL}[0]{\mathbf{L}}
\newcommand{\BM}[0]{\mathbf{M}}
\newcommand{\BN}[0]{\mathbf{N}}
\newcommand{\BO}[0]{\mathbf{O}}
\newcommand{\BP}[0]{\mathbf{P}}
\newcommand{\BQ}[0]{\mathbf{Q}}
\newcommand{\BR}[0]{\mathbf{R}}
\newcommand{\BS}[0]{\mathbf{S}}
\newcommand{\BT}[0]{\mathbf{T}}
\newcommand{\BU}[0]{\mathbf{U}}
\newcommand{\BV}[0]{\mathbf{V}}
\newcommand{\BW}[0]{\mathbf{W}}
\newcommand{\BX}[0]{\mathbf{X}}
\newcommand{\BY}[0]{\mathbf{Y}}
\newcommand{\BZ}[0]{\mathbf{Z}}

\newcommand{\Bzero}[0]{\mathbf{0}}
\newcommand{\Btheta}[0]{\boldsymbol{\theta}}
\newcommand{\Btau}[0]{\boldsymbol{\tau}}
\newcommand{\Bomega}[0]{\boldsymbol{\omega}}

%
% shorthand for unit vectors
%
\newcommand{\acap}[0]{\hat{\Ba}}
\newcommand{\bcap}[0]{\hat{\Bb}}
\newcommand{\ccap}[0]{\hat{\Bc}}
\newcommand{\dcap}[0]{\hat{\Bd}}
\newcommand{\ecap}[0]{\hat{\Be}}
\newcommand{\fcap}[0]{\hat{\Bf}}
\newcommand{\gcap}[0]{\hat{\Bg}}
\newcommand{\hcap}[0]{\hat{\Bh}}
\newcommand{\icap}[0]{\hat{\Bi}}
\newcommand{\jcap}[0]{\hat{\Bj}}
\newcommand{\kcap}[0]{\hat{\Bk}}
\newcommand{\lcap}[0]{\hat{\Bl}}
\newcommand{\mcap}[0]{\hat{\Bm}}
\newcommand{\ncap}[0]{\hat{\Bn}}
\newcommand{\ocap}[0]{\hat{\Bo}}
\newcommand{\pcap}[0]{\hat{\Bp}}
\newcommand{\qcap}[0]{\hat{\Bq}}
\newcommand{\rcap}[0]{\hat{\Br}}
\newcommand{\scap}[0]{\hat{\Bs}}
\newcommand{\tcap}[0]{\hat{\Bt}}
\newcommand{\ucap}[0]{\hat{\Bu}}
\newcommand{\vcap}[0]{\hat{\Bv}}
\newcommand{\wcap}[0]{\hat{\Bw}}
\newcommand{\xcap}[0]{\hat{\Bx}}
\newcommand{\ycap}[0]{\hat{\By}}
\newcommand{\zcap}[0]{\hat{\Bz}}
\newcommand{\thetacap}[0]{\hat{\Btheta}}

%
% to write R^n and C^n in a distinguishable fashion.  Perhaps change this
% to the double lined characters upon figuring out how to do so.
%
\newcommand{\C}[1]{$\mathbb{C}^{#1}$}
\newcommand{\R}[1]{$\mathbb{R}^{#1}$}

%
% various generally useful helpers
%

% derivative of #1 wrt. #2:
\newcommand{\D}[2] {\frac {d#2} {d#1}}

\newcommand{\inv}[1]{\frac{1}{#1}}
\newcommand{\cross}[0]{\times}

\newcommand{\abs}[1]{\lvert{#1}\rvert}
\newcommand{\norm}[1]{\lVert{#1}\rVert}
\newcommand{\innerprod}[2]{\langle{#1}, {#2}\rangle}
\newcommand{\dotprod}[2]{{#1} \cdot {#2}}
\newcommand{\bdotprod}[2]{\left({#1} \cdot {#2}\right)}
\newcommand{\crossprod}[2]{{#1} \cross {#2}}
\newcommand{\tripleprod}[3]{\dotprod{\left(\crossprod{#1}{#2}\right)}{#3}}

\DeclareMathOperator{\Proj}{Proj}
\DeclareMathOperator{\Span}{span}
\DeclareMathOperator{\Sgn}{sgn}
\DeclareMathOperator{\Area}{Area}
\DeclareMathOperator{\Volume}{Volume}

%
% A few miscellaneous things specific to this document
%
\newcommand{\crossop}[1]{\crossprod{#1}{}}

% R2 vector.
\newcommand{\VectorTwo}[2]{
\begin{bmatrix}
 {#1} \\
 {#2}
\end{bmatrix}
}

\newcommand{\VectorN}[1]{
\begin{bmatrix}
{#1}_1 \\
{#1}_2 \\
\vdots \\
{#1}_N \\
\end{bmatrix}
}

\newcommand{\DETuvij}[4]{
\begin{vmatrix}
 {#1}_{#3} & {#1}_{#4} \\
 {#2}_{#3} & {#2}_{#4}
\end{vmatrix}
}

\newcommand{\DETuvwijk}[6]{
\begin{vmatrix}
 {#1}_{#4} & {#1}_{#5} & {#1}_{#6} \\
 {#2}_{#4} & {#2}_{#5} & {#2}_{#6} \\
 {#3}_{#4} & {#3}_{#5} & {#3}_{#6}
\end{vmatrix}
}

\newcommand{\DETuvwxijkl}[8]{
\begin{vmatrix}
 {#1}_{#5} & {#1}_{#6} & {#1}_{#7} & {#1}_{#8} \\
 {#2}_{#5} & {#2}_{#6} & {#2}_{#7} & {#2}_{#8} \\
 {#3}_{#5} & {#3}_{#6} & {#3}_{#7} & {#3}_{#8} \\
 {#4}_{#5} & {#4}_{#6} & {#4}_{#7} & {#4}_{#8} \\
\end{vmatrix}
}

%\newcommand{\DETuvwxyijklm}[10]{
%\begin{vmatrix}
% {#1}_{#6} & {#1}_{#7} & {#1}_{#8} & {#1}_{#9} & {#1}_{#10} \\
% {#2}_{#6} & {#2}_{#7} & {#2}_{#8} & {#2}_{#9} & {#2}_{#10} \\
% {#3}_{#6} & {#3}_{#7} & {#3}_{#8} & {#3}_{#9} & {#3}_{#10} \\
% {#4}_{#6} & {#4}_{#7} & {#4}_{#8} & {#4}_{#9} & {#4}_{#10} \\
% {#5}_{#6} & {#5}_{#7} & {#5}_{#8} & {#5}_{#9} & {#5}_{#10}
%\end{vmatrix}
%}

% R3 vector.
\newcommand{\VectorThree}[3]{
\begin{bmatrix}
 {#1} \\
 {#2} \\
 {#3}
\end{bmatrix}
}



\author{Peeter Joot}
\email{peeter.joot@gmail.com}


%\chapter{Geometric Algebra equivalants for Pauli Matrices}
\chapter{Pauli Matrices}
\index{Pauli matrices}
\label{chap:pauliMatrix}
% does not work with underscore.
%%\blogpage{http://sites.google.com/site/peeterjoot/math/pauli_matrix.pdf}
%\date{Dec 06, 2008.  pauliMatrix.tex}
%\date{Dec 06, 2008}
%%\revisionInfo{\(RCSfile: pauliMatrix.tex,v \) Last \(Revision: 1.29 \) \(Date: 2009/08/01 22:12:18 \)}

\beginArtWithToc

\section{Motivation}

Having learned Geometric (Clifford) Algebra from \citep{doran2003gap}, \citep{hestenes1999nfc}, \citep{dorst2007gac}, and other sources before studying any quantum mechanics, trying to work with (and talk to people familiar with) the Pauli and Dirac matrix notation as used in traditional quantum mechanics becomes difficult.

The aim of these notes is to work through equivalents to many Clifford algebra expressions entirely in commutator and anticommutator notations.  This will show the mapping between the (generalized) dot product and the wedge product, and also show how the different grade elements of the Clifford algebra \(\Clifford{3}{0}\) manifest in their matrix forms.

\section{Pauli Matrices}

The matrices in question are:

\begin{equation}\label{eqn:pauliMatrix:20}
\begin{aligned}
\sigma_1 &= \PauliX \\
\sigma_2 &= \PauliY \\
\sigma_3 &= \PauliZ
\end{aligned}
\end{equation}

These all have positive square as do the traditional Euclidean unit vectors \(\Be_i\), and so can be used algebraically as a vector basis for \R{3}.  So any vector that we can write in coordinates


\begin{equation}\label{eqn:pauliMatrix:40}
\begin{aligned}
\Bx = x^i \Be_i,
\end{aligned}
\end{equation}

we can equivalently write (an isomorphism) in terms of the Pauli matrix's

\begin{equation}\label{eqn:pauliMat:vectorInPauliBasis}
\begin{aligned}
x = x^i \sigma_i.
\end{aligned}
\end{equation}

\subsection{Pauli Vector}
\index{Pauli vector}
%Pauli Matrix article
\citep{wiki:pauli} introduces the Pauli vector as a mechanism for mapping between a vector basis and this matrix basis

\begin{equation}\label{eqn:pauliMatrix:60}
\begin{aligned}
\Bsigma = \sum \sigma_i \Be_i
\end{aligned}
\end{equation}

This is a curious looking construct with products of \(2 x 2\) matrices and \R{3} vectors.  Obviously these are not the usual \(3 x 1\) column vector representations.  This Pauli vector is thus really a notational construct.  If one takes the dot product of a vector expressed using the standard orthonormal Euclidean basis \(\{\Be_i\}\) basis, and then takes the dot product with the Pauli matrix in a mechanical fashion

\begin{equation}\label{eqn:pauliMatrix:80}
\begin{aligned}
\Bx \cdot \Bsigma &=
(x^i \Be_i) \cdot \sum \sigma_j \Be_j \\
&= \sum_{i,j} x^i \sigma_j \Be_i \cdot \Be_j \\
&= x^i \sigma_i \\
\end{aligned}
\end{equation}

one arrives at the matrix representation of the vector in the Pauli basis \(\{\sigma_i\}\).  Does this construct have any value?  That I do not know, but for the rest of these notes the coordinate representation as in equation \eqnref{eqn:pauliMat:vectorInPauliBasis} will be used directly.

\subsection{Matrix squares}

It was stated that the Pauli matrices have unit square.  Direct calculation of this is straightforward, and confirms the assertion

\begin{equation}\label{eqn:pauliMatrix:100}
\begin{aligned}
{\sigma_1}^2 &= \PauliX \PauliX = \PauliI = I \\
{\sigma_2}^2 &= \PauliY \PauliY = i^2 \PauliYNoI \PauliYNoI = \PauliI = I \\
{\sigma_3}^2 &= \PauliZ \PauliZ = \PauliI = I \\
\end{aligned}
\end{equation}

Note that unlike the vector (Clifford) square the identity matrix and not a scalar.

\subsection{Length}
If we are to operate with Pauli matrices how do we express our most basic vector operation, the length?

Examining a vector lying along one direction, say, \(\Ba = \alpha\xcap\) we expect

\begin{equation}\label{eqn:pauliMatrix:120}
\begin{aligned}
\Ba^2 = \Ba \cdot \Ba = \alpha^2 \xcap \cdot \xcap = \alpha^2.
\end{aligned}
\end{equation}

Lets contrast this to the Pauli square for the same vector \(y = \alpha\sigma_1\)

\begin{equation}\label{eqn:pauliMatrix:140}
\begin{aligned}
y^2 = \alpha^2 {\sigma_1}^2 = \alpha^2 I
\end{aligned}
\end{equation}

The wiki article mentions trace, but no application for it.  Since \(\traceB{I} = 2\), an observable application is that the trace operator provides a mechanism to convert a diagonal matrix to a scalar.  In particular for this scaled unit vector \(y\) we have

\begin{equation}\label{eqn:pauliMatrix:160}
\begin{aligned}
\alpha^2 = \inv{2} \traceB{y^2}
\end{aligned}
\end{equation}

It is plausible to guess that the squared length will be related to the matrix square in the general case as well

\begin{equation}\label{eqn:pauliMatrix:180}
\begin{aligned}
\Abs{x}^2 = \inv{2}\traceB{x^2}
\end{aligned}
\end{equation}

Let us see if this works by performing the coordinate expansion

\begin{equation}\label{eqn:pauliMatrix:200}
\begin{aligned}
x^2
&= (x^i \sigma_i)(x^j \sigma_j) \\
&= x^i x^j \sigma_i \sigma_j \\
\end{aligned}
\end{equation}

A split into equal and different indices thus leaves

\begin{equation}\label{eqn:pauliMat:square}
\begin{aligned}
x^2
&= \sum_{i < j} x^i x^j (\sigma_i \sigma_j + \sigma_j \sigma_i) + \sum_i (x^i)^2 {\sigma_i}^2
\end{aligned}
\end{equation}

As an algebra that is isomorphic to the Clifford Algebra \(\Clifford{3}{0}\) it is expected that the \(\sigma_i \sigma_j\) matrices anticommute for \(i \ne j\).  Multiplying these out verifies this

\begin{equation}\label{eqn:pauliMatrix:220}
\begin{aligned}
\begin{array}{l l l l}
\sigma_1 \sigma_2 &= i \PauliX \PauliYNoI    &= i \PauliZ                                  &=  i \sigma_3 \\
\sigma_2 \sigma_1 &= i \PauliYNoI \PauliX    &= i \begin{bmatrix}-1 & 0 \\0 & 1\end{bmatrix} &= -i \sigma_3 \\
\sigma_3 \sigma_1 &= \PauliZ \PauliX         &=   \begin{bmatrix}0 & 1 \\-1 & 0\end{bmatrix} &=  i \sigma_2 \\
\sigma_1 \sigma_3 &= \PauliX \PauliZ         &=   \begin{bmatrix}0 & -1 \\1 & 0\end{bmatrix} &= -i \sigma_2 \\
\sigma_2 \sigma_3 &= i \PauliYNoI \PauliZ    &= i \PauliX                                  &=  i \sigma_1 \\
\sigma_3 \sigma_2 &= i \PauliZ \PauliYNoI    &= i \begin{bmatrix}0 & -1 \\-1 & 0\end{bmatrix} &= -i \sigma_3 \\
\end{array}.
\end{aligned}
\end{equation}

Thus in \eqnref{eqn:pauliMat:square} the sum over the \(\{i < j\} = \{12, 23, 13\}\) indices is zero.

Having computed this, our vector square leaves us with the vector length multiplied by the identity matrix

\begin{equation}\label{eqn:pauliMatrix:240}
\begin{aligned}
x^2 = \sum_i (x^i)^2 I.
\end{aligned}
\end{equation}

Invoking the trace operator will therefore extract just the scalar length desired

\begin{equation}\label{eqn:pauliMatrix:260}
\begin{aligned}
\Abs{x}^2 = \inv{2} \traceB{x^2} = \sum_i (x^i)^2.
\end{aligned}
\end{equation}

\subsubsection{Aside: Summarizing the multiplication table}

It is worth pointing out that the multiplication table above used to confirm the antisymmetric behavior of the Pauli basis can be summarized as

\begin{equation}\label{eqn:pauliMatrix:280}
\begin{aligned}
\sigma_a \sigma_b = 2 i \epsilon_{abc} \sigma_c
\end{aligned}
\end{equation}

\subsection{Scalar product}
\index{scalar product!Pauli matrix}

Having found the expression for the length of a vector in the Pauli basis, the next logical desirable identity is the dot product.  One can guess that this will be the trace of a scaled symmetric product, but can motivate this without guessing in the usual fashion, by calculating the length of an orthonormal sum.

Consider first the length of a general vector sum.  To calculate this we first wish to calculate the matrix square of this sum.

\begin{equation}\label{eqn:pauliMatrix:300}
\begin{aligned}
(x + y)^2 &= x^2 + y^2 + x y + y x \\
\end{aligned}
\end{equation}

If these vectors are perpendicular this equals \(x^2 + y^2\).  Thus orthonormality implies that
\begin{equation}\label{eqn:pauliMatrix:320}
\begin{aligned}
x y + y x &= 0 \\
\end{aligned}
\end{equation}

or,

\begin{equation}\label{eqn:pauliMatrix:340}
\begin{aligned}
y x &= - x y
\end{aligned}
\end{equation}

We have already observed this by direct calculation for the Pauli matrices themselves.  Now, this is not any different than the usual description of perpendicularity in a Clifford Algebra, and it is notable that there are not any references to matrices in this argument.  One only requires that a well defined vector product exists, where the squared vector has a length interpretation.

One matrix dependent observation that can be made is that since the left hand side and the \(x^2\), and \(y^2\) terms are all diagonal, this symmetric sum must also be diagonal.  Additionally, for the length of this vector sum we then have

\begin{equation}\label{eqn:pauliMatrix:360}
\begin{aligned}
{\Abs{x + y}}^2
%&= x^2 + y^2 + x y + y x \\
&= \Abs{x}^2 + \Abs{y}^2 + \inv{2}\traceB{x y + y x} \\
\end{aligned}
\end{equation}

For correspondence with the Euclidean dot product of two vectors we must then have

\begin{equation}\label{eqn:pauliMatrix:380}
\begin{aligned}
\scalarProduct{x}{y} &= \inv{4}\traceB{x y + y x}.
\end{aligned}
\end{equation}

Here \(x \bullet y\) has been used to denote this scalar product (ie: a plain old number), since \(x \cdot y\) will be used later for a matrix dot product (this times the identity matrix) which is more natural in many ways for this Pauli algebra.

Observe the symmetric product that is found embedded in this scalar selection operation.  In physics this is known as the anticommutator, where the commutator is the antisymmetric sum.  In the physics notation the anticommutator (symmetric sum) is

\begin{equation}\label{eqn:pauliMat:anticommutator}
\begin{aligned}
\symmetric{x}{y} &= x y + y x
\end{aligned}
\end{equation}

So this scalar selection can be written

\begin{equation}\label{eqn:pauliMatrix:400}
\begin{aligned}
\scalarProduct{ x}{ y} = \inv{4}\trace{\symmetric{x}{y}}
\end{aligned}
\end{equation}

Similarly, the commutator, an antisymmetric product, is denoted:

\begin{equation}\label{eqn:pauliMat:commutator}
\begin{aligned}
\antisymmetric{x}{y} &= x y - y x,
\end{aligned}
\end{equation}

A close relationship between this commutator and the wedge product of Clifford Algebra is expected.

\subsection{Symmetric and antisymmetric split}
\index{symmetric sum!Pauli matrix}
\index{antisymmetric sum!Pauli matrix}

As with the Clifford product, the symmetric and antisymmetric split of a vector product is a useful concept.  This can be used to write the product of two Pauli basis vectors in terms of the anticommutator and commutator products

\begin{equation}\label{eqn:pauliMatrix:420}
\begin{aligned}
x y &= \inv{2} \symmetric{x}{y} + \inv{2} \antisymmetric{x}{y} \\
y x &= \inv{2} \symmetric{x}{y} - \inv{2} \antisymmetric{x}{y}
\end{aligned}
\end{equation}

These follows from the definition of the anticommutator \eqnref{eqn:pauliMat:anticommutator} and commutator \eqnref{eqn:pauliMat:commutator} products above, and are the equivalents of the Clifford symmetric and antisymmetric split into dot and wedge products

\begin{equation}\label{eqn:pauliMatrix:440}
\begin{aligned}
x y &= {x} \cdot {y} + {x} \wedge {y} \\
y x &= {x} \cdot {y} - {x} \wedge {y}
\end{aligned}
\end{equation}

Where the dot and wedge products are respectively

\begin{equation}\label{eqn:pauliMatrix:460}
\begin{aligned}
x \cdot y &= \inv{2}(x y + y x) \\
x \wedge y &= \inv{2}(x y - y x)
\end{aligned}
\end{equation}

Note the factor of two differences in the two algebraic notations.  In particular very handy Clifford vector product reversal formula

\begin{equation}\label{eqn:pauliMatrix:480}
\begin{aligned}
y x = - x y + 2 x \cdot y
\end{aligned}
\end{equation}

has no factor of two in its Pauli anticommutator equivalent

\begin{equation}\label{eqn:pauliMatrix:500}
\begin{aligned}
y x = - x y + \symmetric{x}{y}
\end{aligned}
\end{equation}

\subsection{Vector inverse}

It has been observed that the square of a vector is diagonal in this matrix representation, and can therefore be inverted for any non-zero vector

\begin{equation}\label{eqn:pauliMatrix:520}
\begin{aligned}
x^2 &= \Abs{x}^2 I \\
(x^2)^{-1} &= \Abs{x}^{-2} I \\
\implies \\
x^2 (x^2)^{-1} &= I \\
\end{aligned}
\end{equation}

So it is therefore quite justifiable to define
\begin{equation}\label{eqn:pauliMatrix:540}
\begin{aligned}
x^{-2} = \inv{x^2} \equiv \Abs{x}^{-2} I \\
\end{aligned}
\end{equation}

This allows for the construction of a dual sided vector inverse operation.

\begin{equation}\label{eqn:pauliMatrix:560}
\begin{aligned}
x^{-1}
&\equiv \inv{\Abs{x}^2} x \\
&= \inv{x^2} x \\
&= x \inv{x^2} \\
\end{aligned}
\end{equation}

This inverse is a scaled version of the vector itself.

The diagonality of the squared matrix or the inverse of that allows for commutation with x.  This diagonality plays the same role as the scalar in a regular Clifford square.  In either case the square can commute with the vector, and that commutation allows the inverse to have both left and right sided action.

Note that like the Clifford vector inverse when the vector is multiplied with this inverse, the product resides outside of the proper \R{3} Pauli basis since the identity matrix is required.

\subsection{Coordinate extraction}

Given a vector in the Pauli basis, we can extract the coordinates using the scalar product

\begin{equation}\label{eqn:pauliMatrix:580}
\begin{aligned}
x = \sum_i \inv{4}\trace{\symmetric{x}{\sigma_i}} \sigma_i
\end{aligned}
\end{equation}

But do not need to convert to strict scalar form if we are multiplying by a Pauli matrix.  So in anticommutator notation this takes the form

\begin{equation}\label{eqn:pauliMat:fourier}
\begin{aligned}
x &= x^i \sigma_i = \sum_i \inv{2}\symmetric{x}{\sigma_i} \sigma_i \\
x^i &= \inv{2}\symmetric{x}{\sigma_i}
\end{aligned}
\end{equation}

\subsection{Projection and rejection}

The usual Clifford algebra trick for projective and rejective split maps naturally to matrix form.  Write

\begin{equation}\label{eqn:pauliMatrix:600}
\begin{aligned}
x
&= x a a^{-1} \\
&= (x a) a^{-1} \\
&= \left( \inv{2}\symmetric{x}{a} + \inv{2}\antisymmetric{x}{a} \right) a^{-1} \\
&= \left( \inv{2}\left(x a + a x \right) + \inv{2} \left(x a - a x \right) \right) a^{-1} \\
&= \inv{2}\left(x + a x a^{-1} \right) + \inv{2} \left(x - a x a^{-1} \right) \\
\end{aligned}
\end{equation}

Since \(\symmetric{x}{a}\) is diagonal, this first term is proportional to \(a^{-1}\), and thus lines in the direction of \(a\) itself.  The second term is perpendicular to \(a\).

These are in fact the projection of \(x\) in the direction of \(a\) and rejection of \(x\) from the direction of \(a\) respectively.

\begin{equation}\label{eqn:pauliMatrix:620}
\begin{aligned}
x &= x_\parallel + x_\perp \\
x_\parallel &= \Proj_a(x) = \inv{2}\symmetric{x}{a}a^{-1} = \inv{2}\left(x + a x a^{-1} \right) \\
x_\perp &= \RejName_a(x) = \inv{2} \antisymmetric{x}{a} a^{-1} = \inv{2} \left( x - a x a^{-1} \right) \\
\end{aligned}
\end{equation}

To complete the verification of this note that the perpendicularity of the \(x_\perp\) term can be verified by taking dot products

\begin{equation}\label{eqn:pauliMatrix:640}
\begin{aligned}
\inv{2}\symmetric{a}{x_\perp}
&= \inv{4}\left( a \left(x - a x a^{-1} \right) +\left(x - a x a^{-1} \right) a \right) \\
&= \inv{4}\left( a x - a a x a^{-1} + x a - a x a^{-1} a \right) \\
&= \inv{4}\left( a x - x a + x a - a x \right) \\
&= 0
\end{aligned}
\end{equation}

\subsection{Space of the vector product}

Expansion of the anticommutator and commutator in coordinate form shows that these entities lie in a different space than the vectors itself.

For real coordinate vectors in the Pauli basis, all the commutator values are imaginary multiples and thus not representable

\begin{equation}\label{eqn:pauliMatrix:660}
\begin{aligned}
\antisymmetric{x}{y}
&= x^a \sigma_a y^b \sigma_b - y^a \sigma_a x^b \sigma_b \\
&= (x^a y^b - y^a x^b) \sigma_a \sigma_b \\
&= 2 i (x^a y^b - y^a x^b) \epsilon_{a b c} \sigma_c
\end{aligned}
\end{equation}

Similarly, the anticommutator is diagonal, which also falls outside the Pauli vector basis:

\begin{equation}\label{eqn:pauliMatrix:680}
\begin{aligned}
\symmetric{x}{y}
&= x^a \sigma_a y^b \sigma_b + y^a \sigma_a x^b \sigma_b \\
&= (x^a y^b + y^a x^b) \sigma_a \sigma_b \\
&= (x^a y^b + y^a x^b) ( I \delta_{a b} + i \epsilon_{a b c} \sigma_c) \\
&= \sum_a (x^a y^a + y^a x^a) I
+\sum_{a<b}(x^a y^b + y^a x^b) i (\mathLabelBox{\epsilon_{a b c} + \epsilon_{b a c}}{\(=0\)}) \sigma_c \\
&= \sum_a (x^a y^a + y^a x^a) I \\
&= 2 \sum_a x^a y^a I,
\end{aligned}
\end{equation}

These correspond to the Clifford dot product being scalar (grade zero), and the wedge defining a grade two space, where grade expresses the minimal degree that a product can be reduced to.  By example a Clifford product of normal unit vectors such as

\begin{equation}\label{eqn:pauliMatrix:700}
\begin{aligned}
\Be_1 \Be_3 \Be_4 \Be_1 \Be_3 \Be_4 \Be_3 &\propto \Be_3 \\
\Be_2 \Be_3 \Be_4 \Be_1 \Be_3 \Be_4 \Be_3 \Be_5 &\propto \Be_1 \Be_2 \Be_3 \Be_5
\end{aligned}
\end{equation}

are grade one and four respectively.  The proportionality constant will be dependent on metric of the underlying vector space and the number of permutations required to group terms in pairs of matching indices.

\subsection{Completely antisymmetrized product of three vectors}

In a Clifford algebra no imaginary number is required to express the antisymmetric (commutator) product.  However, the bivector space can be enumerated using a dual basis defined by multiplication of the vector basis elements with the unit volume trivector.  That is also the case here and gives a geometrical meaning to the imaginaries of the Pauli formulation.

How do we even write the unit volume element in Pauli notation?  This would be

\begin{equation}\label{eqn:pauliMatrix:720}
\begin{aligned}
\sigma_1 \wedge \sigma_2 \wedge \sigma_3
&= (\sigma_1 \wedge \sigma_2) \wedge \sigma_3 \\
&= \inv{2} \antisymmetric{\sigma_1}{\sigma_2} \wedge \sigma_3 \\
&= \inv{4} \left( \antisymmetric{\sigma_1}{\sigma_2} \sigma_3 + \sigma_3 \antisymmetric{\sigma_1}{\sigma_2} \right) \\
\end{aligned}
\end{equation}

So we have
\begin{equation}\label{eqn:pauliMat:triplewedgeproduct}
\begin{aligned}
\sigma_1 \wedge \sigma_2 \wedge \sigma_3
&= \inv{8} \symmetricBladeVecPauli{\sigma_1}{\sigma_2}{\sigma_3}
\end{aligned}
\end{equation}

Similar expansion of \(\sigma_1 \wedge \sigma_2 \wedge \sigma_3 = \sigma_1 \wedge (\sigma_2 \wedge \sigma_3)\), or \(\sigma_1 \wedge \sigma_2 \wedge \sigma_3 = (\sigma_3 \wedge \sigma_1) \wedge \sigma_2\) shows that we must also have

\begin{equation}\label{eqn:pauliMatrix:740}
\begin{aligned}
\symmetricBladeVecPauli{\sigma_1}{\sigma_2}{\sigma_3}
= \symmetricVecBladePauli{\sigma_1}{\sigma_2}{\sigma_3}
= \symmetricBladeVecPauli{\sigma_3}{\sigma_1}{\sigma_2}
\end{aligned}
\end{equation}

Until now the differences in notation between the anticommutator/commutator and the dot/wedge product of the Pauli algebra and Clifford algebra respectively have only differed by factors of two, which is not much of a big deal.  However, having to express the naturally associative wedge product operation in the non-associative looking notation of equation \eqnref{eqn:pauliMat:triplewedgeproduct} is rather unpleasant seeming.  Looking at an expression of the form gives no mnemonic hint of the underlying associativity, and actually seems obfuscating.  I suppose that one could get used to it though.

We expect to get a three by three determinant out of the trivector product.  Let us verify this by expanding this in Pauli notation for three general coordinate vectors

\begin{equation}\label{eqn:pauliMatrix:760}
\begin{aligned}
\symmetricBladeVecPauli{x}{y}{z}
&= \symmetricBladeVecPauli{x^a \sigma_a}{y^b \sigma_b}{ z^c \sigma_c} \\
&= 2 i \epsilon_{a b d} x^a y^b z^c \symmetric{\sigma_d}{\sigma_c} \\
&= 4 i \epsilon_{a b d} x^a y^b z^c \delta_{c d} I \\
&= 4 i \epsilon_{a b c} x^a y^b z^c I \\
&= 4 i
%\DETuvwijk{x}{y}{z}{a}{b}{c}
\begin{vmatrix}
 x_a & x_b & x_c \\
 y_a & y_b & y_c \\
 z_a & z_b & z_c
\end{vmatrix} I
\end{aligned}
\end{equation}

In particular, our unit volume element is

\begin{equation}\label{eqn:pauliMatrix:780}
\begin{aligned}
\sigma_1 \wedge \sigma_2 \wedge \sigma_3
&= \inv{4} \symmetricBladeVecPauli{\sigma_1}{\sigma_2}{\sigma_3} = i I
\end{aligned}
\end{equation}

So one sees that the complex number \(i\) in the Pauli algebra can logically be replaced by the unit pseudoscalar \(i I\), and relations involving \(i\), like the commutator expansion of a vector product, is restored to the expected dual form of Clifford algebra

\begin{equation}\label{eqn:pauliMatrix:800}
\begin{aligned}
\sigma_a \wedge \sigma_b
&= \inv{2} \antisymmetric{\sigma_a}{\sigma_b} \\
&= i \epsilon_{a b c} \sigma_c \\
&= (\sigma_a \wedge \sigma_b \wedge \sigma_c) \sigma_c \\
\end{aligned}
\end{equation}

Or
\begin{equation}\label{eqn:pauliMat:dualBivector}
\begin{aligned}
\sigma_a \wedge \sigma_b &= (\sigma_a \wedge \sigma_b \wedge \sigma_c) \cdot \sigma_c
\end{aligned}
\end{equation}

\subsection{Duality}

We have seen that multiplication by \(i\) is a duality operation, which is expected since \(iI\) is the matrix equivalent of the unit pseudoscalar.  Logically this means that for a vector \(x\), the product \((iI) x\) represents a plane quantity (torque, angular velocity/momentum, ...).  Similarly if \(B\) is a plane object, then \((iI) B\) will have a vector interpretation.

In particular, for the antisymmetric (commutator) part of the vector product \(x y\)

\begin{equation}\label{eqn:pauliMatrix:820}
\begin{aligned}
\inv{2} \antisymmetric{x}{y}
&= \inv{2} x^a y^b \antisymmetric{\sigma_a}{\sigma_b} \\
&= x^a y^b i \epsilon_{a b c}{\sigma_c} \\
\end{aligned}
\end{equation}

a ``vector'' in the dual space spanned by \(\{i \sigma_a\}\) is seen to be more naturally interpreted as a plane quantity (bivector in Clifford algebra).

As in Clifford algebra, we can write the cross product in terms of the antisymmetric product

\begin{equation}\label{eqn:pauliMatrix:840}
\begin{aligned}
a \times b = \inv{2i} \antisymmetric{a}{b}.
\end{aligned}
\end{equation}

With the factor of \(2\) in the denominator here (like the exponential form of sine), it is interesting
to contrast this to the cross product in its trigonometric form

\begin{equation}\label{eqn:pauliMatrix:860}
\begin{aligned}
a \times b
&= \Abs{a}\Abs{b} \sin(\theta) \ncap \\
&= \Abs{a}\Abs{b} \inv{2i} ( e^{i\theta} - e^{-i\theta}) \ncap
\end{aligned}
\end{equation}

This shows we can make the curious identity

\begin{equation}\label{eqn:pauliMatrix:880}
\begin{aligned}
\antisymmetric{\hata}{\hatb} &= ( e^{i\theta} - e^{-i\theta}) \ncap
\end{aligned}
\end{equation}

If one did not already know about the dual sides half angle rotation formulation of Clifford algebra, this is a hint about how one could potentially work towards that.  We have the commutator (or wedge product) as a rotation operator that leaves the normal component of a vector untouched (commutes with the normal vector).

\subsection{Complete algebraic space}

Pauli equivalents for all the elements in the Clifford algebra have now been
determined.

\begin{itemize}
\item scalar
\begin{equation}\label{eqn:pauliMatrix:900}
\begin{aligned}
\alpha \rightarrow \alpha I
\end{aligned}
\end{equation}

\item vector
\begin{equation}\label{eqn:pauliMatrix:920}
\begin{aligned}
u^i \sigma_i &\rightarrow
\begin{bmatrix}
0 & u^1 \\
u^1 & 0 \\
\end{bmatrix}
+
\begin{bmatrix}
0 & -i u^2 \\
i u^2 & 0 \\
\end{bmatrix}
+
\begin{bmatrix}
u^3 & 0 \\
0 & -u^3 \\
\end{bmatrix} \\
&=
\begin{bmatrix}
u^3 & u^1 -i u^2 \\
u^1 + i u^2 & -u^3 \\
\end{bmatrix}
\end{aligned}
\end{equation}

\item bivector
\begin{equation}\label{eqn:pauliMatrix:940}
\begin{aligned}
\sigma_1\sigma_2\sigma_3 v^a \sigma_a &\rightarrow i v^a \sigma_a \\
&= \begin{bmatrix}
iv^3 & iv^1 + v^2 \\
iv^1 - v^2 & -i v^3 \\
\end{bmatrix}
\end{aligned}
\end{equation}
\item pseudoscalar
\begin{equation}\label{eqn:pauliMatrix:960}
\begin{aligned}
\beta \sigma_1\sigma_2\sigma_3 \rightarrow i \beta I
\end{aligned}
\end{equation}
\end{itemize}

Summing these we have the mapping from Clifford basis to Pauli matrix as follows

\begin{equation}\label{eqn:pauliMatrix:980}
\begin{aligned}
\alpha + \beta I
+ u^i \sigma_i
+ I v^a \sigma_a
&\rightarrow
\begin{bmatrix}
(\alpha + u^3) + i(\beta + v^3) & (u^1 + v^2) +i (-u^2 + v^1) \\
(u^1 - v^2) + i (u^2 - v^1) & (\alpha -u^3) + i(\beta - v^3)\\
\end{bmatrix}
\end{aligned}
\end{equation}

Thus for any given sum of scalar, vector, bivector, and trivector elements we can completely express this in Pauli form as a general \(2 x 2\) complex matrix.

Provided that one can also extract the coordinates for each of the grades involved, this also provides a complete Clifford algebra characterization of an arbitrary complex \(2 x 2\) matrix.

Computationally this has some nice looking advantages.  Given any canned complex matrix software, one should be able to easily cook up with little work a working \R{3} Clifford calculator.

As for the coordinate extraction, part of the work can be done by taking real and imaginary components.  Let an element of the general algebra be denoted

\begin{equation}\label{eqn:pauliMatrix:1000}
\begin{aligned}
P =
\begin{bmatrix}
z_{11} & z_{12} \\
z_{21} & z_{22} \\
\end{bmatrix}
\end{aligned}
\end{equation}

We therefore have
\begin{equation}\label{eqn:pauliMatrix:1020}
\begin{aligned}
\Real(P) &=
\begin{bmatrix}
\alpha + u^3   & u^1 + v^2 \\
u^1 - v^2      & \alpha -u^3 \\
\end{bmatrix} \\
\Imag(P) &=
\begin{bmatrix}
\beta + v^3 & -u^2 + v^1 \\
u^2 + v^1 & \beta - v^3 \\
\end{bmatrix}
\end{aligned}
\end{equation}

By inspection, symmetric and antisymmetric sums of the real and imaginary parts recovers the coordinates as follows

\begin{equation}\label{eqn:pauliMatrix:1040}
\begin{aligned}
\alpha   &= \inv{2} \Real( z_{11} + z_{22} ) \\
u^3      &= \inv{2} \Real( z_{11} - z_{22} ) \\
u^1      &= \inv{2} \Real( z_{12} + z_{21} ) \\
v^2      &= \inv{2} \Real( z_{12} - z_{21} ) \\
\beta    &= \inv{2} \Imag( z_{11} + z_{22} ) \\
v^3      &= \inv{2} \Imag( z_{11} - z_{22} ) \\
v^1      &= \inv{2} \Imag( z_{12} + z_{21} ) \\
u^2      &= \inv{2} \Imag( -z_{12} + z_{21} ) \\
\end{aligned}
\end{equation}

In terms of grade selection operations the decomposition by grade

\begin{equation}\label{eqn:pauliMatrix:1060}
\begin{aligned}
P = \gpgradezero{P} +\gpgradeone{P} +\gpgradetwo{P} +\gpgradethree{P},
\end{aligned}
\end{equation}

is

\begin{equation}\label{eqn:pauliMatrix:1080}
\begin{aligned}
\gpgradezero{P} &= \inv{2} \Real( z_{11} + z_{22} ) = \inv{2} \Real(\trace P) \\
\gpgradeone{P} &= \inv{2} \left(
\Real( z_{12} + z_{21} ) \sigma_1
+\Imag( -z_{12} + z_{21} ) \sigma_2
+\Real( z_{11} - z_{22} ) \sigma_3
\right) \\
\gpgradetwo{P}
%&= \inv{2} \left(
%\Imag( z_{12} + z_{21} ) I \sigma_1
%+\Real( z_{12} - z_{21} ) I \sigma_2
%+\Imag( z_{11} - z_{22} ) I \sigma_3
%\right) \\
&= \inv{2} \left(
\Imag( z_{12} + z_{21} ) \sigma_2 \wedge \sigma_3
+\Real( z_{12} - z_{21} ) \sigma_3 \wedge \sigma_1
+\Imag( z_{11} - z_{22} ) \sigma_1 \wedge \sigma_2
\right) \\
\gpgradethree{P} &= \inv{2} \Imag( z_{11} + z_{22} ) I = \inv{2} \Imag(\trace P) \sigma_1 \wedge \sigma_2 \wedge \sigma_3
\end{aligned}
\end{equation}

Employing \(\Imag(z) = \Real(-iz)\), and \(\Real(z) = \Imag(iz)\) this can be made slightly more symmetrical, with Real operations selecting the vector coordinates and imaginary operations selecting the bivector coordinates.

\begin{equation}\label{eqn:pauliMatrix:1100}
\begin{aligned}
\gpgradezero{P} &= \inv{2} \Real( z_{11} + z_{22} ) = \inv{2} \Real(\trace P) \\
\gpgradeone{P} &= \inv{2} \left(
\Real( z_{12} + z_{21} ) \sigma_1
+\Real( iz_{12} - iz_{21} ) \sigma_2
+\Real( z_{11} - z_{22} ) \sigma_3
\right) \\
\gpgradetwo{P}
&= \inv{2} \left(
\Imag( z_{12} + z_{21} ) \sigma_2 \wedge \sigma_3
+\Imag( iz_{12} - iz_{21} ) \sigma_3 \wedge \sigma_1
+\Imag( z_{11} - z_{22} ) \sigma_1 \wedge \sigma_2
\right) \\
\gpgradethree{P} &= \inv{2} \Imag( z_{11} + z_{22} ) I = \inv{2} \Imag(\trace P) \sigma_1 \wedge \sigma_2 \wedge \sigma_3
\end{aligned}
\end{equation}

Finally, returning to the Pauli algebra, this also provides the following split of the Pauli multivector matrix into its geometrically significant components \(P = \gpgradezero{P} +\gpgradeone{P} +\gpgradetwo{P} +\gpgradethree{P}\),

\begin{equation}\label{eqn:pauliMatrix:1120}
\begin{aligned}
\gpgradezero{P} &= \inv{2} \Real( z_{11} + z_{22} ) I \\
\gpgradeone{P} &= \inv{2} \left(
\Real( z_{12} + z_{21} ) \sigma_1
+\Real( iz_{12} - iz_{21} ) \sigma_2
+\Real( z_{11} - z_{22} ) \sigma_3
\right) \\
\gpgradetwo{P}
&= \inv{2} \left(
\Imag( z_{12} + z_{21} ) i\sigma_1
+\Imag( iz_{12} - iz_{21} ) i\sigma_2
+\Imag( z_{11} - z_{22} ) i\sigma_k
\right) \\
\gpgradethree{P} &= \inv{2} \Imag( z_{11} + z_{22} ) iI
\end{aligned}
\end{equation}

\subsection{Reverse operation}

The reversal operation switches the order of the product of perpendicular vectors.  This will change the sign of grade two and three terms in the Pauli algebra.  Since \(\sigma_2\) is imaginary, conjugation does not have the desired effect, but Hermitian conjugation (conjugate transpose) does the trick.

Since the reverse operation can be written as Hermitian conjugation, one can also define the anticommutator and commutator in terms of reversion in a way that seems particularly natural for complex matrices.  That is

\begin{equation}\label{eqn:pauliMatrix:1140}
\begin{aligned}
\symmetric{a}{b} &= ab + (ab)^\conj \\
\antisymmetric{a}{b} &= ab - (ab)^\conj \\
\end{aligned}
\end{equation}

\subsection{Rotations}

Rotations take the normal Clifford, dual sided quaterionic form.  A rotation about a unit normal \(n\) will be

\begin{equation}\label{eqn:pauliMatrix:1160}
\begin{aligned}
R(x) = e^{-i n \theta/2} x e^{i n \theta/2}
\end{aligned}
\end{equation}

The Rotor \(R = e^{-i n \theta/2}\) commutes with any component of the vector x that is parallel to the normal (perpendicular to the plane), whereas it anticommutes with the components
in the plane.  Writing the vector components perpendicular and parallel to the plane respectively as \(x = x_\perp + x_\parallel\), the essence of the rotation action is this selective commutation or anti-commutation behavior

\begin{equation}\label{eqn:pauliMatrix:1180}
\begin{aligned}
R x_\parallel R^\conj &= x_\parallel R^\conj \\
R x_\perp R^\conj &= x_\perp R R^\conj = x_\perp
\end{aligned}
\end{equation}

Here the exponential has the obvious meaning in terms of exponential series, so for this bivector case we have

\begin{equation}\label{eqn:pauliMatrix:1200}
\begin{aligned}
\exp(i\ncap\theta/2) &= \cos(\theta/2) I + i\ncap \sin(\theta/2)
\end{aligned}
\end{equation}

The unit bivector \(B = i\ncap\) can also be defined explicitly in terms of two vectors \(a\), and \(b\) in the plane

\begin{equation}\label{eqn:pauliMatrix:1220}
\begin{aligned}
B = \inv{\Abs{\antisymmetric{a}{b}}} \antisymmetric{a}{b}
\end{aligned}
\end{equation}

Where the bivector length is defined in terms of the conjugate square (bivector times bivector reverse)

\begin{equation}\label{eqn:pauliMatrix:1240}
\begin{aligned}
\Abs{\antisymmetric{a}{b}}^2 = \antisymmetric{a}{b} {\antisymmetric{a}{b}}^\conj
\end{aligned}
\end{equation}

Examples to complete this subsection would make sense.  As one of the most powerful and useful operations in the algebra, it is a shame in terms of completeness to skimp on this.  However, except for some minor differences like substitution of the Hermitian conjugate operation for reversal, the use of the identity matrix \(I\) in place of the scalar in the exponential expansion, the treatment is exactly the same as in the Clifford algebra.

\subsection{Grade selection}

Coordinate equations for grade selection were worked out above, but the observation that reversion and Hermitian conjugation are isomorphic operations can partially clean this up.  In particular a Hermitian conjugate symmetrization and anti-symmetrization of the general matrix provides a nice split into quaternion and dual quaternion parts (say \(P = Q + R\) respectively).  That is

\begin{equation}\label{eqn:pauliMatrix:1260}
\begin{aligned}
Q &= \gpgradezero{P} + \gpgradeone{P} = \inv{2}(P + P^\conj) \\
R &= \gpgradetwo{P} + \gpgradethree{P} = \inv{2}(P - P^\conj)
\end{aligned}
\end{equation}

Now, having done that, how to determine \(\gpgradezero{Q}\), \(\gpgradeone{Q}\), \(\gpgradetwo{R}\), and \(\gpgradethree{R}\) becomes the next question.  Once that is done, the individual coordinates can be picked off easily enough.  For the vector parts, a Fourier decomposition as in equation \eqnref{eqn:pauliMat:fourier} will retrieve the desired coordinates.

The dual vector coordinates can be picked off easily enough taking dot products with the dual basis vectors

\begin{equation}\label{eqn:pauliMat:dualfourier}
\begin{aligned}
B &= B^k i \sigma_k = \sum_k \inv{2}\symmetric{B}{\inv{i \sigma_k}} i\sigma_k \\
B^k &= \inv{2}\symmetric{B}{\inv{i\sigma_k}}
\end{aligned}
\end{equation}

For the quaternion part \(Q\) the aim is to figure out how to isolate or subtract out the scalar part.  This is the only tricky bit because the diagonal bits are all mixed up with the \(\sigma_3\) term which is also real, and diagonal.  Consideration of the sum

\begin{equation}\label{eqn:pauliMatrix:1280}
\begin{aligned}
a I + b \sigma_3 =
\begin{bmatrix}
a + b & 0 \\
0 & a - b \\
\end{bmatrix},
\end{aligned}
\end{equation}

shows that trace will recover the value \(2a\), so we have

\begin{equation}\label{eqn:pauliMatrix:1300}
\begin{aligned}
\gpgradezero{Q} &= \inv{2}\traceB{Q} I \\
\gpgradeone{Q} &= Q - \inv{2}\traceB{Q} I.
\end{aligned}
\end{equation}

Next is isolation of the \textAndIndex{pseudoscalar} part of the dual quaternion \(R\).  As with the scalar part, consideration of the sum of the \(i\sigma_3\) term and the \(iI\) term is required

\begin{equation}\label{eqn:pauliMatrix:1320}
\begin{aligned}
ia I + ib \sigma_3 =
\begin{bmatrix}
ia + ib & 0 \\
0 & ia - ib \\
\end{bmatrix},
\end{aligned}
\end{equation}

So the trace of the dual quaternion provides the \(2a\), leaving the bivector and pseudoscalar grade split

\begin{equation}\label{eqn:pauliMatrix:1340}
\begin{aligned}
\gpgradethree{R} &= \inv{2}\traceB{R} I \\
\gpgradetwo{R} &= R - \inv{2}\traceB{R} I.
\end{aligned}
\end{equation}

A final assembly of these results provides the following coordinate free grade selection operators

\begin{equation}\label{eqn:pauliMatrix:1360}
\begin{aligned}
\gpgradezero{P} &= \inv{4}\traceB{P + P^\conj} I \\
\gpgradeone{P} &= \inv{2} (P + P^\conj) - \inv{4}\traceB{P + P^\conj} I \\
\gpgradetwo{P} &= \inv{2} (P - P^\conj) - \inv{4}\traceB{P - P^\conj} I \\
\gpgradethree{P} &= \inv{4}\traceB{P - P^\conj} I
\end{aligned}
\end{equation}

\subsection{Generalized dot products}

Here the equivalent of the generalized Clifford bivector/vector dot product will be computed, as well as the associated distribution equation

\begin{equation}\label{eqn:pauliMatrix:1380}
\begin{aligned}
( a \wedge b ) \cdot c &= a (b \cdot c) - b (a \cdot c)
\end{aligned}
\end{equation}

To translate this write

\begin{equation}\label{eqn:pauliMat:cliffordBivectorVectorDot}
\begin{aligned}
( a \wedge b ) \cdot c &= \inv{2} \left( (a \wedge b) c - c (a \wedge b) \right)
\end{aligned}
\end{equation}

Then with the identifications

\begin{equation}\label{eqn:pauliMatrix:1400}
\begin{aligned}
a \cdot b &\equiv \inv{2} \symmetric{a}{b} \\
a \wedge b &\equiv \inv{2} \antisymmetric{a}{b}
\end{aligned}
\end{equation}

we have

\begin{equation}\label{eqn:pauliMatrix:1420}
\begin{aligned}
( a \wedge b ) \cdot c
&\equiv \inv{4} \antisymmetric{\antisymmetric{a}{b}}{c}  \\
&= \inv{2} \left( a \symmetric{b}{c} - \symmetric{b}{c} a \right)
\end{aligned}
\end{equation}

From this we also get the strictly Pauli algebra identity

\begin{equation}\label{eqn:pauliMatrix:1440}
\begin{aligned}
\antisymmetric{\antisymmetric{a}{b}}{c} &= {2} \left( a \symmetric{b}{c} - \symmetric{b}{c} a \right)
\end{aligned}
\end{equation}

But the geometric meaning of this is unfortunately somewhat obfuscated by the notation.

\subsection{Generalized dot and wedge product}

The fundamental definitions of dot and wedge products are in terms of grade

\begin{equation}\label{eqn:pauliMatrix:1460}
\begin{aligned}
\gpgrade{A}{r} \cdot \gpgrade{B}{s} = \gpgrade{AB}{{\Abs{r-s}}}
\end{aligned}
\end{equation}
\begin{equation}\label{eqn:pauliMatrix:1480}
\begin{aligned}
\gpgrade{A}{r} \wedge \gpgrade{B}{s} = \gpgrade{AB}{{r+s}}
\end{aligned}
\end{equation}

Use of the trace and Hermitian conjugate split grade selection operations above, we can calculate these for each of the four grades in the Pauli algebra.

\subsubsection{Grade zero}

There are three dot products consider, vector/vector, bivector/bivector, and trivector/trivector.  In each case we want to compute

\begin{equation}\label{eqn:pauliMatrix:1500}
\begin{aligned}
A \cdot B &= \gpgradezero{ A}{ B} \\
&= \inv{4}\traceB{ AB + (AB)^\conj} I \\
&= \inv{4}\traceB{ AB + B^\conj A^\conj} I
\end{aligned}
\end{equation}

For vectors we have \(a^\conj = a\), since the Pauli basis is Hermitian, whereas for bivectors and trivectors we have \(a^\conj = -a\).  Therefore, in all cases where \(A\), and \(B\) have equal grades we have

\begin{equation}\label{eqn:pauliMatrix:1520}
\begin{aligned}
A \cdot B &= \gpgradezero{A}{B} I \\
&= \inv{4}\traceB{ AB + B A} I \\
&= \inv{4}\trace{\symmetric{ A}{ B}} I
\end{aligned}
\end{equation}

\subsubsection{Grade one}

We have two dot products that produce vectors, bivector/vector, and trivector/bivector, and in each case we need to compute

\begin{equation}\label{eqn:pauliMatrix:1540}
\begin{aligned}
\gpgradeone{ A B} &= \inv{2}( AB + (AB)^\conj ) -\inv{4}\traceB{ AB + (AB)^\conj }
\end{aligned}
\end{equation}

For the bivector/vector dot product we have

\begin{equation}\label{eqn:pauliMatrix:1560}
\begin{aligned}
(Ba)^\conj = -a B
\end{aligned}
\end{equation}

For bivector \(B = i b^k \sigma_k\), and vector \(a = a^k \sigma_k\) our symmetric Hermitian sum in coordinates is

\begin{equation}\label{eqn:pauliMatrix:1580}
\begin{aligned}
B a + (Ba)^\conj
&= B a - a B \\
&= i b^k \sigma_k a^m \sigma_m - a^m \sigma_m i b^k \sigma_k \\
\end{aligned}
\end{equation}

Any \(m=k\) terms will vanish, leaving only the bivector terms, which are traceless.  We therefore have

\begin{equation}\label{eqn:pauliMatrix:1600}
\begin{aligned}
B \cdot a &= \gpgradeone{ B a} \\
&= \inv{2}( B a - a B ) \\
&= \inv{2} \antisymmetric{B}{a}.
\end{aligned}
\end{equation}

This result was borrowed without motivation from Clifford algebra in equation \eqnref{eqn:pauliMat:cliffordBivectorVectorDot}, and thus not satisfactory in terms of a logically derived sequence.

For a trivector \(T\) dotted with bivector \(B\) we have

\begin{equation}\label{eqn:pauliMatrix:1620}
\begin{aligned}
(BT)^\conj = (-T)(-B) = TB = BT.
\end{aligned}
\end{equation}

This is also traceless, and the trivector/bivector dot product is therefore reduced to just

\begin{equation}\label{eqn:pauliMatrix:1640}
\begin{aligned}
B \cdot T &= \gpgradeone{ B T} \\
&= \inv{2} \symmetric{B}{T} \\
&= {B}{T} \\
&= T B.
\end{aligned}
\end{equation}

This is the duality relationship for bivectors.  Multiplication by the unit pseudoscalar (or any multiple of it), produces a vector, the dual of the original bivector.

\subsubsection{Grade two}

We have two products that produce a grade two term, the vector wedge product, and the vector/trivector dot product.  For either case we must compute

\begin{equation}\label{eqn:pauliMat:gradeTwo}
\begin{aligned}
\gpgradetwo{ A B} &= \inv{2}( AB - (AB)^\conj ) -\inv{4}\traceB{ AB - (AB)^\conj }
\end{aligned}
\end{equation}

For a vector \(a\), and trivector \(T\) we need the antisymmetric Hermitian sum

\begin{equation}\label{eqn:pauliMatrix:1660}
\begin{aligned}
a T - (a T)^\conj &= a T + T a = 2 a T = 2 T a
\end{aligned}
\end{equation}

This is a pure bivector, and thus traceless, leaving just

\begin{equation}\label{eqn:pauliMatrix:1680}
\begin{aligned}
a \cdot T  &= \gpgradetwo{ a T} \\
&= a T \\
&= T a
\end{aligned}
\end{equation}

Again we have the duality relation, pseudoscalar multiplication with a vector produces a bivector, and is equivalent to the dot product of the two.

Now for the wedge product case, with vector \(a = a^m \sigma_m\), and \(b = b^k \sigma_k\) we must compute

\begin{equation}\label{eqn:pauliMatrix:1700}
\begin{aligned}
a b - (a b)^\conj
&= a b - b a \\
&= a^m \sigma_m b^k \sigma_k - b^k \sigma_k a^m \sigma_m
\end{aligned}
\end{equation}

All the \(m = n\) terms vanish, leaving a pure bivector which is traceless, so only the first term of \eqnref{eqn:pauliMat:gradeTwo} is relevant, and is in this case a commutator

\begin{equation}\label{eqn:pauliMatrix:1720}
\begin{aligned}
a \wedge b  &= \gpgradetwo{ a b} \\
&= \inv{2} \antisymmetric{ a}{b}
\end{aligned}
\end{equation}

\subsubsection{Grade three}

There are two ways we can produce a grade three term in the algebra.  One is a wedge of a vector with a bivector, and the other is the wedge product of three vectors.  The triple wedge product is the grade three term of the product of the three

\begin{equation}\label{eqn:pauliMatrix:1740}
\begin{aligned}
a \wedge b \wedge c
&= \gpgradethree{ a b c} \\
&= \inv{4} \traceB{ a b c - (a b c)^\conj} \\
&= \inv{4} \traceB{ a b c - c b a} \\
\end{aligned}
\end{equation}

With a split of the \(bc\) and \(cb\) terms into symmetric and antisymmetric terms we have

\begin{equation}\label{eqn:pauliMatrix:1760}
\begin{aligned}
a b c - c b a
&=
 \inv{2} (a \symmetric{b}{c} - \symmetric{c}{b} a)
+\inv{2} (a \antisymmetric{b}{c} - \antisymmetric{c}{b} a)
\end{aligned}
\end{equation}

The symmetric term is diagonal so it commutes (equivalent to scalar commutation with a vector in Clifford algebra), and this therefore vanishes.  Writing \(B = b \wedge c = \inv{2}\antisymmetric{b}{c}\), and noting that \(\antisymmetric{b}{c} = -\antisymmetric{c}{b}\) we therefore have

\begin{equation}\label{eqn:pauliMatrix:1780}
\begin{aligned}
a \wedge B
&= \gpgradethree{ a B} \\
&= \inv{4} \traceB{ a B + B a} \\
&= \inv{4} \trace{ \symmetric{a}{B}} \\
\end{aligned}
\end{equation}

In terms of the original three vectors this is

\begin{equation}\label{eqn:pauliMatrix:1800}
\begin{aligned}
a \wedge b \wedge c
&= \gpgradethree{ a B} \\
&= \inv{8} \trace{ \symmetricVecBladePauli{a}{b}{c}}.
%&= \inv{8} \trace{ \symmetricVecBladePauli{a}{b}{c}}.
\end{aligned}
\end{equation}

Since this could have been expanded by grouping \(ab\) instead of \(bc\) we also have

\begin{equation}\label{eqn:pauliMatrix:1820}
\begin{aligned}
a \wedge b \wedge c
&= \inv{8} \trace{ \symmetricBladeVecPauli{a}{b}{c}}.
\end{aligned}
\end{equation}
%symmetricBladeVecPauli

\subsection{Plane projection and rejection}

Projection of a vector onto a plane follows like the vector projection case.  In the Pauli notation this is

\begin{equation}\label{eqn:pauliMatrix:1840}
\begin{aligned}
x
&= x B \inv{B} \\
&= \inv{2} \symmetric{x}{B} \inv{B} + \inv{2} \antisymmetric{x}{B} \inv{B} \\
\end{aligned}
\end{equation}

Here the plane is a bivector, so if vectors \(a\), and \(b\) are in the plane, the orientation and attitude can be represented
by the commutator

%\newcommand{\symmetricVecBladePauli}[3]{\left\{{#1},\left[{#2},{#3}\right]\right\}}

So we have
\begin{equation}\label{eqn:pauliMatrix:1860}
\begin{aligned}
x
%&= \inv{2} \symmetricVecBladePauli{x}{a}{b} \inv{\antisymmetric{a}{b}} + \inv{2} \antisymmetric{x}{\antisymmetric{a}{b}} \inv{\antisymmetric{a}{b}} \\
&= \inv{2} \symmetricVecBladePauli{x}{a}{b} \inv{\antisymmetric{a}{b}} + \inv{2} \antisymmetric{x}{\antisymmetric{a}{b}} \inv{\antisymmetric{a}{b}} \\
\end{aligned}
\end{equation}

Of these the second term is our projection onto the plane, while the first is the normal component of the vector.

\section{Examples}

\subsection{Radial decomposition}

\subsubsection{Velocity and momentum}

A decomposition of velocity into radial and perpendicular components should be straightforward in the Pauli algebra as it is in the Clifford algebra.

With a radially expressed position vector

\begin{equation}\label{eqn:pauliMatrix:1880}
\begin{aligned}
x = \Abs{x} \hatx,
\end{aligned}
\end{equation}

velocity can be written by taking derivatives
\begin{equation}\label{eqn:pauliMatrix:1900}
\begin{aligned}
v = x' = \Abs{x}' \hatx + \Abs{x} \hatx'
\end{aligned}
\end{equation}

or as above in the projection calculation with
\begin{equation}\label{eqn:pauliMatrix:1920}
\begin{aligned}
v
&= v \inv{x} x \\
&= \inv{2}\symmetric{v}{\inv{x}} x + \inv{2}\antisymmetric{v}{\inv{x}} x \\
&= \inv{2}\symmetric{v}{\hatx} \hatx + \inv{2}\antisymmetric{v}{\hatx} \hatx
\end{aligned}
\end{equation}

By comparison we have

\begin{equation}\label{eqn:pauliMatrix:1940}
\begin{aligned}
\Abs{x}' &= \inv{2}\symmetric{v}{\hatx} \\
\hatx' &= \inv{2 \Abs{x}} \antisymmetric{v}{\hatx} \hatx
\end{aligned}
\end{equation}

In assembled form we have

\begin{equation}\label{eqn:pauliMatrix:1960}
\begin{aligned}
v &= \inv{2}\symmetric{v}{\hatx} \hatx + x \omega \\
\end{aligned}
\end{equation}

Here the commutator has been identified with the angular velocity bivector \(\omega\)

\begin{equation}\label{eqn:pauliMatrix:1980}
\begin{aligned}
\omega &= \inv{2 x^2}\antisymmetric{x}{v}.
\end{aligned}
\end{equation}

Similarly, the linear and angular momentum split of a momentum vector is

\begin{equation}\label{eqn:pauliMatrix:2000}
\begin{aligned}
p_\parallel &= \inv{2}\symmetric{p}{\hatx} \hatx \\
p_\perp &= \inv{2}\antisymmetric{p}{\hatx} \hatx
\end{aligned}
\end{equation}

and in vector form

\begin{equation}\label{eqn:pauliMatrix:2020}
\begin{aligned}
p &= \inv{2}\symmetric{p}{\hatx} \hatx + m x \omega \\
\end{aligned}
\end{equation}

Writing \(J = m x^2\) for the moment of inertia we have for our commutator

\begin{equation}\label{eqn:pauliMatrix:2040}
\begin{aligned}
L = \inv{2}\antisymmetric{x}{p} = m x^2 \omega = J \omega
\end{aligned}
\end{equation}

With the identification of the commutator with the angular momentum bivector \(L\) we have the total momentum as

\begin{equation}\label{eqn:pauliMatrix:2060}
\begin{aligned}
p &= \inv{2}\symmetric{p}{\hatx} \hatx + \inv{x} L \\
\end{aligned}
\end{equation}

\subsubsection{Acceleration and force}

Having computed velocity, and its radial split, the next logical thing to try is acceleration.

The acceleration will be

\begin{equation}\label{eqn:pauliMatrix:2080}
\begin{aligned}
a = v' = \Abs{x}'' \hatx + 2 \Abs{x}' \hatx' + \Abs{x} \hatx''
\end{aligned}
\end{equation}

We need to compute \(\hatx''\) to continue, which is

\begin{equation}\label{eqn:pauliMatrix:2100}
\begin{aligned}
\hatx''
&= \left(\frac{1}{2 \Abs{x}^3} \antisymmetric{v}{x} x \right)' \\
&=
\frac{-3}{2 \Abs{x}^4} \Abs{x}' \antisymmetric{v}{x} x
+ \frac{1}{2 \Abs{x}^3} \antisymmetric{a}{x} x
+ \frac{1}{2 \Abs{x}^3} \antisymmetric{v}{x} v \\
&=
\frac{-3}{4 \Abs{x}^5} \symmetric{v}{x} \antisymmetric{v}{x} x
+ \frac{1}{2 \Abs{x}^3} \antisymmetric{a}{x} x
+ \frac{1}{2 \Abs{x}^3} \antisymmetric{v}{x} v
\end{aligned}
\end{equation}

Putting things back together is a bit messy, but starting so gives

\begin{equation}\label{eqn:pauliMatrix:2120}
\begin{aligned}
a
&= \Abs{x}'' \hatx + 2
%\Abs{x}'
%\hatx'
\inv{4 \Abs{x}^4} \symmetric{v}{x} \antisymmetric{v}{x} x
% + \Abs{x} \hatx''
+\frac{-3}{4 \Abs{x}^4} \symmetric{v}{x} \antisymmetric{v}{x} x
+ \frac{1}{2 \Abs{x}^2} \antisymmetric{a}{x} x
+ \frac{1}{2 \Abs{x}^2} \antisymmetric{v}{x} v \\
&=
  \Abs{x}'' \hatx
- \frac{1}{4 \Abs{x}^4} \symmetric{v}{x} \antisymmetric{v}{x} x
+ \frac{1}{2 \Abs{x}^2} \antisymmetric{a}{x} x
+ \frac{1}{2 \Abs{x}^2} \antisymmetric{v}{x} v \\
&=
  \Abs{x}'' \hatx
+ \inv{4 \Abs{x}^4} \antisymmetric{v}{x} \left( -\symmetric{v}{x} x + 2 x^2 v \right)
+ \frac{1}{2 \Abs{x}^2} \antisymmetric{a}{x} x
\end{aligned}
\end{equation}

The anticommutator can be eliminated above using

\begin{equation}\label{eqn:pauliMatrix:2140}
\begin{aligned}
v x &= \inv{2} \symmetric{v}{x} + \inv{2}\antisymmetric{v}{x} \\
\implies \\
-\symmetric{v}{x} x + 2 x^2 v
&= -(2 v x - \antisymmetric{v}{x}) x + 2 x^2 v \\
&= \antisymmetric{v}{x} x
\end{aligned}
\end{equation}

Finally reassembly of the assembly is thus

\begin{equation}\label{eqn:pauliMatrix:2160}
\begin{aligned}
a
&= \Abs{x}'' \hatx
+ \inv{4 \Abs{x}^4} \antisymmetric{v}{x}^2 x
+ \frac{1}{2 \Abs{x}^2} \antisymmetric{a}{x} x \\
&= \Abs{x}'' \hatx
+ \omega^2 x
+ \frac{1}{2} \antisymmetric{a}{x} \inv{x} \\
\end{aligned}
\end{equation}

The second term is the inwards facing radially directed acceleration, while the last is the rejective component of the acceleration.

It is usual to express this last term as the rate of change of angular momentum (torque).  Because \(\antisymmetric{v}{v} = 0\), we have

\begin{equation}\label{eqn:pauliMatrix:2180}
\begin{aligned}
\frac{d \antisymmetric{x}{v}}{dt} = \antisymmetric{x}{a}
\end{aligned}
\end{equation}

So, for constant mass, we can write the torque as

\begin{equation}\label{eqn:pauliMatrix:2200}
\begin{aligned}
\tau
&= \frac{d}{dt} \left( \inv{2} \antisymmetric{x}{p} \right) \\
&= \frac{dL}{dt}
\end{aligned}
\end{equation}

and finally have for the force

\begin{equation}\label{eqn:pauliMatrix:2220}
\begin{aligned}
F
&= m \Abs{x}'' \hatx + m \omega^2 x + \inv{x} \frac{dL}{dt} \\
&= m \left(\Abs{x}'' - \frac{\Abs{\omega^2}}{\Abs{x}} \right) \hatx + \inv{x} \frac{dL}{dt}
\end{aligned}
\end{equation}

\section{Conclusion}

Although many of the GA references that can be found downplay the Pauli algebra as unnecessarily employing matrices as a basis, I believe this shows that there are some nice computational and logical niceties in the complete formulation of the \R{3} Clifford algebra in this complex matrix formulation.  If nothing else it takes some of the abstraction away, which is good for developing intuition.  All of the generalized dot and wedge product relationships are easily derived showing specific examples of the general pattern for the dot and blade product equations which are sometimes supplied as definitions instead of consequences.

Also, the matrix concepts (if presented right which I likely have not done) should also be accessible to most anybody out of high school these days since both matrix algebra and complex numbers are covered as basics these days (at least that is how I recall it from fifteen years back;)

Hopefully, having gone through the exercise of examining all the equivalent constructions will be useful in subsequent Quantum physics study to see how the matrix algebra that is used in that subject is tied to the classical geometrical vector constructions.

Expressions that were scary and mysterious looking like

\begin{equation}\label{eqn:pauliMatrix:2240}
\begin{aligned}
\antisymmetric{L_x}{L_y} = i \Hbar L_z
\end{aligned}
\end{equation}

are no longer so bad since some of the geometric meaning that backs this operator expression is now clear (this is a quantization of angular momentum in a specific plane, and encodes the plane orientation as well as the magnitude).  Knowing that \(\antisymmetric{a}{b}\) was an antisymmetric sum, but not realizing the connection between that and the wedge product previously made me wonder ``where the hell did the i come from''?

This commutator equation is logically and geometrically a plane operation.  It can therefore be expressed with a vector duality relationship employing the \R{3} unit pseudoscalar \(iI = \sigma_1 \sigma_2 \sigma_3\).  This is a good nice step towards taking some of the mystery out of the math behind the physics of the subject (which has enough intrinsic mystery without the mathematical language adding to it).

It is unfortunate that QM uses this matrix operator formulation and none of classical physics does.  By the time one gets to QM learning an entirely new language is required despite the fact that there are many powerful applications of this algebra in the classical domain, not just for rotations which is recognized (in \citep{goldstein1951cm} for example where he uses the Pauli algebra to express his rotation quaternions.)

%\EndArticle

%
% Copyright � 2012 Peeter Joot.  All Rights Reserved.
% Licenced as described in the file LICENSE under the root directory of this GIT repository.
%

% 
% 
\chapter{Gamma Matrices}\label{chap:PJDiracGamma}
\index{gamma matrices}
%\date{Dec 13, 2008.  gamma.tex}

\section{Dirac matrices}

\index{Dirac!matrix}
\index{gamma matrix}
The Dirac matrices \(\gamma^\mu\) can be used as a Minkowski basis.  The basic defining relationship is the Minkowski metric, where the dot products satisfy

\begin{equation}\label{eqn:gamma:20}
\begin{aligned}
\scalarProduct{\gamma^\mu}{\gamma^\nu} &= \pm \delta_{\mu\nu} \\
(\scalarProduct{\gamma^0}{\gamma^0})(\scalarProduct{\gamma^a}{\gamma^a}) &= -1 \quad \text{where \(a \in \{1,2,3\}\)}
\end{aligned}
\end{equation}

There is freedom to pick the positive square for either \(\gamma^0\) or \(\gamma^a\), and both conventions are common.

One of the matrix representations for these vectors listed in the 
\href{http://en.wikipedia.org/wiki/Gamma_matrices}{Dirac matrix wikipedia article}
is

\begin{equation}\label{eqn:gamma:basis}
\begin{aligned}
\gamma^0 &= \begin{bmatrix}
 1  &  0  &  0  &  0  \\
 0  &  1  &  0  &  0  \\
 0  &  0  &  -1  &  0  \\
 0  &  0  &  0  &  -1  \\
\end{bmatrix} \quad
\gamma^1 = \begin{bmatrix}
 0  &  0  &  0  &  1  \\
 0  &  0  &  1  &  0  \\
 0  &  -1  &  0  &  0  \\
 -1  &  0  &  0  &  0  \\
\end{bmatrix} \\
\gamma^2 &= \begin{bmatrix}
 0  &  0  &  0  &  -i  \\
 0  &  0  &  i  &  0  \\
 0  &  i  &  0  &  0  \\
 -i  &  0  &  0  &  0  \\
\end{bmatrix}
\quad \gamma^3 = \begin{bmatrix}
 0  &  0  &  1  &  0  \\
 0  &  0  &  0  &  -1  \\
 -1  &  0  &  0  &  0  \\
 0  &  1  &  0  &  0  \\
\end{bmatrix}
\end{aligned}
\end{equation}

For this particular basis we have a \(+---\) metric signature.  In the matrix form this takes the specific meaning that \((\gamma^0)^2 = I\), and \((\gamma^a)^2 = -I\).

A table of all the possible product variants of \eqnref{eqn:gamma:basis} can be found below in the appendix.

\subsection{anticommutator product}
\index{anticommutator}

Noting that the matrices square in the fashion just described and that they reverse sign when multiplication order is reversed allows for summarizing the dot products relationships as follows

\begin{equation}\label{eqn:gamma:symmetric}
\begin{aligned}
\symmetric{\gamma^\mu}{\gamma^\nu} 
&= {\gamma^\mu}{\gamma^\nu} + {\gamma^\nu}{\gamma^\mu} \\
%&= 2 (\scalarProduct{\gamma^\mu}{\gamma^\nu}) I \\
&= 2 \eta^{\mu\nu} I,
\end{aligned}
\end{equation}

where the metric tensor \(\eta^{\mu\nu} = \scalarProduct{\gamma^\mu}{\gamma^\nu}\) is commonly summarized as coordinates of a matrix as in

\begin{equation}\label{eqn:gamma:40}
\begin{aligned}
\begin{bmatrix}
\eta^{\mu\nu}
\end{bmatrix}
&=
\begin{bmatrix}
1 & 0 & 0 & 0 \\
0 & -1 & 0 & 0 \\
0 & 0 & -1 & 0 \\
0 & 0 & 0 & -1 \\
\end{bmatrix}
\end{aligned}
\end{equation}

The relationship \eqnref{eqn:gamma:symmetric} is taken as the defining relationship for the Dirac matrices, but can be seen to be just a matricized statement of the Clifford vector dot product.

\subsection{Written as Pauli matrices}
\index{Pauli matrices}

Using the Pauli matrices

\begin{equation}\label{eqn:gamma:60}
\begin{aligned}
\sigma_1 = \PauliX \quad \sigma_2 = \PauliY \quad \sigma_3 = \PauliZ
\end{aligned}
\end{equation}

one can write the Dirac matrices and all their products (reading from the multiplication table) more concisely as

\begin{equation}\label{eqn:gamma:80}
\begin{aligned}
\gamma^0 &= 
\begin{bmatrix}
I & 0 \\
0 & -I
\end{bmatrix} \\
\gamma^a &= 
\begin{bmatrix}
0 & \sigma_a \\
-\sigma_a & 0 \\
\end{bmatrix} \\
\gamma^0 \gamma^a &=
\begin{bmatrix}
0 & \sigma_a \\
\sigma_a & 0 \\
\end{bmatrix} \\
\gamma^a \gamma^b &=
- i \epsilon_{a b c}
\begin{bmatrix}
\sigma_c & 0 \\
0 & \sigma_c \\
\end{bmatrix} \\
\gamma^1 \gamma^2 \gamma^3 &= i 
\begin{bmatrix}
0 & -I \\
I & 0
\end{bmatrix} \\
\gamma^0 \gamma^1 \gamma^2 &= i 
\begin{bmatrix}
-\sigma_1 & 0 \\
0 & \sigma_1 \\
\end{bmatrix} \\
\gamma^3 \gamma^0 \gamma^1 &= i 
\begin{bmatrix}
\sigma_2 & 0 \\
0 & -\sigma_2 \\
\end{bmatrix} \\
\gamma^0 \gamma^1 \gamma^2 &= i 
\begin{bmatrix}
-\sigma_3 & 0 \\
0 & \sigma_3 \\
\end{bmatrix}
\end{aligned}
\end{equation}

\subsection{Deriving properties using the Pauli matrices}

From the multiplication table a number of properties can be observed.  Using the Pauli matrices one can arrive at these more directly using the multiplication identity for those
matrices

\begin{equation}\label{eqn:gamma:100}
\begin{aligned}
\sigma_a \sigma_b = 2 i \epsilon_{abc} \sigma_c
\end{aligned}
\end{equation}

Actually taking the time to type this out in full does not seem worthwhile and is a fairly straightforward exercise.

\subsection{Conjugation behavior}
\index{conjugation}

Unlike the Pauli matrices, the Dirac matrices do not split nicely via conjugation.  Instead we have the time basis vector and its dual are Hermitian

\begin{equation}\label{eqn:gamma:120}
\begin{aligned}
(\gamma^0)^\conj &= \gamma^0 \\
(\gamma^1 \gamma^2 \gamma^3)^\conj &= \gamma^1 \gamma^2 \gamma^3
\end{aligned}
\end{equation}

whereas the spacelike basis vectors and their duals are all anti-Hermitian

\begin{equation}\label{eqn:gamma:140}
\begin{aligned}
(\gamma^a)^\conj &= -\gamma^a \\
(\gamma^a \gamma^b \gamma^c)^\conj &= - \gamma^a \gamma^b \gamma^c.
\end{aligned}
\end{equation}

For the scalar and the pseudoscalar parts we have a Hermitian split

\begin{equation}\label{eqn:gamma:160}
\begin{aligned}
I^\conj &= I \\
(\gamma^0 \gamma^1 \gamma^2 \gamma^3)^\conj &= -(\gamma^0 \gamma^1 \gamma^2 \gamma^3)^\conj
\end{aligned}
\end{equation}

and finally, also have a Hermitian split of the bivector parts into spacetime (relative vectors), and the purely spatial bivectors

\begin{equation}\label{eqn:gamma:180}
\begin{aligned}
(\gamma^0 \gamma^a)^\conj &= \gamma^0 \gamma^a \\
(\gamma^a \gamma^b)^\conj &= -\gamma^a \gamma^b
\end{aligned}
\end{equation}

Is there a logical and simple set of matrix operations that splits things nicely into scalar, vector, bivector, trivector, and pseudoscalar parts as there was with the Pauli
matrices?

\section{Appendix.  Table of all generated products}

A small C++ program using boost::numeric::ublas and std::complex,
plus some perl to generate part of that, was
written to generate the multiplication table for the gamma matrix products
for this particular basis.  The metric tensor and the antisymmetry of
the wedge products can be seen from these.

%% <GENERATED>




\begin{equation}\label{eqn:gamma:200}
\begin{aligned}
\gamma^0 \gamma^0 = \begin{bmatrix}
 1  &  0  &  0  &  0  \\
 0  &  1  &  0  &  0  \\
 0  &  0  &  1  &  0  \\
 0  &  0  &  0  &  1  \\
\end{bmatrix} \quad
\gamma^1 \gamma^1 = \begin{bmatrix}
 -1  &  0  &  0  &  0  \\
 0  &  -1  &  0  &  0  \\
 0  &  0  &  -1  &  0  \\
 0  &  0  &  0  &  -1  \\
\end{bmatrix}
\end{aligned}
\end{equation}

\begin{equation}\label{eqn:gamma:220}
\begin{aligned}
\gamma^2 \gamma^2 = \begin{bmatrix}
 -1  &  0  &  0  &  0  \\
 0  &  -1  &  0  &  0  \\
 0  &  0  &  -1  &  0  \\
 0  &  0  &  0  &  -1  \\
\end{bmatrix} \quad
\gamma^3 \gamma^3 = \begin{bmatrix}
 -1  &  0  &  0  &  0  \\
 0  &  -1  &  0  &  0  \\
 0  &  0  &  -1  &  0  \\
 0  &  0  &  0  &  -1  \\
\end{bmatrix}
\end{aligned}
\end{equation}

\begin{equation}\label{eqn:gamma:240}
\begin{aligned}
\gamma^0 \gamma^1 = \begin{bmatrix}
 0  &  0  &  0  &  1  \\
 0  &  0  &  1  &  0  \\
 0  &  1  &  0  &  0  \\
 1  &  0  &  0  &  0  \\
\end{bmatrix} \quad
\gamma^1 \gamma^0 = \begin{bmatrix}
 0  &  0  &  0  &  -1  \\
 0  &  0  &  -1  &  0  \\
 0  &  -1  &  0  &  0  \\
 -1  &  0  &  0  &  0  \\
\end{bmatrix}
\end{aligned}
\end{equation}

\begin{equation}\label{eqn:gamma:260}
\begin{aligned}
\gamma^0 \gamma^2 = \begin{bmatrix}
 0  &  0  &  0  &  -i  \\
 0  &  0  &  i  &  0  \\
 0  &  -i  &  0  &  0  \\
 i  &  0  &  0  &  0  \\
\end{bmatrix} \quad
\gamma^2 \gamma^0 = \begin{bmatrix}
 0  &  0  &  0  &  i  \\
 0  &  0  &  -i  &  0  \\
 0  &  i  &  0  &  0  \\
 -i  &  0  &  0  &  0  \\
\end{bmatrix}
\end{aligned}
\end{equation}

\begin{equation}\label{eqn:gamma:280}
\begin{aligned}
\gamma^0 \gamma^3 = \begin{bmatrix}
 0  &  0  &  1  &  0  \\
 0  &  0  &  0  &  -1  \\
 1  &  0  &  0  &  0  \\
 0  &  -1  &  0  &  0  \\
\end{bmatrix} \quad
\gamma^3 \gamma^0 = \begin{bmatrix}
 0  &  0  &  -1  &  0  \\
 0  &  0  &  0  &  1  \\
 -1  &  0  &  0  &  0  \\
 0  &  1  &  0  &  0  \\
\end{bmatrix}
\end{aligned}
\end{equation}

\begin{equation}\label{eqn:gamma:300}
\begin{aligned}
\gamma^1 \gamma^2 = \begin{bmatrix}
 -i  &  0  &  0  &  0  \\
 0  &  i  &  0  &  0  \\
 0  &  0  &  -i  &  0  \\
 0  &  0  &  0  &  i  \\
\end{bmatrix} \quad
\gamma^2 \gamma^1 = \begin{bmatrix}
 i  &  0  &  0  &  0  \\
 0  &  -i  &  0  &  0  \\
 0  &  0  &  i  &  0  \\
 0  &  0  &  0  &  -i  \\
\end{bmatrix}
\end{aligned}
\end{equation}

\begin{equation}\label{eqn:gamma:320}
\begin{aligned}
\gamma^1 \gamma^3 = \begin{bmatrix}
 0  &  1  &  0  &  0  \\
 -1  &  0  &  0  &  0  \\
 0  &  0  &  0  &  1  \\
 0  &  0  &  -1  &  0  \\
\end{bmatrix} \quad
\gamma^3 \gamma^1 = \begin{bmatrix}
 0  &  -1  &  0  &  0  \\
 1  &  0  &  0  &  0  \\
 0  &  0  &  0  &  -1  \\
 0  &  0  &  1  &  0  \\
\end{bmatrix}
\end{aligned}
\end{equation}

\begin{equation}\label{eqn:gamma:340}
\begin{aligned}
\gamma^2 \gamma^3 = \begin{bmatrix}
 0  &  -i  &  0  &  0  \\
 -i  &  0  &  0  &  0  \\
 0  &  0  &  0  &  -i  \\
 0  &  0  &  -i  &  0  \\
\end{bmatrix} \quad
\gamma^3 \gamma^2 = \begin{bmatrix}
 0  &  i  &  0  &  0  \\
 i  &  0  &  0  &  0  \\
 0  &  0  &  0  &  i  \\
 0  &  0  &  i  &  0  \\
\end{bmatrix}
\end{aligned}
\end{equation}

\begin{equation}\label{eqn:gamma:360}
\begin{aligned}
\gamma^1 \gamma^2 \gamma^3 = \begin{bmatrix}
 0  &  0  &  -i  &  0  \\
 0  &  0  &  0  &  -i  \\
 i  &  0  &  0  &  0  \\
 0  &  i  &  0  &  0  \\
\end{bmatrix} \quad
\gamma^2 \gamma^3 \gamma^0 = \begin{bmatrix}
 0  &  -i  &  0  &  0  \\
 -i  &  0  &  0  &  0  \\
 0  &  0  &  0  &  i  \\
 0  &  0  &  i  &  0  \\
\end{bmatrix}
\end{aligned}
\end{equation}

\begin{equation}\label{eqn:gamma:380}
\begin{aligned}
\gamma^3 \gamma^0 \gamma^1 = \begin{bmatrix}
 0  &  1  &  0  &  0  \\
 -1  &  0  &  0  &  0  \\
 0  &  0  &  0  &  -1  \\
 0  &  0  &  1  &  0  \\
\end{bmatrix} \quad
\gamma^0 \gamma^1 \gamma^2 = \begin{bmatrix}
 -i  &  0  &  0  &  0  \\
 0  &  i  &  0  &  0  \\
 0  &  0  &  i  &  0  \\
 0  &  0  &  0  &  -i  \\
\end{bmatrix}
\end{aligned}
\end{equation}

\begin{equation}\label{eqn:gamma:400}
\begin{aligned}
\gamma^0 \gamma^1 \gamma^2 \gamma^3 = \begin{bmatrix}
 0  &  0  &  -i  &  0  \\
 0  &  0  &  0  &  -i  \\
 -i  &  0  &  0  &  0  \\
 0  &  -i  &  0  &  0  \\
\end{bmatrix}
\end{aligned}
\end{equation}


%% </GENERATED>

\documentclass[]{eliblog}

\usepackage{amsmath}
\usepackage{mathpazo}

%
% shorthand for bold symbols, convenient for vectors and matrices
%
\newcommand{\Ba}[0]{\mathbf{a}}
\newcommand{\Bb}[0]{\mathbf{b}}
\newcommand{\Bc}[0]{\mathbf{c}}
\newcommand{\Bd}[0]{\mathbf{d}}
\newcommand{\Be}[0]{\mathbf{e}}
\newcommand{\Bf}[0]{\mathbf{f}}
\newcommand{\Bg}[0]{\mathbf{g}}
\newcommand{\Bh}[0]{\mathbf{h}}
\newcommand{\Bi}[0]{\mathbf{i}}
\newcommand{\Bj}[0]{\mathbf{j}}
\newcommand{\Bk}[0]{\mathbf{k}}
\newcommand{\Bl}[0]{\mathbf{l}}
\newcommand{\Bm}[0]{\mathbf{m}}
\newcommand{\Bn}[0]{\mathbf{n}}
\newcommand{\Bo}[0]{\mathbf{o}}
\newcommand{\Bp}[0]{\mathbf{p}}
\newcommand{\Bq}[0]{\mathbf{q}}
\newcommand{\Br}[0]{\mathbf{r}}
\newcommand{\Bs}[0]{\mathbf{s}}
\newcommand{\Bt}[0]{\mathbf{t}}
\newcommand{\Bu}[0]{\mathbf{u}}
\newcommand{\Bv}[0]{\mathbf{v}}
\newcommand{\Bw}[0]{\mathbf{w}}
\newcommand{\Bx}[0]{\mathbf{x}}
\newcommand{\By}[0]{\mathbf{y}}
\newcommand{\Bz}[0]{\mathbf{z}}
\newcommand{\BA}[0]{\mathbf{A}}
\newcommand{\BB}[0]{\mathbf{B}}
\newcommand{\BC}[0]{\mathbf{C}}
\newcommand{\BD}[0]{\mathbf{D}}
\newcommand{\BE}[0]{\mathbf{E}}
\newcommand{\BF}[0]{\mathbf{F}}
\newcommand{\BG}[0]{\mathbf{G}}
\newcommand{\BH}[0]{\mathbf{H}}
\newcommand{\BI}[0]{\mathbf{I}}
\newcommand{\BJ}[0]{\mathbf{J}}
\newcommand{\BK}[0]{\mathbf{K}}
\newcommand{\BL}[0]{\mathbf{L}}
\newcommand{\BM}[0]{\mathbf{M}}
\newcommand{\BN}[0]{\mathbf{N}}
\newcommand{\BO}[0]{\mathbf{O}}
\newcommand{\BP}[0]{\mathbf{P}}
\newcommand{\BQ}[0]{\mathbf{Q}}
\newcommand{\BR}[0]{\mathbf{R}}
\newcommand{\BS}[0]{\mathbf{S}}
\newcommand{\BT}[0]{\mathbf{T}}
\newcommand{\BU}[0]{\mathbf{U}}
\newcommand{\BV}[0]{\mathbf{V}}
\newcommand{\BW}[0]{\mathbf{W}}
\newcommand{\BX}[0]{\mathbf{X}}
\newcommand{\BY}[0]{\mathbf{Y}}
\newcommand{\BZ}[0]{\mathbf{Z}}

\newcommand{\Bzero}[0]{\mathbf{0}}
\newcommand{\Btheta}[0]{\boldsymbol{\theta}}
\newcommand{\Btau}[0]{\boldsymbol{\tau}}
\newcommand{\Bomega}[0]{\boldsymbol{\omega}}

%
% shorthand for unit vectors
%
\newcommand{\acap}[0]{\hat{\Ba}}
\newcommand{\bcap}[0]{\hat{\Bb}}
\newcommand{\ccap}[0]{\hat{\Bc}}
\newcommand{\dcap}[0]{\hat{\Bd}}
\newcommand{\ecap}[0]{\hat{\Be}}
\newcommand{\fcap}[0]{\hat{\Bf}}
\newcommand{\gcap}[0]{\hat{\Bg}}
\newcommand{\hcap}[0]{\hat{\Bh}}
\newcommand{\icap}[0]{\hat{\Bi}}
\newcommand{\jcap}[0]{\hat{\Bj}}
\newcommand{\kcap}[0]{\hat{\Bk}}
\newcommand{\lcap}[0]{\hat{\Bl}}
\newcommand{\mcap}[0]{\hat{\Bm}}
\newcommand{\ncap}[0]{\hat{\Bn}}
\newcommand{\ocap}[0]{\hat{\Bo}}
\newcommand{\pcap}[0]{\hat{\Bp}}
\newcommand{\qcap}[0]{\hat{\Bq}}
\newcommand{\rcap}[0]{\hat{\Br}}
\newcommand{\scap}[0]{\hat{\Bs}}
\newcommand{\tcap}[0]{\hat{\Bt}}
\newcommand{\ucap}[0]{\hat{\Bu}}
\newcommand{\vcap}[0]{\hat{\Bv}}
\newcommand{\wcap}[0]{\hat{\Bw}}
\newcommand{\xcap}[0]{\hat{\Bx}}
\newcommand{\ycap}[0]{\hat{\By}}
\newcommand{\zcap}[0]{\hat{\Bz}}
\newcommand{\thetacap}[0]{\hat{\Btheta}}

%
% to write R^n and C^n in a distinguishable fashion.  Perhaps change this
% to the double lined characters upon figuring out how to do so.
%
\newcommand{\C}[1]{$\mathbb{C}^{#1}$}
\newcommand{\R}[1]{$\mathbb{R}^{#1}$}

%
% various generally useful helpers
%

% derivative of #1 wrt. #2:
\newcommand{\D}[2] {\frac {d#2} {d#1}}

\newcommand{\inv}[1]{\frac{1}{#1}}
\newcommand{\cross}[0]{\times}

\newcommand{\abs}[1]{\lvert{#1}\rvert}
\newcommand{\norm}[1]{\lVert{#1}\rVert}
\newcommand{\innerprod}[2]{\langle{#1}, {#2}\rangle}
\newcommand{\dotprod}[2]{{#1} \cdot {#2}}
\newcommand{\bdotprod}[2]{\left({#1} \cdot {#2}\right)}
\newcommand{\crossprod}[2]{{#1} \cross {#2}}
\newcommand{\tripleprod}[3]{\dotprod{\left(\crossprod{#1}{#2}\right)}{#3}}

\DeclareMathOperator{\Proj}{Proj}
\DeclareMathOperator{\Span}{span}
\DeclareMathOperator{\Sgn}{sgn}
\DeclareMathOperator{\Area}{Area}
\DeclareMathOperator{\Volume}{Volume}

%
% A few miscellaneous things specific to this document
%
\newcommand{\crossop}[1]{\crossprod{#1}{}}

% R2 vector.
\newcommand{\VectorTwo}[2]{
\begin{bmatrix}
 {#1} \\
 {#2}
\end{bmatrix}
}

\newcommand{\VectorN}[1]{
\begin{bmatrix}
{#1}_1 \\
{#1}_2 \\
\vdots \\
{#1}_N \\
\end{bmatrix}
}

\newcommand{\DETuvij}[4]{
\begin{vmatrix}
 {#1}_{#3} & {#1}_{#4} \\
 {#2}_{#3} & {#2}_{#4}
\end{vmatrix}
}

\newcommand{\DETuvwijk}[6]{
\begin{vmatrix}
 {#1}_{#4} & {#1}_{#5} & {#1}_{#6} \\
 {#2}_{#4} & {#2}_{#5} & {#2}_{#6} \\
 {#3}_{#4} & {#3}_{#5} & {#3}_{#6}
\end{vmatrix}
}

\newcommand{\DETuvwxijkl}[8]{
\begin{vmatrix}
 {#1}_{#5} & {#1}_{#6} & {#1}_{#7} & {#1}_{#8} \\
 {#2}_{#5} & {#2}_{#6} & {#2}_{#7} & {#2}_{#8} \\
 {#3}_{#5} & {#3}_{#6} & {#3}_{#7} & {#3}_{#8} \\
 {#4}_{#5} & {#4}_{#6} & {#4}_{#7} & {#4}_{#8} \\
\end{vmatrix}
}

%\newcommand{\DETuvwxyijklm}[10]{
%\begin{vmatrix}
% {#1}_{#6} & {#1}_{#7} & {#1}_{#8} & {#1}_{#9} & {#1}_{#10} \\
% {#2}_{#6} & {#2}_{#7} & {#2}_{#8} & {#2}_{#9} & {#2}_{#10} \\
% {#3}_{#6} & {#3}_{#7} & {#3}_{#8} & {#3}_{#9} & {#3}_{#10} \\
% {#4}_{#6} & {#4}_{#7} & {#4}_{#8} & {#4}_{#9} & {#4}_{#10} \\
% {#5}_{#6} & {#5}_{#7} & {#5}_{#8} & {#5}_{#9} & {#5}_{#10}
%\end{vmatrix}
%}

% R3 vector.
\newcommand{\VectorThree}[3]{
\begin{bmatrix}
 {#1} \\
 {#2} \\
 {#3}
\end{bmatrix}
}



\author{Peeter Joot}
\email{peeter.joot@gmail.com}


\chapter{Bivector form of quantum angular momentum operator}
\label{chap:qmAngularMom}
%\useCCL
\blogpage{http://sites.google.com/site/peeterjoot/math2009/qmAngularMom.pdf}
%\date{July 27, 2009}
%\revisionInfo{$RCSfile: qmAngularMom.tex,v $ Last $Revision: 1.5 $ $Date: 2009/08/25 04:37:55 $}

\date{July 27, 2009.  $RCSfile: qmAngularMom.tex,v $ Last $Revision: 1.5 $ $Date: 2009/08/25 04:37:55 $}

\beginArtWithToc

\section{Spatial bivector representation of the angular momentum operator.}

Reading (\cite{bohm1989qt}) on the angular momentum operator, the form of the operator is suggested by analogy where components of $\Bx \cross \Bp$ with 
the position representation $\Bp \sim -i \hbar \spacegrad$ used to expand the coordinate representation of the operator.

The result is the following coordinate representation of the operator

%\begin{align*}
%L_x &= -i \hbar( y \partial_z - z \partial_y ) \\
%L_y &= -i \hbar( z \partial_x - x \partial_z ) \\
%L_z &= -i \hbar( x \partial_y - y \partial_x ) \\
%\end{align*}
\begin{align*}
L_1 &= -i \hbar( x_2 \partial_3 - x_3 \partial_2 ) \\
L_2 &= -i \hbar( x_3 \partial_1 - x_1 \partial_3 ) \\
L_3 &= -i \hbar( x_1 \partial_2 - x_2 \partial_1 ) \\
\end{align*}

It is interesting to put these in vector form, and then employ the freedom to use for $i = \sigma_1 \sigma_2 \sigma_3$ the spatial pseudoscalar.

\begin{align*}
\BL 
&= 
-\sigma_1 (\sigma_1 \sigma_2 \sigma_3) \hbar( x_2 \partial_3 - x_3 \partial_2 ) 
-\sigma_2 (\sigma_2 \sigma_3 \sigma_1) \hbar( x_3 \partial_1 - x_1 \partial_3 ) 
-\sigma_3 (\sigma_3 \sigma_1 \sigma_2) \hbar( x_1 \partial_2 - x_2 \partial_1 ) \\
&= 
-\sigma_2 \sigma_3 \hbar( x_2 \partial_3 - x_3 \partial_2 ) 
-\sigma_3 \sigma_1 \hbar( x_3 \partial_1 - x_1 \partial_3 ) 
-\sigma_1 \sigma_2 \hbar( x_1 \partial_2 - x_2 \partial_1 ) \\
&=
-\hbar ( \sigma_1 x_1 +\sigma_2 x_2 +\sigma_3 x_3 ) \wedge ( \sigma_1 \partial_1 +\sigma_2 \partial_2 +\sigma_3 \partial_3 ) \\
\end{align*}

The choice to use the pseudoscalar for this imaginary seems a logical one and the end result is a pure bivector representation of angular momentum operator

\begin{align}\label{eqn:qmAngularMom:ang}
\BL &= - \hbar \Bx \wedge \spacegrad
\end{align}

The choice to represent angular momentum as a bivector $\Bx \wedge \Bp$ is also natural in classical mechanics (encoding the orientation of the plane and the magnitude of the momentum in the bivector), although its dual form the axial vector $\Bx \cross \Bp$ is more common, at least in introductory mechanics.  Observe that there is no longer any explicit imaginary in (\ref{eqn:qmAngularMom:ang}), since the bivector itself has an implicit complex structure.

\section{Factoring the gradient and Laplacian.}

The form of (\ref{eqn:qmAngularMom:ang}) suggests a more direct way to extract the angular momentum operator from the Hamiltonian (i.e. from the Laplacian).  Bohm uses the spherical polar representation of the Laplacian as the starting point.  Instead let's project the gradient itself in a specific constant direction $\Ba$, much as we can do to find the polar form angular velocity and acceleration components.

Write 

\begin{align*}
\spacegrad 
&=
\inv{\Ba} \Ba \spacegrad \\
&=
\inv{\Ba} (\Ba \cdot \spacegrad + \Ba \wedge \spacegrad) \\
\end{align*}

Or
\begin{align*}
\spacegrad 
&=
\spacegrad \Ba \inv{\Ba} \\
&=
(\spacegrad \cdot \Ba + \spacegrad \wedge \Ba) \inv{\Ba} \\
&=
(\Ba \cdot \spacegrad - \Ba \wedge \spacegrad) \inv{\Ba} \\
\end{align*}

The Laplacian is therefore

\begin{align*}
\spacegrad^2 
&=
\gpgradezero{ \spacegrad^2 } \\
&=
\gpgradezero{ 
(\Ba \cdot \spacegrad - \Ba \wedge \spacegrad) \inv{\Ba} \inv{\Ba} (\Ba \cdot \spacegrad + \Ba \wedge \spacegrad) 
} \\
&=
\inv{\Ba^2} \gpgradezero{ 
(\Ba \cdot \spacegrad - \Ba \wedge \spacegrad) (\Ba \cdot \spacegrad + \Ba \wedge \spacegrad) 
} \\
&=
\inv{\Ba^2} ((\Ba \cdot \spacegrad)^2 - (\Ba \wedge \spacegrad)^2 ) \\
\end{align*}

So we have for the Laplacian a representation in terms of projection and rejection components 

\begin{align*}
\spacegrad^2
&=
(\acap \cdot \spacegrad)^2 - \inv{\Ba^2} (\Ba \wedge \spacegrad)^2 \\
&=
(\acap \cdot \spacegrad)^2 - (\acap \wedge \spacegrad)^2 \\
\end{align*}

The vector $\Ba$ was arbitrary, and just needed to be constant with respect to the factorization operations.  Setting $\Ba = \Bx$, the radial position from the origin one may guess that we have

\begin{align}\label{eqn:qmAngularMom:wrong}
\spacegrad^2 &= \frac{\partial^2 }{\partial r^2} - \inv{\Bx^2} (\Bx \wedge \spacegrad)^2 
\end{align}

however, with the switch to a non-constant position vector $\Bx$, this cannot possibly be right.

\section{The Coriolis term}

The radial factorization of the gradient relied on the direction vector $\Ba$ being constant.  If we evaluate (\ref{eqn:qmAngularMom:wrong}), then there should be a non-zero remainder compared to the Laplacian.  Evaluation by coordinate expansion is one way to verify this, and should produce the difference.  Let's do this in two parts, starting with $(x \wedge \grad)^2$.  Summation will be implied by mixed indexes, and for generality a general basis and associated reciprocal frame will be used.

\begin{align*}
(x \wedge \grad)^2 f 
&=
((x^\mu \gamma_\mu) \wedge (\gamma_\nu \partial^\nu)) \cdot 
((x_\alpha \gamma^\alpha) \wedge (\gamma^\beta \partial_\beta)) \\
&=
(\gamma_\mu \wedge \gamma_\nu) \cdot (\gamma^\alpha \wedge \gamma^\beta) x^\mu \partial^\nu (x_\alpha \partial_\beta) f \\
&=
({\delta_\mu}^\beta {\delta_\nu}^\alpha -{\delta_\mu}^\alpha {\delta_\nu}^\beta) x^\mu \partial^\nu (x_\alpha \partial_\beta) f \\
&=
x^\mu \partial^\nu ((x_\nu \partial_\mu) - x_\mu \partial_\nu) f \\
&=
x^\mu (\partial^\nu x_\nu) \partial_\mu f - x^\mu (\partial^\nu x_\mu) \partial_\nu f \\
&+x^\mu x_\nu \partial^\nu \partial_\mu f - x^\mu x_\mu \partial^\nu \partial_\nu f \\
&=
(n-1) x \cdot \grad f +x^\mu x_\nu \partial^\nu \partial_\mu f - x^2 \grad^2 f \\
\end{align*}

For the dot product we have
\begin{align*}
(x \cdot \grad)^2 f 
&=
x^\mu \partial_\mu( x^\nu \partial_\nu ) f \\
&=
x^\mu (\partial_\mu x^\nu) \partial_\nu  f + x^\mu x^\nu \partial_\mu \partial_\nu f \\
&=
x^\mu \partial_\mu f + x^\mu x_\nu \partial^\nu \partial_\mu f \\
&=
x \cdot \grad f + x^\mu x_\nu \partial^\nu \partial_\mu f \\
\end{align*}

So, forming the difference we have

\begin{align*}
(x \cdot \grad)^2 f - (x \wedge \grad)^2 f &=
-(n - 2) x \cdot \grad f + x^2 \grad^2 f \\
\end{align*}

Or
\begin{align}
\grad^2 &= \inv{x^2} (x \cdot \grad)^2 - \inv{x^2} (x \wedge \grad)^2 + (n - 2) \inv{x} \cdot \grad 
\end{align}

Going back to the quantum Hamiltonian we do still have the angular momentum operator as one of the distinct factors of the Laplacian.  As operators we have something akin to the projection of the gradient onto the radial direction, as well as terms that project the gradient onto the tangential plane to the sphere at the radial point

\begin{align}
-\frac{\hbar^2}{2m} \spacegrad^2 + V
&=
-\frac{\hbar^2}{2m} \left( \inv{\Bx^2} (\Bx \cdot \spacegrad)^2 - \inv{\Bx^2} (\Bx \wedge \spacegrad)^2 + \inv{\Bx} \cdot \spacegrad \right) + V
\end{align}

\section{Correspondence with explicit radial form.}

We've seen above that we can factor the 3D Laplacian as 

\begin{align}\label{eqn:qmAngularMom:foo1}
\spacegrad^2 \psi = \inv{\Bx^2}( (\Bx \cdot \spacegrad)^2 + \Bx \cdot \spacegrad - (\Bx \wedge \spacegrad)^2) \psi
\end{align}

Contrast this to the explicit $r,\theta,\phi$ form as given in (Bohm's \cite{bohm1989qt}, 14.2)

\begin{align}\label{eqn:qmAngularMom:foo2}
\spacegrad^2 \psi = \inv{r} \frac{\partial^2}{\partial r^2} (r\psi) + \inv{r^2} \left(
\inv{\sin\theta} \partial_\theta \sin\theta \partial\theta + \inv{\sin^2\theta} + \partial_{\phi \phi} \right) \psi
\end{align}

Let's expand out the non-angular momentum operator terms explicitly as a partial verification of this factorization.  The radial term in Bohm's Laplacian formula expands out to

\begin{align*}
\inv{r} \frac{\partial^2}{\partial r^2} (r\psi) 
&=
\inv{r} \partial_r (\partial_r r \psi) \\
&=
\inv{r} \partial_r (\psi + r\partial_r \psi) \\
&=
\inv{r} \partial_r \psi + \inv{r}( \partial_r \psi + r \partial_{rr} \psi) \\
&=
\frac{2}{r} \partial_r \psi + \partial_{rr} \psi \\
\end{align*}

On the other hand, with $\Bx = r\rcap$, what we expect to correspond to the radial term in the vector factorization is

\begin{align*}
\inv{\Bx^2}( (\Bx \cdot \spacegrad)^2 + \Bx \cdot \spacegrad ) \psi
&=
\inv{r^2}( (r \rcap \cdot \spacegrad)^2 + r \rcap \cdot \spacegrad  ) \psi \\
&=
\inv{r^2}( (r \partial_r )^2 + r \partial_r  ) \psi \\
&=
\inv{r^2}( r \partial_r \psi + r^2 \partial_{rr} \psi + r \partial_r \psi ) \\
&=
\frac{2}{r} \partial_r \psi + \partial_{rr} \psi 
\end{align*}

Okay, good.  It's a brute force way to verify things, but it works.  With $\Bx \wedge \spacegrad = I (\Bx \cross \spacegrad)$ we can eliminate the wedge product from the factorization expression (\ref{eqn:qmAngularMom:foo1}) and express things completely in quantities that can be understood without any resort to Geometric Algebra.  That is

\begin{align}\label{eqn:qmAngularMom:foo3}
\spacegrad^2 \psi = \inv{r} \frac{\partial^2}{\partial r^2} (r\psi) + \inv{r^2} (\Bx \cross \spacegrad)^2 \psi
\end{align}

Bohm resorts to analogy and an operatorization of $L_c = \epsilon_{abc} (x_a p_b - x_b p_a)$, then later a spherical polar change of coordinates to match exactly the $L^2$ expression with (\ref{eqn:qmAngularMom:foo2}).  With the GA formalism we see this a bit more directly, although it is not the least bit obvious that the operator $\Bx \cross \spacegrad$ has no radial dependence.  Without resorting to a comparison with the explicit $r,\theta,\phi$ form that wouldn't be so easy to see.

\section{Raising and Lowering operators in GA form.}

Having seen in (\chapcite{L1Associated}) that we have a natural GA form for the $l=1$ spherical harmonic eigenfunctions $\psi_1^{m}$, and that we have the vector angular momentum operator $\Bx \cross \spacegrad$ showing up directly in a sort-of-radial factorization of the Laplacian, it is natural to wonder what the GA form of the raising and lowering operators are.  At least for the $l=1$ harmonics use of $i = I \Be_3$ (unit bivector for the $x-y$ plane) for the imaginary ended up providing a nice geometric interpretation.

Let's see what that provides for the raising and lowering operators.  First we need to express $L_x$ and $L_y$ in terms of our bivector angular momentum operator.  Let's switch notations and drop the $-i \hbar$ factor from (\ref{eqn:qmAngularMom:ang}) writing just

\begin{align}\label{eqn:qmAngularMom:Ang}
\BL &= \Bx \wedge \spacegrad
\end{align}

We can now write this in terms of components with respect to the basis bivectors $I \Be_k$.  That is

\begin{align}\label{eqn:qmAngularMom:foo4}
\BL = \sum_k \left( (\Bx \wedge \spacegrad) \cdot \inv{I \Be_k}\right) I \Be_k
\end{align}

These scalar product results are expected to match the $L_x$, $L_y$, and $L_z$ components at least up to a sign.  Let's check, picking $L_z$ as representative

\begin{align*}
(\Bx \wedge \spacegrad) \cdot \inv{I \Be_3}
&=
(\sigma_m \wedge \sigma^k) \cdot {-\sigma_1 \sigma_2 \sigma_3 \sigma_3} x^m \partial_k \\
&=
(\sigma_m \wedge \sigma^k) \cdot {-\sigma_1 \sigma_2} x^m \partial_k \\
&=
-( x^2 \partial_1 - x^1 \partial_2 )
\end{align*}

With the $-i\hbar$ factors dropped this is $L_z = L_3 = x^1 \partial_2 - x^2 \partial_1$, the projection of $\BL$ onto the $x-y$ plane $I \Be_k$.  So, now how about the raising and lowering operators

\begin{align*}
L_x \pm i L_y
&=
L_x \pm I \Be_3 L_y \\
&=
\BL \cdot \inv{I\Be_1} \pm I \Be_3 \BL \cdot \inv{I\Be_2} \\
&=
-\Be_1 I \left( I \Be_1 \BL \cdot \inv{I\Be_1} \pm I \Be_2 \BL \cdot \inv{I\Be_2} \right) \\
\end{align*}

Or
\begin{align}\label{eqn:qmAngularMom:foo5}
(I \Be_1) L_x \pm i L_y &= I \Be_1 \BL \cdot \inv{I\Be_1} \pm I \Be_2 \BL \cdot \inv{I\Be_2} 
\end{align}

Compare this to the projective split of $\BL$ (\ref{eqn:qmAngularMom:foo4}).  We have projections of the bivector angular momentum operator onto the bivector directions $I\Be_1$ and $I\Be_2$ (really the bivectors for the planes perpendicular to the $\xcap$ and $\ycap$ directions).

We have the Laplacian in explicit vector form and have a clue how to vectorize (really bivectorize) the raising and lowering operators.  We have also seen how to geometrize the first spherical harmonics.  The next logical step is to try to apply this vector form of the raising and lowering operators to the vector form of the spherical harmonics.  I don't know yet how that will work out, and it's a game for a different night.

\EndArticle

\part{Fourier treatments}
\documentclass{article}

\usepackage{amsmath}
\usepackage{mathpazo}

%
% shorthand for bold symbols, convenient for vectors and matrices
%
\newcommand{\Ba}[0]{\mathbf{a}}
\newcommand{\Bb}[0]{\mathbf{b}}
\newcommand{\Bc}[0]{\mathbf{c}}
\newcommand{\Bd}[0]{\mathbf{d}}
\newcommand{\Be}[0]{\mathbf{e}}
\newcommand{\Bf}[0]{\mathbf{f}}
\newcommand{\Bg}[0]{\mathbf{g}}
\newcommand{\Bh}[0]{\mathbf{h}}
\newcommand{\Bi}[0]{\mathbf{i}}
\newcommand{\Bj}[0]{\mathbf{j}}
\newcommand{\Bk}[0]{\mathbf{k}}
\newcommand{\Bl}[0]{\mathbf{l}}
\newcommand{\Bm}[0]{\mathbf{m}}
\newcommand{\Bn}[0]{\mathbf{n}}
\newcommand{\Bo}[0]{\mathbf{o}}
\newcommand{\Bp}[0]{\mathbf{p}}
\newcommand{\Bq}[0]{\mathbf{q}}
\newcommand{\Br}[0]{\mathbf{r}}
\newcommand{\Bs}[0]{\mathbf{s}}
\newcommand{\Bt}[0]{\mathbf{t}}
\newcommand{\Bu}[0]{\mathbf{u}}
\newcommand{\Bv}[0]{\mathbf{v}}
\newcommand{\Bw}[0]{\mathbf{w}}
\newcommand{\Bx}[0]{\mathbf{x}}
\newcommand{\By}[0]{\mathbf{y}}
\newcommand{\Bz}[0]{\mathbf{z}}
\newcommand{\BA}[0]{\mathbf{A}}
\newcommand{\BB}[0]{\mathbf{B}}
\newcommand{\BC}[0]{\mathbf{C}}
\newcommand{\BD}[0]{\mathbf{D}}
\newcommand{\BE}[0]{\mathbf{E}}
\newcommand{\BF}[0]{\mathbf{F}}
\newcommand{\BG}[0]{\mathbf{G}}
\newcommand{\BH}[0]{\mathbf{H}}
\newcommand{\BI}[0]{\mathbf{I}}
\newcommand{\BJ}[0]{\mathbf{J}}
\newcommand{\BK}[0]{\mathbf{K}}
\newcommand{\BL}[0]{\mathbf{L}}
\newcommand{\BM}[0]{\mathbf{M}}
\newcommand{\BN}[0]{\mathbf{N}}
\newcommand{\BO}[0]{\mathbf{O}}
\newcommand{\BP}[0]{\mathbf{P}}
\newcommand{\BQ}[0]{\mathbf{Q}}
\newcommand{\BR}[0]{\mathbf{R}}
\newcommand{\BS}[0]{\mathbf{S}}
\newcommand{\BT}[0]{\mathbf{T}}
\newcommand{\BU}[0]{\mathbf{U}}
\newcommand{\BV}[0]{\mathbf{V}}
\newcommand{\BW}[0]{\mathbf{W}}
\newcommand{\BX}[0]{\mathbf{X}}
\newcommand{\BY}[0]{\mathbf{Y}}
\newcommand{\BZ}[0]{\mathbf{Z}}

\newcommand{\Bzero}[0]{\mathbf{0}}
\newcommand{\Btheta}[0]{\boldsymbol{\theta}}
\newcommand{\Btau}[0]{\boldsymbol{\tau}}
\newcommand{\Bomega}[0]{\boldsymbol{\omega}}

%
% shorthand for unit vectors
%
\newcommand{\acap}[0]{\hat{\Ba}}
\newcommand{\bcap}[0]{\hat{\Bb}}
\newcommand{\ccap}[0]{\hat{\Bc}}
\newcommand{\dcap}[0]{\hat{\Bd}}
\newcommand{\ecap}[0]{\hat{\Be}}
\newcommand{\fcap}[0]{\hat{\Bf}}
\newcommand{\gcap}[0]{\hat{\Bg}}
\newcommand{\hcap}[0]{\hat{\Bh}}
\newcommand{\icap}[0]{\hat{\Bi}}
\newcommand{\jcap}[0]{\hat{\Bj}}
\newcommand{\kcap}[0]{\hat{\Bk}}
\newcommand{\lcap}[0]{\hat{\Bl}}
\newcommand{\mcap}[0]{\hat{\Bm}}
\newcommand{\ncap}[0]{\hat{\Bn}}
\newcommand{\ocap}[0]{\hat{\Bo}}
\newcommand{\pcap}[0]{\hat{\Bp}}
\newcommand{\qcap}[0]{\hat{\Bq}}
\newcommand{\rcap}[0]{\hat{\Br}}
\newcommand{\scap}[0]{\hat{\Bs}}
\newcommand{\tcap}[0]{\hat{\Bt}}
\newcommand{\ucap}[0]{\hat{\Bu}}
\newcommand{\vcap}[0]{\hat{\Bv}}
\newcommand{\wcap}[0]{\hat{\Bw}}
\newcommand{\xcap}[0]{\hat{\Bx}}
\newcommand{\ycap}[0]{\hat{\By}}
\newcommand{\zcap}[0]{\hat{\Bz}}
\newcommand{\thetacap}[0]{\hat{\Btheta}}

%
% to write R^n and C^n in a distinguishable fashion.  Perhaps change this
% to the double lined characters upon figuring out how to do so.
%
\newcommand{\C}[1]{$\mathbb{C}^{#1}$}
\newcommand{\R}[1]{$\mathbb{R}^{#1}$}

%
% various generally useful helpers
%

% derivative of #1 wrt. #2:
\newcommand{\D}[2] {\frac {d#2} {d#1}}

\newcommand{\inv}[1]{\frac{1}{#1}}
\newcommand{\cross}[0]{\times}

\newcommand{\abs}[1]{\lvert{#1}\rvert}
\newcommand{\norm}[1]{\lVert{#1}\rVert}
\newcommand{\innerprod}[2]{\langle{#1}, {#2}\rangle}
\newcommand{\dotprod}[2]{{#1} \cdot {#2}}
\newcommand{\bdotprod}[2]{\left({#1} \cdot {#2}\right)}
\newcommand{\crossprod}[2]{{#1} \cross {#2}}
\newcommand{\tripleprod}[3]{\dotprod{\left(\crossprod{#1}{#2}\right)}{#3}}

\DeclareMathOperator{\Proj}{Proj}
\DeclareMathOperator{\Span}{span}
\DeclareMathOperator{\Sgn}{sgn}
\DeclareMathOperator{\Area}{Area}
\DeclareMathOperator{\Volume}{Volume}

%
% A few miscellaneous things specific to this document
%
\newcommand{\crossop}[1]{\crossprod{#1}{}}

% R2 vector.
\newcommand{\VectorTwo}[2]{
\begin{bmatrix}
 {#1} \\
 {#2}
\end{bmatrix}
}

\newcommand{\VectorN}[1]{
\begin{bmatrix}
{#1}_1 \\
{#1}_2 \\
\vdots \\
{#1}_N \\
\end{bmatrix}
}

\newcommand{\DETuvij}[4]{
\begin{vmatrix}
 {#1}_{#3} & {#1}_{#4} \\
 {#2}_{#3} & {#2}_{#4}
\end{vmatrix}
}

\newcommand{\DETuvwijk}[6]{
\begin{vmatrix}
 {#1}_{#4} & {#1}_{#5} & {#1}_{#6} \\
 {#2}_{#4} & {#2}_{#5} & {#2}_{#6} \\
 {#3}_{#4} & {#3}_{#5} & {#3}_{#6}
\end{vmatrix}
}

\newcommand{\DETuvwxijkl}[8]{
\begin{vmatrix}
 {#1}_{#5} & {#1}_{#6} & {#1}_{#7} & {#1}_{#8} \\
 {#2}_{#5} & {#2}_{#6} & {#2}_{#7} & {#2}_{#8} \\
 {#3}_{#5} & {#3}_{#6} & {#3}_{#7} & {#3}_{#8} \\
 {#4}_{#5} & {#4}_{#6} & {#4}_{#7} & {#4}_{#8} \\
\end{vmatrix}
}

%\newcommand{\DETuvwxyijklm}[10]{
%\begin{vmatrix}
% {#1}_{#6} & {#1}_{#7} & {#1}_{#8} & {#1}_{#9} & {#1}_{#10} \\
% {#2}_{#6} & {#2}_{#7} & {#2}_{#8} & {#2}_{#9} & {#2}_{#10} \\
% {#3}_{#6} & {#3}_{#7} & {#3}_{#8} & {#3}_{#9} & {#3}_{#10} \\
% {#4}_{#6} & {#4}_{#7} & {#4}_{#8} & {#4}_{#9} & {#4}_{#10} \\
% {#5}_{#6} & {#5}_{#7} & {#5}_{#8} & {#5}_{#9} & {#5}_{#10}
%\end{vmatrix}
%}

% R3 vector.
\newcommand{\VectorThree}[3]{
\begin{bmatrix}
 {#1} \\
 {#2} \\
 {#3}
\end{bmatrix}
}


%<misc>
%
\newcommand{\Abs}[1]{{\left\lvert{#1}\right\rvert}}
\newcommand{\spacegrad}[0]{\boldsymbol{\nabla}}
\newcommand{\grad}[0]{\nabla}
\newcommand{\LL}[0]{\mathcal{L}}

% == \partial_{#1} {#2}
\newcommand{\PD}[2]{\frac{\partial {#2}}{\partial {#1}}}
% inline variant
\newcommand{\PDi}[2]{{\partial {#2}}/{\partial {#1}}}

\newcommand{\PDD}[3]{\frac{\partial^2 {#3}}{\partial {#1}\partial {#2}}}
%\newcommand{\PDd}[2]{\frac{\partial^2 {#2}}{{\partial{#1}}^2}}
\newcommand{\PDsq}[2]{\frac{\partial^2 {#2}}{(\partial {#1})^2}}

\newcommand{\Partial}[2]{\frac{\partial {#1}}{\partial {#2}}}
\DeclareMathOperator{\RejName}{Rej}
\newcommand{\Rej}[2]{\RejName_{#1}\left( {#2} \right)}
\newcommand{\Rm}[1]{\mathbb{R}^{#1}}
\newcommand{\Cm}[1]{\mathbb{C}^{#1}}
\newcommand{\conj}[0]{{*}}

%</misc>

% <grade selection>
%
\newcommand{\gpgrade}[2] {{\left\langle{{#1}}\right\rangle}_{#2}}

\newcommand{\gpgradezero}[1] {\gpgrade{#1}{}}
%\newcommand{\gpscalargrade}[1] {{\left\langle{{#1}}\right\rangle}}
%\newcommand{\gpgradezero}[1] {\gpgrade{#1}{0}}

%\newcommand{\gpgradeone}[1] {{\left\langle{{#1}}\right\rangle}_{1}}
\newcommand{\gpgradeone}[1] {\gpgrade{#1}{1}}

\newcommand{\gpgradetwo}[1] {\gpgrade{#1}{2}}
\newcommand{\gpgradethree}[1] {\gpgrade{#1}{3}}
\newcommand{\gpgradefour}[1] {\gpgrade{#1}{4}}
%
% </grade selection>



\newcommand{\adot}[0]{{\dot{a}}}
\newcommand{\bdot}[0]{{\dot{b}}}
% taken for centered dot:
%\newcommand{\cdot}[0]{{\dot{c}}}
%\newcommand{\ddot}[0]{{\dot{d}}}
\newcommand{\edot}[0]{{\dot{e}}}
\newcommand{\fdot}[0]{{\dot{f}}}
\newcommand{\gdot}[0]{{\dot{g}}}
\newcommand{\hdot}[0]{{\dot{h}}}
\newcommand{\idot}[0]{{\dot{i}}}
\newcommand{\jdot}[0]{{\dot{j}}}
\newcommand{\kdot}[0]{{\dot{k}}}
\newcommand{\ldot}[0]{{\dot{l}}}
\newcommand{\mdot}[0]{{\dot{m}}}
\newcommand{\ndot}[0]{{\dot{n}}}
%\newcommand{\odot}[0]{{\dot{o}}}
\newcommand{\pdot}[0]{{\dot{p}}}
\newcommand{\qdot}[0]{{\dot{q}}}
\newcommand{\rdot}[0]{{\dot{r}}}
\newcommand{\sdot}[0]{{\dot{s}}}
\newcommand{\tdot}[0]{{\dot{t}}}
\newcommand{\udot}[0]{{\dot{u}}}
\newcommand{\vdot}[0]{{\dot{v}}}
\newcommand{\wdot}[0]{{\dot{w}}}
\newcommand{\xdot}[0]{{\dot{x}}}
\newcommand{\ydot}[0]{{\dot{y}}}
\newcommand{\zdot}[0]{{\dot{z}}}
\newcommand{\addot}[0]{{\ddot{a}}}
\newcommand{\bddot}[0]{{\ddot{b}}}
\newcommand{\cddot}[0]{{\ddot{c}}}
%\newcommand{\dddot}[0]{{\ddot{d}}}
\newcommand{\eddot}[0]{{\ddot{e}}}
\newcommand{\fddot}[0]{{\ddot{f}}}
\newcommand{\gddot}[0]{{\ddot{g}}}
\newcommand{\hddot}[0]{{\ddot{h}}}
\newcommand{\iddot}[0]{{\ddot{i}}}
\newcommand{\jddot}[0]{{\ddot{j}}}
\newcommand{\kddot}[0]{{\ddot{k}}}
\newcommand{\lddot}[0]{{\ddot{l}}}
\newcommand{\mddot}[0]{{\ddot{m}}}
\newcommand{\nddot}[0]{{\ddot{n}}}
\newcommand{\oddot}[0]{{\ddot{o}}}
\newcommand{\pddot}[0]{{\ddot{p}}}
\newcommand{\qddot}[0]{{\ddot{q}}}
\newcommand{\rddot}[0]{{\ddot{r}}}
\newcommand{\sddot}[0]{{\ddot{s}}}
\newcommand{\tddot}[0]{{\ddot{t}}}
\newcommand{\uddot}[0]{{\ddot{u}}}
\newcommand{\vddot}[0]{{\ddot{v}}}
\newcommand{\wddot}[0]{{\ddot{w}}}
\newcommand{\xddot}[0]{{\ddot{x}}}
\newcommand{\yddot}[0]{{\ddot{y}}}
\newcommand{\zddot}[0]{{\ddot{z}}}

%<bold and dot greek symbols>
%

\newcommand{\Deltadot}[0]{{\dot{\Delta}}}
\newcommand{\Gammadot}[0]{{\dot{\Gamma}}}
\newcommand{\Lambdadot}[0]{{\dot{\Lambda}}}
\newcommand{\Omegadot}[0]{{\dot{\Omega}}}
\newcommand{\Phidot}[0]{{\dot{\Phi}}}
\newcommand{\Pidot}[0]{{\dot{\Pi}}}
\newcommand{\Psidot}[0]{{\dot{\Psi}}}
\newcommand{\Sigmadot}[0]{{\dot{\Sigma}}}
\newcommand{\Thetadot}[0]{{\dot{\Theta}}}
\newcommand{\Upsilondot}[0]{{\dot{\Upsilon}}}
\newcommand{\Xidot}[0]{{\dot{\Xi}}}
\newcommand{\alphadot}[0]{{\dot{\alpha}}}
\newcommand{\betadot}[0]{{\dot{\beta}}}
\newcommand{\chidot}[0]{{\dot{\chi}}}
\newcommand{\deltadot}[0]{{\dot{\delta}}}
\newcommand{\epsilondot}[0]{{\dot{\epsilon}}}
\newcommand{\etadot}[0]{{\dot{\eta}}}
\newcommand{\gammadot}[0]{{\dot{\gamma}}}
\newcommand{\kappadot}[0]{{\dot{\kappa}}}
\newcommand{\lambdadot}[0]{{\dot{\lambda}}}
\newcommand{\mudot}[0]{{\dot{\mu}}}
\newcommand{\nudot}[0]{{\dot{\nu}}}
\newcommand{\omegadot}[0]{{\dot{\omega}}}
\newcommand{\phidot}[0]{{\dot{\phi}}}
\newcommand{\pidot}[0]{{\dot{\pi}}}
\newcommand{\psidot}[0]{{\dot{\psi}}}
\newcommand{\rhodot}[0]{{\dot{\rho}}}
\newcommand{\sigmadot}[0]{{\dot{\sigma}}}
\newcommand{\taudot}[0]{{\dot{\tau}}}
\newcommand{\thetadot}[0]{{\dot{\theta}}}
\newcommand{\upsilondot}[0]{{\dot{\upsilon}}}
\newcommand{\varepsilondot}[0]{{\dot{\varepsilon}}}
\newcommand{\varphidot}[0]{{\dot{\varphi}}}
\newcommand{\varpidot}[0]{{\dot{\varpi}}}
\newcommand{\varrhodot}[0]{{\dot{\varrho}}}
\newcommand{\varsigmadot}[0]{{\dot{\varsigma}}}
\newcommand{\varthetadot}[0]{{\dot{\vartheta}}}
\newcommand{\xidot}[0]{{\dot{\xi}}}
\newcommand{\zetadot}[0]{{\dot{\zeta}}}

\newcommand{\Deltaddot}[0]{{\ddot{\Delta}}}
\newcommand{\Gammaddot}[0]{{\ddot{\Gamma}}}
\newcommand{\Lambdaddot}[0]{{\ddot{\Lambda}}}
\newcommand{\Omegaddot}[0]{{\ddot{\Omega}}}
\newcommand{\Phiddot}[0]{{\ddot{\Phi}}}
\newcommand{\Piddot}[0]{{\ddot{\Pi}}}
\newcommand{\Psiddot}[0]{{\ddot{\Psi}}}
\newcommand{\Sigmaddot}[0]{{\ddot{\Sigma}}}
\newcommand{\Thetaddot}[0]{{\ddot{\Theta}}}
\newcommand{\Upsilonddot}[0]{{\ddot{\Upsilon}}}
\newcommand{\Xiddot}[0]{{\ddot{\Xi}}}
\newcommand{\alphaddot}[0]{{\ddot{\alpha}}}
\newcommand{\betaddot}[0]{{\ddot{\beta}}}
\newcommand{\chiddot}[0]{{\ddot{\chi}}}
\newcommand{\deltaddot}[0]{{\ddot{\delta}}}
\newcommand{\epsilonddot}[0]{{\ddot{\epsilon}}}
\newcommand{\etaddot}[0]{{\ddot{\eta}}}
\newcommand{\gammaddot}[0]{{\ddot{\gamma}}}
\newcommand{\kappaddot}[0]{{\ddot{\kappa}}}
\newcommand{\lambdaddot}[0]{{\ddot{\lambda}}}
\newcommand{\muddot}[0]{{\ddot{\mu}}}
\newcommand{\nuddot}[0]{{\ddot{\nu}}}
\newcommand{\omegaddot}[0]{{\ddot{\omega}}}
\newcommand{\phiddot}[0]{{\ddot{\phi}}}
\newcommand{\piddot}[0]{{\ddot{\pi}}}
\newcommand{\psiddot}[0]{{\ddot{\psi}}}
\newcommand{\rhoddot}[0]{{\ddot{\rho}}}
\newcommand{\sigmaddot}[0]{{\ddot{\sigma}}}
\newcommand{\tauddot}[0]{{\ddot{\tau}}}
\newcommand{\thetaddot}[0]{{\ddot{\theta}}}
\newcommand{\upsilonddot}[0]{{\ddot{\upsilon}}}
\newcommand{\varepsilonddot}[0]{{\ddot{\varepsilon}}}
\newcommand{\varphiddot}[0]{{\ddot{\varphi}}}
\newcommand{\varpiddot}[0]{{\ddot{\varpi}}}
\newcommand{\varrhoddot}[0]{{\ddot{\varrho}}}
\newcommand{\varsigmaddot}[0]{{\ddot{\varsigma}}}
\newcommand{\varthetaddot}[0]{{\ddot{\vartheta}}}
\newcommand{\xiddot}[0]{{\ddot{\xi}}}
\newcommand{\zetaddot}[0]{{\ddot{\zeta}}}

\newcommand{\BDelta}[0]{\boldsymbol{\Delta}}
\newcommand{\BGamma}[0]{\boldsymbol{\Gamma}}
\newcommand{\BLambda}[0]{\boldsymbol{\Lambda}}
\newcommand{\BOmega}[0]{\boldsymbol{\Omega}}
\newcommand{\BPhi}[0]{\boldsymbol{\Phi}}
\newcommand{\BPi}[0]{\boldsymbol{\Pi}}
\newcommand{\BPsi}[0]{\boldsymbol{\Psi}}
\newcommand{\BSigma}[0]{\boldsymbol{\Sigma}}
\newcommand{\BTheta}[0]{\boldsymbol{\Theta}}
\newcommand{\BUpsilon}[0]{\boldsymbol{\Upsilon}}
\newcommand{\BXi}[0]{\boldsymbol{\Xi}}
\newcommand{\Balpha}[0]{\boldsymbol{\alpha}}
\newcommand{\Bbeta}[0]{\boldsymbol{\beta}}
\newcommand{\Bchi}[0]{\boldsymbol{\chi}}
\newcommand{\Bdelta}[0]{\boldsymbol{\delta}}
\newcommand{\Bepsilon}[0]{\boldsymbol{\epsilon}}
\newcommand{\Beta}[0]{\boldsymbol{\eta}}
\newcommand{\Bgamma}[0]{\boldsymbol{\gamma}}
\newcommand{\Bkappa}[0]{\boldsymbol{\kappa}}
\newcommand{\Blambda}[0]{\boldsymbol{\lambda}}
\newcommand{\Bmu}[0]{\boldsymbol{\mu}}
\newcommand{\Bnu}[0]{\boldsymbol{\nu}}
%\newcommand{\Bomega}[0]{\boldsymbol{\omega}}
\newcommand{\Bphi}[0]{\boldsymbol{\phi}}
\newcommand{\Bpi}[0]{\boldsymbol{\pi}}
\newcommand{\Bpsi}[0]{\boldsymbol{\psi}}
\newcommand{\Brho}[0]{\boldsymbol{\rho}}
\newcommand{\Bsigma}[0]{\boldsymbol{\sigma}}
%\newcommand{\Btau}[0]{\boldsymbol{\tau}}
%\newcommand{\Btheta}[0]{\boldsymbol{\theta}}
\newcommand{\Bupsilon}[0]{\boldsymbol{\upsilon}}
\newcommand{\Bvarepsilon}[0]{\boldsymbol{\varepsilon}}
\newcommand{\Bvarphi}[0]{\boldsymbol{\varphi}}
\newcommand{\Bvarpi}[0]{\boldsymbol{\varpi}}
\newcommand{\Bvarrho}[0]{\boldsymbol{\varrho}}
\newcommand{\Bvarsigma}[0]{\boldsymbol{\varsigma}}
\newcommand{\Bvartheta}[0]{\boldsymbol{\vartheta}}
\newcommand{\Bxi}[0]{\boldsymbol{\xi}}
\newcommand{\Bzeta}[0]{\boldsymbol{\zeta}}
%
%</bold and dot greek symbols>
%<infrequent>
%
%\newcommand{\AreaOp}[1]{\AName_{#1}}
%\newcommand{\Babs}[0]{\abs{\BB}}
%\newcommand{\Bcap}[0]{\hat{\BB}}
%\newcommand{\BrPrimeRej}[0]{\rcap(\rcap \wedge \Br')}
%\newcommand{\CA}[0]{\mathcal{A}}
%\newcommand{\Cos}[1]{\cos{\left({#1}\right)}}
%\newcommand{\Det}[1] {\abs{#1}}
%\newcommand{\Dsq}[2] {\frac {\partial^2 {#1}} {\partial {#2}^2}}
%\newcommand{\Exp}[1]{\exp{\left({#1}\right)}}
%\newcommand{\Norm}[1]{\left\lVert{#1}\right\rVert}
%\newcommand{\Sin}[1]{\sin{\left({#1}\right)}}
%\newcommand{\T}[0]{\text{T}}
%\newcommand{\VolumeOp}[1]{\VName_{#1}}
%\newcommand{\agrad}[0]{\Ba \cdot \nabla}
%\newcommand{\alphacap}[0]{\hat{\boldsymbol{\alpha}}}
%\newcommand{\Fcap}[0]{\hat{\BF}}
%\newcommand{\bithree}[0]{{\Bi}_3}
%\newcommand{\bxa}[0]{\Bx\Ba}
%\newcommand{\coordvec}[2]{
%\newcommand{\costheta}[0]{\acap \cdot \xcap}
%\newcommand{\ddt}[1]{\ddot{#1}}
%\newcommand{\ddu}[1] {\frac {d{#1}} {du}}
%\newcommand{\dsqxj}[2] {\frac {\partial^2 {#1}} {\partial {x_{#2}}^2}}
%\newcommand{\dtheta}[1]{\frac{d {#1}}{d \theta}}
%\newcommand{\dt}[1]{\dot{#1}}
%\newcommand{\dt}[1]{\frac{d {#1}}{dt}}
%\newcommand{\dxj}[2] {\frac {\partial {#1}} {\partial {x_{#2}}}}
%\newcommand{\halfPhi}[0]{\frac{\phi}{2}}
%\newcommand{\half}[0]{\inv{2}}
%\newcommand{\inv}[1]{\frac{1}{#1}}
%\newcommand{\laplacian}[0]{\nabla^2}
%\newcommand{\matrixoftx}[3]{
%\newcommand{\nrrp}[0]{\norm{\rcap \wedge \Br'}}
%\newcommand{\oiint}{\bigcirc \hspace{-1.4em} \int \hspace{-.8em} \int}
%\newcommand{\transpose}[1]{{#1}^{\text{T}}}
%\newcommand{\transpose}[1]{{{#1}^{\TextTranspose}}}
%\newcommand{\transpose}[1]{{{#1}^{\text{T}}}}
%\newcommand{\barA}[0]{\bar{A}}
%\newcommand{\qbar}[0]{\bar{q}}
%\newcommand{\qdotbar}[0]{\dot{\bar{q}}}
%
%</infrequent>




\newcommand{\PDSq}[2]{\frac{\partial^2 {#2}}{\partial {#1}^2}}
\DeclareMathOperator{\sinc}{sinc}
\newcommand{\FF}[0]{\mathcal{F}}

\usepackage[bookmarks=true]{hyperref}

\usepackage{color,cite,graphicx}
   % use colour in the document, put your citations as [1-4]
   % rather than [1,2,3,4] (it looks nicer, and the extended LaTeX2e
   % graphics package. 
\usepackage{latexsym,amssymb,epsf} % don't remember if these are
   % needed, but their inclusion can't do any damage


\title{ Fourier Series solution to Heat equation. }
\author{Peeter Joot}
\date{ Jan 19, 2009.  Last Revision: $Date: 2009/01/21 02:43:01 $ }

\begin{document}

\maketitle{}

%\tableofcontents

\section{ Motivation. }

Stanford iTunesU has some Fourier transform lectures by Prof. Brad Osgood.
He starts with Fourier series and by Lecture 5 has covered this and
the solution of the Heat equation on a ring as an example.

Now, for these lectures I get only sound on my ipod.  I can listen along and
pick up most of the lectures since this is review material, but here's some
notes to firm things up.

Since this heat equation

\begin{align}
\grad^2 u = \kappa \partial_t u
\end{align}

is also the Schr\"{o}dinger equation for a free particle in one 
dimension (once the 
constant is fixed appropriately), we can also apply the Fourier
technique to a particle
constrained to a circle.  It would be interesting afterwards to 
contrast this with Susskind's solution of the
same problem (where he used the Fourier transform and algebraic techniques
instead).

\section{ Preliminaries. }

\subsection{ Laplacian. }

Osgood wrote the heat equation for the ring as

\begin{align*}
\inv{2} u_{xx} = u_t
\end{align*}

where $x$ represented an angular position on the ring, and where
he set the heat diffusion constant to $1/2$ for convienience.
To apply this to the Schr\"{o}dinger equation retaining all the desired
units we want to be a bit more careful, so let's start with the Laplacian
in polar coordinates.

In polar coordinates our gradient is

\begin{align*}
\grad = \thetacap \inv{r} \PD{\theta}{} +\rcap \PD{r}{} 
\end{align*}

squaring this we have

\begin{align*}
\grad^2 = \grad \cdot \grad
&= 
\thetacap \inv{r} \PD{\theta}{} \cdot \left(\thetacap \inv{r} \PD{\theta}{}\right)
 +
\rcap \PD{r}{} \cdot \left(\rcap \PD{r}{} \right) \\
&= 
\frac{-1}{r^3} \PD{\theta}{r} \PD{\theta}{}
+\inv{r^2} \PDSq{\theta}{}
+ \PDSq{r}{}
\\
&= \inv{r^2} \PDSq{\theta}{} + \PDSq{r}{} \\
\end{align*}

So for the circularly constrained where $r$ is constant case we have simply

\begin{align}
\grad^2 = \inv{r^2} \PDSq{\theta}{}
\end{align}

and our heat equation to solve becomes

\begin{align}
\PDSq{\theta}{u(\theta, t)} = (r^2 \kappa) \PD{t}{u(\theta, t)}
\end{align}

\subsection{ Fourier series. }

Now we also want Fourier series for a given period.  Assuming the absence of the "Rigor Police" as Osgood puts it
we write for a periodic function $f(x)$ known on the interval $I = [a, a+T]$

\begin{align*}
f(x) = \sum c_k e^{2\pi i k x/T}
\end{align*}

\begin{align*}
\int_{\partial I} f(x) e^{- 2 \pi i m x /T} 
&= \sum c_k \int_{\partial I} e^{2\pi i (k -m) x/T} \\
&= c_m T
\end{align*}

So our Fourier coefficient is
\begin{align*}
\hat{f}(k) = c_k = \inv{T} \int_{\partial I} f(x) e^{- 2 \pi i m x /T} 
\end{align*}

\section{ Solution of heat equation. } 

\subsection{ Basic solution. }

Now we are ready to solve the radial heat equation

\begin{align}\label{eqn:heatRadial}
u_{\theta\theta} = r^2 \kappa u_t,
\end{align}

by assuming a Fourier series solution.

Suppose

\begin{align*}
u(\theta, t) 
&= \sum c_m(t) e^{2 \pi i m \theta / T} \\
&= \sum c_m(t) e^{i m \theta} \\
\end{align*}

Taking derivatives of this assumed solution we have
\begin{align*}
u_{\theta\theta} &= \sum (im)^2 c_m e^{i m \theta} \\
u_{t} &= \sum c_m' e^{i m \theta}
\end{align*}

Substituting this back into \ref{eqn:heatRadial} we have

\begin{align*}
\sum - m^2 c_m e^{ i m \theta} = \sum c_m' r^2 \kappa e^{i m \theta}
\end{align*}

equating components we have 

\begin{align*}
c_m' = - \frac{m^2}{ r^2 \kappa } c_m 
\end{align*}

which is also just an exponential.

\begin{align*}
c_m = A_m e^{- \frac{m^2}{ r^2 \kappa } t }
\end{align*}

Reassembling we have the time variation of the solution now fixed and can write

\begin{align}
u(\theta, t) = \sum A_m e^{- \frac{m^2}{ r^2 \kappa } t } e^{i m \theta}
\end{align}

\subsection{ As initial value problem. }

For the heat equation case, we can assume a known initial heat distribution 
$f(\theta)$.
For an initial time $t=0$ we can then write

\begin{align*}
u(\theta, 0) = \sum A_m e^{i m \theta} = f(\theta)
\end{align*}

This is just another Fourier series, with Fourier coefficients

\begin{align*}
A_m = \inv{2\pi} \int_{\partial I} f(v) e^{-i m v} dv
\end{align*}

Final reassembly of the results gives us

\begin{align}
u(\theta, t) = \sum e^{- \frac{m^2}{ r^2 \kappa } t } e^{i m \theta} \inv{2\pi} \int_{\partial I} f(v) e^{-i m v} dv
\end{align}

\subsection{ Convolution. }

Osgood's next step, also with the rigor police in hiding, was to exchange orders of integration and summation, to write

\begin{align*}
u(\theta, t) 
&= 
\int_{\partial I} f(v) dv \inv{2 \pi} \sum_{m=-\infty}^{\infty} e^{- \frac{m^2}{ r^2 \kappa } t } e^{-i m (v -\theta)} \\
\end{align*}

Introducing a Green's function $g(v - \theta, t)$, we then have the complete solution in terms of convolution

\begin{align}
g( v - \theta, t ) &= \inv{2 \pi} \sum_{m=-\infty}^\infty e^{- \frac{m^2}{ r^2 \kappa } t } e^{-i m (v -\theta)} \\
u(\theta, t) &= \int_{\partial I} f(v) g(v - \theta, t) dv 
\end{align}

Now, this Green's function is fairly interesting.  By summing over paired negative and positive indexes, we have a set of
weighted Gaussians.

\begin{align*}
g( v - \theta, t ) &= \inv{2 \pi} + \sum_{m=1}^\infty \exp\left(- \frac{m^2}{ r^2 \kappa } t \right) \frac{\cos(m (v -\theta))}{\pi} \\
\end{align*}

Recalling that the delta function can be expressed as a limit of a $\sinc$ function, seeing something similar
in this Green's function is not entirely unsuprising seeming.

\section{ Wave equation. }

The QM equation for a free particle is

\begin{align*}
-\frac{\hbar^2}{2m} \grad^2 \psi = i \hbar \partial_t \psi
\end{align*}

This has the same form of the heat equation, so for the free particle on a circle our wave equation is

\begin{align*}
\psi_{\theta\theta} = - \frac{2 m i r^2 }{\hbar} \partial_t \psi \quad \mbox{ ie: $r^2 \kappa = - 2 m i /\hbar$ }
\end{align*}

So, if the wave equation was known at an initial time $\psi(\theta, 0) = \phi(\theta)$, we therefore have by comparision the time 
evolution of the particle's wave function is

\begin{align*}
g( w, t ) &= \inv{2 \pi} + \sum_{k=1}^\infty \exp\left(- \frac{i \hbar k^2 t}{ 2 m r^2 } \right) \frac{\cos(k w )}{\pi} \\
\psi(\theta, t) &= \int_{\partial I} \phi(v) g(v - \theta, t) dv 
\end{align*}

%TODO: contrast this to a Fourier transform solution.  Also write this in terms of circular angular momentum since that appears natually in the Green's function.

\section{ Fourier transform solution. }

% also see example (brief on details)
%\href{http://zakuski.utsa.edu/~gokhman/ftp//courses/notes/heat.pdf}{ example of Fourier tx solution. }
Now, lets try this one dimensional heat problem with a Fourier transform instead to compare.  Here we don't try to start with an
assumed solution, but instead take the Fourier transform of both sides of the equation directly.

\begin{align*}
\FF(u_{xx}) = \kappa \FF(u_t)
\end{align*}

Let's start with the left hand side, where we can evaluate by integrating by parts

\begin{align*}
\FF(u_{xx}) 
&= \inv{\sqrt{2\pi}} \int u_{xx}(x, t) e^{- 2 \pi i s x } dx \\
&= \inv{\sqrt{2\pi}} \int \PD{x}{u_x(x, t)} e^{- 2 \pi i s x } dx \\
&= \inv{\sqrt{2\pi}} 
\left(
{\left. u_x(x, t) e^{- 2 \pi i s x } \right\vert}_{x= -\infty}^\infty
-( - 2 \pi i s ) \int u_x(x, t) e^{- 2 \pi i s x } dx 
\right) \\
\end{align*}

So if we assume (or require) that the derivative of our unknown function $u$ is zero at infinity, and then similarily
require the function itself to be zero there, we have

\begin{align*}
\FF(u_{xx}) 
&= \inv{\sqrt{2\pi}} ( 2 \pi i s ) \int \PD{x}{u_x(x, t)} e^{- 2 \pi i s x } dx  \\
&= \inv{\sqrt{2\pi}} ( 2 \pi i s )^2 \int u(x, t) e^{- 2 \pi i s x } dx  \\
&= ( 2 \pi i s )^2 \FF(u)
\end{align*}

Now, for the time derivative.  We want

\begin{align*}
\FF(u_t) &= \inv{\sqrt{2\pi}} \int u_t(x, t) e^{- 2 \pi i s x } dx \\
\end{align*}

But can pull the derivative out of the integral for
\begin{align*}
\FF(u_t)
&= \PD{t}{} \left(\inv{\sqrt{2\pi}} \int u(x, t) e^{- 2 \pi i s x } dx \right) \\
&= \PD{t}{\FF(u)} 
\end{align*}

So, now we have an equation relating time derivatives only of the Fourier transformed solution.

Writing $\FF(u) = \hat{u}$ this is

\begin{align}\label{eqn:toSolveFreq}
( 2 \pi i s )^2 \hat{u} = \kappa \PD{t}{\hat{u}}
\end{align}

With a solution of

\begin{align*}
\hat{u} = A(s) e^{ -4 \pi^2 s^2 t/ \kappa }
\end{align*}

Here $A(s)$ is an arbitrary constant in time integration constant, which may depend on $s$ since it is a solution of our simpler freqency domain partial differential equation
\ref{eqn:toSolveFreq}.

Performing an inverse transform to recover $u(x,t)$ we thus have

\begin{align*}
u(x,t) 
&= \inv{\sqrt{2\pi}} \int \hat{u} e^{2 \pi i x s } ds  \\
&= \inv{\sqrt{2\pi}} \int A(s) e^{ -4 \pi^2 s^2 t/ \kappa } e^{2 \pi i x s } ds  \\
\end{align*}

Now, how about initial conditions.  Suppose we have $u(x,0) = w(x)$, then 

\begin{align*}
w(x) &= \inv{\sqrt{2\pi}} \int A(s) e^{2 \pi i x s } ds \\
\end{align*}

Which is just an inverse Fourier transform in terms of the integration ``constant'' $A(s)$.  We can therefore write the $A(s)$ in terms of the
initial time domain conditions.

\begin{align*}
A(s) &= \inv{\sqrt{2\pi}} \int w(x) e^{-2 \pi i s x } dx \\
&= \hat{w}(s)
\end{align*}

and finally have a complete solution of the one dimensional Heat equation (or wave equation).  That is

\begin{align*}
u(x,t) &= \inv{\sqrt{2\pi}} \int \hat{w}(s) e^{ -4 \pi^2 s^2 t/ \kappa } e^{2 \pi i x s } ds  \\
\end{align*}

%\bibliographystyle{plainnat}
%\bibliography{myrefs}

\end{document}

\documentclass{article}

\usepackage{amsmath}
\usepackage{mathpazo}

%
% shorthand for bold symbols, convenient for vectors and matrices
%
\newcommand{\Ba}[0]{\mathbf{a}}
\newcommand{\Bb}[0]{\mathbf{b}}
\newcommand{\Bc}[0]{\mathbf{c}}
\newcommand{\Bd}[0]{\mathbf{d}}
\newcommand{\Be}[0]{\mathbf{e}}
\newcommand{\Bf}[0]{\mathbf{f}}
\newcommand{\Bg}[0]{\mathbf{g}}
\newcommand{\Bh}[0]{\mathbf{h}}
\newcommand{\Bi}[0]{\mathbf{i}}
\newcommand{\Bj}[0]{\mathbf{j}}
\newcommand{\Bk}[0]{\mathbf{k}}
\newcommand{\Bl}[0]{\mathbf{l}}
\newcommand{\Bm}[0]{\mathbf{m}}
\newcommand{\Bn}[0]{\mathbf{n}}
\newcommand{\Bo}[0]{\mathbf{o}}
\newcommand{\Bp}[0]{\mathbf{p}}
\newcommand{\Bq}[0]{\mathbf{q}}
\newcommand{\Br}[0]{\mathbf{r}}
\newcommand{\Bs}[0]{\mathbf{s}}
\newcommand{\Bt}[0]{\mathbf{t}}
\newcommand{\Bu}[0]{\mathbf{u}}
\newcommand{\Bv}[0]{\mathbf{v}}
\newcommand{\Bw}[0]{\mathbf{w}}
\newcommand{\Bx}[0]{\mathbf{x}}
\newcommand{\By}[0]{\mathbf{y}}
\newcommand{\Bz}[0]{\mathbf{z}}
\newcommand{\BA}[0]{\mathbf{A}}
\newcommand{\BB}[0]{\mathbf{B}}
\newcommand{\BC}[0]{\mathbf{C}}
\newcommand{\BD}[0]{\mathbf{D}}
\newcommand{\BE}[0]{\mathbf{E}}
\newcommand{\BF}[0]{\mathbf{F}}
\newcommand{\BG}[0]{\mathbf{G}}
\newcommand{\BH}[0]{\mathbf{H}}
\newcommand{\BI}[0]{\mathbf{I}}
\newcommand{\BJ}[0]{\mathbf{J}}
\newcommand{\BK}[0]{\mathbf{K}}
\newcommand{\BL}[0]{\mathbf{L}}
\newcommand{\BM}[0]{\mathbf{M}}
\newcommand{\BN}[0]{\mathbf{N}}
\newcommand{\BO}[0]{\mathbf{O}}
\newcommand{\BP}[0]{\mathbf{P}}
\newcommand{\BQ}[0]{\mathbf{Q}}
\newcommand{\BR}[0]{\mathbf{R}}
\newcommand{\BS}[0]{\mathbf{S}}
\newcommand{\BT}[0]{\mathbf{T}}
\newcommand{\BU}[0]{\mathbf{U}}
\newcommand{\BV}[0]{\mathbf{V}}
\newcommand{\BW}[0]{\mathbf{W}}
\newcommand{\BX}[0]{\mathbf{X}}
\newcommand{\BY}[0]{\mathbf{Y}}
\newcommand{\BZ}[0]{\mathbf{Z}}

\newcommand{\Bzero}[0]{\mathbf{0}}
\newcommand{\Btheta}[0]{\boldsymbol{\theta}}
\newcommand{\Btau}[0]{\boldsymbol{\tau}}
\newcommand{\Bomega}[0]{\boldsymbol{\omega}}

%
% shorthand for unit vectors
%
\newcommand{\acap}[0]{\hat{\Ba}}
\newcommand{\bcap}[0]{\hat{\Bb}}
\newcommand{\ccap}[0]{\hat{\Bc}}
\newcommand{\dcap}[0]{\hat{\Bd}}
\newcommand{\ecap}[0]{\hat{\Be}}
\newcommand{\fcap}[0]{\hat{\Bf}}
\newcommand{\gcap}[0]{\hat{\Bg}}
\newcommand{\hcap}[0]{\hat{\Bh}}
\newcommand{\icap}[0]{\hat{\Bi}}
\newcommand{\jcap}[0]{\hat{\Bj}}
\newcommand{\kcap}[0]{\hat{\Bk}}
\newcommand{\lcap}[0]{\hat{\Bl}}
\newcommand{\mcap}[0]{\hat{\Bm}}
\newcommand{\ncap}[0]{\hat{\Bn}}
\newcommand{\ocap}[0]{\hat{\Bo}}
\newcommand{\pcap}[0]{\hat{\Bp}}
\newcommand{\qcap}[0]{\hat{\Bq}}
\newcommand{\rcap}[0]{\hat{\Br}}
\newcommand{\scap}[0]{\hat{\Bs}}
\newcommand{\tcap}[0]{\hat{\Bt}}
\newcommand{\ucap}[0]{\hat{\Bu}}
\newcommand{\vcap}[0]{\hat{\Bv}}
\newcommand{\wcap}[0]{\hat{\Bw}}
\newcommand{\xcap}[0]{\hat{\Bx}}
\newcommand{\ycap}[0]{\hat{\By}}
\newcommand{\zcap}[0]{\hat{\Bz}}
\newcommand{\thetacap}[0]{\hat{\Btheta}}

%
% to write R^n and C^n in a distinguishable fashion.  Perhaps change this
% to the double lined characters upon figuring out how to do so.
%
\newcommand{\C}[1]{$\mathbb{C}^{#1}$}
\newcommand{\R}[1]{$\mathbb{R}^{#1}$}

%
% various generally useful helpers
%

% derivative of #1 wrt. #2:
\newcommand{\D}[2] {\frac {d#2} {d#1}}

\newcommand{\inv}[1]{\frac{1}{#1}}
\newcommand{\cross}[0]{\times}

\newcommand{\abs}[1]{\lvert{#1}\rvert}
\newcommand{\norm}[1]{\lVert{#1}\rVert}
\newcommand{\innerprod}[2]{\langle{#1}, {#2}\rangle}
\newcommand{\dotprod}[2]{{#1} \cdot {#2}}
\newcommand{\bdotprod}[2]{\left({#1} \cdot {#2}\right)}
\newcommand{\crossprod}[2]{{#1} \cross {#2}}
\newcommand{\tripleprod}[3]{\dotprod{\left(\crossprod{#1}{#2}\right)}{#3}}

\DeclareMathOperator{\Proj}{Proj}
\DeclareMathOperator{\Span}{span}
\DeclareMathOperator{\Sgn}{sgn}
\DeclareMathOperator{\Area}{Area}
\DeclareMathOperator{\Volume}{Volume}

%
% A few miscellaneous things specific to this document
%
\newcommand{\crossop}[1]{\crossprod{#1}{}}

% R2 vector.
\newcommand{\VectorTwo}[2]{
\begin{bmatrix}
 {#1} \\
 {#2}
\end{bmatrix}
}

\newcommand{\VectorN}[1]{
\begin{bmatrix}
{#1}_1 \\
{#1}_2 \\
\vdots \\
{#1}_N \\
\end{bmatrix}
}

\newcommand{\DETuvij}[4]{
\begin{vmatrix}
 {#1}_{#3} & {#1}_{#4} \\
 {#2}_{#3} & {#2}_{#4}
\end{vmatrix}
}

\newcommand{\DETuvwijk}[6]{
\begin{vmatrix}
 {#1}_{#4} & {#1}_{#5} & {#1}_{#6} \\
 {#2}_{#4} & {#2}_{#5} & {#2}_{#6} \\
 {#3}_{#4} & {#3}_{#5} & {#3}_{#6}
\end{vmatrix}
}

\newcommand{\DETuvwxijkl}[8]{
\begin{vmatrix}
 {#1}_{#5} & {#1}_{#6} & {#1}_{#7} & {#1}_{#8} \\
 {#2}_{#5} & {#2}_{#6} & {#2}_{#7} & {#2}_{#8} \\
 {#3}_{#5} & {#3}_{#6} & {#3}_{#7} & {#3}_{#8} \\
 {#4}_{#5} & {#4}_{#6} & {#4}_{#7} & {#4}_{#8} \\
\end{vmatrix}
}

%\newcommand{\DETuvwxyijklm}[10]{
%\begin{vmatrix}
% {#1}_{#6} & {#1}_{#7} & {#1}_{#8} & {#1}_{#9} & {#1}_{#10} \\
% {#2}_{#6} & {#2}_{#7} & {#2}_{#8} & {#2}_{#9} & {#2}_{#10} \\
% {#3}_{#6} & {#3}_{#7} & {#3}_{#8} & {#3}_{#9} & {#3}_{#10} \\
% {#4}_{#6} & {#4}_{#7} & {#4}_{#8} & {#4}_{#9} & {#4}_{#10} \\
% {#5}_{#6} & {#5}_{#7} & {#5}_{#8} & {#5}_{#9} & {#5}_{#10}
%\end{vmatrix}
%}

% R3 vector.
\newcommand{\VectorThree}[3]{
\begin{bmatrix}
 {#1} \\
 {#2} \\
 {#3}
\end{bmatrix}
}


%<misc>
%
\newcommand{\Abs}[1]{{\left\lvert{#1}\right\rvert}}
\newcommand{\spacegrad}[0]{\boldsymbol{\nabla}}
\newcommand{\grad}[0]{\nabla}
\newcommand{\LL}[0]{\mathcal{L}}

% == \partial_{#1} {#2}
\newcommand{\PD}[2]{\frac{\partial {#2}}{\partial {#1}}}
% inline variant
\newcommand{\PDi}[2]{{\partial {#2}}/{\partial {#1}}}

\newcommand{\PDD}[3]{\frac{\partial^2 {#3}}{\partial {#1}\partial {#2}}}
%\newcommand{\PDd}[2]{\frac{\partial^2 {#2}}{{\partial{#1}}^2}}
\newcommand{\PDsq}[2]{\frac{\partial^2 {#2}}{(\partial {#1})^2}}

\newcommand{\Partial}[2]{\frac{\partial {#1}}{\partial {#2}}}
\DeclareMathOperator{\RejName}{Rej}
\newcommand{\Rej}[2]{\RejName_{#1}\left( {#2} \right)}
\newcommand{\Rm}[1]{\mathbb{R}^{#1}}
\newcommand{\Cm}[1]{\mathbb{C}^{#1}}
\newcommand{\conj}[0]{{*}}

%</misc>

% <grade selection>
%
\newcommand{\gpgrade}[2] {{\left\langle{{#1}}\right\rangle}_{#2}}

\newcommand{\gpgradezero}[1] {\gpgrade{#1}{}}
%\newcommand{\gpscalargrade}[1] {{\left\langle{{#1}}\right\rangle}}
%\newcommand{\gpgradezero}[1] {\gpgrade{#1}{0}}

%\newcommand{\gpgradeone}[1] {{\left\langle{{#1}}\right\rangle}_{1}}
\newcommand{\gpgradeone}[1] {\gpgrade{#1}{1}}

\newcommand{\gpgradetwo}[1] {\gpgrade{#1}{2}}
\newcommand{\gpgradethree}[1] {\gpgrade{#1}{3}}
\newcommand{\gpgradefour}[1] {\gpgrade{#1}{4}}
%
% </grade selection>



\newcommand{\adot}[0]{{\dot{a}}}
\newcommand{\bdot}[0]{{\dot{b}}}
% taken for centered dot:
%\newcommand{\cdot}[0]{{\dot{c}}}
%\newcommand{\ddot}[0]{{\dot{d}}}
\newcommand{\edot}[0]{{\dot{e}}}
\newcommand{\fdot}[0]{{\dot{f}}}
\newcommand{\gdot}[0]{{\dot{g}}}
\newcommand{\hdot}[0]{{\dot{h}}}
\newcommand{\idot}[0]{{\dot{i}}}
\newcommand{\jdot}[0]{{\dot{j}}}
\newcommand{\kdot}[0]{{\dot{k}}}
\newcommand{\ldot}[0]{{\dot{l}}}
\newcommand{\mdot}[0]{{\dot{m}}}
\newcommand{\ndot}[0]{{\dot{n}}}
%\newcommand{\odot}[0]{{\dot{o}}}
\newcommand{\pdot}[0]{{\dot{p}}}
\newcommand{\qdot}[0]{{\dot{q}}}
\newcommand{\rdot}[0]{{\dot{r}}}
\newcommand{\sdot}[0]{{\dot{s}}}
\newcommand{\tdot}[0]{{\dot{t}}}
\newcommand{\udot}[0]{{\dot{u}}}
\newcommand{\vdot}[0]{{\dot{v}}}
\newcommand{\wdot}[0]{{\dot{w}}}
\newcommand{\xdot}[0]{{\dot{x}}}
\newcommand{\ydot}[0]{{\dot{y}}}
\newcommand{\zdot}[0]{{\dot{z}}}
\newcommand{\addot}[0]{{\ddot{a}}}
\newcommand{\bddot}[0]{{\ddot{b}}}
\newcommand{\cddot}[0]{{\ddot{c}}}
%\newcommand{\dddot}[0]{{\ddot{d}}}
\newcommand{\eddot}[0]{{\ddot{e}}}
\newcommand{\fddot}[0]{{\ddot{f}}}
\newcommand{\gddot}[0]{{\ddot{g}}}
\newcommand{\hddot}[0]{{\ddot{h}}}
\newcommand{\iddot}[0]{{\ddot{i}}}
\newcommand{\jddot}[0]{{\ddot{j}}}
\newcommand{\kddot}[0]{{\ddot{k}}}
\newcommand{\lddot}[0]{{\ddot{l}}}
\newcommand{\mddot}[0]{{\ddot{m}}}
\newcommand{\nddot}[0]{{\ddot{n}}}
\newcommand{\oddot}[0]{{\ddot{o}}}
\newcommand{\pddot}[0]{{\ddot{p}}}
\newcommand{\qddot}[0]{{\ddot{q}}}
\newcommand{\rddot}[0]{{\ddot{r}}}
\newcommand{\sddot}[0]{{\ddot{s}}}
\newcommand{\tddot}[0]{{\ddot{t}}}
\newcommand{\uddot}[0]{{\ddot{u}}}
\newcommand{\vddot}[0]{{\ddot{v}}}
\newcommand{\wddot}[0]{{\ddot{w}}}
\newcommand{\xddot}[0]{{\ddot{x}}}
\newcommand{\yddot}[0]{{\ddot{y}}}
\newcommand{\zddot}[0]{{\ddot{z}}}

%<bold and dot greek symbols>
%

\newcommand{\Deltadot}[0]{{\dot{\Delta}}}
\newcommand{\Gammadot}[0]{{\dot{\Gamma}}}
\newcommand{\Lambdadot}[0]{{\dot{\Lambda}}}
\newcommand{\Omegadot}[0]{{\dot{\Omega}}}
\newcommand{\Phidot}[0]{{\dot{\Phi}}}
\newcommand{\Pidot}[0]{{\dot{\Pi}}}
\newcommand{\Psidot}[0]{{\dot{\Psi}}}
\newcommand{\Sigmadot}[0]{{\dot{\Sigma}}}
\newcommand{\Thetadot}[0]{{\dot{\Theta}}}
\newcommand{\Upsilondot}[0]{{\dot{\Upsilon}}}
\newcommand{\Xidot}[0]{{\dot{\Xi}}}
\newcommand{\alphadot}[0]{{\dot{\alpha}}}
\newcommand{\betadot}[0]{{\dot{\beta}}}
\newcommand{\chidot}[0]{{\dot{\chi}}}
\newcommand{\deltadot}[0]{{\dot{\delta}}}
\newcommand{\epsilondot}[0]{{\dot{\epsilon}}}
\newcommand{\etadot}[0]{{\dot{\eta}}}
\newcommand{\gammadot}[0]{{\dot{\gamma}}}
\newcommand{\kappadot}[0]{{\dot{\kappa}}}
\newcommand{\lambdadot}[0]{{\dot{\lambda}}}
\newcommand{\mudot}[0]{{\dot{\mu}}}
\newcommand{\nudot}[0]{{\dot{\nu}}}
\newcommand{\omegadot}[0]{{\dot{\omega}}}
\newcommand{\phidot}[0]{{\dot{\phi}}}
\newcommand{\pidot}[0]{{\dot{\pi}}}
\newcommand{\psidot}[0]{{\dot{\psi}}}
\newcommand{\rhodot}[0]{{\dot{\rho}}}
\newcommand{\sigmadot}[0]{{\dot{\sigma}}}
\newcommand{\taudot}[0]{{\dot{\tau}}}
\newcommand{\thetadot}[0]{{\dot{\theta}}}
\newcommand{\upsilondot}[0]{{\dot{\upsilon}}}
\newcommand{\varepsilondot}[0]{{\dot{\varepsilon}}}
\newcommand{\varphidot}[0]{{\dot{\varphi}}}
\newcommand{\varpidot}[0]{{\dot{\varpi}}}
\newcommand{\varrhodot}[0]{{\dot{\varrho}}}
\newcommand{\varsigmadot}[0]{{\dot{\varsigma}}}
\newcommand{\varthetadot}[0]{{\dot{\vartheta}}}
\newcommand{\xidot}[0]{{\dot{\xi}}}
\newcommand{\zetadot}[0]{{\dot{\zeta}}}

\newcommand{\Deltaddot}[0]{{\ddot{\Delta}}}
\newcommand{\Gammaddot}[0]{{\ddot{\Gamma}}}
\newcommand{\Lambdaddot}[0]{{\ddot{\Lambda}}}
\newcommand{\Omegaddot}[0]{{\ddot{\Omega}}}
\newcommand{\Phiddot}[0]{{\ddot{\Phi}}}
\newcommand{\Piddot}[0]{{\ddot{\Pi}}}
\newcommand{\Psiddot}[0]{{\ddot{\Psi}}}
\newcommand{\Sigmaddot}[0]{{\ddot{\Sigma}}}
\newcommand{\Thetaddot}[0]{{\ddot{\Theta}}}
\newcommand{\Upsilonddot}[0]{{\ddot{\Upsilon}}}
\newcommand{\Xiddot}[0]{{\ddot{\Xi}}}
\newcommand{\alphaddot}[0]{{\ddot{\alpha}}}
\newcommand{\betaddot}[0]{{\ddot{\beta}}}
\newcommand{\chiddot}[0]{{\ddot{\chi}}}
\newcommand{\deltaddot}[0]{{\ddot{\delta}}}
\newcommand{\epsilonddot}[0]{{\ddot{\epsilon}}}
\newcommand{\etaddot}[0]{{\ddot{\eta}}}
\newcommand{\gammaddot}[0]{{\ddot{\gamma}}}
\newcommand{\kappaddot}[0]{{\ddot{\kappa}}}
\newcommand{\lambdaddot}[0]{{\ddot{\lambda}}}
\newcommand{\muddot}[0]{{\ddot{\mu}}}
\newcommand{\nuddot}[0]{{\ddot{\nu}}}
\newcommand{\omegaddot}[0]{{\ddot{\omega}}}
\newcommand{\phiddot}[0]{{\ddot{\phi}}}
\newcommand{\piddot}[0]{{\ddot{\pi}}}
\newcommand{\psiddot}[0]{{\ddot{\psi}}}
\newcommand{\rhoddot}[0]{{\ddot{\rho}}}
\newcommand{\sigmaddot}[0]{{\ddot{\sigma}}}
\newcommand{\tauddot}[0]{{\ddot{\tau}}}
\newcommand{\thetaddot}[0]{{\ddot{\theta}}}
\newcommand{\upsilonddot}[0]{{\ddot{\upsilon}}}
\newcommand{\varepsilonddot}[0]{{\ddot{\varepsilon}}}
\newcommand{\varphiddot}[0]{{\ddot{\varphi}}}
\newcommand{\varpiddot}[0]{{\ddot{\varpi}}}
\newcommand{\varrhoddot}[0]{{\ddot{\varrho}}}
\newcommand{\varsigmaddot}[0]{{\ddot{\varsigma}}}
\newcommand{\varthetaddot}[0]{{\ddot{\vartheta}}}
\newcommand{\xiddot}[0]{{\ddot{\xi}}}
\newcommand{\zetaddot}[0]{{\ddot{\zeta}}}

\newcommand{\BDelta}[0]{\boldsymbol{\Delta}}
\newcommand{\BGamma}[0]{\boldsymbol{\Gamma}}
\newcommand{\BLambda}[0]{\boldsymbol{\Lambda}}
\newcommand{\BOmega}[0]{\boldsymbol{\Omega}}
\newcommand{\BPhi}[0]{\boldsymbol{\Phi}}
\newcommand{\BPi}[0]{\boldsymbol{\Pi}}
\newcommand{\BPsi}[0]{\boldsymbol{\Psi}}
\newcommand{\BSigma}[0]{\boldsymbol{\Sigma}}
\newcommand{\BTheta}[0]{\boldsymbol{\Theta}}
\newcommand{\BUpsilon}[0]{\boldsymbol{\Upsilon}}
\newcommand{\BXi}[0]{\boldsymbol{\Xi}}
\newcommand{\Balpha}[0]{\boldsymbol{\alpha}}
\newcommand{\Bbeta}[0]{\boldsymbol{\beta}}
\newcommand{\Bchi}[0]{\boldsymbol{\chi}}
\newcommand{\Bdelta}[0]{\boldsymbol{\delta}}
\newcommand{\Bepsilon}[0]{\boldsymbol{\epsilon}}
\newcommand{\Beta}[0]{\boldsymbol{\eta}}
\newcommand{\Bgamma}[0]{\boldsymbol{\gamma}}
\newcommand{\Bkappa}[0]{\boldsymbol{\kappa}}
\newcommand{\Blambda}[0]{\boldsymbol{\lambda}}
\newcommand{\Bmu}[0]{\boldsymbol{\mu}}
\newcommand{\Bnu}[0]{\boldsymbol{\nu}}
%\newcommand{\Bomega}[0]{\boldsymbol{\omega}}
\newcommand{\Bphi}[0]{\boldsymbol{\phi}}
\newcommand{\Bpi}[0]{\boldsymbol{\pi}}
\newcommand{\Bpsi}[0]{\boldsymbol{\psi}}
\newcommand{\Brho}[0]{\boldsymbol{\rho}}
\newcommand{\Bsigma}[0]{\boldsymbol{\sigma}}
%\newcommand{\Btau}[0]{\boldsymbol{\tau}}
%\newcommand{\Btheta}[0]{\boldsymbol{\theta}}
\newcommand{\Bupsilon}[0]{\boldsymbol{\upsilon}}
\newcommand{\Bvarepsilon}[0]{\boldsymbol{\varepsilon}}
\newcommand{\Bvarphi}[0]{\boldsymbol{\varphi}}
\newcommand{\Bvarpi}[0]{\boldsymbol{\varpi}}
\newcommand{\Bvarrho}[0]{\boldsymbol{\varrho}}
\newcommand{\Bvarsigma}[0]{\boldsymbol{\varsigma}}
\newcommand{\Bvartheta}[0]{\boldsymbol{\vartheta}}
\newcommand{\Bxi}[0]{\boldsymbol{\xi}}
\newcommand{\Bzeta}[0]{\boldsymbol{\zeta}}
%
%</bold and dot greek symbols>
%<infrequent>
%
%\newcommand{\AreaOp}[1]{\AName_{#1}}
%\newcommand{\Babs}[0]{\abs{\BB}}
%\newcommand{\Bcap}[0]{\hat{\BB}}
%\newcommand{\BrPrimeRej}[0]{\rcap(\rcap \wedge \Br')}
%\newcommand{\CA}[0]{\mathcal{A}}
%\newcommand{\Cos}[1]{\cos{\left({#1}\right)}}
%\newcommand{\Det}[1] {\abs{#1}}
%\newcommand{\Dsq}[2] {\frac {\partial^2 {#1}} {\partial {#2}^2}}
%\newcommand{\Exp}[1]{\exp{\left({#1}\right)}}
%\newcommand{\Norm}[1]{\left\lVert{#1}\right\rVert}
%\newcommand{\Sin}[1]{\sin{\left({#1}\right)}}
%\newcommand{\T}[0]{\text{T}}
%\newcommand{\VolumeOp}[1]{\VName_{#1}}
%\newcommand{\agrad}[0]{\Ba \cdot \nabla}
%\newcommand{\alphacap}[0]{\hat{\boldsymbol{\alpha}}}
%\newcommand{\Fcap}[0]{\hat{\BF}}
%\newcommand{\bithree}[0]{{\Bi}_3}
%\newcommand{\bxa}[0]{\Bx\Ba}
%\newcommand{\coordvec}[2]{
%\newcommand{\costheta}[0]{\acap \cdot \xcap}
%\newcommand{\ddt}[1]{\ddot{#1}}
%\newcommand{\ddu}[1] {\frac {d{#1}} {du}}
%\newcommand{\dsqxj}[2] {\frac {\partial^2 {#1}} {\partial {x_{#2}}^2}}
%\newcommand{\dtheta}[1]{\frac{d {#1}}{d \theta}}
%\newcommand{\dt}[1]{\dot{#1}}
%\newcommand{\dt}[1]{\frac{d {#1}}{dt}}
%\newcommand{\dxj}[2] {\frac {\partial {#1}} {\partial {x_{#2}}}}
%\newcommand{\halfPhi}[0]{\frac{\phi}{2}}
%\newcommand{\half}[0]{\inv{2}}
%\newcommand{\inv}[1]{\frac{1}{#1}}
%\newcommand{\laplacian}[0]{\nabla^2}
%\newcommand{\matrixoftx}[3]{
%\newcommand{\nrrp}[0]{\norm{\rcap \wedge \Br'}}
%\newcommand{\oiint}{\bigcirc \hspace{-1.4em} \int \hspace{-.8em} \int}
%\newcommand{\transpose}[1]{{#1}^{\text{T}}}
%\newcommand{\transpose}[1]{{{#1}^{\TextTranspose}}}
%\newcommand{\transpose}[1]{{{#1}^{\text{T}}}}
%\newcommand{\barA}[0]{\bar{A}}
%\newcommand{\qbar}[0]{\bar{q}}
%\newcommand{\qdotbar}[0]{\dot{\bar{q}}}
%
%</infrequent>





\usepackage[bookmarks=true]{hyperref}

\usepackage{color,cite,graphicx}
   % use colour in the document, put your citations as [1-4]
   % rather than [1,2,3,4] (it looks nicer, and the extended LaTeX2e
   % graphics package. 
\usepackage{latexsym,amssymb,epsf} % don't remember if these are
   % needed, but their inclusion can't do any damage


\title{ Attempt to make sense of fourier form of Green's function for the Poisson equation. }
\author{Peeter Joot}
\date{ Feb 18, 2009.  Last Revision: $Date: 2009/02/19 05:15:56 $ }

\begin{document}

\maketitle{}
\tableofcontents

%\section{}

Am just playing around, and 
following examples of Fourier transform solutions of the heat equation, tried the same thing for 
the electrostatics Poisson equation
\begin{align*}
\grad^2 \phi &= -\rho/\epsilon_0 \\
\end{align*}

With fourier transform pairs
\begin{align*}
\hat{f}(\mathbf{k}) &= \frac{1}{\sqrt{2\pi}} \iiint f(\mathbf{x}) e^{-i \mathbf{k} \cdot \mathbf{x} } d^3 x \\
{f}(\mathbf{x}) &= \frac{1}{\sqrt{2\pi}} \iiint \hat{f}(\mathbf{k}) e^{i \mathbf{k} \cdot \mathbf{x} } d^3 k \\
\end{align*}

one gets 

\begin{align*}
\phi(\mathbf{x}) &= \frac{1}{\epsilon_0} \int \rho(\mathbf{x}' G(\mathbf{x-x'}) d^3 x' \\
G(\mathbf{x}) &= \frac{1}{(2 \pi)^3} \iiint \frac{1}{\mathbf{k}^2} e^{ i \mathbf{k} \cdot \mathbf{x} } d^3 k
\end{align*}

Now it seems to me that this integral $G$ only has to be evaluated around a small neighbourhood of the origin.  For example if one evaluates one of
the
integrals  
\begin{align*}
\int_{-\infty}^\infty \frac{1}{{k_1}^2 + {k_2}^2 + {k_3}^3 } e^{ i k_1 x_1 } dk_1 
\end{align*}

using a an upper half plane contour the result is zero unless $k_2 = k_3 = 0$.  So one is left with something loosely like

\begin{align*}
G(\mathbf{x}) &= \lim_{\epsilon \rightarrow 0} \frac{1}{(2 \pi)^3} 
\int_{k_1 = -\epsilon}^{\epsilon} dk_1
\int_{k_2 = -\epsilon}^{\epsilon} dk_2
\int_{k_3 = -\epsilon}^{\epsilon} dk_3
 \frac{1}{\mathbf{k}^2} e^{ i \mathbf{k} \cdot \mathbf{x} } 
\end{align*}

However, from electrostatics we also know that the solution to the Poission equation means that $G(\mathbf{x}) = \frac{1}{4\pi\lvert{\mathbf{x}}\rvert}$.
Does anybody know of a technique that would reduce the integral limit expression above for $G$ to the $1/x$ form?  I've played around with this for a bit
without any success.

\bibliographystyle{plainnat}
\bibliography{myrefs}

\end{document}

\documentclass{article}

\usepackage{amsmath}
\usepackage{mathpazo}

%
% shorthand for bold symbols, convenient for vectors and matrices
%
\newcommand{\Ba}[0]{\mathbf{a}}
\newcommand{\Bb}[0]{\mathbf{b}}
\newcommand{\Bc}[0]{\mathbf{c}}
\newcommand{\Bd}[0]{\mathbf{d}}
\newcommand{\Be}[0]{\mathbf{e}}
\newcommand{\Bf}[0]{\mathbf{f}}
\newcommand{\Bg}[0]{\mathbf{g}}
\newcommand{\Bh}[0]{\mathbf{h}}
\newcommand{\Bi}[0]{\mathbf{i}}
\newcommand{\Bj}[0]{\mathbf{j}}
\newcommand{\Bk}[0]{\mathbf{k}}
\newcommand{\Bl}[0]{\mathbf{l}}
\newcommand{\Bm}[0]{\mathbf{m}}
\newcommand{\Bn}[0]{\mathbf{n}}
\newcommand{\Bo}[0]{\mathbf{o}}
\newcommand{\Bp}[0]{\mathbf{p}}
\newcommand{\Bq}[0]{\mathbf{q}}
\newcommand{\Br}[0]{\mathbf{r}}
\newcommand{\Bs}[0]{\mathbf{s}}
\newcommand{\Bt}[0]{\mathbf{t}}
\newcommand{\Bu}[0]{\mathbf{u}}
\newcommand{\Bv}[0]{\mathbf{v}}
\newcommand{\Bw}[0]{\mathbf{w}}
\newcommand{\Bx}[0]{\mathbf{x}}
\newcommand{\By}[0]{\mathbf{y}}
\newcommand{\Bz}[0]{\mathbf{z}}
\newcommand{\BA}[0]{\mathbf{A}}
\newcommand{\BB}[0]{\mathbf{B}}
\newcommand{\BC}[0]{\mathbf{C}}
\newcommand{\BD}[0]{\mathbf{D}}
\newcommand{\BE}[0]{\mathbf{E}}
\newcommand{\BF}[0]{\mathbf{F}}
\newcommand{\BG}[0]{\mathbf{G}}
\newcommand{\BH}[0]{\mathbf{H}}
\newcommand{\BI}[0]{\mathbf{I}}
\newcommand{\BJ}[0]{\mathbf{J}}
\newcommand{\BK}[0]{\mathbf{K}}
\newcommand{\BL}[0]{\mathbf{L}}
\newcommand{\BM}[0]{\mathbf{M}}
\newcommand{\BN}[0]{\mathbf{N}}
\newcommand{\BO}[0]{\mathbf{O}}
\newcommand{\BP}[0]{\mathbf{P}}
\newcommand{\BQ}[0]{\mathbf{Q}}
\newcommand{\BR}[0]{\mathbf{R}}
\newcommand{\BS}[0]{\mathbf{S}}
\newcommand{\BT}[0]{\mathbf{T}}
\newcommand{\BU}[0]{\mathbf{U}}
\newcommand{\BV}[0]{\mathbf{V}}
\newcommand{\BW}[0]{\mathbf{W}}
\newcommand{\BX}[0]{\mathbf{X}}
\newcommand{\BY}[0]{\mathbf{Y}}
\newcommand{\BZ}[0]{\mathbf{Z}}

\newcommand{\Bzero}[0]{\mathbf{0}}
\newcommand{\Btheta}[0]{\boldsymbol{\theta}}
\newcommand{\Btau}[0]{\boldsymbol{\tau}}
\newcommand{\Bomega}[0]{\boldsymbol{\omega}}

%
% shorthand for unit vectors
%
\newcommand{\acap}[0]{\hat{\Ba}}
\newcommand{\bcap}[0]{\hat{\Bb}}
\newcommand{\ccap}[0]{\hat{\Bc}}
\newcommand{\dcap}[0]{\hat{\Bd}}
\newcommand{\ecap}[0]{\hat{\Be}}
\newcommand{\fcap}[0]{\hat{\Bf}}
\newcommand{\gcap}[0]{\hat{\Bg}}
\newcommand{\hcap}[0]{\hat{\Bh}}
\newcommand{\icap}[0]{\hat{\Bi}}
\newcommand{\jcap}[0]{\hat{\Bj}}
\newcommand{\kcap}[0]{\hat{\Bk}}
\newcommand{\lcap}[0]{\hat{\Bl}}
\newcommand{\mcap}[0]{\hat{\Bm}}
\newcommand{\ncap}[0]{\hat{\Bn}}
\newcommand{\ocap}[0]{\hat{\Bo}}
\newcommand{\pcap}[0]{\hat{\Bp}}
\newcommand{\qcap}[0]{\hat{\Bq}}
\newcommand{\rcap}[0]{\hat{\Br}}
\newcommand{\scap}[0]{\hat{\Bs}}
\newcommand{\tcap}[0]{\hat{\Bt}}
\newcommand{\ucap}[0]{\hat{\Bu}}
\newcommand{\vcap}[0]{\hat{\Bv}}
\newcommand{\wcap}[0]{\hat{\Bw}}
\newcommand{\xcap}[0]{\hat{\Bx}}
\newcommand{\ycap}[0]{\hat{\By}}
\newcommand{\zcap}[0]{\hat{\Bz}}
\newcommand{\thetacap}[0]{\hat{\Btheta}}

%
% to write R^n and C^n in a distinguishable fashion.  Perhaps change this
% to the double lined characters upon figuring out how to do so.
%
\newcommand{\C}[1]{$\mathbb{C}^{#1}$}
\newcommand{\R}[1]{$\mathbb{R}^{#1}$}

%
% various generally useful helpers
%

% derivative of #1 wrt. #2:
\newcommand{\D}[2] {\frac {d#2} {d#1}}

\newcommand{\inv}[1]{\frac{1}{#1}}
\newcommand{\cross}[0]{\times}

\newcommand{\abs}[1]{\lvert{#1}\rvert}
\newcommand{\norm}[1]{\lVert{#1}\rVert}
\newcommand{\innerprod}[2]{\langle{#1}, {#2}\rangle}
\newcommand{\dotprod}[2]{{#1} \cdot {#2}}
\newcommand{\bdotprod}[2]{\left({#1} \cdot {#2}\right)}
\newcommand{\crossprod}[2]{{#1} \cross {#2}}
\newcommand{\tripleprod}[3]{\dotprod{\left(\crossprod{#1}{#2}\right)}{#3}}

\DeclareMathOperator{\Proj}{Proj}
\DeclareMathOperator{\Span}{span}
\DeclareMathOperator{\Sgn}{sgn}
\DeclareMathOperator{\Area}{Area}
\DeclareMathOperator{\Volume}{Volume}

%
% A few miscellaneous things specific to this document
%
\newcommand{\crossop}[1]{\crossprod{#1}{}}

% R2 vector.
\newcommand{\VectorTwo}[2]{
\begin{bmatrix}
 {#1} \\
 {#2}
\end{bmatrix}
}

\newcommand{\VectorN}[1]{
\begin{bmatrix}
{#1}_1 \\
{#1}_2 \\
\vdots \\
{#1}_N \\
\end{bmatrix}
}

\newcommand{\DETuvij}[4]{
\begin{vmatrix}
 {#1}_{#3} & {#1}_{#4} \\
 {#2}_{#3} & {#2}_{#4}
\end{vmatrix}
}

\newcommand{\DETuvwijk}[6]{
\begin{vmatrix}
 {#1}_{#4} & {#1}_{#5} & {#1}_{#6} \\
 {#2}_{#4} & {#2}_{#5} & {#2}_{#6} \\
 {#3}_{#4} & {#3}_{#5} & {#3}_{#6}
\end{vmatrix}
}

\newcommand{\DETuvwxijkl}[8]{
\begin{vmatrix}
 {#1}_{#5} & {#1}_{#6} & {#1}_{#7} & {#1}_{#8} \\
 {#2}_{#5} & {#2}_{#6} & {#2}_{#7} & {#2}_{#8} \\
 {#3}_{#5} & {#3}_{#6} & {#3}_{#7} & {#3}_{#8} \\
 {#4}_{#5} & {#4}_{#6} & {#4}_{#7} & {#4}_{#8} \\
\end{vmatrix}
}

%\newcommand{\DETuvwxyijklm}[10]{
%\begin{vmatrix}
% {#1}_{#6} & {#1}_{#7} & {#1}_{#8} & {#1}_{#9} & {#1}_{#10} \\
% {#2}_{#6} & {#2}_{#7} & {#2}_{#8} & {#2}_{#9} & {#2}_{#10} \\
% {#3}_{#6} & {#3}_{#7} & {#3}_{#8} & {#3}_{#9} & {#3}_{#10} \\
% {#4}_{#6} & {#4}_{#7} & {#4}_{#8} & {#4}_{#9} & {#4}_{#10} \\
% {#5}_{#6} & {#5}_{#7} & {#5}_{#8} & {#5}_{#9} & {#5}_{#10}
%\end{vmatrix}
%}

% R3 vector.
\newcommand{\VectorThree}[3]{
\begin{bmatrix}
 {#1} \\
 {#2} \\
 {#3}
\end{bmatrix}
}


%<misc>
%
\newcommand{\Abs}[1]{{\left\lvert{#1}\right\rvert}}
\newcommand{\spacegrad}[0]{\boldsymbol{\nabla}}
\newcommand{\grad}[0]{\nabla}
\newcommand{\LL}[0]{\mathcal{L}}

% == \partial_{#1} {#2}
\newcommand{\PD}[2]{\frac{\partial {#2}}{\partial {#1}}}
% inline variant
\newcommand{\PDi}[2]{{\partial {#2}}/{\partial {#1}}}

\newcommand{\PDD}[3]{\frac{\partial^2 {#3}}{\partial {#1}\partial {#2}}}
%\newcommand{\PDd}[2]{\frac{\partial^2 {#2}}{{\partial{#1}}^2}}
\newcommand{\PDsq}[2]{\frac{\partial^2 {#2}}{(\partial {#1})^2}}

\newcommand{\Partial}[2]{\frac{\partial {#1}}{\partial {#2}}}
\DeclareMathOperator{\RejName}{Rej}
\newcommand{\Rej}[2]{\RejName_{#1}\left( {#2} \right)}
\newcommand{\Rm}[1]{\mathbb{R}^{#1}}
\newcommand{\Cm}[1]{\mathbb{C}^{#1}}
\newcommand{\conj}[0]{{*}}

%</misc>

% <grade selection>
%
\newcommand{\gpgrade}[2] {{\left\langle{{#1}}\right\rangle}_{#2}}

\newcommand{\gpgradezero}[1] {\gpgrade{#1}{}}
%\newcommand{\gpscalargrade}[1] {{\left\langle{{#1}}\right\rangle}}
%\newcommand{\gpgradezero}[1] {\gpgrade{#1}{0}}

%\newcommand{\gpgradeone}[1] {{\left\langle{{#1}}\right\rangle}_{1}}
\newcommand{\gpgradeone}[1] {\gpgrade{#1}{1}}

\newcommand{\gpgradetwo}[1] {\gpgrade{#1}{2}}
\newcommand{\gpgradethree}[1] {\gpgrade{#1}{3}}
\newcommand{\gpgradefour}[1] {\gpgrade{#1}{4}}
%
% </grade selection>



\newcommand{\adot}[0]{{\dot{a}}}
\newcommand{\bdot}[0]{{\dot{b}}}
% taken for centered dot:
%\newcommand{\cdot}[0]{{\dot{c}}}
%\newcommand{\ddot}[0]{{\dot{d}}}
\newcommand{\edot}[0]{{\dot{e}}}
\newcommand{\fdot}[0]{{\dot{f}}}
\newcommand{\gdot}[0]{{\dot{g}}}
\newcommand{\hdot}[0]{{\dot{h}}}
\newcommand{\idot}[0]{{\dot{i}}}
\newcommand{\jdot}[0]{{\dot{j}}}
\newcommand{\kdot}[0]{{\dot{k}}}
\newcommand{\ldot}[0]{{\dot{l}}}
\newcommand{\mdot}[0]{{\dot{m}}}
\newcommand{\ndot}[0]{{\dot{n}}}
%\newcommand{\odot}[0]{{\dot{o}}}
\newcommand{\pdot}[0]{{\dot{p}}}
\newcommand{\qdot}[0]{{\dot{q}}}
\newcommand{\rdot}[0]{{\dot{r}}}
\newcommand{\sdot}[0]{{\dot{s}}}
\newcommand{\tdot}[0]{{\dot{t}}}
\newcommand{\udot}[0]{{\dot{u}}}
\newcommand{\vdot}[0]{{\dot{v}}}
\newcommand{\wdot}[0]{{\dot{w}}}
\newcommand{\xdot}[0]{{\dot{x}}}
\newcommand{\ydot}[0]{{\dot{y}}}
\newcommand{\zdot}[0]{{\dot{z}}}
\newcommand{\addot}[0]{{\ddot{a}}}
\newcommand{\bddot}[0]{{\ddot{b}}}
\newcommand{\cddot}[0]{{\ddot{c}}}
%\newcommand{\dddot}[0]{{\ddot{d}}}
\newcommand{\eddot}[0]{{\ddot{e}}}
\newcommand{\fddot}[0]{{\ddot{f}}}
\newcommand{\gddot}[0]{{\ddot{g}}}
\newcommand{\hddot}[0]{{\ddot{h}}}
\newcommand{\iddot}[0]{{\ddot{i}}}
\newcommand{\jddot}[0]{{\ddot{j}}}
\newcommand{\kddot}[0]{{\ddot{k}}}
\newcommand{\lddot}[0]{{\ddot{l}}}
\newcommand{\mddot}[0]{{\ddot{m}}}
\newcommand{\nddot}[0]{{\ddot{n}}}
\newcommand{\oddot}[0]{{\ddot{o}}}
\newcommand{\pddot}[0]{{\ddot{p}}}
\newcommand{\qddot}[0]{{\ddot{q}}}
\newcommand{\rddot}[0]{{\ddot{r}}}
\newcommand{\sddot}[0]{{\ddot{s}}}
\newcommand{\tddot}[0]{{\ddot{t}}}
\newcommand{\uddot}[0]{{\ddot{u}}}
\newcommand{\vddot}[0]{{\ddot{v}}}
\newcommand{\wddot}[0]{{\ddot{w}}}
\newcommand{\xddot}[0]{{\ddot{x}}}
\newcommand{\yddot}[0]{{\ddot{y}}}
\newcommand{\zddot}[0]{{\ddot{z}}}

%<bold and dot greek symbols>
%

\newcommand{\Deltadot}[0]{{\dot{\Delta}}}
\newcommand{\Gammadot}[0]{{\dot{\Gamma}}}
\newcommand{\Lambdadot}[0]{{\dot{\Lambda}}}
\newcommand{\Omegadot}[0]{{\dot{\Omega}}}
\newcommand{\Phidot}[0]{{\dot{\Phi}}}
\newcommand{\Pidot}[0]{{\dot{\Pi}}}
\newcommand{\Psidot}[0]{{\dot{\Psi}}}
\newcommand{\Sigmadot}[0]{{\dot{\Sigma}}}
\newcommand{\Thetadot}[0]{{\dot{\Theta}}}
\newcommand{\Upsilondot}[0]{{\dot{\Upsilon}}}
\newcommand{\Xidot}[0]{{\dot{\Xi}}}
\newcommand{\alphadot}[0]{{\dot{\alpha}}}
\newcommand{\betadot}[0]{{\dot{\beta}}}
\newcommand{\chidot}[0]{{\dot{\chi}}}
\newcommand{\deltadot}[0]{{\dot{\delta}}}
\newcommand{\epsilondot}[0]{{\dot{\epsilon}}}
\newcommand{\etadot}[0]{{\dot{\eta}}}
\newcommand{\gammadot}[0]{{\dot{\gamma}}}
\newcommand{\kappadot}[0]{{\dot{\kappa}}}
\newcommand{\lambdadot}[0]{{\dot{\lambda}}}
\newcommand{\mudot}[0]{{\dot{\mu}}}
\newcommand{\nudot}[0]{{\dot{\nu}}}
\newcommand{\omegadot}[0]{{\dot{\omega}}}
\newcommand{\phidot}[0]{{\dot{\phi}}}
\newcommand{\pidot}[0]{{\dot{\pi}}}
\newcommand{\psidot}[0]{{\dot{\psi}}}
\newcommand{\rhodot}[0]{{\dot{\rho}}}
\newcommand{\sigmadot}[0]{{\dot{\sigma}}}
\newcommand{\taudot}[0]{{\dot{\tau}}}
\newcommand{\thetadot}[0]{{\dot{\theta}}}
\newcommand{\upsilondot}[0]{{\dot{\upsilon}}}
\newcommand{\varepsilondot}[0]{{\dot{\varepsilon}}}
\newcommand{\varphidot}[0]{{\dot{\varphi}}}
\newcommand{\varpidot}[0]{{\dot{\varpi}}}
\newcommand{\varrhodot}[0]{{\dot{\varrho}}}
\newcommand{\varsigmadot}[0]{{\dot{\varsigma}}}
\newcommand{\varthetadot}[0]{{\dot{\vartheta}}}
\newcommand{\xidot}[0]{{\dot{\xi}}}
\newcommand{\zetadot}[0]{{\dot{\zeta}}}

\newcommand{\Deltaddot}[0]{{\ddot{\Delta}}}
\newcommand{\Gammaddot}[0]{{\ddot{\Gamma}}}
\newcommand{\Lambdaddot}[0]{{\ddot{\Lambda}}}
\newcommand{\Omegaddot}[0]{{\ddot{\Omega}}}
\newcommand{\Phiddot}[0]{{\ddot{\Phi}}}
\newcommand{\Piddot}[0]{{\ddot{\Pi}}}
\newcommand{\Psiddot}[0]{{\ddot{\Psi}}}
\newcommand{\Sigmaddot}[0]{{\ddot{\Sigma}}}
\newcommand{\Thetaddot}[0]{{\ddot{\Theta}}}
\newcommand{\Upsilonddot}[0]{{\ddot{\Upsilon}}}
\newcommand{\Xiddot}[0]{{\ddot{\Xi}}}
\newcommand{\alphaddot}[0]{{\ddot{\alpha}}}
\newcommand{\betaddot}[0]{{\ddot{\beta}}}
\newcommand{\chiddot}[0]{{\ddot{\chi}}}
\newcommand{\deltaddot}[0]{{\ddot{\delta}}}
\newcommand{\epsilonddot}[0]{{\ddot{\epsilon}}}
\newcommand{\etaddot}[0]{{\ddot{\eta}}}
\newcommand{\gammaddot}[0]{{\ddot{\gamma}}}
\newcommand{\kappaddot}[0]{{\ddot{\kappa}}}
\newcommand{\lambdaddot}[0]{{\ddot{\lambda}}}
\newcommand{\muddot}[0]{{\ddot{\mu}}}
\newcommand{\nuddot}[0]{{\ddot{\nu}}}
\newcommand{\omegaddot}[0]{{\ddot{\omega}}}
\newcommand{\phiddot}[0]{{\ddot{\phi}}}
\newcommand{\piddot}[0]{{\ddot{\pi}}}
\newcommand{\psiddot}[0]{{\ddot{\psi}}}
\newcommand{\rhoddot}[0]{{\ddot{\rho}}}
\newcommand{\sigmaddot}[0]{{\ddot{\sigma}}}
\newcommand{\tauddot}[0]{{\ddot{\tau}}}
\newcommand{\thetaddot}[0]{{\ddot{\theta}}}
\newcommand{\upsilonddot}[0]{{\ddot{\upsilon}}}
\newcommand{\varepsilonddot}[0]{{\ddot{\varepsilon}}}
\newcommand{\varphiddot}[0]{{\ddot{\varphi}}}
\newcommand{\varpiddot}[0]{{\ddot{\varpi}}}
\newcommand{\varrhoddot}[0]{{\ddot{\varrho}}}
\newcommand{\varsigmaddot}[0]{{\ddot{\varsigma}}}
\newcommand{\varthetaddot}[0]{{\ddot{\vartheta}}}
\newcommand{\xiddot}[0]{{\ddot{\xi}}}
\newcommand{\zetaddot}[0]{{\ddot{\zeta}}}

\newcommand{\BDelta}[0]{\boldsymbol{\Delta}}
\newcommand{\BGamma}[0]{\boldsymbol{\Gamma}}
\newcommand{\BLambda}[0]{\boldsymbol{\Lambda}}
\newcommand{\BOmega}[0]{\boldsymbol{\Omega}}
\newcommand{\BPhi}[0]{\boldsymbol{\Phi}}
\newcommand{\BPi}[0]{\boldsymbol{\Pi}}
\newcommand{\BPsi}[0]{\boldsymbol{\Psi}}
\newcommand{\BSigma}[0]{\boldsymbol{\Sigma}}
\newcommand{\BTheta}[0]{\boldsymbol{\Theta}}
\newcommand{\BUpsilon}[0]{\boldsymbol{\Upsilon}}
\newcommand{\BXi}[0]{\boldsymbol{\Xi}}
\newcommand{\Balpha}[0]{\boldsymbol{\alpha}}
\newcommand{\Bbeta}[0]{\boldsymbol{\beta}}
\newcommand{\Bchi}[0]{\boldsymbol{\chi}}
\newcommand{\Bdelta}[0]{\boldsymbol{\delta}}
\newcommand{\Bepsilon}[0]{\boldsymbol{\epsilon}}
\newcommand{\Beta}[0]{\boldsymbol{\eta}}
\newcommand{\Bgamma}[0]{\boldsymbol{\gamma}}
\newcommand{\Bkappa}[0]{\boldsymbol{\kappa}}
\newcommand{\Blambda}[0]{\boldsymbol{\lambda}}
\newcommand{\Bmu}[0]{\boldsymbol{\mu}}
\newcommand{\Bnu}[0]{\boldsymbol{\nu}}
%\newcommand{\Bomega}[0]{\boldsymbol{\omega}}
\newcommand{\Bphi}[0]{\boldsymbol{\phi}}
\newcommand{\Bpi}[0]{\boldsymbol{\pi}}
\newcommand{\Bpsi}[0]{\boldsymbol{\psi}}
\newcommand{\Brho}[0]{\boldsymbol{\rho}}
\newcommand{\Bsigma}[0]{\boldsymbol{\sigma}}
%\newcommand{\Btau}[0]{\boldsymbol{\tau}}
%\newcommand{\Btheta}[0]{\boldsymbol{\theta}}
\newcommand{\Bupsilon}[0]{\boldsymbol{\upsilon}}
\newcommand{\Bvarepsilon}[0]{\boldsymbol{\varepsilon}}
\newcommand{\Bvarphi}[0]{\boldsymbol{\varphi}}
\newcommand{\Bvarpi}[0]{\boldsymbol{\varpi}}
\newcommand{\Bvarrho}[0]{\boldsymbol{\varrho}}
\newcommand{\Bvarsigma}[0]{\boldsymbol{\varsigma}}
\newcommand{\Bvartheta}[0]{\boldsymbol{\vartheta}}
\newcommand{\Bxi}[0]{\boldsymbol{\xi}}
\newcommand{\Bzeta}[0]{\boldsymbol{\zeta}}
%
%</bold and dot greek symbols>
%<infrequent>
%
%\newcommand{\AreaOp}[1]{\AName_{#1}}
%\newcommand{\Babs}[0]{\abs{\BB}}
%\newcommand{\Bcap}[0]{\hat{\BB}}
%\newcommand{\BrPrimeRej}[0]{\rcap(\rcap \wedge \Br')}
%\newcommand{\CA}[0]{\mathcal{A}}
%\newcommand{\Cos}[1]{\cos{\left({#1}\right)}}
%\newcommand{\Det}[1] {\abs{#1}}
%\newcommand{\Dsq}[2] {\frac {\partial^2 {#1}} {\partial {#2}^2}}
%\newcommand{\Exp}[1]{\exp{\left({#1}\right)}}
%\newcommand{\Norm}[1]{\left\lVert{#1}\right\rVert}
%\newcommand{\Sin}[1]{\sin{\left({#1}\right)}}
%\newcommand{\T}[0]{\text{T}}
%\newcommand{\VolumeOp}[1]{\VName_{#1}}
%\newcommand{\agrad}[0]{\Ba \cdot \nabla}
%\newcommand{\alphacap}[0]{\hat{\boldsymbol{\alpha}}}
%\newcommand{\Fcap}[0]{\hat{\BF}}
%\newcommand{\bithree}[0]{{\Bi}_3}
%\newcommand{\bxa}[0]{\Bx\Ba}
%\newcommand{\coordvec}[2]{
%\newcommand{\costheta}[0]{\acap \cdot \xcap}
%\newcommand{\ddt}[1]{\ddot{#1}}
%\newcommand{\ddu}[1] {\frac {d{#1}} {du}}
%\newcommand{\dsqxj}[2] {\frac {\partial^2 {#1}} {\partial {x_{#2}}^2}}
%\newcommand{\dtheta}[1]{\frac{d {#1}}{d \theta}}
%\newcommand{\dt}[1]{\dot{#1}}
%\newcommand{\dt}[1]{\frac{d {#1}}{dt}}
%\newcommand{\dxj}[2] {\frac {\partial {#1}} {\partial {x_{#2}}}}
%\newcommand{\halfPhi}[0]{\frac{\phi}{2}}
%\newcommand{\half}[0]{\inv{2}}
%\newcommand{\inv}[1]{\frac{1}{#1}}
%\newcommand{\laplacian}[0]{\nabla^2}
%\newcommand{\matrixoftx}[3]{
%\newcommand{\nrrp}[0]{\norm{\rcap \wedge \Br'}}
%\newcommand{\oiint}{\bigcirc \hspace{-1.4em} \int \hspace{-.8em} \int}
%\newcommand{\transpose}[1]{{#1}^{\text{T}}}
%\newcommand{\transpose}[1]{{{#1}^{\TextTranspose}}}
%\newcommand{\transpose}[1]{{{#1}^{\text{T}}}}
%\newcommand{\barA}[0]{\bar{A}}
%\newcommand{\qbar}[0]{\bar{q}}
%\newcommand{\qdotbar}[0]{\dot{\bar{q}}}
%
%</infrequent>





\newcommand{\PDSq}[2]{\frac{\partial^2 {#2}}{\partial {#1}^2}}
\DeclareMathOperator{\sinc}{sinc}
\DeclareMathOperator{\PV}{PV}
\newcommand{\FF}[0]{\mathcal{F}}
\newcommand{\IIinf}[0]{ \int_{-\infty}^\infty }

\usepackage[bookmarks=true]{hyperref}

\usepackage{color,cite,graphicx}
   % use colour in the document, put your citations as [1-4]
   % rather than [1,2,3,4] (it looks nicer, and the extended LaTeX2e
   % graphics package. 
\usepackage{latexsym,amssymb,epsf} % don't remember if these are
   % needed, but their inclusion can't do any damage


\title{ Fourier transform solutions to the wave equation. }
\author{Peeter Joot}
\date{ Jan 26, 2009.  Last Revision: $Date: 2009/01/26 14:31:58 $ }

\begin{document}

\maketitle{}

%\tableofcontents
\section{ Mechanical wave equation solution. }

We want to solve

\begin{align}\label{eqn:waveTwoDim}
\left(\inv{v^2} \partial_{tt} - \partial_{xx}\right) \psi = 0
\end{align}

A separation of variables treatment of this has been done in
\cite{PJwaveFourVector}, and some logical followup for that done in
\cite{PJemWave} in the context of Maxwell's equation for the vacuum field.

Here the Fourier transform will be used as a tool.

\section{ do it. }

Following the heat equation treatment in \cite{PJheatFourier}, we take Fourier transforms 
of both parts of \ref{eqn:waveTwoDim}.

\begin{align*}
\FF\left( \inv{v^2} \partial_{tt} \psi \right) = \FF\left( \partial_{xx} \psi \right)
\end{align*}

For the $x$ derivatives we can integrate by parts twice

\begin{align*}
\FF\left( \partial_{xx} \psi \right)
&= \inv{\sqrt{2 \pi}} \IIinf \left( \partial_{xx} \psi \right) \exp\left( -i k x \right) dx \\
&= -\inv{\sqrt{2 \pi}} \IIinf \left( \partial_{x} \psi \right) \partial_x\left(\exp\left( -i k x \right) \right) dx \\
&= -\frac{-i k}{\sqrt{2 \pi}} \IIinf \left( \partial_{x} \psi \right) \exp\left( -i k x \right) dx \\
&= \frac{(-i k)^2}{\sqrt{2 \pi}} \IIinf \psi \exp\left( -i k x \right) dx \\
\end{align*}

Note that this integration by parts requires that $\partial_x \psi = \psi = 0$ at $\pm \infty$.  We are left with 

\begin{align*}
\FF\left( \partial_{xx} \psi \right) &= -k^2 \hat{\psi}(k, t)
\end{align*}

Now, for the left hand side, for the fourier transform of the time partials we can pull the derivative operation out of the 
integral

\begin{align*}
\FF\left( \inv{v^2} \partial_{tt} \psi \right) 
&= \inv{\sqrt{2 \pi}} \IIinf \left( \inv{v^2} \partial_{tt} \psi \right) \exp\left( -i k x \right) dx \\
&= \inv{v^2} \partial_{tt} \hat{\psi}(k,t) \\
\end{align*}

We are left with our harmonic oscillator differential equation for the transformed wave function

\begin{align*}
\inv{v^2} \partial_{tt} \hat{\psi}(k,t) &= -k^2 \hat{\psi}(k, t).
\end{align*}

Since we have a partial differential equation, for the integration constant we are free to pick any function of $k$.  The solutions of this are therefore of the form

\begin{align*}
\hat{\psi}(k,t) &= A(k) \exp\left( \pm i v k t \right)
\end{align*}

Performing an inverse Fourier transform we now have the wave equation expressed in terms of this unknown (so far) frequency domain function $A(k)$.  That is

\begin{align}\label{eqn:almostThere}
{\psi}(x,t) &= \inv{\sqrt{2\pi}} \IIinf A(k) \exp\left( \pm i v k t + i k x \right) dk
\end{align}

Now, suppose we fix the boundary value conditions by employing a known value of the wave function at $t = 0$, say $\psi(x,0) = \phi(x)$.  We then have

\begin{align*}
{\phi}(x) &= \inv{\sqrt{2\pi}} \IIinf A(k) \exp\left( i k x \right) dk
\end{align*}

From which we have $A(k)$ in terms of $\phi$ by inverse transform

\begin{align}\label{eqn:freqInitial}
A(k) &= \inv{\sqrt{2\pi}} \IIinf \phi(x) \exp\left( -i k x \right) dx
\end{align}

One could consider the problem fully solved at this point, but it can be carried further.  Let's substitute 
\ref{eqn:freqInitial} back into \ref{eqn:almostThere}.  This is

\begin{align*}
{\psi}(x,t) &= \inv{\sqrt{2\pi}} \IIinf \left( \inv{\sqrt{2\pi}} \IIinf \phi(u) \exp\left( -i k u \right) du \right) \exp\left( \pm i v k t + i k x \right) dk
\end{align*}

With the Rigor police on holiday, exchange the order of integration

\begin{align*}
{\psi}(x,t) 
&= \IIinf \phi(u) du \inv{{2\pi}} \IIinf \exp\left( -i k u \pm i v k t + i k x \right) dk \\
&= \IIinf \phi(u) du \inv{{2\pi}} \IIinf \exp\left( i k (x - u \pm v t ) \right) dk \\
\end{align*}

The principle value of this inner integral is

\begin{align*}
\PV \inv{{2\pi}} \IIinf \exp\left( i k (x - u \pm v t ) \right) dk 
&= \lim_{R\rightarrow \infty} \inv{2\pi} \int_{-R}^R \exp\left( i k (x - u \pm v t ) \right) dk  \\
&= \lim_{R\rightarrow \infty} \frac{\sin\left( R (x - u \pm v t ) \right) }{ \pi (x - u \pm v t) } \\
\end{align*}

And here we make the usual identification with the delta function $\delta( x - u \pm v t )$.  We are left with

\begin{align*}
{\psi}(x,t) 
&= \IIinf \phi(u) \delta( x - u \pm v t ) du \\
&= \phi( x \pm v t )
\end{align*}

We find, amazingly enough, just by application of the Fourier transform, that the time evolution of the
wave function follows propagation of the initial wave packet down the x-axis in one of the two directions with velocity $v$.

This is a statement well known to any first year student taking a vibrations and waves course, but it is nice to see 
it follow from the straightforward application of transform techniques straight out of the Engineer's toolbox.

\bibliographystyle{plainnat}
\bibliography{myrefs}

\end{document}

%
% Copyright � 2012 Peeter Joot.  All Rights Reserved.
% Licenced as described in the file LICENSE under the root directory of this GIT repository.
%

% 
% 
\chapter{Fourier transform solutions to Maxwell's equation}\label{chap:PJfourierMaxwellSecondOrder}
\index{Maxwell's equation!Fourier transform}
%\date{Jan 29, 2009.  fourierMaxwell.tex}

\section{Motivation}

In \chapcite{PJwaveFourier} a Green's function solution to the homogeneous wave equation

\begin{equation}\label{eqn:fourier_maxwell:wave}
\begin{aligned}
\left(\inv{v^2} \partial_{tt} -\partial_{xx} -\partial_{yy} -\partial_{zz} \right)\psi = 0
\end{aligned}
\end{equation}

was found to be

\begin{equation}\label{eqn:fourier_maxwell:greensSolution3d}
\begin{aligned}
{\psi}(x,y,z,t) &= \IIinf \phi(u,w,r) G( x-u, y-w, z-r, t) du d\tau dr \\
G(x,y,z,t) &= \inv{({2\pi})^3} \IIinf \exp\left( i k x + i m y + i n z \pm i \sqrt{k^2 + m^2 + n^2} v t \right) dk dm dn
\end{aligned}
\end{equation}

The aim of this set of notes is to explore the same ideas to the forced wave equations for the four vector potentials of the Lorentz gauge Maxwell equation.

Such solutions can be used to find the Faraday bivector or its associated tensor components.

Note that the specific form of the Fourier transform used in these notes continues to be

\begin{equation}\label{eqn:fourierMaxwell:23}
\begin{aligned}
%\hat{f}(\Bk) &= \inv{(\sqrt{2\pi})^n} \IIinf f(\Bx) \exp\left( -i k_j x^j \right) d^n x \\
%{f}(\Bx) &= \inv{(\sqrt{2\pi})^n} \IIinf \hat{f}(\Bk) \exp\left( i k_j x^j \right) d^n k
\hat{f}(\Bk) &= \inv{(\sqrt{2\pi})^n} \IIinf f(\Bx) \exp\left( -i \Bk \cdot \Bx \right) d^n x \\
{f}(\Bx) &= \inv{(\sqrt{2\pi})^n} \IIinf \hat{f}(\Bk) \exp\left( i \Bk \cdot \Bx \right) d^n k
\end{aligned}
\end{equation}

\section{Forced wave equation}
\index{wave equation!forced}

\subsection{One dimensional case}

A good starting point is the reduced complexity one dimensional forced wave equation

\begin{equation}\label{eqn:fourierMaxwell:43}
\begin{aligned}
\left( \inv{v^2} \partial_{tt} -\partial_{xx} \right)\psi = g
\end{aligned}
\end{equation}

Fourier transforming to to the wave number domain, with application of integration by parts twice (each toggling the sign of the spatial derivative term) we have

\begin{equation}\label{eqn:fourierMaxwell:63}
\begin{aligned}
\inv{v^2}\hat{\psi}_{tt} - (-i k)^2 \hat{\psi} = \hat{g}
\end{aligned}
\end{equation}

This leaves us with a linear differential equation of the following form to solve

\begin{equation}\label{eqn:fourier_maxwell:waveNumEquationToSolve}
\begin{aligned}
f'' + \alpha^2 f = h
\end{aligned}
\end{equation}

Out of line solution of this can be found below in \eqnref{eqn:fourier_maxwell:solutionToWaveNumberDomainEquation}, where we have \(f = \hat{\psi}\), \(\alpha = k v\), and \(h = \hat{g} v^2\).  Our solution for the wave function in the wave number domain is now completely
specified

\begin{equation}\label{eqn:fourierMaxwell:83}
\begin{aligned}
\hat{\psi}(k, t) = \Abs{\frac{v}{k}} \int_{u=t_0(k)}^t \hat{g}(u) \sin( \Abs{k v} (t-u) ) du
\end{aligned}
\end{equation}

Here because of the partial differentiation we have the flexibility to make the initial time a function of the wave number \(k\), but it is probably more natural to just set \(t_0 = -\infty\).  Also let us explicitly pick \(v > 0\) so that absolutes are only required on the factors of \(k\)

\begin{equation}\label{eqn:fourierMaxwell:103}
\begin{aligned}
\hat{\psi}(k, t) = \frac{v}{\Abs{k}} \int_{u = -\infty}^t \hat{g}(k, u) \sin( \Abs{k} v (t-u) ) du
\end{aligned}
\end{equation}

But seeing the integral in this form suggests a change of variables \(\tau = t-u\), which gives us our final wave function in the wave number domain with all the time
dependency removed from the integration limits

\begin{equation}\label{eqn:fourierMaxwell:123}
\begin{aligned}
\hat{\psi}(k, t) = \frac{v}{\Abs{k}} \int_{\tau = 0}^\infty \hat{g}(k, t-\tau) \sin( \Abs{k} v \tau ) d\tau
\end{aligned}
\end{equation}

With this our wave function is

\begin{equation}\label{eqn:fourierMaxwell:143}
\begin{aligned}
{\psi}(x, t)
&=
\inv{\sqrt{2\pi}} \IIinf
\left(
\frac{v}{k} \int_{\tau = 0}^\infty \hat{g}(k, t-\tau) \sin( \Abs{k} v \tau ) d\tau
\right) \exp( i k x ) dk \\
\end{aligned}
\end{equation}


But we also have

\begin{equation}\label{eqn:fourierMaxwell:163}
\begin{aligned}
\hat{g}(k,t) &= \inv{\sqrt{2\pi}} \int_{-\infty}^\infty {g}(x, t) \exp( -i k x ) dx
\end{aligned}
\end{equation}

Reassembling we have

\begin{equation}\label{eqn:fourierMaxwell:183}
\begin{aligned}
{\psi}(x, t)
&= 
\int_{k = -\infty}^\infty
\int_{\tau = 0}^\infty
\int_{y=-\infty}^\infty
\frac{v}{ 2 \pi \Abs{k}}
{g}(y, t-\tau) 
\sin( \Abs{k} v \tau ) 
\exp( i k (x-y) ) 
dy
d\tau
dk
\end{aligned}
\end{equation}

Rearranging a bit, and noting that \(\sinc(\Abs{k}x) = \sinc(kx)\) we have

\begin{equation}\label{eqn:fourier_maxwell:oneDimResult}
\begin{aligned}
{\psi}(x, t)
&=
\int_{x'=-\infty}^\infty
\int_{t' = 0}^\infty {g}(x-x', t-t') G(x', t') dx' dt' \\
G(x, t) &=
\int_{k = -\infty}^\infty
\frac{v}{2\pi {k}}
\sin( {k} v t )
\exp( i k x )
dk
\end{aligned}
\end{equation}

We see that our charge density summed over all space contributes to the wave function, but it is the charge density at that spatial location as it existed at a specific previous time.

The Green's function that we convolve with in \eqnref{eqn:fourier_maxwell:oneDimResult} is a rather complex looking function.  As seen later in \chapcite{PJpoisson} it was possible to evaluate a 3D variant of such an integral in ad-hoc methods to produce a form in terms of retarded time and advanced time delta functions.  A similar reduction, also in \chapcite{PJpoisson}, of the Green's function above yields a unit step function identification

\begin{equation}\label{eqn:fourierMaxwell:203}
\begin{aligned}
G(x, t) &= \frac{v}{2} \left(\theta(x + v t )  - \theta(x - v t) \right) 
\end{aligned}
\end{equation}

(This has to be verified more closely to see if it works).

\subsection{Three dimensional case}

Now, lets move on to the 3D case that is of particular interest for electrodynamics.  Our wave equation is now of the form

\begin{equation}\label{eqn:fourierMaxwell:223}
\begin{aligned}
\left( \inv{v^2} \PDSq{t}{} -\sum_j \PDSq{x^j}{} \right)\psi = g
\end{aligned}
\end{equation}

and our Fourier transformation produces almost the same result, but we have a wave number contribution from each of the three dimensions

\begin{equation}\label{eqn:fourierMaxwell:243}
\begin{aligned}
\inv{v^2}\hat{\psi}_{tt} + \Bk^2 \hat{\psi} = \hat{g}
\end{aligned}
\end{equation}

Our wave number domain solution is therefore
\begin{equation}\label{eqn:fourier_maxwell:waveNumDomainSolution}
\begin{aligned}
\hat{\psi}(\Bk, t) = \frac{v}{\Abs{\Bk}} \int_{\tau = 0}^\infty \hat{g}(\Bk, t-\tau) \sin( \Abs{\Bk} v \tau ) d\tau
\end{aligned}
\end{equation}

But our wave number domain charge density is

\begin{equation}\label{eqn:fourierMaxwell:263}
\begin{aligned}
\hat{g}(\Bk, t) &= \inv{(\sqrt{2\pi})^3} \IIinf g(\Bx, t) \exp\left( -i \Bk \cdot \Bx \right) d^3 x \\
\end{aligned}
\end{equation}

Our wave number domain result in terms of the charge density is therefore

%\hat{g}(\Bk, t-\tau) &= \inv{(\sqrt{2\pi})^3} \IIinf g(\Br, t-\tau) \exp\left( -i \Bk \cdot \Br \right) d^3 r \\
\begin{equation}\label{eqn:fourierMaxwell:283}
\begin{aligned}
\hat{\psi}(\Bk, t) =
\frac{v}{\Abs{\Bk}} \int_{\tau = 0}^\infty
%\hat{g}(\Bk, t-\tau) &=
\left(
\inv{(\sqrt{2\pi})^3} \IIinf g(\Br, t-\tau) \exp\left( -i \Bk \cdot \Br \right) d^3 r
\right)
\sin( \Abs{\Bk} v \tau ) d\tau
\end{aligned}
\end{equation}

And finally inverse transforming back to the spatial domain we have a complete solution for the inhomogeneous wave equation in terms of the spatial and temporal charge density distribution

\begin{equation}\label{eqn:fourierMaxwell:303}
\begin{aligned}
{\psi}(\Bx, t)
%&= \IIinf \int_{t' = 0}^\infty g(\Br, t-t') G(\Bx -\Br) d^3 r dt' \\
% x'_j = x_j - r_j
% r_j = x_j - x'_j
% dr_j = -dx'_j
% III dr_1 dr_2 dr_3 = (-1)^3 III d^3 x'
&= \IIinf \int_{t' = 0}^\infty g(\Bx -\Bx', t-t') G(\Bx', t') d^3 x' dt' \\
G(\Bx, t)
&= \IIinf
\frac{v}{(2\pi)^3 \Abs{\Bk}}
\sin( \Abs{\Bk} v t )
\exp\left( i \Bk \cdot \Bx \right)
d^3 k
\end{aligned}
\end{equation}

For computational purposes we are probably much better off using
\eqnref{eqn:fourier_maxwell:waveNumDomainSolution}, however,
from an abstract point of form this expression is much prettier.

One can also see the elements of the traditional retarded time expressions for the potential hiding in there.  See
\chapcite{PJpoisson} for an evaluation of this integral (in an ad-hoc
non-rigorous fashion) eventually producing the retarded time solution.

\subsubsection{Tweak this a bit to put into proper Green's function form}

Now, it makes sense to redefine \(G(\Bx,t)\) above so that we can integrate uniformly over all space and time.  To do so we can add a unit step function into the definition, so that \(G(\Bx,t<0) = 0\).  Additionally, if we express this convolution it is slightly tidier (and consistent with the normal Green's function notation) to put the parameter differences in the kernel term.  Such a change of variables will alter the sign of the integral limits by a factor of \((-1)^4\), but we also have a \((-1)^4\) term from the differentials.  After making these final adjustments we have a final variation of our integral solution

%
% y_j = x_j - x'_j
% dy_j = x_j - dx'_j
% t' = t - \tau
% dt' = t - d\tau
% 
%&= \int_x'1 \int_x'2 \int_x'3 \int_tau g(\Bx -\Bx', t-\tau) G(\Bx', \tau) d^3 x' d\tau \\
%
% =>
%
%&= (-1)^4 \int_y1 \int_y2 \int_y3 \int_t' g(\By, t') G(\Bx - \By, t - t') (-1)^4 d^3 y dt' \\
% 
% 
\begin{equation}\label{eqn:fourierMaxwell:323}
\begin{aligned}
{\psi}(\Bx, t)
&= \IIinf g(\Bx', t') G(\Bx - \Bx', t - t') d^3 x' dt' \\
G(\Bx, t)
&= \theta(t) \IIinf
\frac{v}{(2\pi)^3 \Abs{\Bk}}
\sin( \Abs{\Bk} v t )
\exp\left( i \Bk \cdot \Bx \right)
d^3 k
\end{aligned}
\end{equation}

Now our inhomogeneous solution is expressed nicely as the convolution of our current density over all space and time with an integral kernel.  That integral kernel is precisely the Green's function for this forced wave equation.

This solution comes with a large number of assumptions.  Along the way we have the assumption that both our wave function and the charge density was Fourier transformable, and that the wave number domain products were inverse transformable.  We also had an assumption that the wave function is sufficiently small at the limits of integration that the intermediate contributions from the integration by parts vanished, and finally the big assumption that we were perfectly free to interchange integration order in an extremely ad-hoc and non-rigorous fashion!

\section{Maxwell equation solution}

Having now found Green's function form for the forced wave equation, we can now move to Maxwell's equation

\begin{equation}\label{eqn:fourierMaxwell:343}
\begin{aligned}
\grad F = J/\epsilon_0 c
\end{aligned}
\end{equation}

In terms of potentials we have \(F = \grad \wedge A\), and may also impose the Lorentz gauge \(\grad \cdot A = 0\), to give us our four charge/current forced wave equations

\begin{equation}\label{eqn:fourierMaxwell:363}
\begin{aligned}
\grad^2 A = J/\epsilon_0 c
\end{aligned}
\end{equation}

As scalar equations these are

\begin{equation}\label{eqn:fourierMaxwell:383}
\begin{aligned}
\left( \inv{c^2} \PDSq{t}{} -\sum_j \PDSq{x^j}{} \right) A^\mu = \frac{J^\mu}{\epsilon_0 c}
\end{aligned}
\end{equation}

So, from above, also writing \(x^0 = ct\), we have

\begin{equation}\label{eqn:fourier_maxwell:fourVectorPotentials}
\begin{aligned}
{A^\mu}(x)
&= \inv{\epsilon_0 c} \int J^\mu(x') G(x - x') d^4 x' \\
G(x)
&= \theta(x^0) \int
\frac{1}{(2\pi)^3 \Abs{\Bk}}
\sin( \Abs{\Bk} x^0 )
\exp\left( i \Bk \cdot \Bx \right)
d^3 k
\end{aligned}
\end{equation}

\subsection{Four vector form for the Green's function}
\index{Green's function}

Can we put the sine and exponential product in a more pleasing form?  It would be nice to merge the \(\Bx\) and \(ct\) terms into a 
single four vector form.  One possibility is merging the two

\begin{equation}\label{eqn:fourierMaxwell:403}
\begin{aligned}
\sin( \Abs{\Bk} x^0 ) &\exp\left( i \Bk \cdot \Bx \right) \\
&=
\inv{2i} \left(
\exp\left( i \left( \Bk \cdot \Bx +\Abs{\Bk} x^0 \right) \right)
-\exp\left( i \left( \Bk \cdot \Bx -\Abs{\Bk} x^0 \right) \right)
\right) \\
&=
\inv{2i} \left(
\exp\left( i \Abs{\Bk} \left( \kcap \cdot \Bx + x^0 \right) \right)
-\exp\left( i \Abs{\Bk} \left( \kcap \cdot \Bx - x^0 \right) \right)
\right)
\end{aligned}
\end{equation}

Here we have a sort of sine like conjugation in the two exponentials.  Can we tidy this up?  Let us write the unit wave number
vector in terms of direction cosines

\begin{equation}\label{eqn:fourierMaxwell:423}
\begin{aligned}
\kcap 
&= \sum_m \sigma_m \alpha_m \\
&= \sum_m \gamma_m \gamma_0 \alpha_m \\
\end{aligned}
\end{equation}

Allowing us to write

\begin{equation}\label{eqn:fourierMaxwell:443}
\begin{aligned}
\sum_m \gamma^m \alpha_m &= -\kcap \gamma_0
\end{aligned}
\end{equation}

This gives us
\begin{equation}\label{eqn:fourierMaxwell:463}
\begin{aligned}
\kcap \cdot \Bx + x^0 
&= \alpha_m x^m + x^0 \\
&= (\alpha_m \gamma^m) \cdot (\gamma_j x^j) + \gamma^0 \cdot \gamma_0 x^0 \\
&= (-\kcap \gamma_0 + \gamma_0) \cdot \gamma_\mu x^\mu \\
&= (-\kcap \gamma_0 + \gamma_0) \cdot x
\end{aligned}
\end{equation}

Similarly we have

\begin{equation}\label{eqn:fourierMaxwell:483}
\begin{aligned}
\kcap \cdot \Bx - x^0  &= (-\kcap \gamma_0 - \gamma_0) \cdot x
\end{aligned}
\end{equation}

and can now put \(G\) in explicit four vector form

\begin{equation}\label{eqn:fourierMaxwell:503}
\begin{aligned}
G(x)
&= 
%\theta(x \cdot \gamma_0) \int
%\frac{1}{(2\pi)^3 2 i \Abs{\Bk}}
%\left(
%\exp\left( i ((\Abs{\Bk} -\Bk)\gamma_0) \cdot x \right)
%-\exp\left( -i ((\Abs{\Bk} +\Bk)\gamma_0) \cdot x \right)
%\right)
%d^3 k
\frac{\theta(x \cdot \gamma_0)}{
(2\pi)^3 2 i 
} \int
\left(
\exp\left( i ((\Abs{\Bk} -\Bk)\gamma_0) \cdot x \right)
-\exp\left( -i ((\Abs{\Bk} +\Bk)\gamma_0) \cdot x \right)
\right)
\frac{d^3 k}{ \Abs{\Bk} }
\end{aligned}
\end{equation}

Hmm, is that really any better?  Intuition says that this whole thing 
can be written as sine with some sort of geometric product conjugate 
terms.

I get as far as writing 

\begin{equation}\label{eqn:fourierMaxwell:523}
\begin{aligned}
i ( \Bk \cdot \Bx \pm \Abs{\Bk} x^0 ) 
&=
(i \gamma_0) \wedge ( \Bk \pm \Abs{\Bk} ) \cdot x
\end{aligned}
\end{equation}

But that does not quite have the conjugate form I was looking for (or does it)?  Have to go back and look at Hestenes's multivector conjugation operation.  Think it had something to do with reversion, but do not recall.

Failing that tidy up the following

\begin{equation}\label{eqn:fourierMaxwell:543}
\begin{aligned}
G(x)
&= 
\frac{\theta(x \cdot \gamma_0)}{ (2\pi)^3 }
\int
\sin( \Abs{\Bk} x \cdot \gamma_0 )
\exp\left( -i (\Bk \gamma_0) \cdot x \right)
\frac{d^3 k}{ \Abs{\Bk} }
\end{aligned}
\end{equation}

is probably about as good as it gets for now.  Note the interesting feature
that we end up essentially integrating over a unit ball in our wave number
space.  This suggests the possibility of simplification using the 
divergence theorem.

\subsection{Faraday tensor}
\index{Faraday tensor}

Attempting to find a tidy four vector form for the four vector potentials was in preparation for taking derivatives.
Specifically, applied to \eqnref{eqn:fourier_maxwell:fourVectorPotentials} we have

\begin{equation}\label{eqn:fourierMaxwell:563}
\begin{aligned}
F^{\mu\nu} = \partial^\mu A^\nu - \partial^\nu A^\mu
\end{aligned}
\end{equation}

subject to the Lorentz gauge constraint
\begin{equation}\label{eqn:fourierMaxwell:583}
\begin{aligned}
0 = \partial_\mu A^\mu
\end{aligned}
\end{equation}

If we switch the convolution indices for our potentials

\begin{equation}\label{eqn:fourierMaxwell:603}
\begin{aligned}
{A^\mu}(\Bx, t) &= \inv{\epsilon_0 c} \int J^\mu(x - x') G(x') d^4 x' \\
\end{aligned}
\end{equation}

Then the Lorentz gauge condition, after differentiation under the integral sign, is

\begin{equation}\label{eqn:fourierMaxwell:623}
\begin{aligned}
0 = \partial_\mu A^\mu &= \inv{\epsilon_0 c} \int \left(\partial_\mu J^\mu(x - x') \right) G(x') d^4 x' \\
\end{aligned}
\end{equation}

So we see that the Lorentz gauge seems to actually imply the continuity equation

\begin{equation}\label{eqn:fourierMaxwell:643}
\begin{aligned}
\partial_\mu J^\mu(x) = 0
\end{aligned}
\end{equation}

Similarly, it appears that we can write our tensor components in terms of current density derivatives

\begin{equation}\label{eqn:fourierMaxwell:663}
\begin{aligned}
F^{\mu\nu} 
%&= \partial^\mu A^\nu - \partial^\nu A^\mu \\
&= \inv{\epsilon_0 c} \int \left(\partial^\mu J^\nu(x - x') - \partial^\nu J^\mu(x - x') \right) G(x') d^4 x'
\end{aligned}
\end{equation}

Logically, I suppose that one can consider the entire problem solved here, pending the completion of this calculus exercise.

In terms of tidiness, it would be nicer seeming use the original convolution, and take derivative differences of the Green's function.  However, how to do this is not clear to me since this function has no defined derivative at the \(t=0\) points due to the unit step.

\section{Appendix.  Mechanical details}

\subsection{Solving the wave number domain differential equation}

We wish to solve equation the inhomogeneous \eqnref{eqn:fourier_maxwell:waveNumEquationToSolve}.  Writing this in terms of a linear operator equation this is

\begin{equation}\label{eqn:fourierMaxwell:683}
\begin{aligned}
L(y) &= y'' + \alpha^2 y \\
L(y) &= h
\end{aligned}
\end{equation}

The solutions of this equation will be formed from linear combinations of the homogeneous problem plus a specific solution of the inhomogeneous problem

By inspection the homogeneous problem has solutions in \(\Span \{ e^{ i \alpha x }, e^{ -i \alpha x }\}\).
We can find a solution to the inhomogeneous problem using the variation of parameters method, assuming a solution of the form

\begin{equation}\label{eqn:fourierMaxwell:703}
\begin{aligned}
y  = u e^{ i \alpha x } + v e^{ -i \alpha x }
\end{aligned}
\end{equation}

Taking derivatives we have
\begin{equation}\label{eqn:fourierMaxwell:723}
\begin{aligned}
y' = u' e^{ i \alpha x } + v' e^{ -i \alpha x } + i \alpha (u e^{ i \alpha x } - v e^{ -i \alpha x })
\end{aligned}
\end{equation}

The trick to solving this is to employ the freedom to set the \(u'\), and \(v'\) terms above to zero

\begin{equation}\label{eqn:fourier_maxwell:firstConstraint}
\begin{aligned}
u' e^{ i \alpha x } + v' e^{ -i \alpha x } = 0
\end{aligned}
\end{equation}

Given this choice we then have
\begin{equation}\label{eqn:fourierMaxwell:743}
\begin{aligned}
y' &= i \alpha (u e^{ i \alpha x } - v e^{ -i \alpha x }) \\
y'' &=
(i \alpha)^2 (u e^{ i \alpha x } + v e^{ -i \alpha x })
i \alpha (u' e^{ i \alpha x } - v' e^{ -i \alpha x })
\end{aligned}
\end{equation}

So we have
\begin{equation}\label{eqn:fourierMaxwell:763}
\begin{aligned}
L(y)
&=
(i \alpha)^2 (u e^{ i \alpha x } + v e^{ -i \alpha x })  \\
&+i \alpha (u' e^{ i \alpha x } - v' e^{ -i \alpha x })
+ (\alpha)^2 (u e^{ i \alpha x } + v e^{ -i \alpha x })  \\
&=
i \alpha (u' e^{ i \alpha x } - v' e^{ -i \alpha x })
\end{aligned}
\end{equation}

With this and \eqnref{eqn:fourier_maxwell:firstConstraint} we have a set of simultaneous first order linear differential equations to solve

\begin{equation}\label{eqn:fourierMaxwell:783}
\begin{aligned}
\begin{bmatrix}
u' \\
v' \\
\end{bmatrix}
&=
{\begin{bmatrix}
 e^{ i \alpha x } &- e^{ -i \alpha x } \\
 e^{ i \alpha x } &  e^{ -i \alpha x } \\
\end{bmatrix}}^{-1}
\begin{bmatrix}
{h}/{i \alpha} \\
0 \\
\end{bmatrix} \\
&=
\inv{2}
{\begin{bmatrix}
 e^{ -i \alpha x } & e^{ -i \alpha x } \\
 -e^{ i \alpha x } &  e^{ i \alpha x } \\
\end{bmatrix}}
\begin{bmatrix}
{h}/{i \alpha} \\
0 \\
\end{bmatrix} \\
&=
\frac{h}{2 i \alpha}
{\begin{bmatrix}
 e^{ -i \alpha x } \\
 -e^{ i \alpha x } \\
\end{bmatrix}}
\end{aligned}
\end{equation}

Substituting back into the assumed solution we have
\begin{equation}\label{eqn:fourierMaxwell:803}
\begin{aligned}
y
&= \frac{1}{2 i \alpha} \left(
  e^{ i \alpha x } \int h e^{ -i \alpha x }
- e^{ -i \alpha x } \int h e^{ i \alpha x }
\right) \\
&= \frac{1}{2 i \alpha} \int_{u=x_0}^x h(u) \left( e^{ -i \alpha (u-x) } -e^{ i \alpha (u-x) } \right) du \\
\end{aligned}
\end{equation}

So our solution appears to be

\begin{equation}\label{eqn:fourier_maxwell:solutionToWaveNumberDomainEquation}
\begin{aligned}
y &= \frac{1}{\alpha} \int_{u=x_0}^x h(u) \sin( \alpha(x-u) ) du
\end{aligned}
\end{equation}

A check to see if this is correct is in order to verify this.  Differentiating using \eqnref{eqn:fourier_maxwell:diffInt} we have

\begin{equation}\label{eqn:fourierMaxwell:823}
\begin{aligned}
y'
&=
{\left.
\frac{1}{\alpha}
h(u) \sin( \alpha(x-u) ) \right\vert}_{u=x}
+\frac{1}{\alpha} \int_{u=x_0}^x \PD{x}{} h(u) \sin( \alpha(x-u) ) du \\
&= \int_{u=x_0}^x h(u) \cos( \alpha(x-u) ) du \\
\end{aligned}
\end{equation}

and for the second derivative we have

\begin{equation}\label{eqn:fourierMaxwell:843}
\begin{aligned}
y''
&=
{\left. h(u) \cos( \alpha(x-u) ) \right\vert}_{u=x}
- \alpha \int_{u=x_0}^x h(u) \sin( \alpha(x-u) ) du \\
&= h(x) - \alpha^2 y(x)
\end{aligned}
\end{equation}

Excellent, we have \(y'' + \alpha^2 y = h\) as desired.

\subsection{Differentiation under the integral sign}

Given an function that is both a function of the integral limits and the integrals kernel

\begin{equation}\label{eqn:fourierMaxwell:863}
\begin{aligned}
f(x) = \int_{u = a(x)}^{b(x)} G(x,u) du,
\end{aligned}
\end{equation}

lets recall how to differentiate the beastie.  First let \(G(x,u) = \PDi{u}{F(x,u)}\) so we have

\begin{equation}\label{eqn:fourierMaxwell:883}
\begin{aligned}
f(x) = F(x,b(x)) - F(x,a(x))
\end{aligned}
\end{equation}

and our derivative is
\begin{equation}\label{eqn:fourierMaxwell:903}
\begin{aligned}
f'(x)
&=
\PD{x}{F}(x,b(x))
\PD{u}{F}(x,b(x)) b'
-\PD{x}{F}(x,a(x))
-\PD{u}{F}(x,a(x)) a' \\
&=
G(x,b(x)) b'
-G(x,a(x)) a'
+\PD{x}{F}(x,b(x))
-\PD{x}{F}(x,a(x))
\\
\end{aligned}
\end{equation}

Now, we want \(\PDi{x}{F}\) in terms of \(G\), and to get there, assuming sufficient continuity, we have from the definition

\begin{equation}\label{eqn:fourierMaxwell:923}
\begin{aligned}
\PD{x}{} G(x,u) 
&= \PD{x}{} \PD{u}{F(x,u)} \\
&= \PD{u}{} \PD{x}{F(x,u)} \\
\end{aligned}
\end{equation}

Integrating both sides with respect to \(u\) we have

\begin{equation}\label{eqn:fourierMaxwell:943}
\begin{aligned}
\int \PD{x}{G} du 
&= \int \PD{u}{} \PD{x}{F(x,u)} du \\
%&= \int \PD{u}{} \left( \PD{x}{F(x,u)} + A(x) \right) du \\
&= \PD{x}{F(x,u)} 
\end{aligned}
\end{equation}

This allows us to write 

\begin{equation}\label{eqn:fourierMaxwell:963}
\begin{aligned}
\PD{x}{F}(x,b(x))
-\PD{x}{F}(x,a(x))
&=
\int_{a}^b \PD{x}{G}(x,u) du
\end{aligned}
\end{equation}

and finally

\begin{equation}\label{eqn:fourier_maxwell:diffInt}
\begin{aligned}
\frac{d}{dx} \int_{u = a(x)}^{b(x)} G(x,u) du
&=
G(x,b(x)) b'
-G(x,a(x)) a'
+ \int_{a(x)}^{b(x)} \PD{x}{G}(x,u) du
\end{aligned}
\end{equation}

\subsubsection{Argument logic error above to understand}

Is the following not also true

\begin{equation}\label{eqn:fourierMaxwell:983}
\begin{aligned}
\int \PD{x}{G} du 
&= \int \PD{u}{} \PD{x}{F(x,u)} du \\
&= \int \PD{u}{} \left( \PD{x}{F(x,u)} + A(x) \right) du \\
&= \PD{x}{F(x,u)} + A(x)u + B
\end{aligned}
\end{equation}

In this case we have
\begin{equation}\label{eqn:fourierMaxwell:1003}
\begin{aligned}
\PD{x}{F}(x,b(x)) -\PD{x}{F}(x,a(x)) &= \int_{a}^b \PD{x}{G}(x,u) du - A(x) ( b(x) - a(x))
\end{aligned}
\end{equation}

How to reconcile this with the answer I expect (and having gotten it, I believe matches my recollection)?

\documentclass{article}

\usepackage{amsmath}
\usepackage{mathpazo}

%
% shorthand for bold symbols, convenient for vectors and matrices
%
\newcommand{\Ba}[0]{\mathbf{a}}
\newcommand{\Bb}[0]{\mathbf{b}}
\newcommand{\Bc}[0]{\mathbf{c}}
\newcommand{\Bd}[0]{\mathbf{d}}
\newcommand{\Be}[0]{\mathbf{e}}
\newcommand{\Bf}[0]{\mathbf{f}}
\newcommand{\Bg}[0]{\mathbf{g}}
\newcommand{\Bh}[0]{\mathbf{h}}
\newcommand{\Bi}[0]{\mathbf{i}}
\newcommand{\Bj}[0]{\mathbf{j}}
\newcommand{\Bk}[0]{\mathbf{k}}
\newcommand{\Bl}[0]{\mathbf{l}}
\newcommand{\Bm}[0]{\mathbf{m}}
\newcommand{\Bn}[0]{\mathbf{n}}
\newcommand{\Bo}[0]{\mathbf{o}}
\newcommand{\Bp}[0]{\mathbf{p}}
\newcommand{\Bq}[0]{\mathbf{q}}
\newcommand{\Br}[0]{\mathbf{r}}
\newcommand{\Bs}[0]{\mathbf{s}}
\newcommand{\Bt}[0]{\mathbf{t}}
\newcommand{\Bu}[0]{\mathbf{u}}
\newcommand{\Bv}[0]{\mathbf{v}}
\newcommand{\Bw}[0]{\mathbf{w}}
\newcommand{\Bx}[0]{\mathbf{x}}
\newcommand{\By}[0]{\mathbf{y}}
\newcommand{\Bz}[0]{\mathbf{z}}
\newcommand{\BA}[0]{\mathbf{A}}
\newcommand{\BB}[0]{\mathbf{B}}
\newcommand{\BC}[0]{\mathbf{C}}
\newcommand{\BD}[0]{\mathbf{D}}
\newcommand{\BE}[0]{\mathbf{E}}
\newcommand{\BF}[0]{\mathbf{F}}
\newcommand{\BG}[0]{\mathbf{G}}
\newcommand{\BH}[0]{\mathbf{H}}
\newcommand{\BI}[0]{\mathbf{I}}
\newcommand{\BJ}[0]{\mathbf{J}}
\newcommand{\BK}[0]{\mathbf{K}}
\newcommand{\BL}[0]{\mathbf{L}}
\newcommand{\BM}[0]{\mathbf{M}}
\newcommand{\BN}[0]{\mathbf{N}}
\newcommand{\BO}[0]{\mathbf{O}}
\newcommand{\BP}[0]{\mathbf{P}}
\newcommand{\BQ}[0]{\mathbf{Q}}
\newcommand{\BR}[0]{\mathbf{R}}
\newcommand{\BS}[0]{\mathbf{S}}
\newcommand{\BT}[0]{\mathbf{T}}
\newcommand{\BU}[0]{\mathbf{U}}
\newcommand{\BV}[0]{\mathbf{V}}
\newcommand{\BW}[0]{\mathbf{W}}
\newcommand{\BX}[0]{\mathbf{X}}
\newcommand{\BY}[0]{\mathbf{Y}}
\newcommand{\BZ}[0]{\mathbf{Z}}

\newcommand{\Bzero}[0]{\mathbf{0}}
\newcommand{\Btheta}[0]{\boldsymbol{\theta}}
\newcommand{\Btau}[0]{\boldsymbol{\tau}}
\newcommand{\Bomega}[0]{\boldsymbol{\omega}}

%
% shorthand for unit vectors
%
\newcommand{\acap}[0]{\hat{\Ba}}
\newcommand{\bcap}[0]{\hat{\Bb}}
\newcommand{\ccap}[0]{\hat{\Bc}}
\newcommand{\dcap}[0]{\hat{\Bd}}
\newcommand{\ecap}[0]{\hat{\Be}}
\newcommand{\fcap}[0]{\hat{\Bf}}
\newcommand{\gcap}[0]{\hat{\Bg}}
\newcommand{\hcap}[0]{\hat{\Bh}}
\newcommand{\icap}[0]{\hat{\Bi}}
\newcommand{\jcap}[0]{\hat{\Bj}}
\newcommand{\kcap}[0]{\hat{\Bk}}
\newcommand{\lcap}[0]{\hat{\Bl}}
\newcommand{\mcap}[0]{\hat{\Bm}}
\newcommand{\ncap}[0]{\hat{\Bn}}
\newcommand{\ocap}[0]{\hat{\Bo}}
\newcommand{\pcap}[0]{\hat{\Bp}}
\newcommand{\qcap}[0]{\hat{\Bq}}
\newcommand{\rcap}[0]{\hat{\Br}}
\newcommand{\scap}[0]{\hat{\Bs}}
\newcommand{\tcap}[0]{\hat{\Bt}}
\newcommand{\ucap}[0]{\hat{\Bu}}
\newcommand{\vcap}[0]{\hat{\Bv}}
\newcommand{\wcap}[0]{\hat{\Bw}}
\newcommand{\xcap}[0]{\hat{\Bx}}
\newcommand{\ycap}[0]{\hat{\By}}
\newcommand{\zcap}[0]{\hat{\Bz}}
\newcommand{\thetacap}[0]{\hat{\Btheta}}

%
% to write R^n and C^n in a distinguishable fashion.  Perhaps change this
% to the double lined characters upon figuring out how to do so.
%
\newcommand{\C}[1]{$\mathbb{C}^{#1}$}
\newcommand{\R}[1]{$\mathbb{R}^{#1}$}

%
% various generally useful helpers
%

% derivative of #1 wrt. #2:
\newcommand{\D}[2] {\frac {d#2} {d#1}}

\newcommand{\inv}[1]{\frac{1}{#1}}
\newcommand{\cross}[0]{\times}

\newcommand{\abs}[1]{\lvert{#1}\rvert}
\newcommand{\norm}[1]{\lVert{#1}\rVert}
\newcommand{\innerprod}[2]{\langle{#1}, {#2}\rangle}
\newcommand{\dotprod}[2]{{#1} \cdot {#2}}
\newcommand{\bdotprod}[2]{\left({#1} \cdot {#2}\right)}
\newcommand{\crossprod}[2]{{#1} \cross {#2}}
\newcommand{\tripleprod}[3]{\dotprod{\left(\crossprod{#1}{#2}\right)}{#3}}

\DeclareMathOperator{\Proj}{Proj}
\DeclareMathOperator{\Span}{span}
\DeclareMathOperator{\Sgn}{sgn}
\DeclareMathOperator{\Area}{Area}
\DeclareMathOperator{\Volume}{Volume}

%
% A few miscellaneous things specific to this document
%
\newcommand{\crossop}[1]{\crossprod{#1}{}}

% R2 vector.
\newcommand{\VectorTwo}[2]{
\begin{bmatrix}
 {#1} \\
 {#2}
\end{bmatrix}
}

\newcommand{\VectorN}[1]{
\begin{bmatrix}
{#1}_1 \\
{#1}_2 \\
\vdots \\
{#1}_N \\
\end{bmatrix}
}

\newcommand{\DETuvij}[4]{
\begin{vmatrix}
 {#1}_{#3} & {#1}_{#4} \\
 {#2}_{#3} & {#2}_{#4}
\end{vmatrix}
}

\newcommand{\DETuvwijk}[6]{
\begin{vmatrix}
 {#1}_{#4} & {#1}_{#5} & {#1}_{#6} \\
 {#2}_{#4} & {#2}_{#5} & {#2}_{#6} \\
 {#3}_{#4} & {#3}_{#5} & {#3}_{#6}
\end{vmatrix}
}

\newcommand{\DETuvwxijkl}[8]{
\begin{vmatrix}
 {#1}_{#5} & {#1}_{#6} & {#1}_{#7} & {#1}_{#8} \\
 {#2}_{#5} & {#2}_{#6} & {#2}_{#7} & {#2}_{#8} \\
 {#3}_{#5} & {#3}_{#6} & {#3}_{#7} & {#3}_{#8} \\
 {#4}_{#5} & {#4}_{#6} & {#4}_{#7} & {#4}_{#8} \\
\end{vmatrix}
}

%\newcommand{\DETuvwxyijklm}[10]{
%\begin{vmatrix}
% {#1}_{#6} & {#1}_{#7} & {#1}_{#8} & {#1}_{#9} & {#1}_{#10} \\
% {#2}_{#6} & {#2}_{#7} & {#2}_{#8} & {#2}_{#9} & {#2}_{#10} \\
% {#3}_{#6} & {#3}_{#7} & {#3}_{#8} & {#3}_{#9} & {#3}_{#10} \\
% {#4}_{#6} & {#4}_{#7} & {#4}_{#8} & {#4}_{#9} & {#4}_{#10} \\
% {#5}_{#6} & {#5}_{#7} & {#5}_{#8} & {#5}_{#9} & {#5}_{#10}
%\end{vmatrix}
%}

% R3 vector.
\newcommand{\VectorThree}[3]{
\begin{bmatrix}
 {#1} \\
 {#2} \\
 {#3}
\end{bmatrix}
}


%<misc>
%
\newcommand{\Abs}[1]{{\left\lvert{#1}\right\rvert}}
\newcommand{\spacegrad}[0]{\boldsymbol{\nabla}}
\newcommand{\grad}[0]{\nabla}
\newcommand{\LL}[0]{\mathcal{L}}

% == \partial_{#1} {#2}
\newcommand{\PD}[2]{\frac{\partial {#2}}{\partial {#1}}}
% inline variant
\newcommand{\PDi}[2]{{\partial {#2}}/{\partial {#1}}}

\newcommand{\PDD}[3]{\frac{\partial^2 {#3}}{\partial {#1}\partial {#2}}}
%\newcommand{\PDd}[2]{\frac{\partial^2 {#2}}{{\partial{#1}}^2}}
\newcommand{\PDsq}[2]{\frac{\partial^2 {#2}}{(\partial {#1})^2}}

\newcommand{\Partial}[2]{\frac{\partial {#1}}{\partial {#2}}}
\DeclareMathOperator{\RejName}{Rej}
\newcommand{\Rej}[2]{\RejName_{#1}\left( {#2} \right)}
\newcommand{\Rm}[1]{\mathbb{R}^{#1}}
\newcommand{\Cm}[1]{\mathbb{C}^{#1}}
\newcommand{\conj}[0]{{*}}

%</misc>

% <grade selection>
%
\newcommand{\gpgrade}[2] {{\left\langle{{#1}}\right\rangle}_{#2}}

\newcommand{\gpgradezero}[1] {\gpgrade{#1}{}}
%\newcommand{\gpscalargrade}[1] {{\left\langle{{#1}}\right\rangle}}
%\newcommand{\gpgradezero}[1] {\gpgrade{#1}{0}}

%\newcommand{\gpgradeone}[1] {{\left\langle{{#1}}\right\rangle}_{1}}
\newcommand{\gpgradeone}[1] {\gpgrade{#1}{1}}

\newcommand{\gpgradetwo}[1] {\gpgrade{#1}{2}}
\newcommand{\gpgradethree}[1] {\gpgrade{#1}{3}}
\newcommand{\gpgradefour}[1] {\gpgrade{#1}{4}}
%
% </grade selection>



\newcommand{\adot}[0]{{\dot{a}}}
\newcommand{\bdot}[0]{{\dot{b}}}
% taken for centered dot:
%\newcommand{\cdot}[0]{{\dot{c}}}
%\newcommand{\ddot}[0]{{\dot{d}}}
\newcommand{\edot}[0]{{\dot{e}}}
\newcommand{\fdot}[0]{{\dot{f}}}
\newcommand{\gdot}[0]{{\dot{g}}}
\newcommand{\hdot}[0]{{\dot{h}}}
\newcommand{\idot}[0]{{\dot{i}}}
\newcommand{\jdot}[0]{{\dot{j}}}
\newcommand{\kdot}[0]{{\dot{k}}}
\newcommand{\ldot}[0]{{\dot{l}}}
\newcommand{\mdot}[0]{{\dot{m}}}
\newcommand{\ndot}[0]{{\dot{n}}}
%\newcommand{\odot}[0]{{\dot{o}}}
\newcommand{\pdot}[0]{{\dot{p}}}
\newcommand{\qdot}[0]{{\dot{q}}}
\newcommand{\rdot}[0]{{\dot{r}}}
\newcommand{\sdot}[0]{{\dot{s}}}
\newcommand{\tdot}[0]{{\dot{t}}}
\newcommand{\udot}[0]{{\dot{u}}}
\newcommand{\vdot}[0]{{\dot{v}}}
\newcommand{\wdot}[0]{{\dot{w}}}
\newcommand{\xdot}[0]{{\dot{x}}}
\newcommand{\ydot}[0]{{\dot{y}}}
\newcommand{\zdot}[0]{{\dot{z}}}
\newcommand{\addot}[0]{{\ddot{a}}}
\newcommand{\bddot}[0]{{\ddot{b}}}
\newcommand{\cddot}[0]{{\ddot{c}}}
%\newcommand{\dddot}[0]{{\ddot{d}}}
\newcommand{\eddot}[0]{{\ddot{e}}}
\newcommand{\fddot}[0]{{\ddot{f}}}
\newcommand{\gddot}[0]{{\ddot{g}}}
\newcommand{\hddot}[0]{{\ddot{h}}}
\newcommand{\iddot}[0]{{\ddot{i}}}
\newcommand{\jddot}[0]{{\ddot{j}}}
\newcommand{\kddot}[0]{{\ddot{k}}}
\newcommand{\lddot}[0]{{\ddot{l}}}
\newcommand{\mddot}[0]{{\ddot{m}}}
\newcommand{\nddot}[0]{{\ddot{n}}}
\newcommand{\oddot}[0]{{\ddot{o}}}
\newcommand{\pddot}[0]{{\ddot{p}}}
\newcommand{\qddot}[0]{{\ddot{q}}}
\newcommand{\rddot}[0]{{\ddot{r}}}
\newcommand{\sddot}[0]{{\ddot{s}}}
\newcommand{\tddot}[0]{{\ddot{t}}}
\newcommand{\uddot}[0]{{\ddot{u}}}
\newcommand{\vddot}[0]{{\ddot{v}}}
\newcommand{\wddot}[0]{{\ddot{w}}}
\newcommand{\xddot}[0]{{\ddot{x}}}
\newcommand{\yddot}[0]{{\ddot{y}}}
\newcommand{\zddot}[0]{{\ddot{z}}}

%<bold and dot greek symbols>
%

\newcommand{\Deltadot}[0]{{\dot{\Delta}}}
\newcommand{\Gammadot}[0]{{\dot{\Gamma}}}
\newcommand{\Lambdadot}[0]{{\dot{\Lambda}}}
\newcommand{\Omegadot}[0]{{\dot{\Omega}}}
\newcommand{\Phidot}[0]{{\dot{\Phi}}}
\newcommand{\Pidot}[0]{{\dot{\Pi}}}
\newcommand{\Psidot}[0]{{\dot{\Psi}}}
\newcommand{\Sigmadot}[0]{{\dot{\Sigma}}}
\newcommand{\Thetadot}[0]{{\dot{\Theta}}}
\newcommand{\Upsilondot}[0]{{\dot{\Upsilon}}}
\newcommand{\Xidot}[0]{{\dot{\Xi}}}
\newcommand{\alphadot}[0]{{\dot{\alpha}}}
\newcommand{\betadot}[0]{{\dot{\beta}}}
\newcommand{\chidot}[0]{{\dot{\chi}}}
\newcommand{\deltadot}[0]{{\dot{\delta}}}
\newcommand{\epsilondot}[0]{{\dot{\epsilon}}}
\newcommand{\etadot}[0]{{\dot{\eta}}}
\newcommand{\gammadot}[0]{{\dot{\gamma}}}
\newcommand{\kappadot}[0]{{\dot{\kappa}}}
\newcommand{\lambdadot}[0]{{\dot{\lambda}}}
\newcommand{\mudot}[0]{{\dot{\mu}}}
\newcommand{\nudot}[0]{{\dot{\nu}}}
\newcommand{\omegadot}[0]{{\dot{\omega}}}
\newcommand{\phidot}[0]{{\dot{\phi}}}
\newcommand{\pidot}[0]{{\dot{\pi}}}
\newcommand{\psidot}[0]{{\dot{\psi}}}
\newcommand{\rhodot}[0]{{\dot{\rho}}}
\newcommand{\sigmadot}[0]{{\dot{\sigma}}}
\newcommand{\taudot}[0]{{\dot{\tau}}}
\newcommand{\thetadot}[0]{{\dot{\theta}}}
\newcommand{\upsilondot}[0]{{\dot{\upsilon}}}
\newcommand{\varepsilondot}[0]{{\dot{\varepsilon}}}
\newcommand{\varphidot}[0]{{\dot{\varphi}}}
\newcommand{\varpidot}[0]{{\dot{\varpi}}}
\newcommand{\varrhodot}[0]{{\dot{\varrho}}}
\newcommand{\varsigmadot}[0]{{\dot{\varsigma}}}
\newcommand{\varthetadot}[0]{{\dot{\vartheta}}}
\newcommand{\xidot}[0]{{\dot{\xi}}}
\newcommand{\zetadot}[0]{{\dot{\zeta}}}

\newcommand{\Deltaddot}[0]{{\ddot{\Delta}}}
\newcommand{\Gammaddot}[0]{{\ddot{\Gamma}}}
\newcommand{\Lambdaddot}[0]{{\ddot{\Lambda}}}
\newcommand{\Omegaddot}[0]{{\ddot{\Omega}}}
\newcommand{\Phiddot}[0]{{\ddot{\Phi}}}
\newcommand{\Piddot}[0]{{\ddot{\Pi}}}
\newcommand{\Psiddot}[0]{{\ddot{\Psi}}}
\newcommand{\Sigmaddot}[0]{{\ddot{\Sigma}}}
\newcommand{\Thetaddot}[0]{{\ddot{\Theta}}}
\newcommand{\Upsilonddot}[0]{{\ddot{\Upsilon}}}
\newcommand{\Xiddot}[0]{{\ddot{\Xi}}}
\newcommand{\alphaddot}[0]{{\ddot{\alpha}}}
\newcommand{\betaddot}[0]{{\ddot{\beta}}}
\newcommand{\chiddot}[0]{{\ddot{\chi}}}
\newcommand{\deltaddot}[0]{{\ddot{\delta}}}
\newcommand{\epsilonddot}[0]{{\ddot{\epsilon}}}
\newcommand{\etaddot}[0]{{\ddot{\eta}}}
\newcommand{\gammaddot}[0]{{\ddot{\gamma}}}
\newcommand{\kappaddot}[0]{{\ddot{\kappa}}}
\newcommand{\lambdaddot}[0]{{\ddot{\lambda}}}
\newcommand{\muddot}[0]{{\ddot{\mu}}}
\newcommand{\nuddot}[0]{{\ddot{\nu}}}
\newcommand{\omegaddot}[0]{{\ddot{\omega}}}
\newcommand{\phiddot}[0]{{\ddot{\phi}}}
\newcommand{\piddot}[0]{{\ddot{\pi}}}
\newcommand{\psiddot}[0]{{\ddot{\psi}}}
\newcommand{\rhoddot}[0]{{\ddot{\rho}}}
\newcommand{\sigmaddot}[0]{{\ddot{\sigma}}}
\newcommand{\tauddot}[0]{{\ddot{\tau}}}
\newcommand{\thetaddot}[0]{{\ddot{\theta}}}
\newcommand{\upsilonddot}[0]{{\ddot{\upsilon}}}
\newcommand{\varepsilonddot}[0]{{\ddot{\varepsilon}}}
\newcommand{\varphiddot}[0]{{\ddot{\varphi}}}
\newcommand{\varpiddot}[0]{{\ddot{\varpi}}}
\newcommand{\varrhoddot}[0]{{\ddot{\varrho}}}
\newcommand{\varsigmaddot}[0]{{\ddot{\varsigma}}}
\newcommand{\varthetaddot}[0]{{\ddot{\vartheta}}}
\newcommand{\xiddot}[0]{{\ddot{\xi}}}
\newcommand{\zetaddot}[0]{{\ddot{\zeta}}}

\newcommand{\BDelta}[0]{\boldsymbol{\Delta}}
\newcommand{\BGamma}[0]{\boldsymbol{\Gamma}}
\newcommand{\BLambda}[0]{\boldsymbol{\Lambda}}
\newcommand{\BOmega}[0]{\boldsymbol{\Omega}}
\newcommand{\BPhi}[0]{\boldsymbol{\Phi}}
\newcommand{\BPi}[0]{\boldsymbol{\Pi}}
\newcommand{\BPsi}[0]{\boldsymbol{\Psi}}
\newcommand{\BSigma}[0]{\boldsymbol{\Sigma}}
\newcommand{\BTheta}[0]{\boldsymbol{\Theta}}
\newcommand{\BUpsilon}[0]{\boldsymbol{\Upsilon}}
\newcommand{\BXi}[0]{\boldsymbol{\Xi}}
\newcommand{\Balpha}[0]{\boldsymbol{\alpha}}
\newcommand{\Bbeta}[0]{\boldsymbol{\beta}}
\newcommand{\Bchi}[0]{\boldsymbol{\chi}}
\newcommand{\Bdelta}[0]{\boldsymbol{\delta}}
\newcommand{\Bepsilon}[0]{\boldsymbol{\epsilon}}
\newcommand{\Beta}[0]{\boldsymbol{\eta}}
\newcommand{\Bgamma}[0]{\boldsymbol{\gamma}}
\newcommand{\Bkappa}[0]{\boldsymbol{\kappa}}
\newcommand{\Blambda}[0]{\boldsymbol{\lambda}}
\newcommand{\Bmu}[0]{\boldsymbol{\mu}}
\newcommand{\Bnu}[0]{\boldsymbol{\nu}}
%\newcommand{\Bomega}[0]{\boldsymbol{\omega}}
\newcommand{\Bphi}[0]{\boldsymbol{\phi}}
\newcommand{\Bpi}[0]{\boldsymbol{\pi}}
\newcommand{\Bpsi}[0]{\boldsymbol{\psi}}
\newcommand{\Brho}[0]{\boldsymbol{\rho}}
\newcommand{\Bsigma}[0]{\boldsymbol{\sigma}}
%\newcommand{\Btau}[0]{\boldsymbol{\tau}}
%\newcommand{\Btheta}[0]{\boldsymbol{\theta}}
\newcommand{\Bupsilon}[0]{\boldsymbol{\upsilon}}
\newcommand{\Bvarepsilon}[0]{\boldsymbol{\varepsilon}}
\newcommand{\Bvarphi}[0]{\boldsymbol{\varphi}}
\newcommand{\Bvarpi}[0]{\boldsymbol{\varpi}}
\newcommand{\Bvarrho}[0]{\boldsymbol{\varrho}}
\newcommand{\Bvarsigma}[0]{\boldsymbol{\varsigma}}
\newcommand{\Bvartheta}[0]{\boldsymbol{\vartheta}}
\newcommand{\Bxi}[0]{\boldsymbol{\xi}}
\newcommand{\Bzeta}[0]{\boldsymbol{\zeta}}
%
%</bold and dot greek symbols>
%<infrequent>
%
%\newcommand{\AreaOp}[1]{\AName_{#1}}
%\newcommand{\Babs}[0]{\abs{\BB}}
%\newcommand{\Bcap}[0]{\hat{\BB}}
%\newcommand{\BrPrimeRej}[0]{\rcap(\rcap \wedge \Br')}
%\newcommand{\CA}[0]{\mathcal{A}}
%\newcommand{\Cos}[1]{\cos{\left({#1}\right)}}
%\newcommand{\Det}[1] {\abs{#1}}
%\newcommand{\Dsq}[2] {\frac {\partial^2 {#1}} {\partial {#2}^2}}
%\newcommand{\Exp}[1]{\exp{\left({#1}\right)}}
%\newcommand{\Norm}[1]{\left\lVert{#1}\right\rVert}
%\newcommand{\Sin}[1]{\sin{\left({#1}\right)}}
%\newcommand{\T}[0]{\text{T}}
%\newcommand{\VolumeOp}[1]{\VName_{#1}}
%\newcommand{\agrad}[0]{\Ba \cdot \nabla}
%\newcommand{\alphacap}[0]{\hat{\boldsymbol{\alpha}}}
%\newcommand{\Fcap}[0]{\hat{\BF}}
%\newcommand{\bithree}[0]{{\Bi}_3}
%\newcommand{\bxa}[0]{\Bx\Ba}
%\newcommand{\coordvec}[2]{
%\newcommand{\costheta}[0]{\acap \cdot \xcap}
%\newcommand{\ddt}[1]{\ddot{#1}}
%\newcommand{\ddu}[1] {\frac {d{#1}} {du}}
%\newcommand{\dsqxj}[2] {\frac {\partial^2 {#1}} {\partial {x_{#2}}^2}}
%\newcommand{\dtheta}[1]{\frac{d {#1}}{d \theta}}
%\newcommand{\dt}[1]{\dot{#1}}
%\newcommand{\dt}[1]{\frac{d {#1}}{dt}}
%\newcommand{\dxj}[2] {\frac {\partial {#1}} {\partial {x_{#2}}}}
%\newcommand{\halfPhi}[0]{\frac{\phi}{2}}
%\newcommand{\half}[0]{\inv{2}}
%\newcommand{\inv}[1]{\frac{1}{#1}}
%\newcommand{\laplacian}[0]{\nabla^2}
%\newcommand{\matrixoftx}[3]{
%\newcommand{\nrrp}[0]{\norm{\rcap \wedge \Br'}}
%\newcommand{\oiint}{\bigcirc \hspace{-1.4em} \int \hspace{-.8em} \int}
%\newcommand{\transpose}[1]{{#1}^{\text{T}}}
%\newcommand{\transpose}[1]{{{#1}^{\TextTranspose}}}
%\newcommand{\transpose}[1]{{{#1}^{\text{T}}}}
%\newcommand{\barA}[0]{\bar{A}}
%\newcommand{\qbar}[0]{\bar{q}}
%\newcommand{\qdotbar}[0]{\dot{\bar{q}}}
%
%</infrequent>





\newcommand{\PDSq}[2]{\frac{\partial^2 {#2}}{\partial {#1}^2}}
\DeclareMathOperator{\sinc}{sinc}
\DeclareMathOperator{\PV}{PV}
\newcommand{\FF}[0]{\mathcal{F}}
\newcommand{\IIinf}[0]{ \int_{-\infty}^\infty }

\usepackage[bookmarks=true]{hyperref}

\usepackage{color,cite,graphicx}
   % use colour in the document, put your citations as [1-4]
   % rather than [1,2,3,4] (it looks nicer, and the extended LaTeX2e
   % graphics package. 
\usepackage{latexsym,amssymb,epsf} % don't remember if these are
   % needed, but their inclusion can't do any damage


\title{ First order Fourier transform solution attempt for Maxwell's equation. }
\author{Peeter Joot}
\date{ Jan 31, 2009.  Last Revision: $Date: 2009/02/01 01:12:33 $ }

\begin{document}

\maketitle{}

\tableofcontents

\section{ Motivation. }

In \cite{PJfourierMaxwellSecondOrder} solutions of Maxwell's equation
via Fourier transformation of the four potential forced wave equations were
explored.

Here a first order solution is attempted, by directly Fourier transforming
the Maxwell's equation in bivector form.

\section{ Setup. }

Again using a 3D spatial fourier transform, we want to put Maxwell's equation into an explicit time dependent form, and can do so by
premultiplying by our observer's time basis vector $\gamma_0$

\begin{align*}
\gamma_0 \grad F &= \gamma_0 \frac{J }{\epsilon_0 c}
\end{align*}

On the left hand side we have
\begin{align*}
\gamma_0 \grad 
&= \gamma_0 \left( \gamma^0 \partial_0 + \gamma^k \partial_k \right) \\
&= \partial_0 - \gamma^k \gamma_0 \partial_k \\
&= \partial_0 + \sigma^k \partial_k \\
&= \partial_0 + \spacegrad \\
\end{align*}

and on the right hand side we have
\begin{align*}
\gamma_0 \frac{J }{\epsilon_0 c}
&= \gamma_0 \frac{c \rho \gamma_0 + J^k \gamma_k }{\epsilon_0 c} \\
&= \frac{c \rho - J^k \sigma_k }{\epsilon_0 c} \\
&= \frac{\rho}{\epsilon_0} - \frac{\Bj}{\epsilon_0 c} \\
\end{align*}

Both the spacetime gradient and the current density four vector have been put in a quaternionic form with scalar and bivector grades in the 
STA basis.  This leaves us with the time centric formulation of Maxwell's equation

\begin{align}
\left(\partial_0 + \spacegrad\right) F &= \frac{\rho}{\epsilon_0} - \frac{\Bj}{\epsilon_0 c} 
\end{align}

Except for the fact that we have objects of various grades here, and that this is a first instead of second order equation,
these equations have the same form as in the previous Fourier transform attacks.
Those were Fourier transform application for the homogeneous and inhomogeneous wave equations, and the heat and 
Schr\"{o}dinger equation.

\section{ Fourier transforming a mixed grade object. }

Now, here we make the assumption that we can apply 3D Fourier transform pairs to mixed grade objects, as in

\begin{align}\label{eqn:FourierTxDefinition}
\hat{\psi}(\Bk, t) &= \inv{(\sqrt{2\pi})^3} \IIinf \psi(\Bx, t) \exp\left( -i \Bk \cdot \Bx \right) d^3 x \\
{\psi}(\Bx, t) &= \PV \inv{(\sqrt{2\pi})^3} \IIinf \hat{\psi}(\Bk, t) \exp\left( i \Bk \cdot \Bx \right) d^3 k
\end{align}

Now, because of linearity, is it clear enough that this will work, provided this is a valid transform pair for any specific grade.
We do however want to be careful of the order of the factors since we want the flexibility to use any particular convienient representation
of $i$, in particular $i = \gamma_0 \gamma_1 \gamma_2 \gamma_3 = \sigma_1 \sigma_2 \sigma_3$.

Let's repeat our an ad-hoc verification that this transform pair works as desired, being careful with the order of products and specifically
allowing for $\psi$ to be a non-scalar function.  
Writing $\Bk = k_m \sigma^m$, $\Br = \sigma_m r^m$, $\Bx = \sigma_m x^m$, that is an expansion of

\begin{align*}
\PV &\inv{(\sqrt{2\pi})^3} \int 
\left( \inv{(\sqrt{2\pi})^3} \int \psi(\Br, t) \exp\left( -i \Bk \cdot \Br \right) d^3 r \right)
\exp\left( i \Bk \cdot \Bx \right) d^3 k \\
&= \int \psi(\Br, t) d^3 r \PV \inv{({2\pi})^3} \int \exp\left( i \Bk \cdot (\Bx -\Br) \right) d^3 k \\
&= \int \psi(\Br, t) d^3 r \Pi_{m=1}^3 \PV \inv{{2\pi}} \int \exp\left( i k_m (x^m -r^m) \right) dk_m \\
&= \int \psi(\Br, t) d^3 r \Pi_{m=1}^3 \lim_{R\rightarrow \infty} \frac{\sin\left( R (x^m -r^m) \right)}{\pi(x^m - r^m)} \\
&\sim \int \psi(\Br, t) \delta(x^1-r^1) \delta(x^2-r^2) \delta(x^3-r^3) d^3 r \\
&= \psi(\Bx, t)
\end{align*}

In the second last step above we make the ad-hoc identification of that $\sinc$ limit with the dirac delta function, and recover
our original function as desired (the Rigor police are on holiday again).

\subsection{ Rotor form of the Fourier transform? }

Although the formulation picked above appears to work, it isn't the only choice to potentially make for the Fourier transform
of multivector.  Would it be more natural to pick an explicit Rotor formulation?  This perhaps makes more sense since it is then
automatically grade preserving.

\begin{align}\label{eqn:rotorFourier}
\hat{\psi}(\Bk, t) &= \inv{(\sqrt{2\pi})^n} \IIinf \exp\left( \inv{2} i \Bk \cdot \Bx \right) \psi(\Bx, t) \exp\left( - \inv{2} i \Bk \cdot \Bx \right) d^n x \\
{\psi}(\Bx, t) &= \PV \inv{(\sqrt{2\pi})^n} \IIinf \exp\left( -\inv{2} i \Bk \cdot \Bx \right) \hat{\psi}(\Bk, t) \exp\left( \inv{2} i \Bk \cdot \Bx \right) d^n k
\end{align}

This isn't a moot question since I later
tried to make an assumption that the grade of a transformed object equals the original grade.  That doesn't work with the
Fourier transform definition that has been picked in equation \ref{eqn:FourierTxDefinition}.  It may be neccessary to revamp the complete treatment, but
for now at least an observation that the grades of transform pairs do not neccessarily match is required.

Does the transform pair work?  For the $n=1$ case this is

\begin{align*}
\FF(f) = \hat{f}(k) &= \inv{\sqrt{2\pi}} \IIinf \exp\left( \inv{2} i k x \right) f(x) \exp\left( - \inv{2} i k x \right) dx \\
\FF^{-1}(\hat{f}) = {f}(x) &= \PV \inv{\sqrt{2\pi}} \IIinf \exp\left( -\inv{2} i k x \right) \hat{f}(k) \exp\left( \inv{2} i k x \right) dk
\end{align*}

Will the computation of $\FF^{-1}(\FF(f(x)))$ produce $f(x)$?  Let's try

\begin{align*}
\FF^{-1}(\FF(f(x)))
&= \\
\PV &\inv{{2\pi}} \IIinf \exp\left( -\inv{2} i k x \right) 
\left(
\IIinf \exp\left( \inv{2} i k u \right) f(u) \exp\left( - \inv{2} i k u \right) du 
\right)
\exp\left( \inv{2} i k x \right) dk \\
&=
\PV \inv{{2\pi}} \IIinf \exp\left( -\inv{2} i k (x -u) \right) f(u) \exp\left( \inv{2} i k (x -u) \right) du dk \\
\end{align*}

Now, this expression can't obviously be identified with the delta function as in the single sided transformation.  Suppose we decompose $f$ into grades that 
commute and anticommute with $i$.  That is

\begin{align*}
f &= f_\parallel + f_\perp \\
f_\parallel i &= i f_\parallel  \\
f_\perp  i &= -i f_\perp 
\end{align*}

This is also sufficient to determine how these components of $f$ behave with respect to the exponentials.  We have

\begin{align*}
e^{i\theta} 
&= \sum_m \frac{(i\theta)^m}{m!} \\
&= \cos(\theta) + i\sin(\theta)
\end{align*}

So we also have

\begin{align*}
f_\parallel e^{i\theta} &= e^{i\theta} f_\parallel  \\
f_\perp e^{i\theta} &= e^{-i\theta} f_\perp 
\end{align*}

This gives us
\begin{align*}
\FF^{-1}(\FF(f(x)))
&=
\PV \inv{{2\pi}} \IIinf f_\parallel(u) du dk 
+\PV \inv{{2\pi}} \IIinf f_\perp(u) \exp\left( i k (x -u) \right) du dk \\
&=
\inv{{2\pi}} \IIinf dk \IIinf f_\parallel(u) du +\IIinf f_\perp(u) \delta( x -u ) du \\
\end{align*}

So, we see two things.  First is that any $f_\parallel \ne 0$ produces an unpleasant infinite result.  One could, in a vague sense, allow for odd valued $f_\parallel$, however, if we were to apply this inversion transformation pair to a function time varying multivector function $f(x,t)$, this would then require that the function is odd for all times.  Such a function must be zero valued. % or some odd construction such as a function that is zero everywhere except at some denumerable set of points.

The second thing that we see is that if $f$ entirely anticommutes with $i$, we do recover it with this transform pair, obtaining $f_\perp(x)$.

With respect to Maxwell's equation
this immediately means that this double sided transform pair is of no use, since our pseudoscalar $i = \gamma_0 \gamma_1\gamma_2 \gamma_3$ commutes with our grade two 
field bivector $F$.

\section{ Fourier transforming the spacetime split gradient equation. }

Now, suppose we have a Maxwell like equation of the form

\begin{align}\label{eqn:spacetimeGradientEquation}
\left(\partial_0 + \spacegrad \right) \psi = \rho
\end{align}

Let's take the Fourier transform of this equation.  This gives us

\begin{align*}
\partial_0 \hat{\psi} + \sigma^m \FF(\partial_m \psi) = \hat{\rho}
\end{align*}

Now, we need to look at the middle term in a bit more detail.  For the wave, and heat equations this was evaluated with just an integration
by parts.  Was there any commutation assumption in that previous treatment?  Let's write this out in full to make sure we are cool.

\begin{align*}
\FF(\partial_m \psi) 
&= \inv{(\sqrt{2\pi})^3} \int \left(\partial_m \psi(\Bx, t)\right) \exp\left( -i \Bk \cdot \Bx \right) d^3 x 
\end{align*}

Let's also expand the integral completely, employing a permutation of indexes $\pi(1,2,3) = (m,n,p)$.

\begin{align*}
\FF(\partial_m \psi) 
&= 
\inv{(\sqrt{2\pi})^3} 
\int_{\partial x^p} dx^p
\int_{\partial x^n} dx^n
\int_{\partial x^m} dx^m
\left(\partial_m \psi(\Bx, t)\right) \exp\left( -i \Bk \cdot \Bx \right) \\
\end{align*}

Okay, now we are ready for the integration by parts.  We want a derivative substitution, based on

\begin{align*}
\partial_m &\left( \psi(\Bx, t) \exp\left( -i \Bk \cdot \Bx \right) \right) \\
&= (\partial_m \psi(\Bx, t)) \exp\left( -i \Bk \cdot \Bx \right) + \psi(\Bx, t) \partial_m \exp\left( -i \Bk \cdot \Bx \right) \\
&= (\partial_m \psi(\Bx, t)) \exp\left( -i \Bk \cdot \Bx \right) + \psi(\Bx, t) ( -i k_m ) \exp\left( -i \Bk \cdot \Bx \right) \\
\end{align*}

Observe that we do not wish to assume that the pseudoscalar $i$ commutes with anything except the exponential term, so we have to leave
it sandwiched or on the far right.  We also must take care to not neccessarily commute the exponential itself with $\psi$ or its derivative.
Having noted this we can rearrange as desired for the integration by parts

\begin{align*}
(\partial_m \psi(\Bx, t)) \exp\left( -i \Bk \cdot \Bx \right)
&=
\partial_m \left( \psi(\Bx, t) \exp\left( -i \Bk \cdot \Bx \right) \right) - \psi(\Bx, t) ( -i k_m ) \exp\left( -i \Bk \cdot \Bx \right) \\
\end{align*}

and substitute back into the integral

\begin{align*}
\sigma^m \FF(\partial_m \psi) 
&= 
\inv{(\sqrt{2\pi})^3} 
\int_{\partial x^p} dx^p
\int_{\partial x^n} dx^n
{\left. {\left(\sigma^m \psi(\Bx, t) \exp\left( -i \Bk \cdot \Bx \right) \right)} \right\vert}_{\partial x^m} \\
&- 
\inv{(\sqrt{2\pi})^3} 
\int_{\partial x^p} dx^p
\int_{\partial x^n} dx^n
\int_{\partial x^m} dx^m
\sigma^m \psi(\Bx, t) ( -i k_m )
\exp\left( -i \Bk \cdot \Bx \right) 
\\
\end{align*}

So, we find that the Fourier transform of our spatial gradient is

\begin{align*}
\FF(\grad \psi) = \Bk \hat{\psi} i
\end{align*}

This has the specific ordering of the vector products for our possiblility of non-communative factors.

From this, without making any assumptions about grade, we have the wave number domain equivalent
for the spacetime split of the gradient equation \ref{eqn:spacetimeGradientEquation}

\begin{align}\label{eqn:waveDomainGeneral}
\partial_0 \hat{\psi} + \Bk \hat{\psi} i = \hat{\rho}
\end{align}

\section{ Back to specifics.  Maxwell's equation in wave number domain. }

For Maxwell's equation our field variable $F$ is grade two in the STA basis, and our specific transform pair is:

\begin{align}
\left(\partial_0 + \spacegrad \right) F &= \gamma_0 J/\epsilon_0 c \\
\partial_0 \hat{F} + \Bk \hat{F} i &= \gamma_0 \hat{J}/ \epsilon_0 c
\end{align}

Now, $\exp(i\theta)$ and $i$ commute, and $i$ also commutes with both $F$ and $\Bk$.  This is true since our field $F$ as well as the spatial vector $\Bk$ are grade two in the STA basis.  
Two sign interchanges occur as we commute with each vector factor of these
bivectors.

This allows us to write our transformed equation in the slightly tider form

\begin{align}\label{eqn:toSolve}
\partial_0 \hat{F} + (i \Bk) \hat{F} &= \gamma_0 \hat{J}/ \epsilon_0 c
\end{align}

We want to find a solution to this equation.  If the objects in question were all scalars this would be simple enough, and is a problem of the form

\begin{align}\label{eqn:firstOrder}
f' + a f &= g
\end{align}

For our electromagnetic field our transform is a summation of the following 
form

\begin{align*}
(\BE + i c \BB) (\cos\theta + i \sin\theta)
&=
(\BE \cos\theta - c \BB \sin\theta) + 
i (\BE \sin\theta + c \BB \cos\theta)
\end{align*}

The summation of the integral itself will not change the grades, so $\hat{F}$
is also a grade two multivector.  The dual of our spatial wave number
vector $i\Bk$ is also grade two with basis bivectors $\gamma_m \gamma_n$ very 
much like the magnetic field portions of our field vector $i c \BB$.

Having figured out the grades of all the terms in \ref{eqn:toSolve}, what
does a grade split of this equation yield?  For the equation to be true
do we not need it to be true for all grades?  This would imply

\begin{align*}
(i \Bk) \cdot \hat{F} &= \hat{\rho}/ \epsilon_0 \\
(i \Bk) \wedge \hat{F} &= 0 \\
\partial_0 \hat{F} + (i \Bk) \times \hat{F} &= -\hat{\Bj}/ \epsilon_0 c
\end{align*}

It is kind of interesting that an unmoving charge density contributes nothing
to the time variation of the field in the wave number domain, instead
only the current density (spatial) vectors contribute to our differential
equation.

\subsection{ Solving this first order inhomogeneous problem. } 

Let's remind ourselves what the form of the solution of this inhomogeneous scalar equation 
\ref{eqn:firstOrder} is.  Having just used it for our second order 
wave equation problems, lets use the variation of parameters method again.

\bibliographystyle{plainnat}
\bibliography{myrefs}

\end{document}

\documentclass{article}

\usepackage{amsmath}
\usepackage{mathpazo}

%
% shorthand for bold symbols, convenient for vectors and matrices
%
\newcommand{\Ba}[0]{\mathbf{a}}
\newcommand{\Bb}[0]{\mathbf{b}}
\newcommand{\Bc}[0]{\mathbf{c}}
\newcommand{\Bd}[0]{\mathbf{d}}
\newcommand{\Be}[0]{\mathbf{e}}
\newcommand{\Bf}[0]{\mathbf{f}}
\newcommand{\Bg}[0]{\mathbf{g}}
\newcommand{\Bh}[0]{\mathbf{h}}
\newcommand{\Bi}[0]{\mathbf{i}}
\newcommand{\Bj}[0]{\mathbf{j}}
\newcommand{\Bk}[0]{\mathbf{k}}
\newcommand{\Bl}[0]{\mathbf{l}}
\newcommand{\Bm}[0]{\mathbf{m}}
\newcommand{\Bn}[0]{\mathbf{n}}
\newcommand{\Bo}[0]{\mathbf{o}}
\newcommand{\Bp}[0]{\mathbf{p}}
\newcommand{\Bq}[0]{\mathbf{q}}
\newcommand{\Br}[0]{\mathbf{r}}
\newcommand{\Bs}[0]{\mathbf{s}}
\newcommand{\Bt}[0]{\mathbf{t}}
\newcommand{\Bu}[0]{\mathbf{u}}
\newcommand{\Bv}[0]{\mathbf{v}}
\newcommand{\Bw}[0]{\mathbf{w}}
\newcommand{\Bx}[0]{\mathbf{x}}
\newcommand{\By}[0]{\mathbf{y}}
\newcommand{\Bz}[0]{\mathbf{z}}
\newcommand{\BA}[0]{\mathbf{A}}
\newcommand{\BB}[0]{\mathbf{B}}
\newcommand{\BC}[0]{\mathbf{C}}
\newcommand{\BD}[0]{\mathbf{D}}
\newcommand{\BE}[0]{\mathbf{E}}
\newcommand{\BF}[0]{\mathbf{F}}
\newcommand{\BG}[0]{\mathbf{G}}
\newcommand{\BH}[0]{\mathbf{H}}
\newcommand{\BI}[0]{\mathbf{I}}
\newcommand{\BJ}[0]{\mathbf{J}}
\newcommand{\BK}[0]{\mathbf{K}}
\newcommand{\BL}[0]{\mathbf{L}}
\newcommand{\BM}[0]{\mathbf{M}}
\newcommand{\BN}[0]{\mathbf{N}}
\newcommand{\BO}[0]{\mathbf{O}}
\newcommand{\BP}[0]{\mathbf{P}}
\newcommand{\BQ}[0]{\mathbf{Q}}
\newcommand{\BR}[0]{\mathbf{R}}
\newcommand{\BS}[0]{\mathbf{S}}
\newcommand{\BT}[0]{\mathbf{T}}
\newcommand{\BU}[0]{\mathbf{U}}
\newcommand{\BV}[0]{\mathbf{V}}
\newcommand{\BW}[0]{\mathbf{W}}
\newcommand{\BX}[0]{\mathbf{X}}
\newcommand{\BY}[0]{\mathbf{Y}}
\newcommand{\BZ}[0]{\mathbf{Z}}

\newcommand{\Bzero}[0]{\mathbf{0}}
\newcommand{\Btheta}[0]{\boldsymbol{\theta}}
\newcommand{\Btau}[0]{\boldsymbol{\tau}}
\newcommand{\Bomega}[0]{\boldsymbol{\omega}}

%
% shorthand for unit vectors
%
\newcommand{\acap}[0]{\hat{\Ba}}
\newcommand{\bcap}[0]{\hat{\Bb}}
\newcommand{\ccap}[0]{\hat{\Bc}}
\newcommand{\dcap}[0]{\hat{\Bd}}
\newcommand{\ecap}[0]{\hat{\Be}}
\newcommand{\fcap}[0]{\hat{\Bf}}
\newcommand{\gcap}[0]{\hat{\Bg}}
\newcommand{\hcap}[0]{\hat{\Bh}}
\newcommand{\icap}[0]{\hat{\Bi}}
\newcommand{\jcap}[0]{\hat{\Bj}}
\newcommand{\kcap}[0]{\hat{\Bk}}
\newcommand{\lcap}[0]{\hat{\Bl}}
\newcommand{\mcap}[0]{\hat{\Bm}}
\newcommand{\ncap}[0]{\hat{\Bn}}
\newcommand{\ocap}[0]{\hat{\Bo}}
\newcommand{\pcap}[0]{\hat{\Bp}}
\newcommand{\qcap}[0]{\hat{\Bq}}
\newcommand{\rcap}[0]{\hat{\Br}}
\newcommand{\scap}[0]{\hat{\Bs}}
\newcommand{\tcap}[0]{\hat{\Bt}}
\newcommand{\ucap}[0]{\hat{\Bu}}
\newcommand{\vcap}[0]{\hat{\Bv}}
\newcommand{\wcap}[0]{\hat{\Bw}}
\newcommand{\xcap}[0]{\hat{\Bx}}
\newcommand{\ycap}[0]{\hat{\By}}
\newcommand{\zcap}[0]{\hat{\Bz}}
\newcommand{\thetacap}[0]{\hat{\Btheta}}

%
% to write R^n and C^n in a distinguishable fashion.  Perhaps change this
% to the double lined characters upon figuring out how to do so.
%
\newcommand{\C}[1]{$\mathbb{C}^{#1}$}
\newcommand{\R}[1]{$\mathbb{R}^{#1}$}

%
% various generally useful helpers
%

% derivative of #1 wrt. #2:
\newcommand{\D}[2] {\frac {d#2} {d#1}}

\newcommand{\inv}[1]{\frac{1}{#1}}
\newcommand{\cross}[0]{\times}

\newcommand{\abs}[1]{\lvert{#1}\rvert}
\newcommand{\norm}[1]{\lVert{#1}\rVert}
\newcommand{\innerprod}[2]{\langle{#1}, {#2}\rangle}
\newcommand{\dotprod}[2]{{#1} \cdot {#2}}
\newcommand{\bdotprod}[2]{\left({#1} \cdot {#2}\right)}
\newcommand{\crossprod}[2]{{#1} \cross {#2}}
\newcommand{\tripleprod}[3]{\dotprod{\left(\crossprod{#1}{#2}\right)}{#3}}

\DeclareMathOperator{\Proj}{Proj}
\DeclareMathOperator{\Span}{span}
\DeclareMathOperator{\Sgn}{sgn}
\DeclareMathOperator{\Area}{Area}
\DeclareMathOperator{\Volume}{Volume}

%
% A few miscellaneous things specific to this document
%
\newcommand{\crossop}[1]{\crossprod{#1}{}}

% R2 vector.
\newcommand{\VectorTwo}[2]{
\begin{bmatrix}
 {#1} \\
 {#2}
\end{bmatrix}
}

\newcommand{\VectorN}[1]{
\begin{bmatrix}
{#1}_1 \\
{#1}_2 \\
\vdots \\
{#1}_N \\
\end{bmatrix}
}

\newcommand{\DETuvij}[4]{
\begin{vmatrix}
 {#1}_{#3} & {#1}_{#4} \\
 {#2}_{#3} & {#2}_{#4}
\end{vmatrix}
}

\newcommand{\DETuvwijk}[6]{
\begin{vmatrix}
 {#1}_{#4} & {#1}_{#5} & {#1}_{#6} \\
 {#2}_{#4} & {#2}_{#5} & {#2}_{#6} \\
 {#3}_{#4} & {#3}_{#5} & {#3}_{#6}
\end{vmatrix}
}

\newcommand{\DETuvwxijkl}[8]{
\begin{vmatrix}
 {#1}_{#5} & {#1}_{#6} & {#1}_{#7} & {#1}_{#8} \\
 {#2}_{#5} & {#2}_{#6} & {#2}_{#7} & {#2}_{#8} \\
 {#3}_{#5} & {#3}_{#6} & {#3}_{#7} & {#3}_{#8} \\
 {#4}_{#5} & {#4}_{#6} & {#4}_{#7} & {#4}_{#8} \\
\end{vmatrix}
}

%\newcommand{\DETuvwxyijklm}[10]{
%\begin{vmatrix}
% {#1}_{#6} & {#1}_{#7} & {#1}_{#8} & {#1}_{#9} & {#1}_{#10} \\
% {#2}_{#6} & {#2}_{#7} & {#2}_{#8} & {#2}_{#9} & {#2}_{#10} \\
% {#3}_{#6} & {#3}_{#7} & {#3}_{#8} & {#3}_{#9} & {#3}_{#10} \\
% {#4}_{#6} & {#4}_{#7} & {#4}_{#8} & {#4}_{#9} & {#4}_{#10} \\
% {#5}_{#6} & {#5}_{#7} & {#5}_{#8} & {#5}_{#9} & {#5}_{#10}
%\end{vmatrix}
%}

% R3 vector.
\newcommand{\VectorThree}[3]{
\begin{bmatrix}
 {#1} \\
 {#2} \\
 {#3}
\end{bmatrix}
}


%<misc>
%
\newcommand{\Abs}[1]{{\left\lvert{#1}\right\rvert}}
\newcommand{\spacegrad}[0]{\boldsymbol{\nabla}}
\newcommand{\grad}[0]{\nabla}
\newcommand{\LL}[0]{\mathcal{L}}

% == \partial_{#1} {#2}
\newcommand{\PD}[2]{\frac{\partial {#2}}{\partial {#1}}}
% inline variant
\newcommand{\PDi}[2]{{\partial {#2}}/{\partial {#1}}}

\newcommand{\PDD}[3]{\frac{\partial^2 {#3}}{\partial {#1}\partial {#2}}}
%\newcommand{\PDd}[2]{\frac{\partial^2 {#2}}{{\partial{#1}}^2}}
\newcommand{\PDsq}[2]{\frac{\partial^2 {#2}}{(\partial {#1})^2}}

\newcommand{\Partial}[2]{\frac{\partial {#1}}{\partial {#2}}}
\DeclareMathOperator{\RejName}{Rej}
\newcommand{\Rej}[2]{\RejName_{#1}\left( {#2} \right)}
\newcommand{\Rm}[1]{\mathbb{R}^{#1}}
\newcommand{\Cm}[1]{\mathbb{C}^{#1}}
\newcommand{\conj}[0]{{*}}

%</misc>

% <grade selection>
%
\newcommand{\gpgrade}[2] {{\left\langle{{#1}}\right\rangle}_{#2}}

\newcommand{\gpgradezero}[1] {\gpgrade{#1}{}}
%\newcommand{\gpscalargrade}[1] {{\left\langle{{#1}}\right\rangle}}
%\newcommand{\gpgradezero}[1] {\gpgrade{#1}{0}}

%\newcommand{\gpgradeone}[1] {{\left\langle{{#1}}\right\rangle}_{1}}
\newcommand{\gpgradeone}[1] {\gpgrade{#1}{1}}

\newcommand{\gpgradetwo}[1] {\gpgrade{#1}{2}}
\newcommand{\gpgradethree}[1] {\gpgrade{#1}{3}}
\newcommand{\gpgradefour}[1] {\gpgrade{#1}{4}}
%
% </grade selection>



\newcommand{\adot}[0]{{\dot{a}}}
\newcommand{\bdot}[0]{{\dot{b}}}
% taken for centered dot:
%\newcommand{\cdot}[0]{{\dot{c}}}
%\newcommand{\ddot}[0]{{\dot{d}}}
\newcommand{\edot}[0]{{\dot{e}}}
\newcommand{\fdot}[0]{{\dot{f}}}
\newcommand{\gdot}[0]{{\dot{g}}}
\newcommand{\hdot}[0]{{\dot{h}}}
\newcommand{\idot}[0]{{\dot{i}}}
\newcommand{\jdot}[0]{{\dot{j}}}
\newcommand{\kdot}[0]{{\dot{k}}}
\newcommand{\ldot}[0]{{\dot{l}}}
\newcommand{\mdot}[0]{{\dot{m}}}
\newcommand{\ndot}[0]{{\dot{n}}}
%\newcommand{\odot}[0]{{\dot{o}}}
\newcommand{\pdot}[0]{{\dot{p}}}
\newcommand{\qdot}[0]{{\dot{q}}}
\newcommand{\rdot}[0]{{\dot{r}}}
\newcommand{\sdot}[0]{{\dot{s}}}
\newcommand{\tdot}[0]{{\dot{t}}}
\newcommand{\udot}[0]{{\dot{u}}}
\newcommand{\vdot}[0]{{\dot{v}}}
\newcommand{\wdot}[0]{{\dot{w}}}
\newcommand{\xdot}[0]{{\dot{x}}}
\newcommand{\ydot}[0]{{\dot{y}}}
\newcommand{\zdot}[0]{{\dot{z}}}
\newcommand{\addot}[0]{{\ddot{a}}}
\newcommand{\bddot}[0]{{\ddot{b}}}
\newcommand{\cddot}[0]{{\ddot{c}}}
%\newcommand{\dddot}[0]{{\ddot{d}}}
\newcommand{\eddot}[0]{{\ddot{e}}}
\newcommand{\fddot}[0]{{\ddot{f}}}
\newcommand{\gddot}[0]{{\ddot{g}}}
\newcommand{\hddot}[0]{{\ddot{h}}}
\newcommand{\iddot}[0]{{\ddot{i}}}
\newcommand{\jddot}[0]{{\ddot{j}}}
\newcommand{\kddot}[0]{{\ddot{k}}}
\newcommand{\lddot}[0]{{\ddot{l}}}
\newcommand{\mddot}[0]{{\ddot{m}}}
\newcommand{\nddot}[0]{{\ddot{n}}}
\newcommand{\oddot}[0]{{\ddot{o}}}
\newcommand{\pddot}[0]{{\ddot{p}}}
\newcommand{\qddot}[0]{{\ddot{q}}}
\newcommand{\rddot}[0]{{\ddot{r}}}
\newcommand{\sddot}[0]{{\ddot{s}}}
\newcommand{\tddot}[0]{{\ddot{t}}}
\newcommand{\uddot}[0]{{\ddot{u}}}
\newcommand{\vddot}[0]{{\ddot{v}}}
\newcommand{\wddot}[0]{{\ddot{w}}}
\newcommand{\xddot}[0]{{\ddot{x}}}
\newcommand{\yddot}[0]{{\ddot{y}}}
\newcommand{\zddot}[0]{{\ddot{z}}}

%<bold and dot greek symbols>
%

\newcommand{\Deltadot}[0]{{\dot{\Delta}}}
\newcommand{\Gammadot}[0]{{\dot{\Gamma}}}
\newcommand{\Lambdadot}[0]{{\dot{\Lambda}}}
\newcommand{\Omegadot}[0]{{\dot{\Omega}}}
\newcommand{\Phidot}[0]{{\dot{\Phi}}}
\newcommand{\Pidot}[0]{{\dot{\Pi}}}
\newcommand{\Psidot}[0]{{\dot{\Psi}}}
\newcommand{\Sigmadot}[0]{{\dot{\Sigma}}}
\newcommand{\Thetadot}[0]{{\dot{\Theta}}}
\newcommand{\Upsilondot}[0]{{\dot{\Upsilon}}}
\newcommand{\Xidot}[0]{{\dot{\Xi}}}
\newcommand{\alphadot}[0]{{\dot{\alpha}}}
\newcommand{\betadot}[0]{{\dot{\beta}}}
\newcommand{\chidot}[0]{{\dot{\chi}}}
\newcommand{\deltadot}[0]{{\dot{\delta}}}
\newcommand{\epsilondot}[0]{{\dot{\epsilon}}}
\newcommand{\etadot}[0]{{\dot{\eta}}}
\newcommand{\gammadot}[0]{{\dot{\gamma}}}
\newcommand{\kappadot}[0]{{\dot{\kappa}}}
\newcommand{\lambdadot}[0]{{\dot{\lambda}}}
\newcommand{\mudot}[0]{{\dot{\mu}}}
\newcommand{\nudot}[0]{{\dot{\nu}}}
\newcommand{\omegadot}[0]{{\dot{\omega}}}
\newcommand{\phidot}[0]{{\dot{\phi}}}
\newcommand{\pidot}[0]{{\dot{\pi}}}
\newcommand{\psidot}[0]{{\dot{\psi}}}
\newcommand{\rhodot}[0]{{\dot{\rho}}}
\newcommand{\sigmadot}[0]{{\dot{\sigma}}}
\newcommand{\taudot}[0]{{\dot{\tau}}}
\newcommand{\thetadot}[0]{{\dot{\theta}}}
\newcommand{\upsilondot}[0]{{\dot{\upsilon}}}
\newcommand{\varepsilondot}[0]{{\dot{\varepsilon}}}
\newcommand{\varphidot}[0]{{\dot{\varphi}}}
\newcommand{\varpidot}[0]{{\dot{\varpi}}}
\newcommand{\varrhodot}[0]{{\dot{\varrho}}}
\newcommand{\varsigmadot}[0]{{\dot{\varsigma}}}
\newcommand{\varthetadot}[0]{{\dot{\vartheta}}}
\newcommand{\xidot}[0]{{\dot{\xi}}}
\newcommand{\zetadot}[0]{{\dot{\zeta}}}

\newcommand{\Deltaddot}[0]{{\ddot{\Delta}}}
\newcommand{\Gammaddot}[0]{{\ddot{\Gamma}}}
\newcommand{\Lambdaddot}[0]{{\ddot{\Lambda}}}
\newcommand{\Omegaddot}[0]{{\ddot{\Omega}}}
\newcommand{\Phiddot}[0]{{\ddot{\Phi}}}
\newcommand{\Piddot}[0]{{\ddot{\Pi}}}
\newcommand{\Psiddot}[0]{{\ddot{\Psi}}}
\newcommand{\Sigmaddot}[0]{{\ddot{\Sigma}}}
\newcommand{\Thetaddot}[0]{{\ddot{\Theta}}}
\newcommand{\Upsilonddot}[0]{{\ddot{\Upsilon}}}
\newcommand{\Xiddot}[0]{{\ddot{\Xi}}}
\newcommand{\alphaddot}[0]{{\ddot{\alpha}}}
\newcommand{\betaddot}[0]{{\ddot{\beta}}}
\newcommand{\chiddot}[0]{{\ddot{\chi}}}
\newcommand{\deltaddot}[0]{{\ddot{\delta}}}
\newcommand{\epsilonddot}[0]{{\ddot{\epsilon}}}
\newcommand{\etaddot}[0]{{\ddot{\eta}}}
\newcommand{\gammaddot}[0]{{\ddot{\gamma}}}
\newcommand{\kappaddot}[0]{{\ddot{\kappa}}}
\newcommand{\lambdaddot}[0]{{\ddot{\lambda}}}
\newcommand{\muddot}[0]{{\ddot{\mu}}}
\newcommand{\nuddot}[0]{{\ddot{\nu}}}
\newcommand{\omegaddot}[0]{{\ddot{\omega}}}
\newcommand{\phiddot}[0]{{\ddot{\phi}}}
\newcommand{\piddot}[0]{{\ddot{\pi}}}
\newcommand{\psiddot}[0]{{\ddot{\psi}}}
\newcommand{\rhoddot}[0]{{\ddot{\rho}}}
\newcommand{\sigmaddot}[0]{{\ddot{\sigma}}}
\newcommand{\tauddot}[0]{{\ddot{\tau}}}
\newcommand{\thetaddot}[0]{{\ddot{\theta}}}
\newcommand{\upsilonddot}[0]{{\ddot{\upsilon}}}
\newcommand{\varepsilonddot}[0]{{\ddot{\varepsilon}}}
\newcommand{\varphiddot}[0]{{\ddot{\varphi}}}
\newcommand{\varpiddot}[0]{{\ddot{\varpi}}}
\newcommand{\varrhoddot}[0]{{\ddot{\varrho}}}
\newcommand{\varsigmaddot}[0]{{\ddot{\varsigma}}}
\newcommand{\varthetaddot}[0]{{\ddot{\vartheta}}}
\newcommand{\xiddot}[0]{{\ddot{\xi}}}
\newcommand{\zetaddot}[0]{{\ddot{\zeta}}}

\newcommand{\BDelta}[0]{\boldsymbol{\Delta}}
\newcommand{\BGamma}[0]{\boldsymbol{\Gamma}}
\newcommand{\BLambda}[0]{\boldsymbol{\Lambda}}
\newcommand{\BOmega}[0]{\boldsymbol{\Omega}}
\newcommand{\BPhi}[0]{\boldsymbol{\Phi}}
\newcommand{\BPi}[0]{\boldsymbol{\Pi}}
\newcommand{\BPsi}[0]{\boldsymbol{\Psi}}
\newcommand{\BSigma}[0]{\boldsymbol{\Sigma}}
\newcommand{\BTheta}[0]{\boldsymbol{\Theta}}
\newcommand{\BUpsilon}[0]{\boldsymbol{\Upsilon}}
\newcommand{\BXi}[0]{\boldsymbol{\Xi}}
\newcommand{\Balpha}[0]{\boldsymbol{\alpha}}
\newcommand{\Bbeta}[0]{\boldsymbol{\beta}}
\newcommand{\Bchi}[0]{\boldsymbol{\chi}}
\newcommand{\Bdelta}[0]{\boldsymbol{\delta}}
\newcommand{\Bepsilon}[0]{\boldsymbol{\epsilon}}
\newcommand{\Beta}[0]{\boldsymbol{\eta}}
\newcommand{\Bgamma}[0]{\boldsymbol{\gamma}}
\newcommand{\Bkappa}[0]{\boldsymbol{\kappa}}
\newcommand{\Blambda}[0]{\boldsymbol{\lambda}}
\newcommand{\Bmu}[0]{\boldsymbol{\mu}}
\newcommand{\Bnu}[0]{\boldsymbol{\nu}}
%\newcommand{\Bomega}[0]{\boldsymbol{\omega}}
\newcommand{\Bphi}[0]{\boldsymbol{\phi}}
\newcommand{\Bpi}[0]{\boldsymbol{\pi}}
\newcommand{\Bpsi}[0]{\boldsymbol{\psi}}
\newcommand{\Brho}[0]{\boldsymbol{\rho}}
\newcommand{\Bsigma}[0]{\boldsymbol{\sigma}}
%\newcommand{\Btau}[0]{\boldsymbol{\tau}}
%\newcommand{\Btheta}[0]{\boldsymbol{\theta}}
\newcommand{\Bupsilon}[0]{\boldsymbol{\upsilon}}
\newcommand{\Bvarepsilon}[0]{\boldsymbol{\varepsilon}}
\newcommand{\Bvarphi}[0]{\boldsymbol{\varphi}}
\newcommand{\Bvarpi}[0]{\boldsymbol{\varpi}}
\newcommand{\Bvarrho}[0]{\boldsymbol{\varrho}}
\newcommand{\Bvarsigma}[0]{\boldsymbol{\varsigma}}
\newcommand{\Bvartheta}[0]{\boldsymbol{\vartheta}}
\newcommand{\Bxi}[0]{\boldsymbol{\xi}}
\newcommand{\Bzeta}[0]{\boldsymbol{\zeta}}
%
%</bold and dot greek symbols>
%<infrequent>
%
%\newcommand{\AreaOp}[1]{\AName_{#1}}
%\newcommand{\Babs}[0]{\abs{\BB}}
%\newcommand{\Bcap}[0]{\hat{\BB}}
%\newcommand{\BrPrimeRej}[0]{\rcap(\rcap \wedge \Br')}
%\newcommand{\CA}[0]{\mathcal{A}}
%\newcommand{\Cos}[1]{\cos{\left({#1}\right)}}
%\newcommand{\Det}[1] {\abs{#1}}
%\newcommand{\Dsq}[2] {\frac {\partial^2 {#1}} {\partial {#2}^2}}
%\newcommand{\Exp}[1]{\exp{\left({#1}\right)}}
%\newcommand{\Norm}[1]{\left\lVert{#1}\right\rVert}
%\newcommand{\Sin}[1]{\sin{\left({#1}\right)}}
%\newcommand{\T}[0]{\text{T}}
%\newcommand{\VolumeOp}[1]{\VName_{#1}}
%\newcommand{\agrad}[0]{\Ba \cdot \nabla}
%\newcommand{\alphacap}[0]{\hat{\boldsymbol{\alpha}}}
%\newcommand{\Fcap}[0]{\hat{\BF}}
%\newcommand{\bithree}[0]{{\Bi}_3}
%\newcommand{\bxa}[0]{\Bx\Ba}
%\newcommand{\coordvec}[2]{
%\newcommand{\costheta}[0]{\acap \cdot \xcap}
%\newcommand{\ddt}[1]{\ddot{#1}}
%\newcommand{\ddu}[1] {\frac {d{#1}} {du}}
%\newcommand{\dsqxj}[2] {\frac {\partial^2 {#1}} {\partial {x_{#2}}^2}}
%\newcommand{\dtheta}[1]{\frac{d {#1}}{d \theta}}
%\newcommand{\dt}[1]{\dot{#1}}
%\newcommand{\dt}[1]{\frac{d {#1}}{dt}}
%\newcommand{\dxj}[2] {\frac {\partial {#1}} {\partial {x_{#2}}}}
%\newcommand{\halfPhi}[0]{\frac{\phi}{2}}
%\newcommand{\half}[0]{\inv{2}}
%\newcommand{\inv}[1]{\frac{1}{#1}}
%\newcommand{\laplacian}[0]{\nabla^2}
%\newcommand{\matrixoftx}[3]{
%\newcommand{\nrrp}[0]{\norm{\rcap \wedge \Br'}}
%\newcommand{\oiint}{\bigcirc \hspace{-1.4em} \int \hspace{-.8em} \int}
%\newcommand{\transpose}[1]{{#1}^{\text{T}}}
%\newcommand{\transpose}[1]{{{#1}^{\TextTranspose}}}
%\newcommand{\transpose}[1]{{{#1}^{\text{T}}}}
%\newcommand{\barA}[0]{\bar{A}}
%\newcommand{\qbar}[0]{\bar{q}}
%\newcommand{\qdotbar}[0]{\dot{\bar{q}}}
%
%</infrequent>





\newcommand{\PDSq}[2]{\frac{\partial^2 {#2}}{\partial {#1}^2}}
\DeclareMathOperator{\sinc}{sinc}
\DeclareMathOperator{\PV}{PV}
\newcommand{\FF}[0]{\mathcal{F}}
\newcommand{\IIinf}[0]{ \int_{-\infty}^\infty }

\usepackage[bookmarks=true]{hyperref}

\usepackage{color,cite,graphicx}
   % use colour in the document, put your citations as [1-4]
   % rather than [1,2,3,4] (it looks nicer, and the extended LaTeX2e
   % graphics package. 
\usepackage{latexsym,amssymb,epsf} % don't remember if these are
   % needed, but their inclusion can't do any damage


\title{ 4D Fourier transforms applied to Maxwell's equation. }
\author{Peeter Joot}
\date{ Feb 1, 2009.  Last Revision: $Date: 2009/02/06 00:44:19 $ }

\begin{document}

\maketitle{}
\tableofcontents

\section{ Notation. }

Please see \ref{eqn:notation} below for a summary of prerequisite notation used here.

\section{ Motivation. }

In \cite{PJfirstOrderMaxwell}, a solution of the first order Maxwell equation

\begin{align}
\grad F &= \frac{J }{\epsilon_0 c}
\end{align}

was found to be

\begin{align}
F(\Bx,t) 
&=
\inv{({2\pi})^3} \int 
e^{-i c \Bk t}
\left(
F(\Bu, 0) + \inv{\epsilon_0} \int_{\tau = -\infty}^{t} e^{i c \Bk \tau } \gamma_0 J(\Bu,\tau)  d\tau  
\right)
e^{i \Bk \cdot (\Bx-\Bu)} 
d^3 u
d^3 k
\end{align}

This doesn't have the spacetime uniformity that is expected for a solution of a Lorentz invariant equation.

Similarily, in \cite{PJfourierMaxwellSecondOrder} solutions of the second order Maxwell equation in the Lorentz gauge
$\grad \cdot A = 0$ 

\begin{align*}
F &= \grad \wedge A \\
\grad^2 A &= J/\epsilon_0 c
\end{align*}

were found to be
\begin{align}\label{eqn:fourVectorPotentials}
{A^\mu}(x)
&= \inv{\epsilon_0 c} \int J^\mu(x') G(x - x') d^4 x' \\
G(x)
&= 
\frac{u(x \cdot \gamma_0)}{ (2\pi)^3 }
\int
\sin( \Abs{\Bk} x \cdot \gamma_0 )
\exp\left( -i (\Bk \gamma_0) \cdot x \right)
\frac{d^3 k}{ \Abs{\Bk} }
\end{align}

Here our convolution kernel $G$ also doesn't exhibit a uniform four vector form that one could logically expect.

In these notes an attempt to rework these problems using a 4D spacetime Fourier transform will be made.

\section{ 4D Fourier transform. }

As before we want a multivector friendly Fourier transform pair, and choose the following

\begin{align}\label{eqn:FourierTxDefinition}
\hat{\psi}(k) &= \inv{(\sqrt{2\pi})^4} \IIinf \psi(x) \exp\left( -i k \cdot x \right) d^4 x \\
{\psi}(x) &= \PV \inv{(\sqrt{2\pi})^4} \IIinf \hat{\psi}(k) \exp\left( i k \cdot x \right) d^4 k
\end{align}

Here we use $i = \gamma_0 \gamma_1 \gamma_2 \gamma_3$ as our pseudoscalar, and have to therefore be careful of order
of operations since this does not neccessarily commute with multivector $\psi$ or $\hat{\psi}$ functions.

For our dot product and vectors, with summation over matched upstairs downstairs indexes implied, we write

\begin{align*}
x &= x^\mu \gamma_\mu = x_\mu \gamma^\mu \\
k &= k^\mu \gamma_\mu = k_\mu \gamma^\mu \\
x \cdot k &= x^\mu k_\mu = x_\mu k^\mu
\end{align*}

Finally our differential volume elements are defined to be

\begin{align*}
d^4 x &= dx^0 dx^1 dx^2 dx^3 \\
d^4 k &= dk_0 dk_1 dk_2 dk_3 \\
\end{align*}

Note the opposite pairing of upstairs and downstairs indexes in the coordinates.

\section{ Potential equations. }

\subsection{ Inhomogeneous case. } 

First for the attack is the Maxwell potential equations.  As well as using a 4D transform, having learned how to do Fourier
transformations of multivectors, we will attack this one in vector form as well.  Our equation to invert is

\begin{align*}
\grad^2 A = J/\epsilon_0 c
\end{align*}

There is nothing special to do for the transformation of the current term, but the left hand side will require two integration
parts

\begin{align*}
\FF(\grad^2 A )
&= \inv{(2 \pi)^2} \IIinf \left(\left(\partial_{00} - \sum_m \partial_{mm}\right) A \right) e^{ -i k_\mu x^\mu } d^4 x \\
&= \inv{(2 \pi)^2} \IIinf A \left( (-i k_0)^2 - \sum_m (-i k_m)^2 \right) e^{ -i k_\mu x^\mu } d^4 x \\
\end{align*}

As usual it is required that $A$ and $\partial_\mu A$ vanish at infinity.  Now for the scalar in the interior we have

\begin{align*}
(-i k_0)^2 - \sum_m (-i k_m)^2 
&= -(k_0)^2 + \sum_m (k_m)^2 \\
\end{align*}

But this is just the (negation) of the square of our wave number vector
\begin{align*}
k^2 
&= k_\mu \gamma^\mu \cdot k_\nu \gamma^\nu \\
&= k_\mu k_\nu \gamma^\mu \cdot \gamma^\nu \\
&= 
k_0 k_0 \gamma_0 \cdot \gamma^0 
-\sum_{a,b} k_a k_b \gamma_a \cdot \gamma^b \\
&= (k_0)^2 - \sum_a (k_a)^2
\end{align*}

Putting things back together we have for our potential vector in the wave number domain

\begin{align*}
\hat{A} &= \frac{\hat{J}}{- k^2 \epsilon_0 c}
\end{align*}

Inverting, and substitution for $\hat{J}$ gives us our spacetime domain potential vector in one fell swoop

\begin{align*}
A(x)
&= 
\inv{(\sqrt{2\pi})^4} \IIinf 
\left(
\inv{- k^2 \epsilon_0 c} \inv{(\sqrt{2\pi})^4} \IIinf {J}(x') e^{-i k \cdot x' } d^4 x'
\right)
e^{ i k \cdot x } d^4 k \\
&= 
\inv{({2\pi})^4} \IIinf {J}(x') \inv{- k^2 \epsilon_0 c} e^{ i k \cdot (x - x') } d^4 k d^4 x'
\\
\end{align*}

This allows us to write this entire specific solution to the forced wave equation problem as a convolution integral

\begin{align}\label{eqn:potentialInHomogeneous}
A(x) &= \inv{\epsilon_0 c} \IIinf {J}(x') G(x-x') d^4 x' \\
G(x) &= \frac{-1}{ ({2\pi})^4} \IIinf \frac{e^{ i k \cdot x }}{ k^2 } d^4 k 
\end{align}

Pretty slick looking, but actually also problematic if one thinks about it.  Since $k^2$ is null in some cases
$G(x)$ may blow up in some conditions.  My assumption however, is that a well defined meaning can be associated
with this integral, I just do not know what it is yet.  A way to define this more exactly may require
picking a more specific orthonormal basis once the exact character of $J$ is known.

\subsection{ The homogeneous case. }

The missing element here is the addition of any allowed homogeneous solutions to the wave equation.
The form of such solutions cannot be obtained with the 4D transform since that produces

\begin{align*}
-k^2 \hat{A} = 0
\end{align*}

and no meaningful inversion of that is possible.

For the homogeneous problem we are forced to reexpress the spacetime Laplacian with an explicit bias towards either time or a specific direction in space, and attack with a Fourier transform on the remaining coordinates.  This has been done previously, but we can
revisit this using our new vector transform.

Now we switch to a spatial Fourier transform

\begin{align}\label{eqn:3DFourierTxDefinition}
\hat{\psi}(\Bk, t) &= \inv{(\sqrt{2\pi})^3} \IIinf \psi(\Bx, t) \exp\left( -i \Bk \cdot \Bx \right) d^3 x \\
{\psi}(\Bx, t) &= \PV \inv{(\sqrt{2\pi})^3} \IIinf \hat{\psi}(\Bk, t) \exp\left( i \Bk \cdot \Bx \right) d^3 k
\end{align}

Using a spatial transform we have

\begin{align*}
\FF((\partial_{00} - \sum_m \partial_{mm}) A)
&= \partial_{00} \hat{A} - \sum_m \hat{A} (-i k_m)^2 
\end{align*}

Carefully keeping the pseudoscalar factors all on the right of our vector as the integration by parts was performed doesn't make a difference since we just end up with a scalar in the end.  Our equation in the wave number domain is then just

\begin{align*}
\partial_{tt} \hat{A}(\Bk,t) + (c^2 \Bk^2) \hat{A}(\Bk,t) &= 0 %c \hat{J}(\Bk,t)/\epsilon_0
\end{align*}

with exponential solutions

\begin{align*}
\hat{A}(\Bk, t) &= C(\Bk) \exp(\pm i c \Abs{\Bk} t)
\end{align*}

In particular, for $t = 0$ we have

\begin{align*}
\hat{A}(\Bk, 0) &= C(\Bk)
\end{align*}

Reassembling then gives us our homogeneous solution

\begin{align*}
{A}(\Bx, t)
&= 
\inv{(\sqrt{2\pi})^3} \IIinf 
\left( \inv{(\sqrt{2\pi})^3} \IIinf A(\Bx', 0) e^{ -i \Bk \cdot \Bx' } d^3 x' \right) e^{\pm i c \Abs{\Bk} t}
e^{ i \Bk \cdot \Bx } d^3 k
\end{align*}

This is

\begin{align}\label{eqn:potentialHomogeneous}
{A}(\Bx, t) &= \IIinf A(\Bx', 0) G( \Bx - \Bx' ) d^3 x' \\
G(\Bx) &= \inv{({2\pi})^3} \IIinf \exp\left( i \Bk \cdot \Bx \pm i c \Abs{\Bk} t \right) d^3 k
\end{align}

Here also we have to be careful to keep the Green's function on the right hand side of $A$ since they 
won't generally commute.

\subsection{ Summarizing. }

Assembling both the homogeneous and inhomogeneous parts for a complete solution we have for the Maxwell
four vector potential

\begin{align}
A(x) &= \IIinf \left( A(\Bx', 0) H( \Bx - \Bx' ) + \inv{\epsilon_0 c} \IIinf {J}(x') G(x-x') dx^0 \right) dx^1 dx^2 dx^3 \\
H(\Bx) &= \inv{({2\pi})^3} \IIinf \exp\left( i \Bk \cdot \Bx  \pm i c \Abs{\Bk} t \right) d^3 k \\
G(x) &= \frac{-1}{ ({2\pi})^4} \IIinf \frac{e^{ i k \cdot x }}{ k^2 } d^4 k
\end{align}

Here for convienence both four vectors and spatial vectors were used with

\begin{align*}
x &= x^\mu \gamma_\mu \\
\Bx &= x^m \sigma_m = x \wedge \gamma_0
\end{align*}

As expected, operating where possible in a Four vector context does produce a simpler convolution kernel for the vector potential.

\section{ First order Maxwell equation treatment. }

Now we want to Fourier transform Maxwell's equation directly.  That is

\begin{align*}
\FF(\grad F = J/\epsilon_0 c)
\end{align*}

For the LHS we have

\begin{align*}
\FF(\grad F)
&= \FF(\gamma^\mu \partial_\mu F) \\
&= \gamma^\mu \inv{(2\pi)^2} \IIinf (\partial_\mu F) e^{ - i k \cdot x } d^4 x \\
&= -\gamma^\mu \inv{(2\pi)^2} \IIinf F \partial_\mu (e^{ - i k_\sigma x^\sigma }) d^4 x \\
&= -\gamma^\mu \inv{(2\pi)^2} \IIinf F (-i k_\mu ) e^{ - i k \cdot x } d^4 x \\
&= -i \gamma^\mu k_\mu \inv{(2\pi)^2} \IIinf F e^{ - i k \cdot x } d^4 x \\
&= -i k \hat{F}
\end{align*}

This gives us

\begin{align*}
-i k \hat{F} = \hat{J}/\epsilon_0 c
\end{align*}

So to solve the forced Maxwell equation we have only to inverse transform the following

\begin{align*}
\hat{F} = \inv{ -i k \epsilon_0 c} \hat{J}
\end{align*}

This is 

\begin{align*}
{F} 
&= \inv{(\sqrt{2\pi})^4} \IIinf \inv{ -i k \epsilon_0 c} \left( \inv{(\sqrt{2\pi})^4} \IIinf J(x') e^{ -i k \cdot x' } d^4 x' \right) e^{ i k \cdot x } d^4 k \\
\end{align*}

Adding to this a solution to the homogeneous equation we now have a complete solution in terms of the given four current density and an
initial field wave packet

\begin{align*}
{F} &= 
\inv{({2\pi})^3} \int e^{ -i c \Bk t } F(\Bx', 0) e^{ i \Bk \cdot (\Bx-\Bx') } d^3 x' d^3 k  
+
\inv{ ({2\pi})^4 \epsilon_0 c} \int \frac{i}{ k } J(x') e^{ i k \cdot (x - x') } d^4 k d^4 x' \\
\end{align*}

Observe that we can't make a single sided Green's function to convolve $J$ with since the vectors $k$ and $J$ may not commute.

As expected working in a relativistic context for our inherently relativistic equation turns out to be much simpler and produce a simpler result.  As before 
trying to actually evaluate these integrals is a different story.

\section{ Appendix.  Summary of Notation used. }

% 
% NOTATION BOILERPLATE BASE STOLEN FROM ../geometric-algebra/lagrangian_field_density.ltx
%
For standalone purposes, here is a summary of the notation and definitions that will be used.  Greek letters range over all indexes and
english indexes range over $1,2,3$.  Bold vectors are spatial enties and non-bold is used for four vectors.

\begin{equation*}\label{eqn:notation}
\begin{array}{l l l}
\gamma_{\mu} & & \quad \mbox{Four vector basis vector)} \\
& & \quad \mbox{($\gamma_{\mu} \cdot \gamma_{\nu} = \pm {\delta^{\mu}}_{\nu}$)} \\
{(\gamma_0)}^2 {(\gamma_i)}^2 &= -1 & \quad \mbox{Minkowski metric} \\
\sigma_i = \sigma^i &= \gamma_{i} \wedge \gamma_0 & \quad \mbox{Spatial basis bivector. ($\sigma_i \cdot \sigma_j = \delta_{ij}$)} \\
%                    &= \gamma_{i0} \\
i &= \gamma_{0} \wedge \gamma_1 \wedge \gamma_{2} \wedge \gamma_3 & \quad \mbox{Four-vector pseudoscalar} \\
%  &= \gamma_{0123} \\
\gamma^{\mu} \cdot \gamma_{\nu} &= {\delta^{\mu}}_{\nu} & \quad \mbox{Reciprocal basis vectors} \\
x^{\mu} &= x \cdot \gamma^{\mu} & \quad \mbox{Vector coordinate} \\
x_{\mu} &= x \cdot \gamma_{\mu} & \quad \mbox{Coordinate for reciprocal basis} \\
x &= \sum \gamma_{\mu} x^{\mu} & \quad \mbox{Four vector in terms of coordinates} \\
  &= \sum \gamma^{\mu} x_{\mu} \\
\BE &= \sum E^i \sigma_i & \quad \mbox{Electric field spatial vector} \\
\BB &= \sum B^i \sigma_i & \quad \mbox{Magnetic field spatial vector} \\
J &= \sum \gamma_{\mu} J^{\mu} & \quad \mbox{Current density four vector.} \\
  &= \sum \gamma^{\mu} J_{\mu} \\
F &= \BE + i c \BB & \quad \mbox{Faraday bivector} \\
  &= F^{\mu\nu} \gamma_\mu \wedge \gamma_\nu & \quad \mbox{in terms of Faraday tensor} \\
x^{0} &= x \cdot \gamma^0 & \quad \mbox{Time coordinate (length dim.)} \\
      &= c t \\
\Bx &= x \wedge \gamma_0 & \quad \mbox{Spatial vector} \\
    &= x^i \sigma_i \\
J^{0} &= J \cdot \gamma^0 & \quad \mbox{Charge density.} \\
      &= c \rho & \quad \mbox{(current density dimensions.)} \\
\BJ &= J \wedge \gamma_0 & \quad \mbox{Current density spatial vector} \\
    &= \sum J^i \sigma_i \\
\partial_{\mu} &= \PDi{x^\mu}{} & \quad \mbox{Partial of index up coord.} \\
\grad &= \sum \gamma^{\mu} \partial/\partial {x^{\mu}} & \quad \mbox{Spacetime gradient} \\
      &= \sum \gamma^{\mu}\partial_{\mu} \\
      &= \sum \gamma_{\mu} \partial/\partial {x_{\mu}} \\
      &= \sum \gamma_{\mu}\partial^{\mu} \\
\spacegrad &= \sum \sigma^{i} \partial/\partial{x^{i}} & \quad \mbox{Spatial gradient} \\
           &= \sum \sigma^{i}\partial_{i} \\
\PV \IIinf &= \lim_{R\rightarrow \infty} \int_{R}^R & \quad \mbox{Integral Principle value} \\
\hat{A}(k) &= \FF(A(x)) & \quad \mbox{Fourier transform of $A$} \\ 
{A}(x) &= \FF^{-1}(A(k)) & \quad \mbox{Inverse Fourier transform} \\ 
\grad^2 A
   &= (\grad \cdot \grad) A & \quad \mbox{Four Laplacian. } \\
   &= (\partial_{00} - \sum_k \partial_{kk}) A & \\
x^2 &= x \cdot x & \quad \mbox{Four vector square. } \\
    &= x^\mu x_\mu \\
\Bx^2 &= \Bx \cdot \Bx & \quad \mbox{Spatial vector square. } \\
    &= {x^k}^2 \\
    &= \Abs{\Bx}^2 \\
d^4 x &= dx^0 dx^1 dx^2 dx^3 & \quad \mbox{Four volume element. } \\
d^3 x &= dx^1 dx^2 dx^3 & \quad \mbox{Spatial volume element. } \\
\exp(i\Bk\phi) &= 
\cos(\Abs{\Bk}\phi) + \frac{i \Bk}{\Abs{i\Bk}} \sin(\Abs{\Bk}\phi) & \quad \mbox{bivector exponential. } \\
\end{array}
\end{equation*}

Summation convention, where summation over all sets of matched upper and lower indexes is implied. %, will be in effect from this point on.

\bibliographystyle{plainnat}
\bibliography{myrefs}

\end{document}

%
% Copyright � 2012 Peeter Joot.  All Rights Reserved.
% Licenced as described in the file LICENSE under the root directory of this GIT repository.
%

% 
% 
\chapter{Fourier series Vacuum Maxwell's equations}\label{chap:PJFourierVacuum}
\index{Maxwell's equations!Fourier series}
%\date{Feb 03, 2009.  fourierSeriesMaxwell.tex}

\section{Motivation}

In \citep{bohm1989qt}, 
after finding a formulation of Maxwell's equations that he likes, his next
step is to assume the electric and magnetic fields can be expressed in 
a 3D Fourier series form, with periodicity in some repeated volume 
of space, and then proceeds to evaluate the energy of the 
field.

\subsection{Notation}

See the notational table \chapcite{notationTable} for much of the notation
assumed here.

\section{Setup}

Let us try this.  Instead of using the sine and cosine Fourier series
which looks more complex than it ought to be, use of a complex exponential
ought to be cleaner.

\subsection{3D Fourier series in complex exponential form}

For a multivector function \(f(\Bx, t)\), periodic in some rectangular spatial volume, let us assume that we have a
3D Fourier series representation.

Define the element of volume for our fundamental wavelengths to be the region bounded by three intervals in the \(x^1, x^2, x^3\) directions respectively

\begin{equation}\label{eqn:fourierSeriesMaxwell:20}
\begin{aligned}
I_1 &= [ a^1, a^1 + \lambda_1 ] \\
I_2 &= [ a^2, a^2 + \lambda_2 ] \\
I_3 &= [ a^3, a^3 + \lambda_3 ] \\
\end{aligned}
\end{equation}

Our assumed Fourier representation is then

\begin{equation}\label{eqn:fourierSeriesMaxwell:40}
\begin{aligned}
f(\Bx, t) &= \sum_{\Bk} \hat{f}_{\Bk}(t) \exp\left( - \sum_j \frac{2 \pi i k_j x^j}{\lambda_j} \right)
\end{aligned}
\end{equation}

Here \(\hat{f}_{\Bk} = \hat{f}_{\{k_1, k_2, k_3\}}\) is indexed over a triplet of integer values, and the \(k_1, k_2, k_3\) indices take on all integer values in the \([-\infty, \infty]\) range.

Note that we also wish to allow \(i\) to not just be a generic complex number, but allow for the use of either the Euclidean or Minkowski pseudoscalar

\begin{equation}\label{eqn:fourierSeriesMaxwell:60}
\begin{aligned}
i = \gamma_0 \gamma_1 \gamma_2 \gamma_3 = \sigma_1 \sigma_2 \sigma_3
\end{aligned}
\end{equation}

Because of this we should not assume that we can commute \(i\), or our exponentials with the functions \(f(\Bx,t)\), or \(\hat{f}_{\Bk}(t)\).

\begin{equation}\label{eqn:fourierSeriesMaxwell:80}
\begin{aligned}
\int_{x^1 = \partial I_1} &\int_{x^2 = \partial I_2} \int_{x^3 = \partial I_3} f(\Bx, t) 
e^{ 2 \pi i m_j x^j/\lambda_j}
dx^1 dx^2 dx^3 \\
&= \sum_{\Bk} \hat{f}_{\Bk}(t) \int_{x^1 = \partial I_1} \int_{x^2 = \partial I_2} \int_{x^3 = \partial I_3} dx^1 dx^2 dx^3 e^{ 2 \pi i (m_j - k_j) x^j/\lambda_j} dx^1 dx^2 dx^3
\end{aligned}
\end{equation}

But each of these integrals is just \(\delta_{\Bk,\Bm} \lambda_1 \lambda_2 \lambda_3\), giving us

\begin{equation}\label{eqn:fourierSeriesMaxwell:100}
\begin{aligned}
\hat{f}_{\Bk}(t)
&= \inv{\lambda_1 \lambda_2 \lambda_3 } \int_{x^1 = \partial I_1} \int_{x^2 = \partial I_2} \int_{x^3 = \partial I_3} f(\Bx, t) \exp\left( \sum_j \frac{2 \pi i k_j x^j}{\lambda_j} \right) dx^1 dx^2 dx^3 \\
\end{aligned}
\end{equation}

To tidy things up 
lets invent (or perhaps abuse) some notation to tidy things up.  As a subscript on our Fourier coefficients we have used \(\Bk\) as an index.
Let us also use it as a vector, and define

\begin{equation}\label{eqn:fourierSeriesMaxwell:120}
\begin{aligned}
\Bk \equiv 2 \pi \sum_m \frac{\sigma^m k_m}{\lambda_m}
\end{aligned}
\end{equation}

With our spatial vector \(\Bx\) written

\begin{equation}\label{eqn:fourierSeriesMaxwell:140}
\begin{aligned}
\Bx = \sum_m \sigma_m x^m
\end{aligned}
\end{equation}

We now have a \(\Bk \cdot \Bx\) term in the exponential, and can remove when desirable the coordinate summation.  If we write \(V = \lambda_1 \lambda_2 \lambda_3\)
it leaves a nice tidy notation for the 3D Fourier series over the volume

\begin{equation}\label{eqn:fourierSeriesMaxwell:160}
\begin{aligned}
f(\Bx, t) &= \sum_{\Bk} \hat{f}_{\Bk}(t) e^{ - i \Bk \cdot \Bx } \\
\hat{f}_{\Bk}( t) &= \inv{V} \int f(\Bx, t) e^{ i \Bk \cdot \Bx } d^3 x
\end{aligned}
\end{equation}

This allows us to proceed without caring about the specifics of the lengths of the sides of the rectangular prism that defines the periodicity of the signal
in question.

\subsection{Vacuum equation}

Now that we have a desirable seeming Fourier series representation, we 
want to apply this to Maxwell's equation for the vacuum.  We will use the 
STA formulation of Maxwell's equation, but use the unit convention of Bohm's
book.

In \chapcite{PJrayleighJeans} the STA equivalent to Bohm's notation 
for Maxwell's equations was found to be

\begin{equation}\label{eqn:fourier_series_maxwell:maxwell}
\begin{aligned}
F &= \bcE + i\bcH \\
J &= (\rho + \Bj) \gamma_0 \\
\grad F &= 4 \pi J
\end{aligned}
\end{equation}

This is the CGS form of Maxwell's equation, but with the old style \(\bcH\) for \(c\BB\), and \(\bcE\) for \(\BE\).  In more recent texts \(\bcE\) (as a non-vector) is reserved for electromotive flux.  In this set of notes I use Bohm's notation, since the aim is to clarify for myself aspects of his treatment.

For the vacuum equation, we make an explicit spacetime split by premultiplying with \(\gamma_0\)

\begin{equation}\label{eqn:fourierSeriesMaxwell:180}
\begin{aligned}
\gamma_0 \grad 
&= \gamma_0 
\lr{ \gamma^0 \partial_0 + \gamma^k \partial_k } \\
&= \partial_0 - \gamma^k \gamma_0 \partial_k \\
&= \partial_0 + \gamma_k \gamma_0 \partial_k \\
&= \partial_0 + \sigma_k \partial_k \\
&= \partial_0 + \spacegrad \\
\end{aligned}
\end{equation}

So our vacuum equation is just

\begin{equation}\label{eqn:fourier_series_maxwell:vacuumMaxwell}
\begin{aligned}
(\partial_0 + \spacegrad) F = 0
\end{aligned}
\end{equation}

\section{First order vacuum solution with Fourier series}

\subsection{Basic solution in terms of undetermined coefficients}

Now that a notation for the 3D Fourier series has been established, we
can assume a series solution for our field of the form

\begin{equation}\label{eqn:fourier_series_maxwell:assumed}
\begin{aligned}
F(\Bx,t) = \sum_{\Bk} \hat{F}_{\Bk}(t) e^{-2\pi i k_j x^j/\lambda_j}
\end{aligned}
\end{equation}

can now apply this to the vacuum Maxwell equation \eqnref{eqn:fourier_series_maxwell:vacuumMaxwell}.
This gives us

\begin{equation}\label{eqn:fourierSeriesMaxwell:200}
\begin{aligned}
\sum_{\Bk} \left(\partial_t \hat{F}_{\Bk}(t) \right) e^{-2\pi i k_j x^j/\lambda_j}
&= -c \sum_{\Bk, m} \sigma^m \hat{F}_{\Bk}(t) \PD{x^m}{} e^{-2\pi i k_j x^j/\lambda_j} \\
&= -c \sum_{\Bk, m} \sigma^m \hat{F}_{\Bk}(t) \left(-2 \pi \frac{k_m}{\lambda_m}\right) e^{-2\pi i k_j x^j/\lambda_j} \\
&= 2 \pi c \sum_{\Bk} \sum_m \frac{\sigma^m k_m}{\lambda_m} \hat{F}_{\Bk}(t) i e^{-2\pi i k_j x^j/\lambda_j} \\
\end{aligned}
\end{equation}


Note that \(i\) commutes with \(\Bk\) and since \(F\) is also an STA bivector \(i\) commutes with \(F\).  Putting all this together we have

\begin{equation}\label{eqn:fourierSeriesMaxwell:220}
\begin{aligned}
\sum_{\Bk} \left(\partial_t \hat{F}_{\Bk}(t) \right) e^{-i \Bk \cdot \Bx }
&= i c \sum_{\Bk} \Bk \hat{F}_{\Bk}(t) e^{- i \Bk \cdot \Bx } \\
\end{aligned}
\end{equation}

Term by term we now have a (big ass, triple infinite) set of very simple first order differential equations, one for each \(\Bk\) triplet of indices.  Specifically this is

\begin{equation}\label{eqn:fourierSeriesMaxwell:240}
\begin{aligned}
\hat{F}_{\Bk}' &= i c \Bk \hat{F}_{\Bk}
\end{aligned}
\end{equation}

With solutions

\begin{equation}\label{eqn:fourierSeriesMaxwell:260}
\begin{aligned}
\hat{F}_{0} &= C_{0} \\
\hat{F}_{\Bk} &= \exp\left(i c \Bk t \right) C_{\Bk} \\
\end{aligned}
\end{equation}

Here \(C_{\Bk}\) is an undetermined STA bivector.  For now we keep this undetermined coefficient on the right hand side of the exponential since no demonstration that it commutes with a factor of the form \(\exp(i\Bk\phi)\).  Substitution back into our assumed solution sum we have a solution to Maxwell's equation, in terms of a set of as yet undetermined (bivector) coefficients

\begin{equation}\label{eqn:fourierSeriesMaxwell:280}
\begin{aligned}
F(\Bx,t) = C_0 + \sum_{\Bk \ne 0} \exp\left(i c \Bk t \right) C_{\Bk} \exp(-i \Bk \cdot \Bx )
\end{aligned}
\end{equation}

The special case of \(\Bk = 0\) is now seen to be not so special and can be brought into the sum.  

\begin{equation}\label{eqn:fourierSeriesMaxwell:300}
\begin{aligned}
F(\Bx,t) = \sum_{\Bk} \exp\left(i c \Bk t \right) C_{\Bk} \exp(-i \Bk \cdot \Bx )
\end{aligned}
\end{equation}

We can also 
take advantage of the bivector nature of \(C_{\Bk}\), which implies the complex exponential can commute to the left, since the two fold commutation with the pseudoscalar with change sign twice.
%  A similar right commutation of the \(i\Bk\) exponential cannot be justified, and without more thought I am unsure if it can be allowed?

\begin{equation}\label{eqn:fourier_series_maxwell:undetermined}
\begin{aligned}
F(\Bx,t) = \sum_{\Bk} 
\exp\left(i \Bk c t \right) 
\exp\left(-i \Bk \cdot \Bx \right) 
C_{\Bk} 
\end{aligned}
\end{equation}

\subsection{Solution as time evolution of initial field}

Now, observe the form of this sum for \(t=0\).  This is

\begin{equation}\label{eqn:fourierSeriesMaxwell:320}
\begin{aligned}
F(\Bx,0) 
&= \sum_{\Bk} C_{\Bk} \exp(-i \Bk \cdot \Bx ) \\
\end{aligned}
\end{equation}

So, the \(C_k\) coefficients are precisely the Fourier coefficients of \(F(\Bx,0)\).  This is to be expected having repeatedly seen similar results in the Fourier transform treatments of 
\chapcite{PJfourierMaxwellSecondOrder}, \chapcite{PJfirstOrderMaxwell}, and \chapcite{PJ4dFourier}.
We then have an equation for the complete time evolution of any spatially periodic electrodynamic field in terms of the field value at all points in the region at some initial time.  Summarizing so far this is

\begin{equation}\label{eqn:fourierSeriesMaxwell:340}
\begin{aligned}
F(\Bx,t) &= \sum_{\Bk} \exp\left(i c \Bk t \right) 
C_{\Bk}
\exp(-i \Bk \cdot \Bx) \\
C_{\Bk}
&= \inv{V} \int F(\Bx', 0) \exp\left( i\Bk \cdot \Bx' \right) d^3 x'
\end{aligned}
\end{equation}

Regrouping slightly we can write this as a convolution with a Fourier kernel (a Green's function).  That is

\begin{equation}\label{eqn:fourier_series_maxwell:bivectorSolNonGreens}
\begin{aligned}
F(\Bx,t) &= \inv{V} \int \sum_{\Bk} \exp\left( i \Bk ct \right) \exp\left( i \Bk \cdot (\Bx' - \Bx) \right) F(\Bx', 0) d^3 x'
\end{aligned}
\end{equation}

Or
\begin{equation}\label{eqn:fourier_series_maxwell:bivectorSolution}
\begin{aligned}
F(\Bx,t) &= \int G(\Bx - \Bx', t) F(\Bx', 0) d^3 x' \\
G(\Bx,t) &= \inv{V} \sum_{\Bk} \exp\left( i \Bk ct \right) \exp\left( -i \Bk \cdot \Bx \right)
\end{aligned}
\end{equation}

Okay, that is cool.  We have now got the basic periodicity result directly from Maxwell's equation in one shot.  No need to drop down to
potentials, or even the separate electric or magnetic components of our field \(F = \bcE + i \bcH\).

\subsection{Prettying it up?  Questions of commutation}

Now, it is tempting here to write 
\eqnref{eqn:fourier_series_maxwell:undetermined}
as a single exponential

% k = kcappa g_0 |k|
% kcap g_0 = kcappa 
\begin{equation}\label{eqn:fourier_series_maxwell:isItValid}
\begin{aligned}
F(\Bx,t) 
%&= \sum_{\Bk} e^{i \Abs{\Bk}( \kcap c t - \kcap \cdot \Bx)} C_{\Bk} \\
&= \sum_{\Bk} \exp\left(i \Bk c t - i\Bk \cdot \Bx \right) C_{\Bk} \quad\quad \mbox{VALID?}
\end{aligned}
\end{equation}

This would probably allow for a prettier four vector form in terms of \(x = x^\mu \gamma_\mu\) replacing the separate \(\Bx\) and \(x^0 = ct\) terms.
However, 
such a grouping is not allowable unless one first demonstrates that \(e^{i \Bu }\), and \(e^{i \alpha }\), for spatial vector \(\Bu\) and scalar \(\alpha\) commute!

To demonstrate that this is in fact the case 
note that exponential
of this dual spatial vector can be written

\begin{equation}\label{eqn:fourierSeriesMaxwell:360}
\begin{aligned}
\exp( i \Bu ) &= \cos( \Bu ) + i \sin( \Bu ) \\
\end{aligned}
\end{equation}

This spatial vector cosine, \(\cos(\Bu)\), is a scalar (even powers only), and our sine, \(\sin(\Bu) \propto \Bu\), is a spatial vector in the direction of \(\Bu\) (odd powers leaves a vector times a scalar).  Spatial vectors commute with \(i\) (toggles sign twice percolating its way through), therefore pseudoscalar exponentials also commute with \(i\).

This will simplify a lot, and it shows that \eqnref{eqn:fourier_series_maxwell:isItValid} is in fact a valid representation.

Now, there is one more question of commutation here.  Namely, does a dual spatial vector exponential commute with the field itself
(or equivalently, one of the Fourier coefficients).

Expanding such a product and attempting term by term commutation should show

\begin{equation}\label{eqn:fourierSeriesMaxwell:380}
\begin{aligned}
e^{i\Bu} F
&= (\cos \Bu + i\sin\Bu) (\bcE + i\bcH) \\
&= i\sin\Bu (\bcE + i\bcH) + (\bcE + i\bcH) \cos\Bu \\
&= i (\sin\Bu) \bcE - (\sin\Bu) \bcH + F \cos\Bu \\
&= i (-\bcE \sin\Bu + 2 \bcE \cdot \sin\Bu ) + (\bcH \sin\Bu - 2 \bcH \cdot \sin\Bu ) + F \cos\Bu \\
&= 2 \sin\Bu \cdot (\bcE - \bcH) + F (\cos\Bu -i\sin\Bu) \\
\end{aligned}
\end{equation}

That is
\begin{equation}\label{eqn:fourier_series_maxwell:anticommutes}
\begin{aligned}
e^{i\Bu} F &= 2 \sin\Bu \cdot (\bcE - \bcH) + F e^{-i\Bu}
\end{aligned}
\end{equation}

This exponential has one anticommuting term, but also has a scalar component introduced by the portions of the electric
and magnetic fields that are colinear with the spatial vector \(\Bu\).

%
%Would this look any tidier in terms of unit wave number vector \(\Bk = \Abs{\Bk} \kcap\)?  Let us see
%
%\begin{align*}
%F(\Bx,t) = \sum_{\Bk} \exp\left(-i \Abs{\Bk}(\kcap \cdot \Bx - \kcap c t) \right) C_{\Bk} 
%\end{align*}
%FIXME:EXP: above.
%
%Perhaps not.
%
%One thing we may do however, is perform a summation swap and sum over all
%triplets \(-\Bk\) instead, with a redefition of the undetermined
%coefficients \(C_{\Bk}\) as \(C_{-\Bk}\) (incorporating the effects of that sign swap into the value of these coefficients).  This takes the sign out of the exponential and pretties it up slightly.
%
%\begin{align*}
%F(\Bx,t) = \sum_{\Bk} e^{i \Bk \cdot \Bx - i\Bk c t} C_{\Bk} 
%\end{align*}
%FIXME:EXP: above.
%
%There was also the notational trick noticed in the Fourier transform treatment 
%where a conversion of these separate space and time exponential factors into a single
%four vector dot product was possible.  That should work here a bit better than in the Fourier transform case.
%
%FIXME: detail that here.  Want to see what that implies for a Lorentz transformation of the field.

\section{Field Energy and momentum}

Given that we have the same structure for our four vector potential solutions as the complete bivector field, it does not appear that there is much
reason to work in the second order quantities.  Following Bohm we should now be prepared to express the field energy density and
momentum density in terms of the Fourier coefficients, however unlike Bohm, let us try this using the first order 
solutions found above.

In CGS units (see \chapcite{PJrayleighJeans} for verification) these field energy and momentum densities (Poynting vector \(\BP\)) are, respectively

\begin{equation}\label{eqn:fourierSeriesMaxwell:400}
\begin{aligned}
E &= \inv{8\pi} 
\lr{ {\bcE}^2 + \bcH^2 } \\
\BP &= \inv{4\pi} (\bcE \cross \bcH )
\end{aligned}
\end{equation}

Given that we have a complete field equation without an explicit separation of electric and magnetic components, perhaps this
is easier to calculate from the stress energy four vector for energy/momentum.  In CGS units this must be

\begin{equation}\label{eqn:fourierSeriesMaxwell:420}
\begin{aligned}
T(\gamma_0) &= \inv{8\pi} F \gamma_0 \tilde{F}
\end{aligned}
\end{equation}

An expansion of this to verify the CGS conversion seems worthwhile.

\begin{equation}\label{eqn:fourierSeriesMaxwell:440}
\begin{aligned}
T(\gamma_0) 
&= \inv{8\pi} F \gamma_0 \tilde{F} \\
&= \frac{-1}{8\pi} (\bcE + i\bcH) \gamma_0 (\bcE + i\bcH) \\
&= \frac{1}{8\pi} (\bcE + i\bcH) (\bcE - i\bcH) \gamma_0 \\
&= \frac{1}{8\pi} \left( \bcE^2 - (i\bcH)^2 + i(\bcH \bcE - \bcE \bcH) \right) \gamma_0 \\
&= \frac{1}{8\pi} \left( \bcE^2 + \bcH^2 + 2 i^2 \bcH \cross \bcE \right) \gamma_0 \\
&= \frac{1}{8\pi} \left( \bcE^2 + \bcH^2 \right) \gamma_0 + \inv{4 \pi} \left(\bcE \cross \bcH \right) \gamma_0 \\
\end{aligned}
\end{equation}

Good, as expected we have 

\begin{equation}\label{eqn:fourierSeriesMaxwell:460}
\begin{aligned}
E &= T(\gamma_0) \cdot \gamma_0 \\
\BP &= T(\gamma_0) \wedge \gamma_0
\end{aligned}
\end{equation}

FIXME: units here for \(\BP\) are off by a factor of \(c\).  This does not matter
so much in four vector form \(T(\gamma_0)\) where the units naturally take care
of themselves.

Okay, let us apply this to our field \eqnref{eqn:fourier_series_maxwell:bivectorSolNonGreens}, and try to percolate the \(\gamma_0\) through all the terms of \(\tilde{F}(\Bx,t)\)

\begin{equation}\label{eqn:fourierSeriesMaxwell:480}
\begin{aligned}
\gamma_0 \tilde{F}(\Bx,t) 
&= -\gamma_0 F(\Bx,t) \\
&= -\gamma_0 \inv{V} \int \sum_{\Bk} \exp\left( i \Bk ct \right) \exp\left( i \Bk \cdot (\Bx' -\Bx) \right) F(\Bx', 0) d^3 x' \\
\end{aligned}
\end{equation}

Taking one factor at a time 

\begin{equation}\label{eqn:fourierSeriesMaxwell:500}
\begin{aligned}
\gamma_0 \exp\left( i \Bk ct \right) 
&= \gamma_0 (\cos\left( \Bk ct \right) + i \sin\left( \Bk ct \right) ) \\
&= \cos\left( \Bk ct \right) \gamma_0 - i \gamma_0 \sin\left( \Bk ct \right) ) \\
&= \cos\left( \Bk ct \right) \gamma_0 - i \sin\left( \Bk ct \right) ) \gamma_0 \\
&= \exp\left( -i \Bk ct \right) \gamma_0
\end{aligned}
\end{equation}


Next, percolate \(\gamma_0\) through the pseudoscalar exponential.

\begin{equation}\label{eqn:fourierSeriesMaxwell:520}
\begin{aligned}
\gamma_0 e^{i\phi} 
&= \gamma_0 (\cos\phi + i \sin\phi) \\
&= \cos\phi \gamma_0 - i \gamma_0 \sin\phi \\
&= e^{-i\phi} \gamma_0
\end{aligned}
\end{equation}

Again, the percolation produces a conjugate effect.  Lastly, as noted previously \(F\) commutes with \(i\).  We have therefore

\begin{equation}\label{eqn:fourierSeriesMaxwell:540}
\begin{aligned}
\tilde{F}(\Bx,t) \gamma_0 {F}(\Bx,t) \gamma_0
&=
\frac{1}{V^2} \int \sum_{\Bk,\Bm} 
F(\Ba, 0) 
e^{i \Bk \cdot (\Ba -\Bx) }
e^{ i \Bk ct }
e^{ -i \Bm ct } e^{ -i \Bm \cdot (\Bb -\Bx) } F(\Bb, 0) d^3 a d^3 b \\
&= \frac{1}{V^2} \int \sum_{\Bk,\Bm} F(\Ba, 0) e^{ i \Bk \cdot \Ba -i \Bm \cdot \Bb + i (\Bk -\Bm) ct -i (\Bk - \Bm) \cdot \Bx } F(\Bb, 0) d^3 a d^3 b \\
&= \frac{1}{V^2} \int \sum_{\Bk} F(\Ba, 0) F(\Bb, 0) e^{ i \Bk \cdot (\Ba - \Bb) } d^3 a d^3 b \\
&\quad + \frac{1}{V^2} \int \sum_{\Bk \ne \Bm} F(\Ba, 0) e^{ i \Bk \cdot \Ba -i \Bm \cdot \Bb + i (\Bk -\Bm) ct -i (\Bk - \Bm) \cdot \Bx } F(\Bb, 0) d^3 a d^3 b \\
&= \frac{1}{V^2} \int \sum_{\Bk} F(\Ba, 0) F(\Bb, 0) e^{ i \Bk \cdot (\Ba - \Bb) } d^3 a d^3 b \\
&\quad + \frac{1}{V^2} \int \sum_{\Bm, \Bk \ne 0} F(\Ba, 0) e^{ 
i \Bm \cdot (\Ba -\Bb) 
+i \Bk \cdot (\Ba -\Bx)
+ i \Bk ct 
} F(\Bb, 0) d^3 a d^3 b \\
\end{aligned}
\end{equation}

Hmm.  Messy.  The scalar bits of the above are our energy.  We have a \(F^2\) like term in the first integral (like the Lagrangian density), but it is at different points, and
we have to integrate those with a sort of vector convolution.  Given the reciprocal relationships between convolution and multiplication moving between the frequency and time domains in Fourier transforms I had expect that this first integral can somehow be turned into the sum of the squares of all the Fourier coefficients

\begin{equation}\label{eqn:fourierSeriesMaxwell:560}
\begin{aligned}
\sum_{\Bk} C_{\Bk}^2 
\end{aligned}
\end{equation}

which is very much like a discrete version of the Rayleigh energy theorem as derived in \chapcite{PJqmFourier}, and is in this case
a constant (not a function of time or space) and is dependent on only the initial field.
That would mean that the remainder is the Poynting vector,
which looks reasonable since it has the appearance of being somewhat antisymmetric.

Hmm, having mostly figured it out without doing the math in this case, the answer pops out.  This first integral can be separated cleanly since the pseudoscalar
exponentials commute with the bivector field.  We then have

\begin{equation}\label{eqn:fourierSeriesMaxwell:580}
\begin{aligned}
\frac{1}{V^2} &\int \sum_{\Bk} F(\Ba, 0) F(\Bb, 0) e^{ i \Bk \cdot (\Ba - \Bb) } d^3 a d^3 b \\
&= \frac{1}{V} \int \sum_{\Bk} F(\Ba, 0) e^{ i \Bk \cdot \Ba } d^3 a \int F(\Bb, 0) e^{ -i \Bk \cdot \Bb } d^3 b \\
&= \sum_{\Bk} \hat{F}_{-\Bk} \hat{F}_{\Bk} \\
\end{aligned}
\end{equation}

A side note on subtle notational sneakiness here.  In the assumed series 
solution of \eqnref{eqn:fourier_series_maxwell:assumed} \(\hat{F}_{\Bk}(t)\) was the \(\Bk\) Fourier coefficient of \(F(\Bx,t)\), whereas here the use of \(\hat{F}_{\Bk}\) has been used to denote the \(\Bk\) Fourier coefficient of \(F(\Bx,0)\).
An alternative considered and rejected was something messier like \(\widehat{F(t=0)}_{\Bk}\), or the use of the original, less physically significant, \(C_{\Bk}\) coefficients.

The second term could also use a simplification, and it looks like we can separate these \(\Ba\) and \(\Bb\) integrals too

\begin{equation}\label{eqn:fourierSeriesMaxwell:600}
\begin{aligned}
\frac{1}{V^2} &\int \sum_{\Bm, \Bk \ne 0} F(\Ba, 0) e^{ 
i \Bm \cdot (\Ba -\Bb) 
+i \Bk \cdot (\Ba -\Bx)
+ i \Bk ct 
} F(\Bb, 0) d^3 a d^3 b \\
&=\frac{1}{V} \int \sum_{\Bm, \Bk \ne 0} F(\Ba, 0) e^{ i (\Bm + \Bk) \cdot \Ba } d^3 a
e^{ i \Bk ct -i \Bk \cdot \Bx }
\inv{V} \int F(\Bb, 0) 
e^{-i \Bm \cdot \Bb}
d^3 b
 \\
&= \sum_{\Bm} \sum_{\Bk \ne 0} \hat{F}_{-\Bm -\Bk} e^{ i \Bk ct -i \Bk \cdot \Bx } \hat{F}_{\Bm} \\
\end{aligned}
\end{equation}

Making an informed guess that the first integral is a scalar, and the second is a spatial vector, our energy and momentum densities (Poynting vector) respectively are

\begin{equation}\label{eqn:fourier_series_maxwell:energyMomentum}
\begin{aligned}
U & 
\questionEquals
 \inv{8 \pi} \sum_{\Bk} \hat{F}_{-\Bk} \hat{F}_{\Bk} \\
\BP &
\questionEquals
 \inv{8 \pi} \sum_{\Bm} \sum_{\Bk \ne 0} \hat{F}_{-\Bm -\Bk} e^{ i \Bk ct -i \Bk \cdot \Bx } \hat{F}_{\Bm}
\end{aligned}
\end{equation}

Now that much of the math is taken care of, more consideration about the physics implications is required.  In particular, relating these
abstract quantities to the frequencies and the harmonic oscillator model as Bohm did is desirable (that was the whole point of the exercise).

On the validity of \eqnref{eqn:fourier_series_maxwell:energyMomentum}, it is not unreasonable to expect that 
\(\PDi{t}{U} = 0\), and \(\spacegrad \cdot \BP = 0\) separately in these current free conditions from the energy momentum conservation relation

\begin{equation}\label{eqn:fourierSeriesMaxwell:620}
\begin{aligned}
\PD{t}{}\frac{1}{8\pi} \left(\bcE^2 + \bcH^2\right) + \inv{4\pi} \spacegrad \cdot (\bcE \cross \bcH) &= -\bcE \cdot \Bj 
\end{aligned}
\end{equation}

Note that an SI derivation of this relation can be found in \chapcite{PJpoynting}.  So it therefore makes some sense that all the time dependence ends
up in what has been labeled as the Poynting vector.  A proof that the spatial divergence of this quantity is zero would help validate
the guess made (or perhaps invalidate it).

Hmm.  Again on the validity of identifying the first sum with the energy.  It does not appear to work for the \(\Bk = 0\) case, since that gives you

\begin{equation}\label{eqn:fourierSeriesMaxwell:640}
\begin{aligned}
\inv{8 \pi V^2} \int F(\Ba, 0) F(\Bb, 0) d^3 a d^3b
\end{aligned}
\end{equation}

That is only a scalar if the somehow all the non-scalar parts of that product somehow magically cancel out.
Perhaps it is true that the second sum has no scalar part, and if that is the case one would have

\begin{equation}\label{eqn:fourierSeriesMaxwell:660}
\begin{aligned}
U
\questionEquals
 \inv{8 \pi} \sum_{\Bk} \gpgradezero{\hat{F}_{-\Bk} \hat{F}_{\Bk}} \\
\end{aligned}
\end{equation}

An explicit calculation of \(T(\gamma_0) \cdot \gamma_0\) is probably justified
to discarding all other grades, and get just the energy.

So, instead of optimistically hoping that the scalar and spatial vector terms will automatically fall out, it appears
that we have to explicitly calculate the dot and wedge products, as in

\begin{equation}\label{eqn:fourierSeriesMaxwell:680}
\begin{aligned}
U &= -\frac{1}{16\pi}( F \gamma_0 F \gamma_0 + \gamma_0 F \gamma_0 F ) \\
\BP &= -\frac{1}{16\pi}( F \gamma_0 F \gamma_0 - \gamma_0 F \gamma_0 F )
\end{aligned}
\end{equation}

and then substitute our Fourier series solution for \(F\) to get the desired result.  This appears to be getting more complex instead of
less so unfortunately, but hopefully following this to a logical conclusion will show in retrospect a faster way to the desired result.
A first attempt to do so shows that we have to return to our assumed Fourier solution and revisit some of the assumptions made.

\section{Return to the assumed solutions to Maxwell's equation}

An initial attempt to expand \eqnref{eqn:fourier_series_maxwell:energyMomentum} properly
given the Fourier specification of the Maxwell solution
gets into trouble.  Consideration of some special cases for specific values
of \(\Bk\) shows that there is a problem with the grades of the solution.

Let us reexamine the assumed solution of \eqnref{eqn:fourier_series_maxwell:bivectorSolNonGreens} with respect to grade
\begin{equation}\label{eqn:fourierSeriesMaxwell:700}
\begin{aligned}
F(\Bx,t) &= \inv{V} \int \sum_{\Bk} \exp\left( i \Bk ct \right) \exp\left( i \Bk \cdot (\Bx' - \Bx) \right) F(\Bx', 0) d^3 x'
\end{aligned}
\end{equation}

For scalar Fourier approximations we are used to the ability to select a subset of the Fourier terms to approximate the field, but
except for the \(\Bk = 0\) term it appears that a term by term approximation actually introduces noise in the form of non-bivector grades.

Consider first the \(\Bk = 0\) term.  This gives us a first order approximation of the field which is

\begin{equation}\label{eqn:fourierSeriesMaxwell:720}
\begin{aligned}
F(\Bx,t) &\approx \inv{V} \int F(\Bx', 0) d^3 x'
\end{aligned}
\end{equation}

As summation is grade preserving this spatial average of the initial field conditions does have the required grade as desired.
Next consider a non-zero Fourier term such as \(\Bk = \{1,0,0\}\).  For this single term approximation of the field let us write
out the field term as

\begin{equation}\label{eqn:fourierSeriesMaxwell:740}
\begin{aligned}
F_{\Bk}(\Bx,t)
&= \inv{V} \int e^{ i \kcap \Abs{\Bk} ct + i \Bk \cdot (\Bx' - \Bx) } (\bcE(\Bx', 0) + i\bcH(\Bx', 0)) d^3 x'
\end{aligned}
\end{equation}

Now, let us expand the exponential.  This was shorthand for the product of the exponentials, which seemed to be a reasonable
shorthand since we showed they commute.  Expanded out this is

\begin{equation}\label{eqn:fourierSeriesMaxwell:760}
\begin{aligned}
\exp&( i \kcap \Abs{\Bk} ct + i \Bk \cdot (\Bx' - \Bx) ) \\
&= (\cos( {\Bk} ct ) + i \kcap \sin( \Abs{\Bk} ct ))( \cos( \Bk \cdot (\Bx' - \Bx) ) + i \sin(\Bk \cdot (\Bx' - \Bx) )) \\
\end{aligned}
\end{equation}

For ease of manipulation write \(\Bk \cdot (\Bx' - \Bx) = k \Delta x\), and \(\Bk c t = \Bomega t\), we have

\begin{equation}\label{eqn:fourierSeriesMaxwell:780}
\begin{aligned}
%(\cos( \Bomega t ) + i \sin( \Bomega t ))( \cos( k \Delta x ) ) + i \sin( k \Delta x ) )) 
\exp( i \Bomega t + i k \Delta x )
&= \cos( \Bomega t ) \cos( k \Delta x ) +i \cos( \Bomega t ) \sin( k \Delta x ) \\
&+i \sin( \Bomega t ) \cos( k \Delta x ) - \sin( \Bomega t ) \sin( k \Delta x )  \\
\end{aligned}
\end{equation}

Note that \(\cos(\Bomega t)\) is a scalar, whereas \(\sin(\Bomega t)\) is a (spatial) vector in the direction of \(\Bk\).
Multiplying this out with the initial time field \(F(\Bx',0) = \bcE(\Bx', 0) + i\bcH(\Bx',0) = \bcE' + i\bcH'\) we can separate into grades.  

\begin{equation}\label{eqn:fourierSeriesMaxwell:800}
\begin{aligned}
\exp&( i \Bomega t + i k \Delta x ) (\bcE' + i\bcH') \\
&= \cos( \Bomega t ) (\bcE' \cos( k \Delta x ) -\bcH' \sin( k \Delta x ) ) +  \sin( \Bomega t ) \cross ( \bcH' \sin( k \Delta x ) - \bcE' \cos( k \Delta x ) ) \\
&+i \cos( \Bomega t ) (\bcE' \sin( k \Delta x ) + \bcH' \cos( k \Delta x ) ) -i \sin( \Bomega t ) \cross (\bcE' \sin( k \Delta x ) + \bcH' \cos( k \Delta x ) ) \\
&-  \sin( \Bomega t ) \cdot (\bcE' \sin( k \Delta x ) +\bcH' \cos( k \Delta x ) ) \\
&+i( \sin( \Bomega t ) \cdot (\bcE' \cos( k \Delta x ) - \bcH' \sin( k \Delta x )) \\
\end{aligned}
\end{equation}

The first two lines, once integrated, produce the electric and magnetic fields, but the last two are rogue scalar and pseudoscalar terms.  These
are allowed in so far as they are still solutions to the differential equation, but do not have the desired physical meaning.

If one explicitly sums over pairs of \(\{\Bk,-\Bk\}\) of index triplets then some cancellation occurs.  The cosine cosine products and sine sine products double
and the sine cosine terms cancel.  We therefore have

\begin{equation}\label{eqn:fourierSeriesMaxwell:820}
\begin{aligned}
\inv{2} &\exp( i \Bomega t + i k \Delta x ) (\bcE' + i\bcH') \\
&= \cos( \Bomega t ) \bcE' \cos( k \Delta x ) +  \sin( \Bomega t ) \cross \bcH' \sin( k \Delta x ) \\
&+i \cos( \Bomega t ) \bcH' \cos( k \Delta x ) -i \sin( \Bomega t ) \cross \bcE' \sin( k \Delta x ) \\
&-  \sin( \Bomega t ) \cdot \bcE' \sin( k \Delta x ) \\
&-i \sin( \Bomega t ) \cdot \bcH' \sin( k \Delta x ) \\
&= (\bcE' + i\bcH') \cos( \Bomega t ) \cos( k \Delta x ) 
 -i \sin( \Bomega t ) \cross (\bcE' + i \bcH') \sin( k \Delta x ) \\
&-  \sin( \Bomega t ) \cdot (\bcE'+i\bcH) \sin( k \Delta x ) \\
\end{aligned}
\end{equation}

Here for grouping purposes \(i\) is treated as a scalar, which should be justifiable in this specific case.  A final grouping produces

\begin{equation}\label{eqn:fourierSeriesMaxwell:840}
\begin{aligned}
\inv{2} \exp( i \Bomega t + i k \Delta x ) (\bcE' + i\bcH') 
&= (\bcE' + i\bcH') \cos( \Bomega t ) \cos( k \Delta x )  \\
&-i \kcap \cross (\bcE' + i \bcH') \sin( \Abs{\Bomega} t ) \sin( k \Delta x ) \\
&-  \sin( \Bomega t ) \cdot (\bcE'+i\bcH') \sin( k \Delta x ) \\
\end{aligned}
\end{equation}

Observe that despite the grouping of the summation over the pairs of complementary sign index triplets we still have a pure scalar and pure pseudoscalar
term above.  Allowable by the math since the differential equation had no way of encoding the grade of the desired solution.  That only came from the
initial time specification of \(F(\Bx',0)\), but that is not enough.

Now, from above, we can see that one way to reconcile this grade requirement is to require both \(\kcap \cdot \bcE' = 0\), and \(\kcap \cdot \bcH' = 0\).
How can such a requirement make sense given that \(\Bk\) ranges over all directions in space, and that both \(\bcE'\) and \(\bcH'\) could conceivably 
range over many different directions in the volume of periodicity.

With no other way out, it seems that we have to impose two requirements, one on the allowable wavenumber vector directions (which in turn means we can only
pick specific orientations of the Fourier volume), and another on the field directions themselves.  The electric and magnetic fields must therefore
be directed only perpendicular to the wave number vector direction.  Wow, that is a pretty severe implication following strictly from a grade requirement!

Thinking back to \eqnref{eqn:fourier_series_maxwell:anticommutes}, it appears that an implication of this is that we have

\begin{equation}\label{eqn:fourierSeriesMaxwell:860}
\begin{aligned}
e^{i\Bomega t} F(\Bx',0) &= F(\Bx',0) e^{-i\Bomega t}
\end{aligned}
\end{equation}

Knowing this is a required condition should considerably simplify the energy and momentum questions.

%\section{FIXME}
%
%Caught myself in these notes abusing notation and probably made mistakes by combining exponentials that probably do not commute into single argument exponentials.  Go back and review all other recent previous Fourier treatments and check for and fix this if neccessary
% ... TURNS OUT THEY DID COMMUTE .... should still review other work.

\documentclass{article}

\usepackage{amsmath}
\usepackage{mathpazo}

%
% shorthand for bold symbols, convenient for vectors and matrices
%
\newcommand{\Ba}[0]{\mathbf{a}}
\newcommand{\Bb}[0]{\mathbf{b}}
\newcommand{\Bc}[0]{\mathbf{c}}
\newcommand{\Bd}[0]{\mathbf{d}}
\newcommand{\Be}[0]{\mathbf{e}}
\newcommand{\Bf}[0]{\mathbf{f}}
\newcommand{\Bg}[0]{\mathbf{g}}
\newcommand{\Bh}[0]{\mathbf{h}}
\newcommand{\Bi}[0]{\mathbf{i}}
\newcommand{\Bj}[0]{\mathbf{j}}
\newcommand{\Bk}[0]{\mathbf{k}}
\newcommand{\Bl}[0]{\mathbf{l}}
\newcommand{\Bm}[0]{\mathbf{m}}
\newcommand{\Bn}[0]{\mathbf{n}}
\newcommand{\Bo}[0]{\mathbf{o}}
\newcommand{\Bp}[0]{\mathbf{p}}
\newcommand{\Bq}[0]{\mathbf{q}}
\newcommand{\Br}[0]{\mathbf{r}}
\newcommand{\Bs}[0]{\mathbf{s}}
\newcommand{\Bt}[0]{\mathbf{t}}
\newcommand{\Bu}[0]{\mathbf{u}}
\newcommand{\Bv}[0]{\mathbf{v}}
\newcommand{\Bw}[0]{\mathbf{w}}
\newcommand{\Bx}[0]{\mathbf{x}}
\newcommand{\By}[0]{\mathbf{y}}
\newcommand{\Bz}[0]{\mathbf{z}}
\newcommand{\BA}[0]{\mathbf{A}}
\newcommand{\BB}[0]{\mathbf{B}}
\newcommand{\BC}[0]{\mathbf{C}}
\newcommand{\BD}[0]{\mathbf{D}}
\newcommand{\BE}[0]{\mathbf{E}}
\newcommand{\BF}[0]{\mathbf{F}}
\newcommand{\BG}[0]{\mathbf{G}}
\newcommand{\BH}[0]{\mathbf{H}}
\newcommand{\BI}[0]{\mathbf{I}}
\newcommand{\BJ}[0]{\mathbf{J}}
\newcommand{\BK}[0]{\mathbf{K}}
\newcommand{\BL}[0]{\mathbf{L}}
\newcommand{\BM}[0]{\mathbf{M}}
\newcommand{\BN}[0]{\mathbf{N}}
\newcommand{\BO}[0]{\mathbf{O}}
\newcommand{\BP}[0]{\mathbf{P}}
\newcommand{\BQ}[0]{\mathbf{Q}}
\newcommand{\BR}[0]{\mathbf{R}}
\newcommand{\BS}[0]{\mathbf{S}}
\newcommand{\BT}[0]{\mathbf{T}}
\newcommand{\BU}[0]{\mathbf{U}}
\newcommand{\BV}[0]{\mathbf{V}}
\newcommand{\BW}[0]{\mathbf{W}}
\newcommand{\BX}[0]{\mathbf{X}}
\newcommand{\BY}[0]{\mathbf{Y}}
\newcommand{\BZ}[0]{\mathbf{Z}}

\newcommand{\Bzero}[0]{\mathbf{0}}
\newcommand{\Btheta}[0]{\boldsymbol{\theta}}
\newcommand{\Btau}[0]{\boldsymbol{\tau}}
\newcommand{\Bomega}[0]{\boldsymbol{\omega}}

%
% shorthand for unit vectors
%
\newcommand{\acap}[0]{\hat{\Ba}}
\newcommand{\bcap}[0]{\hat{\Bb}}
\newcommand{\ccap}[0]{\hat{\Bc}}
\newcommand{\dcap}[0]{\hat{\Bd}}
\newcommand{\ecap}[0]{\hat{\Be}}
\newcommand{\fcap}[0]{\hat{\Bf}}
\newcommand{\gcap}[0]{\hat{\Bg}}
\newcommand{\hcap}[0]{\hat{\Bh}}
\newcommand{\icap}[0]{\hat{\Bi}}
\newcommand{\jcap}[0]{\hat{\Bj}}
\newcommand{\kcap}[0]{\hat{\Bk}}
\newcommand{\lcap}[0]{\hat{\Bl}}
\newcommand{\mcap}[0]{\hat{\Bm}}
\newcommand{\ncap}[0]{\hat{\Bn}}
\newcommand{\ocap}[0]{\hat{\Bo}}
\newcommand{\pcap}[0]{\hat{\Bp}}
\newcommand{\qcap}[0]{\hat{\Bq}}
\newcommand{\rcap}[0]{\hat{\Br}}
\newcommand{\scap}[0]{\hat{\Bs}}
\newcommand{\tcap}[0]{\hat{\Bt}}
\newcommand{\ucap}[0]{\hat{\Bu}}
\newcommand{\vcap}[0]{\hat{\Bv}}
\newcommand{\wcap}[0]{\hat{\Bw}}
\newcommand{\xcap}[0]{\hat{\Bx}}
\newcommand{\ycap}[0]{\hat{\By}}
\newcommand{\zcap}[0]{\hat{\Bz}}
\newcommand{\thetacap}[0]{\hat{\Btheta}}

%
% to write R^n and C^n in a distinguishable fashion.  Perhaps change this
% to the double lined characters upon figuring out how to do so.
%
\newcommand{\C}[1]{$\mathbb{C}^{#1}$}
\newcommand{\R}[1]{$\mathbb{R}^{#1}$}

%
% various generally useful helpers
%

% derivative of #1 wrt. #2:
\newcommand{\D}[2] {\frac {d#2} {d#1}}

\newcommand{\inv}[1]{\frac{1}{#1}}
\newcommand{\cross}[0]{\times}

\newcommand{\abs}[1]{\lvert{#1}\rvert}
\newcommand{\norm}[1]{\lVert{#1}\rVert}
\newcommand{\innerprod}[2]{\langle{#1}, {#2}\rangle}
\newcommand{\dotprod}[2]{{#1} \cdot {#2}}
\newcommand{\bdotprod}[2]{\left({#1} \cdot {#2}\right)}
\newcommand{\crossprod}[2]{{#1} \cross {#2}}
\newcommand{\tripleprod}[3]{\dotprod{\left(\crossprod{#1}{#2}\right)}{#3}}

\DeclareMathOperator{\Proj}{Proj}
\DeclareMathOperator{\Span}{span}
\DeclareMathOperator{\Sgn}{sgn}
\DeclareMathOperator{\Area}{Area}
\DeclareMathOperator{\Volume}{Volume}

%
% A few miscellaneous things specific to this document
%
\newcommand{\crossop}[1]{\crossprod{#1}{}}

% R2 vector.
\newcommand{\VectorTwo}[2]{
\begin{bmatrix}
 {#1} \\
 {#2}
\end{bmatrix}
}

\newcommand{\VectorN}[1]{
\begin{bmatrix}
{#1}_1 \\
{#1}_2 \\
\vdots \\
{#1}_N \\
\end{bmatrix}
}

\newcommand{\DETuvij}[4]{
\begin{vmatrix}
 {#1}_{#3} & {#1}_{#4} \\
 {#2}_{#3} & {#2}_{#4}
\end{vmatrix}
}

\newcommand{\DETuvwijk}[6]{
\begin{vmatrix}
 {#1}_{#4} & {#1}_{#5} & {#1}_{#6} \\
 {#2}_{#4} & {#2}_{#5} & {#2}_{#6} \\
 {#3}_{#4} & {#3}_{#5} & {#3}_{#6}
\end{vmatrix}
}

\newcommand{\DETuvwxijkl}[8]{
\begin{vmatrix}
 {#1}_{#5} & {#1}_{#6} & {#1}_{#7} & {#1}_{#8} \\
 {#2}_{#5} & {#2}_{#6} & {#2}_{#7} & {#2}_{#8} \\
 {#3}_{#5} & {#3}_{#6} & {#3}_{#7} & {#3}_{#8} \\
 {#4}_{#5} & {#4}_{#6} & {#4}_{#7} & {#4}_{#8} \\
\end{vmatrix}
}

%\newcommand{\DETuvwxyijklm}[10]{
%\begin{vmatrix}
% {#1}_{#6} & {#1}_{#7} & {#1}_{#8} & {#1}_{#9} & {#1}_{#10} \\
% {#2}_{#6} & {#2}_{#7} & {#2}_{#8} & {#2}_{#9} & {#2}_{#10} \\
% {#3}_{#6} & {#3}_{#7} & {#3}_{#8} & {#3}_{#9} & {#3}_{#10} \\
% {#4}_{#6} & {#4}_{#7} & {#4}_{#8} & {#4}_{#9} & {#4}_{#10} \\
% {#5}_{#6} & {#5}_{#7} & {#5}_{#8} & {#5}_{#9} & {#5}_{#10}
%\end{vmatrix}
%}

% R3 vector.
\newcommand{\VectorThree}[3]{
\begin{bmatrix}
 {#1} \\
 {#2} \\
 {#3}
\end{bmatrix}
}


%<misc>
%
\newcommand{\Abs}[1]{{\left\lvert{#1}\right\rvert}}
\newcommand{\spacegrad}[0]{\boldsymbol{\nabla}}
\newcommand{\grad}[0]{\nabla}
\newcommand{\LL}[0]{\mathcal{L}}

% == \partial_{#1} {#2}
\newcommand{\PD}[2]{\frac{\partial {#2}}{\partial {#1}}}
% inline variant
\newcommand{\PDi}[2]{{\partial {#2}}/{\partial {#1}}}

\newcommand{\PDD}[3]{\frac{\partial^2 {#3}}{\partial {#1}\partial {#2}}}
%\newcommand{\PDd}[2]{\frac{\partial^2 {#2}}{{\partial{#1}}^2}}
\newcommand{\PDsq}[2]{\frac{\partial^2 {#2}}{(\partial {#1})^2}}

\newcommand{\Partial}[2]{\frac{\partial {#1}}{\partial {#2}}}
\DeclareMathOperator{\RejName}{Rej}
\newcommand{\Rej}[2]{\RejName_{#1}\left( {#2} \right)}
\newcommand{\Rm}[1]{\mathbb{R}^{#1}}
\newcommand{\Cm}[1]{\mathbb{C}^{#1}}
\newcommand{\conj}[0]{{*}}

%</misc>

% <grade selection>
%
\newcommand{\gpgrade}[2] {{\left\langle{{#1}}\right\rangle}_{#2}}

\newcommand{\gpgradezero}[1] {\gpgrade{#1}{}}
%\newcommand{\gpscalargrade}[1] {{\left\langle{{#1}}\right\rangle}}
%\newcommand{\gpgradezero}[1] {\gpgrade{#1}{0}}

%\newcommand{\gpgradeone}[1] {{\left\langle{{#1}}\right\rangle}_{1}}
\newcommand{\gpgradeone}[1] {\gpgrade{#1}{1}}

\newcommand{\gpgradetwo}[1] {\gpgrade{#1}{2}}
\newcommand{\gpgradethree}[1] {\gpgrade{#1}{3}}
\newcommand{\gpgradefour}[1] {\gpgrade{#1}{4}}
%
% </grade selection>



\newcommand{\adot}[0]{{\dot{a}}}
\newcommand{\bdot}[0]{{\dot{b}}}
% taken for centered dot:
%\newcommand{\cdot}[0]{{\dot{c}}}
%\newcommand{\ddot}[0]{{\dot{d}}}
\newcommand{\edot}[0]{{\dot{e}}}
\newcommand{\fdot}[0]{{\dot{f}}}
\newcommand{\gdot}[0]{{\dot{g}}}
\newcommand{\hdot}[0]{{\dot{h}}}
\newcommand{\idot}[0]{{\dot{i}}}
\newcommand{\jdot}[0]{{\dot{j}}}
\newcommand{\kdot}[0]{{\dot{k}}}
\newcommand{\ldot}[0]{{\dot{l}}}
\newcommand{\mdot}[0]{{\dot{m}}}
\newcommand{\ndot}[0]{{\dot{n}}}
%\newcommand{\odot}[0]{{\dot{o}}}
\newcommand{\pdot}[0]{{\dot{p}}}
\newcommand{\qdot}[0]{{\dot{q}}}
\newcommand{\rdot}[0]{{\dot{r}}}
\newcommand{\sdot}[0]{{\dot{s}}}
\newcommand{\tdot}[0]{{\dot{t}}}
\newcommand{\udot}[0]{{\dot{u}}}
\newcommand{\vdot}[0]{{\dot{v}}}
\newcommand{\wdot}[0]{{\dot{w}}}
\newcommand{\xdot}[0]{{\dot{x}}}
\newcommand{\ydot}[0]{{\dot{y}}}
\newcommand{\zdot}[0]{{\dot{z}}}
\newcommand{\addot}[0]{{\ddot{a}}}
\newcommand{\bddot}[0]{{\ddot{b}}}
\newcommand{\cddot}[0]{{\ddot{c}}}
%\newcommand{\dddot}[0]{{\ddot{d}}}
\newcommand{\eddot}[0]{{\ddot{e}}}
\newcommand{\fddot}[0]{{\ddot{f}}}
\newcommand{\gddot}[0]{{\ddot{g}}}
\newcommand{\hddot}[0]{{\ddot{h}}}
\newcommand{\iddot}[0]{{\ddot{i}}}
\newcommand{\jddot}[0]{{\ddot{j}}}
\newcommand{\kddot}[0]{{\ddot{k}}}
\newcommand{\lddot}[0]{{\ddot{l}}}
\newcommand{\mddot}[0]{{\ddot{m}}}
\newcommand{\nddot}[0]{{\ddot{n}}}
\newcommand{\oddot}[0]{{\ddot{o}}}
\newcommand{\pddot}[0]{{\ddot{p}}}
\newcommand{\qddot}[0]{{\ddot{q}}}
\newcommand{\rddot}[0]{{\ddot{r}}}
\newcommand{\sddot}[0]{{\ddot{s}}}
\newcommand{\tddot}[0]{{\ddot{t}}}
\newcommand{\uddot}[0]{{\ddot{u}}}
\newcommand{\vddot}[0]{{\ddot{v}}}
\newcommand{\wddot}[0]{{\ddot{w}}}
\newcommand{\xddot}[0]{{\ddot{x}}}
\newcommand{\yddot}[0]{{\ddot{y}}}
\newcommand{\zddot}[0]{{\ddot{z}}}

%<bold and dot greek symbols>
%

\newcommand{\Deltadot}[0]{{\dot{\Delta}}}
\newcommand{\Gammadot}[0]{{\dot{\Gamma}}}
\newcommand{\Lambdadot}[0]{{\dot{\Lambda}}}
\newcommand{\Omegadot}[0]{{\dot{\Omega}}}
\newcommand{\Phidot}[0]{{\dot{\Phi}}}
\newcommand{\Pidot}[0]{{\dot{\Pi}}}
\newcommand{\Psidot}[0]{{\dot{\Psi}}}
\newcommand{\Sigmadot}[0]{{\dot{\Sigma}}}
\newcommand{\Thetadot}[0]{{\dot{\Theta}}}
\newcommand{\Upsilondot}[0]{{\dot{\Upsilon}}}
\newcommand{\Xidot}[0]{{\dot{\Xi}}}
\newcommand{\alphadot}[0]{{\dot{\alpha}}}
\newcommand{\betadot}[0]{{\dot{\beta}}}
\newcommand{\chidot}[0]{{\dot{\chi}}}
\newcommand{\deltadot}[0]{{\dot{\delta}}}
\newcommand{\epsilondot}[0]{{\dot{\epsilon}}}
\newcommand{\etadot}[0]{{\dot{\eta}}}
\newcommand{\gammadot}[0]{{\dot{\gamma}}}
\newcommand{\kappadot}[0]{{\dot{\kappa}}}
\newcommand{\lambdadot}[0]{{\dot{\lambda}}}
\newcommand{\mudot}[0]{{\dot{\mu}}}
\newcommand{\nudot}[0]{{\dot{\nu}}}
\newcommand{\omegadot}[0]{{\dot{\omega}}}
\newcommand{\phidot}[0]{{\dot{\phi}}}
\newcommand{\pidot}[0]{{\dot{\pi}}}
\newcommand{\psidot}[0]{{\dot{\psi}}}
\newcommand{\rhodot}[0]{{\dot{\rho}}}
\newcommand{\sigmadot}[0]{{\dot{\sigma}}}
\newcommand{\taudot}[0]{{\dot{\tau}}}
\newcommand{\thetadot}[0]{{\dot{\theta}}}
\newcommand{\upsilondot}[0]{{\dot{\upsilon}}}
\newcommand{\varepsilondot}[0]{{\dot{\varepsilon}}}
\newcommand{\varphidot}[0]{{\dot{\varphi}}}
\newcommand{\varpidot}[0]{{\dot{\varpi}}}
\newcommand{\varrhodot}[0]{{\dot{\varrho}}}
\newcommand{\varsigmadot}[0]{{\dot{\varsigma}}}
\newcommand{\varthetadot}[0]{{\dot{\vartheta}}}
\newcommand{\xidot}[0]{{\dot{\xi}}}
\newcommand{\zetadot}[0]{{\dot{\zeta}}}

\newcommand{\Deltaddot}[0]{{\ddot{\Delta}}}
\newcommand{\Gammaddot}[0]{{\ddot{\Gamma}}}
\newcommand{\Lambdaddot}[0]{{\ddot{\Lambda}}}
\newcommand{\Omegaddot}[0]{{\ddot{\Omega}}}
\newcommand{\Phiddot}[0]{{\ddot{\Phi}}}
\newcommand{\Piddot}[0]{{\ddot{\Pi}}}
\newcommand{\Psiddot}[0]{{\ddot{\Psi}}}
\newcommand{\Sigmaddot}[0]{{\ddot{\Sigma}}}
\newcommand{\Thetaddot}[0]{{\ddot{\Theta}}}
\newcommand{\Upsilonddot}[0]{{\ddot{\Upsilon}}}
\newcommand{\Xiddot}[0]{{\ddot{\Xi}}}
\newcommand{\alphaddot}[0]{{\ddot{\alpha}}}
\newcommand{\betaddot}[0]{{\ddot{\beta}}}
\newcommand{\chiddot}[0]{{\ddot{\chi}}}
\newcommand{\deltaddot}[0]{{\ddot{\delta}}}
\newcommand{\epsilonddot}[0]{{\ddot{\epsilon}}}
\newcommand{\etaddot}[0]{{\ddot{\eta}}}
\newcommand{\gammaddot}[0]{{\ddot{\gamma}}}
\newcommand{\kappaddot}[0]{{\ddot{\kappa}}}
\newcommand{\lambdaddot}[0]{{\ddot{\lambda}}}
\newcommand{\muddot}[0]{{\ddot{\mu}}}
\newcommand{\nuddot}[0]{{\ddot{\nu}}}
\newcommand{\omegaddot}[0]{{\ddot{\omega}}}
\newcommand{\phiddot}[0]{{\ddot{\phi}}}
\newcommand{\piddot}[0]{{\ddot{\pi}}}
\newcommand{\psiddot}[0]{{\ddot{\psi}}}
\newcommand{\rhoddot}[0]{{\ddot{\rho}}}
\newcommand{\sigmaddot}[0]{{\ddot{\sigma}}}
\newcommand{\tauddot}[0]{{\ddot{\tau}}}
\newcommand{\thetaddot}[0]{{\ddot{\theta}}}
\newcommand{\upsilonddot}[0]{{\ddot{\upsilon}}}
\newcommand{\varepsilonddot}[0]{{\ddot{\varepsilon}}}
\newcommand{\varphiddot}[0]{{\ddot{\varphi}}}
\newcommand{\varpiddot}[0]{{\ddot{\varpi}}}
\newcommand{\varrhoddot}[0]{{\ddot{\varrho}}}
\newcommand{\varsigmaddot}[0]{{\ddot{\varsigma}}}
\newcommand{\varthetaddot}[0]{{\ddot{\vartheta}}}
\newcommand{\xiddot}[0]{{\ddot{\xi}}}
\newcommand{\zetaddot}[0]{{\ddot{\zeta}}}

\newcommand{\BDelta}[0]{\boldsymbol{\Delta}}
\newcommand{\BGamma}[0]{\boldsymbol{\Gamma}}
\newcommand{\BLambda}[0]{\boldsymbol{\Lambda}}
\newcommand{\BOmega}[0]{\boldsymbol{\Omega}}
\newcommand{\BPhi}[0]{\boldsymbol{\Phi}}
\newcommand{\BPi}[0]{\boldsymbol{\Pi}}
\newcommand{\BPsi}[0]{\boldsymbol{\Psi}}
\newcommand{\BSigma}[0]{\boldsymbol{\Sigma}}
\newcommand{\BTheta}[0]{\boldsymbol{\Theta}}
\newcommand{\BUpsilon}[0]{\boldsymbol{\Upsilon}}
\newcommand{\BXi}[0]{\boldsymbol{\Xi}}
\newcommand{\Balpha}[0]{\boldsymbol{\alpha}}
\newcommand{\Bbeta}[0]{\boldsymbol{\beta}}
\newcommand{\Bchi}[0]{\boldsymbol{\chi}}
\newcommand{\Bdelta}[0]{\boldsymbol{\delta}}
\newcommand{\Bepsilon}[0]{\boldsymbol{\epsilon}}
\newcommand{\Beta}[0]{\boldsymbol{\eta}}
\newcommand{\Bgamma}[0]{\boldsymbol{\gamma}}
\newcommand{\Bkappa}[0]{\boldsymbol{\kappa}}
\newcommand{\Blambda}[0]{\boldsymbol{\lambda}}
\newcommand{\Bmu}[0]{\boldsymbol{\mu}}
\newcommand{\Bnu}[0]{\boldsymbol{\nu}}
%\newcommand{\Bomega}[0]{\boldsymbol{\omega}}
\newcommand{\Bphi}[0]{\boldsymbol{\phi}}
\newcommand{\Bpi}[0]{\boldsymbol{\pi}}
\newcommand{\Bpsi}[0]{\boldsymbol{\psi}}
\newcommand{\Brho}[0]{\boldsymbol{\rho}}
\newcommand{\Bsigma}[0]{\boldsymbol{\sigma}}
%\newcommand{\Btau}[0]{\boldsymbol{\tau}}
%\newcommand{\Btheta}[0]{\boldsymbol{\theta}}
\newcommand{\Bupsilon}[0]{\boldsymbol{\upsilon}}
\newcommand{\Bvarepsilon}[0]{\boldsymbol{\varepsilon}}
\newcommand{\Bvarphi}[0]{\boldsymbol{\varphi}}
\newcommand{\Bvarpi}[0]{\boldsymbol{\varpi}}
\newcommand{\Bvarrho}[0]{\boldsymbol{\varrho}}
\newcommand{\Bvarsigma}[0]{\boldsymbol{\varsigma}}
\newcommand{\Bvartheta}[0]{\boldsymbol{\vartheta}}
\newcommand{\Bxi}[0]{\boldsymbol{\xi}}
\newcommand{\Bzeta}[0]{\boldsymbol{\zeta}}
%
%</bold and dot greek symbols>
%<infrequent>
%
%\newcommand{\AreaOp}[1]{\AName_{#1}}
%\newcommand{\Babs}[0]{\abs{\BB}}
%\newcommand{\Bcap}[0]{\hat{\BB}}
%\newcommand{\BrPrimeRej}[0]{\rcap(\rcap \wedge \Br')}
%\newcommand{\CA}[0]{\mathcal{A}}
%\newcommand{\Cos}[1]{\cos{\left({#1}\right)}}
%\newcommand{\Det}[1] {\abs{#1}}
%\newcommand{\Dsq}[2] {\frac {\partial^2 {#1}} {\partial {#2}^2}}
%\newcommand{\Exp}[1]{\exp{\left({#1}\right)}}
%\newcommand{\Norm}[1]{\left\lVert{#1}\right\rVert}
%\newcommand{\Sin}[1]{\sin{\left({#1}\right)}}
%\newcommand{\T}[0]{\text{T}}
%\newcommand{\VolumeOp}[1]{\VName_{#1}}
%\newcommand{\agrad}[0]{\Ba \cdot \nabla}
%\newcommand{\alphacap}[0]{\hat{\boldsymbol{\alpha}}}
%\newcommand{\Fcap}[0]{\hat{\BF}}
%\newcommand{\bithree}[0]{{\Bi}_3}
%\newcommand{\bxa}[0]{\Bx\Ba}
%\newcommand{\coordvec}[2]{
%\newcommand{\costheta}[0]{\acap \cdot \xcap}
%\newcommand{\ddt}[1]{\ddot{#1}}
%\newcommand{\ddu}[1] {\frac {d{#1}} {du}}
%\newcommand{\dsqxj}[2] {\frac {\partial^2 {#1}} {\partial {x_{#2}}^2}}
%\newcommand{\dtheta}[1]{\frac{d {#1}}{d \theta}}
%\newcommand{\dt}[1]{\dot{#1}}
%\newcommand{\dt}[1]{\frac{d {#1}}{dt}}
%\newcommand{\dxj}[2] {\frac {\partial {#1}} {\partial {x_{#2}}}}
%\newcommand{\halfPhi}[0]{\frac{\phi}{2}}
%\newcommand{\half}[0]{\inv{2}}
%\newcommand{\inv}[1]{\frac{1}{#1}}
%\newcommand{\laplacian}[0]{\nabla^2}
%\newcommand{\matrixoftx}[3]{
%\newcommand{\nrrp}[0]{\norm{\rcap \wedge \Br'}}
%\newcommand{\oiint}{\bigcirc \hspace{-1.4em} \int \hspace{-.8em} \int}
%\newcommand{\transpose}[1]{{#1}^{\text{T}}}
%\newcommand{\transpose}[1]{{{#1}^{\TextTranspose}}}
%\newcommand{\transpose}[1]{{{#1}^{\text{T}}}}
%\newcommand{\barA}[0]{\bar{A}}
%\newcommand{\qbar}[0]{\bar{q}}
%\newcommand{\qdotbar}[0]{\dot{\bar{q}}}
%
%</infrequent>





\usepackage[bookmarks=true]{hyperref}

\usepackage{color,cite,graphicx}
   % use colour in the document, put your citations as [1-4]
   % rather than [1,2,3,4] (it looks nicer, and the extended LaTeX2e
   % graphics package. 
\usepackage{latexsym,amssymb,epsf} % don't remember if these are
   % needed, but their inclusion can't do any damage


\title{ Plane wave Fourier series solutions to the Maxwell vacuum equation. }
\author{Peeter Joot}
\date{ Feb 08, 2009.  Last Revision: $Date: 2009/02/09 02:47:16 $ }

\begin{document}
\maketitle{}
\tableofcontents

\section{ Motivation. }

In \cite{PJFourierVacuum} an exploration of spatially periodic solutions to the electrodynamic vacuum equation was performed using a multivector formulation 
of a 3D Fourier series.
Here a summary of the results obtained will be presented in a more
coherent fashion, followed by an attempt to build on them.
In particular a complete
description of the field energy and momentum is desired.

A conclusion from the first analysis was that the
orientation of both the electric and magnetic field components
must be perpendicular to the angular velocity and wave number vectors 
within the entire spatial volume.  This was a requirement for the field
solutions to retain a bivector grade (STA/Dirac basis).

Here a specific orientation of the Fourier volume so that two of the axis
lie in the direction of the initial time electric and magnetic fields will be
used.  This is expected to simplify the treatment.

Also note that having obtained some results in a first attempt hindsight
now allows a few choices of variables that will be seen to be appropriate.
The natural motivation for any such choices can be found in the initial
treatment.

\subsection{ Notation. }

Conventions, definitions, and notation used here will largely follow
\cite{PJFourierVacuum}.  Also of possible aid in that document is a 
a table of symbols and their definitions.

\section{ A consise review of results. }

A Fourier series and the Fourier coefficients are

\begin{align}
f(\Bx) &= \sum_{\Bk} \hat{f}_{\Bk} e^{ - i \Bk \cdot \Bx } \\
\hat{f}_{\Bk} &= \inv{V} \int f(\Bx) e^{ i \Bk \cdot \Bx } d^3 x
\end{align}

In the vector context $\Bk$ is

\begin{align}
\Bk = 2 \pi \sum_m \sigma^m \frac{k_m}{\lambda_m}
\end{align}

Where $\lambda_m$ are the dimensions of the volume of integration, 
$V = \lambda_1 \lambda_2 \lambda_3$ is the volume, and
in an index context $\Bk = \{k_1, k_2, k_3\}$ is a triplet of integers,
positive, negative or zero.

\subsection{ Fourier series and coefficients. }

We want to find (STA) bivector solutions $F$ to the vacuum Maxwell equation

\begin{align}
\grad F = \gamma_0 (\partial_0 + \spacegrad) F = 0
\end{align}

We start by assuming a Fourier series solution of the form

\begin{align}
F(\Bx,t) &= \sum_{\Bk} \hat{F}_{\Bk} e^{-i \Bk \cdot \Bx} 
\end{align}



\bibliographystyle{plainnat}
\bibliography{myrefs}

\end{document}

%
% Copyright � 2012 Peeter Joot.  All Rights Reserved.
% Licenced as described in the file LICENSE under the root directory of this GIT repository.
%

%
%
\chapter{Lorentz Gauge Fourier Vacuum potential solutions}
\index{Lorentz gauge}
\index{Fourier transform!vacuum potential}
\label{chap:potentialFourier}
%\date{Feb 07, 2009.  potentialFourier.tex}

\section{Motivation}

In \chapcite{PJFourierVacuum} a first order Fourier solution of the Vacuum
Maxwell equation was performed.  Here a comparative potential solution
is obtained.

\subsection{Notation}

The 3D Fourier series notation developed for this treatment can be found
in the original notes \chapcite{PJFourierVacuum}.  Also included there is a
table of notation, much of which is also used here.

\section{Second order treatment with potentials}

\subsection{With the Lorentz gauge}

Now, it appears that Bohm's use of potentials allows a nice comparison with the harmonic oscillator.  Let us also try a Fourier solution of the
potential equations.  Again, use STA instead of the traditional vector equations, writing \(A = (\phi + \Ba)\gamma_0\), and employing the Lorentz gauge
\(\grad \cdot A = 0\) we have for \(F = \grad \wedge A\) in CGS units

FIXME: Add \(\Ba\), and \(\psi\) to notational table below with definitions in terms of \(\bcE\), and \(\bcH\) (or the other way around).

\begin{equation}\label{eqn:potentialFourier:20}
\begin{aligned}
\grad^2 A = 4 \pi J
\end{aligned}
\end{equation}

Again with a spacetime split of the gradient

\begin{equation}\label{eqn:potentialFourier:40}
\begin{aligned}
\grad = \gamma^0(\partial_0 + \spacegrad) = (\partial_0 - \spacegrad) \gamma_0
\end{aligned}
\end{equation}

our four Laplacian can be written

\begin{equation}\label{eqn:potentialFourier:60}
\begin{aligned}
(\partial_0 - \spacegrad) \gamma_0 \gamma^0(\partial_0 + \spacegrad)
&= (\partial_0 - \spacegrad) (\partial_0 + \spacegrad) \\
&= \partial_{00} - \spacegrad^2
\end{aligned}
\end{equation}

Our vacuum field equation for the potential is thus
\begin{equation}\label{eqn:potentialFourier:80}
\begin{aligned}
\partial_{tt} A = c^2 \spacegrad^2 A
\end{aligned}
\end{equation}

Now, as before assume a Fourier solution and see what follows.  That is

\begin{equation}\label{eqn:potential_fourier:assumedPotential}
\begin{aligned}
A(\Bx, t) &= \sum_{\Bk} \hat{A}_{\Bk}(t) e^{ -i \Bk \cdot \Bx}
\end{aligned}
\end{equation}

Applied to each component this gives us
\begin{equation}\label{eqn:potentialFourier:100}
\begin{aligned}
\hat{A}_{\Bk}'' e^{ -i \Bk \cdot \Bx}
&= c^2 \hat{A}_{\Bk}(t) \sum_m \PDsq{x^m}{} e^{ - 2 \pi i \sum_j k_j x^j /\lambda_j} \\
&= c^2 \hat{A}_{\Bk}(t) \sum_m (- 2 \pi i k_m/\lambda_m)^2 e^{ - i \Bk \cdot \Bx } \\
&= -c^2 \Bk^2 \hat{A}_{\Bk} e^{ - i \Bk \cdot \Bx }
\end{aligned}
\end{equation}

So we are left with another big ass set of simplest equations to solve

\begin{equation}\label{eqn:potentialFourier:120}
\begin{aligned}
\hat{A}_{\Bk}'' &= -c^2 \Bk^2 \hat{A}_{\Bk}
\end{aligned}
\end{equation}

Note that again the origin point \(\Bk = (0,0,0)\) is a special case.  Also of note this time is that \(\hat{A}_{\Bk}\) has vector and trivector parts, unlike \(\hat{F}_{\Bk}\) which being derived from dual and non-dual components of a bivector was still a bivector.

It appears that solutions can be found with either left or right handed
vector valued integration constants

\begin{equation}\label{eqn:potentialFourier:140}
\begin{aligned}
\hat{A}_{\Bk}(t) &= \exp(\pm i c \Bk t) C_{\Bk} \\
                 &= D_{\Bk} \exp(\pm i c \Bk t)
\end{aligned}
\end{equation}

Since these are equal at \(t=0\), it appears to imply that these commute with the
complex exponentials as was the case for the bivector field.

For the \(\Bk = 0\) special case we have solutions
\begin{equation}\label{eqn:potentialFourier:160}
\begin{aligned}
\hat{A}_{\Bk}(t) &= D_0 t + C_0
\end{aligned}
\end{equation}

It does not seem unreasonable to require \(D_0 = 0\).  Otherwise this time dependent DC Fourier component will blow up at large and small values, while periodic
solutions are sought.

Putting things back together we have %either of

\begin{equation}\label{eqn:potentialFourier:180}
\begin{aligned}
A(\Bx, t) &= \sum_{\Bk} \exp(\pm i c \Bk t) C_{\Bk} \exp( -i \Bk \cdot \Bx ) \\
%          &= \sum_{\Bk} C_{\Bk} \exp(\pm i c \Bk t) \exp( -i \Bk \cdot \Bx )
\end{aligned}
\end{equation}

Here again for \(t=0\), our integration constants are found to be determined completely by the initial conditions

\begin{equation}\label{eqn:potentialFourier:200}
\begin{aligned}
A(\Bx, 0) &= \sum_{\Bk} C_{\Bk} e^{ -i \Bk \cdot \Bx}
\end{aligned}
\end{equation}

So we can write

\begin{equation}\label{eqn:potentialFourier:220}
\begin{aligned}
C_{\Bk} = \inv{V} \int A(\Bx', 0) e^{ i \Bk \cdot \Bx'} d^3 x'
\end{aligned}
\end{equation}

In integral form this is

\begin{equation}\label{eqn:potential_fourier:potentialSolution}
\begin{aligned}
A(\Bx, t) &= \int \sum_{\Bk} \exp(\pm i \Bk c t ) A(\Bx', 0) \exp( i \Bk \cdot (\Bx -\Bx') )
\end{aligned}
\end{equation}

This, somewhat surprisingly, is strikingly similar to what we had for the bivector field.  That was:

\begin{equation}\label{eqn:potential_fourier:bivectorSolution}
\begin{aligned}
F(\Bx,t) &= \int G(\Bx - \Bx', t) F(\Bx', 0) d^3 x' \\
G(\Bx,t) &= \inv{V} \sum_{\Bk} \exp\left( i \Bk ct \right) \exp\left( -i \Bk \cdot \Bx \right)
\end{aligned}
\end{equation}

We cannot however
commute the time phase term to construct a one sided Green's function for this
potential solution (or perhaps we can but if so shown or attempted to show that this is possible).  We also have a
plus or minus variation in the phase term due to the second order nature of the harmonic oscillator equations for our Fourier coefficients.

\subsection{Comparing the first and second order solutions}

A consequence of working in the Lorentz gauge (\(\grad \cdot A = 0\)) is that our field solution should be a gradient

\begin{equation}\label{eqn:potentialFourier:240}
\begin{aligned}
F
&= \grad \wedge A \\
&= \grad A \\
%&= \int A(\Bx', 0) \left(\grad G_A(\Bx - \Bx', t) \right) d^3 x' \\
\end{aligned}
\end{equation}

%Or with the opposite convolution
%\begin{align*}
%F &= \grad \int A(\Bx - \Bx', 0) G_A(\Bx', t) d^3 x' \\
%\end{align*}

FIXME: expand this out using \eqnref{eqn:potential_fourier:potentialSolution} to compare to the first order solution.

\part{Appendix.}
%
% Copyright � 2012 Peeter Joot.  All Rights Reserved.
% Licenced as described in the file LICENSE under the root directory of this GIT repository.
%

%
%
\chapter{Notation and definitions}\label{chap:notationTable}

Here is a summary of the notation and definitions that will be used.

The following tables summarize a lot of the notation used in these notes.
This largely follows conventions from \citep{doran2003gap}.

\section{Coordinates and basis vectors}

Greek letters range over all indices and English indices range over \(1,2,3\).

Bold vectors are spatial entities and non-bold is used for four vectors and scalars.

Summation convention \index{summation convention} is often used (less so in earlier notes).  This is
summation over all sets of matched upper and lower indices is implied.

While many things could be formulated in a metric signature independent
fashion,
a time positive
\((+,-,-,-)\)
metric signature should be assumed in most cases.  Specifically, that is \((\gamma_0)^2 = 1\), and \((\gamma_k)^2 = -1\).

\begin{equation*}
\begin{array}{l l l}
\gamma_{\mu} & \gamma_{\mu} \cdot \gamma_{\nu} = \pm {\delta^{\mu}}_{\nu} & \quad \mbox{Four vector basis vector} \\
& & \quad \mbox{(\(\gamma_{\mu} \cdot \gamma_{\nu} = \pm {\delta^{\mu}}_{\nu}\))} \\
{(\gamma_0)}^2 {(\gamma_k)}^2 &= -1 & \quad \mbox{Minkowski metric} \\
\sigma_k = \sigma^k &= \gamma_{k} \wedge \gamma_0 & \quad \mbox{Spatial basis bivector. (\(\sigma_k \cdot \sigma_j = \delta_{kj}\))} \\
                    &= \gamma_{k} \gamma_0 & \\
I &= \gamma_{0} \wedge \gamma_1 \wedge \gamma_{2} \wedge \gamma_3 & \quad \mbox{Four-vector pseudoscalar} \\
  &= \gamma_{0} \gamma_1 \gamma_{2} \gamma_3 & \\
  &= \gamma_{0123} \\
\gamma^{\mu} \cdot \gamma_{\nu} &= {\delta^{\mu}}_{\nu} & \quad \mbox{Reciprocal basis vectors} \\
x^{\mu} &= x \cdot \gamma^{\mu} & \quad \mbox{Vector coordinate} \\
x_{\mu} &= x \cdot \gamma_{\mu} & \quad \mbox{Coordinate for reciprocal basis} \\
x &= \gamma_{\mu} x^{\mu} & \quad \mbox{Four vector in terms of coordinates} \\
  &= \gamma^{\mu} x_{\mu} \\
x^{0} &= x \cdot \gamma^0 & \quad \mbox{Time coordinate (length dim.)} \\
      &= c t \\
\Bx &= x \wedge \gamma_0 & \quad \mbox{Spatial vector} \\
    &= x^k \sigma_k \\
x^2 &= x \cdot x & \quad \mbox{Four vector square. } \\
    &= x^\mu x_\mu \\
\Bx^2 &= \Bx \cdot \Bx & \quad \mbox{Spatial vector square. } \\
    &= \sum_{k=1}^3 (x^k)^2 \\
    &= \Abs{\Bx}^2 \\
\end{array}
\end{equation*}

If convient sometimes \(i\) will be used for the pseudoscalar.

\section{Electromagnetism}

SI units are
used in most places, but occasionally natural units are used.  In
some cases, when working with material such as \citep{bohm1989qt},
CGS modifications of the notation are employed.

\begin{equation*}
\begin{array}{l l l}
\BE &= \sum E^k \sigma_k & \quad \mbox{Electric field spatial vector} \\
\BB &= \sum B^k \sigma_k & \quad \mbox{Magnetic field spatial vector} \\
\bcE &= E^k \sigma_k & \quad \mbox{(CGS)Electric field spatial vector} \\
\bcH &= H^k \sigma_k & \quad \mbox{(CGS)Magnetic field spatial vector} \\
J &= \gamma_{\mu} J^{\mu} & \quad \mbox{Current density four vector.} \\
  &= \gamma^{\mu} J_{\mu} \\
F &= \BE + I c \BB & \quad \mbox{Electromagnetic (Faraday) bivector} \\
  &= F^{\mu\nu} \gamma_\mu \wedge \gamma_\nu & \quad \mbox{in terms of Faraday tensor} \\
  &= \bcE + I \bcH & \quad \mbox{(CGS)} \\
J^{0} &= J \cdot \gamma^0 & \quad \mbox{Charge density.} \\
      &= c \rho & \quad \mbox{(current density dimensions.)} \\
      &= \rho & \quad \mbox{(CGS) (current density dimensions.)} \\
\BJ &= J \wedge \gamma_0 & \quad \mbox{Current density spatial vector} \\
    &= J^k \sigma_k \\
\end{array}
\end{equation*}

\section{Differential operators}
\begin{equation*}
\begin{array}{l l l}
\partial_{\mu} &= \PDi{x^\mu}{} & \quad \mbox{Index up partial.} \\
\partial^{\mu} &= \PDi{x_\mu}{} & \quad \mbox{Index down partial.} \\
\partial_{\mu\nu} &= \PDi{x^\mu}{}\PDi{x^\nu}{} & \quad \mbox{Index up partial.} \\
\grad &= \sum \gamma^{\mu} \partial/\partial {x^{\mu}} & \quad \mbox{Spacetime gradient} \\
      &= \gamma^{\mu}\partial_{\mu} \\
      &= \sum \gamma_{\mu} \partial/\partial {x_{\mu}} \\
      &= \gamma_{\mu}\partial^{\mu} \\
\spacegrad &= \sigma^{k} \partial_k & \quad \mbox{Spatial gradient} \\
\hat{A}_{\Bk} &= \hat{A}_{k_1,k_2,k_3} & \quad \mbox{Fourier coefficient, integer indices.} \\
\grad^2 A
   &= (\grad \cdot \grad) A & \quad \mbox{Four Laplacian. } \\
   &= (\partial_{00} - \sum_k \partial_{kk}) A & \\
d^3 x &= dx^1 dx^2 dx^3 & \quad \mbox{Spatial volume element. } \\
d^4 x &= dx^0 dx^1 dx^2 dx^3 & \quad \mbox{Four volume element. } \\
\int_{\partial I} &= \int_{a}^{b} & \quad \mbox{Integration range \(I = [a,b]\) } \\
\text{STA} & & \quad \mbox{Space Time Algebra} \\
(xyz)^{\tilde{}} &= \widetilde{xyz} = z y x & \quad \mbox{Reverse of a vector product.} \\
\end{array}
\end{equation*}

\section{Misc}

The \(\PV\) notation is taken from \citep{lepage1980cva} where the author uses it in his Riemann integral proof of the inverse Fourier integral.

\begin{equation*}
\begin{array}{l l l}
\PV \IIinf &= \lim_{R\rightarrow \infty} \int_{R}^R & \quad \mbox{Integral Principle value} \\
\hat{A}(k) &= \calF(A(x)) & \quad \mbox{Fourier transform of \(A\)} \\
{A}(x) &= \calF^{-1}(A(k)) & \quad \mbox{Inverse Fourier transform} \\
\exp(i\Bk\phi) &=
\cos(\Abs{\Bk}\phi) + \frac{i \Bk}{\Abs{i\Bk}} \sin(\Abs{\Bk}\phi) & \quad \mbox{bivector exponential. } \\
\end{array}
\end{equation*}

\chapter{Some Fourier transform notes.}\label{chap:PJqmFourier}
\date{ Jan 09, 2009.  Last Revision: $Date: 2009/06/04 13:13:27 $ }

\section{Motivation. }

In \cite{mcmahon2005qmd} the Fourier transform pairs are written in a somewhat 
non-orthodox seeming way.

\begin{align*}
\phi(p) &= \FM \Iinf{\psi(x) e^{-ipx/\hbar} dx} \\
\psi(x) &= \FM \Iinf{\phi(p) e^{ipx/\hbar} dp} \\
\end{align*}

The aim here is to do verify this form and do a couple associated calculations (like the Rayleigh energy theorem).

\section{Verify transform pair. }

As an exercise to verify, in a not particularly rigorous fashion, that we get back our original function applying the forward and reverse transformations
in sequence.  Specifically, let's compute

\begin{align*}
\FF^{-1}(\FF(\psi(x)))
&= \PV \FM \Iinf{\left(\FM \Iinf{\psi(u) e^{-ipu/\hbar} du}\right) e^{ipx/\hbar} dp} \\
\end{align*}

Here $\PV$ is the principle value of the integral, which is the specifically symmetric integration

\begin{align*}
\PV \Iinf{} &= \lim_{R \rightarrow \infty} \int_{-R}^R
\end{align*}

We have for the integration
\begin{align*}
\FF^{-1}(\FF(\psi(x)))
&= \PV \inv{{2\pi\hbar}} \int du \psi(u) \int e^{i p (x-u)/\hbar} dp \\
\end{align*}

Now, let $v = (x-u)/\hbar$, or $u=x-v\hbar$ for

\begin{align*}
\inv{{2\pi}} \int dv \psi(x -v\hbar) \int_{-R}^R e^{i p v} dp 
&= \inv{{2\pi}} \int dv \psi(x - v\hbar) \left. \inv{i v} e^{i p v} \right\vert_{p=-R}^R \\
&= \int dv \psi(x - v\hbar) \frac{\sin(R v)}{\pi v} \\
\end{align*}

In a hand-waving (aka. Engineering) fashion, one can identify the limit of $\sin(Rv)/\pi v$ as the Dirac delta function and
then declare that this does in fact recover the value of $\psi(x)$ by a Dirac delta filtering around the point $v=0$.

This does in fact work out, but as a strict integration exercise one ought to be able to do better.
Observe that the integral performed here wasn't really valid for $v=0$ in which case the exponential takes the value of one, so it would be
better to treat the neighborhood of $v=0$ more carefully.  Doing so

\begin{align*}
\inv{{2\pi}} \int dv \psi(x -v\hbar) \int_{-R}^R e^{i p v} dp 
&= \int_{v=-\infty}^{-\epsilon} dv \psi(x - v\hbar) \frac{\sin(R v)}{\pi v} \\
&+ \int_{v=\epsilon}^\infty dv \psi(x - v\hbar) \frac{\sin(R v)}{\pi v} \\
&+ \inv{{2\pi}} \int_{v=-\epsilon}^\epsilon dv \psi(x - v\hbar) \int_{-R}^R e^{i p v} dp \\
&= \int_{v=\epsilon}^\infty dv \left( \psi(x - v\hbar) +\psi(x + v\hbar) \right) \frac{\sin(R v)}{\pi v} \\
&+ \inv{{2\pi}} \int_{v=-\epsilon}^\epsilon dv \psi(x - v\hbar) \int_{-R}^R e^{i p v} dp \\
\end{align*}

Now, evaluating this with $\epsilon$ allowing to tend to zero and $R$ tending to infinity simultaneously is troublesome seeming.  I seem to recall that one
can do something to the effect of setting $\epsilon=1/R$, and then carefully take the limit, but it isn't obvious to me how exactly to do this without
pulling out an old text.
While some kind of ad-hoc limit process can likely be done and justified in some fashion, one can see why the hard core mathematicians had to invent
an alternate stricter mathematics to deal with this stuff rigorously.

That said, from an intuitive point of view, it is fairly clear that the filtering involved here will recover the average of
$\psi(x)$ in the neighborhood assuming that it is piecewise continuous:

\begin{align*}
\FF^{-1}(\FF(\psi(x))) &= \inv{2} \left( \psi(x -\epsilon) + \psi(x + \epsilon) \right)
\end{align*}

After digging through my old texts I found a treatment of the Fourier integral very similar to what I've done above
in \cite{lepage1980cva}, but the important details aren't omitted (like integrability conditions).
I'd read that and some of my treatment is obviously was based on that.  That text 
treats this still with Riemann (and not Lebesgue) integration, but very carefully.

\section{Parseval's theorem. }

In \cite{mcmahon2005qmd} he notes that Parseval's theorem tells us

\begin{align*}
\Iinf{f(x)g(x) dx} &= \Iinf{F(k)G^\conj(k) dk} \\
\Iinf{\Abs{f(x)}^2 dx} &= \Iinf{\Abs{F(k)}^2 dk}
\end{align*}

The last of these in \cite{haykin1994cs} is called Rayleigh's energy theorem.  
As a refresher in 
Fourier manipulation, and to translate to the QM Fourier transform notation, let's go through the arguments
required to prove these.

\subsection{Convolution. }

We'll need convolution in the QM notation as a first step to express
the transform of a product.

Suppose we have two functions 
$\phi_i(x)$
, and their transform
pairs 
$\tilde{\phi}_i(x) = \FF(\phi_i)$, then the transform of the product is

\begin{align*}
\tilde{\Phi}_{12}(p) = \FF(\phi_1(x)\phi_2(x)) &= \FM \Iinf{\phi_1(x) \phi_2(x) e^{-ipx/\hbar} dx}
\end{align*}

Now write $\phi_2(x)$ in terms of its inverse transform

\begin{align*}
\phi_2(x)) &= \FM \Iinf{\tilde{\phi}_2(u) e^{iux/\hbar} du}
\end{align*}

The product transform is now
\begin{align*}
\tilde{\Phi}_{12}(p) 
&= \FM \Iinf{ \phi_1(x)             \FM \Iinf{\tilde{\phi}_2(u) e^{iux/\hbar} du} e^{-ipx/\hbar} dx} \\
&= \FM \Iinf{ du \tilde{\phi}_2(u)  \FM \Iinf{ \phi_1(x) e^{-ix(p-u)/\hbar} dx } } \\
&= \FM \Iinf{ du \tilde{\phi}_2(u)  \tilde{\phi}_1(p-u) } \\
&= \FM \Iinf{ dv \tilde{\phi}_1(v)  \tilde{\phi}_2(p-v)}  \\
\end{align*}

So we have product transform expressed by the convolution integral, but have an extra $1/\sqrt{2\pi\hbar}$ factor in this form

\begin{align}
\phi_1(x) \phi_2(x) \Leftrightarrow \FM \Iinf{ dv \tilde{\phi}_1(v)  \tilde{\phi}_2(p-v) } \\
\end{align}

\subsection{Conjugation. }

Next we need to see how the conjugate transforms.  This is pretty straight forward

\begin{align*}
\phi^\conj(x) 
&\Leftrightarrow \FM \Iinf{ \phi^\conj(x) e^{-ipx/\hbar dx}} \\
&= \left(\FM \Iinf{ \phi(x) e^{ipx/\hbar dx}}\right)^\conj \\
\end{align*}

So we have
\begin{align*}
\phi^\conj(x) \Leftrightarrow \left(\tilde{\phi}(-p) \right)^\conj \\
\end{align*}

\subsection{Rayleigh's Energy Theorem. }

Now, we should be set to prove the energy theorem.  Let's start with the 
momentum domain integral and translate back to position basis

\begin{align*}
\Iinf{ dp \tilde{\phi}(p){\tilde{\phi}}^\conj(p) } 
&= \Iinf{ dp \FM \Iinf{ dx \phi(x) e^{-ipx/\hbar}} {\tilde{\phi}}^\conj(p) }  \\
&= \Iinf{ dx \phi(x) \FM \Iinf{ dp {\tilde{\phi}}^\conj(p) } e^{-ipx/\hbar}} \\
&= \Iinf{ dx \phi(x) \FM \Iinf{ dp {\tilde{\phi}}^\conj(-p) } e^{ipx/\hbar}} \\
&= \Iinf{ dx \phi(x) \FF^{-1}( {\tilde{\phi}}^\conj(-p) )}  \\
\end{align*}

This is exactly our desired result
\begin{align*}
\Iinf{ dp \tilde{\phi}(p){\tilde{\phi}}^\conj(p) } &= 
\Iinf{ dx \phi(x) \phi^\conj(x)}
\end{align*}

Hmm.  Didn't even need the convolution as the systems book did.  Will have to look over how they did this more closely.  Regardless, this method was nicely direct.

%
% Copyright � 2012 Peeter Joot.  All Rights Reserved.
% Licenced as described in the file LICENSE under the root directory of this GIT repository.
%

%
%
\chapter{A cheatsheet for Fourier transform conventions}
\index{Fourier transform}
\label{chap:fourierNotation}
%\date{Jan 21, 2009.  fourierNotation.tex}

\section{A cheatsheet for different Fourier integral notations}

Damn.  There are too many different notations for the Fourier transform.  Examples are:

\begin{equation}\label{eqn:fourierNotation:20}
\begin{aligned}
\tilde{f}(k) &= \IIinf f(x) \exp\left( - 2 \pi i k x \right) dx \\
\tilde{f}(k) &= \inv{\sqrt{2\pi}} \IIinf f(x) \exp\left( -i k x \right) dx \\
\tilde{f}(p) &= \sqrt{\frac{1}{2\pi\Hbar}} \IIinf f(x) \exp\left( \frac{-i p x}{\Hbar} \right) dx \\
\end{aligned}
\end{equation}

There are probably many more, with other variations such as using hats over things instead of twiddles, and so forth.

Unfortunately each of these have different numeric factors for the inverse transform.
Having just been bitten by rogue factors of \(2 \pi\) after innocently switching notations, it seems
worthwhile
to express the Fourier transform with a general fudge factor in the exponential.  Then it can be
seen at a
glance what constants are required in the inverse transform given anybody's particular choice of the transform
definition.

Where to put all the factors can actually be seen from the QM formulation since one is free to treat \(\Hbar\)
as an arbitrary constant, but let us do it from scratch in a mechanical fashion without having to think
back to QM as a fundamental.

Suppose we define the Fourier transform as

\begin{equation}\label{eqn:fourierNotation:40}
\begin{aligned}
\tilde{f}(s) = \kappa \IIinf f(x) \exp\left( - i \alpha s x \right) dx \\
{f}(x) = \kappa' \IIinf \tilde{f}(s) \exp\left( i \alpha x s \right) ds \\
\end{aligned}
\end{equation}

Now, what factor do we need in the inverse transform to make things work out right?  With the Rigor
Police on holiday, let us expand the inverse transform integral in terms of the original transform
and see what these numeric factors must then be to make this work out.

Omitting temporarily the \(\kappa\) factors to be determined we have

\begin{equation}\label{eqn:fourierNotation:60}
\begin{aligned}
f(x)
&\propto \IIinf \left( \IIinf f(u) \exp\left( - i \alpha s u \right) du \right) \exp\left( i \alpha x s \right) ds \\
&= \IIinf f(u) du \IIinf \exp\left( i \alpha s (x-u) \right) ds \\
&= \IIinf f(u) du \lim_{R\rightarrow \infty} 2 \pi \inv{ \pi \alpha(x-u) }\sin\left( \alpha R (x-u) \right) \\
&= \IIinf f(u) du 2 \pi \delta\left( \alpha (x-u) \right) \\
&= \inv{\alpha} \IIinf f(v/\alpha) dv 2 \pi \delta\left( \alpha x - v \right) \\
&= \frac{2 \pi}{\alpha} f((\alpha x)/\alpha) \\
&= \frac{2 \pi}{\alpha} f(x)
\end{aligned}
\end{equation}

Note that to get the result above, after switching order of integration, and assuming that we can take the principle value of the integrals,
the usual ad-hoc \(\sinc\) and exponential integral identification of the delta function was made

\begin{equation}\label{eqn:fourierNotation:80}
\begin{aligned}
\PV \inv{2\pi} \IIinf \exp\left( i  s  x \right) ds
&= \lim_{R \rightarrow \infty} \inv{2\pi} \int_{-R}^R \exp\left( i  s  x \right) ds \\
&= \lim_{R\rightarrow \infty} \frac{\sin\left(  R  x \right)}{ \pi  x } \\
&\equiv \delta\left( x \right) \\
\end{aligned}
\end{equation}

The end result is that we will need to fix

\begin{equation}\label{eqn:fourierNotation:100}
\begin{aligned}
\kappa \kappa' = \frac{\alpha}{2\pi}
\end{aligned}
\end{equation}

to have the transform pair produce the desired result.  Our transform pair is therefore

\begin{equation}\label{eqn:fourier_notation:blah}
\begin{aligned}
\tilde{f}(s) = \kappa \IIinf f(x) \exp\left( - i \alpha s x \right) dx \Leftrightarrow f(x) = {\frac{ \alpha} {2 \pi \kappa} } \IIinf \tilde{f}(s) \exp\left( i \alpha s x \right) ds
\end{aligned}
\end{equation}

\section{A survey of notations}

From \eqnref{eqn:fourier_notation:blah} we can
express the required numeric factors that accompany all the various forward transforms conventions.  Let us do a quick survey of the bookshelf, ignoring differences in the \(i\)'s and \(j\)'s, differences in the transform variables, and so forth.

From my old systems and signals course, with the book \citep{haykin1994cs} we have, \(\kappa = 1\), and \(\alpha = 2 \pi\)

\begin{equation}\label{eqn:fourierNotation:120}
\begin{aligned}
\tilde{f}(s) &= \IIinf f(x) \exp\left( - 2 \pi i s x \right) dx \\
f(x) &= \IIinf \tilde{f}(s) \exp\left( 2 \pi i s x \right) ds \\
\end{aligned}
\end{equation}

The mathematician's preference, and that of
\citep{bohm1989qt}, and \citep{byron1992mca} appears to be the nicely symmetrical version, with \(\kappa = 1/\sqrt{2\pi}\), and \(\alpha = 1\)

\begin{equation}\label{eqn:fourierNotation:140}
\begin{aligned}
\tilde{f}(s) &= \inv{\sqrt{2\pi}} \IIinf f(x) \exp\left( - i  s x \right) dx \\
f(x) &= \inv{\sqrt{2 \pi}} \IIinf \tilde{f}(s) \exp\left( i  s x \right) ds \\
\end{aligned}
\end{equation}

From the old circuits course using \citep{irwin2007bec}, and also in the excellent text \citep{lepage1980cva}, we have \(\kappa = 1\), and \(\alpha = 1\)

\begin{equation}\label{eqn:fourierNotation:160}
\begin{aligned}
\tilde{f}(s) &= \IIinf f(x) \exp\left( - i  s x \right) dx  \\
f(x) &= \inv{{2 \pi}} \IIinf \tilde{f}(s) \exp\left( i  s x \right) ds \\
\end{aligned}
\end{equation}

and finally, the QM specific version from \citep{mcmahon2005qmd}, with \(\alpha = p/\Hbar\), and \(\kappa = 1/\sqrt{2\pi \Hbar}\) we have

\begin{equation}\label{eqn:fourierNotation:180}
\begin{aligned}
\tilde{f}(p) &= \inv{\sqrt{2\pi \Hbar}} \IIinf f(x) \exp\left( - \frac{i  p x}{\Hbar} \right) dx \\
f(x) &= \inv{\sqrt{2 \pi \Hbar}} \IIinf \tilde{f}(p) \exp\left( \frac{i  p x}{\Hbar} \right) dp \\
\end{aligned}
\end{equation}

%
% Copyright � 2012 Peeter Joot.  All Rights Reserved.
% Licenced as described in the file LICENSE under the root directory of this GIT repository.
%

%
%
\chapter{Projection with generalized dot product}\label{chap:PJprojGen}
\index{projection}
%\date{May 17, 2008.  projGeneralizedDotProd.tex}

\imageFigure{../../figures/gabook/visualize_subspace_projection}{Visualizing projection onto a subspace}{fig:Projection_subspace}{0.4}

We can geometrically visualize the projection problem
as in \cref{fig:Projection_subspace}.  Here
the subspace can be pictured
as a plane containing a set of mutually perpendicular basis vectors, as if
one has visually projected all the higher dimensional vectors onto a plane.

For a vector \(\Bx\) that contains some part not in the space we want to find
the component in the space \(\Bp\), or characterize the projection operation
that produces this vector, and also find the space of vectors that lie
perpendicular to the space.

Expressed in terms of
the Euclidean dot product this perpendicularity can be expressed explicitly as
\(U^\T \Bn = 0\).
This is why we say that \(\Bn\) is in the null space of \(U^\T\),
\(N(U^\T)\) not the null space of \(U\) itself (\(N(U)\)).  One perhaps could say this is in the null
or perpendicular space of the set \(\{u_i\}\), but the typical preference to use columns as
vectors makes this not entirely unnatural.

In a complex vector space with \(\Bu \cdot \Bv = \Bu^* \Bv\) transposition no longer expresses this
null space concept, so the null space
is the set of \(\Bn\), such that \(U^* \Bn = 0\), so one would say \(\Bn \in N(U^*)\).

One can generalize this projection and nullity to more general dot products.  Let us examine the projection
matrix calculation with respect to a more arbitrary inner product.  For an inner product that is conjugate linear in the first variable, and linear in second variable
we can write:

\begin{equation}\label{eqn:projGeneralizedDotProd:20}
\innerprod{\Bu}{\Bv} = \Bu^* A \Bv
\end{equation}

This is the most general complex bilinear form, and can thus represent any complex dot product.

The problem is the same as above.  We want to repeat the projection derivation
done with the Euclidean dot product, but
be more careful with ordering of terms since we now using a non-commutative dot (inner) product.

We are looking for vectors \(\Bp = \sum a_i \Bu_i\), and \(\Be\) such that
\begin{equation}\label{eqn:projGeneralizedDotProd:40}
\Bx = \Bp + \Be
\end{equation}

If the inner product defines the projection operation we have for any \(\Bu_i\)

\begin{equation}\label{eqn:projGeneralizedDotProd:60}
\begin{aligned}
0 &= \innerprod{\Bu_i}{\Be} \\
  &= \innerprod{\Bu_i}{\Bx - \Bp} \\
\implies \\
\innerprod{\Bu_i}{\Bx} &= \innerprod{\Bu_i}{\Bp} \\
                       &= \innerprod{\Bu_i}{\sum_j a_j \Bu_j} \\
                       &= \sum_j a_j \innerprod{\Bu_i}{\Bu_j} \\
\end{aligned}
\end{equation}

In matrix form, this is
\begin{equation*}
{
\begin{bmatrix}
\innerprod{\Bu_i}{\Bx}
\end{bmatrix}
}_i
=
{
\begin{bmatrix}
\innerprod{\Bu_i}{\Bu_j}
\end{bmatrix}
}_{ij}
[a_i]_i
\end{equation*}

Or
\begin{equation*}
A = [a_i]_i =
\inv{
   {
   \begin{bmatrix}
   \innerprod{\Bu_i}{\Bu_j}
   \end{bmatrix}
   }_{ij}
}
{
\begin{bmatrix}
\innerprod{\Bu_i}{\Bx}
\end{bmatrix}
}_i
\end{equation*}

We can also write our projection in terms of \(A\):

\begin{equation*}
\Bp =
\begin{bmatrix}
\Bu_1 & \Bu_2 & \cdots & \Bu_k
\end{bmatrix}
A
= U A
\end{equation*}

Thus the projection vector can be written:

\begin{equation*}
\Bp = U
\inv{
   {
   \begin{bmatrix}
   \innerprod{\Bu_i}{\Bu_j}
   \end{bmatrix}
   }_{ij}
}
{
\begin{bmatrix}
\innerprod{\Bu_i}{\Bx}
\end{bmatrix}
}_i
\end{equation*}

In matrix form this is:

\begin{equation}
\Proj_U(\Bx) = \left(U \inv{U^* A U} U^* A\right) \Bx
\end{equation}

Writing \(W^* = U^* A\), this is

\begin{equation*}
\Proj_U(\Bx) = \left(U \inv{W^* U} W^* A\right) \Bx
\end{equation*}

which is what the wikipedia article on projection calls an oblique projection.  Q: Can any oblique projection be expressed using just an alternate dot product?

\documentclass{article}      % Specifies the document class

\usepackage{amsmath}
\usepackage{mathpazo}

%
% shorthand for bold symbols, convenient for vectors and matrices
%
\newcommand{\Ba}[0]{\mathbf{a}}
\newcommand{\Bb}[0]{\mathbf{b}}
\newcommand{\Bc}[0]{\mathbf{c}}
\newcommand{\Bd}[0]{\mathbf{d}}
\newcommand{\Be}[0]{\mathbf{e}}
\newcommand{\Bf}[0]{\mathbf{f}}
\newcommand{\Bg}[0]{\mathbf{g}}
\newcommand{\Bh}[0]{\mathbf{h}}
\newcommand{\Bi}[0]{\mathbf{i}}
\newcommand{\Bj}[0]{\mathbf{j}}
\newcommand{\Bk}[0]{\mathbf{k}}
\newcommand{\Bl}[0]{\mathbf{l}}
\newcommand{\Bm}[0]{\mathbf{m}}
\newcommand{\Bn}[0]{\mathbf{n}}
\newcommand{\Bo}[0]{\mathbf{o}}
\newcommand{\Bp}[0]{\mathbf{p}}
\newcommand{\Bq}[0]{\mathbf{q}}
\newcommand{\Br}[0]{\mathbf{r}}
\newcommand{\Bs}[0]{\mathbf{s}}
\newcommand{\Bt}[0]{\mathbf{t}}
\newcommand{\Bu}[0]{\mathbf{u}}
\newcommand{\Bv}[0]{\mathbf{v}}
\newcommand{\Bw}[0]{\mathbf{w}}
\newcommand{\Bx}[0]{\mathbf{x}}
\newcommand{\By}[0]{\mathbf{y}}
\newcommand{\Bz}[0]{\mathbf{z}}
\newcommand{\BA}[0]{\mathbf{A}}
\newcommand{\BB}[0]{\mathbf{B}}
\newcommand{\BC}[0]{\mathbf{C}}
\newcommand{\BD}[0]{\mathbf{D}}
\newcommand{\BE}[0]{\mathbf{E}}
\newcommand{\BF}[0]{\mathbf{F}}
\newcommand{\BG}[0]{\mathbf{G}}
\newcommand{\BH}[0]{\mathbf{H}}
\newcommand{\BI}[0]{\mathbf{I}}
\newcommand{\BJ}[0]{\mathbf{J}}
\newcommand{\BK}[0]{\mathbf{K}}
\newcommand{\BL}[0]{\mathbf{L}}
\newcommand{\BM}[0]{\mathbf{M}}
\newcommand{\BN}[0]{\mathbf{N}}
\newcommand{\BO}[0]{\mathbf{O}}
\newcommand{\BP}[0]{\mathbf{P}}
\newcommand{\BQ}[0]{\mathbf{Q}}
\newcommand{\BR}[0]{\mathbf{R}}
\newcommand{\BS}[0]{\mathbf{S}}
\newcommand{\BT}[0]{\mathbf{T}}
\newcommand{\BU}[0]{\mathbf{U}}
\newcommand{\BV}[0]{\mathbf{V}}
\newcommand{\BW}[0]{\mathbf{W}}
\newcommand{\BX}[0]{\mathbf{X}}
\newcommand{\BY}[0]{\mathbf{Y}}
\newcommand{\BZ}[0]{\mathbf{Z}}

\newcommand{\Bzero}[0]{\mathbf{0}}
\newcommand{\Btheta}[0]{\boldsymbol{\theta}}
\newcommand{\Btau}[0]{\boldsymbol{\tau}}
\newcommand{\Bomega}[0]{\boldsymbol{\omega}}

%
% shorthand for unit vectors
%
\newcommand{\acap}[0]{\hat{\Ba}}
\newcommand{\bcap}[0]{\hat{\Bb}}
\newcommand{\ccap}[0]{\hat{\Bc}}
\newcommand{\dcap}[0]{\hat{\Bd}}
\newcommand{\ecap}[0]{\hat{\Be}}
\newcommand{\fcap}[0]{\hat{\Bf}}
\newcommand{\gcap}[0]{\hat{\Bg}}
\newcommand{\hcap}[0]{\hat{\Bh}}
\newcommand{\icap}[0]{\hat{\Bi}}
\newcommand{\jcap}[0]{\hat{\Bj}}
\newcommand{\kcap}[0]{\hat{\Bk}}
\newcommand{\lcap}[0]{\hat{\Bl}}
\newcommand{\mcap}[0]{\hat{\Bm}}
\newcommand{\ncap}[0]{\hat{\Bn}}
\newcommand{\ocap}[0]{\hat{\Bo}}
\newcommand{\pcap}[0]{\hat{\Bp}}
\newcommand{\qcap}[0]{\hat{\Bq}}
\newcommand{\rcap}[0]{\hat{\Br}}
\newcommand{\scap}[0]{\hat{\Bs}}
\newcommand{\tcap}[0]{\hat{\Bt}}
\newcommand{\ucap}[0]{\hat{\Bu}}
\newcommand{\vcap}[0]{\hat{\Bv}}
\newcommand{\wcap}[0]{\hat{\Bw}}
\newcommand{\xcap}[0]{\hat{\Bx}}
\newcommand{\ycap}[0]{\hat{\By}}
\newcommand{\zcap}[0]{\hat{\Bz}}
\newcommand{\thetacap}[0]{\hat{\Btheta}}

%
% to write R^n and C^n in a distinguishable fashion.  Perhaps change this
% to the double lined characters upon figuring out how to do so.
%
\newcommand{\C}[1]{$\mathbb{C}^{#1}$}
\newcommand{\R}[1]{$\mathbb{R}^{#1}$}

%
% various generally useful helpers
%

% derivative of #1 wrt. #2:
\newcommand{\D}[2] {\frac {d#2} {d#1}}

\newcommand{\inv}[1]{\frac{1}{#1}}
\newcommand{\cross}[0]{\times}

\newcommand{\abs}[1]{\lvert{#1}\rvert}
\newcommand{\norm}[1]{\lVert{#1}\rVert}
\newcommand{\innerprod}[2]{\langle{#1}, {#2}\rangle}
\newcommand{\dotprod}[2]{{#1} \cdot {#2}}
\newcommand{\bdotprod}[2]{\left({#1} \cdot {#2}\right)}
\newcommand{\crossprod}[2]{{#1} \cross {#2}}
\newcommand{\tripleprod}[3]{\dotprod{\left(\crossprod{#1}{#2}\right)}{#3}}

\DeclareMathOperator{\Proj}{Proj}
\DeclareMathOperator{\Span}{span}
\DeclareMathOperator{\Sgn}{sgn}
\DeclareMathOperator{\Area}{Area}
\DeclareMathOperator{\Volume}{Volume}

%
% A few miscellaneous things specific to this document
%
\newcommand{\crossop}[1]{\crossprod{#1}{}}

% R2 vector.
\newcommand{\VectorTwo}[2]{
\begin{bmatrix}
 {#1} \\
 {#2}
\end{bmatrix}
}

\newcommand{\VectorN}[1]{
\begin{bmatrix}
{#1}_1 \\
{#1}_2 \\
\vdots \\
{#1}_N \\
\end{bmatrix}
}

\newcommand{\DETuvij}[4]{
\begin{vmatrix}
 {#1}_{#3} & {#1}_{#4} \\
 {#2}_{#3} & {#2}_{#4}
\end{vmatrix}
}

\newcommand{\DETuvwijk}[6]{
\begin{vmatrix}
 {#1}_{#4} & {#1}_{#5} & {#1}_{#6} \\
 {#2}_{#4} & {#2}_{#5} & {#2}_{#6} \\
 {#3}_{#4} & {#3}_{#5} & {#3}_{#6}
\end{vmatrix}
}

\newcommand{\DETuvwxijkl}[8]{
\begin{vmatrix}
 {#1}_{#5} & {#1}_{#6} & {#1}_{#7} & {#1}_{#8} \\
 {#2}_{#5} & {#2}_{#6} & {#2}_{#7} & {#2}_{#8} \\
 {#3}_{#5} & {#3}_{#6} & {#3}_{#7} & {#3}_{#8} \\
 {#4}_{#5} & {#4}_{#6} & {#4}_{#7} & {#4}_{#8} \\
\end{vmatrix}
}

%\newcommand{\DETuvwxyijklm}[10]{
%\begin{vmatrix}
% {#1}_{#6} & {#1}_{#7} & {#1}_{#8} & {#1}_{#9} & {#1}_{#10} \\
% {#2}_{#6} & {#2}_{#7} & {#2}_{#8} & {#2}_{#9} & {#2}_{#10} \\
% {#3}_{#6} & {#3}_{#7} & {#3}_{#8} & {#3}_{#9} & {#3}_{#10} \\
% {#4}_{#6} & {#4}_{#7} & {#4}_{#8} & {#4}_{#9} & {#4}_{#10} \\
% {#5}_{#6} & {#5}_{#7} & {#5}_{#8} & {#5}_{#9} & {#5}_{#10}
%\end{vmatrix}
%}

% R3 vector.
\newcommand{\VectorThree}[3]{
\begin{bmatrix}
 {#1} \\
 {#2} \\
 {#3}
\end{bmatrix}
}


\newcommand{\xdot}[0]{\dot{x}}
\newcommand{\xddot}[0]{\ddot{x}}

%
% The real thing:
%

                             % The preamble begins here.
%\title{} % Declares the document's title.
%\author{Peeter Joot}         % Declares the author's name.
%\date{}        % Deleting this command produces today's date.

\begin{document}             % End of preamble and beginning of text.

%\maketitle{}
%\section{}

\begin{equation*}
L = \inv{12}m^2 \xdot^4 - m \xdot^2 V + V^2
\end{equation*}

\begin{align*}
\frac{\partial L}{\partial x} &= \frac{d}{dt} \frac{\partial L}{\partial \xdot} \\
-m \xdot^2 V_x + 2V V_x &= \frac{d}{dt} \left( \inv{3}m^2 \xdot^3 - 2 m \xdot V \right) \\
                        &= m^2 \xdot^2 \xddot - 2m \xddot V \\
-(m \xdot^2 -2V) V_x    &= m \xddot (m \xdot^2 - 2 V) \\
\end{align*}

Cancelling left and right common factors:

\begin{equation*}
m \xddot = -\frac{\partial V}{\partial x}
\end{equation*}

\end{document}               % End of document.

\part{Cronology}
\chapter{Cronological Index}
\begin{itemize}

\item October 13, 2007 \ref{chap:gaWiki} Comparison of many traditional vector and GA identities

\item October 13, 2007 \ref{chap:gaWikiTorque} Torque

\item October 16, 2007 \ref{chap:PJUnitDer} Derivatives of a unit vector

\item October 16, 2007 \ref{chap:gaWikiCramers} Cramer's rule

\item October 22, 2007 \ref{chap:PJRadialDer} Radial components of vector derivatives

\item January 1, 2008 \ref{chap:plane} More details on NFCM plane formulation

\item January 29, 2008 \ref{chap:PJAngVel} Rotational dynamics

\item January 29, 2008 \ref{chap:maxwellsGa} Maxwell's equations expressed with Geometric Algebra

\item February 2, 2008 \ref{chap:quaternion} Quaternions

\item February 4, 2008 \ref{chap:legendre} Legendre Polynomials

\item February 15, 2008 \ref{chap:inertialTensor} Inertia Tensor

\item February 19, 2008 \ref{chap:rotor} Rotor Notes

\item February 28, 2008 \ref{chap:laplace} Exponential Solutions to Laplace Equation in \R{N}

\item March 9, 2008 \ref{chap:bivector} Bivector Geometry

\item March 9, 2008 \ref{chap:trivector} Trivector geometry

\item March 12, 2008 \ref{chap:kvectorExponential} Exponential of a blade

\item March 16, 2008 \ref{chap:scalarCommutes} Multivector product grade zero terms

\item March 17, 2008 \ref{chap:angleBetweenLineAndPlane} Angle between geometric elements

\item March 17, 2008 \ref{chap:gaGradeDotWedge} An earlier attempt to intuitively introduce the dot, wedge, cross, and geometric products

\item March 25, 2008 \ref{chap:bladegradereduction} Blade grade reduction

\item March 29, 2008 \ref{chap:reciprocalFrame} Reciprocal Frame Vectors

\item March 31, 2008 \ref{chap:gradientAndForms} Exterior derivative and chain rule components of the gradient

\item April 1, 2008 \ref{chap:orthodecomp} Orthogonal decomposition take II

\item April 11, 2008 \ref{chap:matrixReview} Matrix review

\item April 13, 2008 \ref{chap:locateSatellite} Satellite triangulation over sphere

\item April 30, 2008 \ref{chap:PJKeRot} Kinetic Energy in rotational frame

\item May 7, 2008 \ref{chap:lorentzRotation} Lorentz Force Trajectory

\item May 16, 2008 \ref{chap:obliqueProj} Oblique projection and reciprocal frame vectors

\item May 16, 2008 \ref{chap:matrixOfLinearTx} Matrix of grade k multivector linear transformations

\item May 16, 2008 \ref{chap:projectionAndMoorePenroseVectorInverse} Projection and Moore-Penrose vector inverse

\item May 17, 2008 \ref{chap:PJprojGen} Projection with generalized dot product

\item June 6, 2008 \ref{chap:tensor} Gradient and tensor notes

\item June 10, 2008 \ref{chap:PJAngAcc} Angular Velocity and Acceleration.  Again

\item June 25, 2008 \ref{chap:lorentz} Wave equation based Lorentz transformation derivation

\item July 8, 2008 \ref{chap:PJAngAccCross} Cross product Radial decomposition

\item July 12, 2008 \ref{chap:PJMaxwell2} Back to Maxwell's equations

\item July 16, 2008 \ref{chap:spacetimegrad} Lorentz transformation of spacetime gradient

\item July 20, 2008 \ref{chap:sgMx41} Magnetic field between two parallel wires

\item August 1, 2008 \ref{chap:fourvecDotinvariance} Four vector dot product invariance and Lorentz rotors

\item August 9, 2008 \ref{chap:newtonianLagrangianAndGradient} Newton's Law from Lagrangian

\item August 13, 2008 \ref{chap:cauchyGradient} Cauchy Equations expressed as a gradient

\item August 13, 2008 \ref{chap:velocityTx} Understanding four velocity transform from rest frame

\item August 15, 2008 \ref{chap:emPotential} Four vector potential

\item August 16, 2008 \ref{chap:PJSrGAFPLorentzForce} Lorentz force Law

\item August 21, 2008 \ref{chap:PJSrLagrangian} Covariant Lagrangian, and electrodynamic potential

\item August 25, 2008 \ref{chap:PJTongMf1} Solutions to David Tong's mf1 Lagrangian problems

\item August 28, 2008 \ref{chap:massVaryLagrangian} Equations of motion given mass variation with spacetime position

\item September 1, 2008 \ref{chap:PJCanMomentum} Vector canonical momentum

\item September 2, 2008 \ref{chap:outermorphismDet} OuterMorphism Question 

\item September 5, 2008 \ref{chap:emBivectorMetricDependencies} Metric signature dependencies

\item September 7, 2008 \ref{chap:PJMaxwellTensor} Tensor relations from bivector field equation

\item September 8, 2008 \ref{chap:PJMaxwellLagrangian} Direct variation of Maxwell equations

\item September 9, 2008 \ref{chap:PJMaxwellProj} Vector forms of Maxwell's equations as projection and rejection operations

\item September 18, 2008 \ref{chap:PJStokes1} Stokes law in wedge product form

\item September 26, 2008 \ref{chap:stokesMaxwellApplication} Application of Stokes Integrals to Maxwell's Equation

\item September 27, 2008 \ref{chap:PJStokes2} Stokes Law revisited with algebraic enumeration of boundary

\item October 8, 2008 \ref{chap:PJSrLorentzForce} Revisit Lorentz force from Lagrangian

\item October 10, 2008 \ref{chap:PJFieldLagrangian} Derivation of Euler-Lagrange field equations

\item October 12, 2008 \ref{chap:maxwellTensorLagrangian} Tensor Derivation of Covariant Lorentz Force from Lagrangian

\item October 13, 2008 \ref{chap:PJEulerLagrange} Euler Lagrange Equations

\item October 19, 2008 \ref{chap:PJBoostMaxwell} Lorentz Invariance of Maxwell Lagrangian

\item October 22, 2008 \ref{chap:PJLorentzTxInteraction} Lorentz transform Noether current for interaction Lagrangian

\item October 26, 2008 \ref{chap:gem} GravitoElectroMagnetism

\item October 29, 2008 \ref{chap:PJNoethersField} Field form of Noether's Law

\item November 1, 2008 \ref{chap:eulerangle} Euler Angle Notes

\item November 8, 2008 \ref{chap:complex} Hyper complex numbers and symplectic structure

\item November 13, 2008 \ref{chap:sphericalPolar} Spherical polar coordinates

\item November 22, 2008 \ref{chap:gaussianSurface} Gaussian Surface invariance for radial field

\item November 23, 2008 \ref{chap:chargeArcElement} Field due to line charge in arc

\item November 23, 2008 \ref{chap:chargeLineElement} Charge line element

\item November 27, 2008 \ref{chap:nfcmCh2} Some NFCM exercise solutions and notes

\item November 30, 2008 \ref{chap:PJwaveFourVector} Expressing wave equation exponential solutions using four vectors

\item November 30, 2008 \ref{chap:slerp} Rotor interpolation calculation

\item December 6, 2008 \ref{chap:pauliMatrix} Pauli Matrixes in Clifford Algebra

\item December 11, 2008 \ref{chap:bohr} Bohr Model

\item December 13, 2008 \ref{chap:PJDiracGamma} Gamma Matrices

\item December 21, 2008 \ref{chap:diracLagrangian} Dirac Lagrangian

\item December 27, 2008 \ref{chap:PJrayleighJeans} Rayleigh-Jeans Law Notes

\item December 29, 2008 \ref{chap:PJpoynting} Poynting vector and Electromagnetic Energy conservation

\item January 1, 2009 \ref{chap:PJemstresstensor} Energy momentum tensor

\item January 3, 2009 \ref{chap:PJelectricFieldEnergy} Field and wave energy and momentum

\item January 5, 2009 \ref{chap:vectorDifferentialIdentities} Vector Differential Identities

\item January 6, 2009 \ref{chap:dcPower} DC Power consumption formula for resistive load

\item January 9, 2009 \ref{chap:PJqmFourier} Some Fourier transform notes

\item January 11, 2009 \ref{chap:schCurrent} Schr\"{o}dinger equation probability conservation

\item January 13, 2009 \ref{chap:radial} Polar velocity and acceleration

\item January 18, 2009 \ref{chap:PJpoyntingRate} Time rate of change of the Poynting vector, and its conservation law

\item January 19, 2009 \ref{chap:PJheatFourier} Fourier Solutions to Heat and Wave equations

\item January 21, 2009 \ref{chap:fourierNotation} A cheatsheet for Fourier transform conventions

\item January 25, 2009 \ref{chap:PJemWave} Electrodynamic wave equation solutions

\item January 26, 2009 \ref{chap:PJwaveFourier} Fourier transform solutions to the wave equation

\item January 29, 2009 \ref{chap:PJfourierMaxwellSecondOrder} Fourier transform solutions to Maxwell's equation

\item January 31, 2009 \ref{chap:PJfirstOrderMaxwell} First order Fourier transform solution of Maxwell's equation

\item February 1, 2009 \ref{chap:PJ4dFourier} 4D Fourier transforms applied to Maxwell's equation

\item February 3, 2009 \ref{chap:PJFourierVacuum} Fourier series Vacuum Maxwell's equations

\item February 7, 2009 \ref{chap:potentialFourier} Lorentz Gauge Fourier Vacuum potential solutions

\item February 8, 2009 \ref{chap:PJplaneWave} Plane wave Fourier series solutions to the Maxwell vacuum equation

\item February 13, 2009 \ref{chap:PJstressEnergyLorentz} Lorentz force relation to the energy momentum tensor

\item February 17, 2009 \ref{chap:en_m_tensor} Energy momentum tensor relation to Lorentz force

\item February 18, 2009 \ref{chap:PJpoisson} Poisson and retarded Potential Green's functions from Fourier kernels

\item February 26, 2009 \ref{chap:nvolume} Spherical and hyperspherical parametrization

\item March 13, 2009 \ref{chap:levi} Levi-Civitica summation identity

\item March 18, 2009 \ref{chap:electronRotor} Lorentz force rotor formulation

\item April 15, 2009 \ref{chap:lorentzForcePQA} Lorentz force Lagrangian with conjugate momentum

\item April 18, 2009 \ref{chap:biotSavart} Biot Savart Derivation

\item April 20, 2009 \ref{chap:maxwellTensorFromLagrangian} Tensor derivation of non-dual Maxwell equation from Lagrangian

\item April 28, 2009 \ref{chap:PJmultiTaylors} Developing some intuition for Multivariable and Multivector Taylor Series

\item May 23, 2009 \ref{chap:lorentzForceTx} Lorentz boost of Lorentz force equations

\item May 28, 2009 \ref{chap:macroscopicMaxwell} Macroscopic Maxwell's equation

\item June 1, 2009 \ref{chap:poincareTx} Poincare transformations

\item June 5, 2009 \ref{chap:stressEnergyNoethers} Canonical energy momentum tensor and Lagrangian translation

\item June 17, 2009 \ref{chap:lForceLag2} Comparison of two covariant Lorentz force Lagrangians

\item June 21, 2009 \ref{chap:emVacWave} Wave equation form of Maxwell's equations

\item June 27, 2009 \ref{chap:frequencyTx} Relativistic Doppler formula

\end{itemize}

\part{Bibliography and Readings.}
%
% Copyright � 2012 Peeter Joot.  All Rights Reserved.
% Licenced as described in the file LICENSE under the root directory of this GIT repository.
%

%
%
%\chapter{Learning Geometric Algebra/Clifford Algebra}
\chapter{Further reading}
\label{chap:gabookmark}

%\setlength{\textwidth}{13in}
%\setlength{\oddsidemargin}{0in}
%\setlength{\evensidemargin}{0in}

%\section{Learning Geometric Algebra and its applications}

There is a wealth of information on the subject available online, but finding information at an appropriate level may be difficult.  Not all resources use the same notation or nomenclature, and one can get lost in a sea of product operators.
Some of the introductory material also assumes knowledge of various levels of physics.  This is natural since the algebra can be utilized well to expresses many physics concepts.  While natural, this can also be intimidating if one is unprepared, so mathematics that one could potentially understand may be presented
in a fashion that is inaccessible.

%Colected here is an attempt to collect some of the available online information.

\section{Geometric Algebra for Computer Science Book}

The book
\href{http://www.geometricalgebra.net/tour.html}{Geometric Algebra For Computer Science.}
by Dorst, Fontijne, and Mann has one of the best introductions to the subject that I have seen.  It is also fairly inexpensive (\\(60 Canadian).  Compared for example to Hestenes's ``From Clifford Algebra to Geometric Calculus'' which I have seen listed on amazon.com with a default price of \\)250, discounted to \$150.

This book contains particularly good introductions to the dot and wedge products, both for vectors, and the generalizations.  How these can be applied and what they can be used to model is covered excellently.

Compromises have been made in this book on the order to present information, and what level of detail to use and when.  Many proofs are deferred or placed only in the appendix.  For example, they introduce (define) a scalar product initially (denoted with an asterisk (*)), and define this using a determinant without motivation.  This allows for development of a working knowledge of how to apply the subject.

Once an ability to apply has been developed they proceed with an axiomatic development.  I would consider an axiomatic approach to the subject very important since there is a sea of identities associated with the algebra.  Figuring out which ones are consequences of the others can be difficult, if one starts with definitions that are not fundamental.  One can easily go in circles and wonder really are the basic rules (this was my first impression starting with the Hestenes book ``New Foundations for Classical Mechanics''.   The book ``Geometric Algebra for Physicists'' has an excellent axiomatic development.  It however notably makes a similar compromise first introducing the algebra with a dot plus wedge product formulation to develop some familiarity.

This book has three parts.  The first is on the algebra, covering the generalized dot and wedge products, rotors, projections, join, linear transformations as outermorphisms, and all the rest of the basic material that one would expect.  It does this excellently.

The second portion of this book is on the use of a 5D conformal model for 3D graphics (adding a point at infinity on top of the normal extra viewport dimension that traditional graphics applications use).  I can not comment too much on this part of the book since I loaned it to a friend after reading the first and last parts of the book.

The last part of the book is on implementation, and makes for an interesting read.  Details on their Gaigen implementation are discussed, as are performance and code size implications of their implementation.

The only thing negative I have to say about this book is the unfortunate introduction of an alternate notation for the generalized dot product (L and backwards L).  This is distracting if one started, like I did, with the Hestenes, Cambridge, or Baylis papers or books, and their notation dominates the literature as far as I can tell.  This does not take too long to adjust, since one mostly just has to mentally substitute dots for L's (although there are some subtle differences where this transposition does not necessarily work).

\section{GAViewer}

Performing the GAViewer tutorial exercises is a great way to build some intuition to go along with the math (putting the geometric back in the algebra).

There are specific GAViewer exercises that you can do independent of the book, and there is also an excellent interactive tutorial 2003 Game Developer Lecture available here:

\href{http://www.science.uva.nl/ga/tutorials/}{Interactive GA tutorial. UvA GA Website: Tutorials}

 (they have hijacked GAViewer here to use as presentation software, but you can go through things at your own pace, and do things such as rotating viewpoints). Quite neat, and worth doing just to play with the graphical cross product manipulation even if you decide not to learn GA.
\section{Other resources from Dorst, Fontijne, and Mann}

There are other web resources available associated with this book that are quite good. The best of these is GAViewer, a graphical geometric calculator that was the product of some of the research that generated this book.

See
%\href{http://staff.science.uva.nl/~fontijne/phd.html}{Daniel Fontijne PhD thesis}
, or his paper itself
%\href{http://staff.science.uva.nl/~fontijne/phd/fontijne_phd.pdf}{fontijne_phd.pdf}
.

Some other links:

\href{http://staff.science.uva.nl/~leo/clifford/index.html}{Geometric algebra (Clifford algebra)}


This is
\href{http://staff.science.uva.nl/~leo/clifford/dorst-mann-I.pdf}{a good tutorial}
, as it focuses on the geometrical rather than have any tie to physics (fun but more to know).  The following looks like a slightly longer updated version:

\href{http://staff.science.uva.nl/~leo/clifford/dorst-mann-I.pdf}{GA: a practical tool for efficient geometric representation (Dorst)}

\section{Learning GA}
Of the various GA primers and workbooks above, here are a couple specific documents that are noteworthy, and some direct links to a few things that can be found by browsing that were noteworthy.
This is an
\href{http://www.science.uva.nl/ga/tutorials/}{interactive GA tutorial/presentation for a game programmers conference}

that provides a really good intro and has a lot of examples that I found helpful to get an intuitive feel for all the various product operations and object types.
Even if you weare not trying to learn GA, if you have done any traditional vector algebra/calculus, IMO its worthwhile to download this just to just to see the animation of how the old cross product varies with changes to the vectors.
You have to download the GAViewer program (graphical vector calculator) to run the presentation. Once you do that you can use it for other calculation examples, such as those available in these examples of how to use GAViewer as a standalone tool.. Note that the book the drills are from use a different notation for dot product (with a slightly different meaning and uses an oriented L symbol dependent on the grades of the blades.

\href{http://www.lomont.org/Math/GeometricAlgebra/Geometric%20Algebra%20Primer%20-%20Suter%20-%202003.pdf}{Jaap Suter's GA primer}.
\href{http://www.jaapsuter.com/}{His website}, which is referenced in various GA papers no longer (at least obviously) has this primer on it any longer (Sept/2008).

\href{http://www.iancgbell.clara.net/maths/geoalg.htm}{Ian Bell's introduction to GA}

    This author has a wide range of GA information, but looking at it will probably give you a headache.


\href{http://en.wikipedia.org/wiki/Geometric_algebra.}{GA wikipedia}

There are a number of comparisons here between GA identities and traditional vector identities, that may be helpful to get oriented.

- Maths - Clifford / Geometric Algebra - Martin Baker

A GA intro, a small part in the much larger Euclidean space website.

- As mentioned above there is a lot of learning GA content available in the Cambridge/Baylis/Hestenes/Dorst/... sites.

\section{Cambridge}
The Cambridge GA group has a number of Geometric Algebra publications, including the book

\href{http://www.mrao.cam.ac.uk/~cjld1/pages/book.htm}{Geometric Algebra for Physicists}

This book has an excellent introductory treatment of a number of basic GA concepts, a number of which are much easier to follow than similar content in Hestenes's "New Foundations for Classical Mechanics".  When it comes to physics content in this book there are a lot of details left out, so it is not the best for learning the physics itself if you are new to the topic in question.

Much of the content of their book
is actually available online in their publications above, but it is hard to beat coherent organization and a paper version that you can mark up.

Some other online learning content from the Cambridge group includes

\href{http://www.mrao.cam.ac.uk/~clifford/introduction/index.html}{Introduction to Geometric Algebra}

This is an HTML version of the
\href{http://www.mrao.cam.ac.uk/~clifford/publications/ps/imag_numbs.pdf}{Imaginary numbers are not real paper.}

A
nice starting point is lect1.pdf from the
\href{http://www.mrao.cam.ac.uk/~clifford/ptIIIcourse/GeometricAlgebraLectures.zip}{Cambridge PartIII physics course on GA applications}
. Only at the very end of this first pdf is any real physics content.
taught to what sounds like final year undergrad physics students.  The first parts of this do not need much physics knowledge.

\section{Baylis}

\href{http://www.uwindsor.ca/users/b/baylis/main.nsf}{Wiliam Baylis GA page}

He uses a scalar plus vector multivector representation for relativity (APS, Algebra of Physical Space), and an associated conjugate length operation.  You will find an intro relativity, GA workbook, and some papers on GA applied to physics here.
Also based on his APS approach is the following wikibook:

\href{http://en.wikibooks.org/wiki/Physics_in_the_Language_of_Geometric_Algebra._An_Approach_with_the_Algebra_of_Physical_Space}{Physics in the Language of Geometric Algebra. An Approach with the Algebra of Physical Space}

\section{Hestenes}
Hestenes main page for GA is
\href{http://modelingnts.la.asu.edu/}{Geometric Calculus R \& D Home Page}

This includes a number of primers and introductions to the subject such as
\href{http://modelingnts.la.asu.edu/pdf/PrimerGeometricAlgebra.pdf}{Geometric Algebra Primer.}
As described in the
\href{http://modelingnts.la.asu.edu/html/IntroPrimerGeometricAlgebra.html}{Introduction page for this primer}, this is a workbook, and reading should not be passive.

Also available is his
\href{http://modelingnts.la.asu.edu/pdf/OerstedMedalLecture.pdf}{Oersted Lecture}, which contains a good introduction.

If you do not have his ``New Foundations of Classical Mechanics'' book, you can find some of the dot-product/wedge-product reduction formulas in the following
\href{http://modelingnts.la.asu.edu/pdf/UGA.pdf}{non-metric treatment of GA.}

Also interesting is this
\href{http://modelingnts.la.asu.edu/pdf/GTG.w.GC.FP.pdf}{Gauge Theory Gravity with Geometric Algebra}
paper.  This has an introduction to STA (Space Time Algebra) as used in the Cambridge books.  This also shows at a high level where one can go with a lot of these ideas (like the grad F = J formulation of Maxwell's equation, a multivector form that incorporates all of the traditional four vector Maxwell's equations).  Nice teaser document if you intend to use GA for physics study, but hard to read even the consumable bits because they are buried in among a lot of other higher level math and physics.


Hestenes, Li and Rockwood in their paper
\href{http://modelingnts.la.asu.edu/pdf/CompGeom-ch1.pdf}{ New Algebraic Tools for Classical Geometry}
in G. Sommer (ed.) Geom.
Computing with Clifford Algebras (Springer, 2001) treat outermorphisms
and determinants in a separate subsection entitled "Outermorphism"
of section 1.3 Linear Transformations:

This is a comprehensive doc.  Content includes:
\begin{itemize}
\item GA intro boilerplate.
\item Projection and Rejection.
\item Meet and Join.
\item Reciprocal vectors (dual frame).
\item Vector differentiation.
\item Linear transformations.
\item Determinants and outermorphisms.
\item Rotations.
\item Simplexes and boundaries
\item Dual quaternions.
\end{itemize}

%\section{Peeter's GA/Physics Topics}
%
%\href{http://en.wikipedia.org/wiki/Geometric_algebra}{GA wikipedia}
%
%I dumped a bunch of info in this doc as I was puzzling things out (initially before I had purchased any books).  I have since stopped contributing since it was getting too big and was no longer appropriate (ie: getting bookish instead of encyclopedic).
%
%\href{http://sites.google.com/site/peeterjoot/}{Instead I have got a bunch of standalone latex writeups of my notes here.}
%
%I have a number of smallish GA related documents, which in reverse chronological order document my own learning/relearning roadmap (for GA and physics).  Hopefully useful for others too.  Please email peeterjoot@protonmail.com if any mistakes are found (other than ones that are already described in the index that I have not gone back to fix).
%

\section{Eckhard M. S. Hitzer (University of Fukui)}

From
\href{http://sinai.mech.fukui-u.ac.jp/gala2/}{Eckhard's Geometric Algebra Topics.}

Since these are all specific documents, and all at a fairly consumable level for a new learner, I have listed them here specifically:

\begin{itemize}
\item
\href{http://sinai.mech.fukui-u.ac.jp/gala2/GAtopics/axioms.pdf}{Axioms of geometric algebra}
\item
\href{http://sinai.mech.fukui-u.ac.jp/gala2/GAtopics/qform.pdf}{The use of quadratic forms in geometric algebra}
\item
\href{http://sinai.mech.fukui-u.ac.jp/gala2/GAtopics/products.pdf}{The geometric product and derived products}
\item
\href{http://sinai.mech.fukui-u.ac.jp/gala2/GAtopics/det.pdf}{Determinants in geometric algebra}
\item
\href{http://sinai.mech.fukui-u.ac.jp/gala2/GAtopics/GS.pdf}{Gram-Schmidt orthogonalization in geometric algebra}
\item
\href{http://sinai.mech.fukui-u.ac.jp/gala2/GAtopics/WhatIsi.pdf}{What is an imaginary number?}
\item
\href{http://sinai.mech.fukui-u.ac.jp/gcj/publications/mvdifcalc/mvdc.pdf}{Simplical calculus:}
\end{itemize}

\section{Electrodynamics}

John Denker has a number of GA docs that all appear very readable.  One such doc is:

\href{http://www.av8n.com/physics/maxwell-ga.pdf}{Electromagnetism using Geometric Algebra versus Components}

This is a nice little doc (there is also an HTML version, but it is very hard to read, and the first time I saw it I actually missed a lot of content).

The oft repeated introduction to GA is not in this doc, so you have to know the basics first.  Denker takes the \(\grad F = J/c \epsilon_0\) equation and unpacks it in a brute force but understandable fashion, and shows that these are identical to the vector differential form of Maxwell's equations.  A few other E\&M constructs are shown in their GA form (covariant form of Lorentz force equation, Lagrangian density, Stress tensor, Poynting Vector.  There are also many good comments on notation issues.

A cautionary note if you have read any of the Cambridge papers.  This doc uses a -+++ metric instead of the +--- used in those docs.

Some other Denker GA papers:
\begin{itemize}
\item
\href{http://www.av8n.com/physics/straight-wire.pdf}{Magnetic field of a straight wire.}
\item
\href{http://www.av8n.com/physics/clifford-intro.pdf}{Clifford Intro.}

Very nice axiomatic introduction with excellent commentary.  Also includes an STA intro.

\item
\href{http://www.av8n.com/physics/complex-clifford.pdf}{Complex numbers.}
\item
\href{http://www.av8n.com/physics/area-volume.pdf}{Area and Volume.}
\item
\href{http://www.av8n.com/physics/rotations.pdf}{Rotations.}
\end{itemize}
(have not read all these yet).


Richard E. Harke,
\href{http://www.harke.org/ps/intro.ps.gz}{An Introduction to the Mathematics of the Space-Time Algebra}

This is a nice complete little doc (\~40 pages), where many basic GA constructs are developed axiomatically with associated proofs.  This includes some simplical calculus and outermorphism content, and eventually moves on to STA and Lorentz rotations.



\section{Misc}

\begin{itemize}
\item
A blog like
\href{http://gaupdate.wordpress.com/}{subscription service that carries abstracts}
for various papers on or using Geometric Algebra.
\end{itemize}

\section{Collections of other GA bookmarks}

\begin{itemize}
\item
\href{http://www.geomerics.com/geometric-algebra.htm}{Geomerics.  Graphics software for Games, Geometric Algebra references and description.}
\item
\href{http://www.xtec.es/~rgonzal1/links.htm}{Ramon Gonz�lez Calvet us GA links.}
\item
\href{http://www.rwgrayprojects.com/GeometricAlgebra/references.html}{R. W. Gray's GA links.}
\item
\href{http://www.mrao.cam.ac.uk/~clifford/pages/links.htm}{Cambridge groups GA urls.}
\end{itemize}

\section{Exterior Algebra and differential forms}

\begin{itemize}
\item
\href{http://www.grassmannalgebra.info/grassmannalgebra/book/index.htm}{Grassmann Algebra Book}

Pdf files of a book draft entitled Grassmann Algebra: Exploring applications of extended vector algebra with Mathematica.

This has some useful info.  In particular, a great example of solving linear systems with the wedge product.
\item
The Cornell Library Historical Mathematics Monographs -
\href{http://historical.library.cornell.edu/cgi-bin/cul.math/docviewer?did=00540001&seq=15&frames=0&view=50}{hyde on grassman}
\item
\href{http://www.math.boun.edu.tr/instructors/ozturk/eskiders/fall04math488/bachman.pdf}{A Geometric Approach to Differential Forms by David Bachman}
\end{itemize}

\section{Software}

\begin{itemize}
\item
\href{http://staff.science.uva.nl/~fontijne/gaigen2.html}{Gaigen 2}
\item
\href{http://users.tkk.fi/~ppuska/mirror/Lounesto/CLICAL.htm}{CLICAL for Clifford Algebra Calculations}
\item
\href{http://www.nklein.com/products/geoma/}{nklein software.  Geoma.}
\end{itemize}


% END INCLUDES.
%-------------------------------------------------------

% from the template:

%\begin{thebibliography}{99}
%  \addcontentsline{toc}{chapter}{Bibliography}
%\bibitem{lamport} L. Lamport. {\bf \LaTeX \ A Document Preparation System}
%Addison-Wesley, California 1986.
%
%\end{thebibliography}

%\bibliographystyle{plainnat}
\bibliographystyle{unsrturl}
  \addcontentsline{toc}{chapter}{Bibliography}
\bibliography{myrefs}

%Note the tag used to make an index entry. You may need to consult Lamport's
%book~\cite{lamport} for details of the procedure to make the index input
%file; \LaTeX \ will create a pre-index by listing all the tagged
%items in the file {\tt bookex.idx} then you edit this into
%a {\tt theindex} environment, as {\tt index.tex}.

%\documentclass[openany]{memoir}
\usepackage[]{makeidx}

\chapterstyle{ell}

\makeindex

\begin{document}

To solve various problems in physics, it can be advantageous
to express any arbitrary piecewise-smooth function as a Fourier Series
\index{Fourier Series}
composed of multiples of sine \index{sine} and cosine \index{cosine} functions.  These are used in \cite{acheson1990elementary}.

\printindex

\bibliography{myrefs}
\bibliographystyle{unsrturl}

\end{document}

%  \addcontentsline{toc}{chapter}{Index}

\end{document}
