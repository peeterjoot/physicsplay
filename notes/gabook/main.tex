
\documentclass[12pt,leqno]{book}
\usepackage{amsmath,amssymb,amsfonts} % Typical maths resource packages
\usepackage{graphics}                 % Packages to allow inclusion of graphics
\usepackage{color}                    % For creating coloured text and background
\usepackage{hyperref}                 % For creating hyperlinks in cross references

\parindent 1cm
\parskip 0.2cm
\topmargin 0.2cm
\oddsidemargin 1cm
\evensidemargin 0.5cm
\textwidth 15cm
\textheight 21cm
\newtheorem{theorem}{Theorem}[section]
\newtheorem{proposition}[theorem]{Proposition}
\newtheorem{corollary}[theorem]{Corollary}
\newtheorem{lemma}[theorem]{Lemma}
\newtheorem{remark}[theorem]{Remark}
\newtheorem{definition}[theorem]{Definition}


\def\R{\mathbb{ R}}
\def\S{\mathbb{ S}}
\def\I{\mathbb{ I}}
\makeindex


\title{A \LaTeX \ Book Skeleton  }

\author{C.T.J.Dodson  \\
{\small\em \copyright \  Draft date \today }}

 \date{ }
\begin{document}
\maketitle
 \addcontentsline{toc}{chapter}{Contents}
\pagenumbering{roman}
\tableofcontents
\listoffigures
\listoftables
\chapter*{Preface}\normalsize
  \addcontentsline{toc}{chapter}{Preface}
\pagestyle{plain}
The book root file {\tt bookex.tex} gives a basic example of how to
use \LaTeX \ for preparation of a book. Note that all
\LaTeX \ commands begin with a
backslash.

Each
Chapter, Appendix and the Index is made as a {\tt *.tex} file and is
called in by the {\tt include} command---thus {\tt ch1.tex} is
the name here of the file containing Chapter~1. The inclusion of any
particular file can be suppressed by prefixing the line by a
percent sign.


 Do not put an {\tt end{document}} command at the end of chapter files;
just one such command is needed at the end of the book.

Note the tag used to make an index entry. You may need to consult Lamport's
book~\cite{lamport} for details of the procedure to make the index input
file; \LaTeX \ will create a pre-index by listing all the tagged
items in the file {\tt bookex.idx} then you edit this into
a {\tt theindex} environment, as {\tt index.tex}.





\pagestyle{headings}
\pagenumbering{arabic}

\include{ch1}
\include{ch2}

\begin{thebibliography}{99}
  \addcontentsline{toc}{chapter}{Bibliography}
\bibitem{lamport} L. Lamport. {\bf \LaTeX \ A Document Preparation System}
Addison-Wesley, California 1986.
\end{thebibliography}

\documentclass[openany]{memoir}
\usepackage[]{makeidx}

\chapterstyle{ell}

\makeindex

\begin{document}

To solve various problems in physics, it can be advantageous
to express any arbitrary piecewise-smooth function as a Fourier Series
\index{Fourier Series}
composed of multiples of sine \index{sine} and cosine \index{cosine} functions.  These are used in \cite{acheson1990elementary}.

\printindex

\bibliography{myrefs}
\bibliographystyle{unsrturl}

\end{document}

  \addcontentsline{toc}{chapter}{Index}
\end{document}
