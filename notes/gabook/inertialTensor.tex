\documentclass{article}      % Specifies the document class

\usepackage{amsmath}
\usepackage{mathpazo}

%
% shorthand for bold symbols, convenient for vectors and matrices
%
\newcommand{\Ba}[0]{\mathbf{a}}
\newcommand{\Bb}[0]{\mathbf{b}}
\newcommand{\Bc}[0]{\mathbf{c}}
\newcommand{\Bd}[0]{\mathbf{d}}
\newcommand{\Be}[0]{\mathbf{e}}
\newcommand{\Bf}[0]{\mathbf{f}}
\newcommand{\Bg}[0]{\mathbf{g}}
\newcommand{\Bh}[0]{\mathbf{h}}
\newcommand{\Bi}[0]{\mathbf{i}}
\newcommand{\Bj}[0]{\mathbf{j}}
\newcommand{\Bk}[0]{\mathbf{k}}
\newcommand{\Bl}[0]{\mathbf{l}}
\newcommand{\Bm}[0]{\mathbf{m}}
\newcommand{\Bn}[0]{\mathbf{n}}
\newcommand{\Bo}[0]{\mathbf{o}}
\newcommand{\Bp}[0]{\mathbf{p}}
\newcommand{\Bq}[0]{\mathbf{q}}
\newcommand{\Br}[0]{\mathbf{r}}
\newcommand{\Bs}[0]{\mathbf{s}}
\newcommand{\Bt}[0]{\mathbf{t}}
\newcommand{\Bu}[0]{\mathbf{u}}
\newcommand{\Bv}[0]{\mathbf{v}}
\newcommand{\Bw}[0]{\mathbf{w}}
\newcommand{\Bx}[0]{\mathbf{x}}
\newcommand{\By}[0]{\mathbf{y}}
\newcommand{\Bz}[0]{\mathbf{z}}
\newcommand{\BA}[0]{\mathbf{A}}
\newcommand{\BB}[0]{\mathbf{B}}
\newcommand{\BC}[0]{\mathbf{C}}
\newcommand{\BD}[0]{\mathbf{D}}
\newcommand{\BE}[0]{\mathbf{E}}
\newcommand{\BF}[0]{\mathbf{F}}
\newcommand{\BG}[0]{\mathbf{G}}
\newcommand{\BH}[0]{\mathbf{H}}
\newcommand{\BI}[0]{\mathbf{I}}
\newcommand{\BJ}[0]{\mathbf{J}}
\newcommand{\BK}[0]{\mathbf{K}}
\newcommand{\BL}[0]{\mathbf{L}}
\newcommand{\BM}[0]{\mathbf{M}}
\newcommand{\BN}[0]{\mathbf{N}}
\newcommand{\BO}[0]{\mathbf{O}}
\newcommand{\BP}[0]{\mathbf{P}}
\newcommand{\BQ}[0]{\mathbf{Q}}
\newcommand{\BR}[0]{\mathbf{R}}
\newcommand{\BS}[0]{\mathbf{S}}
\newcommand{\BT}[0]{\mathbf{T}}
\newcommand{\BU}[0]{\mathbf{U}}
\newcommand{\BV}[0]{\mathbf{V}}
\newcommand{\BW}[0]{\mathbf{W}}
\newcommand{\BX}[0]{\mathbf{X}}
\newcommand{\BY}[0]{\mathbf{Y}}
\newcommand{\BZ}[0]{\mathbf{Z}}

\newcommand{\Bzero}[0]{\mathbf{0}}
\newcommand{\Btheta}[0]{\boldsymbol{\theta}}
\newcommand{\Btau}[0]{\boldsymbol{\tau}}
\newcommand{\Bomega}[0]{\boldsymbol{\omega}}

%
% shorthand for unit vectors
%
\newcommand{\acap}[0]{\hat{\Ba}}
\newcommand{\bcap}[0]{\hat{\Bb}}
\newcommand{\ccap}[0]{\hat{\Bc}}
\newcommand{\dcap}[0]{\hat{\Bd}}
\newcommand{\ecap}[0]{\hat{\Be}}
\newcommand{\fcap}[0]{\hat{\Bf}}
\newcommand{\gcap}[0]{\hat{\Bg}}
\newcommand{\hcap}[0]{\hat{\Bh}}
\newcommand{\icap}[0]{\hat{\Bi}}
\newcommand{\jcap}[0]{\hat{\Bj}}
\newcommand{\kcap}[0]{\hat{\Bk}}
\newcommand{\lcap}[0]{\hat{\Bl}}
\newcommand{\mcap}[0]{\hat{\Bm}}
\newcommand{\ncap}[0]{\hat{\Bn}}
\newcommand{\ocap}[0]{\hat{\Bo}}
\newcommand{\pcap}[0]{\hat{\Bp}}
\newcommand{\qcap}[0]{\hat{\Bq}}
\newcommand{\rcap}[0]{\hat{\Br}}
\newcommand{\scap}[0]{\hat{\Bs}}
\newcommand{\tcap}[0]{\hat{\Bt}}
\newcommand{\ucap}[0]{\hat{\Bu}}
\newcommand{\vcap}[0]{\hat{\Bv}}
\newcommand{\wcap}[0]{\hat{\Bw}}
\newcommand{\xcap}[0]{\hat{\Bx}}
\newcommand{\ycap}[0]{\hat{\By}}
\newcommand{\zcap}[0]{\hat{\Bz}}
\newcommand{\thetacap}[0]{\hat{\Btheta}}

%
% to write R^n and C^n in a distinguishable fashion.  Perhaps change this
% to the double lined characters upon figuring out how to do so.
%
\newcommand{\C}[1]{$\mathbb{C}^{#1}$}
\newcommand{\R}[1]{$\mathbb{R}^{#1}$}

%
% various generally useful helpers
%

% derivative of #1 wrt. #2:
\newcommand{\D}[2] {\frac {d#2} {d#1}}

\newcommand{\inv}[1]{\frac{1}{#1}}
\newcommand{\cross}[0]{\times}

\newcommand{\abs}[1]{\lvert{#1}\rvert}
\newcommand{\norm}[1]{\lVert{#1}\rVert}
\newcommand{\innerprod}[2]{\langle{#1}, {#2}\rangle}
\newcommand{\dotprod}[2]{{#1} \cdot {#2}}
\newcommand{\bdotprod}[2]{\left({#1} \cdot {#2}\right)}
\newcommand{\crossprod}[2]{{#1} \cross {#2}}
\newcommand{\tripleprod}[3]{\dotprod{\left(\crossprod{#1}{#2}\right)}{#3}}

\DeclareMathOperator{\Proj}{Proj}
\DeclareMathOperator{\Span}{span}
\DeclareMathOperator{\Sgn}{sgn}
\DeclareMathOperator{\Area}{Area}
\DeclareMathOperator{\Volume}{Volume}

%
% A few miscellaneous things specific to this document
%
\newcommand{\crossop}[1]{\crossprod{#1}{}}

% R2 vector.
\newcommand{\VectorTwo}[2]{
\begin{bmatrix}
 {#1} \\
 {#2}
\end{bmatrix}
}

\newcommand{\VectorN}[1]{
\begin{bmatrix}
{#1}_1 \\
{#1}_2 \\
\vdots \\
{#1}_N \\
\end{bmatrix}
}

\newcommand{\DETuvij}[4]{
\begin{vmatrix}
 {#1}_{#3} & {#1}_{#4} \\
 {#2}_{#3} & {#2}_{#4}
\end{vmatrix}
}

\newcommand{\DETuvwijk}[6]{
\begin{vmatrix}
 {#1}_{#4} & {#1}_{#5} & {#1}_{#6} \\
 {#2}_{#4} & {#2}_{#5} & {#2}_{#6} \\
 {#3}_{#4} & {#3}_{#5} & {#3}_{#6}
\end{vmatrix}
}

\newcommand{\DETuvwxijkl}[8]{
\begin{vmatrix}
 {#1}_{#5} & {#1}_{#6} & {#1}_{#7} & {#1}_{#8} \\
 {#2}_{#5} & {#2}_{#6} & {#2}_{#7} & {#2}_{#8} \\
 {#3}_{#5} & {#3}_{#6} & {#3}_{#7} & {#3}_{#8} \\
 {#4}_{#5} & {#4}_{#6} & {#4}_{#7} & {#4}_{#8} \\
\end{vmatrix}
}

%\newcommand{\DETuvwxyijklm}[10]{
%\begin{vmatrix}
% {#1}_{#6} & {#1}_{#7} & {#1}_{#8} & {#1}_{#9} & {#1}_{#10} \\
% {#2}_{#6} & {#2}_{#7} & {#2}_{#8} & {#2}_{#9} & {#2}_{#10} \\
% {#3}_{#6} & {#3}_{#7} & {#3}_{#8} & {#3}_{#9} & {#3}_{#10} \\
% {#4}_{#6} & {#4}_{#7} & {#4}_{#8} & {#4}_{#9} & {#4}_{#10} \\
% {#5}_{#6} & {#5}_{#7} & {#5}_{#8} & {#5}_{#9} & {#5}_{#10}
%\end{vmatrix}
%}

% R3 vector.
\newcommand{\VectorThree}[3]{
\begin{bmatrix}
 {#1} \\
 {#2} \\
 {#3}
\end{bmatrix}
}



%
% The real thing:
%

                             % The preamble begins here.
\title{Inertia Tensor} % Declares the document's title.
\author{Peeter Joot}         % Declares the author's name.
%\date{}        % Deleting this command produces today's date.

\begin{document}             % End of preamble and beginning of text.

\maketitle{}

\section{}

GAFP derives the angular momentum for rotational motion in the following form

\[
L = R \left( \int \Bx \wedge (\Bx \cdot \Omega_B) dm \right) R^\dagger
\]

and calls the integral part, the inertia tensor

\[
\emph{I}(B) = \int \Bx \wedge (\Bx \cdot \Omega_B) dm
\]

which is a linear mapping from bivectors to bivectors.  To understand the
form of this I found it helpful to expanding the wedge product
part of this explicitly for the \R{3} case.

Ignoring the sum in this expansion write

\[
f(B) = \Bx \wedge (\Bx \cdot B)
\]

And writing $\Be_{ij} = \Be_i \Be_j$ introduce a basis

\[
b = \{ \Be_1 I, \Be_2 I, \Be_3 I \} = \{ \Be_{23}, \Be_{31}, \Be_{12} \}
\]

for the \R{3} bivector product space.

Now calculate $f(B)$ for each of the basis vectors

\begin{align*}
f(\Be_1 I) 
&= \Bx \wedge (\Bx \cdot \Be_{23}) \\
&= ( x_1 \Be_1 + x_2 \Be_2 + x_3 \Be_3) \wedge (x_2 \Be_3 - x_3 \Be_2) \\
\end{align*}

Completing this calculation for each of the unit basic bivectors, we have
%%+ (           + x_2 \Be_2            ) \wedge (x_2 \Be_3            ) \\
%%+ (                       + x_3 \Be_3) \wedge (          - x_3 \Be_2) \\
%+ ( x_1 \Be_1                        ) \wedge (x_2 \Be_3            ) \\
%+ ( x_1 \Be_1                        ) \wedge (          - x_3 \Be_2) \\
%%+ (           + x_2 \Be_2            ) \wedge (          - x_3 \Be_2) \\ %% = 0
%%+ (                       + x_3 \Be_3) \wedge (x_2 \Be_3            ) \\ %% = 0
\begin{align*}
f(\Be_1 I) &= (x_2^2 + x_3^2) \Be_{23} - (x_1 x_2) \Be_{31} - (x_1 x_3) \Be_{12} \\
f(\Be_2 I) &= -(x_1 x_2) \Be_{23} + (x_1^2 + x_3^2) \Be_{31} - (x_2 x_3) \Be_{12} \\
f(\Be_3 I) &= -(x_1 x_3) \Be_{23} - (x_2 x_3) \Be_{31} - (x_1^2 x_2^2) \Be_{12} \\
\end{align*}

Observe that taking dot products with $(\Be_i I)^\dagger$ will select just the $\Be_i I$ term of the result, so one can
form the matrix of this linear transformation that maps bivectors in basis $b$ to image vectors also in basis $b$ as follows

\[
{\begin{bmatrix}
\emph{I}(B)
\end{bmatrix}}_{b}^{b}
=
{\begin{bmatrix}
\emph{I}(\Be_i I) \cdot (\Be_j I)^\dagger
\end{bmatrix}}_{ij}
=
\int {\begin{bmatrix}
(x_2^2 + x_3^2)  & - (x_1 x_2)  & - (x_1 x_3)  \\
-(x_1 x_2)  & + (x_1^2 + x_3^2)  & - (x_2 x_3)  \\
-(x_1 x_3)  & - (x_2 x_3)  & - (x_1^2 x_2^2)  \\
\end{bmatrix}} dm
\]

Observe that this can also be written in a more typical tensor notation

\[
I_{ij} 
= \emph{I}(\Be_i I) \cdot (\Be_j I)^\dagger 
= \int (\delta_{ij} \Bx^2 - x_i x_j) dm
\]

Where, as usual for tensors, the meaning of the indexes and whether summation is required is implied.  In this case
the coordinate transformation matrix for this linear transformation has components $I_{ij}$ (and no summation).

\subsection{ coordinate transformation matrix for a couple other linear transformations }

Seeing a function of a bivector for the first time is kind of intriging.  We can form the matrix of such a linear transformation
from a basis of the bivector space to the space spanned by function.  For fun, let's calculate that matrix for the basis $b$ above
for the following function:

\[
f(B) = \Be_1 \wedge (\Be_2 \cdot B)
\]

For this function operating on \R{3} bivectors we have:

\begin{align*}
f(\Be_{23}) &= \Be_1 \wedge (\Be_2 \cdot \Be_{23}) = -\Be_{31} \\
f(\Be_{31}) &= \Be_1 \wedge (\Be_2 \cdot \Be_{31}) = 0 \\
f(\Be_{12}) &= \Be_1 \wedge (\Be_2 \cdot \Be_{12}) = 0 \\
\end{align*}

So

\[
{\begin{bmatrix}
f
\end{bmatrix}}_b^b
= 
\begin{bmatrix}
-1 & 0 & 0
\end{bmatrix}
\]

For \R{4} one orthonormal basis is

\[
b = \{ 
\Be_{12},
\Be_{13},
\Be_{14},
\Be_{23},
\Be_{24},
\Be_{34}
\}
\]

A basis for the span of $f$ is $b' = \{
\Be_{13},
\Be_{14}
\}$.  Like any other coordinate transformation associated with a linear transformation we can write the matrix of the transformation that
takes a coordinate vector in one basis into a coordinate vector for the basis for the image:

\[
{\begin{bmatrix}
f(x)
\end{bmatrix}}_{b'}
=
{\begin{bmatrix}
f
\end{bmatrix}}_b^{b'}
{\begin{bmatrix}
x
\end{bmatrix}}_{b}
\] 

For this function $f$ and these pair of basis bivectors we have:

\[
{\begin{bmatrix}
f
\end{bmatrix}}_b^{b'}
= 
\begin{bmatrix}
0 & 0 & 0 & 1 & 0 & 0 \\
0 & 0 & 0 & 0 & 1 & 0 \\
\end{bmatrix}
\]

\subsection{ Equation 3.126 details. }

This statement from GAFP deserves expansion (or at least an exersize):

\[
A \cdot ( \Bx \wedge (\Bx \cdot B) )
= \langle{A \Bx (\Bx \cdot B)}\rangle
= \langle (A \cdot \Bx) \Bx B \rangle
= B \cdot ( \Bx \wedge (\Bx \cdot A) )
\]

Perhaps this is obvious to the author, but wasn't to me.  To clarify this observe the following product

\[
\Bx ( \Bx \cdot B ) = \Bx \cdot ( \Bx \cdot B ) + \Bx \wedge ( \Bx \cdot B ) 
\]

By writing $B = \Bb \Bc = \Bb \wedge \Bc$ we can show that the dot product part of this product is zero:

\begin{align*}
\Bx \cdot ( \Bx \cdot B ) 
&= \Bx \cdot ( (\Bx \cdot \Bb) \Bc - (\Bx \cdot \Bc) \Bb ) \\
&= (\Bx \cdot \Bc) (\Bx \cdot \Bb) - (\Bx \cdot \Bb) (\Bx \cdot \Bc)) \\
&= 0
\end{align*}

This provides the justification for the wedge product removal in the text, since
one can write

\begin{equation}
\Bx \wedge ( \Bx \cdot B ) = \Bx ( \Bx \cdot B )
\end{equation}\label{eqn:inertia_wedge_to_product}

Although it wasn't stated in the text (\ref{eqn:inertia_wedge_to_product}), can
be used to put this inertia product in a pure dot product form

\begin{align*}
A^\dagger \cdot (\Bx \wedge (\Bx \cdot B) )
&= -\langle {A \Bx (\Bx \cdot B)} \rangle \\
&= -\langle (A \cdot \Bx + A \wedge \Bx)(\Bx \cdot B) \rangle \\
\end{align*}

The trivector vector product has only vector and trivector components
\[
(A \wedge \Bx)(\Bx \cdot B) = \langle{ (A \wedge \Bx)(\Bx \cdot B)}\rangle_1 + \langle{(A \wedge \Bx)(\Bx \cdot B)}\rangle_3
\]

So $\langle{(A \wedge \Bx)(\Bx \cdot B)}\rangle_0 = 0$, and one can write

\begin{align*}
A^\dagger \cdot (\Bx \wedge (\Bx \cdot B) )
&= - (A \cdot \Bx ) \cdot (\Bx \cdot B) \\
&= (\Bx \cdot A) \cdot (\Bx \cdot B) \\
\end{align*}

As pointed out in the text this is symetric.  That can't be more clear than above.

\end{document}               % End of document.
