\documentclass[]{eliblog}

\usepackage{amsmath}
\usepackage{mathpazo}

%
% shorthand for bold symbols, convenient for vectors and matrices
%
\newcommand{\Ba}[0]{\mathbf{a}}
\newcommand{\Bb}[0]{\mathbf{b}}
\newcommand{\Bc}[0]{\mathbf{c}}
\newcommand{\Bd}[0]{\mathbf{d}}
\newcommand{\Be}[0]{\mathbf{e}}
\newcommand{\Bf}[0]{\mathbf{f}}
\newcommand{\Bg}[0]{\mathbf{g}}
\newcommand{\Bh}[0]{\mathbf{h}}
\newcommand{\Bi}[0]{\mathbf{i}}
\newcommand{\Bj}[0]{\mathbf{j}}
\newcommand{\Bk}[0]{\mathbf{k}}
\newcommand{\Bl}[0]{\mathbf{l}}
\newcommand{\Bm}[0]{\mathbf{m}}
\newcommand{\Bn}[0]{\mathbf{n}}
\newcommand{\Bo}[0]{\mathbf{o}}
\newcommand{\Bp}[0]{\mathbf{p}}
\newcommand{\Bq}[0]{\mathbf{q}}
\newcommand{\Br}[0]{\mathbf{r}}
\newcommand{\Bs}[0]{\mathbf{s}}
\newcommand{\Bt}[0]{\mathbf{t}}
\newcommand{\Bu}[0]{\mathbf{u}}
\newcommand{\Bv}[0]{\mathbf{v}}
\newcommand{\Bw}[0]{\mathbf{w}}
\newcommand{\Bx}[0]{\mathbf{x}}
\newcommand{\By}[0]{\mathbf{y}}
\newcommand{\Bz}[0]{\mathbf{z}}
\newcommand{\BA}[0]{\mathbf{A}}
\newcommand{\BB}[0]{\mathbf{B}}
\newcommand{\BC}[0]{\mathbf{C}}
\newcommand{\BD}[0]{\mathbf{D}}
\newcommand{\BE}[0]{\mathbf{E}}
\newcommand{\BF}[0]{\mathbf{F}}
\newcommand{\BG}[0]{\mathbf{G}}
\newcommand{\BH}[0]{\mathbf{H}}
\newcommand{\BI}[0]{\mathbf{I}}
\newcommand{\BJ}[0]{\mathbf{J}}
\newcommand{\BK}[0]{\mathbf{K}}
\newcommand{\BL}[0]{\mathbf{L}}
\newcommand{\BM}[0]{\mathbf{M}}
\newcommand{\BN}[0]{\mathbf{N}}
\newcommand{\BO}[0]{\mathbf{O}}
\newcommand{\BP}[0]{\mathbf{P}}
\newcommand{\BQ}[0]{\mathbf{Q}}
\newcommand{\BR}[0]{\mathbf{R}}
\newcommand{\BS}[0]{\mathbf{S}}
\newcommand{\BT}[0]{\mathbf{T}}
\newcommand{\BU}[0]{\mathbf{U}}
\newcommand{\BV}[0]{\mathbf{V}}
\newcommand{\BW}[0]{\mathbf{W}}
\newcommand{\BX}[0]{\mathbf{X}}
\newcommand{\BY}[0]{\mathbf{Y}}
\newcommand{\BZ}[0]{\mathbf{Z}}

\newcommand{\Bzero}[0]{\mathbf{0}}
\newcommand{\Btheta}[0]{\boldsymbol{\theta}}
\newcommand{\Btau}[0]{\boldsymbol{\tau}}
\newcommand{\Bomega}[0]{\boldsymbol{\omega}}

%
% shorthand for unit vectors
%
\newcommand{\acap}[0]{\hat{\Ba}}
\newcommand{\bcap}[0]{\hat{\Bb}}
\newcommand{\ccap}[0]{\hat{\Bc}}
\newcommand{\dcap}[0]{\hat{\Bd}}
\newcommand{\ecap}[0]{\hat{\Be}}
\newcommand{\fcap}[0]{\hat{\Bf}}
\newcommand{\gcap}[0]{\hat{\Bg}}
\newcommand{\hcap}[0]{\hat{\Bh}}
\newcommand{\icap}[0]{\hat{\Bi}}
\newcommand{\jcap}[0]{\hat{\Bj}}
\newcommand{\kcap}[0]{\hat{\Bk}}
\newcommand{\lcap}[0]{\hat{\Bl}}
\newcommand{\mcap}[0]{\hat{\Bm}}
\newcommand{\ncap}[0]{\hat{\Bn}}
\newcommand{\ocap}[0]{\hat{\Bo}}
\newcommand{\pcap}[0]{\hat{\Bp}}
\newcommand{\qcap}[0]{\hat{\Bq}}
\newcommand{\rcap}[0]{\hat{\Br}}
\newcommand{\scap}[0]{\hat{\Bs}}
\newcommand{\tcap}[0]{\hat{\Bt}}
\newcommand{\ucap}[0]{\hat{\Bu}}
\newcommand{\vcap}[0]{\hat{\Bv}}
\newcommand{\wcap}[0]{\hat{\Bw}}
\newcommand{\xcap}[0]{\hat{\Bx}}
\newcommand{\ycap}[0]{\hat{\By}}
\newcommand{\zcap}[0]{\hat{\Bz}}
\newcommand{\thetacap}[0]{\hat{\Btheta}}

%
% to write R^n and C^n in a distinguishable fashion.  Perhaps change this
% to the double lined characters upon figuring out how to do so.
%
\newcommand{\C}[1]{$\mathbb{C}^{#1}$}
\newcommand{\R}[1]{$\mathbb{R}^{#1}$}

%
% various generally useful helpers
%

% derivative of #1 wrt. #2:
\newcommand{\D}[2] {\frac {d#2} {d#1}}

\newcommand{\inv}[1]{\frac{1}{#1}}
\newcommand{\cross}[0]{\times}

\newcommand{\abs}[1]{\lvert{#1}\rvert}
\newcommand{\norm}[1]{\lVert{#1}\rVert}
\newcommand{\innerprod}[2]{\langle{#1}, {#2}\rangle}
\newcommand{\dotprod}[2]{{#1} \cdot {#2}}
\newcommand{\bdotprod}[2]{\left({#1} \cdot {#2}\right)}
\newcommand{\crossprod}[2]{{#1} \cross {#2}}
\newcommand{\tripleprod}[3]{\dotprod{\left(\crossprod{#1}{#2}\right)}{#3}}

\DeclareMathOperator{\Proj}{Proj}
\DeclareMathOperator{\Span}{span}
\DeclareMathOperator{\Sgn}{sgn}
\DeclareMathOperator{\Area}{Area}
\DeclareMathOperator{\Volume}{Volume}

%
% A few miscellaneous things specific to this document
%
\newcommand{\crossop}[1]{\crossprod{#1}{}}

% R2 vector.
\newcommand{\VectorTwo}[2]{
\begin{bmatrix}
 {#1} \\
 {#2}
\end{bmatrix}
}

\newcommand{\VectorN}[1]{
\begin{bmatrix}
{#1}_1 \\
{#1}_2 \\
\vdots \\
{#1}_N \\
\end{bmatrix}
}

\newcommand{\DETuvij}[4]{
\begin{vmatrix}
 {#1}_{#3} & {#1}_{#4} \\
 {#2}_{#3} & {#2}_{#4}
\end{vmatrix}
}

\newcommand{\DETuvwijk}[6]{
\begin{vmatrix}
 {#1}_{#4} & {#1}_{#5} & {#1}_{#6} \\
 {#2}_{#4} & {#2}_{#5} & {#2}_{#6} \\
 {#3}_{#4} & {#3}_{#5} & {#3}_{#6}
\end{vmatrix}
}

\newcommand{\DETuvwxijkl}[8]{
\begin{vmatrix}
 {#1}_{#5} & {#1}_{#6} & {#1}_{#7} & {#1}_{#8} \\
 {#2}_{#5} & {#2}_{#6} & {#2}_{#7} & {#2}_{#8} \\
 {#3}_{#5} & {#3}_{#6} & {#3}_{#7} & {#3}_{#8} \\
 {#4}_{#5} & {#4}_{#6} & {#4}_{#7} & {#4}_{#8} \\
\end{vmatrix}
}

%\newcommand{\DETuvwxyijklm}[10]{
%\begin{vmatrix}
% {#1}_{#6} & {#1}_{#7} & {#1}_{#8} & {#1}_{#9} & {#1}_{#10} \\
% {#2}_{#6} & {#2}_{#7} & {#2}_{#8} & {#2}_{#9} & {#2}_{#10} \\
% {#3}_{#6} & {#3}_{#7} & {#3}_{#8} & {#3}_{#9} & {#3}_{#10} \\
% {#4}_{#6} & {#4}_{#7} & {#4}_{#8} & {#4}_{#9} & {#4}_{#10} \\
% {#5}_{#6} & {#5}_{#7} & {#5}_{#8} & {#5}_{#9} & {#5}_{#10}
%\end{vmatrix}
%}

% R3 vector.
\newcommand{\VectorThree}[3]{
\begin{bmatrix}
 {#1} \\
 {#2} \\
 {#3}
\end{bmatrix}
}



\author{Peeter Joot}
\email{peeter.joot@gmail.com}


\chapter{Covariant Maxwell equation in media}
\label{chap:covariantMedia}
%\useCCL
\blogpage{http://sites.google.com/site/peeterjoot/math2009/covariantMedia.pdf}
\date{Aug 10, 2009}
\revisionInfo{$RCSfile: covariantMedia.tex,v $ Last $Revision: 1.2 $ $Date: 2009/08/10 21:08:07 $}

\beginArtWithToc

\section{Motivation, some notation, and review.}

Adjusting to Jackson's of CGS (\cite{jackson1975cew}) takes some work.  A first pass at a GA form was assembled in (\cite{macroscopicMaxwell}), based on what was in the introduction chapter for media that includes $\BP$, and $\BM$ properties.  He later changes conventions, and also assumes linear media in most cases, so we want something different than what was previously derived.

The non-covariant form of Maxwell's equation in abscence of current and charge has been convient to use in some initial attempts to look at wave propagation.  That was

\begin{align}\label{eqn:foo1}
F &= \BE + I\BB/\sqrt{\mu\epsilon} \\
0 &= (\spacegrad + \sqrt{\mu\epsilon} \partial_0) F
\end{align}

To examine the energy momentum tensor, it is desirable to express this in a fashion that has no such explicit spacetime dependence.  This suggests a spacetime gradient definition that varies throughout the media.

\begin{align}\label{eqn:foo2}
\grad \equiv \gamma^m \partial_m + \sqrt{\mu\epsilon} \gamma^0 \partial_0
\end{align}

Observe that this spacetime gradient is adjusted by the speed of light in the media, and isn't one that is naturally relativistic.  Even though the differential form of Maxwell's equation is implicitly defined only in a neighbourhood of the point it is evaluated at, we now have a reason to say this explicitly, because this non-isotropic condition is now hiding in the (perhaps poor) notation for the operator.  Ignoring the obscuring nature of this operator, and working with it, we can can that Maxwell's equation in the neighbourhood (where $\mu\epsilon$ is ``fixed'') is

\begin{align}\label{eqn:foo3}
\grad F = 0
\end{align}

We also want a varient of this that includes the charge and current terms.

\section{Linear media.}

Lets pick Jackson's equation (6.70) as the starting point.  A partial translation to GA form, with $\BD = \epsilon \BE$, and $\BB = \mu \BH$, and $\partial_0 = \partial/\partial ct$ is

\begin{align}\label{eqn:foo4}
\spacegrad \cdot \BB &= 0 \\
\spacegrad \cdot \epsilon \BE &= 4 \pi \rho \\
-I \spacegrad \wedge \BE + \partial_0 \BB &= 0 \\
-I \spacegrad \wedge \BB/\mu - \partial_0 \epsilon \BE &= \frac{4 \pi}{c} \BJ
\end{align}

Scaling and adding we have

\begin{align}\label{eqn:foo5}
\spacegrad \BE + \partial_0 I \BB &= \frac{4 \pi \rho}{\epsilon} \\
\spacegrad \BB - I \partial_0 \mu \epsilon \BE &= \frac{4 \pi \mu I}{c} \BJ
\end{align}

Once last scaling prepares for addition of these last two equations

\begin{align}\label{eqn:foo6}
\spacegrad \BE + \sqrt{\mu\epsilon}\partial_0 I \BB/\sqrt{\mu\epsilon} &= \frac{4 \pi \rho}{\epsilon} \\
\spacegrad I \BB/\sqrt{\mu\epsilon} + \partial_0 \sqrt{\mu \epsilon} \BE &= -\frac{4 \pi \mu }{c\sqrt{\mu\epsilon}}\BJ
\end{align}

This gives us a non-covariant assembly of Maxwell's equations in linear media

\begin{align}\label{eqn:foo7}
(\spacegrad + \sqrt{\mu\epsilon}\partial_0) F &= \frac{4 \pi}{c} \left( \frac{c \rho}{\epsilon} - \sqrt{\frac{\mu}{\epsilon}} \BJ \right)
\end{align}

Premultiplication by $\gamma_0$, and utilizing the definition of (\ref{eqn:foo2}) we have

\begin{align}\label{eqn:foo8}
\grad F &= \frac{4 \pi}{c} \left( c \frac{\rho}{\epsilon} \gamma_0 + \sqrt{\frac{\mu}{\epsilon}} J^m \gamma_m \right)
\end{align}

We can then define

\begin{align}\label{eqn:foo9}
J \equiv \frac{c \rho}{\epsilon} \gamma_0 + \sqrt{\frac{\mu}{\epsilon}} J^m \gamma_m
\end{align}

and are left with an expression of Maxwell's equation that puts space and time on a similar footing.  It's probably not really right to call this a covariant expression since it isn't naturally relativistic.

\begin{align}\label{eqn:foo10}
\grad F &= \frac{4 \pi}{c} J
\end{align}

\section{Energy momentum tensor.}

My main goal was to find the GA form of the stress energy tensor in media.  With the requirement for both an alternate spacetime gradient and the inclusion of the scaling factors for the media it is not obviously clear to me how to do translate from the vacuum expression in SI units to the CGS in media form.  It makes sense to step back to see how the divergence conservation equation translates with both of these changes.  In SI units our tensor (a four vector parameterized by another direction vector $a$) was

\begin{align}\label{eqn:foo11}
T(a) \equiv \frac{-1}{2\epsilon_0} F a F
\end{align}

Ignoring units temporarily, let's calculate the media-spacetime divergence of $-FaF/2$.  That is

\begin{align*}
-\inv{2} \grad \cdot (FaF)
&=
-\inv{2} \gpgradezero{\grad (FaF)} \\
&=
-\inv{2} \gpgradezero{(F(\rgrad F) + (F\lgrad)F) a} \\
&=
-\frac{4\pi}{c} \gpgradezero{\inv{2}(F J - J F) a} \\
&=
-\frac{4\pi}{c} (F \cdot J) \cdot a \\
\end{align*}

We want the $T^{\mu 0}$ components of the tensor $T(\gamma_0)$.  Noting the anticommutation relation for the pseudoscalar $I \gamma_0 = -\gamma_0 I$, and the anticommutation behaviour for spatial vectors such as $\BE \gamma_0 = -\gamma_0$ we have

\begin{align*}
-\inv{2} (\BE + I\BB/\sqrt{\mu\epsilon}) \gamma_0 (\BE + I\BB/\sqrt{\mu\epsilon})
&=
\frac{\gamma_0}{2} (\BE - I\BB/\sqrt{\mu\epsilon}) (\BE + I\BB/\sqrt{\mu\epsilon}) \\
&=
\frac{\gamma_0}{2} \left( (\BE^2 + \BB^2/{\mu\epsilon}) + I \inv{\sqrt{\mu\epsilon}} (\BE\BB - \BB\BE) \right) \\
&=
\frac{1}{2} (\BE^2 + \BB^2/{\mu\epsilon}) + \gamma_0 I \inv{\sqrt{\mu\epsilon}} (\BE \wedge \BB) \\
&=
\frac{\gamma_0}{2} (\BE^2 + \BB^2/{\mu\epsilon}) - \gamma_0 \inv{\sqrt{\mu\epsilon}} (\BE \cross \BB) \\
&=
\frac{\gamma_0}{2} (\BE^2 + \BB^2/{\mu\epsilon}) - \gamma_0 \inv{\sqrt{\mu\epsilon}} \gamma_m \gamma_0 (\BE \cross \BB)^m \\
&=
\frac{\gamma_0}{2} (\BE^2 + \BB^2/{\mu\epsilon}) + \inv{\sqrt{\mu\epsilon}} \gamma_m (\BE \cross \BB)^m \\
\end{align*}

Calculating the divergence of this using the media spacetime gradient we have

\begin{align*}
\grad \cdot \left( -\inv{2} F \gamma_0 F \right)
&=
\frac{\sqrt{\mu\epsilon}}{c} \frac{\partial}{\partial t} \frac{1}{2} \left(\BE^2 + \inv{\mu\epsilon}\BB^2\right)
+ \sum_m
\frac{\partial}{\partial x^m} \left( \inv{\sqrt{\mu\epsilon}} (\BE \cross \BB)^m \right) \\
&=
\frac{\sqrt{\mu\epsilon}}{c} \frac{\partial}{\partial t} \frac{1}{2} \left(\BE^2 + \inv{\mu\epsilon}\BB^2 \right)
+ \spacegrad \cdot \left( \inv{\sqrt{\mu\epsilon}} (\BE \cross \BB)^m \right)
\end{align*}

Multiplying this by $(c/4\pi) \sqrt{\epsilon/\mu}$, we have

\begin{align*}
\grad \cdot \left( -\frac{c}{8 \pi} \sqrt{\frac{\epsilon}{\mu}} F \gamma_0 F \right)
&=
\frac{\partial}{\partial t} \frac{1}{2} \left(\BE \cdot \BD + \BB \cdot \BH \right) + \spacegrad \cdot \frac{c}{4\pi}(\BE \cross \BH) \\
&=
- \sqrt{\frac{\epsilon}{\mu}}(F \cdot J) \cdot \gamma_0 \\
\end{align*}

Now expand the RHS.  We have

\begin{align*}
\sqrt{\frac{\epsilon}{\mu}}
(F \cdot J) \cdot \gamma_0
&=
\left((\BE + I \BB/\sqrt{\mu\epsilon}) \cdot \left( \frac{\rho}{\sqrt{\mu\epsilon}} \gamma_0 + J^m \gamma_m \right) \right) \cdot \gamma_0  \\
&=
\gpgradezero{E^q \gamma_q \gamma_0 J^m \gamma_m \gamma_0} \\
&=
\BE \cdot \BJ
\end{align*}

Assembling results the energy conservation relation, first in covariant form is

\begin{align}\label{eqn:foo12}
\grad \cdot \left( -\frac{c}{8 \pi} \sqrt{\frac{\epsilon}{\mu}} F a F \right) &= - \sqrt{\frac{\epsilon}{\mu}}(F \cdot J) \cdot a
\end{align}

and the same with an explicit spacetime split in vector quantities is

\begin{align}\label{eqn:foo13}
\frac{\partial}{\partial t} \frac{1}{2} \left(\BE \cdot \BD + \BB \cdot \BH \right) + \spacegrad \cdot \frac{c}{4\pi}(\BE \cross \BH)
&=
-\BE \cdot \BJ
\end{align}

%It appears (6.106) the electrostatic energy density in CGS units is given by
%
%\begin{align}\label{eqn:foo1}
%\frac{dW}{dV} = \inv{8\pi} \BE \cdot \BD
%\end{align}

\EndArticle
