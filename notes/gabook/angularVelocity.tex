\documentclass{article}      % Specifies the document class

\usepackage{amsmath}
\usepackage{mathpazo}

%
% shorthand for bold symbols, convenient for vectors and matrices
%
\newcommand{\Ba}[0]{\mathbf{a}}
\newcommand{\Bb}[0]{\mathbf{b}}
\newcommand{\Bc}[0]{\mathbf{c}}
\newcommand{\Bd}[0]{\mathbf{d}}
\newcommand{\Be}[0]{\mathbf{e}}
\newcommand{\Bf}[0]{\mathbf{f}}
\newcommand{\Bg}[0]{\mathbf{g}}
\newcommand{\Bh}[0]{\mathbf{h}}
\newcommand{\Bi}[0]{\mathbf{i}}
\newcommand{\Bj}[0]{\mathbf{j}}
\newcommand{\Bk}[0]{\mathbf{k}}
\newcommand{\Bl}[0]{\mathbf{l}}
\newcommand{\Bm}[0]{\mathbf{m}}
\newcommand{\Bn}[0]{\mathbf{n}}
\newcommand{\Bo}[0]{\mathbf{o}}
\newcommand{\Bp}[0]{\mathbf{p}}
\newcommand{\Bq}[0]{\mathbf{q}}
\newcommand{\Br}[0]{\mathbf{r}}
\newcommand{\Bs}[0]{\mathbf{s}}
\newcommand{\Bt}[0]{\mathbf{t}}
\newcommand{\Bu}[0]{\mathbf{u}}
\newcommand{\Bv}[0]{\mathbf{v}}
\newcommand{\Bw}[0]{\mathbf{w}}
\newcommand{\Bx}[0]{\mathbf{x}}
\newcommand{\By}[0]{\mathbf{y}}
\newcommand{\Bz}[0]{\mathbf{z}}
\newcommand{\BA}[0]{\mathbf{A}}
\newcommand{\BB}[0]{\mathbf{B}}
\newcommand{\BC}[0]{\mathbf{C}}
\newcommand{\BD}[0]{\mathbf{D}}
\newcommand{\BE}[0]{\mathbf{E}}
\newcommand{\BF}[0]{\mathbf{F}}
\newcommand{\BG}[0]{\mathbf{G}}
\newcommand{\BH}[0]{\mathbf{H}}
\newcommand{\BI}[0]{\mathbf{I}}
\newcommand{\BJ}[0]{\mathbf{J}}
\newcommand{\BK}[0]{\mathbf{K}}
\newcommand{\BL}[0]{\mathbf{L}}
\newcommand{\BM}[0]{\mathbf{M}}
\newcommand{\BN}[0]{\mathbf{N}}
\newcommand{\BO}[0]{\mathbf{O}}
\newcommand{\BP}[0]{\mathbf{P}}
\newcommand{\BQ}[0]{\mathbf{Q}}
\newcommand{\BR}[0]{\mathbf{R}}
\newcommand{\BS}[0]{\mathbf{S}}
\newcommand{\BT}[0]{\mathbf{T}}
\newcommand{\BU}[0]{\mathbf{U}}
\newcommand{\BV}[0]{\mathbf{V}}
\newcommand{\BW}[0]{\mathbf{W}}
\newcommand{\BX}[0]{\mathbf{X}}
\newcommand{\BY}[0]{\mathbf{Y}}
\newcommand{\BZ}[0]{\mathbf{Z}}

\newcommand{\Bzero}[0]{\mathbf{0}}
\newcommand{\Btheta}[0]{\boldsymbol{\theta}}
\newcommand{\Btau}[0]{\boldsymbol{\tau}}
\newcommand{\Bomega}[0]{\boldsymbol{\omega}}

%
% shorthand for unit vectors
%
\newcommand{\acap}[0]{\hat{\Ba}}
\newcommand{\bcap}[0]{\hat{\Bb}}
\newcommand{\ccap}[0]{\hat{\Bc}}
\newcommand{\dcap}[0]{\hat{\Bd}}
\newcommand{\ecap}[0]{\hat{\Be}}
\newcommand{\fcap}[0]{\hat{\Bf}}
\newcommand{\gcap}[0]{\hat{\Bg}}
\newcommand{\hcap}[0]{\hat{\Bh}}
\newcommand{\icap}[0]{\hat{\Bi}}
\newcommand{\jcap}[0]{\hat{\Bj}}
\newcommand{\kcap}[0]{\hat{\Bk}}
\newcommand{\lcap}[0]{\hat{\Bl}}
\newcommand{\mcap}[0]{\hat{\Bm}}
\newcommand{\ncap}[0]{\hat{\Bn}}
\newcommand{\ocap}[0]{\hat{\Bo}}
\newcommand{\pcap}[0]{\hat{\Bp}}
\newcommand{\qcap}[0]{\hat{\Bq}}
\newcommand{\rcap}[0]{\hat{\Br}}
\newcommand{\scap}[0]{\hat{\Bs}}
\newcommand{\tcap}[0]{\hat{\Bt}}
\newcommand{\ucap}[0]{\hat{\Bu}}
\newcommand{\vcap}[0]{\hat{\Bv}}
\newcommand{\wcap}[0]{\hat{\Bw}}
\newcommand{\xcap}[0]{\hat{\Bx}}
\newcommand{\ycap}[0]{\hat{\By}}
\newcommand{\zcap}[0]{\hat{\Bz}}
\newcommand{\thetacap}[0]{\hat{\Btheta}}

%
% to write R^n and C^n in a distinguishable fashion.  Perhaps change this
% to the double lined characters upon figuring out how to do so.
%
\newcommand{\C}[1]{$\mathbb{C}^{#1}$}
\newcommand{\R}[1]{$\mathbb{R}^{#1}$}

%
% various generally useful helpers
%

% derivative of #1 wrt. #2:
\newcommand{\D}[2] {\frac {d#2} {d#1}}

\newcommand{\inv}[1]{\frac{1}{#1}}
\newcommand{\cross}[0]{\times}

\newcommand{\abs}[1]{\lvert{#1}\rvert}
\newcommand{\norm}[1]{\lVert{#1}\rVert}
\newcommand{\innerprod}[2]{\langle{#1}, {#2}\rangle}
\newcommand{\dotprod}[2]{{#1} \cdot {#2}}
\newcommand{\bdotprod}[2]{\left({#1} \cdot {#2}\right)}
\newcommand{\crossprod}[2]{{#1} \cross {#2}}
\newcommand{\tripleprod}[3]{\dotprod{\left(\crossprod{#1}{#2}\right)}{#3}}

\DeclareMathOperator{\Proj}{Proj}
\DeclareMathOperator{\Span}{span}
\DeclareMathOperator{\Sgn}{sgn}
\DeclareMathOperator{\Area}{Area}
\DeclareMathOperator{\Volume}{Volume}

%
% A few miscellaneous things specific to this document
%
\newcommand{\crossop}[1]{\crossprod{#1}{}}

% R2 vector.
\newcommand{\VectorTwo}[2]{
\begin{bmatrix}
 {#1} \\
 {#2}
\end{bmatrix}
}

\newcommand{\VectorN}[1]{
\begin{bmatrix}
{#1}_1 \\
{#1}_2 \\
\vdots \\
{#1}_N \\
\end{bmatrix}
}

\newcommand{\DETuvij}[4]{
\begin{vmatrix}
 {#1}_{#3} & {#1}_{#4} \\
 {#2}_{#3} & {#2}_{#4}
\end{vmatrix}
}

\newcommand{\DETuvwijk}[6]{
\begin{vmatrix}
 {#1}_{#4} & {#1}_{#5} & {#1}_{#6} \\
 {#2}_{#4} & {#2}_{#5} & {#2}_{#6} \\
 {#3}_{#4} & {#3}_{#5} & {#3}_{#6}
\end{vmatrix}
}

\newcommand{\DETuvwxijkl}[8]{
\begin{vmatrix}
 {#1}_{#5} & {#1}_{#6} & {#1}_{#7} & {#1}_{#8} \\
 {#2}_{#5} & {#2}_{#6} & {#2}_{#7} & {#2}_{#8} \\
 {#3}_{#5} & {#3}_{#6} & {#3}_{#7} & {#3}_{#8} \\
 {#4}_{#5} & {#4}_{#6} & {#4}_{#7} & {#4}_{#8} \\
\end{vmatrix}
}

%\newcommand{\DETuvwxyijklm}[10]{
%\begin{vmatrix}
% {#1}_{#6} & {#1}_{#7} & {#1}_{#8} & {#1}_{#9} & {#1}_{#10} \\
% {#2}_{#6} & {#2}_{#7} & {#2}_{#8} & {#2}_{#9} & {#2}_{#10} \\
% {#3}_{#6} & {#3}_{#7} & {#3}_{#8} & {#3}_{#9} & {#3}_{#10} \\
% {#4}_{#6} & {#4}_{#7} & {#4}_{#8} & {#4}_{#9} & {#4}_{#10} \\
% {#5}_{#6} & {#5}_{#7} & {#5}_{#8} & {#5}_{#9} & {#5}_{#10}
%\end{vmatrix}
%}

% R3 vector.
\newcommand{\VectorThree}[3]{
\begin{bmatrix}
 {#1} \\
 {#2} \\
 {#3}
\end{bmatrix}
}



\newcommand{\dt}[1]{\frac{d {#1}}{dt}}
\newcommand{\Norm}[1]{\left\lVert{#1}\right\rVert}
\newcommand{\dtheta}[1]{\frac{d {#1}}{d \theta}}

%
% The real thing:
%

                             % The preamble begins here.
\title{Rotational dynamics}
\author{Peeter Joot}         % Declares the author's name.
%\date{}        % Deleting this command produces today's date.

\begin{document}             % End of preamble and beginning of text.

\maketitle{}

\section{GA introduction of angular velocity}

By taking the first derivative of a radially expressed vector we have the velocity 

\[
\Bv 
   = r'\rcap + \rcap(\rcap \wedge \Br')
   = \rcap( v_r + \rcap \wedge \Bv )
\]

Or,
\[
\rcap \Bv = v_r + \rcap \wedge \Bv
\]
\[
\rcap \Bv = v_r + (1/r)\Br \wedge \Bv
\]

Put this way, the earlier calculus exercise to derive this seems a bit silly, since it is probably clear that $v_r = \rcap \cdot \Bv$.

Anyways, let's work with velocity expressed this way in a few ways.

\subsection{Speed in terms of linear and rotational components}

\[
\Norm{\Bv}^2 = v_r^2 + (\rcap(\rcap \wedge \Bv))^2
\]

And,
\begin{align*}
(\rcap(\rcap \wedge \Bv))^2 
   &= (\Bv \wedge \rcap)\rcap \rcap(\rcap \wedge \Bv) \\
   &= (\Bv \wedge \rcap)(\rcap \wedge \Bv) \\
   &= -(\rcap \wedge \Bv)^2 \\
   &= \Norm{\rcap \wedge \Bv}^2 \\
\end{align*}

\begin{align*}
\implies
\Norm{\Bv}^2 &= v_r^2 + \Norm{\rcap \wedge \Bv}^2 \\
             &= v_r^2 + \Norm{\rcap \wedge \Bv}^2 \\
\end{align*}

So, we can assign a physical significance to the bivector.

\[
\Norm{\rcap \wedge \Bv} = \abs{v_{\perp}} 
\]

The bivector $\Norm{\rcap \wedge \Bv}$ has the magnitude of the non-radial component of the velocity.  This
equals the magnitude of the component of the velocity perpendicular to its radial component (ie: the angular component of the velocity).

\subsection{angular velocity.  Prep.}

Because $\Norm{\rcap \wedge \Bv}$ is the non-radial velocity component, for small angles
${v_\perp}/r$ will equal the angle between the vector and its displacement.

This allows for the calculation of the rate of change of that angle with time, what it called the scalar
angular velocity (dimensions are $1/t$ not $x/t$).  This can be done by taking the $\sin$ as the ratio of the
length of the non-radial component of the delta to the length of the displaced vector.

\begin{align*}
\sin d\theta &= \frac{\Norm{\rcap(\rcap \wedge d\Br)}}{\Norm{\Br + d\Br}} \\
\end{align*}

With $d\Br = \dt{\Br} dt = \Bv dt$, the angular velocity is

\begin{align*}
\sin d\theta
   &= \frac{1}{\Norm{\Br + \Bv dt}} \Norm{ \rcap (\rcap \wedge \Bv) dt } \\
   &= \frac{1}{\Norm{\Br + \Bv dt}} \Norm{ (\rcap \wedge \Bv) dt } \\
\frac{\sin d\theta}{\abs{dt}}
   &= \frac{1}{\Norm{\Br + \Bv dt}} \Norm{ \rcap \wedge \Bv } \\
   &= \frac{1}{\Norm{\Br}\Norm{\Br + \Bv dt}} \Norm{ \Br \wedge \Bv } \\
\end{align*}

In the limit, taking $dt > 0$, this is
\[
\omega = \dt{\theta} = \frac{1}{\Br^2} \Norm{ \Br \wedge \Bv }
\]

\subsection{angular velocity.  Summarizing.}

Here is a summary of calculations so far involving the $\Br \wedge \Bv$ bivector

\begin{align*}
\Bv &= \rcap v_r + \frac{\rcap}{\Norm{\Br}} (\Br \wedge \Bv) \\
\dt{\rcap} &= \frac{\rcap}{\Br^2} (\Br \wedge \Bv) \\
\abs{v_{\perp}} &= \frac{1}{\Norm{\Br}} \Norm{ \Br \wedge \Bv } \\
\omega = \dt{\theta} &= \frac{1}{\Br^2} \Norm{ \Br \wedge \Bv } \\
\end{align*}

It makes sense to give the bivector a name.  Given it's magnitude the 
angular velocity bivector $\Bomega$ is designated

\[
\Bomega = \frac{ \Br \wedge \Bv }{\Br^2} 
\]

So the linear and rotational components of the velocity can thus be expressed in terms of this, as can our
unit vector derivative, scalar angular velocity, and perpendicular velocity magnitude:

\begin{align*}
\omega = \dt{\theta} &= \Norm{ \Bomega } \\
\Bv &= \rcap v_r + \Br \Bomega \\
    &= \rcap( v_r + r \Bomega ) \\
\dt{\rcap} &= \rcap \Bomega \\
\abs{v_{\perp}} &= r \Norm{ \Bomega } \\
\end{align*}

This is similar to the vector angular velocity ($\Bomega = (\Br \times \Bv)/r^2$), but instead of lying perpendicular to the
plane of rotation, it defines the plane of rotation (for a vector $\Ba$, $\Ba \wedge \Bomega$ is zero if the vector is in the plane and non-zero if the vector has a component outside of the plane).

%\begin{align*}
%\Bomega = \frac{1}{\Br^2} (\Br \wedge \Bv)
%\end{align*}
%
%Or,
%\begin{align*}
%\Br \wedge \Bv = \Br^2 \Bomega = r^2\Bomega
%\end{align*}
%
%
%\begin{align*}
%\Bv 
%   &= \rcap(v_r + (1/r)\Br \wedge \Bv) \\
%   &= \rcap(v_r + r\Bomega) \\
%   &= v_r\rcap + \Br\Bomega \\
%\end{align*}

\subsection{Explicit perpendicular unit vector.}

If one introduces a unit vector $\thetacap$ in the direction of rejection of $\Br$ from $d\Br$, the total velocity takes the symetrical form
\begin{align*}
\Bv 
   &= v_r\rcap + r\omega\thetacap \\
   &= \dt{r}\rcap + r\dt{\theta}\thetacap \\
\end{align*}

\subsection{acceleration in terms of angular velocity bivector}

Taking derivatives of velocity, one can with a bit of work,
express acceleration in terms of
radial and non-radial components

%\Br \wedge \Bv = \Br^2 \Bomega = r^2\Bomega

\begin{align*}
\Ba 
   &= (\rcap v_r + \Br \Bomega)' \\
   &= \rcap' v_r + \rcap v_r' + \Br' \Bomega + \Br \Bomega' \\
   &= \rcap \Bomega v_r + \rcap v_r' + \Br' \Bomega + \Br \Bomega' \\
   &= \rcap \Bomega v_r + \rcap a_r + \Bv \Bomega + \Br \Bomega' \\
\end{align*}

But,
\begin{align*}
\Bomega' &= ((1/r^2) (\Br \wedge \Bv))' \\
         &= (-2/r^3) r' (\Br \wedge \Bv) + (1/r^2) (\Bv \wedge \Bv + \Br \wedge \Ba) \\
         &= -(2/r) v_r \Bomega + (1/r^2) (\Br \wedge \Ba) \\
\end{align*}

%\rcap v_r = \Bv - \Br \Bomega
So,

\begin{align*}
\Ba 
   &= \rcap a_r -\rcap \Bomega v_r + \Bv \Bomega + \rcap (\rcap \wedge \Ba) \\
   &= \rcap a_r -( \Bv - \Br \Bomega) \Bomega + \Bv \Bomega + \rcap (\rcap \wedge \Ba) \\
\\
   &= \rcap a_r + \Br \Bomega^2+ \rcap (\rcap \wedge \Ba) \\
   &= \rcap( a_r + r \Bomega^2) + \rcap (\rcap \wedge \Ba) \\
\end{align*}

Note that $\Bomega^2$ is a negative scalar, so as normal writing $\norm{\Bomega}^2 = -\Bomega^2$, we have acceleration in a fashion similar to the
traditional cross product form:

\begin{align*}
\Ba 
   &= \rcap( a_r - r \norm{\Bomega}^2) + \rcap (\rcap \wedge \Ba) \\
   &= \rcap( a_r - r \norm{\Bomega}^2 + \rcap \wedge \Ba) \\
\end{align*}

In the traditional representation, this last term, the non-radial acceleration
component, is often expressed as a derivative.

In terms of the wedge product, this can be done by noting that

\[
(\Br \wedge \Bv)' = \Bv \wedge \Bv + \Br \wedge \Ba = \Br \wedge \Ba
\]

\begin{align*}
\Ba 
   &= \rcap( a_r - r \norm{\Bomega}^2 ) + \frac{\Br}{r^2}(\Br \wedge \Bv)') \\
   &= \rcap( a_r - r \norm{\Bomega}^2 ) + \frac{1}{\Br}\dt{(\Br^2 \Bomega)} \\
\end{align*}

Expressed in terms of force (for constant mass) this is
\begin{align*}
\BF &= m \Ba \\
    &= \rcap (m a_r) + (m \Br) {\Bomega}^2
       + \frac{1}{\Br}\dt{(m \Br^2 \Bomega)} \\
    &= \BF_r + (m \Br) {\Bomega}^2
             + \frac{1}{\Br}\dt{(m \Br^2 \Bomega)} \\
\end{align*}

Alternately, the non-radial term can be expressed in terms of torque

\begin{align*}
\rcap (\rcap \wedge \Ba) 
   &= \rcap (\rcap \wedge m \Ba)  \\
   &= \frac{\Br}{r^2} (\Br \wedge \BF)  \\
   &= \frac{1}{\Br} (\Br \wedge \BF)  \\
   &= \frac{1}{\Br} \Btau \\
\end{align*}

Thus the torque bivector, which in magnitude was the angular derivative of
the work
done by the force $\norm{\Btau} = \tau = \dtheta{W} = \BF \cdot \dtheta{\Br}$
is also expressable as a time derivative

\begin{align*}
\Btau 
&= \dt{( m \Br^2 \Bomega )}  \\
&= \dt{( m \Br \wedge \Bv)}  \\
&= \dt{( \Br \wedge m \Bv)}  \\
&= \dt{( \Br \wedge \Bp  )}  \\
\end{align*}

This bivector $m \Br^2 \Bomega = \Br \wedge \Bp$ is called the angular
momentum, designated $\BJ$.  It is related to the total momentum as follows

\[
\Bp = \rcap (\rcap \cdot \Bp) + \frac{1}{\Br} \BJ 
\]

So the total force is

\begin{align*}
\BF 
    &= \BF_r + m \Br {\Bomega}^2 + \frac{1}{\Br}\dt{\BJ} \\
\end{align*}

Observe that for a purely radial (ie: central) force, we must have
$\dt{\BJ} = 0$
so, the angular
momentum must be constant.

%\begin{align*}
%\Ba 
%   &= \rcap \left(  v_r' +(1/r)(\rcap \wedge \Bv)^2\right) +    \rcap(\rcap \wedge \Ba) \\
%   &= \rcap \left(  a_r +(1/{r^3})(\Br \wedge \Bv)^2\right) +    \rcap(\rcap \wedge \Ba) \\
%   &= \rcap \left(  a_r +(1/{r^3})( r^2\Bomega )^2\right) +    \rcap(\rcap \wedge \Ba) \\
%   &= \rcap \left(  a_r +r( \Bomega )^2\right) +    \rcap(\rcap \wedge \Ba) \\
%   &= \rcap \left(  a_r -r \Norm{\Bomega}^2\right) + \rcap(\rcap \wedge \Ba) \\
%\end{align*}
%
%This can be written in a few different ways:
%\begin{align*}
%\Ba 
%   &= \rcap \left(  a_r -r \Norm{\Bomega}^2 + \rcap \wedge \Ba \right) \\
%   &= \rcap \left(  a_r -(1/r)(\Br\Bomega)^2 + \rcap \wedge \Ba \right) \\
%   &= \rcap \left(  a_r -(1/r)(\Bv - v_r\rcap)^2 + \rcap \wedge \Ba \right) \\
%\end{align*}

\subsection{Circular motion}

For circular motion $v_r = a_r = 0$, so:

\[
\Bv = \Br \Bomega
\]
\[
\Ba = \rcap \left(  -\frac{\Bv^2}{r} + \rcap \wedge \Ba \right) \\
\]

For constant circular motion:
\begin{align*}
\Ba 
   &= \Bv\Bomega + \Br\Bomega' \\
   &= \Bv\Bomega + \Br(\Bzero) \\
   &= \Br(\Bomega)^2 \\
   &= -\Br\Norm{\Bomega}^2 \\
\end{align*}

ie: the $\rcap (\rcap \wedge \Ba )$ term is zero... all accelartion is inwards.

Can also expand this in terms of $\Br$ and $\Bv$:
\begin{align*}
\Ba 
   &= \Br(\Bomega)^2 \\
   &= \Br(\frac{1}{\Br}\Bv)^2 \\
   &= -\Br( \Bv \frac{1}{\Br} \frac{1}{\Br}\Bv ) \\
   &= -\Br( \frac{\Bv^2}{\Br^2}) \\
   &= -\frac{1}{\Br}\Bv^2 \\
\end{align*}

\end{document}             % End of preamble and beginning of text.
