\chapter{Hyper complex numbers and symplectic structure. }
\label{chap:complex}
\date{November 8, 2008.  complex.tex}

\section{On 4.2 Hermitian Norms and Unitary Groups. }

These are some rather rough notes filling in some details 
on the treatment of \citep{DoranHamiltonian}.

Expanding equation 4.17

\begin{align*}
J &= e_i \wedge f_i \\
a &= u_i e_i + v_i f_i \\
b &= x_i e_i + y_i f_i \\
B &= a \wedge b + (a \cdot J) \wedge (b \cdot J)  \\
\end{align*}

\begin{align*}
a \wedge b
&= (u_i e_i + v_i f_i) \wedge (x_j e_j + y_j f_j) \\
&= 
u_i x_j e_i \wedge e_j 
+ u_i y_j e_i \wedge f_j
+ v_i x_j f_i \wedge e_j
+ v_i y_j f_i \wedge f_j \\
\end{align*}

\begin{align*}
a \cdot J
&=
u_i e_i \cdot ( e_j \wedge f_j )
+ v_i f_i \cdot ( e_j \wedge f_j ) \\
&= u_j f_j - v_j e_j
\end{align*}

Search and replace for $b \cdot J$ gives

\begin{align*}
b \cdot J
&=
x_i e_i \cdot ( e_j \wedge f_j )
+ y_i f_i \cdot ( e_j \wedge f_j ) \\
&= x_j f_j - y_j e_j
\end{align*}

So we have

\begin{align*}
(a \cdot J) \wedge (b \cdot J) 
&= (u_i f_i - v_i e_i) \wedge (x_j f_j - y_j e_j) \\
&=
 u_i x_j f_i \wedge f_j 
-u_i y_j f_i \wedge e_j
- v_i x_j e_i \wedge f_j
+ v_i y_j e_i \wedge e_j
\end{align*}

For
\begin{align*}
a \wedge b + (a \cdot J) \wedge (b \cdot J) 
&= 
 ( u_i y_j - v_i x_j ) (e_i \wedge f_j - f_i \wedge e_j)
+ ( u_i x_j + v_i y_j ) (e_i \wedge e_j + f_i \wedge f_j)
\end{align*}

This shows why the elements were picked as a basis
\begin{align*}
e_i \wedge f_j - f_i \wedge e_j
\end{align*}
\begin{align*}
e_i \wedge e_j + f_i \wedge f_j
\end{align*}

The first of which is a multiple of $J_i = e_i \wedge f_i$ when $i=j$, and the second of which is zero if $i=j$.

\section{5.1 Conservation Theorems and Flows. } 

equation 5.10 is

\begin{align*}
\fdot = \xdot \cdot \grad f = (\grad f \wedge \grad H) \cdot J
\end{align*}

This one isn't obvious to me.  For $\fdot$ we have

\begin{align*}
\fdot = \PD{p_i}{f} \pdot_i +\PD{q_i}{f} \qdot_i + \underbrace{\PD{t}{f}}_{=0}
\end{align*}

compare to 

\begin{align*}
\xdot \cdot \grad f 
&= (\pdot_i e_i + \qdot_i f_i) \cdot (e_j \PD{p_j}{f} + f_j \PD{q_j}{f}) \\
&= \pdot_i \PD{p_i}{f} + \qdot_i \PD{q_i}{f}
\end{align*}

Okay, this part matches the first part of (5.10).  Writing this in terms of the Hamiltonian relation (5.9) $\xdot = \grad H \cdot J$ we have

\begin{align*}
\fdot
&= (\grad H \cdot J) \cdot \grad f \\
&= \grad f \cdot (\grad H \cdot J) \\
\end{align*}

The relation $a \cdot (b \cdot (c \wedge d)) = (a \wedge b) \cdot (c \wedge d)$, 
can be used here to factor out the $J$, we have
\begin{align*}
\fdot
&= \grad f \cdot (\grad H \cdot J) \\
&= (\grad f \wedge \grad H) \cdot J \\
\end{align*}

which completes (5.10).

Also with $f=H$ since H was also specified as having no explicit time dependence, one has

\begin{align*}
\dot{H} &= (\grad H \wedge \grad H) \cdot J = 0 \cdot J = 0
\end{align*}

%\bibliographystyle{plainnat}
%\bibliography{myrefs}

%\end{document}
