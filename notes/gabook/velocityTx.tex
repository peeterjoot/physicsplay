\documentclass{article}      % Specifies the document class

\usepackage{amsmath}
\usepackage{mathpazo}

%
% shorthand for bold symbols, convenient for vectors and matrices
%
\newcommand{\Ba}[0]{\mathbf{a}}
\newcommand{\Bb}[0]{\mathbf{b}}
\newcommand{\Bc}[0]{\mathbf{c}}
\newcommand{\Bd}[0]{\mathbf{d}}
\newcommand{\Be}[0]{\mathbf{e}}
\newcommand{\Bf}[0]{\mathbf{f}}
\newcommand{\Bg}[0]{\mathbf{g}}
\newcommand{\Bh}[0]{\mathbf{h}}
\newcommand{\Bi}[0]{\mathbf{i}}
\newcommand{\Bj}[0]{\mathbf{j}}
\newcommand{\Bk}[0]{\mathbf{k}}
\newcommand{\Bl}[0]{\mathbf{l}}
\newcommand{\Bm}[0]{\mathbf{m}}
\newcommand{\Bn}[0]{\mathbf{n}}
\newcommand{\Bo}[0]{\mathbf{o}}
\newcommand{\Bp}[0]{\mathbf{p}}
\newcommand{\Bq}[0]{\mathbf{q}}
\newcommand{\Br}[0]{\mathbf{r}}
\newcommand{\Bs}[0]{\mathbf{s}}
\newcommand{\Bt}[0]{\mathbf{t}}
\newcommand{\Bu}[0]{\mathbf{u}}
\newcommand{\Bv}[0]{\mathbf{v}}
\newcommand{\Bw}[0]{\mathbf{w}}
\newcommand{\Bx}[0]{\mathbf{x}}
\newcommand{\By}[0]{\mathbf{y}}
\newcommand{\Bz}[0]{\mathbf{z}}
\newcommand{\BA}[0]{\mathbf{A}}
\newcommand{\BB}[0]{\mathbf{B}}
\newcommand{\BC}[0]{\mathbf{C}}
\newcommand{\BD}[0]{\mathbf{D}}
\newcommand{\BE}[0]{\mathbf{E}}
\newcommand{\BF}[0]{\mathbf{F}}
\newcommand{\BG}[0]{\mathbf{G}}
\newcommand{\BH}[0]{\mathbf{H}}
\newcommand{\BI}[0]{\mathbf{I}}
\newcommand{\BJ}[0]{\mathbf{J}}
\newcommand{\BK}[0]{\mathbf{K}}
\newcommand{\BL}[0]{\mathbf{L}}
\newcommand{\BM}[0]{\mathbf{M}}
\newcommand{\BN}[0]{\mathbf{N}}
\newcommand{\BO}[0]{\mathbf{O}}
\newcommand{\BP}[0]{\mathbf{P}}
\newcommand{\BQ}[0]{\mathbf{Q}}
\newcommand{\BR}[0]{\mathbf{R}}
\newcommand{\BS}[0]{\mathbf{S}}
\newcommand{\BT}[0]{\mathbf{T}}
\newcommand{\BU}[0]{\mathbf{U}}
\newcommand{\BV}[0]{\mathbf{V}}
\newcommand{\BW}[0]{\mathbf{W}}
\newcommand{\BX}[0]{\mathbf{X}}
\newcommand{\BY}[0]{\mathbf{Y}}
\newcommand{\BZ}[0]{\mathbf{Z}}

\newcommand{\Bzero}[0]{\mathbf{0}}
\newcommand{\Btheta}[0]{\boldsymbol{\theta}}
\newcommand{\Btau}[0]{\boldsymbol{\tau}}
\newcommand{\Bomega}[0]{\boldsymbol{\omega}}

%
% shorthand for unit vectors
%
\newcommand{\acap}[0]{\hat{\Ba}}
\newcommand{\bcap}[0]{\hat{\Bb}}
\newcommand{\ccap}[0]{\hat{\Bc}}
\newcommand{\dcap}[0]{\hat{\Bd}}
\newcommand{\ecap}[0]{\hat{\Be}}
\newcommand{\fcap}[0]{\hat{\Bf}}
\newcommand{\gcap}[0]{\hat{\Bg}}
\newcommand{\hcap}[0]{\hat{\Bh}}
\newcommand{\icap}[0]{\hat{\Bi}}
\newcommand{\jcap}[0]{\hat{\Bj}}
\newcommand{\kcap}[0]{\hat{\Bk}}
\newcommand{\lcap}[0]{\hat{\Bl}}
\newcommand{\mcap}[0]{\hat{\Bm}}
\newcommand{\ncap}[0]{\hat{\Bn}}
\newcommand{\ocap}[0]{\hat{\Bo}}
\newcommand{\pcap}[0]{\hat{\Bp}}
\newcommand{\qcap}[0]{\hat{\Bq}}
\newcommand{\rcap}[0]{\hat{\Br}}
\newcommand{\scap}[0]{\hat{\Bs}}
\newcommand{\tcap}[0]{\hat{\Bt}}
\newcommand{\ucap}[0]{\hat{\Bu}}
\newcommand{\vcap}[0]{\hat{\Bv}}
\newcommand{\wcap}[0]{\hat{\Bw}}
\newcommand{\xcap}[0]{\hat{\Bx}}
\newcommand{\ycap}[0]{\hat{\By}}
\newcommand{\zcap}[0]{\hat{\Bz}}
\newcommand{\thetacap}[0]{\hat{\Btheta}}

%
% to write R^n and C^n in a distinguishable fashion.  Perhaps change this
% to the double lined characters upon figuring out how to do so.
%
\newcommand{\C}[1]{$\mathbb{C}^{#1}$}
\newcommand{\R}[1]{$\mathbb{R}^{#1}$}

%
% various generally useful helpers
%

% derivative of #1 wrt. #2:
\newcommand{\D}[2] {\frac {d#2} {d#1}}

\newcommand{\inv}[1]{\frac{1}{#1}}
\newcommand{\cross}[0]{\times}

\newcommand{\abs}[1]{\lvert{#1}\rvert}
\newcommand{\norm}[1]{\lVert{#1}\rVert}
\newcommand{\innerprod}[2]{\langle{#1}, {#2}\rangle}
\newcommand{\dotprod}[2]{{#1} \cdot {#2}}
\newcommand{\bdotprod}[2]{\left({#1} \cdot {#2}\right)}
\newcommand{\crossprod}[2]{{#1} \cross {#2}}
\newcommand{\tripleprod}[3]{\dotprod{\left(\crossprod{#1}{#2}\right)}{#3}}

\DeclareMathOperator{\Proj}{Proj}
\DeclareMathOperator{\Span}{span}
\DeclareMathOperator{\Sgn}{sgn}
\DeclareMathOperator{\Area}{Area}
\DeclareMathOperator{\Volume}{Volume}

%
% A few miscellaneous things specific to this document
%
\newcommand{\crossop}[1]{\crossprod{#1}{}}

% R2 vector.
\newcommand{\VectorTwo}[2]{
\begin{bmatrix}
 {#1} \\
 {#2}
\end{bmatrix}
}

\newcommand{\VectorN}[1]{
\begin{bmatrix}
{#1}_1 \\
{#1}_2 \\
\vdots \\
{#1}_N \\
\end{bmatrix}
}

\newcommand{\DETuvij}[4]{
\begin{vmatrix}
 {#1}_{#3} & {#1}_{#4} \\
 {#2}_{#3} & {#2}_{#4}
\end{vmatrix}
}

\newcommand{\DETuvwijk}[6]{
\begin{vmatrix}
 {#1}_{#4} & {#1}_{#5} & {#1}_{#6} \\
 {#2}_{#4} & {#2}_{#5} & {#2}_{#6} \\
 {#3}_{#4} & {#3}_{#5} & {#3}_{#6}
\end{vmatrix}
}

\newcommand{\DETuvwxijkl}[8]{
\begin{vmatrix}
 {#1}_{#5} & {#1}_{#6} & {#1}_{#7} & {#1}_{#8} \\
 {#2}_{#5} & {#2}_{#6} & {#2}_{#7} & {#2}_{#8} \\
 {#3}_{#5} & {#3}_{#6} & {#3}_{#7} & {#3}_{#8} \\
 {#4}_{#5} & {#4}_{#6} & {#4}_{#7} & {#4}_{#8} \\
\end{vmatrix}
}

%\newcommand{\DETuvwxyijklm}[10]{
%\begin{vmatrix}
% {#1}_{#6} & {#1}_{#7} & {#1}_{#8} & {#1}_{#9} & {#1}_{#10} \\
% {#2}_{#6} & {#2}_{#7} & {#2}_{#8} & {#2}_{#9} & {#2}_{#10} \\
% {#3}_{#6} & {#3}_{#7} & {#3}_{#8} & {#3}_{#9} & {#3}_{#10} \\
% {#4}_{#6} & {#4}_{#7} & {#4}_{#8} & {#4}_{#9} & {#4}_{#10} \\
% {#5}_{#6} & {#5}_{#7} & {#5}_{#8} & {#5}_{#9} & {#5}_{#10}
%\end{vmatrix}
%}

% R3 vector.
\newcommand{\VectorThree}[3]{
\begin{bmatrix}
 {#1} \\
 {#2} \\
 {#3}
\end{bmatrix}
}


\newcommand{\gpgrade}[2] {{\left\langle{{#1}}\right\rangle}_{#2}}
\newcommand{\gpgradezero}[1] {\gpgrade{#1}{0}}
\newcommand{\gpgradetwo}[1] {\gpgrade{#1}{2}}

\title{Understanding four velocity transform from rest frame.}
\author{Peeter Joot}
\date{August 13, 2008}

\begin{document}             % End of preamble and beginning of text.

\maketitle{}

\section{}

GAFP writes $v = R \gamma_0 R^\dagger$.  This would probably be a lot more
obvious to me if I had fully read chapter 5.  Let's just expand this out
to see how this works.  First thing to note is that there is an ommitted factor
of $c$, and I'll add that back in here, since I'm not comfortable enough
without it explicitly for now.

With:

\begin{align*}
\Bv/c &= tanh\left(\alpha\right)\vcap \\
R &= \exp\left(\alpha \vcap/2\right)
\end{align*}

We want to expansion this Lorentz boost exponential (see details section) and apply it to the rest frame basis vector.  Writing
$C = \cosh\left(\alpha/2\right)$, and $S = \sinh\left(\alpha/2\right)$, we have:

\begin{align*}
v
&= R \left(c \gamma_0\right) R^\dagger \\
&= c \left(C + \vcap S\right) \gamma_0 \left(C - \vcap S\right) \\
&= c \left(C \gamma_0 + S \vcap \gamma_0\right) \left(C - \vcap S\right) \\
&= c \left( C^2 \gamma_0 + SC \vcap \gamma_0 -CS \gamma_0\vcap - S^2 \vcap \gamma_0 \vcap \right) \\
\end{align*}

Now, here things can start to get confusing since $\vcap$ is a spatial quantity with vector-like spacetime basis bivectors $\sigma_i = \gamma_i \gamma_0$.  Factoring out the $\gamma_0$ term, utilizing the fact that $\gamma_0$ and $\sigma_i$ anticommute (see below).

\begin{align*}
v
&= c \left( C^2 + S^2 + 2 SC \vcap \right) \gamma_0 \\
&= c \left( \cosh\left(\alpha\right) + \vcap \sinh\left(\alpha\right) \right) \gamma_0 \\
&= c \cosh\left(\alpha\right) \left( 1 + \vcap \tanh\left(\alpha\right) \right) \gamma_0 \\
&= c \cosh\left(\alpha\right) \left( 1 + \Bv/c \right) \gamma_0 \\
&= c \gamma \left( 1 + \Bv/c \right) \gamma_0 \\
&= \gamma \left( c \gamma_0 + \sum v^i \gamma_i\right) \\
&= \frac{dt}{d\tau}\left( c \gamma_0 + \sum v^i \gamma_i\right) \\
&= \frac{dt}{d\tau} \frac{d}{dt}\left( c t \gamma_0 + \sum x^i \gamma_i\right) \\
&= \frac{dt}{d\tau} \frac{d}{dt} \sum x^{\mu} \gamma_{\mu} \\
&= \frac{d}{d\tau} \sum x^{\mu} \gamma_{\mu} \\
&= \frac{dx}{d\tau}
\end{align*}

So, we get the end result that demonstrates that a Lorentz boost applied to the rest event vector $x = x^0 \gamma_0 = c t \gamma_0$ directly produces the four velocity for the motion from the new viewpoint.  This makes some intuitive sense, but
I don't feel this is neccessarily obvious without demonstration.

This also explains how the text is able to use the wedge and dot product ratios with the $\gamma_0$ basis vector
to produce the relative spatial velocity.  If one introduces a rest frame proper velocity of
$w = \frac{d}{dt}\left(ct \gamma_0\right) = c \gamma_0$, then one has:

\begin{align*}
v \cdot w 
&= \left(\sum \frac{d x^{\mu}}{d\tau} \gamma_{\mu}\right) \cdot \left(c\gamma_0\right) \\
&= c^2 \gamma
\end{align*}

\begin{align*}
v \wedge w 
&= \left(\sum \frac{d x^{\mu}}{d\tau} \gamma_{\mu}\right) \wedge \left(c\gamma_0\right) \\
&= \left(\sum \frac{d x^{i}}{d\tau} \gamma_{i}\right) \wedge \left(c\gamma_0\right) \\
&= c \sum \frac{d x^{i}}{d\tau} \sigma_{i} \\
&= c \frac{dt}{d\tau} \sum \frac{d x^{i}}{dt} \sigma_{i} \\
&= c \gamma \sum \frac{d x^{i}}{dt} \sigma_{i} \\
\end{align*}

Combining these one has the spatial observer dependent (relative) velocity:

\begin{equation}
\frac{v \wedge w}{v \cdot w} = \inv{c} \sum \frac{d x^{i}}{dt} \sigma_{i} = \frac{\Bv}{c}
\end{equation}

What isn't clear to me is whether this can be used to determine the relative velocity between two particles in the general case, when one of them isn't a rest frame velocity (time progression only at a fixed point in space.)
The text seems
to imply this is the case, so perhaps it is obvious to them only and not me;)

Intuitively, I would guess that this is fact the case because when only two particles are considered, the result should be the same independent of which of the
two is considered at rest.

Mathematically, I would express this statement by saying that if one has
a Lorentz boost that takes $v' = T v T^\dagger$ to its rest frame, then application of this to both proper velocities leaves both the wedge and dot product 
parts of this ratio unchanged:

\begin{align*}
v \cdot w 
&= \left(T^\dagger v' T\right) \cdot \left(T^\dagger w' T\right) \\
&= \gpgradezero{\left(T^\dagger v' T\right) \left(T^\dagger w' T\right)} \\
&= \gpgradezero{T^\dagger v' w' T} \\
&= \gpgradezero{T^\dagger v' \cdot w' T} + \underbrace{\gpgradezero{T^\dagger v' \wedge w' T}}_{=0} \\
&= \left(v' \cdot w'\right)\gpgradezero{T^\dagger T} \\
&= v' \cdot w'
\end{align*}

\begin{align*}
v \wedge w 
&= \left(T^\dagger v' T\right) \wedge \left(T^\dagger w' T\right) \\
&= \gpgradetwo{\left(T^\dagger v' T\right) \left(T^\dagger w' T\right)} \\
&= \gpgradetwo{T^\dagger v' w' T} \\
&= \underbrace{\gpgradetwo{T^\dagger v' \cdot w' T}}_{=0} + \gpgradetwo{T^\dagger v' \wedge w' T} \\
&= T^\dagger \left(v' \wedge w'\right) T
\end{align*}

FIXME: can't those last $T$ factors be removed?

\section{ Appendix. Omitted details from above. }

\subsection{ exponential of a vector. } 

Understanding the vector exponential is a prerequisite above.  This is defined
and interpretted by series expansion as with matrix exponentials.
Expanding
in series the exponential of a vector $\Bx = x\xcap$, we have:

\begin{align*}
\exp\left(\Bx\right)
&= \sum \frac{\Bx^{2k}}{\left(2k\right)!} + \sum \frac{\Bx^{2k+1}}{\left(2k+1\right)!} \\
&= \sum \frac{x^{2k}}{\left(2k\right)!} + \xcap \sum \frac{x^{2k+1}}{\left(2k+1\right)!} \\
&= \cosh\left(x\right) + \xcap \sinh\left(x\right)
\end{align*}

Notationally this can also be written:

\begin{equation*}
\exp\left(\Bx\right) = \cosh\left(\Bx\right) + \sinh\left(\Bx\right)
\end{equation*}

But doing so won't really help.

\subsection{ $\Bv$ anticommutes with $\gamma_0$ }

\begin{align*}
\Bv \gamma_0 
&= \sum v^i \sigma_i \gamma_0 \\
&= \sum v^i \gamma_i \gamma_0 \gamma_0 \\
&= -\sum v^i \gamma_0 \gamma_i \gamma_0 \\
&= - \gamma_0 \sum v^i \gamma_i \gamma_0 \\
&= - \gamma_0 \sum v^i \sigma_0 \\
&= - \gamma_0 \Bv
\end{align*}

\end{document}               % End of document.
