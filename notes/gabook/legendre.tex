\chapter{Legendre Polynomials} 
\label{chap:legendre}
\date{Feb 4, 2008.  legendre.tex}

Exercise 8.4, from \cite{hestenes1999nfc}.

Find the first couple terms of the Legendre polynomial expansion of 

\[
\inv{\abs{\Bx - \Ba}}
\]

Write

\[
f(x) = \inv{\abs{\Bx}}
\]

Expanding $f(\Bx - \Ba)$ about $\Bx$ we have

\[
\inv{\abs{\Bx - \Ba}} = 
\sum_{k=0}{ \inv{k!} (-\agrad)^k} \inv{\abs{\Bx}}
\]

Expanding the first term we have

\begin{align*}
-\agrad \inv{\abs{\Bx}} 
&= 
\inv{{\abs{\Bx}}^2} \agrad {\abs{\Bx}} \\
&= 
\inv{{\abs{\Bx}}^2} \agrad (\Bx^2)^{1/2} \\
&= 
\inv{{\abs{\Bx}}^2} \frac{(1/2)}{({\abs{\Bx}}^2)^{1/2}}\agrad \Bx^2 \\
&= 
\frac{\Ba \cdot \Bx}{{\abs{\Bx}}^3} 
\end{align*}

Expansion of the second derivative term is
\begin{align*}
\frac{(-\agrad)}{2}\frac{(-\agrad)}{1}\inv{\abs{\Bx}} 
&= 
\frac{\agrad}{2} \left(\frac{-\Ba \cdot \Bx}{{\abs{\Bx}}^3}\right) \\
&= 
\frac{-1}{2}
\left( 
\frac{\agrad {(\Ba \cdot \Bx)}}{{\abs{\Bx}}^3} + {(\Ba \cdot \Bx)}\agrad \inv{{\abs{\Bx}}^3} \right) \\
\end{align*}

For this we need 
\[
\agrad {(\Ba \cdot \Bx)} = 
\Ba \cdot (\agrad {\Bx}) = \Ba^2
\]

And 
\begin{align*}
\agrad \inv{{\abs{\Bx}}^k}
&=
k \inv{{\abs{\Bx}}^{k-1}} \agrad \inv{{\abs{\Bx}}} \\
&=
k \inv{{\abs{\Bx}}^{k-1}} \frac{- \Ba \cdot \Bx }{{\abs{\Bx}}^3} \\
&=
-k \frac{\Ba \cdot \Bx }{{\abs{\Bx}}^{k+2}} \\
\end{align*}

Thus the second derivative term is 
\begin{align*}
\frac{-1}{2}
\left( 
\frac{\Ba^2}{{\abs{\Bx}}^3} -3 \frac{(\Ba \cdot \Bx)^2} {{\abs{\Bx}}^5} \right) \\
&=
\frac{ (1/2)\left( 3 (\Ba \cdot \Bx)^2 - \Ba^2 \Bx^2 \right) }
{ {{\abs{\Bx}}^5} } 
\\
\end{align*}

Summing these terms we have

\[
\inv{\abs{\Bx -\Ba}} =
\inv{\abs{\Bx}} +
\frac{ \Ba \cdot \Bx } { {\abs{\Bx}}^3 } +
\frac{ (1/2)\left( 3 (\Ba \cdot \Bx)^2 - \Ba^2 \Bx^2 \right) } { {{\abs{\Bx}}^5} } + \cdots
\]

NFCM writes this as
\[
\inv{\abs{\Bx -\Ba}} =
\frac{ P_0(\bxa) } {  \abs{\Bx}} +
\frac{ P_1(\bxa) } { {\abs{\Bx}}^3 } +
\frac{ P_2(\bxa) } { {\abs{\Bx}}^5 } + \cdots
\]

And calls $P_i = P_i(\bxa)$ terms the Legendre polynomials.  This isn't terribly clear since one expects a different form for the Legendre polynomials.

Using the Taylor formula one can derive a recurrence relation for these that makes the calculation a bit
simpler

\begin{align*}
\frac{P_{k+1}}{\abs{\Bx}^{2(k+1)+1}}
&= \frac{-\agrad}{k+1}\left(\frac{P_k}{\abs{\Bx}^{2k+1}}\right) \\
&= 
\frac{-1}{k+1}
\left(
\frac{\agrad({P_k}}
{\abs{\Bx}^{2k+1}}
+
{P_k}\frac{\agrad}
{\abs{\Bx}^{2k+1}}
\right) \\
&= 
\inv{k+1}
\left(
{P_k}(2k+1) \frac{\Ba \cdot \Bx}
{\abs{\Bx}^{2k+3}}
-\Bx^2 \frac{\agrad{P_k}}
{\abs{\Bx}^{2k+3}}
\right) \\
\end{align*}

Or
\begin{align*}
(k+1){P_{k+1}}
=
{P_k}(2k+1) {\Ba \cdot \Bx}
-\Bx^2 {\agrad{P_k}}
\end{align*}

Some of these have been calculated

\begin{align*}
P_0 &= 1 \\
P_1 &= \Ba \cdot \Bx \\
P_2 &= \half(3(\Ba \cdot \Bx)^2 -\Ba^2\Bx^2) \\
\end{align*}

And for the derivatives

\begin{align*}
\agrad P_0 &= 0 \\
\agrad P_1 &= \Ba^2 \\
\agrad P_2 &= \half((3)(2)(\Ba \cdot \Bx)\Ba^2 - 2\Ba^2\Bx \cdot \Ba) \\
           &= 2\Ba^2(\Bx \cdot \Ba) \\
\end{align*}

Using the recurrence relation one can calculate $P_3$ for example.

\begin{align*}
P_3
%(k+1){P_{k+1}} ; k=2
&=
(1/3)\left(
\frac{5}{2}(3(\Ba \cdot \Bx)^2 -\Ba^2\Bx^2)({\Ba \cdot \Bx})
- 2 \Bx^2 \Ba^2(\Bx \cdot \Ba) \right) \\
&=
(1/3) ({\Ba \cdot \Bx}) \left(
\frac{5}{2}(3(\Ba \cdot \Bx)^2 -\Ba^2\Bx^2)
- 2 \Bx^2 \Ba^2 \right) \\
&=
({\Ba \cdot \Bx}) \left( \frac{5}{2}((\Ba \cdot \Bx)^2 ) - 3/2 \Bx^2 \Ba^2 \right) \\
&=
\half({\Ba \cdot \Bx}) ( {5}(\Ba \cdot \Bx)^2 - 3 \Bx^2 \Ba^2 ) \\
\end{align*}

\section{ Putting things in standard Legendre polynomial form.}

This is still pretty laborious to calculate, especially because of not having a closed form recurrence
relation for $\agrad P_k$.  Let's relate these to the standard Legendre polynomial form.

Observe that we can write

\begin{align*}
P_0(\bxa) &= 1 \\
\frac{P_1(\bxa)}{\abs{\Bx} \abs{\Ba}} &= \costheta \\
\frac{P_2(\bxa)}{\abs{\Bx}^2 \abs{\Ba}^2} &= \half(3(\costheta)^2 - 1) \\
\frac{P_3(\bxa)}{\abs{\Bx}^3 \abs{\Ba}^3} &= \half ( {5}(\costheta)^3 - 3 {(\costheta)} ) \\
\end{align*}

With this scaling, we have the standard form for the Legendre polynomials, and can write

\[
\inv{\Bx-\Ba} = \inv{\abs{\Bx}}\left(
P_0 
+ \frac{\abs{\Ba}}{\abs{\Bx}} P_1(\costheta)
+ \left(\frac{\abs{\Ba}}{\abs{\Bx}}\right)^2 P_2(\costheta)
+ \left(\frac{\abs{\Ba}}{\abs{\Bx}}\right)^3 P_3(\costheta)
+ \cdots \right)
\]

\section{ Scaling standard form Legendre polynomials }

Since the odd Legendre polynomials have only odd terms and even have only even terms this allows for
the scaled form that NFCM uses.

\begin{align*}
P_0(\bxa) &= P_0(\costheta) \\
P_1(\bxa) &= \abs{\Bx}\abs{\Ba} P_1(\costheta) = \Ba \cdot \Bx \\
P_2(\bxa) &= \abs{\Bx}^2\abs{\Ba}^2 P_2(\costheta) = \half(3(\Ba \cdot \Bx)^2 - \Bx^2\Ba^2) \\
P_3(\bxa) &= \abs{\Bx}^3\abs{\Ba}^3 P_3(\costheta) = \half(5(\Ba \cdot \Bx)^3 - 3(\Ba \cdot \Bx) \Bx^2\Ba^2) \\
\end{align*}

Every term for the $k^{th}$ polynomial is a permutation of the geometric product $\Bx^k\Ba^k$.

This allows for writing some of these terms using the wedge product.  Using the product expansion:

\[
%\Ba \Bx \Bx \Ba = \Ba^2 \Bx^2 = (\Ba \cdot \Bx + \Ba \wedge \Bx)(\Bx \cdot \Ba + \Bx \wedge \Ba) = (\Ba \cdot \Bx)^2 - ( \Ba \wedge \Bx )^2
%\Ba^2 \Bx^2 = (\Ba \cdot \Bx)^2 - ( \Ba \wedge \Bx )^2
(\Ba \cdot \Bx)^2 = ( \Ba \wedge \Bx )^2 + \Ba^2 \Bx^2
\]

Thus we have:
\begin{align*}
P_2(\bxa)
&= (\Ba \cdot \Bx)^2 + \half(\Ba \wedge \Bx)^2 \\
&= (\Ba \cdot \Bx)^2 - \half\abs{\Ba \wedge \Bx}^2 \\
\end{align*}

This is nice geometrically since the directional dependence of this term on the co-linearity and 
perpendicularity of the vectors $\Ba$ and $\Bx$ is clear.

Doing the same for the $P_3$:

\begin{align*}
P_3(\bxa) &= (\Ba \cdot \Bx)\half(5(\Ba \cdot \Bx)^2 - 3\Bx^2\Ba^2) \\
          &= (\Ba \cdot \Bx)\half(2(\Ba \cdot \Bx)^2 + 3(\Ba \wedge \Bx)^2) \\
          &= (\Ba \cdot \Bx)((\Ba \cdot \Bx)^2 - \frac{3}{2}\abs{\Ba \wedge \Bx}^2) \\
\end{align*}

I suppose that one could get the same geometrical interpretation with a standard Legendre expansion in terms of $\costheta = cos(\theta)$ terms, by collect both $sin(\theta)$ and $cos(\theta)$ powers, but one
can see the power of writing things explicitly in terms of the original vectors.

\section{ Note on NFCM Legendre polynomial notation. }

In NFCM's slightly abusive notation $P_k$ was used with various meanings.  He wrote $P_k(\costheta) = \frac{P_k(\bxa)}{\abs{\Bx}^k \abs{\Ba}^k}$.

Note for example that the standard first degree Legendre polynomial $P_1(x) = x$ evaluated with a $\bxa$ value:

\begin{align*}
\inv {\abs{\Bx}\abs{\Ba}} {P_1(x) \vert_{x=\bxa}} &= \xcap \acap \\
&= \xcap \cdot \acap + \xcap \wedge \acap \\
\end{align*}

This has a bivector component in addition to the component identical to the standard Legendre polynomial
term (the first part).

By luck it happens that the scalar part of this equals $P_1(\costheta)$, but this
isn't the case for other terms.  Example, $P_2(\bxa)$:

\begin{align*}
{P_2(x) \vert_{x=\bxa}} 
&= \half( 3(\Bx \Ba)^2 - 1 ) \\
&= \half( 3(-\Ba \Bx + 2 \Ba \cdot \Bx )(\Bx \Ba) - 1 ) \\
&= \half( 3(-\Ba^2 \Bx^2 + 2(\Ba \cdot \Bx)^2 + 2(\Ba \cdot \Bx)(\Bx \wedge \Ba)) - 1 ) \\
&=  -(3/2)\Ba^2 \Bx^2 + 3(\Ba \cdot \Bx)^2 + 3(\Ba \cdot \Bx)(\Bx \wedge \Ba) - 1/2  \\
\end{align*}

Scaling this by $1/(\Ba^2\Bx^2)$ is
\[
-\frac{3}{2} + 3(\costheta)^2 + 3(\costheta)(\xcap \wedge \acap) - \inv{\Ba^2\Bx^2} \\
\]

The scalar part of this isn't anything recognizable.
