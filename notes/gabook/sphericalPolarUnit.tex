\documentclass[]{eliblog}

\usepackage{amsmath}
\usepackage{mathpazo}

%
% shorthand for bold symbols, convenient for vectors and matrices
%
\newcommand{\Ba}[0]{\mathbf{a}}
\newcommand{\Bb}[0]{\mathbf{b}}
\newcommand{\Bc}[0]{\mathbf{c}}
\newcommand{\Bd}[0]{\mathbf{d}}
\newcommand{\Be}[0]{\mathbf{e}}
\newcommand{\Bf}[0]{\mathbf{f}}
\newcommand{\Bg}[0]{\mathbf{g}}
\newcommand{\Bh}[0]{\mathbf{h}}
\newcommand{\Bi}[0]{\mathbf{i}}
\newcommand{\Bj}[0]{\mathbf{j}}
\newcommand{\Bk}[0]{\mathbf{k}}
\newcommand{\Bl}[0]{\mathbf{l}}
\newcommand{\Bm}[0]{\mathbf{m}}
\newcommand{\Bn}[0]{\mathbf{n}}
\newcommand{\Bo}[0]{\mathbf{o}}
\newcommand{\Bp}[0]{\mathbf{p}}
\newcommand{\Bq}[0]{\mathbf{q}}
\newcommand{\Br}[0]{\mathbf{r}}
\newcommand{\Bs}[0]{\mathbf{s}}
\newcommand{\Bt}[0]{\mathbf{t}}
\newcommand{\Bu}[0]{\mathbf{u}}
\newcommand{\Bv}[0]{\mathbf{v}}
\newcommand{\Bw}[0]{\mathbf{w}}
\newcommand{\Bx}[0]{\mathbf{x}}
\newcommand{\By}[0]{\mathbf{y}}
\newcommand{\Bz}[0]{\mathbf{z}}
\newcommand{\BA}[0]{\mathbf{A}}
\newcommand{\BB}[0]{\mathbf{B}}
\newcommand{\BC}[0]{\mathbf{C}}
\newcommand{\BD}[0]{\mathbf{D}}
\newcommand{\BE}[0]{\mathbf{E}}
\newcommand{\BF}[0]{\mathbf{F}}
\newcommand{\BG}[0]{\mathbf{G}}
\newcommand{\BH}[0]{\mathbf{H}}
\newcommand{\BI}[0]{\mathbf{I}}
\newcommand{\BJ}[0]{\mathbf{J}}
\newcommand{\BK}[0]{\mathbf{K}}
\newcommand{\BL}[0]{\mathbf{L}}
\newcommand{\BM}[0]{\mathbf{M}}
\newcommand{\BN}[0]{\mathbf{N}}
\newcommand{\BO}[0]{\mathbf{O}}
\newcommand{\BP}[0]{\mathbf{P}}
\newcommand{\BQ}[0]{\mathbf{Q}}
\newcommand{\BR}[0]{\mathbf{R}}
\newcommand{\BS}[0]{\mathbf{S}}
\newcommand{\BT}[0]{\mathbf{T}}
\newcommand{\BU}[0]{\mathbf{U}}
\newcommand{\BV}[0]{\mathbf{V}}
\newcommand{\BW}[0]{\mathbf{W}}
\newcommand{\BX}[0]{\mathbf{X}}
\newcommand{\BY}[0]{\mathbf{Y}}
\newcommand{\BZ}[0]{\mathbf{Z}}

\newcommand{\Bzero}[0]{\mathbf{0}}
\newcommand{\Btheta}[0]{\boldsymbol{\theta}}
\newcommand{\Btau}[0]{\boldsymbol{\tau}}
\newcommand{\Bomega}[0]{\boldsymbol{\omega}}

%
% shorthand for unit vectors
%
\newcommand{\acap}[0]{\hat{\Ba}}
\newcommand{\bcap}[0]{\hat{\Bb}}
\newcommand{\ccap}[0]{\hat{\Bc}}
\newcommand{\dcap}[0]{\hat{\Bd}}
\newcommand{\ecap}[0]{\hat{\Be}}
\newcommand{\fcap}[0]{\hat{\Bf}}
\newcommand{\gcap}[0]{\hat{\Bg}}
\newcommand{\hcap}[0]{\hat{\Bh}}
\newcommand{\icap}[0]{\hat{\Bi}}
\newcommand{\jcap}[0]{\hat{\Bj}}
\newcommand{\kcap}[0]{\hat{\Bk}}
\newcommand{\lcap}[0]{\hat{\Bl}}
\newcommand{\mcap}[0]{\hat{\Bm}}
\newcommand{\ncap}[0]{\hat{\Bn}}
\newcommand{\ocap}[0]{\hat{\Bo}}
\newcommand{\pcap}[0]{\hat{\Bp}}
\newcommand{\qcap}[0]{\hat{\Bq}}
\newcommand{\rcap}[0]{\hat{\Br}}
\newcommand{\scap}[0]{\hat{\Bs}}
\newcommand{\tcap}[0]{\hat{\Bt}}
\newcommand{\ucap}[0]{\hat{\Bu}}
\newcommand{\vcap}[0]{\hat{\Bv}}
\newcommand{\wcap}[0]{\hat{\Bw}}
\newcommand{\xcap}[0]{\hat{\Bx}}
\newcommand{\ycap}[0]{\hat{\By}}
\newcommand{\zcap}[0]{\hat{\Bz}}
\newcommand{\thetacap}[0]{\hat{\Btheta}}

%
% to write R^n and C^n in a distinguishable fashion.  Perhaps change this
% to the double lined characters upon figuring out how to do so.
%
\newcommand{\C}[1]{$\mathbb{C}^{#1}$}
\newcommand{\R}[1]{$\mathbb{R}^{#1}$}

%
% various generally useful helpers
%

% derivative of #1 wrt. #2:
\newcommand{\D}[2] {\frac {d#2} {d#1}}

\newcommand{\inv}[1]{\frac{1}{#1}}
\newcommand{\cross}[0]{\times}

\newcommand{\abs}[1]{\lvert{#1}\rvert}
\newcommand{\norm}[1]{\lVert{#1}\rVert}
\newcommand{\innerprod}[2]{\langle{#1}, {#2}\rangle}
\newcommand{\dotprod}[2]{{#1} \cdot {#2}}
\newcommand{\bdotprod}[2]{\left({#1} \cdot {#2}\right)}
\newcommand{\crossprod}[2]{{#1} \cross {#2}}
\newcommand{\tripleprod}[3]{\dotprod{\left(\crossprod{#1}{#2}\right)}{#3}}

\DeclareMathOperator{\Proj}{Proj}
\DeclareMathOperator{\Span}{span}
\DeclareMathOperator{\Sgn}{sgn}
\DeclareMathOperator{\Area}{Area}
\DeclareMathOperator{\Volume}{Volume}

%
% A few miscellaneous things specific to this document
%
\newcommand{\crossop}[1]{\crossprod{#1}{}}

% R2 vector.
\newcommand{\VectorTwo}[2]{
\begin{bmatrix}
 {#1} \\
 {#2}
\end{bmatrix}
}

\newcommand{\VectorN}[1]{
\begin{bmatrix}
{#1}_1 \\
{#1}_2 \\
\vdots \\
{#1}_N \\
\end{bmatrix}
}

\newcommand{\DETuvij}[4]{
\begin{vmatrix}
 {#1}_{#3} & {#1}_{#4} \\
 {#2}_{#3} & {#2}_{#4}
\end{vmatrix}
}

\newcommand{\DETuvwijk}[6]{
\begin{vmatrix}
 {#1}_{#4} & {#1}_{#5} & {#1}_{#6} \\
 {#2}_{#4} & {#2}_{#5} & {#2}_{#6} \\
 {#3}_{#4} & {#3}_{#5} & {#3}_{#6}
\end{vmatrix}
}

\newcommand{\DETuvwxijkl}[8]{
\begin{vmatrix}
 {#1}_{#5} & {#1}_{#6} & {#1}_{#7} & {#1}_{#8} \\
 {#2}_{#5} & {#2}_{#6} & {#2}_{#7} & {#2}_{#8} \\
 {#3}_{#5} & {#3}_{#6} & {#3}_{#7} & {#3}_{#8} \\
 {#4}_{#5} & {#4}_{#6} & {#4}_{#7} & {#4}_{#8} \\
\end{vmatrix}
}

%\newcommand{\DETuvwxyijklm}[10]{
%\begin{vmatrix}
% {#1}_{#6} & {#1}_{#7} & {#1}_{#8} & {#1}_{#9} & {#1}_{#10} \\
% {#2}_{#6} & {#2}_{#7} & {#2}_{#8} & {#2}_{#9} & {#2}_{#10} \\
% {#3}_{#6} & {#3}_{#7} & {#3}_{#8} & {#3}_{#9} & {#3}_{#10} \\
% {#4}_{#6} & {#4}_{#7} & {#4}_{#8} & {#4}_{#9} & {#4}_{#10} \\
% {#5}_{#6} & {#5}_{#7} & {#5}_{#8} & {#5}_{#9} & {#5}_{#10}
%\end{vmatrix}
%}

% R3 vector.
\newcommand{\VectorThree}[3]{
\begin{bmatrix}
 {#1} \\
 {#2} \\
 {#3}
\end{bmatrix}
}



\author{Peeter Joot}
\email{peeter.joot@gmail.com}


\chapter{Spherical Polar unit vectors in exponential form.}
\label{chap:sphericalPolarUnit}
%\useCCL
\blogpage{http://sites.google.com/site/peeterjoot/math2009/sphericalPolarUnit.pdf}
\date{Sept 20, 2009}
\revisionInfo{$RCSfile: sphericalPolarUnit.tex,v $ Last $Revision: 1.5 $ $Date: 2009/10/09 23:07:01 $}

\beginArtWithToc
%\beginArtNoToc

\section{Motivation}

In \chapcite{qmAngularMom} I blundered on a particularly concise exponential non-coordinate form for the unit vectors in a spherical polar coordinate system.  For future reference outside of a quantum mechanical context here is a separate and more concise iteration of these results.

\section{The rotation and notation.}

The spherical polar rotor is a composition of rotations, expressed as half angle exponentials.  Following the normal physics conventions we first apply a $z,x$ plane rotation by angle theta, then an $x,y$ plane rotation by angle $\phi$.  This produces the rotor

\begin{align}\label{eqn:sphericalPolarUnit:foo1}
R = e^{\Be_{31}\theta/2} e^{\Be_{12}\phi/2}
\end{align}

Our triplet of Cartesian unit vectors is therefore rotated as

\begin{align}\label{eqn:sphericalPolarUnit:foo2}
\begin{pmatrix}
\rcap \\
\thetacap \\
\phicap \\
\end{pmatrix}
&=
\tilde{R}
\begin{pmatrix}
\Be_3 \\
\Be_1 \\
\Be_2 \\
\end{pmatrix}
R
\end{align}

In the quantum mechanical context it was convenient to denote the $x,y$ plane unit bivector with the imaginary symbol

\begin{align}\label{eqn:sphericalPolarUnit:foo3}
i = \Be_1 \Be_2
\end{align}

reserving for the spatial pseudoscalar the capital

\begin{align}\label{eqn:sphericalPolarUnit:foo4}
I = \Be_1 \Be_2 \Be_3 = \rcap \thetacap \phicap = i \Be_3
\end{align}

Note the characteristic differences between these two ``imaginaries''.  The planar quantity $i = \Be_1 \Be_2$ commutes with $\Be_3$, but anticommutes with either $\Be_1$ or $\Be_2$.  On the other hand the spatial pseudoscalar $I$ commutes with any vector, bivector or trivector in the algebra.

\section{Application of the rotor.  The spherical polar unit vectors.}

Having fixed notation, lets apply the rotation to each of the unit vectors in sequence, starting with the calculation for $\phicap$.  This is

\begin{align*}
\phicap 
&= e^{-i \phi/2} e^{-\Be_{31}\theta/2} (\Be_2) e^{\Be_{31}\theta/2} e^{i\phi/2} \\
&= \Be_2 e^{i\phi} 
\end{align*}

Here, since $\Be_2$ commutes with the rotor bivector $\Be_3 \Be_1$ the innermost exponentials cancel, leaving just the $i\phi$ rotation.  For $\rcap$ it is a bit messier, and we have

\begin{align*}
\rcap 
&= e^{-i \phi/2} e^{-\Be_{31}\theta/2} (\Be_3) e^{\Be_{31}\theta/2} e^{i\phi/2} \\
&= e^{-i \phi/2} \Be_3 e^{\Be_{31}\theta} e^{i\phi/2} \\
&= e^{-i \phi/2} (\Be_3 \cos\theta + \Be_1 \sin\theta) e^{i\phi/2} \\
&= \Be_3 \cos\theta + \Be_1 \sin\theta e^{i\phi} \\
&= \Be_3 \cos\theta + \Be_1 \Be_2 \sin\theta \Be_2 e^{i\phi} \\
&= \Be_3 \cos\theta + i \sin\theta \phicap \\
&= \Be_3 (\cos\theta + \Be_3 i \sin\theta \phicap) \\
&= \Be_3 e^{I\phicap\theta} 
\end{align*}

Finally for $\thetacap$, we have a similar messy expansion

\begin{align*}
\thetacap 
&= e^{-i \phi/2} e^{-\Be_{31}\theta/2} (\Be_1) e^{\Be_{31}\theta/2} e^{i\phi/2} \\
&= e^{-i \phi/2} \Be_1 e^{\Be_{31}\theta} e^{i\phi/2} \\
&= e^{-i \phi/2} (\Be_1 \cos\theta - \Be_3 \sin\theta) e^{i\phi/2} \\
&= \Be_1 \cos\theta e^{i\phi} - \Be_3 \sin\theta \\
&= i \cos\theta \Be_2 e^{i\phi} - \Be_3 \sin\theta \\
&= i \phicap \cos\theta - \Be_3 \sin\theta \\
&= i \phicap (\cos\theta + \phicap i \Be_3 \sin\theta) \\
&= i \phicap e^{I\phicap\theta}
\end{align*}

Summarizing the three of these relations we have for the rotated unit vectors

\begin{align}\label{eqn:sphericalPolarUnit:foo5}
\rcap &= \Be_3 e^{I \phicap \theta} \\
\thetacap &= i \phicap e^{I \phicap \theta} \\
\phicap &= \Be_2 e^{i\phi} 
\end{align}

and in particular for the radial position vector from the origin, rotating from the polar axis, we have

\begin{align}\label{eqn:sphericalPolarUnit:foo6}
\Bx &= r \rcap = r \Be_3 e^{I\phicap \theta}
\end{align}

Compare this to the coordinate representation

\begin{align}\label{eqn:sphericalPolarUnit:foo7}
\Bx = r(\sin\theta \cos\phi, \sin\theta \sin\phi, \cos\theta)
\end{align}

it is not initially obvious that these $\theta$ and $\phi$ rotations admit such a tidy factorization.  In retrospect, this doesn't seem so suprising, since we can form a quaternion product that acts via multiplication to map a vector to a rotated position.  In fact those quaternions, acting from the right on the initial vectors are 

\begin{align}\label{eqn:sphericalPolarUnit:foo8}
\Be_3 &\rightarrow \rcap = \Be_3 \bigl( e^{I \phicap \theta} \bigr) \\
\Be_1 &\rightarrow \thetacap = \Be_1 \bigl( \Be_2 \phicap e^{I \phicap \theta} \bigr) \\
\Be_2 &\rightarrow \phicap = \Be_2 \bigl( e^{i\phi} \bigr)
\end{align}

FIXME: it should be possible to reduce the quaternion that rotates $\Be_1 \rightarrow \thetacap$ to a single exponential.  What is it?

\section{A consistency check.}

We expect that the dot product between a north pole oriented vector $\Bz = Z \Be_3$ and the spherically polar rotated vector $\Bx = r \Be_3 e^{I\phicap \theta}$ is just

\begin{align}\label{eqn:sphericalPolarUnit:foo13}
\Bx \cdot \Bz = Z r \cos\theta
\end{align}

Lets verify this

\begin{align*}
\Bx \cdot \Bz 
&= 
\gpgradezero{ Z \Be_3 \Be_3 r e^{I\phicap \theta}} \\
&= 
Z r \gpgradezero{ \cos\theta + I \phicap \sin\theta} \\
&= 
Z r \cos\theta &\quad\square
\end{align*}

\section{A small example application.}

Let's use these results to compute the spherical polar volume element.  Pictorially this can be read off simply from a diagram.  If one is less trusting of pictorial means (or want a method more generally applicable), we can also do this particular calculation algebraically, expanding the determinant of partials

\begin{align}\label{eqn:sphericalPolarUnit:foo9}
\begin{vmatrix}
\frac{\partial \Bx}{\partial r} & \frac{\partial \Bx}{\partial \theta} & \frac{\partial \Bx}{\partial \phi} \\
\end{vmatrix} dr d\theta d\phi
&=
\begin{vmatrix}
\sin\theta \cos\phi & \cos\theta \cos\phi & -\sin\theta \sin\phi \\
\sin\theta \sin\phi & \cos\theta \sin\phi & \sin\theta \cos\phi \\
\cos\theta          & -\sin\theta         & 0                   \\
\end{vmatrix} r^2 dr d\theta d\phi
\end{align}

One can chug through the trig reduction for this determinant with not too much trouble, but it isn't particularly fun.

Now compare to the same calculation proceeding directly with the exponential form.  We do still need to compute the partials

\begin{align*}
\frac{\partial \Bx}{\partial r} = \rcap
\end{align*}

\begin{align*}
\frac{\partial \Bx}{\partial \theta} 
&= r \Be_3 \frac{\partial }{\partial \theta} e^{I\phicap \theta} \\
&= r \rcap I \phicap \\
&= r \rcap (\rcap \thetacap \phicap) \phicap \\
&= r \thetacap
\end{align*}

\begin{align*}
\frac{\partial \Bx}{\partial \phi} 
&= r \Be_3 \frac{\partial }{\partial \phi} (\cos\theta + I\phicap \sin\theta) \\
&= -r \Be_3 I i \phicap \sin\theta \\
&= r \phicap \sin\theta 
\end{align*}

So the area element, the oriented area of the parallelogram between the two vectors $d\theta \partial \Bx/\partial \theta$, and $d\phi \partial \Bx/\partial \phi$ on the spherical surface at radius $r$ is

\begin{align}\label{eqn:sphericalPolarUnit:foo10}
d\BS = \left(d\theta \frac{\partial \Bx}{\partial \theta}\right) \wedge \left( d\phi \frac{\partial \Bx}{\partial \phi} \right) 
= r^2 \thetacap \phicap \sin\theta d\theta d\phi
\end{align}

and the volume element in trivector form is just the product
\begin{align}\label{eqn:sphericalPolarUnit:foo11}
d\BV = \left(dr\frac{\partial \Bx}{\partial r}\right) \wedge d\BS
= r^2 \sin\theta I dr d\theta d\phi
\end{align}

\EndArticle
%\EndNoBibArticle
