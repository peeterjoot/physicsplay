\chapter*{Preface}\normalsize
  \addcontentsline{toc}{chapter}{Preface}

This is a somewhat hodge podge compilation of Geometric Algebra related notes on mathematics and physics.

I have little formal education in physics, having been schooled in, as well as employed in, software engineering work.  My engineering undergrad did supply enough basic background in elementary mechanics, electromagnetism and mathematics, that I've been able to pursue a part time home schooling project furthering my understanding of physics.  This is a subject that I find fascinating and enjoyable.

In addition to enjoying physics I've found this study to be complemented nicely by Geometric (or Clifford) Algebra.  I am studying both the physics and the algebra concurrently.  The use of this algebra in physics is still in its infancy.  Some use of Geometric Algebra can be found in advanced treatments like the work of the Cambridge group (\cite{doran2003gap}), there is much less that is easily accessible to someone with undergrad level education like myself.  Even a text like Hestenes's New Foundations (\cite{hestenes1999nfc}), which has a more elementary target audience is fairly difficult to read.  Texts like these leave one having to do a fair amount of figure it out yourself, and this prompted a fair number of the notes in this compilation.

Most of what appear here as chapters were originally disjoint standalone notes.  I eventually accumulated enough of these individual notes that assembling them into a bookish form made some sense, even if only for personal organizational purposes.  Since my original notes were disconnected, this assembled form is not necessarily in a logical sequence, so in some cases reading in a chronological sequence (\ref{chap:Cronology}) may be helpful.

What can be found here is an exploratory record of learning.  This is much different than what you'll find in a polished textbook where the author knows the subject like the back of his hand.  A reader here will not be surprised to find that I don't always know where I am going or what the final result will be ahead of time.  I sometimes hit dead ends, mistakes, or unproductive paths.  These are not always deleted since I feel there is personal educational value to exploring the thought process.  I have at least tried to go back and correct (or at least point out) errors if I realize I've made them.  There will also be a fair amount of repetition that one would not find in a carefully crafted text.

I have found the process of attempting to write notes of what I am learning to be very educational.  This process feeds back on itself, so I learn by writing as well as writing on what I learn.  This process often highlights holes in my understanding or errors in my original messy paper scribblings.  I also find thta an attempt to coherently write what one has learned makes it easier to move on, and and also points out subjects and ideas worthy of followup.

I can't promise that I explain things in a way that is good for anybody else.  My audience is essentially myself as I existed at the time of writing, so the prerequisites, both for the mathematics and the physics, evolve continually. 

I also recognize that in many instances these notes are unfortunately not standalone.  What I cover is often done to clarify aspects of the texts I personally own that didn't happen to have explanations containing the level of detail I desired.  Because I am trying to remedy aspects that I did understand and not the parts that I did not, the complete sort of content required for others to learn from in isolation may very well be missing.  I apologize for that, but make these notes available anyhow since some small subset may still have value to others too.

If you have managed to both somehow find this piece of math and physics play, and also manage to get something out of it please send me an email.  If anybody other than myself gets some value out of these notes, I'd be curious to know who you are.

Additionally, if you have any specific comments, questions, suggestions, and especially observations of errors and typos please let me know.  I know I am also guilty of poor grammar, spelling, run on sentences, letting the math talk instead of using words, and not elaborating on the physics behind the math.  I won't be offended if these are pointed out to me too.

Peeter Joot  \quad peeter.joot@gmail.com 
