This is an informal compilation of Geometric Algebra related notes on
mathematics and physics.

What can be found here is an exploratory record of learning.  This is much different than what you'll find in a polished textbook where the author knows the subject like the back of his hand.  

A reader here should not be surprised to find that I don't always know where I am going or what the final result will be ahead of time.  I sometimes hit dead ends, mistakes, or unproductive paths.  These are not always deleted since I feel there is personal educational value to exploring the thought process.  I have at least tried to go back and correct (or at least point out) errors if I realize I've made them.

I end up writing up notes on what I have learned, or write as I am learning, or to learn by writing (some hybrid combination of all of these).
I have found for myself the process of writing up 
my thoughts on a topic as if explaining to somebody to be very educational.
%This has led many times to additional understanding interesting or discovering that something was not actually as well understood as initially believed.

I can't promise that I explain things in a way that is good for anybody else.  My audience is essentially myself as I existed at the time of writing.  The prerequisites, both for the mathematics and the physics, evolve continually.  What I cover is often done to clarify aspects of the texts I personally own that didn't happen to explain with the level of detail desired when I first read them.  Because I am trying to remedy aspects that I did understand and not the parts that I understood, the complete sort of content required for others to learn from in isolation may very well be missing.

%This is a selfish work.  Each of the original notes was written primarily
%%as a study aid yself.  
%Even this compilation, a concatonation
%of the original notes, was created for my own benefit, so that
%I can find stuff that I've thought about more easily.

%That said, there's enough here that may be of interest to others that
%I thought I'd make it accessible.  I know it doesn't stand on its own,
%it has duplication and reworking of topics that you wouldn't find in a
%polished text, has rigor only as much as I felt like at the time, has
%an ambiguous target audience,
%% (each part written for me and my level of
%%understanding at the time of writing), 
%has many questions, likely has
%many errors, has horrid grammar and spelling and run on sentances (not
%many as bad as this one),
%is not structured in a
%way that somebody completely new to the subject would find consumable,
%has prereqs that are missing, and many other faults especially as a
%whole.
%
%A fair amount of this hodge podge set of Geometric Algebra notes 
%was in response to study of \cite{doran2003gap} and
%\cite{hestenes1999nfc}.  The first of which has many particularily hard to 
%grasp elements since the physics prerequisites are high.  The book itself
%is not self contained since those prerequisites are assumed (it is after
%all a book for Physisists, not systems level software developers with 
%curiousity about physics.)
%
%%However, ... if you are trying to work through a text like
%%Doran/Lasenby's GA for Physicists, that is quite tough due to brevity
%%and assumed knowledge (at least for computer-programmer
%%physicist-wantabee's like me), then there may be parts of this
%%compilation that has value.
%
%Since the original notes of this compilation were disconnected the
%assembled form is not neccessarily in a logical sequence.  There will also
%be a fair amount of repetition that one would not find in a carefully
%crafted text.  Hopefully some subset of this material is useful
%even without the original texts that motivated many of the topics.

I'm open to any comments, questions, suggestions and feedback,
observations of errors and typos, or anything else.

Peeter Joot  \quad peeter.joot@gmail.com 
