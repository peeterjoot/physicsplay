\chapter*{Motivation and Goals}\normalsize
  \addcontentsline{toc}{chapter}{Motivation and Goals}

This is a somewhat hodge podge compilation of Geometric (or Clifford) Algebra related notes on mathematics and Physics.

I have little formal education in Physics, having been schooled in, as well as employed in, software engineering work.  My engineering undergrad education supplied me basic background in elementary mechanics, electromagnetism and mathematics.  This has been enough to allow me to pursue a part time ``home schooling'' project furthering my understanding of Physics.  The Physics as well as the Geometric Algebra (GA) used to explore Physics are both subjects that I find fascinating, enjoyable and complementary.  These notes are the product of concurrent study of both.

The use of this algebra in Physics could be said to be still in its infancy.  There is a fair amount Geometric Algebra in advanced treatments like the work of the Cambridge group (\citep{doran2003gap}).  There is much less that is easily accessible to someone with undergrad level education.  Even a text like Hestenes's New Foundations (\citep{hestenes1999nfc}), which has a more elementary target audience is fairly difficult to read.  Reading these leaves one having to do a fair amount of figure it out yourself.  This prompted a fair number of the notes in this compilation.  Somebody who has studied Physics instead of engineering would probably be better equipt for the subject as it is currently presented.

Most of what appear here as chapters were originally disjoint standalone notes.  I eventually accumulated enough of these individual notes that assembling them into a bookish form made some sense, even if only for personal organizational purposes.  Since my original notes were disconnected, this assembled form is not necessarily in a logical sequence, so in some cases reading in a chronological sequence (\chapcite{Cronology}) may be helpful.

What can be found here is an exploratory record of learning.  I have found the process of attempting to write notes of what I am learning to be very educational.  This process feeds back on itself, so I learn by writing as well as writing on what I learn.  This process often highlights holes in my understanding or errors in my original messy paper scribblings.  I find that an attempt to coherently write what one has learned makes it easier to move on.  This summarization exersize is also an excellent way to observe further topics and ideas worthy of followup.

Because of the journaling nature of many of these notes, a reader will find that I don't always know where I am going or what the final result will be ahead of time.  This is much different than what you'll find in a polished textbook where the author knows the subject like the back of his hand.  I sometimes hit dead ends, mistakes, or unproductive paths.  Not all of these have been removed (although mistakes should at least be pointed out if I am aware of them).  I have a tendancy to rework a topic from scratch if I did not like my initial approach, so one will find in these notes an unfortunate amount of repetition that would not be found in a carefully crafted text.  Eventually I would like to revisit much of this too verbose and too lengthly collection and shorten it to make it more useful.  It is more work to make a good short book, than to make a long poor one.

I can't promise that I explain things in a way that is good for anybody else.  My audience is essentially myself as I existed at the time of writing, so the prerequisites, both for the mathematics and the Physics, evolve continually. 

I recognize that in many instances these notes are unfortunately not standalone.  What I cover is often done to clarify aspects of the texts I personally own that didn't happen to have explanations containing the level of detail I desired.  Because I am trying to remedy aspects that I did understand and not the parts that I did not, the complete sort of content required for others to learn from in isolation may very well be missing.  I apologize for that, but make these notes available anyhow.  Despite deficiencies, due to the size of this compilation, I have a statistical expectation that there is some subset of these notes that somebody other than myself can get some value from (if you can find it!)

If you have managed to somehow blunder upon this piece of math and Physics play, and have gotten something out of it please feel free to send me an email.  I would be curious to know your study goals and what you are studying.

Additionally, if you have any specific comments, questions, suggestions, and especially observations of errors and typos please let me know.  I know I am guilty of poor grammar, spelling, run on sentences, letting the math talk instead of using words, and not elaborating on the Physics behind the math.  I won't be offended if these are pointed out to me, no matter how trivial.

Peeter Joot  \quad peeter.joot@gmail.com 
