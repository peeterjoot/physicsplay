\documentclass{article}      % Specifies the document class

\usepackage{amsmath}
\usepackage{mathpazo}

%
% shorthand for bold symbols, convenient for vectors and matrices
%
\newcommand{\Ba}[0]{\mathbf{a}}
\newcommand{\Bb}[0]{\mathbf{b}}
\newcommand{\Bc}[0]{\mathbf{c}}
\newcommand{\Bd}[0]{\mathbf{d}}
\newcommand{\Be}[0]{\mathbf{e}}
\newcommand{\Bf}[0]{\mathbf{f}}
\newcommand{\Bg}[0]{\mathbf{g}}
\newcommand{\Bh}[0]{\mathbf{h}}
\newcommand{\Bi}[0]{\mathbf{i}}
\newcommand{\Bj}[0]{\mathbf{j}}
\newcommand{\Bk}[0]{\mathbf{k}}
\newcommand{\Bl}[0]{\mathbf{l}}
\newcommand{\Bm}[0]{\mathbf{m}}
\newcommand{\Bn}[0]{\mathbf{n}}
\newcommand{\Bo}[0]{\mathbf{o}}
\newcommand{\Bp}[0]{\mathbf{p}}
\newcommand{\Bq}[0]{\mathbf{q}}
\newcommand{\Br}[0]{\mathbf{r}}
\newcommand{\Bs}[0]{\mathbf{s}}
\newcommand{\Bt}[0]{\mathbf{t}}
\newcommand{\Bu}[0]{\mathbf{u}}
\newcommand{\Bv}[0]{\mathbf{v}}
\newcommand{\Bw}[0]{\mathbf{w}}
\newcommand{\Bx}[0]{\mathbf{x}}
\newcommand{\By}[0]{\mathbf{y}}
\newcommand{\Bz}[0]{\mathbf{z}}
\newcommand{\BA}[0]{\mathbf{A}}
\newcommand{\BB}[0]{\mathbf{B}}
\newcommand{\BC}[0]{\mathbf{C}}
\newcommand{\BD}[0]{\mathbf{D}}
\newcommand{\BE}[0]{\mathbf{E}}
\newcommand{\BF}[0]{\mathbf{F}}
\newcommand{\BG}[0]{\mathbf{G}}
\newcommand{\BH}[0]{\mathbf{H}}
\newcommand{\BI}[0]{\mathbf{I}}
\newcommand{\BJ}[0]{\mathbf{J}}
\newcommand{\BK}[0]{\mathbf{K}}
\newcommand{\BL}[0]{\mathbf{L}}
\newcommand{\BM}[0]{\mathbf{M}}
\newcommand{\BN}[0]{\mathbf{N}}
\newcommand{\BO}[0]{\mathbf{O}}
\newcommand{\BP}[0]{\mathbf{P}}
\newcommand{\BQ}[0]{\mathbf{Q}}
\newcommand{\BR}[0]{\mathbf{R}}
\newcommand{\BS}[0]{\mathbf{S}}
\newcommand{\BT}[0]{\mathbf{T}}
\newcommand{\BU}[0]{\mathbf{U}}
\newcommand{\BV}[0]{\mathbf{V}}
\newcommand{\BW}[0]{\mathbf{W}}
\newcommand{\BX}[0]{\mathbf{X}}
\newcommand{\BY}[0]{\mathbf{Y}}
\newcommand{\BZ}[0]{\mathbf{Z}}

\newcommand{\Bzero}[0]{\mathbf{0}}
\newcommand{\Btheta}[0]{\boldsymbol{\theta}}
\newcommand{\Btau}[0]{\boldsymbol{\tau}}
\newcommand{\Bomega}[0]{\boldsymbol{\omega}}

%
% shorthand for unit vectors
%
\newcommand{\acap}[0]{\hat{\Ba}}
\newcommand{\bcap}[0]{\hat{\Bb}}
\newcommand{\ccap}[0]{\hat{\Bc}}
\newcommand{\dcap}[0]{\hat{\Bd}}
\newcommand{\ecap}[0]{\hat{\Be}}
\newcommand{\fcap}[0]{\hat{\Bf}}
\newcommand{\gcap}[0]{\hat{\Bg}}
\newcommand{\hcap}[0]{\hat{\Bh}}
\newcommand{\icap}[0]{\hat{\Bi}}
\newcommand{\jcap}[0]{\hat{\Bj}}
\newcommand{\kcap}[0]{\hat{\Bk}}
\newcommand{\lcap}[0]{\hat{\Bl}}
\newcommand{\mcap}[0]{\hat{\Bm}}
\newcommand{\ncap}[0]{\hat{\Bn}}
\newcommand{\ocap}[0]{\hat{\Bo}}
\newcommand{\pcap}[0]{\hat{\Bp}}
\newcommand{\qcap}[0]{\hat{\Bq}}
\newcommand{\rcap}[0]{\hat{\Br}}
\newcommand{\scap}[0]{\hat{\Bs}}
\newcommand{\tcap}[0]{\hat{\Bt}}
\newcommand{\ucap}[0]{\hat{\Bu}}
\newcommand{\vcap}[0]{\hat{\Bv}}
\newcommand{\wcap}[0]{\hat{\Bw}}
\newcommand{\xcap}[0]{\hat{\Bx}}
\newcommand{\ycap}[0]{\hat{\By}}
\newcommand{\zcap}[0]{\hat{\Bz}}
\newcommand{\thetacap}[0]{\hat{\Btheta}}

%
% to write R^n and C^n in a distinguishable fashion.  Perhaps change this
% to the double lined characters upon figuring out how to do so.
%
\newcommand{\C}[1]{$\mathbb{C}^{#1}$}
\newcommand{\R}[1]{$\mathbb{R}^{#1}$}

%
% various generally useful helpers
%

% derivative of #1 wrt. #2:
\newcommand{\D}[2] {\frac {d#2} {d#1}}

\newcommand{\inv}[1]{\frac{1}{#1}}
\newcommand{\cross}[0]{\times}

\newcommand{\abs}[1]{\lvert{#1}\rvert}
\newcommand{\norm}[1]{\lVert{#1}\rVert}
\newcommand{\innerprod}[2]{\langle{#1}, {#2}\rangle}
\newcommand{\dotprod}[2]{{#1} \cdot {#2}}
\newcommand{\bdotprod}[2]{\left({#1} \cdot {#2}\right)}
\newcommand{\crossprod}[2]{{#1} \cross {#2}}
\newcommand{\tripleprod}[3]{\dotprod{\left(\crossprod{#1}{#2}\right)}{#3}}

\DeclareMathOperator{\Proj}{Proj}
\DeclareMathOperator{\Span}{span}
\DeclareMathOperator{\Sgn}{sgn}
\DeclareMathOperator{\Area}{Area}
\DeclareMathOperator{\Volume}{Volume}

%
% A few miscellaneous things specific to this document
%
\newcommand{\crossop}[1]{\crossprod{#1}{}}

% R2 vector.
\newcommand{\VectorTwo}[2]{
\begin{bmatrix}
 {#1} \\
 {#2}
\end{bmatrix}
}

\newcommand{\VectorN}[1]{
\begin{bmatrix}
{#1}_1 \\
{#1}_2 \\
\vdots \\
{#1}_N \\
\end{bmatrix}
}

\newcommand{\DETuvij}[4]{
\begin{vmatrix}
 {#1}_{#3} & {#1}_{#4} \\
 {#2}_{#3} & {#2}_{#4}
\end{vmatrix}
}

\newcommand{\DETuvwijk}[6]{
\begin{vmatrix}
 {#1}_{#4} & {#1}_{#5} & {#1}_{#6} \\
 {#2}_{#4} & {#2}_{#5} & {#2}_{#6} \\
 {#3}_{#4} & {#3}_{#5} & {#3}_{#6}
\end{vmatrix}
}

\newcommand{\DETuvwxijkl}[8]{
\begin{vmatrix}
 {#1}_{#5} & {#1}_{#6} & {#1}_{#7} & {#1}_{#8} \\
 {#2}_{#5} & {#2}_{#6} & {#2}_{#7} & {#2}_{#8} \\
 {#3}_{#5} & {#3}_{#6} & {#3}_{#7} & {#3}_{#8} \\
 {#4}_{#5} & {#4}_{#6} & {#4}_{#7} & {#4}_{#8} \\
\end{vmatrix}
}

%\newcommand{\DETuvwxyijklm}[10]{
%\begin{vmatrix}
% {#1}_{#6} & {#1}_{#7} & {#1}_{#8} & {#1}_{#9} & {#1}_{#10} \\
% {#2}_{#6} & {#2}_{#7} & {#2}_{#8} & {#2}_{#9} & {#2}_{#10} \\
% {#3}_{#6} & {#3}_{#7} & {#3}_{#8} & {#3}_{#9} & {#3}_{#10} \\
% {#4}_{#6} & {#4}_{#7} & {#4}_{#8} & {#4}_{#9} & {#4}_{#10} \\
% {#5}_{#6} & {#5}_{#7} & {#5}_{#8} & {#5}_{#9} & {#5}_{#10}
%\end{vmatrix}
%}

% R3 vector.
\newcommand{\VectorThree}[3]{
\begin{bmatrix}
 {#1} \\
 {#2} \\
 {#3}
\end{bmatrix}
}


\newcommand{\gpgrade}[2] {{\left\langle{{#1}}\right\rangle}_{#2}}

%
% The real thing:
%

                             % The preamble begins here.
\title{} % Declares the document's title.
\author{Peeter Joot}         % Declares the author's name.
%\date{}        % Deleting this command produces today's date.

\begin{document}             % End of preamble and beginning of text.

\maketitle{}

\section{ General triple product reduction formula. }

Consideration of the reciprocal frame bivector decomposition required the following identity

\begin{equation}
(\BA_a \wedge \BA_b) \cdot \BA_c =
\BA_a \cdot (\BA_b \cdot \BA_c)
\end{equation}

This holds when $a + b \le c$.  Similar equations for vector/blade/blade reduction can be found in NFCM, but intuition let me to believe this generalized simply.

To prove this use the definition of the generalized dot product of two blades:

\begin{align*}
(\BA_a \wedge \BA_b) \cdot \BA_c
&= \gpgrade{ (\BA_a \wedge \BA_b) \BA_c }{\abs{c-(a+b)}} \\
\end{align*}

%Now if $\Bx = \BA_a$ is a vector, one can write $\Bx \wedge \BA_b = \Bx\BA_b - \Bx \cdot \BA_b$, and one can use this to show in that case:
%
%\[
%\gpgrade{ (\Bx \wedge \BA_b) \BA_c }{\abs{c-(a+b)}}
%=
%\gpgrade{ \Bx \BA_b \BA_c }{\abs{c-(a+b)}}
%\]
%
%The dot product term is a $b-1$ grade vector, so the lowest grade product with $\BA_c$ has grade $c-b+1$.  Since we want the $c-b-1 < c-b+1$ term, so this
%dot product doesn't contribute to the triple product.
%
%For the general case we don't have such a wedge product decomposition relationship, but can write
%

Note first that without loss of generality we can restrict subsequent discussion 
to the $b >= a$ case (reverse the wedge for the $b < a$ case.)

\begin{align*}\label{eqn:bladewedge}
\BA_a \wedge \BA_b 
&= \BA_a \BA_b - \sum_{i=\abs{b-a},i+=2}^{a+b-2}\gpgrade{\BA_a\BA_b}{i} \\
&= \BA_a \BA_b - \sum_{k=0}^{a-1}\gpgrade{\BA_a\BA_b}{2k + b - a} \\
\end{align*}

%\begin{equation}\label{eqn:bladewedge}
%\BA_a \wedge \BA_b = \BA_a \BA_b - \sum_{i=\abs{b-a},i+=2}^{a+b-2}\gpgrade{\BA_a\BA_b}{i}
%\end{equation}

%Reduction to a direct product (removal of the explicit wedge) follows like the vector case.

%The highest grade term in the sum of equation \ref{eqn:bladewedge} is for $i = b+a-2$, so a product with $\BA_c$ has grades $c-(b+a-2), \cdots c+(b+a-2)$.  All of these are greater than $c-(a+b)$, so we are left with just the direct product term as in the vector case:

Back substitution gives:

\begin{align*}
\gpgrade{ (\BA_a \wedge \BA_b) \BA_c }{\abs{c-(a+b)}} 
&=
\gpgrade{ \BA_a \BA_b \BA_c }{\abs{c-(a+b)}} 
-
\sum_{k=0}^{a-1}
\gpgrade{ \gpgrade{\BA_a\BA_b}{2k + b - a} \BA_c }{c-a-b}
\end{align*}

Temporarily writing $\gpgrade{\BA_a\BA_b}{2k + b - a} = \BC_i$,
\begin{align*}
\gpgrade{\BA_a\BA_b}{2k + b - a} \BA_c
&= \sum_{j=c-i,j+=2}^{c+i} \gpgrade{ \BC_i \BA_c }{j} \\
&= \sum_{r=0}^{i} \gpgrade{ \BC_i \BA_c }{c-i+2r} \\
&= \sum_{r=0}^{2k+b-a} \gpgrade{ \BC_i \BA_c }{c-2k-b+a+2r} \\
&= \sum_{r=0}^{2k+b-a} \gpgrade{ \BC_i \BA_c }{c-b+a +2(r-k)} \\
\end{align*}

We want the only the following grade terms:

\[
c-b+a+2(r-k) = c - b - a
\implies
r=k-a
\]

There are many such $k,r$ combinations, but we have a $k \in [0,a-1]$ constraint, which implies $r \in [-a,-1]$.  This contradicts with $r$ strictly
positive,
so there are no such grade elements.

This gives an intermediate result, the reduction of the triple product to a direct product, removing the explicit wedge:

\begin{equation}
(\BA_a \wedge \BA_b) \cdot \BA_c =
\gpgrade{\BA_a \BA_b \BA_c}{c-a-b}
\end{equation}

\begin{align*}
\gpgrade{\BA_a \BA_b \BA_c}{c-a-b}
&= \gpgrade{\BA_a (\BA_b \BA_c)}{c-a-b}
&= \gpgrade{\BA_a \sum_{i}\gpgrade{\BA_b \BA_c}{i}}{c-a-b}
&= \gpgrade{\sum_{j}\gpgrade{\BA_a \sum_{i}\gpgrade{\BA_b \BA_c}{i}}{j}}{c-a-b}
\end{align*}

Explicitly specifying the grades here is omitted for simplicity.  The lowest grade of these is $(c-b)-a$, and all others are higher.  By definition

\[
\gpgrade{\BA_b \BA_c}{c-b} = \BA_b \cdot \BA_c
\]

and that lowest grade term is thus

\[
\gpgrade{\BA_a \gpgrade{\BA_b \BA_c}{c-b}}{c-a-b}
= \gpgrade{\BA_a (\BA_b \cdot \BA_c)}{c-a-b}
= \BA_a \cdot (\BA_b \cdot \BA_c)
\]

This completes the proof.

%\gpgrade{ \left(
%\BA_a \BA_b - \sum_{i=\abs{b-a},i+=2}^{a+b-2}\gpgrade{\BA_a\BA_b}{i}
%\right) \BA_c }{\abs{c-(a+b)}}  \\
%\end{align*}
%\begin{align*}
%&=
%\gpgrade{ \BA_a \BA_b \BA_c }{\abs{c-(a+b)}} 
%- \sum_{i=\abs{b-a},i+=2}^{a+b-2} \gpgrade{ \gpgrade{\BA_a\BA_b}{i} \BA_c }{\abs{c-(a+b)}}  \\
%&=
%\gpgrade{ \BA_a \BA_b \BA_c }{\abs{c-(a+b)}} 
%- \sum_{j=\abs{c-i},j+=2}^{c+i} \sum_{i=\abs{b-a},i+=2}^{a+b-2} \gpgrade{ \gpgrade{ \gpgrade{\BA_a\BA_b}{i} \BA_c }{j} }{\abs{c-(a+b)}}  \\
%&=
%\gpgrade{ \BA_a \BA_b \BA_c }{\abs{c-(a+b)}} \\
%\end{align*}

\end{document}               % End of document.
