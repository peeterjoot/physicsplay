\documentclass{article}

\usepackage{amsmath}
\usepackage{mathpazo}

%
% shorthand for bold symbols, convenient for vectors and matrices
%
\newcommand{\Ba}[0]{\mathbf{a}}
\newcommand{\Bb}[0]{\mathbf{b}}
\newcommand{\Bc}[0]{\mathbf{c}}
\newcommand{\Bd}[0]{\mathbf{d}}
\newcommand{\Be}[0]{\mathbf{e}}
\newcommand{\Bf}[0]{\mathbf{f}}
\newcommand{\Bg}[0]{\mathbf{g}}
\newcommand{\Bh}[0]{\mathbf{h}}
\newcommand{\Bi}[0]{\mathbf{i}}
\newcommand{\Bj}[0]{\mathbf{j}}
\newcommand{\Bk}[0]{\mathbf{k}}
\newcommand{\Bl}[0]{\mathbf{l}}
\newcommand{\Bm}[0]{\mathbf{m}}
\newcommand{\Bn}[0]{\mathbf{n}}
\newcommand{\Bo}[0]{\mathbf{o}}
\newcommand{\Bp}[0]{\mathbf{p}}
\newcommand{\Bq}[0]{\mathbf{q}}
\newcommand{\Br}[0]{\mathbf{r}}
\newcommand{\Bs}[0]{\mathbf{s}}
\newcommand{\Bt}[0]{\mathbf{t}}
\newcommand{\Bu}[0]{\mathbf{u}}
\newcommand{\Bv}[0]{\mathbf{v}}
\newcommand{\Bw}[0]{\mathbf{w}}
\newcommand{\Bx}[0]{\mathbf{x}}
\newcommand{\By}[0]{\mathbf{y}}
\newcommand{\Bz}[0]{\mathbf{z}}
\newcommand{\BA}[0]{\mathbf{A}}
\newcommand{\BB}[0]{\mathbf{B}}
\newcommand{\BC}[0]{\mathbf{C}}
\newcommand{\BD}[0]{\mathbf{D}}
\newcommand{\BE}[0]{\mathbf{E}}
\newcommand{\BF}[0]{\mathbf{F}}
\newcommand{\BG}[0]{\mathbf{G}}
\newcommand{\BH}[0]{\mathbf{H}}
\newcommand{\BI}[0]{\mathbf{I}}
\newcommand{\BJ}[0]{\mathbf{J}}
\newcommand{\BK}[0]{\mathbf{K}}
\newcommand{\BL}[0]{\mathbf{L}}
\newcommand{\BM}[0]{\mathbf{M}}
\newcommand{\BN}[0]{\mathbf{N}}
\newcommand{\BO}[0]{\mathbf{O}}
\newcommand{\BP}[0]{\mathbf{P}}
\newcommand{\BQ}[0]{\mathbf{Q}}
\newcommand{\BR}[0]{\mathbf{R}}
\newcommand{\BS}[0]{\mathbf{S}}
\newcommand{\BT}[0]{\mathbf{T}}
\newcommand{\BU}[0]{\mathbf{U}}
\newcommand{\BV}[0]{\mathbf{V}}
\newcommand{\BW}[0]{\mathbf{W}}
\newcommand{\BX}[0]{\mathbf{X}}
\newcommand{\BY}[0]{\mathbf{Y}}
\newcommand{\BZ}[0]{\mathbf{Z}}

\newcommand{\Bzero}[0]{\mathbf{0}}
\newcommand{\Btheta}[0]{\boldsymbol{\theta}}
\newcommand{\Btau}[0]{\boldsymbol{\tau}}
\newcommand{\Bomega}[0]{\boldsymbol{\omega}}

%
% shorthand for unit vectors
%
\newcommand{\acap}[0]{\hat{\Ba}}
\newcommand{\bcap}[0]{\hat{\Bb}}
\newcommand{\ccap}[0]{\hat{\Bc}}
\newcommand{\dcap}[0]{\hat{\Bd}}
\newcommand{\ecap}[0]{\hat{\Be}}
\newcommand{\fcap}[0]{\hat{\Bf}}
\newcommand{\gcap}[0]{\hat{\Bg}}
\newcommand{\hcap}[0]{\hat{\Bh}}
\newcommand{\icap}[0]{\hat{\Bi}}
\newcommand{\jcap}[0]{\hat{\Bj}}
\newcommand{\kcap}[0]{\hat{\Bk}}
\newcommand{\lcap}[0]{\hat{\Bl}}
\newcommand{\mcap}[0]{\hat{\Bm}}
\newcommand{\ncap}[0]{\hat{\Bn}}
\newcommand{\ocap}[0]{\hat{\Bo}}
\newcommand{\pcap}[0]{\hat{\Bp}}
\newcommand{\qcap}[0]{\hat{\Bq}}
\newcommand{\rcap}[0]{\hat{\Br}}
\newcommand{\scap}[0]{\hat{\Bs}}
\newcommand{\tcap}[0]{\hat{\Bt}}
\newcommand{\ucap}[0]{\hat{\Bu}}
\newcommand{\vcap}[0]{\hat{\Bv}}
\newcommand{\wcap}[0]{\hat{\Bw}}
\newcommand{\xcap}[0]{\hat{\Bx}}
\newcommand{\ycap}[0]{\hat{\By}}
\newcommand{\zcap}[0]{\hat{\Bz}}
\newcommand{\thetacap}[0]{\hat{\Btheta}}

%
% to write R^n and C^n in a distinguishable fashion.  Perhaps change this
% to the double lined characters upon figuring out how to do so.
%
\newcommand{\C}[1]{$\mathbb{C}^{#1}$}
\newcommand{\R}[1]{$\mathbb{R}^{#1}$}

%
% various generally useful helpers
%

% derivative of #1 wrt. #2:
\newcommand{\D}[2] {\frac {d#2} {d#1}}

\newcommand{\inv}[1]{\frac{1}{#1}}
\newcommand{\cross}[0]{\times}

\newcommand{\abs}[1]{\lvert{#1}\rvert}
\newcommand{\norm}[1]{\lVert{#1}\rVert}
\newcommand{\innerprod}[2]{\langle{#1}, {#2}\rangle}
\newcommand{\dotprod}[2]{{#1} \cdot {#2}}
\newcommand{\bdotprod}[2]{\left({#1} \cdot {#2}\right)}
\newcommand{\crossprod}[2]{{#1} \cross {#2}}
\newcommand{\tripleprod}[3]{\dotprod{\left(\crossprod{#1}{#2}\right)}{#3}}

\DeclareMathOperator{\Proj}{Proj}
\DeclareMathOperator{\Span}{span}
\DeclareMathOperator{\Sgn}{sgn}
\DeclareMathOperator{\Area}{Area}
\DeclareMathOperator{\Volume}{Volume}

%
% A few miscellaneous things specific to this document
%
\newcommand{\crossop}[1]{\crossprod{#1}{}}

% R2 vector.
\newcommand{\VectorTwo}[2]{
\begin{bmatrix}
 {#1} \\
 {#2}
\end{bmatrix}
}

\newcommand{\VectorN}[1]{
\begin{bmatrix}
{#1}_1 \\
{#1}_2 \\
\vdots \\
{#1}_N \\
\end{bmatrix}
}

\newcommand{\DETuvij}[4]{
\begin{vmatrix}
 {#1}_{#3} & {#1}_{#4} \\
 {#2}_{#3} & {#2}_{#4}
\end{vmatrix}
}

\newcommand{\DETuvwijk}[6]{
\begin{vmatrix}
 {#1}_{#4} & {#1}_{#5} & {#1}_{#6} \\
 {#2}_{#4} & {#2}_{#5} & {#2}_{#6} \\
 {#3}_{#4} & {#3}_{#5} & {#3}_{#6}
\end{vmatrix}
}

\newcommand{\DETuvwxijkl}[8]{
\begin{vmatrix}
 {#1}_{#5} & {#1}_{#6} & {#1}_{#7} & {#1}_{#8} \\
 {#2}_{#5} & {#2}_{#6} & {#2}_{#7} & {#2}_{#8} \\
 {#3}_{#5} & {#3}_{#6} & {#3}_{#7} & {#3}_{#8} \\
 {#4}_{#5} & {#4}_{#6} & {#4}_{#7} & {#4}_{#8} \\
\end{vmatrix}
}

%\newcommand{\DETuvwxyijklm}[10]{
%\begin{vmatrix}
% {#1}_{#6} & {#1}_{#7} & {#1}_{#8} & {#1}_{#9} & {#1}_{#10} \\
% {#2}_{#6} & {#2}_{#7} & {#2}_{#8} & {#2}_{#9} & {#2}_{#10} \\
% {#3}_{#6} & {#3}_{#7} & {#3}_{#8} & {#3}_{#9} & {#3}_{#10} \\
% {#4}_{#6} & {#4}_{#7} & {#4}_{#8} & {#4}_{#9} & {#4}_{#10} \\
% {#5}_{#6} & {#5}_{#7} & {#5}_{#8} & {#5}_{#9} & {#5}_{#10}
%\end{vmatrix}
%}

% R3 vector.
\newcommand{\VectorThree}[3]{
\begin{bmatrix}
 {#1} \\
 {#2} \\
 {#3}
\end{bmatrix}
}


\newcommand{\LL}[0]{\mathcal{L}}
\newcommand{\PD}[2]{\frac{\partial {#2}}{\partial {#1}}}
\newcommand{\barA}[0]{\bar{A}}

\usepackage[bookmarks=true]{hyperref}

\title{ Derivation of Euler-Lagrange field Lagrangian equations.}
\author{Peeter Joot}
\date{ October 10, 2008.  Last Revision: $Date: 2008/10/11 03:33:21 $ }

\begin{document}

\maketitle{}

\tableofcontents

\section{ Motivation. }

In \cite{PJMaxwellLagrangian} Maxwell's equations were derived from
a Lagrangian action in tensor and STA forms.  This was done with 
Feynman's \cite{feynman1963flp} simple, but somewhat non-rigorous, direct variational technique.

An alternate approach is to use a field form of the Euler-Lagrange
equations as done in the wikipedia article \cite{wikiemtensor}.  I had
trouble understanding that derivation, probably because
I didn't understand the notation, nor what the source of that equation.

Here Feynman's approach will be used to derive the field versions of the Euler-Lagrange
equations, which clarifies the notation.  As a verification of the correctness these
will be applied to derive Maxwell's equation.

\section{ Deriving the field Lagrangian equations }

That essence of Feynman's method from his 
``Principle of Least Action'' entertainment chapter of the Lectures is
to do a first order linear expansion of the function, ignore all the higher order terms,
then do the integration by parts for the remainder.

Looking at the Maxwell field Lagrangian and action for motivation,

\begin{align*}
\LL &= (\partial^\mu A^\nu - \partial^\nu A^\mu) (\partial_\mu A_\nu - \partial_\nu A_\mu) + \kappa J^\sigma A_\sigma \\
S &= \int d^4 x \LL
\end{align*}

where the potential functions $A^\mu$ (or their index lowered variants)
are to be determined by extreme values of the action variation.  Note the use of the shorthand
$\partial_\mu \equiv \PD{x^\mu}{}$.

We want to consider general Lagrangians of this form.  Write

\begin{align*}
\LL = \LL( A^\mu, \partial_\nu A^\sigma) = \LL(A^0, A^1, \cdots, \partial_0 A^0, \partial_1 A^0, \cdots )
\end{align*}

\subsection{ First order Taylor expansion of a multi variable function. }

Given an abstractly specified function like this, with indexes and partials flying around, how to do a first order Taylor series expansion may not be obvious, especially since the variables are all undetermined functions!

Consideration of a simple case guides the way.  Assume that a two variable function can be expressed as a polynomial of some order

\begin{align*}
f(x,y) = a_{i j} x^i y^j
\end{align*}

Evaluation of this function or its partials at $(x,y) = (0,0)$ supply the constants $a_{i j}$.  Simplest is the lowest order constant

\begin{align*}
f(0,0) &= a_{0 0} \\
\end{align*}

\begin{align*}
\partial_x f &= i a_{i j} x^{i-1} y^j \\
\partial_y f &= j a_{i j} x^{i} y^{j-1} \\
\partial_{xx} f &= i(i-1) a_{i j} x^{i-2} y^j \\
\partial_{yy} f &= j(j-1) a_{i j} x^{i} y^{j-2} \\
\partial_{xy} f &= i j a_{i j} x^{i-1} y^{j-1} \\
\hdots \\
\implies \\
a_{1 0} &= (\partial_x f) \vert_0 \\
a_{0 1} &= (\partial_y f) \vert_0 \\
a_{2 0} &= \inv{2!} (\partial_{xx} f) \vert_0 \\
a_{0 2} &= \inv{2!} (\partial_{yy} f) \vert_0 \\
a_{1 1} &= (\partial_{xy} f) \vert_0 = (\partial_{yx} f) \vert_0 \\
\hdots
\end{align*}

Or

\begin{align*}
f(x,y) &= f \vert_0 + x (\partial_x f) \vert_0 + y (\partial_y f) \vert_0 \\
&+ \inv{2} \left( x^2 (\partial_{x x} f) \vert_0 + x y (\partial_{x y} f) \vert_0  + y x (\partial_{y x} f) \vert_0  + y^2(\partial_{y y} f) \vert_0 \right) \\
&+ \sum_{(i+j) > 2} a_{i j} x^i y^j
\end{align*}

\subsection{ First order expansion of the Lagrangian function. }

It isn't hard to see that the same thing can be done for higher degree functions too, although enumerating the 
higher order terms will get messier, however for the purposes of this variational exercise the assumption is that only
the first order differential terms are significant.

How to do the first order Taylor expansion of a multivariable function has been established.  Next write $A^\mu = \barA^\mu + n^\mu$, where the $\barA^\mu$ functions are the desired solutions and each of $n^\mu$ vanishes on the boundaries of the integration region.  Expansion of $\LL$ around the desired solutions one has

\begin{align*}
\LL(\barA^\mu + n^\mu, \partial_\nu( \barA^\sigma + n^\sigma) )
&=
\LL(\barA^\mu, \partial_\nu \barA^\sigma ) \\
&+ (\barA^\mu + n^\mu) \left( \left. \PD{A^\mu}{\LL} \right) \right\vert_{A^\mu = \barA^\mu} \\
&+ (\partial_\nu \barA^\sigma + \partial_\nu n^\sigma) \left( \left. \PD{(\partial_\nu A^\sigma)}{\LL} \right) \right\vert_{\partial_\nu A^\mu = \partial_\nu \barA^\mu} \\
&+ \sum_{i+j>2} (\barA^\mu + n^\mu)^i (\partial_\nu \barA^\sigma + \partial_\nu n^\sigma)^j \underbrace{\left(\cdots\right)}_{\text{higher order derivatives}}
\end{align*}

\subsection{ Example for clarification. }
Here we see the first use of the peculiar looking partials from the wikipedia article

\begin{align*}
\PD{(\partial_\nu A^\sigma)}{\LL}.
\end{align*}

Initially looking at that I couldn't fathom what it meant, but it is just what it says, 
differentiation with respect to a variable $\partial_\nu A^\sigma$.  As an example, for

\begin{align*}
\LL 
&= u A^0 + v A^1 + a \partial_1 A^0 + b \partial_0 A^1 \\
&= u A^0 + v A^1 + a \PD{x^1}{A^0} + b \PD{x^0}{A^1}
\end{align*}

where $u$,$v$,$a$, and $b$ are constants.  Then an corresponding example of such a partial term is

\begin{align*}
\PD{(\partial_1 A^0)}{\LL} = a.
\end{align*}

\subsection{ Calculationg of the action for the general field Lagrangian. }

\begin{align*}
S &= \int d^4 x \LL \\
&= \int d^4 x \LL(\barA^\mu, \partial_\nu \barA^\sigma ) \\
&+ \int d^4 x (\barA^\mu + n^\mu) \left( \left. \PD{A^\mu}{\LL} \right) \right\vert_{A^\mu = \barA^\mu} \\
&+ \int d^4 x (\partial_\nu \barA^\sigma + \partial_\nu n^\sigma) \left( \left. \PD{(\partial_\nu A^\sigma)}{\LL} \right) \right\vert_{\partial_\nu A^\mu = \partial_\nu \barA^\mu} \\
&+ \int d^4 x (\cdots \text{neglected higher order terms} \cdots )
\end{align*}

Grouping this into parts associated with the assumed variational solution, and the varied parts we have

\begin{align*}
S &= \int d^4 x 
\left(
\LL(\barA^\mu, \partial_\nu \barA^\sigma ) + \barA^\mu \left. \left( \PD{A^\mu}{\LL} \right) \right\vert_{A^\mu = \barA^\mu} 
+ \partial_\nu \barA^\sigma \left. \left( \PD{(\partial_\nu A^\sigma)}{\LL} \right) \right\vert_{\partial_\nu A^\mu = \partial_\nu \barA^\mu} 
\right) \\
&+ \int d^4 x 
\left(
n^\mu \left. \left( \PD{A^\mu}{\LL} \right) \right\vert_{A^\mu = \barA^\mu} 
+\partial_\nu n^\sigma \left. \left( \PD{(\partial_\nu A^\sigma)}{\LL} \right) \right\vert_{\partial_\nu A^\mu = \partial_\nu \barA^\mu}
\right) \\
&+ \cdots
\end{align*}

None of the terms in the first integral are of interest since they are fixed.  The second term of the remaining integral is the one to integrate by
parts.  For short, let

\begin{align*}
u &= \left. \left( \PD{(\partial_\nu A^\sigma)}{\LL} \right) \right\vert_{\partial_\nu A^\mu = \partial_\nu \barA^\mu},
\end{align*}

then this integral is
%(u v)' = u' v + u v'
%\int u' v = \int (u v)' - \int u v' 
%\int u' v = u v - \int u v' 
\begin{align*}
\int d^3 x d x^\nu \PD{x^\nu}{n^\sigma} u 
&= \int d^3 x \left. {n^\sigma} u \right\vert_{\partial x^\nu} - \int d^4 x n^\sigma \PD{x^\nu}{u}
\end{align*}

Here $\partial x^\nu$ denotes the boundary of the integration.  Because $n^\sigma$ was by definition zero on all boundaries of the
integral region this first integral is zero.  Denoting the non-variational parts of the action integral by $\delta S$, we have

\begin{align*}
\delta S
&= \int d^4 x \left(
n^\mu \left. \left( \PD{A^\mu}{\LL} \right) \right\vert_{A^\mu = \barA^\mu} 
- n^\sigma \partial_\nu \left. \left( \PD{(\partial_\nu A^\sigma)}{\LL} \right) \right\vert_{\partial_\nu A^\mu = \partial_\nu \barA^\mu}
\right) \\
&= \int d^4 x
n^\sigma
\left(
\left. \left( \PD{A^\sigma}{\LL} \right) \right\vert_{A^\sigma = \barA^\sigma} 
- \partial_\nu \left. \left( \PD{(\partial_\nu A^\sigma)}{\LL} \right) \right\vert_{\partial_\nu A^\mu = \partial_\nu \barA^\mu}
\right)
\end{align*}

\section{ Verifying the equations.  Maxwell's equation derivation from action. }

\bibliographystyle{plainnat} % supposed to allow for \url use.
\bibliography{myrefs}      % expects file "myrefs.bib"

\end{document}               % End of document.
