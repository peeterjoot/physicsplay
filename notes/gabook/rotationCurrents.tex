\documentclass[]{eliblog}

\usepackage{amsmath}
\usepackage{mathpazo}

%
% shorthand for bold symbols, convenient for vectors and matrices
%
\newcommand{\Ba}[0]{\mathbf{a}}
\newcommand{\Bb}[0]{\mathbf{b}}
\newcommand{\Bc}[0]{\mathbf{c}}
\newcommand{\Bd}[0]{\mathbf{d}}
\newcommand{\Be}[0]{\mathbf{e}}
\newcommand{\Bf}[0]{\mathbf{f}}
\newcommand{\Bg}[0]{\mathbf{g}}
\newcommand{\Bh}[0]{\mathbf{h}}
\newcommand{\Bi}[0]{\mathbf{i}}
\newcommand{\Bj}[0]{\mathbf{j}}
\newcommand{\Bk}[0]{\mathbf{k}}
\newcommand{\Bl}[0]{\mathbf{l}}
\newcommand{\Bm}[0]{\mathbf{m}}
\newcommand{\Bn}[0]{\mathbf{n}}
\newcommand{\Bo}[0]{\mathbf{o}}
\newcommand{\Bp}[0]{\mathbf{p}}
\newcommand{\Bq}[0]{\mathbf{q}}
\newcommand{\Br}[0]{\mathbf{r}}
\newcommand{\Bs}[0]{\mathbf{s}}
\newcommand{\Bt}[0]{\mathbf{t}}
\newcommand{\Bu}[0]{\mathbf{u}}
\newcommand{\Bv}[0]{\mathbf{v}}
\newcommand{\Bw}[0]{\mathbf{w}}
\newcommand{\Bx}[0]{\mathbf{x}}
\newcommand{\By}[0]{\mathbf{y}}
\newcommand{\Bz}[0]{\mathbf{z}}
\newcommand{\BA}[0]{\mathbf{A}}
\newcommand{\BB}[0]{\mathbf{B}}
\newcommand{\BC}[0]{\mathbf{C}}
\newcommand{\BD}[0]{\mathbf{D}}
\newcommand{\BE}[0]{\mathbf{E}}
\newcommand{\BF}[0]{\mathbf{F}}
\newcommand{\BG}[0]{\mathbf{G}}
\newcommand{\BH}[0]{\mathbf{H}}
\newcommand{\BI}[0]{\mathbf{I}}
\newcommand{\BJ}[0]{\mathbf{J}}
\newcommand{\BK}[0]{\mathbf{K}}
\newcommand{\BL}[0]{\mathbf{L}}
\newcommand{\BM}[0]{\mathbf{M}}
\newcommand{\BN}[0]{\mathbf{N}}
\newcommand{\BO}[0]{\mathbf{O}}
\newcommand{\BP}[0]{\mathbf{P}}
\newcommand{\BQ}[0]{\mathbf{Q}}
\newcommand{\BR}[0]{\mathbf{R}}
\newcommand{\BS}[0]{\mathbf{S}}
\newcommand{\BT}[0]{\mathbf{T}}
\newcommand{\BU}[0]{\mathbf{U}}
\newcommand{\BV}[0]{\mathbf{V}}
\newcommand{\BW}[0]{\mathbf{W}}
\newcommand{\BX}[0]{\mathbf{X}}
\newcommand{\BY}[0]{\mathbf{Y}}
\newcommand{\BZ}[0]{\mathbf{Z}}

\newcommand{\Bzero}[0]{\mathbf{0}}
\newcommand{\Btheta}[0]{\boldsymbol{\theta}}
\newcommand{\Btau}[0]{\boldsymbol{\tau}}
\newcommand{\Bomega}[0]{\boldsymbol{\omega}}

%
% shorthand for unit vectors
%
\newcommand{\acap}[0]{\hat{\Ba}}
\newcommand{\bcap}[0]{\hat{\Bb}}
\newcommand{\ccap}[0]{\hat{\Bc}}
\newcommand{\dcap}[0]{\hat{\Bd}}
\newcommand{\ecap}[0]{\hat{\Be}}
\newcommand{\fcap}[0]{\hat{\Bf}}
\newcommand{\gcap}[0]{\hat{\Bg}}
\newcommand{\hcap}[0]{\hat{\Bh}}
\newcommand{\icap}[0]{\hat{\Bi}}
\newcommand{\jcap}[0]{\hat{\Bj}}
\newcommand{\kcap}[0]{\hat{\Bk}}
\newcommand{\lcap}[0]{\hat{\Bl}}
\newcommand{\mcap}[0]{\hat{\Bm}}
\newcommand{\ncap}[0]{\hat{\Bn}}
\newcommand{\ocap}[0]{\hat{\Bo}}
\newcommand{\pcap}[0]{\hat{\Bp}}
\newcommand{\qcap}[0]{\hat{\Bq}}
\newcommand{\rcap}[0]{\hat{\Br}}
\newcommand{\scap}[0]{\hat{\Bs}}
\newcommand{\tcap}[0]{\hat{\Bt}}
\newcommand{\ucap}[0]{\hat{\Bu}}
\newcommand{\vcap}[0]{\hat{\Bv}}
\newcommand{\wcap}[0]{\hat{\Bw}}
\newcommand{\xcap}[0]{\hat{\Bx}}
\newcommand{\ycap}[0]{\hat{\By}}
\newcommand{\zcap}[0]{\hat{\Bz}}
\newcommand{\thetacap}[0]{\hat{\Btheta}}

%
% to write R^n and C^n in a distinguishable fashion.  Perhaps change this
% to the double lined characters upon figuring out how to do so.
%
\newcommand{\C}[1]{$\mathbb{C}^{#1}$}
\newcommand{\R}[1]{$\mathbb{R}^{#1}$}

%
% various generally useful helpers
%

% derivative of #1 wrt. #2:
\newcommand{\D}[2] {\frac {d#2} {d#1}}

\newcommand{\inv}[1]{\frac{1}{#1}}
\newcommand{\cross}[0]{\times}

\newcommand{\abs}[1]{\lvert{#1}\rvert}
\newcommand{\norm}[1]{\lVert{#1}\rVert}
\newcommand{\innerprod}[2]{\langle{#1}, {#2}\rangle}
\newcommand{\dotprod}[2]{{#1} \cdot {#2}}
\newcommand{\bdotprod}[2]{\left({#1} \cdot {#2}\right)}
\newcommand{\crossprod}[2]{{#1} \cross {#2}}
\newcommand{\tripleprod}[3]{\dotprod{\left(\crossprod{#1}{#2}\right)}{#3}}

\DeclareMathOperator{\Proj}{Proj}
\DeclareMathOperator{\Span}{span}
\DeclareMathOperator{\Sgn}{sgn}
\DeclareMathOperator{\Area}{Area}
\DeclareMathOperator{\Volume}{Volume}

%
% A few miscellaneous things specific to this document
%
\newcommand{\crossop}[1]{\crossprod{#1}{}}

% R2 vector.
\newcommand{\VectorTwo}[2]{
\begin{bmatrix}
 {#1} \\
 {#2}
\end{bmatrix}
}

\newcommand{\VectorN}[1]{
\begin{bmatrix}
{#1}_1 \\
{#1}_2 \\
\vdots \\
{#1}_N \\
\end{bmatrix}
}

\newcommand{\DETuvij}[4]{
\begin{vmatrix}
 {#1}_{#3} & {#1}_{#4} \\
 {#2}_{#3} & {#2}_{#4}
\end{vmatrix}
}

\newcommand{\DETuvwijk}[6]{
\begin{vmatrix}
 {#1}_{#4} & {#1}_{#5} & {#1}_{#6} \\
 {#2}_{#4} & {#2}_{#5} & {#2}_{#6} \\
 {#3}_{#4} & {#3}_{#5} & {#3}_{#6}
\end{vmatrix}
}

\newcommand{\DETuvwxijkl}[8]{
\begin{vmatrix}
 {#1}_{#5} & {#1}_{#6} & {#1}_{#7} & {#1}_{#8} \\
 {#2}_{#5} & {#2}_{#6} & {#2}_{#7} & {#2}_{#8} \\
 {#3}_{#5} & {#3}_{#6} & {#3}_{#7} & {#3}_{#8} \\
 {#4}_{#5} & {#4}_{#6} & {#4}_{#7} & {#4}_{#8} \\
\end{vmatrix}
}

%\newcommand{\DETuvwxyijklm}[10]{
%\begin{vmatrix}
% {#1}_{#6} & {#1}_{#7} & {#1}_{#8} & {#1}_{#9} & {#1}_{#10} \\
% {#2}_{#6} & {#2}_{#7} & {#2}_{#8} & {#2}_{#9} & {#2}_{#10} \\
% {#3}_{#6} & {#3}_{#7} & {#3}_{#8} & {#3}_{#9} & {#3}_{#10} \\
% {#4}_{#6} & {#4}_{#7} & {#4}_{#8} & {#4}_{#9} & {#4}_{#10} \\
% {#5}_{#6} & {#5}_{#7} & {#5}_{#8} & {#5}_{#9} & {#5}_{#10}
%\end{vmatrix}
%}

% R3 vector.
\newcommand{\VectorThree}[3]{
\begin{bmatrix}
 {#1} \\
 {#2} \\
 {#3}
\end{bmatrix}
}



\author{Peeter Joot}
\email{peeter.joot@gmail.com}


\chapter{Noether currents for rotational changes.}
\label{chap:rotationCurrents}
%\useCCL
\blogpage{http://sites.google.com/site/peeterjoot/math2009/rotationCurrents.pdf}
\date{Sept 4, 2009}
\revisionInfo{$RCSfile: rotationCurrents.tex,v $ Last $Revision: 1.4 $ $Date: 2009/09/07 14:19:48 $}

\beginArtWithToc

\section{Motivation}

The article (\cite{montesinos2006sem}) details the calculation for a conserved current associated with an incremental Poincare transformation.  This is used to directly determine the symmetric energy momentum tensor for Maxwell's equations, in contrast to the canonical energy momentum tensor (arising from spacetime translation) which is not symmetric but can be symmetrized with other arguments.

I believe that I am slowly accumulating the tools required to understand this paper.  One such tool is likely the exponential rotational generator examined in \chapcite{rotationGenerator}, utilizing the angular momentum operator.

Here I review some of the Noether conservation calculations and the associated Noether currents for a few example Lagrangian densities.  Then I hope to see how to apply similar techniques to these using an angular momentum operator alteration of the Lagrangian density.

\section{Field Euler-Lagrange equations.}

The extremization of the action integral

\begin{align}\label{eqn:rotationCurrents:boo1}
S &= \int \LL d^4 x
\end{align}

can be dealt with (following Feynman) as a first order Taylor expansion and integration by parts exercise.  A single field variable example serves to illustrate.  A first order Lagrangian of a single field variable has the form

\begin{align}\label{eqn:rotationCurrents:boo2}
\LL = \LL(\phi, \partial_\mu \phi)
\end{align}

Let us vary the field $\phi \rightarrow \phi + \bar{\phi}$, inducing a corresponding variation in the action

\begin{align*}
S + \delta S
&= \int \LL(\phi + \bar{\phi}, \partial_\mu (phi + \bar{\phi}) d^4 x \\
&= \int d^4 x \left(
\LL(\bar{\phi}, \partial_\mu \bar{\phi})
+
\bar{\phi} \PD{\phi}{\LL}
+\partial_\mu \bar{\phi} \PD{(\partial_\mu \phi)}{\LL}
+ \cdots \right)
\end{align*}

Neglecting any second or higher order terms the change in the action from the assumed solution is

\begin{align}\label{eqn:rotationCurrents:boo3}
\delta S
&=
\int d^4 x \left( \bar{\phi} \PD{\phi}{\LL} +\partial_\mu \bar{\phi} \PD{(\partial_\mu \phi)}{\LL} \right)
\end{align}

This is now integrable by parts yielding

\begin{align}\label{eqn:rotationCurrents:boo4}
\delta S
&=
\int d^3 x \left( {\left. \bar{\phi} \partial_\mu \LL \right\vert}_{\partial x^\mu} \right)
+
\int d^4 x \bar{\phi} \left( \PD{\phi}{\LL} - \partial_\mu \PD{(\partial_\mu \phi)}{\LL} \right)
\end{align}

Here $d^3 x$ is taken to mean that part of the integration not including $dx_\mu$.  The field $\bar{\phi}$ is always required to vanish on the boundary as in the dynamic Lagrangian arguments, so the first integral is zero.  If the remainder is zero for all fields $\bar{\phi}$, then the inner term must be zero, and we the field Euler-Lagrange equations as a result

\begin{align}\label{eqn:rotationCurrents:boo5}
\PD{\phi}{\LL} - \partial_\mu \PD{(\partial_\mu \phi)}{\LL} = 0
\end{align}

When we have multiple field variables, say $A_\nu$, the chain rule expansion leading to (\ref{eqn:rotationCurrents:boo3}) will have to be modified to sum over all the field variables, and we end up instead with

\begin{align}\label{eqn:rotationCurrents:boo6}
\delta S
&=
\int d^4 x \sum_{\nu} \bar{A_\nu} \left( \PD{A_\nu}{\LL} - \partial_\mu \PD{(\partial_\mu A_\nu)}{\LL} \right)
\end{align}

So for $\delta S = 0$ for all $\bar{A}_\nu$ we have a set of equations, one for each $\nu$

\begin{align}\label{eqn:rotationCurrents:boo7}
\PD{A_\nu}{\LL} - \partial_\mu \PD{(\partial_\mu A_\nu)}{\LL} = 0
\end{align}

\section{Field Noether currents.}

The single parameter Noether conservation equation again is mainly application of the chain rule.  Illustrating with the one field variable case, with an altered field variable $\phi \rightarrow \phi'(\theta)$, and

\begin{align}\label{eqn:rotationCurrents:moo1}
\LL' = \LL(\phi', \partial_\mu \phi')
\end{align}

Examining the change of $\LL'$ with $\theta$ we have

\begin{align*}
\frac{d \LL'}{d \theta}
&=
\PD{\phi'}{\LL} \PD{\theta}{\phi'}
+\PD{(\partial_\mu \phi')}{\LL}
\PD{\theta}{(\partial_\mu \phi')}
\end{align*}

For the last term we can switch up the order of differentiation

\begin{align*}
\PD{\theta}{(\partial_\mu \phi')}
&=
\PD{\theta}{}
\PD{x^\mu}{\phi'} \\
&= \PD{x^\mu}{} \PD{\theta}{\phi'}
\end{align*}

Additionally, with substitution of the Euler-Lagrange equations in the first term we have

\begin{align*}
\frac{d \LL'}{d \theta}
&=
\left( \PD{x^\mu}{} \PD{(\partial_\mu \phi')}{\LL} \right) \PD{\theta}{\phi'}
+\PD{(\partial_\mu \phi')}{\LL} \PD{x^\mu}{} \PD{\theta}{\phi'} \\
\end{align*}

But this can be directly anti-differentiated yielding the Noether conservation equation

\begin{align}\label{eqn:rotationCurrents:moo2}
\frac{d \LL'}{d \theta}
=
\PD{x^\mu}{} \left( \PD{(\partial_\mu \phi')}{\LL} \PD{\theta}{\phi'} \right)
\end{align}

With multiple field variables we'll have a term in the chain rule expansion for each field variable.  The end result is pretty much the same, but we have to sum over all the fields

\begin{align}\label{eqn:rotationCurrents:moo3}
\frac{d \LL'}{d \theta}
=
\sum_\nu \PD{x^\mu}{} \left( \PD{(\partial_\mu {A'}_\nu)}{\LL} \PD{\theta}{{A'}_\nu} \right)
\end{align}

Unlike the field Euler-Lagrange equations we have just one here, not one for each field variable.  In this multivariable case, expression in vector form can eliminate the sum over field variables.  With $A' = {A'}_\nu \gamma^\nu$, we have

\begin{align}\label{eqn:rotationCurrents:moo4}
\frac{d \LL'}{d \theta}
=
\PD{x^\mu}{} \left( \gamma_\nu \PD{(\partial_\mu {A'}_\nu)}{\LL} \cdot \PD{\theta}{A'} \right)
\end{align}

With an evaluation at $\theta = 0$, we have finally

\begin{align}\label{eqn:rotationCurrents:moo5}
{\left. \frac{d \LL'}{d \theta} \right\vert}_{\theta=0}
=
\PD{x^\mu}{} \left( \gamma_\nu \PD{(\partial_\mu {A}_\nu)}{\LL} \cdot {\left. \PD{\theta}{A'} \right\vert}_{\theta=0}\right)
\end{align}

When the Lagrangian alteration is independent of $\theta$ (i.e. is invariant), it is said that there is a symmetry.  By (\ref{eqn:rotationCurrents:moo5}) we have a conserved quantity associated with this symmetry, some quantity, say $J$ that has a zero divergence.   That is

\begin{align}\label{eqn:rotationCurrents:moo6}
J^\mu &= \gamma_\nu \PD{(\partial_\mu {A}_\nu)}{\LL} \cdot {\left. \PD{\theta}{A'} \right\vert}_{\theta=0} \\
0 &= \partial_\mu J^\mu
\end{align}

TO BE CONTINUED: Review the Noether derivation associated with spacetime translation, and the associated conservation currents.  Eventually try to find the Noether current for a linearized alteration of the Lagrangian using the angular momentum operator or the full exponential operator.

\EndArticle
