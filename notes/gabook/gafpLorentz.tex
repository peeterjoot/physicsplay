\documentclass{article}      

\usepackage{amsmath}
\usepackage{mathpazo}

%
% shorthand for bold symbols, convenient for vectors and matrices
%
\newcommand{\Ba}[0]{\mathbf{a}}
\newcommand{\Bb}[0]{\mathbf{b}}
\newcommand{\Bc}[0]{\mathbf{c}}
\newcommand{\Bd}[0]{\mathbf{d}}
\newcommand{\Be}[0]{\mathbf{e}}
\newcommand{\Bf}[0]{\mathbf{f}}
\newcommand{\Bg}[0]{\mathbf{g}}
\newcommand{\Bh}[0]{\mathbf{h}}
\newcommand{\Bi}[0]{\mathbf{i}}
\newcommand{\Bj}[0]{\mathbf{j}}
\newcommand{\Bk}[0]{\mathbf{k}}
\newcommand{\Bl}[0]{\mathbf{l}}
\newcommand{\Bm}[0]{\mathbf{m}}
\newcommand{\Bn}[0]{\mathbf{n}}
\newcommand{\Bo}[0]{\mathbf{o}}
\newcommand{\Bp}[0]{\mathbf{p}}
\newcommand{\Bq}[0]{\mathbf{q}}
\newcommand{\Br}[0]{\mathbf{r}}
\newcommand{\Bs}[0]{\mathbf{s}}
\newcommand{\Bt}[0]{\mathbf{t}}
\newcommand{\Bu}[0]{\mathbf{u}}
\newcommand{\Bv}[0]{\mathbf{v}}
\newcommand{\Bw}[0]{\mathbf{w}}
\newcommand{\Bx}[0]{\mathbf{x}}
\newcommand{\By}[0]{\mathbf{y}}
\newcommand{\Bz}[0]{\mathbf{z}}
\newcommand{\BA}[0]{\mathbf{A}}
\newcommand{\BB}[0]{\mathbf{B}}
\newcommand{\BC}[0]{\mathbf{C}}
\newcommand{\BD}[0]{\mathbf{D}}
\newcommand{\BE}[0]{\mathbf{E}}
\newcommand{\BF}[0]{\mathbf{F}}
\newcommand{\BG}[0]{\mathbf{G}}
\newcommand{\BH}[0]{\mathbf{H}}
\newcommand{\BI}[0]{\mathbf{I}}
\newcommand{\BJ}[0]{\mathbf{J}}
\newcommand{\BK}[0]{\mathbf{K}}
\newcommand{\BL}[0]{\mathbf{L}}
\newcommand{\BM}[0]{\mathbf{M}}
\newcommand{\BN}[0]{\mathbf{N}}
\newcommand{\BO}[0]{\mathbf{O}}
\newcommand{\BP}[0]{\mathbf{P}}
\newcommand{\BQ}[0]{\mathbf{Q}}
\newcommand{\BR}[0]{\mathbf{R}}
\newcommand{\BS}[0]{\mathbf{S}}
\newcommand{\BT}[0]{\mathbf{T}}
\newcommand{\BU}[0]{\mathbf{U}}
\newcommand{\BV}[0]{\mathbf{V}}
\newcommand{\BW}[0]{\mathbf{W}}
\newcommand{\BX}[0]{\mathbf{X}}
\newcommand{\BY}[0]{\mathbf{Y}}
\newcommand{\BZ}[0]{\mathbf{Z}}

\newcommand{\Bzero}[0]{\mathbf{0}}
\newcommand{\Btheta}[0]{\boldsymbol{\theta}}
\newcommand{\Btau}[0]{\boldsymbol{\tau}}
\newcommand{\Bomega}[0]{\boldsymbol{\omega}}

%
% shorthand for unit vectors
%
\newcommand{\acap}[0]{\hat{\Ba}}
\newcommand{\bcap}[0]{\hat{\Bb}}
\newcommand{\ccap}[0]{\hat{\Bc}}
\newcommand{\dcap}[0]{\hat{\Bd}}
\newcommand{\ecap}[0]{\hat{\Be}}
\newcommand{\fcap}[0]{\hat{\Bf}}
\newcommand{\gcap}[0]{\hat{\Bg}}
\newcommand{\hcap}[0]{\hat{\Bh}}
\newcommand{\icap}[0]{\hat{\Bi}}
\newcommand{\jcap}[0]{\hat{\Bj}}
\newcommand{\kcap}[0]{\hat{\Bk}}
\newcommand{\lcap}[0]{\hat{\Bl}}
\newcommand{\mcap}[0]{\hat{\Bm}}
\newcommand{\ncap}[0]{\hat{\Bn}}
\newcommand{\ocap}[0]{\hat{\Bo}}
\newcommand{\pcap}[0]{\hat{\Bp}}
\newcommand{\qcap}[0]{\hat{\Bq}}
\newcommand{\rcap}[0]{\hat{\Br}}
\newcommand{\scap}[0]{\hat{\Bs}}
\newcommand{\tcap}[0]{\hat{\Bt}}
\newcommand{\ucap}[0]{\hat{\Bu}}
\newcommand{\vcap}[0]{\hat{\Bv}}
\newcommand{\wcap}[0]{\hat{\Bw}}
\newcommand{\xcap}[0]{\hat{\Bx}}
\newcommand{\ycap}[0]{\hat{\By}}
\newcommand{\zcap}[0]{\hat{\Bz}}
\newcommand{\thetacap}[0]{\hat{\Btheta}}

%
% to write R^n and C^n in a distinguishable fashion.  Perhaps change this
% to the double lined characters upon figuring out how to do so.
%
\newcommand{\C}[1]{$\mathbb{C}^{#1}$}
\newcommand{\R}[1]{$\mathbb{R}^{#1}$}

%
% various generally useful helpers
%

% derivative of #1 wrt. #2:
\newcommand{\D}[2] {\frac {d#2} {d#1}}

\newcommand{\inv}[1]{\frac{1}{#1}}
\newcommand{\cross}[0]{\times}

\newcommand{\abs}[1]{\lvert{#1}\rvert}
\newcommand{\norm}[1]{\lVert{#1}\rVert}
\newcommand{\innerprod}[2]{\langle{#1}, {#2}\rangle}
\newcommand{\dotprod}[2]{{#1} \cdot {#2}}
\newcommand{\bdotprod}[2]{\left({#1} \cdot {#2}\right)}
\newcommand{\crossprod}[2]{{#1} \cross {#2}}
\newcommand{\tripleprod}[3]{\dotprod{\left(\crossprod{#1}{#2}\right)}{#3}}

\DeclareMathOperator{\Proj}{Proj}
\DeclareMathOperator{\Span}{span}
\DeclareMathOperator{\Sgn}{sgn}
\DeclareMathOperator{\Area}{Area}
\DeclareMathOperator{\Volume}{Volume}

%
% A few miscellaneous things specific to this document
%
\newcommand{\crossop}[1]{\crossprod{#1}{}}

% R2 vector.
\newcommand{\VectorTwo}[2]{
\begin{bmatrix}
 {#1} \\
 {#2}
\end{bmatrix}
}

\newcommand{\VectorN}[1]{
\begin{bmatrix}
{#1}_1 \\
{#1}_2 \\
\vdots \\
{#1}_N \\
\end{bmatrix}
}

\newcommand{\DETuvij}[4]{
\begin{vmatrix}
 {#1}_{#3} & {#1}_{#4} \\
 {#2}_{#3} & {#2}_{#4}
\end{vmatrix}
}

\newcommand{\DETuvwijk}[6]{
\begin{vmatrix}
 {#1}_{#4} & {#1}_{#5} & {#1}_{#6} \\
 {#2}_{#4} & {#2}_{#5} & {#2}_{#6} \\
 {#3}_{#4} & {#3}_{#5} & {#3}_{#6}
\end{vmatrix}
}

\newcommand{\DETuvwxijkl}[8]{
\begin{vmatrix}
 {#1}_{#5} & {#1}_{#6} & {#1}_{#7} & {#1}_{#8} \\
 {#2}_{#5} & {#2}_{#6} & {#2}_{#7} & {#2}_{#8} \\
 {#3}_{#5} & {#3}_{#6} & {#3}_{#7} & {#3}_{#8} \\
 {#4}_{#5} & {#4}_{#6} & {#4}_{#7} & {#4}_{#8} \\
\end{vmatrix}
}

%\newcommand{\DETuvwxyijklm}[10]{
%\begin{vmatrix}
% {#1}_{#6} & {#1}_{#7} & {#1}_{#8} & {#1}_{#9} & {#1}_{#10} \\
% {#2}_{#6} & {#2}_{#7} & {#2}_{#8} & {#2}_{#9} & {#2}_{#10} \\
% {#3}_{#6} & {#3}_{#7} & {#3}_{#8} & {#3}_{#9} & {#3}_{#10} \\
% {#4}_{#6} & {#4}_{#7} & {#4}_{#8} & {#4}_{#9} & {#4}_{#10} \\
% {#5}_{#6} & {#5}_{#7} & {#5}_{#8} & {#5}_{#9} & {#5}_{#10}
%\end{vmatrix}
%}

% R3 vector.
\newcommand{\VectorThree}[3]{
\begin{bmatrix}
 {#1} \\
 {#2} \\
 {#3}
\end{bmatrix}
}


\newcommand{\gpgrade}[2] {{\left\langle{{#1}}\right\rangle}_{#2}}
\newcommand{\gpgradeone}[1] {\gpgrade{#1}{1}}

\title{ Some notes on GAFP 5.5.3 The Lorentz force Law.} 
\author{Peeter Joot}         
\date{August 16, 2008}        

\begin{document}             

\maketitle{}

\section{}

The idea behind this derivation, is to express the vector part of the proper force in covarient form, and then
do the same for the energy change part of the proper momentum.  That first part is:

\begin{align*}
\frac{dp}{d\tau} \wedge \gamma_0 
&= \frac{d \Bp}{d\tau} \\
&= \frac{d \Bp}{dt} \frac{dt}{d\tau} \\
&= \frac{dt}{d\tau} q \left( \BE + \Bv \cross \BB \right)
\end{align*}

Now, the spacetime split of velocity is done in the normal fashion:

\begin{align*}
x &= c t \gamma_0 + \sum x^i \gamma_i \\
v &= \frac{dx}{d\tau} = c \frac{dt}{d\tau} \gamma_0 + \sum \frac{dx^i}{d\tau} \gamma_i \\
v \cdot \gamma_0 &= c \frac{dt}{d\tau} = c \gamma \\
v \wedge \gamma_0
&= \sum \frac{dx^i}{dt} \frac{dt}{d\tau} \gamma_i \gamma_0 \\
&= (v \cdot \gamma_0)/c \sum v^i \sigma_i \\
&= (v \cdot \gamma_0) \Bv/c.
\end{align*}

Writing $\dot{p} = dp/d\tau$, substituite the gamma factor into the force equation:

\begin{equation*}
\dot{p} \wedge \gamma_0 = ( v/c \cdot \gamma_0 ) q \left( \BE + \Bv \cross \BB \right)
\end{equation*}

Now, GAFP goes on to show that the $\gamma \BE$ term can be reduced to the form $(\BE \cdot v) \wedge \gamma_0$.  Their
method isn't exactly obvious, for example writing $\BE = (1/2)(\BE + \BE)$ to start.  Let's just do this backwards 
instead, expanding $\BE \cdot v$ to see the form of that term:

\begin{align*}
\BE \cdot v
&= \left(\sum E^i \gamma_{i0}\right) \cdot \left( \sum v^{\mu} \gamma_{\mu}\right) \\
&= \sum E^i v^{\mu} \gpgradeone{ \gamma_{i0\mu}} \\
&= v^0 \sum E^i \gamma_{i} + \sum E^i v^{j} \underbrace{\gpgradeone{ \gamma_{i0j}}}_{-\delta_{ij} \gamma_0} \\
&= v^0 \sum E^i \gamma_i - \sum E^i v^i \gamma_0.
\end{align*}

Wedging with $\gamma_0$ we have the desired result:

\begin{equation*}
(\BE \cdot v) \wedge \gamma_0 = v^0 \sum E^i \gamma_{i0} = (v \cdot \gamma_0) \BE = c \gamma \BE
\end{equation*}

Now, for equation 5.164 there aren't any suprising steps, but lets try this backwards too:

\begin{align*}
(I \BB) \cdot v
&= \left(\sum B^i \underbrace{\gamma_{102030i0}}_{\gamma_{123i}} \right) \cdot \left( \sum v^{\mu} \gamma_{\mu} \right) \\
&= \sum B^i v^{\mu} \gpgradeone{\gamma_{123i\mu}}
\end{align*}

That vector selection does yield the cross product as expected:

\begin{equation*}
\gpgradeone{\gamma_{123i\mu}} =
\left\{ 
\begin{array}{l l}
0 & \quad \mu = 0 \\
0 & \quad i = \mu \\
\gamma_1 & \quad i\mu = 32 \\
-\gamma_2 & \quad i\mu = 31 \\
\gamma_3 & \quad i\mu = 21 \\
\end{array} \right.
\end{equation*}

(with alternation for the missing set of index pairs).

This gives:
\begin{align*}
(I \BB) \cdot v
= (B^3 v^2 - B^2 v^3) \gamma_1
+ (B^1 v^3 - B^3 v^1) \gamma_2
+ (B^2 v^1 - B^1 v^2) \gamma_3,
\end{align*}

thus, since $v^i = \gamma d{x^i}/dt$, this yields the desired result

\begin{equation*}
((I\BB) \cdot v) \wedge \gamma_0 = \gamma \Bv \cross \BB
\end{equation*}

In retrospect, the GAFP way is clearer and easier, than to try to do it the dumb way.

Combining the results we have:

\begin{align*}
\dot{p} \wedge \gamma_0 
&= q \gamma ( \BE + \Bv \cross \BB ) \\
&= q (( \BE + c I \BB ) \cdot (v/c)) \wedge \gamma_0 \\
\end{align*}

Or with $F = \BE + c I \BB$, we have:

\begin{equation}
\dot{p} \wedge \gamma_0 = q ( F \cdot v/c ) \wedge \gamma_0
\end{equation}

It is tempting here to attempt to cancel the $\wedge \gamma_0$ parts of this equation, but that cannot be done
until one also shows:

\begin{equation*}
\dot{p} \cdot \gamma_0 = q ( F \cdot v/c ) \cdot \gamma_0
\end{equation*}

\end{document}               
