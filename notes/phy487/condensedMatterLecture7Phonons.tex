%
% Copyright © 2013 Peeter Joot.  All Rights Reserved.
% Licenced as described in the file LICENSE under the root directory of this GIT repository.
%
\section{Phonons}

Here will do an introductory calculation not in the text.  Consider a 1D chain of $N$ atoms, coupled by harmonic springs with periodic boundary conditions.  We suppose that we have $N \sim 10^{23}$.  This is illustrated in \cref{fig:qmSolidsL7:qmSolidsL7Fig4}.

\imageFigure{figures/qmSolidsL7Fig4}{Coupled period oscillators}{fig:qmSolidsL7:qmSolidsL7Fig4}{0.3}

With equilibrium positions $x_j$, and displacement distances from equilibrium of $u_j$, as in \cref{fig:qmSolidsL7:qmSolidsL7Fig5}.

\imageFigure{figures/qmSolidsL7Fig5}{Equilibrium and displacement positions}{fig:qmSolidsL7:qmSolidsL7Fig5}{0.15}

Our force balance is

\begin{dmath}\label{eqn:condensedMatterLecture7:180}
m \uddot_j = K \lr{ u_{j + 1} - u_j } + K \lr{ u_{j - 1} - u_j} 
\end{dmath}

We have $10^{23}$ coupled equations.

In class we used trial solutions of the form

\begin{dmath}\label{eqn:condensedMatterLecture7:200}
u_j = \inv{\sqrt{m}} \sum_q u_q e^{i \lr{ q x_j - \omega_q t} }.
\end{dmath}

The periodicity requirement imposes a constraint on

\begin{dmath}\label{eqn:condensedMatterLecture7:220}
e^{i q( x_j + N a) },
\end{dmath}

so that 

\begin{dmath}\label{eqn:condensedMatterLecture7:240}
q \mathLabelBox{N a}{$L$} = 2 \pi n,
\end{dmath}

or
\begin{dmath}\label{eqn:condensedMatterLecture7:260}
q = \frac{2 \pi n}{L}.
\end{dmath}

Plugging this in and working through a rough Fourier argument provides a constraint on $\omega$, but we can get to that constraint more easily by first considering one component of that solution in isolation.  This follows the outline in \citep{kdasgupta:ph409} where we assume a trial solution of

\begin{dmath}\label{eqn:condensedMatterLecture7:280}
u_n = \epsilon e^{i q n a - \omega t}.
\end{dmath}

Our derivatives are
\begin{subequations}
\begin{dmath}\label{eqn:condensedMatterLecture7:300}
\dot{u}_n = -i \omega \epsilon e^{i q n a - \omega t}
\end{dmath}
\begin{dmath}\label{eqn:condensedMatterLecture7:320}
\ddot{u}_n = - \omega^2 \epsilon e^{i q n a - \omega t},
\end{dmath}
\end{subequations}

and insertion back into \eqnref{eqn:condensedMatterLecture7:180} gives

\begin{dmath}\label{eqn:condensedMatterLecture7:340}
0 
= \epsilon e^{-i\omega t}
\lr{
\frac{m}{K} \omega^2 e^{i q n a} + e^{i q(n + 1) a} - 2 e^{i q n a} + e^{i q (n-1) a}
}
= \epsilon e^{-i\omega t}
e^{i q n a}
\lr{
\frac{m}{K} \omega^2 + e^{i q a} - 2 + e^{-i q a}
}
= \epsilon e^{-i\omega t}
e^{i q n a}
\lr{
\frac{m}{K} \omega^2 + -2 + 2 \cos q a 
}
= \epsilon e^{-i\omega t}
e^{i q n a}
\lr{
\frac{m}{K} \omega^2 + - 4 \sin^2 \lr{ \frac{ q a}{2} }
}.
\end{dmath}

Requiring equality means that we must have

\begin{dmath}\label{eqn:condensedMatterLecture7:360}
\myBoxed{
\sqrt{\frac{m}{K}} \omega = 2 \sin \lr{ \frac{ q a}{2} }.
}
\end{dmath}

TODO: write out this same argument with the Fourier series treatment we did in class (which puts an index on $\epsilon$ and $\omega$.)

%\EndArticle
