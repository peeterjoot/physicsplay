%
% Copyright � 2013 Peeter Joot.  All Rights Reserved.
% Licenced as described in the file LICENSE under the root directory of this GIT repository.
%
\makeoproblem{Ewald construction for a two-dimensional lattice}{condensedMatter:problemSet3:3}{2013 ps3 p3}{ 
%(Based on question 3.4 in Ibach and Luth.)
Show that the reciprocal lattice of a
two-dimensional lattice can be represented by rods.  Discuss the
Ewald construction for diffraction from a two-dimensional lattice
and determine the diffracted beam for a particular orientation and
magnitude of \(\Bk_\nought\).  (It helps to draw a three-dimensional
diagram showing the plane of the reciprocal lattice, the rods, and
the Ewald sphere.)

Explain why you always see a diffraction pattern from a two-dimensional
crystal, provided that the magnitude of \(\Bk_\nought\) exceeds a
critical value, regardless of its orientation.

} % makeproblem

\makeanswer{condensedMatter:problemSet3:3}{ 

Let's consider first the reciprocal frame for a 2D surface by direct computation.  In matrix form we want to solve 

\begin{dmath}\label{eqn:condensedMatterProblemSet3Problem3:20}
\begin{bmatrix}
\Ba_1^\T \\
\Ba_2^\T
\end{bmatrix}
\begin{bmatrix}
\Bg_1 & \Bg_2
\end{bmatrix}
= 
2 \pi
\begin{bmatrix}
1 & 0 \\
0 & 1
\end{bmatrix}.
\end{dmath}

Inverting yields the reciprocal frame vectors in columnar matrix form

\begin{dmath}\label{eqn:condensedMatterProblemSet3Problem3:40}
\begin{bmatrix}
\Bg_1 & \Bg_2
\end{bmatrix}
=
2 \pi
{\begin{bmatrix}
a_{11} & a_{12} \\
a_{21} & a_{22}
\end{bmatrix}}^{-1}
=
\frac{2 \pi}{ a_{11} a_{22} - a_{12} a_{21} }
\begin{bmatrix}
a_{22} & -a_{12} \\
-a_{21} & a_{11}
\end{bmatrix},
\end{dmath}

or

\begin{subequations}
\label{eqn:condensedMatterProblemSet3Problem3:60a}
\begin{dmath}\label{eqn:condensedMatterProblemSet3Problem3:60}
\Bg_1 
=
\frac{2 \pi}{ a_{11} a_{22} - a_{12} a_{21} }
\begin{bmatrix}
a_{22} \\
-a_{21} 
\end{bmatrix}
\end{dmath}
\begin{dmath}\label{eqn:condensedMatterProblemSet3Problem3:80}
\Bg_2 
=
\frac{2 \pi}{ a_{11} a_{22} - a_{12} a_{21} }
\begin{bmatrix}
-a_{12} \\
a_{11}
\end{bmatrix}
\end{dmath}
\end{subequations}

This is considerably messier than the cross product formulation that we used in 3D.  Given familiarity with the geometric algebra formalism of \citep{doran2003gap} it is clear that can be tidied up nicely.  
Introducing a planar pseudoscalar \(I = \Be_1 \wedge \Be_2\), and using the distribution identity \(\Ba \cdot (\Bb \wedge \Bc) = (\Ba \cdot \Bb) \Bc - (\Ba \cdot \Bc) \Bb\), the end result is somewhat reminiscent of the cross product result from the text and class

\begin{subequations}
\begin{dmath}\label{eqn:condensedMatterProblemSet3Problem3:100}
\Bg_1 
=
\frac{2 \pi}{ (\Ba_1 \wedge \Ba_2) I }
\Ba_2 \cdot I
\end{dmath}
\begin{dmath}\label{eqn:condensedMatterProblemSet3Problem3:120}
\Bg_2 
=
-\frac{2 \pi}{ (\Ba_1 \wedge \Ba_2) I}
\Ba_1 \cdot I,
\end{dmath}
\end{subequations}

We have no reason to introduce a third dimension in either position or momentum space, but as with angular momentum and other quantities that are naturally planar (and thus logically expressed as bivector wedge products), we can also introduce an additional dimension so that we can work with the old familiar 3D toolbox (i.e. cross products).

Let's try such a 3D extension of the lattice space and see if the results are consistent, and what the reciprocal frame vectors are.  To do so we extend the lattice to a triplet \(\{ \Ba_1, \Ba_2, \Ba_3 \}\), where \(\Ba_3 = \Be_3\), a unit vector in a normal direction to the plane, now extended to 3D.  Our reciprocal frame vectors are

\begin{subequations}
\begin{dmath}\label{eqn:condensedMatterProblemSet3Problem3:180}
\Bg_1 = \frac{2 \pi}{ \Be_3 \cdot (\Ba_1 \cross \Ba_2) } \Ba_2 \cross \Be_3
\end{dmath}
\begin{dmath}\label{eqn:condensedMatterProblemSet3Problem3:200}
\Bg_2 = \frac{2 \pi}{ \Be_3 \cdot (\Ba_1 \cross \Ba_2) } \Be_3 \cross \Ba_1
\end{dmath}
\begin{dmath}\label{eqn:condensedMatterProblemSet3Problem3:220}
\Bg_3 = \frac{2 \pi}{ \Be_3 \cdot (\Ba_1 \cross \Ba_2) } \Ba_1 \cross \Ba_2
\end{dmath}
\end{subequations}

Observe that the triple produce is the same determinant that we found by matrix inversion for the duality calculation

\begin{dmath}\label{eqn:condensedMatterProblemSet3Problem3:240}
\Be_3 \cdot (\Ba_1 \cross \Ba_2) 
=
\Be_3 \cdot \lr{
(a_{11} \Be_1 + a_{12} \Be_2) \cross
(a_{21} \Be_1 + a_{22} \Be_2)
}
=
\Be_3 \cdot \lr{
a_{11} a_{22} \Be_3 - a_{12} a_{21} \Be_3
}
=
a_{11} a_{22} - a_{12} a_{21}.
\end{dmath}

Expanding out all the 3D extended reciprocal vectors we have

\begin{subequations}
\begin{dmath}\label{eqn:condensedMatterProblemSet3Problem3:260}
\Bg_1 
=
\frac{2 \pi}{ a_{11} a_{22} - a_{12} a_{21} }
\begin{bmatrix}
a_{22} \\
-a_{21} \\
0
\end{bmatrix}
\end{dmath}
\begin{dmath}\label{eqn:condensedMatterProblemSet3Problem3:280}
\Bg_2 
=
\frac{2 \pi}{ a_{11} a_{22} - a_{12} a_{21} }
\begin{bmatrix}
-a_{12} \\
a_{11} \\
0
\end{bmatrix}
\end{dmath}
\begin{dmath}\label{eqn:condensedMatterProblemSet3Problem3:300}
\Bg_3 
=
2 \pi 
\begin{bmatrix}
0 \\
0 \\
0
\end{bmatrix}
\end{dmath}
\end{subequations}

Observe that the first two reciprocal vectors are consistent with the planar computation of \eqnref{eqn:condensedMatterProblemSet3Problem3:60a}.

We see that extension of our lattice frame by introducing an arbitrary set of normal lattice points, introduces an additional reciprocal vector, also normal to those of the purely planar treatment.  Since the scaling of these normal reciprocal vectors are completely arbitrary, we can think of that vector as the span of possible normal lattice points.  This can be characterized as a rod. 

Having arrived at an algebraic meaning for ``rod'' in this context, let's move on to the diagram that was suggested.  This is drafted in \cref{fig:qmSolidsPs3:qmSolidsPs3Fig4}.

\imageFigure{../../figures/phy487/qmSolidsPs3Fig4}{Ewald sphere for 2D lattice}{fig:qmSolidsPs3:qmSolidsPs3Fig4}{0.3}

Two spheres, depicted as circles, are indicated in the figure.  Both have a \(\Bk_\nought\) vector directed downwards towards the plane from the center of the sphere.  

The larger of these spheres (1) passes through the normal ``rods'' in four potential places (really eight, but the additional dimensions are not indicated in the figure).  Each of those positions allows for the construction of a \(\BG\) vector in the reciprocal space, so as indicated in the problem, there will necessarily be diffraction provided the incident wave \(\Bk_\nought\) is a large enough magnitude.

The smaller sphere, marked (2), depicts an incident wave with magnitude insufficient (not past the critical threshold) to cross any of the ``rods''.  We cannot construct a \(\BG\) that has a component normal to the reciprocal lattice frame for such an incident wave, since it is smaller than the critical value required.
}
