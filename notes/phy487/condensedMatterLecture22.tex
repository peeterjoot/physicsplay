%
% Copyright � 2013 Peeter Joot.  All Rights Reserved.
% Licenced as described in the file LICENSE under the root directory of this GIT repository.
%
%\newcommand{\authorname}{Peeter Joot}
\newcommand{\email}{peeterjoot@protonmail.com}
\newcommand{\basename}{FIXMEbasenameUndefined}
\newcommand{\dirname}{notes/FIXMEdirnameUndefined/}

%\renewcommand{\basename}{condensedMatterLecture22}
%\renewcommand{\dirname}{notes/phy487/}
%\newcommand{\keywords}{Condensed matter physics, PHY487H1F}
%\newcommand{\authorname}{Peeter Joot}
\newcommand{\onlineurl}{http://sites.google.com/site/peeterjoot2/math2013/\basename.pdf}
\newcommand{\sourcepath}{\dirname\basename.tex}
\newcommand{\generatetitle}[1]{\chapter{#1}}

\newcommand{\vcsinfo}{%
\section*{}
\noindent{\color{DarkOliveGreen}{\rule{\linewidth}{0.1mm}}}
\paragraph{Document version}
%\paragraph{\color{Maroon}{Document version}}
{
\small
\begin{itemize}
\item Available online at:\\ 
\href{\onlineurl}{\onlineurl}
\item Git Repository: \input{./.revinfo/gitRepo.tex}
\item Source: \sourcepath
\item last commit: \input{./.revinfo/gitCommitString.tex}
\item commit date: \input{./.revinfo/gitCommitDate.tex}
\end{itemize}
}
}

%\PassOptionsToPackage{dvipsnames,svgnames}{xcolor}
\PassOptionsToPackage{square,numbers}{natbib}
\documentclass{scrreprt}

\usepackage[left=2cm,right=2cm]{geometry}
\usepackage[svgnames]{xcolor}
\usepackage{peeters_layout}

\usepackage{natbib}

\usepackage[
colorlinks=true,
bookmarks=false,
pdfauthor={\authorname, \email},
backref 
]{hyperref}

% http://tex.stackexchange.com/questions/75773/how-to-reference-problems-by-the-text-label-in-an-exercise-envioronment
\usepackage[english]{cleveref}
\crefname{Exercise}{exercise}{exercises}
\Crefname{Exercise}{Exercise}{Exercises}

\RequirePackage{titlesec}
\RequirePackage{ifthen}

% http://stackoverflow.com/questions/4932910/date-in-the-tabular-environment
\makeatletter
\let\insertdate\@date
\makeatother

\titleformat{\chapter}[display]
{\bfseries\Large}
{\color{DarkSlateGrey}\filleft \authorname
\ifthenelse{\isundefined{\studentnumber}}{}{\\ \studentnumber}
\ifthenelse{\isundefined{\email}}{}{\\ \email}
\ifthenelse{\isundefined{\dateintitle}}{}{\\ \insertdate}
%\ifthenelse{\isundefined{\coursename}}{}{\\ \coursename} % put in title instead.
}
{4ex}
{\color{DarkOliveGreen}{\titlerule}\color{Maroon}
\vspace{2ex}%
\filright}
[\vspace{2ex}%
\color{DarkOliveGreen}\titlerule
]

\newcommand{\beginArtWithToc}[0]{\begin{document}\tableofcontents}
\newcommand{\beginArtNoToc}[0]{\begin{document}}
\newcommand{\EndNoBibArticle}[0]{\end{document}}
\newcommand{\EndArticle}[0]{\bibliography{Bibliography}\bibliographystyle{plainnat}\end{document}}

% 
%\newcommand{\citep}[1]{\cite{#1}}

\colorSectionsForArticle


%
%%\citep{harald2003solid} \S x.y
%%\citep{ibach2009solid} \S x.y
%
%%\usepackage{mhchem}
%\usepackage[version=3]{mhchem}
%\usepackage{units}
%\newcommand{\nought}[0]{\circ}
%%\newcommand{\EF}[0]{\epsilon_{\mathrm{F}}}
%\newcommand{\EF}[0]{E_{\mathrm{F}}}
%\newcommand{\kF}[0]{k_{\mathrm{F}}}
%
%\beginArtNoToc
%\generatetitle{PHY487H1F Condensed Matter Physics.  Lecture 22: Intro to semiconductor physics.  Taught by Prof.\ Stephen Julian}
\label{chap:condensedMatterLecture22}

%\section{Disclaimer}
%
%Peeter's lecture notes from class.  May not be entirely coherent.
%
\section{Conduction and valence bands}
\index{valence conduction}

\reading \citep{ibach2009solid} \S 12.1 \ce{Si}, \ce{Ge}, \ce{C}, \ce{GaAs}

All of these have a small gap at the Fermi energy \(\EF\).  The interesting physics all happens at the top of the valence band and at the bottom of the conduction band as sketched in \cref{fig:qmSolidsL22:qmSolidsL22Fig1}.

\imageFigure{../../figures/phy487/qmSolidsL22Fig1}{Conduction and valence bands}{fig:qmSolidsL22:qmSolidsL22Fig1}{0.2}

Elements and compounds that have four valence electrons have a chance at being semi-conductors because they can form \(s p^3\) hybrid orbitals.  Note that there are also pressure dependencies here since putting enough pressure on a semiconductor will force it to metalize.  That pressure changes the interatomic spacing and the associated energy distribution.

What we really want to figure out is how to calculate the density of the electrons in the conduction band, and the density of the \textAndIndex{holes} that are left behind.

In ps9 we calculated the density of states at a band edge, which is what we have here.  We found for \(E > E_{\mathrm{C}}\)

\begin{equation}\label{eqn:qmSL22:20}
D_{\mathrm{C}}(E) = \frac{ \lr{ 2 m_n^\conj }^{3/2}}{2 \pi^2 \Hbar^3 } \sqrt{ E - E_{\mathrm{C}} },
\end{equation}

and for \(E < E_{\mathrm{V}}\)

\begin{equation}\label{eqn:qmSL22:40}
D_{\mathrm{V}}(E) = \frac{ \lr{ 2 m_p^\conj }^{3/2}}{2 \pi^2 \Hbar^3 } \sqrt{ E_{\mathrm{V}} - E }.
\end{equation}

This gives us

\begin{equation}\label{eqn:qmSL22:60}
\begin{aligned}
n &= \int_{E_{\mathrm{C}}} D_{\mathrm{C}}(E) f(E, T) dE \\
p &= \int^{E_{\mathrm{V}}} D_{\mathrm{V}}(E) (1 - f(E, T)) dE,
\end{aligned}
\end{equation}

for which the density of states and contributing regions are sketched in \cref{fig:qmSolidsL22:qmSolidsL22Fig2}.

\imageFigure{../../figures/phy487/qmSolidsL22Fig2}{Conduction and valence density of states}{fig:qmSolidsL22:qmSolidsL22Fig2}{0.2}

The ``intrinsic'' number of electrons in the \textAndIndex{conduction band} equals the number of holes \index{hole} in the \textAndIndex{valence band}, so that \(n = p:\EF\) adjusts accordingly.

So, for \(E > E_{\mathrm{C}}\) we have approximately the Boltzmann factor for the distribution

\begin{equation}\label{eqn:qmSL22:80}
\inv{e^{(E - \EF)/\kB T} + 1} \approx e^{ -(E - \EF)/\kB T},
\end{equation}

and the electron density is

\begin{dmath}\label{eqn:qmSL22:100}
n = 
\frac{ \lr{ 2 m_n^\conj }^{3/2}}{2 \pi^2 \Hbar^3 } 
e^{\EF/\kB T}
\int_{E_{\mathrm{C}}}^\infty 
\sqrt{ E - E_{\mathrm{C}} } e^{-E/\kB T} dE.
\end{dmath}

Since

\begin{dmath}\label{eqn:qmSL22:101}
\int_a^\infty \sqrt{ E - a } e^{-E/\tau} dE = \inv{2} \sqrt{\pi} \tau^{3/2} e^{-a/\tau},
\end{dmath}

we have

\begin{dmath}\label{eqn:qmSL22:102}
n
=
2 \lr{ \frac{ 2 \pi m_n^\conj \kB T}{ h^2 }  }^{3/2}
e^{-(E_{\mathrm{C}} -\EF)/\kB T}
\end{dmath}

This is written as

\begin{dmath}\label{eqn:qmSL22:120}
n = N_{\mathrm{eff}}^{\mathrm{C}} e^{-(E_{\mathrm{C}} -\EF)/\kB T},
\end{dmath}

where we note that \(N_{\mathrm{eff}}^{\mathrm{C}}\) is temperature dependent.

Similarly

\begin{dmath}\label{eqn:qmSL22:140}
p = 
2 \lr{ \frac{ 2 \pi m_p^\conj \kB T}{ h^2 }  }^{3/2}
e^{-(\EF - E_{\mathrm{V}})/\kB T},
\end{dmath}

or

\begin{dmath}\label{eqn:qmSL22:120b}
p = N_{\mathrm{eff}}^{\mathrm{p}} e^{-(\EF - E_{\mathrm{V}})/\kB T} 
\end{dmath}

\begin{equation}\label{eqn:qmSL22:160}
n_i = p_i 
= \sqrt{N_{\mathrm{eff}}^{\mathrm{C}}N_{\mathrm{eff}}^{\mathrm{V}}} 
e^{-(E_{\mathrm{V}} - E_{\mathrm{C}})/\kB T}
= \sqrt{N_{\mathrm{eff}}^{\mathrm{C}}N_{\mathrm{eff}}^{\mathrm{V}}} 
e^{-E_{\mathrm{S}}/\kB T}
\end{equation}

\section{Doped semiconductors}
\index{doped semiconductors}

Reading: \S 12.3 pp 428-430.

We introduce a ``donor'', such as \ce{P} in \ce{Si}, as sketched in \cref{fig:qmSolidsL22:qmSolidsL22Fig3}.  Four electrons make \(s p^3\) hybrid orbitals with one electron left over.

\imageFigure{../../figures/phy487/qmSolidsL22Fig3}{\ce{P} donor, \(T = 0\) (cf. hydrogen atom)}{fig:qmSolidsL22:qmSolidsL22Fig3}{0.2}

\begin{equation}\label{eqn:qmSL22:180}
E_{\mathrm{binding}} = \frac{ m^\conj e^4 }{ 2 \lr{ 4 \pi \epsilon_\nought \epsilon_r \Hbar}^2 } \inv{n^2} \approx 6 \Unit{m e V}
\end{equation}

Here \(m^\conj\) is the conduction electron effective mass \(\approx 0.2 m_e\), and \(\epsilon_r \approx 10\) in \ce{Si}.

%\cref{fig:qmSolidsL22:qmSolidsL22Fig4}.
\imageFigure{../../figures/phy487/qmSolidsL22Fig4}{Donors are ionized at low \(T\)}{fig:qmSolidsL22:qmSolidsL22Fig4}{0.2}

For ``acceptors'', e.g. \ce{B-} in \ce{Si}, we need one electron to form four \(sp^3\) hybrid orbitals.  One is taken from the valence band.  The density distribution is sketched in \cref{fig:qmSolidsL22:qmSolidsL22Fig5}.

\imageFigure{../../figures/phy487/qmSolidsL22Fig5}{N-type (only donor) temperature dependence}{fig:qmSolidsL22:qmSolidsL22Fig5}{0.2}

For ``intrinsic'', \(\kB T \gg E_{\mathrm{g}}\), \(n_i \gg N_{\mathrm{D}}\).
For ``saturation'', \(n_i \ll N_{\mathrm{D}}\), \(n \approx N_{\mathrm{D}}\).  Donors are ionized, intrinsic carriers are frozen out.
For ``freeze out'' \(\kB T < E_{\mathrm{d}}\)

%\cref{fig:qmSolidsL22:qmSolidsL22Fig6}.
\imageFigure{../../figures/phy487/qmSolidsL22Fig6}{Acceptor energy distribution with temperature}{fig:qmSolidsL22:qmSolidsL22Fig6}{0.2}

\paragraph{Details to follow...}

Two ways to write \(n\)

\begin{dmath}\label{eqn:qmSL22:200}
n 
= N_{\mathrm{D}} \lr{ 1 - \inv{ 1 + e^{(E_{\mathrm{D}} - \EF)}/\kB T}}
= N_{\mathrm{D}} \lr{ 1 - \inv{ 1 + e^{E_{\mathrm{d}}}/\kB T}},
\end{dmath}

Also

...

%\EndArticle 
