%
% Copyright � 2013 Peeter Joot.  All Rights Reserved.
% Licenced as described in the file LICENSE under the root directory of this GIT repository.
%
%\newcommand{\authorname}{Peeter Joot}
\newcommand{\email}{peeterjoot@protonmail.com}
\newcommand{\basename}{FIXMEbasenameUndefined}
\newcommand{\dirname}{notes/FIXMEdirnameUndefined/}

%\renewcommand{\basename}{condensedMatterLecture4}
%\renewcommand{\dirname}{notes/phy487/}
%\newcommand{\keywords}{Condensed matter physics, PHY487H1F}
%\newcommand{\authorname}{Peeter Joot}
\newcommand{\onlineurl}{http://sites.google.com/site/peeterjoot2/math2013/\basename.pdf}
\newcommand{\sourcepath}{\dirname\basename.tex}
\newcommand{\generatetitle}[1]{\chapter{#1}}

\newcommand{\vcsinfo}{%
\section*{}
\noindent{\color{DarkOliveGreen}{\rule{\linewidth}{0.1mm}}}
\paragraph{Document version}
%\paragraph{\color{Maroon}{Document version}}
{
\small
\begin{itemize}
\item Available online at:\\ 
\href{\onlineurl}{\onlineurl}
\item Git Repository: \input{./.revinfo/gitRepo.tex}
\item Source: \sourcepath
\item last commit: \input{./.revinfo/gitCommitString.tex}
\item commit date: \input{./.revinfo/gitCommitDate.tex}
\end{itemize}
}
}

%\PassOptionsToPackage{dvipsnames,svgnames}{xcolor}
\PassOptionsToPackage{square,numbers}{natbib}
\documentclass{scrreprt}

\usepackage[left=2cm,right=2cm]{geometry}
\usepackage[svgnames]{xcolor}
\usepackage{peeters_layout}

\usepackage{natbib}

\usepackage[
colorlinks=true,
bookmarks=false,
pdfauthor={\authorname, \email},
backref 
]{hyperref}

% http://tex.stackexchange.com/questions/75773/how-to-reference-problems-by-the-text-label-in-an-exercise-envioronment
\usepackage[english]{cleveref}
\crefname{Exercise}{exercise}{exercises}
\Crefname{Exercise}{Exercise}{Exercises}

\RequirePackage{titlesec}
\RequirePackage{ifthen}

% http://stackoverflow.com/questions/4932910/date-in-the-tabular-environment
\makeatletter
\let\insertdate\@date
\makeatother

\titleformat{\chapter}[display]
{\bfseries\Large}
{\color{DarkSlateGrey}\filleft \authorname
\ifthenelse{\isundefined{\studentnumber}}{}{\\ \studentnumber}
\ifthenelse{\isundefined{\email}}{}{\\ \email}
\ifthenelse{\isundefined{\dateintitle}}{}{\\ \insertdate}
%\ifthenelse{\isundefined{\coursename}}{}{\\ \coursename} % put in title instead.
}
{4ex}
{\color{DarkOliveGreen}{\titlerule}\color{Maroon}
\vspace{2ex}%
\filright}
[\vspace{2ex}%
\color{DarkOliveGreen}\titlerule
]

\newcommand{\beginArtWithToc}[0]{\begin{document}\tableofcontents}
\newcommand{\beginArtNoToc}[0]{\begin{document}}
\newcommand{\EndNoBibArticle}[0]{\end{document}}
\newcommand{\EndArticle}[0]{\bibliography{Bibliography}\bibliographystyle{plainnat}\end{document}}

% 
%\newcommand{\citep}[1]{\cite{#1}}

\colorSectionsForArticle


%
%%\citep{harald2003solid} \S x.y
%
%\usepackage{mhchem}
%
%\beginArtNoToc
%\generatetitle{PHY487H1F Condensed Matter Physics.  Lecture 4: Crystal structures.  Taught by Prof.\ Stephen Julian}
%%\chapter{Crystal structures}
\label{chap:condensedMatterLecture4}
%
%\section{Disclaimer}
%
%Peeter's lecture notes from class.  May not be entirely coherent.
%
\section{Crystal structures}
\index{crystal structure}

\reading pp. 23-24 \citep{ibach2009solid}.
Handout: Bravais lattices in three dimensions \citep{wiki:bravais}.
\index{Bravais lattice}

We have seven possibilities that fill space (see handout).  We need 3 vectors or three lengths angle pairs to describe the geometries, as in \cref{fig:qmSolidsL4:qmSolidsL4Fig1}.

\imageFigure{../../figures/phy487/qmSolidsL4Fig1}{Lattice geometry}{fig:qmSolidsL4:qmSolidsL4Fig1}{0.15}

Lots of study of hexagonal structures since there appears to be a preference for that in many superconducting materials.

\section{Point group symmetry}

\reading \S 2.2 \citep{ibach2009solid}.

One way to distinguish crystal structures is according to the symmetries that they have.  Each crystal structure has symmetry operations that map the crystal onto itself.

To illustrate \cref{fig:qmSolidsL4:qmSolidsL4Fig2} (also on the slide)

\imageFigure{../../figures/phy487/qmSolidsL4Fig2}{Symmetries of a graphene lattice}{fig:qmSolidsL4:qmSolidsL4Fig2}{0.2}

\begin{itemize}
\item A is a 6 fold rotation axis.
\item B is a 2 fold rotation axis.
\item C is a 3 fold rotation axis.
\item D is a 2 fold rotation axis.
\end{itemize}

A and B are inversion centers, so that (\(\Br \rightarrow -\Br\)) maps any lattice point onto an existing lattice point.

The plane perpendicular to the page, containing \(\BD\), is a mirror plane.

C is a 6-fold rotation-inversion axis, so that under rotation then inversion, each point maps onto an existing lattice point.

Have other symmetries.  For example, the glide plane \(\BE\) has a symmetry operation under inversion followed by translation.

Symmetry operations tell you about degeneracies of energy eigenstates, and the number of independent elastic constants, ...

Group theory ideas of symmetries described too briefly in \S 2.3, 2.4 \citep{ibach2009solid}, read if desired.

\section{Simple crystal structures}

\reading \citep{ashcroft1976solid} \ch 4

Assume one atom per lattice point.  This is called a \underlineAndIndex{1 atom basis}.

% not a very good image.  Want something that shows the cropped portions of the sphere 
\textunderline{Simple cubic} \index{simple cubic} Very rare: only \ce{Po}, which is very toxic.  This is in fact the toxic substance that was used as in the assasination by injection in the UK of Alexander Litvinenko, the former KGB and FSB agent turned undesirable reporter for Chechenia \citep{wiki:litvinenkoPo}.

%\cref{fig:simpleCubicPacking:simpleCubicPackingFig1}.
\imageFigure{../../figures/phy487/simpleCubicPackingFig1}{Simple cubic}{fig:simpleCubicPacking:simpleCubicPackingFig1}{0.2}

\textunderline{Face centered cubic} \index{face centered cubic} (FCC) \index{FCC} (\ce{Cu}, \ce{Ni}, \ce{Au}, \ce{Pd})

This one is very common 

\imageFigure{../../figures/phy452/problemSet2Problem2faceCenteredCubicFig4b}{Element of a face centered cubic}{fig:problemSet2Problem2faceCenteredCubic:problemSet2Problem2faceCenteredCubicFig4}{0.2}

Fig 2.8 in the text.

This is a conventional unit cell. \index{conventional unit cell}

Total 4 atoms per conventional unit cell

\begin{subequations}
\begin{dmath}\label{eqn:condensedMatterLecture4:20}
\mbox{8 corners \(\times \inv{8}\) = 1 atoms}
\end{dmath}
\begin{dmath}\label{eqn:condensedMatterLecture4:40}
\mbox{\(6 \times \inv{2}\) faces = 3 atoms}
\end{dmath}
\end{subequations}

The primitive lattice vectors are

\begin{subequations}
\begin{dmath}\label{eqn:condensedMatterLecture4:60}
\Ba = 
\lr{
\inv{\sqrt{2}},
\inv{\sqrt{2}},
0
} a
\end{dmath}
\begin{dmath}\label{eqn:condensedMatterLecture4:80}
\Bb =
\lr{
\inv{\sqrt{2}},
0,
\inv{\sqrt{2}}
} a
\end{dmath}
\begin{dmath}\label{eqn:condensedMatterLecture4:100}
\Bc =
\lr{
0,
\inv{\sqrt{2}},
\inv{\sqrt{2}}
} a
\end{dmath}
\end{subequations}

Describing this \underlineAndIndex{Bravais} lattice as

\begin{dmath}\label{eqn:condensedMatterLecture4:120}
\Br_n = n_1 \Ba + n_2 \Bb + n_3 \Bc.
\end{dmath}

The primitive unit cell contains 1 lattice point.  In \citep{ashcroft1976solid} \ch 4 a distinction is made between a Bravais lattice structure, and a Bravais lattice basis \eqnref{eqn:condensedMatterLecture4:120}.  The Bravais lattice is the smallest cell that can be repeated to recover the periodic structure.  This does not neccessarily contain just a single atom.  Two Bravais lattice structure choices are illustrated for a 2D cubic centered lattic in \cref{fig:2dcubicCenteredBravais:2dcubicCenteredBravaisFig1}.  These are both two point lattice structures, with one containing two whole atoms, and the other containing one whole atom and four quarter atoms.

\imageFigure{../../figures/phy487/2dcubicCenteredBravaisFig1}{Two Bravais lattice choices for 2D cubic centered}{fig:2dcubicCenteredBravais:2dcubicCenteredBravaisFig1}{0.2}

A, B, C are 4-fold axes.

Cube diagonal is a 3-fold axis.

Face centered cubic (FCC) is actually a stack of 2-d \underlineAndIndex{hexagonal close packed} (HCP \cref{fig:HexClosePacking:HexClosePackingFig1}) \index{HCP} planes (fig: see slides 04_handout.pdf).

\imageFigure{../../figures/phy487/HexClosePackingFig1}{Hex close packing}{fig:HexClosePacking:HexClosePackingFig1}{0.2}

A common form of dislocation in crystal structures is a mix of face centered and hexagonal close packed.  Not over layer 1 => FCC.  Both lattices have \underlineAndIndex{filling factor} 0.7406.

Transition metals often like FCC because of covalent d bonding, as roughly illustrated in \cref{fig:qmSolidsL4:qmSolidsL4Fig4}.

\imageFigure{../../figures/phy487/qmSolidsL4Fig4}{D orbital bonding in a plane}{fig:qmSolidsL4:qmSolidsL4Fig4}{0.2}

But, for example \ce{Co} has an HCP phase due to s, p, d hybrid orbitals.

In the p-block:  

\begin{itemize}
\item \(sp_2\) hybrids \(\rightarrow\) hexagonal.
\item \(sp_3\) hybrids \(\rightarrow\) diamond.
\end{itemize}

\textunderline{Body centered cubic} \index{body centered cubic} (BCC) \index{BCC}

% from: statMechProblemSet3.nb
%\cref{fig:bccPacking:bccPackingFig1}.
\imageFigure{../../figures/phy487/bccPackingFig1}{Element of a body centered cubic}{fig:bccPacking:bccPackingFig1}{0.2}

Total 3 atoms per conventional unit cell

\begin{subequations}
\begin{dmath}\label{eqn:condensedMatterLecture4:20a}
\mbox{8 corners \(\times \inv{8}\) = 1 atoms}
\end{dmath}
\begin{dmath}\label{eqn:condensedMatterLecture4:40a}
\mbox{1 center = 1 atom}
\end{dmath}
\end{subequations}

BCC has 8 nn, FCC has 12.

FCC is more common, but column 1 of the periodic table are all body centered cubic.

Because n s orbitals are huge

%\cref{fig:qmSolidsL4:qmSolidsL4Fig6}.
\imageFigure{../../figures/phy487/qmSolidsL4Fig6}{Large extent of \(n s\) orbitals}{fig:qmSolidsL4:qmSolidsL4Fig6}{0.2}

\textunderline{Diamond lattice} \index{diamond lattice} 

See: 04_lecture

This is an FCC + 4 tetrahedral holes that are filled due to \(sp_3\) hybridization.

\reading pp. 35, 36 \citep{ibach2009solid}.  Skip phase diagrams.

%\EndArticle
