%
% Copyright � 2013 Peeter Joot.  All Rights Reserved.
% Licenced as described in the file LICENSE under the root directory of this GIT repository.
%

% 
% 
%\chapter{Preface}
% this suppresses an explicit chapter number for the preface.
\chapter*{Preface}%\normalsize
  \addcontentsline{toc}{chapter}{Preface}

This document is based on my lecture notes for the Winter 2013, University of Toronto Condensed Matter Physics course (PHY487H1F), taught by Prof.\ Stephen Julian.  

\paragraph{Official course description:}

``Introduction to the concepts used in the modern treatment of solids.  The student is assumed to be familiar with elementary quantum mechanics.  Topics include:  bonding in solids, crystal structures, lattice vibrations, free electron model of metals, band structure, thermal properties, magnetism and superconductivity (time permitting)''

%In this course there will be an attempt to cover 
%
%\begin{itemize}
%\item Bonding and crystal structure: Why solids form, describing periodic order mathematically, diffraction.
%\item Lattice vibrations: ``elementary excitation'' of a periodic array of atoms (periodicity allows a \(10^{23}\) body problem to be solved).
%\item Electrons in solids: Explaining this introduces a need for quantum mechanics.  
%Periodicity -> band structure -> insulators vs metals.
%\item Electrical conduction in solids.  (metals and semiconductors).
%\end{itemize}

We worked from the text \citep{ibach2009solid}.  I personally found \citep{ashcroft1976solid} not only helpful, but superior in almost every aspect, containing better diagrams, clearer descriptions, more complete and clearer mathematics, and no typos (that I spotted).

\paragraph{This document contains:}

\begin{itemize}
\item Plain old lecture notes.   These mirror what was covered in class, possibly augmented with additional details.

\item Personal notes exploring details that were not clear to me from the lectures, or from the texts associated with the lecture material.

\ifthenelse{\boolean{redacted}}%
{%
\item Assigned problems.  Like anything else take these as is.  I have attempted to either correct errors or mark them as such.%
}%
{\relax}

\item Some worked problems attempted as course prep, for fun, or for test preparation, or post test reflection.

\item Links to Mathematica workbooks associated with this course.

\end{itemize}

My thanks go to Professor Julian for teaching this course.

Peeter Joot  \quad peeterjoot@protonmail.com 
