%
% Copyright � 2013 Peeter Joot.  All Rights Reserved.
% Licenced as described in the file LICENSE under the root directory of this GIT repository.
%
\newcommand{\authorname}{Peeter Joot}
\newcommand{\email}{peeterjoot@protonmail.com}
\newcommand{\basename}{FIXMEbasenameUndefined}
\newcommand{\dirname}{notes/FIXMEdirnameUndefined/}

\renewcommand{\basename}{coupledHarmonicUsingDisplacements}
\renewcommand{\dirname}{notes/phy487/}
\newcommand{\keywords}{Condensed matter physics, PHY487H1F}
\newcommand{\authorname}{Peeter Joot}
\newcommand{\onlineurl}{http://sites.google.com/site/peeterjoot2/math2013/\basename.pdf}
\newcommand{\sourcepath}{\dirname\basename.tex}
\newcommand{\generatetitle}[1]{\chapter{#1}}

\newcommand{\vcsinfo}{%
\section*{}
\noindent{\color{DarkOliveGreen}{\rule{\linewidth}{0.1mm}}}
\paragraph{Document version}
%\paragraph{\color{Maroon}{Document version}}
{
\small
\begin{itemize}
\item Available online at:\\ 
\href{\onlineurl}{\onlineurl}
\item Git Repository: \input{./.revinfo/gitRepo.tex}
\item Source: \sourcepath
\item last commit: \input{./.revinfo/gitCommitString.tex}
\item commit date: \input{./.revinfo/gitCommitDate.tex}
\end{itemize}
}
}

%\PassOptionsToPackage{dvipsnames,svgnames}{xcolor}
\PassOptionsToPackage{square,numbers}{natbib}
\documentclass{scrreprt}

\usepackage[left=2cm,right=2cm]{geometry}
\usepackage[svgnames]{xcolor}
\usepackage{peeters_layout}

\usepackage{natbib}

\usepackage[
colorlinks=true,
bookmarks=false,
pdfauthor={\authorname, \email},
backref 
]{hyperref}

% http://tex.stackexchange.com/questions/75773/how-to-reference-problems-by-the-text-label-in-an-exercise-envioronment
\usepackage[english]{cleveref}
\crefname{Exercise}{exercise}{exercises}
\Crefname{Exercise}{Exercise}{Exercises}

\RequirePackage{titlesec}
\RequirePackage{ifthen}

% http://stackoverflow.com/questions/4932910/date-in-the-tabular-environment
\makeatletter
\let\insertdate\@date
\makeatother

\titleformat{\chapter}[display]
{\bfseries\Large}
{\color{DarkSlateGrey}\filleft \authorname
\ifthenelse{\isundefined{\studentnumber}}{}{\\ \studentnumber}
\ifthenelse{\isundefined{\email}}{}{\\ \email}
\ifthenelse{\isundefined{\dateintitle}}{}{\\ \insertdate}
%\ifthenelse{\isundefined{\coursename}}{}{\\ \coursename} % put in title instead.
}
{4ex}
{\color{DarkOliveGreen}{\titlerule}\color{Maroon}
\vspace{2ex}%
\filright}
[\vspace{2ex}%
\color{DarkOliveGreen}\titlerule
]

\newcommand{\beginArtWithToc}[0]{\begin{document}\tableofcontents}
\newcommand{\beginArtNoToc}[0]{\begin{document}}
\newcommand{\EndNoBibArticle}[0]{\end{document}}
\newcommand{\EndArticle}[0]{\bibliography{Bibliography}\bibliographystyle{plainnat}\end{document}}

% 
%\newcommand{\citep}[1]{\cite{#1}}

\colorSectionsForArticle



%\citep{harald2003solid} \S x.y
%\citep{ibach2009solid} \S x.y
%\reading \citep{ashcroft1976solid} \chaptext N.

%\usepackage{mhchem}
\usepackage[version=3]{mhchem}
\usepackage{units}
%\usepackage{bm} % \bcE
\newcommand{\nought}[0]{\circ}
%\newcommand{\EF}[0]{\epsilon_{\mathrm{F}}}
\newcommand{\EF}[0]{E_{\mathrm{F}}}
\newcommand{\kF}[0]{k_{\mathrm{F}}}

\beginArtNoToc
\generatetitle{Displacement change of variables for coupled SHO}
%\chapter{Displacement change of variables for coupled SHO}
\label{chap:coupledHarmonicUsingDisplacements}

%\section{Displacement change of variables for coupled SHO}

\makeproblem{Displacement change of variables for coupled SHO}{pr:coupledHarmonicUsingDisplacements:1}{
To build intuition with the use of displacement coordinates in lattice problems, consider the simplest ``lattice'' problem, that of a pair of equal masses with harmonic coupling.  
Make a change of variables to displacement coordinates \(\Bu_i = \Br_i - \expectation{\Br_i}\), and solve the equation of motion for the separation without neglecting the terms linear in \(\Bu_i\).
} % makeproblem

\makeanswer{pr:coupledHarmonicUsingDisplacements:1}{

\paragraph{FIXME:} figure.

The Lagrangian for the system in non-displacement coordinates is

\begin{dmath}\label{eqn:coupledHarmonicUsingDisplacements:20}
\LL 
= \inv{2} m \sum_i \lr{ \dot{\Br}_i }^2 - \frac{K}{2} \lr{ \Br_1 - \Br_2 }^2.
\end{dmath}

Expansion in displacement coordinates gives

\begin{dmath}\label{eqn:coupledHarmonicUsingDisplacements:200}
\LL 
= \inv{2} m \sum_i \lr{ \dot{\Bu}_i }^2 - \frac{K}{2} \lr{ 
\Bu_1 - \Bu_2 
+ \expectation{\Br_1 - \Br_2} 
}^2
= \inv{2} m \sum_i \lr{ \dot{\Bu}_i }^2 
- \frac{K}{2} \lr{ \Bu_1 - \Bu_2 }^2
- K \lr{ \Bu_1 - \Bu_2 } \cdot \expectation{\Br_1 - \Br_2}
- \frac{K}{2} \expectation{\Br_1 - \Br_2}^2.
\end{dmath}

Dropping the constant \(\expectation{\Br_1 - \Br_2}^2\) term the displacement coordinate Lagrangian is

\begin{dmath}\label{eqn:coupledHarmonicUsingDisplacements:40}
\LL'
= \inv{2} m \sum_i \lr{ \dot{\Bu}_i }^2 
- \frac{K}{2} \lr{ \Bu_1 - \Bu_2 }^2
- K \lr{ \Bu_1 - \Bu_2 } \cdot \expectation{\Br_1 - \Br_2}.
\end{dmath}

To evaluate the Euler-Lagrange equations 

\begin{dmath}\label{eqn:coupledHarmonicUsingDisplacements:60}
\ddt{} \spacegrad_{\dot{\Bu}_i} \LL' = 
\spacegrad_{\Bu_i} \LL',
\end{dmath}

we note that \(\spacegrad_\Bx \Ba \cdot \Bx = \Ba\), yielding the coupled system

\begin{dmath}\label{eqn:coupledHarmonicUsingDisplacements:80}
\begin{aligned}
m \ddot{\Bu}_1 &= - K \lr{ \Bu_1 - \Bu_2 } - K \expectation{\Br_1 - \Br_2} \\
m \ddot{\Bu}_2 &= - K \lr{ \Bu_2 - \Bu_1 } - K \expectation{\Br_2 - \Br_1}.
\end{aligned}
\end{dmath}

The use of displacement coordinates have complicated the problem, rather than simplifying it, since we could have arrived at this result directly from \eqnref{eqn:coupledHarmonicUsingDisplacements:20}.

With the intent to continue working in displacement coordinates, observe that we can rewrite \eqnref{eqn:coupledHarmonicUsingDisplacements:80} as

\begin{dmath}\label{eqn:coupledHarmonicUsingDisplacements:80}
\begin{aligned}
m \ddot{\Br}_1 &= - K \lr{ \Br_1 - \Br_2 } \\
m \ddot{\Br}_2 &= - K \lr{ \Br_2 - \Br_1 }.
\end{aligned}
\end{dmath}

Returning to the displacement system, we can solve by taking the difference of the equations

\begin{dmath}\label{eqn:coupledHarmonicUsingDisplacements:100}
\frac{d^2}{dt^2} 
\lr{ \Bu_1 - \Bu_2 } 
= - \frac{2 K}{m} \lr{ \Bu_1 - \Bu_2 } - \frac{2 K}{m} \expectation{\Br_1 - \Br_2}.
\end{dmath}

With no external forces acting on the system, the direction of the displacement distance will not change over time, so this is really a one dimensional system of the form

\begin{dmath}\label{eqn:coupledHarmonicUsingDisplacements:120}
\begin{aligned}
\ddot{u} &= - \Omega^2 u - \Omega^2 a \\
\Omega &= \frac{2 K}{m}.
\end{aligned}
\end{dmath}

We can solve for \(u + a\), noting that \(d^2(u + a)/dt^2 = \ddot{u}\), so

\begin{dmath}\label{eqn:coupledHarmonicUsingDisplacements:140}
u(t) + a = \lr{ u_\nought + a } \cos\lr{ \Omega (t - t_\nought) },
\end{dmath}

where \(t_\nought\) is one of the the times for which the amplitude is greatest.  Switching back to vector notation, we have

\boxedEquation{eqn:coupledHarmonicUsingDisplacements:160}{
\Bu_1(t) - \Bu_2(t)
=
-\expectation{\Br_1 - \Br_2}
+
\lr{ \Bu_1(t_\nought) - \Bu_2(t_\nought) + \expectation{\Br_1 - \Br_2} }
\cos\lr{ \Omega (t - t_\nought) }.
}

%Grouping the \(\expectation{ \Br_1 - \Br_2 }\) terms, this is
%\begin{dmath}\label{eqn:coupledHarmonicUsingDisplacements:220}
%\Bu_1(t) - \Bu_2(t)
%=
%-2 \expectation{\Br_1 - \Br_2} \sin^2\lr{ \Omega (t - t_\nought)/2}
%+
%\lr{ \Bu_1(t_\nought) - \Bu_2(t_\nought) }
%\cos\lr{ \Omega (t - t_\nought) }.
%\end{dmath}

Going full circle, we can write this as

\begin{dmath}\label{eqn:coupledHarmonicUsingDisplacements:180}
\Br_1(t) - \Br_2(t) = 
\lr{ \Br_1(t_\nought) - \Br_2(t_\nought) }
\cos\lr{ \Omega (t - t_\nought) }.
\end{dmath}

\paragraph{Average separation}

Calculation of the time average \(\expectation{f(t)} = \inv{T} \int_0^T f(t) dt\) of \eqnref{eqn:coupledHarmonicUsingDisplacements:180} yields

\begin{dmath}\label{eqn:coupledHarmonicUsingDisplacements:280}
\expectation{\Br_1(t) - \Br_2(t)} = 0,
\end{dmath}

a result that did not jive with my intuition.  I was somewhat loose with the use of average position \(\expectation{\Br_i}\) when defining the displacement coordinates, and assumed along the way that the average operation was linear in the following ways

\begin{dmath}\label{eqn:coupledHarmonicUsingDisplacements:300}
\begin{aligned}
\expectation{\Ba - \Bb} &= -\expectation{\Bb - \Ba} \\
\expectation{\Ba} - \expectation{\Bb} &= \expectation{\Ba} - \expectation{\Bb}
\end{aligned}
\end{dmath}

While the time average of the separation is zero, the mean square average separation is not

\begin{dmath}\label{eqn:coupledHarmonicUsingDisplacements:260}
\expectation{ \Abs{\Br_1(t) - \Br_2(t)}^2 } = \inv{2} 
\Abs{\Br_1(t_0) - \Br_2(t_0)}^2.
\end{dmath}

I.e. the RMS absolute difference in position of the two masses is \(1/\sqrt{2}\) of the maximum absolute amplitude.  This quantity won't vanish due to the oscillatory nature of the vectors, and probably better describes the average geometric separation of the masses.

\paragraph{FIXME:}
Solve for \(\Bu_i\) (and thus \(\Br_i\)) separately.
} % makeanswer

%\EndArticle
\EndNoBibArticle
