%
% Copyright � 2013 Peeter Joot.  All Rights Reserved.
% Licenced as described in the file LICENSE under the root directory of this GIT repository.
%
\newcommand{\authorname}{Peeter Joot}
\newcommand{\email}{peeter.joot@utoronto.ca}
\newcommand{\studentnumber}{920798560}
\newcommand{\basename}{FIXMEbasenameUndefined}
\newcommand{\dirname}{notes/FIXMEdirnameUndefined/}

\renewcommand{\basename}{condensedMatterProblemSet1}
\renewcommand{\dirname}{notes/phy487/}
\newcommand{\keywords}{Condensed Matter Physics, PHY487H1F}
\newcommand{\dateintitle}{}
\newcommand{\authorname}{Peeter Joot}
\newcommand{\onlineurl}{http://sites.google.com/site/peeterjoot2/math2013/\basename.pdf}
\newcommand{\sourcepath}{\dirname\basename.tex}
\newcommand{\generatetitle}[1]{\chapter{#1}}

\newcommand{\vcsinfo}{%
\section*{}
\noindent{\color{DarkOliveGreen}{\rule{\linewidth}{0.1mm}}}
\paragraph{Document version}
%\paragraph{\color{Maroon}{Document version}}
{
\small
\begin{itemize}
\item Available online at:\\ 
\href{\onlineurl}{\onlineurl}
\item Git Repository: \input{./.revinfo/gitRepo.tex}
\item Source: \sourcepath
\item last commit: \input{./.revinfo/gitCommitString.tex}
\item commit date: \input{./.revinfo/gitCommitDate.tex}
\end{itemize}
}
}

%\PassOptionsToPackage{dvipsnames,svgnames}{xcolor}
\PassOptionsToPackage{square,numbers}{natbib}
\documentclass{scrreprt}

\usepackage[left=2cm,right=2cm]{geometry}
\usepackage[svgnames]{xcolor}
\usepackage{peeters_layout}

\usepackage{natbib}

\usepackage[
colorlinks=true,
bookmarks=false,
pdfauthor={\authorname, \email},
backref 
]{hyperref}

% http://tex.stackexchange.com/questions/75773/how-to-reference-problems-by-the-text-label-in-an-exercise-envioronment
\usepackage[english]{cleveref}
\crefname{Exercise}{exercise}{exercises}
\Crefname{Exercise}{Exercise}{Exercises}

\RequirePackage{titlesec}
\RequirePackage{ifthen}

% http://stackoverflow.com/questions/4932910/date-in-the-tabular-environment
\makeatletter
\let\insertdate\@date
\makeatother

\titleformat{\chapter}[display]
{\bfseries\Large}
{\color{DarkSlateGrey}\filleft \authorname
\ifthenelse{\isundefined{\studentnumber}}{}{\\ \studentnumber}
\ifthenelse{\isundefined{\email}}{}{\\ \email}
\ifthenelse{\isundefined{\dateintitle}}{}{\\ \insertdate}
%\ifthenelse{\isundefined{\coursename}}{}{\\ \coursename} % put in title instead.
}
{4ex}
{\color{DarkOliveGreen}{\titlerule}\color{Maroon}
\vspace{2ex}%
\filright}
[\vspace{2ex}%
\color{DarkOliveGreen}\titlerule
]

\newcommand{\beginArtWithToc}[0]{\begin{document}\tableofcontents}
\newcommand{\beginArtNoToc}[0]{\begin{document}}
\newcommand{\EndNoBibArticle}[0]{\end{document}}
\newcommand{\EndArticle}[0]{\bibliography{Bibliography}\bibliographystyle{plainnat}\end{document}}

% 
%\newcommand{\citep}[1]{\cite{#1}}

\colorSectionsForArticle



\usepackage{mhchem}
\newcommand{\nought}[0]{\circ}
%\newcommand{\CV}[0]{C_{\txtV}}
\newcommand{\cA}[0]{c_{\txtA}}

\beginArtNoToc
\generatetitle{PHY487H1F Condensed Matter Physics.  Problem Set 5: Density of states and Debye temperature}
%\chapter{Density of states and Deybe temperature}
\label{chap:condensedMatterProblemSet1}

%\section{Disclaimer}
%
%This is an ungraded set of answers to the problems posed.

%\begin{center}
%{\bf PHY 487/1487 Problem Set \# 5} \\
%Due Friday Oct.\ 18 by 4:30 p.m., MP324 \\
%\end{center}
%
%Note 1: I will be away from the University on Thursday and Friday, so if you 
%have questions you will have to send email to 
%{\em sjulian@physics.utoronto.ca}.  I apologize for any inconvenience. 
%The lecture on Friday the 18th will be given by Professor Young-June Kim, 
%and assignments can be submitted to him. 
%
%Note 2:  This problem set will be graded and returned by Monday, 
% 21 October at 2:00 p.m., so late assignments will not be accepted beyond 
% that time.  If you have a valid medical or personal excuse, and 
% cannot submit the problem set by that time, I will ignore this 
% problem set in calculating your final mark.
%
%Note 3: 
%Collaboration is encouraged, but direct copying is not: make sure you 
%understand your solutions.  (At least some exam questions will be 
%based closely on problem set questions.)  
 
%%
% Copyright � 2013 Peeter Joot.  All Rights Reserved.
% Licenced as described in the file LICENSE under the root directory of this GIT repository.
%
\makeoproblem{Density of states of a 1-d chain}{condensedMatter:problemSet5:1}{2013 ps5 p1}{ 
Calculate and sketch a plot of the density of states, \(Z(\omega)\), for 
the vibrational modes of a 1-d monatomic chain of length \(L\), 
with nearest-neighbour spring constant K, atoms of mass \(M\), and 
lattice constant \(a\). 
Specifically, start from \(\sum_q\) and by transforming this into 
an integral over \(\omega\), obtain \(Z(\omega)\).   Then 
draw a sketch of \(Z(\omega)\) vs.\ \(\omega\), labeling intercepts and 
asymptotes.

} % makeproblem

\makeanswer{condensedMatter:problemSet5:1}{ 

For the 2D and 3D (\(d = 2,3\)) density of states we'd consider solutions for \(Z(\omega)\) of

\begin{dmath}\label{eqn:condensedMatterProblemSet5Problem1:20}
\int Z(\omega) d\omega = \lr{\frac{L}{2\pi}}^d \int \frac{d \Bf_\omega}{\Abs{\spacegrad_\Bq \omega(q)} } d\omega.
\end{dmath}

Should we wish to extend this down to \(d = 1\) we'd have to figure out how to interpret \(d\Bf_\omega\).  In 2D and 3D that was a surface area element, a factor of the differential form \(d^d \Bq = d\Bf_\omega d\omega_\perp\).  In 3D we had \(\int d\Bf_\omega = 4 \pi q^2 = d/dq( 4 \pi q^3/3)\), and for 2D \(\int d\Bf_\omega = 2 \pi q = d/dq( \pi q^2 )\).  

Those 3D and 2D ``volumes'' (differentiated to obtain the ``area'' when \(q\) of the surface for \(q\) held constant) can be obtained by these respective integrals

\begin{subequations}
\begin{dmath}\label{eqn:condensedMatterProblemSet5Problem1:40}
\int_{x^2 + y^2 + z^2 \le q^2} dx dy dz = \frac{4}{3} \pi q^3
\end{dmath}
\begin{dmath}\label{eqn:condensedMatterProblemSet5Problem1:60}
\int_{x^2 + y^2 \le q^2} dx dy  = \pi q^2.
\end{dmath}
\end{subequations}

We can generalize this down to a single dimension by considering

\begin{dmath}\label{eqn:condensedMatterProblemSet5Problem1:80}
\int_{x^2 \le q^2} dx = 2 q
\end{dmath}

for which we could conceivably consider the area of this 1D surface to be the constant \(2\).  However, does this even make sense, since writing \(dq = df_\omega dq_\perp\) would split our 1-form into the product of two 1-forms, which isn't a sensible operation?  Let's step back and consider the density of states definition from scratch.

\paragraph{Starting from scratch}

We wish to sum over all the integer values \(n\), subject to a period constraint \(2 \pi n = q L\), and employ an integral approximation to this sum.

\begin{dmath}\label{eqn:condensedMatterProblemSet5Problem1:100}
\sum_n
\sim \int dn
= \frac{L}{2 \pi} \int dq
= \frac{L}{2 \pi} \int \frac{dq}{d\omega} d\omega
\equiv \int Z(\omega) d\omega,
\end{dmath}

From this we find for one dimension

\begin{dmath}\label{eqn:condensedMatterProblemSet5Problem1:120}
Z(\omega) = \frac{L}{2 \pi} \frac{dq}{d\omega}.
\end{dmath}

Now we are ready to start.  For the 1D chain we had

\begin{dmath}\label{eqn:condensedMatterProblemSet5Problem1:140}
\sqrt{\frac{M}{K}} \omega(q) = 2 \sin \lr{ \frac{ q a}{2} },
\end{dmath}

so

\begin{dmath}\label{eqn:condensedMatterProblemSet5Problem1:160}
\sqrt{\frac{M}{K}} = a \cos \lr{ \frac{ q a}{2} } \frac{dq}{d\omega},
\end{dmath}

or

\begin{dmath}\label{eqn:condensedMatterProblemSet5Problem1:180}
Z(\omega) 
= \frac{L}{2 \pi} 
\frac{\sqrt{\frac{M}{K}}}{a \cos \lr{ \frac{ q a}{2} }}
= 
\sqrt{\frac{M}{K}} \frac{L}{2 \pi a} 
\inv{ \cos \lr{ \frac{ q a}{2} }}
= 
\sqrt{\frac{M}{K}} \frac{L}{2 \pi a} 
\inv{ \cos \sin^{-1} \lr{ 
\inv{2} \sqrt{\frac{M}{K}} \omega } }
= 
\inv{2} \sqrt{\frac{M}{K}} \frac{L}{\pi a} 
\inv{ \sqrt{1 - \inv{4} \frac{M}{K} \omega^2 } }.
\end{dmath}

With \(L = N a\), this is

\boxedEquation{eqn:condensedMatterProblemSet5Problem1:200}{
Z(\omega)
= 
\inv{2} \sqrt{\frac{M}{K}} \frac{N}{\pi} 
\inv{ \sqrt{1 - \inv{4} \frac{M}{K} \omega^2 } }.
}

This has a minimum at \(\omega = 0\), and in that neighborhood is approximately parabolic function

\begin{dmath}\label{eqn:condensedMatterProblemSet5Problem1:220}
Z(\omega \approx 0)
= 
\inv{2} \sqrt{\frac{M}{K}} \frac{N}{\pi} 
\lr{ 1 - \lr{ - \inv{2}} \inv{4} \frac{M}{K} \omega^2 }
= 
\inv{2} \sqrt{\frac{M}{K}} \frac{N}{\pi} 
\lr{ 1 + \inv{8} \frac{M}{K} \omega^2 }.
\end{dmath}

As \(\omega \rightarrow \pm \sqrt{4 K/M}\), the density of states approaches vertical asymptotes \(Z(\omega) \rightarrow \infty\).  Observe that these extremes are the edges of the Brillouin zone where \(q a/2 = \pm \pi/2\).  For \(Z(\omega)\) to be useful for probability calculations, we expect that the integral over this first Brillouin zone will be finite, despite these infinite asymptotes.  Let's verify this

\begin{dmath}\label{eqn:condensedMatterProblemSet5Problem1:240}
\int_{-\sqrt{4K/M}}^{\sqrt{4K/M}} Z(\omega) d\omega
=
\inv{2} \sqrt{\frac{M}{K}} \frac{N}{\pi} 
\int_{-1}^1
\sqrt{ \frac{4 K}{M} } 
dx
\inv{ \sqrt{1 - x^2 } }
= \frac{N}{\pi} \pi
= N.
\end{dmath}

Good, the area under the curve is finite as expected.  This curve is sketched in \cref{fig:1dSpringLatticeDensityOfStates:1dSpringLatticeDensityOfStatesFig1}.

\imageFigure{../../figures/phy487/1dSpringLatticeDensityOfStatesFig1}{1D density of states for Harmonic chain}{fig:1dSpringLatticeDensityOfStates:1dSpringLatticeDensityOfStatesFig1}{0.4}

}

%
% Copyright � 2013 Peeter Joot.  All Rights Reserved.
% Licenced as described in the file LICENSE under the root directory of this GIT repository.
%
\makeproblem{Systematic trends in the Debye temperature}{condensedMatter:problemSet5:2}{ 
%:} (worth 5 marks)
Table 5.1 on page 
120 of Ibach and Luth shows the Debye temperature for  
various solids.  Discuss and explain any trends that you see in the Debye 
temperature, e.g.\ as a function of location in the periodic table, 
bonding type, or atomic mass. 

} % makeproblem

\makeanswer{condensedMatter:problemSet5:2}{ 

TODO.
}


%%
% Copyright � 2013 Peeter Joot.  All Rights Reserved.
% Licenced as described in the file LICENSE under the root directory of this GIT repository.
%
\makeproblem{Debye calculation in two dimensions}{condensedMatter:problemSet5:3}{ 
Repeat the Debye theory calculation that we did in class, but for a 
two-dimensional lattice.  Assume (quite artificially) that the atoms 
are free to move only within the plane, so that there are $2rN$ degrees 
of freedom,  and there is only one transverse acoustic 
phonon mode, instead of two as in the three-dimensional calculation.

Show that the low temperature limit of the specific heat at constant 
area, per unit area, is:
\begin{eqnarray*}
c_A(T) = 7.213\, \frac{4rN}{A}\,k_B\,\frac{T^2}{\Theta^2},
\end{eqnarray*}
where $A$ is the area of the crystal, $rN$ is the number of atoms 
in the crystal, 
$\Theta$ is defined by $k_B\Theta = \hbar\omega_D$, 
and 
\begin{eqnarray*}
\int_{0}^{\infty} \frac{y^3 e^y}{(e^y-1)^2}\,dy \simeq 7.213. 
\end{eqnarray*}

} % makeproblem

\makeanswer{condensedMatter:problemSet5:3}{ 

TODO.
}



\EndArticle
