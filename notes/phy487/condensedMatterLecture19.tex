%
% Copyright � 2013 Peeter Joot.  All Rights Reserved.
% Licenced as described in the file LICENSE under the root directory of this GIT repository.
%
%\newcommand{\authorname}{Peeter Joot}
\newcommand{\email}{peeterjoot@protonmail.com}
\newcommand{\basename}{FIXMEbasenameUndefined}
\newcommand{\dirname}{notes/FIXMEdirnameUndefined/}

%\renewcommand{\basename}{condensedMatterLecture19}
%\renewcommand{\dirname}{notes/phy487/}
%\newcommand{\keywords}{Condensed matter physics, PHY487H1F}
%\newcommand{\authorname}{Peeter Joot}
\newcommand{\onlineurl}{http://sites.google.com/site/peeterjoot2/math2013/\basename.pdf}
\newcommand{\sourcepath}{\dirname\basename.tex}
\newcommand{\generatetitle}[1]{\chapter{#1}}

\newcommand{\vcsinfo}{%
\section*{}
\noindent{\color{DarkOliveGreen}{\rule{\linewidth}{0.1mm}}}
\paragraph{Document version}
%\paragraph{\color{Maroon}{Document version}}
{
\small
\begin{itemize}
\item Available online at:\\ 
\href{\onlineurl}{\onlineurl}
\item Git Repository: \input{./.revinfo/gitRepo.tex}
\item Source: \sourcepath
\item last commit: \input{./.revinfo/gitCommitString.tex}
\item commit date: \input{./.revinfo/gitCommitDate.tex}
\end{itemize}
}
}

%\PassOptionsToPackage{dvipsnames,svgnames}{xcolor}
\PassOptionsToPackage{square,numbers}{natbib}
\documentclass{scrreprt}

\usepackage[left=2cm,right=2cm]{geometry}
\usepackage[svgnames]{xcolor}
\usepackage{peeters_layout}

\usepackage{natbib}

\usepackage[
colorlinks=true,
bookmarks=false,
pdfauthor={\authorname, \email},
backref 
]{hyperref}

% http://tex.stackexchange.com/questions/75773/how-to-reference-problems-by-the-text-label-in-an-exercise-envioronment
\usepackage[english]{cleveref}
\crefname{Exercise}{exercise}{exercises}
\Crefname{Exercise}{Exercise}{Exercises}

\RequirePackage{titlesec}
\RequirePackage{ifthen}

% http://stackoverflow.com/questions/4932910/date-in-the-tabular-environment
\makeatletter
\let\insertdate\@date
\makeatother

\titleformat{\chapter}[display]
{\bfseries\Large}
{\color{DarkSlateGrey}\filleft \authorname
\ifthenelse{\isundefined{\studentnumber}}{}{\\ \studentnumber}
\ifthenelse{\isundefined{\email}}{}{\\ \email}
\ifthenelse{\isundefined{\dateintitle}}{}{\\ \insertdate}
%\ifthenelse{\isundefined{\coursename}}{}{\\ \coursename} % put in title instead.
}
{4ex}
{\color{DarkOliveGreen}{\titlerule}\color{Maroon}
\vspace{2ex}%
\filright}
[\vspace{2ex}%
\color{DarkOliveGreen}\titlerule
]

\newcommand{\beginArtWithToc}[0]{\begin{document}\tableofcontents}
\newcommand{\beginArtNoToc}[0]{\begin{document}}
\newcommand{\EndNoBibArticle}[0]{\end{document}}
\newcommand{\EndArticle}[0]{\bibliography{Bibliography}\bibliographystyle{plainnat}\end{document}}

% 
%\newcommand{\citep}[1]{\cite{#1}}

\colorSectionsForArticle


%
%%\citep{harald2003solid} \S x.y
%%\citep{ibach2009solid} \S x.y
%
%%\usepackage{mhchem}
%\usepackage[version=3]{mhchem}
%\usepackage{units}
%\usepackage{bm}
%\newcommand{\nought}[0]{\circ}
%%\newcommand{\EF}[0]{\epsilon_{\mathrm{F}}}
%\newcommand{\EF}[0]{E_{\mathrm{F}}}
%\newcommand{\kF}[0]{k_{\mathrm{F}}}
%
%\beginArtNoToc
%\generatetitle{PHY487H1F Condensed Matter Physics.  Lecture 19: Electrical transport (cont.).  Taught by Prof.\ Stephen Julian}
%\chapter{Electrical transport (cont.)}
\label{chap:condensedMatterLecture19}

%\section{Disclaimer}
%
%Peeter's lecture notes from class.  May not be entirely coherent.

\paragraph{Electrical transport (cont.)}
\index{electrical transport}

Last time we noted that we can't use plain Bloch waves to model this, but must introduce a wave packet centered on some \(\Bk\), such as the Gaussian of \cref{fig:qmSolidsL19:qmSolidsL19Fig1}, moving with group velocity \index{group velocity}

\mathImageFigure{../../figures/phy487/qmSolidsL19Fig1}{Gaussian wave packet}{fig:qmSolidsL19:qmSolidsL19Fig1}{0.15}{guassianPlotsL18L19.nb}

\begin{dmath}\label{eqn:condensedMatterLecture19:20}
\Bv = \inv{\Hbar} \spacegrad_\Bk E(\Bk).
\end{dmath}

For nearly free electrons where \(E(\Bk) = \Hbar^2 \Bk^2/2m\) this gives the intuitively appealing 

\begin{equation}\label{eqn:condensedMatterLecture19:40}
\Bv = \frac{\Hbar \Bk}{m} = \frac{\Bp}{m},
\end{equation}

where we have velocity as momentum over mass.

For tight binding 

\begin{dmath}\label{eqn:condensedMatterLecture19:60}
E(k) = E_i - A - 2 B \cos k a
\end{dmath}

so that 

\begin{dmath}\label{eqn:condensedMatterLecture19:80}
v = \frac{2 B a}{\Hbar} \sin k a.
\end{dmath}

An important quantity is the \textAndIndex{Fermi velocity}

\begin{equation}\label{eqn:condensedMatterLecture19:100}
\Bv_{\mathrm{F}} = \Bv(\Bk = \Bk_{\mathrm{F}}).
\end{equation}

Linearizing \(E(\Bk)\) around \(\Bk_{\mathrm{F}}\)

\begin{dmath}\label{eqn:condensedMatterLecture19:120}
E(\Bk) 
= E(\Bk_{\mathrm{F}}) + \delta k_\perp \lr{ \frac{dE}{d k_\perp} }.
\equiv E(\Bk_{\mathrm{F}}) + \delta k_\perp 
\mathLabelBox
{
\frac{\Hbar \Bk_{\mathrm{F}}}{m^\conj}
}
{
\(\Bp_{\mathrm{F}}\)
}
\end{dmath}

Here \(m^\conj\) is the \textAndIndex{effective mass}.  Keep in mind that the Fermi surface is often not spherical as in \cref{fig:qmSolidsL19:qmSolidsL19Fig2}, and \cref{fig:qmSolidsL19:qmSolidsL19Fig3}.

\imageFigure{../../figures/phy487/qmSolidsL19Fig2}{Non-spherical Fermi surface}{fig:qmSolidsL19:qmSolidsL19Fig2}{0.15}
\imageFigure{../../figures/phy487/qmSolidsL19Fig3}{Second non-spherical Fermi surface}{fig:qmSolidsL19:qmSolidsL19Fig3}{0.15}

\begin{dmath}\label{eqn:condensedMatterLecture19:140}
\Bv_{\mathrm{F}}(\Bk) = \evalbar{\inv{\Hbar} \spacegrad_\Bk E(\Bk)}{\Bk_{\mathrm{F}}} = \frac{\Hbar \Bk_{\mathrm{F}}}{m^\conj}
\end{dmath}

\paragraph{response to applied electric field \(\EE\)}

Our \(\BF = m\Ba\) equivalent for a wave packet is

\begin{equation}\label{eqn:condensedMatterLecture19:160}
\dot{\Bp} = \Hbar \dot{\Bk} = -e \EE
\end{equation}

Wave vector of wave package advances steadily, as in \cref{fig:qmSolidsL19:qmSolidsL19Fig4}, provided we ignore scattering (for now).

\imageFigure{../../figures/phy487/qmSolidsL19Fig4}{Wave packet along distribution curve}{fig:qmSolidsL19:qmSolidsL19Fig4}{0.15}

The time rate of change of the velocity is

\begin{dmath}\label{eqn:condensedMatterLecture19:180}
\vdot_i 
= \inv{\Hbar} \ddt{} \lr{ \spacegrad_\Bk E(\Bk)}_i
= \inv{\Hbar} \sum_j \frac{\partial^2 E}{\partial k_i \partial k_j} \dot{k}_j
= \sum_j \lr{ \inv{m^\conj}}_{i j} \lr{ -e \calE_j }
\end{dmath}

This time we call 

\begin{dmath}\label{eqn:condensedMatterLecture19:200}
\lr{ \inv{m^\conj} }_{i j}
= \inv{\Hbar^2} 
\frac{\partial^2 E}{\partial k_i \partial k_j} 
\end{dmath}

the \textAndIndex{effective mass tensor}, which represents ``resistance'' to applied force.  For example in \cref{fig:qmSolidsL19:qmSolidsL19Fig5}.

\imageFigure{../../figures/phy487/qmSolidsL19Fig5}{Effective mass tensor}{fig:qmSolidsL19:qmSolidsL19Fig5}{0.15}

Some interesting conditions for the effective mass tensor are 

\begin{itemize}
\item For \(m^\conj > 0\), then \(\dot{\Bv}\) is parallel to the force.
\item For \(\Abs{m^\conj} = \infty\), then \(\dot{\Bv} = 0\).
\item For \(m^\conj < 0\), the wave packet slows down under a force parallel to \(\Bv\).
\end{itemize}

%\cref{fig:qmSolidsL19:qmSolidsL19Fig6}.
\imageFigure{../../figures/phy487/qmSolidsL19Fig6}{Illustration for points above}{fig:qmSolidsL19:qmSolidsL19Fig6}{0.15}

\section{Electric current}
\index{electric current}

\reading \citep{ibach2009solid} \S 9.2, \citep{ashcroft1976solid} \chaptext 12.

We've been considering a single electron.  What about a metal?

\begin{dmath}\label{eqn:condensedMatterLecture19:220}
\Bj 
= -\frac{e}{V} \sum_\Bk \Bv(\Bk) 
= - e\frac{2}{(2 \pi)^3} \int d\Bk v(\Bk)
= - e\frac{2}{8 \pi^3} \int d\Bk 
f(E(\Bk), \EE)
\spacegrad_\Bk E(\Bk) 
\end{dmath}

Here \(f(E(\Bk, \EE)\) is the electron distribution in presence of field \(\EE\).

\paragraph{In equilibrium \(\EE = 0\)}

%\cref{fig:qmSolidsL19:qmSolidsL19Fig7}.
%\imageFigure{../../figures/phy487/qmSolidsL19Fig7}{?}{fig:qmSolidsL19:qmSolidsL19Fig7}{0.15}

%\exists
In 1D for every occupied \(+\Bv(\Bk)\) state there exists a \(\Bv(-\Bk)\) state is occupied so that \(\Bv(-\Bk) = -\Bv(\Bk)\), so

\begin{dmath}\label{eqn:condensedMatterLecture19:260}
\sum \Bv(\Bk) = 0.
\end{dmath}

%\EndArticle
