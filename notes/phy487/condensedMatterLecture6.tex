%
% Copyright � 2013 Peeter Joot.  All Rights Reserved.
% Licenced as described in the file LICENSE under the root directory of this GIT repository.
%
%\newcommand{\authorname}{Peeter Joot}
\newcommand{\email}{peeterjoot@protonmail.com}
\newcommand{\basename}{FIXMEbasenameUndefined}
\newcommand{\dirname}{notes/FIXMEdirnameUndefined/}

%\renewcommand{\basename}{condensedMatterLecture6}
%\renewcommand{\dirname}{notes/phy487/}
%\newcommand{\keywords}{Condensed matter physics, PHY487H1F}
%\newcommand{\authorname}{Peeter Joot}
\newcommand{\onlineurl}{http://sites.google.com/site/peeterjoot2/math2013/\basename.pdf}
\newcommand{\sourcepath}{\dirname\basename.tex}
\newcommand{\generatetitle}[1]{\chapter{#1}}

\newcommand{\vcsinfo}{%
\section*{}
\noindent{\color{DarkOliveGreen}{\rule{\linewidth}{0.1mm}}}
\paragraph{Document version}
%\paragraph{\color{Maroon}{Document version}}
{
\small
\begin{itemize}
\item Available online at:\\ 
\href{\onlineurl}{\onlineurl}
\item Git Repository: \input{./.revinfo/gitRepo.tex}
\item Source: \sourcepath
\item last commit: \input{./.revinfo/gitCommitString.tex}
\item commit date: \input{./.revinfo/gitCommitDate.tex}
\end{itemize}
}
}

%\PassOptionsToPackage{dvipsnames,svgnames}{xcolor}
\PassOptionsToPackage{square,numbers}{natbib}
\documentclass{scrreprt}

\usepackage[left=2cm,right=2cm]{geometry}
\usepackage[svgnames]{xcolor}
\usepackage{peeters_layout}

\usepackage{natbib}

\usepackage[
colorlinks=true,
bookmarks=false,
pdfauthor={\authorname, \email},
backref 
]{hyperref}

% http://tex.stackexchange.com/questions/75773/how-to-reference-problems-by-the-text-label-in-an-exercise-envioronment
\usepackage[english]{cleveref}
\crefname{Exercise}{exercise}{exercises}
\Crefname{Exercise}{Exercise}{Exercises}

\RequirePackage{titlesec}
\RequirePackage{ifthen}

% http://stackoverflow.com/questions/4932910/date-in-the-tabular-environment
\makeatletter
\let\insertdate\@date
\makeatother

\titleformat{\chapter}[display]
{\bfseries\Large}
{\color{DarkSlateGrey}\filleft \authorname
\ifthenelse{\isundefined{\studentnumber}}{}{\\ \studentnumber}
\ifthenelse{\isundefined{\email}}{}{\\ \email}
\ifthenelse{\isundefined{\dateintitle}}{}{\\ \insertdate}
%\ifthenelse{\isundefined{\coursename}}{}{\\ \coursename} % put in title instead.
}
{4ex}
{\color{DarkOliveGreen}{\titlerule}\color{Maroon}
\vspace{2ex}%
\filright}
[\vspace{2ex}%
\color{DarkOliveGreen}\titlerule
]

\newcommand{\beginArtWithToc}[0]{\begin{document}\tableofcontents}
\newcommand{\beginArtNoToc}[0]{\begin{document}}
\newcommand{\EndNoBibArticle}[0]{\end{document}}
\newcommand{\EndArticle}[0]{\bibliography{Bibliography}\bibliographystyle{plainnat}\end{document}}

% 
%\newcommand{\citep}[1]{\cite{#1}}

\colorSectionsForArticle


%
%%\citep{harald2003solid} \S x.y
%
%%\usepackage{mhchem}
%\usepackage[version=3]{mhchem}
%
%\beginArtNoToc
%\generatetitle{PHY487H1F Condensed Matter Physics.  Lecture 6: Diffraction.  Taught by Prof.\ Stephen Julian}
%\chapter{Diffraction}
\label{chap:condensedMatterLecture6}
%
%\section{Disclaimer}
%
%Peeter's lecture notes from class.  May not be entirely coherent.
%
\section{Conditions for constructive interference at the detector}

Constructive interference is diffraction peaks (spots).

Recall

\begin{dmath}\label{eqn:condensedMatterLecture6:20}
I(\Bk) 
= 
\Abs{
\int \rho(\Br) e^{-i \BK \cdot \Br}
d\Br
}^2,
\end{dmath}

so that after a periodic decomposition we have

\begin{dmath}\label{eqn:condensedMatterLecture6:40}
I(\Bk) 
= 
\Abs{
\sum_\BG \rho_\BG
\int
e^{i (\BG - \BK) \cdot \Br}
d\Br
}^2.
\end{dmath}

When

\begin{dmath}\label{eqn:condensedMatterLecture6:60}
\BK = \Bk - \Bk_0 = \BG
\end{dmath}

the integrand is unity, whereas if 

\begin{dmath}\label{eqn:condensedMatterLecture6:80}
\BK \ne \BG
\end{dmath}

integrand oscillates, and the integral $\sim 0$.   

%\cref{fig:qmSolidsL6:qmSolidsL6Fig1}.
\imageFigure{qmSolidsL6Fig1}{Cancellation}{fig:qmSolidsL6:qmSolidsL6Fig1}{0.15}

This is effectively a $\delta_{\BK, \BG}$ condition, defining the \underlineAndIndex{Laue condition for constructive interference}

\begin{equation}\label{eqn:condensedMatterLecture6:100}
\myBoxed{
\BK = \BG
}
\end{equation}

%\cref{fig:qmSolidsL6:qmSolidsL6Fig2}.
\imageFigure{qmSolidsL6Fig2}{Diffraction}{fig:qmSolidsL6:qmSolidsL6Fig2}{0.2}

\section{Ewald sphere}

Use that $\Abs{\Bk} = \Abs{\Bk_0}$ (elastic).

Draw reciprocal lattice (r.l.)

\begin{enumerate}
\item 
draw $\Bk_0$ with the head at a r.l. point
\item draw sphere of radius $\Bk_0$ around the \underline{tail}
\item if there sphere intersects other r.l. points we have diffraction.
\end{enumerate}

%F3 (Fig 3.4 in text).
%\cref{fig:qmSolidsL6:qmSolidsL6Fig3}.
\imageFigure{qmSolidsL6Fig3}{Ewald sphere}{fig:qmSolidsL6:qmSolidsL6Fig3}{0.2}

\section{Scattering in terms of lattice points}

A lattice has an infinite number of sets of parallel planes that contain all lattice points.

A given set of planes is labeled by \underlineAndIndex{Miller indices}.

\begin{enumerate}
\item start at the origin
\item With $m, n, o$ all integers,

\begin{itemize}
\item 
go $m \Ba_1$ to next plane.
\item 
go $n \Ba_2$ to next plane.
\item 
go $o \Ba_3$ to next plane.
\end{itemize}

\item Find $\lr{ \frac{p}{m}, \frac{p}{n}, \frac{p}{o}}$.  These triplets are called the Miller indices.
\end{enumerate}

\makeexample{simple cubic}{example:condensedMatterLecture6:1}{

%\cref{fig:qmSolidsL6:qmSolidsL6Fig4}.
\imageFigure{qmSolidsL6Fig4}{Cubic miller planes}{fig:qmSolidsL6:qmSolidsL6Fig4}{0.2}

\begin{dmath}\label{eqn:condensedMatterLecture6:120}
(1 \times \Ba_1, \infty \times \Ba_2, \infty \times \Ba_3)
\end{dmath}

\begin{dmath}\label{eqn:condensedMatterLecture6:140}
\lr{ \frac{p}{1}, \frac{p}{\infty}, \frac{p}{\infty} }
\end{dmath}

with $p = 1$ this gives Miller indices of
\begin{dmath}\label{eqn:condensedMatterLecture6:160}
\lr{ 1, 0, 0}
\end{dmath}
}

\makeexample{fcc}{example:condensedMatterLecture6:2}{

%\cref{fig:qmSolidsL6:qmSolidsL6Fig5}.
\imageFigure{qmSolidsL6Fig5}{fcc Miller planes}{fig:qmSolidsL6:qmSolidsL6Fig5}{0.2}

}

We use 

\begin{dmath}\label{eqn:condensedMatterLecture6:180}
\BG_{hkl} = 
h \Bg_1 +
k \Bg_2 +
l \Bg_3,
\end{dmath}

with this perpendicular to the $(h, k, l)$ planes.

\makeexample{Miller index demonstration}{example:condensedMatterLecture6:3}{

FIXME: figure stolen from Prof's notes.  Was hard to draw in class.

Here we illustrate two pairs of arbitrary planes each passing through two lattice points.
%\cref{fig:qmSolidsL6:qmSolidsL6Fig6}.
\imageFigure{qmSolidsL6Fig6}{Miller index demonstration}{fig:qmSolidsL6:qmSolidsL6Fig6}{0.2}

\begin{dmath}\label{eqn:condensedMatterLecture6:200}
\lr{ 1 \times \Ba_1, \inv{2} \times \Ba_2, - \Ba_3 }
\implies
\lr{ 
\frac{p}{1},
\frac{p}{\inv{2}},
-\frac{p}{1} 
}
\implies
\lr{ 
1, 2, -1
}
\end{dmath}

To find the perpendicular, define 2 non-parallel vectors in a plane: $\Ba_1 - \Ba_2/2$, $-\Ba_3 - \Ba_2/2$.

Perpendicular vector is

\begin{dmath}\label{eqn:condensedMatterLecture6:220}
%\perp
\propto
(\Ba_1 - \Ba_2/2) \cross (-\Ba_3 - \Ba_2/2)
= 
\inv{2} \Ba_2 \cross \Ba_3
+\Ba_3 \cross \Ba_1
- \inv{2} \Ba_1 \cross \Ba_2
\propto
\Bg_1 + 2 \Bg_2 - \Bg_3 = \BG_{1, 2, -1}
\end{dmath}

}

\citep{ibach2009solid} pg 61 has % \S x.y
``general'' proof.  

If $d_{hkl}$ is the distance between planes, then

\begin{dmath}\label{eqn:condensedMatterLecture6:240}
d_{hkl} = \frac{2 \pi}{ G_{hkl} }
\end{dmath}

\section{Another geometrical construction}

%\cref{fig:qmSolidsL6:qmSolidsL6Fig7}.
\imageFigure{qmSolidsL6Fig7}{Diffraction geometry}{fig:qmSolidsL6:qmSolidsL6Fig7}{0.2}

$hkl$ plane perpendicular to the page.

See:

\begin{dmath}\label{eqn:condensedMatterLecture6:260}
\frac{G_{hkl}}{2} = k_o \sin\theta \implies \frac{2 \pi}{d_{hkl}} = \frac{2 \pi}{\lambda} \sin\theta,
\end{dmath}

or

\begin{equation}\label{eqn:condensedMatterLecture6:280}
\myBoxed{
\lambda = 2 d_{hkl} \sin\theta.
}
\end{equation}

This is the \underlineAndIndex{Bragg condition}, which we showed is equivalent to the Laue condition, and illustrated in \cref{fig:qmSolidsL6:qmSolidsL6Fig8}.

\imageFigure{qmSolidsL6Fig8}{Bragg condition}{fig:qmSolidsL6:qmSolidsL6Fig8}{0.3}

Path difference is $2 d \sin\theta = n \lambda$ for constructive interference.

%\EndArticle
