%
% Copyright � 2013 Peeter Joot.  All Rights Reserved.
% Licenced as described in the file LICENSE under the root directory of this GIT repository.
%
%\newcommand{\authorname}{Peeter Joot}
\newcommand{\email}{peeterjoot@protonmail.com}
\newcommand{\basename}{FIXMEbasenameUndefined}
\newcommand{\dirname}{notes/FIXMEdirnameUndefined/}

%\renewcommand{\basename}{condensedMatterLecture18}
%\renewcommand{\dirname}{notes/phy487/}
%\newcommand{\keywords}{Condensed matter physics, PHY487H1F}
%\newcommand{\authorname}{Peeter Joot}
\newcommand{\onlineurl}{http://sites.google.com/site/peeterjoot2/math2013/\basename.pdf}
\newcommand{\sourcepath}{\dirname\basename.tex}
\newcommand{\generatetitle}[1]{\chapter{#1}}

\newcommand{\vcsinfo}{%
\section*{}
\noindent{\color{DarkOliveGreen}{\rule{\linewidth}{0.1mm}}}
\paragraph{Document version}
%\paragraph{\color{Maroon}{Document version}}
{
\small
\begin{itemize}
\item Available online at:\\ 
\href{\onlineurl}{\onlineurl}
\item Git Repository: \input{./.revinfo/gitRepo.tex}
\item Source: \sourcepath
\item last commit: \input{./.revinfo/gitCommitString.tex}
\item commit date: \input{./.revinfo/gitCommitDate.tex}
\end{itemize}
}
}

%\PassOptionsToPackage{dvipsnames,svgnames}{xcolor}
\PassOptionsToPackage{square,numbers}{natbib}
\documentclass{scrreprt}

\usepackage[left=2cm,right=2cm]{geometry}
\usepackage[svgnames]{xcolor}
\usepackage{peeters_layout}

\usepackage{natbib}

\usepackage[
colorlinks=true,
bookmarks=false,
pdfauthor={\authorname, \email},
backref 
]{hyperref}

% http://tex.stackexchange.com/questions/75773/how-to-reference-problems-by-the-text-label-in-an-exercise-envioronment
\usepackage[english]{cleveref}
\crefname{Exercise}{exercise}{exercises}
\Crefname{Exercise}{Exercise}{Exercises}

\RequirePackage{titlesec}
\RequirePackage{ifthen}

% http://stackoverflow.com/questions/4932910/date-in-the-tabular-environment
\makeatletter
\let\insertdate\@date
\makeatother

\titleformat{\chapter}[display]
{\bfseries\Large}
{\color{DarkSlateGrey}\filleft \authorname
\ifthenelse{\isundefined{\studentnumber}}{}{\\ \studentnumber}
\ifthenelse{\isundefined{\email}}{}{\\ \email}
\ifthenelse{\isundefined{\dateintitle}}{}{\\ \insertdate}
%\ifthenelse{\isundefined{\coursename}}{}{\\ \coursename} % put in title instead.
}
{4ex}
{\color{DarkOliveGreen}{\titlerule}\color{Maroon}
\vspace{2ex}%
\filright}
[\vspace{2ex}%
\color{DarkOliveGreen}\titlerule
]

\newcommand{\beginArtWithToc}[0]{\begin{document}\tableofcontents}
\newcommand{\beginArtNoToc}[0]{\begin{document}}
\newcommand{\EndNoBibArticle}[0]{\end{document}}
\newcommand{\EndArticle}[0]{\bibliography{Bibliography}\bibliographystyle{plainnat}\end{document}}

% 
%\newcommand{\citep}[1]{\cite{#1}}

\colorSectionsForArticle


%
%%\citep{harald2003solid} \S x.y
%
%%\usepackage{mhchem}
%\usepackage{units}
%\usepackage[version=3]{mhchem}
%\newcommand{\nought}[0]{\circ}
%%\newcommand{\EF}[0]{\epsilon_{\txtF}}
%\newcommand{\EF}[0]{E_{\txtF}}
%\newcommand{\kF}[0]{k_{\txtF}}
%
%\beginArtNoToc
%\generatetitle{PHY487H1F Condensed Matter Physics.  Lecture 18: Semiconductors.  Taught by Prof.\ Stephen Julian}
%\chapter{Semiconductors}
\label{chap:condensedMatterLecture18}

%\section{Disclaimer}
%
%Peeter's lecture notes from class.  May not be entirely coherent.

\section{Semiconductors}
\index{semiconductors}

We continue with examination of the band structures of real materials.  In particular, start looking at semiconductors

\begin{itemize}
\item diamond
\item \ce{Si}
\item \ce{Ge}
\item \ce{Ga As}
\item \ce{In P}
\item \ce{In As}
\end{itemize}

\makeexample{Diamond bandstructure}{example:condensedMatterLecture18:1}{

%\cref{fig:qmSolidsL18:qmSolidsL18Fig1}.
\imageFigure{../../figures/phy487/qmSolidsL18Fig1}{Diamond bandstructure}{fig:qmSolidsL18:qmSolidsL18Fig1}{0.2}

The two branches are because we have two atoms per unit cell.  Those are due to antibonding and bonding conditions as in \cref{fig:qmSolidsL18:qmSolidsL18Fig2}.

\imageFigure{../../figures/phy487/qmSolidsL18Fig2}{Bonding and antibonding functions in unit cells}{fig:qmSolidsL18:qmSolidsL18Fig2}{0.1}
}

\makeexample{Band structure of \ce{Ge} and \ce{Si}}{example:condensedMatterLecture18:2}{

Also see slides and \citep{ibach2009solid} \S 7.13.

%\cref{fig:qmSolidsL18:qmSolidsL18Fig3}.
\imageFigure{../../figures/phy487/qmSolidsL18Fig3}{\ce{Ge}, \ce{Si} band structure}{fig:qmSolidsL18:qmSolidsL18Fig3}{0.2}

\(\EF\) falls in a full gap, where there is no Fermi surface.  The gap is small.  Carriers can be thermally excited at room temperature.

The lower energy curves correspond to bonding orbitals.  In general in materials the bonding orbitals will always be occupied, because they are energy favorable.
}

\makeexample{Insulators: large gap semiconductors}{example:condensedMatterLecture18:n}{

\citep{ibach2009solid} fig 7.10.

For \ce{K Cl} as in \cref{fig:qmSolidsL18:qmSolidsL18Fig4}

\imageFigure{../../figures/phy487/qmSolidsL18Fig4}{\ce{K Cl} band structure}{fig:qmSolidsL18:qmSolidsL18Fig4}{0.2}

all bands near \(\EF\) are tight binding like.  There's a large gap at \(\EF\).
}

\section{Density of states}
\index{density of states}

\reading \citep{ashcroft1976solid} \chaptext 8.

Recall 

\begin{dmath}\label{eqn:condensedMatterLecture18:20}
\sum_{k, \sigma = \pm 1/2} \rightarrow 
\mathLabelBox{2}{spin}
\inv{ \lr{ 2\pi/L}^3 }
\int d\Bk
\rightarrow 
V \int D(E) dE.
\end{dmath}

For the free electron model we found

\begin{dmath}\label{eqn:condensedMatterLecture18:40}
D(E) = \text{density of states per unit volume} 
= 
\inv{2 \pi^2} \lr{ \frac{2m}{\Hbar} }^{3/2} \sqrt{E},
\end{dmath}

and in general

\begin{dmath}\label{eqn:condensedMatterLecture18:60}
D(E) dE = \frac{2}{\lr{2\pi}^3} \int \frac{df_{\txtE}}{ \Abs{\spacegrad_\Bk E(\Bk)}} dE.
\end{dmath}

Referring to \cref{fig:qmSolidsL18:qmSolidsL18Fig5}.

\imageFigure{../../figures/phy487/qmSolidsL18Fig5}{Fermi surface for constant energy}{fig:qmSolidsL18:qmSolidsL18Fig5}{0.2}

\begin{dmath}\label{eqn:condensedMatterLecture18:80}
d\Bk = df_{\txtE} dk_\perp 
= df_{\txtE} \frac{dE}{\Abs{\spacegrad_\Bk E(\Bk)}}.
\end{dmath}

Some observations:

\begin{itemize}
\item 
Flat bands due to \(1/\Abs{\spacegrad_\Bk E(\Bk)}\) will have a high density of states.
\item At local extremums of \(E(\Bk)\) you get a peak in \(D(E)\).  This is called a \textAndIndex{van Hove singularity}.  See slides and \citep{ashcroft1976solid} \figtext 8.3.
\end{itemize}

\paragraph{d-metals}

High density of states in the d-orbitals, and small elsewhere.  Recall

\begin{dmath}\label{eqn:condensedMatterLecture18:100}
C(T) = 
\mathLabelBox
{
\gamma T 
}
{electrons}
+ 
\mathLabelBox
[
   labelstyle={below of=m\themathLableNode, below of=m\themathLableNode}
]
{
\beta T^3
}
{phonons}.
\end{dmath}

\begin{dmath}\label{eqn:condensedMatterLecture18:120}
\begin{aligned}
\gamma_{\ce{Cu}} &\sim 0.69 \Unitfrac{mJ}{mole\, K^2} \\
\gamma_{\ce{Fe}} &\sim 4.98 \Unitfrac{mJ}{mole\, K^2}
\end{aligned}
\end{dmath}

\paragraph{Germanium}
\index{Germanium}

Here we have Van Hove singularities.  Note that max/min at \(\Gamma\) do not produce a \textAndIndex{Van Hove singlularity}, because \(d f_{\txtE}\) is vanishingly small.

\section{Electrical transport}
\index{electrical transport}

\reading \citep{ibach2009solid} \S 9.1,\citep{ashcroft1976solid} \chaptext 12.

We want to talk about how to make an electrical current flow.  Imagine that we are considering a block of material with leads on it, as in \cref{fig:qmSolidsL18:qmSolidsL18Fig6}.

\imageFigure{../../figures/phy487/qmSolidsL18Fig6}{Conducting block of metal}{fig:qmSolidsL18:qmSolidsL18Fig6}{0.1}

How do we use our Bloch description for such a volume?  A pure Bloch wave

\begin{dmath}\label{eqn:condensedMatterLecture18:140}
\Psi_\Bk = U_\Bk(\Br) e^{i \Bk \cdot \Br }, 
\end{dmath}

is uniformly spread out over the sample

\begin{dmath}\label{eqn:condensedMatterLecture18:160}
\Abs{ \inv{\sqrt{V}} e^{i \Bk \cdot \Br}}^2 = \inv{V}.
\end{dmath}

Such a plane wave cannot transport charge.  Introduce a wave packet

\begin{dmath}\label{eqn:condensedMatterLecture18:180}
\Psi_\Bk(\Br, t) \sim \int_{\Abs{\Bk'} < \Delta} 
\mathLabelBox{a_{\Bk'}}{example: a Gaussian, centered on \(\Bk\)}
U_\Bk(\Br) e^{ i \lr{\Bk \cdot \Br - \omega t}} d\Bk'
\end{dmath}

This will spread as plotted in \cref{fig:qmSolidsL18:qmSolidsL18Fig7}.

%\imageFigure{../../figures/phy487/qmSolidsL18Fig7}{Packet spreading with time}{fig:qmSolidsL18:qmSolidsL18Fig7}{0.1}
\mathImageFigure{../../figures/phy487/qmSolidsL18PacketSpreadingFig1}{Packet spreading with time}{fig:qmSolidsL18:qmSolidsL18Fig7}{0.3}{guassianPlotsL18L19.nb}

Moves with \textAndIndex{group velocity}

\boxedEquation{eqn:condensedMatterLecture18:n}{
\Bv_\Bk = \frac{d\omega(\Bk)}{d\Bk} = \inv{\Hbar} \spacegrad_\Bk E(\Bk).
}

%\EndArticle
