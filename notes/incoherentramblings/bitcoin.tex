%
% Copyright � 2013 Peeter Joot.   All Rights Reserved.
% Licenced as described in the file LICENSE under the root directory of this GIT repository.
%
\newcommand{\authorname}{Peeter Joot}
\newcommand{\email}{peeterjoot@protonmail.com}
\newcommand{\basename}{FIXMEbasenameUndefined}
\newcommand{\dirname}{notes/FIXMEdirnameUndefined/}

\renewcommand{\basename}{bitcoin}
\renewcommand{\dirname}{notes/incoherentramblings/}
%\newcommand{\dateintitle}{}
%\newcommand{\keywords}{}

\newcommand{\authorname}{Peeter Joot}
\newcommand{\onlineurl}{http://sites.google.com/site/peeterjoot2/math2013/\basename.pdf}
\newcommand{\sourcepath}{\dirname\basename.tex}
\newcommand{\generatetitle}[1]{\chapter{#1}}

\newcommand{\vcsinfo}{%
\section*{}
\noindent{\color{DarkOliveGreen}{\rule{\linewidth}{0.1mm}}}
\paragraph{Document version}
%\paragraph{\color{Maroon}{Document version}}
{
\small
\begin{itemize}
\item Available online at:\\ 
\href{\onlineurl}{\onlineurl}
\item Git Repository: \input{./.revinfo/gitRepo.tex}
\item Source: \sourcepath
\item last commit: \input{./.revinfo/gitCommitString.tex}
\item commit date: \input{./.revinfo/gitCommitDate.tex}
\end{itemize}
}
}

%\PassOptionsToPackage{dvipsnames,svgnames}{xcolor}
\PassOptionsToPackage{square,numbers}{natbib}
\documentclass{scrreprt}

\usepackage[left=2cm,right=2cm]{geometry}
\usepackage[svgnames]{xcolor}
\usepackage{peeters_layout}

\usepackage{natbib}

\usepackage[
colorlinks=true,
bookmarks=false,
pdfauthor={\authorname, \email},
backref 
]{hyperref}

% http://tex.stackexchange.com/questions/75773/how-to-reference-problems-by-the-text-label-in-an-exercise-envioronment
\usepackage[english]{cleveref}
\crefname{Exercise}{exercise}{exercises}
\Crefname{Exercise}{Exercise}{Exercises}

\RequirePackage{titlesec}
\RequirePackage{ifthen}

% http://stackoverflow.com/questions/4932910/date-in-the-tabular-environment
\makeatletter
\let\insertdate\@date
\makeatother

\titleformat{\chapter}[display]
{\bfseries\Large}
{\color{DarkSlateGrey}\filleft \authorname
\ifthenelse{\isundefined{\studentnumber}}{}{\\ \studentnumber}
\ifthenelse{\isundefined{\email}}{}{\\ \email}
\ifthenelse{\isundefined{\dateintitle}}{}{\\ \insertdate}
%\ifthenelse{\isundefined{\coursename}}{}{\\ \coursename} % put in title instead.
}
{4ex}
{\color{DarkOliveGreen}{\titlerule}\color{Maroon}
\vspace{2ex}%
\filright}
[\vspace{2ex}%
\color{DarkOliveGreen}\titlerule
]

\newcommand{\beginArtWithToc}[0]{\begin{document}\tableofcontents}
\newcommand{\beginArtNoToc}[0]{\begin{document}}
\newcommand{\EndNoBibArticle}[0]{\end{document}}
\newcommand{\EndArticle}[0]{\bibliography{Bibliography}\bibliographystyle{plainnat}\end{document}}

% 
%\newcommand{\citep}[1]{\cite{#1}}

\colorSectionsForArticle



\beginArtNoToc

\generatetitle{Some thoughts about bitcoin, money and anarchy}
%\chapter{Some thoughts about bitcoin}
%\label{chap:bitcoin}
\section{Recommended reading: Ellen Brown's ``Web of Debt''}

You may think that you do not care to understand how debt based banking and associated currencies work.  Many of us probably think of money as nothing more than a means to an end.  The mechanism that allows governments and banks to create money does not seem important in day to day life.  Provided we get our paycheck, in day to day life, we only care that we can use it to pay our bills, buy our groceries, provide clothes and shelter for our kids, entertain ourselves, and so forth.

To help explain both the mechanisms of debt based currencies, and some of the implications and history of the current system, I would recommend the book ``Web of Debt'' \citep{brown2008web}.  This book explains many subtle details of modern debt based monetary systems, and has a few elements

\begin{itemize}
\item The mechanism that allows money to be created from debt.
\item A historical (and very unconventional) narrative of warfare from the viewpoint of banking manipulations.
\item A history of the United States ``Fed'', its history, its backers and the sorts of manipulations that it allows.
\item A discussion of alternative currencies and possible solutions.
\end{itemize}


You will also find lots of details about how the world financial system got to be in such a mess.  I found that learning the extent of this mess was almost physically painful, a response that makes it easy to understand why so many people put up with the status-quo.  It is easier and more comfortable to just not know.

In the unconventional historical narrative in this text most warfare is framed as motivated by banking interests.  That perspective is so different from conventional history that many people will probably it flat out unbelievable.  The author did not help that believability by framing it as self evident.  A more scholarly presentation as a thesis with supporting arguments and documentation would have helped lend that history more credence.

A number of possible alternative monetary schemes are considered.  The most weight is put on nationalization of banks, with debt free creation of money used instead of government borrowing the money that they print from private banks.  It seems implausible that enough people will learn the mechanics of our monetary system to force a grass roots movement that could wrest control of money creation from the private banking industry.  If such a paradigm shift is unlikely, reform would probably require some of the smaller scale local monetary ideas catch on, or the invention of alternate currencies (this is where I think bitcoin would fit nicely).

There are a number of such alternative currencies discussed in this book.  These include local (municipal) currencies, currencies based on barter and time expended to provide services.  I do not thing that many such specialized currencies will ever find wide spread acceptance.  Something that is widely tradable in a variety of circumstances is required to break the widespread dependence on the current central banking system.  This book does mention some electronic currencies, but I believe that it predates bitcoin slightly.  Perhaps bitcoin and/or alt-coins will make it into a future edition.

Are these topics that would motivate the average Joe into learning how money is created and manipulated?  Not likely.

Why do I think that this is worthwhile?  I think that this is worth knowing because so much of your money is being spent without consent by the governments that are nominally but not functionally our representatives.  It is very informative to gain some historical perspective of the abuses that are enabled by financial manipulation.  When fifty percent of your hard earned money is stolen from you, and you face fines, imprisonment, and confiscation of property if you object to that stealing or how your resources are consumed, it is clearly worthwhile to understand some of the mechanics involved.

However, most people are content with the status-quo.  Most people accept the delusion that we live in a functioning democracy.  Most people go through their day to day lives believing in the stability of the financial system.  A switch to a monetary system that deviates from our current central bank scheme has so many implications, they are hard to enumerate.  Without some sort of catastrophic motivation the resistance to such a change would be extreme.  If the history of the author is accepted, such a change is likely to starts wars.

\section{What is money?}

Money ought to be nothing more than a mechanism to facilitate exchange.  Goods or services that have value to both the buyer and seller should be exchangeable either directly or using some form of currency.  Currency in such a context is just something that has an agreed upon value that can be used to facilitate multiple party transaction where not all of the goods or services offered are of interest to all parties at the time of the transaction.

Does the currency have to have any intrinsic value?  Except for coins, it has been a long time since the monies that we use had any sort of intrinsic value.  Through our laws we have the strange situation that we authorize ``special'' institutions the power to create money on our behalf.  That is either done directly by central banks or indirectly by regular banks when debt is issued, or still more indirectly by companies that issue tradable stocks.  Banks essentially have the power to issue money because they give themselves this power by controlling the financing of the puppet politicians that write the laws that give them these powers.  It is a truly bizarre system that we have grown up accustomed to, and due to that acclimatization, do not question in our day to day lives.

If money has no intrinsic value, what is it?  Oxford dictionaries \citep{oxford:dictionary:money} includes the following definitions

\begin{enumerate}
\item A current medium of exchange in the form of coins and banknotes; coins and banknotes collectively.
\item The assets, property, and resources owned by someone or something; wealth
\item Financial gain
\item Payment for work; wages
\end{enumerate}

The first definition is clearly dated.  Most of the money that changes hands these days has no corresponding coin or banknotes associated with it.  Paychecks are directly deposited into our bank account.  We can pay our bills online, or write cheques that represent banknotes, but in the vast majority of cases never see any physical monies that represent either what we receive or distribute.  For all intents and purposes our monies are electronic already, but the electronic transactions are in the control of special institutions that have been granted exclusive privilege to profit by handling these transactions.

The second definition has broader applicability, but has no connection to the concept that money is a mechanism of exchange.

The third definition describes stock market and debt based creation of money nicely, but is also not a very broad definition.

I happen to like Ron Hubbard's definition of money \citep{LRH:barterSystem}, ``Money is an idea backed by confidence''.

There is a level of abstraction in this definition that describes the utility of money as a mechanism for barter.  Confidence that money will be accepted by all parties regardless of what exchange it happens to be used for at any point in time is what makes it valuable.  The debt based ``Fiat currencies'' (a popular term used to describe our central bank created monies) have value because we all agree that they do, and have confidence that they can be used in future exchange.  If that confidence is lost, the money becomes worthless.

Should we wish to use currencies that are not under any sort of third party control (i.e. financial institutions and governments), the main hurdle is the requirement that people using any alternate currency will be confident that such an alternative currency has value.

This is a bit of a catch 22 for the adoption of any new currency, including bitcoin.  To give it value, people have to start using it.

\section{My anarchist rant.  Why I like the idea of bitcoin?}

It seems to me that bitcoin or something like it could easily provide a currency that is both widely tradable, and is also not controlled by coercive, warmongering or generally abusive governments and their associated banking systems.

We live in a world where the products of your time and work are taken from you by force to be spent by authorities who know better than you how your money should be spent.  We do not have the option of vetoing spending on things that we find morally objectionable or damaging.  Ironically the only reason that the guys in blue with guns can haul you away for not paying your taxes is because you pay them implicitly with your taxes to do this job, like it or not.  These are the same guys that routinely break the same rules they enforce.

When so many wars are funded by debt, imagine a world where we have the possibility of anonymous person to person transfers of money are possible without the intervention or even monitoring of third parties, and without those third parties taking a cut without providing any valuable service.  Imagine a world where coercion and threats of governments are not possible, because we stop paying them to threaten us.  Imagine a world where consent is required.

That is why I like the idea of bitcoin or a bitcoin like electronic currency.  It provides a decentralized infrastructure that allows for direct and anonymous person to person transfers.  It could facilitate a paradigm shift to a more barter oriented and localized culture that bypasses government power.  It has the potential for eliminating warfare by eliminating the coercive collection of the monies that are used to finance it.  It has the potential to eliminate the banking and financial scum that sucks the life out of the people that actually work for a living, and profits from keeping them underfoot.

I am probably a naive dreamer, but at least it is a good dream.

\EndArticle
