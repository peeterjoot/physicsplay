%
% Copyright � 2012 Peeter Joot.  All Rights Reserved.
% Licenced as described in the file LICENSE under the root directory of this GIT repository.
%
%\newcommand{\authorname}{Peeter Joot}
\newcommand{\email}{peeterjoot@protonmail.com}
\newcommand{\basename}{FIXMEbasenameUndefined}
\newcommand{\dirname}{notes/FIXMEdirnameUndefined/}

%\renewcommand{\basename}{modernOpticsCosineTransforms}
%\renewcommand{\dirname}{notes/phy485/}
%\newcommand{\keywords}{Optics, PHY485H1F}
%\newcommand{\authorname}{Peeter Joot}
\newcommand{\onlineurl}{http://sites.google.com/site/peeterjoot2/math2013/\basename.pdf}
\newcommand{\sourcepath}{\dirname\basename.tex}
\newcommand{\generatetitle}[1]{\chapter{#1}}

\newcommand{\vcsinfo}{%
\section*{}
\noindent{\color{DarkOliveGreen}{\rule{\linewidth}{0.1mm}}}
\paragraph{Document version}
%\paragraph{\color{Maroon}{Document version}}
{
\small
\begin{itemize}
\item Available online at:\\ 
\href{\onlineurl}{\onlineurl}
\item Git Repository: \input{./.revinfo/gitRepo.tex}
\item Source: \sourcepath
\item last commit: \input{./.revinfo/gitCommitString.tex}
\item commit date: \input{./.revinfo/gitCommitDate.tex}
\end{itemize}
}
}

%\PassOptionsToPackage{dvipsnames,svgnames}{xcolor}
\PassOptionsToPackage{square,numbers}{natbib}
\documentclass{scrreprt}

\usepackage[left=2cm,right=2cm]{geometry}
\usepackage[svgnames]{xcolor}
\usepackage{peeters_layout}

\usepackage{natbib}

\usepackage[
colorlinks=true,
bookmarks=false,
pdfauthor={\authorname, \email},
backref 
]{hyperref}

% http://tex.stackexchange.com/questions/75773/how-to-reference-problems-by-the-text-label-in-an-exercise-envioronment
\usepackage[english]{cleveref}
\crefname{Exercise}{exercise}{exercises}
\Crefname{Exercise}{Exercise}{Exercises}

\RequirePackage{titlesec}
\RequirePackage{ifthen}

% http://stackoverflow.com/questions/4932910/date-in-the-tabular-environment
\makeatletter
\let\insertdate\@date
\makeatother

\titleformat{\chapter}[display]
{\bfseries\Large}
{\color{DarkSlateGrey}\filleft \authorname
\ifthenelse{\isundefined{\studentnumber}}{}{\\ \studentnumber}
\ifthenelse{\isundefined{\email}}{}{\\ \email}
\ifthenelse{\isundefined{\dateintitle}}{}{\\ \insertdate}
%\ifthenelse{\isundefined{\coursename}}{}{\\ \coursename} % put in title instead.
}
{4ex}
{\color{DarkOliveGreen}{\titlerule}\color{Maroon}
\vspace{2ex}%
\filright}
[\vspace{2ex}%
\color{DarkOliveGreen}\titlerule
]

\newcommand{\beginArtWithToc}[0]{\begin{document}\tableofcontents}
\newcommand{\beginArtNoToc}[0]{\begin{document}}
\newcommand{\EndNoBibArticle}[0]{\end{document}}
\newcommand{\EndArticle}[0]{\bibliography{Bibliography}\bibliographystyle{plainnat}\end{document}}

% 
%\newcommand{\citep}[1]{\cite{#1}}

\colorSectionsForArticle


%
%\usepackage[draft]{fixme}
%%\usepackage{accents}
%\fxusetheme{color}
%
%\beginArtNoToc
%\generatetitle{Cosine Transforms}
%\label{chap:modernOpticsCosineTransforms}
\index{cosine transform}
\section{Motivation}

Cosine transforms were mentioned in the class notes.  Let's work through a few basic operations ourselves to get a feel for things.

\makeproblem{Fourier transform of an even function}{pr:cosineTransforms:1}{

Given an even function, constructed from any arbitrary function \(f(\tau)\)

\begin{dmath}\label{eqn:cosineTransforms:10}
f(\tau) = \inv{2} \left( g(\tau) + g(-\tau) \right)
\end{dmath}

determine if the Fourier transform is even or odd.

} % makeproblem

\makeanswer{pr:cosineTransforms:1}{

\begin{dmath}\label{eqn:cosineTransforms:30}
\int_{-\infty}^{-\infty} e^{-i\omega \tau} f(\tau) d\tau
=
\inv{2}
\int_{-\infty}^{\infty}
 e^{-i\omega \tau}
\left( g(\tau) + g(-\tau) \right)
d\tau
=
\inv{2}
\tilde{G}(\omega)
-
\inv{2}
\int_{\infty}^{-\infty} e^{i\omega \tau}
g(\tau)
d\tau
=
\inv{2}
\left(
\tilde{G}(\omega)
+\tilde{G}(-\omega)
\right)
\end{dmath}

Yes, the Fourier transform of an even function in time is even in frequency.
} % makeanswer

\makeproblem{Express the Fourier transform of an even function in terms of cosines}{pr:cosineTransforms:2}{ } % makeproblem
\makeanswer{pr:cosineTransforms:2}{

\begin{dmath}\label{eqn:cosineTransforms:50}
\int_{-\infty}^\infty e^{-i\omega \tau} f(\tau) d\tau
=
\int_{-\infty}^0 e^{-i\omega \tau} f(\tau) d\tau
+\int_{0}^\infty e^{-i\omega \tau} f(\tau) d\tau
=
\int_{-\infty}^0 e^{-i\omega \tau} f(-\tau) d\tau
+\int_{0}^\infty e^{-i\omega \tau} f(\tau) d\tau
=
-\int_{\infty}^0 e^{i\omega \tau} f(\tau) d\tau
+\int_{0}^\infty e^{-i\omega \tau} f(\tau) d\tau
=
\int_0^{\infty} e^{i\omega \tau} f(\tau) d\tau
+\int_{0}^\infty e^{-i\omega \tau} f(\tau) d\tau
=
\int_0^{\infty} \left(
e^{i\omega \tau}
+e^{-i\omega \tau}
\right)
f(\tau) d\tau
=
2
\int_0^{\infty}
\cos( \omega \tau )
f(\tau) d\tau
\end{dmath}

Let's write

\boxedEquation{eqn:cosineTransforms:70}{
\tilde{f}_c(\omega) =
\int_0^{\infty}
\cos( \omega \tau )
f(\tau) d\tau,
}

with

\begin{dmath}\label{eqn:cosineTransforms:90}
\tilde{f}(\omega) =
\int_{-\infty}^{\infty}
e^{-i \omega \tau }
f(\tau) d\tau
\end{dmath}

for the normal Fourier transform, so that

\boxedEquation{eqn:cosineTransforms:110}{
\tilde{f}(\omega) = 2 \tilde{f}_c(\omega)
}

} % makeanswer

\makeproblem{Inverse transform (of an even function)}{pr:cosineTransforms:3}{ } % makeproblem
\makeanswer{pr:cosineTransforms:3}{

\begin{dmath}\label{eqn:cosineTransforms:130}
f(\tau)
=
\inv{2\pi} \int_{-\infty}^\infty d\omega e^{i \omega \tau} \tilde{f}(\omega)
=
\inv{2\pi} \int_{0}^\infty d\omega e^{i \omega \tau} \tilde{f}(\omega)
+
\inv{2\pi} \int_{-\infty}^0 d\omega e^{i \omega \tau} \tilde{f}(\omega)
=
\inv{2\pi} \int_{0}^\infty d\omega e^{i \omega \tau} \tilde{f}(\omega)
+
\inv{2\pi} \int_{-\infty}^0 d\omega e^{i \omega \tau} \tilde{f}(-\omega)
=
\inv{2\pi} \int_{0}^\infty d\omega e^{i \omega \tau} \tilde{f}(\omega)
-
\inv{2\pi} \int_{\infty}^0 d\omega e^{-i \omega \tau} \tilde{f}(\omega)
=
\inv{2\pi} \int_{0}^\infty d\omega e^{i \omega \tau} \tilde{f}(\omega)
+
\inv{2\pi} \int_0^{\infty} d\omega e^{-i \omega \tau} \tilde{f}(\omega)
=
\inv{\pi} \int_{0}^\infty d\omega \cos(\omega \tau) \tilde{f}(\omega)
\end{dmath}

This gives us the inverse transform relationship

\boxedEquation{eqn:cosineTransforms:150}{
f(\tau)
=
\frac{2}{\pi} \int_{0}^\infty d\omega \cos(\omega \tau) \tilde{f}_c(\omega)
}

} % makeanswer

\makeproblem{Is the convolution of even functions even?}{pr:cosineTransforms:4}{ } % makeproblem
\makeanswer{pr:cosineTransforms:4}{

with

\begin{dmath}\label{eqn:cosineTransforms:170}
(f \conj g)(\tau)
= \int_{-\infty}^\infty f(\tau') g(\tau' - \tau),
\end{dmath}

is this an even function?  Let's compute at a negative time

\begin{dmath}\label{eqn:cosineTransforms:190}
(f \conj g)(-\tau)
=
\int_{-\infty}^\infty d\tau' f(\tau') g(\tau' + \tau)
=
-\int_{\infty}^{-\infty} d\tau' f(-\tau') g(-\tau' + \tau)
=
\int_{-\infty}^{\infty} d\tau' f(\tau') g(-\tau' + \tau)
=
\int_{-\infty}^{\infty} d\tau' f(\tau') g(\tau' - \tau).
\end{dmath}

Okay, yes, the convolution of an even function is even.
} % makeanswer

\makeproblem{What's the cosine transform of a convolution of even functions}{pr:cosineTransforms:5}{ } % makeproblem
\makeanswer{pr:cosineTransforms:5}{

It's not obvious that we can even do this.  If we start naively

\begin{dmath}\label{eqn:cosineTransforms:210}
\int_0^\infty d\tau \cos(\omega \tau) \int_{-\infty}^\infty f(\tau') g(\tau' - \tau) d\tau'
=
\int_{-\infty}^\infty
d\tau'
f(\tau')
\int_0^\infty d\tau \cos(\omega \tau)
g(\tau' - \tau)
=
\int_{-\infty}^\infty
d\tau'
f(\tau')
\int_{-\tau'}^\infty d\tau \cos(\omega (u + \tau'))
g(u)
=
\int_{-\infty}^\infty
d\tau'
f(\tau')
\int_{-\tau'}^\infty d\tau \left(
\cos(\omega u) \cos(\omega \tau')
-\sin(\omega u) \sin(\omega \tau')
\right)
g(u).
\end{dmath}

We see this \([0, \infty]\) interval causes us some trouble.  If we had a symmetric interval, the sine term would be killed off, but we don't have that.  Some thought, and explicit demonstration (above) that the convolution of even functions is also even, we can still evaluate this, but have to step back and double the interval.

\begin{dmath}\label{eqn:cosineTransforms:230}
\int_0^\infty d\tau \cos(\omega \tau) \int_{-\infty}^\infty f(\tau') g(\tau' - \tau) d\tau'
=
\inv{2} \int_{-\infty}^\infty d\tau \cos(\omega \tau) \int_{-\infty}^\infty f(\tau') g(\tau' - \tau) d\tau'
=
\inv{2} \int_{-\infty}^\infty d\tau e^{-i \omega \tau} \int_{-\infty}^\infty f(\tau') g(-\tau' + \tau) d\tau'
=
\inv{2}
\int_{-\infty}^\infty f(\tau')
d\tau'
\int_{-\infty}^\infty d\tau e^{-i \omega \tau} g(-\tau' + \tau)
=
% u = \tau - \tau'
% u + \tau' = \tau
\inv{2}
\int_{-\infty}^\infty f(\tau')
d\tau'
\int_{-\infty}^\infty du e^{-i \omega (u + \tau')} g(u)
=
\inv{2}
\tilde{g}(\omega)
\int_{-\infty}^\infty f(\tau')
d\tau'
e^{-i \omega \tau'}
=
\inv{2}
\tilde{g}(\omega)
\tilde{f}(\omega)
\end{dmath}

We find then for the Cosine transform of a convolution

\boxedEquation{eqn:cosineTransforms:250}{
\int_0^\infty d\tau \cos(\omega \tau)
% too small:
%\widetilde{
\left( f(\tau) \conj g(\tau) \right)
%}_c
=
2
\tilde{g}_c(\omega)
\tilde{f}_c(\omega).
}

} % makeanswer

%\EndNoBibArticle
