%
% Copyright � 2015 Peeter Joot.  All Rights Reserved.
% Licenced as described in the file LICENSE under the root directory of this GIT repository.
%
\makeproblem{Harmonic oscillator}{gradQuantum:problemSet8:3}{ 
%\makesubproblem{}{gradQuantum:problemSet8:3a}

Consider a 2D harmonic oscillator with

\begin{dmath}\label{eqn:gradQuantumProblemSet8Problem3:20}
H =
\frac{p_x^2}{2m}
+\frac{p_y^2}{2m}
+ \inv{2} m \omega^2 \lr{ x^2 + y^2 }
\end{dmath}

Turn on an anharmonic perturbation 

\begin{dmath}\label{eqn:gradQuantumProblemSet8Problem3:40}
V = \lambda \lr{ x^4 + y^4 } + \lambda^2 x y.
\end{dmath}

Find the unperturbed eigenstates and the corresponding energy shifts upto \( O(\lambda^2) \)
for the ground state and the two-fold degenerate excited states. Ignore terms of \( O(\lambda^3) \).

} % makeproblem

\makeanswer{gradQuantum:problemSet8:3}{ 
%\makeSubAnswer{}{gradQuantum:problemSet8:3a}

With a \( \lambda^2 \) pertubation we have to step back and revisit the derivation of the energy level and perturbed state formulas.  Given

\begin{dmath}\label{eqn:gradQuantumProblemSet8Problem3:60}
H = H_0 + \lambda V_1 + \lambda^2 V_2,
\end{dmath}

with known solution \( H_0 \ket{n^{(0)}} = E^{(0)} \ket{n^{(0)}} \), we seek the a power series representation of the perturbed ket and an energy shift \( \Delta \)

\begin{dmath}\label{eqn:gradQuantumProblemSet8Problem3:80}
\ket{n} = \ket{n_0} + \lambda \ket{n_1} + \lambda^2 \ket{n_2} + \cdots
\end{dmath}
\begin{dmath}\label{eqn:gradQuantumProblemSet8Problem3:100}
\Delta = \lambda \Delta^{(1)} + \lambda^2 \Delta^{(2)} + \cdots
\end{dmath}

where

\begin{dmath}\label{eqn:gradQuantumProblemSet8Problem3:120}
H \ket{n} = \lr{ E^{(0)} + \Delta } \ket{n}.
\end{dmath}

We can assume that the we have the same sort of representation of the perturbed state

\begin{dmath}\label{eqn:gradQuantumProblemSet8Problem3:140}
\ket{n} = \ket{n^{(0)}} + \frac{\overbar{P}_n}{E^{(0)} - H_0} \lr{ \lambda_1 V_1 + \lambda^2 V_2 - \Delta } \ket{n},
\end{dmath}

where

\begin{dmath}\label{eqn:gradQuantumProblemSet8Problem3:160}
\overbar{P}_n = 1 - \ket{n^{(0)}}\bra{n^{(0)}} = \sum_{m \ne n} \ket{m^{(0)}}\bra{m^{(0)}}.
\end{dmath}

To check this, operating with \( E^{(0)} - H_0 \), we have

\begin{dmath}\label{eqn:gradQuantumProblemSet8Problem3:180}
\lr{ E^{(0)} - H_0 } \ket{n}
=
\lr{ E^{(0)} - H_0 } \ket{n^{(0)}} + 
\overbar{P}_n \lr{ \lambda_1 V_1 + \lambda^2 V_2 - \Delta } \ket{n}
=
\lr{ 1 - \ket{n^{(0)}}\bra{n^{(0)}} } \lr{ H - H_0 - \Delta } \ket{n},
\end{dmath}

or
\begin{dmath}\label{eqn:gradQuantumProblemSet8Problem3:200}
\lr{ E^{(0)} - H + \Delta} \ket{n}
=
-\ket{n^{(0)}}\bra{n^{(0)}} \lr{ H - H_0 - \Delta } \ket{n}
=
-\ket{n^{(0)}} \lr{ 
\lr{ E^{(0)} + \Delta} - E^{(0)} -\Delta } \braket{n^{(0)}}{n}
= 0.
\end{dmath}

The LHS is also zero as desired, showing we have the desired pertubation relationship.  For the energy shifts consider the braket

\begin{dmath}\label{eqn:gradQuantumProblemSet8Problem3:220}
\bra{n^{(0)}} H -H_0 \ket{n}
=
\bra{n^{(0)}} { V_1 \Delta^{(1)} + V_2 \Delta^{(2)} } \ket{n}
=
\lr{ \lr{ E^{(0)} + \Delta } -E^{(0)} } \braket{n^{(0)}}{n}
=
\Delta \ket{n^{(0)}}{n},
\end{dmath}

or
%As with the a first order \( \lambda \) perturbation, we can impose a requirement that \( \braket{0^{(0)}}{n} = 1 \), so

\begin{dmath}\label{eqn:gradQuantumProblemSet8Problem3:240}
\Delta \braket{n^{(0)}}{n} = \bra{n^{(0)}} { V_1 \Delta^{(1)} + V_2 \Delta^{(2)} } \ket{n}.
\end{dmath}

Expanding both sides in powers of \( \lambda \) we have

\begin{dmath}\label{eqn:gradQuantumProblemSet8Problem3:260}
\sum_{r = 1, s = 0} \lambda^{r+s} \Delta^{(r)} \braket{n^{(0)}}{n_s}
= 
\sum_{m = 0} \lambda^{m+1} \bra{n^{(0)}} V_1 \ket{n_m} + \lambda^{m+2} \bra{n^{(0)}} V_2 \ket{n_m}
\end{dmath}

With \( \ket{n^{(0)}} = \ket{n_0} \) as required in the \( \lambda \rightarrow 0 \) limit, the \( \lambda = 1 \) contribution to these sums is

%\begin{dmath}\label{eqn:gradQuantumProblemSet8Problem3:280}
\boxedEquation{eqn:gradQuantumProblemSet8Problem3:300}{
\Delta^{(1)} = \bra{n_0} V_1 \ket{n_0}.
}
%\end{dmath}

The second order contribution is

\begin{dmath}\label{eqn:gradQuantumProblemSet8Problem3:320}
\Delta^{(1)} \braket{n^{(0)}}{n_1} + \Delta^{(2)} \ket{n^{(0)}}{n_0}
=
\bra{n^{(0)}} V_1 \ket{n_1} + \bra{n^{(0)}} V_2 \ket{n_0},
\end{dmath}

or
\boxedEquation{eqn:gradQuantumProblemSet8Problem3:340}{
\Delta^{(2)} = \bra{n_0} V_1 \ket{n_1} + \bra{n_0} V_2 \ket{n_0} - \bra{n_0} V_1 \ket{n_0} \braket{n_0}{n_1} 
}

We can write this as
\begin{equation}\label{eqn:gradQuantumProblemSet8Problem3:360}
\begin{aligned}
\Delta^{(2)} &= \bra{n_0} V_1 \ket{n_1}_\perp + \bra{n_0} V_2 \ket{n_0} \\
\ket{n_1}_\perp &= \biglr{ 1 - \ket{n_0}\bra{n_0} } \ket{n_1}.
\end{aligned}
\end{equation}

where \( \ket{n_1}_\perp \) is the rejection of \( \ket{n_0} \) from the first order perturbed state \( \ket{n_1} \), the portions of \( \ket{n_1} \) that are orthogonal to \( \ket{n_0} \).

TODO.
}
