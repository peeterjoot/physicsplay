%
% Copyright � 2015 Peeter Joot.  All Rights Reserved.
% Licenced as described in the file LICENSE under the root directory of this GIT repository.
%
\makeproblem{Aharanov Bohm effect}{gradQuantum:problemSet3:3}{ 

Consider an electron confined to the interior of a finite hollow cylinder with its axis being \( \zcap \).
Let the inner and outer
walls of the cylinder be at radial coordinates \( \rho_a \) and \( \rho_b > \rho_a \) respectively.
Let the cylinder have its top and bottom ends at \( z = 0,L \).

\makesubproblem{}{gradQuantum:problemSet3:3a}
Find the eigenstates for a particle confined to this cylinder (ignore normalization), and show that its energies are given by

\begin{equation}\label{eqn:gradQuantumProblemSet3Problem3:20}
E_{l m n} = \frac{\Hbar^2}{2 m } \lr{ k_{m n}^2 + \lr{ \frac{ \pi l }{L} }^2 } \qquad ( l = 1,2,3, \cdots ; m = 0, 1, 2, \cdots )
\end{equation}

where \( k_{m n} \) is the $n$th root of the equation

\begin{dmath}\label{eqn:gradQuantumProblemSet3Problem3:40}
J_m (k_{m n} \rho_b ) N_m (k_{m n} \rho_a ) - N_m (k_{m n} \rho_b ) J_m (k_{m n} \rho_a ) = 0.
\end{dmath}

\makesubproblem{}{gradQuantum:problemSet3:3b}
Repeat this problem with a uniform magnetic field \( B \zcap \) which is confined to the region \( 0 < \rho < \rho_a \) (i.e., only in the hollow part of the cylinder). Show that there is a periodicity of the energy levels with the field, with the period
being such that \( \pi \rho_a^2 B = 2 \pi N \hbar/e \).
} % makeproblem

\makeanswer{gradQuantum:problemSet3:3}{ 

We can model this geometrical constraints of these two configurations by

\begin{dmath}\label{eqn:gradQuantumProblemSet3Problem3:60}
H = \inv{2 m} \lr{ -i \Hbar \spacegrad - \BA } + V,
\end{dmath}

where

\begin{dmath}\label{eqn:gradQuantumProblemSet3Problem3:80}
\spacegrad = \rhocap \partial_\rho + \inv{\rho} \partial_\phi + \zcap \partial_z,
\end{dmath}

and 

\begin{dmath}\label{eqn:gradQuantumProblemSet3Problem3:100}
V = 
\left\{
\begin{array}{l l}
0& \quad \mbox{if \( \rho \in [\rho_a,\rho_b], z \in [0,L] \) } \\
\infty & \quad \mbox{otherwise} 
\end{array}
\right.
\end{dmath}

The effects of the potential require that a wavefunction \( \psi \) solution of this equation satisfies

\begin{dmath}\label{eqn:gradQuantumProblemSet3Problem3:120}
\evalbar{\psi( z )}{z = 0,L} = 0,
\end{dmath}

and

\begin{dmath}\label{eqn:gradQuantumProblemSet3Problem3:140}
\evalbar{\psi( \rho )}{\rho = \rho_a, \rho_b} = 0,
\end{dmath}

leaving

\begin{dmath}\label{eqn:gradQuantumProblemSet3Problem3:160}
H = \inv{2 m} \lr{ -i \Hbar \spacegrad - \BA }^2.
\end{dmath}

in the interior region of the cylinder where the electron is free to move.

\makeSubAnswer{}{gradQuantum:problemSet3:3a}

Without a magnetic field, in the interior of the cylinder, the Hamiltonian is

\begin{dmath}\label{eqn:gradQuantumProblemSet3Problem3:180}
H \psi
= \frac{-\Hbar^2}{2 m} \spacegrad^2 \psi
=
\frac{-\Hbar^2}{2 m} \lr{ 
\inv{\rho} \partial_\rho \lr{ \rho \partial_\rho \psi } + \inv{\rho^2} \partial_{\phi\phi} \psi + \partial_{z z} \psi
}.
\end{dmath}

Assuming a solution is possible using separation of variables, let

\begin{dmath}\label{eqn:gradQuantumProblemSet3Problem3:200}
\psi(\rho, \phi, z) = P(\rho) \Phi(\phi) Z(z),
\end{dmath}

so that 

\begin{dmath}\label{eqn:gradQuantumProblemSet3Problem3:320}
H \psi = E \psi = E P \Phi Z = 
\frac{-\Hbar^2}{2 m} \lr{
\Phi Z \inv{\rho} \partial_\rho \lr{ \rho \partial_\rho P } + P Z \inv{\rho^2} \partial_{\phi\phi} \Phi + P \Phi \partial_{z z} Z
},
\end{dmath}

or

\begin{dmath}\label{eqn:gradQuantumProblemSet3Problem3:220}
E = 
\frac{-\Hbar^2}{2 m} \lr{
\inv{\rho P} \partial_\rho \lr{ \rho \partial_\rho P } + \inv{\rho^2 \Phi} \partial_{\phi\phi} \Phi + \inv{Z} \partial_{z z} Z
},
\end{dmath}

Let \( E = E' + E_z \), where

\begin{dmath}\label{eqn:gradQuantumProblemSet3Problem3:240}
E_z = \frac{-\Hbar^2}{2 m} \inv{Z} Z'',
\end{dmath}

This has solution

\begin{dmath}\label{eqn:gradQuantumProblemSet3Problem3:260}
Z = e^{i k_z z},
\end{dmath}

where \( k_z = \sqrt{2 m E_z}/\Hbar \).  The \( z = 0,L \) boundary condition requires that

\begin{equation}\label{eqn:gradQuantumProblemSet3Problem3:340}
k_z L = \pi l, \qquad l \in \bbZ
\end{equation}

or 

\begin{dmath}\label{eqn:gradQuantumProblemSet3Problem3:280}
k_z = \frac{\pi l}{L}.
\end{dmath}

That means that the total energy is of the form

\begin{dmath}\label{eqn:gradQuantumProblemSet3Problem3:300}
E = E' + \frac{\Hbar^2}{2 m} \lr{ \frac{\pi l}{L} }^2.
\end{dmath}

Now we consider the remaining subset of the eigenvalue equation

\begin{dmath}\label{eqn:gradQuantumProblemSet3Problem3:360}
E' \rho^2 = 
\frac{-\Hbar^2}{2 m} \lr{
\frac{\rho}{P} \partial_\rho \lr{ \rho \partial_\rho P } + \inv{\Phi} \partial_{\phi\phi} \Phi 
}.
\end{dmath}

We are free to let

\begin{dmath}\label{eqn:gradQuantumProblemSet3Problem3:380}
\frac{-\Hbar^2}{2 m} \inv{\Phi} \partial_{\phi\phi} \Phi = E_\phi,
\end{dmath}

which has solution

\begin{dmath}\label{eqn:gradQuantumProblemSet3Problem3:400}
\Phi = e^{i k_\phi \phi},
\end{dmath}

where \( k_\phi = \sqrt{2 m E_\phi}/\Hbar \).  Geometry requires \( k_\phi (2 \pi) = 2 \pi \nu, \nu \in \bbZ \ge 0 \),

or

\begin{dmath}\label{eqn:gradQuantumProblemSet3Problem3:420}
E_\phi = \frac{\Hbar^2 \nu^2}{2 m}.
\end{dmath}

This leaves

\begin{dmath}\label{eqn:gradQuantumProblemSet3Problem3:440}
E' \rho^2 = \frac{-\Hbar^2}{2 m} 
\frac{\rho}{P} \partial_\rho \lr{ \rho \partial_\rho P } + \frac{\Hbar^2 \nu^2}{2 m}.
\end{dmath}

\makeSubAnswer{}{gradQuantum:problemSet3:3b}

TODO.
}

