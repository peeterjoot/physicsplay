%
% Copyright � 2015 Peeter Joot.  All Rights Reserved.
% Licenced as described in the file LICENSE under the root directory of this GIT repository.
%
\makeproblem{Landau Levels - Symmetric gauge}{gradQuantum:problemSet3:2}{ 

Consider a charged particle moving in two dimensions (\(xy\)-plane) in a uniform magnetic field \( B_0 \zcap \) perpendicular
to the plane.
Let us work in a different gauge from the Landau gauge we discussed in class, namely, let us set

\begin{dmath}\label{eqn:gradQuantumProblemSet3Problem2:20}
\BA = \frac{B_0}{2} \lr{ x \ycap - y \xcap },
\end{dmath}

where \( (x,y) \) denotes the particle position.
This is called the `symmetric gauge'.

In this gauge, 
\makesubproblem{}{gradQuantum:problemSet3:2a}
work out the energy spectrum, and the eigenfunctions, and
\makesubproblem{}{gradQuantum:problemSet3:2b}
provide a crude counting of the number of states per energy level (i.e., the degeneracy) for an electron on a disk of radius \( R \).
} % makeproblem

\makeanswer{gradQuantum:problemSet3:2}{ 

\makeSubAnswer{}{gradQuantum:problemSet3:2a}

Following \citep{desai2009quantum}, the Hamiltonian for magnetic field driven motion constrained to a plane, can be factored into raising and lowering style operators

\begin{dmath}\label{eqn:gradQuantumProblemSet3Problem2:40}
H 
= \inv{2 m} \lr{ \Bp - \frac{e}{c} \BA }^2
= \inv{2 m} \lr{ \Pi_x^2 + \Pi_y^2 }^2
= \inv{2 m} \lr{ \lr{ \Pi_x - i \Pi_y }\lr{ \Pi_x + i \Pi_y } - i \lr{ \Pi_x \Pi_y - \Pi_y \Pi_x } },
\end{dmath}

That commutator term is proportional to the magnetic field strength
\begin{dmath}\label{eqn:gradQuantumProblemSet3Problem2:60}
\Pi_x \Pi_y - \Pi_y \Pi_x
=
\antisymmetric{\Pi_x}{\Pi_y}
=
\antisymmetric{p_x - \frac{e}{c} A_x}{p_y - \frac{e}{c} A_y}
=
\cancel{\antisymmetric{p_x}{p_y}} - \frac{e}{c} \lr{ \antisymmetric{A_x}{p_y} + \antisymmetric{p_x}{A_y} } + \lr{\frac{e}{c}}^2 \cancel{\antisymmetric{A_x}{A_y}}
= 
- \frac{e}{c} (-i \Hbar) \lr{ \PD{x}{A_y} - \PD{y}{A_x} }
= 
i \frac{e \Hbar}{c} B_z
= 
i \frac{e \Hbar B_0}{c},
\end{dmath}

so

\begin{dmath}\label{eqn:gradQuantumProblemSet3Problem2:80}
H 
= \inv{2 m} \lr{ \Pi_x - i \Pi_y }\lr{ \Pi_x + i \Pi_y } + \frac{e \Hbar B_0}{2 m c}.
\end{dmath}

Writing 

\begin{dmath}\label{eqn:gradQuantumProblemSet3Problem2:100}
\omega = \frac{e B_0}{m c},
\end{dmath}

this appears to have the structure of the 1D Harmonic oscillator
\begin{dmath}\label{eqn:gradQuantumProblemSet3Problem2:120}
H 
= \inv{2 m} \lr{ \Pi_x - i \Pi_y }\lr{ \Pi_x + i \Pi_y } + \frac{\Hbar \omega}{2}.
\end{dmath}

Observe that 

\begin{dmath}\label{eqn:gradQuantumProblemSet3Problem2:140}
\antisymmetric{ \Pi_x + i \Pi_y }{ \Pi_x - i \Pi_y }
=
i \antisymmetric{ \Pi_y }{ \Pi_x } - i \antisymmetric{ \Pi_x }{ \Pi_y }
=
- 2 i \antisymmetric{ \Pi_x }{ \Pi_y }
=
\frac{2 e \Hbar B_0}{c}
=
2 m \omega.
\end{dmath}

With

\begin{dmath}\label{eqn:gradQuantumProblemSet3Problem2:160}
b = \inv{\sqrt{2 m \omega \Hbar}} \lr{ \Pi_x + i \Pi_y },
\end{dmath}

the Hamiltonian has the form

\begin{dmath}\label{eqn:gradQuantumProblemSet3Problem2:180}
H = \Hbar \omega \lr{ b^\dagger b + \inv{2} },
\end{dmath}

where

\begin{dmath}\label{eqn:gradQuantumProblemSet3Problem2:200}
\antisymmetric{b}{b^\dagger} = 1,
\end{dmath}

just like the 1D SHO.  The energy levels are therefore

\begin{dmath}\label{eqn:gradQuantumProblemSet3Problem2:220}
E_n = \Hbar \omega \lr{ n + \inv{2} }.
\end{dmath}

For the symmetric gauge where \( A_x = -y, A_y = x \), the lowering operator has the form

\begin{dmath}\label{eqn:gradQuantumProblemSet3Problem2:n}
b 
= \inv{\sqrt{2 m \omega \Hbar}} \lr{ p_x + i p_y - \frac{e}{c} \frac{B_0}{2} ( -y + i x ) }
= \frac{-i \Hbar}{\sqrt{2 m \omega \Hbar}} \lr{ \partial_x + i \partial_y + \frac{e B_0}{2 \Hbar c} ( i y + x ) }
= \frac{-i \Hbar}{\sqrt{2 m \omega \Hbar}} \lr{ \partial_x + i \partial_y + \frac{m \omega}{2 \Hbar} ( x + i y) }.
\end{dmath}

Introducing two independent variables

\begin{dmath}\label{eqn:gradQuantumProblemSet3Problem2:n}
\begin{aligned}
z &= x + i y \\
z^\conj &= x - i y,
\end{aligned}
\end{dmath}

the lowering operator takes the form

\begin{dmath}\label{eqn:gradQuantumProblemSet3Problem2:n}
b = -i \sqrt{\frac{\Hbar}{2 m \omega}} \lr{ \PD{z^\conj} + \frac{m \omega}{2 \Hbar} z }.
\end{dmath}

Solving \( b \ket{0} = 0 \), and writing \( u_0(z, z^\conj) = \braket{z,z^\conj}{0} \), we have

\begin{dmath}\label{eqn:gradQuantumProblemSet3Problem2:n}
u_0(z, z^\conj) = z^q e^{- \frac{m \omega}{2 \Hbar} z z^\conj },
\end{dmath}

%Application of the raising operator gives
%
%u_1(z,z^\conj) 
%= \braket{z,z^\conj}{1} 
%= \bra{z,z^\conj} b^\dagger \ket{0} 
%\propto \lr{ \PD{z} + \frac{m \omega}{2 \Hbar} z^\conj }

\makeSubAnswer{}{gradQuantum:problemSet3:2b}
TODO
}
