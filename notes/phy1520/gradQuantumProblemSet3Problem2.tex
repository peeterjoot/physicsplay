%
% Copyright � 2015 Peeter Joot.  All Rights Reserved.
% Licenced as described in the file LICENSE under the root directory of this GIT repository.
%
%\makeoproblem{Landau Levels - Symmetric gauge}{gradQuantum:problemSet3:2}{phy1520 2015 ps3.2}
\makeproblem{Landau Levels - Symmetric gauge}{gradQuantum:problemSet3:2}
{ 

Consider a charged particle moving in two dimensions (\(xy\)-plane) in a uniform magnetic field \( B_0 \zcap \) perpendicular
to the plane.
Let us work in a different gauge from the Landau gauge we discussed in class, namely, let us set

\begin{dmath}\label{eqn:gradQuantumProblemSet3Problem2:20}
\BA = \frac{B_0}{2} \lr{ x \ycap - y \xcap },
\end{dmath}

where \( (x,y) \) denotes the particle position.
This is called the `symmetric gauge'.

In this gauge, 
\makesubproblem{}{gradQuantum:problemSet3:2a}
work out the energy spectrum, and the eigenfunctions, and
\makesubproblem{}{gradQuantum:problemSet3:2b}
provide a crude counting of the number of states per energy level (i.e., the degeneracy) for an electron on a disk of radius \( R \).
} % makeproblem

\makeanswer{gradQuantum:problemSet3:2}{ 

\makeSubAnswer{}{gradQuantum:problemSet3:2a}

Using the approach suggested by \citep{sakurai2014modern} \textprref{39}, 
%Following \citep{desai2009quantum}, 
the Hamiltonian for magnetic field driven motion constrained to a plane, can be factored into raising and lowering style operators

\begin{dmath}\label{eqn:gradQuantumProblemSet3Problem2:40}
H 
= \inv{2 m} \lr{ \Bp - \frac{e}{c} \BA }^2
= \inv{2 m} \lr{ \Pi_x^2 + \Pi_y^2 }^2
= \inv{2 m} \lr{ \lr{ \Pi_x - i \Pi_y }\lr{ \Pi_x + i \Pi_y } - i \lr{ \Pi_x \Pi_y - \Pi_y \Pi_x } },
\end{dmath}

That commutator term is proportional to the magnetic field strength
\begin{dmath}\label{eqn:gradQuantumProblemSet3Problem2:60}
\Pi_x \Pi_y - \Pi_y \Pi_x
=
\antisymmetric{\Pi_x}{\Pi_y}
=
\antisymmetric{p_x - \frac{e}{c} A_x}{p_y - \frac{e}{c} A_y}
=
\cancel{\antisymmetric{p_x}{p_y}} - \frac{e}{c} \lr{ \antisymmetric{A_x}{p_y} + \antisymmetric{p_x}{A_y} } + \lr{\frac{e}{c}}^2 \cancel{\antisymmetric{A_x}{A_y}}
= 
- \frac{e}{c} (-i \Hbar) \lr{ \PD{x}{A_y} - \PD{y}{A_x} }
= 
i \frac{e \Hbar}{c} B_z
= 
i \frac{e \Hbar B_0}{c},
\end{dmath}

so

\begin{dmath}\label{eqn:gradQuantumProblemSet3Problem2:80}
H 
= \inv{2 m} \lr{ \Pi_x - i \Pi_y }\lr{ \Pi_x + i \Pi_y } + \frac{e \Hbar B_0}{2 m c}.
\end{dmath}

Writing 

\begin{dmath}\label{eqn:gradQuantumProblemSet3Problem2:100}
\omega = \frac{e B_0}{m c},
\end{dmath}

this appears to have the structure of the 1D Harmonic oscillator
\begin{dmath}\label{eqn:gradQuantumProblemSet3Problem2:120}
H 
= \inv{2 m} \lr{ \Pi_x - i \Pi_y }\lr{ \Pi_x + i \Pi_y } + \frac{\Hbar \omega}{2}.
\end{dmath}

Observe that 

\begin{dmath}\label{eqn:gradQuantumProblemSet3Problem2:140}
\antisymmetric{ \Pi_x + i \Pi_y }{ \Pi_x - i \Pi_y }
=
i \antisymmetric{ \Pi_y }{ \Pi_x } - i \antisymmetric{ \Pi_x }{ \Pi_y }
=
- 2 i \antisymmetric{ \Pi_x }{ \Pi_y }
=
\frac{2 e \Hbar B_0}{c}
=
2 m \omega.
\end{dmath}

With

\begin{dmath}\label{eqn:gradQuantumProblemSet3Problem2:160}
b = \inv{\sqrt{2 m \omega \Hbar}} \lr{ \Pi_x + i \Pi_y },
\end{dmath}

the Hamiltonian has the form

\begin{dmath}\label{eqn:gradQuantumProblemSet3Problem2:180}
H = \Hbar \omega \lr{ b^\dagger b + \inv{2} },
\end{dmath}

where

\begin{dmath}\label{eqn:gradQuantumProblemSet3Problem2:200}
\antisymmetric{b}{b^\dagger} = 1,
\end{dmath}

just like the 1D SHO.  The energy levels are therefore

\begin{dmath}\label{eqn:gradQuantumProblemSet3Problem2:220}
E_n = \Hbar \omega \lr{ n + \inv{2} }.
\end{dmath}

For the symmetric gauge where \( A_x = -y, A_y = x \), the lowering operator has the form

\begin{dmath}\label{eqn:gradQuantumProblemSet3Problem2:240}
b 
= \inv{\sqrt{2 m \omega \Hbar}} \lr{ p_x + i p_y - \frac{e}{c} \frac{B_0}{2} ( -y + i x ) }
= \frac{-i \Hbar}{\sqrt{2 m \omega \Hbar}} \lr{ \partial_x + i \partial_y + \frac{e B_0}{2 \Hbar c} ( i y + x ) }
= \frac{-i \Hbar}{\sqrt{2 m \omega \Hbar}} \lr{ \partial_x + i \partial_y + \frac{m \omega}{2 \Hbar} ( x + i y) }.
\end{dmath}

With

\begin{equation}\label{eqn:gradQuantumProblemSet3Problem2:260}
\alpha = \frac{m \omega}{2 \Hbar} = \frac{e B_0}{2 \Hbar c},
\end{equation}

The first state is defined by

\begin{dmath}\label{eqn:gradQuantumProblemSet3Problem2:280}
0 
= 
\bra{x,y} b \ket{0} 
\propto
\lr{ \partial_x + i \partial_y + \alpha ( x + i y) } u_0(x,y).
\end{dmath}

For integer \( q \), this has solutions of the form

\begin{dmath}\label{eqn:gradQuantumProblemSet3Problem2:300}
u_0(x,y) = c_q \lr{ x + i y }^q e^{-\alpha \lr{x^2 + y^2}/2},
\end{dmath}

which can be verified directly

\begin{dmath}\label{eqn:gradQuantumProblemSet3Problem2:320}
\begin{aligned}
\biglr{ &\partial_x + i \partial_y + \alpha ( x + i y) } u_0(x,y) \\
&=
\biglr{ \partial_x + i \partial_y + \alpha ( x + i y) } c_q \lr{ x + i y }^q e^{-\alpha \lr{x^2 + y^2}/2} \\
&=
c_q e^{-\alpha \lr{x^2 + y^2}/2} \lr{
q \lr{ x + i y }^{q-1} \lr{ 1 + i^2 } -\alpha \lr{ x + i y }^q \lr{ 2x + 2 y i }/2 + \alpha \lr{ x + i y }^{q+1}
} \\
&= 0.
\end{aligned}
\end{dmath}

The normalization is given by

\begin{dmath}\label{eqn:gradQuantumProblemSet3Problem2:n}
1 
= \int \Abs{u_0(x,y)}^2 dx dy
= \Abs{c_q}^2 \int \Abs{x + i y}^{2q} e^{-\alpha(x^2 + y^2)} dx dy
= \Abs{c_q}^2 \int_0^\infty 2 \pi r r^{2q} e^{-\alpha r^2} dr.
\end{dmath}

Let \( \alpha r^2 = t \), with \( 2 r dr = dt/\alpha \), for

\begin{dmath}\label{eqn:gradQuantumProblemSet3Problem2:n}
1
= \Abs{c_q}^2 \frac{\pi}{\alpha} \int_0^\infty \lr{\frac{t}{\alpha}}^q e^{-t} dt
= \Abs{c_q}^2 \frac{\pi}{\alpha^{q+1}} \int_0^\infty t^{(q + 1) - 1} e^{-t} dt
= \Abs{c_q}^2 \frac{\pi}{\alpha^{q+1}} \Gamma(q+1)
= \Abs{c_q}^2 \frac{\pi}{\alpha^{q+1}} q!,
\end{dmath}

so

\begin{dmath}\label{eqn:gradQuantumProblemSet3Problem2:n}
c_q = \sqrt{\frac{\alpha^{q+1}}{q! \pi}},
\end{dmath}

and

\boxedEquation{eqn:gradQuantumProblemSet3Problem2:n}{
u_0(x,y) = \sqrt{\frac{\alpha^{q+1}}{q! \pi}} \lr{ x + i y }^q e^{-\alpha (x^2 + y^2)/2}.
}

%The higher order wave functions are given by
%
%\begin{dmath}\label{eqn:gradQuantumProblemSet3Problem2:340}
%\braket{x,y}{n} 
%= \frac{\lr{b^\dagger}^n}{\sqrt{n!}} \braket{x,y}{0}
%= c_q \sqrt{\lr{\frac{-\Hbar}{2 m \omega}}^n \inv{n!}} \biglr{ \partial_x -i \partial_y + \alpha( x - i y ) }^n (x+ iy)^q e^{-\alpha (x^2 + y^2)/2}.
%\end{dmath}
%
%There is a simple closed form for these higher energy wave functions, which can be seen by computing \( \braket{x,y}{1} \)
%
%\begin{dmath}\label{eqn:gradQuantumProblemSet3Problem2:360}
%\begin{aligned}
%\biglr{ &\partial_x - i \partial_y + \alpha ( x - i y ) } c_q (x + i y)^q e^{-\alpha (x^2 + y^2)/2} \\
%&= 
%c_q e^{-\alpha (x^2 + y^2)/2}
%\lr{
%q (x + i y)^{q-1} ( 1 -i^2 ) + (-\alpha/2)( 2 x - i 2 y) (x + i y)^q + \alpha ( x - i y ) (x + i y)^q
%} \\
%&=
%2 q c_q (x + i y)^{q-1} e^{-\alpha (x^2 + y^2)/2},
%\end{aligned}
%\end{dmath}
%
%so the wave functions are
%
%\boxedEquation{eqn:gradQuantumProblemSet3Problem2:380}{
%\braket{x,y}{n}
%= c_q \sqrt{\lr{\frac{- 2\Hbar}{m \omega}}^n \inv{n!}} %2^n 
%%q(q-1)\cdots(q-n+1)
%\frac{q!}{(q-n)!}
%(x + i y)^{q-n} e^{-\alpha (x^2 + y^2)/2}.
%}

\makeSubAnswer{}{gradQuantum:problemSet3:2b}

When the particle is constrained to the unit circle \( x + i y = R e^{i\phi} \), the wave functions for the symmetric gauge take the form

\begin{dmath}\label{eqn:gradQuantumProblemSet3Problem2:400}
\braket{x,y}{n}
= c_q \sqrt{n! \lr{\frac{- 2\Hbar}{m \omega}}^n } 
\binom{q}{n}
R^{q-n} e^{i(q-n)\phi} 
e^{-\alpha R^2/2}.
\end{dmath}

TODO: degeneracy.
}
