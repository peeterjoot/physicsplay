%
% Copyright � 2015 Peeter Joot.  All Rights Reserved.
% Licenced as described in the file LICENSE under the root directory of this GIT repository.
%
\paragraph{Trig}

\begin{dmath}\label{eqn:usefulFormulas:20}
1 + \cos x = 2 \cos^2 \frac{x}{2}
\end{dmath}
\begin{dmath}\label{eqn:usefulFormulas:40}
1 - \cos x = 2 \sin^2 \frac{x}{2}
\end{dmath}
\begin{dmath}\label{eqn:usefulFormulas:60}
\sin x = 2 \sin \frac{x}{2} \cos \frac{x}{2}
\end{dmath}

\begin{dmath}\label{eqn:usefulFormulas:560}
2 \cos a \cos b = \cos(a + b) + \cos(a-b)
\end{dmath}
\begin{dmath}\label{eqn:usefulFormulas:580}
2 \sin a \sin b = \cos(a - b) - \cos(a-b)
\end{dmath}
\begin{dmath}\label{eqn:usefulFormulas:561}
2 \cos a \sin b = \sin(a + b) - \sin(a-b)
\end{dmath}
\begin{dmath}\label{eqn:usefulFormulas:581}
2 \sin a \cos b = \sin(a - b) + \sin(a+b)
\end{dmath}


\begin{dmath}\label{eqn:usefulFormulas:600}
\cos(a \pm b) = \cos a\cos b \mp \sin a \sin b
\end{dmath}
\begin{dmath}\label{eqn:usefulFormulas:620}
\sin(a \pm b) = \sin a\cos b \pm \cos a \sin b
\end{dmath}

\paragraph{Pauli matrices}

\begin{dmath}\label{eqn:usefulFormulas:80}
\begin{aligned}
\sigma_x &= \PauliX \\
\sigma_y &= \PauliY \\
\sigma_z &= \PauliZ
\end{aligned}
\end{dmath}

\begin{subequations}
\label{eqn:usefulFormulas:100}
\begin{dmath}\label{eqn:usefulFormulas:120}
\ket{S_x ; \pm } = 
\inv{\sqrt{2}}
\begin{bmatrix}
1 \\
\pm 1
\end{bmatrix}
\end{dmath}
\begin{dmath}\label{eqn:usefulFormulas:140}
\ket{S_y ; \pm } = 
\inv{\sqrt{2}} 
\begin{bmatrix}
1 \\
\pm i
\end{bmatrix}
\end{dmath}
\begin{dmath}\label{eqn:usefulFormulas:160}
\ket{S_z ; \pm } = 
\begin{bmatrix}
1 \\
0
\end{bmatrix}, 
\begin{bmatrix}
0 \\
1 \\
\end{bmatrix}
\end{dmath}
\end{subequations}

For \( \rcap = \lr{\sin\theta \cos\phi, \sin\theta\sin\phi, \cos\theta} \), the eigenkets of \( \BS \cdot \rcap \) are

\begin{dmath}\label{eqn:usefulFormulas:180}
\ket{+}
=
\begin{bmatrix}
\cos\lr{\theta/2} e^{-i\phi/2} \\
\sin\lr{\theta/2} e^{i\phi/2} \\
\end{bmatrix}
\end{dmath}

\begin{dmath}\label{eqn:usefulFormulas:200}
\ket{-}
=
\begin{bmatrix}
\sin\lr{\theta/2} e^{-i\phi/2} \\
-\cos\lr{\theta/2} e^{i\phi/2} \\
\end{bmatrix}
\end{dmath}

\paragraph{Harmonic oscillator}

\begin{dmath}\label{eqn:usefulFormulas:220}
x_0^2 = \frac{\Hbar}{m \omega}
\end{dmath}
\begin{dmath}\label{eqn:usefulFormulas:240}
x(t) = \frac{x_0}{\sqrt{2}} \lr{ a e^{-i\omega t} + a^\dagger e^{i \omega t} }
\end{dmath}
\begin{dmath}\label{eqn:usefulFormulas:260}
p(t) = \frac{i \Hbar}{\sqrt{2} x_0} \lr{ a e^{-i\omega t} + a^\dagger e^{i \omega t} }
\end{dmath}
\begin{dmath}\label{eqn:usefulFormulas:280}
a \ket{n} = \sqrt{n} \ket{n-1}
\end{dmath}
\begin{dmath}\label{eqn:usefulFormulas:300}
a^\dagger \ket{n} = \sqrt{n+1} \ket{n+1}
\end{dmath}
\begin{dmath}\label{eqn:usefulFormulas:320}
x(t) = x(0) \cos \omega t + \frac{p(0)}{m \omega} \sin \omega t
\end{dmath}
\begin{dmath}\label{eqn:usefulFormulas:340}
p(t) = p(0) \cos \omega t - m \omega x(0) \sin \omega t
\end{dmath}
\begin{dmath}\label{eqn:usefulFormulas:360}
x(t)^2 = \frac{\Hbar \omega}{2} \lr{ a e^{-i\omega t} + a^\dagger e^{i \omega t} }^2
\end{dmath}
\begin{dmath}\label{eqn:usefulFormulas:380}
p(t)^2 = \frac{\Hbar \omega}{2} 
\lr{ a e^{-i\omega t} - a^\dagger e^{i \omega t} }
\lr{ a^\dagger e^{i \omega t} -a e^{-i\omega t} }
\end{dmath}
\begin{dmath}\label{eqn:usefulFormulas:400}
H = \Hbar \omega \lr{ a^\dagger a + \inv{2} }
\end{dmath}
\begin{dmath}\label{eqn:usefulFormulas:420}
\antisymmetric{a}{a^\dagger} = 1
\end{dmath}
\begin{dmath}\label{eqn:usefulFormulas:440}
a = \inv{x_0 \sqrt{2}} \lr{ x + \frac{ i p }{m\omega} }
\end{dmath}

\paragraph{Commutators}

\begin{equation}\label{eqn:usefulFormulas:460}
\begin{aligned}
\antisymmetric{x_j}{F(\Bp)} &= i \Hbar \PD{p_j}{F} \\
\antisymmetric{p_j}{G(\Bx)} &= -i \Hbar \PD{x_j}{G}
\end{aligned}
\end{equation}

\paragraph{Heisenberg picture}

\begin{equation}\label{eqn:usefulFormulas:480}
U = e^{-i H t/\Hbar}
\end{equation}

\begin{equation}\label{eqn:usefulFormulas:500}
A_\txtH = U^\dagger A U
\end{equation}

\begin{equation}\label{eqn:usefulFormulas:520}
e^{A} B e^{-A} = B + \antisymmetric{A}{B} + \inv{2!} \antisymmetric{A}{\antisymmetric{A}{B}} + \cdots
\end{equation}

\paragraph{Current}

\begin{equation}\label{eqn:usefulFormulas:540}
\Bj = \frac{\Hbar}{m} \Imag\lr{ \Psi^\conj \spacegrad \Psi}
\end{equation}

