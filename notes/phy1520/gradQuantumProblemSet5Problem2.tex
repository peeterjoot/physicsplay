%
% Copyright � 2015 Peeter Joot.  All Rights Reserved.
% Licenced as described in the file LICENSE under the root directory of this GIT repository.
%
\makeproblem{Boosts}{gradQuantum:problemSet5:2}{ 

The momentum operator \( \hat{p} \) was shown, in class, to act as the generator of space translations. Show by following the exact same steps that the position operator \( \hat{x} \) acts as the generator of momentum boosts (i.e., \( \hat{x} \) is a generator of `momentum translation').

%\makesubproblem{}{gradQuantum:problemSet5:2a}
} % makeproblem

\makeanswer{gradQuantum:problemSet5:2}{ 
%\makeSubAnswer{}{gradQuantum:problemSet5:2a}

Borrowing the same notation as in class, for an infinitesimal change in momentum \( \epsilon \), define a momentum translation operator with the action on a momentum space state of

\begin{dmath}\label{eqn:gradQuantumProblemSet5Problem2:20}
\hat{T}_\epsilon \ket{p} 
=
\ket{p + \epsilon}.
\end{dmath}

A wave function matrix element for this momentum translation is
\begin{dmath}\label{eqn:gradQuantumProblemSet5Problem2:40}
\bra{p} \hat{T}_\epsilon \ket{\psi}
=
\braket{p - \epsilon}{\psi}
=
\psi_p(p - \epsilon)
\approx
\psi_p(p) - \epsilon \PD{p}{\psi_p(p)}.
\end{dmath}

Since the momentum space representation of the position operator is

\begin{dmath}\label{eqn:gradQuantumProblemSet5Problem2:60}
x \dotEquals i \Hbar \PD{p}{},
\end{dmath}

we have

\begin{dmath}\label{eqn:gradQuantumProblemSet5Problem2:80}
\bra{p} \hat{T}_\epsilon \ket{\psi}
\approx
\psi_p(p) - \epsilon \frac{x}{i \Hbar} \psi_p(p)
=
\lr{ 1 + \frac{\epsilon i x}{\Hbar} } \psi_p(p)
\end{dmath}

Given a finite change of momentum \( \calP = N \epsilon \), the matrix element has the limiting form

\begin{dmath}\label{eqn:gradQuantumProblemSet5Problem2:100}
\bra{p} \hat{T}_{\calP} \ket{\psi}
= \lim_{N \rightarrow \infty, \epsilon \rightarrow 0 } 
\lr{ 1 + \frac{\calP i x}{N \Hbar} }^N \psi_p(p)
=
e^{ i \calP x/\Hbar } \psi_p(p),
\end{dmath}

showing that \( x \) is the generator of the momentum translation operator

%\begin{dmath}\label{eqn:gradQuantumProblemSet5Problem2:120}
\boxedEquation{eqn:gradQuantumProblemSet5Problem2:140}{
\hat{T}_{\calP} = e^{ i \calP x/\Hbar }.
}
%\end{dmath}

}
