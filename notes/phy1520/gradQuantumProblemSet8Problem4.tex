%
% Copyright � 2015 Peeter Joot.  All Rights Reserved.
% Licenced as described in the file LICENSE under the root directory of this GIT repository.
%
\makeproblem{Hyperfine levels}{gradQuantum:problemSet8:4}{ 

We can schematically model the hyperfine interaction between the electron and proton spins as \( A \BS_e \cdot \BS_p \) where \( A \) is the hyperfine interaction energy. 

\makesubproblem{}{gradQuantum:problemSet8:4a}
Consider the spin-1/2 proton interacting with a spin-1/2 electron. 
What are the spin eigenstates and eigenvalues? 

\makesubproblem{}{gradQuantum:problemSet8:4b}
Now consider applying a magnetic field which leads to an extra term  

\begin{dmath}\label{eqn:gradQuantumProblemSet8Problem4:20}
-B \lr{ g_e \mu_e S_e^z + g_p \mu_p S_N^z }
\end{dmath}

with gyromagnetic ratios \( g_e \approx -2 \) and \( g_p \approx 5.5 \), with magnetic moments \( \mu_e = e/2m_e \) and
\( \mu_p = e/2m_p \). The large nuclear mass ensures \( \mu_e/\mu_p \sim 2000 \), so let us simply set \( \mu_p = 0\). For convenience, you can club \( B g_e \mu_e \rightarrow B_{\textrm{eff}} \) so the Hamiltonian becomes

\begin{dmath}\label{eqn:gradQuantumProblemSet8Problem4:40}
H = A \BS_e \cdot \BS_p - B_{\textrm{eff}} S_e^z,
\end{dmath}

so the only dimensionless parameter is \( B_{\textrm{eff}}/A \).

Using perturbation theory (degenerate or nondegenerate as appropriate) find how the coupled hyperfine levels split
for weak field \( B_{\textrm{eff}}/A \ll 1 \).
Also consider the strong field limit \( B_{\textrm{eff}}/A \gg 1 \). 
Compute the full field evolution of the levels and compare with the perturbative low field regime result and the high field regime result.

} % makeproblem

\makeanswer{gradQuantum:problemSet8:4}{ 
\makeSubAnswer{}{gradQuantum:problemSet8:4a}
%What are the spin eigenstates and eigenvalues? 

With respect to the basis \( \beta = \ket{++}, \ket{-+}, \ket{+-}, \ket{--} \), the interaction Hamiltonian is

\begin{dmath}\label{eqn:gradQuantumProblemSet8Problem4:60}
A \BS_e \cdot \BS_p
=
A
\begin{bmatrix}
\bra{++} \BS_e \cdot \BS_p \ket{++} & \bra{++} \BS_e \cdot \BS_p \ket{-+} & \bra{++} \BS_e \cdot \BS_p \ket{+-} & \bra{++} \BS_e \cdot \BS_p \ket{--} \\
\bra{-+} \BS_e \cdot \BS_p \ket{++} & \bra{-+} \BS_e \cdot \BS_p \ket{-+} & \bra{-+} \BS_e \cdot \BS_p \ket{+-} & \bra{-+} \BS_e \cdot \BS_p \ket{--} \\
\bra{+-} \BS_e \cdot \BS_p \ket{++} & \bra{+-} \BS_e \cdot \BS_p \ket{-+} & \bra{+-} \BS_e \cdot \BS_p \ket{+-} & \bra{+-} \BS_e \cdot \BS_p \ket{--} \\
\bra{--} \BS_e \cdot \BS_p \ket{++} & \bra{--} \BS_e \cdot \BS_p \ket{-+} & \bra{--} \BS_e \cdot \BS_p \ket{+-} & \bra{--} \BS_e \cdot \BS_p \ket{--} \\
\end{bmatrix}
=
\frac{A \Hbar^2}{4}
\begin{bmatrix}
\bra{+} \sigma_\txte \ket{+} \bra{+} \sigma_\txtp \ket{+} & \bra{+} \sigma_\txte \ket{-} \bra{+} \sigma_\txtp \ket{+} & \bra{+} \sigma_\txte \ket{+} \bra{+} \sigma_\txtp \ket{-} & \bra{+} \sigma_\txte \ket{-} \bra{+} \sigma_\txtp \ket{-} \\
\bra{-} \sigma_\txte \ket{+} \bra{+} \sigma_\txtp \ket{+} & \bra{-} \sigma_\txte \ket{-} \bra{+} \sigma_\txtp \ket{+} & \bra{-} \sigma_\txte \ket{+} \bra{+} \sigma_\txtp \ket{-} & \bra{-} \sigma_\txte \ket{-} \bra{+} \sigma_\txtp \ket{-} \\
\bra{+} \sigma_\txte \ket{+} \bra{-} \sigma_\txtp \ket{+} & \bra{+} \sigma_\txte \ket{-} \bra{-} \sigma_\txtp \ket{+} & \bra{+} \sigma_\txte \ket{+} \bra{-} \sigma_\txtp \ket{-} & \bra{+} \sigma_\txte \ket{-} \bra{-} \sigma_\txtp \ket{-} \\
\bra{-} \sigma_\txte \ket{+} \bra{-} \sigma_\txtp \ket{+} & \bra{-} \sigma_\txte \ket{-} \bra{-} \sigma_\txtp \ket{+} & \bra{-} \sigma_\txte \ket{+} \bra{-} \sigma_\txtp \ket{-} & \bra{-} \sigma_\txte \ket{-} \bra{-} \sigma_\txtp \ket{-} \\
\end{bmatrix}
=
\frac{A \Hbar^2}{4}
\begin{bmatrix}
(1) (1) & (0) (1) & (1) (0) & (0) (0) \\
(0) (1) & (-1) (1) & (0) (0) & (-1) (0) \\
(1) (0) & (0) (0) & (1) (-1) & (0) (-1) \\
(0) (0) & (-1) (0) & (0) (-1) & (-1) (-1) \\
\end{bmatrix}
=
\frac{A \Hbar^2}{4}
\begin{bmatrix}
\sigma_3 & 0 \\
0 & -\sigma_3
\end{bmatrix}.
\end{dmath}

The spin eigenstates are the basis elements of \( \beta \) above, with respective eigenvalues 

\begin{dmath}\label{eqn:gradQuantumProblemSet8Problem4:80}
\setlr{ A \Hbar^2/4, -A \Hbar^2/4, -A \Hbar^2/4, A \Hbar^2/4}
\end{dmath}

The matrix representation of the pertubation potential is

\begin{dmath}\label{eqn:gradQuantumProblemSet8Problem4:100}
-B_{\textrm{eff}} S^z_e
=
-\frac{B_{\textrm{eff}} \Hbar}{2}
\begin{bmatrix}
\bra{+} \sigma^z_e \ket{+} \braket{+}{+} & \bra{+} \sigma^z_e \ket{-} \braket{+}{+} & \bra{+} \sigma^z_e \ket{+} \braket{+}{-} & \bra{+} \sigma^z_e \ket{-} \braket{+}{-} \\
\bra{-} \sigma^z_e \ket{+} \braket{+}{+} & \bra{-} \sigma^z_e \ket{-} \braket{+}{+} & \bra{-} \sigma^z_e \ket{+} \braket{+}{-} & \bra{-} \sigma^z_e \ket{-} \braket{+}{-} \\
\bra{+} \sigma^z_e \ket{+} \braket{-}{+} & \bra{+} \sigma^z_e \ket{-} \braket{-}{+} & \bra{+} \sigma^z_e \ket{+} \braket{-}{-} & \bra{+} \sigma^z_e \ket{-} \braket{-}{-} \\
\bra{-} \sigma^z_e \ket{+} \braket{-}{+} & \bra{-} \sigma^z_e \ket{-} \braket{-}{+} & \bra{-} \sigma^z_e \ket{+} \braket{-}{-} & \bra{-} \sigma^z_e \ket{-} \braket{-}{-} \\
\end{bmatrix}
=
-\frac{B_{\textrm{eff}} \Hbar}{2}
\begin{bmatrix}
\sigma^z_e & 0 \\
0 & \sigma^z_e
\end{bmatrix},
\end{dmath}

Assuming the \( \BS_e \) operator is directed along \( \ncap = (\sin\theta \cos\phi, \sin\theta \sin\phi, \cos\theta) \) with eigenkets
\begin{dmath}\label{eqn:gradQuantumProblemSet8Problem4:120}
\ket{+} =
\begin{bmatrix}
e^{-i\phi} \cos(\theta/2) \\
\sin(\theta/2) \\
\end{bmatrix}
\end{dmath}
\begin{dmath}\label{eqn:gradQuantumProblemSet8Problem4:140}
\ket{-} =
\begin{bmatrix}
-e^{-i\phi} \sin(\theta/2) \\
\cos(\theta/2) \\
\end{bmatrix},
\end{dmath}

the representation of the \( \sigma^z_\txte \) operator is

\begin{dmath}\label{eqn:gradQuantumProblemSet8Problem4:160}
\sigma^z_\txte
=
\begin{bmatrix}
\cos\theta & -\sin\theta \\
-\sin\theta & -\cos\theta \\
\end{bmatrix}
= 
U \PauliZ U^{-1},
\end{dmath}

where
\begin{equation}\label{eqn:gradQuantumProblemSet8Problem4:180}
U = 
\begin{bmatrix}
-\cos(\theta/2) & \sin(\theta/2) \\
\sin(\theta/2) & \cos(\theta/2)
\end{bmatrix}.
\end{equation}

The full Hamiltonian can now be written in block matrix form

\begin{dmath}\label{eqn:gradQuantumProblemSet8Problem4:200}
H
= 
\frac{A \Hbar^2}{4}
\begin{bmatrix}
\sigma_z & 0 \\
0 & -\sigma_z
\end{bmatrix}
-\frac{B_{\textrm{eff}} \Hbar}{2}
\begin{bmatrix}
U \sigma_z U^{-1} & 0 \\
0 & U \sigma_z U^{-1}
\end{bmatrix}
\end{dmath}

Transforming the Hamiltonian to the \( S^z_\txte \) basis we have

\begin{dmath}\label{eqn:gradQuantumProblemSet8Problem4:220}
H' = 
\frac{A \Hbar^2}{4}
\begin{bmatrix}
U^{-1} \sigma_z U & 0 \\
0 & -U^{-1} \sigma_z U
\end{bmatrix}
-\frac{B_{\textrm{eff}} \Hbar}{2}
\begin{bmatrix}
\sigma_z & 0 \\
0 & \sigma_z
\end{bmatrix}
\end{dmath}

With \( C = \cos(\theta/2), S = \sin(\theta/2) \) these \( U^{-1} \sigma_z U \) block matrices are

\begin{dmath}\label{eqn:gradQuantumProblemSet8Problem4:240}
U^{-1} \sigma_z U 
=
\inv{-C^2 - S^2}
\begin{bmatrix}
C & -S \\
-S & -C
\end{bmatrix}
\begin{bmatrix}
1 & 0 \\
0 & -1
\end{bmatrix}
\begin{bmatrix}
-C & S \\
S & C
\end{bmatrix}
=
\begin{bmatrix}
-C & S \\
S & C
\end{bmatrix}
\begin{bmatrix}
-C & S \\
-S & -C
\end{bmatrix}
=
\begin{bmatrix}
C^2 - S^2 & -2 S C \\
-2 C S & S^2 - C^2
\end{bmatrix}
=
\begin{bmatrix}
\cos\theta & - \sin\theta \\
-\sin\theta & -\cos\theta
\end{bmatrix}.
\end{dmath}

\makeSubAnswer{}{gradQuantum:problemSet8:4b}

TODO.
}
