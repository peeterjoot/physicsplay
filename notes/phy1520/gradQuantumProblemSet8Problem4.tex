%
% Copyright � 2015 Peeter Joot.  All Rights Reserved.
% Licenced as described in the file LICENSE under the root directory of this GIT repository.
%
\makeproblem{Hyperfine levels}{gradQuantum:problemSet8:4}{ 

We can schematically model the hyperfine interaction between the electron and proton spins as \( A \BS_e \cdot \BS_p \) where \( A \) is the hyperfine interaction energy. 

\makesubproblem{}{gradQuantum:problemSet8:4a}
Consider the spin-1/2 proton interacting with a spin-1/2 electron. 
What are the spin eigenstates and eigenvalues? 

\makesubproblem{}{gradQuantum:problemSet8:4b}
Now consider applying a magnetic field which leads to an extra term  

\begin{dmath}\label{eqn:gradQuantumProblemSet8Problem4:20}
-B \lr{ g_e \mu_e S_e^z + g_p \mu_p S_N^z }
\end{dmath}

with gyromagnetic ratios \( g_e \approx -2 \) and \( g_p \approx 5.5 \), with magnetic moments \( \mu_e = e/2m_e \) and
\( \mu_p = e/2m_p \). The large nuclear mass ensures \( \mu_e/\mu_p \sim 2000 \), so let us simply set \( \mu_p = 0\). For convenience, you can club \( B g_e \mu_e \rightarrow B_{\textrm{eff}} \) so the Hamiltonian becomes

\begin{dmath}\label{eqn:gradQuantumProblemSet8Problem4:40}
H = A \BS_e \cdot \BS_p - B_{\textrm{eff}} S_e^z,
\end{dmath}

so the only dimensionless parameter is \( B_{\textrm{eff}}/A \).

Using perturbation theory (degenerate or nondegenerate as appropriate) find how the coupled hyperfine levels split
for weak field \( B_{\textrm{eff}}/A \ll 1 \).
Also consider the strong field limit \( B_{\textrm{eff}}/A \gg 1 \). 
Compute the full field evolution of the levels and compare with the perturbative low field regime result and the high field regime result.

} % makeproblem

\makeanswer{gradQuantum:problemSet8:4}{ 
\makeSubAnswer{}{gradQuantum:problemSet8:4a}
TODO.

\makeSubAnswer{}{gradQuantum:problemSet8:4b}

TODO.
}
