%
% Copyright � 2015 Peeter Joot.  All Rights Reserved.
% Licenced as described in the file LICENSE under the root directory of this GIT repository.
%
\makeproblem{Double well potential}{gradQuantum:problemSet7:1}{ 

Consider a particle in the double well potential

\begin{dmath}\label{eqn:gradQuantumProblemSet7Problem1:20}
V (x) =
\frac{m \omega^2}{ 8 a^2 }
\lr{ x + a }^2 \lr{x - a}^2.
\end{dmath}

Expanding \( V(x) \) around \( x = \pm a \) leads to a harmonic potential with frequency \( \omega \). 
Construct variational states with even/odd parity as \( \psi_\pm(x) = g_\pm \lr{ \phi(x - a) \pm \phi(x + a) } \) where \( \phi(x) \) is the normalized ground state of the usual harmonic oscillator with frequency \( \omega \), i.e.,


\begin{equation}\label{eqn:gradQuantumProblemSet7Problem1:40}
\phi(x) 
= 
\lr{ \inv{ \pi a_0^2 }}^{1/4}
e^{
- \frac{x^2}{ 2 a_0^2 }
}
;
\qquad a_0 = \sqrt{ \frac{\Hbar}{m \omega} }.
\end{equation}

Determine the normalization constants \( g_\pm \). 
Next using these wavefunctions, determine the variational energies of these two states.
Hence determine the `tunnel splitting' between the two states, induced by the tunneling through
the barrier region. 
In your calculations, you can assume \( a \gg a_0 \), so retain only the leading terms in any polynomials you might encounter when you do the integrals.
If we pay attention to these lowest two states (left well and right well) in the full Hilbert space, we can write a phenomenological \( 2 \times 2 \) Hamiltonian

\begin{dmath}\label{eqn:gradQuantumProblemSet7Problem1:60}
H =
\begin{bmatrix}
\epsilon_0 & -\gamma \\
-\gamma & \epsilon_0
\end{bmatrix},
\end{dmath}

where \( \epsilon_0 \) is the energy on each side, and \( \gamma \) leads to tunneling, so if we start off in the left well, \( t \) leads to a nonzero amplitude to find it in the right well at a later time. 
Find its eigenvalues and eigenvectors. 
Comparing with your variational result for the energy splitting, determine the `tunnel coupling' \( \gamma \).

\makesubproblem{}{gradQuantum:problemSet7:1a}
} % makeproblem

\makeanswer{gradQuantum:problemSet7:1}{ 
%\makeSubAnswer{}{gradQuantum:problemSet7:1a}

TODO.
}
