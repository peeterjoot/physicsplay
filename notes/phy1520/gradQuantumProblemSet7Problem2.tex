%
% Copyright � 2015 Peeter Joot.  All Rights Reserved.
% Licenced as described in the file LICENSE under the root directory of this GIT repository.
%
\makeproblem{Helium-4 atom}{gradQuantum:problemSet7:2}{ 

Consider the Helium atom with atomic number \( Z=2 \), which leads to the nuclear charge \( Z=2e \), and two electrons with charge \( -e \) each. 

\makesubproblem{}{gradQuantum:problemSet7:2a}
Show that ignoring electron-electron interactions leads to a ground state energy \( E_{\textrm{He}} = 4 E_\txtH \)
where \( E_\txtH \) is the ground state energy of the hydrogen atom. 

\makesubproblem{}{gradQuantum:problemSet7:2b}
Consider the full problem which retains the Coulomb interaction between the electrons, i.e.

\begin{dmath}\label{eqn:gradQuantumProblemSet7Problem2:20}
H
=
\inv{2m} \lr{ \Bp_1^2 + \Bp_2^2 }
- 
2 e^2
\inv{4 \pi \epsilon_0 }
\lr{ \inv{r_1} + \inv{r_2} }
+ 
e^2
\inv{4 \pi \epsilon_0 }
\inv{ \Abs{\Br_1 - \Br_2} }.
\end{dmath}

and consider the variational wavefunction
\begin{dmath}\label{eqn:gradQuantumProblemSet7Problem2:40}
\psi(\Br_1, \Br_2)
=
N
e^{- \inv{a} \lr{ r_1 + r_2 } }.
\end{dmath}

where \( N \) is the normalization constant, and \( a \) is a variational parameter. Determine the variational ground state energy, and minimize with respect to a to find the best estimate for the ground state energy of Helium. 
Compare with numerical estimates of the energy.
} % makeproblem

\makeanswer{gradQuantum:problemSet7:2}{ 
\makeSubAnswer{}{gradQuantum:problemSet7:2a}

The Laplacian of an exponentially decreasing trial function \( e^{-r/a} \) is

\begin{dmath}\label{eqn:gradQuantumProblemSet7Problem2:340}
\spacegrad^2 e^{-r/a}
=
\inv{r^2} \PD{r}{} \lr{ r^2 \PD{r}{e^{-r/a}} }
=
\inv{r^2} \PD{r}{} \lr{ -\frac{r^2}{a} e^{-r/a} }
=
-\inv{r^2 a} \lr{ 2 r - \frac{r^2}{a} } e^{-r/a} ,
\end{dmath}

%To calculate \( \Bp_j^2 \psi \) first compute the Laplacian of an exponential
%
%\begin{dmath}\label{eqn:gradQuantumProblemSet7Problem2:60}
%\spacegrad^2 e^{\phi}
%=
%\spacegrad \cdot \spacegrad e^\phi
%=
%\spacegrad \cdot \lr{ e^\phi \spacegrad \phi }
%=
%e^\phi \spacegrad^2 \phi + \lr{ \spacegrad \phi }^2 e^\phi
%=
%\lr{ \spacegrad^2 \phi + \lr{ \spacegrad \phi }^2  } e^\phi.
%\end{dmath}
%
%For \( \phi = -r/a \), we have
%
%\begin{dmath}\label{eqn:gradQuantumProblemSet7Problem2:80}
%\spacegrad \phi 
%=
%- \inv{a} \spacegrad \sqrt{ \Bx^2 }
%=
%- \inv{a} \spacegrad \sqrt{ x_k x_k }
%=
%- \inv{a} \inv{2 r} \Be_j ( 2 \partial_j x_k ) x_k 
%=
%- \frac{\Bx}{a r},
%\end{dmath}
%
%and
%\begin{dmath}\label{eqn:gradQuantumProblemSet7Problem2:100}
%\spacegrad^2 \phi
%=
%-\inv{a} \spacegrad \cdot \frac{ \Bx}{ r}
%=
%-\inv{a} \lr{
%\inv{r} \spacegrad \cdot \Bx
%+
%\Bx \cdot \spacegrad \inv{ r }
%}
%=
%-\inv{a} \lr{
%\frac{3}{r}
%+
%\Bx \cdot \lr{ -\inv{r^3} \Bx }
%}
%=
%-\inv{a} \lr{
%\frac{3}{r}
%-
%\frac{1}{r}
%}
%=
%-\frac{2}{a r}.
%\end{dmath}
%
or

\begin{dmath}\label{eqn:gradQuantumProblemSet7Problem2:120}
\spacegrad^2 e^{-r/a} = \inv{a} \lr{ \inv{a} -\frac{2}{r} } e^{-r/a}.
\end{dmath}

For the Hydrogen atom (with \( a = a_0 \)), the Hamiltonian action on the unnormalized ground state wavefunction \( \psi = e^{-r/a} \) is

\begin{dmath}\label{eqn:gradQuantumProblemSet7Problem2:380}
H \psi(r)
=
\frac{\Bp^2}{2m} \psi
- e^2 \inv{4 \pi \epsilon_0 } \inv{r} \psi
=
-\frac{\Hbar^2}{2m} \inv{a} \lr{ \inv{a} -\frac{2}{r} } \psi
- e^2 \inv{4 \pi \epsilon_0 } \inv{r} \psi
=
\lr{ -\frac{\Hbar^2}{2m} \inv{a^2} + \lr{ \frac{\Hbar^2}{m a} - e^2 \inv{4 \pi \epsilon_0 } \inv{r} } } e^{-r/a}.
\end{dmath}

The hydrogen ground state energy is
\begin{dmath}\label{eqn:gradQuantumProblemSet7Problem2:160}
E_\txtH
%= 
%\frac{
%   \bra{ \psi } 
%   \lr{ -\frac{\Hbar^2}{2m a} \lr{ \inv{a} -\frac{2}{ r} } - e^2 \inv{4 \pi \epsilon_0 r} }
%   \ket{ \psi }
%}
%{ \braket{ \psi }{ \psi } }
=
   \lr{ \frac{\Hbar^2}{m a} - e^2 \inv{4 \pi \epsilon_0 } } 
\frac{
\bra{ \psi } \inv{r} \ket{ \psi }
}
{ 
   \braket{ \psi }{ \psi } 
}
   - \frac{\Hbar^2}{2m a^2} .
\end{dmath}

For Helium without electron-electron interaction the kinetic portion of the Hamiltonian action on this trial function \( \psi = e^{-(r_1 + r_2)/a} \) is

\begin{dmath}\label{eqn:gradQuantumProblemSet7Problem2:140}
H \psi(r_1, r_2)
=
\frac{\Bp_1^2}{2m} \psi
+
\frac{\Bp_2^2}{2m} \psi
- 2 e^2 \inv{4 \pi \epsilon_0 } \lr{ \inv{r_1} + \inv{r_2} } \psi
=
-\frac{\Hbar^2}{2m a} \lr{ \frac{2}{a} -\frac{2}{ r_1}  -\frac{2}{ r_2} } \psi
- 2 e^2 \inv{4 \pi \epsilon_0 } \lr{ \inv{r_1} + \inv{r_2} } \psi
=
\lr{ -\frac{\Hbar^2}{m a^2}
+
\lr{ \frac{\Hbar^2}{m a} - \frac{e^2}{2 \pi \epsilon_0} } \lr{ \inv{r_1} + \inv{r_2} }
} 
e^{-(r_1 + r_2)/a}.
\end{dmath}

Now, assuming that \( \psi = e^{-(r_1 + r_2)/a} \) is the unnormalized ground state wavefunction for the Helium atom without electron-electron interaction, that ground state energy is given by

\begin{dmath}\label{eqn:gradQuantumProblemSet7Problem2:260}
E_{\textrm{He}}
%= 
%\frac{
%   \bra{ \psi } 
%   \lr{ 
%      -\frac{\Hbar^2}{m a} \lr{ \inv{a} -\frac{1}{ r_1}  -\frac{1}{ r_2} } 
%      - 2 e^2 \inv{4 \pi \epsilon_0 } \lr{ \inv{r_1} + \inv{r_2} } 
%   }
%   \ket{ \psi }
%}
%{ \braket{ \psi }{ \psi } }
=
   \lr{ \frac{\Hbar^2}{m a} - e^2 \inv{2 \pi \epsilon_0 } } 
\frac{
\bra{ \psi } \inv{r_1} + \inv{r_2} \ket{ \psi }
}
{ 
   \braket{ \psi }{ \psi } 
}
   - \frac{\Hbar^2}{m a^2} .
\end{dmath}

\paragraph{Calculating Hydrogen ground state energy}

For the normalization factor we have

\begin{dmath}\label{eqn:gradQuantumProblemSet7Problem2:180}
\begin{aligned}
\braket{ \psi }{ \psi }
&=
4 \pi \int_0^\infty r^2 dr e^{-2 r/a} \\
&=
{4 \pi}\frac{a^3}{2^3} \int_0^\infty r^2 dr e^{-r} \\
&=
\inv{2} { \pi}{a^3} \int_0^\infty 2r dr e^{-r} \\
&=
{\pi}{a^3} \int_0^\infty dr e^{-r} \\
&=
{\pi}{a^3},
\end{aligned}
\end{dmath}

and for the inverse radial expectation we have

\begin{dmath}\label{eqn:gradQuantumProblemSet7Problem2:200}
\bra{ \psi } \inv{r} \ket{ \psi }
=
4 \pi \int_0^\infty r dr e^{-2 r/a}
=
{4 \pi}\frac{a^2}{2^2} \int_0^\infty r dr e^{-r}
=
{\pi}{a^2},
\end{dmath}

so

\begin{dmath}\label{eqn:gradQuantumProblemSet7Problem2:220}
E_\txtH 
= 
\lr{ \frac{\Hbar^2}{m a} - e^2 \inv{4 \pi \epsilon_0 } } \frac{\pi a^2}{\pi a^3}
   - \frac{\Hbar^2}{2m a^2} ,
\end{dmath}

which simplifies to

%\begin{dmath}\label{eqn:gradQuantumProblemSet7Problem2:240}
\boxedEquation{eqn:gradQuantumProblemSet7Problem2:240}{
E_\txtH 
= 
\frac{\Hbar^2}{2 m a^2} - \frac{e^2}{4 \pi \epsilon_0 a }.
}
%\end{dmath}

\paragraph{Calculating Helium ground state energy}

This time the normalization is given by

\begin{dmath}\label{eqn:gradQuantumProblemSet7Problem2:280}
\begin{aligned}
\braket{ \psi }{ \psi } 
&= 
\lr{ 4 \pi}^2 
\int_0^\infty r_1^2 dr_1 e^{-2 r_1/a}
\int_0^\infty r_2^2 dr_2 e^{-2 r_2/a} \\
&=
\lr{ 4 \pi}^2 \lr{ \frac{a}{2} }^6
\lr{ \int_0^\infty r^2 dr e^{-r} }^2 \\
&=
2^{4 - 6 + 2}
\pi^2 a^6 \\
&= \pi^2 a^6.
\end{aligned}
\end{dmath}

We also need the inverse radial expectations.  Calculating the expectation of \( 1/r_1 \) is sufficient, and is

\begin{dmath}\label{eqn:gradQuantumProblemSet7Problem2:300}
\bra{ \psi } \inv{r_1} \ket{ \psi } 
= 
\lr{ 4 \pi}^2 
\int_0^\infty r_1 dr_1 e^{-2 r_1/a}
\int_0^\infty r_2^2 dr_2 e^{-2 r_2/a}
=
\lr{ 4 \pi}^2 \lr{ \frac{a}{2} }^5
\lr{ \int_0^\infty r dr e^{-r} }
\lr{ \int_0^\infty r^2 dr e^{-r} }
=
2^{4 - 5 + 1}
\pi^2 a^5
= \pi^2 a^5.
\end{dmath}

This gives

\begin{dmath}\label{eqn:gradQuantumProblemSet7Problem2:320}
E_{\textrm{He}}
=
   \lr{ \frac{\Hbar^2}{m a} - e^2 \inv{2 \pi \epsilon_0 } } 
\frac{ 2 \pi^2 a^5 }
{ 
   \pi^2 a^6
}
   - \frac{\Hbar^2}{m a^2},
\end{dmath}

or
\boxedEquation{eqn:gradQuantumProblemSet7Problem2:400}{
E_{\textrm{He}}
=
   \frac{\Hbar^2}{m a^2} - e^2 \inv{\pi \epsilon_0 a }.
}

If the zero point energies for Hydrogen and Helium are set to \( \ifrac{\Hbar^2}{2 m a^2} \) and \( \ifrac{\Hbar^2}{m a^2} \) respectively, then we have \( E_{\textrm{He}} = 4 E_\txtH\) as expected.

%Comparing to \cref{eqn:gradQuantumProblemSet7Problem2:240}, we have
%
%%\begin{equation}\label{eqn:gradQuantumProblemSet7Problem2:360}
%\boxedEquation{eqn:gradQuantumProblemSet7Problem2:360}{
%E_{\textrm{He}} = 4 \lr{ 
%\frac{\Hbar^2}{2 m a^2} - \frac{e^2}{4 \pi \epsilon_0 a }
%}
%= 4 E_\txtH.
%}
%%\end{equation}

\makeSubAnswer{}{gradQuantum:problemSet7:2b}

To evaluate the inverse radial interaction term, a Fourier transform representation of that inverse interaction distance can be employed

\begin{dmath}\label{eqn:gradQuantumProblemSet7Problem2:420}
\inv{\Abs{\Br}} 
= \lim_{\epsilon \rightarrow 0} \inv{2 \pi^2} \int d^3 \frac{e^{i \Bk \cdot \Br}}{\Bk^2 + \epsilon^2},
\end{dmath}

A demonstration of this identity and the contour used to evaluate this integral can be found in \citep{byron1992mca}.  Using this we have

\begin{dmath}\label{eqn:gradQuantumProblemSet7Problem2:440}
\frac{e^2}{4 \pi \epsilon_0} \bra{\psi} \inv{\Abs{\Br_1 - \Br_2}} \ket{\psi}
=
\frac{e^2}{4 \pi \epsilon_0}
(2\pi)^2
\int dr_1 d\theta_1 r_1^2 \sin(\theta_1)
\int dr_2 d\theta_2 r_2^2 \sin(\theta_2)
e^{ -(r_1 + r_2)/a}
\inv{2 \pi^2} 
\int d^3 k \frac{e^{i \Bk \cdot \lr{ \Br_1 - \Br_2} }}{\Bk^2}
=
\frac{e^2}{2 \pi \epsilon_0}
\int d^3 k \inv{\Bk^2}
\int dr_1 d\theta_1 r_1^2 \sin(\theta_1) e^{-r_1/a + i \Bk \cdot \Br_1}
\int dr_2 d\theta_2 r_2^2 \sin(\theta_2) e^{-r_2/a - i \Bk \cdot \Br_2}
\end{dmath}

The spatial domain integrals can now be evaluated separately.  With a coordinate system picked so that \( \Bk = \pm k \zcap \), that gives

\begin{dmath}\label{eqn:gradQuantumProblemSet7Problem2:460}
\begin{aligned}
\int dr d\theta r^2 \sin(\theta) e^{-r/a + i \Bk \cdot \Br}
&=
\int_0^\infty dr r^2 e^{-r/a}
\int_0^\pi d\theta
\frac{d}{d\theta} (-\cos(\theta)) 
e^{\pm i k r \cos\theta} \\
&=
\int_0^\infty dr r^2 e^{-r/a}
\int_{-1}^1 du
e^{\mp i k r u} \\
&=
\int_0^\infty dr r^2 e^{-r/a}
\frac{e^{\mp i k r} - e^{\pm i k r} }{ \mp i k r } \\
&=
\frac{2}{k} \int_0^\infty dr r e^{-r/a} \sin( k r ) \\
&=
\frac{2}{k} \int_0^\infty dr r e^{-r/a} \sin( k r ) \\
&= 
\frac{4 a^3}{\lr{1 + a^2 k^2 }^2}.
\end{aligned}
\end{dmath}

The specific orientation used to evaluate the integral no longer matters, and we have

\begin{dmath}\label{eqn:gradQuantumProblemSet7Problem2:480}
\frac{e^2}{4 \pi \epsilon_0} \bra{\psi} \inv{\Abs{\Br_1 - \Br_2}} \ket{\psi}
=
\frac{e^2}{2 \pi \epsilon_0}
\int d^3 k \inv{\Bk^2}
\frac{4^2 a^6}{\lr{1 + a^2 k^2 }^4}
=
\frac{8 e^2 a^6}{\pi \epsilon_0}
(4\pi) 
\int dk k^2 \inv{\Bk^2}
\inv{\lr{1 + a^2 k^2 }^4}
=
\frac{32 e^2 a^6}{\epsilon_0}
\int dk 
\inv{\lr{1 + a^2 k^2 }^4}
=
\frac{32 e^2 a^6}{\epsilon_0}
\frac{5 \pi}{32 a}
=
\frac{5 e^2 a^5 \pi}{\epsilon_0},
\end{dmath}

and finally

\begin{dmath}\label{eqn:gradQuantumProblemSet7Problem2:500}
\frac{e^2}{4 \pi \epsilon_0} \bra{\psi} \inv{\Abs{\Br_1 - \Br_2}} \ket{\psi}/\braket{\psi}{\psi}
=
\frac{5 e^2 a^5 \pi}{\epsilon_0 } \inv{\pi^2 a^6}
=
\frac{5 e^2 }{\epsilon_0 \pi a}.
\end{dmath}

With the electron-electron interaction, the total energy for the Helium ground state is

\begin{dmath}\label{eqn:gradQuantumProblemSet7Problem2:520}
E_{\textrm{He}}
= 
\frac{\Hbar^2}{m a^2} + 4 e^2 \inv{\pi \epsilon_0 a }.
\end{dmath}

For the minimum we want to solve

\begin{dmath}\label{eqn:gradQuantumProblemSet7Problem2:540}
0
= 
\PD{a}{E}
= 
-2 \frac{\Hbar^2}{m a^3} - 4 e^2 \inv{\pi \epsilon_0 a^2 },
\end{dmath}

which has the minimum at

\begin{dmath}\label{eqn:gradQuantumProblemSet7Problem2:560}
a = - \frac{\Hbar^2 \pi \epsilon_0}{2m e^2}.
\end{dmath}

After substitution, the ground state energy at this value of \( a \) is

%\begin{dmath}\label{eqn:gradQuantumProblemSet7Problem2:580}
\boxedEquation{eqn:gradQuantumProblemSet7Problem2:600}{
E_{\textrm{He}}
= 
-\frac{4 m e^4}{\Hbar^2 \pi^2 \epsilon_0^2}.
}
%\end{dmath}

}
