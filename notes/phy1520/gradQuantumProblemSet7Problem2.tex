%
% Copyright � 2015 Peeter Joot.  All Rights Reserved.
% Licenced as described in the file LICENSE under the root directory of this GIT repository.
%
\makeproblem{Helium-4 atom}{gradQuantum:problemSet7:2}{ 

Consider the Helium atom with atomic number \( Z=2 \), which leads to the nuclear charge \( Z=2e \), and two electrons with charge \( -e \) each. 

\makesubproblem{}{gradQuantum:problemSet7:2a}
Show that ignoring electron-electron interactions leads to a ground state energy \( E_{\textrm{He}} = 4 E_\txtH \)
where \( E_\txtH \) is the ground state energy of the hydrogen atom. 

\makesubproblem{}{gradQuantum:problemSet7:2b}
Consider the full problem which retains the Coulomb interaction between the electrons, i.e.

\begin{dmath}\label{eqn:gradQuantumProblemSet7Problem2:20}
H
=
\inv{2m} \lr{ \Bp_1^2 + \Bp_2^2 }
- 
2 e^2
\inv{4 \pi \epsilon_0 }
\lr{ \inv{r_1} + \inv{r_2} }
+ 
e^2
\inv{4 \pi \epsilon_0 }
\inv{ \Abs{\Br_1 - \Br_2} }.
\end{dmath}

and consider the variational wavefunction
\begin{dmath}\label{eqn:gradQuantumProblemSet7Problem2:40}
\psi(\Br_1, \Br_2)
=
N
e^{- \inv{a} \lr{ r_1 + r_2 } }.
\end{dmath}

where \( N \) is the normalization constant, and \( a \) is a variational parameter. Determine the variational ground state energy, and minimize with respect to a to find the best estimate for the ground state energy of Helium. 
Compare with numerical estimates of the energy.
} % makeproblem

\makeanswer{gradQuantum:problemSet7:2}{ 
\makeSubAnswer{}{gradQuantum:problemSet7:2a}

To calculate \( \Bp_j^2 \psi \) first compute the Laplacian of an exponential

\begin{dmath}\label{eqn:gradQuantumProblemSet7Problem2:60}
\spacegrad^2 e^{\phi}
=
\spacegrad \cdot \spacegrad e^\phi
=
\spacegrad \cdot \lr{ e^\phi \spacegrad \phi }
=
e^\phi \spacegrad^2 \phi + \lr{ \spacegrad \phi }^2 e^\phi
=
\lr{ \spacegrad^2 \phi + \lr{ \spacegrad \phi }^2  } e^\phi.
\end{dmath}

For \( \phi = -r/a \), we have

\begin{dmath}\label{eqn:gradQuantumProblemSet7Problem2:80}
\spacegrad \phi 
=
- \inv{a} \spacegrad \sqrt{ \Bx^2 }
=
- \inv{a} \spacegrad \sqrt{ x_k x_k }
=
- \inv{a} \inv{2 r} \Be_j ( 2 \partial_j x_k ) x_k 
=
- \frac{\Bx}{a r},
\end{dmath}

and
\begin{dmath}\label{eqn:gradQuantumProblemSet7Problem2:100}
\spacegrad^2 \phi
=
-\inv{a} \spacegrad \cdot \frac{ \Bx}{ r}
=
-\inv{a} \lr{
\inv{r} \spacegrad \cdot \Bx
+
\Bx \cdot \spacegrad \inv{ r }
}
=
-\inv{a} \lr{
\frac{3}{r}
+
\Bx \cdot \lr{ -\inv{r^3} \Bx }
}
=
-\inv{a} \lr{
\frac{3}{r}
-
\frac{1}{r}
}
=
-\frac{2}{a r}.
\end{dmath}

This gives

\begin{dmath}\label{eqn:gradQuantumProblemSet7Problem2:120}
\spacegrad^2 e^{-r/a} = \inv{a} \lr{ \inv{a} -\frac{2}{r} } e^{-r/a},
\end{dmath}

so the kinetic portion of the Hamiltonian action on this trial function is

\begin{dmath}\label{eqn:gradQuantumProblemSet7Problem2:140}
\frac{\Bp_1^2}{2m} \psi
+
\frac{\Bp_2^2}{2m} \psi
=
-\frac{\Hbar^2}{2m a} \lr{ \frac{2}{a} -\frac{2}{ r_1}  -\frac{2}{ r_2} } e^{-(r_1 + r_2)/a}
=
\frac{\Hbar^2}{m a} \lr{ \frac{1}{r_1} + \frac{1}{r_2} - \inv{a} } e^{-(r_1 + r_2)/a}.
\end{dmath}

With \( a = a_0 \), the wavefunction \( \psi = e^{-r/a} \) is the unnormalized eigenfunction for the Hydrogen ground state, which means that the Hydrogen ground state energy is given by

\begin{dmath}\label{eqn:gradQuantumProblemSet7Problem2:160}
E_\txtH
= 
\frac{
   \bra{ \psi } 
   \lr{ -\frac{\Hbar^2}{2m a} \lr{ \inv{a} -\frac{2}{ r} } - e^2 \inv{4 \pi \epsilon_0 r} }
   \ket{ \psi }
}
{ \braket{ \psi }{ \psi } }
=
   \lr{ \frac{\Hbar^2}{m a} - e^2 \inv{4 \pi \epsilon_0 } } 
\frac{
\bra{ \psi } \inv{r} \ket{ \psi }
}
{ 
   \braket{ \psi }{ \psi } 
}
   - \frac{\Hbar^2}{2m a^2} .
\end{dmath}

For the normalization factor we have

\begin{dmath}\label{eqn:gradQuantumProblemSet7Problem2:180}
\begin{aligned}
\braket{ \psi }{ \psi }
&=
4 \pi \int_0^\infty r^2 dr e^{-2 r/a} \\
&=
{4 \pi}\frac{a^3}{2^3} \int_0^\infty r^2 dr e^{-r} \\
&=
\inv{2} { \pi}{a^3} \int_0^\infty 2r dr e^{-r} \\
&=
{\pi}{a^3} \int_0^\infty dr e^{-r} \\
&=
{\pi}{a^3},
\end{aligned}
\end{dmath}

and for the inverse radial expectation we have

\begin{dmath}\label{eqn:gradQuantumProblemSet7Problem2:200}
\bra{ \psi } \inv{r} \ket{ \psi }
=
4 \pi \int_0^\infty r dr e^{-2 r/a}
=
{4 \pi}\frac{a^2}{2^2} \int_0^\infty r dr e^{-r}
=
{\pi}{a^2},
\end{dmath}

so

\begin{dmath}\label{eqn:gradQuantumProblemSet7Problem2:220}
E_\txtH 
= 
\lr{ \frac{\Hbar^2}{m a} - e^2 \inv{4 \pi \epsilon_0 } } \frac{\pi a^2}{\pi a^3}
   - \frac{\Hbar^2}{2m a^2} ,
\end{dmath}

which simplifies to

\begin{dmath}\label{eqn:gradQuantumProblemSet7Problem2:240}
E_\txtH 
= 
\frac{\Hbar^2}{2 m a^2} - \frac{e^2}{4 \pi \epsilon_0 a }.
\end{dmath}

Now, assuming that \( \psi = e^{-(r_1 + r_2)/a} \) is the unnormalized ground state wavefunction for the Helium atom without electron-electron interaction, that ground state energy is given by

\begin{dmath}\label{eqn:gradQuantumProblemSet7Problem2:260}
E_{\textrm{He}}
= 
\frac{
   \bra{ \psi } 
   \lr{ 
      -\frac{\Hbar^2}{m a} \lr{ \inv{a} -\frac{1}{ r_1}  -\frac{1}{ r_2} } 
      - 2 e^2 \inv{4 \pi \epsilon_0 } \lr{ \inv{r_1} + \inv{r_2} } 
   }
   \ket{ \psi }
}
{ \braket{ \psi }{ \psi } }
=
   \lr{ \frac{\Hbar^2}{m a} - e^2 \inv{2 \pi \epsilon_0 } } 
\frac{
\bra{ \psi } \inv{r_1} + \inv{r_2} \ket{ \psi }
}
{ 
   \braket{ \psi }{ \psi } 
}
   - \frac{\Hbar^2}{m a^2} .
\end{dmath}

This time the normalization is given by

\begin{dmath}\label{eqn:gradQuantumProblemSet7Problem2:280}
\begin{aligned}
\braket{ \psi }{ \psi } 
&= 
\lr{ 4 \pi}^2 
\int_0^\infty r_1^2 dr_1 e^{-2 r_1/a}
\int_0^\infty r_2^2 dr_2 e^{-2 r_2/a} \\
&=
\lr{ 4 \pi}^2 \lr{ \frac{a}{2} }^6
\lr{ \int_0^\infty r^2 dr e^{-r} }^2 \\
&=
2^{4 - 6 + 2}
\pi^2 a^6 \\
&= \pi^2 a^6.
\end{aligned}
\end{dmath}

We also need the inverse radial expectations, say, that of \( 1/r_1 \), which is

\begin{dmath}\label{eqn:gradQuantumProblemSet7Problem2:300}
\bra{ \psi } \inv{r_1} \ket{ \psi } 
= 
\lr{ 4 \pi}^2 
\int_0^\infty r_1 dr_1 e^{-2 r_1/a}
\int_0^\infty r_2^2 dr_2 e^{-2 r_2/a}
=
\lr{ 4 \pi}^2 \lr{ \frac{a}{2} }^5
\lr{ \int_0^\infty r dr e^{-r} }
\lr{ \int_0^\infty r^2 dr e^{-r} }
=
2^{4 - 5 + 1}
\pi^2 a^5
= \pi^2 a^5.
\end{dmath}

This gives

\begin{dmath}\label{eqn:gradQuantumProblemSet7Problem2:320}
E_{\textrm{He}}
=
   \lr{ \frac{\Hbar^2}{m a} - e^2 \inv{2 \pi \epsilon_0 } } 
\frac{ 2 \pi^2 a^5 }
{ 
   \pi^2 a^6
}
   - \frac{\Hbar^2}{m a^2} 
=
   \frac{\Hbar^2}{m a^2} - e^2 \inv{\pi \epsilon_0 a },
\end{dmath}

%or

\makeSubAnswer{}{gradQuantum:problemSet7:2b}
TODO.

}
