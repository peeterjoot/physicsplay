%
% Copyright © 2015 Peeter Joot.  All Rights Reserved.
% Licenced as described in the file LICENSE under the root directory of this GIT repository.
%

\makeproblem{Calculate the right going diagonalization}{problem:qmLecture9:1}{

Prove \cref{eqn:qmLecture9:160}.
} % problem

\makeanswer{problem:qmLecture9:1}{

To determine the relations for \( \theta_k \) we have to solve 

\begin{dmath}\label{eqn:qmLecture9:280}
\begin{bmatrix}
E_k & 0 \\
0 & -E_k
\end{bmatrix}
= R^{-1} H R.
\end{dmath}

Working with \( \Hbar = c = 1 \) temporarily, and \( C = \cos\theta_k, S = \sin\theta_k \), that is

\begin{dmath}\label{eqn:qmLecture9:300}
\begin{bmatrix}
E_k & 0 \\
0 & -E_k
\end{bmatrix}
=
\begin{bmatrix}
C & S \\
-S & C
\end{bmatrix}
\begin{bmatrix}
k & m \\
m & -k
\end{bmatrix}
\begin{bmatrix}
C & -S \\
S & C
\end{bmatrix}
=
\begin{bmatrix}
C & S \\
-S & C
\end{bmatrix}
\begin{bmatrix}
k C + m S & -k S + m C \\
m C - k S & -m S - k C
\end{bmatrix}
=
\begin{bmatrix}
k C^2 + m S C + m C S - k S^2   & -k S C + m C^2 -m S^2 - k C S \\
-k C S - m S^2 + m C^2 - k S C & k S^2 - m C S -m S C - k C^2
\end{bmatrix}
=
\begin{bmatrix}
k \cos(2 \theta_k) + m \sin(2 \theta_k) & m \cos(2 \theta_k) - k \sin(2 \theta_k) \\
m \cos(2 \theta_k) - k \sin(2 \theta_k) & -k \cos(2 \theta_k) - m \sin(2 \theta_k) \\
\end{bmatrix}.
\end{dmath}

This gives

\begin{dmath}\label{eqn:qmLecture9:320}
E_k 
\begin{bmatrix}
1 \\
0
\end{bmatrix}
=
\begin{bmatrix}
k \cos(2 \theta_k) + m \sin(2 \theta_k) \\
m \cos(2 \theta_k) - k \sin(2 \theta_k) \\
\end{bmatrix}
=
\begin{bmatrix}
k & m \\
m & -k
\end{bmatrix}
\begin{bmatrix}
\cos(2 \theta_k) \\
\sin(2 \theta_k) \\
\end{bmatrix}.
\end{dmath}

Adding back in the \(\Hbar\)'s and \(c\)'s this is

\begin{dmath}\label{eqn:qmLecture9:340}
\begin{bmatrix}
\cos(2 \theta_k) \\
\sin(2 \theta_k) \\
\end{bmatrix}
=
\frac{E_k}{-(\Hbar k c)^2 -(m c^2)^2}
\begin{bmatrix}
- \Hbar k c & - m c^2 \\
- m c^2     & \Hbar k c
\end{bmatrix}
\begin{bmatrix}
1 \\
0
\end{bmatrix}
=
\inv{E_k}
\begin{bmatrix}
\Hbar k c \\
m c^2
\end{bmatrix}.
\end{dmath}
} % answer

\makeproblem{Verify the Dirac current relationship.}{problem:qmLecture9:2}{
Prove \cref{eqn:qmLecture9:240}.
} % problem

\makeanswer{problem:qmLecture9:2}{

The components of the Schr\"{o}dinger equation are

\begin{equation}\label{eqn:qmLecture9:360}
\begin{aligned}
-i \Hbar \PD{t}{\psi_1} &= -i \Hbar c \PD{x}{\psi_1} + m c^2 \psi_2  \\
-i \Hbar \PD{t}{\psi_2} &= m c^2 \psi_1 + i \Hbar c \PD{x}{\psi_2},
\end{aligned}
\end{equation}

The conjugates of these are
\begin{equation}\label{eqn:qmLecture9:380}
\begin{aligned}
i \Hbar \PD{t}{\psi_1^\conj} &= i \Hbar c \PD{x}{\psi_1^\conj} + m c^2 \psi_2^\conj \\
i \Hbar \PD{t}{\psi_2^\conj} &= m c^2 \psi_1^\conj - i \Hbar c \PD{x}{\psi_2^\conj}.
\end{aligned}
\end{equation}

This gives
\begin{dmath}\label{eqn:qmLecture9:400}
\begin{aligned}
i \Hbar \PD{t}{\rho}
&=
\lr{ i \Hbar c \PD{x}{\psi_1^\conj} + m c^2 \psi_2^\conj } \psi_1 \\
&+ \psi_1^\conj \lr{ i \Hbar c \PD{x}{\psi_1} - m c^2 \psi_2 } \\
&+ \lr{ m c^2 \psi_1^\conj - i \Hbar c \PD{x}{\psi_2^\conj} } \psi_2 \\
&+ \psi_2^\conj \lr{ -m c^2 \psi_1 - i \Hbar c \PD{x}{\psi_2} }.
\end{aligned}
\end{dmath}

All the non-derivative terms cancel leaving

\begin{dmath}\label{eqn:qmLecture9:420}
\inv{c} \PD{t}{\rho} 
=
\PD{x}{\psi_1^\conj} \psi_1 
+ \psi_1^\conj \PD{x}{\psi_1}
- \PD{x}{\psi_2^\conj} \psi_2 
- \psi_2^\conj \PD{x}{\psi_2} 
=
\PD{x}{} 
\lr{
\psi_1^\conj \psi_1 -
\psi_2^\conj \psi_2 
}.
\end{dmath}

} % answer
 
