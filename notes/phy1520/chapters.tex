%
% Copyright � 2015 Peeter Joot.  All Rights Reserved.
% Licenced as described in the file LICENSE under the root directory of this GIT repository.
%
%----------------------------------------------------------------------------------------
\part{Reading and Lecture Notes}
   \chapter{Fundamental concepts}
      \section{Lighting review}
         %
% Copyright � 2015 Peeter Joot.  All Rights Reserved.
% Licenced as described in the file LICENSE under the root directory of this GIT repository.
%
\newcommand{\authorname}{Peeter Joot}
\newcommand{\email}{peeterjoot@protonmail.com}
\newcommand{\basename}{FIXMEbasenameUndefined}
\newcommand{\dirname}{notes/FIXMEdirnameUndefined/}

\renewcommand{\basename}{gmLecture1}
\renewcommand{\dirname}{notes/phy1520/}
%\newcommand{\dateintitle}{}
%\newcommand{\keywords}{}

\newcommand{\authorname}{Peeter Joot}
\newcommand{\onlineurl}{http://sites.google.com/site/peeterjoot2/math2013/\basename.pdf}
\newcommand{\sourcepath}{\dirname\basename.tex}
\newcommand{\generatetitle}[1]{\chapter{#1}}

\newcommand{\vcsinfo}{%
\section*{}
\noindent{\color{DarkOliveGreen}{\rule{\linewidth}{0.1mm}}}
\paragraph{Document version}
%\paragraph{\color{Maroon}{Document version}}
{
\small
\begin{itemize}
\item Available online at:\\ 
\href{\onlineurl}{\onlineurl}
\item Git Repository: \input{./.revinfo/gitRepo.tex}
\item Source: \sourcepath
\item last commit: \input{./.revinfo/gitCommitString.tex}
\item commit date: \input{./.revinfo/gitCommitDate.tex}
\end{itemize}
}
}

%\PassOptionsToPackage{dvipsnames,svgnames}{xcolor}
\PassOptionsToPackage{square,numbers}{natbib}
\documentclass{scrreprt}

\usepackage[left=2cm,right=2cm]{geometry}
\usepackage[svgnames]{xcolor}
\usepackage{peeters_layout}

\usepackage{natbib}

\usepackage[
colorlinks=true,
bookmarks=false,
pdfauthor={\authorname, \email},
backref 
]{hyperref}

% http://tex.stackexchange.com/questions/75773/how-to-reference-problems-by-the-text-label-in-an-exercise-envioronment
\usepackage[english]{cleveref}
\crefname{Exercise}{exercise}{exercises}
\Crefname{Exercise}{Exercise}{Exercises}

\RequirePackage{titlesec}
\RequirePackage{ifthen}

% http://stackoverflow.com/questions/4932910/date-in-the-tabular-environment
\makeatletter
\let\insertdate\@date
\makeatother

\titleformat{\chapter}[display]
{\bfseries\Large}
{\color{DarkSlateGrey}\filleft \authorname
\ifthenelse{\isundefined{\studentnumber}}{}{\\ \studentnumber}
\ifthenelse{\isundefined{\email}}{}{\\ \email}
\ifthenelse{\isundefined{\dateintitle}}{}{\\ \insertdate}
%\ifthenelse{\isundefined{\coursename}}{}{\\ \coursename} % put in title instead.
}
{4ex}
{\color{DarkOliveGreen}{\titlerule}\color{Maroon}
\vspace{2ex}%
\filright}
[\vspace{2ex}%
\color{DarkOliveGreen}\titlerule
]

\newcommand{\beginArtWithToc}[0]{\begin{document}\tableofcontents}
\newcommand{\beginArtNoToc}[0]{\begin{document}}
\newcommand{\EndNoBibArticle}[0]{\end{document}}
\newcommand{\EndArticle}[0]{\bibliography{Bibliography}\bibliographystyle{plainnat}\end{document}}

% 
%\newcommand{\citep}[1]{\cite{#1}}

\colorSectionsForArticle



\usepackage{peeters_layout_exercise}
\usepackage{peeters_braket}
\usepackage{peeters_figures}

\beginArtNoToc

\generatetitle{Introduction}
%\chapter{Introduction}
%\label{chap:gmLecture1}

Text \citep{sakurai2014modern} (revised edition).

\paragraph{Classical mechanics}

We'll be talking about one body physics for most of this course.  In classical mechanics we can figure out the particle trajectories using both of \( (\Br, \Bp \), where

\begin{dmath}\label{eqn:qmLecture1:20}
\begin{aligned}
\ddt{\Br} &= \inv{m} \Bp \\
\ddt{\Bp} &= \spacegrad V
\end{aligned}
\end{dmath}

A two dimensional phase space shows the trajectory of a point particle subject to some equations of motion

\paragraph{Quantum mechanics}

%\begin{dmath}\label{eqn:qmLecture1:n}
\boxedEquation{eqn:qmLecture1:n}{
\Hbar = 1.
}
%\end{dmath}

In QM we are no longer allowed to think of position and momentum, but have to start asking about state vectors \( \ket{\Psi} \).

We'll consider the state vector with respect to some basis, for example, in a position basis, we write

\begin{dmath}\label{eqn:qmLecture1:40}
\braket{ x }{\Psi } = \Psi(x),
\end{dmath}

a complex numbered ``wave function'', the probability amplitude for a particle in \( \ket{\Psi} \) to be in the vincinity of \( x \).

We could also consider the state in a momentum basis

\begin{dmath}\label{eqn:qmLecture1:60}
\braket{ p }{\Psi } = \Psi(p),
\end{dmath}

a probability amplitude with respect to momentum \( p \).

More precisely, 

\begin{dmath}\label{eqn:qmLecture1:80}
\Abs{\Psi(x)}^2 dx \ge 0
\end{dmath}

is the probability of finding the particle in the range \( (x, x + dx ) \).  To have meaning as a probability, we require

\begin{dmath}\label{eqn:qmLecture1:100}
\int_{-\infty}^\infty \Abs{\Psi(x)}^2 dx = 1.
\end{dmath}

The average position can be calculated using this PDF.  For example

\begin{dmath}\label{eqn:qmLecture1:120}
\expectation{x} = \int_{-\infty}^\infty \Abs{\Psi(x)}^2 x dx,
\end{dmath}

or 
\begin{dmath}\label{eqn:qmLecture1:140}
\expectation{f(x)} = \int_{-\infty}^\infty \Abs{\Psi(x)}^2 f(x) dx.
\end{dmath}

Similarily, calculation of an average of a function of momentum can be expressed as

\begin{dmath}\label{eqn:qmLecture1:160}
\expectation{f(p)} = \int_{-\infty}^\infty \Abs{\Psi(p)}^2 f(p) dp.
\end{dmath}

\paragraph{Transformation from a position to momentum basis}

We have a problem, if we which to compute an average in momentum space such as \( \expectation{p} \), when given a wavefunction \( \Psi(x) \).

How do we convert 

\begin{dmath}\label{eqn:qmLecture1:180}
\Psi(p) 
\Psi(x),
\stackrel{?}{\leftrightarrow}
\end{dmath}

or equivalently
\begin{dmath}\label{eqn:qmLecture1:200}
\braket{p}{\Psi} = 
\leftrightarrow 
\stackrel{?}{\leftrightarrow}
\braket{x}{\Psi}.
\end{dmath}

Such a conversion can be performed by virtue of an the assumption that we have a complete orthonormal basis, for which we can introduce identity operations such as

\begin{dmath}\label{eqn:qmLecture1:220}
\int_{-\infty}^\infty \ket{p}\bra{p} = 1,
\end{dmath}

or
\begin{dmath}\label{eqn:qmLecture1:240}
\int_{-\infty}^\infty \ket{x}\bra{x} = 1
\end{dmath}

FIXME: justification?:

\begin{dmath}\label{eqn:qmLecture1:260}
\begin{aligned}
\int_{-\infty}^\infty \Psi{p}\ket{p} &= \ket{\Psi} \\
\int_{-\infty}^\infty \Psi{x}\ket{x} &= \ket{\Psi}
\end{aligned}
\end{dmath}

Some interpretations:

\begin{enumerate}
\item \( \ket{x_0} \leftrightarrow \text{sits at} x = x_0 \)
\item \( \braket{x}{x'} \leftrightarrow \delta(x - x') \)
\item \( \braket{p}{p'} \leftrightarrow \delta(p - p') \)
\item \( \braket{x}{p'} = \frac{e^{i p x}}{\sqrt{V}} \), where \( V \) is the volume of the box containing the particle.  We'll treat the case where the box bolume is infinite later.
\end{enumerate}

The conversion from a position basis to momentum space is now possible 

\begin{dmath}\label{eqn:qmLecture1:280}
\braket{p}{\Psi} 
=
\Psi(p)
= \int_{-\infty}^\infty \braket{p}{x} \braket{x}{\Psi} dx 
= \int_{-\infty}^\infty \frac{e^{-ip x}}{\sqrt{V}} \Psi(x) dx
\end{dmath}

where \( \braket{p}{x} = \braket{x}{p}^\conj \).  The momentum space to position space conversion can be written as

\begin{dmath}\label{eqn:qmLecture1:300}
\Psi(x) 
= \int_{-\infty}^\infty \frac{e^{ip x}}{\sqrt{V}} \Psi(p) dp.
\end{dmath}

Now we can go back and figure out the an expectation

\begin{dmath}\label{eqn:qmLecture1:320}
\expectation{p}
=
\int \Psi^\conj(p) \Psi(p) p d p
=
\int dp 
\lr{
\int_{-\infty}^\infty \frac{e^{ip x}}{\sqrt{V}} \Psi^\conj(x) dx
}
\lr{
\int_{-\infty}^\infty \frac{e^{-ip x'}}{\sqrt{V}} \Psi(x') dx'
}
p d p.
\end{dmath}

FIXME : Performing this integral we find

\begin{dmath}\label{eqn:qmLecture1:340}
p e^{i p x} \leftrightarrow -i \PD{x}{} e^{i p x},
\end{dmath}

so

\begin{dmath}\label{eqn:qmLecture1:360}
\expectation{p}
=
\int dp dx dx'
\lr{ \frac{e^{ip x}}{\sqrt{V}} \Psi^\conj(x)  }
\lr{ i \frac{\Psi(x')}{\sqrt{V}} \PD{x'}{} e^{-ip x'} }
= \int \frac{dx dx'}{V} \Psi^\conj(x) \Psi(x) i \PD{x'}{} \lr{ \int dp e^{i p x - i p x'} }
= \int \frac{dx dx'}{V} \Psi^\conj(x) \Psi(x) i \PD{x'} V \delta(x - x')
= \int dx dx' \Psi^\conj(x) \Psi(x) i \PD{x'}{} \delta(x - x')
= \int dx dx' \Psi^\conj(x) \lr{ -i \PD{x'}{\Psi(x')}} \delta(x - x')
= \int dx \Psi^\conj(x) \lr{ -i \PD{x}{} } \Psi(x).
\end{dmath}

Allowing identifications like

\begin{dmath}\label{eqn:qmLecture1:380}
p \leftrightarrow -i \PD{x}{}
\end{dmath}
\begin{dmath}\label{eqn:qmLecture1:400}
p^2 \leftrightarrow - \PDSq{x}{}
\end{dmath}

\paragraph{Matrix interpreation}

\begin{enumerate}
\item Ket's \( \ket{\Psi} \leftrightarrow \text{column vector} \)
\item Bra's \( \bra{\Psi} \leftrightarrow {\text{row vector}}^\conj \)
\item Operators \( \leftrightarrow \) matrices that act on vectors.
\end{enumerate}

\begin{dmath}\label{eqn:qmLecture1:420}
\hat{p} \ket{\Psi} \rightarrow \ket{\Psi'}
\end{dmath}

\paragraph{Time evolution}

For a state subject to the equations of motion given by the Hamiltonian operator \( \hat{H} \)

\begin{dmath}\label{eqn:qmLecture1:440}
i \PD{t}{} \ket{\Psi} = \hat{H} \ket{\Psi},
\end{dmath}

the time evolution is given by
\begin{dmath}\label{eqn:qmLecture1:460}
\ket{\Psi(t)} = e^{-i \hat{H} t} \ket{\Psi(0)}.
\end{dmath}

\paragraph{Incomplete information}

We'll need to introduce the concept of Density matrices.  This will bring us to concepts like entanglement.

\EndArticle

      \section{Basic concepts, time evolution, and density operators}
         \input{lecture2.tex}
      \section{Density matrix (cont.)}
         %
% Copyright � 2015 Peeter Joot.  All Rights Reserved.
% Licenced as described in the file LICENSE under the root directory of this GIT repository.
%
%\newcommand{\authorname}{Peeter Joot}
\newcommand{\email}{peeterjoot@protonmail.com}
\newcommand{\basename}{FIXMEbasenameUndefined}
\newcommand{\dirname}{notes/FIXMEdirnameUndefined/}

%\renewcommand{\basename}{chapter3Notes}
%\renewcommand{\dirname}{notes/phy1520/}
%\newcommand{\keywords}{PHY1520H}
%\newcommand{\authorname}{Peeter Joot}
\newcommand{\onlineurl}{http://sites.google.com/site/peeterjoot2/math2013/\basename.pdf}
\newcommand{\sourcepath}{\dirname\basename.tex}
\newcommand{\generatetitle}[1]{\chapter{#1}}

\newcommand{\vcsinfo}{%
\section*{}
\noindent{\color{DarkOliveGreen}{\rule{\linewidth}{0.1mm}}}
\paragraph{Document version}
%\paragraph{\color{Maroon}{Document version}}
{
\small
\begin{itemize}
\item Available online at:\\ 
\href{\onlineurl}{\onlineurl}
\item Git Repository: \input{./.revinfo/gitRepo.tex}
\item Source: \sourcepath
\item last commit: \input{./.revinfo/gitCommitString.tex}
\item commit date: \input{./.revinfo/gitCommitDate.tex}
\end{itemize}
}
}

%\PassOptionsToPackage{dvipsnames,svgnames}{xcolor}
\PassOptionsToPackage{square,numbers}{natbib}
\documentclass{scrreprt}

\usepackage[left=2cm,right=2cm]{geometry}
\usepackage[svgnames]{xcolor}
\usepackage{peeters_layout}

\usepackage{natbib}

\usepackage[
colorlinks=true,
bookmarks=false,
pdfauthor={\authorname, \email},
backref 
]{hyperref}

% http://tex.stackexchange.com/questions/75773/how-to-reference-problems-by-the-text-label-in-an-exercise-envioronment
\usepackage[english]{cleveref}
\crefname{Exercise}{exercise}{exercises}
\Crefname{Exercise}{Exercise}{Exercises}

\RequirePackage{titlesec}
\RequirePackage{ifthen}

% http://stackoverflow.com/questions/4932910/date-in-the-tabular-environment
\makeatletter
\let\insertdate\@date
\makeatother

\titleformat{\chapter}[display]
{\bfseries\Large}
{\color{DarkSlateGrey}\filleft \authorname
\ifthenelse{\isundefined{\studentnumber}}{}{\\ \studentnumber}
\ifthenelse{\isundefined{\email}}{}{\\ \email}
\ifthenelse{\isundefined{\dateintitle}}{}{\\ \insertdate}
%\ifthenelse{\isundefined{\coursename}}{}{\\ \coursename} % put in title instead.
}
{4ex}
{\color{DarkOliveGreen}{\titlerule}\color{Maroon}
\vspace{2ex}%
\filright}
[\vspace{2ex}%
\color{DarkOliveGreen}\titlerule
]

\newcommand{\beginArtWithToc}[0]{\begin{document}\tableofcontents}
\newcommand{\beginArtNoToc}[0]{\begin{document}}
\newcommand{\EndNoBibArticle}[0]{\end{document}}
\newcommand{\EndArticle}[0]{\bibliography{Bibliography}\bibliographystyle{plainnat}\end{document}}

% 
%\newcommand{\citep}[1]{\cite{#1}}

\colorSectionsForArticle


%
%%\usepackage{phy1520}
%\usepackage{peeters_braket}
%\usepackage{peeters_layout_exercise}
%\usepackage{peeters_figures}
%\usepackage{mathtools}
%
%\beginArtNoToc
%\generatetitle{PHY1520H Graduate Quantum Mechanics.  Lecture 3: Density matrix (cont.).  Taught by Prof.\ Arun Paramekanti}
%%\generatetitle{Density matrix (cont.)}
%%\chapter{Density matrix (cont.)}
%\label{chap:chapter3Notes}
%
%\paragraph{Disclaimer}
%
%Peeter's lecture notes from class.  These may be incoherent and rough.
%
%These are notes for the UofT course PHY1520, Graduate Quantum Mechanics, taught by Prof. Paramekanti, covering \textchapref{{1}} \citep{sakurai2014modern} content.
%
%\paragraph{Density matrix (cont.)}

An example of a partitioned system with four total states (two spin 1/2 particles) is sketched in \cref{fig:twoSpins:twoSpinsFig1}.

\imageFigure{../../figures/phy1520/twoSpinsFig1}{Two spins.}{fig:twoSpins:twoSpinsFig1}{0.2}

An example of a partitioned system with eight total states (three spin 1/2 particles) is sketched in \cref{fig:threeSpins:threeSpinsFig2}.

\imageFigure{../../figures/phy1520/threeSpinsFig2}{Three spins.}{fig:threeSpins:threeSpinsFig2}{0.3}

The density matrix

\begin{dmath}\label{eqn:qmLecture3:20}
\hat\rho = \ket{\Psi}\bra{\Psi}
\end{dmath}

is clearly an operator as can be seen by applying it to a state

\begin{dmath}\label{eqn:qmLecture3:40}
\hat\rho \ket{\phi} = \ket{\Psi} \lr{ \braket{ \Psi }{\phi} }.
\end{dmath}

The quantity in braces is just a complex number.

After expanding the pure state \( \ket{\Psi} \) in terms of basis states for each of the two partitions

\begin{dmath}\label{eqn:qmLecture3:60}
\ket{\Psi}
= \sum_{m,n} C_{m, n} \ket{m}_\txtL \ket{n}_\txtR,
\end{dmath}

With \( \txtL \) and \( \txtR \) implied for \( \ket{m}, \ket{n} \) indexed states respectively, this can be written

\begin{dmath}\label{eqn:qmLecture3:460}
\ket{\Psi}
= \sum_{m,n} C_{m, n} \ket{m} \ket{n}.
\end{dmath}

The density operator is

\begin{dmath}\label{eqn:qmLecture3:80}
\hat\rho =
\sum_{m,n}
C_{m, n}
C_{m', n'}^\conj
\ket{m} \ket{n}
\sum_{m',n'}
\bra{m'} \bra{n'}.
\end{dmath}

Suppose we trace over the right partition of the state space, defining such a trace as the reduced density operator \( \hat\rho_{\textrm{red}} \)

\begin{dmath}\label{eqn:qmLecture3:100}
\hat\rho_{\textrm{red}}
\equiv
\tr_\txtR(\hat\rho)
= \sum_{\tilde{n}} \bra{\tilde{n}} \hat\rho \ket{ \tilde{n}}
= \sum_{\tilde{n}}
\bra{\tilde{n} }
\lr{
\sum_{m,n}
C_{m, n}
\ket{m} \ket{n}
}
\lr{
\sum_{m',n'}
C_{m', n'}^\conj
\bra{m'} \bra{n'}
}
\ket{ \tilde{n} }
=
\sum_{\tilde{n}}
\sum_{m,n}
\sum_{m',n'}
C_{m, n}
C_{m', n'}^\conj
\ket{m} \delta_{\tilde{n} n}
\bra{m' }
\delta_{ \tilde{n} n' }
=
\sum_{\tilde{n}, m, m'}
C_{m, \tilde{n}}
C_{m', \tilde{n}}^\conj
\ket{m} \bra{m' }
\end{dmath}

Computing the matrix element of \( \hat\rho_{\textrm{red}} \), we have

\begin{dmath}\label{eqn:qmLecture3:120}
\bra{\tilde{m}} \hat\rho_{\textrm{red}} \ket{\tilde{m}}
=
\sum_{m, m', \tilde{n}} C_{m, \tilde{n}} C_{m', \tilde{n}}^\conj \braket{ \tilde{m}}{m} \braket{m'}{\tilde{m}}
=
\sum_{\tilde{n}} \Abs{C_{\tilde{m}, \tilde{n}} }^2.
\end{dmath}

This is the probability that the left partition is in state \( \tilde{m} \).

\section{Average of an observable}
\index{density operator!observable expectation}

Suppose we have two spin half particles.  For such a system the total magnetization is

\begin{dmath}\label{eqn:qmLecture3:140}
S_{\textrm{Total}} =
S_1^z
+
S_1^z,
\end{dmath}

as sketched in \cref{fig:magneticMomentTwoSpins:magneticMomentTwoSpinsFig3}.

\imageFigure{../../figures/phy1520/magneticMomentTwoSpinsFig3}{Magnetic moments from two spins.}{fig:magneticMomentTwoSpins:magneticMomentTwoSpinsFig3}{0.1}

The average of some observable is

\begin{dmath}\label{eqn:qmLecture3:160}
\expectation{\hatA}
= \sum_{m, n, m', n'} C_{m, n}^\conj C_{m', n'}
\bra{m}\bra{n} \hatA \ket{n'} \ket{m'}.
\end{dmath}

\index{trace}
Consider the trace of the density operator observable product

\begin{dmath}\label{eqn:qmLecture3:180}
\tr( \hat\rho \hatA )
= \sum_{m, n} \braket{m n}{\Psi} \bra{\Psi} \hatA \ket{m, n}.
\end{dmath}

Let

\begin{dmath}\label{eqn:qmLecture3:200}
\ket{\Psi} = \sum_{m, n} C_{m n} \ket{m, n},
\end{dmath}

so that

\begin{dmath}\label{eqn:qmLecture3:220}
\tr( \hat\rho \hatA )
= \sum_{m, n, m', n', m'', n''} C_{m', n'} C_{m'', n''}^\conj
\braket{m n}{m', n'} \bra{m'', n''} \hatA \ket{m, n}
= \sum_{m, n, m'', n''} C_{m, n} C_{m'', n''}^\conj
\bra{m'', n''} \hatA \ket{m, n}.
\end{dmath}

This is just

%\begin{dmath}\label{eqn:qmLecture3:240}
\boxedEquation{eqn:qmLecture3:240}{
\bra{\Psi} \hatA \ket{\Psi} = \tr( \hat\rho \hatA ).
}
%\end{dmath}

\section{Left observables}
\index{density operator!left observable}

Consider

\begin{dmath}\label{eqn:qmLecture3:260}
\bra{\Psi} \hatA_\txtL \ket{\Psi}
= \tr(\hat\rho \hatA_\txtL)
=
\tr_\txtL
\tr_\txtR
(\hat\rho \hatA_\txtL)
=
\tr_\txtL
\lr{
\lr{
\tr_\txtR \hat\rho
}
\hatA_\txtL)
}
=
\tr_\txtL
\lr{
\hat\rho_{\textrm{red}}
\hatA_\txtL)
}.
\end{dmath}

We see

\begin{dmath}\label{eqn:qmLecture3:280}
\bra{\Psi} \hatA_\txtL \ket{\Psi}
=
\tr_\txtL \lr{ \hat\rho_{\textrm{red}, \txtL} \hatA_\txtL }.
\end{dmath}

We find that we don't need to know the state of the complete system to answer questions about portions of the system, but instead just need \( \hat\rho \), a ``probability operator'' that provides all the required information about the partitioning of the system.

\section{Pure states vs. mixed states}

For pure states we can assign a state vector and talk about reduced scenarios.  For mixed states we must work with reduced density matrices.

\index{spin half!two particles}
\makeexample{Two particle spin half pure states}{example:qmLecture3:1}{

Consider

\begin{dmath}\label{eqn:qmLecture3:300}
\ket{\psi_1} = \inv{\sqrt{2}} \lr{ \ket{ \uparrow \downarrow } - \ket{ \downarrow \uparrow } }
\end{dmath}

\begin{dmath}\label{eqn:qmLecture3:320}
\ket{\psi_2} = \inv{\sqrt{2}} \lr{ \ket{ \uparrow \downarrow } + \ket{ \uparrow \uparrow } }.
\end{dmath}

For the first pure state the density operator is
\begin{dmath}\label{eqn:qmLecture3:360}
\hat\rho = \inv{2}
\lr{ \ket{ \uparrow \downarrow } - \ket{ \downarrow \uparrow } }
\lr{ \bra{ \uparrow \downarrow } - \bra{ \downarrow \uparrow } }
\end{dmath}

What are the reduced density matrices?
\index{reduced density!operator}

\begin{dmath}\label{eqn:qmLecture3:340}
\hat\rho_\txtL
= \tr_\txtR \lr{ \hat\rho }
=
\inv{2} (-1)(-1) \ket{\downarrow}\bra{\downarrow}
+\inv{2} (+1)(+1) \ket{\uparrow}\bra{\uparrow},
\end{dmath}

so the matrix representation of this reduced density operator is

\begin{dmath}\label{eqn:qmLecture3:380}
\hat\rho_\txtL
=
\inv{2}
\begin{bmatrix}
1 & 0 \\
0 & 1
\end{bmatrix}.
\end{dmath}

For the second pure state the density operator is
\begin{dmath}\label{eqn:qmLecture3:400}
\hat\rho = \inv{2}
\lr{ \ket{ \uparrow \downarrow } + \ket{ \uparrow \uparrow } }
\lr{ \bra{ \uparrow \downarrow } + \bra{ \uparrow \uparrow } }.
\end{dmath}

This has a reduced density matrix

\begin{dmath}\label{eqn:qmLecture3:420}
\hat\rho_\txtL
= \tr_\txtR \lr{ \hat\rho }
=
\inv{2} \ket{\uparrow}\bra{\uparrow}
+\inv{2} \ket{\uparrow}\bra{\uparrow}
=
\ket{\uparrow}\bra{\uparrow} .
\end{dmath}

This has a matrix representation

\begin{dmath}\label{eqn:qmLecture3:440}
\hat\rho_\txtL
=
\begin{bmatrix}
1 & 0 \\
0 & 0
\end{bmatrix}.
\end{dmath}

\index{entanglement entropy}
In this second example, we have more information about the left partition.  That will be seen as a zero entanglement entropy in the problem set.  In contrast we have less information about the first state, and will find a non-zero positive entanglement entropy in that case.

} % example

%\EndArticle

      \section{Schwartz inequality in bra-ket notation}
         %
% Copyright � 2015 Peeter Joot.  All Rights Reserved.
% Licenced as described in the file LICENSE under the root directory of this GIT repository.
%
%\newcommand{\authorname}{Peeter Joot}
\newcommand{\email}{peeterjoot@protonmail.com}
\newcommand{\basename}{FIXMEbasenameUndefined}
\newcommand{\dirname}{notes/FIXMEdirnameUndefined/}

%\renewcommand{\basename}{qmSchwartz}
%\renewcommand{\dirname}{notes/phy1520/}
%%\newcommand{\dateintitle}{}
%%\newcommand{\keywords}{}
%
%\newcommand{\authorname}{Peeter Joot}
\newcommand{\onlineurl}{http://sites.google.com/site/peeterjoot2/math2013/\basename.pdf}
\newcommand{\sourcepath}{\dirname\basename.tex}
\newcommand{\generatetitle}[1]{\chapter{#1}}

\newcommand{\vcsinfo}{%
\section*{}
\noindent{\color{DarkOliveGreen}{\rule{\linewidth}{0.1mm}}}
\paragraph{Document version}
%\paragraph{\color{Maroon}{Document version}}
{
\small
\begin{itemize}
\item Available online at:\\ 
\href{\onlineurl}{\onlineurl}
\item Git Repository: \input{./.revinfo/gitRepo.tex}
\item Source: \sourcepath
\item last commit: \input{./.revinfo/gitCommitString.tex}
\item commit date: \input{./.revinfo/gitCommitDate.tex}
\end{itemize}
}
}

%\PassOptionsToPackage{dvipsnames,svgnames}{xcolor}
\PassOptionsToPackage{square,numbers}{natbib}
\documentclass{scrreprt}

\usepackage[left=2cm,right=2cm]{geometry}
\usepackage[svgnames]{xcolor}
\usepackage{peeters_layout}

\usepackage{natbib}

\usepackage[
colorlinks=true,
bookmarks=false,
pdfauthor={\authorname, \email},
backref 
]{hyperref}

% http://tex.stackexchange.com/questions/75773/how-to-reference-problems-by-the-text-label-in-an-exercise-envioronment
\usepackage[english]{cleveref}
\crefname{Exercise}{exercise}{exercises}
\Crefname{Exercise}{Exercise}{Exercises}

\RequirePackage{titlesec}
\RequirePackage{ifthen}

% http://stackoverflow.com/questions/4932910/date-in-the-tabular-environment
\makeatletter
\let\insertdate\@date
\makeatother

\titleformat{\chapter}[display]
{\bfseries\Large}
{\color{DarkSlateGrey}\filleft \authorname
\ifthenelse{\isundefined{\studentnumber}}{}{\\ \studentnumber}
\ifthenelse{\isundefined{\email}}{}{\\ \email}
\ifthenelse{\isundefined{\dateintitle}}{}{\\ \insertdate}
%\ifthenelse{\isundefined{\coursename}}{}{\\ \coursename} % put in title instead.
}
{4ex}
{\color{DarkOliveGreen}{\titlerule}\color{Maroon}
\vspace{2ex}%
\filright}
[\vspace{2ex}%
\color{DarkOliveGreen}\titlerule
]

\newcommand{\beginArtWithToc}[0]{\begin{document}\tableofcontents}
\newcommand{\beginArtNoToc}[0]{\begin{document}}
\newcommand{\EndNoBibArticle}[0]{\end{document}}
\newcommand{\EndArticle}[0]{\bibliography{Bibliography}\bibliographystyle{plainnat}\end{document}}

% 
%\newcommand{\citep}[1]{\cite{#1}}

\colorSectionsForArticle


%
%%\usepackage{braket}
%\usepackage{peeters_braket}
%
%\beginArtNoToc
%
%\generatetitle{Schwartz inequality in bra-ket notation}
%\label{chap:qmSchwartz}
\paragraph{Motivation}

In \citep{sakurai2014modern} the \textAndIndex{Schwartz inequality}

\boxedEquation{eqn:qmSchwartz:20}{
%\begin{dmath}\label{eqn:qmSchwartz:20}
\braket{a}{a}
\braket{b}{b}
\ge \Abs{\braket{a}{b}}^2,
%\end{dmath}
}

is used in the derivation of the uncertainty relation.  The proof of the Schwartz inequality uses a sneaky substitution that doesn't seem obvious, and is even less obvious since there is a typo in the value to be substituted.  Let's understand where that sneakiness is coming from.

\paragraph{Without being sneaky}

My ancient first year linear algebra text \citep{nicholson1990elementary} contains a non-sneaky proof, but it only works for real vector spaces.  Recast in bra-ket notation, this method examines the bounds of the norms of sums and differences of unit state vectors (i.e. \( \braket{a}{a} = \braket{b}{b} = 1 \).)

\begin{dmath}\label{eqn:qmSchwartz:40}
\braket{a - b}{a - b}
= \braket{a}{a} + \braket{b}{b} - \braket{a}{b} - \braket{b}{a}
= 2 - 2 \Real \braket{a}{b}
\ge 0,
\end{dmath}

so
\begin{dmath}\label{eqn:qmSchwartz:60}
1 \ge \Real \braket{a}{b}.
\end{dmath}

Similarly

\begin{dmath}\label{eqn:qmSchwartz:80}
\braket{a + b}{a + b}
= \braket{a}{a} + \braket{b}{b} + \braket{a}{b} + \braket{b}{a}
= 2 + 2 \Real \braket{a}{b}
\ge 0,
\end{dmath}

so
\begin{dmath}\label{eqn:qmSchwartz:100}
\Real \braket{a}{b} \ge -1.
\end{dmath}

This means that for normalized state vectors

\begin{equation}\label{eqn:qmSchwartz:120}
-1 \le \Real \braket{a}{b} \le 1,
\end{equation}

or
\begin{dmath}\label{eqn:qmSchwartz:140}
\Abs{\Real \braket{a}{b}} \le 1.
\end{dmath}

%Because this bra-ket is just a complex number, we must also have
%
%\begin{equation}\label{eqn:qmSchwartz:160}
%1 \ge \Abs{\Real \braket{a}{b}} \ge \Abs{\braket{a}{b}}.
%\end{equation}

Writing out the unit vectors explicitly, that last inequality is

\begin{dmath}\label{eqn:qmSchwartz:180}
\Abs{ \Real \braket{ \frac{a}{\sqrt{\braket{a}{a}}} }{ \frac{b}{\sqrt{\braket{b}{b}}} } } \le 1,
%1 \le \Abs{ \Braket{ \frac{a}{\sqrt{\braket{a|a}}} | \frac{b}{\sqrt{\braket{b|b}}} } },
%1 \le \Abs{ \Bra{ \frac{a}{\sqrt{\braket{a|a}}} } \cdot \Ket{ \frac{b}{\sqrt{\braket{b|b}}} } },
\end{dmath}

squaring and rearranging gives

\begin{dmath}\label{eqn:qmSchwartz:200}
\Abs{\Real \braket{a}{b}}^2 \le
\braket{a}{a}
\braket{b}{b}.
\end{dmath}

This is similar to, but not identical to the Schwartz inequality.  Since \( \Abs{\Real \braket{a}{b}} \le \Abs{\braket{a}{b}} \) the Schwartz inequality cannot be demonstrated with this argument.  This first year algebra method works nicely for demonstrating the inequality for real vector spaces, so a different argument is required for a complex vector space (i.e. quantum mechanics state space.)

\paragraph{Arguing with projected and rejected components}

Notice that the equality condition holds when the vectors are colinear, and the largest inequality (\(0 \le 1\)) holds when the vectors are normal to each other.  Given those geometrical observations, it seems reasonable to examine the norms of projected or rejected components of a vector.  To do so in bra-ket notation, the correct form of a projection operation must be determined.  Care is required to get the ordering of the bra-kets right when expressing such a projection (or rejection)

Suppose we wish to calculation the rejection of \( \ket{a} \) from \( \ket{b} \), that is \( \ket{b - \alpha a}\), such that

\begin{dmath}\label{eqn:qmSchwartz:220}
0
= \braket{a}{b - \alpha a}
= \braket{a}{b} - \alpha \braket{a}{a},
\end{dmath}

or
\begin{dmath}\label{eqn:qmSchwartz:240}
\alpha =
\frac{\braket{a}{b} }{ \braket{a}{a} }.
\end{dmath}

Therefore, the projection of \( \ket{b} \) on \( \ket{a} \) is

\begin{equation}\label{eqn:qmSchwartz:260}
\Proj_{\ket{a}} \ket{b}
= \frac{\braket{a}{b} }{ \braket{a}{a} } \ket{a}
= \frac{\braket{b}{a}^\conj }{ \braket{a}{a} } \ket{a}.
\end{equation}

The conventional way to write this in QM is in the operator form

\begin{dmath}\label{eqn:qmSchwartz:300}
\Proj_{\ket{a}} \ket{b}
= \frac{\ket{a}\bra{a}}{\braket{a}{a}} \ket{b}.
\end{dmath}

In this form the rejection of \( \ket{a} \) from \( \ket{b} \) can be expressed as

\begin{dmath}\label{eqn:qmSchwartz:280}
\RejName_{\ket{a}} \ket{b} = \ket{b} - \frac{\ket{a}\bra{a}}{\braket{a}{a}} \ket{b}.
\end{dmath}

This state vector is normal to \( \ket{a} \) as desired

\begin{dmath}\label{eqn:qmSchwartz:320}
\braket{a}{b - \frac{\braket{a}{b} }{ \braket{a}{a} } a }
=
\braket{a}{ b}  - \frac{ \braket{a}{b} }{ \cancel{\braket{a}{a}} } \cancel{\braket{a}{a}}
= 0.
\end{dmath}

How about it's length?  That is

\begin{equation}\label{eqn:qmSchwartz:340}
\begin{aligned}
\braket{b - \frac{\braket{a}{b} }{ \braket{a}{a} } a}{b - \frac{\braket{a}{b} }{ \braket{a}{a} } a }
&=
\braket{b}{b} - 2 \frac{\Abs{\braket{a}{b}}^2}{\braket{a}{a}} +\frac{\Abs{\braket{a}{b}}^2 }{ \braket{a}{a}^2 } \braket{a}{a} \\
&=
\braket{b}{b} - \frac{\Abs{\braket{a}{b}}^2}{\braket{a}{a}}.
\end{aligned}
\end{equation}

Observe that this must be greater to or equal to zero, so

\begin{equation}\label{eqn:qmSchwartz:360}
\braket{b}{b} \ge \frac{ \Abs{ \braket{a}{b} }^2 }{ \braket{a}{a} }.
\end{equation}

Rearranging this gives \cref{eqn:qmSchwartz:20} as desired.  The Schwartz proof in \citep{sakurai2014modern} obscures the geometry involved and starts with

\begin{dmath}\label{eqn:qmSchwartz:380}
\braket{b + \lambda a}{b + \lambda a} \ge 0,
\end{dmath}

where the ``proof'' is nothing more than a statement that one can ``pick'' \( \lambda = -\braket{b}{a}/\braket{a}{a} \).  The Pythagorean context of the Schwartz inequality is not mentioned, and without thinking about it, one is left wondering what sort of magic hat that \( \lambda \) selection came from.

%\EndArticle

      \section{An observation about the geometry of Pauli x,y matrices}
         %
% Copyright � 2015 Peeter Joot.  All Rights Reserved.
% Licenced as described in the file LICENSE under the root directory of this GIT repository.
%
%\newcommand{\authorname}{Peeter Joot}
\newcommand{\email}{peeterjoot@protonmail.com}
\newcommand{\basename}{FIXMEbasenameUndefined}
\newcommand{\dirname}{notes/FIXMEdirnameUndefined/}

%\renewcommand{\basename}{pauliMatrixXYgeometry}
%\renewcommand{\dirname}{notes/phy1540/}
%%\newcommand{\dateintitle}{}
%%\newcommand{\keywords}{}
%
%\newcommand{\authorname}{Peeter Joot}
\newcommand{\onlineurl}{http://sites.google.com/site/peeterjoot2/math2013/\basename.pdf}
\newcommand{\sourcepath}{\dirname\basename.tex}
\newcommand{\generatetitle}[1]{\chapter{#1}}

\newcommand{\vcsinfo}{%
\section*{}
\noindent{\color{DarkOliveGreen}{\rule{\linewidth}{0.1mm}}}
\paragraph{Document version}
%\paragraph{\color{Maroon}{Document version}}
{
\small
\begin{itemize}
\item Available online at:\\ 
\href{\onlineurl}{\onlineurl}
\item Git Repository: \input{./.revinfo/gitRepo.tex}
\item Source: \sourcepath
\item last commit: \input{./.revinfo/gitCommitString.tex}
\item commit date: \input{./.revinfo/gitCommitDate.tex}
\end{itemize}
}
}

%\PassOptionsToPackage{dvipsnames,svgnames}{xcolor}
\PassOptionsToPackage{square,numbers}{natbib}
\documentclass{scrreprt}

\usepackage[left=2cm,right=2cm]{geometry}
\usepackage[svgnames]{xcolor}
\usepackage{peeters_layout}

\usepackage{natbib}

\usepackage[
colorlinks=true,
bookmarks=false,
pdfauthor={\authorname, \email},
backref 
]{hyperref}

% http://tex.stackexchange.com/questions/75773/how-to-reference-problems-by-the-text-label-in-an-exercise-envioronment
\usepackage[english]{cleveref}
\crefname{Exercise}{exercise}{exercises}
\Crefname{Exercise}{Exercise}{Exercises}

\RequirePackage{titlesec}
\RequirePackage{ifthen}

% http://stackoverflow.com/questions/4932910/date-in-the-tabular-environment
\makeatletter
\let\insertdate\@date
\makeatother

\titleformat{\chapter}[display]
{\bfseries\Large}
{\color{DarkSlateGrey}\filleft \authorname
\ifthenelse{\isundefined{\studentnumber}}{}{\\ \studentnumber}
\ifthenelse{\isundefined{\email}}{}{\\ \email}
\ifthenelse{\isundefined{\dateintitle}}{}{\\ \insertdate}
%\ifthenelse{\isundefined{\coursename}}{}{\\ \coursename} % put in title instead.
}
{4ex}
{\color{DarkOliveGreen}{\titlerule}\color{Maroon}
\vspace{2ex}%
\filright}
[\vspace{2ex}%
\color{DarkOliveGreen}\titlerule
]

\newcommand{\beginArtWithToc}[0]{\begin{document}\tableofcontents}
\newcommand{\beginArtNoToc}[0]{\begin{document}}
\newcommand{\EndNoBibArticle}[0]{\end{document}}
\newcommand{\EndArticle}[0]{\bibliography{Bibliography}\bibliographystyle{plainnat}\end{document}}

% 
%\newcommand{\citep}[1]{\cite{#1}}

\colorSectionsForArticle


%
%
%\beginArtNoToc
%
%\generatetitle{An observation about the geometry of Pauli x,y matrices}
%\chapter{An observation about the geometry of Pauli x,y matrices}
%\label{chap:pauliMatrixXYgeometry}

\paragraph{Motivation}

\index{Pauli matrix!orthonality}
The conventional form for the Pauli matrices is

\begin{equation}\label{eqn:pauliMatrixXYgeometry:20}
\begin{aligned}
\sigma_x &= \PauliX \\
\sigma_y &= \PauliY \\
\sigma_z &= \PauliZ
\end{aligned}.
\end{equation}

In \citep{desai2009quantum} these forms are derived based on the commutation relations

\begin{equation}\label{eqn:pauliMatrixXYgeometry:40}
\antisymmetric{\sigma_r}{\sigma_s} = 2 i \epsilon_{r s t} \sigma_t,
\end{equation}

by defining raising and lowering operators \( \sigma_{\pm} = \sigma_x \pm i \sigma_y \) and figuring out what form the matrix must take.  I noticed an interesting geometrical relation hiding in that derivation if \( \sigma_{+} \) is not assumed to be real.

\paragraph{Derivation}

For completeness, I'll repeat the argument of \citep{desai2009quantum}, which builds on the commutation relations of the raising and lowering operators.  Those are

\begin{dmath}\label{eqn:pauliMatrixXYgeometry:60}
\antisymmetric{\sigma_z}{\sigma_{\pm}}
=
\sigma_z \lr{ \sigma_x \pm i \sigma_y }
-\lr{ \sigma_x \pm i \sigma_y } \sigma_z
=
\antisymmetric{\sigma_z}{\sigma_x} \pm i \antisymmetric{\sigma_z}{\sigma_y}
=
2 i \sigma_y \pm i (-2 i) \sigma_x
= \pm 2 \lr{ \sigma_x \pm i \sigma_y }
= \pm 2 \sigma_{\pm},
\end{dmath}

and

\begin{dmath}\label{eqn:pauliMatrixXYgeometry:80}
\antisymmetric{\sigma_{+}}{\sigma_{-}}
=
\lr{ \sigma_x + i \sigma_y } \lr{ \sigma_x - i \sigma_y }
-\lr{ \sigma_x - i \sigma_y } \lr{ \sigma_x + i \sigma_y }
=
-i \sigma_x \sigma_y + i \sigma_y \sigma_x
- i \sigma_x \sigma_y + i \sigma_y \sigma_x
= 2 i \antisymmetric{ \sigma_y }{\sigma_x}
= 2 i (-2i) \sigma_z
= 4 \sigma_z
\end{dmath}

From these a matrix representation containing unknown values can be assumed.  Let

\begin{dmath}\label{eqn:pauliMatrixXYgeometry:100}
\sigma_{+} =
\begin{bmatrix}
a & b \\
c & d
\end{bmatrix}.
\end{dmath}

The commutator with \( \sigma_z \) can be computed

\begin{dmath}\label{eqn:pauliMatrixXYgeometry:120}
\antisymmetric{\sigma_z}{\sigma_{+}}
=
\PauliZ
\begin{bmatrix}
a & b \\
c & d
\end{bmatrix}
-
\begin{bmatrix}
a & b \\
c & d
\end{bmatrix}
\PauliZ
=
\begin{bmatrix}
a & b \\
-c & -d
\end{bmatrix}
-
\begin{bmatrix}
a & -b \\
c & -d
\end{bmatrix}
=
2
\begin{bmatrix}
0 & b \\
-c & 0
\end{bmatrix}
\end{dmath}

Now compare this with \cref{eqn:pauliMatrixXYgeometry:60}

\begin{dmath}\label{eqn:pauliMatrixXYgeometry:140}
2
\begin{bmatrix}
0 & b \\
-c & 0
\end{bmatrix}
=
2 \sigma_{+}
=
2
\begin{bmatrix}
a & b \\
d & d
\end{bmatrix}.
\end{dmath}

This shows that \( a = 0 \), and \( d = 0 \).  Similarly the \( \sigma_z \) commutator with the lowering operator is

\begin{dmath}\label{eqn:pauliMatrixXYgeometry:160}
\antisymmetric{\sigma_z}{\sigma_{-}}
=
\PauliZ
\begin{bmatrix}
0 & -c^\conj \\
b^\conj & 0
\end{bmatrix}
-
\begin{bmatrix}
0 & -c^\conj \\
b^\conj & 0
\end{bmatrix}
\PauliZ
=
\begin{bmatrix}
0 & -c^\conj \\
-b^\conj & 0
\end{bmatrix}
-
\begin{bmatrix}
0 & c^\conj \\
b^\conj & 0
\end{bmatrix}
=
-2
\begin{bmatrix}
0 & c^\conj \\
b^\conj & 0
\end{bmatrix}
\end{dmath}

Again comparing to \cref{eqn:pauliMatrixXYgeometry:60}, we have
\begin{dmath}\label{eqn:pauliMatrixXYgeometry:180}
-2
\begin{bmatrix}
0 & c^\conj \\
b^\conj & 0
\end{bmatrix}
= - 2 \sigma_{-}
= - 2
\begin{bmatrix}
0 & -c^\conj \\
b^\conj & 0
\end{bmatrix},
\end{dmath}

so \( c = 0 \).  Computing the commutator of the raising and lowering operators fixes \( b \)

\begin{dmath}\label{eqn:pauliMatrixXYgeometry:200}
\antisymmetric{\sigma_{+}}{\sigma_{-}}
=
\begin{bmatrix}
0 & b \\
0 & 0 \\
\end{bmatrix}
\begin{bmatrix}
0 & 0 \\
b^\conj & 0 \\
\end{bmatrix}
-
\begin{bmatrix}
0 & 0 \\
b^\conj & 0 \\
\end{bmatrix}
\begin{bmatrix}
0 & b \\
0 & 0 \\
\end{bmatrix}
=
\begin{bmatrix}
\Abs{b}^2 & 0 \\
0 & 0
\end{bmatrix}
-
\begin{bmatrix}
0 & 0
0 & -\Abs{b}^2 \\
\end{bmatrix}
=
\Abs{b}^2 \PauliZ
=
\Abs{b}^2 \sigma_z.
\end{dmath}

From \cref{eqn:pauliMatrixXYgeometry:80} it must be that \( \Abs{b}^2 = 4\), so the most general form of the raising operator is

\begin{dmath}\label{eqn:pauliMatrixXYgeometry:220}
\sigma_{+}
=
2
\begin{bmatrix}
0 & e^{i \phi}  \\
0 & 0
\end{bmatrix}.
\end{dmath}

\paragraph{Observation}

The conventional choice is to set \( \phi = 0 \), but I found it interesting to see the form of \( \sigma_x, \sigma_y \) without that choice.  That is

\begin{dmath}\label{eqn:pauliMatrixXYgeometry:240}
\sigma_x = \inv{2} \lr{ \sigma_{+} + \sigma_{-} }
=
\begin{bmatrix}
0 & e^{i \phi}  \\
e^{-i \phi} & 0 \\
\end{bmatrix}
\end{dmath}

\begin{dmath}\label{eqn:pauliMatrixXYgeometry:260}
\sigma_y = \inv{2 i} \lr{ \sigma_{+} - \sigma_{-} }
=
\begin{bmatrix}
0 & -i e^{i \phi}  \\
-i e^{-i \phi} & 0 \\
\end{bmatrix}
=
\begin{bmatrix}
0 & e^{i (\phi - \pi/2) }  \\
e^{-i (\phi - \pi/2)} & 0 \\
\end{bmatrix}.
\end{dmath}

Notice that the Pauli matrices \( \sigma_x \) and \( \sigma_y \) actually both have the same form as \( \sigma_x \), but the phase of the complex argument of each differs by \ang{90}.  That \ang{90} separation isn't obvious in the standard form \cref{eqn:pauliMatrixXYgeometry:20}.

It's a small detail, but I thought it was kind of cool that the orthogonality of these matrix unit vector representations is built directly into the structure of their matrix representations.

%\EndArticle

      \section{Operator matrix element}
         %
% Copyright � 2015 Peeter Joot.  All Rights Reserved.
% Licenced as described in the file LICENSE under the root directory of this GIT repository.
%
%\newcommand{\authorname}{Peeter Joot}
\newcommand{\email}{peeterjoot@protonmail.com}
\newcommand{\basename}{FIXMEbasenameUndefined}
\newcommand{\dirname}{notes/FIXMEdirnameUndefined/}

%\renewcommand{\basename}{operatorMatrixElement}
%\renewcommand{\dirname}{notes/phy1520/}
%%\newcommand{\dateintitle}{}
%%\newcommand{\keywords}{}
%
%\newcommand{\authorname}{Peeter Joot}
\newcommand{\onlineurl}{http://sites.google.com/site/peeterjoot2/math2013/\basename.pdf}
\newcommand{\sourcepath}{\dirname\basename.tex}
\newcommand{\generatetitle}[1]{\chapter{#1}}

\newcommand{\vcsinfo}{%
\section*{}
\noindent{\color{DarkOliveGreen}{\rule{\linewidth}{0.1mm}}}
\paragraph{Document version}
%\paragraph{\color{Maroon}{Document version}}
{
\small
\begin{itemize}
\item Available online at:\\ 
\href{\onlineurl}{\onlineurl}
\item Git Repository: \input{./.revinfo/gitRepo.tex}
\item Source: \sourcepath
\item last commit: \input{./.revinfo/gitCommitString.tex}
\item commit date: \input{./.revinfo/gitCommitDate.tex}
\end{itemize}
}
}

%\PassOptionsToPackage{dvipsnames,svgnames}{xcolor}
\PassOptionsToPackage{square,numbers}{natbib}
\documentclass{scrreprt}

\usepackage[left=2cm,right=2cm]{geometry}
\usepackage[svgnames]{xcolor}
\usepackage{peeters_layout}

\usepackage{natbib}

\usepackage[
colorlinks=true,
bookmarks=false,
pdfauthor={\authorname, \email},
backref 
]{hyperref}

% http://tex.stackexchange.com/questions/75773/how-to-reference-problems-by-the-text-label-in-an-exercise-envioronment
\usepackage[english]{cleveref}
\crefname{Exercise}{exercise}{exercises}
\Crefname{Exercise}{Exercise}{Exercises}

\RequirePackage{titlesec}
\RequirePackage{ifthen}

% http://stackoverflow.com/questions/4932910/date-in-the-tabular-environment
\makeatletter
\let\insertdate\@date
\makeatother

\titleformat{\chapter}[display]
{\bfseries\Large}
{\color{DarkSlateGrey}\filleft \authorname
\ifthenelse{\isundefined{\studentnumber}}{}{\\ \studentnumber}
\ifthenelse{\isundefined{\email}}{}{\\ \email}
\ifthenelse{\isundefined{\dateintitle}}{}{\\ \insertdate}
%\ifthenelse{\isundefined{\coursename}}{}{\\ \coursename} % put in title instead.
}
{4ex}
{\color{DarkOliveGreen}{\titlerule}\color{Maroon}
\vspace{2ex}%
\filright}
[\vspace{2ex}%
\color{DarkOliveGreen}\titlerule
]

\newcommand{\beginArtWithToc}[0]{\begin{document}\tableofcontents}
\newcommand{\beginArtNoToc}[0]{\begin{document}}
\newcommand{\EndNoBibArticle}[0]{\end{document}}
\newcommand{\EndArticle}[0]{\bibliography{Bibliography}\bibliographystyle{plainnat}\end{document}}

% 
%\newcommand{\citep}[1]{\cite{#1}}

\colorSectionsForArticle


%
%\usepackage{peeters_layout_exercise}
%\usepackage{peeters_braket}
%\usepackage{peeters_figures}
%
%\beginArtNoToc
%
%\generatetitle{Operator matrix element}
%\chapter{Operator matrix element}
%\label{chap:operatorMatrixElement}

\paragraph{Weird dreams}
\index{matrix element}

I woke up today having a dream still in my head from the night, but it was a strange one.  I was expanding out the Dirac notation representation of an operator in matrix form, but the symbols in the kets were elaborate pictures of Disney princesses that I was drawing with forestry scenery in the background, including little bears.  At the point that I woke up from the dream, I noticed that I'd gotten the proportion of the bears wrong in one of the pictures, and they looked like they were ready to eat one of the princess characters.

\paragraph{Guts}

As a side effect of this weird dream I actually started thinking about matrix element representation of operators.

When forming the matrix element of an operator using Dirac notation the elements are of the form \( \bra{\textrm{row}} A \ket{\textrm{column}} \).  I've gotten that mixed up a couple of times, so I thought it would be helpful to write this out explicitly for a \( 2 \times 2 \) operator representation for clarity.

To start, consider a change of basis for a single matrix element from basis \( \setlr{\ket{q}, \ket{r} } \), to basis \( \setlr{\ket{a}, \ket{b} } \)

\begin{dmath}\label{eqn:operatorMatrixElement:20}
\begin{aligned}
\bra{q} A \ket{r}
&=
\braket{q}{a} \bra{a} A \ket{r}
+
\braket{q}{b} \bra{b} A \ket{r} \\
&=
\braket{q}{a} \bra{a} A \ket{a}\braket{a}{r}
+ \braket{q}{a} \bra{a} A \ket{b}\braket{b}{r} \\
&+ \braket{q}{b} \bra{b} A \ket{a}\braket{a}{r}
+ \braket{q}{b} \bra{b} A \ket{b}\braket{b}{r} \\
&=
\braket{q}{a}
\begin{bmatrix}
\bra{a} A \ket{a} & \bra{a} A \ket{b}
\end{bmatrix}
\begin{bmatrix}
\braket{a}{r} \\
\braket{b}{r}
\end{bmatrix}
+
\braket{q}{b}
\begin{bmatrix}
\bra{b} A \ket{a} & \bra{b} A \ket{b}
\end{bmatrix}
\begin{bmatrix}
\braket{a}{r} \\
\braket{b}{r}
\end{bmatrix} \\
&=
\begin{bmatrix}
\braket{q}{a} &
\braket{q}{b}
\end{bmatrix}
\begin{bmatrix}
\bra{a} A \ket{a} & \bra{a} A \ket{b} \\
\bra{b} A \ket{a} & \bra{b} A \ket{b}
\end{bmatrix}
\begin{bmatrix}
\braket{a}{r} \\
\braket{b}{r}
\end{bmatrix}.
\end{aligned}
\end{dmath}

Suppose the matrix representation of \( \ket{q}, \ket{r} \) are respectively

\begin{dmath}\label{eqn:operatorMatrixElement:40}
\begin{aligned}
\ket{q} &\sim
\begin{bmatrix}
\braket{a}{q} \\
\braket{b}{q} \\
\end{bmatrix} \\
\ket{r} &\sim
\begin{bmatrix}
\braket{a}{r} \\
\braket{b}{r} \\
\end{bmatrix} \\
\end{aligned},
\end{dmath}

then

\begin{dmath}\label{eqn:operatorMatrixElement:60}
\bra{q} \sim
{\begin{bmatrix}
\braket{a}{q} \\
\braket{b}{q} \\
\end{bmatrix}}^\dagger
=
\begin{bmatrix}
\braket{q}{a} &
\braket{q}{b}
\end{bmatrix}.
\end{dmath}

The matrix element is then

\begin{dmath}\label{eqn:operatorMatrixElement:80}
\bra{q} A \ket{r}
\sim
\bra{q}
\begin{bmatrix}
\bra{a} A \ket{a} & \bra{a} A \ket{b} \\
\bra{b} A \ket{a} & \bra{b} A \ket{b}
\end{bmatrix}
\ket{r},
\end{dmath}

and the corresponding matrix representation of the operator is

\begin{dmath}\label{eqn:operatorMatrixElement:100}
A \sim
\begin{bmatrix}
\bra{a} A \ket{a} & \bra{a} A \ket{b} \\
\bra{b} A \ket{a} & \bra{b} A \ket{b}
\end{bmatrix}.
\end{dmath}

%This particular matrix representation is largely convention, because we could have easily have picked an alternate representation of the kets.

%\EndNoBibArticle

      \section{Problems}
         %
% Copyright � 2015 Peeter Joot.  All Rights Reserved.
% Licenced as described in the file LICENSE under the root directory of this GIT repository.
%
%\newcommand{\authorname}{Peeter Joot}
\newcommand{\email}{peeterjoot@protonmail.com}
\newcommand{\basename}{FIXMEbasenameUndefined}
\newcommand{\dirname}{notes/FIXMEdirnameUndefined/}

%\renewcommand{\basename}{pauliProblems}
%\renewcommand{\dirname}{notes/phy1520/}
%%\newcommand{\dateintitle}{}
%%\newcommand{\keywords}{}
%
%\newcommand{\authorname}{Peeter Joot}
\newcommand{\onlineurl}{http://sites.google.com/site/peeterjoot2/math2013/\basename.pdf}
\newcommand{\sourcepath}{\dirname\basename.tex}
\newcommand{\generatetitle}[1]{\chapter{#1}}

\newcommand{\vcsinfo}{%
\section*{}
\noindent{\color{DarkOliveGreen}{\rule{\linewidth}{0.1mm}}}
\paragraph{Document version}
%\paragraph{\color{Maroon}{Document version}}
{
\small
\begin{itemize}
\item Available online at:\\ 
\href{\onlineurl}{\onlineurl}
\item Git Repository: \input{./.revinfo/gitRepo.tex}
\item Source: \sourcepath
\item last commit: \input{./.revinfo/gitCommitString.tex}
\item commit date: \input{./.revinfo/gitCommitDate.tex}
\end{itemize}
}
}

%\PassOptionsToPackage{dvipsnames,svgnames}{xcolor}
\PassOptionsToPackage{square,numbers}{natbib}
\documentclass{scrreprt}

\usepackage[left=2cm,right=2cm]{geometry}
\usepackage[svgnames]{xcolor}
\usepackage{peeters_layout}

\usepackage{natbib}

\usepackage[
colorlinks=true,
bookmarks=false,
pdfauthor={\authorname, \email},
backref 
]{hyperref}

% http://tex.stackexchange.com/questions/75773/how-to-reference-problems-by-the-text-label-in-an-exercise-envioronment
\usepackage[english]{cleveref}
\crefname{Exercise}{exercise}{exercises}
\Crefname{Exercise}{Exercise}{Exercises}

\RequirePackage{titlesec}
\RequirePackage{ifthen}

% http://stackoverflow.com/questions/4932910/date-in-the-tabular-environment
\makeatletter
\let\insertdate\@date
\makeatother

\titleformat{\chapter}[display]
{\bfseries\Large}
{\color{DarkSlateGrey}\filleft \authorname
\ifthenelse{\isundefined{\studentnumber}}{}{\\ \studentnumber}
\ifthenelse{\isundefined{\email}}{}{\\ \email}
\ifthenelse{\isundefined{\dateintitle}}{}{\\ \insertdate}
%\ifthenelse{\isundefined{\coursename}}{}{\\ \coursename} % put in title instead.
}
{4ex}
{\color{DarkOliveGreen}{\titlerule}\color{Maroon}
\vspace{2ex}%
\filright}
[\vspace{2ex}%
\color{DarkOliveGreen}\titlerule
]

\newcommand{\beginArtWithToc}[0]{\begin{document}\tableofcontents}
\newcommand{\beginArtNoToc}[0]{\begin{document}}
\newcommand{\EndNoBibArticle}[0]{\end{document}}
\newcommand{\EndArticle}[0]{\bibliography{Bibliography}\bibliographystyle{plainnat}\end{document}}

% 
%\newcommand{\citep}[1]{\cite{#1}}

\colorSectionsForArticle


%\usepackage{peeters_layout_exercise}
%
%\beginArtNoToc
%
%\generatetitle{Pauli matrix problems}
%\chapter{Pauli matrix problems}
%\label{chap:pauliProblems}

\makeoproblem{Representation of \( 2 \times 2 \) matrix with Pauli matrices.}{problem:pauliProblems:1.2}{\citep{sakurai2014modern} pr. 1.2}{
Given an arbitrary \( 2 \times 2 \) matrix \( X = a_0 + \Bsigma \cdot \Ba \), show the relationships between \( a_\mu \) and \( \trace(X), \trace(\sigma_k X) \), and \( X_{ij} \).
\index{Pauli matrix}
\index{Pauli matrix!trace}
} % problem

\makeanswer{problem:pauliProblems:1.2}{

Observe that each of the Pauli matrices \( \sigma_k \) are traceless

\begin{equation}\label{eqn:pauliProblems:20}
\begin{aligned}
\sigma_x &= \PauliX \\
\sigma_y &= \PauliY \\
\sigma_z &= \PauliZ \\
\end{aligned},
\end{equation}

so \( \trace(X) = 2 a_0 \).  Note that \( \trace(\sigma_k \sigma_m) = 2 \delta_{k m} \), so \( \trace(\sigma_k X) = 2 a_k \).

Notationally, it would seem to make sense to define \( \sigma_0 \equiv I \), so that \( \trace(\sigma_\mu X) = a_\mu \).  I don't know if that is common practice.

For the opposite relations, given

\begin{dmath}\label{eqn:pauliProblems:40}
X
= a_0 + \Bsigma \cdot \Ba
= \PauliI a_0 + \PauliX a_1 + \PauliY a_2 + \PauliZ a_3
=
\begin{bmatrix}
a_0 + a_3 & a_1 - i a_2 \\
a_1 + i a_2 & a_0 - a_3
\end{bmatrix}
=
\begin{bmatrix}
X_{11} & X_{12} \\
X_{21} & X_{22} \\
\end{bmatrix},
\end{dmath}

so
%\begin{equation}\label{eqn:pauliProblems:60}
%\begin{aligned}
%X_{11} &= a_0 + a_3 \\
%X_{22} &= a_0 - a_3 \\
%X_{12} &= a_1 - i a_2 \\
%X_{21} &= a_1 + i a_2 \\
%\end{aligned},
%\end{equation}
%
%or
\begin{equation}\label{eqn:pauliProblems:80}
\begin{aligned}
a_0 &= \inv{2} \lr{ X_{11} + X_{22} } \\
a_1 &= \inv{2} \lr{ X_{12} + X_{21} } \\
a_2 &= \inv{2 i} \lr{ X_{21} - X_{12} } \\
a_3 &= \inv{2} \lr{ X_{11} - X_{22} }
\end{aligned}.
\end{equation}
} % answer

\makeoproblem{Rotation transformation.}{problem:pauliProblems:1.3}{\citep{sakurai2014modern} pr. 1.3}{
Determine the structure and determinant of the transformation
\index{rotation}

\begin{equation}\label{eqn:pauliProblems:100}
\Bsigma \cdot \Ba \rightarrow
\Bsigma \cdot \Ba' =
\exp\lr{ i \Bsigma \cdot \ncap \phi/2}
\Bsigma \cdot \Ba
\exp\lr{ -i \Bsigma \cdot \ncap \phi/2}.
\end{equation}

} % problem

\makeanswer{problem:pauliProblems:1.3}{

Knowing Geometric Algebra, this is recognized as a rotation transformation.  In GA, \( i \) is treated as a pseudoscalar (which commutes with all grades in \R{3}), and the expression can be reduced to one involving dot and wedge products.  Let's see how can this be reduced using only the Pauli matrix toolbox.

First, consider the determinant of one of the exponentials.  Showing that one such exponential has unit determinant is sufficient.  The matrix representation of the unit normal is

\begin{dmath}\label{eqn:pauliProblems:120}
\Bsigma \cdot \ncap
= n_x \PauliX
+ n_y \PauliY
+ n_z \PauliZ
=
\begin{bmatrix}
n_z & n_x - i n_y \\
n_x + i n_y & -n_z
\end{bmatrix}.
\end{dmath}

This is expected to have a unit square, and does

\begin{dmath}\label{eqn:pauliProblems:140}
\lr{ \Bsigma \cdot \ncap }^2
=
\begin{bmatrix}
n_z & n_x - i n_y \\
n_x + i n_y & -n_z
\end{bmatrix}
\begin{bmatrix}
n_z & n_x - i n_y \\
n_x + i n_y & -n_z
\end{bmatrix}
=
\lr{ n_x^2 + n_y^2 + n_z^2 }
\begin{bmatrix}
1 & 0 \\
0 & 1
\end{bmatrix}
=
1.
\end{dmath}

This allows for a cosine and sine expansion of the exponential, as in

\begin{dmath}\label{eqn:pauliProblems:160}
\exp\lr{ i \Bsigma \cdot \ncap \theta}
=
\cos\theta + i \Bsigma \cdot \ncap \sin\theta
=
\cos\theta
\begin{bmatrix}
1 & 0 \\
0 & 1
\end{bmatrix}
+
i \sin\theta
\begin{bmatrix}
n_z & n_x - i n_y \\
n_x + i n_y & -n_z
\end{bmatrix}
=
\begin{bmatrix}
\cos\theta + i n_z \sin\theta & \lr{ n_x - i n_y } i \sin\theta \\
\lr{ n_x + i n_y } i \sin\theta & \cos\theta - i n_z \sin\theta \\
\end{bmatrix}.
\end{dmath}

\index{rotation!determinant}
This has determinant

\begin{dmath}\label{eqn:pauliProblems:180}
\Abs{\exp\lr{ i \Bsigma \cdot \ncap \theta} }
=
\cos^2\theta + n_z^2 \sin^2\theta
-
\lr{ -n_x^2 + -n_y^2 } \sin^2\theta
=
\cos^2\theta + \lr{ n_x^2 + n_y^2 + n_z^2 } \sin^2\theta
= 1,
\end{dmath}

as expected.

Next step is to show that this transformation is a rotation, and determine the sense of the rotation.  Let \( C = \cos\phi/2, S = \sin\phi/2 \), so that

\begin{dmath}\label{eqn:pauliProblems:200}
\Bsigma \cdot \Ba'
=
\exp\lr{ i \Bsigma \cdot \ncap \phi/2}
\Bsigma \cdot \Ba
\exp\lr{ -i \Bsigma \cdot \ncap \phi/2}
=
\lr{ C + i \Bsigma \cdot \ncap S }
\Bsigma \cdot \Ba
\lr{ C - i \Bsigma \cdot \ncap S }
=
\lr{ C + i \Bsigma \cdot \ncap S }
\lr{ C \Bsigma \cdot \Ba - i \Bsigma \cdot \Ba \Bsigma \cdot \ncap S }
=
C^2 \Bsigma \cdot \Ba + \Bsigma \cdot \ncap \Bsigma \cdot \Ba \Bsigma \cdot \ncap S^2
+ i \lr{
-\Bsigma \cdot \Ba \Bsigma \cdot \ncap
+ \Bsigma \cdot \ncap \Bsigma \cdot \Ba
} S C
=
\inv{2} \lr{ 1 + \cos\phi}
\Bsigma \cdot \Ba
+ \Bsigma \cdot \ncap \Bsigma \cdot \Ba \Bsigma \cdot \ncap \inv{2} \lr{ 1 - \cos\phi}
+ i
\antisymmetric{
\Bsigma \cdot \ncap }{\Bsigma \cdot \Ba }
\inv{2} \sin\phi
=
\inv{2}
\Bsigma \cdot \ncap
\symmetric{
\Bsigma \cdot \ncap }{\Bsigma \cdot \Ba }
+ \inv{2}
\Bsigma \cdot \ncap
\antisymmetric{
\Bsigma \cdot \ncap }{\Bsigma \cdot \Ba } \cos\phi
+
\inv{2}
i
\antisymmetric{
\Bsigma \cdot \ncap }{\Bsigma \cdot \Ba }
\sin\phi.
\end{dmath}

Observe that the angle dependent portion can be written in a compact exponential form

\begin{dmath}\label{eqn:pauliProblems:220}
\Bsigma \cdot \Ba'
=
\inv{2}
\Bsigma \cdot \ncap
\symmetric{
\Bsigma \cdot \ncap }{\Bsigma \cdot \Ba }
+
\lr{
\cos\phi
+
i
\Bsigma \cdot \ncap
\sin\phi
}
\inv{2}
\Bsigma \cdot \ncap
\antisymmetric{
\Bsigma \cdot \ncap }{\Bsigma \cdot \Ba }
=
\inv{2}
\Bsigma \cdot \ncap
\symmetric{
\Bsigma \cdot \ncap }{\Bsigma \cdot \Ba }
+
\exp\lr{ i \Bsigma \cdot \ncap \phi }
\inv{2}
\Bsigma \cdot \ncap
\antisymmetric{
\Bsigma \cdot \ncap }{\Bsigma \cdot \Ba }.
\end{dmath}

The anticommutator and commutator products with the unit normal can be identified as projections and rejections respectively.  Consider the symmetric product first

\begin{dmath}\label{eqn:pauliProblems:240}
\inv{2}
\symmetric{
\Bsigma \cdot \ncap }{\Bsigma \cdot \Ba }
=
\inv{2}
\sum n_r a_s \lr{ \sigma_r \sigma_s + \sigma_s \sigma_r }
=
\inv{2}
\sum_{r \ne s} n_r a_s \lr{ \sigma_r \sigma_s + \sigma_s \sigma_r }
+
\inv{2}
\sum_{r } n_r a_r 2
= 2 \ncap \cdot \Ba.
\end{dmath}

This shows that
\begin{dmath}\label{eqn:pauliProblems:260}
\inv{2}
\Bsigma \cdot \ncap
\symmetric{
\Bsigma \cdot \ncap }{\Bsigma \cdot \Ba }
=
\lr{ \ncap \cdot \Ba } \Bsigma \cdot \ncap,
\end{dmath}

which is the projection of \( \Ba \) in the direction of the normal \( \ncap \).  To show that the commutator term is the rejection, consider the sum of the two

\begin{dmath}\label{eqn:pauliProblems:280}
\inv{2}
\Bsigma \cdot \ncap
\symmetric{
\Bsigma \cdot \ncap }{\Bsigma \cdot \Ba }
+
\inv{2}
\Bsigma \cdot \ncap
\antisymmetric{
\Bsigma \cdot \ncap }{\Bsigma \cdot \Ba }
=
\Bsigma \cdot \ncap
\Bsigma \cdot \ncap \Bsigma \cdot \Ba
=
\Bsigma \cdot \Ba,
\end{dmath}

so we must have

\begin{dmath}\label{eqn:pauliProblems:300}
\Bsigma \cdot \Ba - \lr{ \ncap \cdot \Ba } \Bsigma \cdot \ncap
=
\inv{2}
\Bsigma \cdot \ncap
\antisymmetric{
\Bsigma \cdot \ncap }{\Bsigma \cdot \Ba }.
\end{dmath}

This is the component of \( \Ba \) that has the projection in the \( \ncap \) direction removed.  Looking back to \cref{eqn:pauliProblems:220}, the transformation leaves components of the vector that are colinear with the unit normal unchanged, and applies an exponential operation to the component that lies in what is presumed to be the rotation plane.  To verify that this latter portion of the transformation is a rotation, and to determine the sense of the rotation, let's expand the factor of the sine of \cref{eqn:pauliProblems:200}.

That is

\begin{dmath}\label{eqn:pauliProblems:320}
\frac{i}{2} \antisymmetric{ \Bsigma \cdot \ncap }{\Bsigma \cdot \Ba }
=
\frac{i}{2} \sum n_r a_s \antisymmetric{ \sigma_r }{\sigma_s }
=
\frac{i}{2} \sum n_r a_s 2 i \epsilon_{r s t} \sigma_t
=
- \sum \sigma_t n_r a_s \epsilon_{r s t}
=
-\Bsigma \cdot \lr{ \ncap \cross \Ba }
=
\Bsigma \cdot \lr{ \Ba \cross \ncap }.
\end{dmath}

Since \( \Ba \cross \ncap = \lr{ \Ba - \ncap (\ncap \cdot \Ba) } \cross \ncap \), this vector is seen to lie in the plane normal to \( \ncap \), but perpendicular to the rejection of \( \ncap \) from \( \Ba \).  That completes the demonstration that this is a rotation transformation.

To understand the sense of this rotation, consider \( \ncap = \zcap, \Ba = \xcap \), so

\begin{dmath}\label{eqn:pauliProblems:340}
\Bsigma \cdot \lr{ \Ba \cross \ncap }
=
\Bsigma \cdot \lr{ \xcap \cross \zcap }
=
-\Bsigma \cdot \ycap,
\end{dmath}

and
\begin{dmath}\label{eqn:pauliProblems:360}
\Bsigma \cdot \Ba'
=
\xcap \cos\phi - \ycap \sin\phi,
\end{dmath}

showing that this rotation transformation has a clockwise sense.
} % answer

%\EndArticle

         %
% Copyright � 2015 Peeter Joot.  All Rights Reserved.
% Licenced as described in the file LICENSE under the root directory of this GIT repository.
%
%\newcommand{\authorname}{Peeter Joot}
\newcommand{\email}{peeterjoot@protonmail.com}
\newcommand{\basename}{FIXMEbasenameUndefined}
\newcommand{\dirname}{notes/FIXMEdirnameUndefined/}

%\renewcommand{\basename}{braketManip}
%\renewcommand{\dirname}{notes/phy1520/}
%%\newcommand{\dateintitle}{}
%%\newcommand{\keywords}{}
%
%\newcommand{\authorname}{Peeter Joot}
\newcommand{\onlineurl}{http://sites.google.com/site/peeterjoot2/math2013/\basename.pdf}
\newcommand{\sourcepath}{\dirname\basename.tex}
\newcommand{\generatetitle}[1]{\chapter{#1}}

\newcommand{\vcsinfo}{%
\section*{}
\noindent{\color{DarkOliveGreen}{\rule{\linewidth}{0.1mm}}}
\paragraph{Document version}
%\paragraph{\color{Maroon}{Document version}}
{
\small
\begin{itemize}
\item Available online at:\\ 
\href{\onlineurl}{\onlineurl}
\item Git Repository: \input{./.revinfo/gitRepo.tex}
\item Source: \sourcepath
\item last commit: \input{./.revinfo/gitCommitString.tex}
\item commit date: \input{./.revinfo/gitCommitDate.tex}
\end{itemize}
}
}

%\PassOptionsToPackage{dvipsnames,svgnames}{xcolor}
\PassOptionsToPackage{square,numbers}{natbib}
\documentclass{scrreprt}

\usepackage[left=2cm,right=2cm]{geometry}
\usepackage[svgnames]{xcolor}
\usepackage{peeters_layout}

\usepackage{natbib}

\usepackage[
colorlinks=true,
bookmarks=false,
pdfauthor={\authorname, \email},
backref 
]{hyperref}

% http://tex.stackexchange.com/questions/75773/how-to-reference-problems-by-the-text-label-in-an-exercise-envioronment
\usepackage[english]{cleveref}
\crefname{Exercise}{exercise}{exercises}
\Crefname{Exercise}{Exercise}{Exercises}

\RequirePackage{titlesec}
\RequirePackage{ifthen}

% http://stackoverflow.com/questions/4932910/date-in-the-tabular-environment
\makeatletter
\let\insertdate\@date
\makeatother

\titleformat{\chapter}[display]
{\bfseries\Large}
{\color{DarkSlateGrey}\filleft \authorname
\ifthenelse{\isundefined{\studentnumber}}{}{\\ \studentnumber}
\ifthenelse{\isundefined{\email}}{}{\\ \email}
\ifthenelse{\isundefined{\dateintitle}}{}{\\ \insertdate}
%\ifthenelse{\isundefined{\coursename}}{}{\\ \coursename} % put in title instead.
}
{4ex}
{\color{DarkOliveGreen}{\titlerule}\color{Maroon}
\vspace{2ex}%
\filright}
[\vspace{2ex}%
\color{DarkOliveGreen}\titlerule
]

\newcommand{\beginArtWithToc}[0]{\begin{document}\tableofcontents}
\newcommand{\beginArtNoToc}[0]{\begin{document}}
\newcommand{\EndNoBibArticle}[0]{\end{document}}
\newcommand{\EndArticle}[0]{\bibliography{Bibliography}\bibliographystyle{plainnat}\end{document}}

% 
%\newcommand{\citep}[1]{\cite{#1}}

\colorSectionsForArticle


%
%\usepackage{peeters_layout_exercise}
%\usepackage{peeters_braket}
%
%\beginArtNoToc
%
%\generatetitle{bra-ket manipulation problems}
%%\chapter{bra-ket manipulation problems}
%%\label{chap:braketManip}
\makeoproblem{Some bra-ket manipulation problems.}{problem:braketManip:1.4}{\citep{sakurai2014modern} pr. 1.4}{
\index{braket}
Using braket logic expand

\makesubproblem{}{problem:braketManip:1.4:a}
\begin{equation}\label{eqn:braketManip:20}
\trace{X Y}
\end{equation}
\makesubproblem{}{problem:braketManip:1.4:b}
\begin{equation}\label{eqn:braketManip:40}
(X Y)^\dagger
\end{equation}
\makesubproblem{}{problem:braketManip:1.4:c}
\begin{equation}\label{eqn:braketManip:60}
e^{i f(A)},
\end{equation}

where \( A \) is Hermitian with a complete set of eigenvalues.

\makesubproblem{}{problem:braketManip:1.4:d}
\begin{equation}\label{eqn:braketManip:80}
\sum_{a'} \Psi_{a'}(\Bx')^\conj \Psi_{a'}(\Bx''),
\end{equation}

where \( \Psi_{a'}(\Bx'') = \braket{\Bx'}{a'} \).

} % problem

\makeanswer{problem:braketManip:1.4}{

\makeSubAnswer{}{problem:braketManip:1.4:a}

\begin{dmath}\label{eqn:braketManip:100}
\trace{X Y}
= \sum_a \bra{a} X Y \ket{a}
= \sum_{a,b} \bra{a} X \ket{b}\bra{b} Y \ket{a}
= \sum_{a,b}
\bra{b} Y \ket{a}
\bra{a} X \ket{b}
= \sum_{a,b}
\bra{b} Y
X \ket{b}
= \trace{ Y X }.
\end{dmath}

\makeSubAnswer{}{problem:braketManip:1.4:b}

\begin{dmath}\label{eqn:braketManip:120}
\bra{a} \lr{ X Y}^\dagger \ket{b}
=
\lr{ \bra{b} X Y \ket{a} }^\conj
=
\sum_c \lr{ \bra{b} X \ket{c}\bra{c} Y \ket{a} }^\conj
=
\sum_c \lr{ \bra{b} X \ket{c} }^\conj \lr{ \bra{c} Y \ket{a} }^\conj
=
\sum_c
\lr{ \bra{c} Y \ket{a} }^\conj
\lr{ \bra{b} X \ket{c} }^\conj
=
\sum_c
\bra{a} Y^\dagger \ket{c}
\bra{c} X^\dagger \ket{b}
=
\bra{a} Y^\dagger
X^\dagger \ket{b},
\end{dmath}

so \( \lr{ X Y }^\dagger = Y^\dagger X^\dagger \).

\makeSubAnswer{}{problem:braketManip:1.4:c}

Let's presume that the function \( f \) has a Taylor series representation

\begin{dmath}\label{eqn:braketManip:140}
f(A) = \sum_r b_r A^r.
\end{dmath}

If the eigenvalues of \( A \) are given by

\begin{dmath}\label{eqn:braketManip:160}
A \ket{a_s} = a_s \ket{a_s},
\end{dmath}

this operator can be expanded like

\begin{dmath}\label{eqn:braketManip:180}
A
= \sum_{a_s} A \ket{a_s} \bra{a_s}
= \sum_{a_s} a_s \ket{a_s} \bra{a_s},
\end{dmath}

To compute powers of this operator, consider first the square

\begin{dmath}\label{eqn:braketManip:200}
A^2 =
=
\sum_{a_s} a_s \ket{a_s} \bra{a_s}
\sum_{a_r} a_r \ket{a_r} \bra{a_r}
=
\sum_{a_s, a_r} a_s a_r \ket{a_s} \bra{a_s} \ket{a_r} \bra{a_r}
=
\sum_{a_s, a_r} a_s a_r \ket{a_s} \delta_{s r} \bra{a_r}
=
\sum_{a_s} a_s^2 \ket{a_s} \bra{a_s}.
\end{dmath}

The pattern for higher powers will clearly just be

\begin{dmath}\label{eqn:braketManip:220}
A^k =
\sum_{a_s} a_s^k \ket{a_s} \bra{a_s},
\end{dmath}

so the expansion of \( f(A) \) will be

\begin{dmath}\label{eqn:braketManip:240}
f(A)
= \sum_r b_r A^r
= \sum_r b_r
\sum_{a_s} a_s^r \ket{a_s} \bra{a_s}
=
\sum_{a_s} \lr{ \sum_r b_r a_s^r } \ket{a_s} \bra{a_s}
=
\sum_{a_s} f(a_s) \ket{a_s} \bra{a_s}.
\end{dmath}

The exponential expansion is

\begin{dmath}\label{eqn:braketManip:260}
e^{i f(A)}
=
\sum_t \frac{i^t}{t!} f^t(A)
=
\sum_t \frac{i^t}{t!}
\lr{ \sum_{a_s} f(a_s) \ket{a_s} \bra{a_s} }^t
=
\sum_t \frac{i^t}{t!}
\sum_{a_s} f^t(a_s) \ket{a_s} \bra{a_s}
=
\sum_{a_s}
e^{i f(a_s) }
\ket{a_s} \bra{a_s}.
\end{dmath}

\makeSubAnswer{}{problem:braketManip:1.4:d}

\begin{dmath}\label{eqn:braketManip:99}
\sum_{a'} \Psi_{a'}(\Bx')^\conj \Psi_{a'}(\Bx'')
=
\sum_{a'}
\braket{\Bx'}{a'}^\conj
\braket{\Bx''}{a'}
=
\sum_{a'}
\braket{a'}{\Bx'}
\braket{\Bx''}{a'}
=
\sum_{a'}
\braket{\Bx''}{a'}
\braket{a'}{\Bx'}
=
\braket{\Bx''}{\Bx'}
= \delta\lr{\Bx'' - \Bx'}.
\end{dmath}

} % answer

%\EndArticle

         %
% Copyright � 2015 Peeter Joot.  All Rights Reserved.
% Licenced as described in the file LICENSE under the root directory of this GIT repository.
%
\newcommand{\authorname}{Peeter Joot}
\newcommand{\email}{peeterjoot@protonmail.com}
\newcommand{\basename}{FIXMEbasenameUndefined}
\newcommand{\dirname}{notes/FIXMEdirnameUndefined/}

\renewcommand{\basename}{moreBraKetProblems}
\renewcommand{\dirname}{notes/FIXMEwheretodirname/}
%\newcommand{\dateintitle}{}
%\newcommand{\keywords}{}

\newcommand{\authorname}{Peeter Joot}
\newcommand{\onlineurl}{http://sites.google.com/site/peeterjoot2/math2013/\basename.pdf}
\newcommand{\sourcepath}{\dirname\basename.tex}
\newcommand{\generatetitle}[1]{\chapter{#1}}

\newcommand{\vcsinfo}{%
\section*{}
\noindent{\color{DarkOliveGreen}{\rule{\linewidth}{0.1mm}}}
\paragraph{Document version}
%\paragraph{\color{Maroon}{Document version}}
{
\small
\begin{itemize}
\item Available online at:\\ 
\href{\onlineurl}{\onlineurl}
\item Git Repository: \input{./.revinfo/gitRepo.tex}
\item Source: \sourcepath
\item last commit: \input{./.revinfo/gitCommitString.tex}
\item commit date: \input{./.revinfo/gitCommitDate.tex}
\end{itemize}
}
}

%\PassOptionsToPackage{dvipsnames,svgnames}{xcolor}
\PassOptionsToPackage{square,numbers}{natbib}
\documentclass{scrreprt}

\usepackage[left=2cm,right=2cm]{geometry}
\usepackage[svgnames]{xcolor}
\usepackage{peeters_layout}

\usepackage{natbib}

\usepackage[
colorlinks=true,
bookmarks=false,
pdfauthor={\authorname, \email},
backref 
]{hyperref}

% http://tex.stackexchange.com/questions/75773/how-to-reference-problems-by-the-text-label-in-an-exercise-envioronment
\usepackage[english]{cleveref}
\crefname{Exercise}{exercise}{exercises}
\Crefname{Exercise}{Exercise}{Exercises}

\RequirePackage{titlesec}
\RequirePackage{ifthen}

% http://stackoverflow.com/questions/4932910/date-in-the-tabular-environment
\makeatletter
\let\insertdate\@date
\makeatother

\titleformat{\chapter}[display]
{\bfseries\Large}
{\color{DarkSlateGrey}\filleft \authorname
\ifthenelse{\isundefined{\studentnumber}}{}{\\ \studentnumber}
\ifthenelse{\isundefined{\email}}{}{\\ \email}
\ifthenelse{\isundefined{\dateintitle}}{}{\\ \insertdate}
%\ifthenelse{\isundefined{\coursename}}{}{\\ \coursename} % put in title instead.
}
{4ex}
{\color{DarkOliveGreen}{\titlerule}\color{Maroon}
\vspace{2ex}%
\filright}
[\vspace{2ex}%
\color{DarkOliveGreen}\titlerule
]

\newcommand{\beginArtWithToc}[0]{\begin{document}\tableofcontents}
\newcommand{\beginArtNoToc}[0]{\begin{document}}
\newcommand{\EndNoBibArticle}[0]{\end{document}}
\newcommand{\EndArticle}[0]{\bibliography{Bibliography}\bibliographystyle{plainnat}\end{document}}

% 
%\newcommand{\citep}[1]{\cite{#1}}

\colorSectionsForArticle



\usepackage{peeters_layout_exercise}
\usepackage{peeters_braket}

\beginArtNoToc

\generatetitle{A couple more bra-ket problems}
%\chapter{A couple more bra-ket problems}
%\label{chap:moreBraKetProblems}


\makeoproblem{Operator matrix representation}{problem:moreBraKetProblems:1.5}{\citep{sakurai2014modern} pr. 1.5}{

\makesubproblem{}{problem:moreBraKetProblems:1.5:a}

Determine the matrix representation of \( \ket{\alpha}\bra{\beta} \) given a complete set of eigenvectors \( \ket{a^r} \).

\makesubproblem{}{problem:moreBraKetProblems:1.5:b}

Verify with \( \ket{\alpha} = \ket{s_z = \hbar/2}, \ket{s_x = \hbar/2} \).

} % problem

\makeanswer{problem:moreBraKetProblems:1.5}{

\makeSubAnswer{}{problem:moreBraKetProblems:1.5:a}

Forming the matrix element

\begin{dmath}\label{eqn:moreBraKetProblems:20}
\bra{a^r} \lr{ \ket{\alpha}\bra{\beta} } \ket{a^s}
=
\braket{a^r}{\alpha}\braket{\beta}{a^s}
=
\braket{a^r}{\alpha}
\braket{a^s}{\beta}^\conj,
\end{dmath}

the matrix representation is seen to be

\begin{dmath}\label{eqn:moreBraKetProblems:40}
\ket{\alpha}\bra{\beta}
\sim
\begin{bmatrix}
\bra{a^1} \lr{ \ket{\alpha}\bra{\beta} } \ket{a^1} & \bra{a^1} \lr{ \ket{\alpha}\bra{\beta} } \ket{a^2} & \cdots \\
\bra{a^2} \lr{ \ket{\alpha}\bra{\beta} } \ket{a^1} & \bra{a^2} \lr{ \ket{\alpha}\bra{\beta} } \ket{a^2} & \cdots \\
\vdots & \vdots & \ddots \\
\end{bmatrix}
=
\begin{bmatrix}
\braket{a^1}{\alpha} \braket{a^1}{\beta}^\conj & \braket{a^1}{\alpha} \braket{a^2}{\beta}^\conj & \cdots \\
\braket{a^2}{\alpha} \braket{a^1}{\beta}^\conj & \braket{a^2}{\alpha} \braket{a^2}{\beta}^\conj & \cdots \\
\vdots & \vdots & \ddots \\
\end{bmatrix}.
\end{dmath}

\makeSubAnswer{}{problem:moreBraKetProblems:1.5:b}

First compute the spin-z representation of \( \ket{s_x = \hbar/2 } \).  

\begin{dmath}\label{eqn:moreBraKetProblems:60}
\lr{ S_x - \hbar/2 I }
\begin{bmatrix}
a \\
b
\end{bmatrix}
=
\lr{
\begin{bmatrix}
0 & \hbar/2 \\
\hbar/2 & 0 \\
\end{bmatrix}
-
\begin{bmatrix}
\hbar/2 & 0 \\
0 & \hbar/2 \\
\end{bmatrix}
}
=
\begin{bmatrix}
a \\
b
\end{bmatrix}
=
\frac{\hbar}{2}
\begin{bmatrix}
-1 & 1 \\
1 & -1 \\
\end{bmatrix}
\begin{bmatrix}
a \\
b
\end{bmatrix},
\end{dmath}

so \( \ket{s_x = \hbar/2 } \propto (1,1) \).

Normalized we have

\begin{equation}\label{eqn:moreBraKetProblems:80}
\begin{aligned}
\ket{\alpha} &= \ket{s_z = \hbar/2 } = 
\begin{bmatrix}
1 \\
0
\end{bmatrix} \\
\ket{\beta} &= \ket{s_z = \hbar/2 }
\inv{\sqrt{2}}
\begin{bmatrix}
1 \\
1
\end{bmatrix}.
\end{aligned}
\end{equation}

Using \cref{eqn:moreBraKetProblems:40} the matrix representation is

\begin{dmath}\label{eqn:moreBraKetProblems:100}
\ket{\alpha}\bra{\beta}
\sim
\begin{bmatrix}
(1) (1/\sqrt{2})^\conj & (1) (1/\sqrt{2})^\conj \\
(0) (1/\sqrt{2})^\conj & (0) (1/\sqrt{2})^\conj \\
\end{bmatrix}
=
\inv{\sqrt{2}}
\begin{bmatrix}
1 & 1 \\
0 & 0
\end{bmatrix}.
\end{dmath}

This can be confirmed with direct computation
\begin{dmath}\label{eqn:moreBraKetProblems:120}
\ket{\alpha}\bra{\beta}
=
\begin{bmatrix}
1 \\
0
\end{bmatrix}
\inv{\sqrt{2}}
\begin{bmatrix}
1 & 1
\end{bmatrix}
=
\inv{\sqrt{2}}
\begin{bmatrix}
1 & 1 \\
0 & 0
\end{bmatrix}.
\end{dmath}

} % answer

\EndArticle

         %
% Copyright © 2015 Peeter Joot.  All Rights Reserved.
% Licenced as described in the file LICENSE under the root directory of this GIT repository.
%

\makeoproblem{Spin half probability and dispersion.}{problem:moreBraKetProblems:12}{\citep{sakurai2014modern} pr. 1.12, phy1520 2015 ps1.3}{
\index{spin half!dispersion}
A spin \( 1/2 \) system \( \BS \cdot \ncap \), with \( \ncap = \sin \theta \xcap + \cos\theta \zcap \), is in state with eigenvalue \( \Hbar/2 \).

\makesubproblem{}{problem:moreBraKetProblems:12:a}

If \( S_x \) is measured.  What is the probability of getting \( + \Hbar/2 \)?

\makesubproblem{}{problem:moreBraKetProblems:12:b}

Evaluate the dispersion in \( S_x \), that is,

\begin{equation}\label{eqn:moreBraKetProblems:660}
\expectation{\lr{ S_x - \expectation{S_x}}^2}.
\end{equation}

} % problem

\makeanswer{problem:moreBraKetProblems:12}{

\makeSubAnswer{}{problem:moreBraKetProblems:12:a}

In matrix form the spin operator for the system is

\begin{dmath}\label{eqn:moreBraKetProblems:680}
\BS \cdot \ncap
= \frac{\Hbar}{2} \lr{ \cos\theta \PauliZ + \sin\theta \PauliX}
= \frac{\Hbar}{2}
\begin{bmatrix}
\cos\theta & \sin\theta \\
\sin\theta & -\cos\theta \\
\end{bmatrix}
\end{dmath}

An eigenket \( \ket{\BS \cdot \ncap ; + } = (a,b)^\T \) must satisfy

\begin{dmath}\label{eqn:moreBraKetProblems:700}
0
= \lr{ \cos \theta - 1 } a + \sin\theta b
= \lr{ -2 \sin^2 \frac{\theta}{2} } a + 2 \sin\frac{\theta}{2} \cos\frac{\theta}{2} b
= -\sin \frac{\theta}{2} a + \cos\frac{\theta}{2} b,
\end{dmath}

so the eigenstate is
\begin{dmath}\label{eqn:moreBraKetProblems:720}
\ket{\BS \cdot \ncap ; + }
=
\begin{bmatrix}
\cos\frac{\theta}{2} \\
\sin\frac{\theta}{2}
\end{bmatrix}.
\end{dmath}

Pick \( \ket{S_x ; \pm } = \inv{\sqrt{2}}
\begin{bmatrix}
1 \\ \pm 1
\end{bmatrix} \) as the basis for the \( S_x \) operator.  Then, for the probability that the system will end up in the \( + \Hbar/2 \) state of \( S_x \), we have

\begin{dmath}\label{eqn:moreBraKetProblems:740}
P
= \Abs{\braket{ S_x ; + }{ \BS \cdot \ncap ; + } }^2
= \Abs{ \inv{\sqrt{2} }
{
\begin{bmatrix}
1 \\
1
\end{bmatrix}}^\dagger
\begin{bmatrix}
\cos\frac{\theta}{2} \\
\sin\frac{\theta}{2}
\end{bmatrix}
}^2
=\inv{2}
\Abs{
\begin{bmatrix}
1 & 1
\end{bmatrix}
\begin{bmatrix}
\cos\frac{\theta}{2} \\
\sin\frac{\theta}{2}
\end{bmatrix}
}^2
=
\inv{2}
\lr{
\cos\frac{\theta}{2} +
\sin\frac{\theta}{2}
}^2
=
\inv{2}
\lr{ 1 + 2 \cos\frac{\theta}{2} \sin\frac{\theta}{2} }
=
\inv{2}
\lr{ 1 + \sin\theta }.
\end{dmath}

This is a reasonable seeming result, with \( P \in [0, 1] \).  Some special values also further validate this

\begin{equation}\label{eqn:moreBraKetProblems:760}
\begin{aligned}
\theta &= 0, \ket{\BS \cdot \ncap ; + } =
\begin{bmatrix}
1 \\
0
\end{bmatrix}
=
\ket{S_z ; +}
=
\inv{\sqrt{2}} \ket{S_x;+}
+\inv{\sqrt{2}} \ket{S_x;-}
\\
\theta &= \pi/2, \ket{\BS \cdot \ncap ; + } =
\inv{\sqrt{2}}
\begin{bmatrix}
1 \\
1
\end{bmatrix}
=
\ket{S_x ; +}
\\
\theta &= \pi, \ket{\BS \cdot \ncap ; + } =
\begin{bmatrix}
0 \\
1
\end{bmatrix}
=
\ket{S_z ; -}
=
\inv{\sqrt{2}} \ket{S_x;+}
-\inv{\sqrt{2}} \ket{S_x;-},
\end{aligned}
\end{equation}

where we see that the probabilities are in proportion to the projection of the initial state onto the measured state \( \ket{S_x ; +} \).

\makeSubAnswer{}{problem:moreBraKetProblems:12:b}

The \( S_x \) expectation is

\begin{dmath}\label{eqn:moreBraKetProblems:780}
\expectation{S_x}
=
\frac{\Hbar}{2}
\begin{bmatrix}
\cos\frac{\theta}{2} & \sin\frac{\theta}{2}
\end{bmatrix}
\PauliX
\begin{bmatrix}
\cos\frac{\theta}{2} \\
\sin\frac{\theta}{2}
\end{bmatrix}
=
\frac{\Hbar}{2}
\begin{bmatrix}
\cos\frac{\theta}{2} & \sin\frac{\theta}{2}
\end{bmatrix}
\begin{bmatrix}
\sin\frac{\theta}{2} \\
\cos\frac{\theta}{2}
\end{bmatrix}
=
\frac{\Hbar}{2} 2 \sin\frac{\theta}{2} \cos\frac{\theta}{2}
=
\frac{\Hbar}{2} \sin\theta.
\end{dmath}

Note that \( S_x^2 = (\Hbar/2)^2I \), so

\begin{dmath}\label{eqn:moreBraKetProblems:800}
\expectation{S_x^2}
=
\lr{\frac{\Hbar}{2}}^2
\begin{bmatrix}
\cos\frac{\theta}{2} & \sin\frac{\theta}{2}
\end{bmatrix}
\begin{bmatrix}
\cos\frac{\theta}{2} \\
\sin\frac{\theta}{2}
\end{bmatrix}
=
\lr{ \frac{\Hbar}{2} }^2
\cos^2\frac{\theta}{2} + \sin^2 \frac{\theta}{2}
=
\lr{ \frac{\Hbar}{2} }^2.
\end{dmath}

The dispersion is

\begin{dmath}\label{eqn:moreBraKetProblems:820}
\expectation{\lr{ S_x - \expectation{S_x}}^2}
=
\expectation{S_x^2} - \expectation{S_x}^2
=
\lr{ \frac{\Hbar}{2} }^2
\lr{1 - \sin^2 \theta}
=
\lr{ \frac{\Hbar}{2} }^2
\cos^2 \theta.
\end{dmath}

At \( \theta = \pi/2 \) the dispersion is 0, which is expected since \( \ket{\BS \cdot \ncap ; + } = \ket{ S_x ; + } \) at that point.  Similarly, the dispersion is maximized at \( \theta = 0,\pi \) where the \( \ket{\BS \cdot \ncap ; + } \) component in the \( \ket{S_x ; + } \) direction is minimized.

} % answer

         % p13:
         %
% Copyright � 2015 Peeter Joot.  All Rights Reserved.
% Licenced as described in the file LICENSE under the root directory of this GIT repository.
%
%\newcommand{\authorname}{Peeter Joot}
\newcommand{\email}{peeterjoot@protonmail.com}
\newcommand{\basename}{FIXMEbasenameUndefined}
\newcommand{\dirname}{notes/FIXMEdirnameUndefined/}

%\renewcommand{\basename}{sg}
%\renewcommand{\dirname}{notes/phy1520/}
%%\newcommand{\dateintitle}{}
%%\newcommand{\keywords}{}
%
%\newcommand{\authorname}{Peeter Joot}
\newcommand{\onlineurl}{http://sites.google.com/site/peeterjoot2/math2013/\basename.pdf}
\newcommand{\sourcepath}{\dirname\basename.tex}
\newcommand{\generatetitle}[1]{\chapter{#1}}

\newcommand{\vcsinfo}{%
\section*{}
\noindent{\color{DarkOliveGreen}{\rule{\linewidth}{0.1mm}}}
\paragraph{Document version}
%\paragraph{\color{Maroon}{Document version}}
{
\small
\begin{itemize}
\item Available online at:\\ 
\href{\onlineurl}{\onlineurl}
\item Git Repository: \input{./.revinfo/gitRepo.tex}
\item Source: \sourcepath
\item last commit: \input{./.revinfo/gitCommitString.tex}
\item commit date: \input{./.revinfo/gitCommitDate.tex}
\end{itemize}
}
}

%\PassOptionsToPackage{dvipsnames,svgnames}{xcolor}
\PassOptionsToPackage{square,numbers}{natbib}
\documentclass{scrreprt}

\usepackage[left=2cm,right=2cm]{geometry}
\usepackage[svgnames]{xcolor}
\usepackage{peeters_layout}

\usepackage{natbib}

\usepackage[
colorlinks=true,
bookmarks=false,
pdfauthor={\authorname, \email},
backref 
]{hyperref}

% http://tex.stackexchange.com/questions/75773/how-to-reference-problems-by-the-text-label-in-an-exercise-envioronment
\usepackage[english]{cleveref}
\crefname{Exercise}{exercise}{exercises}
\Crefname{Exercise}{Exercise}{Exercises}

\RequirePackage{titlesec}
\RequirePackage{ifthen}

% http://stackoverflow.com/questions/4932910/date-in-the-tabular-environment
\makeatletter
\let\insertdate\@date
\makeatother

\titleformat{\chapter}[display]
{\bfseries\Large}
{\color{DarkSlateGrey}\filleft \authorname
\ifthenelse{\isundefined{\studentnumber}}{}{\\ \studentnumber}
\ifthenelse{\isundefined{\email}}{}{\\ \email}
\ifthenelse{\isundefined{\dateintitle}}{}{\\ \insertdate}
%\ifthenelse{\isundefined{\coursename}}{}{\\ \coursename} % put in title instead.
}
{4ex}
{\color{DarkOliveGreen}{\titlerule}\color{Maroon}
\vspace{2ex}%
\filright}
[\vspace{2ex}%
\color{DarkOliveGreen}\titlerule
]

\newcommand{\beginArtWithToc}[0]{\begin{document}\tableofcontents}
\newcommand{\beginArtNoToc}[0]{\begin{document}}
\newcommand{\EndNoBibArticle}[0]{\end{document}}
\newcommand{\EndArticle}[0]{\bibliography{Bibliography}\bibliographystyle{plainnat}\end{document}}

% 
%\newcommand{\citep}[1]{\cite{#1}}

\colorSectionsForArticle


%
%\usepackage{peeters_layout_exercise}
%\usepackage{peeters_braket}
%\usepackage{peeters_figures}
%
%\beginArtNoToc
%
%\generatetitle{Cascading Stern-Gerlach}
%\chapter{Cascading Stern-Gerlach}
%\label{chap:sg}

\makeoproblem{Cascading Stern-Gerlach.}{problem:sg:13}{\citep{sakurai2014modern} pr. 1.13}{
\index{Stern-Gerlach}

Three Stern-Gerlach type measurements are performed, the first that prepares the state in a \( \ket{S_z ; + } \) state, the next in a \( \ket{ \BS \cdot \ncap ; + } \) state where \( \ncap = \cos\beta \zcap + \sin\beta \xcap \), and the last performing a \( S_z \) \( \Hbar/2 \) state measurement, as illustrated in \cref{fig:sternGerlach:sternGerlachFig1}.

\imageFigure{../../figures/phy1520/sternGerlachFig1}{Cascaded Stern-Gerlach type measurements.}{fig:sternGerlach:sternGerlachFig1}{0.3}

What is the intensity of the final \( s_z = -\Hbar/2 \) beam?  What is the orientation for the second measuring apparatus to maximize the intensity of this beam?
} % problem

\makeanswer{problem:sg:13}{

The spin operator for the second apparatus is

\begin{dmath}\label{eqn:sg:20}
\BS \cdot \ncap
= \frac{\Hbar}{2} \lr{ \sin\beta \PauliX + \cos\beta \PauliZ }
= \frac{\Hbar}{2}
\begin{bmatrix}
\cos\beta & \sin\beta \\
\sin\beta & -\cos\beta
\end{bmatrix}.
\end{dmath}

The intensity of the final \( \ket{S_z ; -} \) beam is

\begin{dmath}\label{eqn:sg:40}
P
= \Abs{ \braket{-}{\BS \cdot \ncap ; +} \braket{\BS \cdot \ncap ; +}{+} }^2,
\end{dmath}

(i.e. the second apparatus applies a projection operator \( \ket{\BS \cdot \ncap ; +}\bra{\BS \cdot \ncap ; +} \) to the initial \( \ket{+} \) state, and then the \( \ket{-} \) states are selected out of that.

The \( \BS \cdot \ncap \) eigenket is found to be

\begin{dmath}\label{eqn:sg:60}
\ket{\BS \cdot \ncap ; +} =
\begin{bmatrix}
\cos\frac{\beta}{2} \\
\sin\frac{\beta}{2} \\
\end{bmatrix},
\end{dmath}

so

\begin{dmath}\label{eqn:sg:80}
P
= \Abs{
\begin{bmatrix}
0 & 1
\end{bmatrix}
\begin{bmatrix}
\cos\frac{\beta}{2} \\
\sin\frac{\beta}{2} \\
\end{bmatrix}
\begin{bmatrix}
\cos\frac{\beta}{2} &
\sin\frac{\beta}{2} \\
\end{bmatrix}
\begin{bmatrix}
1 \\
0
\end{bmatrix}
}^2
=
\Abs{
\cos\frac{\beta}{2}
\sin\frac{\beta}{2}
}^2
=
\Abs{\inv{2} \sin\beta}^2
=
\inv{4} \sin^2\beta.
\end{dmath}

This is maximized when \( \beta = \pi/2 \), or \( \ncap = \xcap \).  At this angle the state leaving the second apparatus is

\begin{dmath}\label{eqn:sg:100}
\begin{bmatrix}
\cos\frac{\beta}{2} \\
\sin\frac{\beta}{2} \\
\end{bmatrix}
\begin{bmatrix}
\cos\frac{\beta}{2} &
\sin\frac{\beta}{2} \\
\end{bmatrix}
\begin{bmatrix}
1 \\
0
\end{bmatrix}
=
\inv{2}
\begin{bmatrix}
1 \\ 1
\end{bmatrix}
\begin{bmatrix}
1 & 1
\end{bmatrix}
\begin{bmatrix}
1 \\ 0
\end{bmatrix}
=
\inv{2}
\begin{bmatrix}
1 \\ 1
\end{bmatrix}
=\inv{2} \ket{+} + \inv{2}\ket{-},
\end{dmath}

so the state after filtering the \( \ket{-} \) states is \( \inv{2} \ket{-} \) with intensity (probability density) of \( 1/4 \) relative to a unit normalize input \( \ket{+} \) state to the \( \BS \cdot \ncap \) apparatus.

} % answer

%\EndArticle

         % p16:
         %
% Copyright � 2015 Peeter Joot.  All Rights Reserved.
% Licenced as described in the file LICENSE under the root directory of this GIT repository.
%
%\newcommand{\authorname}{Peeter Joot}
\newcommand{\email}{peeterjoot@protonmail.com}
\newcommand{\basename}{FIXMEbasenameUndefined}
\newcommand{\dirname}{notes/FIXMEdirnameUndefined/}

%\renewcommand{\basename}{anticommutingOperatorWithSimulaneousEigenket}
%\renewcommand{\dirname}{notes/phy1520/}
%%\newcommand{\dateintitle}{}
%%\newcommand{\keywords}{}
%
%\newcommand{\authorname}{Peeter Joot}
\newcommand{\onlineurl}{http://sites.google.com/site/peeterjoot2/math2013/\basename.pdf}
\newcommand{\sourcepath}{\dirname\basename.tex}
\newcommand{\generatetitle}[1]{\chapter{#1}}

\newcommand{\vcsinfo}{%
\section*{}
\noindent{\color{DarkOliveGreen}{\rule{\linewidth}{0.1mm}}}
\paragraph{Document version}
%\paragraph{\color{Maroon}{Document version}}
{
\small
\begin{itemize}
\item Available online at:\\ 
\href{\onlineurl}{\onlineurl}
\item Git Repository: \input{./.revinfo/gitRepo.tex}
\item Source: \sourcepath
\item last commit: \input{./.revinfo/gitCommitString.tex}
\item commit date: \input{./.revinfo/gitCommitDate.tex}
\end{itemize}
}
}

%\PassOptionsToPackage{dvipsnames,svgnames}{xcolor}
\PassOptionsToPackage{square,numbers}{natbib}
\documentclass{scrreprt}

\usepackage[left=2cm,right=2cm]{geometry}
\usepackage[svgnames]{xcolor}
\usepackage{peeters_layout}

\usepackage{natbib}

\usepackage[
colorlinks=true,
bookmarks=false,
pdfauthor={\authorname, \email},
backref 
]{hyperref}

% http://tex.stackexchange.com/questions/75773/how-to-reference-problems-by-the-text-label-in-an-exercise-envioronment
\usepackage[english]{cleveref}
\crefname{Exercise}{exercise}{exercises}
\Crefname{Exercise}{Exercise}{Exercises}

\RequirePackage{titlesec}
\RequirePackage{ifthen}

% http://stackoverflow.com/questions/4932910/date-in-the-tabular-environment
\makeatletter
\let\insertdate\@date
\makeatother

\titleformat{\chapter}[display]
{\bfseries\Large}
{\color{DarkSlateGrey}\filleft \authorname
\ifthenelse{\isundefined{\studentnumber}}{}{\\ \studentnumber}
\ifthenelse{\isundefined{\email}}{}{\\ \email}
\ifthenelse{\isundefined{\dateintitle}}{}{\\ \insertdate}
%\ifthenelse{\isundefined{\coursename}}{}{\\ \coursename} % put in title instead.
}
{4ex}
{\color{DarkOliveGreen}{\titlerule}\color{Maroon}
\vspace{2ex}%
\filright}
[\vspace{2ex}%
\color{DarkOliveGreen}\titlerule
]

\newcommand{\beginArtWithToc}[0]{\begin{document}\tableofcontents}
\newcommand{\beginArtNoToc}[0]{\begin{document}}
\newcommand{\EndNoBibArticle}[0]{\end{document}}
\newcommand{\EndArticle}[0]{\bibliography{Bibliography}\bibliographystyle{plainnat}\end{document}}

% 
%\newcommand{\citep}[1]{\cite{#1}}

\colorSectionsForArticle


%
%\usepackage{peeters_layout_exercise}
%\usepackage{peeters_braket}
%\usepackage{peeters_figures}
%
%\beginArtNoToc
%
%\generatetitle{Can anticommuting operators have a simultaneous eigenket?}
%\chapter{Can anticommuting operators have a simulaneous eigenket?}
%\label{chap:anticommutingOperatorWithSimulaneousEigenket}

\makeoproblem{Can anticommuting operators have a simultaneous eigenket?}{problem:anticommutingOperatorWithSimulaneousEigenket:1}{\citep{sakurai2014modern} pr. 1.16}{
\index{simultaneous eigenstate}

Two Hermitian operators anticommute

\begin{dmath}\label{eqn:anticommutingOperatorWithSimulaneousEigenket:20}
\symmetric{A}{B} = A B + B A = 0.
\end{dmath}

Is it possible to have a simultaneous eigenket of \( A \) and \( B \)?  Prove or illustrate your assertion.
} % problem

\makeanswer{problem:anticommutingOperatorWithSimulaneousEigenket:1}{

Suppose that such a simultaneous non-zero eigenket \( \ket{\alpha} \) exists, then

\begin{dmath}\label{eqn:anticommutingOperatorWithSimulaneousEigenket:40}
A \ket{\alpha} = a \ket{\alpha},
\end{dmath}

and

\begin{dmath}\label{eqn:anticommutingOperatorWithSimulaneousEigenket:60}
B \ket{\alpha} = b \ket{\alpha}
\end{dmath}

This gives

\begin{dmath}\label{eqn:anticommutingOperatorWithSimulaneousEigenket:80}
\lr{ A B + B A } \ket{\alpha}
=
\lr{A b + B a} \ket{\alpha}
= 2 a b \ket{\alpha}.
\end{dmath}

If this is zero, one of the operators must have a zero eigenvalue.  Knowing that we can construct an example of such operators.  In matrix form, let

\begin{subequations}
\label{eqn:anticommutingOperatorWithSimulaneousEigenket:100}
\begin{dmath}\label{eqn:anticommutingOperatorWithSimulaneousEigenket:120}
A =
\begin{bmatrix}
1 & 0 & 0 \\
0 & -1 & 0 \\
0 & 0 & a \\
\end{bmatrix}
\end{dmath}
\begin{dmath}\label{eqn:anticommutingOperatorWithSimulaneousEigenket:140}
B =
\begin{bmatrix}
0 & 1 & 0 \\
1 & 0 & 0 \\
0 & 0 & b \\
\end{bmatrix}.
\end{dmath}
\end{subequations}

These are both Hermitian, and anticommute provided at least one of \( a, b\) is zero.  These have a common eigenket

\begin{dmath}\label{eqn:anticommutingOperatorWithSimulaneousEigenket:160}
\ket{\alpha} =
\begin{bmatrix}
0 \\
0 \\
1
\end{bmatrix}.
\end{dmath}

A zero eigenvalue of one of the commuting operators may not be a sufficient condition for such anticommutation.
%It also appears that not all Hermitian matrices that anticommute, where one has a zero eigenvalue, necessarily have a common eigenket.  An example is
%
%\begin{subequations}
%\label{eqn:anticommutingOperatorWithSimulaneousEigenket:180}
%\begin{dmath}\label{eqn:anticommutingOperatorWithSimulaneousEigenket:200}
%A =
%\begin{bmatrix}
%1 & 0 & 0 & 0 \\
%0 & -1 & 0  & 0\\
%0 & 0 & 1  & 0\\
%0 & 0 & 0  & 0\\
%\end{bmatrix}
%\end{dmath}
%\begin{dmath}\label{eqn:anticommutingOperatorWithSimulaneousEigenket:220}
%B =
%\begin{bmatrix}
%0 & 1 & 0 & 0 \\
%1 & 0 & 0 & 0 \\
%0 & 0 & 0 & 0 \\
%0 & 0 & 0 & 1 \\
%\end{bmatrix}.
%\end{dmath}
%\end{subequations}
%
%The eigenkets for the zero eigenvalues for \( A \) and \( B \) are \( (0,0,0,1) \) and \( (0,0,1,0) \) respectively, but neither of these are a common eigenket.
} % answer

%\EndArticle

         % p17:
         %
% Copyright � 2015 Peeter Joot.  All Rights Reserved.
% Licenced as described in the file LICENSE under the root directory of this GIT repository.
%
%\newcommand{\authorname}{Peeter Joot}
\newcommand{\email}{peeterjoot@protonmail.com}
\newcommand{\basename}{FIXMEbasenameUndefined}
\newcommand{\dirname}{notes/FIXMEdirnameUndefined/}

%\renewcommand{\basename}{angularMomentumAndCentralForceCommutators}
%\renewcommand{\dirname}{notes/phy1520/}
%%\newcommand{\dateintitle}{}
%%\newcommand{\keywords}{}
%
%\newcommand{\authorname}{Peeter Joot}
\newcommand{\onlineurl}{http://sites.google.com/site/peeterjoot2/math2013/\basename.pdf}
\newcommand{\sourcepath}{\dirname\basename.tex}
\newcommand{\generatetitle}[1]{\chapter{#1}}

\newcommand{\vcsinfo}{%
\section*{}
\noindent{\color{DarkOliveGreen}{\rule{\linewidth}{0.1mm}}}
\paragraph{Document version}
%\paragraph{\color{Maroon}{Document version}}
{
\small
\begin{itemize}
\item Available online at:\\ 
\href{\onlineurl}{\onlineurl}
\item Git Repository: \input{./.revinfo/gitRepo.tex}
\item Source: \sourcepath
\item last commit: \input{./.revinfo/gitCommitString.tex}
\item commit date: \input{./.revinfo/gitCommitDate.tex}
\end{itemize}
}
}

%\PassOptionsToPackage{dvipsnames,svgnames}{xcolor}
\PassOptionsToPackage{square,numbers}{natbib}
\documentclass{scrreprt}

\usepackage[left=2cm,right=2cm]{geometry}
\usepackage[svgnames]{xcolor}
\usepackage{peeters_layout}

\usepackage{natbib}

\usepackage[
colorlinks=true,
bookmarks=false,
pdfauthor={\authorname, \email},
backref 
]{hyperref}

% http://tex.stackexchange.com/questions/75773/how-to-reference-problems-by-the-text-label-in-an-exercise-envioronment
\usepackage[english]{cleveref}
\crefname{Exercise}{exercise}{exercises}
\Crefname{Exercise}{Exercise}{Exercises}

\RequirePackage{titlesec}
\RequirePackage{ifthen}

% http://stackoverflow.com/questions/4932910/date-in-the-tabular-environment
\makeatletter
\let\insertdate\@date
\makeatother

\titleformat{\chapter}[display]
{\bfseries\Large}
{\color{DarkSlateGrey}\filleft \authorname
\ifthenelse{\isundefined{\studentnumber}}{}{\\ \studentnumber}
\ifthenelse{\isundefined{\email}}{}{\\ \email}
\ifthenelse{\isundefined{\dateintitle}}{}{\\ \insertdate}
%\ifthenelse{\isundefined{\coursename}}{}{\\ \coursename} % put in title instead.
}
{4ex}
{\color{DarkOliveGreen}{\titlerule}\color{Maroon}
\vspace{2ex}%
\filright}
[\vspace{2ex}%
\color{DarkOliveGreen}\titlerule
]

\newcommand{\beginArtWithToc}[0]{\begin{document}\tableofcontents}
\newcommand{\beginArtNoToc}[0]{\begin{document}}
\newcommand{\EndNoBibArticle}[0]{\end{document}}
\newcommand{\EndArticle}[0]{\bibliography{Bibliography}\bibliographystyle{plainnat}\end{document}}

% 
%\newcommand{\citep}[1]{\cite{#1}}

\colorSectionsForArticle


%
%\usepackage{peeters_layout_exercise}
%\usepackage{peeters_braket}
%\usepackage{peeters_figures}
%
%\beginArtNoToc
%
%\generatetitle{Commutators of angular momentum and a central force Hamiltonian}
%%\chapter{Commutators of angular momentum and a central force Hamiltonian}
%%\label{chap:angularMomentumAndCentralForceCommutators}

\makeoproblem{Degeneracy in non-commuting observables that both commute with the Hamiltonian.}{problem:angularMomentumAndCentralForceCommutators:1}{\citep{sakurai2014modern} pr. 1.17}{
\index{degeneracy}
\index{non-commuting observables}

Show that non-commuting operators that both commute with the Hamiltonian, have, in general, degenerate energy eigenvalues.  That is

\begin{equation}\label{eqn:angularMomentumAndCentralForceCommutators:320}
[A,H] = [B,H] = 0,
\end{equation}

but

\begin{dmath}\label{eqn:angularMomentumAndCentralForceCommutators:340}
[A,B] \ne 0.
\end{dmath}

\makesubproblem{}{problem:angularMomentumAndCentralForceCommutators:1:a}

Consider \( L_x, L_z \) and a central force Hamiltonian \( H = \Bp^2/2m + V(r) \) as examples.

\makesubproblem{}{problem:angularMomentumAndCentralForceCommutators:1:b}

Construct some simple matrix examples that illustrate the degeneracy conditions.

\makesubproblem{}{problem:angularMomentumAndCentralForceCommutators:1:c}

Prove the general case.

} % problem

\makeanswer{problem:angularMomentumAndCentralForceCommutators:1}{

\makeSubAnswer{}{problem:angularMomentumAndCentralForceCommutators:1:a}

Let's start with demonstrate these commutators act as expected in these cases.

With \( \BL = \Bx \cross \Bp \), we have

\begin{equation}\label{eqn:angularMomentumAndCentralForceCommutators:20}
\begin{aligned}
L_x &= y p_z - z p_y \\
L_y &= z p_x - x p_z \\
L_z &= x p_y - y p_x.
\end{aligned}
\end{equation}

The \( L_x, L_z \) commutator is

\begin{equation}\label{eqn:angularMomentumAndCentralForceCommutators:40}
\begin{aligned}
\antisymmetric{L_x}{L_z}
&=
\antisymmetric{y p_z - z p_y }{x p_y - y p_x} \\
&=
\antisymmetric{y p_z}{x p_y}
-\antisymmetric{y p_z}{y p_x}
-\antisymmetric{z p_y }{x p_y}
+\antisymmetric{z p_y }{y p_x} \\
&=
x p_z \antisymmetric{y}{p_y}
+ z p_x \antisymmetric{p_y }{y} \\
&=
i \Hbar \lr{ x p_z - z p_x } \\
&=
- i \Hbar L_y
\end{aligned}
\end{equation}

cyclically permuting the indexes shows that no pairs of different \( \BL \) components commute.  For \( L_y, L_x \) that is

\begin{equation}\label{eqn:angularMomentumAndCentralForceCommutators:60}
\begin{aligned}
\antisymmetric{L_y}{L_x}
&=
\antisymmetric{z p_x - x p_z }{y p_z - z p_y} \\
&=
\antisymmetric{z p_x}{y p_z}
-\antisymmetric{z p_x}{z p_y}
-\antisymmetric{x p_z }{y p_z}
+\antisymmetric{x p_z }{z p_y} \\
&=
y p_x \antisymmetric{z}{p_z}
+ x p_y \antisymmetric{p_z }{z} \\
&=
i \Hbar \lr{ y p_x - x p_y } \\
&=
- i \Hbar L_z,
\end{aligned}
\end{equation}

and for \( L_z, L_y \)

\begin{equation}\label{eqn:angularMomentumAndCentralForceCommutators:80}
\begin{aligned}
\antisymmetric{L_z}{L_y}
&=
\antisymmetric{x p_y - y p_x }{z p_x - x p_z} \\
&=
\antisymmetric{x p_y}{z p_x}
-\antisymmetric{x p_y}{x p_z}
-\antisymmetric{y p_x }{z p_x}
+\antisymmetric{y p_x }{x p_z} \\
&=
z p_y \antisymmetric{x}{p_x}
+ y p_z \antisymmetric{p_x }{x} \\
&=
i \Hbar \lr{ z p_y - y p_z } \\
&=
- i \Hbar L_x.
\end{aligned}
\end{equation}

If these angular momentum components are also shown to commute with themselves (which they do), the commutator relations above can be summarized as

\index{central force potential}
\begin{equation}\label{eqn:angularMomentumAndCentralForceCommutators:100}
\antisymmetric{L_a}{L_b} = i \Hbar \epsilon_{a b c} L_c.
\end{equation}

In the example to consider, we'll have to consider the commutators with \( \Bp^2 \) and \( V(r) \).  Picking any one component of \( \BL \) is sufficient due to the symmetries of the problem.  For example

\begin{equation}\label{eqn:angularMomentumAndCentralForceCommutators:120}
\begin{aligned}
\antisymmetric{L_x}{\Bp^2}
&=
\antisymmetric{y p_z - z p_y}{p_x^2 + p_y^2 + p_z^2} \\
&=
\antisymmetric{y p_z}{\cancel{p_x^2} + p_y^2 + \cancel{p_z^2}}
-\antisymmetric{z p_y}{\cancel{p_x^2} + \cancel{p_y^2} + p_z^2} \\
&=
p_z \antisymmetric{y}{p_y^2}
-p_y \antisymmetric{z}{p_z^2} \\
&=
p_z 2 i \Hbar p_y
-p_y 2 i \Hbar p_z  \\
&=
0.
\end{aligned}
\end{equation}

How about the commutator of \( \BL \) with the potential?  It is sufficient to consider one component again, for example

\begin{equation}\label{eqn:angularMomentumAndCentralForceCommutators:140}
\begin{aligned}
\antisymmetric{L_x}{V}
&=
\antisymmetric{y p_z - z p_y}{V} \\
&=
y \antisymmetric{p_z}{V} - z \antisymmetric{p_y}{V} \\
&=
-i \Hbar y \PD{z}{V(r)} + i \Hbar z \PD{y}{V(r)} \\
&=
-i \Hbar y \PD{r}{V}\PD{z}{r} + i \Hbar z \PD{r}{V}\PD{y}{r}  \\
&=
-i \Hbar y \PD{r}{V} \frac{z}{r} + i \Hbar z \PD{r}{V}\frac{y}{r}  \\
&=
0.
\end{aligned}
\end{equation}

This has shown that all the components of \( \BL \) commute with a central force Hamiltonian, and each different component of \( \BL \) do not commute.  It does not demonstrate the degeneracy, but I do recall that exists for this system.

%\paragraph{Matrix example of non-commuting commutators}
\makeSubAnswer{}{problem:angularMomentumAndCentralForceCommutators:1:b}

I thought perhaps the problem at hand would be easier if I were to construct some example matrices representing operators that did not commute, but did commuted with a Hamiltonian.  I came up with

\begin{equation}\label{eqn:angularMomentumAndCentralForceCommutators:360}
\begin{aligned}
A &=
\begin{bmatrix}
\sigma_z & 0 \\
0 & 1
\end{bmatrix}
=
\begin{bmatrix}
 1 & 0 & 0 \\
 0 & -1 & 0 \\
 0 & 0 & 1 \\
\end{bmatrix} \\
B &=
\begin{bmatrix}
\sigma_x & 0 \\
0 & 1
\end{bmatrix}
=
\begin{bmatrix}
 0 & 1 & 0 \\
 1 & 0 & 0 \\
 0 & 0 & 1 \\
\end{bmatrix} \\
H &=
\begin{bmatrix}
 0 & 0 & 0 \\
 0 & 0 & 0 \\
 0 & 0 & 1 \\
\end{bmatrix}
\end{aligned}
\end{equation}

This system has \( \antisymmetric{A}{H} = \antisymmetric{B}{H} = 0 \), and

\begin{dmath}\label{eqn:angularMomentumAndCentralForceCommutators:380}
\antisymmetric{A}{B}
=
\begin{bmatrix}
 0 & 2 & 0 \\
-2 & 0 & 0 \\
 0 & 0 & 0 \\
\end{bmatrix}
\end{dmath}

There is one shared eigenvector between all of \( A, B, H \)

\begin{dmath}\label{eqn:angularMomentumAndCentralForceCommutators:400}
\ket{3} =
\begin{bmatrix}
0 \\
0 \\
1
\end{bmatrix}.
\end{dmath}

The other eigenvectors for \( A \) are
\begin{equation}\label{eqn:angularMomentumAndCentralForceCommutators:420}
\begin{aligned}
\ket{a_1} &=
\begin{bmatrix}
1 \\
0 \\
0
\end{bmatrix} \\
\ket{a_2} &=
\begin{bmatrix}
0 \\
1 \\
0
\end{bmatrix},
\end{aligned}
\end{equation}

and for \( B \)
\begin{equation}\label{eqn:angularMomentumAndCentralForceCommutators:440}
\begin{aligned}
\ket{b_1} &=
\inv{\sqrt{2}}
\begin{bmatrix}
1 \\
1 \\
0
\end{bmatrix} \\
\ket{b_2} &=
\inv{\sqrt{2}}
\begin{bmatrix}
1 \\
-1 \\
0
\end{bmatrix},
\end{aligned}
\end{equation}

This clearly has the degeneracy sought.

Looking to \citep{commutingMatrices}, it appears that it is possible to construct an even simpler example.  Let

\begin{equation}\label{eqn:angularMomentumAndCentralForceCommutators:460}
\begin{aligned}
A &=
\begin{bmatrix}
0 & 1 \\
0 & 0
\end{bmatrix} \\
B &=
\begin{bmatrix}
1 & 0 \\
0 & 0
\end{bmatrix} \\
H &=
\begin{bmatrix}
0 & 0 \\
0 & 0
\end{bmatrix}.
\end{aligned}
\end{equation}

Here \( \antisymmetric{A}{B} = -A \), and \( \antisymmetric{A}{H} = \antisymmetric{B}{H} = 0 \), but the Hamiltonian isn't interesting at all physically.

A less boring example builds on this.  Let

\begin{equation}\label{eqn:angularMomentumAndCentralForceCommutators:480}
\begin{aligned}
A &=
\begin{bmatrix}
0 & 1 & 0 \\
0 & 0 & 0 \\
0 & 0 & 1
\end{bmatrix} \\
B &=
\begin{bmatrix}
1 & 0 & 0 \\
0 & 0 & 0 \\
0 & 0 & 1
\end{bmatrix} \\
H &=
\begin{bmatrix}
0 & 0 & 0 \\
0 & 0 & 0 \\
0 & 0 & 1 \\
\end{bmatrix}.
\end{aligned}
\end{equation}

Here \( \antisymmetric{A}{B} \ne 0 \), and \( \antisymmetric{A}{H} = \antisymmetric{B}{H} = 0 \).  I don't see a way for any exception to be constructed.

%\paragraph{The problem}
\makeSubAnswer{}{problem:angularMomentumAndCentralForceCommutators:1:c}

The concrete examples above give some intuition for solving the more abstract problem.  Suppose that we are working in a basis that simultaneously diagonalizes operator \( A \) and the Hamiltonian \( H \).  To make life easy consider the simplest case where this basis is also an eigenbasis for the second operator \( B \) for all but two of that operators eigenvectors.  For such a system let's write

\begin{equation}\label{eqn:angularMomentumAndCentralForceCommutators:160}
\begin{aligned}
H \ket{1} &= \epsilon_1 \ket{1} \\
H \ket{2} &= \epsilon_2 \ket{2} \\
A \ket{1} &= a_1 \ket{1} \\
A \ket{2} &= a_2 \ket{2},
\end{aligned}
\end{equation}

where \( \ket{1}\), and \( \ket{2} \) are not eigenkets of \( B \).  Because \( B \) also commutes with \( H \), we must have

\begin{dmath}\label{eqn:angularMomentumAndCentralForceCommutators:180}
H B \ket{1}
= H \ket{n}\bra{n} B \ket{1}
= \epsilon_n \ket{n} B_{n 1},
\end{dmath}

and
\begin{dmath}\label{eqn:angularMomentumAndCentralForceCommutators:200}
B H \ket{1}
= B \epsilon_1 \ket{1}
= \epsilon_1 \ket{n}\bra{n} B \ket{1}
= \epsilon_1 \ket{n} B_{n 1}.
\end{dmath}

The commutator is
\begin{dmath}\label{eqn:angularMomentumAndCentralForceCommutators:220}
\antisymmetric{B}{H} \ket{1}
=
\lr{ \epsilon_1 - \epsilon_n } \ket{n} B_{n 1}.
\end{dmath}

Similarly
\begin{dmath}\label{eqn:angularMomentumAndCentralForceCommutators:240}
\antisymmetric{B}{H} \ket{2}
=
\lr{ \epsilon_2 - \epsilon_n } \ket{n} B_{n 2}.
\end{dmath}

For those kets \( \ket{m} \in \setlr{ \ket{3}, \ket{4}, \cdots } \) that are eigenkets of \( B \), with \( B \ket{m} = b_m \ket{m} \), we have

\begin{dmath}\label{eqn:angularMomentumAndCentralForceCommutators:280}
\antisymmetric{B}{H} \ket{m}
=
B \epsilon_m \ket{m} - H b_m \ket{m}
=
b_m \epsilon_m \ket{m} - \epsilon_m b_m \ket{m}
=
0.
\end{dmath}

If the commutator is zero, then we require all its matrix elements
\begin{equation}\label{eqn:angularMomentumAndCentralForceCommutators:260}
\begin{aligned}
\bra{1} \antisymmetric{B}{H} \ket{1} &= \lr{ \epsilon_1 - \epsilon_1 } B_{1 1} \\
\bra{2} \antisymmetric{B}{H} \ket{1} &= \lr{ \epsilon_1 - \epsilon_2 } B_{2 1} \\
\bra{1} \antisymmetric{B}{H} \ket{2} &= \lr{ \epsilon_2 - \epsilon_1 } B_{1 2} \\
\bra{2} \antisymmetric{B}{H} \ket{2} &= \lr{ \epsilon_2 - \epsilon_2 } B_{2 2},
\end{aligned}
\end{equation}

to be zero.  Because of \cref{eqn:angularMomentumAndCentralForceCommutators:280} only the matrix elements with respect to states \( \ket{1}, \ket{2} \) need be considered.  Two of the matrix elements above are clearly zero, regardless of the values of \( B_{1 1}\), and \(B_{2 2} \), and for the other two to be zero, we must either have

\begin{itemize}
\item \( B_{2 1} = B_{1 2} = 0 \), or
\item \( \epsilon_1 = \epsilon_2 \).
\end{itemize}

If the first condition were true we would have

\begin{dmath}\label{eqn:angularMomentumAndCentralForceCommutators:300}
B \ket{1}
=
\ket{n}\bra{n} B \ket{1}
=
\ket{n} B_{n 1}
=
\ket{1} B_{1 1},
\end{dmath}

and \( B \ket{2} = B_{2 2} \ket{2} \).  This contradicts the requirement that \( \ket{1}, \ket{2} \) not be eigenkets of \( B \), leaving only the second option.  That second option means there must be a degeneracy in the system.

} % answer

%\EndArticle

         %
% Copyright � 2015 Peeter Joot.  All Rights Reserved.
% Licenced as described in the file LICENSE under the root directory of this GIT repository.
%
\newcommand{\authorname}{Peeter Joot}
\newcommand{\email}{peeterjoot@protonmail.com}
\newcommand{\basename}{FIXMEbasenameUndefined}
\newcommand{\dirname}{notes/FIXMEdirnameUndefined/}

\renewcommand{\basename}{moreKet}
\renewcommand{\dirname}{notes/phy1520/}
%\newcommand{\dateintitle}{}
%\newcommand{\keywords}{}

\newcommand{\authorname}{Peeter Joot}
\newcommand{\onlineurl}{http://sites.google.com/site/peeterjoot2/math2013/\basename.pdf}
\newcommand{\sourcepath}{\dirname\basename.tex}
\newcommand{\generatetitle}[1]{\chapter{#1}}

\newcommand{\vcsinfo}{%
\section*{}
\noindent{\color{DarkOliveGreen}{\rule{\linewidth}{0.1mm}}}
\paragraph{Document version}
%\paragraph{\color{Maroon}{Document version}}
{
\small
\begin{itemize}
\item Available online at:\\ 
\href{\onlineurl}{\onlineurl}
\item Git Repository: \input{./.revinfo/gitRepo.tex}
\item Source: \sourcepath
\item last commit: \input{./.revinfo/gitCommitString.tex}
\item commit date: \input{./.revinfo/gitCommitDate.tex}
\end{itemize}
}
}

%\PassOptionsToPackage{dvipsnames,svgnames}{xcolor}
\PassOptionsToPackage{square,numbers}{natbib}
\documentclass{scrreprt}

\usepackage[left=2cm,right=2cm]{geometry}
\usepackage[svgnames]{xcolor}
\usepackage{peeters_layout}

\usepackage{natbib}

\usepackage[
colorlinks=true,
bookmarks=false,
pdfauthor={\authorname, \email},
backref 
]{hyperref}

% http://tex.stackexchange.com/questions/75773/how-to-reference-problems-by-the-text-label-in-an-exercise-envioronment
\usepackage[english]{cleveref}
\crefname{Exercise}{exercise}{exercises}
\Crefname{Exercise}{Exercise}{Exercises}

\RequirePackage{titlesec}
\RequirePackage{ifthen}

% http://stackoverflow.com/questions/4932910/date-in-the-tabular-environment
\makeatletter
\let\insertdate\@date
\makeatother

\titleformat{\chapter}[display]
{\bfseries\Large}
{\color{DarkSlateGrey}\filleft \authorname
\ifthenelse{\isundefined{\studentnumber}}{}{\\ \studentnumber}
\ifthenelse{\isundefined{\email}}{}{\\ \email}
\ifthenelse{\isundefined{\dateintitle}}{}{\\ \insertdate}
%\ifthenelse{\isundefined{\coursename}}{}{\\ \coursename} % put in title instead.
}
{4ex}
{\color{DarkOliveGreen}{\titlerule}\color{Maroon}
\vspace{2ex}%
\filright}
[\vspace{2ex}%
\color{DarkOliveGreen}\titlerule
]

\newcommand{\beginArtWithToc}[0]{\begin{document}\tableofcontents}
\newcommand{\beginArtNoToc}[0]{\begin{document}}
\newcommand{\EndNoBibArticle}[0]{\end{document}}
\newcommand{\EndArticle}[0]{\bibliography{Bibliography}\bibliographystyle{plainnat}\end{document}}

% 
%\newcommand{\citep}[1]{\cite{#1}}

\colorSectionsForArticle



\usepackage{peeters_layout_exercise}
\usepackage{peeters_braket}
\usepackage{peeters_figures}

\beginArtNoToc

\generatetitle{more ket problems}
%\chapter{more ket problems}
%\label{chap:moreKet}


\makeoproblem{degenerate ket space example}{problem:moreKet:23}{\citep{sakurai2014modern} pr. X.23}{

Consider operators with representation

\begin{equation}\label{eqn:moreKet:n}
A = 
\begin{bmatrix}
a & 0 & 0 \\
0 & -a & 0 \\
0 & 0 & -a
\end{bmatrix}
,
\qquad
B = 
\begin{bmatrix}
b & 0 & 0 \\
0 & 0 & -ib \\
0 & ib & 0
\end{bmatrix}.
\end{equation}

Show that these both have degeneracies, commute, and compute a simultaneous ket space for both operators.

} % problem

\makeanswer{problem:moreKet:23}{
} % answer

\EndArticle

         %
% Copyright � 2015 Peeter Joot.  All Rights Reserved.
% Licenced as described in the file LICENSE under the root directory of this GIT repository.
%
\newcommand{\authorname}{Peeter Joot}
\newcommand{\email}{peeterjoot@protonmail.com}
\newcommand{\basename}{FIXMEbasenameUndefined}
\newcommand{\dirname}{notes/FIXMEdirnameUndefined/}

\renewcommand{\basename}{translation}
\renewcommand{\dirname}{notes/phy1520/}
%\newcommand{\dateintitle}{}
%\newcommand{\keywords}{}

\newcommand{\authorname}{Peeter Joot}
\newcommand{\onlineurl}{http://sites.google.com/site/peeterjoot2/math2013/\basename.pdf}
\newcommand{\sourcepath}{\dirname\basename.tex}
\newcommand{\generatetitle}[1]{\chapter{#1}}

\newcommand{\vcsinfo}{%
\section*{}
\noindent{\color{DarkOliveGreen}{\rule{\linewidth}{0.1mm}}}
\paragraph{Document version}
%\paragraph{\color{Maroon}{Document version}}
{
\small
\begin{itemize}
\item Available online at:\\ 
\href{\onlineurl}{\onlineurl}
\item Git Repository: \input{./.revinfo/gitRepo.tex}
\item Source: \sourcepath
\item last commit: \input{./.revinfo/gitCommitString.tex}
\item commit date: \input{./.revinfo/gitCommitDate.tex}
\end{itemize}
}
}

%\PassOptionsToPackage{dvipsnames,svgnames}{xcolor}
\PassOptionsToPackage{square,numbers}{natbib}
\documentclass{scrreprt}

\usepackage[left=2cm,right=2cm]{geometry}
\usepackage[svgnames]{xcolor}
\usepackage{peeters_layout}

\usepackage{natbib}

\usepackage[
colorlinks=true,
bookmarks=false,
pdfauthor={\authorname, \email},
backref 
]{hyperref}

% http://tex.stackexchange.com/questions/75773/how-to-reference-problems-by-the-text-label-in-an-exercise-envioronment
\usepackage[english]{cleveref}
\crefname{Exercise}{exercise}{exercises}
\Crefname{Exercise}{Exercise}{Exercises}

\RequirePackage{titlesec}
\RequirePackage{ifthen}

% http://stackoverflow.com/questions/4932910/date-in-the-tabular-environment
\makeatletter
\let\insertdate\@date
\makeatother

\titleformat{\chapter}[display]
{\bfseries\Large}
{\color{DarkSlateGrey}\filleft \authorname
\ifthenelse{\isundefined{\studentnumber}}{}{\\ \studentnumber}
\ifthenelse{\isundefined{\email}}{}{\\ \email}
\ifthenelse{\isundefined{\dateintitle}}{}{\\ \insertdate}
%\ifthenelse{\isundefined{\coursename}}{}{\\ \coursename} % put in title instead.
}
{4ex}
{\color{DarkOliveGreen}{\titlerule}\color{Maroon}
\vspace{2ex}%
\filright}
[\vspace{2ex}%
\color{DarkOliveGreen}\titlerule
]

\newcommand{\beginArtWithToc}[0]{\begin{document}\tableofcontents}
\newcommand{\beginArtNoToc}[0]{\begin{document}}
\newcommand{\EndNoBibArticle}[0]{\end{document}}
\newcommand{\EndArticle}[0]{\bibliography{Bibliography}\bibliographystyle{plainnat}\end{document}}

% 
%\newcommand{\citep}[1]{\cite{#1}}

\colorSectionsForArticle



\usepackage{peeters_layout_exercise}
\usepackage{peeters_braket}
\usepackage{peeters_figures}

\beginArtNoToc

\generatetitle{Translation operator problems}
%\chapter{Translation operator problems}
%\label{chap:translation}

\makeoproblem{Polynomial commutators.}{problem:translation:29}{\citep{sakurai2014modern} pr. 1.29}{

\makesubproblem{}{problem:translation:29:a}
For power series \( F, G \), verify

\begin{equation}\label{eqn:translation:180}
\antisymmetric{x_i}{G(\Bp)} = i \Hbar \PD{p_i}{G}, \qquad
\antisymmetric{p_i}{F(\Bx)} = -i \Hbar \PD{x_i}{F}.
\end{equation}

\makesubproblem{}{problem:translation:29:b}

Evaluate \( \antisymmetric{x^2}{p^2} \), and compare to the classical Poisson bracket \( \antisymmetric{x^2}{p^2}_{\textrm{classical}} \).

} % problem

\makeanswer{problem:translation:29}{
\makeSubAnswer{}{problem:translation:29:a}
\makeSubAnswer{}{problem:translation:29:b}
} % answer

\makeoproblem{Translation operator and position expectation.}{problem:translation:30}{\citep{sakurai2014modern} pr. 1.30}{

The translation operator for a finite spatial displacement is given by

\begin{dmath}\label{eqn:translation:20}
J(\Bl) = \exp\lr{ -i \Bp \cdot \Bl/\Hbar },
\end{dmath}

where \( \Bp \) is the momentum operator.

\makesubproblem{}{problem:translation:1.30:a}

Evaluate 

\begin{dmath}\label{eqn:translation:40}
\antisymmetric{x_i}{J(\Bl)}.
\end{dmath}

\makesubproblem{}{problem:translation:1.30:b}
Demonstrate how the expectation value \( \expectation{\Bx} \) changes under translation.

} % problem

\makeanswer{problem:translation:30}{

\makeSubAnswer{}{problem:translation:1.30:a}

For clarity, let's set \( x_i = y \).  The general result will be clear despite doing so.

\begin{dmath}\label{eqn:translation:60}
\antisymmetric{y}{J(\Bl)}
=
\sum_{k= 0} \inv{k!} \lr{\frac{-i}{\Hbar}} 
\antisymmetric{y}{
\lr{ \Bp \cdot \Bl }^k
}.
\end{dmath}

The commutator expands as

\begin{dmath}\label{eqn:translation:80}
\antisymmetric{y}{
\lr{ \Bp \cdot \Bl }^k 
}
+ \lr{ \Bp \cdot \Bl }^k y
=
y \lr{ \Bp \cdot \Bl }^k 
=
y \lr{ p_x l_x + p_y l_y + p_z l_z } \lr{ \Bp \cdot \Bl }^{k-1} 
=
\lr{ p_x l_x y + y p_y l_y + p_z l_z y } \lr{ \Bp \cdot \Bl }^{k-1} 
=
\lr{ p_x l_x y + l_y \lr{ p_y y + i \Hbar } + p_z l_z y } \lr{ \Bp \cdot \Bl }^{k-1} 
=
\lr{ \Bp \cdot \Bl } y \lr{ \Bp \cdot \Bl }^{k-1} 
+ i \Hbar l_y \lr{ \Bp \cdot \Bl }^{k-1} 
= \cdots
=
\lr{ \Bp \cdot \Bl }^{k-1} y \lr{ \Bp \cdot \Bl }^{k-(k-1)} 
+ (k-1) i \Hbar l_y \lr{ \Bp \cdot \Bl }^{k-1} 
=
\lr{ \Bp \cdot \Bl }^{k} y 
+ k i \Hbar l_y \lr{ \Bp \cdot \Bl }^{k-1}.
\end{dmath}

In the above expansion, the commutation of \( y \) with \( p_x, p_z \) has been used.  This gives, for \( k \ne 0 \),

\begin{dmath}\label{eqn:translation:100}
\antisymmetric{y}{
\lr{ \Bp \cdot \Bl }^k 
}
=
k i \Hbar l_y \lr{ \Bp \cdot \Bl }^{k-1}.
\end{dmath}

Note that this also holds for the \( k = 0 \) case, since \( y \) commutes with the identity operator.  Plugging back into the \( J \) commutator, we have

\begin{dmath}\label{eqn:translation:120}
\antisymmetric{y}{J(\Bl)}
=
\sum_{k = 1} \inv{k!} \lr{\frac{-i}{\Hbar}} 
k i \Hbar l_y \lr{ \Bp \cdot \Bl }^{k-1}
=
l_y \sum_{k = 1} \inv{(k-1)!} \lr{\frac{-i}{\Hbar}} 
\lr{ \Bp \cdot \Bl }^{k-1}
=
l_y J(\Bl).
\end{dmath}

The same pattern clearly applies with the other \( x_i \) values, providing the desired relation.

\begin{equation}\label{eqn:translation:140}
\antisymmetric{\Bx}{J(\Bl)} = \sum_{m = 1}^3 \Be_m l_m J(\Bl) = \Bl J(\Bl).
\end{equation}

\makeSubAnswer{}{problem:translation:1.30:b}

Suppose that the translated state is defined as \( \ket{\alpha_{\Bl}} = J(\Bl) \ket{\alpha} \).  The expectation value with respect to this state is

\begin{dmath}\label{eqn:translation:160}
\expectation{\Bx'} 
= 
\bra{\alpha_{\Bl}} \Bx \ket{\alpha_{\Bl}}
= 
\bra{\alpha} J^\dagger(\Bl) \Bx J(\Bl) \ket{\alpha}
= 
\bra{\alpha} J^\dagger(\Bl) \lr{ \Bx J(\Bl) } \ket{\alpha}
= 
\bra{\alpha} J^\dagger(\Bl) \lr{ J(\Bl) \Bx + \Bl J(\Bl) } \ket{\alpha}
= 
\bra{\alpha} J^\dagger J \Bx + \Bl J^\dagger J \ket{\alpha}
= 
\bra{\alpha} \Bx \ket{\alpha} + \Bl \braket{\alpha}{\alpha}
= 
\expectation{\Bx} + \Bl.
\end{dmath}

} % answer

\EndArticle

   \chapter{Quantum Dynamics}
      \section{Quantum Harmonic oscillator and coherent states}
         %
% Copyright � 2015 Peeter Joot.  All Rights Reserved.
% Licenced as described in the file LICENSE under the root directory of this GIT repository.
%
%\newcommand{\authorname}{Peeter Joot}
\newcommand{\email}{peeterjoot@protonmail.com}
\newcommand{\basename}{FIXMEbasenameUndefined}
\newcommand{\dirname}{notes/FIXMEdirnameUndefined/}

%\renewcommand{\basename}{gmLecture4}
%\renewcommand{\dirname}{notes/phy1520/}
%%\newcommand{\dateintitle}{}
%\newcommand{\keywords}{PHY1520H}
%
%\newcommand{\authorname}{Peeter Joot}
\newcommand{\onlineurl}{http://sites.google.com/site/peeterjoot2/math2013/\basename.pdf}
\newcommand{\sourcepath}{\dirname\basename.tex}
\newcommand{\generatetitle}[1]{\chapter{#1}}

\newcommand{\vcsinfo}{%
\section*{}
\noindent{\color{DarkOliveGreen}{\rule{\linewidth}{0.1mm}}}
\paragraph{Document version}
%\paragraph{\color{Maroon}{Document version}}
{
\small
\begin{itemize}
\item Available online at:\\ 
\href{\onlineurl}{\onlineurl}
\item Git Repository: \input{./.revinfo/gitRepo.tex}
\item Source: \sourcepath
\item last commit: \input{./.revinfo/gitCommitString.tex}
\item commit date: \input{./.revinfo/gitCommitDate.tex}
\end{itemize}
}
}

%\PassOptionsToPackage{dvipsnames,svgnames}{xcolor}
\PassOptionsToPackage{square,numbers}{natbib}
\documentclass{scrreprt}

\usepackage[left=2cm,right=2cm]{geometry}
\usepackage[svgnames]{xcolor}
\usepackage{peeters_layout}

\usepackage{natbib}

\usepackage[
colorlinks=true,
bookmarks=false,
pdfauthor={\authorname, \email},
backref 
]{hyperref}

% http://tex.stackexchange.com/questions/75773/how-to-reference-problems-by-the-text-label-in-an-exercise-envioronment
\usepackage[english]{cleveref}
\crefname{Exercise}{exercise}{exercises}
\Crefname{Exercise}{Exercise}{Exercises}

\RequirePackage{titlesec}
\RequirePackage{ifthen}

% http://stackoverflow.com/questions/4932910/date-in-the-tabular-environment
\makeatletter
\let\insertdate\@date
\makeatother

\titleformat{\chapter}[display]
{\bfseries\Large}
{\color{DarkSlateGrey}\filleft \authorname
\ifthenelse{\isundefined{\studentnumber}}{}{\\ \studentnumber}
\ifthenelse{\isundefined{\email}}{}{\\ \email}
\ifthenelse{\isundefined{\dateintitle}}{}{\\ \insertdate}
%\ifthenelse{\isundefined{\coursename}}{}{\\ \coursename} % put in title instead.
}
{4ex}
{\color{DarkOliveGreen}{\titlerule}\color{Maroon}
\vspace{2ex}%
\filright}
[\vspace{2ex}%
\color{DarkOliveGreen}\titlerule
]

\newcommand{\beginArtWithToc}[0]{\begin{document}\tableofcontents}
\newcommand{\beginArtNoToc}[0]{\begin{document}}
\newcommand{\EndNoBibArticle}[0]{\end{document}}
\newcommand{\EndArticle}[0]{\bibliography{Bibliography}\bibliographystyle{plainnat}\end{document}}

% 
%\newcommand{\citep}[1]{\cite{#1}}

\colorSectionsForArticle


%%\usepackage{phy1520}
%\usepackage{peeters_braket}
%%\usepackage{peeters_layout_exercise}
%\usepackage{peeters_figures}
%\usepackage{mathtools}
%
%\beginArtNoToc
%\generatetitle{PHY1520H Graduate Quantum Mechanics.  Lecture 4: Quantum Harmonic oscillator and coherent states.  Taught by Prof.\ Arun Paramekanti}
%%\chapter{Quantum Harmonic oscillator and coherent states}
%\label{chap:lecture4}
%
%\paragraph{Disclaimer}
%
%Peeter's lecture notes from class.  These may be incoherent and rough.  This lecture reviewed a lot of quantum harmonic oscillator theory, and wouldn't make sense without having seen raising and lowering operators (ladder operators), number operators, and the like.
%
%These are notes for the UofT course PHY1520, Graduate Quantum Mechanics, taught by Prof. Paramekanti, covering \textchapref{{2}} \citep{sakurai2014modern} content.
%
\section{Classical Harmonic Oscillator}
Recall the classical Harmonic oscillator equations in their Hamiltonian form

\begin{subequations}
\label{eqn:qmLecture4:20}
\begin{dmath}\label{eqn:qmLecture4:40}
\ddt{x} = \frac{p}{m}
\end{dmath}
\begin{dmath}\label{eqn:qmLecture4:60}
\ddt{p} = -k x.
\end{dmath}
\end{subequations}

With

\begin{equation}\label{eqn:qmLecture4:140}
\begin{aligned}
x(t = 0) &= x_0 \\
p(t = 0) &= p_0 \\
k &= m \omega^2,
\end{aligned}
\end{equation}

\index{classical harmonic oscillator}
the solutions are ellipses in phase space

\begin{subequations}
\label{eqn:qmLecture4:80}
\begin{dmath}\label{eqn:qmLecture4:100}
x(t) = x_0 \cos(\omega t) + \frac{p_0}{m \omega} \sin(\omega t)
\end{dmath}
\begin{dmath}\label{eqn:qmLecture4:120}
p(t) = p_0 \cos(\omega t) - m \omega x_0 \sin(\omega t).
\end{dmath}
\end{subequations}

After a suitable scaling of the variables, these elliptical orbits can be transformed into circular trajectories.

\section{Quantum Harmonic Oscillator}
\index{harmonic oscillator}

\begin{dmath}\label{eqn:qmLecture4:160}
\hatH = \frac{\hatp^2}{2 m} + \inv{2} k \hatx^2
\end{dmath}

Set

\begin{subequations}
\label{eqn:qmLecture4:180}
\begin{equation}\label{eqn:qmLecture4:200}
\hatX = \sqrt{\frac{m \omega}{\Hbar}} \hatx
\end{equation}
\begin{equation}\label{eqn:qmLecture4:220}
\hatP = \sqrt{\inv{m \omega \Hbar}} \hatp
\end{equation}
\end{subequations}

The commutators after this change of variables goes from

\begin{equation}\label{eqn:qmLecture4:240}
\antisymmetric{ \hatx}{\hatp} = i \Hbar,
\end{equation}

to
\begin{equation}\label{eqn:qmLecture4:260}
\antisymmetric{ \hatX}{\hatP} = i.
\end{equation}

The Hamiltonian takes the form

\begin{dmath}\label{eqn:qmLecture4:280}
\hatH
= \frac{\Hbar \omega}{2} \lr{ \hatX^2 + \hatP^2 }
= \Hbar \omega \lr{ \lr{ \frac{\hatX -i \hatP}{\sqrt{2}} } \lr{ \frac{\hatX +i \hatP}{\sqrt{2}}} + \inv{2} }.
\end{dmath}

\index{ladder operator}
\index{raising operator}
\index{lowering operator}
Define ladder operators (raising and lowering operators respectively)

\begin{subequations}
\label{eqn:qmLecture4:300}
\begin{dmath}\label{eqn:qmLecture4:320}
\hata^\dagger = \frac{\hatX -i \hatP}{\sqrt{2}}
\end{dmath}
\begin{dmath}\label{eqn:qmLecture4:340}
\hata = \frac{\hatX +i \hatP}{\sqrt{2}}
\end{dmath}
\end{subequations}

so

\begin{dmath}\label{eqn:qmLecture4:360}
\hatH = \Hbar \omega \lr{ \hata^\dagger \hata + \inv{2} }.
\end{dmath}

We can show

\begin{dmath}\label{eqn:qmLecture4:380}
\antisymmetric{\hata}{\hata^\dagger} = 1,
\end{dmath}

and

\index{number operator}
\begin{equation}\label{eqn:qmLecture4:400}
N \ket{n} \equiv \hata^\dagger a = n \ket{n},
\end{equation}

where \( n \ge 0 \) is an integer.  Recall that

\begin{dmath}\label{eqn:qmLecture4:420}
\hata \ket{0} = 0,
\end{dmath}

and

\begin{dmath}\label{eqn:qmLecture4:440}
\bra{X} X + i P \ket{0} = 0.
\end{dmath}

With

\begin{dmath}\label{eqn:qmLecture4:460}
\braket{x}{0} = \Psi_0(x),
\end{dmath}

we can show

\begin{dmath}\label{eqn:qmLecture4:480}
\inv{\sqrt{2}} \lr{ X + \PD{X}{} } \Psi_0(X) = 0.
\end{dmath}

Also recall that

\begin{subequations}
\label{eqn:qmLecture4:500}
\begin{dmath}\label{eqn:qmLecture4:520}
\hata \ket{n} = \sqrt{n} \ket{n-1}
\end{dmath}
\begin{dmath}\label{eqn:qmLecture4:540}
\hata^\dagger \ket{n} = \sqrt{n + 1} \ket{n+1}
\end{dmath}
\end{subequations}

\section{Coherent states}
\index{coherent state}

Coherent states for the quantum harmonic oscillator are the eigenkets for the creation and annihilation operators

\begin{subequations}
\label{eqn:qmLecture4:560}
\begin{dmath}\label{eqn:qmLecture4:580}
\hata \ket{z} = z \ket{z}
\end{dmath}
\begin{dmath}\label{eqn:qmLecture4:600}
\hata^\dagger \ket{\tilde{z}} = \tilde{z} \ket{\tilde{z}} ,
\end{dmath}
\end{subequations}

where

\begin{dmath}\label{eqn:qmLecture4:620}
\ket{z} = \sum_{n = 0}^\infty c_n \ket{n},
\end{dmath}

and \( z \) is allowed to be a complex number.

Looking for such a state, we compute

\begin{dmath}\label{eqn:qmLecture4:640}
\hata \ket{z}
= \sum_{n=1}^\infty c_n \hata \ket{n}
= \sum_{n=1}^\infty c_n \sqrt{n} \ket{n-1}
\end{dmath}

compare this to

\begin{dmath}\label{eqn:qmLecture4:660}
z \ket{z}
=
z \sum_{n=0}^\infty c_n \ket{n}
=
\sum_{n=1}^\infty c_n \sqrt{n} \ket{n-1}
=
\sum_{n=0}^\infty c_{n+1} \sqrt{n+1} \ket{n},
\end{dmath}

so

\begin{dmath}\label{eqn:qmLecture4:680}
c_{n+1} \sqrt{n+1} = z c_n
\end{dmath}

This gives

\begin{dmath}\label{eqn:qmLecture4:700}
c_{n+1} = \frac{z c_n}{\sqrt{n+1}}
\end{dmath}

\begin{equation}\label{eqn:qmLecture4:720}
\begin{aligned}
c_1 &= c_0 z \\
c_2 &= \frac{z c_1}{\sqrt{2}} = \frac{z^2 c_0}{\sqrt{2}} \\
\vdots &
\end{aligned}
\end{equation}

or

\begin{dmath}\label{eqn:qmLecture4:740}
c_n = \frac{z^n}{\sqrt{n!}}.
\end{dmath}

So the desired state is

\begin{dmath}\label{eqn:qmLecture4:760}
\ket{z} = c_0 \sum_{n=0}^\infty \frac{z^n}{\sqrt{n!}} \ket{n}.
\end{dmath}

Also recall that

\begin{dmath}\label{eqn:qmLecture4:780}
\ket{n} = \frac{\lr{ \hata^\dagger }^n}{\sqrt{n!}} \ket{0},
\end{dmath}

which gives

\begin{dmath}\label{eqn:qmLecture4:800}
\ket{z}
= c_0 \sum_{n=0}^\infty \frac{\lr{z \hata^\dagger}^n }{n!} \ket{0}
= c_0 e^{z \hata^\dagger}  \ket{0}.
\end{dmath}

The normalization is

\begin{dmath}\label{eqn:qmLecture4:820}
c_0 = e^{-\Abs{z}^2/2}.
\end{dmath}

While we have \( \braket{n_1}{n_2} = \delta_{n_1, n_2} \), these \( \ket{z} \) states are not orthonormal.  Figuring out that this overlap

\begin{dmath}\label{eqn:qmLecture4:840}
\braket{z_1}{z_2} \ne 0,
\end{dmath}

will be left for homework.

\section{Coherent state time evolution}
\index{coherent state!time evolution}

We don't know much about these coherent states.  For example does a coherent state at time zero evolve to a coherent state?

\begin{dmath}\label{eqn:qmLecture4:860}
\ket{z} \overset{?}{\rightarrow} \ket{z(t)}
\end{dmath}

\index{Heisenberg picture!coherent state}
It turns out that these questions are best tackled in the Heisenberg picture, considering

\begin{dmath}\label{eqn:qmLecture4:880}
e^{-i \hatH t/\Hbar } \ket{z}.
\end{dmath}

For example, what is the average of the position operator

\begin{dmath}\label{eqn:qmLecture4:900}
\bra{z} e^{i \hatH t/\Hbar } \hatx e^{-i \hatH t/\Hbar } \ket{z}
=
\sum_{n, n' = 0}^\infty
\bra{n} c_n^\conj e^{i E_n t/\Hbar}
\lr{ a + a^\dagger} \sqrt{ \frac{\Hbar}{m \omega} }
c_{n'} e^{i E_{n'} t/\Hbar}
\ket{n}.
\end{dmath}

This is very messy to attempt.  Instead if we know how the operator evolves we can calculate

\begin{dmath}\label{eqn:qmLecture4:920}
\bra{z} \hatx_\txtH(t) \ket{z},
\end{dmath}

that is

\begin{dmath}\label{eqn:qmLecture4:940}
\expectation{\hatx}(t) = \bra{z} \hatx_\txtH(t) \ket{z},
\end{dmath}

and for momentum

\begin{dmath}\label{eqn:qmLecture4:960}
\expectation{\hatp}(t) = \bra{z} \hatp_\txtH(t) \ket{z}.
\end{dmath}

The question to ask is what are the expansions of

\begin{subequations}
\label{eqn:qmLecture4:980}
\begin{dmath}\label{eqn:qmLecture4:1000}
\hata_\txtH(t) = e^{i \hatH t/\Hbar} \hata e^{-i \hatH t/\Hbar}.
\end{dmath}
\begin{dmath}\label{eqn:qmLecture4:1020}
\hata^\dagger_\txtH(t) = e^{i \hatH t/\Hbar} \hata^\dagger e^{-i \hatH t/\Hbar}.
\end{dmath}
\end{subequations}

The question to ask is how do these operators ask on the basis states

\begin{dmath}\label{eqn:qmLecture4:1040}
\hata_\txtH(t) \ket{n}
= e^{i \hatH t/\Hbar} \hata e^{-i \hatH t/\Hbar} \ket{n}
= e^{i \hatH t/\Hbar} \hata e^{-i t \omega (n + 1/2)} \ket{n}
=
e^{-i t \omega (n + 1/2)}
e^{i \hatH t/\Hbar}
\sqrt{n} \ket{n-1}
=
\sqrt{n}
e^{-i t \omega (n + 1/2)}
e^{i t \omega (n - 1/2)}
\ket{n-1}
=
\sqrt{n}  e^{-i \omega t} \ket{n-1}
=
e^{-i \omega t} \ket{n}.
\end{dmath}

So we have found

\begin{dmath}\label{eqn:qmLecture4:1060}
\begin{aligned}
\hata_\txtH(t) &= a e^{-i\omega t} \\
\hata^\dagger_\txtH(t) &= a^\dagger e^{i\omega t}
\end{aligned}
\end{dmath}

%\paragraph{Position and momentum operator time evolution}

%\EndArticle

      \section{time evolution of coherent states, and charged particles in a magnetic field}
         %
% Copyright � 2015 Peeter Joot.  All Rights Reserved.
% Licenced as described in the file LICENSE under the root directory of this GIT repository.
%
%\newcommand{\authorname}{Peeter Joot}
\newcommand{\email}{peeterjoot@protonmail.com}
\newcommand{\basename}{FIXMEbasenameUndefined}
\newcommand{\dirname}{notes/FIXMEdirnameUndefined/}

%\renewcommand{\basename}{qmLecture5}
%\renewcommand{\dirname}{notes/phy1520/}
%\newcommand{\keywords}{PHY1520H}
%\newcommand{\authorname}{Peeter Joot}
\newcommand{\onlineurl}{http://sites.google.com/site/peeterjoot2/math2013/\basename.pdf}
\newcommand{\sourcepath}{\dirname\basename.tex}
\newcommand{\generatetitle}[1]{\chapter{#1}}

\newcommand{\vcsinfo}{%
\section*{}
\noindent{\color{DarkOliveGreen}{\rule{\linewidth}{0.1mm}}}
\paragraph{Document version}
%\paragraph{\color{Maroon}{Document version}}
{
\small
\begin{itemize}
\item Available online at:\\ 
\href{\onlineurl}{\onlineurl}
\item Git Repository: \input{./.revinfo/gitRepo.tex}
\item Source: \sourcepath
\item last commit: \input{./.revinfo/gitCommitString.tex}
\item commit date: \input{./.revinfo/gitCommitDate.tex}
\end{itemize}
}
}

%\PassOptionsToPackage{dvipsnames,svgnames}{xcolor}
\PassOptionsToPackage{square,numbers}{natbib}
\documentclass{scrreprt}

\usepackage[left=2cm,right=2cm]{geometry}
\usepackage[svgnames]{xcolor}
\usepackage{peeters_layout}

\usepackage{natbib}

\usepackage[
colorlinks=true,
bookmarks=false,
pdfauthor={\authorname, \email},
backref 
]{hyperref}

% http://tex.stackexchange.com/questions/75773/how-to-reference-problems-by-the-text-label-in-an-exercise-envioronment
\usepackage[english]{cleveref}
\crefname{Exercise}{exercise}{exercises}
\Crefname{Exercise}{Exercise}{Exercises}

\RequirePackage{titlesec}
\RequirePackage{ifthen}

% http://stackoverflow.com/questions/4932910/date-in-the-tabular-environment
\makeatletter
\let\insertdate\@date
\makeatother

\titleformat{\chapter}[display]
{\bfseries\Large}
{\color{DarkSlateGrey}\filleft \authorname
\ifthenelse{\isundefined{\studentnumber}}{}{\\ \studentnumber}
\ifthenelse{\isundefined{\email}}{}{\\ \email}
\ifthenelse{\isundefined{\dateintitle}}{}{\\ \insertdate}
%\ifthenelse{\isundefined{\coursename}}{}{\\ \coursename} % put in title instead.
}
{4ex}
{\color{DarkOliveGreen}{\titlerule}\color{Maroon}
\vspace{2ex}%
\filright}
[\vspace{2ex}%
\color{DarkOliveGreen}\titlerule
]

\newcommand{\beginArtWithToc}[0]{\begin{document}\tableofcontents}
\newcommand{\beginArtNoToc}[0]{\begin{document}}
\newcommand{\EndNoBibArticle}[0]{\end{document}}
\newcommand{\EndArticle}[0]{\bibliography{Bibliography}\bibliographystyle{plainnat}\end{document}}

% 
%\newcommand{\citep}[1]{\cite{#1}}

\colorSectionsForArticle


%
%%\usepackage{phy1520}
%\usepackage{peeters_braket}
%%\usepackage{peeters_layout_exercise}
%\usepackage{peeters_figures}
%\usepackage{mathtools}
%
%\beginArtNoToc
%\generatetitle{PHY1520H Graduate Quantum Mechanics.  Lecture 5: time evolution of coherent states, and charged particles in a magnetic field.  Taught by Prof.\ Arun Paramekanti}
%\label{chap:qmLecture5}
%
%\paragraph{Disclaimer}
%
%Peeter's lecture notes from class.  These may be incoherent and rough.
%
%These are notes for the UofT course PHY1520, Graduate Quantum Mechanics, taught by Prof. Paramekanti, covering \textchapref{{1}} \citep{sakurai2014modern} content.
%
\section{Expectation with respect to coherent states}

\index{expectation!coherent state}
A coherent state for the SHO \( H = \lr{ N + \inv{2} } \Hbar \omega \) was given by

\begin{equation}\label{eqn:qmLecture5:20}
a \ket{z} = z \ket{z},
\end{equation}

where we showed that

\begin{equation}\label{eqn:qmLecture5:40}
\ket{z} = c_0 e^{ z a^\dagger } \ket{0}.
\end{equation}

In the Heisenberg picture we found

\begin{equation}\label{eqn:qmLecture5:60}
\begin{aligned}
a_\txtH(t) &= e^{i H t/\Hbar} a e^{-i H t/\Hbar} = a e^{-i\omega t} \\
a_\txtH^\dagger(t) &= e^{i H t/\Hbar} a^\dagger e^{-i H t/\Hbar} = a^\dagger e^{i\omega t}.
\end{aligned}
\end{equation}

Recall that the position and momentum representation of the ladder operators was

\begin{equation}\label{eqn:qmLecture5:80}
\begin{aligned}
a &= \inv{\sqrt{2}} \lr{ \hatx \sqrt{\frac{m \omega}{\Hbar}} + i \hatp \sqrt{\inv{m \Hbar \omega}} } \\
a^\dagger &= \inv{\sqrt{2}} \lr{ \hatx \sqrt{\frac{m \omega}{\Hbar}} - i \hatp \sqrt{\inv{m \Hbar \omega}} },
\end{aligned}
\end{equation}

or equivalently
\begin{equation}\label{eqn:qmLecture5:100}
\begin{aligned}
\hatx &= \lr{ a + a^\dagger } \sqrt{\frac{\Hbar}{ 2 m \omega}} \\
\hatp &= i \lr{ a^\dagger - a } \sqrt{\frac{m \Hbar \omega}{2}}.
\end{aligned}
\end{equation}

Given this we can compute expectation value of position operator

\begin{dmath}\label{eqn:qmLecture5:120}
\bra{z} \hatx \ket{z}
=
\sqrt{\frac{\Hbar}{ 2 m \omega}}
\bra{z}
\lr{ a + a^\dagger }
\ket{z}
=
\lr{ z + z^\conj } \sqrt{\frac{\Hbar}{ 2 m \omega}}
=
2 \Real z \sqrt{\frac{\Hbar}{ 2 m \omega}} .
\end{dmath}

Similarly

\begin{dmath}\label{eqn:qmLecture5:140}
\bra{z} \hatp \ket{z}
=
i \sqrt{\frac{m \Hbar \omega}{2}}
\bra{z}
\lr{ a^\dagger - a }
\ket{z}
=
\sqrt{\frac{m \Hbar \omega}{2}}
2 \Imag z.
\end{dmath}

How about the expectation of the Heisenberg position operator?  That is

\begin{dmath}\label{eqn:qmLecture5:160}
\bra{z} \hatx_\txtH(t) \ket{z}
=
\sqrt{\frac{\Hbar}{2 m \omega}} \bra{z} \lr{ a + a^\dagger } \ket{z}
=
\sqrt{\frac{\Hbar}{2 m \omega}} \lr{ z e^{-i \omega t} + z^\conj e^{i \omega t}}
=
\sqrt{\frac{\Hbar}{2 m \omega}} \lr{ \lr{z + z^\conj} \cos( \omega t ) -i \lr{ z - z^\conj } \sin( \omega t) }
=
\sqrt{\frac{\Hbar}{2 m \omega}} \lr{ \expectation{x(0)} \sqrt{ \frac{2 m \omega}{\Hbar}} \cos( \omega t ) -i \expectation{p(0)} i \sqrt{\frac{2 m \omega}{\Hbar} } \sin( \omega t) }
=
\expectation{x(0)} \cos( \omega t ) + \frac{\expectation{p(0)}}{m \omega} \sin( \omega t) .
\end{dmath}

\index{coherent state!position operator}
We find that the average of the Heisenberg position operator evolves in time in exactly the same fashion as position in the classical Harmonic oscillator.  This phase space like trajectory is sketched in \cref{fig:zStateProjectionsAverageXandP:zStateProjectionsAverageXandPFig1}.

\imageFigure{../../figures/phy1520/zStateProjectionsAverageXandPFig1}{Phase space like trajectory.}{fig:zStateProjectionsAverageXandP:zStateProjectionsAverageXandPFig1}{0.3}

In the text it is shown that we have the same structure for the Heisenberg operator itself, before taking expectations

\begin{dmath}\label{eqn:qmLecture5:220}
\hatx_\txtH(t)
=
{x(0)} \cos( \omega t ) + \frac{{p(0)}}{m \omega} \sin( \omega t).
\end{dmath}

Where the coherent states become useful is that we will see that the second moments of position and momentum are not time dependent with respect to the coherent states.  Such states remain localized.

\section{Coherent state uncertainty}
\index{coherent state!uncertainty}

First note that using the commutator relationship we have

\begin{dmath}\label{eqn:qmLecture5:180}
\bra{z} a a^\dagger \ket{z}
=
\bra{z} \lr{ \antisymmetric{a}{a^\dagger} + a^\dagger a } \ket{z}
=
\bra{z} \lr{ 1 + a^\dagger a } \ket{z}.
\end{dmath}

For the second moment we have

\begin{dmath}\label{eqn:qmLecture5:200}
\bra{z} \hatx^2 \ket{z}
=
\frac{\Hbar}{ 2 m \omega}
\bra{z} \lr{a + a^\dagger } \lr{a + a^\dagger }  \ket{z}
=
\frac{\Hbar}{ 2 m \omega}
\bra{z} \lr{
a^2 + {(a^\dagger)}^2 + a a^\dagger + a^\dagger a
} \ket{z}
=
\frac{\Hbar}{ 2 m \omega}
\bra{z} \lr{
a^2 + {(a^\dagger)}^2 + 2 a^\dagger a + 1
} \ket{z}
=
\frac{\Hbar}{ 2 m \omega}
\lr{ z^2 + {(z^\conj)}^2 + 2 z^\conj z + 1}  \ket{z}
=
\frac{\Hbar}{ 2 m \omega}
\lr{ z + z^\conj }^2
+
\frac{\Hbar}{ 2 m \omega}.
\end{dmath}

We find

\begin{dmath}\label{eqn:qmLecture5:240}
\sigma_x^2 = \frac{\Hbar}{ 2 m \omega},
\end{dmath}

and

\begin{dmath}\label{eqn:qmLecture5:260}
\sigma_p^2 = \frac{m \Hbar \omega}{2}
\end{dmath}

so

\begin{dmath}\label{eqn:qmLecture5:280}
\sigma_x^2 \sigma_p^2 = \frac{\Hbar^2}{4},
\end{dmath}

or

\begin{dmath}\label{eqn:qmLecture5:300}
\sigma_x \sigma_p = \frac{\Hbar}{2}.
\end{dmath}

This is the minimum uncertainty.

\section{Quantum Field theory}
\index{field theory}

In Quantum Field theory the ideas of isolated oscillators is used to model particle creation.  The lowest energy state (a no particle, vacuum state) is given the lowest energy level, with each additional quantum level modeling a new particle creation state as sketched in \cref{fig:qftEnergyLevels:qftEnergyLevelsFig2}.

\imageFigure{../../figures/phy1520/qftEnergyLevelsFig2}{QFT energy levels.}{fig:qftEnergyLevels:qftEnergyLevelsFig2}{0.3}

We have to imagine many oscillators, each with a distinct vacuum energy \( \sim \Bk^2 \) .  The Harmonic oscillator can be used to model the creation of particles with \( \Hbar \omega \) energy differences from that ``vacuum energy''.

\section{Charged particle in a magnetic field}
\index{magnetic field}

In the classical case ( with SI units or \( c = 1 \) ) we have

\begin{dmath}\label{eqn:qmLecture5:320}
\BF = q \BE + q \Bv \cross \BB.
\end{dmath}

Alternately, we can look at the Hamiltonian view of the system, written in terms of potentials

\begin{dmath}\label{eqn:qmLecture5:340}
\BB = \spacegrad \cross \BA,
\end{dmath}
\begin{dmath}\label{eqn:qmLecture5:360}
\BE = - \spacegrad \phi - \PD{t}{\BA}.
\end{dmath}

Note that the curl form for the magnetic field implies one of the required Maxwell's equations \( \spacegrad \cdot \BB = 0 \).

Ignoring time dependence of the potentials, the Hamiltonian can be expressed as

\begin{dmath}\label{eqn:qmLecture5:380}
H = \inv{2 m} \lr{ \Bp - q \BA }^2 + q \phi.
\end{dmath}

In this Hamiltonian the vector \( \Bp \) is called the canonical momentum, the momentum conjugate to position in phase space.

It is left as an exercise to show that the Lorentz force equation results from application of the Hamiltonian equations of motion, and that the velocity is given by \( \Bv = (\Bp - q \BA)/m \).

For quantum mechanics, we use the same Hamiltonian, but promote our position, momentum and potentials to operators.

\begin{dmath}\label{eqn:qmLecture5:400}
\hatH = \inv{2 m} \lr{ \pcap - q \Acap(\Br, t) }^2 + q \hat\phi(\Br, t).
\end{dmath}

\section{Gauge invariance}
\index{gauge invariance}

Can we say anything about this before looking at the question of a particle in a magnetic field?

Recall that the we can make a gauge transformation of the form

\begin{subequations}
\label{eqn:qmLecture5:420a}
\begin{dmath}\label{eqn:qmLecture5:420}
\BA \rightarrow \BA + \spacegrad \chi
\end{dmath}
\begin{dmath}\label{eqn:qmLecture5:440}
\phi \rightarrow \phi - \PD{t}{\chi}
\end{dmath}
\end{subequations}

Does this notion of gauge invariance also carry over to the Quantum Hamiltonian.  After gauge transformation we have

\begin{dmath}\label{eqn:qmLecture5:460}
\hatH'
= \inv{2 m} \lr{ \pcap - q \BA - q \spacegrad \chi }^2 + q \lr{ \phi - \PD{t}{\chi} }
\end{dmath}

Now we are in a mess, since this function \( \chi \) can make the Hamiltonian horribly complicated.  We don't see how gauge invariance can easily be applied to the quantum problem.  Next time we will introduce a transformation that resolves some of this mess.

%\EndArticle

      \section{Electromagnetic gauge transformation and Aharonov-Bohm effect}
         %
% Copyright � 2015 Peeter Joot.  All Rights Reserved.
% Licenced as described in the file LICENSE under the root directory of this GIT repository.
%
%\newcommand{\authorname}{Peeter Joot}
\newcommand{\email}{peeterjoot@protonmail.com}
\newcommand{\basename}{FIXMEbasenameUndefined}
\newcommand{\dirname}{notes/FIXMEdirnameUndefined/}

%\renewcommand{\basename}{qmLecture6}
%\renewcommand{\dirname}{notes/phy1520/}
%\newcommand{\keywords}{PHY1520H}
%\newcommand{\authorname}{Peeter Joot}
\newcommand{\onlineurl}{http://sites.google.com/site/peeterjoot2/math2013/\basename.pdf}
\newcommand{\sourcepath}{\dirname\basename.tex}
\newcommand{\generatetitle}[1]{\chapter{#1}}

\newcommand{\vcsinfo}{%
\section*{}
\noindent{\color{DarkOliveGreen}{\rule{\linewidth}{0.1mm}}}
\paragraph{Document version}
%\paragraph{\color{Maroon}{Document version}}
{
\small
\begin{itemize}
\item Available online at:\\ 
\href{\onlineurl}{\onlineurl}
\item Git Repository: \input{./.revinfo/gitRepo.tex}
\item Source: \sourcepath
\item last commit: \input{./.revinfo/gitCommitString.tex}
\item commit date: \input{./.revinfo/gitCommitDate.tex}
\end{itemize}
}
}

%\PassOptionsToPackage{dvipsnames,svgnames}{xcolor}
\PassOptionsToPackage{square,numbers}{natbib}
\documentclass{scrreprt}

\usepackage[left=2cm,right=2cm]{geometry}
\usepackage[svgnames]{xcolor}
\usepackage{peeters_layout}

\usepackage{natbib}

\usepackage[
colorlinks=true,
bookmarks=false,
pdfauthor={\authorname, \email},
backref 
]{hyperref}

% http://tex.stackexchange.com/questions/75773/how-to-reference-problems-by-the-text-label-in-an-exercise-envioronment
\usepackage[english]{cleveref}
\crefname{Exercise}{exercise}{exercises}
\Crefname{Exercise}{Exercise}{Exercises}

\RequirePackage{titlesec}
\RequirePackage{ifthen}

% http://stackoverflow.com/questions/4932910/date-in-the-tabular-environment
\makeatletter
\let\insertdate\@date
\makeatother

\titleformat{\chapter}[display]
{\bfseries\Large}
{\color{DarkSlateGrey}\filleft \authorname
\ifthenelse{\isundefined{\studentnumber}}{}{\\ \studentnumber}
\ifthenelse{\isundefined{\email}}{}{\\ \email}
\ifthenelse{\isundefined{\dateintitle}}{}{\\ \insertdate}
%\ifthenelse{\isundefined{\coursename}}{}{\\ \coursename} % put in title instead.
}
{4ex}
{\color{DarkOliveGreen}{\titlerule}\color{Maroon}
\vspace{2ex}%
\filright}
[\vspace{2ex}%
\color{DarkOliveGreen}\titlerule
]

\newcommand{\beginArtWithToc}[0]{\begin{document}\tableofcontents}
\newcommand{\beginArtNoToc}[0]{\begin{document}}
\newcommand{\EndNoBibArticle}[0]{\end{document}}
\newcommand{\EndArticle}[0]{\bibliography{Bibliography}\bibliographystyle{plainnat}\end{document}}

% 
%\newcommand{\citep}[1]{\cite{#1}}

\colorSectionsForArticle


%
%%\usepackage{phy1520}
%\usepackage{peeters_braket}
%%\usepackage{peeters_layout_exercise}
%\usepackage{peeters_figures}
%\usepackage{mathtools}
%
%\beginArtNoToc
%\generatetitle{PHY1520H Graduate Quantum Mechanics.  Lecture 6: Electromagnetic gauge transformation and Aharonov-Bohm effect.  Taught by Prof.\ Arun Paramekanti}
%%\chapter{Electromagnetic gauge transformation and Aharonov-Bohm effect}
%\label{chap:qmLecture6}
%
%\paragraph{Disclaimer}
%
%Peeter's lecture notes from class.  These may be incoherent and rough.
%
%These are notes for the UofT course PHY1520, Graduate Quantum Mechanics, taught by Prof. Paramekanti, covering \textchapref{{2}} \citep{sakurai2014modern} content.
%
\paragraph{Particle with \( \BE, \BB \) fields}
\index{electric field}
\index{magnetic field}

We express our fields with vector and scalar potentials

\begin{dmath}\label{eqn:qmLecture6:20}
\BE, \BB \rightarrow \BA, \phi
\end{dmath}

and apply a gauge transformed Hamiltonian

\begin{dmath}\label{eqn:qmLecture6:40}
H = \inv{2m} \lr{ \Bp - q \BA }^2 + q \phi.
\end{dmath}

Recall that in classical mechanics we have

\begin{dmath}\label{eqn:qmLecture6:60}
\Bp - q \BA = m \Bv
\end{dmath}

where \( \Bp \) is not gauge invariant, but the classical momentum \( m \Bv \) is.

If given a point in phase space we must also specify the gauge that we are working with.

For the quantum case, temporarily considering a Hamiltonian without any scalar potential, but introducing a gauge transformation

\begin{dmath}\label{eqn:qmLecture6:80}
\BA \rightarrow \BA + \spacegrad \chi,
\end{dmath}

which takes the Hamiltonian from

\begin{dmath}\label{eqn:qmLecture6:100}
H = \inv{2m} \lr{ \Bp - q \BA }^2,
\end{dmath}

to
\begin{dmath}\label{eqn:qmLecture6:120}
H = \inv{2m} \lr{ \Bp - q \BA -q \spacegrad \chi }^2.
\end{dmath}

We care that the position and momentum operators obey

\begin{dmath}\label{eqn:qmLecture6:140}
\antisymmetric{\hatr_i}{\hatp_j} = i \Hbar \delta_{i j}.
\end{dmath}

We can apply a transformation that keeps \( \Br \) the same, but changes the momentum

\begin{dmath}\label{eqn:qmLecture6:160}
\begin{aligned}
\rcap' &= \rcap  \\
\pcap' &= \pcap  - q \spacegrad \chi(\Br)
\end{aligned}
\end{dmath}

This maps the Hamiltonian to

\begin{dmath}\label{eqn:qmLecture6:101}
H = \inv{2m} \lr{ \Bp' - q \BA -q \spacegrad \chi }^2,
\end{dmath}

We want to check if the commutator relationships have the desired structure, that is

\begin{dmath}\label{eqn:qmLecture6:180}
\begin{aligned}
\antisymmetric{r_i'}{r_j'} &= 0 \\
\antisymmetric{p_i'}{p_j'} &= 0
\end{aligned}
\end{dmath}

This is confirmed in \cref{problem:qmLecture6:1}.

Another thing of interest is how are the wave functions altered by this change of variables?  The wave functions must change in response to this transformation if the energies of the Hamiltonian are to remain the same.

Considering a plane wave specified by

\begin{dmath}\label{eqn:qmLecture6:200}
e^{i \Bk \cdot \Br},
\end{dmath}

where we alter the momentum by

\begin{dmath}\label{eqn:qmLecture6:220}
\Bk \rightarrow \Bk - e \spacegrad \chi.
\end{dmath}

This takes the plane wave to

\begin{dmath}\label{eqn:qmLecture6:240}
e^{i \lr{ \Bk - q \spacegrad \chi } \cdot \Br}.
\end{dmath}

We want to try to find a wave function for the new Hamiltonian

\begin{dmath}\label{eqn:qmLecture6:260}
H' = \inv{2m} \lr{ \Bp' - q \BA -q \spacegrad \chi }^2,
\end{dmath}

of the form

\begin{dmath}\label{eqn:qmLecture6:280}
\psi'(\Br)
\questionEquals
e^{i \theta(\Br)} \psi(\Br),
\end{dmath}

where the new wave function differs from a wave function for the original Hamiltonian by only a position dependent phase factor.

Let's look at the action of the Hamiltonian on the new wave function

\begin{dmath}\label{eqn:qmLecture6:300}
H' \psi'(\Br) .
\end{dmath}

Looking at just the first action

\begin{dmath}\label{eqn:qmLecture6:320}
\lr{ -i \Hbar \spacegrad - q \BA - q \spacegrad \chi } e^{i \theta(\Br)} \psi(\Br)
=
e^{i\theta}
\lr{ -i \Hbar \spacegrad - q \BA - q \spacegrad \chi }
\psi(\Br)
+
\lr{
-i \Hbar i \spacegrad \theta
}
e^{i\theta}
\psi(\Br)
=
e^{i\theta}
\lr{ -i \Hbar \spacegrad - q \BA - q \spacegrad \chi
+ \Hbar \spacegrad \theta
}
\psi(\Br).
\end{dmath}

If we choose

\begin{dmath}\label{eqn:qmLecture6:340}
\theta = \frac{q \chi}{\Hbar},
\end{dmath}

then we are left with

\begin{dmath}\label{eqn:qmLecture6:360}
\lr{ -i \Hbar \spacegrad - q \BA - q \spacegrad \chi } e^{i \theta(\Br)} \psi(\Br)
=
e^{i\theta}
\lr{ -i \Hbar \spacegrad - q \BA }
\psi(\Br).
\end{dmath}

Let \( \BM = -i \Hbar \spacegrad - q \BA \), and act again with \( \lr{ -i \Hbar \spacegrad - q \BA - q \spacegrad \chi } \)

\begin{dmath}\label{eqn:qmLecture6:700}
\lr{ -i \Hbar \spacegrad - q \BA - q \spacegrad \chi } e^{i \theta} \BM \psi
=
e^{i\theta}
\lr{ -i \Hbar i \spacegrad \theta - q \BA - q \spacegrad \chi } e^{i \theta} \BM \psi
+
e^{i\theta}
\lr{ -i \Hbar \spacegrad } \BM \psi
=
e^{i\theta}
\lr{ -i \Hbar \spacegrad -q \BA + \spacegrad \lr{ \Hbar \theta - q \chi} } \BM \psi
=
e^{i\theta} \BM^2 \psi.
\end{dmath}

Restoring factors of \( m \), we've shown that for a choice of \( \Hbar \theta - q \chi \), we have

\begin{dmath}\label{eqn:qmLecture6:400}
\inv{2m} \lr{ -i \Hbar \spacegrad - q \BA - q \spacegrad \chi }^2 e^{i \theta} \psi = e^{i\theta}
\inv{2m} \lr{ -i \Hbar \spacegrad - q \BA }^2 \psi.
\end{dmath}

When \( \psi \) is an energy eigenfunction, this means

\begin{equation}\label{eqn:qmLecture6:420}
H' e^{i\theta} \psi = e^{i \theta} H \psi = e^{i\theta} E\psi = E (e^{i\theta} \psi).
\end{equation}

We've found a transformation of the wave function that has the same energy eigenvalues as the corresponding wave functions for the original untransformed Hamiltonian.

In summary
%%\begin{equation}\label{eqn:qmLecture6:440}
%%\end{equation}
\boxedEquation{eqn:qmLecture6:440}{
\begin{aligned}
H' &= \inv{2m} \lr{ \Bp - q \BA - q \spacegrad \chi}^2 \\
\psi'(\Br) &= e^{i \theta(\Br)} \psi(\Br), \qquad \text{where}\, \theta(\Br) = q \chi(\Br)/\Hbar
\end{aligned}
}

\paragraph{Aharonov-Bohm effect}
\index{Aharonov-Bohm effect}

Consider a periodic motion in a fixed ring as sketched in \cref{fig:l6:l6Fig1}.

\imageFigure{../../figures/phy1520/l6Fig1}{Particle confined to a ring.}{fig:l6:l6Fig1}{0.3}

Here the displacement around the perimeter is \( s = R \phi \) and the Hamiltonian

\begin{equation}\label{eqn:qmLecture6:460}
H = - \frac{\Hbar^2}{2 m} \PDSq{s}{} = - \frac{\Hbar^2}{2 m R^2} \PDSq{\phi}{}.
\end{equation}

Now assume that there is a magnetic field squeezed into the point at the origin, by virtue of a flux at the origin

\begin{equation}\label{eqn:qmLecture6:480}
\BB = \Phi_0 \delta(\Br) \zcap.
\end{equation}

We know that

\begin{equation}\label{eqn:qmLecture6:500}
\oint \BA \cdot d\Bl = \Phi_0,
\end{equation}

so that

\begin{equation}\label{eqn:qmLecture6:520}
\BA = \frac{\Phi_0}{2 \pi r} \phicap.
\end{equation}

The Hamiltonian for the new configuration is

\begin{equation}\label{eqn:qmLecture6:540}
H
= - \lr{ -i \Hbar \spacegrad - q \frac{\Phi_0}{2 \pi r } \phicap }^2
= - \inv{2 m} \lr{ -i \Hbar \inv{R} \PD{\phi}{} - q \frac{\Phi_0}{2 \pi R } }^2.
\end{equation}

Here the replacement \( r \rightarrow R \) makes use of the fact that this problem as been posed with the particle forced to move around the ring at the fixed radius \( R \).

For this transformed Hamiltonian, what are the wave functions?

\begin{dmath}\label{eqn:qmLecture6:560}
\psi(\phi)'
\questionEquals
% \inv{\sqrt{2 \pi}}
e^{i n \phi}.
\end{dmath}

\begin{dmath}\label{eqn:qmLecture6:580}
H \psi
= \inv{2 m}
\lr{ -i \Hbar \inv{R} (i n) - q \frac{\Phi_0}{2 \pi R } }^2 e^{i n \phi}
=
\mathLabelBox
[ labelstyle={below of=m\themathLableNode, below of=m\themathLableNode} ]
{\inv{2 m}
\lr{ \frac{\Hbar n}{R} - q \frac{\Phi_0}{2 \pi R } }^2}{\(E_n\)} e^{i n \phi}.
\end{dmath}

This is very unclassical, since the energy changes in a way that depends on the flux, because particles are seeing magnetic fields that are not present at the point of the particle.

This is sketched in \cref{fig:l6:l6Fig2}.

\imageFigure{../../figures/phy1520/l6Fig2}{Energy variation with flux.}{fig:l6:l6Fig2}{0.3}

we see that there are multiple points that the energies hit the minimum levels


%\EndArticle
%\EndNoBibArticle

         %
% Copyright � 2015 Peeter Joot.  All Rights Reserved.
% Licenced as described in the file LICENSE under the root directory of this GIT repository.
%
%\newcommand{\authorname}{Peeter Joot}
\newcommand{\email}{peeterjoot@protonmail.com}
\newcommand{\basename}{FIXMEbasenameUndefined}
\newcommand{\dirname}{notes/FIXMEdirnameUndefined/}

%\renewcommand{\basename}{qmLecture7}
%\renewcommand{\dirname}{notes/phy1520/}
%\newcommand{\keywords}{PHY1520H}
%\newcommand{\authorname}{Peeter Joot}
\newcommand{\onlineurl}{http://sites.google.com/site/peeterjoot2/math2013/\basename.pdf}
\newcommand{\sourcepath}{\dirname\basename.tex}
\newcommand{\generatetitle}[1]{\chapter{#1}}

\newcommand{\vcsinfo}{%
\section*{}
\noindent{\color{DarkOliveGreen}{\rule{\linewidth}{0.1mm}}}
\paragraph{Document version}
%\paragraph{\color{Maroon}{Document version}}
{
\small
\begin{itemize}
\item Available online at:\\ 
\href{\onlineurl}{\onlineurl}
\item Git Repository: \input{./.revinfo/gitRepo.tex}
\item Source: \sourcepath
\item last commit: \input{./.revinfo/gitCommitString.tex}
\item commit date: \input{./.revinfo/gitCommitDate.tex}
\end{itemize}
}
}

%\PassOptionsToPackage{dvipsnames,svgnames}{xcolor}
\PassOptionsToPackage{square,numbers}{natbib}
\documentclass{scrreprt}

\usepackage[left=2cm,right=2cm]{geometry}
\usepackage[svgnames]{xcolor}
\usepackage{peeters_layout}

\usepackage{natbib}

\usepackage[
colorlinks=true,
bookmarks=false,
pdfauthor={\authorname, \email},
backref 
]{hyperref}

% http://tex.stackexchange.com/questions/75773/how-to-reference-problems-by-the-text-label-in-an-exercise-envioronment
\usepackage[english]{cleveref}
\crefname{Exercise}{exercise}{exercises}
\Crefname{Exercise}{Exercise}{Exercises}

\RequirePackage{titlesec}
\RequirePackage{ifthen}

% http://stackoverflow.com/questions/4932910/date-in-the-tabular-environment
\makeatletter
\let\insertdate\@date
\makeatother

\titleformat{\chapter}[display]
{\bfseries\Large}
{\color{DarkSlateGrey}\filleft \authorname
\ifthenelse{\isundefined{\studentnumber}}{}{\\ \studentnumber}
\ifthenelse{\isundefined{\email}}{}{\\ \email}
\ifthenelse{\isundefined{\dateintitle}}{}{\\ \insertdate}
%\ifthenelse{\isundefined{\coursename}}{}{\\ \coursename} % put in title instead.
}
{4ex}
{\color{DarkOliveGreen}{\titlerule}\color{Maroon}
\vspace{2ex}%
\filright}
[\vspace{2ex}%
\color{DarkOliveGreen}\titlerule
]

\newcommand{\beginArtWithToc}[0]{\begin{document}\tableofcontents}
\newcommand{\beginArtNoToc}[0]{\begin{document}}
\newcommand{\EndNoBibArticle}[0]{\end{document}}
\newcommand{\EndArticle}[0]{\bibliography{Bibliography}\bibliographystyle{plainnat}\end{document}}

% 
%\newcommand{\citep}[1]{\cite{#1}}

\colorSectionsForArticle


%
%%\usepackage{phy1520}
%\usepackage{peeters_braket}
%%\usepackage{peeters_layout_exercise}
%\usepackage{peeters_figures}
%\usepackage{mathtools}
%\usepackage{mhchem}
%
%
%\beginArtNoToc
%\generatetitle{PHY1520H Graduate Quantum Mechanics.  Lecture 7: Aharonov-Bohm effect and Landau levels.  Taught by Prof.\ Arun Paramekanti}
%%\chapter{Aharonov-Bohm effect and Landau levels}
%\label{chap:qmLecture7}
%
%\paragraph{Disclaimer}
%
%Peeter's lecture notes from class.  These may be incoherent and rough.
%
%These are notes for the UofT course PHY1520, Graduate Quantum Mechanics, taught by Prof. Paramekanti, covering \textchapref{{1}} \citep{sakurai2014modern} content.
%
\paragraph{problem set note.}

In the problem set we'll look at interference patterns for two slit electron interference like that of \cref{fig:lecture7:lecture7Fig1}, where a magnetic whisker that introduces flux is added to the configuration.

\imageFigure{../../figures/phy1520/lecture7Fig1}{Two slit interference with magnetic whisker.}{fig:lecture7:lecture7Fig1}{0.2}

\paragraph{Aharonov-Bohm effect (cont.)}

Why do we have the zeros at integral multiples of \( h/q \)?  Consider a particle in a circular trajectory as sketched in \cref{fig:lecture7:lecture7Fig3}

\imageFigure{../../figures/phy1520/lecture7Fig3}{Circular trajectory.}{fig:lecture7:lecture7Fig3}{0.1}

FIXME: Prof mentioned:

\begin{dmath}\label{eqn:qmLecture7:20}
\phi_{\textrm{loop}} = q \frac{ h p/ q }{\Hbar} = 2 \pi p
\end{dmath}

... I'm not sure what that was about now.

In classical mechanics we have

\begin{dmath}\label{eqn:qmLecture7:40}
\oint p dq
\end{dmath}

The integral zero points are related to such a loop, but the \( q \BA \) portion of the momentum \( \Bp - q \BA \) needs to be considered.

\paragraph{Superconductors}
\index{superconductor}

After cooling some materials sufficiently, superconductivity, a complete lack of resistance to electrical flow can be observed.  A resistivity vs temperature plot of such a material is sketched in \cref{fig:lecture7:lecture7Fig4}.

\imageFigure{../../figures/phy1520/lecture7Fig4}{Superconductivity with comparison to superfluidity.}{fig:lecture7:lecture7Fig4}{0.2}

Just like \ce{He^4} can undergo Bose condensation, superconductivity can be explained by a hybrid Bosonic state where electrons are paired into one state containing integral spin.

The Little-Parks experiment puts a superconducting ring around a magnetic whisker as sketched in \cref{fig:lecture7:lecture7Fig6}.

\imageFigure{../../figures/phy1520/lecture7Fig6}{Little-Parks superconducting ring.}{fig:lecture7:lecture7Fig6}{0.1}

This experiment shows that the effective charge of the circulating charge was \( 2 e \), validating the concept of Cooper-pairing, the Bosonic combination (integral spin) of electrons in superconduction.

\paragraph{Motion around magnetic field}
\index{magnetic field}
\index{Little-Parks superconductor}

%F7
%\cref{fig:lecture7:lecture7Fig7}.
%\imageFigure{../../figures/phy1520/lecture7Fig7}{CAPTION: lecture7Fig7}{fig:lecture7:lecture7Fig7}{0.2}

\begin{dmath}\label{eqn:qmLecture7:140}
\omega_{\textrm{c}} = \frac{e B}{m}
\end{dmath}

We work with what is now called the Landau gauge

\begin{dmath}\label{eqn:qmLecture7:60}
\BA = \lr{ 0, B x, 0 }
\end{dmath}

This gives

\begin{dmath}\label{eqn:qmLecture7:80}
\BB
= \spacegrad \cross \BA
= \lr{ \partial_x A_y - \partial_y A_x } \zcap
= B \zcap.
\end{dmath}

An alternate gauge choice, the symmetric gauge, is

\begin{dmath}\label{eqn:qmLecture7:100}
\BA = \lr{ -\frac{B y}{2}, \frac{B x}{2}, 0 },
\end{dmath}

that also has the same magnetic field

\begin{dmath}\label{eqn:qmLecture7:120}
\BB
= \spacegrad \BA
= \lr{ \partial_x A_y - \partial_y A_x } \zcap
= \lr{ \frac{B}{2} - \lr{ - \frac{B}{2} } } \zcap
= B \zcap.
\end{dmath}

We expect the physics for each to have the same results, although the wave functions in one gauge may be more complicated than in the other.

Our Hamiltonian is

\begin{dmath}\label{eqn:qmLecture7:160}
H
= \inv{2 m} \lr{ \Bp - e \BA }^2
= \inv{2 m} \hatp_x^2 + \inv{2 m} \lr{ \hatp_y - e B \hatx }^2
\end{dmath}

We can solve after noting that

\begin{dmath}\label{eqn:qmLecture7:180}
\antisymmetric{\hatp_y}{H} = 0
\end{dmath}

means that

\begin{dmath}\label{eqn:qmLecture7:200}
\Psi(x,y) = e^{i k_y y} \phi(x)
\end{dmath}

The eigensystem

\begin{dmath}\label{eqn:qmLecture7:220}
H \psi(x, y) = E \phi(x, y) ,
\end{dmath}

becomes

\begin{dmath}\label{eqn:qmLecture7:240}
\lr{ \inv{2 m} \hatp_x^2 + \inv{2 m} \lr{ \Hbar k_y - e B \hatx}^2 } \phi(x)
= E \phi(x).
\end{dmath}

This reduced Hamiltonian can be rewritten as

\begin{dmath}\label{eqn:qmLecture7:320}
H_x
= \inv{2 m} p_x^2 + \inv{2 m} e^2 B^2 \lr{ \hatx - \frac{\Hbar k_y}{e B} }^2
\equiv \inv{2 m} p_x^2 + \inv{2} m \omega^2 \lr{ \hatx - x_0 }^2
\end{dmath}

where

\begin{dmath}\label{eqn:qmLecture7:260}
\inv{2 m} e^2 B^2 = \inv{2} m \omega^2,
\end{dmath}

or
\begin{dmath}\label{eqn:qmLecture7:280}
\omega = \frac{ e B}{m} \equiv \omega_\txtc.
\end{dmath}

and

\begin{dmath}\label{eqn:qmLecture7:300}
x_0 = \frac{\Hbar}{k_y}{e B}.
\end{dmath}

But what is this \( x_0 \)?  Because \( k_y \) is not really specified in this problem, we can consider that we have a zero point energy for every \( k_y \), but the oscillator position is shifted for every such value of \( k_y \).  For each set of energy levels \cref{fig:lecture7:lecture7Fig8} we can consider that there is a different zero point energy for each possible \( k_y \).

\imageFigure{../../figures/phy1520/lecture7Fig8}{Energy levels, and Energy vs flux.}{fig:lecture7:lecture7Fig8}{0.1}

\index{degeneracy}
This is an infinitely degenerate system with an infinite number of states for any given energy level.

This tells us that there is a problem, and have to reconsider the assumption that any \( k_y \) is acceptable.

To resolve this we can introduce periodic boundary conditions, imagining that a square is rotated in space forming a cylinder as sketched in \cref{fig:lecture7:lecture7Fig9}.

\imageFigure{../../figures/phy1520/lecture7Fig9}{Landau degeneracy region.}{fig:lecture7:lecture7Fig9}{0.1}

Requiring quantized momentum

\begin{dmath}\label{eqn:qmLecture7:340}
k_y L_y = 2 \pi n,
\end{dmath}

or

\begin{equation}\label{eqn:qmLecture7:360}
k_y = \frac{2 \pi n}{L_y}, \qquad n \in \bbZ,
\end{equation}

gives

\begin{dmath}\label{eqn:qmLecture7:380}
x_0(n) = \frac{\Hbar}{e B} \frac{ 2 \pi n}{L_y},
\end{dmath}

with \( x_0 \le L_x \).  The range is thus restricted to

\begin{dmath}\label{eqn:qmLecture7:400}
\frac{\Hbar}{e B} \frac{ 2 \pi n_{\textrm{max}}}{L_y} = L_x,
\end{dmath}

or

\begin{dmath}\label{eqn:qmLecture7:420}
n_{\textrm{max}} =
\mathLabelBox
[ labelstyle={below of=m\themathLableNode, below of=m\themathLableNode} ]
{L_x L_y}{area} \frac{ e B }{2 \pi \Hbar }
\end{dmath}

That is

\begin{dmath}\label{eqn:qmLecture7:440}
n_{\textrm{max}}
= \frac{\Phi_{\textrm{total}}}{h/e}
= \frac{\Phi_{\textrm{total}}}{\Phi_0}.
\end{dmath}

%F10
%\cref{fig:lecture7:lecture7Fig10}.
%\imageFigure{../../figures/phy1520/lecture7Fig10}{CAPTION: lecture7Fig10}{fig:lecture7:lecture7Fig10}{0.2}

Attempting to measure Hall-effect systems, it was found that the Hall conductivity was quantized like

\begin{dmath}\label{eqn:qmLecture7:460}
\sigma_{x y} = p \frac{e^2}{h}.
\end{dmath}

\index{Landau levels}
This quantization is explained by these Landau levels, and this experimental apparatus provides one of the more accurate ways to measure the fine structure constant.

%\cref{fig:lecture7:lecture7Fig11}.
%\imageFigure{../../figures/phy1520/lecture7Fig11}{CAPTION: lecture7Fig11}{fig:lecture7:lecture7Fig11}{0.2}

%\EndArticle

      \section{Diagonalizating the Quantum Harmonic Oscillator}
         %
% Copyright � 2015 Peeter Joot.  All Rights Reserved.
% Licenced as described in the file LICENSE under the root directory of this GIT repository.
%
%\newcommand{\authorname}{Peeter Joot}
\newcommand{\email}{peeterjoot@protonmail.com}
\newcommand{\basename}{FIXMEbasenameUndefined}
\newcommand{\dirname}{notes/FIXMEdirnameUndefined/}

%\renewcommand{\basename}{harmonicOscDiagonalize}
%\renewcommand{\dirname}{notes/phy1520/}
%%\newcommand{\dateintitle}{}
%%\newcommand{\keywords}{}
%
%\newcommand{\authorname}{Peeter Joot}
\newcommand{\onlineurl}{http://sites.google.com/site/peeterjoot2/math2013/\basename.pdf}
\newcommand{\sourcepath}{\dirname\basename.tex}
\newcommand{\generatetitle}[1]{\chapter{#1}}

\newcommand{\vcsinfo}{%
\section*{}
\noindent{\color{DarkOliveGreen}{\rule{\linewidth}{0.1mm}}}
\paragraph{Document version}
%\paragraph{\color{Maroon}{Document version}}
{
\small
\begin{itemize}
\item Available online at:\\ 
\href{\onlineurl}{\onlineurl}
\item Git Repository: \input{./.revinfo/gitRepo.tex}
\item Source: \sourcepath
\item last commit: \input{./.revinfo/gitCommitString.tex}
\item commit date: \input{./.revinfo/gitCommitDate.tex}
\end{itemize}
}
}

%\PassOptionsToPackage{dvipsnames,svgnames}{xcolor}
\PassOptionsToPackage{square,numbers}{natbib}
\documentclass{scrreprt}

\usepackage[left=2cm,right=2cm]{geometry}
\usepackage[svgnames]{xcolor}
\usepackage{peeters_layout}

\usepackage{natbib}

\usepackage[
colorlinks=true,
bookmarks=false,
pdfauthor={\authorname, \email},
backref 
]{hyperref}

% http://tex.stackexchange.com/questions/75773/how-to-reference-problems-by-the-text-label-in-an-exercise-envioronment
\usepackage[english]{cleveref}
\crefname{Exercise}{exercise}{exercises}
\Crefname{Exercise}{Exercise}{Exercises}

\RequirePackage{titlesec}
\RequirePackage{ifthen}

% http://stackoverflow.com/questions/4932910/date-in-the-tabular-environment
\makeatletter
\let\insertdate\@date
\makeatother

\titleformat{\chapter}[display]
{\bfseries\Large}
{\color{DarkSlateGrey}\filleft \authorname
\ifthenelse{\isundefined{\studentnumber}}{}{\\ \studentnumber}
\ifthenelse{\isundefined{\email}}{}{\\ \email}
\ifthenelse{\isundefined{\dateintitle}}{}{\\ \insertdate}
%\ifthenelse{\isundefined{\coursename}}{}{\\ \coursename} % put in title instead.
}
{4ex}
{\color{DarkOliveGreen}{\titlerule}\color{Maroon}
\vspace{2ex}%
\filright}
[\vspace{2ex}%
\color{DarkOliveGreen}\titlerule
]

\newcommand{\beginArtWithToc}[0]{\begin{document}\tableofcontents}
\newcommand{\beginArtNoToc}[0]{\begin{document}}
\newcommand{\EndNoBibArticle}[0]{\end{document}}
\newcommand{\EndArticle}[0]{\bibliography{Bibliography}\bibliographystyle{plainnat}\end{document}}

% 
%\newcommand{\citep}[1]{\cite{#1}}

\colorSectionsForArticle


%
%\usepackage{peeters_layout_exercise}
%\usepackage{peeters_braket}
%\usepackage{peeters_figures}
%
%\beginArtNoToc
%
%\generatetitle{Quantum SHO ladder operators as a diagonal change of basis for the Heisenberg EOMs}
%\generatetitle{Diagonalizing the Quantum Harmonic Oscillator}
%\label{chap:harmonicOscDiagonalize}

Many authors pull the definitions of the raising and lowering (or ladder) operators out of their butt with no attempt at motivation.  This is pointed out nicely in \citep{eli:quantumLadderOperators} by Eli along with one justification based on factoring the Hamiltonian.

In \citep{sakurai2014modern:shoTimeEvolution} is a small exception to the usual presentation.  In that text, these operators are defined as usual with no motivation.  However, after the utility of these operators has been shown, the raising and lowering operators show up in a context that does provide that missing motivation as a side effect.
It doesn't look like the author was trying to provide a motivation, but it can be interpreted that way.

When seeking the time evolution of Heisenberg-picture position and momentum operators, we will see that those solutions can be trivially expressed using the raising and lowering operators.  No special tools nor black magic is required to find the structure of these operators.  Unfortunately, we must first switch to both the Heisenberg picture representation of the position and momentum operators, and also employ the Heisenberg equations of motion.  Neither of these last two fit into standard narrative of most introductory quantum mechanics treatments.  We will also see that these raising and lowering ``operators'' could also be introduced in classical mechanics, provided we were attempting to solve the SHO system using the Hamiltonian equations of motion.

I'll outline this route to finding the structure of the ladder operators below.  Because these are encountered trying to solve the time evolution problem, I'll first show a simpler way to solve that problem.  Because that simpler method depends a bit on lucky observation and is somewhat unstructured, I'll then outline a more structured procedure that leads to the ladder operators directly, also providing the solution to the time evolution problem as a side effect.

The starting point is the Heisenberg equations of motion.  For a time independent Hamiltonian \( H \), and a Heisenberg operator \( A^{(H)} \), those equations are

\begin{dmath}\label{eqn:harmonicOscDiagonalize:20}
\ddt{A^{(H)}} = \inv{i \Hbar} \antisymmetric{A^{(H)}}{H}.
\end{dmath}

Here the Heisenberg operator \( A^{(H)} \) is related to the Schr\"{o}dinger operator \( A^{(S)} \) by

\begin{dmath}\label{eqn:harmonicOscDiagonalize:60}
A^{(H)} = U^\dagger A^{(S)} U,
\end{dmath}

where \( U \) is the time evolution operator.  For this discussion, we need only know that \( U \) commutes with \( H \), and do not need to know the specific structure of that operator.  In particular, the Heisenberg equations of motion take the form

\begin{dmath}\label{eqn:harmonicOscDiagonalize:80}
\ddt{A^{(H)}}
= \inv{i \Hbar}
\antisymmetric{A^{(H)}}{H}
= \inv{i \Hbar}
\antisymmetric{U^\dagger A^{(S)} U}{H}
= \inv{i \Hbar}
\lr{
U^\dagger A^{(S)} U H
- H U^\dagger A^{(S)} U
}
= \inv{i \Hbar}
\lr{
U^\dagger A^{(S)} H U
- U^\dagger H A^{(S)} U
}
= \inv{i \Hbar} U^\dagger \antisymmetric{A^{(S)}}{H} U.
\end{dmath}

The Hamiltonian for the harmonic oscillator, with Schr\"{o}dinger-picture position and momentum operators \( x, p \) is

\begin{dmath}\label{eqn:harmonicOscDiagonalize:40}
H = \frac{p^2}{2m} + \inv{2} m \omega^2 x^2,
\end{dmath}

so the equations of motions are

\begin{dmath}\label{eqn:harmonicOscDiagonalize:100}
\ddt{x^{(H)}}
= \inv{i \Hbar} U^\dagger \antisymmetric{x}{H} U
= \inv{i \Hbar} U^\dagger \antisymmetric{x}{\frac{p^2}{2m}} U
= \inv{2 m i \Hbar} U^\dagger \lr{ i \Hbar \PD{p}{p^2} } U
= \inv{m } U^\dagger p U
= \inv{m } p^{(H)},
\end{dmath}

and
\begin{dmath}\label{eqn:harmonicOscDiagonalize:120}
\ddt{p^{(H)}}
= \inv{i \Hbar} U^\dagger \antisymmetric{p}{H} U
= \inv{i \Hbar} U^\dagger \antisymmetric{p}{\inv{2} m \omega^2 x^2 } U
= \frac{m \omega^2}{2 i \Hbar} U^\dagger \lr{ -i \Hbar \PD{x}{x^2} } U
= -m \omega^2 U^\dagger x U
= -m \omega^2 x^{(H)}.
\end{dmath}

In the Heisenberg picture the equations of motion are precisely those of classical Hamiltonian mechanics, except that we are dealing with operators instead of scalars

\begin{dmath}\label{eqn:harmonicOscDiagonalize:140}
\begin{aligned}
\ddt{p^{(H)}} &= -m \omega^2 x^{(H)} \\
\ddt{x^{(H)}} &= \inv{m } p^{(H)}.
\end{aligned}
\end{dmath}

In the text the ladder operators are used to simplify the solution of these coupled equations, since they can decouple them.  That's not really required since we can solve them directly in matrix form with little work

\begin{dmath}\label{eqn:harmonicOscDiagonalize:160}
\ddt{}
\begin{bmatrix}
p^{(H)} \\
x^{(H)}
\end{bmatrix}
=
\begin{bmatrix}
0 & -m \omega^2 \\
\inv{m} & 0
\end{bmatrix}
\begin{bmatrix}
p^{(H)} \\
x^{(H)}
\end{bmatrix},
\end{dmath}

or, with length scaled variables

\begin{dmath}\label{eqn:harmonicOscDiagonalize:180}
\ddt{}
\begin{bmatrix}
\frac{p^{(H)}}{m \omega} \\
x^{(H)}
\end{bmatrix}
=
\begin{bmatrix}
0 & -\omega \\
\omega & 0
\end{bmatrix}
\begin{bmatrix}
\frac{p^{(H)}}{m \omega} \\
x^{(H)}
\end{bmatrix}
=
-i \omega
\PauliY
\begin{bmatrix}
\frac{p^{(H)}}{m \omega} \\
x^{(H)}
\end{bmatrix}
=
-i \omega
\sigma_y
\begin{bmatrix}
\frac{p^{(H)}}{m \omega} \\
x^{(H)}
\end{bmatrix}.
\end{dmath}

Writing \( y = \begin{bmatrix} \frac{p^{(H)}}{m \omega} \\ x^{(H)} \end{bmatrix} \), the solution can then be written immediately as

\begin{dmath}\label{eqn:harmonicOscDiagonalize:200}
y(t)
=
\exp\lr{ -i \omega \sigma_y t } y(0)
=
\lr{ \cos \lr{ \omega t } I - i \sigma_y \sin\lr{ \omega t } } y(0)
=
\begin{bmatrix}
\cos\lr{ \omega t } & \sin\lr{ \omega t } \\
-\sin\lr{ \omega t } & \cos\lr{ \omega t }
\end{bmatrix}
y(0),
\end{dmath}

or

\begin{dmath}\label{eqn:harmonicOscDiagonalize:220}
\begin{aligned}
\frac{p^{(H)}(t)}{m \omega} &= \cos\lr{ \omega t } \frac{p^{(H)}(0)}{m \omega} + \sin\lr{ \omega t } x^{(H)}(0) \\
x^{(H)}(t) &= -\sin\lr{ \omega t } \frac{p^{(H)}(0)}{m \omega} + \cos\lr{ \omega t } x^{(H)}(0).
\end{aligned}
\end{dmath}

This solution depends on being lucky enough to recognize that the matrix has a Pauli matrix as a factor (which squares to unity, and allows the exponential to be evaluated easily.)

If we hadn't been that observant, then the first tool we'd have used instead would have been to diagonalize the matrix.  For such diagonalization, it's natural to work in completely dimensionless variables.  Such a non-dimensionalisation can be had by defining

\begin{dmath}\label{eqn:harmonicOscDiagonalize:240}
x_0 = \sqrt{\frac{\Hbar}{m \omega}},
\end{dmath}

and dividing the working (operator) variables through by those values.  Let \( z = \inv{x_0} y \), and \( \tau = \omega t \) so that the equations of motion are

\begin{dmath}\label{eqn:harmonicOscDiagonalize:260}
\frac{dz}{d\tau}
=
\begin{bmatrix}
0 & -1 \\
1 & 0
\end{bmatrix}
z.
\end{dmath}

This matrix can be diagonalized as

\begin{dmath}\label{eqn:harmonicOscDiagonalize:280}
A =
\begin{bmatrix}
0 & -1 \\
1 & 0
\end{bmatrix}
=
V
\begin{bmatrix}
i & 0  \\
0 & -i
\end{bmatrix}
V^{-1},
\end{dmath}

where

\begin{dmath}\label{eqn:harmonicOscDiagonalize:300}
V =
\inv{\sqrt{2}}
\begin{bmatrix}
i & -i \\
1 & 1
\end{bmatrix}.
\end{dmath}

The equations of motion can now be written

\begin{dmath}\label{eqn:harmonicOscDiagonalize:320}
\frac{d}{d\tau} \lr{ V^{-1} z } =
\begin{bmatrix}
i & 0  \\
0 & -i
\end{bmatrix}
\lr{ V^{-1} z }.
\end{dmath}

This final change of variables \( V^{-1} z \) decouples the system as desired.  Expanding that gives

\begin{dmath}\label{eqn:harmonicOscDiagonalize:340}
V^{-1} z
=
\inv{\sqrt{2}}
\begin{bmatrix}
-i & 1 \\
 i & 1
\end{bmatrix}
\begin{bmatrix}
\frac{p^{(H)}}{x_0 m \omega} \\
\frac{x^{(H)}}{x_0}
\end{bmatrix}
=
\inv{\sqrt{2} x_0}
\begin{bmatrix}
-i \frac{p^{(H)}}{m \omega} + x^{(H)} \\
i \frac{p^{(H)}}{m \omega} + x^{(H)}
\end{bmatrix}
=
\begin{bmatrix}
a^\dagger \\
a
\end{bmatrix},
\end{dmath}

where
\begin{dmath}\label{eqn:harmonicOscDiagonalize:400}
\begin{aligned}
a^\dagger &= \sqrt{\frac{m \omega}{2 \Hbar}} \lr{ -i \frac{p^{(H)}}{m \omega} + x^{(H)} } \\
a &= \sqrt{\frac{m \omega}{2 \Hbar}} \lr{ i \frac{p^{(H)}}{m \omega} + x^{(H)} }.
\end{aligned}
\end{dmath}

Lo and behold, we have the standard form of the raising and lowering operators, and can write the system equations as

\begin{dmath}\label{eqn:harmonicOscDiagonalize:360}
\begin{aligned}
\ddt{a^\dagger} &= i \omega a^\dagger \\
\ddt{a} &= -i \omega a.
\end{aligned}
\end{dmath}

It is actually a bit fluky that this matched exactly, since we could have chosen eigenvectors that differ by constant phase factors, like

\begin{dmath}\label{eqn:harmonicOscDiagonalize:380}
V = \inv{\sqrt{2}}
\begin{bmatrix}
i e^{i\phi} & -i e^{i \psi} \\
1 e^{i\phi} & e^{i \psi}
\end{bmatrix},
\end{dmath}

so

\begin{dmath}\label{eqn:harmonicOscDiagonalize:341}
V^{-1} z
=
\frac{e^{-i(\phi + \psi)}}{\sqrt{2}}
\begin{bmatrix}
-i e^{i\psi} & e^{i \psi} \\
i e^{i\phi} & e^{i \phi}
\end{bmatrix}
\begin{bmatrix}
\frac{p^{(H)}}{x_0 m \omega} \\
\frac{x^{(H)}}{x_0}
\end{bmatrix}
=
\inv{\sqrt{2} x_0}
\begin{bmatrix}
-i e^{i\phi} \frac{p^{(H)}}{m \omega} + e^{i\phi} x^{(H)} \\
i e^{i\psi} \frac{p^{(H)}}{m \omega} + e^{i\psi} x^{(H)}
\end{bmatrix}
=
\begin{bmatrix}
e^{i\phi} a^\dagger \\
e^{i\psi} a
\end{bmatrix}.
\end{dmath}

To make the resulting pairs of operators Hermitian conjugates, we'd want to constrain those constant phase factors by setting \( \phi = -\psi \).  If we were only interested in solving the time evolution problem no such additional constraints are required.

The raising and lowering operators are seen to naturally occur when seeking the solution of the Heisenberg equations of motion.  This is found using the standard technique of non-dimensionalisation and then seeking a change of basis that diagonalizes the system matrix.  Because the Heisenberg equations of motion are identical to the classical Hamiltonian equations of motion in this case, what we call the raising and lowering operators in quantum mechanics could also be utilized in the classical simple harmonic oscillator problem.  However, in a classical context we wouldn't have a justification to call this more than a change of basis.

%\EndArticle

      \section{Generalized Gaussian integrals}
         %
% Copyright � 2015 Peeter Joot.  All Rights Reserved.
% Licenced as described in the file LICENSE under the root directory of this GIT repository.
%
%\newcommand{\authorname}{Peeter Joot}
\newcommand{\email}{peeterjoot@protonmail.com}
\newcommand{\basename}{FIXMEbasenameUndefined}
\newcommand{\dirname}{notes/FIXMEdirnameUndefined/}

%\renewcommand{\basename}{generalizedGaussian}
%\renewcommand{\dirname}{notes/phy1520/}
%\newcommand{\dateintitle}{}
%\newcommand{\keywords}{}

%\newcommand{\authorname}{Peeter Joot}
\newcommand{\onlineurl}{http://sites.google.com/site/peeterjoot2/math2013/\basename.pdf}
\newcommand{\sourcepath}{\dirname\basename.tex}
\newcommand{\generatetitle}[1]{\chapter{#1}}

\newcommand{\vcsinfo}{%
\section*{}
\noindent{\color{DarkOliveGreen}{\rule{\linewidth}{0.1mm}}}
\paragraph{Document version}
%\paragraph{\color{Maroon}{Document version}}
{
\small
\begin{itemize}
\item Available online at:\\ 
\href{\onlineurl}{\onlineurl}
\item Git Repository: \input{./.revinfo/gitRepo.tex}
\item Source: \sourcepath
\item last commit: \input{./.revinfo/gitCommitString.tex}
\item commit date: \input{./.revinfo/gitCommitDate.tex}
\end{itemize}
}
}

%\PassOptionsToPackage{dvipsnames,svgnames}{xcolor}
\PassOptionsToPackage{square,numbers}{natbib}
\documentclass{scrreprt}

\usepackage[left=2cm,right=2cm]{geometry}
\usepackage[svgnames]{xcolor}
\usepackage{peeters_layout}

\usepackage{natbib}

\usepackage[
colorlinks=true,
bookmarks=false,
pdfauthor={\authorname, \email},
backref 
]{hyperref}

% http://tex.stackexchange.com/questions/75773/how-to-reference-problems-by-the-text-label-in-an-exercise-envioronment
\usepackage[english]{cleveref}
\crefname{Exercise}{exercise}{exercises}
\Crefname{Exercise}{Exercise}{Exercises}

\RequirePackage{titlesec}
\RequirePackage{ifthen}

% http://stackoverflow.com/questions/4932910/date-in-the-tabular-environment
\makeatletter
\let\insertdate\@date
\makeatother

\titleformat{\chapter}[display]
{\bfseries\Large}
{\color{DarkSlateGrey}\filleft \authorname
\ifthenelse{\isundefined{\studentnumber}}{}{\\ \studentnumber}
\ifthenelse{\isundefined{\email}}{}{\\ \email}
\ifthenelse{\isundefined{\dateintitle}}{}{\\ \insertdate}
%\ifthenelse{\isundefined{\coursename}}{}{\\ \coursename} % put in title instead.
}
{4ex}
{\color{DarkOliveGreen}{\titlerule}\color{Maroon}
\vspace{2ex}%
\filright}
[\vspace{2ex}%
\color{DarkOliveGreen}\titlerule
]

\newcommand{\beginArtWithToc}[0]{\begin{document}\tableofcontents}
\newcommand{\beginArtNoToc}[0]{\begin{document}}
\newcommand{\EndNoBibArticle}[0]{\end{document}}
\newcommand{\EndArticle}[0]{\bibliography{Bibliography}\bibliographystyle{plainnat}\end{document}}

% 
%\newcommand{\citep}[1]{\cite{#1}}

\colorSectionsForArticle


%
%\usepackage{peeters_layout_exercise}
%\usepackage{peeters_braket}
%\usepackage{peeters_figures}
%\usepackage{txfonts} % \ointclockwise
%
%\beginArtNoToc
%
%\generatetitle{Generalized Gaussian integrals}
%\chapter{Generalized Gaussian integrals}
%\label{chap:generalizedGaussian}

\index{Gaussian integral}
Both \citep{sakurai2014modern} and \citep{zee2005quantum} use Gaussian integrals with both (negative) real, and imaginary arguments, which give the impression that the following is true:

\begin{dmath}\label{eqn:generalizedGaussian:20}
\int_{-\infty}^\infty \exp\lr{ a x^2 } dx = \sqrt{\frac{-\pi}{a}},
\end{dmath}

even when \( a \) is not a real negative constant, and in particular, with values \( a = \pm i \).  Clearly this doesn't follow by just making a substitution \( x \rightarrow x/\sqrt{a} \), since that moves the integration range onto a rotated path in the complex plane when \( a \) is \( \pm i \).  However, with some care, it can be shown that \cref{eqn:generalizedGaussian:20} holds provided \( \Real a \le 0 \).

To show this, this integral will be considered for the pure real case, purely imaginary, and finally the complex case with non-zero real and imaginary parts for \( a \).

\paragraph{Real (negative) case.}

The first special case is \( \int_{-\infty}^\infty \exp\lr{ - x^2 } dx = \sqrt{\pi} \) which is easy to derive using the usual square it and integrate in circular coordinates trick.

\paragraph{Purely imaginary cases.}

Let's handle the \( a = \pm i \) cases next.  These can be evaluated by considering integrals over the contours of \cref{fig:twoContours45Degrees:twoContours45DegreesFig1}, where the upper plane contour is used for \( a = i \) and the lower plane contour for \( a = -i \).

\imageFigure{../../figures/phy1520/twoContours45DegreesFig1}{Contours for \( a = \pm i \).}{fig:twoContours45Degrees:twoContours45DegreesFig1}{0.3}

Since there are no poles, the integral over either such contour is zero.  Credit for figuring out how to tackle that integral and what contour to use goes to Dr MV, on stackexchange \citep{se:evalDefiniteIntegralOf}.

For the upper plane contour we have

\begin{dmath}\label{eqn:generalizedGaussian:40}
0
= \ointctrclockwise \exp\lr{ i z^2 } dz
= \int_0^R \exp\lr{ i x^2 } dx
+ \int_0^{\pi/4} \exp\lr{ i R^2 e^{2 i \theta} } R i e^{i\theta} d\theta
+ \int_R^0 \exp\lr{ i^2 t^2 } e^{i\pi/4} dt.
\end{dmath}

Observe that \( i e^{2 i \theta} = i\cos(2 \theta) - \sin(2\theta) \) which has a negative real part for all values of \( \theta \ne 0 \).  Provided the contour is slightly deformed from the axis, that second integral has a term of the form \( \sim R e^{-R^2} \) which tends to zero as \( R \rightarrow \infty \).  So in the limit, this is

\begin{dmath}\label{eqn:generalizedGaussian:60}
\int_0^\infty \exp\lr{ i x^2 } dx
= \sqrt{\pi} e^{i\pi/4}/2,
\end{dmath}

or
\begin{dmath}\label{eqn:generalizedGaussian:80}
\int_{-\infty}^\infty \exp\lr{ i x^2 } dx
= \sqrt{i \pi},
\end{dmath}

a special case of \cref{eqn:generalizedGaussian:20} as desired.  For \( a = -i \) integrating around the lower plane contour, we have

\begin{dmath}\label{eqn:generalizedGaussian:100}
0
= \ointclockwise \exp\lr{ -i z^2 } dz
= \int_0^R \exp\lr{ i x^2 } dx
+ \int_0^{-\pi/4} \exp\lr{ -i R^2 e^{2 i \theta} } R i e^{i\theta} d\theta
+ \int_R^0 \exp\lr{ -i (-i) t^2 } e^{-i\pi/4} dt.
\end{dmath}

This time, in the second integral we also have \( -i R^2 e^{2 i \theta} = i R^2 \cos(2 \theta) + \sin(2 \theta) \), which also has a negative real part for \( \theta \in (0, \pi/4] \).  Again the contour needs to be infinitesimally deformed\footnote{Distorting the contour in this fashion seems somewhat like handwaving.  A better approach would probably follow \citep{lepage1980cva:theoremForTrigIntegrals} where Jordan's lemma is covered.  It doesn't look like Jordan's lemma applies as is to this case, but the arguments look like they could be adapted appropriately.}, placed just lower than the axis.

This time we find

\begin{dmath}\label{eqn:generalizedGaussian:120}
\int_{-\infty}^\infty \exp\lr{ -i x^2 } dx
= \sqrt{-i \pi},
\end{dmath}

another special case of \cref{eqn:generalizedGaussian:20}.

\paragraph{Completely complex case.}

A similar trick can be used to evaluate the more general cases, but a bit of thought is required to figure out the contours required.  More precisely, while these contours will still have a wedge of pie shape, as sketched in \cref{fig:twoContoursTheta:twoContoursThetaFig2}, we have to figure out the angle subtended by the edge of this piece of pie.

\imageFigure{../../figures/phy1520/twoContoursThetaFig2}{Contours for complex \( a \).}{fig:twoContoursTheta:twoContoursThetaFig2}{0.3}

To evaluate an integral consider

\begin{dmath}\label{eqn:generalizedGaussian:140}
0
= \oint \exp\lr{ e^{i\phi} z^2 } dz
= \int_0^R \exp\lr{ e^{i\phi} x^2 } dx
+ \int_0^{\theta} \exp\lr{ e^{i\phi} R^2 e^{2 i \mu} } R i e^{i\mu} d\mu
+ \int_R^0 \exp\lr{ e^{i\phi} e^{2 i \theta} t^2 } e^{i\theta} dt,
\end{dmath}

where \( \phi \in (\pi/2, \pi) \cup (\pi,3\pi/2) \).  We have a hope of evaluating this last integral if \( \phi + 2 \theta = \pi \), or

\begin{dmath}\label{eqn:generalizedGaussian:160}
\theta = (\pi -\phi)/2,
\end{dmath}

giving

\begin{dmath}\label{eqn:generalizedGaussian:180}
\int_0^R \exp\lr{ e^{i\phi} x^2 } dx
=
e^{i\lr{\pi - \phi}/2} \int_0^R \exp\lr{ -t^2 } dt
- \int_0^{\theta} \exp\lr{ R^2 \lr{ \cos\lr{\phi + 2 \mu} + i \sin\lr{\phi + 2 \mu}} } R i e^{i\mu} d\mu.
\end{dmath}

If the cosine is always negative on the chosen contours, then that integral will vanish in the \( R \rightarrow \infty \) limit.  This turns out to be the case, which can be confirmed by considering each of the contours in sequence.  If the upper plane contour is used to evaluate \cref{eqn:generalizedGaussian:140} for the \( \phi \in (\pi/2,\pi) \) case, we have

\begin{dmath}\label{eqn:generalizedGaussian:200}
\theta \in (0, \pi/4).
\end{dmath}

Since \( \phi + 2\theta = \pi \), we have

\begin{dmath}\label{eqn:generalizedGaussian:220}
\phi + 2 \mu \in (\pi/2, \pi),
\end{dmath}

and find that the cosine is strictly negative on that contour for that range of \( \phi \).  Picking the lower plane contour for the \( \phi \in (\pi, 3\pi/2) \) range, we have

\begin{dmath}\label{eqn:generalizedGaussian:240}
\theta \in (-\pi/4, 0),
\end{dmath}

and so

\begin{dmath}\label{eqn:generalizedGaussian:260}
\phi + 2 \mu \in (\pi/2, 3\pi/2).
\end{dmath}

For this range of \( \phi \) the cosine on the lower plane contour is again negative as desired, so in the infinite \( R \) limit we have

\begin{dmath}\label{eqn:generalizedGaussian:280}
\int_0^\infty \exp\lr{ e^{i\phi} x^2 } dx
=
\inv{2} \sqrt{ -\pi e^{-i\phi} }.
\end{dmath}

The points at \( \phi = \pi/2, \pi, 3\pi/2 \) were omitted, but we've found the same result at those points, completing the verification of \cref{eqn:generalizedGaussian:20} for all \( \Real a \le 0 \).

%\EndArticle

      \section{Constant magnetic solenoid field}
         %
% Copyright � 2015 Peeter Joot.  All Rights Reserved.
% Licenced as described in the file LICENSE under the root directory of this GIT repository.
%
%\newcommand{\authorname}{Peeter Joot}
\newcommand{\email}{peeterjoot@protonmail.com}
\newcommand{\basename}{FIXMEbasenameUndefined}
\newcommand{\dirname}{notes/FIXMEdirnameUndefined/}

%\renewcommand{\basename}{solenoidConstantField}
%\renewcommand{\dirname}{notes/phy1520/}
%%\newcommand{\dateintitle}{}
%%\newcommand{\keywords}{}
%
%\newcommand{\authorname}{Peeter Joot}
\newcommand{\onlineurl}{http://sites.google.com/site/peeterjoot2/math2013/\basename.pdf}
\newcommand{\sourcepath}{\dirname\basename.tex}
\newcommand{\generatetitle}[1]{\chapter{#1}}

\newcommand{\vcsinfo}{%
\section*{}
\noindent{\color{DarkOliveGreen}{\rule{\linewidth}{0.1mm}}}
\paragraph{Document version}
%\paragraph{\color{Maroon}{Document version}}
{
\small
\begin{itemize}
\item Available online at:\\ 
\href{\onlineurl}{\onlineurl}
\item Git Repository: \input{./.revinfo/gitRepo.tex}
\item Source: \sourcepath
\item last commit: \input{./.revinfo/gitCommitString.tex}
\item commit date: \input{./.revinfo/gitCommitDate.tex}
\end{itemize}
}
}

%\PassOptionsToPackage{dvipsnames,svgnames}{xcolor}
\PassOptionsToPackage{square,numbers}{natbib}
\documentclass{scrreprt}

\usepackage[left=2cm,right=2cm]{geometry}
\usepackage[svgnames]{xcolor}
\usepackage{peeters_layout}

\usepackage{natbib}

\usepackage[
colorlinks=true,
bookmarks=false,
pdfauthor={\authorname, \email},
backref 
]{hyperref}

% http://tex.stackexchange.com/questions/75773/how-to-reference-problems-by-the-text-label-in-an-exercise-envioronment
\usepackage[english]{cleveref}
\crefname{Exercise}{exercise}{exercises}
\Crefname{Exercise}{Exercise}{Exercises}

\RequirePackage{titlesec}
\RequirePackage{ifthen}

% http://stackoverflow.com/questions/4932910/date-in-the-tabular-environment
\makeatletter
\let\insertdate\@date
\makeatother

\titleformat{\chapter}[display]
{\bfseries\Large}
{\color{DarkSlateGrey}\filleft \authorname
\ifthenelse{\isundefined{\studentnumber}}{}{\\ \studentnumber}
\ifthenelse{\isundefined{\email}}{}{\\ \email}
\ifthenelse{\isundefined{\dateintitle}}{}{\\ \insertdate}
%\ifthenelse{\isundefined{\coursename}}{}{\\ \coursename} % put in title instead.
}
{4ex}
{\color{DarkOliveGreen}{\titlerule}\color{Maroon}
\vspace{2ex}%
\filright}
[\vspace{2ex}%
\color{DarkOliveGreen}\titlerule
]

\newcommand{\beginArtWithToc}[0]{\begin{document}\tableofcontents}
\newcommand{\beginArtNoToc}[0]{\begin{document}}
\newcommand{\EndNoBibArticle}[0]{\end{document}}
\newcommand{\EndArticle}[0]{\bibliography{Bibliography}\bibliographystyle{plainnat}\end{document}}

% 
%\newcommand{\citep}[1]{\cite{#1}}

\colorSectionsForArticle


%
%\usepackage{peeters_layout_exercise}
%%\usepackage{peeters_braket}
%\usepackage{peeters_figures}
%
%\beginArtNoToc
%
%\generatetitle{Constant magnetic solenoid field}
%\label{chap:solenoidConstantField}

In \citep{sakurai2014modern} the following vector potential

\begin{dmath}\label{eqn:solenoidConstantField:20}
\BA = \frac{B \rho_a^2}{2 \rho} \phicap,
\end{dmath}

is introduced in a discussion on the Aharonov-Bohm effect, for configurations where the interior field of a solenoid is either a constant \( \BB \) or zero.

I wasn't able to make sense of this since the field I was calculating was zero for all \( \rho \ne 0 \)

\begin{dmath}\label{eqn:solenoidConstantField:40}
\BB
= \spacegrad \cross \BA
= \lr{ \rhocap \partial_\rho + \zcap \partial_z + \frac{\phicap}{\rho} \partial_\phi } \cross \frac{B \rho_a^2}{2 \rho} \phicap
= \lr{ \rhocap \partial_\rho + \frac{\phicap}{\rho} \partial_\phi } \cross \frac{B \rho_a^2}{2 \rho} \phicap
=
\frac{B \rho_a^2}{2}
\rhocap \cross \phicap \partial_\rho \lr{ \inv{\rho} }
+
\frac{B \rho_a^2}{2 \rho}
\frac{\phicap}{\rho} \cross \partial_\phi \phicap
=
\frac{B \rho_a^2}{2 \rho^2} \lr{ -\zcap + \phicap \cross \partial_\phi \phicap}.
\end{dmath}

Note that the \( \rho \) partial requires that \( \rho \ne 0 \).  To expand the cross product in the second term let \( j = \Be_1 \Be_2 \), and expand using a Geometric Algebra representation of the unit vector

\begin{dmath}\label{eqn:solenoidConstantField:60}
\phicap \cross \partial_\phi \phicap
=
\Be_2 e^{j \phi} \cross \lr{ \Be_2 \Be_1 \Be_2 e^{j \phi} }
=
- \Be_1 \Be_2 \Be_3
\gpgradetwo{
\Be_2 e^{j \phi} (-\Be_1) e^{j \phi}
}
=
\Be_1 \Be_2 \Be_3 \Be_2 \Be_1
= \Be_3
= \zcap.
\end{dmath}

So, provided \( \rho \ne 0 \), \( \BB = 0 \).

The errata \citep{sakurai2014modernErrata} provides the clarification, showing that a \( \rho > \rho_a \) constraint is required for this potential to produce the desired results.  Continuity at \( \rho = \rho_a \) means that in the interior (or at least on the boundary) we must have one of

\begin{dmath}\label{eqn:solenoidConstantField:80}
\BA = \frac{B \rho_a}{2} \phicap,
\end{dmath}

or

\begin{dmath}\label{eqn:solenoidConstantField:100}
\BA = \frac{B \rho}{2} \phicap.
\end{dmath}

The first doesn't work, but the second does

\begin{dmath}\label{eqn:solenoidConstantField:120}
\BB
= \spacegrad \cross \BA
= \lr{ \rhocap \partial_\rho + \zcap \partial_z + \frac{\phicap}{\rho} \partial_\phi } \cross \frac{B \rho}{2 } \phicap
=
\frac{B }{2 } \rhocap \cross \phicap
+
\frac{B \rho}{2 }
\frac{\phicap}{\rho} \cross \partial_\phi \phicap
= B \zcap.
\end{dmath}

So the vector potential that we want for a constant \( B \zcap \) field in the interior \( \rho < \rho_a \) of a cylindrical space, we need

\begin{dmath}\label{eqn:solenoidConstantField:140}
\BA =
\left\{
\begin{array}{l l}
\frac{B \rho_a^2}{2 \rho} \phicap & \quad \mbox{if \( \rho \ge \rho_a \) } \\
\frac{B \rho}{2} \phicap & \quad \mbox{if \( \rho < \rho_a \).}
\end{array}
\right.
\end{dmath}

An example of the magnitude of potential is graphed in \cref{fig:solenoidPotential:solenoidPotentialFig1}.

\mathImageFigure{../../figures/phy1520/solenoidPotentialFig1}{Vector potential for constant field in cylindrical region.}{fig:solenoidPotential:solenoidPotentialFig1}{0.3}{vectorSolenoid.jl}

%\EndArticle

      \section{Lagrangian for magnetic portion of Lorentz force}
         %
% Copyright � 2015 Peeter Joot.  All Rights Reserved.
% Licenced as described in the file LICENSE under the root directory of this GIT repository.
%
%\newcommand{\authorname}{Peeter Joot}
\newcommand{\email}{peeterjoot@protonmail.com}
\newcommand{\basename}{FIXMEbasenameUndefined}
\newcommand{\dirname}{notes/FIXMEdirnameUndefined/}

%\renewcommand{\basename}{magneticLorentzForceLagrangian}
%\renewcommand{\dirname}{notes/phy1520/}
%%\newcommand{\dateintitle}{}
%%\newcommand{\keywords}{}
%
%\newcommand{\authorname}{Peeter Joot}
\newcommand{\onlineurl}{http://sites.google.com/site/peeterjoot2/math2013/\basename.pdf}
\newcommand{\sourcepath}{\dirname\basename.tex}
\newcommand{\generatetitle}[1]{\chapter{#1}}

\newcommand{\vcsinfo}{%
\section*{}
\noindent{\color{DarkOliveGreen}{\rule{\linewidth}{0.1mm}}}
\paragraph{Document version}
%\paragraph{\color{Maroon}{Document version}}
{
\small
\begin{itemize}
\item Available online at:\\ 
\href{\onlineurl}{\onlineurl}
\item Git Repository: \input{./.revinfo/gitRepo.tex}
\item Source: \sourcepath
\item last commit: \input{./.revinfo/gitCommitString.tex}
\item commit date: \input{./.revinfo/gitCommitDate.tex}
\end{itemize}
}
}

%\PassOptionsToPackage{dvipsnames,svgnames}{xcolor}
\PassOptionsToPackage{square,numbers}{natbib}
\documentclass{scrreprt}

\usepackage[left=2cm,right=2cm]{geometry}
\usepackage[svgnames]{xcolor}
\usepackage{peeters_layout}

\usepackage{natbib}

\usepackage[
colorlinks=true,
bookmarks=false,
pdfauthor={\authorname, \email},
backref 
]{hyperref}

% http://tex.stackexchange.com/questions/75773/how-to-reference-problems-by-the-text-label-in-an-exercise-envioronment
\usepackage[english]{cleveref}
\crefname{Exercise}{exercise}{exercises}
\Crefname{Exercise}{Exercise}{Exercises}

\RequirePackage{titlesec}
\RequirePackage{ifthen}

% http://stackoverflow.com/questions/4932910/date-in-the-tabular-environment
\makeatletter
\let\insertdate\@date
\makeatother

\titleformat{\chapter}[display]
{\bfseries\Large}
{\color{DarkSlateGrey}\filleft \authorname
\ifthenelse{\isundefined{\studentnumber}}{}{\\ \studentnumber}
\ifthenelse{\isundefined{\email}}{}{\\ \email}
\ifthenelse{\isundefined{\dateintitle}}{}{\\ \insertdate}
%\ifthenelse{\isundefined{\coursename}}{}{\\ \coursename} % put in title instead.
}
{4ex}
{\color{DarkOliveGreen}{\titlerule}\color{Maroon}
\vspace{2ex}%
\filright}
[\vspace{2ex}%
\color{DarkOliveGreen}\titlerule
]

\newcommand{\beginArtWithToc}[0]{\begin{document}\tableofcontents}
\newcommand{\beginArtNoToc}[0]{\begin{document}}
\newcommand{\EndNoBibArticle}[0]{\end{document}}
\newcommand{\EndArticle}[0]{\bibliography{Bibliography}\bibliographystyle{plainnat}\end{document}}

% 
%\newcommand{\citep}[1]{\cite{#1}}

\colorSectionsForArticle


%
%\usepackage{peeters_layout_exercise}
%\usepackage{peeters_braket}
%\usepackage{peeters_figures}
%
%\beginArtNoToc

%\generatetitle{Lagrangian for magnetic portion of Lorentz force}
%\label{chap:magneticLorentzForceLagrangian}

In \citep{sakurai2014modern} it is claimed in an Aharonov-Bohm discussion that a Lagrangian modification to include electromagnetism is

\begin{dmath}\label{eqn:magneticLorentzForceLagrangian:20}
\LL \rightarrow \LL + \frac{e}{c} \Bv \cdot \BA.
\end{dmath}

That can't be the full Lagrangian since there is no \( \phi \) term, so what exactly do we get?

If you have somehow, like I did, forgot the exact form of the Euler-Lagrange equations (i.e. where do the dots go), then the derivation of those equations can come to your rescue.  The starting point is the action

\begin{dmath}\label{eqn:magneticLorentzForceLagrangian:40}
S = \int \LL(x, \xdot, t) dt,
\end{dmath}

where the end points of the integral are fixed, and we assume we have no variation at the end points.  The variational calculation is

\begin{dmath}\label{eqn:magneticLorentzForceLagrangian:60}
\delta S
= \int \delta \LL(x, \xdot, t) dt
= \int \lr{ \PD{x}{\LL} \delta x + \PD{\xdot}{\LL} \delta \xdot } dt
= \int \lr{ \PD{x}{\LL} \delta x + \PD{\xdot}{\LL} \delta \ddt{x} } dt
= \int \lr{ \PD{x}{\LL} - \ddt{}\lr{\PD{\xdot}{\LL}} } \delta x dt
+ \delta x \PD{\xdot}{\LL}.
\end{dmath}

The boundary term is killed after evaluation at the end points where the variation is zero.  For the result to hold for all variations \( \delta x \), we must have

%\begin{dmath}\label{eqn:magneticLorentzForceLagrangian:80}
\boxedEquation{eqn:magneticLorentzForceLagrangian:80}{
\PD{x}{\LL} = \ddt{}\lr{\PD{\xdot}{\LL}}.
}
%\end{dmath}

Now lets apply this to the Lagrangian at hand.  For the position derivative we have

\begin{dmath}\label{eqn:magneticLorentzForceLagrangian:100}
\PD{x_i}{\LL}
=
\frac{e}{c} v_j \PD{x_i}{A_j}.
\end{dmath}

For the canonical momentum term, assuming \( \BA = \BA(\Bx) \) we have

\begin{dmath}\label{eqn:magneticLorentzForceLagrangian:120}
\ddt{} \PD{\xdot_i}{\LL}
=
\ddt{}
\lr{ m \xdot_i
+
\frac{e}{c} A_i
}
=
m \ddot{x}_i
+
\frac{e}{c}
\ddt{A_i}
=
m \ddot{x}_i
+
\frac{e}{c}
\PD{x_j}{A_i} \frac{dx_j}{dt}.
\end{dmath}

Assembling the results, we've got

\begin{dmath}\label{eqn:magneticLorentzForceLagrangian:140}
0
=
\ddt{} \PD{\xdot_i}{\LL}
-
\PD{x_i}{\LL}
=
m \ddot{x}_i
+
\frac{e}{c}
\PD{x_j}{A_i} \frac{dx_j}{dt}
-
\frac{e}{c} v_j \PD{x_i}{A_j},
\end{dmath}

or
\begin{dmath}\label{eqn:magneticLorentzForceLagrangian:160}
m \ddot{x}_i
=
\frac{e}{c} v_j \PD{x_i}{A_j}
-
\frac{e}{c}
\PD{x_j}{A_i} v_j
=
\frac{e}{c} v_j
\lr{
\PD{x_i}{A_j}
-
\PD{x_j}{A_i}
}
=
\frac{e}{c} v_j B_k \epsilon_{i j k}.
\end{dmath}

In vector form that is

%\begin{dmath}\label{eqn:magneticLorentzForceLagrangian:180}
\boxedEquation{eqn:magneticLorentzForceLagrangian:180}{
m \ddot{\Bx}
=
\frac{e}{c} \Bv \cross \BB.
}
%\end{dmath}

So, we get the magnetic term of the Lorentz force.  Also note that this shows the Lagrangian (and the end result), was not in SI units.  The \( 1/c \) term would have to be dropped for SI.

%\EndArticle

      \section{Problems}
         %
% Copyright © 2015 Peeter Joot.  All Rights Reserved.
% Licenced as described in the file LICENSE under the root directory of this GIT repository.
%
\makeproblem{Lorentz force from classical electrodynamic Hamiltonian.}{problem:qmLecture5:1}{
\index{classical Hamiltonian!Lorentz force}

Given the classical Hamiltonian

\begin{dmath}\label{eqn:qmLecture5:381}
H = \inv{2 m} \lr{ \Bp - q \BA }^2 + q \phi.
\end{dmath}

apply the Hamiltonian equations of motion

\begin{equation}\label{eqn:qmLecture5:480}
\begin{aligned}
\ddt{\Bp} &= - \PD{\Bq}{H} \\
\ddt{\Bq} &= \PD{\Bp}{H},
\end{aligned}
\end{equation}

to show that this is the Hamiltonian that describes the Lorentz force equation, and to find the velocity in terms of the canonical momentum and vector potential.
} % problem

\makeanswer{problem:qmLecture5:1}{

The particle velocity follows easily

\begin{dmath}\label{eqn:qmLecture5:500}
\Bv
= \ddt{\Br}
= \PD{\Bp}{H}
= \inv{m} \lr{ \Bp - q \BA }.
\end{dmath}

For the Lorentz force we can proceed in the coordinate representation

\begin{dmath}\label{eqn:qmLecture5:520}
\ddt{p_k}
= - \PD{x_k}{H}
= - \frac{2}{2m} \lr{ p_m - q A_m } \PD{x_k}{}\lr{ p_m - q A_m } - q \PD{x_k}{\phi}
= q v_m \PD{x_k}{A_m} - q \PD{x_k}{\phi},
\end{dmath}

We also have

\begin{dmath}\label{eqn:qmLecture5:540}
\ddt{p_k}
=
\ddt{} \lr{m x_k + q A_k }
=
m \frac{d^2 x_k}{dt^2} + q \PD{x_m}{A_k} \frac{d x_m}{dt} + q \PD{t}{A_k}.
\end{dmath}

Putting these together we've got

\begin{dmath}\label{eqn:qmLecture5:560}
m \frac{d^2 x_k}{dt^2}
= q v_m \PD{x_k}{A_m} - q \PD{x_k}{\phi},
- q \PD{x_m}{A_k} \frac{d x_m}{dt} - q \PD{t}{A_k}
=
q v_m \lr{ \PD{x_k}{A_m} - \PD{x_m}{A_k} } + q E_k
=
q v_m \epsilon_{k m s} B_s + q E_k,
\end{dmath}

or

\begin{dmath}\label{eqn:qmLecture5:580}
m \frac{d^2 \Bx}{dt^2}
=
q \Be_k v_m \epsilon_{k m s} B_s + q E_k
= q \Bv \cross \BB + q \BE.
\end{dmath}

} % answer

\makeproblem{Show gauge invariance of the magnetic and electric fields.}{problem:qmLecture5:2}{
\index{gauge invariance}

After the gauge transformation of \cref{eqn:qmLecture5:420a} show that the electric and magnetic fields are unaltered.
} % problem

\makeanswer{problem:qmLecture5:2}{

For the magnetic field the transformed field is

\begin{dmath}\label{eqn:qmLecture5:600}
\BB'
= \spacegrad \cross \lr{ \BA + \spacegrad \chi }
= \spacegrad \cross \BA + \spacegrad \cross \lr{ \spacegrad \chi }
= \spacegrad \cross \BA
= \BB.
\end{dmath}

\begin{dmath}\label{eqn:qmLecture5:620}
\BE'
=
- \PD{t}{\BA'} - \spacegrad \phi'
=
- \PD{t}{}\lr{\BA + \spacegrad \chi} - \spacegrad \lr{ \phi - \PD{t}{\chi}}
=
- \PD{t}{\BA} - \spacegrad \phi
=
\BE.
\end{dmath}

} % answer

         %
% Copyright © 2015 Peeter Joot.  All Rights Reserved.
% Licenced as described in the file LICENSE under the root directory of this GIT repository.
%
\makeproblem{Gauge transformation.}{problem:qmLecture6:1}{
\index{gauge transformation}

Show that after a transformation of position and momentum of the following form

\begin{dmath}\label{eqn:qmLecture6:600}
\begin{aligned}
\rcap' &= \rcap  \\
\pcap' &= \pcap  - q \spacegrad \chi(\Br)
\end{aligned}
\end{dmath}

all the commutators have the expected values.
} % problem

\makeanswer{problem:qmLecture6:1}{

The position commutators don't need consideration.  Of interest is the momentum-position commutators

\begin{dmath}\label{eqn:qmLecture6:620}
\antisymmetric{\hatp_k'}{\hatx_k'}
=
\antisymmetric{\hatp_k - q \partial_k \chi}{\hatx_k}
=
\antisymmetric{\hatp_k}{\hatx_k} - q \antisymmetric{\partial_k \chi}{\hatx_k}
=
\antisymmetric{\hatp_k}{\hatx_k},
\end{dmath}

and the momentum commutators

\begin{dmath}\label{eqn:qmLecture6:640}
\antisymmetric{\hatp_k'}{\hatp_j'}
=
\antisymmetric{\hatp_k - q \partial_k \chi}{\hatp_j - q \partial_j \chi}
=
\antisymmetric{\hatp_k}{\hatp_j}
- q \lr{ \antisymmetric{\partial_k \chi}{\hatp_j} + \antisymmetric{\hatp_k}{\partial_j \chi} }.
\end{dmath}

That last sum of commutators is

\begin{dmath}\label{eqn:qmLecture6:660}
\antisymmetric{\partial_k \chi}{\hatp_j} + \antisymmetric{\hatp_k}{\partial_j \chi}
=
- i \Hbar \lr{ \PD{k}{(\partial_j \chi)} -  \PD{j}{(\partial_k \chi)} }
= 0.
\end{dmath}

We've shown that

\begin{dmath}\label{eqn:qmLecture6:680}
\begin{aligned}
\antisymmetric{\hatp_k'}{\hatx_k'} &= \antisymmetric{\hatp_k}{\hatx_k} \\
\antisymmetric{\hatp_k'}{\hatp_j'} &= \antisymmetric{\hatp_k}{\hatp_j}.
\end{aligned}
\end{dmath}

All the other commutators clearly have the desired transformation properties.
} % answer

         % p1
         %
% Copyright � 2015 Peeter Joot.  All Rights Reserved.
% Licenced as described in the file LICENSE under the root directory of this GIT repository.
%
%\newcommand{\authorname}{Peeter Joot}
\newcommand{\email}{peeterjoot@protonmail.com}
\newcommand{\basename}{FIXMEbasenameUndefined}
\newcommand{\dirname}{notes/FIXMEdirnameUndefined/}

%\renewcommand{\basename}{heisenbergSpinPrecession}
%\renewcommand{\dirname}{notes/phy1520/}
%%\newcommand{\dateintitle}{}
%%\newcommand{\keywords}{}
%
%\newcommand{\authorname}{Peeter Joot}
\newcommand{\onlineurl}{http://sites.google.com/site/peeterjoot2/math2013/\basename.pdf}
\newcommand{\sourcepath}{\dirname\basename.tex}
\newcommand{\generatetitle}[1]{\chapter{#1}}

\newcommand{\vcsinfo}{%
\section*{}
\noindent{\color{DarkOliveGreen}{\rule{\linewidth}{0.1mm}}}
\paragraph{Document version}
%\paragraph{\color{Maroon}{Document version}}
{
\small
\begin{itemize}
\item Available online at:\\ 
\href{\onlineurl}{\onlineurl}
\item Git Repository: \input{./.revinfo/gitRepo.tex}
\item Source: \sourcepath
\item last commit: \input{./.revinfo/gitCommitString.tex}
\item commit date: \input{./.revinfo/gitCommitDate.tex}
\end{itemize}
}
}

%\PassOptionsToPackage{dvipsnames,svgnames}{xcolor}
\PassOptionsToPackage{square,numbers}{natbib}
\documentclass{scrreprt}

\usepackage[left=2cm,right=2cm]{geometry}
\usepackage[svgnames]{xcolor}
\usepackage{peeters_layout}

\usepackage{natbib}

\usepackage[
colorlinks=true,
bookmarks=false,
pdfauthor={\authorname, \email},
backref 
]{hyperref}

% http://tex.stackexchange.com/questions/75773/how-to-reference-problems-by-the-text-label-in-an-exercise-envioronment
\usepackage[english]{cleveref}
\crefname{Exercise}{exercise}{exercises}
\Crefname{Exercise}{Exercise}{Exercises}

\RequirePackage{titlesec}
\RequirePackage{ifthen}

% http://stackoverflow.com/questions/4932910/date-in-the-tabular-environment
\makeatletter
\let\insertdate\@date
\makeatother

\titleformat{\chapter}[display]
{\bfseries\Large}
{\color{DarkSlateGrey}\filleft \authorname
\ifthenelse{\isundefined{\studentnumber}}{}{\\ \studentnumber}
\ifthenelse{\isundefined{\email}}{}{\\ \email}
\ifthenelse{\isundefined{\dateintitle}}{}{\\ \insertdate}
%\ifthenelse{\isundefined{\coursename}}{}{\\ \coursename} % put in title instead.
}
{4ex}
{\color{DarkOliveGreen}{\titlerule}\color{Maroon}
\vspace{2ex}%
\filright}
[\vspace{2ex}%
\color{DarkOliveGreen}\titlerule
]

\newcommand{\beginArtWithToc}[0]{\begin{document}\tableofcontents}
\newcommand{\beginArtNoToc}[0]{\begin{document}}
\newcommand{\EndNoBibArticle}[0]{\end{document}}
\newcommand{\EndArticle}[0]{\bibliography{Bibliography}\bibliographystyle{plainnat}\end{document}}

% 
%\newcommand{\citep}[1]{\cite{#1}}

\colorSectionsForArticle


%
%\usepackage{peeters_layout_exercise}
%\usepackage{peeters_braket}
%\usepackage{peeters_figures}
%
%\beginArtNoToc
%
%\generatetitle{Heisenberg picture spin precession}
%\chapter{Heisenberg picture spin precession}
%\label{chap:heisenbergSpinPrecession}

\makeoproblem{Heisenberg picture spin precession.}{problem:heisenbergSpinPrecession:2.1}{\citep{sakurai2014modern} pr. 2.1}{
For the spin Hamiltonian
\index{spin precession}

\begin{dmath}\label{eqn:heisenbergSpinPrecession:20}
H = -\frac{e B}{m c} S_z = \omega S_z,
\end{dmath}

express and solve the Heisenberg equations of motion for \( S_x(t), S_y(t) \), and \( S_z(t) \).
} % problem

\makeanswer{problem:heisenbergSpinPrecession:2.1}{

The equations of motion are of the form

\begin{dmath}\label{eqn:heisenbergSpinPrecession:40}
\frac{dS_i^\txtH}{dt}
= \inv{i \Hbar} \antisymmetric{S_i^\txtH}{H}
= \inv{i \Hbar} \antisymmetric{U^\dagger S_i U}{H}
= \inv{i \Hbar} \lr{U^\dagger S_i U H - H U^\dagger S_i U }
= \inv{i \Hbar} U^\dagger \lr{ S_i H - H S_i } U
= \frac{\omega}{i \Hbar} U^\dagger \antisymmetric{ S_i}{S_z } U.
\end{dmath}

These are

\begin{dmath}\label{eqn:heisenbergSpinPrecession:60}
\begin{aligned}
\frac{dS_x^\txtH}{dt} &= -\omega U^\dagger S_y U \\
\frac{dS_y^\txtH}{dt} &= \omega U^\dagger S_x U \\
\frac{dS_z^\txtH}{dt} &= 0.
\end{aligned}
\end{dmath}

To completely specify these equations, we need to expand \( U(t) \), which is

\begin{dmath}\label{eqn:heisenbergSpinPrecession:80}
U(t)
= e^{-i H t /\Hbar}
= e^{-i \omega S_z t /\Hbar}
= e^{-i \omega \sigma_z t /2}
= \cos\lr{ \omega t/2 } -i \sigma_z \sin\lr{ \omega t/2 }
=
\begin{bmatrix}
\cos\lr{ \omega t/2 } -i \sin\lr{ \omega t/2 } & 0 \\
0 & \cos\lr{ \omega t/2 } + i \sin\lr{ \omega t/2 }
\end{bmatrix}
=
\begin{bmatrix}
e^{-i\omega t/2} & 0 \\
0 & e^{i\omega t/2}
\end{bmatrix}.
\end{dmath}

The equations of motion can now be written out in full.  To do so seems a bit silly since we also know that \( S_x^\txtH = U^\dagger S_x U, S_y^\txtH U^\dagger S_x U \).  However, if that is temporarily forgotten, we can show that the Heisenberg equations of motion can be solved for these too.

\begin{dmath}\label{eqn:heisenbergSpinPrecession:100}
U^\dagger S_x U
=
\frac{\Hbar}{2}
\begin{bmatrix}
e^{i\omega t/2} & 0 \\
0 & e^{-i\omega t/2}
\end{bmatrix}
\PauliX
\begin{bmatrix}
e^{-i\omega t/2} & 0 \\
0 & e^{i\omega t/2}
\end{bmatrix}
=
\frac{\Hbar}{2}
\begin{bmatrix}
0 & e^{i\omega t/2} \\
e^{-i\omega t/2} & 0
\end{bmatrix}
\begin{bmatrix}
e^{-i\omega t/2} & 0 \\
0 & e^{i\omega t/2}
\end{bmatrix}
=
\frac{\Hbar}{2}
\begin{bmatrix}
0 & e^{i\omega t} \\
e^{-i\omega t} & 0
\end{bmatrix},
\end{dmath}

and
\begin{dmath}\label{eqn:heisenbergSpinPrecession:120}
U^\dagger S_y U
=
\frac{\Hbar}{2}
\begin{bmatrix}
e^{i\omega t/2} & 0 \\
0 & e^{-i\omega t/2}
\end{bmatrix}
\PauliY
\begin{bmatrix}
e^{-i\omega t/2} & 0 \\
0 & e^{i\omega t/2}
\end{bmatrix}
=
\frac{i\Hbar}{2}
\begin{bmatrix}
0 & -e^{i\omega t/2} \\
e^{-i\omega t/2} & 0
\end{bmatrix}
\begin{bmatrix}
e^{-i\omega t/2} & 0 \\
0 & e^{i\omega t/2}
\end{bmatrix}
=
\frac{i \Hbar}{2}
\begin{bmatrix}
0 & -e^{i\omega t} \\
e^{-i\omega t} & 0
\end{bmatrix}.
\end{dmath}

The equations of motion are now fully specified

\begin{dmath}\label{eqn:heisenbergSpinPrecession:140}
\begin{aligned}
\frac{dS_x^\txtH}{dt} &=
-\frac{i \Hbar \omega}{2}
\begin{bmatrix}
0 & -e^{i\omega t} \\
e^{-i\omega t} & 0
\end{bmatrix} \\
\frac{dS_y^\txtH}{dt} &=
\frac{\Hbar \omega}{2}
\begin{bmatrix}
0 & e^{i\omega t} \\
e^{-i\omega t} & 0
\end{bmatrix} \\
\frac{dS_z^\txtH}{dt} &= 0.
\end{aligned}
\end{dmath}

Integration gives

\begin{dmath}\label{eqn:heisenbergSpinPrecession:160}
\begin{aligned}
S_x^\txtH &=
\frac{\Hbar}{2}
\begin{bmatrix}
0 & e^{i\omega t} \\
e^{-i\omega t} & 0
\end{bmatrix} + C \\
S_y^\txtH &=
\frac{\Hbar}{2}
\begin{bmatrix}
0 & -i e^{i\omega t} \\
i e^{-i\omega t} & 0
\end{bmatrix} + C \\
S_z^\txtH &= C.
\end{aligned}
\end{dmath}

The integration constants are fixed by the boundary condition \( S_i^\txtH(0) = S_i \), so

\begin{dmath}\label{eqn:heisenbergSpinPrecession:180}
\begin{aligned}
S_x^\txtH &=
\frac{\Hbar}{2}
\begin{bmatrix}
0 & e^{i\omega t} \\
e^{-i\omega t} & 0
\end{bmatrix} \\
S_y^\txtH &=
\frac{i \Hbar}{2}
\begin{bmatrix}
0 & - e^{i\omega t} \\
 e^{-i\omega t} & 0
\end{bmatrix} \\
S_z^\txtH &= S_z.
\end{aligned}
\end{dmath}

Observe that these integrated values \( S_x^\txtH, S_y^\txtH \) match \cref{eqn:heisenbergSpinPrecession:100}, and \cref{eqn:heisenbergSpinPrecession:120} as expected.

} % answer

%\EndArticle

         % p2
         %
% Copyright � 2015 Peeter Joot.  All Rights Reserved.
% Licenced as described in the file LICENSE under the root directory of this GIT repository.
%
%\newcommand{\authorname}{Peeter Joot}
\newcommand{\email}{peeterjoot@protonmail.com}
\newcommand{\basename}{FIXMEbasenameUndefined}
\newcommand{\dirname}{notes/FIXMEdirnameUndefined/}

%\renewcommand{\basename}{dynamicsNonHermitian}
%\renewcommand{\dirname}{notes/phy1520/}
%\newcommand{\dateintitle}{}
%\newcommand{\keywords}{}

%\newcommand{\authorname}{Peeter Joot}
\newcommand{\onlineurl}{http://sites.google.com/site/peeterjoot2/math2013/\basename.pdf}
\newcommand{\sourcepath}{\dirname\basename.tex}
\newcommand{\generatetitle}[1]{\chapter{#1}}

\newcommand{\vcsinfo}{%
\section*{}
\noindent{\color{DarkOliveGreen}{\rule{\linewidth}{0.1mm}}}
\paragraph{Document version}
%\paragraph{\color{Maroon}{Document version}}
{
\small
\begin{itemize}
\item Available online at:\\ 
\href{\onlineurl}{\onlineurl}
\item Git Repository: \input{./.revinfo/gitRepo.tex}
\item Source: \sourcepath
\item last commit: \input{./.revinfo/gitCommitString.tex}
\item commit date: \input{./.revinfo/gitCommitDate.tex}
\end{itemize}
}
}

%\PassOptionsToPackage{dvipsnames,svgnames}{xcolor}
\PassOptionsToPackage{square,numbers}{natbib}
\documentclass{scrreprt}

\usepackage[left=2cm,right=2cm]{geometry}
\usepackage[svgnames]{xcolor}
\usepackage{peeters_layout}

\usepackage{natbib}

\usepackage[
colorlinks=true,
bookmarks=false,
pdfauthor={\authorname, \email},
backref 
]{hyperref}

% http://tex.stackexchange.com/questions/75773/how-to-reference-problems-by-the-text-label-in-an-exercise-envioronment
\usepackage[english]{cleveref}
\crefname{Exercise}{exercise}{exercises}
\Crefname{Exercise}{Exercise}{Exercises}

\RequirePackage{titlesec}
\RequirePackage{ifthen}

% http://stackoverflow.com/questions/4932910/date-in-the-tabular-environment
\makeatletter
\let\insertdate\@date
\makeatother

\titleformat{\chapter}[display]
{\bfseries\Large}
{\color{DarkSlateGrey}\filleft \authorname
\ifthenelse{\isundefined{\studentnumber}}{}{\\ \studentnumber}
\ifthenelse{\isundefined{\email}}{}{\\ \email}
\ifthenelse{\isundefined{\dateintitle}}{}{\\ \insertdate}
%\ifthenelse{\isundefined{\coursename}}{}{\\ \coursename} % put in title instead.
}
{4ex}
{\color{DarkOliveGreen}{\titlerule}\color{Maroon}
\vspace{2ex}%
\filright}
[\vspace{2ex}%
\color{DarkOliveGreen}\titlerule
]

\newcommand{\beginArtWithToc}[0]{\begin{document}\tableofcontents}
\newcommand{\beginArtNoToc}[0]{\begin{document}}
\newcommand{\EndNoBibArticle}[0]{\end{document}}
\newcommand{\EndArticle}[0]{\bibliography{Bibliography}\bibliographystyle{plainnat}\end{document}}

% 
%\newcommand{\citep}[1]{\cite{#1}}

\colorSectionsForArticle


%
%\usepackage{peeters_layout_exercise}
%\usepackage{peeters_braket}
%\usepackage{peeters_figures}
%\usepackage{peeters_qed}
%
%\beginArtNoToc

%\generatetitle{Dynamics of non-Hermitian Hamiltonian}
%\chapter{Dynamics of non-Hermitian Hamiltonian}
%\label{chap:dynamicsNonHermitian}

\makeoproblem{Dynamics of non-Hermitian Hamiltonian.}{problem:dynamicsNonHermitian:2.2}{\citep{sakurai2014modern} pr. 2.2}{
\index{Hamiltonian!non-Hermitian}
Revisiting an earlier Hamiltonian, but assuming it was entered incorrectly as

\begin{dmath}\label{eqn:dynamicsNonHermitian:20}
H = H_{11} \ket{1}\bra{1}
  + H_{22} \ket{2}\bra{2}
  + H_{12} \ket{1}\bra{2}.
\end{dmath}

What principle is now violated?  Illustrate your point explicitly by attempting to solve the most general time-dependent problem using an illegal Hamiltonian of this kind.  You may assume that \( H_{11} = H_{22} \) for simplicity.
} % problem

\makeanswer{problem:dynamicsNonHermitian:2.2}{

In matrix form this Hamiltonian is

\begin{dmath}\label{eqn:dynamicsNonHermitian:40}
H =
\begin{bmatrix}
\bra{1} H \ket{1} & \bra{1} H \ket{2} \\
\bra{2} H \ket{1} & \bra{2} H \ket{2} \\
\end{bmatrix}
=
\begin{bmatrix}
H_{11} & H_{12} \\
0      & H_{22} \\
\end{bmatrix}.
\end{dmath}

This is not a Hermitian operator.  What is the physical implication of this non-Hermicity?  Consider the simpler case where \( H_{11} = H_{22} \).  Such a Hamiltonian has the form

\begin{dmath}\label{eqn:dynamicsNonHermitian:60}
H =
\begin{bmatrix}
a & b \\
0 & a
\end{bmatrix}.
\end{dmath}

This has only one unique eigenvector ( \( (1,0) \), but we can still solve the time evolution equation

\begin{dmath}\label{eqn:dynamicsNonHermitian:80}
i \Hbar \PD{t}{U} = H U,
\end{dmath}

since for constant \( H \), we have

\begin{dmath}\label{eqn:dynamicsNonHermitian:100}
U = e^{-i H t/\Hbar}.
\end{dmath}

To exponentiate, note that we have

\begin{dmath}\label{eqn:dynamicsNonHermitian:120}
{\begin{bmatrix}
a & b \\
0 & a
\end{bmatrix}}^n
=
\begin{bmatrix}
a^n & n a^{n-1} b \\
0 & a^n
\end{bmatrix}.
\end{dmath}

To prove the induction, the \( n = 2 \) case follows easily

\begin{dmath}\label{eqn:dynamicsNonHermitian:140}
\begin{bmatrix}
a & b \\
0 & a
\end{bmatrix}
\begin{bmatrix}
a & b \\
0 & a
\end{bmatrix}
=
\begin{bmatrix}
a^2 & 2 a b \\
0 & a^2
\end{bmatrix},
\end{dmath}

as does the general case

\begin{dmath}\label{eqn:dynamicsNonHermitian:160}
\begin{bmatrix}
a^n & n a^{n-1} b \\
0 & a^n
\end{bmatrix}
\begin{bmatrix}
a & b \\
0 & a
\end{bmatrix}
=
\begin{bmatrix}
a^{n+1} & (n +1 ) a^{n} b \\
0 & a^{n+1}
\end{bmatrix}.
\end{dmath}

The exponential sum is thus
\begin{dmath}\label{eqn:dynamicsNonHermitian:180}
e^{H \tau}
=
\begin{bmatrix}
e^{a \tau} & 0 + \frac{b \tau}{1!} + \frac{2 a b \tau^2}{2!} + \frac{3 a^2 b \tau^3}{3!} + \cdots \\
0 & e^{a \tau}
\end{bmatrix}.
\end{dmath}

That sum simplifies to

\begin{dmath}\label{eqn:dynamicsNonHermitian:200}
\frac{b \tau}{0!} + \frac{a b \tau^2}{1!} + \frac{a^2 b \tau^3}{2!} + \cdots \\
=
b \tau \lr{ 1 + \frac{a \tau}{1!} + \frac{(a \tau)^2}{2!} + \cdots }
=
b \tau e^{a \tau}.
\end{dmath}

The exponential is thus
\begin{dmath}\label{eqn:dynamicsNonHermitian:220}
e^{H \tau} =
\begin{bmatrix}
e^{a\tau} & b \tau e^{a\tau} \\
0 & e^{a\tau}
\end{bmatrix}
=
\begin{bmatrix}
1 & b \tau \\
0 & 1
\end{bmatrix}
e^{a\tau}.
\end{dmath}

In particular

\begin{dmath}\label{eqn:dynamicsNonHermitian:240}
U = e^{-i H t/\Hbar} =
\begin{bmatrix}
1 & -i b t/\Hbar \\
0 & 1
\end{bmatrix}
e^{-i a t /\Hbar }.
\end{dmath}

We can verify that this is a solution to \cref{eqn:dynamicsNonHermitian:80}.  The left hand side is

\begin{dmath}\label{eqn:dynamicsNonHermitian:260}
i \Hbar \PD{t}{U}
=
i \Hbar
\begin{bmatrix}
-i a/\Hbar & -i b /\Hbar + (-i b t/\Hbar)(-i a/\Hbar) \\
0 & -i a /\Hbar
\end{bmatrix}
e^{-i a t /\Hbar }
=
\begin{bmatrix}
a & b - i a b t/\Hbar \\
0 & a
\end{bmatrix}
e^{-i a t /\Hbar },
\end{dmath}

and for the right hand side
\begin{dmath}\label{eqn:dynamicsNonHermitian:280}
H U
=
\begin{bmatrix}
a & b \\
0 & a
\end{bmatrix}
\begin{bmatrix}
1 & -i b t/\Hbar \\
0 & 1
\end{bmatrix}
e^{-i a t /\Hbar }
=
\begin{bmatrix}
a & b - i a b t/\Hbar \\
0 & a
\end{bmatrix}
e^{-i a t /\Hbar }
=
i \Hbar \PD{t}{U}. \qedmarker
\end{dmath}

While the Schr\"{o}dinger is satisfied, we don't have the unitary inversion physical property that is desired for the time evolution operator \( U \).  Namely

\begin{dmath}\label{eqn:dynamicsNonHermitian:300}
U^\dagger U
=
\begin{bmatrix}
1 & 0 \\
i b t/\Hbar & 1
\end{bmatrix}
e^{i a t /\Hbar }
\begin{bmatrix}
1 & -i b t/\Hbar \\
0 & 1
\end{bmatrix}
e^{-i a t /\Hbar }
=
\begin{bmatrix}
1 & -i b t/\Hbar \\
i b t/\Hbar & (b t)^2/\Hbar^2
\end{bmatrix}
\ne I.
\end{dmath}

We required \( U^\dagger U = I \) for the time evolution operator, but don't have that property for this non-Hermitian Hamiltonian.
} % answer

%\EndArticle

         % p2.3
         %
% Copyright � 2015 Peeter Joot.  All Rights Reserved.
% Licenced as described in the file LICENSE under the root directory of this GIT repository.
%
%\newcommand{\authorname}{Peeter Joot}
\newcommand{\email}{peeterjoot@protonmail.com}
\newcommand{\basename}{FIXMEbasenameUndefined}
\newcommand{\dirname}{notes/FIXMEdirnameUndefined/}

%\renewcommand{\basename}{spinTimeEvolution}
%\renewcommand{\dirname}{notes/phy1520/}
%%\newcommand{\dateintitle}{}
%%\newcommand{\keywords}{}
%
%\newcommand{\authorname}{Peeter Joot}
\newcommand{\onlineurl}{http://sites.google.com/site/peeterjoot2/math2013/\basename.pdf}
\newcommand{\sourcepath}{\dirname\basename.tex}
\newcommand{\generatetitle}[1]{\chapter{#1}}

\newcommand{\vcsinfo}{%
\section*{}
\noindent{\color{DarkOliveGreen}{\rule{\linewidth}{0.1mm}}}
\paragraph{Document version}
%\paragraph{\color{Maroon}{Document version}}
{
\small
\begin{itemize}
\item Available online at:\\ 
\href{\onlineurl}{\onlineurl}
\item Git Repository: \input{./.revinfo/gitRepo.tex}
\item Source: \sourcepath
\item last commit: \input{./.revinfo/gitCommitString.tex}
\item commit date: \input{./.revinfo/gitCommitDate.tex}
\end{itemize}
}
}

%\PassOptionsToPackage{dvipsnames,svgnames}{xcolor}
\PassOptionsToPackage{square,numbers}{natbib}
\documentclass{scrreprt}

\usepackage[left=2cm,right=2cm]{geometry}
\usepackage[svgnames]{xcolor}
\usepackage{peeters_layout}

\usepackage{natbib}

\usepackage[
colorlinks=true,
bookmarks=false,
pdfauthor={\authorname, \email},
backref 
]{hyperref}

% http://tex.stackexchange.com/questions/75773/how-to-reference-problems-by-the-text-label-in-an-exercise-envioronment
\usepackage[english]{cleveref}
\crefname{Exercise}{exercise}{exercises}
\Crefname{Exercise}{Exercise}{Exercises}

\RequirePackage{titlesec}
\RequirePackage{ifthen}

% http://stackoverflow.com/questions/4932910/date-in-the-tabular-environment
\makeatletter
\let\insertdate\@date
\makeatother

\titleformat{\chapter}[display]
{\bfseries\Large}
{\color{DarkSlateGrey}\filleft \authorname
\ifthenelse{\isundefined{\studentnumber}}{}{\\ \studentnumber}
\ifthenelse{\isundefined{\email}}{}{\\ \email}
\ifthenelse{\isundefined{\dateintitle}}{}{\\ \insertdate}
%\ifthenelse{\isundefined{\coursename}}{}{\\ \coursename} % put in title instead.
}
{4ex}
{\color{DarkOliveGreen}{\titlerule}\color{Maroon}
\vspace{2ex}%
\filright}
[\vspace{2ex}%
\color{DarkOliveGreen}\titlerule
]

\newcommand{\beginArtWithToc}[0]{\begin{document}\tableofcontents}
\newcommand{\beginArtNoToc}[0]{\begin{document}}
\newcommand{\EndNoBibArticle}[0]{\end{document}}
\newcommand{\EndArticle}[0]{\bibliography{Bibliography}\bibliographystyle{plainnat}\end{document}}

% 
%\newcommand{\citep}[1]{\cite{#1}}

\colorSectionsForArticle


%
%\usepackage{peeters_layout_exercise}
%\usepackage{peeters_braket}
%\usepackage{peeters_figures}
%
%\beginArtNoToc
%
%\generatetitle{Time evolution of spin half probability and dispersion}
%%\label{chap:spinTimeEvolution}

\makeoproblem{Time evolution of spin half probability and dispersion.}{problem:spinTimeEvolution:3}{\citep{sakurai2014modern} pr. 2.3}{
\index{spin half!dispersion}
\index{dispersion}
A spin \( 1/2 \) system \( \BS \cdot \ncap \), with \( \ncap = \sin \beta \xcap + \cos\beta \zcap \), is in state with eigenvalue \( \Hbar/2 \), acted on by a magnetic field of strength \( B \) in the \( +z \) direction.

\makesubproblem{}{problem:spinTimeEvolution:3:a}

If \( S_x \) is measured at time \( t \), what is the probability of getting \( + \Hbar/2 \)?

\makesubproblem{}{problem:spinTimeEvolution:3:b}

Evaluate the dispersion in \( S_x \) as a function of t, that is,

\begin{equation}\label{eqn:spinTimeEvolution:20}
\expectation{\lr{ S_x - \expectation{S_x}}^2}.
\end{equation}

\makesubproblem{}{problem:spinTimeEvolution:3:c}

Check your answers for \( \beta \rightarrow 0, \pi/2 \) to see if they make sense.

} % problem

\makeanswer{problem:spinTimeEvolution:3}{

\makeSubAnswer{}{problem:spinTimeEvolution:3:a}

The spin operator in matrix form is
\begin{dmath}\label{eqn:spinTimeEvolution:40}
S \cdot \ncap
=
\frac{\Hbar}{2} \lr{ \sigma_z \cos\beta + \sigma_x \sin\beta }
=
\frac{\Hbar}{2} \lr{ \PauliZ \cos\beta + \PauliX \sin\beta }
=
\frac{\Hbar}{2}
\begin{bmatrix}
\cos\beta & \sin\beta \\
\sin\beta & -\cos\beta
\end{bmatrix}.
\end{dmath}

The \( \ket{S \cdot \ncap ; + } \) eigenstate is found from

\begin{dmath}\label{eqn:spinTimeEvolution:60}
\lr{ S \cdot \ncap - \Hbar/2}
\begin{bmatrix}
a \\
b
\end{bmatrix}
= 0,
\end{dmath}

or

\begin{dmath}\label{eqn:spinTimeEvolution:80}
0
=
\lr{ \cos\beta - 1 } a + \sin\beta b
=
\lr{ -2 \sin^2(\beta/2) } a + 2 \sin(\beta/2) \cos(\beta/2) b
=
\lr{ - \sin(\beta/2) } a + \cos(\beta/2) b,
\end{dmath}

or

\begin{dmath}\label{eqn:spinTimeEvolution:100}
\ket{ S \cdot \ncap ; + }
=
\begin{bmatrix}
\cos(\beta/2) \\
\sin(\beta/2) \\
\end{bmatrix}.
\end{dmath}

The Hamiltonian is

\begin{equation}\label{eqn:spinTimeEvolution:120}
H
= - \frac{e B}{m c} S_z
= - \frac{e B \Hbar}{2 m c} \sigma_z,
\end{equation}

so the time evolution operator is

\begin{dmath}\label{eqn:spinTimeEvolution:140}
U
= e^{-i H t/\Hbar}
= e^{ \frac{i e B t }{2 m c} \sigma_z }.
\end{dmath}

Let \( \omega = e B/(2 m c) \), so

\begin{dmath}\label{eqn:spinTimeEvolution:160}
U
=
e^{i \sigma_z \omega t}
=
\cos(\omega t) + i \sigma_z \sin(\omega t)
=
\begin{bmatrix}
1 & 0 \\
0 & 1
\end{bmatrix}
\cos(\omega t)
+
i  \PauliZ \sin(\omega t)
=
\begin{bmatrix}
e^{i \omega t} & 0 \\
0 & e^{-i \omega t}
\end{bmatrix}.
\end{dmath}

The time evolution of the initial state is

\begin{dmath}\label{eqn:spinTimeEvolution:180}
\ket{S \cdot \ncap ; + }(t)
=
U \ket{S \cdot \ncap ; + }(0)
=
\begin{bmatrix}
e^{i \omega t} & 0 \\
0 & e^{-i \omega t}
\end{bmatrix}
\begin{bmatrix}
\cos(\beta/2) \\
\sin(\beta/2) \\
\end{bmatrix}
=
\begin{bmatrix}
\cos(\beta/2) e^{i \omega t} \\
\sin(\beta/2) e^{-i \omega t} \\
\end{bmatrix}.
\end{dmath}

The probability of finding the state in \( \ket{S \cdot \xcap ; + } \) at time \( t \) (i.e. measuring \( S_x \) and finding \( \Hbar/2 \)) is

\begin{dmath}\label{eqn:spinTimeEvolution:200}
\begin{aligned}
\Abs{\braket{S \cdot \xcap ; + }{S \cdot \ncap ; + }}^2
&=
\Abs{\inv{\sqrt{2}}
\begin{bmatrix}
1 & 1 \\
\end{bmatrix}
\begin{bmatrix}
\cos(\beta/2) e^{i \omega t} \\
\sin(\beta/2) e^{-i \omega t} \\
\end{bmatrix}
}^2 \\
&=
\inv{2}
\Abs{
\cos(\beta/2) e^{i \omega t} +
\sin(\beta/2) e^{-i \omega t} }^2 \\
&=
\inv{2} \lr{ 1 + 2 \cos(\beta/2) \sin(\beta/2) \cos(2 \omega t) } \\
&=
\inv{2} \lr{ 1 + \sin(\beta) \cos( 2 \omega t) }.
\end{aligned}
\end{dmath}

\makeSubAnswer{}{problem:spinTimeEvolution:3:b}

To calculate the dispersion first note that

\begin{dmath}\label{eqn:spinTimeEvolution:300}
S_x^2
= \lr{ \frac{\Hbar}{2} }^2 \PauliX^2
= \lr{ \frac{\Hbar}{2} }^2,
\end{dmath}

so only the first order expectation is non-trivial to calculate.  That is

\begin{dmath}\label{eqn:spinTimeEvolution:320}
\expectation{S_x}
=
\frac{\Hbar}{2}
\begin{bmatrix}
\cos(\beta/2) e^{-i \omega t} &
\sin(\beta/2) e^{i \omega t}
\end{bmatrix}
\PauliX
\begin{bmatrix}
\cos(\beta/2) e^{i \omega t} \\
\sin(\beta/2) e^{-i \omega t} \\
\end{bmatrix}
=
\frac{\Hbar}{2}
\begin{bmatrix}
\cos(\beta/2) e^{-i \omega t} &
\sin(\beta/2) e^{i \omega t}
\end{bmatrix}
\begin{bmatrix}
\sin(\beta/2) e^{-i \omega t} \\
\cos(\beta/2) e^{i \omega t} \\
\end{bmatrix}
=
\frac{\Hbar}{2}
\sin(\beta/2) \cos(\beta/2) \lr{ e^{-2 i \omega t} + e^{ 2 i \omega t} }
=
\frac{\Hbar}{2} \sin\beta \cos( 2 \omega t ).
\end{dmath}

This gives

\boxedEquation{eqn:spinTimeEvolution:340}{
\expectation{(\Delta S_x)^2}
=
\lr{ \frac{\Hbar}{2} }^2 \lr{ 1 - \sin^2\beta \cos^2( 2 \omega t ) }
}

\makeSubAnswer{}{problem:spinTimeEvolution:3:c}

For \( \beta = 0 \), \( \ncap = \zcap \), and \( \beta = \pi/2 \), \( \ncap = \xcap \).  For the first case, the state is in an eigenstate of \( S_z \), so must evolve as

\begin{dmath}\label{eqn:spinTimeEvolution:220}
\ket{S \cdot \ncap ; + }(t) = \ket{S \cdot \ncap ; + }(0) e^{i \omega t}.
\end{dmath}

The probability of finding it in state \( \ket{S \cdot \xcap ; + } \) is therefore

\begin{dmath}\label{eqn:spinTimeEvolution:240}
\Abs{
\inv{\sqrt{2}}
\begin{bmatrix}
1 & 1
\end{bmatrix}
\begin{bmatrix}
e^{i \omega t} \\
0
\end{bmatrix}
}^2
=
\inv{2} \Abs{ e^{i\omega t} }^2
=
\inv{2}
=
\inv{2} \lr{ 1 + \sin(0) \cos(2 \omega t) }.
\end{dmath}

This matches \cref{eqn:spinTimeEvolution:200} as expected.

For \( \beta = \pi/2 \) we have

\begin{dmath}\label{eqn:spinTimeEvolution:260}
\ket{S \cdot \xcap ; + }(t) =
\inv{\sqrt{2}}
\begin{bmatrix}
e^{i \omega t} & 0 \\
0 & e^{-i \omega t}
\end{bmatrix}
\begin{bmatrix}
1 \\
1
\end{bmatrix}
=
\inv{\sqrt{2}}
\begin{bmatrix}
e^{i \omega t} \\
e^{-i \omega t}
\end{bmatrix}.
\end{dmath}

The probability for the \( \Hbar/2 \) \( S_x \) measurement at time \( t \) is
\begin{dmath}\label{eqn:spinTimeEvolution:280}
\Abs{
\inv{2}
\begin{bmatrix}
1 & 1
\end{bmatrix}
\begin{bmatrix}
e^{i \omega t} \\
e^{-i \omega t}
\end{bmatrix}
}^2
=
\inv{4} \Abs{ e^{i \omega t}  + e^{-i \omega t} }^2
=
\cos^2(\omega t)
=
\inv{2}\lr{ 1 + \sin(\pi/2) \cos( 2 \omega t )}.
\end{dmath}

Again, this matches the expected value.

For the dispersions, at \( \beta = 0 \), the dispersion is

\begin{dmath}\label{eqn:spinTimeEvolution:360}
\lr{\frac{\Hbar}{2}}^2
\end{dmath}

This is the maximum dispersion, which makes sense since we are measuring \( S_x \) when the initial state is \( \ket{S \cdot \zcap ; + } \).  For \( \beta = \pi/2 \) the dispersion is

\begin{dmath}\label{eqn:spinTimeEvolution:380}
\lr{\frac{\Hbar}{2}}^2 \sin^2 ( 2 \omega t ).
\end{dmath}

This starts off as zero dispersion (because the initial state is \( \ket{ S \cdot \xcap ; + } \), but then oscillates.

} % answer

%\EndArticle

         % p5:
         %
% Copyright � 2015 Peeter Joot.  All Rights Reserved.
% Licenced as described in the file LICENSE under the root directory of this GIT repository.
%
%\newcommand{\authorname}{Peeter Joot}
\newcommand{\email}{peeterjoot@protonmail.com}
\newcommand{\basename}{FIXMEbasenameUndefined}
\newcommand{\dirname}{notes/FIXMEdirnameUndefined/}

%\renewcommand{\basename}{positionCommutator}
%\renewcommand{\dirname}{notes/phy1520/}
%%\newcommand{\dateintitle}{}
%%\newcommand{\keywords}{}
%
%\newcommand{\authorname}{Peeter Joot}
\newcommand{\onlineurl}{http://sites.google.com/site/peeterjoot2/math2013/\basename.pdf}
\newcommand{\sourcepath}{\dirname\basename.tex}
\newcommand{\generatetitle}[1]{\chapter{#1}}

\newcommand{\vcsinfo}{%
\section*{}
\noindent{\color{DarkOliveGreen}{\rule{\linewidth}{0.1mm}}}
\paragraph{Document version}
%\paragraph{\color{Maroon}{Document version}}
{
\small
\begin{itemize}
\item Available online at:\\ 
\href{\onlineurl}{\onlineurl}
\item Git Repository: \input{./.revinfo/gitRepo.tex}
\item Source: \sourcepath
\item last commit: \input{./.revinfo/gitCommitString.tex}
\item commit date: \input{./.revinfo/gitCommitDate.tex}
\end{itemize}
}
}

%\PassOptionsToPackage{dvipsnames,svgnames}{xcolor}
\PassOptionsToPackage{square,numbers}{natbib}
\documentclass{scrreprt}

\usepackage[left=2cm,right=2cm]{geometry}
\usepackage[svgnames]{xcolor}
\usepackage{peeters_layout}

\usepackage{natbib}

\usepackage[
colorlinks=true,
bookmarks=false,
pdfauthor={\authorname, \email},
backref 
]{hyperref}

% http://tex.stackexchange.com/questions/75773/how-to-reference-problems-by-the-text-label-in-an-exercise-envioronment
\usepackage[english]{cleveref}
\crefname{Exercise}{exercise}{exercises}
\Crefname{Exercise}{Exercise}{Exercises}

\RequirePackage{titlesec}
\RequirePackage{ifthen}

% http://stackoverflow.com/questions/4932910/date-in-the-tabular-environment
\makeatletter
\let\insertdate\@date
\makeatother

\titleformat{\chapter}[display]
{\bfseries\Large}
{\color{DarkSlateGrey}\filleft \authorname
\ifthenelse{\isundefined{\studentnumber}}{}{\\ \studentnumber}
\ifthenelse{\isundefined{\email}}{}{\\ \email}
\ifthenelse{\isundefined{\dateintitle}}{}{\\ \insertdate}
%\ifthenelse{\isundefined{\coursename}}{}{\\ \coursename} % put in title instead.
}
{4ex}
{\color{DarkOliveGreen}{\titlerule}\color{Maroon}
\vspace{2ex}%
\filright}
[\vspace{2ex}%
\color{DarkOliveGreen}\titlerule
]

\newcommand{\beginArtWithToc}[0]{\begin{document}\tableofcontents}
\newcommand{\beginArtNoToc}[0]{\begin{document}}
\newcommand{\EndNoBibArticle}[0]{\end{document}}
\newcommand{\EndArticle}[0]{\bibliography{Bibliography}\bibliographystyle{plainnat}\end{document}}

% 
%\newcommand{\citep}[1]{\cite{#1}}

\colorSectionsForArticle


%
%\usepackage{peeters_layout_exercise}
%\usepackage{peeters_braket}
%\usepackage{peeters_figures}
%
%\beginArtNoToc
%
%\generatetitle{Heisenberg picture position commutator}
%\chapter{Heisenberg picture position commutator}
%\label{chap:positionCommutator}


\makeoproblem{Heisenberg picture position commutator.}{problem:positionCommutator:2.5}{\citep{sakurai2014modern} pr. 2.5}{
\index{Heisenberg picture}
Evaluate

\begin{dmath}\label{eqn:positionCommutator:20}
\antisymmetric{x(t)}{x(0)},
\end{dmath}

for a Heisenberg picture operator \( x(t) \) for a free particle.
} % problem

\makeanswer{problem:positionCommutator:2.5}{

The free particle Hamiltonian is

\begin{dmath}\label{eqn:positionCommutator:40}
H = \frac{p^2}{2m},
\end{dmath}

so the time evolution operator is

\begin{dmath}\label{eqn:positionCommutator:60}
U(t) = e^{-i p^2 t/(2 m \Hbar)}.
\end{dmath}

The Heisenberg picture position operator is

\begin{dmath}\label{eqn:positionCommutator:80}
x^\txtH
= U^\dagger x U
= e^{i p^2 t/(2 m \Hbar)} x e^{-i p^2 t/(2 m \Hbar)}
= \sum_{k = 0}^\infty \inv{k!} \lr{ \frac{i p^2 t}{2 m \Hbar} }^k
x
e^{-i p^2 t/(2 m \Hbar)}
= \sum_{k = 0}^\infty \inv{k!} \lr{ \frac{i t}{2 m \Hbar} }^k p^{2k} x
e^{-i p^2 t/(2 m \Hbar)}
=
\sum_{k = 0}^\infty \inv{k!} \lr{ \frac{i t}{2 m \Hbar} }^k \lr{ \antisymmetric{p^{2k}}{x} + x p^{2k} }
e^{-i p^2 t/(2 m \Hbar)}
= x +
\sum_{k = 0}^\infty \inv{k!} \lr{ \frac{i t}{2 m \Hbar} }^k \antisymmetric{p^{2k}}{x}
e^{-i p^2 t/(2 m \Hbar)}
= x +
\sum_{k = 0}^\infty \inv{k!} \lr{ \frac{i t}{2 m \Hbar} }^k \lr{ -i \Hbar \PD{p}{p^{2k}} }
e^{-i p^2 t/(2 m \Hbar)}
= x +
\sum_{k = 0}^\infty \inv{k!} \lr{ \frac{i t}{2 m \Hbar} }^k \lr{ -i \Hbar 2 k p^{2 k -1} }
e^{-i p^2 t/(2 m \Hbar)}
= x +
-2 i \Hbar p \frac{i t}{2 m \Hbar} \sum_{k = 1}^\infty \inv{(k-1)!} \lr{ \frac{i t}{2 m \Hbar} }^{k-1} p^{2(k - 1)}
e^{-i p^2 t/(2 m \Hbar)}
= x + t \frac{p}{m}.
\end{dmath}

This has the structure of a classical free particle \( x(t) = x + v t \), but in this case \( x,p \) are operators.

The evolved position commutator is
\begin{dmath}\label{eqn:positionCommutator:100}
\antisymmetric{x(t)}{x(0)}
=
\antisymmetric{x + t p/m}{x}
=
\frac{t}{m} \antisymmetric{p}{x}
=
-i \Hbar \frac{t}{m}.
\end{dmath}

Compare this to the classical Poisson bracket
\begin{dmath}\label{eqn:positionCommutator:120}
\antisymmetric{x(t)}{x(0)}_{\textrm{classical}}
=
\PD{x}{}\lr{x + p t/m} \PD{p}{x} - \PD{p}{}\lr{x + p t/m} \PD{x}{x}
=
- \frac{t}{m}.
\end{dmath}

This has the expected relation \( \antisymmetric{x(t)}{x(0)} = i \Hbar \antisymmetric{x(t)}{x(0)}_{\textrm{classical}} \).

} % answer

%\EndArticle

         % p7
         %%
% Copyright � 2015 Peeter Joot.  All Rights Reserved.
% Licenced as described in the file LICENSE under the root directory of this GIT repository.
%
%\newcommand{\authorname}{Peeter Joot}
\newcommand{\email}{peeterjoot@protonmail.com}
\newcommand{\basename}{FIXMEbasenameUndefined}
\newcommand{\dirname}{notes/FIXMEdirnameUndefined/}

%\renewcommand{\basename}{qmVirialTheorem}
%\renewcommand{\dirname}{notes/phy1520/}
%%\newcommand{\dateintitle}{}
%%\newcommand{\keywords}{}
%
%\newcommand{\authorname}{Peeter Joot}
\newcommand{\onlineurl}{http://sites.google.com/site/peeterjoot2/math2013/\basename.pdf}
\newcommand{\sourcepath}{\dirname\basename.tex}
\newcommand{\generatetitle}[1]{\chapter{#1}}

\newcommand{\vcsinfo}{%
\section*{}
\noindent{\color{DarkOliveGreen}{\rule{\linewidth}{0.1mm}}}
\paragraph{Document version}
%\paragraph{\color{Maroon}{Document version}}
{
\small
\begin{itemize}
\item Available online at:\\ 
\href{\onlineurl}{\onlineurl}
\item Git Repository: \input{./.revinfo/gitRepo.tex}
\item Source: \sourcepath
\item last commit: \input{./.revinfo/gitCommitString.tex}
\item commit date: \input{./.revinfo/gitCommitDate.tex}
\end{itemize}
}
}

%\PassOptionsToPackage{dvipsnames,svgnames}{xcolor}
\PassOptionsToPackage{square,numbers}{natbib}
\documentclass{scrreprt}

\usepackage[left=2cm,right=2cm]{geometry}
\usepackage[svgnames]{xcolor}
\usepackage{peeters_layout}

\usepackage{natbib}

\usepackage[
colorlinks=true,
bookmarks=false,
pdfauthor={\authorname, \email},
backref 
]{hyperref}

% http://tex.stackexchange.com/questions/75773/how-to-reference-problems-by-the-text-label-in-an-exercise-envioronment
\usepackage[english]{cleveref}
\crefname{Exercise}{exercise}{exercises}
\Crefname{Exercise}{Exercise}{Exercises}

\RequirePackage{titlesec}
\RequirePackage{ifthen}

% http://stackoverflow.com/questions/4932910/date-in-the-tabular-environment
\makeatletter
\let\insertdate\@date
\makeatother

\titleformat{\chapter}[display]
{\bfseries\Large}
{\color{DarkSlateGrey}\filleft \authorname
\ifthenelse{\isundefined{\studentnumber}}{}{\\ \studentnumber}
\ifthenelse{\isundefined{\email}}{}{\\ \email}
\ifthenelse{\isundefined{\dateintitle}}{}{\\ \insertdate}
%\ifthenelse{\isundefined{\coursename}}{}{\\ \coursename} % put in title instead.
}
{4ex}
{\color{DarkOliveGreen}{\titlerule}\color{Maroon}
\vspace{2ex}%
\filright}
[\vspace{2ex}%
\color{DarkOliveGreen}\titlerule
]

\newcommand{\beginArtWithToc}[0]{\begin{document}\tableofcontents}
\newcommand{\beginArtNoToc}[0]{\begin{document}}
\newcommand{\EndNoBibArticle}[0]{\end{document}}
\newcommand{\EndArticle}[0]{\bibliography{Bibliography}\bibliographystyle{plainnat}\end{document}}

% 
%\newcommand{\citep}[1]{\cite{#1}}

\colorSectionsForArticle


%
%\usepackage{peeters_layout_exercise}
%\usepackage{peeters_braket}
%\usepackage{peeters_figures}
%
%\beginArtNoToc
%
%\generatetitle{Quantum Virial Theorem}
%\chapter{Quantum Virial Theorem}
%\label{chap:qmVirialTheorem}

\makeoproblem{Quantum virial Theorem.}{problem:qmVirialTheorem:7}{\citep{sakurai2014modern} pr. 2.7}{
\index{virial theorem}
Consider a particle with Hamiltonian

\begin{dmath}\label{eqn:qmVirialTheorem:20}
H = \frac{\Bp^2}{2 m} + V(\Bx),
\end{dmath}

By calculating the time evolution of \( \antisymmetric{\Bx \cdot \Bp}{H} \), identify the quantum virial theorem and show the conditions where it is satisfied.

} % problem

\makeanswer{problem:qmVirialTheorem:7}{

\begin{dmath}\label{eqn:qmVirialTheorem:40}
\antisymmetric{\Bx \cdot \Bp}{H}
=
\inv{2 m} \antisymmetric{\Bx \cdot \Bp}{\Bp^2} + \antisymmetric{\Bx \cdot \Bp}{V(\Bx)}
=
\inv{2 m} \lr{ x_r p_r \Bp^2 - \Bp^2 x_r p_r}
+
\lr{ x_r p_r V(\Bx) - V(\Bx) x_r p_r }
=
\inv{2 m} \antisymmetric{ x_r }{\Bp^2} p_r
+
x_r \antisymmetric{ p_r}{ V(\Bx)},
\end{dmath}

Evaluating those commutators separately, gives

\begin{dmath}\label{eqn:qmVirialTheorem:60}
\begin{aligned}
\antisymmetric{ x_r }{\Bp^2}
&=
\antisymmetric{ x_r }{p_r^2}\qquad \text{no sum} \\
&=
2 i \Hbar p_r,
\end{aligned}
\end{dmath}

and

\begin{dmath}\label{eqn:qmVirialTheorem:80}
\antisymmetric{ p_r}{ V(\Bx)}
= -i \Hbar \PD{x_r}{V(\Bx)},
\end{dmath}

so
\begin{dmath}\label{eqn:qmVirialTheorem:100}
\ddt{}\lr{\Bx \cdot \Bp}
=
\inv{i \Hbar}
\antisymmetric{\Bx \cdot \Bp}{H}
=
\inv{2 m} 2 p_r p_r - x_r \PD{x_r}{V(\Bx)}
=
\frac{\Bp^2}{m} - \Bx \cdot \spacegrad V(\Bx).
\end{dmath}

Taking expectation values, assuming that the states are independent of time, we have

\begin{dmath}\label{eqn:qmVirialTheorem:120}
0
= \ddt{} \expectation{ \Bx \cdot \Bp }
= \expectation{\frac{\Bp^2}{m}} - \expectation{\Bx \cdot \spacegrad V(\Bx)}.
\end{dmath}

Note that taking the expectation with respect to stationary states was required to reverse the order of the time derivative with the expectation operation.

The right hand side is the quantum equivalent of the virial theorem, relating the average kinetic energy to the potential

\begin{dmath}\label{eqn:qmVirialTheorem:140}
2 \expectation{T} = \expectation{\Bx \cdot \spacegrad V(\Bx)}
\end{dmath}

} % answer

%\EndArticle

         % ps2. virial theorem
         %
% Copyright � 2015 Peeter Joot.  All Rights Reserved.
% Licenced as described in the file LICENSE under the root directory of this GIT repository.
%
\makeproblem{Virial theorem}{gradQuantum:problemSet2:2}{ 

Consider a three-dimensional Hamiltonian 

\begin{dmath}\label{eqn:gradQuantumProblemSet2Problem2:20}
H = \frac{\Bp^2}{2m} + V(\Bx).
\end{dmath}

Calculate \( \antisymmetric{\Bx \cdot \Bp}{H} \) and show that

\begin{dmath}\label{eqn:gradQuantumProblemSet2Problem2:40}
\ddt{} \expectation{ \Bx \cdot \Bp } = \expectation{ \frac{\Bp^2}{2m} } - \expectation{ \Bx \cdot \spacegrad V }.
\end{dmath}


\makesubproblem{}{gradQuantum:problemSet2:2a}
Under what conditions does the left-hand side vanish?  

\makesubproblem{}{gradQuantum:problemSet2:2b}
When the l.h.s. vanishes, the result that the r.h.s. is zero is called the quantum virial theorem. 
Consider the 3D isotropic harmonic oscillator, and show explicitly that its eigenstates obey the virial theorem. 

\makesubproblem{}{gradQuantum:problemSet2:2c}
Evaluate the r.h.s. for the superposition state \( \ket{0,0,0} + \ket{0,0,2} \) where the notation stands for \( \ket{ n_x, n_y, n_z } \) occupation numbers.

} % makeproblem

%%%%%%%%%%%%%%%%%
%
% ../phy1520/qmVirialTheorem.tex
%
%
\makeanswer{gradQuantum:problemSet2:2}{ 

TODO.
\makeSubAnswer{}{gradQuantum:problemSet2:2a}

TODO.
\makeSubAnswer{}{gradQuantum:problemSet2:2b}

TODO.
\makeSubAnswer{}{gradQuantum:problemSet2:2c}

TODO.
}

         % p9:
         %
% Copyright � 2015 Peeter Joot.  All Rights Reserved.
% Licenced as described in the file LICENSE under the root directory of this GIT repository.
%
%\newcommand{\authorname}{Peeter Joot}
\newcommand{\email}{peeterjoot@protonmail.com}
\newcommand{\basename}{FIXMEbasenameUndefined}
\newcommand{\dirname}{notes/FIXMEdirnameUndefined/}

%\renewcommand{\basename}{symmetricHamiltonianEvolution}
%\renewcommand{\dirname}{notes/phy1520/}
%%\newcommand{\dateintitle}{}
%%\newcommand{\keywords}{}
%
%\newcommand{\authorname}{Peeter Joot}
\newcommand{\onlineurl}{http://sites.google.com/site/peeterjoot2/math2013/\basename.pdf}
\newcommand{\sourcepath}{\dirname\basename.tex}
\newcommand{\generatetitle}[1]{\chapter{#1}}

\newcommand{\vcsinfo}{%
\section*{}
\noindent{\color{DarkOliveGreen}{\rule{\linewidth}{0.1mm}}}
\paragraph{Document version}
%\paragraph{\color{Maroon}{Document version}}
{
\small
\begin{itemize}
\item Available online at:\\ 
\href{\onlineurl}{\onlineurl}
\item Git Repository: \input{./.revinfo/gitRepo.tex}
\item Source: \sourcepath
\item last commit: \input{./.revinfo/gitCommitString.tex}
\item commit date: \input{./.revinfo/gitCommitDate.tex}
\end{itemize}
}
}

%\PassOptionsToPackage{dvipsnames,svgnames}{xcolor}
\PassOptionsToPackage{square,numbers}{natbib}
\documentclass{scrreprt}

\usepackage[left=2cm,right=2cm]{geometry}
\usepackage[svgnames]{xcolor}
\usepackage{peeters_layout}

\usepackage{natbib}

\usepackage[
colorlinks=true,
bookmarks=false,
pdfauthor={\authorname, \email},
backref 
]{hyperref}

% http://tex.stackexchange.com/questions/75773/how-to-reference-problems-by-the-text-label-in-an-exercise-envioronment
\usepackage[english]{cleveref}
\crefname{Exercise}{exercise}{exercises}
\Crefname{Exercise}{Exercise}{Exercises}

\RequirePackage{titlesec}
\RequirePackage{ifthen}

% http://stackoverflow.com/questions/4932910/date-in-the-tabular-environment
\makeatletter
\let\insertdate\@date
\makeatother

\titleformat{\chapter}[display]
{\bfseries\Large}
{\color{DarkSlateGrey}\filleft \authorname
\ifthenelse{\isundefined{\studentnumber}}{}{\\ \studentnumber}
\ifthenelse{\isundefined{\email}}{}{\\ \email}
\ifthenelse{\isundefined{\dateintitle}}{}{\\ \insertdate}
%\ifthenelse{\isundefined{\coursename}}{}{\\ \coursename} % put in title instead.
}
{4ex}
{\color{DarkOliveGreen}{\titlerule}\color{Maroon}
\vspace{2ex}%
\filright}
[\vspace{2ex}%
\color{DarkOliveGreen}\titlerule
]

\newcommand{\beginArtWithToc}[0]{\begin{document}\tableofcontents}
\newcommand{\beginArtNoToc}[0]{\begin{document}}
\newcommand{\EndNoBibArticle}[0]{\end{document}}
\newcommand{\EndArticle}[0]{\bibliography{Bibliography}\bibliographystyle{plainnat}\end{document}}

% 
%\newcommand{\citep}[1]{\cite{#1}}

\colorSectionsForArticle


%
%\usepackage{peeters_layout_exercise}
%\usepackage{peeters_braket}
%\usepackage{peeters_figures}
%
%\beginArtNoToc
%
%\generatetitle{A symmetric real Hamiltonian}
%\chapter{A symmetric real Hamiltonian}
%\label{chap:symmetricHamiltonianEvolution}

\makeoproblem{A symmetric real Hamiltonian.}{problem:symmetricHamiltonianEvolution:9}{\citep{sakurai2014modern} pr. 2.9}{
\index{Hamiltonian!symmetric real}

Find the time evolution for the state \( \ket{a'} \) for a Hamiltonian of the form

\begin{dmath}\label{eqn:symmetricHamiltonianEvolution:20}
H = \delta \lr{ \ket{a'}\bra{a'} + \ket{a''}\bra{a''} }
\end{dmath}
} % problem

\makeanswer{problem:symmetricHamiltonianEvolution:9}{

This Hamiltonian has the matrix representation

\begin{dmath}\label{eqn:symmetricHamiltonianEvolution:40}
H =
\begin{bmatrix}
0 & \delta \\
\delta & 0
\end{bmatrix},
\end{dmath}

which has a characteristic equation of

\begin{dmath}\label{eqn:symmetricHamiltonianEvolution:60}
\lambda^2 -\delta^2 = 0,
\end{dmath}

so the energy eigenvalues are \( \pm \delta \).

The diagonal basis states are respectively

\begin{dmath}\label{eqn:symmetricHamiltonianEvolution:80}
\ket{\pm\delta} =
\inv{\sqrt{2}}
\begin{bmatrix}
\pm 1 \\
1
\end{bmatrix}.
\end{dmath}

The time evolution operator is

\begin{dmath}\label{eqn:symmetricHamiltonianEvolution:100}
U
= e^{-i H t/\Hbar}
=
  e^{-i \delta t/\Hbar} \ket{+\delta}\bra{+\delta}
+ e^{i \delta t/\Hbar} \ket{-\delta}\bra{-\delta}
=
\frac{e^{-i \delta t/\Hbar} }{2}
\begin{bmatrix}
1 & 1
\end{bmatrix}
\begin{bmatrix}
1  \\
1
\end{bmatrix}
+ \frac{e^{i \delta t/\Hbar} }{2}
\begin{bmatrix}
-1 & 1
\end{bmatrix}
\begin{bmatrix}
-1  \\
1
\end{bmatrix}
=
\frac{e^{-i \delta t/\Hbar} }{2}
\begin{bmatrix}
1 & 1 \\
1 & 1
\end{bmatrix}
+\frac{e^{i \delta t/\Hbar} }{2}
\begin{bmatrix}
1 & -1 \\
-1 & 1
\end{bmatrix}
=
\begin{bmatrix}
\cos(\delta t/\Hbar) & -i\sin(\delta t/\Hbar) \\
-i \sin(\delta t/\Hbar) & \cos(\delta t/\Hbar) \\
\end{bmatrix}.
\end{dmath}

%The non-diagonal states have the matrix representation
%
%\begin{dmath}\label{eqn:symmetricHamiltonianEvolution:120}
%\begin{aligned}
%\ket{a'} &= \inv{\sqrt{2}} \lr{ \ket{+\delta} - \ket{-\delta} } \\
%\ket{a''} &= \inv{\sqrt{2}} \lr{ \ket{+\delta} + \ket{-\delta} },
%\end{aligned}
%\end{dmath}
%
%so
The desired time evolution in the original basis is

\begin{dmath}\label{eqn:symmetricHamiltonianEvolution:140}
\ket{a', t}
=
e^{-i H t/\Hbar}
\ket{a', 0}
=
\begin{bmatrix}
\cos(\delta t/\Hbar) & -i\sin(\delta t/\Hbar) \\
-i \sin(\delta t/\Hbar) & \cos(\delta t/\Hbar) \\
\end{bmatrix}
\begin{bmatrix}
1 \\
0
\end{bmatrix}
=
\begin{bmatrix}
\cos(\delta t/\Hbar) \\
-i \sin(\delta t/\Hbar)
\end{bmatrix}
=
\cos(\delta t/\Hbar) \ket{a',0} -i \sin(\delta t/\Hbar) \ket{a'',0}.
\end{dmath}

This evolution has the same structure as left circularly polarized light.

The probability of finding the system in state \( \ket{a''} \) given an initial state of \( \ket{a',0} \) is

\begin{dmath}\label{eqn:symmetricHamiltonianEvolution:160}
P
=
\Abs{\braket{a''}{a',t}}^2
=
\sin^2 \lr{ \delta t/\Hbar }.
\end{dmath}
} % answer

%\EndArticle

         % p12
         %
% Copyright � 2015 Peeter Joot.  All Rights Reserved.
% Licenced as described in the file LICENSE under the root directory of this GIT repository.
%
%\newcommand{\authorname}{Peeter Joot}
\newcommand{\email}{peeterjoot@protonmail.com}
\newcommand{\basename}{FIXMEbasenameUndefined}
\newcommand{\dirname}{notes/FIXMEdirnameUndefined/}

%\renewcommand{\basename}{translationExpectation}
%\renewcommand{\dirname}{notes/phy1520/}
%%\newcommand{\dateintitle}{}
%%\newcommand{\keywords}{}
%
%\newcommand{\authorname}{Peeter Joot}
\newcommand{\onlineurl}{http://sites.google.com/site/peeterjoot2/math2013/\basename.pdf}
\newcommand{\sourcepath}{\dirname\basename.tex}
\newcommand{\generatetitle}[1]{\chapter{#1}}

\newcommand{\vcsinfo}{%
\section*{}
\noindent{\color{DarkOliveGreen}{\rule{\linewidth}{0.1mm}}}
\paragraph{Document version}
%\paragraph{\color{Maroon}{Document version}}
{
\small
\begin{itemize}
\item Available online at:\\ 
\href{\onlineurl}{\onlineurl}
\item Git Repository: \input{./.revinfo/gitRepo.tex}
\item Source: \sourcepath
\item last commit: \input{./.revinfo/gitCommitString.tex}
\item commit date: \input{./.revinfo/gitCommitDate.tex}
\end{itemize}
}
}

%\PassOptionsToPackage{dvipsnames,svgnames}{xcolor}
\PassOptionsToPackage{square,numbers}{natbib}
\documentclass{scrreprt}

\usepackage[left=2cm,right=2cm]{geometry}
\usepackage[svgnames]{xcolor}
\usepackage{peeters_layout}

\usepackage{natbib}

\usepackage[
colorlinks=true,
bookmarks=false,
pdfauthor={\authorname, \email},
backref 
]{hyperref}

% http://tex.stackexchange.com/questions/75773/how-to-reference-problems-by-the-text-label-in-an-exercise-envioronment
\usepackage[english]{cleveref}
\crefname{Exercise}{exercise}{exercises}
\Crefname{Exercise}{Exercise}{Exercises}

\RequirePackage{titlesec}
\RequirePackage{ifthen}

% http://stackoverflow.com/questions/4932910/date-in-the-tabular-environment
\makeatletter
\let\insertdate\@date
\makeatother

\titleformat{\chapter}[display]
{\bfseries\Large}
{\color{DarkSlateGrey}\filleft \authorname
\ifthenelse{\isundefined{\studentnumber}}{}{\\ \studentnumber}
\ifthenelse{\isundefined{\email}}{}{\\ \email}
\ifthenelse{\isundefined{\dateintitle}}{}{\\ \insertdate}
%\ifthenelse{\isundefined{\coursename}}{}{\\ \coursename} % put in title instead.
}
{4ex}
{\color{DarkOliveGreen}{\titlerule}\color{Maroon}
\vspace{2ex}%
\filright}
[\vspace{2ex}%
\color{DarkOliveGreen}\titlerule
]

\newcommand{\beginArtWithToc}[0]{\begin{document}\tableofcontents}
\newcommand{\beginArtNoToc}[0]{\begin{document}}
\newcommand{\EndNoBibArticle}[0]{\end{document}}
\newcommand{\EndArticle}[0]{\bibliography{Bibliography}\bibliographystyle{plainnat}\end{document}}

% 
%\newcommand{\citep}[1]{\cite{#1}}

\colorSectionsForArticle


%
%\usepackage{peeters_layout_exercise}
%\usepackage{peeters_braket}
%\usepackage{peeters_figures}
%
%\beginArtNoToc
%
%\generatetitle{SHO translation operator expectation}
%\chapter{SHO translation operator expectation}
%\label{chap:translationExpectation}

\makeoproblem{SHO translation operator expectation.}{problem:translationExpectation:2.12}{\citep{sakurai2014modern} pr. 2.12}{

\index{translation!expectation}
\index{harmonic oscillator!translation operator}
Using the Heisenberg picture evaluate the expectation of the position operator \( \expectation{x} \) with respect to the initial time state

\begin{dmath}\label{eqn:translationExpectation:20}
\ket{\alpha, 0} = e^{-i p_0 a/\Hbar} \ket{0},
\end{dmath}

where \( p_0 \) is the initial time position operator, and \( a \) is a constant with dimensions of position.

} % problem

\makeanswer{problem:translationExpectation:2.12}{

Recall that the Heisenberg picture position operator expands to

\begin{dmath}\label{eqn:translationExpectation:40}
x^\txtH(t)
= U^\dagger x U
= x_0 \cos(\omega t) + \frac{p_0}{m \omega} \sin(\omega t),
\end{dmath}

so the expectation of the position operator is
\begin{dmath}\label{eqn:translationExpectation:60}
\expectation{x}
=
\bra{0} e^{i p_0 a/\Hbar} \lr{ x_0 \cos(\omega t) + \frac{p_0}{m \omega} \sin(\omega t) } e^{-i p_0 a/\Hbar} \ket{0}
=
\bra{0} \lr{ e^{i p_0 a/\Hbar} x_0 \cos(\omega t) e^{-i p_0 a/\Hbar} \cos(\omega t) + \frac{p_0}{m \omega} \sin(\omega t) } \ket{0}.
\end{dmath}

The exponential sandwich above can be expanded using the Baker-Campbell-Hausdorff \citep{wiki:bakercampbellHausdorff} formula

\begin{dmath}\label{eqn:translationExpectation:80}
\begin{aligned}
e^{i p_0 a/\Hbar} x_0 e^{-i p_0 a/\Hbar}
&=
x_0
+ \frac{i a}{\Hbar} \antisymmetric{p_0}{x_0}
+ \inv{2!} \lr{\frac{i a}{\Hbar}}^2 \antisymmetric{p_0}{\antisymmetric{p_0}{x_0}}
+ \cdots \\
&=
x_0
+ \frac{i a}{\Hbar} \lr{ -i \Hbar }
+ \inv{2!} \lr{\frac{i a}{\Hbar}}^2 \antisymmetric{p_0}{-i \Hbar}
+ \cdots \\
&=
x_0 + a.
\end{aligned}
\end{dmath}

The position expectation with respect to this translated state is

\begin{dmath}\label{eqn:translationExpectation:100}
\expectation{x}
= \bra{0} \lr{ (x_0 + a)\cos(\omega t) + \frac{p_0}{m \omega} \sin(\omega t) }\ket{0}
= a \cos(\omega t).
\end{dmath}

The final simplification above follows from \( \bra{n} x \ket{n} = \bra{n} p \ket{n} = 0 \).

} % answer

%\EndArticle

         % p14
         %
% Copyright � 2015 Peeter Joot.  All Rights Reserved.
% Licenced as described in the file LICENSE under the root directory of this GIT repository.
%
%\newcommand{\authorname}{Peeter Joot}
\newcommand{\email}{peeterjoot@protonmail.com}
\newcommand{\basename}{FIXMEbasenameUndefined}
\newcommand{\dirname}{notes/FIXMEdirnameUndefined/}

%\renewcommand{\basename}{shoExpectations}
%\renewcommand{\dirname}{notes/phy1520/}
%%\newcommand{\dateintitle}{}
%%\newcommand{\keywords}{}
%
%\newcommand{\authorname}{Peeter Joot}
\newcommand{\onlineurl}{http://sites.google.com/site/peeterjoot2/math2013/\basename.pdf}
\newcommand{\sourcepath}{\dirname\basename.tex}
\newcommand{\generatetitle}[1]{\chapter{#1}}

\newcommand{\vcsinfo}{%
\section*{}
\noindent{\color{DarkOliveGreen}{\rule{\linewidth}{0.1mm}}}
\paragraph{Document version}
%\paragraph{\color{Maroon}{Document version}}
{
\small
\begin{itemize}
\item Available online at:\\ 
\href{\onlineurl}{\onlineurl}
\item Git Repository: \input{./.revinfo/gitRepo.tex}
\item Source: \sourcepath
\item last commit: \input{./.revinfo/gitCommitString.tex}
\item commit date: \input{./.revinfo/gitCommitDate.tex}
\end{itemize}
}
}

%\PassOptionsToPackage{dvipsnames,svgnames}{xcolor}
\PassOptionsToPackage{square,numbers}{natbib}
\documentclass{scrreprt}

\usepackage[left=2cm,right=2cm]{geometry}
\usepackage[svgnames]{xcolor}
\usepackage{peeters_layout}

\usepackage{natbib}

\usepackage[
colorlinks=true,
bookmarks=false,
pdfauthor={\authorname, \email},
backref 
]{hyperref}

% http://tex.stackexchange.com/questions/75773/how-to-reference-problems-by-the-text-label-in-an-exercise-envioronment
\usepackage[english]{cleveref}
\crefname{Exercise}{exercise}{exercises}
\Crefname{Exercise}{Exercise}{Exercises}

\RequirePackage{titlesec}
\RequirePackage{ifthen}

% http://stackoverflow.com/questions/4932910/date-in-the-tabular-environment
\makeatletter
\let\insertdate\@date
\makeatother

\titleformat{\chapter}[display]
{\bfseries\Large}
{\color{DarkSlateGrey}\filleft \authorname
\ifthenelse{\isundefined{\studentnumber}}{}{\\ \studentnumber}
\ifthenelse{\isundefined{\email}}{}{\\ \email}
\ifthenelse{\isundefined{\dateintitle}}{}{\\ \insertdate}
%\ifthenelse{\isundefined{\coursename}}{}{\\ \coursename} % put in title instead.
}
{4ex}
{\color{DarkOliveGreen}{\titlerule}\color{Maroon}
\vspace{2ex}%
\filright}
[\vspace{2ex}%
\color{DarkOliveGreen}\titlerule
]

\newcommand{\beginArtWithToc}[0]{\begin{document}\tableofcontents}
\newcommand{\beginArtNoToc}[0]{\begin{document}}
\newcommand{\EndNoBibArticle}[0]{\end{document}}
\newcommand{\EndArticle}[0]{\bibliography{Bibliography}\bibliographystyle{plainnat}\end{document}}

% 
%\newcommand{\citep}[1]{\cite{#1}}

\colorSectionsForArticle


%
%\usepackage{peeters_layout_exercise}
%\usepackage{peeters_braket}
%\usepackage{peeters_figures}
%
%\beginArtNoToc
%
%\generatetitle{Expectations for SHO Hamiltonian, and virial theorem.}
%%\label{chap:shoExpectations}

\makeoproblem{Expectations for SHO Hamiltonian, and virial theorem.}{problem:shoExpectations:1}{\citep{sakurai2014modern} pr. 2.14}{
\index{harmonic oscillator}
\index{virial theorem}
\makesubproblem{}{problem:shoExpectations:1:a}

For a 1D SHO, compute \(
\bra{m} x \ket{n},
\bra{m} x^2 \ket{n},
\bra{m} p \ket{n},
\bra{m} p^2 \ket{n} \) and \( \bra{m} \symmetric{x}{p} \ket{n} \).

\makesubproblem{}{problem:shoExpectations:1:b}

Verify the virial theorem is satisfied for energy eigenstates.
} % problem

\makeanswer{problem:shoExpectations:1}{

\makeSubAnswer{}{problem:shoExpectations:1:a}

Using

\begin{equation}\label{eqn:shoExpectations:20}
\begin{aligned}
x &= \frac{x_0}{\sqrt{2}} \lr{ a + a^\dagger } \\
p &= \frac{i\Hbar}{x_0 \sqrt{2}} \lr{ a^\dagger - a} \\
a(t) &= a(0) e^{-i \omega t} \\
a(0) \ket{n} &= \sqrt{n} \ket{n-1} \\
a^\dagger(0) \ket{n} &= \sqrt{n+1} \ket{n+1} \\
x_0^2 &= \frac{\Hbar}{\omega m},
\end{aligned}
\end{equation}

we have

\begin{dmath}\label{eqn:shoExpectations:40}
\bra{m} x \ket{n}
=
\frac{x_0}{\sqrt{2}} \bra{m} \lr{ a + a^\dagger } \ket{n}
=
\frac{x_0}{\sqrt{2}} \bra{m}
\lr{
e^{-i \omega t} \sqrt{n} \ket{n-1}
+
e^{i \omega t} \sqrt{n+1} \ket{n+1}
}
=
\frac{x_0}{\sqrt{2}} \lr{
\delta_{m, n-1} e^{-i \omega t} \sqrt{n}
+
\delta_{m, n+1} e^{i \omega t} \sqrt{n+1}
},
\end{dmath}

\begin{dmath}\label{eqn:shoExpectations:60}
\bra{m} x^2 \ket{n}
=
\frac{x_0^2}{2} \bra{m} \lr{ a + a^\dagger }^2 \ket{n}
=
\frac{x_0^2}{2}
\lr{
e^{i \omega t} \sqrt{m} \bra{m-1}
+
e^{-i \omega t} \sqrt{m+1} \bra{m+1}
}
\lr{
e^{-i \omega t} \sqrt{n} \ket{n-1}
+
e^{i \omega t} \sqrt{n+1} \ket{n+1}
}
=
\frac{x_0^2}{2}
\lr{
\delta_{m+1,n+1} \sqrt{(m+1)(n+1)}
+\delta_{m+1,n-1} \sqrt{(m+1)n} e^{-2 i \omega t}
+\delta_{m-1,n+1} \sqrt{m(n+1)} e^{2 i \omega t}
+\delta_{m-1,n-1} \sqrt{m n}
},
\end{dmath}

\begin{dmath}\label{eqn:shoExpectations:80}
\bra{m} p \ket{n}
=
\frac{i\Hbar}{\sqrt{2} x_0} \bra{m} \lr{ a^\dagger - a} \ket{n}
=
\frac{i\Hbar}{\sqrt{2} x_0} \bra{m} \lr{
e^{i \omega t} \sqrt{n+1} \ket{n+1}
-
e^{-i \omega t} \sqrt{n} \ket{n-1}
}
=
\frac{i\Hbar}{\sqrt{2} x_0} \lr{
\delta_{m,n+1} e^{i \omega t} \sqrt{n+1}
-
\delta_{m,n-1} e^{-i \omega t} \sqrt{n}
},
\end{dmath}

\begin{dmath}\label{eqn:shoExpectations:100}
\bra{m} p^2 \ket{n}
=
\frac{\Hbar^2}{2 x_0^2} \ket{m} \lr{ a - a^\dagger }  \lr{ a^\dagger - a} \ket{n}
=
\frac{\Hbar^2}{2 x_0^2}
\lr{
-e^{-i \omega t} \sqrt{m+1} \bra{m+1}
+
e^{i \omega t} \sqrt{m} \bra{m-1}
}
\lr{
e^{i \omega t} \sqrt{n+1} \ket{n+1}
-
e^{-i \omega t} \sqrt{n} \ket{n-1}
}
=
\frac{\Hbar^2}{2 x_0^2}
\lr{
\delta_{m+1,n+1} \sqrt{(m+1)(n+1)}
+\delta_{m+1,n-1} \sqrt{(m+1)n} e^{-2 i \omega t}
+\delta_{m-1,n+1} \sqrt{m(n+1)} e^{2 i \omega t}
+\delta_{m-1,n-1} \sqrt{m n}
}.
\end{dmath}

For the anticommutator \( \symmetric{x}{p} \), we have

\begin{dmath}\label{eqn:shoExpectations:120}
\symmetric{x}{p}
=
\frac{i\Hbar}{2}
\lr{
\lr{ a e^{-i \omega t} + a^\dagger e^{i \omega t} } \lr{ a^\dagger e^{i \omega t} - a e^{-i \omega t} }
-
\lr{ a^\dagger e^{i \omega t} - a e^{-i \omega t} }
\lr{ a e^{-i \omega t} + a^\dagger e^{i \omega t} }
}
=
\frac{i\Hbar}{2}
\lr{
- a^2 e^{- 2 i \omega t}
+ (a^\dagger)^2 e^{ 2 i \omega t}
+ a a^\dagger
- a^\dagger a
+ a^2 e^{- 2 i \omega t}
- (a^\dagger)^2 e^{ 2 i \omega t}
- a^\dagger a
+ a a^\dagger
}
=
i\Hbar
\lr{
a a^\dagger - a^\dagger a
},
\end{dmath}

so

\begin{dmath}\label{eqn:shoExpectations:140}
\bra{m} \symmetric{x}{p} \ket{n}
=
i\Hbar
\bra{m}
\lr{
a a^\dagger - a^\dagger a
}
\ket{n}
=
i\Hbar
\bra{m}
\lr{
\sqrt{(n+1)^2}\ket{n}
-\sqrt{n^2}\ket{n}
}
=
i\Hbar
\bra{m}
\lr{
2 n + 1
}
\ket{n}.
\end{dmath}

\makeSubAnswer{}{problem:shoExpectations:1:b}

For the SHO, the virial theorem requires \( \expectation{p^2/m} = \expectation{m \omega x^2} \).  That momentum expectation with respect to the eigenstate \( \ket{n} \) is

\begin{dmath}\label{eqn:shoExpectations:160}
\expectation{p^2/m}
=
\frac{\Hbar^2}{2 x_0^2 m}
\lr{
\sqrt{(n+1)(n+1)}
+
\sqrt{n n}
}
=
\frac{\Hbar^2 m \omega}{2 \Hbar m} \lr{ 2 n + 1 }
=
\Hbar \omega \lr{ n + \inv{2} }.
\end{dmath}

For the position expectation we've got

\begin{dmath}\label{eqn:shoExpectations:180}
\expectation{m \omega x^2}
=
\frac{m \omega^2 x_0^2}{2}
\lr{
\sqrt{(n+1)(n+1)}
+ \sqrt{n n}
}
=
\frac{m \omega^2 \Hbar}{2 m \omega}
\lr{
\sqrt{(n+1)(n+1)}
+ \sqrt{n n}
}
=
\frac{\omega \Hbar}{2 }
\lr{ 2 n + 1 }
=
\omega \Hbar
\lr{ n + \inv{2} }.
\end{dmath}

This shows that the virial theorem holds for the SHO Hamiltonian for eigenstates.

} % answer

%\EndArticle

         % p15
         %
% Copyright � 2015 Peeter Joot.  All Rights Reserved.
% Licenced as described in the file LICENSE under the root directory of this GIT repository.
%
%\newcommand{\authorname}{Peeter Joot}
\newcommand{\email}{peeterjoot@protonmail.com}
\newcommand{\basename}{FIXMEbasenameUndefined}
\newcommand{\dirname}{notes/FIXMEdirnameUndefined/}

%\renewcommand{\basename}{shoMomentumSpace}
%\renewcommand{\dirname}{notes/phy1520/}
%%\newcommand{\dateintitle}{}
%%\newcommand{\keywords}{}
%
%\newcommand{\authorname}{Peeter Joot}
\newcommand{\onlineurl}{http://sites.google.com/site/peeterjoot2/math2013/\basename.pdf}
\newcommand{\sourcepath}{\dirname\basename.tex}
\newcommand{\generatetitle}[1]{\chapter{#1}}

\newcommand{\vcsinfo}{%
\section*{}
\noindent{\color{DarkOliveGreen}{\rule{\linewidth}{0.1mm}}}
\paragraph{Document version}
%\paragraph{\color{Maroon}{Document version}}
{
\small
\begin{itemize}
\item Available online at:\\ 
\href{\onlineurl}{\onlineurl}
\item Git Repository: \input{./.revinfo/gitRepo.tex}
\item Source: \sourcepath
\item last commit: \input{./.revinfo/gitCommitString.tex}
\item commit date: \input{./.revinfo/gitCommitDate.tex}
\end{itemize}
}
}

%\PassOptionsToPackage{dvipsnames,svgnames}{xcolor}
\PassOptionsToPackage{square,numbers}{natbib}
\documentclass{scrreprt}

\usepackage[left=2cm,right=2cm]{geometry}
\usepackage[svgnames]{xcolor}
\usepackage{peeters_layout}

\usepackage{natbib}

\usepackage[
colorlinks=true,
bookmarks=false,
pdfauthor={\authorname, \email},
backref 
]{hyperref}

% http://tex.stackexchange.com/questions/75773/how-to-reference-problems-by-the-text-label-in-an-exercise-envioronment
\usepackage[english]{cleveref}
\crefname{Exercise}{exercise}{exercises}
\Crefname{Exercise}{Exercise}{Exercises}

\RequirePackage{titlesec}
\RequirePackage{ifthen}

% http://stackoverflow.com/questions/4932910/date-in-the-tabular-environment
\makeatletter
\let\insertdate\@date
\makeatother

\titleformat{\chapter}[display]
{\bfseries\Large}
{\color{DarkSlateGrey}\filleft \authorname
\ifthenelse{\isundefined{\studentnumber}}{}{\\ \studentnumber}
\ifthenelse{\isundefined{\email}}{}{\\ \email}
\ifthenelse{\isundefined{\dateintitle}}{}{\\ \insertdate}
%\ifthenelse{\isundefined{\coursename}}{}{\\ \coursename} % put in title instead.
}
{4ex}
{\color{DarkOliveGreen}{\titlerule}\color{Maroon}
\vspace{2ex}%
\filright}
[\vspace{2ex}%
\color{DarkOliveGreen}\titlerule
]

\newcommand{\beginArtWithToc}[0]{\begin{document}\tableofcontents}
\newcommand{\beginArtNoToc}[0]{\begin{document}}
\newcommand{\EndNoBibArticle}[0]{\end{document}}
\newcommand{\EndArticle}[0]{\bibliography{Bibliography}\bibliographystyle{plainnat}\end{document}}

% 
%\newcommand{\citep}[1]{\cite{#1}}

\colorSectionsForArticle


%
%\usepackage{peeters_layout_exercise}
%\usepackage{peeters_braket}
%\usepackage{peeters_figures}
%\usepackage{macros_qed}
%
%\beginArtNoToc
%
%\generatetitle{Momentum space representation of Schr\"{o}dinger equation}
%\chapter{Momentum space representation of Schr\"{o}dinger equation}
%\label{chap:shoMomentumSpace}

\makeoproblem{Momentum space representation of Schr\"{o}dinger equation.}{problem:shoMomentumSpace:15}{\citep{sakurai2014modern} pr. 2.15}{
\index{Schr\"{o}dinger equation!momentum space}

Using

\begin{dmath}\label{eqn:shoMomentumSpace:20}
\braket{x'}{p'} = \inv{\sqrt{2 \pi \Hbar}} e^{i p' x'/\Hbar},
\end{dmath}

show that

\begin{dmath}\label{eqn:shoMomentumSpace:40}
\bra{p'} x \ket{\alpha} = i \Hbar \PD{p'}{} \braket{p'}{\alpha}.
\end{dmath}

Use this to find the momentum space representation of the Schr\"{o}dinger equation for the one dimensional SHO and the energy eigenfunctions in their momentum representation.

} % problem

\makeanswer{problem:shoMomentumSpace:15}{
To expand the matrix element, introduce both momentum and position space identity operators

\begin{dmath}\label{eqn:shoMomentumSpace:60}
\bra{p'} x \ket{\alpha}
=
\int dx' dp'' \braket{p'}{x'}\bra{x'}x \ket{p''}\braket{p''}{\alpha}
=
\int dx' dp'' \braket{p'}{x'}x'\braket{x'}{p''}\braket{p''}{\alpha}
=
\inv{2 \pi \Hbar}
\int dx' dp'' e^{-i p' x'/\Hbar} x' e^{i p'' x'/\Hbar} \braket{p''}{\alpha}
=
\inv{2 \pi \Hbar}
\int dx' dp'' x' e^{i (p'' - p') x'/\Hbar} \braket{p''}{\alpha}
=
\inv{2 \pi \Hbar}
\int dx' dp'' \frac{d}{dp''}\lr{ \frac{-i \Hbar} e^{i (p'' - p') x'/\Hbar} } \braket{p''}{\alpha}
=
i \Hbar
\int dp''
\lr{ \inv{2 \pi \Hbar}
\int dx' e^{i (p'' - p') x'/\Hbar} } \frac{d}{dp''} \braket{p''}{\alpha}
=
i \Hbar
\int dp'' \delta(p''- p')
\frac{d}{dp''} \braket{p''}{\alpha}
=
i \Hbar
\frac{d}{dp'} \braket{p'}{\alpha}. \qedmarker
\end{dmath}

Schr\"{o}dinger's equation for a time dependent state \( \ket{\alpha} = U(t) \ket{\alpha,0} \) is

\begin{dmath}\label{eqn:shoMomentumSpace:80}
i \Hbar \PD{t}{} \ket{\alpha} = H \ket{\alpha},
\end{dmath}

with the momentum representation

\begin{dmath}\label{eqn:shoMomentumSpace:100}
i \Hbar \PD{t}{} \braket{p'}{\alpha} = \bra{p'} H \ket{\alpha}.
\end{dmath}

Expansion of the Hamiltonian matrix element for a strictly spatial dependent potential \( V(x) \) gives

\begin{dmath}\label{eqn:shoMomentumSpace:120}
\bra{p'} H \ket{\alpha}
=
\bra{p'} \lr{\frac{p^2}{2m} + V(x) } \ket{\alpha}
=
\frac{(p')^2}{2m}
+ \bra{p'} V(x) \ket{\alpha}.
\end{dmath}

Assuming a Taylor representation of the potential \( V(x) = \sum c_k x^k \), we want to calculate

\begin{dmath}\label{eqn:shoMomentumSpace:140}
\bra{p'} V(x) \ket{\alpha}
= \sum c_k \bra{p'} x^k \ket{\alpha}.
\end{dmath}

With \( \ket{\alpha} = \ket{p''} \) \cref{eqn:shoMomentumSpace:40} provides the \( k = 1 \) term

\begin{dmath}\label{eqn:shoMomentumSpace:160}
\bra{p'} x \ket{p''}
= i \Hbar \frac{d}{dp'} \braket{p'}{p''}
= i \Hbar \frac{d}{dp'} \delta(p' - p''),
\end{dmath}

where it is implied here that the derivative is operating on not just the delta function, but on all else that follows.

Using this the higher powers of \( \bra{p'} x^k \ket{\alpha} \) can be found easily.  For example for \( k = 2 \) we have

\begin{dmath}\label{eqn:shoMomentumSpace:180}
\bra{p'} x^2 \ket{\alpha}
=
\int dp''
\bra{p'} x \ket{p''}\bra{p''} x \ket{\alpha}
=
\int dp''
i \Hbar
\frac{d}{dp'} \delta(p' - p'') i \Hbar \frac{d}{dp''} \braket{p''}{\alpha}
=
\lr{ i \Hbar }^2 \frac{d^2}{d(p')^2} \braket{p'}{\alpha}.
\end{dmath}

This means that the potential matrix element is

\begin{dmath}\label{eqn:shoMomentumSpace:200}
\bra{p'} V(x) \ket{\alpha}
=
\sum c_k \lr{ i \Hbar \frac{d}{dp'} }^k \braket{p'}{\alpha}
= V\lr{ i \Hbar \frac{d}{dp'} }.
\end{dmath}

Writing \( \Psi_\alpha(p') = \braket{p'}{\alpha} \), the momentum space representation of Schr\"{o}dinger's equation for a position dependent potential is

%\begin{dmath}\label{eqn:shoMomentumSpace:220}
\boxedEquation{eqn:shoMomentumSpace:220}{
i \Hbar \PD{t}{} \Psi_\alpha(p')
=
\lr{ \frac{(p')^2}{2m} + V\lr{ i \Hbar \PDi{p'}{} } } \Psi_\alpha(p').
}
%\end{dmath}

For the SHO Hamiltonian the potential is \( V(x) = (1/2) m \omega^2 x^2 \), so the Schr\"{o}dinger equation is

\begin{dmath}\label{eqn:shoMomentumSpace:240}
i \Hbar \PD{t}{} \Psi_\alpha(p')
=
\lr{ \frac{(p')^2}{2m} - \inv{2} m \omega^2 \Hbar^2 \frac{\partial^2}{\partial(p')^2} } \Psi_\alpha(p')
=
\inv{2 m} \lr{ (p')^2 - m^2 \omega^2 \Hbar^2 \frac{\partial^2}{\partial(p')^2} } \Psi_\alpha(p').
\end{dmath}

To determine the wave functions, let's non-dimensionalize this and compare to the position space Schr\"{o}dinger equation.  Let

\begin{dmath}\label{eqn:shoMomentumSpace:260}
p_0^2 = m \omega \hbar,
\end{dmath}

so
\begin{dmath}\label{eqn:shoMomentumSpace:280}
i \Hbar \PD{t}{} \Psi_\alpha(p')
=
\frac{p_0^2}{2 m} \lr{ \lr{\frac{p'}{p_0}}^2 - \frac{\partial^2}{\partial(p'/p_0)^2} } \Psi_\alpha(p')
=
\frac{\omega \Hbar}{2}\lr{
- \frac{\partial^2}{\partial(p'/p_0)^2} +
\lr{\frac{p'}{p_0}}^2
} \Psi_\alpha(p').
\end{dmath}

Compare this to the position space equation with \( x_0^2 = m \omega/\Hbar \),

\begin{dmath}\label{eqn:shoMomentumSpace:300}
i \Hbar \PD{t}{} \Psi_\alpha(x')
=
\lr{ -\frac{\Hbar^2}{2m} \frac{\partial^2}{\partial(x')^2}
+
\inv{2} m \omega^2 (x')^2 }
\Psi_\alpha(x')
=
\frac{\Hbar^2}{2m}
\lr{ -\frac{\partial^2}{\partial(x')^2}
+
\frac{m^2 \omega^2}{\Hbar^2} (x')^2 }
\Psi_\alpha(x')
=
\frac{\Hbar^2 x_0^2}{2m}
\lr{
-\frac{\partial^2}{\partial(x'/x_0)^2}
+
\lr{\frac{x'}{x_0}}^2
}
\Psi_\alpha(x')
=
\frac{\Hbar \omega}{2}
\lr{
-\frac{\partial^2}{\partial(x'/x_0)^2}
+
\lr{\frac{x'}{x_0}}^2
}
\Psi_\alpha(x').
\end{dmath}

It's clear that there is a straightforward duality relationship between the respective wave functions.  Since

\begin{dmath}\label{eqn:shoMomentumSpace:320}
\braket{x'}{n} =
\inv{\pi^{1/4} \sqrt{2^n n!} x_0^{n + 1/2}}  \lr{ x' - x_0^2 \frac{d}{dx'} }^n \exp\lr{ -\inv{2} \lr{\frac{x'}{x_0}}^2 },
\end{dmath}

\index{wave function!momentum space}
the momentum space wave functions are

\begin{dmath}\label{eqn:shoMomentumSpace:340}
\braket{p'}{n} =
\inv{\pi^{1/4} \sqrt{2^n n!} p_0^{n + 1/2}}  \lr{ p' - p_0^2 \frac{d}{dp'} }^n \exp\lr{ -\inv{2} \lr{\frac{p'}{p_0}}^2 }.
\end{dmath}

} % answer

%\EndArticle

         % p16:
         %%
% Copyright � 2015 Peeter Joot.  All Rights Reserved.
% Licenced as described in the file LICENSE under the root directory of this GIT repository.
%
%\newcommand{\authorname}{Peeter Joot}
\newcommand{\email}{peeterjoot@protonmail.com}
\newcommand{\basename}{FIXMEbasenameUndefined}
\newcommand{\dirname}{notes/FIXMEdirnameUndefined/}

%\renewcommand{\basename}{correlationSHO}
%\renewcommand{\dirname}{notes/phy1520/}
%%\newcommand{\dateintitle}{}
%%\newcommand{\keywords}{}
%
%\newcommand{\authorname}{Peeter Joot}
\newcommand{\onlineurl}{http://sites.google.com/site/peeterjoot2/math2013/\basename.pdf}
\newcommand{\sourcepath}{\dirname\basename.tex}
\newcommand{\generatetitle}[1]{\chapter{#1}}

\newcommand{\vcsinfo}{%
\section*{}
\noindent{\color{DarkOliveGreen}{\rule{\linewidth}{0.1mm}}}
\paragraph{Document version}
%\paragraph{\color{Maroon}{Document version}}
{
\small
\begin{itemize}
\item Available online at:\\ 
\href{\onlineurl}{\onlineurl}
\item Git Repository: \input{./.revinfo/gitRepo.tex}
\item Source: \sourcepath
\item last commit: \input{./.revinfo/gitCommitString.tex}
\item commit date: \input{./.revinfo/gitCommitDate.tex}
\end{itemize}
}
}

%\PassOptionsToPackage{dvipsnames,svgnames}{xcolor}
\PassOptionsToPackage{square,numbers}{natbib}
\documentclass{scrreprt}

\usepackage[left=2cm,right=2cm]{geometry}
\usepackage[svgnames]{xcolor}
\usepackage{peeters_layout}

\usepackage{natbib}

\usepackage[
colorlinks=true,
bookmarks=false,
pdfauthor={\authorname, \email},
backref 
]{hyperref}

% http://tex.stackexchange.com/questions/75773/how-to-reference-problems-by-the-text-label-in-an-exercise-envioronment
\usepackage[english]{cleveref}
\crefname{Exercise}{exercise}{exercises}
\Crefname{Exercise}{Exercise}{Exercises}

\RequirePackage{titlesec}
\RequirePackage{ifthen}

% http://stackoverflow.com/questions/4932910/date-in-the-tabular-environment
\makeatletter
\let\insertdate\@date
\makeatother

\titleformat{\chapter}[display]
{\bfseries\Large}
{\color{DarkSlateGrey}\filleft \authorname
\ifthenelse{\isundefined{\studentnumber}}{}{\\ \studentnumber}
\ifthenelse{\isundefined{\email}}{}{\\ \email}
\ifthenelse{\isundefined{\dateintitle}}{}{\\ \insertdate}
%\ifthenelse{\isundefined{\coursename}}{}{\\ \coursename} % put in title instead.
}
{4ex}
{\color{DarkOliveGreen}{\titlerule}\color{Maroon}
\vspace{2ex}%
\filright}
[\vspace{2ex}%
\color{DarkOliveGreen}\titlerule
]

\newcommand{\beginArtWithToc}[0]{\begin{document}\tableofcontents}
\newcommand{\beginArtNoToc}[0]{\begin{document}}
\newcommand{\EndNoBibArticle}[0]{\end{document}}
\newcommand{\EndArticle}[0]{\bibliography{Bibliography}\bibliographystyle{plainnat}\end{document}}

% 
%\newcommand{\citep}[1]{\cite{#1}}

\colorSectionsForArticle


%
%\usepackage{peeters_layout_exercise}
%\usepackage{peeters_braket}
%\usepackage{peeters_figures}
%
%\beginArtNoToc
%
%\generatetitle{Correlation function}
%\chapter{Correlation function}
%\label{chap:correlationSHO}

\makeoproblem{Correlation function.}{problem:correlationSHO:16}{\citep{sakurai2014modern} pr. 2.16}{
\index{correlation function}

A correlation function can be defined as

\begin{dmath}\label{eqn:correlationSHO:20}
C(t) = \expectation{ x(t) x(0) }.
\end{dmath}

Using a Heisenberg picture \( x(t) \) calculate this correlation for the one dimensional SHO ground state.

} % problem

\makeanswer{problem:correlationSHO:16}{
The time dependent Heisenberg picture position operator was found to be

\begin{dmath}\label{eqn:correlationSHO:40}
x(t) = x(0) \cos(\omega t) + \frac{p(0)}{m \omega} \sin(\omega t),
\end{dmath}

so the correlation function is

\begin{dmath}\label{eqn:correlationSHO:60}
C(t)
=
\bra{0} \lr{ x(0) \cos(\omega t) + \frac{p(0)}{m \omega} \sin(\omega t)} x(0) \ket{0}
=
\cos(\omega t) \bra{0} x(0)^2 \ket{0} + \frac{\sin(\omega t)}{m \omega} \bra{0} p(0) x(0) \ket{0}
=
\frac{\Hbar \cos(\omega t) }{2 m \omega} \bra{0} \lr{ a + a^\dagger}^2 \ket{0} - \frac{i \Hbar}{m \omega} \sin(\omega t),
\end{dmath}

But
\begin{dmath}\label{eqn:correlationSHO:80}
\lr{ a + a^\dagger} \ket{0}
=
a^\dagger \ket{0}
=
\sqrt{1} \ket{1}
=
\ket{1},
\end{dmath}

so

\begin{dmath}\label{eqn:correlationSHO:100}
C(t) = x_0^2 \lr{ \inv{2} \cos(\omega t) - i \sin(\omega t) },
\end{dmath}

where \( x_0^2 = \Hbar/(m \omega) \), not to be confused with \( x(0)^2 \).

} % answer

%\EndArticle

         % ps2. correlation
         %
% Copyright � 2015 Peeter Joot.  All Rights Reserved.
% Licenced as described in the file LICENSE under the root directory of this GIT repository.
%

\makeoproblem{Correlation function.}{gradQuantum:problemSet2:3}{phy1520 2015 ps2.3}{
\index{correlation function}
Consider \( \expectation{x(0)x(t)} \) and \( \expectation{p(0)p(t)} \) where operators are in the Heisenberg representation. These are called correlation functions. Evaluate this for the 1D harmonic oscillator in an energy eigenstate \( \ket{n} \).
} % makeproblem

%
% ground state version of this in ../phy1520/correlationSHO.tex
%
\makeanswer{gradQuantum:problemSet2:3}{
\withproblemsetsParagraph{

In the Heisenberg representation the position and momentum operators evolve as

\begin{equation}\label{eqn:gradQuantumProblemSet2Problem3:20}
\begin{aligned}
x(t) &= x(0) \cos(\omega t) + \frac{p(0)}{m \omega} \sin(\omega t) \\
p(t) &= p(0) \cos(\omega t) - m \omega x(0) \sin(\omega t).
\end{aligned}
\end{equation}

To evaluate the expectation operations, we'll want the ladder operator representations of the position and momentum operators

\begin{equation}\label{eqn:gradQuantumProblemSet2Problem3:40}
\begin{aligned}
x(0) &= \frac{x_0}{\sqrt{2}} \evalbar{\lr{ a + a^\dagger }}{t = 0} \\
p(0) &= \frac{i \Hbar}{x_0 \sqrt{2}} \evalbar{\lr{ a^\dagger - a}}{t = 0},
\end{aligned}
\end{equation}

where

\begin{equation}\label{eqn:gradQuantumProblemSet2Problem3:60}
x_0 = \sqrt{\frac{\Hbar}{m \omega}}.
\end{equation}

The expectations of interest, with the raising and lowering operators evaluated at \( t = 0 \), are

\begin{dmath}\label{eqn:gradQuantumProblemSet2Problem3:80}
\bra{n} x(0) x(0) \ket{n}
=
\frac{x_0^2}{2} \bra{n} \lr{ a + a^\dagger }^2 \ket{n}
=
\frac{x_0^2}{2}
\lr{ \sqrt{n+1} \bra{n+1} + \sqrt{n} \bra{n-1} }
\lr{ \sqrt{n+1} \ket{n+1} + \sqrt{n} \ket{n-1} }
=
\frac{x_0^2}{2} \lr{ 2 n + 1 },
\end{dmath}

and
\begin{dmath}\label{eqn:gradQuantumProblemSet2Problem3:100}
\bra{n} x(0) p(0) \ket{n}
=
\frac{i \Hbar}{2}
\bra{n} \lr{ a + a^\dagger } \lr{ a^\dagger - a }  \ket{n}
=
\frac{i \Hbar}{2}
\lr{ \sqrt{n+1} \bra{n+1} + \sqrt{n} \bra{n-1} }
\lr{ \sqrt{n+1} \ket{n+1} - \sqrt{n} \ket{n-1} }
=
\frac{i \Hbar}{2},
\end{dmath}

\begin{dmath}\label{eqn:gradQuantumProblemSet2Problem3:101}
\bra{n} p(0) p(0) \ket{n}
=
\frac{-\Hbar^2 }{ 2 x_0^2}
\bra{n} \lr{ a^\dagger - a } \lr{ a^\dagger - a }  \ket{n}
=
\frac{-\Hbar^2 }{ 2 x_0^2}
\lr{ \sqrt{n+1} \bra{n+1} - \sqrt{n} \bra{n-1} }
\lr{ \sqrt{n+1} \ket{n+1} - \sqrt{n} \ket{n-1} }
=
\frac{(-\Hbar^2) }{ 2 x_0^2}
\lr{
2 n + 1
},
\end{dmath}

and finally

\begin{dmath}\label{eqn:gradQuantumProblemSet2Problem3:120}
\bra{n} p(0) x(0) \ket{n}
=
\bra{n}
\lr{
\antisymmetric{p(0)}{x(0)} + x(0) p(0)
}
\ket{n}
=
-i \Hbar + \frac{i\Hbar}{2}
=
\frac{i \Hbar}{2}.
\end{dmath}

Now we are ready to compute the correlations.  The position correlation is

\begin{dmath}\label{eqn:gradQuantumProblemSet2Problem3:140}
\bra{n} x(0) x(t) \ket{n}
=
\bra{n} x(0) \lr{
x(0) \cos(\omega t) + \frac{p(0)}{m \omega} \sin(\omega t)
} \ket{n}
=
\cos(\omega t) \bra{n} x(0) x(0) \ket{n}
+
\frac{1}{m \omega} \sin(\omega t) \bra{n} x(0) p(0) \ket{n}
=
\cos(\omega t)
\frac{x_0^2}{2} \lr{ 2 n + 1 }
+ \frac{1}{m \omega} \sin(\omega t) \frac{i \Hbar}{2},
\end{dmath}

which is

%\begin{equation}\label{eqn:gradQuantumProblemSet2Problem3:160}
\boxedEquation{eqn:gradQuantumProblemSet2Problem3:180}{
\bra{n} x(0) x(t) \ket{n}
=
\frac{x_0^2}{2}
\lr{
\lr{ 2 n + 1 }
\cos(\omega t)
+ i \sin(\omega t)
}.
}
%\end{equation}

The momentum correlation is

\begin{dmath}\label{eqn:gradQuantumProblemSet2Problem3:200}
\bra{n} p(0) p(t) \ket{n}
=
\bra{n} p(0) \lr{
p(0) \cos(\omega t) - m \omega x(0) \sin(\omega t)
} \ket{n}
=
\cos(\omega t)
\lr{ 2 n + 1 }
\frac{(-\Hbar^2) }{ 2 x_0^2}
- m \omega \sin(\omega t) \frac{i \Hbar}{2}.
\end{dmath}

With \( p_0^2 = m \omega \Hbar \), this is

\boxedEquation{eqn:gradQuantumProblemSet2Problem3:n}{
\bra{n} p(0) p(t) \ket{n}
=
-\frac{p_0^2}{2} \lr{
\lr{ 2 n + 1 }
\cos(\omega t)
+ i \sin(\omega t)
}.
}

}
}

         % p17
         %
% Copyright � 2015 Peeter Joot.  All Rights Reserved.
% Licenced as described in the file LICENSE under the root directory of this GIT repository.
%
%\newcommand{\authorname}{Peeter Joot}
\newcommand{\email}{peeterjoot@protonmail.com}
\newcommand{\basename}{FIXMEbasenameUndefined}
\newcommand{\dirname}{notes/FIXMEdirnameUndefined/}

%\renewcommand{\basename}{shoSuperposition}
%\renewcommand{\dirname}{notes/phy1520/}
%%\newcommand{\dateintitle}{}
%%\newcommand{\keywords}{}
%
%\newcommand{\authorname}{Peeter Joot}
\newcommand{\onlineurl}{http://sites.google.com/site/peeterjoot2/math2013/\basename.pdf}
\newcommand{\sourcepath}{\dirname\basename.tex}
\newcommand{\generatetitle}[1]{\chapter{#1}}

\newcommand{\vcsinfo}{%
\section*{}
\noindent{\color{DarkOliveGreen}{\rule{\linewidth}{0.1mm}}}
\paragraph{Document version}
%\paragraph{\color{Maroon}{Document version}}
{
\small
\begin{itemize}
\item Available online at:\\ 
\href{\onlineurl}{\onlineurl}
\item Git Repository: \input{./.revinfo/gitRepo.tex}
\item Source: \sourcepath
\item last commit: \input{./.revinfo/gitCommitString.tex}
\item commit date: \input{./.revinfo/gitCommitDate.tex}
\end{itemize}
}
}

%\PassOptionsToPackage{dvipsnames,svgnames}{xcolor}
\PassOptionsToPackage{square,numbers}{natbib}
\documentclass{scrreprt}

\usepackage[left=2cm,right=2cm]{geometry}
\usepackage[svgnames]{xcolor}
\usepackage{peeters_layout}

\usepackage{natbib}

\usepackage[
colorlinks=true,
bookmarks=false,
pdfauthor={\authorname, \email},
backref 
]{hyperref}

% http://tex.stackexchange.com/questions/75773/how-to-reference-problems-by-the-text-label-in-an-exercise-envioronment
\usepackage[english]{cleveref}
\crefname{Exercise}{exercise}{exercises}
\Crefname{Exercise}{Exercise}{Exercises}

\RequirePackage{titlesec}
\RequirePackage{ifthen}

% http://stackoverflow.com/questions/4932910/date-in-the-tabular-environment
\makeatletter
\let\insertdate\@date
\makeatother

\titleformat{\chapter}[display]
{\bfseries\Large}
{\color{DarkSlateGrey}\filleft \authorname
\ifthenelse{\isundefined{\studentnumber}}{}{\\ \studentnumber}
\ifthenelse{\isundefined{\email}}{}{\\ \email}
\ifthenelse{\isundefined{\dateintitle}}{}{\\ \insertdate}
%\ifthenelse{\isundefined{\coursename}}{}{\\ \coursename} % put in title instead.
}
{4ex}
{\color{DarkOliveGreen}{\titlerule}\color{Maroon}
\vspace{2ex}%
\filright}
[\vspace{2ex}%
\color{DarkOliveGreen}\titlerule
]

\newcommand{\beginArtWithToc}[0]{\begin{document}\tableofcontents}
\newcommand{\beginArtNoToc}[0]{\begin{document}}
\newcommand{\EndNoBibArticle}[0]{\end{document}}
\newcommand{\EndArticle}[0]{\bibliography{Bibliography}\bibliographystyle{plainnat}\end{document}}

% 
%\newcommand{\citep}[1]{\cite{#1}}

\colorSectionsForArticle


%
%\usepackage{peeters_layout_exercise}
%\usepackage{peeters_braket}
%\usepackage{peeters_figures}
%
%\beginArtNoToc
%
%\generatetitle{1D SHO linear superposition that maximizes expectation}
%%\chapter{1D SHO linear superposition that maximizes expectation}
%%\label{chap:shoSuperposition}

\makeoproblem{1D SHO linear superposition that maximizes expectation.}{problem:shoSuperposition:1}{\citep{sakurai2014modern} pr. 2.17}{

For a 1D SHO

\makesubproblem{}{problem:shoSuperposition:1:a}

Construct a linear combination of \( \ket{0}, \ket{1} \) that maximizes \( \expectation{x} \) without using wave functions.

\makesubproblem{}{problem:shoSuperposition:1:b}

How does this state evolve with time?

\makesubproblem{}{problem:shoSuperposition:1:c}

Evaluate \( \expectation{x} \) using the Schr\"{o}dinger picture.

\makesubproblem{}{problem:shoSuperposition:1:d}

Evaluate \( \expectation{x} \) using the Heisenberg picture.

\makesubproblem{}{problem:shoSuperposition:1:e}

Evaluate \( \expectation{(\Delta x)^2} \).

} % problem

\makeanswer{problem:shoSuperposition:1}{

\makeSubAnswer{}{problem:shoSuperposition:1:a}

Forming

\begin{dmath}\label{eqn:shoSuperposition:20}
\ket{\psi} = \frac{\ket{0} + \sigma \ket{1}}{\sqrt{1 + \Abs{\sigma}^2}}
\end{dmath}

the position expectation is

\begin{dmath}\label{eqn:shoSuperposition:40}
\bra{\psi} x \ket{\psi}
=
\inv{1 + \Abs{\sigma}^2} \lr{ \bra{0} + \sigma^\conj \bra{1} } \frac{x_0}{\sqrt{2}} \lr{ a^\dagger + a } \lr{ \ket{0} + \sigma \ket{1} }.
\end{dmath}

Evaluating the action of the operators on the kets, we've got

\begin{dmath}\label{eqn:shoSuperposition:60}
\lr{ a^\dagger + a } \lr{ \ket{0} + \sigma \ket{1} }
=
\ket{1} + \sqrt{2} \sigma \ket{2} + \sigma \ket{0}.
\end{dmath}

The \( \ket{2} \) term is killed by the bras, leaving

\begin{dmath}\label{eqn:shoSuperposition:80}
\expectation{x}
=
\inv{1 + \Abs{\sigma}^2} \frac{x_0}{\sqrt{2}} \lr{ \sigma + \sigma^\conj}
=
\frac{\sqrt{2} x_0 \Real \sigma}{1 + \Abs{\sigma}^2}.
\end{dmath}

Any imaginary component in \( \sigma \) will reduce the expectation, so we are constrained to picking a real value.

The derivative of

\begin{dmath}\label{eqn:shoSuperposition:100}
f(\sigma) = \frac{\sigma}{1 + \sigma^2},
\end{dmath}

is

\begin{dmath}\label{eqn:shoSuperposition:120}
f'(\sigma) = \frac{1 - \sigma^2}{(1 + \sigma^2)^2}.
\end{dmath}

That has zeros at \( \sigma = \pm 1 \).  The second derivative is

\begin{dmath}\label{eqn:shoSuperposition:140}
f''(\sigma) = \frac{-2 \sigma (3 - \sigma^2)}{(1 + \sigma^2)^3}.
\end{dmath}

That will be negative (maximum for the extreme value) at \( \sigma = 1 \), so the linear superposition of these first two energy eigenkets that maximizes the position expectation is

\begin{dmath}\label{eqn:shoSuperposition:160}
\psi = \inv{\sqrt{2}}\lr{ \ket{0} + \ket{1} }.
\end{dmath}

That maximized position expectation is

\begin{dmath}\label{eqn:shoSuperposition:180}
\expectation{x}
=
\frac{x_0}{\sqrt{2}}.
\end{dmath}

\makeSubAnswer{}{problem:shoSuperposition:1:b}

The time evolution is given by

\begin{dmath}\label{eqn:shoSuperposition:200}
\ket{\Psi(t)}
= e^{-i H t/\Hbar} \inv{\sqrt{2}}\lr{ \ket{0} + \ket{1} }
= \inv{\sqrt{2}}\lr{ e^{-i(0+ \ifrac{1}{2})\Hbar \omega t/\Hbar} \ket{0} + e^{-i(1+ \ifrac{1}{2})\Hbar \omega t/\Hbar} \ket{1} }
= \inv{\sqrt{2}}\lr{ e^{-i \omega t/2} \ket{0} + e^{-3 i \omega t/2} \ket{1} }.
\end{dmath}

\makeSubAnswer{}{problem:shoSuperposition:1:c}

The position expectation in the Schr\"{o}dinger representation is

\begin{dmath}\label{eqn:shoSuperposition:220}
\expectation{x(t)}
=
\inv{2}
\lr{ e^{i \omega t/2} \bra{0} + e^{3 i \omega t/2} \bra{1} } \frac{x_0}{\sqrt{2}} \lr{ a^\dagger + a }
\lr{ e^{-i \omega t/2} \ket{0} + e^{-3 i \omega t/2} \ket{1} }
=
\frac{x_0}{2\sqrt{2}}
\lr{ e^{i \omega t/2} \bra{0} + e^{3 i \omega t/2} \bra{1} }
\lr{ e^{-i \omega t/2} \ket{1} + e^{-3 i \omega t/2} \sqrt{2} \ket{2} + e^{-3 i \omega t/2} \ket{0} }
=
\frac{x_0}{\sqrt{2}} \cos(\omega t).
\end{dmath}

\makeSubAnswer{}{problem:shoSuperposition:1:d}

\begin{dmath}\label{eqn:shoSuperposition:240}
\expectation{x(t)}
=
\inv{2}
\lr{ \bra{0} + \bra{1} } \frac{x_0}{\sqrt{2}}
\lr{ a^\dagger e^{i\omega t} + a e^{-i \omega t} }
\lr{ \ket{0} + \ket{1} }
=
\frac{x_0}{2 \sqrt{2}}
\lr{ \bra{0} + \bra{1} }
\lr{ e^{i\omega t} \ket{1} + \sqrt{2} e^{i\omega t} \ket{2} + e^{-i \omega t} \ket{0} }
=
\frac{x_0}{\sqrt{2}} \cos(\omega t),
\end{dmath}

matching the calculation using the Schr\"{o}dinger picture.

\makeSubAnswer{}{problem:shoSuperposition:1:e}

Let's use the Heisenberg picture for the uncertainty calculation.  Using the calculation above we have

\begin{dmath}\label{eqn:shoSuperposition:260}
\expectation{x^2}
=
\inv{2} \frac{x_0^2}{2}
\lr{ e^{-i\omega t} \bra{1} + \sqrt{2} e^{-i\omega t} \bra{2} + e^{i \omega t} \bra{0} }
\lr{ e^{i\omega t} \ket{1} + \sqrt{2} e^{i\omega t} \ket{2} + e^{-i \omega t} \ket{0} }
=
\frac{x_0^2}{4} \lr{ 1 + 2 + 1}
=
x_0^2.
\end{dmath}

The uncertainty is
\begin{dmath}\label{eqn:shoSuperposition:280}
\expectation{(\Delta x)^2} =
\expectation{x^2} - \expectation{x}^2
=
x_0^2 - \frac{x_0^2}{2} \cos^2(\omega t)
=
\frac{x_0^2}{2} \lr{ 2 - \cos^2(\omega t) }
=
\frac{x_0^2}{2} \lr{ 1 + \sin^2(\omega t) }
\end{dmath}
} % answer

%\EndArticle

         % p.18:
         %
% Copyright � 2015 Peeter Joot.  All Rights Reserved.
% Licenced as described in the file LICENSE under the root directory of this GIT repository.
%
%\newcommand{\authorname}{Peeter Joot}
\newcommand{\email}{peeterjoot@protonmail.com}
\newcommand{\basename}{FIXMEbasenameUndefined}
\newcommand{\dirname}{notes/FIXMEdirnameUndefined/}

%\renewcommand{\basename}{exponentialExpectationGroundState}
%\renewcommand{\dirname}{notes/phy1520/}
%%\newcommand{\dateintitle}{}
%%\newcommand{\keywords}{}
%
%\newcommand{\authorname}{Peeter Joot}
\newcommand{\onlineurl}{http://sites.google.com/site/peeterjoot2/math2013/\basename.pdf}
\newcommand{\sourcepath}{\dirname\basename.tex}
\newcommand{\generatetitle}[1]{\chapter{#1}}

\newcommand{\vcsinfo}{%
\section*{}
\noindent{\color{DarkOliveGreen}{\rule{\linewidth}{0.1mm}}}
\paragraph{Document version}
%\paragraph{\color{Maroon}{Document version}}
{
\small
\begin{itemize}
\item Available online at:\\ 
\href{\onlineurl}{\onlineurl}
\item Git Repository: \input{./.revinfo/gitRepo.tex}
\item Source: \sourcepath
\item last commit: \input{./.revinfo/gitCommitString.tex}
\item commit date: \input{./.revinfo/gitCommitDate.tex}
\end{itemize}
}
}

%\PassOptionsToPackage{dvipsnames,svgnames}{xcolor}
\PassOptionsToPackage{square,numbers}{natbib}
\documentclass{scrreprt}

\usepackage[left=2cm,right=2cm]{geometry}
\usepackage[svgnames]{xcolor}
\usepackage{peeters_layout}

\usepackage{natbib}

\usepackage[
colorlinks=true,
bookmarks=false,
pdfauthor={\authorname, \email},
backref 
]{hyperref}

% http://tex.stackexchange.com/questions/75773/how-to-reference-problems-by-the-text-label-in-an-exercise-envioronment
\usepackage[english]{cleveref}
\crefname{Exercise}{exercise}{exercises}
\Crefname{Exercise}{Exercise}{Exercises}

\RequirePackage{titlesec}
\RequirePackage{ifthen}

% http://stackoverflow.com/questions/4932910/date-in-the-tabular-environment
\makeatletter
\let\insertdate\@date
\makeatother

\titleformat{\chapter}[display]
{\bfseries\Large}
{\color{DarkSlateGrey}\filleft \authorname
\ifthenelse{\isundefined{\studentnumber}}{}{\\ \studentnumber}
\ifthenelse{\isundefined{\email}}{}{\\ \email}
\ifthenelse{\isundefined{\dateintitle}}{}{\\ \insertdate}
%\ifthenelse{\isundefined{\coursename}}{}{\\ \coursename} % put in title instead.
}
{4ex}
{\color{DarkOliveGreen}{\titlerule}\color{Maroon}
\vspace{2ex}%
\filright}
[\vspace{2ex}%
\color{DarkOliveGreen}\titlerule
]

\newcommand{\beginArtWithToc}[0]{\begin{document}\tableofcontents}
\newcommand{\beginArtNoToc}[0]{\begin{document}}
\newcommand{\EndNoBibArticle}[0]{\end{document}}
\newcommand{\EndArticle}[0]{\bibliography{Bibliography}\bibliographystyle{plainnat}\end{document}}

% 
%\newcommand{\citep}[1]{\cite{#1}}

\colorSectionsForArticle


%
%\usepackage{peeters_layout_exercise}
%\usepackage{peeters_braket}
%\usepackage{peeters_figures}
%
%\beginArtNoToc
%
%\generatetitle{Plane wave ground state expectation}
%%\chapter{Plane wave ground state expectation}
%%\label{chap:exponentialExpectationGroundState}

\makeoproblem{Plane wave ground state expectation for 1D SHO.}{problem:exponentialExpectationGroundState:1}{\citep{sakurai2014modern} pr. 2.18}{
\index{harmonic oscillator!ground state}

For a 1D SHO, show that

\begin{dmath}\label{eqn:exponentialExpectationGroundState:20}
\bra{0} e^{i k x} \ket{0} = \exp\lr{ -k^2 \bra{0} x^2 \ket{0}/2 }.
\end{dmath}

} % problem

\makeanswer{problem:exponentialExpectationGroundState:1}{
Despite the simple appearance of this problem, I found this quite involved to show.  To do so, start with a series expansion of the expectation

\begin{dmath}\label{eqn:exponentialExpectationGroundState:40}
\bra{0} e^{i k x} \ket{0}
=
\sum_{m=0}^\infty \frac{(i k)^m}{m!} \bra{0} x^m \ket{0}.
\end{dmath}

Let

\begin{dmath}\label{eqn:exponentialExpectationGroundState:60}
X = \lr{ a + a^\dagger },
\end{dmath}

so that

\begin{equation}\label{eqn:exponentialExpectationGroundState:80}
x
= \sqrt{\frac{\Hbar}{2 \omega m}} X
= \frac{x_0}{\sqrt{2}} X.
\end{equation}

Consider the first few values of \( \bra{0} X^n \ket{0} \)

\begin{dmath}\label{eqn:exponentialExpectationGroundState:100}
\bra{0} X \ket{0}
=
\bra{0} \lr{ a + a^\dagger } \ket{0}
=
\braket{0}{1}
=
0,
\end{dmath}

\begin{dmath}\label{eqn:exponentialExpectationGroundState:120}
\bra{0} X^2 \ket{0}
=
\bra{0} \lr{ a + a^\dagger }^2 \ket{0}
=
\braket{1}{1}
=
1,
\end{dmath}

\begin{dmath}\label{eqn:exponentialExpectationGroundState:140}
\bra{0} X^3 \ket{0}
=
\bra{0} \lr{ a + a^\dagger }^3 \ket{0}
=
\bra{1} \lr{ \sqrt{2} \ket{2} + \ket{0} }
=
0.
\end{dmath}

Whenever the power \( n \) in \( X^n \) is even, the braket can be split into a bra that has only contributions from odd eigenstates and a ket with even eigenstates.  We conclude that \( \bra{0} X^n \ket{0} = 0 \) when \( n \) is odd.

Noting that \( \bra{0} x^2 \ket{0} = \ifrac{x_0^2}{2} \), this leaves

\begin{dmath}\label{eqn:exponentialExpectationGroundState:160}
\bra{0} e^{i k x} \ket{0}
=
\sum_{m=0}^\infty \frac{(i k)^{2 m}}{(2 m)!} \bra{0} x^{2m} \ket{0}
=
\sum_{m=0}^\infty \frac{(i k)^{2 m}}{(2 m)!} \lr{ \frac{x_0^2}{2} }^m \bra{0} X^{2m} \ket{0}
=
\sum_{m=0}^\infty \frac{1}{(2 m)!} \lr{ -k^2 \bra{0} x^2 \ket{0} }^m \bra{0} X^{2m} \ket{0}.
\end{dmath}

This problem is now reduced to showing that

\begin{dmath}\label{eqn:exponentialExpectationGroundState:180}
\frac{1}{(2 m)!} \bra{0} X^{2m} \ket{0} = \inv{m! 2^m},
\end{dmath}

or

\begin{dmath}\label{eqn:exponentialExpectationGroundState:200}
\bra{0} X^{2m} \ket{0}
= \frac{(2m)!}{m! 2^m}
= \frac{ (2m)(2m-1)(2m-2) \cdots (2)(1) }{2^m m!}
= \frac{ 2^m (m)(2m-1)(m-1)(2m-3)(m-2) \cdots (2)(3)(1)(1) }{2^m m!}
= (2m-1)!!,
\end{dmath}

where \( n!! = n(n-2)(n-4)\cdots \).

It looks like \( \bra{0} X^{2m} \ket{0} \) can be expanded by inserting an identity operator and proceeding recursively, like

\begin{dmath}\label{eqn:exponentialExpectationGroundState:220}
\bra{0} X^{2m} \ket{0}
=
\bra{0} X^2 \lr{ \sum_{n=0}^\infty \ket{n}\bra{n} } X^{2m-2} \ket{0}
=
\bra{0} X^2 \lr{ \ket{0}\bra{0} + \ket{2}\bra{2} } X^{2m-2} \ket{0}
=
\bra{0} X^{2m-2} \ket{0} + \bra{0} X^2 \ket{2} \bra{2} X^{2m-2} \ket{0}.
\end{dmath}

This has made use of the observation that \( \bra{0} X^2 \ket{n} = 0 \) for all \( n \ne 0,2 \).  The remaining term includes the factor

\begin{dmath}\label{eqn:exponentialExpectationGroundState:240}
\bra{0} X^2 \ket{2}
=
\bra{0} \lr{a + a^\dagger}^2 \ket{2}
=
\lr{ \bra{0} + \sqrt{2} \bra{2} } \ket{2}
=
\sqrt{2},
\end{dmath}

Since \( \sqrt{2} \ket{2} = \lr{a^\dagger}^2 \ket{0} \), the expectation of interest can be written

\begin{dmath}\label{eqn:exponentialExpectationGroundState:260}
\bra{0} X^{2m} \ket{0}
=
\bra{0} X^{2m-2} \ket{0} + \bra{0} a^2 X^{2m-2} \ket{0}.
\end{dmath}

How do we expand the second term.  Let's look at how \( a \) and \( X \) commute

\begin{dmath}\label{eqn:exponentialExpectationGroundState:280}
a X
=
\antisymmetric{a}{X} + X a
=
\antisymmetric{a}{a + a^\dagger} + X a
=
\antisymmetric{a}{a^\dagger} + X a
=
1 + X a,
\end{dmath}

\begin{dmath}\label{eqn:exponentialExpectationGroundState:300}
a^2 X
=
a \lr{ a X }
=
a \lr{ 1 + X a }
=
a + a X a
=
a + \lr{ 1 + X a } a
=
2 a + X a^2.
\end{dmath}

Proceeding to expand \( a^2 X^n \) we find
\begin{equation}\label{eqn:exponentialExpectationGroundState:320}
\begin{aligned}
a^2 X^3 &= 6 X + 6 X^2 a + X^3 a^2 \\
a^2 X^4 &= 12 X^2 + 8 X^3 a + X^4 a^2 \\
a^2 X^5 &= 20 X^3 + 10 X^4 a + X^5 a^2 \\
a^2 X^6 &= 30 X^4 + 12 X^5 a + X^6 a^2.
\end{aligned}
\end{equation}

It appears that we have
\begin{equation}\label{eqn:exponentialExpectationGroundState:340}
\antisymmetric{a^2 X^n}{X^n a^2} = \beta_n X^{n-2} + 2 n X^{n-1} a,
\end{equation}

where

\begin{equation}\label{eqn:exponentialExpectationGroundState:360}
\beta_n = \beta_{n-1} + 2 (n-1),
\end{equation}

and \( \beta_2 = 2 \).  Some goofing around shows that \( \beta_n = n(n-1) \), so the induction hypothesis is

\begin{equation}\label{eqn:exponentialExpectationGroundState:380}
\antisymmetric{a^2 X^n}{X^n a^2} = n(n-1) X^{n-2} + 2 n X^{n-1} a.
\end{equation}

Let's check the induction
\begin{dmath}\label{eqn:exponentialExpectationGroundState:400}
a^2 X^{n+1}
=
a^2 X^{n} X
=
\lr{ n(n-1) X^{n-2} + 2 n X^{n-1} a + X^n a^2 } X
=
n(n-1) X^{n-1} + 2 n X^{n-1} a X + X^n a^2 X
=
n(n-1) X^{n-1} + 2 n X^{n-1} \lr{ 1 + X a } + X^n \lr{ 2 a + X a^2 }
=
n(n-1) X^{n-1} + 2 n X^{n-1}  + 2 n X^{n} a
+ 2 X^n a
+ X^{n+1} a^2
=
X^{n+1} a^2 + (2 + 2 n) X^{n} a + \lr{ 2 n + n(n-1) }  X^{n-1}
=
X^{n+1} a^2 + 2(n + 1) X^{n} a + (n+1) n X^{n-1},
\end{dmath}

which concludes the induction, giving

\begin{dmath}\label{eqn:exponentialExpectationGroundState:420}
\bra{ 0 } a^2 X^{n} \ket{0 } = n(n-1) \bra{0} X^{n-2} \ket{0},
\end{dmath}

and

\begin{dmath}\label{eqn:exponentialExpectationGroundState:440}
\bra{0} X^{2m} \ket{0}
=
\bra{0} X^{2m-2} \ket{0} + (2m-2)(2m-3) \bra{0} X^{2m-4} \ket{0}.
\end{dmath}

Let

\begin{dmath}\label{eqn:exponentialExpectationGroundState:460}
\sigma_{n} = \bra{0} X^n \ket{0},
\end{dmath}

so that the recurrence relation, for \( 2n \ge 4 \) is

\begin{dmath}\label{eqn:exponentialExpectationGroundState:480}
\sigma_{2n} = \sigma_{2n -2} + (2n-2)(2n-3) \sigma_{2n -4}
\end{dmath}

We want to show that this simplifies to

\begin{dmath}\label{eqn:exponentialExpectationGroundState:500}
\sigma_{2n} = (2n-1)!!
\end{dmath}

The first values are

\begin{subequations}
\label{eqn:exponentialExpectationGroundState:520}
\begin{equation}\label{eqn:exponentialExpectationGroundState:540}
\sigma_0 = \bra{0} X^0 \ket{0} = 1
\end{equation}
\begin{equation}\label{eqn:exponentialExpectationGroundState:560}
\sigma_2 = \bra{0} X^2 \ket{0} = 1
\end{equation}
\end{subequations}

which gives us the right result for the first term in the induction

\begin{dmath}\label{eqn:exponentialExpectationGroundState:580}
\sigma_4
= \sigma_2 + 2 \times 1 \times \sigma_0
= 1 + 2
= 3!!
\end{dmath}

For the general induction term, consider

\begin{dmath}\label{eqn:exponentialExpectationGroundState:600}
\sigma_{2n + 2}
= \sigma_{2n} + 2 n (2n - 1) \sigma_{2n -2}
= (2n-1)!! + 2n ( 2n - 1) (2n -3)!!
= (2n + 1) (2n -1)!!
= (2n + 1)!!,
\end{dmath}

which completes the final induction.  That was also the last thing required to complete the proof, so we are done!
} % answer

%\EndArticle

         % made up problem:
         %
% Copyright � 2015 Peeter Joot.  All Rights Reserved.
% Licenced as described in the file LICENSE under the root directory of this GIT repository.
%
%\newcommand{\authorname}{Peeter Joot}
\newcommand{\email}{peeterjoot@protonmail.com}
\newcommand{\basename}{FIXMEbasenameUndefined}
\newcommand{\dirname}{notes/FIXMEdirnameUndefined/}

%\renewcommand{\basename}{fluxAndMomentum}
%\renewcommand{\dirname}{notes/phy1520/}
%\newcommand{\dateintitle}{}
%\newcommand{\keywords}{}

%\newcommand{\authorname}{Peeter Joot}
\newcommand{\onlineurl}{http://sites.google.com/site/peeterjoot2/math2013/\basename.pdf}
\newcommand{\sourcepath}{\dirname\basename.tex}
\newcommand{\generatetitle}[1]{\chapter{#1}}

\newcommand{\vcsinfo}{%
\section*{}
\noindent{\color{DarkOliveGreen}{\rule{\linewidth}{0.1mm}}}
\paragraph{Document version}
%\paragraph{\color{Maroon}{Document version}}
{
\small
\begin{itemize}
\item Available online at:\\ 
\href{\onlineurl}{\onlineurl}
\item Git Repository: \input{./.revinfo/gitRepo.tex}
\item Source: \sourcepath
\item last commit: \input{./.revinfo/gitCommitString.tex}
\item commit date: \input{./.revinfo/gitCommitDate.tex}
\end{itemize}
}
}

%\PassOptionsToPackage{dvipsnames,svgnames}{xcolor}
\PassOptionsToPackage{square,numbers}{natbib}
\documentclass{scrreprt}

\usepackage[left=2cm,right=2cm]{geometry}
\usepackage[svgnames]{xcolor}
\usepackage{peeters_layout}

\usepackage{natbib}

\usepackage[
colorlinks=true,
bookmarks=false,
pdfauthor={\authorname, \email},
backref 
]{hyperref}

% http://tex.stackexchange.com/questions/75773/how-to-reference-problems-by-the-text-label-in-an-exercise-envioronment
\usepackage[english]{cleveref}
\crefname{Exercise}{exercise}{exercises}
\Crefname{Exercise}{Exercise}{Exercises}

\RequirePackage{titlesec}
\RequirePackage{ifthen}

% http://stackoverflow.com/questions/4932910/date-in-the-tabular-environment
\makeatletter
\let\insertdate\@date
\makeatother

\titleformat{\chapter}[display]
{\bfseries\Large}
{\color{DarkSlateGrey}\filleft \authorname
\ifthenelse{\isundefined{\studentnumber}}{}{\\ \studentnumber}
\ifthenelse{\isundefined{\email}}{}{\\ \email}
\ifthenelse{\isundefined{\dateintitle}}{}{\\ \insertdate}
%\ifthenelse{\isundefined{\coursename}}{}{\\ \coursename} % put in title instead.
}
{4ex}
{\color{DarkOliveGreen}{\titlerule}\color{Maroon}
\vspace{2ex}%
\filright}
[\vspace{2ex}%
\color{DarkOliveGreen}\titlerule
]

\newcommand{\beginArtWithToc}[0]{\begin{document}\tableofcontents}
\newcommand{\beginArtNoToc}[0]{\begin{document}}
\newcommand{\EndNoBibArticle}[0]{\end{document}}
\newcommand{\EndArticle}[0]{\bibliography{Bibliography}\bibliographystyle{plainnat}\end{document}}

% 
%\newcommand{\citep}[1]{\cite{#1}}

\colorSectionsForArticle



%\usepackage{peeters_layout_exercise}
%\usepackage{peeters_braket}
%\usepackage{peeters_figures}
%
%\beginArtNoToc

%\generatetitle{Relation of probability flux to momentum}
%\chapter{Relation of probability flux to momentum}
%\label{chap:fluxAndMomentum}

\makeproblem{Relation of probability flux to momentum.}{problem:fluxAndMomentum:1}{
\index{probability!flux}
\index{momentum!expectation}

Show that the probability flux

\begin{dmath}\label{eqn:fluxAndMomentum:20}
\Bj(\Bx, t) = -\frac{i\Hbar}{2 m} \lr{ \psi^\conj \spacegrad \psi - \psi \spacegrad \psi^\conj },
\end{dmath}

is related to the momentum expectation at a given time by the integral of the flux over all space

\begin{dmath}\label{eqn:fluxAndMomentum:40}
\int d^3 x \Bj(\Bx, t) = \frac{\expectation{\Bp}_t}{m}.
\end{dmath}

} % problem

\makeanswer{problem:fluxAndMomentum:1}{

This can be seen by recasting the integral in bra-ket form.  Let

\begin{dmath}\label{eqn:fluxAndMomentum:60}
\psi(\Bx, t) = \braket{\Bx}{\psi(t)},
\end{dmath}

and note that the momentum portions of the flux can be written as

\begin{dmath}\label{eqn:fluxAndMomentum:80}
-i \Hbar \spacegrad \psi(\Bx, t) = \bra{\Bx} \Bp \ket{\psi(t)}.
\end{dmath}

The current is therefore

\begin{dmath}\label{eqn:fluxAndMomentum:100}
\Bj(\Bx, t)
= \frac{1}{2 m}
\lr{
\psi^\conj \bra{\Bx} \Bp \ket{\psi(t)}
+\psi {\bra{\Bx} \Bp \ket{\psi(t)} }^\conj
}
= \frac{1}{2 m}
\lr{
{\braket{\Bx}{\psi(t)}}^\conj \bra{\Bx} \Bp \ket{\psi(t)}
+ \braket{\Bx}{\psi(t)} {\bra{\Bx} \Bp \ket{\psi(t)} }^\conj
}
= \frac{1}{2 m}
\lr{
\braket{\psi(t)}{\Bx} \bra{\Bx} \Bp \ket{\psi(t)}
+
\bra{\psi(t)} \Bp \ket{\Bx} \braket{\Bx}{\psi(t)}
}.
\end{dmath}

Integrating and noting that the spatial identity is \( 1 = \int d^3 x \ket{\Bx}\bra{\Bx} \), we have

\begin{dmath}\label{eqn:fluxAndMomentum:n}
\int d^3 x \Bj(\Bx, t)
=
\bra{\psi(t)} \Bp \ket{\psi(t)},
\end{dmath}

This is just the expectation of \( \Bp \) with respect to a specific time-instance state, demonstrating the desired relationship.
} % answer

%\EndArticle

         % p21:
         %
% Copyright � 2015 Peeter Joot.  All Rights Reserved.
% Licenced as described in the file LICENSE under the root directory of this GIT repository.
%
%\newcommand{\authorname}{Peeter Joot}
\newcommand{\email}{peeterjoot@protonmail.com}
\newcommand{\basename}{FIXMEbasenameUndefined}
\newcommand{\dirname}{notes/FIXMEdirnameUndefined/}

%\renewcommand{\basename}{hermiteOrtho}
%\renewcommand{\dirname}{notes/phy1520/}
%%\newcommand{\dateintitle}{}
%%\newcommand{\keywords}{}
%
%\newcommand{\authorname}{Peeter Joot}
\newcommand{\onlineurl}{http://sites.google.com/site/peeterjoot2/math2013/\basename.pdf}
\newcommand{\sourcepath}{\dirname\basename.tex}
\newcommand{\generatetitle}[1]{\chapter{#1}}

\newcommand{\vcsinfo}{%
\section*{}
\noindent{\color{DarkOliveGreen}{\rule{\linewidth}{0.1mm}}}
\paragraph{Document version}
%\paragraph{\color{Maroon}{Document version}}
{
\small
\begin{itemize}
\item Available online at:\\ 
\href{\onlineurl}{\onlineurl}
\item Git Repository: \input{./.revinfo/gitRepo.tex}
\item Source: \sourcepath
\item last commit: \input{./.revinfo/gitCommitString.tex}
\item commit date: \input{./.revinfo/gitCommitDate.tex}
\end{itemize}
}
}

%\PassOptionsToPackage{dvipsnames,svgnames}{xcolor}
\PassOptionsToPackage{square,numbers}{natbib}
\documentclass{scrreprt}

\usepackage[left=2cm,right=2cm]{geometry}
\usepackage[svgnames]{xcolor}
\usepackage{peeters_layout}

\usepackage{natbib}

\usepackage[
colorlinks=true,
bookmarks=false,
pdfauthor={\authorname, \email},
backref 
]{hyperref}

% http://tex.stackexchange.com/questions/75773/how-to-reference-problems-by-the-text-label-in-an-exercise-envioronment
\usepackage[english]{cleveref}
\crefname{Exercise}{exercise}{exercises}
\Crefname{Exercise}{Exercise}{Exercises}

\RequirePackage{titlesec}
\RequirePackage{ifthen}

% http://stackoverflow.com/questions/4932910/date-in-the-tabular-environment
\makeatletter
\let\insertdate\@date
\makeatother

\titleformat{\chapter}[display]
{\bfseries\Large}
{\color{DarkSlateGrey}\filleft \authorname
\ifthenelse{\isundefined{\studentnumber}}{}{\\ \studentnumber}
\ifthenelse{\isundefined{\email}}{}{\\ \email}
\ifthenelse{\isundefined{\dateintitle}}{}{\\ \insertdate}
%\ifthenelse{\isundefined{\coursename}}{}{\\ \coursename} % put in title instead.
}
{4ex}
{\color{DarkOliveGreen}{\titlerule}\color{Maroon}
\vspace{2ex}%
\filright}
[\vspace{2ex}%
\color{DarkOliveGreen}\titlerule
]

\newcommand{\beginArtWithToc}[0]{\begin{document}\tableofcontents}
\newcommand{\beginArtNoToc}[0]{\begin{document}}
\newcommand{\EndNoBibArticle}[0]{\end{document}}
\newcommand{\EndArticle}[0]{\bibliography{Bibliography}\bibliographystyle{plainnat}\end{document}}

% 
%\newcommand{\citep}[1]{\cite{#1}}

\colorSectionsForArticle


%
%\usepackage{peeters_layout_exercise}
%\usepackage{peeters_braket}
%\usepackage{peeters_figures}
%
%\beginArtNoToc
%
%\generatetitle{Hermite polynomial normalization constant}
%\chapter{Hermite polynomial normalization constant}
%\label{chap:hermiteOrtho}

\makeoproblem{Hermite polynomial normalization constant.}{problem:hermiteOrtho:2.21}{\citep{sakurai2014modern} pr. 2.21}{
\index{Hermite polynomial}
Derive the normalization constant \( c_n \) for the Harmonic oscillator solution

\begin{dmath}\label{eqn:hermiteOrtho:20}
u_n(x) = c_n H_n\lr{ x \sqrt{\frac{m\omega}{\Hbar}} } e^{-m \omega x^2/2 \Hbar},
\end{dmath}

by deriving the orthogonality relationship using generating functions

\begin{equation}\label{eqn:hermiteOrtho:40}
g(x,t) = e^{-t^2 + 2 t x} = \sum_{n=0}^\infty H_n(x) \frac{t^n}{n!}.
\end{equation}

Start by working out the integral

\begin{equation}\label{eqn:hermiteOrtho:60}
I = \int_{-\infty}^\infty g(x, t) g(x, s) e^{-x^2} dx,
\end{equation}

consider the integral twice with each side definition of the generating function.

} % problem

\makeanswer{problem:hermiteOrtho:2.21}{

First using the exponential definition of the generating function

\begin{dmath}\label{eqn:hermiteOrtho:80}
\int_{-\infty}^\infty g(x, t) g(x, s) e^{-x^2} dx
=
\int_{-\infty}^\infty
e^{-t^2 + 2 t x}
e^{-s^2 + 2 s x} e^{-x^2} dx
=
e^{-t^2 -s^2}
\int_{-\infty}^\infty
e^{-(x^2- 2 t x - 2 s x)} dx
=
e^{-t^2 -s^2 + (s + t)^2}
\int_{-\infty}^\infty
e^{-(x - t - s)^2} dx
=
e^{2 st}
\int_{-\infty}^\infty
e^{-u^2} du
= \sqrt{\pi} e^{2 st}.
\end{dmath}

With the Hermite polynomial definition of the generating function, this integral is

\begin{dmath}\label{eqn:hermiteOrtho:100}
\int_{-\infty}^\infty g(x, t) g(x, s) e^{-x^2} dx
=
\int_{-\infty}^\infty
\sum_{n=0}^\infty H_n(x) \frac{t^n}{n!}
\sum_{m=0}^\infty H_m(x) \frac{s^m}{m!}
e^{-x^2} dx
=
\sum_{n=0}^\infty \frac{t^n}{n!}
\sum_{m=0}^\infty \frac{s^m}{m!}
\int_{-\infty}^\infty H_n(x) H_m(x) e^{-x^2} dx.
\end{dmath}

Let

\begin{dmath}\label{eqn:hermiteOrtho:120}
\alpha_{n m} = \int_{-\infty}^\infty H_n(x) H_m(x) e^{-x^2} dx,
\end{dmath}

and equate the two expansions of this integral

\begin{dmath}\label{eqn:hermiteOrtho:140}
\sqrt{\pi} \sum_{n=0}^\infty \frac{(2st)^n}{n!}
=
\sum_{n=0}^\infty \frac{t^n}{n!}
\sum_{m=0}^\infty \frac{s^m}{m!}
\alpha_{n m},
\end{dmath}

or, after equating powers of \( t^n \)

\begin{dmath}\label{eqn:hermiteOrtho:160}
\sqrt{\pi} (2 s)^n =
\sum_{m=0}^\infty \frac{s^m}{m!} \alpha_{n m}.
\end{dmath}

This requires \( \alpha_{n m} \) to be zero for \( n \ne m \), so

\begin{dmath}\label{eqn:hermiteOrtho:180}
\sqrt{\pi} 2^n = \frac{1}{n!} \alpha_{n n},
\end{dmath}

and

\begin{dmath}\label{eqn:hermiteOrtho:200}
\int_{-\infty}^\infty H_n(x) H_m(x) e^{-x^2} dx = \delta_{n m} \sqrt{\pi} 2^n n!.
\end{dmath}

The SHO normalization is fixed by

\begin{dmath}\label{eqn:hermiteOrtho:220}
\int_{-\infty}^\infty u_n^2(x) dx
= c_n^2
\int_{-\infty}^\infty H_n^2(x/x_0) e^{-(x/x_0)^2} dx
= c_n^2 x_0 \sqrt{\pi} 2^n n!,
\end{dmath}

or

\begin{dmath}\label{eqn:hermiteOrtho:240}
c_n
= \inv{\sqrt{ \sqrt{\pi} 2^n n! \sqrt{\frac{\Hbar}{m \omega}}}}
= \lr{ \frac{m \omega}{\Hbar \pi} }^{1/4} 2^{-n/2} \inv{\sqrt{n!}}
\end{dmath}

} % answer

%\EndArticle

         % p31
         %
% Copyright � 2015 Peeter Joot.  All Rights Reserved.
% Licenced as described in the file LICENSE under the root directory of this GIT repository.
%
%\newcommand{\authorname}{Peeter Joot}
\newcommand{\email}{peeterjoot@protonmail.com}
\newcommand{\basename}{FIXMEbasenameUndefined}
\newcommand{\dirname}{notes/FIXMEdirnameUndefined/}

%\renewcommand{\basename}{freeParticlePropagator}
%\renewcommand{\dirname}{notes/phy1520/}
%%\newcommand{\dateintitle}{}
%%\newcommand{\keywords}{}
%
%\newcommand{\authorname}{Peeter Joot}
\newcommand{\onlineurl}{http://sites.google.com/site/peeterjoot2/math2013/\basename.pdf}
\newcommand{\sourcepath}{\dirname\basename.tex}
\newcommand{\generatetitle}[1]{\chapter{#1}}

\newcommand{\vcsinfo}{%
\section*{}
\noindent{\color{DarkOliveGreen}{\rule{\linewidth}{0.1mm}}}
\paragraph{Document version}
%\paragraph{\color{Maroon}{Document version}}
{
\small
\begin{itemize}
\item Available online at:\\ 
\href{\onlineurl}{\onlineurl}
\item Git Repository: \input{./.revinfo/gitRepo.tex}
\item Source: \sourcepath
\item last commit: \input{./.revinfo/gitCommitString.tex}
\item commit date: \input{./.revinfo/gitCommitDate.tex}
\end{itemize}
}
}

%\PassOptionsToPackage{dvipsnames,svgnames}{xcolor}
\PassOptionsToPackage{square,numbers}{natbib}
\documentclass{scrreprt}

\usepackage[left=2cm,right=2cm]{geometry}
\usepackage[svgnames]{xcolor}
\usepackage{peeters_layout}

\usepackage{natbib}

\usepackage[
colorlinks=true,
bookmarks=false,
pdfauthor={\authorname, \email},
backref 
]{hyperref}

% http://tex.stackexchange.com/questions/75773/how-to-reference-problems-by-the-text-label-in-an-exercise-envioronment
\usepackage[english]{cleveref}
\crefname{Exercise}{exercise}{exercises}
\Crefname{Exercise}{Exercise}{Exercises}

\RequirePackage{titlesec}
\RequirePackage{ifthen}

% http://stackoverflow.com/questions/4932910/date-in-the-tabular-environment
\makeatletter
\let\insertdate\@date
\makeatother

\titleformat{\chapter}[display]
{\bfseries\Large}
{\color{DarkSlateGrey}\filleft \authorname
\ifthenelse{\isundefined{\studentnumber}}{}{\\ \studentnumber}
\ifthenelse{\isundefined{\email}}{}{\\ \email}
\ifthenelse{\isundefined{\dateintitle}}{}{\\ \insertdate}
%\ifthenelse{\isundefined{\coursename}}{}{\\ \coursename} % put in title instead.
}
{4ex}
{\color{DarkOliveGreen}{\titlerule}\color{Maroon}
\vspace{2ex}%
\filright}
[\vspace{2ex}%
\color{DarkOliveGreen}\titlerule
]

\newcommand{\beginArtWithToc}[0]{\begin{document}\tableofcontents}
\newcommand{\beginArtNoToc}[0]{\begin{document}}
\newcommand{\EndNoBibArticle}[0]{\end{document}}
\newcommand{\EndArticle}[0]{\bibliography{Bibliography}\bibliographystyle{plainnat}\end{document}}

% 
%\newcommand{\citep}[1]{\cite{#1}}

\colorSectionsForArticle


%
%\usepackage{peeters_layout_exercise}
%\usepackage{peeters_braket}
%\usepackage{peeters_figures}
%
%\beginArtNoToc
%
%\generatetitle{Free particle propagator}
%%\chapter{Free particle propagator}
%\label{chap:freeParticlePropagator}

\makeoproblem{Free particle propagator.}{problem:freeParticlePropagator:31}{\citep{sakurai2014modern} pr. 2.31}{
\index{propagator!free particle}
Derive the free particle propagator in one and three dimensions.
} % problem

\makeanswer{problem:freeParticlePropagator:31}{

I found the description in the text confusing, so let's start from scratch with the definition of the propagator.  This is the kernel of the spatial convolution integral that encodes time evolution, and can be expressed by expanding a general time state with two sets of identity operators.  Let the position relative state at time \( t \), relative to an initial time \( t_0 \) be given by \( \braket{\Bx}{\alpha, t ; t_0 } \), and expand this in terms of a complete basis of energy eigenstates \( \ket{a'} \) and the time evolution operator

\begin{dmath}\label{eqn:freeParticlePropagator:20}
\begin{aligned}
\braket{\Bx''}{\alpha, t ; t_0 }
&= \bra{\Bx''} U \ket{\alpha, t_0 } \\
&= \bra{\Bx''} e^{-i H (t -t_0)/\Hbar} \ket{\alpha, t_0 } \\
&= \bra{\Bx''} e^{-i H (t -t_0)/\Hbar} \lr{ \sum_{a'} \ket{a'} \bra{a' }} \ket{\alpha, t_0 } \\
&= \bra{\Bx''} \sum_{a'} e^{-i E_{a'} (t -t_0)/\Hbar} \ket{a'} \braket{a' }{\alpha, t_0 } \\
&=
\bra{\Bx''} \sum_{a'} e^{-i E_{a'} (t -t_0)/\Hbar} \ket{a'} \bra{a' }
\lr{ \int d^3 \Bx'
\ket{\Bx'}\bra{\Bx'}
}
\ket{\alpha, t_0 } \\
&=
\int d^3 \Bx'
\lr{
\bra{\Bx''} \sum_{a'} e^{-i E_{a'} (t -t_0)/\Hbar} \ket{a'} \braket{a' }{\Bx'}
}
\braket{\Bx'}{\alpha, t_0 } \\
&=
\int d^3 \Bx' K(\Bx'', t ; \Bx', t_0) \braket{\Bx'}{\alpha, t_0 },
\end{aligned}
\end{dmath}

where

\begin{dmath}\label{eqn:freeParticlePropagator:40}
K(\Bx'', t ; \Bx', t_0) =
\sum_{a'}
\braket{\Bx''}{a'}\braket{a' }{\Bx'}
e^{-i E_{a'} (t -t_0)/\Hbar},
\end{dmath}

the propagator, is the kernel of the convolution integral that takes the state \( \ket{\alpha, t_0} \) to state \( \ket{\alpha, t ; t_0} \).  Evaluating this over the momentum states (where integration and not plain summation is required), we have

\begin{dmath}\label{eqn:freeParticlePropagator:60}
\begin{aligned}
K(\Bx'', t ; \Bx', t_0)
&=
\int d^3 \Bp'
\braket{\Bx''}{\Bp'}\braket{\Bp' }{\Bx'}
e^{-i E_{\Bp'} (t -t_0)/\Hbar} \\
&=
\int d^3 \Bp'
\braket{\Bx''}{\Bp'}\braket{\Bp' }{\Bx'}
\exp\lr{-i \frac{(\Bp')^2 (t -t_0)}{2 m \Hbar}} \\
&=
\int d^3 \Bp'
\frac{e^{i \Bx'' \cdot \Bp'/\Hbar}}{(\sqrt{2 \pi \Hbar})^3}
\frac{e^{-i \Bx' \cdot \Bp'/\Hbar}}{(\sqrt{2 \pi \Hbar})^3}
\exp\lr{-i \frac{(\Bp')^2 (t -t_0)}{2 m \Hbar}} \\
&=
\inv{(2 \pi \Hbar)^3}
\int d^3 \Bp'
e^{i (\Bx'' -\Bx') \cdot \Bp'/\Hbar}
\exp\lr{-i \frac{(\Bp')^2 (t -t_0)}{2 m \Hbar}} \\
&=
\inv{ 2 \pi \Hbar }
\int_{-\infty}^\infty dp_1'
e^{i (x_1'' -x_1') p_1'/\Hbar}
\exp\lr{-i \frac{(p_1')^2 (t -t_0)}{2 m \Hbar}} \times \\
&\quad \inv{ 2 \pi \Hbar }
\int_{-\infty}^\infty dp_2'
e^{i (x_2'' -x_2') p_2'/\Hbar}
\exp\lr{-i \frac{(p_2')^2 (t -t_0)}{2 m \Hbar}} \times \\
&\quad \inv{ 2 \pi \Hbar }
\int_{-\infty}^\infty dp_3'
e^{i (x_3'' -x_3') p_3'/\Hbar}
\exp\lr{-i \frac{(p_3')^2 (t -t_0)}{2 m \Hbar}}
\end{aligned}
\end{dmath}

With \( a = \ifrac{(t -t_0)}{2 m \Hbar} \), each of these three integral factors is of the form

\begin{dmath}\label{eqn:freeParticlePropagator:80}
\begin{aligned}
\inv{ 2 \pi \Hbar }
\int_{-\infty}^\infty dp
e^{i \Delta x p/\Hbar }
\exp\lr{-i a p^2}
&=
\inv{2 \pi \Hbar \sqrt{a}}
\int_{-\infty}^\infty du
e^{i \Delta x u/(\sqrt{a}\Hbar) }
\exp\lr{-i u^2} \\
&=
\inv{2 \pi \Hbar \sqrt{a}}
\int_{-\infty}^\infty du
e^{i \Delta x u/(\sqrt{a} \Hbar) }
\exp\lr{-i (u - \Delta x /(2\sqrt{a}\Hbar))^2 + i(\Delta x/(2\sqrt{a}\Hbar))^2} \\
&=
\inv{2 \pi \Hbar \sqrt{a}}
\exp\lr{ \frac{i(\Delta x)^2 2 m \Hbar}{4 (t -t_0) \Hbar^2} }
\int_{-\infty}^\infty dz
e^{-i z^2} \\
&= \sqrt{ \frac{ -i \pi 2 m \Hbar}{ 4 \pi^2 \Hbar^2 (t -t_0)} }
\exp\lr{ \frac{i(\Delta x)^2 m}{2 (t -t_0) \Hbar} } \\
&= \sqrt{ \frac{ m }{ 2 \pi i \Hbar (t -t_0)} }
\exp\lr{ \frac{i(\Delta x)^2 m}{2 (t -t_0) \Hbar} }.
\end{aligned}
\end{dmath}

Note that the integral above has value \( \sqrt{-i\pi} \) which can be found by integrating over the contour of \cref{fig:contour:contourFig1}, letting \( R \rightarrow \infty \).

\imageFigure{../../figures/phy1520/contourFig1}{Integration contour for \( \int e^{-i z^2} \).}{fig:contour:contourFig1}{0.2}

Multiplying out each of the spatial direction factors gives the propagator in its closed form
\boxedEquation{eqn:freeParticlePropagator:120}{
K(\Bx'', t ; \Bx', t_0)
= \lr{ \sqrt{ \frac{ m }{ 2 \pi i \Hbar (t -t_0)} } }^3
\exp\lr{ \frac{i(\Bx'' - \Bx')^2 m}{2 (t -t_0) \Hbar} }.
}
%\begin{equation}\label{eqn:freeParticlePropagator:120}
%\boxed{
%}
%\end{equation}

In one or two dimensions the exponential power \( 3 \) need only be adjusted appropriately.

} % answer

         % p32
         %
% Copyright � 2015 Peeter Joot.  All Rights Reserved.
% Licenced as described in the file LICENSE under the root directory of this GIT repository.
%
%\newcommand{\authorname}{Peeter Joot}
\newcommand{\email}{peeterjoot@protonmail.com}
\newcommand{\basename}{FIXMEbasenameUndefined}
\newcommand{\dirname}{notes/FIXMEdirnameUndefined/}

%\renewcommand{\basename}{partitionFunction}
%\renewcommand{\dirname}{notes/phy1520/}
%%\newcommand{\dateintitle}{}
%%\newcommand{\keywords}{}
%
%\newcommand{\authorname}{Peeter Joot}
\newcommand{\onlineurl}{http://sites.google.com/site/peeterjoot2/math2013/\basename.pdf}
\newcommand{\sourcepath}{\dirname\basename.tex}
\newcommand{\generatetitle}[1]{\chapter{#1}}

\newcommand{\vcsinfo}{%
\section*{}
\noindent{\color{DarkOliveGreen}{\rule{\linewidth}{0.1mm}}}
\paragraph{Document version}
%\paragraph{\color{Maroon}{Document version}}
{
\small
\begin{itemize}
\item Available online at:\\ 
\href{\onlineurl}{\onlineurl}
\item Git Repository: \input{./.revinfo/gitRepo.tex}
\item Source: \sourcepath
\item last commit: \input{./.revinfo/gitCommitString.tex}
\item commit date: \input{./.revinfo/gitCommitDate.tex}
\end{itemize}
}
}

%\PassOptionsToPackage{dvipsnames,svgnames}{xcolor}
\PassOptionsToPackage{square,numbers}{natbib}
\documentclass{scrreprt}

\usepackage[left=2cm,right=2cm]{geometry}
\usepackage[svgnames]{xcolor}
\usepackage{peeters_layout}

\usepackage{natbib}

\usepackage[
colorlinks=true,
bookmarks=false,
pdfauthor={\authorname, \email},
backref 
]{hyperref}

% http://tex.stackexchange.com/questions/75773/how-to-reference-problems-by-the-text-label-in-an-exercise-envioronment
\usepackage[english]{cleveref}
\crefname{Exercise}{exercise}{exercises}
\Crefname{Exercise}{Exercise}{Exercises}

\RequirePackage{titlesec}
\RequirePackage{ifthen}

% http://stackoverflow.com/questions/4932910/date-in-the-tabular-environment
\makeatletter
\let\insertdate\@date
\makeatother

\titleformat{\chapter}[display]
{\bfseries\Large}
{\color{DarkSlateGrey}\filleft \authorname
\ifthenelse{\isundefined{\studentnumber}}{}{\\ \studentnumber}
\ifthenelse{\isundefined{\email}}{}{\\ \email}
\ifthenelse{\isundefined{\dateintitle}}{}{\\ \insertdate}
%\ifthenelse{\isundefined{\coursename}}{}{\\ \coursename} % put in title instead.
}
{4ex}
{\color{DarkOliveGreen}{\titlerule}\color{Maroon}
\vspace{2ex}%
\filright}
[\vspace{2ex}%
\color{DarkOliveGreen}\titlerule
]

\newcommand{\beginArtWithToc}[0]{\begin{document}\tableofcontents}
\newcommand{\beginArtNoToc}[0]{\begin{document}}
\newcommand{\EndNoBibArticle}[0]{\end{document}}
\newcommand{\EndArticle}[0]{\bibliography{Bibliography}\bibliographystyle{plainnat}\end{document}}

% 
%\newcommand{\citep}[1]{\cite{#1}}

\colorSectionsForArticle


%
%\usepackage{peeters_layout_exercise}
%\usepackage{peeters_braket}
%\usepackage{peeters_figures}
%
%\beginArtNoToc
%
%\generatetitle{Partition function and ground state energy}
%\chapter{Partition function and ground state energy}
%\label{chap:partitionFunction}

\makeoproblem{Partition function and ground state energy.}{problem:partitionFunction:32}{\citep{sakurai2014modern} pr. 2.32}{
\index{partition function}
\index{ground state}

Define the partition function as

\begin{dmath}\label{eqn:partitionFunction:20}
Z = \int d^3 x' \evalbar{ K( \Bx', t ; \Bx', 0 ) }{\beta = i t/\Hbar},
\end{dmath}

Show that the ground state energy is given by

\begin{dmath}\label{eqn:partitionFunction:40}
-\inv{Z} \PD{\beta}{Z}, \qquad \beta \rightarrow \infty.
\end{dmath}

} % problem

\makeanswer{problem:partitionFunction:32}{

The propagator evaluated at the same point is

\begin{dmath}\label{eqn:partitionFunction:60}
K( \Bx', t ; \Bx', 0 )
=
\sum_{a'} \braket{\Bx'}{a'} \ket{a'}{\Bx'} \exp\lr{ -\frac{i E_{a'} t}{\Hbar}}
=
\sum_{a'} \Abs{\braket{\Bx'}{a'}}^2 \exp\lr{ -\frac{i E_{a'} t}{\Hbar}}
=
\sum_{a'} \Abs{\braket{\Bx'}{a'}}^2 \exp\lr{ -E_{a'} \beta}.
\end{dmath}

The derivative is
\begin{dmath}\label{eqn:partitionFunction:80}
\PD{\beta}{Z}
=
-\int d^3 x' \sum_{a'} E_{a'} \Abs{\braket{\Bx'}{a'}}^2 \exp\lr{ -E_{a'} \beta}.
\end{dmath}

In the \( \beta \rightarrow \infty \) this sum will be dominated by the term with the lowest value of \( E_{a'} \).  Suppose that state is \( a' = 0 \), then

\begin{dmath}\label{eqn:partitionFunction:100}
\lim_{ \beta \rightarrow \infty }
-\inv{Z} \PD{\beta}{Z}
= \frac{
\int d^3 x' E_{0} \Abs{\braket{\Bx'}{0}}^2 \exp\lr{ -E_{0} \beta}
}
{
\int d^3 x' \Abs{\braket{\Bx'}{0}}^2 \exp\lr{ -E_{0} \beta}
}
= E_0.
\end{dmath}

This this stat mech like result seems very striking and profound, and makes me want to go off and study the QM formulation of stat mech that I recall seeing in \citep{pathriastatistical}, but not covered back in phy452.
} % answer

%\EndArticle

         % p33
         %
% Copyright © 2015 Peeter Joot.  All Rights Reserved.
% Licenced as described in the file LICENSE under the root directory of this GIT repository.
%
\makeoproblem{Momentum space free particle propagator.}{problem:freeParticlePropagator:33}{\citep{sakurai2014modern} pr. 2.33}{
\index{free particle propagator!momentum space}
Derive the free particle propagator in momentum space.
} % problem

\makeanswer{problem:freeParticlePropagator:33}{

The momentum space propagator follows in the same fashion as the spatial propagator

\begin{dmath}\label{eqn:freeParticlePropagator:140}
\begin{aligned}
\braket{\Bp''}{\alpha, t ; t_0 }
&= \bra{\Bp''} U \ket{\alpha, t_0 } \\
&= \bra{\Bp''} e^{-i H (t -t_0)/\Hbar} \ket{\alpha, t_0 } \\
&= \bra{\Bp''} e^{-i H (t -t_0)/\Hbar} \lr{ \sum_{a'} \ket{a'} \bra{a' }} \ket{\alpha, t_0 } \\
&= \bra{\Bp''} \sum_{a'} e^{-i E_{a'} (t -t_0)/\Hbar} \ket{a'} \braket{a' }{\alpha, t_0 } \\
&=
\bra{\Bp''} \sum_{a'} e^{-i E_{a'} (t -t_0)/\Hbar} \ket{a'} \bra{a' }
\lr{ \int d^3 \Bp'
\ket{\Bp'}\bra{\Bp'}
}
\ket{\alpha, t_0 } \\
&=
\int d^3 \Bp'
\lr{
\bra{\Bp''} \sum_{a'} e^{-i E_{a'} (t -t_0)/\Hbar} \ket{a'} \braket{a' }{\Bp'}
}
\braket{\Bp'}{\alpha, t_0 } \\
&=
\int d^3 \Bp' K(\Bp'', t ; \Bp', t_0) \braket{\Bp'}{\alpha, t_0 },
\end{aligned}
\end{dmath}

so

\begin{dmath}\label{eqn:freeParticlePropagator:160}
K(\Bp'', t ; \Bp', t_0)
=
\sum_{a'}
\braket{\Bp''}{a'}
\braket{a' }{\Bp'}
e^{-i E_{a'} (t -t_0)/\Hbar}.
\end{dmath}

For the free particle Hamiltonian, this can be evaluated over a momentum space basis

\begin{dmath}\label{eqn:freeParticlePropagator:170}
K(\Bp'', t ; \Bp', t_0)
=
\int d^3 \Bp'''
\braket{\Bp''}{\Bp'''}
\braket{\Bp''' }{\Bp'}
e^{-i E_{\Bp'''} (t -t_0)/\Hbar}
=
\int d^3 \Bp'''
\braket{\Bp''}{\Bp'''}
\delta(\Bp''' - \Bp')
\exp\lr{ -i \frac{(\Bp''')^2 (t -t_0)}{2 m \Hbar}}
=
\braket{\Bp''}{\Bp'}
\exp\lr{ -i \frac{(\Bp')^2 (t -t_0)}{2 m \Hbar}}
\end{dmath}

or

%\begin{dmath}\label{eqn:freeParticlePropagator:200}
\boxedEquation{eqn:freeParticlePropagator:200}{
K(\Bp'', t ; \Bp', t_0)
=
\delta( \Bp'' - \Bp' )
\exp\lr{ -i \frac{(\Bp')^2 (t -t_0)}{2 m \Hbar}}.
}
%\end{dmath}

This is what we expect since the time evolution is given by just this exponential factor

\begin{dmath}\label{eqn:freeParticlePropagator:220}
\braket{\Bp'}{\alpha, t_0 ; t}
= \bra{\Bp'} \exp\lr{ -i \frac{(\Bp')^2 (t -t_0)}{2 m \Hbar}} \ket{\alpha, t_0}
=
\exp\lr{ -i \frac{(\Bp')^2 (t -t_0)}{2 m \Hbar}}
\braket{\Bp'}
{\alpha, t_0}.
\end{dmath}

} % answer

%\EndArticle

         % includes p37(a):
         %
% Copyright � 2015 Peeter Joot.  All Rights Reserved.
% Licenced as described in the file LICENSE under the root directory of this GIT repository.
%
\newcommand{\authorname}{Peeter Joot}
\newcommand{\email}{peeterjoot@protonmail.com}
\newcommand{\basename}{FIXMEbasenameUndefined}
\newcommand{\dirname}{notes/FIXMEdirnameUndefined/}

\renewcommand{\basename}{gaugeTx}
\renewcommand{\dirname}{notes/phy1520/}
%\newcommand{\dateintitle}{}
%\newcommand{\keywords}{}

\newcommand{\authorname}{Peeter Joot}
\newcommand{\onlineurl}{http://sites.google.com/site/peeterjoot2/math2013/\basename.pdf}
\newcommand{\sourcepath}{\dirname\basename.tex}
\newcommand{\generatetitle}[1]{\chapter{#1}}

\newcommand{\vcsinfo}{%
\section*{}
\noindent{\color{DarkOliveGreen}{\rule{\linewidth}{0.1mm}}}
\paragraph{Document version}
%\paragraph{\color{Maroon}{Document version}}
{
\small
\begin{itemize}
\item Available online at:\\ 
\href{\onlineurl}{\onlineurl}
\item Git Repository: \input{./.revinfo/gitRepo.tex}
\item Source: \sourcepath
\item last commit: \input{./.revinfo/gitCommitString.tex}
\item commit date: \input{./.revinfo/gitCommitDate.tex}
\end{itemize}
}
}

%\PassOptionsToPackage{dvipsnames,svgnames}{xcolor}
\PassOptionsToPackage{square,numbers}{natbib}
\documentclass{scrreprt}

\usepackage[left=2cm,right=2cm]{geometry}
\usepackage[svgnames]{xcolor}
\usepackage{peeters_layout}

\usepackage{natbib}

\usepackage[
colorlinks=true,
bookmarks=false,
pdfauthor={\authorname, \email},
backref 
]{hyperref}

% http://tex.stackexchange.com/questions/75773/how-to-reference-problems-by-the-text-label-in-an-exercise-envioronment
\usepackage[english]{cleveref}
\crefname{Exercise}{exercise}{exercises}
\Crefname{Exercise}{Exercise}{Exercises}

\RequirePackage{titlesec}
\RequirePackage{ifthen}

% http://stackoverflow.com/questions/4932910/date-in-the-tabular-environment
\makeatletter
\let\insertdate\@date
\makeatother

\titleformat{\chapter}[display]
{\bfseries\Large}
{\color{DarkSlateGrey}\filleft \authorname
\ifthenelse{\isundefined{\studentnumber}}{}{\\ \studentnumber}
\ifthenelse{\isundefined{\email}}{}{\\ \email}
\ifthenelse{\isundefined{\dateintitle}}{}{\\ \insertdate}
%\ifthenelse{\isundefined{\coursename}}{}{\\ \coursename} % put in title instead.
}
{4ex}
{\color{DarkOliveGreen}{\titlerule}\color{Maroon}
\vspace{2ex}%
\filright}
[\vspace{2ex}%
\color{DarkOliveGreen}\titlerule
]

\newcommand{\beginArtWithToc}[0]{\begin{document}\tableofcontents}
\newcommand{\beginArtNoToc}[0]{\begin{document}}
\newcommand{\EndNoBibArticle}[0]{\end{document}}
\newcommand{\EndArticle}[0]{\bibliography{Bibliography}\bibliographystyle{plainnat}\end{document}}

% 
%\newcommand{\citep}[1]{\cite{#1}}

\colorSectionsForArticle



\usepackage{peeters_layout_exercise}
\usepackage{peeters_braket}
\usepackage{peeters_figures}

\beginArtNoToc

\generatetitle{Gauge transformation}
%\chapter{Gauge transformation}
%\label{chap:gaugeTx}

\makeproblem{}{problem:gaugeTx:0}{
Given a gauge transformation of the free particle Hamiltonian to

\begin{dmath}\label{eqn:gaugeTx:20}
H = \inv{2 m} \BPi \cdot \BPi + e \phi,
\end{dmath}

where

\begin{dmath}\label{eqn:gaugeTx:40}
\BPi = \Bp - \frac{e}{c} \BA,
\end{dmath}

calculate 
\( m d\Bx/dt \), 
\( \antisymmetric{\Pi_i}{\Pi_j} \), and
\( m d^2\Bx/dt^2 \), where 
\( \Bx \) is the Heisenburg picture position operator, and the fields are functions only of position \( \phi = \phi(\Bx), \BA = \BA(\Bx) \).

} % problem

\makeanswer{problem:gaugeTx:0}{
The final results for these calculations are found in \citep{sakurai2014modern}, but seem worth deriving to exersize our commutator muscles.

\paragraph{Heisenberg picture velocity operator}

The first order of business is the Heisenberg picture velocity operator, but first note

\begin{dmath}\label{eqn:gaugeTx:60}
\BPi \cdot \BPi
= \lr{ \Bp - \frac{e}{c} \BA} \cdot \lr{ \Bp - \frac{e}{c} \BA}
= \Bp^2 - \frac{e}{c} \lr{ \BA \cdot \Bp + \Bp \cdot \BA } + \frac{e^2}{c^2} \BA^2.
\end{dmath}

The time evolution of the Heisenburg picture position operator is therefore

\begin{dmath}\label{eqn:gaugeTx:80}
\ddt{\Bx} 
= \inv{i\Hbar} \antisymmetric{\Bx}{H}
= \inv{i\Hbar 2 m} \antisymmetric{\Bx}{\BPi^2}
= \inv{i\Hbar 2 m} \antisymmetric{\Bx}{\Bp^2 - \frac{e}{c} \lr{ \BA \cdot \Bp + \Bp \cdot \BA } + \frac{e^2}{c^2} \BA^2 }
= \inv{i\Hbar 2 m} 
\lr{ 
\antisymmetric{\Bx}{\Bp^2}
- \frac{e}{c} \antisymmetric{\Bx}{ \BA \cdot \Bp + \Bp \cdot \BA }
}
.
\end{dmath}

For the \( \Bp^2 \) commutator we have

\begin{dmath}\label{eqn:gaugeTx:100}
\antisymmetric{x_r}{\Bp^2}
=
i \Hbar \PD{p_r}{\Bp^2}
=
2 i \Hbar p_r,
\end{dmath}

or
\begin{dmath}\label{eqn:gaugeTx:120}
\antisymmetric{\Bx}{\Bp^2}
=
2 i \Hbar \Bp.
\end{dmath}

Computing the remaining commutator, we've got

\begin{equation}\label{eqn:gaugeTx:140}
\begin{aligned}
\antisymmetric{x_r}{\Bp \cdot \BA + \BA \cdot \Bp}
&= x_r p_s A_s - p_s A_s x_r \\
&\quad+ x_r A_s p_s - A_s p_s x_r \\
&= \lr{ \antisymmetric{x_r}{p_s} + p_s x_r } A_s - p_s A_s x_r \\
&\quad+ x_r A_s p_s - A_s \lr{ \antisymmetric{p_s}{x_r} + x_r p_s } \\
&= \antisymmetric{x_r}{p_s} A_s + \cancel{p_s A_s x_r   - p_s A_s x_r} \\
&\quad+ \cancel{x_r A_s p_s - x_r A_s p_s} + A_s \antisymmetric{x_r}{p_s} \\
&= 2 i \Hbar \delta_{r s} A_s \\
&= 2 i \Hbar A_r,
\end{aligned}
\end{equation}

so

\begin{equation}\label{eqn:gaugeTx:160}
\antisymmetric{\Bx}{\Bp \cdot \BA + \BA \cdot \Bp} = 2 i \Hbar \BA.
\end{equation}

Assembling these results gives

%\begin{equation}\label{eqn:gaugeTx:180}
\boxedEquation{eqn:gaugeTx:180}{
\ddt{\Bx} = \inv{m} \lr{ \Bp - \frac{e}{c} \BA } = \inv{m} \BPi,
}
%\end{equation}

as asserted in the text.

\paragraph{Kinetic Momentum communtators}

\begin{dmath}\label{eqn:gaugeTx:200}
\antisymmetric{\Pi_r}{\Pi_s}
=
\antisymmetric{p_r - e A_r/c}{p_s - e A_s/c}
=
\cancel{\antisymmetric{p_r}{p_s}}
- \frac{e}{c} \lr{ \antisymmetric{p_r}{A_s} + \antisymmetric{A_r}{p_s}}
+ \frac{e^2}{c^2} \cancel{\antisymmetric{A_r}{A_s}}.
=
- \frac{e}{c} \lr{ (-i\Hbar) \PD{x_r}{A_s} + (i\Hbar) \PD{x_s}{A_r} }
=
- \frac{i e \Hbar}{c} \lr{ -\PD{x_r}{A_s} + \PD{x_s}{A_r} }.
= 
- \frac{i e \Hbar}{c} \epsilon_{t s r} B_t,
\end{dmath}

or
%\begin{dmath}\label{eqn:gaugeTx:220}
\boxedEquation{eqn:gaugeTx:220}{
\antisymmetric{\Pi_r}{\Pi_s}
= 
\frac{i e \Hbar}{c} \epsilon_{r s t} B_t.
}
%\end{dmath}

\paragraph{Quantum Lorentz force}

For the force equation we have

\begin{dmath}\label{eqn:gaugeTx:240}
m \frac{d^2 \Bx}{dt^2} 
= \ddt{\BPi}
= \inv{i \Hbar} \antisymmetric{\BPi}{H}
= \inv{i \Hbar 2 m } \antisymmetric{\BPi}{\BPi^2}
+ \inv{i \Hbar } \antisymmetric{\BPi}{e \phi}.
\end{dmath}

For the \( \phi \) commutator consider one component

\begin{dmath}\label{eqn:gaugeTx:260}
\antisymmetric{\Pi_r}{e \phi} 
=
e \antisymmetric{p_r - \frac{e}{c} A_r}{\phi}
=
e \antisymmetric{p_r}{\phi}
=
e (-i\Hbar) \PD{x_r}{\phi},
\end{dmath}

or
\begin{equation}\label{eqn:gaugeTx:280}
\inv{i \Hbar} \antisymmetric{\BPi}{e \phi} 
=
- e \spacegrad \phi
=
e \BE.
\end{equation}



} % answer

\EndArticle

         % p37(b):
         %
% Copyright � 2015 Peeter Joot.  All Rights Reserved.
% Licenced as described in the file LICENSE under the root directory of this GIT repository.
%
%\newcommand{\authorname}{Peeter Joot}
\newcommand{\email}{peeterjoot@protonmail.com}
\newcommand{\basename}{FIXMEbasenameUndefined}
\newcommand{\dirname}{notes/FIXMEdirnameUndefined/}

%\renewcommand{\basename}{gaugeTxCurrent}
%\renewcommand{\dirname}{notes/phy1520/}
%%\newcommand{\dateintitle}{}
%%\newcommand{\keywords}{}
%
%\newcommand{\authorname}{Peeter Joot}
\newcommand{\onlineurl}{http://sites.google.com/site/peeterjoot2/math2013/\basename.pdf}
\newcommand{\sourcepath}{\dirname\basename.tex}
\newcommand{\generatetitle}[1]{\chapter{#1}}

\newcommand{\vcsinfo}{%
\section*{}
\noindent{\color{DarkOliveGreen}{\rule{\linewidth}{0.1mm}}}
\paragraph{Document version}
%\paragraph{\color{Maroon}{Document version}}
{
\small
\begin{itemize}
\item Available online at:\\ 
\href{\onlineurl}{\onlineurl}
\item Git Repository: \input{./.revinfo/gitRepo.tex}
\item Source: \sourcepath
\item last commit: \input{./.revinfo/gitCommitString.tex}
\item commit date: \input{./.revinfo/gitCommitDate.tex}
\end{itemize}
}
}

%\PassOptionsToPackage{dvipsnames,svgnames}{xcolor}
\PassOptionsToPackage{square,numbers}{natbib}
\documentclass{scrreprt}

\usepackage[left=2cm,right=2cm]{geometry}
\usepackage[svgnames]{xcolor}
\usepackage{peeters_layout}

\usepackage{natbib}

\usepackage[
colorlinks=true,
bookmarks=false,
pdfauthor={\authorname, \email},
backref 
]{hyperref}

% http://tex.stackexchange.com/questions/75773/how-to-reference-problems-by-the-text-label-in-an-exercise-envioronment
\usepackage[english]{cleveref}
\crefname{Exercise}{exercise}{exercises}
\Crefname{Exercise}{Exercise}{Exercises}

\RequirePackage{titlesec}
\RequirePackage{ifthen}

% http://stackoverflow.com/questions/4932910/date-in-the-tabular-environment
\makeatletter
\let\insertdate\@date
\makeatother

\titleformat{\chapter}[display]
{\bfseries\Large}
{\color{DarkSlateGrey}\filleft \authorname
\ifthenelse{\isundefined{\studentnumber}}{}{\\ \studentnumber}
\ifthenelse{\isundefined{\email}}{}{\\ \email}
\ifthenelse{\isundefined{\dateintitle}}{}{\\ \insertdate}
%\ifthenelse{\isundefined{\coursename}}{}{\\ \coursename} % put in title instead.
}
{4ex}
{\color{DarkOliveGreen}{\titlerule}\color{Maroon}
\vspace{2ex}%
\filright}
[\vspace{2ex}%
\color{DarkOliveGreen}\titlerule
]

\newcommand{\beginArtWithToc}[0]{\begin{document}\tableofcontents}
\newcommand{\beginArtNoToc}[0]{\begin{document}}
\newcommand{\EndNoBibArticle}[0]{\end{document}}
\newcommand{\EndArticle}[0]{\bibliography{Bibliography}\bibliographystyle{plainnat}\end{document}}

% 
%\newcommand{\citep}[1]{\cite{#1}}

\colorSectionsForArticle


%
%\usepackage{peeters_layout_exercise}
%\usepackage{peeters_braket}
%\usepackage{peeters_figures}
%
%\beginArtNoToc
%
%\generatetitle{Gauge transformed probability current}
%%\chapter{Gauge transformed probability current}
%\label{chap:gaugeTxCurrent}

\makeoproblem{Gauge transformed probability current.}{problem:gaugeTxCurrent:1}{\citep{sakurai2014modern} pr. 2.37 (b)}{
\makesubproblem{}{problem:gaugeTxCurrent:1:a}
\index{probability!current}
\index{gauge transformation!probability current}

For the gauge transformed Schr\"{o}dinger equation

\begin{dmath}\label{eqn:gaugeTxCurrent:20}
\inv{2m} \BPi(\Bx) \cdot \BPi(\Bx) \psi(\Bx, t) + e \phi(\Bx) \psi(\Bx, t) = i \Hbar \PD{t}{}\psi(\Bx, t),
\end{dmath}

where

\begin{dmath}\label{eqn:gaugeTxCurrent:40}
\BPi(\Bx) = -i \Hbar \spacegrad - \frac{e}{c} \BA(\Bx),
\end{dmath}

find the probability current defined by

\begin{dmath}\label{eqn:gaugeTxCurrent:60}
\PD{t}{\psi} + \spacegrad \cdot \Bj.
\end{dmath}

\makesubproblem{}{problem:gaugeTxCurrent:1:b}
Once obtained, let use a \( \psi = \sqrt{\rho} e^{i S/\Hbar} \) wavefunction representation, and find the corresponding form for the probability current.

\makesubproblem{}{problem:gaugeTxCurrent:1:c}
Evaluate \( \int d^3 x \Bj \).

} % problem

\makeanswer{problem:gaugeTxCurrent:1}{

\makeSubAnswer{}{problem:gaugeTxCurrent:1:a}

Equation \cref{eqn:gaugeTxCurrent:20} and its conjugate are

\begin{dmath}\label{eqn:gaugeTxCurrent:22}
\begin{aligned}
\inv{2m} \BPi \cdot \BPi \psi + e \phi \psi &= i \Hbar \PD{t}{\psi} \\
\inv{2m} \BPi^\conj \cdot \BPi^\conj \psi^\conj + e \phi \psi^\conj &= -i \Hbar \PD{t}{\psi^\conj}
\end{aligned}
\end{dmath}

which can be used immediately in a chain rule expansion of the probability time derivative

\begin{dmath}\label{eqn:gaugeTxCurrent:80}
i \Hbar \PD{t}{\rho}
=
i \Hbar \psi^\conj \PD{t}{\psi} +
i \Hbar \psi \PD{t}{\psi^\conj}
=
\psi^\conj \lr{ \inv{2m} \BPi \cdot \BPi \psi + e \phi \psi } -
\psi \lr{ \inv{2m} \BPi^\conj \cdot \BPi^\conj \psi^\conj + e \phi \psi^\conj }
=
\inv{2m} \lr{
\psi^\conj \BPi \cdot \BPi \psi
-\psi \BPi^\conj \cdot \BPi^\conj \psi^\conj
}.
\end{dmath}

We have a difference of conjugates, so can get away with expanding just the first term

\begin{dmath}\label{eqn:gaugeTxCurrent:100}
\psi^\conj \BPi \cdot \BPi \psi
=
\psi^\conj
\psi
=
\psi^\conj
\lr{ -i \Hbar \spacegrad - \frac{e}{c} \BA } \cdot \lr{ -i \Hbar \spacegrad - \frac{e}{c} \BA }
\psi
=
\psi^\conj
\lr{
-\Hbar^2 \spacegrad^2 + \frac{i \Hbar e}{c} \lr{ \BA \cdot \spacegrad + \spacegrad \cdot \BA }
+ \frac{e^2}{c^2} \BA^2
}
\psi.
\end{dmath}

Note that in the directional derivative terms, the gradient operates on everything to its right, including \( \BA \).  Also note that the last term has no imaginary component, so it will not contribute to the difference of conjugates.

This gives

\begin{dmath}\label{eqn:gaugeTxCurrent:120}
\begin{aligned}
\psi^\conj \BPi \cdot \BPi \psi - \psi \BPi^\conj \cdot \BPi^\conj \psi^\conj
&=
\psi^\conj
\lr{
-\Hbar^2 \spacegrad^2 \psi + \frac{i \Hbar e}{c} \lr{ \BA \cdot \spacegrad \psi + \spacegrad \cdot (\BA \psi) }
}  \\
&\quad -
\psi
\lr{
-\Hbar^2 \spacegrad^2 \psi^\conj - \frac{i \Hbar e}{c} \lr{ \BA \cdot \spacegrad \psi^\conj + \spacegrad \cdot (\BA \psi^\conj) }
}  \\
&=
-\Hbar^2 \lr{ \psi^\conj \spacegrad^2 \psi - \psi \spacegrad^2 \psi^\conj } \\
&\quad +
\frac{i \Hbar e}{c}
\lr{
\psi^\conj
\BA \cdot \spacegrad \psi + \psi^\conj \spacegrad \cdot (\BA \psi)
+
\psi
\BA \cdot \spacegrad \psi^\conj + \psi \spacegrad \cdot (\BA \psi^\conj)
}
\end{aligned}
\end{dmath}

The first term is recognized as a divergence

\begin{dmath}\label{eqn:gaugeTxCurrent:140}
\spacegrad \cdot \lr{ \psi^\conj \spacegrad \psi - \psi \spacegrad \psi^\conj }
=
\psi^\conj \spacegrad \cdot \spacegrad \psi
+
\spacegrad \psi \cdot \spacegrad \psi^\conj
-
\psi \spacegrad \cdot \spacegrad \psi^\conj
-
\spacegrad \psi^\conj \cdot \spacegrad \psi
= \psi^\conj \spacegrad^2 \psi - \psi \spacegrad^2 \psi^\conj.
\end{dmath}

The second term can also be factored into a divergence operation

\begin{dmath}\label{eqn:gaugeTxCurrent:160}
\begin{aligned}
\psi^\conj
\BA \cdot \spacegrad \psi &+ \psi^\conj \spacegrad \cdot (\BA \psi)
+
\psi
\BA \cdot \spacegrad \psi^\conj + \psi \spacegrad \cdot (\BA \psi^\conj)  \\
%&=
%\BA \cdot \lr{
%\psi^\conj \spacegrad \psi
%+
%\psi \spacegrad \psi^\conj
%}
%+\psi^\conj \spacegrad \cdot (\BA \psi)
%+\psi \spacegrad \cdot (\BA \psi^\conj) \\
&=
\lr{ \psi^\conj\BA \cdot \spacegrad \psi
+\psi \spacegrad \cdot (\BA \psi^\conj)
}
+\lr{
\psi \BA \cdot \spacegrad \psi^\conj
+\psi^\conj \spacegrad \cdot (\BA \psi)
} \\
&= 2 \spacegrad \cdot \lr{ \BA \psi \psi^\conj } \\
%&= 2 \spacegrad \cdot \lr{ \rho \BA }
\end{aligned}
\end{dmath}

Putting all the pieces back together we have

\begin{dmath}\label{eqn:gaugeTxCurrent:180}
\PD{t}{\rho}
=
\inv{2m i \Hbar} \lr{
\psi^\conj \BPi \cdot \BPi \psi
-\psi \BPi^\conj \cdot \BPi^\conj \psi^\conj
}
=
\spacegrad \cdot
\inv{2m i \Hbar} \lr{
-\Hbar^2
\lr{ \psi^\conj \spacegrad \psi - \psi \spacegrad \psi^\conj }
+ \frac{ i \Hbar e}{c} 2 \BA \psi \psi^\conj
}
=
\spacegrad \cdot
\lr{
\frac{i \Hbar}{2 m} \lr{ \psi^\conj \spacegrad \psi - \psi \spacegrad \psi^\conj }
+ \frac{e}{m c} \BA \psi \psi^\conj
}.
\end{dmath}

From \cref{eqn:gaugeTxCurrent:60}, the probability current must be

\begin{dmath}\label{eqn:gaugeTxCurrent:200}
\Bj
=
\frac{\Hbar}{2 i m} \lr{ \psi^\conj \spacegrad \psi - \psi \spacegrad \psi^\conj }
- \frac{e}{m c} \BA \psi \psi^\conj,
\end{dmath}

or
%\begin{dmath}\label{eqn:gaugeTxCurrent:220}
\boxedEquation{eqn:gaugeTxCurrent:220}{
\Bj
=
\frac{\Hbar}{m} \Imag \lr{ \psi^\conj \spacegrad \psi }
- \frac{e}{m c} \BA \psi \psi^\conj.
}
%\end{dmath}

\makeSubAnswer{}{problem:gaugeTxCurrent:1:b}

To find the \( \psi = \sqrt{\rho} e^{i S/\Hbar} \) form of the current, note that

\begin{dmath}\label{eqn:gaugeTxCurrent:240}
\spacegrad \psi = e^{i S/\Hbar} \spacegrad \sqrt{\rho} + \sqrt{\rho} e^{i S/\Hbar} \spacegrad \lr{ i S/\Hbar },
\end{dmath}

so
\begin{dmath}\label{eqn:gaugeTxCurrent:260}
\psi^\conj \spacegrad \psi
=
\sqrt{\rho} \spacegrad \sqrt{\rho} + \frac{ i \rho}{\Hbar} \spacegrad S.
\end{dmath}

Discarding the real part of this product, we have

\begin{dmath}\label{eqn:gaugeTxCurrent:280}
\Bj
= \frac{\Hbar}{m} \rho \spacegrad S - \frac{e }{m c} \BA \rho,
\end{dmath}

or
%\begin{dmath}\label{eqn:gaugeTxCurrent:300}
\boxedEquation{eqn:gaugeTxCurrent:300}{
\Bj = \frac{\rho}{m} \lr{ \spacegrad S - \frac{e}{c} \BA }.
}
%\end{dmath}

\makeSubAnswer{}{problem:gaugeTxCurrent:1:c}
Finally, note that

\begin{dmath}\label{eqn:gaugeTxCurrent:320}
-i \Hbar \spacegrad \psi = \bra{ \Bx } \Bp \ket{\psi},
\end{dmath}

so

\begin{dmath}\label{eqn:gaugeTxCurrent:340}
\Bj
= \frac{\Hbar}{m} \Imag \lr{ \braket{\Psi}{\Bx} \lr{ \frac{i}{ \Hbar} } \bra{\Bx} \Bp \ket{\psi} }
- \frac{e}{m c} \BA \braket{ \psi}{\Bx} \braket{\Bx}{\psi}.
\end{dmath}

Integrating over all space to eliminate the identity operators, this is

\begin{dmath}\label{eqn:gaugeTxCurrent:360}
\int d^3 x \Bj
=
\frac{1}{m} \Imag \lr{ i \bra{\Psi} \Bp \ket{\psi} }
- \frac{e}{m c} \BA \braket{ \psi}{\psi}
=
\inv{m} \bra{\psi} \lr{ \Bp - \frac{e}{c} \BA } \ket{\psi}
=
\inv{m} \expectation{ \BPi }.
\end{dmath}

} % answer

%\EndArticle

         % ps2. coherent states
         %
% Copyright � 2015 Peeter Joot.  All Rights Reserved.
% Licenced as described in the file LICENSE under the root directory of this GIT repository.
%

\makeoproblem{Coherent States.}{gradQuantum:problemSet2:1}{phy1520 2015 ps2.1, and \citep{sakurai2014modern} pr. 2.19(c)}{
\index{coherent state}

Consider the harmonic oscillator Hamiltonian \( H = p^2/2m + m \omega^2 x^2/2\). Define the coherent state \( \ket{z} \) as the eigenfunction of the annihilation operator, via \( a \ket{z} = z \ket{z} \), where \( a \) is the oscillator annihilation operator and \( z \) is some complex number which characterizes the coherent state.

\makesubproblem{}{gradQuantum:problemSet2:1a}
Expanding \( \ket{ z } \) in terms of oscillator energy eigenstates \( \ket{n} \), show that \( \ket{z} = C e^{z a^\dagger} \ket{0} \). Find the normalization constant C.

\makesubproblem{}{gradQuantum:problemSet2:1b}
Calculate the overlap \( \braket{z}{z'} \) for normalized coherent states \( \ket{z} \).

\makesubproblem{}{gradQuantum:problemSet2:1c}
Using the wavefunction \( \ket{z} \), compute \( \expectation{x}, \expectation{p}, \expectation{x^2}\), and \( \expectation{p^2} \) by defining \( x,p \) in terms of \( a, a^\dagger \).

\makesubproblem{}{gradQuantum:problemSet2:1d}

The time evolution of any observed quantity in quantum mechanics can be described in two ways:

\begin{enumerate}[(a)]
\item Schr\"{o}dinger: the wavefunction evolves as \( \ket{\psi(t)} = e^{-i H t/\Hbar} \ket{\psi(0)} \) and the operator \( A \) is time-independent, or
\item Heisenberg: the wavefunction is fixed to its value at \( t = 0 \), say \(  \ket{\psi} \) , and operators evolve as \( A(t) = e^{i H t/\Hbar} A e^{-i H t/\Hbar}\) .
\end{enumerate}

Show that both prescriptions yield the same result for any matrix elements or measured quantities.

\makesubproblem{}{gradQuantum:problemSet2:1e}
Using the Heisenberg picture, compute the time evolution of \( \expectation{x(t)}, \expectation{p(t)}, \expectation{x^2(t)}\), and \( \expectation{p^2(t)} \) in the coherent state \( \ket{z} \).  Comment on connections to classical dynamics of the oscillator in phase space.

\makesubproblem{}{gradQuantum:problemSet2:1f}

Show that \( \Abs{f(n)}^2 \) for a coherent state written as

\begin{dmath}\label{eqn:gradQuantumProblemSet2Problem1:561}
\ket{z} = \sum_{n=0}^\infty f(n) \ket{n}
\end{dmath}

has the form of a Poisson distribution, and find the most probable value of \( n\), and thus the most probable energy.

} % makeproblem

\makeanswer{gradQuantum:problemSet2:1}{
\withproblemsetsParagraph{
\makeSubAnswer{}{gradQuantum:problemSet2:1a}

Let

\begin{dmath}\label{eqn:gradQuantumProblemSet2Problem1:20}
\ket{z} = \sum_{n=0}^\infty c_n \ket{n},
\end{dmath}

The defining identity for \( \ket{z} \) becomes

\begin{dmath}\label{eqn:gradQuantumProblemSet2Problem1:40}
a \ket{z}
=
\sum_{n=0}^\infty c_n a \ket{n}
=
\sum_{n=1}^\infty c_n \sqrt{n} \ket{n-1}
=
\sum_{n=0}^\infty c_{n+1} \sqrt{n+1} \ket{n}
=
\sum_{n=0}^\infty c_{n} z \ket{n}.
\end{dmath}

Equating like terms provides a recurrence relation for \( c_n \)

\begin{dmath}\label{eqn:gradQuantumProblemSet2Problem1:60}
c_{n} = \frac{c_{n-1} z}{\sqrt{n}},
\end{dmath}

or

\begin{dmath}\label{eqn:gradQuantumProblemSet2Problem1:80}
\begin{aligned}
c_1 &= \frac{c_0 z}{\sqrt{1}} \\
c_2 &= \frac{c_1 z}{\sqrt{2}} = \frac{c_0 z^2}{\sqrt{2 \times 1}}  \\
c_3 &= \frac{c_2 z}{\sqrt{3}} = \frac{c_0 z^3}{\sqrt{3!}},
\end{aligned}
\end{dmath}

or more generally

\boxedEquation{eqn:gradQuantumProblemSet2Problem1:100}{
c_n = \frac{c_0 z^n}{\sqrt{n!}}.
}

or
\begin{dmath}\label{eqn:gradQuantumProblemSet2Problem1:120}
\ket{z}
= c_0
\sum_{n=0}^\infty \frac{z^n}{\sqrt{n!}} \ket{n}.
\end{dmath}

A similar recurrence relation can be constructed for \( \ket{n} \)

\begin{dmath}\label{eqn:gradQuantumProblemSet2Problem1:140}
\ket{n}
= a^\dagger \frac{\ket{ n - 1 }}{\sqrt{n}}
= (a^\dagger)^2 \frac{\ket{ n - 2 }}{\sqrt{(n)(n-1)}}
= (a^\dagger)^{n-1} \frac{\ket{ n - (n-1) }}{\sqrt{n(n-1)(n-2)...(n- (n-2))}}
= (a^\dagger)^{n} \frac{\ket{ 0 }}{\sqrt{n!}},
\end{dmath}

so that

\begin{dmath}\label{eqn:gradQuantumProblemSet2Problem1:160}
\ket{z}
= c_0
\sum_{n=0}^\infty \frac{(a^\dagger z)^n}{n!} \ket{0},
\end{dmath}

or
\boxedEquation{eqn:gradQuantumProblemSet2Problem1:180}{
\ket{z}
= c_0 e^{a^\dagger z} \ket{0}.
}

The normalization follows nicely from the \cref{eqn:gradQuantumProblemSet2Problem1:120} representation

\begin{dmath}\label{eqn:gradQuantumProblemSet2Problem1:200}
\begin{aligned}
\braket{z}{z}
&=
\Abs{c_0}^2
\sum_{n,m=0}^\infty
\bra{m} \frac{(z^\conj)^m}{\sqrt{m!}}
\frac{z^n}{\sqrt{n!}} \ket{n} \\
&= \Abs{c_0}^2
\sum_{n=0}^\infty
\bra{n} \frac{\Abs{z}^{2n}}{n!} \ket{n} \\
&= \Abs{c_0}^2
\sum_{n=0}^\infty
\frac{\Abs{z}^{2n}}{n!}  \\
&= \Abs{c_0}^2 e^{\Abs{z}^2} \\
&= 1.
\end{aligned}
\end{dmath}

Picking a real value for the constant provides a z-dependent normalization for the state

\boxedEquation{eqn:gradQuantumProblemSet2Problem1:220}{
c_0 = e^{-\Abs{z}^2/2},
}

or
\begin{dmath}\label{eqn:gradQuantumProblemSet2Problem1:240}
\ket{z} = e^{-\Abs{z}^2/2 + a^\dagger z} \ket{0}.
\end{dmath}

\makeSubAnswer{}{gradQuantum:problemSet2:1b}

\begin{dmath}\label{eqn:gradQuantumProblemSet2Problem1:260}
\begin{aligned}
\braket{z}{z'}
&=
e^{-\Abs{z}^2/2 } e^{-\Abs{z'}^2/2 }
\sum_{n,m=0}^\infty
\bra{m} \frac{(z^\conj)^m}{\sqrt{m!}}
\frac{(z')^n}{\sqrt{n!}} \ket{n}  \\
&=
e^{-\Abs{z}^2/2 -\Abs{z'}^2/2 }
\sum_{n=0}^\infty
\bra{n} \frac{(z^\conj)^n}{\sqrt{n!}}
\frac{(z')^n}{\sqrt{n!}} \ket{n}  \\
&=
\exp\lr{ -\Abs{z}^2/2 -\Abs{z'}^2/2 + z^\conj z' }.
\end{aligned}
\end{dmath}

This can be rewritten in terms of the absolute difference between the two z values

\begin{dmath}\label{eqn:gradQuantumProblemSet2Problem1:280}
\braket{z}{z'} =
\exp\lr{ -\inv{2} \lr{ \Abs{z - z'}^2 - 2 i \Imag\lr{ z' z^\conj } } },
\end{dmath}

however I'm not sure that's any prettier.

\makeSubAnswer{}{gradQuantum:problemSet2:1c}

First note that

\begin{dmath}\label{eqn:gradQuantumProblemSet2Problem1:300}
\bra{z} a^\dagger
=
\lr{a \ket{z}}^\dagger
=
\lr{z \ket{z}}^\dagger
=
\bra{z} z^\conj,
\end{dmath}

so

\begin{dmath}\label{eqn:gradQuantumProblemSet2Problem1:320}
\expectation{x}
=
\frac{x_0}{\sqrt{2}} \bra{z} a + a^\dagger \ket{z}
=
\frac{x_0}{\sqrt{2}} \bra{z} z + z^\conj \ket{z}
=
\frac{x_0}{\sqrt{2}} \lr{ z + z^\conj }
%=
%\frac{2 x_0}{\sqrt{2}} \Real z
%=
%\sqrt{2} x_0 \Real z
%=
%\sqrt{\frac{2 \Hbar}{ m \omega}} \Real z,
=
\sqrt{\frac{2 \Hbar}{ m \omega}} \frac{ z + z^\conj }{2},
\end{dmath}

\begin{dmath}\label{eqn:gradQuantumProblemSet2Problem1:340}
\expectation{p}
=
\frac{i \Hbar}{x_0 \sqrt{2}} \bra{z} a^\dagger - a\ket{z}
=
\frac{-i \Hbar}{x_0 \sqrt{2}} \bra{z} z - z^\conj \ket{z}
%=
%\frac{\sqrt{2} \Hbar}{x_0} \Imag z
%=
%\sqrt{ 2 m \Hbar \omega } \Imag z,
=
\sqrt{ 2 m \Hbar \omega } \frac{z - z^\conj}{2i}
\end{dmath}

\begin{dmath}\label{eqn:gradQuantumProblemSet2Problem1:360}
\expectation{x^2}
=
\frac{x_0^2}{2} \bra{z} \lr{a + a^\dagger}^2 \ket{z}
=
\frac{x_0^2}{2} \bra{z} \lr{a^2 + (a^\dagger)^2 + a a^\dagger + a^\dagger a} \ket{z}
=
\frac{x_0^2}{2} \bra{z} \lr{a^2 + (a^\dagger)^2 + \antisymmetric{a}{a^\dagger} + 2 a^\dagger a} \ket{z}
=
\frac{x_0^2}{2} \lr{z^2 + (z^\conj)^2 + 1 + 2 z^\conj z}
=
\frac{x_0^2}{2} \lr{(z + z^\conj)^2 + 1}
=
\frac{\Hbar}{2 m \omega} \lr{(z + z^\conj)^2 + 1},
\end{dmath}

and
\begin{dmath}\label{eqn:gradQuantumProblemSet2Problem1:380}
\expectation{p^2}
=
\frac{- \Hbar^2}{2 x_0^2} \bra{z} \lr{a^\dagger - a}^2 \ket{z}
=
\frac{- \Hbar^2}{2 x_0^2} \bra{z} \lr{(a^\dagger)^2 + a^2 - a a^\dagger - a^\dagger a} \ket{z}
=
\frac{- \Hbar^2}{2 x_0^2} \bra{z} \lr{(a^\dagger)^2 + a^2 - \antisymmetric{a}{a^\dagger} - 2 a^\dagger a} \ket{z}
=
\frac{- \Hbar^2}{2 x_0^2} \lr{(z^\conj)^2 + z^2 - 1 - 2 z^\conj z}
=
\frac{m \Hbar \omega}{2} \lr{ 1 - (z - z^\conj)^2 }.
\end{dmath}

As a check against what was stated in class, observe that the minimum uncertainty are satisfied

\begin{dmath}\label{eqn:gradQuantumProblemSet2Problem1:400}
\expectation{x^2} - \expectation{x}^2
=
\frac{x_0^2}{2} \lr{ (z + z^\conj)^2 + 1 - (z + z^\conj)^2 }
= \frac{x_0^2}{2},
\end{dmath}

and
\begin{dmath}\label{eqn:gradQuantumProblemSet2Problem1:420}
\expectation{p^2} - \expectation{p}^2
=
\frac{\Hbar^2}{2 x_0^2} \lr{ 1 - (z + z^\conj)^2 - -(z - z^\conj)^2 }
=
\frac{\Hbar^2}{2 x_0^2},
\end{dmath}

so we have
\begin{dmath}\label{eqn:gradQuantumProblemSet2Problem1:440}
\Delta x \Delta p = \frac{\Hbar}{2}.
\end{dmath}

\makeSubAnswer{}{gradQuantum:problemSet2:1d}

Suppose that \( \setlr{ \ket{\psi} } \) is a basis for the observable \( A \).  In the Heisenberg picture the matrix element for that operator is

\begin{dmath}\label{eqn:gradQuantumProblemSet2Problem1:501}
\bra{\psi} A(t) \ket{\psi'}
=
\bra{\psi} e^{i H t/\Hbar} A e^{-i H t/\Hbar} \ket{\psi'}
=
\sum_{\psi'',\psi'''}
\bra{\psi} e^{i H t/\Hbar} \ket{\psi''} \bra{\psi''} A \ket{\psi'''} \bra{\psi'''} e^{-i H t/\Hbar} \ket{\psi'}.
\end{dmath}

This product of three matrix elements has the structure of a similarity transformation \( \tilde{U}^\dagger \tilde{A} \tilde{U} \), where \( \tilde{U} \) is the matrix element of the time evolution operator and \( \tilde{A} \) is the matrix element of the observable \( A \).

Compare this to the Schr\"{o}dinger picture matrix element with respect to time evolved states

\begin{dmath}\label{eqn:gradQuantumProblemSet2Problem1:521}
\bra{\psi(t)} A \ket{\psi'(t)}
=
\lr{ \bra{\psi} e^{i H t/\Hbar} } A \lr{ e^{-i H t/\Hbar} \ket{\psi'} }
=
\sum_{\psi'',\psi'''}
\bra{\psi} e^{i H t/\Hbar} \ket{\psi''} \bra{\psi''} A \ket{\psi'''} \bra{\psi'''} e^{-i H t/\Hbar} \ket{\psi'}.
\end{dmath}

This has exactly the same structure as in the Heisenberg picture.  Since average quantities are matrix elements with respect to the same pair of states, this shows that measurements are independent of whether the Heisenberg or Schr\"{o}dinger picture is used to describe those measurements.

\makeSubAnswer{}{gradQuantum:problemSet2:1e}

With time evolution in the mix using the Heisenberg representation of the annihilation operator \( a(t) = a e^{-i \omega t} \), the \( x \) expectation is

\begin{dmath}\label{eqn:gradQuantumProblemSet2Problem1:460}
\expectation{x(t)}
=
\frac{x_0}{\sqrt{2}} \bra{z} \lr{ a e^{-i \omega t} + a^\dagger e^{i \omega t} } \ket{z}
=
\frac{x_0}{\sqrt{2}} \lr{ z e^{-i \omega t} + z^\conj e^{i \omega t} } .
\end{dmath}

It's clear how to generalize the stationary state calculations in \partref{gradQuantum:problemSet2:1c}, and can do so by inspection

\begin{dmath}\label{eqn:gradQuantumProblemSet2Problem1:480}
\begin{aligned}
\expectation{x} &= \sqrt{ \frac{\Hbar}{2 m \omega} } \lr{ z e^{-i \omega t} + z^\conj e^{i \omega t} } \\
\expectation{xp} &= \frac{i \Hbar}{2} \lr{ z e^{-i \omega t} + z^\conj e^{i \omega t} } \lr{ z e^{-i \omega t} - z^\conj e^{i \omega t} } \\
\expectation{p} &= -i \sqrt{ \frac{m \Hbar \omega}{2} } \lr{ z e^{-i \omega t} - z^\conj e^{i \omega t} } \\
\expectation{x^2} &= \frac{\Hbar}{2 m \omega} \lr{(z e^{-i \omega t} + z^\conj e^{i \omega t} )^2 + 1} = \expectation{x}^2 + \frac{\Hbar}{2 m \omega} \\
\expectation{p^2} &= \frac{m \Hbar \omega}{2} \lr{ 1 - (z e^{-i \omega t} - z^\conj e^{i \omega t})^2 } = \expectation{p}^2 + \frac{m \Hbar \omega}{2}.
\end{aligned}
\end{dmath}

In class, symmetric and antisymmetric conjugate sums of \( z \) were identified as position and momentum, with

\begin{dmath}\label{eqn:gradQuantumProblemSet2Problem1:541}
\begin{aligned}
x_0 &\equiv \sqrt{\frac{2 \Hbar}{m \omega}} \Real z = \sqrt{\frac{\Hbar}{2 m \omega}} \lr{ z + z^\conj } \\
p_0 &\equiv \sqrt{2 m \Hbar \omega} \Imag z = -i \sqrt{\frac{m \Hbar \omega}{2}} \lr{ z - z^\conj }.
\end{aligned}
\end{dmath}

With that identification the expectation values above are

\begin{dmath}\label{eqn:gradQuantumProblemSet2Problem1:481}
\begin{aligned}
\expectation{x} &= x_0 \cos(\omega t) + \frac{p_0}{m \omega} \sin(\omega t) \\
\expectation{p} &= p_0 \cos(\omega t) - m \omega x_0 \sin(\omega t) \\
\end{aligned}
\end{dmath}

These expectations are analogous to the phase space trajectories of classical particles.

\makeSubAnswer{}{gradQuantum:problemSet2:1f}

The Poisson distribution has the form

\begin{dmath}\label{eqn:gradQuantumProblemSet2Problem1:581}
P(n) = \frac{\mu^{n} e^{-\mu}}{n!}.
\end{dmath}

Here \( \mu \) is the mean of the distribution

\begin{dmath}\label{eqn:gradQuantumProblemSet2Problem1:601}
\expectation{n}
= \sum_{n=0}^\infty n P(n)
= \sum_{n=1}^\infty n \frac{\mu^{n} e^{-\mu}}{n!}
= \mu e^{-\mu} \sum_{n=1}^\infty \frac{\mu^{n-1}}{(n-1)!}
= \mu e^{-\mu} e^{\mu}
= \mu.
\end{dmath}

We found that the coherent state had the form

\begin{dmath}\label{eqn:gradQuantumProblemSet2Problem1:621}
\ket{z} = c_0 \sum_{n=0} \frac{z^n}{\sqrt{n!}} \ket{n},
\end{dmath}

so the probability coefficients for \( \ket{n} \) are

\begin{dmath}\label{eqn:gradQuantumProblemSet2Problem1:641}
P(n)
= c_0^2 \frac{\Abs{z^n}^2}{n!}
= e^{-\Abs{z}^2} \frac{\Abs{z^n}^2}{n!}.
\end{dmath}

This has the structure of the Poisson distribution with mean \( \mu = \Abs{z}^2 \).  The most probable value of \( n \) is that for which \( \Abs{f(n)}^2 \) is the largest.  This is, in general, hard to compute, since we have a maximization problem in the integer domain that falls outside the normal toolbox.  If we assume that \( n \) is large, so that Stirling's approximation can be used to approximate the factorial, and also seek a non-integer value that maximizes the distribution, the most probable value will be the closest integer to that, and this can be computed.  Let

\begin{dmath}\label{eqn:gradQuantumProblemSet2Problem1:661}
g(n)
= \Abs{f(n)}^2
= \frac{e^{-\mu} \mu^n}{n!}
= \frac{e^{-\mu} \mu^n}{e^{\ln n!}}
\approx e^{-\mu - n \ln n + n } \mu^n.
= e^{-\mu - n \ln n + n + n \ln \mu }
\end{dmath}

This is maximized when

\begin{dmath}\label{eqn:gradQuantumProblemSet2Problem1:681}
0
= \frac{dg}{dn}
= \lr{ - \ln n - 1 + 1 + \ln \mu } g(n),
\end{dmath}

which is maximized at \( n = \mu \).  One of the integers \( n = \lfloor \mu \rfloor \) or \( n = \lceil \mu \rceil \) that brackets this value \( \mu = \Abs{z}^2 \) is the most probable.  So, if an energy measurement is made of a coherent state \( \ket{z} \), the most probable value will be one of

\begin{dmath}\label{eqn:gradQuantumProblemSet2Problem1:701}
E = \Hbar \lr{
\largestIntLessThan{\Abs{z}^2}
 + \inv{2} },
\end{dmath}

or

\begin{dmath}\label{eqn:gradQuantumProblemSet2Problem1:721}
E = \Hbar \lr{
\largestIntGreaterThan{\Abs{z}^2}
 + \inv{2} },
\end{dmath}

}
}

         % Bohm effect, magnetic interaction, and Landau levels:
         %
% Copyright � 2015 Peeter Joot.  All Rights Reserved.
% Licenced as described in the file LICENSE under the root directory of this GIT repository.
%
\makeoproblem{Aharonov Bohm effect.}{gradQuantum:problemSet3:1}{phy1520 2015 ps3.1}
{
\index{Aharonov-Bohm effect}

\makesubproblem{}{gradQuantum:problemSet3:1a}
Consider Young's double slit experiment with electrons, having a monoenergetic source of electrons hitting a double slit with slit spacing \( d \), with the electrons then landing on a screen at a distance \( D \) away from the double slit.
For electrons with energy E, find the de Broglie wavelength \( \lambda \), and hence the spacing between the fringes on the screen.
You can ignore the drop in intensity as the electron beam `spreads' when it travels from the slits to the screen (recall that the slits act as effective point sources), so just take phase changes into account along the travel path.

\makesubproblem{}{gradQuantum:problemSet3:1b}
Next, imagine a thin solenoidal flux \( \Phi \) being placed between the two slits, so that electron paths which encircle the flux once will pick up an Aharonov Bohm phase \( e \Phi/\Hbar c \).
Compute the resulting shift in the interference pattern on the screen.
Show that when the flux \( \Phi \) is increased from \( 0 \rightarrow h c/e \), the interference pattern shifts by exactly one fringe, so the new pattern appears the same as the old.
This is the same flux periodicity we saw in class for the energy levels versus flux for a particle on a ring.

} % makeproblem

\makeanswer{gradQuantum:problemSet3:1}{
\withproblemsetsParagraph{
\makeSubAnswer{}{gradQuantum:problemSet3:1a}

In general, the superposition of two equal amplitude wave packets with the same wavelength can be factored into a phase and amplitude

\begin{dmath}\label{eqn:gradQuantumProblemSet3Problem1:20}
e^{i (\omega t - k L_1)} + e^{i (\omega t - k L_1)}
=
e^{i (\omega t - k L_1/2 - k L_2/2)}
\lr{
e^{-i k L_1/2 + i k L_2/2}
+
e^{i k L_1/2 - i k L_2/2}
}
=
2 e^{i(\omega t - k L_1/2 - k L_2/2)} \cos\lr{ k (L_1 - L_2)/2 }.
\end{dmath}

Now consider the geometry of this screen configuration as sketched in \cref{fig:ps3p1:ps3p1Fig1}.

\imageFigure{../../figures/phy1520/ps3p1Fig1}{Double slit interference.}{fig:ps3p1:ps3p1Fig1}{0.2}

The upper and lower path lengths are

\begin{dmath}\label{eqn:gradQuantumProblemSet3Problem1:40}
L_{1,2}
= \sqrt{ D^2 + (y \mp d/2)^2 }
= D \sqrt{ 1 + \frac{(y \mp d/2)^2}{D^2} }
\approx D \lr{ 1 + \inv{2} \frac{(y \mp d/2)^2}{D^2} }
= D + \inv{2 D} (y \mp d/2)^2.
\end{dmath}

To first order the length difference is

\begin{dmath}\label{eqn:gradQuantumProblemSet3Problem1:60}
L_1 - L_2
=
\inv{2 D} (y - d/2)^2
-\inv{2 D} (y + d/2)^2
=
-\frac{y d}{D}.
\end{dmath}

The amplitude of the interference pattern, at height \( y \) on the screen is gated by the cosine

\begin{dmath}\label{eqn:gradQuantumProblemSet3Problem1:80}
\cos\lr{ \frac{k y d}{ 2 D} }.
\end{dmath}

This has peaks and zeros separated by

\begin{dmath}\label{eqn:gradQuantumProblemSet3Problem1:100}
\frac{k \Delta y d}{ 2 D} = \pi,
\end{dmath}

or
\begin{dmath}\label{eqn:gradQuantumProblemSet3Problem1:120}
\Delta y = \frac{2 \pi D}{k d}.
\end{dmath}

The electron wave number ( \( k = 2 \pi/\lambda \) ) is

\begin{dmath}\label{eqn:gradQuantumProblemSet3Problem1:140}
k = \frac{\sqrt{2 m E}}{\Hbar},
\end{dmath}

so the peak separation is

\boxedEquation{eqn:gradQuantumProblemSet3Problem1:160}{
\Delta y
%= \frac{2 \pi D \Hbar}{d \sqrt{2 m E}}
= \frac{D h}{d \sqrt{2 m E}}.
}

\makeSubAnswer{}{gradQuantum:problemSet3:1b}

Suppose the upper electron path has a positive orientation with respect to the vector potential direction, while the lower electron path has a negative orientation.

The sum of the wave packets will have the form

\begin{dmath}\label{eqn:gradQuantumProblemSet3Problem1:180}
\begin{aligned}
e^{i \omega t - k L_1 - e \Phi/\Hbar c}
&+
e^{i \omega t - k L_2 + e \Phi/\Hbar c} \\
&=
2
e^{i (\omega t - k L_1/2 - k L_2/2)}
\lr{
e^{-i k L_1/2 + i k L_2/2 - e \Phi/\Hbar c}
+
e^{i k L_1/2 - i k L_2/2 + e \Phi/\Hbar c}
} \\
&=
2 e^{i(\omega t - k L_1/2 - k L_2/2)} \cos\lr{ k (L_1 - L_2)/2 + \frac{e \Phi}{\Hbar c}}.
\end{aligned}
\end{dmath}

As \( \Phi \rightarrow c h/e \) the additional phase term approaches

\begin{dmath}\label{eqn:gradQuantumProblemSet3Problem1:200}
\frac{h}{\Hbar} = 2 \pi,
\end{dmath}

so the entire interference pattern is shifted exactly one full cycle.
}
}

         %
% Copyright � 2015 Peeter Joot.  All Rights Reserved.
% Licenced as described in the file LICENSE under the root directory of this GIT repository.
%
\makeproblem{Landau Levels - Symmetric gauge}{gradQuantum:problemSet3:2}{ 

Consider a charged particle moving in two dimensions (\(xy\)-plane) in a uniform magnetic field \( B_0 \zcap \) perpendicular
to the plane.
Let us work in a different gauge from the Landau gauge we discussed in class, namely, let us set

\begin{dmath}\label{eqn:gradQuantumProblemSet3Problem2:20}
\BA = \frac{B}{2} \lr{ x \ycap - y \xcap },
\end{dmath}

where \( (x,y) \) denotes the particle position.
This is called the `symmetric gauge'.

In this gauge, 
\makesubproblem{}{gradQuantum:problemSet3:2a}
work out the energy spectrum,
\makesubproblem{}{gradQuantum:problemSet3:2b}
the eigenfunctions,
\makesubproblem{}{gradQuantum:problemSet3:2c}
and provide a crude counting of the number of states per energy level (i.e., the degeneracy) for an electron on a disk of radius \( R \).
} % makeproblem

\makeanswer{gradQuantum:problemSet3:2}{ 

\makeSubAnswer{}{gradQuantum:problemSet3:2a}
TODO.
\makeSubAnswer{}{gradQuantum:problemSet3:2b}
TODO.
\makeSubAnswer{}{gradQuantum:problemSet3:2c}
TODO.
}


         %
% Copyright � 2015 Peeter Joot.  All Rights Reserved.
% Licenced as described in the file LICENSE under the root directory of this GIT repository.
%
\makeproblem{Aharanov Bohm effect}{gradQuantum:problemSet3:3}{ 

Consider an electron confined to the interior of a finite hollow cylinder with its axis being \( \zcap \).
Let the inner and outer
walls of the cylinder be at radial coordinates \( \rho_a \) and \( \rho_b > \rho_a \) respectively.
Let the cylinder have its top and bottom ends at \( z = 0,L \).

\makesubproblem{}{gradQuantum:problemSet3:3a}
Find the eigenstates for a particle confined to this cylinder (ignore normalization), and show that its energies are given by

\begin{equation}\label{eqn:gradQuantumProblemSet3Problem3:20}
E_{l m n} = \frac{\Hbar^2}{2 m } \lr{ k_{m n}^2 + \lr{ \frac{ \pi l }{L} }^2 } \qquad ( l = 1,2,3, \cdots ; m = 0, 1, 2, \cdots )
\end{equation}

where \( k_{m n} \) is the $n$th root of the equation

\begin{dmath}\label{eqn:gradQuantumProblemSet3Problem3:40}
J_m (k_{m n} \rho_b ) N_m (k_{m n} \rho_a ) - N_m (k_{m n} \rho_b ) J_m (k_{m n} \rho_a ) = 0.
\end{dmath}

\makesubproblem{}{gradQuantum:problemSet3:3b}
Repeat this problem with a uniform magnetic field \( B \zcap \) which is confined to the region \( 0 < \rho < \rho_a \) (i.e., only in the hollow part of the cylinder). Show that there is a periodicity of the energy levels with the field, with the period
being such that \( \pi \rho_a^2 B = 2 \pi N \hbar/e \).
} % makeproblem

\makeanswer{gradQuantum:problemSet3:3}{ 
\makeSubAnswer{}{gradQuantum:problemSet3:3a}

TODO.
\makeSubAnswer{}{gradQuantum:problemSet3:3b}

TODO.
}


   \chapter{Theory of angular momentum}
      %
% Copyright � 2015 Peeter Joot.  All Rights Reserved.
% Licenced as described in the file LICENSE under the root directory of this GIT repository.
%
%\newcommand{\authorname}{Peeter Joot}
\newcommand{\email}{peeterjoot@protonmail.com}
\newcommand{\basename}{FIXMEbasenameUndefined}
\newcommand{\dirname}{notes/FIXMEdirnameUndefined/}

%\renewcommand{\basename}{densityMatrixEntropy}
%\renewcommand{\dirname}{notes/phy1520/}
%%\newcommand{\dateintitle}{}
%%\newcommand{\keywords}{}
%
%\newcommand{\authorname}{Peeter Joot}
\newcommand{\onlineurl}{http://sites.google.com/site/peeterjoot2/math2013/\basename.pdf}
\newcommand{\sourcepath}{\dirname\basename.tex}
\newcommand{\generatetitle}[1]{\chapter{#1}}

\newcommand{\vcsinfo}{%
\section*{}
\noindent{\color{DarkOliveGreen}{\rule{\linewidth}{0.1mm}}}
\paragraph{Document version}
%\paragraph{\color{Maroon}{Document version}}
{
\small
\begin{itemize}
\item Available online at:\\ 
\href{\onlineurl}{\onlineurl}
\item Git Repository: \input{./.revinfo/gitRepo.tex}
\item Source: \sourcepath
\item last commit: \input{./.revinfo/gitCommitString.tex}
\item commit date: \input{./.revinfo/gitCommitDate.tex}
\end{itemize}
}
}

%\PassOptionsToPackage{dvipsnames,svgnames}{xcolor}
\PassOptionsToPackage{square,numbers}{natbib}
\documentclass{scrreprt}

\usepackage[left=2cm,right=2cm]{geometry}
\usepackage[svgnames]{xcolor}
\usepackage{peeters_layout}

\usepackage{natbib}

\usepackage[
colorlinks=true,
bookmarks=false,
pdfauthor={\authorname, \email},
backref 
]{hyperref}

% http://tex.stackexchange.com/questions/75773/how-to-reference-problems-by-the-text-label-in-an-exercise-envioronment
\usepackage[english]{cleveref}
\crefname{Exercise}{exercise}{exercises}
\Crefname{Exercise}{Exercise}{Exercises}

\RequirePackage{titlesec}
\RequirePackage{ifthen}

% http://stackoverflow.com/questions/4932910/date-in-the-tabular-environment
\makeatletter
\let\insertdate\@date
\makeatother

\titleformat{\chapter}[display]
{\bfseries\Large}
{\color{DarkSlateGrey}\filleft \authorname
\ifthenelse{\isundefined{\studentnumber}}{}{\\ \studentnumber}
\ifthenelse{\isundefined{\email}}{}{\\ \email}
\ifthenelse{\isundefined{\dateintitle}}{}{\\ \insertdate}
%\ifthenelse{\isundefined{\coursename}}{}{\\ \coursename} % put in title instead.
}
{4ex}
{\color{DarkOliveGreen}{\titlerule}\color{Maroon}
\vspace{2ex}%
\filright}
[\vspace{2ex}%
\color{DarkOliveGreen}\titlerule
]

\newcommand{\beginArtWithToc}[0]{\begin{document}\tableofcontents}
\newcommand{\beginArtNoToc}[0]{\begin{document}}
\newcommand{\EndNoBibArticle}[0]{\end{document}}
\newcommand{\EndArticle}[0]{\bibliography{Bibliography}\bibliographystyle{plainnat}\end{document}}

% 
%\newcommand{\citep}[1]{\cite{#1}}

\colorSectionsForArticle


%
%\usepackage{peeters_layout_exercise}
%\usepackage{peeters_braket}
%\usepackage{peeters_figures}
%
%\beginArtNoToc
%
%\generatetitle{Entropy when density operator has zero eigenvalues}
%%\label{chap:densityMatrixEntropy}

In the class notes and the text \citep{sakurai2014modern} the Von Neumann entropy is defined as

\begin{dmath}\label{eqn:densityMatrixEntropy:20}
S = -\tr(\rho \ln \rho).
\end{dmath}

In one of our problems I had trouble evaluating this, having calculated a density operator matrix representation

\begin{dmath}\label{eqn:densityMatrixEntropy:40}
\rho = E \wedge E^{-1},
\end{dmath}

where

\begin{dmath}\label{eqn:densityMatrixEntropy:60}
E = \inv{\sqrt{2}}
\begin{bmatrix}
1 & 1 \\
1 & -1
\end{bmatrix},
\end{dmath}

and
\begin{dmath}\label{eqn:densityMatrixEntropy:100}
\wedge =
\begin{bmatrix}
1 & 0 \\
0 & 0
\end{bmatrix}.
\end{dmath}

The usual method of evaluating a function of a matrix is to assume the function has a power series representation, and that a similarity transformation of the form \( A = E \wedge E^{-1} \) is possible, so that

\begin{dmath}\label{eqn:densityMatrixEntropy:80}
f(A) = E f(\wedge) E^{-1},
\end{dmath}

however, when attempting to do this with the matrix of \cref{eqn:densityMatrixEntropy:40} leads to an undesirable result

\begin{dmath}\label{eqn:densityMatrixEntropy:120}
\ln \rho =
\inv{2}
\begin{bmatrix}
1 & 1 \\
1 & -1
\end{bmatrix}
\begin{bmatrix}
\ln 1 & 0 \\
0 & \ln 0
\end{bmatrix}
\begin{bmatrix}
1 & 1 \\
1 & -1
\end{bmatrix}.
\end{dmath}

The \( \ln 0 \) makes the evaluation of this matrix logarithm rather unpleasant.  To give meaning to the entropy expression, we have to do two things, the first is treating the trace operation as a higher precedence than the logarithms that it contains.  That is

\begin{dmath}\label{eqn:densityMatrixEntropy:140}
-\tr ( \rho \ln \rho )
=
-\tr ( E \wedge E^{-1} E \ln \wedge E^{-1} )
=
-\tr ( E \wedge \ln \wedge E^{-1} )
=
-\tr ( E^{-1} E \wedge \ln \wedge )
=
-\tr ( \wedge \ln \wedge )
=
- \sum_k \wedge_{kk} \ln \wedge_{kk}.
\end{dmath}

Now the matrix of the logarithm need not be evaluated, but we still need to give meaning to \( \wedge_{kk} \ln \wedge_{kk} \) for zero diagonal entries.  This can be done by considering a limiting scenario

\begin{dmath}\label{eqn:densityMatrixEntropy:160}
-\lim_{a \rightarrow 0} a \ln a
=
-\lim_{x \rightarrow \infty} e^{-x} \ln e^{-x}
=
\lim_{x \rightarrow \infty} x e^{-x}
=
0.
\end{dmath}

The entropy can now be expressed in the unambiguous form, summing over all the non-zero eigenvalues of the density operator

%\begin{dmath}\label{eqn:densityMatrixEntropy:180}
\boxedEquation{eqn:densityMatrixEntropy:180}{
S = - \sum_{ \wedge_{kk} \ne 0} \wedge_{kk} \ln \wedge_{kk}.
}
%\end{dmath}

%\EndArticle

      \section{Problems}
         %
% Copyright � 2015 Peeter Joot.  All Rights Reserved.
% Licenced as described in the file LICENSE under the root directory of this GIT repository.
%
\makeoproblem{Density matrix.}{gradQuantum:problemSet1:1}{phy1520 2015 ps1.1}{
\index{density matrix}

Consider a spin-1/2 particle. The Hilbert space is two-dimensional, let us label the two states as \( \ket{\uparrow} \) and \( \ket{\downarrow} \).
Write down the \( 2 \times 2 \) density matrix which corresponds to the following pure states.

\begin{enumerate}[(i)]
\item \( \ket{\uparrow} \)
\item \( \inv{\sqrt{2}} \lr{ \ket{\uparrow} + \ket{\downarrow} } \)
\item \( \inv{\sqrt{2}} \lr{ \ket{\uparrow} + i \ket{\downarrow} } \)
\item At time \( t = 0 \), let us start with the state \( \inv{\sqrt{2}} \lr{ \ket{\uparrow} + i \ket{\downarrow} } \),
and consider time-evolution under the Hamiltonian \( \hatH = - B S_z \),
where \( S_z \) is the z-component of the spin operator.
This leads to eigenstates \( \ket{\uparrow} \) with energy \( -B \Hbar/2 \), and
\( \ket{\downarrow} \)
with energy \( +B \Hbar/2 \).
The state
\( \inv{\sqrt{2}} \lr{ \ket{\uparrow} + i \ket{\downarrow} } \)
is not an eigenstate of this Hamiltonian and it will evolve in time.
Find the state and the corresponding \( 2 \times 2 \) density matrix of this system at a later time \( t \).
\end{enumerate}

} % makeproblem

\makeanswer{gradQuantum:problemSet1:1}{
\withproblemsetsParagraph{
\begin{enumerate}[(i)]
\item
This state in matrix form is

\begin{dmath}\label{eqn:gradQuantumProblemSet1Problem1:20}
\ket{\psi} =
\begin{bmatrix}
1 \\
0
\end{bmatrix},
\end{dmath}

for which the density operator representation is

\begin{dmath}\label{eqn:gradQuantumProblemSet1Problem1:40}
\hat\rho
=
\begin{bmatrix}
1 \\
0
\end{bmatrix}
\begin{bmatrix}
1 &
0
\end{bmatrix}
=
\begin{bmatrix}
1 & 0 \\
0 & 0
\end{bmatrix}.
\end{dmath}

\item

This state in matrix form is

\begin{dmath}\label{eqn:gradQuantumProblemSet1Problem1:60}
\ket{\psi} =
\inv{\sqrt{2}}
\lr{
\begin{bmatrix}
1 \\
0
\end{bmatrix}
+
\begin{bmatrix}
0 \\
1
\end{bmatrix}
}
=
\inv{\sqrt{2}}
\begin{bmatrix}
1 \\
1
\end{bmatrix}
\end{dmath}

for which the density operator representation is

\begin{dmath}\label{eqn:gradQuantumProblemSet1Problem1:80}
\hat\rho
=
\inv{2}
\begin{bmatrix}
1 \\
1
\end{bmatrix}
\begin{bmatrix}
1 &
1
\end{bmatrix}
=
\inv{2}
\begin{bmatrix}
1 & 1 \\
1 & 1
\end{bmatrix}.
\end{dmath}

\item

This state in matrix form is

\begin{dmath}\label{eqn:gradQuantumProblemSet1Problem1:100}
\ket{\psi} =
\inv{\sqrt{2}}
\lr{
\begin{bmatrix}
1 \\
0
\end{bmatrix}
+
\begin{bmatrix}
0 \\
i
\end{bmatrix}
}
=
\inv{\sqrt{2}}
\begin{bmatrix}
1 \\
i
\end{bmatrix}
\end{dmath}

for which the density operator representation is

\begin{dmath}\label{eqn:gradQuantumProblemSet1Problem1:120}
\hat\rho
=
\inv{2}
\begin{bmatrix}
1 \\
i
\end{bmatrix}
\begin{bmatrix}
1 &
-i
\end{bmatrix}
=
\inv{2}
\begin{bmatrix}
1 & -i \\
i & 1
\end{bmatrix}.
\end{dmath}

\item

The time evolution operator is

\begin{dmath}\label{eqn:gradQuantumProblemSet1Problem1:140}
U(t)
= e^{-i \hatH t/\Hbar}
= e^{i B S_z t/\Hbar}
= e^{i B \sigma_z t/2}
= \cos( B t/2 ) + i \sigma_z \sin( B t/2)
=
\begin{bmatrix}
\cos(B t/2) & 0 \\
0 & \cos(B t/2) \\
\end{bmatrix}
+
\begin{bmatrix}
i \sin(B t/2) & 0 \\
0 & -i \sin(B t/2) \\
\end{bmatrix}
=
\begin{bmatrix}
e^{i B t/2} & 0 \\
0 & e^{-i B t/2}
\end{bmatrix}.
\end{dmath}

so the time evolved state is

\begin{dmath}\label{eqn:gradQuantumProblemSet1Problem1:160}
\ket{\psi(t)} =
\inv{\sqrt{2}}
\begin{bmatrix}
e^{i B t/2} & 0 \\
0 & e^{-i B t/2}
\end{bmatrix}
\begin{bmatrix}
1 \\
i
\end{bmatrix}
=
\inv{\sqrt{2}}
\begin{bmatrix}
e^{i B t/2} \\
i e^{-i B t/2}
\end{bmatrix}.
\end{dmath}

The density operator matrix representation is

\begin{dmath}\label{eqn:gradQuantumProblemSet1Problem1:180}
\hat\rho(t)
=
\inv{2}
\begin{bmatrix}
e^{i B t/2} \\
i e^{-i B t/2}
\end{bmatrix}
\begin{bmatrix}
e^{-i B t/2} & -i e^{i B t/2}
\end{bmatrix}
=
\inv{2}
\begin{bmatrix}
1 & -i e^{i B t} \\
i e^{-i B t} & 1
\end{bmatrix}.
\end{dmath}

As a check observe that this has the right value for \( t = 0 \).  This also checks against the slightly messier computation \( \hatp(t) = U \hatp(0) U^\dagger \).

\end{enumerate}
}
}

         %
% Copyright � 2015 Peeter Joot.  All Rights Reserved.
% Licenced as described in the file LICENSE under the root directory of this GIT repository.
%
\makeproblem{Reduced density matrix}{gradQuantum:problemSet1:2}{ 

Consider two spin-1/2 particles, the Hilbert space is now 4-dimensional,
with states 
\( \ket{ \uparrow \uparrow } \),
\( \ket{ \uparrow \downarrow } \),
\( \ket{ \downarrow \uparrow } \),
\( \ket{ \downarrow \downarrow } \).
Let us consider the following pure states:

\begin{enumerate}
\item \( \inv{2} \lr{
 \ket{ \uparrow \uparrow } 
- \ket{ \uparrow \downarrow } 
- \ket{ \downarrow \downarrow } 
+ \ket{ \downarrow \uparrow }
} \)
\item \( \inv{\sqrt{2}} \lr{
 \ket{ \uparrow \uparrow } 
+ \ket{ \downarrow \downarrow } 
} \)
\item \( \inv{\sqrt{5}} \lr{
 \ket{ \uparrow \uparrow } 
+ 2  \ket{ \downarrow \downarrow } 
} \)
\end{enumerate}
In each case, obtain the reduced \( 2 \times 2 \) density matrix which describes the first spin, when we trace over the
second spin. The von Neumann entanglement entropy is defined via 
\( S_{\textrm{v N}} = -\Tr(\rho_R \ln \rho_R ) \) where \( \rho_R \) is the reduced
density matrix you have obtained above and the \( \Tr \) now refers to tracing over the first spin. Using the reduced
density matrices you have obtained above, compute the corresponding 
\( S_{\textrm{v N}} \),
and simply explain your result in words.
Consider the Renyi entropy 
\( S_n =
\inv{1-n}
\ln \lr{ \Tr( \rho_R^n ) } \).
Prove that 
\( S_{n \rightarrow 1} = S_{\textrm{v N}} \),
and compute \( S_{n=2} \) for the above \( \rho_R \).
} % makeproblem

\makeanswer{gradQuantum:problemSet1:2}{ 

TODO.
}

   \chapter{Dirac equation in 1D}
      %
% Copyright � 2015 Peeter Joot.  All Rights Reserved.
% Licenced as described in the file LICENSE under the root directory of this GIT repository.
%
%\newcommand{\authorname}{Peeter Joot}
\newcommand{\email}{peeterjoot@protonmail.com}
\newcommand{\basename}{FIXMEbasenameUndefined}
\newcommand{\dirname}{notes/FIXMEdirnameUndefined/}

%\renewcommand{\basename}{qmLecture8}
%\renewcommand{\dirname}{notes/phy1520/}
%\newcommand{\keywords}{PHY1520H}
%\newcommand{\authorname}{Peeter Joot}
\newcommand{\onlineurl}{http://sites.google.com/site/peeterjoot2/math2013/\basename.pdf}
\newcommand{\sourcepath}{\dirname\basename.tex}
\newcommand{\generatetitle}[1]{\chapter{#1}}

\newcommand{\vcsinfo}{%
\section*{}
\noindent{\color{DarkOliveGreen}{\rule{\linewidth}{0.1mm}}}
\paragraph{Document version}
%\paragraph{\color{Maroon}{Document version}}
{
\small
\begin{itemize}
\item Available online at:\\ 
\href{\onlineurl}{\onlineurl}
\item Git Repository: \input{./.revinfo/gitRepo.tex}
\item Source: \sourcepath
\item last commit: \input{./.revinfo/gitCommitString.tex}
\item commit date: \input{./.revinfo/gitCommitDate.tex}
\end{itemize}
}
}

%\PassOptionsToPackage{dvipsnames,svgnames}{xcolor}
\PassOptionsToPackage{square,numbers}{natbib}
\documentclass{scrreprt}

\usepackage[left=2cm,right=2cm]{geometry}
\usepackage[svgnames]{xcolor}
\usepackage{peeters_layout}

\usepackage{natbib}

\usepackage[
colorlinks=true,
bookmarks=false,
pdfauthor={\authorname, \email},
backref 
]{hyperref}

% http://tex.stackexchange.com/questions/75773/how-to-reference-problems-by-the-text-label-in-an-exercise-envioronment
\usepackage[english]{cleveref}
\crefname{Exercise}{exercise}{exercises}
\Crefname{Exercise}{Exercise}{Exercises}

\RequirePackage{titlesec}
\RequirePackage{ifthen}

% http://stackoverflow.com/questions/4932910/date-in-the-tabular-environment
\makeatletter
\let\insertdate\@date
\makeatother

\titleformat{\chapter}[display]
{\bfseries\Large}
{\color{DarkSlateGrey}\filleft \authorname
\ifthenelse{\isundefined{\studentnumber}}{}{\\ \studentnumber}
\ifthenelse{\isundefined{\email}}{}{\\ \email}
\ifthenelse{\isundefined{\dateintitle}}{}{\\ \insertdate}
%\ifthenelse{\isundefined{\coursename}}{}{\\ \coursename} % put in title instead.
}
{4ex}
{\color{DarkOliveGreen}{\titlerule}\color{Maroon}
\vspace{2ex}%
\filright}
[\vspace{2ex}%
\color{DarkOliveGreen}\titlerule
]

\newcommand{\beginArtWithToc}[0]{\begin{document}\tableofcontents}
\newcommand{\beginArtNoToc}[0]{\begin{document}}
\newcommand{\EndNoBibArticle}[0]{\end{document}}
\newcommand{\EndArticle}[0]{\bibliography{Bibliography}\bibliographystyle{plainnat}\end{document}}

% 
%\newcommand{\citep}[1]{\cite{#1}}

\colorSectionsForArticle


%
%%\usepackage{phy1520}
%\usepackage{peeters_braket}
%%\usepackage{peeters_layout_exercise}
%\usepackage{peeters_figures}
%\usepackage{mathtools}
%
%\beginArtNoToc
%\generatetitle{PHY1520H Graduate Quantum Mechanics.  Lecture 8: Dirac equation in 1D.  Taught by Prof.\ Arun Paramekanti}
\label{chap:qmLecture8}

%\paragraph{Disclaimer}
%
%Peeter's lecture notes from class.  These may be incoherent and rough.
%
%These are notes for the UofT course PHY1520, Graduate Quantum Mechanics, taught by Prof. Paramekanti.
%
\section{Construction of the Dirac equation}
\index{Dirac equation}
\paragraph{Schr\"{o}dinger Derivation}

Recall that a ``derivation'' of the Schr\"{o}dinger equation can be associated with the following equivalences

\begin{equation}\label{eqn:qmLecture8:300}
E \leftrightarrow \Hbar \omega \leftrightarrow i \Hbar \PD{t}{}
\end{equation}
\begin{equation}\label{eqn:qmLecture8:320}
p \leftrightarrow \Hbar k \leftrightarrow -i \Hbar \PD{t}{}
\end{equation}

so that the classical energy relationship

\begin{dmath}\label{eqn:qmLecture8:20}
E = \frac{p^2}{2m}
\end{dmath}

takes the form

\begin{dmath}\label{eqn:qmLecture8:40}
i \Hbar \PD{t}{} = -\frac{\Hbar^2}{2m}.
\end{dmath}

How do we do this in a relativistic context where the energy momentum relationship is

\begin{dmath}\label{eqn:qmLecture8:60}
E = \sqrt{ p^2 c^2 + m^2 c^4 } \approx m c^2 + \frac{p^2}{2m} + \cdots
\end{dmath}

where \( m \) is the rest mass and \( c \) is the speed of light.

\paragraph{Attempt I}

\begin{dmath}\label{eqn:qmLecture8:80}
E = m c^2 + \frac{p^2}{2m} + (...) p^4 + (...) p^6 + \cdots
\end{dmath}

First order in time, but infinite order in space \( \partial/\partial x \).  Useless.

\paragraph{Attempt II}

\begin{dmath}\label{eqn:qmLecture8:100}
E^2 = p^2 c^2 + m^2 c^4.
\end{dmath}

This gives

\begin{dmath}\label{eqn:qmLecture8:120}
-\Hbar^2 \PDSq{t}{\psi} = - \Hbar^2 c^2 \PDSq{x}{\psi} + m^2 c^4 \psi.
\end{dmath}

\index{Klein-Gordon equation}
This is the Klein-Gordon equation, which is second order in time.

\paragraph{Attempt III}

Suppose that we have the matrix

\begin{dmath}\label{eqn:qmLecture8:140}
\begin{bmatrix}
p c & m c^2 \\
m c^2 & - p c
\end{bmatrix},
\end{dmath}

or

\begin{dmath}\label{eqn:qmLecture8:160}
\begin{bmatrix}
m c^2 & i p c \\
-i p c & - m c^2
\end{bmatrix},
\end{dmath}

These both happen to have eigenvalues \( \lambda_{\pm} = \pm \sqrt{p^2 c^2} \).  For those familiar with the Dirac matrices, this amounts to a choice for different representations of the gamma matrices.

Working with \cref{eqn:qmLecture8:140}, which has some nicer features than other possible representations, we seek a state

\begin{dmath}\label{eqn:qmLecture8:180}
\Bpsi =
\begin{bmatrix}
\psi_1(x, t) \\
\psi_2(x, t) \\
\end{bmatrix},
\end{dmath}

where we aim to write down an equation for this composite state.

\begin{dmath}\label{eqn:qmLecture8:200}
i \Hbar \PD{t}{\Bpsi} = \BH \Bpsi
\end{dmath}

Assuming the matrix is the Hamiltonian, multiplying that with the composite state gives

\begin{dmath}\label{eqn:qmLecture8:220}
\begin{bmatrix}
i \Hbar \PD{t}{\psi_1} \\
i \Hbar \PD{t}{\psi_1}
\end{bmatrix}
=
\begin{bmatrix}
\hatp c & m c^2 \\
m c^2 & - \hatp c
\end{bmatrix}
\begin{bmatrix}
\psi_1(x, t) \\
\psi_2(x, t) \\
\end{bmatrix}
=
\begin{bmatrix}
\hatp c \psi_1 + m c^2 \psi_2  \\
m c^2 \psi_1 - \hatp c \psi_2
\end{bmatrix}.
\end{dmath}

What happens when we square this

\begin{dmath}\label{eqn:qmLecture8:240}
\lr{ i \Hbar \PD{t}{} }^2 \Bpsi
= \BH \BH \Bpsi
=
\begin{bmatrix}
\hatp c & m c^2 \\
m c^2 & - \hatp c
\end{bmatrix}
\begin{bmatrix}
\hatp c & m c^2 \\
m c^2 & - \hatp c
\end{bmatrix}
\Bpsi
=
\begin{bmatrix}
\hatp^2 c^2 + m^2 c^4 & 0 \\
0 & \hatp^2 c^2 + m^2 c^4 \\
\end{bmatrix}
\end{dmath}

\begin{dmath}\label{eqn:qmLecture8:260}
- \Hbar^2 \PDSq{t}{} \Bpsi
=
\lr{ \hatp^2 c^2 + m^2 c^4 } \BOne \Bpsi,
\end{dmath}

or more exactly

\begin{dmath}\label{eqn:qmLecture8:280}
- \Hbar^2 \PDSq{t}{} \psi_{1,2}
=
\lr{ \hatp^2 c^2 + m^2 c^4 } \psi_{1,2}.
\end{dmath}

This recovers the Klein Gordon equation for each of the wave functions \( \psi_1, \psi_2 \).

\section{Plane wave solution}
\index{Dirac equation!plane wave}

Instead of squaring the operators, lets try to solve the first order equation.     To do so we'll want to diagonalize \( \BH \).

Before doing that, let's write out the Hamiltonian in an alternate but useful form

\begin{dmath}\label{eqn:qmLecture8:340}
\BH =
\hatp c
\begin{bmatrix}
1 & 0 \\
0 & -1
\end{bmatrix}
+
m c^2
\begin{bmatrix}
0 & 1 \\
1 & 0
\end{bmatrix}
= \hatp c \hat\sigma_z + m c^2 \hat\sigma_x.
\end{dmath}

We have two types of operators in the mix here.  We have matrix operators that act on the wave function matrices, as well as derivative operators that act on the components of those matrices.

We have

\begin{dmath}\label{eqn:qmLecture8:360}
\hat\sigma_z
\begin{bmatrix}
\psi_1 \\
\psi_2 \\
\end{bmatrix}
=
\begin{bmatrix}
\psi_1 \\
-\psi_2 \\
\end{bmatrix},
\end{dmath}

and

\begin{dmath}\label{eqn:qmLecture8:380}
\hat\sigma_x
\begin{bmatrix}
\psi_1 \\
\psi_2 \\
\end{bmatrix}
=
\begin{bmatrix}
\psi_2 \\
\psi_1 \\
\end{bmatrix}.
\end{dmath}

Because the derivative actions of \( \hatp \) and the matrix operators are independent, we see that these operators commute.  For example

\begin{dmath}\label{eqn:qmLecture8:400}
\hat\sigma_z \hatp
\begin{bmatrix}
\psi_1 \\
\psi_2 \\
\end{bmatrix}
=
\hat\sigma_z
\begin{bmatrix}
-i \Hbar \PD{x}{\psi_1} \\
-i \Hbar \PD{x}{\psi_2} \\
\end{bmatrix}
=
\begin{bmatrix}
-i \Hbar \PD{x}{\psi_1} \\
i \Hbar \PD{x}{\psi_2} \\
\end{bmatrix}
=
\hatp
\hat\sigma_z
\begin{bmatrix}
\psi_1 \\
\psi_2 \\
\end{bmatrix}.
\end{dmath}

\paragraph{Diagonalizing it}
\index{Dirac equation!diagonalization}

Suppose the wave function matrix has the structure

\begin{dmath}\label{eqn:qmLecture8:420}
\Bpsi =
\begin{bmatrix}
f_{+} \\
f_{-} \\
\end{bmatrix}
e^{i k x}.
\end{dmath}

We'll plug this into the Schr\"{o}dinger equation and see what we get.

%\EndNoBibArticle

      %
% Copyright � 2015 Peeter Joot.  All Rights Reserved.
% Licenced as described in the file LICENSE under the root directory of this GIT repository.
%
%\newcommand{\authorname}{Peeter Joot}
\newcommand{\email}{peeterjoot@protonmail.com}
\newcommand{\basename}{FIXMEbasenameUndefined}
\newcommand{\dirname}{notes/FIXMEdirnameUndefined/}

%\renewcommand{\basename}{qmLecture9}
%\renewcommand{\dirname}{notes/phy1520/}
%\newcommand{\keywords}{PHY1520H}
%\newcommand{\authorname}{Peeter Joot}
\newcommand{\onlineurl}{http://sites.google.com/site/peeterjoot2/math2013/\basename.pdf}
\newcommand{\sourcepath}{\dirname\basename.tex}
\newcommand{\generatetitle}[1]{\chapter{#1}}

\newcommand{\vcsinfo}{%
\section*{}
\noindent{\color{DarkOliveGreen}{\rule{\linewidth}{0.1mm}}}
\paragraph{Document version}
%\paragraph{\color{Maroon}{Document version}}
{
\small
\begin{itemize}
\item Available online at:\\ 
\href{\onlineurl}{\onlineurl}
\item Git Repository: \input{./.revinfo/gitRepo.tex}
\item Source: \sourcepath
\item last commit: \input{./.revinfo/gitCommitString.tex}
\item commit date: \input{./.revinfo/gitCommitDate.tex}
\end{itemize}
}
}

%\PassOptionsToPackage{dvipsnames,svgnames}{xcolor}
\PassOptionsToPackage{square,numbers}{natbib}
\documentclass{scrreprt}

\usepackage[left=2cm,right=2cm]{geometry}
\usepackage[svgnames]{xcolor}
\usepackage{peeters_layout}

\usepackage{natbib}

\usepackage[
colorlinks=true,
bookmarks=false,
pdfauthor={\authorname, \email},
backref 
]{hyperref}

% http://tex.stackexchange.com/questions/75773/how-to-reference-problems-by-the-text-label-in-an-exercise-envioronment
\usepackage[english]{cleveref}
\crefname{Exercise}{exercise}{exercises}
\Crefname{Exercise}{Exercise}{Exercises}

\RequirePackage{titlesec}
\RequirePackage{ifthen}

% http://stackoverflow.com/questions/4932910/date-in-the-tabular-environment
\makeatletter
\let\insertdate\@date
\makeatother

\titleformat{\chapter}[display]
{\bfseries\Large}
{\color{DarkSlateGrey}\filleft \authorname
\ifthenelse{\isundefined{\studentnumber}}{}{\\ \studentnumber}
\ifthenelse{\isundefined{\email}}{}{\\ \email}
\ifthenelse{\isundefined{\dateintitle}}{}{\\ \insertdate}
%\ifthenelse{\isundefined{\coursename}}{}{\\ \coursename} % put in title instead.
}
{4ex}
{\color{DarkOliveGreen}{\titlerule}\color{Maroon}
\vspace{2ex}%
\filright}
[\vspace{2ex}%
\color{DarkOliveGreen}\titlerule
]

\newcommand{\beginArtWithToc}[0]{\begin{document}\tableofcontents}
\newcommand{\beginArtNoToc}[0]{\begin{document}}
\newcommand{\EndNoBibArticle}[0]{\end{document}}
\newcommand{\EndArticle}[0]{\bibliography{Bibliography}\bibliographystyle{plainnat}\end{document}}

% 
%\newcommand{\citep}[1]{\cite{#1}}

\colorSectionsForArticle


%
%%\usepackage{phy1520}
%\usepackage{peeters_braket}
%%\usepackage{peeters_layout_exercise}
%\usepackage{peeters_figures}
%\usepackage{mathtools}
%
%\beginArtNoToc
%\generatetitle{PHY1520H Graduate Quantum Mechanics.  Lecture 9: Dirac equation (cont.).  Taught by Prof.\ Arun Paramekanti}
%%\chapter{Dirac equation (cont.)}
%\label{chap:qmLecture9}
%
%\paragraph{Disclaimer}
%
%Peeter's lecture notes from class.  These may be incoherent and rough.
%
%These are notes for the UofT course PHY1520, Graduate Quantum Mechanics, taught by Prof. Paramekanti.
%
\paragraph{Where we left off}

\begin{dmath}\label{eqn:qmLecture9:20}
-i \Hbar \PD{t}{}
\begin{bmatrix}
\psi_1 \\
\psi_2
\end{bmatrix}
=
\begin{bmatrix}
-i \Hbar c \PD{x}{} & m c^2 \\
m c^2 & i \Hbar c \PD{x}{} \\
\end{bmatrix}.
\end{dmath}

\index{Dirac equation!with potential}
With a potential this would be

\begin{dmath}\label{eqn:qmLecture9:40}
-i \Hbar \PD{t}{}
\begin{bmatrix}
\psi_1 \\
\psi_2
\end{bmatrix}
=
\begin{bmatrix}
-i \Hbar c \PD{x}{} + V(x) & m c^2 \\
m c^2 & i \Hbar c \PD{x}{} + V(x) \\
\end{bmatrix}.
\end{dmath}

This means that the potential is raising the energy eigenvalue of the system.

\paragraph{Free Particle}
\index{Dirac equation!free particle}

Assuming a form

\begin{dmath}\label{eqn:qmLecture9:60}
\begin{bmatrix}
\psi_1(x,t) \\
\psi_2(x,t)
\end{bmatrix}
=
e^{i k x}
\begin{bmatrix}
f_1(t) \\
f_2(t) \\
\end{bmatrix},
\end{dmath}

and plugging back into the Dirac equation we have

\begin{dmath}\label{eqn:qmLecture9:80}
-i \Hbar \PD{t}{}
\begin{bmatrix}
f_1 \\
f_2
\end{bmatrix}
=
\begin{bmatrix}
k \Hbar c & m c^2 \\
m c^2 & - \Hbar k c \\
\end{bmatrix}
\begin{bmatrix}
f_1 \\
f_2
\end{bmatrix}.
\end{dmath}

We can use a diagonalizing rotation

\begin{dmath}\label{eqn:qmLecture9:100}
\begin{bmatrix}
f_1 \\
f_2
\end{bmatrix}
=
\begin{bmatrix}
\cos\theta_k & -\sin\theta_k \\
\sin\theta_k & \cos\theta_k \\
\end{bmatrix}
\begin{bmatrix}
f_{+} \\
f_{-} \\
\end{bmatrix}.
\end{dmath}

Plugging this in reduces the system to the form

\begin{dmath}\label{eqn:qmLecture9:140}
-i \Hbar \PD{t}{}
\begin{bmatrix}
f_{+} \\
f_{-} \\
\end{bmatrix}
=
\begin{bmatrix}
E_k & 0 \\
0 & -E_k
\end{bmatrix}
\begin{bmatrix}
f_{+} \\
f_{-} \\
\end{bmatrix}.
\end{dmath}

Where the rotation angle is found to be given by

\begin{dmath}\label{eqn:qmLecture9:160}
\begin{aligned}
\sin(2 \theta_k) &= \frac{m c^2}{\sqrt{(\Hbar k c)^2 + m^2 c^4}} \\
\cos(2 \theta_k) &= \frac{\Hbar k c}{\sqrt{(\Hbar k c)^2 + m^2 c^4}} \\
E_k &= \sqrt{(\Hbar k c)^2 + m^2 c^4}
\end{aligned}
\end{dmath}

\section{Dirac sea and pair creation}
\index{Dirac sea}
\index{pair creation}

See \cref{fig:l9:l9Fig1} for a sketch of energy vs momentum.  The asymptotes are the limiting cases when \( m c^2 \rightarrow 0 \).  The \( + \) branch is what we usually associate with particles.  What about the other energy states.  For Fermions Dirac argued that the lower energy states could be thought of as ``filled up'', using the Pauli principle to leave only the positive energy states available.  This was called the ``Dirac Sea''.  This isn't a good solution, and won't work for example for Bosons.

\imageFigure{../../figures/phy1520/l9Fig1}{Dirac equation solution space.}{fig:l9:l9Fig1}{0.2}

Another way to rationalize this is to employ ideas from solid state theory.  For example consider a semiconductor with a valence and conduction band as sketched in \cref{fig:l9:l9Fig2}.

\imageFigure{../../figures/phy1520/l9Fig2}{Solid state valence and conduction band transition.}{fig:l9:l9Fig2}{0.2}

A photon can excite an electron from the valence band to the conduction band, leaving all the valence band states filled except for one (a hole).  For an electron we can use almost the same picture, as sketched in \cref{fig:l9:l9Fig3}.

\imageFigure{../../figures/phy1520/l9Fig3}{Pair creation.}{fig:l9:l9Fig3}{0.2}

A photon with energy \( E_k - (-E_k) \) can create a positron-electron pair from the vacuum, where the energy of the electron and positron pair is \( E_k \).
%, and the energy of the positron is \( E_k \).
At high enough energies, we can see this pair creation occur.

\section{Zitterbewegung}
\index{zitterbewegung}

If a particle is created at a non-eigenstate such as one on the asymptotes, then oscillations between the positive and negative branches are possible as sketched in \cref{fig:l9:l9Fig4}.

\imageFigure{../../figures/phy1520/l9Fig4}{Zitterbewegung oscillation.}{fig:l9:l9Fig4}{0.15}

Only ``vertical" oscillations between the positive and negative locations on these branches is possible since those are the points that match the particle momentum.  Examining this will be the aim of one of the problem set problems.

\section{Probability and current density}
\index{Dirac equation!probability density}
\index{Dirac equation!current density}

If we define a probability density

\begin{dmath}\label{eqn:qmLecture9:180}
\rho(x, t) = \Abs{\psi_1}^2 + \Abs{\psi_2}^2,
\end{dmath}

does this satisfy a probability conservation relation

\begin{dmath}\label{eqn:qmLecture9:200}
\PD{t}{\rho} + \PD{x}{j} = 0,
\end{dmath}

where \( j \) is the probability current.  Plugging in the density, we have

\begin{dmath}\label{eqn:qmLecture9:220}
\PD{t}{\rho}
=
\PD{t}{\psi_1^\conj} \psi_1
+
\psi_1^\conj \PD{t}{\psi_1}
+
\PD{t}{\psi_2^\conj} \psi_2
+
\psi_2^\conj \PD{t}{\psi_2}.
\end{dmath}

It turns out that the probability current has the form

\begin{dmath}\label{eqn:qmLecture9:240}
j(x,t) = c \lr{ \psi_1^\conj \psi_1 - \psi_2^\conj \psi_2 }.
\end{dmath}

Here the speed of light \( c \) is the slope of the line in the plots above.  We can think of this current density as right movers minus the left movers.  Any state that is given can be thought of as a combination of right moving and left moving states, neither of which are eigenstates of the free particle Hamiltonian.

\section{Potential step}

The next logical thing to think about, as in non-relativistic quantum mechanics, is to think about what occurs when the particle hits a potential step, as in \cref{fig:l9:l9Fig5}.

\imageFigure{../../figures/phy1520/l9Fig5}{Reflection off a potential barrier.}{fig:l9:l9Fig5}{0.2}

The approach is the same.  We write down the wave functions for the \( V = 0 \) region (I), and the higher potential region (II).

The eigenstates are found on the solid lines above the asymptotes on the right hand movers side as sketched in \cref{fig:l9:l9Fig6}.  The right and left moving designations are based on the phase velocity \( \PDi{k}{E} \) (approaching \( \pm c \) on the top-right and top-left quadrants respectively).

\imageFigure{../../figures/phy1520/l9Fig6}{Right movers and left movers.}{fig:l9:l9Fig6}{0.2}

For \( k > 0 \), an eigenstate for the incident wave is

\begin{dmath}\label{eqn:qmLecture9:261}
\Bpsi_{\textrm{inc}}(x) =
\begin{bmatrix}
\cos\theta_k \\
\sin\theta_k
\end{bmatrix}
e^{i k x},
\end{dmath}

For the reflected wave function, we pick a function on the left moving side of the positive energy branch.

\begin{dmath}\label{eqn:qmLecture9:260}
\Bpsi_{\textrm{ref}}(x) =
\begin{bmatrix}
? \\
?
\end{bmatrix}
e^{-i k x},
\end{dmath}

We'll go through this in more detail next time.

%\EndNoBibArticle

      %
% Copyright � 2015 Peeter Joot.  All Rights Reserved.
% Licenced as described in the file LICENSE under the root directory of this GIT repository.
%
%\newcommand{\authorname}{Peeter Joot}
\newcommand{\email}{peeterjoot@protonmail.com}
\newcommand{\basename}{FIXMEbasenameUndefined}
\newcommand{\dirname}{notes/FIXMEdirnameUndefined/}

%\renewcommand{\basename}{qmLecture10}
%\renewcommand{\dirname}{notes/phy1520/}
%\newcommand{\keywords}{PHY1520H}
%\newcommand{\authorname}{Peeter Joot}
\newcommand{\onlineurl}{http://sites.google.com/site/peeterjoot2/math2013/\basename.pdf}
\newcommand{\sourcepath}{\dirname\basename.tex}
\newcommand{\generatetitle}[1]{\chapter{#1}}

\newcommand{\vcsinfo}{%
\section*{}
\noindent{\color{DarkOliveGreen}{\rule{\linewidth}{0.1mm}}}
\paragraph{Document version}
%\paragraph{\color{Maroon}{Document version}}
{
\small
\begin{itemize}
\item Available online at:\\ 
\href{\onlineurl}{\onlineurl}
\item Git Repository: \input{./.revinfo/gitRepo.tex}
\item Source: \sourcepath
\item last commit: \input{./.revinfo/gitCommitString.tex}
\item commit date: \input{./.revinfo/gitCommitDate.tex}
\end{itemize}
}
}

%\PassOptionsToPackage{dvipsnames,svgnames}{xcolor}
\PassOptionsToPackage{square,numbers}{natbib}
\documentclass{scrreprt}

\usepackage[left=2cm,right=2cm]{geometry}
\usepackage[svgnames]{xcolor}
\usepackage{peeters_layout}

\usepackage{natbib}

\usepackage[
colorlinks=true,
bookmarks=false,
pdfauthor={\authorname, \email},
backref 
]{hyperref}

% http://tex.stackexchange.com/questions/75773/how-to-reference-problems-by-the-text-label-in-an-exercise-envioronment
\usepackage[english]{cleveref}
\crefname{Exercise}{exercise}{exercises}
\Crefname{Exercise}{Exercise}{Exercises}

\RequirePackage{titlesec}
\RequirePackage{ifthen}

% http://stackoverflow.com/questions/4932910/date-in-the-tabular-environment
\makeatletter
\let\insertdate\@date
\makeatother

\titleformat{\chapter}[display]
{\bfseries\Large}
{\color{DarkSlateGrey}\filleft \authorname
\ifthenelse{\isundefined{\studentnumber}}{}{\\ \studentnumber}
\ifthenelse{\isundefined{\email}}{}{\\ \email}
\ifthenelse{\isundefined{\dateintitle}}{}{\\ \insertdate}
%\ifthenelse{\isundefined{\coursename}}{}{\\ \coursename} % put in title instead.
}
{4ex}
{\color{DarkOliveGreen}{\titlerule}\color{Maroon}
\vspace{2ex}%
\filright}
[\vspace{2ex}%
\color{DarkOliveGreen}\titlerule
]

\newcommand{\beginArtWithToc}[0]{\begin{document}\tableofcontents}
\newcommand{\beginArtNoToc}[0]{\begin{document}}
\newcommand{\EndNoBibArticle}[0]{\end{document}}
\newcommand{\EndArticle}[0]{\bibliography{Bibliography}\bibliographystyle{plainnat}\end{document}}

% 
%\newcommand{\citep}[1]{\cite{#1}}

\colorSectionsForArticle


%
%%\usepackage{phy1520}
%\usepackage{peeters_braket}
%%\usepackage{peeters_layout_exercise}
%\usepackage{peeters_figures}
%\usepackage{mathtools}
%
%\beginArtNoToc
%\generatetitle{PHY1520H Graduate Quantum Mechanics.  Lecture 10: 1D Dirac scattering off potential step.  Taught by Prof.\ Arun Paramekanti}
%%\chapter{1D Dirac scattering off potential step}
%\label{chap:qmLecture10}
%
%\paragraph{Disclaimer}
%
%Peeter's lecture notes from class.  These may be incoherent and rough.
%
%These are notes for the UofT course PHY1520, Graduate Quantum Mechanics, taught by Prof. Paramekanti.

\section{Dirac scattering off a potential step}
\index{Dirac equation!scattering}

For the non-relativistic case we have

\begin{equation}\label{eqn:qmLecture10:20}
\begin{aligned}
E < V_0 &\implies T = 0, R = 1 \\
E > V_0 &\implies T > 0, R < 1.
\end{aligned}
\end{equation}

What happens for a relativistic 1D particle?

Referring to \cref{fig:lecture10:lecture10Fig1}.

\imageFigure{../../figures/phy1520/lecture10Fig1}{Potential step.}{fig:lecture10:lecture10Fig1}{0.1}

the region I Hamiltonian is

\begin{equation}\label{eqn:qmLecture10:40}
H =
\begin{bmatrix}
\hatp c & m c^2 \\
m c^2 & - \hatp c
\end{bmatrix},
\end{equation}

for which the solution is

\begin{dmath}\label{eqn:qmLecture10:60}
\Phi = e^{i k_1 x }
\begin{bmatrix}
\cos \theta_1 \\
\sin \theta_1
\end{bmatrix},
\end{dmath}

where
\begin{dmath}\label{eqn:qmLecture10:80}
\begin{aligned}
\cos 2 \theta_1 &= \frac{ \Hbar c k_1 }{E_{k_1}} \\
\sin 2 \theta_1 &= \frac{ m c^2 }{E_{k_1}} \\
\end{aligned}
\end{dmath}

To consider the \( k_1 < 0 \) case, note that

\begin{equation}\label{eqn:qmLecture10:100}
\begin{aligned}
\cos^2 \theta_1 - \sin^2 \theta_1 &= \cos 2 \theta_1 \\
2 \sin\theta_1 \cos\theta_1 &= \sin 2 \theta_1
\end{aligned}
\end{equation}

so after flipping the signs on all the \( k_1 \) terms we find for the reflected wave

\begin{dmath}\label{eqn:qmLecture10:120}
\Phi = e^{-i k_1 x}
\begin{bmatrix}
\sin\theta_1 \\
\cos\theta_1
\end{bmatrix}.
\end{dmath}

FIXME: this reasoning doesn't entirely make sense to me.  Make sense of this by trying this solution as was done for the form of the incident wave solution.

The region I wave has the form

\begin{dmath}\label{eqn:qmLecture10:140}
\Phi_I
=
A e^{i k_1 x}
\begin{bmatrix}
\cos\theta_1 \\
\sin\theta_1 \\
\end{bmatrix}
+
B e^{-i k_1 x}
\begin{bmatrix}
\sin\theta_1 \\
\cos\theta_1 \\
\end{bmatrix}.
\end{dmath}

By the time we are done we want to have computed the reflection coefficient

\begin{dmath}\label{eqn:qmLecture10:160}
R =
\frac{\Abs{B}^2}{\Abs{A}^2}.
\end{dmath}

The region I energy is

\begin{dmath}\label{eqn:qmLecture10:180}
E = \sqrt{ \lr{ m c^2}^2 + \lr{ \Hbar c k_1 }^2 }.
\end{dmath}

We must have
\begin{equation}\label{eqn:qmLecture10:200}
E
=
\sqrt{ \lr{ m c^2}^2 + \lr{ \Hbar c k_2 }^2 } + V_0
=
\sqrt{ \lr{ m c^2}^2 + \lr{ \Hbar c k_1 }^2 },
\end{equation}

so

\begin{dmath}\label{eqn:qmLecture10:220}
\lr{ \Hbar c k_2 }^2
=
\lr{ E - V_0 }^2 - \lr{ m c^2}^2
=
\mathLabelBox
[ labelstyle={below of=m\themathLableNode, below of=m\themathLableNode} ]
{\lr{ E - V_0 + m c^2 }}{\(r_1\)}
\mathLabelBox
[ labelstyle={below of=m\themathLableNode, below of=m\themathLableNode} ]
{\lr{ E - V_0 - m c^2 }}{\(r_2\)}.
\end{dmath}

The \( r_1 \) and \( r_2 \) branches are sketched in \cref{fig:lecture10:lecture10Fig2}.

\imageFigure{../../figures/phy1520/lecture10Fig2}{Energy signs.}{fig:lecture10:lecture10Fig2}{0.2}

For low energies, we have a set of potentials for which we will have propagation, despite having a potential barrier.  For still higher values of the potential barrier the product \( r_1 r_2 \) will be negative, so the solutions will be decaying.  Finally, for even higher energies, there will again be propagation.

The non-relativistic case is sketched in \cref{fig:lecture10:lecture10Fig3}.

\imageFigure{../../figures/phy1520/lecture10Fig3}{Effects of increasing potential for non-relativistic case.}{fig:lecture10:lecture10Fig3}{0.1}

For the relativistic case we must consider three different cases, sketched in
\cref{fig:lecture10:lecture10Fig4a},
\cref{fig:lecture10:lecture10Fig4b}, and
\cref{fig:lecture10:lecture10Fig4c} respectively.  For the low potential energy, a particle with positive group velocity (what we've called right moving) can be matched to an equal energy portion of the potential shifted parabola in region II.  This is a case where we have transmission, but no antiparticle creation.  There will be an energy region where the region II wave function has only a dissipative term, since there is no region of either of the region II parabolic branches available at the incident energy.  When the potential is shifted still higher so that \( V_0 > E + m c^2 \), a positive group velocity in region I with a given energy can be matched to an antiparticle branch in the region II parabolic energy curve.

\imageFigure{../../figures/phy1520/lecture10Fig4a}{Low potential energy.}{fig:lecture10:lecture10Fig4a}{0.1}
\imageFigure{../../figures/phy1520/lecture10Fig4b}{High enough potential energy for no propagation.}{fig:lecture10:lecture10Fig4b}{0.1}
\imageFigure{../../figures/phy1520/lecture10Fig4c}{High potential energy.}{fig:lecture10:lecture10Fig4c}{0.1}

\paragraph{Boundary value conditions}
\index{Dirac equation!boundary conditions}
We want to ensure that the current across the barrier is conserved (no particles are lost), as sketched in \cref{fig:lecture10:lecture10Fig5}.

\imageFigure{../../figures/phy1520/lecture10Fig5}{Transmitted, reflected and incident components.}{fig:lecture10:lecture10Fig5}{0.1}

Recall that given a wave function

\begin{dmath}\label{eqn:qmLecture10:240}
\Psi =
\begin{bmatrix}
\psi_1 \\
\psi_2
\end{bmatrix},
\end{dmath}

the density and currents are respectively

\begin{equation}\label{eqn:qmLecture10:260}
\begin{aligned}
\rho &= \psi_1^\conj \psi_1 + \psi_2^\conj \psi_2 \\
j &= \psi_1^\conj \psi_1 - \psi_2^\conj \psi_2
\end{aligned}
\end{equation}

Matching boundary value conditions requires

\begin{enumerate}
\item For both the relativistic and non-relativistic cases we must have

\begin{equation}\label{eqn:qmLecture10:280}
\Psi_\txtL = \Psi_\txtR, \qquad \mbox{at \( x = 0 \).}
\end{equation}
\item For the non-relativistic case we want
\begin{equation}\label{eqn:qmLecture10:300}
\int_{-\epsilon}^\epsilon -\frac{\Hbar^2}{2m} \PDSq{x}{\Psi} =
\cancel{\int_{-\epsilon}^\epsilon \lr{ E - V(x) } \Psi(x)}
\end{equation}

\begin{equation}\label{eqn:qmLecture10:320}
-\frac{\Hbar^2}{2m} \lr{ \evalbar{\PD{x}{\Psi}}{\txtR} - \evalbar{\PD{x}{\Psi}}{\txtL} }  = 0.
\end{equation}

We have to match

For the relativistic case

\begin{dmath}\label{eqn:qmLecture10:460}
-i \Hbar \sigma_z \int_{-\epsilon}^\epsilon \PD{x}{\Psi} +
\cancel{m c^2 \sigma_x \int_{-\epsilon}^\epsilon \psi}
=
\cancel{\int_{-\epsilon}^\epsilon \lr{ E - V_0 } \psi},
\end{dmath}
\end{enumerate}

so

\begin{equation}\label{eqn:qmLecture10:340}
-i \Hbar c \sigma_z \lr{ \psi(\epsilon) - \psi(-\epsilon) }
=
-i \Hbar c \sigma_z \lr{ \psi_\txtR - \psi_\txtL }.
\end{equation}

so we must match

\begin{equation}\label{eqn:qmLecture10:360}
\sigma_z \psi_\txtR = \sigma_z \psi_\txtL .
\end{equation}

It appears that things are simpler, because we only have to match the wave function values at the boundary, and don't have to match the derivatives too.  However, we have a two component wave function, so there are still two tasks.

\paragraph{Solving the system}

Let's look for a solution for the \( E + m c^2 > V_0 \) case on the right branch, as sketched in \cref{fig:lecture10:lecture10Fig6}.

\imageFigure{../../figures/phy1520/lecture10Fig6}{High potential region.  Anti-particle transmission.}{fig:lecture10:lecture10Fig6}{0.15}

While the right branch in this case is left going, this might work out since that is an antiparticle.  We could try both.

Try

\begin{dmath}\label{eqn:qmLecture10:480}
\Psi_{II} = D e^{i k_2 x}
\begin{bmatrix}
-\sin\theta_2 \\
\cos\theta_2
\end{bmatrix}.
\end{dmath}

This is justified by

\begin{dmath}\label{eqn:qmLecture10:500}
+E \rightarrow
\begin{bmatrix}
\cos\theta \\
\sin\theta
\end{bmatrix},
\end{dmath}

so

\begin{dmath}\label{eqn:qmLecture10:520}
-E \rightarrow
\begin{bmatrix}
-\sin\theta \\
\cos\theta \\
\end{bmatrix}
\end{dmath}

At \( x = 0 \) the exponentials vanish, so equating the waves at that point means

\begin{dmath}\label{eqn:qmLecture10:380}
\begin{bmatrix}
\cos\theta_1 \\
\sin\theta_1 \\
\end{bmatrix}
+
\frac{B}{A}
\begin{bmatrix}
\sin\theta_1 \\
\cos\theta_1 \\
\end{bmatrix}
=
\frac{D}{A}
\begin{bmatrix}
-\sin\theta_2 \\
\cos\theta_2
\end{bmatrix}.
\end{dmath}

Solving this yields

\begin{dmath}\label{eqn:qmLecture10:400}
\frac{B}{A} = - \frac{\cos(\theta_1 - \theta_2)}{\sin(\theta_1 + \theta_2)}.
\end{dmath}

This yields

\boxedEquation{eqn:qmLecture10:420}{
R = \frac{1 + \cos( 2 \theta_1 - 2 \theta_2) }{1 - \cos( 2 \theta_1 - 2 \theta_2)}.
}

As \( V_0 \rightarrow \infty \) this simplifies to

\begin{dmath}\label{eqn:qmLecture10:440}
R = \frac{ E - \sqrt{ E^2 - \lr{ m c^2 }^2 } }{ E + \sqrt{ E^2 - \lr{ m c^2 }^2 } }.
\end{dmath}

Filling in the details for these results part of problem set 4.

%\EndNoBibArticle

      \section{Problems}
         %
% Copyright © 2015 Peeter Joot.  All Rights Reserved.
% Licenced as described in the file LICENSE under the root directory of this GIT repository.
%

\makeproblem{Calculate the right going diagonalization}{problem:qmLecture9:1}{

Prove \cref{eqn:qmLecture9:160}.
} % problem

\makeanswer{problem:qmLecture9:1}{

To determine the relations for \( \theta_k \) we have to solve 

\begin{dmath}\label{eqn:qmLecture9:280}
\begin{bmatrix}
E_k & 0 \\
0 & -E_k
\end{bmatrix}
= R^{-1} H R.
\end{dmath}

Working with \( \Hbar = c = 1 \) temporarily, and \( C = \cos\theta_k, S = \sin\theta_k \), that is

\begin{dmath}\label{eqn:qmLecture9:300}
\begin{bmatrix}
E_k & 0 \\
0 & -E_k
\end{bmatrix}
=
\begin{bmatrix}
C & S \\
-S & C
\end{bmatrix}
\begin{bmatrix}
k & m \\
m & -k
\end{bmatrix}
\begin{bmatrix}
C & -S \\
S & C
\end{bmatrix}
=
\begin{bmatrix}
C & S \\
-S & C
\end{bmatrix}
\begin{bmatrix}
k C + m S & -k S + m C \\
m C - k S & -m S - k C
\end{bmatrix}
=
\begin{bmatrix}
k C^2 + m S C + m C S - k S^2   & -k S C + m C^2 -m S^2 - k C S \\
-k C S - m S^2 + m C^2 - k S C & k S^2 - m C S -m S C - k C^2
\end{bmatrix}
=
\begin{bmatrix}
k \cos(2 \theta_k) + m \sin(2 \theta_k) & m \cos(2 \theta_k) - k \sin(2 \theta_k) \\
m \cos(2 \theta_k) - k \sin(2 \theta_k) & -k \cos(2 \theta_k) - m \sin(2 \theta_k) \\
\end{bmatrix}.
\end{dmath}

This gives

\begin{dmath}\label{eqn:qmLecture9:320}
E_k 
\begin{bmatrix}
1 \\
0
\end{bmatrix}
=
\begin{bmatrix}
k \cos(2 \theta_k) + m \sin(2 \theta_k) \\
m \cos(2 \theta_k) - k \sin(2 \theta_k) \\
\end{bmatrix}
=
\begin{bmatrix}
k & m \\
m & -k
\end{bmatrix}
\begin{bmatrix}
\cos(2 \theta_k) \\
\sin(2 \theta_k) \\
\end{bmatrix}.
\end{dmath}

Adding back in the \(\Hbar\)'s and \(c\)'s this is

\begin{dmath}\label{eqn:qmLecture9:340}
\begin{bmatrix}
\cos(2 \theta_k) \\
\sin(2 \theta_k) \\
\end{bmatrix}
=
\frac{E_k}{-(\Hbar k c)^2 -(m c^2)^2}
\begin{bmatrix}
- \Hbar k c & - m c^2 \\
- m c^2     & \Hbar k c
\end{bmatrix}
\begin{bmatrix}
1 \\
0
\end{bmatrix}
=
\inv{E_k}
\begin{bmatrix}
\Hbar k c \\
m c^2
\end{bmatrix}.
\end{dmath}
} % answer

\makeproblem{Verify the Dirac current relationship.}{problem:qmLecture9:2}{
Prove \cref{eqn:qmLecture9:240}.
} % problem

\makeanswer{problem:qmLecture9:2}{

The components of the Schr\"{o}dinger equation are

\begin{equation}\label{eqn:qmLecture9:360}
\begin{aligned}
-i \Hbar \PD{t}{\psi_1} &= -i \Hbar c \PD{x}{\psi_1} + m c^2 \psi_2  \\
-i \Hbar \PD{t}{\psi_2} &= m c^2 \psi_1 + i \Hbar c \PD{x}{\psi_2},
\end{aligned}
\end{equation}

The conjugates of these are
\begin{equation}\label{eqn:qmLecture9:380}
\begin{aligned}
i \Hbar \PD{t}{\psi_1^\conj} &= i \Hbar c \PD{x}{\psi_1^\conj} + m c^2 \psi_2^\conj \\
i \Hbar \PD{t}{\psi_2^\conj} &= m c^2 \psi_1^\conj - i \Hbar c \PD{x}{\psi_2^\conj}.
\end{aligned}
\end{equation}

This gives
\begin{dmath}\label{eqn:qmLecture9:400}
\begin{aligned}
i \Hbar \PD{t}{\rho}
&=
\lr{ i \Hbar c \PD{x}{\psi_1^\conj} + m c^2 \psi_2^\conj } \psi_1 \\
&+ \psi_1^\conj \lr{ i \Hbar c \PD{x}{\psi_1} - m c^2 \psi_2 } \\
&+ \lr{ m c^2 \psi_1^\conj - i \Hbar c \PD{x}{\psi_2^\conj} } \psi_2 \\
&+ \psi_2^\conj \lr{ -m c^2 \psi_1 - i \Hbar c \PD{x}{\psi_2} }.
\end{aligned}
\end{dmath}

All the non-derivative terms cancel leaving

\begin{dmath}\label{eqn:qmLecture9:420}
\inv{c} \PD{t}{\rho} 
=
\PD{x}{\psi_1^\conj} \psi_1 
+ \psi_1^\conj \PD{x}{\psi_1}
- \PD{x}{\psi_2^\conj} \psi_2 
- \psi_2^\conj \PD{x}{\psi_2} 
=
\PD{x}{} 
\lr{
\psi_1^\conj \psi_1 -
\psi_2^\conj \psi_2 
}.
\end{dmath}

} % answer
 

         %
% Copyright � 2015 Peeter Joot.  All Rights Reserved.
% Licenced as described in the file LICENSE under the root directory of this GIT repository.
%
\makeproblem{Zitterbewegung in one dimension}{gradQuantum:problemSet4:1}{ 
\index{Zitterbewegung}
Consider the Dirac Hamiltonian \( H = c \hat{p} \sigma_z + m c^2 \sigma_x \) . Using the Heisenberg equations of motion, derive a second order equation of motion (eom) for the velocity operator \( \hat{v} = d\hat{x}/dt \). For a state

\begin{dmath}\label{eqn:gradQuantumProblemSet4Problem1:20}
\Psi = 
\begin{bmatrix}
1 \\
0
\end{bmatrix}
e^{i k x}
\end{dmath}

with \( k = 0 \), average this eom in state \( \Psi \) to get a homogeneous second order differential equation for \( \expectation{\hat{v}} \). Using this equation of motion and the initial conditions on the velocity and its time-derivative, obtain \( \expectation{\hat{v}}(t) \) and \( \expectation{\hat{x}(t)} \).
Show that these oscillate with a rapid frequency \( 2 m c^2/h \), with the oscillation of the position having an amplitude \( \Hbar/m c \) which is the Compton wavelength.  This `trembling' motion is called Zitterbewegung.

%\makesubproblem{}{gradQuantum:problemSet4:1a}
} % makeproblem

\makeanswer{gradQuantum:problemSet4:1}{ 
%\makeSubAnswer{}{gradQuantum:problemSet4:1a}

Using the hint from class that Zitterbewegung is associated with oscillation of the particle between ordinary-particle and anti-particle states, let's look for a combination of such \( k > 0 \) states to represent the state of this problem

\begin{dmath}\label{eqn:gradQuantumProblemSet4Problem1:40}
\begin{bmatrix}
1 \\
0
\end{bmatrix}
e^{i k x}
=
\evalbar{
\lr{
a 
\begin{bmatrix}
\cos\theta \\
\sin\theta \\
\end{bmatrix}
e^{i k x - i E t/\Hbar}
+ b
\begin{bmatrix}
-\sin\theta \\
\cos\theta \\
\end{bmatrix}
e^{i k x + i E t/\Hbar}
}
}{t = 0}.
\end{dmath}

Observe that \( \begin{bmatrix}
-\sin\theta \\
\cos\theta \\
\end{bmatrix} \) and \( \begin{bmatrix}
\cos\theta \\
\sin\theta \\
\end{bmatrix} \) are orthogonal, so 

\begin{equation}\label{eqn:gradQuantumProblemSet4Problem1:60}
a = 
\begin{bmatrix}
1 & 0 
\end{bmatrix}
\begin{bmatrix}
\cos\theta \\
\sin\theta \\
\end{bmatrix}
= \cos\theta.
\end{equation}
\begin{equation}\label{eqn:gradQuantumProblemSet4Problem1:80}
b =
\begin{bmatrix}
1 & 0 
\end{bmatrix}
\begin{bmatrix}
-\sin\theta \\
\cos\theta \\
\end{bmatrix}
= -\sin\theta.
\end{equation}

With 

\begin{equation}\label{eqn:gradQuantumProblemSet4Problem1:100}
\begin{aligned}
\ket{a} &=
\begin{bmatrix}
\cos\theta \\
\sin\theta \\
\end{bmatrix} 
e^{i k x - i E t/\Hbar}
\\
\ket{b} &=
\begin{bmatrix}
-\sin\theta \\
\cos\theta \\
\end{bmatrix} 
e^{i k x + i E t/\Hbar}
\end{aligned}
\end{equation}

The wave function of \cref{eqn:gradQuantumProblemSet4Problem1:20} can be represented as

\begin{dmath}\label{eqn:gradQuantumProblemSet4Problem1:120}
\Psi = 
\cos\theta
\ket{a}
- \sin\theta
\ket{b}.
\end{dmath}

The action of the Hamiltonian on this wave function is

\begin{dmath}\label{eqn:gradQuantumProblemSet4Problem1:140}
H \psi 
= i \Hbar \lr{ -i \frac{E}{\Hbar} C \ket{a} - i \frac{E}{\Hbar} S \ket{b} } 
= E \lr{ C \ket{a} + S \ket{b} },
\end{dmath}

Writing \( \epsilon = \sqrt{ (m c^2)^2 + (\Hbar k c)^2 } \), and noting that \( \ket{a} \), and \( \ket{b} \) are positive and negative eigenstates respectively,
%using the diagonalization from class, 
the spatial action of the Hamiltonian on this wave function is
\begin{dmath}\label{eqn:gradQuantumProblemSet4Problem1:160}
H \Psi
=
%\begin{bmatrix}
%\hat{p} c & m c^2 \\
%m c^2 & -\hat{p} c
%\end{bmatrix}
%\Psi
%=
%\begin{bmatrix}
%\Hbar k c & m c^2 \\
%m c^2 & -\Hbar k c
%\end{bmatrix}
%\Psi
%=
%\begin{bmatrix}
%C & -S \\
%S & C 
%\end{bmatrix}
%\begin{bmatrix}
%\epsilon & 0 \\
%0 & -\epsilon
%\end{bmatrix}
%\begin{bmatrix}
%C & S \\
%-S & C 
%\end{bmatrix}
%\Psi
%=
%\begin{bmatrix}
%C & -S \\
%S & C 
%\end{bmatrix}
%\begin{bmatrix}
%\epsilon & 0 \\
%0 & -\epsilon
%\end{bmatrix}
%\begin{bmatrix}
%C & S \\
%-S & C 
%\end{bmatrix}
%\lr{ 
%C 
%\begin{bmatrix}
%C \\
%S
%\end{bmatrix}
%e^{i k x - i E t/\Hbar}
%-S
%\begin{bmatrix}
%-S \\
%C \\
%\end{bmatrix}
%e^{i k x + i E t/\Hbar}
%}
%=
%\begin{bmatrix}
%C & -S \\
%S & C 
%\end{bmatrix}
%\begin{bmatrix}
%\epsilon & 0 \\
%0 & -\epsilon
%\end{bmatrix}
%\lr{ 
%C 
%\begin{bmatrix}
%C^2 + S^2 \\
%-S C + C S
%\end{bmatrix}
%e^{i k x - i E t/\Hbar}
%-S
%\begin{bmatrix}
%-S C + S C \\
%S^2 + C^2
%\end{bmatrix}
%e^{i k x + i E t/\Hbar}
%}
%=
%\begin{bmatrix}
%C & -S \\
%S & C 
%\end{bmatrix}
%\lr{ 
%C \epsilon
%\begin{bmatrix}
%1 \\
%0
%\end{bmatrix}
%e^{i k x - i E t/\Hbar}
%+ S \epsilon
%\begin{bmatrix}
%0 \\
%1
%\end{bmatrix}
%e^{i k x + i E t/\Hbar}
%}
%=
%\lr{ 
%C \epsilon
%\begin{bmatrix}
%C \\
%S
%\end{bmatrix}
%e^{i k x - i E t/\Hbar}
%+S \epsilon
%\begin{bmatrix}
%-S \\
%C
%\end{bmatrix}
%e^{i k x + i E t/\Hbar}
%}
= (+\epsilon) C \ket{a} - (-\epsilon) S \ket{b} 
= \epsilon \lr{ C \ket{a} + S\ket{b} }.
\end{dmath}

This provides a relationship between the energy \( E \) and the eigenvalues

\begin{equation}\label{eqn:gradQuantumProblemSet4Problem1:180}
E = \epsilon = \sqrt{ (m c^2)^2 + (\Hbar k c)^2 }.
\end{equation}

In particular, when \( k = 0 \), this is \( E = m c^2 \).

Using the Heisenberg equations of motion, the velocity operator is

\begin{dmath}\label{eqn:gradQuantumProblemSet4Problem1:200}
\hat{v} I
=
\ddt{\hat{x}} I
= \inv{i\Hbar }\antisymmetric{\hat{x} I}{ \hat{p} c \sigma_z + m c^2 \sigma_x }
= 
\inv{i\Hbar }
\lr{
\hat{x} \hat{p} c \sigma_z - \hat{p} c \sigma_z \hat{x}
+ \cancel{\hat{x} m c^2 \sigma_x} - \cancel{m c^2 \sigma_x \hat{x}}
}
= 
\inv{i\Hbar }
c \sigma_z \antisymmetric{\hat{x}}{\hat{p}}
= 
c \sigma_z.
\end{dmath}

As we've seen with the Harmonic oscillator, an expectation with respect to a state that is not a single eigenstate, will have non-zero time dependence.  With
\( C = \cos\theta \), \( S = \sin\theta \)
that expectation value with respect to the state as expressed in \cref{eqn:gradQuantumProblemSet4Problem1:120} is

\begin{dmath}\label{eqn:gradQuantumProblemSet4Problem1:220}
\expectation{\hat{v}}
=
\bra{\Psi} \hat{v} \ket{\Psi}
=
c \lr{ C \bra{a} - S \bra{b} } \sigma_z \lr{ C \ket{a} - S \ket{b} }
=
c \lr{ 
C \bra{a} - S \bra{b} } 
\begin{bmatrix}
1 & 0 \\
0 & -1
\end{bmatrix}
\lr{ 
\begin{bmatrix}
C \\
S
\end{bmatrix} 
e^{i k x - i E t/\Hbar}
- S
\begin{bmatrix}
-S \\
C
\end{bmatrix} 
e^{i k x + i E t/\Hbar}
}
=
c 
\lr{ 
C
\begin{bmatrix}
C & 
S
\end{bmatrix} 
e^{-i k x + i E t/\Hbar}
- S
\begin{bmatrix}
-S & C
\end{bmatrix} 
e^{-i k x - i E t/\Hbar}
}
\lr{ 
C
\begin{bmatrix}
C \\
-S
\end{bmatrix} 
e^{i k x - i E t/\Hbar}
+ S
\begin{bmatrix}
S \\
C
\end{bmatrix} 
e^{i k x + i E t/\Hbar}
}
= c \lr{
C^2 \cos(2\theta)
- S^2 \cos( 2 \theta )
+ S C \sin( 2 \theta ) e^{-2 i E t/\Hbar}
+ S C \sin( 2 \theta ) e^{2 i E t/\Hbar}
}
= 
c \lr{ \cos^2( 2 \theta ) + \sin^2( 2 \theta ) \cos(2 E t/\Hbar) }
= 
c \lr{ 
\frac{ (\Hbar c k)^2 }{ (m c^2)^2 + (\Hbar k c)^2 }
+\frac{ (m c^2)^2 }{ (m c^2)^2 + (\Hbar k c)^2 } \cos\lr{ 2 \pi \frac{2 E}{h} t }
}
= 
\frac{c}{ (m c^2)^2 + (\Hbar k c)^2 } \lr{ 
(\Hbar c k)^2 
+ 
(m c^2)^2 
\cos\lr{ 2 \pi \frac{2 E}{h} t }
}.
\end{dmath}

For \( k = 0 \), this is

\boxedEquation{eqn:gradQuantumProblemSet4Problem1:240}{
\expectation{\hat{v}} = c \cos\lr{ 2 \pi \frac{2 m c^2}{h} t }.
}

The frequency of this oscillation is \( \ifrac{2 m c^2}{h} \) as the problem states.  For the position expectation with respect to this state, we have

\begin{dmath}\label{eqn:gradQuantumProblemSet4Problem1:260}
\expectation{\hat{x}} 
= \expectation{\hat{x}}(0) + \frac{c \Hbar}{2 m c^2} \sin \lr{ \frac{2 m c^2}{\Hbar} t }
= \expectation{\hat{x}}(0) + \frac{\Hbar}{2 m c} \sin \lr{ \frac{2 m c^2}{\Hbar} t }.
\end{dmath}

The amplitude of this oscillation is

\begin{dmath}\label{eqn:gradQuantumProblemSet4Problem1:280}
2 \times \frac{\Hbar}{2 m c}
=
\frac{\Hbar}{m c},
\end{dmath}

as expected.
}

         %
% Copyright � 2015 Peeter Joot.  All Rights Reserved.
% Licenced as described in the file LICENSE under the root directory of this GIT repository.
%
\makeproblem{Jackiw-Rebbi problem}{gradQuantum:problemSet4:2}{ 

Recall that the energy of a relativistic particle is \( E(p) = \sqrt{p^2 c^2 + m^2 c^4 } \), which is independent of the sign of \( m \).
Thus \( m > 0 \) and \( m < 0 \) lead to the same dispersion relation.
Set aside for now, the physical meaning of \( m < 0 \).
Assume \( V(x) = 0 \) but let us assume the mass \( m \) is a function of position \( m(x) \).
This leads to

\begin{dmath}\label{eqn:gradQuantumProblemSet4Problem2:20}
H =
\begin{bmatrix}
c \hat{p} & m(x) c^2 \\
m(x) c^2 & -c \hat{p}
\end{bmatrix}
.
\end{dmath}

Let \( m(x) \) be such that \( m(-x) = -m(x) \), i.e., an odd function of position which changes sign at \( x = 0 \). Show that the operator \( \hat{\calP}_{\textrm{Dirac}} = \sigma_y \hat{\calP} \) commutes with the Hamiltonian, where

\begin{dmath}\label{eqn:gradQuantumProblemSet4Problem2:40}
\sigma_y
=
\PauliY
\end{dmath}

is the y-Pauli matrix, and \( \hat{\calP} \) is the parity operator which sends \( x \rightarrow -x\).
Consider the wavefunction

\begin{dmath}\label{eqn:gradQuantumProblemSet4Problem2:60}
\Phi(x) =
\begin{bmatrix}
f(x) \\
- i f(x)
\end{bmatrix},
\end{dmath}

where \( f(-x) = f(x) \) is an even function. Show \( \Phi(x) \) is an eigenstate of \( \hat{\calP}_{\textrm{Dirac}} \) with eigenvalue \( -1 \).
Next, assuming \( m(x > 0) = m_0 \) and \( m(x < 0) = -m_0 \) , where \( m_0 > 0 \), find \( f(x) \) such that \( \Phi(x) \) is an eigenstate of H with zero energy.
Normalize the wavefunction \( \Phi(x) \).

%\makesubproblem{}{gradQuantum:problemSet4:2a}
} % makeproblem

\makeanswer{gradQuantum:problemSet4:2}{ 
%\makeSubAnswer{}{gradQuantum:problemSet4:2a}

TODO.
}

         %
% Copyright � 2015 Peeter Joot.  All Rights Reserved.
% Licenced as described in the file LICENSE under the root directory of this GIT repository.
%
\makeoproblem{Scattering off a potential step.}{gradQuantum:problemSet4:3}{2015 ps4 p3}{
\index{Dirac equation!scattering}
\index{Dirac equation!potential step}

Consider the 1D Dirac Hamiltonian as

\begin{dmath}\label{eqn:gradQuantumProblemSet4Problem3:21}
H =
\begin{bmatrix}
c \hatp + V(x) & m c^2 \\
m c^2 & - c \hatp + V(x)
\end{bmatrix}
\end{dmath}

where the operator \( \hatp = - i \Hbar \PDi{x}{} \) is the rest mass, and \( c \) is the speed of light, and with 2-component wavefunctions

\begin{dmath}\label{eqn:gradQuantumProblemSet4Problem3:41}
\Psi(x,t) \equiv
\begin{bmatrix}
\psi_1(x, t) \\
\psi_2(x, t) \\
\end{bmatrix}
\end{dmath}

such that \( i \Hbar \PDi{t}{\Psi(x,t)} = H \Psi(x, t) \).  Assuming a potential step, where \( V(x < 0) = 0 \) and \( V(x > 0) = V_0 \), with \( V_0 > 0 \) as in class, complete the details of the scattering onto the step which was done in class. Discuss the incident current, reflected current, and transmitted current for the case where the incident energy is such that \( E > 0 \) and \( V_0 > 2 m c^2 \) , and \( E > 0 \) and \( V_0 < 2 m c^2 \). Draw pictures to illustrate the parabola and the location in momentum of the incident and reflected particles.

%\makesubproblem{}{gradQuantum:problemSet4:3a}
} % makeproblem

\makeanswer{gradQuantum:problemSet4:3}{
\withproblemsetsParagraph{
%\makeSubAnswer{}{gradQuantum:problemSet4:3a}

%
% Copyright © 2016 Peeter Joot.  All Rights Reserved.
% Licenced as described in the file LICENSE under the root directory of this GIT repository.
%

\paragraph{Mostly background.}
% (really for myself).  Grading can probably skip to around \cref{eqn:gradQuantumProblemSet4Problem3:560}

We talked about diagonalization of the Dirac Hamiltonian by introducing a rotation, and then figuring out the rotation angle required.

To understand the form form of the eigenkets for particles and antiparticles in both forward and backwards moving configurations, lets do this diagonalization explicitly for both forwards and backwards solutions.

For the forward solution, given \( \Psi = \Psi_0 e^{i(k x - E t/\Hbar) } \), the Dirac equation is

\begin{dmath}\label{eqn:gradQuantumProblemSet4Problem3:20}
i \Hbar (-i E/\Hbar) \Psi =
\begin{bmatrix}
-i \Hbar (i k) + V_0 & m c^2 \\
m c^2 & i \Hbar (i k)
\end{bmatrix},
\Psi
\end{dmath}

or
\begin{dmath}\label{eqn:gradQuantumProblemSet4Problem3:40}
\begin{bmatrix}
E - V_0 & 0 \\
0 & E - V_0
\end{bmatrix}
\Psi
=
\begin{bmatrix}
\Hbar k & m c^2 \\
m c^2 &  - \Hbar k
\end{bmatrix}
\Psi.
\end{dmath}

Similarly, for the backwards moving wave \( \Psi = e^{i(-k x - E t/\Hbar)} \), we have

\begin{dmath}\label{eqn:gradQuantumProblemSet4Problem3:60}
\begin{bmatrix}
E - V_0 & 0 \\
0 & E - V_0
\end{bmatrix}
\Psi
=
\begin{bmatrix}
-\Hbar k & m c^2 \\
m c^2 & \Hbar k
\end{bmatrix}
\Psi.
\end{dmath}

Working with \( \Hbar = c = 1 \) temporarily, we want to compute the eigensolutions for the matrix

\begin{dmath}\label{eqn:gradQuantumProblemSet4Problem3:80}
H_{\pm k}
=
\begin{bmatrix}
\pm k & m \\
m & \mp k
\end{bmatrix}.
\end{dmath}

The eigenvalues \( \epsilon \) of both are the same

\begin{dmath}\label{eqn:gradQuantumProblemSet4Problem3:100}
0
=
\Abs{ H_{\pm k} - \epsilon }
=
(\pm k - \epsilon)(\mp k - \epsilon) - m^2
=
(\mp k + \epsilon)(\pm k + \epsilon) - m^2
=
\epsilon^2 - k^2 - m^2,
\end{dmath}

or

\begin{dmath}\label{eqn:gradQuantumProblemSet4Problem3:120}
\epsilon = \pm \sqrt{k^2 + m^2}.
\end{dmath}

\paragraph{Eigenkets for \( H_k \)}

For the positive(negative) energy eigenvalues, we have

\begin{dmath}\label{eqn:gradQuantumProblemSet4Problem3:140}
0
=
\begin{bmatrix}
k \mp \epsilon & m \\
m & -k \mp \epsilon
\end{bmatrix}
\begin{bmatrix}
a \\
b
\end{bmatrix},
\end{dmath}

for some \( a, b\).  That is

\begin{dmath}\label{eqn:gradQuantumProblemSet4Problem3:160}
(k \mp \epsilon) a + m b = 0,
\end{dmath}

or

\begin{dmath}\label{eqn:gradQuantumProblemSet4Problem3:180}
\begin{bmatrix}
a \\
b
\end{bmatrix}
\propto
\begin{bmatrix}
- m \\
k \mp \epsilon
\end{bmatrix}.
\end{dmath}

For the normalization note that

\begin{dmath}\label{eqn:gradQuantumProblemSet4Problem3:200}
m^2 + \lr{ k \mp \epsilon }^2
=
m^2 + k^2 + \epsilon^2 \mp 2 k \epsilon
=
2 \epsilon^2 \mp 2 k \epsilon
=
2 \epsilon (\epsilon \mp k),
\end{dmath}

so the normalized kets are

\begin{dmath}\label{eqn:gradQuantumProblemSet4Problem3:220}
\ket{k; \pm\epsilon} =
\inv{\sqrt{2 \epsilon(\epsilon \mp k)}}
\begin{bmatrix}
\pm m \\
\epsilon \mp k
\end{bmatrix}.
\end{dmath}

\paragraph{Eigenkets for \( H_{-k} \)}

This time, for the positive(negative) energy eigenvalues, we have

\begin{dmath}\label{eqn:gradQuantumProblemSet4Problem3:141}
0
=
\begin{bmatrix}
-k \mp \epsilon & m \\
m & k \mp \epsilon
\end{bmatrix}
\begin{bmatrix}
a \\
b
\end{bmatrix},
\end{dmath}

for some \( a, b\).  That is

\begin{dmath}\label{eqn:gradQuantumProblemSet4Problem3:161}
(-k \mp \epsilon) a + m b = 0,
\end{dmath}

or

\begin{dmath}\label{eqn:gradQuantumProblemSet4Problem3:181}
\begin{bmatrix}
a \\
b
\end{bmatrix}
\propto
\begin{bmatrix}
- m \\
-k \mp \epsilon
\end{bmatrix}
\propto
\begin{bmatrix}
m \\
k \pm \epsilon
\end{bmatrix}.
\end{dmath}

For the normalization note that

\begin{dmath}\label{eqn:gradQuantumProblemSet4Problem3:201}
m^2 + \lr{ k \pm \epsilon }^2
=
m^2 + k^2 + \epsilon^2 \pm 2 k \epsilon
=
2 \epsilon^2 \pm 2 k \epsilon
=
2 \epsilon (\epsilon \pm k),
\end{dmath}

so the normalized kets are

\begin{dmath}\label{eqn:gradQuantumProblemSet4Problem3:221}
\ket{-k; \pm\epsilon} =
\inv{\sqrt{2 \epsilon(\epsilon \mp k)}}
\begin{bmatrix}
\pm m \\
\epsilon \pm k
\end{bmatrix}.
\end{dmath}

\paragraph{Simplification of the rotation matrices}

The eigenvalue equations have the form

\begin{dmath}\label{eqn:gradQuantumProblemSet4Problem3:240}
H
\begin{bmatrix}
\ket{+} & \ket{-}
\end{bmatrix}
=
\begin{bmatrix}
\ket{+} & \ket{-}
\end{bmatrix}
\begin{bmatrix}
\epsilon & 0 \\
0 & -\epsilon
\end{bmatrix}
\end{dmath}

With \( R = \begin{bmatrix} \ket{+} & \ket{-} \end{bmatrix} \), this has the form \( H R = R \Omega \), or \( H = R \Omega R^{-1} \).  The rotation matrices have been found to be

\begin{dmath}\label{eqn:gradQuantumProblemSet4Problem3:260}
R_{+k}
=
\inv{\sqrt{2\epsilon}}
\begin{bmatrix}
\frac{m}{\sqrt{\epsilon - k}} & \frac{-m}{\sqrt{\epsilon + k}} \\
\sqrt{\epsilon - k} & \sqrt{\epsilon + k}
\end{bmatrix},
\end{dmath}

and
\begin{dmath}\label{eqn:gradQuantumProblemSet4Problem3:280}
R_{-k}
=
\inv{\sqrt{2\epsilon}}
\begin{bmatrix}
\frac{m}{\sqrt{\epsilon + k}} & \frac{-m}{\sqrt{\epsilon - k}} \\
\sqrt{\epsilon + k} & \sqrt{\epsilon - k}
\end{bmatrix}.
\end{dmath}

These don't look very much like rotation matrices as is, but not all the terms are independent.  Writing

\begin{dmath}\label{eqn:gradQuantumProblemSet4Problem3:300}
\begin{aligned}
a &= \frac{m}{\sqrt{2\epsilon(\epsilon - k)}}  \\
b &= \frac{m}{\sqrt{2\epsilon(\epsilon + k)}} \\
\alpha &= \sqrt{\frac{\epsilon + k}{2 \epsilon}}  \\
\beta &= \sqrt{\frac{\epsilon - k}{2 \epsilon}}
\end{aligned},
\end{dmath}

the rotation matrices are

\begin{dmath}\label{eqn:gradQuantumProblemSet4Problem3:320}
R_{+k}
=
\begin{bmatrix}
a & - b \\
\beta & \alpha
\end{bmatrix},
\end{dmath}

and
\begin{dmath}\label{eqn:gradQuantumProblemSet4Problem3:340}
R_{-k}
=
\begin{bmatrix}
b & -a \\
\alpha & \beta
\end{bmatrix}.
\end{dmath}

Note that
\begin{equation}\label{eqn:gradQuantumProblemSet4Problem3:360}
\begin{aligned}
\frac{a}{\alpha} &= \frac{b}{\beta} \\
&= \frac{m}{\sqrt{\epsilon^2 - k^2}}  \\
&= \frac{m}{\sqrt{m^2}}  \\
&= 1.
\end{aligned}
\end{equation}

So
\begin{dmath}\label{eqn:gradQuantumProblemSet4Problem3:380}
R_{+k}
=
\begin{bmatrix}
a & - b \\
b & a
\end{bmatrix},
\end{dmath}

and
\begin{dmath}\label{eqn:gradQuantumProblemSet4Problem3:400}
R_{-k}
=
\begin{bmatrix}
b & -a \\
a & b
\end{bmatrix}.
\end{dmath}

These are expected to have a unit determinant, which is verified easily

\begin{dmath}\label{eqn:gradQuantumProblemSet4Problem3:420}
a^2 + b^2
=
\frac{m^2}{2 \epsilon}
\lr{
\frac{1}{\epsilon + k}
+
\frac{1}{\epsilon - k}
}
=
\frac{m^2}{2 \epsilon}
\frac{2 \epsilon}{\epsilon_2 - k^2}
= 1.
\end{dmath}

We see that both sets of matrices invert by transposition, so we are free to make a trigonometric identification

\begin{equation}\label{eqn:gradQuantumProblemSet4Problem3:440}
a = \cos\theta
=
\frac{m}{\sqrt{2\epsilon(\epsilon - k)}}
\end{equation}

\begin{equation}\label{eqn:gradQuantumProblemSet4Problem3:460}
b = \sin\theta
=
\frac{m}{\sqrt{2\epsilon(\epsilon + k)}}.
\end{equation}

Using the double angle formulation used in class these are

\begin{dmath}\label{eqn:gradQuantumProblemSet4Problem3:480}
\cos(2 \theta)
=
\frac{m^2}{2\epsilon(\epsilon - k)} -
\frac{m^2}{2\epsilon(\epsilon + k)}
=
\frac{m^2}{2 \epsilon} \lr{ \inv{\epsilon - k} - \inv{\epsilon + k} }
=
\frac{2 k m^2}{2 \epsilon (\epsilon^2 - k^2)}
=
\frac{k}{\epsilon},
\end{dmath}

and
\begin{dmath}\label{eqn:gradQuantumProblemSet4Problem3:500}
\sin(2 \theta)
=
2 \frac{m^2}{2 \epsilon} \inv{\sqrt{\epsilon^2 - k^2}}
=
\frac{m}{\epsilon}.
\end{dmath}

Putting back in the \( \Hbar \), and \( c\) factors, the diagonalizing transformation is now fully specified

\begin{equation}\label{eqn:gradQuantumProblemSet4Problem3:520}
\begin{aligned}
H_{\pm k} &=
\begin{bmatrix}
\pm \Hbar k c & m c^2 \\
m c^2 & \mp \Hbar k c
\end{bmatrix}
=
R_{\pm k} \Omega R_{\pm k}^\T \\
R_{+ k} &=
\begin{bmatrix}
\cos\theta & -\sin\theta  \\
\sin\theta & \cos\theta
\end{bmatrix}
=
\begin{bmatrix}
\ket{+k;+\epsilon} &
\ket{+k;-\epsilon}
\end{bmatrix}
\\
R_{- k} &=
\begin{bmatrix}
\sin\theta & -\cos\theta  \\
\cos\theta & \sin\theta
\end{bmatrix}
=
\begin{bmatrix}
\ket{-k;+\epsilon} &
\ket{-k;-\epsilon}
\end{bmatrix}
\\
\tan(2 \theta) &= \frac{m c}{\Hbar \Abs{k}} \\
\Omega &=
\begin{bmatrix}
\epsilon & 0 \\
0 & -\epsilon
\end{bmatrix} \\
\epsilon &= \sqrt{(\Hbar k c)^2 + (m c^2)^2}.
\end{aligned}
\end{equation}

The functions \( \Psi = \ket{\pm k; \pm \epsilon} e^{\pm i k x - i E t/\Hbar} \) are eigenfunctions of the Hamiltonian.  For example, for a non-antiparticle forward moving state

\begin{dmath}\label{eqn:gradQuantumProblemSet4Problem3:540}
(E - V_0) \ket{k; +\epsilon}
=
H_k \ket{k; +\epsilon}
=
\begin{bmatrix}
\ket{k;+\epsilon} &
\ket{k;-\epsilon}
\end{bmatrix}
\begin{bmatrix}
\epsilon & 0 \\
0 & -\epsilon
\end{bmatrix}
\begin{bmatrix}
\bra{k;+\epsilon} \\
\bra{k;-\epsilon}
\end{bmatrix}
\ket{k; \epsilon}
=
\begin{bmatrix}
\ket{k;+\epsilon} &
\ket{k;-\epsilon}
\end{bmatrix}
\begin{bmatrix}
\epsilon & 0 \\
0 & -\epsilon
\end{bmatrix}
\begin{bmatrix}
\braket{k;+\epsilon}{k; \epsilon} \\
\braket{k;-\epsilon}{k; \epsilon}
\end{bmatrix}
=
\begin{bmatrix}
\ket{k;+\epsilon} &
\ket{k;-\epsilon}
\end{bmatrix}
\begin{bmatrix}
\epsilon & 0 \\
0 & -\epsilon
\end{bmatrix}
\begin{bmatrix}
1 \\
0
\end{bmatrix}
=
\begin{bmatrix}
\ket{k;+\epsilon} &
\ket{k;-\epsilon}
\end{bmatrix}
\begin{bmatrix}
\epsilon \\
0
\end{bmatrix}
=
\epsilon
\ket{k;+\epsilon}.
\end{dmath}

\paragraph{Back to the core problem.}

The total energy for this particle is

\begin{dmath}\label{eqn:gradQuantumProblemSet4Problem3:560}
E = V_0 \pm \sqrt{(\Hbar k c)^2 + (m c^2)^2}.
\end{dmath}

This can be plotted nicely, as \( \frac{E}{m c^2} \) vs. \( \frac{\Hbar k}{m c} \) if non-dimensionalized as

\begin{dmath}\label{eqn:gradQuantumProblemSet4Problem3:580}
\frac{E}{m c^2} = \frac{V_0}{m c^2} + \sqrt{1 + \frac{(\Hbar k)^2}{(m c)^2} }.
\end{dmath}

For the step potential we have

\begin{dmath}\label{eqn:gradQuantumProblemSet4Problem3:600}
\frac{E}{m c^2}
= \sqrt{ 1 + {\frac{ \Hbar k_1}{m c} }^2 }
= \pm \sqrt{ 1 + {\frac{ \Hbar k_2}{m c} }^2 } + \frac{V_0}{m c^2},
\end{dmath}

or
\begin{dmath}\label{eqn:gradQuantumProblemSet4Problem3:620}
\frac{ \Hbar k_2}{m c}
=
\lr{ \lr{ \pm \sqrt{ 1 + {\frac{ \Hbar k_1}{m c} }^2 } - \frac{V_0}{m c^2} }^2 - 1 }^{1/2}.
\end{dmath}

When this is real valued, there is transmission in the barrier region.  There are a few cases of interest, the \( V_0 < 2 m c^2 \) cases are plotted in \cref{fig:ps4DiracStepPotential:ps4DiracStepPotentialFig1} showing the transmission (\( V_0 = 0.7 m c^2 \)) and decaying cases (\( V_0 = 1.4 m c^2 \)) respectively.
Three \( V_0 > 2 m c^2 \) cases are plotted in \cref{fig:ps4DiracStepPotential:ps4DiracStepPotentialFig3},
showing the
anti-particle transmission,
decaying solution,
and ordinary-particle transmission
(\( (V_0/m c^2,k \Hbar/m c) = (3.4,1.7), (2.6,1.9), (2.7,6.3) \)).  At low enough momentum there is antiparticle transmission, then reflection as momentum is increased, and finally at high enough momentum ordinary particle transmission.

\mathImageTwoFigures
{../../figures/phy1520/ps4DiracStepPotentialFig2}
{../../figures/phy1520/ps4DiracStepPotentialFig1}
{\( V_0 < 2 m c^2 \)}{fig:ps4DiracStepPotential:ps4DiracStepPotentialFig1}{scale=0.3}
{energyVsMomentumForDiracStepPotentialWaveFunctions.nb}
%%
% v_0, k
% 3.4, 1.7
% 2.6, 1.9
% 2.7, 6.3
\mathImageThreeFiguresOneLine
{../../figures/phy1520/ps4DiracStepPotentialFig3}
{../../figures/phy1520/ps4DiracStepPotentialFig4}
{../../figures/phy1520/ps4DiracStepPotentialFig5}
{\( V_0 > 2 m c^2 \)}{fig:ps4DiracStepPotential:ps4DiracStepPotentialFig3}{scale=0.3}
{energyVsMomentumForDiracStepPotentialWaveFunctions.nb}

We want to find the wave incident, reflected, and transmitted wave function.  From the diagonalization above, these are

\begin{dmath}\label{eqn:gradQuantumProblemSet4Problem3:640}
\Psi_{\textrm{inc}} =
\begin{bmatrix}
\cos\theta_{k_1} \\
\sin\theta_{k_1} \\
\end{bmatrix}
e^{i (k_1 x - E t/\Hbar)}
\end{dmath}
\begin{dmath}\label{eqn:gradQuantumProblemSet4Problem3:660}
\Psi_{\textrm{ref}} =
\begin{bmatrix}
\sin\theta_{k_1} \\
\cos\theta_{k_1} \\
\end{bmatrix}
e^{i (-k_1 x - E t/\Hbar)}
\end{dmath}

The currents \( j = c (\psi_1^\conj \psi_1 - \psi_2^\conj \psi_2) \) for these wave functions are

\begin{equation}\label{eqn:gradQuantumProblemSet4Problem3:960}
j_{\textrm{inc}} = c (\cos^2 \theta_{k_1} - \sin^2 \theta_{k_1}) = c \cos(2 \theta_{k_1} ),
\end{equation}

and

\begin{equation}\label{eqn:gradQuantumProblemSet4Problem3:980}
j_{\textrm{ref}} = c (\sin^2 \theta_{k_1} - \cos^2 \theta_{k_1}) = -c \cos(2 \theta_{k_1} ).
\end{equation}

When there is a transmitted particle (anti-particle) in region II, the transmitted wave function are respectively

\begin{dmath}\label{eqn:gradQuantumProblemSet4Problem3:680}
\Psi_{\textrm{trans}} =
\begin{bmatrix}
\cos\theta_{k_2} \\
\sin\theta_{k_2} \\
\end{bmatrix}
e^{i (k_2 x - E t/\Hbar)}
\end{dmath}
\begin{dmath}\label{eqn:gradQuantumProblemSet4Problem3:700}
\Psi_{\textrm{trans}} =
\begin{bmatrix}
-\sin\theta_{k_2} \\
\cos\theta_{k_2} \\
\end{bmatrix}
e^{i (k_2 x - E t/\Hbar)}.
\end{dmath}

The currents for these are respectively
\begin{dmath}\label{eqn:gradQuantumProblemSet4Problem3:1321}
j = c \lr{ \cos^2\theta_{k_2} - \sin^2\theta_{k_2} } = c \cos(2\theta_{k_2}),
\end{dmath}

and

\begin{dmath}\label{eqn:gradQuantumProblemSet4Problem3:1341}
j = c \lr{ \sin^2\theta_{k_2} - \cos^2\theta_{k_2} } = -c \cos(2\theta_{k_2}).
\end{dmath}

To determine the weighting of the wave functions, we require matching at the boundary.  There are a few cases to consider

\begin{enumerate}[(i)]
\item Total reflection.  Are there any special values of momentum that allow for total reflection?  If so, at the boundary both components must be zero

\begin{dmath}\label{eqn:gradQuantumProblemSet4Problem3:720}
\begin{bmatrix}
\cos\theta_{k_1} \\
\sin\theta_{k_1} \\
\end{bmatrix}
+
\frac{B}{A}
\begin{bmatrix}
\sin\theta_{k_1} \\
\cos\theta_{k_1} \\
\end{bmatrix}
=
\begin{bmatrix}
0 \\
0
\end{bmatrix}
\end{dmath}

This is possible if \( -B/A = \cot\theta_{k_1} = \tan\theta_{k_1} \), which requires

\begin{equation}\label{eqn:gradQuantumProblemSet4Problem3:740}
\theta_{k_1} = \frac{\pi}{4} \lr{ 1 + 2 n }, \qquad n \in \bbZ.
\end{equation}

However, we must also have

\begin{dmath}\label{eqn:gradQuantumProblemSet4Problem3:760}
\tan 2 \theta_1 = \frac{m c}{\Hbar k},
\end{dmath}

so

\begin{dmath}\label{eqn:gradQuantumProblemSet4Problem3:780}
\theta_1 = \inv{2} \Atan\lr{ \frac{m c}{\Hbar k} }.
\end{dmath}

Simultaneous solutions of \cref{eqn:gradQuantumProblemSet4Problem3:740}, \cref{eqn:gradQuantumProblemSet4Problem3:780} only occur at \( k = \pm 0 \), where \( \tan( (1 + 2 n)\pi/2 ) = \pm \infty \).  Since a particle at rest is not at interest in a reflection scenario, this shows that a decaying solution in region II must be introduced to match the boundary value constraints.

\item Ordinary matter transmission.

For this case at \( x = 0 \), we have

\begin{dmath}\label{eqn:gradQuantumProblemSet4Problem3:920}
A
\begin{bmatrix}
\cos\theta_{k_1} \\
\sin\theta_{k_1} \\
\end{bmatrix}
+
B
\begin{bmatrix}
\sin\theta_{k_1} \\
\cos\theta_{k_1} \\
\end{bmatrix}
=
D
\begin{bmatrix}
\cos \theta_2 \\
\sin \theta_2
\end{bmatrix}.
\end{dmath}

With \( a = B/A \) and \( b = D/A \), \( S_{1,2} = \sin\theta_{k_{1,2}}, C_{1,2} = \cos\theta_{k_{1,2}} \), this is

\begin{dmath}\label{eqn:gradQuantumProblemSet4Problem3:800}
\begin{bmatrix}
C_1 \\
S_1
\end{bmatrix}
=
\begin{bmatrix}
- S_1 & C_2 \\
- C_1 & S_2
\end{bmatrix}
\begin{bmatrix}
a \\
b
\end{bmatrix},
\end{dmath}

or
\begin{dmath}\label{eqn:gradQuantumProblemSet4Problem3:820}
\begin{bmatrix}
a \\
b
\end{bmatrix}
=
\inv{- S_1 S_2 + C_1 C_2 }
\begin{bmatrix}
S_2 & -C_2 \\
C_1 & -S_1
\end{bmatrix}
\begin{bmatrix}
C_1 \\
S_1
\end{bmatrix}
=
\inv{\cos(\theta_{k_1} + \theta_{k_2})}
\begin{bmatrix}
C_1 S_2 - S_1 C_2 \\
C_1^2 - S_1^2
\end{bmatrix},
\end{dmath}

which is
\boxedEquation{eqn:gradQuantumProblemSet4Problem3:840}{
\begin{aligned}
\frac{B}{A} &= \frac{ \cos(\theta_{k_2} - \theta_{k_1}) }{\cos(\theta_{k_1} + \theta_{k_2})} \\
\frac{D}{A} &= \frac{ \cos(2 \theta_{k_1}) }{\cos(\theta_{k_1} + \theta_{k_2})} \\
\end{aligned}
}

Let's verify that the currents in both regions match.  With \( A = 1 \), the region I current sum is

\begin{dmath}\label{eqn:gradQuantumProblemSet4Problem3:1000}
j_{\textrm{inc}}
+ j_{\textrm{ref}}
=
c \cos( 2 \theta_{k_1} ) - B^2 c \cos( 2 \theta_{k_1} )
=
c \cos( 2 \theta_{k_1} )
\lr{ 1 -
\frac{ \sin^2(\theta_{k_2} - \theta_{k_1}) }{\cos^2(\theta_{k_1} + \theta_{k_2})} }
=
c \cos( 2 \theta_{k_1} )
\frac{ \cos^2(\theta_{k_1} + \theta_{k_2}) - \sin^2(\theta_{k_2} - \theta_{k_1}) }{\cos^2(\theta_{k_1} + \theta_{k_2})}
=
c
\frac{ \cos( 2 \theta_{k_1} ) \cos(2 \theta_{k_1}) \cos(2 \theta_{k_2})}
{\cos^2(\theta_{k_1} + \theta_{k_2})}.
\end{dmath}

Whereas, the transmitted (region II) current is
\begin{dmath}\label{eqn:gradQuantumProblemSet4Problem3:1020}
j_{\textrm{trans}}
=
 c D^2 \cos( 2 \theta_{k_2} )
=
 c \cos( 2 \theta_{k_2} )
\frac{ \cos^2(2 \theta_{k_1}) }{\cos^2(\theta_{k_1} + \theta_{k_2})},
\end{dmath}

so we see \( j_{\textrm{inc}} + j_{\textrm{ref}} = j_{\textrm{trans}} \), as expected.

\item Anti-matter transmission.
\index{anti-matter transmission}

For this case at \( x = 0 \), we have

\begin{dmath}\label{eqn:gradQuantumProblemSet4Problem3:940}
A
\begin{bmatrix}
\cos\theta_{k_1} \\
\sin\theta_{k_1} \\
\end{bmatrix}
+
B
\begin{bmatrix}
\sin\theta_{k_1} \\
\cos\theta_{k_1} \\
\end{bmatrix}
=
D
\begin{bmatrix}
-\sin \theta_2 \\
\cos \theta_2 \\
\end{bmatrix}.
\end{dmath}

With the same substitutions as above, this is

\begin{dmath}\label{eqn:gradQuantumProblemSet4Problem3:860}
\begin{bmatrix}
C_1 \\
S_1
\end{bmatrix}
=
\begin{bmatrix}
- S_1 & -S_2 \\
- C_1 & C_2
\end{bmatrix}
\begin{bmatrix}
a \\
b
\end{bmatrix},
\end{dmath}

or
\begin{dmath}\label{eqn:gradQuantumProblemSet4Problem3:880}
\begin{bmatrix}
a \\
b
\end{bmatrix}
=
\inv{- (S_1 C_2 + S_2 C_1) }
\begin{bmatrix}
C_2 & -S_2 \\
C_1 & -S_1
\end{bmatrix}
\begin{bmatrix}
C_1 \\
S_1
\end{bmatrix}
=
\inv{-\sin(\theta_{k_1} + \theta_{k_2})}
\begin{bmatrix}
C_1 C_2 + S_1 S_2 \\
C_1^2 - S_1^2
\end{bmatrix},
\end{dmath}

which is
\boxedEquation{eqn:gradQuantumProblemSet4Problem3:900}{
\begin{aligned}
\frac{B}{A} &= \frac{ \cos(\theta_{k_1} - \theta_{k_2}) }{\sin(\theta_{k_1} + \theta_{k_2})} \\
\frac{D}{A} &= -\frac{ \cos(2 \theta_{k_1}) }{\sin(\theta_{k_1} + \theta_{k_2})} \\
\end{aligned}
}

For the anti-particle transmission, the region I current is

\begin{dmath}\label{eqn:gradQuantumProblemSet4Problem3:1001}
j_{\textrm{inc}}
+ j_{\textrm{ref}}
=
c \cos( 2 \theta_{k_1} ) - B^2 c \cos( 2 \theta_{k_1} )
=
c \cos( 2 \theta_{k_1} )
\lr{ 1 -
\frac{ \cos^2(\theta_{k_1} - \theta_{k_2}) }{\sin^2(\theta_{k_1} + \theta_{k_2})} }
=
c \cos( 2 \theta_{k_1} )
\frac{ \sin^2(\theta_{k_1} + \theta_{k_2}) - \cos^2(\theta_{k_1} - \theta_{k_2}) }{\cos^2(\theta_{k_1} + \theta_{k_2})}
=
-c
\frac{ \cos( 2 \theta_{k_1} ) \cos(2 \theta_{k_1}) \cos(2 \theta_{k_2})}
{\sin^2(\theta_{k_1} + \theta_{k_2})}.
\end{dmath}

Whereas, the transmitted (region II) current is
\begin{dmath}\label{eqn:gradQuantumProblemSet4Problem3:1021}
j_{\textrm{trans}}
=
 -c D^2 \cos( 2 \theta_{k_2} )
=
 -c \cos( 2 \theta_{k_2} )
\frac{ \cos^2(2 \theta_{k_1}) }{\sin^2(\theta_{k_1} + \theta_{k_2})},
\end{dmath}

and again we see \( j_{\textrm{inc}} + j_{\textrm{ref}} = j_{\textrm{trans}} \), as expected.

\item Decaying transmission.

In units where \( \Hbar = c = 1 \), the eigenkets found for the Dirac Hamiltonian, before normalization, were found to be

\begin{dmath}\label{eqn:gradQuantumProblemSet4Problem3:1041}
\ket{\pm} \propto
\begin{bmatrix}
\mp m \\
\epsilon_1 \pm k
\end{bmatrix}
e^{\pm i k x - i E t/\Hbar},
\end{dmath}

where \( \epsilon_1^2 = m^2 + k^2 \).  Letting \( k \rightarrow i k \), this provides the form of the non-oscillatory wavefunctions in region II when there is no ordinary nor anti-particle transmission.

\begin{dmath}\label{eqn:gradQuantumProblemSet4Problem3:1061}
\ket{\pm} \propto
\begin{bmatrix}
\mp m \\
\epsilon_2 \pm i k
\end{bmatrix}
e^{\mp k x - i E t/\Hbar},
\end{dmath}

where

\begin{dmath}\label{eqn:gradQuantumProblemSet4Problem3:1081}
\epsilon_2^2 = m^2 - k^2
\end{dmath}

Observe that the \( \psi_2 \) component of this wave function sits on a circle in the complex plane

\begin{dmath}\label{eqn:gradQuantumProblemSet4Problem3:1101}
\Abs{\epsilon_2 \pm i k}^2
= \epsilon^2 + k^2
= m^2 - k^2 + k^2
= m^2.
\end{dmath}

Putting back in the factors of \( \Hbar \), \( c\), this allows the identification

\begin{dmath}\label{eqn:gradQuantumProblemSet4Problem3:1121}
\epsilon_2 \pm i \Hbar k c = m c^2 e^{i\phi}.
\end{dmath}

So, like the trigonometric representation of the oscillatory wave function, the normalized wave functions for exponential decay(increase) can be written

\begin{dmath}\label{eqn:gradQuantumProblemSet4Problem3:1141}
\ket{\pm}
=
\inv{\sqrt{2}}
\begin{bmatrix}
\pm e^{\pm i \phi/2} \\
e^{\mp i \phi/2}
\end{bmatrix}
e^{\mp k x - i E t/\Hbar}
,
\end{dmath}

where the respective eigenvalues are \( \epsilon_2 = \pm \sqrt{ (m c^2)^2 - (\Hbar k c)^2} \), and \( \Hbar k c < m c^2 \).

Observe that the currents for the exponential wave functions are both light-like

\begin{equation}\label{eqn:gradQuantumProblemSet4Problem3:1161}
j_{\pm} = c \lr{ \Abs{\pm e^{\pm i \phi/2}}^2 - \Abs{e^{\mp i \phi/2}}^2 } = 0,
\end{equation}

which makes some sense since the particle is not able to propagate freely in the barrier region.

At the interface, we wish to solve

\begin{dmath}\label{eqn:gradQuantumProblemSet4Problem3:1181}
A
\begin{bmatrix}
C_1 \\
S_1
\end{bmatrix}
+
B
\begin{bmatrix}
S_1 \\
C_1 \\
\end{bmatrix}
=
\frac{D}{\sqrt{2}}
\begin{bmatrix}
e^{i \phi/2} \\
e^{-i \phi/2} \\
\end{bmatrix}.
\end{dmath}

With \( a = A/D \), and \( b = B/D \), this has solution

\begin{dmath}\label{eqn:gradQuantumProblemSet4Problem3:1201}
\begin{bmatrix}
a \\
b
\end{bmatrix}
=
\inv{\sqrt{2}}
{\begin{bmatrix}
C_1 & S_1 \\
S_1 & C_1
\end{bmatrix}}^{-1}
\begin{bmatrix}
e^{i \phi/2} \\
e^{-i \phi/2} \\
\end{bmatrix}
=
\inv{\sqrt{2} \cos(2 \theta_{k_1})}
\begin{bmatrix}
C_1 & -S_1 \\
-S_1 & C_1
\end{bmatrix}
\begin{bmatrix}
e^{i \phi/2} \\
e^{-i \phi/2} \\
\end{bmatrix}
=
\inv{\sqrt{2} \cos(2 \theta_{k_1})}
\begin{bmatrix}
C_1 e^{i \phi/2} -S_1 e^{-i \phi/2} \\
-S_1 e^{i \phi/2} + C_1 e^{-i \phi/2}
\end{bmatrix}.
\end{dmath}

The reflection coefficient is unity
\begin{dmath}\label{eqn:gradQuantumProblemSet4Problem3:1221}
R
= \Abs{\frac{B}{A}}^2
= \Abs{\frac{B/D}{A/D}}^2
= \frac{\Abs{-S_1 e^{i \phi/2} + C_1 e^{-i \phi/2}}^2}
{\Abs{C_1 e^{i \phi/2} -S_1 e^{-i \phi/2} }^2}
=
\frac
{ C_1^2 + S_1^2 - 2 S_1 C_1 \Real e^{-i\phi} }
{ C_1^2 + S_1^2 - 2 S_1 C_1 \Real e^{i\phi} }
=
\frac
{ 1 - \sin( 2 \theta_{k_1}) \cos \phi }
{ 1 - \sin( 2 \theta_{k_1}) \cos \phi }
=
1.
\end{dmath}

%%%\paragraph{Confusion:}
%%%Completely counter to (my) intuition, the transmission coefficient is not 0, so we don't appear to have \( R + T = 1 \).  Instead
%%%
%%%\begin{dmath}\label{eqn:gradQuantumProblemSet4Problem3:1241}
%%%T
%%%= \Abs{\frac{D}{A}}^2
%%%=
%%%\frac{2 \cos^2(2 \theta_{k_1})}
%%%{
%%%1 - \sin( 2 \theta_{k_1}) \cos \phi
%%%}
%%%\ne 0.
%%%\end{dmath}
%%%
%%%
%%%Temporarily reverting to natural units, note that
%%%
%%%\begin{dmath}\label{eqn:gradQuantumProblemSet4Problem3:1261}
%%%\cos\phi
%%%= \Real e^{i \phi}
%%%= \Real \lr{ \frac{\epsilon_2}{m} + i \frac{k_2}{m} }
%%%%=
%%%%\frac{
%%%%\sqrt{
%%%%m^2 - k_2^2
%%%%}
%%%%}{m}
%%%\frac{\epsilon_2}{m},
%%%\end{dmath}
%%%
%%%so
%%%\begin{dmath}\label{eqn:gradQuantumProblemSet4Problem3:1281}
%%%T
%%%=
%%%\frac{
%%%   2 \lr{\frac{k_1}{\epsilon_1}}^2
%%%}{
%%%1 - \frac{m}{\epsilon_1} \frac{\epsilon_2}{m}
%%%}
%%%=
%%%\frac{
%%%   2 k_1^2
%%%}{
%%%\epsilon_1^2 - \epsilon_1 \epsilon_2
%%%}
%%%\end{dmath}
%%%
%%%In terms of \(k_1, k_2 \) after putting back \( \Hbar, c \)'s this is
%%%
%%%\begin{dmath}\label{eqn:gradQuantumProblemSet4Problem3:1301}
%%%T
%%%=
%%%\frac{
%%%   2 (\Hbar k_1 c)^2
%%%}{
%%%\lr{ m c^2 }^2 + \lr{ \Hbar k_1 c }^2
%%%- \sqrt{
%%%\lr{(m c^2)^2 + \lr{ \Hbar k_1 c}^2 }
%%%\lr{(m c^2)^2 - \lr{ \Hbar k_2 c}^2 }
%%%}
%%%}.
%%%\end{dmath}

\end{enumerate}

}
}

