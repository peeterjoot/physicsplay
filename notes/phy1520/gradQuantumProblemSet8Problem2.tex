%
% Copyright � 2015 Peeter Joot.  All Rights Reserved.
% Licenced as described in the file LICENSE under the root directory of this GIT repository.
%
\makeproblem{Quadrupolar potential}{gradQuantum:problemSet8:2}{ 

Consider a p-orbital electron of hydrogen with \( \ket{ n,l = 1, m } \), with \( m = 0, \pm 1 \), subject to an external potential

\begin{dmath}\label{eqn:gradQuantumProblemSet8Problem2:20}
V(x, y, z) = \lambda(x^2 - y^2),
\end{dmath}

with \( \lambda \) being a constant. For fixed \( n \), obtain the correct eigenstates which diagonalize
the perturbation, without worrying about doing radial integrals explicitly. Show that the three-fold degeneracy of the
p-orbital is completely broken by the perturbation to linear order in \( \lambda \).

%\makesubproblem{}{gradQuantum:problemSet8:2a}
} % makeproblem

\makeanswer{gradQuantum:problemSet8:2}{ 
%\makeSubAnswer{}{gradQuantum:problemSet8:2a}

The potential in spherical coordinates is

\begin{dmath}\label{eqn:gradQuantumProblemSet8Problem2:40}
V = \lambda r^2 \sin^2\theta \lr{ \cos^2\phi - \sin^2\phi } = \lambda r^2 \sin^2\theta \cos(2 \phi).
\end{dmath}

The p-orbital wave functions are

\begin{dmath}\label{eqn:gradQuantumProblemSet8Problem2:60}
\psi_{n1m} = R_n(r) Y_{1,m}(\theta, \phi),
\end{dmath}

where
\begin{equation}\label{eqn:gradQuantumProblemSet8Problem2:80}
\begin{aligned}
Y_{1,1} &= -\frac{1}{2} \sqrt{\frac{3}{2 \pi }} e^{i \phi } \sin  \theta  \\
Y_{1,0} &= \frac{1}{2} \sqrt{\frac{3}{\pi }} \cos  \theta \\
Y_{1,-1} &= \frac{1}{2} \sqrt{\frac{3}{2 \pi }} e^{-i \phi } \sin  \theta.
\end{aligned}
\end{equation}

That is enough information to construct the matrix element of the perturbing potential with respect to these states.  Those are

\begin{dmath}\label{eqn:gradQuantumProblemSet8Problem2:100}
\bra{n' 1 m'} V \ket{n 1 m}
=
\int_0^\infty r^2 dr \int_0^\pi \sin\theta d\theta \int_0^{2 \pi} d\phi R_n(r) Y^\conj_{1, m'}(\theta, \phi) \lambda r^2 \sin^2\theta \cos(2 \phi) R_n(r) Y_{1, m}(\theta, \phi)
=
\lambda \int_0^\infty r^4 dr R^2_n(r)
\int_0^\pi \sin^3\theta \cos(2\phi) d\theta \int_0^{2 \pi} d\phi Y^\conj_{1, m'}(\theta, \phi) Y_{1, m}(\theta, \phi).
=
\lambda \int_0^\infty r^4 dr R^2_n(r)
\begin{bmatrix}
0 & 0 & -\frac{2}{5} \\
0 & 0 & 0 \\
-\frac{2}{5} & 0 & 0 \\
\end{bmatrix}.
\end{dmath}

This matrix has eigenvalues

\begin{dmath}\label{eqn:gradQuantumProblemSet8Problem2:120}
\lambda \int_0^\infty r^4 dr R^2_n(r)
\setlr{
 -\frac{2}{5}, 
 \frac{2}{5}, 
 0 
},
\end{dmath}

and respective eigenvectors
\begin{dmath}\label{eqn:gradQuantumProblemSet8Problem2:140}
\setlr{
\inv{\sqrt{2}}
\begin{bmatrix}
1 \\
0 \\
1
\end{bmatrix},
\begin{bmatrix}
0 \\
1 \\
0
\end{bmatrix},
\inv{\sqrt{2}}
\begin{bmatrix}
1 \\
0 \\
-1
\end{bmatrix}
}.
\end{dmath}

The wave functions that diagonalize this perturbation potential are 
\begin{dmath}\label{eqn:gradQuantumProblemSet8Problem2:160}
\begin{aligned}
\frac{R_n(r)}{\sqrt{2}} \lr{ Y_{-1,1} + Y_{1,1} } &= \frac{R_n(r)}{2 i} \sqrt{\frac{3}{\pi }} \sin\phi \sin\theta  \\
\frac{R_n(r)}{\sqrt{2}} \lr{ Y_{-1,1} - Y_{1,1} } &= -\frac{R_n(r)}{2} \sqrt{\frac{3}{\pi }} \cos\phi \sin\theta  \\
R_n(r) Y_{1,0} &= \frac{R_n(r)}{2} \sqrt{\frac{3}{\pi }} \cos\theta.
\end{aligned}
\end{dmath}

Regardless of the form of the radial factor of the wave function, we see that the energy levels are split according to \cref{eqn:gradQuantumProblemSet8Problem2:120}.
}
