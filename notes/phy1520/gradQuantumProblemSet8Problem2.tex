%
% Copyright � 2015 Peeter Joot.  All Rights Reserved.
% Licenced as described in the file LICENSE under the root directory of this GIT repository.
%
\makeproblem{Quadrupolar potential}{gradQuantum:problemSet8:2}{ 

Consider a p-orbital electron of hydrogen with \( \ket{ n,l = 1, m } \), with \( m = 0, \pm 1 \), subject to an external potential

\begin{dmath}\label{eqn:gradQuantumProblemSet8Problem2:20}
V(x, y, z) = \lambda(x^2 - y^2),
\end{dmath}

with \( \lambda \) being a constant. For fixed \( n \), obtain the correct eigenstates which diagonalize
the perturbation, without worrying about doing radial integrals explicitly. Show that the three-fold degeneracy of the
p-orbital is completely broken by the perturbation to linear order in \( \lambda \).

%\makesubproblem{}{gradQuantum:problemSet8:2a}
} % makeproblem

\makeanswer{gradQuantum:problemSet8:2}{ 
%\makeSubAnswer{}{gradQuantum:problemSet8:2a}

The potential in spherical coordinates is

\begin{dmath}\label{eqn:gradQuantumProblemSet8Problem2:n}
V = \lambda r^2 \sin^2\theta \lr{ \cos^2\phi - \sin^2\phi } = \lambda r^2 \sin^2\theta \cos(2 \phi).
\end{dmath}

The p-orbital wave functions are

\begin{dmath}\label{eqn:gradQuantumProblemSet8Problem2:n}
\psi_{n1m} = R_n(r) Y_{1,m}(\theta, \phi),
\end{dmath}

where
\begin{equation}\label{eqn:gradQuantumProblemSet8Problem2:n}
\begin{aligned}
Y_{1,1}
\end{aligned}
\end{equation}
so the matrix element 

}

\left(
\begin{array}{c}
 -\frac{1}{2} \sqrt{\frac{3}{2 \pi }} e^{i \phi } (\sin  \theta ) \\
 \frac{1}{2} \sqrt{\frac{3}{\pi }} (\cos  \theta ) \\
 \frac{1}{2} \sqrt{\frac{3}{2 \pi }} e^{-i \phi } (\sin  \theta ) \\
\end{array}
\right)

\left(
\begin{array}{ccc}
 0 & 0 & -\frac{2}{5} \\
 0 & 0 & 0 \\
 -\frac{2}{5} & 0 & 0 \\
\end{array}
\right)

\begin{array}{l}
 -\frac{2}{5} \\
 \frac{2}{5} \\
 0 \\
\end{array}

\left\{\frac{1}{\sqrt{2}},0,\frac{1}{\sqrt{2}}\right\}

\left(
\begin{array}{c}
 \frac{1}{\sqrt{2}} \\
 0 \\
 \frac{1}{\sqrt{2}} \\
\end{array}
\right)

\left\{-\frac{1}{\sqrt{2}},0,\frac{1}{\sqrt{2}}\right\}

\{0,1,0\}

\frac{Y_{1,-1}\text{($\theta $, $\phi $)}}{\sqrt{2}}+\frac{Y_{1,1}\text{($\theta $, $\phi $)}}{\sqrt{2}}

\frac{Y_{1,-1}\text{($\theta $, $\phi $)}}{\sqrt{2}}-\frac{Y_{1,1}\text{($\theta $, $\phi $)}}{\sqrt{2}}

\text{$\unicode{f3b5}\unicode{f7c1}\unicode{f7c9}\unicode{f7c8}$SubscriptBox[$\unicode{f7c9}$Y$\unicode{f7c0}$, $\unicode{f7c9}$1, 0$\unicode{f7c0}$]$\unicode{f7c0}$($\theta $, $\phi $)$\unicode{f3b5}$}
