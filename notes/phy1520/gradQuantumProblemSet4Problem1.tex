%
% Copyright � 2015 Peeter Joot.  All Rights Reserved.
% Licenced as described in the file LICENSE under the root directory of this GIT repository.
%
\makeproblem{Zitterbewegung in one dimension}{gradQuantum:problemSet4:1}{ 
Consider the Dirac Hamiltonian \( H = c \hat{p} \sigma_z + m c^2 \sigma_x \) . Using the Heisenberg equations of motion, derive a second order equation of motion (eom) for the velocity operator \( \hat{v} = d\hat{x}/dt \). For a state

\begin{dmath}\label{eqn:gradQuantumProblemSet4Problem1:20}
\Psi = 
\begin{bmatrix}
1 \\
0
\end{bmatrix}
e^{i k x}
\end{dmath}

with \( k = 0 \), average this eom in state \( \Psi \) to get a homogeneous second order differential equation for \( \expectation{\hat{v}} \). Using this equation of motion and the initial conditions on the velocity and its time-derivative, obtain \( \expectation{\hat{v}}(t) \) and \( \expectation{\hat{x}(t)} \).
Show that these oscillate with a rapid frequency \( 2 m c^2 \), with the oscillation of the position having an amplitude \( \Hbar/m c \) which is the Compton wavelength.  This `trembling' motion is called Zitterbewegung.

%\makesubproblem{}{gradQuantum:problemSet4:1a}
} % makeproblem

\makeanswer{gradQuantum:problemSet4:1}{ 
%\makeSubAnswer{}{gradQuantum:problemSet4:1a}

TODO.
}

