%
% Copyright � 2015 Peeter Joot.  All Rights Reserved.
% Licenced as described in the file LICENSE under the root directory of this GIT repository.
%
\makeproblem{Zitterbewegung in one dimension}{gradQuantum:problemSet4:1}{ 
\index{Zitterbewegung}
Consider the Dirac Hamiltonian \( H = c \hat{p} \sigma_z + m c^2 \sigma_x \) . Using the Heisenberg equations of motion, derive a second order equation of motion (eom) for the velocity operator \( \hat{v} = d\hat{x}/dt \). For a state

\begin{dmath}\label{eqn:gradQuantumProblemSet4Problem1:20}
\Psi = 
\begin{bmatrix}
1 \\
0
\end{bmatrix}
e^{i k x}
\end{dmath}

with \( k = 0 \), average this eom in state \( \Psi \) to get a homogeneous second order differential equation for \( \expectation{\hat{v}} \). Using this equation of motion and the initial conditions on the velocity and its time-derivative, obtain \( \expectation{\hat{v}}(t) \) and \( \expectation{\hat{x}(t)} \).
Show that these oscillate with a rapid frequency \( 2 m c^2/h \), with the oscillation of the position having an amplitude \( \Hbar/m c \) which is the Compton wavelength.  This `trembling' motion is called Zitterbewegung.

%\makesubproblem{}{gradQuantum:problemSet4:1a}
} % makeproblem

\makeanswer{gradQuantum:problemSet4:1}{ 
%\makeSubAnswer{}{gradQuantum:problemSet4:1a}

Using the hint from class that Zitterbewegung is associated with oscillation of the particle between ordinary-particle and anti-particle states, let's look for a combination of such \( k > 0 \) states to represent the state of this problem

\begin{dmath}\label{eqn:gradQuantumProblemSet4Problem1:40}
\begin{bmatrix}
1 \\
0
\end{bmatrix}
e^{i k x}
=
\evalbar{
\lr{
a 
\begin{bmatrix}
\cos\theta \\
\sin\theta \\
\end{bmatrix}
e^{i k x - i E t/\Hbar}
+ b
\begin{bmatrix}
-\sin\theta \\
\cos\theta \\
\end{bmatrix}
e^{i k x + i E t/\Hbar}
}
}{t = 0}.
\end{dmath}

Observe that \( \begin{bmatrix}
-\sin\theta \\
\cos\theta \\
\end{bmatrix} \) and \( \begin{bmatrix}
\cos\theta \\
\sin\theta \\
\end{bmatrix} \) are orthogonal, so 

\begin{equation}\label{eqn:gradQuantumProblemSet4Problem1:60}
a = 
\begin{bmatrix}
1 & 0 
\end{bmatrix}
\begin{bmatrix}
\cos\theta \\
\sin\theta \\
\end{bmatrix}
= \cos\theta.
\end{equation}
\begin{equation}\label{eqn:gradQuantumProblemSet4Problem1:80}
b =
\begin{bmatrix}
1 & 0 
\end{bmatrix}
\begin{bmatrix}
-\sin\theta \\
\cos\theta \\
\end{bmatrix}
= -\sin\theta.
\end{equation}

With 

\begin{equation}\label{eqn:gradQuantumProblemSet4Problem1:100}
\begin{aligned}
\ket{a} &=
\begin{bmatrix}
\cos\theta \\
\sin\theta \\
\end{bmatrix} 
e^{i k x - i E t/\Hbar}
\\
\ket{b} &=
\begin{bmatrix}
-\sin\theta \\
\cos\theta \\
\end{bmatrix} 
e^{i k x + i E t/\Hbar}
\end{aligned}
\end{equation}

The wave function of \cref{eqn:gradQuantumProblemSet4Problem1:20} can be represented as

\begin{dmath}\label{eqn:gradQuantumProblemSet4Problem1:120}
\Psi = 
\cos\theta
\ket{a}
- \sin\theta
\ket{b}.
\end{dmath}

The action of the Hamiltonian on this wave function is

\begin{dmath}\label{eqn:gradQuantumProblemSet4Problem1:140}
H \psi 
= i \Hbar \lr{ -i \frac{E}{\Hbar} C \ket{a} - i \frac{E}{\Hbar} S \ket{b} } 
= E \lr{ C \ket{a} + S \ket{b} },
\end{dmath}

Writing \( \epsilon = \sqrt{ (m c^2)^2 + (\Hbar k c)^2 } \), and noting that \( \ket{a} \), and \( \ket{b} \) are positive and negative eigenstates respectively,
%using the diagonalization from class, 
the spatial action of the Hamiltonian on this wave function is
\begin{dmath}\label{eqn:gradQuantumProblemSet4Problem1:160}
H \Psi
=
%\begin{bmatrix}
%\hat{p} c & m c^2 \\
%m c^2 & -\hat{p} c
%\end{bmatrix}
%\Psi
%=
%\begin{bmatrix}
%\Hbar k c & m c^2 \\
%m c^2 & -\Hbar k c
%\end{bmatrix}
%\Psi
%=
%\begin{bmatrix}
%C & -S \\
%S & C 
%\end{bmatrix}
%\begin{bmatrix}
%\epsilon & 0 \\
%0 & -\epsilon
%\end{bmatrix}
%\begin{bmatrix}
%C & S \\
%-S & C 
%\end{bmatrix}
%\Psi
%=
%\begin{bmatrix}
%C & -S \\
%S & C 
%\end{bmatrix}
%\begin{bmatrix}
%\epsilon & 0 \\
%0 & -\epsilon
%\end{bmatrix}
%\begin{bmatrix}
%C & S \\
%-S & C 
%\end{bmatrix}
%\lr{ 
%C 
%\begin{bmatrix}
%C \\
%S
%\end{bmatrix}
%e^{i k x - i E t/\Hbar}
%-S
%\begin{bmatrix}
%-S \\
%C \\
%\end{bmatrix}
%e^{i k x + i E t/\Hbar}
%}
%=
%\begin{bmatrix}
%C & -S \\
%S & C 
%\end{bmatrix}
%\begin{bmatrix}
%\epsilon & 0 \\
%0 & -\epsilon
%\end{bmatrix}
%\lr{ 
%C 
%\begin{bmatrix}
%C^2 + S^2 \\
%-S C + C S
%\end{bmatrix}
%e^{i k x - i E t/\Hbar}
%-S
%\begin{bmatrix}
%-S C + S C \\
%S^2 + C^2
%\end{bmatrix}
%e^{i k x + i E t/\Hbar}
%}
%=
%\begin{bmatrix}
%C & -S \\
%S & C 
%\end{bmatrix}
%\lr{ 
%C \epsilon
%\begin{bmatrix}
%1 \\
%0
%\end{bmatrix}
%e^{i k x - i E t/\Hbar}
%+ S \epsilon
%\begin{bmatrix}
%0 \\
%1
%\end{bmatrix}
%e^{i k x + i E t/\Hbar}
%}
%=
%\lr{ 
%C \epsilon
%\begin{bmatrix}
%C \\
%S
%\end{bmatrix}
%e^{i k x - i E t/\Hbar}
%+S \epsilon
%\begin{bmatrix}
%-S \\
%C
%\end{bmatrix}
%e^{i k x + i E t/\Hbar}
%}
= (+\epsilon) C \ket{a} - (-\epsilon) S \ket{b} 
= \epsilon \lr{ C \ket{a} + S\ket{b} }.
\end{dmath}

This provides a relationship between the energy \( E \) and the eigenvalues

\begin{equation}\label{eqn:gradQuantumProblemSet4Problem1:180}
E = \epsilon = \sqrt{ (m c^2)^2 + (\Hbar k c)^2 }.
\end{equation}

In particular, when \( k = 0 \), this is \( E = m c^2 \).

Using the Heisenberg equations of motion, the velocity operator is

\begin{dmath}\label{eqn:gradQuantumProblemSet4Problem1:200}
\hat{v} I
=
\ddt{\hat{x}} I
= \inv{i\Hbar }\antisymmetric{\hat{x} I}{ \hat{p} c \sigma_z + m c^2 \sigma_x }
= 
\inv{i\Hbar }
\lr{
\hat{x} \hat{p} c \sigma_z - \hat{p} c \sigma_z \hat{x}
+ \cancel{\hat{x} m c^2 \sigma_x} - \cancel{m c^2 \sigma_x \hat{x}}
}
= 
\inv{i\Hbar }
c \sigma_z \antisymmetric{\hat{x}}{\hat{p}}
= 
c \sigma_z.
\end{dmath}

As we've seen with the Harmonic oscillator, an expectation with respect to a state that is not a single eigenstate, will have non-zero time dependence.  With
\( C = \cos\theta \), \( S = \sin\theta \)
that expectation value with respect to the state as expressed in \cref{eqn:gradQuantumProblemSet4Problem1:120} is

\begin{dmath}\label{eqn:gradQuantumProblemSet4Problem1:220}
\expectation{\hat{v}}
=
\bra{\Psi} \hat{v} \ket{\Psi}
=
c \lr{ C \bra{a} - S \bra{b} } \sigma_z \lr{ C \ket{a} - S \ket{b} }
=
c \lr{ 
C \bra{a} - S \bra{b} } 
\begin{bmatrix}
1 & 0 \\
0 & -1
\end{bmatrix}
\lr{ 
\begin{bmatrix}
C \\
S
\end{bmatrix} 
e^{i k x - i E t/\Hbar}
- S
\begin{bmatrix}
-S \\
C
\end{bmatrix} 
e^{i k x + i E t/\Hbar}
}
=
c 
\lr{ 
C
\begin{bmatrix}
C & 
S
\end{bmatrix} 
e^{-i k x + i E t/\Hbar}
- S
\begin{bmatrix}
-S & C
\end{bmatrix} 
e^{-i k x - i E t/\Hbar}
}
\lr{ 
C
\begin{bmatrix}
C \\
-S
\end{bmatrix} 
e^{i k x - i E t/\Hbar}
+ S
\begin{bmatrix}
S \\
C
\end{bmatrix} 
e^{i k x + i E t/\Hbar}
}
= c \lr{
C^2 \cos(2\theta)
- S^2 \cos( 2 \theta )
+ S C \sin( 2 \theta ) e^{-2 i E t/\Hbar}
+ S C \sin( 2 \theta ) e^{2 i E t/\Hbar}
}
= 
c \lr{ \cos^2( 2 \theta ) + \sin^2( 2 \theta ) \cos(2 E t/\Hbar) }
= 
c \lr{ 
\frac{ (\Hbar c k)^2 }{ (m c^2)^2 + (\Hbar k c)^2 }
+\frac{ (m c^2)^2 }{ (m c^2)^2 + (\Hbar k c)^2 } \cos\lr{ 2 \pi \frac{2 E}{h} t }
}
= 
\frac{c}{ (m c^2)^2 + (\Hbar k c)^2 } \lr{ 
(\Hbar c k)^2 
+ 
(m c^2)^2 
\cos\lr{ 2 \pi \frac{2 E}{h} t }
}.
\end{dmath}

For \( k = 0 \), this is

\boxedEquation{eqn:gradQuantumProblemSet4Problem1:240}{
\expectation{\hat{v}} = c \cos\lr{ 2 \pi \frac{2 m c^2}{h} t }.
}

The frequency of this oscillation is \( \ifrac{2 m c^2}{h} \) as the problem states.  For the position expectation with respect to this state, we have

\begin{dmath}\label{eqn:gradQuantumProblemSet4Problem1:260}
\expectation{\hat{x}} 
= \expectation{\hat{x}}(0) + \frac{c \Hbar}{2 m c^2} \sin \lr{ \frac{2 m c^2}{\Hbar} t }
= \expectation{\hat{x}}(0) + \frac{\Hbar}{2 m c} \sin \lr{ \frac{2 m c^2}{\Hbar} t }.
\end{dmath}

The amplitude of this oscillation is

\begin{dmath}\label{eqn:gradQuantumProblemSet4Problem1:280}
2 \times \frac{\Hbar}{2 m c}
=
\frac{\Hbar}{m c},
\end{dmath}

as expected.
}
