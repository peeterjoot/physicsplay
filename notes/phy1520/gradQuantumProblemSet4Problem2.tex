%
% Copyright � 2015 Peeter Joot.  All Rights Reserved.
% Licenced as described in the file LICENSE under the root directory of this GIT repository.
%
\makeproblem{Jackiw-Rebbi problem}{gradQuantum:problemSet4:2}{ 

Recall that the energy of a relativistic particle is \( E(p) = \sqrt{p^2 c^2 + m^2 c^4 } \), which is independent of the sign of \( m \).
Thus \( m > 0 \) and \( m < 0 \) lead to the same dispersion relation.
Set aside for now, the physical meaning of \( m < 0 \).
Assume \( V(x) = 0 \) but let us assume the mass \( m \) is a function of position \( m(x) \).
This leads to

\begin{dmath}\label{eqn:gradQuantumProblemSet4Problem2:20}
H =
\begin{bmatrix}
c \hat{p} & m(x) c^2 \\
m(x) c^2 & -c \hat{p}
\end{bmatrix}
.
\end{dmath}

Let \( m(x) \) be such that \( m(-x) = -m(x) \), i.e., an odd function of position which changes sign at \( x = 0 \). Show that the operator \( \hat{\calP}_{\textrm{Dirac}} = \sigma_y \hat{\calP} \) commutes with the Hamiltonian, where

\begin{dmath}\label{eqn:gradQuantumProblemSet4Problem2:40}
\sigma_y
=
\PauliY
\end{dmath}

is the y-Pauli matrix, and \( \hat{\calP} \) is the parity operator which sends \( x \rightarrow -x\).
Consider the wavefunction

\begin{dmath}\label{eqn:gradQuantumProblemSet4Problem2:60}
\Phi(x) =
\begin{bmatrix}
f(x) \\
- i f(x)
\end{bmatrix},
\end{dmath}

where \( f(-x) = f(x) \) is an even function. Show \( \Phi(x) \) is an eigenstate of \( \hat{\calP}_{\textrm{Dirac}} \) with eigenvalue \( -1 \).
Next, assuming \( m(x > 0) = m_0 \) and \( m(x < 0) = -m_0 \) , where \( m_0 > 0 \), find \( f(x) \) such that \( \Phi(x) \) is an eigenstate of H with zero energy.
Normalize the wavefunction \( \Phi(x) \).

%\makesubproblem{}{gradQuantum:problemSet4:2a}
} % makeproblem

\makeanswer{gradQuantum:problemSet4:2}{ 
%\makeSubAnswer{}{gradQuantum:problemSet4:2a}

TODO.
}
