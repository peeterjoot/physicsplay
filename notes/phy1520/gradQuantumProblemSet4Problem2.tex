%
% Copyright � 2015 Peeter Joot.  All Rights Reserved.
% Licenced as described in the file LICENSE under the root directory of this GIT repository.
%
\makeproblem{Jackiw-Rebbi problem}{gradQuantum:problemSet4:2}{

Recall that the energy of a relativistic particle is \( E(p) = \sqrt{p^2 c^2 + m^2 c^4 } \), which is independent of the sign of \( m \).
Thus \( m > 0 \) and \( m < 0 \) lead to the same dispersion relation.
Set aside for now, the physical meaning of \( m < 0 \).
Assume \( V(x) = 0 \) but let us assume the mass \( m \) is a function of position \( m(x) \).
This leads to

\begin{dmath}\label{eqn:gradQuantumProblemSet4Problem2:20}
H =
\begin{bmatrix}
c \hat{p} & m(x) c^2 \\
m(x) c^2 & -c \hat{p}
\end{bmatrix}
.
\end{dmath}

Let \( m(x) \) be such that \( m(-x) = -m(x) \), i.e., an odd function of position which changes sign at \( x = 0 \).

\makesubproblem{}{gradQuantum:problemSet4:2a}
Show that the operator \( \hat{\calP}_{\textrm{Dirac}} = \sigma_y \hat{\calP} \) commutes with the Hamiltonian, where

\begin{dmath}\label{eqn:gradQuantumProblemSet4Problem2:40}
\sigma_y
=
\PauliY
\end{dmath}

is the y-Pauli matrix, and \( \hat{\calP} \) is the parity operator which sends \( x \rightarrow -x\).
\makesubproblem{}{gradQuantum:problemSet4:2b}
Consider the wavefunction

\begin{dmath}\label{eqn:gradQuantumProblemSet4Problem2:60}
\Phi(x) =
\begin{bmatrix}
f(x) \\
- i f(x)
\end{bmatrix},
\end{dmath}

where \( f(-x) = f(x) \) is an even function.  Show \( \Phi(x) \) is an eigenstate of \( \hat{\calP}_{\textrm{Dirac}} \) with eigenvalue \( -1 \).

\makesubproblem{}{gradQuantum:problemSet4:2c}
Next, assuming \( m(x > 0) = m_0 \) and \( m(x < 0) = -m_0 \) , where \( m_0 > 0 \), find \( f(x) \) such that \( \Phi(x) \) is an eigenstate of H with zero energy.

\makesubproblem{}{gradQuantum:problemSet4:2d}
Normalize the wavefunction \( \Phi(x) \).

} % makeproblem

\makeanswer{gradQuantum:problemSet4:2}{
\makeSubAnswer{}{gradQuantum:problemSet4:2a}

Let \( \pi \) represent a parity operator that acts on a scalar (operator), so the Dirac parity operator \( \hat{\calP}_{\textrm{Dirac}} \) takes the form

\begin{dmath}\label{eqn:gradQuantumProblemSet4Problem2:80}
\hat{\calP}_{\textrm{Dirac}} =
\begin{bmatrix}
0 & -i \pi \\
i \pi & 0
\end{bmatrix}.
\end{dmath}

The commutator with the Hamiltonian is

\begin{dmath}\label{eqn:gradQuantumProblemSet4Problem2:100}
\antisymmetric{H}{\hat{\calP}_{\textrm{Dirac}}}
= H \hat{\calP}_{\textrm{Dirac}} - \hat{\calP}_{\textrm{Dirac}} H
=
\begin{bmatrix}
c \hat{p} & m(x) c^2 \\
m(x) c^2 & -c \hat{p}
\end{bmatrix}
\begin{bmatrix}
0 & -i \pi \\
i \pi & 0
\end{bmatrix}
-
\begin{bmatrix}
0 & -i \pi \\
i \pi & 0
\end{bmatrix}
\begin{bmatrix}
c \hat{p} & m(x) c^2 \\
m(x) c^2 & -c \hat{p}
\end{bmatrix}
=
\begin{bmatrix}
i c^2 m(x) \pi & -i c \hat{p} \pi \\
-i c \hat{p} \pi & - i c^2 m(x) \pi
\end{bmatrix}
-
\begin{bmatrix}
-i c^2 \pi m(x) & i c \pi \hat{p} \\
i c \pi \hat{p} & i c^2 \pi m(x)
\end{bmatrix}
=
\begin{bmatrix}
i c^2 \symmetric{m(x)}{ \pi } & -i c \symmetric{\hat{p}}{\pi} \\
-i c \symmetric{\hat{p}}{\pi} & - i c^2 \symmetric{m(x)}{\pi}
\end{bmatrix}.
\end{dmath}

Now consider the matrix element of this commutator with respect to a position basis wavefunction \( \bra{x} \antisymmetric{H}{\hat{\calP}_{\textrm{Dirac}}} \ket{\Psi} \).  To compute this we must first understand the behaviour of the anticommutators of scalar operators \( m, \hat{p} \) with \( \pi \).  That is

\begin{dmath}\label{eqn:gradQuantumProblemSet4Problem2:120}
\bra{x} \symmetric{m(x)}{\pi} \ket{\psi}
=
\symmetric{m(x)}{\pi} \psi(x)
=
m(x) \pi \psi(x) + \pi ( m(x) \psi(x) )
=
m(x) \psi(-x) + m(-x) \psi(-x)
=
m(x) \psi(-x) - m(x) \psi(-x)
= 
0,
\end{dmath}

and

\begin{dmath}\label{eqn:gradQuantumProblemSet4Problem2:140}
\bra{x} \symmetric{\hat{p}}{\pi} \ket{\psi}
=
\symmetric{-i \Hbar \PD{x}{} }{\pi} \psi(x)
=
-i \Hbar \PD{x}{} \pi \psi(x) + \pi ( -i \Hbar \PD{x}{} \psi(x) )
=
-i \Hbar \PD{x}{} \psi(-x) + (-i \Hbar) \PD{(-x)}{} \psi(-x)
=
i \Hbar \PD{(-x)}{} \psi(-x) - i \Hbar \PD{(-x)}{} \psi(-x)
= 
0.
\end{dmath}

This shows that \( \bra{x} \antisymmetric{H}{\hat{\calP}_{\textrm{Dirac}}} \ket{\Psi} = 0 \).  Since this is true for any \( \ket{\Psi} \), we've shown that the Hamiltonian commutes with the Dirac parity operator

\boxedEquation{eqn:gradQuantumProblemSet4Problem2:160}{
H \hat{\calP}_{\textrm{Dirac}} = \hat{\calP}_{\textrm{Dirac}} H.
}

\makeSubAnswer{}{gradQuantum:problemSet4:2b}

This follows with direct substitution

\begin{dmath}\label{eqn:gradQuantumProblemSet4Problem2:180}
\hat{\calP}_{\textrm{Dirac}} \Phi(x)
=
\begin{bmatrix}
0 & -i \pi \\
i \pi & 0
\end{bmatrix}
\begin{bmatrix}
f(x) \\
-i f(x)
\end{bmatrix}
=
\begin{bmatrix}
- \pi f(x) \\
i \pi f(x)
\end{bmatrix}
=
\begin{bmatrix}
- f(-x) \\
i f(-x)
\end{bmatrix}
=
-
\begin{bmatrix}
f(x) \\
-i f(x)
\end{bmatrix}.
\end{dmath}

\makeSubAnswer{}{gradQuantum:problemSet4:2c}

TODO.
\makeSubAnswer{}{gradQuantum:problemSet4:2d}

TODO.
}
