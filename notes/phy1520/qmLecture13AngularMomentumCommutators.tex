%
% Copyright © 2015 Peeter Joot.  All Rights Reserved.
% Licenced as described in the file LICENSE under the root directory of this GIT repository.
%
\makeproblem{Angular momentum commutators}{problem:qmLecture13:1}{

Using \( \hat{L}_i = \epsilon_{ijk} \hat{r}_j \hat{p}_k \), show that 

\begin{dmath}\label{eqn:qmLecture13:1620}
\antisymmetric{\hat{L}_i}{\hat{L}_j} = i \Hbar \epsilon_{ijk} \hat{L}_k
\end{dmath}

} % problem

\makeanswer{problem:qmLecture13:1}{

Let's start without using abstract index expressions, computing the commutator for \( \hat{L}_1, \hat{L}_2 \), which should show the basic steps required

\begin{dmath}\label{eqn:qmLecture13:1640}
\antisymmetric{\hat{L}_1}{\hat{L}_2} 
=
\antisymmetric{\hat{r}_2 \hat{p}_3 - \hat{r}_3 \hat{p}_2}{\hat{r}_3 \hat{p}_1 - \hat{r}_1 \hat{p}_3}
=
\antisymmetric{\hat{r}_2 \hat{p}_3}{\hat{r}_3 \hat{p}_1}
-\antisymmetric{\hat{r}_2 \hat{p}_3}{\hat{r}_1 \hat{p}_3}
-\antisymmetric{\hat{r}_3 \hat{p}_2}{\hat{r}_3 \hat{p}_1}
+\antisymmetric{\hat{r}_3 \hat{p}_2}{\hat{r}_1 \hat{p}_3}.
\end{dmath}

The first of these commutators is

\begin{dmath}\label{eqn:qmLecture13:1660}
\antisymmetric{\hat{r}_2 \hat{p}_3}{\hat{r}_3 \hat{p}_1}
=
{\hat{r}_2 \hat{p}_3}{\hat{r}_3 \hat{p}_1}
-
{\hat{r}_3 \hat{p}_1}
{\hat{r}_2 \hat{p}_3}
=
\hat{r}_2 \hat{p}_1 \antisymmetric{\hat{p}_3}{\hat{r}_3}
=
-i \Hbar \hat{r}_2 \hat{p}_1.
\end{dmath}

We see that any factors in the commutator don't have like indexes (i.e. \( \hat{r}_k, \hat{p}_k \)) on both position and momentum terms, can be pulled out of the commutator.  This leaves

\begin{dmath}\label{eqn:qmLecture13:1680}
\antisymmetric{\hat{L}_1}{\hat{L}_2} 
=
\hat{r}_2 \hat{p}_1 \antisymmetric{\hat{p}_3}{\hat{r}_3}
-\cancel{\antisymmetric{\hat{r}_2 \hat{p}_3}{\hat{r}_1 \hat{p}_3}}
-\cancel{\antisymmetric{\hat{r}_3 \hat{p}_2}{\hat{r}_3 \hat{p}_1}}
+\hat{r}_1 \hat{p}_2 \antisymmetric{\hat{r}_3}{\hat{p}_3}
=
i \Hbar \lr{ \hat{r}_1 \hat{p}_2 - \hat{r}_2 \hat{p}_1 }
=
i \Hbar \hat{L}_3.
\end{dmath}

With cyclic permutation this is really enough to consider \cref{eqn:qmLecture13:1620} proven.  However, can we do this in the general case with the abstract index expression?  The quantity to simplify looks forbidding

\begin{dmath}\label{eqn:qmLecture13:1700}
\antisymmetric{\hat{L}_i}{\hat{L}_j}
=
\epsilon_{i a b }
\epsilon_{j s t }
\antisymmetric{ \hat{r}_a \hat{p}_b }{ \hat{r}_s \hat{p}_t }
\end{dmath}

Because there are no repeated indexes, this doesn't submit to any of the normal reduction identities.  Note however, since we only care about the \( i \ne j \) case, that one of the indexes \( a, b \) must be \( j \) for this quantity to be non-zero.  Therefore (for \( i \ne j \))

\begin{dmath}\label{eqn:qmLecture13:1720}
\antisymmetric{\hat{L}_i}{\hat{L}_j}
=
\epsilon_{i j b }
\epsilon_{j s t }
\antisymmetric{ \hat{r}_j \hat{p}_b }{ \hat{r}_s \hat{p}_t   }
+
\epsilon_{i a j }
\epsilon_{j s t }
\antisymmetric{ \hat{r}_a \hat{p}_j }{ \hat{r}_s \hat{p}_t   }
=
\epsilon_{i j b }
\epsilon_{j s t }
\lr{
\antisymmetric{ \hat{r}_j \hat{p}_b }{ \hat{r}_s \hat{p}_t   }
-
\antisymmetric{ \hat{r}_b \hat{p}_j }{ \hat{r}_s \hat{p}_t   }
}
=
-\delta^{s t}_{[i b]}
\antisymmetric{ \hat{r}_j \hat{p}_b - \hat{r}_b \hat{p}_j }{ \hat{r}_s \hat{p}_t }
=
\antisymmetric{ \hat{r}_j \hat{p}_b - \hat{r}_b \hat{p}_j }{ \hat{r}_b \hat{p}_i - \hat{r}_i \hat{p}_b }
=
  \antisymmetric{ \hat{r}_j \hat{p}_b }{ \hat{r}_b \hat{p}_i }
- \cancel{\antisymmetric{ \hat{r}_j \hat{p}_b }{ \hat{r}_i \hat{p}_b }}
- \cancel{\antisymmetric{ \hat{r}_b \hat{p}_j }{ \hat{r}_b \hat{p}_i }}
+ \antisymmetric{ \hat{r}_b \hat{p}_j }{ \hat{r}_i \hat{p}_b }
=
\hat{r}_j \hat{p}_i  \antisymmetric{ \hat{p}_b }{ \hat{r}_b }
+ \hat{r}_i \hat{p}_j \antisymmetric{ \hat{r}_b }{ \hat{p}_b }
=
 i \Hbar \lr{ \hat{r}_i \hat{p}_j - \hat{r}_j \hat{p}_i }
=
 i \Hbar \epsilon_{i j k} \hat{r}_i \hat{p}_j .
\end{dmath}

} % answer

%\EndArticle
