%
% Copyright � 2015 Peeter Joot.  All Rights Reserved.
% Licenced as described in the file LICENSE under the root directory of this GIT repository.
%
\section{Angular momentum}

In classical mechanics the (orbital) angular momentum is
\index{orbital angular momentum}
\index{angular momentum operator}

\begin{dmath}\label{eqn:qmLecture13:880}
\BL = \Br \cross \Bp.
\end{dmath}

Here ``orbital'' is to distinguish from spin angular momentum.

In quantum mechanics, the mapping to operators, in component form, is

\begin{dmath}\label{eqn:qmLecture13:900}
\hat{L}_i = \epsilon_{ijk} \hat{r}_j \hat{p}_k.
\end{dmath}

These operators do not commute
\begin{dmath}\label{eqn:qmLecture13:920}
\antisymmetric{\hat{L}_i}{\hat{L}_j} 
%= \epsilon_{i m n} \epsilon_{ijk} 
%\antisymmetric{\hat{r}_m \hat{p}_n}{\hat{r}_k \hat{p}_l}
%= 
%\delta^{[mn]}_{jk}
%\lr{
%\hat{r}_m \hat{p}_n \hat{r}_k \hat{p}_l
%-
%\hat{r}_k \hat{p}_l
%\hat{r}_m \hat{p}_n 
%}
%=
%\lr{
%\hat{r}_j \hat{p}_k \hat{r}_k \hat{p}_l
%-
%\hat{r}_k \hat{p}_j \hat{r}_k \hat{p}_l
%-
%\hat{r}_k \hat{p}_l
%\hat{r}_j \hat{p}_k 
%+
%\hat{r}_k \hat{p}_l
%\hat{r}_k \hat{p}_j 
%}
=
i \Hbar \epsilon_{ijk} \hat{L}_k.
\end{dmath}

%FIXME: fill in the details.

\index{vector operator}
which means that we can't simultaneously determine \( \hat{L}_i \) for all \( i \).  

Aside: In quantum mechanics, we define an operator \( \hat{\BV} \) to be a vector operator if

\begin{dmath}\label{eqn:qmLecture13:940}
\antisymmetric{\hat{L}_i}{\hatV_j} 
=
i \Hbar \epsilon_{ijk} \hatV_k.
\end{dmath}

The commutator of the squared angular momentum operator with any \( \hat{L}_i \), say \( \hat{L}_x \) is zero

\begin{dmath}\label{eqn:qmLecture13:960}
\begin{aligned}
\antisymmetric{
\hat{L}_x^2 +
\hat{L}_y^2 +
\hat{L}_z^2
}
{\hat{L}_x}
&=
\hat{L}_y \hat{L}_y \hat{L}_x
- \hat{L}_x \hat{L}_y \hat{L}_y
+
\hat{L}_z \hat{L}_z \hat{L}_x
- \hat{L}_x \hat{L}_z \hat{L}_z \\
&=
\hat{L}_y \lr{ \antisymmetric{\hat{L}_y}{\hat{L}_x} + \cancel{\hat{L}_x \hat{L}_y} }
-\lr{ \antisymmetric{\hat{L}_x}{\hat{L}_y} + \cancel{\hat{L}_y \hat{L}_x} } \hat{L}_y \\
&\quad +\hat{L}_z \lr{ \antisymmetric{\hat{L}_z}{\hat{L}_x} + \cancel{\hat{L}_x \hat{L}_z} }
-\lr{ \antisymmetric{\hat{L}_x}{\hat{L}_z} + \cancel{\hat{L}_z \hat{L}_x} } \hat{L}_z \\
&=
\hat{L}_y \antisymmetric{\hat{L}_y}{\hat{L}_x} 
-\antisymmetric{\hat{L}_x}{\hat{L}_y} \hat{L}_y
+\hat{L}_z \antisymmetric{\hat{L}_z}{\hat{L}_x} 
-\antisymmetric{\hat{L}_x}{\hat{L}_z} \hat{L}_z \\
&=
i \Hbar \lr{
-\hat{L}_y \hat{L}_z
- \hat{L}_z \hat{L}_y
+\hat{L}_z \hat{L}_y
+ \hat{L}_y \hat{L}_z
} \\
&= 
0.
\end{aligned}
\end{dmath}
%
%In fact 
%\begin{dmath}\label{eqn:qmLecture13:980}
%\antisymmetric{\hat{\BL}^2 }{\hat{L}_i} = 0.
%\end{dmath}

Suppose we have a state \( \ket{\Psi} \) with a well defined \( \hat{L}_z \) eigenvalue and well defined \( \hat{\BL^2} \) eigenvalue, written as

\begin{dmath}\label{eqn:qmLecture13:1000}
\ket{\Psi} = \ket{a, b},
\end{dmath}

where the label \( a \) is used for the eigenvalue of \( \hat{\BL}^2 \) and \( b \) labels the eigenvalue of \( \hat{L}_z \).  Then

\begin{equation}\label{eqn:qmLecture13:1020}
\begin{aligned}
\hat{\BL}^2 \ket{a , b} &= \Hbar^2 a \ket{a ,b} \\
\hat{L}_z \ket{a , b} &= \Hbar b \ket{a ,b}.
\end{aligned}
\end{equation}

Things aren't so nice when we act with other angular momentum operators, producing a scrambled mess

\begin{equation}\label{eqn:qmLecture13:1040}
\begin{aligned}
\hat{L}_x \ket{a , b} &= \sum_{a', b'} \calA^x_{a, b, a', b'} \ket{a', b'} \\
\hat{L}_y \ket{a , b} &= \sum_{a', b'} \calA^y_{a, b, a', b'} \ket{a', b'} \\
\end{aligned}
\end{equation}

With this representation, we have

\begin{dmath}\label{eqn:qmLecture13:1060}
\hat{L}_x \hat{\BL}^2 \ket{a, b}
=
\hat{L}_x \Hbar^2 a
\sum_{a', b'} \calA^x_{a, b, a', b'} \ket{a', b'}.
\end{dmath}

\begin{dmath}\label{eqn:qmLecture13:1080}
\hat{\BL}^2 \hat{L}_x \ket{a, b}
=
\Hbar^2 
\sum_{a', b'} a' \calA^x_{a, b, a', b'} \ket{a', b'}.
\end{dmath}

Since \( \hat{\BL}^2, \hat{L}_x \) commute, we must have

\begin{dmath}\label{eqn:qmLecture13:1100}
\calA^x_{a, b, a', b'} = \delta_{a, a'} \calA^x_{a'; b, b'},
\end{dmath}

and similarly
\begin{dmath}\label{eqn:qmLecture13:1120}
\calA^y_{a, b, a', b'} = \delta_{a, a'} \calA^y_{a'; b, b'}.
\end{dmath}

Simplifying things we can write the action of \( \hat{L}_x, \hat{L}_y \) on the state as

\begin{dmath}\label{eqn:qmLecture13:1140}
\begin{aligned}
\hat{L}_x \ket{a , b} &= \sum_{ b'} \calA^x_{a; b, b'} \ket{a, b'} \\
\hat{L}_y \ket{a , b} &= \sum_{ b'} \calA^y_{a; b, b'} \ket{a, b'} \\
\end{aligned}
\end{dmath}

Let's define
\begin{dmath}\label{eqn:qmLecture13:1160}
\begin{aligned}
\hat{L}_{+} &\equiv \hat{L}_x + i \hat{L}_y \\
\hat{L}_{-} &\equiv \hat{L}_x - i \hat{L}_y \\
\end{aligned}
\end{dmath}

Because these are sums of \( \hat{L}_x, \hat{L}_y \) they must also commute with \( \hat{\BL}^2 \)

\begin{dmath}\label{eqn:qmLecture13:1180}
\antisymmetric{\hat{\BL}^2}{\hat{L}_{\pm}} = 0.
\end{dmath}

The commutators with \( \hat{L}_z \) are non-zero

\begin{dmath}\label{eqn:qmLecture13:1740}
\antisymmetric{\hat{L}_z}{\hat{L}_{\pm}} 
= 
\hat{L}_z \lr{ \hat{L}_x \pm i \hat{L}_y }
- \lr{ \hat{L}_x \pm i \hat{L}_y } \hat{L}_z
=
\antisymmetric{\hat{L}_z}{\hat{L}_x}
\pm i 
\antisymmetric{\hat{L}_z}{\hat{L}_y}
=
i \Hbar \lr{ 
\hat{L}_y \mp i \hat{L}_x
}
=
\Hbar \lr{ i \hat{L}_y \pm \hat{L}_x }
=
\pm \Hbar \lr{ \hat{L}_x \pm i \hat{L}_y }
=
\pm \Hbar \hat{L}_{\pm}.
\end{dmath}

%\begin{dmath}\label{eqn:qmLecture13:1200}
%\antisymmetric{\hat{L}_z}{\hat{L}_{\pm}} = \pm \Hbar \hat{L}_{\pm}.
%\end{dmath}
%
Explicitly, that is

\begin{dmath}\label{eqn:qmLecture13:1220}
\begin{aligned}
\hat{L}_z \hat{L}_{+} - \hat{L}_{+} \hat{L}_z &= \Hbar \hat{L}_{+} \\
\hat{L}_z \hat{L}_{-} - \hat{L}_{-} \hat{L}_z &= -\Hbar \hat{L}_{-}
\end{aligned}
\end{dmath}

Now we are set to compute actions of these (assumed) raising and lowering operators on the eigenstate of \( \hat{L}_z, \hat{\BL}^2 \)

\begin{dmath}\label{eqn:qmLecture13:1240}
\hat{L}_z \hat{L}_{\pm} \ket{a, b} 
= 
\Hbar \hat{L}_{\pm} \ket{a,b} \pm \hat{L}_{\pm} \hat{L}_z \ket{a,b}
= 
\Hbar \hat{L}_{\pm} \ket{a,b} \pm \Hbar b \hat{L}_{\pm} \ket{a,b}
=
\Hbar \lr{ b \pm 1 } \hat{L}_{\pm} \ket{a, b} .
\end{dmath}

There must be a proportionality of the form

\begin{dmath}\label{eqn:qmLecture13:1260}
\ket{\hat{L}_{\pm}} \propto \ket{a, b \pm 1},
\end{dmath}

The products of the raising and lowering operators are

\begin{dmath}\label{eqn:qmLecture13:1280}
\hat{L}_{-} \hat{L}_{+} 
=
\lr{ \hat{L}_x - i \hat{L}_y }
\lr{ \hat{L}_x + i \hat{L}_y }
=
\hat{L}_x^2 + \hat{L}_y^2 + i \hat{L}_x \hat{L}_y - i \hat{L}_y \hat{L}_x
=
\lr{ \hat{\BL}^2 - \hat{L}_z^2 } + i \antisymmetric{\hat{L}_x}{\hat{L}_y}
=
\hat{\BL}^2 - \hat{L}_z^2 - \Hbar \hat{L}_z,
\end{dmath}

and
\begin{dmath}\label{eqn:qmLecture13:1300}
\hat{L}_{+} \hat{L}_{-} 
=
\lr{ \hat{L}_x + i \hat{L}_y }
\lr{ \hat{L}_x - i \hat{L}_y }
=
\hat{L}_x^2 + \hat{L}_y^2 - i \hat{L}_x \hat{L}_y + i \hat{L}_y \hat{L}_x
=
\lr{ \hat{\BL}^2 - \hat{L}_z^2 } - i \antisymmetric{\hat{L}_x}{\hat{L}_y}
=
\hat{\BL}^2 - \hat{L}_z^2 + \Hbar \hat{L}_z,
\end{dmath}

So we must have

\begin{equation}\label{eqn:qmLecture13:1320}
0 
\le \bra{a, b} \hat{L}_{-} \hat{L}_{+} \ket{a, b} 
= 
\bra{a, b} 
\lr{ \hat{\BL}^2 - \hat{L}_z^2 - \Hbar \hat{L}_z }
\ket{a, b} 
=
\Hbar^2 a - \Hbar^2 b^2 - \Hbar^2 b,
\end{equation}

and

\begin{equation}\label{eqn:qmLecture13:1340}
0 
\le \bra{a, b} \hat{L}_{+} \hat{L}_{-} \ket{a, b} 
= 
\bra{a, b} 
\lr{ \hat{\BL}^2 - \hat{L}_z^2 + \Hbar \hat{L}_z }
\ket{a, b} 
=
\Hbar^2 a - \Hbar^2 b^2 + \Hbar^2 b.
\end{equation}

This puts constraints on \( a, b \), roughly of the form

\begin{enumerate}
\item
\begin{equation}\label{eqn:qmLecture13:1360}
a - b( b + 1) \ge 0
\end{equation}

With \( b_{\textrm{max}} > 0 \), \( b_{\textrm{max}} \approx \sqrt{a} \).

\item
\begin{equation}\label{eqn:qmLecture13:1380}
a - b( b - 1) \ge 0
\end{equation}

With \( b_{\textrm{min}} < 0 \), \( b_{\textrm{max}} \approx -\sqrt{a} \).

\end{enumerate}

%\paragraph{I Love and Desire Sofia Always!}

