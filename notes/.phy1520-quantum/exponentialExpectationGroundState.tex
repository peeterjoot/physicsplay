%
% Copyright � 2015 Peeter Joot.  All Rights Reserved.
% Licenced as described in the file LICENSE under the root directory of this GIT repository.
%
%\newcommand{\authorname}{Peeter Joot}
\newcommand{\email}{peeterjoot@protonmail.com}
\newcommand{\basename}{FIXMEbasenameUndefined}
\newcommand{\dirname}{notes/FIXMEdirnameUndefined/}

%\renewcommand{\basename}{exponentialExpectationGroundState}
%\renewcommand{\dirname}{notes/phy1520/}
%%\newcommand{\dateintitle}{}
%%\newcommand{\keywords}{}
%
%\newcommand{\authorname}{Peeter Joot}
\newcommand{\onlineurl}{http://sites.google.com/site/peeterjoot2/math2013/\basename.pdf}
\newcommand{\sourcepath}{\dirname\basename.tex}
\newcommand{\generatetitle}[1]{\chapter{#1}}

\newcommand{\vcsinfo}{%
\section*{}
\noindent{\color{DarkOliveGreen}{\rule{\linewidth}{0.1mm}}}
\paragraph{Document version}
%\paragraph{\color{Maroon}{Document version}}
{
\small
\begin{itemize}
\item Available online at:\\ 
\href{\onlineurl}{\onlineurl}
\item Git Repository: \input{./.revinfo/gitRepo.tex}
\item Source: \sourcepath
\item last commit: \input{./.revinfo/gitCommitString.tex}
\item commit date: \input{./.revinfo/gitCommitDate.tex}
\end{itemize}
}
}

%\PassOptionsToPackage{dvipsnames,svgnames}{xcolor}
\PassOptionsToPackage{square,numbers}{natbib}
\documentclass{scrreprt}

\usepackage[left=2cm,right=2cm]{geometry}
\usepackage[svgnames]{xcolor}
\usepackage{peeters_layout}

\usepackage{natbib}

\usepackage[
colorlinks=true,
bookmarks=false,
pdfauthor={\authorname, \email},
backref 
]{hyperref}

% http://tex.stackexchange.com/questions/75773/how-to-reference-problems-by-the-text-label-in-an-exercise-envioronment
\usepackage[english]{cleveref}
\crefname{Exercise}{exercise}{exercises}
\Crefname{Exercise}{Exercise}{Exercises}

\RequirePackage{titlesec}
\RequirePackage{ifthen}

% http://stackoverflow.com/questions/4932910/date-in-the-tabular-environment
\makeatletter
\let\insertdate\@date
\makeatother

\titleformat{\chapter}[display]
{\bfseries\Large}
{\color{DarkSlateGrey}\filleft \authorname
\ifthenelse{\isundefined{\studentnumber}}{}{\\ \studentnumber}
\ifthenelse{\isundefined{\email}}{}{\\ \email}
\ifthenelse{\isundefined{\dateintitle}}{}{\\ \insertdate}
%\ifthenelse{\isundefined{\coursename}}{}{\\ \coursename} % put in title instead.
}
{4ex}
{\color{DarkOliveGreen}{\titlerule}\color{Maroon}
\vspace{2ex}%
\filright}
[\vspace{2ex}%
\color{DarkOliveGreen}\titlerule
]

\newcommand{\beginArtWithToc}[0]{\begin{document}\tableofcontents}
\newcommand{\beginArtNoToc}[0]{\begin{document}}
\newcommand{\EndNoBibArticle}[0]{\end{document}}
\newcommand{\EndArticle}[0]{\bibliography{Bibliography}\bibliographystyle{plainnat}\end{document}}

% 
%\newcommand{\citep}[1]{\cite{#1}}

\colorSectionsForArticle


%
%\usepackage{peeters_layout_exercise}
%\usepackage{peeters_braket}
%\usepackage{peeters_figures}
%
%\beginArtNoToc
%
%\generatetitle{Plane wave ground state expectation}
%%\chapter{Plane wave ground state expectation}
%%\label{chap:exponentialExpectationGroundState}

\makeoproblem{Plane wave ground state expectation for 1D SHO.}{problem:exponentialExpectationGroundState:1}{\citep{sakurai2014modern} pr. 2.18}{
\index{harmonic oscillator!ground state}

For a 1D SHO, show that

\begin{dmath}\label{eqn:exponentialExpectationGroundState:20}
\bra{0} e^{i k x} \ket{0} = \exp\lr{ -k^2 \bra{0} x^2 \ket{0}/2 }.
\end{dmath}

} % problem

\makeanswer{problem:exponentialExpectationGroundState:1}{
Despite the simple appearance of this problem, I found this quite involved to show.  To do so, start with a series expansion of the expectation

\begin{dmath}\label{eqn:exponentialExpectationGroundState:40}
\bra{0} e^{i k x} \ket{0}
=
\sum_{m=0}^\infty \frac{(i k)^m}{m!} \bra{0} x^m \ket{0}.
\end{dmath}

Let

\begin{dmath}\label{eqn:exponentialExpectationGroundState:60}
X = \lr{ a + a^\dagger },
\end{dmath}

so that

\begin{equation}\label{eqn:exponentialExpectationGroundState:80}
x
= \sqrt{\frac{\Hbar}{2 \omega m}} X
= \frac{x_0}{\sqrt{2}} X.
\end{equation}

Consider the first few values of \( \bra{0} X^n \ket{0} \)

\begin{dmath}\label{eqn:exponentialExpectationGroundState:100}
\bra{0} X \ket{0}
=
\bra{0} \lr{ a + a^\dagger } \ket{0}
=
\braket{0}{1}
=
0,
\end{dmath}

\begin{dmath}\label{eqn:exponentialExpectationGroundState:120}
\bra{0} X^2 \ket{0}
=
\bra{0} \lr{ a + a^\dagger }^2 \ket{0}
=
\braket{1}{1}
=
1,
\end{dmath}

\begin{dmath}\label{eqn:exponentialExpectationGroundState:140}
\bra{0} X^3 \ket{0}
=
\bra{0} \lr{ a + a^\dagger }^3 \ket{0}
=
\bra{1} \lr{ \sqrt{2} \ket{2} + \ket{0} }
=
0.
\end{dmath}

Whenever the power \( n \) in \( X^n \) is even, the braket can be split into a bra that has only contributions from odd eigenstates and a ket with even eigenstates.  We conclude that \( \bra{0} X^n \ket{0} = 0 \) when \( n \) is odd.

Noting that \( \bra{0} x^2 \ket{0} = \ifrac{x_0^2}{2} \), this leaves

\begin{dmath}\label{eqn:exponentialExpectationGroundState:160}
\bra{0} e^{i k x} \ket{0}
=
\sum_{m=0}^\infty \frac{(i k)^{2 m}}{(2 m)!} \bra{0} x^{2m} \ket{0}
=
\sum_{m=0}^\infty \frac{(i k)^{2 m}}{(2 m)!} \lr{ \frac{x_0^2}{2} }^m \bra{0} X^{2m} \ket{0}
=
\sum_{m=0}^\infty \frac{1}{(2 m)!} \lr{ -k^2 \bra{0} x^2 \ket{0} }^m \bra{0} X^{2m} \ket{0}.
\end{dmath}

This problem is now reduced to showing that

\begin{dmath}\label{eqn:exponentialExpectationGroundState:180}
\frac{1}{(2 m)!} \bra{0} X^{2m} \ket{0} = \inv{m! 2^m},
\end{dmath}

or

\begin{dmath}\label{eqn:exponentialExpectationGroundState:200}
\bra{0} X^{2m} \ket{0}
= \frac{(2m)!}{m! 2^m}
= \frac{ (2m)(2m-1)(2m-2) \cdots (2)(1) }{2^m m!}
= \frac{ 2^m (m)(2m-1)(m-1)(2m-3)(m-2) \cdots (2)(3)(1)(1) }{2^m m!}
= (2m-1)!!,
\end{dmath}

where \( n!! = n(n-2)(n-4)\cdots \).

It looks like \( \bra{0} X^{2m} \ket{0} \) can be expanded by inserting an identity operator and proceeding recursively, like

\begin{dmath}\label{eqn:exponentialExpectationGroundState:220}
\bra{0} X^{2m} \ket{0}
=
\bra{0} X^2 \lr{ \sum_{n=0}^\infty \ket{n}\bra{n} } X^{2m-2} \ket{0}
=
\bra{0} X^2 \lr{ \ket{0}\bra{0} + \ket{2}\bra{2} } X^{2m-2} \ket{0}
=
\bra{0} X^{2m-2} \ket{0} + \bra{0} X^2 \ket{2} \bra{2} X^{2m-2} \ket{0}.
\end{dmath}

This has made use of the observation that \( \bra{0} X^2 \ket{n} = 0 \) for all \( n \ne 0,2 \).  The remaining term includes the factor

\begin{dmath}\label{eqn:exponentialExpectationGroundState:240}
\bra{0} X^2 \ket{2}
=
\bra{0} \lr{a + a^\dagger}^2 \ket{2}
=
\lr{ \bra{0} + \sqrt{2} \bra{2} } \ket{2}
=
\sqrt{2},
\end{dmath}

Since \( \sqrt{2} \ket{2} = \lr{a^\dagger}^2 \ket{0} \), the expectation of interest can be written

\begin{dmath}\label{eqn:exponentialExpectationGroundState:260}
\bra{0} X^{2m} \ket{0}
=
\bra{0} X^{2m-2} \ket{0} + \bra{0} a^2 X^{2m-2} \ket{0}.
\end{dmath}

How do we expand the second term.  Let's look at how \( a \) and \( X \) commute

\begin{dmath}\label{eqn:exponentialExpectationGroundState:280}
a X
=
\antisymmetric{a}{X} + X a
=
\antisymmetric{a}{a + a^\dagger} + X a
=
\antisymmetric{a}{a^\dagger} + X a
=
1 + X a,
\end{dmath}

\begin{dmath}\label{eqn:exponentialExpectationGroundState:300}
a^2 X
=
a \lr{ a X }
=
a \lr{ 1 + X a }
=
a + a X a
=
a + \lr{ 1 + X a } a
=
2 a + X a^2.
\end{dmath}

Proceeding to expand \( a^2 X^n \) we find
\begin{equation}\label{eqn:exponentialExpectationGroundState:320}
\begin{aligned}
a^2 X^3 &= 6 X + 6 X^2 a + X^3 a^2 \\
a^2 X^4 &= 12 X^2 + 8 X^3 a + X^4 a^2 \\
a^2 X^5 &= 20 X^3 + 10 X^4 a + X^5 a^2 \\
a^2 X^6 &= 30 X^4 + 12 X^5 a + X^6 a^2.
\end{aligned}
\end{equation}

It appears that we have
\begin{equation}\label{eqn:exponentialExpectationGroundState:340}
\antisymmetric{a^2 X^n}{X^n a^2} = \beta_n X^{n-2} + 2 n X^{n-1} a,
\end{equation}

where

\begin{equation}\label{eqn:exponentialExpectationGroundState:360}
\beta_n = \beta_{n-1} + 2 (n-1),
\end{equation}

and \( \beta_2 = 2 \).  Some goofing around shows that \( \beta_n = n(n-1) \), so the induction hypothesis is

\begin{equation}\label{eqn:exponentialExpectationGroundState:380}
\antisymmetric{a^2 X^n}{X^n a^2} = n(n-1) X^{n-2} + 2 n X^{n-1} a.
\end{equation}

Let's check the induction
\begin{dmath}\label{eqn:exponentialExpectationGroundState:400}
a^2 X^{n+1}
=
a^2 X^{n} X
=
\lr{ n(n-1) X^{n-2} + 2 n X^{n-1} a + X^n a^2 } X
=
n(n-1) X^{n-1} + 2 n X^{n-1} a X + X^n a^2 X
=
n(n-1) X^{n-1} + 2 n X^{n-1} \lr{ 1 + X a } + X^n \lr{ 2 a + X a^2 }
=
n(n-1) X^{n-1} + 2 n X^{n-1}  + 2 n X^{n} a
+ 2 X^n a
+ X^{n+1} a^2
=
X^{n+1} a^2 + (2 + 2 n) X^{n} a + \lr{ 2 n + n(n-1) }  X^{n-1}
=
X^{n+1} a^2 + 2(n + 1) X^{n} a + (n+1) n X^{n-1},
\end{dmath}

which concludes the induction, giving

\begin{dmath}\label{eqn:exponentialExpectationGroundState:420}
\bra{ 0 } a^2 X^{n} \ket{0 } = n(n-1) \bra{0} X^{n-2} \ket{0},
\end{dmath}

and

\begin{dmath}\label{eqn:exponentialExpectationGroundState:440}
\bra{0} X^{2m} \ket{0}
=
\bra{0} X^{2m-2} \ket{0} + (2m-2)(2m-3) \bra{0} X^{2m-4} \ket{0}.
\end{dmath}

Let

\begin{dmath}\label{eqn:exponentialExpectationGroundState:460}
\sigma_{n} = \bra{0} X^n \ket{0},
\end{dmath}

so that the recurrence relation, for \( 2n \ge 4 \) is

\begin{dmath}\label{eqn:exponentialExpectationGroundState:480}
\sigma_{2n} = \sigma_{2n -2} + (2n-2)(2n-3) \sigma_{2n -4}
\end{dmath}

We want to show that this simplifies to

\begin{dmath}\label{eqn:exponentialExpectationGroundState:500}
\sigma_{2n} = (2n-1)!!
\end{dmath}

The first values are

\begin{subequations}
\label{eqn:exponentialExpectationGroundState:520}
\begin{equation}\label{eqn:exponentialExpectationGroundState:540}
\sigma_0 = \bra{0} X^0 \ket{0} = 1
\end{equation}
\begin{equation}\label{eqn:exponentialExpectationGroundState:560}
\sigma_2 = \bra{0} X^2 \ket{0} = 1
\end{equation}
\end{subequations}

which gives us the right result for the first term in the induction

\begin{dmath}\label{eqn:exponentialExpectationGroundState:580}
\sigma_4
= \sigma_2 + 2 \times 1 \times \sigma_0
= 1 + 2
= 3!!
\end{dmath}

For the general induction term, consider

\begin{dmath}\label{eqn:exponentialExpectationGroundState:600}
\sigma_{2n + 2}
= \sigma_{2n} + 2 n (2n - 1) \sigma_{2n -2}
= (2n-1)!! + 2n ( 2n - 1) (2n -3)!!
= (2n + 1) (2n -1)!!
= (2n + 1)!!,
\end{dmath}

which completes the final induction.  That was also the last thing required to complete the proof, so we are done!
} % answer

%\EndArticle
