%
% Copyright � 2015 Peeter Joot.  All Rights Reserved.
% Licenced as described in the file LICENSE under the root directory of this GIT repository.
%
%\newcommand{\authorname}{Peeter Joot}
\newcommand{\email}{peeterjoot@protonmail.com}
\newcommand{\basename}{FIXMEbasenameUndefined}
\newcommand{\dirname}{notes/FIXMEdirnameUndefined/}

%\renewcommand{\basename}{sg}
%\renewcommand{\dirname}{notes/phy1520/}
%%\newcommand{\dateintitle}{}
%%\newcommand{\keywords}{}
%
%\newcommand{\authorname}{Peeter Joot}
\newcommand{\onlineurl}{http://sites.google.com/site/peeterjoot2/math2013/\basename.pdf}
\newcommand{\sourcepath}{\dirname\basename.tex}
\newcommand{\generatetitle}[1]{\chapter{#1}}

\newcommand{\vcsinfo}{%
\section*{}
\noindent{\color{DarkOliveGreen}{\rule{\linewidth}{0.1mm}}}
\paragraph{Document version}
%\paragraph{\color{Maroon}{Document version}}
{
\small
\begin{itemize}
\item Available online at:\\ 
\href{\onlineurl}{\onlineurl}
\item Git Repository: \input{./.revinfo/gitRepo.tex}
\item Source: \sourcepath
\item last commit: \input{./.revinfo/gitCommitString.tex}
\item commit date: \input{./.revinfo/gitCommitDate.tex}
\end{itemize}
}
}

%\PassOptionsToPackage{dvipsnames,svgnames}{xcolor}
\PassOptionsToPackage{square,numbers}{natbib}
\documentclass{scrreprt}

\usepackage[left=2cm,right=2cm]{geometry}
\usepackage[svgnames]{xcolor}
\usepackage{peeters_layout}

\usepackage{natbib}

\usepackage[
colorlinks=true,
bookmarks=false,
pdfauthor={\authorname, \email},
backref 
]{hyperref}

% http://tex.stackexchange.com/questions/75773/how-to-reference-problems-by-the-text-label-in-an-exercise-envioronment
\usepackage[english]{cleveref}
\crefname{Exercise}{exercise}{exercises}
\Crefname{Exercise}{Exercise}{Exercises}

\RequirePackage{titlesec}
\RequirePackage{ifthen}

% http://stackoverflow.com/questions/4932910/date-in-the-tabular-environment
\makeatletter
\let\insertdate\@date
\makeatother

\titleformat{\chapter}[display]
{\bfseries\Large}
{\color{DarkSlateGrey}\filleft \authorname
\ifthenelse{\isundefined{\studentnumber}}{}{\\ \studentnumber}
\ifthenelse{\isundefined{\email}}{}{\\ \email}
\ifthenelse{\isundefined{\dateintitle}}{}{\\ \insertdate}
%\ifthenelse{\isundefined{\coursename}}{}{\\ \coursename} % put in title instead.
}
{4ex}
{\color{DarkOliveGreen}{\titlerule}\color{Maroon}
\vspace{2ex}%
\filright}
[\vspace{2ex}%
\color{DarkOliveGreen}\titlerule
]

\newcommand{\beginArtWithToc}[0]{\begin{document}\tableofcontents}
\newcommand{\beginArtNoToc}[0]{\begin{document}}
\newcommand{\EndNoBibArticle}[0]{\end{document}}
\newcommand{\EndArticle}[0]{\bibliography{Bibliography}\bibliographystyle{plainnat}\end{document}}

% 
%\newcommand{\citep}[1]{\cite{#1}}

\colorSectionsForArticle


%
%\usepackage{peeters_layout_exercise}
%\usepackage{peeters_braket}
%\usepackage{peeters_figures}
%
%\beginArtNoToc
%
%\generatetitle{Cascading Stern-Gerlach}
%\chapter{Cascading Stern-Gerlach}
%\label{chap:sg}

\makeoproblem{Cascading Stern-Gerlach.}{problem:sg:13}{\citep{sakurai2014modern} pr. 1.13}{
\index{Stern-Gerlach}

Three Stern-Gerlach type measurements are performed, the first that prepares the state in a \( \ket{S_z ; + } \) state, the next in a \( \ket{ \BS \cdot \ncap ; + } \) state where \( \ncap = \cos\beta \zcap + \sin\beta \xcap \), and the last performing a \( S_z \) \( \Hbar/2 \) state measurement, as illustrated in \cref{fig:sternGerlach:sternGerlachFig1}.

\imageFigure{../../figures/phy1520/sternGerlachFig1}{Cascaded Stern-Gerlach type measurements.}{fig:sternGerlach:sternGerlachFig1}{0.3}

What is the intensity of the final \( s_z = -\Hbar/2 \) beam?  What is the orientation for the second measuring apparatus to maximize the intensity of this beam?
} % problem

\makeanswer{problem:sg:13}{

The spin operator for the second apparatus is

\begin{dmath}\label{eqn:sg:20}
\BS \cdot \ncap
= \frac{\Hbar}{2} \lr{ \sin\beta \PauliX + \cos\beta \PauliZ }
= \frac{\Hbar}{2}
\begin{bmatrix}
\cos\beta & \sin\beta \\
\sin\beta & -\cos\beta
\end{bmatrix}.
\end{dmath}

The intensity of the final \( \ket{S_z ; -} \) beam is

\begin{dmath}\label{eqn:sg:40}
P
= \Abs{ \braket{-}{\BS \cdot \ncap ; +} \braket{\BS \cdot \ncap ; +}{+} }^2,
\end{dmath}

(i.e. the second apparatus applies a projection operator \( \ket{\BS \cdot \ncap ; +}\bra{\BS \cdot \ncap ; +} \) to the initial \( \ket{+} \) state, and then the \( \ket{-} \) states are selected out of that.

The \( \BS \cdot \ncap \) eigenket is found to be

\begin{dmath}\label{eqn:sg:60}
\ket{\BS \cdot \ncap ; +} =
\begin{bmatrix}
\cos\frac{\beta}{2} \\
\sin\frac{\beta}{2} \\
\end{bmatrix},
\end{dmath}

so

\begin{dmath}\label{eqn:sg:80}
P
= \Abs{
\begin{bmatrix}
0 & 1
\end{bmatrix}
\begin{bmatrix}
\cos\frac{\beta}{2} \\
\sin\frac{\beta}{2} \\
\end{bmatrix}
\begin{bmatrix}
\cos\frac{\beta}{2} &
\sin\frac{\beta}{2} \\
\end{bmatrix}
\begin{bmatrix}
1 \\
0
\end{bmatrix}
}^2
=
\Abs{
\cos\frac{\beta}{2}
\sin\frac{\beta}{2}
}^2
=
\Abs{\inv{2} \sin\beta}^2
=
\inv{4} \sin^2\beta.
\end{dmath}

This is maximized when \( \beta = \pi/2 \), or \( \ncap = \xcap \).  At this angle the state leaving the second apparatus is

\begin{dmath}\label{eqn:sg:100}
\begin{bmatrix}
\cos\frac{\beta}{2} \\
\sin\frac{\beta}{2} \\
\end{bmatrix}
\begin{bmatrix}
\cos\frac{\beta}{2} &
\sin\frac{\beta}{2} \\
\end{bmatrix}
\begin{bmatrix}
1 \\
0
\end{bmatrix}
=
\inv{2}
\begin{bmatrix}
1 \\ 1
\end{bmatrix}
\begin{bmatrix}
1 & 1
\end{bmatrix}
\begin{bmatrix}
1 \\ 0
\end{bmatrix}
=
\inv{2}
\begin{bmatrix}
1 \\ 1
\end{bmatrix}
=\inv{2} \ket{+} + \inv{2}\ket{-},
\end{dmath}

so the state after filtering the \( \ket{-} \) states is \( \inv{2} \ket{-} \) with intensity (probability density) of \( 1/4 \) relative to a unit normalize input \( \ket{+} \) state to the \( \BS \cdot \ncap \) apparatus.

} % answer

%\EndArticle
