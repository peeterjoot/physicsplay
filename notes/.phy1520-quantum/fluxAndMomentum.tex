%
% Copyright � 2015 Peeter Joot.  All Rights Reserved.
% Licenced as described in the file LICENSE under the root directory of this GIT repository.
%
%\newcommand{\authorname}{Peeter Joot}
\newcommand{\email}{peeterjoot@protonmail.com}
\newcommand{\basename}{FIXMEbasenameUndefined}
\newcommand{\dirname}{notes/FIXMEdirnameUndefined/}

%\renewcommand{\basename}{fluxAndMomentum}
%\renewcommand{\dirname}{notes/phy1520/}
%\newcommand{\dateintitle}{}
%\newcommand{\keywords}{}

%\newcommand{\authorname}{Peeter Joot}
\newcommand{\onlineurl}{http://sites.google.com/site/peeterjoot2/math2013/\basename.pdf}
\newcommand{\sourcepath}{\dirname\basename.tex}
\newcommand{\generatetitle}[1]{\chapter{#1}}

\newcommand{\vcsinfo}{%
\section*{}
\noindent{\color{DarkOliveGreen}{\rule{\linewidth}{0.1mm}}}
\paragraph{Document version}
%\paragraph{\color{Maroon}{Document version}}
{
\small
\begin{itemize}
\item Available online at:\\ 
\href{\onlineurl}{\onlineurl}
\item Git Repository: \input{./.revinfo/gitRepo.tex}
\item Source: \sourcepath
\item last commit: \input{./.revinfo/gitCommitString.tex}
\item commit date: \input{./.revinfo/gitCommitDate.tex}
\end{itemize}
}
}

%\PassOptionsToPackage{dvipsnames,svgnames}{xcolor}
\PassOptionsToPackage{square,numbers}{natbib}
\documentclass{scrreprt}

\usepackage[left=2cm,right=2cm]{geometry}
\usepackage[svgnames]{xcolor}
\usepackage{peeters_layout}

\usepackage{natbib}

\usepackage[
colorlinks=true,
bookmarks=false,
pdfauthor={\authorname, \email},
backref 
]{hyperref}

% http://tex.stackexchange.com/questions/75773/how-to-reference-problems-by-the-text-label-in-an-exercise-envioronment
\usepackage[english]{cleveref}
\crefname{Exercise}{exercise}{exercises}
\Crefname{Exercise}{Exercise}{Exercises}

\RequirePackage{titlesec}
\RequirePackage{ifthen}

% http://stackoverflow.com/questions/4932910/date-in-the-tabular-environment
\makeatletter
\let\insertdate\@date
\makeatother

\titleformat{\chapter}[display]
{\bfseries\Large}
{\color{DarkSlateGrey}\filleft \authorname
\ifthenelse{\isundefined{\studentnumber}}{}{\\ \studentnumber}
\ifthenelse{\isundefined{\email}}{}{\\ \email}
\ifthenelse{\isundefined{\dateintitle}}{}{\\ \insertdate}
%\ifthenelse{\isundefined{\coursename}}{}{\\ \coursename} % put in title instead.
}
{4ex}
{\color{DarkOliveGreen}{\titlerule}\color{Maroon}
\vspace{2ex}%
\filright}
[\vspace{2ex}%
\color{DarkOliveGreen}\titlerule
]

\newcommand{\beginArtWithToc}[0]{\begin{document}\tableofcontents}
\newcommand{\beginArtNoToc}[0]{\begin{document}}
\newcommand{\EndNoBibArticle}[0]{\end{document}}
\newcommand{\EndArticle}[0]{\bibliography{Bibliography}\bibliographystyle{plainnat}\end{document}}

% 
%\newcommand{\citep}[1]{\cite{#1}}

\colorSectionsForArticle



%\usepackage{peeters_layout_exercise}
%\usepackage{peeters_braket}
%\usepackage{peeters_figures}
%
%\beginArtNoToc

%\generatetitle{Relation of probability flux to momentum}
%\chapter{Relation of probability flux to momentum}
%\label{chap:fluxAndMomentum}

\makeproblem{Relation of probability flux to momentum.}{problem:fluxAndMomentum:1}{
\index{probability!flux}
\index{momentum!expectation}

Show that the probability flux

\begin{dmath}\label{eqn:fluxAndMomentum:20}
\Bj(\Bx, t) = -\frac{i\Hbar}{2 m} \lr{ \psi^\conj \spacegrad \psi - \psi \spacegrad \psi^\conj },
\end{dmath}

is related to the momentum expectation at a given time by the integral of the flux over all space

\begin{dmath}\label{eqn:fluxAndMomentum:40}
\int d^3 x \Bj(\Bx, t) = \frac{\expectation{\Bp}_t}{m}.
\end{dmath}

} % problem

\makeanswer{problem:fluxAndMomentum:1}{

This can be seen by recasting the integral in bra-ket form.  Let

\begin{dmath}\label{eqn:fluxAndMomentum:60}
\psi(\Bx, t) = \braket{\Bx}{\psi(t)},
\end{dmath}

and note that the momentum portions of the flux can be written as

\begin{dmath}\label{eqn:fluxAndMomentum:80}
-i \Hbar \spacegrad \psi(\Bx, t) = \bra{\Bx} \Bp \ket{\psi(t)}.
\end{dmath}

The current is therefore

\begin{dmath}\label{eqn:fluxAndMomentum:100}
\Bj(\Bx, t)
= \frac{1}{2 m}
\lr{
\psi^\conj \bra{\Bx} \Bp \ket{\psi(t)}
+\psi {\bra{\Bx} \Bp \ket{\psi(t)} }^\conj
}
= \frac{1}{2 m}
\lr{
{\braket{\Bx}{\psi(t)}}^\conj \bra{\Bx} \Bp \ket{\psi(t)}
+ \braket{\Bx}{\psi(t)} {\bra{\Bx} \Bp \ket{\psi(t)} }^\conj
}
= \frac{1}{2 m}
\lr{
\braket{\psi(t)}{\Bx} \bra{\Bx} \Bp \ket{\psi(t)}
+
\bra{\psi(t)} \Bp \ket{\Bx} \braket{\Bx}{\psi(t)}
}.
\end{dmath}

Integrating and noting that the spatial identity is \( 1 = \int d^3 x \ket{\Bx}\bra{\Bx} \), we have

\begin{dmath}\label{eqn:fluxAndMomentum:n}
\int d^3 x \Bj(\Bx, t)
=
\bra{\psi(t)} \Bp \ket{\psi(t)},
\end{dmath}

This is just the expectation of \( \Bp \) with respect to a specific time-instance state, demonstrating the desired relationship.
} % answer

%\EndArticle
