%
% Copyright � 2015 Peeter Joot.  All Rights Reserved.
% Licenced as described in the file LICENSE under the root directory of this GIT repository.
%
\makeoproblem{Jackiw-Rebbi problem.}{gradQuantum:problemSet4:2}{2015 ps4 p2}{
\index{Jackiw-Rebbi problem}

Recall that the energy of a relativistic particle is \( E(p) = \sqrt{p^2 c^2 + m^2 c^4 } \), which is independent of the sign of \( m \).
Thus \( m > 0 \) and \( m < 0 \) lead to the same dispersion relation.
Set aside for now, the physical meaning of \( m < 0 \).
Assume \( V(x) = 0 \) but let us assume the mass \( m \) is a function of position \( m(x) \).
This leads to

\begin{dmath}\label{eqn:gradQuantumProblemSet4Problem2:20}
H =
\begin{bmatrix}
c \hatp & m(x) c^2 \\
m(x) c^2 & -c \hatp
\end{bmatrix}
.
\end{dmath}

Let \( m(x) \) be such that \( m(-x) = -m(x) \), i.e., an odd function of position which changes sign at \( x = 0 \).

\makesubproblem{}{gradQuantum:problemSet4:2a}
Show that the operator \( \hatcalP_{\textrm{Dirac}} = \sigma_y \hatcalP \) commutes with the Hamiltonian, where

\begin{dmath}\label{eqn:gradQuantumProblemSet4Problem2:40}
\sigma_y
=
\PauliY
\end{dmath}

is the y-Pauli matrix, and \( \hatcalP \) is the parity operator which sends \( x \rightarrow -x\).
\makesubproblem{}{gradQuantum:problemSet4:2b}
Consider the wavefunction

\begin{dmath}\label{eqn:gradQuantumProblemSet4Problem2:60}
\Phi(x) =
\begin{bmatrix}
f(x) \\
- i f(x)
\end{bmatrix},
\end{dmath}

where \( f(-x) = f(x) \) is an even function.  Show \( \Phi(x) \) is an eigenstate of \( \hatcalP_{\textrm{Dirac}} \) with eigenvalue \( -1 \).

\makesubproblem{}{gradQuantum:problemSet4:2c}
Next, assuming \( m(x > 0) = m_0 \) and \( m(x < 0) = -m_0 \) , where \( m_0 > 0 \), find \( f(x) \) such that \( \Phi(x) \) is an eigenstate of H with zero energy.

\makesubproblem{}{gradQuantum:problemSet4:2d}
Normalize the wavefunction \( \Phi(x) \).

} % makeproblem

\makeanswer{gradQuantum:problemSet4:2}{
\withproblemsetsParagraph{
\makeSubAnswer{}{gradQuantum:problemSet4:2a}

Let \( \pi \) represent a parity operator that acts on a scalar (operator), so the Dirac parity operator \( \hatcalP_{\textrm{Dirac}} \) takes the form

\begin{dmath}\label{eqn:gradQuantumProblemSet4Problem2:80}
\hatcalP_{\textrm{Dirac}} =
\begin{bmatrix}
0 & -i \pi \\
i \pi & 0
\end{bmatrix}.
\end{dmath}

The commutator with the Hamiltonian is

\begin{dmath}\label{eqn:gradQuantumProblemSet4Problem2:100}
\antisymmetric{H}{\hatcalP_{\textrm{Dirac}}}
= H \hatcalP_{\textrm{Dirac}} - \hatcalP_{\textrm{Dirac}} H
=
\begin{bmatrix}
c \hatp & m(x) c^2 \\
m(x) c^2 & -c \hatp
\end{bmatrix}
\begin{bmatrix}
0 & -i \pi \\
i \pi & 0
\end{bmatrix}
-
\begin{bmatrix}
0 & -i \pi \\
i \pi & 0
\end{bmatrix}
\begin{bmatrix}
c \hatp & m(x) c^2 \\
m(x) c^2 & -c \hatp
\end{bmatrix}
=
\begin{bmatrix}
i c^2 m(x) \pi & -i c \hatp \pi \\
-i c \hatp \pi & - i c^2 m(x) \pi
\end{bmatrix}
-
\begin{bmatrix}
-i c^2 \pi m(x) & i c \pi \hatp \\
i c \pi \hatp & i c^2 \pi m(x)
\end{bmatrix}
=
\begin{bmatrix}
i c^2 \symmetric{m(x)}{ \pi } & -i c \symmetric{\hatp}{\pi} \\
-i c \symmetric{\hatp}{\pi} & - i c^2 \symmetric{m(x)}{\pi}
\end{bmatrix}.
\end{dmath}

Now consider the matrix element of this commutator with respect to a position basis wavefunction \( \bra{x} \antisymmetric{H}{\hatcalP_{\textrm{Dirac}}} \ket{\Psi} \).  To compute this we must first understand the behaviour of the anticommutators of scalar operators \( m, \hatp \) with \( \pi \).  That is

\begin{dmath}\label{eqn:gradQuantumProblemSet4Problem2:120}
\bra{x} \symmetric{m(x)}{\pi} \ket{\psi}
=
\symmetric{m(x)}{\pi} \psi(x)
=
m(x) \pi \psi(x) + \pi ( m(x) \psi(x) )
=
m(x) \psi(-x) + m(-x) \psi(-x)
=
m(x) \psi(-x) - m(x) \psi(-x)
=
0,
\end{dmath}

and

\begin{dmath}\label{eqn:gradQuantumProblemSet4Problem2:140}
\bra{x} \symmetric{\hatp}{\pi} \ket{\psi}
=
\symmetric{-i \Hbar \PD{x}{} }{\pi} \psi(x)
=
-i \Hbar \PD{x}{} \pi \psi(x) + \pi ( -i \Hbar \PD{x}{} \psi(x) )
=
-i \Hbar \PD{x}{} \psi(-x) + (-i \Hbar) \PD{(-x)}{} \psi(-x)
=
i \Hbar \PD{(-x)}{} \psi(-x) - i \Hbar \PD{(-x)}{} \psi(-x)
=
0.
\end{dmath}

This shows that \( \bra{x} \antisymmetric{H}{\hatcalP_{\textrm{Dirac}}} \ket{\Psi} = 0 \).  Since this is true for any \( \ket{\Psi} \), we've shown that the Hamiltonian commutes with the Dirac parity operator

\boxedEquation{eqn:gradQuantumProblemSet4Problem2:160}{
H \hatcalP_{\textrm{Dirac}} = \hatcalP_{\textrm{Dirac}} H.
}

\makeSubAnswer{}{gradQuantum:problemSet4:2b}

This follows with direct substitution

\begin{dmath}\label{eqn:gradQuantumProblemSet4Problem2:180}
\hatcalP_{\textrm{Dirac}} \Phi(x)
=
\begin{bmatrix}
0 & -i \pi \\
i \pi & 0
\end{bmatrix}
\begin{bmatrix}
f(x) \\
-i f(x)
\end{bmatrix}
=
\begin{bmatrix}
- \pi f(x) \\
i \pi f(x)
\end{bmatrix}
=
\begin{bmatrix}
- f(-x) \\
i f(-x)
\end{bmatrix}
=
-
\begin{bmatrix}
f(x) \\
-i f(x)
\end{bmatrix}.
\end{dmath}

\makeSubAnswer{}{gradQuantum:problemSet4:2c}

The eigenvalue equation has the form

\begin{dmath}\label{eqn:gradQuantumProblemSet4Problem2:380}
\lambda \Phi
= H \Phi
=
\begin{bmatrix}
c \hatp & m(x) c^2 \\
m(x) c^2 & -c \hatp
\end{bmatrix}
\begin{bmatrix}
f \\
-i f
\end{bmatrix}
=
\begin{bmatrix}
c \hatp f -i m(x) c^2 f \\
m(x) c^2 f + i c \hatp f,
\end{bmatrix}
\end{dmath}

or
\begin{dmath}\label{eqn:gradQuantumProblemSet4Problem2:200}
\begin{aligned}
c (-i \Hbar) \PD{x}{f} -i m(x) c^2 f &= \lambda f \\
i c (-i \Hbar) \PD{x}{f} + m(x) c^2 f &= -i \lambda f \\
\end{aligned}.
\end{dmath}

This special selection of \( \Phi \) leads to two identical equations, which should simplify things nicely.  For the zero energy eigenstate, we must solve

\begin{dmath}\label{eqn:gradQuantumProblemSet4Problem2:220}
\PD{x}{f} = - \inv{c \Hbar} m(x) c^2 f,
\end{dmath}

which integrates to

\begin{dmath}\label{eqn:gradQuantumProblemSet4Problem2:240}
\ln f
= \ln f(0) - \int_0^x \inv{c \Hbar} m(x) c^2 dx,
\end{dmath}

or

\begin{dmath}\label{eqn:gradQuantumProblemSet4Problem2:260}
f(x) = f(0) \exp\lr{ - \int_0^x \frac{c}{\Hbar} m(x) dx }.
\end{dmath}

For \( x > 0 \) this is

\begin{dmath}\label{eqn:gradQuantumProblemSet4Problem2:280}
f(x)
= f(0) \exp\lr{ - \int_0^x \frac{c}{\Hbar} m_0 dx }
= f(0) \exp\lr{ - \frac{m_0 c x}{\Hbar}},
\end{dmath}

and for \( x < 0 \) this is
\begin{dmath}\label{eqn:gradQuantumProblemSet4Problem2:300}
f(x)
= f(0) \exp\lr{ \int_0^x \frac{c}{\Hbar} m_0 dx }
= f(0) \exp\lr{ \frac{m_0 c x}{\Hbar}}.
\end{dmath}

These can be combined as

\begin{dmath}\label{eqn:gradQuantumProblemSet4Problem2:320}
f(x) = f(0) \exp\lr{ - \frac{m_0 c \Abs{x}}{\Hbar}}.
\end{dmath}

\makeSubAnswer{}{gradQuantum:problemSet4:2d}

The normalization is given by

\begin{dmath}\label{eqn:gradQuantumProblemSet4Problem2:340}
1
=
\int_{-\infty}^\infty \Phi^\dagger \Phi dx
=
\int_{-\infty}^\infty 2 \Abs{f}^2 dx
=
2 f^2(0) \int_{-\infty}^\infty e^{- 2 m_0 c \Abs{x}/\Hbar} dx
=
4 f^2(0) \int_{0}^\infty e^{- 2 m_0 c x/\Hbar} dx
=
4 f^2(0) \evalrange{\frac{e^{- 2 m_0 c x/\Hbar}}{ - 2 m_0 c/\Hbar}}{0}{\infty}
=
\frac{2 \Hbar}{ m_0 c} f^2(0),
\end{dmath}

or

\begin{dmath}\label{eqn:gradQuantumProblemSet4Problem2:360}
f(0) = \sqrt{ \frac{ m_0 c}{2 \Hbar} }.
\end{dmath}

The fully normalized wave function is therefore

\boxedEquation{eqn:gradQuantumProblemSet4Problem2:400}{
\Phi(x) = \sqrt{ \frac{ m_0 c}{2 \Hbar} } e^{ - m_0 c \Abs{x}/\Hbar}
\begin{bmatrix}
1 \\
-i
\end{bmatrix}.
}
}
}
