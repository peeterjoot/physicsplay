%
% Copyright � 2015 Peeter Joot.  All Rights Reserved.
% Licenced as described in the file LICENSE under the root directory of this GIT repository.
%
\makeoproblem{Angular momentum.}{gradQuantum:problemSet5:4}{2015 ps5 p4}{
\index{angular momentum}
\index{time reversal}
\makesubproblem{}{gradQuantum:problemSet5:4a}

What is the time-reversed version of \( \calD(R) \ket{ j, m}\)?
\makesubproblem{}{gradQuantum:problemSet5:4b}
Prove that time-reversal implies \( \Theta \ket{j , m} = (-1)^{m} \ket{j, -m}\).

} % makeproblem

\makeanswer{gradQuantum:problemSet5:4}{
\withproblemsetsParagraph{
\makeSubAnswer{}{gradQuantum:problemSet5:4a}

Assuming the rotation is around the normal \( \ncap = (n_1, n_2, n_3) \) by \( \phi \) radians, the action of the time reversal operator on the rotated state can be expanded in series

\begin{dmath}\label{eqn:gradQuantumProblemSet5Problem4:20}
\Theta \calD(R) \ket{j, m}
=
\Theta e^{-i \BJ \cdot \ncap \phi/\Hbar} \ket{j, m}
=
\sum_{k = 0}^\infty \inv{k!} \lr{\frac{-i \phi}{\Hbar}}^k \lr{ \BJ \cdot \ncap }^k \ket{j, m}
=
\sum_{k = 0}^\infty \inv{k!} \lr{\frac{-i \phi}{\Hbar}}^k
\Theta \lr{ \BJ \cdot \ncap }^k
\ket{j, m}.
\end{dmath}

For any \( J_i \), we know that \( \Theta J_i = -J_i \Theta \), so

\begin{dmath}\label{eqn:gradQuantumProblemSet5Problem4:40}
\Theta (J_m n_m)^k
=
-J_r n_r \Theta (J_m n_m)^{k-1}
=
(-J_r n_r)^2 \Theta (J_m n_m)^{k-2}
=
\cdots
=
(-J_r n_r)^k \Theta.
\end{dmath}

This effectively flips the sign of the rotation angle, so we have

\begin{dmath}\label{eqn:gradQuantumProblemSet5Problem4:60}
\Theta \calD(R) \ket{j, m}
=
\calD(R^{-1}) \Theta \ket{j, m}.
\end{dmath}

\makeSubAnswer{}{gradQuantum:problemSet5:4b}

%Using spherical harmonics \( Y_l^m \) \citep{sakurai2014modern} shows that for orbital angular momentum we have \( \Theta \ket{l m} = (-1)^m \ket{l, -m} \).
For general total angular momentum states \( \BJ + \BL + \BS \), we can utilize the postulated properties of the time reversal operator.  In particular

\begin{dmath}\label{eqn:gradQuantumProblemSet5Problem4:80}
\Theta J_i = - J_i \Theta.
\end{dmath}

Considering each of \( J_z \) and \( J_{\pm} \) in turn, we first have

\begin{dmath}\label{eqn:gradQuantumProblemSet5Problem4:100}
\Theta J_z \ket{j, m}
=
\Hbar m \Theta \ket{j, m}
=
- J_z \Theta \ket{j, m},
\end{dmath}

or

\begin{dmath}\label{eqn:gradQuantumProblemSet5Problem4:120}
J_z \lr{ \Theta \ket{j, m} } = - m \Hbar \lr{ \Theta \ket{j, m} },
\end{dmath}

This means that \( \Theta \ket{j, m} \) is an eigenket of \( J_z \) with eigenvalue \( - m \Hbar \), so that we must have

\begin{dmath}\label{eqn:gradQuantumProblemSet5Problem4:140}
\Theta \ket{j, m} = c_m \ket{j, -m} .
\end{dmath}

The interplay of the ladder and time-reversal operators can be used to determine the factor \( c_m \).  Note that

\begin{dmath}\label{eqn:gradQuantumProblemSet5Problem4:160}
\Theta J_{+} \Theta^{-1}
=
\Theta \lr{ J_x + i J_y } \Theta^{-1}
=
-J_x - i \Theta J_y \Theta^{-1}
=
-J_x + i J_y
=
-\lr{ J_x - i J_y }
=
-J_{-},
\end{dmath}

and

\begin{dmath}\label{eqn:gradQuantumProblemSet5Problem4:180}
\Theta J_{-} \Theta^{-1}
=
\Theta \lr{ J_x - i J_y } \Theta^{-1}
=
-J_x + i \Theta J_y \Theta^{-1}
=
-J_x - i J_y
=
-\lr{ J_x + i J_y }
=
-J_{+},
\end{dmath}

or
\begin{equation}\label{eqn:gradQuantumProblemSet5Problem4:200}
\begin{aligned}
J_{-} \Theta &= -\Theta J_{+} \\
J_{+} \Theta &= -\Theta J_{-}.
\end{aligned}
\end{equation}

Acting with the raising operator on a time reversed state we have

\begin{dmath}\label{eqn:gradQuantumProblemSet5Problem4:220}
J_{+} \Theta \ket{j, m}
=
J_{+} c_m \ket{j, -m}
=
\Hbar \sqrt{ (j-(-m))(j+(-m)+1) } c_{m} \ket{j, -m + 1}.
\end{dmath}

But we must also have

\begin{dmath}\label{eqn:gradQuantumProblemSet5Problem4:240}
J_{+} \Theta \ket{j, m}
=
-\Theta J_{-} \ket{j, m}
=
-\Theta \Hbar \sqrt{(j+m)(j-m+1)} \ket{j, m - 1}
=
- \Hbar \sqrt{(j+m)(j-m+1)} c_{m-1} \ket{j, -m + 1}.
\end{dmath}

This provides a recurrence relation for the undetermined \( c_m \) factors

\begin{dmath}\label{eqn:gradQuantumProblemSet5Problem4:260}
c_m = - c_{m-1} = (-1)^k c_{m-k}
\end{dmath}

In particular for \( m -k = -j \), this is

\begin{dmath}\label{eqn:gradQuantumProblemSet5Problem4:280}
c_m = (-1)^{m + j} c_{-j}.
\end{dmath}

Acting twice with the time reversal operator

\begin{dmath}\label{eqn:gradQuantumProblemSet5Problem4:300}
\Theta^2 \ket{j, m}
=
\Theta (-1)^{m + j} c_{-j} \ket{j, -m}
=
(-1)^{ m -m + 2 j} c_{-j}^\conj c_{-j} \ket{j, m}
=
 (-1)^{2 j} \Abs{c_{-j}}^2 \ket{j, m}.
\end{dmath}

The \( c_{-j} \) factor can be absorbed into the definition of the time reversal operator, so that, up to a phase factor, we have

%\begin{dmath}\label{eqn:gradQuantumProblemSet5Problem4:320}
\boxedEquation{eqn:gradQuantumProblemSet5Problem4:340}{
\Theta \ket{j, m} = (-1)^{m} \ket{j, m},
}
%\end{dmath}

and
%\begin{dmath}\label{eqn:gradQuantumProblemSet5Problem4:360}
\boxedEquation{eqn:gradQuantumProblemSet5Problem4:380}{
\Theta^2 = (-1)^{2 j}.
}
%\end{dmath}
}
}
