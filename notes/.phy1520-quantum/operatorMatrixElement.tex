%
% Copyright � 2015 Peeter Joot.  All Rights Reserved.
% Licenced as described in the file LICENSE under the root directory of this GIT repository.
%
%\newcommand{\authorname}{Peeter Joot}
\newcommand{\email}{peeterjoot@protonmail.com}
\newcommand{\basename}{FIXMEbasenameUndefined}
\newcommand{\dirname}{notes/FIXMEdirnameUndefined/}

%\renewcommand{\basename}{operatorMatrixElement}
%\renewcommand{\dirname}{notes/phy1520/}
%%\newcommand{\dateintitle}{}
%%\newcommand{\keywords}{}
%
%\newcommand{\authorname}{Peeter Joot}
\newcommand{\onlineurl}{http://sites.google.com/site/peeterjoot2/math2013/\basename.pdf}
\newcommand{\sourcepath}{\dirname\basename.tex}
\newcommand{\generatetitle}[1]{\chapter{#1}}

\newcommand{\vcsinfo}{%
\section*{}
\noindent{\color{DarkOliveGreen}{\rule{\linewidth}{0.1mm}}}
\paragraph{Document version}
%\paragraph{\color{Maroon}{Document version}}
{
\small
\begin{itemize}
\item Available online at:\\ 
\href{\onlineurl}{\onlineurl}
\item Git Repository: \input{./.revinfo/gitRepo.tex}
\item Source: \sourcepath
\item last commit: \input{./.revinfo/gitCommitString.tex}
\item commit date: \input{./.revinfo/gitCommitDate.tex}
\end{itemize}
}
}

%\PassOptionsToPackage{dvipsnames,svgnames}{xcolor}
\PassOptionsToPackage{square,numbers}{natbib}
\documentclass{scrreprt}

\usepackage[left=2cm,right=2cm]{geometry}
\usepackage[svgnames]{xcolor}
\usepackage{peeters_layout}

\usepackage{natbib}

\usepackage[
colorlinks=true,
bookmarks=false,
pdfauthor={\authorname, \email},
backref 
]{hyperref}

% http://tex.stackexchange.com/questions/75773/how-to-reference-problems-by-the-text-label-in-an-exercise-envioronment
\usepackage[english]{cleveref}
\crefname{Exercise}{exercise}{exercises}
\Crefname{Exercise}{Exercise}{Exercises}

\RequirePackage{titlesec}
\RequirePackage{ifthen}

% http://stackoverflow.com/questions/4932910/date-in-the-tabular-environment
\makeatletter
\let\insertdate\@date
\makeatother

\titleformat{\chapter}[display]
{\bfseries\Large}
{\color{DarkSlateGrey}\filleft \authorname
\ifthenelse{\isundefined{\studentnumber}}{}{\\ \studentnumber}
\ifthenelse{\isundefined{\email}}{}{\\ \email}
\ifthenelse{\isundefined{\dateintitle}}{}{\\ \insertdate}
%\ifthenelse{\isundefined{\coursename}}{}{\\ \coursename} % put in title instead.
}
{4ex}
{\color{DarkOliveGreen}{\titlerule}\color{Maroon}
\vspace{2ex}%
\filright}
[\vspace{2ex}%
\color{DarkOliveGreen}\titlerule
]

\newcommand{\beginArtWithToc}[0]{\begin{document}\tableofcontents}
\newcommand{\beginArtNoToc}[0]{\begin{document}}
\newcommand{\EndNoBibArticle}[0]{\end{document}}
\newcommand{\EndArticle}[0]{\bibliography{Bibliography}\bibliographystyle{plainnat}\end{document}}

% 
%\newcommand{\citep}[1]{\cite{#1}}

\colorSectionsForArticle


%
%\usepackage{peeters_layout_exercise}
%\usepackage{peeters_braket}
%\usepackage{peeters_figures}
%
%\beginArtNoToc
%
%\generatetitle{Operator matrix element}
%\chapter{Operator matrix element}
%\label{chap:operatorMatrixElement}

\paragraph{Weird dreams}
\index{matrix element}

I woke up today having a dream still in my head from the night, but it was a strange one.  I was expanding out the Dirac notation representation of an operator in matrix form, but the symbols in the kets were elaborate pictures of Disney princesses that I was drawing with forestry scenery in the background, including little bears.  At the point that I woke up from the dream, I noticed that I'd gotten the proportion of the bears wrong in one of the pictures, and they looked like they were ready to eat one of the princess characters.

\paragraph{Guts}

As a side effect of this weird dream I actually started thinking about matrix element representation of operators.

When forming the matrix element of an operator using Dirac notation the elements are of the form \( \bra{\textrm{row}} A \ket{\textrm{column}} \).  I've gotten that mixed up a couple of times, so I thought it would be helpful to write this out explicitly for a \( 2 \times 2 \) operator representation for clarity.

To start, consider a change of basis for a single matrix element from basis \( \setlr{\ket{q}, \ket{r} } \), to basis \( \setlr{\ket{a}, \ket{b} } \)

\begin{dmath}\label{eqn:operatorMatrixElement:20}
\begin{aligned}
\bra{q} A \ket{r}
&=
\braket{q}{a} \bra{a} A \ket{r}
+
\braket{q}{b} \bra{b} A \ket{r} \\
&=
\braket{q}{a} \bra{a} A \ket{a}\braket{a}{r}
+ \braket{q}{a} \bra{a} A \ket{b}\braket{b}{r} \\
&+ \braket{q}{b} \bra{b} A \ket{a}\braket{a}{r}
+ \braket{q}{b} \bra{b} A \ket{b}\braket{b}{r} \\
&=
\braket{q}{a}
\begin{bmatrix}
\bra{a} A \ket{a} & \bra{a} A \ket{b}
\end{bmatrix}
\begin{bmatrix}
\braket{a}{r} \\
\braket{b}{r}
\end{bmatrix}
+
\braket{q}{b}
\begin{bmatrix}
\bra{b} A \ket{a} & \bra{b} A \ket{b}
\end{bmatrix}
\begin{bmatrix}
\braket{a}{r} \\
\braket{b}{r}
\end{bmatrix} \\
&=
\begin{bmatrix}
\braket{q}{a} &
\braket{q}{b}
\end{bmatrix}
\begin{bmatrix}
\bra{a} A \ket{a} & \bra{a} A \ket{b} \\
\bra{b} A \ket{a} & \bra{b} A \ket{b}
\end{bmatrix}
\begin{bmatrix}
\braket{a}{r} \\
\braket{b}{r}
\end{bmatrix}.
\end{aligned}
\end{dmath}

Suppose the matrix representation of \( \ket{q}, \ket{r} \) are respectively

\begin{dmath}\label{eqn:operatorMatrixElement:40}
\begin{aligned}
\ket{q} &\sim
\begin{bmatrix}
\braket{a}{q} \\
\braket{b}{q} \\
\end{bmatrix} \\
\ket{r} &\sim
\begin{bmatrix}
\braket{a}{r} \\
\braket{b}{r} \\
\end{bmatrix} \\
\end{aligned},
\end{dmath}

then

\begin{dmath}\label{eqn:operatorMatrixElement:60}
\bra{q} \sim
{\begin{bmatrix}
\braket{a}{q} \\
\braket{b}{q} \\
\end{bmatrix}}^\dagger
=
\begin{bmatrix}
\braket{q}{a} &
\braket{q}{b}
\end{bmatrix}.
\end{dmath}

The matrix element is then

\begin{dmath}\label{eqn:operatorMatrixElement:80}
\bra{q} A \ket{r}
\sim
\bra{q}
\begin{bmatrix}
\bra{a} A \ket{a} & \bra{a} A \ket{b} \\
\bra{b} A \ket{a} & \bra{b} A \ket{b}
\end{bmatrix}
\ket{r},
\end{dmath}

and the corresponding matrix representation of the operator is

\begin{dmath}\label{eqn:operatorMatrixElement:100}
A \sim
\begin{bmatrix}
\bra{a} A \ket{a} & \bra{a} A \ket{b} \\
\bra{b} A \ket{a} & \bra{b} A \ket{b}
\end{bmatrix}.
\end{dmath}

%This particular matrix representation is largely convention, because we could have easily have picked an alternate representation of the kets.

%\EndNoBibArticle
