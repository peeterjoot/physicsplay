%
% Copyright � 2015 Peeter Joot.  All Rights Reserved.
% Licenced as described in the file LICENSE under the root directory of this GIT repository.
%
%{
%\newcommand{\authorname}{Peeter Joot}
\newcommand{\email}{peeterjoot@protonmail.com}
\newcommand{\basename}{FIXMEbasenameUndefined}
\newcommand{\dirname}{notes/FIXMEdirnameUndefined/}

%\renewcommand{\basename}{symmetryOperatorCommutators}
%\renewcommand{\dirname}{notes/phy1520/}
%%\newcommand{\dateintitle}{}
%%\newcommand{\keywords}{}
%
%\newcommand{\authorname}{Peeter Joot}
\newcommand{\onlineurl}{http://sites.google.com/site/peeterjoot2/math2013/\basename.pdf}
\newcommand{\sourcepath}{\dirname\basename.tex}
\newcommand{\generatetitle}[1]{\chapter{#1}}

\newcommand{\vcsinfo}{%
\section*{}
\noindent{\color{DarkOliveGreen}{\rule{\linewidth}{0.1mm}}}
\paragraph{Document version}
%\paragraph{\color{Maroon}{Document version}}
{
\small
\begin{itemize}
\item Available online at:\\ 
\href{\onlineurl}{\onlineurl}
\item Git Repository: \input{./.revinfo/gitRepo.tex}
\item Source: \sourcepath
\item last commit: \input{./.revinfo/gitCommitString.tex}
\item commit date: \input{./.revinfo/gitCommitDate.tex}
\end{itemize}
}
}

%\PassOptionsToPackage{dvipsnames,svgnames}{xcolor}
\PassOptionsToPackage{square,numbers}{natbib}
\documentclass{scrreprt}

\usepackage[left=2cm,right=2cm]{geometry}
\usepackage[svgnames]{xcolor}
\usepackage{peeters_layout}

\usepackage{natbib}

\usepackage[
colorlinks=true,
bookmarks=false,
pdfauthor={\authorname, \email},
backref 
]{hyperref}

% http://tex.stackexchange.com/questions/75773/how-to-reference-problems-by-the-text-label-in-an-exercise-envioronment
\usepackage[english]{cleveref}
\crefname{Exercise}{exercise}{exercises}
\Crefname{Exercise}{Exercise}{Exercises}

\RequirePackage{titlesec}
\RequirePackage{ifthen}

% http://stackoverflow.com/questions/4932910/date-in-the-tabular-environment
\makeatletter
\let\insertdate\@date
\makeatother

\titleformat{\chapter}[display]
{\bfseries\Large}
{\color{DarkSlateGrey}\filleft \authorname
\ifthenelse{\isundefined{\studentnumber}}{}{\\ \studentnumber}
\ifthenelse{\isundefined{\email}}{}{\\ \email}
\ifthenelse{\isundefined{\dateintitle}}{}{\\ \insertdate}
%\ifthenelse{\isundefined{\coursename}}{}{\\ \coursename} % put in title instead.
}
{4ex}
{\color{DarkOliveGreen}{\titlerule}\color{Maroon}
\vspace{2ex}%
\filright}
[\vspace{2ex}%
\color{DarkOliveGreen}\titlerule
]

\newcommand{\beginArtWithToc}[0]{\begin{document}\tableofcontents}
\newcommand{\beginArtNoToc}[0]{\begin{document}}
\newcommand{\EndNoBibArticle}[0]{\end{document}}
\newcommand{\EndArticle}[0]{\bibliography{Bibliography}\bibliographystyle{plainnat}\end{document}}

% 
%\newcommand{\citep}[1]{\cite{#1}}

\colorSectionsForArticle


%
%\usepackage{peeters_layout_exercise}
%\usepackage{peeters_braket}
%\usepackage{peeters_figures}
%\usepackage{enumerate}
%\usepackage{macros_cal}
%
%\beginArtNoToc
%
%\generatetitle{Commutators for some symmetry operators}
%\chapter{Commutators for some symmetry operators}
%\label{chap:symmetryOperatorCommutators}

\makeoproblem{Commutators for some symmetry operators.}{problem:symmetryOperatorCommutators:1}{\citep{sakurai2014modern} pr. 4.2}{

If \( \calT_\Bd \), \( \calD(\ncap, \phi) \), and \( \pi \) denote the translation, rotation, and parity operators respectively.  Which of the following commute and why

\makesubproblem{}{problem:symmetryOperatorCommutators:1:a}
\( \calT_\Bd \) and \( \calT_{\Bd'} \), translations in different directions.
\makesubproblem{}{problem:symmetryOperatorCommutators:1:b}
\( \calD(\ncap, \phi) \) and \( \calD(\ncap', \phi') \), rotations in different directions.
\makesubproblem{}{problem:symmetryOperatorCommutators:1:c}
\( \calT_\Bd \) and \( \pi \).
\makesubproblem{}{problem:symmetryOperatorCommutators:1:d}
\( \calD(\ncap,\phi)\) and \( \pi \).

} % problem

\makeanswer{problem:symmetryOperatorCommutators:1}{

\makeSubAnswer{}{problem:symmetryOperatorCommutators:1:a}

Consider
\begin{dmath}\label{eqn:symmetryOperatorCommutators:20}
\calT_\Bd \calT_{\Bd'} \ket{\Bx}
=
\calT_\Bd \ket{\Bx + \Bd'}
=
\ket{\Bx + \Bd' + \Bd},
\end{dmath}

and the reverse application of the translation operators
\begin{dmath}\label{eqn:symmetryOperatorCommutators:40}
\calT_{\Bd'} \calT_{\Bd} \ket{\Bx}
=
\calT_{\Bd'} \ket{\Bx + \Bd}
=
\ket{\Bx + \Bd + \Bd'}
=
\ket{\Bx + \Bd' + \Bd}.
\end{dmath}

so we see that

\begin{dmath}\label{eqn:symmetryOperatorCommutators:60}
\antisymmetric{\calT_\Bd}{\calT_{\Bd'}} \ket{\Bx} = 0,
\end{dmath}

for any position state \( \ket{\Bx} \), and therefore in general they commute.

\makeSubAnswer{}{problem:symmetryOperatorCommutators:1:b}

That rotations do not commute when they are in different directions (like any two orthogonal directions) need not be belaboured.

\makeSubAnswer{}{problem:symmetryOperatorCommutators:1:c}
We have
\begin{dmath}\label{eqn:symmetryOperatorCommutators:80}
\calT_\Bd \pi \ket{\Bx}
=
\calT_\Bd \ket{-\Bx}
=
\ket{-\Bx + \Bd},
\end{dmath}

yet
\begin{dmath}\label{eqn:symmetryOperatorCommutators:100}
\pi \calT_\Bd \ket{\Bx}
=
\pi \ket{\Bx + \Bd}
=
\ket{-\Bx - \Bd}
\ne
\ket{-\Bx + \Bd}.
\end{dmath}

so, in general \( \antisymmetric{\calT_\Bd}{\pi} \ne 0 \).

\makeSubAnswer{}{problem:symmetryOperatorCommutators:1:d}

We have

\begin{dmath}\label{eqn:symmetryOperatorCommutators:120}
\pi \calD(\ncap, \phi) \ket{\Bx}
=
\pi \calD(\ncap, \phi) \pi^\dagger \pi \ket{\Bx}
=
\pi \calD(\ncap, \phi) \pi^\dagger \pi \ket{\Bx}
=
\pi \lr{ \sum_{k=0}^\infty \frac{(-i \BJ \cdot \ncap)^k}{k!} } \pi^\dagger \pi \ket{\Bx}
=
\sum_{k=0}^\infty \frac{(-i (\pi \BJ \pi^\dagger) \cdot (\pi \ncap \pi^\dagger) )^k}{k!} \pi \ket{\Bx}
=
\sum_{k=0}^\infty \frac{(-i \BJ \cdot \ncap)^k}{k!} \pi \ket{\Bx}
=
\calD(\ncap, \phi) \pi \ket{\Bx},
\end{dmath}

so \( \antisymmetric{\calD(\ncap, \phi)}{\pi} \ket{\Bx} = 0 \), for any position state \( \ket{\Bx} \), and therefore these operators commute in general.
} % answer

%}
%\EndArticle
