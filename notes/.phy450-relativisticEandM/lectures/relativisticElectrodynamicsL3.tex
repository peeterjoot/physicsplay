%
% Copyright � 2012 Peeter Joot.  All Rights Reserved.
% Licenced as described in the file LICENSE under the root directory of this GIT repository.
%

%\chapter{Spacetime, events, worldlines, spacetime intervals, and invariance}
\index{spacetime}
\index{events}
\index{worldlines}
\index{spacetime interval}
\index{invariance}
\label{chap:relativisticElectrodynamicsL3}
%\blogpage{http://sites.google.com/site/peeterjoot/math2011/relativisticElectrodynamicsL3.pdf}
%\date{Jan 13, 2011}

\paragraph{Reading}

Still covering chapter 1 material from the text \citep{landau1980classical}, and
\popcite{RelEM12-26.pdf}{lecture notes RelEM12-26.pdf}.
%: geometry of spacetime, lightlike, spacelike, timelike intervals, and worldlines (18-22); proper time (23-24); invariance of finite intervals (25-26).

\section{Geometry of spacetime: lightlike, spacelike, timelike intervals}
\index{lightlike}
\index{spacelike}
\index{timelike}

Last time we introduced the (squared) interval

\begin{equation}\label{eqn:relativisticElectrodynamicsL3:10}
s_{12}^2 = c^2 dt^2 - d\Br^2.
\end{equation}

This spacetime interval is of great importance to relativity, and is as important as the spatial distance \(\Abs{\Br_2 -\Br_1}\) in Newtonian physics.  This distance determines the Euclidean geometry of space.

Similarly, the interval \eqnref{eqn:relativisticElectrodynamicsL3:10} determines the ``distance'' in space time.

Symmetries are the guiding principles of physics, and this quantity we will see to be related to spacetime symmetries.  Last time we argued that the constancy of the speed of light in all frames implies that if \(s_{12}^2 = 0\) in one frame, then \({s_{12}'}^2 = 0\).

We were considering infinitesimal \(1,2\) separation with \(ds = F(V) ds'\) where \(V\) is the relative speed of the two frames.  Relating the two incremental intervals we have a function \(F\) and its inverse

\begin{equation}\label{eqn:relativisticElectrodynamicsL3:150}
\begin{aligned}
ds' &= \tilde{F(V)} ds = \tilde{F}(V) F(V) ds,
\end{aligned}
\end{equation}

But we can also argue that
\begin{equation}\label{eqn:relativisticElectrodynamicsL3:170}
\begin{aligned}
\tilde{F} &= F \qquad \mbox{by \(O' \sim O\)},
\end{aligned}
\end{equation}

and thus that

\begin{equation}\label{eqn:relativisticElectrodynamicsL3:190}
\begin{aligned}
F^2 &= 1,
\end{aligned}
\end{equation}

or
\begin{equation}\label{eqn:relativisticElectrodynamicsL3:210}
\begin{aligned}
F &= \pm 1.
\end{aligned}
\end{equation}

Since we wish this to hold for \(V =0\), we require the positive root, and can conclude that \(F = 1\).

Note that \(ds\) (or \(s_{12}\)) requires a sign convention, since it is \(s_{12}^2 = c^2(t_2 - t_1)^2 - (\Br_2 - \Br_1)^2\) that is the object which (we will argue) is invariant.

This is similar to the Euclidean case where it is the quantity \((\Br_2 - \Br_1)^2\) is invariant, and our convention is to always  pick the positive sign.

Possible conventions for \(s_{12}\) are

\begin{equation}\label{eqn:relativisticElectrodynamicsL3:50}
s_{12} = \sqrt{ c^2(t_2 - t_1)^2 - (\Br_2 - \Br_1)^2 },
\end{equation}

if \(s_{12}^2 > 0\), and when \(s_{12}^2 < 0\), the alternate convention is

\begin{equation}\label{eqn:relativisticElectrodynamicsL3:55}
s_{12} = i \sqrt{\Abs{s_{12}}^2}.
\end{equation}

Later we will argue that \(ds = ds'\) implies \(s_{12} = s_{12}'\) for any finite interval.

\section{Relativity principle in mathematical formulation}
\index{relativity principle}

The Relativity principle (in mathematical formulation): the spacetime interval \(s_{12}, \forall 1,2\) (spacetime points) is the same in all frames.

In other words, the transformations \((t, \Br) \rightarrow (t', \Br')\) have to leave \(s_{12}^2\) invariant for all \(1\) and \(2\).

These transformations, that is to say these coordinate transformations

\begin{equation}\label{eqn:relativisticElectrodynamicsL3:230}
\begin{aligned}
(t, \Br) &\rightarrow (t', \Br') \\
O &\rightarrow O'
\end{aligned}
\end{equation}

leave the laws of nature invariant.

We will see later how such invariance, like the spatial invariance in Newtonian physics, defines the dynamics of spacetime.  We will also answer the question about what are these transformations that leave the interval invariant.  In the Newtonian case those transformations were rotations, and we will be looking for similar transformations.  The negative sign in the spacetime interval  will complicate things a bit, but not actually too much.

Next week: we will find the ``Lorentz transformation''.

\section{Geometry of spacetime}
\index{spacetime}

We now want to study a bit of the geometry of spacetime implied by \(s_{12}^2 = c^2(t_2 - t_1)^2 - (\Br_2 - \Br_1)^2\).  Consider two spacetime points 1,2, where \((t_1, \Br_1), (t_2, \Br_2)\) are points in some frame.

PICTURE: two points plotted on the x-axis, with time \(t_1=0\), and \(t = t_2\)

The points are

\begin{enumerate}
\item \((0, \Bzero)\)
\item \((t, x, 0, 0)\)
\end{enumerate}

The interval is

\begin{equation}\label{eqn:relativisticElectrodynamicsL3:100}
s_{12}^2 = c^2 t^2 - x^2
\end{equation}

PICTURE: ``flat'' light cone.  2d cross-section of space time surface \(c^2 t^2 = x^2 + y^2\).
\index{light cone}

PICTURE: conic light cone.  3d (2 space + 1 time) cross-section of space time surface \(c^2 t^2 = x^2 + y^2\).  One diagonal for the trajectory \(ct = -x\), and another for \(ct = x\).  The bottom section is the past light cone, since light that is absorbed at the origin must have been emitted at some point in the past.  Similarly, light emitted from the origin, takes trajectories on the future light cone.

Observe that on the light cone, \(s_{12}^2 = 0\).  The intervals \(s_{12}^2 = 0\) separates any given set of spacetime points into ``lightlike'', ``spacelike'', and ``timelike'' regions.

For events (or spacetime points) separated by a timelike interval, there always exists a frame such that the occur at same point in space (since \(s_{12}^2 = c^2 t^2 - \Br^2 > 0\) (region II) it is consistent to imagine that there exists a frame where \(\Br' = 0\) and \(s_{12}^2 = c^2 t^2 > 0\).  This is very much like we can always find a rotation in Euclidean space that orients two points so that they lie along the \(x\) (or any other arbitrary) axis.

We have not yet proven this, but will see it shortly.  What we will see is that we can never make these two events have the same time (\(t' \ne 0\)).  This is because if we make \(t' = 0\) we will get a negative interval in some frame.

For points in spacetime separated by spacelike intervals, one can always choose a frame such that they occur at same \(t\).  (i.e. for us \(t' = 0\)).  Since \(s_{12}^2 = c^2 t^2 - \Br^2 < 0\), \(s_{12}^2 = -\Br^2 < 0\).

Similar to light rays that move along the light cone, particles that move at speeds less than the speed of light propagate along \textunderline{worldlines} within region II (in the interior of the light cone).  At at arbitrary point in the worldline of a particle draw a 45 degree cone.  Tangent to world line should lie inside the figure lightcone of that space time point.

\section{Proper time}
\index{proper time}

PICTURE4: velocity at \((t, \Bx) = v\) (say).  Consider an inertial frame with speed \(v\), centered at the momentary position of the particle.  Call this the primed frame.  In this frame \(ds^2 = c^2 {dt'}^2\) (particle is at rest in this frame).

In the original frame \(ds^2 = c^2 dt^2 - d\Bx^2\).  Since these are equal we have

\begin{equation}\label{eqn:relativisticElectrodynamicsL3:110}
c^2 dt^2 - d\Bx^2 = c^2 {dt'}^2
\end{equation}

Dividing by \(c^2\) we have

\begin{equation}\label{eqn:relativisticElectrodynamicsL3:115}
{dt'}^2 = dt^2 - \inv{c^2} d\Bx^2 .
\end{equation}

Here \(dt^2\) is the (squared) time elapsed in the frame where it is moving.  The time elapsed in the rest frame of the particle, we call the ``proper time'', and we have \(dt' < dt\) because \(1 - \Bv^2/c^2 < 1\).  This is described as

More exactly, we write

\begin{equation}\label{eqn:relativisticElectrodynamicsL3:120}
d\tau^2 = \frac{ds^2}{dt^2} = dt^2 \left( 1 - \inv{c^2} \left(\frac{d\Bx}{dt}\right)^2 \right)
\end{equation}

In general, for noninfinitesimal \(dt\), to find the proper time one has to integrate

\begin{equation}\label{eqn:relativisticElectrodynamicsL3:130}
\tau_{ab} = \inv{c} \int_a^b ds
\end{equation}

\paragraph{Plan for next class:}  Talk about causality.  Derive the Lorentz transformation.
