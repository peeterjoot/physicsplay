%
% Copyright © 2012 Peeter Joot.  All Rights Reserved.
% Licenced as described in the file LICENSE under the root directory of this GIT repository.
%

\makeproblem{Collision of photon and electron}{pr:relativisticElectrodynamicsExamReflection:2}{

Determine the velocity of an electron, initially at rest, after absorbing a photon.

} % makeproblem

\makeanswer{pr:relativisticElectrodynamicsExamReflection:2}{

I made a dumb error on the exam on this one.  I setup the four momentum conservation statement, but then did not multiply out the cross terms properly.  This led me to incorrectly assume that I had to try doing this the hard way (something akin to what I did on the midterm).  Simon later told us in the tutorial the simple way, and that is all we needed here too.  Here is the setup.

An electron at rest initially has four momentum

\begin{equation}\label{eqn:relativisticElectrodynamicsExamReflection:310}
(m c, 0)
\end{equation}

where the incoming photon has four momentum

\begin{equation}\label{eqn:relativisticElectrodynamicsExamReflection:330}
\left(\Hbar \frac{\omega}{c}, \Hbar \Bk \right)
\end{equation}

After the collision our electron has some velocity so its four momentum becomes (say)

\begin{equation}\label{eqn:relativisticElectrodynamicsExamReflection:350}
\gamma (m c, m \Bv),
\end{equation}

and our new photon, going off on an angle \(\theta\) relative to \(\Bk\) has four momentum

\begin{equation}\label{eqn:relativisticElectrodynamicsExamReflection:370}
\left(\Hbar \frac{\omega'}{c}, \Hbar \Bk' \right)
\end{equation}

Our conservation relationship is thus
\begin{equation}\label{eqn:relativisticElectrodynamicsExamReflection:390}
(m c, 0) + \left(\Hbar \frac{\omega}{c}, \Hbar \Bk \right)
=
\gamma (m c, m \Bv)
+
\left(\Hbar \frac{\omega'}{c}, \Hbar \Bk' \right)
\end{equation}

I squared both sides, but dropped my cross terms, which was just plain wrong, and costly for both time and effort on the exam.  What I should have done was just

\begin{equation}\label{eqn:relativisticElectrodynamicsExamReflection:410}
\gamma (m c, m \Bv) =
(m c, 0) + \left(\Hbar \frac{\omega}{c}, \Hbar \Bk \right)
-\left(\Hbar \frac{\omega'}{c}, \Hbar \Bk' \right),
\end{equation}

and then square this (really making contractions of the form \(p_i p^i\)).  That gives (and this time keeping my cross terms)

\begin{equation}\label{eqn:relativisticElectrodynamicsExamReflection:1350}
\begin{aligned}
(\gamma (m c, m \Bv) )^2
&= \gamma^2 m^2 (c^2 - \Bv^2) \\
&= m^2 c^2 \\
&=
m^2 c^2 + 0 + 0
+ 2 (m c, 0)
\cdot \left(\Hbar \frac{\omega}{c}, \Hbar \Bk \right) \\
&\qquad - 2 (m c, 0) \cdot \left(\Hbar \frac{\omega'}{c}, \Hbar \Bk' \right)
- 2
\left(\Hbar \frac{\omega}{c}, \Hbar \Bk \right)
\cdot \left(\Hbar \frac{\omega'}{c}, \Hbar \Bk' \right) \\
&=
m^2 c^2 + 2 m c \Hbar \frac{\omega}{c} - 2 m c \Hbar \frac{\omega'}{c}
- 2\Hbar^2 \left(
\frac{\omega}{c} \frac{\omega'}{c}
-
\Bk \cdot \Bk'
\right) \\
&=
m^2 c^2 + 2 m c \Hbar \frac{\omega}{c} - 2 m c \Hbar \frac{\omega'}{c}
- 2\Hbar^2
\frac{\omega}{c} \frac{\omega'}{c} (1 - \cos\theta)
\end{aligned}
\end{equation}

Rearranging a bit we have

\begin{equation}\label{eqn:relativisticElectrodynamicsExamReflection:430}
\omega' \left( m + \frac{\Hbar \omega}{c^2} ( 1 - \cos\theta ) \right) = m \omega,
\end{equation}

or
\begin{equation}\label{eqn:relativisticElectrodynamicsExamReflection:450}
\omega' = \frac{\omega}{
1 + \frac{\Hbar \omega}{m c^2} ( 1 - \cos\theta )
}
\end{equation}
} % makeanswer

