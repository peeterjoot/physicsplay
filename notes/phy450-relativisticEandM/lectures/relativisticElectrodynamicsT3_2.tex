%
% Copyright © 2012 Peeter Joot.  All Rights Reserved.
% Licenced as described in the file LICENSE under the root directory of this GIT repository.
%

%\chapter{Relativistic motion in constant uniform electric or magnetic fields}
%\label{chap:relativisticElectrodynamicsT3}
%\blogpage{http://sites.google.com/site/peeterjoot/math2011/relativisticElectrodynamicsT3.pdf}
%\date{Feb 3, 2011}


\makeproblem{Motion in an constant uniform Magnetic field}{pr:relativisticElectrodynamicsT3_2:2}{

Calculate a particle motion in a uniform magnetic field.

} % makeproblem

\makeanswer{pr:relativisticElectrodynamicsT3_2:2}{


\paragraph{Work by the magnetic field}
\index{magnetic field!work}

Note that the magnetic field does no work

\begin{equation}\label{eqn:relativisticElectrodynamicsT3:500}
\BF = \frac{e}{c} \Bv \cross \BB
\end{equation}

\begin{equation}\label{eqn:relativisticElectrodynamicsT3:1030}
\begin{aligned}
dW &=
\BF \cdot d\Bl \\
&=
\frac{e}{c} (\Bv \cross \BB) \cdot d\Bl \\
&=
\frac{e}{c} (\Bv \cross \BB) \cdot \Bv dt \\
&= 0
\end{aligned}
\end{equation}

Because \(\Bv\) and \(\Bv \cross \BB\) are necessarily perpendicular we are reminded that the magnetic field does no work (even in this relativistic sense).

\paragraph{Initial energy of the particle}

Because no work is done, the particle's energy is only the initial time value

\begin{equation}\label{eqn:relativisticElectrodynamicsT3:520}
\calE = .... + e A^0
\end{equation}

Simon asked if we would calculated this (i.e. the Hamiltonian in class).  We would calculated the conservation for time invariance, the Hamiltonian (and called it \(E\)).  We would also calculated the Hamiltonian for the free particle

\begin{equation}\label{eqn:relativisticElectrodynamicsT3:530}
\calE^2 = \Bp^2 c^2 + (m c^2)^2.
\end{equation}

We had not done this calculation for the Lorentz force Lagrangian, so lets do it now.  Recall that this Lagrangian was

\begin{equation}\label{eqn:relativisticElectrodynamicsT3:531}
\LL =
- m c^2 \sqrt{1 - \frac{\Bv^2}{c^2}} - e \phi + \frac{e}{c} \Bv \cdot \BA,
\end{equation}

with generalized momentum of
\begin{equation}\label{eqn:relativisticElectrodynamicsT3:532}
\PD{\Bv}{\LL} = \frac{m \Bv}{\sqrt{1 - \frac{\Bv^2}{c^2}}} + \frac{e}{c} \BA.
\end{equation}

Our Hamiltonian is thus

\begin{equation}\label{eqn:relativisticElectrodynamicsT3:1050}
\begin{aligned}
\calE
=
\frac{m \Bv^2}{\sqrt{1 - \frac{\Bv^2}{c^2}}} + \frac{e}{c} \BA \cdot \Bv
+ m c^2 \sqrt{1 - \frac{\Bv^2}{c^2}} + e \phi - \frac{e}{c} \Bv \cdot \BA,
\end{aligned}
\end{equation}

which gives us

\begin{equation}\label{eqn:relativisticElectrodynamicsT3:1070}
\begin{aligned}
\calE = e \phi + \frac{m c^2}{\sqrt{1 - \frac{\Bv^2}{c^2}}}
\end{aligned}
\end{equation}

So we see that our ``energy'', defined as a quantity that is conserved, as a result of the symmetry of time translation invariance, has a component due to the electric field (but not the vector potential field \(\BA\)), plus the free particle ``energy''.

Is this right?  With \(\BA\) and \(\phi\) being functions of space and time, perhaps we need to be more careful with this argument.  Perhaps this actually only applies to a statics case where \(\BA\) and \(\phi\) are constant.

Since it was hinted to us that the energy component of the Lorentz force equation was proportional to \(F^{0j} u_j\), and we can peek ahead to find that \(F^{ij} = \partial^i A^j - \partial^j A^i\), let us compare to that

\begin{equation}\label{eqn:relativisticElectrodynamicsT3:1090}
\begin{aligned}
e F^{0 j} u_j
&=
e (\partial^0 A^j - \partial^j A^0) u_j \\
&=
e (\partial^0 A^\alpha - \partial^\alpha A^0) u_\alpha \\
&=
e \left( \inv{c} \PD{t}{A^\alpha} + \partial_\alpha A^0 \right) \inv{c} \frac{dx_\alpha}{d\tau} \\
&=
-e \left( \inv{c} \PD{t}{A^\alpha} + \PD{x^\alpha}{\phi} \right) \inv{c} \frac{dx^\alpha}{dt} \gamma,
\end{aligned}
\end{equation}

which is

\begin{equation}\label{eqn:relativisticElectrodynamicsT3:910}
e F^{0 j} u_j = e \left(\BE \cdot \frac{\Bv}{c}\right) \gamma.
\end{equation}

So if we have

\begin{equation}\label{eqn:relativisticElectrodynamicsT3:920}
\frac{d\Bp}{dt} = e \left( \BE + \frac{\Bv}{c} \cross \BB \right)
\end{equation}

I had guess that we have

\begin{equation}\label{eqn:relativisticElectrodynamicsT3:940}
\frac{d(\calE/c)}{d\tau} \sim e F^{0 j} u_j,
\end{equation}

which is, using \eqnref{eqn:relativisticElectrodynamicsT3:910}

\begin{equation}\label{eqn:relativisticElectrodynamicsT3:950}
\frac{d(\calE/c)}{dt} \sim e \left(\BE \cdot \frac{\Bv}{c}\right)
\end{equation}

Can the left hand side be integrated to yield \(e \phi\)?  Yes, but only in the statics case when \(\PDi{t}{\BA} = 0\), and \(\phi(\Bx,t) = \phi(\Bx)\) for which we have

\begin{equation}\label{eqn:relativisticElectrodynamicsT3:1110}
\begin{aligned}
\calE
&\sim e \int_a^b \BE \cdot \Bv dt \\
&= -e \int_a^b (\spacegrad \phi) \cdot \frac{d \Bx}{dt} dt \\
&= -e \int_a^b (\spacegrad \phi) \cdot d \Bx \\
&= -e \int_a^b \PD{x^\alpha}{\phi} d x^\alpha \\
&= -e (\phi_b - \phi_a)
\end{aligned}
\end{equation}

FIXME: My suspicion is that the result \eqnref{eqn:relativisticElectrodynamicsT3:950}, is generally true, but that we have dropped terms from the Hamiltonian calculation that need to be retained when \(\phi\) and \(\BA\) are functions of time.

\paragraph{Expressing the field and the force equation}

We will align our field with the \(z\) axis, and write

\begin{equation}\label{eqn:relativisticElectrodynamicsT3:560}
\BB = H \zcap,
\end{equation}

or, in components

\begin{equation}\label{eqn:relativisticElectrodynamicsT3:580}
\delta_{\alpha 3} H = H_\alpha.
\end{equation}

Because the energy is only due to the initial value, we write

\begin{equation}\label{eqn:relativisticElectrodynamicsT3:540}
\calE(t) = \calE_0
\end{equation}

\begin{equation}\label{eqn:relativisticElectrodynamicsT3:600}
\Bp = \calE \frac{\Bv}{c^2} = \calE_0 \frac{\Bv}{c^2}
\end{equation}

implies

\begin{equation}\label{eqn:relativisticElectrodynamicsT3:620}
\Bv = \Bp \frac{c^2}{\calE_0}
\end{equation}

\begin{equation}\label{eqn:relativisticElectrodynamicsT3:640}
\dot{\Bv} = \dot{\Bp} \frac{c^2}{\calE_0}
\end{equation}

\begin{equation}\label{eqn:relativisticElectrodynamicsT3:660}
\vdot_\alpha = \frac{e c}{\calE_0} \epsilon_{\alpha \beta \gamma} v_\beta H_\gamma
\end{equation}

write

\begin{equation}\label{eqn:relativisticElectrodynamicsT3:680}
\omega = \frac{e c H}{\calE_0}
\end{equation}

Evaluating the delta
\begin{equation}\label{eqn:relativisticElectrodynamicsT3:700}
\vdot_\alpha = \omega \epsilon_{\alpha \beta 3} v_\beta
\end{equation}

\begin{equation}\label{eqn:relativisticElectrodynamicsT3:720}
\begin{aligned}
\vdot_1 &= \omega \epsilon_{1 \beta 3} v_\beta = \omega v_2 \\
\vdot_2 &= \omega \epsilon_{2 \beta 3} v_\beta = - \omega v_1 \\
\vdot_3 &= \omega \epsilon_{3 \beta 3} v_\beta = 0
\end{aligned}
\end{equation}

Looks like circular motion, so it is natural to use complex variables.  With

\begin{equation}\label{eqn:relativisticElectrodynamicsT3:740}
z = v_1 + i v_2
\end{equation}

Using this we have

\begin{equation}\label{eqn:relativisticElectrodynamicsT3:1130}
\begin{aligned}
\frac{d}{dt} ( v_1 + i v_2 )
&=
\omega v_2 - i \omega v_1  \\
&= -i \omega ( v_1 + i v_2 ).
\end{aligned}
\end{equation}

which comes out nicely

\begin{equation}\label{eqn:relativisticElectrodynamicsT3:751}
\frac{dz}{dt} = -i \omega z
\end{equation}

for
\begin{equation}\label{eqn:relativisticElectrodynamicsT3:780}
z = V_0 e^{-i \omega z t + i \alpha}
\end{equation}

Real and imaginary parts

\begin{equation}\label{eqn:relativisticElectrodynamicsT3:800}
\begin{aligned}
v_1(t) &= V_0 \cos( \omega z t + \alpha) \\
v_2(t) &= -V_0 \sin( \omega z t + \alpha)
\end{aligned}
\end{equation}

Integrating
\begin{equation}\label{eqn:relativisticElectrodynamicsT3:820}
\begin{aligned}
x_1(t) &= x_1(0) + V_0 \sin( \omega z t + \alpha) \\
x_2(t) &= x_2(0) + V_0 \cos( \omega z t + \alpha)
\end{aligned}
\end{equation}

Which is a helix.
PICTURE: ...
} % makeanswer
