%
% Copyright � 2012 Peeter Joot.  All Rights Reserved.
% Licenced as described in the file LICENSE under the root directory of this GIT repository.
%

%
%
%\documentclass[]{eliblog}

\usepackage{amsmath}
\usepackage{mathpazo}

%
% shorthand for bold symbols, convenient for vectors and matrices
%
\newcommand{\Ba}[0]{\mathbf{a}}
\newcommand{\Bb}[0]{\mathbf{b}}
\newcommand{\Bc}[0]{\mathbf{c}}
\newcommand{\Bd}[0]{\mathbf{d}}
\newcommand{\Be}[0]{\mathbf{e}}
\newcommand{\Bf}[0]{\mathbf{f}}
\newcommand{\Bg}[0]{\mathbf{g}}
\newcommand{\Bh}[0]{\mathbf{h}}
\newcommand{\Bi}[0]{\mathbf{i}}
\newcommand{\Bj}[0]{\mathbf{j}}
\newcommand{\Bk}[0]{\mathbf{k}}
\newcommand{\Bl}[0]{\mathbf{l}}
\newcommand{\Bm}[0]{\mathbf{m}}
\newcommand{\Bn}[0]{\mathbf{n}}
\newcommand{\Bo}[0]{\mathbf{o}}
\newcommand{\Bp}[0]{\mathbf{p}}
\newcommand{\Bq}[0]{\mathbf{q}}
\newcommand{\Br}[0]{\mathbf{r}}
\newcommand{\Bs}[0]{\mathbf{s}}
\newcommand{\Bt}[0]{\mathbf{t}}
\newcommand{\Bu}[0]{\mathbf{u}}
\newcommand{\Bv}[0]{\mathbf{v}}
\newcommand{\Bw}[0]{\mathbf{w}}
\newcommand{\Bx}[0]{\mathbf{x}}
\newcommand{\By}[0]{\mathbf{y}}
\newcommand{\Bz}[0]{\mathbf{z}}
\newcommand{\BA}[0]{\mathbf{A}}
\newcommand{\BB}[0]{\mathbf{B}}
\newcommand{\BC}[0]{\mathbf{C}}
\newcommand{\BD}[0]{\mathbf{D}}
\newcommand{\BE}[0]{\mathbf{E}}
\newcommand{\BF}[0]{\mathbf{F}}
\newcommand{\BG}[0]{\mathbf{G}}
\newcommand{\BH}[0]{\mathbf{H}}
\newcommand{\BI}[0]{\mathbf{I}}
\newcommand{\BJ}[0]{\mathbf{J}}
\newcommand{\BK}[0]{\mathbf{K}}
\newcommand{\BL}[0]{\mathbf{L}}
\newcommand{\BM}[0]{\mathbf{M}}
\newcommand{\BN}[0]{\mathbf{N}}
\newcommand{\BO}[0]{\mathbf{O}}
\newcommand{\BP}[0]{\mathbf{P}}
\newcommand{\BQ}[0]{\mathbf{Q}}
\newcommand{\BR}[0]{\mathbf{R}}
\newcommand{\BS}[0]{\mathbf{S}}
\newcommand{\BT}[0]{\mathbf{T}}
\newcommand{\BU}[0]{\mathbf{U}}
\newcommand{\BV}[0]{\mathbf{V}}
\newcommand{\BW}[0]{\mathbf{W}}
\newcommand{\BX}[0]{\mathbf{X}}
\newcommand{\BY}[0]{\mathbf{Y}}
\newcommand{\BZ}[0]{\mathbf{Z}}

\newcommand{\Bzero}[0]{\mathbf{0}}
\newcommand{\Btheta}[0]{\boldsymbol{\theta}}
\newcommand{\Btau}[0]{\boldsymbol{\tau}}
\newcommand{\Bomega}[0]{\boldsymbol{\omega}}

%
% shorthand for unit vectors
%
\newcommand{\acap}[0]{\hat{\Ba}}
\newcommand{\bcap}[0]{\hat{\Bb}}
\newcommand{\ccap}[0]{\hat{\Bc}}
\newcommand{\dcap}[0]{\hat{\Bd}}
\newcommand{\ecap}[0]{\hat{\Be}}
\newcommand{\fcap}[0]{\hat{\Bf}}
\newcommand{\gcap}[0]{\hat{\Bg}}
\newcommand{\hcap}[0]{\hat{\Bh}}
\newcommand{\icap}[0]{\hat{\Bi}}
\newcommand{\jcap}[0]{\hat{\Bj}}
\newcommand{\kcap}[0]{\hat{\Bk}}
\newcommand{\lcap}[0]{\hat{\Bl}}
\newcommand{\mcap}[0]{\hat{\Bm}}
\newcommand{\ncap}[0]{\hat{\Bn}}
\newcommand{\ocap}[0]{\hat{\Bo}}
\newcommand{\pcap}[0]{\hat{\Bp}}
\newcommand{\qcap}[0]{\hat{\Bq}}
\newcommand{\rcap}[0]{\hat{\Br}}
\newcommand{\scap}[0]{\hat{\Bs}}
\newcommand{\tcap}[0]{\hat{\Bt}}
\newcommand{\ucap}[0]{\hat{\Bu}}
\newcommand{\vcap}[0]{\hat{\Bv}}
\newcommand{\wcap}[0]{\hat{\Bw}}
\newcommand{\xcap}[0]{\hat{\Bx}}
\newcommand{\ycap}[0]{\hat{\By}}
\newcommand{\zcap}[0]{\hat{\Bz}}
\newcommand{\thetacap}[0]{\hat{\Btheta}}

%
% to write R^n and C^n in a distinguishable fashion.  Perhaps change this
% to the double lined characters upon figuring out how to do so.
%
\newcommand{\C}[1]{$\mathbb{C}^{#1}$}
\newcommand{\R}[1]{$\mathbb{R}^{#1}$}

%
% various generally useful helpers
%

% derivative of #1 wrt. #2:
\newcommand{\D}[2] {\frac {d#2} {d#1}}

\newcommand{\inv}[1]{\frac{1}{#1}}
\newcommand{\cross}[0]{\times}

\newcommand{\abs}[1]{\lvert{#1}\rvert}
\newcommand{\norm}[1]{\lVert{#1}\rVert}
\newcommand{\innerprod}[2]{\langle{#1}, {#2}\rangle}
\newcommand{\dotprod}[2]{{#1} \cdot {#2}}
\newcommand{\bdotprod}[2]{\left({#1} \cdot {#2}\right)}
\newcommand{\crossprod}[2]{{#1} \cross {#2}}
\newcommand{\tripleprod}[3]{\dotprod{\left(\crossprod{#1}{#2}\right)}{#3}}

\DeclareMathOperator{\Proj}{Proj}
\DeclareMathOperator{\Span}{span}
\DeclareMathOperator{\Sgn}{sgn}
\DeclareMathOperator{\Area}{Area}
\DeclareMathOperator{\Volume}{Volume}

%
% A few miscellaneous things specific to this document
%
\newcommand{\crossop}[1]{\crossprod{#1}{}}

% R2 vector.
\newcommand{\VectorTwo}[2]{
\begin{bmatrix}
 {#1} \\
 {#2}
\end{bmatrix}
}

\newcommand{\VectorN}[1]{
\begin{bmatrix}
{#1}_1 \\
{#1}_2 \\
\vdots \\
{#1}_N \\
\end{bmatrix}
}

\newcommand{\DETuvij}[4]{
\begin{vmatrix}
 {#1}_{#3} & {#1}_{#4} \\
 {#2}_{#3} & {#2}_{#4}
\end{vmatrix}
}

\newcommand{\DETuvwijk}[6]{
\begin{vmatrix}
 {#1}_{#4} & {#1}_{#5} & {#1}_{#6} \\
 {#2}_{#4} & {#2}_{#5} & {#2}_{#6} \\
 {#3}_{#4} & {#3}_{#5} & {#3}_{#6}
\end{vmatrix}
}

\newcommand{\DETuvwxijkl}[8]{
\begin{vmatrix}
 {#1}_{#5} & {#1}_{#6} & {#1}_{#7} & {#1}_{#8} \\
 {#2}_{#5} & {#2}_{#6} & {#2}_{#7} & {#2}_{#8} \\
 {#3}_{#5} & {#3}_{#6} & {#3}_{#7} & {#3}_{#8} \\
 {#4}_{#5} & {#4}_{#6} & {#4}_{#7} & {#4}_{#8} \\
\end{vmatrix}
}

%\newcommand{\DETuvwxyijklm}[10]{
%\begin{vmatrix}
% {#1}_{#6} & {#1}_{#7} & {#1}_{#8} & {#1}_{#9} & {#1}_{#10} \\
% {#2}_{#6} & {#2}_{#7} & {#2}_{#8} & {#2}_{#9} & {#2}_{#10} \\
% {#3}_{#6} & {#3}_{#7} & {#3}_{#8} & {#3}_{#9} & {#3}_{#10} \\
% {#4}_{#6} & {#4}_{#7} & {#4}_{#8} & {#4}_{#9} & {#4}_{#10} \\
% {#5}_{#6} & {#5}_{#7} & {#5}_{#8} & {#5}_{#9} & {#5}_{#10}
%\end{vmatrix}
%}

% R3 vector.
\newcommand{\VectorThree}[3]{
\begin{bmatrix}
 {#1} \\
 {#2} \\
 {#3}
\end{bmatrix}
}



\author{Peeter Joot}
\email{peeter.joot@gmail.com}


\chapter{Wave equation form of Maxwell's equations}\label{chap:emVacWave}
\index{wave equation}
\index{Maxwell's equation}
%\blogpage{http://sites.google.com/site/peeterjoot/math2009/emVacWave.pdf}
%\date{ June 21, 2009.  \(RCSfile: emVacWave.tex,v \) Last \(Revision: 1.5 \) \(Date: 2009/07/11 05:49:02 \) }

\beginArtWithToc

\section{Motivation}

In \citep{jackson1975cew}, on plane waves, he writes "we find easily..." to show that the wave equation for each of the components of \(\BE\), and \(\BB\) in the absence of current and charge satisfy the wave equation.  Do this calculation.

\section{Vacuum case}

Avoiding the non-vacuum medium temporarily, Maxwell's vacuum equations (in SI units) are

\begin{equation}\label{eqn:emVacWave:divE}
\begin{aligned}
\spacegrad \cdot \BE = 0
\end{aligned}
\end{equation}
\begin{equation}\label{eqn:emVacWave:divB}
\begin{aligned}
\spacegrad \cdot \BB = 0
\end{aligned}
\end{equation}
\begin{equation}\label{eqn:emVacWave:curlB}
\begin{aligned}
\spacegrad \cross \BB = \inv{c^2} \frac{\partial \BE}{\partial t}
\end{aligned}
\end{equation}
\begin{equation}\label{eqn:emVacWave:curlE}
\begin{aligned}
\spacegrad \cross \BE = -\frac{\partial \BB}{\partial t}
\end{aligned}
\end{equation}

The last two curl equations can be decoupled by once more calculating the curl.  Illustrating by example

\begin{equation}\label{eqn:emVacWave:curlCurlE}
\begin{aligned}
\spacegrad \cross (\spacegrad \cross \BE) = -\frac{\partial }{\partial t} \spacegrad \cross \BB = -\inv{c^2} \frac{\partial^2 \BE}{\partial t^2}
\end{aligned}
\end{equation}

Digging out vector identities and utilizing the zero divergence we have

\begin{equation}\label{eqn:emVacWave:identForcurlCurlE}
\begin{aligned}
\spacegrad \cross (\spacegrad \cross \BE) = \spacegrad (\spacegrad \cdot \BE) - \spacegrad^2 \BE = -\spacegrad^2 \BE
\end{aligned}
\end{equation}

Putting \eqnref{eqn:emVacWave:curlCurlE}, and \eqnref{eqn:emVacWave:identForcurlCurlE} together provides a wave equation for the electric field vector

\begin{equation}\label{eqn:emVacWave:waveE}
\begin{aligned}
\inv{c^2} \frac{\partial^2 \BE}{\partial t^2} - \spacegrad^2 \BE = 0
\end{aligned}
\end{equation}

Operating with curl on the remaining Maxwell equation similarly produces a wave equation for the magnetic field vector

\begin{equation}\label{eqn:emVacWave:waveB}
\begin{aligned}
\inv{c^2} \frac{\partial^2 \BB}{\partial t^2} - \spacegrad^2 \BB = 0
\end{aligned}
\end{equation}

This is really six wave equations, one for each of the field coordinates.

\section{With Geometric Algebra}

Arriving at \eqnref{eqn:emVacWave:waveE}, and \eqnref{eqn:emVacWave:waveB} is much easier using the GA formalism of \citep{doran2003gap}.

Pre or post multiplication of the gradient with the observer frame time basis unit vector \(\gamma_0\) has a conjugate like
action

\begin{equation}\label{eqn:emVacWave:20}
\begin{aligned}
\grad \gamma_0
&=
\gamma^0 \gamma_0 \partial_0 + \gamma^k \gamma_0 \partial_k \\
&=
\partial_0 - \spacegrad \\
\end{aligned}
\end{equation}

(where as usual our spatial basis is \(\sigma_k = \gamma_k \gamma_0\)).

Similarly
\begin{equation}\label{eqn:emVacWave:40}
\begin{aligned}
\gamma_0 \grad
&=
\partial_0 + \spacegrad \\
\end{aligned}
\end{equation}

For the vacuum Maxwell's equation is just
\begin{equation}\label{eqn:emVacWave:60}
\begin{aligned}
\grad F = \grad (\BE + I c \BB) = 0
\end{aligned}
\end{equation}

With nothing more than an algebraic operation we have

\begin{equation}\label{eqn:emVacWave:80}
\begin{aligned}
0
&= \grad \gamma_0 \gamma_0 \grad F \\
&=
( \partial_0 - \spacegrad ) ( \partial_0 + \spacegrad ) (\BE + I c \BB) \\
&=
\left( \inv{c^2} \frac{\partial^2}{\partial t^2} - \spacegrad^2 \right) (\BE + I c \BB) \\
\end{aligned}
\end{equation}

This equality is true independently for each of the components of \(\BE\) and \(\BB\), so we have as before

These wave equations are still subject to the constraints of the original Maxwell equations.

\begin{equation}\label{eqn:emVacWave:100}
\begin{aligned}
0 &= \gamma_0 \grad F \\
&= (\partial_0 + \spacegrad) (\BE + I c \BB) \\
&=
  \spacegrad \cdot \BE
+ (\partial_0 \BE - c \spacegrad \cross \BB)
+ I ( c \partial_0 \BB + \spacegrad \cross \BE )
+ I c \spacegrad \cdot \BB
\\
\end{aligned}
\end{equation}

\section{Tensor approach?}

In both the traditional vector and the GA form one can derive the wave equation relations of \eqnref{eqn:emVacWave:waveE}, \eqnref{eqn:emVacWave:waveB}.  One can obviously summarize these in tensor form as

\begin{equation}\label{eqn:emVacWave:waveFaraday}
\begin{aligned}
\partial_\mu\partial^\mu F^{\alpha\beta} = 0
\end{aligned}
\end{equation}

working backwards from the vector or GA result.  In this notation, the coupling constraint would be that the field variables \(F^{\alpha\beta}\) are subject to the Maxwell divergence equation (name?)

\begin{equation}\label{eqn:emVacWave:divergenceFaraday}
\begin{aligned}
\partial_\mu F^{\mu\nu} = 0
\end{aligned}
\end{equation}

and also the dual tensor relation

\begin{equation}\label{eqn:emVacWave:dualFaraday}
\begin{aligned}
\epsilon^{\sigma\mu\alpha\beta} \partial_\mu F_{\alpha\beta} = 0
\end{aligned}
\end{equation}

I cannot seem to figure out how to derive \eqnref{eqn:emVacWave:waveFaraday} starting from these tensor relations?

This probably has something to do with the fact that we require both the divergence and the dual relations \eqnref{eqn:emVacWave:divergenceFaraday}, \eqnref{eqn:emVacWave:dualFaraday} expressed together to do this.

\section{Electromagnetic waves in media}

Jackson lists the Macroscopic Maxwell equations in (6.70) as

\begin{equation}\label{eqn:emVacWave:120}
\begin{aligned}
\spacegrad \cdot \BB &= 0 \\
\spacegrad \cdot \BD &= 4 \pi \rho \\
\spacegrad \cross \BE + \inv{c}\PD{t}{\BB} &= 0 \\
\spacegrad \cross \BH - \inv{c}\PD{t}{\BD} &= \frac{4 \pi}{c} \BJ  \\
\end{aligned}
\end{equation}

(for this note this means unfortunately a switch from SI to CGS midstream)

For linear material (\(\BB = \mu \BH\), and \(\BD = \epsilon \BE\)) that is devoid of unbound charge and current (\(\rho =0\), and \(\BJ =0\)), we can assemble these into his (7.1) equations

\begin{equation}\label{eqn:emVacWave:140}
\begin{aligned}
\spacegrad \cdot \BB &= 0 \\
\spacegrad \cdot \BE &= 0 \\
\spacegrad \cross \BE + \inv{c}\PD{t}{\BB} &= 0 \\
\spacegrad \cross \BB - \frac{\epsilon \mu}{c}\PD{t}{\BE} &= 0  \\
\end{aligned}
\end{equation}

In this macroscopic form, it is not obvious how to assemble the equations into a nice tidy GA form.  A compromise is

\begin{equation}\label{eqn:emVacWave:compromise}
\begin{aligned}
\spacegrad \BE + \partial_0 (I\BB) &= 0 \\
\spacegrad (I \BB) + \epsilon \mu \partial_0 \BE &= 0
\end{aligned}
\end{equation}

Although not as pretty, we can at least derive the wave equations from these.  For example for \(\BE\), we apply one additional spatial gradient

\begin{equation}\label{eqn:emVacWave:160}
\begin{aligned}
0
&= \spacegrad^2 \BE + \partial_0 (\spacegrad I \BB) \\
&= \spacegrad^2 \BE + \partial_0 ( -\epsilon \mu \partial_0 \BE ) \\
\end{aligned}
\end{equation}

For \(\BB\) we get the same, and have two wave equations

\begin{equation}\label{eqn:emVacWave:waveEquationsMedia}
\begin{aligned}
\frac{\mu \epsilon}{c^2} \frac{\partial^2 \BE}{\partial t^2} - \spacegrad^2 \BE &= 0 \\
\frac{\mu \epsilon}{c^2} \frac{\partial^2 \BB}{\partial t^2} - \spacegrad^2 \BB &= 0
\end{aligned}
\end{equation}

The wave velocity is thus not \(c\), but instead the reduced speed of \(c/{\sqrt{\mu\epsilon}}\).

The fact that it is possible to assemble wave equations of this form means that there must also be a simpler form than \eqnref{eqn:emVacWave:compromise}.  The reduced velocity is the clue, and that can be used to refactor the constants

\begin{equation}\label{eqn:emVacWave:180}
\begin{aligned}
\spacegrad \BE + \sqrt{\mu\epsilon}\partial_0 \left(\frac{I\BB}{\sqrt{\mu\epsilon}}\right) &= 0 \\
\spacegrad \left(\frac{I \BB}{\sqrt{\mu\epsilon}}\right) + \sqrt{\mu\epsilon} \partial_0 \BE &= 0
\end{aligned}
\end{equation}

These can now be added

\begin{equation}\label{eqn:emVacWave:200}
\begin{aligned}
\left(\spacegrad + \sqrt{\mu\epsilon} \partial_0\right)\left(\BE + \frac{I\BB}{\sqrt{\mu\epsilon}}\right) = 0
\end{aligned}
\end{equation}

This allows for the one liner derivation of \eqnref{eqn:emVacWave:waveEquationsMedia} by premultiplying by the conjugate
operator \(-\spacegrad + \sqrt{\mu\epsilon} \partial_0\)

\begin{equation}\label{eqn:emVacWave:220}
\begin{aligned}
0
&=
\left(-\spacegrad + \sqrt{\mu\epsilon} \partial_0\right)
\left(\spacegrad + \sqrt{\mu\epsilon} \partial_0\right)
\left(\BE + \frac{I\BB}{\sqrt{\mu\epsilon}}\right) \\
&=
\left(-\spacegrad^2 + \frac{\mu\epsilon}{c^2} \partial_{tt}\right)
\left(\BE + \frac{I\BB}{\sqrt{\mu\epsilon}}\right)
\end{aligned}
\end{equation}

Using the same hint, and doing some rearrangement, we can write Jackson's equations (6.70) as

\begin{equation}\label{eqn:emVacWave:maxwellEquationMedia}
\begin{aligned}
\left(\spacegrad + \sqrt{\mu\epsilon} \partial_0\right)\left(\BE + \frac{I\BB}{\sqrt{\mu\epsilon}}\right) =
\frac{4\pi}{\epsilon}\left( \rho - \frac{\sqrt{\mu\epsilon}}{c}\BJ \right)
\end{aligned}
\end{equation}

%\EndArticle
