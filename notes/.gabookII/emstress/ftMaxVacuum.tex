%
% Copyright � 2012 Peeter Joot.  All Rights Reserved.
% Licenced as described in the file LICENSE under the root directory of this GIT repository.
%

%
%
%\documentclass[]{eliblogwidescreen}

\usepackage{amsmath}
\usepackage{mathpazo}

%
% shorthand for bold symbols, convenient for vectors and matrices
%
\newcommand{\Ba}[0]{\mathbf{a}}
\newcommand{\Bb}[0]{\mathbf{b}}
\newcommand{\Bc}[0]{\mathbf{c}}
\newcommand{\Bd}[0]{\mathbf{d}}
\newcommand{\Be}[0]{\mathbf{e}}
\newcommand{\Bf}[0]{\mathbf{f}}
\newcommand{\Bg}[0]{\mathbf{g}}
\newcommand{\Bh}[0]{\mathbf{h}}
\newcommand{\Bi}[0]{\mathbf{i}}
\newcommand{\Bj}[0]{\mathbf{j}}
\newcommand{\Bk}[0]{\mathbf{k}}
\newcommand{\Bl}[0]{\mathbf{l}}
\newcommand{\Bm}[0]{\mathbf{m}}
\newcommand{\Bn}[0]{\mathbf{n}}
\newcommand{\Bo}[0]{\mathbf{o}}
\newcommand{\Bp}[0]{\mathbf{p}}
\newcommand{\Bq}[0]{\mathbf{q}}
\newcommand{\Br}[0]{\mathbf{r}}
\newcommand{\Bs}[0]{\mathbf{s}}
\newcommand{\Bt}[0]{\mathbf{t}}
\newcommand{\Bu}[0]{\mathbf{u}}
\newcommand{\Bv}[0]{\mathbf{v}}
\newcommand{\Bw}[0]{\mathbf{w}}
\newcommand{\Bx}[0]{\mathbf{x}}
\newcommand{\By}[0]{\mathbf{y}}
\newcommand{\Bz}[0]{\mathbf{z}}
\newcommand{\BA}[0]{\mathbf{A}}
\newcommand{\BB}[0]{\mathbf{B}}
\newcommand{\BC}[0]{\mathbf{C}}
\newcommand{\BD}[0]{\mathbf{D}}
\newcommand{\BE}[0]{\mathbf{E}}
\newcommand{\BF}[0]{\mathbf{F}}
\newcommand{\BG}[0]{\mathbf{G}}
\newcommand{\BH}[0]{\mathbf{H}}
\newcommand{\BI}[0]{\mathbf{I}}
\newcommand{\BJ}[0]{\mathbf{J}}
\newcommand{\BK}[0]{\mathbf{K}}
\newcommand{\BL}[0]{\mathbf{L}}
\newcommand{\BM}[0]{\mathbf{M}}
\newcommand{\BN}[0]{\mathbf{N}}
\newcommand{\BO}[0]{\mathbf{O}}
\newcommand{\BP}[0]{\mathbf{P}}
\newcommand{\BQ}[0]{\mathbf{Q}}
\newcommand{\BR}[0]{\mathbf{R}}
\newcommand{\BS}[0]{\mathbf{S}}
\newcommand{\BT}[0]{\mathbf{T}}
\newcommand{\BU}[0]{\mathbf{U}}
\newcommand{\BV}[0]{\mathbf{V}}
\newcommand{\BW}[0]{\mathbf{W}}
\newcommand{\BX}[0]{\mathbf{X}}
\newcommand{\BY}[0]{\mathbf{Y}}
\newcommand{\BZ}[0]{\mathbf{Z}}

\newcommand{\Bzero}[0]{\mathbf{0}}
\newcommand{\Btheta}[0]{\boldsymbol{\theta}}
\newcommand{\Btau}[0]{\boldsymbol{\tau}}
\newcommand{\Bomega}[0]{\boldsymbol{\omega}}

%
% shorthand for unit vectors
%
\newcommand{\acap}[0]{\hat{\Ba}}
\newcommand{\bcap}[0]{\hat{\Bb}}
\newcommand{\ccap}[0]{\hat{\Bc}}
\newcommand{\dcap}[0]{\hat{\Bd}}
\newcommand{\ecap}[0]{\hat{\Be}}
\newcommand{\fcap}[0]{\hat{\Bf}}
\newcommand{\gcap}[0]{\hat{\Bg}}
\newcommand{\hcap}[0]{\hat{\Bh}}
\newcommand{\icap}[0]{\hat{\Bi}}
\newcommand{\jcap}[0]{\hat{\Bj}}
\newcommand{\kcap}[0]{\hat{\Bk}}
\newcommand{\lcap}[0]{\hat{\Bl}}
\newcommand{\mcap}[0]{\hat{\Bm}}
\newcommand{\ncap}[0]{\hat{\Bn}}
\newcommand{\ocap}[0]{\hat{\Bo}}
\newcommand{\pcap}[0]{\hat{\Bp}}
\newcommand{\qcap}[0]{\hat{\Bq}}
\newcommand{\rcap}[0]{\hat{\Br}}
\newcommand{\scap}[0]{\hat{\Bs}}
\newcommand{\tcap}[0]{\hat{\Bt}}
\newcommand{\ucap}[0]{\hat{\Bu}}
\newcommand{\vcap}[0]{\hat{\Bv}}
\newcommand{\wcap}[0]{\hat{\Bw}}
\newcommand{\xcap}[0]{\hat{\Bx}}
\newcommand{\ycap}[0]{\hat{\By}}
\newcommand{\zcap}[0]{\hat{\Bz}}
\newcommand{\thetacap}[0]{\hat{\Btheta}}

%
% to write R^n and C^n in a distinguishable fashion.  Perhaps change this
% to the double lined characters upon figuring out how to do so.
%
\newcommand{\C}[1]{$\mathbb{C}^{#1}$}
\newcommand{\R}[1]{$\mathbb{R}^{#1}$}

%
% various generally useful helpers
%

% derivative of #1 wrt. #2:
\newcommand{\D}[2] {\frac {d#2} {d#1}}

\newcommand{\inv}[1]{\frac{1}{#1}}
\newcommand{\cross}[0]{\times}

\newcommand{\abs}[1]{\lvert{#1}\rvert}
\newcommand{\norm}[1]{\lVert{#1}\rVert}
\newcommand{\innerprod}[2]{\langle{#1}, {#2}\rangle}
\newcommand{\dotprod}[2]{{#1} \cdot {#2}}
\newcommand{\bdotprod}[2]{\left({#1} \cdot {#2}\right)}
\newcommand{\crossprod}[2]{{#1} \cross {#2}}
\newcommand{\tripleprod}[3]{\dotprod{\left(\crossprod{#1}{#2}\right)}{#3}}

\DeclareMathOperator{\Proj}{Proj}
\DeclareMathOperator{\Span}{span}
\DeclareMathOperator{\Sgn}{sgn}
\DeclareMathOperator{\Area}{Area}
\DeclareMathOperator{\Volume}{Volume}

%
% A few miscellaneous things specific to this document
%
\newcommand{\crossop}[1]{\crossprod{#1}{}}

% R2 vector.
\newcommand{\VectorTwo}[2]{
\begin{bmatrix}
 {#1} \\
 {#2}
\end{bmatrix}
}

\newcommand{\VectorN}[1]{
\begin{bmatrix}
{#1}_1 \\
{#1}_2 \\
\vdots \\
{#1}_N \\
\end{bmatrix}
}

\newcommand{\DETuvij}[4]{
\begin{vmatrix}
 {#1}_{#3} & {#1}_{#4} \\
 {#2}_{#3} & {#2}_{#4}
\end{vmatrix}
}

\newcommand{\DETuvwijk}[6]{
\begin{vmatrix}
 {#1}_{#4} & {#1}_{#5} & {#1}_{#6} \\
 {#2}_{#4} & {#2}_{#5} & {#2}_{#6} \\
 {#3}_{#4} & {#3}_{#5} & {#3}_{#6}
\end{vmatrix}
}

\newcommand{\DETuvwxijkl}[8]{
\begin{vmatrix}
 {#1}_{#5} & {#1}_{#6} & {#1}_{#7} & {#1}_{#8} \\
 {#2}_{#5} & {#2}_{#6} & {#2}_{#7} & {#2}_{#8} \\
 {#3}_{#5} & {#3}_{#6} & {#3}_{#7} & {#3}_{#8} \\
 {#4}_{#5} & {#4}_{#6} & {#4}_{#7} & {#4}_{#8} \\
\end{vmatrix}
}

%\newcommand{\DETuvwxyijklm}[10]{
%\begin{vmatrix}
% {#1}_{#6} & {#1}_{#7} & {#1}_{#8} & {#1}_{#9} & {#1}_{#10} \\
% {#2}_{#6} & {#2}_{#7} & {#2}_{#8} & {#2}_{#9} & {#2}_{#10} \\
% {#3}_{#6} & {#3}_{#7} & {#3}_{#8} & {#3}_{#9} & {#3}_{#10} \\
% {#4}_{#6} & {#4}_{#7} & {#4}_{#8} & {#4}_{#9} & {#4}_{#10} \\
% {#5}_{#6} & {#5}_{#7} & {#5}_{#8} & {#5}_{#9} & {#5}_{#10}
%\end{vmatrix}
%}

% R3 vector.
\newcommand{\VectorThree}[3]{
\begin{bmatrix}
 {#1} \\
 {#2} \\
 {#3}
\end{bmatrix}
}



\author{Peeter Joot}
\email{peeter.joot@gmail.com}


% reformated equations 63, 8, 16, 17 to break long lines.
%%
% Copyright � 2015 Peeter Joot.  All Rights Reserved.
% Licenced as described in the file LICENSE under the root directory of this GIT repository.
%
\documentclass[]{eliblog}

\usepackage{amsmath}
\usepackage{mathpazo}

%
% shorthand for bold symbols, convenient for vectors and matrices
%
\newcommand{\Ba}[0]{\mathbf{a}}
\newcommand{\Bb}[0]{\mathbf{b}}
\newcommand{\Bc}[0]{\mathbf{c}}
\newcommand{\Bd}[0]{\mathbf{d}}
\newcommand{\Be}[0]{\mathbf{e}}
\newcommand{\Bf}[0]{\mathbf{f}}
\newcommand{\Bg}[0]{\mathbf{g}}
\newcommand{\Bh}[0]{\mathbf{h}}
\newcommand{\Bi}[0]{\mathbf{i}}
\newcommand{\Bj}[0]{\mathbf{j}}
\newcommand{\Bk}[0]{\mathbf{k}}
\newcommand{\Bl}[0]{\mathbf{l}}
\newcommand{\Bm}[0]{\mathbf{m}}
\newcommand{\Bn}[0]{\mathbf{n}}
\newcommand{\Bo}[0]{\mathbf{o}}
\newcommand{\Bp}[0]{\mathbf{p}}
\newcommand{\Bq}[0]{\mathbf{q}}
\newcommand{\Br}[0]{\mathbf{r}}
\newcommand{\Bs}[0]{\mathbf{s}}
\newcommand{\Bt}[0]{\mathbf{t}}
\newcommand{\Bu}[0]{\mathbf{u}}
\newcommand{\Bv}[0]{\mathbf{v}}
\newcommand{\Bw}[0]{\mathbf{w}}
\newcommand{\Bx}[0]{\mathbf{x}}
\newcommand{\By}[0]{\mathbf{y}}
\newcommand{\Bz}[0]{\mathbf{z}}
\newcommand{\BA}[0]{\mathbf{A}}
\newcommand{\BB}[0]{\mathbf{B}}
\newcommand{\BC}[0]{\mathbf{C}}
\newcommand{\BD}[0]{\mathbf{D}}
\newcommand{\BE}[0]{\mathbf{E}}
\newcommand{\BF}[0]{\mathbf{F}}
\newcommand{\BG}[0]{\mathbf{G}}
\newcommand{\BH}[0]{\mathbf{H}}
\newcommand{\BI}[0]{\mathbf{I}}
\newcommand{\BJ}[0]{\mathbf{J}}
\newcommand{\BK}[0]{\mathbf{K}}
\newcommand{\BL}[0]{\mathbf{L}}
\newcommand{\BM}[0]{\mathbf{M}}
\newcommand{\BN}[0]{\mathbf{N}}
\newcommand{\BO}[0]{\mathbf{O}}
\newcommand{\BP}[0]{\mathbf{P}}
\newcommand{\BQ}[0]{\mathbf{Q}}
\newcommand{\BR}[0]{\mathbf{R}}
\newcommand{\BS}[0]{\mathbf{S}}
\newcommand{\BT}[0]{\mathbf{T}}
\newcommand{\BU}[0]{\mathbf{U}}
\newcommand{\BV}[0]{\mathbf{V}}
\newcommand{\BW}[0]{\mathbf{W}}
\newcommand{\BX}[0]{\mathbf{X}}
\newcommand{\BY}[0]{\mathbf{Y}}
\newcommand{\BZ}[0]{\mathbf{Z}}

\newcommand{\Bzero}[0]{\mathbf{0}}
\newcommand{\Btheta}[0]{\boldsymbol{\theta}}
\newcommand{\Btau}[0]{\boldsymbol{\tau}}
\newcommand{\Bomega}[0]{\boldsymbol{\omega}}

%
% shorthand for unit vectors
%
\newcommand{\acap}[0]{\hat{\Ba}}
\newcommand{\bcap}[0]{\hat{\Bb}}
\newcommand{\ccap}[0]{\hat{\Bc}}
\newcommand{\dcap}[0]{\hat{\Bd}}
\newcommand{\ecap}[0]{\hat{\Be}}
\newcommand{\fcap}[0]{\hat{\Bf}}
\newcommand{\gcap}[0]{\hat{\Bg}}
\newcommand{\hcap}[0]{\hat{\Bh}}
\newcommand{\icap}[0]{\hat{\Bi}}
\newcommand{\jcap}[0]{\hat{\Bj}}
\newcommand{\kcap}[0]{\hat{\Bk}}
\newcommand{\lcap}[0]{\hat{\Bl}}
\newcommand{\mcap}[0]{\hat{\Bm}}
\newcommand{\ncap}[0]{\hat{\Bn}}
\newcommand{\ocap}[0]{\hat{\Bo}}
\newcommand{\pcap}[0]{\hat{\Bp}}
\newcommand{\qcap}[0]{\hat{\Bq}}
\newcommand{\rcap}[0]{\hat{\Br}}
\newcommand{\scap}[0]{\hat{\Bs}}
\newcommand{\tcap}[0]{\hat{\Bt}}
\newcommand{\ucap}[0]{\hat{\Bu}}
\newcommand{\vcap}[0]{\hat{\Bv}}
\newcommand{\wcap}[0]{\hat{\Bw}}
\newcommand{\xcap}[0]{\hat{\Bx}}
\newcommand{\ycap}[0]{\hat{\By}}
\newcommand{\zcap}[0]{\hat{\Bz}}
\newcommand{\thetacap}[0]{\hat{\Btheta}}

%
% to write R^n and C^n in a distinguishable fashion.  Perhaps change this
% to the double lined characters upon figuring out how to do so.
%
\newcommand{\C}[1]{$\mathbb{C}^{#1}$}
\newcommand{\R}[1]{$\mathbb{R}^{#1}$}

%
% various generally useful helpers
%

% derivative of #1 wrt. #2:
\newcommand{\D}[2] {\frac {d#2} {d#1}}

\newcommand{\inv}[1]{\frac{1}{#1}}
\newcommand{\cross}[0]{\times}

\newcommand{\abs}[1]{\lvert{#1}\rvert}
\newcommand{\norm}[1]{\lVert{#1}\rVert}
\newcommand{\innerprod}[2]{\langle{#1}, {#2}\rangle}
\newcommand{\dotprod}[2]{{#1} \cdot {#2}}
\newcommand{\bdotprod}[2]{\left({#1} \cdot {#2}\right)}
\newcommand{\crossprod}[2]{{#1} \cross {#2}}
\newcommand{\tripleprod}[3]{\dotprod{\left(\crossprod{#1}{#2}\right)}{#3}}

\DeclareMathOperator{\Proj}{Proj}
\DeclareMathOperator{\Span}{span}
\DeclareMathOperator{\Sgn}{sgn}
\DeclareMathOperator{\Area}{Area}
\DeclareMathOperator{\Volume}{Volume}

%
% A few miscellaneous things specific to this document
%
\newcommand{\crossop}[1]{\crossprod{#1}{}}

% R2 vector.
\newcommand{\VectorTwo}[2]{
\begin{bmatrix}
 {#1} \\
 {#2}
\end{bmatrix}
}

\newcommand{\VectorN}[1]{
\begin{bmatrix}
{#1}_1 \\
{#1}_2 \\
\vdots \\
{#1}_N \\
\end{bmatrix}
}

\newcommand{\DETuvij}[4]{
\begin{vmatrix}
 {#1}_{#3} & {#1}_{#4} \\
 {#2}_{#3} & {#2}_{#4}
\end{vmatrix}
}

\newcommand{\DETuvwijk}[6]{
\begin{vmatrix}
 {#1}_{#4} & {#1}_{#5} & {#1}_{#6} \\
 {#2}_{#4} & {#2}_{#5} & {#2}_{#6} \\
 {#3}_{#4} & {#3}_{#5} & {#3}_{#6}
\end{vmatrix}
}

\newcommand{\DETuvwxijkl}[8]{
\begin{vmatrix}
 {#1}_{#5} & {#1}_{#6} & {#1}_{#7} & {#1}_{#8} \\
 {#2}_{#5} & {#2}_{#6} & {#2}_{#7} & {#2}_{#8} \\
 {#3}_{#5} & {#3}_{#6} & {#3}_{#7} & {#3}_{#8} \\
 {#4}_{#5} & {#4}_{#6} & {#4}_{#7} & {#4}_{#8} \\
\end{vmatrix}
}

%\newcommand{\DETuvwxyijklm}[10]{
%\begin{vmatrix}
% {#1}_{#6} & {#1}_{#7} & {#1}_{#8} & {#1}_{#9} & {#1}_{#10} \\
% {#2}_{#6} & {#2}_{#7} & {#2}_{#8} & {#2}_{#9} & {#2}_{#10} \\
% {#3}_{#6} & {#3}_{#7} & {#3}_{#8} & {#3}_{#9} & {#3}_{#10} \\
% {#4}_{#6} & {#4}_{#7} & {#4}_{#8} & {#4}_{#9} & {#4}_{#10} \\
% {#5}_{#6} & {#5}_{#7} & {#5}_{#8} & {#5}_{#9} & {#5}_{#10}
%\end{vmatrix}
%}

% R3 vector.
\newcommand{\VectorThree}[3]{
\begin{bmatrix}
 {#1} \\
 {#2} \\
 {#3}
\end{bmatrix}
}



\author{Peeter Joot}
\email{peeter.joot@gmail.com}


\chapter{Fourier transform solutions and associated energy and momentum for the homogeneous Maxwell equation}
\index{Maxwell equation!Fourier transform}
\index{Maxwell equation!energy and momentum}
\label{chap:ftMaxVacuum}
%\blogpage{http://sites.google.com/site/peeterjoot/math2009/ftMaxVacuum.pdf}
%\date{Dec 21, 2009}
%\revisionInfo{ftMaxVacuum.tex}

%\beginArtWithToc
\beginArtNoToc

%FIXME: toggled from not using 'we have' to doing so.  Fix the inconsistencies.
\section{Motivation and notation}

In \chapcite{fourierMaxVac}, building on \chapcite{complexFieldEnergy} a derivation for the energy and momentum density was derived for an assumed Fourier series solution to the homogeneous Maxwell's equation.  Here we move to the continuous case examining Fourier transform solutions and the associated energy and momentum density.

A complex (phasor) representation is implied, so taking real parts when all is said and done is required of the fields.  For the energy momentum tensor the Geometric Algebra form, modified for complex fields, is used

\begin{equation}
\label{eqn:ftMaxVacuum:1}
T(a) = -\frac{\epsilon_0}{2} \Real \Bigl( \conjugateStar{F} a F \Bigr).
\end{equation}

The assumed four vector potential will be written

\begin{equation}
\label{eqn:ftMaxVacuum:2}
A(\Bx, t) = A^\mu(\Bx, t) \gamma_\mu = \inv{(\sqrt{2 \pi})^3} \int A(\Bk, t) e^{i \Bk \cdot \Bx } d^3 \Bk.
\end{equation}

Subject to the requirement that \(A\) is a solution of Maxwell's equation

\begin{equation}
\label{eqn:ftMaxVacuum:3}
\grad (\grad \wedge A) = 0.
\end{equation}

To avoid latex hell, no special notation will be used for the Fourier coefficients,

\begin{equation}
\label{eqn:ftMaxVacuum:3a}
A(\Bk, t) = \inv{(\sqrt{2 \pi})^3} \int A(\Bx, t) e^{-i \Bk \cdot \Bx } d^3 \Bx.
\end{equation}

When convenient and unambiguous, this \((\Bk,t)\) dependence will be implied.

Having picked a time and space representation for the field, it will be natural to express both the four potential and the gradient as scalar plus spatial vector, instead of using the Dirac basis.  For the gradient this is

\begin{equation}
\label{eqn:ftMaxVacuum:4}
\grad = \gamma^\mu \partial_\mu = (\partial_0 - \spacegrad) \gamma_0 = \gamma_0 (\partial_0 + \spacegrad),
\end{equation}

and for the four potential (or the Fourier transform functions), this is

\begin{equation}
\label{eqn:ftMaxVacuum:5}
A = \gamma_\mu A^\mu = (\phi + \BA) \gamma_0 = \gamma_0 (\phi - \BA).
\end{equation}

\section{Setup}

The field bivector \(F = \grad \wedge A\) is required for the energy momentum tensor.  This is

\begin{equation}\label{eqn:ftMaxVacuum:222}
\begin{aligned}
\grad \wedge A
&= \inv{2}\left( \rgrad A - A \lgrad \right) \\
&= \inv{2}\left( (\rpartial_0 - \rspacegrad) \gamma_0 \gamma_0 (\phi - \BA)
-
(\phi + \BA) \gamma_0 \gamma_0 (\lpartial_0 + \lspacegrad)
\right) \\
&= -\spacegrad \phi -\partial_0 \BA + \inv{2}(\rspacegrad \BA - \BA \lspacegrad)
\end{aligned}
\end{equation}

This last term is a spatial curl and the field is then

\begin{equation}
\label{eqn:ftMaxVacuum:6}
F = -\spacegrad \phi -\partial_0 \BA + \spacegrad \wedge \BA
\end{equation}

Applied to the Fourier representation this is

\begin{equation}
\label{eqn:ftMaxVacuum:7}
F =
\inv{(\sqrt{2 \pi})^3} \int
\left(
- \inv{c} \dot{\BA}
- i \Bk \phi
+ i \Bk \wedge \BA
\right)
e^{i \Bk \cdot \Bx } d^3 \Bk.
\end{equation}

It is only the real parts of this that we are actually interested in, unless physical meaning can be assigned to the complete complex vector field.

\section{Constraints supplied by Maxwell's equation}

A Fourier transform solution of Maxwell's vacuum equation \(\grad F = 0\) has been assumed.  Having expressed the Faraday bivector in terms of spatial vector quantities, it is more convenient to do this back substitution into after pre-multiplying Maxwell's equation by \(\gamma_0\), namely

\begin{equation}
\label{eqn:ftMaxVacuum:20}
\begin{aligned}
0
&= \gamma_0 \grad F \\
&= (\partial_0 + \spacegrad) F.
\end{aligned}
\end{equation}

Applied to the spatially decomposed field as specified in \autoref{eqn:ftMaxVacuum:6}, this is

\begin{equation}\label{eqn:ftMaxVacuum:302}
\begin{aligned}
0
&=
-\partial_0 \spacegrad \phi
-\partial_{00} \BA
+ \partial_0 \spacegrad \wedge \BA
-\spacegrad^2 \phi
- \spacegrad \partial_0 \BA
+ \spacegrad \cdot (\spacegrad \wedge \BA ) \\
&=
- \partial_0 \spacegrad \phi - \spacegrad^2 \phi
- \partial_{00} \BA
- \spacegrad \cdot \partial_0 \BA
+ \spacegrad^2 \BA
- \spacegrad (\spacegrad \cdot \BA ) \\
\end{aligned}
\end{equation}

All grades of this equation must simultaneously equal zero, and the bivector grades have canceled (assuming commuting space and time partials), leaving two equations of constraint for the system

\begin{subequations}
\label{eqn:ftMaxVacuum:22}
\begin{align}
0 &=
\spacegrad^2 \phi + \spacegrad \cdot \partial_0 \BA
\label{eqn:ftMaxVacuum:22a}
\\
0 &=
\partial_{00} \BA - \spacegrad^2 \BA
+ \spacegrad \partial_0 \phi
+ \spacegrad ( \spacegrad \cdot \BA )
\label{eqn:ftMaxVacuum:22b}
\end{align}
\end{subequations}

It is immediately evident that a gauge transformation could be immediately helpful to simplify things.  In \citep{bohm1989qt} the gauge choice \(\spacegrad \cdot \BA = 0\) is used.  From \autoref{eqn:ftMaxVacuum:22a} this implies that \(\spacegrad^2 \phi = 0\).  Bohm argues that for this current and charge free case this implies \(\phi = 0\), but he also has a periodicity constraint.  Without a periodicity constraint it is easy to manufacture non-zero counterexamples.  One is a linear function in the space and time coordinates

\begin{equation}
\label{eqn:ftMaxVacuum:23}
\phi = p x + q y + r z + s t
\end{equation}

This is a valid scalar potential provided that the wave equation for the vector potential is also a solution.  We can however, force \(\phi = 0\) by making the transformation \(A^\mu \rightarrow A^\mu + \partial^\mu \psi\), which in non-covariant notation is

\begin{equation}
\label{eqn:ftMaxVacuum:24}
\begin{aligned}
\phi &\rightarrow \phi + \inv{c} \partial_t \psi \\
\BA &\rightarrow \phi - \spacegrad \psi
\end{aligned}
\end{equation}

If the transformed field \(\phi' = \phi + \partial_t \psi/c\) can be forced to zero, then the complexity of the associated Maxwell equations are reduced.  In particular, antidifferentiation of \(\phi = -(1/c) \partial_t \psi\), yields

\begin{equation}
\label{eqn:ftMaxVacuum:25}
\psi(\Bx,t) = \psi(\Bx, 0) - c \int_{\tau=0}^t \phi(\Bx, \tau) d\tau.
\end{equation}

Dropping primes, the transformed Maxwell equations now take the form

\begin{subequations}
\label{eqn:ftMaxVacuum:26}
\begin{align}
0 &= \partial_t( \spacegrad \cdot \BA )
\label{eqn:ftMaxVacuum:26a}
\\
0 &=
\partial_{00} \BA - \spacegrad^2 \BA + \spacegrad (\spacegrad \cdot \BA ).
\label{eqn:ftMaxVacuum:26b}
\end{align}
\end{subequations}

There are two classes of solutions that stand out for these equations.  If the vector potential is constant in time \(\BA(\Bx,t) = \BA(\Bx)\), Maxwell's equations are reduced to the single equation

\begin{equation}
\label{eqn:ftMaxVacuum:28}
0
= - \spacegrad^2 \BA + \spacegrad (\spacegrad \cdot \BA ).
\end{equation}

Observe that a gradient can be factored out of this equation

\begin{equation}\label{eqn:ftMaxVacuum:442}
\begin{aligned}
- \spacegrad^2 \BA + \spacegrad (\spacegrad \cdot \BA )
&=
\spacegrad (-\spacegrad \BA + \spacegrad \cdot \BA ) \\
&=
-\spacegrad (\spacegrad \wedge \BA).
\end{aligned}
\end{equation}

The solutions are then those \(\BA\)s that satisfy both
\begin{subequations}
\begin{align}
\label{eqn:ftMaxVacuum:28b}
0 &= \partial_t \BA \\
0 &= \spacegrad (\spacegrad \wedge \BA).
\end{align}
\end{subequations}

In particular any non-time dependent potential \(\BA\) with constant curl provides a solution to Maxwell's equations.  There may be other solutions to \autoref{eqn:ftMaxVacuum:28} too that are more general.  Returning to \autoref{eqn:ftMaxVacuum:26} a second way to satisfy these equations stands out.  Instead of requiring of \(\BA\) constant curl, constant divergence with respect to the time partial eliminates \autoref{eqn:ftMaxVacuum:26a}.  The simplest resulting equations are those for which the divergence is a constant in time and space (such as zero).  The solution set are then spanned by the vectors \(\BA\) for which

\begin{subequations}
\label{eqn:ftMaxVacuum:29}
\begin{align}
\text{constant} &= \spacegrad \cdot \BA
\label{eqn:ftMaxVacuum:29a}
\\
0 &= \inv{c^2} \partial_{tt} \BA - \spacegrad^2 \BA.
\label{eqn:ftMaxVacuum:29b}
\end{align}
\end{subequations}

Any \(\BA\) that both has constant divergence and satisfies the wave equation will via \autoref{eqn:ftMaxVacuum:6} then produce a solution to Maxwell's equation.

%Also consider the gauge transformation here.  Note that the gauge transformation shows that the \(\Abs{\BA}^2\) belongs in the potential, while the \(\Abs{\dot{\BA}/c}^2\) belongs in the kinetic.  Assume that the remainder of the position only and velocity only terms are grouped that way and see where it leads.

\section{Maxwell equation constraints applied to the assumed Fourier solutions}

Let us consider Maxwell's equations in all three forms, \autoref{eqn:ftMaxVacuum:22}, \autoref{eqn:ftMaxVacuum:28b}, and \autoref{eqn:ftMaxVacuum:29} and apply these constraints to the assumed Fourier solution.

In all cases the starting point is a pair of Fourier transform relationships, where the Fourier transforms are the functions to be determined

\begin{subequations}
\label{eqn:ftMaxVacuum:40}
\begin{align}
\phi(\Bx, t) &= (2 \pi)^{-3/2} \int \phi(\Bk, t) e^{i \Bk \cdot \Bx } d^3 \Bk
\label{eqn:ftMaxVacuum:40a}
\\
\BA(\Bx, t) &= (2 \pi)^{-3/2} \int \BA(\Bk, t) e^{i \Bk \cdot \Bx } d^3 \Bk
\label{eqn:ftMaxVacuum:40b}
\end{align}
\end{subequations}

\subsection{Case I.  Constant time vector potential.  Scalar potential eliminated by gauge transformation}
\index{gauge transformation}
\index{vector potential}

From \autoref{eqn:ftMaxVacuum:40a} we require

\begin{equation}\label{eqn:ftMaxVacuum:50}
0 = (2 \pi)^{-3/2} \int \partial_t \BA(\Bk, t) e^{i \Bk \cdot \Bx } d^3 \Bk.
\end{equation}

So the Fourier transform also cannot have any time dependence, and we have

\begin{equation}\label{eqn:ftMaxVacuum:51}
\BA(\Bx, t) = (2 \pi)^{-3/2} \int \BA(\Bk) e^{i \Bk \cdot \Bx } d^3 \Bk
\end{equation}

What is the curl of this?  Temporarily falling back to coordinates is easiest for this calculation

\begin{equation}\label{eqn:ftMaxVacuum:522}
\begin{aligned}
\spacegrad \wedge \BA(\Bk) e^{i\Bk \cdot \Bx}
&=
\sigma_m \partial_m \wedge \sigma_n A^n(\Bk) e^{i \Bx \cdot \Bx} \\
&=
\sigma_m \wedge \sigma_n A^n(\Bk) i k^m e^{i \Bx \cdot \Bx} \\
&=
i\Bk \wedge \BA(\Bk) e^{i \Bx \cdot \Bx} \\
\end{aligned}
\end{equation}

This gives

\begin{equation}\label{eqn:ftMaxVacuum:52}
\spacegrad \wedge \BA(\Bx, t) = (2 \pi)^{-3/2} \int i \Bk \wedge \BA(\Bk) e^{i \Bk \cdot \Bx } d^3 \Bk.
\end{equation}

We want to equate the divergence of this to zero.  Neglecting the integral and constant factor this requires

\begin{equation}\label{eqn:ftMaxVacuum:542}
\begin{aligned}
0
&=
\spacegrad \cdot \left( i \Bk \wedge \BA e^{i\Bk \cdot \Bx} \right) \\
&=
\gpgradeone{ \sigma_m \partial_m i (\Bk \wedge \BA) e^{i\Bk \cdot \Bx} } \\
&=
-\gpgradeone{ \sigma_m (\Bk \wedge \BA) k^m e^{i\Bk \cdot \Bx} } \\
&=
-\Bk \cdot (\Bk \wedge \BA) e^{i\Bk \cdot \Bx} \\
\end{aligned}
\end{equation}

Requiring that the plane spanned by \(\Bk\) and \(\BA(\Bk)\) be perpendicular to \(\Bk\) implies that \(\BA \propto \Bk\).  The solution set is then completely described by functions of the form

\begin{equation}\label{eqn:ftMaxVacuum:53}
\BA(\Bx, t) = (2 \pi)^{-3/2} \int \Bk \psi(\Bk) e^{i \Bk \cdot \Bx } d^3 \Bk,
\end{equation}

where \(\psi(\Bk)\) is an arbitrary scalar valued function.  This is however, an extremely uninteresting solution since the curl is uniformly zero

\begin{equation}\label{eqn:ftMaxVacuum:562}
\begin{aligned}
F
&= \spacegrad \wedge \BA \\
&= (2 \pi)^{-3/2} \int (i \Bk) \wedge \Bk \psi(\Bk) e^{i \Bk \cdot \Bx } d^3 \Bk.
\end{aligned}
\end{equation}

Since \(\Bk \wedge \Bk = 0\), when all is said and done the \(\phi = 0\), \(\partial_t \BA = 0\) case appears to have no non-trivial (zero) solutions.  Moving on, ...

\subsection{Case II.  Constant vector potential divergence.  Scalar potential eliminated by gauge transformation}
\index{divergence}
\index{gauge transformation}

Next in the order of complexity is consideration of the case \autoref{eqn:ftMaxVacuum:29}.  Here we also have \(\phi = 0\), eliminated by gauge transformation, and are looking for solutions with the constraint

\begin{equation}\label{eqn:ftMaxVacuum:582}
\begin{aligned}
\text{constant}
&= \spacegrad \cdot \BA(\Bx, t) \\
&= (2 \pi)^{-3/2} \int i \Bk \cdot \BA(\Bk, t) e^{i \Bk \cdot \Bx } d^3 \Bk.
\end{aligned}
\end{equation}

How can this constraint be enforced?  The only obvious way is a requirement for \(\Bk \cdot \BA(\Bk, t)\) to be zero for all \((\Bk,t)\), meaning that our to be determined Fourier transform coefficients are required to be perpendicular to the wave number vector parameters at all times.

The remainder of Maxwell's equations, \autoref{eqn:ftMaxVacuum:29b} impose the addition constraint on the Fourier transform \(\BA(\Bk,t)\)

\begin{equation}\label{eqn:ftMaxVacuum:60}
0 = (2 \pi)^{-3/2} \int \left( \inv{c^2} \partial_{tt} \BA(\Bk, t) - i^2 \Bk^2 \BA(\Bk, t)\right) e^{i \Bk \cdot \Bx } d^3 \Bk.
\end{equation}

For zero equality for all \(\Bx\) it appears that we require the Fourier transforms \(\BA(\Bk)\) to be harmonic in time

\begin{equation}\label{eqn:ftMaxVacuum:61}
\partial_{tt} \BA(\Bk, t) = - c^2 \Bk^2 \BA(\Bk, t).
\end{equation}

This has the familiar exponential solutions

\begin{equation}\label{eqn:ftMaxVacuum:62}
\BA(\Bk, t) = \BA_{\pm}(\Bk) e^{ \pm i c \Abs{\Bk} t },
\end{equation}

also subject to a requirement that \(\Bk \cdot \BA(\Bk) = 0\).  Our field, where the \(\BA_{\pm}(\Bk)\) are to be determined by initial time conditions, is by \autoref{eqn:ftMaxVacuum:6} of the form

\begin{equation}\label{eqn:ftMaxVacuum:602}
\begin{aligned}
F(\Bx, t)
&=
\Real \frac{i}{(\sqrt{2\pi})^3} \int \Bigl( -\Abs{\Bk} \BA_{+}(\Bk) + \Bk \wedge \BA_{+}(\Bk) \Bigr) \exp(i \Bk \cdot \Bx + i c \Abs{\Bk} t) d^3 \Bk \\
&+ \Real \frac{i}{(\sqrt{2\pi})^3} \int \Bigl( \Abs{\Bk} \BA_{-}(\Bk) + \Bk \wedge \BA_{-}(\Bk) \Bigr) \exp(i \Bk \cdot \Bx - i c \Abs{\Bk} t) d^3 \Bk.
\end{aligned}
\end{equation}

Since \(0 = \Bk \cdot \BA_{\pm}(\Bk)\), we have \(\Bk \wedge \BA_{\pm}(\Bk) = \Bk \BA_{\pm}\).  This allows for factoring out of \(\Abs{\Bk}\).  The structure of the solution is not changed by incorporating the \(i (2\pi)^{-3/2} \Abs{\Bk}\) factors into \(\BA_{\pm}\), leaving the field having the general form

\begin{equation}\label{eqn:ftMaxVacuum:64}
\begin{aligned}
&F(\Bx, t) \\
&=
\Real \int ( \kcap - 1 ) \BA_{+}(\Bk) \exp(i \Bk \cdot \Bx + i c \Abs{\Bk} t) d^3 \Bk
+ \Real \int ( \kcap + 1 ) \BA_{-}(\Bk) \exp(i \Bk \cdot \Bx - i c \Abs{\Bk} t) d^3 \Bk.
\end{aligned}
\end{equation}

The original meaning of \(\BA_{\pm}\) as Fourier transforms of the vector potential is obscured by the tidy up change to absorb \(\Abs{\Bk}\), but the geometry of the solution is clearer this way.

It is also particularly straightforward to confirm that \(\gamma_0 \grad F = 0\) separately for either half of \autoref{eqn:ftMaxVacuum:64}.

\subsection{Case III.  Non-zero scalar potential.  No gauge transformation}

Now lets work from \autoref{eqn:ftMaxVacuum:22}.  In particular, a divergence operation can be factored from \autoref{eqn:ftMaxVacuum:22a}, for

\begin{equation}\label{eqn:ftMaxVacuum:70}
0 = \spacegrad \cdot (\spacegrad \phi + \partial_0 \BA).
\end{equation}

Right off the top, there is a requirement for

\begin{equation}\label{eqn:ftMaxVacuum:71}
\text{constant} = \spacegrad \phi + \partial_0 \BA.
\end{equation}

In terms of the Fourier transforms this is

\begin{equation}\label{eqn:ftMaxVacuum:72}
\text{constant} =
\inv{(\sqrt{2 \pi})^3} \int
\Bigl(
i \Bk \phi(\Bk, t) + \inv{c} \partial_t \BA(\Bk, t)
\Bigr)
e^{i \Bk \cdot \Bx } d^3 \Bk.
\end{equation}

Are there any ways for this to equal a constant for all \(\Bx\) without requiring that constant to be zero?  Assuming no for now, and that this constant must be zero, this implies a coupling between the \(\phi\) and \(\BA\) Fourier transforms of the form

\begin{equation}
\label{eqn:ftMaxVacuum:73}
\phi(\Bk, t) = -\inv{i c \Bk} \partial_t \BA(\Bk, t)
\end{equation}

A secondary implication is that \(\partial_t \BA(\Bk, t) \propto \Bk\) or else \(\phi(\Bk, t)\) is not a scalar.  We had a transverse solution by requiring via gauge transformation that \(\phi = 0\), and here we have instead the vector potential in the propagation direction.

A secondary confirmation that this is a required coupling between the scalar and vector potential can be had by evaluating the divergence equation of \autoref{eqn:ftMaxVacuum:70}

\begin{equation}\label{eqn:ftMaxVacuum:72b}
0 =
\inv{(\sqrt{2 \pi})^3} \int
\Bigl(
- \Bk^2 \phi(\Bk, t) + \frac{i\Bk}{c} \cdot \partial_t \BA(\Bk, t)
\Bigr)
e^{i \Bk \cdot \Bx } d^3 \Bk.
\end{equation}

Rearranging this also produces \autoref{eqn:ftMaxVacuum:73}.  We want to now substitute this relationship into \autoref{eqn:ftMaxVacuum:22b}.

Starting with just the \(\partial_0 \phi - \spacegrad \cdot \BA\) part we have

\begin{equation}\label{eqn:ftMaxVacuum:74}
\partial_0 \phi + \spacegrad \cdot \BA
=
\inv{(\sqrt{2 \pi})^3} \int
\Bigl(
\frac{i}{c^2 \Bk} \partial_{tt} \BA(\Bk, t) + i \Bk \cdot \BA
\Bigr)
e^{i \Bk \cdot \Bx } d^3 \Bk.
\end{equation}

Taking the gradient of this brings down a factor of \(i\Bk\) for

\begin{equation}\label{eqn:ftMaxVacuum:74b}
\spacegrad (\partial_0 \phi + \spacegrad \cdot \BA)
=
-\inv{(\sqrt{2 \pi})^3} \int
\Bigl(
\frac{1}{c^2} \partial_{tt} \BA(\Bk, t) + \Bk (\Bk \cdot \BA)
\Bigr)
e^{i \Bk \cdot \Bx } d^3 \Bk.
\end{equation}

\autoref{eqn:ftMaxVacuum:22b} in its entirety is now

\begin{equation}\label{eqn:ftMaxVacuum:75}
0 =
\inv{(\sqrt{2 \pi})^3} \int
\Bigl(
%\frac{2}{c^2} \partial_{tt} \BA(\Bk, t)
- (i\Bk)^2 \BA
+ \Bk (\Bk \cdot \BA)
\Bigr)
e^{i \Bk \cdot \Bx } d^3 \Bk.
\end{equation}

This is not terribly pleasant looking.  Perhaps going the other direction.  We could write

\begin{equation}\label{eqn:ftMaxVacuum:76}
\phi = \frac{i}{c \Bk} \PD{t}{\BA} = \frac{i}{c} \PD{t}{\psi},
\end{equation}

so that

\begin{equation}\label{eqn:ftMaxVacuum:77}
\BA(\Bk, t) = \Bk \psi(\Bk, t).
\end{equation}

\begin{equation}\label{eqn:ftMaxVacuum:642}
0
=
\inv{(\sqrt{2 \pi})^3} \int
\Bigl(
\inv{c^2} \Bk \psi_{tt}
- \spacegrad^2 \Bk \psi
+ \spacegrad \frac{i}{c^2} \psi_{tt}
+\spacegrad( \spacegrad \cdot (\Bk \psi) )
\Bigr)
e^{i \Bk \cdot \Bx } d^3 \Bk \\
\end{equation}

Note that the gradients here operate on everything to the right, including and especially the exponential.  Each application of the gradient brings down an additional \(i\Bk\) factor, and we have

\begin{equation}\label{eqn:ftMaxVacuum:662}
\inv{(\sqrt{2 \pi})^3} \int
\Bk \Bigl(
\inv{c^2} \psi_{tt}
- i^2 \Bk^2 \psi
+ \frac{i^2}{c^2} \psi_{tt}
+i^2 \Bk^2 \psi
\Bigr)
e^{i \Bk \cdot \Bx } d^3 \Bk.
\end{equation}

This is identically zero, so we see that this second equation provides no additional information.  That is somewhat surprising since there is not a whole lot of constraints supplied by the first equation.  The function \(\psi(\Bk, t)\) can be anything.  Understanding of this curiosity comes from computation of the Faraday bivector itself.  From \autoref{eqn:ftMaxVacuum:6}, that is

\begin{equation}\label{eqn:ftMaxVacuum:78}
F =
\inv{(\sqrt{2 \pi})^3} \int
\Bigl(
-i \Bk \frac{i}{c}\psi_t - \inv{c} \Bk \psi_t + i \Bk \wedge \Bk \psi
\Bigr)
e^{i \Bk \cdot \Bx } d^3 \Bk.
\end{equation}

All terms cancel, so we see that a non-zero \(\phi\) leads to \(F = 0\), as was the case when considering \autoref{eqn:ftMaxVacuum:40a} (a case that also resulted in \(\BA(\Bk) \propto \Bk\)).

Can this Fourier representation lead to a non-transverse solution to Maxwell's equation?  If so, it is not obvious how.

\section{The energy momentum tensor}

The energy momentum tensor is then

\begin{equation}\label{eqn:ftMaxVacuum:682}
\begin{aligned}
T(a) = -\frac{\epsilon_0}{2 (2 \pi)^3} \Real \iint &
\left(
- \inv{c} \conjugateStar{\dot{\BA}}(\Bk',t)
+ i \Bk' \conjugateStar{\phi}(\Bk', t)
- i \Bk' \wedge \conjugateStar{\BA}(\Bk', t)
\right) \\
&a
\left(
- \inv{c} \dot{\BA}(\Bk, t)
- i \Bk \phi(\Bk, t)
+ i \Bk \wedge \BA(\Bk, t)
\right)
e^{i (\Bk -\Bk') \cdot \Bx } d^3 \Bk d^3 \Bk'.
\end{aligned}
\end{equation}

Observing that \(\gamma_0\) commutes with spatial bivectors and anticommutes with spatial vectors, and writing \(\sigma_\mu = \gamma_\mu \gamma_0\), the tensor splits neatly into scalar and spatial vector components

%\begin{subequations}
%\label{eqn:ftMaxVacuum:16}
\begin{equation}\label{eqn:ftMaxVacuum:702}
\begin{aligned}
&T(\gamma_\mu) \cdot \gamma_0 = \frac{\epsilon_0}{2 (2 \pi)^3} \Real \iint
e^{i (\Bk -\Bk') \cdot \Bx } d^3 \Bk d^3 \Bk' \\
&\gpgradezero{
\left(
\inv{c} \conjugateStar{\dot{\BA}}(\Bk',t)
- i \Bk' \conjugateStar{\phi}(\Bk', t)
+ i \Bk' \wedge \conjugateStar{\BA}(\Bk', t)
\right)
\sigma_\mu
\left(
\inv{c} \dot{\BA}(\Bk, t)
+ i \Bk \phi(\Bk, t)
+ i \Bk \wedge \BA(\Bk, t)
\right)
} \\
&T(\gamma_\mu) \wedge \gamma_0 = \frac{\epsilon_0}{2 (2 \pi)^3} \Real \iint
e^{i (\Bk -\Bk') \cdot \Bx } d^3 \Bk d^3 \Bk' \\
&\gpgradeone{
\left(
\inv{c} \conjugateStar{\dot{\BA}}(\Bk',t)
- i \Bk' \conjugateStar{\phi}(\Bk', t)
+ i \Bk' \wedge \conjugateStar{\BA}(\Bk', t)
\right)
\sigma_\mu
\left(
\inv{c} \dot{\BA}(\Bk, t)
+ i \Bk \phi(\Bk, t)
+ i \Bk \wedge \BA(\Bk, t)
\right)
}.
\end{aligned}
\end{equation}
%\end{subequations}

In particular for \(\mu = 0\), we have

%\begin{subequations}
%\label{eqn:ftMaxVacuum:17}
\begin{equation}\label{eqn:ftMaxVacuum:722}
\begin{aligned}
H &\equiv
T(\gamma_0) \cdot \gamma_0 = \frac{\epsilon_0}{2 (2 \pi)^3} \Real \iint
e^{i (\Bk -\Bk') \cdot \Bx } d^3 \Bk d^3 \Bk' \\
&\left(
\left(
\inv{c} \conjugateStar{\dot{\BA}}(\Bk',t)
- i \Bk' \conjugateStar{\phi}(\Bk', t)
\right)
\cdot
\left(
\inv{c} \dot{\BA}(\Bk, t)
+ i \Bk \phi(\Bk, t)
\right)
- (\Bk' \wedge \conjugateStar{\BA}(\Bk', t)) \cdot (\Bk \wedge \BA(\Bk, t))
\right)
\\
\BP &\equiv
T(\gamma_\mu) \wedge \gamma_0 = \frac{\epsilon_0}{2 (2 \pi)^3} \Real \iint
e^{i (\Bk -\Bk') \cdot \Bx } d^3 \Bk d^3 \Bk' \\
&\left(
i
\left(
\inv{c} \conjugateStar{\dot{\BA}}(\Bk',t)
- i \Bk' \conjugateStar{\phi}(\Bk', t)
\right) \cdot
\left(
\Bk \wedge \BA(\Bk, t)
\right)
-i
\left(
\inv{c} \dot{\BA}(\Bk, t)
+ i \Bk \phi(\Bk, t)
\right)
\cdot
\left(
\Bk' \wedge \conjugateStar{\BA}(\Bk', t)
\right)
\right).
\end{aligned}
\end{equation}
%\end{subequations}

Integrating this over all space and identification of the delta function

\begin{equation}
\label{eqn:ftMaxVacuum:9}
\delta(\Bk) \equiv \inv{(2 \pi)^3} \int e^{i \Bk \cdot \Bx} d^3 \Bx,
\end{equation}

reduces the tensor to a single integral in the continuous angular wave number space of \(\Bk\).

\begin{equation}
\label{eqn:ftMaxVacuum:10}
\int T(a) d^3 \Bx = -\frac{\epsilon_0}{2} \Real \int
\left(
- \inv{c} \conjugateStar{\dot{\BA}}
+ i \Bk \conjugateStar{\phi}
- i \Bk \wedge \conjugateStar{\BA}
\right)
a
\left(
- \inv{c} \dot{\BA}
- i \Bk \phi
+ i \Bk \wedge \BA
\right)
d^3 \Bk.
\end{equation}

Or,

\begin{equation}
\label{eqn:ftMaxVacuum:12}
\int T(\gamma_\mu) \gamma_0 d^3 \Bx =
\frac{\epsilon_0}{2} \Real \int
\gpgrade{
\left(
\inv{c} \conjugateStar{\dot{\BA}}
- i \Bk \conjugateStar{\phi}
+ i \Bk \wedge \conjugateStar{\BA}
\right)
\sigma_\mu
\left(
\inv{c} \dot{\BA}
+ i \Bk \phi
+ i \Bk \wedge \BA
\right)
}{0,1}
d^3 \Bk.
\end{equation}

%%\begin{align}
%%\label{eqn:ftMaxVacuum:13b}
%%\PD{t}{H} + \spacegrad \cdot (c \BP) = 0.
%%\end{align}

Multiplying out \autoref{eqn:ftMaxVacuum:12} yields for \(\int H\)

\begin{equation}
\label{eqn:ftMaxVacuum:14}
\int H d^3 \Bx =
\frac{\epsilon_0}{2} \int d^3 \Bk \left(
\inv{c^2} \Abs{\dot{\BA}}^2 + \Bk^2 (\Abs{\phi}^2 + \Abs{\BA}^2 )
- \Abs{\Bk \cdot \BA}^2
+ 2 \frac{\Bk}{c} \cdot \Real( i \conjugateStar{\phi} \dot{\BA} )
\right)
\end{equation}

Recall that the only non-trivial solution we found for the assumed Fourier transform representation of \(F\) was for \(\phi = 0\), \(\Bk \cdot \BA(\Bk, t) = 0\).  Thus we have for the energy density integrated over all space, just

\begin{equation}
\label{eqn:ftMaxVacuum:14b}
\int H d^3 \Bx =
\frac{\epsilon_0}{2} \int d^3 \Bk \left(
\inv{c^2} \Abs{\dot{\BA}}^2 + \Bk^2 \Abs{\BA}^2
\right).
\end{equation}

Observe that we have the structure of a Harmonic oscillator for the energy of the radiation system.  What is the canonical momentum for this system?  Will it correspond to the Poynting vector, integrated over all space?

Let us reduce the vector component of \autoref{eqn:ftMaxVacuum:12}, after first imposing the \(\phi=0\), and \(\Bk \cdot \BA = 0\) conditions used to above for our harmonic oscillator form energy relationship.  This is

\begin{equation}\label{eqn:ftMaxVacuum:842}
\begin{aligned}
\int \BP d^3 \Bx
&=
\frac{\epsilon_0}{2 c} \Real
\int d^3 \Bk \left(
i \conjugateStar{\BA}_t \cdot (\Bk \wedge \BA)
+ i (\Bk \wedge \conjugateStar{\BA}) \cdot \BA_t
\right) \\
&=
\frac{\epsilon_0}{2 c} \Real
\int d^3 \Bk \left(
-i (\conjugateStar{\BA}_t \cdot \BA) \Bk
+ i \Bk (\conjugateStar{\BA} \cdot \BA_t)
\right)
\end{aligned}
\end{equation}

This is just

\begin{equation}
\label{eqn:ftMaxVacuum:15}
\int \BP d^3 \Bx
=
\frac{\epsilon_0}{c} \Real
i \int
\Bk (\conjugateStar{\BA} \cdot \BA_t) d^3 \Bk.
\end{equation}

Recall that the Fourier transforms for the transverse propagation case had the form \(\BA(\Bk, t) = \BA_{\pm}(\Bk) e^{\pm i c \Abs{\Bk} t}\), where the minus generated the advanced wave, and the plus the receding wave.  With substitution of the vector potential for the advanced wave into the energy and momentum results of \autoref{eqn:ftMaxVacuum:14b} and \autoref{eqn:ftMaxVacuum:15} respectively, we have

\begin{equation}\label{eqn:ftMaxVacuum:80}
\begin{aligned}
\int H d^3 \Bx   &= \epsilon_0 \int \Bk^2 \Abs{\BA(\Bk)}^2 d^3 \Bk \\
\int \BP d^3 \Bx &= \epsilon_0 \int \kcap \Bk^2 \Abs{\BA(\Bk)}^2 d^3 \Bk.
\end{aligned}
\end{equation}

After a somewhat circuitous route, this has the relativistic symmetry that is expected.  In particular the for the complete \(\mu=0\) tensor we have after integration over all space

\begin{equation}\label{eqn:ftMaxVacuum:81}
\int
T(\gamma_0) \gamma_0 d^3 \Bx = \epsilon_0 \int (1 + \kcap) \Bk^2 \Abs{\BA(\Bk)}^2 d^3 \Bk.
\end{equation}

The receding wave solution would give the same result, but directed as \(1 - \kcap\) instead.

Observe that we also have the four divergence conservation statement that is expected

\begin{equation}\label{eqn:ftMaxVacuum:82}
\PD{t}{} \int H d^3 \Bx + \spacegrad \cdot \int c \BP d^3 \Bx = 0.
\end{equation}

This follows trivially since both the derivatives are zero.  If the integration region was to be more specific instead of a \(0 + 0 = 0\) relationship, we would have the power flux \(\PDi{t}{H}\) equal in magnitude to the momentum change through a bounding surface.  For a more general surface the time and spatial dependencies should not necessarily vanish, but we should still have this radiation energy momentum conservation.

%%FIXME: Given an energy H and a set of generalized coordinates, how does one construct the canonical momenta without the Lagrangian.  How can one find the Lagrangian from the Hamiltonian, when the separate potential and kinetic terms are not known.  What is the potential and the Kinetic in this system (ie: \(H = E^2 + B^2\)).  Looks like there is a separation into velocity and non-velocity parts.  Using that could potentially produce a Lagrangian.

%\EndArticle
