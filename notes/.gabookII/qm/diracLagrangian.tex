%
% Copyright � 2012 Peeter Joot.  All Rights Reserved.
% Licenced as described in the file LICENSE under the root directory of this GIT repository.
%

%
%
\chapter{Dirac Lagrangian}
\index{Dirac Lagrangian}
\label{chap:diracLagrangian}
%\date{Dec 21, 2008.  diracLagrangian.tex}

\section{Dirac Lagrangian with Feynman slash notation}

Wikipedia's \href{http://en.wikipedia.org/wiki/Lagrangian#Dirac_Lagrangian}{Dirac Lagrangian} entry lists the Lagrangian as

\begin{equation}\label{eqn:diracLagrangian:24}
\begin{aligned}
\LL = \overbar{\psi} (i \Hbar c \Dslash - mc^2) \psi
\end{aligned}
\end{equation}

"where \(\psi\!\) is a Dirac spinor, \(\overbar{\psi} = \psi^\dagger \gamma^0\) is its Dirac adjoint, \(D\!\) is the gauge covariant derivative, and \(\Dslash\) is Feynman slash notation|Feynman notation for \(\gamma^\sigma D_\sigma\!\)."

Let us decode this.  First, what is \(D_\sigma\)?

From \href{http://en.wikipedia.org/wiki/Gauge_covariant_derivative}{Gauge theory}

\begin{equation}\label{eqn:diracLagrangian:44}
\begin{aligned}
D_\mu := \partial_\mu - i e A_\mu
\end{aligned}
\end{equation}

where \(A_\mu\) is the electromagnetic vector potential.

So, in four-vector notation we have

\begin{equation}\label{eqn:diracLagrangian:64}
\begin{aligned}
\Dslash
&= \gamma^\mu \partial_\mu - i e \gamma^\mu A_\mu \\
&= \grad - i e A \\
\end{aligned}
\end{equation}

So our Lagrangian written out in full is left as

\begin{equation}\label{eqn:diracLag:lag1}
\begin{aligned}
\LL = \psi^\dagger \gamma^0 ( i \Hbar c \grad + \Hbar c e A - mc^2) \psi
\end{aligned}
\end{equation}

How about this \(\gamma^0 i \grad\) term?  If we assume that \(i = \gamma_0 \gamma_1 \gamma_2 \gamma_3\) is the four space pseudoscalar, then this is

\begin{equation}\label{eqn:diracLagrangian:84}
\begin{aligned}
\gamma^0 i \grad
&= - i \gamma^0 (\gamma^0 \partial_0 + \gamma^i \partial_i) \\
&= - i (\partial_0 + \sigma_i \partial_i) \\
\end{aligned}
\end{equation}

So, operationally, we have the dual of a quaternion like gradient operator.  If \(\psi\) is an even grade object, as I had guess can be implied by
the term spinor, then there is some sense to requiring a gradient operation that has scalar and spacetime bivector components.

Let us write this

\begin{equation}\label{eqn:diracLagrangian:104}
\begin{aligned}
\gamma^0 \grad &= \partial_0 + \sigma_i \partial_i = {\grad}_{0,2}
\end{aligned}
\end{equation}

Now, how about the meaning of \(\overbar{\psi} = \psi^\dagger \gamma^0\)?  I initially assumed that \(\psi^\dagger\) was the reverse operation.
However, looking in the quantum treatment of \citep{doran2003gap} and their earlier relativity content, I see that they explicitly avoid dagger as a reverse in a relativistic context since it is used for ``something-else'' in a quantum context.  It appears that their mapping from matrix algebra to Clifford
algebra is

\begin{equation}\label{eqn:diracLagrangian:124}
\begin{aligned}
\psi^\dagger \equiv \gamma_0 \tilde{\psi} \gamma_0,
\end{aligned}
\end{equation}

where tilde is used for the reverse operation.

This then implies that

\begin{equation}\label{eqn:diracLagrangian:144}
\begin{aligned}
\overbar{\psi} = \psi^\dagger \gamma^0 = \gamma_0 \tilde{\psi}
\end{aligned}
\end{equation}

We now have an expression of the Lagrangian in full in terms of geometric objects

\begin{equation}\label{eqn:diracLag:lag2}
\begin{aligned}
\LL = \gamma_0 \tilde{\psi} ( i \Hbar c \grad + \Hbar c e A - mc^2) \psi.
\end{aligned}
\end{equation}

Assuming that this is now the correct geometric interpretation of the Lagrangian, why bother having that first \(\gamma_0\) factor.  It should not change the field equations (just as a constant factor should not make a difference).  It seems more natural to instead write the Lagrangian as just

\begin{equation}\label{eqn:diracLag:lag3}
\begin{aligned}
\LL = \tilde{\psi} \left( i \grad + e A - \frac{mc}{\Hbar} \right) \psi,
\end{aligned}
\end{equation}

where both the constant vector factor \(\gamma_0\), the redundant common factor of \(c\) have been removed, and we divide throughout by \(\Hbar\) to tidy up a bit.  Perhaps this tidy up should be omitted since it sacrifices the
energy dimensionality of the original.

\subsection{Dirac adjoint field}

%Now, it is somewhat interesting to note that treating \(\psi^\dagger\) as a reverse operation resulted, after calculation of the field equations, in the quantity \(\gamma_0 \psi^\dagger \gamma_0\) showing up as a significant quanity.  It was almost like the math was auto-correcting itself attempting to show that the this was the real quantity of interest.

The reverse sandwich operation of \(\gamma_0 \tilde{\psi} \gamma_0\) to produce the Dirac adjoint field from \(\psi\)
can be recognized
as very similar to the mechanism used to split the Faraday bivector for the electromagnetic field into electric and magnetic terms.  There addition and subtraction of the sandwich'ed fields with the original acted as a spacetime split operation, producing separate electric field spacetime (Pauli) bivectors and pure spatial bivectors (magnetic components) from the total field.  Here we have a
quaternion like field variable with scalar and bivector terms.  Is there a physical (observables) significance
only for a subset of the six possible bivectors that make up the spinor field?
%only in the
%spacetime bivector parts (ie: Pauli matrix components) of the six possible bivectors,
If so, then this adjoint operation can be used as a filter to select only the desired components.

%Observe that we have an almost recognizable term in the effective field variable \(\gamma_0 \psi \gamma_0\) of
%\eqnref{eqn:diracLag:antimatter}.  In particular recall that the electric field and magnetic field components of the faraday bivector can be obtained from a
%\(\gamma^0 F \gamma_0\) spacetime split.
Recall that the Faraday bivector is

\begin{equation}\label{eqn:diracLagrangian:164}
\begin{aligned}
F
&= \BE + i c \BB \\
&= E^j \sigma_j + i c B^j \sigma_j \\
&= E^j \gamma_j \gamma_0 + i c B^j \gamma_j \gamma_0 \\
\end{aligned}
\end{equation}

So we have

\begin{equation}\label{eqn:diracLagrangian:184}
\begin{aligned}
\gamma_0 F \gamma_0
&= E^j \gamma_0 \gamma_j + \gamma_0 i c B^j \gamma_j \\
&= -E^j \sigma_j + i c B^j \sigma_j \\
&= -\BE + i c \BB
\end{aligned}
\end{equation}

So we have

\begin{equation}\label{eqn:diracLagrangian:204}
\begin{aligned}
\inv{2} \left(F - \gamma_0 F \gamma_0 \right) &= \BE \\
\inv{2i} \left(F + \gamma_0 F \gamma_0 \right) &= c \BB
\end{aligned}
\end{equation}

How does this sandwich operation act on other grade objects?

\begin{itemize}
\item scalar

\begin{equation}\label{eqn:diracLagrangian:224}
\begin{aligned}
\gamma_0 \alpha \gamma_0 = \alpha
\end{aligned}
\end{equation}

\item vector

\begin{equation}\label{eqn:diracLagrangian:244}
\begin{aligned}
\gamma_0 \gamma_\mu \gamma_0
&= \left(2 \gamma_0 \cdot \gamma_\mu - \gamma_\mu \gamma_0\right) \gamma_0 \\
&= 2 (\gamma_0 \cdot \gamma_\mu) \gamma_0 - \gamma_\mu \\
&=
\left\{
\begin{array}{l l}
\gamma_0 & \quad \mbox{if \(\mu = 0\)} \\
-\gamma_i & \quad \mbox{if \(\mu = i \ne 0\)} \\
\end{array} \right.
\end{aligned}
\end{equation}

\item trivector

For the duals of the vectors we have the opposite split, where for the dual of \(\gamma_0\) we have a sign
toggle

\begin{equation}\label{eqn:diracLagrangian:264}
\begin{aligned}
\gamma_0 \gamma_i \gamma_j \gamma_k \gamma_0 = -\gamma_i \gamma_j \gamma_k
\end{aligned}
\end{equation}

whereas for the duals of \(\gamma_k\) we have invariant sign under sandwich
\begin{equation}\label{eqn:diracLagrangian:284}
\begin{aligned}
\gamma_0 \gamma_i \gamma_j \gamma_0 \gamma_0 = \gamma_i \gamma_j \gamma_0
\end{aligned}
\end{equation}

\item pseudoscalar

\begin{equation}\label{eqn:diracLagrangian:304}
\begin{aligned}
\gamma_0 i \gamma_0
&= \gamma_0 \gamma_0 \gamma_1 \gamma_2 \gamma_3 \gamma_0 \\
&= -i
\end{aligned}
\end{equation}
\end{itemize}

Ah ha!  Recalling the conjugation results from \chapcite{PJDiracGamma}, one sees that this sandwich operation is in fact just the equivalent of the conjugate operation on Dirac matrix algebra elements.  So we can write

\begin{equation}\label{eqn:diracLagrangian:324}
\begin{aligned}
\psi^\conj \equiv \gamma_0 \psi \gamma_0
\end{aligned}
\end{equation}

and can thus identify \(\gamma_0 \tilde{\psi} \gamma_0 = \psi^\dagger\) as the reverse of that conjugate quantity.  That is

\begin{equation}\label{eqn:diracLagrangian:344}
\begin{aligned}
\psi^\dagger = (\psi^\conj)^{\tilde{}}
\end{aligned}
\end{equation}

This does not really help identify the significance of this term but this identification may prove useful later.

\subsection{Field equations}

Now, how to recover the field equation from \eqnref{eqn:diracLag:lag3}
%or \eqnref{eqn:diracLag:lag2}
?  If one assumes that the Euler-Lagrange field equations

\begin{equation}\label{eqn:diracLagrangian:364}
\begin{aligned}
\PD{\eta}{\LL} - \partial_\mu \PD{(\partial_\mu \eta)}{\LL} = 0
\end{aligned}
\end{equation}

hold for these even grade field variables \(\psi\), then treating \(\psi\) and \(\overbar{\psi}\) as separate field variables one has for the reversed field variable

\begin{equation}\label{eqn:diracLagrangian:384}
\begin{aligned}
\PD{\tilde{\psi}}{\LL} - \partial_\mu \PD{(\partial_\mu \tilde{\psi})}{\LL} &= 0 \\
\left( i \grad + e A - \frac{mc}{\Hbar}\right) \psi - (0) &= 0
\end{aligned}
\end{equation}

Or
\begin{equation}\label{eqn:diracLagrangian:404}
\begin{aligned}
\Hbar (i \grad + e A) \psi = mc \psi
\end{aligned}
\end{equation}

Except for the additional \(e A\) term here, this is the Dirac equation that we get from taking square roots of the Klein-Gordon equation.  Should \(A\) be considered a field variable?  More likely is that this is a supplied potential as in the \(V\) of the non-relativistic
Schr\"{o}dinger's equation.

Being so loose with the math here (ie: taking partials with respect to non-scalar variables) is somewhat disturbing but developing some intuition is worthwhile before getting the details down.

\subsection{Conjugate field equation}

Our Lagrangian is not at all symmetric looking, having derivatives of \(\psi\), but not \(\overbar{\psi}\).  Compare this to the Lagrangians for the
Schr\"{o}dinger's equation, and Klein-Gordon equation respectively, which are

\begin{equation}\label{eqn:diracLagrangian:424}
\begin{aligned}
\LL &= \frac{\Hbar^2}{2m}
(\spacegrad \psi) \cdot (\spacegrad \psi^\conj) + V \psi \psi^\conj + {i \Hbar} \left( \psi \partial_t \psi^\conj - \psi^\conj \partial_t \psi \right) \\
\LL &= -\partial^\nu \psi \partial_\nu \psi^\conj + \frac{m^2 c^2}{\Hbar^2} \psi \psi^\conj.
\end{aligned}
\end{equation}

With these Lagrangians one gets the field equation for \(\psi\), differentiating with respect to the conjugate field \(\psi^\conj\), and the conjugate equation with differentiation with respect to \(\psi\) (where \(\psi\) and \(\psi^\conj\) are treated as independent field variables).

It is not obvious that evaluating the Euler-Lagrange equations will produce a similarly regular result, so
%we would get a similarly nice symmetric reversal result with respect to \(\psi\) and \(\tilde{\psi}\).
let us compute the derivatives with respect to the \(\psi\) field variables to compute the equations for \(\overbar{\psi}\) or \(\tilde{\psi}\) to see what results.  Written out in coordinates so that we can apply the Euler-Lagrange equations, our Lagrangian (with \(A\) terms omitted) is

\begin{equation}\label{eqn:diracLag:lag4}
\begin{aligned}
\LL = \tilde{\psi}\left( i \gamma^\mu \partial_\mu + e A - \frac{m c}{\Hbar} \right) \psi
\end{aligned}
\end{equation}

Again abusing the Euler Lagrange equations, ignoring the possible issues with commuting partials taken
with respect to spinors (not scalar variables), blinding plugging into the formulas we have

\begin{equation}\label{eqn:diracLagrangian:444}
\begin{aligned}
\PD{\psi}{\LL} &= \partial_\mu \PD{\partial_\mu \psi}{\LL} \\
\tilde{\psi}\left( e A - \frac{m c}{\Hbar} \right) &= \partial_\mu \left(\tilde{\psi} i \gamma^\mu \right)
\end{aligned}
\end{equation}

reversing this entire equation we have

\begin{equation}\label{eqn:diracLagrangian:464}
\begin{aligned}
\left( e A - \frac{m c}{\Hbar} \right) \psi &= \gamma^\mu i \partial_\mu \psi = - i \grad \psi
\end{aligned}
\end{equation}

Or
\begin{equation}\label{eqn:diracLagrangian:484}
\begin{aligned}
\Hbar \left( i \grad + e A \right) \psi = m c \psi
\end{aligned}
\end{equation}

So we do in fact get the same field equation regardless of which of the two field variables one differentiates with.  That is not obvious looking at the Lagrangian.

\section{Alternate Dirac Lagrangian with antisymmetric terms}

Now, the wikipedia article
\href{http://en.wikipedia.org/wiki/Dirac_equation#Adjoint_equation_and_Dirac_current}{Adjoint equation and Dirac current} lists the Lagrangian as

\begin{equation}\label{eqn:diracLagrangian:504}
\begin{aligned}
\LL = mc \overbar{\psi}\psi - {\inv{2}i\Hbar}(\overbar{\psi}\gamma^\mu (\partial_\mu\psi) - (\partial_\mu\overbar{\psi})\gamma^\mu \psi)
\end{aligned}
\end{equation}

Computing the Euler Lagrange equations for this potential free Lagrangian we have

\begin{equation}\label{eqn:diracLagrangian:524}
\begin{aligned}
m c \psi - \inv{2} i \Hbar \gamma^\mu \partial_\mu \psi = \partial_\mu \left( \inv{2} i \Hbar \gamma^\mu \psi \right)
\end{aligned}
\end{equation}

Or,
\begin{equation}\label{eqn:diracLagrangian:544}
\begin{aligned}
m c \psi = i \Hbar \grad \psi
\end{aligned}
\end{equation}

And the same computation, treating \(\overbar{\psi}\) as the independent field variable of interest we have

\begin{equation}\label{eqn:diracLagrangian:564}
\begin{aligned}
m c \psi + \inv{2} i \Hbar \partial_\mu \overbar{\psi} \gamma^\mu = -\inv{2} i \Hbar \partial_\mu \overbar{\psi} \gamma^\mu
\end{aligned}
\end{equation}

which is

\begin{equation}\label{eqn:diracLagrangian:584}
\begin{aligned}
m c \overbar{\psi} &= - i \Hbar \partial_\mu \overbar{\psi} \gamma^\mu \\
m c \gamma_0 \tilde{\psi} &= - i \Hbar \partial_\mu \gamma_0 \tilde{\psi} \gamma^\mu \\
m c \tilde{\psi} &= i \Hbar \partial_\mu \tilde{\psi} \gamma^\mu \\
m c \psi &= \Hbar \grad \psi i \\
\end{aligned}
\end{equation}

Or,
\begin{equation}\label{eqn:diracLagrangian:604}
\begin{aligned}
i \Hbar \grad \psi &= -m c \psi
\end{aligned}
\end{equation}

FIXME: This differs in sign from the same calculation with the Lagrangian of \eqnref{eqn:diracLag:lag4}.  Based on
the possibility of both roots in the Klein-Gordon equation, I suspect I have made a sign error in the first
calculation.

\section{Appendix}

\subsection{Pseudoscalar reversal}
\index{pseudoscalar}

The pseudoscalar reverses to itself

\begin{equation}\label{eqn:diracLagrangian:624}
\begin{aligned}
\tilde{i}
&= \gamma_{3210} \\
&= -\gamma_{2103} \\
&= -\gamma_{1023} \\
&= \gamma_{0123} \\
&= i,
\end{aligned}
\end{equation}

\subsection{Form of the spinor}

The specific structure of the spinor has not been defined here.  It has been assumed to be quaternion like,
and contain only even grades, but in the Dirac/Minkowski algebra that gives us two possibilities

\begin{equation}\label{eqn:diracLagrangian:644}
\begin{aligned}
\psi
&= \alpha + P^a \gamma_a \gamma_0 \\
&= \alpha + P^a \sigma_a
\end{aligned}
\end{equation}

Or
\begin{equation}\label{eqn:diracLagrangian:664}
\begin{aligned}
\psi
&= \alpha + P^c \gamma_a \wedge \gamma_b \\
&= \alpha - P^c \sigma_a \wedge \sigma_b \\
&= \alpha - i \epsilon_{a b c} P^c \sigma_c \\
\end{aligned}
\end{equation}

Spinors in Doran/Lasenby appear to use the latter form of dual Pauli vectors (wedge products of the Pauli spatial basis elements).  This actually makes sense since one wants a spatial bivector for rotation (ie: ``spin''), and not the spacetime bivectors, which provide a Lorentz boost action.
