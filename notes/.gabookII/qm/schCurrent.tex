%
% Copyright � 2012 Peeter Joot.  All Rights Reserved.
% Licenced as described in the file LICENSE under the root directory of this GIT repository.
%

%
%
\chapter{Schr\"{o}dinger equation probability conservation}
\index{Schr\"{o}dinger equation!probability conservation}
\label{chap:schCurrent}
%\date{Jan 11, 2009.  schCurrent.tex}
\section{Motivation}

In \citep{mcmahon2005qmd} is a one dimensional probability conservation
derivation from
Schr\"{o}dinger's equation.  Do this for the three dimensional case.

\section{}

Consider the time rate of change of the probability as expressed
in terms of the wave function

\begin{equation}\label{eqn:schCurrent:20}
\begin{aligned}
\PD{t}{\rho}
&= \PD{t}{\psi^\conj \psi} \\
&= \PD{t}{\psi^\conj} \psi + \psi^\conj \PD{t}{\psi} \\
\end{aligned}
\end{equation}

This can be calculated from Schr\"{o}dinger's equation and its complex
conjugate

\begin{equation}\label{eqn:schCurrent:40}
\begin{aligned}
\partial_t \psi &= \left(-\frac{\Hbar}{2mi}\spacegrad^2 + \inv{i\Hbar} V \right) \psi \\
\partial_t \psi^\conj &= \left(\frac{\Hbar}{2mi}\spacegrad^2 - \inv{i\Hbar} V \right) \psi^\conj \\
\end{aligned}
\end{equation}

Multiplying by the conjugate wave functions and adding we have
\begin{equation}\label{eqn:schCurrent:60}
\begin{aligned}
\PD{t}{\rho}
&=
\psi^\conj \left(-\frac{\Hbar}{2mi}\spacegrad^2 + \inv{i\Hbar} V \right) \psi +
\psi \left(\frac{\Hbar}{2mi}\spacegrad^2 - \inv{i\Hbar} V \right) \psi^\conj \\
&=
\frac{\Hbar}{2mi} \left(
-\psi^\conj \spacegrad^2 \psi + \psi \spacegrad^2 \psi^\conj \right) \\
\end{aligned}
\end{equation}

So we have the following conservation law
\begin{equation}\label{eqn:sch_current:intermediate}
\begin{aligned}
\PD{t}{\rho} + \frac{\Hbar}{2mi} \left( \psi^\conj \spacegrad^2 \psi - \psi \spacegrad^2 \psi^\conj \right) = 0
\end{aligned}
\end{equation}

The text indicates that the second order terms here can be written as a divergence.  Somewhat loosely, by treating \(\psi\) as a scalar field one can show that this is the case

\begin{equation}\label{eqn:schCurrent:80}
\begin{aligned}
\spacegrad \cdot \left( \psi^\conj \spacegrad \psi - \psi \spacegrad \psi^\conj \right)
&=
\gpgradezero{
\spacegrad \left( \psi^\conj \spacegrad \psi - \psi \spacegrad \psi^\conj \right)
} \\
&=
\gpgradezero{
(\spacegrad \psi^\conj) (\spacegrad \psi) - (\spacegrad \psi) (\spacegrad \psi^\conj)
+\psi^\conj \spacegrad^2 \psi - \psi \spacegrad^2 \psi^\conj
} \\
&=
\gpgradezero{
2 (\spacegrad \psi^\conj) \wedge (\spacegrad \psi)
+\psi^\conj \spacegrad^2 \psi - \psi \spacegrad^2 \psi^\conj
} \\
&=
\psi^\conj \spacegrad^2 \psi - \psi \spacegrad^2 \psi^\conj
 \\
\end{aligned}
\end{equation}

Assuming that this procedure is justified.
\Eqnref{eqn:sch_current:intermediate} therefore can be written
in terms of a probability current very reminiscent of the current density vector of electrodynamics

\begin{equation}\label{eqn:sch_current:pcons}
\begin{aligned}
\BJ &= \frac{\Hbar}{2mi} \left( \psi^\conj \spacegrad \psi - \psi \spacegrad \psi^\conj \right) \\
0 &= \PD{t}{\rho} + \spacegrad \cdot \BJ
\end{aligned}
\end{equation}

Regarding justification, this should be revisited.
It appears to give the right answer, despite the fact that \(\psi\) is a complex (mixed grade) object, which
likely has some additional significance.

\section{}

Now, having calculated the probability conservation \eqnref{eqn:sch_current:pcons}, it is interesting to
note the similarity to the relativistic spacetime divergence from Maxwell's equation.

We can write
\begin{equation}\label{eqn:schCurrent:100}
\begin{aligned}
0 = \PD{t}{\rho} + \spacegrad \cdot \BJ &= \grad \cdot \left( c\rho \gamma_0 + \BJ \gamma_0 \right)
\end{aligned}
\end{equation}

and form something that has the appearance of a relativistic four vector, re-writing the conservation equation as

\begin{equation}\label{eqn:schCurrent:120}
\begin{aligned}
J &= c\rho \gamma_0 + \BJ \gamma_0 \\
0 &= \grad \cdot J
\end{aligned}
\end{equation}

Expanding this four component vector shows an interesting form:

\begin{equation}\label{eqn:schCurrent:140}
\begin{aligned}
J &= c \rho \gamma_0 +
\frac{\Hbar}{2mi} \left( \psi^\conj \spacegrad \psi - \psi \spacegrad \psi^\conj \right) \gamma_0 \\
\end{aligned}
\end{equation}

Now, if one assumes the wave function can be represented as a even grade object with the following complex
structure
\begin{equation}\label{eqn:schCurrent:160}
\begin{aligned}
\psi &= \alpha + \gamma^m \wedge \gamma^n \beta_{mn}
\end{aligned}
\end{equation}

then \(\gamma_0\) will commute with \(\psi\).  Noting that \(\spacegrad \gamma_0 = \sum_k \gamma_k \partial_k = -\gamma^k \partial_k\), we have

\begin{equation}\label{eqn:schCurrent:180}
\begin{aligned}
m J &= m c \psi^\conj \psi \gamma_0 +
\frac{i\Hbar}{2} \left( \psi^\conj \gamma^k \partial_k \psi - \psi \gamma^k \partial_k \psi^\conj \right)
\end{aligned}
\end{equation}

Now, this is an interesting form.  In particular compare this to the Dirac Lagrangian, as given in
the \href{http://en.wikipedia.org/wiki/Dirac_equation#Adjoint_equation_and_Dirac_current}{wikipedia Dirac equation} article.

\begin{equation}\label{eqn:schCurrent:200}
\begin{aligned}
L = mc \overbar{\psi}\psi - \frac{i\Hbar}{2}(\overbar{\psi}\gamma^\mu (\partial_\mu\psi) - (\partial_\mu\overbar{\psi})\gamma^\mu \psi)
\end{aligned}
\end{equation}

Although the Schr\"{o}dinger equation is a non-relativistic equation, it appears that the probability current,
when we add the \(\gamma^0 \partial_0\) term required to put this into a covariant form, is in fact the Lagrangian density
for the Dirac equation (when scaled by mass).

I do not know enough yet about QM to see what exactly the implications of this are, but I suspect that there is something
of some interesting significance to this particular observation.

\section{On the grades of the QM complex numbers}

To get to \eqnref{eqn:sch_current:intermediate}, no assumptions about the representation of the field variable \(\psi\) were
required.  However, to make the identification

\begin{equation}\label{eqn:schCurrent:220}
\begin{aligned}
\psi^\conj \spacegrad^2 \psi - \psi \spacegrad^2 \psi^\conj
&= \spacegrad \cdot \left( \psi^\conj \spacegrad^2 \psi - \psi \spacegrad^2 \psi^\conj \right)
\end{aligned}
\end{equation}

we need some knowledge or assumptions about the representation.  The assumption made initially was that we could treat
\(\psi\) as a scalar, but then we later see there is value trying to switch to the Dirac representation (which appears
to be the logical way to relativistically extend the probability current).

For example, with a geometric algebra multivector representation we have many ways to construct complex quantities.  Assuming a
Euclidean basis we can construct a complex number we can factor out one of the basis vectors

\begin{equation}\label{eqn:schCurrent:240}
\begin{aligned}
\sigma_1 x_1 + \sigma_2 x_2 = \sigma_1 ( x_1 + \sigma_1 \sigma_2 x_2 )
\end{aligned}
\end{equation}

However, this is not going to commute with vectors (ie: such as the gradient), unless that vector is perpendicular to the
plane spanned by this vector.  As an example

\begin{equation}\label{eqn:schCurrent:260}
\begin{aligned}
i = \sigma_1 \sigma_2
\end{aligned}
\end{equation}

\begin{equation}\label{eqn:schCurrent:280}
\begin{aligned}
i \sigma_1 &= -\sigma_1 i \\
i \sigma_2 &= -\sigma_2 i \\
i \sigma_3 &=  \sigma_3 i
\end{aligned}
\end{equation}

What would work is a complex representation using the \R{3} pseudoscalar (aka the Dirac pseudoscalar).

\begin{equation}\label{eqn:schCurrent:300}
\begin{aligned}
\psi = \alpha + \sigma_1 \sigma_2 \sigma_3 \beta = \alpha + \gamma_0 \gamma_1 \gamma_2 \gamma_3 \beta
\end{aligned}
\end{equation}
