%
% Copyright � 2012 Peeter Joot.  All Rights Reserved.
% Licenced as described in the file LICENSE under the root directory of this GIT repository.
%

%
%
%\documentclass{article}

%\usepackage{amsmath}
\usepackage{mathpazo}

%
% shorthand for bold symbols, convenient for vectors and matrices
%
\newcommand{\Ba}[0]{\mathbf{a}}
\newcommand{\Bb}[0]{\mathbf{b}}
\newcommand{\Bc}[0]{\mathbf{c}}
\newcommand{\Bd}[0]{\mathbf{d}}
\newcommand{\Be}[0]{\mathbf{e}}
\newcommand{\Bf}[0]{\mathbf{f}}
\newcommand{\Bg}[0]{\mathbf{g}}
\newcommand{\Bh}[0]{\mathbf{h}}
\newcommand{\Bi}[0]{\mathbf{i}}
\newcommand{\Bj}[0]{\mathbf{j}}
\newcommand{\Bk}[0]{\mathbf{k}}
\newcommand{\Bl}[0]{\mathbf{l}}
\newcommand{\Bm}[0]{\mathbf{m}}
\newcommand{\Bn}[0]{\mathbf{n}}
\newcommand{\Bo}[0]{\mathbf{o}}
\newcommand{\Bp}[0]{\mathbf{p}}
\newcommand{\Bq}[0]{\mathbf{q}}
\newcommand{\Br}[0]{\mathbf{r}}
\newcommand{\Bs}[0]{\mathbf{s}}
\newcommand{\Bt}[0]{\mathbf{t}}
\newcommand{\Bu}[0]{\mathbf{u}}
\newcommand{\Bv}[0]{\mathbf{v}}
\newcommand{\Bw}[0]{\mathbf{w}}
\newcommand{\Bx}[0]{\mathbf{x}}
\newcommand{\By}[0]{\mathbf{y}}
\newcommand{\Bz}[0]{\mathbf{z}}
\newcommand{\BA}[0]{\mathbf{A}}
\newcommand{\BB}[0]{\mathbf{B}}
\newcommand{\BC}[0]{\mathbf{C}}
\newcommand{\BD}[0]{\mathbf{D}}
\newcommand{\BE}[0]{\mathbf{E}}
\newcommand{\BF}[0]{\mathbf{F}}
\newcommand{\BG}[0]{\mathbf{G}}
\newcommand{\BH}[0]{\mathbf{H}}
\newcommand{\BI}[0]{\mathbf{I}}
\newcommand{\BJ}[0]{\mathbf{J}}
\newcommand{\BK}[0]{\mathbf{K}}
\newcommand{\BL}[0]{\mathbf{L}}
\newcommand{\BM}[0]{\mathbf{M}}
\newcommand{\BN}[0]{\mathbf{N}}
\newcommand{\BO}[0]{\mathbf{O}}
\newcommand{\BP}[0]{\mathbf{P}}
\newcommand{\BQ}[0]{\mathbf{Q}}
\newcommand{\BR}[0]{\mathbf{R}}
\newcommand{\BS}[0]{\mathbf{S}}
\newcommand{\BT}[0]{\mathbf{T}}
\newcommand{\BU}[0]{\mathbf{U}}
\newcommand{\BV}[0]{\mathbf{V}}
\newcommand{\BW}[0]{\mathbf{W}}
\newcommand{\BX}[0]{\mathbf{X}}
\newcommand{\BY}[0]{\mathbf{Y}}
\newcommand{\BZ}[0]{\mathbf{Z}}

\newcommand{\Bzero}[0]{\mathbf{0}}
\newcommand{\Btheta}[0]{\boldsymbol{\theta}}
\newcommand{\Btau}[0]{\boldsymbol{\tau}}
\newcommand{\Bomega}[0]{\boldsymbol{\omega}}

%
% shorthand for unit vectors
%
\newcommand{\acap}[0]{\hat{\Ba}}
\newcommand{\bcap}[0]{\hat{\Bb}}
\newcommand{\ccap}[0]{\hat{\Bc}}
\newcommand{\dcap}[0]{\hat{\Bd}}
\newcommand{\ecap}[0]{\hat{\Be}}
\newcommand{\fcap}[0]{\hat{\Bf}}
\newcommand{\gcap}[0]{\hat{\Bg}}
\newcommand{\hcap}[0]{\hat{\Bh}}
\newcommand{\icap}[0]{\hat{\Bi}}
\newcommand{\jcap}[0]{\hat{\Bj}}
\newcommand{\kcap}[0]{\hat{\Bk}}
\newcommand{\lcap}[0]{\hat{\Bl}}
\newcommand{\mcap}[0]{\hat{\Bm}}
\newcommand{\ncap}[0]{\hat{\Bn}}
\newcommand{\ocap}[0]{\hat{\Bo}}
\newcommand{\pcap}[0]{\hat{\Bp}}
\newcommand{\qcap}[0]{\hat{\Bq}}
\newcommand{\rcap}[0]{\hat{\Br}}
\newcommand{\scap}[0]{\hat{\Bs}}
\newcommand{\tcap}[0]{\hat{\Bt}}
\newcommand{\ucap}[0]{\hat{\Bu}}
\newcommand{\vcap}[0]{\hat{\Bv}}
\newcommand{\wcap}[0]{\hat{\Bw}}
\newcommand{\xcap}[0]{\hat{\Bx}}
\newcommand{\ycap}[0]{\hat{\By}}
\newcommand{\zcap}[0]{\hat{\Bz}}
\newcommand{\thetacap}[0]{\hat{\Btheta}}

%
% to write R^n and C^n in a distinguishable fashion.  Perhaps change this
% to the double lined characters upon figuring out how to do so.
%
\newcommand{\C}[1]{$\mathbb{C}^{#1}$}
\newcommand{\R}[1]{$\mathbb{R}^{#1}$}

%
% various generally useful helpers
%

% derivative of #1 wrt. #2:
\newcommand{\D}[2] {\frac {d#2} {d#1}}

\newcommand{\inv}[1]{\frac{1}{#1}}
\newcommand{\cross}[0]{\times}

\newcommand{\abs}[1]{\lvert{#1}\rvert}
\newcommand{\norm}[1]{\lVert{#1}\rVert}
\newcommand{\innerprod}[2]{\langle{#1}, {#2}\rangle}
\newcommand{\dotprod}[2]{{#1} \cdot {#2}}
\newcommand{\bdotprod}[2]{\left({#1} \cdot {#2}\right)}
\newcommand{\crossprod}[2]{{#1} \cross {#2}}
\newcommand{\tripleprod}[3]{\dotprod{\left(\crossprod{#1}{#2}\right)}{#3}}

\DeclareMathOperator{\Proj}{Proj}
\DeclareMathOperator{\Span}{span}
\DeclareMathOperator{\Sgn}{sgn}
\DeclareMathOperator{\Area}{Area}
\DeclareMathOperator{\Volume}{Volume}

%
% A few miscellaneous things specific to this document
%
\newcommand{\crossop}[1]{\crossprod{#1}{}}

% R2 vector.
\newcommand{\VectorTwo}[2]{
\begin{bmatrix}
 {#1} \\
 {#2}
\end{bmatrix}
}

\newcommand{\VectorN}[1]{
\begin{bmatrix}
{#1}_1 \\
{#1}_2 \\
\vdots \\
{#1}_N \\
\end{bmatrix}
}

\newcommand{\DETuvij}[4]{
\begin{vmatrix}
 {#1}_{#3} & {#1}_{#4} \\
 {#2}_{#3} & {#2}_{#4}
\end{vmatrix}
}

\newcommand{\DETuvwijk}[6]{
\begin{vmatrix}
 {#1}_{#4} & {#1}_{#5} & {#1}_{#6} \\
 {#2}_{#4} & {#2}_{#5} & {#2}_{#6} \\
 {#3}_{#4} & {#3}_{#5} & {#3}_{#6}
\end{vmatrix}
}

\newcommand{\DETuvwxijkl}[8]{
\begin{vmatrix}
 {#1}_{#5} & {#1}_{#6} & {#1}_{#7} & {#1}_{#8} \\
 {#2}_{#5} & {#2}_{#6} & {#2}_{#7} & {#2}_{#8} \\
 {#3}_{#5} & {#3}_{#6} & {#3}_{#7} & {#3}_{#8} \\
 {#4}_{#5} & {#4}_{#6} & {#4}_{#7} & {#4}_{#8} \\
\end{vmatrix}
}

%\newcommand{\DETuvwxyijklm}[10]{
%\begin{vmatrix}
% {#1}_{#6} & {#1}_{#7} & {#1}_{#8} & {#1}_{#9} & {#1}_{#10} \\
% {#2}_{#6} & {#2}_{#7} & {#2}_{#8} & {#2}_{#9} & {#2}_{#10} \\
% {#3}_{#6} & {#3}_{#7} & {#3}_{#8} & {#3}_{#9} & {#3}_{#10} \\
% {#4}_{#6} & {#4}_{#7} & {#4}_{#8} & {#4}_{#9} & {#4}_{#10} \\
% {#5}_{#6} & {#5}_{#7} & {#5}_{#8} & {#5}_{#9} & {#5}_{#10}
%\end{vmatrix}
%}

% R3 vector.
\newcommand{\VectorThree}[3]{
\begin{bmatrix}
 {#1} \\
 {#2} \\
 {#3}
\end{bmatrix}
}


%%<misc>
%
\newcommand{\Abs}[1]{{\left\lvert{#1}\right\rvert}}
\newcommand{\spacegrad}[0]{\boldsymbol{\nabla}}
\newcommand{\grad}[0]{\nabla}
\newcommand{\LL}[0]{\mathcal{L}}

% == \partial_{#1} {#2}
\newcommand{\PD}[2]{\frac{\partial {#2}}{\partial {#1}}}
% inline variant
\newcommand{\PDi}[2]{{\partial {#2}}/{\partial {#1}}}

\newcommand{\PDD}[3]{\frac{\partial^2 {#3}}{\partial {#1}\partial {#2}}}
%\newcommand{\PDd}[2]{\frac{\partial^2 {#2}}{{\partial{#1}}^2}}
\newcommand{\PDsq}[2]{\frac{\partial^2 {#2}}{(\partial {#1})^2}}

\newcommand{\Partial}[2]{\frac{\partial {#1}}{\partial {#2}}}
\DeclareMathOperator{\RejName}{Rej}
\newcommand{\Rej}[2]{\RejName_{#1}\left( {#2} \right)}
\newcommand{\Rm}[1]{\mathbb{R}^{#1}}
\newcommand{\Cm}[1]{\mathbb{C}^{#1}}
\newcommand{\conj}[0]{{*}}

%</misc>

% <grade selection>
%
\newcommand{\gpgrade}[2] {{\left\langle{{#1}}\right\rangle}_{#2}}

\newcommand{\gpgradezero}[1] {\gpgrade{#1}{}}
%\newcommand{\gpscalargrade}[1] {{\left\langle{{#1}}\right\rangle}}
%\newcommand{\gpgradezero}[1] {\gpgrade{#1}{0}}

%\newcommand{\gpgradeone}[1] {{\left\langle{{#1}}\right\rangle}_{1}}
\newcommand{\gpgradeone}[1] {\gpgrade{#1}{1}}

\newcommand{\gpgradetwo}[1] {\gpgrade{#1}{2}}
\newcommand{\gpgradethree}[1] {\gpgrade{#1}{3}}
\newcommand{\gpgradefour}[1] {\gpgrade{#1}{4}}
%
% </grade selection>



\newcommand{\adot}[0]{{\dot{a}}}
\newcommand{\bdot}[0]{{\dot{b}}}
% taken for centered dot:
%\newcommand{\cdot}[0]{{\dot{c}}}
%\newcommand{\ddot}[0]{{\dot{d}}}
\newcommand{\edot}[0]{{\dot{e}}}
\newcommand{\fdot}[0]{{\dot{f}}}
\newcommand{\gdot}[0]{{\dot{g}}}
\newcommand{\hdot}[0]{{\dot{h}}}
\newcommand{\idot}[0]{{\dot{i}}}
\newcommand{\jdot}[0]{{\dot{j}}}
\newcommand{\kdot}[0]{{\dot{k}}}
\newcommand{\ldot}[0]{{\dot{l}}}
\newcommand{\mdot}[0]{{\dot{m}}}
\newcommand{\ndot}[0]{{\dot{n}}}
%\newcommand{\odot}[0]{{\dot{o}}}
\newcommand{\pdot}[0]{{\dot{p}}}
\newcommand{\qdot}[0]{{\dot{q}}}
\newcommand{\rdot}[0]{{\dot{r}}}
\newcommand{\sdot}[0]{{\dot{s}}}
\newcommand{\tdot}[0]{{\dot{t}}}
\newcommand{\udot}[0]{{\dot{u}}}
\newcommand{\vdot}[0]{{\dot{v}}}
\newcommand{\wdot}[0]{{\dot{w}}}
\newcommand{\xdot}[0]{{\dot{x}}}
\newcommand{\ydot}[0]{{\dot{y}}}
\newcommand{\zdot}[0]{{\dot{z}}}
\newcommand{\addot}[0]{{\ddot{a}}}
\newcommand{\bddot}[0]{{\ddot{b}}}
\newcommand{\cddot}[0]{{\ddot{c}}}
%\newcommand{\dddot}[0]{{\ddot{d}}}
\newcommand{\eddot}[0]{{\ddot{e}}}
\newcommand{\fddot}[0]{{\ddot{f}}}
\newcommand{\gddot}[0]{{\ddot{g}}}
\newcommand{\hddot}[0]{{\ddot{h}}}
\newcommand{\iddot}[0]{{\ddot{i}}}
\newcommand{\jddot}[0]{{\ddot{j}}}
\newcommand{\kddot}[0]{{\ddot{k}}}
\newcommand{\lddot}[0]{{\ddot{l}}}
\newcommand{\mddot}[0]{{\ddot{m}}}
\newcommand{\nddot}[0]{{\ddot{n}}}
\newcommand{\oddot}[0]{{\ddot{o}}}
\newcommand{\pddot}[0]{{\ddot{p}}}
\newcommand{\qddot}[0]{{\ddot{q}}}
\newcommand{\rddot}[0]{{\ddot{r}}}
\newcommand{\sddot}[0]{{\ddot{s}}}
\newcommand{\tddot}[0]{{\ddot{t}}}
\newcommand{\uddot}[0]{{\ddot{u}}}
\newcommand{\vddot}[0]{{\ddot{v}}}
\newcommand{\wddot}[0]{{\ddot{w}}}
\newcommand{\xddot}[0]{{\ddot{x}}}
\newcommand{\yddot}[0]{{\ddot{y}}}
\newcommand{\zddot}[0]{{\ddot{z}}}

%<bold and dot greek symbols>
%

\newcommand{\Deltadot}[0]{{\dot{\Delta}}}
\newcommand{\Gammadot}[0]{{\dot{\Gamma}}}
\newcommand{\Lambdadot}[0]{{\dot{\Lambda}}}
\newcommand{\Omegadot}[0]{{\dot{\Omega}}}
\newcommand{\Phidot}[0]{{\dot{\Phi}}}
\newcommand{\Pidot}[0]{{\dot{\Pi}}}
\newcommand{\Psidot}[0]{{\dot{\Psi}}}
\newcommand{\Sigmadot}[0]{{\dot{\Sigma}}}
\newcommand{\Thetadot}[0]{{\dot{\Theta}}}
\newcommand{\Upsilondot}[0]{{\dot{\Upsilon}}}
\newcommand{\Xidot}[0]{{\dot{\Xi}}}
\newcommand{\alphadot}[0]{{\dot{\alpha}}}
\newcommand{\betadot}[0]{{\dot{\beta}}}
\newcommand{\chidot}[0]{{\dot{\chi}}}
\newcommand{\deltadot}[0]{{\dot{\delta}}}
\newcommand{\epsilondot}[0]{{\dot{\epsilon}}}
\newcommand{\etadot}[0]{{\dot{\eta}}}
\newcommand{\gammadot}[0]{{\dot{\gamma}}}
\newcommand{\kappadot}[0]{{\dot{\kappa}}}
\newcommand{\lambdadot}[0]{{\dot{\lambda}}}
\newcommand{\mudot}[0]{{\dot{\mu}}}
\newcommand{\nudot}[0]{{\dot{\nu}}}
\newcommand{\omegadot}[0]{{\dot{\omega}}}
\newcommand{\phidot}[0]{{\dot{\phi}}}
\newcommand{\pidot}[0]{{\dot{\pi}}}
\newcommand{\psidot}[0]{{\dot{\psi}}}
\newcommand{\rhodot}[0]{{\dot{\rho}}}
\newcommand{\sigmadot}[0]{{\dot{\sigma}}}
\newcommand{\taudot}[0]{{\dot{\tau}}}
\newcommand{\thetadot}[0]{{\dot{\theta}}}
\newcommand{\upsilondot}[0]{{\dot{\upsilon}}}
\newcommand{\varepsilondot}[0]{{\dot{\varepsilon}}}
\newcommand{\varphidot}[0]{{\dot{\varphi}}}
\newcommand{\varpidot}[0]{{\dot{\varpi}}}
\newcommand{\varrhodot}[0]{{\dot{\varrho}}}
\newcommand{\varsigmadot}[0]{{\dot{\varsigma}}}
\newcommand{\varthetadot}[0]{{\dot{\vartheta}}}
\newcommand{\xidot}[0]{{\dot{\xi}}}
\newcommand{\zetadot}[0]{{\dot{\zeta}}}

\newcommand{\Deltaddot}[0]{{\ddot{\Delta}}}
\newcommand{\Gammaddot}[0]{{\ddot{\Gamma}}}
\newcommand{\Lambdaddot}[0]{{\ddot{\Lambda}}}
\newcommand{\Omegaddot}[0]{{\ddot{\Omega}}}
\newcommand{\Phiddot}[0]{{\ddot{\Phi}}}
\newcommand{\Piddot}[0]{{\ddot{\Pi}}}
\newcommand{\Psiddot}[0]{{\ddot{\Psi}}}
\newcommand{\Sigmaddot}[0]{{\ddot{\Sigma}}}
\newcommand{\Thetaddot}[0]{{\ddot{\Theta}}}
\newcommand{\Upsilonddot}[0]{{\ddot{\Upsilon}}}
\newcommand{\Xiddot}[0]{{\ddot{\Xi}}}
\newcommand{\alphaddot}[0]{{\ddot{\alpha}}}
\newcommand{\betaddot}[0]{{\ddot{\beta}}}
\newcommand{\chiddot}[0]{{\ddot{\chi}}}
\newcommand{\deltaddot}[0]{{\ddot{\delta}}}
\newcommand{\epsilonddot}[0]{{\ddot{\epsilon}}}
\newcommand{\etaddot}[0]{{\ddot{\eta}}}
\newcommand{\gammaddot}[0]{{\ddot{\gamma}}}
\newcommand{\kappaddot}[0]{{\ddot{\kappa}}}
\newcommand{\lambdaddot}[0]{{\ddot{\lambda}}}
\newcommand{\muddot}[0]{{\ddot{\mu}}}
\newcommand{\nuddot}[0]{{\ddot{\nu}}}
\newcommand{\omegaddot}[0]{{\ddot{\omega}}}
\newcommand{\phiddot}[0]{{\ddot{\phi}}}
\newcommand{\piddot}[0]{{\ddot{\pi}}}
\newcommand{\psiddot}[0]{{\ddot{\psi}}}
\newcommand{\rhoddot}[0]{{\ddot{\rho}}}
\newcommand{\sigmaddot}[0]{{\ddot{\sigma}}}
\newcommand{\tauddot}[0]{{\ddot{\tau}}}
\newcommand{\thetaddot}[0]{{\ddot{\theta}}}
\newcommand{\upsilonddot}[0]{{\ddot{\upsilon}}}
\newcommand{\varepsilonddot}[0]{{\ddot{\varepsilon}}}
\newcommand{\varphiddot}[0]{{\ddot{\varphi}}}
\newcommand{\varpiddot}[0]{{\ddot{\varpi}}}
\newcommand{\varrhoddot}[0]{{\ddot{\varrho}}}
\newcommand{\varsigmaddot}[0]{{\ddot{\varsigma}}}
\newcommand{\varthetaddot}[0]{{\ddot{\vartheta}}}
\newcommand{\xiddot}[0]{{\ddot{\xi}}}
\newcommand{\zetaddot}[0]{{\ddot{\zeta}}}

\newcommand{\BDelta}[0]{\boldsymbol{\Delta}}
\newcommand{\BGamma}[0]{\boldsymbol{\Gamma}}
\newcommand{\BLambda}[0]{\boldsymbol{\Lambda}}
\newcommand{\BOmega}[0]{\boldsymbol{\Omega}}
\newcommand{\BPhi}[0]{\boldsymbol{\Phi}}
\newcommand{\BPi}[0]{\boldsymbol{\Pi}}
\newcommand{\BPsi}[0]{\boldsymbol{\Psi}}
\newcommand{\BSigma}[0]{\boldsymbol{\Sigma}}
\newcommand{\BTheta}[0]{\boldsymbol{\Theta}}
\newcommand{\BUpsilon}[0]{\boldsymbol{\Upsilon}}
\newcommand{\BXi}[0]{\boldsymbol{\Xi}}
\newcommand{\Balpha}[0]{\boldsymbol{\alpha}}
\newcommand{\Bbeta}[0]{\boldsymbol{\beta}}
\newcommand{\Bchi}[0]{\boldsymbol{\chi}}
\newcommand{\Bdelta}[0]{\boldsymbol{\delta}}
\newcommand{\Bepsilon}[0]{\boldsymbol{\epsilon}}
\newcommand{\Beta}[0]{\boldsymbol{\eta}}
\newcommand{\Bgamma}[0]{\boldsymbol{\gamma}}
\newcommand{\Bkappa}[0]{\boldsymbol{\kappa}}
\newcommand{\Blambda}[0]{\boldsymbol{\lambda}}
\newcommand{\Bmu}[0]{\boldsymbol{\mu}}
\newcommand{\Bnu}[0]{\boldsymbol{\nu}}
%\newcommand{\Bomega}[0]{\boldsymbol{\omega}}
\newcommand{\Bphi}[0]{\boldsymbol{\phi}}
\newcommand{\Bpi}[0]{\boldsymbol{\pi}}
\newcommand{\Bpsi}[0]{\boldsymbol{\psi}}
\newcommand{\Brho}[0]{\boldsymbol{\rho}}
\newcommand{\Bsigma}[0]{\boldsymbol{\sigma}}
%\newcommand{\Btau}[0]{\boldsymbol{\tau}}
%\newcommand{\Btheta}[0]{\boldsymbol{\theta}}
\newcommand{\Bupsilon}[0]{\boldsymbol{\upsilon}}
\newcommand{\Bvarepsilon}[0]{\boldsymbol{\varepsilon}}
\newcommand{\Bvarphi}[0]{\boldsymbol{\varphi}}
\newcommand{\Bvarpi}[0]{\boldsymbol{\varpi}}
\newcommand{\Bvarrho}[0]{\boldsymbol{\varrho}}
\newcommand{\Bvarsigma}[0]{\boldsymbol{\varsigma}}
\newcommand{\Bvartheta}[0]{\boldsymbol{\vartheta}}
\newcommand{\Bxi}[0]{\boldsymbol{\xi}}
\newcommand{\Bzeta}[0]{\boldsymbol{\zeta}}
%
%</bold and dot greek symbols>
%<infrequent>
%
%\newcommand{\AreaOp}[1]{\AName_{#1}}
%\newcommand{\Babs}[0]{\abs{\BB}}
%\newcommand{\Bcap}[0]{\hat{\BB}}
%\newcommand{\BrPrimeRej}[0]{\rcap(\rcap \wedge \Br')}
%\newcommand{\CA}[0]{\mathcal{A}}
%\newcommand{\Cos}[1]{\cos{\left({#1}\right)}}
%\newcommand{\Det}[1] {\abs{#1}}
%\newcommand{\Dsq}[2] {\frac {\partial^2 {#1}} {\partial {#2}^2}}
%\newcommand{\Exp}[1]{\exp{\left({#1}\right)}}
%\newcommand{\Norm}[1]{\left\lVert{#1}\right\rVert}
%\newcommand{\Sin}[1]{\sin{\left({#1}\right)}}
%\newcommand{\T}[0]{\text{T}}
%\newcommand{\VolumeOp}[1]{\VName_{#1}}
%\newcommand{\agrad}[0]{\Ba \cdot \nabla}
%\newcommand{\alphacap}[0]{\hat{\boldsymbol{\alpha}}}
%\newcommand{\Fcap}[0]{\hat{\BF}}
%\newcommand{\bithree}[0]{{\Bi}_3}
%\newcommand{\bxa}[0]{\Bx\Ba}
%\newcommand{\coordvec}[2]{
%\newcommand{\costheta}[0]{\acap \cdot \xcap}
%\newcommand{\ddt}[1]{\ddot{#1}}
%\newcommand{\ddu}[1] {\frac {d{#1}} {du}}
%\newcommand{\dsqxj}[2] {\frac {\partial^2 {#1}} {\partial {x_{#2}}^2}}
%\newcommand{\dtheta}[1]{\frac{d {#1}}{d \theta}}
%\newcommand{\dt}[1]{\dot{#1}}
%\newcommand{\dt}[1]{\frac{d {#1}}{dt}}
%\newcommand{\dxj}[2] {\frac {\partial {#1}} {\partial {x_{#2}}}}
%\newcommand{\halfPhi}[0]{\frac{\phi}{2}}
%\newcommand{\half}[0]{\inv{2}}
%\newcommand{\inv}[1]{\frac{1}{#1}}
%\newcommand{\laplacian}[0]{\nabla^2}
%\newcommand{\matrixoftx}[3]{
%\newcommand{\nrrp}[0]{\norm{\rcap \wedge \Br'}}
%\newcommand{\oiint}{\bigcirc \hspace{-1.4em} \int \hspace{-.8em} \int}
%\newcommand{\transpose}[1]{{#1}^{\text{T}}}
%\newcommand{\transpose}[1]{{{#1}^{\TextTranspose}}}
%\newcommand{\transpose}[1]{{{#1}^{\text{T}}}}
%\newcommand{\barA}[0]{\bar{A}}
%\newcommand{\qbar}[0]{\bar{q}}
%\newcommand{\qdotbar}[0]{\dot{\bar{q}}}
%
%</infrequent>





%\usepackage{listings}
%\usepackage{txfonts} % for ointctr... (also appears to make "prettier" \int and \sum's)
% makes \grad look funny though (almost like spacegrad, but narrower)
%\usepackage[bookmarks=true]{hyperref}

%\usepackage{color,cite,graphicx}
   % use colour in the document, put your citations as [1-4]
   % rather than [1,2,3,4] (it looks nicer, and the extended LaTeX2e
   % graphics package.
%\usepackage{latexsym,amssymb,epsf} % do not remember if these are
   % needed, but their inclusion can not do any damage


\chapter{VERSION: First try at one dimensional rectangular Quantum barrier problem}
\label{chap:qmbFirstTry}
%\author{Peeter Joot \quad peeterjoot@protonmail.com }
\date{ May 9, 2009.  \(RCSfile: qmbFirstTry.tex,v \) Last \(Revision: 1.8 \) \(Date: 2009/06/14 23:51:45 \) }

%\begin{document}

%\maketitle{}
%\tableofcontents
\section{}

\subsection{Problem 4.  Barrier penetration}

\subsubsection{Setup}

%In the barrier \(x \in [0,a]\),
%
%\begin{align*}
%\psi
%&=
%\frac{A}{2}\left(1 + i \frac{p_1}{p_2}\right) \exp\left( (ip_1 - p_2)\frac{a}{\Hbar}
%+\frac{p_2 x}{\Hbar}\right)  \\
%&+
%\frac{A}{2}\left(1 - i \frac{p_1}{p_2}\right) \exp\left( (ip_1 + p_2)\frac{a}{\Hbar}
%-\frac{p_2 x}{\Hbar}\right) \\
%\end{align*}
%
%Or
%\begin{align*}
%\psi
%&=
%A
%\exp\left( ip_1 \frac{a}{\Hbar} \right)
%\cosh\left( p_2 \frac{(a-x)}{\Hbar} \right) \\
%&+
%i A
%\exp\left( ip_1 \frac{a}{\Hbar} \right)
%\inv{2}\left(
%\frac{p_1}{p_2}\exp\left( p_2 \frac{(x-a)}{\Hbar} \right)
%-\frac{p_2}{p_1}\exp\left( p_2 \frac{(a-x)}{\Hbar} \right)
%\right)
%\end{align*}
%
%Past the barrier, \(x >a\),
%\begin{align*}
%\psi =
%A \exp\left( \frac{ip_1 x}{\Hbar}\right)
%\end{align*}
%
%and before the barrier, \(x<0\)
%
%\begin{align*}
%\psi &=
%D \exp\left( \frac{ip_1 x}{\Hbar}\right)
%+E \exp\left( \frac{-ip_1 x}{\Hbar}\right) \\
%D
%&=
%\frac{C}{2} \left( 1 + i \frac{p_2 }{p_1}\right)
%+\frac{B}{2} \left( 1 - i \frac{p_2 }{p_1}\right) \\
%E
%&=
%\frac{C}{2} \left( 1 - i \frac{p_2 }{p_1}\right)
%+\frac{B}{2} \left( 1 + i \frac{p_2 }{p_1}\right) \\
%B &= \frac{A}{2}\left(1 + i \frac{p_1}{p_2}\right) \exp\left( (ip_1 - p_2)\frac{a}{\Hbar}\right) \\
%C &= \frac{A}{2}\left(1 - i \frac{p_1}{p_2}\right) \exp\left( (ip_1 + p_2)\frac{a}{\Hbar}\right) \\
%\end{align*}
%
%That is
%
%\begin{align*}
%D
%&=
%\frac{A}{4}\left(1 - i \frac{p_1}{p_2}\right) \exp\left( (ip_1 + p_2)\frac{a}{\Hbar}\right)
%\left( 1 + i \frac{p_2 }{p_1}\right) \\
%&+
%\frac{A}{4}\left(1 + i \frac{p_1}{p_2}\right) \exp\left( (ip_1 - p_2)\frac{a}{\Hbar}\right)
%\left( 1 - i \frac{p_2 }{p_1}\right) \\
%E
%&=
%\frac{A}{4}\left(1 - i \frac{p_1}{p_2}\right) \exp\left( (ip_1 + p_2)\frac{a}{\Hbar}\right)
%\left( 1 - i \frac{p_2 }{p_1}\right) \\
%&+
%\frac{A}{4}\left(1 + i \frac{p_1}{p_2}\right) \exp\left( (ip_1 - p_2)\frac{a}{\Hbar}\right)
%\left( 1 + i \frac{p_2 }{p_1}\right) \\
%\end{align*}
%
%Expanding the products first for \(D\) we have
%
%\begin{align*}
%D
%&=
%\frac{A}{4}
%\left(2 - i \frac{p_1^2 - p_2^2}{p_1 p_2}\right)
%\exp\left( (ip_1 + p_2)\frac{a}{\Hbar}\right)
%\\
%&+
%\frac{A}{4}
%\left(2 + i \frac{p_1^2 - p_2^2}{p_1 p_2}\right)
%\exp\left( (ip_1 - p_2)\frac{a}{\Hbar}\right)
%\\
%&=
%A
%\exp\left( ip_1 \frac{a}{\Hbar}\right)
%\cosh\left( p_2 \frac{a}{\Hbar}\right)
%%%%
%+\frac{iA}{2}\frac{p_2^2 - p_1^2}{p_1 p_2}
%\exp\left( ip_1 \frac{a}{\Hbar}\right)
%\sinh\left( p_2 \frac{a}{\Hbar}\right)
%%%%
%\end{align*}
%
%and for \(E\), we have
%\begin{align*}
%E
%&=
%\frac{-iA}{4}\frac{p_1^2 + p_2^2}{p_1 p_2}
%\exp\left( (ip_1 + p_2)\frac{a}{\Hbar}\right)
%\\
%&+
%\frac{iA}{4}\frac{p_1^2 + p_2^2}{p_1 p_2}
%\exp\left( (ip_1 - p_2)\frac{a}{\Hbar}\right)
%\\
%&=
%\frac{iA}{2}\frac{p_1^2 + p_2^2}{p_1 p_2}
%\exp\left( ip_1 \frac{a}{\Hbar}\right)
%\sinh\left( p_2 \frac{a}{\Hbar}\right)
%\end{align*}
%
%Reassembling, the sum of the incident and reflected wave functions in the \(x<0\) region is
%
%%\begin{align*}
%%D &=
%%A \exp\left( ip_1 \frac{a}{\Hbar}\right) \cosh\left( p_2 \frac{a}{\Hbar}\right)
%%+\frac{iA}{2}\frac{p_2^2 - p_1^2}{p_1 p_2} \exp\left( ip_1 \frac{a}{\Hbar}\right) \sinh\left( p_2 \frac{a}{\Hbar}\right)
%%\end{align*}
%%
%%\begin{align*}
%%E &= \frac{iA}{2}\frac{p_1^2 + p_2^2}{p_1 p_2} \exp\left( ip_1 \frac{a}{\Hbar}\right) \sinh\left( p_2 \frac{a}{\Hbar}\right)
%%\end{align*}
%
%\begin{align*}
%\psi
%&= A \exp\left( ip_1 \frac{(a+x)}{\Hbar}\right) \cosh\left( p_2 \frac{a}{\Hbar}\right) \\
%&+\frac{iA}{2}\frac{p_2^2 - p_1^2}{p_1 p_2} \exp\left( ip_1 \frac{(a+x)}{\Hbar}\right) \sinh\left( p_2 \frac{a}{\Hbar}\right) \\
%&+ \frac{iA}{2}\frac{p_1^2 + p_2^2}{p_1 p_2} \exp\left( ip_1 \frac{(a-x)}{\Hbar}\right) \sinh\left( p_2 \frac{a}{\Hbar}\right)  \\
%%%%
%&= A \exp\left( ip_1 \frac{(a+x)}{\Hbar}\right) \cosh\left( p_2 \frac{a}{\Hbar}\right) \\
%&+{iA}\frac{p_2 }{p_1 }
%\exp\left( ip_1 \frac{a}{\Hbar}\right)
%\cos\left( p_1 \frac{x}{\Hbar}\right)
%\sinh\left( p_2 \frac{a}{\Hbar}\right) \\
%&+A \frac{p_1}{p_2}
%\exp\left( ip_1 \frac{a}{\Hbar}\right)
%\sin\left( p_1 \frac{x}{\Hbar}\right)
%\sinh\left( p_2 \frac{a}{\Hbar}\right) \\
%&= A \exp\left( ip_1 \frac{(a+x)}{\Hbar}\right) \cosh\left( p_2 \frac{a}{\Hbar}\right) \\
%&
%+A \exp\left( ip_1 \frac{a}{\Hbar}\right) \sinh\left( p_2 \frac{a}{\Hbar}\right)
%\left(
%+{i}\frac{p_2 }{p_1 }
%\cos\left( p_1 \frac{x}{\Hbar}\right)
%+ \frac{p_1}{p_2}
%\sin\left( p_1 \frac{x}{\Hbar}\right)
%\right)
%\\
%\end{align*}
%
%This can be made slightly more symmetrical by expanding one of the complex exponential terms.  Doing so, also writing \(\alpha = A e^{i p_1 a/\Hbar}\),
%we can summarize the wave functions in each of the intervals.
%
%For \(x<0\)
%\begin{align*}
%\psi
%&=
%\alpha
%\cosh\left( p_2 \frac{a}{\Hbar}\right)
%%\left( \cos\left( p_1 \frac{x}{\Hbar}\right) + i \sin\left( p_1 \frac{x}{\Hbar}\right) \right) \\
%\exp\left( i p_1 \frac{x}{\Hbar}\right)
%%\\
%%&
%+\alpha
%\sinh\left( p_2 \frac{a}{\Hbar}\right)
%\left( {i}\frac{p_2 }{p_1 } \cos\left( p_1 \frac{x}{\Hbar}\right) + \frac{p_1}{p_2} \sin\left( p_1 \frac{x}{\Hbar}\right) \right)
%\\
%\end{align*}
%
%For \(x \in [0,a]\)
%\begin{align*}
%\psi
%&=
%\alpha
%\cosh\left( p_2 \frac{(x-a)}{\Hbar} \right)
%+
%\frac{i\alpha}{2}\left(
%\frac{p_1}{p_2}\exp\left( p_2 \frac{(x-a)}{\Hbar} \right)
%-\frac{p_2}{p_1}\exp\left( p_2 \frac{(a-x)}{\Hbar} \right)
%\right)
%\end{align*}
%
%and past the barrier, \(x >a\),
%\begin{align*}
%\psi =
%\alpha \exp\left( \frac{ip_1 (x-a)}{\Hbar}\right)
%\end{align*}

Potential barrier, height V, in \([0,a]\), with potential zero everywhere else.  With \(p_1 = \sqrt{2mE}\), and \(p_2 = \sqrt{2m(V-E)}\), a
solution is calculated in the text.  Attempting to substitute for the coefficients found, and reduce them to a form
that looks amenable to calculation of the currents, leads to a mess.  It does however suggest a simpler structure for the
wave functions.  Let us start over and seek solutions in the \(x>a\) region of the form

\begin{equation}\label{eqn:qmbFirstTry:20}
\begin{aligned}
\psi &= \alpha e^{ i p_1 (x-a)/\Hbar}
\end{aligned}
\end{equation}

and in the barrier region \(x \in [0,a]\)

\begin{equation}\label{eqn:qmbFirstTry:40}
\begin{aligned}
\psi &= A e^{ p_1 (x-a)/\Hbar} +B e^{ -p_1 (x-a)/\Hbar}
\end{aligned}
\end{equation}

continuity (value and derivative) at \(x=a\) gives

\begin{equation}\label{eqn:qmbFirstTry:60}
\begin{aligned}
A + B &= \alpha  \\
A - B &= i \alpha \frac{p_1}{p_2}
\end{aligned}
\end{equation}

Solving and reducing we have

\begin{equation}\label{eqn:qmbFirstTry:80}
\begin{aligned}
\psi &=
\alpha \cosh\left( p_2 (x-a)/\Hbar \right)
+ i \alpha \frac{p_1}{p_2} \sinh\left( p_2 (x-a)/\Hbar \right)
\end{aligned}
\end{equation}

For the \(x<0\) region assume again a solution of the form

\begin{equation}\label{eqn:qmbFirstTry:100}
\begin{aligned}
\psi = u e^{i p_1 x/\Hbar} +v e^{-i p_1 x/\Hbar}
\end{aligned}
\end{equation}

Matching at \(x=0\) we have

\begin{equation}\label{eqn:qmbFirstTry:120}
\begin{aligned}
u + v &= \alpha \cosh( p_2 a/\Hbar ) - i \alpha \frac{p_1}{p_2} \sinh( p_2 a /\Hbar ) \\
u - v &= i \alpha \frac{p_2}{p_1} \sinh( p_2 a/\Hbar ) + \alpha \cosh( p_2 a /\Hbar ) \\
\end{aligned}
\end{equation}

Solving
\begin{equation}\label{eqn:qmb_first_try:uAndv}
\begin{aligned}
u &= \alpha \cosh( p_2 a/\Hbar )  + \frac{i\alpha}{2} \sinh( p_2 a /\Hbar ) \left( \frac{p_2}{p_1} -  \frac{p_1}{p_2} \right) \\
v &= - \frac{i\alpha}{2} \sinh( p_2 a /\Hbar ) \left( \frac{p_2}{p_1} + \frac{p_1}{p_2} \right)
\end{aligned}
\end{equation}

\subsubsection{Currents}

Current past the barrier is simply calculated

\begin{equation}\label{eqn:qmb_first_try:JtBarrier}
\begin{aligned}
J_t
&= \frac{p_1}{m} \Abs{\alpha}^2
\end{aligned}
\end{equation}

Which we can write directly in terms of the probability density
\begin{equation}\label{eqn:qmbFirstTry:140}
\begin{aligned}
J_t
&= \frac{p_1}{m} \rho
\end{aligned}
\end{equation}

So, again, the probability current is a velocity scaling of the probably density.

In the barrier region, writing
\(C = \cosh(p_2(x-a)/\Hbar)\), and
\(S = \sinh(p_2(x-a)/\Hbar)\), we have

\begin{equation}\label{eqn:qmbFirstTry:160}
\begin{aligned}
J &=
\frac{\Abs{\alpha}^2}{ 2 m i }
\left(
\left(C - i \frac{p_1}{p_2}S\right)
\left(p_2 S + i p_1 C\right)
-
\left(C + i \frac{p_1}{p_2}S\right)
\left(p_2 S - i p_1 C\right)
\right)
\end{aligned}
\end{equation}

Reducing these products we get
\begin{equation}\label{eqn:qmbFirstTry:180}
\begin{aligned}
J
&=
\frac{p_1 \Abs{\alpha}^2}{m} (C^2 - S^2) \\
&=
\frac{p_1 \Abs{\alpha}^2}{m}
\end{aligned}
\end{equation}

So the current equals that in the region past the barrier.  This
equality makes me suspect an error in the treatment of problem 1.

Note however
that the probability density in this region is

\begin{equation}\label{eqn:qmbFirstTry:200}
\begin{aligned}
\rho
&= \Abs{\alpha}^2 \left(
C^2 + S^2 \left(\frac{p_1}{p_2}\right)^2
\right)
\end{aligned}
\end{equation}

So, we have

\begin{equation}\label{eqn:qmbFirstTry:220}
\begin{aligned}
J
&=
\frac{p_1}{m} \rho \inv{C^2 + {\left(\frac{p_1}{p_2}\right)}^2 S^2 } \\
&=
\frac{p_1}{m} \rho \inv{
\cosh^2(p_2(x-a)/\Hbar)
 +
 {\left(\frac{p_1}{p_2}\right)}^2
\sinh^2(p_2(x-a)/\Hbar)
 } \\
\end{aligned}
\end{equation}

There is a spatially dependency between the current density and the probability density that we do not have past the barrier region.  The current density
can still
be thought of as a velocity scaling of the probability density, but
we have a position dependent velocity inside the barrier

\begin{equation}\label{eqn:qmbFirstTry:240}
\begin{aligned}
v_b = \frac{p_1}{m} \inv{
\cosh^2(p_2(x-a)/\Hbar)
 +
 {\left(\frac{p_1}{p_2}\right)}^2
\sinh^2(p_2(x-a)/\Hbar)
 } \\
\end{aligned}
\end{equation}

Here the optical analogy seems appropriate, and the barrier with its sharp
boundaries ends up manifesting like a material with varying index of
refraction.  Some of that variation of index of refraction (related by Snell's
law to the velocity of the material) appears to be impacted by the shape of
the boundary, and loosely it seems like a sharp potential transition acts
as an impedance.  TODO: do some math to back this up.  Am kind of guessing
here.

SNIP.  MOVED to \(qm_barrier.ltx\)

We are now in shape to substitute \eqnref{eqn:qmb_first_try:uAndv} into these.  For short we
write

\begin{equation}\label{eqn:qmbFirstTry:260}
\begin{aligned}
u
&= \alpha \cosh( p_2 a/\Hbar )  + \frac{i\alpha}{2} \sinh( p_2 a /\Hbar ) \left( \frac{p_2}{p_1} -  \frac{p_1}{p_2} \right) \\
&= \alpha C + \frac{i\alpha}{2} S \left( p_{21} - p_{12} \right) \\
&= \alpha C + \frac{i\alpha}{2} S \delta p
\end{aligned}
\end{equation}

and
\begin{equation}\label{eqn:qmbFirstTry:280}
\begin{aligned}
v
&= - \frac{i\alpha}{2} \sinh( p_2 a /\Hbar ) \left( \frac{p_2}{p_1} + \frac{p_1}{p_2} \right) \\
&= \frac{i\alpha}{2} S \left( p_{12} - p_{21} \right) \\
&= \frac{i\alpha}{2} S \delta p
\end{aligned}
\end{equation}

Absolute squares of these are
\begin{equation}\label{eqn:qmbFirstTry:300}
\begin{aligned}
\Abs{u}^2 &= \Abs{\alpha}^2 (C^2 + S^2 (\delta p)^2/4)
\end{aligned}
\end{equation}

and
\begin{equation}\label{eqn:qmbFirstTry:320}
\begin{aligned}
\Abs{v}^2 &= \Abs{\alpha}^2 S^2 (\delta p)^2/4
\end{aligned}
\end{equation}

This gives a total incident plus reflected current in the \(x<0\) region as

\begin{equation}\label{eqn:qmb_first_try:jTotBarrier}
\begin{aligned}
J
&=
\frac{p_1}{m} \Abs{\alpha}^2 \cosh^2(p_2 a/\Hbar)
\end{aligned}
\end{equation}

For the probability density, we need

\begin{equation}\label{eqn:qmbFirstTry:340}
\begin{aligned}
\Abs{u}^2 + \Abs{v}^2 + 2 \Re(v^\conj u \epsilon^2)
&=
\Abs{\alpha}^2 \left(C^2 + S^2 (\delta p)^2/2
+ 2\Re\left(
-\frac{i}{2} S (\delta p) \left(C + \frac{i}{2} S (\delta p)\right) e^{i p_1 x/\Hbar}
\right)\right) \\
&=
\Abs{\alpha}^2 \left(C^2 + S^2 (\delta p)^2/2
+ S C (\delta p) \sin( 2 p_1 x/\Hbar)
+\frac{1}{2} S^2 (\delta p)^2 \cos( 2 p_1 x/\Hbar)
\right) \\
&=
\Abs{\alpha}^2 \left(C^2 + S^2 (\delta p)^2 \inv{2}(1 + \cos( 2 p_1 x/\Hbar))
+ S C (\delta p) \sin( 2 p_1 x/\Hbar)
\right) \\
&=
\Abs{\alpha}^2 \left(C^2 + S^2 (\delta p)^2 \cos^2( p_1 x/\Hbar)
+ S C (\delta p) \sin( 2 p_1 x/\Hbar)
\right) \\
\end{aligned}
\end{equation}

So the probability density for \(x<0\) is
\begin{equation}\label{eqn:qmbFirstTry:360}
\begin{aligned}
\rho
&=
\Abs{\alpha}^2 \left(\cosh^2(p_2 a/\Hbar) + \sinh^2(p_2 a/\Hbar) (\delta p)^2 \cos^2( p_1 x/\Hbar)
%+ \sinh(p_2 a/\Hbar)\cosh(p_2 a/\Hbar) (\delta p) \sin( 2 p_1 x/\Hbar)
+ \inv{2} \sinh(2 p_2 a/\Hbar) (\delta p) \sin( 2 p_1 x/\Hbar)
\right)
\end{aligned}
\end{equation}

The last two currents of interest to calculate in the \(x<0\) region are the incident and reflected probability currents.  The
incident current is

\begin{equation}\label{eqn:qmbFirstTry:380}
\begin{aligned}
J_i
&=
\frac{\Hbar}{2m i}\left(
u^\conj \epsilon^\conj (ip_1/\Hbar) u \epsilon
-u \epsilon (-ip_1/\Hbar) u^\conj \epsilon^\conj
\right)
\\
&=
\frac{p_1}{m} \Abs{u}^2
\end{aligned}
\end{equation}

and the reflected current is
\begin{equation}\label{eqn:qmbFirstTry:400}
\begin{aligned}
J_r
&=
\frac{\Hbar}{2m i}\left(
v^\conj \epsilon (-ip_1/\Hbar) v^\conj \epsilon^\conj
-v \epsilon^\conj (ip_1/\Hbar) v \epsilon
\right) \\
&=
\frac{-p_1}{m} \Abs{v}^2
\end{aligned}
\end{equation}

So we have

\begin{equation}\label{eqn:qmb_first_try:JiBarrier}
\begin{aligned}
J_i
&= \frac{p_1}{m} \Abs{\alpha}^2 (\cosh^2(p_2 a/\Hbar) + \sinh^2(p_2 a/\Hbar) (\delta p)^2/4)
\end{aligned}
\end{equation}

\begin{equation}\label{eqn:qmb_first_try:JrBarrier}
\begin{aligned}
J_r
&= \frac{-p_1}{m} \Abs{\alpha}^2 \sinh^2(p_2 a/\Hbar) (\delta p)^2/4
\end{aligned}
\end{equation}

\subsubsection{Transmission and reflection coefficients}

From equations \eqnref{eqn:qmb_first_try:JtBarrier} \eqnref{eqn:qmb_first_try:JiBarrier} \eqnref{eqn:qmb_first_try:JrBarrier}, the transmission and reflection coefficients can be calculated.
These are

\begin{equation}\label{eqn:qmbFirstTry:420}
\begin{aligned}
R &=
\frac{\Abs{J_r}} {\Abs{J_i}} \\
&=
\frac{\sinh^2(p_2 a/\Hbar) (\delta p)^2/4}{ \cosh^2(p_2 a/\Hbar) + \sinh^2(p_2 a/\Hbar) (\delta p)^2/4 }
\end{aligned}
\end{equation}

and

\begin{equation}\label{eqn:qmbFirstTry:440}
\begin{aligned}
T &=
\frac{\Abs{J_t}} {\Abs{J_i}} \\
&=
\frac{1}{ \cosh^2(p_2 a/\Hbar) + \sinh^2(p_2 a/\Hbar) (\delta p)^2/4 }
\end{aligned}
\end{equation}

Again, these do not sum to unity as expected.  I either misunderstand something or am making mistakes.  Review this all, and compare to the approximate treatment in the book (for the case \(p_2 a/\Hbar \gg 1\)).

A peek in \citep{mcmahon2005qmd} has got a result similar to this
but with \(1/2\) instead of a \(\cosh\) term in the denominator for \(T\).  That
said there are so many obvious typos in that treatment it is hard
to trust it.  Review
can probably focus on my own, but looking for an error in \(J_i\) or \(\psi_i\).

%\bibliographystyle{plainnat}
%\bibliography{myrefs}

%\end{document}
