%
% Copyright � 2013 Peeter Joot.  All Rights Reserved.
% Licenced as described in the file LICENSE under the root directory of this GIT repository.
%
%\newcommand{\authorname}{Peeter Joot}
\newcommand{\email}{peeterjoot@protonmail.com}
\newcommand{\basename}{FIXMEbasenameUndefined}
\newcommand{\dirname}{notes/FIXMEdirnameUndefined/}

%\renewcommand{\basename}{basicStatMechLecture1}
%\renewcommand{\dirname}{notes/phy452/}
%\newcommand{\keywords}{Optics, PHY452H1S}
%\newcommand{\authorname}{Peeter Joot}
\newcommand{\onlineurl}{http://sites.google.com/site/peeterjoot2/math2013/\basename.pdf}
\newcommand{\sourcepath}{\dirname\basename.tex}
\newcommand{\generatetitle}[1]{\chapter{#1}}

\newcommand{\vcsinfo}{%
\section*{}
\noindent{\color{DarkOliveGreen}{\rule{\linewidth}{0.1mm}}}
\paragraph{Document version}
%\paragraph{\color{Maroon}{Document version}}
{
\small
\begin{itemize}
\item Available online at:\\ 
\href{\onlineurl}{\onlineurl}
\item Git Repository: \input{./.revinfo/gitRepo.tex}
\item Source: \sourcepath
\item last commit: \input{./.revinfo/gitCommitString.tex}
\item commit date: \input{./.revinfo/gitCommitDate.tex}
\end{itemize}
}
}

%\PassOptionsToPackage{dvipsnames,svgnames}{xcolor}
\PassOptionsToPackage{square,numbers}{natbib}
\documentclass{scrreprt}

\usepackage[left=2cm,right=2cm]{geometry}
\usepackage[svgnames]{xcolor}
\usepackage{peeters_layout}

\usepackage{natbib}

\usepackage[
colorlinks=true,
bookmarks=false,
pdfauthor={\authorname, \email},
backref 
]{hyperref}

% http://tex.stackexchange.com/questions/75773/how-to-reference-problems-by-the-text-label-in-an-exercise-envioronment
\usepackage[english]{cleveref}
\crefname{Exercise}{exercise}{exercises}
\Crefname{Exercise}{Exercise}{Exercises}

\RequirePackage{titlesec}
\RequirePackage{ifthen}

% http://stackoverflow.com/questions/4932910/date-in-the-tabular-environment
\makeatletter
\let\insertdate\@date
\makeatother

\titleformat{\chapter}[display]
{\bfseries\Large}
{\color{DarkSlateGrey}\filleft \authorname
\ifthenelse{\isundefined{\studentnumber}}{}{\\ \studentnumber}
\ifthenelse{\isundefined{\email}}{}{\\ \email}
\ifthenelse{\isundefined{\dateintitle}}{}{\\ \insertdate}
%\ifthenelse{\isundefined{\coursename}}{}{\\ \coursename} % put in title instead.
}
{4ex}
{\color{DarkOliveGreen}{\titlerule}\color{Maroon}
\vspace{2ex}%
\filright}
[\vspace{2ex}%
\color{DarkOliveGreen}\titlerule
]

\newcommand{\beginArtWithToc}[0]{\begin{document}\tableofcontents}
\newcommand{\beginArtNoToc}[0]{\begin{document}}
\newcommand{\EndNoBibArticle}[0]{\end{document}}
\newcommand{\EndArticle}[0]{\bibliography{Bibliography}\bibliographystyle{plainnat}\end{document}}

% 
%\newcommand{\citep}[1]{\cite{#1}}

\colorSectionsForArticle


%
%\beginArtNoToc
%\generatetitle{PHY452H1S Basic Statistical Mechanics.  Lecture 1: What is statistical mechanics and equilibrium.  Taught by Prof.\ Arun Paramekanti}
\label{chap:basicStatMechLecture1}

%\section{Disclaimer}
%
%Peeter's lecture notes from class.  May not be entirely coherent.

\section{Equilibrium}

This is the study of systems in \underlineAndIndex{equilibrium}.  What is equilibrium? \footnote{There's an excellent discussion of this in \S 1 \citep{ma1985statistical}}

\begin{itemize}
\item Mechanical equilibrium example: ball in bowl
\item Chemical equilibrium: when all the reactants have been consumed, or rates of forward and backwards reactions have become more constant.
\end{itemize}

More generally, equilibrium is a matter of time scales!

\paragraph{Feynman}
``Fast things have happened and all slow things have not''

\makeexample{Water in a cup}{example:basicStatMechLecture1:1}
{
After 1 min - 1 hour, when any sloshing has stopped, we can say it's in equilibrium.  However, if we consider a time scale like 10 days, we see that there are changes occurring (evaporation).
}

\makeexample{Hot water in a cup}{example:basicStatMechLecture1:2}{
Less than 10 minute time scale: Not in equilibrium (evaporating).  At a longer time scale we may say it's reached equilibrium, but again on a, say, 10 day time scale, we'd again reach the conclusion that this system is also not in equilibrium.

We go through state transitions
\begin{itemize}
\item Cooling: Not in equilibrium
\item Stay: In equilibrium
\item Evaporate: Not in equilibrium
\end{itemize}
%FIXME: tabularize
%		Hot water -> Cooling -> Stay -> evaporate
%equil				No    	Yes 	No
}

\makeexample{Window glass}{example:basicStatMechLecture1:3}{

\begin{itemize}
\item \(\sim 10\) years \(\rightarrow\) equilibrium.
\item \(\sim 100-1000\) years \(\rightarrow\) not in equilibrium.
\end{itemize}

While this was given in class as an example, \citep{stokes1999flowing} refutes this.
}

\makeexample{Battery and resistor}{example:basicStatMechLecture1:4}{

%\cref{fig:batteryAndResistor:batteryAndResistorFig1}.
\imageFigure{../../figures/phy452/batteryAndResistorFig1}{Steady non-equilibrium state}{fig:batteryAndResistor:batteryAndResistorFig1}{0.4}

Steady current over a small time scale, but we are persistently generating heat and draining the battery.
}

\paragraph{How do we reach equilibrium?}

\begin{itemize}
\item We'll be looking at small systems connected to a very much larger ``heat bath'' or ``environment''.  Such a system will eventually, after perhaps exchange of particles, radiation, ... will eventually take on a state that is dictated by the state of the environment.
\item Completely isolated system!  Molecules in a box, say, may all initially be a corner of a box.  We'll have exchange of energy and momentum, and interaction with the walls of the box.  The ``final'' state of the system will be determined by the initial state of the molecules.  Experiments of this sort have recently been performed and studied in depth (actually a very tricky problem).
\end{itemize}

\section{Probabilities}

\paragraph{Why do I need probabilities?}

\begin{itemize}
\item QM: Ultimately there's an underlying quantum state, and we can only talk about probabilities.  Any measurement has uncertainties, and this microscopic state forces us to use statistical methods.
\item Classical chaos and unpredictability: In a many particle system, or even a system of a few interacting particles with non-linear interaction, even given an initial state to 10 decimal places, we'll end up with uncertainties.  Given enough particles, even with simple interactions, we'll be forced to use statistical methods.
\item Too much information: Even if we suppose that we could answer the question of where is and how fast every particle in a large collection was, what would we do with that info.  We don't care about such a fine granularity state.  We want to know about things like gas pressure.  We are forced to discard information.
\end{itemize}

We need a systematic way of dealing with and discarding specific details.  We want to use statistical methods to throw away useless information.

\paragraph{Analogy: Election}

We can poll a sample of the population if we want to attempt to predict results.  We want to know things, but not what every person thinks.

%To come: lighting review of probability.

%\EndArticle
