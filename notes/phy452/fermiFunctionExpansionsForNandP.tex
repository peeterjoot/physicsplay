%
% Copyright � 2013 Peeter Joot.  All Rights Reserved.
% Licenced as described in the file LICENSE under the root directory of this GIT repository.
%
%\newcommand{\authorname}{Peeter Joot}
\newcommand{\email}{peeterjoot@protonmail.com}
\newcommand{\basename}{FIXMEbasenameUndefined}
\newcommand{\dirname}{notes/FIXMEdirnameUndefined/}

%\renewcommand{\basename}{fermiFunctionExpansionsForNandP}
%\renewcommand{\dirname}{notes/phy452/}
%\newcommand{\keywords}{Statistics mechanics, PHY452H1S, Fermion, Fermi-Dirac function, pressure, energy, average occupancy, density of states, volume, number density}
%
%\newcommand{\authorname}{Peeter Joot}
\newcommand{\onlineurl}{http://sites.google.com/site/peeterjoot2/math2013/\basename.pdf}
\newcommand{\sourcepath}{\dirname\basename.tex}
\newcommand{\generatetitle}[1]{\chapter{#1}}

\newcommand{\vcsinfo}{%
\section*{}
\noindent{\color{DarkOliveGreen}{\rule{\linewidth}{0.1mm}}}
\paragraph{Document version}
%\paragraph{\color{Maroon}{Document version}}
{
\small
\begin{itemize}
\item Available online at:\\ 
\href{\onlineurl}{\onlineurl}
\item Git Repository: \input{./.revinfo/gitRepo.tex}
\item Source: \sourcepath
\item last commit: \input{./.revinfo/gitCommitString.tex}
\item commit date: \input{./.revinfo/gitCommitDate.tex}
\end{itemize}
}
}

%\PassOptionsToPackage{dvipsnames,svgnames}{xcolor}
\PassOptionsToPackage{square,numbers}{natbib}
\documentclass{scrreprt}

\usepackage[left=2cm,right=2cm]{geometry}
\usepackage[svgnames]{xcolor}
\usepackage{peeters_layout}

\usepackage{natbib}

\usepackage[
colorlinks=true,
bookmarks=false,
pdfauthor={\authorname, \email},
backref 
]{hyperref}

% http://tex.stackexchange.com/questions/75773/how-to-reference-problems-by-the-text-label-in-an-exercise-envioronment
\usepackage[english]{cleveref}
\crefname{Exercise}{exercise}{exercises}
\Crefname{Exercise}{Exercise}{Exercises}

\RequirePackage{titlesec}
\RequirePackage{ifthen}

% http://stackoverflow.com/questions/4932910/date-in-the-tabular-environment
\makeatletter
\let\insertdate\@date
\makeatother

\titleformat{\chapter}[display]
{\bfseries\Large}
{\color{DarkSlateGrey}\filleft \authorname
\ifthenelse{\isundefined{\studentnumber}}{}{\\ \studentnumber}
\ifthenelse{\isundefined{\email}}{}{\\ \email}
\ifthenelse{\isundefined{\dateintitle}}{}{\\ \insertdate}
%\ifthenelse{\isundefined{\coursename}}{}{\\ \coursename} % put in title instead.
}
{4ex}
{\color{DarkOliveGreen}{\titlerule}\color{Maroon}
\vspace{2ex}%
\filright}
[\vspace{2ex}%
\color{DarkOliveGreen}\titlerule
]

\newcommand{\beginArtWithToc}[0]{\begin{document}\tableofcontents}
\newcommand{\beginArtNoToc}[0]{\begin{document}}
\newcommand{\EndNoBibArticle}[0]{\end{document}}
\newcommand{\EndArticle}[0]{\bibliography{Bibliography}\bibliographystyle{plainnat}\end{document}}

% 
%\newcommand{\citep}[1]{\cite{#1}}

\colorSectionsForArticle


%
%\beginArtNoToc
%
%\generatetitle{Fermi-Dirac function expansion for thermodynamic quantities}
\label{chap:fermiFunctionExpansionsForNandP}

In \S 8.1 of \citep{pathriastatistical} are some Fermi-Dirac \index{Fermi-Dirac function} expansions for \(P\), \(N\), and \(U\).  Let's work through these in detail.

Our starting point is the relations

\begin{subequations}
\begin{equation}\label{eqn:fermiFunctionExpansionsForNandP:20}
P V \beta = \ln \ZG = 
\sum \ln 
\lr{ 1 + z e^{-\beta \epsilon} }
\end{equation}
\begin{equation}\label{eqn:fermiFunctionExpansionsForNandP:40}
N = \sum \inv{ z^{-1} e^{\beta \epsilon} + 1 }.
\end{equation}
\end{subequations}

\paragraph{Recap.  Density of states}

We'll employ the 3D non-relativistic \textAndIndex{density of states}

\begin{dmath}\label{eqn:fermiFunctionExpansionsForNandP:60}
\mathcal{D}(\epsilon) 
=
\sum_\Bk \delta(\epsilon - \epsilon_\Bk)
\sim
V \int 
\frac{d^3 \Bk}{(2 \pi)^3}
\delta(\epsilon - \epsilon_\Bk)
=
\frac{4 \pi V}{(2 \pi)^3}
\int dk k^2 
\delta
\lr{ \epsilon - \frac{\Hbar^2 k^2}{2 m} }
=
\frac{4 \pi V}{(2 \pi)^3}
\int dk k^2 
\frac
{
   \delta
\lr{ k - \sqrt{2 m \epsilon}/\Hbar }
}
{ 
   \frac{\Hbar^2}{m} 
   \frac{\sqrt{2 m \epsilon}}{\Hbar}
}
=
\frac{2 V}{(2 \pi)^2 }
\frac{m}{\Hbar^2}
\sqrt{\frac{2 m \epsilon}{\Hbar^2}},
\end{dmath}

or 

\boxedEquation{eqn:fermiFunctionExpansionsForNandP:80}{
\mathcal{D}(\epsilon)
=
\frac{V}{(2 \pi)^2 }
\lr{ \frac{2 m}{\Hbar^2} }
^{3/2}
\epsilon^{1/2}.
}

\paragraph{Density}

Now let's make our integral approximation of the sum for \(N\).  That is

\begin{dmath}\label{eqn:fermiFunctionExpansionsForNandP:100}
N 
= g \int d\epsilon \mathcal{D}(\epsilon) \inv{ z^{-1} e^{\beta \epsilon} + 1 }
= g 
\frac{V}{(2 \pi)^2 }
\lr{ \frac{2 m}{\Hbar^2} }
^{3/2}
\int_0^\infty d\epsilon \frac{\epsilon^{1/2}}{ z^{-1} e^{\beta \epsilon} + 1 }
= g 
\frac{V}{(2 \pi)^2 \beta^{3/2}}
\lr{ \frac{2 m}{\Hbar^2} }
^{3/2}
\int_0^\infty du \frac{u^{1/2}}{ z^{-1} e^{u} + 1 }
= g 
\frac{V}{(2 \pi)^2 \beta^{3/2}}
\lr{ \frac{2 m}{\Hbar^2} }
^{3/2}
\Gamma(3/2) f_{3/2}(z)
= g 
\frac{V}{(2 \pi)^2 \beta^{3/2}}
\frac{
\lr{ 2 m \kB T }
^{3/2}
}{\Hbar^3}
\inv{2} \sqrt{\pi}
f_{3/2}(z)
= 
g 
V \cancel{2} \pi
\frac{
\lr{ 2 m \kB T }
^{3/2}
}{h^3}
\inv{\cancel{2}} \sqrt{\pi}
f_{3/2}(z),
\end{dmath}

or
\begin{equation}\label{eqn:fermiFunctionExpansionsForNandP:120}
\frac{N}{V} 
= 
g 
\frac{
\lr{ 2 \pi m \kB T }
^{3/2}
}{h^3}
f_{3/2}(z).
\end{equation}

With

\begin{equation}\label{eqn:fermiFunctionExpansionsForNandP:140}
\lambda = \frac{h}{\sqrt{ 2 \pi m \kB T }},
\end{equation}

this gives us the desired density result from the text

\boxedEquation{eqn:fermiFunctionExpansionsForNandP:160}{
\frac{N}{V}
=
\frac{g}{\lambda^3} f_{3/2}(z).
}

\paragraph{Pressure}

For the pressure, we can do the same, but have to integrate by parts

\begin{dmath}\label{eqn:fermiFunctionExpansionsForNandP:180}
P V \beta 
= 
g \sum \ln 
\lr{ 1 + z e^{-\beta \epsilon} }
\sim
g \frac{V}{(2 \pi)^2 }
\lr{ \frac{2 m}{\Hbar^2} }
^{3/2}
\int_0^\infty d\epsilon \epsilon^{1/2} \ln 
\lr{ 1 + z e^{-\beta \epsilon} }
=
- g \frac{V}{(2 \pi)^2 }
\lr{ \frac{2 m}{\Hbar^2} }
^{3/2}
\int_0^\infty d\epsilon \frac{2}{3} \epsilon^{3/2} \frac{-\beta z e^{-\beta \epsilon} }{ 1 + z e^{-\beta \epsilon} }
=
g
\frac{V}{(2 \pi)^2 }
\lr{ \frac{2 m}{\Hbar^2} }
^{3/2}
\frac{2}{3} 
\inv{\beta^{3/2}}
\int_0^\infty dx
\frac{x^{3/2}}{z^{-1} e^{x} + 1 }
=
g
\frac{2}{3} 
2 \pi V
\frac{
\lr{2 m \kB T}
^{3/2}
}
{h^3 }
\Gamma(5/2)
f_{5/2}(z)
=
g
\frac{2}{3} 
2 \pi V
\frac{
\lr{2 m \kB T}
^{3/2}
}
{h^3 }
\frac{3}{2} \frac{1}{2} \sqrt{\pi}
f_{5/2}(z)
=
g
V
\frac{
\lr{2 \pi m \kB T}
^{3/2}
}
{h^3 }
f_{5/2}(z),
\end{dmath}

or

\boxedEquation{eqn:fermiFunctionExpansionsForNandP:200}{
P \beta = \frac{g}{\lambda^3} f_{5/2}(z).
} 

\paragraph{Energy}

The average energy is the last thermodynamic quantity to come very easily.  We have

\begin{dmath}\label{eqn:fermiFunctionExpansionsForNandP:220}
U 
= - \PD{\beta}{} \ln \ZG
= - \PD{\beta}{T} \PD{T}{} \ln \ZG
= - \PD{\beta}{(1/\kB T)} \PD{T}{} P V \beta
= \inv{\kB \beta^2}
\PD{T}{} 
\frac{g V}{\lambda^3} f_{5/2}(z)
= g V \kB T^2
f_{5/2}(z)
\PD{T}{} 
\frac
{
\lr{ 2 \pi m \kB T }
^{3/2}}
{h^3}
= \frac{3}{2} \frac{g V \kB T}{\lambda^3}
f_{5/2}(z).
\end{dmath}

From \eqnref{eqn:fermiFunctionExpansionsForNandP:160}, we have

\begin{equation}\label{eqn:fermiFunctionExpansionsForNandP:240}
\frac{g V}{\lambda^3} 
=
\frac{N}{f_{3/2}(z) },
\end{equation}

so the energy takes the form

\boxedEquation{eqn:fermiFunctionExpansionsForNandP:260}{
U
= \frac{3}{2} N \kB T 
\frac{f_{5/2}(z)}{
f_{3/2}(z) 
}.
}

We can compare this to the ratio of pressure to density

\begin{dmath}\label{eqn:fermiFunctionExpansionsForNandP:280}
\frac{P \beta}{n} = 
\frac{f_{5/2}(z)}{
f_{3/2}(z) 
},
\end{dmath}

to find

\begin{dmath}\label{eqn:fermiFunctionExpansionsForNandP:300}
U
= \frac{3}{2} N \kB T \frac{P V \beta}{N}
= \frac{3}{2} P V,
\end{dmath}

or

\boxedEquation{eqn:fermiFunctionExpansionsForNandP:320}{
P V = \frac{2}{3} U.
}

%\EndArticle
