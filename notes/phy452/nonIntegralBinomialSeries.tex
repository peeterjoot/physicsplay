%
% Copyright � 2013 Peeter Joot.  All Rights Reserved.
% Licenced as described in the file LICENSE under the root directory of this GIT repository.
%
%\newcommand{\authorname}{Peeter Joot}
\newcommand{\email}{peeterjoot@protonmail.com}
\newcommand{\basename}{FIXMEbasenameUndefined}
\newcommand{\dirname}{notes/FIXMEdirnameUndefined/}

%\renewcommand{\basename}{nonIntegralBinomialSeries}
%\renewcommand{\dirname}{notes/phy452/}
%\newcommand{\keywords}{binomial coefficient, factorial, gamma function, Taylor expansion, binomial series}
%
%\newcommand{\authorname}{Peeter Joot}
\newcommand{\onlineurl}{http://sites.google.com/site/peeterjoot2/math2013/\basename.pdf}
\newcommand{\sourcepath}{\dirname\basename.tex}
\newcommand{\generatetitle}[1]{\chapter{#1}}

\newcommand{\vcsinfo}{%
\section*{}
\noindent{\color{DarkOliveGreen}{\rule{\linewidth}{0.1mm}}}
\paragraph{Document version}
%\paragraph{\color{Maroon}{Document version}}
{
\small
\begin{itemize}
\item Available online at:\\ 
\href{\onlineurl}{\onlineurl}
\item Git Repository: \input{./.revinfo/gitRepo.tex}
\item Source: \sourcepath
\item last commit: \input{./.revinfo/gitCommitString.tex}
\item commit date: \input{./.revinfo/gitCommitDate.tex}
\end{itemize}
}
}

%\PassOptionsToPackage{dvipsnames,svgnames}{xcolor}
\PassOptionsToPackage{square,numbers}{natbib}
\documentclass{scrreprt}

\usepackage[left=2cm,right=2cm]{geometry}
\usepackage[svgnames]{xcolor}
\usepackage{peeters_layout}

\usepackage{natbib}

\usepackage[
colorlinks=true,
bookmarks=false,
pdfauthor={\authorname, \email},
backref 
]{hyperref}

% http://tex.stackexchange.com/questions/75773/how-to-reference-problems-by-the-text-label-in-an-exercise-envioronment
\usepackage[english]{cleveref}
\crefname{Exercise}{exercise}{exercises}
\Crefname{Exercise}{Exercise}{Exercises}

\RequirePackage{titlesec}
\RequirePackage{ifthen}

% http://stackoverflow.com/questions/4932910/date-in-the-tabular-environment
\makeatletter
\let\insertdate\@date
\makeatother

\titleformat{\chapter}[display]
{\bfseries\Large}
{\color{DarkSlateGrey}\filleft \authorname
\ifthenelse{\isundefined{\studentnumber}}{}{\\ \studentnumber}
\ifthenelse{\isundefined{\email}}{}{\\ \email}
\ifthenelse{\isundefined{\dateintitle}}{}{\\ \insertdate}
%\ifthenelse{\isundefined{\coursename}}{}{\\ \coursename} % put in title instead.
}
{4ex}
{\color{DarkOliveGreen}{\titlerule}\color{Maroon}
\vspace{2ex}%
\filright}
[\vspace{2ex}%
\color{DarkOliveGreen}\titlerule
]

\newcommand{\beginArtWithToc}[0]{\begin{document}\tableofcontents}
\newcommand{\beginArtNoToc}[0]{\begin{document}}
\newcommand{\EndNoBibArticle}[0]{\end{document}}
\newcommand{\EndArticle}[0]{\bibliography{Bibliography}\bibliographystyle{plainnat}\end{document}}

% 
%\newcommand{\citep}[1]{\cite{#1}}

\colorSectionsForArticle


%
%\beginArtNoToc
%
%\generatetitle{Non integral binomial coefficient}
\label{chap:nonIntegralBinomialSeries}

In \citep{pathriastatistical} appendix \S F was the use of binomial coefficients in a non-integral binomial expansion.  This surprised me, since I'd never seen that before.  However, on reflection, this is a very sensible notation, provided the binomial coefficients are defined in terms of the gamma function.  Let's explore this little detail explicitly, starting with a Taylor expansion of

\begin{dmath}\label{eqn:nonIntegralBinomialSeries:20}
f(x) = (a + x)^b.
\end{dmath}

Our derivatives are

\begin{equation}\label{eqn:nonIntegralBinomialSeries:40}
\begin{aligned}
f'(x) &= b (a + x)^{b-1} \\
f''(x) &= b(b-1) (a + x)^{b-2} \\
f^3(x) &= b(b-1)(b-(3-1)) (a + x)^{b-3} \\
\vdots & \\
f^k(x) &= b(b-1)\cdots (b-(k-1)) (a + x)^{b-k}.
\end{aligned}
\end{equation}

Our Taylor series is then

\begin{dmath}\label{eqn:nonIntegralBinomialSeries:60}
(a + x)^b = \sum_{k = 0}^\infty \inv{k!} b(b-1)\cdots (b-(k-1)) a^{b-k} x^k.
\end{dmath}

If $b$ is integral, then all the elements of this series become zero at $b = k - 1$, or

\begin{dmath}\label{eqn:nonIntegralBinomialSeries:80}
(a + x)^b = \sum_{k = 0}^k \inv{k!} b(b-1)\cdots (b-(k-1)) a^{b-k} x^k.
\end{dmath}

Let's now relate this to the gamma function.  From \citep{abramowitz1964handbook} \S 6.1.1 we have

\begin{dmath}\label{eqn:nonIntegralBinomialSeries:100}
\Gamma(z) = \int_0^\infty t^{z-1} e^{-t} dt.
\end{dmath}

Iteratively integrating by parts, we find the usual relation between gamma functions of integral separation

\begin{dmath}\label{eqn:nonIntegralBinomialSeries:120}
\Gamma(z + 1) 
= \int_0^\infty t^{z} e^{-t} dt
= \int_0^\infty t^{z} d \lr{ \frac{ e^{-t}}{-1} }
= 
\evalrange{
t^z \frac{e^{-t}}{-1}
}{0}{\infty}
- \int_0^\infty z t^{z-1} \frac{e^{-t}}{-1} dt
=
z \int_0^\infty t^{z-1} e^{-t} dt
=
z (z - 1)\int_0^\infty t^{z-2} e^{-t} dt
=
z (z - 1)(z - (3-1))\int_0^\infty t^{z-3} e^{-t} dt
=
z (z - 1) \cdots (z - (k-1))\int_0^\infty t^{z-k} e^{-t} dt
=
z (z - 1) \cdots (z - (k-1))\int_0^\infty t^{(z + 1 - k) - 1} e^{-t} dt,
\end{dmath}

or

\begin{dmath}\label{eqn:nonIntegralBinomialSeries:140}
\Gamma(z + 1) = z (z - 1) \cdots (z - (k-1)) \Gamma( z - (k-1) ).
\end{dmath}

Flipping this gives us a nice closed form expression for the products of a number of unit separated values, integer or otherwise

\begin{dmath}\label{eqn:nonIntegralBinomialSeries:160}
z (z - 1) \cdots (z - k)
=
\frac{\Gamma(z + 1)}{\Gamma( z - (k-1) )}.
\end{dmath}

We can now use this in our Taylor expansion \eqnref{eqn:nonIntegralBinomialSeries:60}

\begin{dmath}\label{eqn:nonIntegralBinomialSeries:180}
(a + x)^b 
= \sum_{k = 0}^\infty \inv{k!} \frac{\Gamma(b+1)}{\Gamma{b - k + 1}}
a^{b-k} x^k
= \sum_{k = 0}^\infty \frac{\Gamma(b + 1)}{\Gamma(k + 1)\Gamma{b - k + 1}}
a^{b-k} x^k.
\end{dmath}

Observe that when $b$ is integral we have

\begin{equation}\label{eqn:nonIntegralBinomialSeries:200}
\frac{\Gamma(b + 1)}{\Gamma(k + 1)\Gamma{b - k + 1}}
=
\frac{b!}{k!(b-k)!}
= \binom{b}{k}.
\end{equation}

We see that is then very reasonable to define the binomial coefficient \index{binomial coefficient} explicitly in terms of the gamma function

\begin{equation}\label{eqn:nonIntegralBinomialSeries:220}
\binom{b}{k} \equiv
\frac{\Gamma(b + 1)}{\Gamma(k + 1)\Gamma(b - k + 1)}.
\end{equation}

If we do that, then the binomial expansion for non-integral values of $b$ is simply

\begin{dmath}\label{eqn:nonIntegralBinomialSeries:240}
(a + x)^b = \sum_{k = 0}^\infty \binom{b}{k} a^{b-k} x^k.
\end{dmath}

%\EndArticle
