%
% Copyright � 2013 Peeter Joot.  All Rights Reserved.
% Licenced as described in the file LICENSE under the root directory of this GIT repository.
%
%\newcommand{\authorname}{Peeter Joot}
\newcommand{\email}{peeterjoot@protonmail.com}
\newcommand{\basename}{FIXMEbasenameUndefined}
\newcommand{\dirname}{notes/FIXMEdirnameUndefined/}

%\renewcommand{\basename}{basicStatMechLecture18}
%\renewcommand{\dirname}{notes/phy452/}
%\newcommand{\keywords}{Statistical mechanics, PHY452H1S, Fermi gas, specific heat, density of states, graphene, relativistic gas, chemical potential, energy, Fermi distribution, hole, electron}
%\newcommand{\authorname}{Peeter Joot}
\newcommand{\onlineurl}{http://sites.google.com/site/peeterjoot2/math2013/\basename.pdf}
\newcommand{\sourcepath}{\dirname\basename.tex}
\newcommand{\generatetitle}[1]{\chapter{#1}}

\newcommand{\vcsinfo}{%
\section*{}
\noindent{\color{DarkOliveGreen}{\rule{\linewidth}{0.1mm}}}
\paragraph{Document version}
%\paragraph{\color{Maroon}{Document version}}
{
\small
\begin{itemize}
\item Available online at:\\ 
\href{\onlineurl}{\onlineurl}
\item Git Repository: \input{./.revinfo/gitRepo.tex}
\item Source: \sourcepath
\item last commit: \input{./.revinfo/gitCommitString.tex}
\item commit date: \input{./.revinfo/gitCommitDate.tex}
\end{itemize}
}
}

%\PassOptionsToPackage{dvipsnames,svgnames}{xcolor}
\PassOptionsToPackage{square,numbers}{natbib}
\documentclass{scrreprt}

\usepackage[left=2cm,right=2cm]{geometry}
\usepackage[svgnames]{xcolor}
\usepackage{peeters_layout}

\usepackage{natbib}

\usepackage[
colorlinks=true,
bookmarks=false,
pdfauthor={\authorname, \email},
backref 
]{hyperref}

% http://tex.stackexchange.com/questions/75773/how-to-reference-problems-by-the-text-label-in-an-exercise-envioronment
\usepackage[english]{cleveref}
\crefname{Exercise}{exercise}{exercises}
\Crefname{Exercise}{Exercise}{Exercises}

\RequirePackage{titlesec}
\RequirePackage{ifthen}

% http://stackoverflow.com/questions/4932910/date-in-the-tabular-environment
\makeatletter
\let\insertdate\@date
\makeatother

\titleformat{\chapter}[display]
{\bfseries\Large}
{\color{DarkSlateGrey}\filleft \authorname
\ifthenelse{\isundefined{\studentnumber}}{}{\\ \studentnumber}
\ifthenelse{\isundefined{\email}}{}{\\ \email}
\ifthenelse{\isundefined{\dateintitle}}{}{\\ \insertdate}
%\ifthenelse{\isundefined{\coursename}}{}{\\ \coursename} % put in title instead.
}
{4ex}
{\color{DarkOliveGreen}{\titlerule}\color{Maroon}
\vspace{2ex}%
\filright}
[\vspace{2ex}%
\color{DarkOliveGreen}\titlerule
]

\newcommand{\beginArtWithToc}[0]{\begin{document}\tableofcontents}
\newcommand{\beginArtNoToc}[0]{\begin{document}}
\newcommand{\EndNoBibArticle}[0]{\end{document}}
\newcommand{\EndArticle}[0]{\bibliography{Bibliography}\bibliographystyle{plainnat}\end{document}}

% 
%\newcommand{\citep}[1]{\cite{#1}}

\colorSectionsForArticle


%
%\beginArtNoToc
%\generatetitle{PHY452H1S Basic Statistical Mechanics.  Lecture 18: Fermi gas thermodynamics.  Taught by Prof.\ Arun Paramekanti}
%\chapter{Fermi gas thermodynamics}
\label{chap:basicStatMechLecture18}

%\section{Disclaimer}
%
%Peeter's lecture notes from class.  May not be entirely coherent.
%
\paragraph{Review}

Last time we found that the low temperature behavior or the chemical potential was quadratic as in \cref{fig:lecture18:lecture18Fig1}.

\begin{dmath}\label{eqn:basicStatMechLecture18:20}
\mu =
\mu(0) - a \frac{T^2}{T_{\mathrm{F}}}
%^
%\kB T ?
\end{dmath}

\imageFigure{figures/lecture18Fig1}{Fermi gas chemical potential}{fig:lecture18:lecture18Fig1}{0.2}

\paragraph{Specific heat}

\begin{dmath}\label{eqn:basicStatMechLecture18:40}
E = \sum_\Bk n_{\mathrm{F}}(\epsilon_\Bk, T) \epsilon_\Bk
\end{dmath}

\begin{dmath}\label{eqn:basicStatMechLecture18:60}
\frac{E}{V}
= \inv{(2\pi)^3} \int d^3 \Bk n_{\mathrm{F}}(\epsilon_\Bk, T) \epsilon_\Bk
= \int d\epsilon N(\epsilon) n_{\mathrm{F}}(\epsilon, T) \epsilon,
\end{dmath}

where

\begin{dmath}\label{eqn:basicStatMechLecture18:80}
N(\epsilon) = \inv{4 \pi^2}
\lr{\frac{2m}{\hbar^2}}
^{3/2}
\sqrt{\epsilon}.
\end{dmath}

\paragraph{Low temperature $\CV$}

\begin{dmath}\label{eqn:basicStatMechLecture18:100}
\frac{\Delta E(T)}{V}
=
\int_0^\infty d\epsilon N(\epsilon)
\lr{ n_{\mathrm{F}}(\epsilon, T) - n_{\mathrm{F}}(\epsilon, 0)}
\end{dmath}

The only change in the distribution \cref{fig:lecture18:lecture18Fig2}, that is of interest is over the step portion of the distribution, and over this range of interest $N(\epsilon)$ is approximately constant as in \cref{fig:lecture18:lecture18Fig3}.

\imageFigure{figures/lecture18Fig2}{Fermi distribution}{fig:lecture18:lecture18Fig2}{0.2}
\imageFigure{figures/lecture18Fig3}{Fermi gas density of states}{fig:lecture18:lecture18Fig3}{0.2}

\begin{subequations}
\begin{dmath}\label{eqn:basicStatMechLecture18:120}
N(\epsilon) \approx N(\mu)
\end{dmath}
\begin{dmath}\label{eqn:basicStatMechLecture18:140}
\mu \approx \epsilon_{\mathrm{F}},
\end{dmath}
\end{subequations}

so that
\begin{dmath}\label{eqn:basicStatMechLecture18:160}
\Delta e \equiv
\frac{\Delta E(T)}{V}
\approx
N(\epsilon_{\mathrm{F}})
\int_0^\infty d\epsilon
\lr{ n_{\mathrm{F}}(\epsilon, T) - n_{\mathrm{F}}(\epsilon, 0)}
=
N(\epsilon_{\mathrm{F}})
\int_{-\epsilon_{\mathrm{F}}}^\infty d x (\epsilon_{\mathrm{F}} + x)
\lr{ n_{\mathrm{F}}(\epsilon + x, T) - n_{\mathrm{F}}(\epsilon_{\mathrm{F}} + x, 0)}.
\end{dmath}

Here we've made a change of variables $\epsilon = \epsilon_{\mathrm{F}} + x$, so that we have near cancellation of the $\epsilon_{\mathrm{F}}$ factor

\begin{dmath}\label{eqn:basicStatMechLecture18:180}
\Delta e 
=
N(\epsilon_{\mathrm{F}})
\epsilon_{\mathrm{F}}
\int_{-\epsilon_{\mathrm{F}}}^\infty d x 
\mathLabelBox{
\lr{ n_{\mathrm{F}}(\epsilon + x, T) - n_{\mathrm{F}}(\epsilon_{\mathrm{F}} + x, 0)}
}{almost equal everywhere}
+
N(\epsilon_{\mathrm{F}})
\int_{-\epsilon_{\mathrm{F}}}^\infty d x x
\lr{ n_{\mathrm{F}}(\epsilon + x, T) - n_{\mathrm{F}}(\epsilon_{\mathrm{F}} + x, 0)}
\approx
N(\epsilon_{\mathrm{F}})
\int_{-\infty}^\infty d x x
\lr{ 
\inv{ e^{\beta x} +1 }
-
\evalbar{\inv{ e^{\beta x} +1 }}{T \rightarrow 0}
}.
\end{dmath}

Here we've extended the integration range to $-\infty$ since this doesn't change much.  FIXME: justify this to myself?  Taking derivatives with respect to temperature we have

\begin{dmath}\label{eqn:basicStatMechLecture18:200}
\frac{\delta e}{T}
= 
-N(\epsilon_{\mathrm{F}})
\int_{-\infty}^\infty d x x
\inv{(e^{\beta x} + 1)^2}
\frac{d}{dT} e^{\beta x}
= 
N(\epsilon_{\mathrm{F}})
\int_{-\infty}^\infty d x x
\inv{(e^{\beta x} + 1)^2}
e^{\beta x}
\frac{x}{\kB T^2}
\end{dmath}

With $\beta x = y$, we have for $T \ll T_{\mathrm{F}}$

\begin{dmath}\label{eqn:basicStatMechLecture18:220}
\frac{C}{V} 
= 
N(\epsilon_{\mathrm{F}})
\int_{-\infty}^\infty \frac{ dy y^2 e^y }{ (e^y + 1)^2 \kB T^2} (\kB T)^3
= 
N(\epsilon_{\mathrm{F}}) \kB^2 T
\mathLabelBox
[
   labelstyle={xshift=2cm},
   linestyle={out=270,in=90, latex-}
]
{
\int_{-\infty}^\infty \frac{ dy y^2 e^y }{ (e^y + 1)^2 } 
}{$\pi^2/3$}
= 
\frac{\pi^2}{3} N(\epsilon_{\mathrm{F}}) \kB (\kB T).
\end{dmath}

Using \eqnref{eqn:basicStatMechLecture18:80} at the Fermi energy and

\begin{subequations}
\begin{dmath}\label{eqn:basicStatMechLecture18:240}
\frac{N}{V} = \rho
\end{dmath}
\begin{dmath}\label{eqn:basicStatMechLecture18:260}
\epsilon_{\mathrm{F}} = \frac{\hbar^2 \kF^2}{2 m}
\end{dmath}
\begin{dmath}\label{eqn:basicStatMechLecture18:280}
\kF = \lr{6 \pi^2 \rho}
^{1/3},
\end{dmath}
\end{subequations}

we have

\begin{dmath}\label{eqn:basicStatMechLecture18:320}
N(\epsilon_{\mathrm{F}}) 
= \inv{4 \pi^2}
\lr{\frac{2m}{\hbar^2}}
^{3/2}
\sqrt{\epsilon_{\mathrm{F}}}
= \inv{4 \pi^2}
\lr{\frac{2m}{\hbar^2}}
^{3/2}
\frac{\hbar \kF}{\sqrt{2m}}
= \inv{4 \pi^2}
\lr{\frac{2m}{\hbar^2}}
^{3/2}
\frac{\hbar }{\sqrt{2m}} \lr{6 \pi^2 \rho}^{1/3}
= 
\inv{4 \pi^2}
\lr{\frac{2m}{\hbar^2}}
\lr{6 \pi^2 \frac{N}{V}}^{1/3}
\end{dmath}

Giving

\begin{dmath}\label{eqn:basicStatMechLecture18:480}
\frac{C}{N} 
= 
\frac{\pi^2}{3} 
\frac{V}{N}
\inv{4 \pi^2}
\lr{\frac{2m}{\hbar^2}}
\lr{6 \pi^2 \frac{N}{V}}
^{1/3}
\kB (\kB T)
= 
\lr{\frac{m}{6 \hbar^2}}
\lr{\frac{V}{N}}^{2/3}
\lr{6 \pi^2}
^{1/3}
\kB (\kB T)
= 
\lr{\frac{ \pi^2 m}{3 \hbar^2}}
\lr{\frac{V}{\pi^2 N}}^{2/3}
\kB (\kB T)
= 
\lr{\frac{ \pi^2 m}{\hbar^2}}
\frac{\hbar^2}{2 m \epsilon_{\mathrm{F}}}
\kB (\kB T),
\end{dmath}

or

\begin{dmath}\label{eqn:basicStatMechLecture18:300}
\myBoxed{
\frac{C}{N} = 
\frac{\pi^2}{2} \kB \frac{ \kB T}{\epsilon_{\mathrm{F}}}.
}
\end{dmath}

This is illustrated in \cref{fig:lecture18:lecture18Fig4}.

\imageFigure{figures/lecture18Fig4}{Specific heat per Fermion}{fig:lecture18:lecture18Fig4}{0.25}

\paragraph{Relativistic gas}

\begin{itemize}
\item Relativistic gas

\begin{dmath}\label{eqn:basicStatMechLecture18:340}
\epsilon_\Bk = \pm \hbar v \Abs{\Bk}.
\end{dmath}
\begin{dmath}\label{eqn:basicStatMechLecture18:360}
\epsilon = \sqrt{(m_0 c^2)^2 + c^2 (\hbar \Bk)^2}
\end{dmath}

\item graphene

\item massless Dirac Fermion

%\cref{fig:lecture18:lecture18Fig5}.
\imageFigure{figures/lecture18Fig5}{Relativistic gas energy distribution}{fig:lecture18:lecture18Fig5}{0.3}

We can think of this state distribution in a condensed matter view, where we can have a hole to electron state transition by supplying energy to the system (i.e. shining light on the substrate).  This can also be thought of in a relativistic particle view where the same state transition can be thought of as a positron electron pair transition.  A round trip transition will have to supply energy like $2 m_0 c^2$ as illustrated in \cref{fig:lecture18:lecture18Fig6}.

\imageFigure{figures/lecture18Fig6}{Hole to electron round trip transition energy requirement}{fig:lecture18:lecture18Fig6}{0.2}

\end{itemize}

\paragraph{Graphene}

Consider graphene, a 2D system.  We want to determine the density of states $N(\epsilon)$, 

\begin{dmath}\label{eqn:basicStatMechLecture18:380}
\int \frac{d^2 \Bk}{(2 \pi)^2} \rightarrow \int_{-\infty}^\infty d\epsilon N(\epsilon),
\end{dmath}

We'll find a density of states distribution like \cref{fig:lecture18:lecture18Fig7}.

\imageFigure{figures/lecture18Fig7}{Density of states for 2D linear energy momentum distribution}{fig:lecture18:lecture18Fig7}{0.2}

\begin{dmath}\label{eqn:basicStatMechLecture18:400}
N(\epsilon) = \text{constant factor} \frac{\Abs{\epsilon}}{v},
\end{dmath}

\begin{dmath}\label{eqn:basicStatMechLecture18:420}
C \sim \frac{d}{dT} \int N(\epsilon) n_{\mathrm{F}}(\epsilon) \epsilon d\epsilon,
\end{dmath}

\begin{dmath}\label{eqn:basicStatMechLecture18:440}
\Delta E 
\sim 
\mathLabelBox
{
T}{window}
\times
\mathLabelBox
[
   labelstyle={below of=m\themathLableNode, below of=m\themathLableNode}
]
{
T}{energy}
\times
\mathLabelBox
[
   labelstyle={xshift=2cm},
   linestyle={out=270,in=90, latex-}
]
{
T}{number of states}
\sim T^3
\end{dmath}

so that

\begin{dmath}\label{eqn:basicStatMechLecture18:460}
C_{\mathrm{Dimensionless}} \sim T^2
\end{dmath}

%\EndNoBibArticle
