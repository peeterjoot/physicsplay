%
% Copyright � 2013 Peeter Joot.  All Rights Reserved.
% Licenced as described in the file LICENSE under the root directory of this GIT repository.
%
\makeoproblem{Relativisitic Fermi gas}{basicStatMech:problemSet6:5}{\citep{huang2001introduction}, pr 9.3}{
Consider a relativisitic gas of $N$ particles of spin $1/2$ obeying Fermi statistics, enclosed in volume $V$, at absolute zero.  The energy-momentum relation is $E = \sqrt{(p c)^2 + 
\lr{m c^2}
^2}$, where $m$ is the rest mass.
\makesubproblem{}{basicStatMech:problemSet6:5a}
Find the Fermi energy at density $n$.
\makesubproblem{}{basicStatMech:problemSet6:5b}
Define the internal energy $U$ as the average $E - m c^2$, and the pressure $P$ as the average force per unit area exerted on a perfectly-reflecting wall of the container.  Set up expressions for these quantities in the form of integrals, but you need not evaluate them.
\makesubproblem{}{basicStatMech:problemSet6:5c}
Show that $P C = 2 U/3$ at low densities, and $P V = U/3$ at high densities.  State the criteria for low and high densities.
\makesubproblem{}{basicStatMech:problemSet6:5d}
There may exist a gas of neutinos (and/or antineutinos) in the cosmos.  (Neutrinos are massless fermions of spin $1/2$.)  Calculate the Fermi energy (in eV) of such a gas, assuming a density of one particle per $\text{cm}^3$.
} % makeoproblem

\makeanswer{basicStatMech:problemSet6:5}{ 
\makeSubAnswer{}{basicStatMech:problemSet6:5a}

We've found that the density of states associated with a 3D relativisitic system is

\begin{equation}\label{eqn:huang93:20}
\mathcal{D}(p) = \frac{4 \pi V}{(c h)^3} \epsilon \sqrt{\epsilon^2 - 
\lr{m c^2}
^2},
\end{equation}

For a given density $\rho$, we can find the Fermi energy in the same way as we did for the non-relativisitic energies, with the exception that we have to integrate from a lowest energy of $m c^2$ instead of $0$ (the energy at $\Bp = 0$).  That is

\begin{dmath}\label{eqn:huang93:40}
\rho 
= \frac{N}{V} 
= 
\lr{2 \inv{2} + 1}
\frac{4 \pi}{(c h)^3} \int_{m c^2}^{\epsilon_{\mathrm{F}}}
d\epsilon \epsilon \sqrt{ \epsilon^2 - 
\lr{m c^2}^2
}
= \frac{2 \pi}{(c h)^3} 
\inv{3} \evalrange{
\lr{x^2 - 
\lr{m c^2}^2
}
^{3/2}
}{m c^2}{\epsilon_{\mathrm{F}}}
= \frac{2 \pi}{3 (c h)^3} 
\lr{\epsilon_{\mathrm{F}}^2 - 
\lr{m c^2}^2
}
^{3/2}.
\end{dmath}

Solving for $\epsilon_{\mathrm{F}}$ we have
\begin{dmath}\label{eqn:huang93:60}
\epsilon_{\mathrm{F}} = 
\sqrt{
\lr{ \frac{3 (c h)^3 \rho}{2 \pi} }
^{2/3}
+ \lr{m c^2}^2
}.
\end{dmath}

\makeSubAnswer{}{basicStatMech:problemSet6:5b}

TODO.
\makeSubAnswer{}{basicStatMech:problemSet6:5c}

TODO.
\makeSubAnswer{}{basicStatMech:problemSet6:5d}

\begin{dmath}\label{eqn:huang93:80}
\evalbar{\epsilon_{\mathrm{F}}}{\rho = 1/(0.01)^3}
%ef[3 10^8, 6.62606957 10^(-34), rho, 0]
%= 6.12402 \times 10^{-35} \text{J}
= 6.12402 \times 10^{-35} \text{J} \times 6.24150934 \times 10^{18} \frac{\text{eV}}{\text{J}}
= 3.82231 \times 10^{-16} \text{eV}
\end{dmath}

Wow.  That's pretty low!
}
