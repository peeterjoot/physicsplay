%
% Copyright � 2013 Peeter Joot.  All Rights Reserved.
% Licenced as described in the file LICENSE under the root directory of this GIT repository.
%
\makeoproblem{description}{basicStatMech:problemSet6:5}{\citep{huang2001introduction}, pr 9.3}{
Consider a relativisitic gas of $N$ particles of spin $1/2$ obeying Fermi statistics, enclosed in volume $V$, at absolute zero.  The energy-momentum relation is $E = \sqrt{(p c)^2 + 
\lr{m c^2}
^2}$, where $m$ is the rest mass.
\makesubproblem{}{basicStatMech:problemSet6:5a}
Find the Fermi energy at density $n$.
\makesubproblem{}{basicStatMech:problemSet6:5b}
Define the internal energy $U$ as the average $E - m c^2$, and the pressure $P$ as the average force per unit area exerted on a perfectly-reflecting wall of the container.  Set up expressions for these quantities in the form of integrals, but you need not evaluate them.
\makesubproblem{}{basicStatMech:problemSet6:5c}
Show that $P C = 2 U/3$ at low densities, and $P V = U/3$ at high densities.  State the criteria for low and high densities.
\makesubproblem{}{basicStatMech:problemSet6:5d}
There may exist a gas of neutinos (and/or antineutinos) in the cosmos.  (Neutrinos are massless fermions of spin $1/2$.)  Calculate the Fermi energy (in eV) of such a gas, assuming a density of one particle per $\text{cm}^3$.
} % makeoproblem

\makeanswer{basicStatMech:problemSet6:5}{ 
\makeSubAnswer{}{basicStatMech:problemSet6:5a}

We've found that the density of states associated with a 3D relativisitic system is

\DD(p) = \frac{4 \pi V}{(c h)^3} \ee \sqrt{\ee^2 - 
\lr{m c^2}
^2}

TODO.
\makeSubAnswer{}{basicStatMech:problemSet6:5b}

TODO.
\makeSubAnswer{}{basicStatMech:problemSet6:5c}

TODO.
\makeSubAnswer{}{basicStatMech:problemSet6:5d}

TODO.
}

