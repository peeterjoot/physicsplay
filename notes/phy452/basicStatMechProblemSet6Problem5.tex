%
% Copyright � 2013 Peeter Joot.  All Rights Reserved.
% Licenced as described in the file LICENSE under the root directory of this GIT repository.
%
\makeoproblem{Relativisitic Fermi gas}{basicStatMech:problemSet6:5}{\citep{huang2001introduction}, pr 9.3}{
Consider a relativisitic gas of $N$ particles of spin $1/2$ obeying Fermi statistics, enclosed in volume $V$, at absolute zero.  The energy-momentum relation is
\begin{dmath}\label{eqn:huang93:100}
\epsilon = \sqrt{(p c)^2 + \epsilon_0^2
},
\end{dmath}

where $\epsilon_0 = m c^2$, and $m$ is the rest mass.

\makesubproblem{}{basicStatMech:problemSet6:5a}
Find the Fermi energy at density $n$.
\makesubproblem{}{basicStatMech:problemSet6:5b}
With the pressure $P$ defined as the average force per unit area exerted on a perfectly-reflecting wall of the container.  
Set up expressions for this in the form of an integral.
\makesubproblem{}{basicStatMech:problemSet6:5be}
Define the internal energy $U$ as the average $\epsilon - \epsilon_0$.
Set up expressions for this in the form of an integral.

\makesubproblem{}{basicStatMech:problemSet6:5c}
Show that $P V = 2 U/3$ at low densities, and $P V = U/3$ at high densities.  State the criteria for low and high densities.
\makesubproblem{}{basicStatMech:problemSet6:5d}
There may exist a gas of neutrinos (and/or antineutrinos) in the cosmos.  (Neutrinos are massless Fermions of spin $1/2$.)  Calculate the Fermi energy (in eV) of such a gas, assuming a density of one particle per $\text{cm}^3$.
} % makeoproblem

\makeanswer{basicStatMech:problemSet6:5}{
\makeSubAnswer{}{basicStatMech:problemSet6:5a}

We've found \citep{phy452:relativisiticDensityOfStates} that the density of states associated with a 3D relativisitic system is

\begin{equation}\label{eqn:huang93:20}
\mathcal{D}(\epsilon) = \frac{4 \pi V}{(c h)^3} \epsilon \sqrt{\epsilon^2 -
\epsilon_0
^2},
\end{equation}

For a given density $n$, we can find the Fermi energy in the same way as we did for the non-relativisitic energies, with the exception that we have to integrate from a lowest energy of $\epsilon_0$ instead of $0$ (the energy at $\Bp = 0$).  That is

\begin{dmath}\label{eqn:huang93:40}
n
= \frac{N}{V}
=
\lr{2 \inv{2} + 1}
\frac{4 \pi}{(c h)^3} \int_{\epsilon_0}^{\epsilon_{\mathrm{F}}}
d\epsilon \epsilon \sqrt{ \epsilon^2 -
\epsilon_0^2
}
= \frac{8 \pi}{(c h)^3}
\inv{3} \evalrange{
\lr{x^2 -
\epsilon_0^2
}
^{3/2}
}{\epsilon_0}{\epsilon_{\mathrm{F}}}
= \frac{8 \pi}{3 (c h)^3}
\lr{\epsilon_{\mathrm{F}}^2 -
\epsilon_0^2
}
^{3/2}.
\end{dmath}

Solving for $\epsilon_{\mathrm{F}}/\epsilon_0$ we have

\begin{dmath}\label{eqn:huang93:60}
\frac{\epsilon_{\mathrm{F}}}{\epsilon_0} 
=
\sqrt{
\lr{ \frac{3 (c h)^3 n}{8 \pi \epsilon_0^3} }
^{2/3}
+ 1
}.
\end{dmath}

We'll see the constant factor above a number of times below and designate it

\begin{dmath}\label{eqn:huang93:620}
n_0 = \frac{8 \pi}{3} \lr{ \frac{\epsilon_0}{c h} }^3,
\end{dmath}

so that the Fermi energy is

\begin{dmath}\label{eqn:huang93:700}
\frac{\epsilon_{\mathrm{F}}}{\epsilon_0} 
=
\sqrt{
\lr{\frac{n}{n_0}}
^{2/3}
+ 1
}.
\end{dmath}

\makeSubAnswer{}{basicStatMech:problemSet6:5b}

For the pressure calculation, let's suppose that we have a configuration with a plane in the $x,y$ orientation as in \cref{fig:huang93:huang93Fig1}.

\imageFigure{../blogit/huang93Fig1}{Pressure against $x,y$ oriented plane}{fig:huang93:huang93Fig1}{0.3}

It's argued in \citep{pathriastatistical} \S 6.4 that the pressure for such a configuration is

\begin{dmath}\label{eqn:huang93:120}
P = n \int p_z u_z f(\Bu) d^3 \Bu,
\end{dmath}

where $n$ is the number density and $f(\Bu)$ is a normalized distribution function for the velocities.  The velocity and momentum components are related by the Hamiltonian equations.  From the Hamiltonian \eqnref{eqn:huang93:100} we find \footnote{ Observe that by squaring and summing one can show that this is equivalent to the standard relativisitic momentum $p_x = \frac{m v_x}{\sqrt{ 1 - \Bu^2/c^2}}$.} (for the x-component which is representative)

\begin{dmath}\label{eqn:huang93:140}
u_x
%=
%\xdot
= \PD{p_x}{\epsilon}
= \PD{p_x}{}
\sqrt{(p c)^2 +
\epsilon_0
^2}
=
\frac{ p_x c^2 }{
\sqrt{(p c)^2 +
\epsilon_0
^2}
}.
\end{dmath}

For $\alpha \in \{1, 2, 3\}$ we can summarize these velocity-momentum relationships as

\begin{dmath}\label{eqn:huang93:220}
\frac{u_\alpha}{c} = \frac{ c p_\alpha }{ \epsilon }.
\end{dmath}

Should we attempt to calculate the pressure with this parameterization of the velocity space we end up with convergence problems, and can't express the results in terms of $f^+_\nu(z)$.  Let's try instead with a distribution over momentum space

\begin{dmath}\label{eqn:huang93:340}
P
=
n \int \frac{(c p_z)^2}{\epsilon} f(c \Bp) d^3 (c \Bp).
\end{dmath}

Here the momenta have been scaled to have units of energy since we want to express this integral in terms of energy in the end.  Our normalized distribution function is

\begin{dmath}\label{eqn:huang93:360}
f(c \Bp)
\propto \frac{
\inv{ z^{-1} e^{\beta \epsilon} + 1 }}
{\int \inv{ z^{-1} e^{\beta \epsilon} + 1 } d^3 (c \Bp)},
\end{dmath}

but before evaluating anything, we first want to change our integration variable from momentum to energy.  In spherical coordinates our volume element takes the form

\begin{dmath}\label{eqn:huang93:380}
d^3 (c \Bp)
= 2 \pi (c p)^2 d (c p) \sin\theta d\theta
= 2 \pi (c p)^2 \frac{d (cp)}{d \epsilon} d \epsilon \sin\theta d\theta.
\end{dmath}

Implicit derivatives of

\begin{dmath}\label{eqn:huang93:240}
c^2 p^2 = \epsilon^2 - \epsilon_0^2
,
\end{dmath}

gives us
%\begin{dmath}\label{eqn:huang93:260}
%2 c p \frac{d (c p)}{d\epsilon} = 2 \epsilon,
%\end{dmath}
%
%or

\begin{equation}\label{eqn:huang93:280}
\frac{d (c p)}{d\epsilon}
= \frac{\epsilon}{c p}
=
\frac{\epsilon}{\sqrt{\epsilon^2 -
\epsilon_0^2
}}
.
\end{equation}

Our momentum volume element becomes

\begin{dmath}\label{eqn:huang93:400}
d^3 (c \Bp)
=
2 \pi (c p)^2 \frac{\epsilon}{\sqrt{\epsilon^2 - \epsilon_0^2 }}
d \epsilon \sin\theta d\theta
=
2 \pi \lr{ \epsilon^2 - \epsilon_0^2} \frac{\epsilon}{\sqrt{\epsilon^2 - \epsilon_0^2 }}
d \epsilon \sin\theta d\theta
=
2 \pi \epsilon \sqrt{ \epsilon^2 - \epsilon_0^2} d \epsilon \sin\theta d\theta.
\end{dmath}

For our distribution function, we can now write

\begin{dmath}\label{eqn:huang93:420}
f(c \Bp) d^3 (c \Bp)
= C 
\frac
{
\epsilon \sqrt{ \epsilon^2 - \epsilon_0^2} d \epsilon 
}
{ z^{-1} e^{\beta \epsilon} + 1 }
\frac{ 2 \pi \sin\theta d\theta }{ 4 \pi \epsilon_0^3 },
\end{dmath}

where $C$ is determined by the requirement $\int f(c \Bp) d^3 (c \Bp) = 1$

\begin{dmath}\label{eqn:huang93:600}
C^{-1} = 
\int_{0}^\infty 
\frac{(y + 1)\sqrt{ (y + 1)^2 - 1} dy }
{ z^{-1} e^{\beta \epsilon_0 (y + 1)} + 1 }.
\end{dmath}

The z component of our momentum can be written in spherical coordinates as

\begin{equation}\label{eqn:huang93:440}
(c p_z)^2
= (c p)^2 \cos^2\theta
= \lr{ \epsilon^2 - \epsilon_0^2}
\cos^2\theta,
\end{equation}

Noting that

\begin{equation}\label{eqn:huang93:460}
\int_0^\pi \cos^2\theta \sin\theta d\theta =
-\int_0^\pi \cos^2\theta d(\cos\theta)
= \frac{2}{3},
\end{equation}

all the bits come together as

\begin{dmath}\label{eqn:huang93:480}
P
= \frac{C n}{3 \epsilon_0^3 }
   \int_{\epsilon_0}^\infty
   \lr{ \epsilon^2 - \epsilon_0^2}^{3/2}
   \inv{ z^{-1} e^{\beta \epsilon} + 1 }
   d \epsilon
= \frac{n \epsilon_0}{3}
   \int_{1}^\infty
   \lr{ x^2 - 1}^{3/2}
   \inv{ z^{-1} e^{\beta \epsilon_0 x} + 1 }
   dx.
\end{dmath}

Letting $y = x - 1$, this is

\begin{dmath}\label{eqn:huang93:500}
P
= \frac{C n \epsilon_0}{3}
   \int_{0}^\infty
   \frac{ \lr{ (y + 1)^2 - 1}^{3/2} }
   { z^{-1} e^{\beta \epsilon_0 (y + 1)} + 1 }
   dy.
\end{dmath}

We could concievable expand the numerators of each of these integrals in power series, which could then be evaluated as a sum of $f^+_\nu(z e^{-\beta \epsilon_0})$ terms.

Note that above the Fermi energy $n$ also has an integral representation

\begin{dmath}\label{eqn:huang93:560}
n 
= 
\lr{2\lr{\inv{2}} + 1}
\int_{\epsilon_0}^\infty d\epsilon \mathcal{D}(\epsilon) 
\inv
{
   z^{-1} e^{\beta \epsilon} + 1
}
= 
\frac{8 \pi}{(c h)^3} 
\int_{\epsilon_0}^\infty d\epsilon
\frac{
\epsilon \sqrt{\epsilon^2 - \epsilon_0 ^2} 
}
{
   z^{-1} e^{\beta \epsilon} + 1
}
= 
\frac{8 \pi \epsilon_0^3}{(c h)^3} 
\int_{0}^\infty dy
\frac{
(y + 1)\sqrt{(y + 1)^2 - 1} 
}
{
   z^{-1} e^{\beta \epsilon_0 (y + 1)} + 1
},
\end{dmath}

or

\begin{equation}\label{eqn:huang93:580}
\myBoxed{
n 
= \frac{3 n_0}{C}.
}
\end{equation}

Observe that we can use this result to remove the dependence of pressure on this constant $C$

\begin{equation}\label{eqn:huang93:640}
\myBoxed{
\frac{P}{n_0 \epsilon_0}
= 
   \int_{0}^\infty dy
   \frac{ \lr{ (y + 1)^2 - 1}^{3/2} }
   { z^{-1} e^{\beta \epsilon_0 (y + 1)} + 1 }
.
}
\end{equation}

\makeSubAnswer{}{basicStatMech:problemSet6:5be}

Now for the average energy difference from the rest energy $\epsilon_0$

\begin{dmath}\label{eqn:huang93:520}
U 
= \expectation{\epsilon - \epsilon_0} 
= 
\int_{\epsilon_0}^\infty d\epsilon \mathcal{D}(\epsilon) f(\epsilon) (\epsilon - \epsilon_0)
=
\frac{8 \pi V}{(c h)^3}
\int_{\epsilon_0}^\infty d\epsilon 
\frac
{
   \epsilon(\epsilon - \epsilon_0) \sqrt{ \epsilon^2 - \epsilon_0 } 
}
{ 
   z^{-1} e^{\beta \epsilon} + 1
}
=
\frac{8 \pi V \epsilon_0^4}{(c h)^3}
\int_{0}^\infty dy
\frac
{
   y ( y - 1 ) \sqrt{ (y + 1)^2 - 1 }
}
{ 
   z^{-1} e^{\beta \epsilon} + 1
}.
\end{dmath}

So the average energy density difference from the rest energy, relative to the rest energy, is

\begin{equation}\label{eqn:huang93:540}
\myBoxed{
\frac{\expectation{\epsilon - \epsilon_0}}{V \epsilon_0} 
=
3 n_0
   \int_{0}^\infty dy
   \frac
   {
      y (y + 1)\sqrt{(y + 1)^2 - 1} 
   }
   {
      z^{-1} e^{\beta \epsilon_0 (y + 1)} + 1
   }
.
}
\end{equation}

\makeSubAnswer{}{basicStatMech:problemSet6:5c}

From \eqnref{eqn:huang93:640} and \eqnref{eqn:huang93:540} we have

\begin{dmath}\label{eqn:basicStatMechProblemSet6Problem5:660}
\inv{n_0}
=
3 \frac
{V \epsilon_0} 
{\expectation{\epsilon - \epsilon_0}}
   \int_{0}^\infty
   \frac
   {
      y (y + 1)\sqrt{(y + 1)^2 - 1} dy
   }
   {
      z^{-1} e^{\beta \epsilon_0 (y + 1)} + 1
   }
=
\frac
{\epsilon_0}
{P}
   \int_{0}^\infty
   \frac{ \lr{ (y + 1)^2 - 1}^{3/2} }
   { z^{-1} e^{\beta \epsilon_0 (y + 1)} + 1 }
   dy,
\end{dmath}

or

\begin{dmath}\label{eqn:basicStatMechProblemSet6Problem5:680}
P V 
=
\frac{U}{3}
\frac
{
   \int_{0}^\infty
   \frac{ \lr{ (y + 1)^2 - 1}^{3/2} }
   { z^{-1} e^{\beta \epsilon_0 (y + 1)} + 1 }
   dy
}
{
   \int_{0}^\infty
   \frac
   {
      y (y + 1)\sqrt{(y + 1)^2 - 1} dy
   }
   {
      z^{-1} e^{\beta \epsilon_0 (y + 1)} + 1
   }
}.
\end{dmath}

This ratio of integrals is supposed to resolve to 1 and 2 in the low and high density limits.  To consider this let's perform one final non-dimensionalization, writing

\begin{equation}\label{eqn:huang93:720}
\begin{aligned} \\
x &= \beta \epsilon_0 y \\
\theta &= \inv{\beta \epsilon_0} = \frac{\kB T}{\epsilon_0} \\
\barmu &= \mu - \epsilon_0 \\
\barz &= e^{\beta \barmu}.
\end{aligned}
\end{equation}

The density, pressure, and energy take the form

\begin{subequations}
\begin{equation}\label{eqn:huang93:740}
\frac{n}{n_0}
= 
3 \theta
\int_{0}^\infty dx
\frac{
(\theta x + 1)\sqrt{(\theta x + 1)^2 - 1} 
}
{
   \barz^{-1} e^{x} + 1
}
\end{equation}
\begin{equation}\label{eqn:huang93:760}
\frac{P}{n_0 \epsilon_0}
= 
   \theta
   \int_{0}^\infty
   dx
   \frac{ \lr{ (\theta x + 1)^2 - 1}^{3/2} }
   { \barz^{-1} e^{x} + 1 }
\end{equation}
\begin{equation}\label{eqn:huang93:780}
\frac{\expectation{\epsilon - \epsilon_0}}{V \epsilon_0 n_0} 
=
3 \theta^2
   \int_{0}^\infty dx
   \frac
   {
      x (\theta x + 1)\sqrt{(\theta x + 1)^2 - 1} 
   }
   {
      \barz^{-1} e^{x} + 1
   }
.
\end{equation}
\end{subequations}

We can rewrite the square roots in the number density and energy density expressions by expanding out the completion of the square

\begin{dmath}\label{eqn:huang93:800}
(1 + \theta x) \sqrt{ (1 + \theta x)^2 - 1}
=
(1 + \theta x) 
\sqrt{ 1 + \theta x + 1 }
\sqrt{ 1 + \theta x - 1 }
= \sqrt{2 \theta} x^{1/2} 
(1 + \theta x) \sqrt{ 1 + \frac{\theta x}{2}}
,
\end{dmath}

Expanding the distribution about $\bar{z} e^{-x} = 0$, we have

\begin{equation}\label{eqn:huang93:820}
\frac{1}
{
   \barz^{-1} e^{x}
}
=
\frac{z e^{-x}}
{
   1 + \barz e^{-x}
}
=
z e^{-x} \sum_{s = 0}^\infty (-1)^s \lr{ \barz e^{-x} }^k,
\end{equation}

allowing us to write, in the low density limit with respect to $\barz$

\begin{subequations}
\begin{equation}\label{eqn:huang93:740}
\frac{n}{n_0}
= 
3 
\sqrt{2}
\theta^{3/2} 
\sum_{s=0}^\infty
(-1)^s
\barz^{s + 1}
\int_{0}^\infty dx x^{1/2}
(1 + \theta x) \sqrt{ 1 + \frac{\theta x}{2}} 
e^{-x(1 + s)} 
\end{equation}
\begin{equation}\label{eqn:huang93:760}
\frac{P}{n_0 \epsilon_0}
= 
   \theta
\sum_{s=0}^\infty
(-1)^s
\barz^{s + 1}
   \int_{0}^\infty
   dx
   \lr{ (\theta x + 1)^2 - 1}^{3/2} 
e^{-x(1 + s)} 
\end{equation}
\begin{equation}\label{eqn:huang93:780}
\frac{\expectation{\epsilon - \epsilon_0}}{V \epsilon_0 n_0} 
=
3 \sqrt{2} \theta^{5/2} 
\sum_{s=0}^\infty
(-1)^s
\barz^{s + 1}
   \int_{0}^\infty dx
      x^{3/2} 
(1 + \theta x) \sqrt{ 1 + \frac{\theta x}{2}} 
e^{-x(1 + s)} 
.
\end{equation}
\end{subequations}

Of these the pressure integral is yields directly to Mathematica

\begin{equation}\label{eqn:huang93:840}
   \int_{0}^\infty
   dx
   \lr{ (\theta x + 1)^2 - 1}^{3/2} 
e^{-x(1 + s)} 
=
\frac{e^{\frac{s+1}{\theta }} K_1\left(\frac{s+1}{\theta }\right)}{s+1},
\end{equation}

where $K_1(z)$ is a modified Bessel function \citep{wolframBesselK} of the second kind as plotted in \cref{fig:huang93:huang93Fig2}.

\imageFigure{huang93Fig2sv}{Modified Bessel function of the second kind}{fig:huang93:huang93Fig2}{0.3}

Our pressure is 
\begin{equation}\label{eqn:huang93:860}
\frac{P}{n_0 \epsilon_0}
= 
\frac{\kB T}{\epsilon_0}
\sum_{s=0}^\infty
(-1)^s
\frac{
\lr{ \barz e^{\epsilon_0/\kB T} }^{s + 1}
}{s + 1}
K_1\left( (s+1) \epsilon_0/\kB T \right).
\end{equation}

We've got Bessel and linear and exponential functions all in the mix, so the plot of \cref{fig:huang93:huang93Fig3} is helpful to get a first glance at the asympotic behavior of this function.

\imageFigure{huang93Fig3sv}{Pressure summands $(-1)^s \frac{\theta}{s + 1} e^{ (s + 1)/\theta} K_1\left((s+1)/\theta\right)$ for $\barz = 1$}{fig:huang93:huang93Fig3}{0.3}

Plotting the sum in \cref{fig:huang93:huang93Fig4sv} numerically (up to 10 terms, with no visible change after that), we end up with something that looks fairly exponential

\imageFigure{huang93Fig4sv}{Ten terms of pressure sum with $\barz = 1$}{fig:huang93:huang93Fig4sv}{0.3}

Plotting this sum in \cref{fig:huang93:huang93Fig5sv} (also numerically to 10 terms) as a function of both $\barz$ and $\theta$ we have something that looks roughly exponential in 

\imageFigure{huang93Fig5sv}{Pressure to ten terms in $\barz$ and $\theta$}{fig:huang93:huang93Fig5sv}{0.3}

\makeSubAnswer{}{basicStatMech:problemSet6:5d}

\begin{dmath}\label{eqn:huang93:80}
\evalbar{\epsilon_{\mathrm{F}}}{n = 1/(0.01)^3}
%ef[3 10^8, 6.62606957 10^(-34), rho, 0]
%= 6.12402 \times 10^{-35} \text{J}
= 6.12402 \times 10^{-35} \text{J} \times 6.24150934 \times 10^{18} \frac{\text{eV}}{\text{J}}
= 3.82231 \times 10^{-16} \text{eV}
\end{dmath}

Wow.  That's pretty low!
}
