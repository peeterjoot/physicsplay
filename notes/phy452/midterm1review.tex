\chapter{Midterm 1 review}

\begin{enumerate}
\item Product of IID random vars.  What can we say about this with the CLT

\begin{equation}\label{eqn:basicStatMechLecture8:20}
X = x_1 x_2 \cdots x_N
\end{equation}

We can't apply CLT directly, but we can apply it to $\ln X$

\begin{equation}\label{eqn:basicStatMechLecture8:40}
Y = \ln X = \sum_i \ln x_i = \sum_i y_i
\end{equation}

So

\begin{equation}\label{eqn:basicStatMechLecture8:60}
\tilde{P}(Y) = \inv{\sqrt{ 2 \pi N \sigma_y^2}} \exp
\lr{
-\frac{(Y - N \mu_y)^2}{N \sigma_y^2}
}
\end{equation}

Of interest here is how probabilities change under change of variables.  If $Y = \ln X$, this works like

\begin{equation}\label{eqn:basicStatMechLecture8:80}
P(X) dX = \tilde{P}(Y) dY
\end{equation}
\begin{equation}\label{eqn:basicStatMechLecture8:100}
P(X) = \tilde{P}(Y = \ln X) \frac{dY}{dX}.
\end{equation}

This is illustrated roughly in \cref{fig:lecture8:lecture8Fig1}

\imageFigure{lecture8Fig1}{Change of variables for a probability distribution}{fig:lecture8:lecture8Fig1}{0.3}

\item Random walk

This was a unit stepping problem as illustrated in \cref{fig:lecture8:lecture8Fig2}.

\imageFigure{lecture8Fig2}{Unit one dimensional random walk}{fig:lecture8:lecture8Fig2}{0.15}

\begin{dmath}\label{eqn:basicStatMechLecture8:120}
\expectation{X} = \sum_i \expectation{x_i}
\expectation{X^2} = 
\expectation{
\lr{\sum_{i = 1}^N x_i}
\lr{\sum_{j = 1}^N x_i}
}
= 
\expectation{
\lr{\sum_{i = 1}^N x_i^2}
\lr{\sum_{i \ne j = 1}^N x_i x_j}
}
\end{dmath}

\item State and prove Liouville's theorem

\begin{dmath}\label{eqn:basicStatMechLecture8:140}
\frac{d\rho}{dt} = 
\PD{t}{\rho} + 
\sum_{i = 1}^{3N} 
\PD{x_i}{
\lr{\dot{x}_i \rho}
}
+
\PD{x_i}{
\lr{\dot{x}_i \rho}
}
\end{dmath}

\item Collisions in 1D

This was a one direction collision problem for $N$ particles, a portion of which is illustrated in \cref{fig:lecture8:lecture8Fig3}.

\imageFigure{lecture8Fig3}{One dimensional collision of particles}{fig:lecture8:lecture8Fig3}{0.15}

The statement here that the collision was in 1D was a hint that we can actually calculate the result.

Start with a pair of collisions and work out that the velocities are exchanged.

\begin{equation}\label{eqn:basicStatMechLecture8:160}
(v_i, v_{i+1}) \rightarrow (v_{i+1}, v_{i})
\end{equation}

We have only exchange velocities.  This won't result in all the phase space being explored, and was meant to show that things are extremely restrictive in 1D.

\item Harmonic oscillator in 1D.

This problem ends up essentially requiring the evaluation of the area in phase space of an ellipse in phase space as in \cref{fig:lecture8:lecture8Fig4}.

\imageFigure{lecture8Fig4}{1D classical SHO phase space}{fig:lecture8:lecture8Fig4}{0.2}

The result is $\pi a b \times \text{area}$.

\end{enumerate}
