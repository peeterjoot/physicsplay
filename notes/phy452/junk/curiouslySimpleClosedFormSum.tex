%
% Copyright � 2013 Peeter Joot.  All Rights Reserved.
% Licenced as described in the file LICENSE under the root directory of this GIT repository.
%
\newcommand{\authorname}{Peeter Joot}
\newcommand{\email}{peeterjoot@protonmail.com}
\newcommand{\basename}{FIXMEbasenameUndefined}
\newcommand{\dirname}{notes/FIXMEdirnameUndefined/}

\renewcommand{\basename}{curiouslySimpleClosedFormSum}
\renewcommand{\dirname}{notes/phy452/}
\newcommand{\keywords}{binomial coefficient, gamma function}

\newcommand{\authorname}{Peeter Joot}
\newcommand{\onlineurl}{http://sites.google.com/site/peeterjoot2/math2013/\basename.pdf}
\newcommand{\sourcepath}{\dirname\basename.tex}
\newcommand{\generatetitle}[1]{\chapter{#1}}

\newcommand{\vcsinfo}{%
\section*{}
\noindent{\color{DarkOliveGreen}{\rule{\linewidth}{0.1mm}}}
\paragraph{Document version}
%\paragraph{\color{Maroon}{Document version}}
{
\small
\begin{itemize}
\item Available online at:\\ 
\href{\onlineurl}{\onlineurl}
\item Git Repository: \input{./.revinfo/gitRepo.tex}
\item Source: \sourcepath
\item last commit: \input{./.revinfo/gitCommitString.tex}
\item commit date: \input{./.revinfo/gitCommitDate.tex}
\end{itemize}
}
}

%\PassOptionsToPackage{dvipsnames,svgnames}{xcolor}
\PassOptionsToPackage{square,numbers}{natbib}
\documentclass{scrreprt}

\usepackage[left=2cm,right=2cm]{geometry}
\usepackage[svgnames]{xcolor}
\usepackage{peeters_layout}

\usepackage{natbib}

\usepackage[
colorlinks=true,
bookmarks=false,
pdfauthor={\authorname, \email},
backref 
]{hyperref}

% http://tex.stackexchange.com/questions/75773/how-to-reference-problems-by-the-text-label-in-an-exercise-envioronment
\usepackage[english]{cleveref}
\crefname{Exercise}{exercise}{exercises}
\Crefname{Exercise}{Exercise}{Exercises}

\RequirePackage{titlesec}
\RequirePackage{ifthen}

% http://stackoverflow.com/questions/4932910/date-in-the-tabular-environment
\makeatletter
\let\insertdate\@date
\makeatother

\titleformat{\chapter}[display]
{\bfseries\Large}
{\color{DarkSlateGrey}\filleft \authorname
\ifthenelse{\isundefined{\studentnumber}}{}{\\ \studentnumber}
\ifthenelse{\isundefined{\email}}{}{\\ \email}
\ifthenelse{\isundefined{\dateintitle}}{}{\\ \insertdate}
%\ifthenelse{\isundefined{\coursename}}{}{\\ \coursename} % put in title instead.
}
{4ex}
{\color{DarkOliveGreen}{\titlerule}\color{Maroon}
\vspace{2ex}%
\filright}
[\vspace{2ex}%
\color{DarkOliveGreen}\titlerule
]

\newcommand{\beginArtWithToc}[0]{\begin{document}\tableofcontents}
\newcommand{\beginArtNoToc}[0]{\begin{document}}
\newcommand{\EndNoBibArticle}[0]{\end{document}}
\newcommand{\EndArticle}[0]{\bibliography{Bibliography}\bibliographystyle{plainnat}\end{document}}

% 
%\newcommand{\citep}[1]{\cite{#1}}

\colorSectionsForArticle



\beginArtNoToc

\generatetitle{A curious Taylor series}
%\chapter{A curious Taylor series}
\label{chap:curiouslySimpleClosedFormSum}
%\section{Motivation}
%\section{Guts}

As part of a problem, I thought I needed the Taylor expansion of

\begin{dmath}\label{eqn:curiouslySimpleClosedFormSum:20}
(1 + \theta x)\lr{1 + \theta x/2}^{1/2}.
\end{dmath}

Take a look at the suprising simple final form form of the sum once evaluated.

\begin{equation}\label{eqn:curiouslySimpleClosedFormSum:40}
\begin{aligned}
(1 + \theta x)\lr{1 + \theta x/2}^{1/2}
&=
(1 + \theta x)
\sum_{s = 0}^\infty \binom{1/2}{s} \lr{\theta x/2}^s \\
&=
\sum_{s = 0}^\infty \binom{1/2}{s} \lr{\theta x/2}^s
+ 2 \frac{\theta x}{2}
\sum_{s = 0}^\infty \binom{1/2}{s} \lr{\theta x/2}^{s} \\
&=
1 + 
\sum_{s = 1}^\infty \binom{1/2}{s} \lr{\theta x/2}^s
+ 2
\sum_{s = 0}^\infty \binom{1/2}{s} \lr{\theta x/2}^{s + 1} \\
&= 
1
+
% t = s - 1
% t + 1 = s
\sum_{t = 0}^\infty \binom{1/2}{t + 1} \lr{\theta x/2}^{t + 1}
+
2
\sum_{s = 0}^\infty \binom{1/2}{s} \lr{\theta x/2}^{s + 1} \\
&=
1 +
\sum_{s = 0}^\infty 
\lr{ 
\binom{1/2}{s + 1} 
+
2
\binom{1/2}{s}
} \lr{\theta x/2}^{s + 1}.
\end{aligned}
\end{equation}

Our binomial coefficient has the usual definition

\begin{dmath}\label{eqn:curiouslySimpleClosedFormSum:60}
\binom{a}{b} 
=
\frac{a!}{b!(a-b)!},
\end{dmath}

however, because of the fractional powers here we require the gamma function generalization of the factorials

\begin{dmath}\label{eqn:curiouslySimpleClosedFormSum:80}
r! \equiv \Gamma(r+1),
\end{dmath}

for any \(r\) that is non-integral.

Now lets sum the binomial coefficients

\begin{equation}\label{eqn:curiouslySimpleClosedFormSum:100}
\begin{aligned}
\binom{1/2}{s + 1} 
+
2
\binom{1/2}{s}
&=
\frac{(1/2)!}{(s+1)!(1/2 -(s+1))!}
+ 2
\frac{(1/2)!}{s!(1/2 -s)!} \\
&=
\frac{(1/2)!}{s!(1/2 -s -1)!}
\lr{
\inv{s +1} + \frac{2}{1/2 - s}
} \\
&=
\cdots
\end{aligned}
\end{equation}

Never mind.  Doesn't look like it works after all?

%\EndArticle
\EndNoBibArticle
