\chapter{Cronological Index}
\label{chap:Cronology}

\begin{itemize}

\item January 17, 2000 \ref{chap:starDistance} Triangulating distance to a star from orbital angle measurements.

Derive the star distance calculation in the Feynman lectures.  One of my earliest latex attempts ever, adding details from Feynman's star distance discussion in the lectures\item March 25, 2000 \ref{chap:maxwell} Various formulations of Maxwell's equations

\item May 1, 2000 \ref{chap:crossOld} Early cross product generalization and motivation attempt

\item October 12, 2007 \ref{chap:cross} The cross product in three and more dimensions

\item February 2, 2008 \ref{chap:taylors} Taylor's theorem deviation

\item March 17, 2008 \ref{chap:pythagoras} Pythagoras law

2D diagramatic proof, and justification of the 3D/ND coordinate length rule/definition.\item May 7, 2008. \ref{chap:lorentzRotation} Lorentz Force Trajectory.

\item May 15, 2008 \ref{chap:mpInverseSvdRoughNotes} Singular Value Decomposition

Rough notes on Moore Penrose Inverse and SVD\item June 10, 2008 \ref{chap:fvec} Relativistic dynamics from Lagrangian

Summarizing for myself the various four-vectors of mechanics.\item July 28, 2008 \ref{chap:pe} Potential and Kinetic Energy

\item August 30, 2008 \ref{chap:lorentzMetricTensor} Short metric tensor explanation

Metric tensor and Lorentz diagonality.\item Sept 2, 2008 \ref{chap:goldsteinCh12} Attempts at solutions for some Goldstein Mechanics problems

Solutions to selected Goldstein Mechanics problems from chapter I and II.

Some of the Goldstein problems in chapter I were also in the Tong problem set. This is some remaining ones and a start at chapter II problems.

Problem 8 from Chapter I was never really completed in my first pass.  It looks like I missed the Kinetic term in the Lagrangian too.  The question of if angular momentum is conserved in that problem is considered in more detail, and a Noether's derivation that is specific to the calculation of the conserved ``current'' for a rotational symmetry is performed.  I'd be curious what attack on that question Goldstein was originally thinking of.  Although I believe this Noether's current treatment answers the question in full detail, since it wasn't covered yet in the text, is there an easier way to get at the result?
\item October 25, 2008 \ref{chap:debroglie} Some notes on DeBroglie paper

Some very rough notes on a reading of a translation of the DeBroglie thesis. 

I'd like to revisit this paper, but have to revisit wave basics better first (ie: phase and group velocity, doppler shift, ...).   These notes are strictly SR clarifications for myself at this point.\item December 02, 2008 \ref{chap:waveLagrangian} Compare some wave equation's and their Lagrangians

A summary of some wave equation Lagrangians, including wave equations of quantum mechanics.\item December 13, 2008 \ref{chap:sch} Ad-hoc motivation of some QM wave equations

Ad hoc motivation of the Schrodinger, Klein-Gordan, and Dirac equations in the usual introductory energy conservation forms.\item December 18, 2008 \ref{chap:rapidity} Some rapidity angle notes

\item December 23, 2008 \ref{chap:qmSusskind} Notes on Susskind's QM Lecture

Some very rough notes on Susskind's photon polarization lectures from itunes U.  Bra and Ket notations, projections, dual space, Hermitian operators, position and momentum eigenfunctions and states, QM postulates.\item December 25, 2008 \ref{chap:velocityAddition} Some notes on Pauli Relativity Velocity addition

Walk through the details of the relativistic velocity addition from Pauli's book.\item January 05, 2009 \ref{chap:gaussian} Evaluating the Gaussian integral

Nothing fancy, but I couldn't remember how to do it at first.  Also added the next two degree Gaussian integrals, and a derivation of the recurrance relations for the higher degree variations. \item January 12, 2009 \ref{chap:qmdCh2quiz} Chapter II quiz problems from Quantum Mechanics Demystified.

\item January 22, 2009 \ref{chap:ehrenfest} Ehrenfest's theorem

A dumbed down derivation of the one dimensional Ehrenfest theorem.  This theorem expresses the classical limit of QM in terms of expectation values of the position and momentum operators.  Rather than use the slick and fancy operator commutator formalism, which I haven't taken the time to learn yet, do this with just integration by parts. \item January 24, 2009 \ref{chap:pauliQmRelativityIntro} Pauli's relativity background in QM intro from "Wave Mechanics"

Relativity equations of motion from a Hamiltonian treatment.  A walk through of Pauli's relativity intro from his "Wave Mechanics" book at my own pace.\item February 16, 2009 \ref{chap:wavepacket} Simple Wave Packet Examples

Perform some of the wave packet integrals from Bohm's book.   Work the integrals for the three wave packet examples in the text in full detail (unweighted plane wave summed over a small continuous range of frequencies, gaussian wave packet, and gaussian wave packet with frequency expanded up to second order frequency taylor series terms).\item March 1, 2009 \ref{chap:fletcher} fletcher64

\item March 4, 2009 \ref{chap:distributions} Applications of Fourier distribution theory to some PDEs

Solve the homogeneous first order advanced wave equation in one variable using classical Fourier techniques and also the distribution formalism to compare the two, and get a feel for the latter.  This applies the distribution techniques from Prof Brad Osgood's Fourier lectures, as heard on Stanford on iTunesU.  After listening to the distribution lectures, I wasn't convinced that this method would be practical, but the proof is in the application.  I was surprised that it is actually simpler, with no "so many words" requirements to pull delta functions out of magic hats from PV sinc evaluations of the exponential integral. \item March 6, 2009 \ref{chap:bohmCh9} Bohm Chapter 9 problems

Have done all but one of these problems, and now written up solutions to a subset of these. \item March 8, 2009 \ref{chap:deltaOrthoSeries} Dirac delta function in terms of orthogonal functions

Explore the delta function representation given in Pauli's book, and express some of Pauli's content in my own words to build understanding, and relate fourier series and transforms under the general umbrella of orthonormal inner product spaces. \item March 13, 2009 \ref{chap:dotLinearity} Dot product linearity by construction

Dot product linearity using cosine form.  Also demonstrate the law of cosines.\item March 21, 2009 \ref{chap:byronFullerCalcVar} Worked calculus of variations problems from Byron and Fuller

Pi-Meson problem from the calculus of variations problems of Byron and Fuller.  Also compute the Noether current for phase invariance, and the attempt to show how a local gauge phase transformation leads to the interaction Lagrangian.  I end up with one term having a different sign than in the original, and can't spot an error.  Is there a sign error in the problem of the text?   Do the snell's law problem, at least the first part, showing the angle relation by minimizing the path over time.\item March 26, 2009 \ref{chap:binomial} Integer binomial theorem induction, the easy dumb way

\item March 27, 2009 \ref{chap:kleinGordon} Some Klein-Gordon equation notes

To get the feel for global and local guage transformations, work through this for the KG equation, as Susskind did in his relativity lecture.  As context, examined the equation and its solution a bit as well. \item April 8, 2009 \ref{chap:pauliFourVectorV} Four vector velocity addition notes

Reconcile the covariant velocity addition formula, via Lorentz boost, with the traditional introductory differential division presentation.  Further reading of Pauli's Theory of relativity' text prompted this bit of self clarification.\item April 10, 2009 \ref{chap:accFourVector} Relativistic acceleration

Acceleration four vector notes.  Also from reading Pauli.\item April 13, 2009 \ref{chap:commutatorHerm} Commutator and Anti-Commutator Hermitian-ness

Proving that the i-commutator (imaginary scaled) and anticommutator operators are Hermitian (given Hermitian operators).  Notes from reading of Pauli's wave mechanics.\item April 19, 2009 \ref{chap:harmonicOsc} Quantum Harmonic Oscillator

First two energy level solutions without using the fancier operator raising or lowering methods. \item April 23, 2009 \ref{chap:bohmCh10} Select problems from Chapter 10 of Bohm's Quantum Theory.

\item April 27, 2009 \ref{chap:wavevariation} Simple minded variation of one dimensional wave equation Lagrangian

Getting from the Lagrangian to the wave equation, without using the field form of the Euler-Lagrange equations. \item May 8, 2009 \ref{chap:bohm11} QM notes and problems for Bohm, chapter 11

\item May 11, 2009 \ref{chap:qmBarrier} One dimensional rectangular Quantum barrier penetration problem

Work through all the messy algebra, for the wave function calculation, and determine the coefficients, the probability densities, and the probability currents, in full gory detail.  This was in response to problem 11.4 of Bohm, and an attempt to reconcile a first busted attempt at that problem with other sources such as wikipedia and QMD. \item June 21, 2009 \ref{chap:invarianceEnMom} Lorentz invariance of energy momentum four vector

\item Aug 18, 2009 \ref{chap:sphericalHarmonicRaising} Spherical harmonic Eigenfunctions by application of the raising operator

\item Sept 19, 2009 \ref{chap:jacksonRetarded} Reader notes for Jackson 12.11, Retarded time solution to the wave equation.

\item Sept 26, 2009 \ref{chap:hamiltonian} Hamiltonian notes.

\item Oct 4, 2009 \ref{chap:quadraticForm} Linear transformations that retain two by two positive definiteness.

\item Nov 13, 2009 \ref{chap:linearizeDE} Linearizing a set of regular differential equations.

\item Nov 15, 2009 \ref{chap:constFourMomentum} Force free relativistic motion.

\item Nov 26, 2009 \ref{chap:multiPendulumSphericalMatrix} Lagrangian and Euler-Lagrange equation evaluation for the spherical N-pendulum problem

\item Nov 30, 2009 \ref{chap:twoParticleCMLaplacian} Two particle center of mass Laplacian change of variables.

\item Dec 6, 2009 \ref{chap:jacobianSphericalPolar} Jacobians and spherical polar gradient.

\item Jan 1, 2010 \ref{chap:1dpotentialIntegral} Integrating the equation of motion for a one dimensional problem.

Solve for time for an arbitary one dimensional potential.\item Feb 19, 2010 \ref{chap:1dharmonicOsc} 1D forced harmonic oscillator.  Quick solution of non-homogeneous problem.

Solve the one dimensional harmonic oscillator problem using matrix methods.\item Mar 3, 2010 \ref{chap:goldsteinRouth} Notes on Goldstein's Routh's procedure.

Puzzle through Routh's procedure as outlined in Goldstein.\item May 23, 2010 \ref{chap:liboff314} Time evolution of some wave functions

Liboff, problem 3.14.\item May 23, 2010 \ref{chap:liboff319} Effect of sinusoid operators

Liboff, problem 3.19.\item May 28, 2010 \ref{chap:feynmanQEDerrata} Errata for Feynman's Quantum Electrodynamics (Addison-Wesley)?

My collection of errata notes for some Feynman lecture notes on QED compiled by a student.\item May 29, 2010 \ref{chap:pauliFourier} Fourier transformation of the Pauli QED wave equation (Take I).

Unsuccessful attempt to find a solution to the Pauli QM Hamiltonian using Fourier transforms.  Also try to figure out the notation from the Feynman book where I saw this.\item May 30, 2010 \ref{chap:exponentialCommutation} On commutation of exponentials

Show that commutation of exponentials occurs if exponentiated terms also commute.\item May 31, 2010 \ref{chap:liboff41} Infinite square well wavefunction.

A QM problem from Liboff chapter 4.\item June 19, 2010 \ref{chap:hoopSpring} Hoop and spring oscillator problem.

A linear appromation to a hoop and spring problem.\item June 25, 2010 \ref{chap:liboff43} More problems from Liboff chapter 4

Liboff problems 4.11, 4.12, 4.14\item July 27, 2010 \ref{chap:rotationUnitary} Rotations using matrix exponentials

Calculating the exponential form for a unitary operator.  A unitary operator can be expressed as the exponential of a Hermitian operator.  Show how this can be calculated for the matrix representation of an operator.  Explicitly calculate this matrix for a plane rotionation yields one of the Pauli spin matrices.  While not unitary, the same procedure can be used to calculate such a rotation like angle for a Lorentz boost, and we also find that the result can be expressed in terms of one of the Pauli spin matrices.\item Feb 20, 2011 \ref{chap:stokesTensor} Exploring Stokes Theorem in tensor form.

Exploring Stokes Theorem in tensor form.\item May 25, 2011 \ref{chap:desaiDiracAdjoint} Dirac spinor notes.

Notes on Dirac spinor solutions.\item June 20, 2011 \ref{chap:zeeTauMatrix} On tensor product generators of the gamma matrices.

\item Aug 4, 2011 \ref{chap:diracGauge} Gauge transformation of the Dirac equation.

non-Covariant gauge transformation notes for the Dirac equation, and the corresponding non-relativistic approximation.\item Sept 1, 2011 \ref{chap:diracCovariant} Covariant gauge transformation of Dirac equation to get the spin adjusted Klein-Gordon equation.

GA use of dot and wedge products to arrive at some of the Dirac equation results in Desai.  In particular multiplication of the (gauge transformed) Dirac equation by a conjugate of sorts results in the Klein-Gordon equation, but with the spin terms required for electrons. \end{itemize}
