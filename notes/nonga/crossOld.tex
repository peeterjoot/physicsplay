\documentclass{article}      % Specifies the document class

\usepackage{amsmath}
\newcommand{\norm}[1]{\lVert#1\rVert}

% partial derivative of #1 wrt. #2:
\newcommand{\D}[2] {\frac {\partial #2} {\partial #1}}
% second partial derivative of #1 wrt. #2:
\newcommand{\Dsq}[2] {\frac {\partial^2 #2} {\partial {#1}^2}}

%
% shorthand for bold symbols:
%
\newcommand{\Bx}[0]{\mathbf{x}}
\newcommand{\By}[0]{\mathbf{y}}
\newcommand{\Bz}[0]{\mathbf{z}}
\newcommand{\Bu}[0]{\mathbf{u}}
\newcommand{\Br}[0]{\mathbf{r}}
\newcommand{\BR}[0]{\mathbf{R}}
\newcommand{\BF}[0]{\mathbf{F}}
\newcommand{\BM}[0]{\mathbf{M}}
\newcommand{\Bl}[0]{\mathbf{l}}
\newcommand{\Cn}[0]{\mathbf{C}^n}
\newcommand{\Rn}[0]{\mathbf{R}^n}
\newcommand{\Btheta}[0]{\boldsymbol{\theta}}

\newcommand{\rcap}[0]{\hat{\Br}}
\newcommand{\ucap}[1]{\hat{\Bu}_{#1}}
\newcommand{\xcap}[0]{\hat{\Bx}}
\newcommand{\ycap}[0]{\hat{\By}}
\newcommand{\zcap}[0]{\hat{\Bz}}
\newcommand{\thetacap}[0]{\hat{\Btheta}}

\newcommand{\innerprod}[2]{\langle{#1}, {#2}\rangle}

                             % The preamble begins here.
\title{The cross product in three and more dimensions} % Declares the document's title.
\author{Peeter Joot}         % Declares the author's name.
%\date{}        % Deleting this command produces today's date.

\begin{document}             % End of preamble and beginning of text.

\maketitle{}

\section{Introduction}

The cross product is an ugly arbitrary seeming sort of beast, but it is a beast that 
describes many sorts of physical and mathematical situations.  In vector calculus 
cross product terms and it relative the determinant end up occuring all over the place, 
and in physics the cross product also occurs in many contexts.
Examples are Stokes theorem, Jacobian transformations, normal equations, the 
curl operator, Maxwell's equations, torque, and the list goes on.  In many of 
these cases we have mathematics that has no logical tie to three dimensions, but
the cross product is an explicitly three dimensional sort of beast and one is 
left quickly with the open question of how to generalize it and the math that
is related to it to higher dimensions and other mathematical fields that that 
of real numbers.  One can even wonder how to generalize the cross product to
lower dimensions than three, since the cross product isn't defined in two or
one dimension as is the inner product in $\Cn$.

\section{The cross product and torque}

The basic definition of torque as a scalar quantity is the product of the radial distance times 
the perpendicular force.  The formula in terms of components in three dimensions given a force vector 
$\BF = (F_x, F_y, F_z)$ and the
radial distance $\Br = (x, y, z)$ is pretty messy, which is the reason it 
is typically described by means of a cross product, and
a generalized torque ``vector'' with a magnitude and direction.

My Feynman book gives a derivation of for the formula for torque in one dimension as 
the differential work per unit rotation.  This derivation is interesting 
because it yields in a simple fashion a torque formula without having to 
introduce the complexities of the cross product or the torque pseudo-vector.  I will 
not reproduce it here, but will go through a generalized derivation for the torque 
equation when the plane of rotation has an arbitrary orientation in space, rather 
than being restricted to the x,y plane (or y,z or z,x).

To start things off, some basic vector algebra results will be presented.

\subsection{change of basis, transformations, and rotations}

Given an orthagonal basis $(\ucap{i})_i$ in one coordinate system and an
orthagonal basis $(\ucap{i}')_i$ for the same coordinate system, how are 
the two related?

We can relate the two sets of unit vectors by a set of linear equations

\begin{equation}
\ucap{i}' = \sum_{j=1}^n{a_{ij}\ucap{j}}
\end{equation}

What the values of $a_{ij}$ are can be determined by taking inner products and by using the 
orthagonality constraints.

\begin{align}
\innerprod{\ucap{i}'}{\ucap{k}} &= \sum_{j=1}^n{a_{ij}\innerprod{\ucap{j}}{\ucap{k}}} \\
                                &= \sum_{j=1}^n{a_{ij}\delta_{jk}} \\
                                &= a_{ik}
\end{align}

So we have the relationship between $\ucap{i}'$ and the unprimed coordinate system basis, and by 
symmetry, the equivalent relationship for $\ucap{i}$

\begin{align}
\ucap{i}' &= \sum_{j=1}^n{
\innerprod{\ucap{i}'}{\ucap{j}}
\ucap{j}
} 
&= \sum_{j=1}^n{
a_{ij}
\ucap{j}
} 
\\
\ucap{i} &= \sum_{j=1}^n{
\innerprod{\ucap{i}}{\ucap{j}'}
\ucap{j}'
}
&= \sum_{j=1}^n{
\overline{a_{ji}}
\ucap{j}'
} 
\end{align}

Note that we can express these two relationships with a transformation 
matrix $\BM$ and it's hermition transpose $\BM^*$

\begin{equation}
\begin{bmatrix}
\ucap{1}' \\
\ucap{2}' \\
\vdots	  \\
\ucap{n}'
\end{bmatrix} 
= 
\begin{bmatrix}
	a_{11} & a_{12} & \dots  & a_{1n} \\
        a_{21} & a_{22} & 	  &        \\
	\vdots &        & \ddots &        \\
	a_{n1} & \dots  &        & a_{nn}
\end{bmatrix}
\begin{bmatrix}
\ucap{1}  \\
\ucap{2}  \\
\vdots	  \\
\ucap{n} 
\end{bmatrix}
= \BM 
\begin{bmatrix}
\ucap{1}  \\
\ucap{2}  \\
\vdots	  \\
\ucap{n} 
\end{bmatrix}
\end{equation}
\begin{equation}
\begin{bmatrix}
\ucap{1} \\
\ucap{2} \\
\vdots	  \\
\ucap{n}
\end{bmatrix} 
= 
\begin{bmatrix}
	\overline{a_{11}} & \overline{a_{21}} & \dots  & \overline{a_{n1}} \\
        \overline{a_{12}} & \overline{a_{22}} & 	  &        \\
	\vdots &        & \ddots &        \\
	\overline{a_{1n}} & \dots  &        & \overline{a_{nn}}
\end{bmatrix}
\begin{bmatrix}
\ucap{1}' \\
\ucap{2}' \\
\vdots	  \\
\ucap{n}'
\end{bmatrix}
= \BM^*
\begin{bmatrix}
\ucap{1}' \\
\ucap{2}' \\
\vdots	  \\
\ucap{n}'
\end{bmatrix}
\end{equation}

Given an arbitrary vector $\Br = [r_j]_j$ in the primary coordinate system, one 
can express this vector $\Br' = [r_j']_j$ in the secondary coordinate system using 
the same sort procedure used to derive the transformation matrix $\BM$.

\begin{align}
\Br' &= 
      \sum_{k=1}^n
      {
       r_k 
       \ucap{k} 
      } \\
      &= 
      \sum_{k=1}^n
      {
       r_k
\sum_{j=1}^n
{
\overline{a_{jk}}
\ucap{j}'
}
      } \\
      &= 
\sum_{j=1}^n
      {
\ucap{j}'
      \sum_{k=1}^n
{
\overline{a_{jk}}
       r_k
}
      } \\
      &=
\sum_{j=1}^n
      {
\ucap{j}'
r_j'
      }
\end{align}

Since $r_j' = 
      \sum_{k=1}^n
{
\overline{a_{jk}}
       r_k
}
$
one can see that the components of the vectors transform 
in a similar fashion the 
basis vectors, and we can write $\Br = \BM^T \Br'$ and $\Br' = \overline{\BM} \Br$.
\footnote{this doesn't seem right.  I find myself wondering if have I messed up, despite the fact that everything looks okay?  Note that on paper I only derived this case for $\Rn$ and not $\Cn$}

%If the work done is $dW$, the object moves through an angle $d\theta$, and the distance from the point of 
%force application to the point of rotation is $ r = \norm{\Br}$ where $\Br = (x, y) = r(\cos\theta,\sin\theta)$, then 
%
%\begin{align*}
%dW &= \BF \cdot d\Bl \\
%   &= \BF \cdot r d\thetacap
%\end{align*}
%
%If we transform to a coordinate system with unit vectors 
%$\ucap{1} = \rcap, \ucap{2} = \thetacap$ 
%then 
%$d\thetacap = d\theta \ucap{2}$ 
%and 
%$dW = F_\theta r d\theta$ 
%where 
%$\BF' = (F_r, F_\theta)$ 
%is the force as measured in this 
%coordinate system.

\end{document}               % End of document.
