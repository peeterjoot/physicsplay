%\documentclass[]{eliblog}
%
%\usepackage{color}
%
%\usepackage{amsmath}
\usepackage{mathpazo}

%
% shorthand for bold symbols, convenient for vectors and matrices
%
\newcommand{\Ba}[0]{\mathbf{a}}
\newcommand{\Bb}[0]{\mathbf{b}}
\newcommand{\Bc}[0]{\mathbf{c}}
\newcommand{\Bd}[0]{\mathbf{d}}
\newcommand{\Be}[0]{\mathbf{e}}
\newcommand{\Bf}[0]{\mathbf{f}}
\newcommand{\Bg}[0]{\mathbf{g}}
\newcommand{\Bh}[0]{\mathbf{h}}
\newcommand{\Bi}[0]{\mathbf{i}}
\newcommand{\Bj}[0]{\mathbf{j}}
\newcommand{\Bk}[0]{\mathbf{k}}
\newcommand{\Bl}[0]{\mathbf{l}}
\newcommand{\Bm}[0]{\mathbf{m}}
\newcommand{\Bn}[0]{\mathbf{n}}
\newcommand{\Bo}[0]{\mathbf{o}}
\newcommand{\Bp}[0]{\mathbf{p}}
\newcommand{\Bq}[0]{\mathbf{q}}
\newcommand{\Br}[0]{\mathbf{r}}
\newcommand{\Bs}[0]{\mathbf{s}}
\newcommand{\Bt}[0]{\mathbf{t}}
\newcommand{\Bu}[0]{\mathbf{u}}
\newcommand{\Bv}[0]{\mathbf{v}}
\newcommand{\Bw}[0]{\mathbf{w}}
\newcommand{\Bx}[0]{\mathbf{x}}
\newcommand{\By}[0]{\mathbf{y}}
\newcommand{\Bz}[0]{\mathbf{z}}
\newcommand{\BA}[0]{\mathbf{A}}
\newcommand{\BB}[0]{\mathbf{B}}
\newcommand{\BC}[0]{\mathbf{C}}
\newcommand{\BD}[0]{\mathbf{D}}
\newcommand{\BE}[0]{\mathbf{E}}
\newcommand{\BF}[0]{\mathbf{F}}
\newcommand{\BG}[0]{\mathbf{G}}
\newcommand{\BH}[0]{\mathbf{H}}
\newcommand{\BI}[0]{\mathbf{I}}
\newcommand{\BJ}[0]{\mathbf{J}}
\newcommand{\BK}[0]{\mathbf{K}}
\newcommand{\BL}[0]{\mathbf{L}}
\newcommand{\BM}[0]{\mathbf{M}}
\newcommand{\BN}[0]{\mathbf{N}}
\newcommand{\BO}[0]{\mathbf{O}}
\newcommand{\BP}[0]{\mathbf{P}}
\newcommand{\BQ}[0]{\mathbf{Q}}
\newcommand{\BR}[0]{\mathbf{R}}
\newcommand{\BS}[0]{\mathbf{S}}
\newcommand{\BT}[0]{\mathbf{T}}
\newcommand{\BU}[0]{\mathbf{U}}
\newcommand{\BV}[0]{\mathbf{V}}
\newcommand{\BW}[0]{\mathbf{W}}
\newcommand{\BX}[0]{\mathbf{X}}
\newcommand{\BY}[0]{\mathbf{Y}}
\newcommand{\BZ}[0]{\mathbf{Z}}

\newcommand{\Bzero}[0]{\mathbf{0}}
\newcommand{\Btheta}[0]{\boldsymbol{\theta}}
\newcommand{\Btau}[0]{\boldsymbol{\tau}}
\newcommand{\Bomega}[0]{\boldsymbol{\omega}}

%
% shorthand for unit vectors
%
\newcommand{\acap}[0]{\hat{\Ba}}
\newcommand{\bcap}[0]{\hat{\Bb}}
\newcommand{\ccap}[0]{\hat{\Bc}}
\newcommand{\dcap}[0]{\hat{\Bd}}
\newcommand{\ecap}[0]{\hat{\Be}}
\newcommand{\fcap}[0]{\hat{\Bf}}
\newcommand{\gcap}[0]{\hat{\Bg}}
\newcommand{\hcap}[0]{\hat{\Bh}}
\newcommand{\icap}[0]{\hat{\Bi}}
\newcommand{\jcap}[0]{\hat{\Bj}}
\newcommand{\kcap}[0]{\hat{\Bk}}
\newcommand{\lcap}[0]{\hat{\Bl}}
\newcommand{\mcap}[0]{\hat{\Bm}}
\newcommand{\ncap}[0]{\hat{\Bn}}
\newcommand{\ocap}[0]{\hat{\Bo}}
\newcommand{\pcap}[0]{\hat{\Bp}}
\newcommand{\qcap}[0]{\hat{\Bq}}
\newcommand{\rcap}[0]{\hat{\Br}}
\newcommand{\scap}[0]{\hat{\Bs}}
\newcommand{\tcap}[0]{\hat{\Bt}}
\newcommand{\ucap}[0]{\hat{\Bu}}
\newcommand{\vcap}[0]{\hat{\Bv}}
\newcommand{\wcap}[0]{\hat{\Bw}}
\newcommand{\xcap}[0]{\hat{\Bx}}
\newcommand{\ycap}[0]{\hat{\By}}
\newcommand{\zcap}[0]{\hat{\Bz}}
\newcommand{\thetacap}[0]{\hat{\Btheta}}

%
% to write R^n and C^n in a distinguishable fashion.  Perhaps change this
% to the double lined characters upon figuring out how to do so.
%
\newcommand{\C}[1]{$\mathbb{C}^{#1}$}
\newcommand{\R}[1]{$\mathbb{R}^{#1}$}

%
% various generally useful helpers
%

% derivative of #1 wrt. #2:
\newcommand{\D}[2] {\frac {d#2} {d#1}}

\newcommand{\inv}[1]{\frac{1}{#1}}
\newcommand{\cross}[0]{\times}

\newcommand{\abs}[1]{\lvert{#1}\rvert}
\newcommand{\norm}[1]{\lVert{#1}\rVert}
\newcommand{\innerprod}[2]{\langle{#1}, {#2}\rangle}
\newcommand{\dotprod}[2]{{#1} \cdot {#2}}
\newcommand{\bdotprod}[2]{\left({#1} \cdot {#2}\right)}
\newcommand{\crossprod}[2]{{#1} \cross {#2}}
\newcommand{\tripleprod}[3]{\dotprod{\left(\crossprod{#1}{#2}\right)}{#3}}

\DeclareMathOperator{\Proj}{Proj}
\DeclareMathOperator{\Span}{span}
\DeclareMathOperator{\Sgn}{sgn}
\DeclareMathOperator{\Area}{Area}
\DeclareMathOperator{\Volume}{Volume}

%
% A few miscellaneous things specific to this document
%
\newcommand{\crossop}[1]{\crossprod{#1}{}}

% R2 vector.
\newcommand{\VectorTwo}[2]{
\begin{bmatrix}
 {#1} \\
 {#2}
\end{bmatrix}
}

\newcommand{\VectorN}[1]{
\begin{bmatrix}
{#1}_1 \\
{#1}_2 \\
\vdots \\
{#1}_N \\
\end{bmatrix}
}

\newcommand{\DETuvij}[4]{
\begin{vmatrix}
 {#1}_{#3} & {#1}_{#4} \\
 {#2}_{#3} & {#2}_{#4}
\end{vmatrix}
}

\newcommand{\DETuvwijk}[6]{
\begin{vmatrix}
 {#1}_{#4} & {#1}_{#5} & {#1}_{#6} \\
 {#2}_{#4} & {#2}_{#5} & {#2}_{#6} \\
 {#3}_{#4} & {#3}_{#5} & {#3}_{#6}
\end{vmatrix}
}

\newcommand{\DETuvwxijkl}[8]{
\begin{vmatrix}
 {#1}_{#5} & {#1}_{#6} & {#1}_{#7} & {#1}_{#8} \\
 {#2}_{#5} & {#2}_{#6} & {#2}_{#7} & {#2}_{#8} \\
 {#3}_{#5} & {#3}_{#6} & {#3}_{#7} & {#3}_{#8} \\
 {#4}_{#5} & {#4}_{#6} & {#4}_{#7} & {#4}_{#8} \\
\end{vmatrix}
}

%\newcommand{\DETuvwxyijklm}[10]{
%\begin{vmatrix}
% {#1}_{#6} & {#1}_{#7} & {#1}_{#8} & {#1}_{#9} & {#1}_{#10} \\
% {#2}_{#6} & {#2}_{#7} & {#2}_{#8} & {#2}_{#9} & {#2}_{#10} \\
% {#3}_{#6} & {#3}_{#7} & {#3}_{#8} & {#3}_{#9} & {#3}_{#10} \\
% {#4}_{#6} & {#4}_{#7} & {#4}_{#8} & {#4}_{#9} & {#4}_{#10} \\
% {#5}_{#6} & {#5}_{#7} & {#5}_{#8} & {#5}_{#9} & {#5}_{#10}
%\end{vmatrix}
%}

% R3 vector.
\newcommand{\VectorThree}[3]{
\begin{bmatrix}
 {#1} \\
 {#2} \\
 {#3}
\end{bmatrix}
}


%
%\author{Peeter Joot}
%\email{peeter.joot@utoronto.ca, 920798560}
%%
% Copyright � 2015 Peeter Joot.  All Rights Reserved.
% Licenced as described in the file LICENSE under the root directory of this GIT repository.
%
\documentclass[]{eliblog}

\usepackage{amsmath}
\usepackage{mathpazo}

%
% shorthand for bold symbols, convenient for vectors and matrices
%
\newcommand{\Ba}[0]{\mathbf{a}}
\newcommand{\Bb}[0]{\mathbf{b}}
\newcommand{\Bc}[0]{\mathbf{c}}
\newcommand{\Bd}[0]{\mathbf{d}}
\newcommand{\Be}[0]{\mathbf{e}}
\newcommand{\Bf}[0]{\mathbf{f}}
\newcommand{\Bg}[0]{\mathbf{g}}
\newcommand{\Bh}[0]{\mathbf{h}}
\newcommand{\Bi}[0]{\mathbf{i}}
\newcommand{\Bj}[0]{\mathbf{j}}
\newcommand{\Bk}[0]{\mathbf{k}}
\newcommand{\Bl}[0]{\mathbf{l}}
\newcommand{\Bm}[0]{\mathbf{m}}
\newcommand{\Bn}[0]{\mathbf{n}}
\newcommand{\Bo}[0]{\mathbf{o}}
\newcommand{\Bp}[0]{\mathbf{p}}
\newcommand{\Bq}[0]{\mathbf{q}}
\newcommand{\Br}[0]{\mathbf{r}}
\newcommand{\Bs}[0]{\mathbf{s}}
\newcommand{\Bt}[0]{\mathbf{t}}
\newcommand{\Bu}[0]{\mathbf{u}}
\newcommand{\Bv}[0]{\mathbf{v}}
\newcommand{\Bw}[0]{\mathbf{w}}
\newcommand{\Bx}[0]{\mathbf{x}}
\newcommand{\By}[0]{\mathbf{y}}
\newcommand{\Bz}[0]{\mathbf{z}}
\newcommand{\BA}[0]{\mathbf{A}}
\newcommand{\BB}[0]{\mathbf{B}}
\newcommand{\BC}[0]{\mathbf{C}}
\newcommand{\BD}[0]{\mathbf{D}}
\newcommand{\BE}[0]{\mathbf{E}}
\newcommand{\BF}[0]{\mathbf{F}}
\newcommand{\BG}[0]{\mathbf{G}}
\newcommand{\BH}[0]{\mathbf{H}}
\newcommand{\BI}[0]{\mathbf{I}}
\newcommand{\BJ}[0]{\mathbf{J}}
\newcommand{\BK}[0]{\mathbf{K}}
\newcommand{\BL}[0]{\mathbf{L}}
\newcommand{\BM}[0]{\mathbf{M}}
\newcommand{\BN}[0]{\mathbf{N}}
\newcommand{\BO}[0]{\mathbf{O}}
\newcommand{\BP}[0]{\mathbf{P}}
\newcommand{\BQ}[0]{\mathbf{Q}}
\newcommand{\BR}[0]{\mathbf{R}}
\newcommand{\BS}[0]{\mathbf{S}}
\newcommand{\BT}[0]{\mathbf{T}}
\newcommand{\BU}[0]{\mathbf{U}}
\newcommand{\BV}[0]{\mathbf{V}}
\newcommand{\BW}[0]{\mathbf{W}}
\newcommand{\BX}[0]{\mathbf{X}}
\newcommand{\BY}[0]{\mathbf{Y}}
\newcommand{\BZ}[0]{\mathbf{Z}}

\newcommand{\Bzero}[0]{\mathbf{0}}
\newcommand{\Btheta}[0]{\boldsymbol{\theta}}
\newcommand{\Btau}[0]{\boldsymbol{\tau}}
\newcommand{\Bomega}[0]{\boldsymbol{\omega}}

%
% shorthand for unit vectors
%
\newcommand{\acap}[0]{\hat{\Ba}}
\newcommand{\bcap}[0]{\hat{\Bb}}
\newcommand{\ccap}[0]{\hat{\Bc}}
\newcommand{\dcap}[0]{\hat{\Bd}}
\newcommand{\ecap}[0]{\hat{\Be}}
\newcommand{\fcap}[0]{\hat{\Bf}}
\newcommand{\gcap}[0]{\hat{\Bg}}
\newcommand{\hcap}[0]{\hat{\Bh}}
\newcommand{\icap}[0]{\hat{\Bi}}
\newcommand{\jcap}[0]{\hat{\Bj}}
\newcommand{\kcap}[0]{\hat{\Bk}}
\newcommand{\lcap}[0]{\hat{\Bl}}
\newcommand{\mcap}[0]{\hat{\Bm}}
\newcommand{\ncap}[0]{\hat{\Bn}}
\newcommand{\ocap}[0]{\hat{\Bo}}
\newcommand{\pcap}[0]{\hat{\Bp}}
\newcommand{\qcap}[0]{\hat{\Bq}}
\newcommand{\rcap}[0]{\hat{\Br}}
\newcommand{\scap}[0]{\hat{\Bs}}
\newcommand{\tcap}[0]{\hat{\Bt}}
\newcommand{\ucap}[0]{\hat{\Bu}}
\newcommand{\vcap}[0]{\hat{\Bv}}
\newcommand{\wcap}[0]{\hat{\Bw}}
\newcommand{\xcap}[0]{\hat{\Bx}}
\newcommand{\ycap}[0]{\hat{\By}}
\newcommand{\zcap}[0]{\hat{\Bz}}
\newcommand{\thetacap}[0]{\hat{\Btheta}}

%
% to write R^n and C^n in a distinguishable fashion.  Perhaps change this
% to the double lined characters upon figuring out how to do so.
%
\newcommand{\C}[1]{$\mathbb{C}^{#1}$}
\newcommand{\R}[1]{$\mathbb{R}^{#1}$}

%
% various generally useful helpers
%

% derivative of #1 wrt. #2:
\newcommand{\D}[2] {\frac {d#2} {d#1}}

\newcommand{\inv}[1]{\frac{1}{#1}}
\newcommand{\cross}[0]{\times}

\newcommand{\abs}[1]{\lvert{#1}\rvert}
\newcommand{\norm}[1]{\lVert{#1}\rVert}
\newcommand{\innerprod}[2]{\langle{#1}, {#2}\rangle}
\newcommand{\dotprod}[2]{{#1} \cdot {#2}}
\newcommand{\bdotprod}[2]{\left({#1} \cdot {#2}\right)}
\newcommand{\crossprod}[2]{{#1} \cross {#2}}
\newcommand{\tripleprod}[3]{\dotprod{\left(\crossprod{#1}{#2}\right)}{#3}}

\DeclareMathOperator{\Proj}{Proj}
\DeclareMathOperator{\Span}{span}
\DeclareMathOperator{\Sgn}{sgn}
\DeclareMathOperator{\Area}{Area}
\DeclareMathOperator{\Volume}{Volume}

%
% A few miscellaneous things specific to this document
%
\newcommand{\crossop}[1]{\crossprod{#1}{}}

% R2 vector.
\newcommand{\VectorTwo}[2]{
\begin{bmatrix}
 {#1} \\
 {#2}
\end{bmatrix}
}

\newcommand{\VectorN}[1]{
\begin{bmatrix}
{#1}_1 \\
{#1}_2 \\
\vdots \\
{#1}_N \\
\end{bmatrix}
}

\newcommand{\DETuvij}[4]{
\begin{vmatrix}
 {#1}_{#3} & {#1}_{#4} \\
 {#2}_{#3} & {#2}_{#4}
\end{vmatrix}
}

\newcommand{\DETuvwijk}[6]{
\begin{vmatrix}
 {#1}_{#4} & {#1}_{#5} & {#1}_{#6} \\
 {#2}_{#4} & {#2}_{#5} & {#2}_{#6} \\
 {#3}_{#4} & {#3}_{#5} & {#3}_{#6}
\end{vmatrix}
}

\newcommand{\DETuvwxijkl}[8]{
\begin{vmatrix}
 {#1}_{#5} & {#1}_{#6} & {#1}_{#7} & {#1}_{#8} \\
 {#2}_{#5} & {#2}_{#6} & {#2}_{#7} & {#2}_{#8} \\
 {#3}_{#5} & {#3}_{#6} & {#3}_{#7} & {#3}_{#8} \\
 {#4}_{#5} & {#4}_{#6} & {#4}_{#7} & {#4}_{#8} \\
\end{vmatrix}
}

%\newcommand{\DETuvwxyijklm}[10]{
%\begin{vmatrix}
% {#1}_{#6} & {#1}_{#7} & {#1}_{#8} & {#1}_{#9} & {#1}_{#10} \\
% {#2}_{#6} & {#2}_{#7} & {#2}_{#8} & {#2}_{#9} & {#2}_{#10} \\
% {#3}_{#6} & {#3}_{#7} & {#3}_{#8} & {#3}_{#9} & {#3}_{#10} \\
% {#4}_{#6} & {#4}_{#7} & {#4}_{#8} & {#4}_{#9} & {#4}_{#10} \\
% {#5}_{#6} & {#5}_{#7} & {#5}_{#8} & {#5}_{#9} & {#5}_{#10}
%\end{vmatrix}
%}

% R3 vector.
\newcommand{\VectorThree}[3]{
\begin{bmatrix}
 {#1} \\
 {#2} \\
 {#3}
\end{bmatrix}
}



\author{Peeter Joot}
\email{peeter.joot@gmail.com}

%\documentclass[]{eliblogwidescreen}

\usepackage{amsmath}
\usepackage{mathpazo}

%
% shorthand for bold symbols, convenient for vectors and matrices
%
\newcommand{\Ba}[0]{\mathbf{a}}
\newcommand{\Bb}[0]{\mathbf{b}}
\newcommand{\Bc}[0]{\mathbf{c}}
\newcommand{\Bd}[0]{\mathbf{d}}
\newcommand{\Be}[0]{\mathbf{e}}
\newcommand{\Bf}[0]{\mathbf{f}}
\newcommand{\Bg}[0]{\mathbf{g}}
\newcommand{\Bh}[0]{\mathbf{h}}
\newcommand{\Bi}[0]{\mathbf{i}}
\newcommand{\Bj}[0]{\mathbf{j}}
\newcommand{\Bk}[0]{\mathbf{k}}
\newcommand{\Bl}[0]{\mathbf{l}}
\newcommand{\Bm}[0]{\mathbf{m}}
\newcommand{\Bn}[0]{\mathbf{n}}
\newcommand{\Bo}[0]{\mathbf{o}}
\newcommand{\Bp}[0]{\mathbf{p}}
\newcommand{\Bq}[0]{\mathbf{q}}
\newcommand{\Br}[0]{\mathbf{r}}
\newcommand{\Bs}[0]{\mathbf{s}}
\newcommand{\Bt}[0]{\mathbf{t}}
\newcommand{\Bu}[0]{\mathbf{u}}
\newcommand{\Bv}[0]{\mathbf{v}}
\newcommand{\Bw}[0]{\mathbf{w}}
\newcommand{\Bx}[0]{\mathbf{x}}
\newcommand{\By}[0]{\mathbf{y}}
\newcommand{\Bz}[0]{\mathbf{z}}
\newcommand{\BA}[0]{\mathbf{A}}
\newcommand{\BB}[0]{\mathbf{B}}
\newcommand{\BC}[0]{\mathbf{C}}
\newcommand{\BD}[0]{\mathbf{D}}
\newcommand{\BE}[0]{\mathbf{E}}
\newcommand{\BF}[0]{\mathbf{F}}
\newcommand{\BG}[0]{\mathbf{G}}
\newcommand{\BH}[0]{\mathbf{H}}
\newcommand{\BI}[0]{\mathbf{I}}
\newcommand{\BJ}[0]{\mathbf{J}}
\newcommand{\BK}[0]{\mathbf{K}}
\newcommand{\BL}[0]{\mathbf{L}}
\newcommand{\BM}[0]{\mathbf{M}}
\newcommand{\BN}[0]{\mathbf{N}}
\newcommand{\BO}[0]{\mathbf{O}}
\newcommand{\BP}[0]{\mathbf{P}}
\newcommand{\BQ}[0]{\mathbf{Q}}
\newcommand{\BR}[0]{\mathbf{R}}
\newcommand{\BS}[0]{\mathbf{S}}
\newcommand{\BT}[0]{\mathbf{T}}
\newcommand{\BU}[0]{\mathbf{U}}
\newcommand{\BV}[0]{\mathbf{V}}
\newcommand{\BW}[0]{\mathbf{W}}
\newcommand{\BX}[0]{\mathbf{X}}
\newcommand{\BY}[0]{\mathbf{Y}}
\newcommand{\BZ}[0]{\mathbf{Z}}

\newcommand{\Bzero}[0]{\mathbf{0}}
\newcommand{\Btheta}[0]{\boldsymbol{\theta}}
\newcommand{\Btau}[0]{\boldsymbol{\tau}}
\newcommand{\Bomega}[0]{\boldsymbol{\omega}}

%
% shorthand for unit vectors
%
\newcommand{\acap}[0]{\hat{\Ba}}
\newcommand{\bcap}[0]{\hat{\Bb}}
\newcommand{\ccap}[0]{\hat{\Bc}}
\newcommand{\dcap}[0]{\hat{\Bd}}
\newcommand{\ecap}[0]{\hat{\Be}}
\newcommand{\fcap}[0]{\hat{\Bf}}
\newcommand{\gcap}[0]{\hat{\Bg}}
\newcommand{\hcap}[0]{\hat{\Bh}}
\newcommand{\icap}[0]{\hat{\Bi}}
\newcommand{\jcap}[0]{\hat{\Bj}}
\newcommand{\kcap}[0]{\hat{\Bk}}
\newcommand{\lcap}[0]{\hat{\Bl}}
\newcommand{\mcap}[0]{\hat{\Bm}}
\newcommand{\ncap}[0]{\hat{\Bn}}
\newcommand{\ocap}[0]{\hat{\Bo}}
\newcommand{\pcap}[0]{\hat{\Bp}}
\newcommand{\qcap}[0]{\hat{\Bq}}
\newcommand{\rcap}[0]{\hat{\Br}}
\newcommand{\scap}[0]{\hat{\Bs}}
\newcommand{\tcap}[0]{\hat{\Bt}}
\newcommand{\ucap}[0]{\hat{\Bu}}
\newcommand{\vcap}[0]{\hat{\Bv}}
\newcommand{\wcap}[0]{\hat{\Bw}}
\newcommand{\xcap}[0]{\hat{\Bx}}
\newcommand{\ycap}[0]{\hat{\By}}
\newcommand{\zcap}[0]{\hat{\Bz}}
\newcommand{\thetacap}[0]{\hat{\Btheta}}

%
% to write R^n and C^n in a distinguishable fashion.  Perhaps change this
% to the double lined characters upon figuring out how to do so.
%
\newcommand{\C}[1]{$\mathbb{C}^{#1}$}
\newcommand{\R}[1]{$\mathbb{R}^{#1}$}

%
% various generally useful helpers
%

% derivative of #1 wrt. #2:
\newcommand{\D}[2] {\frac {d#2} {d#1}}

\newcommand{\inv}[1]{\frac{1}{#1}}
\newcommand{\cross}[0]{\times}

\newcommand{\abs}[1]{\lvert{#1}\rvert}
\newcommand{\norm}[1]{\lVert{#1}\rVert}
\newcommand{\innerprod}[2]{\langle{#1}, {#2}\rangle}
\newcommand{\dotprod}[2]{{#1} \cdot {#2}}
\newcommand{\bdotprod}[2]{\left({#1} \cdot {#2}\right)}
\newcommand{\crossprod}[2]{{#1} \cross {#2}}
\newcommand{\tripleprod}[3]{\dotprod{\left(\crossprod{#1}{#2}\right)}{#3}}

\DeclareMathOperator{\Proj}{Proj}
\DeclareMathOperator{\Span}{span}
\DeclareMathOperator{\Sgn}{sgn}
\DeclareMathOperator{\Area}{Area}
\DeclareMathOperator{\Volume}{Volume}

%
% A few miscellaneous things specific to this document
%
\newcommand{\crossop}[1]{\crossprod{#1}{}}

% R2 vector.
\newcommand{\VectorTwo}[2]{
\begin{bmatrix}
 {#1} \\
 {#2}
\end{bmatrix}
}

\newcommand{\VectorN}[1]{
\begin{bmatrix}
{#1}_1 \\
{#1}_2 \\
\vdots \\
{#1}_N \\
\end{bmatrix}
}

\newcommand{\DETuvij}[4]{
\begin{vmatrix}
 {#1}_{#3} & {#1}_{#4} \\
 {#2}_{#3} & {#2}_{#4}
\end{vmatrix}
}

\newcommand{\DETuvwijk}[6]{
\begin{vmatrix}
 {#1}_{#4} & {#1}_{#5} & {#1}_{#6} \\
 {#2}_{#4} & {#2}_{#5} & {#2}_{#6} \\
 {#3}_{#4} & {#3}_{#5} & {#3}_{#6}
\end{vmatrix}
}

\newcommand{\DETuvwxijkl}[8]{
\begin{vmatrix}
 {#1}_{#5} & {#1}_{#6} & {#1}_{#7} & {#1}_{#8} \\
 {#2}_{#5} & {#2}_{#6} & {#2}_{#7} & {#2}_{#8} \\
 {#3}_{#5} & {#3}_{#6} & {#3}_{#7} & {#3}_{#8} \\
 {#4}_{#5} & {#4}_{#6} & {#4}_{#7} & {#4}_{#8} \\
\end{vmatrix}
}

%\newcommand{\DETuvwxyijklm}[10]{
%\begin{vmatrix}
% {#1}_{#6} & {#1}_{#7} & {#1}_{#8} & {#1}_{#9} & {#1}_{#10} \\
% {#2}_{#6} & {#2}_{#7} & {#2}_{#8} & {#2}_{#9} & {#2}_{#10} \\
% {#3}_{#6} & {#3}_{#7} & {#3}_{#8} & {#3}_{#9} & {#3}_{#10} \\
% {#4}_{#6} & {#4}_{#7} & {#4}_{#8} & {#4}_{#9} & {#4}_{#10} \\
% {#5}_{#6} & {#5}_{#7} & {#5}_{#8} & {#5}_{#9} & {#5}_{#10}
%\end{vmatrix}
%}

% R3 vector.
\newcommand{\VectorThree}[3]{
\begin{bmatrix}
 {#1} \\
 {#2} \\
 {#3}
\end{bmatrix}
}



\author{Peeter Joot}
\email{peeter.joot@gmail.com}


\chapter{PHY356 Problem Set 4.}
\label{chap:qmIproblemSet4}
%\useCCL
\blogpage{http://sites.google.com/site/peeterjoot/math2010/qmIproblemSet4.pdf}
\date{Nov 16, 2010}
\revisionInfo{qmIproblemSet4.tex}

\beginArtNoToc
\section{Disclaimer.}

This problem set is as yet ungraded.

\section{Problem 1.}
\subsection{Statement}

Is it possible to derive the eigenvalues and eigenvectors presented in Section 8.2 from those in Section 8.1.2?  What does this say about the potential energy operator in these two situations?

For reference 8.1.2 was a finite potential barrier, $V(x) = V_0, \Abs{x} > a$, and zero in the interior of the well.  This had trigonometric solutions in the interior, and died off exponentially past the boundary of the well.

On the other hand, 8.2 was a delta function potential $V(x) = -g \delta(x)$, which had the solution $u(x) = \sqrt{\beta} e^{-\beta \Abs{x}}$, where $\beta = m g/\hbar^2$.

%%Note: some references I found to help provide some of the context for WHY to consider the delta function potential in the first place are:
%%
%%\href{http://en.wikipedia.org/wiki/Delta_potential}{wikipedia Delta potential},
%%\href{http://quantummechanics.ucsd.edu/ph130a/130_notes/node156.html}{ucsd},
%%\href{http://www.physics.csbsju.edu/QM/delta.01.html}{csbsju},
%%\href{http://panda.unm.edu/Courses/Fields/Phys491/Notes/TISEDelta.pdf}{unm},
%%\href{http://www.phys.ufl.edu/~rfield/PHY4604/images/Chapter2_20.pdf}{ufl}.

\subsection{Solution}

The pair of figures in the text \cite{desai2009quantum} for these potentials doesn't make it clear that there are possibly any similarities.  The attractive delta function potential isn't illustrated (although the delta function is, but with opposite sign), and the scaling and the reference energy levels are different.  Let's illustrate these using the same reference energy level and sign conventions to make the similarities more obvious.

% FIXME: accidentally deleted the original, and left a symlink.
%\begin{figure}[htp]
%\centering
%\includegraphics[totalheight=0.4\textheight]{FiniteWellPotential}
%\caption{8.1.2 Finite Well potential (with energy shifted downwards by $V_0$)}\label{fig:FiniteWellPotential}
%\end{figure}

\begin{figure}[htp]
\centering
\includegraphics[totalheight=0.4\textheight]{deltaFunctionPotential}
\caption{8.2 Delta function potential.}\label{fig:deltaFunctionPotential}
\end{figure}
%figure (\ref{fig:deltaFunctionPotential})

The physics isn't changed by picking a different point for the reference energy level, so let's compare the two potentials, and their solutions using $V(x) = 0$ outside of the well for both cases.  The method used to solve the finite well problem in the text is hard to follow, so re-doing this from scratch in a slightly tidier way doesn't hurt.

Schr\"{o}dinger's equation for the finite well, in the $\Abs{x} > a$ region is

\begin{align}\label{eqn:qmIproblemSet4:110}
-\frac{\hbar^2}{2m} u'' = E u = - E_B u,
\end{align}

where a positive bound state energy $E_B = -E > 0$ has been introduced.

Writing 
\begin{align}\label{eqn:qmIproblemSet4:115}
\beta = \sqrt{\frac{2 m E_B}{\hbar^2}},
\end{align}

the wave functions outside of the well are 
\begin{align}\label{eqn:qmIproblemSet4:120}
u(x) =
\left\{
\begin{array}{l l}
u(-a) e^{\beta(x+a)} &\quad \mbox{$x < -a$} \\
u(a) e^{-\beta(x-a)} &\quad \mbox{$x > a$} \\
\end{array}
\right.
\end{align}

Within the well Schr\"{o}dinger's equation is
\begin{align}\label{eqn:qmIproblemSet4:125}
-\frac{\hbar^2}{2m} u'' - V_0 u = E u = - E_B u,
\end{align}

or
\begin{align}\label{eqn:qmIproblemSet4:126}
\frac{\hbar^2}{2m} u'' = - \frac{2m}{\hbar^2} (V_0 - E_B) u,
\end{align}

Noting that the bound state energies are the $E_B < V_0$ values, let $\alpha^2 = 2m (V_0 - E_B)/\hbar^2$, so that the solutions are of the form
\begin{align}\label{eqn:qmIproblemSet4:130}
u(x) = A e^{i\alpha x} + B e^{-i\alpha x}.
\end{align}

As was done for the wave functions outside of the well, the normalization constants can be expressed in terms of the values of the wave functions on the boundary.  That provides a pair of equations to solve

\begin{align}\label{eqn:qmIproblemSet4:135}
\begin{bmatrix}
u(a) \\
u(-a)
\end{bmatrix}
=
\begin{bmatrix}
e^{i \alpha a} & e^{-i \alpha a} \\
e^{-i \alpha a} & e^{i \alpha a}
\end{bmatrix}
\begin{bmatrix}
A \\
B
\end{bmatrix}.
\end{align}

Inverting this and substitution back into \ref{eqn:qmIproblemSet4:130} yields
\begin{align*}
u(x) 
&=
\begin{bmatrix}
e^{i\alpha x} & e^{-i\alpha x}
\end{bmatrix}
\begin{bmatrix}
A \\
B
\end{bmatrix} \\
&=
\begin{bmatrix}
e^{i\alpha x} & e^{-i\alpha x}
\end{bmatrix}
\inv{e^{2 i \alpha a} - e^{-2 i \alpha a}}
\begin{bmatrix}
e^{i \alpha a} & -e^{-i \alpha a} \\
-e^{-i \alpha a} & e^{i \alpha a}
\end{bmatrix}
\begin{bmatrix}
u(a) \\
u(-a)
\end{bmatrix} \\
&=
\begin{bmatrix}
\frac{\sin(\alpha (a + x))}{\sin(2 \alpha a)} &
\frac{\sin(\alpha (a - x))}{\sin(2 \alpha a)}
\end{bmatrix}
\begin{bmatrix}
u(a) \\
u(-a)
\end{bmatrix}.
\end{align*}

Expanding the last of these matrix products the wave function is close to completely specified.

\begin{equation}\label{eqn:qmIproblemSet4:140}
u(x) =
\left\{
\begin{array}{l l}
u(-a) e^{\beta(x+a)}
 & \quad \mbox{$x < -a$} \\
u(a) \frac{\sin(\alpha (a + x))}{\sin(2 \alpha a)} +
u(-a) \frac{\sin(\alpha (a - x))}{\sin(2 \alpha a)}
 & \quad \mbox{$\Abs{x} < a$} \\
u(a) e^{-\beta(x-a)}
 & \quad \mbox{$x > a$} \\
\end{array}
\right.
\end{equation}

There are still two unspecified constants $u(\pm a)$ and the constraints on $E_B$ have not been determined (both $\alpha$ and $\beta$ are functions of that energy level).  It should be possible to eliminate at least one of the $u(\pm a)$ by computing the wavefunction normalization, and since the well is being narrowed the $\alpha$ term will not be relevant.  Since only the vanishingly narrow case where $a \rightarrow 0, x \in [-a,a]$ is of interest, the wave function in that interval approaches

\begin{equation}\label{eqn:qmIproblemSet4:145}
u(x) \rightarrow \inv{2} (u(a) + u(-a)) + \frac{x}{2} ( u(a) - u(-a) ) \rightarrow \inv{2} (u(a) + u(-a)).
\end{equation}

Since no discontinuity is expected this is just $u(a) = u(-a)$.  Let's write $\lim_{a\rightarrow 0} u(a) = A$ for short, and the limited width well wave function becomes

\begin{equation}\label{eqn:qmIproblemSet4:150}
u(x) =
\left\{
\begin{array}{l l}
A e^{\beta x}
 & \quad \mbox{$x < 0$} \\
A e^{-\beta x}
 & \quad \mbox{$x > 0$} \\
\end{array}
\right.
\end{equation}

This is now the same form as the delta function potential, and normalization also gives $A = \sqrt{\beta}$.

One task remains before the attractive delta function potential can be considered a limiting case for the finite well, since the relation between $a, V_0$, and $g$ has not been established.  To do so integrate the Schr\"{o}dinger equation over the infinitesimal range $[-a,a]$.  This was done in the text for the delta function potential, and that provided the relation

\begin{equation}\label{eqn:qmIproblemSet4:155a}
\beta = \frac{mg}{\hbar^2}
\end{equation}

For the finite well this is

\begin{equation}\label{eqn:qmIproblemSet4:151}
\int_{-a}^a -\frac{\hbar^2}{2m} u'' - V_0 \int_{-a}^a u = -E_B \int_{-a}^a u \\
\end{equation}

In the limit as $a \rightarrow 0$ this is
\begin{equation}\label{eqn:qmIproblemSet4:152}
\frac{\hbar^2}{2m} (u'(a) - u'(-a)) + V_0 2 a u(0) = 2 E_B a u(0).
\end{equation}

Some care is required with the $V_0 a$ term since $a \rightarrow 0$ as $V_0 \rightarrow \infty$, but the $E_B$ term is unambiguously killed, leaving
\begin{equation}\label{eqn:qmIproblemSet4:153}
\frac{\hbar^2}{2m} u(0) (-2\beta e^{-\beta a}) = -V_0 2 a u(0).
\end{equation}

The exponential vanishes in the limit and leaves

\begin{equation}\label{eqn:qmIproblemSet4:155}
\beta = \frac{m (2 a) V_0}{\hbar^2}
\end{equation}

Comparing to \ref{eqn:qmIproblemSet4:155a} from the attractive delta function completes the problem.  The conclusion is that when the finite well is narrowed with $a \rightarrow 0$, also letting $V_0 \rightarrow \infty$ such that the absolute area of the well $g = (2 a) V_0$ is maintained, the finite potential well produces exactly the attractive delta function wave function and associated bound state energy.

\section{Problem 2.}
\subsection{Statement}

For the hydrogen atom, determine $\bra{nlm}(1/R)\ket{nlm}$ and $1/\bra{nlm}R\ket{nlm}$ such that $(nlm)=(211)$ and $R$ is the radial position operator $(X^2+Y^2+Z^2)^{1/2}$. What do these quantities represent physically and are they the same?

\subsection{Solution}

Both of the computation tasks for the hydrogen like atom require expansion of a braket of the following form

\begin{equation}\label{eqn:qmIproblemSet4:200}
\bra{nlm} A(R) \ket{nlm},
\end{equation}

where $A(R) = R = (X^2 + Y^2 + Z^2)^{1/2}$ or $A(R) = 1/R$.

The spherical representation of the identity resolution is required to convert this braket into integral form

\begin{equation}\label{eqn:qmIproblemSet4:202}
\BOne = \int r^2 \sin\theta dr d\theta d\phi 
\ket{ r \theta \phi}
\bra{ r \theta \phi},
\end{equation}

where the spherical wave function is given by the braket $\braket{ r \theta \phi}{nlm} = R_{nl}(r) Y_{lm}(\theta,\phi)$.

Additionally, the radial form of the delta function will be required, which is
\begin{equation}\label{eqn:qmIproblemSet4:204}
\delta(\Bx - \Bx') = \inv{r^2 \sin\theta} \delta(r - r') \delta(\theta - \theta') \delta(\phi - \phi')
\end{equation}

Two applications of the identity operator to the braket yield
\begin{align*}
&\bra{nlm} A(R) \ket{nlm} \\
&=\bra{nlm} \BOne A(R) \BOne \ket{nlm} \\
&=
\int 
dr d\theta d\phi 
dr' d\theta' d\phi'
r^2 \sin\theta 
{r'}^2 \sin\theta' 
\braket{nlm}{ r \theta \phi}
\bra{ r \theta \phi} A(R) 
\ket{ r' \theta' \phi'}
\braket{ r' \theta' \phi'}{nlm} \\
&=
\int 
dr d\theta d\phi 
dr' d\theta' d\phi'
r^2 \sin\theta 
{r'}^2 \sin\theta' 
R_{nl}(r) Y_{lm}^\conj(\theta, \phi)
\bra{ r \theta \phi} A(R) \ket{ r' \theta' \phi'}
R_{nl}(r') Y_{lm}(\theta', \phi') \\
\end{align*}

To continue an assumption about the matrix element $\bra{ r \theta \phi} A(R) \ket{ r' \theta' \phi'}$ is required.  It seems reasonable that this would be

\begin{equation}\label{eqn:qmIproblemSet4:206}
\bra{ r \theta \phi} A(R) \ket{ r' \theta' \phi'} = \\
\delta(\Bx - \Bx') A(r) = \inv{r^2 \sin\theta} \delta(r-r') \delta(\theta -\theta')\delta(\phi-\phi') A(r).
\end{equation}

The braket can now be written completely in integral form as
\begin{align*}
&\bra{nlm} A(R) \ket{nlm} \\
&=
\int 
dr d\theta d\phi 
dr' d\theta' d\phi'
r^2 \sin\theta 
{r'}^2 \sin\theta' 
R_{nl}(r) Y_{lm}^\conj(\theta, \phi)
\inv{r^2 \sin\theta} \delta(r-r') \delta(\theta -\theta')\delta(\phi-\phi') A(r)
R_{nl}(r') Y_{lm}(\theta', \phi') \\
&=
\int 
dr d\theta d\phi 
{r'}^2 \sin\theta' dr' d\theta' d\phi'
R_{nl}(r) Y_{lm}^\conj(\theta, \phi)
\delta(r-r') \delta(\theta -\theta')\delta(\phi-\phi') A(r)
R_{nl}(r') Y_{lm}(\theta', \phi') \\
\end{align*}

Application of the delta functions then reduces the integral, since the only $\theta$, and $\phi$ dependence is in the (orthonormal) $Y_{lm}$ terms they are found to drop out

\begin{align*}
\bra{nlm} A(R) \ket{nlm}
&=
\int 
dr d\theta d\phi 
r^2 \sin\theta 
R_{nl}(r) Y_{lm}^\conj(\theta, \phi)
A(r)
R_{nl}(r) Y_{lm}(\theta, \phi) \\
&=
\int 
dr 
r^2 
R_{nl}(r) 
A(r)
R_{nl}(r) 
\underbrace{\int
\sin\theta d\theta d\phi 
Y_{lm}^\conj(\theta, \phi)
Y_{lm}(\theta, \phi) }_{=1}
\\
\end{align*}

This leaves just the radial wave functions in the integral
\begin{equation}\label{eqn:qmIproblemSet4:208}
\bra{nlm} A(R) \ket{nlm}
=
\int 
dr 
r^2 
R_{nl}^2(r) 
A(r)
\end{equation}

As a consistency check, observe that with $A(r) = 1$, this integral evaluates to 1 according to equation (8.274) in the text, so we can think of $(r R_{nl}(r))^2$ as the radial probability density for functions of $r$.

The problem asks specifically for these expectation values for the $\ket{211}$ state.  For that state the radial wavefunction is found in (8.277) as

\begin{equation}\label{eqn:qmIproblemSet4:210}
R_{21}(r) = 
\left(\frac{Z}{2a_0}\right)^{3/2} \frac{ Z r }{a_0 \sqrt{3}} e^{-Z r/2 a_0}
\end{equation}

The braket can now be written explicitly 
\begin{equation}\label{eqn:qmIproblemSet4:212}
\bra{21m} A(R) \ket{21m}
=
%\left(\frac{Z}{2a_0}\right)^{3} \frac{ Z^2 }{3 a_0^2 } 
\inv{24} \left(\frac{ Z }{a_0 } \right)^5
\int_0^\infty
dr 
r^4 
e^{-Z r/ a_0}
A(r)
\end{equation}

Now, let's consider the two functions $A(r)$ separately.  First for $A(r) = r$ we have
\begin{align*}
\bra{21m} R \ket{21m}
&=
\inv{24} \left(\frac{ Z }{a_0 } \right)^5
\int_0^\infty
dr 
r^5 
e^{-Z r/ a_0} 
\\
&=
\frac{ a_0 }{ 24 Z } 
\int_0^\infty
du 
u^5 
e^{-u} 
\\
\end{align*}

% 5 a_0 / Z
The last integral evaluates to $120$, leaving
\begin{equation}\label{eqn:qmIproblemSet4:220a}
\bra{21m} R \ket{21m}
=
\frac{ 5 a_0 }{ Z }.
\end{equation}

The expectation value associated with this $\ket{21m}$ state for the radial position is found to be proportional to the Bohr radius.  For the hydrogen atom where $Z=1$ this average value for repeated measurements of the physical quantity associated with the operator $R$ is found to be 5 times the Bohr radius for $n=2, l=1$ states.

Our problem actually asks for the inverse of this expectation value, and for reference this is
\begin{equation}\label{eqn:qmIproblemSet4:220}
1/ \bra{21m} R \ket{21m}
=
\frac{ Z }{ 5 a_0 } %= n^2 \frac{a_0}{Z}
\end{equation}

Performing the same task for $A(R) = 1/R$
\begin{align*}
\bra{21m} 1/R \ket{21m}
&=
\inv{24} \left(\frac{ Z }{a_0 } \right)^5
\int_0^\infty
dr 
r^3
e^{-Z r/ a_0} 
\\
&=
\inv{24} \frac{ Z }{ a_0 } 
\int_0^\infty
du 
u^3
e^{-u}.
\end{align*}

This last integral has value $6$, and we have the second part of the computational task complete
\begin{equation}\label{eqn:qmIproblemSet4:225}
\bra{21m} 1/R \ket{21m} = \inv{4} \frac{ Z }{ a_0 } 
\end{equation}

The question of whether or not \ref{eqn:qmIproblemSet4:220}, and \ref{eqn:qmIproblemSet4:225} are equal is answered.  They are not.

Still remaining for this problem is the question of the what these quantities represent physically.

The quantity $\bra{nlm} R \ket{nlm}$ is the expectation value for the radial position of the particle measured from the center of mass of the system.  This is the average outcome for many measurements of this radial distance when the system is prepared in the state $\ket{nlm}$ prior to each measurement.

Interestingly, the physical quantity that we associate with the operator $R$ has a different measurable value than the inverse of the expectation value for the inverted operator $1/R$.  Regardless, we have a physical (observable) quantity associated with the operator $1/R$, and when the system is prepared in state $\ket{21m}$ prior to each measurement, the average outcome of many measurements of this physical quantity produces this value $\bra{21m} 1/R \ket{21m} = Z/n^2 a_0$, a quantity inversely proportional to the Bohr radius.

\subsection{ASIDE: Comparing to the general case.}

As a confirmation of the results obtained, we can check \ref{eqn:qmIproblemSet4:220}, and \ref{eqn:qmIproblemSet4:225} against the general form of the expectation values $\expectation{R^s}$ for various powers $s$ of the radial position operator.  These can be found in locations such as \href{http://farside.ph.utexas.edu/teaching/qmech/lectures/node81.html}{farside.ph.utexas.edu} which gives for $Z=1$ (without proof), and in \cite{liboff2003iqm} (where these and harder looking ones expectation values are left as an exercise for the reader to prove).  Both of those give:

\begin{align}\label{eqn:qmIproblemSet4:226}
\expectation{R} &= \frac{a_0}{2} ( 3 n^2 -l (l+1) ) \\
\expectation{1/R} &= \frac{1}{n^2 a_0} 
\end{align}

It is curious to me that the general expectation values noted in \ref{eqn:qmIproblemSet4:226} we have a $l$ quantum number dependence for $\expectation{R}$, but only the $n$ quantum number dependence for $\expectation{1/R}$.  It is not obvious to me why this would be the case.

%%%\section{Suggested non-submission problem 1.}
%%%\subsection{Statement}
%%%In Section 8.1.1, why are boundary conditions imposed on $u(x)$ at $x=\pm a$? 
%%%\subsection{Solution}
%%%
%%%\section{Suggested non-submission problem 2.}
%%%\subsection{Statement}
%%%
%%%In Section 8.1.2, what is the probability current density for $u(x)$. Does this make physical sense? 
%%%
%%%\subsection{Solution}
%%%
%%%\section{Suggested non-submission problem 3.}
%%%\subsection{Statement}
%%%
%%%Why are the states described in Sections 8.6.1 and 8.8 degenerate?
%%%
%%%\subsection{Solution}

%\subsection{Sanity check.}
%\begin{align*}
%\braket{nlm}{nlm}
%&=
%\inv{24} \left(\frac{ Z }{a_0 } \right)^5
%\int_0^\infty
%dr 
%r^4 
%e^{-Z r/ a_0} \\
%&=
%\inv{24} 
%\int_0^\infty
%du 
%u^4 
%e^{-u} \\
%&= 1
%\end{align*}

\EndArticle
%\EndNoBibArticle
