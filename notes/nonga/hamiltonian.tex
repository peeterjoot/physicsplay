\documentclass[]{eliblog}

\usepackage{amsmath}
\usepackage{mathpazo}

%
% shorthand for bold symbols, convenient for vectors and matrices
%
\newcommand{\Ba}[0]{\mathbf{a}}
\newcommand{\Bb}[0]{\mathbf{b}}
\newcommand{\Bc}[0]{\mathbf{c}}
\newcommand{\Bd}[0]{\mathbf{d}}
\newcommand{\Be}[0]{\mathbf{e}}
\newcommand{\Bf}[0]{\mathbf{f}}
\newcommand{\Bg}[0]{\mathbf{g}}
\newcommand{\Bh}[0]{\mathbf{h}}
\newcommand{\Bi}[0]{\mathbf{i}}
\newcommand{\Bj}[0]{\mathbf{j}}
\newcommand{\Bk}[0]{\mathbf{k}}
\newcommand{\Bl}[0]{\mathbf{l}}
\newcommand{\Bm}[0]{\mathbf{m}}
\newcommand{\Bn}[0]{\mathbf{n}}
\newcommand{\Bo}[0]{\mathbf{o}}
\newcommand{\Bp}[0]{\mathbf{p}}
\newcommand{\Bq}[0]{\mathbf{q}}
\newcommand{\Br}[0]{\mathbf{r}}
\newcommand{\Bs}[0]{\mathbf{s}}
\newcommand{\Bt}[0]{\mathbf{t}}
\newcommand{\Bu}[0]{\mathbf{u}}
\newcommand{\Bv}[0]{\mathbf{v}}
\newcommand{\Bw}[0]{\mathbf{w}}
\newcommand{\Bx}[0]{\mathbf{x}}
\newcommand{\By}[0]{\mathbf{y}}
\newcommand{\Bz}[0]{\mathbf{z}}
\newcommand{\BA}[0]{\mathbf{A}}
\newcommand{\BB}[0]{\mathbf{B}}
\newcommand{\BC}[0]{\mathbf{C}}
\newcommand{\BD}[0]{\mathbf{D}}
\newcommand{\BE}[0]{\mathbf{E}}
\newcommand{\BF}[0]{\mathbf{F}}
\newcommand{\BG}[0]{\mathbf{G}}
\newcommand{\BH}[0]{\mathbf{H}}
\newcommand{\BI}[0]{\mathbf{I}}
\newcommand{\BJ}[0]{\mathbf{J}}
\newcommand{\BK}[0]{\mathbf{K}}
\newcommand{\BL}[0]{\mathbf{L}}
\newcommand{\BM}[0]{\mathbf{M}}
\newcommand{\BN}[0]{\mathbf{N}}
\newcommand{\BO}[0]{\mathbf{O}}
\newcommand{\BP}[0]{\mathbf{P}}
\newcommand{\BQ}[0]{\mathbf{Q}}
\newcommand{\BR}[0]{\mathbf{R}}
\newcommand{\BS}[0]{\mathbf{S}}
\newcommand{\BT}[0]{\mathbf{T}}
\newcommand{\BU}[0]{\mathbf{U}}
\newcommand{\BV}[0]{\mathbf{V}}
\newcommand{\BW}[0]{\mathbf{W}}
\newcommand{\BX}[0]{\mathbf{X}}
\newcommand{\BY}[0]{\mathbf{Y}}
\newcommand{\BZ}[0]{\mathbf{Z}}

\newcommand{\Bzero}[0]{\mathbf{0}}
\newcommand{\Btheta}[0]{\boldsymbol{\theta}}
\newcommand{\Btau}[0]{\boldsymbol{\tau}}
\newcommand{\Bomega}[0]{\boldsymbol{\omega}}

%
% shorthand for unit vectors
%
\newcommand{\acap}[0]{\hat{\Ba}}
\newcommand{\bcap}[0]{\hat{\Bb}}
\newcommand{\ccap}[0]{\hat{\Bc}}
\newcommand{\dcap}[0]{\hat{\Bd}}
\newcommand{\ecap}[0]{\hat{\Be}}
\newcommand{\fcap}[0]{\hat{\Bf}}
\newcommand{\gcap}[0]{\hat{\Bg}}
\newcommand{\hcap}[0]{\hat{\Bh}}
\newcommand{\icap}[0]{\hat{\Bi}}
\newcommand{\jcap}[0]{\hat{\Bj}}
\newcommand{\kcap}[0]{\hat{\Bk}}
\newcommand{\lcap}[0]{\hat{\Bl}}
\newcommand{\mcap}[0]{\hat{\Bm}}
\newcommand{\ncap}[0]{\hat{\Bn}}
\newcommand{\ocap}[0]{\hat{\Bo}}
\newcommand{\pcap}[0]{\hat{\Bp}}
\newcommand{\qcap}[0]{\hat{\Bq}}
\newcommand{\rcap}[0]{\hat{\Br}}
\newcommand{\scap}[0]{\hat{\Bs}}
\newcommand{\tcap}[0]{\hat{\Bt}}
\newcommand{\ucap}[0]{\hat{\Bu}}
\newcommand{\vcap}[0]{\hat{\Bv}}
\newcommand{\wcap}[0]{\hat{\Bw}}
\newcommand{\xcap}[0]{\hat{\Bx}}
\newcommand{\ycap}[0]{\hat{\By}}
\newcommand{\zcap}[0]{\hat{\Bz}}
\newcommand{\thetacap}[0]{\hat{\Btheta}}

%
% to write R^n and C^n in a distinguishable fashion.  Perhaps change this
% to the double lined characters upon figuring out how to do so.
%
\newcommand{\C}[1]{$\mathbb{C}^{#1}$}
\newcommand{\R}[1]{$\mathbb{R}^{#1}$}

%
% various generally useful helpers
%

% derivative of #1 wrt. #2:
\newcommand{\D}[2] {\frac {d#2} {d#1}}

\newcommand{\inv}[1]{\frac{1}{#1}}
\newcommand{\cross}[0]{\times}

\newcommand{\abs}[1]{\lvert{#1}\rvert}
\newcommand{\norm}[1]{\lVert{#1}\rVert}
\newcommand{\innerprod}[2]{\langle{#1}, {#2}\rangle}
\newcommand{\dotprod}[2]{{#1} \cdot {#2}}
\newcommand{\bdotprod}[2]{\left({#1} \cdot {#2}\right)}
\newcommand{\crossprod}[2]{{#1} \cross {#2}}
\newcommand{\tripleprod}[3]{\dotprod{\left(\crossprod{#1}{#2}\right)}{#3}}

\DeclareMathOperator{\Proj}{Proj}
\DeclareMathOperator{\Span}{span}
\DeclareMathOperator{\Sgn}{sgn}
\DeclareMathOperator{\Area}{Area}
\DeclareMathOperator{\Volume}{Volume}

%
% A few miscellaneous things specific to this document
%
\newcommand{\crossop}[1]{\crossprod{#1}{}}

% R2 vector.
\newcommand{\VectorTwo}[2]{
\begin{bmatrix}
 {#1} \\
 {#2}
\end{bmatrix}
}

\newcommand{\VectorN}[1]{
\begin{bmatrix}
{#1}_1 \\
{#1}_2 \\
\vdots \\
{#1}_N \\
\end{bmatrix}
}

\newcommand{\DETuvij}[4]{
\begin{vmatrix}
 {#1}_{#3} & {#1}_{#4} \\
 {#2}_{#3} & {#2}_{#4}
\end{vmatrix}
}

\newcommand{\DETuvwijk}[6]{
\begin{vmatrix}
 {#1}_{#4} & {#1}_{#5} & {#1}_{#6} \\
 {#2}_{#4} & {#2}_{#5} & {#2}_{#6} \\
 {#3}_{#4} & {#3}_{#5} & {#3}_{#6}
\end{vmatrix}
}

\newcommand{\DETuvwxijkl}[8]{
\begin{vmatrix}
 {#1}_{#5} & {#1}_{#6} & {#1}_{#7} & {#1}_{#8} \\
 {#2}_{#5} & {#2}_{#6} & {#2}_{#7} & {#2}_{#8} \\
 {#3}_{#5} & {#3}_{#6} & {#3}_{#7} & {#3}_{#8} \\
 {#4}_{#5} & {#4}_{#6} & {#4}_{#7} & {#4}_{#8} \\
\end{vmatrix}
}

%\newcommand{\DETuvwxyijklm}[10]{
%\begin{vmatrix}
% {#1}_{#6} & {#1}_{#7} & {#1}_{#8} & {#1}_{#9} & {#1}_{#10} \\
% {#2}_{#6} & {#2}_{#7} & {#2}_{#8} & {#2}_{#9} & {#2}_{#10} \\
% {#3}_{#6} & {#3}_{#7} & {#3}_{#8} & {#3}_{#9} & {#3}_{#10} \\
% {#4}_{#6} & {#4}_{#7} & {#4}_{#8} & {#4}_{#9} & {#4}_{#10} \\
% {#5}_{#6} & {#5}_{#7} & {#5}_{#8} & {#5}_{#9} & {#5}_{#10}
%\end{vmatrix}
%}

% R3 vector.
\newcommand{\VectorThree}[3]{
\begin{bmatrix}
 {#1} \\
 {#2} \\
 {#3}
\end{bmatrix}
}



\author{Peeter Joot}
\email{peeter.joot@gmail.com}


\chapter{Hamiltonian notes.}
\label{chap:hamiltonian}
%\useCCL
\blogpage{http://sites.google.com/site/peeterjoot/math2009/hamiltonian.pdf}
\date{Sept 26, 2009}
\revisionInfo{$RCSfile: hamiltonian.tex,v $ Last $Revision: 1.3 $ $Date: 2009/09/27 03:56:00 $}

\beginArtWithToc

\section{Motivation}

I've now seen Hamiltonians used, mostly in a Quantum context, and think that I understand at least some of the math associated with the Hamiltonian and the Hamiltonian principle.  I have, however, not used either of these enough that it seems natural to do so.

Here I attempt to summarize for myself what I know about Hamiltonians, and work through a number of examples.  Some of the examples considered will be ones already treated with the Lagrangian formalism \cite{PJTongMf1}.

Some notation will be invented along the way as reasonable, since I'd like to try to also relate the usual coordinate representation of the Hamiltonian, the Hamiltonian principle, and the Poisson bracket, with the bivector representation of the 2N complex configuration space introduced in \cite{doran2003gap}.

\section{Hamiltonian as a conserved quantity.}

Starting with the Lagrangian formalism the Hamiltonian can be found as a conserved quantity associated with time translation when the Lagrangian has no explicit time dependence.  This follows directly by considering the time derivative of the Lagrangian $\LL = \LL(q^i, \qdot^i)$.

\begin{align*}
\frac{d\LL}{dt} 
&= \PD{q^i}{\LL} \frac{dq^i}{dt} +\PD{\qdot^i}{\LL} \frac{d\qdot^i}{dt} \\
&= \qdot^i \frac{d}{dt}\PD{\qdot^i}{\LL} +\PD{\qdot^i}{\LL} \frac{d\qdot^i}{dt} \\
&= \frac{d}{dt} \left( \qdot^i \PD{\qdot^i}{\LL} \right) 
\end{align*}

We can therefore form the difference 

\begin{align}\label{eqn:hamiltonian:foo1}
\frac{d}{dt} \left( \qdot^i \PD{\qdot^i}{\LL} -\LL \right) = 0
\end{align}

and find that this quantity, labelled H, is a constant of motion for the system

\begin{align}\label{eqn:hamiltonian:foo2}
H \equiv \qdot^i \PD{\qdot^i}{\LL} -\LL = \text{constant}
\end{align}

We'll see later that this constant is sometimes the total energy of the system.

The $\qdot^i$ partials of the Lagrangian are called the canonical momentum conjugate to $q^i$.  Quite a mouthful, so just canonical momenta seems like a good compromize.  We will write (reserving $p^i = m q^i$ for the non-canonical momenta)

\begin{align}\label{eqn:hamiltonian:foo2b}
P_i \equiv \PD{\qdot^i}{\LL}
\end{align}

and note that these are the coordinates of a sort of velocity gradient of the Lagrangian.  We've seen these canonical momenta in velocity gradient form previously where it was noted that we could write the Euler-Lagrange equations in vector form in an orthonormal reciprocal frame space as

\begin{align}\label{eqn:hamiltonian:foo9}
\grad \LL = \frac{d}{dt} \grad_v \LL
\end{align}

where $\grad_v = e^i \partial \LL/\partial \xdot^i = e^i P_i$, $\grad = e^i \partial/\partial x^i$, and $x = e_i x^i$.

%Here we will be exploring phase space relationships where the position and velocity basis pairs are treated independently, but also will not have much requirement for direct use of the Euler Lagrange equations.

\section{Some syntaxic sugar.  In vector form.}

Following Jackson \cite{jackson1975cew} (section 12.1, relativistic Lorentz force Hamiltonian), this can be written in vector form if the velocity gradient, the vector sum of the momenta conjugate to the $q^i$'s is given its own symbol $\BP$.  He writes

\begin{align}\label{eqn:hamiltonian:foo3}
H = \Bv \cdot \BP - \LL
\end{align}

This makes most sense when working in othonormal coordinates, but can be generalized.  Suppose we introduce a pair of reciprocal frame basis for the generalized position and velocity coordinates, writing as vectors in configuration space

\begin{align}\label{eqn:hamiltonian:foo4}
q &= e_i q^i \\
v &= f_i \qdot^i 
\end{align}

Following \cite{doran2003gap} (who use this for their bivector complexification of the configuration space), we have the freedom to impose orthonormal constraints on this configuration space basis

\begin{align}\label{eqn:hamiltonian:foo5}
e^i \cdot e_j &= {\delta^i}_j \\
f^i \cdot f_j &= {\delta^i}_j \\
e^i \cdot f_j &= {\delta^i}_j
\end{align}

We can now define configuration space position and velocity gradients

\begin{align}\label{eqn:hamiltonian:foo6}
\grad &\equiv e^i \PD{q^i}{} \\
\grad_v &\equiv f^i \PD{\qdot^i}{}
\end{align}

so the conjugate momenta in vector form is now

\begin{align}\label{eqn:hamiltonian:foo7}
P \equiv \grad_v \LL = f^i \PD{\qdot^i}{\LL}
\end{align}

Our Hamiltonian takes the form

\begin{align}\label{eqn:hamiltonian:foo8}
H = v \cdot P - \LL
\end{align}

\section{The Hamiltonian principle.}

We want to take partials of \ref{eqn:hamiltonian:foo2} with respect to $P_i$ and $q^i$.  In terms of the canonical momenta we want to differentiate

\begin{align}\label{eqn:hamiltonian:hoo1}
H \equiv \qdot^i P_i -\LL(q^i, \qdot^i, t)
\end{align}

for the $P_i$ partial we have

\begin{align*}
\PD{P_i}{H} = \qdot^i
\end{align*}

and for the $q^i$ partial

\begin{align*}
\PD{q^i}{H} 
&= -\PD{q^i}{\LL} \\
&= - \frac{d}{dt} \PD{\qdot^i}{\LL} 
\end{align*}

These two results taken together form what I believe is called the Hamiltonian principle

\begin{align}\label{eqn:hamiltonian:hoo3}
\PD{P_i}{H} &= \qdot^i \\
\PD{q^i}{H} &= - \dot{P}_i \\
P_i &= \PD{\qdot^i}{\LL} 
\end{align}

A set of 2N first order equations equivalent to the second order Euler-Lagrange equations.  These appear to follow straight from the definitions.  Given that I'm curious why the more complex method of derivation is chosen in \cite{goldstein1951cm}.  There the total differential of the Hamiltonian is computed

\begin{align*}
dH &= 
\qdot^i dP_i 
+ d\qdot^i P_i 
- dq^i \PD{q^i}{\LL}
- d \qdot^i \PD{\qdot^i}{\LL}
- dt \PD{t}{\LL} \\
&= 
\qdot^i dP_i 
+ d\qdot^i \left( P_i - \PD{\qdot^i}{\LL} \right)
- dq^i \PD{q^i}{\LL}
- dt \PD{t}{\LL} \\
&= 
\qdot^i dP_i 
- dq^i \underbrace{\PD{q^i}{\LL}}_{= d/dt P_i}
- dt \PD{t}{\LL} \\
\end{align*}

A term by term comparision to the total differential written out explicitly

\begin{align}\label{eqn:hamiltonian:hoo4}
dH &= 
\PD{q^i}{H} d q^i
+\PD{P_i}{H} d P_i
+\PD{t}{H} dt
\end{align}

allows the Hamiltonian equations to be picked off.

\begin{align}\label{eqn:hamiltonian:hoo5}
\PD{P_i}{H} &= \qdot^i  \\
\PD{q^i}{H} &= - \dot{P}_i  \\
\PD{t}{H}   &= - \PD{t}{\LL} 
\end{align}

I guess that isn't that much more complicated and it does yield a relation between the Hamiltonian and Lagrangian time derivatives.

\section{Examples.}

Now, that's just about the most abstract way we can start things off isn't it?  Getting some initial feel for this constant of motion can be had by considering a sequence of Lagrangians, starting with the very simplest.

\subsection{Force free motion}

Our very simplest Lagrangian is that of one dimensional purely kinetic motion

\begin{align}\label{eqn:hamiltonian:boo1}
\LL = \inv{2} m v^2 = \inv{2} m \xdot^2
\end{align}

Our Hamiltonian is in this case just

\begin{align}\label{eqn:hamiltonian:boo2}
H = \xdot m \xdot - \inv{2} m \xdot = \inv{2} m v^2
\end{align}

The Hamiltonian is just the kinetic energy.  The canonical momentum in this case is also equal to the momentum, so eliminating $v$ to apply the Hamiltonian equations we have

\begin{align}\label{eqn:hamiltonian:boo3}
H = \inv{2m} p^2
\end{align}

We have then

\begin{align*}
\PD{p}{H} &= \frac{p}{m} = \dot{x} \\
\PD{x}{H} &= 0 = -\dot{p} 
\end{align*}

Just for fun we can put this simple linear system in matrix form

\begin{align}\label{eqn:hamiltonian:boo4}
\frac{d}{dt}
\begin{bmatrix}
p \\
x
\end{bmatrix}
=
\inv{m}
\begin{bmatrix}
0 & 0 \\
1 & 0
\end{bmatrix}
\begin{bmatrix}
p \\
x
\end{bmatrix}
\end{align}

A linear system of this form $y' = A y$ can be solved by exponentiation with solution

\begin{align}\label{eqn:hamiltonian:boo5}
y = e^{A t} y_0
\end{align}

In this case our matrix is nilpotent degree 2 so we can exponentiate only requiring up to the first order power

\begin{align}\label{eqn:hamiltonian:boo6}
e^{A t} = I + A t
\end{align}

specifically

\begin{align}\label{eqn:hamiltonian:boo7}
\begin{bmatrix}
p \\
x
\end{bmatrix}
=
\begin{bmatrix}
1 & 0 \\
\frac{t}{m} & 1
\end{bmatrix}
\begin{bmatrix}
p_0 \\
x_0
\end{bmatrix}
\end{align}

Written out in full this is just

\begin{align}\label{eqn:hamiltonian:boo8}
p &= p_0 \\
x &= \frac{p_0}{m} t + x_0
\end{align}

Since the canonical momentum is the regular momentum $p = m v$ in this case, we have the usual constant rate change of position $x = v_0 t + x_0$ that we could have gotten in many easier ways.  I'd hazzard a guess that any single variable Lagrangian that is at most quadratic in position or velocity will yield a linear system.

The generalization of this Hamiltonian to three dimensions is straightforward, and we get

\begin{align}\label{eqn:hamiltonian:boo9}
H &= \inv{m} \Bp^2 
\end{align}

\begin{align}\label{eqn:hamiltonian:boo10}
\frac{d}{dt}
\begin{bmatrix}
p_x \\
x \\
p_y \\
y \\
p_z \\
z \\
\end{bmatrix}
=
\inv{m}
\begin{bmatrix}
0 & 0 &   &   &   &   \\
1 & 0 &   &   &   &   \\
  &   & 0 & 0 &   &   \\
  &   & 1 & 0 &   &   \\
  &   &   &   & 0 & 0 \\
  &   &   &   & 1 & 0 \\
\end{bmatrix}
\begin{bmatrix}
p_x \\
x \\
p_y \\
y \\
p_z \\
z \\
\end{bmatrix}
\end{align}

Since there is no coupling (nilpotent matrixes down the diagonal) between the coordinates this can be treated as three independent sets of equations of the form \ref{eqn:hamiltonian:boo4}, and we have

\begin{align}\label{eqn:hamiltonian:boo11}
p_i(t) &= p_i(0) \\
x_i(t) &= \frac{p_i(0)}{m} t + x_i(0)
\end{align}

Or just

\begin{align}\label{eqn:hamiltonian:boo12}
\Bp(t) &= \Bp(0) \\
\Bx(t) &= \frac{\Bp(0)}{m} t + \Bx(0)
\end{align}

\subsection{Linear potential (surface gravitation)}

For the gravitational force $F = - m g \zcap = - \spacegrad \phi$, we have $\phi = m g z$, and a Lagrangian of

\begin{align}\label{eqn:hamiltonian:roo1}
\LL = \inv{2} m \Bv^2 - \phi = \inv{2} m \Bv^2 - m g z
\end{align}

Without velocity dependence the canonical momentum is the momentum $m \Bv$, and our Hamiltonian is

\begin{align}\label{eqn:hamiltonian:roo2}
H = \inv{2 m} \Bp^2 + m g z
\end{align}

The Hamiltonian equations are

\begin{align}\label{eqn:hamiltonian:roo3}
\PD{p_i}{H} &= \xdot_i = \inv{m} p_i \\
\sigma_i \PD{x_i}{H} &= -\sigma_i \pdot_i = \begin{bmatrix}0 \\ 0 \\ m g \end{bmatrix}
\end{align}

In matrix form we have

\begin{align}\label{eqn:hamiltonian:roo4}
\frac{d}{dt}
\begin{bmatrix}
p_x \\
x \\
p_y \\
y \\
p_z \\
z \\
\end{bmatrix}
=
\inv{m}
\begin{bmatrix}
0 & 0 &   &   &   &   \\
1 & 0 &   &   &   &   \\
  &   & 0 & 0 &   &   \\
  &   & 1 & 0 &   &   \\
  &   &   &   & 0 & 0 \\
  &   &   &   & 1 & 0 \\
\end{bmatrix}
\begin{bmatrix}
p_x \\
x \\
p_y \\
y \\
p_z \\
z \\
\end{bmatrix}
+
\begin{bmatrix}
0 \\
0 \\
0 \\
0 \\
-m g \\
0 \\
\end{bmatrix}
\end{align}

So our problem is now reduced to solving a linear system of the form

\begin{align}\label{eqn:hamiltonian:roo5}
y' = A y + b
\end{align}

That extra little term $b$ throws a wrench into things and I'm no longer sure how to integrate by inspection.  What can be noted is that we really only have to consider the $z$ components since we've solved the problem for the $x$ and $y$ coordinates in the force free case.  That leaves

\begin{align}\label{eqn:hamiltonian:roo6}
\frac{d}{dt}
\begin{bmatrix}
p_z \\
z \\
\end{bmatrix}
=
\inv{m}
\begin{bmatrix}
0 & 0 \\
1 & 0 \\
\end{bmatrix}
\begin{bmatrix}
p_z \\
z \\
\end{bmatrix}
+
\begin{bmatrix}
-m g \\
0 \\
\end{bmatrix}
\end{align}

Is there any reason that we have to solve in matrix form?  Except for a coolness factor, not really, and we can integrate each equation directly.  For the momentum equation we have

\begin{align}\label{eqn:hamiltonian:roo7}
p_z = - m g t + p_z(0)
\end{align}

This can be substituted into the position equation for

\begin{align}\label{eqn:hamiltonian:roo8}
\dot{z} = \inv{m} (p_z(0) - m g t)
\end{align}

Direct integration is now possible for the final solution

\begin{align*}
z 
&= \inv{m} (p_z(0) t - m g t^2/2) + z_0 \\
&= \frac{p_z(0)}{m} t - \frac{g}{2} t^2 + z_0 
\end{align*}

Again something that we could have gotten in many easier ways.  Using the result we see that the solution to \ref{eqn:hamiltonian:roo6} in matrix form, again with $A = \inv{m}\begin{bmatrix}0 & 0 \\ 1 & 0\end{bmatrix}$ is

\begin{align}\label{eqn:hamiltonian:roo9}
\begin{bmatrix}
p_z \\
z \\
\end{bmatrix}
= e^{At} 
\begin{bmatrix}
p_z(0) \\
z(0) \\
\end{bmatrix}
- m g 
\begin{bmatrix}
t \\
\inv{2m} t^2
\end{bmatrix}
\end{align}

I thought if I wrote this out how to solve \ref{eqn:hamiltonian:roo5} may be more obvious, but that path is still unclear.  If $A$ were invertable, which it isn't, then writing $b = A c$ would allow for a change of variables.  Does this matter for consideration of a physical problem.  Not really, so I'll fight the urge to play with the math for a while and perhaps revisit this later separately.

\subsection{Harmonic oscillator (spring potential)}

\subsection{Gravitational potential}

\subsection{Pendulum}

\subsection{Spherical pendulum}

\subsection{Double pendulum}

\subsection{Particle in non-velocity dependent potential.}

\subsection{Velocity dependent potential.}

\subsection{Dangling mass connected by string to another.}

\subsection{Lorentz force Hamiltonian.}

\EndArticle
