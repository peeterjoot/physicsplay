\documentclass[12pt,leqno]{book}

\usepackage{amsmath,amssymb,amsfonts} % Typical maths resource packages
\usepackage{graphicx}
\usepackage{color}                   % For creating coloured text and background
\usepackage{txfonts} 
\usepackage{listings}
\usepackage[bookmarks=true,plainpages=false]{hyperref}

\parindent 1cm
\parskip 0.2cm
\topmargin 0.2cm
\oddsidemargin 1cm
\evensidemargin 0.5cm
\textwidth 15cm
\textheight 21cm

%% how do these ones work?
%\newtheorem{theorem}{Theorem}[section]
%\newtheorem{proposition}[theorem]{Proposition}
%\newtheorem{corollary}[theorem]{Corollary}
%\newtheorem{lemma}[theorem]{Lemma}
%\newtheorem{remark}[theorem]{Remark}
%\newtheorem{definition}[theorem]{Definition}

\usepackage{amsmath}
\usepackage{mathpazo}

%
% shorthand for bold symbols, convenient for vectors and matrices
%
\newcommand{\Ba}[0]{\mathbf{a}}
\newcommand{\Bb}[0]{\mathbf{b}}
\newcommand{\Bc}[0]{\mathbf{c}}
\newcommand{\Bd}[0]{\mathbf{d}}
\newcommand{\Be}[0]{\mathbf{e}}
\newcommand{\Bf}[0]{\mathbf{f}}
\newcommand{\Bg}[0]{\mathbf{g}}
\newcommand{\Bh}[0]{\mathbf{h}}
\newcommand{\Bi}[0]{\mathbf{i}}
\newcommand{\Bj}[0]{\mathbf{j}}
\newcommand{\Bk}[0]{\mathbf{k}}
\newcommand{\Bl}[0]{\mathbf{l}}
\newcommand{\Bm}[0]{\mathbf{m}}
\newcommand{\Bn}[0]{\mathbf{n}}
\newcommand{\Bo}[0]{\mathbf{o}}
\newcommand{\Bp}[0]{\mathbf{p}}
\newcommand{\Bq}[0]{\mathbf{q}}
\newcommand{\Br}[0]{\mathbf{r}}
\newcommand{\Bs}[0]{\mathbf{s}}
\newcommand{\Bt}[0]{\mathbf{t}}
\newcommand{\Bu}[0]{\mathbf{u}}
\newcommand{\Bv}[0]{\mathbf{v}}
\newcommand{\Bw}[0]{\mathbf{w}}
\newcommand{\Bx}[0]{\mathbf{x}}
\newcommand{\By}[0]{\mathbf{y}}
\newcommand{\Bz}[0]{\mathbf{z}}
\newcommand{\BA}[0]{\mathbf{A}}
\newcommand{\BB}[0]{\mathbf{B}}
\newcommand{\BC}[0]{\mathbf{C}}
\newcommand{\BD}[0]{\mathbf{D}}
\newcommand{\BE}[0]{\mathbf{E}}
\newcommand{\BF}[0]{\mathbf{F}}
\newcommand{\BG}[0]{\mathbf{G}}
\newcommand{\BH}[0]{\mathbf{H}}
\newcommand{\BI}[0]{\mathbf{I}}
\newcommand{\BJ}[0]{\mathbf{J}}
\newcommand{\BK}[0]{\mathbf{K}}
\newcommand{\BL}[0]{\mathbf{L}}
\newcommand{\BM}[0]{\mathbf{M}}
\newcommand{\BN}[0]{\mathbf{N}}
\newcommand{\BO}[0]{\mathbf{O}}
\newcommand{\BP}[0]{\mathbf{P}}
\newcommand{\BQ}[0]{\mathbf{Q}}
\newcommand{\BR}[0]{\mathbf{R}}
\newcommand{\BS}[0]{\mathbf{S}}
\newcommand{\BT}[0]{\mathbf{T}}
\newcommand{\BU}[0]{\mathbf{U}}
\newcommand{\BV}[0]{\mathbf{V}}
\newcommand{\BW}[0]{\mathbf{W}}
\newcommand{\BX}[0]{\mathbf{X}}
\newcommand{\BY}[0]{\mathbf{Y}}
\newcommand{\BZ}[0]{\mathbf{Z}}

\newcommand{\Bzero}[0]{\mathbf{0}}
\newcommand{\Btheta}[0]{\boldsymbol{\theta}}
\newcommand{\Btau}[0]{\boldsymbol{\tau}}
\newcommand{\Bomega}[0]{\boldsymbol{\omega}}

%
% shorthand for unit vectors
%
\newcommand{\acap}[0]{\hat{\Ba}}
\newcommand{\bcap}[0]{\hat{\Bb}}
\newcommand{\ccap}[0]{\hat{\Bc}}
\newcommand{\dcap}[0]{\hat{\Bd}}
\newcommand{\ecap}[0]{\hat{\Be}}
\newcommand{\fcap}[0]{\hat{\Bf}}
\newcommand{\gcap}[0]{\hat{\Bg}}
\newcommand{\hcap}[0]{\hat{\Bh}}
\newcommand{\icap}[0]{\hat{\Bi}}
\newcommand{\jcap}[0]{\hat{\Bj}}
\newcommand{\kcap}[0]{\hat{\Bk}}
\newcommand{\lcap}[0]{\hat{\Bl}}
\newcommand{\mcap}[0]{\hat{\Bm}}
\newcommand{\ncap}[0]{\hat{\Bn}}
\newcommand{\ocap}[0]{\hat{\Bo}}
\newcommand{\pcap}[0]{\hat{\Bp}}
\newcommand{\qcap}[0]{\hat{\Bq}}
\newcommand{\rcap}[0]{\hat{\Br}}
\newcommand{\scap}[0]{\hat{\Bs}}
\newcommand{\tcap}[0]{\hat{\Bt}}
\newcommand{\ucap}[0]{\hat{\Bu}}
\newcommand{\vcap}[0]{\hat{\Bv}}
\newcommand{\wcap}[0]{\hat{\Bw}}
\newcommand{\xcap}[0]{\hat{\Bx}}
\newcommand{\ycap}[0]{\hat{\By}}
\newcommand{\zcap}[0]{\hat{\Bz}}
\newcommand{\thetacap}[0]{\hat{\Btheta}}

%
% to write R^n and C^n in a distinguishable fashion.  Perhaps change this
% to the double lined characters upon figuring out how to do so.
%
\newcommand{\C}[1]{$\mathbb{C}^{#1}$}
\newcommand{\R}[1]{$\mathbb{R}^{#1}$}

%
% various generally useful helpers
%

% derivative of #1 wrt. #2:
\newcommand{\D}[2] {\frac {d#2} {d#1}}

\newcommand{\inv}[1]{\frac{1}{#1}}
\newcommand{\cross}[0]{\times}

\newcommand{\abs}[1]{\lvert{#1}\rvert}
\newcommand{\norm}[1]{\lVert{#1}\rVert}
\newcommand{\innerprod}[2]{\langle{#1}, {#2}\rangle}
\newcommand{\dotprod}[2]{{#1} \cdot {#2}}
\newcommand{\bdotprod}[2]{\left({#1} \cdot {#2}\right)}
\newcommand{\crossprod}[2]{{#1} \cross {#2}}
\newcommand{\tripleprod}[3]{\dotprod{\left(\crossprod{#1}{#2}\right)}{#3}}

\DeclareMathOperator{\Proj}{Proj}
\DeclareMathOperator{\Span}{span}
\DeclareMathOperator{\Sgn}{sgn}
\DeclareMathOperator{\Area}{Area}
\DeclareMathOperator{\Volume}{Volume}

%
% A few miscellaneous things specific to this document
%
\newcommand{\crossop}[1]{\crossprod{#1}{}}

% R2 vector.
\newcommand{\VectorTwo}[2]{
\begin{bmatrix}
 {#1} \\
 {#2}
\end{bmatrix}
}

\newcommand{\VectorN}[1]{
\begin{bmatrix}
{#1}_1 \\
{#1}_2 \\
\vdots \\
{#1}_N \\
\end{bmatrix}
}

\newcommand{\DETuvij}[4]{
\begin{vmatrix}
 {#1}_{#3} & {#1}_{#4} \\
 {#2}_{#3} & {#2}_{#4}
\end{vmatrix}
}

\newcommand{\DETuvwijk}[6]{
\begin{vmatrix}
 {#1}_{#4} & {#1}_{#5} & {#1}_{#6} \\
 {#2}_{#4} & {#2}_{#5} & {#2}_{#6} \\
 {#3}_{#4} & {#3}_{#5} & {#3}_{#6}
\end{vmatrix}
}

\newcommand{\DETuvwxijkl}[8]{
\begin{vmatrix}
 {#1}_{#5} & {#1}_{#6} & {#1}_{#7} & {#1}_{#8} \\
 {#2}_{#5} & {#2}_{#6} & {#2}_{#7} & {#2}_{#8} \\
 {#3}_{#5} & {#3}_{#6} & {#3}_{#7} & {#3}_{#8} \\
 {#4}_{#5} & {#4}_{#6} & {#4}_{#7} & {#4}_{#8} \\
\end{vmatrix}
}

%\newcommand{\DETuvwxyijklm}[10]{
%\begin{vmatrix}
% {#1}_{#6} & {#1}_{#7} & {#1}_{#8} & {#1}_{#9} & {#1}_{#10} \\
% {#2}_{#6} & {#2}_{#7} & {#2}_{#8} & {#2}_{#9} & {#2}_{#10} \\
% {#3}_{#6} & {#3}_{#7} & {#3}_{#8} & {#3}_{#9} & {#3}_{#10} \\
% {#4}_{#6} & {#4}_{#7} & {#4}_{#8} & {#4}_{#9} & {#4}_{#10} \\
% {#5}_{#6} & {#5}_{#7} & {#5}_{#8} & {#5}_{#9} & {#5}_{#10}
%\end{vmatrix}
%}

% R3 vector.
\newcommand{\VectorThree}[3]{
\begin{bmatrix}
 {#1} \\
 {#2} \\
 {#3}
\end{bmatrix}
}



%\newcommand{\Dslash}[0]{{\not{}}D}
\newcommand{\Dslash}[0]{D\!\!\!/}
\newcommand{\chapcite}[1]{\ref{chap:#1}}

%-----------------------------------------
%
% stubs for article class.
%
\newcommand{\blogpage}[1]{}
\newcommand{\email}[1]{}
\newcommand{\beginArtWithToc}[0]{}
\newcommand{\beginArtNoToc}[0]{}
\newcommand{\EndArticle}[0]{}
\newcommand{\EndNoBibArticle}[0]{}
\newcommand{\revisionInfo}[1]{}
%-----------------------------------------
\DeclareMathOperator{\Atan}{atan}

%%
% Copyright � 2012 Peeter Joot.  All Rights Reserved.
% Licenced as described in the file LICENSE under the root directory of this GIT repository.
%

% 
% 
\DeclareMathOperator{\Div}{div}
\DeclareMathOperator{\Mod}{mod}
\DeclareMathOperator{\PV}{PV}
\DeclareMathOperator{\Prob}{Prob}
\DeclareMathOperator{\rank}{rank}
\DeclareMathOperator{\sgn}{sgn}
\DeclareMathOperator{\sinc}{sinc}
%\DeclareMathOperator{\Atan2}{atan2}
\DeclareMathOperator{\atan}{atan}


\newcommand{\expectation}[1]{\langle{#1}\rangle}
%\newcommand{\gpgradefour}[1] {\gpgrade{#1}{4}}
%\newcommand{\gpgradeone}[1] {\gpgrade{#1}{1}}
%\newcommand{\gpgradethree}[1] {\gpgrade{#1}{3}}
%\newcommand{\gpgradetwo}[1] {\gpgrade{#1}{2}}
%\newcommand{\gpgradezero}[1] {\gpgrade{#1}{}}
%\newcommand{\gpgrade}[2] {{\left\langle{{#1}}\right\rangle}_{#2}}
%\newcommand{\grad}[0]{\boldsymbol{\nabla}}
%\newcommand{\grad}[0]{\nabla}


\newcommand{\ketbra}[2]{\ket{#1}\bra{#2}}
\newcommand{\ket}[1]{\lvert {#1} \rangle}
%\newcommand{\norm}[1]{\lVert#1\rVert}
\newcommand{\questionEquals}[0]{\stackrel{?}{=}}
\newcommand{\rightshift}[0]{\gg}
%\newcommand{\spacegrad}[0]{\boldsymbol{\nabla}}
\newcommand{\symmetric}[2]{{\left\{{#1},{#2}\right\}}}
\newcommand{\antisymmetric}[2]{\left[{#1},{#2}\right]}

%\newcommand{\Abs}[1]{\left\lvert{#1}\right\rvert}

%\newcommand{\BB}[0]{\mathbf{B}}
%\newcommand{\BE}[0]{\mathbf{E}}
%\newcommand{\BF}[0]{\mathbf{F}}
%\newcommand{\BS}[0]{\mathbf{S}}
%\newcommand{\BV}[0]{\mathbf{V}}
%\newcommand{\Bj}[0]{\mathbf{j}}

\newcommand{\BraOpKet}[3]{\bra{#1} \hat{#2} \ket{#3} }
%\newcommand{\Brho}[0]{\boldsymbol{\rho}}
\newcommand{\CC}[0]{c^2}
\newcommand{\Cos}[1]{\cos{\left({#1}\right)}}

% not working anymore.  think it's a conflicting macro for \not.
% compared to original usage in klien_gordon.ltx
%
%\newcommand{\Dslash}[0]{{\not}D}
%\newcommand{\Dslash}[0]{{\not{}}D}
% switched to cancel in macros.tex
%\newcommand{\Dslash}[0]{D\!\!\!/}

\newcommand{\Expectation}[1]{\left\langle {#1} \right\rangle}
\newcommand{\Exp}[1]{\exp{\left({#1}\right)}}
\newcommand{\FF}[0]{\mathcal{F}}
\newcommand{\FM}[0]{\inv{\sqrt{2\pi\hbar}}}
\newcommand{\IIinf}[0]{ \int_{-\infty}^\infty }
\newcommand{\Innerprod}[2]{\left\langle{#1}, {#2}\right\rangle}
%\newcommand{\LL}[0]{\mathcal{L}}

%\newcommand{\PD}[2] {\frac {\partial #2} {\partial #1}}

% backwards from ../peeterj_macros2:
\newcommand{\PDb}[2]{ \frac{\partial{#1}}{\partial {#2}} }

%\newcommand{\PDD}[3]{\frac{\partial^2 {#3}}{\partial {#1}\partial {#2}}}
\newcommand{\PDN}[3]{\frac{\partial^{#3} {#2}}{\partial {#1}^{#3}}}

\newcommand{\PDSq}[2]{\frac{\partial^2 {#2}}{\partial {#1}^2}}
\newcommand{\PDsQ}[2]{\frac{\partial^2 {#2}}{\partial^2 {#1}}}

\newcommand{\Sch}[0]{{Schr\"{o}dinger} }
\newcommand{\Sin}[1]{\sin{\left({#1}\right)}}
\newcommand{\Sw}[0]{\mathcal{S}}
%\newcommand{\T}[0]{\text{T}}
\newcommand{\T}[0]{{\text{T}}}

\newcommand{\braket}[2]{\langle{#1} \vert {#2}\rangle}
\newcommand{\bra}[1]{\langle {#1} \rvert}
\newcommand{\curl}[0]{\grad \times}
\newcommand{\delambert}[0]{\sum_{\alpha = 1}^4{\PDSq{x_\alpha}{}}}
\newcommand{\delsquared}[0]{\nabla^2}
\newcommand{\diverg}[0]{\grad \cdot}

\newcommand{\halfPhi}[0]{\frac{\phi}{2}}
\newcommand{\hatH}[0]{\hat{H}}
\newcommand{\hatS}[0]{\hat{S}}
\newcommand{\hatk}[0]{\hat{k}}
\newcommand{\hatp}[0]{\hat{p}}
\newcommand{\hatx}[0]{\hat{x}}


\newcommand{\Rdot}[0]{\dot{R}}
%\newcommand{\addot}[0]{\ddot{a}}
%\newcommand{\adot}[0]{\dot{a}}
%\newcommand{\fddot}[0]{\ddot{f}}
%\newcommand{\fdot}[0]{\dot{f}}
%\newcommand{\bddot}[0]{\ddot{b}}
%\newcommand{\bdot}[0]{\dot{b}}
\newcommand{\ddotOmega}[0]{\ddot{\Omega}}
\newcommand{\ddotalpha}[0]{\ddot{\alpha}}
\newcommand{\ddotomega}[0]{\ddot{\omega}}
\newcommand{\ddotphi}[0]{\ddot{\phi}}
\newcommand{\ddotpsi}[0]{\ddot{\psi}}
\newcommand{\ddottheta}[0]{\ddot{\theta}}
\newcommand{\dotOmega}[0]{\dot{\Omega}}
\newcommand{\dotalpha}[0]{\dot{\alpha}}
\newcommand{\dotomega}[0]{\dot{\omega}}
\newcommand{\dotphi}[0]{\dot{\phi}}
\newcommand{\dotpsi}[0]{\dot{\psi}}
\newcommand{\dottheta}[0]{\dot{\theta}}
%\newcommand{\pddot}[0]{\ddot{p}}
%\newcommand{\pdot}[0]{\dot{p}}
%\newcommand{\qddot}[0]{\ddot{q}}
%\newcommand{\qdot}[0]{\dot{q}}
%\newcommand{\rddot}[0]{\ddot{r}}
%\newcommand{\rdot}[0]{\dot{r}}
%\newcommand{\tddot}[0]{\ddot{t}}
%\newcommand{\tdot}[0]{\dot{t}}
%\newcommand{\uddot}[0]{\ddot{u}}
%\newcommand{\udot}[0]{\dot{u}}
%\newcommand{\xddot}[0]{\ddot{x}}
%\newcommand{\xdot}[0]{\dot{x}}
%\newcommand{\yddot}[0]{\ddot{y}}
%\newcommand{\ydot}[0]{\dot{y}}
%\newcommand{\zddot}[0]{\ddot{z}}
%\newcommand{\zdot}[0]{\dot{z}}








%-------------------------------------------------------------------
% ORIGINS:
%
% bohm11.tex

%\DeclareMathOperator{\sgn}{sgn}
%\newcommand{\PDSq}[2]{\frac{\partial^2 {#2}}{\partial {#1}^2}}
%\newcommand{\PDN}[3]{\frac{\partial^{#3} {#2}}{\partial {#1}^{#3}}}
%\DeclareMathOperator{\sinc}{sinc}
%\DeclareMathOperator{\PV}{PV}
%\newcommand{\FF}[0]{\mathcal{F}}
%\newcommand{\Sw}[0]{\mathcal{S}}
%\newcommand{\IIinf}[0]{ \int_{-\infty}^\infty }
%\newcommand{\FM}[0]{\inv{\sqrt{2\pi\hbar}}}
%\newcommand{\expectation}[1]{\langle{#1}\rangle}
%
%

% bohm_ch10.tex

%\DeclareMathOperator{\sgn}{sgn}
%\newcommand{\expectation}[1]{\langle{#1}\rangle}
%\newcommand{\IIinf}[0]{ \int_{-\infty}^\infty }
%\DeclareMathOperator{\PV}{PV}
%
%

% bohm_ch9.tex

%\newcommand{\PDSq}[2]{\frac{\partial^2 {#2}}{\partial {#1}^2}}
%\newcommand{\PDN}[3]{\frac{\partial^{#3} {#2}}{\partial {#1}^{#3}}}
%\DeclareMathOperator{\sinc}{sinc}
%\DeclareMathOperator{\PV}{PV}
%\newcommand{\FF}[0]{\mathcal{F}}
%\newcommand{\Sw}[0]{\mathcal{S}}
%\newcommand{\IIinf}[0]{ \int_{-\infty}^\infty }
%\newcommand{\FM}[0]{\inv{\sqrt{2\pi\hbar}}}
%\newcommand{\expectation}[1]{\langle{#1}\rangle}
%
%

% commutator_herm.tex

%\newcommand{\symmetric}[2]{{\left\{{#1},{#2}\right\}}}
%\newcommand{\antisymmetric}[2]{\left[{#1},{#2}\right]}
%
%%\newcommand{\ket}[1]{\lvert {#1} \rangle}
%%\newcommand{\bra}[1]{\langle {#1} \rvert}
%%\newcommand{\braket}[2]{\langle{#1} \vert {#2}\rangle}
%%\newcommand{\ketbra}[2]{\ket{#1}\bra{#2}}
%%\newcommand{\BraOpKet}[3]{\bra{#1} \hat{#2} \ket{#3} }
%%\newcommand{\Innerprod}[2]{\left\langle{#1}, {#2}\right\rangle}
%\newcommand{\Expectation}[1]{\left\langle {#1} \right\rangle}
%
%

% delta_ortho_series.tex

%\newcommand{\IIinf}[0]{ \int_{-\infty}^\infty }
%\newcommand{\ket}[1]{\lvert {#1} \rangle}
%\newcommand{\bra}[1]{\langle {#1} \rvert}
%\newcommand{\braket}[2]{\langle{#1} \vert {#2}\rangle}
%\newcommand{\ketbra}[2]{\ket{#1}\bra{#2}}
%\newcommand{\BraOpKet}[3]{\bra{#1} \hat{#2} \ket{#3} }
%\newcommand{\Innerprod}[2]{\left\langle{#1}, {#2}\right\rangle}
%
%

% distributions.tex

%\newcommand{\PDSq}[2]{\frac{\partial^2 {#2}}{\partial {#1}^2}}
%\DeclareMathOperator{\sinc}{sinc}
%\DeclareMathOperator{\PV}{PV}
%\newcommand{\FF}[0]{\mathcal{F}}
%\newcommand{\Sw}[0]{\mathcal{S}}
%\newcommand{\IIinf}[0]{ \int_{-\infty}^\infty }
%
%

% ehrenfest.tex

%\newcommand{\PDSq}[2]{\frac{\partial^2 {#2}}{\partial {#1}^2}}
%
%

% fletcher.tex

%\DeclareMathOperator{\Div}{div}
%\DeclareMathOperator{\Mod}{mod}
%\newcommand{\rightshift}[0]{\gg}
%\newcommand{\questionEquals}[0]{\stackrel{?}{=}}
%
%

% fvec.tex

%\newcommand{\grad}[0]{\nabla}
%\newcommand{\PD}[2]{ \frac{\partial{#1}}{\partial {#2}} }
%
%

% gacs_q8_8.tex

%\newcommand{\halfPhi}[0]{\frac{\phi}{2}}
%\newcommand{\Sin}[1]{\sin{\left({#1}\right)}}
%\newcommand{\Cos}[1]{\cos{\left({#1}\right)}}
%\newcommand{\Exp}[1]{\exp{\left({#1}\right)}}
%
%

% goldstein_ch1_2.tex

%\newcommand{\spacegrad}[0]{\boldsymbol{\nabla}}
%\newcommand{\Brho}[0]{\boldsymbol{\rho}}
%\newcommand{\LL}[0]{\mathcal{L}}
%\newcommand{\Abs}[1]{\left\lvert{#1}\right\rvert}
%\newcommand{\qdot}[0]{\dot{q}}
%\newcommand{\qddot}[0]{\ddot{q}}
%\newcommand{\xdot}[0]{\dot{x}}
%\newcommand{\xddot}[0]{\ddot{x}}
%\newcommand{\ydot}[0]{\dot{y}}
%\newcommand{\yddot}[0]{\ddot{y}}
%\newcommand{\dotalpha}[0]{\dot{\alpha}}
%\newcommand{\ddotalpha}[0]{\ddot{\alpha}}
%\newcommand{\dottheta}[0]{\dot{\theta}}
%\newcommand{\ddottheta}[0]{\ddot{\theta}}
%\newcommand{\dotphi}[0]{\dot{\phi}}
%\newcommand{\ddotphi}[0]{\ddot{\phi}}
%% == \partial_{#1} {#2}
%\newcommand{\PD}[2]{\frac{\partial {#2}}{\partial {#1}}}
%\newcommand{\PDD}[3]{\frac{\partial^2 {#3}}{\partial {#1}\partial {#2}}}
%
%% <grade selection>
%%
%\newcommand{\gpgrade}[2] {{\left\langle{{#1}}\right\rangle}_{#2}}
%
%\newcommand{\gpgradezero}[1] {\gpgrade{#1}{}}
%%\newcommand{\gpscalargrade}[1] {{\left\langle{{#1}}\right\rangle}}
%%\newcommand{\gpgradezero}[1] {\gpgrade{#1}{0}}
%
%%\newcommand{\gpgradeone}[1] {{\left\langle{{#1}}\right\rangle}_{1}}
%\newcommand{\gpgradeone}[1] {\gpgrade{#1}{1}}
%
%\newcommand{\gpgradetwo}[1] {\gpgrade{#1}{2}}
%\newcommand{\gpgradethree}[1] {\gpgrade{#1}{3}}
%\newcommand{\gpgradefour}[1] {\gpgrade{#1}{4}}
%%
%% </grade selection>
%
%
%

% harmonic_osc.tex

%\newcommand{\IIinf}[0]{ \int_{-\infty}^\infty }
%
%

% klein_gordon.tex

%\newcommand{\PDSq}[2]{\frac{\partial^2 {#2}}{\partial {#1}^2}}
%%\newcommand{\Dslash}[0]{D\!\!\!/}
%\newcommand{\Dslash}[0]{{\not}D}
%
%

% matrix_to_operator.tex

%\newcommand{\T}[0]{{\text{T}}}
%
%

% maxwell.tex

%\newcommand{\norm}[1]{\lVert#1\rVert}
%\newcommand{\grad}[0]{\boldsymbol{\nabla}}
%\newcommand{\curl}[0]{\grad \times}
%\newcommand{\diverg}[0]{\grad \cdot}
%\newcommand{\delsquared}[0]{\nabla^2}
%\newcommand{\delambert}[0]{\sum_{\alpha = 1}^4{\PDSq{x_\alpha}{}}}
%
%% partial derivative of #1 wrt. #2:
%\newcommand{\PD}[2] {\frac {\partial #2} {\partial #1}}
%% second partial derivative of #1 wrt. #2:
%\newcommand{\PDSq}[2] {\frac {\partial^2 #2} {\partial {#1}^2}}
%
%%
%% shorthand for bold symbols:
%%
%\newcommand{\Bj}[0]{\mathbf{j}}
%\newcommand{\BB}[0]{\mathbf{B}}
%\newcommand{\BE}[0]{\mathbf{E}}
%\newcommand{\BF}[0]{\mathbf{F}}
%\newcommand{\BS}[0]{\mathbf{S}}
%\newcommand{\BV}[0]{\mathbf{V}}
%
%

% mp_inverse_svd_rough_notes.tex

%\newcommand{\T}[0]{\text{T}}
%\DeclareMathOperator{\rank}{rank}
%
%

% outermorphism_det.tex

%\newcommand{\gpgrade}[2] {{\left\langle{{#1}}\right\rangle}_{#2}}
%\newcommand{\gpgradeone}[1] {\gpgrade{#1}{1}}
%\newcommand{\gpgradetwo}[1] {\gpgrade{#1}{2}}
%
%

% pauli_qm_relativity_intro.tex

%\newcommand{\Sch}[0]{{Schr\"{o}dinger} }
%
%

% pe.tex

%
%\newcommand{\grad}[0]{\nabla}

% qm_susskind.tex

%\newcommand{\ket}[1]{\lvert {#1} \rangle}
%\newcommand{\bra}[1]{\langle {#1} \rvert}
%\newcommand{\braket}[2]{\langle{#1} \vert {#2}\rangle}
%\newcommand{\ketbra}[2]{\ket{#1}\bra{#2}}
%\newcommand{\BraOpKet}[3]{\bra{#1} \hat{#2} \ket{#3} }
%\newcommand{\hatH}[0]{\hat{H}}
%\newcommand{\hatS}[0]{\hat{S}}
%\newcommand{\hatk}[0]{\hat{k}}
%\newcommand{\hatx}[0]{\hat{x}}
%\newcommand{\hatp}[0]{\hat{p}}
%\DeclareMathOperator{\Prob}{Prob}
%
%

% schwartzchild_metric.tex

%\newcommand{\grad}[0]{\nabla}
%\newcommand{\Abs}[1]{\left\lvert{#1}\right\rvert}
%\newcommand{\spacegrad}[0]{\boldsymbol{\nabla}}
%\newcommand{\LL}[0]{\mathcal{L}}
%\newcommand{\PD}[2]{\frac{\partial {#2}}{\partial {#1}}}
%\newcommand{\PDsQ}[2]{\frac{\partial^2 {#2}}{\partial^2 {#1}}}
%\newcommand{\dotalpha}[0]{\dot{\alpha}}
%\newcommand{\ddotalpha}[0]{\ddot{\alpha}}
%
%\newcommand{\dotomega}[0]{\dot{\omega}}
%\newcommand{\ddotomega}[0]{\ddot{\omega}}
%
%\newcommand{\dotOmega}[0]{\dot{\Omega}}
%\newcommand{\ddotOmega}[0]{\ddot{\Omega}}
%
%\newcommand{\CC}[0]{c^2}
%
%\newcommand{\dottheta}[0]{\dot{\theta}}
%\newcommand{\ddottheta}[0]{\ddot{\theta}}
%
%\newcommand{\dotpsi}[0]{\dot{\psi}}
%\newcommand{\ddotpsi}[0]{\ddot{\psi}}
%
%\newcommand{\adot}[0]{\dot{a}}
%\newcommand{\addot}[0]{\ddot{a}}
%\newcommand{\udot}[0]{\dot{u}}
%\newcommand{\uddot}[0]{\ddot{u}}
%\newcommand{\fdot}[0]{\dot{f}}
%\newcommand{\fddot}[0]{\ddot{f}}
%\newcommand{\bdot}[0]{\dot{b}}
%\newcommand{\bddot}[0]{\ddot{b}}
%\newcommand{\qdot}[0]{\dot{q}}
%\newcommand{\qddot}[0]{\ddot{q}}
%\newcommand{\tdot}[0]{\dot{t}}
%\newcommand{\tddot}[0]{\ddot{t}}
%
%\newcommand{\Rdot}[0]{\dot{R}}
%
%\newcommand{\pdot}[0]{\dot{p}}
%\newcommand{\pddot}[0]{\ddot{p}}
%
%\newcommand{\xdot}[0]{\dot{x}}
%\newcommand{\xddot}[0]{\ddot{x}}
%
%\newcommand{\zdot}[0]{\dot{z}}
%\newcommand{\zddot}[0]{\ddot{z}}
%
%\newcommand{\rdot}[0]{\dot{r}}
%\newcommand{\rddot}[0]{\ddot{r}}
%
%

% shear.tex

%\newcommand{\gpgrade}[2] {{\left\langle{{#1}}\right\rangle}_{#2}}
%
%

% tong_mf1.tex

%\newcommand{\Abs}[1]{\left\lvert{#1}\right\rvert}
%\newcommand{\grad}[0]{\nabla}
%\newcommand{\LL}[0]{\mathcal{L}}
%
%\newcommand{\dotalpha}[0]{\dot{\alpha}}
%\newcommand{\ddotalpha}[0]{\ddot{\alpha}}
%
%\newcommand{\dotomega}[0]{\dot{\omega}}
%\newcommand{\ddotomega}[0]{\ddot{\omega}}
%
%\newcommand{\dottheta}[0]{\dot{\theta}}
%\newcommand{\ddottheta}[0]{\ddot{\theta}}
%
%\newcommand{\dotpsi}[0]{\dot{\psi}}
%\newcommand{\ddotpsi}[0]{\ddot{\psi}}
%
%\newcommand{\qdot}[0]{\dot{q}}
%\newcommand{\qddot}[0]{\ddot{q}}
%
%\newcommand{\Rdot}[0]{\dot{R}}
%
%\newcommand{\pdot}[0]{\dot{p}}
%\newcommand{\pddot}[0]{\ddot{p}}
%
%\newcommand{\xdot}[0]{\dot{x}}
%\newcommand{\xddot}[0]{\ddot{x}}
%
%\newcommand{\zdot}[0]{\dot{z}}
%\newcommand{\zddot}[0]{\ddot{z}}
%
%\newcommand{\rdot}[0]{\dot{r}}
%\newcommand{\rddot}[0]{\ddot{r}}
%
%% == \partial_{#1} {#2}
%\newcommand{\PD}[2]{\frac{\partial {#2}}{\partial {#1}}}
%\newcommand{\PDD}[3]{\frac{\partial^2 {#3}}{\partial {#1}\partial {#2}}}
%
%

% wavepacket.tex

%\newcommand{\PDSq}[2]{\frac{\partial^2 {#2}}{\partial {#1}^2}}
%\newcommand{\IIinf}[0]{ \int_{-\infty}^\infty }
%
%

% wavevariation.tex

%\newcommand{\PDSq}[2]{\frac{\partial^2 {#2}}{\partial {#1}^2}}
%
%

% qm_barrier
%\DeclareMathOperator{\Atan2}{atan2}
%\DeclareMathOperator{\atan}{atan}

% twobodies.tex
%\DeclareMathOperator{\sgn}{sgn}


% sr_lagrangian_q.tex

%\newcommand{\PD}[2]{\frac{\partial {#2}}{\partial {#1}}}
%\newcommand{\xdot}[0]{\dot{x}}
%\newcommand{\xddot}[0]{\ddot{x}}

% stub_em_fields.tex

%\newcommand{\EE}[0]{\boldsymbol{\mathcal{E}}}
%\newcommand{\HH}[0]{\boldsymbol{\mathcal{H}}}
%\newcommand{\PDSq}[2]{\frac{\partial^2 {#2}}{\partial {#1}^2}}

% long_wire_q.tex

%\newcommand{\grad}[0]{\nabla}

% lorentz_tx_em_potential.tex
%\newcommand{\LL}[0]{\mathcal{L}}
%\newcommand{\grad}[0]{\nabla}
%\newcommand{\pdot}[0]{\dot{p}}
%\newcommand{\pddot}[0]{\ddot{p}}

%------------------------------------------------------
% cross_old.tex

%%
%% shorthand for bold symbols, convenient for vectors and matrices
%%
%\newcommand{\Ba}[0]{\mathbf{a}}
%\newcommand{\Bb}[0]{\mathbf{b}}
%\newcommand{\Bc}[0]{\mathbf{c}}
%\newcommand{\Bd}[0]{\mathbf{d}}
%\newcommand{\Be}[0]{\mathbf{e}}
%\newcommand{\Bf}[0]{\mathbf{f}}
%\newcommand{\Bg}[0]{\mathbf{g}}
%\newcommand{\Bh}[0]{\mathbf{h}}
%\newcommand{\Bi}[0]{\mathbf{i}}
%\newcommand{\Bj}[0]{\mathbf{j}}
%\newcommand{\Bk}[0]{\mathbf{k}}
%\newcommand{\Bl}[0]{\mathbf{l}}
%\newcommand{\Bm}[0]{\mathbf{m}}
%\newcommand{\Bn}[0]{\mathbf{n}}
%\newcommand{\Bo}[0]{\mathbf{o}}
%\newcommand{\Bp}[0]{\mathbf{p}}
%\newcommand{\Bq}[0]{\mathbf{q}}
%\newcommand{\Br}[0]{\mathbf{r}}
%\newcommand{\Bs}[0]{\mathbf{s}}
%\newcommand{\Bt}[0]{\mathbf{t}}
%\newcommand{\Bu}[0]{\mathbf{u}}
%\newcommand{\Bv}[0]{\mathbf{v}}
%\newcommand{\Bw}[0]{\mathbf{w}}
%\newcommand{\Bx}[0]{\mathbf{x}}
%\newcommand{\By}[0]{\mathbf{y}}
%\newcommand{\Bz}[0]{\mathbf{z}}
%\newcommand{\BA}[0]{\mathbf{A}}
%\newcommand{\BB}[0]{\mathbf{B}}
%\newcommand{\BC}[0]{\mathbf{C}}
%\newcommand{\BD}[0]{\mathbf{D}}
%\newcommand{\BE}[0]{\mathbf{E}}
%\newcommand{\BF}[0]{\mathbf{F}}
%\newcommand{\BG}[0]{\mathbf{G}}
%\newcommand{\BH}[0]{\mathbf{H}}
%\newcommand{\BI}[0]{\mathbf{I}}
%\newcommand{\BJ}[0]{\mathbf{J}}
%\newcommand{\BK}[0]{\mathbf{K}}
%\newcommand{\BL}[0]{\mathbf{L}}
%\newcommand{\BM}[0]{\mathbf{M}}
%\newcommand{\BN}[0]{\mathbf{N}}
%\newcommand{\BO}[0]{\mathbf{O}}
%\newcommand{\BP}[0]{\mathbf{P}}
%\newcommand{\BQ}[0]{\mathbf{Q}}
%\newcommand{\BR}[0]{\mathbf{R}}
%\newcommand{\BS}[0]{\mathbf{S}}
%\newcommand{\BT}[0]{\mathbf{T}}
%\newcommand{\BU}[0]{\mathbf{U}}
%\newcommand{\BV}[0]{\mathbf{V}}
%\newcommand{\BW}[0]{\mathbf{W}}
%\newcommand{\BX}[0]{\mathbf{X}}
%\newcommand{\BY}[0]{\mathbf{Y}}
%\newcommand{\BZ}[0]{\mathbf{Z}}
%
%\newcommand{\Bzero}[0]{\mathbf{0}}
%\newcommand{\Btheta}[0]{\boldsymbol{\theta}}
%\newcommand{\Btau}[0]{\boldsymbol{\tau}}
%\newcommand{\Bomega}[0]{\boldsymbol{\omega}}
%
%%
%% shorthand for unit vectors
%%
%\newcommand{\acap}[0]{\hat{\Ba}}
%\newcommand{\bcap}[0]{\hat{\Bb}}
%\newcommand{\ccap}[0]{\hat{\Bc}}
%\newcommand{\dcap}[0]{\hat{\Bd}}
%\newcommand{\ecap}[0]{\hat{\Be}}
%\newcommand{\fcap}[0]{\hat{\Bf}}
%\newcommand{\gcap}[0]{\hat{\Bg}}
%\newcommand{\hcap}[0]{\hat{\Bh}}
%\newcommand{\icap}[0]{\hat{\Bi}}
%\newcommand{\jCap}[0]{\hat{\Bj}}
%\newcommand{\kcap}[0]{\hat{\Bk}}
%\newcommand{\lcap}[0]{\hat{\Bl}}
%\newcommand{\mcap}[0]{\hat{\Bm}}
%\newcommand{\ncap}[0]{\hat{\Bn}}
%\newcommand{\ocap}[0]{\hat{\Bo}}
%\newcommand{\pcap}[0]{\hat{\Bp}}
%\newcommand{\qcap}[0]{\hat{\Bq}}
%\newcommand{\rcap}[0]{\hat{\Br}}
%\newcommand{\scap}[0]{\hat{\Bs}}
%\newcommand{\tcap}[0]{\hat{\Bt}}
%\newcommand{\ucap}[0]{\hat{\Bu}}
%\newcommand{\vcap}[0]{\hat{\Bv}}
%\newcommand{\wcap}[0]{\hat{\Bw}}
%\newcommand{\xcap}[0]{\hat{\Bx}}
%\newcommand{\ycap}[0]{\hat{\By}}
%\newcommand{\zcap}[0]{\hat{\Bz}}
%\newcommand{\thetacap}[0]{\hat{\Btheta}}
%
%%
%% to write R^n and C^n in a distinguishable fashion.  Perhaps change this
%% to the double lined characters upon figuring out how to do so.
%%
%\newcommand{\C}[1]{${\BC}^{#1}$}
%\newcommand{\R}[1]{${\BR}^{#1}$}
%
%%
%% various generally useful helpers
%%
%
%% derivative of #1 wrt. #2:
%\newcommand{\D}[2] {\frac {d#2} {d#1}}

%\newcommand{\inv}[1]{\frac{1}{#1}}
%\newcommand{\cross}[0]{\times}

%\newcommand{\abs}[1]{\lvert#1\rvert}
%\newcommand{\norm}[1]{\lVert#1\rVert}
%\newcommand{\innerprod}[2]{\langle{#1}, {#2}\rangle}
%\newcommand{\dotprod}[2]{#1 \cdot #2}
%\newcommand{\crossprod}[2]{#1 \cross #2}
%\newcommand{\tripleprod}[3]{\dotprod{\crossprod{#1}{#2}}{#3}}

%
% A few miscellaneous things specific to this document
%
%\newcommand{\crossop}[1]{\crossprod{#1}{}}

\newcommand{\PDP}[2]{\BP^{#1}\BD{\BP^{#2}}}
\newcommand{\PDPDP}[3]{\Bv^T\BP^{#1}\BD\BP^{#2}\BD\BP^{#3}\Bv}

\newcommand{\Mp}[0]{
\begin{bmatrix}
0 & 1 & 0 & 0 \\
0 & 0 & 1 & 0 \\
0 & 0 & 0 & 1 \\
1 & 0 & 0 & 0
\end{bmatrix}
}
\newcommand{\Mpp}[0]{
\begin{bmatrix}
0 & 0 & 1 & 0 \\
0 & 0 & 0 & 1 \\
1 & 0 & 0 & 0 \\
0 & 1 & 0 & 0
\end{bmatrix}
}
\newcommand{\Mppp}[0]{
\begin{bmatrix}
0 & 0 & 0 & 1 \\
1 & 0 & 0 & 0 \\
0 & 1 & 0 & 0 \\
0 & 0 & 1 & 0
\end{bmatrix}
}
\newcommand{\Mpu}[0]{
\begin{bmatrix}
u_1 & 0 & 0 & 0 \\
0 & u_2 & 0 & 0 \\
0 & 0 & u_3 & 0 \\
0 & 0 & 0 & u_4
\end{bmatrix}
}

%------------------------------------------------------



%\makeindex

%\date{ May 31, 2009.  $RCSfile: main.tex,v $ Last $Revision: 1.2 $ $Date: 2009/12/03 03:24:40 $ }
\date{ Last $Revision: 1.2 $ }
\begin{document}
\pagenumbering{alph}

\title{Misc Physics and Math Play.}
\author{Peeter Joot  \quad peeter.joot@gmail.com \\
{\small\em \copyright \  Draft date \today }}

\maketitle

\clearpage\pagenumbering{roman}
\tableofcontents

\clearpage\pagenumbering{arabic}

\pagestyle{plain}

%
% Copyright � 2015 Peeter Joot.  All Rights Reserved.
% Licenced as described in the file LICENSE under the root directory of this GIT repository.
%

% 
%\chapter{Preface}
% this suppresses an explicit chapter number for the preface.
\chapter*{Preface}%\normalsize
  \addcontentsline{toc}{chapter}{Preface}

This document was produced while taking the Spring 2016, University of Toronto Microwave Circuits course (ECE1236H), taught by Prof.\ G. V. Eleftheriades.

\paragraph{Course Syllabus}

This course outlines the principles of designing modern microwave and RF circuits.  Signal-integrity issues in high-speed digital circuits are also examined.

\begin{itemize}
\item The wave equation.
\item Ideal transmission lines.
\item Transients on transmission-lines.
\item Planar transmission lines and introduction to MMIC's.
\item Designing with scattering parameters.
\item Planar power dividers.
\item Directional couplers.
\item Microwave filters.
\item Solid-state microwave amplifiers.
\item Noise.
\item Diode-mixers.
\item RF receiver chains.
\item Oscillators.
\end{itemize}

\withproblemsetsMessage{
\textcolor{Maroon}{
\textit{THIS DOCUMENT IS REDACTED.  THE PROBLEM SET SOLUTIONS AND ASSOCIATED MATHEMATICA CODE IS NOT VISIBLE.  PLEASE EMAIL ME FOR THE FULL VERSION IF YOU ARE NOT TAKING ECE1236.}
}
}

\paragraph{This document contains:}

\begin{itemize}
\item Lecture notes.
\item Personal notes exploring auxiliary details.
\item Worked practice problems.

\ifthenelse{\boolean{redacted}}%
{%
\item Links to Mathematica notebooks associated with the course material and problems (but not problem sets).
}%
{
\item Assigned problems.%
\item Links to Mathematica notebooks associated with problems and course material.%
}
\end{itemize}

%This set of notes is significantly different from my notes for many other classes.  With the class taught on slides (and some of those slides mirroring the text closely), I did not take live notes in class.
%These notes fill in details that I felt deserved clarification, contain problem sets solutions, as well as a number of loosely related musings on Geometric Algebra equivalents to some of the generalized concepts of electromagnetic theory encountered in this class (i.e. magnetic sources).
%
My thanks go to Professor Eleftheriades for teaching this course.

Peeter Joot  \quad peeterjoot@protonmail.com 


%-------------------------------------------------------

\part{Algebra}
\documentclass{article}

\usepackage{amsmath}
\usepackage{mathpazo}

%
% shorthand for bold symbols, convenient for vectors and matrices
%
\newcommand{\Ba}[0]{\mathbf{a}}
\newcommand{\Bb}[0]{\mathbf{b}}
\newcommand{\Bc}[0]{\mathbf{c}}
\newcommand{\Bd}[0]{\mathbf{d}}
\newcommand{\Be}[0]{\mathbf{e}}
\newcommand{\Bf}[0]{\mathbf{f}}
\newcommand{\Bg}[0]{\mathbf{g}}
\newcommand{\Bh}[0]{\mathbf{h}}
\newcommand{\Bi}[0]{\mathbf{i}}
\newcommand{\Bj}[0]{\mathbf{j}}
\newcommand{\Bk}[0]{\mathbf{k}}
\newcommand{\Bl}[0]{\mathbf{l}}
\newcommand{\Bm}[0]{\mathbf{m}}
\newcommand{\Bn}[0]{\mathbf{n}}
\newcommand{\Bo}[0]{\mathbf{o}}
\newcommand{\Bp}[0]{\mathbf{p}}
\newcommand{\Bq}[0]{\mathbf{q}}
\newcommand{\Br}[0]{\mathbf{r}}
\newcommand{\Bs}[0]{\mathbf{s}}
\newcommand{\Bt}[0]{\mathbf{t}}
\newcommand{\Bu}[0]{\mathbf{u}}
\newcommand{\Bv}[0]{\mathbf{v}}
\newcommand{\Bw}[0]{\mathbf{w}}
\newcommand{\Bx}[0]{\mathbf{x}}
\newcommand{\By}[0]{\mathbf{y}}
\newcommand{\Bz}[0]{\mathbf{z}}
\newcommand{\BA}[0]{\mathbf{A}}
\newcommand{\BB}[0]{\mathbf{B}}
\newcommand{\BC}[0]{\mathbf{C}}
\newcommand{\BD}[0]{\mathbf{D}}
\newcommand{\BE}[0]{\mathbf{E}}
\newcommand{\BF}[0]{\mathbf{F}}
\newcommand{\BG}[0]{\mathbf{G}}
\newcommand{\BH}[0]{\mathbf{H}}
\newcommand{\BI}[0]{\mathbf{I}}
\newcommand{\BJ}[0]{\mathbf{J}}
\newcommand{\BK}[0]{\mathbf{K}}
\newcommand{\BL}[0]{\mathbf{L}}
\newcommand{\BM}[0]{\mathbf{M}}
\newcommand{\BN}[0]{\mathbf{N}}
\newcommand{\BO}[0]{\mathbf{O}}
\newcommand{\BP}[0]{\mathbf{P}}
\newcommand{\BQ}[0]{\mathbf{Q}}
\newcommand{\BR}[0]{\mathbf{R}}
\newcommand{\BS}[0]{\mathbf{S}}
\newcommand{\BT}[0]{\mathbf{T}}
\newcommand{\BU}[0]{\mathbf{U}}
\newcommand{\BV}[0]{\mathbf{V}}
\newcommand{\BW}[0]{\mathbf{W}}
\newcommand{\BX}[0]{\mathbf{X}}
\newcommand{\BY}[0]{\mathbf{Y}}
\newcommand{\BZ}[0]{\mathbf{Z}}

\newcommand{\Bzero}[0]{\mathbf{0}}
\newcommand{\Btheta}[0]{\boldsymbol{\theta}}
\newcommand{\Btau}[0]{\boldsymbol{\tau}}
\newcommand{\Bomega}[0]{\boldsymbol{\omega}}

%
% shorthand for unit vectors
%
\newcommand{\acap}[0]{\hat{\Ba}}
\newcommand{\bcap}[0]{\hat{\Bb}}
\newcommand{\ccap}[0]{\hat{\Bc}}
\newcommand{\dcap}[0]{\hat{\Bd}}
\newcommand{\ecap}[0]{\hat{\Be}}
\newcommand{\fcap}[0]{\hat{\Bf}}
\newcommand{\gcap}[0]{\hat{\Bg}}
\newcommand{\hcap}[0]{\hat{\Bh}}
\newcommand{\icap}[0]{\hat{\Bi}}
\newcommand{\jcap}[0]{\hat{\Bj}}
\newcommand{\kcap}[0]{\hat{\Bk}}
\newcommand{\lcap}[0]{\hat{\Bl}}
\newcommand{\mcap}[0]{\hat{\Bm}}
\newcommand{\ncap}[0]{\hat{\Bn}}
\newcommand{\ocap}[0]{\hat{\Bo}}
\newcommand{\pcap}[0]{\hat{\Bp}}
\newcommand{\qcap}[0]{\hat{\Bq}}
\newcommand{\rcap}[0]{\hat{\Br}}
\newcommand{\scap}[0]{\hat{\Bs}}
\newcommand{\tcap}[0]{\hat{\Bt}}
\newcommand{\ucap}[0]{\hat{\Bu}}
\newcommand{\vcap}[0]{\hat{\Bv}}
\newcommand{\wcap}[0]{\hat{\Bw}}
\newcommand{\xcap}[0]{\hat{\Bx}}
\newcommand{\ycap}[0]{\hat{\By}}
\newcommand{\zcap}[0]{\hat{\Bz}}
\newcommand{\thetacap}[0]{\hat{\Btheta}}

%
% to write R^n and C^n in a distinguishable fashion.  Perhaps change this
% to the double lined characters upon figuring out how to do so.
%
\newcommand{\C}[1]{$\mathbb{C}^{#1}$}
\newcommand{\R}[1]{$\mathbb{R}^{#1}$}

%
% various generally useful helpers
%

% derivative of #1 wrt. #2:
\newcommand{\D}[2] {\frac {d#2} {d#1}}

\newcommand{\inv}[1]{\frac{1}{#1}}
\newcommand{\cross}[0]{\times}

\newcommand{\abs}[1]{\lvert{#1}\rvert}
\newcommand{\norm}[1]{\lVert{#1}\rVert}
\newcommand{\innerprod}[2]{\langle{#1}, {#2}\rangle}
\newcommand{\dotprod}[2]{{#1} \cdot {#2}}
\newcommand{\bdotprod}[2]{\left({#1} \cdot {#2}\right)}
\newcommand{\crossprod}[2]{{#1} \cross {#2}}
\newcommand{\tripleprod}[3]{\dotprod{\left(\crossprod{#1}{#2}\right)}{#3}}

\DeclareMathOperator{\Proj}{Proj}
\DeclareMathOperator{\Span}{span}
\DeclareMathOperator{\Sgn}{sgn}
\DeclareMathOperator{\Area}{Area}
\DeclareMathOperator{\Volume}{Volume}

%
% A few miscellaneous things specific to this document
%
\newcommand{\crossop}[1]{\crossprod{#1}{}}

% R2 vector.
\newcommand{\VectorTwo}[2]{
\begin{bmatrix}
 {#1} \\
 {#2}
\end{bmatrix}
}

\newcommand{\VectorN}[1]{
\begin{bmatrix}
{#1}_1 \\
{#1}_2 \\
\vdots \\
{#1}_N \\
\end{bmatrix}
}

\newcommand{\DETuvij}[4]{
\begin{vmatrix}
 {#1}_{#3} & {#1}_{#4} \\
 {#2}_{#3} & {#2}_{#4}
\end{vmatrix}
}

\newcommand{\DETuvwijk}[6]{
\begin{vmatrix}
 {#1}_{#4} & {#1}_{#5} & {#1}_{#6} \\
 {#2}_{#4} & {#2}_{#5} & {#2}_{#6} \\
 {#3}_{#4} & {#3}_{#5} & {#3}_{#6}
\end{vmatrix}
}

\newcommand{\DETuvwxijkl}[8]{
\begin{vmatrix}
 {#1}_{#5} & {#1}_{#6} & {#1}_{#7} & {#1}_{#8} \\
 {#2}_{#5} & {#2}_{#6} & {#2}_{#7} & {#2}_{#8} \\
 {#3}_{#5} & {#3}_{#6} & {#3}_{#7} & {#3}_{#8} \\
 {#4}_{#5} & {#4}_{#6} & {#4}_{#7} & {#4}_{#8} \\
\end{vmatrix}
}

%\newcommand{\DETuvwxyijklm}[10]{
%\begin{vmatrix}
% {#1}_{#6} & {#1}_{#7} & {#1}_{#8} & {#1}_{#9} & {#1}_{#10} \\
% {#2}_{#6} & {#2}_{#7} & {#2}_{#8} & {#2}_{#9} & {#2}_{#10} \\
% {#3}_{#6} & {#3}_{#7} & {#3}_{#8} & {#3}_{#9} & {#3}_{#10} \\
% {#4}_{#6} & {#4}_{#7} & {#4}_{#8} & {#4}_{#9} & {#4}_{#10} \\
% {#5}_{#6} & {#5}_{#7} & {#5}_{#8} & {#5}_{#9} & {#5}_{#10}
%\end{vmatrix}
%}

% R3 vector.
\newcommand{\VectorThree}[3]{
\begin{bmatrix}
 {#1} \\
 {#2} \\
 {#3}
\end{bmatrix}
}


%<misc>
%
\newcommand{\Abs}[1]{{\left\lvert{#1}\right\rvert}}
\newcommand{\spacegrad}[0]{\boldsymbol{\nabla}}
\newcommand{\grad}[0]{\nabla}
\newcommand{\LL}[0]{\mathcal{L}}

% == \partial_{#1} {#2}
\newcommand{\PD}[2]{\frac{\partial {#2}}{\partial {#1}}}
% inline variant
\newcommand{\PDi}[2]{{\partial {#2}}/{\partial {#1}}}

\newcommand{\PDD}[3]{\frac{\partial^2 {#3}}{\partial {#1}\partial {#2}}}
%\newcommand{\PDd}[2]{\frac{\partial^2 {#2}}{{\partial{#1}}^2}}
\newcommand{\PDsq}[2]{\frac{\partial^2 {#2}}{(\partial {#1})^2}}

\newcommand{\Partial}[2]{\frac{\partial {#1}}{\partial {#2}}}
\DeclareMathOperator{\RejName}{Rej}
\newcommand{\Rej}[2]{\RejName_{#1}\left( {#2} \right)}
\newcommand{\Rm}[1]{\mathbb{R}^{#1}}
\newcommand{\Cm}[1]{\mathbb{C}^{#1}}
\newcommand{\conj}[0]{{*}}

%</misc>

% <grade selection>
%
\newcommand{\gpgrade}[2] {{\left\langle{{#1}}\right\rangle}_{#2}}

\newcommand{\gpgradezero}[1] {\gpgrade{#1}{}}
%\newcommand{\gpscalargrade}[1] {{\left\langle{{#1}}\right\rangle}}
%\newcommand{\gpgradezero}[1] {\gpgrade{#1}{0}}

%\newcommand{\gpgradeone}[1] {{\left\langle{{#1}}\right\rangle}_{1}}
\newcommand{\gpgradeone}[1] {\gpgrade{#1}{1}}

\newcommand{\gpgradetwo}[1] {\gpgrade{#1}{2}}
\newcommand{\gpgradethree}[1] {\gpgrade{#1}{3}}
\newcommand{\gpgradefour}[1] {\gpgrade{#1}{4}}
%
% </grade selection>



\newcommand{\adot}[0]{{\dot{a}}}
\newcommand{\bdot}[0]{{\dot{b}}}
% taken for centered dot:
%\newcommand{\cdot}[0]{{\dot{c}}}
%\newcommand{\ddot}[0]{{\dot{d}}}
\newcommand{\edot}[0]{{\dot{e}}}
\newcommand{\fdot}[0]{{\dot{f}}}
\newcommand{\gdot}[0]{{\dot{g}}}
\newcommand{\hdot}[0]{{\dot{h}}}
\newcommand{\idot}[0]{{\dot{i}}}
\newcommand{\jdot}[0]{{\dot{j}}}
\newcommand{\kdot}[0]{{\dot{k}}}
\newcommand{\ldot}[0]{{\dot{l}}}
\newcommand{\mdot}[0]{{\dot{m}}}
\newcommand{\ndot}[0]{{\dot{n}}}
%\newcommand{\odot}[0]{{\dot{o}}}
\newcommand{\pdot}[0]{{\dot{p}}}
\newcommand{\qdot}[0]{{\dot{q}}}
\newcommand{\rdot}[0]{{\dot{r}}}
\newcommand{\sdot}[0]{{\dot{s}}}
\newcommand{\tdot}[0]{{\dot{t}}}
\newcommand{\udot}[0]{{\dot{u}}}
\newcommand{\vdot}[0]{{\dot{v}}}
\newcommand{\wdot}[0]{{\dot{w}}}
\newcommand{\xdot}[0]{{\dot{x}}}
\newcommand{\ydot}[0]{{\dot{y}}}
\newcommand{\zdot}[0]{{\dot{z}}}
\newcommand{\addot}[0]{{\ddot{a}}}
\newcommand{\bddot}[0]{{\ddot{b}}}
\newcommand{\cddot}[0]{{\ddot{c}}}
%\newcommand{\dddot}[0]{{\ddot{d}}}
\newcommand{\eddot}[0]{{\ddot{e}}}
\newcommand{\fddot}[0]{{\ddot{f}}}
\newcommand{\gddot}[0]{{\ddot{g}}}
\newcommand{\hddot}[0]{{\ddot{h}}}
\newcommand{\iddot}[0]{{\ddot{i}}}
\newcommand{\jddot}[0]{{\ddot{j}}}
\newcommand{\kddot}[0]{{\ddot{k}}}
\newcommand{\lddot}[0]{{\ddot{l}}}
\newcommand{\mddot}[0]{{\ddot{m}}}
\newcommand{\nddot}[0]{{\ddot{n}}}
\newcommand{\oddot}[0]{{\ddot{o}}}
\newcommand{\pddot}[0]{{\ddot{p}}}
\newcommand{\qddot}[0]{{\ddot{q}}}
\newcommand{\rddot}[0]{{\ddot{r}}}
\newcommand{\sddot}[0]{{\ddot{s}}}
\newcommand{\tddot}[0]{{\ddot{t}}}
\newcommand{\uddot}[0]{{\ddot{u}}}
\newcommand{\vddot}[0]{{\ddot{v}}}
\newcommand{\wddot}[0]{{\ddot{w}}}
\newcommand{\xddot}[0]{{\ddot{x}}}
\newcommand{\yddot}[0]{{\ddot{y}}}
\newcommand{\zddot}[0]{{\ddot{z}}}

%<bold and dot greek symbols>
%

\newcommand{\Deltadot}[0]{{\dot{\Delta}}}
\newcommand{\Gammadot}[0]{{\dot{\Gamma}}}
\newcommand{\Lambdadot}[0]{{\dot{\Lambda}}}
\newcommand{\Omegadot}[0]{{\dot{\Omega}}}
\newcommand{\Phidot}[0]{{\dot{\Phi}}}
\newcommand{\Pidot}[0]{{\dot{\Pi}}}
\newcommand{\Psidot}[0]{{\dot{\Psi}}}
\newcommand{\Sigmadot}[0]{{\dot{\Sigma}}}
\newcommand{\Thetadot}[0]{{\dot{\Theta}}}
\newcommand{\Upsilondot}[0]{{\dot{\Upsilon}}}
\newcommand{\Xidot}[0]{{\dot{\Xi}}}
\newcommand{\alphadot}[0]{{\dot{\alpha}}}
\newcommand{\betadot}[0]{{\dot{\beta}}}
\newcommand{\chidot}[0]{{\dot{\chi}}}
\newcommand{\deltadot}[0]{{\dot{\delta}}}
\newcommand{\epsilondot}[0]{{\dot{\epsilon}}}
\newcommand{\etadot}[0]{{\dot{\eta}}}
\newcommand{\gammadot}[0]{{\dot{\gamma}}}
\newcommand{\kappadot}[0]{{\dot{\kappa}}}
\newcommand{\lambdadot}[0]{{\dot{\lambda}}}
\newcommand{\mudot}[0]{{\dot{\mu}}}
\newcommand{\nudot}[0]{{\dot{\nu}}}
\newcommand{\omegadot}[0]{{\dot{\omega}}}
\newcommand{\phidot}[0]{{\dot{\phi}}}
\newcommand{\pidot}[0]{{\dot{\pi}}}
\newcommand{\psidot}[0]{{\dot{\psi}}}
\newcommand{\rhodot}[0]{{\dot{\rho}}}
\newcommand{\sigmadot}[0]{{\dot{\sigma}}}
\newcommand{\taudot}[0]{{\dot{\tau}}}
\newcommand{\thetadot}[0]{{\dot{\theta}}}
\newcommand{\upsilondot}[0]{{\dot{\upsilon}}}
\newcommand{\varepsilondot}[0]{{\dot{\varepsilon}}}
\newcommand{\varphidot}[0]{{\dot{\varphi}}}
\newcommand{\varpidot}[0]{{\dot{\varpi}}}
\newcommand{\varrhodot}[0]{{\dot{\varrho}}}
\newcommand{\varsigmadot}[0]{{\dot{\varsigma}}}
\newcommand{\varthetadot}[0]{{\dot{\vartheta}}}
\newcommand{\xidot}[0]{{\dot{\xi}}}
\newcommand{\zetadot}[0]{{\dot{\zeta}}}

\newcommand{\Deltaddot}[0]{{\ddot{\Delta}}}
\newcommand{\Gammaddot}[0]{{\ddot{\Gamma}}}
\newcommand{\Lambdaddot}[0]{{\ddot{\Lambda}}}
\newcommand{\Omegaddot}[0]{{\ddot{\Omega}}}
\newcommand{\Phiddot}[0]{{\ddot{\Phi}}}
\newcommand{\Piddot}[0]{{\ddot{\Pi}}}
\newcommand{\Psiddot}[0]{{\ddot{\Psi}}}
\newcommand{\Sigmaddot}[0]{{\ddot{\Sigma}}}
\newcommand{\Thetaddot}[0]{{\ddot{\Theta}}}
\newcommand{\Upsilonddot}[0]{{\ddot{\Upsilon}}}
\newcommand{\Xiddot}[0]{{\ddot{\Xi}}}
\newcommand{\alphaddot}[0]{{\ddot{\alpha}}}
\newcommand{\betaddot}[0]{{\ddot{\beta}}}
\newcommand{\chiddot}[0]{{\ddot{\chi}}}
\newcommand{\deltaddot}[0]{{\ddot{\delta}}}
\newcommand{\epsilonddot}[0]{{\ddot{\epsilon}}}
\newcommand{\etaddot}[0]{{\ddot{\eta}}}
\newcommand{\gammaddot}[0]{{\ddot{\gamma}}}
\newcommand{\kappaddot}[0]{{\ddot{\kappa}}}
\newcommand{\lambdaddot}[0]{{\ddot{\lambda}}}
\newcommand{\muddot}[0]{{\ddot{\mu}}}
\newcommand{\nuddot}[0]{{\ddot{\nu}}}
\newcommand{\omegaddot}[0]{{\ddot{\omega}}}
\newcommand{\phiddot}[0]{{\ddot{\phi}}}
\newcommand{\piddot}[0]{{\ddot{\pi}}}
\newcommand{\psiddot}[0]{{\ddot{\psi}}}
\newcommand{\rhoddot}[0]{{\ddot{\rho}}}
\newcommand{\sigmaddot}[0]{{\ddot{\sigma}}}
\newcommand{\tauddot}[0]{{\ddot{\tau}}}
\newcommand{\thetaddot}[0]{{\ddot{\theta}}}
\newcommand{\upsilonddot}[0]{{\ddot{\upsilon}}}
\newcommand{\varepsilonddot}[0]{{\ddot{\varepsilon}}}
\newcommand{\varphiddot}[0]{{\ddot{\varphi}}}
\newcommand{\varpiddot}[0]{{\ddot{\varpi}}}
\newcommand{\varrhoddot}[0]{{\ddot{\varrho}}}
\newcommand{\varsigmaddot}[0]{{\ddot{\varsigma}}}
\newcommand{\varthetaddot}[0]{{\ddot{\vartheta}}}
\newcommand{\xiddot}[0]{{\ddot{\xi}}}
\newcommand{\zetaddot}[0]{{\ddot{\zeta}}}

\newcommand{\BDelta}[0]{\boldsymbol{\Delta}}
\newcommand{\BGamma}[0]{\boldsymbol{\Gamma}}
\newcommand{\BLambda}[0]{\boldsymbol{\Lambda}}
\newcommand{\BOmega}[0]{\boldsymbol{\Omega}}
\newcommand{\BPhi}[0]{\boldsymbol{\Phi}}
\newcommand{\BPi}[0]{\boldsymbol{\Pi}}
\newcommand{\BPsi}[0]{\boldsymbol{\Psi}}
\newcommand{\BSigma}[0]{\boldsymbol{\Sigma}}
\newcommand{\BTheta}[0]{\boldsymbol{\Theta}}
\newcommand{\BUpsilon}[0]{\boldsymbol{\Upsilon}}
\newcommand{\BXi}[0]{\boldsymbol{\Xi}}
\newcommand{\Balpha}[0]{\boldsymbol{\alpha}}
\newcommand{\Bbeta}[0]{\boldsymbol{\beta}}
\newcommand{\Bchi}[0]{\boldsymbol{\chi}}
\newcommand{\Bdelta}[0]{\boldsymbol{\delta}}
\newcommand{\Bepsilon}[0]{\boldsymbol{\epsilon}}
\newcommand{\Beta}[0]{\boldsymbol{\eta}}
\newcommand{\Bgamma}[0]{\boldsymbol{\gamma}}
\newcommand{\Bkappa}[0]{\boldsymbol{\kappa}}
\newcommand{\Blambda}[0]{\boldsymbol{\lambda}}
\newcommand{\Bmu}[0]{\boldsymbol{\mu}}
\newcommand{\Bnu}[0]{\boldsymbol{\nu}}
%\newcommand{\Bomega}[0]{\boldsymbol{\omega}}
\newcommand{\Bphi}[0]{\boldsymbol{\phi}}
\newcommand{\Bpi}[0]{\boldsymbol{\pi}}
\newcommand{\Bpsi}[0]{\boldsymbol{\psi}}
\newcommand{\Brho}[0]{\boldsymbol{\rho}}
\newcommand{\Bsigma}[0]{\boldsymbol{\sigma}}
%\newcommand{\Btau}[0]{\boldsymbol{\tau}}
%\newcommand{\Btheta}[0]{\boldsymbol{\theta}}
\newcommand{\Bupsilon}[0]{\boldsymbol{\upsilon}}
\newcommand{\Bvarepsilon}[0]{\boldsymbol{\varepsilon}}
\newcommand{\Bvarphi}[0]{\boldsymbol{\varphi}}
\newcommand{\Bvarpi}[0]{\boldsymbol{\varpi}}
\newcommand{\Bvarrho}[0]{\boldsymbol{\varrho}}
\newcommand{\Bvarsigma}[0]{\boldsymbol{\varsigma}}
\newcommand{\Bvartheta}[0]{\boldsymbol{\vartheta}}
\newcommand{\Bxi}[0]{\boldsymbol{\xi}}
\newcommand{\Bzeta}[0]{\boldsymbol{\zeta}}
%
%</bold and dot greek symbols>
%<infrequent>
%
%\newcommand{\AreaOp}[1]{\AName_{#1}}
%\newcommand{\Babs}[0]{\abs{\BB}}
%\newcommand{\Bcap}[0]{\hat{\BB}}
%\newcommand{\BrPrimeRej}[0]{\rcap(\rcap \wedge \Br')}
%\newcommand{\CA}[0]{\mathcal{A}}
%\newcommand{\Cos}[1]{\cos{\left({#1}\right)}}
%\newcommand{\Det}[1] {\abs{#1}}
%\newcommand{\Dsq}[2] {\frac {\partial^2 {#1}} {\partial {#2}^2}}
%\newcommand{\Exp}[1]{\exp{\left({#1}\right)}}
%\newcommand{\Norm}[1]{\left\lVert{#1}\right\rVert}
%\newcommand{\Sin}[1]{\sin{\left({#1}\right)}}
%\newcommand{\T}[0]{\text{T}}
%\newcommand{\VolumeOp}[1]{\VName_{#1}}
%\newcommand{\agrad}[0]{\Ba \cdot \nabla}
%\newcommand{\alphacap}[0]{\hat{\boldsymbol{\alpha}}}
%\newcommand{\Fcap}[0]{\hat{\BF}}
%\newcommand{\bithree}[0]{{\Bi}_3}
%\newcommand{\bxa}[0]{\Bx\Ba}
%\newcommand{\coordvec}[2]{
%\newcommand{\costheta}[0]{\acap \cdot \xcap}
%\newcommand{\ddt}[1]{\ddot{#1}}
%\newcommand{\ddu}[1] {\frac {d{#1}} {du}}
%\newcommand{\dsqxj}[2] {\frac {\partial^2 {#1}} {\partial {x_{#2}}^2}}
%\newcommand{\dtheta}[1]{\frac{d {#1}}{d \theta}}
%\newcommand{\dt}[1]{\dot{#1}}
%\newcommand{\dt}[1]{\frac{d {#1}}{dt}}
%\newcommand{\dxj}[2] {\frac {\partial {#1}} {\partial {x_{#2}}}}
%\newcommand{\halfPhi}[0]{\frac{\phi}{2}}
%\newcommand{\half}[0]{\inv{2}}
%\newcommand{\inv}[1]{\frac{1}{#1}}
%\newcommand{\laplacian}[0]{\nabla^2}
%\newcommand{\matrixoftx}[3]{
%\newcommand{\nrrp}[0]{\norm{\rcap \wedge \Br'}}
%\newcommand{\oiint}{\bigcirc \hspace{-1.4em} \int \hspace{-.8em} \int}
%\newcommand{\transpose}[1]{{#1}^{\text{T}}}
%\newcommand{\transpose}[1]{{{#1}^{\TextTranspose}}}
%\newcommand{\transpose}[1]{{{#1}^{\text{T}}}}
%\newcommand{\barA}[0]{\bar{A}}
%\newcommand{\qbar}[0]{\bar{q}}
%\newcommand{\qdotbar}[0]{\dot{\bar{q}}}
%
%</infrequent>





%\usepackage{listings}
\usepackage{txfonts} % for ointctr... (also appears to make "prettier" \int and \sum's)
\usepackage[bookmarks=true]{hyperref}

\usepackage{color,cite,graphicx}
   % use colour in the document, put your citations as [1-4]
   % rather than [1,2,3,4] (it looks nicer, and the extended LaTeX2e
   % graphics package. 
\usepackage{latexsym,amssymb,epsf} % don't remember if these are
   % needed, but their inclusion can't do any damage


\title{ Integer binomial theorem induction, the easy dumb way. }
\author{Peeter Joot \quad peeter.joot@gmail.com }
\date{ March 26, 2009.  Last Revision: $Date: 2009/03/27 00:59:32 $ }

\begin{document}
\maketitle{}

%\tableofcontents

\section{ Motivation. }

While working a problem with an induction requirement similar to but more 
complicated than the binomial theorem, I steped back and thought I'd try
this as an easier first step.  Had some trouble doing it, until I tried it
explicitly with for the power of three case.  Ironically, working it
out for an explicit index takes the abstraction out of the problem, and
generalizing further really only requires a search and replace.

\section{ Do it. }

Want to prove

\begin{align}
(t + x)^k = \sum_{m=0}^k \binom{k}{m} t^m x^{k-m}
\end{align}

where

\begin{align}
\binom{k}{m} = \frac{k!}{(k-m)!m!}
\end{align}

In particular want to prove this for the $k+1$ case, given $k$.

\subsection{ step with k+1 = 3 }

Three isn't actually the best way to start since it is almost too trivial, but
one gets the idea easily by doing it.

\begin{align*}
(t + x)^3 
&= (t + x)(t + x)^2  \\
&= (t + x)\sum_{m=0}^2 \binom{2}{m} t^m x^{2-m} \\
&= 
\sum_{m=0}^2 \binom{2}{m} t^{m+1} x^{2-m} 
+ \sum_{m=0}^2 \binom{2}{m} t^{m} x^{2-m + 1} \\
&= 
\sum_{m=1}^{2 + 1} \binom{2}{m-1} t^{m} x^{2 - m + 1} 
+ \sum_{m=0}^2 \binom{2}{m} t^{m} x^{2-m + 1} \\
\end{align*}

Now pull the lowest and highest order terms out of the sums, and group the
remaining bits.

\begin{align*}
(t + x)^3 
&= 
 \binom{2}{0} t^{0} x^{2 - 0 + 1} 
+ \sum_{m=1}^{2} \left( \binom{2}{m} + \binom{2}{m-1} \right) t^{m} x^{2 - m + 1} 
+ \binom{2}{2 + 1 -1} t^{2 + 1} x^{2 - (2 + 1) + 1} 
\\
\end{align*}

Now, observe that in all the steps above everywhere if all places that $2$, a nice easy to think with and concrete number, we could have used some
abstract index.

\subsection{ step with k+1 }

A straight text search and replace on $2$ with $k$ gives

\begin{align*}
(t + x)^{k+1}
&= (t + x)(t + x)^k  \\
&= (t + x)\sum_{m=0}^k \binom{k}{m} t^m x^{k-m} \\
&= 
\sum_{m=0}^k \binom{k}{m} t^{m+1} x^{k-m} 
+ \sum_{m=0}^k \binom{k}{m} t^{m} x^{k-m + 1} \\
&= 
\sum_{m=1}^{k + 1} \binom{k}{m-1} t^{m} x^{k - m + 1} 
+ \sum_{m=0}^k \binom{k}{m} t^{m} x^{k-m + 1} \\
&= 
 \binom{k}{0} t^{0} x^{k - 0 + 1} 
+ \sum_{m=1}^{k} \left( \binom{k}{m} + \binom{k}{m-1} \right) t^{m} x^{k - m + 1} 
+ \binom{k}{k + 1 -1} t^{k + 1} x^{k - (k + 1) + 1} \\
\end{align*}

This is EXACTLY the same as above with $k=2$ but it sure looks more complicated with an an abstract index.

To finish off we need a couple observations, that 
$\binom{k}{0} = 1 = \binom{k+1}{0}$, and $\binom{k}{k} = 1 = \binom{k+1}{k+1}$.  This leaves us with

\begin{align*}
(t + x)^{k+1}
= \binom{k+1}{0} t^{0} x^{k + 1} 
+ \sum_{m=1}^{k} \left( \binom{k}{m} + \binom{k}{m-1} \right) t^{m} x^{k - m + 1} 
+ \binom{k + 1}{k + 1} t^{k + 1} x^{0} \\
\end{align*}

So if we can show
\begin{align*}
\binom{k}{m} + \binom{k}{m-1} = \binom{k+1}{m}
\end{align*}

then we would have

\begin{align*}
(t + x)^{k+1}
&= \sum_{m=0}^{k+1} \binom{k+1}{m} t^{m} x^{k + 1 - m} 
\end{align*}

which is what was desired.

\subsection{ That last little piece. }

To prove that last little piece, let's do it again the dumb way, and let a regular expression $s/2/k/g$ in vim do the hard work.

\begin{align*}
\binom{2}{m} + \binom{2}{m-1} 
&=
\frac{2!}{(2-m)!m!}
+ \frac{2!}{(2-m + 1)!(m-1)!} \\
&=
\frac{2!}{(2-m)!m(m-1)!}
+ \frac{2!}{(3-m)!(m-1)!} \\
&=
\frac{2!}{(2-m)!m(m-1)!}
+ \frac{2!}{(3-m)(2-m)!(m-1)!} \\
&=
\frac{2!}{(2-m)!(m-1)!} \left( \inv{m} + \inv{3-m}\right) \\
&=
\frac{2!}{(2-m)!(m-1)!} \frac{3 -m + m }{m(3-m)} \\
&=
\frac{3!}{(3-m)!(m)!} \\
\end{align*}

So, generalizing the easy way with $s/3/k+1/g$, we have 

\begin{align*}
\binom{k}{m} + \binom{k}{m-1} 
&=
\frac{k!}{(k-m)!m!}
+ \frac{k!}{(k-m + 1)!(m-1)!} \\
&=
\frac{k!}{(k-m)!m(m-1)!}
+ \frac{k!}{((k+1)-m)!(m-1)!} \\
&=
\frac{k!}{(k-m)!m(m-1)!}
+ \frac{k!}{((k+1)-m)(k-m)!(m-1)!} \\
&=
\frac{k!}{(k-m)!(m-1)!} \left( \inv{m} + \inv{(k+1)-m}\right) \\
&=
\frac{k!}{(k-m)!(m-1)!} \frac{(k+1) -m + m }{m((k+1)-m)} \\
&=
\frac{(k+1)!}{((k+1)-m)!(m)!} \\
\end{align*}

Every step is EXACTLY the same as with $k=2$, the only differences were straight text substitution.  That leaves us with

\begin{align}
\binom{k}{m} + \binom{k}{m-1} 
&=
\binom{k+1}{m}
\end{align}

That's all we needed to complete the proof.

I think this is a superior way to do inductive proofs.  Just do the absolute easiest case and do it with a number that is easy to think with.  Search and replace in an editor does all the bits that would make you look clever if you were to leave off the fact that you were really only doing the easy version!

%\bibliographystyle{plainnat}
%\bibliography{myrefs}

\end{document}

%\documentclass{article}

%\usepackage{amsmath}
\usepackage{mathpazo}

%
% shorthand for bold symbols, convenient for vectors and matrices
%
\newcommand{\Ba}[0]{\mathbf{a}}
\newcommand{\Bb}[0]{\mathbf{b}}
\newcommand{\Bc}[0]{\mathbf{c}}
\newcommand{\Bd}[0]{\mathbf{d}}
\newcommand{\Be}[0]{\mathbf{e}}
\newcommand{\Bf}[0]{\mathbf{f}}
\newcommand{\Bg}[0]{\mathbf{g}}
\newcommand{\Bh}[0]{\mathbf{h}}
\newcommand{\Bi}[0]{\mathbf{i}}
\newcommand{\Bj}[0]{\mathbf{j}}
\newcommand{\Bk}[0]{\mathbf{k}}
\newcommand{\Bl}[0]{\mathbf{l}}
\newcommand{\Bm}[0]{\mathbf{m}}
\newcommand{\Bn}[0]{\mathbf{n}}
\newcommand{\Bo}[0]{\mathbf{o}}
\newcommand{\Bp}[0]{\mathbf{p}}
\newcommand{\Bq}[0]{\mathbf{q}}
\newcommand{\Br}[0]{\mathbf{r}}
\newcommand{\Bs}[0]{\mathbf{s}}
\newcommand{\Bt}[0]{\mathbf{t}}
\newcommand{\Bu}[0]{\mathbf{u}}
\newcommand{\Bv}[0]{\mathbf{v}}
\newcommand{\Bw}[0]{\mathbf{w}}
\newcommand{\Bx}[0]{\mathbf{x}}
\newcommand{\By}[0]{\mathbf{y}}
\newcommand{\Bz}[0]{\mathbf{z}}
\newcommand{\BA}[0]{\mathbf{A}}
\newcommand{\BB}[0]{\mathbf{B}}
\newcommand{\BC}[0]{\mathbf{C}}
\newcommand{\BD}[0]{\mathbf{D}}
\newcommand{\BE}[0]{\mathbf{E}}
\newcommand{\BF}[0]{\mathbf{F}}
\newcommand{\BG}[0]{\mathbf{G}}
\newcommand{\BH}[0]{\mathbf{H}}
\newcommand{\BI}[0]{\mathbf{I}}
\newcommand{\BJ}[0]{\mathbf{J}}
\newcommand{\BK}[0]{\mathbf{K}}
\newcommand{\BL}[0]{\mathbf{L}}
\newcommand{\BM}[0]{\mathbf{M}}
\newcommand{\BN}[0]{\mathbf{N}}
\newcommand{\BO}[0]{\mathbf{O}}
\newcommand{\BP}[0]{\mathbf{P}}
\newcommand{\BQ}[0]{\mathbf{Q}}
\newcommand{\BR}[0]{\mathbf{R}}
\newcommand{\BS}[0]{\mathbf{S}}
\newcommand{\BT}[0]{\mathbf{T}}
\newcommand{\BU}[0]{\mathbf{U}}
\newcommand{\BV}[0]{\mathbf{V}}
\newcommand{\BW}[0]{\mathbf{W}}
\newcommand{\BX}[0]{\mathbf{X}}
\newcommand{\BY}[0]{\mathbf{Y}}
\newcommand{\BZ}[0]{\mathbf{Z}}

\newcommand{\Bzero}[0]{\mathbf{0}}
\newcommand{\Btheta}[0]{\boldsymbol{\theta}}
\newcommand{\Btau}[0]{\boldsymbol{\tau}}
\newcommand{\Bomega}[0]{\boldsymbol{\omega}}

%
% shorthand for unit vectors
%
\newcommand{\acap}[0]{\hat{\Ba}}
\newcommand{\bcap}[0]{\hat{\Bb}}
\newcommand{\ccap}[0]{\hat{\Bc}}
\newcommand{\dcap}[0]{\hat{\Bd}}
\newcommand{\ecap}[0]{\hat{\Be}}
\newcommand{\fcap}[0]{\hat{\Bf}}
\newcommand{\gcap}[0]{\hat{\Bg}}
\newcommand{\hcap}[0]{\hat{\Bh}}
\newcommand{\icap}[0]{\hat{\Bi}}
\newcommand{\jcap}[0]{\hat{\Bj}}
\newcommand{\kcap}[0]{\hat{\Bk}}
\newcommand{\lcap}[0]{\hat{\Bl}}
\newcommand{\mcap}[0]{\hat{\Bm}}
\newcommand{\ncap}[0]{\hat{\Bn}}
\newcommand{\ocap}[0]{\hat{\Bo}}
\newcommand{\pcap}[0]{\hat{\Bp}}
\newcommand{\qcap}[0]{\hat{\Bq}}
\newcommand{\rcap}[0]{\hat{\Br}}
\newcommand{\scap}[0]{\hat{\Bs}}
\newcommand{\tcap}[0]{\hat{\Bt}}
\newcommand{\ucap}[0]{\hat{\Bu}}
\newcommand{\vcap}[0]{\hat{\Bv}}
\newcommand{\wcap}[0]{\hat{\Bw}}
\newcommand{\xcap}[0]{\hat{\Bx}}
\newcommand{\ycap}[0]{\hat{\By}}
\newcommand{\zcap}[0]{\hat{\Bz}}
\newcommand{\thetacap}[0]{\hat{\Btheta}}

%
% to write R^n and C^n in a distinguishable fashion.  Perhaps change this
% to the double lined characters upon figuring out how to do so.
%
\newcommand{\C}[1]{$\mathbb{C}^{#1}$}
\newcommand{\R}[1]{$\mathbb{R}^{#1}$}

%
% various generally useful helpers
%

% derivative of #1 wrt. #2:
\newcommand{\D}[2] {\frac {d#2} {d#1}}

\newcommand{\inv}[1]{\frac{1}{#1}}
\newcommand{\cross}[0]{\times}

\newcommand{\abs}[1]{\lvert{#1}\rvert}
\newcommand{\norm}[1]{\lVert{#1}\rVert}
\newcommand{\innerprod}[2]{\langle{#1}, {#2}\rangle}
\newcommand{\dotprod}[2]{{#1} \cdot {#2}}
\newcommand{\bdotprod}[2]{\left({#1} \cdot {#2}\right)}
\newcommand{\crossprod}[2]{{#1} \cross {#2}}
\newcommand{\tripleprod}[3]{\dotprod{\left(\crossprod{#1}{#2}\right)}{#3}}

\DeclareMathOperator{\Proj}{Proj}
\DeclareMathOperator{\Span}{span}
\DeclareMathOperator{\Sgn}{sgn}
\DeclareMathOperator{\Area}{Area}
\DeclareMathOperator{\Volume}{Volume}

%
% A few miscellaneous things specific to this document
%
\newcommand{\crossop}[1]{\crossprod{#1}{}}

% R2 vector.
\newcommand{\VectorTwo}[2]{
\begin{bmatrix}
 {#1} \\
 {#2}
\end{bmatrix}
}

\newcommand{\VectorN}[1]{
\begin{bmatrix}
{#1}_1 \\
{#1}_2 \\
\vdots \\
{#1}_N \\
\end{bmatrix}
}

\newcommand{\DETuvij}[4]{
\begin{vmatrix}
 {#1}_{#3} & {#1}_{#4} \\
 {#2}_{#3} & {#2}_{#4}
\end{vmatrix}
}

\newcommand{\DETuvwijk}[6]{
\begin{vmatrix}
 {#1}_{#4} & {#1}_{#5} & {#1}_{#6} \\
 {#2}_{#4} & {#2}_{#5} & {#2}_{#6} \\
 {#3}_{#4} & {#3}_{#5} & {#3}_{#6}
\end{vmatrix}
}

\newcommand{\DETuvwxijkl}[8]{
\begin{vmatrix}
 {#1}_{#5} & {#1}_{#6} & {#1}_{#7} & {#1}_{#8} \\
 {#2}_{#5} & {#2}_{#6} & {#2}_{#7} & {#2}_{#8} \\
 {#3}_{#5} & {#3}_{#6} & {#3}_{#7} & {#3}_{#8} \\
 {#4}_{#5} & {#4}_{#6} & {#4}_{#7} & {#4}_{#8} \\
\end{vmatrix}
}

%\newcommand{\DETuvwxyijklm}[10]{
%\begin{vmatrix}
% {#1}_{#6} & {#1}_{#7} & {#1}_{#8} & {#1}_{#9} & {#1}_{#10} \\
% {#2}_{#6} & {#2}_{#7} & {#2}_{#8} & {#2}_{#9} & {#2}_{#10} \\
% {#3}_{#6} & {#3}_{#7} & {#3}_{#8} & {#3}_{#9} & {#3}_{#10} \\
% {#4}_{#6} & {#4}_{#7} & {#4}_{#8} & {#4}_{#9} & {#4}_{#10} \\
% {#5}_{#6} & {#5}_{#7} & {#5}_{#8} & {#5}_{#9} & {#5}_{#10}
%\end{vmatrix}
%}

% R3 vector.
\newcommand{\VectorThree}[3]{
\begin{bmatrix}
 {#1} \\
 {#2} \\
 {#3}
\end{bmatrix}
}


%%<misc>
%
\newcommand{\Abs}[1]{{\left\lvert{#1}\right\rvert}}
\newcommand{\spacegrad}[0]{\boldsymbol{\nabla}}
\newcommand{\grad}[0]{\nabla}
\newcommand{\LL}[0]{\mathcal{L}}

% == \partial_{#1} {#2}
\newcommand{\PD}[2]{\frac{\partial {#2}}{\partial {#1}}}
% inline variant
\newcommand{\PDi}[2]{{\partial {#2}}/{\partial {#1}}}

\newcommand{\PDD}[3]{\frac{\partial^2 {#3}}{\partial {#1}\partial {#2}}}
%\newcommand{\PDd}[2]{\frac{\partial^2 {#2}}{{\partial{#1}}^2}}
\newcommand{\PDsq}[2]{\frac{\partial^2 {#2}}{(\partial {#1})^2}}

\newcommand{\Partial}[2]{\frac{\partial {#1}}{\partial {#2}}}
\DeclareMathOperator{\RejName}{Rej}
\newcommand{\Rej}[2]{\RejName_{#1}\left( {#2} \right)}
\newcommand{\Rm}[1]{\mathbb{R}^{#1}}
\newcommand{\Cm}[1]{\mathbb{C}^{#1}}
\newcommand{\conj}[0]{{*}}

%</misc>

% <grade selection>
%
\newcommand{\gpgrade}[2] {{\left\langle{{#1}}\right\rangle}_{#2}}

\newcommand{\gpgradezero}[1] {\gpgrade{#1}{}}
%\newcommand{\gpscalargrade}[1] {{\left\langle{{#1}}\right\rangle}}
%\newcommand{\gpgradezero}[1] {\gpgrade{#1}{0}}

%\newcommand{\gpgradeone}[1] {{\left\langle{{#1}}\right\rangle}_{1}}
\newcommand{\gpgradeone}[1] {\gpgrade{#1}{1}}

\newcommand{\gpgradetwo}[1] {\gpgrade{#1}{2}}
\newcommand{\gpgradethree}[1] {\gpgrade{#1}{3}}
\newcommand{\gpgradefour}[1] {\gpgrade{#1}{4}}
%
% </grade selection>



\newcommand{\adot}[0]{{\dot{a}}}
\newcommand{\bdot}[0]{{\dot{b}}}
% taken for centered dot:
%\newcommand{\cdot}[0]{{\dot{c}}}
%\newcommand{\ddot}[0]{{\dot{d}}}
\newcommand{\edot}[0]{{\dot{e}}}
\newcommand{\fdot}[0]{{\dot{f}}}
\newcommand{\gdot}[0]{{\dot{g}}}
\newcommand{\hdot}[0]{{\dot{h}}}
\newcommand{\idot}[0]{{\dot{i}}}
\newcommand{\jdot}[0]{{\dot{j}}}
\newcommand{\kdot}[0]{{\dot{k}}}
\newcommand{\ldot}[0]{{\dot{l}}}
\newcommand{\mdot}[0]{{\dot{m}}}
\newcommand{\ndot}[0]{{\dot{n}}}
%\newcommand{\odot}[0]{{\dot{o}}}
\newcommand{\pdot}[0]{{\dot{p}}}
\newcommand{\qdot}[0]{{\dot{q}}}
\newcommand{\rdot}[0]{{\dot{r}}}
\newcommand{\sdot}[0]{{\dot{s}}}
\newcommand{\tdot}[0]{{\dot{t}}}
\newcommand{\udot}[0]{{\dot{u}}}
\newcommand{\vdot}[0]{{\dot{v}}}
\newcommand{\wdot}[0]{{\dot{w}}}
\newcommand{\xdot}[0]{{\dot{x}}}
\newcommand{\ydot}[0]{{\dot{y}}}
\newcommand{\zdot}[0]{{\dot{z}}}
\newcommand{\addot}[0]{{\ddot{a}}}
\newcommand{\bddot}[0]{{\ddot{b}}}
\newcommand{\cddot}[0]{{\ddot{c}}}
%\newcommand{\dddot}[0]{{\ddot{d}}}
\newcommand{\eddot}[0]{{\ddot{e}}}
\newcommand{\fddot}[0]{{\ddot{f}}}
\newcommand{\gddot}[0]{{\ddot{g}}}
\newcommand{\hddot}[0]{{\ddot{h}}}
\newcommand{\iddot}[0]{{\ddot{i}}}
\newcommand{\jddot}[0]{{\ddot{j}}}
\newcommand{\kddot}[0]{{\ddot{k}}}
\newcommand{\lddot}[0]{{\ddot{l}}}
\newcommand{\mddot}[0]{{\ddot{m}}}
\newcommand{\nddot}[0]{{\ddot{n}}}
\newcommand{\oddot}[0]{{\ddot{o}}}
\newcommand{\pddot}[0]{{\ddot{p}}}
\newcommand{\qddot}[0]{{\ddot{q}}}
\newcommand{\rddot}[0]{{\ddot{r}}}
\newcommand{\sddot}[0]{{\ddot{s}}}
\newcommand{\tddot}[0]{{\ddot{t}}}
\newcommand{\uddot}[0]{{\ddot{u}}}
\newcommand{\vddot}[0]{{\ddot{v}}}
\newcommand{\wddot}[0]{{\ddot{w}}}
\newcommand{\xddot}[0]{{\ddot{x}}}
\newcommand{\yddot}[0]{{\ddot{y}}}
\newcommand{\zddot}[0]{{\ddot{z}}}

%<bold and dot greek symbols>
%

\newcommand{\Deltadot}[0]{{\dot{\Delta}}}
\newcommand{\Gammadot}[0]{{\dot{\Gamma}}}
\newcommand{\Lambdadot}[0]{{\dot{\Lambda}}}
\newcommand{\Omegadot}[0]{{\dot{\Omega}}}
\newcommand{\Phidot}[0]{{\dot{\Phi}}}
\newcommand{\Pidot}[0]{{\dot{\Pi}}}
\newcommand{\Psidot}[0]{{\dot{\Psi}}}
\newcommand{\Sigmadot}[0]{{\dot{\Sigma}}}
\newcommand{\Thetadot}[0]{{\dot{\Theta}}}
\newcommand{\Upsilondot}[0]{{\dot{\Upsilon}}}
\newcommand{\Xidot}[0]{{\dot{\Xi}}}
\newcommand{\alphadot}[0]{{\dot{\alpha}}}
\newcommand{\betadot}[0]{{\dot{\beta}}}
\newcommand{\chidot}[0]{{\dot{\chi}}}
\newcommand{\deltadot}[0]{{\dot{\delta}}}
\newcommand{\epsilondot}[0]{{\dot{\epsilon}}}
\newcommand{\etadot}[0]{{\dot{\eta}}}
\newcommand{\gammadot}[0]{{\dot{\gamma}}}
\newcommand{\kappadot}[0]{{\dot{\kappa}}}
\newcommand{\lambdadot}[0]{{\dot{\lambda}}}
\newcommand{\mudot}[0]{{\dot{\mu}}}
\newcommand{\nudot}[0]{{\dot{\nu}}}
\newcommand{\omegadot}[0]{{\dot{\omega}}}
\newcommand{\phidot}[0]{{\dot{\phi}}}
\newcommand{\pidot}[0]{{\dot{\pi}}}
\newcommand{\psidot}[0]{{\dot{\psi}}}
\newcommand{\rhodot}[0]{{\dot{\rho}}}
\newcommand{\sigmadot}[0]{{\dot{\sigma}}}
\newcommand{\taudot}[0]{{\dot{\tau}}}
\newcommand{\thetadot}[0]{{\dot{\theta}}}
\newcommand{\upsilondot}[0]{{\dot{\upsilon}}}
\newcommand{\varepsilondot}[0]{{\dot{\varepsilon}}}
\newcommand{\varphidot}[0]{{\dot{\varphi}}}
\newcommand{\varpidot}[0]{{\dot{\varpi}}}
\newcommand{\varrhodot}[0]{{\dot{\varrho}}}
\newcommand{\varsigmadot}[0]{{\dot{\varsigma}}}
\newcommand{\varthetadot}[0]{{\dot{\vartheta}}}
\newcommand{\xidot}[0]{{\dot{\xi}}}
\newcommand{\zetadot}[0]{{\dot{\zeta}}}

\newcommand{\Deltaddot}[0]{{\ddot{\Delta}}}
\newcommand{\Gammaddot}[0]{{\ddot{\Gamma}}}
\newcommand{\Lambdaddot}[0]{{\ddot{\Lambda}}}
\newcommand{\Omegaddot}[0]{{\ddot{\Omega}}}
\newcommand{\Phiddot}[0]{{\ddot{\Phi}}}
\newcommand{\Piddot}[0]{{\ddot{\Pi}}}
\newcommand{\Psiddot}[0]{{\ddot{\Psi}}}
\newcommand{\Sigmaddot}[0]{{\ddot{\Sigma}}}
\newcommand{\Thetaddot}[0]{{\ddot{\Theta}}}
\newcommand{\Upsilonddot}[0]{{\ddot{\Upsilon}}}
\newcommand{\Xiddot}[0]{{\ddot{\Xi}}}
\newcommand{\alphaddot}[0]{{\ddot{\alpha}}}
\newcommand{\betaddot}[0]{{\ddot{\beta}}}
\newcommand{\chiddot}[0]{{\ddot{\chi}}}
\newcommand{\deltaddot}[0]{{\ddot{\delta}}}
\newcommand{\epsilonddot}[0]{{\ddot{\epsilon}}}
\newcommand{\etaddot}[0]{{\ddot{\eta}}}
\newcommand{\gammaddot}[0]{{\ddot{\gamma}}}
\newcommand{\kappaddot}[0]{{\ddot{\kappa}}}
\newcommand{\lambdaddot}[0]{{\ddot{\lambda}}}
\newcommand{\muddot}[0]{{\ddot{\mu}}}
\newcommand{\nuddot}[0]{{\ddot{\nu}}}
\newcommand{\omegaddot}[0]{{\ddot{\omega}}}
\newcommand{\phiddot}[0]{{\ddot{\phi}}}
\newcommand{\piddot}[0]{{\ddot{\pi}}}
\newcommand{\psiddot}[0]{{\ddot{\psi}}}
\newcommand{\rhoddot}[0]{{\ddot{\rho}}}
\newcommand{\sigmaddot}[0]{{\ddot{\sigma}}}
\newcommand{\tauddot}[0]{{\ddot{\tau}}}
\newcommand{\thetaddot}[0]{{\ddot{\theta}}}
\newcommand{\upsilonddot}[0]{{\ddot{\upsilon}}}
\newcommand{\varepsilonddot}[0]{{\ddot{\varepsilon}}}
\newcommand{\varphiddot}[0]{{\ddot{\varphi}}}
\newcommand{\varpiddot}[0]{{\ddot{\varpi}}}
\newcommand{\varrhoddot}[0]{{\ddot{\varrho}}}
\newcommand{\varsigmaddot}[0]{{\ddot{\varsigma}}}
\newcommand{\varthetaddot}[0]{{\ddot{\vartheta}}}
\newcommand{\xiddot}[0]{{\ddot{\xi}}}
\newcommand{\zetaddot}[0]{{\ddot{\zeta}}}

\newcommand{\BDelta}[0]{\boldsymbol{\Delta}}
\newcommand{\BGamma}[0]{\boldsymbol{\Gamma}}
\newcommand{\BLambda}[0]{\boldsymbol{\Lambda}}
\newcommand{\BOmega}[0]{\boldsymbol{\Omega}}
\newcommand{\BPhi}[0]{\boldsymbol{\Phi}}
\newcommand{\BPi}[0]{\boldsymbol{\Pi}}
\newcommand{\BPsi}[0]{\boldsymbol{\Psi}}
\newcommand{\BSigma}[0]{\boldsymbol{\Sigma}}
\newcommand{\BTheta}[0]{\boldsymbol{\Theta}}
\newcommand{\BUpsilon}[0]{\boldsymbol{\Upsilon}}
\newcommand{\BXi}[0]{\boldsymbol{\Xi}}
\newcommand{\Balpha}[0]{\boldsymbol{\alpha}}
\newcommand{\Bbeta}[0]{\boldsymbol{\beta}}
\newcommand{\Bchi}[0]{\boldsymbol{\chi}}
\newcommand{\Bdelta}[0]{\boldsymbol{\delta}}
\newcommand{\Bepsilon}[0]{\boldsymbol{\epsilon}}
\newcommand{\Beta}[0]{\boldsymbol{\eta}}
\newcommand{\Bgamma}[0]{\boldsymbol{\gamma}}
\newcommand{\Bkappa}[0]{\boldsymbol{\kappa}}
\newcommand{\Blambda}[0]{\boldsymbol{\lambda}}
\newcommand{\Bmu}[0]{\boldsymbol{\mu}}
\newcommand{\Bnu}[0]{\boldsymbol{\nu}}
%\newcommand{\Bomega}[0]{\boldsymbol{\omega}}
\newcommand{\Bphi}[0]{\boldsymbol{\phi}}
\newcommand{\Bpi}[0]{\boldsymbol{\pi}}
\newcommand{\Bpsi}[0]{\boldsymbol{\psi}}
\newcommand{\Brho}[0]{\boldsymbol{\rho}}
\newcommand{\Bsigma}[0]{\boldsymbol{\sigma}}
%\newcommand{\Btau}[0]{\boldsymbol{\tau}}
%\newcommand{\Btheta}[0]{\boldsymbol{\theta}}
\newcommand{\Bupsilon}[0]{\boldsymbol{\upsilon}}
\newcommand{\Bvarepsilon}[0]{\boldsymbol{\varepsilon}}
\newcommand{\Bvarphi}[0]{\boldsymbol{\varphi}}
\newcommand{\Bvarpi}[0]{\boldsymbol{\varpi}}
\newcommand{\Bvarrho}[0]{\boldsymbol{\varrho}}
\newcommand{\Bvarsigma}[0]{\boldsymbol{\varsigma}}
\newcommand{\Bvartheta}[0]{\boldsymbol{\vartheta}}
\newcommand{\Bxi}[0]{\boldsymbol{\xi}}
\newcommand{\Bzeta}[0]{\boldsymbol{\zeta}}
%
%</bold and dot greek symbols>
%<infrequent>
%
%\newcommand{\AreaOp}[1]{\AName_{#1}}
%\newcommand{\Babs}[0]{\abs{\BB}}
%\newcommand{\Bcap}[0]{\hat{\BB}}
%\newcommand{\BrPrimeRej}[0]{\rcap(\rcap \wedge \Br')}
%\newcommand{\CA}[0]{\mathcal{A}}
%\newcommand{\Cos}[1]{\cos{\left({#1}\right)}}
%\newcommand{\Det}[1] {\abs{#1}}
%\newcommand{\Dsq}[2] {\frac {\partial^2 {#1}} {\partial {#2}^2}}
%\newcommand{\Exp}[1]{\exp{\left({#1}\right)}}
%\newcommand{\Norm}[1]{\left\lVert{#1}\right\rVert}
%\newcommand{\Sin}[1]{\sin{\left({#1}\right)}}
%\newcommand{\T}[0]{\text{T}}
%\newcommand{\VolumeOp}[1]{\VName_{#1}}
%\newcommand{\agrad}[0]{\Ba \cdot \nabla}
%\newcommand{\alphacap}[0]{\hat{\boldsymbol{\alpha}}}
%\newcommand{\Fcap}[0]{\hat{\BF}}
%\newcommand{\bithree}[0]{{\Bi}_3}
%\newcommand{\bxa}[0]{\Bx\Ba}
%\newcommand{\coordvec}[2]{
%\newcommand{\costheta}[0]{\acap \cdot \xcap}
%\newcommand{\ddt}[1]{\ddot{#1}}
%\newcommand{\ddu}[1] {\frac {d{#1}} {du}}
%\newcommand{\dsqxj}[2] {\frac {\partial^2 {#1}} {\partial {x_{#2}}^2}}
%\newcommand{\dtheta}[1]{\frac{d {#1}}{d \theta}}
%\newcommand{\dt}[1]{\dot{#1}}
%\newcommand{\dt}[1]{\frac{d {#1}}{dt}}
%\newcommand{\dxj}[2] {\frac {\partial {#1}} {\partial {x_{#2}}}}
%\newcommand{\halfPhi}[0]{\frac{\phi}{2}}
%\newcommand{\half}[0]{\inv{2}}
%\newcommand{\inv}[1]{\frac{1}{#1}}
%\newcommand{\laplacian}[0]{\nabla^2}
%\newcommand{\matrixoftx}[3]{
%\newcommand{\nrrp}[0]{\norm{\rcap \wedge \Br'}}
%\newcommand{\oiint}{\bigcirc \hspace{-1.4em} \int \hspace{-.8em} \int}
%\newcommand{\transpose}[1]{{#1}^{\text{T}}}
%\newcommand{\transpose}[1]{{{#1}^{\TextTranspose}}}
%\newcommand{\transpose}[1]{{{#1}^{\text{T}}}}
%\newcommand{\barA}[0]{\bar{A}}
%\newcommand{\qbar}[0]{\bar{q}}
%\newcommand{\qdotbar}[0]{\dot{\bar{q}}}
%
%</infrequent>





%\usepackage[bookmarks=true]{hyperref}

%\usepackage{color,cite,graphicx}
   % use colour in the document, put your citations as [1-4]
   % rather than [1,2,3,4] (it looks nicer, and the extended LaTeX2e
   % graphics package. 
%\usepackage{latexsym,amssymb,epsf} % don't remember if these are
   % needed, but their inclusion can't do any damage


\chapter{Dot product linearity by construction. }
%\author{Peeter Joot \quad peeter.joot@gmail.com }
%\date{ March 13, 2009.  Last Revision: $Date: 2009/06/03 22:13:06 $ }

%\begin{document}

%\maketitle{}
%\tableofcontents
\section{Motivation. }

Reading of \cite{byron1992mca} it is observed that the dot product when defined via geometrical constructs such as

\begin{align*}
\Bx \cdot \By = \Abs{\Bx} \Abs{\By} \cos\theta
\end{align*}

is linear

\begin{align*}
\Bx \cdot (\By + \Bz) = \Bx \cdot \By + \Bx \cdot \Bz 
\end{align*}

Despite the fact that this is obvious when the dot product is given in algebraic form, this doesn't look obvious geometrically, so my 
immediate thought was ``how would you show this geometrically''.  Sure enough, in the next paragraph is the statement that the reader will
want to show this by construction.  Here's such a demonstration and construction.

\begin{figure}[htp]
\centering
\includegraphics[totalheight=0.4\textheight]{dot_lin}
\caption{Sum of two vectors and their angles with another.}\label{fig:dot_linearity}
\end{figure}

\section{Info from the figure. }
\subsection{Law of cosines. }

From 
figure \ref{fig:dot_linearity}, with $\Be_1 = \xcap$ we have

\begin{align*}
\By &= \Abs{\By} \cos\theta \Be_1 + \Abs{\By} \sin\theta \Be_2 \\
\Bz &= \Abs{\Bz} \cos\alpha \Be_1 + \Abs{\Bz} \sin\alpha \Be_2
\end{align*}

The vector sum $\By + \Bz$ is therefore

\begin{align*}
\By + \Bz &= (\Abs{\By} \cos\theta + \Abs{\Bz} \cos\alpha )\Be_1 + (\Abs{\By} \sin\theta + \Abs{\Bz} \sin\alpha ) \Be_2
\end{align*}

By using Pythagoras's law, a calculation of the squared length, produces the law of cosines

\begin{align*}
\Abs{\By + \Bz}^2 
&= (\Abs{\By} \cos\theta + \Abs{\Bz} \cos\alpha )^2 + (\Abs{\By} \sin\theta + \Abs{\Bz} \sin\alpha )^2 \\
&= \Abs{\By}^2 + \Abs{\Bz}^2 + 2 \Abs{\By} \Abs{\Bz} (\cos\theta \cos\alpha + \sin\theta \sin\alpha) \\
&= \Abs{\By}^2 + \Abs{\Bz}^2 + 2 \Abs{\By} \Abs{\Bz} \cos(\theta -\alpha) \\
\end{align*}

\subsection{Linearity by construction. }

Okay, that is a digression, ... back to the original problem.  We get the dot product linearity follows from direct calculation
of the cosine of the angle between $\xcap$ and $\By + \Bz$.  Again from the figure we have

\begin{align*}
\cos\beta &= \frac{\Abs{\By}\cos\theta + \Abs{\By}\cos\alpha}{\Abs{\By + \Bz}}
\end{align*}

or
\begin{align*}
{\Abs{\By + \Bz}} \cos\beta &= {\Abs{\By}\cos\theta + \Abs{\By}\cos\alpha} \\
\end{align*}

But $\xcap \cdot \By = \Abs{\By}\cos\theta $, and $\xcap \cdot \Bz = \Abs{\By}\cos\alpha $, so we have

\begin{align*}
{\Abs{\By + \Bz}} \cos\beta &= \xcap \cdot \By + \xcap \cdot \Bz
\end{align*}

Multiplying by $\Abs{\Bx}$ we have

\begin{align*}
\Abs{\Bx} {\Abs{\By + \Bz}} \cos\beta &= \Bx \cdot \By + \Bx \cdot \Bz
\end{align*}

The left hand side is $\Bx \cdot (\By + \Bz)$, which completes the desired demonstration.

%\bibliographystyle{plainnat}
%\bibliography{myrefs}

%\end{document}

%
% Copyright � 2012 Peeter Joot.  All Rights Reserved.
% Licenced as described in the file LICENSE under the root directory of this GIT repository.
%

% 
% 
%\documentclass{article}      % Specifies the document class

%\usepackage{amsmath}
\usepackage{mathpazo}

%
% shorthand for bold symbols, convenient for vectors and matrices
%
\newcommand{\Ba}[0]{\mathbf{a}}
\newcommand{\Bb}[0]{\mathbf{b}}
\newcommand{\Bc}[0]{\mathbf{c}}
\newcommand{\Bd}[0]{\mathbf{d}}
\newcommand{\Be}[0]{\mathbf{e}}
\newcommand{\Bf}[0]{\mathbf{f}}
\newcommand{\Bg}[0]{\mathbf{g}}
\newcommand{\Bh}[0]{\mathbf{h}}
\newcommand{\Bi}[0]{\mathbf{i}}
\newcommand{\Bj}[0]{\mathbf{j}}
\newcommand{\Bk}[0]{\mathbf{k}}
\newcommand{\Bl}[0]{\mathbf{l}}
\newcommand{\Bm}[0]{\mathbf{m}}
\newcommand{\Bn}[0]{\mathbf{n}}
\newcommand{\Bo}[0]{\mathbf{o}}
\newcommand{\Bp}[0]{\mathbf{p}}
\newcommand{\Bq}[0]{\mathbf{q}}
\newcommand{\Br}[0]{\mathbf{r}}
\newcommand{\Bs}[0]{\mathbf{s}}
\newcommand{\Bt}[0]{\mathbf{t}}
\newcommand{\Bu}[0]{\mathbf{u}}
\newcommand{\Bv}[0]{\mathbf{v}}
\newcommand{\Bw}[0]{\mathbf{w}}
\newcommand{\Bx}[0]{\mathbf{x}}
\newcommand{\By}[0]{\mathbf{y}}
\newcommand{\Bz}[0]{\mathbf{z}}
\newcommand{\BA}[0]{\mathbf{A}}
\newcommand{\BB}[0]{\mathbf{B}}
\newcommand{\BC}[0]{\mathbf{C}}
\newcommand{\BD}[0]{\mathbf{D}}
\newcommand{\BE}[0]{\mathbf{E}}
\newcommand{\BF}[0]{\mathbf{F}}
\newcommand{\BG}[0]{\mathbf{G}}
\newcommand{\BH}[0]{\mathbf{H}}
\newcommand{\BI}[0]{\mathbf{I}}
\newcommand{\BJ}[0]{\mathbf{J}}
\newcommand{\BK}[0]{\mathbf{K}}
\newcommand{\BL}[0]{\mathbf{L}}
\newcommand{\BM}[0]{\mathbf{M}}
\newcommand{\BN}[0]{\mathbf{N}}
\newcommand{\BO}[0]{\mathbf{O}}
\newcommand{\BP}[0]{\mathbf{P}}
\newcommand{\BQ}[0]{\mathbf{Q}}
\newcommand{\BR}[0]{\mathbf{R}}
\newcommand{\BS}[0]{\mathbf{S}}
\newcommand{\BT}[0]{\mathbf{T}}
\newcommand{\BU}[0]{\mathbf{U}}
\newcommand{\BV}[0]{\mathbf{V}}
\newcommand{\BW}[0]{\mathbf{W}}
\newcommand{\BX}[0]{\mathbf{X}}
\newcommand{\BY}[0]{\mathbf{Y}}
\newcommand{\BZ}[0]{\mathbf{Z}}

\newcommand{\Bzero}[0]{\mathbf{0}}
\newcommand{\Btheta}[0]{\boldsymbol{\theta}}
\newcommand{\Btau}[0]{\boldsymbol{\tau}}
\newcommand{\Bomega}[0]{\boldsymbol{\omega}}

%
% shorthand for unit vectors
%
\newcommand{\acap}[0]{\hat{\Ba}}
\newcommand{\bcap}[0]{\hat{\Bb}}
\newcommand{\ccap}[0]{\hat{\Bc}}
\newcommand{\dcap}[0]{\hat{\Bd}}
\newcommand{\ecap}[0]{\hat{\Be}}
\newcommand{\fcap}[0]{\hat{\Bf}}
\newcommand{\gcap}[0]{\hat{\Bg}}
\newcommand{\hcap}[0]{\hat{\Bh}}
\newcommand{\icap}[0]{\hat{\Bi}}
\newcommand{\jcap}[0]{\hat{\Bj}}
\newcommand{\kcap}[0]{\hat{\Bk}}
\newcommand{\lcap}[0]{\hat{\Bl}}
\newcommand{\mcap}[0]{\hat{\Bm}}
\newcommand{\ncap}[0]{\hat{\Bn}}
\newcommand{\ocap}[0]{\hat{\Bo}}
\newcommand{\pcap}[0]{\hat{\Bp}}
\newcommand{\qcap}[0]{\hat{\Bq}}
\newcommand{\rcap}[0]{\hat{\Br}}
\newcommand{\scap}[0]{\hat{\Bs}}
\newcommand{\tcap}[0]{\hat{\Bt}}
\newcommand{\ucap}[0]{\hat{\Bu}}
\newcommand{\vcap}[0]{\hat{\Bv}}
\newcommand{\wcap}[0]{\hat{\Bw}}
\newcommand{\xcap}[0]{\hat{\Bx}}
\newcommand{\ycap}[0]{\hat{\By}}
\newcommand{\zcap}[0]{\hat{\Bz}}
\newcommand{\thetacap}[0]{\hat{\Btheta}}

%
% to write R^n and C^n in a distinguishable fashion.  Perhaps change this
% to the double lined characters upon figuring out how to do so.
%
\newcommand{\C}[1]{$\mathbb{C}^{#1}$}
\newcommand{\R}[1]{$\mathbb{R}^{#1}$}

%
% various generally useful helpers
%

% derivative of #1 wrt. #2:
\newcommand{\D}[2] {\frac {d#2} {d#1}}

\newcommand{\inv}[1]{\frac{1}{#1}}
\newcommand{\cross}[0]{\times}

\newcommand{\abs}[1]{\lvert{#1}\rvert}
\newcommand{\norm}[1]{\lVert{#1}\rVert}
\newcommand{\innerprod}[2]{\langle{#1}, {#2}\rangle}
\newcommand{\dotprod}[2]{{#1} \cdot {#2}}
\newcommand{\bdotprod}[2]{\left({#1} \cdot {#2}\right)}
\newcommand{\crossprod}[2]{{#1} \cross {#2}}
\newcommand{\tripleprod}[3]{\dotprod{\left(\crossprod{#1}{#2}\right)}{#3}}

\DeclareMathOperator{\Proj}{Proj}
\DeclareMathOperator{\Span}{span}
\DeclareMathOperator{\Sgn}{sgn}
\DeclareMathOperator{\Area}{Area}
\DeclareMathOperator{\Volume}{Volume}

%
% A few miscellaneous things specific to this document
%
\newcommand{\crossop}[1]{\crossprod{#1}{}}

% R2 vector.
\newcommand{\VectorTwo}[2]{
\begin{bmatrix}
 {#1} \\
 {#2}
\end{bmatrix}
}

\newcommand{\VectorN}[1]{
\begin{bmatrix}
{#1}_1 \\
{#1}_2 \\
\vdots \\
{#1}_N \\
\end{bmatrix}
}

\newcommand{\DETuvij}[4]{
\begin{vmatrix}
 {#1}_{#3} & {#1}_{#4} \\
 {#2}_{#3} & {#2}_{#4}
\end{vmatrix}
}

\newcommand{\DETuvwijk}[6]{
\begin{vmatrix}
 {#1}_{#4} & {#1}_{#5} & {#1}_{#6} \\
 {#2}_{#4} & {#2}_{#5} & {#2}_{#6} \\
 {#3}_{#4} & {#3}_{#5} & {#3}_{#6}
\end{vmatrix}
}

\newcommand{\DETuvwxijkl}[8]{
\begin{vmatrix}
 {#1}_{#5} & {#1}_{#6} & {#1}_{#7} & {#1}_{#8} \\
 {#2}_{#5} & {#2}_{#6} & {#2}_{#7} & {#2}_{#8} \\
 {#3}_{#5} & {#3}_{#6} & {#3}_{#7} & {#3}_{#8} \\
 {#4}_{#5} & {#4}_{#6} & {#4}_{#7} & {#4}_{#8} \\
\end{vmatrix}
}

%\newcommand{\DETuvwxyijklm}[10]{
%\begin{vmatrix}
% {#1}_{#6} & {#1}_{#7} & {#1}_{#8} & {#1}_{#9} & {#1}_{#10} \\
% {#2}_{#6} & {#2}_{#7} & {#2}_{#8} & {#2}_{#9} & {#2}_{#10} \\
% {#3}_{#6} & {#3}_{#7} & {#3}_{#8} & {#3}_{#9} & {#3}_{#10} \\
% {#4}_{#6} & {#4}_{#7} & {#4}_{#8} & {#4}_{#9} & {#4}_{#10} \\
% {#5}_{#6} & {#5}_{#7} & {#5}_{#8} & {#5}_{#9} & {#5}_{#10}
%\end{vmatrix}
%}

% R3 vector.
\newcommand{\VectorThree}[3]{
\begin{bmatrix}
 {#1} \\
 {#2} \\
 {#3}
\end{bmatrix}
}



%\usepackage{color,cite,graphicx}
   % use colour in the document, put your citations as [1-4]
   % rather than [1,2,3,4] (it looks nicer, and the extended LaTeX2e
   % graphics package. 
%\usepackage{latexsym,amssymb,epsf} % do not remember if these are
   % needed, but their inclusion can not do any damage


%
% The real thing:
%

                             % The preamble begins here.
\chapter{Pythagoras law}
\label{chap:pythagoras}
%\author{Peeter Joot}         % Declares the author's name.
\date{ March 17, 2008.  pythagoras.tex }

%\begin{document}             % End of preamble and beginning of text.

%\maketitle{}

\section{Length}

To base vector multiplication on length, and examine all the consequences of having done so, it is first necessary to 
The geometrical definition of length for vectors generalizes Pythagoras theorem to higher dimensions.

In two dimensions this theorem can be proved with the aid of the following diagram

\imageFigure{../../figures/miscphysics/square_in_square}{Geometrical Proof of Pythagoras Theorem for Right Triangle}{fig:phthagoras}{0.3}

The area of the interior and exterior squares is \(c^2\), and \((a+b)^2\) respectively.  The interior area can also be calculated by subtracting the area of the triangles from the exterior area:

\begin{equation}\label{eqn:pythagoras:20}
(a+b)^2 - 4(ab/2) = a^2 + b^2 + 2ab - 2ab
\end{equation}

Thus proving Pythagoras theorem for the length of the diagonal in a right angle triangle

\begin{equation}\label{eqn:pythagoras:40}
c^2 = a^2 + b^2
\end{equation}

%for a geometrical proof like this one should perhaps show that the inscribed shape is a square ; all the lengths being equal is sufficient IMO.

The length of a vector in three dimensions can be found by repeated application of Pythagoras theorem, as in the following figure 

\imageFigure{../../figures/miscphysics/3d_vector_len}{Length of vector in three dimensions}{fig:3dveclen}{0.3}

The vector \(\Be = \Ba + \Bb + \Bc\), where each of the vectors \(\Ba\), \(\Bb\), and \(\Bc\) are mutually perpendicular can be found by first calculating

\begin{equation}\label{eqn:pythagoras:60}
d^2 = a^2 + b^2
\end{equation}

Then

\begin{equation}\label{eqn:pythagoras:80}
e^2 = d^2 + c^2 = a^2 + b^2 + c^2
\end{equation}

This process can be repeated for any number of higher dimensions.  Having calculated the length of a \(N-1\) dimensional vector

\begin{equation}\label{eqn:pythagoras:100}
{L(\Bv)}^2 = \sum_{i=1}^{N-1} {{l_i}^2}
\end{equation}

Once an additional component of length \(l_N\) is added to that vector in a direction mutually perpendicular to all previous components the new length of this vector becomes

\begin{equation}\label{eqn:pythagoras:120}
\sum_{i=1}^{N-1} {{l_i}^2} + l_N^2 = \sum_{i=0}^{N} {{l_i}^2}
\end{equation}

This is what we mean by the geometrical length of a vector.

%\subsection{Pythagoras Law, and the vector product}
%
%We have a rule for vector multiplication when two vectors are collinear.  Comparison to Pythagoras law will provide an additional rule for vector multiplication when the vectors are completely perpendicular.
%

%\end{document}               % End of document.

%
% Copyright � 2012 Peeter Joot.  All Rights Reserved.
% Licenced as described in the file LICENSE under the root directory of this GIT repository.
%

% 
% 
\chapter{Singular Value Decomposition}
\label{chap:mpInverseSvdRoughNotes}
\date{ May 15, 2008.  mpInverseSvdRoughNotes.tex }

\section{blah}

\subsection{Application of projection as left pseudoinverse (ie: linear fitting)}

%We have shown that the left pseudoinverse product with the matrix can
%be expressed as a projection matrix (sum of the projection matrices
%associated with a set of orthonormal vectors)

%\begin{equation}\label{eqn:mpInvRough:pseudoprojmatsum}
%A^{+} A =
%\sum_{k=1}^r {u_k}u_k^\T
%\end{equation}
%

%(note this is a different ``\(V\)'' than the \(V\) in \(A = U \Sigma V^\T\) since it only includes the first \(r\) columns).
%This allows us to write the matrix of \eqnref{eqn:mpInvRough:pseudoprojmatsum} as

%\begin{equation}
%A^{+} A = V V^\T
%\end{equation}

Equation (FIXME)
%\eqnref{eqn:mpInvRough:projectiongeneralmatrix} %% ??
provides us a way to find best solutions to general equations of the form:


\begin{equation}\label{eqn:mpInverseSvdRoughNotes:20}
A x = b
\end{equation}

Here \(A\) is the matrix of a linear transformation, \(A : \mathbb{R}^k \rightarrow \mathbb{R}^n\), for some \(k<n\).
By ``best solutions'' here, we give this the geometrical meaning, namely, the solution matching the projection of \(b\) onto the space.

If b is not completely in the column space \(C(A)\) of \(A\), this can have no solution.  However, writing

\begin{equation}\label{eqn:mpInverseSvdRoughNotes:40}
b = \Proj_A(b) + b_\perp
\end{equation}

as the components of \(b\) in \(C(A)\) and not in \(C(A)\) respectively we can at least solve the reduced equation for \(\hat{x}\):


\begin{equation}\label{eqn:mpInvRough:reducedinverseproblem}
A \hat{x} = \Proj_A(b)
\end{equation}


This will be possible even in circumstances that the original equation had no solution.  Specifically, the vector b when projected onto the plane can be expressed as some
linear combination of the columns of \(A\) (a basis for the subspace).

Substitution of our projection result into \eqnref{eqn:mpInvRough:reducedinverseproblem} yields:

\begin{equation}\label{eqn:mpInverseSvdRoughNotes:180}
\begin{aligned}
A \hat{x} 
&= \Proj_{A}\left(b\right) = A (A^\T A)^{-1} A^\T b
\end{aligned}
\end{equation}

The simplest case here is when \(A\) is of full column rank since one can pre-multiply this complete equation by \(A^\T\) without any possibility of nulling
\(A \hat{x}\).

\begin{equation}\label{eqn:mpInverseSvdRoughNotes:200}
\begin{aligned}
A^\T A \hat{x} 
&= A^\T A (A^\T A)^{-1} A^\T b \\
&= A^\T b \\
\end{aligned}
\end{equation}

Thus our best fit vector is

\begin{equation}
\hat{x} 
= (A^\T A)^{-1} A^\T b
\end{equation}

Another way to view this is for any vector \(x\) that is not in the null space \(N(A)\), then the matrix:

\begin{equation}
A^{+}= (A^\T A)^{-1} A^\T
\end{equation}

has the action of a left inverse for any full column rank matrix \(A\).  Thus when there is a solution to:

\begin{equation}
A x = b.
\end{equation}

It can be obtained by pre-multiplication using this "left" inverse.

\begin{equation}
A^{+} A x = x = A^{+} b
\end{equation}











































\section{SVD connection}


SVT decomposition is an factoring of \(A \in M^{m \times n}\) with orthonormal matrices \(U \in M^{m \times m}\)

and \(V \in M^{n \times n} \) producing the following form:

\begin{equation}\label{eqn:mpInverseSvdRoughNotes:60}
A = U \Sigma V^\T
\end{equation}

Sigma has the form:

\begin{equation}\label{eqn:mpInverseSvdRoughNotes:80}
\Sigma = 
\begin{bmatrix}
D_{r,r} & 0_{r,n-r} \\
0_{m-r,r} & 0_{m-r,n-r} \\
\end{bmatrix}
\end{equation}

where \(r = \rank(A)\), and \(D\) is a diagonal matrix with the root of the (positive) eigenvalues of \(A^\T A\).

This provides a generalized spectral decomposition and similarity that applies to both non-square matrices and matrices not otherwise diagonalizable
(ie: square matrix with similarity to a Jordon form matrix).  Given this decomposition we can write:

\begin{equation}\label{eqn:mpInverseSvdRoughNotes:100}
\Sigma = U^\T A V
\end{equation}

If one were to ask the question of what is the closest that one could get to inverting such a matrix.  It is pretty clear that the closest one could get to
identity will be with multiplication of a \(\Sigma^{+}\) of the following form:

\begin{equation}\label{eqn:mpInverseSvdRoughNotes:120}
\Sigma^{+} \Sigma
=
\begin{bmatrix}
(D_{r,r})^{-1} & 0_{r,m-r} \\
0_{n-r,r} & 0_{n-r,m-r} \\
\end{bmatrix}
\begin{bmatrix}
D_{r,r} & 0_{r,n-r} \\
0_{m-r,r} & 0_{m-r,n-r} \\
\end{bmatrix}
=
\begin{bmatrix}
I_{r,r} & 0_{r,n-r} \\
0_{n-r,r} & 0_{n-r,n-r} \\
\end{bmatrix}
\end{equation}

For a right pseudoinverse we have a similar result:

\begin{equation}\label{eqn:mpInverseSvdRoughNotes:140}
\Sigma
\Sigma^{+}
=
\begin{bmatrix}
D_{r,r} & 0_{r,n-r} \\
0_{m-r,r} & 0_{m-r,n-r} \\
\end{bmatrix}
\begin{bmatrix}
(D_{r,r})^{-1} & 0_{r,m-r} \\
0_{n-r,r} & 0_{n-r,m-r} \\
\end{bmatrix}
=
\begin{bmatrix}
I_{r,r} & 0_{r,m-r} \\
0_{m-r,r} & 0_{m-r,m-r} \\
\end{bmatrix}
\end{equation}

With either of these one can define a corresponding pseudoinverse (left or right) as:

\begin{equation}
A^{+} = V \Sigma^{+} U^\T
\end{equation}

This is a logical definition, but how close is it to the projective
left inverse we calculated above in the case where \(A\) is not of full column 
rank?

Multiplication gives: 

\begin{equation}\label{eqn:mpInverseSvdRoughNotes:220}
\begin{aligned}
A^{+} A 
&= V \Sigma^{+} U^\T U \Sigma V^\T \\
&= V \Sigma^{+} \Sigma V^\T \\
&=
\begin{bmatrix}
v_1 & v_2 & \cdots & v_r & v_{r+1} & \cdots & v_n \\
\end{bmatrix}
\begin{bmatrix}
(D_{r,r})^{-1} & 0_{r,m-r} \\
0_{n-r,r} & 0_{n-r,m-r} \\
\end{bmatrix}
\begin{bmatrix}
D_{r,r} & 0_{r,n-r} \\
0_{m-r,r} & 0_{m-r,n-r} \\
\end{bmatrix}
\begin{bmatrix}
v_1^\T \\ v_2^\T \\ \vdots \\ v_r^\T \\ {v_{r+1}}^\T \\ \vdots \\ v_n^\T \\
\end{bmatrix}
\end{aligned}
\end{equation}
%Embeded in that is the same "as-close-to" identity as calculated above.

Writing \(D_{r,r} = [\delta_{ij}\sigma_i]_{ij}\), we have:

\begin{equation}\label{eqn:mpInvRough:VIrVt}
V \Sigma^{+} \Sigma V^\T 
=
\begin{bmatrix}
\frac{v_1}{\sigma_1} & \frac{v_2}{\sigma_2} & \cdots & \frac{v_r}{\sigma_2} & 0 & \cdots & 0 
\end{bmatrix}
\begin{bmatrix}
v_1^\T \sigma_1 \\ v_2^\T \sigma_2 \\ \vdots \\ v_r^\T \sigma_r \\ 0 \\ \vdots \\ 0 \\
\end{bmatrix}
\end{equation}

Considering this as the product of block matrices we have a product here of the form

\begin{equation}\label{eqn:mpInverseSvdRoughNotes:240}
\begin{aligned}
\begin{bmatrix}
A_{n,r} & 0_{n,n-r}
\end{bmatrix}
\begin{bmatrix}
B_{r,n} \\ 0_{n-r,n}
\end{bmatrix} \\
&=
\begin{bmatrix}
A_{n,r} B_{r,n} + 0_{n,n-r} 0_{n-r,n}
\end{bmatrix} \\
&=
\begin{bmatrix}
A_{n,r} B_{r,n} + 0_{n,n}
\end{bmatrix} \\
&=
\begin{bmatrix}
A_{n,r} B_{r,n}
\end{bmatrix}
\end{aligned}
\end{equation}

Thus we can strip the block zero matrices from \eqnref{eqn:mpInvRough:VIrVt} and write

\begin{equation}\label{eqn:mpInvRough:pseudoinversetimesmatrix}
A^{+} A =
V \Sigma^{+} \Sigma V^\T 
=
\begin{bmatrix}
\frac{v_1}{\sigma_1} & \frac{v_2}{\sigma_2} & \cdots & \frac{v_r}{\sigma_2} 
\end{bmatrix}
\begin{bmatrix}
v_1^\T \sigma_1 \\ v_2^\T \sigma_2 \\ \vdots \\ v_r^\T \sigma_r 
\end{bmatrix}
\end{equation}

Eliminating the \(\sigma\) terms we have:

\begin{equation}\label{eqn:mpInvRough:pseudoinversetimesmatrixsum}
A^{+} A =
\begin{bmatrix}
\sum_{k=1}^r {v_k}v_k^\T
\end{bmatrix}
=
\begin{bmatrix}
v_1 & v_2 & \cdots & v_r 
\end{bmatrix}
\begin{bmatrix}
v_1^\T \\ v_2^\T \\ \vdots \\ v_r^\T 
\end{bmatrix}
\end{equation}

We previously calculated a left inverse using the projection matrix associated with a full column rank matrix.  For this product to have the properties of a
left acting inverse we also expect it to be a projection.
Let us digress
slightly before looking at whether equation
\eqnref{eqn:mpInvRough:pseudoinversetimesmatrixsum} satisfies this expectation.

\subsection{Correlating the SVD derived projection matrix back to \texorpdfstring{\(A\)}{A}}

We now have to show that this is also the projection matrix associated

with the columns of the 
original matrix that we have an SVD factorization for

\begin{equation}\label{eqn:mpInverseSvdRoughNotes:160}
A = U \Sigma V^\T
\end{equation}

Once we show this, then we have also demonstrated that the first \(r\) 
(orthonormal) column vectors in the matrix \(V\) of this decomposition
are a basis for the column space of \(A\) itself.  Note that we are
switching back to the original definition of \(V \in M^{n,n}\) here, and
not the \(V \in M^{n,r}\) of equation (FIXME)
%\eqnref{eqn:mpInvRough:projOrthonormal}. % ???


\part{Calculus}
\documentclass{article}

\usepackage{amsmath}
\usepackage{mathpazo}

%
% shorthand for bold symbols, convenient for vectors and matrices
%
\newcommand{\Ba}[0]{\mathbf{a}}
\newcommand{\Bb}[0]{\mathbf{b}}
\newcommand{\Bc}[0]{\mathbf{c}}
\newcommand{\Bd}[0]{\mathbf{d}}
\newcommand{\Be}[0]{\mathbf{e}}
\newcommand{\Bf}[0]{\mathbf{f}}
\newcommand{\Bg}[0]{\mathbf{g}}
\newcommand{\Bh}[0]{\mathbf{h}}
\newcommand{\Bi}[0]{\mathbf{i}}
\newcommand{\Bj}[0]{\mathbf{j}}
\newcommand{\Bk}[0]{\mathbf{k}}
\newcommand{\Bl}[0]{\mathbf{l}}
\newcommand{\Bm}[0]{\mathbf{m}}
\newcommand{\Bn}[0]{\mathbf{n}}
\newcommand{\Bo}[0]{\mathbf{o}}
\newcommand{\Bp}[0]{\mathbf{p}}
\newcommand{\Bq}[0]{\mathbf{q}}
\newcommand{\Br}[0]{\mathbf{r}}
\newcommand{\Bs}[0]{\mathbf{s}}
\newcommand{\Bt}[0]{\mathbf{t}}
\newcommand{\Bu}[0]{\mathbf{u}}
\newcommand{\Bv}[0]{\mathbf{v}}
\newcommand{\Bw}[0]{\mathbf{w}}
\newcommand{\Bx}[0]{\mathbf{x}}
\newcommand{\By}[0]{\mathbf{y}}
\newcommand{\Bz}[0]{\mathbf{z}}
\newcommand{\BA}[0]{\mathbf{A}}
\newcommand{\BB}[0]{\mathbf{B}}
\newcommand{\BC}[0]{\mathbf{C}}
\newcommand{\BD}[0]{\mathbf{D}}
\newcommand{\BE}[0]{\mathbf{E}}
\newcommand{\BF}[0]{\mathbf{F}}
\newcommand{\BG}[0]{\mathbf{G}}
\newcommand{\BH}[0]{\mathbf{H}}
\newcommand{\BI}[0]{\mathbf{I}}
\newcommand{\BJ}[0]{\mathbf{J}}
\newcommand{\BK}[0]{\mathbf{K}}
\newcommand{\BL}[0]{\mathbf{L}}
\newcommand{\BM}[0]{\mathbf{M}}
\newcommand{\BN}[0]{\mathbf{N}}
\newcommand{\BO}[0]{\mathbf{O}}
\newcommand{\BP}[0]{\mathbf{P}}
\newcommand{\BQ}[0]{\mathbf{Q}}
\newcommand{\BR}[0]{\mathbf{R}}
\newcommand{\BS}[0]{\mathbf{S}}
\newcommand{\BT}[0]{\mathbf{T}}
\newcommand{\BU}[0]{\mathbf{U}}
\newcommand{\BV}[0]{\mathbf{V}}
\newcommand{\BW}[0]{\mathbf{W}}
\newcommand{\BX}[0]{\mathbf{X}}
\newcommand{\BY}[0]{\mathbf{Y}}
\newcommand{\BZ}[0]{\mathbf{Z}}

\newcommand{\Bzero}[0]{\mathbf{0}}
\newcommand{\Btheta}[0]{\boldsymbol{\theta}}
\newcommand{\Btau}[0]{\boldsymbol{\tau}}
\newcommand{\Bomega}[0]{\boldsymbol{\omega}}

%
% shorthand for unit vectors
%
\newcommand{\acap}[0]{\hat{\Ba}}
\newcommand{\bcap}[0]{\hat{\Bb}}
\newcommand{\ccap}[0]{\hat{\Bc}}
\newcommand{\dcap}[0]{\hat{\Bd}}
\newcommand{\ecap}[0]{\hat{\Be}}
\newcommand{\fcap}[0]{\hat{\Bf}}
\newcommand{\gcap}[0]{\hat{\Bg}}
\newcommand{\hcap}[0]{\hat{\Bh}}
\newcommand{\icap}[0]{\hat{\Bi}}
\newcommand{\jcap}[0]{\hat{\Bj}}
\newcommand{\kcap}[0]{\hat{\Bk}}
\newcommand{\lcap}[0]{\hat{\Bl}}
\newcommand{\mcap}[0]{\hat{\Bm}}
\newcommand{\ncap}[0]{\hat{\Bn}}
\newcommand{\ocap}[0]{\hat{\Bo}}
\newcommand{\pcap}[0]{\hat{\Bp}}
\newcommand{\qcap}[0]{\hat{\Bq}}
\newcommand{\rcap}[0]{\hat{\Br}}
\newcommand{\scap}[0]{\hat{\Bs}}
\newcommand{\tcap}[0]{\hat{\Bt}}
\newcommand{\ucap}[0]{\hat{\Bu}}
\newcommand{\vcap}[0]{\hat{\Bv}}
\newcommand{\wcap}[0]{\hat{\Bw}}
\newcommand{\xcap}[0]{\hat{\Bx}}
\newcommand{\ycap}[0]{\hat{\By}}
\newcommand{\zcap}[0]{\hat{\Bz}}
\newcommand{\thetacap}[0]{\hat{\Btheta}}

%
% to write R^n and C^n in a distinguishable fashion.  Perhaps change this
% to the double lined characters upon figuring out how to do so.
%
\newcommand{\C}[1]{$\mathbb{C}^{#1}$}
\newcommand{\R}[1]{$\mathbb{R}^{#1}$}

%
% various generally useful helpers
%

% derivative of #1 wrt. #2:
\newcommand{\D}[2] {\frac {d#2} {d#1}}

\newcommand{\inv}[1]{\frac{1}{#1}}
\newcommand{\cross}[0]{\times}

\newcommand{\abs}[1]{\lvert{#1}\rvert}
\newcommand{\norm}[1]{\lVert{#1}\rVert}
\newcommand{\innerprod}[2]{\langle{#1}, {#2}\rangle}
\newcommand{\dotprod}[2]{{#1} \cdot {#2}}
\newcommand{\bdotprod}[2]{\left({#1} \cdot {#2}\right)}
\newcommand{\crossprod}[2]{{#1} \cross {#2}}
\newcommand{\tripleprod}[3]{\dotprod{\left(\crossprod{#1}{#2}\right)}{#3}}

\DeclareMathOperator{\Proj}{Proj}
\DeclareMathOperator{\Span}{span}
\DeclareMathOperator{\Sgn}{sgn}
\DeclareMathOperator{\Area}{Area}
\DeclareMathOperator{\Volume}{Volume}

%
% A few miscellaneous things specific to this document
%
\newcommand{\crossop}[1]{\crossprod{#1}{}}

% R2 vector.
\newcommand{\VectorTwo}[2]{
\begin{bmatrix}
 {#1} \\
 {#2}
\end{bmatrix}
}

\newcommand{\VectorN}[1]{
\begin{bmatrix}
{#1}_1 \\
{#1}_2 \\
\vdots \\
{#1}_N \\
\end{bmatrix}
}

\newcommand{\DETuvij}[4]{
\begin{vmatrix}
 {#1}_{#3} & {#1}_{#4} \\
 {#2}_{#3} & {#2}_{#4}
\end{vmatrix}
}

\newcommand{\DETuvwijk}[6]{
\begin{vmatrix}
 {#1}_{#4} & {#1}_{#5} & {#1}_{#6} \\
 {#2}_{#4} & {#2}_{#5} & {#2}_{#6} \\
 {#3}_{#4} & {#3}_{#5} & {#3}_{#6}
\end{vmatrix}
}

\newcommand{\DETuvwxijkl}[8]{
\begin{vmatrix}
 {#1}_{#5} & {#1}_{#6} & {#1}_{#7} & {#1}_{#8} \\
 {#2}_{#5} & {#2}_{#6} & {#2}_{#7} & {#2}_{#8} \\
 {#3}_{#5} & {#3}_{#6} & {#3}_{#7} & {#3}_{#8} \\
 {#4}_{#5} & {#4}_{#6} & {#4}_{#7} & {#4}_{#8} \\
\end{vmatrix}
}

%\newcommand{\DETuvwxyijklm}[10]{
%\begin{vmatrix}
% {#1}_{#6} & {#1}_{#7} & {#1}_{#8} & {#1}_{#9} & {#1}_{#10} \\
% {#2}_{#6} & {#2}_{#7} & {#2}_{#8} & {#2}_{#9} & {#2}_{#10} \\
% {#3}_{#6} & {#3}_{#7} & {#3}_{#8} & {#3}_{#9} & {#3}_{#10} \\
% {#4}_{#6} & {#4}_{#7} & {#4}_{#8} & {#4}_{#9} & {#4}_{#10} \\
% {#5}_{#6} & {#5}_{#7} & {#5}_{#8} & {#5}_{#9} & {#5}_{#10}
%\end{vmatrix}
%}

% R3 vector.
\newcommand{\VectorThree}[3]{
\begin{bmatrix}
 {#1} \\
 {#2} \\
 {#3}
\end{bmatrix}
}


%<misc>
%
\newcommand{\Abs}[1]{{\left\lvert{#1}\right\rvert}}
\newcommand{\spacegrad}[0]{\boldsymbol{\nabla}}
\newcommand{\grad}[0]{\nabla}
\newcommand{\LL}[0]{\mathcal{L}}

% == \partial_{#1} {#2}
\newcommand{\PD}[2]{\frac{\partial {#2}}{\partial {#1}}}
% inline variant
\newcommand{\PDi}[2]{{\partial {#2}}/{\partial {#1}}}

\newcommand{\PDD}[3]{\frac{\partial^2 {#3}}{\partial {#1}\partial {#2}}}
%\newcommand{\PDd}[2]{\frac{\partial^2 {#2}}{{\partial{#1}}^2}}
\newcommand{\PDsq}[2]{\frac{\partial^2 {#2}}{(\partial {#1})^2}}

\newcommand{\Partial}[2]{\frac{\partial {#1}}{\partial {#2}}}
\DeclareMathOperator{\RejName}{Rej}
\newcommand{\Rej}[2]{\RejName_{#1}\left( {#2} \right)}
\newcommand{\Rm}[1]{\mathbb{R}^{#1}}
\newcommand{\Cm}[1]{\mathbb{C}^{#1}}
\newcommand{\conj}[0]{{*}}

%</misc>

% <grade selection>
%
\newcommand{\gpgrade}[2] {{\left\langle{{#1}}\right\rangle}_{#2}}

\newcommand{\gpgradezero}[1] {\gpgrade{#1}{}}
%\newcommand{\gpscalargrade}[1] {{\left\langle{{#1}}\right\rangle}}
%\newcommand{\gpgradezero}[1] {\gpgrade{#1}{0}}

%\newcommand{\gpgradeone}[1] {{\left\langle{{#1}}\right\rangle}_{1}}
\newcommand{\gpgradeone}[1] {\gpgrade{#1}{1}}

\newcommand{\gpgradetwo}[1] {\gpgrade{#1}{2}}
\newcommand{\gpgradethree}[1] {\gpgrade{#1}{3}}
\newcommand{\gpgradefour}[1] {\gpgrade{#1}{4}}
%
% </grade selection>



\newcommand{\adot}[0]{{\dot{a}}}
\newcommand{\bdot}[0]{{\dot{b}}}
% taken for centered dot:
%\newcommand{\cdot}[0]{{\dot{c}}}
%\newcommand{\ddot}[0]{{\dot{d}}}
\newcommand{\edot}[0]{{\dot{e}}}
\newcommand{\fdot}[0]{{\dot{f}}}
\newcommand{\gdot}[0]{{\dot{g}}}
\newcommand{\hdot}[0]{{\dot{h}}}
\newcommand{\idot}[0]{{\dot{i}}}
\newcommand{\jdot}[0]{{\dot{j}}}
\newcommand{\kdot}[0]{{\dot{k}}}
\newcommand{\ldot}[0]{{\dot{l}}}
\newcommand{\mdot}[0]{{\dot{m}}}
\newcommand{\ndot}[0]{{\dot{n}}}
%\newcommand{\odot}[0]{{\dot{o}}}
\newcommand{\pdot}[0]{{\dot{p}}}
\newcommand{\qdot}[0]{{\dot{q}}}
\newcommand{\rdot}[0]{{\dot{r}}}
\newcommand{\sdot}[0]{{\dot{s}}}
\newcommand{\tdot}[0]{{\dot{t}}}
\newcommand{\udot}[0]{{\dot{u}}}
\newcommand{\vdot}[0]{{\dot{v}}}
\newcommand{\wdot}[0]{{\dot{w}}}
\newcommand{\xdot}[0]{{\dot{x}}}
\newcommand{\ydot}[0]{{\dot{y}}}
\newcommand{\zdot}[0]{{\dot{z}}}
\newcommand{\addot}[0]{{\ddot{a}}}
\newcommand{\bddot}[0]{{\ddot{b}}}
\newcommand{\cddot}[0]{{\ddot{c}}}
%\newcommand{\dddot}[0]{{\ddot{d}}}
\newcommand{\eddot}[0]{{\ddot{e}}}
\newcommand{\fddot}[0]{{\ddot{f}}}
\newcommand{\gddot}[0]{{\ddot{g}}}
\newcommand{\hddot}[0]{{\ddot{h}}}
\newcommand{\iddot}[0]{{\ddot{i}}}
\newcommand{\jddot}[0]{{\ddot{j}}}
\newcommand{\kddot}[0]{{\ddot{k}}}
\newcommand{\lddot}[0]{{\ddot{l}}}
\newcommand{\mddot}[0]{{\ddot{m}}}
\newcommand{\nddot}[0]{{\ddot{n}}}
\newcommand{\oddot}[0]{{\ddot{o}}}
\newcommand{\pddot}[0]{{\ddot{p}}}
\newcommand{\qddot}[0]{{\ddot{q}}}
\newcommand{\rddot}[0]{{\ddot{r}}}
\newcommand{\sddot}[0]{{\ddot{s}}}
\newcommand{\tddot}[0]{{\ddot{t}}}
\newcommand{\uddot}[0]{{\ddot{u}}}
\newcommand{\vddot}[0]{{\ddot{v}}}
\newcommand{\wddot}[0]{{\ddot{w}}}
\newcommand{\xddot}[0]{{\ddot{x}}}
\newcommand{\yddot}[0]{{\ddot{y}}}
\newcommand{\zddot}[0]{{\ddot{z}}}

%<bold and dot greek symbols>
%

\newcommand{\Deltadot}[0]{{\dot{\Delta}}}
\newcommand{\Gammadot}[0]{{\dot{\Gamma}}}
\newcommand{\Lambdadot}[0]{{\dot{\Lambda}}}
\newcommand{\Omegadot}[0]{{\dot{\Omega}}}
\newcommand{\Phidot}[0]{{\dot{\Phi}}}
\newcommand{\Pidot}[0]{{\dot{\Pi}}}
\newcommand{\Psidot}[0]{{\dot{\Psi}}}
\newcommand{\Sigmadot}[0]{{\dot{\Sigma}}}
\newcommand{\Thetadot}[0]{{\dot{\Theta}}}
\newcommand{\Upsilondot}[0]{{\dot{\Upsilon}}}
\newcommand{\Xidot}[0]{{\dot{\Xi}}}
\newcommand{\alphadot}[0]{{\dot{\alpha}}}
\newcommand{\betadot}[0]{{\dot{\beta}}}
\newcommand{\chidot}[0]{{\dot{\chi}}}
\newcommand{\deltadot}[0]{{\dot{\delta}}}
\newcommand{\epsilondot}[0]{{\dot{\epsilon}}}
\newcommand{\etadot}[0]{{\dot{\eta}}}
\newcommand{\gammadot}[0]{{\dot{\gamma}}}
\newcommand{\kappadot}[0]{{\dot{\kappa}}}
\newcommand{\lambdadot}[0]{{\dot{\lambda}}}
\newcommand{\mudot}[0]{{\dot{\mu}}}
\newcommand{\nudot}[0]{{\dot{\nu}}}
\newcommand{\omegadot}[0]{{\dot{\omega}}}
\newcommand{\phidot}[0]{{\dot{\phi}}}
\newcommand{\pidot}[0]{{\dot{\pi}}}
\newcommand{\psidot}[0]{{\dot{\psi}}}
\newcommand{\rhodot}[0]{{\dot{\rho}}}
\newcommand{\sigmadot}[0]{{\dot{\sigma}}}
\newcommand{\taudot}[0]{{\dot{\tau}}}
\newcommand{\thetadot}[0]{{\dot{\theta}}}
\newcommand{\upsilondot}[0]{{\dot{\upsilon}}}
\newcommand{\varepsilondot}[0]{{\dot{\varepsilon}}}
\newcommand{\varphidot}[0]{{\dot{\varphi}}}
\newcommand{\varpidot}[0]{{\dot{\varpi}}}
\newcommand{\varrhodot}[0]{{\dot{\varrho}}}
\newcommand{\varsigmadot}[0]{{\dot{\varsigma}}}
\newcommand{\varthetadot}[0]{{\dot{\vartheta}}}
\newcommand{\xidot}[0]{{\dot{\xi}}}
\newcommand{\zetadot}[0]{{\dot{\zeta}}}

\newcommand{\Deltaddot}[0]{{\ddot{\Delta}}}
\newcommand{\Gammaddot}[0]{{\ddot{\Gamma}}}
\newcommand{\Lambdaddot}[0]{{\ddot{\Lambda}}}
\newcommand{\Omegaddot}[0]{{\ddot{\Omega}}}
\newcommand{\Phiddot}[0]{{\ddot{\Phi}}}
\newcommand{\Piddot}[0]{{\ddot{\Pi}}}
\newcommand{\Psiddot}[0]{{\ddot{\Psi}}}
\newcommand{\Sigmaddot}[0]{{\ddot{\Sigma}}}
\newcommand{\Thetaddot}[0]{{\ddot{\Theta}}}
\newcommand{\Upsilonddot}[0]{{\ddot{\Upsilon}}}
\newcommand{\Xiddot}[0]{{\ddot{\Xi}}}
\newcommand{\alphaddot}[0]{{\ddot{\alpha}}}
\newcommand{\betaddot}[0]{{\ddot{\beta}}}
\newcommand{\chiddot}[0]{{\ddot{\chi}}}
\newcommand{\deltaddot}[0]{{\ddot{\delta}}}
\newcommand{\epsilonddot}[0]{{\ddot{\epsilon}}}
\newcommand{\etaddot}[0]{{\ddot{\eta}}}
\newcommand{\gammaddot}[0]{{\ddot{\gamma}}}
\newcommand{\kappaddot}[0]{{\ddot{\kappa}}}
\newcommand{\lambdaddot}[0]{{\ddot{\lambda}}}
\newcommand{\muddot}[0]{{\ddot{\mu}}}
\newcommand{\nuddot}[0]{{\ddot{\nu}}}
\newcommand{\omegaddot}[0]{{\ddot{\omega}}}
\newcommand{\phiddot}[0]{{\ddot{\phi}}}
\newcommand{\piddot}[0]{{\ddot{\pi}}}
\newcommand{\psiddot}[0]{{\ddot{\psi}}}
\newcommand{\rhoddot}[0]{{\ddot{\rho}}}
\newcommand{\sigmaddot}[0]{{\ddot{\sigma}}}
\newcommand{\tauddot}[0]{{\ddot{\tau}}}
\newcommand{\thetaddot}[0]{{\ddot{\theta}}}
\newcommand{\upsilonddot}[0]{{\ddot{\upsilon}}}
\newcommand{\varepsilonddot}[0]{{\ddot{\varepsilon}}}
\newcommand{\varphiddot}[0]{{\ddot{\varphi}}}
\newcommand{\varpiddot}[0]{{\ddot{\varpi}}}
\newcommand{\varrhoddot}[0]{{\ddot{\varrho}}}
\newcommand{\varsigmaddot}[0]{{\ddot{\varsigma}}}
\newcommand{\varthetaddot}[0]{{\ddot{\vartheta}}}
\newcommand{\xiddot}[0]{{\ddot{\xi}}}
\newcommand{\zetaddot}[0]{{\ddot{\zeta}}}

\newcommand{\BDelta}[0]{\boldsymbol{\Delta}}
\newcommand{\BGamma}[0]{\boldsymbol{\Gamma}}
\newcommand{\BLambda}[0]{\boldsymbol{\Lambda}}
\newcommand{\BOmega}[0]{\boldsymbol{\Omega}}
\newcommand{\BPhi}[0]{\boldsymbol{\Phi}}
\newcommand{\BPi}[0]{\boldsymbol{\Pi}}
\newcommand{\BPsi}[0]{\boldsymbol{\Psi}}
\newcommand{\BSigma}[0]{\boldsymbol{\Sigma}}
\newcommand{\BTheta}[0]{\boldsymbol{\Theta}}
\newcommand{\BUpsilon}[0]{\boldsymbol{\Upsilon}}
\newcommand{\BXi}[0]{\boldsymbol{\Xi}}
\newcommand{\Balpha}[0]{\boldsymbol{\alpha}}
\newcommand{\Bbeta}[0]{\boldsymbol{\beta}}
\newcommand{\Bchi}[0]{\boldsymbol{\chi}}
\newcommand{\Bdelta}[0]{\boldsymbol{\delta}}
\newcommand{\Bepsilon}[0]{\boldsymbol{\epsilon}}
\newcommand{\Beta}[0]{\boldsymbol{\eta}}
\newcommand{\Bgamma}[0]{\boldsymbol{\gamma}}
\newcommand{\Bkappa}[0]{\boldsymbol{\kappa}}
\newcommand{\Blambda}[0]{\boldsymbol{\lambda}}
\newcommand{\Bmu}[0]{\boldsymbol{\mu}}
\newcommand{\Bnu}[0]{\boldsymbol{\nu}}
%\newcommand{\Bomega}[0]{\boldsymbol{\omega}}
\newcommand{\Bphi}[0]{\boldsymbol{\phi}}
\newcommand{\Bpi}[0]{\boldsymbol{\pi}}
\newcommand{\Bpsi}[0]{\boldsymbol{\psi}}
\newcommand{\Brho}[0]{\boldsymbol{\rho}}
\newcommand{\Bsigma}[0]{\boldsymbol{\sigma}}
%\newcommand{\Btau}[0]{\boldsymbol{\tau}}
%\newcommand{\Btheta}[0]{\boldsymbol{\theta}}
\newcommand{\Bupsilon}[0]{\boldsymbol{\upsilon}}
\newcommand{\Bvarepsilon}[0]{\boldsymbol{\varepsilon}}
\newcommand{\Bvarphi}[0]{\boldsymbol{\varphi}}
\newcommand{\Bvarpi}[0]{\boldsymbol{\varpi}}
\newcommand{\Bvarrho}[0]{\boldsymbol{\varrho}}
\newcommand{\Bvarsigma}[0]{\boldsymbol{\varsigma}}
\newcommand{\Bvartheta}[0]{\boldsymbol{\vartheta}}
\newcommand{\Bxi}[0]{\boldsymbol{\xi}}
\newcommand{\Bzeta}[0]{\boldsymbol{\zeta}}
%
%</bold and dot greek symbols>
%<infrequent>
%
%\newcommand{\AreaOp}[1]{\AName_{#1}}
%\newcommand{\Babs}[0]{\abs{\BB}}
%\newcommand{\Bcap}[0]{\hat{\BB}}
%\newcommand{\BrPrimeRej}[0]{\rcap(\rcap \wedge \Br')}
%\newcommand{\CA}[0]{\mathcal{A}}
%\newcommand{\Cos}[1]{\cos{\left({#1}\right)}}
%\newcommand{\Det}[1] {\abs{#1}}
%\newcommand{\Dsq}[2] {\frac {\partial^2 {#1}} {\partial {#2}^2}}
%\newcommand{\Exp}[1]{\exp{\left({#1}\right)}}
%\newcommand{\Norm}[1]{\left\lVert{#1}\right\rVert}
%\newcommand{\Sin}[1]{\sin{\left({#1}\right)}}
%\newcommand{\T}[0]{\text{T}}
%\newcommand{\VolumeOp}[1]{\VName_{#1}}
%\newcommand{\agrad}[0]{\Ba \cdot \nabla}
%\newcommand{\alphacap}[0]{\hat{\boldsymbol{\alpha}}}
%\newcommand{\Fcap}[0]{\hat{\BF}}
%\newcommand{\bithree}[0]{{\Bi}_3}
%\newcommand{\bxa}[0]{\Bx\Ba}
%\newcommand{\coordvec}[2]{
%\newcommand{\costheta}[0]{\acap \cdot \xcap}
%\newcommand{\ddt}[1]{\ddot{#1}}
%\newcommand{\ddu}[1] {\frac {d{#1}} {du}}
%\newcommand{\dsqxj}[2] {\frac {\partial^2 {#1}} {\partial {x_{#2}}^2}}
%\newcommand{\dtheta}[1]{\frac{d {#1}}{d \theta}}
%\newcommand{\dt}[1]{\dot{#1}}
%\newcommand{\dt}[1]{\frac{d {#1}}{dt}}
%\newcommand{\dxj}[2] {\frac {\partial {#1}} {\partial {x_{#2}}}}
%\newcommand{\halfPhi}[0]{\frac{\phi}{2}}
%\newcommand{\half}[0]{\inv{2}}
%\newcommand{\inv}[1]{\frac{1}{#1}}
%\newcommand{\laplacian}[0]{\nabla^2}
%\newcommand{\matrixoftx}[3]{
%\newcommand{\nrrp}[0]{\norm{\rcap \wedge \Br'}}
%\newcommand{\oiint}{\bigcirc \hspace{-1.4em} \int \hspace{-.8em} \int}
%\newcommand{\transpose}[1]{{#1}^{\text{T}}}
%\newcommand{\transpose}[1]{{{#1}^{\TextTranspose}}}
%\newcommand{\transpose}[1]{{{#1}^{\text{T}}}}
%\newcommand{\barA}[0]{\bar{A}}
%\newcommand{\qbar}[0]{\bar{q}}
%\newcommand{\qdotbar}[0]{\dot{\bar{q}}}
%
%</infrequent>





%\usepackage{listings}
\usepackage{txfonts} % for ointctr... (also appears to make "prettier" \int and \sum's)
\usepackage[bookmarks=true]{hyperref}

\usepackage{color,cite,graphicx}
   % use colour in the document, put your citations as [1-4]
   % rather than [1,2,3,4] (it looks nicer, and the extended LaTeX2e
   % graphics package. 
\usepackage{latexsym,amssymb,epsf} % don't remember if these are
   % needed, but their inclusion can't do any damage


\title{ Worked calculus of variations problems from Byron and Fuller. }
\author{Peeter Joot \quad peeter.joot@gmail.com }
\date{ March 21, 2009.  Last Revision: $Date: 2009/03/22 05:32:43 $ }

\begin{document}

\maketitle{}
\tableofcontents

\section{ Worked calculus of variations problems. }

Select problems from chapter II of \cite{byron1992mca}.

\subsection{ Problem 1.  Shortest line between points in polar coordinates. }

Problem.  Variational calculus exersize to find shortest distance between two points using polar coordinates.
 
The line element is:
 
\begin{align*}
ds^2 = r^2 d\theta^2 + {r'}^2
\end{align*}
 
So, the integral to minimize is
 
\begin{align*}
I = \int \sqrt{ r^2 + {r'}^2 } d\theta
\end{align*}
 
Application of the Euler-Lagrange equations yields
 
\begin{align*}
0 
&= \left( \PD{r}{} - \frac{d}{d\theta} \PD{r'}{} \right) \sqrt{ r^2 + {r'}^2 }  \\
&= \frac{r}{\sqrt{{r'}^2 + r^2}} - \frac{d}{d\theta} \left( \frac{{r'}}{\sqrt{{r'}^2 + r^2}}\right) \\
\end{align*}
 
Dividing through by $r$ and writing $v = u' = {r'}/r$ this is
 
\begin{align*}
\frac{1}{\sqrt{v^2 + 1}}
&= \frac{d}{d\theta} \left( \frac{v}{\sqrt{v^2 + 1}}\right) \\
&= \frac{v'}{\sqrt{v^2 + 1}} -\frac{v^2 v'}{(\sqrt{v^2 + 1})^3} \\
\end{align*}
 
\begin{align*}
1 
&= v' \left( 1 -\frac{v^2 }{v^2 + 1} \right) \\
&= \frac{v'}{v^2 + 1} \\
\end{align*}

This is now separable, and can be integrated directly

\begin{align*}
\theta - \theta_0
&= \int \frac{dv}{v^2 + 1} \\
&= \arctan(v) \\
\end{align*}
 
\begin{align*}
\tan(\theta - \theta_0)
&= \frac{{r'}}{r} \\
&= \frac{d \ln(r) }{d\theta}
\end{align*}

That solves the first of the second order differential equations resulting from the Euler-Lagrange equations, and the 
last becomes
\begin{align*}
\ln(r) &= \int \tan(\theta - \theta_0) d\theta \\
&= -\ln(\cos(\theta - \theta_0)) + \ln(r_0)
\end{align*}

Finally a polar parametric equation is obtained
 
\begin{align*}
\frac{r}{r_0} \cos(\theta - \theta_0) = 1
\end{align*}
 
If all went right, this should be the equation for a straight line in polar form.

It doesn't look like one, but if the cosine is expanded 

\begin{align*}
\cos(\theta - \theta_0) 
&= \Re\left( e^{i\theta}e^{-i\theta_0} \right) \\
&= \Re\left( (\cos\theta + i\sin\theta)(\cos\theta_0 - i\sin\theta_0) \right) \\
&= \cos\theta\cos\theta_0 + \sin\theta\sin\theta_0 \\
\end{align*}

With $x = r\cos\theta$, and $y = r\sin\theta$ this gives

\begin{align*}
r_0 
&= r \left( \cos\theta\cos\theta_0 + \sin\theta\sin\theta_0 \right) \\
&= x \cos\theta_0 + y\sin\theta_0 \\
\end{align*}

So, sure enough, following the math gives an equation for a straight line in a recognizable form.

\subsection{ Problem 2. Shortest line, in 3D. }

Did this one in \cite{PJgoldch1}

\subsection{ Problem 3.  Spherical geodesics. }

First calculate the line element (this was given in the problem, but I feel like working it out).
The position vector with $i = \Be_1 \Be_2$, is given by

\begin{align*}
\Br = a ( \sin\theta \Be_1 e^{i\phi} + \Be_3 \cos\theta )
\end{align*}

So, the differential given constant radius $a$ is

\begin{align*}
d\Br 
&= 
a \thetadot ( \cos\theta \Be_1 e^{i\phi} - \Be_3 \sin\theta )
+ a \phidot ( \sin\theta \Be_1 \Be_1 \Be_2 e^{i\phi} ) \\
&= 
a \left( \thetadot \cos\theta \Be_1 + \phidot \sin\theta \Be_2 \right) e^{i\phi} - a \thetadot \Be_3 \sin\theta \\
\end{align*}

And the square is
\begin{align*}
d\Br^2
&= a^2 \left( \thetadot^2 \cos^2\theta + \phidot^2 \sin^2\theta + \thetadot^2 \sin^2\theta \right) \\
&= a^2 \left( \thetadot^2 + \phidot^2 \sin^2\theta \right) \\
\end{align*}

Here the derivatives are with respect to some implicit variable that parameterizes the differential displacement.
This can be taken to be $\theta$, which gives the distance along any two points on the sphere as

\begin{align*}
S &= a^2 \int d\theta \sqrt{1 + \left(\frac{d\phi}{d\theta}\right)^2 \sin^2\theta } \\
\end{align*}

Writing $f(\theta, \phi, \phidot) = \sqrt{1 + \phidot^2 \sin^2\theta}$, the Euler-Lagrange equations can be
applied

\begin{align*}
0 
&= \left( \PD{\phi}{} - \frac{d}{d\theta} \PD{\phidot}{} \right) f \\
&= 0 - \frac{d}{d\theta} \frac{(1/2)(2\phidot) \sin^2\theta}{\sqrt{1 + \phidot^2 \sin^2\theta}} \\
\end{align*}

Introducing an integration constant $\kappa$, this is

\begin{align*}
\phidot \sin^2\theta = \kappa \sqrt{1 + \phidot^2 \sin^2\theta}
\end{align*}

squaring
\begin{align*}
\phidot^2 \sin^4\theta &= \kappa^2 \left(1 + \phidot^2 \sin^2\theta \right) \\
\phidot^2 \sin^2\theta \left( \sin^2 \theta - \kappa^2 \right) &= \kappa^2 \\
\end{align*}

\begin{align*}
\phi - \phi_0 &= \kappa \int \frac{d\theta}{\sin\theta \sqrt{ \sin^2 \theta - \kappa^2 }} \\
\end{align*}

This doesn't look particularily nice to integrate.  Instead, let's try writing the arc length integral as

\begin{align*}
\frac{S}{a^2} &= \int d\phi \sqrt{{\frac{d\theta}{d\phi}}^2 + \sin^2\theta } \\
\end{align*}

\begin{align*}
0 
&= \left( \PD{\theta}{} - \frac{d}{d\phi} \PD{\thetadot}{} \right) \sqrt{\thetadot^2 + \sin^2\theta } \\
&= \frac{\sin\theta \cos\theta}{ \sqrt{\thetadot^2 + \sin^2\theta } } -
\frac{d}{d\phi} \frac{\thetadot} { \sqrt{\thetadot^2 + \sin^2\theta } } \\
\end{align*}

\bibliographystyle{plainnat}
\bibliography{myrefs}

\end{document}

%
% Copyright � 2012 Peeter Joot.  All Rights Reserved.
% Licenced as described in the file LICENSE under the root directory of this GIT repository.
%

% 
% 
%\documentclass{article}

%\usepackage{amsmath}
\usepackage{mathpazo}

%
% shorthand for bold symbols, convenient for vectors and matrices
%
\newcommand{\Ba}[0]{\mathbf{a}}
\newcommand{\Bb}[0]{\mathbf{b}}
\newcommand{\Bc}[0]{\mathbf{c}}
\newcommand{\Bd}[0]{\mathbf{d}}
\newcommand{\Be}[0]{\mathbf{e}}
\newcommand{\Bf}[0]{\mathbf{f}}
\newcommand{\Bg}[0]{\mathbf{g}}
\newcommand{\Bh}[0]{\mathbf{h}}
\newcommand{\Bi}[0]{\mathbf{i}}
\newcommand{\Bj}[0]{\mathbf{j}}
\newcommand{\Bk}[0]{\mathbf{k}}
\newcommand{\Bl}[0]{\mathbf{l}}
\newcommand{\Bm}[0]{\mathbf{m}}
\newcommand{\Bn}[0]{\mathbf{n}}
\newcommand{\Bo}[0]{\mathbf{o}}
\newcommand{\Bp}[0]{\mathbf{p}}
\newcommand{\Bq}[0]{\mathbf{q}}
\newcommand{\Br}[0]{\mathbf{r}}
\newcommand{\Bs}[0]{\mathbf{s}}
\newcommand{\Bt}[0]{\mathbf{t}}
\newcommand{\Bu}[0]{\mathbf{u}}
\newcommand{\Bv}[0]{\mathbf{v}}
\newcommand{\Bw}[0]{\mathbf{w}}
\newcommand{\Bx}[0]{\mathbf{x}}
\newcommand{\By}[0]{\mathbf{y}}
\newcommand{\Bz}[0]{\mathbf{z}}
\newcommand{\BA}[0]{\mathbf{A}}
\newcommand{\BB}[0]{\mathbf{B}}
\newcommand{\BC}[0]{\mathbf{C}}
\newcommand{\BD}[0]{\mathbf{D}}
\newcommand{\BE}[0]{\mathbf{E}}
\newcommand{\BF}[0]{\mathbf{F}}
\newcommand{\BG}[0]{\mathbf{G}}
\newcommand{\BH}[0]{\mathbf{H}}
\newcommand{\BI}[0]{\mathbf{I}}
\newcommand{\BJ}[0]{\mathbf{J}}
\newcommand{\BK}[0]{\mathbf{K}}
\newcommand{\BL}[0]{\mathbf{L}}
\newcommand{\BM}[0]{\mathbf{M}}
\newcommand{\BN}[0]{\mathbf{N}}
\newcommand{\BO}[0]{\mathbf{O}}
\newcommand{\BP}[0]{\mathbf{P}}
\newcommand{\BQ}[0]{\mathbf{Q}}
\newcommand{\BR}[0]{\mathbf{R}}
\newcommand{\BS}[0]{\mathbf{S}}
\newcommand{\BT}[0]{\mathbf{T}}
\newcommand{\BU}[0]{\mathbf{U}}
\newcommand{\BV}[0]{\mathbf{V}}
\newcommand{\BW}[0]{\mathbf{W}}
\newcommand{\BX}[0]{\mathbf{X}}
\newcommand{\BY}[0]{\mathbf{Y}}
\newcommand{\BZ}[0]{\mathbf{Z}}

\newcommand{\Bzero}[0]{\mathbf{0}}
\newcommand{\Btheta}[0]{\boldsymbol{\theta}}
\newcommand{\Btau}[0]{\boldsymbol{\tau}}
\newcommand{\Bomega}[0]{\boldsymbol{\omega}}

%
% shorthand for unit vectors
%
\newcommand{\acap}[0]{\hat{\Ba}}
\newcommand{\bcap}[0]{\hat{\Bb}}
\newcommand{\ccap}[0]{\hat{\Bc}}
\newcommand{\dcap}[0]{\hat{\Bd}}
\newcommand{\ecap}[0]{\hat{\Be}}
\newcommand{\fcap}[0]{\hat{\Bf}}
\newcommand{\gcap}[0]{\hat{\Bg}}
\newcommand{\hcap}[0]{\hat{\Bh}}
\newcommand{\icap}[0]{\hat{\Bi}}
\newcommand{\jcap}[0]{\hat{\Bj}}
\newcommand{\kcap}[0]{\hat{\Bk}}
\newcommand{\lcap}[0]{\hat{\Bl}}
\newcommand{\mcap}[0]{\hat{\Bm}}
\newcommand{\ncap}[0]{\hat{\Bn}}
\newcommand{\ocap}[0]{\hat{\Bo}}
\newcommand{\pcap}[0]{\hat{\Bp}}
\newcommand{\qcap}[0]{\hat{\Bq}}
\newcommand{\rcap}[0]{\hat{\Br}}
\newcommand{\scap}[0]{\hat{\Bs}}
\newcommand{\tcap}[0]{\hat{\Bt}}
\newcommand{\ucap}[0]{\hat{\Bu}}
\newcommand{\vcap}[0]{\hat{\Bv}}
\newcommand{\wcap}[0]{\hat{\Bw}}
\newcommand{\xcap}[0]{\hat{\Bx}}
\newcommand{\ycap}[0]{\hat{\By}}
\newcommand{\zcap}[0]{\hat{\Bz}}
\newcommand{\thetacap}[0]{\hat{\Btheta}}

%
% to write R^n and C^n in a distinguishable fashion.  Perhaps change this
% to the double lined characters upon figuring out how to do so.
%
\newcommand{\C}[1]{$\mathbb{C}^{#1}$}
\newcommand{\R}[1]{$\mathbb{R}^{#1}$}

%
% various generally useful helpers
%

% derivative of #1 wrt. #2:
\newcommand{\D}[2] {\frac {d#2} {d#1}}

\newcommand{\inv}[1]{\frac{1}{#1}}
\newcommand{\cross}[0]{\times}

\newcommand{\abs}[1]{\lvert{#1}\rvert}
\newcommand{\norm}[1]{\lVert{#1}\rVert}
\newcommand{\innerprod}[2]{\langle{#1}, {#2}\rangle}
\newcommand{\dotprod}[2]{{#1} \cdot {#2}}
\newcommand{\bdotprod}[2]{\left({#1} \cdot {#2}\right)}
\newcommand{\crossprod}[2]{{#1} \cross {#2}}
\newcommand{\tripleprod}[3]{\dotprod{\left(\crossprod{#1}{#2}\right)}{#3}}

\DeclareMathOperator{\Proj}{Proj}
\DeclareMathOperator{\Span}{span}
\DeclareMathOperator{\Sgn}{sgn}
\DeclareMathOperator{\Area}{Area}
\DeclareMathOperator{\Volume}{Volume}

%
% A few miscellaneous things specific to this document
%
\newcommand{\crossop}[1]{\crossprod{#1}{}}

% R2 vector.
\newcommand{\VectorTwo}[2]{
\begin{bmatrix}
 {#1} \\
 {#2}
\end{bmatrix}
}

\newcommand{\VectorN}[1]{
\begin{bmatrix}
{#1}_1 \\
{#1}_2 \\
\vdots \\
{#1}_N \\
\end{bmatrix}
}

\newcommand{\DETuvij}[4]{
\begin{vmatrix}
 {#1}_{#3} & {#1}_{#4} \\
 {#2}_{#3} & {#2}_{#4}
\end{vmatrix}
}

\newcommand{\DETuvwijk}[6]{
\begin{vmatrix}
 {#1}_{#4} & {#1}_{#5} & {#1}_{#6} \\
 {#2}_{#4} & {#2}_{#5} & {#2}_{#6} \\
 {#3}_{#4} & {#3}_{#5} & {#3}_{#6}
\end{vmatrix}
}

\newcommand{\DETuvwxijkl}[8]{
\begin{vmatrix}
 {#1}_{#5} & {#1}_{#6} & {#1}_{#7} & {#1}_{#8} \\
 {#2}_{#5} & {#2}_{#6} & {#2}_{#7} & {#2}_{#8} \\
 {#3}_{#5} & {#3}_{#6} & {#3}_{#7} & {#3}_{#8} \\
 {#4}_{#5} & {#4}_{#6} & {#4}_{#7} & {#4}_{#8} \\
\end{vmatrix}
}

%\newcommand{\DETuvwxyijklm}[10]{
%\begin{vmatrix}
% {#1}_{#6} & {#1}_{#7} & {#1}_{#8} & {#1}_{#9} & {#1}_{#10} \\
% {#2}_{#6} & {#2}_{#7} & {#2}_{#8} & {#2}_{#9} & {#2}_{#10} \\
% {#3}_{#6} & {#3}_{#7} & {#3}_{#8} & {#3}_{#9} & {#3}_{#10} \\
% {#4}_{#6} & {#4}_{#7} & {#4}_{#8} & {#4}_{#9} & {#4}_{#10} \\
% {#5}_{#6} & {#5}_{#7} & {#5}_{#8} & {#5}_{#9} & {#5}_{#10}
%\end{vmatrix}
%}

% R3 vector.
\newcommand{\VectorThree}[3]{
\begin{bmatrix}
 {#1} \\
 {#2} \\
 {#3}
\end{bmatrix}
}


%%<misc>
%
\newcommand{\Abs}[1]{{\left\lvert{#1}\right\rvert}}
\newcommand{\spacegrad}[0]{\boldsymbol{\nabla}}
\newcommand{\grad}[0]{\nabla}
\newcommand{\LL}[0]{\mathcal{L}}

% == \partial_{#1} {#2}
\newcommand{\PD}[2]{\frac{\partial {#2}}{\partial {#1}}}
% inline variant
\newcommand{\PDi}[2]{{\partial {#2}}/{\partial {#1}}}

\newcommand{\PDD}[3]{\frac{\partial^2 {#3}}{\partial {#1}\partial {#2}}}
%\newcommand{\PDd}[2]{\frac{\partial^2 {#2}}{{\partial{#1}}^2}}
\newcommand{\PDsq}[2]{\frac{\partial^2 {#2}}{(\partial {#1})^2}}

\newcommand{\Partial}[2]{\frac{\partial {#1}}{\partial {#2}}}
\DeclareMathOperator{\RejName}{Rej}
\newcommand{\Rej}[2]{\RejName_{#1}\left( {#2} \right)}
\newcommand{\Rm}[1]{\mathbb{R}^{#1}}
\newcommand{\Cm}[1]{\mathbb{C}^{#1}}
\newcommand{\conj}[0]{{*}}

%</misc>

% <grade selection>
%
\newcommand{\gpgrade}[2] {{\left\langle{{#1}}\right\rangle}_{#2}}

\newcommand{\gpgradezero}[1] {\gpgrade{#1}{}}
%\newcommand{\gpscalargrade}[1] {{\left\langle{{#1}}\right\rangle}}
%\newcommand{\gpgradezero}[1] {\gpgrade{#1}{0}}

%\newcommand{\gpgradeone}[1] {{\left\langle{{#1}}\right\rangle}_{1}}
\newcommand{\gpgradeone}[1] {\gpgrade{#1}{1}}

\newcommand{\gpgradetwo}[1] {\gpgrade{#1}{2}}
\newcommand{\gpgradethree}[1] {\gpgrade{#1}{3}}
\newcommand{\gpgradefour}[1] {\gpgrade{#1}{4}}
%
% </grade selection>



\newcommand{\adot}[0]{{\dot{a}}}
\newcommand{\bdot}[0]{{\dot{b}}}
% taken for centered dot:
%\newcommand{\cdot}[0]{{\dot{c}}}
%\newcommand{\ddot}[0]{{\dot{d}}}
\newcommand{\edot}[0]{{\dot{e}}}
\newcommand{\fdot}[0]{{\dot{f}}}
\newcommand{\gdot}[0]{{\dot{g}}}
\newcommand{\hdot}[0]{{\dot{h}}}
\newcommand{\idot}[0]{{\dot{i}}}
\newcommand{\jdot}[0]{{\dot{j}}}
\newcommand{\kdot}[0]{{\dot{k}}}
\newcommand{\ldot}[0]{{\dot{l}}}
\newcommand{\mdot}[0]{{\dot{m}}}
\newcommand{\ndot}[0]{{\dot{n}}}
%\newcommand{\odot}[0]{{\dot{o}}}
\newcommand{\pdot}[0]{{\dot{p}}}
\newcommand{\qdot}[0]{{\dot{q}}}
\newcommand{\rdot}[0]{{\dot{r}}}
\newcommand{\sdot}[0]{{\dot{s}}}
\newcommand{\tdot}[0]{{\dot{t}}}
\newcommand{\udot}[0]{{\dot{u}}}
\newcommand{\vdot}[0]{{\dot{v}}}
\newcommand{\wdot}[0]{{\dot{w}}}
\newcommand{\xdot}[0]{{\dot{x}}}
\newcommand{\ydot}[0]{{\dot{y}}}
\newcommand{\zdot}[0]{{\dot{z}}}
\newcommand{\addot}[0]{{\ddot{a}}}
\newcommand{\bddot}[0]{{\ddot{b}}}
\newcommand{\cddot}[0]{{\ddot{c}}}
%\newcommand{\dddot}[0]{{\ddot{d}}}
\newcommand{\eddot}[0]{{\ddot{e}}}
\newcommand{\fddot}[0]{{\ddot{f}}}
\newcommand{\gddot}[0]{{\ddot{g}}}
\newcommand{\hddot}[0]{{\ddot{h}}}
\newcommand{\iddot}[0]{{\ddot{i}}}
\newcommand{\jddot}[0]{{\ddot{j}}}
\newcommand{\kddot}[0]{{\ddot{k}}}
\newcommand{\lddot}[0]{{\ddot{l}}}
\newcommand{\mddot}[0]{{\ddot{m}}}
\newcommand{\nddot}[0]{{\ddot{n}}}
\newcommand{\oddot}[0]{{\ddot{o}}}
\newcommand{\pddot}[0]{{\ddot{p}}}
\newcommand{\qddot}[0]{{\ddot{q}}}
\newcommand{\rddot}[0]{{\ddot{r}}}
\newcommand{\sddot}[0]{{\ddot{s}}}
\newcommand{\tddot}[0]{{\ddot{t}}}
\newcommand{\uddot}[0]{{\ddot{u}}}
\newcommand{\vddot}[0]{{\ddot{v}}}
\newcommand{\wddot}[0]{{\ddot{w}}}
\newcommand{\xddot}[0]{{\ddot{x}}}
\newcommand{\yddot}[0]{{\ddot{y}}}
\newcommand{\zddot}[0]{{\ddot{z}}}

%<bold and dot greek symbols>
%

\newcommand{\Deltadot}[0]{{\dot{\Delta}}}
\newcommand{\Gammadot}[0]{{\dot{\Gamma}}}
\newcommand{\Lambdadot}[0]{{\dot{\Lambda}}}
\newcommand{\Omegadot}[0]{{\dot{\Omega}}}
\newcommand{\Phidot}[0]{{\dot{\Phi}}}
\newcommand{\Pidot}[0]{{\dot{\Pi}}}
\newcommand{\Psidot}[0]{{\dot{\Psi}}}
\newcommand{\Sigmadot}[0]{{\dot{\Sigma}}}
\newcommand{\Thetadot}[0]{{\dot{\Theta}}}
\newcommand{\Upsilondot}[0]{{\dot{\Upsilon}}}
\newcommand{\Xidot}[0]{{\dot{\Xi}}}
\newcommand{\alphadot}[0]{{\dot{\alpha}}}
\newcommand{\betadot}[0]{{\dot{\beta}}}
\newcommand{\chidot}[0]{{\dot{\chi}}}
\newcommand{\deltadot}[0]{{\dot{\delta}}}
\newcommand{\epsilondot}[0]{{\dot{\epsilon}}}
\newcommand{\etadot}[0]{{\dot{\eta}}}
\newcommand{\gammadot}[0]{{\dot{\gamma}}}
\newcommand{\kappadot}[0]{{\dot{\kappa}}}
\newcommand{\lambdadot}[0]{{\dot{\lambda}}}
\newcommand{\mudot}[0]{{\dot{\mu}}}
\newcommand{\nudot}[0]{{\dot{\nu}}}
\newcommand{\omegadot}[0]{{\dot{\omega}}}
\newcommand{\phidot}[0]{{\dot{\phi}}}
\newcommand{\pidot}[0]{{\dot{\pi}}}
\newcommand{\psidot}[0]{{\dot{\psi}}}
\newcommand{\rhodot}[0]{{\dot{\rho}}}
\newcommand{\sigmadot}[0]{{\dot{\sigma}}}
\newcommand{\taudot}[0]{{\dot{\tau}}}
\newcommand{\thetadot}[0]{{\dot{\theta}}}
\newcommand{\upsilondot}[0]{{\dot{\upsilon}}}
\newcommand{\varepsilondot}[0]{{\dot{\varepsilon}}}
\newcommand{\varphidot}[0]{{\dot{\varphi}}}
\newcommand{\varpidot}[0]{{\dot{\varpi}}}
\newcommand{\varrhodot}[0]{{\dot{\varrho}}}
\newcommand{\varsigmadot}[0]{{\dot{\varsigma}}}
\newcommand{\varthetadot}[0]{{\dot{\vartheta}}}
\newcommand{\xidot}[0]{{\dot{\xi}}}
\newcommand{\zetadot}[0]{{\dot{\zeta}}}

\newcommand{\Deltaddot}[0]{{\ddot{\Delta}}}
\newcommand{\Gammaddot}[0]{{\ddot{\Gamma}}}
\newcommand{\Lambdaddot}[0]{{\ddot{\Lambda}}}
\newcommand{\Omegaddot}[0]{{\ddot{\Omega}}}
\newcommand{\Phiddot}[0]{{\ddot{\Phi}}}
\newcommand{\Piddot}[0]{{\ddot{\Pi}}}
\newcommand{\Psiddot}[0]{{\ddot{\Psi}}}
\newcommand{\Sigmaddot}[0]{{\ddot{\Sigma}}}
\newcommand{\Thetaddot}[0]{{\ddot{\Theta}}}
\newcommand{\Upsilonddot}[0]{{\ddot{\Upsilon}}}
\newcommand{\Xiddot}[0]{{\ddot{\Xi}}}
\newcommand{\alphaddot}[0]{{\ddot{\alpha}}}
\newcommand{\betaddot}[0]{{\ddot{\beta}}}
\newcommand{\chiddot}[0]{{\ddot{\chi}}}
\newcommand{\deltaddot}[0]{{\ddot{\delta}}}
\newcommand{\epsilonddot}[0]{{\ddot{\epsilon}}}
\newcommand{\etaddot}[0]{{\ddot{\eta}}}
\newcommand{\gammaddot}[0]{{\ddot{\gamma}}}
\newcommand{\kappaddot}[0]{{\ddot{\kappa}}}
\newcommand{\lambdaddot}[0]{{\ddot{\lambda}}}
\newcommand{\muddot}[0]{{\ddot{\mu}}}
\newcommand{\nuddot}[0]{{\ddot{\nu}}}
\newcommand{\omegaddot}[0]{{\ddot{\omega}}}
\newcommand{\phiddot}[0]{{\ddot{\phi}}}
\newcommand{\piddot}[0]{{\ddot{\pi}}}
\newcommand{\psiddot}[0]{{\ddot{\psi}}}
\newcommand{\rhoddot}[0]{{\ddot{\rho}}}
\newcommand{\sigmaddot}[0]{{\ddot{\sigma}}}
\newcommand{\tauddot}[0]{{\ddot{\tau}}}
\newcommand{\thetaddot}[0]{{\ddot{\theta}}}
\newcommand{\upsilonddot}[0]{{\ddot{\upsilon}}}
\newcommand{\varepsilonddot}[0]{{\ddot{\varepsilon}}}
\newcommand{\varphiddot}[0]{{\ddot{\varphi}}}
\newcommand{\varpiddot}[0]{{\ddot{\varpi}}}
\newcommand{\varrhoddot}[0]{{\ddot{\varrho}}}
\newcommand{\varsigmaddot}[0]{{\ddot{\varsigma}}}
\newcommand{\varthetaddot}[0]{{\ddot{\vartheta}}}
\newcommand{\xiddot}[0]{{\ddot{\xi}}}
\newcommand{\zetaddot}[0]{{\ddot{\zeta}}}

\newcommand{\BDelta}[0]{\boldsymbol{\Delta}}
\newcommand{\BGamma}[0]{\boldsymbol{\Gamma}}
\newcommand{\BLambda}[0]{\boldsymbol{\Lambda}}
\newcommand{\BOmega}[0]{\boldsymbol{\Omega}}
\newcommand{\BPhi}[0]{\boldsymbol{\Phi}}
\newcommand{\BPi}[0]{\boldsymbol{\Pi}}
\newcommand{\BPsi}[0]{\boldsymbol{\Psi}}
\newcommand{\BSigma}[0]{\boldsymbol{\Sigma}}
\newcommand{\BTheta}[0]{\boldsymbol{\Theta}}
\newcommand{\BUpsilon}[0]{\boldsymbol{\Upsilon}}
\newcommand{\BXi}[0]{\boldsymbol{\Xi}}
\newcommand{\Balpha}[0]{\boldsymbol{\alpha}}
\newcommand{\Bbeta}[0]{\boldsymbol{\beta}}
\newcommand{\Bchi}[0]{\boldsymbol{\chi}}
\newcommand{\Bdelta}[0]{\boldsymbol{\delta}}
\newcommand{\Bepsilon}[0]{\boldsymbol{\epsilon}}
\newcommand{\Beta}[0]{\boldsymbol{\eta}}
\newcommand{\Bgamma}[0]{\boldsymbol{\gamma}}
\newcommand{\Bkappa}[0]{\boldsymbol{\kappa}}
\newcommand{\Blambda}[0]{\boldsymbol{\lambda}}
\newcommand{\Bmu}[0]{\boldsymbol{\mu}}
\newcommand{\Bnu}[0]{\boldsymbol{\nu}}
%\newcommand{\Bomega}[0]{\boldsymbol{\omega}}
\newcommand{\Bphi}[0]{\boldsymbol{\phi}}
\newcommand{\Bpi}[0]{\boldsymbol{\pi}}
\newcommand{\Bpsi}[0]{\boldsymbol{\psi}}
\newcommand{\Brho}[0]{\boldsymbol{\rho}}
\newcommand{\Bsigma}[0]{\boldsymbol{\sigma}}
%\newcommand{\Btau}[0]{\boldsymbol{\tau}}
%\newcommand{\Btheta}[0]{\boldsymbol{\theta}}
\newcommand{\Bupsilon}[0]{\boldsymbol{\upsilon}}
\newcommand{\Bvarepsilon}[0]{\boldsymbol{\varepsilon}}
\newcommand{\Bvarphi}[0]{\boldsymbol{\varphi}}
\newcommand{\Bvarpi}[0]{\boldsymbol{\varpi}}
\newcommand{\Bvarrho}[0]{\boldsymbol{\varrho}}
\newcommand{\Bvarsigma}[0]{\boldsymbol{\varsigma}}
\newcommand{\Bvartheta}[0]{\boldsymbol{\vartheta}}
\newcommand{\Bxi}[0]{\boldsymbol{\xi}}
\newcommand{\Bzeta}[0]{\boldsymbol{\zeta}}
%
%</bold and dot greek symbols>
%<infrequent>
%
%\newcommand{\AreaOp}[1]{\AName_{#1}}
%\newcommand{\Babs}[0]{\abs{\BB}}
%\newcommand{\Bcap}[0]{\hat{\BB}}
%\newcommand{\BrPrimeRej}[0]{\rcap(\rcap \wedge \Br')}
%\newcommand{\CA}[0]{\mathcal{A}}
%\newcommand{\Cos}[1]{\cos{\left({#1}\right)}}
%\newcommand{\Det}[1] {\abs{#1}}
%\newcommand{\Dsq}[2] {\frac {\partial^2 {#1}} {\partial {#2}^2}}
%\newcommand{\Exp}[1]{\exp{\left({#1}\right)}}
%\newcommand{\Norm}[1]{\left\lVert{#1}\right\rVert}
%\newcommand{\Sin}[1]{\sin{\left({#1}\right)}}
%\newcommand{\T}[0]{\text{T}}
%\newcommand{\VolumeOp}[1]{\VName_{#1}}
%\newcommand{\agrad}[0]{\Ba \cdot \nabla}
%\newcommand{\alphacap}[0]{\hat{\boldsymbol{\alpha}}}
%\newcommand{\Fcap}[0]{\hat{\BF}}
%\newcommand{\bithree}[0]{{\Bi}_3}
%\newcommand{\bxa}[0]{\Bx\Ba}
%\newcommand{\coordvec}[2]{
%\newcommand{\costheta}[0]{\acap \cdot \xcap}
%\newcommand{\ddt}[1]{\ddot{#1}}
%\newcommand{\ddu}[1] {\frac {d{#1}} {du}}
%\newcommand{\dsqxj}[2] {\frac {\partial^2 {#1}} {\partial {x_{#2}}^2}}
%\newcommand{\dtheta}[1]{\frac{d {#1}}{d \theta}}
%\newcommand{\dt}[1]{\dot{#1}}
%\newcommand{\dt}[1]{\frac{d {#1}}{dt}}
%\newcommand{\dxj}[2] {\frac {\partial {#1}} {\partial {x_{#2}}}}
%\newcommand{\halfPhi}[0]{\frac{\phi}{2}}
%\newcommand{\half}[0]{\inv{2}}
%\newcommand{\inv}[1]{\frac{1}{#1}}
%\newcommand{\laplacian}[0]{\nabla^2}
%\newcommand{\matrixoftx}[3]{
%\newcommand{\nrrp}[0]{\norm{\rcap \wedge \Br'}}
%\newcommand{\oiint}{\bigcirc \hspace{-1.4em} \int \hspace{-.8em} \int}
%\newcommand{\transpose}[1]{{#1}^{\text{T}}}
%\newcommand{\transpose}[1]{{{#1}^{\TextTranspose}}}
%\newcommand{\transpose}[1]{{{#1}^{\text{T}}}}
%\newcommand{\barA}[0]{\bar{A}}
%\newcommand{\qbar}[0]{\bar{q}}
%\newcommand{\qdotbar}[0]{\dot{\bar{q}}}
%
%</infrequent>




%\usepackage{listings}
%\usepackage{txfonts} % for ointctr... (also appears to make "prettier" \int and \sum's)
%\usepackage[bookmarks=true]{hyperref}

%\usepackage{color,cite,graphicx}
   % use colour in the document, put your citations as [1-4]
   % rather than [1,2,3,4] (it looks nicer, and the extended LaTeX2e
   % graphics package. 
%\usepackage{latexsym,amssymb,epsf} % do not remember if these are
   % needed, but their inclusion can not do any damage


\chapter{Simple minded variation of one dimensional wave equation Lagrangian}
\label{chap:wavevariation}
%\author{Peeter Joot \quad peeterjoot@protonmail.com }
\date{ April 27, 2009.  wavevariation.tex }

%\begin{document}

%\maketitle{}
%\tableofcontents
%\section{}

From the action

\begin{equation}\label{eqn:wavevariation:20}
\begin{aligned}
S_\eta = \int dx dt \inv{2} \left( \left(\PD{x}{\eta}\right)^2 - \inv{v^2} \left(\PD{t}{\eta}\right)^2 \right)
\end{aligned}
\end{equation}

We can vary the field \(\eta = \psi + \epsilon\), where \(\psi\) is the field variable to be determined, and \(\epsilon\) is the field variable allowed to vary within the volume of integration.

Forming the difference, subtracts off the ``constant'' parts of the action due only to the optimal field variable \(\psi\)

\begin{equation}\label{eqn:wavevariation:40}
\begin{aligned}
S_{\psi + \epsilon} - S_\psi
&=
\int dx dt \inv{2} \left( \left(\PD{x}{(\psi + \epsilon)}\right)^2 - \inv{v^2} \left(\PD{t}{(\psi + \epsilon)}\right)^2 \right)
-\int dx dt \inv{2} \left( \left(\PD{x}{\psi}\right)^2 - \inv{v^2} \left(\PD{t}{\psi}\right)^2 \right) \\
&=
\int dx dt \left( \PD{x}{\psi}\PD{x}{\epsilon} - \PD{t}{\psi}\PD{t}{\epsilon} \right)
+ \int dx dt \inv{2} \left( \left(\PD{x}{\epsilon }\right)^2 - \inv{v^2} \left(\PD{t}{\epsilon }\right)^2 \right) \\
\end{aligned}
\end{equation}

Now integrating by parts

\begin{equation}\label{eqn:wavevariation:60}
\begin{aligned}
S_{\psi + \epsilon} - S_\psi
&=
\int dx dt \left( \inv{v^2} \PDSq{t}{\psi} -\PDSq{x}{\psi} \right) {\epsilon} 
+ \int dx dt \inv{2} \left( -\PDSq{x}{\epsilon } + \inv{v^2} \PDSq{t}{\epsilon } \right) \epsilon \\
\end{aligned}
\end{equation}

Roughly speaking the terms that are quadratic in \(\epsilon\) can be discarded as small, and if the remaining differential
is to be zero for all \(\epsilon\), we are left with the wave equation

\begin{equation}\label{eqn:wavevariation:80}
\begin{aligned}
\inv{v^2} \PDSq{t}{\psi} -\PDSq{x}{\psi} = 0 
\end{aligned}
\end{equation}

with solutions 
%(like your exponential) 
of the form 

\begin{equation}\label{eqn:wavevariation:100}
\begin{aligned}
\psi = f(x \pm vt)
\end{aligned}
\end{equation}

%\bibliographystyle{plainnat}
%\bibliography{myrefs}

%\end{document}

% 
% 
% 
% Copyright � 2012 Peeter Joot
% All Rights Reserved
% 
% This file may be reproduced and distributed in whole or in part, without fee, subject to the following conditions:
% 
% o The copyright notice above and this permission notice must be preserved complete on all complete or partial copies.
% 
% o Any translation or derived work must be approved by the author in writing before distribution.
% 
% o If you distribute this work in part, instructions for obtaining the complete version of this file must be included, and a means for obtaining a complete version provided.
% 
% 
% Exceptions to these rules may be granted for academic purposes: Write to the author and ask.
% 
% 
% 
%\documentclass{article}

%\usepackage{amsmath}
\usepackage{mathpazo}

%
% shorthand for bold symbols, convenient for vectors and matrices
%
\newcommand{\Ba}[0]{\mathbf{a}}
\newcommand{\Bb}[0]{\mathbf{b}}
\newcommand{\Bc}[0]{\mathbf{c}}
\newcommand{\Bd}[0]{\mathbf{d}}
\newcommand{\Be}[0]{\mathbf{e}}
\newcommand{\Bf}[0]{\mathbf{f}}
\newcommand{\Bg}[0]{\mathbf{g}}
\newcommand{\Bh}[0]{\mathbf{h}}
\newcommand{\Bi}[0]{\mathbf{i}}
\newcommand{\Bj}[0]{\mathbf{j}}
\newcommand{\Bk}[0]{\mathbf{k}}
\newcommand{\Bl}[0]{\mathbf{l}}
\newcommand{\Bm}[0]{\mathbf{m}}
\newcommand{\Bn}[0]{\mathbf{n}}
\newcommand{\Bo}[0]{\mathbf{o}}
\newcommand{\Bp}[0]{\mathbf{p}}
\newcommand{\Bq}[0]{\mathbf{q}}
\newcommand{\Br}[0]{\mathbf{r}}
\newcommand{\Bs}[0]{\mathbf{s}}
\newcommand{\Bt}[0]{\mathbf{t}}
\newcommand{\Bu}[0]{\mathbf{u}}
\newcommand{\Bv}[0]{\mathbf{v}}
\newcommand{\Bw}[0]{\mathbf{w}}
\newcommand{\Bx}[0]{\mathbf{x}}
\newcommand{\By}[0]{\mathbf{y}}
\newcommand{\Bz}[0]{\mathbf{z}}
\newcommand{\BA}[0]{\mathbf{A}}
\newcommand{\BB}[0]{\mathbf{B}}
\newcommand{\BC}[0]{\mathbf{C}}
\newcommand{\BD}[0]{\mathbf{D}}
\newcommand{\BE}[0]{\mathbf{E}}
\newcommand{\BF}[0]{\mathbf{F}}
\newcommand{\BG}[0]{\mathbf{G}}
\newcommand{\BH}[0]{\mathbf{H}}
\newcommand{\BI}[0]{\mathbf{I}}
\newcommand{\BJ}[0]{\mathbf{J}}
\newcommand{\BK}[0]{\mathbf{K}}
\newcommand{\BL}[0]{\mathbf{L}}
\newcommand{\BM}[0]{\mathbf{M}}
\newcommand{\BN}[0]{\mathbf{N}}
\newcommand{\BO}[0]{\mathbf{O}}
\newcommand{\BP}[0]{\mathbf{P}}
\newcommand{\BQ}[0]{\mathbf{Q}}
\newcommand{\BR}[0]{\mathbf{R}}
\newcommand{\BS}[0]{\mathbf{S}}
\newcommand{\BT}[0]{\mathbf{T}}
\newcommand{\BU}[0]{\mathbf{U}}
\newcommand{\BV}[0]{\mathbf{V}}
\newcommand{\BW}[0]{\mathbf{W}}
\newcommand{\BX}[0]{\mathbf{X}}
\newcommand{\BY}[0]{\mathbf{Y}}
\newcommand{\BZ}[0]{\mathbf{Z}}

\newcommand{\Bzero}[0]{\mathbf{0}}
\newcommand{\Btheta}[0]{\boldsymbol{\theta}}
\newcommand{\Btau}[0]{\boldsymbol{\tau}}
\newcommand{\Bomega}[0]{\boldsymbol{\omega}}

%
% shorthand for unit vectors
%
\newcommand{\acap}[0]{\hat{\Ba}}
\newcommand{\bcap}[0]{\hat{\Bb}}
\newcommand{\ccap}[0]{\hat{\Bc}}
\newcommand{\dcap}[0]{\hat{\Bd}}
\newcommand{\ecap}[0]{\hat{\Be}}
\newcommand{\fcap}[0]{\hat{\Bf}}
\newcommand{\gcap}[0]{\hat{\Bg}}
\newcommand{\hcap}[0]{\hat{\Bh}}
\newcommand{\icap}[0]{\hat{\Bi}}
\newcommand{\jcap}[0]{\hat{\Bj}}
\newcommand{\kcap}[0]{\hat{\Bk}}
\newcommand{\lcap}[0]{\hat{\Bl}}
\newcommand{\mcap}[0]{\hat{\Bm}}
\newcommand{\ncap}[0]{\hat{\Bn}}
\newcommand{\ocap}[0]{\hat{\Bo}}
\newcommand{\pcap}[0]{\hat{\Bp}}
\newcommand{\qcap}[0]{\hat{\Bq}}
\newcommand{\rcap}[0]{\hat{\Br}}
\newcommand{\scap}[0]{\hat{\Bs}}
\newcommand{\tcap}[0]{\hat{\Bt}}
\newcommand{\ucap}[0]{\hat{\Bu}}
\newcommand{\vcap}[0]{\hat{\Bv}}
\newcommand{\wcap}[0]{\hat{\Bw}}
\newcommand{\xcap}[0]{\hat{\Bx}}
\newcommand{\ycap}[0]{\hat{\By}}
\newcommand{\zcap}[0]{\hat{\Bz}}
\newcommand{\thetacap}[0]{\hat{\Btheta}}

%
% to write R^n and C^n in a distinguishable fashion.  Perhaps change this
% to the double lined characters upon figuring out how to do so.
%
\newcommand{\C}[1]{$\mathbb{C}^{#1}$}
\newcommand{\R}[1]{$\mathbb{R}^{#1}$}

%
% various generally useful helpers
%

% derivative of #1 wrt. #2:
\newcommand{\D}[2] {\frac {d#2} {d#1}}

\newcommand{\inv}[1]{\frac{1}{#1}}
\newcommand{\cross}[0]{\times}

\newcommand{\abs}[1]{\lvert{#1}\rvert}
\newcommand{\norm}[1]{\lVert{#1}\rVert}
\newcommand{\innerprod}[2]{\langle{#1}, {#2}\rangle}
\newcommand{\dotprod}[2]{{#1} \cdot {#2}}
\newcommand{\bdotprod}[2]{\left({#1} \cdot {#2}\right)}
\newcommand{\crossprod}[2]{{#1} \cross {#2}}
\newcommand{\tripleprod}[3]{\dotprod{\left(\crossprod{#1}{#2}\right)}{#3}}

\DeclareMathOperator{\Proj}{Proj}
\DeclareMathOperator{\Span}{span}
\DeclareMathOperator{\Sgn}{sgn}
\DeclareMathOperator{\Area}{Area}
\DeclareMathOperator{\Volume}{Volume}

%
% A few miscellaneous things specific to this document
%
\newcommand{\crossop}[1]{\crossprod{#1}{}}

% R2 vector.
\newcommand{\VectorTwo}[2]{
\begin{bmatrix}
 {#1} \\
 {#2}
\end{bmatrix}
}

\newcommand{\VectorN}[1]{
\begin{bmatrix}
{#1}_1 \\
{#1}_2 \\
\vdots \\
{#1}_N \\
\end{bmatrix}
}

\newcommand{\DETuvij}[4]{
\begin{vmatrix}
 {#1}_{#3} & {#1}_{#4} \\
 {#2}_{#3} & {#2}_{#4}
\end{vmatrix}
}

\newcommand{\DETuvwijk}[6]{
\begin{vmatrix}
 {#1}_{#4} & {#1}_{#5} & {#1}_{#6} \\
 {#2}_{#4} & {#2}_{#5} & {#2}_{#6} \\
 {#3}_{#4} & {#3}_{#5} & {#3}_{#6}
\end{vmatrix}
}

\newcommand{\DETuvwxijkl}[8]{
\begin{vmatrix}
 {#1}_{#5} & {#1}_{#6} & {#1}_{#7} & {#1}_{#8} \\
 {#2}_{#5} & {#2}_{#6} & {#2}_{#7} & {#2}_{#8} \\
 {#3}_{#5} & {#3}_{#6} & {#3}_{#7} & {#3}_{#8} \\
 {#4}_{#5} & {#4}_{#6} & {#4}_{#7} & {#4}_{#8} \\
\end{vmatrix}
}

%\newcommand{\DETuvwxyijklm}[10]{
%\begin{vmatrix}
% {#1}_{#6} & {#1}_{#7} & {#1}_{#8} & {#1}_{#9} & {#1}_{#10} \\
% {#2}_{#6} & {#2}_{#7} & {#2}_{#8} & {#2}_{#9} & {#2}_{#10} \\
% {#3}_{#6} & {#3}_{#7} & {#3}_{#8} & {#3}_{#9} & {#3}_{#10} \\
% {#4}_{#6} & {#4}_{#7} & {#4}_{#8} & {#4}_{#9} & {#4}_{#10} \\
% {#5}_{#6} & {#5}_{#7} & {#5}_{#8} & {#5}_{#9} & {#5}_{#10}
%\end{vmatrix}
%}

% R3 vector.
\newcommand{\VectorThree}[3]{
\begin{bmatrix}
 {#1} \\
 {#2} \\
 {#3}
\end{bmatrix}
}



%\usepackage[bookmarks=true]{hyperref}

\chapter{Taylor's theorem deviation.}
\label{chap:taylors}
%\author{Peeter Joot \quad peeter.joot@gmail.com}
\date{ Feb. 2, 2008. taylors.tex }

%\begin{document}

%\maketitle{}

\citep{hestenes1999nfc} presents a very simple derivation of Taylor's Theorem,
but I feel a
slightly different (dumber, but longer) presentation would be more effective.

In the same fashion, form the integral

\[
I = \int_{t}^{t+s} F'(u) du
\]

Now, observe that the this first order derivative can be written in
terms of it's second order derivative

\[
(u F'(u))' = F'(u) + u F''(u)
\]

So we could write

\begin{align*}
I &= \int_{t}^{t+s} ((u F'(u))' - u F''(u)) du \\
  &= {u F'(u)} \vert_{u=t}^{t+s} - \int_{t}^{t+s} u F''(u)) du \\
  &= (t+s) F'(t+s) - t F'(t) - \int_{t}^{t+s} u F''(u)) du \\
\end{align*}

This is true, but not the Taylor expansion we are used to.  Adjusting things slightly leaves a zero term at $u=t+s$, as follows:

\[
\left((t + s - u) F'(u)\right)' = -F'(u) + (t+s-u) F''(u)
\]

\begin{align*}
I &= F(t+s) - F(t) \\
 &= \int_{t}^{t+s} ( - ((t + s - u) F'(u))' + (t+s-u) F''(u) ) du \\
 &= - {(t + s - u) F'(u)} \vert_{u=t}^{t+s} + \int_{t}^{t+s} (t+s-u) F''(u) du \\
 &= s F'(t) + \int_{t}^{t+s} (t+s-u) F''(u) du \\
\end{align*}

This results in the first two order terms of the Tailor series, with an explicit remainder term

\[
F(t+s) = F(t) + s F'(t) + \int_{t}^{t+s} (t+s-u) F''(u) du
\]

This process can be continued for as many terms as desired.  Doing the calculation for the second order term yields

\[
\left(\frac{(t + s - u)^2}{2} F''(u)\right)' = -( t + s - u ) F''(u) + \frac{(t+s-u)}{2} F'''(u)
\]

And substituting that into the first order expansion above we have

\[
F(t+s) = F(t) + s F'(t) + \frac{s^2}{2} F''(t) + \int_{t}^{t+s} \frac{1}{2}(t+s-u)^2 F'''(u) du
\]

Induction produces n terms of Taylor's series with an explicit remainder

\[
F(t+s) = \sum_{k=0}^{n} \frac{s^k}{k!} \frac{d^k}{dt^k} F(t) +
                        \int_{t}^{t+s} \frac{1}{n!}(t+s-u)^n \frac{d^{n+1}}{dt^{n+1}} F(u) du
\]

To truly prove the infinite series result one would have to show that the remainder term tends to zero.

%\bibliographystyle{plainnat} % supposed to allow for \url use.
%\bibliography{myrefs}      % expects file "myrefs.bib"

%\end{document}               % End of document.

\documentclass[]{eliblog}

\usepackage{amsmath}
\usepackage{mathpazo}

%
% shorthand for bold symbols, convenient for vectors and matrices
%
\newcommand{\Ba}[0]{\mathbf{a}}
\newcommand{\Bb}[0]{\mathbf{b}}
\newcommand{\Bc}[0]{\mathbf{c}}
\newcommand{\Bd}[0]{\mathbf{d}}
\newcommand{\Be}[0]{\mathbf{e}}
\newcommand{\Bf}[0]{\mathbf{f}}
\newcommand{\Bg}[0]{\mathbf{g}}
\newcommand{\Bh}[0]{\mathbf{h}}
\newcommand{\Bi}[0]{\mathbf{i}}
\newcommand{\Bj}[0]{\mathbf{j}}
\newcommand{\Bk}[0]{\mathbf{k}}
\newcommand{\Bl}[0]{\mathbf{l}}
\newcommand{\Bm}[0]{\mathbf{m}}
\newcommand{\Bn}[0]{\mathbf{n}}
\newcommand{\Bo}[0]{\mathbf{o}}
\newcommand{\Bp}[0]{\mathbf{p}}
\newcommand{\Bq}[0]{\mathbf{q}}
\newcommand{\Br}[0]{\mathbf{r}}
\newcommand{\Bs}[0]{\mathbf{s}}
\newcommand{\Bt}[0]{\mathbf{t}}
\newcommand{\Bu}[0]{\mathbf{u}}
\newcommand{\Bv}[0]{\mathbf{v}}
\newcommand{\Bw}[0]{\mathbf{w}}
\newcommand{\Bx}[0]{\mathbf{x}}
\newcommand{\By}[0]{\mathbf{y}}
\newcommand{\Bz}[0]{\mathbf{z}}
\newcommand{\BA}[0]{\mathbf{A}}
\newcommand{\BB}[0]{\mathbf{B}}
\newcommand{\BC}[0]{\mathbf{C}}
\newcommand{\BD}[0]{\mathbf{D}}
\newcommand{\BE}[0]{\mathbf{E}}
\newcommand{\BF}[0]{\mathbf{F}}
\newcommand{\BG}[0]{\mathbf{G}}
\newcommand{\BH}[0]{\mathbf{H}}
\newcommand{\BI}[0]{\mathbf{I}}
\newcommand{\BJ}[0]{\mathbf{J}}
\newcommand{\BK}[0]{\mathbf{K}}
\newcommand{\BL}[0]{\mathbf{L}}
\newcommand{\BM}[0]{\mathbf{M}}
\newcommand{\BN}[0]{\mathbf{N}}
\newcommand{\BO}[0]{\mathbf{O}}
\newcommand{\BP}[0]{\mathbf{P}}
\newcommand{\BQ}[0]{\mathbf{Q}}
\newcommand{\BR}[0]{\mathbf{R}}
\newcommand{\BS}[0]{\mathbf{S}}
\newcommand{\BT}[0]{\mathbf{T}}
\newcommand{\BU}[0]{\mathbf{U}}
\newcommand{\BV}[0]{\mathbf{V}}
\newcommand{\BW}[0]{\mathbf{W}}
\newcommand{\BX}[0]{\mathbf{X}}
\newcommand{\BY}[0]{\mathbf{Y}}
\newcommand{\BZ}[0]{\mathbf{Z}}

\newcommand{\Bzero}[0]{\mathbf{0}}
\newcommand{\Btheta}[0]{\boldsymbol{\theta}}
\newcommand{\Btau}[0]{\boldsymbol{\tau}}
\newcommand{\Bomega}[0]{\boldsymbol{\omega}}

%
% shorthand for unit vectors
%
\newcommand{\acap}[0]{\hat{\Ba}}
\newcommand{\bcap}[0]{\hat{\Bb}}
\newcommand{\ccap}[0]{\hat{\Bc}}
\newcommand{\dcap}[0]{\hat{\Bd}}
\newcommand{\ecap}[0]{\hat{\Be}}
\newcommand{\fcap}[0]{\hat{\Bf}}
\newcommand{\gcap}[0]{\hat{\Bg}}
\newcommand{\hcap}[0]{\hat{\Bh}}
\newcommand{\icap}[0]{\hat{\Bi}}
\newcommand{\jcap}[0]{\hat{\Bj}}
\newcommand{\kcap}[0]{\hat{\Bk}}
\newcommand{\lcap}[0]{\hat{\Bl}}
\newcommand{\mcap}[0]{\hat{\Bm}}
\newcommand{\ncap}[0]{\hat{\Bn}}
\newcommand{\ocap}[0]{\hat{\Bo}}
\newcommand{\pcap}[0]{\hat{\Bp}}
\newcommand{\qcap}[0]{\hat{\Bq}}
\newcommand{\rcap}[0]{\hat{\Br}}
\newcommand{\scap}[0]{\hat{\Bs}}
\newcommand{\tcap}[0]{\hat{\Bt}}
\newcommand{\ucap}[0]{\hat{\Bu}}
\newcommand{\vcap}[0]{\hat{\Bv}}
\newcommand{\wcap}[0]{\hat{\Bw}}
\newcommand{\xcap}[0]{\hat{\Bx}}
\newcommand{\ycap}[0]{\hat{\By}}
\newcommand{\zcap}[0]{\hat{\Bz}}
\newcommand{\thetacap}[0]{\hat{\Btheta}}

%
% to write R^n and C^n in a distinguishable fashion.  Perhaps change this
% to the double lined characters upon figuring out how to do so.
%
\newcommand{\C}[1]{$\mathbb{C}^{#1}$}
\newcommand{\R}[1]{$\mathbb{R}^{#1}$}

%
% various generally useful helpers
%

% derivative of #1 wrt. #2:
\newcommand{\D}[2] {\frac {d#2} {d#1}}

\newcommand{\inv}[1]{\frac{1}{#1}}
\newcommand{\cross}[0]{\times}

\newcommand{\abs}[1]{\lvert{#1}\rvert}
\newcommand{\norm}[1]{\lVert{#1}\rVert}
\newcommand{\innerprod}[2]{\langle{#1}, {#2}\rangle}
\newcommand{\dotprod}[2]{{#1} \cdot {#2}}
\newcommand{\bdotprod}[2]{\left({#1} \cdot {#2}\right)}
\newcommand{\crossprod}[2]{{#1} \cross {#2}}
\newcommand{\tripleprod}[3]{\dotprod{\left(\crossprod{#1}{#2}\right)}{#3}}

\DeclareMathOperator{\Proj}{Proj}
\DeclareMathOperator{\Span}{span}
\DeclareMathOperator{\Sgn}{sgn}
\DeclareMathOperator{\Area}{Area}
\DeclareMathOperator{\Volume}{Volume}

%
% A few miscellaneous things specific to this document
%
\newcommand{\crossop}[1]{\crossprod{#1}{}}

% R2 vector.
\newcommand{\VectorTwo}[2]{
\begin{bmatrix}
 {#1} \\
 {#2}
\end{bmatrix}
}

\newcommand{\VectorN}[1]{
\begin{bmatrix}
{#1}_1 \\
{#1}_2 \\
\vdots \\
{#1}_N \\
\end{bmatrix}
}

\newcommand{\DETuvij}[4]{
\begin{vmatrix}
 {#1}_{#3} & {#1}_{#4} \\
 {#2}_{#3} & {#2}_{#4}
\end{vmatrix}
}

\newcommand{\DETuvwijk}[6]{
\begin{vmatrix}
 {#1}_{#4} & {#1}_{#5} & {#1}_{#6} \\
 {#2}_{#4} & {#2}_{#5} & {#2}_{#6} \\
 {#3}_{#4} & {#3}_{#5} & {#3}_{#6}
\end{vmatrix}
}

\newcommand{\DETuvwxijkl}[8]{
\begin{vmatrix}
 {#1}_{#5} & {#1}_{#6} & {#1}_{#7} & {#1}_{#8} \\
 {#2}_{#5} & {#2}_{#6} & {#2}_{#7} & {#2}_{#8} \\
 {#3}_{#5} & {#3}_{#6} & {#3}_{#7} & {#3}_{#8} \\
 {#4}_{#5} & {#4}_{#6} & {#4}_{#7} & {#4}_{#8} \\
\end{vmatrix}
}

%\newcommand{\DETuvwxyijklm}[10]{
%\begin{vmatrix}
% {#1}_{#6} & {#1}_{#7} & {#1}_{#8} & {#1}_{#9} & {#1}_{#10} \\
% {#2}_{#6} & {#2}_{#7} & {#2}_{#8} & {#2}_{#9} & {#2}_{#10} \\
% {#3}_{#6} & {#3}_{#7} & {#3}_{#8} & {#3}_{#9} & {#3}_{#10} \\
% {#4}_{#6} & {#4}_{#7} & {#4}_{#8} & {#4}_{#9} & {#4}_{#10} \\
% {#5}_{#6} & {#5}_{#7} & {#5}_{#8} & {#5}_{#9} & {#5}_{#10}
%\end{vmatrix}
%}

% R3 vector.
\newcommand{\VectorThree}[3]{
\begin{bmatrix}
 {#1} \\
 {#2} \\
 {#3}
\end{bmatrix}
}



\author{Peeter Joot}
\email{peeter.joot@gmail.com}


\chapter{Linearizing a set of regular differential equations.}
%\label{chap:template}
%\useCCL
%\blogpage{http://sites.google.com/site/peeterjoot/math2009/template.pdf}
%\date{Sept XX, 2009}
\revisionInfo{$RCSfile: linearizeDE.tex,v $ Last $Revision: 1.1 $ $Date: 2009/11/13 04:44:23 $}

%\beginArtWithToc
\beginArtNoToc

%\section{Motivation}
%\section{Guts}

Hi Lut,

Sorry for the over the head remarks.  Here's an elaboration.  All of these discrete multiple particle systems appear to generate coupled differential equations of the following form

\begin{align*}
A z' = b(z)
\end{align*}

Where $A = A(z)$ is a matrix, and $b(z)$ a column vector valued non-linear function.  Consider the nicely behaved case where $A(z)$ is invertable for all $z$.  Then we can write

\begin{align*}
z' = A^{-1} b(z)
\end{align*}

Now with a non-linear function $b$ (like the sines that we have in the pendulum problem from $-\grad \cos\phi$), we can't solve this thing easily, but in some small-enough neighbourhood of some point (i.e. a point in phase space containing $z$) we can make a linear approximation.  Suppose our initial phase space point is $z_0$, and we wish to solve for differential displacement from that point $x$, namely $z = z_0 + x$.  Then we have for our system

\begin{align*}
x' = A^{-1} b(z_0) + 
\evalnobar{
\begin{bmatrix}
\PD{z_1}{[A^{-1}b(z)]_1} & \PD{z_2}{[A^{-1}b(z)]_1} & \hdots & \PD{z_N}{[A^{-1}b(z)]_1} \\
\PD{z_1}{[A^{-1}b(z)]_2} & \PD{z_2}{[A^{-1}b(z)]_2} & \hdots & \PD{z_N}{[A^{-1}b(z)]_2} \\
\vdots \\
\PD{z_1}{[A^{-1}b(z)]_N} & \PD{z_2}{[A^{-1}b(z)]_N} & \hdots & \PD{z_N}{[A^{-1}b(z)]_N} \\
\end{bmatrix}}{z = z_0} x
\end{align*}

Now we have a linear matrix, corresponding roughly to a first order Taylor expansion of the original system of equations.

%\EndArticle
\EndNoBibArticle

%
% Copyright � 2012 Peeter Joot.  All Rights Reserved.
% Licenced as described in the file LICENSE under the root directory of this GIT repository.
%

% 
% 
%\documentclass[]{eliblog}

\usepackage{amsmath}
\usepackage{mathpazo}

%
% shorthand for bold symbols, convenient for vectors and matrices
%
\newcommand{\Ba}[0]{\mathbf{a}}
\newcommand{\Bb}[0]{\mathbf{b}}
\newcommand{\Bc}[0]{\mathbf{c}}
\newcommand{\Bd}[0]{\mathbf{d}}
\newcommand{\Be}[0]{\mathbf{e}}
\newcommand{\Bf}[0]{\mathbf{f}}
\newcommand{\Bg}[0]{\mathbf{g}}
\newcommand{\Bh}[0]{\mathbf{h}}
\newcommand{\Bi}[0]{\mathbf{i}}
\newcommand{\Bj}[0]{\mathbf{j}}
\newcommand{\Bk}[0]{\mathbf{k}}
\newcommand{\Bl}[0]{\mathbf{l}}
\newcommand{\Bm}[0]{\mathbf{m}}
\newcommand{\Bn}[0]{\mathbf{n}}
\newcommand{\Bo}[0]{\mathbf{o}}
\newcommand{\Bp}[0]{\mathbf{p}}
\newcommand{\Bq}[0]{\mathbf{q}}
\newcommand{\Br}[0]{\mathbf{r}}
\newcommand{\Bs}[0]{\mathbf{s}}
\newcommand{\Bt}[0]{\mathbf{t}}
\newcommand{\Bu}[0]{\mathbf{u}}
\newcommand{\Bv}[0]{\mathbf{v}}
\newcommand{\Bw}[0]{\mathbf{w}}
\newcommand{\Bx}[0]{\mathbf{x}}
\newcommand{\By}[0]{\mathbf{y}}
\newcommand{\Bz}[0]{\mathbf{z}}
\newcommand{\BA}[0]{\mathbf{A}}
\newcommand{\BB}[0]{\mathbf{B}}
\newcommand{\BC}[0]{\mathbf{C}}
\newcommand{\BD}[0]{\mathbf{D}}
\newcommand{\BE}[0]{\mathbf{E}}
\newcommand{\BF}[0]{\mathbf{F}}
\newcommand{\BG}[0]{\mathbf{G}}
\newcommand{\BH}[0]{\mathbf{H}}
\newcommand{\BI}[0]{\mathbf{I}}
\newcommand{\BJ}[0]{\mathbf{J}}
\newcommand{\BK}[0]{\mathbf{K}}
\newcommand{\BL}[0]{\mathbf{L}}
\newcommand{\BM}[0]{\mathbf{M}}
\newcommand{\BN}[0]{\mathbf{N}}
\newcommand{\BO}[0]{\mathbf{O}}
\newcommand{\BP}[0]{\mathbf{P}}
\newcommand{\BQ}[0]{\mathbf{Q}}
\newcommand{\BR}[0]{\mathbf{R}}
\newcommand{\BS}[0]{\mathbf{S}}
\newcommand{\BT}[0]{\mathbf{T}}
\newcommand{\BU}[0]{\mathbf{U}}
\newcommand{\BV}[0]{\mathbf{V}}
\newcommand{\BW}[0]{\mathbf{W}}
\newcommand{\BX}[0]{\mathbf{X}}
\newcommand{\BY}[0]{\mathbf{Y}}
\newcommand{\BZ}[0]{\mathbf{Z}}

\newcommand{\Bzero}[0]{\mathbf{0}}
\newcommand{\Btheta}[0]{\boldsymbol{\theta}}
\newcommand{\Btau}[0]{\boldsymbol{\tau}}
\newcommand{\Bomega}[0]{\boldsymbol{\omega}}

%
% shorthand for unit vectors
%
\newcommand{\acap}[0]{\hat{\Ba}}
\newcommand{\bcap}[0]{\hat{\Bb}}
\newcommand{\ccap}[0]{\hat{\Bc}}
\newcommand{\dcap}[0]{\hat{\Bd}}
\newcommand{\ecap}[0]{\hat{\Be}}
\newcommand{\fcap}[0]{\hat{\Bf}}
\newcommand{\gcap}[0]{\hat{\Bg}}
\newcommand{\hcap}[0]{\hat{\Bh}}
\newcommand{\icap}[0]{\hat{\Bi}}
\newcommand{\jcap}[0]{\hat{\Bj}}
\newcommand{\kcap}[0]{\hat{\Bk}}
\newcommand{\lcap}[0]{\hat{\Bl}}
\newcommand{\mcap}[0]{\hat{\Bm}}
\newcommand{\ncap}[0]{\hat{\Bn}}
\newcommand{\ocap}[0]{\hat{\Bo}}
\newcommand{\pcap}[0]{\hat{\Bp}}
\newcommand{\qcap}[0]{\hat{\Bq}}
\newcommand{\rcap}[0]{\hat{\Br}}
\newcommand{\scap}[0]{\hat{\Bs}}
\newcommand{\tcap}[0]{\hat{\Bt}}
\newcommand{\ucap}[0]{\hat{\Bu}}
\newcommand{\vcap}[0]{\hat{\Bv}}
\newcommand{\wcap}[0]{\hat{\Bw}}
\newcommand{\xcap}[0]{\hat{\Bx}}
\newcommand{\ycap}[0]{\hat{\By}}
\newcommand{\zcap}[0]{\hat{\Bz}}
\newcommand{\thetacap}[0]{\hat{\Btheta}}

%
% to write R^n and C^n in a distinguishable fashion.  Perhaps change this
% to the double lined characters upon figuring out how to do so.
%
\newcommand{\C}[1]{$\mathbb{C}^{#1}$}
\newcommand{\R}[1]{$\mathbb{R}^{#1}$}

%
% various generally useful helpers
%

% derivative of #1 wrt. #2:
\newcommand{\D}[2] {\frac {d#2} {d#1}}

\newcommand{\inv}[1]{\frac{1}{#1}}
\newcommand{\cross}[0]{\times}

\newcommand{\abs}[1]{\lvert{#1}\rvert}
\newcommand{\norm}[1]{\lVert{#1}\rVert}
\newcommand{\innerprod}[2]{\langle{#1}, {#2}\rangle}
\newcommand{\dotprod}[2]{{#1} \cdot {#2}}
\newcommand{\bdotprod}[2]{\left({#1} \cdot {#2}\right)}
\newcommand{\crossprod}[2]{{#1} \cross {#2}}
\newcommand{\tripleprod}[3]{\dotprod{\left(\crossprod{#1}{#2}\right)}{#3}}

\DeclareMathOperator{\Proj}{Proj}
\DeclareMathOperator{\Span}{span}
\DeclareMathOperator{\Sgn}{sgn}
\DeclareMathOperator{\Area}{Area}
\DeclareMathOperator{\Volume}{Volume}

%
% A few miscellaneous things specific to this document
%
\newcommand{\crossop}[1]{\crossprod{#1}{}}

% R2 vector.
\newcommand{\VectorTwo}[2]{
\begin{bmatrix}
 {#1} \\
 {#2}
\end{bmatrix}
}

\newcommand{\VectorN}[1]{
\begin{bmatrix}
{#1}_1 \\
{#1}_2 \\
\vdots \\
{#1}_N \\
\end{bmatrix}
}

\newcommand{\DETuvij}[4]{
\begin{vmatrix}
 {#1}_{#3} & {#1}_{#4} \\
 {#2}_{#3} & {#2}_{#4}
\end{vmatrix}
}

\newcommand{\DETuvwijk}[6]{
\begin{vmatrix}
 {#1}_{#4} & {#1}_{#5} & {#1}_{#6} \\
 {#2}_{#4} & {#2}_{#5} & {#2}_{#6} \\
 {#3}_{#4} & {#3}_{#5} & {#3}_{#6}
\end{vmatrix}
}

\newcommand{\DETuvwxijkl}[8]{
\begin{vmatrix}
 {#1}_{#5} & {#1}_{#6} & {#1}_{#7} & {#1}_{#8} \\
 {#2}_{#5} & {#2}_{#6} & {#2}_{#7} & {#2}_{#8} \\
 {#3}_{#5} & {#3}_{#6} & {#3}_{#7} & {#3}_{#8} \\
 {#4}_{#5} & {#4}_{#6} & {#4}_{#7} & {#4}_{#8} \\
\end{vmatrix}
}

%\newcommand{\DETuvwxyijklm}[10]{
%\begin{vmatrix}
% {#1}_{#6} & {#1}_{#7} & {#1}_{#8} & {#1}_{#9} & {#1}_{#10} \\
% {#2}_{#6} & {#2}_{#7} & {#2}_{#8} & {#2}_{#9} & {#2}_{#10} \\
% {#3}_{#6} & {#3}_{#7} & {#3}_{#8} & {#3}_{#9} & {#3}_{#10} \\
% {#4}_{#6} & {#4}_{#7} & {#4}_{#8} & {#4}_{#9} & {#4}_{#10} \\
% {#5}_{#6} & {#5}_{#7} & {#5}_{#8} & {#5}_{#9} & {#5}_{#10}
%\end{vmatrix}
%}

% R3 vector.
\newcommand{\VectorThree}[3]{
\begin{bmatrix}
 {#1} \\
 {#2} \\
 {#3}
\end{bmatrix}
}



\author{Peeter Joot}
\email{peeter.joot@gmail.com}


\chapter{Jacobians and spherical polar gradient}
\label{chap:jacobianSphericalPolar}
%\useCCL
\blogpage{http://sites.google.com/site/peeterjoot/math2009/jacobianSphericalPolar.pdf}
\date{Dec 6, 2009}
\revisionInfo{jacobianSphericalPolar.tex}

\beginArtWithToc
%\beginArtNoToc

\section{Motivation}

The dumbest and most obvious way to do a chain of variables for the gradient is to utilize a chain rule expansion producing the Jacobian matrix to transform the coordinates.  Here we do this to calculate the spherical polar representation of the gradient.

There are smarter and easier ways to do this, but there is some surprising simple structure to the resulting Jacobians that seems worth noting.

\section{Spherical polar gradient coordinates in terms of Cartesian}

We wish to do a change of variables for each of the differential operators of the gradient.  This is essentially just application of the chain rule, as in

\begin{equation}\label{eqn:jacobianSphericalPolar:1}
\begin{aligned}
\PD{r}{} 
= 
\PD{r}{x} \PD{x}{}
+\PD{r}{y} \PD{y}{}
+\PD{r}{z} \PD{z}{}.
\end{aligned}
\end{equation}

Collecting all such derivatives we have in column vector form 
\begin{equation}\label{eqn:jacobianSphericalPolar:2}
\begin{aligned}
\begin{bmatrix}
\partial_r \\
\partial_\theta \\
\partial_\phi
\end{bmatrix}
= 
\begin{bmatrix}
\PD{r}{x} &\PD{r}{y} &\PD{r}{z}  \\
\PD{\theta}{x} &\PD{\theta}{y} &\PD{\theta}{z}  \\
\PD{\phi}{x} &\PD{\phi}{y} &\PD{\phi}{z} 
\end{bmatrix}
\begin{bmatrix}
\partial_x \\
\partial_y \\
\partial_z
\end{bmatrix}.
\end{aligned}
\end{equation}

This becomes a bit more tractable with the Jacobian notation

\begin{equation}\label{eqn:jacobianSphericalPolar:3}
\begin{aligned}
\frac{\partial (x,y,z)}{\partial (r,\theta,\phi)}
=
\begin{bmatrix}
\PD{r}{x} &\PD{r}{y} &\PD{r}{z}  \\
\PD{\theta}{x} &\PD{\theta}{y} &\PD{\theta}{z}  \\
\PD{\phi}{x} &\PD{\phi}{y} &\PD{\phi}{z}
\end{bmatrix}.
\end{aligned}
\end{equation}

The change of variables for the operator triplet is then just
\begin{equation}\label{eqn:jacobianSphericalPolar:4}
\begin{aligned}
\begin{bmatrix}
\partial_r \\
\partial_\theta \\
\partial_\phi
\end{bmatrix}
= 
\frac{\partial (x,y,z)}{\partial (r,\theta,\phi)}
\begin{bmatrix}
\partial_x \\
\partial_y \\
\partial_z
\end{bmatrix}.
\end{aligned}
\end{equation}

This Jacobian matrix is also not even too hard to calculate.  With \(\Bx = r \rcap\), we have \(x_k = r \rcap \cdot \Be_k\), and

\begin{subequations}
\begin{equation}\label{eqn:jacobianSphericalPolar:5}
\begin{aligned}
\PD{r}{x_k} &= \rcap \cdot \Be_k \\
\PD{\theta}{x_k} &= r \PD{\theta}{\rcap} \cdot \Be_k \\
\PD{\phi}{x_k} &= r \PD{\phi}{\rcap} \cdot \Be_k.
\end{aligned}
\end{equation}
\end{subequations}

The last two derivatives can be calculated easily if the radial unit vector is written out explicitly, with \(S\) and \(C\) for sine and cosine respectively, these are

\begin{subequations}\label{eqn:jacobianSphericalPolar:6}
\begin{equation}\label{eqn:jacobianSphericalPolar:51}
\begin{aligned}
\rcap &= 
\begin{bmatrix}
S_\theta C_\phi \\
S_\theta S_\phi \\
C_\theta 
\end{bmatrix} \\
\PD{\theta}{\rcap} &= 
\begin{bmatrix}
C_\theta C_\phi \\
C_\theta S_\phi \\
-S_\theta 
\end{bmatrix} \\
\PD{\phi}{\rcap} &= 
\begin{bmatrix}
-S_\theta S_\phi \\
S_\theta C_\phi \\
0
\end{bmatrix} 
.
\end{aligned}
\end{equation}
\end{subequations}

We can plug these into the elements of the Jacobian matrix explicitly, which produces

\begin{equation}\label{eqn:jacobianSphericalPolar:7}
\begin{aligned}
\frac{\partial (x,y,z)}{\partial (r,\theta,\phi)}
=
\begin{bmatrix} 
S_\theta C_\phi & S_\theta S_\phi & C_\theta \\
r C_\theta C_\phi & r C_\theta S_\phi & - r S_\theta \\
-r S_\theta S_\phi & rS_\theta C_\phi & 0
\end{bmatrix},
\end{aligned}
\end{equation}

however, we are probably better off just referring back to \eqnref{eqn:jacobianSphericalPolar:6}, and writing 

\begin{equation}\label{eqn:jacobianSphericalPolar:7a}
\begin{aligned}
\frac{\partial (x,y,z)}{\partial (r,\theta,\phi)}
=
\begin{bmatrix} 
\rcap^\T \\
r \PD{\theta}{\rcap^\T} \\
r \PD{\phi}{\rcap^\T} 
\end{bmatrix}.
\end{aligned}
\end{equation}

Unfortunately, this is actually a bit of a dead end.  We really want the inverse of this matrix because the desired quantity is

\begin{equation}\label{eqn:jacobianSphericalPolar:8}
\begin{aligned}
\spacegrad = 
\begin{bmatrix}
\Be_1 & \Be_2 & \Be_3  
\end{bmatrix}
\begin{bmatrix}
\partial_{x_1} \\
\partial_{x_2} \\
\partial_{x_3}
\end{bmatrix}.
\end{aligned}
\end{equation}

(Here my matrix of unit vectors treats these abusively as single elements and not as column vectors).

The matrix of \eqnref{eqn:jacobianSphericalPolar:7a} does not look particularly fun to invert directly, and that is what we need to substitute into
\eqnref{eqn:jacobianSphericalPolar:8}.  One knows that in the end if it was attempted things should mystically simplify (presuming this was done error free).

\section{Cartesian gradient coordinates in terms of spherical polar partials}

Let us flip things upside down and calculate the inverse Jacobian matrix directly.  This is a messier job, but it appears less messy than the matrix inversion above.

\begin{subequations}\label{eqn:jacobianSphericalPolar:9}
\begin{equation}\label{eqn:jacobianSphericalPolar:71}
\begin{aligned}
r^2 &= x^2 + y^2 + z^2  \\
\sin^2 \theta &= \frac{x^2 + y^2}{x^2 + y^2 + z^2} \\
\tan\phi &= \frac{y}{x}
.
\end{aligned}
\end{equation}
\end{subequations}

The messy task is now the calculation of these derivatives.

For the first, from \(r^2 = x^2 + y^2 + z^2\), taking partials on both sides, we have

\begin{equation}\label{eqn:jacobianSphericalPolar:10}
\begin{aligned}
\PD{x_k}{r} = \frac{x_k}{r}.
\end{aligned}
\end{equation}

But these are just the direction cosines, the components of our polar unit vector \(\rcap\).  We can then write for all of these derivatives in column matrix form

\begin{equation}\label{eqn:jacobianSphericalPolar:11}
\begin{aligned}
\spacegrad r = \rcap
\end{aligned}
\end{equation}

Next from \(\sin^2\theta = (x^2 + y^2)/r^2\), we get after some reduction

\begin{subequations}\label{eqn:jacobianSphericalPolar:12}
\begin{equation}\label{eqn:jacobianSphericalPolar:91}
\begin{aligned}
\PD{x}{\theta} &= \inv{r} C_\theta C_\phi \\
\PD{y}{\phi} &= \inv{r} C_\theta S_\phi \\
\PD{z}{\phi} &= -\frac{S_\theta}{r}.
\end{aligned}
\end{equation}
\end{subequations}

Observe that we can antidifferentiate with respect to theta and obtain

\begin{equation}\label{eqn:jacobianSphericalPolar:111}
\begin{aligned}
\spacegrad \theta &= 
\inv{r}
\begin{bmatrix}
C_\theta C_\phi \\
C_\theta S_\phi \\
-S_\theta
\end{bmatrix} \\
&=
\inv{r}
\PD{\theta}{}
\begin{bmatrix}
S_\theta C_\phi \\
S_\theta S_\phi \\
C_\theta
\end{bmatrix}.
\end{aligned}
\end{equation}

This last column vector is our friend the unit polar vector again, and we have

\begin{equation}\label{eqn:jacobianSphericalPolar:13}
\begin{aligned}
\spacegrad \theta &= 
\inv{r}
\PD{\theta}{\rcap}
\end{aligned}
\end{equation}

Finally for the \(\phi\) dependence we have after some reduction

\begin{equation}\label{eqn:jacobianSphericalPolar:14}
\begin{aligned}
\spacegrad \phi &=
\inv{r S_\theta}
\begin{bmatrix}
-S_\phi \\
C_\phi \\
0
\end{bmatrix}.
\end{aligned}
\end{equation}

Again, we can antidifferentiate

\begin{equation}\label{eqn:jacobianSphericalPolar:131}
\begin{aligned}
\spacegrad \phi 
&=
\inv{r (S_\theta)^2}
\begin{bmatrix}
-S_\theta S_\phi \\
S_\theta C_\phi \\
0
\end{bmatrix} \\
&=
\inv{r (S_\theta)^2}
\PD{\phi}{}
\begin{bmatrix}
S_\theta C_\phi \\
S_\theta S_\phi \\
C_\theta
\end{bmatrix}.
\end{aligned}
\end{equation}

We have our unit polar vector again, and our \(\phi\) partials nicely summarized by

\begin{equation}\label{eqn:jacobianSphericalPolar:15}
\begin{aligned}
\spacegrad \phi 
&=
\inv{r (S_\theta)^2}
\PD{\phi}{\rcap}.
\end{aligned}
\end{equation}

With this we can now write out the Jacobian matrix either explicitly, or in column vector form in terms of \(\rcap\).  First a reminder of why we want this matrix, for the following change of variables

\begin{equation}\label{eqn:jacobianSphericalPolar:16}
\begin{aligned}
\begin{bmatrix}
\partial_x \\
\partial_y \\
\partial_z
\end{bmatrix}
= 
\begin{bmatrix}
\PD{x}{r} &\PD{x}{\theta} &\PD{x}{\phi}  \\
\PD{y}{r} &\PD{y}{\theta} &\PD{y}{\phi}  \\
\PD{z}{r} &\PD{z}{\theta} &\PD{z}{\phi} 
\end{bmatrix}
\begin{bmatrix}
\partial_r \\
\partial_\theta \\
\partial_\phi
\end{bmatrix}
.
\end{aligned}
\end{equation}

We want the Jacobian matrix

\begin{equation}\label{eqn:jacobianSphericalPolar:17}
\begin{aligned}
\frac{\partial (r,\theta,\phi)}{\partial (x, y, z)}
=
\begin{bmatrix}
\spacegrad r & \spacegrad \theta & \spacegrad \phi
\end{bmatrix}
=
\begin{bmatrix}
\rcap & \inv{r} \PD{\theta}{\rcap} & \inv{r \sin^2\theta} \PD{\phi}{\rcap}
\end{bmatrix}.
\end{aligned}
\end{equation}

Explicitly this is

\begin{equation}\label{eqn:jacobianSphericalPolar:18}
\begin{aligned}
\frac{\partial (r,\theta,\phi)}{\partial (x, y, z)}
=
\begin{bmatrix}
S_\theta C_\phi & \inv{r} C_\theta C_\phi & -\inv{r S_\theta} S_\phi \\
S_\theta S_\phi & \inv{r} C_\theta S_\phi & \frac{C_\phi}{r S_\theta} \\
C_\theta        & -\inv{r} S_\theta       &  0
\end{bmatrix}.
\end{aligned}
\end{equation} 

As a verification of correctness multiplication of this with \eqnref{eqn:jacobianSphericalPolar:7} should produce identity.  That is a mess of trig that I do not really feel like trying, but we can get a rough idea why it should all be the identity matrix by multiplying it out in block matrix form

\begin{equation}\label{eqn:jacobianSphericalPolar:151}
\begin{aligned}
\frac{\partial (x,y,z)}{\partial (r,\theta,\phi)}
\frac{\partial (r,\theta,\phi)}{\partial (x, y, z)}
&=
\begin{bmatrix} 
\rcap^\T \\
r \PD{\theta}{\rcap^\T} \\
r \PD{\phi}{\rcap^\T} 
\end{bmatrix}
\begin{bmatrix}
\rcap & \inv{r} \PD{\theta}{\rcap} & \inv{r \sin^2\theta} \PD{\phi}{\rcap}
\end{bmatrix} \\
&=
\begin{bmatrix} 
\rcap^\T \rcap                & \inv{r} \rcap^\T \PD{\theta}{\rcap}      & \inv{r \sin^2 \theta} \rcap^\T \PD{\phi}{\rcap} \\
r \PD{\theta}{\rcap^\T} \rcap & \PD{\theta}{\rcap^\T} \PD{\theta}{\rcap} & \inv{\sin^2\theta} \PD{\theta}{\rcap^\T} \PD{\phi}{\rcap} \\
r \PD{\phi}{\rcap^\T} \rcap   & \PD{\phi}{\rcap^\T} \PD{\theta}{\rcap}   & \inv{\sin^2\theta} \PD{\phi}{\rcap^\T} \PD{\phi}{\rcap}
\end{bmatrix}.
\end{aligned}
\end{equation}

The derivatives are vectors that lie tangential to the unit sphere.  We can calculate this to verify, or we can look at the off diagonal terms which say just this if we trust the math that says these should all be zeros.  For each of the off diagonal terms to be zero must mean that we have

\begin{equation}\label{eqn:jacobianSphericalPolar:19}
\begin{aligned}
0 = \PD{\theta}{\rcap} \cdot \rcap = \PD{\theta}{\rcap} \cdot \PD{\phi}{\rcap} = \PD{\phi}{\rcap} \cdot \rcap 
\end{aligned}
\end{equation}

This makes intuitive sense.  We can also verify quickly enough that \((\PDi{\theta}{\rcap})^2 = 1\), and \((\PDi{\phi}{\rcap})^2 = \sin^2\theta\) (I did this with a back of the envelope calculation using geometric algebra).  That is consistent with what this matrix product implies it should equal.

\section{Completing the gradient change of variables to spherical polar coordinates}

We are now set to calculate the gradient in spherical polar coordinates from our Cartesian representation.  From \eqnref{eqn:jacobianSphericalPolar:8} and
\eqnref{eqn:jacobianSphericalPolar:16}, and \eqnref{eqn:jacobianSphericalPolar:17} we have

\begin{equation}\label{eqn:jacobianSphericalPolar:20}
\begin{aligned}
\spacegrad =
\begin{bmatrix}
\Be_1 & \Be_2 & \Be_3  
\end{bmatrix}
\begin{bmatrix}
\rcap \cdot \Be_1 & \inv{r} \PD{\theta}{\rcap} \cdot \Be_1 & \inv{r \sin^2\theta} \PD{\phi}{\rcap} \cdot \Be_1 \\
\rcap \cdot \Be_2 & \inv{r} \PD{\theta}{\rcap} \cdot \Be_2 & \inv{r \sin^2\theta} \PD{\phi}{\rcap} \cdot \Be_2 \\
\rcap \cdot \Be_3 & \inv{r} \PD{\theta}{\rcap} \cdot \Be_3 & \inv{r \sin^2\theta} \PD{\phi}{\rcap} \cdot \Be_3 
\end{bmatrix}
\begin{bmatrix}
\partial_r \\
\partial_\theta \\
\partial_\phi
\end{bmatrix}.
\end{aligned}
\end{equation}

The Jacobian matrix has been written out explicitly as scalars because we are now switching to an abusive notation using matrices of vector elements.  Our Jacobian, a matrix of scalars happened to have a nice compact representation in column vector form, but we cannot use this when multiplying out with our matrix elements (or perhaps could if we invented more conventions, but lets avoid that).  Having written it out in full we see that we recover our original compact Jacobian representation, and have just

\begin{equation}\label{eqn:jacobianSphericalPolar:21}
\begin{aligned}
\spacegrad = 
\begin{bmatrix}
\rcap & \inv{r} \PD{\theta}{\rcap} & \inv{r \sin^2\theta} \PD{\phi}{\rcap} 
\end{bmatrix}
\begin{bmatrix}
\partial_r \\
\partial_\theta \\
\partial_\phi
\end{bmatrix}.
\end{aligned}
\end{equation}

Expanding this last product we have the gradient in its spherical polar representation

\begin{equation}\label{eqn:jacobianSphericalPolar:22}
\begin{aligned}
\spacegrad = 
\begin{bmatrix}
\rcap \PD{r}{} + \inv{r} \PD{\theta}{\rcap} \PD{\theta}{} + \inv{r \sin\theta} \inv{\sin\theta} \PD{\phi}{\rcap} \PD{\phi}{}
\end{bmatrix}.
\end{aligned}
\end{equation}

With the labels 
\begin{subequations}
\begin{equation}\label{eqn:jacobianSphericalPolar:23}
\begin{aligned}
\thetacap &= \PD{\theta}{\rcap} \\
\phicap &= \inv{\sin\theta} \PD{\phi}{\rcap},
\end{aligned}
\end{equation}
\end{subequations}

(having confirmed that these are unit vectors), we have the final result for the gradient in this representation

\begin{equation}\label{eqn:jacobianSphericalPolar:24}
\begin{aligned}
\spacegrad = 
\rcap \PD{r}{} + \inv{r} \thetacap \PD{\theta}{} + \inv{r \sin\theta} \phicap \PD{\phi}{}.
\end{aligned}
\end{equation}

Here the matrix delimiters for the remaining one by one matrix term were also dropped.

\section{General expression for gradient in orthonormal frames}

Having done the computation for the spherical polar case, we get the result for any orthonormal frame for free.  That is just

\begin{equation}\label{eqn:jacobianSphericalPolar:30}
\begin{aligned}
\spacegrad = \sum_i (\spacegrad q_i) \PD{q_i}{}.
\end{aligned}
\end{equation}

From each of the gradients we can factor out a unit vector in the direction of the gradient, and have an expression that structurally has the same form as \eqnref{eqn:jacobianSphericalPolar:24}.  Writing \(\hat{\Bq}_i = (\spacegrad q_i)/\Abs{\spacegrad q_i}\), this is

\begin{equation}\label{eqn:jacobianSphericalPolar:31}
\begin{aligned}
\spacegrad = \sum_i \Abs{\spacegrad q_i} \hat{\Bq}_i \PD{q_i}{}.
\end{aligned}
\end{equation}

These individual direction gradients are not necessarily easy to compute.  The procedures outlined in \citep{byron1992mca} are a more effective way of dealing with this general computational task.  However, if we want, we can at proceed this dumb obvious way and be able to get the desired result knowing only how to apply the chain rule, and the Cartesian definition of the gradient.

\EndArticle


\part{Feynman}
% 
% 
% 
% Copyright � 2012 Peeter Joot
% All Rights Reserved
% 
% This file may be reproduced and distributed in whole or in part, without fee, subject to the following conditions:
% 
% o The copyright notice above and this permission notice must be preserved complete on all complete or partial copies.
% 
% o Any translation or derived work must be approved by the author in writing before distribution.
% 
% o If you distribute this work in part, instructions for obtaining the complete version of this file must be included, and a means for obtaining a complete version provided.
% 
% 
% Exceptions to these rules may be granted for academic purposes: Write to the author and ask.
% 
% 
% 
%\documentclass{article}

%\usepackage{amsmath}
\usepackage{mathpazo}

%
% shorthand for bold symbols, convenient for vectors and matrices
%
\newcommand{\Ba}[0]{\mathbf{a}}
\newcommand{\Bb}[0]{\mathbf{b}}
\newcommand{\Bc}[0]{\mathbf{c}}
\newcommand{\Bd}[0]{\mathbf{d}}
\newcommand{\Be}[0]{\mathbf{e}}
\newcommand{\Bf}[0]{\mathbf{f}}
\newcommand{\Bg}[0]{\mathbf{g}}
\newcommand{\Bh}[0]{\mathbf{h}}
\newcommand{\Bi}[0]{\mathbf{i}}
\newcommand{\Bj}[0]{\mathbf{j}}
\newcommand{\Bk}[0]{\mathbf{k}}
\newcommand{\Bl}[0]{\mathbf{l}}
\newcommand{\Bm}[0]{\mathbf{m}}
\newcommand{\Bn}[0]{\mathbf{n}}
\newcommand{\Bo}[0]{\mathbf{o}}
\newcommand{\Bp}[0]{\mathbf{p}}
\newcommand{\Bq}[0]{\mathbf{q}}
\newcommand{\Br}[0]{\mathbf{r}}
\newcommand{\Bs}[0]{\mathbf{s}}
\newcommand{\Bt}[0]{\mathbf{t}}
\newcommand{\Bu}[0]{\mathbf{u}}
\newcommand{\Bv}[0]{\mathbf{v}}
\newcommand{\Bw}[0]{\mathbf{w}}
\newcommand{\Bx}[0]{\mathbf{x}}
\newcommand{\By}[0]{\mathbf{y}}
\newcommand{\Bz}[0]{\mathbf{z}}
\newcommand{\BA}[0]{\mathbf{A}}
\newcommand{\BB}[0]{\mathbf{B}}
\newcommand{\BC}[0]{\mathbf{C}}
\newcommand{\BD}[0]{\mathbf{D}}
\newcommand{\BE}[0]{\mathbf{E}}
\newcommand{\BF}[0]{\mathbf{F}}
\newcommand{\BG}[0]{\mathbf{G}}
\newcommand{\BH}[0]{\mathbf{H}}
\newcommand{\BI}[0]{\mathbf{I}}
\newcommand{\BJ}[0]{\mathbf{J}}
\newcommand{\BK}[0]{\mathbf{K}}
\newcommand{\BL}[0]{\mathbf{L}}
\newcommand{\BM}[0]{\mathbf{M}}
\newcommand{\BN}[0]{\mathbf{N}}
\newcommand{\BO}[0]{\mathbf{O}}
\newcommand{\BP}[0]{\mathbf{P}}
\newcommand{\BQ}[0]{\mathbf{Q}}
\newcommand{\BR}[0]{\mathbf{R}}
\newcommand{\BS}[0]{\mathbf{S}}
\newcommand{\BT}[0]{\mathbf{T}}
\newcommand{\BU}[0]{\mathbf{U}}
\newcommand{\BV}[0]{\mathbf{V}}
\newcommand{\BW}[0]{\mathbf{W}}
\newcommand{\BX}[0]{\mathbf{X}}
\newcommand{\BY}[0]{\mathbf{Y}}
\newcommand{\BZ}[0]{\mathbf{Z}}

\newcommand{\Bzero}[0]{\mathbf{0}}
\newcommand{\Btheta}[0]{\boldsymbol{\theta}}
\newcommand{\Btau}[0]{\boldsymbol{\tau}}
\newcommand{\Bomega}[0]{\boldsymbol{\omega}}

%
% shorthand for unit vectors
%
\newcommand{\acap}[0]{\hat{\Ba}}
\newcommand{\bcap}[0]{\hat{\Bb}}
\newcommand{\ccap}[0]{\hat{\Bc}}
\newcommand{\dcap}[0]{\hat{\Bd}}
\newcommand{\ecap}[0]{\hat{\Be}}
\newcommand{\fcap}[0]{\hat{\Bf}}
\newcommand{\gcap}[0]{\hat{\Bg}}
\newcommand{\hcap}[0]{\hat{\Bh}}
\newcommand{\icap}[0]{\hat{\Bi}}
\newcommand{\jcap}[0]{\hat{\Bj}}
\newcommand{\kcap}[0]{\hat{\Bk}}
\newcommand{\lcap}[0]{\hat{\Bl}}
\newcommand{\mcap}[0]{\hat{\Bm}}
\newcommand{\ncap}[0]{\hat{\Bn}}
\newcommand{\ocap}[0]{\hat{\Bo}}
\newcommand{\pcap}[0]{\hat{\Bp}}
\newcommand{\qcap}[0]{\hat{\Bq}}
\newcommand{\rcap}[0]{\hat{\Br}}
\newcommand{\scap}[0]{\hat{\Bs}}
\newcommand{\tcap}[0]{\hat{\Bt}}
\newcommand{\ucap}[0]{\hat{\Bu}}
\newcommand{\vcap}[0]{\hat{\Bv}}
\newcommand{\wcap}[0]{\hat{\Bw}}
\newcommand{\xcap}[0]{\hat{\Bx}}
\newcommand{\ycap}[0]{\hat{\By}}
\newcommand{\zcap}[0]{\hat{\Bz}}
\newcommand{\thetacap}[0]{\hat{\Btheta}}

%
% to write R^n and C^n in a distinguishable fashion.  Perhaps change this
% to the double lined characters upon figuring out how to do so.
%
\newcommand{\C}[1]{$\mathbb{C}^{#1}$}
\newcommand{\R}[1]{$\mathbb{R}^{#1}$}

%
% various generally useful helpers
%

% derivative of #1 wrt. #2:
\newcommand{\D}[2] {\frac {d#2} {d#1}}

\newcommand{\inv}[1]{\frac{1}{#1}}
\newcommand{\cross}[0]{\times}

\newcommand{\abs}[1]{\lvert{#1}\rvert}
\newcommand{\norm}[1]{\lVert{#1}\rVert}
\newcommand{\innerprod}[2]{\langle{#1}, {#2}\rangle}
\newcommand{\dotprod}[2]{{#1} \cdot {#2}}
\newcommand{\bdotprod}[2]{\left({#1} \cdot {#2}\right)}
\newcommand{\crossprod}[2]{{#1} \cross {#2}}
\newcommand{\tripleprod}[3]{\dotprod{\left(\crossprod{#1}{#2}\right)}{#3}}

\DeclareMathOperator{\Proj}{Proj}
\DeclareMathOperator{\Span}{span}
\DeclareMathOperator{\Sgn}{sgn}
\DeclareMathOperator{\Area}{Area}
\DeclareMathOperator{\Volume}{Volume}

%
% A few miscellaneous things specific to this document
%
\newcommand{\crossop}[1]{\crossprod{#1}{}}

% R2 vector.
\newcommand{\VectorTwo}[2]{
\begin{bmatrix}
 {#1} \\
 {#2}
\end{bmatrix}
}

\newcommand{\VectorN}[1]{
\begin{bmatrix}
{#1}_1 \\
{#1}_2 \\
\vdots \\
{#1}_N \\
\end{bmatrix}
}

\newcommand{\DETuvij}[4]{
\begin{vmatrix}
 {#1}_{#3} & {#1}_{#4} \\
 {#2}_{#3} & {#2}_{#4}
\end{vmatrix}
}

\newcommand{\DETuvwijk}[6]{
\begin{vmatrix}
 {#1}_{#4} & {#1}_{#5} & {#1}_{#6} \\
 {#2}_{#4} & {#2}_{#5} & {#2}_{#6} \\
 {#3}_{#4} & {#3}_{#5} & {#3}_{#6}
\end{vmatrix}
}

\newcommand{\DETuvwxijkl}[8]{
\begin{vmatrix}
 {#1}_{#5} & {#1}_{#6} & {#1}_{#7} & {#1}_{#8} \\
 {#2}_{#5} & {#2}_{#6} & {#2}_{#7} & {#2}_{#8} \\
 {#3}_{#5} & {#3}_{#6} & {#3}_{#7} & {#3}_{#8} \\
 {#4}_{#5} & {#4}_{#6} & {#4}_{#7} & {#4}_{#8} \\
\end{vmatrix}
}

%\newcommand{\DETuvwxyijklm}[10]{
%\begin{vmatrix}
% {#1}_{#6} & {#1}_{#7} & {#1}_{#8} & {#1}_{#9} & {#1}_{#10} \\
% {#2}_{#6} & {#2}_{#7} & {#2}_{#8} & {#2}_{#9} & {#2}_{#10} \\
% {#3}_{#6} & {#3}_{#7} & {#3}_{#8} & {#3}_{#9} & {#3}_{#10} \\
% {#4}_{#6} & {#4}_{#7} & {#4}_{#8} & {#4}_{#9} & {#4}_{#10} \\
% {#5}_{#6} & {#5}_{#7} & {#5}_{#8} & {#5}_{#9} & {#5}_{#10}
%\end{vmatrix}
%}

% R3 vector.
\newcommand{\VectorThree}[3]{
\begin{bmatrix}
 {#1} \\
 {#2} \\
 {#3}
\end{bmatrix}
}


%%<misc>
%
\newcommand{\Abs}[1]{{\left\lvert{#1}\right\rvert}}
\newcommand{\spacegrad}[0]{\boldsymbol{\nabla}}
\newcommand{\grad}[0]{\nabla}
\newcommand{\LL}[0]{\mathcal{L}}

% == \partial_{#1} {#2}
\newcommand{\PD}[2]{\frac{\partial {#2}}{\partial {#1}}}
% inline variant
\newcommand{\PDi}[2]{{\partial {#2}}/{\partial {#1}}}

\newcommand{\PDD}[3]{\frac{\partial^2 {#3}}{\partial {#1}\partial {#2}}}
%\newcommand{\PDd}[2]{\frac{\partial^2 {#2}}{{\partial{#1}}^2}}
\newcommand{\PDsq}[2]{\frac{\partial^2 {#2}}{(\partial {#1})^2}}

\newcommand{\Partial}[2]{\frac{\partial {#1}}{\partial {#2}}}
\DeclareMathOperator{\RejName}{Rej}
\newcommand{\Rej}[2]{\RejName_{#1}\left( {#2} \right)}
\newcommand{\Rm}[1]{\mathbb{R}^{#1}}
\newcommand{\Cm}[1]{\mathbb{C}^{#1}}
\newcommand{\conj}[0]{{*}}

%</misc>

% <grade selection>
%
\newcommand{\gpgrade}[2] {{\left\langle{{#1}}\right\rangle}_{#2}}

\newcommand{\gpgradezero}[1] {\gpgrade{#1}{}}
%\newcommand{\gpscalargrade}[1] {{\left\langle{{#1}}\right\rangle}}
%\newcommand{\gpgradezero}[1] {\gpgrade{#1}{0}}

%\newcommand{\gpgradeone}[1] {{\left\langle{{#1}}\right\rangle}_{1}}
\newcommand{\gpgradeone}[1] {\gpgrade{#1}{1}}

\newcommand{\gpgradetwo}[1] {\gpgrade{#1}{2}}
\newcommand{\gpgradethree}[1] {\gpgrade{#1}{3}}
\newcommand{\gpgradefour}[1] {\gpgrade{#1}{4}}
%
% </grade selection>



\newcommand{\adot}[0]{{\dot{a}}}
\newcommand{\bdot}[0]{{\dot{b}}}
% taken for centered dot:
%\newcommand{\cdot}[0]{{\dot{c}}}
%\newcommand{\ddot}[0]{{\dot{d}}}
\newcommand{\edot}[0]{{\dot{e}}}
\newcommand{\fdot}[0]{{\dot{f}}}
\newcommand{\gdot}[0]{{\dot{g}}}
\newcommand{\hdot}[0]{{\dot{h}}}
\newcommand{\idot}[0]{{\dot{i}}}
\newcommand{\jdot}[0]{{\dot{j}}}
\newcommand{\kdot}[0]{{\dot{k}}}
\newcommand{\ldot}[0]{{\dot{l}}}
\newcommand{\mdot}[0]{{\dot{m}}}
\newcommand{\ndot}[0]{{\dot{n}}}
%\newcommand{\odot}[0]{{\dot{o}}}
\newcommand{\pdot}[0]{{\dot{p}}}
\newcommand{\qdot}[0]{{\dot{q}}}
\newcommand{\rdot}[0]{{\dot{r}}}
\newcommand{\sdot}[0]{{\dot{s}}}
\newcommand{\tdot}[0]{{\dot{t}}}
\newcommand{\udot}[0]{{\dot{u}}}
\newcommand{\vdot}[0]{{\dot{v}}}
\newcommand{\wdot}[0]{{\dot{w}}}
\newcommand{\xdot}[0]{{\dot{x}}}
\newcommand{\ydot}[0]{{\dot{y}}}
\newcommand{\zdot}[0]{{\dot{z}}}
\newcommand{\addot}[0]{{\ddot{a}}}
\newcommand{\bddot}[0]{{\ddot{b}}}
\newcommand{\cddot}[0]{{\ddot{c}}}
%\newcommand{\dddot}[0]{{\ddot{d}}}
\newcommand{\eddot}[0]{{\ddot{e}}}
\newcommand{\fddot}[0]{{\ddot{f}}}
\newcommand{\gddot}[0]{{\ddot{g}}}
\newcommand{\hddot}[0]{{\ddot{h}}}
\newcommand{\iddot}[0]{{\ddot{i}}}
\newcommand{\jddot}[0]{{\ddot{j}}}
\newcommand{\kddot}[0]{{\ddot{k}}}
\newcommand{\lddot}[0]{{\ddot{l}}}
\newcommand{\mddot}[0]{{\ddot{m}}}
\newcommand{\nddot}[0]{{\ddot{n}}}
\newcommand{\oddot}[0]{{\ddot{o}}}
\newcommand{\pddot}[0]{{\ddot{p}}}
\newcommand{\qddot}[0]{{\ddot{q}}}
\newcommand{\rddot}[0]{{\ddot{r}}}
\newcommand{\sddot}[0]{{\ddot{s}}}
\newcommand{\tddot}[0]{{\ddot{t}}}
\newcommand{\uddot}[0]{{\ddot{u}}}
\newcommand{\vddot}[0]{{\ddot{v}}}
\newcommand{\wddot}[0]{{\ddot{w}}}
\newcommand{\xddot}[0]{{\ddot{x}}}
\newcommand{\yddot}[0]{{\ddot{y}}}
\newcommand{\zddot}[0]{{\ddot{z}}}

%<bold and dot greek symbols>
%

\newcommand{\Deltadot}[0]{{\dot{\Delta}}}
\newcommand{\Gammadot}[0]{{\dot{\Gamma}}}
\newcommand{\Lambdadot}[0]{{\dot{\Lambda}}}
\newcommand{\Omegadot}[0]{{\dot{\Omega}}}
\newcommand{\Phidot}[0]{{\dot{\Phi}}}
\newcommand{\Pidot}[0]{{\dot{\Pi}}}
\newcommand{\Psidot}[0]{{\dot{\Psi}}}
\newcommand{\Sigmadot}[0]{{\dot{\Sigma}}}
\newcommand{\Thetadot}[0]{{\dot{\Theta}}}
\newcommand{\Upsilondot}[0]{{\dot{\Upsilon}}}
\newcommand{\Xidot}[0]{{\dot{\Xi}}}
\newcommand{\alphadot}[0]{{\dot{\alpha}}}
\newcommand{\betadot}[0]{{\dot{\beta}}}
\newcommand{\chidot}[0]{{\dot{\chi}}}
\newcommand{\deltadot}[0]{{\dot{\delta}}}
\newcommand{\epsilondot}[0]{{\dot{\epsilon}}}
\newcommand{\etadot}[0]{{\dot{\eta}}}
\newcommand{\gammadot}[0]{{\dot{\gamma}}}
\newcommand{\kappadot}[0]{{\dot{\kappa}}}
\newcommand{\lambdadot}[0]{{\dot{\lambda}}}
\newcommand{\mudot}[0]{{\dot{\mu}}}
\newcommand{\nudot}[0]{{\dot{\nu}}}
\newcommand{\omegadot}[0]{{\dot{\omega}}}
\newcommand{\phidot}[0]{{\dot{\phi}}}
\newcommand{\pidot}[0]{{\dot{\pi}}}
\newcommand{\psidot}[0]{{\dot{\psi}}}
\newcommand{\rhodot}[0]{{\dot{\rho}}}
\newcommand{\sigmadot}[0]{{\dot{\sigma}}}
\newcommand{\taudot}[0]{{\dot{\tau}}}
\newcommand{\thetadot}[0]{{\dot{\theta}}}
\newcommand{\upsilondot}[0]{{\dot{\upsilon}}}
\newcommand{\varepsilondot}[0]{{\dot{\varepsilon}}}
\newcommand{\varphidot}[0]{{\dot{\varphi}}}
\newcommand{\varpidot}[0]{{\dot{\varpi}}}
\newcommand{\varrhodot}[0]{{\dot{\varrho}}}
\newcommand{\varsigmadot}[0]{{\dot{\varsigma}}}
\newcommand{\varthetadot}[0]{{\dot{\vartheta}}}
\newcommand{\xidot}[0]{{\dot{\xi}}}
\newcommand{\zetadot}[0]{{\dot{\zeta}}}

\newcommand{\Deltaddot}[0]{{\ddot{\Delta}}}
\newcommand{\Gammaddot}[0]{{\ddot{\Gamma}}}
\newcommand{\Lambdaddot}[0]{{\ddot{\Lambda}}}
\newcommand{\Omegaddot}[0]{{\ddot{\Omega}}}
\newcommand{\Phiddot}[0]{{\ddot{\Phi}}}
\newcommand{\Piddot}[0]{{\ddot{\Pi}}}
\newcommand{\Psiddot}[0]{{\ddot{\Psi}}}
\newcommand{\Sigmaddot}[0]{{\ddot{\Sigma}}}
\newcommand{\Thetaddot}[0]{{\ddot{\Theta}}}
\newcommand{\Upsilonddot}[0]{{\ddot{\Upsilon}}}
\newcommand{\Xiddot}[0]{{\ddot{\Xi}}}
\newcommand{\alphaddot}[0]{{\ddot{\alpha}}}
\newcommand{\betaddot}[0]{{\ddot{\beta}}}
\newcommand{\chiddot}[0]{{\ddot{\chi}}}
\newcommand{\deltaddot}[0]{{\ddot{\delta}}}
\newcommand{\epsilonddot}[0]{{\ddot{\epsilon}}}
\newcommand{\etaddot}[0]{{\ddot{\eta}}}
\newcommand{\gammaddot}[0]{{\ddot{\gamma}}}
\newcommand{\kappaddot}[0]{{\ddot{\kappa}}}
\newcommand{\lambdaddot}[0]{{\ddot{\lambda}}}
\newcommand{\muddot}[0]{{\ddot{\mu}}}
\newcommand{\nuddot}[0]{{\ddot{\nu}}}
\newcommand{\omegaddot}[0]{{\ddot{\omega}}}
\newcommand{\phiddot}[0]{{\ddot{\phi}}}
\newcommand{\piddot}[0]{{\ddot{\pi}}}
\newcommand{\psiddot}[0]{{\ddot{\psi}}}
\newcommand{\rhoddot}[0]{{\ddot{\rho}}}
\newcommand{\sigmaddot}[0]{{\ddot{\sigma}}}
\newcommand{\tauddot}[0]{{\ddot{\tau}}}
\newcommand{\thetaddot}[0]{{\ddot{\theta}}}
\newcommand{\upsilonddot}[0]{{\ddot{\upsilon}}}
\newcommand{\varepsilonddot}[0]{{\ddot{\varepsilon}}}
\newcommand{\varphiddot}[0]{{\ddot{\varphi}}}
\newcommand{\varpiddot}[0]{{\ddot{\varpi}}}
\newcommand{\varrhoddot}[0]{{\ddot{\varrho}}}
\newcommand{\varsigmaddot}[0]{{\ddot{\varsigma}}}
\newcommand{\varthetaddot}[0]{{\ddot{\vartheta}}}
\newcommand{\xiddot}[0]{{\ddot{\xi}}}
\newcommand{\zetaddot}[0]{{\ddot{\zeta}}}

\newcommand{\BDelta}[0]{\boldsymbol{\Delta}}
\newcommand{\BGamma}[0]{\boldsymbol{\Gamma}}
\newcommand{\BLambda}[0]{\boldsymbol{\Lambda}}
\newcommand{\BOmega}[0]{\boldsymbol{\Omega}}
\newcommand{\BPhi}[0]{\boldsymbol{\Phi}}
\newcommand{\BPi}[0]{\boldsymbol{\Pi}}
\newcommand{\BPsi}[0]{\boldsymbol{\Psi}}
\newcommand{\BSigma}[0]{\boldsymbol{\Sigma}}
\newcommand{\BTheta}[0]{\boldsymbol{\Theta}}
\newcommand{\BUpsilon}[0]{\boldsymbol{\Upsilon}}
\newcommand{\BXi}[0]{\boldsymbol{\Xi}}
\newcommand{\Balpha}[0]{\boldsymbol{\alpha}}
\newcommand{\Bbeta}[0]{\boldsymbol{\beta}}
\newcommand{\Bchi}[0]{\boldsymbol{\chi}}
\newcommand{\Bdelta}[0]{\boldsymbol{\delta}}
\newcommand{\Bepsilon}[0]{\boldsymbol{\epsilon}}
\newcommand{\Beta}[0]{\boldsymbol{\eta}}
\newcommand{\Bgamma}[0]{\boldsymbol{\gamma}}
\newcommand{\Bkappa}[0]{\boldsymbol{\kappa}}
\newcommand{\Blambda}[0]{\boldsymbol{\lambda}}
\newcommand{\Bmu}[0]{\boldsymbol{\mu}}
\newcommand{\Bnu}[0]{\boldsymbol{\nu}}
%\newcommand{\Bomega}[0]{\boldsymbol{\omega}}
\newcommand{\Bphi}[0]{\boldsymbol{\phi}}
\newcommand{\Bpi}[0]{\boldsymbol{\pi}}
\newcommand{\Bpsi}[0]{\boldsymbol{\psi}}
\newcommand{\Brho}[0]{\boldsymbol{\rho}}
\newcommand{\Bsigma}[0]{\boldsymbol{\sigma}}
%\newcommand{\Btau}[0]{\boldsymbol{\tau}}
%\newcommand{\Btheta}[0]{\boldsymbol{\theta}}
\newcommand{\Bupsilon}[0]{\boldsymbol{\upsilon}}
\newcommand{\Bvarepsilon}[0]{\boldsymbol{\varepsilon}}
\newcommand{\Bvarphi}[0]{\boldsymbol{\varphi}}
\newcommand{\Bvarpi}[0]{\boldsymbol{\varpi}}
\newcommand{\Bvarrho}[0]{\boldsymbol{\varrho}}
\newcommand{\Bvarsigma}[0]{\boldsymbol{\varsigma}}
\newcommand{\Bvartheta}[0]{\boldsymbol{\vartheta}}
\newcommand{\Bxi}[0]{\boldsymbol{\xi}}
\newcommand{\Bzeta}[0]{\boldsymbol{\zeta}}
%
%</bold and dot greek symbols>
%<infrequent>
%
%\newcommand{\AreaOp}[1]{\AName_{#1}}
%\newcommand{\Babs}[0]{\abs{\BB}}
%\newcommand{\Bcap}[0]{\hat{\BB}}
%\newcommand{\BrPrimeRej}[0]{\rcap(\rcap \wedge \Br')}
%\newcommand{\CA}[0]{\mathcal{A}}
%\newcommand{\Cos}[1]{\cos{\left({#1}\right)}}
%\newcommand{\Det}[1] {\abs{#1}}
%\newcommand{\Dsq}[2] {\frac {\partial^2 {#1}} {\partial {#2}^2}}
%\newcommand{\Exp}[1]{\exp{\left({#1}\right)}}
%\newcommand{\Norm}[1]{\left\lVert{#1}\right\rVert}
%\newcommand{\Sin}[1]{\sin{\left({#1}\right)}}
%\newcommand{\T}[0]{\text{T}}
%\newcommand{\VolumeOp}[1]{\VName_{#1}}
%\newcommand{\agrad}[0]{\Ba \cdot \nabla}
%\newcommand{\alphacap}[0]{\hat{\boldsymbol{\alpha}}}
%\newcommand{\Fcap}[0]{\hat{\BF}}
%\newcommand{\bithree}[0]{{\Bi}_3}
%\newcommand{\bxa}[0]{\Bx\Ba}
%\newcommand{\coordvec}[2]{
%\newcommand{\costheta}[0]{\acap \cdot \xcap}
%\newcommand{\ddt}[1]{\ddot{#1}}
%\newcommand{\ddu}[1] {\frac {d{#1}} {du}}
%\newcommand{\dsqxj}[2] {\frac {\partial^2 {#1}} {\partial {x_{#2}}^2}}
%\newcommand{\dtheta}[1]{\frac{d {#1}}{d \theta}}
%\newcommand{\dt}[1]{\dot{#1}}
%\newcommand{\dt}[1]{\frac{d {#1}}{dt}}
%\newcommand{\dxj}[2] {\frac {\partial {#1}} {\partial {x_{#2}}}}
%\newcommand{\halfPhi}[0]{\frac{\phi}{2}}
%\newcommand{\half}[0]{\inv{2}}
%\newcommand{\inv}[1]{\frac{1}{#1}}
%\newcommand{\laplacian}[0]{\nabla^2}
%\newcommand{\matrixoftx}[3]{
%\newcommand{\nrrp}[0]{\norm{\rcap \wedge \Br'}}
%\newcommand{\oiint}{\bigcirc \hspace{-1.4em} \int \hspace{-.8em} \int}
%\newcommand{\transpose}[1]{{#1}^{\text{T}}}
%\newcommand{\transpose}[1]{{{#1}^{\TextTranspose}}}
%\newcommand{\transpose}[1]{{{#1}^{\text{T}}}}
%\newcommand{\barA}[0]{\bar{A}}
%\newcommand{\qbar}[0]{\bar{q}}
%\newcommand{\qdotbar}[0]{\dot{\bar{q}}}
%
%</infrequent>





%\usepackage[bookmarks=true]{hyperref}

%\usepackage{color,cite,graphicx}
   % use colour in the document, put your citations as [1-4]
   % rather than [1,2,3,4] (it looks nicer, and the extended LaTeX2e
   % graphics package. 
%\usepackage{latexsym,amssymb,epsf} % don't remember if these are
   % needed, but their inclusion can't do any damage

\chapter{Derive the star distance calculation in the Feynman lectures. }
\label{chap:starDistance}
%\author{Peeter Joot \quad peeter.joot@gmail.com}
\date{ Jan 17, 2000.  starDistance.tex }

%\begin{document}

%\maketitle{}

%\tableofcontents
\section{Motivation. }

\begin{figure}[htp]
\centering
\includegraphics[totalheight=0.4\textheight]{feynman_star}
\caption{Triangulating the distance to a star}\label{fig:feynman_star}
\end{figure}

Derivation of the "locate Sputnik" formula of Fig 5-5 in \citep{feynman1963flp}, as illustrated in \ref{fig:feynman_star}.

\section{}

Elliptical orbit $x^2/a^2 + y^2/b^2 = c^2$, with orbital diameter $2ac = L$.  Angles to the star, measured relative to the Sun are $\alpha$, and $\pi - (\alpha+\epsilon)$.  Using the sine rule for triangles, 

\begin{displaymath}
  \frac{ l_1}{\sin(\pi - (\alpha + \epsilon))} 
= \frac{ l_2 }{\sin{\alpha}}
= \frac{L}{\sin{\epsilon}}
\end{displaymath}

So the lengths to the star from each end of the orbit we have 
\begin{eqnarray*}
l_1 & = & L \frac{ \sin(\alpha + \epsilon) }   { \sin{\epsilon}} \\
l_2 & = & L \frac{ \sin{\alpha} }   { \sin{\epsilon}}
\end{eqnarray*}

and so the average length $\overline{l}$ to the star is 
\begin{eqnarray*}
\overline{l} & = & L \frac{ \sin(\alpha + \epsilon) + \sin{\alpha} }{    \sin{\epsilon}} \\
             & = & 2L \frac{ \sin(\alpha + \epsilon/2)\cos(\epsilon/2) }{    \sin{\epsilon}}
\end{eqnarray*}

Since the angular difference $\epsilon << 0$, the average distance to the star can be approximated as
\begin{displaymath}
\overline{l} = 2L \frac{ \sin{\alpha} }{\epsilon }
\end{displaymath}
%\end{multicols}

%\pagebreak

%\bibliographystyle{plainnat}
%\bibliography{myrefs}

%\end{document}


\part{Fletcher}
%
% Copyright � 2012 Peeter Joot.  All Rights Reserved.
% Licenced as described in the file LICENSE under the root directory of this GIT repository.
%

% 
% 
%\documentclass{article}

%\usepackage{amsmath}
\usepackage{mathpazo}

%
% shorthand for bold symbols, convenient for vectors and matrices
%
\newcommand{\Ba}[0]{\mathbf{a}}
\newcommand{\Bb}[0]{\mathbf{b}}
\newcommand{\Bc}[0]{\mathbf{c}}
\newcommand{\Bd}[0]{\mathbf{d}}
\newcommand{\Be}[0]{\mathbf{e}}
\newcommand{\Bf}[0]{\mathbf{f}}
\newcommand{\Bg}[0]{\mathbf{g}}
\newcommand{\Bh}[0]{\mathbf{h}}
\newcommand{\Bi}[0]{\mathbf{i}}
\newcommand{\Bj}[0]{\mathbf{j}}
\newcommand{\Bk}[0]{\mathbf{k}}
\newcommand{\Bl}[0]{\mathbf{l}}
\newcommand{\Bm}[0]{\mathbf{m}}
\newcommand{\Bn}[0]{\mathbf{n}}
\newcommand{\Bo}[0]{\mathbf{o}}
\newcommand{\Bp}[0]{\mathbf{p}}
\newcommand{\Bq}[0]{\mathbf{q}}
\newcommand{\Br}[0]{\mathbf{r}}
\newcommand{\Bs}[0]{\mathbf{s}}
\newcommand{\Bt}[0]{\mathbf{t}}
\newcommand{\Bu}[0]{\mathbf{u}}
\newcommand{\Bv}[0]{\mathbf{v}}
\newcommand{\Bw}[0]{\mathbf{w}}
\newcommand{\Bx}[0]{\mathbf{x}}
\newcommand{\By}[0]{\mathbf{y}}
\newcommand{\Bz}[0]{\mathbf{z}}
\newcommand{\BA}[0]{\mathbf{A}}
\newcommand{\BB}[0]{\mathbf{B}}
\newcommand{\BC}[0]{\mathbf{C}}
\newcommand{\BD}[0]{\mathbf{D}}
\newcommand{\BE}[0]{\mathbf{E}}
\newcommand{\BF}[0]{\mathbf{F}}
\newcommand{\BG}[0]{\mathbf{G}}
\newcommand{\BH}[0]{\mathbf{H}}
\newcommand{\BI}[0]{\mathbf{I}}
\newcommand{\BJ}[0]{\mathbf{J}}
\newcommand{\BK}[0]{\mathbf{K}}
\newcommand{\BL}[0]{\mathbf{L}}
\newcommand{\BM}[0]{\mathbf{M}}
\newcommand{\BN}[0]{\mathbf{N}}
\newcommand{\BO}[0]{\mathbf{O}}
\newcommand{\BP}[0]{\mathbf{P}}
\newcommand{\BQ}[0]{\mathbf{Q}}
\newcommand{\BR}[0]{\mathbf{R}}
\newcommand{\BS}[0]{\mathbf{S}}
\newcommand{\BT}[0]{\mathbf{T}}
\newcommand{\BU}[0]{\mathbf{U}}
\newcommand{\BV}[0]{\mathbf{V}}
\newcommand{\BW}[0]{\mathbf{W}}
\newcommand{\BX}[0]{\mathbf{X}}
\newcommand{\BY}[0]{\mathbf{Y}}
\newcommand{\BZ}[0]{\mathbf{Z}}

\newcommand{\Bzero}[0]{\mathbf{0}}
\newcommand{\Btheta}[0]{\boldsymbol{\theta}}
\newcommand{\Btau}[0]{\boldsymbol{\tau}}
\newcommand{\Bomega}[0]{\boldsymbol{\omega}}

%
% shorthand for unit vectors
%
\newcommand{\acap}[0]{\hat{\Ba}}
\newcommand{\bcap}[0]{\hat{\Bb}}
\newcommand{\ccap}[0]{\hat{\Bc}}
\newcommand{\dcap}[0]{\hat{\Bd}}
\newcommand{\ecap}[0]{\hat{\Be}}
\newcommand{\fcap}[0]{\hat{\Bf}}
\newcommand{\gcap}[0]{\hat{\Bg}}
\newcommand{\hcap}[0]{\hat{\Bh}}
\newcommand{\icap}[0]{\hat{\Bi}}
\newcommand{\jcap}[0]{\hat{\Bj}}
\newcommand{\kcap}[0]{\hat{\Bk}}
\newcommand{\lcap}[0]{\hat{\Bl}}
\newcommand{\mcap}[0]{\hat{\Bm}}
\newcommand{\ncap}[0]{\hat{\Bn}}
\newcommand{\ocap}[0]{\hat{\Bo}}
\newcommand{\pcap}[0]{\hat{\Bp}}
\newcommand{\qcap}[0]{\hat{\Bq}}
\newcommand{\rcap}[0]{\hat{\Br}}
\newcommand{\scap}[0]{\hat{\Bs}}
\newcommand{\tcap}[0]{\hat{\Bt}}
\newcommand{\ucap}[0]{\hat{\Bu}}
\newcommand{\vcap}[0]{\hat{\Bv}}
\newcommand{\wcap}[0]{\hat{\Bw}}
\newcommand{\xcap}[0]{\hat{\Bx}}
\newcommand{\ycap}[0]{\hat{\By}}
\newcommand{\zcap}[0]{\hat{\Bz}}
\newcommand{\thetacap}[0]{\hat{\Btheta}}

%
% to write R^n and C^n in a distinguishable fashion.  Perhaps change this
% to the double lined characters upon figuring out how to do so.
%
\newcommand{\C}[1]{$\mathbb{C}^{#1}$}
\newcommand{\R}[1]{$\mathbb{R}^{#1}$}

%
% various generally useful helpers
%

% derivative of #1 wrt. #2:
\newcommand{\D}[2] {\frac {d#2} {d#1}}

\newcommand{\inv}[1]{\frac{1}{#1}}
\newcommand{\cross}[0]{\times}

\newcommand{\abs}[1]{\lvert{#1}\rvert}
\newcommand{\norm}[1]{\lVert{#1}\rVert}
\newcommand{\innerprod}[2]{\langle{#1}, {#2}\rangle}
\newcommand{\dotprod}[2]{{#1} \cdot {#2}}
\newcommand{\bdotprod}[2]{\left({#1} \cdot {#2}\right)}
\newcommand{\crossprod}[2]{{#1} \cross {#2}}
\newcommand{\tripleprod}[3]{\dotprod{\left(\crossprod{#1}{#2}\right)}{#3}}

\DeclareMathOperator{\Proj}{Proj}
\DeclareMathOperator{\Span}{span}
\DeclareMathOperator{\Sgn}{sgn}
\DeclareMathOperator{\Area}{Area}
\DeclareMathOperator{\Volume}{Volume}

%
% A few miscellaneous things specific to this document
%
\newcommand{\crossop}[1]{\crossprod{#1}{}}

% R2 vector.
\newcommand{\VectorTwo}[2]{
\begin{bmatrix}
 {#1} \\
 {#2}
\end{bmatrix}
}

\newcommand{\VectorN}[1]{
\begin{bmatrix}
{#1}_1 \\
{#1}_2 \\
\vdots \\
{#1}_N \\
\end{bmatrix}
}

\newcommand{\DETuvij}[4]{
\begin{vmatrix}
 {#1}_{#3} & {#1}_{#4} \\
 {#2}_{#3} & {#2}_{#4}
\end{vmatrix}
}

\newcommand{\DETuvwijk}[6]{
\begin{vmatrix}
 {#1}_{#4} & {#1}_{#5} & {#1}_{#6} \\
 {#2}_{#4} & {#2}_{#5} & {#2}_{#6} \\
 {#3}_{#4} & {#3}_{#5} & {#3}_{#6}
\end{vmatrix}
}

\newcommand{\DETuvwxijkl}[8]{
\begin{vmatrix}
 {#1}_{#5} & {#1}_{#6} & {#1}_{#7} & {#1}_{#8} \\
 {#2}_{#5} & {#2}_{#6} & {#2}_{#7} & {#2}_{#8} \\
 {#3}_{#5} & {#3}_{#6} & {#3}_{#7} & {#3}_{#8} \\
 {#4}_{#5} & {#4}_{#6} & {#4}_{#7} & {#4}_{#8} \\
\end{vmatrix}
}

%\newcommand{\DETuvwxyijklm}[10]{
%\begin{vmatrix}
% {#1}_{#6} & {#1}_{#7} & {#1}_{#8} & {#1}_{#9} & {#1}_{#10} \\
% {#2}_{#6} & {#2}_{#7} & {#2}_{#8} & {#2}_{#9} & {#2}_{#10} \\
% {#3}_{#6} & {#3}_{#7} & {#3}_{#8} & {#3}_{#9} & {#3}_{#10} \\
% {#4}_{#6} & {#4}_{#7} & {#4}_{#8} & {#4}_{#9} & {#4}_{#10} \\
% {#5}_{#6} & {#5}_{#7} & {#5}_{#8} & {#5}_{#9} & {#5}_{#10}
%\end{vmatrix}
%}

% R3 vector.
\newcommand{\VectorThree}[3]{
\begin{bmatrix}
 {#1} \\
 {#2} \\
 {#3}
\end{bmatrix}
}


%%<misc>
%
\newcommand{\Abs}[1]{{\left\lvert{#1}\right\rvert}}
\newcommand{\spacegrad}[0]{\boldsymbol{\nabla}}
\newcommand{\grad}[0]{\nabla}
\newcommand{\LL}[0]{\mathcal{L}}

% == \partial_{#1} {#2}
\newcommand{\PD}[2]{\frac{\partial {#2}}{\partial {#1}}}
% inline variant
\newcommand{\PDi}[2]{{\partial {#2}}/{\partial {#1}}}

\newcommand{\PDD}[3]{\frac{\partial^2 {#3}}{\partial {#1}\partial {#2}}}
%\newcommand{\PDd}[2]{\frac{\partial^2 {#2}}{{\partial{#1}}^2}}
\newcommand{\PDsq}[2]{\frac{\partial^2 {#2}}{(\partial {#1})^2}}

\newcommand{\Partial}[2]{\frac{\partial {#1}}{\partial {#2}}}
\DeclareMathOperator{\RejName}{Rej}
\newcommand{\Rej}[2]{\RejName_{#1}\left( {#2} \right)}
\newcommand{\Rm}[1]{\mathbb{R}^{#1}}
\newcommand{\Cm}[1]{\mathbb{C}^{#1}}
\newcommand{\conj}[0]{{*}}

%</misc>

% <grade selection>
%
\newcommand{\gpgrade}[2] {{\left\langle{{#1}}\right\rangle}_{#2}}

\newcommand{\gpgradezero}[1] {\gpgrade{#1}{}}
%\newcommand{\gpscalargrade}[1] {{\left\langle{{#1}}\right\rangle}}
%\newcommand{\gpgradezero}[1] {\gpgrade{#1}{0}}

%\newcommand{\gpgradeone}[1] {{\left\langle{{#1}}\right\rangle}_{1}}
\newcommand{\gpgradeone}[1] {\gpgrade{#1}{1}}

\newcommand{\gpgradetwo}[1] {\gpgrade{#1}{2}}
\newcommand{\gpgradethree}[1] {\gpgrade{#1}{3}}
\newcommand{\gpgradefour}[1] {\gpgrade{#1}{4}}
%
% </grade selection>



\newcommand{\adot}[0]{{\dot{a}}}
\newcommand{\bdot}[0]{{\dot{b}}}
% taken for centered dot:
%\newcommand{\cdot}[0]{{\dot{c}}}
%\newcommand{\ddot}[0]{{\dot{d}}}
\newcommand{\edot}[0]{{\dot{e}}}
\newcommand{\fdot}[0]{{\dot{f}}}
\newcommand{\gdot}[0]{{\dot{g}}}
\newcommand{\hdot}[0]{{\dot{h}}}
\newcommand{\idot}[0]{{\dot{i}}}
\newcommand{\jdot}[0]{{\dot{j}}}
\newcommand{\kdot}[0]{{\dot{k}}}
\newcommand{\ldot}[0]{{\dot{l}}}
\newcommand{\mdot}[0]{{\dot{m}}}
\newcommand{\ndot}[0]{{\dot{n}}}
%\newcommand{\odot}[0]{{\dot{o}}}
\newcommand{\pdot}[0]{{\dot{p}}}
\newcommand{\qdot}[0]{{\dot{q}}}
\newcommand{\rdot}[0]{{\dot{r}}}
\newcommand{\sdot}[0]{{\dot{s}}}
\newcommand{\tdot}[0]{{\dot{t}}}
\newcommand{\udot}[0]{{\dot{u}}}
\newcommand{\vdot}[0]{{\dot{v}}}
\newcommand{\wdot}[0]{{\dot{w}}}
\newcommand{\xdot}[0]{{\dot{x}}}
\newcommand{\ydot}[0]{{\dot{y}}}
\newcommand{\zdot}[0]{{\dot{z}}}
\newcommand{\addot}[0]{{\ddot{a}}}
\newcommand{\bddot}[0]{{\ddot{b}}}
\newcommand{\cddot}[0]{{\ddot{c}}}
%\newcommand{\dddot}[0]{{\ddot{d}}}
\newcommand{\eddot}[0]{{\ddot{e}}}
\newcommand{\fddot}[0]{{\ddot{f}}}
\newcommand{\gddot}[0]{{\ddot{g}}}
\newcommand{\hddot}[0]{{\ddot{h}}}
\newcommand{\iddot}[0]{{\ddot{i}}}
\newcommand{\jddot}[0]{{\ddot{j}}}
\newcommand{\kddot}[0]{{\ddot{k}}}
\newcommand{\lddot}[0]{{\ddot{l}}}
\newcommand{\mddot}[0]{{\ddot{m}}}
\newcommand{\nddot}[0]{{\ddot{n}}}
\newcommand{\oddot}[0]{{\ddot{o}}}
\newcommand{\pddot}[0]{{\ddot{p}}}
\newcommand{\qddot}[0]{{\ddot{q}}}
\newcommand{\rddot}[0]{{\ddot{r}}}
\newcommand{\sddot}[0]{{\ddot{s}}}
\newcommand{\tddot}[0]{{\ddot{t}}}
\newcommand{\uddot}[0]{{\ddot{u}}}
\newcommand{\vddot}[0]{{\ddot{v}}}
\newcommand{\wddot}[0]{{\ddot{w}}}
\newcommand{\xddot}[0]{{\ddot{x}}}
\newcommand{\yddot}[0]{{\ddot{y}}}
\newcommand{\zddot}[0]{{\ddot{z}}}

%<bold and dot greek symbols>
%

\newcommand{\Deltadot}[0]{{\dot{\Delta}}}
\newcommand{\Gammadot}[0]{{\dot{\Gamma}}}
\newcommand{\Lambdadot}[0]{{\dot{\Lambda}}}
\newcommand{\Omegadot}[0]{{\dot{\Omega}}}
\newcommand{\Phidot}[0]{{\dot{\Phi}}}
\newcommand{\Pidot}[0]{{\dot{\Pi}}}
\newcommand{\Psidot}[0]{{\dot{\Psi}}}
\newcommand{\Sigmadot}[0]{{\dot{\Sigma}}}
\newcommand{\Thetadot}[0]{{\dot{\Theta}}}
\newcommand{\Upsilondot}[0]{{\dot{\Upsilon}}}
\newcommand{\Xidot}[0]{{\dot{\Xi}}}
\newcommand{\alphadot}[0]{{\dot{\alpha}}}
\newcommand{\betadot}[0]{{\dot{\beta}}}
\newcommand{\chidot}[0]{{\dot{\chi}}}
\newcommand{\deltadot}[0]{{\dot{\delta}}}
\newcommand{\epsilondot}[0]{{\dot{\epsilon}}}
\newcommand{\etadot}[0]{{\dot{\eta}}}
\newcommand{\gammadot}[0]{{\dot{\gamma}}}
\newcommand{\kappadot}[0]{{\dot{\kappa}}}
\newcommand{\lambdadot}[0]{{\dot{\lambda}}}
\newcommand{\mudot}[0]{{\dot{\mu}}}
\newcommand{\nudot}[0]{{\dot{\nu}}}
\newcommand{\omegadot}[0]{{\dot{\omega}}}
\newcommand{\phidot}[0]{{\dot{\phi}}}
\newcommand{\pidot}[0]{{\dot{\pi}}}
\newcommand{\psidot}[0]{{\dot{\psi}}}
\newcommand{\rhodot}[0]{{\dot{\rho}}}
\newcommand{\sigmadot}[0]{{\dot{\sigma}}}
\newcommand{\taudot}[0]{{\dot{\tau}}}
\newcommand{\thetadot}[0]{{\dot{\theta}}}
\newcommand{\upsilondot}[0]{{\dot{\upsilon}}}
\newcommand{\varepsilondot}[0]{{\dot{\varepsilon}}}
\newcommand{\varphidot}[0]{{\dot{\varphi}}}
\newcommand{\varpidot}[0]{{\dot{\varpi}}}
\newcommand{\varrhodot}[0]{{\dot{\varrho}}}
\newcommand{\varsigmadot}[0]{{\dot{\varsigma}}}
\newcommand{\varthetadot}[0]{{\dot{\vartheta}}}
\newcommand{\xidot}[0]{{\dot{\xi}}}
\newcommand{\zetadot}[0]{{\dot{\zeta}}}

\newcommand{\Deltaddot}[0]{{\ddot{\Delta}}}
\newcommand{\Gammaddot}[0]{{\ddot{\Gamma}}}
\newcommand{\Lambdaddot}[0]{{\ddot{\Lambda}}}
\newcommand{\Omegaddot}[0]{{\ddot{\Omega}}}
\newcommand{\Phiddot}[0]{{\ddot{\Phi}}}
\newcommand{\Piddot}[0]{{\ddot{\Pi}}}
\newcommand{\Psiddot}[0]{{\ddot{\Psi}}}
\newcommand{\Sigmaddot}[0]{{\ddot{\Sigma}}}
\newcommand{\Thetaddot}[0]{{\ddot{\Theta}}}
\newcommand{\Upsilonddot}[0]{{\ddot{\Upsilon}}}
\newcommand{\Xiddot}[0]{{\ddot{\Xi}}}
\newcommand{\alphaddot}[0]{{\ddot{\alpha}}}
\newcommand{\betaddot}[0]{{\ddot{\beta}}}
\newcommand{\chiddot}[0]{{\ddot{\chi}}}
\newcommand{\deltaddot}[0]{{\ddot{\delta}}}
\newcommand{\epsilonddot}[0]{{\ddot{\epsilon}}}
\newcommand{\etaddot}[0]{{\ddot{\eta}}}
\newcommand{\gammaddot}[0]{{\ddot{\gamma}}}
\newcommand{\kappaddot}[0]{{\ddot{\kappa}}}
\newcommand{\lambdaddot}[0]{{\ddot{\lambda}}}
\newcommand{\muddot}[0]{{\ddot{\mu}}}
\newcommand{\nuddot}[0]{{\ddot{\nu}}}
\newcommand{\omegaddot}[0]{{\ddot{\omega}}}
\newcommand{\phiddot}[0]{{\ddot{\phi}}}
\newcommand{\piddot}[0]{{\ddot{\pi}}}
\newcommand{\psiddot}[0]{{\ddot{\psi}}}
\newcommand{\rhoddot}[0]{{\ddot{\rho}}}
\newcommand{\sigmaddot}[0]{{\ddot{\sigma}}}
\newcommand{\tauddot}[0]{{\ddot{\tau}}}
\newcommand{\thetaddot}[0]{{\ddot{\theta}}}
\newcommand{\upsilonddot}[0]{{\ddot{\upsilon}}}
\newcommand{\varepsilonddot}[0]{{\ddot{\varepsilon}}}
\newcommand{\varphiddot}[0]{{\ddot{\varphi}}}
\newcommand{\varpiddot}[0]{{\ddot{\varpi}}}
\newcommand{\varrhoddot}[0]{{\ddot{\varrho}}}
\newcommand{\varsigmaddot}[0]{{\ddot{\varsigma}}}
\newcommand{\varthetaddot}[0]{{\ddot{\vartheta}}}
\newcommand{\xiddot}[0]{{\ddot{\xi}}}
\newcommand{\zetaddot}[0]{{\ddot{\zeta}}}

\newcommand{\BDelta}[0]{\boldsymbol{\Delta}}
\newcommand{\BGamma}[0]{\boldsymbol{\Gamma}}
\newcommand{\BLambda}[0]{\boldsymbol{\Lambda}}
\newcommand{\BOmega}[0]{\boldsymbol{\Omega}}
\newcommand{\BPhi}[0]{\boldsymbol{\Phi}}
\newcommand{\BPi}[0]{\boldsymbol{\Pi}}
\newcommand{\BPsi}[0]{\boldsymbol{\Psi}}
\newcommand{\BSigma}[0]{\boldsymbol{\Sigma}}
\newcommand{\BTheta}[0]{\boldsymbol{\Theta}}
\newcommand{\BUpsilon}[0]{\boldsymbol{\Upsilon}}
\newcommand{\BXi}[0]{\boldsymbol{\Xi}}
\newcommand{\Balpha}[0]{\boldsymbol{\alpha}}
\newcommand{\Bbeta}[0]{\boldsymbol{\beta}}
\newcommand{\Bchi}[0]{\boldsymbol{\chi}}
\newcommand{\Bdelta}[0]{\boldsymbol{\delta}}
\newcommand{\Bepsilon}[0]{\boldsymbol{\epsilon}}
\newcommand{\Beta}[0]{\boldsymbol{\eta}}
\newcommand{\Bgamma}[0]{\boldsymbol{\gamma}}
\newcommand{\Bkappa}[0]{\boldsymbol{\kappa}}
\newcommand{\Blambda}[0]{\boldsymbol{\lambda}}
\newcommand{\Bmu}[0]{\boldsymbol{\mu}}
\newcommand{\Bnu}[0]{\boldsymbol{\nu}}
%\newcommand{\Bomega}[0]{\boldsymbol{\omega}}
\newcommand{\Bphi}[0]{\boldsymbol{\phi}}
\newcommand{\Bpi}[0]{\boldsymbol{\pi}}
\newcommand{\Bpsi}[0]{\boldsymbol{\psi}}
\newcommand{\Brho}[0]{\boldsymbol{\rho}}
\newcommand{\Bsigma}[0]{\boldsymbol{\sigma}}
%\newcommand{\Btau}[0]{\boldsymbol{\tau}}
%\newcommand{\Btheta}[0]{\boldsymbol{\theta}}
\newcommand{\Bupsilon}[0]{\boldsymbol{\upsilon}}
\newcommand{\Bvarepsilon}[0]{\boldsymbol{\varepsilon}}
\newcommand{\Bvarphi}[0]{\boldsymbol{\varphi}}
\newcommand{\Bvarpi}[0]{\boldsymbol{\varpi}}
\newcommand{\Bvarrho}[0]{\boldsymbol{\varrho}}
\newcommand{\Bvarsigma}[0]{\boldsymbol{\varsigma}}
\newcommand{\Bvartheta}[0]{\boldsymbol{\vartheta}}
\newcommand{\Bxi}[0]{\boldsymbol{\xi}}
\newcommand{\Bzeta}[0]{\boldsymbol{\zeta}}
%
%</bold and dot greek symbols>
%<infrequent>
%
%\newcommand{\AreaOp}[1]{\AName_{#1}}
%\newcommand{\Babs}[0]{\abs{\BB}}
%\newcommand{\Bcap}[0]{\hat{\BB}}
%\newcommand{\BrPrimeRej}[0]{\rcap(\rcap \wedge \Br')}
%\newcommand{\CA}[0]{\mathcal{A}}
%\newcommand{\Cos}[1]{\cos{\left({#1}\right)}}
%\newcommand{\Det}[1] {\abs{#1}}
%\newcommand{\Dsq}[2] {\frac {\partial^2 {#1}} {\partial {#2}^2}}
%\newcommand{\Exp}[1]{\exp{\left({#1}\right)}}
%\newcommand{\Norm}[1]{\left\lVert{#1}\right\rVert}
%\newcommand{\Sin}[1]{\sin{\left({#1}\right)}}
%\newcommand{\T}[0]{\text{T}}
%\newcommand{\VolumeOp}[1]{\VName_{#1}}
%\newcommand{\agrad}[0]{\Ba \cdot \nabla}
%\newcommand{\alphacap}[0]{\hat{\boldsymbol{\alpha}}}
%\newcommand{\Fcap}[0]{\hat{\BF}}
%\newcommand{\bithree}[0]{{\Bi}_3}
%\newcommand{\bxa}[0]{\Bx\Ba}
%\newcommand{\coordvec}[2]{
%\newcommand{\costheta}[0]{\acap \cdot \xcap}
%\newcommand{\ddt}[1]{\ddot{#1}}
%\newcommand{\ddu}[1] {\frac {d{#1}} {du}}
%\newcommand{\dsqxj}[2] {\frac {\partial^2 {#1}} {\partial {x_{#2}}^2}}
%\newcommand{\dtheta}[1]{\frac{d {#1}}{d \theta}}
%\newcommand{\dt}[1]{\dot{#1}}
%\newcommand{\dt}[1]{\frac{d {#1}}{dt}}
%\newcommand{\dxj}[2] {\frac {\partial {#1}} {\partial {x_{#2}}}}
%\newcommand{\halfPhi}[0]{\frac{\phi}{2}}
%\newcommand{\half}[0]{\inv{2}}
%\newcommand{\inv}[1]{\frac{1}{#1}}
%\newcommand{\laplacian}[0]{\nabla^2}
%\newcommand{\matrixoftx}[3]{
%\newcommand{\nrrp}[0]{\norm{\rcap \wedge \Br'}}
%\newcommand{\oiint}{\bigcirc \hspace{-1.4em} \int \hspace{-.8em} \int}
%\newcommand{\transpose}[1]{{#1}^{\text{T}}}
%\newcommand{\transpose}[1]{{{#1}^{\TextTranspose}}}
%\newcommand{\transpose}[1]{{{#1}^{\text{T}}}}
%\newcommand{\barA}[0]{\bar{A}}
%\newcommand{\qbar}[0]{\bar{q}}
%\newcommand{\qdotbar}[0]{\dot{\bar{q}}}
%
%</infrequent>





%\usepackage[bookmarks=true]{hyperref}

%\usepackage{color,cite,graphicx}
   % use colour in the document, put your citations as [1-4]
   % rather than [1,2,3,4] (it looks nicer, and the extended LaTeX2e
   % graphics package. 
%\usepackage{latexsym,amssymb,epsf} % do not remember if these are
   % needed, but their inclusion can not do any damage


\chapter{fletcher64}
\label{chap:fletcher}
%\author{Peeter Joot \quad peeter.joot@gmail.com }
\date{ March dd, 2009.  fletcher.tex }

%\begin{document}

%\maketitle{}
%\tableofcontents
\section{Fast modulus by one less than power of 2?}

Statement in some code is

``\((x \& 0xFFFFFFFF) + (x \rightshift 32)\) is a faster way to do \(x \Mod 0xFFFFFFFF\), except the max result may be
\(0x1FFFFFFFE\), so need another round of this.''

How can this be justified?

\subsection{Translating the bit ops to math ops}

Suppose we have 

\begin{equation}\label{eqn:fletcher:modPower2}
\begin{aligned}
x = y \times 2^B + r \quad \mbox{where \(r \le 2^{B-1}\)}
\end{aligned}
\end{equation}

then we have

\begin{equation}\label{eqn:fletcher:22}
\begin{aligned}
y &= x \Div 2^{B} = x \rightshift B \\
r &= x \Mod 2^B = x \& 2^{B-1}
\end{aligned}
\end{equation}

So we have
\begin{equation}\label{eqn:fletcher:42}
\begin{aligned}
x 
&= (x \& 2^{B-1}) \times 2^B + x \rightshift B \\
&= (x \Mod 2^B) \times 2^B + x \Div 2^B \\
\end{aligned}
\end{equation}

\subsection{The shift expression}

From \eqnref{eqn:fletcher:modPower2} we have

\begin{equation}\label{eqn:fletcher:62}
\begin{aligned}
x 
&= y \times 2^B + r \\
&= y \times (2^B -1) + y + r \\
\end{aligned}
\end{equation}

Therefore we have

\begin{equation}\label{eqn:fletcher:82}
\begin{aligned}
x \Mod (2^B -1) &= (y + r) \Mod (2^B -1) \\
\end{aligned}
\end{equation}

The net effect is that the original desired modulus calculation has been reduced to one of a lesser numerical order.  In particular, for the
code in question we have \(B=32\), and an unsigned 64-bit value \(x\).  This means the value \(y \le 0xFFFFFFFF\), whereas \(r \le 0xFFFFFFFF\), so the new value
\(y + r \le 0x1FFFFFFFE\) with up to 31 bits knocked off.  A second round of reduction can give a value no more than \(0xFFFFFFFE + 1 = 0xFFFFFFFF\).

The conclusion is that this ``fast modulus'' is not exactly a modulus replacement.  Instead it is either the true modulus (ie: \(x \Mod 0xFFFFFFFF < 0xFFFFFFFF\)), but may for some \(x\) actually be equal to \(0xFFFFFFFF\).  This almost modulus operation was however sufficient in the code in question, since the idea was only to reduce the magnitude to a 32-bit quantity.

\section{Fletcher64}

A 32-bit word extension of the \href{http://en.wikipedia.org/wiki/Fletcher%27s_checksum}{wikipedia fletcher algorithm} is

\begin{equation}\label{eqn:fletcher:102}
\begin{aligned}
&\text{while} ( i < n )  \\
&\quad   S_1 += A_i \\
&\quad   S_2 += S_1 \\
\\
&C = \Mod_{32}(S_2) | \Mod_{32}(S_1)
\end{aligned}
\end{equation}

Where \(\Mod_{32}(x)\) is the fast modulus operation described above.

The summation equivalent to the loop above is
\begin{equation}\label{eqn:fletcher:122}
\begin{aligned}
S_1 &= \sum_{i=0}^{n-1} A_i \\
S_2 &= \sum_{i=0}^{n-1} (n-i)A_i \\
\end{aligned}
\end{equation}

Note that the above loop assumes that the data is constrained enough that no arithmetic overflows occur within the loop iteration.
Strictly speaking a more accurate transcription of the wiki fletcher32 algorithm using 64-bit accumulators is

\begin{equation}\label{eqn:fletcher:142}
\begin{aligned}
&\text{while} ( i < n )  \\
&\quad   S_1 += A_i \\
&\quad   S_2 += S_1 \\
&\quad   S_1 = \Mod_{32}(S_1) \\
&\quad   S_2 = \Mod_{32}(S_2) \\
\\
&C = S_2 | S_1
\end{aligned}
\end{equation}

Optimizations (like the use of the magic number 360 in the wikipedia article) are possible to avoid the modulus reduction in each iteration of the loop.

In particular, the modulus accumulation of residues is equivalent to the modulus of the sums.  Say

\begin{equation}\label{eqn:fletcher:162}
\begin{aligned}
a_1 &= (a_1 \Div m ) m + r_1 \\
a_2 &= (a_2 \Div m ) m + r_2
\end{aligned}
\end{equation}

So we have
\begin{equation}\label{eqn:fletcher:182}
\begin{aligned}
a_1 + a_2 &= (a_1 \Div m + a_2 \Div m ) m + r_1 + r_2 \\
\implies \\
(a_1 + a_2) \Mod m &= (r_1 + r_2) \Mod m \\
\end{aligned}
\end{equation}

Or more generally

\begin{equation}\label{eqn:fletcher:202}
\begin{aligned}
\left(\sum_i a_i \right) \Mod m &= \sum_i (a_i \Mod m) \Mod m \\
\end{aligned}
\end{equation}

\subsection{Max loops before truncation}

With 32K page sizes and 4 byte words we have a max number of iterations for the loop of

\begin{equation}\label{eqn:fletcher:222}
\begin{aligned}
\frac{32 \times 1024 }{4} = 8192 
\end{aligned}
\end{equation}

Assuming the biggest size value of \(M = A_i = 0xFFFFFFFF\) when do our summation variables wrap?

\begin{equation}\label{eqn:fletcher:242}
\begin{aligned}
S_1 &= \sum_{i=0}^{n-1} A_i \le n M \\
S_2 &= \sum_{i=0}^{n-1} (n-i)A_i <= M \sum_{i=0}^{n-1} (n-i) = M n(n-1)/2 < M n^2/2
\end{aligned}
\end{equation}

The overflow point is dominated by the \(S_2\) sum, so if our max accumulator value is \(M_a\) we want

\begin{equation}\label{eqn:fletcher:262}
\begin{aligned}
M_a \ge M n^2/2 \\
\implies \\
\sqrt{\frac{ 2 M_a }{ M}} \ge n \\
\end{aligned}
\end{equation}

For 64-bit max accumulator and 32-bit words we have
\begin{equation}\label{eqn:fletcher:282}
\begin{aligned}
\sqrt{2 \frac{2^{64} - 1}{2^{32} - 1}} = 92 681.9
\end{aligned}
\end{equation}

which is far bigger than \(8192\).

% google calc:
%log( (2^64-1)/8192 + 1)/log( 2 )

The biggest number of bits in the word that we can use without having to worry about carries are as follows

\begin{tabular}{|l|l|l|}
\hline
Page Size & N & bits \\
\hline
4K & 712 & 46 \\
8K & 1524 & 43 \\
16K & 3196 & 41 \\
32K & 6721 & 39 \\
\hline
\end{tabular}


%\section{Scratch notes}
%
%\begin{align*}
%f(x)
%&= (x \& 2^{B-1}) + x \rightshift B \\
%&= (x \Mod 2^B) + x \Div 2^B \\
%&\questionEquals x \Mod (2^B-1)
%\end{align*}

%\bibliographystyle{plainnat}
%\bibliography{myrefs}

%\end{document}


\part{Fourier treatments}
\documentclass{article}

\usepackage{amsmath}
\usepackage{mathpazo}

%
% shorthand for bold symbols, convenient for vectors and matrices
%
\newcommand{\Ba}[0]{\mathbf{a}}
\newcommand{\Bb}[0]{\mathbf{b}}
\newcommand{\Bc}[0]{\mathbf{c}}
\newcommand{\Bd}[0]{\mathbf{d}}
\newcommand{\Be}[0]{\mathbf{e}}
\newcommand{\Bf}[0]{\mathbf{f}}
\newcommand{\Bg}[0]{\mathbf{g}}
\newcommand{\Bh}[0]{\mathbf{h}}
\newcommand{\Bi}[0]{\mathbf{i}}
\newcommand{\Bj}[0]{\mathbf{j}}
\newcommand{\Bk}[0]{\mathbf{k}}
\newcommand{\Bl}[0]{\mathbf{l}}
\newcommand{\Bm}[0]{\mathbf{m}}
\newcommand{\Bn}[0]{\mathbf{n}}
\newcommand{\Bo}[0]{\mathbf{o}}
\newcommand{\Bp}[0]{\mathbf{p}}
\newcommand{\Bq}[0]{\mathbf{q}}
\newcommand{\Br}[0]{\mathbf{r}}
\newcommand{\Bs}[0]{\mathbf{s}}
\newcommand{\Bt}[0]{\mathbf{t}}
\newcommand{\Bu}[0]{\mathbf{u}}
\newcommand{\Bv}[0]{\mathbf{v}}
\newcommand{\Bw}[0]{\mathbf{w}}
\newcommand{\Bx}[0]{\mathbf{x}}
\newcommand{\By}[0]{\mathbf{y}}
\newcommand{\Bz}[0]{\mathbf{z}}
\newcommand{\BA}[0]{\mathbf{A}}
\newcommand{\BB}[0]{\mathbf{B}}
\newcommand{\BC}[0]{\mathbf{C}}
\newcommand{\BD}[0]{\mathbf{D}}
\newcommand{\BE}[0]{\mathbf{E}}
\newcommand{\BF}[0]{\mathbf{F}}
\newcommand{\BG}[0]{\mathbf{G}}
\newcommand{\BH}[0]{\mathbf{H}}
\newcommand{\BI}[0]{\mathbf{I}}
\newcommand{\BJ}[0]{\mathbf{J}}
\newcommand{\BK}[0]{\mathbf{K}}
\newcommand{\BL}[0]{\mathbf{L}}
\newcommand{\BM}[0]{\mathbf{M}}
\newcommand{\BN}[0]{\mathbf{N}}
\newcommand{\BO}[0]{\mathbf{O}}
\newcommand{\BP}[0]{\mathbf{P}}
\newcommand{\BQ}[0]{\mathbf{Q}}
\newcommand{\BR}[0]{\mathbf{R}}
\newcommand{\BS}[0]{\mathbf{S}}
\newcommand{\BT}[0]{\mathbf{T}}
\newcommand{\BU}[0]{\mathbf{U}}
\newcommand{\BV}[0]{\mathbf{V}}
\newcommand{\BW}[0]{\mathbf{W}}
\newcommand{\BX}[0]{\mathbf{X}}
\newcommand{\BY}[0]{\mathbf{Y}}
\newcommand{\BZ}[0]{\mathbf{Z}}

\newcommand{\Bzero}[0]{\mathbf{0}}
\newcommand{\Btheta}[0]{\boldsymbol{\theta}}
\newcommand{\Btau}[0]{\boldsymbol{\tau}}
\newcommand{\Bomega}[0]{\boldsymbol{\omega}}

%
% shorthand for unit vectors
%
\newcommand{\acap}[0]{\hat{\Ba}}
\newcommand{\bcap}[0]{\hat{\Bb}}
\newcommand{\ccap}[0]{\hat{\Bc}}
\newcommand{\dcap}[0]{\hat{\Bd}}
\newcommand{\ecap}[0]{\hat{\Be}}
\newcommand{\fcap}[0]{\hat{\Bf}}
\newcommand{\gcap}[0]{\hat{\Bg}}
\newcommand{\hcap}[0]{\hat{\Bh}}
\newcommand{\icap}[0]{\hat{\Bi}}
\newcommand{\jcap}[0]{\hat{\Bj}}
\newcommand{\kcap}[0]{\hat{\Bk}}
\newcommand{\lcap}[0]{\hat{\Bl}}
\newcommand{\mcap}[0]{\hat{\Bm}}
\newcommand{\ncap}[0]{\hat{\Bn}}
\newcommand{\ocap}[0]{\hat{\Bo}}
\newcommand{\pcap}[0]{\hat{\Bp}}
\newcommand{\qcap}[0]{\hat{\Bq}}
\newcommand{\rcap}[0]{\hat{\Br}}
\newcommand{\scap}[0]{\hat{\Bs}}
\newcommand{\tcap}[0]{\hat{\Bt}}
\newcommand{\ucap}[0]{\hat{\Bu}}
\newcommand{\vcap}[0]{\hat{\Bv}}
\newcommand{\wcap}[0]{\hat{\Bw}}
\newcommand{\xcap}[0]{\hat{\Bx}}
\newcommand{\ycap}[0]{\hat{\By}}
\newcommand{\zcap}[0]{\hat{\Bz}}
\newcommand{\thetacap}[0]{\hat{\Btheta}}

%
% to write R^n and C^n in a distinguishable fashion.  Perhaps change this
% to the double lined characters upon figuring out how to do so.
%
\newcommand{\C}[1]{$\mathbb{C}^{#1}$}
\newcommand{\R}[1]{$\mathbb{R}^{#1}$}

%
% various generally useful helpers
%

% derivative of #1 wrt. #2:
\newcommand{\D}[2] {\frac {d#2} {d#1}}

\newcommand{\inv}[1]{\frac{1}{#1}}
\newcommand{\cross}[0]{\times}

\newcommand{\abs}[1]{\lvert{#1}\rvert}
\newcommand{\norm}[1]{\lVert{#1}\rVert}
\newcommand{\innerprod}[2]{\langle{#1}, {#2}\rangle}
\newcommand{\dotprod}[2]{{#1} \cdot {#2}}
\newcommand{\bdotprod}[2]{\left({#1} \cdot {#2}\right)}
\newcommand{\crossprod}[2]{{#1} \cross {#2}}
\newcommand{\tripleprod}[3]{\dotprod{\left(\crossprod{#1}{#2}\right)}{#3}}

\DeclareMathOperator{\Proj}{Proj}
\DeclareMathOperator{\Span}{span}
\DeclareMathOperator{\Sgn}{sgn}
\DeclareMathOperator{\Area}{Area}
\DeclareMathOperator{\Volume}{Volume}

%
% A few miscellaneous things specific to this document
%
\newcommand{\crossop}[1]{\crossprod{#1}{}}

% R2 vector.
\newcommand{\VectorTwo}[2]{
\begin{bmatrix}
 {#1} \\
 {#2}
\end{bmatrix}
}

\newcommand{\VectorN}[1]{
\begin{bmatrix}
{#1}_1 \\
{#1}_2 \\
\vdots \\
{#1}_N \\
\end{bmatrix}
}

\newcommand{\DETuvij}[4]{
\begin{vmatrix}
 {#1}_{#3} & {#1}_{#4} \\
 {#2}_{#3} & {#2}_{#4}
\end{vmatrix}
}

\newcommand{\DETuvwijk}[6]{
\begin{vmatrix}
 {#1}_{#4} & {#1}_{#5} & {#1}_{#6} \\
 {#2}_{#4} & {#2}_{#5} & {#2}_{#6} \\
 {#3}_{#4} & {#3}_{#5} & {#3}_{#6}
\end{vmatrix}
}

\newcommand{\DETuvwxijkl}[8]{
\begin{vmatrix}
 {#1}_{#5} & {#1}_{#6} & {#1}_{#7} & {#1}_{#8} \\
 {#2}_{#5} & {#2}_{#6} & {#2}_{#7} & {#2}_{#8} \\
 {#3}_{#5} & {#3}_{#6} & {#3}_{#7} & {#3}_{#8} \\
 {#4}_{#5} & {#4}_{#6} & {#4}_{#7} & {#4}_{#8} \\
\end{vmatrix}
}

%\newcommand{\DETuvwxyijklm}[10]{
%\begin{vmatrix}
% {#1}_{#6} & {#1}_{#7} & {#1}_{#8} & {#1}_{#9} & {#1}_{#10} \\
% {#2}_{#6} & {#2}_{#7} & {#2}_{#8} & {#2}_{#9} & {#2}_{#10} \\
% {#3}_{#6} & {#3}_{#7} & {#3}_{#8} & {#3}_{#9} & {#3}_{#10} \\
% {#4}_{#6} & {#4}_{#7} & {#4}_{#8} & {#4}_{#9} & {#4}_{#10} \\
% {#5}_{#6} & {#5}_{#7} & {#5}_{#8} & {#5}_{#9} & {#5}_{#10}
%\end{vmatrix}
%}

% R3 vector.
\newcommand{\VectorThree}[3]{
\begin{bmatrix}
 {#1} \\
 {#2} \\
 {#3}
\end{bmatrix}
}


%<misc>
%
\newcommand{\Abs}[1]{{\left\lvert{#1}\right\rvert}}
\newcommand{\spacegrad}[0]{\boldsymbol{\nabla}}
\newcommand{\grad}[0]{\nabla}
\newcommand{\LL}[0]{\mathcal{L}}

% == \partial_{#1} {#2}
\newcommand{\PD}[2]{\frac{\partial {#2}}{\partial {#1}}}
% inline variant
\newcommand{\PDi}[2]{{\partial {#2}}/{\partial {#1}}}

\newcommand{\PDD}[3]{\frac{\partial^2 {#3}}{\partial {#1}\partial {#2}}}
%\newcommand{\PDd}[2]{\frac{\partial^2 {#2}}{{\partial{#1}}^2}}
\newcommand{\PDsq}[2]{\frac{\partial^2 {#2}}{(\partial {#1})^2}}

\newcommand{\Partial}[2]{\frac{\partial {#1}}{\partial {#2}}}
\DeclareMathOperator{\RejName}{Rej}
\newcommand{\Rej}[2]{\RejName_{#1}\left( {#2} \right)}
\newcommand{\Rm}[1]{\mathbb{R}^{#1}}
\newcommand{\Cm}[1]{\mathbb{C}^{#1}}
\newcommand{\conj}[0]{{*}}

%</misc>

% <grade selection>
%
\newcommand{\gpgrade}[2] {{\left\langle{{#1}}\right\rangle}_{#2}}

\newcommand{\gpgradezero}[1] {\gpgrade{#1}{}}
%\newcommand{\gpscalargrade}[1] {{\left\langle{{#1}}\right\rangle}}
%\newcommand{\gpgradezero}[1] {\gpgrade{#1}{0}}

%\newcommand{\gpgradeone}[1] {{\left\langle{{#1}}\right\rangle}_{1}}
\newcommand{\gpgradeone}[1] {\gpgrade{#1}{1}}

\newcommand{\gpgradetwo}[1] {\gpgrade{#1}{2}}
\newcommand{\gpgradethree}[1] {\gpgrade{#1}{3}}
\newcommand{\gpgradefour}[1] {\gpgrade{#1}{4}}
%
% </grade selection>



\newcommand{\adot}[0]{{\dot{a}}}
\newcommand{\bdot}[0]{{\dot{b}}}
% taken for centered dot:
%\newcommand{\cdot}[0]{{\dot{c}}}
%\newcommand{\ddot}[0]{{\dot{d}}}
\newcommand{\edot}[0]{{\dot{e}}}
\newcommand{\fdot}[0]{{\dot{f}}}
\newcommand{\gdot}[0]{{\dot{g}}}
\newcommand{\hdot}[0]{{\dot{h}}}
\newcommand{\idot}[0]{{\dot{i}}}
\newcommand{\jdot}[0]{{\dot{j}}}
\newcommand{\kdot}[0]{{\dot{k}}}
\newcommand{\ldot}[0]{{\dot{l}}}
\newcommand{\mdot}[0]{{\dot{m}}}
\newcommand{\ndot}[0]{{\dot{n}}}
%\newcommand{\odot}[0]{{\dot{o}}}
\newcommand{\pdot}[0]{{\dot{p}}}
\newcommand{\qdot}[0]{{\dot{q}}}
\newcommand{\rdot}[0]{{\dot{r}}}
\newcommand{\sdot}[0]{{\dot{s}}}
\newcommand{\tdot}[0]{{\dot{t}}}
\newcommand{\udot}[0]{{\dot{u}}}
\newcommand{\vdot}[0]{{\dot{v}}}
\newcommand{\wdot}[0]{{\dot{w}}}
\newcommand{\xdot}[0]{{\dot{x}}}
\newcommand{\ydot}[0]{{\dot{y}}}
\newcommand{\zdot}[0]{{\dot{z}}}
\newcommand{\addot}[0]{{\ddot{a}}}
\newcommand{\bddot}[0]{{\ddot{b}}}
\newcommand{\cddot}[0]{{\ddot{c}}}
%\newcommand{\dddot}[0]{{\ddot{d}}}
\newcommand{\eddot}[0]{{\ddot{e}}}
\newcommand{\fddot}[0]{{\ddot{f}}}
\newcommand{\gddot}[0]{{\ddot{g}}}
\newcommand{\hddot}[0]{{\ddot{h}}}
\newcommand{\iddot}[0]{{\ddot{i}}}
\newcommand{\jddot}[0]{{\ddot{j}}}
\newcommand{\kddot}[0]{{\ddot{k}}}
\newcommand{\lddot}[0]{{\ddot{l}}}
\newcommand{\mddot}[0]{{\ddot{m}}}
\newcommand{\nddot}[0]{{\ddot{n}}}
\newcommand{\oddot}[0]{{\ddot{o}}}
\newcommand{\pddot}[0]{{\ddot{p}}}
\newcommand{\qddot}[0]{{\ddot{q}}}
\newcommand{\rddot}[0]{{\ddot{r}}}
\newcommand{\sddot}[0]{{\ddot{s}}}
\newcommand{\tddot}[0]{{\ddot{t}}}
\newcommand{\uddot}[0]{{\ddot{u}}}
\newcommand{\vddot}[0]{{\ddot{v}}}
\newcommand{\wddot}[0]{{\ddot{w}}}
\newcommand{\xddot}[0]{{\ddot{x}}}
\newcommand{\yddot}[0]{{\ddot{y}}}
\newcommand{\zddot}[0]{{\ddot{z}}}

%<bold and dot greek symbols>
%

\newcommand{\Deltadot}[0]{{\dot{\Delta}}}
\newcommand{\Gammadot}[0]{{\dot{\Gamma}}}
\newcommand{\Lambdadot}[0]{{\dot{\Lambda}}}
\newcommand{\Omegadot}[0]{{\dot{\Omega}}}
\newcommand{\Phidot}[0]{{\dot{\Phi}}}
\newcommand{\Pidot}[0]{{\dot{\Pi}}}
\newcommand{\Psidot}[0]{{\dot{\Psi}}}
\newcommand{\Sigmadot}[0]{{\dot{\Sigma}}}
\newcommand{\Thetadot}[0]{{\dot{\Theta}}}
\newcommand{\Upsilondot}[0]{{\dot{\Upsilon}}}
\newcommand{\Xidot}[0]{{\dot{\Xi}}}
\newcommand{\alphadot}[0]{{\dot{\alpha}}}
\newcommand{\betadot}[0]{{\dot{\beta}}}
\newcommand{\chidot}[0]{{\dot{\chi}}}
\newcommand{\deltadot}[0]{{\dot{\delta}}}
\newcommand{\epsilondot}[0]{{\dot{\epsilon}}}
\newcommand{\etadot}[0]{{\dot{\eta}}}
\newcommand{\gammadot}[0]{{\dot{\gamma}}}
\newcommand{\kappadot}[0]{{\dot{\kappa}}}
\newcommand{\lambdadot}[0]{{\dot{\lambda}}}
\newcommand{\mudot}[0]{{\dot{\mu}}}
\newcommand{\nudot}[0]{{\dot{\nu}}}
\newcommand{\omegadot}[0]{{\dot{\omega}}}
\newcommand{\phidot}[0]{{\dot{\phi}}}
\newcommand{\pidot}[0]{{\dot{\pi}}}
\newcommand{\psidot}[0]{{\dot{\psi}}}
\newcommand{\rhodot}[0]{{\dot{\rho}}}
\newcommand{\sigmadot}[0]{{\dot{\sigma}}}
\newcommand{\taudot}[0]{{\dot{\tau}}}
\newcommand{\thetadot}[0]{{\dot{\theta}}}
\newcommand{\upsilondot}[0]{{\dot{\upsilon}}}
\newcommand{\varepsilondot}[0]{{\dot{\varepsilon}}}
\newcommand{\varphidot}[0]{{\dot{\varphi}}}
\newcommand{\varpidot}[0]{{\dot{\varpi}}}
\newcommand{\varrhodot}[0]{{\dot{\varrho}}}
\newcommand{\varsigmadot}[0]{{\dot{\varsigma}}}
\newcommand{\varthetadot}[0]{{\dot{\vartheta}}}
\newcommand{\xidot}[0]{{\dot{\xi}}}
\newcommand{\zetadot}[0]{{\dot{\zeta}}}

\newcommand{\Deltaddot}[0]{{\ddot{\Delta}}}
\newcommand{\Gammaddot}[0]{{\ddot{\Gamma}}}
\newcommand{\Lambdaddot}[0]{{\ddot{\Lambda}}}
\newcommand{\Omegaddot}[0]{{\ddot{\Omega}}}
\newcommand{\Phiddot}[0]{{\ddot{\Phi}}}
\newcommand{\Piddot}[0]{{\ddot{\Pi}}}
\newcommand{\Psiddot}[0]{{\ddot{\Psi}}}
\newcommand{\Sigmaddot}[0]{{\ddot{\Sigma}}}
\newcommand{\Thetaddot}[0]{{\ddot{\Theta}}}
\newcommand{\Upsilonddot}[0]{{\ddot{\Upsilon}}}
\newcommand{\Xiddot}[0]{{\ddot{\Xi}}}
\newcommand{\alphaddot}[0]{{\ddot{\alpha}}}
\newcommand{\betaddot}[0]{{\ddot{\beta}}}
\newcommand{\chiddot}[0]{{\ddot{\chi}}}
\newcommand{\deltaddot}[0]{{\ddot{\delta}}}
\newcommand{\epsilonddot}[0]{{\ddot{\epsilon}}}
\newcommand{\etaddot}[0]{{\ddot{\eta}}}
\newcommand{\gammaddot}[0]{{\ddot{\gamma}}}
\newcommand{\kappaddot}[0]{{\ddot{\kappa}}}
\newcommand{\lambdaddot}[0]{{\ddot{\lambda}}}
\newcommand{\muddot}[0]{{\ddot{\mu}}}
\newcommand{\nuddot}[0]{{\ddot{\nu}}}
\newcommand{\omegaddot}[0]{{\ddot{\omega}}}
\newcommand{\phiddot}[0]{{\ddot{\phi}}}
\newcommand{\piddot}[0]{{\ddot{\pi}}}
\newcommand{\psiddot}[0]{{\ddot{\psi}}}
\newcommand{\rhoddot}[0]{{\ddot{\rho}}}
\newcommand{\sigmaddot}[0]{{\ddot{\sigma}}}
\newcommand{\tauddot}[0]{{\ddot{\tau}}}
\newcommand{\thetaddot}[0]{{\ddot{\theta}}}
\newcommand{\upsilonddot}[0]{{\ddot{\upsilon}}}
\newcommand{\varepsilonddot}[0]{{\ddot{\varepsilon}}}
\newcommand{\varphiddot}[0]{{\ddot{\varphi}}}
\newcommand{\varpiddot}[0]{{\ddot{\varpi}}}
\newcommand{\varrhoddot}[0]{{\ddot{\varrho}}}
\newcommand{\varsigmaddot}[0]{{\ddot{\varsigma}}}
\newcommand{\varthetaddot}[0]{{\ddot{\vartheta}}}
\newcommand{\xiddot}[0]{{\ddot{\xi}}}
\newcommand{\zetaddot}[0]{{\ddot{\zeta}}}

\newcommand{\BDelta}[0]{\boldsymbol{\Delta}}
\newcommand{\BGamma}[0]{\boldsymbol{\Gamma}}
\newcommand{\BLambda}[0]{\boldsymbol{\Lambda}}
\newcommand{\BOmega}[0]{\boldsymbol{\Omega}}
\newcommand{\BPhi}[0]{\boldsymbol{\Phi}}
\newcommand{\BPi}[0]{\boldsymbol{\Pi}}
\newcommand{\BPsi}[0]{\boldsymbol{\Psi}}
\newcommand{\BSigma}[0]{\boldsymbol{\Sigma}}
\newcommand{\BTheta}[0]{\boldsymbol{\Theta}}
\newcommand{\BUpsilon}[0]{\boldsymbol{\Upsilon}}
\newcommand{\BXi}[0]{\boldsymbol{\Xi}}
\newcommand{\Balpha}[0]{\boldsymbol{\alpha}}
\newcommand{\Bbeta}[0]{\boldsymbol{\beta}}
\newcommand{\Bchi}[0]{\boldsymbol{\chi}}
\newcommand{\Bdelta}[0]{\boldsymbol{\delta}}
\newcommand{\Bepsilon}[0]{\boldsymbol{\epsilon}}
\newcommand{\Beta}[0]{\boldsymbol{\eta}}
\newcommand{\Bgamma}[0]{\boldsymbol{\gamma}}
\newcommand{\Bkappa}[0]{\boldsymbol{\kappa}}
\newcommand{\Blambda}[0]{\boldsymbol{\lambda}}
\newcommand{\Bmu}[0]{\boldsymbol{\mu}}
\newcommand{\Bnu}[0]{\boldsymbol{\nu}}
%\newcommand{\Bomega}[0]{\boldsymbol{\omega}}
\newcommand{\Bphi}[0]{\boldsymbol{\phi}}
\newcommand{\Bpi}[0]{\boldsymbol{\pi}}
\newcommand{\Bpsi}[0]{\boldsymbol{\psi}}
\newcommand{\Brho}[0]{\boldsymbol{\rho}}
\newcommand{\Bsigma}[0]{\boldsymbol{\sigma}}
%\newcommand{\Btau}[0]{\boldsymbol{\tau}}
%\newcommand{\Btheta}[0]{\boldsymbol{\theta}}
\newcommand{\Bupsilon}[0]{\boldsymbol{\upsilon}}
\newcommand{\Bvarepsilon}[0]{\boldsymbol{\varepsilon}}
\newcommand{\Bvarphi}[0]{\boldsymbol{\varphi}}
\newcommand{\Bvarpi}[0]{\boldsymbol{\varpi}}
\newcommand{\Bvarrho}[0]{\boldsymbol{\varrho}}
\newcommand{\Bvarsigma}[0]{\boldsymbol{\varsigma}}
\newcommand{\Bvartheta}[0]{\boldsymbol{\vartheta}}
\newcommand{\Bxi}[0]{\boldsymbol{\xi}}
\newcommand{\Bzeta}[0]{\boldsymbol{\zeta}}
%
%</bold and dot greek symbols>
%<infrequent>
%
%\newcommand{\AreaOp}[1]{\AName_{#1}}
%\newcommand{\Babs}[0]{\abs{\BB}}
%\newcommand{\Bcap}[0]{\hat{\BB}}
%\newcommand{\BrPrimeRej}[0]{\rcap(\rcap \wedge \Br')}
%\newcommand{\CA}[0]{\mathcal{A}}
%\newcommand{\Cos}[1]{\cos{\left({#1}\right)}}
%\newcommand{\Det}[1] {\abs{#1}}
%\newcommand{\Dsq}[2] {\frac {\partial^2 {#1}} {\partial {#2}^2}}
%\newcommand{\Exp}[1]{\exp{\left({#1}\right)}}
%\newcommand{\Norm}[1]{\left\lVert{#1}\right\rVert}
%\newcommand{\Sin}[1]{\sin{\left({#1}\right)}}
%\newcommand{\T}[0]{\text{T}}
%\newcommand{\VolumeOp}[1]{\VName_{#1}}
%\newcommand{\agrad}[0]{\Ba \cdot \nabla}
%\newcommand{\alphacap}[0]{\hat{\boldsymbol{\alpha}}}
%\newcommand{\Fcap}[0]{\hat{\BF}}
%\newcommand{\bithree}[0]{{\Bi}_3}
%\newcommand{\bxa}[0]{\Bx\Ba}
%\newcommand{\coordvec}[2]{
%\newcommand{\costheta}[0]{\acap \cdot \xcap}
%\newcommand{\ddt}[1]{\ddot{#1}}
%\newcommand{\ddu}[1] {\frac {d{#1}} {du}}
%\newcommand{\dsqxj}[2] {\frac {\partial^2 {#1}} {\partial {x_{#2}}^2}}
%\newcommand{\dtheta}[1]{\frac{d {#1}}{d \theta}}
%\newcommand{\dt}[1]{\dot{#1}}
%\newcommand{\dt}[1]{\frac{d {#1}}{dt}}
%\newcommand{\dxj}[2] {\frac {\partial {#1}} {\partial {x_{#2}}}}
%\newcommand{\halfPhi}[0]{\frac{\phi}{2}}
%\newcommand{\half}[0]{\inv{2}}
%\newcommand{\inv}[1]{\frac{1}{#1}}
%\newcommand{\laplacian}[0]{\nabla^2}
%\newcommand{\matrixoftx}[3]{
%\newcommand{\nrrp}[0]{\norm{\rcap \wedge \Br'}}
%\newcommand{\oiint}{\bigcirc \hspace{-1.4em} \int \hspace{-.8em} \int}
%\newcommand{\transpose}[1]{{#1}^{\text{T}}}
%\newcommand{\transpose}[1]{{{#1}^{\TextTranspose}}}
%\newcommand{\transpose}[1]{{{#1}^{\text{T}}}}
%\newcommand{\barA}[0]{\bar{A}}
%\newcommand{\qbar}[0]{\bar{q}}
%\newcommand{\qdotbar}[0]{\dot{\bar{q}}}
%
%</infrequent>





\newcommand{\PDSq}[2]{\frac{\partial^2 {#2}}{\partial {#1}^2}}
\DeclareMathOperator{\sinc}{sinc}
\DeclareMathOperator{\PV}{PV}
\newcommand{\FF}[0]{\mathcal{F}}
\newcommand{\IIinf}[0]{ \int_{-\infty}^\infty }

\usepackage[bookmarks=true]{hyperref}

\usepackage{color,cite,graphicx}
   % use colour in the document, put your citations as [1-4]
   % rather than [1,2,3,4] (it looks nicer, and the extended LaTeX2e
   % graphics package. 
\usepackage{latexsym,amssymb,epsf} % don't remember if these are
   % needed, but their inclusion can't do any damage


\title{ Applications of Fourier distribution theory to some PDEs. }
\author{Peeter Joot \quad peeter.joot@gmail.com }
\date{ March 4, 2009.  Last Revision: $Date: 2009/03/04 22:35:55 $ }

\begin{document}
\maketitle{}
\tableofcontents

\section{ Motivation. }

I recently listened to Prof Brad Osgood's lectures on distributions in Fourier
transform theory and read the associated lecture notes
\cite{osgoodFourier}.  Here is an attempt to apply these ideas to solution
of some of the common PDEs of physics (wave and Poisson equations).

Some of these were tackled recently using ``classical'' Fourier methods
in \cite{PJpoisson}. 
This requires ad-hoc PV associations of integrals with delta and step
functions.  Such solutions do not inspire confidence.  Without
validation of the solutions by substituion back into the generating PDE
or comparison to a known solution one is left wondering
if all the right fudges were actually performed to get the answer.

\section{ Conventions and definitions. }

\subsection{ Transform pairs. }

The form of the Fourier transform pairs used here are
\begin{align*}
\hat{f}(\mathbf{k}) &= \frac{1}{(\sqrt{2\pi})^n} \iiint f(\mathbf{x}) e^{-i \mathbf{k} \cdot \mathbf{x} } d^n x \\
{f}(\mathbf{x}) &= \frac{1}{(\sqrt{2\pi})^n} \iiint \hat{f}(\mathbf{k}) e^{i \mathbf{k} \cdot \mathbf{x} } d^n k \\
\end{align*}

\subsection{ Space of Schwarz functions. }

\section{ 1D first order homogeneous wave equation. }

This is the simplest PDE that I can think of that one should be able 
to apply Fourier techniques to.  We seek solutions $f(x,t)$ of

\begin{align}
\inv{v} \PD{t}{f} - \PD{x}{f} = 0
\end{align}

\subsection{ Classical way. }

\subsection{ Using distributions. }

\bibliographystyle{plainnat}
\bibliography{myrefs}

\end{document}


\part{Old cross product musings}
%
% Copyright � 2012 Peeter Joot.  All Rights Reserved.
% Licenced as described in the file LICENSE under the root directory of this GIT repository.
%

% 
% 
%\documentclass{article}      

%\usepackage{amsmath}
\usepackage{mathpazo}

%
% shorthand for bold symbols, convenient for vectors and matrices
%
\newcommand{\Ba}[0]{\mathbf{a}}
\newcommand{\Bb}[0]{\mathbf{b}}
\newcommand{\Bc}[0]{\mathbf{c}}
\newcommand{\Bd}[0]{\mathbf{d}}
\newcommand{\Be}[0]{\mathbf{e}}
\newcommand{\Bf}[0]{\mathbf{f}}
\newcommand{\Bg}[0]{\mathbf{g}}
\newcommand{\Bh}[0]{\mathbf{h}}
\newcommand{\Bi}[0]{\mathbf{i}}
\newcommand{\Bj}[0]{\mathbf{j}}
\newcommand{\Bk}[0]{\mathbf{k}}
\newcommand{\Bl}[0]{\mathbf{l}}
\newcommand{\Bm}[0]{\mathbf{m}}
\newcommand{\Bn}[0]{\mathbf{n}}
\newcommand{\Bo}[0]{\mathbf{o}}
\newcommand{\Bp}[0]{\mathbf{p}}
\newcommand{\Bq}[0]{\mathbf{q}}
\newcommand{\Br}[0]{\mathbf{r}}
\newcommand{\Bs}[0]{\mathbf{s}}
\newcommand{\Bt}[0]{\mathbf{t}}
\newcommand{\Bu}[0]{\mathbf{u}}
\newcommand{\Bv}[0]{\mathbf{v}}
\newcommand{\Bw}[0]{\mathbf{w}}
\newcommand{\Bx}[0]{\mathbf{x}}
\newcommand{\By}[0]{\mathbf{y}}
\newcommand{\Bz}[0]{\mathbf{z}}
\newcommand{\BA}[0]{\mathbf{A}}
\newcommand{\BB}[0]{\mathbf{B}}
\newcommand{\BC}[0]{\mathbf{C}}
\newcommand{\BD}[0]{\mathbf{D}}
\newcommand{\BE}[0]{\mathbf{E}}
\newcommand{\BF}[0]{\mathbf{F}}
\newcommand{\BG}[0]{\mathbf{G}}
\newcommand{\BH}[0]{\mathbf{H}}
\newcommand{\BI}[0]{\mathbf{I}}
\newcommand{\BJ}[0]{\mathbf{J}}
\newcommand{\BK}[0]{\mathbf{K}}
\newcommand{\BL}[0]{\mathbf{L}}
\newcommand{\BM}[0]{\mathbf{M}}
\newcommand{\BN}[0]{\mathbf{N}}
\newcommand{\BO}[0]{\mathbf{O}}
\newcommand{\BP}[0]{\mathbf{P}}
\newcommand{\BQ}[0]{\mathbf{Q}}
\newcommand{\BR}[0]{\mathbf{R}}
\newcommand{\BS}[0]{\mathbf{S}}
\newcommand{\BT}[0]{\mathbf{T}}
\newcommand{\BU}[0]{\mathbf{U}}
\newcommand{\BV}[0]{\mathbf{V}}
\newcommand{\BW}[0]{\mathbf{W}}
\newcommand{\BX}[0]{\mathbf{X}}
\newcommand{\BY}[0]{\mathbf{Y}}
\newcommand{\BZ}[0]{\mathbf{Z}}

\newcommand{\Bzero}[0]{\mathbf{0}}
\newcommand{\Btheta}[0]{\boldsymbol{\theta}}
\newcommand{\Btau}[0]{\boldsymbol{\tau}}
\newcommand{\Bomega}[0]{\boldsymbol{\omega}}

%
% shorthand for unit vectors
%
\newcommand{\acap}[0]{\hat{\Ba}}
\newcommand{\bcap}[0]{\hat{\Bb}}
\newcommand{\ccap}[0]{\hat{\Bc}}
\newcommand{\dcap}[0]{\hat{\Bd}}
\newcommand{\ecap}[0]{\hat{\Be}}
\newcommand{\fcap}[0]{\hat{\Bf}}
\newcommand{\gcap}[0]{\hat{\Bg}}
\newcommand{\hcap}[0]{\hat{\Bh}}
\newcommand{\icap}[0]{\hat{\Bi}}
\newcommand{\jcap}[0]{\hat{\Bj}}
\newcommand{\kcap}[0]{\hat{\Bk}}
\newcommand{\lcap}[0]{\hat{\Bl}}
\newcommand{\mcap}[0]{\hat{\Bm}}
\newcommand{\ncap}[0]{\hat{\Bn}}
\newcommand{\ocap}[0]{\hat{\Bo}}
\newcommand{\pcap}[0]{\hat{\Bp}}
\newcommand{\qcap}[0]{\hat{\Bq}}
\newcommand{\rcap}[0]{\hat{\Br}}
\newcommand{\scap}[0]{\hat{\Bs}}
\newcommand{\tcap}[0]{\hat{\Bt}}
\newcommand{\ucap}[0]{\hat{\Bu}}
\newcommand{\vcap}[0]{\hat{\Bv}}
\newcommand{\wcap}[0]{\hat{\Bw}}
\newcommand{\xcap}[0]{\hat{\Bx}}
\newcommand{\ycap}[0]{\hat{\By}}
\newcommand{\zcap}[0]{\hat{\Bz}}
\newcommand{\thetacap}[0]{\hat{\Btheta}}

%
% to write R^n and C^n in a distinguishable fashion.  Perhaps change this
% to the double lined characters upon figuring out how to do so.
%
\newcommand{\C}[1]{$\mathbb{C}^{#1}$}
\newcommand{\R}[1]{$\mathbb{R}^{#1}$}

%
% various generally useful helpers
%

% derivative of #1 wrt. #2:
\newcommand{\D}[2] {\frac {d#2} {d#1}}

\newcommand{\inv}[1]{\frac{1}{#1}}
\newcommand{\cross}[0]{\times}

\newcommand{\abs}[1]{\lvert{#1}\rvert}
\newcommand{\norm}[1]{\lVert{#1}\rVert}
\newcommand{\innerprod}[2]{\langle{#1}, {#2}\rangle}
\newcommand{\dotprod}[2]{{#1} \cdot {#2}}
\newcommand{\bdotprod}[2]{\left({#1} \cdot {#2}\right)}
\newcommand{\crossprod}[2]{{#1} \cross {#2}}
\newcommand{\tripleprod}[3]{\dotprod{\left(\crossprod{#1}{#2}\right)}{#3}}

\DeclareMathOperator{\Proj}{Proj}
\DeclareMathOperator{\Span}{span}
\DeclareMathOperator{\Sgn}{sgn}
\DeclareMathOperator{\Area}{Area}
\DeclareMathOperator{\Volume}{Volume}

%
% A few miscellaneous things specific to this document
%
\newcommand{\crossop}[1]{\crossprod{#1}{}}

% R2 vector.
\newcommand{\VectorTwo}[2]{
\begin{bmatrix}
 {#1} \\
 {#2}
\end{bmatrix}
}

\newcommand{\VectorN}[1]{
\begin{bmatrix}
{#1}_1 \\
{#1}_2 \\
\vdots \\
{#1}_N \\
\end{bmatrix}
}

\newcommand{\DETuvij}[4]{
\begin{vmatrix}
 {#1}_{#3} & {#1}_{#4} \\
 {#2}_{#3} & {#2}_{#4}
\end{vmatrix}
}

\newcommand{\DETuvwijk}[6]{
\begin{vmatrix}
 {#1}_{#4} & {#1}_{#5} & {#1}_{#6} \\
 {#2}_{#4} & {#2}_{#5} & {#2}_{#6} \\
 {#3}_{#4} & {#3}_{#5} & {#3}_{#6}
\end{vmatrix}
}

\newcommand{\DETuvwxijkl}[8]{
\begin{vmatrix}
 {#1}_{#5} & {#1}_{#6} & {#1}_{#7} & {#1}_{#8} \\
 {#2}_{#5} & {#2}_{#6} & {#2}_{#7} & {#2}_{#8} \\
 {#3}_{#5} & {#3}_{#6} & {#3}_{#7} & {#3}_{#8} \\
 {#4}_{#5} & {#4}_{#6} & {#4}_{#7} & {#4}_{#8} \\
\end{vmatrix}
}

%\newcommand{\DETuvwxyijklm}[10]{
%\begin{vmatrix}
% {#1}_{#6} & {#1}_{#7} & {#1}_{#8} & {#1}_{#9} & {#1}_{#10} \\
% {#2}_{#6} & {#2}_{#7} & {#2}_{#8} & {#2}_{#9} & {#2}_{#10} \\
% {#3}_{#6} & {#3}_{#7} & {#3}_{#8} & {#3}_{#9} & {#3}_{#10} \\
% {#4}_{#6} & {#4}_{#7} & {#4}_{#8} & {#4}_{#9} & {#4}_{#10} \\
% {#5}_{#6} & {#5}_{#7} & {#5}_{#8} & {#5}_{#9} & {#5}_{#10}
%\end{vmatrix}
%}

% R3 vector.
\newcommand{\VectorThree}[3]{
\begin{bmatrix}
 {#1} \\
 {#2} \\
 {#3}
\end{bmatrix}
}







                             
\chapter{The cross product in three and more dimensions} 
\label{chap:cross}
%\author{Peeter Joot \quad peeter.joot@gmail.com }
\date{ October 12, 2007.  cross.tex }

%\begin{document}             

%\maketitle{}

\section{Introduction}

We are initially taught that there are two ways to multiply vectors.  One is the dot product and one is the cross product.  However, we are also taught
how to work with higher dimensional vectors.  There is lots of real applications of higher dimensional vectors that do not require one to waste time puzzling about what the fourth dimension is or how to visualize it.  All we need are four, or N measured qualities for something and we have higher dimensions (position, color, texture, ...).

We see the dot product in a lot of very basic math and physics.  It is inherently simple to understand and to teach, and work with it for a variety of
sorts of calculations.

On the other hand, the cross product is an ugly arbitrary seeming sort of beast.  However, it is a beast that
describes many sorts of physical and mathematical situations.  In vector calculus
cross product terms and it relative the determinant end up occurring all over the place,
and in physics the cross product also occurs in many contexts.
Examples are Stokes theorem, Jacobian transformations, normal equations, the
curl operator, Maxwell's equations, torque, and the list goes on.

The cross product and the dot product have some similarities in form
yet the cross product is only defined for \R{3}, while the dot product
can be defined for \R{n} including \(n > 3\), and even be extended easily in many other ways.  Examples are complex vector spaces, and
the integral dot products that underpin the simple and beautiful Fourier theory.
Note that we call the dot product the inner product when being general, leaving dot or scalar product for the our standard Cartesian vector space beastie.

In many of these cases the mathematics ought to have no logical tie to three dimensions, yet
the cross product is an explicitly three dimensional sort of beast.

It is a bit surprising that one can get all the way through four years of engineering school and still not have an answer about how to do all the sort
of routine vector calculations that we can do in \R{3} in \R{N} (or even \R{4}).
With the second of our two ways to multiply vectors valid only in \R{3}, there is something wrong or
at least missing.

This paper was initially a write up of my scribblings on dot and cross products, where I tried to go back to some of the basic
physical situations that we first the dot and cross products and see for myself how each of these surface in a natural way.  I felt this ought to help indicate where the
explicit three dimensionality of the cross product really came from.

I had some success with this and would say that I had a better feel for the underlying structure of the cross product,
so next came an attempt to investigate possible generalizations of the cross
product to higher dimensions and other mathematical fields.

Since writing some initial notes on this (and producing a generalized cross product that is probably
not entirely useful), I have stumbled across the calculus of differential forms and its wedge product.
I have found, unsurprisingly in retrospect, that I am not the first one to try to
generalize the cross product and the math related to it.
However, differential forms (or calculus on manifolds) can be presented in a horrendously abstract unfriendly fashion, with few of the
geometric considerations that
likely inspired the subject in the first place.
It would be interesting to look at some of the original works by the founders of the subject and see how they presented it.
What we find in texts now is a presentation of the subject in the fashion required to demonstrate its proof from principle to principle.  This hides the natural progression of the subject and is hard to learn from in my opinion.  Obviously there is value to having a thoroughly proven set of
theorems underlying the subject, but that should not inhibit learning or teaching something that appears to have the potential to simplify
the way we work with vector calculus and physics in general.
I have also found a severe lack of clearly worked examples and easy to understand spelled out details in any of my books on this subject, and I have accumulated some of these for myself here.  It may not all make sense without also reading a or some books on differential forms too, but if you are reading this and you are not me, then I hope you get something out of it!

What is this wedge product?  The wedge product is not a mapping into \R{1} or \R{N} from the product of two (or more) vectors in \R{N}, but is instead a value in a ``product space'' that has a dimension usually different from the original.  This is not unreasonable, since we should not have any specific reason to require the product of two vectors itself be a vector of the same dimension.  In fact this is one of the aspects of the cross product that is particularly awkward since we should expect that we can take two vectors in \R{2} and multiply them without having to introduce a normal in a direction that was originally not defined.

We are shown that the wedge product is like the \R{3} cross or triple product in many ways.   It can be used to express the area of the
parallelogram formed by two vectors, or the parallelepiped formed by three, and can also be used to express normals.
My texts on the subject either leave it to you to demonstrate
these for yourself for \R{2} or \R{3} cases, or do not touch on the details at all.  Perhaps it was too ``obvious'' to the author to bother with.

This paper will provide a natural progressive discussion on some of the ways we can end up with a cross product.  An explicit calculation
of the area and volume of an \R{N} parallelogram and parallelepiped will be presented.  Details on how to calculate and express normals to lines,
planes, and volumes will be given in a few different ways.

In the end all this will be tied to the wedge product, and I will hopefully have
an intuitive underpinning that will help me learn differential calculus a bit better.  I had like to see how basic vector math and physics
ends up expressed in the language of differential forms and whether or not it simplifies things and ends up as an easier way to
calculate (I have the feeling it will).

\section{Vector products in geometry}

One can define a product of two vectors in any way you see fit.  The dot product provides a mapping
from \R{n} \(\otimes\) \R{n} to \R{1}, whereas the cross product is a mapping from
from \R{3} \(\otimes\) \R{3} to \R{3}.

The introduction of the cross product with a direction that seems so arbitrarily picked in a normal direction to both vectors was one of my reasons for needing to examine the underlying structure.
It is not unreasonable to ask for a vector product definition that
maps a pair of vectors in \R{2} into \R{2}, without introducing a third dimension to express the result
as is required by the cross product.

It seems irregular to have to introduce a product that is in a space different from the original.  In retrospect that is not too irregular, the dot product does just that, as it provides a mapping to a scalar space.

We should expect to be able to define vector products in many different spaces, and the dot and cross products are just two such potential vector products.

Much more general than either of the dot or cross products would be a product that included all the possible pairs of products of the components

\begin{equation}
\Bu \times \Bv = \sum_{i,j}{u_i v_j F_{i,j}}
\end{equation}

Here, \(F_{i,j}\) is some arbitrary function of the indices, perhaps a vector or component of a matrix or a value in some arbitrary field.  One also does not have to assume that there is any relation between \(F_{i,j}\) and \(F_{j,i}\).
This is generally called a tensor product I believe and like the wedge product it is in a space different than the original.  One could for example, express such a product as a NxN matrix or a vector in \R{N^2}.

Let us start with the saner (well, at least more regular and intuitive) older brother of the cross product, the dot product.  Later, examinations of physical concepts, area's, normals, and
volumes will lead to the cross and wedge products.

\subsection{The dot product}

The first time
we see the dot product in physics is in the context of vector projections onto two or three axis.  For example, when drawing
vector force diagrams and calculating work done against a force applied in a direction different than the motion.

Quantifying this projective operation requires nothing more than basic trigonometry, and we can
express the component of a vector
\(\Bv\) in the direction of \(\Bu\) , as:

\begin{equation}\label{eqn:cross:21}
\Proj_{\ucap}(\Bv) = \norm{\Bv} \cos(\theta) \ucap
\end{equation}

Here \(\ucap = \Bu/\norm{\Bu}\) is the unit vector in the direction of \(\Bu\), and \(\theta\) is the angle between the vector \(\Bu\) and \(\Bv\) where the angle is measured with counterclockwise rotation positive as usual.

Unlike the
cross product, both vector projection, and the lengths of the sides of
a triangle defined by vectors in a plane are just as valid in \R{n}
as in \R{3}, and in fact do not even have a particularly strong tie to
the field of real numbers.  There is however a requirement for both concepts
to introduce a distance metric.

The length of a line from the origin \((0, 0)\) to a point \((v_1, v_2)\), can be
shown to be equal to \(\sqrt{v_1^2 + v_2^2}\) (sophisticated math is not required for this and one can show this with the classic square inscribed in a square diagram ... probably dating back to Pythagoras).
Successive applications of this result shows that this length for a point \((v_1, v_2, v_3)\) equals \(\sqrt{v_1^2 + v_2^2 + v_3^2}\).  It is thus natural to define the length of an \R{n} vector in the same fashion: \(\norm{\Bv} = \sqrt{\sum_{i}{v_i^2}}\).

Taking the length of a vector sum (the opposing side of a triangle formed by these two vectors end to end) we have:

\begin{equation}\label{eqn:cross:901}
\begin{aligned}
\norm{\Bu + \Bv}^2 &= \sum_i{(u_i + v_i)(u_i + v_i)} \\
                   &= \sum_i{{u_i}^2 + 2 u_i v_i + {v_i}^2} \\
                   &= \sum_i{{u_i}^2} + 2 \sum_i{u_i v_i} + \sum{{v_i}^2} \\
                   &= \norm{\Bu}^2 + 2 \sum_i{u_i v_i} + \norm{\Bv}^2
\end{aligned}
\end{equation}

So, if these vectors form a right triangle, the middle term \(\sum_i{u_i v_i}\) must equal zero.

Similarly, for a triangle formed by the difference of two vectors both at the origin the length of the opposing side is:

\begin{equation}\label{eqn:cross:921}
\begin{aligned}
\norm{\Bv - \Bu}^2 &= \sum_i{(v_i - u_i)(v_i - u_i)} \\
                   &= \sum_i{{v_i}^2 - 2 u_i v_i + {u_i}^2} \\
                   &= \sum_i{{v_i}^2} - 2 \sum_i{u_i v_i} + \sum{{u_i}^2} \\
                   &= \norm{\Bv}^2 - 2 \sum_i{u_i v_i} + \norm{\Bu}^2
\end{aligned}
\end{equation}

Note that this is just the triangle law.

\begin{equation}\label{eqn:cross:trianglelaw}
\norm{\Bu - \Bv}^2
= \norm{\Bu}^2 + \norm{\Bv}^2 - 2 \norm{\Bu} \norm{\Bv} \cos(\theta)
\end{equation}

This middle term
\(\sum_i{u_i v_i}\)
in both expansions, we give the name (ie: define) ``dot product'', and write that as:

\begin{equation}
\dotprod{\Bu}{\Bv} = \sum_i{u_i v_i}
\end{equation}

By the triangle law comparison above this can also be expressed, writing \(\theta = (\Bu, \Bv)\), in the projective form

\begin{equation}
\dotprod{\Bu}{\Bv} = \norm{\Bu} \norm{\Bv} \cos(\Bu, \Bv)
\end{equation}

Alternatively the projective operation itself can be expressed in terms of the dot product:

\begin{equation}\label{eqn:cross:941}
\begin{aligned}
\Proj_{\ucap}(\Bv) &= (\norm{\Bv} \cos(\theta)) \ucap \\
                  &= \left(\frac{\dotprod{\Bu}{\Bv}}{\norm{\Bu}}\right) \ucap \\
                  &= \bdotprod{\ucap}{\Bv} \ucap
\end{aligned}
\end{equation}

\subsection{Proof of the projective form of the dot product}

Since the proof of the triangle law was not given, our result for the projective form of the dot product is also unproven.
However, a proof of this would also implicitly prove the triangle law
by comparison as above.

Let us do that proof of the projective form of the dot product without requiring a previous (trigonometric) proof of the triangle law.  To calculate \(\cos(\theta)\) where \(\theta\) is the angle between two vectors \(\Bu\) and \(\Bv\), we let

\begin{equation}\label{eqn:cross:961}
\begin{aligned}
\Bx &= \Proj_{\ucap}(\Bv) \\
    &= \alpha\Bu
\end{aligned}
\end{equation}

and,
\begin{equation}\label{eqn:cross:41}
\By = \Bv - \alpha\Bu
\end{equation}

Temporarily imposing a restriction \(\theta \in [0, \pi/2]\) so that \(\alpha\) is positive, we can now express the vector \(\Bv\) in terms of its perpendicular components \(\Bx\) and \(\By\).

\begin{equation}\label{eqn:cross:dotcosine}
\cos(\theta) = \frac{\alpha \norm{\Bu}}{\norm{\Bv}}
\end{equation}

\begin{equation}\label{eqn:cross:981}
\begin{aligned}
\norm{\Bv}^2       &= \\
\norm{\Bx + \By}^2 &= \norm{\alpha\Bu}^2 + \norm{\Bv - \alpha\Bu}^2 \\
                   &= 2 \alpha^2 \norm{\Bu}^2 + \norm{\Bv}^2 - 2\alpha\dotprod{\Bu}{\Bv}
\end{aligned}
\end{equation}

So,
\begin{equation}\label{eqn:cross:1001}
\begin{aligned}
\alpha\dotprod{\Bu}{\Bv} &= \alpha^2 \norm{\Bu}^2 \\
      \dotprod{\Bu}{\Bv} &= \alpha \norm{\Bu}^2 \\
                         &= (\alpha \norm{\Bu}) \norm{\Bu} \\
                         &= \left(\frac{\alpha \norm{\Bu}}{\norm{\Bv}}\right) \norm{\Bu} \norm{\Bv} \\
                         &= \cos(\theta) \norm{\Bu} \norm{\Bv}
\end{aligned}
\end{equation}

Lifting the restriction and considering the \(\theta \in [\pi/2, \pi]\) range, then by the above:

\begin{equation}\label{eqn:cross:1021}
\begin{aligned}
      \dotprod{-\Bu}{\Bv} &= \alpha \norm{\Bu}^2 \\
                          &= (\cos(\pi - \theta)) \norm{-\Bu} \norm{\Bv}
\end{aligned}
\end{equation}

So, again we have:

\begin{equation}\label{eqn:cross:projectivedotprod}
      \dotprod{\Bu}{\Bv}  = \cos(\theta) \norm{\Bu} \norm{\Bv}
\end{equation}

Proof of \eqnref{eqn:cross:projectivedotprod} for the third and fourth quadrants is similar, proving the following:

\begin{equation}\label{eqn:cross:1041}
\begin{aligned}
\cos(\theta)      &= \cos(\Bu,\Bv) \\
                  &= \frac{\dotprod{\Bu}{\Bv}}{\norm{\Bu} \norm{\Bv}} \\
\cos(\ucap,\vcap) &= \dotprod{\ucap}{\vcap}
\end{aligned}
\end{equation}

Now a triangle law, which gave us the significance of the dot product before even naming it, is proven as a side effect.

\subsection{Projective form of the cross product.  Part I.  Normal to vector in direction of other vector}

Since we started with the projective form of the dot product, it is natural to also start with the
projective form of the cross product.

The cross product is first seen (by me at least), in high school was in the following projective form:

\begin{equation}\label{eqn:cross:61}
\crossprod{\Bu}{\Bv} = \norm{\Bu}\norm{\Bv} \sin(\theta) \ncap
\end{equation}

Now, comparing to the projective form of the dot product we can expect that this is going to be
related to the
component of \(\Bv\) that is perpendicular to \(\Bu\), since that vector has magnitude:

\begin{equation}\label{eqn:cross:81}
\norm{\Bv - \Proj_{\ucap}(\Bv)} = \norm{\Bv} | \sin(\theta) |
\end{equation}

Let us calculate the Cartesian representation for the component of \(\Bv\) normal to \(\Bu\), a
perpendicular projection, \(\Proj_{\perp\ucap}(\Bv) = \Bv - \Proj_{\ucap}(\Bv)\):

\begin{equation}\label{eqn:cross:1061}
\begin{aligned}
\Proj_{\perp\ucap}(\Bv) = \Bv - \Proj_{\ucap}(\Bv) &= \Bv - \bdotprod{\ucap}{\Bv} \ucap \\
                                                 &= \frac{1}{\norm{\Bu}^2} \left(\Bv \norm{\Bu}^2 - \bdotprod{\Bu}{\Bv} \Bu \right) \\
                                                 &= \frac{1}{\norm{\Bu}^2} \sum_{i,j}(v_i \ecap_i u_j u_j - u_j v_j u_i \ecap_i) \\
                                                 &= -\frac{1}{\norm{\Bu}^2} \sum_{i,j}u_j \ecap_i (u_i v_j - u_j v_i) \\
                                                 &= -\frac{1}{\norm{\Bu}^2} \sum_{i,j}u_j \ecap_i \DETuvij{u}{v}{i}{j}
\end{aligned}
\end{equation}

For brevity, let us introduce a shorthand notation for this determinant:

\begin{equation}
D_{ij}^{\Bu \Bv} = \DETuvij{u}{v}{i}{j}
\end{equation}

Since \(D_{ii}^{\Bu \Bv} = 0\), we can write:
\begin{equation}\label{eqn:cross:1081}
\begin{aligned}
\Proj_{\perp\ucap}(\Bv) = \Bv - \Proj_{\ucap}(\Bv)
   &= -\frac{1}{\norm{\Bu}^2} \sum_{i,j} u_j \ecap_i D_{ij}^{\Bu \Bv} \\
   &= -\frac{1}{\norm{\Bu}^2} \left(\sum_{i<j} u_j \ecap_i D_{ij}^{\Bu \Bv} + \sum_{j<i} u_{j} \ecap_{i} D_{ij}^{\Bu \Bv}\right) \\
   &= -\frac{1}{\norm{\Bu}^2} \left(\sum_{i<j} u_j \ecap_i D_{ij}^{\Bu \Bv} + \sum_{j'<i'} u_{j'} \ecap_{i'} D_{i'j'}^{\Bu \Bv}\right) \\
   &= -\frac{1}{\norm{\Bu}^2} \left(\sum_{i<j} u_j \ecap_i D_{ij}^{\Bu \Bv} + \sum_{i<j} u_i \ecap_j D_{ji}^{\Bu \Bv}\right) \\
   &= \frac{1}{\norm{\Bu}^2} \sum_{i<j} (u_i \ecap_j - u_j \ecap_i) D_{ij}^{\Bu \Bv}
\end{aligned}
\end{equation}

But \(u_i \ecap_j - u_j \ecap_i\) is also a determinant, so writing \(\Be = (\ecap_1, \dots, \ecap_N)\), we have:

\begin{equation}\label{eqn:cross:1101}
\begin{aligned}
\Proj_{\perp\ucap}(\Bv) = \Bv - \Proj_{\ucap}(\Bv)
   &= \frac{1}{\norm{\Bu}^2} \sum_{i<j} (u_i \ecap_j - u_j \ecap_i) D_{ij}^{\Bu \Bv} \\
   &= \frac{1}{\norm{\Bu}^2} \sum_{i<j} D_{ij}^{\Bu \Bv} D_{ij}^{\Bu \Be}
\end{aligned}
\end{equation}

This has squared magnitude:

\begin{equation}\label{eqn:cross:1121}
\begin{aligned}
\norm{\Proj_{\perp\ucap}(\Bv)}^2
   &= \dotprod{\Bv}{\left(\Bv - \Proj_{\ucap}(\Bv)\right)} \\
   &= \frac{1}{\norm{\Bu}^2} \sum_{i<j} (D_{ij}^{\Bu \Bv})^2
\end{aligned}
\end{equation}

Taking the root:
\begin{equation}\label{eqn:cross:101}
\norm{\Proj_{\perp\ucap}(\Bv)}
   = \frac{1}{\norm{\Bu}} \left(\sum_{i<j} (D_{ij}^{\Bu \Bv})^2\right)^{1/2}
\end{equation}

This also yields the area of the \R{N} parallelogram, with the two vectors \(\Bu\), \(\Bv\) as edges:
\begin{equation}\label{eqn:cross:1141}
\begin{aligned}
\Area(\Bu,\Bv) &= \norm{\Bu} \norm{\Proj_{\perp\ucap}(\Bv)} \\
              &= \left(\sum_{i<j} (D_{ij}^{\Bu \Bv})^2\right)^{1/2}
\end{aligned}
\end{equation}

\subsection{Projective form of the cross product.  Part II.  Normal to two vectors in direction of other vector}

The result \(\Proj_{\perp\ucap}(\Bv) = \frac{1}{\norm{\Bu}^2} \sum_{i<j} D_{ij}^{\Bu \Bv} D_{ij}^{\Bu \Be}\)
does not look like the cross product, and it is not.  However, it is also not a normal to two vectors as the cross product is, only one.

If we continue with the calculation of the normal to two vectors (in the direction of a third) something that we can
calculate in \R{N}, it is expected that the result will have similar aspects to the cross product, especially for \R{3}.

Like the one vector normal
\(\Proj_{\perp\ucap}(\Bv)\)
, we can only calculate this definitively for most dimensions when that calculation is with respect to an additional
vector.  Without a reference vector, we can calculate only specific cases.  These are the \R{2} normal to one vector, and a \R{3} normal to two vectors (cross product), or the \R{N} normal to \(N-1\) vectors (and for all of these the result can vary by an arbitrary scalar multiplier).

It is a bit laborious, but let us calculate the normal to two vectors in the direction of a third (the component that is perpendicular to the plane formed by all the linear combinations of the first two vectors).

Let \(\Bm = \Proj_{\perp\ucap}(\Bv)\)

\begin{equation}\label{eqn:cross:1161}
\begin{aligned}
\Proj_{\perp\ucap,\vcap}(\Bw) &= \Bw - \bdotprod{\Bw}{\ucap}
 \ucap - \bdotprod{\Bw}{\mcap} \mcap \\
                             &= \Bw - \frac{1}{\norm{\Bu}^2}
\bdotprod{\Bu}{\Bw}
 \Bu - \frac{1}{\norm{\Bm}^2}\bdotprod{\Bw}{\Bm} \Bm \\
                             &= \frac{1}{\norm{\Bu}^2 \norm{\Bm}^2}\left(\Bw\norm{\Bu}^2\norm{\Bm}^2 - \norm{\Bm}^2
\bdotprod{\Bu}{\Bw}
 \Bu - {\norm{\Bu}^2}\bdotprod{\Bw}{\Bm} \Bm\right)
\end{aligned}
\end{equation}

Expanding, \(\norm{\Bm}^2\), yields:

\begin{equation}\label{eqn:cross:1181}
\begin{aligned}
\norm{\Bm}^2 &=
\dotprod{\left( \Bv - \bdotprod{\ucap}{\Bv} \ucap \right)} {\Bv} \\
             &= \frac{1}{\norm{\Bu}^2}
\bdotprod{\Bv \norm{\Bu}^2 - \bdotprod{\Bu}{\Bv} \Bu}{\Bv} \\
             &= \frac{1}{\norm{\Bu}^2}
\left( \norm{\Bu}^2 \norm{\Bv}^2 - {\bdotprod{\Bu}{\Bv}}^2 \right) \\
\end{aligned}
\end{equation}
%\norm{\Bm}^2 \norm{\Bu}^2 = \norm{\Bu}^2 \norm{\Bv}^2 - {\bdotprod{\Bu}{\Bv}}^2
%\Bm = \Bv - \bdotprod{\ucap}{\Bv}\ucap

\begin{equation}\label{eqn:cross:1201}
\begin{aligned}
\Rightarrow \Proj_{\perp\ucap,\vcap}(\Bw)
&= \frac{1}{\sum_{i<j} (D_{ij}^{\Bu \Bv})^2}
   \left(\Bw\norm{\Bu}^2\norm{\Bm}^2 - \norm{\Bm}^2
\bdotprod{\Bu}{\Bw}
\Bu - {\norm{\Bu}^2}\bdotprod{\Bw}{\Bm} \Bm\right) \\
\end{aligned}
\end{equation}

\begin{equation}\label{eqn:cross:1221}
\begin{aligned}
\Rightarrow \Proj_{\perp\ucap,\vcap}(\Bw) {\sum_{i<j} (D_{ij}^{\Bu \Bv})^2} &=
\Bw \left(\norm{\Bu}^2 \norm{\Bv}^2 - {\bdotprod{\Bu}{\Bv}}^2\right) \\
&- \left(\frac{1}{\norm{\Bu}^2}\left( \norm{\Bu}^2 \norm{\Bv}^2 -
{\bdotprod{\Bu}{\Bv}}^2\right)\right) {\bdotprod{\Bu}{\Bw}} \Bu \\
&- {\norm{\Bu}^2}
\bdotprod{\Bw}{(\Bv - {\bdotprod{\ucap}{\Bv}}\ucap)}
 \left( \Bv - {\bdotprod{\ucap}{\Bv}}\ucap \right)
\end{aligned}
\end{equation}

\begin{equation}\label{eqn:cross:1241}
\begin{aligned}
\Rightarrow \Proj_{\perp\ucap,\vcap}(\Bw) {\sum_{i<j} (D_{ij}^{\Bu \Bv})^2} \norm{\Bu}^2
&= \Bw \left(\norm{\Bu}^4 \norm{\Bv}^2 - {\bdotprod{\Bu}{\Bv}}^2\norm{\Bu}^2\right) \\
&- \left( \norm{\Bu}^2 \norm{\Bv}^2 - {\bdotprod{\Bu}{\Bv}}
^2\right) {\bdotprod{\Bu}{\Bw}} \Bu \\
&- \left( \dotprod{\Bw}({ {\norm{\Bu}^2} \Bv -
{\bdotprod{\Bu}{\Bv}}
\Bu})\right) \left( \Bv {\norm{\Bu}^2} - {\bdotprod{\Bu}{\Bv}}\Bu\right)
\end{aligned}
\end{equation}

\begin{equation}\label{eqn:cross:1261}
\begin{aligned}
&=
   \left( \norm{\Bu}^4 \norm{\Bv}^2 - {\bdotprod{\Bu}{\Bv}}^2\norm{\Bu}^2 \right)                                             \Bw \\
&- \left( {\bdotprod{\Bv}{\Bw}}
 \norm{\Bu}^4 - {\bdotprod{\Bu}{\Bw}}
{\bdotprod{\Bu}{\Bv}} {\norm{\Bu}^2} \right)                \Bv \\
&+ \left( {\bdotprod{\Bv}{\Bw}}
 {\bdotprod{\Bu}{\Bv}}
 {\norm{\Bu}^2} - {\bdotprod{\Bu}{\Bw}} \norm{\Bu}^2 \norm{\Bv}^2 \right)  \Bu
\end{aligned}
\end{equation}

\begin{equation}\label{eqn:cross:1281}
\begin{aligned}
\Rightarrow \Proj_{\perp\ucap,\vcap}(\Bw) {\sum_{i<j} (D_{ij}^{\Bu \Bv})^2}
&= \left( \norm{\Bu}^2 \norm{\Bv}^2 - {\bdotprod{\Bu}{\Bv}}^2 \right)                              \Bw \\
&- \left( \norm{\Bu}^2 {\bdotprod{\Bv}{\Bw}}
 - {\bdotprod{\Bu}{\Bv}}
 {\bdotprod{\Bu}{\Bw}}\right)    \Bv \\
&+ \left( {\bdotprod{\Bu}{\Bv}}
 {\bdotprod{\Bv}{\Bw}}
 - {\bdotprod{\Bu}{\Bw}} \norm{\Bv}^2 \right)   \Bu \\
\end{aligned}
\end{equation}
\begin{equation}\label{eqn:cross:1301}
\begin{aligned}
= \sum_{ijk} \ecap_i ( &\left( u_j u_j v_k v_k - u_j v_j u_k v_k \right) w_i \\
                      +&\left( u_j w_j u_k v_k - u_j u_j v_k w_k \right) v_i \\
                      +&\left( u_j v_j v_k w_k - u_j w_j v_k v_k \right) u_i )
\end{aligned}
\end{equation}
\begin{equation}\label{eqn:cross:1321}
\begin{aligned}
= \sum_{ijk} u_j v_k \ecap_i ( &\left( u_j v_k - v_j u_k \right) w_i \\
                              +&\left( w_j u_k - u_j w_k \right) v_i \\
                              +&\left( v_j w_k - w_j v_k \right) u_i )
\end{aligned}
\end{equation}
\begin{equation}\label{eqn:cross:1341}
\begin{aligned}
&= \sum_{ijk} u_j v_k \ecap_i \left(
u_i D_{jk}^{\Bv \Bw}
+v_i D_{jk}^{\Bw \Bu}
+w_i D_{jk}^{\Bu \Bv}\right)  \\
&= \sum_{i,j<k} \left(u_j v_k - u_k v_j\right) \ecap_i \left( u_i D_{jk}^{\Bv \Bw} + v_i D_{jk}^{\Bw \Bu} + w_i D_{jk}^{\Bu \Bv} \right) \\
&= \sum_{i,j<k} \ecap_i D_{jk}^{\Bu \Bv} \left( u_i D_{jk}^{\Bv \Bw} + v_i D_{jk}^{\Bw \Bu} + w_i D_{jk}^{\Bu \Bv} \right) \\
&= \sum_{i,j<k} \ecap_i D_{jk}^{\Bu \Bv} D_{ijk}^{\Bu \Bv \Bw} \\
&= \sum_{i<j<k} \ecap_i D_{jk}^{\Bu \Bv} D_{ijk}^{\Bu \Bv \Bw}
+ \sum_{j'<i'<k} \ecap_i' D_{j'k}^{\Bu \Bv} D_{i'j'k}^{\Bu \Bv \Bw}
+ \sum_{j'<k'<i'} \ecap_i' D_{j'k'}^{\Bu \Bv} D_{i'j'k'}^{\Bu \Bv \Bw} \\
&= \sum_{i<j<k} \ecap_i D_{jk}^{\Bu \Bv} D_{ijk}^{\Bu \Bv \Bw}
+ \sum_{i<j<k} \ecap_j D_{ik}^{\Bu \Bv} D_{jik}^{\Bu \Bv \Bw}
+ \sum_{i<j<k} \ecap_k D_{ij}^{\Bu \Bv} D_{kij}^{\Bu \Bv \Bw} \\
\end{aligned}
\end{equation}
\begin{equation}\label{eqn:cross:1361}
\begin{aligned}
&= \sum_{i<j<k} \left(\ecap_i D_{jk}^{\Bu \Bv} + \ecap_j D_{ki}^{\Bu \Bv} + \ecap_k D_{ij}^{\Bu \Bv}\right) D_{ijk}^{\Bu \Bv \Bw} \\
&= \sum_{i<j<k} D_{ijk}^{\Bu \Bv \Bw} D_{ijk}^{\Bu \Bv \Be} \\
\end{aligned}
\end{equation}
\begin{equation}\label{eqn:cross:121}
\Rightarrow \Proj_{\perp\ucap,\vcap}(\Bw) =
\frac{1}{\sum_{i<j} \left(D_{ij}^{\Bu \Bv}\right)^2}\sum_{i<j<k} D_{ijk}^{\Bu \Bv \Bw} D_{ijk}^{\Bu \Bv \Be} \\
\end{equation}

Calculating the magnitude of this vector yields a formula for the volume of the \R{N} parallelepiped spanned by three vectors \(\Bu\), \(\Bv\), and \(\Bw\).

\begin{equation}\label{eqn:cross:1381}
\begin{aligned}
\Volume(\Bu, \Bv, \Bw) &= \Area(\Bu, \Bv) \norm{\Proj_{\perp\ucap, \vcap}(\Bw)} \\
                      &= \left(\sum_{i<j<k} \left(D_{ijk}^{\Bu \Bv \Bw}\right)^{2}\right)^{1/2}
\end{aligned}
\end{equation}

\subsection{Summary of \texorpdfstring{\R{N}}{ND} directed normal results}

\begin{equation}\label{eqn:cross:1401}
\begin{aligned}
\Proj_{\perp\ucap}(\Bv)
   &= \Bv - {\bdotprod{\ucap}{\Bv}}\ucap \\
   &= \Bv - \norm{\Bu}^{-2}{\bdotprod{\Bu}{\Bv}}\Bu \\
   &= \frac{1}{\norm{\Bu}^2} \sum_{i<j} D_{ij}^{\Bu \Bv} D_{ij}^{\Bu \Be} \\
   &= \norm{\Bu}^{-2} \sum_{i<j} \DETuvij{u}{v}{i}{j} \DETuvij{u}{\ecap}{i}{j}
\end{aligned}
\end{equation}
\begin{equation}\label{eqn:cross:1421}
\begin{aligned}
\Proj_{\perp\ucap,\vcap}(\Bw)
&= \Bw - \norm{\Bu}^{-2} {\bdotprod{\Bu}{\Bw}}\Bu \\
&- \norm{\Proj_{\perp\ucap}\left(\Bv\right)}^{-2}
\bdotprod{\Proj_{\perp\ucap}\left(\Bv\right)}{\Bw}
 \Proj_{\perp\ucap}(\Bv) \\
&= \frac{1}{\sum_{i<j} \left(D_{ij}^{\Bu \Bv}\right)^2}\sum_{i<j<k} D_{ijk}^{\Bu \Bv \Bw} D_{ijk}^{\Bu \Bv \Be} \\
&= \left(\sum_{i<j} {\DETuvij{u}{v}{i}{j}}^2\right)^{-1} \sum_{i<j<k} \DETuvwijk{u}{v}{w}{i}{j}{k} \DETuvwijk{u}{v}{\ecap}{i}{j}{k} \\
\end{aligned}
\end{equation}

These results are valid for \R{3}, as well as other dimensions \R{N} (including \R{2} which the cross product is not).

For \R{2} the one vector normal, and for \R{3} the two vector normals become:

\begin{equation}\label{eqn:cross:1441}
\begin{aligned}
\Proj_{\perp\ucap}(\Bv)        &= \left(\norm{\Bu}^{-2} \DETuvij{u}{v}{1}{2}\right) \DETuvij{u}{\ecap}{1}{2} \\
                              &= (scalar value) \DETuvij{u}{\ecap}{1}{2} \\
                              &= (scalar value) \VectorTwo{-u_2}{u_1} \\
\Proj_{\perp\ucap,\vcap}(\Bw)
&= \left(\left(\sum_{1 \leq i<j \leq 3} {\DETuvij{u}{v}{i}{j}}^2\right)^{-1} \DETuvwijk{u}{v}{w}{1}{2}{3}\right) \DETuvwijk{u}{v}{\ecap}{1}{2}{3} \\
&= (scalar value) \DETuvwijk{u}{v}{\ecap}{1}{2}{3} \\
%&= (scalar value) \VectorThree {D_{23}^{\Bu\Bv}} {D_{31}^{\Bu\Bv}} {D_{12}^{\Bu\Bv}} \\
&= (scalar value) \VectorThree {u_2 v_3 - u_3 v_2} {u_3 v_1 - u_1 v_3} {u_1 v_2 - u_2 v_1} \\
%&= (scalar value) (\crossprod{\Bu}{\Bv})
\end{aligned}
\end{equation}

The first is a (scaled) normal vector to \(\Bu\), and the second is a (scaled) normal vector to \(\Bu\) and \(\Bv\) (ie: cross product)

\subsection{Summary of \texorpdfstring{\R{N}}{ND} Area, Volume results}

\begin{equation}\label{eqn:cross:1461}
\begin{aligned}
\Area(\Bu, \Bv)
   &= \left(\sum_{i<j} \left(D_{ij}^{\Bu \Bv}\right)^2\right)^{1/2} \\
   &= \left(\sum_{i<j} {\DETuvij{u}{v}{i}{j}}^2\right)^{1/2} \\
\end{aligned}
\end{equation}
\begin{equation}\label{eqn:cross:1481}
\begin{aligned}
\Volume(\Bu, \Bv, \Bw)
&= \left(\sum_{i<j<k} \left(D_{ijk}^{\Bu \Bv \Bw}\right)^2\right)^{1/2} \\
&= \left(\sum_{i<j<k} {\DETuvwijk{u}{v}{w}{i}{j}{k}}^2\right)^{1/2}
\end{aligned}
\end{equation}

For the \R{2} Area, and \R{3} Volume these becomes the familiar determinant and triplet product results:

\begin{equation}\label{eqn:cross:1501}
\begin{aligned}
\Area(\Bu, \Bv)
&= \abs{ D_{ij}^{uv} } \\
&= abs \DETuvij{u}{v}{1}{2}
\end{aligned}
\end{equation}

\begin{equation}\label{eqn:cross:1521}
\begin{aligned}
\Volume(\Bu, \Bv, \Bw)
&= \abs{\dotprod{(\crossprod{\Bu}{\Bv})}{\Bw}} \\
&= abs \DETuvwijk{u}{v}{w}{1}{2}{3}
\end{aligned}
\end{equation}

What was not proved is that this generalizes, and that the m-parallelepiped volume is what we would expect:

\begin{equation}\label{eqn:cross:141}
\Volume(\Bu_1, \dotsc, \Bu_{m}) =
\sum_{i_1 < \dotsb < i_m} \left(D_{i_1,\dotsc,i_m}^{\Bu_1, \dotsc, \Bu_{m}}\right)^2
\end{equation}

This \(\Volume()\) result corresponds to
the length of a vector, area of a parallelogram, and the volume of a parallelepiped for the 1 vector, 2 vector and 3 vector cases respectively, and this has been proved for \R{N} (not just \R{2} or \R{3}).  To prove it for a
\(m+1\) vector parallelogram, in terms of the
\(\Volume(\Bu_1, \dotsc, \Bu_{m})\)
 we need to take the component of this \(m+1\)'th vector that is perpendicular to the the span of all the vectors in the m-parallelepiped and multiply the length of that projection by \(\Volume(\Bu_1, \dotsc, \Bu_{m})\)

Without calculation, it is expected that this perpendicular projection is:

\begin{equation}\label{eqn:cross:1541}
\begin{aligned}
\Proj_{\perp \ucap_1\dotsb\ucap_{m-1}}(\Bu_m)
&=
\Bu_m - \Proj_{\ucap_1\dotsb\ucap_{m-1}}(\Bu_m) \\
&=
\left(\Volume(\Bu_1, \dotsc, \Bu_{m-1})\right)^{-2}
\sum_{i_1 < \dotsb < i_m}
D_{i_1,\dotsc,i_m}^{\Bu_1, \dotsc, \Bu_{m}}
D_{i_1,\dotsc,i_m}^{\Bu_1, \dotsc, \Bu_{m-1}, \Be}
\end{aligned}
\end{equation}

From which the \(m>3\) volume result would follow.  I have not been successful proving this for myself inductively even for
\(m=3\) based on the \(m=2\) result in the form above.  I also note that the books that I have also do not prove this.  Some have a kind of
sneaky way of dealing with this by defining the generalized volume in terms of the wedge product, and never really
demonstrating the geometrical validity of doing so except for \(m=2\) or \(m=3\), or in some cases only for \R{3}.  Since the generalized volume result seems to be uniformly accepted, I am sure some
sufficiently talented mathematician has done the inductive proof for this, perhaps tackling the problem
from some other direction where the result follows more easily.

\section{Normals without a reference vector}

Calculation of a normal above required a reference vector, since there can be normals in many different directions to a set of \(n \in\) \R{N} vectors unless \(n = N-1\).  In the \(n = N-1\) case, the normal only varies by a scalar multiplier.

In this section the normal to a set of vectors will be calculated without introducing a
reference vector.  This is closer to the formulation one would expect of a generalized cross
product, but does generally introduce a set of undetermined coefficients.  We can then
compare the to results for the normals taken in the direction of a reference vector.

\subsection{orthogonality and the Null Space of two vectors}

Using just an
orthogonality condition is not enough to uniquely define a ``cross product''
even in \R{3}, but for \R{3} that is good within at least a scalar multiple.

For \R{N}, lets calculate via row reduction the Null Space of a matrix with rows formed of the elements of the two vectors \(\Bu\) and \(\Bv\), and then solve for \(\Bn\).

\begin{equation}\label{eqn:cross:161}
\begin{bmatrix}
u_1 & u_2 & \dotsb & u_N \\
v_1 & v_2 & \dotsb & v_N
\end{bmatrix}
\VectorN{n}
= \Bzero
\end{equation}

The row reduction can be performed with any set of two columns, not just the first two (the first two could be all zeros for example).  So for generality, we Row reduce based on columns \(i\) and \(j\):

\begin{equation}\label{eqn:cross:181}
\begin{bmatrix}
v_j & -u_j \\
-v_i & u_i
\end{bmatrix}
\begin{bmatrix}
u_1 & u_2 & \dotsb & u_N \\
v_1 & v_2 & \dotsb & v_N
\end{bmatrix}
\VectorN{n} = \Bzero
\end{equation}

For column \(k\) this is:
\begin{equation}\label{eqn:cross:201}
\begin{bmatrix}
v_j & -u_j \\
-v_i & u_i
\end{bmatrix}
\VectorTwo{u_k}{v_k}
=
\VectorTwo{ u_k v_j - u_j v_k }{ u_i v_k - u_k v_i }
=
\VectorTwo{ D_{kj}^{\Bu\Bv} }{ D_{ik}^{\Bu\Bv} }
\end{equation}

In particular, for columns \(i\), and \(j\) respectively this is:

\begin{equation}\label{eqn:cross:221}
\VectorTwo{ D_{ij}^{\Bu\Bv} }{ 0 }
,
\VectorTwo{ 0 }{ D_{ij}^{\Bu\Bv} }
\end{equation}

And we are left with two sets of equations in \(N-2\) dependent variables.

\begin{equation}\label{eqn:cross:1561}
\begin{aligned}
D_{ij}^{\Bu\Bv} n_i &= \sum_{k \neq i,j}D_{jk}^{\Bu\Bv} n_k \\
D_{ij}^{\Bu\Bv} n_j &= \sum_{k \neq i,j}D_{ki}^{\Bu\Bv} n_k \\
\end{aligned}
\end{equation}

Now, let \(n_k = t_k\) for \(k \neq i,j\), and \(t_k\) is an arbitrary constant, and combining these equations:

\begin{equation}\label{eqn:cross:1581}
\begin{aligned}
D_{ij}^{\Bu\Bv} \Bn
&= \sum_s D_{ij}^{\Bu\Bv} n_s \ecap_s \\
&= D_{ij}^{\Bu\Bv} n_i \ecap_i + D_{ij}^{\Bu\Bv} n_j \ecap_j +
\sum_{k \neq i,j} n_k \ecap_k D_{ij}^{\Bu\Bv} \\
&=
\ecap_i \sum_{k \neq i,j}{D_{jk}^{\Bu\Bv} t_k} +
\ecap_j \sum_{k \neq i,j}{D_{ki}^{\Bu\Bv} t_k} +
\sum_{k \neq i,j} t_k \ecap_k D_{ij}^{\Bu\Bv} \\
&=
\sum_{k \neq i,j} t_k
\left(
\ecap_i {D_{jk}^{\Bu\Bv}} +
\ecap_j {D_{ki}^{\Bu\Bv}} +
\ecap_k {D_{ij}^{\Bu\Bv}}
\right) \\
&=
\sum_{k \neq i,j} t_k D_{ijk}^{\Bu\Bv\Be}
\end{aligned}
\end{equation}

Since \(i\) and \(j\) were chosen arbitrarily, this is really the sum over all sets of unique combinations of \(i\), \(j\), and \(k\), so we can write the most
generic normal to a pair of vectors in \R{N} as
\begin{equation}\label{eqn:cross:1601}
\begin{aligned}
\Bn
&= \sum_{i,j,k} s_{ijk} D_{ijk}^{\Bu\Bv\Be} \\
&= \sum_{i<j<k} \left( s_{ijk} - s_{ikj} + s_{kij} - s_{kji} + s_{jki} - s_{jik} \right) D_{ijk}^{\Bu\Bv\Be} \\
&= \sum_{i<j<k} \left( \sum_{\pi_x(i,j,k)} s_{\pi_x}\Sgn(\pi_x) \right) D_{ijk}^{\Bu\Bv\Be} \\
&= \sum_{i<j<k} \left( \sum_{\pi_x \in \pi(i,j,k)} s_{\pi_x}\Sgn(\pi_x) \right) \DETuvwijk{u}{v}{\ecap}{i}{j}{k} \\
\end{aligned}
\end{equation}

Here \(s_{ijk} = t_k/{D_{ij}^{\Bu\Bv}}\), and \(\pi(i,j,k)\) are the permutations of the indices \(i\), \(j\), and \(k\), and \(\Sgn\) is the sign of the individual permutation \(\pi_x\) in that set (-1 for odd numbers of index switches, 1 for even numbers of switches).

For \R{3}, we once again have a scaled cross product.  Also observe that the coefficient term is much like a determinant, the sign alternates with the switch of any two indices and zero if any indices match.  This is not surprising given the earlier calculation of the normal in the direction of a reference vector, was in fact a determinant.

Because the values \(s_{ijk}\) were arbitrary constants, so is the composite value \(s_{ijk}' = \sum_{\pi_x \in \pi(i,j,k)} s_{\pi_x}\Sgn(\pi_x)\), so really this is just a statement that:

\begin{equation}\label{eqn:cross:241}
\Bn \in \Span \left\lbrace
\DETuvwijk{u}{v}{\ecap}{i}{j}{k}
\right\rbrace
\end{equation}

A small note here about the dependence of this result on the field of real numbers.  The normal that was calculated here is not normal for \C{N}, but is a sort of
conjugate normal.  One would have to row reduce the complex conjugates of the vectors to produce a result that is valid for \C{N}.  If one replaces the components in the determinants with their conjugates that should correct the result.

\subsection{orthogonality and the Null Space of one vector}

Having done the calculation for the two vector case, a result like \(\Proj_{\perp\ucap}(\Bv)\) is expected.  Here is the calculation that verifies this:

\begin{equation}\label{eqn:cross:1621}
\begin{aligned}
 (n_i u_i) \ecap_i &= \left( -\sum_{j \neq i}n_j u_j \right) \ecap_i \\
                   &= \left( -\sum_{j \neq i}t_j u_j \right) \ecap_i
\end{aligned}
\end{equation}
\begin{equation}\label{eqn:cross:1641}
\begin{aligned}
 u_i (n_j \ecap_j) &= u_i ( t_j \ecap_j)
\end{aligned}
\end{equation}
\begin{equation}\label{eqn:cross:1661}
\begin{aligned}
 \Rightarrow  u_i \Bn  &= \sum_{j \neq i}( u_i t_j \ecap_j - t_j u_j \ecap_i ) \\
                       &= \sum_{j \neq i} t_j ( u_i \ecap_j - u_j \ecap_i ) \\
                       &= \sum_{j \neq i} t_j D_{ij}^{\Bu\Be}
\end{aligned}
\end{equation}
Let \(s_{ij} = t_j/u_i\),
\begin{equation}\label{eqn:cross:1681}
\begin{aligned}
 \Rightarrow \Bn       &= \sum_{i<j} (s_{ij} - s_{ji}) D_{ij}^{\Bu\Be} \\
                       &= \sum_{i<j} \left(\sum_{\pi_x \in \pi(i,j)}s_{\pi_x} \Sgn(\pi_x)\right) D_{ij}^{\Bu\Be} \\
                       &= \sum_{i<j} s_{ij}' D_{ij}^{\Bu\Be}
\end{aligned}
\end{equation}

With the expected result:

\begin{equation}\label{eqn:cross:261}
\Bn \in \Span \left\lbrace
\DETuvij{u}{\ecap}{i}{j}
\right\rbrace
\end{equation}

\subsection{orthogonality and the Null Space of three vectors}

The calculation of \(\Proj_{\perp\ucap\vcap}(\Bw)\) was pretty laborious.  Without actually calculating it
the expected result for, it is expected that:

\begin{equation}\label{eqn:cross:281}
\Proj_{\perp\ucap\vcap\wcap}(\Bx) = {\Volume(\Bu, \Bv, \Bw)}^{-2} \sum_{i<j<k<l} D_{ijkl}^{\Bu\Bv\Bw\Bx} D_{ijkl}^{\Bu\Bv\Bw\Be}
\end{equation}

There is probably a way to verify this inductively, but it is not obvious to me how to approach this.  One the
other hand the calculation of the Null Space of three vectors is not too hard.

First calculate the Cofactor matrix for
\begin{equation}\label{eqn:cross:301}
\begin{bmatrix}
u_i & u_j & u_k \\
v_i & v_j & v_k \\
w_i & w_j & w_k \\
\end{bmatrix}
\xrightarrow{Transpose}
\begin{bmatrix}
u_i & v_i & w_i \\
u_j & v_j & w_j \\
u_k & v_k & w_k \\
\end{bmatrix}
\xrightarrow{Cofactors}
\begin{bmatrix}
D_{jk}^{\Bv\Bw} & -D_{jk}^{\Bu\Bw} & D_{jk}^{\Bu\Bv} \\
-D_{ik}^{\Bv\Bw} & D_{ik}^{\Bu\Bw} & -D_{ik}^{\Bu\Bv} \\
D_{ij}^{\Bv\Bw} & -D_{ij}^{\Bu\Bw} & D_{ij}^{\Bu\Bv}
\end{bmatrix}
\end{equation}
\begin{equation}\label{eqn:cross:321}
\Rightarrow
\begin{bmatrix}
D_{jk}^{\Bv\Bw} & -D_{jk}^{\Bu\Bw} & D_{jk}^{\Bu\Bv} \\
-D_{ik}^{\Bv\Bw} & D_{ik}^{\Bu\Bw} & -D_{ik}^{\Bu\Bv} \\
D_{ij}^{\Bv\Bw} & -D_{ij}^{\Bu\Bw} & D_{ij}^{\Bu\Bv}
\end{bmatrix}
\VectorThree{u_m}{v_m}{w_m}
=
\begin{bmatrix}
u_m D_{jk}^{\Bv\Bw} & -v_m D_{jk}^{\Bu\Bw} & w_m D_{jk}^{\Bu\Bv} \\
u_m D_{ki}^{\Bv\Bw} & -v_m D_{ki}^{\Bu\Bw} & w_m D_{ki}^{\Bu\Bv} \\
u_m D_{ij}^{\Bv\Bw} & -v_m D_{ij}^{\Bu\Bw} & w_m D_{ij}^{\Bu\Bv}
\end{bmatrix}
=
\VectorThree
{D_{mjk}^{\Bu\Bv\Bw}}
{D_{mki}^{\Bu\Bv\Bw}}
{D_{mij}^{\Bu\Bv\Bw}}
\end{equation}

Columns \(m=i\), \(m=j\), and \(m=k\) are respectively,
\begin{equation}\label{eqn:cross:341}
\VectorThree
{D_{ijk}^{\Bu\Bv\Bw}}
{0}
{0}
,
\VectorThree
{0}
{D_{ijk}^{\Bu\Bv\Bw}}
{0}
,
\VectorThree
{0}
{0}
{D_{ijk}^{\Bu\Bv\Bw}}
\end{equation}

With the introduction of free parameters \(n_m = t_m\) we have four equations,

\begin{equation}\label{eqn:cross:1701}
\begin{aligned}
{D_{ijk}^{\Bu\Bv\Bw}} n_i \ecap_i &= - \ecap_i \sum_{m \neq i,j,k} t_m {D_{mjk}^{\Bu\Bv\Bw}} \\
{D_{ijk}^{\Bu\Bv\Bw}} n_j \ecap_j &= - \ecap_j \sum_{m \neq i,j,k} t_m {D_{mki}^{\Bu\Bv\Bw}} \\
{D_{ijk}^{\Bu\Bv\Bw}} n_k \ecap_k &= - \ecap_k \sum_{m \neq i,j,k} t_m {D_{mij}^{\Bu\Bv\Bw}} \\
\sum_{m \neq i,j,k} {D_{ijk}^{\Bu\Bv\Bw}} n_m \ecap_m &= \sum_{m \neq i,j,k} {D_{ijk}^{\Bu\Bv\Bw}} t_m \ecap_m
\end{aligned}
\end{equation}

Adding these
\begin{equation}\label{eqn:cross:1721}
\begin{aligned}
{D_{ijk}^{\Bu\Bv\Bw}} \Bn
&= \sum_{m} {D_{ijk}^{\Bu\Bv\Bw}} n_m \ecap_m \\
&=
\sum_{m \neq i,j,k} t_m \left(
  \ecap_m {D_{ijk}^{\Bu\Bv\Bw}}
- \ecap_i {D_{jkm}^{\Bu\Bv\Bw}}
+ \ecap_j {D_{kmi}^{\Bu\Bv\Bw}}
- \ecap_k {D_{mij}^{\Bu\Bv\Bw}}
\right) \\
&=
\sum_{m \neq i,j,k} t_m D_{mijk}^{\Bu\Bv\Bw\Be}
\end{aligned}
\end{equation}

A result exactly like the one and two vector cases, with the same conclusion:

\begin{equation}\label{eqn:cross:361}
\Rightarrow
\Bn \in \Span \left\lbrace
\DETuvwxijkl{u}{v}{w}{\ecap}{i}{j}{k}{l}
\right\rbrace
\end{equation}

\subsection{orthogonality and complex vector products}
The same thing can be done for the complex inner product, where
for orthogonality the term,
\begin{equation}\label{eqn:cross:381}
\sum_i{ u_i \overline{v_i} + v_i \overline{u_i}}
\end{equation}
must be zero.

If \(\sum_i{ u_i \overline{v_i}} = 0\), this implies
\(\overline{\sum_i{ u_i \overline{v_i}}} = \sum_i{v_i \overline{u_i}} = 0\), so the definitions of both the
complex and the real inner products arise naturally from an examination of orthogonality constraints.

\section{Introducing the wedge product}

Now, some clever mathematicians have observed that the underlying properties of the determinant is what is important in all these problems of normal, area, and volume.

In particular the property that it changes sign if two of its elements (rows or columns) are switched, and is zero if any of its elements are the same, and it is linear in either variable.

%linearity demo/proof (using cofactor expansion on first row ... any other would do).
% det(a+b, c) = (a_1 + b_1)c_2 - (a_2 + b_2)c_1 = det(a,c) + det(b,c)
% det(a+b, ..., c) = \sum(a_i + b_i)C_1i = \sum(a_i)C_1i +\sum(b_i)C_1i = det(a,...c) + det(b,...,c)

The wedge product is an operator defined based on these two properties.  Symbolically, for two vectors \(\Bu\) and \(\Bv\), these rules are:

\begin{equation}\label{eqn:cross:1741}
\begin{aligned}
\Bu \wedge \Bv &= - (\Bv \wedge \Bu) \\
\Bu \wedge \Bu &= \Bzero \\
(\Bu + \Bv) \wedge \Bw &= \Bu \wedge \Bw + \Bv \wedge \Bw
\end{aligned}
\end{equation}

And for a third vector the wedge defined as:
\begin{equation}\label{eqn:cross:401}
\Bu \wedge \Bv \wedge \Bw = (\Bu \wedge \Bv) \wedge \Bw = \Bu \wedge (\Bv \wedge \Bw)
\end{equation}

Now, intuitively this is a more awkward sort of product than the dot or cross product.  It is it is own thing, in general having no direct mapping to a vector (like the cross product), nor to a scalar (like the dot product).  We will see the dimensions of this beast is not even necessarily close to the dimensions of the original vector space (and that dimension varies according to how many wedge products are composed).

To get a feel for this quantity, an expansion of this in terms of components is helpful (I would have liked to have seen this spelled out in my books for the dumb reader like me).

With
\( \Bu = \sum_i{u_i \ecap_i} \), and \( \Bv = \sum_i{v_i \ecap_i}\), here is the expansion of \(\Bu \wedge \Bv\) .

\begin{equation}\label{eqn:cross:1761}
\begin{aligned}
\Bu \wedge \Bv
&= \left(\sum_i{u_i \ecap_i}\right) \wedge \left(\sum_j{v_j \ecap_j}\right) \\
&= \sum_{i,j}{u_i v_j \left(\ecap_i \wedge \ecap_j\right)} \\
&= \sum_{i \neq j}{u_i v_j \left(\ecap_i \wedge \ecap_j\right)} \\
&= \sum_{i < j}{u_i v_j \left(\ecap_i \wedge \ecap_j\right)} +
   \sum_{j' < i'}{u_{i'} v_{j'} \left(\ecap_{i'} \wedge \ecap_{j'}\right)} \\
&= \sum_{i < j}\left(u_i v_j - u_{j} v_{i}\right)\left(\ecap_i \wedge \ecap_j\right) \\
&= \sum_{i < j}\DETuvij{u}{v}{i}{j}\left(\ecap_i \wedge \ecap_j\right) \\
&= \sum_{i < j}D_{ij}^{\Bu\Bv}\left(\ecap_i \wedge \ecap_j\right)
\end{aligned}
\end{equation}

Introducing a third wedge:
\begin{equation}\label{eqn:cross:1781}
\begin{aligned}
\Bu \wedge \Bv \wedge \Bw
&= \left(\sum_{i < j}D_{ij}^{\Bu\Bv}\left(\ecap_i \wedge \ecap_j\right)\right) \wedge
\Bw = \sum_k{w_k \ecap_k} \\
&= \sum_{i < j, k}w_k D_{ij}^{\Bu\Bv}\left(\ecap_i \wedge \ecap_j \wedge \ecap_k\right) \\
&= \sum_{i < j, k \neq i,j}w_k D_{ij}^{\Bu\Bv}\left(\ecap_i \wedge \ecap_j \wedge \ecap_k\right) \\
&=
\left(\sum_{i < j < k} +
\sum_{i < k < j} +
\sum_{k < i < j}\right)
w_k D_{ij}^{\Bu\Bv}\left(\ecap_i \wedge \ecap_j \wedge \ecap_k\right) \\
&=
\sum_{i < j < k} \left(w_k D_{ij}^{\Bu\Bv}\left(\ecap_i \wedge \ecap_j \wedge \ecap_k\right) +
w_j D_{ik}^{\Bu\Bv}\left(\ecap_i \wedge \ecap_k \wedge \ecap_j\right) +
w_i D_{jk}^{\Bu\Bv}\left(\ecap_j \wedge \ecap_k \wedge \ecap_i\right)\right)
 \\
&=
\sum_{i < j < k} \left(w_k D_{ij}^{\Bu\Bv} - w_j D_{ik}^{\Bu\Bv} + w_i D_{jk}^{\Bu\Bv}\right)
\left(\ecap_i \wedge \ecap_j \wedge \ecap_k\right)
 \\
&=
\sum_{i < j < k} D_{ijk}^{\Bu\Bv\Bw}\left(\ecap_i \wedge \ecap_j \wedge \ecap_k\right)
 \\
\end{aligned}
\end{equation}

Written out in full, these are:
\begin{equation}\label{eqn:cross:1801}
\begin{aligned}
\Bu \wedge \Bv
&=
\sum_{i < j} \DETuvij{u}{v}{i}{j}\left(\ecap_i \wedge \ecap_j\right)  \\
\Bu \wedge \Bv \wedge \Bw
&=
\sum_{i < j < k} \DETuvwijk{u}{v}{w}{i}{j}{k}\left(\ecap_i \wedge \ecap_j \wedge \ecap_k\right)
 \\
\end{aligned}
\end{equation}

A couple observations.

The set
$\left\lbrace \ecap_i \wedge \ecap_j
| i < j
 \right\rbrace
$ is
a basis for the two vector ``wedge product space'', and the set
$\left\lbrace \ecap_i \wedge \ecap_j \wedge \ecap_k
| i < j < k
\right\rbrace
$ is
a basis for the three vector ``wedge product space''.

With these values taken as the basis ``vectors'' for the product space, the natural dot product of two
vectors would be:

\begin{equation}\label{eqn:cross:1821}
\begin{aligned}
\dotprod{(\Bu \wedge \Bv)}{(\Bw \wedge \Bx)}
&=
\dotprod{\left(\sum_{i < j}D_{ij}^{\Bu\Bv}\left(\ecap_i \wedge \ecap_j\right)\right)}
{\left(\sum_{{s} < {t}}D_{{s}{t}}^{\Bw\Bx}\left(\ecap_{s} \wedge \ecap_{t}\right)\right)} \\
&=
\sum_{i < j}\sum_{s < t} D_{ij}^{\Bu\Bv} D_{{s}{t}}^{\Bw\Bx}
\dotprod{\left(\ecap_i \wedge \ecap_j\right)}{\left(\ecap_{s} \wedge \ecap_{t}\right)} \\
&=
\sum_{i < j}\sum_{s < t} D_{ij}^{\Bu\Bv} D_{{s}{t}}^{\Bw\Bx} \delta_{ij,st} \\
&=
\sum_{i < j}D_{ij}^{\Bu\Bv} D_{ij}^{\Bw\Bx}
\end{aligned}
\end{equation}

And in particular, this defines the length of a wedge product, which can be written in terms of the area of the parallelogram spanned by the two ``wedged'' vectors.

\begin{equation}\label{eqn:cross:1841}
\begin{aligned}
\norm{(\Bu \wedge \Bv)}^2
&= \sum_{i < j}\left(D_{ij}^{\Bu\Bv}\right)^2 \\
&= \left(\Area(\Bu,\Bv)\right)^2
\end{aligned}
\end{equation}

For the length of a three vector wedge, we have a length equivalent to the volume of the parallelepiped spanned by the three vectors:
% I think of a proof of this is not neccessary.  Exactly like the above.
\begin{equation}\label{eqn:cross:1861}
\begin{aligned}
\norm{(\Bu \wedge \Bv \wedge \Bw)}^2
&= \sum_{i < j < k}\left(D_{ijk}^{\Bu\Bv\Bw}\right)^2 \\
&= \left(\Volume(\Bu,\Bv,\Bw)\right)^2
\end{aligned}
\end{equation}

So, the elements
\(D_{i_1 \dotsb i_M}^{\Bu^1 \dotsb \Bu^M}\left( \ecap_{i_1} \wedge \dotsb \wedge \ecap_{i_M} \right)\)
of the wedge product
can be thought of as oriented ``\(\Volume\)'' elements of the subspace spanned by the \(M\) vectors.

\subsection{Comparing the wedge product to the normal of \texorpdfstring{\(N-1\)}{N - 1} independent vectors in \texorpdfstring{\R{N}}{R N}}

We have three variations now that generalize the cross product of two vectors in different ways:

\begin{equation}\label{eqn:cross:1881}
\begin{aligned}
\Bu \wedge \Bv &= \sum_{i < j} \DETuvij{u}{v}{i}{j}\left(\ecap_i \wedge \ecap_j\right)  \\
\Proj_{\perp\ucap,\vcap}(\Bw)
&= \left({\sum_{i<j} \left(D_{ij}^{\Bu \Bv}\right)^2}\right)^{-2}\sum_{i<j<k} \DETuvwijk{u}{v}{w}{i}{j}{k} \DETuvwijk{u}{v}{\ecap}{i}{j}{k} \\
\Bn(\Bu,\Bv)
&= \sum_{i<j<k} \left( \sum_{\pi_x \in \pi(i,j,k)} s_{\pi_x}\Sgn(\pi_x) \right) \DETuvwijk{u}{v}{\ecap}{i}{j}{k} \\
\end{aligned}
\end{equation}

The wedge product of \(N-1\) \R{N} vectors and the normal those \(N-1\) vectors (assuming that they are all linearly independent) are a very close match.  Let us compare these for \R{3}, \R{4} and \R{5}.

%\R{2}:
%\begin{align*}
%\Bn(\Bu)
%& \propto \DETuvij{u}{\ecap}{1}{2} \\
%&= (+\ecap_1) |u_2| \\
%&+ (-\ecap_2) |u_1|
%\end{align*}

For \R{3} the normal to two vectors is of the following form:
\begin{equation}\label{eqn:cross:1901}
\begin{aligned}
\Bn(\Bu,\Bv)
&\propto \DETuvwijk{u}{v}{\ecap}{1}{2}{3} \\
&= (+\ecap_1) \DETuvij{u}{v}{2}{3} \\
&+ (-\ecap_2) \DETuvij{u}{v}{1}{3} \\
&+ (+\ecap_3) \DETuvij{u}{v}{1}{2} \\
\end{aligned}
\end{equation}

Compare this to the wedge product of two \R{3} vectors:
\begin{equation}\label{eqn:cross:1921}
\begin{aligned}
\Bu \wedge \Bv
&= (\ecap_2 \wedge \ecap_3) \DETuvij{u}{v}{2}{3} \\
&+ (\ecap_1 \wedge \ecap_3) \DETuvij{u}{v}{1}{3} \\
&+ (\ecap_1 \wedge \ecap_2) \DETuvij{u}{v}{1}{2} \\
\end{aligned}
\end{equation}

Similarly, for \R{4} the normal to three vectors is of the following form:

\R{4}:
\begin{equation}\label{eqn:cross:1941}
\begin{aligned}
\Bn(\Bu,\Bv,\Bw)
&\propto \DETuvwxijkl{u}{v}{w}{\ecap}{1}{2}{3}{4} \\
&= (+\ecap_1) \DETuvwijk{u}{v}{w}{2}{3}{4} + (-\ecap_2) \DETuvwijk{u}{v}{w}{1}{3}{4} \\
&+ (+\ecap_3) \DETuvwijk{u}{v}{w}{1}{2}{4} + (-\ecap_4) \DETuvwijk{u}{v}{w}{1}{2}{3} \\
\end{aligned}
\end{equation}

Compare this to the wedge product of three \R{4} vectors:

\begin{equation}\label{eqn:cross:1961}
\begin{aligned}
\Bu \wedge \Bv \wedge \Bw
&= (\ecap_{2} \wedge \ecap_{3} \wedge \ecap_{4}) \DETuvwijk{u}{v}{w}{2}{3}{4}
+ (\ecap_{1} \wedge \ecap_{3} \wedge \ecap_{4}) \DETuvwijk{u}{v}{w}{1}{3}{4} \\
&+ (\ecap_{1} \wedge \ecap_{2} \wedge \ecap_{4}) \DETuvwijk{u}{v}{w}{1}{2}{4}
+ (\ecap_{1} \wedge \ecap_{2} \wedge \ecap_{3}) \DETuvwijk{u}{v}{w}{1}{2}{3} \\
%\sum_{i < j < k} \DETuvwijk{u}{v}{w}{i}{j}{k}\left(\ecap_i \wedge \ecap_j \wedge \ecap_k\right)
\end{aligned}
\end{equation}

And finally, for \R{5} the normal to four vectors is of the following form:
\begin{equation}\label{eqn:cross:1981}
\begin{aligned}
\Bn(\Bu,\Bv,\Bw,\Bx)
&\propto
%\DETuvwxyijklm{u}{v}{w}{x}{\ecap}{1}{2}{3}{4}{5} \\
\begin{vmatrix}
 {u}_{1} & {u}_{2} & {u}_{3} & {u}_{4} & {u}_{5} \\
 {v}_{1} & {v}_{2} & {v}_{3} & {v}_{4} & {v}_{5} \\
 {w}_{1} & {w}_{2} & {w}_{3} & {w}_{4} & {w}_{5} \\
 {x}_{1} & {x}_{2} & {x}_{3} & {x}_{4} & {x}_{5} \\
 {\ecap}_{1} & {\ecap}_{2} & {\ecap}_{3} & {\ecap}_{4} & {\ecap}_{5}
\end{vmatrix} \\
&= (+\ecap_1) \DETuvwxijkl{u}{v}{w}{x}{2}{3}{4}{5} \\
&+ (-\ecap_2) \DETuvwxijkl{u}{v}{w}{x}{1}{3}{4}{5} \\
&+ (+\ecap_3) \DETuvwxijkl{u}{v}{w}{x}{1}{2}{4}{5} \\
&+ (-\ecap_4) \DETuvwxijkl{u}{v}{w}{x}{1}{2}{3}{4} \\
&+ (+\ecap_5) \DETuvwxijkl{u}{v}{w}{x}{1}{2}{3}{4} \\
\end{aligned}
\end{equation}

Compare this to the wedge product of four \R{5} vectors:
\begin{equation}\label{eqn:cross:2001}
\begin{aligned}
\Bu \wedge \Bv \wedge \Bw \wedge \Bx
%%% \sum_{i < j < k < l} \DETuvwijkl{u}{v}{w}{x}{i}{j}{k}{l}\left(\ecap_i \wedge \ecap_j \wedge \ecap_k \wedge \ecap_l \right)
&= \left(\ecap_{2} \wedge \ecap_{3} \wedge \ecap_{4} \wedge \ecap_{5} \right) \DETuvwxijkl{u}{v}{w}{x}{2}{3}{4}{5} \\
&+ \left(\ecap_{1} \wedge \ecap_{3} \wedge \ecap_{4} \wedge \ecap_{5} \right) \DETuvwxijkl{u}{v}{w}{x}{1}{3}{4}{5} \\
&+ \left(\ecap_{1} \wedge \ecap_{2} \wedge \ecap_{4} \wedge \ecap_{5} \right) \DETuvwxijkl{u}{v}{w}{x}{1}{2}{4}{5} \\
&+ \left(\ecap_{1} \wedge \ecap_{2} \wedge \ecap_{3} \wedge \ecap_{4} \right) \DETuvwxijkl{u}{v}{w}{x}{1}{2}{3}{4} \\
&+ \left(\ecap_{1} \wedge \ecap_{2} \wedge \ecap_{3} \wedge \ecap_{4} \right) \DETuvwxijkl{u}{v}{w}{x}{1}{2}{3}{4} \\
\end{aligned}
\end{equation}

We see that the wedge product is a normal to two vectors in \R{3} (in fact is the cross product) if we make the identities:
\begin{equation}\label{eqn:cross:2021}
\begin{aligned}
 \ecap_1 &= \ecap_2 \wedge \ecap_3 \\
-\ecap_2 &= \ecap_1 \wedge \ecap_3 \\
 \ecap_3 &= \ecap_1 \wedge \ecap_2 \\
\end{aligned}
\end{equation}

And, the wedge product is a normal to three vectors in \R{4} if we make the identities:
\begin{equation}\label{eqn:cross:2041}
\begin{aligned}
  \ecap_1 &= \ecap_{2} \wedge \ecap_{3} \wedge \ecap_{4} \\
 -\ecap_2 &= \ecap_{1} \wedge \ecap_{3} \wedge \ecap_{4} \\
  \ecap_3 &= \ecap_{1} \wedge \ecap_{2} \wedge \ecap_{4} \\
 -\ecap_4 &= \ecap_{1} \wedge \ecap_{2} \wedge \ecap_{3} \\
\end{aligned}
\end{equation}

And, the wedge product is a normal to four vectors in \R{5} if we make the identities:
\begin{equation}\label{eqn:cross:2061}
\begin{aligned}
  \ecap_1 &= \ecap_{2} \wedge \ecap_{3} \wedge \ecap_{4} \wedge \ecap_{5}   \\
 -\ecap_2 &= \ecap_{1} \wedge \ecap_{3} \wedge \ecap_{4} \wedge \ecap_{5}   \\
  \ecap_3 &= \ecap_{1} \wedge \ecap_{2} \wedge \ecap_{4} \wedge \ecap_{5}   \\
 -\ecap_4 &= \ecap_{1} \wedge \ecap_{2} \wedge \ecap_{3} \wedge \ecap_{4}   \\
  \ecap_5 &= \ecap_{1} \wedge \ecap_{2} \wedge \ecap_{3} \wedge \ecap_{4}   \\
\end{aligned}
\end{equation}

So, as well as \(\Bu \wedge \Bv\) being an oriented parallelogram area vector, and \(\Bu \wedge \Bv \wedge \Bw\) being an oriented parallelepiped volume vector, 
the wedge product of \(N-1\) linearly independent vectors in \R{N} 
is also normal to those vectors if we make the identity:

\begin{equation}\label{eqn:cross:421}
  \ecap_j = \Sgn(j, i_1, \dotsc, i_{N-1}) \ecap_{i_1} \wedge \dotsb \wedge \ecap_{i_{N-1}}
\end{equation}
Or,
\begin{equation}\label{eqn:cross:2081}
\begin{aligned}
\ecap_{i_1} \wedge \dotsb \wedge \ecap_{i_{N-1}} &= \Sgn(j, i_1, \dotsc, i_{N-1}) \ecap_j \\
                                                 &= (-1)^{j+1} \ecap_j 
\end{aligned}
\end{equation}

Where \(i_1 < i_2 < \dotsb < i_{N-1}\) and \(j \notin \lbrace i_1, \dotsc, i_{N-1} \rbrace\).

\subsection{Comparing the wedge product and the cross product and triple product}

For the wedge of \(N-1\) vectors in \R{N}, and using the mapping from this wedge product to a normal to these vectors we have:

\begin{equation}\label{eqn:cross:2101}
\begin{aligned}
\Bu^{1} \wedge \dotsb \wedge \Bu^{N-1} &=
\sum_{i_1<\dotsb <i_{N-1}}{ D_{i_1 \dotsb i_{N-1}}^{\Bu^{1}\dotsc\Bu^{N-1}}\left( \ecap_{i_1} \wedge \dotsb \wedge \ecap_{i_{N-1}} \right) } \\
&=
\sum_{i_1<\dotsb <i_{N-1}}{ D_{i_1 \dotsb i_{N-1}}^{\Bu^{1}\dotsc\Bu^{N-1}}
\left( 
%\ecap_{i_1} \wedge \dotsb \wedge \ecap_{i_{N-1}} 
\Sgn(j, i_1, \dotsc, i_{N-1}) \ecap_j
\right)
} \\
\end{aligned}
\end{equation}

\begin{equation}\label{eqn:cross:Nminus1Product}
\Rightarrow
\Bu^{1} \wedge \dotsb \wedge \Bu^{N-1} =
\sum_{i_1<\dotsb <i_{N-1}}{ D_{i_1 \dotsb i_{N-1}}^{\Bu^{1}\dotsc\Bu^{N-1}}
\left( 
%\Sgn(j, i_1, \dotsc, i_{N-1}) \ecap_j
(-1)^{j+1} \ecap_j 
\right)
}
\end{equation}

Where, as before \(j \notin \lbrace i_1, \dotsc, i_{N-1} \rbrace\).

(Note that for \R{3} this is exactly the cross product \(\crossprod{\Bu^1}{\Bu^2}\).)

Continuing with the wedge of \(N-1\) vectors as a normal in \R{N} we can write:

\begin{equation}\label{eqn:cross:2121}
\begin{aligned}
\dotprod{\left(\Bu^{1} \wedge \dotsb \wedge \Bu^{N-1}\right)}{\Bu^N}
&=
\sum_{j,i_1<\dotsb <i_{N-1}}
   { 
      D_{i_1 \dotsb i_{N-1}}^{\Bu^{1}\dotsc\Bu^{N-1}}
      \left( (-1)^{j+1} u_j^N \right)
   } \\
&=
D_{1 \dotsb N}^{\Bu^{1}\dotsc\Bu^{N}}
\end{aligned}
\end{equation}

But,
\begin{equation}\label{eqn:cross:441}
\Bu^{1} \wedge \dotsb \wedge \Bu^{N} =
D_{1 \dotsb N}^{\Bu^{1}\dotsc\Bu^{N}}
\left(\ecap_{1} \wedge \dotsb \wedge \ecap_{N}\right)
\end{equation}
\begin{equation}\label{eqn:cross:461}
\Rightarrow
\Bu^{1} \wedge \dotsb \wedge \Bu^{N} =
\left(\dotprod{\left(\Bu^{1} \wedge \dotsb \wedge \Bu^{N-1}\right)}{\Bu^N} \right)
\left(\ecap_{1} \wedge \dotsb \wedge \ecap_{N}\right)
\end{equation}

This result is very much like the \R{3} triple product:

\begin{equation}\label{eqn:cross:481}
\tripleprod{\Bu}{\Bv}{\Bw} = \DETuvwijk{u}{v}{w}{1}{2}{3}
\end{equation}

This should not be surprising since \eqnref{eqn:cross:Nminus1Product} was the \R{3} cross product.



Explicitly expanding this \R{N} wedge of \(N\) vectors for a couple of dimensions to illustrate.

For \R{2}:
\begin{equation}\label{eqn:cross:501}
\Bu \wedge \Bv = \DETuvij{u}{v}{1}{2} (\ecap_1 \wedge \ecap_2)
\end{equation}
(Here the determinant is the oriented area of the \R{2} parallelogram formed by vectors \(\Bu\) and \(\Bv\).)

For \R{3}:
\begin{equation}\label{eqn:cross:521}
\Bu \wedge \Bv \wedge \Bw = \DETuvwijk{u}{v}{w}{1}{2}{3} (\ecap_1 \wedge \ecap_2 \wedge \ecap_3)
\end{equation}
(Here the determinant is the oriented volume of the \R{3} parallelepiped formed by vectors \(\Bu\), \(\Bv\), and \(\Bw\).)

And for \R{4}:
\begin{equation}\label{eqn:cross:541}
\Bu \wedge \Bv \wedge \Bw \wedge \Bx = \DETuvwxijkl{u}{v}{w}{x}{1}{2}{3}{4} (\ecap_1 \wedge \ecap_2 \wedge \ecap_3 \wedge \ecap_4)
\end{equation}
(Here the determinant is the oriented \(4\Volume\) of the \R{4} parallelo-4gram formed by vectors \(\Bu\), \(\Bv\), \(\Bw\), and \(\Bx\))

Each of these are one dimensional wedge product space vectors can be thought of as directed 
areas, volumes, or N-volumes.  There are two ways that the sign can vary, one is due to the ordering of the vectors themselves, and the 
other is due to the ordering of the \(\ecap_{i_{1}} \wedge \dotsb \wedge \ecap_{i_{N}}\) term. Picking the ordering of that
term is equivalent to picking a basis for the wedge product space.  Once that basis is picked it defines an isomorphism with \R{1}.


What this does do is put the wedge product into a context that we are used to.  

We can identify the \(N-1\) vector wedge product with the cross product at least with respect to its normal properties.

We can identify the wedge product of \(N\) vectors with the triple product (which is better described as an \(N\) product, valid for \(N \geq 2\)).
This \(N\) product is a scalar, and in particular has 
a geometric meaning which is the area, volume, ... of the parallelo-Ngram formed by the span of the vectors in question.

\subsection{Comparing the wedge product and the general normal to some vectors}

In general can we put the normal equations in a form closer to that of the wedge product?

\begin{equation}\label{eqn:cross:2141}
\begin{aligned}
\Proj_{\perp\ucap,\vcap}(\Bw)
\left({\sum_{i<j} \left(D_{ij}^{\Bu \Bv}\right)^2}\right)^2
&=
\sum_{i<j<k} \DETuvwijk{u}{v}{w}{i}{j}{k} \DETuvwijk{u}{v}{\ecap}{i}{j}{k} \\
&=
\sum_{i<j<k} \DETuvwijk{u}{v}{w}{i}{j}{k}
\left(\ecap_i \DETuvij{u}{v}{j}{k}
+\ecap_j \DETuvij{u}{v}{k}{i}
+\ecap_k \DETuvij{u}{v}{i}{j} \right) \\
&=
\sum_{t<i<j} \DETuvwijk{u}{v}{w}{t}{i}{j}
\ecap_t \DETuvij{u}{v}{i}{j} \\
&-
\sum_{i<t<j} \DETuvwijk{u}{v}{w}{i}{t}{j}
\ecap_t \DETuvij{u}{v}{i}{j} \\
&+
\sum_{i<j<t} \DETuvwijk{u}{v}{w}{i}{j}{t}
\ecap_t \DETuvij{u}{v}{i}{j} \\
&=
\sum_{i<j} \left( \sum_t \DETuvwijk{u}{v}{w}{t}{i}{j} \ecap_t\right) \DETuvij{u}{v}{i}{j} \\
\end{aligned}
\end{equation}

The same can be done for the general normal to two vectors,

\begin{equation}\label{eqn:cross:2161}
\begin{aligned}
\Bn(\Bu,\Bv)
&= \sum_{i<j<k} \left( \sum_{\pi_x \in \pi(i,j,k)} s_{\pi_x}\Sgn(\pi_x) \right)
\left(\ecap_i \DETuvij{u}{v}{j}{k}
+\ecap_j \DETuvij{u}{v}{k}{i}
+\ecap_k \DETuvij{u}{v}{i}{j} \right) \\
&=
\sum_{t<i<j}
\left( \sum_{\pi_x \in \pi(t,i,j)} s_{\pi_x}\Sgn(\pi_x) \right)
\ecap_t \DETuvij{u}{v}{i}{j} \\
&- \sum_{i<t<j}
\left( \sum_{\pi_x \in \pi(i,t,j)} s_{\pi_x}\Sgn(\pi_x) \right)
\ecap_t \DETuvij{u}{v}{i}{j} \\
&+\sum_{i<j<t}
\left( \sum_{\pi_x \in \pi(i,j,t)} s_{\pi_x}\Sgn(\pi_x) \right)
\ecap_t \DETuvij{u}{v}{i}{j} \\
&=
\sum_{i<j}
\left( \sum_{t=1}^{n} \ecap_t \sum_{\pi_x \in \pi(t,i,j)} s_{\pi_x}\Sgn(\pi_x) \right)
\DETuvij{u}{v}{i}{j} \\
\end{aligned}
\end{equation}

Let us expand the \(\Proj_{\perp\ucap,\vcap}(\Bw)\) result for a few specific examples \R{3}, \R{4} and \R{5} to get a feel for it.

Here is the \R{3} case:
\begin{equation}\label{eqn:cross:2181}
\begin{aligned}
\Bn_{(\Bu,\Bv)}(\Bw)
&=
\Proj_{\perp\ucap,\vcap}(\Bw)
\left({\sum_{ij=12,23,13} \left(D_{ij}^{\Bu \Bv}\right)^2}\right)^2 \\
&=
\sum_{ij=12,23,13}
\left(\sum_{t=1}^{3} \ecap_t \DETuvwijk{u}{v}{w}{t}{i}{j} \right)\DETuvij{u}{v}{i}{j} \\
&=
\sum_{ij=12,23,13}
\left(\sum_{t=1}^{3} \ecap_t D_{{t}{i}{j}}^{\Bu\Bv\Bw} \right)
\DETuvij{u}{v}{i}{j} \\
&=
\sum_{t=1}^{3} \ecap_t D_{{t}{1}{2}}^{\Bu\Bv\Bw} \DETuvij{u}{v}{1}{2}
+ \sum_{t=1}^{3} \ecap_t D_{{t}{2}{3}}^{\Bu\Bv\Bw} \DETuvij{u}{v}{2}{3}
+ \sum_{t=1}^{3} \ecap_t D_{{t}{1}{3}}^{\Bu\Bv\Bw} \DETuvij{u}{v}{1}{3} \\
&=
 \ecap_3 D_{{3}{1}{2}}^{\Bu\Bv\Bw} \DETuvij{u}{v}{1}{2}
+ \ecap_1 D_{{1}{2}{3}}^{\Bu\Bv\Bw} \DETuvij{u}{v}{2}{3}
+ \ecap_2 D_{{2}{1}{3}}^{\Bu\Bv\Bw} \DETuvij{u}{v}{1}{3} \\
&=
D_{{1}{2}{3}}^{\Bu\Bv\Bw} \left(
 \ecap_3 \DETuvij{u}{v}{1}{2}
+\ecap_1 \DETuvij{u}{v}{2}{3}
+\ecap_2 \DETuvij{u}{v}{3}{1}
\right) \\
&=
\DETuvwijk{u}{v}{w}{1}{2}{3} \left(
 \ecap_3 \DETuvij{u}{v}{1}{2}
+\ecap_1 \DETuvij{u}{v}{2}{3}
+\ecap_2 \DETuvij{u}{v}{3}{1}
\right)
\end{aligned}
\end{equation}

And here is the \R{4} case:
\begin{equation}\label{eqn:cross:2201}
\begin{aligned}
\Bn_{(\Bu,\Bv)}(\Bw)
&=
\Proj_{\perp\ucap,\vcap}(\Bw)
\left({\sum_{ij=12,13,14,23,24,34} \left(D_{ij}^{\Bu \Bv}\right)^2}\right)^2 \\
&=
\sum_{ij=12,13,14,23,24,34} \left( \sum_t \DETuvwijk{u}{v}{w}{t}{i}{j} \ecap_t\right) \DETuvij{u}{v}{i}{j} \\
&=
\sum_{t=1}^{4} \ecap_t D_{{t}{1}{2}}^{\Bu\Bv\Bw} \DETuvij{u}{v}{1}{2}
+
\sum_{t=1}^{4} \ecap_t D_{{t}{1}{3}}^{\Bu\Bv\Bw} \DETuvij{u}{v}{1}{3} \\
&+
\sum_{t=1}^{4} \ecap_t D_{{t}{1}{4}}^{\Bu\Bv\Bw} \DETuvij{u}{v}{1}{4}
+
\sum_{t=1}^{4} \ecap_t D_{{t}{2}{3}}^{\Bu\Bv\Bw} \DETuvij{u}{v}{2}{3} \\
&+
\sum_{t=1}^{4} \ecap_t D_{{t}{2}{4}}^{\Bu\Bv\Bw} \DETuvij{u}{v}{2}{4}
+
\sum_{t=1}^{4} \ecap_t D_{{t}{3}{4}}^{\Bu\Bv\Bw} \DETuvij{u}{v}{3}{4} \\
&=
\left(
\ecap_{3} D_{{3}{1}{2}}^{\Bu\Bv\Bw}
+\ecap_{4} D_{{4}{1}{2}}^{\Bu\Bv\Bw}
\right)
\DETuvij{u}{v}{1}{2}
+
\left(
\ecap_{2} D_{{2}{1}{3}}^{\Bu\Bv\Bw}
+\ecap_{4} D_{{4}{1}{3}}^{\Bu\Bv\Bw}
\right)
\DETuvij{u}{v}{1}{3} \\
&+
\left(
\ecap_{2} D_{{2}{1}{4}}^{\Bu\Bv\Bw}
+\ecap_{3} D_{{3}{1}{4}}^{\Bu\Bv\Bw}
\right)
\DETuvij{u}{v}{1}{4}
+
\left(
\ecap_{1} D_{{1}{2}{3}}^{\Bu\Bv\Bw}
+\ecap_{4} D_{{4}{2}{3}}^{\Bu\Bv\Bw}
\right)
\DETuvij{u}{v}{2}{3} \\
&+
\left(
\ecap_{1} D_{{1}{2}{4}}^{\Bu\Bv\Bw}
+\ecap_{3} D_{{3}{2}{4}}^{\Bu\Bv\Bw}
\right)
\DETuvij{u}{v}{2}{4}
+
\left(
\ecap_{1} D_{{1}{3}{4}}^{\Bu\Bv\Bw}
+\ecap_{2} D_{{2}{3}{4}}^{\Bu\Bv\Bw}
\right)
\DETuvij{u}{v}{3}{4} \\
\end{aligned}
\end{equation}

For \R{5} the set of indices \(\lbrace{ij}\rbrace = \lbrace{12,13,14,15,23,24,25,34,35,45}\rbrace\).  The
coefficients of the \(\DETuvij{u}{v}{i}{j}\) terms are \(\sum_{t \neq i,j} \ecap_t D_{tij}^{\Bu\Bv\Bw}\), so for
\R{5} this will be three terms.

So, we can equate the vector normal to the plane of \(\Bu\) and \(\Bv\) in the direction of \(\Bw\) if one introduces the following mapping:

\begin{equation}\label{eqn:cross:561}
\ecap_i \wedge \ecap_j = \frac{\sum_{t \neq i,j} \ecap_t D_{tij}^{\Bu\Bv\Bw}}
{\sum_{i<j} \left(D_{ij}^{\Bu\Bv}\right)^2}
\end{equation}

Comparing to the normal to the \(\Bu,\Bv\) plane in an unspecified direction:
\(\Bn = \sum s_{ijk}D_{ijk}^{\Bu\Bv\Be}\)
we can treat \(\Bu \wedge \Bv\) as a normal if we write:

\begin{equation}\label{eqn:cross:581}
\ecap_i \wedge \ecap_j =
\left( \sum_{t \neq i \neq j} \ecap_t \sum_{\pi_x \in \pi(t,i,j)} s_{\pi_x}\Sgn(\pi_x) \right)
\end{equation}

Similarly we can think of
\(\Bu \wedge \Bv \wedge \Bw\) as a normal to the \(\Bu,\Bv,\Bw\) volume, \(\Bn = \sum{s_{ijkl}D_{ijkl}^{\Bu\Bv\Bw\Be}}\),  if we write:

\begin{equation}\label{eqn:cross:601}
\ecap_i \wedge \ecap_j \wedge \ecap_k =
\left( \sum_{t \neq i \neq j \neq k} \ecap_t \sum_{\pi_x \in \pi(t,i,j,k)} s_{\pi_x}\Sgn(\pi_x) \right)
\end{equation}

%  Example for \R{4}
%
%\begin{align*}
%\Bn(\Bu,\Bv,\Bw) =
%\left( \sum_{t \neq i \neq j \neq k} \sum_{\pi_x \in \pi(t,i,j,k)} s_{\pi_x}\Sgn(\pi_x) \right)
%\left( \sum_{i<j<k<l} \ecap_
%\sum_{\pi_x \in \pi(t,i,j,k)} s_{\pi_x}\Sgn(\pi_x) \right)
%\end{align*}

This is not exactly a natural seeming correspondence, unlike the \(N-1\) vector case.  There it did make some sense to treat
the wedge product as a normal form, but only in the \(N-1\) vector case is there an unambiguous normal form (apart from the scalar multiplier).

\subsection{A first attempt to tie in to differential forms.  Dot product of wedge products as normal times magnitude}

There is a natural scenario where it does make sense to equate the normal and the wedge.

By example, if one has a surface function with components f, and g in the \(x_1,x_2\) and \(x_2,x_3\) planes respectively, one could in \R{3} express this as:

\begin{equation}\label{eqn:cross:621}
\Bv(\Br) = f(\Br) \ecap_3 + g(\Br) \ecap_1
\end{equation}

As illustrated by the generic normal calculations, in \R{4}, 
the normal to the \(x_1,x_2\) plane does not have a unique direction and potentially has components in one or more of the \(\ecap_3\) and \(\ecap_4\) directions, and
the normal to the \(x_2,x_3\) plane potentially has \(\ecap_1\) and \(\ecap_4\) components.

So, it is much more natural to not try to express a directed surface function in terms of a normal that is not well defined unless the ``surface'' is of dimension \(N-1\).  Instead such a directed surface function is best expressed exclusively in terms of its components in each of its planes.  Restating the above in such terms we have:

\begin{equation}\label{eqn:cross:641}
\Bv(\Br) = f(\Br) \ecap_{1,2} + g(\Br) \ecap_{2,3}
\end{equation}

Here \(\ecap_{1,2}\) indicates that \(f(\Br)\) is the component of the surface function \(\Bv\) that is in the \(x_1,x_2\) plane and \(\ecap_{2,3}\) means that \(g(\Br)\) is the component of \(\Bv\) in the \(x_2,x_3\) plane.

It is natural to use the wedge product to express each of these components, writing:

\begin{equation}\label{eqn:cross:661}
\Bv(\Br) = f(\Br) (\ecap_1 \wedge \ecap_2) + g(\Br) (\ecap_2 \wedge \ecap_3)
\end{equation}

A formulation like this is well regardless of whether the space is \R{3}, \R{4}, \R{5}, or \R{N}, though it does not have any specific physical interpretation.

More generally, a surface function in \R{N} can be expressed as
\begin{equation}\label{eqn:cross:681}
\Bv(\Br) = \sum_{i \neq j} f_{ij}(\Br) \ecap_{i,j}
\end{equation}

and write this as:

\begin{equation}\label{eqn:cross:701}
\Bv(\Br) = \sum_{i<j} f_{ij}(\Br) \ecap_i \wedge \ecap_j
\end{equation}

This and the fact that we can express length, area, volume, n-volumes, as wedge products thus gives us a way to describe flux and work like quantities.  

Recall that a line integral of the following form describes phenomena such as ``work done''.

\begin{equation}\label{eqn:cross:702}
\int_{\Br} \dotprod{\Bv(\Br)}{d\Br} 
\end{equation}

This is the component of a directed field and multiply by the length of a vector in the specified direction.

Similarly a surface integral of the following form describes ``flux through a surface'' like quantities.  We write this as:

\begin{equation}\label{eqn:cross:721}
\int_{\BA} \dotprod{\Bv(\Br)}{d\BA} 
\end{equation}

This is the component of a field in the direction of a surface, is multiplied by the area element for that surface.

Using the example above with components f, and g in the \(x_1,x_2\) and \(x_2,x_3\) planes respectively, one could in \R{3} express the flux for this function as:
 
\begin{equation}\label{eqn:cross:741}
\int_{\Br} \dotprod{\left(f(\Br) \ecap_3 + g(\Br) \ecap_1\right)}{\left(dx_1 dx_2 \ecap_3 + dx_2 dx_3 \ecap_1 + dx_3 dx_1 \ecap_2\right)}
\end{equation}

But again, this is a formulation that is only good for \R{3}.  Also note that there is an implied common orientation
of the normal and the area elements.

Because we can express the component of a surface function as a wedge product and can also express an oriented area element as a wedge product we can 
express this flux quantity in terms of the wedge product and have a formulation that is valid for \R{N} as well as \R{3}.

\begin{equation}\label{eqn:cross:761}
\int
\dotprod{\left(f(\Br) (\ecap_1 \wedge \ecap_3) + g(\Br) (\ecap_2 \wedge \ecap_3)\right)}{\left(\sum_{i<j} d\ecap_i \wedge d\ecap_j\right)}
=
\int f(\Br) dx_1 dx_3 + g(\Br) dx_2 dx_3
\end{equation}

Similarly, say one has a volume function in \R{4} with components f, and g with \(x_1,x_2,x_3\) and \(x_1,x_2,x_4\) cubes, one could express it as:

\begin{equation}\label{eqn:cross:781}
\Bv(\Br) = f(\Br) \ecap_4 + g(\Br) \ecap_3
\end{equation}

But this is only a good description for \R{4}.  If one expressed the same thing as:

\begin{equation}\label{eqn:cross:801}
\Bv(\Br) = f(\Br) (\ecap_1 \wedge \ecap_2 \wedge \ecap_3) + g(\Br) (\ecap_1 \wedge \ecap_2 \wedge \ecap_4)
\end{equation}

This is a good description of the directed volume function for \R{4} as well as any other dimension \R{N}, and one could write a \R{N} volume ``flux'' like integral
for this function as:

\begin{equation}\label{eqn:cross:821}
\int
\dotprod{\left(
f(\Br) (\ecap_1 \wedge \ecap_2 \wedge \ecap_3) + g(\Br) (\ecap_1 \wedge \ecap_2 \wedge \ecap_4)
\right)}
{\left(\sum_{i<j<k} d\ecap_i \wedge d\ecap_j \wedge d\ecap_k\right)}
\end{equation}
\begin{equation}\label{eqn:cross:841}
=
\int f(\Br) dx_1 dx_2 dx_3 + g(\Br) dx_1 dx_2 dx_4
\end{equation}

Here like the surface flux formulation, the volume function has a specific orientation.  That orientation has been defined by 
considering its \(i,j,k\) element as positive in the ``direction'' of \(\ecap_i \wedge \ecap_j \wedge \ecap_k\).

Now, it may not look like much has been gained by introducing the wedge product, but when it comes to parameterizing the surface-integral, or volume-integral 
this allows for a great deal of flexibility.  For example, if the piece of the surface element can be parametrized as a parallelogram in terms of two vectors \(d\Bu\) and \(d\Bv\), then that element of the flux integral above can be expressed as

\begin{equation}\label{eqn:cross:861}
\int
\dotprod{\left(f(\Br) (\ecap_1 \wedge \ecap_3) + g(\Br) (\ecap_2 \wedge \ecap_3)\right)}{\left(d\Bu \wedge d\Bv\right)}
\end{equation}

and using the volume flux example above, if a parallelepiped volume element is parametrized in terms of three vectors 
\(d\Bu\), \(d\Bv\) and \(d\Bw\), then the flux element can be expressed as

\begin{equation}\label{eqn:cross:881}
\int
\dotprod{\left(
f(\Br) (\ecap_1 \wedge \ecap_2 \wedge \ecap_3) + g(\Br) (\ecap_1 \wedge \ecap_2 \wedge \ecap_4)
\right)}{\left(d\Bu \wedge d\Bv \wedge d\Bw\right)}
\end{equation}

%\end{document}               % End of document.

%
% Copyright � 2012 Peeter Joot.  All Rights Reserved.
% Licenced as described in the file LICENSE under the root directory of this GIT repository.
%

%
%
% forked from here everything after 1.15 version of cross.ltx
% this is the original bits.

%\documentclass{article}      % Specifies the document class

%\usepackage{amsmath}

%
% The real thing:
%

\chapter{Early cross product generalization and motivation attempt}
\label{chap:crossOld}
%\author{Peeter Joot}         % Declares the author's name.
\date{ May 2000.  crossOld.tex }

%\begin{document}             % End of preamble and beginning of text.

%\maketitle{}

\section{Introduction/Abstract}

The cross product is an ugly arbitrary seeming sort of beast, but it is a beast that
describes many sorts of physical and mathematical situations.  In vector calculus
cross product terms and it relative the determinant end up occurring all over the place,
and in physics the cross product also occurs in many contexts.
Examples are Stokes theorem, Jacobian transformations, normal equations, the
curl operator, Maxwell's equations, torque, and the list goes on.
In many of
these cases the mathematics has no logical tie to three dimensions, yet
the cross product is an explicitly three dimensional sort of beast.
The cross product and the dot product have some similarities in form
yet the cross product is only defined for \R{3}, while the dot product
can be defined for \R{n} including \(n < 3\), and even extended easily to \C{n}.
The open question remains of how to generalize it and the math that
is related to it to higher dimensions and other mathematical fields.

\section{The cross product in physical situations}

On of the common places where the cross product appears naturally is in the
definition of torque.
The basic definition of torque as a scalar quantity is the product of the radial distance times
the perpendicular force.  The formula in terms of components in three dimensions given a force vector
\(\BF = (F_x, F_y, F_z)\) and the
radial distance \(\Br = (x, y, z)\) is pretty messy, which is the reason it
is typically described by means of a cross product, and
a generalized torque ``vector'' with a magnitude and direction.

%My Feynman book gives a derivation of for the formula for torque in one dimension as
%the differential work per unit rotation.  This derivation is interesting
%because it yields in a simple fashion a torque formula without having to
%introduce the complexities of the cross product or the torque pseudo-vector.  I will
%not reproduce it here, but will go through a generalized derivation for the torque
%equation when the plane of rotation has an arbitrary orientation in space, rather
%than being restricted to the x,y plane (or y,z or z,x).

The torque expression can be seen to be a natural result of the examination
of the differential work per unit rotation.
\footnote{The Feynman lectures, where a one dimension derivation
of torque is given in this fashion for the (x,y) plane.}
A derivation of this torque expression for an arbitrary rotation
in space will be given in the following sections, first in two dimensions then in three.
The expression for angular velocity for a rotational motion will also be derived.  In each of
these physical scenarios it will be seen that the expression for the cross product arises.
These physical preliminaries will
lead to a technique for which a possible higher dimensional cross product can
be formed and also show how a cross product operator can be defined in a convenient
and natural matrix formulation.

%To start things off, some basic vector algebra results will be presented.

\subsection{torque in two dimensions}

%Feynman's derivation of torque in two dimensions was geometrical and also was quite simple.
In
modern physics where torque is a vector in three dimensions
%even if the rotation is constrained to two dimensions
it does not make sense to talk of a two dimensional torque, but
%There is no reason
the magnitude of the torque for a rotation confined to a plane can be defined without
reference to the plane's normal (ie: the third dimension).
%in two dimensions can not be determined even if the
%the third dimension or rotational axis has not been defined.
This
%, strictly speaking,
is what is meant by torque
in this section.
Application of transformations to and from a rotated frame will be used to define an expression for torque in
\R{2}.  This approach can be applied to do the same in \R{3}, yielding a natural occurrence of the mathematical
form known as the cross product.  The cross product is typically first introduced from its projective
definition, but this form does not easily lead to generalizations in higher dimensions.  Using this procedure
the cross product will be shown to be an expression of incremental rotation, and an \R{n} cross product will
be defined by examination of what we will call an \R{n} rotation.

%arrive at the same result, but have the additional benefit of indicating an approach for the same
%problem for \R{3} and paves the way for generalizing the cross product for \R{n}.

If a rotation inducing force \(\BF\) is applied to an object in space with position \(\Br\) then the only component of the
force that will do work is the component perpendicular to the direction \(\Br\).  Let a new coordinate system
with unit vectors \(\{\rcap = \Br/\norm{\Br}\), \(\thetacap\}\) be defined where \(\thetacap\) is the unit vector perpendicular to \(\Br\) in the direction of positive angular increase.  In this coordinate system
for the force \(\BF' = (F_r, F_\theta)\), only the \(F_\theta\)
component does any work.

To transform to the \(r,\theta\) basis, it can be noted that \(\thetacap \propto (-y, x)\).  Thus
\(\rcap = \inv{r}(x,y)\), \(\thetacap = \inv{r}(-y,x)\), and
\footnote{see appendix for a refresh on change of basis calculations and for the \(\BM\) notation used here}
$\BM =
\inv{r}
\Bigl[
\begin{smallmatrix}
 x & y \\
-y & x
\end{smallmatrix}
\Bigr]
$

The work done \(dW\) is
\begin{equation}\label{eqn:crossOld:20}
\begin{aligned}
dW &=\dotprod{\BF}{d\Bl} \\
   &=\dotprod{\BF'} r d\thetacap \\
   &= F_\theta r d\theta
\end{aligned}
\end{equation}

and
\begin{equation}\label{eqn:crossOld:40}
\begin{aligned}
\BF' &= \BM \BF \\
     &=
\inv{r}
\begin{bmatrix}
 x & y \\
-y & x
\end{bmatrix}
\begin{bmatrix}
 F_x \\
 F_y
\end{bmatrix} \\
     &=
\inv{r}
\begin{bmatrix}
 x Fx - y F_y \\
-y Fx + x F_y
\end{bmatrix}
\end{aligned}
\end{equation}

so
\begin{equation*}
dW = (x F_y - y F_x) d\theta
\end{equation*}

What is being called the torque
\(\tau\)
is this scalar quantity
\(\tau = \D{\theta}{W} = x F_y - y F_x\),
the work per unit rotation for a force \(\BF = (F_x, F_y)\)
applied at a point \(\Br = (x, y)\) from the origin about which the rotation occurs.

It is also easily noted that the transformation
\begin{equation*}
\BM =
\inv{r}
\begin{bmatrix}
 x & y \\
-y & x
\end{bmatrix}
=
\begin{bmatrix}
 x/r & y/r \\
-y/r & x/r
\end{bmatrix}
=
\begin{bmatrix}
 \cos\theta & \sin\theta \\
-\sin\theta & \cos\theta
\end{bmatrix}
=-\BR_\theta
={\BR_\theta}^T
={\BR_\theta}^{-1}
\end{equation*}

Where
\(\BR_\theta\) is the transformation matrix for a rotation through an angle \(\theta\).

The torque can also be calculated in an alternate fashion by using the
rotation matrix.
For a rotation
through a small angle \(d\theta\) this transformation becomes,

\begin{equation*}
\BR_{d\theta} =
\begin{bmatrix}
 \cos{d\theta} & -\sin{d\theta} \\
 \sin{d\theta} & \cos{d\theta}
\end{bmatrix}
=
\begin{bmatrix}
 1 & -d\theta \\
 d\theta & 1
\end{bmatrix}
\end{equation*}

and so the displaced vector is
\begin{equation*}
\Br' = \BR_{d\theta} \Br =
\Bigr( \BI +
\begin{bmatrix}
 0 & -d\theta \\
 d\theta & 0
\end{bmatrix}
\Bigl) \Br
\end{equation*}

which gives the differential change in position
\begin{equation*}
d\Br = \Br' - \Br =
(\BR_{d\theta}-\BI) \Br
=
\begin{bmatrix}
 0 & -d\theta \\
 d\theta & 0
\end{bmatrix}
\Br
=
\begin{bmatrix}
 0 & -1 \\
 1 & 0
\end{bmatrix}
\Br d\theta
\end{equation*}

and the work done is

\begin{equation}\label{eqn:crossOld:60}
\begin{aligned}
dW &=\dotprod{\BF}{d\Br} \\
   &=
\begin{bmatrix}
F_x & F_y
\end{bmatrix}
\begin{bmatrix}
 0 & -1 \\
 1 & 0
\end{bmatrix}
\Br d\theta \\
   &=
\begin{bmatrix}
F_y & -F_x
\end{bmatrix}
\begin{bmatrix}
x \\
y
\end{bmatrix}
d\theta \\
   &=
(x F_y - y F_x) d\theta
\end{aligned}
\end{equation}

This same technique can be applied in three and more dimensions, which will be done in
the following sections.

\subsection{torque in three dimensions}
For three dimensions successive rotations in the \(xy, yz\) and \(zx\) planes can be applied

\begin{equation}\label{eqn:crossOld:80}
\begin{aligned}
\BR_{d\theta_{xy}}
&=
\BR_{d\theta_z}
&=
\begin{bmatrix}
1 & -d\theta_z & 0 \\
 d\theta_z & 1 & 0 \\
0 & 0 & 1
\end{bmatrix} \\
\BR_{d\theta_{yz}}
&=
\BR_{d\theta_x}
&=
\begin{bmatrix}
 1 & 0 & 0 \\
 0 & 1 & -d\theta_x \\
 0 & d\theta_x & 1
\end{bmatrix} \\
\BR_{d\theta_{zx}}
&=
\BR_{d\theta_y}
&=
\begin{bmatrix}
 1 & 0 & d\theta_y \\
 0 & 1 & 0 \\
 -d\theta_y & 0 & 1
\end{bmatrix} \\
\end{aligned}
\end{equation}

Applying these transformations in sequence is a bit messy, but certainly easier than
applying three successive large rotations in sequence.  The mess of sine and cosine terms
for that is horrendous if you care to try!

The calculation for the sequential application of
\(\BR_{d\theta_{xy}}\)
,
\(\BR_{d\theta_{yz}}\)
and
\(\BR_{d\theta_{zx}}\)
is below.

\begin{multline*}
\BR_{d\theta_{zx}}
\BR_{d\theta_{yz}}
\BR_{d\theta_{xy}}
%
=
%
\BR_{d\theta_y}
\BR_{d\theta_x}
\BR_{d\theta_z} \\
%
=
%
% \BR_\theta_y
\begin{bmatrix}
 1 & 0 & d\theta_y \\
 0 & 1 & 0 \\
 -d\theta_y & 0 & 1
\end{bmatrix}
% \BR_{d\theta_x}
\begin{bmatrix}
 1 & 0 & 0 \\
 0 & 1 & -d\theta_x \\
 0 & d\theta_x & 1
\end{bmatrix}
% \BR_{d\theta_z}
\begin{bmatrix}
1 & -d\theta_z & 0 \\
d\theta_z & 1 & 0 \\
0 & 0 & 1
\end{bmatrix} \\
%
=
%
\begin{bmatrix}
1 & d\theta_x\,d\theta_y & d\theta_y \\
0 & 1 & -d\theta_x \\
-d\theta_y & 0 & 1
\end{bmatrix}
% \BR_{d\theta_z}
\begin{bmatrix}
1 & -d\theta_z & 0 \\
d\theta_z & 1 & 0 \\
0 & 0 & 1
\end{bmatrix} \\
%
=
%
\begin{bmatrix}
1+d\theta_x\,d\theta_y\,d\theta_z & -d\theta_z + d\theta_x\,d\theta_y & d\theta_y \\
d\theta_z & 1 & -d\theta_x \\
-d\theta_y + d\theta_x\,d\theta_z & d\theta_y\,d\theta_z d\theta_x & 1
\end{bmatrix} \\
%
=
%
\BI
+
\begin{bmatrix}
0 & -d\theta_z & d\theta_y \\
d\theta_z & 0 & -d\theta_x \\
-d\theta_y & d\theta_x & 0
\end{bmatrix}
+
\begin{bmatrix}
0 & d\theta_x\,d\theta_y & 0 \\
0 & 0 & 0 \\
d\theta_x\,d\theta_z & d\theta_y\,d\theta_z & 0
\end{bmatrix} \\
+
\begin{bmatrix}
d\theta_x\,d\theta_y\,d\theta_z & 0 & 0 \\
0 & 0 & 0 \\
0 & 0 & 0
\end{bmatrix} \\
\end{multline*}

Note that if the second and third order terms are neglected then
\begin{equation*}
\BR_{d\theta_{zx}}
\BR_{d\theta_{yz}}
\BR_{d\theta_{xy}} - \BI
%
=
%
\BR_{d\theta_y}
\BR_{d\theta_x}
\BR_{d\theta_z} - \BI
%
\approx
%
\begin{bmatrix}
0 & -d\theta_z & d\theta_y \\
d\theta_z & 0 & -d\theta_x \\
-d\theta_y & d\theta_x & 0
\end{bmatrix}
\end{equation*}

and that

\begin{multline*}
\begin{bmatrix}
0 & -d\theta_z & d\theta_y \\
d\theta_z & 0 & -d\theta_x \\
-d\theta_y & d\theta_x & 0
\end{bmatrix}
= \\
% \BR_\theta_y
\begin{bmatrix}
 0 & 0 & d\theta_y \\
 0 & 0 & 0 \\
-d\theta_y & 0 & 0
\end{bmatrix}
+
% \BR_{d\theta_x}
\begin{bmatrix}
 0 & 0 & 0 \\
 0 & 0 & -d\theta_x \\
 0 & d\theta_x & 0
\end{bmatrix}
% \BR_{d\theta_z}
+
\begin{bmatrix}
0 & -d\theta_z & 0 \\
d\theta_z & 0 & 0 \\
0 & 0 & 0
\end{bmatrix}
\end{multline*}

The following result

\begin{multline*}
\BR_{d\theta_{zx}}
\BR_{d\theta_{yz}}
\BR_{d\theta_{xy}}
%
=
%
\BR_{d\theta_y}
\BR_{d\theta_x}
\BR_{d\theta_z} \\
\approx
\BI +
(\BR_{d\theta_y} - \BI)
+
(\BR_{d\theta_x} - \BI)
+
(\BR_{d\theta_z} - \BI)
=
\BR_{d\theta_{xyz}}
\end{multline*}
is independent of the order of application of the rotations, which is not true for the case
when the rotations are not infinitesimal.

Using this a differential change in \(\Br\) due to the rotation is

\begin{equation}\label{eqn:crossOld:100}
\begin{aligned}
d\Br = \Br' - \Br =
(\BR_{d\theta_{xyx}}-\BI) \Br
&=
\begin{bmatrix}
0 & -d\theta_z & d\theta_y \\
d\theta_z & 0 & -d\theta_x \\
-d\theta_y & d\theta_x & 0
\end{bmatrix} \Br \\
&=
-
\begin{bmatrix}
0 & -z & y \\
z & 0 & -x \\
-y & x & 0
\end{bmatrix}
\begin{bmatrix}
d\theta_x \\
d\theta_y \\
d\theta_z
\end{bmatrix}
\end{aligned}
\end{equation}

This vector of \(d\theta_i\) components can be written
\begin{equation*}
d\Btheta =
\begin{bmatrix}
d\theta_x \\
d\theta_y \\
d\theta_z
\end{bmatrix}
\end{equation*}

In the same fashion as in the two dimensional case above,
this can be applied
to
calculate the work done,

\begin{equation}\label{eqn:crossOld:120}
\begin{aligned}
dW &=\dotprod{\BF}{d\Br} \\
   &=
\begin{bmatrix}
F_x & F_y & F_z
\end{bmatrix}
\Biggl(
   -
\begin{bmatrix}
0 & -z & y \\
z & 0 & -x \\
-y & x & 0
\end{bmatrix} d\Btheta
\Biggr)
\\
 &=
-
{\Biggl(
{\begin{bmatrix}
0 & -z & y \\
z & 0 & -x \\
-y & x & 0
\end{bmatrix}}^T
\begin{bmatrix}
F_x \\
F_y \\
F_z
\end{bmatrix}
\Biggr)}^T
d\Btheta
\\
 &=
 \Biggl(
\begin{bmatrix}
0 & -z & y \\
z & 0 & -x \\
-y & x & 0
\end{bmatrix}
\BF
\Biggr) \cdot d\Btheta
\end{aligned}
\end{equation}

So in the three dimensional case we can write
\begin{equation*}
dW =\dotprod{\BF}{d\Br} =\dotprod{\Btau}{d\Btheta}
\end{equation*}

Where in analogy with the two dimensional case
\(dW =\dotprod{\BF}{d\Br} = \tau d\theta\) we can define the torque as this
quantity \(\Btau\),

\begin{equation*}
\Btau =
\begin{bmatrix}
0 & -z & y \\
z & 0 & -x \\
-y & x & 0
\end{bmatrix}
\BF
=
\begin{bmatrix}
y F_z - z F_y \\
z F_x - x F_z \\
x F_y - y F_x
\end{bmatrix}
=
\crossprod{\Br}{\BF}
\end{equation*}

the work per unit ``angle'' of rotation in space.

\subsection{angular velocity in three dimensions}

One of the formulas that I recall was always just presented and never derived (for the three dimensional case)
was that for
\(\D{t}{\Br}\) in terms of a vector angular velocity.  This is another equation that the cross product
comes up, and an equation whose derivation is easily done given some of the work above.

Let \(\Bomega = \D{t}{}\Bigl(\theta_x, \theta_y, \theta_z\Bigr)\) then in the limit

\begin{equation}\label{eqn:crossOld:140}
\begin{aligned}
\D{t}{\Br} &= \frac{\Br(t + dt) - \Br(t)}{dt} \\
%           &= \frac{\Br' - \Br}{dt} \\
           &= \frac{\BR_{d\theta_{xyx}}\Br -\Br}{dt} \\
           &= \frac{(\BR_{d\theta_{xyx}}-\BI) \Br}{dt} \\
	   &= \inv{dt}
\begin{bmatrix}
0 & -d\theta_z & d\theta_y \\
d\theta_z & 0 & -d\theta_x \\
-d\theta_y & d\theta_x & 0
\end{bmatrix} \Br \\
           &=
\begin{bmatrix}
\omega_y z -\omega_z y \\
\omega_z x -\omega_x z \\
\omega_x y -\omega_y x
\end{bmatrix}
\end{aligned}
\end{equation}

Thus
\begin{equation*}
\Bv = \crossprod{\Bomega}{\Br}
\end{equation*}

Note that these were not derivations of the cross product, but just
physical situations in which the cross product occurs.

\section{generalizing the cross product}

\subsection{defining a cross product operator}

In each of the physical situations where the cross product occurs
above (ie: torque and angular velocity) the derivation of
these formulas was closely tied to the
differential change in position \(d\Br\) after application of a rotation
transformation \(\BR_{d\theta_{xyz}}\),
%limiting expression for
%a three dimensional differential rotation
%\(\BR_{d\theta_{xyz}}\)
%applied to a vector \(\Br\)

\begin{equation*}
d\Br =
(\BR_{d\theta_{xyz}} - \BI) \Br =
\begin{bmatrix}
0 & -d\theta_z & d\theta_y \\
d\theta_z & 0 & -d\theta_x \\
-d\theta_y & d\theta_x & 0
\end{bmatrix} \Br
=
%-
\begin{bmatrix}
0 & -z & y \\
z & 0 & -x \\
-y & x & 0
\end{bmatrix} d\Btheta
\end{equation*}

As a side effect of the derivations above the cross
product shows up
tied
to a
matrix form since the rotation transformation was defined in matrix form.
Expressing the cross product in this matrix form has a certain aesthetic pleasantness
that the
component form lacks.  It can also be seen that
as an operational quantity
the matrices above,
whether they contain the \(d\theta_i\) or the \(r_i\) terms,
are the important part of the equation, and we can drop the vector multiplication
part of the expression.

A cross product operator \(\Bu \cross\) can be defined for and vector \(\Bu\)
in \R{3},

\begin{equation*}
\crossop{\Bu}=
\begin{bmatrix}
0 & -u_z & u_y \\
u_z & 0 & -u_x \\
-u_y & u_x & 0
\end{bmatrix}
\end{equation*}

When applied to a vector \(\Bv\)
\begin{equation*}
(\Bu \cross) \Bv =
\begin{bmatrix}
0 & -u_z & u_y \\
u_z & 0 & -u_x \\
-u_y & u_x & 0
\end{bmatrix}
\begin{bmatrix}
v_x \\
v_y \\
v_z
\end{bmatrix}
= \crossprod{\Bu}{\Bv}
\end{equation*}

which is what we typically define as \(\cross{\Bu}{\Bv}\).  Using this new notation the change
in position can be written,

\begin{equation*}
d\Br
=  (\crossprod{d\Btheta}{})\Br
= -(\crossprod{\Br}{})d\Btheta
\end{equation*}

By defining the cross product in this fashion, the rotational and physical origins have been discarded,
%As an operational entity there is no reason to refer
%to the differential rotation vector \(d\Btheta = (d\theta_x, d\theta_y, d\theta_z)\) anymore,
but it is interesting to note the way that the cross product
and a rotation in space are related in a fundamental way.

\subsection{decomposition of the cross product operator}

The cross product operator as defined above can be antisymetrically
decomposed into its positive and negative portions as follows
\begin{equation*}
\crossop{\Bu}=
\begin{bmatrix}
0 & -u_z & u_y \\
u_z & 0 & -u_x \\
-u_y & u_x & 0
\end{bmatrix}
=
\begin{bmatrix}
0 & 0 & u_y \\
u_z & 0 & 0 \\
0 & u_x & 0
\end{bmatrix}
-
\begin{bmatrix}
0 & 0 & u_y \\
u_z & 0 & 0 \\
0 & u_x & 0
\end{bmatrix}^T
\end{equation*}

Each half of the right hand side can be diagonalized, but not via a change of basis
\footnote
{
I was playing around with this at one point
and I believe it was possible to diagonalized the
matrix, but not in a simple fashion, as it required complex eigenvalues and
eigenvectors, and the solution of a cubic equation.
}
, as follows:
\begin{multline*}
\crossop{\Bu}=
\begin{bmatrix}
0 & 1 & 0 \\
0 & 0 & 1 \\
1 & 0 & 0
\end{bmatrix}
\begin{bmatrix}
u_x & 0 & 0 \\
0 & u_y & 0 \\
0 & 0 & u_z
\end{bmatrix}
\begin{bmatrix}
0 & 1 & 0 \\
0 & 0 & 1 \\
1 & 0 & 0
\end{bmatrix} \\
-
\begin{bmatrix}
0 & 0 & 1 \\
1 & 0 & 0 \\
0 & 1 & 0
\end{bmatrix}
\begin{bmatrix}
u_x & 0 & 0 \\
0 & u_y & 0 \\
0 & 0 & u_z
\end{bmatrix}
\begin{bmatrix}
0 & 0 & 1 \\
1 & 0 & 0 \\
0 & 1 & 0
\end{bmatrix}
\end{multline*}

The matrix
\( \BP =
\Bigl[
\begin{smallmatrix}
0 & 1 & 0 \\
0 & 0 & 1 \\
1 & 0 & 0
\end{smallmatrix}
\Bigr]
\) above is an interesting one, as \(\BP^{-1} = \BP^T = \BP^2\).

%From this it can be seen that one can write
%
%\begin{multline*}
%\begin{bmatrix}
%0 & 0 & 1 \\
%1 & 0 & 0 \\
%0 & 1 & 0
%\end{bmatrix}
%\begin{bmatrix}
%u_x & 0 & 0 \\
%0 & u_y & 0 \\
%0 & 0 & u_z
%%\end{bmatrix}
%\begin{bmatrix}
%0 & 0 & 1 \\
%1 & 0 & 0 \\
%0 & 1 & 0
%\end{bmatrix} \\
%=
%\begin{bmatrix}
%0 & 1 & 0 \\
%0 & 0 & 1 \\
%1 & 0 & 0
%\end{bmatrix}
%\Biggl(
%\begin{bmatrix}
%0 & 1 & 0 \\
%0 & 0 & 1 \\
%1 & 0 & 0
%\end{bmatrix}
%\begin{bmatrix}
%u_x & 0 & 0 \\
%0 & u_y & 0 \\
%0 & 0 & u_z
%\end{bmatrix}
%\begin{bmatrix}
%0 & 0 & 1 \\
%1 & 0 & 0 \\
%0 & 1 & 0
%\end{bmatrix}
%\Biggr) \\
%=
%\Biggl(
%\begin{bmatrix}
%0 & 0 & 1 \\
%1 & 0 & 0 \\
%%0 & 1 & 0
%\end{bmatrix}
%\begin{bmatrix}
%u_x & 0 & 0 \\
%0 & u_y & 0 \\
%0 & 0 & u_z
%\end{bmatrix}
%\begin{bmatrix}
%0 & 1 & 0 \\
%0 & 0 & 1 \\
%1 & 0 & 0
%\end{bmatrix}
%\Biggr)
%\begin{bmatrix}
%0 & 1 & 0 \\
%0 & 0 & 1 \\
%1 & 0 & 0
%\end{bmatrix}
%\end{multline*}
%
%where the terms inside the braces represent a change of basis transformation.

If one defines \(D(\Bu)\) as the matrix with the terms of \(\Bu\) are along the diagonal, then
there is now a nice and concise way of writing the cross product operator.

\begin{equation*}
\crossop{\Bu}= \BP D(\Bu) \BP - \BP^T D(\Bu) \BP^T
\end{equation*}

This is also
possibly suggestive of how to define the cross product in greater than three dimensions, for
it could possibly be of the same form where \(\BP\) is a permutation or some other transformation.

%An alternative form is also possible, by taking advantage of the \(\BP\) decomposition noted above.
%Let \(G(\Bu) = \BP^T D(\Bu) \BP = G(\Bu)^T\), then
%\begin{align*}
%\crossop{\Bu}&= \BP^T (\BP^T D(\Bu) \BP) - (\BP^T D(\Bu) \BP) \BP \\
%             &= \BP^T G(\Bu) - G(\Bu)^T \BP
%\end{align*}
%
% -- this does not really buy us anything to mention.
%

\subsection{cross product via four dimensional rotation}

As was noted above the cross product is closely related to a rotation in space.
This leads to a possible means for generalizing the cross product to
higher dimensions by examining the four dimensional rotation operator.

Some clarification here is probably in order.  What is meant by a four
dimensional rotation?  When the three dimensional rotation
operator was defined, it was the product in the limit of applying a possible rotation
\(d\theta_{xy}\) around the \(z\) axis, a possible rotation
\(d\theta_{yz}\) around the \(x\) axis, and a possible rotation
\(d\theta_{zx}\) around the \(y\) axis.  It is important to note that the
magnitude of these rotations is small, because otherwise the result
is different according to which order each of the rotations is applied.
\footnote{
I am not even sure that this composite rotation is a
single rotation in the traditional sense of a rotation along a plane in
space.
}
A four dimension differential rotation
operator can be defined in the same fashion as the limit of the products of the rotations
in each plane.  This is slightly more complicated in four dimensions than
in three since a rotation can be
simultaneously
perpendicular to two different axis, rather than one.  For example,
a rotation in the \(xy\) plane is perpendicular to both the \(z\) axis and the
\(w\) axis, if the space is defined as having \(w\), \(x\), \(y\), and \(z\)
components.

Each of the possible differential rotations will be enumerated below.  Let
\(\Bu = (d\theta_w, d\theta_x, d\theta_y, d\theta_z)\)
where each of the \(d\theta_i\) terms is a rotation perpendicular to the \(i\)
axis.

Differential rotations in the \(1, 2, 3\) subspace,
in the \(1, 2\) plane perpendicular to \(3\),
in the \(2, 3\) plane perpendicular to \(1\),
in the \(3, 1\) plane perpendicular to \(2\), respectively.
\begin{equation*}
\begin{bmatrix}
 1   &-u_3  & 0    & 0   \\
 u_3 & 1    & 0    & 0   \\
 0   & 0    & 1    & 0   \\
 0   & 0    & 0    & 1
\end{bmatrix}
%\end{equation*}
,
%\begin{equation*}
\begin{bmatrix}
   1 &  0   & 0    & 0    \\
   0 &  1   &-u_1  & 0    \\
   0 &  u_1 & 1    & 0    \\
   0 &  0   & 0    & 1
\end{bmatrix}
%\end{equation*}
,
%\begin{equation*}
\begin{bmatrix}
 1   & 0    &  u_2 & 0    \\
 0   & 1    & 0    & 0    \\
-u_2 & 0    & 1    & 0    \\
   0 &  0   & 0    & 1
\end{bmatrix}
\end{equation*}

Differential rotations in the \(1, 2, 4\) subspace,
in the \(1, 2\) plane perpendicular to \(4\),
in the \(2, 4\) plane perpendicular to \(1\),
in the \(4, 1\) plane perpendicular to \(2\), respectively.
\begin{equation*}
\begin{bmatrix}
 1   &-u_4  & 0    & 0   \\
 u_4 & 1    & 0    & 0   \\
 0   & 0    & 1    & 0   \\
 0   & 0    & 0    & 1
\end{bmatrix}
%\end{equation*}
,
%\begin{equation*}
\begin{bmatrix}
   1 &  0   & 0    & 0   \\
   0 &  1   & 0    &-u_1 \\
 0   & 0    & 1    & 0   \\
   0 &  u_1 & 0    & 1
\end{bmatrix}
%\end{equation*}
,
%\begin{equation*}
\begin{bmatrix}
 1   & 0    & 0    & u_2 \\
 0   & 1    & 0    & 0    \\
 0   & 0    & 1    & 0    \\
-u_2 & 0    & 0    & 1
\end{bmatrix}
\end{equation*}

Differential rotations in the \(1, 3, 4\) subspace,
in the \(1, 3\) plane perpendicular to \(4\),
in the \(3, 4\) plane perpendicular to \(1\),
in the \(4, 1\) plane perpendicular to \(3\), respectively.
\begin{equation*}
\begin{bmatrix}
 1   & 0    &-u_4  & 0   \\
 0   & 1    & 0    & 0   \\
 u_4 & 0    & 1    & 0   \\
 0   & 0    & 0    & 1
\end{bmatrix}
%\end{equation*}
,
%\begin{equation*}
\begin{bmatrix}
   1 & 0    &  0   & 0   \\
 0   & 1    & 0    & 0   \\
   0 & 0    &  1   &-u_1 \\
   0 & 0    &  u_1 & 1
\end{bmatrix}
%\end{equation*}
,
%\begin{equation*}
\begin{bmatrix}
 1   & 0    & 0    &  u_3 \\
 0   & 1    & 0    & 0    \\
 0   & 0    & 1    & 0    \\
-u_3 & 0    & 0    & 1
\end{bmatrix}
\end{equation*}

Differential rotations in the \(2, 3, 4\) subspace
in the \(2, 3\) plane perpendicular to \(4\),
in the \(3, 4\) plane perpendicular to \(2\),
in the \(4, 2\) plane perpendicular to \(3\), respectively.
\begin{equation*}
\begin{bmatrix}
   1 & 0    &  0   & 0   \\
   0 & 1    &-u_4  & 0   \\
   0 &  u_4 & 1    & 0   \\
   0 & 0    & 0    & 1
\end{bmatrix}
%\end{equation*}
,
%\begin{equation*}
\begin{bmatrix}
   1 & 0 &  0   & 0   \\
   0 & 1 &  0   & 0   \\
   0 & 0 &  1   &-u_2 \\
   0 & 0 &  u_2 & 1
\end{bmatrix}
%\end{equation*}
,
%\begin{equation*}
\begin{bmatrix}
 1   & 0   & 0    & 0   \\
 0   & 1   & 0    & u_3 \\
 0   & 0   & 1    & 0    \\
 0   &-u_3 & 0    & 1
\end{bmatrix}
\end{equation*}

In the limit, where any \(d\theta_i d\theta_j\) or higher order terms are
neglected, the product of all of these matrices \(\BR_i\) is

\begin{multline*}
\BR = \prod_i\BR_i \approx \BI + \sum_i(\BR_i-\BI) = \\
\begin{bmatrix}
%
%=================================
% 0   &-u_3  & 0    & 0   \\
% 0   & 0    &  u_2 & 0    \\
% 0   &-u_4  & 0    & 0   \\
% 0   & 0    & 0    &  u_2 \\
% 0   & 0    &-u_4  & 0   \\
% 0   & 0    & 0    &  u_3 \\
% ---------------------
1 &-u_3-u_4 &u_2-u_4 & u_2+u_3 \\
%=================================
% u_3 & 0    & 0    & 0   \\
%   0 &  0   &-u_1  & 0    \\
% u_4 & 0    & 0    & 0   \\
%   0 &  0   & 0    &-u_1 \\
%   0 & 0    &-u_4  & 0   \\
% 0   & 0   & 0    &  u_3 \\
% ---------------------
 u_3+u_4 & 1 &-u_1-u_4 &-u_1+u_3 \\
%=================================
%   0 &  u_1 & 0    & 0    \\
%-u_2 & 0    & 0    & 0    \\
% u_4 & 0    & 0    & 0   \\
%   0 & 0    &  0   &-u_1 \\
%   0 &  u_4 & 0    & 0   \\
%   0 & 0 &  0   &-u_2 \\
% ---------------------
u_4-u_2 &  u_1+u_4 & 1 &-u_1-u_2 \\
%=================================
%   0 &  u_1 & 0    & 0
%-u_2 & 0    & 0    & 0
%   0 & 0    &  u_1 & 0
%-u_3 & 0    & 0    & 0
%   0 & 0 &  u_2 & 0
% 0   &-u_3 & 0    & 0
% ---------------------
-u_2-u_3 &u_1-u_3 &  u_1+u_2 & 1
\end{bmatrix}
\end{multline*}

From this in the same fashion as was done for three dimensions
a cross product operator can be defined for \(\Bu\) as \(\crossop{\Bu}= \BR - \BI\)
to give a cross product definition for \R{4}

\begin{equation*}
\crossop{\Bu}=
\begin{bmatrix}
           0 & - u_3 - u_4 &   u_2 - u_4 &   u_2 + u_3 \\
   u_3 + u_4 & 0           & - u_1 - u_4 &   u_3 - u_1 \\
 - u_2 + u_4 & u_1 + u_4   & 0           & - u_1 - u_2 \\
 - u_2 - u_3 & u_1 - u_3   & u_1 + u_2   &   0
\end{bmatrix}
\end{equation*}

It has yet to be shown that this ``cross product'' has characteristics
similar to the three dimension cross product.  It can be verified that
it satisfies at least the following orthogonality conditions as does
the standard \R{3} cross product.

\begin{equation}\label{eqn:crossOld:160}
\begin{aligned}
\tripleprod{\Bu}{\Bv}{\Bu} &= 0 \\
\tripleprod{\Bu}{\Bv}{\Bv} &= 0 \\
      \crossprod{\Bu}{\Bu} &= \Bzero \\
      \crossprod{\Bu}{\Bv} &= -\crossprod{\Bv}{\Bu} \\
      \crossprod{\Bu}{(a\Bv + b\Bw)} &= a\crossprod{\Bu}{\Bv} + b\crossprod{\Bu}{\Bw} \\
      \crossprod{(a\Bu + b\Bv)}{\Bw} &= a\crossprod{\Bu}{\Bw} + b\crossprod{\Bv}{\Bw} \\
\tripleprod{\Bu}{\Bv}{\Bw} &= -\tripleprod{\Bu}{\Bw}{\Bv}
\end{aligned}
\end{equation}

Not all of these conditions are independent,
\(\crossprod{\Bu}{\Bv}\) is implied by \(\crossprod{\Bu}{\Bu} = -\crossprod{\Bu}{\Bu}\),
and \(\tripleprod{\Bu}{\Bv}{\Bu} = 0\) is implied since,
\begin{equation}\label{eqn:crossOld:180}
\begin{aligned}
\tripleprod{\Bu}{\Bv}{\Bu} &= -\tripleprod{\Bu}{\Bu}{\Bv} \\
                           &= -\dotprod{(\crossprod{\Bu}{\Bu})}{\Bv} \\
                           &= 0
\end{aligned}
\end{equation}
The key properties are probably the bilinearity, and the negation on exchange, but I have not yet
spent the time proving that all the rest follow.
% TODO...

\subsection{orthogonality and vector products}
I had arrived at the above result for a \R{4} cross product in a few
different ways, where this was one of the later methods.
The first ways that I arrived at this result was
by looking at orthogonality conditions and trying to extend the three
dimensional cross product in component form.  Using just the
orthogonality conditions is not enough to uniquely define a ``cross product''
even in \R{3}.

It is interesting to note that the dot product can be seen to be a statement of
the orthogonality conditions of Pythagoras law
%(the sum of
%squares of the lengths of two perpendicular line segments
%is the square of the length of the hypotonus).
\begin{equation*}
\norm{\Bu + \Bv}^2 = \norm{\Bu}^2 + \norm{\Bv}^2
\end{equation*}

In terms of components this is
\begin{equation}\label{eqn:crossOld:200}
\begin{aligned}
\norm{\Bu + \Bv}^2 &= \sum_i{(u_i + v_i)(u_i + v_i)} \\
                   &= \sum_i{{u_i}^2 + 2 u_i v_i + {v_i}^2} \\
                   &= \sum_i{{u_i}^2} + 2 \sum_i{u_i v_i} + \sum{{v_i}^2} \\
                   &= \norm{\Bu}^2 + 2 \sum_i{u_i v_i} + \norm{\Bv}^2
\end{aligned}
\end{equation}

So, if the Pythagorean condition is to hold the term, the dot product,
\begin{equation*}
\sum_i{u_i v_i}
\end{equation*}
must be zero.

The same thing can be done for the complex inner product, where
for orthogonality the term,
\begin{equation*}
\sum_i{ u_i \overline{v_i} + v_i \overline{u_i}}
\end{equation*}
must be zero.

If \(\sum_i{ u_i \overline{v_i}} = 0\), this implies
\(\overline{\sum_i{ u_i \overline{v_i}}} = \sum_i{v_i \overline{u_i}} = 0\), so the definitions of both the
complex and the real inner products arise naturally from an examination of orthogonality constraints.

The cross product is also closely related to orthogonality constraints and the \R{3}
cross product can be derived by looking specifically at these constraints.
This can be seen by
calculating the null space of a matrix with rows formed of the elements of two vectors
\(\Bu\) and \(\Bv\)

\begin{equation*}
\begin{bmatrix}
u_1 & u_2 & u_3 \\
v_1 & v_2 & v_3
\end{bmatrix}
\end{equation*}

Any vector that is in the null space is not a linear combination of the two vectors and then
must be perpendicular to it.
\footnote{a proof of this should be inserted.  Note the implicit dependence on the real
inner product here.}
Row reducing this matrix gives

\begin{equation*}
\begin{bmatrix}
u_1(u_2 v_1 - u_1 v_2) & 0                      & u_1(u_2 v_3 - u_3 v_2) \\
0                      & u_2(u_2 v_1 - u_1 v_2) & u_2(u_3 v_1 - u_1 v_3)
\end{bmatrix}
\end{equation*}

Provided that \(u_1 \neq 0\), \(u_2 \neq 0\), and \(u_2 v_1 - u_1 v_2 \neq 0\), then the
fully reduced form of this matrix is
\begin{equation*}
\begin{bmatrix}
1 & 0 & (u_2 v_3 - u_3 v_2)/(u_2 v_1 - u_1 v_2) \\
0 & 1 & (u_3 v_1 - u_1 v_3)/(u_2 v_1 - u_1 v_2)
\end{bmatrix}
\end{equation*}

So
%, where \(t\) is an arbitrary constant,
the null space is composed of the set of scalar multiples of
the vector
\begin{equation*}
\begin{bmatrix}
(u_2 v_3 - u_3 v_2)/(u_2 v_1 - u_1 v_2) \\
(u_3 v_1 - u_1 v_3)/(u_2 v_1 - u_1 v_2) \\
-1
\end{bmatrix}
\end{equation*}

of which, provided the \R{3} cross product \(\crossprod{\Bu}{\Bv}\) is one of
\begin{equation*}
\begin{bmatrix}
{u_2 v_3 - u_3 v_2} \\
{u_3 v_1 - u_1 v_3} \\
{u_1 v_2 - u_2 v_1}
\end{bmatrix}
\end{equation*}

This shows that orthogonality is not enough to uniquely define the cross product.

Doing the same null space calculations in \R{n} for the two vectors \(\Bu\) and \(\Bv\) gives the
null space as the set of vectors \({\Bn}\), where \(t_i\) are all arbitrary constants and \(\Bn\) is
defined as follows

\begin{equation*}
\Bn =
t_1
\begin{bmatrix}
u_2 v_3 - u_3 v_2 \\
u_3 v_1 - u_1 v_3 \\
u_1 v_2 - u_2 v_1 \\
0 \\
0 \\
\vdots \\
0
\end{bmatrix}
+ t_2
\begin{bmatrix}
u_2 v_4 - u_4 v_2 \\
u_4 v_1 - u_1 v_4 \\
0 \\
u_1 v_2 - u_2 v_1 \\
0 \\
\vdots \\
0
\end{bmatrix}
\hdots + t_{n-2}
\begin{bmatrix}
u_2 v_n - u_n v_2 \\
u_n v_1 - u_1 v_n \\
0 \\
\vdots \\
0 \\
0 \\
u_1 v_2 - u_2 v_1
\end{bmatrix}
\end{equation*}

This is the set of vectors that are orthogonal to both \(\Bu\) and \(\Bv\), but at a glance
no particular vector from that
set is appealing as a choice for a vector product.

Suppose, for \R{4}, setting \(t_1 = 1\) and \(t_2 = 1\), then a vector from the
null space is

\begin{equation*}
\begin{bmatrix}
u_2 v_3 - u_3 v_2 \\
u_3 v_1 - u_1 v_3 \\
u_1 v_2 - u_2 v_1 \\
0
\end{bmatrix}
+
\begin{bmatrix}
u_2 v_4 - u_4 v_2 \\
u_4 v_1 - u_1 v_4 \\
0 \\
u_1 v_2 - u_2 v_1
\end{bmatrix}
\end{equation*}

If this is compared to what was called the \R{4} cross product above, it can be seen that
the cross product has these two terms, plus two more.

\begin{equation*}
\begin{bmatrix}
u_3 v_4 - u_4 v_3 \\
0 \\
u_4 v_1 - u_1 v_4 \\
u_1 v_3 - u_3 v_1
\end{bmatrix}
+
\begin{bmatrix}
0 \\
u_3 v_4 - u_4 v_3 \\
u_4 v_2 - u_2 v_4 \\
u_2 v_3 - u_3 v_2
\end{bmatrix}
\end{equation*}

It should be possible to form these last two terms via a linear combination
of the first two, but this has not been tried.
%\footnote{
%an exercise for the reader ;)}

\subsection{more on the cross product in four dimensions}
Going back to the original decomposition of the three dimensional
cross product, a possible higher dimensional
cross product can be defined in the same fashion
$
\Bu \cross_4 = \BP_4 D(\Bu) \BP_4 - {\BP_4}^T D(\Bu) {\BP_4}^T
$

or perhaps as some more general quantity
$
\Bu \cross_4 = G(\Bu) - G(\Bu)^T
$

but how are the \(\BP_4\) or \(G\) matrices selected, so that
the result
has properties comparable to the \R{3}
cross product.

It can be noted above that the \(\BP\) matrix above is a permutation matrix. This is the identity
matrix with its rows shifted
up by one, or it columns shifted over right by one.

In four dimensions there are three permutation matrices that be can created by similarly shifting the
rows or columns of the identity matrix.  These are

\begin{equation*}
\BP = \Mp
\end{equation*}
\begin{equation*}
\BP^2 = \Mpp
\end{equation*}
\begin{equation*}
\BP^3 = \Mppp
\end{equation*}

They matrices are unitary, so for each the inverse is the transpose
\begin{equation}\label{eqn:crossOld:220}
\begin{aligned}
{\BP}^{-1} &= {\BP}^T = \BP^3 \\
{\BP^2}^{-1} &= {\BP^2}^T = \BP^2 \\
{\BP^3}^{-1} &= {\BP^3}^T = \BP
\end{aligned}
\end{equation}

With the hopes of discovering a suitable cross product operator with the form of the \R{3}
cross product operator, calculation for \({\BP^i} D(\Bu) {\BP^i}\) follows.

\begin{equation}\label{eqn:crossOld:240}
\begin{aligned}
{\BP} D(\Bu) {\BP} &=
\Mp
\Mpu
\Mp \\
& =
\begin{bmatrix}
0 & u_2 & 0 & 0 \\
0 & 0 & u_3 & 0 \\
0 & 0 & 0 & u_4 \\
u_1 & 0 & 0 & 0
\end{bmatrix}
\Mp
=
\begin{bmatrix}
0   & 0   & u_2 & 0   \\
0   & 0   & 0   & u_3 \\
u_4 & 0   & 0   & 0   \\
0   & u_1 & 0   & 0
\end{bmatrix}
\end{aligned}
\end{equation}

\begin{equation}\label{eqn:crossOld:260}
\begin{aligned}
{\BP^2} D(\Bu) {\BP^2} &=
\Mpp
\Mpu
\Mpp \\
&=
\begin{bmatrix}
0   & 0   & u_3 & 0   \\
0   & 0   & 0   & u_4 \\
u_1 & 0   & 0   & 0   \\
0   & u_2 & 0   & 0
\end{bmatrix}
\Mpp
=
\begin{bmatrix}
u_3 & 0   & 0   & 0   \\
0   & u_4 & 0   & 0   \\
0   & 0   & u_1 & 0   \\
0   & 0   & 0   & u_2
\end{bmatrix}
\end{aligned}
\end{equation}

\begin{equation}\label{eqn:crossOld:280}
\begin{aligned}
{\BP^3} D(\Bu) {\BP^3} &=
\Mppp
\Mpu
\Mppp \\
&=
\begin{bmatrix}
0   & 0   & 0   & u_4 \\
u_1 & 0   & 0   & 0   \\
0   & u_2 & 0   & 0   \\
0   & 0   & u_3 & 0
\end{bmatrix}
\Mppp
=
\begin{bmatrix}
0   & 0   & u_4 & 0   \\
0   & 0   & 0   & u_1 \\
u_2 & 0   & 0   & 0   \\
0   & u_3 & 0   & 0
\end{bmatrix}
\end{aligned}
\end{equation}

The potential cross product operators can be defined as
\begin{equation*}
\crossop{\Bu} = {\BP^i} D(\Bu) {\BP^i} - (\BP^i)^T D(\Bu) (\BP^i)^T
\end{equation*}

For \(\BP\) the following cross product operator is generated
\begin{multline*}
\crossop{\Bu} = {\BP} D(\Bu) {\BP} - {\BP}^T D(\Bu) {\BP}^T \\
=
\begin{bmatrix}
0         & 0         & u_2 - u_4 & 0         \\
0         & 0         & 0         & u_3 - u_1 \\
u_4 - u_2 & 0         & 0         & 0         \\
0         & u_1 - u_3 & 0         & 0
\end{bmatrix}
\end{multline*}

For \(\BP^2\) a trivial cross product operator is generated
\begin{equation*}
\crossop{\Bu} = {\BP^2} D(\Bu) {\BP^2} - (\BP^2)^T D(\Bu) (\BP^2)^T = \Bzero
\end{equation*}

And the cross product generated by \(\BP^3\) is just the transpose of that
for \(\BP\)
\begin{multline*}
\crossop{\Bu} = {\BP^3} D(\Bu) {\BP^3} - (\BP^3)^T D(\Bu) (\BP^3)^T \\
=
\begin{bmatrix}
0         & 0         & u_4 - u_2 & 0         \\
0         & 0         & 0         & u_1 - u_3 \\
u_2 - u_4 & 0         & 0         & 0         \\
0         & u_3 - u_1 & 0         & 0
\end{bmatrix}
\end{multline*}

Obviously the second of these does not generate a useful cross product.  Since the other two are
transposes of each other, either of those can be chosen for investigation.
Examining the first of these relations shows that
\begin{multline*}
\crossprod{\Bu}{\Bv} = ({\BP} D(\Bu) {\BP} - {\BP}^T D(\Bu) {\BP}^T) \Bv
= \\
\begin{bmatrix}
0         & 0         & u_2 - u_4 & 0         \\
0         & 0         & 0         & u_3 - u_1 \\
u_4 - u_2 & 0         & 0         & 0         \\
0         & u_1 - u_3 & 0         & 0
\end{bmatrix}
\begin{bmatrix}
v_1 \\
v_2 \\
v_3 \\
v_4
\end{bmatrix}
=
\begin{bmatrix}
(u_2 - u_4)v_3 \\
(u_3 - u_1)v_4 \\
(u_4 - u_2)v_1 \\
(u_1 - u_3)v_2
\end{bmatrix}
\end{multline*}

One of the properties that the cross product in three dimensions had was
\(\tripleprod{\Bu}{\Bv}{\Bv} = 0\) and
\(\tripleprod{\Bu}{\Bv}{\Bu} = 0\).  Does this potential cross product have the same properties?
\begin{equation*}
\tripleprod{\Bu}{\Bv}{\Bv} =
%\\
\begin{matrix}
  &v_3 v_1(u_2 - u_4) \\
+ &v_4 v_2(u_3 - u_1) \\
+ &v_1 v_3(u_4 - u_2) \\
+ &v_2 v_4(u_1 - u_3)
\end{matrix}
=
\begin{matrix}
  & v_1 v_3 ( u_2 - u_4 + u_4 - u_2) \\
+ & v_2 v_4 ( u_3 - u_1 + u_1 - u_3) \\
\end{matrix}
= 0
\end{equation*}
\begin{equation*}
\tripleprod{\Bu}{\Bv}{\Bu} =
%\\
\begin{matrix}
  &(u_2 - u_4)v_3 u_1 \\
+ &(u_3 - u_1)v_4 u_2 \\
+ &(u_4 - u_2)v_1 u_3 \\
+ &(u_1 - u_3)v_2 u_4
\end{matrix}
=
\begin{matrix}
  &u_1 u_2 ( v_3 - v_4 ) \\
+ &u_1 u_4 ( v_2 - v_3 ) \\
+ &u_2 u_3 ( v_4 - v_1 ) \\
+ &u_3 u_4 ( v_1 - v_2 )
\end{matrix}
\neq 0
\end{equation*}

Although
\(\tripleprod{\Bu}{\Bv}{\Bv} = 0\), and
\(\tripleprod{\Bu}{\Bv}{\Bw} = -\tripleprod{\Bu}{\Bw}{\Bv}\) as the \R{3}
cross product, this product seems incomplete.  There are no
\(v_1 v_2\), \(v_1 v_4\), or \(v_2 v_3\) terms.  In
\(\tripleprod{\Bu}{\Bv}{\Bu}\) there are no
\(u_1 u_3\) or \(u_2 u_4\) terms and the result is not zero as would be expected in a cross product.
\footnote{
The fact that
\(\tripleprod{\Bu}{\Bv}{\Bu} \ne 0\) can also be viewed as a consequence
of \(\crossprod{\Bu}{\Bu} \ne 0\) for this cross product, given that
\(\tripleprod{\Bu}{\Bv}{\Bu} = -\tripleprod{\Bu}{\Bu}{\Bv} \ne 0\).
}

Some of the terms that are missing can be added to generate a cross product which satisfy the same
orthogonality conditions that are true for \R{3}.
For example a \(u_1 v_3 \xcap_1\) term and a
\(u_3 v_1 \xcap_1\) term could be added.  Similarly a \(-u_3 v_2 \xcap_1\) term and a
\(u_2 v_3 \xcap_1\) term can be added.  The result for
\(\tripleprod{\Bu}{\Bv}{\Bu} = \Bu\)
had a
\(u_1 u_4 v_2\) term that resulted from the \(u_1 v_2 \xcap_4\) term of \(\crossprod{\Bu}{\Bv}\).  If
a \(-u_4 v_2 \xcap_1\) term is added then it would have canceled out.
If terms are added until each term has a ``match'' and each term of \(\tripleprod{\Bu}{\Bv}{\Bu}\)
cancels out leaving zero then the following revised cross product is generated.

\begin{equation*}
\begin{matrix}
 &(u_2 v_3 - u_3 v_2)\xcap_1 \\
+&(u_3 v_4 - u_4 v_3)\xcap_2 \\
+&(u_4 v_1 - u_1 v_4)\xcap_3 \\
+&(u_1 v_2 - u_2 v_1)\xcap_4 \\
 & \\
+&(u_3 v_4 - u_4 v_3)\xcap_1 \\
+&(u_4 v_1 - u_1 v_4)\xcap_2 \\
+&(u_1 v_2 - u_2 v_1)\xcap_3 \\
+&(u_2 v_3 - u_3 v_2)\xcap_4 \\
 & \\
+&(u_2 v_4 - u_4 v_2)\xcap_1 \\
+&(u_3 v_1 - u_1 v_3)\xcap_2 \\
+&(u_4 v_2 - u_2 v_4)\xcap_3 \\
+&(u_1 v_3 - u_3 v_1)\xcap_4
\end{matrix}
\end{equation*}

Note that this can also written as
\begin{equation*}
\begin{matrix}
% 1, 2, 3
 &(u_2 v_3 - u_3 v_2)\xcap_1 \\
+&(u_3 v_1 - u_1 v_3)\xcap_2 \\
+&(u_1 v_2 - u_2 v_1)\xcap_3 \\
 & \\
% 2, 3, 4
+&(u_3 v_4 - u_4 v_3)\xcap_2 \\
+&(u_4 v_2 - u_2 v_4)\xcap_3 \\
+&(u_2 v_3 - u_3 v_2)\xcap_4 \\
 & \\
% 1, 2, 4
+&(u_2 v_4 - u_4 v_2)\xcap_1 \\
+&(u_4 v_1 - u_1 v_4)\xcap_2 \\
+&(u_1 v_2 - u_2 v_1)\xcap_4 \\
 & \\
% 1, 3, 4
+&(u_3 v_4 - u_4 v_3)\xcap_1 \\
+&(u_4 v_1 - u_1 v_4)\xcap_3 \\
+&(u_1 v_3 - u_3 v_1)\xcap_4
\end{matrix}
\end{equation*}
where the terms are grouped into 4 sets of the three dimensional cross products on the
\((1, 2, 3)\),
\((2, 3, 4)\),
\((1, 2, 4)\), and
\((1, 3, 4)\) subspaces.

If this is put back into the matrix form \(\crossprod{\Bu}{}\) as
\begin{equation*}
\begin{bmatrix}
  0         & - u_3 - u_4 &   u_2 - u_4 &   u_2 + u_3 \\
  u_3 + u_4 &   0         & - u_1 - u_4 &   u_3 - u_1 \\
  u_4 - u_2 &   u_1 + u_4 &   0         & - u_1 - u_2 \\
- u_2 - u_3 &   u_1 - u_3 &   u_1 + u_2 &   0
\end{bmatrix}
\begin{bmatrix}
v_1 \\
v_2 \\
v_3 \\
v_4
\end{bmatrix}
\end{equation*}

Then the left hand side is the same as obtained via the \R{4} rotation method.

%%%%%% verification calculations:
%%%%%\begin{equation*}
%%%%%\crossop{\Bu}=
%%%%%\begin{bmatrix}
%%%%%  0         & - u_3 - u_4 &   u_2 - u_4 &   u_2 + u_3 \\
%%%%%  u_3 + u_4 &   0         & - u_1 - u_4 &   u_3 - u_1 \\
%%%%%  u_4 - u_2 &   u_1 + u_4 &   0         & - u_1 - u_2 \\
%%%%%- u_2 - u_3 &   u_1 - u_3 &   u_1 + u_2 &   0
%%%%%\end{bmatrix}
%%%%%\end{equation*}
%%%%%
%%%%%\begin{equation*}
%%%%%\crossop{\Bu}= \Bv
%%%%%\begin{bmatrix}
%%%%%(  0        )v_1 +(- u_3 - u_4)v_2 +(  u_2 - u_4)v_3 +(  u_2 + u_3)v_4 \\
%%%%%(  u_3 + u_4)v_1 +(  0        )v_2 +(- u_1 - u_4)v_3 +(  u_3 - u_1)v_4 \\
%%%%%(  u_4 - u_2)v_1 +(  u_1 + u_4)v_2 +(  0        )v_3 +(- u_1 - u_2)v_4 \\
%%%%%(- u_2 - u_3)v_1 +(  u_1 - u_3)v_2 +(  u_1 + u_2)v_3 +(  0        )v_4
%%%%%\end{bmatrix}
%%%%%\end{equation*}
%%%%%
%%%%%\begin{multline*}
%%%%%\tripleprod{\Bu}{\Bv}{\Bv} = \\
%%%%%\begin{bmatrix}
%%%%%(- u_3 - u_4)v_2 v_1 +(  u_2 - u_4)v_3 v_1 +(  u_2 + u_3)v_4 v_1 \\
%%%%%(  u_3 + u_4)v_1 v_2 +(- u_1 - u_4)v_3 v_2 +(  u_3 - u_1)v_4 v_2 \\
%%%%%(  u_4 - u_2)v_1 v_3 +(  u_1 + u_4)v_2 v_3 +(- u_1 - u_2)v_4 v_3 \\
%%%%%(- u_2 - u_3)v_1 v_4 +(  u_1 - u_3)v_2 v_4 +(  u_1 + u_2)v_3 v_4
%%%%%\end{bmatrix}
%%%%%= \\
%%%%%(  u_1 + u_2)v_3 v_4 + \\
%%%%%(- u_1 - u_2)v_4 v_3 + \\
%%%%%(  u_1 + u_4)v_2 v_3 + \\
%%%%%(- u_1 - u_4)v_3 v_2 + \\
%%%%%(  u_1 - u_3)v_2 v_4 + \\
%%%%%(  u_3 - u_1)v_4 v_2 + \\
%%%%%(  u_2 + u_3)v_4 v_1 + \\
%%%%%(- u_2 - u_3)v_1 v_4 + \\
%%%%%(  u_2 - u_4)v_3 v_1 + \\
%%%%%(  u_4 - u_2)v_1 v_3 + \\
%%%%%(  u_3 + u_4)v_1 v_2 + \\
%%%%%(- u_3 - u_4)v_2 v_1 \\
%%%%%= 0
%%%%%\end{multline*}
%%%%%
%%%%%\begin{multline*}
%%%%%\tripleprod{\Bu}{\Bv}{\Bu} = \\
%%%%%\begin{bmatrix}
%%%%%(- u_3 - u_4)v_2 u_1 +(  u_2 - u_4)v_3 u_1 +(  u_2 + u_3)v_4 u_1 \\
%%%%%(  u_3 + u_4)v_1 u_2 +(- u_1 - u_4)v_3 u_2 +(  u_3 - u_1)v_4 u_2 \\
%%%%%(  u_4 - u_2)v_1 u_3 +(  u_1 + u_4)v_2 u_3 +(- u_1 - u_2)v_4 u_3 \\
%%%%%(- u_2 - u_3)v_1 u_4 +(  u_1 - u_3)v_2 u_4 +(  u_1 + u_2)v_3 u_4
%%%%%\end{bmatrix} \\
%%%%% = \\
%%%%%u_1 u_2 v_3 + \\
%%%%%u_1 u_2 v_3 (-1) + \\
%%%%%u_1 u_2 v_4 + \\ + \\
%%%%%u_1 u_2 v_4 (-1) + \\
%%%%%u_1 u_3 v_2 + \\
%%%%%u_1 u_3 v_2 (-1) + \\
%%%%%u_1 u_3 v_4 + \\
%%%%%u_1 u_3 v_4 (-1) + \\
%%%%%u_1 u_4 v_2 + \\
%%%%%u_1 u_4 v_2 (-1) + \\
%%%%%u_1 u_4 v_3 + \\
%%%%%u_1 u_4 v_3 (-1) + \\
%%%%%u_2 u_3 v_1 + \\
%%%%%u_2 u_3 v_1 (-1) + \\
%%%%%u_2 u_3 v_4 + \\
%%%%%u_2 u_3 v_4 (-1) + \\
%%%%%u_2 u_4 v_1 + \\
%%%%%u_2 u_4 v_1 (-1) + \\
%%%%%u_2 u_4 v_3 + \\
%%%%%u_2 u_4 v_3 (-1) + \\
%%%%%u_3 u_4 v_1 + \\
%%%%%u_3 u_4 v_1 (-1) + \\
%%%%%u_3 u_4 v_2 + \\
%%%%%u_3 u_4 v_2 (-1) \\
%%%%% = 0
%%%%%\end{multline*}
%%%%%
%%%%%\begin{multline*}
%%%%%\crossprod{\Bu}{\Bu} = \\
%%%%%\begin{bmatrix}
%%%%%(- u_3 - u_4)u_2 +(  u_2 - u_4)u_3 +(  u_2 + u_3)u_4 \\
%%%%%(  u_3 + u_4)u_1 +(- u_1 - u_4)u_3 +(  u_3 - u_1)u_4 \\
%%%%%(  u_4 - u_2)u_1 +(  u_1 + u_4)u_2 +(- u_1 - u_2)u_4 \\
%%%%%(- u_2 - u_3)u_1 +(  u_1 - u_3)u_2 +(  u_1 + u_2)u_3
%%%%%\end{bmatrix} \\
%%%%%=0
%%%%%\end{multline*}
%%%%%% end verification

\subsection{components of the \texorpdfstring{\R{4}}{4D} cross product operator}

The \R{4} cross product operator that has been defined above was
arrived at by two different methods.  One was via an \R{4} rotation, and
the second was by starting with the decomposed form of $\Bu \cross_3 =
\BP\BD\BP - (\BP\BD\BP)^T$ and adding terms until it was ``complete'' with
respect to various orthogonality conditions that hold in \R{3}.  A
additional method of arriving at the same operator can be seen by
decomposing this operator.

To do so, \(\crossop{\Bu}\) can be written
\(G(\Bu) - {G(\Bu)}^T\) where
\begin{equation}\label{eqn:crossOld:300}
\begin{aligned}
G(\Bu)
=&
\begin{bmatrix}
  0         &   0         &   u_2       &   u_2 + u_3 \\
  u_3 + u_4 &   0         &   0         &   u_3       \\
  u_4       &   u_1 + u_4 &   0         &   0         \\
  0         &   u_1       &   u_1 + u_2 &   0
\end{bmatrix} \\
=&
\begin{bmatrix}
  0         &   0         &   u_2       &   0         \\
  0         &   0         &   0         &   u_3       \\
  u_4       &   0         &   0         &   0         \\
  0         &   u_1       &   0         &   0
\end{bmatrix} \\
+&
\begin{bmatrix}
  0         &   0         &   0         &   u_3       \\
  u_4       &   0         &   0         &   0         \\
  0         &   u_1       &   0         &   0         \\
  0         &   0         &   u_2       &   0
\end{bmatrix} \\
+&
\begin{bmatrix}
  0         &   0         &   0         &   u_2       \\
  u_3       &   0         &   0         &   0         \\
  0         &   u_4       &   0         &   0         \\
  0         &   0         &   u_1       &   0
\end{bmatrix}
\end{aligned}
\end{equation}

If this is decomposed into four sets of three dimension cross product operators on each of
the subspaces where one component is removed then
\begin{equation}\label{eqn:crossOld:320}
\begin{aligned}
G(\Bu)
=&
\begin{bmatrix}
  0         &   0         &   u_2       &   0         \\
  u_3       &   0         &   0         &   0         \\
  0         &   u_1       &   0         &   0         \\
  0         &   0         &   0         &   0
\end{bmatrix} \\
 +&
\begin{bmatrix}
  0         &   0         &   0         &   0         \\
  0         &   0         &   0         &   u_3       \\
  0         &   u_4       &   0         &   0         \\
  0         &             &   u_2       &   0
\end{bmatrix} \\
+&
\begin{bmatrix}
  0         &   0         &   0         &   u_2       \\
  u_4       &   0         &   0         &   0         \\
  0         &   0         &   0         &   0         \\
  0         &   u_1       &   0         &   0
\end{bmatrix} \\
+&
\begin{bmatrix}
  0         &   0         &   0         &   u_3       \\
  0         &   0         &   0         &   0         \\
  u_4       &   0         &   0         &   0         \\
  0         &   0         &   u_1       &   0
\end{bmatrix}
\end{aligned}
\end{equation}

Thus the \R{4} cross product operator can be generated by adding all of the \R{3} cross product
operators for each subspace where one component is removed or zeroed out.

Each of the terms of the sum for \(G(\Bu)\) above can be decomposed using the
\(\BP\), \(\BP^2\), and \(\BP^3\)
permutation matrices
\begin{equation}\label{eqn:crossOld:340}
\begin{aligned}
\begin{bmatrix}
  0         &   0         &   u_2       &   0         \\
  0         &   0         &   0         &   u_3       \\
  u_4       &   0         &   0         &   0         \\
  0         &   u_1       &   0         &   0
\end{bmatrix}
&=
\Mp \Mpu \Mp \\
&=
\BP D(\Bu) \BP
\end{aligned}
\end{equation}
\begin{equation}\label{eqn:crossOld:360}
\begin{aligned}
\begin{bmatrix}
  0         &   0         &   0         &   u_3       \\
  u_4       &   0         &   0         &   0         \\
  0         &   u_1       &   0         &   0         \\
  0         &   0         &   u_2       &   0
\end{bmatrix}
&=
\Mpp \Mpu \Mp \\
&=
\BP^2 D(\Bu) \BP
%=
%\BP \BP D(\Bu) \BP
\end{aligned}
\end{equation}
\begin{equation}\label{eqn:crossOld:380}
\begin{aligned}
\begin{bmatrix}
  0         &   0         &   0         &   u_2       \\
  u_3       &   0         &   0         &   0         \\
  0         &   u_4       &   0         &   0         \\
  0         &   0         &   u_1       &   0
\end{bmatrix}
&=
\Mp \Mpu \Mpp \\
&=
\BP D(\Bu) \BP^2
%=
%\BP D(\Bu) \BP \BP
\end{aligned}
\end{equation}

And can write in summary, that the four dimensional cross product operator is
\begin{equation*}
G(\Bu)
=
\BP D(\Bu) \BP +
\BP D(\Bu) \BP^2
+
\BP^2 D(\Bu) \BP
\end{equation*}

Defining,
\begin{equation*}
F(\Bu) = \BP D(\Bu) \BP,
\end{equation*}
the three and four dimensional cross products operator matrices can be written,

\begin{multline*}
\Bu \cross_3 =
F(\Bu) - {F(\Bu)}^T \\
\text{where}
\BP =
\begin{bmatrix}
0 & 1 & 0 \\
0 & 0 & 1 \\
1 & 0 & 0
\end{bmatrix}
\end{multline*}
\begin{multline*}
\Bu \cross_4 =
      F(\Bu)     -         {F(\Bu)}^T \\
+ \BP F(\Bu)     -         {F(\Bu)}^T {\BP}^T
    + F(\Bu) \BP - {\BP}^T {F(\Bu)}^T
\\
\text{where}
\BP = \Mp
\end{multline*}

\subsection{possible cross products in five dimensions}
Possible cross products were constructed earlier
for which the
\(\tripleprod{\Bu}{\Bv}{\Bv} = 0\)
and for which
\(\tripleprod{\Bu}{\Bv}{\Bu}\) could possibly be zero under certain conditions, but was not true generally.
It was shown that the sum of the cross products of the 4 possible \R{3} subspaces of \R{4}
was a more suitable choice for a cross product than the other construction as it generates a
result that is orthogonal to both of its component vectors.  That result could have been obtained
more directly, but the process used to arrive at it indirectly was useful or at least interesting.

Let us form the sum of the \R{4} cross products of the five possible \R{4} subspaces of \R{5} and see what the result is.
It will be simpler to use just the positive parts of the \R{4} cross product operator,
then to expand this all out explicitly.

For the \((1,2,3,4)\) subspace
\begin{equation*}
\begin{bmatrix}
  0         &   0         &   u_2       &   u_2 + u_3 &   0         \\
  u_3 + u_4 &   0         &   0         &   u_3       &   0         \\
  u_4       &   u_1 + u_4 &   0         &   0         &   0         \\
  0         &   u_1       &   u_1 + u_2 &   0         &   0         \\
  0         &   0         &   0         &   0         &   0
\end{bmatrix}
\end{equation*}
For the \((1,2,3,5)\) subspace
\begin{equation*}
\begin{bmatrix}
  0         &   0         &   u_2       &   0         &   u_2 + u_3 \\
  u_3 + u_5 &   0         &   0         &   0         &   u_3       \\
  u_5       &   u_1 + u_5 &   0         &   0         &   0         \\
  0         &   0         &   0         &   0         &   0         \\
  0         &   u_1       &   u_1 + u_2 &   0         &   0
\end{bmatrix}
\end{equation*}
For the \((1,2,4,5)\) subspace
\begin{equation*}
\begin{bmatrix}
  0         &   0         &   0         &   u_2       &   u_2 + u_4 \\
  u_4 + u_5 &   0         &   0         &   0         &   u_4       \\
  0         &   0         &   0         &   0         &   0         \\
  u_5       &   u_1 + u_5 &   0         &   0         &   0         \\
  0         &   u_1       &   0         &   u_1 + u_2 &   0
\end{bmatrix}
\end{equation*}
For the \((1,3,4,5)\) subspace
\begin{equation*}
\begin{bmatrix}
  0         &   0         &   0         &   u_3       &   u_3 + u_4 \\
  0         &   0         &   0         &   0         &   0         \\
  u_4 + u_5 &   0         &   0         &   0         &   u_4       \\
  u_5       &   0         &   u_1 + u_5 &   0         &   0         \\
  0         &   0         &   u_1       &   u_1 + u_3 &   0
\end{bmatrix}
\end{equation*}
For the \((2,3,4,5)\) subspace
\begin{equation*}
\begin{bmatrix}
  0         & 0         &   0         &   0         &   0         \\
  0         & 0         &   0         &   u_3       &   u_3 + u_4 \\
  0         & u_4 + u_5 &   0         &   0         &   u_4       \\
  0         & u_5       &   u_2 + u_5 &   0         &   0         \\
  0         & 0         &   u_2       &   u_2 + u_3 &   0
\end{bmatrix}
\end{equation*}

%\begin{bmatrix}
%  0         &   0         &   u_2       &   u_2 + u_3 &   0         \\
%  0         &   0         &   u_2       &   0         &   u_2 + u_3 \\
%  0         &   0         &   0         &   u_2       &   u_2 + u_4 \\
%  0         &   0         &   0         &   u_3       &   u_3 + u_4 \\
%  0         &   0         &   0         &   0         &   0         \\
%----------------------------------------------------------------------
%  0         &   0         &   2 u_2     &   2 u_2     &   2 u_2     \\
%                                          + 2 u_3       + 2 u_3
%                                                        + 2 u_4
%----------------------------------------------------------------------
%  0         &   0         &   2 u_2     & 2 u_2 +2u_3&2u_2+2u_3+2u_4 \\
%
%
%  u_3 + u_4 &   0         &   0         &   u_3       &   0         \\
%  u_3 + u_5 &   0         &   0         &   0         &   u_3       \\
%  u_4 + u_5 &   0         &   0         &   0         &   u_4       \\
%  0         &   0         &   0         &   0         &   0         \\
%  0         &   0         &   0         &   u_3       &   u_3 + u_4 \\
%----------------------------------------------------------------------
%  2 u_3     &   0         &   0         &   2 u_3     &   2 u_3     \\
%+ 2 u_4                                                 + 2 u_4
%+ 2 u_5
%----------------------------------------------------------------------
%2u_3+2u_4+2u_5& 0         &   0         &   2 u_3     &   2 u_3 +2u_4    \\
%
%
%  u_4       &   u_1 + u_4 &   0         &   0         &   0         \\
%  u_5       &   u_1 + u_5 &   0         &   0         &   0         \\
%  0         &   0         &   0         &   0         &   0         \\
%  u_4 + u_5 &   0         &   0         &   0         &   u_4       \\
%  0         &   u_4 + u_5 &   0         &   0         &   u_4       \\
%----------------------------------------------------------------------
%  2 u_4     &   2 u_4     &   0         &   0         &   2 u_4     \\
%+ 2 u_5       + 2 u_5
%              + 2 u_1
%----------------------------------------------------------------------
%  2 u_4+2u_5&2u_1+2u_4+2u_5 &  0        &   0         &   2 u_4     \\
%
%
%  0         &   u_1       &   u_1 + u_2 &   0         &   0         \\
%  0         &   0         &   0         &   0         &   0         \\
%  u_5       &   u_1 + u_5 &   0         &   0         &   0         \\
%  u_5       &   0         &   u_1 + u_5 &   0         &   0         \\
%  0         &   u_5       &   u_2 + u_5 &   0         &   0         \\
%----------------------------------------------------------------------
%  2 u_5     &   2 u_1     &   2 u_1     &   0         &   0         \\
%              + 2 u_5     & + 2 u_2                                 \\
%                            + 2 u_5                                 \\
%----------------------------------------------------------------------
%  2 u_5     & 2 u_1 + 2u_5&2u_1+2u_2+2u_5&  0         &   0         \\
%
%
%  0         &   0         &   0         &   0         &   0
%  0         &   u_1       &   u_1 + u_2 &   0         &   0
%  0         &   u_1       &   0         &   u_1 + u_2 &   0
%  0         &   0         &   u_1       &   u_1 + u_3 &   0
%  0         &   0         &   u_2       &   u_2 + u_3 &   0
%----------------------------------------------------------------------
%  0         &   2 u_1     &  2u_1+2u_2  &2u_1+2u_2+2u_3 &   0
%\end{bmatrix}
%
%
%
%Adding these yields (too wide to fit across one line)
%\begin{bmatrix}
%0                     & 0                     &         2 u_2         &         2 u_2 + 2 u_3 & 2 u_2 + 2 u_3 + 2 u_4 \\
%2 u_3 + 2 u_4 + 2 u_5 & 0                     & 0                     &         2 u_3         &         2 u_3 + 2 u_4 \\
%        2 u_4 + 2 u_5 & 2 u_1 + 2 u_4 + 2 u_5 & 0                     & 0                     &                 2 u_4 \\
%                2 u_5 & 2 u_1 +         2 u_5 & 2 u_1 + 2 u_2 + 2 u_5 & 0                     & 0                     \\
%0                     & 2 u_1                 & 2 u_1 + 2 u_2         & 2 u_1 + 2 u_2 + 2 u_3 & 0
%\end{bmatrix}
%
%
%
%Adding these yields (where the first matrix is the first three columns and the second is the last two)
%\begin{multline*}
%\begin{bmatrix}
%%0                     & 0                     &         2 u_2         &\\
%2 u_3 + 2 u_4 + 2 u_5 & 0                     & 0                     &\\
%        2 u_4 + 2 u_5 & 2 u_1 + 2 u_4 + 2 u_5 & 0                     &\\
%                2 u_5 & 2 u_1 +         2 u_5 & 2 u_1 + 2 u_2 + 2 u_5 &\\
%0                     & 2 u_1                 & 2 u_1 + 2 u_2         &
%\end{bmatrix} \\
%\begin{bmatrix}
%&         2 u_2 + 2 u_3 & 2 u_2 + 2 u_3 + 2 u_4 \\
%&         2 u_3         &         2 u_3 + 2 u_4 \\
%& 0                     &                 2 u_4 \\
%& 0                     & 0                     \\
%& 2 u_1 + 2 u_2 + 2 u_3 & 0
%\end{bmatrix}
%\end{multline*}

Which sums to the following
\begin{equation}\label{eqn:crossOld:400}
\begin{aligned}
2&
\begin{bmatrix}
0                     & 0                     &           u_2         &           u_2 +   u_3 &   u_2 +   u_3 +   u_4 \\
  u_3 +   u_4 +   u_5 & 0                     & 0                     &           u_3         &           u_3 +   u_4 \\
          u_4 +   u_5 &   u_1 +   u_4 +   u_5 & 0                     & 0                     &                   u_4 \\
                  u_5 &   u_1 +           u_5 &   u_1 +   u_2 +   u_5 & 0                     & 0                     \\
0                     &   u_1                 &   u_1 +   u_2         &   u_1 +   u_2 +   u_3 & 0
\end{bmatrix} \\
=&
2
\begin{bmatrix}
 0   & 0   & 0   & 0   & u_2 \\
 u_3 & 0   & 0   & 0   & 0 \\
 0   & u_4 & 0   & 0   & 0 \\
 0   & 0   & u_5 & 0   & 0 \\
 0   & 0   & 0   & u_1 & 0
\end{bmatrix}
+2
\begin{bmatrix}
 0   & 0   & 0   & 0   & u_3 \\
 u_4 & 0   & 0   & 0   & 0 \\
 0   & u_5 & 0   & 0   & 0 \\
 0   & 0   & u_1 & 0   & 0 \\
 0   & 0   & 0   & u_2 & 0
\end{bmatrix} \\
+&2
\begin{bmatrix}
 0   & 0   & 0   & 0   & u_4 \\
 u_5 & 0   & 0   & 0   & 0 \\
 0   & u_1 & 0   & 0   & 0 \\
 0   & 0   & u_2 & 0   & 0 \\
 0   & 0   & 0   & u_3 & 0
\end{bmatrix}
+2
\begin{bmatrix}
 0   & 0   & 0   & u_2 & 0 \\
 0   & 0   & 0   & 0   & u_3 \\
 u_4 & 0   & 0   & 0   & 0 \\
 0   & u_5 & 0   & 0   & 0 \\
 0   & 0   & u_1 & 0   & 0
\end{bmatrix} \\
+&2
\begin{bmatrix}
 0   & 0   & 0   & u_3 & 0 \\
 0   & 0   & 0   & 0   & u_4 \\
 u_5 & 0   & 0   & 0   & 0 \\
 0   & u_1 & 0   & 0   & 0 \\
 0   & 0   & u_2 & 0   & 0
\end{bmatrix}
+2
\begin{bmatrix}
 0   & 0   & u_2 & 0   & 0 \\
 0   & 0   & 0   & u_3 & 0 \\
 0   & 0   & 0   & 0   & u_4 \\
 u_5 & 0   & 0   & 0   & 0 \\
 0   & u_1 & 0   & 0   & 0
\end{bmatrix}
\end{aligned}
\end{equation}

If \(\BP\) is defined as
\begin{equation*}
\begin{bmatrix}
 0 & 1 & 0 & 0 & 0 \\
 0 & 0 & 1 & 0 & 0 \\
 0 & 0 & 0 & 1 & 0 \\
 0 & 0 & 0 & 0 & 1 \\
 1 & 0 & 0 & 0 & 0
\end{bmatrix}
\end{equation*}
then the \R{5} cross product operator can be written as
\begin{equation*}
\Bu \cross_5 = G(\Bu) - {G(\Bu)}^T
\end{equation*}
where
\begin{multline*}
G(\Bu) = 2 (
\BP D(\Bu) \BP
+ \BP D(\Bu) \BP^2
+ \BP^2 D(\Bu) \BP^2
+ \BP^2 D(\Bu) \BP \\
+ \BP D(\Bu) \BP^3
+ \BP^3 D(\Bu) \BP
)
\end{multline*}

Reiterating the results for each of
\R{3},
\R{4}, and
\R{5}, where \(\BD = D(\Bu)\)

\begin{equation*}
\Bu \cross_3 =
\BP\BD\BP
-
(\BP\BD\BP)^T
\end{equation*}
\begin{equation*}
\BP =
\begin{bmatrix}
0 & 1 & 0 \\
0 & 0 & 1 \\
1 & 0 & 0
\end{bmatrix}
\end{equation*}
\begin{equation*}
\Bu \cross_4 =
\BP\BD\BP + \BP\BD\BP^2 + \BP^2\BD\BP
-
(\BP\BD\BP + \BP\BD\BP^2 + \BP^2\BD\BP)^T
\end{equation*}
\begin{equation*}
\BP = \Mp
\end{equation*}
\begin{multline*}
\Bu \cross_5 =
2 (
  \BP\BD\BP
+ \BP\BD\BP^2
+ \BP^2\BD\BP^2
+ \BP^2\BD\BP
+ \BP\BD\BP^3
+ \BP^3\BD\BP
)
- \\
2 (
  \BP\BD\BP
+ \BP\BD\BP^2
+ \BP^2\BD\BP^2
+ \BP^2\BD\BP
+ \BP\BD\BP^3
+ \BP^3\BD\BP
)^T
\end{multline*}
\begin{equation*}
\BP =
\begin{bmatrix}
 0 & 1 & 0 & 0 & 0 \\
 0 & 0 & 1 & 0 & 0 \\
 0 & 0 & 0 & 1 & 0 \\
 0 & 0 & 0 & 0 & 1 \\
 1 & 0 & 0 & 0 & 0
\end{bmatrix}
\end{equation*}

There is a definite pattern here.  Looking at the positive parts, this pattern can extrapolated to higher dimensions
\begin{equation}\label{eqn:crossOld:420}
\begin{aligned}
\BR^3 : &\PDP{1}{1} \\
\BR^4 : &\PDP{1}{1} + \PDP{1}{2} \\
      + &\PDP{2}{1} \\
\BR^5 : &\PDP{1}{1} + \PDP{1}{2} + \PDP{1}{3} \\
      + &\PDP{2}{1} + \PDP{2}{2} \\
      + &\PDP{3}{1} \\
\BR^6 : &\PDP{1}{1} + \PDP{1}{2} + \PDP{1}{3} + \PDP{1}{4} \\
      + &\PDP{2}{1} + \PDP{2}{2} + \PDP{2}{3} \\
      + &\PDP{3}{1} + \PDP{3}{2} \\
      + &\PDP{4}{1} \\
\BR^7 : &\hdots
\end{aligned}
\end{equation}
In the above all of the sets of \(\PDP{i}{j}\) have been included such that \(i+j < n\), the dimension of the
vector space.  The number of these matrices is \(\sum_{i=1}^{n-2}{i} = \frac{1}{2}(n-2)(n-1)\).
Because of the way these operators have been constructed, for \(\Bu \cross\)
in \R{4} there are \((4) (3) = 12 \) positive terms (in \R{3} there are \(3\) positive terms),
in \R{5} there are \((5) (12)\) positive terms, and
in \R{n} it can be expected that there will be \((n) (n-1) ... (3) = n!/2\) positive terms.

Each of the matrices \(\PDP{i}{j}\) contributes \(n\) positive terms to
the cross product, and so the following multiplicative factor can be added to the terms above
to have a definition consistent with the \R{5} cross product operator derived above.
\begin{equation*}
\frac{\frac{1}{n}\frac{n!}{2}}{\frac{1}{2}(n-2)(n-1)} = (n-3)!
\end{equation*}

Using this pattern, the general cross product operator matrix for \R{n} can be written as
\begin{equation}\label{eqn:crossOld:440}
\begin{aligned}
\Bu \cross_n &= (n-3)! \sum_{i=1}^{n-2}\sum_{j=1}^{n-1-i}(\PDP{i}{j} - (\PDP{i}{j})^T)
\\
%\end{equation*}
%\begin{equation*}
\text{where}\:
\BP &=
\begin{bmatrix}
0 & 1 & 0 & \hdots & 0 \\
0 & 0 & 1 & 0      & 0 \\
0 & 0 & 0 & \ddots & 0 \\
0 & 0 & 0 & 0      & 1 \\
1 & 0 & 0 & \hdots & 0
\end{bmatrix}
\\
%\end{equation*}
%\begin{equation*}
\text{and}\:
\BD &=
\begin{bmatrix}
u_1 & 0   & \hdots & 0       & 0 \\
0   & u_2 & 0      & 0       & 0 \\
0   & 0   & \ddots & 0       & 0 \\
0   & 0   & 0      & u_{n-1} & 0 \\
0   & 0   & \hdots & 0       & u_n
\end{bmatrix}
\end{aligned}
\end{equation}

Note that for higher than \R{5} it has not yet been verified that
\(\tripleprod{\Bu}{\Bv}{\Bu} = 0\)
or that
\(\tripleprod{\Bu}{\Bv}{\Bv} = 0\)
or that
\(\crossprod{\Bu}{\Bu} = \Bzero\)
.  These have been verified explicitly for \R{4} and \R{5}, but the calculations are too tedious to show.

%%%%%% verification:
%%%%%\subsection{validity check for generalized cross product operator}
%%%%%
%%%%%For \R{5} let us calculate \(\crossprod{\Bu}{\Bv}\) and see if this is orthogonal to both \(\Bu\) and \(\Bv\),
%%%%%and see if \(\crossprod{\Bu}{\Bu} = 0\).
%%%%%\begin{multline*}
%%%%%\frac{1}{2} \Bu \cross_5 = \\
%%%%%\begin{bmatrix}
%%%%% 0           &-u_3-u_4-u_5 & u_2-u_4-u_5 & u_2+u_3-u_5 & u_2+u_3+u_4 \\
%%%%% u_3+u_4+u_5 & 0           &-u_1-u_4-u_5 & u_3-u_1-u_5 & u_3+u_4-u_1 \\
%%%%% u_4+u_5-u_2 & u_1+u_4+u_5 & 0           &-u_1-u_2-u_5 & u_4-u_1-u_2 \\
%%%%% u_5-u_2-u_3 & u_1+u_5-u_3 & u_1+u_2+u_5 & 0           &-u_1-u_2-u_3 \\
%%%%%-u_2-u_3-u_4 & u_1-u_3-u_4 & u_1+u_2-u_4 & u_1+u_2+u_3 & 0
%%%%%\end{bmatrix}
%%%%%\end{multline*}
%%%%%
%%%%%\begin{multline*}
%%%%%\frac{1}{2} \Bu \cross_5 \Bv = \\
%%%%%\begin{bmatrix}
%%%%%(-u_3-u_4-u_5)v_2 +( u_2-u_4-u_5)v_3 +( u_2+u_3-u_5)v_4 +( u_2+u_3+u_4)v_5 \\
%%%%%( u_3+u_4+u_5)v_1 +(-u_1-u_4-u_5)v_3 +( u_3-u_1-u_5)v_4 +( u_3+u_4-u_1)v_5 \\
%%%%%( u_4+u_5-u_2)v_1 +( u_1+u_4+u_5)v_2 +(-u_1-u_2-u_5)v_4 +( u_4-u_1-u_2)v_5 \\
%%%%%( u_5-u_2-u_3)v_1 +( u_1+u_5-u_3)v_2 +( u_1+u_2+u_5)v_3 +(-u_1-u_2-u_3)v_5 \\
%%%%%(-u_2-u_3-u_4)v_1 +( u_1-u_3-u_4)v_2 +( u_1+u_2-u_4)v_3 +( u_1+u_2+u_3)v_4
%%%%%\end{bmatrix}
%%%%%\end{multline*}
%%%%%
%%%%%\begin{multline*}
%%%%%\frac{1}{2} \Bu \cross_5 \Bv \cdot \Bv = \\
%%%%%(-u_3-u_4-u_5)v_2 v_1 +( u_2-u_4-u_5)v_3 v_1 +( u_2+u_3-u_5)v_4 v_1 +( u_2+u_3+u_4)v_5 v_1 \\
%%%%%( u_3+u_4+u_5)v_1 v_2 +(-u_1-u_4-u_5)v_3 v_2 +( u_3-u_1-u_5)v_4 v_2 +( u_3+u_4-u_1)v_5 v_2 \\
%%%%%( u_4+u_5-u_2)v_1 v_3 +( u_1+u_4+u_5)v_2 v_3 +(-u_1-u_2-u_5)v_4 v_3 +( u_4-u_1-u_2)v_5 v_3 \\
%%%%%( u_5-u_2-u_3)v_1 v_4 +( u_1+u_5-u_3)v_2 v_4 +( u_1+u_2+u_5)v_3 v_4 +(-u_1-u_2-u_3)v_5 v_4 \\
%%%%%(-u_2-u_3-u_4)v_1 v_5 +( u_1-u_3-u_4)v_2 v_5 +( u_1+u_2-u_4)v_3 v_5 +( u_1+u_2+u_3)v_4 v_5 \\
%%%%%= \\
%%%%%v_1 v_2( u_3+u_4+u_5)) + \\
%%%%%v_1 v_2(-u_3-u_4-u_5)) + \\
%%%%%v_1 v_3( u_2-u_4-u_5)) + \\
%%%%%v_1 v_3( u_4+u_5-u_2)) + \\
%%%%%v_1 v_4( u_2+u_3-u_5)) + \\
%%%%%v_1 v_4( u_5-u_2-u_3)) + \\
%%%%%v_1 v_5( u_2+u_3+u_4)) + \\
%%%%%v_1 v_5(-u_2-u_3-u_4)) + \\
%%%%%v_2 v_3( u_1+u_4+u_5)) + \\
%%%%%v_2 v_3(-u_1-u_4-u_5)) + \\
%%%%%v_2 v_4( u_1+u_5-u_3)) + \\
%%%%%v_2 v_4( u_3-u_1-u_5)) + \\
%%%%%v_2 v_5( u_1-u_3-u_4)) + \\
%%%%%v_2 v_5( u_3+u_4-u_1)) + \\
%%%%%v_3 v_4( u_1+u_2+u_5)) + \\
%%%%%v_3 v_4(-u_1-u_2-u_5)) + \\
%%%%%v_3 v_5( u_1+u_2-u_4)) + \\
%%%%%v_3 v_5( u_4-u_1-u_2)) + \\
%%%%%v_4 v_5( u_1+u_2+u_3)) + \\
%%%%%v_4 v_5(-u_1-u_2-u_3)) \\
%%%%%= 0
%%%%%\end{multline*}
%%%%%
%%%%%\begin{multline*}
%%%%%\frac{1}{2} \Bu \cross_5 \Bv \cdot \Bu = \\
%%%%%(-u_3-u_4-u_5)v_2 u_1 +( u_2-u_4-u_5)v_3 u_1 +( u_2+u_3-u_5)v_4 u_1 +( u_2+u_3+u_4)v_5 u_1 \\
%%%%%( u_3+u_4+u_5)v_1 u_2 +(-u_1-u_4-u_5)v_3 u_2 +( u_3-u_1-u_5)v_4 u_2 +( u_3+u_4-u_1)v_5 u_2 \\
%%%%%( u_4+u_5-u_2)v_1 u_3 +( u_1+u_4+u_5)v_2 u_3 +(-u_1-u_2-u_5)v_4 u_3 +( u_4-u_1-u_2)v_5 u_3 \\
%%%%%( u_5-u_2-u_3)v_1 u_4 +( u_1+u_5-u_3)v_2 u_4 +( u_1+u_2+u_5)v_3 u_4 +(-u_1-u_2-u_3)v_5 u_4 \\
%%%%%(-u_2-u_3-u_4)v_1 u_5 +( u_1-u_3-u_4)v_2 u_5 +( u_1+u_2-u_4)v_3 u_5 +( u_1+u_2+u_3)v_4 u_5 \\
%%%%%= \\
%%%%%(-u_3-u_4-u_5)v_2 u_1 + \\
%%%%%(+u_2-u_4-u_5)v_3 u_1 + \\
%%%%%(+u_2+u_3-u_5)v_4 u_1 + \\
%%%%%(+u_2+u_3+u_4)v_5 u_1 + \\
%%%%%(+u_3+u_4+u_5)v_1 u_2 + \\
%%%%%(-u_1-u_4-u_5)v_3 u_2 + \\
%%%%%(+u_3-u_1-u_5)v_4 u_2 + \\
%%%%%(+u_3+u_4-u_1)v_5 u_2 + \\
%%%%%(+u_4+u_5-u_2)v_1 u_3 + \\
%%%%%(+u_1+u_4+u_5)v_2 u_3 + \\
%%%%%(-u_1-u_2-u_5)v_4 u_3 + \\
%%%%%(+u_4-u_1-u_2)v_5 u_3 + \\
%%%%%(+u_5-u_2-u_3)v_1 u_4 + \\
%%%%%(+u_1+u_5-u_3)v_2 u_4 + \\
%%%%%(+u_1+u_2+u_5)v_3 u_4 + \\
%%%%%(-u_1-u_2-u_3)v_5 u_4 + \\
%%%%%(-u_2-u_3-u_4)v_1 u_5 + \\
%%%%%(+u_1-u_3-u_4)v_2 u_5 + \\
%%%%%(+u_1+u_2-u_4)v_3 u_5 + \\
%%%%%(+u_1+u_2+u_3)v_4 u_5 + \\
%%%%%= \\
%%%%%u_1 u_2 v_3 + \\
%%%%%u_1 u_2 v_4 + \\
%%%%%u_1 u_2 v_5 + \\
%%%%%u_1 u_2(-1) v_3 + \\
%%%%%u_1 u_2(-1) v_4 + \\
%%%%%u_1 u_2(-1) v_5 + \\
%%%%%u_1 u_3 v_2 + \\
%%%%%u_1 u_3 v_4 + \\
%%%%%u_1 u_3 v_5 + \\
%%%%%u_1 u_3(-1) v_2 + \\
%%%%%u_1 u_3(-1) v_4 + \\
%%%%%u_1 u_3(-1) v_5 + \\
%%%%%u_1 u_4 v_2 + \\
%%%%%u_1 u_4 v_3 + \\
%%%%%u_1 u_4 v_5 + \\
%%%%%u_1 u_4(-1) v_2 + \\
%%%%%u_1 u_4(-1) v_3 + \\
%%%%%u_1 u_4(-1) v_5 + \\
%%%%%u_1 u_5 v_2 + \\
%%%%%u_1 u_5 v_3 + \\
%%%%%u_1 u_5 v_4 + \\
%%%%%u_1 u_5(-1) v_2 + \\
%%%%%u_1 u_5(-1) v_3 + \\
%%%%%u_1 u_5(-1) v_4 + \\
%%%%%u_2 u_3 v_1 + \\
%%%%%u_2 u_3 v_4 + \\
%%%%%u_2 u_3 v_5 + \\
%%%%%u_2 u_3(-1) v_1 + \\
%%%%%u_2 u_3(-1) v_4 + \\
%%%%%u_2 u_3(-1) v_5 + \\
%%%%%u_2 u_4 v_1 + \\
%%%%%u_2 u_4 v_3 + \\
%%%%%u_2 u_4 v_5 + \\
%%%%%u_2 u_4(-1) v_1 + \\
%%%%%u_2 u_4(-1) v_3 + \\
%%%%%u_2 u_4(-1) v_5 + \\
%%%%%u_2 u_5 v_1 + \\
%%%%%u_2 u_5 v_3 + \\
%%%%%u_2 u_5 v_4 + \\
%%%%%u_2 u_5(-1) v_1 + \\
%%%%%u_2 u_5(-1) v_3 + \\
%%%%%u_2 u_5(-1) v_4 + \\
%%%%%u_3 u_4 v_1 + \\
%%%%%u_3 u_4 v_2 + \\
%%%%%u_3 u_4 v_5 + \\
%%%%%u_3 u_4(-1) v_1 + \\
%%%%%u_3 u_4(-1) v_2 + \\
%%%%%u_3 u_4(-1) v_5 + \\
%%%%%u_3 u_5 v_1 + \\
%%%%%u_3 u_5 v_2 + \\
%%%%%u_3 u_5 v_4 + \\
%%%%%u_3 u_5(-1) v_1 + \\
%%%%%u_3 u_5(-1) v_2 + \\
%%%%%u_3 u_5(-1) v_4 + \\
%%%%%u_4 u_5 v_1 + \\
%%%%%u_4 u_5 v_2 + \\
%%%%%u_4 u_5 v_3 + \\
%%%%%u_4 u_5(-1) v_1 + \\
%%%%%u_4 u_5(-1) v_2 + \\
%%%%%u_4 u_5(-1) v_3
%%%%%\end{multline*}
%%%%%
%%%%%\begin{multline*}
%%%%%\frac{1}{2} \Bu \cross_5 \Bu = \\
%%%%%\begin{bmatrix}
%%%%%(-u_3-u_4-u_5)u_2 +( u_2-u_4-u_5)u_3 +( u_2+u_3-u_5)u_4 +( u_2+u_3+u_4)u_5 \\
%%%%%( u_3+u_4+u_5)u_1 +(-u_1-u_4-u_5)u_3 +( u_3-u_1-u_5)u_4 +( u_3+u_4-u_1)u_5 \\
%%%%%( u_4+u_5-u_2)u_1 +( u_1+u_4+u_5)u_2 +(-u_1-u_2-u_5)u_4 +( u_4-u_1-u_2)u_5 \\
%%%%%( u_5-u_2-u_3)u_1 +( u_1+u_5-u_3)u_2 +( u_1+u_2+u_5)u_3 +(-u_1-u_2-u_3)u_5 \\
%%%%%(-u_2-u_3-u_4)u_1 +( u_1-u_3-u_4)u_2 +( u_1+u_2-u_4)u_3 +( u_1+u_2+u_3)u_4
%%%%%\end{bmatrix}
%%%%%= 0
%%%%%\end{multline*}
%%%%%% end verification.

One additional property that holds for the three dimensional cross product
that also holds for the \R{n} version is
\(\tripleprod{\Bu}{\Bv}{\Bw} = -\tripleprod{\Bu}{\Bw}{\Bv}\).
If \(\crossprod{\Bu}{\Bu} = 0\) is true for the \R{n} cross product
defined above, then this implies that
\(\tripleprod{\Bu}{\Bv}{\Bu} = 0\) too.  This first property can be shown by
writing \(\crossop{\Bu} = \BG - {\BG}^T = [g_{ij}]\), so that

\begin{equation*}
(\crossprod{\Bu}{\Bv})_i = \sum_{s=1}^{n}{g_{is}v_s}
\end{equation*}

thus for the triple-product
\begin{equation}\label{eqn:crossOld:460}
\begin{aligned}
\tripleprod{\Bu}{\Bv}{\Bw} &= \sum_{t=1}^{n}{(\sum_{s=1}^{n}g_{ts}v_s)w_t} \\
                           &= \sum_{s=1}^{n}{(\sum_{t=1}^{n}g_{ts}w_t)v_s} \\
			   &= \dotprod{{([g_{ij}]}^T \Bw)}{\Bv} \\
			   &= \dotprod{(\BG-\BG^T)^T \Bw}{\Bv} \\
			   &= -\dotprod{(\BG-\BG^T) \Bw}{\Bv} \\
			   &= -\tripleprod{\Bu}{\Bw}{\Bv}
\end{aligned}
\end{equation}

I suspect that \(\crossprod{\Bu}{\Bu} = \Bzero\) and that \(\tripleprod{\Bu}{\Bv}{\Bv} = 0\) also hold for \(n>5\) in the \R{n} cross product as defined above.
Some sort of recursive proof for this is probably required to show this.

\subsection{on the magnitude of the cross product operator}

The orthogonality properties of the cross product operator are not the only
ones of interest, since the cross product in \R{3} has a specific
magnitude as well as direction.

The projective form
\(\crossprod{\Bu}{\Bv} = \norm{\Bu}\norm{\Bv}\sin(\Bu,\Bv) \ncap\) may give
some indication of what to expect for \R{n}, where \((\Bu,\Bv)\) is the
angle between the two vectors \(\Bu\) and \(\Bv\), and \(\ncap\) is a unit
vector in the direction of \(\crossprod{\Bu}{\Bv}\).  However, how would the
angle be defined for \R{n}.

For this we can go to the projective form of the dot product
\(\dotprod{\Bu}{\Bv} = \norm{\Bu}\norm{\Bv}\cos(\Bu,\Bv)\).

Note that this form of the dot product comes directly from the
triangle law of trigonometry.

\begin{equation}\label{eqn:crossOld:480}
\begin{aligned}
\norm{\Bu - \Bv}^2 &= \norm{\Bu}^2 + \norm{\Bv}^2 -2\norm{\Bu}\norm{\Bv}\cos(\Bu,\Bv) \\
                  &= \innerprod{\Bu-\Bv}{\Bu-\Bv} \\
                  &=
		  \innerprod{\Bu}{\Bu}
		  +\innerprod{\Bv}{\Bv}
		  -\innerprod{\Bu}{\Bv}
		  -\innerprod{\Bv}{\Bu} \\
                  &= \norm{\Bu}^2 + \norm{\Bv}^2 -\innerprod{\Bu}{\Bv} -\innerprod{\Bv}{\Bu}
\end{aligned}
\end{equation}

The result follows, since for the real case, \(\innerprod{\Bu}{\Bv} + \innerprod{\Bv}{\Bu} = 2\dotprod{\Bu}{\Bv}\).

If \(\cos(\Bu,\Bv) = \frac{\dotprod{\Bu}{\Bv}}{\norm{\Bu}\norm{\Bv}}\) is taken to implicitly define the
angle between two vectors in \R{n}, then the magnitude of the \R{n} cross product could be defined in
the following fashion as is true for \R{3}

\begin{dmath}\label{eqn:crossOld:500}
\norm{\crossprod{\Bu}{\Bv}}^2
=
\norm{\Bu}^2\norm{\Bv}^2
\lr{
1-
\lr{
\frac{\dotprod{\Bu}{\Bv}}{\norm{\Bu}\norm{\Bv}}
}^2
}
= \norm{\Bu}^2\norm{\Bv}^2 - (\dotprod{\Bu}{\Bv})^2
\end{dmath}

Note
that using the norm squared as a measure of magnitude looses the sign of the magnitude.  There may be
a better way of defining \(\sin(\Bu,\Bv) = \sqrt{1 - \frac{\dotprod{\Bu}{\Bv}}{\norm{\Bu}\norm{\Bv}}}\)
because this has an implied sign ambiguity.  Then again the \(\ncap\) term has also been ignored, so
perhaps the positive root is an acceptable angular measure.

I still need to check if this is true for the \R{n} cross product operator as defined above for \(n>3\).
In order to calculate
\(\norm{\Bu \cross_n \Bv}^2\)
the following sum has to be evaluated

\begin{multline*}
((n-3)!)^2
\Bigl(
\sum_{i=1}^{n-2}\sum_{j=1}^{n-1-i}(\PDP{i}{j}\Bv - (\PDP{i}{j})^T\Bv)
\Bigr)
\cdot \\
\Bigl(
\sum_{i'=1}^{n-2}\sum_{j'=1}^{n-1-i'}(\PDP{i'}{j'}\Bv - (\PDP{i'}{j'})^T\Bv)
\Bigr)
\end{multline*}

The dot product can be brought into the sum
\begin{multline*}
((n-3)!)^2
\sum_{i=1}^{n-2}\sum_{j=1}^{n-1-i}
\sum_{i'=1}^{n-2}\sum_{j'=1}^{n-1-i'}
\Bigl(
(\PDP{i}{j}\Bv - (\PDP{i}{j})^T\Bv)
\\
\cdot
(\PDP{i'}{j'}\Bv - (\PDP{i'}{j'})^T\Bv)
\Bigr)
\end{multline*}

and this can be expanded
\begin{multline*}
((n-3)!)^2
\sum_{i=1}^{n-2}\sum_{j=1}^{n-1-i}
\sum_{i'=1}^{n-2}\sum_{j'=1}^{n-1-i'} \\
  \dotprod
   {\PDP{i}{j}\Bv}
   {\PDP{i'}{j'}\Bv}
 -\dotprod
   {\PDP{i}{j}\Bv}
   {(\PDP{i'}{j'})^T\Bv} \\
 +\dotprod
   {(\PDP{i}{j})^T\Bv}
   {(\PDP{i'}{j'})^T\Bv}
 -\dotprod
   {(\PDP{i}{j})^T\Bv}
   {\PDP{i'}{j'}\Bv}
\end{multline*}

There are a couple further manipulations that can be done, since
\(\dotprod{\Ba}{\Bb} = \Ba^T\Bv\), \(\BP^T = \BP^{-1}\), and
\(\BP^{-i} = \BP^{n-i}\).

\begin{multline*}
((n-3)!)^2
\sum_{i=1}^{n-2}\sum_{j=1}^{n-1-i}
\sum_{i'=1}^{n-2}\sum_{j'=1}^{n-1-i'} \\
   \PDPDP{-j}{i'-i}{j'}
  -\PDPDP{-j}{-i-j'}{-i'} \\
  +\PDPDP{i}{j-j'}{-i'}
  -\PDPDP{i}{j+i'}{j'}
\end{multline*}

Well, after all this, I am not actually any closer to getting \(\norm{\crossprod{\Bu}{\Bv}}^2\) evaluated.
Perhaps there is an easier way.  This matrix formulation for \(\crossop{\Bu}\) is a nice way for expression
but it has turned out to be a bit awkward for manipulation, at least without a way to expand \(\PDP{i}{j}\),
which I have not tried for the general case.

\section{Appendix 1}
\subsection{Change of basis, transformations, and rotations}

Given an orthogonal basis \((\ucap_i)_i\) in one coordinate system and an
orthogonal basis \(({\ucap_i}')_i\) for the same coordinate system, how are
the two related?

The two sets of unit vectors can be related by a set of linear equations

\begin{equation}\label{eqn:crossOld:520}
\begin{aligned}
{\ucap_i}' &= \sum_{s=1}^n{a_{is}\ucap_s} \\
\ucap_i &= \sum_{s=1}^n{b_{is}{\ucap_s}'}
\end{aligned}
\end{equation}

What the values of \(a_{ij}\) or \(b_{ij}\) are can be determined by taking inner products and by using the
orthogonality constraints.

\begin{equation}\label{eqn:crossOld:540}
\begin{aligned}
\innerprod{{\ucap_i}'}{\ucap_j} &= \sum_{s=1}^n{a_{is}\innerprod{\ucap_s}{\ucap_j}} \\
                                &= \sum_{s=1}^n{a_{is}\delta_{sj}} \\
                                &= a_{ij} \\
\innerprod{\ucap_i}{{\ucap_j}'} &= \sum_{s=1}^n{b_{is}\innerprod{{\ucap_s}'}{{\ucap_j}'}} \\
                                &= \sum_{s=1}^n{b_{is}\delta_{sj}} \\
                                &= b_{ij} \\
                                &= \overline{a_{ji}} \\
\end{aligned}
\end{equation}

So the relationships between the two sets of basis vectors \({\ucap_i}'\) and \(\ucap_i\) are

\begin{equation}\label{eqn:crossOld:560}
\begin{aligned}
{\ucap_i}'
&= \sum_{s=1}^n{
a_{is}
\ucap_s
}
&=
\sum_{s=1}^n{
\innerprod{{\ucap_i}'}{\ucap_s}
\ucap_s
}
\\
\ucap_i
&= \sum_{s=1}^n{
\overline{a_{si}}
{\ucap_s}'
}
&=
\sum_{s=1}^n{
\innerprod{\ucap_i}{{\ucap_s}'}
{\ucap_s}'
}
\end{aligned}
\end{equation}

Note that these two relationships can be expressed with a transformation
matrix \(\BM\) and its Hermitian transpose \(\BM^*\)

\begin{equation*}
\begin{bmatrix}
{\ucap_1}' \\
{\ucap_2}' \\
\vdots	  \\
{\ucap_n}'
\end{bmatrix}
=
\begin{bmatrix}
	a_{11} & a_{12} & \dots  & a_{1n} \\
        a_{21} & a_{22} & 	  &        \\
	\vdots &        & \ddots &        \\
	a_{n1} & \dots  &        & a_{nn}
\end{bmatrix}
\begin{bmatrix}
\ucap_1  \\
\ucap_2  \\
\vdots	  \\
\ucap_n
\end{bmatrix}
= \BM
\begin{bmatrix}
\ucap_1  \\
\ucap_2  \\
\vdots	  \\
\ucap_n
\end{bmatrix}
\end{equation*}

\begin{equation*}
\begin{bmatrix}
\ucap_1 \\
\ucap_2 \\
\vdots	  \\
\ucap_n
\end{bmatrix}
=
\begin{bmatrix}
	\overline{a_{11}} & \overline{a_{21}} & \dots  & \overline{a_{n1}} \\
        \overline{a_{12}} & \overline{a_{22}} & 	  &        \\
	\vdots &        & \ddots &        \\
	\overline{a_{1n}} & \dots  &        & \overline{a_{nn}}
\end{bmatrix}
\begin{bmatrix}
{\ucap_1}' \\
{\ucap_2}' \\
\vdots	  \\
{\ucap_n}'
\end{bmatrix}
= \BM^*
\begin{bmatrix}
{\ucap_1}' \\
{\ucap_2}' \\
\vdots	  \\
{\ucap_n}'
\end{bmatrix}
\end{equation*}

or

\begin{equation*}
\begin{bmatrix}
{\ucap_1}' \\
{\ucap_2}' \\
\vdots	  \\
{\ucap_n}'
\end{bmatrix}
=
\begin{bmatrix}
	\innerprod{{\ucap_1}'}{\ucap_1} & \innerprod{{\ucap_1}'}{\ucap_2} & \dots  & \innerprod{{\ucap_1}'}{\ucap_n} \\
        \innerprod{{\ucap_2}'}{\ucap_1} & \innerprod{{\ucap_2}'}{\ucap_2} & 	  &        \\
	\vdots &        & \ddots &        \\
	\innerprod{{\ucap_n}'}{\ucap_1} & \dots  &        & \innerprod{{\ucap_n}'}{\ucap_n}
\end{bmatrix}
\begin{bmatrix}
\ucap_1  \\
\ucap_2  \\
\vdots	  \\
\ucap_n
\end{bmatrix}
\end{equation*}

Given an arbitrary vector \(\Br = [r_j]_j\) in the primary coordinate system, one
can express this vector \(\Br' = [r_j']_j\) in the secondary coordinate system using
the same sort procedure used to derive the transformation matrix \(\BM\).

\begin{equation}\label{eqn:crossOld:580}
\begin{aligned}
\Br' &=
      \sum_{s=1}^n
      {
       r_s
       \ucap_s
      } \\
      &=
      \sum_{s=1}^n
      {
       r_s
\sum_{t=1}^n
{
\overline{a_{ts}}
{\ucap_t}'
}
      } \\
      &=
\sum_{t=1}^n
      {
{\ucap_t}'
      \sum_{s=1}^n
{
\overline{a_{ts}}
       r_s
}
      } \\
      &=
\sum_{t=1}^n
      {
{\ucap_t}'
r_t'
      }
\end{aligned}
\end{equation}

Since $r_i' =
      \sum_{s=1}^n
{
\overline{a_{is}}
       r_s
}
$
one can see that the components of the vectors transform
in a similar fashion the
basis vectors, and this can be written \(\Br = \BM^T \Br'\) and \(\Br' = \overline{\BM} \Br\).

When deriving this this result seemed odd at first, and found myself wondering if have I messed up despite the fact everything looked okay?  On paper I had only derived this case for \R{n} and not \C{n}.\footnote{
A worked example showed that transformation of the coordinate vectors and the basis vectors do differ by a complex conjugate factor.

Setting \({\ucap_1}' = \inv{\sqrt{2}}(1,i), {\ucap_2}'=\inv{\sqrt{2}}(1,-i)\),
\(\ucap_i = \ecap_i\) the unit vectors in \R{2}, then
$\BM =
\inv{\sqrt{2}}
\Bigl[
\begin{smallmatrix}
1 & i \\
1 & -i
\end{smallmatrix}
\Bigr]
$.  Picking an arbitrary test vector
\(\Br = (1,1) = \ucap_1 + \ucap_2 = \inv{\sqrt{2}}((1-i){\ucap_1}' + (1+i){\ucap_2}')\) the application of the
transformation formulas shows $\Br = \BM^T \Br' =
\inv{\sqrt{2}}
\Bigl[
\begin{smallmatrix}
1 & 1 \\
i & -i
\end{smallmatrix}
\Bigr]
\inv{\sqrt{2}}
\Bigl[
\begin{smallmatrix}
1 - i \\
1 + i
\end{smallmatrix}
\Bigr]
=
\Bigl[
\begin{smallmatrix}
1 \\
1
\end{smallmatrix}
\Bigr]
$ as expected.
}

It does not matter too much, because I do not need the result for the general case in the torque examination anyhow.

%\end{document}               % End of document.


\part{Maxwell}
\documentclass{article}      % Specifies the document class

\usepackage{amsmath}
\newcommand{\abs}[1]{\lvert#1\rvert}
\newcommand{\norm}[1]{\lVert#1\rVert}
\newcommand{\grad}[1]{\nabla#1}
\newcommand{\curl}[1]{\nabla \times #1}
\newcommand{\Curl}[1]{\nabla \times \mathbf{#1}}
\newcommand{\diverg}[1]{\nabla \cdot #1}
\newcommand{\Diverg}[1]{\nabla \cdot \mathbf{#1}}
\newcommand{\curlcurl}[1]{\curl\curl{#1}}
\newcommand{\curlCurl}[1]{\curl\Curl{#1}}
\newcommand{\delsquared}[1]{\nabla^2{#1}}
\newcommand{\Delsquared}[1]{{\nabla^2}{\mathbf{#1}}}
\newcommand{\ddt}[1]{ {{\partial{#1}} \over {\partial{t}}}}
\newcommand{\Ddt}[1]{ {{\partial{\mathbf{#1}}} \over {\partial{t}}}}
\newcommand{\ddts}[1]{ {{\partial^2{#1}} \over {\partial{t}^2}}}
\newcommand{\Ddts}[1]{ {{\partial^2{\mathbf{#1}}} \over {\partial{t}^2}}}
\newcommand{\Bj}[0]{\mathbf{j}}
\newcommand{\BB}[0]{\mathbf{B}}
\newcommand{\BE}[0]{\mathbf{E}}

                             % The preamble begins here.
\title{An Example Document}  % Declares the document's title.
\author{Peeter Joot}         % Declares the author's name.
\date{March 25, 2000}        % Deleting this command produces today's date.

\begin{document}             % End of preamble and beginning of text.

%\maketitle                  % Produces the title.

\section{various formulations of Maxwell's equations}

It is interesting to look at the various formulations of Maxwell's 
equations.  The standard formulation these days is the differential
form, which isn't the most intuitive form.  In cgs units, the differential
form is as follows:

\begin{equation*}
\Diverg E = 4\pi\rho
\end{equation*}
\begin{equation*}
\Diverg B = 0
\end{equation*}
\begin{equation*}
\Curl{B} = 4\pi \Bj - {1 \over c} \Ddt{E}
\end{equation*}
\begin{equation*}
\Curl{E} = {1 \over c} \Ddt{B}
\end{equation*}

It is interesting to note that these are probably not the form that Maxwell 
originally formulated his equations in.  In my 1960 version of encyclodedia 
britianica Maxwell's equations were given in a differential form.  I had 
trouble seeing how these two forms were related at first, but the relation
is via the standard integral transformations.  For example, the formula $\Diverg E = 4\pi\rho$ is an alternate formulation of Gauss' law
\begin{equation*}
\BE = \int_V{{\rho\,dV' \over \norm{\mathbf r - \mathbf r'}}}
\end{equation*}

This can be shown via application of Gauss's theorm.  Similarily the following 
integral forms of Maxwell's equations:

%[ see paper notes ].

One of the awkward things with this standard formulation is that there 
is an awkward interdependence between the electric and magnetic fields, and 
the solution of either $\BE$ or $\BB$ interdependently seems difficult.

The interdependence of the $\BE$ and $\BB$ field equations can be removed by taking the
curl of each of the curl equations, and using the fact that:

$\curlCurl{V} = \grad (\Diverg V) - \Delsquared{V}$

\begin{align*}
\curlCurl{B} &= \grad (\Diverg{B}) - \Delsquared{B} \\
\curl(4\pi\Bj - {1 \over c} \Ddt{E}) &= \\
4\pi \Curl{j} - {1 \over c^2} \Ddts{B} &= \\
      	     &= 		   - \Delsquared{B}
\end{align*}

We end up with a single differential equation for the magnetic field $\BB$.

$\delsquared{\BB} - {1 \over c^2} \Ddts{B} = - 4\pi \Curl{j}$

We can do the same calculation for the electric field.

\begin{align*}
\curlCurl{E} &= \grad (\Diverg{E}) - \Delsquared{E} \\
\curl({1 \over c} \Ddt{B}) &= \\
{1 \over c} \ddt(4\pi \Bj - {1 \over c} \Ddt{E}) &= \\
{4\pi \over c} \Ddt{j} - {1 \over c^2} \Ddts{E} &= \\
             &= \grad (4\pi\rho) - \Delsquared{E} \\
\end{align*}

which gives a single differential equation for the electric field $\BE$.

%$\Delsquared{E} - {1 \over c^2} \Ddt{E} = {4\pi \over c} \Ddt{j} - 4\pi \grad{\rho}$
$\Delsquared{E} - {1 \over c^2} \Ddt{E}$
$ = {4\pi \over c} \Ddt{j} - 4\pi \grad{\rho}$

In the absense of current density and charge these equations take the simple form of standard wave
equations.

\begin{equation*}
\Delsquared{E} - {1 \over c^2} \Ddt{E} = 0
\end{equation*}
\begin{equation*}
\Delsquared{B} - {1 \over c^2} \Ddt{B} = 0
\end{equation*}

Any solution to these is also a solution to the original form where there is current density $\Bj$, and 
charge density $\rho$ since 

\end{document}               % End of document.

% 
% 
% 
% Copyright � 2012 Peeter Joot
% All Rights Reserved
% 
% This file may be reproduced and distributed in whole or in part, without fee, subject to the following conditions:
% 
% o The copyright notice above and this permission notice must be preserved complete on all complete or partial copies.
% 
% o Any translation or derived work must be approved by the author in writing before distribution.
% 
% o If you distribute this work in part, instructions for obtaining the complete version of this file must be included, and a means for obtaining a complete version provided.
% 
% 
% Exceptions to these rules may be granted for academic purposes: Write to the author and ask.
% 
% 
% 
%\documentclass[]{eliblog}

\usepackage{amsmath}
\usepackage{mathpazo}

%
% shorthand for bold symbols, convenient for vectors and matrices
%
\newcommand{\Ba}[0]{\mathbf{a}}
\newcommand{\Bb}[0]{\mathbf{b}}
\newcommand{\Bc}[0]{\mathbf{c}}
\newcommand{\Bd}[0]{\mathbf{d}}
\newcommand{\Be}[0]{\mathbf{e}}
\newcommand{\Bf}[0]{\mathbf{f}}
\newcommand{\Bg}[0]{\mathbf{g}}
\newcommand{\Bh}[0]{\mathbf{h}}
\newcommand{\Bi}[0]{\mathbf{i}}
\newcommand{\Bj}[0]{\mathbf{j}}
\newcommand{\Bk}[0]{\mathbf{k}}
\newcommand{\Bl}[0]{\mathbf{l}}
\newcommand{\Bm}[0]{\mathbf{m}}
\newcommand{\Bn}[0]{\mathbf{n}}
\newcommand{\Bo}[0]{\mathbf{o}}
\newcommand{\Bp}[0]{\mathbf{p}}
\newcommand{\Bq}[0]{\mathbf{q}}
\newcommand{\Br}[0]{\mathbf{r}}
\newcommand{\Bs}[0]{\mathbf{s}}
\newcommand{\Bt}[0]{\mathbf{t}}
\newcommand{\Bu}[0]{\mathbf{u}}
\newcommand{\Bv}[0]{\mathbf{v}}
\newcommand{\Bw}[0]{\mathbf{w}}
\newcommand{\Bx}[0]{\mathbf{x}}
\newcommand{\By}[0]{\mathbf{y}}
\newcommand{\Bz}[0]{\mathbf{z}}
\newcommand{\BA}[0]{\mathbf{A}}
\newcommand{\BB}[0]{\mathbf{B}}
\newcommand{\BC}[0]{\mathbf{C}}
\newcommand{\BD}[0]{\mathbf{D}}
\newcommand{\BE}[0]{\mathbf{E}}
\newcommand{\BF}[0]{\mathbf{F}}
\newcommand{\BG}[0]{\mathbf{G}}
\newcommand{\BH}[0]{\mathbf{H}}
\newcommand{\BI}[0]{\mathbf{I}}
\newcommand{\BJ}[0]{\mathbf{J}}
\newcommand{\BK}[0]{\mathbf{K}}
\newcommand{\BL}[0]{\mathbf{L}}
\newcommand{\BM}[0]{\mathbf{M}}
\newcommand{\BN}[0]{\mathbf{N}}
\newcommand{\BO}[0]{\mathbf{O}}
\newcommand{\BP}[0]{\mathbf{P}}
\newcommand{\BQ}[0]{\mathbf{Q}}
\newcommand{\BR}[0]{\mathbf{R}}
\newcommand{\BS}[0]{\mathbf{S}}
\newcommand{\BT}[0]{\mathbf{T}}
\newcommand{\BU}[0]{\mathbf{U}}
\newcommand{\BV}[0]{\mathbf{V}}
\newcommand{\BW}[0]{\mathbf{W}}
\newcommand{\BX}[0]{\mathbf{X}}
\newcommand{\BY}[0]{\mathbf{Y}}
\newcommand{\BZ}[0]{\mathbf{Z}}

\newcommand{\Bzero}[0]{\mathbf{0}}
\newcommand{\Btheta}[0]{\boldsymbol{\theta}}
\newcommand{\Btau}[0]{\boldsymbol{\tau}}
\newcommand{\Bomega}[0]{\boldsymbol{\omega}}

%
% shorthand for unit vectors
%
\newcommand{\acap}[0]{\hat{\Ba}}
\newcommand{\bcap}[0]{\hat{\Bb}}
\newcommand{\ccap}[0]{\hat{\Bc}}
\newcommand{\dcap}[0]{\hat{\Bd}}
\newcommand{\ecap}[0]{\hat{\Be}}
\newcommand{\fcap}[0]{\hat{\Bf}}
\newcommand{\gcap}[0]{\hat{\Bg}}
\newcommand{\hcap}[0]{\hat{\Bh}}
\newcommand{\icap}[0]{\hat{\Bi}}
\newcommand{\jcap}[0]{\hat{\Bj}}
\newcommand{\kcap}[0]{\hat{\Bk}}
\newcommand{\lcap}[0]{\hat{\Bl}}
\newcommand{\mcap}[0]{\hat{\Bm}}
\newcommand{\ncap}[0]{\hat{\Bn}}
\newcommand{\ocap}[0]{\hat{\Bo}}
\newcommand{\pcap}[0]{\hat{\Bp}}
\newcommand{\qcap}[0]{\hat{\Bq}}
\newcommand{\rcap}[0]{\hat{\Br}}
\newcommand{\scap}[0]{\hat{\Bs}}
\newcommand{\tcap}[0]{\hat{\Bt}}
\newcommand{\ucap}[0]{\hat{\Bu}}
\newcommand{\vcap}[0]{\hat{\Bv}}
\newcommand{\wcap}[0]{\hat{\Bw}}
\newcommand{\xcap}[0]{\hat{\Bx}}
\newcommand{\ycap}[0]{\hat{\By}}
\newcommand{\zcap}[0]{\hat{\Bz}}
\newcommand{\thetacap}[0]{\hat{\Btheta}}

%
% to write R^n and C^n in a distinguishable fashion.  Perhaps change this
% to the double lined characters upon figuring out how to do so.
%
\newcommand{\C}[1]{$\mathbb{C}^{#1}$}
\newcommand{\R}[1]{$\mathbb{R}^{#1}$}

%
% various generally useful helpers
%

% derivative of #1 wrt. #2:
\newcommand{\D}[2] {\frac {d#2} {d#1}}

\newcommand{\inv}[1]{\frac{1}{#1}}
\newcommand{\cross}[0]{\times}

\newcommand{\abs}[1]{\lvert{#1}\rvert}
\newcommand{\norm}[1]{\lVert{#1}\rVert}
\newcommand{\innerprod}[2]{\langle{#1}, {#2}\rangle}
\newcommand{\dotprod}[2]{{#1} \cdot {#2}}
\newcommand{\bdotprod}[2]{\left({#1} \cdot {#2}\right)}
\newcommand{\crossprod}[2]{{#1} \cross {#2}}
\newcommand{\tripleprod}[3]{\dotprod{\left(\crossprod{#1}{#2}\right)}{#3}}

\DeclareMathOperator{\Proj}{Proj}
\DeclareMathOperator{\Span}{span}
\DeclareMathOperator{\Sgn}{sgn}
\DeclareMathOperator{\Area}{Area}
\DeclareMathOperator{\Volume}{Volume}

%
% A few miscellaneous things specific to this document
%
\newcommand{\crossop}[1]{\crossprod{#1}{}}

% R2 vector.
\newcommand{\VectorTwo}[2]{
\begin{bmatrix}
 {#1} \\
 {#2}
\end{bmatrix}
}

\newcommand{\VectorN}[1]{
\begin{bmatrix}
{#1}_1 \\
{#1}_2 \\
\vdots \\
{#1}_N \\
\end{bmatrix}
}

\newcommand{\DETuvij}[4]{
\begin{vmatrix}
 {#1}_{#3} & {#1}_{#4} \\
 {#2}_{#3} & {#2}_{#4}
\end{vmatrix}
}

\newcommand{\DETuvwijk}[6]{
\begin{vmatrix}
 {#1}_{#4} & {#1}_{#5} & {#1}_{#6} \\
 {#2}_{#4} & {#2}_{#5} & {#2}_{#6} \\
 {#3}_{#4} & {#3}_{#5} & {#3}_{#6}
\end{vmatrix}
}

\newcommand{\DETuvwxijkl}[8]{
\begin{vmatrix}
 {#1}_{#5} & {#1}_{#6} & {#1}_{#7} & {#1}_{#8} \\
 {#2}_{#5} & {#2}_{#6} & {#2}_{#7} & {#2}_{#8} \\
 {#3}_{#5} & {#3}_{#6} & {#3}_{#7} & {#3}_{#8} \\
 {#4}_{#5} & {#4}_{#6} & {#4}_{#7} & {#4}_{#8} \\
\end{vmatrix}
}

%\newcommand{\DETuvwxyijklm}[10]{
%\begin{vmatrix}
% {#1}_{#6} & {#1}_{#7} & {#1}_{#8} & {#1}_{#9} & {#1}_{#10} \\
% {#2}_{#6} & {#2}_{#7} & {#2}_{#8} & {#2}_{#9} & {#2}_{#10} \\
% {#3}_{#6} & {#3}_{#7} & {#3}_{#8} & {#3}_{#9} & {#3}_{#10} \\
% {#4}_{#6} & {#4}_{#7} & {#4}_{#8} & {#4}_{#9} & {#4}_{#10} \\
% {#5}_{#6} & {#5}_{#7} & {#5}_{#8} & {#5}_{#9} & {#5}_{#10}
%\end{vmatrix}
%}

% R3 vector.
\newcommand{\VectorThree}[3]{
\begin{bmatrix}
 {#1} \\
 {#2} \\
 {#3}
\end{bmatrix}
}



\author{Peeter Joot}
\email{peeter.joot@gmail.com}


\chapter{Reader notes for Jackson 12.11, Retarded time solution to the wave equation.}
\label{chap:jacksonRetarded}
%\useCCL
\blogpage{http://sites.google.com/site/peeterjoot/math2009/jacksonRetarded.pdf}
\date{Sept 19, 2009}
\revisionInfo{jacksonRetarded.tex }

%\beginArtWithToc
\beginArtNoToc

\section{Motivation}

In \citep{gabook:PJpoisson} I blundered my way towards the retarded time Green's function solution to the 3D wave equation.  Jackson's \citep{jackson1975cew} (section 12.11) covers this in a much more coherent fashion.  It is however somewhat terse, and some details that were not immediately obvious to me were omitted.

Here are my notes for this section in case I want to refer to it again later.

\section{Guts}

The starting point is the electrodynamic wave equation

\begin{align}\label{eqn:jacksonRet:boo1}
\partial_\alpha F^{\alpha\beta} = \frac{4 \pi}{c} J^\beta
\end{align}

A substitution of $F^{\alpha \beta} = \partial^\alpha A^\beta - \partial^\beta A^\alpha$ gives us

\begin{align}\label{eqn:jacksonRet:boo2}
\partial_\alpha F^{\alpha\beta} = \partial_\alpha \partial^\alpha A^\beta - \partial_\alpha \partial^\beta A^\alpha
= \square A^\beta - \partial^\beta  (\partial_\alpha A^\alpha)
\end{align}

Thus with the Lorentz condition $\partial_\alpha A^\alpha = 0$ we have 

\begin{align}\label{eqn:jacksonRet:boo3}
\square A^\beta = \frac{4 \pi}{c} J^\beta
\end{align}

A set of four non-homogeneous wave equations to solve.  It is assumed that a Green's function of the form

\begin{align}\label{eqn:jacksonRet:boo4}
\square_x D(x - x') = \delta^4(x - x')
\end{align}

can be found.  Jackson states that this is possible in the absence of boundary surfaces, which seems to imply that the more general case would require $\square_x D(x, x') = \delta^4(x - x')$, where $D$ is not necessarily a function of the four vector difference $x - x'$.

What is really meant by this Green's function?  It only takes meaning in the context of the convolution integral.  Namely

\begin{align}\label{eqn:jacksonRet:boo5}
A^\beta = \int d^4 x' D(x, x') \frac{4 \pi}{c} J^\beta(x') 
\end{align}

So that
\begin{align*}
\square_x A^\beta 
&= \int d^4 x' \square_x D(x, x') \frac{4 \pi}{c} J^\beta(x') \\
&= \frac{4 \pi}{c} \int d^4 x' \delta^4(x - x') J^\beta(x') \\
&= \frac{4 \pi}{c} J^\beta(x) \\
\end{align*}

So if a function with this delta filtering property under the DeLambertian can be found we can find the non-homogeneous solutions directly by four-volume convolution.

It is implied in the text (probably stated explicitly somewhere earlier) that the asymmetric convention for the Fourier transform pairs is being used

\begin{align}\label{eqn:jacksonRet:boo6}
\tilde{f}(k) &= \int d^4 z f(z) e^{i k \cdot z} \\
f(z) &= \inv{(2\pi)^4} \int d^4 k \tilde{f}(k) e^{-i k \cdot z} 
\end{align}

where $d^4 k = dk_0 dk_1 dk_2 dk_3$, and $d^4 z = dz^0 dz^1 dz^2 dz^3$, and $k \cdot z = k_\mu z^\mu = k^\mu z_\mu$.

Assuming the validity of this transform pair, even for the delta distribution, we can find an integral representation of the delta using the transform pairs.  For the Fourier transform of delta we have

\begin{align*}
\tilde{\delta^4}(k) 
&= \int d^4 z \delta^4(z) e^{i k \cdot z} \\
&= e^{i k \cdot 0} \\
&= 1
\end{align*}

Performing the inverse transformation provides the delta function exponential integral representation 

\begin{align*}
\delta^4(z) 
&= \inv{(2\pi)^4} \int d^4 k \tilde{\delta^4}(k) e^{-i k \cdot z} \\
&= \inv{(2\pi)^4} \int d^4 k e^{-i k \cdot z} \\
\end{align*}

Just as a Fourier representation of the delta can be found, we can integrate by parts to find an integral representation of the Green's function that we seek.  Taking Fourier transforms

\begin{align*}
\FF(\square_x D(z))(k) 
&= \int d^4 z \partial_\alpha \partial^\alpha D(z) e^{i k \cdot z} \\
&= -\int d^4 z \partial^\alpha D(z) \partial_\alpha e^{i k_\beta z^\beta} \\
&= -\int d^4 z \partial^\alpha D(z) i k_\alpha e^{i k_\beta z^\beta} \\
&= \int d^4 z D(z) i k_\alpha \partial^\alpha e^{i k^\beta z_\beta} \\
&= -\int d^4 z D(z) k_\alpha k^\alpha e^{i k \cdot z } \\
&= - k^2 \tilde{D}(k)
\end{align*}

Using the assumed delta function property of this Green's function we also have

\begin{align*}
\FF(\square_x D(z))(k) 
&= \int d^4 z \delta^4(z) e^{i k \cdot z} \\
&= 1
\end{align*}

This completely specifies the Fourier transform of the Green's function
\begin{align}\label{eqn:jacksonRet:boo7}
\tilde{D}(k) &= - \inv{k^2}
\end{align}

and we can inverse transform to complete the task of finding an initial representation of the Green's function itself.  That is

\begin{align}\label{eqn:jacksonRet:boo8}
D(z) = -\inv{(2\pi)^4} \int d^4 k \inv{k^2} e^{-i k \cdot z} 
\end{align}

With an explicit spacetime split we have our integral prepped for the contour integration

\begin{align}\label{eqn:jacksonRet:boo9}
D(z) = -\inv{(2\pi)^4} \int d^3 k e^{i \Bk \cdot \Bz} \int_{-\infty}^\infty dk_0 \inv{k_0^2 - \Bk^2} e^{-i k_0 z_0} 
\end{align}

Here $\kappa = \Abs{\Bk}$ is used as in the text.  If we let $k_0 = R e^{i\theta}$ take on complex values, integrating over a semicircular arc, we have for the exponential 

\begin{align*}
\Abs{e^{-i k_0 z_0}}
&= \Abs{e^{-i R (\cos\theta + i \sin\theta) z_0} } \\
&= \Abs{e^{ z_0 R \sin\theta} e^{ -i z_0 R \cos\theta } } \\
&= \Abs{e^{ z_0 R \sin\theta} }
\end{align*}

In the upper half plane $\theta \in [0,\pi]$, so $\sin\theta$ is never negative, and the integral on an upper half plane semi-circular contour can only vanish as desired for $z_0 < 0$.  Similarly the infinite arc integral can only be zero for $z_0 > 0$ for a lower half plane contour.  This is mentioned in the text but I felt it more clear just writing out the exponential as above.

\begin{figure}[htp]
\centering
\includegraphics[totalheight=0.4\textheight]{retardedContourBoth}
\caption{Contours strictly above the $k_0 = 0$ axis}\label{fig:jacksonRet:retardedContourBoth}
\end{figure}

\begin{figure}[htp]
\centering
\includegraphics[totalheight=0.4\textheight]{retardedContourAroundPole}
\caption{Contour around pole}\label{fig:jacksonRet:retardedContourAroundPole}
\end{figure}

Having established the value on the loop at infinity we can now integrate over the contour $r_1$ as depicted in figure (\ref{fig:jacksonRet:retardedContourBoth}).  The problem is mainly reduced to an integral of the form figure (\ref{fig:jacksonRet:retardedContourAroundPole}) around the simple poles at $\alpha = \pm \kappa$

\begin{align}\label{eqn:jacksonRet:boo10}
I_\alpha = \oint \frac{f(z)}{z - \alpha} dz
\end{align}

With $z = \alpha + R e^{i\theta}$, and $\theta \in [\pi/2, 5\pi/2]$, we have

\begin{align}\label{eqn:jacksonRet:boo11}
I_\alpha = \int \frac{f(z)}{R e^{i\theta}} R i e^{i\theta} d\theta
\end{align}

with $R \rightarrow 0$, we are left with 

\begin{align}\label{eqn:jacksonRet:boo12}
I_\alpha = 2 \pi i f(\alpha)
\end{align}

There are six arcs on the contour of interest.  For the first two around the poles lets label the integral contributions $I_\kappa$ and $I_{-\kappa}$.  Along the infinite semicircular contour the integral vanishes with the right sign choice for $z_0$.  For the remainder lets write the integral contributions $I$.

Summing over the complete contour, specially chosen to enclose no poles, we have

\begin{align}\label{eqn:jacksonRet:boo13}
I + I_\kappa + I_{-\kappa} + 0 = 0 
\end{align}

For this $z_0 > 0$ integral we are left with the residue sum

\begin{align*}
\int_{-\infty}^\infty dk_0 \inv{k_0^2 - \Bk^2} e^{-i k_0 z_0} 
&= 
- 2 \pi i \left( 
{\left. \inv{k_0 - \kappa} e^{-i k_0 z_0} \right\vert}_{k_0 = -\kappa}
+{\left. \inv{k_0 + \kappa} e^{-i k_0 z_0} \right\vert}_{k_0 = \kappa}
\right) \\
&= 
\frac{2 \pi i^2}{\kappa} \sin(\kappa z_0)
\end{align*}

Since I can never remember the signs and integral orientations for the residue formula so I've always done it ``manually'' as above picking a zero valued contour.

\begin{figure}[htp]
\centering
\includegraphics[totalheight=0.4\textheight]{retardedContourOnAxis}
\caption{Contour exactly on the $k_0 = 0$ axis?}\label{fig:jacksonRet:retardedContourOnAxis}
\end{figure}

Now, the issue of where to place the contour wasn't really discussed mathematically.  Physically this makes the difference between causal and acausal behavior, but why put the contour strictly above or below the axis and not right on it.  If we put the contour exactly on the $k_0 = 0$ axis as in (\ref{fig:jacksonRet:retardedContourOnAxis}), then our integrals around the two half circular poles gives us a result off by a factor of two?  There is also an (implied) limiting procedure required to place the contour strictly above the axis, and the details of this aren't mentioned (and I also haven't thought them through).  Some of these would be worth thinking through in more detail, but for now lets ignore these.  We are left with

\begin{align}\label{eqn:jacksonRet:boo14}
D(z) = \frac{\theta(z_0)}{(2\pi)^3} \int d^3 k e^{i \Bk \cdot \Bz} \inv{\kappa} \sin(\kappa z_0)
\end{align}

How to reduce this to the single variable integral in $\kappa$ was not immediately clear to me.  Aligning $\Bz$ with the $\Be_3$ axis, and using a spherical polar representation for $\Bk$ we can write $\Bz \cdot \Bk = R \kappa \cos\theta$.  With this and the volume element $d^3 k = \kappa^2 \sin\theta d\theta d\phi d\kappa$, we have

\begin{align}\label{eqn:jacksonRet:boo15}
D(z) = \frac{\theta(z_0)}{(2\pi)^3} \int_0^\infty d\kappa \sin(\kappa z_0) \int_0^{2\pi} d\phi \int_0^\pi d\theta \kappa e^{i R \kappa \cos\theta} \sin\theta
\end{align}

This now happily submits to a nice variable substitution, unlike an integral like $\int e^{i \mu \cos\theta} d\theta = J_0(\Abs{\mu})$ which can be evaluated, but only in terms of Bessel functions or messy series expansion.  Writing $\tau = \kappa \cos\theta$, and $-d\tau = \kappa \sin\theta d\theta$ we have

\begin{align*}
\int_0^\pi d\theta \kappa e^{i R \kappa \cos\theta} \sin\theta
&=
-\int_{\kappa}^{-\kappa} d\tau e^{i R \tau} \\
&=
\frac{e^{i R \kappa}}{i R} -\frac{e^{-i R \kappa}}{i R} \\
&=
2 \inv{R} \sin(R \kappa)
\end{align*}

Our Green's function is now reduced to

\begin{align}\label{eqn:jacksonRet:boo16}
D(z) = \frac{\theta(z_0)}{2 \pi^2 R} \int_0^\infty d\kappa \sin(\kappa z_0) \sin(\kappa R)
\end{align}

Expanding out these sines in terms of exponentials we have

\begin{align*}
D(z) 
&= -\frac{\theta(z_0)}{8 \pi^2 R} \int_0^\infty d\kappa ( e^{i\kappa(z_0+R)} + e^{-i\kappa(z_0+R)} -e^{i\kappa(R-z_0)} - e^{i\kappa(z_0-R)} ) \\
&= -\frac{\theta(z_0)}{8 \pi^2 R} \left(
\int_0^\infty d\kappa \left( e^{i\kappa(z_0+R)} -e^{i\kappa(R-z_0)} \right) 
+\int_0^{-\infty} -d\kappa \left( e^{i\kappa(z_0+R)} - e^{-i\kappa(z_0-R)} \right) 
\right)
\\
&= \frac{\theta(z_0)}{8 \pi^2 R} \int_{-\infty}^\infty d\kappa \left( e^{i\kappa(R-z_0)} -e^{i\kappa(z_0+R)} \right) 
\\
\end{align*}

the sign in this first exponential differs from what Jackson obtained but it won't change the end result.  Did I make a mistake or did he?  Wonder what the third edition shows?  Using $\delta(x) = \int e^{-ikx} dk/2\pi$ we have

\begin{align}\label{eqn:jacksonRet:boo17}
D(z) = \frac{\theta(z_0)}{4 \pi R} \left( \delta(z_0 -R) - \delta(-(z_0 + R)) \right)
\end{align}

With $R = \Abs{\Bx - \Bx'} \ge 0$, and $z_0 = c(t - t') > 0$, this second delta cannot contribute, and we are left with the retarded Green's function

\begin{align}\label{eqn:jacksonRet:boo18}
D_r(z) = \frac{\theta(z_0)}{4 \pi R} \delta(c(t -t') - \Abs{\Bx - \Bx'}) 
\end{align}

Very slick.  I like the procedure, despite a few magic steps (like the choice to offset the contour).

\EndArticle
%\EndNoBibArticle


\part{Mechanics}
\documentclass{article}      % Specifies the document class

\usepackage{amsmath}
\usepackage{mathpazo}

%
% shorthand for bold symbols, convenient for vectors and matrices
%
\newcommand{\Ba}[0]{\mathbf{a}}
\newcommand{\Bb}[0]{\mathbf{b}}
\newcommand{\Bc}[0]{\mathbf{c}}
\newcommand{\Bd}[0]{\mathbf{d}}
\newcommand{\Be}[0]{\mathbf{e}}
\newcommand{\Bf}[0]{\mathbf{f}}
\newcommand{\Bg}[0]{\mathbf{g}}
\newcommand{\Bh}[0]{\mathbf{h}}
\newcommand{\Bi}[0]{\mathbf{i}}
\newcommand{\Bj}[0]{\mathbf{j}}
\newcommand{\Bk}[0]{\mathbf{k}}
\newcommand{\Bl}[0]{\mathbf{l}}
\newcommand{\Bm}[0]{\mathbf{m}}
\newcommand{\Bn}[0]{\mathbf{n}}
\newcommand{\Bo}[0]{\mathbf{o}}
\newcommand{\Bp}[0]{\mathbf{p}}
\newcommand{\Bq}[0]{\mathbf{q}}
\newcommand{\Br}[0]{\mathbf{r}}
\newcommand{\Bs}[0]{\mathbf{s}}
\newcommand{\Bt}[0]{\mathbf{t}}
\newcommand{\Bu}[0]{\mathbf{u}}
\newcommand{\Bv}[0]{\mathbf{v}}
\newcommand{\Bw}[0]{\mathbf{w}}
\newcommand{\Bx}[0]{\mathbf{x}}
\newcommand{\By}[0]{\mathbf{y}}
\newcommand{\Bz}[0]{\mathbf{z}}
\newcommand{\BA}[0]{\mathbf{A}}
\newcommand{\BB}[0]{\mathbf{B}}
\newcommand{\BC}[0]{\mathbf{C}}
\newcommand{\BD}[0]{\mathbf{D}}
\newcommand{\BE}[0]{\mathbf{E}}
\newcommand{\BF}[0]{\mathbf{F}}
\newcommand{\BG}[0]{\mathbf{G}}
\newcommand{\BH}[0]{\mathbf{H}}
\newcommand{\BI}[0]{\mathbf{I}}
\newcommand{\BJ}[0]{\mathbf{J}}
\newcommand{\BK}[0]{\mathbf{K}}
\newcommand{\BL}[0]{\mathbf{L}}
\newcommand{\BM}[0]{\mathbf{M}}
\newcommand{\BN}[0]{\mathbf{N}}
\newcommand{\BO}[0]{\mathbf{O}}
\newcommand{\BP}[0]{\mathbf{P}}
\newcommand{\BQ}[0]{\mathbf{Q}}
\newcommand{\BR}[0]{\mathbf{R}}
\newcommand{\BS}[0]{\mathbf{S}}
\newcommand{\BT}[0]{\mathbf{T}}
\newcommand{\BU}[0]{\mathbf{U}}
\newcommand{\BV}[0]{\mathbf{V}}
\newcommand{\BW}[0]{\mathbf{W}}
\newcommand{\BX}[0]{\mathbf{X}}
\newcommand{\BY}[0]{\mathbf{Y}}
\newcommand{\BZ}[0]{\mathbf{Z}}

\newcommand{\Bzero}[0]{\mathbf{0}}
\newcommand{\Btheta}[0]{\boldsymbol{\theta}}
\newcommand{\Btau}[0]{\boldsymbol{\tau}}
\newcommand{\Bomega}[0]{\boldsymbol{\omega}}

%
% shorthand for unit vectors
%
\newcommand{\acap}[0]{\hat{\Ba}}
\newcommand{\bcap}[0]{\hat{\Bb}}
\newcommand{\ccap}[0]{\hat{\Bc}}
\newcommand{\dcap}[0]{\hat{\Bd}}
\newcommand{\ecap}[0]{\hat{\Be}}
\newcommand{\fcap}[0]{\hat{\Bf}}
\newcommand{\gcap}[0]{\hat{\Bg}}
\newcommand{\hcap}[0]{\hat{\Bh}}
\newcommand{\icap}[0]{\hat{\Bi}}
\newcommand{\jcap}[0]{\hat{\Bj}}
\newcommand{\kcap}[0]{\hat{\Bk}}
\newcommand{\lcap}[0]{\hat{\Bl}}
\newcommand{\mcap}[0]{\hat{\Bm}}
\newcommand{\ncap}[0]{\hat{\Bn}}
\newcommand{\ocap}[0]{\hat{\Bo}}
\newcommand{\pcap}[0]{\hat{\Bp}}
\newcommand{\qcap}[0]{\hat{\Bq}}
\newcommand{\rcap}[0]{\hat{\Br}}
\newcommand{\scap}[0]{\hat{\Bs}}
\newcommand{\tcap}[0]{\hat{\Bt}}
\newcommand{\ucap}[0]{\hat{\Bu}}
\newcommand{\vcap}[0]{\hat{\Bv}}
\newcommand{\wcap}[0]{\hat{\Bw}}
\newcommand{\xcap}[0]{\hat{\Bx}}
\newcommand{\ycap}[0]{\hat{\By}}
\newcommand{\zcap}[0]{\hat{\Bz}}
\newcommand{\thetacap}[0]{\hat{\Btheta}}

%
% to write R^n and C^n in a distinguishable fashion.  Perhaps change this
% to the double lined characters upon figuring out how to do so.
%
\newcommand{\C}[1]{$\mathbb{C}^{#1}$}
\newcommand{\R}[1]{$\mathbb{R}^{#1}$}

%
% various generally useful helpers
%

% derivative of #1 wrt. #2:
\newcommand{\D}[2] {\frac {d#2} {d#1}}

\newcommand{\inv}[1]{\frac{1}{#1}}
\newcommand{\cross}[0]{\times}

\newcommand{\abs}[1]{\lvert{#1}\rvert}
\newcommand{\norm}[1]{\lVert{#1}\rVert}
\newcommand{\innerprod}[2]{\langle{#1}, {#2}\rangle}
\newcommand{\dotprod}[2]{{#1} \cdot {#2}}
\newcommand{\bdotprod}[2]{\left({#1} \cdot {#2}\right)}
\newcommand{\crossprod}[2]{{#1} \cross {#2}}
\newcommand{\tripleprod}[3]{\dotprod{\left(\crossprod{#1}{#2}\right)}{#3}}

\DeclareMathOperator{\Proj}{Proj}
\DeclareMathOperator{\Span}{span}
\DeclareMathOperator{\Sgn}{sgn}
\DeclareMathOperator{\Area}{Area}
\DeclareMathOperator{\Volume}{Volume}

%
% A few miscellaneous things specific to this document
%
\newcommand{\crossop}[1]{\crossprod{#1}{}}

% R2 vector.
\newcommand{\VectorTwo}[2]{
\begin{bmatrix}
 {#1} \\
 {#2}
\end{bmatrix}
}

\newcommand{\VectorN}[1]{
\begin{bmatrix}
{#1}_1 \\
{#1}_2 \\
\vdots \\
{#1}_N \\
\end{bmatrix}
}

\newcommand{\DETuvij}[4]{
\begin{vmatrix}
 {#1}_{#3} & {#1}_{#4} \\
 {#2}_{#3} & {#2}_{#4}
\end{vmatrix}
}

\newcommand{\DETuvwijk}[6]{
\begin{vmatrix}
 {#1}_{#4} & {#1}_{#5} & {#1}_{#6} \\
 {#2}_{#4} & {#2}_{#5} & {#2}_{#6} \\
 {#3}_{#4} & {#3}_{#5} & {#3}_{#6}
\end{vmatrix}
}

\newcommand{\DETuvwxijkl}[8]{
\begin{vmatrix}
 {#1}_{#5} & {#1}_{#6} & {#1}_{#7} & {#1}_{#8} \\
 {#2}_{#5} & {#2}_{#6} & {#2}_{#7} & {#2}_{#8} \\
 {#3}_{#5} & {#3}_{#6} & {#3}_{#7} & {#3}_{#8} \\
 {#4}_{#5} & {#4}_{#6} & {#4}_{#7} & {#4}_{#8} \\
\end{vmatrix}
}

%\newcommand{\DETuvwxyijklm}[10]{
%\begin{vmatrix}
% {#1}_{#6} & {#1}_{#7} & {#1}_{#8} & {#1}_{#9} & {#1}_{#10} \\
% {#2}_{#6} & {#2}_{#7} & {#2}_{#8} & {#2}_{#9} & {#2}_{#10} \\
% {#3}_{#6} & {#3}_{#7} & {#3}_{#8} & {#3}_{#9} & {#3}_{#10} \\
% {#4}_{#6} & {#4}_{#7} & {#4}_{#8} & {#4}_{#9} & {#4}_{#10} \\
% {#5}_{#6} & {#5}_{#7} & {#5}_{#8} & {#5}_{#9} & {#5}_{#10}
%\end{vmatrix}
%}

% R3 vector.
\newcommand{\VectorThree}[3]{
\begin{bmatrix}
 {#1} \\
 {#2} \\
 {#3}
\end{bmatrix}
}


\newcommand{\grad}[0]{\nabla}

%
% The real thing:
%

\title{ Potential and Kinetic Energy.} 
\author{Peeter Joot}         
\date{}

\begin{document}             

\maketitle{}

\section{}

Attempting some Lagrangian calculation problems I found I got all the signs of my potential energy terms wrong.  Here's a quick step back to basics to clarify for myself what the definition of potential energy is, and thus implicitly determine the correct signs.

Starting with kinetic energy, expressed in vector form:

\begin{equation*}
K 
= \inv{2} m \Br' \cdot \Br' 
= \inv{2} \Bp \cdot \Br',
\end{equation*}

one can calculate the rate of change of that energy:

\begin{align*}
\frac{dK }{dt}
&= \inv{2} \left(\Bp' \cdot \Br' + \Bp \cdot \Br''\right) \\
&= \inv{2} \left(\Bp' \cdot \Br' + \Br' \cdot \Bp'\right) \\
&= \Bp' \cdot \Br'.
\end{align*}

Note that the mass has been assumed constant above.

Integrating this time rate of change of kinetic energy produces a force
line 
integral:

\begin{align*}
K_2 - K_1 
&= \int_{t1}^{t2} \frac{dK}{dt} dt \\
&= \int_{t1}^{t2} \Bp' \cdot \Br' dt \\
&= \int_{t1}^{t2} \Bp' \cdot \frac{d\Br'}{dt} dt \\
&= \int_{\Br_1}^{\Br_2} \BF \cdot d\Br
\end{align*}

For the path integral to depend on only the end points or the corresponding end times requires a conservative force that can be expressed as a gradient.
Let's say that $\BF = \grad f$, then integrating:

\begin{align*}
K_2 - K_1 
&= \int_{\Br_1}^{\Br_2} \BF \cdot d\Br \\
&= \int_{\Br_1}^{\Br_2} \grad f \cdot d\Br \\
&= {\text{limit}}_{\epsilon \rightarrow 0} \int_{\Br_1}^{\Br_1 + \epsilon\rcap}
   \left(\rcap \frac{f(\Br + \epsilon \rcap)}{\epsilon}\right)
      \cdot d\Br \\
&= \text{handwaving} \\
&= f(\Br_2) - f(\Br_1).
\end{align*}

Assembling the quantities for times $1$, and $2$, we have

\begin{equation}
K_2 -f(\Br_2) = K_1 - f(\Br_1) = \text{constant}.
\end{equation}

This constant is what we give the name Energy.  The quantities $-f(\Br_i)$ we label potential energy $V_i$, and finally write the total energy as the sum of the kinetic and potential energies for a particle at a point in time and space:

\begin{equation}
K_2 + V_2 = K_1 + V_1 = E
\end{equation}
\begin{equation}
\BF = -\grad V
\end{equation}

\subsection{ Work with a specific example.  Newtonian gravitational force.}

Take the gravitional force:

\begin{equation}
F = -\frac{GmM}{r^2} \rcap
\end{equation}

The rate of change of kinetic energy with respect to such a force (FIXME: think though signs ... with or against?), is:

\begin{align*}
\frac{dK}{dt} 
&= \Bp' \cdot \Br' \\
&= -\frac{GmM}{r^2} \rcap \cdot \frac{d\Br}{dt} \\
&= -\frac{GmM}{r^3} \Br \cdot \frac{d\Br}{dt}.
\end{align*}

The vector dot products above can be eliminated with the standard trick:

\begin{align*}
\frac{dr^2}{dt} 
&= \frac{\Br \cdot \Br}{dt} \\
&= 2 \frac{d\Br}{dt} \cdot \Br.
\end{align*}

Thus,
\begin{align*}
\frac{dK}{dt} 
&= -\frac{GmM }{2r^3} \frac{dr^2}{dt} \\
&= -\frac{GmM }{r^2} \frac{dr}{dt} \\
&= \frac{d}{dt} \left( \frac{GmM }{r} \right).
\end{align*}

This can be integrated to find the kinetic energy difference associated with a change of position in a gravitational field:

\begin{align*}
K_2 - K_1 
&= \int_{t_1}^{t_2} \frac{d}{dt} \left( \frac{GmM }{r} \right) dt \\
&= GmM \left( \inv{r_2} - \inv{r_1} \right).
\end{align*}

Or, 

\begin{align*}
K_2 - \frac{GmM}{r_2} = K_1 - \frac{GmM}{r_1} = E.
\end{align*}

Taking gradients of this negative term:

\begin{align*}
\grad \left( - \frac{GmM}{r} \right)
&= \rcap \frac{\partial}{\partial r} \left( - \frac{GmM}{r} \right) \\
&= \rcap \frac{GmM}{r^2},
\end{align*}

returns the negation of the original force, so if we write $V = -Gmm/r^2$, it implies the force is:

\begin{equation}
\BF = -\grad V.
\end{equation}

By this example we see how one arrives at the negative sign convention for the potential energy, and also how we end up with strictly positive terms in the Lagrangian associated with a gravitational force:

\begin{equation}
L = \inv{2} m \Bv^2 + \frac{GmM}{r}.
\end{equation}

\end{document}               

\documentclass{article}      % Specifies the document class

\usepackage{amsmath}
\usepackage{mathpazo}

%
% shorthand for bold symbols, convenient for vectors and matrices
%
\newcommand{\Ba}[0]{\mathbf{a}}
\newcommand{\Bb}[0]{\mathbf{b}}
\newcommand{\Bc}[0]{\mathbf{c}}
\newcommand{\Bd}[0]{\mathbf{d}}
\newcommand{\Be}[0]{\mathbf{e}}
\newcommand{\Bf}[0]{\mathbf{f}}
\newcommand{\Bg}[0]{\mathbf{g}}
\newcommand{\Bh}[0]{\mathbf{h}}
\newcommand{\Bi}[0]{\mathbf{i}}
\newcommand{\Bj}[0]{\mathbf{j}}
\newcommand{\Bk}[0]{\mathbf{k}}
\newcommand{\Bl}[0]{\mathbf{l}}
\newcommand{\Bm}[0]{\mathbf{m}}
\newcommand{\Bn}[0]{\mathbf{n}}
\newcommand{\Bo}[0]{\mathbf{o}}
\newcommand{\Bp}[0]{\mathbf{p}}
\newcommand{\Bq}[0]{\mathbf{q}}
\newcommand{\Br}[0]{\mathbf{r}}
\newcommand{\Bs}[0]{\mathbf{s}}
\newcommand{\Bt}[0]{\mathbf{t}}
\newcommand{\Bu}[0]{\mathbf{u}}
\newcommand{\Bv}[0]{\mathbf{v}}
\newcommand{\Bw}[0]{\mathbf{w}}
\newcommand{\Bx}[0]{\mathbf{x}}
\newcommand{\By}[0]{\mathbf{y}}
\newcommand{\Bz}[0]{\mathbf{z}}
\newcommand{\BA}[0]{\mathbf{A}}
\newcommand{\BB}[0]{\mathbf{B}}
\newcommand{\BC}[0]{\mathbf{C}}
\newcommand{\BD}[0]{\mathbf{D}}
\newcommand{\BE}[0]{\mathbf{E}}
\newcommand{\BF}[0]{\mathbf{F}}
\newcommand{\BG}[0]{\mathbf{G}}
\newcommand{\BH}[0]{\mathbf{H}}
\newcommand{\BI}[0]{\mathbf{I}}
\newcommand{\BJ}[0]{\mathbf{J}}
\newcommand{\BK}[0]{\mathbf{K}}
\newcommand{\BL}[0]{\mathbf{L}}
\newcommand{\BM}[0]{\mathbf{M}}
\newcommand{\BN}[0]{\mathbf{N}}
\newcommand{\BO}[0]{\mathbf{O}}
\newcommand{\BP}[0]{\mathbf{P}}
\newcommand{\BQ}[0]{\mathbf{Q}}
\newcommand{\BR}[0]{\mathbf{R}}
\newcommand{\BS}[0]{\mathbf{S}}
\newcommand{\BT}[0]{\mathbf{T}}
\newcommand{\BU}[0]{\mathbf{U}}
\newcommand{\BV}[0]{\mathbf{V}}
\newcommand{\BW}[0]{\mathbf{W}}
\newcommand{\BX}[0]{\mathbf{X}}
\newcommand{\BY}[0]{\mathbf{Y}}
\newcommand{\BZ}[0]{\mathbf{Z}}

\newcommand{\Bzero}[0]{\mathbf{0}}
\newcommand{\Btheta}[0]{\boldsymbol{\theta}}
\newcommand{\Btau}[0]{\boldsymbol{\tau}}
\newcommand{\Bomega}[0]{\boldsymbol{\omega}}

%
% shorthand for unit vectors
%
\newcommand{\acap}[0]{\hat{\Ba}}
\newcommand{\bcap}[0]{\hat{\Bb}}
\newcommand{\ccap}[0]{\hat{\Bc}}
\newcommand{\dcap}[0]{\hat{\Bd}}
\newcommand{\ecap}[0]{\hat{\Be}}
\newcommand{\fcap}[0]{\hat{\Bf}}
\newcommand{\gcap}[0]{\hat{\Bg}}
\newcommand{\hcap}[0]{\hat{\Bh}}
\newcommand{\icap}[0]{\hat{\Bi}}
\newcommand{\jcap}[0]{\hat{\Bj}}
\newcommand{\kcap}[0]{\hat{\Bk}}
\newcommand{\lcap}[0]{\hat{\Bl}}
\newcommand{\mcap}[0]{\hat{\Bm}}
\newcommand{\ncap}[0]{\hat{\Bn}}
\newcommand{\ocap}[0]{\hat{\Bo}}
\newcommand{\pcap}[0]{\hat{\Bp}}
\newcommand{\qcap}[0]{\hat{\Bq}}
\newcommand{\rcap}[0]{\hat{\Br}}
\newcommand{\scap}[0]{\hat{\Bs}}
\newcommand{\tcap}[0]{\hat{\Bt}}
\newcommand{\ucap}[0]{\hat{\Bu}}
\newcommand{\vcap}[0]{\hat{\Bv}}
\newcommand{\wcap}[0]{\hat{\Bw}}
\newcommand{\xcap}[0]{\hat{\Bx}}
\newcommand{\ycap}[0]{\hat{\By}}
\newcommand{\zcap}[0]{\hat{\Bz}}
\newcommand{\thetacap}[0]{\hat{\Btheta}}

%
% to write R^n and C^n in a distinguishable fashion.  Perhaps change this
% to the double lined characters upon figuring out how to do so.
%
\newcommand{\C}[1]{$\mathbb{C}^{#1}$}
\newcommand{\R}[1]{$\mathbb{R}^{#1}$}

%
% various generally useful helpers
%

% derivative of #1 wrt. #2:
\newcommand{\D}[2] {\frac {d#2} {d#1}}

\newcommand{\inv}[1]{\frac{1}{#1}}
\newcommand{\cross}[0]{\times}

\newcommand{\abs}[1]{\lvert{#1}\rvert}
\newcommand{\norm}[1]{\lVert{#1}\rVert}
\newcommand{\innerprod}[2]{\langle{#1}, {#2}\rangle}
\newcommand{\dotprod}[2]{{#1} \cdot {#2}}
\newcommand{\bdotprod}[2]{\left({#1} \cdot {#2}\right)}
\newcommand{\crossprod}[2]{{#1} \cross {#2}}
\newcommand{\tripleprod}[3]{\dotprod{\left(\crossprod{#1}{#2}\right)}{#3}}

\DeclareMathOperator{\Proj}{Proj}
\DeclareMathOperator{\Span}{span}
\DeclareMathOperator{\Sgn}{sgn}
\DeclareMathOperator{\Area}{Area}
\DeclareMathOperator{\Volume}{Volume}

%
% A few miscellaneous things specific to this document
%
\newcommand{\crossop}[1]{\crossprod{#1}{}}

% R2 vector.
\newcommand{\VectorTwo}[2]{
\begin{bmatrix}
 {#1} \\
 {#2}
\end{bmatrix}
}

\newcommand{\VectorN}[1]{
\begin{bmatrix}
{#1}_1 \\
{#1}_2 \\
\vdots \\
{#1}_N \\
\end{bmatrix}
}

\newcommand{\DETuvij}[4]{
\begin{vmatrix}
 {#1}_{#3} & {#1}_{#4} \\
 {#2}_{#3} & {#2}_{#4}
\end{vmatrix}
}

\newcommand{\DETuvwijk}[6]{
\begin{vmatrix}
 {#1}_{#4} & {#1}_{#5} & {#1}_{#6} \\
 {#2}_{#4} & {#2}_{#5} & {#2}_{#6} \\
 {#3}_{#4} & {#3}_{#5} & {#3}_{#6}
\end{vmatrix}
}

\newcommand{\DETuvwxijkl}[8]{
\begin{vmatrix}
 {#1}_{#5} & {#1}_{#6} & {#1}_{#7} & {#1}_{#8} \\
 {#2}_{#5} & {#2}_{#6} & {#2}_{#7} & {#2}_{#8} \\
 {#3}_{#5} & {#3}_{#6} & {#3}_{#7} & {#3}_{#8} \\
 {#4}_{#5} & {#4}_{#6} & {#4}_{#7} & {#4}_{#8} \\
\end{vmatrix}
}

%\newcommand{\DETuvwxyijklm}[10]{
%\begin{vmatrix}
% {#1}_{#6} & {#1}_{#7} & {#1}_{#8} & {#1}_{#9} & {#1}_{#10} \\
% {#2}_{#6} & {#2}_{#7} & {#2}_{#8} & {#2}_{#9} & {#2}_{#10} \\
% {#3}_{#6} & {#3}_{#7} & {#3}_{#8} & {#3}_{#9} & {#3}_{#10} \\
% {#4}_{#6} & {#4}_{#7} & {#4}_{#8} & {#4}_{#9} & {#4}_{#10} \\
% {#5}_{#6} & {#5}_{#7} & {#5}_{#8} & {#5}_{#9} & {#5}_{#10}
%\end{vmatrix}
%}

% R3 vector.
\newcommand{\VectorThree}[3]{
\begin{bmatrix}
 {#1} \\
 {#2} \\
 {#3}
\end{bmatrix}
}


\newcommand{\Brho}[0]{\boldsymbol{\rho}}
\newcommand{\LL}[0]{\mathcal{L}}
\newcommand{\Abs}[1]{\left\lvert{#1}\right\rvert}
\newcommand{\qdot}[0]{\dot{q}}
\newcommand{\qddot}[0]{\ddot{q}}
\newcommand{\xdot}[0]{\dot{x}}
\newcommand{\xddot}[0]{\ddot{x}}
\newcommand{\ydot}[0]{\dot{y}}
\newcommand{\yddot}[0]{\ddot{y}}
\newcommand{\dotalpha}[0]{\dot{\alpha}}
\newcommand{\ddotalpha}[0]{\ddot{\alpha}}
\newcommand{\dottheta}[0]{\dot{\theta}}
\newcommand{\ddottheta}[0]{\ddot{\theta}}
\newcommand{\dotphi}[0]{\dot{\phi}}
\newcommand{\ddotphi}[0]{\ddot{\phi}}
% == \partial_{#1} {#2}
\newcommand{\PD}[2]{\frac{\partial {#2}}{\partial {#1}}}
\newcommand{\PDD}[3]{\frac{\partial^2 {#3}}{\partial {#1}\partial {#2}}}

%
% The real thing:
%

                             % The preamble begins here.
\title{Attempts at solutions for some Goldstein Mechanics problems.} % Declares the document's title.
\author{Peeter Joot}         % Declares the author's name.
\date{ }        % Deleting this command produces today's date.

\begin{document}             % End of preamble and beginning of text.

\maketitle{}

\section{ Problem 1.7 }

Barbell shape, equal masses.  center of rod between masses constrained to circular motion.

Assuming motion in a plane, the equation for the center of the rod is:

\begin{equation*}
c = a e^{i\theta}
\end{equation*}

and the two mass points positions are:
\begin{align*}
q_1 &= c + (l/2) e^{i\alpha} \\
q_2 &= c - (l/2) e^{i\alpha}
\end{align*}

taking derivatives:
\begin{align*}
\qdot_1 &= a i \dottheta e^{i\theta} + (l/2) i \dotalpha e^{i\alpha} \\
\qdot_2 &= a i \dottheta e^{i\theta} - (l/2) i \dotalpha e^{i\alpha} \\
\end{align*}

and squared magnitudes:

\begin{align*}
\qdot_{\pm}
&= \Abs{a \dottheta \pm (l/2) \dotalpha e^{i(\alpha - \theta)}}^2 \\
&= \left(a \dottheta   \pm   \inv{2} l \dotalpha \cos(\alpha - \theta)\right)^2 + \left(\inv{2} l \dotalpha \sin(\alpha - \theta)\right)^2
\end{align*}

Summing the kinetic terms yeilds

\begin{equation*}
K = m \left(a \dottheta \right)^2 + m \left(\inv{2} l \dotalpha\right)^2
\end{equation*}

Summing the potential energies, presuming that the motion is verticle, we have:

\begin{equation*}
V = m g (l/2) \cos\theta - m g (l/2) \cos \theta
\end{equation*}

So, the Lagrangian is just the Kinetic energy.

Taking derivatives to get the OEMs we have:

\begin{align*}
(m a^2 \dottheta)' &= 0 \\
\left(\inv{4} m l^2 \dotalpha \right)' &= 0
\end{align*}

This is suprising seeming.  Is this correct?

\section{ Problem 1.8 }

Hopefully, not a copyright violation, but here is the problem verbatim:

A system is composed of three particles of equal mass m.  Between any two of them there are forces derivable from a potential

\begin{equation*}
V = -g e^{-\mu r}
\end{equation*}

where r is the disance between the two particles.  In addition, two of the particles each exert a force on the third which can be obtained from a generalized potential of the form

\begin{equation*}
U = -f \Bv \cdot \Br
\end{equation*}

$\Bv$ being the relative velocity of the interacting particles and f a constant.  Set up the Lagragian for the system, using as coordinates the radius vector $\BR$ of the center of mass and the two vectors

\begin{align*}
\Brho_1 &= \Br_1 - \Br_3 \\
\Brho_2 &= \Br_2 - \Br_3
\end{align*}

Is the total angular momentum of the system conserved?

\subsection{ Solution attempt. }

The center of mass vector is:

\begin{equation*}
\BR = \inv{3}(\Br_1 + \Br_2 + \Br_3)
\end{equation*}

This can be used to express each of the position vectors in terms of the $\Brho_i$ vectors:

\begin{align*}
3 m \BR &= m (\Brho_1 + \Br_3) + m(\Brho_2 + \Br_3) + m \Br_3 \\
        &= 2 m (\Brho_1 + \Brho_2) + 3 m \Br_3 \\
  \Br_3 &= \BR - \inv{3}(\Brho_1 + \Brho_2) \\
\Br_2 = \Brho_2 + \Br_3 &= \Brho_2 + \Br_3 = \frac{2}{3} \Brho_2 - \inv{2} \Brho_1 + \BR \\
\Br_1 = \Brho_1 + \Br_3 &= \frac{2}{3} \Brho_1 - \inv{2} \Brho_2 + \BR \\
\end{align*}

Now, that is enough to specify the part of the Lagrangian from the potentials that act between all the particles

\begin{equation*}
\LL_U = \sum -U_{ij} = g \left( e^{-\mu \Abs{\Brho_1}} + e^{-\mu \Abs{\Brho_2}} + e^{-\mu \Abs{ \Brho_1 - \Brho_2 }} \right)
\end{equation*}

Now, we need to calculate the two $V$ potentials in terms.  If we consider with positions $\Br_1$, and $\Br_2$ to be the ones
that can exert a force on the third, the velocities of those masses relative to $\Br_3$ are:

\begin{equation*}
(\Br_3 - \Br_i)' = \dot{\Brho_i}
\end{equation*}

Adding this to the first half of the Lagrangian we have:
So, for the second half of the Lagrangian we have:

\begin{equation*}
\LL =
g \left( e^{-\mu \Abs{\Brho_1}} + e^{-\mu \Abs{\Brho_2}} + e^{-\mu \Abs{ \Brho_1 - \Brho_2 }} \right)
+ f \left(\BR - \inv{3}(\Brho_1 + \Brho_2) \right) \cdot \left( \dot{\Brho_1} + \dot{\Brho_2} \right)
\end{equation*}

So, there's the Lagrangian.

How about the angular momentum conservation question?  How to answer that?  One way would be to compute the forces from the Lagrangian, and take cross products but is that really the best way?  Perhaps the answer is as simple as observing that there are no external torque's on the system, thus $d\BL/dt = 0$, or angular momentum for the system is constant (conserved).

FIXME: FOLLOWUP: it has been suggested to me on PF that I should look at how this Lagrangian transforms under rotation.  The relative vectors (both speed and velocity) between the different points will be rotation invarient.  Think that is the case for the CM too.

\section{ Problem 2.1 }

Prove that the shortest length curve between two points in space is a straight line.

A first attempt of this I used:

\begin{equation*}
ds = \sqrt{ 1 + (dy/dx)^2 + (dz/dx)^2 } dx
\end{equation*}

Application of the Euler-Lagrange equations does show that one ends up with a linear relation between the y and z coordinates, but no mention of x.  Rather than write that up, consider instead a parameterization of the coordinates:

\begin{align*}
x &= x_1(\lambda) \\
y &= x_2(\lambda) \\
z &= x_3(\lambda)
\end{align*}

in terms of this arbitrary parameterization we have a segment length of:

\begin{equation*}
ds = \sqrt{ \sum \left(\frac{d x_i}{d\lambda}\right)^2 } d \lambda = f\left(x_i\right) d\lambda
\end{equation*}

Application of the Euler-Lagrange equation to $f$ we have:

\begin{align*}
\PD{x_i}{f} 
&= 0 \\
&= \frac{d}{d\lambda} \PD{\xdot_i}{} \sqrt{ \sum {\xdot_j}^2 } \\
&= \frac{d}{d\lambda} \frac{ \xdot_i }{\sqrt{ \sum {\xdot_j}^2 }}
\end{align*}

Therefore each of these quotients can be equated to a constant:

\begin{align*}
\frac{ \xdot_i }{\sqrt{ \sum {\xdot_j}^2 }} &= {c_i}^{-2} \\
{c_i}^2 \xdot_i^2 &= \sum {\xdot_j}^2 \\
({c_i}^2 -1)\xdot_i^2 &= \sum_{j \ne i} {\xdot_j}^2 \\
(1 - {c_i}^2)\xdot_i^2 + \sum_{j \ne i} {\xdot_j}^2 &= 0 
\end{align*}

This last form shows explicitly that not all of these squared derivative terms can be linearly independent.  In particular, we have a
zero determinant:

\begin{equation*}
0 =
\begin{vmatrix}
1 - c_1^2   & 1            & 1         & 1 & \hdots \\
1           & 1 - c_2^2    & 1         & 1 & \vdots \\
1           & 1            & 1 - c_3^2 & 1 & \\
            &              &           & \ddots & \\
            &              &           &        & 1 - {c_n}^2
\end{vmatrix}
\end{equation*}

Now, expanding this for a couple specific cases isn't too hard.  For $n=2$ we have:

\begin{align*}
0 &= (1 - c_1^2)(1-c_2^2) - 1 \\
c_1^2 + c_2^2 &= c_1^2 c_2^2 \\
c_1^2 &= \frac{c_2^2}{ c_2^2 - 1 } \\
c_2^2 - 1 &= \frac{c_2^2}{ c_1^2 }
\end{align*}

This can be substuited back into one our $c_2^2$ equation:

\begin{align*}
({c_2}^2 -1)\xdot_2^2 &= {\xdot_1}^2 \\
\frac{c_2^2}{ c_1^2 } \xdot_2^2 &= {\xdot_1}^2 \\
\pm \frac{c_2}{ c_1 } \xdot_2 &= {\xdot_1} \\
\pm \frac{c_2}{ c_1 } x_2 &= x_1 + \kappa \\
\end{align*}

This is precisely the straight line that was desired, but we have setup for proving that consideration of all path variations from two points 
in \R{N} space has the shortest distance when that path is a straight line.

Despite the general setup, I'm going to chicken out and show this only for the \R{3} case.  In that case our determinant expands to:

\begin{equation*}
c_1^2 + c_2^2 + c_3^2 = c_1^2 c_2^2 c_3^2
\end{equation*}

Since not all of the $\xdot_i^2$ can be linearly independent, one can be eliminated:

\begin{align*}
(1 - c_1^2) \xdot_1^2 + \xdot_2^2 + \xdot_3^2 &= 0 \\
(1 - c_2^2) \xdot_2^2 + \xdot_3^2 + \xdot_1^2 &= 0 \\
(1 - c_3^2) \xdot_3^2 + \xdot_1^2 + \xdot_2^2 &= 0
\end{align*}

Let's pick $\xdot_1^2$ to eliminate, and subst 2 into 3:

\begin{align*}
%(1 - c_1^2) (-(1 - c_2^2) \xdot_2^2 - \xdot_3^2) + \xdot_2^2 + \xdot_3^2 &= 0 \\
(1 - c_3^2) \xdot_3^2 + (-(1 - c_2^2) \xdot_2^2 - \xdot_3^2) + \xdot_2^2 &= 0
\implies \\
%\xdot_2^2 ( 1 - (1 - c_1^2)(1 - c_2^2) ) + \xdot_3^2 ( 1 - (1 - c_1^2) ) &= 0 \\
- c_3^2 \xdot_3^2 + c_2^2 \xdot_2 &= 0 \\
\pm c_3 \xdot_3 &= c_2 \xdot_2 \\
\end{align*}

%Which is, once again a straight line:
%
%\begin{equation*}
%\pm c_3 x_3 = c_2 x_2 + \kappa
%\end{equation*}

Since these equations are symmetric, we can do this for all, with the result:
\begin{align*}
\pm c_3 \xdot_3 &= c_2 \xdot_2 \\
\pm c_3 \xdot_3 &= c_1 \xdot_1 \\
\pm c_2 \xdot_2 &= c_1 \xdot_1 \\
\end{align*}

Since the $c_i$ constants are arbitrary, then we can for example pick the negative sign for $\pm c_2$, and the positive for the rest, then add all of these, and scale by two:

\begin{equation*}
c_3 \xdot_3 - c_2 \xdot_2 = c_1 \xdot_1
\end{equation*}

and integrating:

\begin{equation*}
c_3 x_3 - c_2 x_2 = c_1 x_1 + \kappa
\end{equation*}

Again, we have the general equation of a line, subject to the desired constraints on the end points.  In the end we didn't need to 
evaluate the determinant after all, as done in the 
\R{2} case.

\section{ Problem 2.2 }

Prove that the geodesics (shortest length paths) on a spherical surface are great circles.

As a variational problem, the first step is to formulate an element of length on the surface.  If we write our vector in spherical coordinates ($\phi$ on the equator, and $\theta$ measuring from the north pole) we have:

FIXME: Scan picture.

\begin{equation*}
\Br = (x, y, z) = R( \sin\theta cos\phi, \sin\theta \sin\phi, \cos\theta)
\end{equation*}

A differential vector element on the surface is (set $R=1$ without loss of generality) :

\begin{align*}
d \Br 
&= \frac{d\Br}{d \theta} \frac{d \theta}{d \lambda} d \lambda + \frac{d\Br}{d \phi} \frac{d \phi}{d \lambda} d \lambda \\
&=
 ( \cos\theta \cos\phi, \cos\theta \sin\phi, -\sin\theta) \dottheta d\lambda
+( -\sin\theta \sin\phi, \sin\theta \cos\phi, 0) \dotphi d\lambda \\
&=
 ( \cos\theta \cos\phi \dottheta - \sin\theta \sin\phi \dotphi,
   \cos\theta \sin\phi \dottheta + \sin\theta \cos\phi \dotphi,
  -\sin\theta \dottheta) d\lambda
\end{align*}

Calculation of the length $ds$ of this vector yields:

\begin{equation*}
ds = \Abs{ d\Br} = \sqrt{\dottheta^2 + (\sin\theta)^2 \dotphi^2} d\lambda
\end{equation*}

This completes the setup for the minimization problem, and we want to 
minimize:

\begin{equation*}
s = \int \sqrt{\dottheta^2 + ( \dotphi \sin\theta )^2 } d\lambda
\end{equation*}

and can therefore apply the Euler-Lagrange equations to the function

\begin{equation*}
f(\theta, \phi, \dottheta, \dotphi, \lambda) = 
\sqrt{\dottheta^2 + ( \dotphi \sin\theta )^2 }
\end{equation*}

The $\phi$ is cyclic, and we have:

\begin{equation*}
\PD{\phi}{f} = 0 = \frac{d}{d\lambda} \frac{\dotphi \sin^2\theta}{f}
\end{equation*}

Thus we have:
\begin{align*}
\dotphi^2 \sin^4\theta &= K^2 \left(\dottheta^2 + ( \dotphi \sin\theta )^2 \right) \\
\dotphi^2 \sin^2\theta( \sin^2\theta - K^2 ) &= K^2 \dottheta^2 \\
\dotphi^2 
&= \frac{K^2 \dottheta^2 }{ \sin^2\theta ( \sin^2\theta - K^2 ) } \\
\dotphi
&= \frac{K \dottheta }{ \sin\theta \sqrt{ \sin^2\theta - K^2 } } \\
\end{align*}

This is in a nicely separated form, but it is not obvious that this describes paths that are great circles.

Let's have a look at the second equation.
\begin{align*}
\PD{\theta}{f} &= \frac{d}{d\lambda} \PD{\dottheta}{f} \\
\frac{\sin\theta\cos\theta \dotphi^2}{f}
&= \frac{d}{d\lambda} \frac{\dottheta}{f} \\
&= \frac{\ddottheta}{f} - \inv{2} \frac{ (\dottheta^2 + ( \dotphi \sin\theta )^2 )' }{f^3} \\
&= \frac{\ddottheta}{f} - \frac{ \dottheta \ddottheta + \dotphi \sin\theta ( \ddotphi \sin\theta + \dotphi \cos\theta \dottheta ) }{f^3} \\
\implies
-\sin\theta\cos\theta \dotphi^2 ( \dottheta^2 + ( \dotphi \sin\theta )^2 )
&= -\ddottheta ( \dottheta^2 + ( \dotphi \sin\theta )^2 )
   + \dottheta \ddottheta 
   + \dotphi \sin\theta ( \ddotphi \sin\theta + \dotphi \cos\theta \dottheta ) \\
\end{align*}

Or,
\begin{equation*}
- \ddottheta \dottheta^2 
- \ddottheta \dotphi^2 \sin^2\theta 
+ \dottheta \ddottheta 
+ \dotphi \ddotphi \sin^2\theta
+ \dotphi^2 \dottheta \sin\theta \cos\theta
+ \dotphi^2 \dottheta^2 \sin\theta \cos\theta 
+ \dotphi^4 \sin^3\theta \cos\theta 
= 0
\end{equation*}

What a mess!  I don't feel inclined to try to reduce this at the moment.  I'll come back to this problem later.  Perhaps there's a better parameterization?

\section{ Problem 2.3 }

For $f = f( y, \ydot, \yddot, x )$, find the equations for extreme values of

\begin{equation*}
I = \int_a^b f dx
\end{equation*}

Here we want $y$ and $\ydot$ fixed at the end points.  Following the first derivative derivation write the 
functions in terms of the desired extremum functions plus a set of arbitrary functions:

\begin{align*}
y( x, \alpha ) &= y( x, 0 ) + \alpha n(x) \\
\ydot( x, \alpha ) &= \ydot( x, 0 ) + \alpha m(x)
\end{align*}

Here we specify that these arbitrary variational functions vanish at the endpoints:

\begin{equation*}
n(a) = n(b) = m(a) = m(b) = 0
\end{equation*}

The functions $y(x, 0)$, and $\ydot(x, 0)$ are the functions we are looking for as solutions to the min/max problem.

Calculating derivatives we have:

\begin{equation*}
\frac{dI}{d\alpha} = 
\int \left( 
\PD{y}{f} \PD{\alpha}{y}
+\PD{\ydot}{f} \PD{\alpha}{\ydot}
+\PD{\yddot}{f} \PD{\alpha}{\yddot}
\right) d x
\end{equation*}

Assuming sufficient continuity look at the second term where we have:

\begin{align*}
\PD{\alpha}{\ydot} 
&= \PD{\alpha}{} \PD{x}{y} \\
&= \PD{x}{} \PD{\alpha}{y} \\
&= \PD{x}{} n(x) \\
&= \frac{d}{ d x} n(x) \\
&= \frac{d}{ d x} \PD{\alpha}{y} \\
\end{align*}

Similarily for the third term we have:

\begin{equation*}
\PD{\alpha}{\ydot} = \frac{d}{ d x} \PD{\alpha}{\ydot}
\end{equation*}

\begin{equation*}
\frac{dI}{d\alpha} = 
\int \PD{y}{f} \PD{\alpha}{y} d x +
\underbrace{\PD{\ydot}{f} \frac{d}{ d x} \PD{\alpha}{y}}_{ u v' = (u v)' - u' v } d x
+\PD{\yddot}{f} \frac{d}{ d x} \PD{\alpha}{\ydot} d x
\end{equation*}

Now integrating by parts:
\begin{align*}
\frac{dI}{d\alpha} &= 
 \int \PD{y}{f} \PD{\alpha}{y} d x
+\int \PD{\ydot}{f} \frac{d}{ d x} \PD{\alpha}{y} d x
+\int \PD{\yddot}{f} \frac{d}{ d x} \PD{\alpha}{\ydot} d x \\
\frac{dI}{d\alpha} &= 
 \int \PD{y}{f} \PD{\alpha}{y} d x
+\left(\PD{\ydot}{f} \underbrace{\PD{\alpha}{y}}_{n(x)}\right)_a^b - \int \PD{\alpha}{y} \frac{d}{ d x} \PD{\ydot}{f} d x
+\left(\PD{\yddot}{f} \underbrace{\PD{\alpha}{\ydot}}_{m(x)} \right)_a^b
-\int \PD{\alpha}{\ydot} \frac{d}{ d x} \PD{\yddot}{f} d x
\end{align*}

Because $m(a) = m(b) = n(a) = n(b)$ the non-integral terms are all zero, leaving:

\begin{align*}
\frac{dI}{d\alpha} &= 
  \int \PD{y}{f} \PD{\alpha}{y} d x
- \int \PD{\alpha}{y} \frac{d}{ d x} \PD{\ydot}{f} d x
- \int \PD{\alpha}{\ydot} \frac{d}{ d x} \PD{\yddot}{f} d x
\end{align*}

Now consider just this last integral, which we can again integrate by parts:
\begin{align*}
\int \PD{\alpha}{\ydot} \frac{d}{ d x} \PD{\yddot}{f} d x
&= \int \underbrace{\frac{d}{dx} \PD{\alpha}{y}}_{u'} \underbrace{\frac{d}{ d x} \PD{\yddot}{f}}_{v} d x \\
&= 
\left( \underbrace{\PD{\alpha}{y}}_{n(x)} {\frac{d}{ d x} \PD{\yddot}{f}} \right)_a^b
-\int \PD{\alpha}{y} \frac{d}{dx} {\frac{d}{ d x} \PD{\yddot}{f}} d x \\
&= 
-\int \PD{\alpha}{y} \frac{d^2}{dx^2} \PD{\yddot}{f} d x \\
\end{align*}

This gives:
\begin{align*}
\frac{dI}{d\alpha} &= 
  \int \PD{y}{f} \PD{\alpha}{y} d x
- \int \PD{\alpha}{y} \frac{d}{ d x} \PD{\ydot}{f} d x
+ \int \PD{\alpha}{y} \frac{d^2}{dx^2} \PD{\yddot}{f} d x \\
\frac{dI}{d\alpha} 
&= \int d x \PD{\alpha}{y} \left( \PD{y}{f} - \frac{d}{ d x} \PD{\ydot}{f} + \frac{d^2}{dx^2} \PD{\yddot}{f} \right) \\
&= \int d x n(x) \left( \PD{y}{f} - \frac{d}{ d x} \PD{\ydot}{f} + \frac{d^2}{dx^2} \PD{\yddot}{f} \right)
\end{align*}

So, if we want this derivative to equal zero for any $n(x)$ we require the inner quantity to by zero:

\begin{equation}
\PD{y}{f} - \frac{d}{ d x} \PD{\ydot}{f} + \frac{d^2}{dx^2} \PD{\yddot}{f} = 0
\end{equation}

Question.  Goldstein writes this in total differential form instead of a derivative:

\begin{align*}
dI &= \frac{dI}{d\alpha} d\alpha \\
&= \int d x \left(\PD{\alpha}{y} d \alpha\right) \left( \PD{y}{f} - \frac{d}{ d x} \PD{\ydot}{f} + \frac{d^2}{dx^2} \PD{\yddot}{f} \right) \\
\end{align*}

and then calls this quantity $\PD{\alpha}{y} d \alpha = \delta y$, the variation of $y$.  There must be a mathematical subtlety which motivates this
but it isn't clear to me what that is.  Since the variational calculus texts go a different route, with norms on functional spaces and so forth, perhaps
understanding that motivation isn't worthwhile.  In the end, the conclusion is the same, namely that the inner expression must equal zero for the extremum
condition.

\end{document}               % End of document.

%
% Copyright � 2012 Peeter Joot.  All Rights Reserved.
% Licenced as described in the file LICENSE under the root directory of this GIT repository.
%

\chapter{Compare some wave equation's and their Lagrangians}
\index{wave equation}
\label{chap:waveLagrangian}
%\date{ Dec 02, 2008.  waveLagrangian.tex }

\section{Motivation}

Compare the Lagrangians for the classical wave equation of a vibrating string/film with the wave equation Lagrangian for electromagnetism and Quantum mechanics.

Observe the similarities and differences, and come back to this later after grasping some of the concepts of Field energy and momentum (energy in vibration and electromagnetism and momentum in quantum mechanics).  Do the ideas of field momentum carry in quantum have equivalents in electromagnetism?

\section{Vibrating object equations}

\subsection{One dimensional wave equation}

\citep{goldstein1951cm} does a nice derivation of the one dimensional wave
equation Lagrangian, using a limiting argument applied to an infinite
sequence of connected masses on springs.

\begin{equation}\label{eqn:wave_lagrangian:oneDimensionalWaveLagrangian}
\begin{aligned}
\LL = \inv{2} \left(\mu \left(\PD{t}{\eta}\right)^2 - Y \left(\PD{x}{\eta}\right)^2 \right)
\end{aligned}
\end{equation}

here \(\eta\) was the displacement from the equilibrium position, \(\mu\) is the mass line density and \(Y\) is Young's modulus.

Taking derivatives confirms that this is the correct form.  The Euler-Lagrange
equations for this equation are:

\begin{equation}\label{eqn:waveLagrangian:20}
\begin{aligned}
\PD{\eta}{\LL} &= \PD{t}{} \PD{\PD{t}{\eta}}{\LL} +\PD{x}{} \PD{\PD{x}{\eta}}{\LL} \\
0 &= \PD{t}{} \mu \PD{t}{\eta} -\PD{x}{} Y \PD{x}{\eta} \\
\end{aligned}
\end{equation}

Which has the expected form

\begin{equation}\label{eqn:waveLagrangian:40}
\begin{aligned}
\mu \PDsq{t}{\eta} - Y \PDsq{x}{\eta} &= 0 \\
\end{aligned}
\end{equation}

\subsection{Higher dimension wave equation}

For a string or film or other wavy material with more degrees of freedom than a string with back and forth motion one can guess the Lagrangian from \eqnref{eqn:wave_lagrangian:oneDimensionalWaveLagrangian}.

\begin{equation}\label{eqn:wave_lagrangian:moreDimensionalWaveLagrangian}
\begin{aligned}
\LL = \inv{2} \left(\mu \left(\PD{t}{\eta}\right)^2 - Y \sum_i \left(\PD{x^i}{\eta}\right)^2 \right)
\end{aligned}
\end{equation}

Calculating the Euler-Lagrange equations gives
\begin{equation}\label{eqn:waveLagrangian:60}
\begin{aligned}
\PD{\eta}{\LL} &= \PD{t}{} \PD{\PD{t}{\eta}}{\LL} +\sum_i \PD{x^i}{} \PD{\PD{x^i}{\eta}}{\LL} \\
0 &= \PD{t}{} \mu \PD{t}{\eta} - \sum_i \PD{x^i}{} Y \PD{x^i}{\eta} \\
\end{aligned}
\end{equation}

Which also has the expected form

\begin{equation}\label{eqn:waveLagrangian:80}
\begin{aligned}
\mu \PDsq{t}{\eta} - Y \sum_i \PDsq{x^i}{\eta} &= 0 \\
\end{aligned}
\end{equation}

\section{Electrodynamics wave equation}

From \eqnref{eqn:wave_lagrangian:moreDimensionalWaveLagrangian} one can guess the Lagrangian for the electrodynamic potential wave equations.  Maxwell's equation in potential form are:

\begin{equation}\label{eqn:wave_lagrangian:maxwellPotential}
\begin{aligned}
\grad^2 A &= J/\epsilon_0 c
\end{aligned}
\end{equation}

Which has the following split into four scalar equations
\begin{equation}\label{eqn:waveLagrangian:100}
\begin{aligned}
\grad^2 A^\mu \gamma_\mu &= J^\mu \gamma_\mu/\epsilon_0 c \\
\grad^2 A^\mu &= J^\mu /\epsilon_0 c
\end{aligned}
\end{equation}

For the \(A^\mu\) coordinate try the Lagrangian

\begin{equation}\label{eqn:waveLagrangian:120}
\begin{aligned}
\LL
&= \sum_\nu \inv{2} (\gamma^\nu)^2 \left( \PD{x^\nu}{A^\mu} \right)^2 + J^\nu A^\nu / \epsilon_0 c \\
&= \sum_\nu \inv{2} (\gamma^\nu)^2 \left(\partial_{\nu}{A^\mu}\right)^2 + J^\nu A^\nu / \epsilon_0 c \\
\end{aligned}
\end{equation}

With evaluation of the Euler-Lagrange equations we have

\begin{equation}\label{eqn:waveLagrangian:140}
\begin{aligned}
\PD{A^\mu}{\LL} &= \sum_\alpha \partial_{\alpha} \PD{(\partial{\alpha}{A^\mu})}{\LL} \\
\implies \\
J^\mu/\epsilon_0 c
&= \sum \partial_\alpha (\gamma^\alpha)^2 \partial_{\alpha}{A^\mu} \\
&= \partial^\alpha \partial_{\alpha}{A^\mu} \\
&= \grad^2 A^\mu \\
\end{aligned}
\end{equation}

Which recovers Maxwell's equation.  Having done that the Lagrangian can be tidied slightly introducing the spacetime gradient:

\begin{equation}\label{eqn:wave_lagrangian:potentialLagrangianWithGrad}
\begin{aligned}
\LL &= \inv{2} \left( \grad {A^\alpha} \right)^2 + J^\alpha A^\alpha / \epsilon_0 c
\end{aligned}
\end{equation}

\subsection{Comparing with complex (bivector) version of Maxwell Lagrangian}
Previously, in \bookchapcite{PJMaxwellLagrangian}{phy354} and \bookchapcite{PJMaxwellLagrangian}{phy354}, Maxwell's equation

\begin{equation}\label{eqn:wave_lagrangian:maxwell}
\begin{aligned}
\grad (\grad \wedge A) &= J/ \epsilon_0 c
\end{aligned}
\end{equation}

was seen as the result of evaluating the Lagrangian

\begin{equation}\label{eqn:wave_lagrangian:maxlag}
\begin{aligned}
\LL &= -\frac{\epsilon_0 c}{2} (\grad \wedge A)^2 + J \cdot A
\end{aligned}
\end{equation}

\Eqnref{eqn:wave_lagrangian:maxwell} with the gauge condition \(\grad \cdot A = 0\)
is where we get the potential form \eqnref{eqn:wave_lagrangian:maxwellPotential} from.

For comparison it should be possible to reconcile this with
\eqnref{eqn:wave_lagrangian:potentialLagrangianWithGrad}.  We can multiply by \((\gamma_\alpha)^2\), which is \((\pm 1)\) dependent on \(\alpha\), as well as multiply by \(\epsilon_0 c\)

\begin{equation}\label{eqn:waveLagrangian:160}
\begin{aligned}
\LL &= \frac{\epsilon_0 c}{2} \grad {A^\alpha} \grad {A_\alpha} + J^\alpha A_\alpha \\
\end{aligned}
\end{equation}

No sum need be implied here, but since the field variables are independent we can sum them without changing the field equations.  So, instead of having four independent Lagrangians, we are now left with a (sums now implied) single density that can be evaluated for each of the potential coordinate variables:

\begin{equation}\label{eqn:waveLagrangian:180}
\begin{aligned}
\LL &= \frac{\epsilon_0 c}{2} \grad {A^\alpha} \grad {A_\alpha} + J \cdot A \\
\end{aligned}
\end{equation}

This is looking more like \eqnref{eqn:wave_lagrangian:maxlag} now.  It is expected that the gauge condition can be used to complete the reconciliation.  However, I have had trouble actually doing this, despite the fact that both Lagrangians appear to correctly
lead to equivalent results.

Also notable perhaps is a comparison to the four potential Lagrangian in Goldstein:

\begin{equation}\label{eqn:waveLagrangian:200}
\begin{aligned}
\LL =
-\inv{16\pi} \sum_{\mu,\nu} \left(
\PD{x_\nu}{A_\mu} - \PD{x_\mu}{A_\nu}
\right)^2
-\inv{8\pi}
\sum_\mu
\left(
\PD{x_\mu}{A_\mu}
\right)^2
+
\sum_\mu \frac{j_\mu A_\mu}{c}
\end{aligned}
\end{equation}

This one is considerably more complex looking, and
it should be possible to see how exactly this is related to the wave
equation guessed by comparison to the vibrating string.

\section{Quantum Mechanics}

\subsection{Non-relativistic case}

The non-relativistic Lagrangian given by Goldstein (problem 11.3) is

\begin{equation}\label{eqn:waveLagrangian:220}
\begin{aligned}
\LL = \frac{\Hbar^2}{2m}
(\spacegrad \psi) \cdot (\spacegrad \psi^\conj) + V \psi \psi^\conj + {i \Hbar} \left( \psi \partial_t \psi^\conj - \psi^\conj \partial_t \psi \right)
\end{aligned}
\end{equation}

Again we see the square of the spatial gradient so we expect a (spatial) Laplacian
in the field equation, which one has:

\begin{equation}\label{eqn:waveLagrangian:240}
\begin{aligned}
\left( \frac{-\Hbar^2}{2m} \spacegrad^2 + V \right) \psi = i \Hbar \PD{t}{\psi}
\end{aligned}
\end{equation}

\subsection{Relativistic case. Klein-Gordon}

The Klein-Gordon Lagrangian is
\begin{equation}\label{eqn:waveLagrangian:260}
\begin{aligned}
\LL
&= -(\grad \psi) \cdot (\grad \psi^\conj) + \frac{m^2 c^2}{\Hbar^2} \psi \psi^\conj \\
\end{aligned}
\end{equation}

from which we can recover the Klein-Gordon scalar wave equation which applies to a
specific subset of quantum phenomena (what exactly?)

\begin{equation}\label{eqn:waveLagrangian:280}
\begin{aligned}
\left(\frac{\Hbar^2}{2m} \grad^2 + \inv{2} m c^2\right) \psi = 0 \\
\end{aligned}
\end{equation}

\subsection{Dirac wave equation}

The Dirac wave equation, for vector wave function \(\psi\) can be formally
obtained by
taking vector roots of the scalar operators in the Klein-Gordon equation
to yield:

\begin{equation}\label{eqn:waveLagrangian:300}
\begin{aligned}
i \Hbar \grad \psi = \pm m c \psi
\end{aligned}
\end{equation}

The Lagrangian for this field equation is

\begin{equation}\label{eqn:waveLagrangian:320}
\begin{aligned}
\LL = mc \overbar{\psi}\psi - {\inv{2}i\Hbar}(\overbar{\psi}\gamma^\mu (\partial_\mu\psi) - (\partial_\mu\overbar{\psi})\gamma^\mu \psi)
\end{aligned}
\end{equation}

Where \(\overbar{\psi} = \gamma_0 \tilde{\psi}\), and \(\tilde{\psi}\) is the reversed field spinor.

\section{Summary comparison of all the second order wave equations}

\begin{itemize}

\item Vibration wave equation.

\begin{equation}\label{eqn:waveLagrangian:340}
\begin{aligned}
\LL &= \mu \left( \PD{t}{\eta} \right)^2 - Y \left( \spacegrad \eta \right)^2 \\
0 &= \mu \PDsq{t}{\eta} - Y \spacegrad^2 {\eta}
\end{aligned}
\end{equation}

\item Maxwell wave equation.

\begin{equation}\label{eqn:waveLagrangian:360}
\begin{aligned}
\LL &= \inv{2} \left( \grad {A^\alpha} \right)^2 + J^\alpha A^\alpha / \epsilon_0 c \\
\grad^2 A^\alpha &= J^\alpha/\epsilon_0 c
\end{aligned}
\end{equation}

\item Schr\"{o}dinger non-relativistic wave equation.

\begin{equation}\label{eqn:waveLagrangian:380}
\begin{aligned}
\LL &= \frac{\Hbar^2}{2m}
(\spacegrad \psi) \cdot (\spacegrad \psi^\conj) + V \psi \psi^\conj + {i \Hbar} \left( \psi \partial_t \psi^\conj - \psi^\conj \partial_t \psi \right) \\
\left( \frac{-\Hbar}{2m} \spacegrad^2 + V \right) \psi &= \Hbar i \PD{t}{\psi}
\end{aligned}
\end{equation}

\item Klein-Gordon wave equation.

\begin{equation}\label{eqn:waveLagrangian:400}
\begin{aligned}
\LL &= -(\grad \psi) \cdot (\grad \psi^\conj) + \frac{m^2 c^2}{\Hbar^2} \psi \psi^\conj \\
-\grad^2 \psi &= \frac{m^2 c^2}{\Hbar^2} \psi
\end{aligned}
\end{equation}

\end{itemize}

%
% Copyright � 2012 Peeter Joot.  All Rights Reserved.
% Licenced as described in the file LICENSE under the root directory of this GIT repository.
%

\chapter{Short metric tensor explanation}
\index{metric tensor}
\label{chap:lorentzMetricTensor}
%\date{August 30, 2008.  lorentzMetricTensor.tex }

\section{}

\href{http://www.physicsforums.com/showthread.php?p=1853416}{PF thread.}

I have found it helpful to think about the metric tensor in terms of vector dot products, and a corresponding basis.

You can cut relativity completely out of the question, and ask the same question for Euclidean space, where the metric tensor it the identity matrix when you pick an orthonormal basis.

That diagonality is due to orthogonality conditions of the basis chosen.  For, example, in 3D we can express vectors in terms of an
orthonormal frame, but if we choose not to, say picking \(e_1 + e_2\), \(e_1-e_2\), and \(e_1 + e_3\) as our basis vectors then how do we calculate the coordinates?

The trick is to calculate, or assume calculated, an alternate set of basis vectors, called the reciprocal frame.  Provided the initial set of vectors spans the space, one can always calculate (and that part is a linear algebra exercise) this second pair such that they meet the following relationships:

\begin{equation*}
e^i \cdot e_j = {\delta^i}_j
\end{equation*}

So, if a vector is specified in terms of the \(e_i\)
\begin{equation*}
x = \sum e_j a_j
\end{equation*}

Dotting with \(e^i\) one has:

\begin{equation*}
x \cdot e^i = \sum (e_j a_j) \cdot e^i = \sum {\delta^j}_i a_j = a_i
\end{equation*}

It is customary to write \(a_i = x^i\), which allows for the entire vector to
be written in the mixed upper and lower index method where sums are assumed:

\begin{equation*}
x = \sum e_j x^j = e_j x^j
\end{equation*}

Now, if one calculates dot product here, say with \(x\), and a second vector

\begin{equation*}
y = \sum e_j y^j
\end{equation*}

you have:

\begin{equation*}
x \cdot y = \sum (e_j \cdot e_k) x^j y^k
\end{equation*}

The coefficient of this \(x^j y^k\) term is symmetric, and if you choose, you
can write \(g_{jk} = e_j \cdot e_k\), and you have the dot product in
tensor form:

\begin{equation*}
x \cdot y = \sum g_{jk} x^j y^k = g_{jk} x^j y^k
\end{equation*}

Now, for relativity, you have four instead of three basis vectors, so if you choose your spatial basis vectors orthonormally, and a timelike basis vector normal to all of those (ie: no mixing of space and time vectors in anything but a Lorentz fashion), then you get a diagonal metric tensor.  You can choose not to work in an "orthonormal" spacetime basis, and a non-diagonal metric tensor will show up in all your dot products.  That decision is perfectly valid, just makes everything harder.  When it comes down to why, it all boils down to your choice of basis.

Now, just like you can think of a rotation as a linear transformation that preserves angles in Euclidean space, the Lorentz transformation preserves the spacetime relationships appropriately.  So, if one transforms from a "orthonormal" spacetime frame to an alternate "orthonormal" spacetime frame (and a Lorentz transformation is just that) you still have the same "angles" (ie: dot products) between an event coordinates, and the metric will still be diagonal as described.  This could be viewed as just a rather long winded way of saying exactly what jdstokes said, but its the explanation coming from somebody who is also just learning this (so I had need such a longer explanation if I was explaining to myself).

%
% Copyright � 2012 Peeter Joot.  All Rights Reserved.
% Licenced as described in the file LICENSE under the root directory of this GIT repository.
%

\chapter{Hamiltonian notes}
\index{Hamiltonian}
\label{chap:hamiltonian}
%\blogpage{http://sites.google.com/site/peeterjoot/math2009/hamiltonian.pdf?revision=6}
%\date{Sept 26, 2009}

\section{Motivation}

I have now seen Hamiltonian's used, mostly in a Quantum context, and think that I understand at least some of the math associated with the Hamiltonian and the Hamiltonian principle.  I have, however, not used either of these enough that it seems natural to do so.

Here I attempt to summarize for myself what I know about Hamiltonian's, and work through a number of examples.  Some of the examples considered will be ones already treated with the Lagrangian formalism \bookchapcite{PJTongMf1}{phy354}.

Some notation will be invented along the way as reasonable, since I had like to try to also relate the usual coordinate representation of the Hamiltonian, the Hamiltonian principle, and the Poisson bracket, with the bivector representation of the 2N complex configuration space introduced in \citep{doran2003gap}. (NOT YET DONE).

\section{Hamiltonian as a conserved quantity}

Starting with the Lagrangian formalism the Hamiltonian can be found as a conserved quantity associated with time translation when the Lagrangian has no explicit time dependence.  This follows directly by considering the time derivative of the Lagrangian \(\LL = \LL(q^i, \qdot^i)\).

\begin{equation}\label{eqn:hamiltonian:63}
\begin{aligned}
\frac{d\LL}{dt}
&= \PD{q^i}{\LL} \frac{dq^i}{dt} +\PD{\qdot^i}{\LL} \frac{d\qdot^i}{dt} \\
&= \qdot^i \frac{d}{dt}\PD{\qdot^i}{\LL} +\PD{\qdot^i}{\LL} \frac{d\qdot^i}{dt} \\
&= \frac{d}{dt} \left( \qdot^i \PD{\qdot^i}{\LL} \right)
\end{aligned}
\end{equation}

We can therefore form the difference

\begin{equation}\label{eqn:hamiltonian:foo1}
\begin{aligned}
\frac{d}{dt} \left( \qdot^i \PD{\qdot^i}{\LL} -\LL \right) = 0
\end{aligned}
\end{equation}

and find that this quantity, labeled H, is a constant of motion for the system

\begin{equation}\label{eqn:hamiltonian:foo2}
\begin{aligned}
H \equiv \qdot^i \PD{\qdot^i}{\LL} -\LL = \text{constant}
\end{aligned}
\end{equation}

We will see later that this constant is sometimes the total energy of the system.

The \(\qdot^i\) partials of the Lagrangian are called the canonical momentum conjugate to \(q^i\).  Quite a mouthful, so just canonical momenta seems like a good compromise.  We will write (reserving \(p^i = m q^i\) for the non-canonical momenta)

\begin{equation}\label{eqn:hamiltonian:foo2b}
\begin{aligned}
P_i \equiv \PD{\qdot^i}{\LL}
\end{aligned}
\end{equation}

and note that these are the coordinates of a sort of velocity gradient of the Lagrangian.  We have seen these canonical momenta in velocity gradient form previously where it was noted that we could write the Euler-Lagrange equations in vector form in an orthonormal reciprocal frame space as

\begin{equation}\label{eqn:hamiltonian:foo9}
\begin{aligned}
\grad \LL = \frac{d}{dt} \grad_v \LL
\end{aligned}
\end{equation}

where \(\grad_v = e^i \partial \LL/\partial \xdot^i = e^i P_i\), \(\grad = e^i \partial/\partial x^i\), and \(x = e_i x^i\).

%Here we will be exploring phase space relationships where the position and velocity basis pairs are treated independently, but also will not have much requirement for direct use of the Euler Lagrange equations.

\section{Some syntactic sugar.  In vector form}

Following Jackson \citep{jackson1975cew} (section 12.1, relativistic Lorentz force Hamiltonian), this can be written in vector form if the velocity gradient, the vector sum of the momenta conjugate to the \(q^i\)'s is given its own symbol \(\BP\).  He writes

\begin{equation}\label{eqn:hamiltonian:foo3}
\begin{aligned}
H = \Bv \cdot \BP - \LL
\end{aligned}
\end{equation}

This makes most sense when working in orthonormal coordinates, but can be generalized.  Suppose we introduce a pair of reciprocal frame basis for the generalized position and velocity coordinates, writing as vectors in configuration space

\begin{equation}\label{eqn:hamiltonian:foo4}
\begin{aligned}
q &= e_i q^i \\
v &= f_i \qdot^i
\end{aligned}
\end{equation}

Following \citep{doran2003gap} (who use this for their bivector complexification of the configuration space), we have the freedom to impose orthonormal constraints on this configuration space basis

\begin{equation}\label{eqn:hamiltonian:foo5}
\begin{aligned}
e^i \cdot e_j &= {\delta^i}_j \\
f^i \cdot f_j &= {\delta^i}_j \\
e^i \cdot f_j &= {\delta^i}_j
\end{aligned}
\end{equation}

We can now define configuration space position and velocity gradients

\begin{equation}\label{eqn:hamiltonian:foo6}
\begin{aligned}
\grad &\equiv e^i \PD{q^i}{} \\
\grad_v &\equiv f^i \PD{\qdot^i}{}
\end{aligned}
\end{equation}

so the conjugate momenta in vector form is now

\begin{equation}\label{eqn:hamiltonian:foo7}
\begin{aligned}
P \equiv \grad_v \LL = f^i \PD{\qdot^i}{\LL}
\end{aligned}
\end{equation}

Our Hamiltonian takes the form

\begin{equation}\label{eqn:hamiltonian:foo8}
\begin{aligned}
H = v \cdot P - \LL
\end{aligned}
\end{equation}

\section{The Hamiltonian principle}

We want to take partials of \eqnref{eqn:hamiltonian:foo2} with respect to \(P_i\) and \(q^i\).  In terms of the canonical momenta we want to differentiate

\begin{equation}\label{eqn:hamiltonian:hoo1}
\begin{aligned}
H \equiv \qdot^i P_i -\LL(q^i, \qdot^i, t)
\end{aligned}
\end{equation}

for the \(P_i\) partial we have

\begin{equation}\label{eqn:hamiltonian:83}
\begin{aligned}
\PD{P_i}{H} = \qdot^i
\end{aligned}
\end{equation}

and for the \(q^i\) partial

\begin{equation}\label{eqn:hamiltonian:103}
\begin{aligned}
\PD{q^i}{H}
&= -\PD{q^i}{\LL} \\
&= - \frac{d}{dt} \PD{\qdot^i}{\LL}
\end{aligned}
\end{equation}

These two results taken together form what I believe is called the Hamiltonian principle

\begin{equation}\label{eqn:hamiltonian:hoo3}
\begin{aligned}
\PD{P_i}{H} &= \qdot^i \\
\PD{q^i}{H} &= - \dot{P}_i \\
P_i &= \PD{\qdot^i}{\LL}
\end{aligned}
\end{equation}

A set of 2N first order equations equivalent to the second order Euler-Lagrange equations.  These appear to follow straight from the definitions.  Given that I am curious why the more complex method of derivation is chosen in \citep{goldstein1951cm}.  There the total differential of the Hamiltonian is computed

\begin{equation}\label{eqn:hamiltonian:123}
\begin{aligned}
dH &=
\qdot^i dP_i
+ d\qdot^i P_i
- dq^i \PD{q^i}{\LL}
- d \qdot^i \PD{\qdot^i}{\LL}
- dt \PD{t}{\LL} \\
&=
\qdot^i dP_i
+ d\qdot^i \left( P_i - \PD{\qdot^i}{\LL} \right)
- dq^i \PD{q^i}{\LL}
- dt \PD{t}{\LL} \\
&=
\qdot^i dP_i
- dq^i
\mathLabelBox
[
   labelstyle={below of=m\themathLableNode, below of=m\themathLableNode}
]
{\PD{q^i}{\LL}}{\(= dP_i/dt\)}
- dt \PD{t}{\LL} \\
\end{aligned}
\end{equation}

A term by term comparison to the total differential written out explicitly

\begin{equation}\label{eqn:hamiltonian:hoo4}
\begin{aligned}
dH &=
\PD{q^i}{H} d q^i
+\PD{P_i}{H} d P_i
+\PD{t}{H} dt
\end{aligned}
\end{equation}

allows the Hamiltonian equations to be picked off.

\begin{equation}\label{eqn:hamiltonian:hoo5}
\begin{aligned}
\PD{P_i}{H} &= \qdot^i  \\
\PD{q^i}{H} &= - \dot{P}_i  \\
\PD{t}{H}   &= - \PD{t}{\LL}
\end{aligned}
\end{equation}

I guess that is not that much more complicated and it does yield a relation between the Hamiltonian and Lagrangian time derivatives.

\section{Examples}

Now, that is just about the most abstract way we can start things off is not it?  Getting some initial feel for this constant of motion can be had by considering a sequence of Lagrangians, starting with the very simplest.

\subsection{Force free motion}

Our very simplest Lagrangian is that of one dimensional purely kinetic motion

\begin{equation}\label{eqn:hamiltonian:boo1}
\begin{aligned}
\LL = \inv{2} m v^2 = \inv{2} m \xdot^2
\end{aligned}
\end{equation}

Our Hamiltonian is in this case just

\begin{equation}\label{eqn:hamiltonian:boo2}
\begin{aligned}
H = \xdot m \xdot - \inv{2} m \xdot = \inv{2} m v^2
\end{aligned}
\end{equation}

The Hamiltonian is just the kinetic energy.  The canonical momentum in this case is also equal to the momentum, so eliminating \(v\) to apply the Hamiltonian equations we have

\begin{equation}\label{eqn:hamiltonian:boo3}
\begin{aligned}
H = \inv{2m} p^2
\end{aligned}
\end{equation}

We have then

\begin{equation}\label{eqn:hamiltonian:143}
\begin{aligned}
\PD{p}{H} &= \frac{p}{m} = \dot{x} \\
\PD{x}{H} &= 0 = -\dot{p}
\end{aligned}
\end{equation}

Just for fun we can put this simple linear system in matrix form

\begin{equation}\label{eqn:hamiltonian:boo4}
\begin{aligned}
\frac{d}{dt}
\begin{bmatrix}
p \\
x
\end{bmatrix}
=
\inv{m}
\begin{bmatrix}
0 & 0 \\
1 & 0
\end{bmatrix}
\begin{bmatrix}
p \\
x
\end{bmatrix}
\end{aligned}
\end{equation}

A linear system of this form \(y' = A y\) can be solved by exponentiation with solution

\begin{equation}\label{eqn:hamiltonian:boo5}
\begin{aligned}
y = e^{A t} y_0
\end{aligned}
\end{equation}

In this case our matrix is nilpotent degree 2 so we can exponentiate only requiring up to the first order power

\begin{equation}\label{eqn:hamiltonian:boo6}
\begin{aligned}
e^{A t} = I + A t
\end{aligned}
\end{equation}

specifically

\begin{equation}\label{eqn:hamiltonian:boo7}
\begin{aligned}
\begin{bmatrix}
p \\
x
\end{bmatrix}
=
\begin{bmatrix}
1 & 0 \\
\frac{t}{m} & 1
\end{bmatrix}
\begin{bmatrix}
p_0 \\
x_0
\end{bmatrix}
\end{aligned}
\end{equation}

Written out in full this is just

\begin{equation}\label{eqn:hamiltonian:boo8}
\begin{aligned}
p &= p_0 \\
x &= \frac{p_0}{m} t + x_0
\end{aligned}
\end{equation}

Since the canonical momentum is the regular momentum \(p = m v\) in this case, we have the usual constant rate change of position \(x = v_0 t + x_0\) that we could have gotten in many easier ways.  I had hazard a guess that any single variable Lagrangian that is at most quadratic in position or velocity will yield a linear system.

The generalization of this Hamiltonian to three dimensions is straightforward, and we get

\begin{equation}\label{eqn:hamiltonian:boo9}
\begin{aligned}
H &= \inv{m} \Bp^2
\end{aligned}
\end{equation}

\begin{equation}\label{eqn:hamiltonian:boo10}
\begin{aligned}
\frac{d}{dt}
\begin{bmatrix}
p_x \\
x \\
p_y \\
y \\
p_z \\
z \\
\end{bmatrix}
=
\inv{m}
\begin{bmatrix}
0 & 0 &   &   &   &   \\
1 & 0 &   &   &   &   \\
  &   & 0 & 0 &   &   \\
  &   & 1 & 0 &   &   \\
  &   &   &   & 0 & 0 \\
  &   &   &   & 1 & 0 \\
\end{bmatrix}
\begin{bmatrix}
p_x \\
x \\
p_y \\
y \\
p_z \\
z \\
\end{bmatrix}
\end{aligned}
\end{equation}

Since there is no coupling (nilpotent matrices down the diagonal) between the coordinates this can be treated as three independent sets of equations of the form \eqnref{eqn:hamiltonian:boo4}, and we have

\begin{equation}\label{eqn:hamiltonian:boo11}
\begin{aligned}
p_i(t) &= p_i(0) \\
x_i(t) &= \frac{p_i(0)}{m} t + x_i(0)
\end{aligned}
\end{equation}

Or just

\begin{equation}\label{eqn:hamiltonian:boo12}
\begin{aligned}
\Bp(t) &= \Bp(0) \\
\Bx(t) &= \frac{\Bp(0)}{m} t + \Bx(0)
\end{aligned}
\end{equation}

\subsection{Linear potential (surface gravitation)}

For the gravitational force \(F = - m g \zcap = - \spacegrad \phi\), we have \(\phi = m g z\), and a Lagrangian of

\begin{equation}\label{eqn:hamiltonian:roo1}
\begin{aligned}
\LL = \inv{2} m \Bv^2 - \phi = \inv{2} m \Bv^2 - m g z
\end{aligned}
\end{equation}

Without velocity dependence the canonical momentum is the momentum \(m \Bv\), and our Hamiltonian is

\begin{equation}\label{eqn:hamiltonian:roo2}
\begin{aligned}
H = \inv{2 m} \Bp^2 + m g z
\end{aligned}
\end{equation}

The Hamiltonian equations are

\begin{equation}\label{eqn:hamiltonian:roo3}
\begin{aligned}
\PD{p_i}{H} &= \xdot_i = \inv{m} p_i \\
\sigma_i \PD{x_i}{H} &= -\sigma_i \pdot_i = \begin{bmatrix}0 \\ 0 \\ m g \end{bmatrix}
\end{aligned}
\end{equation}

In matrix form we have

\begin{equation}\label{eqn:hamiltonian:roo4}
\begin{aligned}
\frac{d}{dt}
\begin{bmatrix}
p_x \\
x \\
p_y \\
y \\
p_z \\
z \\
\end{bmatrix}
=
\inv{m}
\begin{bmatrix}
0 & 0 &   &   &   &   \\
1 & 0 &   &   &   &   \\
  &   & 0 & 0 &   &   \\
  &   & 1 & 0 &   &   \\
  &   &   &   & 0 & 0 \\
  &   &   &   & 1 & 0 \\
\end{bmatrix}
\begin{bmatrix}
p_x \\
x \\
p_y \\
y \\
p_z \\
z \\
\end{bmatrix}
+
\begin{bmatrix}
0 \\
0 \\
0 \\
0 \\
-m g \\
0 \\
\end{bmatrix}
\end{aligned}
\end{equation}

So our problem is now reduced to solving a linear system of the form

\begin{equation}\label{eqn:hamiltonian:roo5}
\begin{aligned}
y' = A y + b
\end{aligned}
\end{equation}

That extra little term \(b\) throws a wrench into things and I am no longer sure how to integrate by inspection.  What can be noted is that we really only have to consider the \(z\) components since we have solved the problem for the \(x\) and \(y\) coordinates in the force free case.  That leaves

\begin{equation}\label{eqn:hamiltonian:roo6}
\begin{aligned}
\frac{d}{dt}
\begin{bmatrix}
p_z \\
z \\
\end{bmatrix}
=
\inv{m}
\begin{bmatrix}
0 & 0 \\
1 & 0 \\
\end{bmatrix}
\begin{bmatrix}
p_z \\
z \\
\end{bmatrix}
+
\begin{bmatrix}
-m g \\
0 \\
\end{bmatrix}
\end{aligned}
\end{equation}

Is there any reason that we have to solve in matrix form?  Except for a coolness factor, not really, and we can integrate each equation directly.  For the momentum equation we have

\begin{equation}\label{eqn:hamiltonian:roo7}
\begin{aligned}
p_z = - m g t + p_z(0)
\end{aligned}
\end{equation}

This can be substituted into the position equation for

\begin{equation}\label{eqn:hamiltonian:roo8}
\begin{aligned}
\dot{z} = \inv{m} (p_z(0) - m g t)
\end{aligned}
\end{equation}

Direct integration is now possible for the final solution

\begin{equation}\label{eqn:hamiltonian:163}
\begin{aligned}
z
&= \inv{m} (p_z(0) t - m g t^2/2) + z_0 \\
&= \frac{p_z(0)}{m} t - \frac{g}{2} t^2 + z_0
\end{aligned}
\end{equation}

Again something that we could have gotten in many easier ways.  Using the result we see that the solution to \eqnref{eqn:hamiltonian:roo6} in matrix form, again with \(A = \inv{m}\begin{bmatrix}0 & 0 \\ 1 & 0\end{bmatrix}\) is

\begin{equation}\label{eqn:hamiltonian:roo9}
\begin{aligned}
\begin{bmatrix}
p_z \\
z \\
\end{bmatrix}
= e^{At}
\begin{bmatrix}
p_z(0) \\
z(0) \\
\end{bmatrix}
- m g
\begin{bmatrix}
t \\
\inv{2m} t^2
\end{bmatrix}
\end{aligned}
\end{equation}

I thought if I wrote this out how to solve \eqnref{eqn:hamiltonian:roo5} may be more obvious, but that path is still unclear.  If \(A\) were invertible, which it is not, then writing \(b = A c\) would allow for a change of variables.  Does this matter for consideration of a physical problem.  Not really, so I will fight the urge to play with the math for a while and perhaps revisit this later separately.

\subsection{Harmonic oscillator (spring potential)}

Like the free particle, the harmonic oscillator is very tractable in a phase space representation.  For a restoring force \(F = - k x \xcap = -\spacegrad \phi\), we have \(\phi = k x^2/2\), and a Lagrangian of

\begin{equation}\label{eqn:hamiltonian:woo1}
\begin{aligned}
\LL = \inv{2} m \Bv^2 - \inv{2} k \Bx^2
\end{aligned}
\end{equation}

Our Hamiltonian is again just the total energy

\begin{equation}\label{eqn:hamiltonian:woo2}
\begin{aligned}
H = \inv{2m} \Bp^2 + \inv{2} k \Bx^2
\end{aligned}
\end{equation}

Evaluating the Hamiltonian equations we have

\begin{equation}\label{eqn:hamiltonian:woo3}
\begin{aligned}
\PD{p_i}{H} &= \dot{x_i} = p_i/m \\
\PD{x_i}{H} &= -\dot{p_i} = k x_i
\end{aligned}
\end{equation}

Considering just the \(x\) dimension (the others have the free particle behavior), our matrix phase space representation is

\begin{equation}\label{eqn:hamiltonian:woo4}
\begin{aligned}
\frac{d}{dt}
\begin{bmatrix}
p \\
x \\
\end{bmatrix}
=
\begin{bmatrix}
0 & - k \\
1/m & 0 \\
\end{bmatrix}
\begin{bmatrix}
p \\
x \\
\end{bmatrix}
\end{aligned}
\end{equation}

So with

\begin{equation}\label{eqn:hamiltonian:woo5}
\begin{aligned}
A =
\begin{bmatrix}
0 & - k \\
1/m & 0 \\
\end{bmatrix}
\end{aligned}
\end{equation}

Our solution is

\begin{equation}\label{eqn:hamiltonian:woo6}
\begin{aligned}
\begin{bmatrix}
p \\
x \\
\end{bmatrix}
=
e^{At}
\begin{bmatrix}
p_0 \\
x_0 \\
\end{bmatrix}
\end{aligned}
\end{equation}

The stateful nature of the phase space solution is interesting.  Provided we can make a simultaneous measurement of position and momentum, this initial state determines a next position and momentum state at a new time \(t = t_0 + \Delta t_1\), and we have a trajectory through phase space of discrete transitions from one state to another

\begin{equation}\label{eqn:hamiltonian:woo6a}
\begin{aligned}
{\begin{bmatrix}
p \\
x \\
\end{bmatrix}}_{i+1}
=
e^{A \Delta t_{i+1}}
{\begin{bmatrix}
p \\
x \\
\end{bmatrix}}_i
\end{aligned}
\end{equation}

Or

\begin{equation}\label{eqn:hamiltonian:woo6b}
\begin{aligned}
{\begin{bmatrix}
p \\
x \\
\end{bmatrix}}_{i+1}
=
e^{A \Delta t_{i+1}} e^{A \Delta t_{i}} \cdots e^{A \Delta t_1}
{\begin{bmatrix}
p \\
x \\
\end{bmatrix}}_0
\end{aligned}
\end{equation}

As for solving the system, we require again the exponential of our matrix.  This matrix being antisymmetric, has complex eigenvalues and again cannot be exponentiated easily by diagonalization.  However,  this antisymmetric matrix is very much like the complex imaginary and its square is a negative scalar multiple of identity, so we can proceed directly forming the power series

\begin{equation}\label{eqn:hamiltonian:woo7}
\begin{aligned}
A^2 =
\begin{bmatrix}
0 & - k \\
1/m & 0 \\
\end{bmatrix}
\begin{bmatrix}
0 & - k \\
1/m & 0 \\
\end{bmatrix}
=
-\frac{k}{m} I
\end{aligned}
\end{equation}

The first few powers are
\begin{equation}\label{eqn:hamiltonian:woo8}
\begin{aligned}
A^2 &= -\frac{k}{m} I \\
A^3 &= -\frac{k}{m} A \\
A^4 &= \left(\frac{k}{m}\right)^2 I \\
A^5 &= \left(\frac{k}{m}\right)^2 A
\end{aligned}
\end{equation}

So exponentiating we can collect cosine and sine terms
\begin{equation}\label{eqn:hamiltonian:183}
\begin{aligned}
e^{At}
&= I \left( 1 - \frac{k}{m} \frac{t^2}{2!} + \left( \frac{k}{m} \right)^2 \frac{t^4}{4!} + \cdots \right)
+ A\sqrt{\frac{m}{k}} \left( \sqrt{\frac{k}{m}} - \left(\sqrt{\frac{k}{m}}\right)^3 \frac{t^3}{3!} + \left(\sqrt{\frac{k}{m}}\right)^5 \frac{t^5}{5!} \right) \\
&=
I \cos\left(\sqrt{\frac{k}{m}} t\right) + \sqrt{\frac{m}{k}} A \sin\left(\sqrt{\frac{k}{m}} t\right)
\end{aligned}
\end{equation}

As a check it is readily verified that this satisfies the desired \(d(e^{At})/dt = A e^{At}\) property.

The full solution in phase space representation is therefore

\begin{equation}\label{eqn:hamiltonian:woo9}
\begin{aligned}
\begin{bmatrix}
p \\
x \\
\end{bmatrix}
=
\begin{bmatrix}
p_0 \\
x_0 \\
\end{bmatrix}
\cos\left(\sqrt{\frac{k}{m}} t\right)
+ \sqrt{\frac{m}{k}}
\begin{bmatrix}
-k x_0 \\
p_0/m \\
\end{bmatrix}
\sin\left(\sqrt{\frac{k}{m}} t\right)
\end{aligned}
\end{equation}

Written out separately this is clearer

\begin{equation}\label{eqn:hamiltonian:woo10}
\begin{aligned}
p &= p_0 \cos\left(\sqrt{\frac{k}{m}} t\right) - \sqrt{\frac{m}{k}} k x_0 \sin\left(\sqrt{\frac{k}{m}} t\right) \\
x &= x_0 \cos\left(\sqrt{\frac{k}{m}} t\right) + \sqrt{\frac{m}{k}} \frac{p_0}{m} \sin\left(\sqrt{\frac{k}{m}} t\right)
\end{aligned}
\end{equation}

One can readily verify that \(m \xdot = p\), and \(m \ddot{x} = -k x\) as expected.

Let us pause before leaving the harmonic oscillator to see if \eqnref{eqn:hamiltonian:woo10} seems to make sense.  Consider the position solution.  With only initial position and no initial velocity \(p_0/m\) we have oscillation that has no dependence on the mass or spring constant.  This is the unmoving mass about to be let go at the end of a spring case, and since we have no damping force the magnitude of the oscillation is exactly the initial position of the mass.  If the instantaneous velocity is measured at position zero, it makes sense in this case that the oscillation amplitude does depend on both the mass and the spring constant.  The stronger the spring (\(k\)), the bigger the oscillation, and the smaller the mass, the bigger the oscillation.

It is definitely no easier to work with the phase space formulation than just solving the second order system directly.  The fact that we have a linear system to solve, at least in this particular case is kind of nice.  Perhaps this methodology can be helpful considering linear approximation solutions in a neighborhood of some phase space point for more complex non-linear systems.

\subsection{Harmonic oscillator (change of variables.)}

It was pointed out to me by Lut that the following rather strange looking change of variables has nice properties

\begin{equation}\label{eqn:hamiltonian:zoo1}
\begin{aligned}
P &= x \sqrt{\frac{k}{2}} + \frac{ p }{\sqrt{2 m}} \\
Q &= x \sqrt{\frac{k}{2}} - \frac{ p }{\sqrt{2 m}}
\end{aligned}
\end{equation}

In particular the Hamiltonian is then just

\begin{equation}\label{eqn:hamiltonian:zoo2}
\begin{aligned}
H = P^2 + Q^2
\end{aligned}
\end{equation}

Part of this change of variables, which rotates in phase space, as well as scales, looks like just a way of putting the system into natural units.  We do not however, need the rotation to do that.  Suppose we write for just the scaling change of variables

\begin{equation}\label{eqn:hamiltonian:zoo3}
\begin{aligned}
p &= \sqrt{2m} P_s \\
x &= \sqrt{\frac{2}{k}} Q_s
\end{aligned}
\end{equation}

or

\begin{equation}\label{eqn:hamiltonian:zoo5}
\begin{aligned}
\begin{bmatrix}
p \\
x \\
\end{bmatrix}
=
\begin{bmatrix}
\sqrt{2m} & 0 \\
0 & \sqrt{\frac{2}{k}}
\end{bmatrix}
\begin{bmatrix}
P_s \\
Q_s \\
\end{bmatrix}
\end{aligned}
\end{equation}

This also gives the Hamiltonian \eqnref{eqn:hamiltonian:zoo2}, and the Hamiltonian equations are transformed to

\begin{equation}\label{eqn:hamiltonian:203}
\begin{aligned}
\frac{d}{dt}
\begin{bmatrix}
P_s \\
Q_s \\
\end{bmatrix}
&=
\begin{bmatrix}
1/\sqrt{2m} & 0 \\
0 & \sqrt{\frac{k}{2}}
\end{bmatrix}
\begin{bmatrix}
0 & - k \\
1/m & 0 \\
\end{bmatrix}
\begin{bmatrix}
\sqrt{2m} & 0 \\
0 & \sqrt{\frac{2}{k}}
\end{bmatrix}
\begin{bmatrix}
P_s \\
Q_s \\
\end{bmatrix} \\
&=
\begin{bmatrix}
0 & - \sqrt{\frac{k}{m}} \\
\sqrt{\frac{k}{m}} & 0 \\
\end{bmatrix}
\begin{bmatrix}
P_s \\
Q_s \\
\end{bmatrix}
\end{aligned}
\end{equation}

This first change of variables is nice since it groups the two factors \(k\) and \(m\) into a reciprocal pair.  Since only the ratio is significant to the kinetics it is nice to have that explicit.  Since \(\sqrt{k/m}\) is in fact the angular frequency, we can define

\begin{equation}\label{eqn:hamiltonian:zoo6}
\begin{aligned}
\omega \equiv \sqrt{\frac{k}{m}}
\end{aligned}
\end{equation}

and our system is reduced to

\begin{equation}\label{eqn:hamiltonian:zoo7}
\begin{aligned}
\frac{d}{dt}
\begin{bmatrix}
P_s \\
Q_s \\
\end{bmatrix}
&=
\omega
\begin{bmatrix}
0 & -1 \\
1 & 0
\end{bmatrix}
\begin{bmatrix}
P_s \\
Q_s \\
\end{bmatrix}
\end{aligned}
\end{equation}

Solution of this system now becomes particularly easy, especially if one notes that the matrix factor above can be expressed in terms of the \(y\) axis Pauli matrix \(\sigma_2\).  That is

\begin{equation}\label{eqn:hamiltonian:zoo8}
\begin{aligned}
\sigma_2 =
i
\begin{bmatrix}
0 & -1 \\
1 & 0
\end{bmatrix}
\end{aligned}
\end{equation}

Inverting this, and labeling this matrix \(\calI\) we can write

\begin{equation}\label{eqn:hamiltonian:zoo9}
\begin{aligned}
\calI \equiv
\begin{bmatrix}
0 & -1 \\
1 & 0
\end{bmatrix}
=
-i \sigma_2
\end{aligned}
\end{equation}

Recalling that \(\sigma_2^2 = I\), we then have \(\calI^2 = -I\), and see that this matrix behaves exactly like a unit imaginary.  This reduces the Hamiltonian system to

\begin{equation}\label{eqn:hamiltonian:zoo10}
\begin{aligned}
\frac{d}{dt}
\begin{bmatrix}
P_s \\
Q_s \\
\end{bmatrix}
&=
\calI \omega
\begin{bmatrix}
P_s \\
Q_s \\
\end{bmatrix}
\end{aligned}
\end{equation}

We can now solve the system directly.  Writing \(\Bz_s = \bigl(\begin{smallmatrix} P_s \\ Q_s \\ \end{smallmatrix}\bigr)\), this is just

\begin{equation}\label{eqn:hamiltonian:zoo11}
\begin{aligned}
\Bz_s(t)
=
e^{\calI \omega t} \Bz_s(0)
&=
\left( I \cos(\omega t) + \calI \sin(\omega t) \right) \Bz_s(0)
\end{aligned}
\end{equation}

With just the scaling giving both the simple Hamiltonian, and a simple solution, what is the advantage of the further change of variables that mixes (rotates in phase space by 45 degrees with a factor of two scaling) the momentum and position coordinates?  That second transformation is

\begin{equation}\label{eqn:hamiltonian:zoo1a}
\begin{aligned}
P &= Q_s + P_s \\
Q &= Q_s - P_s
\end{aligned}
\end{equation}

Inverting this we have

\begin{equation}\label{eqn:hamiltonian:zoo12}
\begin{aligned}
\begin{bmatrix}
P_s \\
Q_s \\
\end{bmatrix}
=
\inv{2}
\begin{bmatrix}
1 & -1 \\
1 & 1 \\
\end{bmatrix}
\begin{bmatrix}
P \\
Q \\
\end{bmatrix}
\end{aligned}
\end{equation}

The Hamiltonian after this change of variables is now

\begin{equation}\label{eqn:hamiltonian:zoo13}
\begin{aligned}
\frac{d}{dt}
\begin{bmatrix}
P \\
Q \\
\end{bmatrix}
&=
\frac{\omega}{2}
\begin{bmatrix}
1 & 1 \\
-1 & 1 \\
\end{bmatrix}
\begin{bmatrix}
0 & -1 \\
1 & 0 \\
\end{bmatrix}
\begin{bmatrix}
1 & -1 \\
1 & 1 \\
\end{bmatrix}
\begin{bmatrix}
P \\
Q \\
\end{bmatrix}
\end{aligned}
\end{equation}

But multiplying this out one finds that the equations of motion for the state space vector are unchanged by the rotation, and writing \(\Bz = \bigl(\begin{smallmatrix} P \\ Q \\ \end{smallmatrix}\bigr)\) for the state vector, the Hamiltonian equations are

\begin{equation}\label{eqn:hamiltonian:zoo14}
\begin{aligned}
\Bz' = \calI \omega \Bz
\end{aligned}
\end{equation}

This is just as we had before the rotation-like mixing of position and momentum coordinates.  Now it looks like the rotational change of coordinates is related to the raising and lowering operators in the Quantum treatment of the Harmonic oscillator, but it is not clear to me what the advantage is in the classical context?  Perhaps the point is, that at least for the classical Harmonic oscillator, we are free to rotate the phase space vector arbitrarily and not change the equations of motion.  A restriction to the classical domain is required since squaring the results of this 45 degree rotation of the phase space vector requires commutation of the position and momentum coordinates in order for the cross terms to cancel out.

Is there a deeper meaning to this rotational freedom?  It seems to me that one ought to be able to relate the rotation and the quantum ladder operators in a more natural way, but it is not clear to me exactly how.

\subsection{Force free system dependent on only differences}

In gravitational and electrostatic problems are forces are all functions of only the difference in positions of the particles.  Lets look at how the purely kinetic Lagrangian and Hamiltonian change when one or more of the vector positions is reexpressed in terms of a difference in position change of variables.  In the force free case this is primarily a task of rewriting the Hamiltonian in terms of the conjugate momenta after such a change of variables.

The very simplest case is the two particle single dimensional Kinetic Lagrangian,

\begin{equation}\label{eqn:hamiltonian:noo1}
\begin{aligned}
\LL = \inv{2} m_1 {\rdot_1}^2 + \inv{2} m_2 {\rdot_2}^2
\end{aligned}
\end{equation}

With a change of variables

\begin{equation}\label{eqn:hamiltonian:noo2}
\begin{aligned}
x &= r_2 - r_1 \\
y &= r_2
\end{aligned}
\end{equation}

and elimination of \(r_1\), and \(r_2\) we have

\begin{equation}\label{eqn:hamiltonian:noo3}
\begin{aligned}
\LL = \inv{2} m_1 (\ydot - \xdot)^2 + \inv{2} m_2 {\ydot}^2
\end{aligned}
\end{equation}

We now need the conjugate momenta in terms of \(\xdot\) and \(\ydot\).  These are

\begin{equation}\label{eqn:hamiltonian:noo4}
\begin{aligned}
P_x &= \PD{\xdot}{\LL} = -m_1 (\ydot - \xdot) \\
P_y &= \PD{\ydot}{\LL} = m_1 (\ydot - \xdot) + m_2 \ydot
\end{aligned}
\end{equation}

We must now rewrite the Lagrangian in terms of \(P_x\) and \(P_y\), essentially requiring the inversion of this which amounts to the inversion of the two by two linear system of \eqnref{eqn:hamiltonian:noo4}.  That is

\begin{equation}\label{eqn:hamiltonian:noo5}
\begin{aligned}
\begin{bmatrix}
\xdot \\
\ydot
\end{bmatrix}
=
{\begin{bmatrix}
m_1 & -m_1 \\
-m_1 & (m_1 + m_2) \\
\end{bmatrix}}^{-1}
\begin{bmatrix}
P_x \\
P_y
\end{bmatrix}
\end{aligned}
\end{equation}

This is

\begin{equation}\label{eqn:hamiltonian:noo6}
\begin{aligned}
\begin{bmatrix}
\xdot \\
\ydot
\end{bmatrix}
=
\inv{m_1}\inv{m_2}
\begin{bmatrix}
m_1 + m_2 & m_1 \\
m_1 & m_1 \\
\end{bmatrix}
\begin{bmatrix}
P_x \\
P_y
\end{bmatrix}
\end{aligned}
\end{equation}

Of these only \(\ydot\) and \(\ydot - \xdot\) are of interest and after a bit of manipulation we find

\begin{equation}\label{eqn:hamiltonian:noo7}
\begin{aligned}
\ydot &= \inv{m_2}(P_x + P_y) \\
\xdot &= \inv{m_1}\inv{m_2}((m_1 + m_2)P_x + m_1 P_y)
\end{aligned}
\end{equation}

From this we find the Lagrangian in terms of the conjugate momenta

\begin{equation}\label{eqn:hamiltonian:noo8}
\begin{aligned}
\LL = \inv{2 m_1} {P_x}^2 + \inv{2 m_2} (P_x + P_y)^2
\end{aligned}
\end{equation}

A quick check shows that \(P_x + P_y = m_2 \rdot_2\), and \(P_x = -m_1 \rdot_1\), so we have agreement with the original Lagrangian.  Generalizing to the three dimensional case is straightforward, and we have

\begin{equation}\label{eqn:hamiltonian:noo9}
\begin{aligned}
\LL &= \inv{2} m_1 {\mathbf{\rdot}_1}^2 + \inv{2} m_2 {\mathbf{\rdot}_2}^2 - \phi(\Bx_1 - \Bx_2)
\end{aligned}
\end{equation}

With
\begin{equation}\label{eqn:hamiltonian:noo10}
\begin{aligned}
\Bx &= \Bx_1 - \Bx_2 \\
\By &= \Bx_2
\end{aligned}
\end{equation}

The 3D generalization of the above (following by adding indices then summing) becomes

\begin{equation}\label{eqn:hamiltonian:noo11}
\begin{aligned}
\BP_x &= \sigma_j \PD{\xdot^j}{\LL} = -m_1 (\mathbf{\ydot} - \mathbf{\xdot}) \\
\BP_y &= \sigma_j \PD{\ydot^j}{\LL} = m_1 (\mathbf{\ydot} - \mathbf{\xdot}) + m_2 \mathbf{\ydot}
\end{aligned}
\end{equation}

\begin{equation}\label{eqn:hamiltonian:noo12}
\begin{aligned}
\LL &= \inv{2 m_1} {\BP_x}^2 + \inv{2 m_2} (\BP_x + \BP_y)^2 - \phi(\Bx) \\
H &= \inv{2 m_1} {\BP_x}^2 + \inv{2 m_2} (\BP_x + \BP_y)^2 + \phi(\Bx)
\end{aligned}
\end{equation}

Finally, evaluation of the Hamiltonian equations we have

\begin{equation}\label{eqn:hamiltonian:223}
\begin{aligned}
\sigma_j \PD{P_x^j}{H} &= \mathbf{\xdot} \\
&= \sigma_j \left( \inv{m_1} P_x^j + \inv{m_2} (P_x^j + P_y^j) \right) \\
&= \inv{m_1} \BP_x + \inv{m_2} (\BP_x + \BP_y)
\end{aligned}
\end{equation}

\begin{equation}\label{eqn:hamiltonian:243}
\begin{aligned}
\sigma_j \PD{P_y^j}{H} &= \mathbf{\ydot} \\
&= \sigma_j \inv{m_2} (P_x^j + P_y^j) \\
&= \inv{m_2} (\BP_x + \BP_y)
\end{aligned}
\end{equation}

\begin{equation}\label{eqn:hamiltonian:263}
\begin{aligned}
\sigma_j \PD{x^j}{H} &= -\mathbf{\dot{P}}_x \\
&= -\sigma_j \PD{x^j}{\phi} \\
&= -\spacegrad_{\Bx} \phi(\Bx)
\end{aligned}
\end{equation}

\begin{equation}\label{eqn:hamiltonian:283}
\begin{aligned}
\sigma_j \PD{y^j}{H} &= -\mathbf{\dot{P}}_y \\
&= -\sigma_j \PD{y^j}{\phi} \\
&= 0
\end{aligned}
\end{equation}

Summarizing we have four first order equations

\begin{equation}\label{eqn:hamiltonian:noo13}
\begin{aligned}
\mathbf{\xdot} &= \left( \inv{m_1} + \inv{m_2} \right) \BP_x + \inv{m_2} \BP_y \\
\mathbf{\ydot} &= \inv{m_2} (\BP_x + \BP_y)  \\
\mathbf{\dot{P}}_x &= \spacegrad_{\Bx} \phi(\Bx) \\
\mathbf{\dot{P}}_y &= 0
\end{aligned}
\end{equation}

FIXME: what would we get if using the center of mass position as one of the variables.  A parametrization with three vector variables should also still work, even if it includes additional redundancy.

\subsection{Gravitational potential}

Next I had like to consider a two particle gravitational interaction.  However, to start we need the Lagrangian, and what should the potential term be in a two particle gravitational Lagrangian?  I had guess something with a \(1/x\) form, but do we need one potential term for each mass or something interrelated?  Whatever the Lagrangian is, we want it to produce the pair of force relationships

\begin{equation}\label{eqn:hamiltonian:moo1}
\begin{aligned}
\text{Force on 2} &= - G m_1 m_2 \frac{(\Br_2 - \Br_1)}{\Abs{\Br_2 - \Br_1}} \\
\text{Force on 1} &= G m_1 m_2 \frac{(\Br_2 - \Br_1)}{\Abs{\Br_2 - \Br_1}}
\end{aligned}
\end{equation}

Guessing that the Lagrangian has a single term for the interaction potential

\begin{equation}\label{eqn:hamiltonian:moo2}
\begin{aligned}
\phi_{21} &= \kappa \inv{\Abs{\Br_2 - \Br_1}}
\end{aligned}
\end{equation}

so that we have
\begin{equation}\label{eqn:hamiltonian:moo3}
\begin{aligned}
\LL &= \inv{2} m {\Bv_1}^2 +\inv{2} m {\Bv_2}^2 - \phi_{21}
\end{aligned}
\end{equation}

We can evaluate the Euler-Lagrange equations and see if the result is consistent with the Newtonian force laws of \eqnref{eqn:hamiltonian:moo1}.  Suppose we write the coordinates of \(\Br_i\) as \({x^k}_i\).  There are then six Euler-Lagrange equations

\begin{equation}\label{eqn:hamiltonian:303}
\begin{aligned}
\PD{{x^j}_i}{\LL} &= \frac{d}{dt} \PD{{\dot{x}^j}_i}{\LL} \\
-\PD{{x^j}_i}{\phi_{21}} &= m_i {\ddot{x}^j}_i
\end{aligned}
\end{equation}

Evaluating the potential derivatives separately.  Consider the \(i=2\) derivative

\begin{equation}\label{eqn:hamiltonian:323}
\begin{aligned}
\PD{{x^j}_2}{\phi_{21}}
&=
\kappa \PD{{x^j}_2}{} \left( \sum_k ({x^k}_2 - {x^k}_1)^2 \right)^{-1/2} \\
&=
-\kappa \inv{\Abs{\Br_2 - \Br_1}^3} \sum_k ({x^k}_2 - {x^k}_1) \PD{{x^j}_2}{} ({x^k}_2 - {x^k}_1) \\
&=
-\kappa \inv{\Abs{\Br_2 - \Br_1}^3} ({x^j}_2 - {x^j}_1)
\end{aligned}
\end{equation}

%Somewhat problematic in this derivative is the fact that the derivative of an absolute or square is multivalued.  Consider the graph of \(y = (x-a)^2\), the derivative is \(y' = \pm 2 (x-a)\) depending upon which side of the parabola one is evaluating the slope.  It appears that everything will work out, obtaining the Newtonian gravitation equations for two particle interaction with either sign, but we have to use the same sign for all derivatives.  With this sign picked arbitrarily following the sense convention of this above calculation, we have from the Euler-Lagrange equations

Therefore the final result of the Euler-Lagrange equations is

\begin{equation}\label{eqn:hamiltonian:moo4}
\begin{aligned}
\kappa \inv{\Abs{\Br_2 - \Br_1}^3} ({x^j}_2 - {x^j}_1) &= m_2 {\ddot{x}^j}_2 \\
-\kappa \inv{\Abs{\Br_2 - \Br_1}^3} ({x^j}_2 - {x^j}_1) &= m_1 {\ddot{x}^j}_1
\end{aligned}
\end{equation}

which confirms the Lagrangian and potential guess and fixes the constant \(\kappa = - G m_1 m_2\).  With the sign fixed, our potential, Lagrangian, and Hamiltonian are respectively

\begin{equation}\label{eqn:hamiltonian:moo5}
\begin{aligned}
\phi_{21} &= -\frac{G m_2 m_1 }{\Abs{\Br_2 - \Br_1}}  \\
\LL &= \inv{2} m_1 {\Bv_1}^2 +\inv{2} m_2 {\Bv_2}^2 - \phi_{21} \\
H &= \inv{2 m_1} {\Bp_1}^2 +\inv{2 m_2} {\Bp_2}^2 + \phi_{21}
\end{aligned}
\end{equation}

There is however an undesirable asymmetry to this expression, in particular a formulation that extends to multiple particles seems desirable.  Let us write instead a slight variation

\begin{equation}\label{eqn:hamiltonian:moo6}
\begin{aligned}
\phi_{ij} = - \frac{G m_i m_j}{\Abs{\Br_i - \Br_j}}
\end{aligned}
\end{equation}

and form a scaled by two double summation over all pairs of potentials

\begin{equation}\label{eqn:hamiltonian:moo7}
\begin{aligned}
\LL &= \sum_i \inv{2} m_i {\Bv_i}^2 - \inv{2} \sum_{i \ne j} \phi_{ij}
\end{aligned}
\end{equation}

%Again since \(\), for this to have meaning an agreement to evaluate the Euler-Lagrange partials according to the following ``positive'' sense is required
%
%\begin{align}\label{eqn:hamiltonian:moo8}
%\PD{{x^j}_i}{} (\Br_a - \Br_b)^2 \equiv (\Br_a - \Br_b) \cdot \PD{{x^j}_i}{} (\Br_a - \Br_b)
%\end{align}

Having established what seems like an appropriate form for the Lagrangian, we can write the Hamiltonian for the multiparticle gravitational interaction by inspection

\begin{equation}\label{eqn:hamiltonian:moo9}
\begin{aligned}
H = \sum_i \inv{2 m_i} {\Bp_i}^2 + \inv{2} \sum_{i \ne j} \phi_{ij}
\end{aligned}
\end{equation}

This leaves us finally in position to evaluate the Hamiltonian equations, but the result of doing so is rudely nothing more than the Newtonian equations in coordinate form.  We get, for the \(k\)th component of the \(i\)th particle

\begin{equation}\label{eqn:hamiltonian:moo10}
\begin{aligned}
\PD{{p^k}_i}{H} = {\dot{x}^k}_i = \inv{m_i} {p^k}_i
\end{aligned}
\end{equation}

\begin{equation}\label{eqn:hamiltonian:moo11}
\begin{aligned}
\PD{{x^k}_i}{H} = -{\dot{p}^k}_i = G \sum_{j \ne i} m_i m_j \frac{{x^k}_i -{x^k}_j}{\Abs{\Br_i - \Br_j}^3}
\end{aligned}
\end{equation}

The state space vector for this system of equations is brutally ugly, and could be put into the following form for example

\begin{equation}\label{eqn:hamiltonian:moo12}
\begin{aligned}
\Bz =
\begin{bmatrix}
{p^1}_1 \\
{p^2}_1 \\
{p^3}_1 \\
{x^1}_1 \\
{x^2}_1 \\
{x^3}_1 \\
{p^1}_2 \\
{p^2}_2 \\
{p^3}_2 \\
{x^1}_2 \\
\vdots \\
\end{bmatrix}
\end{aligned}
\end{equation}

Where the Hamiltonian equations take the form of a non-linear function on such state space vectors
We have a somewhat sparse equation of the form

\begin{equation}\label{eqn:hamiltonian:moo13}
\begin{aligned}
\frac{d\Bz}{dt} = A(\Bz)
\end{aligned}
\end{equation}

One thing that is possible in such a representation is calculating the first order approximate change in position and momentum moving from one time to a small time later

\begin{equation}\label{eqn:hamiltonian:moo14}
\begin{aligned}
\Bz(t_0 + \Delta t) = \Bz(t_0) + A(\Bz(t_0)) \Delta t
\end{aligned}
\end{equation}

One could conceivably calculate the trajectories in phase space using such increments, and if a small enough time increment is used this can be thought of as solving the gravitational system.  I recall that Feynman did something like this in his lectures, but set up the problem in a more computationally efficient form (it definitely did not have the redundancy built into the Hamiltonian equations).

FIXME: should be able to solve this for an arbitrary \(\Delta t\) later time if this was extended to the higher order terms.  Need something like the \(e^{z \cdot \grad}\) chain rule expansion.  Think this through.  Will be a little different since we are already starting with the first order contribution.

What does this system of equations look like with a reduction of order through center of mass change of variables?
% Lut was exploring this and having trouble.

\subsection{Pendulum}

FIXME: picture.  x-axis down, y-axis right.

%For the position of the pendulum bob, lets write
%
%\begin{align}\label{eqn:hamiltonian:yoo1}
%\Bx &= r \xcap e^{i \theta} \\
%%\Bv &= \xcap ( \rdot + r i \thetadot ) e^{i \theta} \\
%i &= \xcap \ycap
%\end{align}

The bob speed for a stiff rod of length \(l\) is \((l \thetadot)^2\), and our potential is \(m g h = m g l (1 -\cos\theta)\).  The Lagrangian is therefore

\begin{equation}\label{eqn:hamiltonian:yoo2}
\begin{aligned}
\LL = \inv{2} m l^2 \thetadot^2 - mg l (1 -\cos\theta) \\
\end{aligned}
\end{equation}

The constant \(m g l\) term can be dropped, and our canonical momentum conjugate to \(\thetadot\) is \(p_\theta = m l^2 \thetadot\), so our Hamiltonian is

\begin{equation}\label{eqn:hamiltonian:yoo3}
\begin{aligned}
H = \inv{2 m l^2} {p_\theta}^2 - mg l \cos\theta \\
\end{aligned}
\end{equation}

We can now compute the Hamiltonian equations

\begin{equation}\label{eqn:hamiltonian:yoo4}
\begin{aligned}
\PD{p_\theta}{H}  &= \thetadot         = \inv{ m l^2 } p_\theta \\
\PD{q}{H}         &= -\dot{p}_\theta   = m g l \sin\theta
\end{aligned}
\end{equation}

Only in the neighborhood of a particular angle can we write this in matrix form.  Suppose we expand this around \(\theta = \theta_0 + \alpha\).  The sine is then

\begin{equation}\label{eqn:hamiltonian:yoo5}
\begin{aligned}
\sin\theta \approx \sin\theta_0 + \cos\theta_0 \alpha
\end{aligned}
\end{equation}

The linear approximation of the Hamiltonian equations after a change of variables become

\begin{equation}\label{eqn:hamiltonian:yoo6}
\begin{aligned}
\frac{d}{dt}
\begin{bmatrix}
p_\theta \\
\alpha
\end{bmatrix}
=
\begin{bmatrix}
0 & -m g l \cos\theta_0 \\
1/ m l^2 & 0
\end{bmatrix}
\begin{bmatrix}
p_\theta \\
\alpha
\end{bmatrix}
+
\begin{bmatrix}
-
m g l \sin\theta_0 \\
\dot{\theta}_0
\end{bmatrix}
\end{aligned}
\end{equation}

A change of variables that scales the factors in the matrix to have equal magnitude and equivalent dimensions is helpful.  Writing

\begin{equation}\label{eqn:hamiltonian:yoo7}
\begin{aligned}
\begin{bmatrix}
p_\theta \\
\alpha
\end{bmatrix}
=
\begin{bmatrix}
a & 0 \\
0 & 1
\end{bmatrix}
\Bz
\end{aligned}
\end{equation}

one finds

\begin{equation}\label{eqn:hamiltonian:yoo8a}
\begin{aligned}
\frac{d\Bz}{dt}
%&=
%\begin{bmatrix}
%1/a & 0 \\
%0 & 1
%\end{bmatrix}
%\begin{bmatrix}
%0 & -m g l \cos\theta_0 \\
%1/ m l^2 & 0
%\end{bmatrix}
%\begin{bmatrix}
%a & 0 \\
%0 & 1
%\end{bmatrix}
%\Bz
%-
%\frac{m g l \sin\theta_0 }{a}
%\begin{bmatrix}
%1 \\
%0
%\end{bmatrix} \\
%&=
%\begin{bmatrix}
%0 & -m g l \cos\theta_0/a \\
%1/ m l^2 & 0
%\end{bmatrix}
%\begin{bmatrix}
%a & 0 \\
%0 & 1
%\end{bmatrix}
%\Bz
%-
%\frac{m g l \sin\theta_0 }{a}
%\begin{bmatrix}
%1 \\
%0
%\end{bmatrix} \\
&=
\begin{bmatrix}
0 & -m g l \cos\theta_0/a \\
a/ m l^2 & 0
\end{bmatrix}
\Bz
+
\inv{a}
\begin{bmatrix}
- m g l \sin\theta_0 \\
\dot{\theta}_0
\end{bmatrix}
\end{aligned}
\end{equation}

To tidy this up, we want

\begin{equation}\label{eqn:hamiltonian:yoo9}
\begin{aligned}
\Abs{\frac{a}{m l^2}} = \Abs{\frac{m g l \cos\theta_0}{a}}
%a^2 = m^2 l^4 \frac{g}{l} \cos\theta_0
\end{aligned}
\end{equation}

Or
\begin{equation}\label{eqn:hamiltonian:yoo9b}
\begin{aligned}
a = m l^2 \sqrt{\frac{g}{l} \Abs{\cos\theta_0}}
\end{aligned}
\end{equation}

The result of applying this scaling is quite different above and below the horizontal due to the sign difference in the cosine.  Below the horizontal where \(\theta_0 \in (-\pi/2, \pi/2)\) we get

\begin{equation}\label{eqn:hamiltonian:yoo11}
\begin{aligned}
\frac{d\Bz}{dt}
&=
\sqrt{\frac{g}{l} \cos\theta_0}
\begin{bmatrix}
0 & -1 \\
1 & 0
\end{bmatrix}
\Bz
+ \inv{m l^2 \sqrt{\frac{g}{l} \cos\theta_0}}
\begin{bmatrix}
- m g l \sin\theta_0 \\
\dot{\theta}_0
\end{bmatrix}
\end{aligned}
\end{equation}

and above the horizontal where \(\theta_0 \in (\pi/2, 3\pi/2)\) we get

\begin{equation}\label{eqn:hamiltonian:yoo12}
\begin{aligned}
\frac{d\Bz}{dt}
&=
\sqrt{-\frac{g}{l} \cos\theta_0}
\begin{bmatrix}
0 & 1 \\
1 & 0
\end{bmatrix}
\Bz
+ \inv{m l^2 \sqrt{-\frac{g}{l} \cos\theta_0}}
\begin{bmatrix}
- m g l \sin\theta_0 \\
\dot{\theta}_0
\end{bmatrix}
\end{aligned}
\end{equation}

Since \(\Bigl(\begin{smallmatrix} 0 & -1 \\ 1 & 0 \end{smallmatrix}\Bigl)\) has the characteristics of an imaginary number (squaring to the negative of the identity) the homogeneous part of the solution for the change of the phase space vector in the vicinity of any initial angle in the lower half plane is trigonometric.  Similarly the solutions are necessarily hyperbolic in the upper half plane since \(\Bigl(\begin{smallmatrix} 0 & 1 \\ 1 & 0 \end{smallmatrix}\Bigl)\) squares to identity.  And around \(\pm \pi/2\) something totally different (return to this later).  The problem is now reduced to solving a non-homogeneous first order matrix equation of the form

\begin{equation}\label{eqn:hamiltonian:yoo13}
\begin{aligned}
\Bz' = \Omega \Bz + \Bb
\end{aligned}
\end{equation}

But we have the good fortune of being able to easily exponentiate and invert this matrix \(\Omega\).  The homogeneous problem

\begin{equation}\label{eqn:hamiltonian:yoo13h}
\begin{aligned}
\Bz' = \Omega \Bz
\end{aligned}
\end{equation}

has the solution

\begin{equation}\label{eqn:hamiltonian:yoo14}
\begin{aligned}
\Bz_h(t) = e^{\Omega t} \Bz_{t=0}
\end{aligned}
\end{equation}

Assuming a specific solution \(z = e^{\Omega t}f(t)\) for the non-homogeneous equation, one finds \(z = \Omega^{-1} (e^{\Omega t} - I) \Bb\).  The complete solution with both the homogeneous and non-homogeneous parts is thus

\begin{equation}\label{eqn:hamiltonian:yoo15}
\begin{aligned}
\Bz(t) = e^{\Omega t} \Bz_0 + \Omega^{-1} (e^{\Omega t} - I) \Bb
\end{aligned}
\end{equation}

Going back to the pendulum problem, lets write

\begin{equation}\label{eqn:hamiltonian:yoo12o}
\begin{aligned}
\omega = \sqrt{\frac{g}{l} \abs{\cos\theta_0}}
\end{aligned}
\end{equation}

Below the horizontal we have

\begin{equation}\label{eqn:hamiltonian:yoo16}
\begin{aligned}
\Omega
&= \omega
\begin{bmatrix}
0 & -1 \\
1 & 0
\end{bmatrix} \\
\Omega^{-1}
&= -\inv{\omega} \begin{bmatrix}
0 & -1 \\
1 & 0
\end{bmatrix} \\
e^{\Omega t}
&=
\cos(\omega t)
\begin{bmatrix}
1 & 0 \\
0 & 1
\end{bmatrix}
+\sin(\omega t)
\begin{bmatrix}
0 & -1 \\
1 & 0
\end{bmatrix}
\end{aligned}
\end{equation}

Whereas above the horizontal we have

\begin{equation}\label{eqn:hamiltonian:yoo17}
\begin{aligned}
\Omega
&= \omega
\begin{bmatrix}
0 & 1 \\
1 & 0
\end{bmatrix} \\
\Omega^{-1}
&= \inv{\omega} \begin{bmatrix}
0 & 1 \\
1 & 0
\end{bmatrix} \\
e^{\Omega t}
&=
\cosh(\omega t)
\begin{bmatrix}
1 & 0 \\
0 & 1
\end{bmatrix}
+\sinh(\omega t)
\begin{bmatrix}
0 & 1 \\
1 & 0
\end{bmatrix}
\end{aligned}
\end{equation}

In both cases we have
\begin{equation}\label{eqn:hamiltonian:yoo18}
\begin{aligned}
\begin{bmatrix}
p_\theta \\
\alpha
\end{bmatrix}
&=
\begin{bmatrix}
m l^2 \omega & 0 \\
0 & 1
\end{bmatrix}
\Bz \\
\Bb &=
\inv{\omega}
\begin{bmatrix}
- \frac{g}{l} \sin\theta_0 \\
\frac{\dot{\theta}_0}{m l^2}
\end{bmatrix}
\end{aligned}
\end{equation}

(where the real angle was \(\theta = \theta_0 + \alpha\)).  Since in this case \(\Omega^{-1}\) and \(e^{\Omega t}\) commute, we have below the horizontal

\begin{equation}\label{eqn:hamiltonian:343}
\begin{aligned}
\Bz(t)
&= e^{\Omega t} (\Bz_0 - \Omega^{-1} \Bb) - \Omega^{-1} \Bb \\
&=
\left(
\cos(\omega t)
\begin{bmatrix}
1 & 0 \\
0 & 1
\end{bmatrix}
+\sin(\omega t)
\begin{bmatrix}
0 & -1 \\
1 & 0
\end{bmatrix}
\right)
\left(\Bz_0 +\inv{\omega} \begin{bmatrix}
0 & -1 \\
1 & 0
\end{bmatrix}
\Bb \right)
+\inv{\omega} \begin{bmatrix}
0 & -1 \\
1 & 0
\end{bmatrix}
\Bb
\end{aligned}
\end{equation}

Expanding out the \(\Bb\) terms and doing the same for above the horizontal we have respectively (below and above)

\begin{equation}\label{eqn:hamiltonian:yoo21}
\begin{aligned}
\Bz_{\text{low}}(t)
&=
\left(
\cos(\omega t)
\begin{bmatrix}
1 & 0 \\
0 & 1
\end{bmatrix}
+\sin(\omega t)
\begin{bmatrix}
0 & -1 \\
1 & 0
\end{bmatrix}
\right)
\left(\Bz_0
-\inv{\omega^2}
\begin{bmatrix}
\frac{\dot{\theta}_0}{m l^2} \\
\frac{g}{l} \sin\theta_0 \\
\end{bmatrix}
\right)
-\inv{\omega^2}
\begin{bmatrix}
\frac{\dot{\theta}_0}{m l^2} \\
\frac{g}{l} \sin\theta_0 \\
\end{bmatrix} \\
\Bz_{\text{high}}(t)
&=
\left(
\cosh(\omega t)
\begin{bmatrix}
1 & 0 \\
0 & 1
\end{bmatrix}
+\sinh(\omega t)
\begin{bmatrix}
0 & 1 \\
1 & 0
\end{bmatrix}
\right)
\left(\Bz_0
+\inv{\omega^2}
\begin{bmatrix}
\frac{\dot{\theta}_0}{m l^2} \\
\frac{g}{l} \sin\theta_0 \\
\end{bmatrix}
\right)
+\inv{\omega^2}
\begin{bmatrix}
\frac{\dot{\theta}_0}{m l^2} \\
\frac{g}{l} \sin\theta_0 \\
\end{bmatrix}
\end{aligned}
\end{equation}

The only thing that is really left is re-insertion of the original momentum and position variables using the inverse relation

\begin{equation}\label{eqn:hamiltonian:yoo20}
\begin{aligned}
\Bz
&=
\begin{bmatrix}
1/(m l^2 \omega) & 0 \\
0 & 1
\end{bmatrix}
\begin{bmatrix}
p_\theta \\
\theta - \theta_0
\end{bmatrix}
\end{aligned}
\end{equation}

Will that final insertion do anything more than make things messier?  Observe that the \(\Bz_0\) only has a momentum component when expressed back in terms of the total angle \(\theta\).  Also recall that \(p_\theta = m l^2 \dot{\theta}\), so we have

\begin{equation}\label{eqn:hamiltonian:yoo20a}
\begin{aligned}
\Bz
&=
\begin{bmatrix}
\dot{\theta}/\omega \\
\theta - \theta_0
\end{bmatrix} \\
\Bz_0
&=
\begin{bmatrix}
\dot{\theta}_{t=0}/\omega \\
0
\end{bmatrix} \\
\end{aligned}
\end{equation}

If this is somehow mystically free of all math mistakes then we have the final solution

\begin{equation}\label{eqn:hamiltonian:yoo19}
\begin{aligned}
{\begin{bmatrix}
\dot{\theta}(t)/\omega \\
\theta(t) - \theta_0
\end{bmatrix}}_{\text{low}}
&=
\left(
\cos(\omega t)
\begin{bmatrix}
1 & 0 \\
0 & 1
\end{bmatrix}
+\sin(\omega t)
\begin{bmatrix}
0 & -1 \\
1 & 0
\end{bmatrix}
\right)
\left(
\frac{\dot{\theta}(0)}{\omega}
\begin{bmatrix}
1 \\
0
\end{bmatrix}
%\Bz_0
-\inv{\omega^2}
\begin{bmatrix}
\frac{\dot{\theta}_0}{m l^2} \\
\frac{g}{l} \sin\theta_0 \\
\end{bmatrix}
\right)
-\inv{\omega^2}
\begin{bmatrix}
\frac{\dot{\theta}_0}{m l^2} \\
\frac{g}{l} \sin\theta_0 \\
\end{bmatrix}
\\
{\begin{bmatrix}
\dot{\theta}(t)/\omega \\
\theta(t) - \theta_0
\end{bmatrix}}_{\text{high}}
&=
\left(
\cosh(\omega t)
\begin{bmatrix}
1 & 0 \\
0 & 1
\end{bmatrix}
+\sinh(\omega t)
\begin{bmatrix}
0 & 1 \\
1 & 0
\end{bmatrix}
\right)
\left(
%\Bz_0
\frac{\dot{\theta}(0)}{\omega}
\begin{bmatrix}
1 \\
0
\end{bmatrix}
+\inv{\omega^2}
\begin{bmatrix}
\frac{\dot{\theta}_0}{m l^2} \\
\frac{g}{l} \sin\theta_0 \\
\end{bmatrix}
\right)
+\inv{\omega^2}
\begin{bmatrix}
\frac{\dot{\theta}_0}{m l^2} \\
\frac{g}{l} \sin\theta_0 \\
\end{bmatrix}
\end{aligned}
\end{equation}

A qualification is required to call this a solution since it is only a solution is the restricted range where \(\theta\) is close enough to \(\theta_0\) (in some imprecisely specified sense).  One could conceivably apply this in a recursive fashion however, calculating the result for a small incremental change, yielding the new phase space point, and repeating at the new angle.

The question of what the form of the solution in the neighborhood of \(\pm \pi/2\) has also been ignored.  That is probably also worth considering but I do not feel like trying now.

\subsection{Spherical pendulum}

For the spherical rigid pendulum of length \(l\), we have for the distance above the lowest point

\begin{equation}\label{eqn:hamiltonian:uoo1}
\begin{aligned}
h = l (1 + \cos\theta)
\end{aligned}
\end{equation}

(measuring \(\theta\) down from the North pole as conventional).  The Lagrangian is therefore

\begin{equation}\label{eqn:hamiltonian:uoo2}
\begin{aligned}
\LL = \inv{2} m l^2 (\thetadot^2 + \sin^2\theta \phidot^2) - m g l (1 + \cos\theta)
\end{aligned}
\end{equation}

We can drop the constant term, using the simpler Lagrangian

\begin{equation}\label{eqn:hamiltonian:uoo3}
\begin{aligned}
\LL = \inv{2} m l^2 (\thetadot^2 + \sin^2\theta \phidot^2) - m g l \cos\theta
\end{aligned}
\end{equation}

To express the Hamiltonian we need first the conjugate momenta, which are

\begin{equation}\label{eqn:hamiltonian:uoo4}
\begin{aligned}
P_\theta &= \PD{\thetadot}{\LL} = m l^2 \thetadot \\
P_\phi &= \PD{\phidot}{\LL} = m l^2 \sin^2\theta \phidot
\end{aligned}
\end{equation}

We can now write the Hamiltonian

\begin{equation}\label{eqn:hamiltonian:uoo5}
\begin{aligned}
H = \inv{2 m l^2} \left({P_\theta}^2 + \inv{\sin^2\theta} {P_\phi}^2\right) + m g l \cos\theta
\end{aligned}
\end{equation}

Before going further one sees that there is going to be trouble where \(\sin\theta = 0\).  Curiously, this is at the poles, the most dangling position and the upright.  The south pole is the usual point where we solve the planar pendulum problem using the harmonic oscillator approximation, so it is somewhat curious that the energy of the system appears to go undefined at this point where the position is becoming more defined.  It seems almost like a quantum uncertainty phenomena until one realizes that the momentum conjugate to \(\phi\) is itself proportional to \(\sin^2 \theta\).  By expressing the energy in terms of this \(P_\phi\) momentum we have to avoid looking at the poles for a solution to the equations.  If we go back to the Lagrangian and the Euler-Lagrange equations, this point becomes perfectly tractable since we are no longer dividing through by \(\sin^2\theta\).

Examining the polar solutions is something to return to.  For now, let us avoid that region.  For regions where \(\sin\theta\) is nicely non-zero, we get for the Hamiltonian equations

\begin{equation}\label{eqn:hamiltonian:uoo6}
\begin{aligned}
\PD{P_\phi}{H} &= \dot{\phi} = \inv{ m l^2 \sin^2 \theta} P_\phi \\
\PD{P_\theta}{H} &= \dot{\theta} = \inv{ m l^2 } P_\theta \\
\PD{\phi}{H} &= -\dot{P}_\phi = 0 \\
\PD{\theta}{H} &= -\dot{P}_\theta = -\frac{\cos\theta}{ m l^2 \sin^3 \theta} {P_\phi}^2 - m g l \sin\theta
\end{aligned}
\end{equation}

These now expressing the dynamics of the system.  The first two equations are just the definitions of the canonical momenta that we started with using the Lagrangian.  Not surprisingly, but unfortunate, we have a non-linear system here like the planar rigid pendulum, so despite this being one of the most simple systems it does not look terribly tractable.  What would it take to linearize this system of equations?

Lets write the state space vector for the system as

\begin{equation}\label{eqn:hamiltonian:uoo7}
\begin{aligned}
\Bx =
\begin{bmatrix}
P_\theta \\
\theta \\
P_\phi \\
\phi \\
\end{bmatrix}
\end{aligned}
\end{equation}

lets also suppose that we are interested in the change to the state vector in the neighborhood of an initial state

\begin{equation}\label{eqn:hamiltonian:uoo8}
\begin{aligned}
\Bx =
\begin{bmatrix}
P_\theta \\
\theta \\
P_\phi \\
\phi \\
\end{bmatrix}
=
{
\begin{bmatrix}
P_\theta \\
\theta \\
P_\phi \\
\phi \\
\end{bmatrix}
}_0
+
\Bz
\end{aligned}
\end{equation}

The Hamiltonian equations can then be written

\begin{equation}\label{eqn:hamiltonian:uoo9}
\begin{aligned}
\frac{d\Bz}{dt} &=
\begin{bmatrix}
\frac{\cos\theta}{ m l^2 \sin^3 \theta} {P_\phi}^2 + m g l \sin\theta \\
\inv{ m l^2 } P_\theta \\
0 \\
\inv{ m l^2 \sin^2 \theta} P_\phi \\
\end{bmatrix}
\end{aligned}
\end{equation}

Getting away from the specifics of this particular system is temporarily helpful.  We have a set of equations that we wish to calculate a linear approximation for

\begin{equation}\label{eqn:hamiltonian:uoo10}
\begin{aligned}
\frac{dz_\mu}{dt} = A_\mu(x_\nu) \approx A_\mu(\Bx_0) + \sum_\alpha {\left. \PD{x_\alpha}{A_\mu} \right\vert}_{\Bx_0} z_\alpha
\end{aligned}
\end{equation}

Our linear approximation is thus

\begin{equation}\label{eqn:hamiltonian:uoo11}
\begin{aligned}
\frac{d\Bz}{dt} &\approx
{
\begin{bmatrix}
\frac{\cos\theta}{ m l^2 \sin^3 \theta} {P_\phi}^2 + m g l \sin\theta \\
\inv{ m l^2 } P_\theta \\
0 \\
\inv{ m l^2 \sin^2 \theta} P_\phi \\
\end{bmatrix}
}_0
+
%{
%\begin{bmatrix}
%0 & \PD{\theta}{}\left( \frac{\cos\theta}{ m l^2 \sin^3 \theta} {P_\phi}^2 + m g l \sin\theta \right) & \frac{2 \cos\theta}{ m l^2 \sin^3 \theta} P_\phi & 0 \\
%1/m l^2 & 0 & 0 & 0 \\
%0 & 0 & 0 & 0 \\
%0 & \frac{P_\phi}{m l^2} \PD{\theta}{\inv{\sin^2\theta}} & \inv{ m l^2 \sin^2 \theta} & 0 \\
%\end{bmatrix}
%}_0
{
\begin{bmatrix}
0 &
-\frac{{P_\phi}^2 (1 + 2 \cos^2 \theta)}{m l^2 \sin^4 \theta} +m g l \cos\theta
& \frac{2 \cos\theta}{ m l^2 \sin^3 \theta} P_\phi & 0 \\
\inv{m l^2} & 0 & 0 & 0 \\
0 & 0 & 0 & 0 \\
0 & \frac{-2 P_\phi \cos\theta}{m l^2 \sin^3\theta} & \inv{ m l^2 \sin^2 \theta} & 0 \\
\end{bmatrix}
}_0
\Bz
\end{aligned}
\end{equation}
%- 1/S^2 + C (-3) S^2 C/ S^6
%=
%- (1/S^2 + 3 C^2 / S^4 )
%=
%- 1/S^4( S^2 + 3 C^2 )
%=
%- 1/S^4( S^2 + 3 (1-S^2)) = - 1/S^4( 3 - 2 S^2 )
% or:
%=
%- 1/S^4( 1 - C^2 + 3 C^2 )
%=
%- 1/S^4( 1 + 2 C^2 )

Now, this is what we get blindly trying to set up the linear approximation of the state space differential equation.  We see that the cyclic coordinate \(\phi\) leads to a bit of trouble since no explicit \(\phi\) dependence in the Hamiltonian makes the resulting matrix factor non-invertible.  It appears that we would be better explicitly utilizing this cyclic coordinate to note that \(P_\phi = \text{constant}\), and to omit this completely from the state vector.  Our equations in raw form are now

\begin{equation}\label{eqn:hamiltonian:uoo12}
\begin{aligned}
\dot{\theta} &= \inv{ m l^2 } P_\theta \\
\dot{P}_\theta &= \frac{\cos\theta}{ m l^2 \sin^3 \theta} {P_\phi}^2 + m g l \sin\theta \\
\dot{\phi} &= \inv{ m l^2 \sin^2 \theta} P_\phi
\end{aligned}
\end{equation}

We can treat the \(\phi\) dependence later once we have solved for \(\theta\).  That equation to later solve is just this last

\begin{equation}\label{eqn:hamiltonian:uoo13}
\begin{aligned}
\dot{\phi} &= \inv{ m l^2 \sin^2 \theta} P_\phi
\end{aligned}
\end{equation}

This integrates directly, presuming \(\theta = \theta(t)\) is known, and we have

\begin{equation}\label{eqn:hamiltonian:uoo13b}
\begin{aligned}
\phi - \phi(0) &= \frac{P_\phi}{ m l^2} \int_0^t \frac{d\tau}{\sin^2 \theta(\tau)}
\end{aligned}
\end{equation}

Now the state vector and its perturbation can be redefined omitting all but the \(\theta\) dependence.  Namely

\begin{equation}\label{eqn:hamiltonian:uoo7p}
\begin{aligned}
\Bx =
\begin{bmatrix}
P_\theta \\
\theta \\
\end{bmatrix}
\end{aligned}
\end{equation}

\begin{equation}\label{eqn:hamiltonian:uoo8p}
\begin{aligned}
\Bx =
\begin{bmatrix}
P_\theta \\
\theta \\
\end{bmatrix}
=
{
\begin{bmatrix}
P_\theta \\
\theta \\
\end{bmatrix}
}_0
+
\Bz
\end{aligned}
\end{equation}

We can now write the remainder of this non-linear system as
\begin{equation}\label{eqn:hamiltonian:uoo9p}
\begin{aligned}
\frac{d\Bz}{dt} &=
\begin{bmatrix}
\frac{\cos\theta}{ m l^2 \sin^3 \theta} {P_\phi}^2 + m g l \sin\theta \\
\inv{ m l^2 } P_\theta \\
\end{bmatrix}
\end{aligned}
\end{equation}

and make the linear approximation around \(\Bx_0\) as

\begin{equation}\label{eqn:hamiltonian:uoo11p}
\begin{aligned}
\frac{d\Bz}{dt} &\approx
{
\begin{bmatrix}
\frac{\cos\theta}{ m l^2 \sin^3 \theta} {P_\phi}^2 + m g l \sin\theta \\
\inv{ m l^2 } P_\theta \\
\end{bmatrix}
}_0
+
{
\begin{bmatrix}
0 & -\frac{{P_\phi}^2 (1 + 2 \cos^2 \theta)}{m l^2 \sin^4 \theta} +m g l \cos\theta  \\
\inv{m l^2} & 0 \\
\end{bmatrix}
}_0
\Bz
\end{aligned}
\end{equation}

This now looks a lot more tractable, and is in fact exactly the same form now as the equation for the linearized planar pendulum.  The only difference is the normalization required to switch to less messy dimensionless variables.  The main effect of allowing the trajectory to have a non-planar component is a change in the angular frequency in the \(\theta\) dependent motion.  That frequency will no longer be \(\sqrt{\Abs{\cos\theta_0} g/l}\), but also has a \(P_\phi\) and other more complex trigonometric \(\theta\) dependencies.  It also appears that we can probably have hyperbolic or trigonometric solutions in the neighborhood of any point, regardless of whether it is a northern hemispherical point or a southern one.  In the planar pendulum the unambiguous sign of the matrix terms led to hyperbolic only above the horizon, and trigonometric only below.

\subsection{Double and multiple pendulums, and general quadratic velocity dependence}

In the following section I started off with the goal of treating two connected pendulums moving in a plane.  Even setting up the Hamiltonian's for this turned out to be a bit messy, requiring a matrix inversion.  Tackling the problem in the guise of using a more general quadratic form (which works for the two particle as well as \(N\) particle cases) seemed like it would actually be simpler than using the specifics from the angular velocity dependence of the specific pendulum problem.  Once the Hamiltonian equations were found in this form, an attempt to do the first order Taylor expansion as done for the single planar pendulum and the spherical pendulum was performed.  This turned out to be a nasty mess and is seen to not be particularly illuminating.  I did not know that is how it would turn out ahead of time since I had my fingers crossed for some sort of magic simplification once the final substitution were made.  If such a simplification is possible, the procedure to do so is not obvious.

Although the Hamiltonian equations for a spherical pendulum have been considered previously, for the double pendulum case it seems prudent to avoid temptation, and to first see what happens with a simpler first step, a planar double pendulum.

Setting up coordinates \(x\) axis down, and \(y\) axis to the left with \(i = \xcap \ycap\) we have for the position of the first mass \(m_1\), at angle \({\theta_1}\) and length \(l_1\)

\begin{equation}\label{eqn:hamiltonian:voo1}
\begin{aligned}
z_1 = \xcap l_1 e^{i{\theta_1}}
\end{aligned}
\end{equation}

If the second mass, dangling from this is at an angle \({\theta_2}\) from the \(x\) axis, its position is

\begin{equation}\label{eqn:hamiltonian:voo2}
\begin{aligned}
z_2 = z_1 + \xcap l_2 e^{i{\theta_2}}
\end{aligned}
\end{equation}

We need the velocities, and their magnitudes.  For \(z_1\) this is

\begin{equation}\label{eqn:hamiltonian:voo3}
\begin{aligned}
\Abs{\dot{z}_1}^2 = {l_1}^2 {\dot{\theta}_1}^2
\end{aligned}
\end{equation}

For the second mass

\begin{equation}\label{eqn:hamiltonian:voo4}
\begin{aligned}
\dot{z}_2 = \xcap i \left( l_1 {\dot{\theta}_1} e^{i{\theta_1}} + l_2 {\dot{\theta}_2} e^{i{\theta_2}} \right)
\end{aligned}
\end{equation}

Taking conjugates and multiplying out we have

\begin{equation}\label{eqn:hamiltonian:voo5}
\begin{aligned}
\Abs{\dot{z}_2}^2 =
{l_1}^2 {\dot{\theta}_1}^2
+ 2 l_1 l_2 {\dot{\theta}_1} {\dot{\theta}_2} \cos({\theta_1} - {\theta_2})
+{l_2}^2 {\dot{\theta}_2}^2
\end{aligned}
\end{equation}

That is all that we need for the Kinetic terms in the Lagrangian.  Now we need the height for the \(m g h\) terms.  If we set the reference point at the lowest point for the double pendulum system, the height of the first particle is

\begin{equation}\label{eqn:hamiltonian:voo6}
\begin{aligned}
h_1 = l_2 + l_1 (1 -\cos{\theta_1})
\end{aligned}
\end{equation}

For the second particle, the distance from the horizontal is
\begin{equation}\label{eqn:hamiltonian:voo7}
\begin{aligned}
d = l_1 \cos{\theta_1} + l_2 \cos{\theta_2}
\end{aligned}
\end{equation}

So the total distance from the reference point is
\begin{equation}\label{eqn:hamiltonian:voo8}
\begin{aligned}
h_2 = l_1 (1 - \cos{\theta_1}) + l_2 (1 -\cos{\theta_2})
\end{aligned}
\end{equation}

We now have the Lagrangian
\begin{equation}\label{eqn:hamiltonian:voo9}
\begin{aligned}
\LL' &=
\inv{2} m_1 {l_1}^2 {\dot{\theta}_1}^2
+ \inv{2} m_2 \left(
{l_1}^2 {\dot{\theta}_1}^2
+ 2 l_1 l_2 {\dot{\theta}_1} {\dot{\theta}_2} \cos({\theta_1} - {\theta_2})
+{l_2}^2 {\dot{\theta}_2}^2
\right) \\
&-
m_1 g (l_2 + l_1 (1 -\cos{\theta_1}))
-
m_2 g ( l_1 (1 - \cos{\theta_1}) + l_2 (1 -\cos{\theta_2}) )
\end{aligned}
\end{equation}

Dropping constant terms (effectively choosing a difference reference point for the potential) and rearranging a bit, also writing \(M = m_1 + m_2\), we have the simpler Lagrangian

\begin{equation}\label{eqn:hamiltonian:voo10}
\begin{aligned}
\LL &=
\inv{2} M {l_1}^2 {\dot{\theta}_1}^2 + \inv{2} m_2 {l_2}^2 {\dot{\theta}_2}^2
+ m_2 l_1 l_2 {\dot{\theta}_1} {\dot{\theta}_2} \cos({\theta_1} - {\theta_2})
+
M l_1 g \cos{\theta_1}
+
m_2 l_2 g \cos{\theta_2}
\end{aligned}
\end{equation}

The conjugate momenta that we need for the Hamiltonian are

\begin{equation}\label{eqn:hamiltonian:voo11}
\begin{aligned}
P_{\theta_1} &=
M {l_1}^2 {\dot{\theta}_1}
+ m_2 l_1 l_2 {\dot{\theta}_2} \cos({\theta_1} - {\theta_2}) \\
P_{\theta_2} &=
m_2 {l_2}^2 {\dot{\theta}_2}
+ m_2 l_1 l_2 {\dot{\theta}_1} \cos({\theta_1} - {\theta_2})
\end{aligned}
\end{equation}

Unlike any of the other simpler Hamiltonian systems considered so far, the coupling between the velocities means that we have a system of equations that we must first invert before we can even express the Hamiltonian in terms of the respective momenta.

That is
\begin{equation}\label{eqn:hamiltonian:voo12}
\begin{aligned}
\begin{bmatrix}
P_{\theta_1} \\
P_{\theta_2}
\end{bmatrix}
=
\begin{bmatrix}
M {l_1}^2 & m_2 l_1 l_2 \cos({\theta_1} - {\theta_2}) \\
m_2 l_1 l_2 \cos({\theta_1} - {\theta_2}) & m_2 {l_2}^2
\end{bmatrix}
\begin{bmatrix}
{\dot{\theta}_1} \\
{\dot{\theta}_2}
\end{bmatrix}
\end{aligned}
\end{equation}

While this is easily invertible, doing so and attempting to substitute it back, results in an unholy mess (albeit perhaps one that can be simplified).  Is there a better way?  A possibly promising way is motivated by observing that this matrix, a function of the angular difference \(\delta = {\theta_1} - {\theta_2}\), looks like it is something like a moment of inertia tensor.  If we call this \(\calI\), and write

\begin{equation}\label{eqn:hamiltonian:voo13}
\begin{aligned}
\BTheta
&\equiv
\begin{bmatrix}
{\theta_1} \\
{\theta_2}
\end{bmatrix}
\end{aligned}
\end{equation}

Then the relation between the conjugate momenta in vector form

\begin{equation}\label{eqn:hamiltonian:voo14}
\begin{aligned}
\Bp &\equiv
\begin{bmatrix}
P_{\theta_1} \\
P_{\theta_2}
\end{bmatrix}
\end{aligned}
\end{equation}

and the angular velocity vector can be written

\begin{equation}\label{eqn:hamiltonian:voo15}
\begin{aligned}
\Bp &= \calI(\delta)
\dot{\BTheta}
\end{aligned}
\end{equation}

Can we write the Lagrangian in terms of \(\dot{\BTheta}\)?  The first Kinetic term is easy, just

\begin{equation}\label{eqn:hamiltonian:voo16}
\begin{aligned}
\inv{2} m_1 l^2 {\dot{\theta}_1}^2 = \inv{2} m_1
\dot{\BTheta}^\T
\begin{bmatrix}
l_1^2 & 0 \\
0 & 0
\end{bmatrix}
\dot{\BTheta}
\end{aligned}
\end{equation}

For the second mass, going back to \eqnref{eqn:hamiltonian:voo4}, we can write

\begin{equation}\label{eqn:hamiltonian:voo4m}
\begin{aligned}
\dot{z}_2 = \xcap i
\begin{bmatrix}
l_1 e^{i{\theta_1}} & l_2 e^{i{\theta_2}}
\end{bmatrix}
\dot{\BTheta}
\end{aligned}
\end{equation}

Writing \(\Br\) for this \(1x2\) matrix, we can utilize the associative property for compatible sized matrices to rewrite the speed for the second particle in terms of a quadratic form

\begin{equation}\label{eqn:hamiltonian:voo5m}
\begin{aligned}
\Abs{\dot{z}_2}^2
=
(\Br \dot{\BTheta}) (\overbar{\Br} \dot{\BTheta})
=
\dot{\BTheta}^\T (\Br^\T \overbar{\Br}) \dot{\BTheta}
\end{aligned}
\end{equation}

The Lagrangian kinetic can all now be grouped into a single quadratic form

\begin{equation}\label{eqn:hamiltonian:voo17}
\begin{aligned}
Q \equiv
m_1
\begin{bmatrix}
l_1 \\
0
\end{bmatrix}
\begin{bmatrix}
l_1 & 0
\end{bmatrix}
+m_2
\begin{bmatrix}
l_1 e^{i{\theta_1}} \\
l_2 e^{i{\theta_2}}
\end{bmatrix}
\begin{bmatrix}
l_1 e^{-i{\theta_1}} & l_2 e^{-i{\theta_2}}
\end{bmatrix}
\end{aligned}
\end{equation}

\begin{equation}\label{eqn:hamiltonian:voo18}
\begin{aligned}
\LL =
\inv{2} \dot{\BTheta}^\T Q \dot{\BTheta}
+
M l_1 g \cos{\theta_1}
+
m_2 l_2 g \cos{\theta_2}
\end{aligned}
\end{equation}

It is also clear that this generalize easily to multiple connected pendulums, as follows

\begin{equation}\label{eqn:hamiltonian:voo18g}
\begin{aligned}
K &= \inv{2} \dot{\BTheta}^\T \sum_k m_k Q_k \dot{\BTheta} \\
Q_k &=
{\begin{bmatrix}
l_r l_s e^{i(\theta_r - \theta_s)}
\end{bmatrix}}_{r,s \le k} \\
\phi &= - g \sum_k l_k \cos\theta_k \sum_{j=k}^N m_j \\
\LL &= K - \phi
\end{aligned}
\end{equation}

In the expression for \(Q_k\) above, it is implied that the matrix is zero for any indices \(r,s > k\), so it would perhaps be better to write explicitly

\begin{equation}\label{eqn:hamiltonian:voo18q}
\begin{aligned}
Q = \sum_k m_k Q_k
=
{\begin{bmatrix}
\sum_{j=\max(r,s)}^N m_j l_r l_s e^{i(\theta_r - \theta_s)}
\end{bmatrix}}_{r,s}
\end{aligned}
\end{equation}

Returning to the problem, it is convenient and sufficient in many cases to only discuss the representative double pendulum case.  For that we can calculate the conjugate momenta from \eqnref{eqn:hamiltonian:voo18} directly

\begin{equation}\label{eqn:hamiltonian:363}
\begin{aligned}
P_{\theta_1}
&=
\PD{{\dot{\theta}_1}}{}
\inv{2} \dot{\BTheta}^\T Q \dot{\BTheta} \\
&=
\PD{{\dot{\theta}_1}}{}
\inv{2} \dot{\BTheta}^\T Q
\begin{bmatrix}
1 \\
0
\end{bmatrix}
+
\inv{2}
\begin{bmatrix}
1 & 0
\end{bmatrix}
Q \dot{\BTheta} \\
&=
\begin{bmatrix}
1 & 0
\end{bmatrix}
\left(\inv{2}(Q + Q^\T)\right) \dot{\BTheta} \\
\end{aligned}
\end{equation}

Similarly the \({\theta_2}\) conjugate momentum is

\begin{equation}\label{eqn:hamiltonian:383}
\begin{aligned}
P_{\theta_2}
=
\begin{bmatrix}
0 & 1
\end{bmatrix}
\left(\inv{2}(Q + Q^\T)\right) \dot{\BTheta} \\
\end{aligned}
\end{equation}

Putting both together, it is straightforward to verify that this recovers \eqnref{eqn:hamiltonian:voo12}, which can now be written

\begin{equation}\label{eqn:hamiltonian:voo19}
\begin{aligned}
\Bp
&=
\inv{2}(Q + Q^\T) \dot{\BTheta} = \calI \dot{\BTheta}
\end{aligned}
\end{equation}

Observing that \(\calI = \calI^\T\), and thus \((\calI^\T)^{-1} = \calI^{-1}\), we now have everything required to express the Hamiltonian in terms of the conjugate momenta

\begin{equation}\label{eqn:hamiltonian:voo20}
\begin{aligned}
H = \Bp^\T \left( \inv{2} \calI^{-1} Q \calI^{-1} \right) \Bp - M g l_1 \cos{\theta_1} - m_2 l_2 g \cos{\theta_2}
\end{aligned}
\end{equation}

This is now in a convenient form to calculate the first set of Hamiltonian equations.

\begin{equation}\label{eqn:hamiltonian:403}
\begin{aligned}
\dot{\theta}_k &=
\PD{P_{\theta_k}}{H} \\
&=
\PD{P_{\theta_k}}{\Bp^\T} \inv{2} \calI^{-1} Q \calI^{-1} \Bp
+ \Bp^\T \inv{2} \calI^{-1} Q \calI^{-1} \PD{P_{\theta_k}}{\Bp^\T} \\
&=
{\begin{bmatrix}
\delta_{kj}
\end{bmatrix}}_j
\inv{2} \calI^{-1} Q \calI^{-1} \Bp
+ \Bp^\T \inv{2} \calI^{-1} Q \calI^{-1}
{\begin{bmatrix}
\delta_{ik}
\end{bmatrix}}_i \\
&=
{\begin{bmatrix}
\delta_{kj}
\end{bmatrix}}_j
\calI^{-1}
\mathLabelBox{\inv{2}(Q + Q^\T)}{\(\calI\)}
\calI^{-1} \Bp \\
&=
{\begin{bmatrix}
\delta_{kj}
\end{bmatrix}}_j
\calI^{-1} \Bp \\
\end{aligned}
\end{equation}

So, when the velocity dependence is a quadratic form as identified in \eqnref{eqn:hamiltonian:voo17}, the first half of the Hamiltonian equations in vector form are just

\begin{equation}\label{eqn:hamiltonian:voo21}
\begin{aligned}
\dot{\BTheta} &=
{\begin{bmatrix}
\PD{P_{\theta_1}}{} & \cdots & \PD{P_{\theta_N}}{}
\end{bmatrix}}^\T H
=
\calI^{-1} \Bp
\end{aligned}
\end{equation}

This is exactly the relation we used in the first place to re-express the Lagrangian in terms of the conjugate momenta in preparation for this calculation.  The remaining Hamiltonian equations are trickier, and what we now want to calculate.  Without specific reference to the pendulum problem, lets do this calculation for the general Hamiltonian for a non-velocity dependent potential.  That is

\begin{equation}\label{eqn:hamiltonian:voo22}
\begin{aligned}
H = \Bp^\T \left( \inv{2} \calI^{-1} Q \calI^{-1} \right) \Bp + \phi(\BTheta)
\end{aligned}
\end{equation}

The remaining Hamiltonian equations are \(\PDi{\theta_a}{H} = -\dot{P}_{\theta_a}\), and the tricky part of evaluating this is going to all reside in the Kinetic term.  Diving right in this is

\begin{equation}\label{eqn:hamiltonian:423}
\begin{aligned}
\PD{\theta_a}{K}
&=
\Bp^\T \left( \inv{2} \PD{\theta_a}{(\calI^{-1})} Q \calI^{-1} \right) \Bp
+\Bp^\T \left( \inv{2} \calI^{-1} \PD{\theta_a}{Q} \calI^{-1} \right) \Bp
+\Bp^\T \left( \inv{2} \calI^{-1} Q \PD{\theta_a}{(\calI^{-1})} \right) \Bp \\
&=
\Bp^\T \PD{\theta_a}{(\calI^{-1})}
\mathLabelBox{\inv{2}(Q + Q^\T)}{\(=\calI\)}
 \calI^{-1} \Bp
+\Bp^\T \left( \inv{2} \calI^{-1} \PD{\theta_a}{Q} \calI^{-1} \right) \Bp  \\
&=
\Bp^\T \PD{\theta_a}{(\calI^{-1})} \Bp
+\Bp^\T \left( \inv{2} \calI^{-1} \PD{\theta_a}{Q} \calI^{-1} \right) \Bp  \\
\end{aligned}
\end{equation}

For the two particle case we can expand this inverse easily enough, and then take derivatives to evaluate this, but this is messier and intractable for the general case.  We can however, calculate the derivative of the identity matrix using the standard trick from rigid body mechanics

\begin{equation}\label{eqn:hamiltonian:443}
\begin{aligned}
0
&= \PD{\theta_a}{I} \\
&= \PD{\theta_a}{(\calI \calI^{-1})} \\
&=
\PD{\theta_a}{\calI} \calI^{-1}
+\calI \PD{\theta_a}{(\calI^{-1})}
\\
\end{aligned}
\end{equation}

Thus the derivative of the inverse (moment of inertia?) matrix is

\begin{equation}\label{eqn:hamiltonian:463}
\begin{aligned}
\PD{\theta_a}{(\calI^{-1})}
&= -\calI^{-1}\PD{\theta_a}{\calI} \calI^{-1} \\
&= -\calI^{-1}\inv{2}\left( \PD{\theta_a}{Q} + \PD{\theta_a}{Q^\T} \right) \calI^{-1}
\end{aligned}
\end{equation}

This gives us for the Hamiltonian equation

\begin{equation}\label{eqn:hamiltonian:voo24}
\begin{aligned}
\PD{\theta_a}{H}
%&=
%-\Bp^\T \calI^{-1}\inv{2}\left( \PD{\theta_a}{Q} + \PD{\theta_a}{Q^\T} \right) \calI^{-1} \Bp
%+\Bp^\T \left( \inv{2} \calI^{-1} \PD{\theta_a}{Q} \calI^{-1} \right) \Bp
%+ \PD{\theta_a}{\phi} \\
&=
-\inv{2} \Bp^\T \calI^{-1} \left(\PD{\theta_a}{Q}\right)^\T \calI^{-1} \Bp
+ \PD{\theta_a}{\phi}
\end{aligned}
\end{equation}

%If we introduce position and momentum gradients in phase space
If we introduce a phase space position gradients

\begin{equation}\label{eqn:hamiltonian:voo25}
\begin{aligned}
%\grad_{\BTheta} &\equiv
\grad &\equiv
{\begin{bmatrix}
\PD{\theta_1}{} & \cdots & \PD{\theta_N}{}
\end{bmatrix}}^\T \\
%\grad_{\Bp} &\equiv
%{\begin{bmatrix}
%\PD{P_{\theta_1}}{} & \cdots & \PD{P_{\theta_N}}{}
%\end{bmatrix}}^\T
\end{aligned}
\end{equation}

Then for the second half of the Hamiltonian equations we have the vector form

\begin{equation}\label{eqn:hamiltonian:voo26}
\begin{aligned}
-\grad H = \dot{\Bp} =
{\begin{bmatrix}
\inv{2} \Bp^\T \calI^{-1} \left(\PD{\theta_r}{Q}\right)^\T \calI^{-1} \Bp
\end{bmatrix}}_r
- \grad \phi
\end{aligned}
\end{equation}

The complete set of Hamiltonian equations for \eqnref{eqn:hamiltonian:voo22}, in block matrix form, describing all the phase space change of the system is therefore

\begin{equation}\label{eqn:hamiltonian:veryGeneral}
\begin{aligned}
\frac{d}{dt}
\begin{bmatrix}
\Bp \\
\BTheta
\end{bmatrix}
=
\begin{bmatrix}
{\begin{bmatrix}
\inv{2} \Bp^\T \calI^{-1} \left(\PD{\theta_r}{Q}\right)^\T \calI^{-1} \Bp
\end{bmatrix}}_r
- \grad \phi \\
%\begin{bmatrix}
%\calI^{-1} & 0
%\end{bmatrix}
%\begin{bmatrix}
%\Bp \\
%\BTheta
%\end{bmatrix}
\calI^{-1} \Bp \\
\end{bmatrix}
=
\begin{bmatrix}
{\begin{bmatrix}
\inv{2} \dot{\BTheta} \left(\PD{\theta_r}{Q}\right)^\T \dot{\BTheta}
\end{bmatrix}}_r
- \grad \phi \\
\dot{\BTheta}
\end{bmatrix}
\end{aligned}
\end{equation}

This is a very general relation, much more so than required for the original two particle problem.  We have the same non-linearity that prevents this from being easily solved.  If we want a linear expansion around a phase space point to find an approximate first order solution, we can get that applying the chain rule, calculating all the \(\PDi{\theta_k}{}\), and \(\PDi{P_{\theta_k}}{}\) derivatives of the top \(N\) rows of this matrix.

If we write

\begin{equation}\label{eqn:hamiltonian:voo28}
\begin{aligned}
\Bz \equiv
\begin{bmatrix}
\Bp \\
\BTheta
\end{bmatrix}
-
\evalnobar{
\begin{bmatrix}
\Bp \\
\BTheta
\end{bmatrix}
}{t=0}
\end{aligned}
\end{equation}

and the Hamiltonian equations \eqnref{eqn:hamiltonian:veryGeneral} as

\begin{equation}\label{eqn:hamiltonian:voo27v}
\begin{aligned}
\frac{d}{dt}
\begin{bmatrix}
\Bp \\
\BTheta
\end{bmatrix}
= A(\Bp, \BTheta)
\end{aligned}
\end{equation}

Then the linearization, without simplifying or making explicit yet is

\begin{equation}\label{eqn:hamiltonian:voo29}
\begin{aligned}
\dot{\Bz}
\approx
\evalnobar{
\begin{bmatrix}
{\begin{bmatrix}
\inv{2} \dot{\BTheta} \left(\PD{\theta_r}{Q}\right)^\T \dot{\BTheta}
\end{bmatrix}}_r
- \grad \phi \\
\dot{\BTheta}
\end{bmatrix}
}{t=0}
+
\evalbar{
\begin{bmatrix}
\PD{P_{\theta_1}}{A} & \cdots & \PD{P_{\theta_N}}{A} & \PD{\theta_1}{A} & \cdots & \PD{\theta_N}{A}
\end{bmatrix}
}{t=0} \Bz
\end{aligned}
\end{equation}

For brevity the constant term evaluated at \(t=0\) is expressed in terms of the original angular velocity vector from our Lagrangian.  The task is now evaluating the derivatives in the first order term of this Taylor series.  Let us do these one at a time and then reassemble all the results afterward.

So that we can discuss just the first order terms lets write \(\Delta\) for the matrix of first order derivatives in our Taylor expansion, as in

\begin{equation}\label{eqn:hamiltonian:voo30p}
\begin{aligned}
f(\Bp, \BTheta) = \evalbar{f(\Bp, \BTheta)}{0} + \evalbar{\Delta f}{0} \Bz + \cdots
\end{aligned}
\end{equation}

First, lets do the potential gradient.

\begin{equation}\label{eqn:hamiltonian:voo30}
\begin{aligned}
\Delta (\grad \phi) =
\begin{bmatrix}
0 &
{ \begin{bmatrix}
\frac{\partial^2 \phi}{ \partial \theta_r \partial \theta_c }
\end{bmatrix} }_{r,c}
\end{bmatrix}
\end{aligned}
\end{equation}

Next in terms of complexity is the first order term of \(\dot{\BTheta}\), for which we have

\begin{equation}\label{eqn:hamiltonian:483}
\begin{aligned}
\Delta (\calI^{-1} \Bp)
&=
\begin{bmatrix}
{
\begin{bmatrix}
\calI^{-1}
{\begin{bmatrix}
\delta_{rc}
\end{bmatrix}}_{r}
\end{bmatrix}}_{c}
&
{\begin{bmatrix}
\PD{\theta_c}{(\calI^{-1})} \Bp \\
\end{bmatrix}}_{c}
\end{bmatrix} \\
\end{aligned}
\end{equation}

The \(\delta\) over all rows \(r\) and columns \(c\) is the identity matrix and we are left with

\begin{equation}\label{eqn:hamiltonian:voo30b}
\begin{aligned}
\Delta (\calI^{-1} \Bp)
&=
\begin{bmatrix}
\calI^{-1}
&
{\begin{bmatrix}
\PD{\theta_c}{(\calI^{-1})} \Bp \\
\end{bmatrix}}_{c}
\end{bmatrix}
\end{aligned}
\end{equation}

Next, consider just the \(P_\theta\) dependence in the elements of the row vector

\begin{equation}\label{eqn:hamiltonian:voo30c}
\begin{aligned}
{\begin{bmatrix}
\inv{2} \Bp^\T \calI^{-1} \left(\PD{\theta_r}{Q}\right)^\T \calI^{-1} \Bp
\end{bmatrix}}_r
\end{aligned}
\end{equation}

We can take derivatives of this, and exploiting the fact that these elements are scalars, so they equal their transpose.  Also noting that \({A^{-1}}^\T = {A^\T}^{-1}\), and \(\calI = \calI^\T\), we have

\begin{equation}\label{eqn:hamiltonian:503}
\begin{aligned}
\PD{P_{\theta_c}}{}
\left( \inv{2} \Bp^\T \calI^{-1} \left(\PD{\theta_r}{Q}\right)^\T \calI^{-1} \Bp  \right)
&=
\inv{2} \Bp^\T \calI^{-1} \left(\PD{\theta_r}{Q}\right)^\T \calI^{-1}
{\begin{bmatrix}
\delta_{rc}
\end{bmatrix}}_r
+
\inv{2}
\left({\begin{bmatrix}
\delta_{rc}
\end{bmatrix}}_r \right)^\T
\calI^{-1} \left(\PD{\theta_r}{Q}\right)^\T \calI^{-1} \Bp
\\
&=
\Bp^\T \calI^{-1} \left( \PD{\theta_r}{}
\inv{2} \left( Q + Q^T \right) \right)
\calI^{-1}
{\begin{bmatrix}
\delta_{rc}
\end{bmatrix}}_r \\
&=
\Bp^\T \calI^{-1} \PD{\theta_r}{\calI}
\calI^{-1}
{\begin{bmatrix}
\delta_{rc}
\end{bmatrix}}_r \\
\end{aligned}
\end{equation}

Since we also have \(B' B^{-1} + B (B^{-1})' = 0\), for invertible matrixes \(B\), this reduces to

\begin{equation}\label{eqn:hamiltonian:523}
\begin{aligned}
\PD{P_{\theta_c}}{}
\left( \inv{2} \Bp^\T \calI^{-1} \left(\PD{\theta_r}{Q}\right)^\T \calI^{-1} \Bp  \right)
&=
-\Bp^\T \PD{\theta_r}{(\calI^{-1})}
{\begin{bmatrix}
\delta_{rc}
\end{bmatrix}}_r \\
\end{aligned}
\end{equation}

Forming the matrix over all rows \(r\), and columns \(c\), we get a trailing identity multiplying from the right, and are left with

\begin{equation}\label{eqn:hamiltonian:voo31}
\begin{aligned}
{\begin{bmatrix}
\PD{P_{\theta_c}}{}
\left( \inv{2} \Bp^\T \calI^{-1} \left(\PD{\theta_r}{Q}\right)^\T \calI^{-1} \Bp  \right)
\end{bmatrix}}_{r,c}
&=
{\begin{bmatrix}
-\Bp^\T \PD{\theta_r}{(\calI^{-1})}
\end{bmatrix}}_{r}
=
{\begin{bmatrix}
-\PD{\theta_c}{(\calI^{-1})} \Bp
\end{bmatrix}}_{c}
\end{aligned}
\end{equation}

Okay, getting closer.  The only thing left is to consider the remaining \(\theta\) dependence of \eqnref{eqn:hamiltonian:voo30c}, and now want the theta partials of the scalar matrix elements

\begin{equation}\label{eqn:hamiltonian:543}
\begin{aligned}
\PD{\theta_c}{}&
\left( \inv{2} \Bp^\T \calI^{-1} \left(\PD{\theta_r}{Q}\right)^\T \calI^{-1} \Bp  \right) \\
&=
\Bp^\T
\left(
\PD{\theta_c}{}
\left(
\inv{2} \calI^{-1} \left(\PD{\theta_r}{Q}\right)^\T \calI^{-1}
\right)
\right)
\Bp  \\
&=
\Bp^\T
\inv{2} \calI^{-1} \frac{\partial^2 Q^\T}{\partial \theta_c \partial \theta_r} \calI^{-1}
\Bp
+
\Bp^\T
\inv{2} \left(
\PD{\theta_c}{(\calI^{-1})} \left(\PD{\theta_r}{Q}\right)^\T \calI^{-1}
+\calI^{-1} \left(\PD{\theta_r}{Q}\right)^\T \PD{\theta_c}{(\calI^{-1} )}
\right)
\Bp \\
&=
\Bp^\T
\inv{2} \calI^{-1} \frac{\partial^2 Q^\T}{\partial \theta_c \partial \theta_r} \calI^{-1}
\Bp
+
\Bp^\T
\PD{\theta_c}{(\calI^{-1})} \PD{\theta_r}{\calI} \calI^{-1}
\Bp \\
\end{aligned}
\end{equation}

There is a slight asymmetry between the first and last terms here that can possibly be eliminated.  Using \({B^{-1}}' = -B^{-1} B' B^{-1}\), we can factor out the \(\calI^{-1}\Bp = \dot{\BTheta}\) terms
% I' = - I {I^{-1}}' I

\begin{equation}\label{eqn:hamiltonian:563}
\begin{aligned}
\PD{\theta_c}{}
\left( \inv{2} \Bp^\T \calI^{-1} \left(\PD{\theta_r}{Q}\right)^\T \calI^{-1} \Bp  \right)
&=
\dot{\BTheta}^\T
\left(
\inv{2} \frac{\partial^2 Q^\T}{\partial \theta_c \partial \theta_r}
-
\PD{\theta_c}{\calI}
\calI^{-1}
\PD{\theta_r}{\calI}
\right)
\dot{\BTheta}
\end{aligned}
\end{equation}

Is this any better?  Maybe a bit.  Since we are forming the matrix over all \(r,c\) indices and can assume mixed partial commutation the transpose can be dropped leaving us with

\begin{equation}\label{eqn:hamiltonian:voo32}
\begin{aligned}
{\begin{bmatrix}
\PD{\theta_c}{}
\left( \inv{2} \Bp^\T \calI^{-1} \left(\PD{\theta_r}{Q}\right)^\T \calI^{-1} \Bp  \right)
\end{bmatrix}}_{r,c}
&=
{\begin{bmatrix}
\dot{\BTheta}^\T
\left(
\inv{2} \frac{\partial^2 Q}{\partial \theta_c \partial \theta_r}
-
\PD{\theta_c}{\calI}
\calI^{-1}
\PD{\theta_r}{\calI}
\right)
\dot{\BTheta}
\end{bmatrix}}_{r,c}
\end{aligned}
\end{equation}

We can now assemble all these individual derivatives

\begin{equation}\label{eqn:hamiltonian:voo29a}
\begin{aligned}
\dot{\Bz}
&\approx
\evalnobar{
\begin{bmatrix}
{\begin{bmatrix}
\inv{2} \dot{\BTheta} \left(\PD{\theta_r}{Q}\right)^\T \dot{\BTheta}
\end{bmatrix}}_r
- \grad \phi \\
\dot{\BTheta}
\end{bmatrix}
}{t=0}
%\\
%&
+
\evalnobar{
\begin{bmatrix}
-{\begin{bmatrix}
\PD{\theta_c}{(\calI^{-1})} \Bp
\end{bmatrix}}_{c} &
{\begin{bmatrix}
\dot{\BTheta}^\T
\left(
\inv{2} \frac{\partial^2 Q}{\partial \theta_c \partial \theta_r}
-
\PD{\theta_c}{\calI}
\calI^{-1}
\PD{\theta_r}{\calI}
\right)
\dot{\BTheta}
-\frac{\partial^2 \phi}{ \partial \theta_r \partial \theta_c }
\end{bmatrix}}_{r,c} \\
\\
\calI^{-1}
&
{\begin{bmatrix}
\PD{\theta_c}{(\calI^{-1})} \Bp \\
\end{bmatrix}}_{c}
\end{bmatrix}
}{t=0} \Bz
\end{aligned}
\end{equation}

We have both \(\PDi{\theta_k}{(\calI^{-1})}\) and \(\PDi{\theta_k}{\calI}\) derivatives above, which will complicate things when trying to evaluate this for any specific system.  A final elimination of the derivatives of the inverse inertial matrix leaves us with

\begin{equation}\label{eqn:hamiltonian:voo29b}
\begin{aligned}
\dot{\Bz}
&\approx
\evalnobar{
\begin{bmatrix}
{\begin{bmatrix}
\inv{2} \dot{\BTheta} \left(\PD{\theta_r}{Q}\right)^\T \dot{\BTheta}
\end{bmatrix}}_r
- \grad \phi \\
\dot{\BTheta}
\end{bmatrix}
}{t=0}
+
\evalnobar{
\begin{bmatrix}
{\begin{bmatrix}
\calI^{-1} \PD{\theta_c}{\calI} \dot{\BTheta}
\end{bmatrix}}_{c} &
{\begin{bmatrix}
\dot{\BTheta}^\T
\left(
\inv{2} \frac{\partial^2 Q}{\partial \theta_c \partial \theta_r}
-
\PD{\theta_c}{\calI}
\calI^{-1}
\PD{\theta_r}{\calI}
\right)
\dot{\BTheta}
-\frac{\partial^2 \phi}{ \partial \theta_r \partial \theta_c }
\end{bmatrix}}_{r,c} \\
\\
\calI^{-1}
&
-
{\begin{bmatrix}
\calI^{-1} \PD{\theta_c}{\calI} \dot{\BTheta}
\end{bmatrix}}_{c}
\end{bmatrix}
}{t=0} \Bz
\end{aligned}
\end{equation}

\subsubsection{Single pendulum verification}

Having accumulated this unholy mess of abstraction, lets verify this first against the previous result obtained for the single planar pendulum.  Then if that checks out, calculate these matrices explicitly for the double and multiple pendulum cases.  For the single mass pendulum we have

\begin{equation}\label{eqn:hamiltonian:voo40}
\begin{aligned}
Q &= \calI = m l^2 \\
\phi &= - m g l \cos\theta
\end{aligned}
\end{equation}

So all the \(\theta\) partials except that of the potential are zero.  For the potential we have

\begin{equation}\label{eqn:hamiltonian:voo41}
\begin{aligned}
\evalbar{- \frac{\partial^2 \phi}{\partial^2 \theta}}{0} = - m g l \cos\theta_0
\end{aligned}
\end{equation}

and for the angular gradient

\begin{equation}\label{eqn:hamiltonian:voo42}
\begin{aligned}
\evalbar{-\grad \phi}{0} =
\begin{bmatrix}
- m g l \sin\theta_0
\end{bmatrix}
\end{aligned}
\end{equation}

Putting these all together in this simplest application of \eqnref{eqn:hamiltonian:voo29b} we have for the linear approximation of a single point mass pendulum about some point in phase space at time zero:

\begin{equation}\label{eqn:hamiltonian:voo43}
\begin{aligned}
\dot{\Bz} \approx
\begin{bmatrix}
- m g l \sin\theta_0 \\
\dot{\theta}_0
\end{bmatrix}
+\begin{bmatrix}
0 & - m g l \cos\theta_0 \\
\inv{m l^2} & 0
\end{bmatrix} \Bz
\end{aligned}
\end{equation}

Excellent.  Have not gotten into too much trouble with the math so far.  This is consistent with the previous results obtained considering the simple pendulum directly (it actually pointed out an error in the earlier pendulum treatment which is now fixed (I had dropped the \(\dot{\theta}_0\) term)).

\subsubsection{Double pendulum explicitly}

For the double pendulum, with \(\delta = \theta_1 - \theta_2\), and \(M = m_1 + m_2\), we have

\begin{equation}\label{eqn:hamiltonian:xoo1}
\begin{aligned}
Q =
\begin{bmatrix}
M {l_1}^2 & m_2 l_2 l_1 e^{i(\theta_2 - \theta_1)} \\
m_2 l_1 l_2 e^{i(\theta_1 - \theta_2)} & m_2 {l_2}^2 \\
\end{bmatrix}
=
\begin{bmatrix}
M {l_1}^2 & m_2 l_2 l_1 e^{-i\delta} \\
m_2 l_1 l_2 e^{i\delta} & m_2 {l_2}^2 \\
\end{bmatrix}
\end{aligned}
\end{equation}

\begin{equation}\label{eqn:hamiltonian:583}
\begin{aligned}
\inv{2} \dot{\BTheta}^\T \left(\PD{\theta_1}{Q} \right)^\T \dot{\BTheta}
&=
\inv{2} m_2 l_1 l_2 i
\dot{\BTheta}^\T
{\begin{bmatrix}
0 & - e^{-i\delta} \\
e^{i\delta} & 0
\end{bmatrix}}^\T
\dot{\BTheta} \\
&=
\inv{2} m_2 l_1 l_2 i
\dot{\BTheta}^\T
\begin{bmatrix}
e^{i\delta} \dot{\theta}_2 \\
-e^{-i\delta} \dot{\theta}_1
\end{bmatrix} \\
&=
\inv{2} m_2 l_1 l_2 i \dot{\theta}_1 \dot{\theta}_2 (e^{i\delta} -e^{-i\delta}) \\
&=
-m_2 l_1 l_2 \dot{\theta}_1 \dot{\theta}_2 \sin\delta
\end{aligned}
\end{equation}

The \(\theta_2\) derivative is the same but inverted in sign, so we have most of the constant term calculated.  We need the potential gradient to complete.  Our potential was

\begin{equation}\label{eqn:hamiltonian:xoo2}
\begin{aligned}
\phi = - M l_1 g \cos{\theta_1} - m_2 l_2 g \cos{\theta_2}
\end{aligned}
\end{equation}

So, the gradient is

\begin{equation}\label{eqn:hamiltonian:xoo3}
\begin{aligned}
\grad \phi =
\begin{bmatrix}
M l_1 g \sin{\theta_1} \\
m_2 l_2 g \sin{\theta_2}
\end{bmatrix}
\end{aligned}
\end{equation}

Putting things back together we have for the linear approximation of the two pendulum system

\begin{equation}\label{eqn:hamiltonian:xoo4}
\begin{aligned}
\dot{\Bz} =
\evalnobar{
\begin{bmatrix}
m_2 l_1 l_2 \dot{\theta}_1 \dot{\theta}_2 \sin(\theta_1 - \theta_2)
\begin{bmatrix}
-1 \\
1
\end{bmatrix}
& - g
\begin{bmatrix}
M l_1 \sin{\theta_1} \\
m_2 l_2 \sin{\theta_2} \\
\end{bmatrix}
\\
\dot{\theta}_1 \\
\dot{\theta}_2 \\
\end{bmatrix}}{t=0}
+ A \Bz
\end{aligned}
\end{equation}

Where \(A\) is still to be determined (from \eqnref{eqn:hamiltonian:voo29b}).

One of the elements of \(A\) are the matrix of potential derivatives.  These are

\begin{equation}\label{eqn:hamiltonian:potentialDerivatives}
\begin{aligned}
\begin{bmatrix}
\PD{\theta_1}{\grad \phi} & \PD{\theta_2}{\grad \phi}
\end{bmatrix}
=
\begin{bmatrix}
M l_1 g \cos\theta_1 & 0 \\
0 & m_2 l_2 g \cos\theta_2
\end{bmatrix}
\end{aligned}
\end{equation}

We also need the inertial matrix and its inverse.  These are

\begin{equation}\label{eqn:hamiltonian:xoo6}
\begin{aligned}
\calI
=
\begin{bmatrix}
M {l_1}^2 & m_2 l_2 l_1 \cos\delta \\
m_2 l_1 l_2 \cos\delta & m_2 {l_2}^2 \\
\end{bmatrix}
\end{aligned}
\end{equation}

\begin{equation}\label{eqn:hamiltonian:Iinverse}
\begin{aligned}
\calI^{-1}
=
\inv{{l_1}^2 {l_2}^2 m_2 (M - m_2 \cos^2\delta)}
\begin{bmatrix}
m_2 {l_2}^2 & -m_2 l_2 l_1 \cos\delta \\
-m_2 l_1 l_2 \cos\delta & M {l_1}^2 \\
\end{bmatrix}
\end{aligned}
\end{equation}

Since

\begin{equation}\label{eqn:hamiltonian:xoo8}
\begin{aligned}
\PD{\theta_1}{Q}
&=
m_2 l_1 l_2 i
\begin{bmatrix}
0 & - e^{-i\delta} \\
e^{i\delta} & 0
\end{bmatrix}
\end{aligned}
\end{equation}

We have
\begin{equation}\label{eqn:hamiltonian:xoo9}
\begin{aligned}
\PD{\theta_1}{} \PD{\theta_1}{Q} &=
-m_2 l_1 l_2
\begin{bmatrix}
0 & e^{-i\delta} \\
e^{i\delta} & 0
\end{bmatrix} \\
\PD{\theta_2}{} \PD{\theta_1}{Q} &=
m_2 l_1 l_2
\begin{bmatrix}
0 & e^{-i\delta} \\
e^{i\delta} & 0
\end{bmatrix} \\
\PD{\theta_1}{} \PD{\theta_2}{Q} &=
m_2 l_1 l_2
\begin{bmatrix}
0 & e^{-i\delta} \\
e^{i\delta} & 0
\end{bmatrix}  \\
\PD{\theta_2}{} \PD{\theta_2}{Q} &=
-m_2 l_1 l_2
\begin{bmatrix}
0 & e^{-i\delta} \\
e^{i\delta} & 0
\end{bmatrix}
\end{aligned}
\end{equation}

and the matrix of derivatives becomes

\begin{equation}\label{eqn:hamiltonian:matrixOfSecondPartials}
\begin{aligned}
\inv{2}\dot{\BTheta}^\T \PD{\theta_c}{}\PD{\theta_r}{Q} \dot{\BTheta}
=
m_2 l_1 l_2
\dot{\theta_1}
\dot{\theta_2} \cos(\theta_1 - \theta_2)
\begin{bmatrix}
-1 & 1 \\
1 & -1
\end{bmatrix}
\end{aligned}
\end{equation}

For the remaining two types of terms in the matrix \(A\) we need \(\calI^{-1} \PDi{\theta_k}{\calI}\).   The derivative of the inertial matrix is

\begin{equation}\label{eqn:hamiltonian:xoo11}
\begin{aligned}
\PD{\theta_k}{\calI}
=
-m_2 l_1 l_2 (\delta_{k1} - \delta_{k2})
\begin{bmatrix}
0 & \sin\delta \\
\sin\delta & 0
\end{bmatrix}
\end{aligned}
\end{equation}

Computing the product
\begin{equation}\label{eqn:hamiltonian:603}
\begin{aligned}
\calI^{-1} \PD{\theta_k}{\calI}
&=
\frac{ -m_2 l_1 l_2 (\delta_{k1} - \delta_{k2}) }{{l_1}^2 {l_2}^2 m_2 (M - m_2 \cos^2\delta)}
\begin{bmatrix}
m_2 {l_2}^2 & -m_2 l_2 l_1 \cos\delta \\
-m_2 l_1 l_2 \cos\delta & M {l_1}^2 \\
\end{bmatrix}
\begin{bmatrix}
0 & \sin\delta \\
\sin\delta & 0
\end{bmatrix} \\
&=
\frac{ -m_2 l_1 l_2 (\delta_{k1} - \delta_{k2}) \sin\delta}{{l_1}^2 {l_2}^2 m_2 (M - m_2 \cos^2\delta)}
\begin{bmatrix}
-m_2 l_2 l_1 \cos\delta & m_2 {l_2}^2 \\
M {l_1}^2 & -m_2 l_1 l_2 \cos\delta \\
\end{bmatrix}
\end{aligned}
\end{equation}

We want the matrix of \(\calI^{-1} \PDi{\theta_c}{\calI} \dot{\BTheta}\) over columns \(c\), and this is

\begin{equation}\label{eqn:hamiltonian:veryMessy}
\begin{aligned}
{\begin{bmatrix}
\calI^{-1} \PDi{\theta_c}{\calI} \dot{\BTheta}
\end{bmatrix}}_{c}
=
\frac{ m_2 l_1 l_2 \sin\delta}{{l_1}^2 {l_2}^2 m_2 (M - m_2 \cos^2\delta)}
\begin{bmatrix}
m_2 l_2 l_1 \cos\delta \dot{\theta}_1 - m_2 {l_2}^2 \dot{\theta}_2 & -m_2 l_2 l_1 \cos\delta \dot{\theta}_1 + m_2 {l_2}^2 \dot{\theta}_2 \\
-M {l_1}^2 \dot{\theta}_1 +m_2 l_1 l_2 \cos\delta \dot{\theta}_2 & M {l_1}^2 \dot{\theta}_1 -m_2 l_1 l_2 \cos\delta \dot{\theta}_2
\end{bmatrix}
\end{aligned}
\end{equation}

Very messy.  Perhaps it would be better not even bothering to expand this explicitly?  The last term in the matrix \(A\) is probably no better.  For that we want

\begin{equation}\label{eqn:hamiltonian:623}
\begin{aligned}
-\PD{\theta_c}{\calI} \calI^{-1} \PD{\theta_r}{\calI}
&=
\frac{ -{m_2}^2 {l_1}^2 {l_2}^2
(\delta_{c1} - \delta_{c2}) (\delta_{r1} - \delta_{r2}) \sin^2\delta}{{l_1}^2 {l_2}^2 m_2 (M - m_2 \cos^2\delta)}
\begin{bmatrix}
0 & 1 \\
1 & 0
\end{bmatrix}
\begin{bmatrix}
-m_2 l_2 l_1 \cos\delta & m_2 {l_2}^2 \\
M {l_1}^2 & -m_2 l_1 l_2 \cos\delta \\
\end{bmatrix} \\
&=
\frac{ -{m_2}^2 {l_1}^2 {l_2}^2
(\delta_{c1} - \delta_{c2}) (\delta_{r1} - \delta_{r2}) \sin^2\delta}{{l_1}^2 {l_2}^2 m_2 (M - m_2 \cos^2\delta)}
\begin{bmatrix}
M {l_1}^2 & -m_2 l_1 l_2 \cos\delta \\
-m_2 l_2 l_1 \cos\delta & m_2 {l_2}^2 \\
\end{bmatrix} \\
\end{aligned}
\end{equation}

With a sandwich of this between \(\dot{\BTheta}^\T\) and \(\dot{\BTheta}\) we are almost there

\begin{equation}\label{eqn:hamiltonian:643}
\begin{aligned}
-
\dot{\BTheta}^\T
\PD{\theta_c}{\calI} \calI^{-1} \PD{\theta_r}{\calI}
\dot{\BTheta}
&=
\frac{ -{m_2}^2 {l_1}^2 {l_2}^2
(\delta_{c1} - \delta_{c2}) (\delta_{r1} - \delta_{r2}) \sin^2\delta}{{l_1}^2 {l_2}^2 m_2 (M - m_2 \cos^2\delta)}
\left( M {l_1}^2 {\dot{\theta}_1}^2
-
2
m_2 l_1 l_2 \cos\delta \dot{\theta_1} \dot{\theta}_2 +
%-m_2 l_2 l_1 \cos\delta \dot{\theta}_1 \dot{\theta}_2
+ m_2 {l_2}^2 {\dot{\theta}_2}^2 \right)
\end{aligned}
\end{equation}

we have a matrix of these scalars over \(r,c\), and that is

\begin{equation}\label{eqn:hamiltonian:quadraticPartials}
\begin{aligned}
{\begin{bmatrix}
-
\dot{\BTheta}^\T
\PD{\theta_c}{\calI} \calI^{-1} \PD{\theta_r}{\calI}
\dot{\BTheta}
\end{bmatrix}}_{rc}
&=
\frac{ {m_2}^2 {l_1}^2 {l_2}^2
\sin^2\delta}{{l_1}^2 {l_2}^2 m_2 (M - m_2 \cos^2\delta)}
\left(
M {l_1}^2 {\dot{\theta}_1}^2 - 2 m_2 l_1 l_2 \cos\delta \dot{\theta_1} \dot{\theta}_2
+ m_2 {l_2}^2 {\dot{\theta}_2}^2  \right)
\begin{bmatrix}
-1 &  1 \\
 1 & -1
\end{bmatrix}
%(\delta_{c1} - \delta_{c2}) (\delta_{r1} - \delta_{r2})
\end{aligned}
\end{equation}

Putting all the results for the matrix \(A\) together is going to make a disgusting mess, so lets summarize in block matrix form

\begin{equation}\label{eqn:hamiltonian:663}
\begin{aligned}%\label{eqn:hamiltonian:xoo12}
A &=
\evalnobar{\begin{bmatrix}
B & C \\
\calI^{-1} & -B
\end{bmatrix}}{t=0} \\
B &=
%{eqn:hamiltonian:veryMessy}
\frac{ m_2 l_1 l_2 \sin\delta}{{l_1}^2 {l_2}^2 m_2 (M - m_2 \cos^2\delta)}
\begin{bmatrix}
m_2 l_2 l_1 \cos\delta \dot{\theta}_1 - m_2 {l_2}^2 \dot{\theta}_2 & -m_2 l_2 l_1 \cos\delta \dot{\theta}_1 + m_2 {l_2}^2 \dot{\theta}_2 \\
-M {l_1}^2 \dot{\theta}_1 +m_2 l_1 l_2 \cos\delta \dot{\theta}_2 & M {l_1}^2 \dot{\theta}_1 -m_2 l_1 l_2 \cos\delta \dot{\theta}_2
\end{bmatrix} \\
C &=
\left(
% {eqn:hamiltonian:matrixOfSecondPartials}
m_2 l_1 l_2
\dot{\theta_1}
\dot{\theta_2} \cos \delta
+
% {eqn:hamiltonian:quadraticPartials}
\frac{ {m_2}^2 {l_1}^2 {l_2}^2
\sin^2\delta}{{l_1}^2 {l_2}^2 m_2 (M - m_2 \cos^2\delta)}
\left(
M {l_1}^2 {\dot{\theta}_1}^2 - 2 m_2 l_1 l_2 \cos\delta \dot{\theta_1} \dot{\theta}_2
+ m_2 {l_2}^2 {\dot{\theta}_2}^2  \right)
\right)
\begin{bmatrix}
-1 & 1 \\
1 & -1
\end{bmatrix} \\
&\qquad+
% {eqn:hamiltonian:potentialDerivatives}
\begin{bmatrix}
M l_1 g \cos\theta_1 & 0 \\
0 & m_2 l_2 g \cos\theta_2
\end{bmatrix} \\
%{eqn:hamiltonian:Iinverse}
\calI^{-1}
&=
\inv{{l_1}^2 {l_2}^2 m_2 (M - m_2 \cos^2\delta)}
\begin{bmatrix}
m_2 {l_2}^2 & -m_2 l_2 l_1 \cos\delta \\
-m_2 l_1 l_2 \cos\delta & M {l_1}^2 \\
\end{bmatrix} \\
%{eqn:hamiltonian:xoo4}
b &=
\begin{bmatrix}
m_2 l_1 l_2 \dot{\theta}_1 \dot{\theta}_2 \sin(\theta_1 - \theta_2)
\begin{bmatrix}
-1 \\
1
\end{bmatrix}
& - g
\begin{bmatrix}
M l_1 \sin{\theta_1} \\
m_2 l_2 \sin{\theta_2} \\
\end{bmatrix}
\\
\dot{\theta}_1 \\
\dot{\theta}_2 \\
\end{bmatrix}
\end{aligned}
\end{equation}

where these are all related by the first order matrix equation

\begin{equation}\label{eqn:hamiltonian:xoo20}
\begin{aligned}
\frac{d\Bz}{dt} &= \evalbar{\Bb}{t=0} + \evalbar{A}{t=0} \Bz
\end{aligned}
\end{equation}

Wow, even to just write down the equations required to get a linear approximation of the two pendulum system is horrendously messy, and this is not even trying to solve it.  Numerical and or symbolic computation is really called for here.  If one elected to do this numerically, which looks pretty much mandatory since the analytic way did not turn out to be simple even for just the two pendulum system, then one is probably better off going all the way back to \eqnref{eqn:hamiltonian:veryGeneral} and just calculating the increment for the trajectory using a very small time increment, and do this repeatedly (i.e. do a zeroth order numerical procedure instead of the first order which turns out much more complicated).

\subsection{Dangling mass connected by string to another}

TODO.

% not sure what I wanted to do for these two.  Do not quite fit with the rest being so specific.
%\subsection{Particle in non-velocity dependent potential}
%
%TODO.
%
%\subsection{Velocity dependent potential}
%
%TODO.

\subsection{Non-covariant Lorentz force}

In \citep{jackson1975cew}, the Lagrangian for a charged particle is given as (12.9) as

\begin{equation}\label{eqn:hamiltonian:em1}
\begin{aligned}
\LL = -m c^2 \sqrt{1 - \Bu^2/c^2} + \frac{e}{c} \Bu \cdot \BA - e \Phi.
\end{aligned}
\end{equation}

Let us work in detail from this to the Lorentz force law and the Hamiltonian and from the Hamiltonian again to the Lorentz force law using the Hamiltonian equations.  We should get the same results in each case, and have enough details in doing so to render the text a bit more comprehensible.

\subsubsection{Canonical momenta}

We need the conjugate momenta for both the Euler-Lagrange evaluation and the Hamiltonian, so lets get that first.  The components of this are

\begin{equation}\label{eqn:hamiltonian:683}
\begin{aligned}
\PD{\dot{x}_i}{\LL}
&= - \inv{2} m c^2 \gamma (-2/c^2) \dot{x}_i + \frac{e}{c} A_i \\
&= m \gamma \dot{x}_i + \frac{e}{c} A_i.
\end{aligned}
\end{equation}

In vector form the canonical momenta are then

\begin{equation}\label{eqn:hamiltonian:em2}
\begin{aligned}
\BP &= \gamma m \Bu + \frac{e}{c} \BA.
\end{aligned}
\end{equation}

\subsubsection{Euler-Lagrange evaluation}

Completing the Euler-Lagrange equation evaluation is the calculation of

\begin{equation}\label{eqn:hamiltonian:em2b}
\begin{aligned}
\frac{d\BP}{dt} = \spacegrad \LL.
\end{aligned}
\end{equation}

On the left hand side we have

\begin{equation}\label{eqn:hamiltonian:em2l}
\begin{aligned}
\frac{d\BP}{dt} = \frac{d(\gamma m \Bu)}{dt} + \frac{e}{c} \frac{d\BA }{dt},
\end{aligned}
\end{equation}

and on the right, with implied summation over repeated indices, we have

\begin{equation}\label{eqn:hamiltonian:em2r}
\begin{aligned}
\spacegrad \LL = \frac{e}{c} \Be_k (\Bu \cdot \partial_k \BA) - e \spacegrad \Phi.
\end{aligned}
\end{equation}

Putting things together we have

\begin{equation}\label{eqn:hamiltonian:703}
\begin{aligned}
\frac{d(\gamma m \Bu)}{dt}
&=
-e \left(
\spacegrad \Phi + \inv{c} \PD{t}{\BA}
+ \frac{1}{c}
\left(
\PD{x_a}{\BA} \PD{t}{x_a} - \Be_k (\Bu \cdot \partial_k \BA)
\right)
\right) \\
&=
-e \left(
\spacegrad \Phi + \inv{c} \PD{t}{\BA}
+ \frac{1}{c} \Be_b u_a
\left(
\PD{x_a}{A_b}
-
\PD{x_b}{A_a}
\right)
\right).
\end{aligned}
\end{equation}

With

\begin{equation}\label{eqn:hamiltonian:em3}
\begin{aligned}
\BE = -\spacegrad \Phi - \inv{c} \PD{t}{\BA},
\end{aligned}
\end{equation}

the first two terms are recognizable as the electric field.  To put some structure in the remainder start by writing

\begin{equation}\label{eqn:hamiltonian:em4}
\begin{aligned}
\PD{x_a}{A_b} - \PD{x_b}{A_a} = \epsilon^{fab} {(\spacegrad \cross \BA)}_f.
\end{aligned}
\end{equation}

The remaining term, with \(\BB = \spacegrad \cross \BA\) is now

\begin{equation}\label{eqn:hamiltonian:723}
\begin{aligned}
- \frac{e}{c} \Be_b u_a \epsilon^{gab} B_g
&=
\frac{e}{c}
\Be_a u_b \epsilon^{abg} B_g \\
&=
\frac{e}{c} \Bu \cross \BB.
\end{aligned}
\end{equation}

We are left with the momentum portion of the Lorentz force law as expected

\begin{equation}\label{eqn:hamiltonian:em5}
\begin{aligned}
\frac{d(\gamma m \Bu)}{dt} = e \left( \BE + \frac{1}{c} \Bu \cross \BB \right).
\end{aligned}
\end{equation}

Observe that with a small velocity Taylor expansion of the Lagrangian we obtain the approximation

\begin{equation}\label{eqn:hamiltonian:em6}
\begin{aligned}
-m c^2 \sqrt{ 1 -\Bu^2/c^2} \approx - m c^2 \left( 1 - \inv{2} \Bu^2/c^2 \right) = \inv{2} m \Bu^2
\end{aligned}
\end{equation}

If that is our starting place, we can only obtain the non-relativistic approximation of the momentum change by evaluating the Euler-Lagrange equations

\begin{equation}\label{eqn:hamiltonian:em5a}
\begin{aligned}
\frac{d (m \Bu)}{dt} = e \left( \BE + \frac{1}{c} \Bu \cross \BB \right).
\end{aligned}
\end{equation}

That was an exercise previously attempting working the Tong Lagrangian problem set \citep{TongMf1}.

\subsubsection{Hamiltonian}

Having confirmed the by old fashioned Euler-Lagrange equation evaluation that our Lagrangian provides the desired equations of motion, let us now try it using the Hamiltonian approach.  First we need the Hamiltonian, which is nothing more than

\begin{equation}\label{eqn:hamiltonian:em10}
\begin{aligned}
H = \BP \cdot \Bu - \LL
\end{aligned}
\end{equation}

However, in the Lagrangian and the dot product we have velocity terms that we must eliminate in favor of the canonical momenta.  The Hamiltonian remains valid in either form, but to apply the Hamiltonian equations we need \(H = H(\BP, \Bx)\), and not \(H = H(\Bu, \BP, \Bx)\).

\begin{equation}\label{eqn:hamiltonian:em11}
\begin{aligned}
H = \BP \cdot \Bu + m c^2 \sqrt{1 - \Bu^2/c^2} - \frac{e}{c} \Bu \cdot \BA + e \Phi.
\end{aligned}
\end{equation}

Or
\begin{equation}\label{eqn:hamiltonian:em11b}
\begin{aligned}
H = \Bu \cdot \left(\BP - \frac{e}{c} \BA\right) + m c^2 \sqrt{1 - \Bu^2/c^2} + e \Phi.
\end{aligned}
\end{equation}

We can rearrange \eqnref{eqn:hamiltonian:em2} for \(\Bu\)

\begin{equation}\label{eqn:hamiltonian:em12}
\begin{aligned}
\Bu = \inv{m \gamma} \left( \BP - \frac{e}{c} \BA \right),
\end{aligned}
\end{equation}

but \(\gamma\) also has a \(\Bu\) dependence, so this is not complete.  Squaring gets us closer

\begin{equation}\label{eqn:hamiltonian:em13}
\begin{aligned}
\Bu^2 = \frac{1 - \Bu^2/c^2}{m^2} {\left( \BP - \frac{e}{c} \BA \right)}^2,
\end{aligned}
\end{equation}

and a bit of final rearrangement yields

\begin{equation}\label{eqn:hamiltonian:em14}
\begin{aligned}
\Bu^2 = \frac{(c \BP - e \BA)^2}{m^2 c^2 + {\left( \BP - \frac{e}{c} \BA \right)}^2}.
\end{aligned}
\end{equation}

Writing \(\Bp = \BP - e \BA/c\), we can rearrange and find

\begin{equation}\label{eqn:hamiltonian:em14a}
\begin{aligned}
\sqrt{1 - \Bu^2/c^2} = \frac{m c }{\sqrt{m^2 c^2 +\Bp^2}}
\end{aligned}
\end{equation}

Also, taking roots of \eqnref{eqn:hamiltonian:em14} we must have the directions of \(\Bu\) and \(\left( \BP - \frac{e}{c} \BA \right)\) differ only by a rotation.  From \eqnref{eqn:hamiltonian:em12} we also know that these are colinear, and therefore have

\begin{equation}\label{eqn:hamiltonian:em15}
\begin{aligned}
\Bu = \frac{c \BP - e \BA}{\sqrt{m^2 c^2 + {\left( \BP - \frac{e}{c} \BA \right)}^2}}.
\end{aligned}
\end{equation}

This and \eqnref{eqn:hamiltonian:em14a} can now be substituted into \eqnref{eqn:hamiltonian:em11b}, for

\begin{equation}\label{eqn:hamiltonian:em11c}
\begin{aligned}
H = \frac{c}{m^2 c^2 + \Bp^2}
\left(
{\left(\BP - \frac{e}{c} \BA\right)}^2 + m^2 c^2
\right)
+ e \Phi.
\end{aligned}
\end{equation}

Dividing out the common factors we finally have the Hamiltonian in a tidy form

\begin{equation}\label{eqn:hamiltonian:em20}
\begin{aligned}
H = \sqrt{ (c \BP - e \BA)^2 + m^2 c^4 } + e\Phi.
\end{aligned}
\end{equation}

\subsubsection{Hamiltonian equation evaluation}

Let us now go through the exercise of evaluating the Hamiltonian equations.  We want the starting point to be just the energy expression \eqnref{eqn:hamiltonian:em20}, and the use of the Hamiltonian equations and none of what led up to that.  If we were given only this Hamiltonian and the Hamiltonian principle

\begin{subequations}
\begin{equation}\label{eqn:hamiltonian:em21}
\begin{aligned}
\PD{P_k}{H} &= u_k \\
\PD{x_k}{H} &= -\dot{P}_k,
\end{aligned}
\end{equation}
\end{subequations}

how far can we go?

For the particle velocity we have no \(\Phi\) dependence and get

\begin{equation}\label{eqn:hamiltonian:em22}
\begin{aligned}
u_k &= \frac{c (c P_k -e A_k)}{\sqrt{ (c \BP - e \BA)^2 + m^2 c^4 }}
\end{aligned}
\end{equation}

This is \eqnref{eqn:hamiltonian:em15} in coordinate form, one of our stepping stones on the way to the Hamiltonian, and we recover it quickly with our first set of derivatives.  We have the gradient part \(\dot{\BP} = -\spacegrad H\) of the Hamiltonian left to evaluate

\begin{equation}\label{eqn:hamiltonian:em23}
\begin{aligned}
\frac{d\BP}{dt} =
\frac{e (c P_k -e A_k) \spacegrad A_k }{\sqrt{ (c \BP - e \BA)^2 + m^2 c^4 }} - e \spacegrad \Phi.
\end{aligned}
\end{equation}

Or
\begin{equation}\label{eqn:hamiltonian:em23b}
\begin{aligned}
\frac{d\BP}{dt} = e \left( \frac{u_k}{c} \spacegrad A_k - \spacegrad \Phi \right)
\end{aligned}
\end{equation}

This looks nothing like the Lorentz force law.  Knowing that \(\BP - e\BA/c\) is of significance (because we know where we started which is kind of a cheat), we can subtract derivatives of this from both sides, and use the convective derivative operator \(d/dt = \PDi{t}{} + \Bu \cdot \spacegrad\) (ie. chain rule) yielding

\begin{equation}\label{eqn:hamiltonian:em23c}
\begin{aligned}
\frac{d}{dt}(\BP - e\BA/c) = e \left( -\inv{c}\PD{t}{\BA} - \inv{c} (\Bu \cdot \spacegrad) \BA + \frac{u_k}{c} \spacegrad A_k - \spacegrad \Phi \right).
\end{aligned}
\end{equation}

The first and last terms sum to the electric field, and we seen evaluating the Euler-Lagrange equations that the remainder is \(u_k \spacegrad A_k - (\Bu \cdot \spacegrad) \BA = \Bu \cross (\spacegrad \cross \BA)\).  We have therefore gotten close to the familiar Lorentz force law, and have

\begin{equation}\label{eqn:hamiltonian:em24}
\begin{aligned}
\frac{d}{dt}(\BP - e\BA/c) = e \left( \BE + \frac{\Bu}{c} \cross \BB \right).
\end{aligned}
\end{equation}

The only untidy detail left is that \(\BP - e \BA/c\) does not look much like \(\gamma m \Bu\), what we recognize as the relativistically corrected momentum.  We ought to have that implied somewhere and \eqnref{eqn:hamiltonian:em22} looks like the place.  That turns out to be the case, and some rearrangement gives us this directly

\begin{equation}\label{eqn:hamiltonian:em25}
\begin{aligned}
\BP - \frac{e}{c}\BA = \frac{m \Bu}{\sqrt{1 - \Bu^2/c^2}}
\end{aligned}
\end{equation}

This completes the exercise, and we have now obtained the momentum part of the Lorentz force law.  This is still unsatisfactory from a relativistic context since we do not have momentum and energy on equal footing.  We likely need to utilize a covariant Lagrangian and Hamiltonian formulation to fix up that deficiency.

\subsection{Covariant force free case}

TODO.

\subsection{Covariant Lorentz force}

TODO.

%
% Copyright � 2012 Peeter Joot.  All Rights Reserved.
% Licenced as described in the file LICENSE under the root directory of this GIT repository.
%

\chapter{Linear transformations that retain two by two positive definiteness}
\index{positive definite}
\index{linear transformation!positive definite}
\label{chap:quadraticForm}
%\blogpage{http://sites.google.com/site/peeterjoot/math2009/quadraticForm.pdf}
%\date{Oct 4, 2009}

\section{Motivation}

Purely for fun, lets study the classes of linear transformations that retain the positive definiteness of a diagonal two by two quadratic form.  Namely, the Hamiltonian

\begin{equation}\label{eqn:quadratricForm:voo1}
\begin{aligned}
H = P^2 + Q^2
\end{aligned}
\end{equation}

under a change of variables that mixes position and momenta coordinates in phase space

\begin{equation}\label{eqn:quadratricForm:voo2}
\begin{aligned}
\Bz' =
\begin{bmatrix}
p \\
q
\end{bmatrix}
=
\begin{bmatrix}
\alpha & \beta \\
a      & b \\
\end{bmatrix}
\begin{bmatrix}
P \\
Q
\end{bmatrix}
= A \Bz
\end{aligned}
\end{equation}

We want the conditions on the matrix \(A\) such that the quadratic form retains the diagonal nature

\begin{equation}\label{eqn:quadratricForm:voo3a}
\begin{aligned}
H = P^2 + Q^2 = p^2 + q^2
\end{aligned}
\end{equation}

which in matrix form is

\begin{equation}\label{eqn:quadratricForm:voo3}
\begin{aligned}
H = \Bz^\T \Bz = {\Bz'}^\T \Bz'
\end{aligned}
\end{equation}

So the task is to solve for the constants on the matrix elements for

\begin{equation}\label{eqn:quadratricForm:voo4}
\begin{aligned}
I = A^\T A =
\begin{bmatrix}
\alpha & a     \\
\beta  & b \\
\end{bmatrix}
\begin{bmatrix}
\alpha & \beta \\
a      & b \\
\end{bmatrix}
\end{aligned}
\end{equation}

Strictly speaking we can also scale and retain positive definiteness, but that case is not of interest to me right now so I will use this term as described above.

\section{Guts}

The expectation is that this will necessarily include all rotations.  Will there be any other allowable linear transformations?  Written out in full we want the solutions of

\begin{equation}\label{eqn:quadratricForm:voo5a}
\begin{aligned}
\begin{bmatrix}
1 & 0 \\
0 & 1 \\
\end{bmatrix}
=
\begin{bmatrix}
\alpha & a     \\
\beta  & b \\
\end{bmatrix}
\begin{bmatrix}
\alpha & \beta \\
a      & b \\
\end{bmatrix}
=
\begin{bmatrix}
\alpha^2 + a^2 & \alpha \beta + a b \\
\alpha \beta + a b & \beta^2 + b^2
\end{bmatrix}
\end{aligned}
\end{equation}

Written out explicitly we have three distinct equations to reduce

\begin{equation}\label{eqn:quadratricForm:voo5}
\begin{aligned}
1 = \alpha^2 + a^2
\end{aligned}
\end{equation}
\begin{equation}\label{eqn:quadratricForm:voo6}
\begin{aligned}
1 = \beta^2 + b^2
\end{aligned}
\end{equation}
\begin{equation}\label{eqn:quadratricForm:voo7}
\begin{aligned}
0 = \alpha \beta + a b
\end{aligned}
\end{equation}

Solving for \(a\) in \eqnref{eqn:quadratricForm:voo7} we have

\begin{equation}\label{eqn:quadraticForm:39}
\begin{aligned}
a &= -\frac{\alpha \beta}{b} \\
\implies \\
1
&= \alpha^2 \left( 1 + \left(-\frac{\beta}{b} \right)^2 \right) \\
&= \frac{\alpha^2}{b^2} \left( b^2 + \beta^2 \right) \\
&= \frac{\alpha^2}{b^2}
\end{aligned}
\end{equation}

So, provided \(b \ne 0)\), we have a first simplifying identity

\begin{equation}\label{eqn:quadratricForm:voo8}
\begin{aligned}
\alpha^2 = b^2
\end{aligned}
\end{equation}

Written out to check, this reduces our system of equations

\begin{equation}\label{eqn:quadratricForm:voo9}
\begin{aligned}
\begin{bmatrix}
1 & 0 \\
0 & 1 \\
\end{bmatrix}
=
\begin{bmatrix}
\alpha & a     \\
\beta  & \pm \alpha \\
\end{bmatrix}
\begin{bmatrix}
\alpha & \beta \\
a      & \pm \alpha \\
\end{bmatrix}
=
\begin{bmatrix}
\alpha^2 + a^2 & \alpha \beta \pm a \alpha \\
\alpha \beta \pm a \alpha & \beta^2 + \alpha^2
\end{bmatrix}
\end{aligned}
\end{equation}

so our equations are now

\begin{equation}\label{eqn:quadratricForm:voo10}
\begin{aligned}
1 = \alpha^2 + a^2
\end{aligned}
\end{equation}
\begin{equation}\label{eqn:quadratricForm:voo11}
\begin{aligned}
1 = \beta^2 + \alpha^2
\end{aligned}
\end{equation}
\begin{equation}\label{eqn:quadratricForm:voo12}
\begin{aligned}
0 = \alpha (\beta \pm a )
\end{aligned}
\end{equation}

There are two cases to distinguish here.  The first is the more trivial \(\alpha = 0\) case, for which we find \(a^2 = \beta^2 = 1\).  For the other case we have

\begin{equation}\label{eqn:quadratricForm:voo13}
\begin{aligned}
\beta = \mp a
\end{aligned}
\end{equation}

Again, writing out in full to check, this reduces our system of equations

\begin{equation}\label{eqn:quadratricForm:voo14}
\begin{aligned}
\begin{bmatrix}
1 & 0 \\
0 & 1 \\
\end{bmatrix}
=
\begin{bmatrix}
\alpha & a     \\
\mp a  & \pm \alpha \\
\end{bmatrix}
\begin{bmatrix}
\alpha & \mp a \\
a      & \pm \alpha \\
\end{bmatrix}
=
\begin{bmatrix}
\alpha^2 + a^2 & 0 \\
0 & a^2 + \alpha^2
\end{bmatrix}
\end{aligned}
\end{equation}

We have now only one constraint left, and have reduced things to a single degree of freedom

\begin{equation}\label{eqn:quadratricForm:voo15}
\begin{aligned}
1 = \alpha^2 + a^2
\end{aligned}
\end{equation}

Or
\begin{equation}\label{eqn:quadratricForm:voo16}
\begin{aligned}
\alpha = (1 - a^2)^{1/2}
\end{aligned}
\end{equation}

We have already used \(\pm\) to distinguish the roots of \(\alpha = \pm b\), so here lets imply that this square root can take either positive or negative values, but that we are treating the sign of this the same where ever seen.  Our transformation, employing \(a\) as the free variable is now known to take any of the following forms

\begin{equation}\label{eqn:quadratricForm:voo17}
\begin{aligned}
A
&=
\begin{bmatrix}
(1 - a^2)^{1/2} & \mp a \\
a  & \pm (1 - a^2)^{1/2} \\
\end{bmatrix} \\
&=
\begin{bmatrix}
0 & \pm 1 \\
(1)^{1/2} & 0 \\
\end{bmatrix} \\
&=
\begin{bmatrix}
\alpha & \beta \\
a & 0 \\
\end{bmatrix}
\end{aligned}
\end{equation}

The last of these (the \(b=0\) case from earlier) was not considered, but doing so one finds that it produces nothing different from the second form of the transformation above.  That leaves us with two possible forms of linear transformations that are allowable for the desired constraints, the first of which screams for a trigonometric parametrization.

For \(\Abs{a} \le 0\) we can parametrize with \(a = \sin\theta\).  Should we allow complex valued linear transformations?  If so \(a = \cosh(\theta)\) is a reasonable way to parametrize the matrix for the \(a > 0\) case.  The complete set of allowable linear transformations in matrix form are now

\begin{equation}\label{eqn:quadratricForm:voo18}
\begin{aligned}
A &=
\begin{bmatrix}
1^{1/2} \cos\theta & \mp \sin\theta \\
\sin\theta & \pm 1^{1/2} \cos\theta \\
\end{bmatrix} \\
A &=
\begin{bmatrix}
(-1)^{1/2} \sinh\theta & \mp \cosh\theta \\
\cosh\theta  & \pm (-1)^{1/2} \sinh\theta \\
\end{bmatrix} \\
A &=
\begin{bmatrix}
0 & \pm 1 \\
1^{1/2} & 0 \\
\end{bmatrix}
\end{aligned}
\end{equation}

There are really four different matrices in each of the above.  Removing all the shorthand for clarity we have finally

\begin{equation}\label{eqn:quadraticForm:59}
\begin{aligned}
A \in \Bigr\{
&\begin{bmatrix}
\cos\theta & - \sin\theta \\
\sin\theta & \cos\theta \\
\end{bmatrix},
\begin{bmatrix}
-\cos\theta & - \sin\theta \\
\sin\theta & -\cos\theta \\
\end{bmatrix},
\begin{bmatrix}
\cos\theta & \sin\theta \\
\sin\theta & - \cos\theta \\
\end{bmatrix},
\begin{bmatrix}
-\cos\theta & \sin\theta \\
\sin\theta & \cos\theta \\
\end{bmatrix}, \\
&
\begin{bmatrix}
i \sinh\theta & -\cosh\theta \\
\cosh\theta  & i \sinh\theta \\
\end{bmatrix},
\begin{bmatrix}
-i \sinh\theta & -\cosh\theta \\
\cosh\theta  & -i \sinh\theta \\
\end{bmatrix},
\begin{bmatrix}
i \sinh\theta & \cosh\theta \\
\cosh\theta  & - i \sinh\theta \\
\end{bmatrix},
\begin{bmatrix}
-i \sinh\theta & \cosh\theta \\
\cosh\theta  & i \sinh\theta \\
\end{bmatrix},
\\
&
\begin{bmatrix}
0 & 1 \\
1 & 0 \\
\end{bmatrix},
\begin{bmatrix}
0 & 1 \\
-1 & 0 \\
\end{bmatrix},
\begin{bmatrix}
0 & -1 \\
1 & 0 \\
\end{bmatrix},
\begin{bmatrix}
0 & -1 \\
-1 & 0 \\
\end{bmatrix}
\Bigl\}
\end{aligned}
\end{equation}

The last four possibilities are now seen to be redundant since they can be incorporated into the \(\theta = \pm \pi/2\) cases of the real trig parameterizations where \(\sin\theta = \pm 1\), and \(\cos\theta = 0\).   Employing a \(\theta' = -\theta\) change of variables, we find that two of the hyperbolic parameterizations are also redundant and can express the reduced solution set as

\begin{equation}\label{eqn:quadratricForm:voo19}
\begin{aligned}
A \in \Bigr\{
\pm \begin{bmatrix}
\cos\theta & - \sin\theta \\
\sin\theta & \cos\theta \\
\end{bmatrix},
\pm \begin{bmatrix}
\cos\theta & \sin\theta \\
\sin\theta & - \cos\theta \\
\end{bmatrix},
\pm \begin{bmatrix}
i \sinh\theta & -\cosh\theta \\
\cosh\theta  & i \sinh\theta \\
\end{bmatrix},
\pm \begin{bmatrix}
i \sinh\theta & \cosh\theta \\
\cosh\theta  & - i \sinh\theta \\
\end{bmatrix}
\Bigl\}
\end{aligned}
\end{equation}

I suspect this class of transformations has a name in the grand group classification scheme, but I do not know what it is.

%
% Copyright � 2012 Peeter Joot.  All Rights Reserved.
% Licenced as described in the file LICENSE under the root directory of this GIT repository.
%
%{
\newcommand{\authorname}{Peeter Joot}
\newcommand{\email}{peeterjoot@protonmail.com}
\newcommand{\basename}{FIXMEbasenameUndefined}
\newcommand{\dirname}{notes/FIXMEdirnameUndefined/}

\renewcommand{\basename}{multiPendulumSphericalMatrix}
\renewcommand{\dirname}{notes/classicalmechanics/}
%\newcommand{\dateintitle}{}
%\newcommand{\keywords}{}

\newcommand{\authorname}{Peeter Joot}
\newcommand{\onlineurl}{http://sites.google.com/site/peeterjoot2/math2013/\basename.pdf}
\newcommand{\sourcepath}{\dirname\basename.tex}
\newcommand{\generatetitle}[1]{\chapter{#1}}

\newcommand{\vcsinfo}{%
\section*{}
\noindent{\color{DarkOliveGreen}{\rule{\linewidth}{0.1mm}}}
\paragraph{Document version}
%\paragraph{\color{Maroon}{Document version}}
{
\small
\begin{itemize}
\item Available online at:\\ 
\href{\onlineurl}{\onlineurl}
\item Git Repository: \input{./.revinfo/gitRepo.tex}
\item Source: \sourcepath
\item last commit: \input{./.revinfo/gitCommitString.tex}
\item commit date: \input{./.revinfo/gitCommitDate.tex}
\end{itemize}
}
}

%\PassOptionsToPackage{dvipsnames,svgnames}{xcolor}
\PassOptionsToPackage{square,numbers}{natbib}
\documentclass{scrreprt}

\usepackage[left=2cm,right=2cm]{geometry}
\usepackage[svgnames]{xcolor}
\usepackage{peeters_layout}

\usepackage{natbib}

\usepackage[
colorlinks=true,
bookmarks=false,
pdfauthor={\authorname, \email},
backref 
]{hyperref}

% http://tex.stackexchange.com/questions/75773/how-to-reference-problems-by-the-text-label-in-an-exercise-envioronment
\usepackage[english]{cleveref}
\crefname{Exercise}{exercise}{exercises}
\Crefname{Exercise}{Exercise}{Exercises}

\RequirePackage{titlesec}
\RequirePackage{ifthen}

% http://stackoverflow.com/questions/4932910/date-in-the-tabular-environment
\makeatletter
\let\insertdate\@date
\makeatother

\titleformat{\chapter}[display]
{\bfseries\Large}
{\color{DarkSlateGrey}\filleft \authorname
\ifthenelse{\isundefined{\studentnumber}}{}{\\ \studentnumber}
\ifthenelse{\isundefined{\email}}{}{\\ \email}
\ifthenelse{\isundefined{\dateintitle}}{}{\\ \insertdate}
%\ifthenelse{\isundefined{\coursename}}{}{\\ \coursename} % put in title instead.
}
{4ex}
{\color{DarkOliveGreen}{\titlerule}\color{Maroon}
\vspace{2ex}%
\filright}
[\vspace{2ex}%
\color{DarkOliveGreen}\titlerule
]

\newcommand{\beginArtWithToc}[0]{\begin{document}\tableofcontents}
\newcommand{\beginArtNoToc}[0]{\begin{document}}
\newcommand{\EndNoBibArticle}[0]{\end{document}}
\newcommand{\EndArticle}[0]{\bibliography{Bibliography}\bibliographystyle{plainnat}\end{document}}

% 
%\newcommand{\citep}[1]{\cite{#1}}

\colorSectionsForArticle



%\usepackage{peeters_layout_exercise}
%\usepackage{peeters_braket}
\usepackage{peeters_figures}
\usepackage{macros_cal}

\newcommand{\chapcite}[1]{\ref{chap:#1}}
\newcommand{\gpgradezeroNoOp}[1]{{#1}}

%\newcommand{\small}[1]{#1}

%%% %\documentclass[11pt,twocolumn]{article}
%%%
%%% % small font:
%%% %\documentclass[11pt]{article}
%%% %\setlength\topmargin{-0.5in}
%%% %\setlength\columnsep{0.2in}
%%% %\setlength\headsep{0.0in}
%%% %\setlength\textheight{9.5in}
%%% %\setlength\textwidth{7in}
%%% %\setlength\oddsidemargin{-0.25in}
%%% %\setlength\evensidemargin{-0.25in}
%%%
%%% % readable?  Font still seems small?
%%% \documentclass[11pt]{article}
%%% \setlength{\textwidth}{\paperwidth}
%%% \addtolength{\textwidth}{-2in}
%%% \setlength{\oddsidemargin}{0pt}
%%% \setlength{\evensidemargin}{0pt}
%%% \setlength{\headheight}{15pt}
%%% \setlength{\headsep}{15pt}
%%% \setlength{\topmargin}{0in}
%%% \addtolength{\topmargin}{-\headheight}
%%% \addtolength{\topmargin}{-\headsep}
%%% \setlength{\footskip}{29pt}
%%% \setlength{\textheight}{\paperheight}
%%% \addtolength{\textheight}{-2.2in}
%%% \setlength{\marginparwidth}{.5in}
%%% \setlength{\marginparsep}{5pt}
%%%
%%%
%%% %\usepackage{mathpazo}
%%% \usepackage{color}
%%% \usepackage{amsmath}
%%% \usepackage{amsfonts}
%%% \usepackage{graphicx}
%%% %\usepackage[bookmarks=false]{hyperref}
%%% \usepackage{hyperref}
%%% \usepackage{subfigure}
%%% \usepackage{titlesec}
%%% \usepackage{indentfirst}
%%%
%%% \newtheorem{theorem}{Theorem}[section]
%%% \newtheorem{definition}[theorem]{Definition}
%%% \newtheorem{axiom}[theorem]{Axiom}
%%%
%%% \newcommand{\Abs}[1]{{\left\lvert{#1}\right\rvert}}
%%% \newcommand{\evalbar}[2]{{\left.{#1}\right\vert}_{#2}}
%%% \newcommand{\evalnobar}[2]{{#1}_{#2}}
%%% \newcommand{\Bu}[0]{\mathbf{u}}
%%% \newcommand{\Bv}[0]{\mathbf{v}}
%%% \newcommand{\Be}[0]{\mathbf{e}}
%%% \newcommand{\Ba}[0]{\mathbf{a}}
%%% \newcommand{\Bb}[0]{\mathbf{b}}
%%% \newcommand{\Bc}[0]{\mathbf{c}}
%%% \newcommand{\Br}[0]{\mathbf{r}}
%%% \newcommand{\Bx}[0]{\mathbf{x}}
%%% \newcommand{\Bk}[0]{\mathbf{k}}
%%% \newcommand{\Bq}[0]{\mathbf{q}}
%%% \newcommand{\Bs}[0]{\mathbf{s}}
%%% \newcommand{\Bt}[0]{\mathbf{t}}
%%% \newcommand{\rcap}[0]{\hat{\Br}}
%%% \newcommand{\inv}[1]{\frac{1}{#1}}
%%% \newcommand{\grad}[0]{\nabla}
%%% \newcommand{\LL}[0]{\calL}
%%% \newcommand{\PD}[2]{\frac{\partial {#2}}{\partial {#1}}}
%%% \newcommand{\PDi}[2]{{\partial {#2}}/{\partial {#1}}}
%%% \newcommand{\gpgrade}[2] {{\left\langle{{#1}}\right\rangle}_{#2}}
%%% \newcommand{\gpgradezero}[1] {\gpgrade{#1}{}}
%%% \newcommand{\gpgradeone}[1] {\gpgrade{#1}{1}}
%%% \newcommand{\BTheta}[0]{\boldsymbol{\Theta}}
%%% \newcommand{\Brho}[0]{\boldsymbol{\rho}}
%%% \newcommand{\T}[0]{\text{T}}
%%%
%%% \begin{document}

%\chapter{Lagrangian and Euler-Lagrange equation evaluation for the spherical N-pendulum problem}
%\author{Peeter Joot
%\smallskip\\
%\small{e-mail: peeterjoot\(@\)protonmail.com}}
%\date{\small{\today}}
%\maketitle

\newcommand{\nbref}[1]{%
%\itemRef{classicalmechanics}{#1}%
%\index{Mathematica}%
fixme:nbref:#1
}

\beginArtNoToc

\generatetitle{Lagrangian and Euler-Lagrange equation evaluation for the spherical N-pendulum problem}
%
% Copyright � 2012 Peeter Joot.  All Rights Reserved.
% Licenced as described in the file LICENSE under the root directory of this GIT repository.
%
%{
\newcommand{\authorname}{Peeter Joot}
\newcommand{\email}{peeterjoot@protonmail.com}
\newcommand{\basename}{FIXMEbasenameUndefined}
\newcommand{\dirname}{notes/FIXMEdirnameUndefined/}

\renewcommand{\basename}{multiPendulumSphericalMatrix}
\renewcommand{\dirname}{notes/classicalmechanics/}
%\newcommand{\dateintitle}{}
%\newcommand{\keywords}{}

\newcommand{\authorname}{Peeter Joot}
\newcommand{\onlineurl}{http://sites.google.com/site/peeterjoot2/math2013/\basename.pdf}
\newcommand{\sourcepath}{\dirname\basename.tex}
\newcommand{\generatetitle}[1]{\chapter{#1}}

\newcommand{\vcsinfo}{%
\section*{}
\noindent{\color{DarkOliveGreen}{\rule{\linewidth}{0.1mm}}}
\paragraph{Document version}
%\paragraph{\color{Maroon}{Document version}}
{
\small
\begin{itemize}
\item Available online at:\\ 
\href{\onlineurl}{\onlineurl}
\item Git Repository: \input{./.revinfo/gitRepo.tex}
\item Source: \sourcepath
\item last commit: \input{./.revinfo/gitCommitString.tex}
\item commit date: \input{./.revinfo/gitCommitDate.tex}
\end{itemize}
}
}

%\PassOptionsToPackage{dvipsnames,svgnames}{xcolor}
\PassOptionsToPackage{square,numbers}{natbib}
\documentclass{scrreprt}

\usepackage[left=2cm,right=2cm]{geometry}
\usepackage[svgnames]{xcolor}
\usepackage{peeters_layout}

\usepackage{natbib}

\usepackage[
colorlinks=true,
bookmarks=false,
pdfauthor={\authorname, \email},
backref 
]{hyperref}

% http://tex.stackexchange.com/questions/75773/how-to-reference-problems-by-the-text-label-in-an-exercise-envioronment
\usepackage[english]{cleveref}
\crefname{Exercise}{exercise}{exercises}
\Crefname{Exercise}{Exercise}{Exercises}

\RequirePackage{titlesec}
\RequirePackage{ifthen}

% http://stackoverflow.com/questions/4932910/date-in-the-tabular-environment
\makeatletter
\let\insertdate\@date
\makeatother

\titleformat{\chapter}[display]
{\bfseries\Large}
{\color{DarkSlateGrey}\filleft \authorname
\ifthenelse{\isundefined{\studentnumber}}{}{\\ \studentnumber}
\ifthenelse{\isundefined{\email}}{}{\\ \email}
\ifthenelse{\isundefined{\dateintitle}}{}{\\ \insertdate}
%\ifthenelse{\isundefined{\coursename}}{}{\\ \coursename} % put in title instead.
}
{4ex}
{\color{DarkOliveGreen}{\titlerule}\color{Maroon}
\vspace{2ex}%
\filright}
[\vspace{2ex}%
\color{DarkOliveGreen}\titlerule
]

\newcommand{\beginArtWithToc}[0]{\begin{document}\tableofcontents}
\newcommand{\beginArtNoToc}[0]{\begin{document}}
\newcommand{\EndNoBibArticle}[0]{\end{document}}
\newcommand{\EndArticle}[0]{\bibliography{Bibliography}\bibliographystyle{plainnat}\end{document}}

% 
%\newcommand{\citep}[1]{\cite{#1}}

\colorSectionsForArticle



%\usepackage{peeters_layout_exercise}
%\usepackage{peeters_braket}
\usepackage{peeters_figures}
\usepackage{macros_cal}

\newcommand{\chapcite}[1]{\ref{chap:#1}}
\newcommand{\gpgradezeroNoOp}[1]{{#1}}

%\newcommand{\small}[1]{#1}

%%% %\documentclass[11pt,twocolumn]{article}
%%%
%%% % small font:
%%% %\documentclass[11pt]{article}
%%% %\setlength\topmargin{-0.5in}
%%% %\setlength\columnsep{0.2in}
%%% %\setlength\headsep{0.0in}
%%% %\setlength\textheight{9.5in}
%%% %\setlength\textwidth{7in}
%%% %\setlength\oddsidemargin{-0.25in}
%%% %\setlength\evensidemargin{-0.25in}
%%%
%%% % readable?  Font still seems small?
%%% \documentclass[11pt]{article}
%%% \setlength{\textwidth}{\paperwidth}
%%% \addtolength{\textwidth}{-2in}
%%% \setlength{\oddsidemargin}{0pt}
%%% \setlength{\evensidemargin}{0pt}
%%% \setlength{\headheight}{15pt}
%%% \setlength{\headsep}{15pt}
%%% \setlength{\topmargin}{0in}
%%% \addtolength{\topmargin}{-\headheight}
%%% \addtolength{\topmargin}{-\headsep}
%%% \setlength{\footskip}{29pt}
%%% \setlength{\textheight}{\paperheight}
%%% \addtolength{\textheight}{-2.2in}
%%% \setlength{\marginparwidth}{.5in}
%%% \setlength{\marginparsep}{5pt}
%%%
%%%
%%% %\usepackage{mathpazo}
%%% \usepackage{color}
%%% \usepackage{amsmath}
%%% \usepackage{amsfonts}
%%% \usepackage{graphicx}
%%% %\usepackage[bookmarks=false]{hyperref}
%%% \usepackage{hyperref}
%%% \usepackage{subfigure}
%%% \usepackage{titlesec}
%%% \usepackage{indentfirst}
%%%
%%% \newtheorem{theorem}{Theorem}[section]
%%% \newtheorem{definition}[theorem]{Definition}
%%% \newtheorem{axiom}[theorem]{Axiom}
%%%
%%% \newcommand{\Abs}[1]{{\left\lvert{#1}\right\rvert}}
%%% \newcommand{\evalbar}[2]{{\left.{#1}\right\vert}_{#2}}
%%% \newcommand{\evalnobar}[2]{{#1}_{#2}}
%%% \newcommand{\Bu}[0]{\mathbf{u}}
%%% \newcommand{\Bv}[0]{\mathbf{v}}
%%% \newcommand{\Be}[0]{\mathbf{e}}
%%% \newcommand{\Ba}[0]{\mathbf{a}}
%%% \newcommand{\Bb}[0]{\mathbf{b}}
%%% \newcommand{\Bc}[0]{\mathbf{c}}
%%% \newcommand{\Br}[0]{\mathbf{r}}
%%% \newcommand{\Bx}[0]{\mathbf{x}}
%%% \newcommand{\Bk}[0]{\mathbf{k}}
%%% \newcommand{\Bq}[0]{\mathbf{q}}
%%% \newcommand{\Bs}[0]{\mathbf{s}}
%%% \newcommand{\Bt}[0]{\mathbf{t}}
%%% \newcommand{\rcap}[0]{\hat{\Br}}
%%% \newcommand{\inv}[1]{\frac{1}{#1}}
%%% \newcommand{\grad}[0]{\nabla}
%%% \newcommand{\LL}[0]{\calL}
%%% \newcommand{\PD}[2]{\frac{\partial {#2}}{\partial {#1}}}
%%% \newcommand{\PDi}[2]{{\partial {#2}}/{\partial {#1}}}
%%% \newcommand{\gpgrade}[2] {{\left\langle{{#1}}\right\rangle}_{#2}}
%%% \newcommand{\gpgradezero}[1] {\gpgrade{#1}{}}
%%% \newcommand{\gpgradeone}[1] {\gpgrade{#1}{1}}
%%% \newcommand{\BTheta}[0]{\boldsymbol{\Theta}}
%%% \newcommand{\Brho}[0]{\boldsymbol{\rho}}
%%% \newcommand{\T}[0]{\text{T}}
%%%
%%% \begin{document}

%\chapter{Lagrangian and Euler-Lagrange equation evaluation for the spherical N-pendulum problem}
%\author{Peeter Joot
%\smallskip\\
%\small{e-mail: peeterjoot\(@\)protonmail.com}}
%\date{\small{\today}}
%\maketitle

\newcommand{\nbref}[1]{%
%\itemRef{classicalmechanics}{#1}%
%\index{Mathematica}%
fixme:nbref:#1
}

\beginArtNoToc

\generatetitle{Lagrangian and Euler-Lagrange equation evaluation for the spherical N-pendulum problem}
%
% Copyright � 2012 Peeter Joot.  All Rights Reserved.
% Licenced as described in the file LICENSE under the root directory of this GIT repository.
%
%{
\input{../latex/blogpost.tex}
\renewcommand{\basename}{multiPendulumSphericalMatrix}
\renewcommand{\dirname}{notes/classicalmechanics/}
%\newcommand{\dateintitle}{}
%\newcommand{\keywords}{}

\input{../latex/peeter_prologue_print2.tex}

%\usepackage{peeters_layout_exercise}
%\usepackage{peeters_braket}
\usepackage{peeters_figures}
\usepackage{macros_cal}

\newcommand{\chapcite}[1]{\ref{chap:#1}}
\newcommand{\gpgradezeroNoOp}[1]{{#1}}

%\newcommand{\small}[1]{#1}

%%% %\documentclass[11pt,twocolumn]{article}
%%%
%%% % small font:
%%% %\documentclass[11pt]{article}
%%% %\setlength\topmargin{-0.5in}
%%% %\setlength\columnsep{0.2in}
%%% %\setlength\headsep{0.0in}
%%% %\setlength\textheight{9.5in}
%%% %\setlength\textwidth{7in}
%%% %\setlength\oddsidemargin{-0.25in}
%%% %\setlength\evensidemargin{-0.25in}
%%%
%%% % readable?  Font still seems small?
%%% \documentclass[11pt]{article}
%%% \setlength{\textwidth}{\paperwidth}
%%% \addtolength{\textwidth}{-2in}
%%% \setlength{\oddsidemargin}{0pt}
%%% \setlength{\evensidemargin}{0pt}
%%% \setlength{\headheight}{15pt}
%%% \setlength{\headsep}{15pt}
%%% \setlength{\topmargin}{0in}
%%% \addtolength{\topmargin}{-\headheight}
%%% \addtolength{\topmargin}{-\headsep}
%%% \setlength{\footskip}{29pt}
%%% \setlength{\textheight}{\paperheight}
%%% \addtolength{\textheight}{-2.2in}
%%% \setlength{\marginparwidth}{.5in}
%%% \setlength{\marginparsep}{5pt}
%%%
%%%
%%% %\usepackage{mathpazo}
%%% \usepackage{color}
%%% \usepackage{amsmath}
%%% \usepackage{amsfonts}
%%% \usepackage{graphicx}
%%% %\usepackage[bookmarks=false]{hyperref}
%%% \usepackage{hyperref}
%%% \usepackage{subfigure}
%%% \usepackage{titlesec}
%%% \usepackage{indentfirst}
%%%
%%% \newtheorem{theorem}{Theorem}[section]
%%% \newtheorem{definition}[theorem]{Definition}
%%% \newtheorem{axiom}[theorem]{Axiom}
%%%
%%% \newcommand{\Abs}[1]{{\left\lvert{#1}\right\rvert}}
%%% \newcommand{\evalbar}[2]{{\left.{#1}\right\vert}_{#2}}
%%% \newcommand{\evalnobar}[2]{{#1}_{#2}}
%%% \newcommand{\Bu}[0]{\mathbf{u}}
%%% \newcommand{\Bv}[0]{\mathbf{v}}
%%% \newcommand{\Be}[0]{\mathbf{e}}
%%% \newcommand{\Ba}[0]{\mathbf{a}}
%%% \newcommand{\Bb}[0]{\mathbf{b}}
%%% \newcommand{\Bc}[0]{\mathbf{c}}
%%% \newcommand{\Br}[0]{\mathbf{r}}
%%% \newcommand{\Bx}[0]{\mathbf{x}}
%%% \newcommand{\Bk}[0]{\mathbf{k}}
%%% \newcommand{\Bq}[0]{\mathbf{q}}
%%% \newcommand{\Bs}[0]{\mathbf{s}}
%%% \newcommand{\Bt}[0]{\mathbf{t}}
%%% \newcommand{\rcap}[0]{\hat{\Br}}
%%% \newcommand{\inv}[1]{\frac{1}{#1}}
%%% \newcommand{\grad}[0]{\nabla}
%%% \newcommand{\LL}[0]{\calL}
%%% \newcommand{\PD}[2]{\frac{\partial {#2}}{\partial {#1}}}
%%% \newcommand{\PDi}[2]{{\partial {#2}}/{\partial {#1}}}
%%% \newcommand{\gpgrade}[2] {{\left\langle{{#1}}\right\rangle}_{#2}}
%%% \newcommand{\gpgradezero}[1] {\gpgrade{#1}{}}
%%% \newcommand{\gpgradeone}[1] {\gpgrade{#1}{1}}
%%% \newcommand{\BTheta}[0]{\boldsymbol{\Theta}}
%%% \newcommand{\Brho}[0]{\boldsymbol{\rho}}
%%% \newcommand{\T}[0]{\text{T}}
%%%
%%% \begin{document}

%\chapter{Lagrangian and Euler-Lagrange equation evaluation for the spherical N-pendulum problem}
%\author{Peeter Joot
%\smallskip\\
%\small{e-mail: peeterjoot\(@\)protonmail.com}}
%\date{\small{\today}}
%\maketitle

\newcommand{\nbref}[1]{%
%\itemRef{classicalmechanics}{#1}%
%\index{Mathematica}%
fixme:nbref:#1
}

\beginArtNoToc

\generatetitle{Lagrangian and Euler-Lagrange equation evaluation for the spherical N-pendulum problem}
\input{../classicalmechanics/mine/multiPendulumSphericalMatrix.tex}

%\bibliography{myrefs}
%\bibliographystyle{unsrturl}

%\section{graphics experiment}
%
%% TIP: draw the picture on graph paper and copy the numbers.
%% http://www.ursoswald.ch/LaTeXGraphics/picture/picture.html
%
%Forces on the Catenary
%
%% features: arrows, axis's, gravity vector.
%
%\setlength{\unitlength}{3cm}
%\begin{picture}(1.75, 2.75)(0, -0.1)
%  \put(0,0){\vector(1,0){1.75}}
%  \put(1.85, -0.03){\(x\)}
%  \put(0,0){\vector(0,1){2.75}}
%  \put(0.07, 2.70){\(y\)}
%  \thicklines
%  \qbezier(0.399, 0.467)(0.797, 0.998)
%          (1.118, 1.962)
%  \thinlines
%  \multiput(0.399, 0)(0, 0.1){3}
%           {\line(0,1){0.05}}
%  \multiput(1.118, 0)(0, 0.1){20}
%           {\line(0,1){0.05}}
%  \put(0.399, -0.08){\makebox(0,0){\(x_1\)}}
%  \put(1.118, -0.08){\makebox(0,0){\(x_2\)}}
%  \multiput(0.399, 0.467)(0, -0.267){2}
%           {\line(-1, 0){0.2}}
%  \multiput(0.399, 0.467)(-0.2, 0){2}
%           {\line(0, -1){0.267}}
%  \put(0.399, 0.467){\vector(-1, 0){0.2}}
%  \put(0.399, 0.467){\vector(0, -1){0.267}}
%  \put(0.399, 0.467){\vector(-3, -4){0.2}}
%  \put(0.32,0.55){\makebox(0,0){\(H\)}}
%  \put(0.48, 0.33){\makebox(0,0){\(V_1\)}}
%  \multiput(1.118, 1.962)(0, 0.6){2}
%           {\line(1,0){0.2}}
%  \multiput(1.118, 1.962)(0.2, 0){2}
%           {\line(0,1){0.6}}
%  \put(1.118, 1.962){\vector(1,0){0.2}}
%  \put(1.118, 1.962){\vector(0,1){0.6}}
%  \put(1.118, 1.962){\vector(1,3){0.2}}
%  \put(1.22, 1.87){\makebox(0,0){\(H\)}}
%  \put(1.02, 2.22){\makebox(0,0){\(V_2\)}}
%  \put(0.797, 1.195){\vector(0,-1){0.333}}
%  \put(0.777, 1.195){\line(1,0){0.04}}
%  \put(0.83,1.0){\(G\)}
%\end{picture}
%
%% feature: big dots.
%Simultaneousness
%
%\setlength{\unitlength}{1mm}
%\begin{picture}(60,50)
%  \put(0 ,15){\vector(1,0){53}}
%  \put(54,14){\(x_A\)}
%  \put( 8,10){\vector(0,1){37}}
%  \put( 0,46){\(ct_A\)}
%  \multiput(13,9)(15,0){3}{\line(0,1){35}}
%  \put(11,5){\(A''\)}
%  \put(26,5){\(A'\)}
%  \put(41,5){\(A'''\)}
%  \multiput(13,37)(30,0){2}{\circle*{2}}
%  \put(28,22){\circle*{2}}
%  \put(28,22){\vector(1,1){14}}
%  \put(15,39){\(E_1\)}
%  \put(28,22){\vector(-1,1){14}}
%  \put(45,39){\(E_2\)}
%  \multiput(0,37)(4,0){13}{\line(1,0){2}}
%\end{picture}
%
%
%\begin{picture}(0,0)%
%\includegraphics{3axisPl.pdf}%
%\end{picture}%
%\setlength{\unitlength}{3947sp}%
%%
%\begingroup\makeatletter\ifx\SetFigFont\undefined%
%\gdef\SetFigFont#1#2#3#4#5{%
%  \reset@font\fontsize{#1}{#2pt}%
%  \fontfamily{#3}\fontseries{#4}\fontshape{#5}%
%  \selectfont}%
%\fi\endgroup%
%\begin{picture}(6624,7149)(1189,-7498)
%\end{picture}%

%}
\EndArticle
%\EndNoBibArticle


%\bibliography{myrefs}
%\bibliographystyle{unsrturl}

%\section{graphics experiment}
%
%% TIP: draw the picture on graph paper and copy the numbers.
%% http://www.ursoswald.ch/LaTeXGraphics/picture/picture.html
%
%Forces on the Catenary
%
%% features: arrows, axis's, gravity vector.
%
%\setlength{\unitlength}{3cm}
%\begin{picture}(1.75, 2.75)(0, -0.1)
%  \put(0,0){\vector(1,0){1.75}}
%  \put(1.85, -0.03){\(x\)}
%  \put(0,0){\vector(0,1){2.75}}
%  \put(0.07, 2.70){\(y\)}
%  \thicklines
%  \qbezier(0.399, 0.467)(0.797, 0.998)
%          (1.118, 1.962)
%  \thinlines
%  \multiput(0.399, 0)(0, 0.1){3}
%           {\line(0,1){0.05}}
%  \multiput(1.118, 0)(0, 0.1){20}
%           {\line(0,1){0.05}}
%  \put(0.399, -0.08){\makebox(0,0){\(x_1\)}}
%  \put(1.118, -0.08){\makebox(0,0){\(x_2\)}}
%  \multiput(0.399, 0.467)(0, -0.267){2}
%           {\line(-1, 0){0.2}}
%  \multiput(0.399, 0.467)(-0.2, 0){2}
%           {\line(0, -1){0.267}}
%  \put(0.399, 0.467){\vector(-1, 0){0.2}}
%  \put(0.399, 0.467){\vector(0, -1){0.267}}
%  \put(0.399, 0.467){\vector(-3, -4){0.2}}
%  \put(0.32,0.55){\makebox(0,0){\(H\)}}
%  \put(0.48, 0.33){\makebox(0,0){\(V_1\)}}
%  \multiput(1.118, 1.962)(0, 0.6){2}
%           {\line(1,0){0.2}}
%  \multiput(1.118, 1.962)(0.2, 0){2}
%           {\line(0,1){0.6}}
%  \put(1.118, 1.962){\vector(1,0){0.2}}
%  \put(1.118, 1.962){\vector(0,1){0.6}}
%  \put(1.118, 1.962){\vector(1,3){0.2}}
%  \put(1.22, 1.87){\makebox(0,0){\(H\)}}
%  \put(1.02, 2.22){\makebox(0,0){\(V_2\)}}
%  \put(0.797, 1.195){\vector(0,-1){0.333}}
%  \put(0.777, 1.195){\line(1,0){0.04}}
%  \put(0.83,1.0){\(G\)}
%\end{picture}
%
%% feature: big dots.
%Simultaneousness
%
%\setlength{\unitlength}{1mm}
%\begin{picture}(60,50)
%  \put(0 ,15){\vector(1,0){53}}
%  \put(54,14){\(x_A\)}
%  \put( 8,10){\vector(0,1){37}}
%  \put( 0,46){\(ct_A\)}
%  \multiput(13,9)(15,0){3}{\line(0,1){35}}
%  \put(11,5){\(A''\)}
%  \put(26,5){\(A'\)}
%  \put(41,5){\(A'''\)}
%  \multiput(13,37)(30,0){2}{\circle*{2}}
%  \put(28,22){\circle*{2}}
%  \put(28,22){\vector(1,1){14}}
%  \put(15,39){\(E_1\)}
%  \put(28,22){\vector(-1,1){14}}
%  \put(45,39){\(E_2\)}
%  \multiput(0,37)(4,0){13}{\line(1,0){2}}
%\end{picture}
%
%
%\begin{picture}(0,0)%
%\includegraphics{3axisPl.pdf}%
%\end{picture}%
%\setlength{\unitlength}{3947sp}%
%%
%\begingroup\makeatletter\ifx\SetFigFont\undefined%
%\gdef\SetFigFont#1#2#3#4#5{%
%  \reset@font\fontsize{#1}{#2pt}%
%  \fontfamily{#3}\fontseries{#4}\fontshape{#5}%
%  \selectfont}%
%\fi\endgroup%
%\begin{picture}(6624,7149)(1189,-7498)
%\end{picture}%

%}
\EndArticle
%\EndNoBibArticle


%\bibliography{myrefs}
%\bibliographystyle{unsrturl}

%\section{graphics experiment}
%
%% TIP: draw the picture on graph paper and copy the numbers.
%% http://www.ursoswald.ch/LaTeXGraphics/picture/picture.html
%
%Forces on the Catenary
%
%% features: arrows, axis's, gravity vector.
%
%\setlength{\unitlength}{3cm}
%\begin{picture}(1.75, 2.75)(0, -0.1)
%  \put(0,0){\vector(1,0){1.75}}
%  \put(1.85, -0.03){\(x\)}
%  \put(0,0){\vector(0,1){2.75}}
%  \put(0.07, 2.70){\(y\)}
%  \thicklines
%  \qbezier(0.399, 0.467)(0.797, 0.998)
%          (1.118, 1.962)
%  \thinlines
%  \multiput(0.399, 0)(0, 0.1){3}
%           {\line(0,1){0.05}}
%  \multiput(1.118, 0)(0, 0.1){20}
%           {\line(0,1){0.05}}
%  \put(0.399, -0.08){\makebox(0,0){\(x_1\)}}
%  \put(1.118, -0.08){\makebox(0,0){\(x_2\)}}
%  \multiput(0.399, 0.467)(0, -0.267){2}
%           {\line(-1, 0){0.2}}
%  \multiput(0.399, 0.467)(-0.2, 0){2}
%           {\line(0, -1){0.267}}
%  \put(0.399, 0.467){\vector(-1, 0){0.2}}
%  \put(0.399, 0.467){\vector(0, -1){0.267}}
%  \put(0.399, 0.467){\vector(-3, -4){0.2}}
%  \put(0.32,0.55){\makebox(0,0){\(H\)}}
%  \put(0.48, 0.33){\makebox(0,0){\(V_1\)}}
%  \multiput(1.118, 1.962)(0, 0.6){2}
%           {\line(1,0){0.2}}
%  \multiput(1.118, 1.962)(0.2, 0){2}
%           {\line(0,1){0.6}}
%  \put(1.118, 1.962){\vector(1,0){0.2}}
%  \put(1.118, 1.962){\vector(0,1){0.6}}
%  \put(1.118, 1.962){\vector(1,3){0.2}}
%  \put(1.22, 1.87){\makebox(0,0){\(H\)}}
%  \put(1.02, 2.22){\makebox(0,0){\(V_2\)}}
%  \put(0.797, 1.195){\vector(0,-1){0.333}}
%  \put(0.777, 1.195){\line(1,0){0.04}}
%  \put(0.83,1.0){\(G\)}
%\end{picture}
%
%% feature: big dots.
%Simultaneousness
%
%\setlength{\unitlength}{1mm}
%\begin{picture}(60,50)
%  \put(0 ,15){\vector(1,0){53}}
%  \put(54,14){\(x_A\)}
%  \put( 8,10){\vector(0,1){37}}
%  \put( 0,46){\(ct_A\)}
%  \multiput(13,9)(15,0){3}{\line(0,1){35}}
%  \put(11,5){\(A''\)}
%  \put(26,5){\(A'\)}
%  \put(41,5){\(A'''\)}
%  \multiput(13,37)(30,0){2}{\circle*{2}}
%  \put(28,22){\circle*{2}}
%  \put(28,22){\vector(1,1){14}}
%  \put(15,39){\(E_1\)}
%  \put(28,22){\vector(-1,1){14}}
%  \put(45,39){\(E_2\)}
%  \multiput(0,37)(4,0){13}{\line(1,0){2}}
%\end{picture}
%
%
%\begin{picture}(0,0)%
%\includegraphics{3axisPl.pdf}%
%\end{picture}%
%\setlength{\unitlength}{3947sp}%
%%
%\begingroup\makeatletter\ifx\SetFigFont\undefined%
%\gdef\SetFigFont#1#2#3#4#5{%
%  \reset@font\fontsize{#1}{#2pt}%
%  \fontfamily{#3}\fontseries{#4}\fontshape{#5}%
%  \selectfont}%
%\fi\endgroup%
%\begin{picture}(6624,7149)(1189,-7498)
%\end{picture}%

%}
\EndArticle
%\EndNoBibArticle

%
% Copyright � 2012 Peeter Joot.  All Rights Reserved.
% Licenced as described in the file LICENSE under the root directory of this GIT repository.
%

\chapter{1D forced harmonic oscillator.  Quick solution of non-homogeneous problem}
\index{forced harmonic oscillator}
\label{chap:1dharmonicOsc}
%\blogpage{http://sites.google.com/site/peeterjoot/math2010/1dharmonicOsc.pdf}
%\date{Feb 19, 2010}

\section{Motivation}

In \citep{brown1954feynman} equation (25) we have a forced harmonic oscillator equation

\begin{equation}\label{eqn:1dharmonicOsc:1}
\begin{aligned}
m \ddot{x} + m \omega^2 x = \gamma(t).
\end{aligned}
\end{equation}

The solution of this equation is provided, but for fun lets derive it.

\section{Guts}

Writing

\begin{equation}\label{eqn:1dharmonicOsc:2}
\begin{aligned}
\omega u = \dot{x},
\end{aligned}
\end{equation}

we can rewrite the second order equation as a first order linear system

\begin{equation}\label{eqn:1dharmonicOsc:3}
\begin{aligned}
\dot{u} + \omega x &= \gamma(t)/m \omega \\
\dot{x} - \omega u &= 0,
\end{aligned}
\end{equation}

Or, with \(X = (u, x)\), in matrix form

\begin{equation}\label{eqn:1dharmonicOsc:4}
\begin{aligned}
\dot{X} + \omega
\begin{bmatrix}
0 & 1 \\
-1 & 0
\end{bmatrix}
X
&=
\begin{bmatrix}
\gamma(t)/m \omega \\
0
\end{bmatrix}.
\end{aligned}
\end{equation}

The two by two matrix has the same properties as the complex imaginary, squaring to the identity matrix, so the equation to solve is now of the form

\begin{equation}\label{eqn:1dharmonicOsc:5}
\begin{aligned}
\dot{X} + \omega i X &= \Gamma.
\end{aligned}
\end{equation}

The homogeneous part of the solution is just the matrix

\begin{equation}\label{eqn:1dharmonicOsc:28}
\begin{aligned}
X
&= e^{-i \omega t} A \\
&=
\left(
\cos(\omega t)
\begin{bmatrix}
1 & 0 \\
0 & 1
\end{bmatrix}
-
\sin(\omega t)
\begin{bmatrix}
0 & 1 \\
-1 & 0
\end{bmatrix}
\right) A,
\end{aligned}
\end{equation}

where \(A\) is a two by one column matrix of constants.  Assuming for the specific solution \(X = e^{-i \omega t} A(t)\), and substituting we have

\begin{equation}\label{eqn:1dharmonicOsc:6}
\begin{aligned}
e^{-i \omega t} \dot{A} = \Gamma(t).
\end{aligned}
\end{equation}

This integrates directly, fixing the unknown column vector function \(A(t)\)

\begin{equation}\label{eqn:1dharmonicOsc:7}
\begin{aligned}
A(t) = A(0) + \int_0^t e^{i \omega \tau} \Gamma(\tau).
\end{aligned}
\end{equation}

Thus the non-homogeneous solution takes the form

\begin{equation}\label{eqn:1dharmonicOsc:8}
\begin{aligned}
X = e^{-i \omega t} A(0) + \int_0^t e^{i \omega (\tau - t)} \Gamma(\tau).
\end{aligned}
\end{equation}

Note that \(A(0) = (\dot{x}_0/\omega, x_0)\).  Multiplying this out, and discarding all but the second row of the matrix product gives \(x(t)\), and Feynman's equation (26) follows directly.

\documentclass[]{eliblog}

\usepackage{amsmath}
\usepackage{mathpazo}

%
% shorthand for bold symbols, convenient for vectors and matrices
%
\newcommand{\Ba}[0]{\mathbf{a}}
\newcommand{\Bb}[0]{\mathbf{b}}
\newcommand{\Bc}[0]{\mathbf{c}}
\newcommand{\Bd}[0]{\mathbf{d}}
\newcommand{\Be}[0]{\mathbf{e}}
\newcommand{\Bf}[0]{\mathbf{f}}
\newcommand{\Bg}[0]{\mathbf{g}}
\newcommand{\Bh}[0]{\mathbf{h}}
\newcommand{\Bi}[0]{\mathbf{i}}
\newcommand{\Bj}[0]{\mathbf{j}}
\newcommand{\Bk}[0]{\mathbf{k}}
\newcommand{\Bl}[0]{\mathbf{l}}
\newcommand{\Bm}[0]{\mathbf{m}}
\newcommand{\Bn}[0]{\mathbf{n}}
\newcommand{\Bo}[0]{\mathbf{o}}
\newcommand{\Bp}[0]{\mathbf{p}}
\newcommand{\Bq}[0]{\mathbf{q}}
\newcommand{\Br}[0]{\mathbf{r}}
\newcommand{\Bs}[0]{\mathbf{s}}
\newcommand{\Bt}[0]{\mathbf{t}}
\newcommand{\Bu}[0]{\mathbf{u}}
\newcommand{\Bv}[0]{\mathbf{v}}
\newcommand{\Bw}[0]{\mathbf{w}}
\newcommand{\Bx}[0]{\mathbf{x}}
\newcommand{\By}[0]{\mathbf{y}}
\newcommand{\Bz}[0]{\mathbf{z}}
\newcommand{\BA}[0]{\mathbf{A}}
\newcommand{\BB}[0]{\mathbf{B}}
\newcommand{\BC}[0]{\mathbf{C}}
\newcommand{\BD}[0]{\mathbf{D}}
\newcommand{\BE}[0]{\mathbf{E}}
\newcommand{\BF}[0]{\mathbf{F}}
\newcommand{\BG}[0]{\mathbf{G}}
\newcommand{\BH}[0]{\mathbf{H}}
\newcommand{\BI}[0]{\mathbf{I}}
\newcommand{\BJ}[0]{\mathbf{J}}
\newcommand{\BK}[0]{\mathbf{K}}
\newcommand{\BL}[0]{\mathbf{L}}
\newcommand{\BM}[0]{\mathbf{M}}
\newcommand{\BN}[0]{\mathbf{N}}
\newcommand{\BO}[0]{\mathbf{O}}
\newcommand{\BP}[0]{\mathbf{P}}
\newcommand{\BQ}[0]{\mathbf{Q}}
\newcommand{\BR}[0]{\mathbf{R}}
\newcommand{\BS}[0]{\mathbf{S}}
\newcommand{\BT}[0]{\mathbf{T}}
\newcommand{\BU}[0]{\mathbf{U}}
\newcommand{\BV}[0]{\mathbf{V}}
\newcommand{\BW}[0]{\mathbf{W}}
\newcommand{\BX}[0]{\mathbf{X}}
\newcommand{\BY}[0]{\mathbf{Y}}
\newcommand{\BZ}[0]{\mathbf{Z}}

\newcommand{\Bzero}[0]{\mathbf{0}}
\newcommand{\Btheta}[0]{\boldsymbol{\theta}}
\newcommand{\Btau}[0]{\boldsymbol{\tau}}
\newcommand{\Bomega}[0]{\boldsymbol{\omega}}

%
% shorthand for unit vectors
%
\newcommand{\acap}[0]{\hat{\Ba}}
\newcommand{\bcap}[0]{\hat{\Bb}}
\newcommand{\ccap}[0]{\hat{\Bc}}
\newcommand{\dcap}[0]{\hat{\Bd}}
\newcommand{\ecap}[0]{\hat{\Be}}
\newcommand{\fcap}[0]{\hat{\Bf}}
\newcommand{\gcap}[0]{\hat{\Bg}}
\newcommand{\hcap}[0]{\hat{\Bh}}
\newcommand{\icap}[0]{\hat{\Bi}}
\newcommand{\jcap}[0]{\hat{\Bj}}
\newcommand{\kcap}[0]{\hat{\Bk}}
\newcommand{\lcap}[0]{\hat{\Bl}}
\newcommand{\mcap}[0]{\hat{\Bm}}
\newcommand{\ncap}[0]{\hat{\Bn}}
\newcommand{\ocap}[0]{\hat{\Bo}}
\newcommand{\pcap}[0]{\hat{\Bp}}
\newcommand{\qcap}[0]{\hat{\Bq}}
\newcommand{\rcap}[0]{\hat{\Br}}
\newcommand{\scap}[0]{\hat{\Bs}}
\newcommand{\tcap}[0]{\hat{\Bt}}
\newcommand{\ucap}[0]{\hat{\Bu}}
\newcommand{\vcap}[0]{\hat{\Bv}}
\newcommand{\wcap}[0]{\hat{\Bw}}
\newcommand{\xcap}[0]{\hat{\Bx}}
\newcommand{\ycap}[0]{\hat{\By}}
\newcommand{\zcap}[0]{\hat{\Bz}}
\newcommand{\thetacap}[0]{\hat{\Btheta}}

%
% to write R^n and C^n in a distinguishable fashion.  Perhaps change this
% to the double lined characters upon figuring out how to do so.
%
\newcommand{\C}[1]{$\mathbb{C}^{#1}$}
\newcommand{\R}[1]{$\mathbb{R}^{#1}$}

%
% various generally useful helpers
%

% derivative of #1 wrt. #2:
\newcommand{\D}[2] {\frac {d#2} {d#1}}

\newcommand{\inv}[1]{\frac{1}{#1}}
\newcommand{\cross}[0]{\times}

\newcommand{\abs}[1]{\lvert{#1}\rvert}
\newcommand{\norm}[1]{\lVert{#1}\rVert}
\newcommand{\innerprod}[2]{\langle{#1}, {#2}\rangle}
\newcommand{\dotprod}[2]{{#1} \cdot {#2}}
\newcommand{\bdotprod}[2]{\left({#1} \cdot {#2}\right)}
\newcommand{\crossprod}[2]{{#1} \cross {#2}}
\newcommand{\tripleprod}[3]{\dotprod{\left(\crossprod{#1}{#2}\right)}{#3}}

\DeclareMathOperator{\Proj}{Proj}
\DeclareMathOperator{\Span}{span}
\DeclareMathOperator{\Sgn}{sgn}
\DeclareMathOperator{\Area}{Area}
\DeclareMathOperator{\Volume}{Volume}

%
% A few miscellaneous things specific to this document
%
\newcommand{\crossop}[1]{\crossprod{#1}{}}

% R2 vector.
\newcommand{\VectorTwo}[2]{
\begin{bmatrix}
 {#1} \\
 {#2}
\end{bmatrix}
}

\newcommand{\VectorN}[1]{
\begin{bmatrix}
{#1}_1 \\
{#1}_2 \\
\vdots \\
{#1}_N \\
\end{bmatrix}
}

\newcommand{\DETuvij}[4]{
\begin{vmatrix}
 {#1}_{#3} & {#1}_{#4} \\
 {#2}_{#3} & {#2}_{#4}
\end{vmatrix}
}

\newcommand{\DETuvwijk}[6]{
\begin{vmatrix}
 {#1}_{#4} & {#1}_{#5} & {#1}_{#6} \\
 {#2}_{#4} & {#2}_{#5} & {#2}_{#6} \\
 {#3}_{#4} & {#3}_{#5} & {#3}_{#6}
\end{vmatrix}
}

\newcommand{\DETuvwxijkl}[8]{
\begin{vmatrix}
 {#1}_{#5} & {#1}_{#6} & {#1}_{#7} & {#1}_{#8} \\
 {#2}_{#5} & {#2}_{#6} & {#2}_{#7} & {#2}_{#8} \\
 {#3}_{#5} & {#3}_{#6} & {#3}_{#7} & {#3}_{#8} \\
 {#4}_{#5} & {#4}_{#6} & {#4}_{#7} & {#4}_{#8} \\
\end{vmatrix}
}

%\newcommand{\DETuvwxyijklm}[10]{
%\begin{vmatrix}
% {#1}_{#6} & {#1}_{#7} & {#1}_{#8} & {#1}_{#9} & {#1}_{#10} \\
% {#2}_{#6} & {#2}_{#7} & {#2}_{#8} & {#2}_{#9} & {#2}_{#10} \\
% {#3}_{#6} & {#3}_{#7} & {#3}_{#8} & {#3}_{#9} & {#3}_{#10} \\
% {#4}_{#6} & {#4}_{#7} & {#4}_{#8} & {#4}_{#9} & {#4}_{#10} \\
% {#5}_{#6} & {#5}_{#7} & {#5}_{#8} & {#5}_{#9} & {#5}_{#10}
%\end{vmatrix}
%}

% R3 vector.
\newcommand{\VectorThree}[3]{
\begin{bmatrix}
 {#1} \\
 {#2} \\
 {#3}
\end{bmatrix}
}



\author{Peeter Joot}
\email{peeter.joot@gmail.com}


\chapter{Integrating the equation of motion for a one dimensional problem.}
\label{chap:1dpotentialIntegral}
%\useCCL
\blogpage{http://sites.google.com/site/peeterjoot/math2010/1dpotentialIntegral.pdf}
\date{Jan 1, 2010}
\revisionInfo{1dpotentialIntegral.tex}

%\beginArtWithToc
\beginArtNoToc

\section{Motivation.}

While linear approximations, such as the small angle approximation for the pendum, are often used to understand the dynamics of non-linear systems, exact solutions may be possible in some cases.  Walk through the setup for such an exact solution.

\section{Guts}

The equation to consider solutions of has the form

\begin{align}
\label{eqn:1dpotentialIntegral:1}
\frac{d}{dt} \left( m \frac{dx}{dt} \right) = -\PD{x}{U(x)}.
\end{align}

We have an unpleasant mix of space and time derivatives, preventing any sort of antidifferentiation.  Assuming constant mass $m$, and employing the chain rule a refactoring in terms of velocities is possible.

\begin{align*}
\frac{d}{dt} \left( \frac{dx}{dt} \right) 
&= 
\frac{dx}{dt} \frac{d}{dx} \left( \frac{dx}{dt} \right)  \\
&= 
\inv{2} \frac{d}{dx} \left( \frac{dx}{dt} \right)^2  \\
\end{align*}

The one dimensional Newton's law \autoref{eqn:1dpotentialIntegral:1} now takes the form
\begin{align}
\label{eqn:1dpotentialIntegral:2}
\frac{d}{dx} \left( \frac{dx}{dt} \right)^2 &= -\frac{2}{m} \PD{x}{U(x)}.
\end{align}

This can now be antidifferentiated for

\begin{align}\label{eqn:1dpotentialIntegral:3}
\left( \frac{dx}{dt} \right)^2 &= \frac{2}{m} (E - U(x)).
\end{align}

NOTE: this is a dumb way to get here.  This is just the Hamiltonian, rearranged, so one could start with the physics, instead of the calculus to get to this point.

Taking roots and rearranging produces an implicit differential form $x$ in terms of time

\begin{align}\label{eqn:1dpotentialIntegral:4}
dt = \frac{dx}{\sqrt{ \frac{2}{m} (E - U(x)) } }.
\end{align}

One can concievably integrate this and invert to solve for position as a function of time, but substitution of a more specific potential is required to go further.

\begin{align}\label{eqn:1dpotentialIntegral:5}
t(x) = t(x_0) + \int_{y=x_0}^{x} \frac{dy}{\sqrt{ \frac{2}{m} (E - U(y)) } }.
\end{align}


%\EndArticle
\EndNoBibArticle

%%
% Copyright � 2015 Peeter Joot.  All Rights Reserved.
% Licenced as described in the file LICENSE under the root directory of this GIT repository.
%
\documentclass[]{eliblog}

\usepackage{amsmath}
\usepackage{mathpazo}

%
% shorthand for bold symbols, convenient for vectors and matrices
%
\newcommand{\Ba}[0]{\mathbf{a}}
\newcommand{\Bb}[0]{\mathbf{b}}
\newcommand{\Bc}[0]{\mathbf{c}}
\newcommand{\Bd}[0]{\mathbf{d}}
\newcommand{\Be}[0]{\mathbf{e}}
\newcommand{\Bf}[0]{\mathbf{f}}
\newcommand{\Bg}[0]{\mathbf{g}}
\newcommand{\Bh}[0]{\mathbf{h}}
\newcommand{\Bi}[0]{\mathbf{i}}
\newcommand{\Bj}[0]{\mathbf{j}}
\newcommand{\Bk}[0]{\mathbf{k}}
\newcommand{\Bl}[0]{\mathbf{l}}
\newcommand{\Bm}[0]{\mathbf{m}}
\newcommand{\Bn}[0]{\mathbf{n}}
\newcommand{\Bo}[0]{\mathbf{o}}
\newcommand{\Bp}[0]{\mathbf{p}}
\newcommand{\Bq}[0]{\mathbf{q}}
\newcommand{\Br}[0]{\mathbf{r}}
\newcommand{\Bs}[0]{\mathbf{s}}
\newcommand{\Bt}[0]{\mathbf{t}}
\newcommand{\Bu}[0]{\mathbf{u}}
\newcommand{\Bv}[0]{\mathbf{v}}
\newcommand{\Bw}[0]{\mathbf{w}}
\newcommand{\Bx}[0]{\mathbf{x}}
\newcommand{\By}[0]{\mathbf{y}}
\newcommand{\Bz}[0]{\mathbf{z}}
\newcommand{\BA}[0]{\mathbf{A}}
\newcommand{\BB}[0]{\mathbf{B}}
\newcommand{\BC}[0]{\mathbf{C}}
\newcommand{\BD}[0]{\mathbf{D}}
\newcommand{\BE}[0]{\mathbf{E}}
\newcommand{\BF}[0]{\mathbf{F}}
\newcommand{\BG}[0]{\mathbf{G}}
\newcommand{\BH}[0]{\mathbf{H}}
\newcommand{\BI}[0]{\mathbf{I}}
\newcommand{\BJ}[0]{\mathbf{J}}
\newcommand{\BK}[0]{\mathbf{K}}
\newcommand{\BL}[0]{\mathbf{L}}
\newcommand{\BM}[0]{\mathbf{M}}
\newcommand{\BN}[0]{\mathbf{N}}
\newcommand{\BO}[0]{\mathbf{O}}
\newcommand{\BP}[0]{\mathbf{P}}
\newcommand{\BQ}[0]{\mathbf{Q}}
\newcommand{\BR}[0]{\mathbf{R}}
\newcommand{\BS}[0]{\mathbf{S}}
\newcommand{\BT}[0]{\mathbf{T}}
\newcommand{\BU}[0]{\mathbf{U}}
\newcommand{\BV}[0]{\mathbf{V}}
\newcommand{\BW}[0]{\mathbf{W}}
\newcommand{\BX}[0]{\mathbf{X}}
\newcommand{\BY}[0]{\mathbf{Y}}
\newcommand{\BZ}[0]{\mathbf{Z}}

\newcommand{\Bzero}[0]{\mathbf{0}}
\newcommand{\Btheta}[0]{\boldsymbol{\theta}}
\newcommand{\Btau}[0]{\boldsymbol{\tau}}
\newcommand{\Bomega}[0]{\boldsymbol{\omega}}

%
% shorthand for unit vectors
%
\newcommand{\acap}[0]{\hat{\Ba}}
\newcommand{\bcap}[0]{\hat{\Bb}}
\newcommand{\ccap}[0]{\hat{\Bc}}
\newcommand{\dcap}[0]{\hat{\Bd}}
\newcommand{\ecap}[0]{\hat{\Be}}
\newcommand{\fcap}[0]{\hat{\Bf}}
\newcommand{\gcap}[0]{\hat{\Bg}}
\newcommand{\hcap}[0]{\hat{\Bh}}
\newcommand{\icap}[0]{\hat{\Bi}}
\newcommand{\jcap}[0]{\hat{\Bj}}
\newcommand{\kcap}[0]{\hat{\Bk}}
\newcommand{\lcap}[0]{\hat{\Bl}}
\newcommand{\mcap}[0]{\hat{\Bm}}
\newcommand{\ncap}[0]{\hat{\Bn}}
\newcommand{\ocap}[0]{\hat{\Bo}}
\newcommand{\pcap}[0]{\hat{\Bp}}
\newcommand{\qcap}[0]{\hat{\Bq}}
\newcommand{\rcap}[0]{\hat{\Br}}
\newcommand{\scap}[0]{\hat{\Bs}}
\newcommand{\tcap}[0]{\hat{\Bt}}
\newcommand{\ucap}[0]{\hat{\Bu}}
\newcommand{\vcap}[0]{\hat{\Bv}}
\newcommand{\wcap}[0]{\hat{\Bw}}
\newcommand{\xcap}[0]{\hat{\Bx}}
\newcommand{\ycap}[0]{\hat{\By}}
\newcommand{\zcap}[0]{\hat{\Bz}}
\newcommand{\thetacap}[0]{\hat{\Btheta}}

%
% to write R^n and C^n in a distinguishable fashion.  Perhaps change this
% to the double lined characters upon figuring out how to do so.
%
\newcommand{\C}[1]{$\mathbb{C}^{#1}$}
\newcommand{\R}[1]{$\mathbb{R}^{#1}$}

%
% various generally useful helpers
%

% derivative of #1 wrt. #2:
\newcommand{\D}[2] {\frac {d#2} {d#1}}

\newcommand{\inv}[1]{\frac{1}{#1}}
\newcommand{\cross}[0]{\times}

\newcommand{\abs}[1]{\lvert{#1}\rvert}
\newcommand{\norm}[1]{\lVert{#1}\rVert}
\newcommand{\innerprod}[2]{\langle{#1}, {#2}\rangle}
\newcommand{\dotprod}[2]{{#1} \cdot {#2}}
\newcommand{\bdotprod}[2]{\left({#1} \cdot {#2}\right)}
\newcommand{\crossprod}[2]{{#1} \cross {#2}}
\newcommand{\tripleprod}[3]{\dotprod{\left(\crossprod{#1}{#2}\right)}{#3}}

\DeclareMathOperator{\Proj}{Proj}
\DeclareMathOperator{\Span}{span}
\DeclareMathOperator{\Sgn}{sgn}
\DeclareMathOperator{\Area}{Area}
\DeclareMathOperator{\Volume}{Volume}

%
% A few miscellaneous things specific to this document
%
\newcommand{\crossop}[1]{\crossprod{#1}{}}

% R2 vector.
\newcommand{\VectorTwo}[2]{
\begin{bmatrix}
 {#1} \\
 {#2}
\end{bmatrix}
}

\newcommand{\VectorN}[1]{
\begin{bmatrix}
{#1}_1 \\
{#1}_2 \\
\vdots \\
{#1}_N \\
\end{bmatrix}
}

\newcommand{\DETuvij}[4]{
\begin{vmatrix}
 {#1}_{#3} & {#1}_{#4} \\
 {#2}_{#3} & {#2}_{#4}
\end{vmatrix}
}

\newcommand{\DETuvwijk}[6]{
\begin{vmatrix}
 {#1}_{#4} & {#1}_{#5} & {#1}_{#6} \\
 {#2}_{#4} & {#2}_{#5} & {#2}_{#6} \\
 {#3}_{#4} & {#3}_{#5} & {#3}_{#6}
\end{vmatrix}
}

\newcommand{\DETuvwxijkl}[8]{
\begin{vmatrix}
 {#1}_{#5} & {#1}_{#6} & {#1}_{#7} & {#1}_{#8} \\
 {#2}_{#5} & {#2}_{#6} & {#2}_{#7} & {#2}_{#8} \\
 {#3}_{#5} & {#3}_{#6} & {#3}_{#7} & {#3}_{#8} \\
 {#4}_{#5} & {#4}_{#6} & {#4}_{#7} & {#4}_{#8} \\
\end{vmatrix}
}

%\newcommand{\DETuvwxyijklm}[10]{
%\begin{vmatrix}
% {#1}_{#6} & {#1}_{#7} & {#1}_{#8} & {#1}_{#9} & {#1}_{#10} \\
% {#2}_{#6} & {#2}_{#7} & {#2}_{#8} & {#2}_{#9} & {#2}_{#10} \\
% {#3}_{#6} & {#3}_{#7} & {#3}_{#8} & {#3}_{#9} & {#3}_{#10} \\
% {#4}_{#6} & {#4}_{#7} & {#4}_{#8} & {#4}_{#9} & {#4}_{#10} \\
% {#5}_{#6} & {#5}_{#7} & {#5}_{#8} & {#5}_{#9} & {#5}_{#10}
%\end{vmatrix}
%}

% R3 vector.
\newcommand{\VectorThree}[3]{
\begin{bmatrix}
 {#1} \\
 {#2} \\
 {#3}
\end{bmatrix}
}



\author{Peeter Joot}
\email{peeter.joot@gmail.com}

\documentclass[]{eliblogwidescreen}

\usepackage{amsmath}
\usepackage{mathpazo}

%
% shorthand for bold symbols, convenient for vectors and matrices
%
\newcommand{\Ba}[0]{\mathbf{a}}
\newcommand{\Bb}[0]{\mathbf{b}}
\newcommand{\Bc}[0]{\mathbf{c}}
\newcommand{\Bd}[0]{\mathbf{d}}
\newcommand{\Be}[0]{\mathbf{e}}
\newcommand{\Bf}[0]{\mathbf{f}}
\newcommand{\Bg}[0]{\mathbf{g}}
\newcommand{\Bh}[0]{\mathbf{h}}
\newcommand{\Bi}[0]{\mathbf{i}}
\newcommand{\Bj}[0]{\mathbf{j}}
\newcommand{\Bk}[0]{\mathbf{k}}
\newcommand{\Bl}[0]{\mathbf{l}}
\newcommand{\Bm}[0]{\mathbf{m}}
\newcommand{\Bn}[0]{\mathbf{n}}
\newcommand{\Bo}[0]{\mathbf{o}}
\newcommand{\Bp}[0]{\mathbf{p}}
\newcommand{\Bq}[0]{\mathbf{q}}
\newcommand{\Br}[0]{\mathbf{r}}
\newcommand{\Bs}[0]{\mathbf{s}}
\newcommand{\Bt}[0]{\mathbf{t}}
\newcommand{\Bu}[0]{\mathbf{u}}
\newcommand{\Bv}[0]{\mathbf{v}}
\newcommand{\Bw}[0]{\mathbf{w}}
\newcommand{\Bx}[0]{\mathbf{x}}
\newcommand{\By}[0]{\mathbf{y}}
\newcommand{\Bz}[0]{\mathbf{z}}
\newcommand{\BA}[0]{\mathbf{A}}
\newcommand{\BB}[0]{\mathbf{B}}
\newcommand{\BC}[0]{\mathbf{C}}
\newcommand{\BD}[0]{\mathbf{D}}
\newcommand{\BE}[0]{\mathbf{E}}
\newcommand{\BF}[0]{\mathbf{F}}
\newcommand{\BG}[0]{\mathbf{G}}
\newcommand{\BH}[0]{\mathbf{H}}
\newcommand{\BI}[0]{\mathbf{I}}
\newcommand{\BJ}[0]{\mathbf{J}}
\newcommand{\BK}[0]{\mathbf{K}}
\newcommand{\BL}[0]{\mathbf{L}}
\newcommand{\BM}[0]{\mathbf{M}}
\newcommand{\BN}[0]{\mathbf{N}}
\newcommand{\BO}[0]{\mathbf{O}}
\newcommand{\BP}[0]{\mathbf{P}}
\newcommand{\BQ}[0]{\mathbf{Q}}
\newcommand{\BR}[0]{\mathbf{R}}
\newcommand{\BS}[0]{\mathbf{S}}
\newcommand{\BT}[0]{\mathbf{T}}
\newcommand{\BU}[0]{\mathbf{U}}
\newcommand{\BV}[0]{\mathbf{V}}
\newcommand{\BW}[0]{\mathbf{W}}
\newcommand{\BX}[0]{\mathbf{X}}
\newcommand{\BY}[0]{\mathbf{Y}}
\newcommand{\BZ}[0]{\mathbf{Z}}

\newcommand{\Bzero}[0]{\mathbf{0}}
\newcommand{\Btheta}[0]{\boldsymbol{\theta}}
\newcommand{\Btau}[0]{\boldsymbol{\tau}}
\newcommand{\Bomega}[0]{\boldsymbol{\omega}}

%
% shorthand for unit vectors
%
\newcommand{\acap}[0]{\hat{\Ba}}
\newcommand{\bcap}[0]{\hat{\Bb}}
\newcommand{\ccap}[0]{\hat{\Bc}}
\newcommand{\dcap}[0]{\hat{\Bd}}
\newcommand{\ecap}[0]{\hat{\Be}}
\newcommand{\fcap}[0]{\hat{\Bf}}
\newcommand{\gcap}[0]{\hat{\Bg}}
\newcommand{\hcap}[0]{\hat{\Bh}}
\newcommand{\icap}[0]{\hat{\Bi}}
\newcommand{\jcap}[0]{\hat{\Bj}}
\newcommand{\kcap}[0]{\hat{\Bk}}
\newcommand{\lcap}[0]{\hat{\Bl}}
\newcommand{\mcap}[0]{\hat{\Bm}}
\newcommand{\ncap}[0]{\hat{\Bn}}
\newcommand{\ocap}[0]{\hat{\Bo}}
\newcommand{\pcap}[0]{\hat{\Bp}}
\newcommand{\qcap}[0]{\hat{\Bq}}
\newcommand{\rcap}[0]{\hat{\Br}}
\newcommand{\scap}[0]{\hat{\Bs}}
\newcommand{\tcap}[0]{\hat{\Bt}}
\newcommand{\ucap}[0]{\hat{\Bu}}
\newcommand{\vcap}[0]{\hat{\Bv}}
\newcommand{\wcap}[0]{\hat{\Bw}}
\newcommand{\xcap}[0]{\hat{\Bx}}
\newcommand{\ycap}[0]{\hat{\By}}
\newcommand{\zcap}[0]{\hat{\Bz}}
\newcommand{\thetacap}[0]{\hat{\Btheta}}

%
% to write R^n and C^n in a distinguishable fashion.  Perhaps change this
% to the double lined characters upon figuring out how to do so.
%
\newcommand{\C}[1]{$\mathbb{C}^{#1}$}
\newcommand{\R}[1]{$\mathbb{R}^{#1}$}

%
% various generally useful helpers
%

% derivative of #1 wrt. #2:
\newcommand{\D}[2] {\frac {d#2} {d#1}}

\newcommand{\inv}[1]{\frac{1}{#1}}
\newcommand{\cross}[0]{\times}

\newcommand{\abs}[1]{\lvert{#1}\rvert}
\newcommand{\norm}[1]{\lVert{#1}\rVert}
\newcommand{\innerprod}[2]{\langle{#1}, {#2}\rangle}
\newcommand{\dotprod}[2]{{#1} \cdot {#2}}
\newcommand{\bdotprod}[2]{\left({#1} \cdot {#2}\right)}
\newcommand{\crossprod}[2]{{#1} \cross {#2}}
\newcommand{\tripleprod}[3]{\dotprod{\left(\crossprod{#1}{#2}\right)}{#3}}

\DeclareMathOperator{\Proj}{Proj}
\DeclareMathOperator{\Span}{span}
\DeclareMathOperator{\Sgn}{sgn}
\DeclareMathOperator{\Area}{Area}
\DeclareMathOperator{\Volume}{Volume}

%
% A few miscellaneous things specific to this document
%
\newcommand{\crossop}[1]{\crossprod{#1}{}}

% R2 vector.
\newcommand{\VectorTwo}[2]{
\begin{bmatrix}
 {#1} \\
 {#2}
\end{bmatrix}
}

\newcommand{\VectorN}[1]{
\begin{bmatrix}
{#1}_1 \\
{#1}_2 \\
\vdots \\
{#1}_N \\
\end{bmatrix}
}

\newcommand{\DETuvij}[4]{
\begin{vmatrix}
 {#1}_{#3} & {#1}_{#4} \\
 {#2}_{#3} & {#2}_{#4}
\end{vmatrix}
}

\newcommand{\DETuvwijk}[6]{
\begin{vmatrix}
 {#1}_{#4} & {#1}_{#5} & {#1}_{#6} \\
 {#2}_{#4} & {#2}_{#5} & {#2}_{#6} \\
 {#3}_{#4} & {#3}_{#5} & {#3}_{#6}
\end{vmatrix}
}

\newcommand{\DETuvwxijkl}[8]{
\begin{vmatrix}
 {#1}_{#5} & {#1}_{#6} & {#1}_{#7} & {#1}_{#8} \\
 {#2}_{#5} & {#2}_{#6} & {#2}_{#7} & {#2}_{#8} \\
 {#3}_{#5} & {#3}_{#6} & {#3}_{#7} & {#3}_{#8} \\
 {#4}_{#5} & {#4}_{#6} & {#4}_{#7} & {#4}_{#8} \\
\end{vmatrix}
}

%\newcommand{\DETuvwxyijklm}[10]{
%\begin{vmatrix}
% {#1}_{#6} & {#1}_{#7} & {#1}_{#8} & {#1}_{#9} & {#1}_{#10} \\
% {#2}_{#6} & {#2}_{#7} & {#2}_{#8} & {#2}_{#9} & {#2}_{#10} \\
% {#3}_{#6} & {#3}_{#7} & {#3}_{#8} & {#3}_{#9} & {#3}_{#10} \\
% {#4}_{#6} & {#4}_{#7} & {#4}_{#8} & {#4}_{#9} & {#4}_{#10} \\
% {#5}_{#6} & {#5}_{#7} & {#5}_{#8} & {#5}_{#9} & {#5}_{#10}
%\end{vmatrix}
%}

% R3 vector.
\newcommand{\VectorThree}[3]{
\begin{bmatrix}
 {#1} \\
 {#2} \\
 {#3}
\end{bmatrix}
}



\author{Peeter Joot}
\email{peeter.joot@gmail.com}


\chapter{Notes on Goldstein's Route's procedure.}
\label{chap:goldsteinRouth}
%\useCCL
\blogpage{http://sites.google.com/site/peeterjoot/math2010/goldsteinRouth.pdf}
\date{Mar 3, 2010}
\revisionInfo{goldsteinRouth.tex}

%\beginArtWithToc
\beginArtNoToc

\section{Motivation.}

Attempting study of \cite{goldstein1951cm} section 7-2 on Routh's procedure has been giving me some trouble.  It wasn't ``sinking in'', indicating a fundamental misunderstanding, or at least a requirement to work some examples.  Here I pick a system, the spherical pendulum, which has the required ignorable coordinate, to illustrate the ideas for myself with something less abstract.

\section{Spherical pendulum example.}

The Lagrangian for the pendulum is

\begin{align}\label{eqn:goldsteinRouth:1}
\LL = \inv{2} m r^2 \left( \dot{\theta}^2 + \dot{\phi}^2 \sin^2 \theta \right) - m g r ( 1 + \cos\theta ),
\end{align}

and our conjugate momenta are therefore

\begin{align}\label{eqn:goldsteinRouth:2}
p_\theta &= \PD{\dot{\theta}}{\LL} = m r^2 \dot{\theta} \\
p_\phi &= \PD{\dot{\phi}}{\LL} = m r^2 \sin^2\theta \dot{\phi}.
\end{align}

That's enough to now formulate the Hamiltonian $H = \dot{\theta} p_\theta + \dot{\phi} p_\phi - \LL$, which is

\begin{align}\label{eqn:goldsteinRouth:3}
H = H(\theta, p_\theta, p_\phi) = \inv{2 m r^2 } (p_\theta)^2 + \inv{2 m r^2 \sin^2\theta} (p_\phi)^2 + m g r ( 1 + \cos\theta ).
\end{align}

We've got the ignorable coordinate $\phi$ here, since the Hamiltonian has no explicit dependence on it.  In the Hamiltonian formalism the constant of motion associated with this comes as a consequence of evaluating the Hamiltonian equations.  For this system, those are

\begin{align}\label{eqn:goldsteinRouth:4}
\PD{\theta}{H} &= - \dot{p}_\theta \\
\PD{\phi}{H} &= - \dot{p}_\phi \\
\PD{p_\theta}{H} &= \dot{\theta} \\
\PD{p_\phi}{H} &= \dot{\phi},
\end{align}

Or, explicitly,
\begin{align}\label{eqn:goldsteinRouth:5}
- \dot{p}_\theta &= -m g r \sin\theta - \frac{\cos\theta}{2 m r^2 \sin^3 \theta} (p_\phi)^2  \\
- \dot{p}_\phi &= 0 \\
\dot{\theta} &= \inv{m r^2 } p_\theta \\
\dot{\phi} &= \inv{m r^2 \sin^2 \theta} p_\phi.
\end{align}

The second of these provides the integration constant, allowing us to write, $p_\phi = \alpha$.  Once this is done, our Hamiltonian example is reduced to one complete set of conjugate coordinates,

\begin{align}\label{eqn:goldsteinRouth:6}
H(\theta, p_\theta, \alpha) = \inv{2 m r^2 } (p_\theta)^2 + \inv{2 m r^2 \sin^2\theta} \alpha^2 + m g r ( 1 + \cos\theta ).
\end{align}

Goldstein notes that the behavior of the cyclic coordinate follows by integrating

\begin{align}\label{eqn:goldsteinRouth:7}
\dot{q}_n = \PD{\alpha}{H}.
\end{align}

In this example $\alpha = p_\theta$, so this is really just one of our Hamiltonian equations

\begin{align}\label{eqn:goldsteinRouth:8}
\dot{\phi} = \PD{p_\phi}{H}.
\end{align}

Okay, good.  First part of the mission is accomplished.  The setup for Routh's procedure no longer has anything mysterious to it.

Now, Goldstein defines the Routhian as

\begin{align}\label{eqn:goldsteinRouth:9}
R = p_i \dot{q}_i - \LL,
\end{align}

where the index $i$ is summed only over the cyclic (ignorable) coordinates.  For this spherical pendulum example, this is $q_i = \phi$, and $p_i = m r^2 \sin^2 \theta \dot{\phi}$, for

\begin{align}\label{eqn:goldsteinRouth:10}
R = \inv{2} m r^2 \left( -\dot{\theta}^2 + \dot{\phi}^2 \sin^2 \theta \right) + m g r ( 1 + \cos\theta ).
\end{align}

Now, we should also have for the non-cyclic coordinates, just like the Euler-Lagrange equations 

\begin{align}\label{eqn:goldsteinRouth:11}
\PD{\theta}{R} = \frac{d}{dt} \PD{\dot{\theta}}{R}.
\end{align}

Evaluating this we have

\begin{align}\label{eqn:goldsteinRouth:12}
m r^2 \sin\theta \cos\theta \dot{\phi}^2 - m g r \sin\theta = \frac{d}{dt} \left( -m r^2 \dot{\theta} \right).
\end{align}

It would be reasonable now to compare this the $\theta$ Euler-Lagrange equations, but evaluating those we get

\begin{align}\label{eqn:goldsteinRouth:13}
m r^2 \sin\theta \cos\theta \dot{\phi}^2 + m g r \sin\theta = \frac{d}{dt} \left( m r^2 \dot{\theta} \right).
\end{align}

Bugger.  We've got a sign difference on the $\dot{\phi}^2$ term.

\section{Simpler planar example.}

Having found an inconsistency with Routhian formalism and the concrete example of the spherical pendulum which has a cyclic coordinate as desired, let's step back slightly, and try a simpler example, artificially constructed

\begin{align}\label{eqn:goldsteinRouth:20}
\LL = \inv{2} m (\dot{x}^2 + \dot{y}^2 ) - V(x).
\end{align}

Our Hamiltonian and Routhian functions are

\begin{align}\label{eqn:goldsteinRouth:21}
H &= \inv{2} m (\dot{x}^2 + \dot{y}^2 ) + V(x) \\
R &= \inv{2} m (-\dot{x}^2 + \dot{y}^2 ) + V(x) 
\end{align}

For the non-cyclic coordinate we should have

\begin{align}\label{eqn:goldsteinRouth:22}
\PD{x}{R} = \frac{d}{dt} \PD{\dot{x}}{R},
\end{align}

which is

\begin{align}\label{eqn:goldsteinRouth:23}
V'(x) = \frac{d}{dt}\left( - m \dot{x} \right)
\end{align}

Okay, good, that's what is expected, and exactly what we get from the Euler-Lagrange equations.  This looks good, so where did things go wrong in the spherical pendulum evaluation.

\EndArticle

%
% Copyright � 2015 Peeter Joot.  All Rights Reserved.
% Licenced as described in the file LICENSE under the root directory of this GIT repository.
%
\documentclass[]{eliblog}

\usepackage{amsmath}
\usepackage{mathpazo}

%
% shorthand for bold symbols, convenient for vectors and matrices
%
\newcommand{\Ba}[0]{\mathbf{a}}
\newcommand{\Bb}[0]{\mathbf{b}}
\newcommand{\Bc}[0]{\mathbf{c}}
\newcommand{\Bd}[0]{\mathbf{d}}
\newcommand{\Be}[0]{\mathbf{e}}
\newcommand{\Bf}[0]{\mathbf{f}}
\newcommand{\Bg}[0]{\mathbf{g}}
\newcommand{\Bh}[0]{\mathbf{h}}
\newcommand{\Bi}[0]{\mathbf{i}}
\newcommand{\Bj}[0]{\mathbf{j}}
\newcommand{\Bk}[0]{\mathbf{k}}
\newcommand{\Bl}[0]{\mathbf{l}}
\newcommand{\Bm}[0]{\mathbf{m}}
\newcommand{\Bn}[0]{\mathbf{n}}
\newcommand{\Bo}[0]{\mathbf{o}}
\newcommand{\Bp}[0]{\mathbf{p}}
\newcommand{\Bq}[0]{\mathbf{q}}
\newcommand{\Br}[0]{\mathbf{r}}
\newcommand{\Bs}[0]{\mathbf{s}}
\newcommand{\Bt}[0]{\mathbf{t}}
\newcommand{\Bu}[0]{\mathbf{u}}
\newcommand{\Bv}[0]{\mathbf{v}}
\newcommand{\Bw}[0]{\mathbf{w}}
\newcommand{\Bx}[0]{\mathbf{x}}
\newcommand{\By}[0]{\mathbf{y}}
\newcommand{\Bz}[0]{\mathbf{z}}
\newcommand{\BA}[0]{\mathbf{A}}
\newcommand{\BB}[0]{\mathbf{B}}
\newcommand{\BC}[0]{\mathbf{C}}
\newcommand{\BD}[0]{\mathbf{D}}
\newcommand{\BE}[0]{\mathbf{E}}
\newcommand{\BF}[0]{\mathbf{F}}
\newcommand{\BG}[0]{\mathbf{G}}
\newcommand{\BH}[0]{\mathbf{H}}
\newcommand{\BI}[0]{\mathbf{I}}
\newcommand{\BJ}[0]{\mathbf{J}}
\newcommand{\BK}[0]{\mathbf{K}}
\newcommand{\BL}[0]{\mathbf{L}}
\newcommand{\BM}[0]{\mathbf{M}}
\newcommand{\BN}[0]{\mathbf{N}}
\newcommand{\BO}[0]{\mathbf{O}}
\newcommand{\BP}[0]{\mathbf{P}}
\newcommand{\BQ}[0]{\mathbf{Q}}
\newcommand{\BR}[0]{\mathbf{R}}
\newcommand{\BS}[0]{\mathbf{S}}
\newcommand{\BT}[0]{\mathbf{T}}
\newcommand{\BU}[0]{\mathbf{U}}
\newcommand{\BV}[0]{\mathbf{V}}
\newcommand{\BW}[0]{\mathbf{W}}
\newcommand{\BX}[0]{\mathbf{X}}
\newcommand{\BY}[0]{\mathbf{Y}}
\newcommand{\BZ}[0]{\mathbf{Z}}

\newcommand{\Bzero}[0]{\mathbf{0}}
\newcommand{\Btheta}[0]{\boldsymbol{\theta}}
\newcommand{\Btau}[0]{\boldsymbol{\tau}}
\newcommand{\Bomega}[0]{\boldsymbol{\omega}}

%
% shorthand for unit vectors
%
\newcommand{\acap}[0]{\hat{\Ba}}
\newcommand{\bcap}[0]{\hat{\Bb}}
\newcommand{\ccap}[0]{\hat{\Bc}}
\newcommand{\dcap}[0]{\hat{\Bd}}
\newcommand{\ecap}[0]{\hat{\Be}}
\newcommand{\fcap}[0]{\hat{\Bf}}
\newcommand{\gcap}[0]{\hat{\Bg}}
\newcommand{\hcap}[0]{\hat{\Bh}}
\newcommand{\icap}[0]{\hat{\Bi}}
\newcommand{\jcap}[0]{\hat{\Bj}}
\newcommand{\kcap}[0]{\hat{\Bk}}
\newcommand{\lcap}[0]{\hat{\Bl}}
\newcommand{\mcap}[0]{\hat{\Bm}}
\newcommand{\ncap}[0]{\hat{\Bn}}
\newcommand{\ocap}[0]{\hat{\Bo}}
\newcommand{\pcap}[0]{\hat{\Bp}}
\newcommand{\qcap}[0]{\hat{\Bq}}
\newcommand{\rcap}[0]{\hat{\Br}}
\newcommand{\scap}[0]{\hat{\Bs}}
\newcommand{\tcap}[0]{\hat{\Bt}}
\newcommand{\ucap}[0]{\hat{\Bu}}
\newcommand{\vcap}[0]{\hat{\Bv}}
\newcommand{\wcap}[0]{\hat{\Bw}}
\newcommand{\xcap}[0]{\hat{\Bx}}
\newcommand{\ycap}[0]{\hat{\By}}
\newcommand{\zcap}[0]{\hat{\Bz}}
\newcommand{\thetacap}[0]{\hat{\Btheta}}

%
% to write R^n and C^n in a distinguishable fashion.  Perhaps change this
% to the double lined characters upon figuring out how to do so.
%
\newcommand{\C}[1]{$\mathbb{C}^{#1}$}
\newcommand{\R}[1]{$\mathbb{R}^{#1}$}

%
% various generally useful helpers
%

% derivative of #1 wrt. #2:
\newcommand{\D}[2] {\frac {d#2} {d#1}}

\newcommand{\inv}[1]{\frac{1}{#1}}
\newcommand{\cross}[0]{\times}

\newcommand{\abs}[1]{\lvert{#1}\rvert}
\newcommand{\norm}[1]{\lVert{#1}\rVert}
\newcommand{\innerprod}[2]{\langle{#1}, {#2}\rangle}
\newcommand{\dotprod}[2]{{#1} \cdot {#2}}
\newcommand{\bdotprod}[2]{\left({#1} \cdot {#2}\right)}
\newcommand{\crossprod}[2]{{#1} \cross {#2}}
\newcommand{\tripleprod}[3]{\dotprod{\left(\crossprod{#1}{#2}\right)}{#3}}

\DeclareMathOperator{\Proj}{Proj}
\DeclareMathOperator{\Span}{span}
\DeclareMathOperator{\Sgn}{sgn}
\DeclareMathOperator{\Area}{Area}
\DeclareMathOperator{\Volume}{Volume}

%
% A few miscellaneous things specific to this document
%
\newcommand{\crossop}[1]{\crossprod{#1}{}}

% R2 vector.
\newcommand{\VectorTwo}[2]{
\begin{bmatrix}
 {#1} \\
 {#2}
\end{bmatrix}
}

\newcommand{\VectorN}[1]{
\begin{bmatrix}
{#1}_1 \\
{#1}_2 \\
\vdots \\
{#1}_N \\
\end{bmatrix}
}

\newcommand{\DETuvij}[4]{
\begin{vmatrix}
 {#1}_{#3} & {#1}_{#4} \\
 {#2}_{#3} & {#2}_{#4}
\end{vmatrix}
}

\newcommand{\DETuvwijk}[6]{
\begin{vmatrix}
 {#1}_{#4} & {#1}_{#5} & {#1}_{#6} \\
 {#2}_{#4} & {#2}_{#5} & {#2}_{#6} \\
 {#3}_{#4} & {#3}_{#5} & {#3}_{#6}
\end{vmatrix}
}

\newcommand{\DETuvwxijkl}[8]{
\begin{vmatrix}
 {#1}_{#5} & {#1}_{#6} & {#1}_{#7} & {#1}_{#8} \\
 {#2}_{#5} & {#2}_{#6} & {#2}_{#7} & {#2}_{#8} \\
 {#3}_{#5} & {#3}_{#6} & {#3}_{#7} & {#3}_{#8} \\
 {#4}_{#5} & {#4}_{#6} & {#4}_{#7} & {#4}_{#8} \\
\end{vmatrix}
}

%\newcommand{\DETuvwxyijklm}[10]{
%\begin{vmatrix}
% {#1}_{#6} & {#1}_{#7} & {#1}_{#8} & {#1}_{#9} & {#1}_{#10} \\
% {#2}_{#6} & {#2}_{#7} & {#2}_{#8} & {#2}_{#9} & {#2}_{#10} \\
% {#3}_{#6} & {#3}_{#7} & {#3}_{#8} & {#3}_{#9} & {#3}_{#10} \\
% {#4}_{#6} & {#4}_{#7} & {#4}_{#8} & {#4}_{#9} & {#4}_{#10} \\
% {#5}_{#6} & {#5}_{#7} & {#5}_{#8} & {#5}_{#9} & {#5}_{#10}
%\end{vmatrix}
%}

% R3 vector.
\newcommand{\VectorThree}[3]{
\begin{bmatrix}
 {#1} \\
 {#2} \\
 {#3}
\end{bmatrix}
}



\author{Peeter Joot}
\email{peeter.joot@gmail.com}

%\documentclass[]{eliblogwidescreen}

\usepackage{amsmath}
\usepackage{mathpazo}

%
% shorthand for bold symbols, convenient for vectors and matrices
%
\newcommand{\Ba}[0]{\mathbf{a}}
\newcommand{\Bb}[0]{\mathbf{b}}
\newcommand{\Bc}[0]{\mathbf{c}}
\newcommand{\Bd}[0]{\mathbf{d}}
\newcommand{\Be}[0]{\mathbf{e}}
\newcommand{\Bf}[0]{\mathbf{f}}
\newcommand{\Bg}[0]{\mathbf{g}}
\newcommand{\Bh}[0]{\mathbf{h}}
\newcommand{\Bi}[0]{\mathbf{i}}
\newcommand{\Bj}[0]{\mathbf{j}}
\newcommand{\Bk}[0]{\mathbf{k}}
\newcommand{\Bl}[0]{\mathbf{l}}
\newcommand{\Bm}[0]{\mathbf{m}}
\newcommand{\Bn}[0]{\mathbf{n}}
\newcommand{\Bo}[0]{\mathbf{o}}
\newcommand{\Bp}[0]{\mathbf{p}}
\newcommand{\Bq}[0]{\mathbf{q}}
\newcommand{\Br}[0]{\mathbf{r}}
\newcommand{\Bs}[0]{\mathbf{s}}
\newcommand{\Bt}[0]{\mathbf{t}}
\newcommand{\Bu}[0]{\mathbf{u}}
\newcommand{\Bv}[0]{\mathbf{v}}
\newcommand{\Bw}[0]{\mathbf{w}}
\newcommand{\Bx}[0]{\mathbf{x}}
\newcommand{\By}[0]{\mathbf{y}}
\newcommand{\Bz}[0]{\mathbf{z}}
\newcommand{\BA}[0]{\mathbf{A}}
\newcommand{\BB}[0]{\mathbf{B}}
\newcommand{\BC}[0]{\mathbf{C}}
\newcommand{\BD}[0]{\mathbf{D}}
\newcommand{\BE}[0]{\mathbf{E}}
\newcommand{\BF}[0]{\mathbf{F}}
\newcommand{\BG}[0]{\mathbf{G}}
\newcommand{\BH}[0]{\mathbf{H}}
\newcommand{\BI}[0]{\mathbf{I}}
\newcommand{\BJ}[0]{\mathbf{J}}
\newcommand{\BK}[0]{\mathbf{K}}
\newcommand{\BL}[0]{\mathbf{L}}
\newcommand{\BM}[0]{\mathbf{M}}
\newcommand{\BN}[0]{\mathbf{N}}
\newcommand{\BO}[0]{\mathbf{O}}
\newcommand{\BP}[0]{\mathbf{P}}
\newcommand{\BQ}[0]{\mathbf{Q}}
\newcommand{\BR}[0]{\mathbf{R}}
\newcommand{\BS}[0]{\mathbf{S}}
\newcommand{\BT}[0]{\mathbf{T}}
\newcommand{\BU}[0]{\mathbf{U}}
\newcommand{\BV}[0]{\mathbf{V}}
\newcommand{\BW}[0]{\mathbf{W}}
\newcommand{\BX}[0]{\mathbf{X}}
\newcommand{\BY}[0]{\mathbf{Y}}
\newcommand{\BZ}[0]{\mathbf{Z}}

\newcommand{\Bzero}[0]{\mathbf{0}}
\newcommand{\Btheta}[0]{\boldsymbol{\theta}}
\newcommand{\Btau}[0]{\boldsymbol{\tau}}
\newcommand{\Bomega}[0]{\boldsymbol{\omega}}

%
% shorthand for unit vectors
%
\newcommand{\acap}[0]{\hat{\Ba}}
\newcommand{\bcap}[0]{\hat{\Bb}}
\newcommand{\ccap}[0]{\hat{\Bc}}
\newcommand{\dcap}[0]{\hat{\Bd}}
\newcommand{\ecap}[0]{\hat{\Be}}
\newcommand{\fcap}[0]{\hat{\Bf}}
\newcommand{\gcap}[0]{\hat{\Bg}}
\newcommand{\hcap}[0]{\hat{\Bh}}
\newcommand{\icap}[0]{\hat{\Bi}}
\newcommand{\jcap}[0]{\hat{\Bj}}
\newcommand{\kcap}[0]{\hat{\Bk}}
\newcommand{\lcap}[0]{\hat{\Bl}}
\newcommand{\mcap}[0]{\hat{\Bm}}
\newcommand{\ncap}[0]{\hat{\Bn}}
\newcommand{\ocap}[0]{\hat{\Bo}}
\newcommand{\pcap}[0]{\hat{\Bp}}
\newcommand{\qcap}[0]{\hat{\Bq}}
\newcommand{\rcap}[0]{\hat{\Br}}
\newcommand{\scap}[0]{\hat{\Bs}}
\newcommand{\tcap}[0]{\hat{\Bt}}
\newcommand{\ucap}[0]{\hat{\Bu}}
\newcommand{\vcap}[0]{\hat{\Bv}}
\newcommand{\wcap}[0]{\hat{\Bw}}
\newcommand{\xcap}[0]{\hat{\Bx}}
\newcommand{\ycap}[0]{\hat{\By}}
\newcommand{\zcap}[0]{\hat{\Bz}}
\newcommand{\thetacap}[0]{\hat{\Btheta}}

%
% to write R^n and C^n in a distinguishable fashion.  Perhaps change this
% to the double lined characters upon figuring out how to do so.
%
\newcommand{\C}[1]{$\mathbb{C}^{#1}$}
\newcommand{\R}[1]{$\mathbb{R}^{#1}$}

%
% various generally useful helpers
%

% derivative of #1 wrt. #2:
\newcommand{\D}[2] {\frac {d#2} {d#1}}

\newcommand{\inv}[1]{\frac{1}{#1}}
\newcommand{\cross}[0]{\times}

\newcommand{\abs}[1]{\lvert{#1}\rvert}
\newcommand{\norm}[1]{\lVert{#1}\rVert}
\newcommand{\innerprod}[2]{\langle{#1}, {#2}\rangle}
\newcommand{\dotprod}[2]{{#1} \cdot {#2}}
\newcommand{\bdotprod}[2]{\left({#1} \cdot {#2}\right)}
\newcommand{\crossprod}[2]{{#1} \cross {#2}}
\newcommand{\tripleprod}[3]{\dotprod{\left(\crossprod{#1}{#2}\right)}{#3}}

\DeclareMathOperator{\Proj}{Proj}
\DeclareMathOperator{\Span}{span}
\DeclareMathOperator{\Sgn}{sgn}
\DeclareMathOperator{\Area}{Area}
\DeclareMathOperator{\Volume}{Volume}

%
% A few miscellaneous things specific to this document
%
\newcommand{\crossop}[1]{\crossprod{#1}{}}

% R2 vector.
\newcommand{\VectorTwo}[2]{
\begin{bmatrix}
 {#1} \\
 {#2}
\end{bmatrix}
}

\newcommand{\VectorN}[1]{
\begin{bmatrix}
{#1}_1 \\
{#1}_2 \\
\vdots \\
{#1}_N \\
\end{bmatrix}
}

\newcommand{\DETuvij}[4]{
\begin{vmatrix}
 {#1}_{#3} & {#1}_{#4} \\
 {#2}_{#3} & {#2}_{#4}
\end{vmatrix}
}

\newcommand{\DETuvwijk}[6]{
\begin{vmatrix}
 {#1}_{#4} & {#1}_{#5} & {#1}_{#6} \\
 {#2}_{#4} & {#2}_{#5} & {#2}_{#6} \\
 {#3}_{#4} & {#3}_{#5} & {#3}_{#6}
\end{vmatrix}
}

\newcommand{\DETuvwxijkl}[8]{
\begin{vmatrix}
 {#1}_{#5} & {#1}_{#6} & {#1}_{#7} & {#1}_{#8} \\
 {#2}_{#5} & {#2}_{#6} & {#2}_{#7} & {#2}_{#8} \\
 {#3}_{#5} & {#3}_{#6} & {#3}_{#7} & {#3}_{#8} \\
 {#4}_{#5} & {#4}_{#6} & {#4}_{#7} & {#4}_{#8} \\
\end{vmatrix}
}

%\newcommand{\DETuvwxyijklm}[10]{
%\begin{vmatrix}
% {#1}_{#6} & {#1}_{#7} & {#1}_{#8} & {#1}_{#9} & {#1}_{#10} \\
% {#2}_{#6} & {#2}_{#7} & {#2}_{#8} & {#2}_{#9} & {#2}_{#10} \\
% {#3}_{#6} & {#3}_{#7} & {#3}_{#8} & {#3}_{#9} & {#3}_{#10} \\
% {#4}_{#6} & {#4}_{#7} & {#4}_{#8} & {#4}_{#9} & {#4}_{#10} \\
% {#5}_{#6} & {#5}_{#7} & {#5}_{#8} & {#5}_{#9} & {#5}_{#10}
%\end{vmatrix}
%}

% R3 vector.
\newcommand{\VectorThree}[3]{
\begin{bmatrix}
 {#1} \\
 {#2} \\
 {#3}
\end{bmatrix}
}



\author{Peeter Joot}
\email{peeter.joot@gmail.com}


\chapter{Hoop and spring oscillator problem.}
\label{chap:hoopSpring}
%\useCCL
\blogpage{http://sites.google.com/site/peeterjoot/math2010/hoopSpring.pdf}
\date{June 19, 2010}
\revisionInfo{hoopSpring.tex}

%\beginArtWithToc
\beginArtNoToc

\section{Motivation.}

Nolan was attempting to setup and solve the equations for the following system (\ref{fig:hoopSpring})

\begin{figure}[htp]
\centering
\includegraphics[totalheight=0.4\textheight]{hoopSpring}
\caption{Coupled hoop and spring system.}\label{fig:hoopSpring}
\end{figure}

One mass is connected between two springs to a bar.  That bar moves up and down as forced by the motion of the other mass along a immovable hoop.  While Nolan didn't include any gravitational force in his potential terms (ie: system lying on a table perhaps) it doesn't take much more to include that, and I'll do so.  I also include the distance $L$ to the center of the hoop, which I believe required.

\section{Guts}

The Lagrangian can be written by inspection.  Writing $x = x_1$, and $x_2 = R \sin\theta$, we have

\begin{align}\label{eqn:hoopSpring:1}
\LL = 
\inv{2} m_1 \dot{x}^2 
+ \inv{2} m_2 R^2 \dot{\theta}^2 
- \inv{2} k_1 x^2 
- \inv{2} k_2 ( L + R \sin\theta - x )^2
- m_1 g x
- m_2 g ( L + R \sin\theta).
\end{align}

Evaluation of the Euler-Lagrange equations gives

\begin{subequations}
\label{eqn:hoopSpring:2}
\begin{align}
m_1 \ddot{x} &= - k_1 x + k_2 ( L + R \sin\theta - x ) - m_1 g \\
m_2 R^2 \ddot{\theta} &= - k_2 ( L + R \sin\theta - x ) R \cos\theta - m_2 g R \cos\theta,
\end{align}
\end{subequations}

or

\begin{subequations}
\label{eqn:hoopSpring:3}
\begin{align}
\ddot{x} &= 
%- \frac{k_1}{m_1} x + \frac{k_2}{m_1} ( L + R \sin\theta - x ) - g \\
-x \frac{k_1 + k_2}{m_1} 
+ \frac{k_2 R \sin\theta}{m_1} 
- g + \frac{k_2 L }{m_1} \\
\ddot{\theta} &= 
%1/(m_2 R^2) (- k_2 ( L + R \sin\theta - x ) R \cos\theta - m_2 g R \cos\theta)
- \inv{R}\left( \frac{k_2}{m_2} \left( L + R \sin\theta - x \right) +g \right) \cos\theta
.
\end{align}
\end{subequations}

Just like any other coupled pendulum system, this one is non-linear.  There's no obvious way to solve this in closed form, but we could determine a solution in the neighbourhood of a point $(x, \theta) = (x_0, \theta_0)$.  Let's switch our dynamical variables to ones that express the deviation from the initial point $\delta x = x - x_0$, and $\delta \theta = \theta - \theta_0$, with $u = (\delta x)'$, and $v = (\delta \theta)'$.  Our system then takes the form

\begin{subequations}
\label{eqn:hoopSpring:4}
\begin{align}
u' &= f(x,\theta) =
-x \frac{k_1 + k_2}{m_1} 
+ \frac{k_2 R \sin\theta}{m_1} 
- g + \frac{k_2 L }{m_1} \\
v' &= g(x,\theta) =
- \inv{R}\left( \frac{k_2}{m_2} \left( L + R \sin\theta - x \right) +g \right) \cos\theta \\
(\delta x)' &= u \\
(\delta \theta)' &= v
.
\end{align}
\end{subequations}

We can use a first order Taylor approximation of the form $f(x, \theta) = f(x_0, \theta_0) + f_x(x_0,\theta_0) (\delta x) + f_\theta(x_0,\theta_0) (\delta \theta)$.  So, to first order, our system has the approximation

\begin{subequations}
\label{eqn:hoopSpring:5}
\begin{align}
u' &= 
-x_0 \frac{k_1 + k_2}{m_1} 
+ \frac{k_2 R \sin\theta_0}{m_1} 
- g + \frac{k_2 L }{m_1} 
-(\delta x) \frac{k_1 + k_2}{m_1} 
+ \frac{k_2 R \cos\theta_0}{m_1} (\delta \theta)
\\
v' &= 
- \inv{R}\left( \frac{k_2}{m_2} \left( L + R \sin\theta_0 - x_0 \right) +g \right) \cos\theta_0
+ \frac{k_2 \cos\theta_0}{m_2 R} (\delta x)
- \inv{R}\left( \frac{k_2}{m_2} \left( \left( L - x_0 \right) \sin\theta_0 + R \right) + g \sin\theta_0 \right) (\delta \theta)
\\
(\delta x)' &= u \\
(\delta \theta)' &= v
.
\end{align}
\end{subequations}

This would be tidier in matrix form with $\Bx = (u, v, \delta x, \delta \theta)$

\begin{align}
\label{eqn:hoopSpring:6}
\Bx' &=
\begin{bmatrix}
-x_0 \frac{k_1 + k_2}{m_1} 
+ \frac{k_2 R \sin\theta_0}{m_1} 
- g + \frac{k_2 L }{m_1} \\
- \inv{R}\left( \frac{k_2}{m_2} \left( L + R \sin\theta_0 - x_0 \right) +g \right) \cos\theta_0 \\
0 \\
0
\end{bmatrix}
+
\begin{bmatrix}
0 & 0 &
-\frac{k_1 + k_2}{m_1} & \frac{k_2 R \cos\theta_0}{m_1} \\
0 & 0 & 
\frac{k_2 \cos\theta_0}{m_2 R} &
- \inv{R}\left( \frac{k_2}{m_2} \left( \left( L - x_0 \right) \sin\theta_0 + R \right) + g \sin\theta_0 \right) \\
1 & 0 & 0 & 0 \\
0 & 1 & 0 & 0 \\
\end{bmatrix}\Bx
.
\end{align}

This reduces the problem to the solutions of first order equations of the form

\begin{align}\label{eqn:hoopSpring:7}
\Bx' = \Ba 
+ \begin{bmatrix}
0 & A \\
I & 0
\end{bmatrix}\Bx,
\end{align}

where $\Ba$, and $A$ are constant matrices.

%\EndArticle
\EndNoBibArticle


\part{Electrodynamics}
%
% Copyright � 2012 Peeter Joot.  All Rights Reserved.
% Licenced as described in the file LICENSE under the root directory of this GIT repository.
%

%
%
%%
% Copyright � 2015 Peeter Joot.  All Rights Reserved.
% Licenced as described in the file LICENSE under the root directory of this GIT repository.
%
\documentclass[]{eliblog}

\usepackage{amsmath}
\usepackage{mathpazo}

%
% shorthand for bold symbols, convenient for vectors and matrices
%
\newcommand{\Ba}[0]{\mathbf{a}}
\newcommand{\Bb}[0]{\mathbf{b}}
\newcommand{\Bc}[0]{\mathbf{c}}
\newcommand{\Bd}[0]{\mathbf{d}}
\newcommand{\Be}[0]{\mathbf{e}}
\newcommand{\Bf}[0]{\mathbf{f}}
\newcommand{\Bg}[0]{\mathbf{g}}
\newcommand{\Bh}[0]{\mathbf{h}}
\newcommand{\Bi}[0]{\mathbf{i}}
\newcommand{\Bj}[0]{\mathbf{j}}
\newcommand{\Bk}[0]{\mathbf{k}}
\newcommand{\Bl}[0]{\mathbf{l}}
\newcommand{\Bm}[0]{\mathbf{m}}
\newcommand{\Bn}[0]{\mathbf{n}}
\newcommand{\Bo}[0]{\mathbf{o}}
\newcommand{\Bp}[0]{\mathbf{p}}
\newcommand{\Bq}[0]{\mathbf{q}}
\newcommand{\Br}[0]{\mathbf{r}}
\newcommand{\Bs}[0]{\mathbf{s}}
\newcommand{\Bt}[0]{\mathbf{t}}
\newcommand{\Bu}[0]{\mathbf{u}}
\newcommand{\Bv}[0]{\mathbf{v}}
\newcommand{\Bw}[0]{\mathbf{w}}
\newcommand{\Bx}[0]{\mathbf{x}}
\newcommand{\By}[0]{\mathbf{y}}
\newcommand{\Bz}[0]{\mathbf{z}}
\newcommand{\BA}[0]{\mathbf{A}}
\newcommand{\BB}[0]{\mathbf{B}}
\newcommand{\BC}[0]{\mathbf{C}}
\newcommand{\BD}[0]{\mathbf{D}}
\newcommand{\BE}[0]{\mathbf{E}}
\newcommand{\BF}[0]{\mathbf{F}}
\newcommand{\BG}[0]{\mathbf{G}}
\newcommand{\BH}[0]{\mathbf{H}}
\newcommand{\BI}[0]{\mathbf{I}}
\newcommand{\BJ}[0]{\mathbf{J}}
\newcommand{\BK}[0]{\mathbf{K}}
\newcommand{\BL}[0]{\mathbf{L}}
\newcommand{\BM}[0]{\mathbf{M}}
\newcommand{\BN}[0]{\mathbf{N}}
\newcommand{\BO}[0]{\mathbf{O}}
\newcommand{\BP}[0]{\mathbf{P}}
\newcommand{\BQ}[0]{\mathbf{Q}}
\newcommand{\BR}[0]{\mathbf{R}}
\newcommand{\BS}[0]{\mathbf{S}}
\newcommand{\BT}[0]{\mathbf{T}}
\newcommand{\BU}[0]{\mathbf{U}}
\newcommand{\BV}[0]{\mathbf{V}}
\newcommand{\BW}[0]{\mathbf{W}}
\newcommand{\BX}[0]{\mathbf{X}}
\newcommand{\BY}[0]{\mathbf{Y}}
\newcommand{\BZ}[0]{\mathbf{Z}}

\newcommand{\Bzero}[0]{\mathbf{0}}
\newcommand{\Btheta}[0]{\boldsymbol{\theta}}
\newcommand{\Btau}[0]{\boldsymbol{\tau}}
\newcommand{\Bomega}[0]{\boldsymbol{\omega}}

%
% shorthand for unit vectors
%
\newcommand{\acap}[0]{\hat{\Ba}}
\newcommand{\bcap}[0]{\hat{\Bb}}
\newcommand{\ccap}[0]{\hat{\Bc}}
\newcommand{\dcap}[0]{\hat{\Bd}}
\newcommand{\ecap}[0]{\hat{\Be}}
\newcommand{\fcap}[0]{\hat{\Bf}}
\newcommand{\gcap}[0]{\hat{\Bg}}
\newcommand{\hcap}[0]{\hat{\Bh}}
\newcommand{\icap}[0]{\hat{\Bi}}
\newcommand{\jcap}[0]{\hat{\Bj}}
\newcommand{\kcap}[0]{\hat{\Bk}}
\newcommand{\lcap}[0]{\hat{\Bl}}
\newcommand{\mcap}[0]{\hat{\Bm}}
\newcommand{\ncap}[0]{\hat{\Bn}}
\newcommand{\ocap}[0]{\hat{\Bo}}
\newcommand{\pcap}[0]{\hat{\Bp}}
\newcommand{\qcap}[0]{\hat{\Bq}}
\newcommand{\rcap}[0]{\hat{\Br}}
\newcommand{\scap}[0]{\hat{\Bs}}
\newcommand{\tcap}[0]{\hat{\Bt}}
\newcommand{\ucap}[0]{\hat{\Bu}}
\newcommand{\vcap}[0]{\hat{\Bv}}
\newcommand{\wcap}[0]{\hat{\Bw}}
\newcommand{\xcap}[0]{\hat{\Bx}}
\newcommand{\ycap}[0]{\hat{\By}}
\newcommand{\zcap}[0]{\hat{\Bz}}
\newcommand{\thetacap}[0]{\hat{\Btheta}}

%
% to write R^n and C^n in a distinguishable fashion.  Perhaps change this
% to the double lined characters upon figuring out how to do so.
%
\newcommand{\C}[1]{$\mathbb{C}^{#1}$}
\newcommand{\R}[1]{$\mathbb{R}^{#1}$}

%
% various generally useful helpers
%

% derivative of #1 wrt. #2:
\newcommand{\D}[2] {\frac {d#2} {d#1}}

\newcommand{\inv}[1]{\frac{1}{#1}}
\newcommand{\cross}[0]{\times}

\newcommand{\abs}[1]{\lvert{#1}\rvert}
\newcommand{\norm}[1]{\lVert{#1}\rVert}
\newcommand{\innerprod}[2]{\langle{#1}, {#2}\rangle}
\newcommand{\dotprod}[2]{{#1} \cdot {#2}}
\newcommand{\bdotprod}[2]{\left({#1} \cdot {#2}\right)}
\newcommand{\crossprod}[2]{{#1} \cross {#2}}
\newcommand{\tripleprod}[3]{\dotprod{\left(\crossprod{#1}{#2}\right)}{#3}}

\DeclareMathOperator{\Proj}{Proj}
\DeclareMathOperator{\Span}{span}
\DeclareMathOperator{\Sgn}{sgn}
\DeclareMathOperator{\Area}{Area}
\DeclareMathOperator{\Volume}{Volume}

%
% A few miscellaneous things specific to this document
%
\newcommand{\crossop}[1]{\crossprod{#1}{}}

% R2 vector.
\newcommand{\VectorTwo}[2]{
\begin{bmatrix}
 {#1} \\
 {#2}
\end{bmatrix}
}

\newcommand{\VectorN}[1]{
\begin{bmatrix}
{#1}_1 \\
{#1}_2 \\
\vdots \\
{#1}_N \\
\end{bmatrix}
}

\newcommand{\DETuvij}[4]{
\begin{vmatrix}
 {#1}_{#3} & {#1}_{#4} \\
 {#2}_{#3} & {#2}_{#4}
\end{vmatrix}
}

\newcommand{\DETuvwijk}[6]{
\begin{vmatrix}
 {#1}_{#4} & {#1}_{#5} & {#1}_{#6} \\
 {#2}_{#4} & {#2}_{#5} & {#2}_{#6} \\
 {#3}_{#4} & {#3}_{#5} & {#3}_{#6}
\end{vmatrix}
}

\newcommand{\DETuvwxijkl}[8]{
\begin{vmatrix}
 {#1}_{#5} & {#1}_{#6} & {#1}_{#7} & {#1}_{#8} \\
 {#2}_{#5} & {#2}_{#6} & {#2}_{#7} & {#2}_{#8} \\
 {#3}_{#5} & {#3}_{#6} & {#3}_{#7} & {#3}_{#8} \\
 {#4}_{#5} & {#4}_{#6} & {#4}_{#7} & {#4}_{#8} \\
\end{vmatrix}
}

%\newcommand{\DETuvwxyijklm}[10]{
%\begin{vmatrix}
% {#1}_{#6} & {#1}_{#7} & {#1}_{#8} & {#1}_{#9} & {#1}_{#10} \\
% {#2}_{#6} & {#2}_{#7} & {#2}_{#8} & {#2}_{#9} & {#2}_{#10} \\
% {#3}_{#6} & {#3}_{#7} & {#3}_{#8} & {#3}_{#9} & {#3}_{#10} \\
% {#4}_{#6} & {#4}_{#7} & {#4}_{#8} & {#4}_{#9} & {#4}_{#10} \\
% {#5}_{#6} & {#5}_{#7} & {#5}_{#8} & {#5}_{#9} & {#5}_{#10}
%\end{vmatrix}
%}

% R3 vector.
\newcommand{\VectorThree}[3]{
\begin{bmatrix}
 {#1} \\
 {#2} \\
 {#3}
\end{bmatrix}
}



\author{Peeter Joot}
\email{peeter.joot@gmail.com}

%\documentclass[]{eliblogwidescreen}

\usepackage{amsmath}
\usepackage{mathpazo}

%
% shorthand for bold symbols, convenient for vectors and matrices
%
\newcommand{\Ba}[0]{\mathbf{a}}
\newcommand{\Bb}[0]{\mathbf{b}}
\newcommand{\Bc}[0]{\mathbf{c}}
\newcommand{\Bd}[0]{\mathbf{d}}
\newcommand{\Be}[0]{\mathbf{e}}
\newcommand{\Bf}[0]{\mathbf{f}}
\newcommand{\Bg}[0]{\mathbf{g}}
\newcommand{\Bh}[0]{\mathbf{h}}
\newcommand{\Bi}[0]{\mathbf{i}}
\newcommand{\Bj}[0]{\mathbf{j}}
\newcommand{\Bk}[0]{\mathbf{k}}
\newcommand{\Bl}[0]{\mathbf{l}}
\newcommand{\Bm}[0]{\mathbf{m}}
\newcommand{\Bn}[0]{\mathbf{n}}
\newcommand{\Bo}[0]{\mathbf{o}}
\newcommand{\Bp}[0]{\mathbf{p}}
\newcommand{\Bq}[0]{\mathbf{q}}
\newcommand{\Br}[0]{\mathbf{r}}
\newcommand{\Bs}[0]{\mathbf{s}}
\newcommand{\Bt}[0]{\mathbf{t}}
\newcommand{\Bu}[0]{\mathbf{u}}
\newcommand{\Bv}[0]{\mathbf{v}}
\newcommand{\Bw}[0]{\mathbf{w}}
\newcommand{\Bx}[0]{\mathbf{x}}
\newcommand{\By}[0]{\mathbf{y}}
\newcommand{\Bz}[0]{\mathbf{z}}
\newcommand{\BA}[0]{\mathbf{A}}
\newcommand{\BB}[0]{\mathbf{B}}
\newcommand{\BC}[0]{\mathbf{C}}
\newcommand{\BD}[0]{\mathbf{D}}
\newcommand{\BE}[0]{\mathbf{E}}
\newcommand{\BF}[0]{\mathbf{F}}
\newcommand{\BG}[0]{\mathbf{G}}
\newcommand{\BH}[0]{\mathbf{H}}
\newcommand{\BI}[0]{\mathbf{I}}
\newcommand{\BJ}[0]{\mathbf{J}}
\newcommand{\BK}[0]{\mathbf{K}}
\newcommand{\BL}[0]{\mathbf{L}}
\newcommand{\BM}[0]{\mathbf{M}}
\newcommand{\BN}[0]{\mathbf{N}}
\newcommand{\BO}[0]{\mathbf{O}}
\newcommand{\BP}[0]{\mathbf{P}}
\newcommand{\BQ}[0]{\mathbf{Q}}
\newcommand{\BR}[0]{\mathbf{R}}
\newcommand{\BS}[0]{\mathbf{S}}
\newcommand{\BT}[0]{\mathbf{T}}
\newcommand{\BU}[0]{\mathbf{U}}
\newcommand{\BV}[0]{\mathbf{V}}
\newcommand{\BW}[0]{\mathbf{W}}
\newcommand{\BX}[0]{\mathbf{X}}
\newcommand{\BY}[0]{\mathbf{Y}}
\newcommand{\BZ}[0]{\mathbf{Z}}

\newcommand{\Bzero}[0]{\mathbf{0}}
\newcommand{\Btheta}[0]{\boldsymbol{\theta}}
\newcommand{\Btau}[0]{\boldsymbol{\tau}}
\newcommand{\Bomega}[0]{\boldsymbol{\omega}}

%
% shorthand for unit vectors
%
\newcommand{\acap}[0]{\hat{\Ba}}
\newcommand{\bcap}[0]{\hat{\Bb}}
\newcommand{\ccap}[0]{\hat{\Bc}}
\newcommand{\dcap}[0]{\hat{\Bd}}
\newcommand{\ecap}[0]{\hat{\Be}}
\newcommand{\fcap}[0]{\hat{\Bf}}
\newcommand{\gcap}[0]{\hat{\Bg}}
\newcommand{\hcap}[0]{\hat{\Bh}}
\newcommand{\icap}[0]{\hat{\Bi}}
\newcommand{\jcap}[0]{\hat{\Bj}}
\newcommand{\kcap}[0]{\hat{\Bk}}
\newcommand{\lcap}[0]{\hat{\Bl}}
\newcommand{\mcap}[0]{\hat{\Bm}}
\newcommand{\ncap}[0]{\hat{\Bn}}
\newcommand{\ocap}[0]{\hat{\Bo}}
\newcommand{\pcap}[0]{\hat{\Bp}}
\newcommand{\qcap}[0]{\hat{\Bq}}
\newcommand{\rcap}[0]{\hat{\Br}}
\newcommand{\scap}[0]{\hat{\Bs}}
\newcommand{\tcap}[0]{\hat{\Bt}}
\newcommand{\ucap}[0]{\hat{\Bu}}
\newcommand{\vcap}[0]{\hat{\Bv}}
\newcommand{\wcap}[0]{\hat{\Bw}}
\newcommand{\xcap}[0]{\hat{\Bx}}
\newcommand{\ycap}[0]{\hat{\By}}
\newcommand{\zcap}[0]{\hat{\Bz}}
\newcommand{\thetacap}[0]{\hat{\Btheta}}

%
% to write R^n and C^n in a distinguishable fashion.  Perhaps change this
% to the double lined characters upon figuring out how to do so.
%
\newcommand{\C}[1]{$\mathbb{C}^{#1}$}
\newcommand{\R}[1]{$\mathbb{R}^{#1}$}

%
% various generally useful helpers
%

% derivative of #1 wrt. #2:
\newcommand{\D}[2] {\frac {d#2} {d#1}}

\newcommand{\inv}[1]{\frac{1}{#1}}
\newcommand{\cross}[0]{\times}

\newcommand{\abs}[1]{\lvert{#1}\rvert}
\newcommand{\norm}[1]{\lVert{#1}\rVert}
\newcommand{\innerprod}[2]{\langle{#1}, {#2}\rangle}
\newcommand{\dotprod}[2]{{#1} \cdot {#2}}
\newcommand{\bdotprod}[2]{\left({#1} \cdot {#2}\right)}
\newcommand{\crossprod}[2]{{#1} \cross {#2}}
\newcommand{\tripleprod}[3]{\dotprod{\left(\crossprod{#1}{#2}\right)}{#3}}

\DeclareMathOperator{\Proj}{Proj}
\DeclareMathOperator{\Span}{span}
\DeclareMathOperator{\Sgn}{sgn}
\DeclareMathOperator{\Area}{Area}
\DeclareMathOperator{\Volume}{Volume}

%
% A few miscellaneous things specific to this document
%
\newcommand{\crossop}[1]{\crossprod{#1}{}}

% R2 vector.
\newcommand{\VectorTwo}[2]{
\begin{bmatrix}
 {#1} \\
 {#2}
\end{bmatrix}
}

\newcommand{\VectorN}[1]{
\begin{bmatrix}
{#1}_1 \\
{#1}_2 \\
\vdots \\
{#1}_N \\
\end{bmatrix}
}

\newcommand{\DETuvij}[4]{
\begin{vmatrix}
 {#1}_{#3} & {#1}_{#4} \\
 {#2}_{#3} & {#2}_{#4}
\end{vmatrix}
}

\newcommand{\DETuvwijk}[6]{
\begin{vmatrix}
 {#1}_{#4} & {#1}_{#5} & {#1}_{#6} \\
 {#2}_{#4} & {#2}_{#5} & {#2}_{#6} \\
 {#3}_{#4} & {#3}_{#5} & {#3}_{#6}
\end{vmatrix}
}

\newcommand{\DETuvwxijkl}[8]{
\begin{vmatrix}
 {#1}_{#5} & {#1}_{#6} & {#1}_{#7} & {#1}_{#8} \\
 {#2}_{#5} & {#2}_{#6} & {#2}_{#7} & {#2}_{#8} \\
 {#3}_{#5} & {#3}_{#6} & {#3}_{#7} & {#3}_{#8} \\
 {#4}_{#5} & {#4}_{#6} & {#4}_{#7} & {#4}_{#8} \\
\end{vmatrix}
}

%\newcommand{\DETuvwxyijklm}[10]{
%\begin{vmatrix}
% {#1}_{#6} & {#1}_{#7} & {#1}_{#8} & {#1}_{#9} & {#1}_{#10} \\
% {#2}_{#6} & {#2}_{#7} & {#2}_{#8} & {#2}_{#9} & {#2}_{#10} \\
% {#3}_{#6} & {#3}_{#7} & {#3}_{#8} & {#3}_{#9} & {#3}_{#10} \\
% {#4}_{#6} & {#4}_{#7} & {#4}_{#8} & {#4}_{#9} & {#4}_{#10} \\
% {#5}_{#6} & {#5}_{#7} & {#5}_{#8} & {#5}_{#9} & {#5}_{#10}
%\end{vmatrix}
%}

% R3 vector.
\newcommand{\VectorThree}[3]{
\begin{bmatrix}
 {#1} \\
 {#2} \\
 {#3}
\end{bmatrix}
}



\author{Peeter Joot}
\email{peeter.joot@gmail.com}


\chapter{Classical Electrodynamic gauge interaction}
\label{chap:gaugeInteractionHamiltonian}
%\useCCL
\blogpage{http://sites.google.com/site/peeterjoot/math2010/gaugeInteractionHamiltonian.pdf}
\date{Oct 22, 2010}
\revisionInfo{gaugeInteractionHamiltonian.tex}

\beginArtWithToc
%\beginArtNoToc

\section{Motivation}

In \citep{desai2009quantum} chapter 6, we have a statement that in classical mechanics the electromagnetic interaction is due to a transformation of the following form

\begin{equation}\label{eqn:gaugeInteractionHamiltonian:1}
\begin{aligned}
\Bp &\rightarrow \Bp - \frac{e}{c} \BA \\
E &\rightarrow E - e \phi
\end{aligned}
\end{equation}

Let us verify that this does produce the classical interaction law.  Putting a more familiar label on this, we should see that we obtain the Lorentz force law from a transformation of the Hamiltonian.

\section{Hamiltonian equations}

Recall that the Hamiltonian was defined in terms of conjugate momentum components \(p_k\) as
\begin{equation}\label{eqn:gaugeInteractionHamiltonian:4}
H(x_k, p_k) = \xdot_k p_k - \LL(x_k, \xdot_k),
\end{equation}

we can take \(x_k\) partials to obtain the first of the Hamiltonian system of equations for the motion
\begin{equation}\label{eqn:gaugeInteractionHamiltonian:50}
\begin{aligned}
\PD{x_k}{H}
&= - \PD{x_k}{\LL}  \\
&= - \frac{d}{dt} \PD{\xdot_k}{\LL}
\end{aligned}
\end{equation}

With \(p_k \equiv \PDi{\xdot_k}{\LL}\), and taking \(p_k\) partials too, we have the system of equations

\begin{subequations}
\begin{equation}\label{eqn:gaugeInteractionHamiltonian:5}
\PD{x_k}{H} = - \frac{d p_k}{dt}
\end{equation}
\begin{equation}\label{eqn:gaugeInteractionHamiltonian:6}
\PD{p_k}{H} = \xdot_k
\end{equation}
\end{subequations}

\section{Classical interaction}

Starting with the free particle Hamiltonian

\begin{equation}\label{eqn:gaugeInteractionHamiltonian:20}
H = \frac{\Bp}{2m},
\end{equation}

we make the transformation required to both the energy and momentum terms

\begin{equation}\label{eqn:gaugeInteractionHamiltonian:21}
H - e\phi = \frac{\left(\Bp - \frac{e}{c} \BA\right)^2 }{2m} = \inv{2m} \Bp^2 - \frac{e}{m c} \Bp \cdot \BA + \inv{2m} \left(\frac{e}{c}\right)^2 \BA^2
\end{equation}

From \eqnref{eqn:gaugeInteractionHamiltonian:6} we find

\begin{equation}\label{eqn:gaugeInteractionHamiltonian:22}
\frac{d x_k}{dt} = \PD{p_k}{H} = \inv{m} \left( p_k - \frac{e}{c} A_k \right),
\end{equation}

or
\begin{equation}\label{eqn:gaugeInteractionHamiltonian:23}
p_k = m \frac{d x_k}{dt} + \frac{e}{c} A_k.
\end{equation}

Taking derivatives and employing \eqnref{eqn:gaugeInteractionHamiltonian:5} we have
\begin{equation}\label{eqn:gaugeInteractionHamiltonian:70}
\begin{aligned}
\frac{d p_k}{dt}
&= m \frac{d^2 x_k}{dt^2} + \frac{e}{c} \frac{d A_k}{dt}  \\
&= -\PD{x_k}{H} \\
&=
\inv{m} \frac{e}{c} p_n \PD{x_k}{A_n} - e \PD{x_k}{\phi}
- \inv{m} \left(\frac{e}{c}\right)^2 A_k \PD{x_k}{A_k} \\
&=
\inv{m} \frac{e}{c} \left(
m \frac{d x_n}{dt} + \frac{e}{c} A_n
\right)
\PD{x_k}{A_n}
 - e \PD{x_k}{\phi}
- \inv{m} \left(\frac{e}{c}\right)^2 A_k \PD{x_k}{A_k} \\
&=
\frac{e}{c} \frac{d x_n}{dt}
\PD{x_k}{A_n}
 - e \PD{x_k}{\phi}
\end{aligned}
\end{equation}

Rearranging and utilizing the convective derivative expansion \(d/dt = (d x_a/dt) \PDi{x_a}{}\) (ie: chain rule), we have

\begin{equation}\label{eqn:gaugeInteractionHamiltonian:25}
\begin{aligned}
m \frac{d^2 x_k}{dt^2}
&=
\frac{e}{c}
\frac{d x_n}{dt}
\left(
\PD{x_k}{A_n}
-
\PD{x_n}{A_k}
\right)
 - e \PD{x_k}{\phi}
\end{aligned}
\end{equation}

We guess and expect that the first term of \eqnref{eqn:gaugeInteractionHamiltonian:25} is \(e (\Bv/c \cross \BB)_k\).  Let us verify this

\begin{equation}\label{eqn:gaugeInteractionHamiltonian:90}
\begin{aligned}
(\Bv \cross \BB)_k
&= \xdot_m B_d \epsilon_{k m d} \\
&= \xdot_m ( \epsilon_{d a b} \partial_a A_b ) \epsilon_{k m d} \\
&= \xdot_m \partial_a A_b \epsilon_{d a b} \epsilon_{d k m}
\end{aligned}
\end{equation}

Since \(\epsilon_{d a b} \epsilon_{d k m} = \delta_{a k} \delta_{b m} - \delta_{a m} \delta_{b k}\) we have

\begin{equation}\label{eqn:gaugeInteractionHamiltonian:110}
\begin{aligned}
(\Bv \cross \BB)_k
&= \xdot_m \partial_a A_b \epsilon_{d a b} \epsilon_{d k m} \\
&=
\xdot_m \partial_a A_b \delta_{a k} \delta_{b m}
-\xdot_m \partial_a A_b \delta_{a m} \delta_{b k} \\
&=
\xdot_m ( \partial_k A_m - \partial_m A_k )
\end{aligned}
\end{equation}

Except for a difference in dummy summation variables, this matches what we had in \eqnref{eqn:gaugeInteractionHamiltonian:25}.  Thus we are able to put that into the traditional Lorentz force vector form

\begin{equation}\label{eqn:gaugeInteractionHamiltonian:30}
m \frac{d^2 \Bx}{dt^2} = e \frac{\Bv}{c} \cross \BB + e \BE.
\end{equation}

It is good to see that we get the classical interaction from this transformation before moving on to the trickier seeming QM interaction.

\EndArticle


\part{Quantum Mechanics.}
% 
% 
% 
% Copyright � 2012 Peeter Joot
% All Rights Reserved
% 
% This file may be reproduced and distributed in whole or in part, without fee, subject to the following conditions:
% 
% o The copyright notice above and this permission notice must be preserved complete on all complete or partial copies.
% 
% o Any translation or derived work must be approved by the author in writing before distribution.
% 
% o If you distribute this work in part, instructions for obtaining the complete version of this file must be included, and a means for obtaining a complete version provided.
% 
% 
% Exceptions to these rules may be granted for academic purposes: Write to the author and ask.
% 
% 
% 
%\documentclass{article}

%\usepackage{amsmath}
\usepackage{mathpazo}

%
% shorthand for bold symbols, convenient for vectors and matrices
%
\newcommand{\Ba}[0]{\mathbf{a}}
\newcommand{\Bb}[0]{\mathbf{b}}
\newcommand{\Bc}[0]{\mathbf{c}}
\newcommand{\Bd}[0]{\mathbf{d}}
\newcommand{\Be}[0]{\mathbf{e}}
\newcommand{\Bf}[0]{\mathbf{f}}
\newcommand{\Bg}[0]{\mathbf{g}}
\newcommand{\Bh}[0]{\mathbf{h}}
\newcommand{\Bi}[0]{\mathbf{i}}
\newcommand{\Bj}[0]{\mathbf{j}}
\newcommand{\Bk}[0]{\mathbf{k}}
\newcommand{\Bl}[0]{\mathbf{l}}
\newcommand{\Bm}[0]{\mathbf{m}}
\newcommand{\Bn}[0]{\mathbf{n}}
\newcommand{\Bo}[0]{\mathbf{o}}
\newcommand{\Bp}[0]{\mathbf{p}}
\newcommand{\Bq}[0]{\mathbf{q}}
\newcommand{\Br}[0]{\mathbf{r}}
\newcommand{\Bs}[0]{\mathbf{s}}
\newcommand{\Bt}[0]{\mathbf{t}}
\newcommand{\Bu}[0]{\mathbf{u}}
\newcommand{\Bv}[0]{\mathbf{v}}
\newcommand{\Bw}[0]{\mathbf{w}}
\newcommand{\Bx}[0]{\mathbf{x}}
\newcommand{\By}[0]{\mathbf{y}}
\newcommand{\Bz}[0]{\mathbf{z}}
\newcommand{\BA}[0]{\mathbf{A}}
\newcommand{\BB}[0]{\mathbf{B}}
\newcommand{\BC}[0]{\mathbf{C}}
\newcommand{\BD}[0]{\mathbf{D}}
\newcommand{\BE}[0]{\mathbf{E}}
\newcommand{\BF}[0]{\mathbf{F}}
\newcommand{\BG}[0]{\mathbf{G}}
\newcommand{\BH}[0]{\mathbf{H}}
\newcommand{\BI}[0]{\mathbf{I}}
\newcommand{\BJ}[0]{\mathbf{J}}
\newcommand{\BK}[0]{\mathbf{K}}
\newcommand{\BL}[0]{\mathbf{L}}
\newcommand{\BM}[0]{\mathbf{M}}
\newcommand{\BN}[0]{\mathbf{N}}
\newcommand{\BO}[0]{\mathbf{O}}
\newcommand{\BP}[0]{\mathbf{P}}
\newcommand{\BQ}[0]{\mathbf{Q}}
\newcommand{\BR}[0]{\mathbf{R}}
\newcommand{\BS}[0]{\mathbf{S}}
\newcommand{\BT}[0]{\mathbf{T}}
\newcommand{\BU}[0]{\mathbf{U}}
\newcommand{\BV}[0]{\mathbf{V}}
\newcommand{\BW}[0]{\mathbf{W}}
\newcommand{\BX}[0]{\mathbf{X}}
\newcommand{\BY}[0]{\mathbf{Y}}
\newcommand{\BZ}[0]{\mathbf{Z}}

\newcommand{\Bzero}[0]{\mathbf{0}}
\newcommand{\Btheta}[0]{\boldsymbol{\theta}}
\newcommand{\Btau}[0]{\boldsymbol{\tau}}
\newcommand{\Bomega}[0]{\boldsymbol{\omega}}

%
% shorthand for unit vectors
%
\newcommand{\acap}[0]{\hat{\Ba}}
\newcommand{\bcap}[0]{\hat{\Bb}}
\newcommand{\ccap}[0]{\hat{\Bc}}
\newcommand{\dcap}[0]{\hat{\Bd}}
\newcommand{\ecap}[0]{\hat{\Be}}
\newcommand{\fcap}[0]{\hat{\Bf}}
\newcommand{\gcap}[0]{\hat{\Bg}}
\newcommand{\hcap}[0]{\hat{\Bh}}
\newcommand{\icap}[0]{\hat{\Bi}}
\newcommand{\jcap}[0]{\hat{\Bj}}
\newcommand{\kcap}[0]{\hat{\Bk}}
\newcommand{\lcap}[0]{\hat{\Bl}}
\newcommand{\mcap}[0]{\hat{\Bm}}
\newcommand{\ncap}[0]{\hat{\Bn}}
\newcommand{\ocap}[0]{\hat{\Bo}}
\newcommand{\pcap}[0]{\hat{\Bp}}
\newcommand{\qcap}[0]{\hat{\Bq}}
\newcommand{\rcap}[0]{\hat{\Br}}
\newcommand{\scap}[0]{\hat{\Bs}}
\newcommand{\tcap}[0]{\hat{\Bt}}
\newcommand{\ucap}[0]{\hat{\Bu}}
\newcommand{\vcap}[0]{\hat{\Bv}}
\newcommand{\wcap}[0]{\hat{\Bw}}
\newcommand{\xcap}[0]{\hat{\Bx}}
\newcommand{\ycap}[0]{\hat{\By}}
\newcommand{\zcap}[0]{\hat{\Bz}}
\newcommand{\thetacap}[0]{\hat{\Btheta}}

%
% to write R^n and C^n in a distinguishable fashion.  Perhaps change this
% to the double lined characters upon figuring out how to do so.
%
\newcommand{\C}[1]{$\mathbb{C}^{#1}$}
\newcommand{\R}[1]{$\mathbb{R}^{#1}$}

%
% various generally useful helpers
%

% derivative of #1 wrt. #2:
\newcommand{\D}[2] {\frac {d#2} {d#1}}

\newcommand{\inv}[1]{\frac{1}{#1}}
\newcommand{\cross}[0]{\times}

\newcommand{\abs}[1]{\lvert{#1}\rvert}
\newcommand{\norm}[1]{\lVert{#1}\rVert}
\newcommand{\innerprod}[2]{\langle{#1}, {#2}\rangle}
\newcommand{\dotprod}[2]{{#1} \cdot {#2}}
\newcommand{\bdotprod}[2]{\left({#1} \cdot {#2}\right)}
\newcommand{\crossprod}[2]{{#1} \cross {#2}}
\newcommand{\tripleprod}[3]{\dotprod{\left(\crossprod{#1}{#2}\right)}{#3}}

\DeclareMathOperator{\Proj}{Proj}
\DeclareMathOperator{\Span}{span}
\DeclareMathOperator{\Sgn}{sgn}
\DeclareMathOperator{\Area}{Area}
\DeclareMathOperator{\Volume}{Volume}

%
% A few miscellaneous things specific to this document
%
\newcommand{\crossop}[1]{\crossprod{#1}{}}

% R2 vector.
\newcommand{\VectorTwo}[2]{
\begin{bmatrix}
 {#1} \\
 {#2}
\end{bmatrix}
}

\newcommand{\VectorN}[1]{
\begin{bmatrix}
{#1}_1 \\
{#1}_2 \\
\vdots \\
{#1}_N \\
\end{bmatrix}
}

\newcommand{\DETuvij}[4]{
\begin{vmatrix}
 {#1}_{#3} & {#1}_{#4} \\
 {#2}_{#3} & {#2}_{#4}
\end{vmatrix}
}

\newcommand{\DETuvwijk}[6]{
\begin{vmatrix}
 {#1}_{#4} & {#1}_{#5} & {#1}_{#6} \\
 {#2}_{#4} & {#2}_{#5} & {#2}_{#6} \\
 {#3}_{#4} & {#3}_{#5} & {#3}_{#6}
\end{vmatrix}
}

\newcommand{\DETuvwxijkl}[8]{
\begin{vmatrix}
 {#1}_{#5} & {#1}_{#6} & {#1}_{#7} & {#1}_{#8} \\
 {#2}_{#5} & {#2}_{#6} & {#2}_{#7} & {#2}_{#8} \\
 {#3}_{#5} & {#3}_{#6} & {#3}_{#7} & {#3}_{#8} \\
 {#4}_{#5} & {#4}_{#6} & {#4}_{#7} & {#4}_{#8} \\
\end{vmatrix}
}

%\newcommand{\DETuvwxyijklm}[10]{
%\begin{vmatrix}
% {#1}_{#6} & {#1}_{#7} & {#1}_{#8} & {#1}_{#9} & {#1}_{#10} \\
% {#2}_{#6} & {#2}_{#7} & {#2}_{#8} & {#2}_{#9} & {#2}_{#10} \\
% {#3}_{#6} & {#3}_{#7} & {#3}_{#8} & {#3}_{#9} & {#3}_{#10} \\
% {#4}_{#6} & {#4}_{#7} & {#4}_{#8} & {#4}_{#9} & {#4}_{#10} \\
% {#5}_{#6} & {#5}_{#7} & {#5}_{#8} & {#5}_{#9} & {#5}_{#10}
%\end{vmatrix}
%}

% R3 vector.
\newcommand{\VectorThree}[3]{
\begin{bmatrix}
 {#1} \\
 {#2} \\
 {#3}
\end{bmatrix}
}


%%<misc>
%
\newcommand{\Abs}[1]{{\left\lvert{#1}\right\rvert}}
\newcommand{\spacegrad}[0]{\boldsymbol{\nabla}}
\newcommand{\grad}[0]{\nabla}
\newcommand{\LL}[0]{\mathcal{L}}

% == \partial_{#1} {#2}
\newcommand{\PD}[2]{\frac{\partial {#2}}{\partial {#1}}}
% inline variant
\newcommand{\PDi}[2]{{\partial {#2}}/{\partial {#1}}}

\newcommand{\PDD}[3]{\frac{\partial^2 {#3}}{\partial {#1}\partial {#2}}}
%\newcommand{\PDd}[2]{\frac{\partial^2 {#2}}{{\partial{#1}}^2}}
\newcommand{\PDsq}[2]{\frac{\partial^2 {#2}}{(\partial {#1})^2}}

\newcommand{\Partial}[2]{\frac{\partial {#1}}{\partial {#2}}}
\DeclareMathOperator{\RejName}{Rej}
\newcommand{\Rej}[2]{\RejName_{#1}\left( {#2} \right)}
\newcommand{\Rm}[1]{\mathbb{R}^{#1}}
\newcommand{\Cm}[1]{\mathbb{C}^{#1}}
\newcommand{\conj}[0]{{*}}

%</misc>

% <grade selection>
%
\newcommand{\gpgrade}[2] {{\left\langle{{#1}}\right\rangle}_{#2}}

\newcommand{\gpgradezero}[1] {\gpgrade{#1}{}}
%\newcommand{\gpscalargrade}[1] {{\left\langle{{#1}}\right\rangle}}
%\newcommand{\gpgradezero}[1] {\gpgrade{#1}{0}}

%\newcommand{\gpgradeone}[1] {{\left\langle{{#1}}\right\rangle}_{1}}
\newcommand{\gpgradeone}[1] {\gpgrade{#1}{1}}

\newcommand{\gpgradetwo}[1] {\gpgrade{#1}{2}}
\newcommand{\gpgradethree}[1] {\gpgrade{#1}{3}}
\newcommand{\gpgradefour}[1] {\gpgrade{#1}{4}}
%
% </grade selection>



\newcommand{\adot}[0]{{\dot{a}}}
\newcommand{\bdot}[0]{{\dot{b}}}
% taken for centered dot:
%\newcommand{\cdot}[0]{{\dot{c}}}
%\newcommand{\ddot}[0]{{\dot{d}}}
\newcommand{\edot}[0]{{\dot{e}}}
\newcommand{\fdot}[0]{{\dot{f}}}
\newcommand{\gdot}[0]{{\dot{g}}}
\newcommand{\hdot}[0]{{\dot{h}}}
\newcommand{\idot}[0]{{\dot{i}}}
\newcommand{\jdot}[0]{{\dot{j}}}
\newcommand{\kdot}[0]{{\dot{k}}}
\newcommand{\ldot}[0]{{\dot{l}}}
\newcommand{\mdot}[0]{{\dot{m}}}
\newcommand{\ndot}[0]{{\dot{n}}}
%\newcommand{\odot}[0]{{\dot{o}}}
\newcommand{\pdot}[0]{{\dot{p}}}
\newcommand{\qdot}[0]{{\dot{q}}}
\newcommand{\rdot}[0]{{\dot{r}}}
\newcommand{\sdot}[0]{{\dot{s}}}
\newcommand{\tdot}[0]{{\dot{t}}}
\newcommand{\udot}[0]{{\dot{u}}}
\newcommand{\vdot}[0]{{\dot{v}}}
\newcommand{\wdot}[0]{{\dot{w}}}
\newcommand{\xdot}[0]{{\dot{x}}}
\newcommand{\ydot}[0]{{\dot{y}}}
\newcommand{\zdot}[0]{{\dot{z}}}
\newcommand{\addot}[0]{{\ddot{a}}}
\newcommand{\bddot}[0]{{\ddot{b}}}
\newcommand{\cddot}[0]{{\ddot{c}}}
%\newcommand{\dddot}[0]{{\ddot{d}}}
\newcommand{\eddot}[0]{{\ddot{e}}}
\newcommand{\fddot}[0]{{\ddot{f}}}
\newcommand{\gddot}[0]{{\ddot{g}}}
\newcommand{\hddot}[0]{{\ddot{h}}}
\newcommand{\iddot}[0]{{\ddot{i}}}
\newcommand{\jddot}[0]{{\ddot{j}}}
\newcommand{\kddot}[0]{{\ddot{k}}}
\newcommand{\lddot}[0]{{\ddot{l}}}
\newcommand{\mddot}[0]{{\ddot{m}}}
\newcommand{\nddot}[0]{{\ddot{n}}}
\newcommand{\oddot}[0]{{\ddot{o}}}
\newcommand{\pddot}[0]{{\ddot{p}}}
\newcommand{\qddot}[0]{{\ddot{q}}}
\newcommand{\rddot}[0]{{\ddot{r}}}
\newcommand{\sddot}[0]{{\ddot{s}}}
\newcommand{\tddot}[0]{{\ddot{t}}}
\newcommand{\uddot}[0]{{\ddot{u}}}
\newcommand{\vddot}[0]{{\ddot{v}}}
\newcommand{\wddot}[0]{{\ddot{w}}}
\newcommand{\xddot}[0]{{\ddot{x}}}
\newcommand{\yddot}[0]{{\ddot{y}}}
\newcommand{\zddot}[0]{{\ddot{z}}}

%<bold and dot greek symbols>
%

\newcommand{\Deltadot}[0]{{\dot{\Delta}}}
\newcommand{\Gammadot}[0]{{\dot{\Gamma}}}
\newcommand{\Lambdadot}[0]{{\dot{\Lambda}}}
\newcommand{\Omegadot}[0]{{\dot{\Omega}}}
\newcommand{\Phidot}[0]{{\dot{\Phi}}}
\newcommand{\Pidot}[0]{{\dot{\Pi}}}
\newcommand{\Psidot}[0]{{\dot{\Psi}}}
\newcommand{\Sigmadot}[0]{{\dot{\Sigma}}}
\newcommand{\Thetadot}[0]{{\dot{\Theta}}}
\newcommand{\Upsilondot}[0]{{\dot{\Upsilon}}}
\newcommand{\Xidot}[0]{{\dot{\Xi}}}
\newcommand{\alphadot}[0]{{\dot{\alpha}}}
\newcommand{\betadot}[0]{{\dot{\beta}}}
\newcommand{\chidot}[0]{{\dot{\chi}}}
\newcommand{\deltadot}[0]{{\dot{\delta}}}
\newcommand{\epsilondot}[0]{{\dot{\epsilon}}}
\newcommand{\etadot}[0]{{\dot{\eta}}}
\newcommand{\gammadot}[0]{{\dot{\gamma}}}
\newcommand{\kappadot}[0]{{\dot{\kappa}}}
\newcommand{\lambdadot}[0]{{\dot{\lambda}}}
\newcommand{\mudot}[0]{{\dot{\mu}}}
\newcommand{\nudot}[0]{{\dot{\nu}}}
\newcommand{\omegadot}[0]{{\dot{\omega}}}
\newcommand{\phidot}[0]{{\dot{\phi}}}
\newcommand{\pidot}[0]{{\dot{\pi}}}
\newcommand{\psidot}[0]{{\dot{\psi}}}
\newcommand{\rhodot}[0]{{\dot{\rho}}}
\newcommand{\sigmadot}[0]{{\dot{\sigma}}}
\newcommand{\taudot}[0]{{\dot{\tau}}}
\newcommand{\thetadot}[0]{{\dot{\theta}}}
\newcommand{\upsilondot}[0]{{\dot{\upsilon}}}
\newcommand{\varepsilondot}[0]{{\dot{\varepsilon}}}
\newcommand{\varphidot}[0]{{\dot{\varphi}}}
\newcommand{\varpidot}[0]{{\dot{\varpi}}}
\newcommand{\varrhodot}[0]{{\dot{\varrho}}}
\newcommand{\varsigmadot}[0]{{\dot{\varsigma}}}
\newcommand{\varthetadot}[0]{{\dot{\vartheta}}}
\newcommand{\xidot}[0]{{\dot{\xi}}}
\newcommand{\zetadot}[0]{{\dot{\zeta}}}

\newcommand{\Deltaddot}[0]{{\ddot{\Delta}}}
\newcommand{\Gammaddot}[0]{{\ddot{\Gamma}}}
\newcommand{\Lambdaddot}[0]{{\ddot{\Lambda}}}
\newcommand{\Omegaddot}[0]{{\ddot{\Omega}}}
\newcommand{\Phiddot}[0]{{\ddot{\Phi}}}
\newcommand{\Piddot}[0]{{\ddot{\Pi}}}
\newcommand{\Psiddot}[0]{{\ddot{\Psi}}}
\newcommand{\Sigmaddot}[0]{{\ddot{\Sigma}}}
\newcommand{\Thetaddot}[0]{{\ddot{\Theta}}}
\newcommand{\Upsilonddot}[0]{{\ddot{\Upsilon}}}
\newcommand{\Xiddot}[0]{{\ddot{\Xi}}}
\newcommand{\alphaddot}[0]{{\ddot{\alpha}}}
\newcommand{\betaddot}[0]{{\ddot{\beta}}}
\newcommand{\chiddot}[0]{{\ddot{\chi}}}
\newcommand{\deltaddot}[0]{{\ddot{\delta}}}
\newcommand{\epsilonddot}[0]{{\ddot{\epsilon}}}
\newcommand{\etaddot}[0]{{\ddot{\eta}}}
\newcommand{\gammaddot}[0]{{\ddot{\gamma}}}
\newcommand{\kappaddot}[0]{{\ddot{\kappa}}}
\newcommand{\lambdaddot}[0]{{\ddot{\lambda}}}
\newcommand{\muddot}[0]{{\ddot{\mu}}}
\newcommand{\nuddot}[0]{{\ddot{\nu}}}
\newcommand{\omegaddot}[0]{{\ddot{\omega}}}
\newcommand{\phiddot}[0]{{\ddot{\phi}}}
\newcommand{\piddot}[0]{{\ddot{\pi}}}
\newcommand{\psiddot}[0]{{\ddot{\psi}}}
\newcommand{\rhoddot}[0]{{\ddot{\rho}}}
\newcommand{\sigmaddot}[0]{{\ddot{\sigma}}}
\newcommand{\tauddot}[0]{{\ddot{\tau}}}
\newcommand{\thetaddot}[0]{{\ddot{\theta}}}
\newcommand{\upsilonddot}[0]{{\ddot{\upsilon}}}
\newcommand{\varepsilonddot}[0]{{\ddot{\varepsilon}}}
\newcommand{\varphiddot}[0]{{\ddot{\varphi}}}
\newcommand{\varpiddot}[0]{{\ddot{\varpi}}}
\newcommand{\varrhoddot}[0]{{\ddot{\varrho}}}
\newcommand{\varsigmaddot}[0]{{\ddot{\varsigma}}}
\newcommand{\varthetaddot}[0]{{\ddot{\vartheta}}}
\newcommand{\xiddot}[0]{{\ddot{\xi}}}
\newcommand{\zetaddot}[0]{{\ddot{\zeta}}}

\newcommand{\BDelta}[0]{\boldsymbol{\Delta}}
\newcommand{\BGamma}[0]{\boldsymbol{\Gamma}}
\newcommand{\BLambda}[0]{\boldsymbol{\Lambda}}
\newcommand{\BOmega}[0]{\boldsymbol{\Omega}}
\newcommand{\BPhi}[0]{\boldsymbol{\Phi}}
\newcommand{\BPi}[0]{\boldsymbol{\Pi}}
\newcommand{\BPsi}[0]{\boldsymbol{\Psi}}
\newcommand{\BSigma}[0]{\boldsymbol{\Sigma}}
\newcommand{\BTheta}[0]{\boldsymbol{\Theta}}
\newcommand{\BUpsilon}[0]{\boldsymbol{\Upsilon}}
\newcommand{\BXi}[0]{\boldsymbol{\Xi}}
\newcommand{\Balpha}[0]{\boldsymbol{\alpha}}
\newcommand{\Bbeta}[0]{\boldsymbol{\beta}}
\newcommand{\Bchi}[0]{\boldsymbol{\chi}}
\newcommand{\Bdelta}[0]{\boldsymbol{\delta}}
\newcommand{\Bepsilon}[0]{\boldsymbol{\epsilon}}
\newcommand{\Beta}[0]{\boldsymbol{\eta}}
\newcommand{\Bgamma}[0]{\boldsymbol{\gamma}}
\newcommand{\Bkappa}[0]{\boldsymbol{\kappa}}
\newcommand{\Blambda}[0]{\boldsymbol{\lambda}}
\newcommand{\Bmu}[0]{\boldsymbol{\mu}}
\newcommand{\Bnu}[0]{\boldsymbol{\nu}}
%\newcommand{\Bomega}[0]{\boldsymbol{\omega}}
\newcommand{\Bphi}[0]{\boldsymbol{\phi}}
\newcommand{\Bpi}[0]{\boldsymbol{\pi}}
\newcommand{\Bpsi}[0]{\boldsymbol{\psi}}
\newcommand{\Brho}[0]{\boldsymbol{\rho}}
\newcommand{\Bsigma}[0]{\boldsymbol{\sigma}}
%\newcommand{\Btau}[0]{\boldsymbol{\tau}}
%\newcommand{\Btheta}[0]{\boldsymbol{\theta}}
\newcommand{\Bupsilon}[0]{\boldsymbol{\upsilon}}
\newcommand{\Bvarepsilon}[0]{\boldsymbol{\varepsilon}}
\newcommand{\Bvarphi}[0]{\boldsymbol{\varphi}}
\newcommand{\Bvarpi}[0]{\boldsymbol{\varpi}}
\newcommand{\Bvarrho}[0]{\boldsymbol{\varrho}}
\newcommand{\Bvarsigma}[0]{\boldsymbol{\varsigma}}
\newcommand{\Bvartheta}[0]{\boldsymbol{\vartheta}}
\newcommand{\Bxi}[0]{\boldsymbol{\xi}}
\newcommand{\Bzeta}[0]{\boldsymbol{\zeta}}
%
%</bold and dot greek symbols>
%<infrequent>
%
%\newcommand{\AreaOp}[1]{\AName_{#1}}
%\newcommand{\Babs}[0]{\abs{\BB}}
%\newcommand{\Bcap}[0]{\hat{\BB}}
%\newcommand{\BrPrimeRej}[0]{\rcap(\rcap \wedge \Br')}
%\newcommand{\CA}[0]{\mathcal{A}}
%\newcommand{\Cos}[1]{\cos{\left({#1}\right)}}
%\newcommand{\Det}[1] {\abs{#1}}
%\newcommand{\Dsq}[2] {\frac {\partial^2 {#1}} {\partial {#2}^2}}
%\newcommand{\Exp}[1]{\exp{\left({#1}\right)}}
%\newcommand{\Norm}[1]{\left\lVert{#1}\right\rVert}
%\newcommand{\Sin}[1]{\sin{\left({#1}\right)}}
%\newcommand{\T}[0]{\text{T}}
%\newcommand{\VolumeOp}[1]{\VName_{#1}}
%\newcommand{\agrad}[0]{\Ba \cdot \nabla}
%\newcommand{\alphacap}[0]{\hat{\boldsymbol{\alpha}}}
%\newcommand{\Fcap}[0]{\hat{\BF}}
%\newcommand{\bithree}[0]{{\Bi}_3}
%\newcommand{\bxa}[0]{\Bx\Ba}
%\newcommand{\coordvec}[2]{
%\newcommand{\costheta}[0]{\acap \cdot \xcap}
%\newcommand{\ddt}[1]{\ddot{#1}}
%\newcommand{\ddu}[1] {\frac {d{#1}} {du}}
%\newcommand{\dsqxj}[2] {\frac {\partial^2 {#1}} {\partial {x_{#2}}^2}}
%\newcommand{\dtheta}[1]{\frac{d {#1}}{d \theta}}
%\newcommand{\dt}[1]{\dot{#1}}
%\newcommand{\dt}[1]{\frac{d {#1}}{dt}}
%\newcommand{\dxj}[2] {\frac {\partial {#1}} {\partial {x_{#2}}}}
%\newcommand{\halfPhi}[0]{\frac{\phi}{2}}
%\newcommand{\half}[0]{\inv{2}}
%\newcommand{\inv}[1]{\frac{1}{#1}}
%\newcommand{\laplacian}[0]{\nabla^2}
%\newcommand{\matrixoftx}[3]{
%\newcommand{\nrrp}[0]{\norm{\rcap \wedge \Br'}}
%\newcommand{\oiint}{\bigcirc \hspace{-1.4em} \int \hspace{-.8em} \int}
%\newcommand{\transpose}[1]{{#1}^{\text{T}}}
%\newcommand{\transpose}[1]{{{#1}^{\TextTranspose}}}
%\newcommand{\transpose}[1]{{{#1}^{\text{T}}}}
%\newcommand{\barA}[0]{\bar{A}}
%\newcommand{\qbar}[0]{\bar{q}}
%\newcommand{\qdotbar}[0]{\dot{\bar{q}}}
%
%</infrequent>





%\usepackage[bookmarks=true]{hyperref}

\chapter{Some notes on DeBroglie paper.}
\label{chap:debroglie}
%\author{Peeter Joot \quad peeter.joot@gmail.com}
\date{ Oct. 25, 2008.  debroglie.tex }

%\begin{document}

%\maketitle{}
%\tableofcontents

\section{Motivation. }

The translation of the DeBroglie thesis \citep{AFkracklauerDeBroglie}
appears to have a quite readable introduction to many relativity and 
quantum phenomena.  Here I collect additional notes on things that were
not clear to me.

\section{Chapter 2. }

\subsection{Equation 2.2.10}

Let 

\begin{align*}
\LL = k = g_{ij} \qdot^i \qdot^j
\end{align*}
\begin{align*}
\PD{\qdot^k}{\LL} = 2 g_{ik} \qdot^i \\
\PD{\qdot^k}{\LL} \qdot^k = 2 g_{ik} \qdot^i \qdot^k = 2 K
\end{align*}


\subsection{Equation 2.3.3}

A worldline velocity with respect to some parametrization is

\begin{align*}
\left(\frac{ds}{d\lambda}\right)^2 = \left( \frac{dx}{d\lambda} \cdot \frac{dx}{d\lambda} \right)^2
&= \left(\frac{dx^4}{d\lambda}\right)^2 - \sum_i \left(\frac{dx^i}{d\lambda}\right)^2
\end{align*}

For $\lambda = s$, we can therefore calculate $u^4$:

\begin{align*}
\left(\frac{ds}{ds}\right)^2 = 1 
&= \left(\frac{dx^4}{ds}\right)^2 - \sum_i \left(\frac{dx^i}{ds}\right)^2 \\
&= \left(\frac{dx^4}{ds}\right)^2 - \sum_i \left( \frac{dx^i}{dx^4} \frac{dx^4}{ds} \right)^2 \\
&= \left(\frac{dx^4}{ds}\right)^2 \left( 1 - \sum_i \left( \frac{dx^i}{dx^4} \right)^2 \right) \\
\end{align*}

Or
\begin{align}\label{eqn:debroglie:timearc}
\left(\frac{dx^4}{ds}\right)^2 = \inv{1 - \Bv^2/c^2 } \\
\end{align}

There is a freedom to pick either plus or minus here.  Returning to that later, first 
calculate the remainder of this table of derivatives.  Picking $x^1$ as representative

\begin{align*}
1 &= \inv{1 - \Bv^2/c^2 } 
- \left(\frac{dx^1}{ds}\right)^2
- \left( \frac{dy}{dx^4} \frac{dx^4}{ds} \right)^2
- \left( \frac{dz}{dx^4} \frac{dx^4}{ds} \right)^2 \\
\left(\frac{dx^1}{ds}\right)^2
&= \frac{\Bv^2/c^2 }{1 - \Bv^2/c^2 } 
-\inv{1 - \Bv^2/c^2 } \left( \left( \frac{dy}{dx^4} \right)^2 + \left( \frac{dz}{dx^4} \right)^2 \right) \\
&= \inv{c^2 (1 - \Bv^2/c^2) } \left( \Bv^2 - \left( \frac{dy}{dt} \right)^2 - \left( \frac{dz}{dt} \right)^2 \right) \\
&= \frac{v_x^2/c^2}{1 - \Bv^2/c^2}
\end{align*}

Now, express the coordinate vector for the worldline differential in its entirety:

\begin{align*}
\frac{dx}{ds} =
\frac{d}{ds}(x^1, x^2, x^3, x^4)
&= \frac{dx^4}{ds} \left( \frac{dx^1}{dx^4}, \frac{dx^2}{dx^4}, \frac{dx^3}{dx^4}, 1 \right) \\
&= \frac{dx^4}{ds} ( v_x/c, v_y/c, v_z/c, 1) \\
\end{align*}

This shows that the flexibility to choose a sign for the square roots to obtain $dx^\mu/ds$ must all match the sign for the $dx^4/ds$ term.  Considering a particle at rest in the implied frame associated with these coordinates, one has, by \ref{eqn:debroglie:timearc}

\begin{align*}
\frac{dx}{ds} 
&= \pm \left(0, 0, 0, \inv{\sqrt{1 - \Bv^2/c^2}} \right) \\
&= \pm \left(0, 0, 0, 1 \right) \\
\end{align*}

If we take positive $ds$ to measure increase of time in the rest frame, then there is some sense to picking the positive root.  One
wouldn't have to, since there is also a corresponding freedom to bury a sign adjustment in the $dx_\mu/ds$ derivatives.

%\bibliographystyle{plainnat}
%\bibliography{myrefs}

%\end{document}
           % Oct 25/08
%
% Copyright � 2012 Peeter Joot.  All Rights Reserved.
% Licenced as described in the file LICENSE under the root directory of this GIT repository.
%

% 
% 
%\documentclass{article}

%\usepackage{amsmath}
\usepackage{mathpazo}

%
% shorthand for bold symbols, convenient for vectors and matrices
%
\newcommand{\Ba}[0]{\mathbf{a}}
\newcommand{\Bb}[0]{\mathbf{b}}
\newcommand{\Bc}[0]{\mathbf{c}}
\newcommand{\Bd}[0]{\mathbf{d}}
\newcommand{\Be}[0]{\mathbf{e}}
\newcommand{\Bf}[0]{\mathbf{f}}
\newcommand{\Bg}[0]{\mathbf{g}}
\newcommand{\Bh}[0]{\mathbf{h}}
\newcommand{\Bi}[0]{\mathbf{i}}
\newcommand{\Bj}[0]{\mathbf{j}}
\newcommand{\Bk}[0]{\mathbf{k}}
\newcommand{\Bl}[0]{\mathbf{l}}
\newcommand{\Bm}[0]{\mathbf{m}}
\newcommand{\Bn}[0]{\mathbf{n}}
\newcommand{\Bo}[0]{\mathbf{o}}
\newcommand{\Bp}[0]{\mathbf{p}}
\newcommand{\Bq}[0]{\mathbf{q}}
\newcommand{\Br}[0]{\mathbf{r}}
\newcommand{\Bs}[0]{\mathbf{s}}
\newcommand{\Bt}[0]{\mathbf{t}}
\newcommand{\Bu}[0]{\mathbf{u}}
\newcommand{\Bv}[0]{\mathbf{v}}
\newcommand{\Bw}[0]{\mathbf{w}}
\newcommand{\Bx}[0]{\mathbf{x}}
\newcommand{\By}[0]{\mathbf{y}}
\newcommand{\Bz}[0]{\mathbf{z}}
\newcommand{\BA}[0]{\mathbf{A}}
\newcommand{\BB}[0]{\mathbf{B}}
\newcommand{\BC}[0]{\mathbf{C}}
\newcommand{\BD}[0]{\mathbf{D}}
\newcommand{\BE}[0]{\mathbf{E}}
\newcommand{\BF}[0]{\mathbf{F}}
\newcommand{\BG}[0]{\mathbf{G}}
\newcommand{\BH}[0]{\mathbf{H}}
\newcommand{\BI}[0]{\mathbf{I}}
\newcommand{\BJ}[0]{\mathbf{J}}
\newcommand{\BK}[0]{\mathbf{K}}
\newcommand{\BL}[0]{\mathbf{L}}
\newcommand{\BM}[0]{\mathbf{M}}
\newcommand{\BN}[0]{\mathbf{N}}
\newcommand{\BO}[0]{\mathbf{O}}
\newcommand{\BP}[0]{\mathbf{P}}
\newcommand{\BQ}[0]{\mathbf{Q}}
\newcommand{\BR}[0]{\mathbf{R}}
\newcommand{\BS}[0]{\mathbf{S}}
\newcommand{\BT}[0]{\mathbf{T}}
\newcommand{\BU}[0]{\mathbf{U}}
\newcommand{\BV}[0]{\mathbf{V}}
\newcommand{\BW}[0]{\mathbf{W}}
\newcommand{\BX}[0]{\mathbf{X}}
\newcommand{\BY}[0]{\mathbf{Y}}
\newcommand{\BZ}[0]{\mathbf{Z}}

\newcommand{\Bzero}[0]{\mathbf{0}}
\newcommand{\Btheta}[0]{\boldsymbol{\theta}}
\newcommand{\Btau}[0]{\boldsymbol{\tau}}
\newcommand{\Bomega}[0]{\boldsymbol{\omega}}

%
% shorthand for unit vectors
%
\newcommand{\acap}[0]{\hat{\Ba}}
\newcommand{\bcap}[0]{\hat{\Bb}}
\newcommand{\ccap}[0]{\hat{\Bc}}
\newcommand{\dcap}[0]{\hat{\Bd}}
\newcommand{\ecap}[0]{\hat{\Be}}
\newcommand{\fcap}[0]{\hat{\Bf}}
\newcommand{\gcap}[0]{\hat{\Bg}}
\newcommand{\hcap}[0]{\hat{\Bh}}
\newcommand{\icap}[0]{\hat{\Bi}}
\newcommand{\jcap}[0]{\hat{\Bj}}
\newcommand{\kcap}[0]{\hat{\Bk}}
\newcommand{\lcap}[0]{\hat{\Bl}}
\newcommand{\mcap}[0]{\hat{\Bm}}
\newcommand{\ncap}[0]{\hat{\Bn}}
\newcommand{\ocap}[0]{\hat{\Bo}}
\newcommand{\pcap}[0]{\hat{\Bp}}
\newcommand{\qcap}[0]{\hat{\Bq}}
\newcommand{\rcap}[0]{\hat{\Br}}
\newcommand{\scap}[0]{\hat{\Bs}}
\newcommand{\tcap}[0]{\hat{\Bt}}
\newcommand{\ucap}[0]{\hat{\Bu}}
\newcommand{\vcap}[0]{\hat{\Bv}}
\newcommand{\wcap}[0]{\hat{\Bw}}
\newcommand{\xcap}[0]{\hat{\Bx}}
\newcommand{\ycap}[0]{\hat{\By}}
\newcommand{\zcap}[0]{\hat{\Bz}}
\newcommand{\thetacap}[0]{\hat{\Btheta}}

%
% to write R^n and C^n in a distinguishable fashion.  Perhaps change this
% to the double lined characters upon figuring out how to do so.
%
\newcommand{\C}[1]{$\mathbb{C}^{#1}$}
\newcommand{\R}[1]{$\mathbb{R}^{#1}$}

%
% various generally useful helpers
%

% derivative of #1 wrt. #2:
\newcommand{\D}[2] {\frac {d#2} {d#1}}

\newcommand{\inv}[1]{\frac{1}{#1}}
\newcommand{\cross}[0]{\times}

\newcommand{\abs}[1]{\lvert{#1}\rvert}
\newcommand{\norm}[1]{\lVert{#1}\rVert}
\newcommand{\innerprod}[2]{\langle{#1}, {#2}\rangle}
\newcommand{\dotprod}[2]{{#1} \cdot {#2}}
\newcommand{\bdotprod}[2]{\left({#1} \cdot {#2}\right)}
\newcommand{\crossprod}[2]{{#1} \cross {#2}}
\newcommand{\tripleprod}[3]{\dotprod{\left(\crossprod{#1}{#2}\right)}{#3}}

\DeclareMathOperator{\Proj}{Proj}
\DeclareMathOperator{\Span}{span}
\DeclareMathOperator{\Sgn}{sgn}
\DeclareMathOperator{\Area}{Area}
\DeclareMathOperator{\Volume}{Volume}

%
% A few miscellaneous things specific to this document
%
\newcommand{\crossop}[1]{\crossprod{#1}{}}

% R2 vector.
\newcommand{\VectorTwo}[2]{
\begin{bmatrix}
 {#1} \\
 {#2}
\end{bmatrix}
}

\newcommand{\VectorN}[1]{
\begin{bmatrix}
{#1}_1 \\
{#1}_2 \\
\vdots \\
{#1}_N \\
\end{bmatrix}
}

\newcommand{\DETuvij}[4]{
\begin{vmatrix}
 {#1}_{#3} & {#1}_{#4} \\
 {#2}_{#3} & {#2}_{#4}
\end{vmatrix}
}

\newcommand{\DETuvwijk}[6]{
\begin{vmatrix}
 {#1}_{#4} & {#1}_{#5} & {#1}_{#6} \\
 {#2}_{#4} & {#2}_{#5} & {#2}_{#6} \\
 {#3}_{#4} & {#3}_{#5} & {#3}_{#6}
\end{vmatrix}
}

\newcommand{\DETuvwxijkl}[8]{
\begin{vmatrix}
 {#1}_{#5} & {#1}_{#6} & {#1}_{#7} & {#1}_{#8} \\
 {#2}_{#5} & {#2}_{#6} & {#2}_{#7} & {#2}_{#8} \\
 {#3}_{#5} & {#3}_{#6} & {#3}_{#7} & {#3}_{#8} \\
 {#4}_{#5} & {#4}_{#6} & {#4}_{#7} & {#4}_{#8} \\
\end{vmatrix}
}

%\newcommand{\DETuvwxyijklm}[10]{
%\begin{vmatrix}
% {#1}_{#6} & {#1}_{#7} & {#1}_{#8} & {#1}_{#9} & {#1}_{#10} \\
% {#2}_{#6} & {#2}_{#7} & {#2}_{#8} & {#2}_{#9} & {#2}_{#10} \\
% {#3}_{#6} & {#3}_{#7} & {#3}_{#8} & {#3}_{#9} & {#3}_{#10} \\
% {#4}_{#6} & {#4}_{#7} & {#4}_{#8} & {#4}_{#9} & {#4}_{#10} \\
% {#5}_{#6} & {#5}_{#7} & {#5}_{#8} & {#5}_{#9} & {#5}_{#10}
%\end{vmatrix}
%}

% R3 vector.
\newcommand{\VectorThree}[3]{
\begin{bmatrix}
 {#1} \\
 {#2} \\
 {#3}
\end{bmatrix}
}


%%<misc>
%
\newcommand{\Abs}[1]{{\left\lvert{#1}\right\rvert}}
\newcommand{\spacegrad}[0]{\boldsymbol{\nabla}}
\newcommand{\grad}[0]{\nabla}
\newcommand{\LL}[0]{\mathcal{L}}

% == \partial_{#1} {#2}
\newcommand{\PD}[2]{\frac{\partial {#2}}{\partial {#1}}}
% inline variant
\newcommand{\PDi}[2]{{\partial {#2}}/{\partial {#1}}}

\newcommand{\PDD}[3]{\frac{\partial^2 {#3}}{\partial {#1}\partial {#2}}}
%\newcommand{\PDd}[2]{\frac{\partial^2 {#2}}{{\partial{#1}}^2}}
\newcommand{\PDsq}[2]{\frac{\partial^2 {#2}}{(\partial {#1})^2}}

\newcommand{\Partial}[2]{\frac{\partial {#1}}{\partial {#2}}}
\DeclareMathOperator{\RejName}{Rej}
\newcommand{\Rej}[2]{\RejName_{#1}\left( {#2} \right)}
\newcommand{\Rm}[1]{\mathbb{R}^{#1}}
\newcommand{\Cm}[1]{\mathbb{C}^{#1}}
\newcommand{\conj}[0]{{*}}

%</misc>

% <grade selection>
%
\newcommand{\gpgrade}[2] {{\left\langle{{#1}}\right\rangle}_{#2}}

\newcommand{\gpgradezero}[1] {\gpgrade{#1}{}}
%\newcommand{\gpscalargrade}[1] {{\left\langle{{#1}}\right\rangle}}
%\newcommand{\gpgradezero}[1] {\gpgrade{#1}{0}}

%\newcommand{\gpgradeone}[1] {{\left\langle{{#1}}\right\rangle}_{1}}
\newcommand{\gpgradeone}[1] {\gpgrade{#1}{1}}

\newcommand{\gpgradetwo}[1] {\gpgrade{#1}{2}}
\newcommand{\gpgradethree}[1] {\gpgrade{#1}{3}}
\newcommand{\gpgradefour}[1] {\gpgrade{#1}{4}}
%
% </grade selection>



\newcommand{\adot}[0]{{\dot{a}}}
\newcommand{\bdot}[0]{{\dot{b}}}
% taken for centered dot:
%\newcommand{\cdot}[0]{{\dot{c}}}
%\newcommand{\ddot}[0]{{\dot{d}}}
\newcommand{\edot}[0]{{\dot{e}}}
\newcommand{\fdot}[0]{{\dot{f}}}
\newcommand{\gdot}[0]{{\dot{g}}}
\newcommand{\hdot}[0]{{\dot{h}}}
\newcommand{\idot}[0]{{\dot{i}}}
\newcommand{\jdot}[0]{{\dot{j}}}
\newcommand{\kdot}[0]{{\dot{k}}}
\newcommand{\ldot}[0]{{\dot{l}}}
\newcommand{\mdot}[0]{{\dot{m}}}
\newcommand{\ndot}[0]{{\dot{n}}}
%\newcommand{\odot}[0]{{\dot{o}}}
\newcommand{\pdot}[0]{{\dot{p}}}
\newcommand{\qdot}[0]{{\dot{q}}}
\newcommand{\rdot}[0]{{\dot{r}}}
\newcommand{\sdot}[0]{{\dot{s}}}
\newcommand{\tdot}[0]{{\dot{t}}}
\newcommand{\udot}[0]{{\dot{u}}}
\newcommand{\vdot}[0]{{\dot{v}}}
\newcommand{\wdot}[0]{{\dot{w}}}
\newcommand{\xdot}[0]{{\dot{x}}}
\newcommand{\ydot}[0]{{\dot{y}}}
\newcommand{\zdot}[0]{{\dot{z}}}
\newcommand{\addot}[0]{{\ddot{a}}}
\newcommand{\bddot}[0]{{\ddot{b}}}
\newcommand{\cddot}[0]{{\ddot{c}}}
%\newcommand{\dddot}[0]{{\ddot{d}}}
\newcommand{\eddot}[0]{{\ddot{e}}}
\newcommand{\fddot}[0]{{\ddot{f}}}
\newcommand{\gddot}[0]{{\ddot{g}}}
\newcommand{\hddot}[0]{{\ddot{h}}}
\newcommand{\iddot}[0]{{\ddot{i}}}
\newcommand{\jddot}[0]{{\ddot{j}}}
\newcommand{\kddot}[0]{{\ddot{k}}}
\newcommand{\lddot}[0]{{\ddot{l}}}
\newcommand{\mddot}[0]{{\ddot{m}}}
\newcommand{\nddot}[0]{{\ddot{n}}}
\newcommand{\oddot}[0]{{\ddot{o}}}
\newcommand{\pddot}[0]{{\ddot{p}}}
\newcommand{\qddot}[0]{{\ddot{q}}}
\newcommand{\rddot}[0]{{\ddot{r}}}
\newcommand{\sddot}[0]{{\ddot{s}}}
\newcommand{\tddot}[0]{{\ddot{t}}}
\newcommand{\uddot}[0]{{\ddot{u}}}
\newcommand{\vddot}[0]{{\ddot{v}}}
\newcommand{\wddot}[0]{{\ddot{w}}}
\newcommand{\xddot}[0]{{\ddot{x}}}
\newcommand{\yddot}[0]{{\ddot{y}}}
\newcommand{\zddot}[0]{{\ddot{z}}}

%<bold and dot greek symbols>
%

\newcommand{\Deltadot}[0]{{\dot{\Delta}}}
\newcommand{\Gammadot}[0]{{\dot{\Gamma}}}
\newcommand{\Lambdadot}[0]{{\dot{\Lambda}}}
\newcommand{\Omegadot}[0]{{\dot{\Omega}}}
\newcommand{\Phidot}[0]{{\dot{\Phi}}}
\newcommand{\Pidot}[0]{{\dot{\Pi}}}
\newcommand{\Psidot}[0]{{\dot{\Psi}}}
\newcommand{\Sigmadot}[0]{{\dot{\Sigma}}}
\newcommand{\Thetadot}[0]{{\dot{\Theta}}}
\newcommand{\Upsilondot}[0]{{\dot{\Upsilon}}}
\newcommand{\Xidot}[0]{{\dot{\Xi}}}
\newcommand{\alphadot}[0]{{\dot{\alpha}}}
\newcommand{\betadot}[0]{{\dot{\beta}}}
\newcommand{\chidot}[0]{{\dot{\chi}}}
\newcommand{\deltadot}[0]{{\dot{\delta}}}
\newcommand{\epsilondot}[0]{{\dot{\epsilon}}}
\newcommand{\etadot}[0]{{\dot{\eta}}}
\newcommand{\gammadot}[0]{{\dot{\gamma}}}
\newcommand{\kappadot}[0]{{\dot{\kappa}}}
\newcommand{\lambdadot}[0]{{\dot{\lambda}}}
\newcommand{\mudot}[0]{{\dot{\mu}}}
\newcommand{\nudot}[0]{{\dot{\nu}}}
\newcommand{\omegadot}[0]{{\dot{\omega}}}
\newcommand{\phidot}[0]{{\dot{\phi}}}
\newcommand{\pidot}[0]{{\dot{\pi}}}
\newcommand{\psidot}[0]{{\dot{\psi}}}
\newcommand{\rhodot}[0]{{\dot{\rho}}}
\newcommand{\sigmadot}[0]{{\dot{\sigma}}}
\newcommand{\taudot}[0]{{\dot{\tau}}}
\newcommand{\thetadot}[0]{{\dot{\theta}}}
\newcommand{\upsilondot}[0]{{\dot{\upsilon}}}
\newcommand{\varepsilondot}[0]{{\dot{\varepsilon}}}
\newcommand{\varphidot}[0]{{\dot{\varphi}}}
\newcommand{\varpidot}[0]{{\dot{\varpi}}}
\newcommand{\varrhodot}[0]{{\dot{\varrho}}}
\newcommand{\varsigmadot}[0]{{\dot{\varsigma}}}
\newcommand{\varthetadot}[0]{{\dot{\vartheta}}}
\newcommand{\xidot}[0]{{\dot{\xi}}}
\newcommand{\zetadot}[0]{{\dot{\zeta}}}

\newcommand{\Deltaddot}[0]{{\ddot{\Delta}}}
\newcommand{\Gammaddot}[0]{{\ddot{\Gamma}}}
\newcommand{\Lambdaddot}[0]{{\ddot{\Lambda}}}
\newcommand{\Omegaddot}[0]{{\ddot{\Omega}}}
\newcommand{\Phiddot}[0]{{\ddot{\Phi}}}
\newcommand{\Piddot}[0]{{\ddot{\Pi}}}
\newcommand{\Psiddot}[0]{{\ddot{\Psi}}}
\newcommand{\Sigmaddot}[0]{{\ddot{\Sigma}}}
\newcommand{\Thetaddot}[0]{{\ddot{\Theta}}}
\newcommand{\Upsilonddot}[0]{{\ddot{\Upsilon}}}
\newcommand{\Xiddot}[0]{{\ddot{\Xi}}}
\newcommand{\alphaddot}[0]{{\ddot{\alpha}}}
\newcommand{\betaddot}[0]{{\ddot{\beta}}}
\newcommand{\chiddot}[0]{{\ddot{\chi}}}
\newcommand{\deltaddot}[0]{{\ddot{\delta}}}
\newcommand{\epsilonddot}[0]{{\ddot{\epsilon}}}
\newcommand{\etaddot}[0]{{\ddot{\eta}}}
\newcommand{\gammaddot}[0]{{\ddot{\gamma}}}
\newcommand{\kappaddot}[0]{{\ddot{\kappa}}}
\newcommand{\lambdaddot}[0]{{\ddot{\lambda}}}
\newcommand{\muddot}[0]{{\ddot{\mu}}}
\newcommand{\nuddot}[0]{{\ddot{\nu}}}
\newcommand{\omegaddot}[0]{{\ddot{\omega}}}
\newcommand{\phiddot}[0]{{\ddot{\phi}}}
\newcommand{\piddot}[0]{{\ddot{\pi}}}
\newcommand{\psiddot}[0]{{\ddot{\psi}}}
\newcommand{\rhoddot}[0]{{\ddot{\rho}}}
\newcommand{\sigmaddot}[0]{{\ddot{\sigma}}}
\newcommand{\tauddot}[0]{{\ddot{\tau}}}
\newcommand{\thetaddot}[0]{{\ddot{\theta}}}
\newcommand{\upsilonddot}[0]{{\ddot{\upsilon}}}
\newcommand{\varepsilonddot}[0]{{\ddot{\varepsilon}}}
\newcommand{\varphiddot}[0]{{\ddot{\varphi}}}
\newcommand{\varpiddot}[0]{{\ddot{\varpi}}}
\newcommand{\varrhoddot}[0]{{\ddot{\varrho}}}
\newcommand{\varsigmaddot}[0]{{\ddot{\varsigma}}}
\newcommand{\varthetaddot}[0]{{\ddot{\vartheta}}}
\newcommand{\xiddot}[0]{{\ddot{\xi}}}
\newcommand{\zetaddot}[0]{{\ddot{\zeta}}}

\newcommand{\BDelta}[0]{\boldsymbol{\Delta}}
\newcommand{\BGamma}[0]{\boldsymbol{\Gamma}}
\newcommand{\BLambda}[0]{\boldsymbol{\Lambda}}
\newcommand{\BOmega}[0]{\boldsymbol{\Omega}}
\newcommand{\BPhi}[0]{\boldsymbol{\Phi}}
\newcommand{\BPi}[0]{\boldsymbol{\Pi}}
\newcommand{\BPsi}[0]{\boldsymbol{\Psi}}
\newcommand{\BSigma}[0]{\boldsymbol{\Sigma}}
\newcommand{\BTheta}[0]{\boldsymbol{\Theta}}
\newcommand{\BUpsilon}[0]{\boldsymbol{\Upsilon}}
\newcommand{\BXi}[0]{\boldsymbol{\Xi}}
\newcommand{\Balpha}[0]{\boldsymbol{\alpha}}
\newcommand{\Bbeta}[0]{\boldsymbol{\beta}}
\newcommand{\Bchi}[0]{\boldsymbol{\chi}}
\newcommand{\Bdelta}[0]{\boldsymbol{\delta}}
\newcommand{\Bepsilon}[0]{\boldsymbol{\epsilon}}
\newcommand{\Beta}[0]{\boldsymbol{\eta}}
\newcommand{\Bgamma}[0]{\boldsymbol{\gamma}}
\newcommand{\Bkappa}[0]{\boldsymbol{\kappa}}
\newcommand{\Blambda}[0]{\boldsymbol{\lambda}}
\newcommand{\Bmu}[0]{\boldsymbol{\mu}}
\newcommand{\Bnu}[0]{\boldsymbol{\nu}}
%\newcommand{\Bomega}[0]{\boldsymbol{\omega}}
\newcommand{\Bphi}[0]{\boldsymbol{\phi}}
\newcommand{\Bpi}[0]{\boldsymbol{\pi}}
\newcommand{\Bpsi}[0]{\boldsymbol{\psi}}
\newcommand{\Brho}[0]{\boldsymbol{\rho}}
\newcommand{\Bsigma}[0]{\boldsymbol{\sigma}}
%\newcommand{\Btau}[0]{\boldsymbol{\tau}}
%\newcommand{\Btheta}[0]{\boldsymbol{\theta}}
\newcommand{\Bupsilon}[0]{\boldsymbol{\upsilon}}
\newcommand{\Bvarepsilon}[0]{\boldsymbol{\varepsilon}}
\newcommand{\Bvarphi}[0]{\boldsymbol{\varphi}}
\newcommand{\Bvarpi}[0]{\boldsymbol{\varpi}}
\newcommand{\Bvarrho}[0]{\boldsymbol{\varrho}}
\newcommand{\Bvarsigma}[0]{\boldsymbol{\varsigma}}
\newcommand{\Bvartheta}[0]{\boldsymbol{\vartheta}}
\newcommand{\Bxi}[0]{\boldsymbol{\xi}}
\newcommand{\Bzeta}[0]{\boldsymbol{\zeta}}
%
%</bold and dot greek symbols>
%<infrequent>
%
%\newcommand{\AreaOp}[1]{\AName_{#1}}
%\newcommand{\Babs}[0]{\abs{\BB}}
%\newcommand{\Bcap}[0]{\hat{\BB}}
%\newcommand{\BrPrimeRej}[0]{\rcap(\rcap \wedge \Br')}
%\newcommand{\CA}[0]{\mathcal{A}}
%\newcommand{\Cos}[1]{\cos{\left({#1}\right)}}
%\newcommand{\Det}[1] {\abs{#1}}
%\newcommand{\Dsq}[2] {\frac {\partial^2 {#1}} {\partial {#2}^2}}
%\newcommand{\Exp}[1]{\exp{\left({#1}\right)}}
%\newcommand{\Norm}[1]{\left\lVert{#1}\right\rVert}
%\newcommand{\Sin}[1]{\sin{\left({#1}\right)}}
%\newcommand{\T}[0]{\text{T}}
%\newcommand{\VolumeOp}[1]{\VName_{#1}}
%\newcommand{\agrad}[0]{\Ba \cdot \nabla}
%\newcommand{\alphacap}[0]{\hat{\boldsymbol{\alpha}}}
%\newcommand{\Fcap}[0]{\hat{\BF}}
%\newcommand{\bithree}[0]{{\Bi}_3}
%\newcommand{\bxa}[0]{\Bx\Ba}
%\newcommand{\coordvec}[2]{
%\newcommand{\costheta}[0]{\acap \cdot \xcap}
%\newcommand{\ddt}[1]{\ddot{#1}}
%\newcommand{\ddu}[1] {\frac {d{#1}} {du}}
%\newcommand{\dsqxj}[2] {\frac {\partial^2 {#1}} {\partial {x_{#2}}^2}}
%\newcommand{\dtheta}[1]{\frac{d {#1}}{d \theta}}
%\newcommand{\dt}[1]{\dot{#1}}
%\newcommand{\dt}[1]{\frac{d {#1}}{dt}}
%\newcommand{\dxj}[2] {\frac {\partial {#1}} {\partial {x_{#2}}}}
%\newcommand{\halfPhi}[0]{\frac{\phi}{2}}
%\newcommand{\half}[0]{\inv{2}}
%\newcommand{\inv}[1]{\frac{1}{#1}}
%\newcommand{\laplacian}[0]{\nabla^2}
%\newcommand{\matrixoftx}[3]{
%\newcommand{\nrrp}[0]{\norm{\rcap \wedge \Br'}}
%\newcommand{\oiint}{\bigcirc \hspace{-1.4em} \int \hspace{-.8em} \int}
%\newcommand{\transpose}[1]{{#1}^{\text{T}}}
%\newcommand{\transpose}[1]{{{#1}^{\TextTranspose}}}
%\newcommand{\transpose}[1]{{{#1}^{\text{T}}}}
%\newcommand{\barA}[0]{\bar{A}}
%\newcommand{\qbar}[0]{\bar{q}}
%\newcommand{\qdotbar}[0]{\dot{\bar{q}}}
%
%</infrequent>





%\usepackage[bookmarks=true]{hyperref}

%\usepackage{color,cite,graphicx}
   % use colour in the document, put your citations as [1-4]
   % rather than [1,2,3,4] (it looks nicer, and the extended LaTeX2e
   % graphics package. 
%\usepackage{latexsym,amssymb,epsf} % do not remember if these are
   % needed, but their inclusion can not do any damage

\chapter{Ad-hoc motivation of some QM wave equations}
\label{chap:sch}
%\author{Peeter Joot \quad peeterjoot@protonmail.com}
\date{ Dec 13, 2008.  sch.tex }

%\begin{document}

%\maketitle{}

%\tableofcontents

\section{Motivation}

\subsection{Non-Relativistic case}

A common (cf: wikipedia and \citep{french1998iqp}) introductory motivation for the non-relativistic Schr\"{o}dinger's equation appears to follow the following lines.  Assume that
we desire a plane wave equation of the following form

\begin{equation}\label{eqn:sch:20}
\begin{aligned}
\psi = \exp(i (\Bk \cdot \Bx - \omega t))
\end{aligned}
\end{equation}

plus a requirement that we have a total energy that can be expressed in terms of kinetic plus potential

\begin{equation}\label{eqn:sch:energy}
\begin{aligned}
E = \frac{\Bp^2}{2m} + V
\end{aligned}
\end{equation}

We also have the Einstein relationship

\begin{equation}\label{eqn:sch:40}
\begin{aligned}
E = h \nu = \frac{h}{2\pi} 2 \pi \nu = \Hbar \omega
\end{aligned}
\end{equation}

and the DeBroglie Hypothesis for the magnitude of the momentum of a particle

\begin{equation}\label{eqn:sch:60}
\begin{aligned}
p = \frac{h}{\lambda}.
\end{aligned}
\end{equation}

In terms of wave number \(2 \pi k = \inv{\lambda}\) this last is

\begin{equation}\label{eqn:sch:80}
\begin{aligned}
p = \frac{h}{2\pi} k = \Hbar k.
\end{aligned}
\end{equation}

or in three dimensions

\begin{equation}\label{eqn:sch:100}
\begin{aligned}
\Bp = \Hbar \Bk.
\end{aligned}
\end{equation}

Taking derivatives of the postulated wave function one has

\begin{equation}\label{eqn:sch:timepartial}
\begin{aligned}
\frac{\partial \psi}{\partial t} = -i \omega \psi
\end{aligned}
\end{equation}

and 
\begin{equation}\label{eqn:sch:laplacian}
\begin{aligned}
\spacegrad^2 \psi = i^2 \sum_j k_j^2 \psi = - \Bk^2 \psi
\end{aligned}
\end{equation}

From the energy relationship, if one requires that
\begin{equation}\label{eqn:sch:120}
\begin{aligned}
E \psi &= \left(\frac{\Bp^2}{2m} + V\right) \psi \\
\Hbar \omega \psi &= \left(\Hbar^2 \frac{\Bk^2}{2m} + V\right) \psi \\
\end{aligned}
\end{equation}

and then substituting the derivatives from equations \eqnref{eqn:sch:timepartial} and \eqnref{eqn:sch:laplacian} we have

\begin{equation}\label{eqn:sch:140}
\begin{aligned}
i \Hbar \PD{t}{\psi} &= \left(-\frac{\Hbar^2}{2m} \spacegrad^2  + V\right) \psi \\
\end{aligned}
\end{equation}

\subsection{Relativistic case}

The relativistic force free Schr\"{o}dinger's equation is motivated by \citep{srednicki2007qft} replacing the Hamiltonian operator \(H = \BP^2/{2 m}\) with

\begin{equation}\label{eqn:sch:160}
\begin{aligned}
H = \sqrt{m^2 c^4 + \BP^2 c^2} \approx m c^2 + \BP^2/2m
\end{aligned}
\end{equation}

for

\begin{equation}\label{eqn:sch:180}
\begin{aligned}
i \Hbar \PD{t}{\psi} = \sqrt{ -\Hbar^2 c^2 \spacegrad^2 + m^2 c^4} \psi
\end{aligned}
\end{equation}

then squaring the operators on both sides, removing the root:

\begin{equation}\label{eqn:sch:200}
\begin{aligned}
- \Hbar^2 \PDsq{x^0}{\psi} &= \left(-\Hbar^2 \spacegrad^2 + m^2 c^2 \right) \psi \\
- \Hbar^2 \PDsq{x^0}{\psi} + \Hbar^2 \spacegrad^2 &= m^2 c^2 \psi \\
\end{aligned}
\end{equation}

Which is called the Klein-Gordon equation:
\begin{equation}\label{eqn:sch:220}
\begin{aligned}
\left(\frac{\Hbar^2}{2m} \grad^2 + \inv{2} m c^2\right) \psi = 0 \\
\end{aligned}
\end{equation}

The Noether's theorem wikipedia article, and Klein-Gordon pages give the action
as (removing use of natural units and changing to a \(+---\) metric) gives:

\begin{equation}\label{eqn:sch:240}
\begin{aligned}
\LL &= -\eta^{\mu\nu} \partial_\mu \psi \partial_\nu \psi^\conj + \frac{m^2 c^2}{\Hbar^2} \psi \psi^\conj \\
&= -\partial^\nu \psi \partial_\nu \psi^\conj + \frac{m^2 c^2}{\Hbar^2} \psi \psi^\conj \\
\end{aligned}
\end{equation}

This first term is a squared spacetime gradient

\begin{equation}\label{eqn:sch:260}
\begin{aligned}
(\grad \psi) \cdot (\grad \psi^\conj) 
&= (\gamma^\mu \partial_\mu \psi) \cdot (\gamma_\nu \partial^\nu \psi^\conj) \\
&= {\delta^\mu}_\nu \partial_\mu \psi \partial^\nu \psi^\conj \\
&= \partial_\mu \psi \partial^\mu \psi^\conj \\
\end{aligned}
\end{equation}

so we have
\begin{equation}\label{eqn:sch:280}
\begin{aligned}
\LL 
&= -\eta^{\mu\nu} \partial_\mu \psi \partial_\nu \psi^\conj + \frac{m^2 c^2}{\Hbar^2} \psi \psi^\conj \\
&= -(\grad \psi) \cdot (\grad \psi^\conj) + \frac{m^2 c^2}{\Hbar^2} \psi \psi^\conj \\
\end{aligned}
\end{equation}

So, here we expect to get a spacetime Laplacian, like the Maxwell potential equation, and one does:

\begin{equation}\label{eqn:sch:300}
\begin{aligned}
-\grad^2 \psi = \frac{m^2 c^2}{\Hbar^2} \psi
\end{aligned}
\end{equation}

Now, the Srednicki text indicates that Dirac linearized this equation, to get back something that was
first order in the time derivative.

I do not yet follow that description, but see that a linearization is possible by taking roots of the operators above,
undoing the somewhat fishy seeming squaring done to arrive at the Klein-Gordon equation in the first place

\begin{equation}\label{eqn:sch:dirac}
\begin{aligned}
i \Hbar \grad \psi = \pm m c \psi
\end{aligned}
\end{equation}

Is this equivalent to Dirac's formulation?  Comparison to 
\href{http://en.wikipedia.org/wiki/Dirac_equation#Covariant_form_and_relativistic_invariance}, where the Dirac equation is given in covariant
form

\begin{equation}\label{eqn:sch:320}
\begin{aligned}
i \Hbar \gamma^\mu \partial_\mu \psi - m c \psi = 0
\end{aligned}
\end{equation}

which is the positive variant of \eqnref{eqn:sch:dirac}.

What is the Lagrangian that is associated with this?  The most probable interpretation of \(i\) here is the Minkowski pseudoscalar, as opposed to
a unit bivector of spacetime vectors or some other geometrical object with \(-1\) square.

%\bibliographystyle{plainnat}
%\bibliography{myrefs}

%\end{document}
                 % Dec 13/08
\documentclass{article}

\usepackage{amsmath}
\usepackage{mathpazo}

%
% shorthand for bold symbols, convenient for vectors and matrices
%
\newcommand{\Ba}[0]{\mathbf{a}}
\newcommand{\Bb}[0]{\mathbf{b}}
\newcommand{\Bc}[0]{\mathbf{c}}
\newcommand{\Bd}[0]{\mathbf{d}}
\newcommand{\Be}[0]{\mathbf{e}}
\newcommand{\Bf}[0]{\mathbf{f}}
\newcommand{\Bg}[0]{\mathbf{g}}
\newcommand{\Bh}[0]{\mathbf{h}}
\newcommand{\Bi}[0]{\mathbf{i}}
\newcommand{\Bj}[0]{\mathbf{j}}
\newcommand{\Bk}[0]{\mathbf{k}}
\newcommand{\Bl}[0]{\mathbf{l}}
\newcommand{\Bm}[0]{\mathbf{m}}
\newcommand{\Bn}[0]{\mathbf{n}}
\newcommand{\Bo}[0]{\mathbf{o}}
\newcommand{\Bp}[0]{\mathbf{p}}
\newcommand{\Bq}[0]{\mathbf{q}}
\newcommand{\Br}[0]{\mathbf{r}}
\newcommand{\Bs}[0]{\mathbf{s}}
\newcommand{\Bt}[0]{\mathbf{t}}
\newcommand{\Bu}[0]{\mathbf{u}}
\newcommand{\Bv}[0]{\mathbf{v}}
\newcommand{\Bw}[0]{\mathbf{w}}
\newcommand{\Bx}[0]{\mathbf{x}}
\newcommand{\By}[0]{\mathbf{y}}
\newcommand{\Bz}[0]{\mathbf{z}}
\newcommand{\BA}[0]{\mathbf{A}}
\newcommand{\BB}[0]{\mathbf{B}}
\newcommand{\BC}[0]{\mathbf{C}}
\newcommand{\BD}[0]{\mathbf{D}}
\newcommand{\BE}[0]{\mathbf{E}}
\newcommand{\BF}[0]{\mathbf{F}}
\newcommand{\BG}[0]{\mathbf{G}}
\newcommand{\BH}[0]{\mathbf{H}}
\newcommand{\BI}[0]{\mathbf{I}}
\newcommand{\BJ}[0]{\mathbf{J}}
\newcommand{\BK}[0]{\mathbf{K}}
\newcommand{\BL}[0]{\mathbf{L}}
\newcommand{\BM}[0]{\mathbf{M}}
\newcommand{\BN}[0]{\mathbf{N}}
\newcommand{\BO}[0]{\mathbf{O}}
\newcommand{\BP}[0]{\mathbf{P}}
\newcommand{\BQ}[0]{\mathbf{Q}}
\newcommand{\BR}[0]{\mathbf{R}}
\newcommand{\BS}[0]{\mathbf{S}}
\newcommand{\BT}[0]{\mathbf{T}}
\newcommand{\BU}[0]{\mathbf{U}}
\newcommand{\BV}[0]{\mathbf{V}}
\newcommand{\BW}[0]{\mathbf{W}}
\newcommand{\BX}[0]{\mathbf{X}}
\newcommand{\BY}[0]{\mathbf{Y}}
\newcommand{\BZ}[0]{\mathbf{Z}}

\newcommand{\Bzero}[0]{\mathbf{0}}
\newcommand{\Btheta}[0]{\boldsymbol{\theta}}
\newcommand{\Btau}[0]{\boldsymbol{\tau}}
\newcommand{\Bomega}[0]{\boldsymbol{\omega}}

%
% shorthand for unit vectors
%
\newcommand{\acap}[0]{\hat{\Ba}}
\newcommand{\bcap}[0]{\hat{\Bb}}
\newcommand{\ccap}[0]{\hat{\Bc}}
\newcommand{\dcap}[0]{\hat{\Bd}}
\newcommand{\ecap}[0]{\hat{\Be}}
\newcommand{\fcap}[0]{\hat{\Bf}}
\newcommand{\gcap}[0]{\hat{\Bg}}
\newcommand{\hcap}[0]{\hat{\Bh}}
\newcommand{\icap}[0]{\hat{\Bi}}
\newcommand{\jcap}[0]{\hat{\Bj}}
\newcommand{\kcap}[0]{\hat{\Bk}}
\newcommand{\lcap}[0]{\hat{\Bl}}
\newcommand{\mcap}[0]{\hat{\Bm}}
\newcommand{\ncap}[0]{\hat{\Bn}}
\newcommand{\ocap}[0]{\hat{\Bo}}
\newcommand{\pcap}[0]{\hat{\Bp}}
\newcommand{\qcap}[0]{\hat{\Bq}}
\newcommand{\rcap}[0]{\hat{\Br}}
\newcommand{\scap}[0]{\hat{\Bs}}
\newcommand{\tcap}[0]{\hat{\Bt}}
\newcommand{\ucap}[0]{\hat{\Bu}}
\newcommand{\vcap}[0]{\hat{\Bv}}
\newcommand{\wcap}[0]{\hat{\Bw}}
\newcommand{\xcap}[0]{\hat{\Bx}}
\newcommand{\ycap}[0]{\hat{\By}}
\newcommand{\zcap}[0]{\hat{\Bz}}
\newcommand{\thetacap}[0]{\hat{\Btheta}}

%
% to write R^n and C^n in a distinguishable fashion.  Perhaps change this
% to the double lined characters upon figuring out how to do so.
%
\newcommand{\C}[1]{$\mathbb{C}^{#1}$}
\newcommand{\R}[1]{$\mathbb{R}^{#1}$}

%
% various generally useful helpers
%

% derivative of #1 wrt. #2:
\newcommand{\D}[2] {\frac {d#2} {d#1}}

\newcommand{\inv}[1]{\frac{1}{#1}}
\newcommand{\cross}[0]{\times}

\newcommand{\abs}[1]{\lvert{#1}\rvert}
\newcommand{\norm}[1]{\lVert{#1}\rVert}
\newcommand{\innerprod}[2]{\langle{#1}, {#2}\rangle}
\newcommand{\dotprod}[2]{{#1} \cdot {#2}}
\newcommand{\bdotprod}[2]{\left({#1} \cdot {#2}\right)}
\newcommand{\crossprod}[2]{{#1} \cross {#2}}
\newcommand{\tripleprod}[3]{\dotprod{\left(\crossprod{#1}{#2}\right)}{#3}}

\DeclareMathOperator{\Proj}{Proj}
\DeclareMathOperator{\Span}{span}
\DeclareMathOperator{\Sgn}{sgn}
\DeclareMathOperator{\Area}{Area}
\DeclareMathOperator{\Volume}{Volume}

%
% A few miscellaneous things specific to this document
%
\newcommand{\crossop}[1]{\crossprod{#1}{}}

% R2 vector.
\newcommand{\VectorTwo}[2]{
\begin{bmatrix}
 {#1} \\
 {#2}
\end{bmatrix}
}

\newcommand{\VectorN}[1]{
\begin{bmatrix}
{#1}_1 \\
{#1}_2 \\
\vdots \\
{#1}_N \\
\end{bmatrix}
}

\newcommand{\DETuvij}[4]{
\begin{vmatrix}
 {#1}_{#3} & {#1}_{#4} \\
 {#2}_{#3} & {#2}_{#4}
\end{vmatrix}
}

\newcommand{\DETuvwijk}[6]{
\begin{vmatrix}
 {#1}_{#4} & {#1}_{#5} & {#1}_{#6} \\
 {#2}_{#4} & {#2}_{#5} & {#2}_{#6} \\
 {#3}_{#4} & {#3}_{#5} & {#3}_{#6}
\end{vmatrix}
}

\newcommand{\DETuvwxijkl}[8]{
\begin{vmatrix}
 {#1}_{#5} & {#1}_{#6} & {#1}_{#7} & {#1}_{#8} \\
 {#2}_{#5} & {#2}_{#6} & {#2}_{#7} & {#2}_{#8} \\
 {#3}_{#5} & {#3}_{#6} & {#3}_{#7} & {#3}_{#8} \\
 {#4}_{#5} & {#4}_{#6} & {#4}_{#7} & {#4}_{#8} \\
\end{vmatrix}
}

%\newcommand{\DETuvwxyijklm}[10]{
%\begin{vmatrix}
% {#1}_{#6} & {#1}_{#7} & {#1}_{#8} & {#1}_{#9} & {#1}_{#10} \\
% {#2}_{#6} & {#2}_{#7} & {#2}_{#8} & {#2}_{#9} & {#2}_{#10} \\
% {#3}_{#6} & {#3}_{#7} & {#3}_{#8} & {#3}_{#9} & {#3}_{#10} \\
% {#4}_{#6} & {#4}_{#7} & {#4}_{#8} & {#4}_{#9} & {#4}_{#10} \\
% {#5}_{#6} & {#5}_{#7} & {#5}_{#8} & {#5}_{#9} & {#5}_{#10}
%\end{vmatrix}
%}

% R3 vector.
\newcommand{\VectorThree}[3]{
\begin{bmatrix}
 {#1} \\
 {#2} \\
 {#3}
\end{bmatrix}
}


%<misc>
%
\newcommand{\Abs}[1]{{\left\lvert{#1}\right\rvert}}
\newcommand{\spacegrad}[0]{\boldsymbol{\nabla}}
\newcommand{\grad}[0]{\nabla}
\newcommand{\LL}[0]{\mathcal{L}}

% == \partial_{#1} {#2}
\newcommand{\PD}[2]{\frac{\partial {#2}}{\partial {#1}}}
% inline variant
\newcommand{\PDi}[2]{{\partial {#2}}/{\partial {#1}}}

\newcommand{\PDD}[3]{\frac{\partial^2 {#3}}{\partial {#1}\partial {#2}}}
%\newcommand{\PDd}[2]{\frac{\partial^2 {#2}}{{\partial{#1}}^2}}
\newcommand{\PDsq}[2]{\frac{\partial^2 {#2}}{(\partial {#1})^2}}

\newcommand{\Partial}[2]{\frac{\partial {#1}}{\partial {#2}}}
\DeclareMathOperator{\RejName}{Rej}
\newcommand{\Rej}[2]{\RejName_{#1}\left( {#2} \right)}
\newcommand{\Rm}[1]{\mathbb{R}^{#1}}
\newcommand{\Cm}[1]{\mathbb{C}^{#1}}
\newcommand{\conj}[0]{{*}}

%</misc>

% <grade selection>
%
\newcommand{\gpgrade}[2] {{\left\langle{{#1}}\right\rangle}_{#2}}

\newcommand{\gpgradezero}[1] {\gpgrade{#1}{}}
%\newcommand{\gpscalargrade}[1] {{\left\langle{{#1}}\right\rangle}}
%\newcommand{\gpgradezero}[1] {\gpgrade{#1}{0}}

%\newcommand{\gpgradeone}[1] {{\left\langle{{#1}}\right\rangle}_{1}}
\newcommand{\gpgradeone}[1] {\gpgrade{#1}{1}}

\newcommand{\gpgradetwo}[1] {\gpgrade{#1}{2}}
\newcommand{\gpgradethree}[1] {\gpgrade{#1}{3}}
\newcommand{\gpgradefour}[1] {\gpgrade{#1}{4}}
%
% </grade selection>



\newcommand{\adot}[0]{{\dot{a}}}
\newcommand{\bdot}[0]{{\dot{b}}}
% taken for centered dot:
%\newcommand{\cdot}[0]{{\dot{c}}}
%\newcommand{\ddot}[0]{{\dot{d}}}
\newcommand{\edot}[0]{{\dot{e}}}
\newcommand{\fdot}[0]{{\dot{f}}}
\newcommand{\gdot}[0]{{\dot{g}}}
\newcommand{\hdot}[0]{{\dot{h}}}
\newcommand{\idot}[0]{{\dot{i}}}
\newcommand{\jdot}[0]{{\dot{j}}}
\newcommand{\kdot}[0]{{\dot{k}}}
\newcommand{\ldot}[0]{{\dot{l}}}
\newcommand{\mdot}[0]{{\dot{m}}}
\newcommand{\ndot}[0]{{\dot{n}}}
%\newcommand{\odot}[0]{{\dot{o}}}
\newcommand{\pdot}[0]{{\dot{p}}}
\newcommand{\qdot}[0]{{\dot{q}}}
\newcommand{\rdot}[0]{{\dot{r}}}
\newcommand{\sdot}[0]{{\dot{s}}}
\newcommand{\tdot}[0]{{\dot{t}}}
\newcommand{\udot}[0]{{\dot{u}}}
\newcommand{\vdot}[0]{{\dot{v}}}
\newcommand{\wdot}[0]{{\dot{w}}}
\newcommand{\xdot}[0]{{\dot{x}}}
\newcommand{\ydot}[0]{{\dot{y}}}
\newcommand{\zdot}[0]{{\dot{z}}}
\newcommand{\addot}[0]{{\ddot{a}}}
\newcommand{\bddot}[0]{{\ddot{b}}}
\newcommand{\cddot}[0]{{\ddot{c}}}
%\newcommand{\dddot}[0]{{\ddot{d}}}
\newcommand{\eddot}[0]{{\ddot{e}}}
\newcommand{\fddot}[0]{{\ddot{f}}}
\newcommand{\gddot}[0]{{\ddot{g}}}
\newcommand{\hddot}[0]{{\ddot{h}}}
\newcommand{\iddot}[0]{{\ddot{i}}}
\newcommand{\jddot}[0]{{\ddot{j}}}
\newcommand{\kddot}[0]{{\ddot{k}}}
\newcommand{\lddot}[0]{{\ddot{l}}}
\newcommand{\mddot}[0]{{\ddot{m}}}
\newcommand{\nddot}[0]{{\ddot{n}}}
\newcommand{\oddot}[0]{{\ddot{o}}}
\newcommand{\pddot}[0]{{\ddot{p}}}
\newcommand{\qddot}[0]{{\ddot{q}}}
\newcommand{\rddot}[0]{{\ddot{r}}}
\newcommand{\sddot}[0]{{\ddot{s}}}
\newcommand{\tddot}[0]{{\ddot{t}}}
\newcommand{\uddot}[0]{{\ddot{u}}}
\newcommand{\vddot}[0]{{\ddot{v}}}
\newcommand{\wddot}[0]{{\ddot{w}}}
\newcommand{\xddot}[0]{{\ddot{x}}}
\newcommand{\yddot}[0]{{\ddot{y}}}
\newcommand{\zddot}[0]{{\ddot{z}}}

%<bold and dot greek symbols>
%

\newcommand{\Deltadot}[0]{{\dot{\Delta}}}
\newcommand{\Gammadot}[0]{{\dot{\Gamma}}}
\newcommand{\Lambdadot}[0]{{\dot{\Lambda}}}
\newcommand{\Omegadot}[0]{{\dot{\Omega}}}
\newcommand{\Phidot}[0]{{\dot{\Phi}}}
\newcommand{\Pidot}[0]{{\dot{\Pi}}}
\newcommand{\Psidot}[0]{{\dot{\Psi}}}
\newcommand{\Sigmadot}[0]{{\dot{\Sigma}}}
\newcommand{\Thetadot}[0]{{\dot{\Theta}}}
\newcommand{\Upsilondot}[0]{{\dot{\Upsilon}}}
\newcommand{\Xidot}[0]{{\dot{\Xi}}}
\newcommand{\alphadot}[0]{{\dot{\alpha}}}
\newcommand{\betadot}[0]{{\dot{\beta}}}
\newcommand{\chidot}[0]{{\dot{\chi}}}
\newcommand{\deltadot}[0]{{\dot{\delta}}}
\newcommand{\epsilondot}[0]{{\dot{\epsilon}}}
\newcommand{\etadot}[0]{{\dot{\eta}}}
\newcommand{\gammadot}[0]{{\dot{\gamma}}}
\newcommand{\kappadot}[0]{{\dot{\kappa}}}
\newcommand{\lambdadot}[0]{{\dot{\lambda}}}
\newcommand{\mudot}[0]{{\dot{\mu}}}
\newcommand{\nudot}[0]{{\dot{\nu}}}
\newcommand{\omegadot}[0]{{\dot{\omega}}}
\newcommand{\phidot}[0]{{\dot{\phi}}}
\newcommand{\pidot}[0]{{\dot{\pi}}}
\newcommand{\psidot}[0]{{\dot{\psi}}}
\newcommand{\rhodot}[0]{{\dot{\rho}}}
\newcommand{\sigmadot}[0]{{\dot{\sigma}}}
\newcommand{\taudot}[0]{{\dot{\tau}}}
\newcommand{\thetadot}[0]{{\dot{\theta}}}
\newcommand{\upsilondot}[0]{{\dot{\upsilon}}}
\newcommand{\varepsilondot}[0]{{\dot{\varepsilon}}}
\newcommand{\varphidot}[0]{{\dot{\varphi}}}
\newcommand{\varpidot}[0]{{\dot{\varpi}}}
\newcommand{\varrhodot}[0]{{\dot{\varrho}}}
\newcommand{\varsigmadot}[0]{{\dot{\varsigma}}}
\newcommand{\varthetadot}[0]{{\dot{\vartheta}}}
\newcommand{\xidot}[0]{{\dot{\xi}}}
\newcommand{\zetadot}[0]{{\dot{\zeta}}}

\newcommand{\Deltaddot}[0]{{\ddot{\Delta}}}
\newcommand{\Gammaddot}[0]{{\ddot{\Gamma}}}
\newcommand{\Lambdaddot}[0]{{\ddot{\Lambda}}}
\newcommand{\Omegaddot}[0]{{\ddot{\Omega}}}
\newcommand{\Phiddot}[0]{{\ddot{\Phi}}}
\newcommand{\Piddot}[0]{{\ddot{\Pi}}}
\newcommand{\Psiddot}[0]{{\ddot{\Psi}}}
\newcommand{\Sigmaddot}[0]{{\ddot{\Sigma}}}
\newcommand{\Thetaddot}[0]{{\ddot{\Theta}}}
\newcommand{\Upsilonddot}[0]{{\ddot{\Upsilon}}}
\newcommand{\Xiddot}[0]{{\ddot{\Xi}}}
\newcommand{\alphaddot}[0]{{\ddot{\alpha}}}
\newcommand{\betaddot}[0]{{\ddot{\beta}}}
\newcommand{\chiddot}[0]{{\ddot{\chi}}}
\newcommand{\deltaddot}[0]{{\ddot{\delta}}}
\newcommand{\epsilonddot}[0]{{\ddot{\epsilon}}}
\newcommand{\etaddot}[0]{{\ddot{\eta}}}
\newcommand{\gammaddot}[0]{{\ddot{\gamma}}}
\newcommand{\kappaddot}[0]{{\ddot{\kappa}}}
\newcommand{\lambdaddot}[0]{{\ddot{\lambda}}}
\newcommand{\muddot}[0]{{\ddot{\mu}}}
\newcommand{\nuddot}[0]{{\ddot{\nu}}}
\newcommand{\omegaddot}[0]{{\ddot{\omega}}}
\newcommand{\phiddot}[0]{{\ddot{\phi}}}
\newcommand{\piddot}[0]{{\ddot{\pi}}}
\newcommand{\psiddot}[0]{{\ddot{\psi}}}
\newcommand{\rhoddot}[0]{{\ddot{\rho}}}
\newcommand{\sigmaddot}[0]{{\ddot{\sigma}}}
\newcommand{\tauddot}[0]{{\ddot{\tau}}}
\newcommand{\thetaddot}[0]{{\ddot{\theta}}}
\newcommand{\upsilonddot}[0]{{\ddot{\upsilon}}}
\newcommand{\varepsilonddot}[0]{{\ddot{\varepsilon}}}
\newcommand{\varphiddot}[0]{{\ddot{\varphi}}}
\newcommand{\varpiddot}[0]{{\ddot{\varpi}}}
\newcommand{\varrhoddot}[0]{{\ddot{\varrho}}}
\newcommand{\varsigmaddot}[0]{{\ddot{\varsigma}}}
\newcommand{\varthetaddot}[0]{{\ddot{\vartheta}}}
\newcommand{\xiddot}[0]{{\ddot{\xi}}}
\newcommand{\zetaddot}[0]{{\ddot{\zeta}}}

\newcommand{\BDelta}[0]{\boldsymbol{\Delta}}
\newcommand{\BGamma}[0]{\boldsymbol{\Gamma}}
\newcommand{\BLambda}[0]{\boldsymbol{\Lambda}}
\newcommand{\BOmega}[0]{\boldsymbol{\Omega}}
\newcommand{\BPhi}[0]{\boldsymbol{\Phi}}
\newcommand{\BPi}[0]{\boldsymbol{\Pi}}
\newcommand{\BPsi}[0]{\boldsymbol{\Psi}}
\newcommand{\BSigma}[0]{\boldsymbol{\Sigma}}
\newcommand{\BTheta}[0]{\boldsymbol{\Theta}}
\newcommand{\BUpsilon}[0]{\boldsymbol{\Upsilon}}
\newcommand{\BXi}[0]{\boldsymbol{\Xi}}
\newcommand{\Balpha}[0]{\boldsymbol{\alpha}}
\newcommand{\Bbeta}[0]{\boldsymbol{\beta}}
\newcommand{\Bchi}[0]{\boldsymbol{\chi}}
\newcommand{\Bdelta}[0]{\boldsymbol{\delta}}
\newcommand{\Bepsilon}[0]{\boldsymbol{\epsilon}}
\newcommand{\Beta}[0]{\boldsymbol{\eta}}
\newcommand{\Bgamma}[0]{\boldsymbol{\gamma}}
\newcommand{\Bkappa}[0]{\boldsymbol{\kappa}}
\newcommand{\Blambda}[0]{\boldsymbol{\lambda}}
\newcommand{\Bmu}[0]{\boldsymbol{\mu}}
\newcommand{\Bnu}[0]{\boldsymbol{\nu}}
%\newcommand{\Bomega}[0]{\boldsymbol{\omega}}
\newcommand{\Bphi}[0]{\boldsymbol{\phi}}
\newcommand{\Bpi}[0]{\boldsymbol{\pi}}
\newcommand{\Bpsi}[0]{\boldsymbol{\psi}}
\newcommand{\Brho}[0]{\boldsymbol{\rho}}
\newcommand{\Bsigma}[0]{\boldsymbol{\sigma}}
%\newcommand{\Btau}[0]{\boldsymbol{\tau}}
%\newcommand{\Btheta}[0]{\boldsymbol{\theta}}
\newcommand{\Bupsilon}[0]{\boldsymbol{\upsilon}}
\newcommand{\Bvarepsilon}[0]{\boldsymbol{\varepsilon}}
\newcommand{\Bvarphi}[0]{\boldsymbol{\varphi}}
\newcommand{\Bvarpi}[0]{\boldsymbol{\varpi}}
\newcommand{\Bvarrho}[0]{\boldsymbol{\varrho}}
\newcommand{\Bvarsigma}[0]{\boldsymbol{\varsigma}}
\newcommand{\Bvartheta}[0]{\boldsymbol{\vartheta}}
\newcommand{\Bxi}[0]{\boldsymbol{\xi}}
\newcommand{\Bzeta}[0]{\boldsymbol{\zeta}}
%
%</bold and dot greek symbols>
%<infrequent>
%
%\newcommand{\AreaOp}[1]{\AName_{#1}}
%\newcommand{\Babs}[0]{\abs{\BB}}
%\newcommand{\Bcap}[0]{\hat{\BB}}
%\newcommand{\BrPrimeRej}[0]{\rcap(\rcap \wedge \Br')}
%\newcommand{\CA}[0]{\mathcal{A}}
%\newcommand{\Cos}[1]{\cos{\left({#1}\right)}}
%\newcommand{\Det}[1] {\abs{#1}}
%\newcommand{\Dsq}[2] {\frac {\partial^2 {#1}} {\partial {#2}^2}}
%\newcommand{\Exp}[1]{\exp{\left({#1}\right)}}
%\newcommand{\Norm}[1]{\left\lVert{#1}\right\rVert}
%\newcommand{\Sin}[1]{\sin{\left({#1}\right)}}
%\newcommand{\T}[0]{\text{T}}
%\newcommand{\VolumeOp}[1]{\VName_{#1}}
%\newcommand{\agrad}[0]{\Ba \cdot \nabla}
%\newcommand{\alphacap}[0]{\hat{\boldsymbol{\alpha}}}
%\newcommand{\Fcap}[0]{\hat{\BF}}
%\newcommand{\bithree}[0]{{\Bi}_3}
%\newcommand{\bxa}[0]{\Bx\Ba}
%\newcommand{\coordvec}[2]{
%\newcommand{\costheta}[0]{\acap \cdot \xcap}
%\newcommand{\ddt}[1]{\ddot{#1}}
%\newcommand{\ddu}[1] {\frac {d{#1}} {du}}
%\newcommand{\dsqxj}[2] {\frac {\partial^2 {#1}} {\partial {x_{#2}}^2}}
%\newcommand{\dtheta}[1]{\frac{d {#1}}{d \theta}}
%\newcommand{\dt}[1]{\dot{#1}}
%\newcommand{\dt}[1]{\frac{d {#1}}{dt}}
%\newcommand{\dxj}[2] {\frac {\partial {#1}} {\partial {x_{#2}}}}
%\newcommand{\halfPhi}[0]{\frac{\phi}{2}}
%\newcommand{\half}[0]{\inv{2}}
%\newcommand{\inv}[1]{\frac{1}{#1}}
%\newcommand{\laplacian}[0]{\nabla^2}
%\newcommand{\matrixoftx}[3]{
%\newcommand{\nrrp}[0]{\norm{\rcap \wedge \Br'}}
%\newcommand{\oiint}{\bigcirc \hspace{-1.4em} \int \hspace{-.8em} \int}
%\newcommand{\transpose}[1]{{#1}^{\text{T}}}
%\newcommand{\transpose}[1]{{{#1}^{\TextTranspose}}}
%\newcommand{\transpose}[1]{{{#1}^{\text{T}}}}
%\newcommand{\barA}[0]{\bar{A}}
%\newcommand{\qbar}[0]{\bar{q}}
%\newcommand{\qdotbar}[0]{\dot{\bar{q}}}
%
%</infrequent>




\newcommand{\ket}[1]{\lvert {#1} \rangle}
\newcommand{\bra}[1]{\langle {#1} \rvert}
\newcommand{\braket}[2]{\langle{#1} \vert {#2}\rangle}
\newcommand{\ketbra}[2]{\ket{#1}\bra{#2}}
\newcommand{\BraOpKet}[3]{\bra{#1} \hat{#2} \ket{#3} }

\usepackage[bookmarks=true]{hyperref}

\usepackage{color,cite,graphicx}
   % use colour in the document, put your citations as [1-4]
   % rather than [1,2,3,4] (it looks nicer, and the extended LaTeX2e
   % graphics package. 
\usepackage{latexsym,amssymb,epsf} % don't remember if these are
   % needed, but their inclusion can't do any damage


\title{ Notes on Susskind's QM Lecture 3. }
\author{Peeter Joot}
\date{ Dec 23, 2008.  Last Revision: $Date: 2008/12/24 05:52:30 $ }

\begin{document}

\maketitle{}
%\tableofcontents
\section{ Bra and Ket vectors. }

An odd looking vector notation is introduced.  Instead of just using a letter, say $A$, for a vector in \C{N}, such a
vector is instead written

\begin{align*}
\ket{A}
\end{align*}

This is called a ``ket'' or ket-vector, but really just means complex vector.   The complex conjugate of this
vector is then written as a ``bra'' like so

\begin{align*}
\bra{A}
\end{align*}

The inner product of two vectors can then be written by combining this bra and ket by butting them up together, as
in

\begin{align*}
\braket{A}{B}
\end{align*}

Contrast this to the explicit complex column vector representation

\begin{align*}
{A} = 
\begin{bmatrix}
a_1 \\
a_2 \\
\vdots \\
a_n
\end{bmatrix}
\quad
{B} = 
\begin{bmatrix}
b_1 \\
b_2 \\
\vdots \\
b_n
\end{bmatrix}
\end{align*}

in a finite dimensional space.  The usual convention is to employ an inner product notation like

\begin{align*}
\innerprod{A}{B} = {A}^\text{T} \bar{B} = \sum_i {a_i} \bar{b_i}
\end{align*}

or,
\begin{align*}
\innerprod{A}{B} = {A}^\conj {B} = \sum_i \bar{a_i} {b_i}
\end{align*}

Observe that the braket notation is closer to this last form with the conjugation on the first term.  However, note that the metric associated with the braket notation has not been specified yet.  This is in fact an integral over space, where the ket vectors are complex valued functions.

\subsection{ Coordinates and basis notation. }

Ket vectors represent states, and the lables that are used for these are pretty loose.  For example, instead
of writing the n'th basis vector as 

\begin{align*}
\ket{a_n}
\end{align*}

just n was used like so

\begin{align*}
\ket{n}
\end{align*}

so if $\{\ket{n}\}$ is a basis, the coordinates of a vector $\ket{A}$ can be written

\begin{align*}
\ket{A} = \sum_n \alpha_n \ket{n}
\end{align*}

Note that here in the summation sign, and the subscript $n$ is an index, and also implicitly indexes the basis vectors, but in that context is not a number but a label for the basis vector itself.

Assuming that the $\ket{n}$ vectors are orthonormal, we can take inner products (brackets) to compute the $\alpha_n$ coordinates.

Writing $\ket{k}$ as an alternate labeling for the same basis, the congugate bra vectors when sandwiched against this bra representation denotes the inner product

\begin{align*}
\braket{n}{A} 
&= \sum_k \alpha_k \braket{n}{k} \\
&= \sum_k \alpha_k \delta_{nk} \\
&= \alpha_n \\
\end{align*}

So we have

\begin{align*}
\ket{A} = \sum_n \braket{n}{A} \ket{n}
\end{align*}

Since $\ket{n}$ is a vector and $\braket{n}{A}$ is just a complex number, this can be rearranged to butt the points
together as a mnemonic reminder that this is a projective operation.

\begin{align*}
\ket{A} = \sum_n \ket{n} \braket{n}{A} 
\end{align*}

As is the case with an othornormal split by projection matrixes, this can be observed to be more than a memory
device since the object

\begin{align*}
\ketbra{n}{n}
\end{align*}

is in fact the orthonormal projection operator onto the $\ket{n}$ direction.  ie:

\begin{align*}
\Proj_{\ket{n}}(\ket{A}) = \left( \ketbra{n}{n} \right) \ket{A}
\end{align*}

Note this is not summed over indexes $n$.

In the matrix representation, this is really nothing more than writing

\begin{align*}
\Proj_{e} (A) = e \innerprod{e}{A} = e (e^\conj A) = (e e^\conj) A
\end{align*}

So one can think of this funny looking $\ketbra{n}{n}$ as nothing more than the orthonormal projector matrix of the form $e e^\conj$.

\subsection{ Dual space. }

Susskind called the set of the conjugate vectors (the bras), the dual space.  If the basis is not orthonormal
are the conjugates really the duals (reciprocals) of the basis vectors?  Let's see with an example:

\begin{align*}
\ket{1} = 
\inv{\sqrt{2}}
\begin{bmatrix}
1 \\
i
\end{bmatrix}
, \quad
\ket{2} = 
\inv{\sqrt{5}}
\begin{bmatrix}
i \\
2
\end{bmatrix}
\end{align*}

Here we have

\begin{align*}
\braket{1}{1} = 
\inv{2}
\begin{bmatrix}
1 & -i
\end{bmatrix}
\begin{bmatrix}
1 \\
i
\end{bmatrix}
= 1
\end{align*}

and 

\begin{align*}
\braket{2}{2} = 
\inv{5}
\begin{bmatrix}
-i & 2 
\end{bmatrix}
\begin{bmatrix}
i \\
2
\end{bmatrix}
= 1
\end{align*}

\begin{align*}
\braket{1}{2} = 
\inv{\sqrt{10}}
\begin{bmatrix}
1 & -i
\end{bmatrix}
\begin{bmatrix}
i \\
2
\end{bmatrix}
= -\frac{i}{\sqrt{10}}
\end{align*}

\begin{align*}
\braket{2}{1} = 
\inv{\sqrt{10}}
\begin{bmatrix}
-i & 2
\end{bmatrix}
\begin{bmatrix}
1 \\
i
\end{bmatrix}
= \frac{i}{\sqrt{10}}
\end{align*}

Definitely not the dual space.  The conjugates are only going to be the dual basis when the primary basis is orthonormal.

Switching back to abstraction temporarily, let's calculate the coordinates with respect to a non-orthonormal
basis of dimension $k$.  That is, determine the $\alpha_i$ given a decomposition by basis vectors

\begin{align*}
\ket{x} = \sum_i \alpha_i \ket{a_i}
\end{align*}

We have 
\begin{align*}
\braket{a_j}{x} 
&= \sum_i \alpha_i \braket{a_j}{a_i} \\
&=
\begin{bmatrix}
\braket{a_j}{a_1} & \braket{a_j}{a_2} & \hdots & \braket{a_j}{a_k}
\end{bmatrix}
\begin{bmatrix}
\alpha_1 \\
\alpha_2 \\
\vdots \\
\alpha_k \\
\end{bmatrix}
\end{align*}

Assembling these into a matrix with a column for each $j$, we have
\begin{align*}
\begin{bmatrix}
\braket{a_1}{x} \\
\braket{a_2}{x} \\
\vdots \\
\braket{a_k}{x} \\
\end{bmatrix}
&=
{\begin{bmatrix}
\braket{a_i}{a_j}
\end{bmatrix}}_{ij}
\Balpha
\end{align*}

With,

\begin{align*}
A = 
\begin{bmatrix}
\ket{a_1} & \ket{a_2} & \hdots & \ket{a_k}
\end{bmatrix},
\end{align*}

this is

\begin{align*}
\Balpha = \inv{A^\conj A} A^\conj \ket{x}
\end{align*}

or 
\begin{align*}
\ket{x} = A \Balpha = A \inv{A^\conj A} A^\conj \ket{x}
\end{align*}

From this we can pick off the reciprocal frame vectors, which are the columns of 

\begin{align*}
\begin{bmatrix}
\ket{a^1} & \ket{a^2} & \hdots & \ket{a^k}
\end{bmatrix} = 
A \inv{A^\conj A}
\end{align*}

To verify we calculate 
\begin{align*}
{\begin{bmatrix}
\ket{a^1} & \ket{a^2} & \hdots & \ket{a^k}
\end{bmatrix}}^\conj
\begin{bmatrix}
\ket{a_1} & \ket{a_2} & \hdots & \ket{a_k}
\end{bmatrix}
&=
\begin{bmatrix}
\bra{a^1} \\ \bra{a^2} \\ \vdots \\ \bra{a^k}
\end{bmatrix}
\begin{bmatrix}
\ket{a_1} & \ket{a_2} & \hdots & \ket{a_k}
\end{bmatrix} \\
&= 
{
\begin{bmatrix}
\braket{a^i}{a_j}
\end{bmatrix}}_{ij}
\end{align*}

with the expectation that this is the identity matrix.  That product is

\begin{align*}
&\left(
\begin{bmatrix}
\ket{a_1} & \ket{a_2} & \hdots & \ket{a_k}
\end{bmatrix}
\inv{{\begin{bmatrix}
\braket{a_i}{a_j}
\end{bmatrix}}_{ij}} \right)^\conj
\begin{bmatrix}
\ket{a_1} & \ket{a_2} & \hdots & \ket{a_k}
\end{bmatrix} \\
&=
\inv{{\begin{bmatrix}
\braket{a_j}{a_i}^\conj
\end{bmatrix}}_{ij}} 
\begin{bmatrix}
\bra{a_1} \\ \bra{a_2} \\ \vdots \\ \bra{a_k}
\end{bmatrix}
\begin{bmatrix}
\ket{a_1} & \ket{a_2} & \hdots & \ket{a_k}
\end{bmatrix} \\
&=
\inv{{\begin{bmatrix}
\braket{a_i}{a_j}
\end{bmatrix}}_{ij}} 
{{\begin{bmatrix}
\braket{a_i}{a_j}
\end{bmatrix}}_{ij}} \\
&= I
\end{align*}

This proves the desired result, that we can calculate $\braket{a^i}{a_j} = \delta_{ij}$ where 

\begin{align*}
\begin{bmatrix}
\ket{a^1} & \ket{a^2} & \hdots & \ket{a^k}
\end{bmatrix} = 
\begin{bmatrix}
\ket{a_1} & \ket{a_2} & \hdots & \ket{a_k}
\end{bmatrix}
\inv{{\begin{bmatrix}
\braket{a_i}{a_j}
\end{bmatrix}}_{ij}}
\end{align*}

While kind of fun to see how to express this in the bra ket notation, is this useful.  Probably not since
all the operators of QM are Hermitian, and thus have orthonormal basis (the eigenvectors).  Oh well... it has
provided some comfort with the notation if nothing else.

\subsection{ Hermitian operators. }

The operators of QM are written with hats, and are applied to vectors (states).  For example for an 
operator $\hat{H}$ applied to $\ket{x}$ we write

\begin{align*}
\hat{H} \ket{x}
\end{align*}

FIXME: Express op applied to a vector in coordinates.

Additionally, the inner product is written with a sandwich bra ket format like

\begin{align*}
\braket{y}{ \left(\hat{H} \ket{x} \right) }
=
\BraOpKet{y}{H}{x}
\end{align*}

In the braket notation a Hermitian operator $\hat{H}$ is defined as one 

FIXME.

and anti-Hermitian is:

FIXME:

\subsubsection{ Tricky eigenvalue notation. } 



\section{ Postulates of QM. }



\section{ Position operator. }

\section{ Derivative (momentum) operator. }

\subsection{ Proper characterization as momentum. }

There is a discussion in the lectures mentioning that a wave function that is peaked at a position in space
corresponds to the position of a particle in an approximate, but intuitively reasonable seeming fashion.

It is mentioned however, to properly show the same sort of characterization for the 
equivalence of the QM momentum construction and momentum of classical
physics requires the consideration of a (large and massive) wave packet.  It can be shown that the equations that
govern the motion
of such an object approximate familiar newtonian dynamics in this limit.  It will be interesting to see how
this pans out.  My assumption is that if we start from the relativistic (Dirac) wave equations, we can also
get the relativistic dynamics equations, and presumably also ideas like the classical current density vector
of electromagnetism.  Very exciting to see that it is at least possible to formulate the classical results from
more fundamental underlying principles, even if I don't know what those are yet.  How will all of this relate 
to the Lagrangian formulation that we can express newtonian and relativistic dynamics and electromagnetism using.
Can one in fact produce the classical Lagrangians for (proper) Lorentz force, and Maxwell's equation (or at least the $A \cdot J$ term) directly from the QM Lagrangians?

%\bibliographystyle{plainnat}
%\bibliography{myrefs}

\end{document}
          % dec 23/08
%
% Copyright � 2012 Peeter Joot.  All Rights Reserved.
% Licenced as described in the file LICENSE under the root directory of this GIT repository.
%

% 
% 
%\documentclass{article}

%\usepackage{amsmath}
\usepackage{mathpazo}

%
% shorthand for bold symbols, convenient for vectors and matrices
%
\newcommand{\Ba}[0]{\mathbf{a}}
\newcommand{\Bb}[0]{\mathbf{b}}
\newcommand{\Bc}[0]{\mathbf{c}}
\newcommand{\Bd}[0]{\mathbf{d}}
\newcommand{\Be}[0]{\mathbf{e}}
\newcommand{\Bf}[0]{\mathbf{f}}
\newcommand{\Bg}[0]{\mathbf{g}}
\newcommand{\Bh}[0]{\mathbf{h}}
\newcommand{\Bi}[0]{\mathbf{i}}
\newcommand{\Bj}[0]{\mathbf{j}}
\newcommand{\Bk}[0]{\mathbf{k}}
\newcommand{\Bl}[0]{\mathbf{l}}
\newcommand{\Bm}[0]{\mathbf{m}}
\newcommand{\Bn}[0]{\mathbf{n}}
\newcommand{\Bo}[0]{\mathbf{o}}
\newcommand{\Bp}[0]{\mathbf{p}}
\newcommand{\Bq}[0]{\mathbf{q}}
\newcommand{\Br}[0]{\mathbf{r}}
\newcommand{\Bs}[0]{\mathbf{s}}
\newcommand{\Bt}[0]{\mathbf{t}}
\newcommand{\Bu}[0]{\mathbf{u}}
\newcommand{\Bv}[0]{\mathbf{v}}
\newcommand{\Bw}[0]{\mathbf{w}}
\newcommand{\Bx}[0]{\mathbf{x}}
\newcommand{\By}[0]{\mathbf{y}}
\newcommand{\Bz}[0]{\mathbf{z}}
\newcommand{\BA}[0]{\mathbf{A}}
\newcommand{\BB}[0]{\mathbf{B}}
\newcommand{\BC}[0]{\mathbf{C}}
\newcommand{\BD}[0]{\mathbf{D}}
\newcommand{\BE}[0]{\mathbf{E}}
\newcommand{\BF}[0]{\mathbf{F}}
\newcommand{\BG}[0]{\mathbf{G}}
\newcommand{\BH}[0]{\mathbf{H}}
\newcommand{\BI}[0]{\mathbf{I}}
\newcommand{\BJ}[0]{\mathbf{J}}
\newcommand{\BK}[0]{\mathbf{K}}
\newcommand{\BL}[0]{\mathbf{L}}
\newcommand{\BM}[0]{\mathbf{M}}
\newcommand{\BN}[0]{\mathbf{N}}
\newcommand{\BO}[0]{\mathbf{O}}
\newcommand{\BP}[0]{\mathbf{P}}
\newcommand{\BQ}[0]{\mathbf{Q}}
\newcommand{\BR}[0]{\mathbf{R}}
\newcommand{\BS}[0]{\mathbf{S}}
\newcommand{\BT}[0]{\mathbf{T}}
\newcommand{\BU}[0]{\mathbf{U}}
\newcommand{\BV}[0]{\mathbf{V}}
\newcommand{\BW}[0]{\mathbf{W}}
\newcommand{\BX}[0]{\mathbf{X}}
\newcommand{\BY}[0]{\mathbf{Y}}
\newcommand{\BZ}[0]{\mathbf{Z}}

\newcommand{\Bzero}[0]{\mathbf{0}}
\newcommand{\Btheta}[0]{\boldsymbol{\theta}}
\newcommand{\Btau}[0]{\boldsymbol{\tau}}
\newcommand{\Bomega}[0]{\boldsymbol{\omega}}

%
% shorthand for unit vectors
%
\newcommand{\acap}[0]{\hat{\Ba}}
\newcommand{\bcap}[0]{\hat{\Bb}}
\newcommand{\ccap}[0]{\hat{\Bc}}
\newcommand{\dcap}[0]{\hat{\Bd}}
\newcommand{\ecap}[0]{\hat{\Be}}
\newcommand{\fcap}[0]{\hat{\Bf}}
\newcommand{\gcap}[0]{\hat{\Bg}}
\newcommand{\hcap}[0]{\hat{\Bh}}
\newcommand{\icap}[0]{\hat{\Bi}}
\newcommand{\jcap}[0]{\hat{\Bj}}
\newcommand{\kcap}[0]{\hat{\Bk}}
\newcommand{\lcap}[0]{\hat{\Bl}}
\newcommand{\mcap}[0]{\hat{\Bm}}
\newcommand{\ncap}[0]{\hat{\Bn}}
\newcommand{\ocap}[0]{\hat{\Bo}}
\newcommand{\pcap}[0]{\hat{\Bp}}
\newcommand{\qcap}[0]{\hat{\Bq}}
\newcommand{\rcap}[0]{\hat{\Br}}
\newcommand{\scap}[0]{\hat{\Bs}}
\newcommand{\tcap}[0]{\hat{\Bt}}
\newcommand{\ucap}[0]{\hat{\Bu}}
\newcommand{\vcap}[0]{\hat{\Bv}}
\newcommand{\wcap}[0]{\hat{\Bw}}
\newcommand{\xcap}[0]{\hat{\Bx}}
\newcommand{\ycap}[0]{\hat{\By}}
\newcommand{\zcap}[0]{\hat{\Bz}}
\newcommand{\thetacap}[0]{\hat{\Btheta}}

%
% to write R^n and C^n in a distinguishable fashion.  Perhaps change this
% to the double lined characters upon figuring out how to do so.
%
\newcommand{\C}[1]{$\mathbb{C}^{#1}$}
\newcommand{\R}[1]{$\mathbb{R}^{#1}$}

%
% various generally useful helpers
%

% derivative of #1 wrt. #2:
\newcommand{\D}[2] {\frac {d#2} {d#1}}

\newcommand{\inv}[1]{\frac{1}{#1}}
\newcommand{\cross}[0]{\times}

\newcommand{\abs}[1]{\lvert{#1}\rvert}
\newcommand{\norm}[1]{\lVert{#1}\rVert}
\newcommand{\innerprod}[2]{\langle{#1}, {#2}\rangle}
\newcommand{\dotprod}[2]{{#1} \cdot {#2}}
\newcommand{\bdotprod}[2]{\left({#1} \cdot {#2}\right)}
\newcommand{\crossprod}[2]{{#1} \cross {#2}}
\newcommand{\tripleprod}[3]{\dotprod{\left(\crossprod{#1}{#2}\right)}{#3}}

\DeclareMathOperator{\Proj}{Proj}
\DeclareMathOperator{\Span}{span}
\DeclareMathOperator{\Sgn}{sgn}
\DeclareMathOperator{\Area}{Area}
\DeclareMathOperator{\Volume}{Volume}

%
% A few miscellaneous things specific to this document
%
\newcommand{\crossop}[1]{\crossprod{#1}{}}

% R2 vector.
\newcommand{\VectorTwo}[2]{
\begin{bmatrix}
 {#1} \\
 {#2}
\end{bmatrix}
}

\newcommand{\VectorN}[1]{
\begin{bmatrix}
{#1}_1 \\
{#1}_2 \\
\vdots \\
{#1}_N \\
\end{bmatrix}
}

\newcommand{\DETuvij}[4]{
\begin{vmatrix}
 {#1}_{#3} & {#1}_{#4} \\
 {#2}_{#3} & {#2}_{#4}
\end{vmatrix}
}

\newcommand{\DETuvwijk}[6]{
\begin{vmatrix}
 {#1}_{#4} & {#1}_{#5} & {#1}_{#6} \\
 {#2}_{#4} & {#2}_{#5} & {#2}_{#6} \\
 {#3}_{#4} & {#3}_{#5} & {#3}_{#6}
\end{vmatrix}
}

\newcommand{\DETuvwxijkl}[8]{
\begin{vmatrix}
 {#1}_{#5} & {#1}_{#6} & {#1}_{#7} & {#1}_{#8} \\
 {#2}_{#5} & {#2}_{#6} & {#2}_{#7} & {#2}_{#8} \\
 {#3}_{#5} & {#3}_{#6} & {#3}_{#7} & {#3}_{#8} \\
 {#4}_{#5} & {#4}_{#6} & {#4}_{#7} & {#4}_{#8} \\
\end{vmatrix}
}

%\newcommand{\DETuvwxyijklm}[10]{
%\begin{vmatrix}
% {#1}_{#6} & {#1}_{#7} & {#1}_{#8} & {#1}_{#9} & {#1}_{#10} \\
% {#2}_{#6} & {#2}_{#7} & {#2}_{#8} & {#2}_{#9} & {#2}_{#10} \\
% {#3}_{#6} & {#3}_{#7} & {#3}_{#8} & {#3}_{#9} & {#3}_{#10} \\
% {#4}_{#6} & {#4}_{#7} & {#4}_{#8} & {#4}_{#9} & {#4}_{#10} \\
% {#5}_{#6} & {#5}_{#7} & {#5}_{#8} & {#5}_{#9} & {#5}_{#10}
%\end{vmatrix}
%}

% R3 vector.
\newcommand{\VectorThree}[3]{
\begin{bmatrix}
 {#1} \\
 {#2} \\
 {#3}
\end{bmatrix}
}


%%<misc>
%
\newcommand{\Abs}[1]{{\left\lvert{#1}\right\rvert}}
\newcommand{\spacegrad}[0]{\boldsymbol{\nabla}}
\newcommand{\grad}[0]{\nabla}
\newcommand{\LL}[0]{\mathcal{L}}

% == \partial_{#1} {#2}
\newcommand{\PD}[2]{\frac{\partial {#2}}{\partial {#1}}}
% inline variant
\newcommand{\PDi}[2]{{\partial {#2}}/{\partial {#1}}}

\newcommand{\PDD}[3]{\frac{\partial^2 {#3}}{\partial {#1}\partial {#2}}}
%\newcommand{\PDd}[2]{\frac{\partial^2 {#2}}{{\partial{#1}}^2}}
\newcommand{\PDsq}[2]{\frac{\partial^2 {#2}}{(\partial {#1})^2}}

\newcommand{\Partial}[2]{\frac{\partial {#1}}{\partial {#2}}}
\DeclareMathOperator{\RejName}{Rej}
\newcommand{\Rej}[2]{\RejName_{#1}\left( {#2} \right)}
\newcommand{\Rm}[1]{\mathbb{R}^{#1}}
\newcommand{\Cm}[1]{\mathbb{C}^{#1}}
\newcommand{\conj}[0]{{*}}

%</misc>

% <grade selection>
%
\newcommand{\gpgrade}[2] {{\left\langle{{#1}}\right\rangle}_{#2}}

\newcommand{\gpgradezero}[1] {\gpgrade{#1}{}}
%\newcommand{\gpscalargrade}[1] {{\left\langle{{#1}}\right\rangle}}
%\newcommand{\gpgradezero}[1] {\gpgrade{#1}{0}}

%\newcommand{\gpgradeone}[1] {{\left\langle{{#1}}\right\rangle}_{1}}
\newcommand{\gpgradeone}[1] {\gpgrade{#1}{1}}

\newcommand{\gpgradetwo}[1] {\gpgrade{#1}{2}}
\newcommand{\gpgradethree}[1] {\gpgrade{#1}{3}}
\newcommand{\gpgradefour}[1] {\gpgrade{#1}{4}}
%
% </grade selection>



\newcommand{\adot}[0]{{\dot{a}}}
\newcommand{\bdot}[0]{{\dot{b}}}
% taken for centered dot:
%\newcommand{\cdot}[0]{{\dot{c}}}
%\newcommand{\ddot}[0]{{\dot{d}}}
\newcommand{\edot}[0]{{\dot{e}}}
\newcommand{\fdot}[0]{{\dot{f}}}
\newcommand{\gdot}[0]{{\dot{g}}}
\newcommand{\hdot}[0]{{\dot{h}}}
\newcommand{\idot}[0]{{\dot{i}}}
\newcommand{\jdot}[0]{{\dot{j}}}
\newcommand{\kdot}[0]{{\dot{k}}}
\newcommand{\ldot}[0]{{\dot{l}}}
\newcommand{\mdot}[0]{{\dot{m}}}
\newcommand{\ndot}[0]{{\dot{n}}}
%\newcommand{\odot}[0]{{\dot{o}}}
\newcommand{\pdot}[0]{{\dot{p}}}
\newcommand{\qdot}[0]{{\dot{q}}}
\newcommand{\rdot}[0]{{\dot{r}}}
\newcommand{\sdot}[0]{{\dot{s}}}
\newcommand{\tdot}[0]{{\dot{t}}}
\newcommand{\udot}[0]{{\dot{u}}}
\newcommand{\vdot}[0]{{\dot{v}}}
\newcommand{\wdot}[0]{{\dot{w}}}
\newcommand{\xdot}[0]{{\dot{x}}}
\newcommand{\ydot}[0]{{\dot{y}}}
\newcommand{\zdot}[0]{{\dot{z}}}
\newcommand{\addot}[0]{{\ddot{a}}}
\newcommand{\bddot}[0]{{\ddot{b}}}
\newcommand{\cddot}[0]{{\ddot{c}}}
%\newcommand{\dddot}[0]{{\ddot{d}}}
\newcommand{\eddot}[0]{{\ddot{e}}}
\newcommand{\fddot}[0]{{\ddot{f}}}
\newcommand{\gddot}[0]{{\ddot{g}}}
\newcommand{\hddot}[0]{{\ddot{h}}}
\newcommand{\iddot}[0]{{\ddot{i}}}
\newcommand{\jddot}[0]{{\ddot{j}}}
\newcommand{\kddot}[0]{{\ddot{k}}}
\newcommand{\lddot}[0]{{\ddot{l}}}
\newcommand{\mddot}[0]{{\ddot{m}}}
\newcommand{\nddot}[0]{{\ddot{n}}}
\newcommand{\oddot}[0]{{\ddot{o}}}
\newcommand{\pddot}[0]{{\ddot{p}}}
\newcommand{\qddot}[0]{{\ddot{q}}}
\newcommand{\rddot}[0]{{\ddot{r}}}
\newcommand{\sddot}[0]{{\ddot{s}}}
\newcommand{\tddot}[0]{{\ddot{t}}}
\newcommand{\uddot}[0]{{\ddot{u}}}
\newcommand{\vddot}[0]{{\ddot{v}}}
\newcommand{\wddot}[0]{{\ddot{w}}}
\newcommand{\xddot}[0]{{\ddot{x}}}
\newcommand{\yddot}[0]{{\ddot{y}}}
\newcommand{\zddot}[0]{{\ddot{z}}}

%<bold and dot greek symbols>
%

\newcommand{\Deltadot}[0]{{\dot{\Delta}}}
\newcommand{\Gammadot}[0]{{\dot{\Gamma}}}
\newcommand{\Lambdadot}[0]{{\dot{\Lambda}}}
\newcommand{\Omegadot}[0]{{\dot{\Omega}}}
\newcommand{\Phidot}[0]{{\dot{\Phi}}}
\newcommand{\Pidot}[0]{{\dot{\Pi}}}
\newcommand{\Psidot}[0]{{\dot{\Psi}}}
\newcommand{\Sigmadot}[0]{{\dot{\Sigma}}}
\newcommand{\Thetadot}[0]{{\dot{\Theta}}}
\newcommand{\Upsilondot}[0]{{\dot{\Upsilon}}}
\newcommand{\Xidot}[0]{{\dot{\Xi}}}
\newcommand{\alphadot}[0]{{\dot{\alpha}}}
\newcommand{\betadot}[0]{{\dot{\beta}}}
\newcommand{\chidot}[0]{{\dot{\chi}}}
\newcommand{\deltadot}[0]{{\dot{\delta}}}
\newcommand{\epsilondot}[0]{{\dot{\epsilon}}}
\newcommand{\etadot}[0]{{\dot{\eta}}}
\newcommand{\gammadot}[0]{{\dot{\gamma}}}
\newcommand{\kappadot}[0]{{\dot{\kappa}}}
\newcommand{\lambdadot}[0]{{\dot{\lambda}}}
\newcommand{\mudot}[0]{{\dot{\mu}}}
\newcommand{\nudot}[0]{{\dot{\nu}}}
\newcommand{\omegadot}[0]{{\dot{\omega}}}
\newcommand{\phidot}[0]{{\dot{\phi}}}
\newcommand{\pidot}[0]{{\dot{\pi}}}
\newcommand{\psidot}[0]{{\dot{\psi}}}
\newcommand{\rhodot}[0]{{\dot{\rho}}}
\newcommand{\sigmadot}[0]{{\dot{\sigma}}}
\newcommand{\taudot}[0]{{\dot{\tau}}}
\newcommand{\thetadot}[0]{{\dot{\theta}}}
\newcommand{\upsilondot}[0]{{\dot{\upsilon}}}
\newcommand{\varepsilondot}[0]{{\dot{\varepsilon}}}
\newcommand{\varphidot}[0]{{\dot{\varphi}}}
\newcommand{\varpidot}[0]{{\dot{\varpi}}}
\newcommand{\varrhodot}[0]{{\dot{\varrho}}}
\newcommand{\varsigmadot}[0]{{\dot{\varsigma}}}
\newcommand{\varthetadot}[0]{{\dot{\vartheta}}}
\newcommand{\xidot}[0]{{\dot{\xi}}}
\newcommand{\zetadot}[0]{{\dot{\zeta}}}

\newcommand{\Deltaddot}[0]{{\ddot{\Delta}}}
\newcommand{\Gammaddot}[0]{{\ddot{\Gamma}}}
\newcommand{\Lambdaddot}[0]{{\ddot{\Lambda}}}
\newcommand{\Omegaddot}[0]{{\ddot{\Omega}}}
\newcommand{\Phiddot}[0]{{\ddot{\Phi}}}
\newcommand{\Piddot}[0]{{\ddot{\Pi}}}
\newcommand{\Psiddot}[0]{{\ddot{\Psi}}}
\newcommand{\Sigmaddot}[0]{{\ddot{\Sigma}}}
\newcommand{\Thetaddot}[0]{{\ddot{\Theta}}}
\newcommand{\Upsilonddot}[0]{{\ddot{\Upsilon}}}
\newcommand{\Xiddot}[0]{{\ddot{\Xi}}}
\newcommand{\alphaddot}[0]{{\ddot{\alpha}}}
\newcommand{\betaddot}[0]{{\ddot{\beta}}}
\newcommand{\chiddot}[0]{{\ddot{\chi}}}
\newcommand{\deltaddot}[0]{{\ddot{\delta}}}
\newcommand{\epsilonddot}[0]{{\ddot{\epsilon}}}
\newcommand{\etaddot}[0]{{\ddot{\eta}}}
\newcommand{\gammaddot}[0]{{\ddot{\gamma}}}
\newcommand{\kappaddot}[0]{{\ddot{\kappa}}}
\newcommand{\lambdaddot}[0]{{\ddot{\lambda}}}
\newcommand{\muddot}[0]{{\ddot{\mu}}}
\newcommand{\nuddot}[0]{{\ddot{\nu}}}
\newcommand{\omegaddot}[0]{{\ddot{\omega}}}
\newcommand{\phiddot}[0]{{\ddot{\phi}}}
\newcommand{\piddot}[0]{{\ddot{\pi}}}
\newcommand{\psiddot}[0]{{\ddot{\psi}}}
\newcommand{\rhoddot}[0]{{\ddot{\rho}}}
\newcommand{\sigmaddot}[0]{{\ddot{\sigma}}}
\newcommand{\tauddot}[0]{{\ddot{\tau}}}
\newcommand{\thetaddot}[0]{{\ddot{\theta}}}
\newcommand{\upsilonddot}[0]{{\ddot{\upsilon}}}
\newcommand{\varepsilonddot}[0]{{\ddot{\varepsilon}}}
\newcommand{\varphiddot}[0]{{\ddot{\varphi}}}
\newcommand{\varpiddot}[0]{{\ddot{\varpi}}}
\newcommand{\varrhoddot}[0]{{\ddot{\varrho}}}
\newcommand{\varsigmaddot}[0]{{\ddot{\varsigma}}}
\newcommand{\varthetaddot}[0]{{\ddot{\vartheta}}}
\newcommand{\xiddot}[0]{{\ddot{\xi}}}
\newcommand{\zetaddot}[0]{{\ddot{\zeta}}}

\newcommand{\BDelta}[0]{\boldsymbol{\Delta}}
\newcommand{\BGamma}[0]{\boldsymbol{\Gamma}}
\newcommand{\BLambda}[0]{\boldsymbol{\Lambda}}
\newcommand{\BOmega}[0]{\boldsymbol{\Omega}}
\newcommand{\BPhi}[0]{\boldsymbol{\Phi}}
\newcommand{\BPi}[0]{\boldsymbol{\Pi}}
\newcommand{\BPsi}[0]{\boldsymbol{\Psi}}
\newcommand{\BSigma}[0]{\boldsymbol{\Sigma}}
\newcommand{\BTheta}[0]{\boldsymbol{\Theta}}
\newcommand{\BUpsilon}[0]{\boldsymbol{\Upsilon}}
\newcommand{\BXi}[0]{\boldsymbol{\Xi}}
\newcommand{\Balpha}[0]{\boldsymbol{\alpha}}
\newcommand{\Bbeta}[0]{\boldsymbol{\beta}}
\newcommand{\Bchi}[0]{\boldsymbol{\chi}}
\newcommand{\Bdelta}[0]{\boldsymbol{\delta}}
\newcommand{\Bepsilon}[0]{\boldsymbol{\epsilon}}
\newcommand{\Beta}[0]{\boldsymbol{\eta}}
\newcommand{\Bgamma}[0]{\boldsymbol{\gamma}}
\newcommand{\Bkappa}[0]{\boldsymbol{\kappa}}
\newcommand{\Blambda}[0]{\boldsymbol{\lambda}}
\newcommand{\Bmu}[0]{\boldsymbol{\mu}}
\newcommand{\Bnu}[0]{\boldsymbol{\nu}}
%\newcommand{\Bomega}[0]{\boldsymbol{\omega}}
\newcommand{\Bphi}[0]{\boldsymbol{\phi}}
\newcommand{\Bpi}[0]{\boldsymbol{\pi}}
\newcommand{\Bpsi}[0]{\boldsymbol{\psi}}
\newcommand{\Brho}[0]{\boldsymbol{\rho}}
\newcommand{\Bsigma}[0]{\boldsymbol{\sigma}}
%\newcommand{\Btau}[0]{\boldsymbol{\tau}}
%\newcommand{\Btheta}[0]{\boldsymbol{\theta}}
\newcommand{\Bupsilon}[0]{\boldsymbol{\upsilon}}
\newcommand{\Bvarepsilon}[0]{\boldsymbol{\varepsilon}}
\newcommand{\Bvarphi}[0]{\boldsymbol{\varphi}}
\newcommand{\Bvarpi}[0]{\boldsymbol{\varpi}}
\newcommand{\Bvarrho}[0]{\boldsymbol{\varrho}}
\newcommand{\Bvarsigma}[0]{\boldsymbol{\varsigma}}
\newcommand{\Bvartheta}[0]{\boldsymbol{\vartheta}}
\newcommand{\Bxi}[0]{\boldsymbol{\xi}}
\newcommand{\Bzeta}[0]{\boldsymbol{\zeta}}
%
%</bold and dot greek symbols>
%<infrequent>
%
%\newcommand{\AreaOp}[1]{\AName_{#1}}
%\newcommand{\Babs}[0]{\abs{\BB}}
%\newcommand{\Bcap}[0]{\hat{\BB}}
%\newcommand{\BrPrimeRej}[0]{\rcap(\rcap \wedge \Br')}
%\newcommand{\CA}[0]{\mathcal{A}}
%\newcommand{\Cos}[1]{\cos{\left({#1}\right)}}
%\newcommand{\Det}[1] {\abs{#1}}
%\newcommand{\Dsq}[2] {\frac {\partial^2 {#1}} {\partial {#2}^2}}
%\newcommand{\Exp}[1]{\exp{\left({#1}\right)}}
%\newcommand{\Norm}[1]{\left\lVert{#1}\right\rVert}
%\newcommand{\Sin}[1]{\sin{\left({#1}\right)}}
%\newcommand{\T}[0]{\text{T}}
%\newcommand{\VolumeOp}[1]{\VName_{#1}}
%\newcommand{\agrad}[0]{\Ba \cdot \nabla}
%\newcommand{\alphacap}[0]{\hat{\boldsymbol{\alpha}}}
%\newcommand{\Fcap}[0]{\hat{\BF}}
%\newcommand{\bithree}[0]{{\Bi}_3}
%\newcommand{\bxa}[0]{\Bx\Ba}
%\newcommand{\coordvec}[2]{
%\newcommand{\costheta}[0]{\acap \cdot \xcap}
%\newcommand{\ddt}[1]{\ddot{#1}}
%\newcommand{\ddu}[1] {\frac {d{#1}} {du}}
%\newcommand{\dsqxj}[2] {\frac {\partial^2 {#1}} {\partial {x_{#2}}^2}}
%\newcommand{\dtheta}[1]{\frac{d {#1}}{d \theta}}
%\newcommand{\dt}[1]{\dot{#1}}
%\newcommand{\dt}[1]{\frac{d {#1}}{dt}}
%\newcommand{\dxj}[2] {\frac {\partial {#1}} {\partial {x_{#2}}}}
%\newcommand{\halfPhi}[0]{\frac{\phi}{2}}
%\newcommand{\half}[0]{\inv{2}}
%\newcommand{\inv}[1]{\frac{1}{#1}}
%\newcommand{\laplacian}[0]{\nabla^2}
%\newcommand{\matrixoftx}[3]{
%\newcommand{\nrrp}[0]{\norm{\rcap \wedge \Br'}}
%\newcommand{\oiint}{\bigcirc \hspace{-1.4em} \int \hspace{-.8em} \int}
%\newcommand{\transpose}[1]{{#1}^{\text{T}}}
%\newcommand{\transpose}[1]{{{#1}^{\TextTranspose}}}
%\newcommand{\transpose}[1]{{{#1}^{\text{T}}}}
%\newcommand{\barA}[0]{\bar{A}}
%\newcommand{\qbar}[0]{\bar{q}}
%\newcommand{\qdotbar}[0]{\dot{\bar{q}}}
%
%</infrequent>





%\usepackage{listings}
%\usepackage{txfonts} % for ointctr... (also appears to make "prettier" \int and \sum's)
%\usepackage[bookmarks=true]{hyperref}

%\usepackage{color,cite,graphicx}
   % use colour in the document, put your citations as [1-4]
   % rather than [1,2,3,4] (it looks nicer, and the extended LaTeX2e
   % graphics package. 
%\usepackage{latexsym,amssymb,epsf} % do not remember if these are
   % needed, but their inclusion can not do any damage


\chapter{Commutator and Anti-Commutator Hermitian-ness}
\label{chap:commutatorHerm}
%\author{Peeter Joot \quad peeter.joot@gmail.com }
\date{ April 13, 2009.  commutatorHerm.tex }

%\begin{document}

%\maketitle{}

%\tableofcontents

\section{Motivation}

Reading of the proof of chapter 3, equation 12.7, in \citep{pauli2000wm},
that the anti-commutator 

\begin{equation}\label{eqn:commutatorHerm:20}
\begin{aligned}
\symmetric{F}{G} = {F G + G F}
\end{aligned}
\end{equation}

is Hermitian, is unclear to me.  Fill in the missing details.  Also prove 12.8, that 

\begin{equation}\label{eqn:commutatorHerm:40}
\begin{aligned}
i \antisymmetric{F}{G} = i (F G - G F)
\end{aligned}
\end{equation}

is Hermitian.

\section{Hermitian operator examples}

\subsection{Hermitian definition}

Pauli defines Hermitian in terms of the operator expectation value.  An operator \(H\) is Hermitian if

\begin{equation}\label{eqn:commutator_herm:Hermitian}
\begin{aligned}
\Expectation{H} = \int \psi^\conj (H \psi) d^3 x = \int \psi (H \psi)^\conj d^3 x
\end{aligned}
\end{equation}

Or
\begin{equation}\label{eqn:commutator_herm:HermitianInt}
\begin{aligned}
0 &= \Expectation{H} - \Expectation{H}^\conj = \int d^3 x \left( \psi^\conj (H \psi) - \psi (H \psi)^\conj \right)
\end{aligned}
\end{equation}

\subsection{Two operator form}

For completeness, let us derive the two wave function form of the Hermitian operator definition in full detail (omitted from the text).  With \(\psi = \psi_1 + \psi_2\)
\eqnref{eqn:commutator_herm:HermitianInt} becomes

\begin{equation}\label{eqn:commutatorHerm:60}
\begin{aligned}
\Expectation{H} - \Expectation{H}^\conj
&= \int (\psi_1 + \psi_2)^\conj (H (\psi_1 + \psi_2)) d^3 x - \int (\psi_1 + \psi_2) (H (\psi_1 + \psi_2))^\conj d^3 x \\
&= \int d^3 x \\
&\psi_1^\conj (H \psi_1) - \psi_1 (H \psi_1)^\conj 
+\psi_1^\conj (H \psi_2) - \psi_1 (H \psi_2)^\conj \\
&+\psi_2^\conj (H \psi_2) - \psi_2 (H \psi_2)^\conj 
+\psi_2^\conj (H \psi_1) - \psi_2 (H \psi_1)^\conj \\
&= \int d^3 x
\left( \psi_1^\conj (H \psi_2) - \psi_1 (H \psi_2)^\conj 
+\psi_2^\conj (H \psi_1) - \psi_2 (H \psi_1)^\conj \right) \\
\end{aligned}
\end{equation}

Grouping terms, we have 

\begin{equation}\label{eqn:commutatorHerm:80}
\begin{aligned}
\int d^3 x \left( \psi_1^\conj (H \psi_2) - \psi_2 (H \psi_1)^\conj \right) = \int d^3 x \left( \psi_1 (H \psi_2)^\conj - \psi_2^\conj (H \psi_1) \right)
\end{aligned}
\end{equation}

This is quite a bit different than both sides being separately zero, and the key to that further statement (as pointed out in 9.17 in \citep{bohm1989qt}) 
is that this is also true if the two wave function are adjusted by constant phase factors
\(\psi_1 \rightarrow \psi_1 e^{ia}\), 
\(\psi_2 \rightarrow \psi_2 e^{ib}\).  Doing so we have

\begin{equation}\label{eqn:commutatorHerm:100}
\begin{aligned}
e^{i(a-b)} \int d^3 x \left( \psi_1^\conj (H \psi_2) - \psi_2 (H \psi_1)^\conj \right) = e^{i(b-a)} \int d^3 x \left( \psi_1 (H \psi_2)^\conj - \psi_2^\conj (H \psi_1) \right)
\end{aligned}
\end{equation}

For this to hold for any \(a\), \(b\) both sides of the equation must separately equal zero, and we have

\begin{equation}\label{eqn:commutator_herm:HermitianTwo}
\begin{aligned}
\int d^3 x \psi_1^\conj (H \psi_2) = \int d^3 x \psi_2 (H \psi_1)^\conj
\end{aligned}
\end{equation}

\subsection{Lemma for repeated operators}

Next, examine the reversion behavior of repeated operators.  This appears to be used in the text (or is proved implicitly via some other operation not explained).

Given an pair of Hermitian operators, \(H_1\), and \(H_2\), 

\begin{equation}\label{eqn:commutatorHerm:120}
\begin{aligned}
\int d^3 x \psi_1^\conj (H_1 H_2 \psi_2)
\end{aligned}
\end{equation}

what do we get by reversing the operator action?  With the introduction of a couple of helper wave function variables, \(\epsilon = H_2 \psi_2\), and \(\beta = H_1 \psi_1\), 
this becomes straight forward to determine

\begin{equation}\label{eqn:commutatorHerm:140}
\begin{aligned}
\int d^3 x \psi_1^\conj (H_1 \epsilon)
&=
\int d^3 x \epsilon (H_1 \psi_1)^\conj \\
&=
\int d^3 x (H_2 \psi_2) (H_1 \psi_1)^\conj \\
&=
\int d^3 x \beta^\conj (H_2 \psi_2) \\
&=
\int d^3 x \psi_2 (H_2 \beta)^\conj \\
\end{aligned}
\end{equation}

So we have
\begin{equation}\label{eqn:commutator_herm:reverse}
\begin{aligned}
\int d^3 x \psi_1^\conj (H_1 H_2 \psi_2) &= \int d^3 x \psi_2 (H_2 H_1 \psi_1)^\conj 
\end{aligned}
\end{equation}

\subsection{Anti-commutator}

The statement that the anti-commutator is Hermitian means that we have

\begin{equation}\label{eqn:commutatorHerm:160}
\begin{aligned}
\int d^3 x \psi_2^\conj \left( \symmetric{F}{G} \psi_1 \right) &= \int d^3 x \psi_1 \left(\symmetric{F}{G} \psi_2 \right)^\conj \\
\text{or} \\
\int d^3 x \psi_2^\conj \left( (F G + G F) \psi_1 \right) &= \int d^3 x \psi_1 \left((F G + G F) \psi_2 \right)^\conj \\
\end{aligned}
\end{equation}

Let us expand the right hand side and see if we can get back the LHS
Or
\begin{equation}\label{eqn:commutatorHerm:180}
\begin{aligned}
\int d^3 x \psi_1 \left(\symmetric{F}{G} \psi_2 \right)^\conj 
&=
\int d^3 x \psi_1 \left((F G + G F) \psi_2 \right)^\conj \\
&=
\int d^3 x \psi_2^\conj \left((G F + F G) \psi_1 \right) \quad\quad \text{(applying \eqnref{eqn:commutator_herm:reverse} twice)} \\
&=
\int d^3 x \psi_2^\conj \left( \symmetric{F}{G} \psi_1 \right) \\
\end{aligned}
\end{equation}

Hmm.  That is the Hermitian identity of \eqnref{eqn:commutator_herm:HermitianTwo}, so we are done.  Not at all complicated after all (albeit less
general than the text where a result for a more general pair of operators was given).

\subsection{Commutator}

Now, how about the (imaginary scaled) commutator case?

\begin{equation}\label{eqn:commutatorHerm:200}
\begin{aligned}
\int d^3 x \psi_2^\conj \left( i \antisymmetric{F}{G} \psi_1 \right) &= \int d^3 x \psi_1 \left( i \antisymmetric{F}{G} \psi_2 \right)^\conj \\
\text{or} \\
\int d^3 x \psi_2^\conj \left( i(F G - G F) \psi_1 \right) &= \int d^3 x \psi_1 \left(i(F G - G F) \psi_2 \right)^\conj \\
\end{aligned}
\end{equation}

Again, let us try just expanding out the RHS

\begin{equation}\label{eqn:commutatorHerm:220}
\begin{aligned}
\int d^3 x \psi_1 \left( i \antisymmetric{F}{G} \psi_2 \right)^\conj 
&= \int d^3 x \psi_1 \left(i(F G - G F) \psi_2 \right)^\conj \\
&= -i \int d^3 x \psi_1 \left((F G - G F) \psi_2 \right)^\conj \\
&= -i \int d^3 x \psi_2^\conj \left((G F - F G) \psi_1 \right) \\
&= \int d^3 x \psi_2^\conj \left(i (F G - G F) \psi_1 \right) \\
&= \int d^3 x \psi_2^\conj \left(i \antisymmetric{F}{G}\psi_1 \right) \\
\end{aligned}
\end{equation}

QED.

\section{Future: Relation to Clifford product}

For vector spaces, as noted in \citep{gabook:PJpauliMatrix},
we can write the Clifford product of two \R{N} vectors in terms of commutators and anti-commutators

\begin{equation}\label{eqn:commutatorHerm:240}
\begin{aligned}
\Bf \Bg = \inv{2} \left( \symmetric{\Bf}{\Bg} + i \antisymmetric{\Bf}{\Bg} \right)
\end{aligned}
\end{equation}

where \(i\) is the pseudoscalar for the space.  So, while \(FG\) is not necessarily Hermitian, it is interesting that the composite operator
\(\symmetric{F}{G} + i \antisymmetric{F}{G}\), which is so close to the product operator of Euclidean vector spaces, is Hermitian.
Explore this geometric analogy later.

%\bibliographystyle{plainnat}
%\bibliography{myrefs}

%\end{document}
      % apr 13/09
%\documentclass{article}

%\usepackage{amsmath}
\usepackage{mathpazo}

%
% shorthand for bold symbols, convenient for vectors and matrices
%
\newcommand{\Ba}[0]{\mathbf{a}}
\newcommand{\Bb}[0]{\mathbf{b}}
\newcommand{\Bc}[0]{\mathbf{c}}
\newcommand{\Bd}[0]{\mathbf{d}}
\newcommand{\Be}[0]{\mathbf{e}}
\newcommand{\Bf}[0]{\mathbf{f}}
\newcommand{\Bg}[0]{\mathbf{g}}
\newcommand{\Bh}[0]{\mathbf{h}}
\newcommand{\Bi}[0]{\mathbf{i}}
\newcommand{\Bj}[0]{\mathbf{j}}
\newcommand{\Bk}[0]{\mathbf{k}}
\newcommand{\Bl}[0]{\mathbf{l}}
\newcommand{\Bm}[0]{\mathbf{m}}
\newcommand{\Bn}[0]{\mathbf{n}}
\newcommand{\Bo}[0]{\mathbf{o}}
\newcommand{\Bp}[0]{\mathbf{p}}
\newcommand{\Bq}[0]{\mathbf{q}}
\newcommand{\Br}[0]{\mathbf{r}}
\newcommand{\Bs}[0]{\mathbf{s}}
\newcommand{\Bt}[0]{\mathbf{t}}
\newcommand{\Bu}[0]{\mathbf{u}}
\newcommand{\Bv}[0]{\mathbf{v}}
\newcommand{\Bw}[0]{\mathbf{w}}
\newcommand{\Bx}[0]{\mathbf{x}}
\newcommand{\By}[0]{\mathbf{y}}
\newcommand{\Bz}[0]{\mathbf{z}}
\newcommand{\BA}[0]{\mathbf{A}}
\newcommand{\BB}[0]{\mathbf{B}}
\newcommand{\BC}[0]{\mathbf{C}}
\newcommand{\BD}[0]{\mathbf{D}}
\newcommand{\BE}[0]{\mathbf{E}}
\newcommand{\BF}[0]{\mathbf{F}}
\newcommand{\BG}[0]{\mathbf{G}}
\newcommand{\BH}[0]{\mathbf{H}}
\newcommand{\BI}[0]{\mathbf{I}}
\newcommand{\BJ}[0]{\mathbf{J}}
\newcommand{\BK}[0]{\mathbf{K}}
\newcommand{\BL}[0]{\mathbf{L}}
\newcommand{\BM}[0]{\mathbf{M}}
\newcommand{\BN}[0]{\mathbf{N}}
\newcommand{\BO}[0]{\mathbf{O}}
\newcommand{\BP}[0]{\mathbf{P}}
\newcommand{\BQ}[0]{\mathbf{Q}}
\newcommand{\BR}[0]{\mathbf{R}}
\newcommand{\BS}[0]{\mathbf{S}}
\newcommand{\BT}[0]{\mathbf{T}}
\newcommand{\BU}[0]{\mathbf{U}}
\newcommand{\BV}[0]{\mathbf{V}}
\newcommand{\BW}[0]{\mathbf{W}}
\newcommand{\BX}[0]{\mathbf{X}}
\newcommand{\BY}[0]{\mathbf{Y}}
\newcommand{\BZ}[0]{\mathbf{Z}}

\newcommand{\Bzero}[0]{\mathbf{0}}
\newcommand{\Btheta}[0]{\boldsymbol{\theta}}
\newcommand{\Btau}[0]{\boldsymbol{\tau}}
\newcommand{\Bomega}[0]{\boldsymbol{\omega}}

%
% shorthand for unit vectors
%
\newcommand{\acap}[0]{\hat{\Ba}}
\newcommand{\bcap}[0]{\hat{\Bb}}
\newcommand{\ccap}[0]{\hat{\Bc}}
\newcommand{\dcap}[0]{\hat{\Bd}}
\newcommand{\ecap}[0]{\hat{\Be}}
\newcommand{\fcap}[0]{\hat{\Bf}}
\newcommand{\gcap}[0]{\hat{\Bg}}
\newcommand{\hcap}[0]{\hat{\Bh}}
\newcommand{\icap}[0]{\hat{\Bi}}
\newcommand{\jcap}[0]{\hat{\Bj}}
\newcommand{\kcap}[0]{\hat{\Bk}}
\newcommand{\lcap}[0]{\hat{\Bl}}
\newcommand{\mcap}[0]{\hat{\Bm}}
\newcommand{\ncap}[0]{\hat{\Bn}}
\newcommand{\ocap}[0]{\hat{\Bo}}
\newcommand{\pcap}[0]{\hat{\Bp}}
\newcommand{\qcap}[0]{\hat{\Bq}}
\newcommand{\rcap}[0]{\hat{\Br}}
\newcommand{\scap}[0]{\hat{\Bs}}
\newcommand{\tcap}[0]{\hat{\Bt}}
\newcommand{\ucap}[0]{\hat{\Bu}}
\newcommand{\vcap}[0]{\hat{\Bv}}
\newcommand{\wcap}[0]{\hat{\Bw}}
\newcommand{\xcap}[0]{\hat{\Bx}}
\newcommand{\ycap}[0]{\hat{\By}}
\newcommand{\zcap}[0]{\hat{\Bz}}
\newcommand{\thetacap}[0]{\hat{\Btheta}}

%
% to write R^n and C^n in a distinguishable fashion.  Perhaps change this
% to the double lined characters upon figuring out how to do so.
%
\newcommand{\C}[1]{$\mathbb{C}^{#1}$}
\newcommand{\R}[1]{$\mathbb{R}^{#1}$}

%
% various generally useful helpers
%

% derivative of #1 wrt. #2:
\newcommand{\D}[2] {\frac {d#2} {d#1}}

\newcommand{\inv}[1]{\frac{1}{#1}}
\newcommand{\cross}[0]{\times}

\newcommand{\abs}[1]{\lvert{#1}\rvert}
\newcommand{\norm}[1]{\lVert{#1}\rVert}
\newcommand{\innerprod}[2]{\langle{#1}, {#2}\rangle}
\newcommand{\dotprod}[2]{{#1} \cdot {#2}}
\newcommand{\bdotprod}[2]{\left({#1} \cdot {#2}\right)}
\newcommand{\crossprod}[2]{{#1} \cross {#2}}
\newcommand{\tripleprod}[3]{\dotprod{\left(\crossprod{#1}{#2}\right)}{#3}}

\DeclareMathOperator{\Proj}{Proj}
\DeclareMathOperator{\Span}{span}
\DeclareMathOperator{\Sgn}{sgn}
\DeclareMathOperator{\Area}{Area}
\DeclareMathOperator{\Volume}{Volume}

%
% A few miscellaneous things specific to this document
%
\newcommand{\crossop}[1]{\crossprod{#1}{}}

% R2 vector.
\newcommand{\VectorTwo}[2]{
\begin{bmatrix}
 {#1} \\
 {#2}
\end{bmatrix}
}

\newcommand{\VectorN}[1]{
\begin{bmatrix}
{#1}_1 \\
{#1}_2 \\
\vdots \\
{#1}_N \\
\end{bmatrix}
}

\newcommand{\DETuvij}[4]{
\begin{vmatrix}
 {#1}_{#3} & {#1}_{#4} \\
 {#2}_{#3} & {#2}_{#4}
\end{vmatrix}
}

\newcommand{\DETuvwijk}[6]{
\begin{vmatrix}
 {#1}_{#4} & {#1}_{#5} & {#1}_{#6} \\
 {#2}_{#4} & {#2}_{#5} & {#2}_{#6} \\
 {#3}_{#4} & {#3}_{#5} & {#3}_{#6}
\end{vmatrix}
}

\newcommand{\DETuvwxijkl}[8]{
\begin{vmatrix}
 {#1}_{#5} & {#1}_{#6} & {#1}_{#7} & {#1}_{#8} \\
 {#2}_{#5} & {#2}_{#6} & {#2}_{#7} & {#2}_{#8} \\
 {#3}_{#5} & {#3}_{#6} & {#3}_{#7} & {#3}_{#8} \\
 {#4}_{#5} & {#4}_{#6} & {#4}_{#7} & {#4}_{#8} \\
\end{vmatrix}
}

%\newcommand{\DETuvwxyijklm}[10]{
%\begin{vmatrix}
% {#1}_{#6} & {#1}_{#7} & {#1}_{#8} & {#1}_{#9} & {#1}_{#10} \\
% {#2}_{#6} & {#2}_{#7} & {#2}_{#8} & {#2}_{#9} & {#2}_{#10} \\
% {#3}_{#6} & {#3}_{#7} & {#3}_{#8} & {#3}_{#9} & {#3}_{#10} \\
% {#4}_{#6} & {#4}_{#7} & {#4}_{#8} & {#4}_{#9} & {#4}_{#10} \\
% {#5}_{#6} & {#5}_{#7} & {#5}_{#8} & {#5}_{#9} & {#5}_{#10}
%\end{vmatrix}
%}

% R3 vector.
\newcommand{\VectorThree}[3]{
\begin{bmatrix}
 {#1} \\
 {#2} \\
 {#3}
\end{bmatrix}
}


%%<misc>
%
\newcommand{\Abs}[1]{{\left\lvert{#1}\right\rvert}}
\newcommand{\spacegrad}[0]{\boldsymbol{\nabla}}
\newcommand{\grad}[0]{\nabla}
\newcommand{\LL}[0]{\mathcal{L}}

% == \partial_{#1} {#2}
\newcommand{\PD}[2]{\frac{\partial {#2}}{\partial {#1}}}
% inline variant
\newcommand{\PDi}[2]{{\partial {#2}}/{\partial {#1}}}

\newcommand{\PDD}[3]{\frac{\partial^2 {#3}}{\partial {#1}\partial {#2}}}
%\newcommand{\PDd}[2]{\frac{\partial^2 {#2}}{{\partial{#1}}^2}}
\newcommand{\PDsq}[2]{\frac{\partial^2 {#2}}{(\partial {#1})^2}}

\newcommand{\Partial}[2]{\frac{\partial {#1}}{\partial {#2}}}
\DeclareMathOperator{\RejName}{Rej}
\newcommand{\Rej}[2]{\RejName_{#1}\left( {#2} \right)}
\newcommand{\Rm}[1]{\mathbb{R}^{#1}}
\newcommand{\Cm}[1]{\mathbb{C}^{#1}}
\newcommand{\conj}[0]{{*}}

%</misc>

% <grade selection>
%
\newcommand{\gpgrade}[2] {{\left\langle{{#1}}\right\rangle}_{#2}}

\newcommand{\gpgradezero}[1] {\gpgrade{#1}{}}
%\newcommand{\gpscalargrade}[1] {{\left\langle{{#1}}\right\rangle}}
%\newcommand{\gpgradezero}[1] {\gpgrade{#1}{0}}

%\newcommand{\gpgradeone}[1] {{\left\langle{{#1}}\right\rangle}_{1}}
\newcommand{\gpgradeone}[1] {\gpgrade{#1}{1}}

\newcommand{\gpgradetwo}[1] {\gpgrade{#1}{2}}
\newcommand{\gpgradethree}[1] {\gpgrade{#1}{3}}
\newcommand{\gpgradefour}[1] {\gpgrade{#1}{4}}
%
% </grade selection>



\newcommand{\adot}[0]{{\dot{a}}}
\newcommand{\bdot}[0]{{\dot{b}}}
% taken for centered dot:
%\newcommand{\cdot}[0]{{\dot{c}}}
%\newcommand{\ddot}[0]{{\dot{d}}}
\newcommand{\edot}[0]{{\dot{e}}}
\newcommand{\fdot}[0]{{\dot{f}}}
\newcommand{\gdot}[0]{{\dot{g}}}
\newcommand{\hdot}[0]{{\dot{h}}}
\newcommand{\idot}[0]{{\dot{i}}}
\newcommand{\jdot}[0]{{\dot{j}}}
\newcommand{\kdot}[0]{{\dot{k}}}
\newcommand{\ldot}[0]{{\dot{l}}}
\newcommand{\mdot}[0]{{\dot{m}}}
\newcommand{\ndot}[0]{{\dot{n}}}
%\newcommand{\odot}[0]{{\dot{o}}}
\newcommand{\pdot}[0]{{\dot{p}}}
\newcommand{\qdot}[0]{{\dot{q}}}
\newcommand{\rdot}[0]{{\dot{r}}}
\newcommand{\sdot}[0]{{\dot{s}}}
\newcommand{\tdot}[0]{{\dot{t}}}
\newcommand{\udot}[0]{{\dot{u}}}
\newcommand{\vdot}[0]{{\dot{v}}}
\newcommand{\wdot}[0]{{\dot{w}}}
\newcommand{\xdot}[0]{{\dot{x}}}
\newcommand{\ydot}[0]{{\dot{y}}}
\newcommand{\zdot}[0]{{\dot{z}}}
\newcommand{\addot}[0]{{\ddot{a}}}
\newcommand{\bddot}[0]{{\ddot{b}}}
\newcommand{\cddot}[0]{{\ddot{c}}}
%\newcommand{\dddot}[0]{{\ddot{d}}}
\newcommand{\eddot}[0]{{\ddot{e}}}
\newcommand{\fddot}[0]{{\ddot{f}}}
\newcommand{\gddot}[0]{{\ddot{g}}}
\newcommand{\hddot}[0]{{\ddot{h}}}
\newcommand{\iddot}[0]{{\ddot{i}}}
\newcommand{\jddot}[0]{{\ddot{j}}}
\newcommand{\kddot}[0]{{\ddot{k}}}
\newcommand{\lddot}[0]{{\ddot{l}}}
\newcommand{\mddot}[0]{{\ddot{m}}}
\newcommand{\nddot}[0]{{\ddot{n}}}
\newcommand{\oddot}[0]{{\ddot{o}}}
\newcommand{\pddot}[0]{{\ddot{p}}}
\newcommand{\qddot}[0]{{\ddot{q}}}
\newcommand{\rddot}[0]{{\ddot{r}}}
\newcommand{\sddot}[0]{{\ddot{s}}}
\newcommand{\tddot}[0]{{\ddot{t}}}
\newcommand{\uddot}[0]{{\ddot{u}}}
\newcommand{\vddot}[0]{{\ddot{v}}}
\newcommand{\wddot}[0]{{\ddot{w}}}
\newcommand{\xddot}[0]{{\ddot{x}}}
\newcommand{\yddot}[0]{{\ddot{y}}}
\newcommand{\zddot}[0]{{\ddot{z}}}

%<bold and dot greek symbols>
%

\newcommand{\Deltadot}[0]{{\dot{\Delta}}}
\newcommand{\Gammadot}[0]{{\dot{\Gamma}}}
\newcommand{\Lambdadot}[0]{{\dot{\Lambda}}}
\newcommand{\Omegadot}[0]{{\dot{\Omega}}}
\newcommand{\Phidot}[0]{{\dot{\Phi}}}
\newcommand{\Pidot}[0]{{\dot{\Pi}}}
\newcommand{\Psidot}[0]{{\dot{\Psi}}}
\newcommand{\Sigmadot}[0]{{\dot{\Sigma}}}
\newcommand{\Thetadot}[0]{{\dot{\Theta}}}
\newcommand{\Upsilondot}[0]{{\dot{\Upsilon}}}
\newcommand{\Xidot}[0]{{\dot{\Xi}}}
\newcommand{\alphadot}[0]{{\dot{\alpha}}}
\newcommand{\betadot}[0]{{\dot{\beta}}}
\newcommand{\chidot}[0]{{\dot{\chi}}}
\newcommand{\deltadot}[0]{{\dot{\delta}}}
\newcommand{\epsilondot}[0]{{\dot{\epsilon}}}
\newcommand{\etadot}[0]{{\dot{\eta}}}
\newcommand{\gammadot}[0]{{\dot{\gamma}}}
\newcommand{\kappadot}[0]{{\dot{\kappa}}}
\newcommand{\lambdadot}[0]{{\dot{\lambda}}}
\newcommand{\mudot}[0]{{\dot{\mu}}}
\newcommand{\nudot}[0]{{\dot{\nu}}}
\newcommand{\omegadot}[0]{{\dot{\omega}}}
\newcommand{\phidot}[0]{{\dot{\phi}}}
\newcommand{\pidot}[0]{{\dot{\pi}}}
\newcommand{\psidot}[0]{{\dot{\psi}}}
\newcommand{\rhodot}[0]{{\dot{\rho}}}
\newcommand{\sigmadot}[0]{{\dot{\sigma}}}
\newcommand{\taudot}[0]{{\dot{\tau}}}
\newcommand{\thetadot}[0]{{\dot{\theta}}}
\newcommand{\upsilondot}[0]{{\dot{\upsilon}}}
\newcommand{\varepsilondot}[0]{{\dot{\varepsilon}}}
\newcommand{\varphidot}[0]{{\dot{\varphi}}}
\newcommand{\varpidot}[0]{{\dot{\varpi}}}
\newcommand{\varrhodot}[0]{{\dot{\varrho}}}
\newcommand{\varsigmadot}[0]{{\dot{\varsigma}}}
\newcommand{\varthetadot}[0]{{\dot{\vartheta}}}
\newcommand{\xidot}[0]{{\dot{\xi}}}
\newcommand{\zetadot}[0]{{\dot{\zeta}}}

\newcommand{\Deltaddot}[0]{{\ddot{\Delta}}}
\newcommand{\Gammaddot}[0]{{\ddot{\Gamma}}}
\newcommand{\Lambdaddot}[0]{{\ddot{\Lambda}}}
\newcommand{\Omegaddot}[0]{{\ddot{\Omega}}}
\newcommand{\Phiddot}[0]{{\ddot{\Phi}}}
\newcommand{\Piddot}[0]{{\ddot{\Pi}}}
\newcommand{\Psiddot}[0]{{\ddot{\Psi}}}
\newcommand{\Sigmaddot}[0]{{\ddot{\Sigma}}}
\newcommand{\Thetaddot}[0]{{\ddot{\Theta}}}
\newcommand{\Upsilonddot}[0]{{\ddot{\Upsilon}}}
\newcommand{\Xiddot}[0]{{\ddot{\Xi}}}
\newcommand{\alphaddot}[0]{{\ddot{\alpha}}}
\newcommand{\betaddot}[0]{{\ddot{\beta}}}
\newcommand{\chiddot}[0]{{\ddot{\chi}}}
\newcommand{\deltaddot}[0]{{\ddot{\delta}}}
\newcommand{\epsilonddot}[0]{{\ddot{\epsilon}}}
\newcommand{\etaddot}[0]{{\ddot{\eta}}}
\newcommand{\gammaddot}[0]{{\ddot{\gamma}}}
\newcommand{\kappaddot}[0]{{\ddot{\kappa}}}
\newcommand{\lambdaddot}[0]{{\ddot{\lambda}}}
\newcommand{\muddot}[0]{{\ddot{\mu}}}
\newcommand{\nuddot}[0]{{\ddot{\nu}}}
\newcommand{\omegaddot}[0]{{\ddot{\omega}}}
\newcommand{\phiddot}[0]{{\ddot{\phi}}}
\newcommand{\piddot}[0]{{\ddot{\pi}}}
\newcommand{\psiddot}[0]{{\ddot{\psi}}}
\newcommand{\rhoddot}[0]{{\ddot{\rho}}}
\newcommand{\sigmaddot}[0]{{\ddot{\sigma}}}
\newcommand{\tauddot}[0]{{\ddot{\tau}}}
\newcommand{\thetaddot}[0]{{\ddot{\theta}}}
\newcommand{\upsilonddot}[0]{{\ddot{\upsilon}}}
\newcommand{\varepsilonddot}[0]{{\ddot{\varepsilon}}}
\newcommand{\varphiddot}[0]{{\ddot{\varphi}}}
\newcommand{\varpiddot}[0]{{\ddot{\varpi}}}
\newcommand{\varrhoddot}[0]{{\ddot{\varrho}}}
\newcommand{\varsigmaddot}[0]{{\ddot{\varsigma}}}
\newcommand{\varthetaddot}[0]{{\ddot{\vartheta}}}
\newcommand{\xiddot}[0]{{\ddot{\xi}}}
\newcommand{\zetaddot}[0]{{\ddot{\zeta}}}

\newcommand{\BDelta}[0]{\boldsymbol{\Delta}}
\newcommand{\BGamma}[0]{\boldsymbol{\Gamma}}
\newcommand{\BLambda}[0]{\boldsymbol{\Lambda}}
\newcommand{\BOmega}[0]{\boldsymbol{\Omega}}
\newcommand{\BPhi}[0]{\boldsymbol{\Phi}}
\newcommand{\BPi}[0]{\boldsymbol{\Pi}}
\newcommand{\BPsi}[0]{\boldsymbol{\Psi}}
\newcommand{\BSigma}[0]{\boldsymbol{\Sigma}}
\newcommand{\BTheta}[0]{\boldsymbol{\Theta}}
\newcommand{\BUpsilon}[0]{\boldsymbol{\Upsilon}}
\newcommand{\BXi}[0]{\boldsymbol{\Xi}}
\newcommand{\Balpha}[0]{\boldsymbol{\alpha}}
\newcommand{\Bbeta}[0]{\boldsymbol{\beta}}
\newcommand{\Bchi}[0]{\boldsymbol{\chi}}
\newcommand{\Bdelta}[0]{\boldsymbol{\delta}}
\newcommand{\Bepsilon}[0]{\boldsymbol{\epsilon}}
\newcommand{\Beta}[0]{\boldsymbol{\eta}}
\newcommand{\Bgamma}[0]{\boldsymbol{\gamma}}
\newcommand{\Bkappa}[0]{\boldsymbol{\kappa}}
\newcommand{\Blambda}[0]{\boldsymbol{\lambda}}
\newcommand{\Bmu}[0]{\boldsymbol{\mu}}
\newcommand{\Bnu}[0]{\boldsymbol{\nu}}
%\newcommand{\Bomega}[0]{\boldsymbol{\omega}}
\newcommand{\Bphi}[0]{\boldsymbol{\phi}}
\newcommand{\Bpi}[0]{\boldsymbol{\pi}}
\newcommand{\Bpsi}[0]{\boldsymbol{\psi}}
\newcommand{\Brho}[0]{\boldsymbol{\rho}}
\newcommand{\Bsigma}[0]{\boldsymbol{\sigma}}
%\newcommand{\Btau}[0]{\boldsymbol{\tau}}
%\newcommand{\Btheta}[0]{\boldsymbol{\theta}}
\newcommand{\Bupsilon}[0]{\boldsymbol{\upsilon}}
\newcommand{\Bvarepsilon}[0]{\boldsymbol{\varepsilon}}
\newcommand{\Bvarphi}[0]{\boldsymbol{\varphi}}
\newcommand{\Bvarpi}[0]{\boldsymbol{\varpi}}
\newcommand{\Bvarrho}[0]{\boldsymbol{\varrho}}
\newcommand{\Bvarsigma}[0]{\boldsymbol{\varsigma}}
\newcommand{\Bvartheta}[0]{\boldsymbol{\vartheta}}
\newcommand{\Bxi}[0]{\boldsymbol{\xi}}
\newcommand{\Bzeta}[0]{\boldsymbol{\zeta}}
%
%</bold and dot greek symbols>
%<infrequent>
%
%\newcommand{\AreaOp}[1]{\AName_{#1}}
%\newcommand{\Babs}[0]{\abs{\BB}}
%\newcommand{\Bcap}[0]{\hat{\BB}}
%\newcommand{\BrPrimeRej}[0]{\rcap(\rcap \wedge \Br')}
%\newcommand{\CA}[0]{\mathcal{A}}
%\newcommand{\Cos}[1]{\cos{\left({#1}\right)}}
%\newcommand{\Det}[1] {\abs{#1}}
%\newcommand{\Dsq}[2] {\frac {\partial^2 {#1}} {\partial {#2}^2}}
%\newcommand{\Exp}[1]{\exp{\left({#1}\right)}}
%\newcommand{\Norm}[1]{\left\lVert{#1}\right\rVert}
%\newcommand{\Sin}[1]{\sin{\left({#1}\right)}}
%\newcommand{\T}[0]{\text{T}}
%\newcommand{\VolumeOp}[1]{\VName_{#1}}
%\newcommand{\agrad}[0]{\Ba \cdot \nabla}
%\newcommand{\alphacap}[0]{\hat{\boldsymbol{\alpha}}}
%\newcommand{\Fcap}[0]{\hat{\BF}}
%\newcommand{\bithree}[0]{{\Bi}_3}
%\newcommand{\bxa}[0]{\Bx\Ba}
%\newcommand{\coordvec}[2]{
%\newcommand{\costheta}[0]{\acap \cdot \xcap}
%\newcommand{\ddt}[1]{\ddot{#1}}
%\newcommand{\ddu}[1] {\frac {d{#1}} {du}}
%\newcommand{\dsqxj}[2] {\frac {\partial^2 {#1}} {\partial {x_{#2}}^2}}
%\newcommand{\dtheta}[1]{\frac{d {#1}}{d \theta}}
%\newcommand{\dt}[1]{\dot{#1}}
%\newcommand{\dt}[1]{\frac{d {#1}}{dt}}
%\newcommand{\dxj}[2] {\frac {\partial {#1}} {\partial {x_{#2}}}}
%\newcommand{\halfPhi}[0]{\frac{\phi}{2}}
%\newcommand{\half}[0]{\inv{2}}
%\newcommand{\inv}[1]{\frac{1}{#1}}
%\newcommand{\laplacian}[0]{\nabla^2}
%\newcommand{\matrixoftx}[3]{
%\newcommand{\nrrp}[0]{\norm{\rcap \wedge \Br'}}
%\newcommand{\oiint}{\bigcirc \hspace{-1.4em} \int \hspace{-.8em} \int}
%\newcommand{\transpose}[1]{{#1}^{\text{T}}}
%\newcommand{\transpose}[1]{{{#1}^{\TextTranspose}}}
%\newcommand{\transpose}[1]{{{#1}^{\text{T}}}}
%\newcommand{\barA}[0]{\bar{A}}
%\newcommand{\qbar}[0]{\bar{q}}
%\newcommand{\qdotbar}[0]{\dot{\bar{q}}}
%
%</infrequent>





%\usepackage[bookmarks=true]{hyperref}

%\usepackage{color,cite,graphicx}
   % use colour in the document, put your citations as [1-4]
   % rather than [1,2,3,4] (it looks nicer, and the extended LaTeX2e
   % graphics package. 
%\usepackage{latexsym,amssymb,epsf} % don't remember if these are
   % needed, but their inclusion can't do any damage

\chapter{Evaluating the Gaussian integral. }
%\author{Peeter Joot \quad peeter.joot@gmail.com}
\date{ Jan 05, 2009.  $RCSfile: gaussian.tex,v $ Last $Revision: 1.12 $ $Date: 2009/06/11 17:00:37 $ }

%\begin{document}

%\maketitle{}
%\tableofcontents
\section{Plain old Gaussian. }

QM solutions appear to involve a lot of Gaussian integrals.  Looking at one
of the problems in \cite{mcmahon2005qmd} I tried to recall how to evaluate
the simplest of these.  Google says the trick is squaring and polar 
substitution.  Let's try this.

Solve
\begin{align*}
I = \int_{-\infty}^\infty e^{-\alpha x^2} dx
\end{align*}

\begin{align*}
I^2 
&= \int_{x= -\infty}^\infty e^{-\alpha x^2} dx \int_{y = -\infty}^\infty e^{-\alpha y^2} dy \\
&= \int_{\theta=0}^{2\pi}\int_{r= 0}^\infty e^{-\alpha r^2} r dr d\theta \\
&= 2\pi 
{\left.
\frac{e^{-\alpha r^2}}{-2\alpha}
\right\vert}_{r= 0}^\infty  \\
&= \frac{\pi}{\alpha}
\end{align*}

So we have

\begin{align*}
I = \sqrt{\frac{\pi}{\alpha}}
\end{align*}

\section{A couple higher order Gaussian's and normalization exercise. }

In order to do the normalization exercise for

\begin{align}\label{eqn:gaussian:exercise}
\psi = \left(A e^{-\frac{x^2}{a}} +B x e^{-\frac{x^2}{b}}\right) e^{-ict}
\end{align}

We want to calculate
\begin{align*}
\int \psi^\conj \psi = 
\Abs{A}^2 e^{-2\frac{x^2}{a}} +\Abs{B}^2 x^2 e^{-2\frac{x^2}{b}}
+ (A\bar{B} + B\bar{A}) x e^{-x^2\left(\inv{a} + \inv{b}\right)}
\end{align*}

so we need the $n=1,2$ versions of the following Gaussian integrals

\begin{align*}
I_n = \int x^n e^{-\alpha x^2} dx
\end{align*}

The $n=1$ case is directly integrable:

\begin{align*}
I_1 
&= \int x e^{-\alpha x^2} dx \\
&= \int \left(\frac{e^{-\alpha x^2}}{-2\alpha}\right)' dx \\
&= 0
\end{align*}

(integration bounds are $\pm \infty$ so the exponential vanishes).

Next.  $I_2$ follows with integration by parts

\begin{align*}
I_2 
&= \int x^2 e^{-\alpha x^2} dx \\
&= \int x \left(x e^{-\alpha x^2}\right) dx \\
&= \int x \left(\frac{e^{-\alpha x^2}}{-2\alpha}\right)' dx \\
&= -\int \frac{e^{-\alpha x^2}}{-2\alpha} dx \\
&= \inv{2\alpha}\int e^{-\alpha x^2} dx \\
&= \inv{2\alpha} \sqrt{\frac{\pi}{\alpha}}
\end{align*}

So the normalization required for \ref{eqn:gaussian:exercise} is
\begin{align*}
1 = \Abs{A}^2 \sqrt{\frac{\pi a}{2}} + \frac{\Abs{B}^2}{2} 
\sqrt{\frac{\pi b}{2}} \frac{b}{2}
\end{align*}

The values $a$, and $b$ are presumably due to boundary conditions, and this then fixes $\Abs{A}$ in terms of $\Abs{B}$ or the other way around.

\section{Generalized. }

Let 

\begin{align}
I_n = \int_{-\infty}^\infty x^n e^{-\alpha x^2} dx
\end{align}

We've solved this for $I_0 = \sqrt{\pi/\alpha}$, $I_1 = 0$, and $I_2$.  A quick calculation shows that $I_{2k+1} = 0$ too:

\begin{align*}
I_n 
&= \int_{-\infty}^\infty x^n e^{-\alpha x^2} dx \\
&= \int_{0}^\infty x^n e^{-\alpha x^2} dx +\int_{-\infty}^0 x^n e^{-\alpha x^2} dx \\
&= \int_{0}^\infty x^n e^{-\alpha x^2} dx +\int_{\infty}^0 (-x)^n e^{-\alpha x^2} (-dx) \\
&= \int_{0}^\infty x^n e^{-\alpha x^2} dx +\int_0^{\infty} (-x)^n e^{-\alpha x^2} dx \\
&= \int_{0}^\infty (x^n + (-x)^n)e^{-\alpha x^2} dx \\
\end{align*}

But if $n$ is odd $(-x)^n = -x^n$, so this is zero.

Now, for $n$ even, we can integrate by parts, as done for $I_2$.

\begin{align*}
I_{2m}
&= \int x^{2m} e^{-\alpha x^2} dx \\
&= \int x^{2m-1} \left(x e^{-\alpha x^2}\right) dx \\
&= \int x^{2m-1} \left(\frac{e^{-\alpha x^2}}{-2\alpha}\right)' dx \\
&= -\int (2m-1) x^{2m-2} \frac{e^{-\alpha x^2}}{-2\alpha} dx \\
\end{align*}

This gives us a recurrence relationship for the even order terms
\begin{align}
I_{2m} &= \frac{2m-1}{2\alpha} I_{2m-2}.
%&= \inv{2\alpha} \sqrt{\frac{\pi}{\alpha}}
\end{align}

Expanding this explicitly for the first few $m$ shows the pattern

\begin{align*}
\begin{array}{l l l}
I_2 &= \frac{2-1}{2\alpha} I_0 &= \frac{1}{2\alpha} \sqrt{\frac{\pi}{\alpha}} \\
I_4 &= \frac{4-1}{2\alpha}\frac{2-1}{2\alpha} I_0 &= \frac{3.1}{(2\alpha)^2} \sqrt{\frac{\pi}{\alpha}} \\
I_6 &= \frac{6-1}{2\alpha}\frac{4-1}{2\alpha}\frac{2-1}{2\alpha} I_0 &= \frac{5.3.1}{(2\alpha)^3} \sqrt{\frac{\pi}{\alpha}} \\
\end{array}
\end{align*}

Or
\begin{align}
I_{0} &= \sqrt{\frac{\pi}{\alpha}} \\
I_{2m-1} &= 0 \\
I_{2m} &= \frac{(2m-1)(2m-3)\cdots(3)(1)}{(2\alpha)^m} \sqrt{\frac{\pi}{\alpha}}
\end{align}

%\bibliographystyle{plainnat}
%\bibliography{myrefs}

%\end{document}
            % jan 05/09
\documentclass{article}

\usepackage{amsmath}
\usepackage{mathpazo}

%
% shorthand for bold symbols, convenient for vectors and matrices
%
\newcommand{\Ba}[0]{\mathbf{a}}
\newcommand{\Bb}[0]{\mathbf{b}}
\newcommand{\Bc}[0]{\mathbf{c}}
\newcommand{\Bd}[0]{\mathbf{d}}
\newcommand{\Be}[0]{\mathbf{e}}
\newcommand{\Bf}[0]{\mathbf{f}}
\newcommand{\Bg}[0]{\mathbf{g}}
\newcommand{\Bh}[0]{\mathbf{h}}
\newcommand{\Bi}[0]{\mathbf{i}}
\newcommand{\Bj}[0]{\mathbf{j}}
\newcommand{\Bk}[0]{\mathbf{k}}
\newcommand{\Bl}[0]{\mathbf{l}}
\newcommand{\Bm}[0]{\mathbf{m}}
\newcommand{\Bn}[0]{\mathbf{n}}
\newcommand{\Bo}[0]{\mathbf{o}}
\newcommand{\Bp}[0]{\mathbf{p}}
\newcommand{\Bq}[0]{\mathbf{q}}
\newcommand{\Br}[0]{\mathbf{r}}
\newcommand{\Bs}[0]{\mathbf{s}}
\newcommand{\Bt}[0]{\mathbf{t}}
\newcommand{\Bu}[0]{\mathbf{u}}
\newcommand{\Bv}[0]{\mathbf{v}}
\newcommand{\Bw}[0]{\mathbf{w}}
\newcommand{\Bx}[0]{\mathbf{x}}
\newcommand{\By}[0]{\mathbf{y}}
\newcommand{\Bz}[0]{\mathbf{z}}
\newcommand{\BA}[0]{\mathbf{A}}
\newcommand{\BB}[0]{\mathbf{B}}
\newcommand{\BC}[0]{\mathbf{C}}
\newcommand{\BD}[0]{\mathbf{D}}
\newcommand{\BE}[0]{\mathbf{E}}
\newcommand{\BF}[0]{\mathbf{F}}
\newcommand{\BG}[0]{\mathbf{G}}
\newcommand{\BH}[0]{\mathbf{H}}
\newcommand{\BI}[0]{\mathbf{I}}
\newcommand{\BJ}[0]{\mathbf{J}}
\newcommand{\BK}[0]{\mathbf{K}}
\newcommand{\BL}[0]{\mathbf{L}}
\newcommand{\BM}[0]{\mathbf{M}}
\newcommand{\BN}[0]{\mathbf{N}}
\newcommand{\BO}[0]{\mathbf{O}}
\newcommand{\BP}[0]{\mathbf{P}}
\newcommand{\BQ}[0]{\mathbf{Q}}
\newcommand{\BR}[0]{\mathbf{R}}
\newcommand{\BS}[0]{\mathbf{S}}
\newcommand{\BT}[0]{\mathbf{T}}
\newcommand{\BU}[0]{\mathbf{U}}
\newcommand{\BV}[0]{\mathbf{V}}
\newcommand{\BW}[0]{\mathbf{W}}
\newcommand{\BX}[0]{\mathbf{X}}
\newcommand{\BY}[0]{\mathbf{Y}}
\newcommand{\BZ}[0]{\mathbf{Z}}

\newcommand{\Bzero}[0]{\mathbf{0}}
\newcommand{\Btheta}[0]{\boldsymbol{\theta}}
\newcommand{\Btau}[0]{\boldsymbol{\tau}}
\newcommand{\Bomega}[0]{\boldsymbol{\omega}}

%
% shorthand for unit vectors
%
\newcommand{\acap}[0]{\hat{\Ba}}
\newcommand{\bcap}[0]{\hat{\Bb}}
\newcommand{\ccap}[0]{\hat{\Bc}}
\newcommand{\dcap}[0]{\hat{\Bd}}
\newcommand{\ecap}[0]{\hat{\Be}}
\newcommand{\fcap}[0]{\hat{\Bf}}
\newcommand{\gcap}[0]{\hat{\Bg}}
\newcommand{\hcap}[0]{\hat{\Bh}}
\newcommand{\icap}[0]{\hat{\Bi}}
\newcommand{\jcap}[0]{\hat{\Bj}}
\newcommand{\kcap}[0]{\hat{\Bk}}
\newcommand{\lcap}[0]{\hat{\Bl}}
\newcommand{\mcap}[0]{\hat{\Bm}}
\newcommand{\ncap}[0]{\hat{\Bn}}
\newcommand{\ocap}[0]{\hat{\Bo}}
\newcommand{\pcap}[0]{\hat{\Bp}}
\newcommand{\qcap}[0]{\hat{\Bq}}
\newcommand{\rcap}[0]{\hat{\Br}}
\newcommand{\scap}[0]{\hat{\Bs}}
\newcommand{\tcap}[0]{\hat{\Bt}}
\newcommand{\ucap}[0]{\hat{\Bu}}
\newcommand{\vcap}[0]{\hat{\Bv}}
\newcommand{\wcap}[0]{\hat{\Bw}}
\newcommand{\xcap}[0]{\hat{\Bx}}
\newcommand{\ycap}[0]{\hat{\By}}
\newcommand{\zcap}[0]{\hat{\Bz}}
\newcommand{\thetacap}[0]{\hat{\Btheta}}

%
% to write R^n and C^n in a distinguishable fashion.  Perhaps change this
% to the double lined characters upon figuring out how to do so.
%
\newcommand{\C}[1]{$\mathbb{C}^{#1}$}
\newcommand{\R}[1]{$\mathbb{R}^{#1}$}

%
% various generally useful helpers
%

% derivative of #1 wrt. #2:
\newcommand{\D}[2] {\frac {d#2} {d#1}}

\newcommand{\inv}[1]{\frac{1}{#1}}
\newcommand{\cross}[0]{\times}

\newcommand{\abs}[1]{\lvert{#1}\rvert}
\newcommand{\norm}[1]{\lVert{#1}\rVert}
\newcommand{\innerprod}[2]{\langle{#1}, {#2}\rangle}
\newcommand{\dotprod}[2]{{#1} \cdot {#2}}
\newcommand{\bdotprod}[2]{\left({#1} \cdot {#2}\right)}
\newcommand{\crossprod}[2]{{#1} \cross {#2}}
\newcommand{\tripleprod}[3]{\dotprod{\left(\crossprod{#1}{#2}\right)}{#3}}

\DeclareMathOperator{\Proj}{Proj}
\DeclareMathOperator{\Span}{span}
\DeclareMathOperator{\Sgn}{sgn}
\DeclareMathOperator{\Area}{Area}
\DeclareMathOperator{\Volume}{Volume}

%
% A few miscellaneous things specific to this document
%
\newcommand{\crossop}[1]{\crossprod{#1}{}}

% R2 vector.
\newcommand{\VectorTwo}[2]{
\begin{bmatrix}
 {#1} \\
 {#2}
\end{bmatrix}
}

\newcommand{\VectorN}[1]{
\begin{bmatrix}
{#1}_1 \\
{#1}_2 \\
\vdots \\
{#1}_N \\
\end{bmatrix}
}

\newcommand{\DETuvij}[4]{
\begin{vmatrix}
 {#1}_{#3} & {#1}_{#4} \\
 {#2}_{#3} & {#2}_{#4}
\end{vmatrix}
}

\newcommand{\DETuvwijk}[6]{
\begin{vmatrix}
 {#1}_{#4} & {#1}_{#5} & {#1}_{#6} \\
 {#2}_{#4} & {#2}_{#5} & {#2}_{#6} \\
 {#3}_{#4} & {#3}_{#5} & {#3}_{#6}
\end{vmatrix}
}

\newcommand{\DETuvwxijkl}[8]{
\begin{vmatrix}
 {#1}_{#5} & {#1}_{#6} & {#1}_{#7} & {#1}_{#8} \\
 {#2}_{#5} & {#2}_{#6} & {#2}_{#7} & {#2}_{#8} \\
 {#3}_{#5} & {#3}_{#6} & {#3}_{#7} & {#3}_{#8} \\
 {#4}_{#5} & {#4}_{#6} & {#4}_{#7} & {#4}_{#8} \\
\end{vmatrix}
}

%\newcommand{\DETuvwxyijklm}[10]{
%\begin{vmatrix}
% {#1}_{#6} & {#1}_{#7} & {#1}_{#8} & {#1}_{#9} & {#1}_{#10} \\
% {#2}_{#6} & {#2}_{#7} & {#2}_{#8} & {#2}_{#9} & {#2}_{#10} \\
% {#3}_{#6} & {#3}_{#7} & {#3}_{#8} & {#3}_{#9} & {#3}_{#10} \\
% {#4}_{#6} & {#4}_{#7} & {#4}_{#8} & {#4}_{#9} & {#4}_{#10} \\
% {#5}_{#6} & {#5}_{#7} & {#5}_{#8} & {#5}_{#9} & {#5}_{#10}
%\end{vmatrix}
%}

% R3 vector.
\newcommand{\VectorThree}[3]{
\begin{bmatrix}
 {#1} \\
 {#2} \\
 {#3}
\end{bmatrix}
}


%<misc>
%
\newcommand{\Abs}[1]{{\left\lvert{#1}\right\rvert}}
\newcommand{\spacegrad}[0]{\boldsymbol{\nabla}}
\newcommand{\grad}[0]{\nabla}
\newcommand{\LL}[0]{\mathcal{L}}

% == \partial_{#1} {#2}
\newcommand{\PD}[2]{\frac{\partial {#2}}{\partial {#1}}}
% inline variant
\newcommand{\PDi}[2]{{\partial {#2}}/{\partial {#1}}}

\newcommand{\PDD}[3]{\frac{\partial^2 {#3}}{\partial {#1}\partial {#2}}}
%\newcommand{\PDd}[2]{\frac{\partial^2 {#2}}{{\partial{#1}}^2}}
\newcommand{\PDsq}[2]{\frac{\partial^2 {#2}}{(\partial {#1})^2}}

\newcommand{\Partial}[2]{\frac{\partial {#1}}{\partial {#2}}}
\DeclareMathOperator{\RejName}{Rej}
\newcommand{\Rej}[2]{\RejName_{#1}\left( {#2} \right)}
\newcommand{\Rm}[1]{\mathbb{R}^{#1}}
\newcommand{\Cm}[1]{\mathbb{C}^{#1}}
\newcommand{\conj}[0]{{*}}

%</misc>

% <grade selection>
%
\newcommand{\gpgrade}[2] {{\left\langle{{#1}}\right\rangle}_{#2}}

\newcommand{\gpgradezero}[1] {\gpgrade{#1}{}}
%\newcommand{\gpscalargrade}[1] {{\left\langle{{#1}}\right\rangle}}
%\newcommand{\gpgradezero}[1] {\gpgrade{#1}{0}}

%\newcommand{\gpgradeone}[1] {{\left\langle{{#1}}\right\rangle}_{1}}
\newcommand{\gpgradeone}[1] {\gpgrade{#1}{1}}

\newcommand{\gpgradetwo}[1] {\gpgrade{#1}{2}}
\newcommand{\gpgradethree}[1] {\gpgrade{#1}{3}}
\newcommand{\gpgradefour}[1] {\gpgrade{#1}{4}}
%
% </grade selection>



\newcommand{\adot}[0]{{\dot{a}}}
\newcommand{\bdot}[0]{{\dot{b}}}
% taken for centered dot:
%\newcommand{\cdot}[0]{{\dot{c}}}
%\newcommand{\ddot}[0]{{\dot{d}}}
\newcommand{\edot}[0]{{\dot{e}}}
\newcommand{\fdot}[0]{{\dot{f}}}
\newcommand{\gdot}[0]{{\dot{g}}}
\newcommand{\hdot}[0]{{\dot{h}}}
\newcommand{\idot}[0]{{\dot{i}}}
\newcommand{\jdot}[0]{{\dot{j}}}
\newcommand{\kdot}[0]{{\dot{k}}}
\newcommand{\ldot}[0]{{\dot{l}}}
\newcommand{\mdot}[0]{{\dot{m}}}
\newcommand{\ndot}[0]{{\dot{n}}}
%\newcommand{\odot}[0]{{\dot{o}}}
\newcommand{\pdot}[0]{{\dot{p}}}
\newcommand{\qdot}[0]{{\dot{q}}}
\newcommand{\rdot}[0]{{\dot{r}}}
\newcommand{\sdot}[0]{{\dot{s}}}
\newcommand{\tdot}[0]{{\dot{t}}}
\newcommand{\udot}[0]{{\dot{u}}}
\newcommand{\vdot}[0]{{\dot{v}}}
\newcommand{\wdot}[0]{{\dot{w}}}
\newcommand{\xdot}[0]{{\dot{x}}}
\newcommand{\ydot}[0]{{\dot{y}}}
\newcommand{\zdot}[0]{{\dot{z}}}
\newcommand{\addot}[0]{{\ddot{a}}}
\newcommand{\bddot}[0]{{\ddot{b}}}
\newcommand{\cddot}[0]{{\ddot{c}}}
%\newcommand{\dddot}[0]{{\ddot{d}}}
\newcommand{\eddot}[0]{{\ddot{e}}}
\newcommand{\fddot}[0]{{\ddot{f}}}
\newcommand{\gddot}[0]{{\ddot{g}}}
\newcommand{\hddot}[0]{{\ddot{h}}}
\newcommand{\iddot}[0]{{\ddot{i}}}
\newcommand{\jddot}[0]{{\ddot{j}}}
\newcommand{\kddot}[0]{{\ddot{k}}}
\newcommand{\lddot}[0]{{\ddot{l}}}
\newcommand{\mddot}[0]{{\ddot{m}}}
\newcommand{\nddot}[0]{{\ddot{n}}}
\newcommand{\oddot}[0]{{\ddot{o}}}
\newcommand{\pddot}[0]{{\ddot{p}}}
\newcommand{\qddot}[0]{{\ddot{q}}}
\newcommand{\rddot}[0]{{\ddot{r}}}
\newcommand{\sddot}[0]{{\ddot{s}}}
\newcommand{\tddot}[0]{{\ddot{t}}}
\newcommand{\uddot}[0]{{\ddot{u}}}
\newcommand{\vddot}[0]{{\ddot{v}}}
\newcommand{\wddot}[0]{{\ddot{w}}}
\newcommand{\xddot}[0]{{\ddot{x}}}
\newcommand{\yddot}[0]{{\ddot{y}}}
\newcommand{\zddot}[0]{{\ddot{z}}}

%<bold and dot greek symbols>
%

\newcommand{\Deltadot}[0]{{\dot{\Delta}}}
\newcommand{\Gammadot}[0]{{\dot{\Gamma}}}
\newcommand{\Lambdadot}[0]{{\dot{\Lambda}}}
\newcommand{\Omegadot}[0]{{\dot{\Omega}}}
\newcommand{\Phidot}[0]{{\dot{\Phi}}}
\newcommand{\Pidot}[0]{{\dot{\Pi}}}
\newcommand{\Psidot}[0]{{\dot{\Psi}}}
\newcommand{\Sigmadot}[0]{{\dot{\Sigma}}}
\newcommand{\Thetadot}[0]{{\dot{\Theta}}}
\newcommand{\Upsilondot}[0]{{\dot{\Upsilon}}}
\newcommand{\Xidot}[0]{{\dot{\Xi}}}
\newcommand{\alphadot}[0]{{\dot{\alpha}}}
\newcommand{\betadot}[0]{{\dot{\beta}}}
\newcommand{\chidot}[0]{{\dot{\chi}}}
\newcommand{\deltadot}[0]{{\dot{\delta}}}
\newcommand{\epsilondot}[0]{{\dot{\epsilon}}}
\newcommand{\etadot}[0]{{\dot{\eta}}}
\newcommand{\gammadot}[0]{{\dot{\gamma}}}
\newcommand{\kappadot}[0]{{\dot{\kappa}}}
\newcommand{\lambdadot}[0]{{\dot{\lambda}}}
\newcommand{\mudot}[0]{{\dot{\mu}}}
\newcommand{\nudot}[0]{{\dot{\nu}}}
\newcommand{\omegadot}[0]{{\dot{\omega}}}
\newcommand{\phidot}[0]{{\dot{\phi}}}
\newcommand{\pidot}[0]{{\dot{\pi}}}
\newcommand{\psidot}[0]{{\dot{\psi}}}
\newcommand{\rhodot}[0]{{\dot{\rho}}}
\newcommand{\sigmadot}[0]{{\dot{\sigma}}}
\newcommand{\taudot}[0]{{\dot{\tau}}}
\newcommand{\thetadot}[0]{{\dot{\theta}}}
\newcommand{\upsilondot}[0]{{\dot{\upsilon}}}
\newcommand{\varepsilondot}[0]{{\dot{\varepsilon}}}
\newcommand{\varphidot}[0]{{\dot{\varphi}}}
\newcommand{\varpidot}[0]{{\dot{\varpi}}}
\newcommand{\varrhodot}[0]{{\dot{\varrho}}}
\newcommand{\varsigmadot}[0]{{\dot{\varsigma}}}
\newcommand{\varthetadot}[0]{{\dot{\vartheta}}}
\newcommand{\xidot}[0]{{\dot{\xi}}}
\newcommand{\zetadot}[0]{{\dot{\zeta}}}

\newcommand{\Deltaddot}[0]{{\ddot{\Delta}}}
\newcommand{\Gammaddot}[0]{{\ddot{\Gamma}}}
\newcommand{\Lambdaddot}[0]{{\ddot{\Lambda}}}
\newcommand{\Omegaddot}[0]{{\ddot{\Omega}}}
\newcommand{\Phiddot}[0]{{\ddot{\Phi}}}
\newcommand{\Piddot}[0]{{\ddot{\Pi}}}
\newcommand{\Psiddot}[0]{{\ddot{\Psi}}}
\newcommand{\Sigmaddot}[0]{{\ddot{\Sigma}}}
\newcommand{\Thetaddot}[0]{{\ddot{\Theta}}}
\newcommand{\Upsilonddot}[0]{{\ddot{\Upsilon}}}
\newcommand{\Xiddot}[0]{{\ddot{\Xi}}}
\newcommand{\alphaddot}[0]{{\ddot{\alpha}}}
\newcommand{\betaddot}[0]{{\ddot{\beta}}}
\newcommand{\chiddot}[0]{{\ddot{\chi}}}
\newcommand{\deltaddot}[0]{{\ddot{\delta}}}
\newcommand{\epsilonddot}[0]{{\ddot{\epsilon}}}
\newcommand{\etaddot}[0]{{\ddot{\eta}}}
\newcommand{\gammaddot}[0]{{\ddot{\gamma}}}
\newcommand{\kappaddot}[0]{{\ddot{\kappa}}}
\newcommand{\lambdaddot}[0]{{\ddot{\lambda}}}
\newcommand{\muddot}[0]{{\ddot{\mu}}}
\newcommand{\nuddot}[0]{{\ddot{\nu}}}
\newcommand{\omegaddot}[0]{{\ddot{\omega}}}
\newcommand{\phiddot}[0]{{\ddot{\phi}}}
\newcommand{\piddot}[0]{{\ddot{\pi}}}
\newcommand{\psiddot}[0]{{\ddot{\psi}}}
\newcommand{\rhoddot}[0]{{\ddot{\rho}}}
\newcommand{\sigmaddot}[0]{{\ddot{\sigma}}}
\newcommand{\tauddot}[0]{{\ddot{\tau}}}
\newcommand{\thetaddot}[0]{{\ddot{\theta}}}
\newcommand{\upsilonddot}[0]{{\ddot{\upsilon}}}
\newcommand{\varepsilonddot}[0]{{\ddot{\varepsilon}}}
\newcommand{\varphiddot}[0]{{\ddot{\varphi}}}
\newcommand{\varpiddot}[0]{{\ddot{\varpi}}}
\newcommand{\varrhoddot}[0]{{\ddot{\varrho}}}
\newcommand{\varsigmaddot}[0]{{\ddot{\varsigma}}}
\newcommand{\varthetaddot}[0]{{\ddot{\vartheta}}}
\newcommand{\xiddot}[0]{{\ddot{\xi}}}
\newcommand{\zetaddot}[0]{{\ddot{\zeta}}}

\newcommand{\BDelta}[0]{\boldsymbol{\Delta}}
\newcommand{\BGamma}[0]{\boldsymbol{\Gamma}}
\newcommand{\BLambda}[0]{\boldsymbol{\Lambda}}
\newcommand{\BOmega}[0]{\boldsymbol{\Omega}}
\newcommand{\BPhi}[0]{\boldsymbol{\Phi}}
\newcommand{\BPi}[0]{\boldsymbol{\Pi}}
\newcommand{\BPsi}[0]{\boldsymbol{\Psi}}
\newcommand{\BSigma}[0]{\boldsymbol{\Sigma}}
\newcommand{\BTheta}[0]{\boldsymbol{\Theta}}
\newcommand{\BUpsilon}[0]{\boldsymbol{\Upsilon}}
\newcommand{\BXi}[0]{\boldsymbol{\Xi}}
\newcommand{\Balpha}[0]{\boldsymbol{\alpha}}
\newcommand{\Bbeta}[0]{\boldsymbol{\beta}}
\newcommand{\Bchi}[0]{\boldsymbol{\chi}}
\newcommand{\Bdelta}[0]{\boldsymbol{\delta}}
\newcommand{\Bepsilon}[0]{\boldsymbol{\epsilon}}
\newcommand{\Beta}[0]{\boldsymbol{\eta}}
\newcommand{\Bgamma}[0]{\boldsymbol{\gamma}}
\newcommand{\Bkappa}[0]{\boldsymbol{\kappa}}
\newcommand{\Blambda}[0]{\boldsymbol{\lambda}}
\newcommand{\Bmu}[0]{\boldsymbol{\mu}}
\newcommand{\Bnu}[0]{\boldsymbol{\nu}}
%\newcommand{\Bomega}[0]{\boldsymbol{\omega}}
\newcommand{\Bphi}[0]{\boldsymbol{\phi}}
\newcommand{\Bpi}[0]{\boldsymbol{\pi}}
\newcommand{\Bpsi}[0]{\boldsymbol{\psi}}
\newcommand{\Brho}[0]{\boldsymbol{\rho}}
\newcommand{\Bsigma}[0]{\boldsymbol{\sigma}}
%\newcommand{\Btau}[0]{\boldsymbol{\tau}}
%\newcommand{\Btheta}[0]{\boldsymbol{\theta}}
\newcommand{\Bupsilon}[0]{\boldsymbol{\upsilon}}
\newcommand{\Bvarepsilon}[0]{\boldsymbol{\varepsilon}}
\newcommand{\Bvarphi}[0]{\boldsymbol{\varphi}}
\newcommand{\Bvarpi}[0]{\boldsymbol{\varpi}}
\newcommand{\Bvarrho}[0]{\boldsymbol{\varrho}}
\newcommand{\Bvarsigma}[0]{\boldsymbol{\varsigma}}
\newcommand{\Bvartheta}[0]{\boldsymbol{\vartheta}}
\newcommand{\Bxi}[0]{\boldsymbol{\xi}}
\newcommand{\Bzeta}[0]{\boldsymbol{\zeta}}
%
%</bold and dot greek symbols>
%<infrequent>
%
%\newcommand{\AreaOp}[1]{\AName_{#1}}
%\newcommand{\Babs}[0]{\abs{\BB}}
%\newcommand{\Bcap}[0]{\hat{\BB}}
%\newcommand{\BrPrimeRej}[0]{\rcap(\rcap \wedge \Br')}
%\newcommand{\CA}[0]{\mathcal{A}}
%\newcommand{\Cos}[1]{\cos{\left({#1}\right)}}
%\newcommand{\Det}[1] {\abs{#1}}
%\newcommand{\Dsq}[2] {\frac {\partial^2 {#1}} {\partial {#2}^2}}
%\newcommand{\Exp}[1]{\exp{\left({#1}\right)}}
%\newcommand{\Norm}[1]{\left\lVert{#1}\right\rVert}
%\newcommand{\Sin}[1]{\sin{\left({#1}\right)}}
%\newcommand{\T}[0]{\text{T}}
%\newcommand{\VolumeOp}[1]{\VName_{#1}}
%\newcommand{\agrad}[0]{\Ba \cdot \nabla}
%\newcommand{\alphacap}[0]{\hat{\boldsymbol{\alpha}}}
%\newcommand{\Fcap}[0]{\hat{\BF}}
%\newcommand{\bithree}[0]{{\Bi}_3}
%\newcommand{\bxa}[0]{\Bx\Ba}
%\newcommand{\coordvec}[2]{
%\newcommand{\costheta}[0]{\acap \cdot \xcap}
%\newcommand{\ddt}[1]{\ddot{#1}}
%\newcommand{\ddu}[1] {\frac {d{#1}} {du}}
%\newcommand{\dsqxj}[2] {\frac {\partial^2 {#1}} {\partial {x_{#2}}^2}}
%\newcommand{\dtheta}[1]{\frac{d {#1}}{d \theta}}
%\newcommand{\dt}[1]{\dot{#1}}
%\newcommand{\dt}[1]{\frac{d {#1}}{dt}}
%\newcommand{\dxj}[2] {\frac {\partial {#1}} {\partial {x_{#2}}}}
%\newcommand{\halfPhi}[0]{\frac{\phi}{2}}
%\newcommand{\half}[0]{\inv{2}}
%\newcommand{\inv}[1]{\frac{1}{#1}}
%\newcommand{\laplacian}[0]{\nabla^2}
%\newcommand{\matrixoftx}[3]{
%\newcommand{\nrrp}[0]{\norm{\rcap \wedge \Br'}}
%\newcommand{\oiint}{\bigcirc \hspace{-1.4em} \int \hspace{-.8em} \int}
%\newcommand{\transpose}[1]{{#1}^{\text{T}}}
%\newcommand{\transpose}[1]{{{#1}^{\TextTranspose}}}
%\newcommand{\transpose}[1]{{{#1}^{\text{T}}}}
%\newcommand{\barA}[0]{\bar{A}}
%\newcommand{\qbar}[0]{\bar{q}}
%\newcommand{\qdotbar}[0]{\dot{\bar{q}}}
%
%</infrequent>





\usepackage[bookmarks=true]{hyperref}

\usepackage{color,cite,graphicx}
   % use colour in the document, put your citations as [1-4]
   % rather than [1,2,3,4] (it looks nicer, and the extended LaTeX2e
   % graphics package. 
\usepackage{latexsym,amssymb,epsf} % don't remember if these are
   % needed, but their inclusion can't do any damage


\title{ Chapter 2 Quiz solutions for QM Demystified book. }
\author{Peeter Joot}
\date{ Jan 12, 2009.  Last Revision: $Date: 2009/01/13 04:41:51 $ }

\begin{document}

\maketitle{}

%\tableofcontents
\section{ Problem 1. Separation of variables. }

This problem was to show that separation of variables leads to an exponential energy/phase
term.  Let's try this, but do it instead for three dimensions and explore a bit.

We try a test solution of the form

\begin{align*}
\phi &= X(x) Y(y) Z(z) T(t) \\
\end{align*}

and substitute into 

\begin{align*}
\left(-\frac{\hbar^2}{2m}\grad^2 + V\right) \phi &= i \hbar \partial_t \phi
\end{align*}

differentiating and dividing by $\phi$ we have
\begin{align*}
-\frac{\hbar^2}{2m}
\left(
\frac{X''}{X}
+\frac{Y''}{Y}
+\frac{Z''}{Z}
\right)
 + V &= i \hbar \frac{T'}{T}
\end{align*}

We set the right hand side equal to a constant $E$ to be determined by boundary value conditions.
According to the dimensionals of $V$ this $E$ constant can be seen to neccessarily be an energy
of some sort.  In terms of this energy, we have for the function $T$

\begin{align*}
i \hbar \frac{T'}{T} &= E
\end{align*}

With a solution of
\begin{align*}
T(t) &= e^{-i E t/\hbar}
\end{align*}

Now, the left hand side imposes some constraints on $E$, but these will be potential dependent.
The simplest case, for the wave function of a free particle, is where $V=0$.

In that case we have
\begin{align*}
\frac{X''}{X} +\frac{Y''}{Y} +\frac{Z''}{Z} &= - \frac{2 m E}{\hbar^2}
\end{align*}

The sum of each of the terms involved all identically equal a constant, which is perhaps
reasonable to assume to be negative.  If we do so and impose the usual sort of separation of
variables constrait, requiring each of the $X''/X$, $Y''/Y$, and $Z''/Z$ terms to separately
equal some negative constant (to be fixed by boundary conditions), we can write

\begin{align*}
\frac{X''}{X} &= -{k_1}^2 \\
\frac{Y''}{Y} &= -{k_2}^2 \\
\frac{Z''}{Z} &= -{k_3}^2 \\
\end{align*}

So we have for the complete equation a solution proportional to

\begin{align*}
\phi = X Y Z T &= \exp(i(\Bk \cdot \Bx - E t/\hbar))
\end{align*}

With the additonal boundary value constraint of
\begin{align*}
\Bk^2 &= \frac{2 m E}{\hbar^2}
\end{align*}

\subsection{ constant potential variation. }

Let's consider a slightly more general potential with $V$ constant in some interval.

By inspection, in the interval we now have as the solution

\begin{align*}
\phi &= \exp(i(\Bk \cdot \Bx - (E - V) t/\hbar)) \\
\Bk^2 &= \frac{2 m (E - V)}{\hbar^2}
\end{align*}

Based on just the math, we don't know that $E$ is neccessarily positive, so in addition to the trigonometric solution above is it also reasonable to
allow for possible
hyperbolic solutions

\begin{align*}
\phi &= \exp(\Bk \cdot \Bx - i(E - V) t/\hbar) \\
\Bk^2 &= \frac{2 m (V - E)}{\hbar^2}
\end{align*}

We need some physics
to augment the math in order to determine what form of solution is actually valid.  Some of that physics likely comes in the form of the boundary conditions and perhaps other constraints such as normalization.

\subsection{ General solution. }

Using superposition we should be able to form a wave packet in integral form
by allowing for any set of $\Bk$ vectors.  Suppose we assemble a test solution by summing over possible wave numbers

\begin{align*}
\phi = \int A(\Bk) \exp(\alpha(\Bk \cdot \Bx) + i(E - V) t/\hbar) dk_1 dk_2 dk_3 \\
\end{align*}

FIXME: time term sign fudged to make the $V$ cancel out.  Must have a mistake above.

Does this work?  Let's take derivatives and see what constraints we require if it does.

\begin{align*}
\grad^2 \phi &= \alpha^2 \Bk^2 \phi \\
i \hbar \partial_t \phi &= -(E - V) \phi
\end{align*}

So we have
\begin{align*}
-\frac{\hbar^2}{2m} \alpha^2 \Bk^2 \phi + V\phi = -(E-V) \phi
\end{align*}

and the constraint required for the wave number $\Bk$ is 

\begin{align*}
\alpha^2 \Bk^2 = \frac{2 m E }{\hbar^2}
\end{align*}

So, for trig solutions (plane waves) where $\alpha \propto i$ our energy integration constant $E$ takes a negative value.
For hyperbolic solutions, where $\alpha = \pm 1$ (say), our energy has a positive value.

This screams for the boundary values of a 3D particle in a box problem, which for infinite potential outside the
box will require sinusoidal solutions (to vanish on the boundaries since the particle can't get past the potential
barrier).

\subsection{ Particle in a box. }

\section{ Problem 2. Probabilities for a polynomial wavefunction. }

\subsection{ normalize it. }

This part can be done directly with a contour integration.

\subsection{ definite integral of probablility. }

While a partial fractions split of the probability density can be done
around the poles $\{\pm\sqrt{i}, \pm i \sqrt{i}\}$

\begin{align*}
\frac{1 + x^2}{1 + x^4} &=
\frac{A}{x -\sqrt{i}}
+\frac{B}{x +\sqrt{i}}
+\frac{C}{x -i\sqrt{i}}
+\frac{D}{x +i\sqrt{i}}
\end{align*}

Integrating these is straight forward (logarithms), but
the algebra gets messy to simplify the resulting expression (perhaps not so
much for the $[0,1]$ interval of the question).

A lazier way is to invoke \href{http://integrals.wolfram.com/index.jsp}{webmathematica}, which gives

\begin{align*}
\int \frac{1 + x^2}{1 + x^4} dx &=
\frac{-\tan^{-1}(1 - \sqrt{2} x) + \tan^{-1}(1 + \sqrt{2} x)}{ \sqrt{2} }
\end{align*}

\bibliographystyle{plainnat}
\bibliography{myrefs}

\end{document}
          % jan 12/09
%
% Copyright � 2012 Peeter Joot.  All Rights Reserved.
% Licenced as described in the file LICENSE under the root directory of this GIT repository.
%

%
%
%\documentclass{article}

%\usepackage{amsmath}
\usepackage{mathpazo}

%
% shorthand for bold symbols, convenient for vectors and matrices
%
\newcommand{\Ba}[0]{\mathbf{a}}
\newcommand{\Bb}[0]{\mathbf{b}}
\newcommand{\Bc}[0]{\mathbf{c}}
\newcommand{\Bd}[0]{\mathbf{d}}
\newcommand{\Be}[0]{\mathbf{e}}
\newcommand{\Bf}[0]{\mathbf{f}}
\newcommand{\Bg}[0]{\mathbf{g}}
\newcommand{\Bh}[0]{\mathbf{h}}
\newcommand{\Bi}[0]{\mathbf{i}}
\newcommand{\Bj}[0]{\mathbf{j}}
\newcommand{\Bk}[0]{\mathbf{k}}
\newcommand{\Bl}[0]{\mathbf{l}}
\newcommand{\Bm}[0]{\mathbf{m}}
\newcommand{\Bn}[0]{\mathbf{n}}
\newcommand{\Bo}[0]{\mathbf{o}}
\newcommand{\Bp}[0]{\mathbf{p}}
\newcommand{\Bq}[0]{\mathbf{q}}
\newcommand{\Br}[0]{\mathbf{r}}
\newcommand{\Bs}[0]{\mathbf{s}}
\newcommand{\Bt}[0]{\mathbf{t}}
\newcommand{\Bu}[0]{\mathbf{u}}
\newcommand{\Bv}[0]{\mathbf{v}}
\newcommand{\Bw}[0]{\mathbf{w}}
\newcommand{\Bx}[0]{\mathbf{x}}
\newcommand{\By}[0]{\mathbf{y}}
\newcommand{\Bz}[0]{\mathbf{z}}
\newcommand{\BA}[0]{\mathbf{A}}
\newcommand{\BB}[0]{\mathbf{B}}
\newcommand{\BC}[0]{\mathbf{C}}
\newcommand{\BD}[0]{\mathbf{D}}
\newcommand{\BE}[0]{\mathbf{E}}
\newcommand{\BF}[0]{\mathbf{F}}
\newcommand{\BG}[0]{\mathbf{G}}
\newcommand{\BH}[0]{\mathbf{H}}
\newcommand{\BI}[0]{\mathbf{I}}
\newcommand{\BJ}[0]{\mathbf{J}}
\newcommand{\BK}[0]{\mathbf{K}}
\newcommand{\BL}[0]{\mathbf{L}}
\newcommand{\BM}[0]{\mathbf{M}}
\newcommand{\BN}[0]{\mathbf{N}}
\newcommand{\BO}[0]{\mathbf{O}}
\newcommand{\BP}[0]{\mathbf{P}}
\newcommand{\BQ}[0]{\mathbf{Q}}
\newcommand{\BR}[0]{\mathbf{R}}
\newcommand{\BS}[0]{\mathbf{S}}
\newcommand{\BT}[0]{\mathbf{T}}
\newcommand{\BU}[0]{\mathbf{U}}
\newcommand{\BV}[0]{\mathbf{V}}
\newcommand{\BW}[0]{\mathbf{W}}
\newcommand{\BX}[0]{\mathbf{X}}
\newcommand{\BY}[0]{\mathbf{Y}}
\newcommand{\BZ}[0]{\mathbf{Z}}

\newcommand{\Bzero}[0]{\mathbf{0}}
\newcommand{\Btheta}[0]{\boldsymbol{\theta}}
\newcommand{\Btau}[0]{\boldsymbol{\tau}}
\newcommand{\Bomega}[0]{\boldsymbol{\omega}}

%
% shorthand for unit vectors
%
\newcommand{\acap}[0]{\hat{\Ba}}
\newcommand{\bcap}[0]{\hat{\Bb}}
\newcommand{\ccap}[0]{\hat{\Bc}}
\newcommand{\dcap}[0]{\hat{\Bd}}
\newcommand{\ecap}[0]{\hat{\Be}}
\newcommand{\fcap}[0]{\hat{\Bf}}
\newcommand{\gcap}[0]{\hat{\Bg}}
\newcommand{\hcap}[0]{\hat{\Bh}}
\newcommand{\icap}[0]{\hat{\Bi}}
\newcommand{\jcap}[0]{\hat{\Bj}}
\newcommand{\kcap}[0]{\hat{\Bk}}
\newcommand{\lcap}[0]{\hat{\Bl}}
\newcommand{\mcap}[0]{\hat{\Bm}}
\newcommand{\ncap}[0]{\hat{\Bn}}
\newcommand{\ocap}[0]{\hat{\Bo}}
\newcommand{\pcap}[0]{\hat{\Bp}}
\newcommand{\qcap}[0]{\hat{\Bq}}
\newcommand{\rcap}[0]{\hat{\Br}}
\newcommand{\scap}[0]{\hat{\Bs}}
\newcommand{\tcap}[0]{\hat{\Bt}}
\newcommand{\ucap}[0]{\hat{\Bu}}
\newcommand{\vcap}[0]{\hat{\Bv}}
\newcommand{\wcap}[0]{\hat{\Bw}}
\newcommand{\xcap}[0]{\hat{\Bx}}
\newcommand{\ycap}[0]{\hat{\By}}
\newcommand{\zcap}[0]{\hat{\Bz}}
\newcommand{\thetacap}[0]{\hat{\Btheta}}

%
% to write R^n and C^n in a distinguishable fashion.  Perhaps change this
% to the double lined characters upon figuring out how to do so.
%
\newcommand{\C}[1]{$\mathbb{C}^{#1}$}
\newcommand{\R}[1]{$\mathbb{R}^{#1}$}

%
% various generally useful helpers
%

% derivative of #1 wrt. #2:
\newcommand{\D}[2] {\frac {d#2} {d#1}}

\newcommand{\inv}[1]{\frac{1}{#1}}
\newcommand{\cross}[0]{\times}

\newcommand{\abs}[1]{\lvert{#1}\rvert}
\newcommand{\norm}[1]{\lVert{#1}\rVert}
\newcommand{\innerprod}[2]{\langle{#1}, {#2}\rangle}
\newcommand{\dotprod}[2]{{#1} \cdot {#2}}
\newcommand{\bdotprod}[2]{\left({#1} \cdot {#2}\right)}
\newcommand{\crossprod}[2]{{#1} \cross {#2}}
\newcommand{\tripleprod}[3]{\dotprod{\left(\crossprod{#1}{#2}\right)}{#3}}

\DeclareMathOperator{\Proj}{Proj}
\DeclareMathOperator{\Span}{span}
\DeclareMathOperator{\Sgn}{sgn}
\DeclareMathOperator{\Area}{Area}
\DeclareMathOperator{\Volume}{Volume}

%
% A few miscellaneous things specific to this document
%
\newcommand{\crossop}[1]{\crossprod{#1}{}}

% R2 vector.
\newcommand{\VectorTwo}[2]{
\begin{bmatrix}
 {#1} \\
 {#2}
\end{bmatrix}
}

\newcommand{\VectorN}[1]{
\begin{bmatrix}
{#1}_1 \\
{#1}_2 \\
\vdots \\
{#1}_N \\
\end{bmatrix}
}

\newcommand{\DETuvij}[4]{
\begin{vmatrix}
 {#1}_{#3} & {#1}_{#4} \\
 {#2}_{#3} & {#2}_{#4}
\end{vmatrix}
}

\newcommand{\DETuvwijk}[6]{
\begin{vmatrix}
 {#1}_{#4} & {#1}_{#5} & {#1}_{#6} \\
 {#2}_{#4} & {#2}_{#5} & {#2}_{#6} \\
 {#3}_{#4} & {#3}_{#5} & {#3}_{#6}
\end{vmatrix}
}

\newcommand{\DETuvwxijkl}[8]{
\begin{vmatrix}
 {#1}_{#5} & {#1}_{#6} & {#1}_{#7} & {#1}_{#8} \\
 {#2}_{#5} & {#2}_{#6} & {#2}_{#7} & {#2}_{#8} \\
 {#3}_{#5} & {#3}_{#6} & {#3}_{#7} & {#3}_{#8} \\
 {#4}_{#5} & {#4}_{#6} & {#4}_{#7} & {#4}_{#8} \\
\end{vmatrix}
}

%\newcommand{\DETuvwxyijklm}[10]{
%\begin{vmatrix}
% {#1}_{#6} & {#1}_{#7} & {#1}_{#8} & {#1}_{#9} & {#1}_{#10} \\
% {#2}_{#6} & {#2}_{#7} & {#2}_{#8} & {#2}_{#9} & {#2}_{#10} \\
% {#3}_{#6} & {#3}_{#7} & {#3}_{#8} & {#3}_{#9} & {#3}_{#10} \\
% {#4}_{#6} & {#4}_{#7} & {#4}_{#8} & {#4}_{#9} & {#4}_{#10} \\
% {#5}_{#6} & {#5}_{#7} & {#5}_{#8} & {#5}_{#9} & {#5}_{#10}
%\end{vmatrix}
%}

% R3 vector.
\newcommand{\VectorThree}[3]{
\begin{bmatrix}
 {#1} \\
 {#2} \\
 {#3}
\end{bmatrix}
}


%%<misc>
%
\newcommand{\Abs}[1]{{\left\lvert{#1}\right\rvert}}
\newcommand{\spacegrad}[0]{\boldsymbol{\nabla}}
\newcommand{\grad}[0]{\nabla}
\newcommand{\LL}[0]{\mathcal{L}}

% == \partial_{#1} {#2}
\newcommand{\PD}[2]{\frac{\partial {#2}}{\partial {#1}}}
% inline variant
\newcommand{\PDi}[2]{{\partial {#2}}/{\partial {#1}}}

\newcommand{\PDD}[3]{\frac{\partial^2 {#3}}{\partial {#1}\partial {#2}}}
%\newcommand{\PDd}[2]{\frac{\partial^2 {#2}}{{\partial{#1}}^2}}
\newcommand{\PDsq}[2]{\frac{\partial^2 {#2}}{(\partial {#1})^2}}

\newcommand{\Partial}[2]{\frac{\partial {#1}}{\partial {#2}}}
\DeclareMathOperator{\RejName}{Rej}
\newcommand{\Rej}[2]{\RejName_{#1}\left( {#2} \right)}
\newcommand{\Rm}[1]{\mathbb{R}^{#1}}
\newcommand{\Cm}[1]{\mathbb{C}^{#1}}
\newcommand{\conj}[0]{{*}}

%</misc>

% <grade selection>
%
\newcommand{\gpgrade}[2] {{\left\langle{{#1}}\right\rangle}_{#2}}

\newcommand{\gpgradezero}[1] {\gpgrade{#1}{}}
%\newcommand{\gpscalargrade}[1] {{\left\langle{{#1}}\right\rangle}}
%\newcommand{\gpgradezero}[1] {\gpgrade{#1}{0}}

%\newcommand{\gpgradeone}[1] {{\left\langle{{#1}}\right\rangle}_{1}}
\newcommand{\gpgradeone}[1] {\gpgrade{#1}{1}}

\newcommand{\gpgradetwo}[1] {\gpgrade{#1}{2}}
\newcommand{\gpgradethree}[1] {\gpgrade{#1}{3}}
\newcommand{\gpgradefour}[1] {\gpgrade{#1}{4}}
%
% </grade selection>



\newcommand{\adot}[0]{{\dot{a}}}
\newcommand{\bdot}[0]{{\dot{b}}}
% taken for centered dot:
%\newcommand{\cdot}[0]{{\dot{c}}}
%\newcommand{\ddot}[0]{{\dot{d}}}
\newcommand{\edot}[0]{{\dot{e}}}
\newcommand{\fdot}[0]{{\dot{f}}}
\newcommand{\gdot}[0]{{\dot{g}}}
\newcommand{\hdot}[0]{{\dot{h}}}
\newcommand{\idot}[0]{{\dot{i}}}
\newcommand{\jdot}[0]{{\dot{j}}}
\newcommand{\kdot}[0]{{\dot{k}}}
\newcommand{\ldot}[0]{{\dot{l}}}
\newcommand{\mdot}[0]{{\dot{m}}}
\newcommand{\ndot}[0]{{\dot{n}}}
%\newcommand{\odot}[0]{{\dot{o}}}
\newcommand{\pdot}[0]{{\dot{p}}}
\newcommand{\qdot}[0]{{\dot{q}}}
\newcommand{\rdot}[0]{{\dot{r}}}
\newcommand{\sdot}[0]{{\dot{s}}}
\newcommand{\tdot}[0]{{\dot{t}}}
\newcommand{\udot}[0]{{\dot{u}}}
\newcommand{\vdot}[0]{{\dot{v}}}
\newcommand{\wdot}[0]{{\dot{w}}}
\newcommand{\xdot}[0]{{\dot{x}}}
\newcommand{\ydot}[0]{{\dot{y}}}
\newcommand{\zdot}[0]{{\dot{z}}}
\newcommand{\addot}[0]{{\ddot{a}}}
\newcommand{\bddot}[0]{{\ddot{b}}}
\newcommand{\cddot}[0]{{\ddot{c}}}
%\newcommand{\dddot}[0]{{\ddot{d}}}
\newcommand{\eddot}[0]{{\ddot{e}}}
\newcommand{\fddot}[0]{{\ddot{f}}}
\newcommand{\gddot}[0]{{\ddot{g}}}
\newcommand{\hddot}[0]{{\ddot{h}}}
\newcommand{\iddot}[0]{{\ddot{i}}}
\newcommand{\jddot}[0]{{\ddot{j}}}
\newcommand{\kddot}[0]{{\ddot{k}}}
\newcommand{\lddot}[0]{{\ddot{l}}}
\newcommand{\mddot}[0]{{\ddot{m}}}
\newcommand{\nddot}[0]{{\ddot{n}}}
\newcommand{\oddot}[0]{{\ddot{o}}}
\newcommand{\pddot}[0]{{\ddot{p}}}
\newcommand{\qddot}[0]{{\ddot{q}}}
\newcommand{\rddot}[0]{{\ddot{r}}}
\newcommand{\sddot}[0]{{\ddot{s}}}
\newcommand{\tddot}[0]{{\ddot{t}}}
\newcommand{\uddot}[0]{{\ddot{u}}}
\newcommand{\vddot}[0]{{\ddot{v}}}
\newcommand{\wddot}[0]{{\ddot{w}}}
\newcommand{\xddot}[0]{{\ddot{x}}}
\newcommand{\yddot}[0]{{\ddot{y}}}
\newcommand{\zddot}[0]{{\ddot{z}}}

%<bold and dot greek symbols>
%

\newcommand{\Deltadot}[0]{{\dot{\Delta}}}
\newcommand{\Gammadot}[0]{{\dot{\Gamma}}}
\newcommand{\Lambdadot}[0]{{\dot{\Lambda}}}
\newcommand{\Omegadot}[0]{{\dot{\Omega}}}
\newcommand{\Phidot}[0]{{\dot{\Phi}}}
\newcommand{\Pidot}[0]{{\dot{\Pi}}}
\newcommand{\Psidot}[0]{{\dot{\Psi}}}
\newcommand{\Sigmadot}[0]{{\dot{\Sigma}}}
\newcommand{\Thetadot}[0]{{\dot{\Theta}}}
\newcommand{\Upsilondot}[0]{{\dot{\Upsilon}}}
\newcommand{\Xidot}[0]{{\dot{\Xi}}}
\newcommand{\alphadot}[0]{{\dot{\alpha}}}
\newcommand{\betadot}[0]{{\dot{\beta}}}
\newcommand{\chidot}[0]{{\dot{\chi}}}
\newcommand{\deltadot}[0]{{\dot{\delta}}}
\newcommand{\epsilondot}[0]{{\dot{\epsilon}}}
\newcommand{\etadot}[0]{{\dot{\eta}}}
\newcommand{\gammadot}[0]{{\dot{\gamma}}}
\newcommand{\kappadot}[0]{{\dot{\kappa}}}
\newcommand{\lambdadot}[0]{{\dot{\lambda}}}
\newcommand{\mudot}[0]{{\dot{\mu}}}
\newcommand{\nudot}[0]{{\dot{\nu}}}
\newcommand{\omegadot}[0]{{\dot{\omega}}}
\newcommand{\phidot}[0]{{\dot{\phi}}}
\newcommand{\pidot}[0]{{\dot{\pi}}}
\newcommand{\psidot}[0]{{\dot{\psi}}}
\newcommand{\rhodot}[0]{{\dot{\rho}}}
\newcommand{\sigmadot}[0]{{\dot{\sigma}}}
\newcommand{\taudot}[0]{{\dot{\tau}}}
\newcommand{\thetadot}[0]{{\dot{\theta}}}
\newcommand{\upsilondot}[0]{{\dot{\upsilon}}}
\newcommand{\varepsilondot}[0]{{\dot{\varepsilon}}}
\newcommand{\varphidot}[0]{{\dot{\varphi}}}
\newcommand{\varpidot}[0]{{\dot{\varpi}}}
\newcommand{\varrhodot}[0]{{\dot{\varrho}}}
\newcommand{\varsigmadot}[0]{{\dot{\varsigma}}}
\newcommand{\varthetadot}[0]{{\dot{\vartheta}}}
\newcommand{\xidot}[0]{{\dot{\xi}}}
\newcommand{\zetadot}[0]{{\dot{\zeta}}}

\newcommand{\Deltaddot}[0]{{\ddot{\Delta}}}
\newcommand{\Gammaddot}[0]{{\ddot{\Gamma}}}
\newcommand{\Lambdaddot}[0]{{\ddot{\Lambda}}}
\newcommand{\Omegaddot}[0]{{\ddot{\Omega}}}
\newcommand{\Phiddot}[0]{{\ddot{\Phi}}}
\newcommand{\Piddot}[0]{{\ddot{\Pi}}}
\newcommand{\Psiddot}[0]{{\ddot{\Psi}}}
\newcommand{\Sigmaddot}[0]{{\ddot{\Sigma}}}
\newcommand{\Thetaddot}[0]{{\ddot{\Theta}}}
\newcommand{\Upsilonddot}[0]{{\ddot{\Upsilon}}}
\newcommand{\Xiddot}[0]{{\ddot{\Xi}}}
\newcommand{\alphaddot}[0]{{\ddot{\alpha}}}
\newcommand{\betaddot}[0]{{\ddot{\beta}}}
\newcommand{\chiddot}[0]{{\ddot{\chi}}}
\newcommand{\deltaddot}[0]{{\ddot{\delta}}}
\newcommand{\epsilonddot}[0]{{\ddot{\epsilon}}}
\newcommand{\etaddot}[0]{{\ddot{\eta}}}
\newcommand{\gammaddot}[0]{{\ddot{\gamma}}}
\newcommand{\kappaddot}[0]{{\ddot{\kappa}}}
\newcommand{\lambdaddot}[0]{{\ddot{\lambda}}}
\newcommand{\muddot}[0]{{\ddot{\mu}}}
\newcommand{\nuddot}[0]{{\ddot{\nu}}}
\newcommand{\omegaddot}[0]{{\ddot{\omega}}}
\newcommand{\phiddot}[0]{{\ddot{\phi}}}
\newcommand{\piddot}[0]{{\ddot{\pi}}}
\newcommand{\psiddot}[0]{{\ddot{\psi}}}
\newcommand{\rhoddot}[0]{{\ddot{\rho}}}
\newcommand{\sigmaddot}[0]{{\ddot{\sigma}}}
\newcommand{\tauddot}[0]{{\ddot{\tau}}}
\newcommand{\thetaddot}[0]{{\ddot{\theta}}}
\newcommand{\upsilonddot}[0]{{\ddot{\upsilon}}}
\newcommand{\varepsilonddot}[0]{{\ddot{\varepsilon}}}
\newcommand{\varphiddot}[0]{{\ddot{\varphi}}}
\newcommand{\varpiddot}[0]{{\ddot{\varpi}}}
\newcommand{\varrhoddot}[0]{{\ddot{\varrho}}}
\newcommand{\varsigmaddot}[0]{{\ddot{\varsigma}}}
\newcommand{\varthetaddot}[0]{{\ddot{\vartheta}}}
\newcommand{\xiddot}[0]{{\ddot{\xi}}}
\newcommand{\zetaddot}[0]{{\ddot{\zeta}}}

\newcommand{\BDelta}[0]{\boldsymbol{\Delta}}
\newcommand{\BGamma}[0]{\boldsymbol{\Gamma}}
\newcommand{\BLambda}[0]{\boldsymbol{\Lambda}}
\newcommand{\BOmega}[0]{\boldsymbol{\Omega}}
\newcommand{\BPhi}[0]{\boldsymbol{\Phi}}
\newcommand{\BPi}[0]{\boldsymbol{\Pi}}
\newcommand{\BPsi}[0]{\boldsymbol{\Psi}}
\newcommand{\BSigma}[0]{\boldsymbol{\Sigma}}
\newcommand{\BTheta}[0]{\boldsymbol{\Theta}}
\newcommand{\BUpsilon}[0]{\boldsymbol{\Upsilon}}
\newcommand{\BXi}[0]{\boldsymbol{\Xi}}
\newcommand{\Balpha}[0]{\boldsymbol{\alpha}}
\newcommand{\Bbeta}[0]{\boldsymbol{\beta}}
\newcommand{\Bchi}[0]{\boldsymbol{\chi}}
\newcommand{\Bdelta}[0]{\boldsymbol{\delta}}
\newcommand{\Bepsilon}[0]{\boldsymbol{\epsilon}}
\newcommand{\Beta}[0]{\boldsymbol{\eta}}
\newcommand{\Bgamma}[0]{\boldsymbol{\gamma}}
\newcommand{\Bkappa}[0]{\boldsymbol{\kappa}}
\newcommand{\Blambda}[0]{\boldsymbol{\lambda}}
\newcommand{\Bmu}[0]{\boldsymbol{\mu}}
\newcommand{\Bnu}[0]{\boldsymbol{\nu}}
%\newcommand{\Bomega}[0]{\boldsymbol{\omega}}
\newcommand{\Bphi}[0]{\boldsymbol{\phi}}
\newcommand{\Bpi}[0]{\boldsymbol{\pi}}
\newcommand{\Bpsi}[0]{\boldsymbol{\psi}}
\newcommand{\Brho}[0]{\boldsymbol{\rho}}
\newcommand{\Bsigma}[0]{\boldsymbol{\sigma}}
%\newcommand{\Btau}[0]{\boldsymbol{\tau}}
%\newcommand{\Btheta}[0]{\boldsymbol{\theta}}
\newcommand{\Bupsilon}[0]{\boldsymbol{\upsilon}}
\newcommand{\Bvarepsilon}[0]{\boldsymbol{\varepsilon}}
\newcommand{\Bvarphi}[0]{\boldsymbol{\varphi}}
\newcommand{\Bvarpi}[0]{\boldsymbol{\varpi}}
\newcommand{\Bvarrho}[0]{\boldsymbol{\varrho}}
\newcommand{\Bvarsigma}[0]{\boldsymbol{\varsigma}}
\newcommand{\Bvartheta}[0]{\boldsymbol{\vartheta}}
\newcommand{\Bxi}[0]{\boldsymbol{\xi}}
\newcommand{\Bzeta}[0]{\boldsymbol{\zeta}}
%
%</bold and dot greek symbols>
%<infrequent>
%
%\newcommand{\AreaOp}[1]{\AName_{#1}}
%\newcommand{\Babs}[0]{\abs{\BB}}
%\newcommand{\Bcap}[0]{\hat{\BB}}
%\newcommand{\BrPrimeRej}[0]{\rcap(\rcap \wedge \Br')}
%\newcommand{\CA}[0]{\mathcal{A}}
%\newcommand{\Cos}[1]{\cos{\left({#1}\right)}}
%\newcommand{\Det}[1] {\abs{#1}}
%\newcommand{\Dsq}[2] {\frac {\partial^2 {#1}} {\partial {#2}^2}}
%\newcommand{\Exp}[1]{\exp{\left({#1}\right)}}
%\newcommand{\Norm}[1]{\left\lVert{#1}\right\rVert}
%\newcommand{\Sin}[1]{\sin{\left({#1}\right)}}
%\newcommand{\T}[0]{\text{T}}
%\newcommand{\VolumeOp}[1]{\VName_{#1}}
%\newcommand{\agrad}[0]{\Ba \cdot \nabla}
%\newcommand{\alphacap}[0]{\hat{\boldsymbol{\alpha}}}
%\newcommand{\Fcap}[0]{\hat{\BF}}
%\newcommand{\bithree}[0]{{\Bi}_3}
%\newcommand{\bxa}[0]{\Bx\Ba}
%\newcommand{\coordvec}[2]{
%\newcommand{\costheta}[0]{\acap \cdot \xcap}
%\newcommand{\ddt}[1]{\ddot{#1}}
%\newcommand{\ddu}[1] {\frac {d{#1}} {du}}
%\newcommand{\dsqxj}[2] {\frac {\partial^2 {#1}} {\partial {x_{#2}}^2}}
%\newcommand{\dtheta}[1]{\frac{d {#1}}{d \theta}}
%\newcommand{\dt}[1]{\dot{#1}}
%\newcommand{\dt}[1]{\frac{d {#1}}{dt}}
%\newcommand{\dxj}[2] {\frac {\partial {#1}} {\partial {x_{#2}}}}
%\newcommand{\halfPhi}[0]{\frac{\phi}{2}}
%\newcommand{\half}[0]{\inv{2}}
%\newcommand{\inv}[1]{\frac{1}{#1}}
%\newcommand{\laplacian}[0]{\nabla^2}
%\newcommand{\matrixoftx}[3]{
%\newcommand{\nrrp}[0]{\norm{\rcap \wedge \Br'}}
%\newcommand{\oiint}{\bigcirc \hspace{-1.4em} \int \hspace{-.8em} \int}
%\newcommand{\transpose}[1]{{#1}^{\text{T}}}
%\newcommand{\transpose}[1]{{{#1}^{\TextTranspose}}}
%\newcommand{\transpose}[1]{{{#1}^{\text{T}}}}
%\newcommand{\barA}[0]{\bar{A}}
%\newcommand{\qbar}[0]{\bar{q}}
%\newcommand{\qdotbar}[0]{\dot{\bar{q}}}
%
%</infrequent>





%\usepackage[bookmarks=true]{hyperref}

%\usepackage{color,cite,graphicx}
   % use colour in the document, put your citations as [1-4]
   % rather than [1,2,3,4] (it looks nicer, and the extended LaTeX2e
   % graphics package.
%\usepackage{latexsym,amssymb,epsf} % do not remember if these are
   % needed, but their inclusion can not do any damage


\chapter{Simple Wave Packet Examples}
\label{chap:wavepacket}
%\author{Peeter Joot \quad peeterjoot@protonmail.com}
\date{ Feb 16, 2009.  wavepacket.tex }

%\begin{document}

%\maketitle{}
%\tableofcontents
\section{Wave packet examples}

In \citep{bohm1989qt}, chapter 3, a couple explicit wave packet examples are given.
Some of this is a little hard to follow in the small set font of the text, and some details are missing.  Here the integrals are performed
in detail.

\subsection{Unweighted example. Equation 1}

Summing plane waves over a small range of frequencies

\begin{equation}\label{eqn:wavepacket:20}
\begin{aligned}
E_z(x)
&= \int_{k_0 -\Delta k}^{k_0 -\Delta k} dk e^{ik(x-x_0)} \\
&= {\left. \frac{e^{ik(x-x_0)}}{i(x-x_0)} \right\vert}_{k_0 -\Delta k}^{k_0 -\Delta k}  \\
&= \frac{e^{i(k_0 + \Delta k)(x-x_0)}}{i(x-x_0)} -\frac{e^{i(k_0 - \Delta k)(x-x_0)}}{i(x-x_0)} \\
&= 2 {e^{i k_0 (x-x_0)}}
 \frac{{e^{i \Delta k(x-x_0)}} -{e^{-i \Delta k(x-x_0)}}}{2 i (x-x_0)} \\
\end{aligned}
\end{equation}

Which is Bohm's equation 1.
\begin{equation}\label{eqn:wavepacket:40}
\begin{aligned}
E_z(x) &= 2 {e^{i k_0 (x-x_0)}} \frac{\sin( \Delta k (x-x_0))}{ x-x_0}
\end{aligned}
\end{equation}

\subsection{Gaussian weighting example. Equation 4}

Next example, also chosen for simplicity, uses a Gaussian weighting function

\begin{equation}\label{eqn:wavepacket:60}
\begin{aligned}
\psi
&= \int_\infty^\infty
\exp\left( - \frac{(k-k_0)^2}{2(\Delta k)^2} \right)
\exp\left( i k(x-x_0) \right) dk \\
\end{aligned}
\end{equation}

Next, is the sneaky/clever step of adding and subtracting \(i k_0 (x-x_0) - (x-x_0)^2 (\Delta k)^2/2\) to the exponentials, which gives

\begin{equation}\label{eqn:wavepacket:80}
\begin{aligned}
\psi
&=
\exp\left( i k_0 (x-x_0) - \frac{(x-x_0)^2}{2} (\Delta k)^2 \right) \\
&\quad \int_\infty^\infty
\exp\left(
- \frac{(k-k_0)^2}{2(\Delta k)^2}
+i (k-k_0)(x-x_0)
+\frac{(x-x_0)^2}{2} (\Delta k)^2
\right)
dk \\
\end{aligned}
\end{equation}

Looking at just the remaining integral part, say \(I\), we have
\begin{equation}\label{eqn:wavepacket:100}
\begin{aligned}
I &=
\int_\infty^\infty
\exp\left( \inv{2} \left(
i^2 \frac{(k-k_0)^2}{(\Delta k)^2}
+2i (k-k_0)(x-x_0)
+{(x-x_0)^2}{} (\Delta k)^2
\right)
\right)
dk \\
&=
\int_\infty^\infty
\exp\left( \inv{2} \left(
i \frac{k-k_0}{\Delta k}
+(x-x_0) \Delta k
\right)^2
\right)
dk \\
&=
\int_\infty^\infty
\exp\left( -\inv{2} \left(
\frac{k-k_0}{\Delta k}
-i(x-x_0) \Delta k
\right)^2
\right)
dk \\
\end{aligned}
\end{equation}

A change of variables \(u = (k-k_0)/{\Delta k} -i(x-x_0) \Delta k\), \(du = dk/{\Delta k}\) gives

\begin{equation}\label{eqn:wavepacket:120}
\begin{aligned}
I
&= \Delta k \int_\infty^\infty e^{ -u^2/2 } du \\
&= \sqrt{2 \pi} \Delta k
\end{aligned}
\end{equation}

Which gives

\begin{equation}\label{eqn:wavepacket:140}
\begin{aligned}
\psi(x)
&= \sqrt{2 \pi} \Delta k
\exp\left( i k_0 (x-x_0) - \frac{(x-x_0)^2}{2} (\Delta k)^2 \right) \\
\end{aligned}
\end{equation}

Off by a factor of \(\sqrt{\Delta k}\) compared to the text?  Typo in the book or a mistake above?

\subsection{Gaussian weighting with angular velocity and acceleration}

Next covered is a wave packet where the angular frequency is a function of the wave number, as in equation 8

\begin{equation}\label{eqn:wavepacket:160}
\begin{aligned}
E(x,t) &= \IIinf f(k - k_0) \exp\left( i k(x-x_0) - i\omega(k) t\right) dk
\end{aligned}
\end{equation}

With a Taylor series expansion of the frequency about \(k_0\)

\begin{equation}\label{eqn:wavepacket:180}
\begin{aligned}
\omega(k)
&= \omega(k_0) + \left(\PD{k}{\omega}\right)_{k=k_0} (k-k_0) + \left(\PDSq{k}{\omega}\right)_{k=k_0} \frac{(k-k_0)^2}{2} + \cdots \\
&= \omega_0 + V_g (k-k_0) + \alpha \frac{(k-k_0)^2}{2} + \cdots \\
\end{aligned}
\end{equation}

Here \(V_g\), and \(\alpha\) are the group velocity and accelerations respectively.  Now, I had never seen the group velocity expressed
this way, which seems a particularly simple way of putting it.
The example of how \(\omega = 2 \pi c/\lambda n(\lambda)\) can vary with index of refraction and wavelength is also nice.  I
imagine a light wave going through a water oil air transition and the angular frequency in each region causing dispersion and
reflection and path alteration effects.

Back to the second order approximation of the frequency, substituting back into the wave packet integral one has

\begin{equation}\label{eqn:wavepacket:200}
\begin{aligned}
E(x,t) &\approx \IIinf f(k - k_0) \exp\left( i k(x-x_0) - i
\left(\omega_0 + V_g (k-k_0) + \alpha \frac{(k-k_0)^2}{2} \right)
t\right) dk
\end{aligned}
\end{equation}

With \(\kappa = k - k_0\), \(\Delta x = x -x_0\) and the Gaussian weighting \(f(\kappa) = e^{-\kappa^2/2(\Delta k)^2}\) this is

\begin{equation}\label{eqn:wavepacket:220}
\begin{aligned}
\exp&( i k_0 (x-x_0) )
\IIinf
\exp\left( -\frac{\kappa^2}{2(\Delta k)^2} +i \kappa (x-x_0)
- i \left(\omega_0 + V_g \kappa + \alpha \frac{\kappa^2}{2} \right) t\right) dk \\
&=
\exp( i k_0 \Delta x -i \omega_0 t)
\IIinf
\exp\left(
i \kappa ( \Delta x - V_g t )
-\frac{\kappa^2}{2} \left(\inv{(\Delta k)^2} + i \alpha t \right)
\right) d\kappa \\
\end{aligned}
\end{equation}

With \(a = \Delta x - V_g t\) and \(b = \inv{(\Delta k)^2} + i \alpha t\), the exponential in the integral takes the form
\begin{equation}\label{eqn:wavepacket:240}
\begin{aligned}
\exp\left( i \kappa a - \kappa^2 \frac{b}{2} \right)
&= \exp\left( - \frac{b}{2}\left(-2 i \kappa \frac{a}{b} + \kappa^2 \right) \right)  \\
&= \exp\left( - \frac{b}{2}\left( \kappa - i \frac{a}{b} \right)^2 + \frac{b}{2}\left(\frac{ia}{b}\right)^2 \right)  \\
&= \exp\left( - \frac{b}{2}\left( \kappa - i \frac{a}{b} \right)^2 - \frac{a^2}{2b} \right)  \\
\end{aligned}
\end{equation}

Our wave packet is now

\begin{equation}\label{eqn:wavepacket:260}
\begin{aligned}
E(x,t)
&= \exp\left( i k_0 \Delta x -i \omega_0 t - \frac{( \Delta x - V_g t)^2 (\Delta k)^2}{2(1 + i \alpha t (\Delta k)^2)} \right)
\IIinf
\exp\left( - \frac{b}{2}\left( \kappa - i \frac{a}{b} \right)^2 \right)
d\kappa \\
\end{aligned}
\end{equation}

A change of vars \(u = \sqrt{b}(\kappa - ia/b)\) gives

\begin{equation}\label{eqn:wavepacket:280}
\begin{aligned}
E(x,t)
&= \exp\left( i k_0 \Delta x -i \omega_0 t - \frac{( \Delta x - V_g t)^2 (\Delta k)^2}{2(1 + i \alpha t (\Delta k)^2)} \right)
\frac{\Delta k}{\sqrt{1 + i \alpha t (\Delta k)^2}}
\IIinf \exp\left( - \frac{u^2}{2} \right) d\kappa \\
&= \exp\left( i k_0 \Delta x -i \omega_0 t - \frac{( \Delta x - V_g t)^2 (\Delta k)^2}{2(1 + i \alpha t (\Delta k)^2)} \right)
\frac{\sqrt{2\pi}\Delta k}{\sqrt{1 + i \alpha t (\Delta k)^2}} \\
&=
\frac{\sqrt{2\pi}\Delta k}{\sqrt{1 + i \alpha t (\Delta k)^2}}
\exp\left( i \left(k_0 \Delta x - \omega_0 t
+
\alpha t (\Delta k)^2 \frac{( \Delta x - V_g t)^2 (\Delta k)^2}{2(1 + \alpha^2 t^2 (\Delta k)^4)}
\right) \right) \times \\
&\exp\left(
-
\frac{( \Delta x - V_g t)^2 (\Delta k)^2}{2(1 + \alpha^2 t^2 (\Delta k)^4)}
\right) \\
\end{aligned}
\end{equation}

This is consistent with the result in the text and with \(\alpha = 0\) confirms that equation 4 did have a typo (irrelevant to the
intensity discussion).

%gnuplot> set xlabel "x-axis"
%gnuplot> set ylabel "t-axis"
%gnuplot> splot [x=75:100] [t=-10:10] exp(-10*(x-10*t)**2/(1+10*t*t*100))
%gnuplot> splot [x=75:100] [t=-10:100] exp(-10*(x-10*t)**2/(1+10*t*t*100))
%gnuplot> splot [x=75:100] [t=-100:100] exp(-10*(x-10*t)**2/(1+10*t*t*100))
%gnuplot> splot [x=75:100] [t=-100:100] exp(-10*(x-1*t)**2/(1+10*t*t*100))
%gnuplot> splot [x=75:100] [t=-100:100] exp(-10*(x-1*t)**2/(1+10*t*t*1000))
%gnuplot> splot [x=75:100] [t=-100:100] exp(-10*(x-1*t)**2/(1+10*t*t*10000))
%gnuplot> splot [x=75:100] [t=-100:100] exp(-10*(x-1*t)**2/(1+10*t*t*2))
%gnuplot> splot [x=-75:100] [t=-100:100] exp(-10*(x-1*t)**2/(1+10*t*t*2))

%\bibliographystyle{plainnat}
%\bibliography{myrefs}

%\end{document}
          % feb 16/09
\documentclass{article}

\usepackage{amsmath}
\usepackage{mathpazo}

%
% shorthand for bold symbols, convenient for vectors and matrices
%
\newcommand{\Ba}[0]{\mathbf{a}}
\newcommand{\Bb}[0]{\mathbf{b}}
\newcommand{\Bc}[0]{\mathbf{c}}
\newcommand{\Bd}[0]{\mathbf{d}}
\newcommand{\Be}[0]{\mathbf{e}}
\newcommand{\Bf}[0]{\mathbf{f}}
\newcommand{\Bg}[0]{\mathbf{g}}
\newcommand{\Bh}[0]{\mathbf{h}}
\newcommand{\Bi}[0]{\mathbf{i}}
\newcommand{\Bj}[0]{\mathbf{j}}
\newcommand{\Bk}[0]{\mathbf{k}}
\newcommand{\Bl}[0]{\mathbf{l}}
\newcommand{\Bm}[0]{\mathbf{m}}
\newcommand{\Bn}[0]{\mathbf{n}}
\newcommand{\Bo}[0]{\mathbf{o}}
\newcommand{\Bp}[0]{\mathbf{p}}
\newcommand{\Bq}[0]{\mathbf{q}}
\newcommand{\Br}[0]{\mathbf{r}}
\newcommand{\Bs}[0]{\mathbf{s}}
\newcommand{\Bt}[0]{\mathbf{t}}
\newcommand{\Bu}[0]{\mathbf{u}}
\newcommand{\Bv}[0]{\mathbf{v}}
\newcommand{\Bw}[0]{\mathbf{w}}
\newcommand{\Bx}[0]{\mathbf{x}}
\newcommand{\By}[0]{\mathbf{y}}
\newcommand{\Bz}[0]{\mathbf{z}}
\newcommand{\BA}[0]{\mathbf{A}}
\newcommand{\BB}[0]{\mathbf{B}}
\newcommand{\BC}[0]{\mathbf{C}}
\newcommand{\BD}[0]{\mathbf{D}}
\newcommand{\BE}[0]{\mathbf{E}}
\newcommand{\BF}[0]{\mathbf{F}}
\newcommand{\BG}[0]{\mathbf{G}}
\newcommand{\BH}[0]{\mathbf{H}}
\newcommand{\BI}[0]{\mathbf{I}}
\newcommand{\BJ}[0]{\mathbf{J}}
\newcommand{\BK}[0]{\mathbf{K}}
\newcommand{\BL}[0]{\mathbf{L}}
\newcommand{\BM}[0]{\mathbf{M}}
\newcommand{\BN}[0]{\mathbf{N}}
\newcommand{\BO}[0]{\mathbf{O}}
\newcommand{\BP}[0]{\mathbf{P}}
\newcommand{\BQ}[0]{\mathbf{Q}}
\newcommand{\BR}[0]{\mathbf{R}}
\newcommand{\BS}[0]{\mathbf{S}}
\newcommand{\BT}[0]{\mathbf{T}}
\newcommand{\BU}[0]{\mathbf{U}}
\newcommand{\BV}[0]{\mathbf{V}}
\newcommand{\BW}[0]{\mathbf{W}}
\newcommand{\BX}[0]{\mathbf{X}}
\newcommand{\BY}[0]{\mathbf{Y}}
\newcommand{\BZ}[0]{\mathbf{Z}}

\newcommand{\Bzero}[0]{\mathbf{0}}
\newcommand{\Btheta}[0]{\boldsymbol{\theta}}
\newcommand{\Btau}[0]{\boldsymbol{\tau}}
\newcommand{\Bomega}[0]{\boldsymbol{\omega}}

%
% shorthand for unit vectors
%
\newcommand{\acap}[0]{\hat{\Ba}}
\newcommand{\bcap}[0]{\hat{\Bb}}
\newcommand{\ccap}[0]{\hat{\Bc}}
\newcommand{\dcap}[0]{\hat{\Bd}}
\newcommand{\ecap}[0]{\hat{\Be}}
\newcommand{\fcap}[0]{\hat{\Bf}}
\newcommand{\gcap}[0]{\hat{\Bg}}
\newcommand{\hcap}[0]{\hat{\Bh}}
\newcommand{\icap}[0]{\hat{\Bi}}
\newcommand{\jcap}[0]{\hat{\Bj}}
\newcommand{\kcap}[0]{\hat{\Bk}}
\newcommand{\lcap}[0]{\hat{\Bl}}
\newcommand{\mcap}[0]{\hat{\Bm}}
\newcommand{\ncap}[0]{\hat{\Bn}}
\newcommand{\ocap}[0]{\hat{\Bo}}
\newcommand{\pcap}[0]{\hat{\Bp}}
\newcommand{\qcap}[0]{\hat{\Bq}}
\newcommand{\rcap}[0]{\hat{\Br}}
\newcommand{\scap}[0]{\hat{\Bs}}
\newcommand{\tcap}[0]{\hat{\Bt}}
\newcommand{\ucap}[0]{\hat{\Bu}}
\newcommand{\vcap}[0]{\hat{\Bv}}
\newcommand{\wcap}[0]{\hat{\Bw}}
\newcommand{\xcap}[0]{\hat{\Bx}}
\newcommand{\ycap}[0]{\hat{\By}}
\newcommand{\zcap}[0]{\hat{\Bz}}
\newcommand{\thetacap}[0]{\hat{\Btheta}}

%
% to write R^n and C^n in a distinguishable fashion.  Perhaps change this
% to the double lined characters upon figuring out how to do so.
%
\newcommand{\C}[1]{$\mathbb{C}^{#1}$}
\newcommand{\R}[1]{$\mathbb{R}^{#1}$}

%
% various generally useful helpers
%

% derivative of #1 wrt. #2:
\newcommand{\D}[2] {\frac {d#2} {d#1}}

\newcommand{\inv}[1]{\frac{1}{#1}}
\newcommand{\cross}[0]{\times}

\newcommand{\abs}[1]{\lvert{#1}\rvert}
\newcommand{\norm}[1]{\lVert{#1}\rVert}
\newcommand{\innerprod}[2]{\langle{#1}, {#2}\rangle}
\newcommand{\dotprod}[2]{{#1} \cdot {#2}}
\newcommand{\bdotprod}[2]{\left({#1} \cdot {#2}\right)}
\newcommand{\crossprod}[2]{{#1} \cross {#2}}
\newcommand{\tripleprod}[3]{\dotprod{\left(\crossprod{#1}{#2}\right)}{#3}}

\DeclareMathOperator{\Proj}{Proj}
\DeclareMathOperator{\Span}{span}
\DeclareMathOperator{\Sgn}{sgn}
\DeclareMathOperator{\Area}{Area}
\DeclareMathOperator{\Volume}{Volume}

%
% A few miscellaneous things specific to this document
%
\newcommand{\crossop}[1]{\crossprod{#1}{}}

% R2 vector.
\newcommand{\VectorTwo}[2]{
\begin{bmatrix}
 {#1} \\
 {#2}
\end{bmatrix}
}

\newcommand{\VectorN}[1]{
\begin{bmatrix}
{#1}_1 \\
{#1}_2 \\
\vdots \\
{#1}_N \\
\end{bmatrix}
}

\newcommand{\DETuvij}[4]{
\begin{vmatrix}
 {#1}_{#3} & {#1}_{#4} \\
 {#2}_{#3} & {#2}_{#4}
\end{vmatrix}
}

\newcommand{\DETuvwijk}[6]{
\begin{vmatrix}
 {#1}_{#4} & {#1}_{#5} & {#1}_{#6} \\
 {#2}_{#4} & {#2}_{#5} & {#2}_{#6} \\
 {#3}_{#4} & {#3}_{#5} & {#3}_{#6}
\end{vmatrix}
}

\newcommand{\DETuvwxijkl}[8]{
\begin{vmatrix}
 {#1}_{#5} & {#1}_{#6} & {#1}_{#7} & {#1}_{#8} \\
 {#2}_{#5} & {#2}_{#6} & {#2}_{#7} & {#2}_{#8} \\
 {#3}_{#5} & {#3}_{#6} & {#3}_{#7} & {#3}_{#8} \\
 {#4}_{#5} & {#4}_{#6} & {#4}_{#7} & {#4}_{#8} \\
\end{vmatrix}
}

%\newcommand{\DETuvwxyijklm}[10]{
%\begin{vmatrix}
% {#1}_{#6} & {#1}_{#7} & {#1}_{#8} & {#1}_{#9} & {#1}_{#10} \\
% {#2}_{#6} & {#2}_{#7} & {#2}_{#8} & {#2}_{#9} & {#2}_{#10} \\
% {#3}_{#6} & {#3}_{#7} & {#3}_{#8} & {#3}_{#9} & {#3}_{#10} \\
% {#4}_{#6} & {#4}_{#7} & {#4}_{#8} & {#4}_{#9} & {#4}_{#10} \\
% {#5}_{#6} & {#5}_{#7} & {#5}_{#8} & {#5}_{#9} & {#5}_{#10}
%\end{vmatrix}
%}

% R3 vector.
\newcommand{\VectorThree}[3]{
\begin{bmatrix}
 {#1} \\
 {#2} \\
 {#3}
\end{bmatrix}
}


%<misc>
%
\newcommand{\Abs}[1]{{\left\lvert{#1}\right\rvert}}
\newcommand{\spacegrad}[0]{\boldsymbol{\nabla}}
\newcommand{\grad}[0]{\nabla}
\newcommand{\LL}[0]{\mathcal{L}}

% == \partial_{#1} {#2}
\newcommand{\PD}[2]{\frac{\partial {#2}}{\partial {#1}}}
% inline variant
\newcommand{\PDi}[2]{{\partial {#2}}/{\partial {#1}}}

\newcommand{\PDD}[3]{\frac{\partial^2 {#3}}{\partial {#1}\partial {#2}}}
%\newcommand{\PDd}[2]{\frac{\partial^2 {#2}}{{\partial{#1}}^2}}
\newcommand{\PDsq}[2]{\frac{\partial^2 {#2}}{(\partial {#1})^2}}

\newcommand{\Partial}[2]{\frac{\partial {#1}}{\partial {#2}}}
\DeclareMathOperator{\RejName}{Rej}
\newcommand{\Rej}[2]{\RejName_{#1}\left( {#2} \right)}
\newcommand{\Rm}[1]{\mathbb{R}^{#1}}
\newcommand{\Cm}[1]{\mathbb{C}^{#1}}
\newcommand{\conj}[0]{{*}}

%</misc>

% <grade selection>
%
\newcommand{\gpgrade}[2] {{\left\langle{{#1}}\right\rangle}_{#2}}

\newcommand{\gpgradezero}[1] {\gpgrade{#1}{}}
%\newcommand{\gpscalargrade}[1] {{\left\langle{{#1}}\right\rangle}}
%\newcommand{\gpgradezero}[1] {\gpgrade{#1}{0}}

%\newcommand{\gpgradeone}[1] {{\left\langle{{#1}}\right\rangle}_{1}}
\newcommand{\gpgradeone}[1] {\gpgrade{#1}{1}}

\newcommand{\gpgradetwo}[1] {\gpgrade{#1}{2}}
\newcommand{\gpgradethree}[1] {\gpgrade{#1}{3}}
\newcommand{\gpgradefour}[1] {\gpgrade{#1}{4}}
%
% </grade selection>



\newcommand{\adot}[0]{{\dot{a}}}
\newcommand{\bdot}[0]{{\dot{b}}}
% taken for centered dot:
%\newcommand{\cdot}[0]{{\dot{c}}}
%\newcommand{\ddot}[0]{{\dot{d}}}
\newcommand{\edot}[0]{{\dot{e}}}
\newcommand{\fdot}[0]{{\dot{f}}}
\newcommand{\gdot}[0]{{\dot{g}}}
\newcommand{\hdot}[0]{{\dot{h}}}
\newcommand{\idot}[0]{{\dot{i}}}
\newcommand{\jdot}[0]{{\dot{j}}}
\newcommand{\kdot}[0]{{\dot{k}}}
\newcommand{\ldot}[0]{{\dot{l}}}
\newcommand{\mdot}[0]{{\dot{m}}}
\newcommand{\ndot}[0]{{\dot{n}}}
%\newcommand{\odot}[0]{{\dot{o}}}
\newcommand{\pdot}[0]{{\dot{p}}}
\newcommand{\qdot}[0]{{\dot{q}}}
\newcommand{\rdot}[0]{{\dot{r}}}
\newcommand{\sdot}[0]{{\dot{s}}}
\newcommand{\tdot}[0]{{\dot{t}}}
\newcommand{\udot}[0]{{\dot{u}}}
\newcommand{\vdot}[0]{{\dot{v}}}
\newcommand{\wdot}[0]{{\dot{w}}}
\newcommand{\xdot}[0]{{\dot{x}}}
\newcommand{\ydot}[0]{{\dot{y}}}
\newcommand{\zdot}[0]{{\dot{z}}}
\newcommand{\addot}[0]{{\ddot{a}}}
\newcommand{\bddot}[0]{{\ddot{b}}}
\newcommand{\cddot}[0]{{\ddot{c}}}
%\newcommand{\dddot}[0]{{\ddot{d}}}
\newcommand{\eddot}[0]{{\ddot{e}}}
\newcommand{\fddot}[0]{{\ddot{f}}}
\newcommand{\gddot}[0]{{\ddot{g}}}
\newcommand{\hddot}[0]{{\ddot{h}}}
\newcommand{\iddot}[0]{{\ddot{i}}}
\newcommand{\jddot}[0]{{\ddot{j}}}
\newcommand{\kddot}[0]{{\ddot{k}}}
\newcommand{\lddot}[0]{{\ddot{l}}}
\newcommand{\mddot}[0]{{\ddot{m}}}
\newcommand{\nddot}[0]{{\ddot{n}}}
\newcommand{\oddot}[0]{{\ddot{o}}}
\newcommand{\pddot}[0]{{\ddot{p}}}
\newcommand{\qddot}[0]{{\ddot{q}}}
\newcommand{\rddot}[0]{{\ddot{r}}}
\newcommand{\sddot}[0]{{\ddot{s}}}
\newcommand{\tddot}[0]{{\ddot{t}}}
\newcommand{\uddot}[0]{{\ddot{u}}}
\newcommand{\vddot}[0]{{\ddot{v}}}
\newcommand{\wddot}[0]{{\ddot{w}}}
\newcommand{\xddot}[0]{{\ddot{x}}}
\newcommand{\yddot}[0]{{\ddot{y}}}
\newcommand{\zddot}[0]{{\ddot{z}}}

%<bold and dot greek symbols>
%

\newcommand{\Deltadot}[0]{{\dot{\Delta}}}
\newcommand{\Gammadot}[0]{{\dot{\Gamma}}}
\newcommand{\Lambdadot}[0]{{\dot{\Lambda}}}
\newcommand{\Omegadot}[0]{{\dot{\Omega}}}
\newcommand{\Phidot}[0]{{\dot{\Phi}}}
\newcommand{\Pidot}[0]{{\dot{\Pi}}}
\newcommand{\Psidot}[0]{{\dot{\Psi}}}
\newcommand{\Sigmadot}[0]{{\dot{\Sigma}}}
\newcommand{\Thetadot}[0]{{\dot{\Theta}}}
\newcommand{\Upsilondot}[0]{{\dot{\Upsilon}}}
\newcommand{\Xidot}[0]{{\dot{\Xi}}}
\newcommand{\alphadot}[0]{{\dot{\alpha}}}
\newcommand{\betadot}[0]{{\dot{\beta}}}
\newcommand{\chidot}[0]{{\dot{\chi}}}
\newcommand{\deltadot}[0]{{\dot{\delta}}}
\newcommand{\epsilondot}[0]{{\dot{\epsilon}}}
\newcommand{\etadot}[0]{{\dot{\eta}}}
\newcommand{\gammadot}[0]{{\dot{\gamma}}}
\newcommand{\kappadot}[0]{{\dot{\kappa}}}
\newcommand{\lambdadot}[0]{{\dot{\lambda}}}
\newcommand{\mudot}[0]{{\dot{\mu}}}
\newcommand{\nudot}[0]{{\dot{\nu}}}
\newcommand{\omegadot}[0]{{\dot{\omega}}}
\newcommand{\phidot}[0]{{\dot{\phi}}}
\newcommand{\pidot}[0]{{\dot{\pi}}}
\newcommand{\psidot}[0]{{\dot{\psi}}}
\newcommand{\rhodot}[0]{{\dot{\rho}}}
\newcommand{\sigmadot}[0]{{\dot{\sigma}}}
\newcommand{\taudot}[0]{{\dot{\tau}}}
\newcommand{\thetadot}[0]{{\dot{\theta}}}
\newcommand{\upsilondot}[0]{{\dot{\upsilon}}}
\newcommand{\varepsilondot}[0]{{\dot{\varepsilon}}}
\newcommand{\varphidot}[0]{{\dot{\varphi}}}
\newcommand{\varpidot}[0]{{\dot{\varpi}}}
\newcommand{\varrhodot}[0]{{\dot{\varrho}}}
\newcommand{\varsigmadot}[0]{{\dot{\varsigma}}}
\newcommand{\varthetadot}[0]{{\dot{\vartheta}}}
\newcommand{\xidot}[0]{{\dot{\xi}}}
\newcommand{\zetadot}[0]{{\dot{\zeta}}}

\newcommand{\Deltaddot}[0]{{\ddot{\Delta}}}
\newcommand{\Gammaddot}[0]{{\ddot{\Gamma}}}
\newcommand{\Lambdaddot}[0]{{\ddot{\Lambda}}}
\newcommand{\Omegaddot}[0]{{\ddot{\Omega}}}
\newcommand{\Phiddot}[0]{{\ddot{\Phi}}}
\newcommand{\Piddot}[0]{{\ddot{\Pi}}}
\newcommand{\Psiddot}[0]{{\ddot{\Psi}}}
\newcommand{\Sigmaddot}[0]{{\ddot{\Sigma}}}
\newcommand{\Thetaddot}[0]{{\ddot{\Theta}}}
\newcommand{\Upsilonddot}[0]{{\ddot{\Upsilon}}}
\newcommand{\Xiddot}[0]{{\ddot{\Xi}}}
\newcommand{\alphaddot}[0]{{\ddot{\alpha}}}
\newcommand{\betaddot}[0]{{\ddot{\beta}}}
\newcommand{\chiddot}[0]{{\ddot{\chi}}}
\newcommand{\deltaddot}[0]{{\ddot{\delta}}}
\newcommand{\epsilonddot}[0]{{\ddot{\epsilon}}}
\newcommand{\etaddot}[0]{{\ddot{\eta}}}
\newcommand{\gammaddot}[0]{{\ddot{\gamma}}}
\newcommand{\kappaddot}[0]{{\ddot{\kappa}}}
\newcommand{\lambdaddot}[0]{{\ddot{\lambda}}}
\newcommand{\muddot}[0]{{\ddot{\mu}}}
\newcommand{\nuddot}[0]{{\ddot{\nu}}}
\newcommand{\omegaddot}[0]{{\ddot{\omega}}}
\newcommand{\phiddot}[0]{{\ddot{\phi}}}
\newcommand{\piddot}[0]{{\ddot{\pi}}}
\newcommand{\psiddot}[0]{{\ddot{\psi}}}
\newcommand{\rhoddot}[0]{{\ddot{\rho}}}
\newcommand{\sigmaddot}[0]{{\ddot{\sigma}}}
\newcommand{\tauddot}[0]{{\ddot{\tau}}}
\newcommand{\thetaddot}[0]{{\ddot{\theta}}}
\newcommand{\upsilonddot}[0]{{\ddot{\upsilon}}}
\newcommand{\varepsilonddot}[0]{{\ddot{\varepsilon}}}
\newcommand{\varphiddot}[0]{{\ddot{\varphi}}}
\newcommand{\varpiddot}[0]{{\ddot{\varpi}}}
\newcommand{\varrhoddot}[0]{{\ddot{\varrho}}}
\newcommand{\varsigmaddot}[0]{{\ddot{\varsigma}}}
\newcommand{\varthetaddot}[0]{{\ddot{\vartheta}}}
\newcommand{\xiddot}[0]{{\ddot{\xi}}}
\newcommand{\zetaddot}[0]{{\ddot{\zeta}}}

\newcommand{\BDelta}[0]{\boldsymbol{\Delta}}
\newcommand{\BGamma}[0]{\boldsymbol{\Gamma}}
\newcommand{\BLambda}[0]{\boldsymbol{\Lambda}}
\newcommand{\BOmega}[0]{\boldsymbol{\Omega}}
\newcommand{\BPhi}[0]{\boldsymbol{\Phi}}
\newcommand{\BPi}[0]{\boldsymbol{\Pi}}
\newcommand{\BPsi}[0]{\boldsymbol{\Psi}}
\newcommand{\BSigma}[0]{\boldsymbol{\Sigma}}
\newcommand{\BTheta}[0]{\boldsymbol{\Theta}}
\newcommand{\BUpsilon}[0]{\boldsymbol{\Upsilon}}
\newcommand{\BXi}[0]{\boldsymbol{\Xi}}
\newcommand{\Balpha}[0]{\boldsymbol{\alpha}}
\newcommand{\Bbeta}[0]{\boldsymbol{\beta}}
\newcommand{\Bchi}[0]{\boldsymbol{\chi}}
\newcommand{\Bdelta}[0]{\boldsymbol{\delta}}
\newcommand{\Bepsilon}[0]{\boldsymbol{\epsilon}}
\newcommand{\Beta}[0]{\boldsymbol{\eta}}
\newcommand{\Bgamma}[0]{\boldsymbol{\gamma}}
\newcommand{\Bkappa}[0]{\boldsymbol{\kappa}}
\newcommand{\Blambda}[0]{\boldsymbol{\lambda}}
\newcommand{\Bmu}[0]{\boldsymbol{\mu}}
\newcommand{\Bnu}[0]{\boldsymbol{\nu}}
%\newcommand{\Bomega}[0]{\boldsymbol{\omega}}
\newcommand{\Bphi}[0]{\boldsymbol{\phi}}
\newcommand{\Bpi}[0]{\boldsymbol{\pi}}
\newcommand{\Bpsi}[0]{\boldsymbol{\psi}}
\newcommand{\Brho}[0]{\boldsymbol{\rho}}
\newcommand{\Bsigma}[0]{\boldsymbol{\sigma}}
%\newcommand{\Btau}[0]{\boldsymbol{\tau}}
%\newcommand{\Btheta}[0]{\boldsymbol{\theta}}
\newcommand{\Bupsilon}[0]{\boldsymbol{\upsilon}}
\newcommand{\Bvarepsilon}[0]{\boldsymbol{\varepsilon}}
\newcommand{\Bvarphi}[0]{\boldsymbol{\varphi}}
\newcommand{\Bvarpi}[0]{\boldsymbol{\varpi}}
\newcommand{\Bvarrho}[0]{\boldsymbol{\varrho}}
\newcommand{\Bvarsigma}[0]{\boldsymbol{\varsigma}}
\newcommand{\Bvartheta}[0]{\boldsymbol{\vartheta}}
\newcommand{\Bxi}[0]{\boldsymbol{\xi}}
\newcommand{\Bzeta}[0]{\boldsymbol{\zeta}}
%
%</bold and dot greek symbols>
%<infrequent>
%
%\newcommand{\AreaOp}[1]{\AName_{#1}}
%\newcommand{\Babs}[0]{\abs{\BB}}
%\newcommand{\Bcap}[0]{\hat{\BB}}
%\newcommand{\BrPrimeRej}[0]{\rcap(\rcap \wedge \Br')}
%\newcommand{\CA}[0]{\mathcal{A}}
%\newcommand{\Cos}[1]{\cos{\left({#1}\right)}}
%\newcommand{\Det}[1] {\abs{#1}}
%\newcommand{\Dsq}[2] {\frac {\partial^2 {#1}} {\partial {#2}^2}}
%\newcommand{\Exp}[1]{\exp{\left({#1}\right)}}
%\newcommand{\Norm}[1]{\left\lVert{#1}\right\rVert}
%\newcommand{\Sin}[1]{\sin{\left({#1}\right)}}
%\newcommand{\T}[0]{\text{T}}
%\newcommand{\VolumeOp}[1]{\VName_{#1}}
%\newcommand{\agrad}[0]{\Ba \cdot \nabla}
%\newcommand{\alphacap}[0]{\hat{\boldsymbol{\alpha}}}
%\newcommand{\Fcap}[0]{\hat{\BF}}
%\newcommand{\bithree}[0]{{\Bi}_3}
%\newcommand{\bxa}[0]{\Bx\Ba}
%\newcommand{\coordvec}[2]{
%\newcommand{\costheta}[0]{\acap \cdot \xcap}
%\newcommand{\ddt}[1]{\ddot{#1}}
%\newcommand{\ddu}[1] {\frac {d{#1}} {du}}
%\newcommand{\dsqxj}[2] {\frac {\partial^2 {#1}} {\partial {x_{#2}}^2}}
%\newcommand{\dtheta}[1]{\frac{d {#1}}{d \theta}}
%\newcommand{\dt}[1]{\dot{#1}}
%\newcommand{\dt}[1]{\frac{d {#1}}{dt}}
%\newcommand{\dxj}[2] {\frac {\partial {#1}} {\partial {x_{#2}}}}
%\newcommand{\halfPhi}[0]{\frac{\phi}{2}}
%\newcommand{\half}[0]{\inv{2}}
%\newcommand{\inv}[1]{\frac{1}{#1}}
%\newcommand{\laplacian}[0]{\nabla^2}
%\newcommand{\matrixoftx}[3]{
%\newcommand{\nrrp}[0]{\norm{\rcap \wedge \Br'}}
%\newcommand{\oiint}{\bigcirc \hspace{-1.4em} \int \hspace{-.8em} \int}
%\newcommand{\transpose}[1]{{#1}^{\text{T}}}
%\newcommand{\transpose}[1]{{{#1}^{\TextTranspose}}}
%\newcommand{\transpose}[1]{{{#1}^{\text{T}}}}
%\newcommand{\barA}[0]{\bar{A}}
%\newcommand{\qbar}[0]{\bar{q}}
%\newcommand{\qdotbar}[0]{\dot{\bar{q}}}
%
%</infrequent>




\newcommand{\PDSq}[2]{\frac{\partial^2 {#2}}{\partial {#1}^2}}

\usepackage[bookmarks=true]{hyperref}

\usepackage{color,cite,graphicx}
   % use colour in the document, put your citations as [1-4]
   % rather than [1,2,3,4] (it looks nicer, and the extended LaTeX2e
   % graphics package. 
\usepackage{latexsym,amssymb,epsf} % don't remember if these are
   % needed, but their inclusion can't do any damage

\title{ Ehrenfest's theorem. }
\author{Peeter Joot}
\date{ Jan 22, 2009.  Last Revision: $Date: 2009/01/24 02:41:28 $ }

\begin{document}

\maketitle{}

\tableofcontents
\section{ Motivation. }

\cite{mcmahon2005qmd} has a one dimensional treatment of Ehrenfest's theorem,
that the expectation values of the position and momentum operators behave
like Newton's law.

However, he makes use
of commutator and braket notation before either is defined.

That looks like a natural way to do the derivation easily, but let's try
this using instead what is defined up to this point in the text.

\section{ Review.  What do we know so far? }

\subsection{ Position and momentum operators. }

We have been given the definitions of two specific operators, position and momentum, 
whos action on a wave function is

\begin{align*}
\hat{x} \psi &= x \psi \\
\hat{p} \psi &= \left(-i \hbar \PD{x}{}\right) \psi
\end{align*}

In operator form, with the omission of the explicit wave function being operated on this is

\begin{align*}
\hat{x} &\equiv x  \\
\hat{p} &\equiv -i \hbar \PD{x}{}
\end{align*}

These are perfectly valid operator definitions, but the validity of using the 
classical names for these really comes from this upcoming Ehrenfest result where
the average of the action of these operators on a wave function is examined.

\subsection{ Expectation (average) value of an operator. }

We also have a definition for the expectation value
of an operator $\hat{A}$, given its specific action $A$.
This is defined very much like a weighted inner product
and is essentially a field weighted average of the operators action

\begin{align*}
<\hat{A}> \equiv \int \psi^\conj (A \psi)
\end{align*}

The braces show that the operator action $A$ here applies to the rightmost field variable $\psi$, and not to its conjugate.

For the position and momentum operators respectively, we have the
expectation values

\begin{align*}
<\hat{x}> &\equiv \int \psi^\conj (x \psi) \\
<\hat{p}> &\equiv \int \psi^\conj \left(-i \hbar \PD{x}{}\right) \psi
\end{align*}

\subsection{ Hermitian operator. }

The notation of a Hermitian operator as also been introduced in terms of 
left acting operators.  That is, an operator $\hat{A}$ is hermitian if

\begin{align}\label{eqn:hermitian1}
\int \psi^\conj (A \psi) = \int (\psi A)^\conj \psi
\end{align}

This is a somewhat non-Demystified seeming definition to me since I'd seen
Hermitian defined more directly in terms of ``normal'' right acting 
expectation integrals.  That is, an operator $\hat{A}$ is Hermitian if

\begin{align*}
<\hat{A}>^\conj = <\hat{A}>
\end{align*}

The conjugate of an operator's expectation value is
\begin{align*}
\left(\int \psi^\conj (A \psi)\right)^\conj 
&= \int \psi (A^\conj \psi^\conj) \\
&= \int (A^\conj \psi^\conj) \psi \\
\end{align*}

So, this second Hermitian definition means that an operator is Hermitian if

\begin{align*}
\int (A^\conj \psi^\conj) \psi &= \int \psi^\conj (A \psi)
\end{align*}

This highlights why the left acting operator notation is pretty reasonable
seeming.  Allowing the conjugation operation to switch an operators action
from right acting to left acting makes the equation prettier, and 
recovers equation \ref{eqn:hermitian1}

\begin{align*}
(\psi A)^\conj \equiv (A^\conj \psi^\conj) 
\end{align*}

Here braces have been used to express the limitation of the scope of the action of the operator.

Another way to express this is that one can say that a Hermitian operator when put 
in its wave function sandwhich has a 
conjugate action acting to the left on the conjugate wave function and a non-conjugate
action to the right.  This allows for a final notation nicety, where one can omit the 
braces entirely as in

\begin{align*}
\int \psi^\conj (A \psi) \equiv \int \psi^\conj A \psi \equiv \int (A^\conj \psi^\conj) \psi
\end{align*}
or in terms of right and left operator notation the equivalent
\begin{align*}
\int \psi^\conj (A \psi) \equiv \int \psi^\conj A \psi \equiv \int (\psi A)^\conj \psi
\end{align*}

And finally, there is one last way to express this the concept of Hermitian.
We have our definition of a left acting operator

\begin{align*}
(\psi A)^\conj = A^\conj \psi^\conj
\end{align*}

And can make the observation that conjugation of a product is the 
product of the conjugates
\begin{align*}
(\psi A)^\conj = \psi^\conj A^\conj
\end{align*}

So we must also have $A = A^\conj$ for a Hermitian operator.  

From this one can observe that the position operator $\hat{x}$ is Hermitian, but the momentum operator is not (but $\hat{p}^2$ is ).

\subsection{ Variance and Heisenberg principle. }

Various calculations have been done to calcualate expectation values.

In a few places we have had to show that the product of variances

\begin{align*}
\Delta A = \sqrt{<A^2> - <A>^2}
\end{align*}

for position and momentum all satisfy the famous Heisenberg uncertainty
principle

\begin{align*}
\Delta x \Delta p \ge \hbar/2
\end{align*}

(in a couple places this formulation is a bit fuzzy since our squared
momentum variance $(\Delta p)^2$ has been negative).

\subsection{ The wave equation. }

We are also given Schr\"{o}dinger equation in Hamiltonian form

\begin{align*}
\hat{H} \psi = i \hbar \PD{t}{\psi}
\end{align*}

and have worked with the specific form of the Hamiltonian that applies to
a non-relativistic particle (and not to photons).

\begin{align*}
\hat{H} = \frac{\hat{p}^2}{2m} + V = -\frac{\hbar^2}{2m} \grad^2 + V
\end{align*}

Most of the text up to this point has been about calculating and interpretting
specific solutions of this equation.

\subsection{ Other stuff. }

A number of other fundamental topics have been covered, probabilities, normalization, probability current, energy, phase, orthogonality, and so forth.  However, summarizing the rest of these in detail is not required as 
background for the Ehrenfest result.

\section{ Ehrenfest theorem. }

We want to calculate the time derivatives of the expectation values
for position and momentum OPERATORS, and show that these reproduce the
familiar velocity, momentum and force concepts from classical mechanics.

\subsection{ Velocity from the derivative of the position operator expectation. }

Diving straight in we have

\begin{align*}
\PD{t}{<\hat{x}>}
&= \PD{t}{} \left( \int \psi^\conj x \psi \right) \\
&= \int \PD{t}{\psi^\conj} x \psi + \int \psi^\conj x \PD{t}{\psi} 
\end{align*}

Now, here the Hamiltonian can be introduced, replacing the time derivatives.

We have
\begin{align*}
\PD{t}{\psi} &= -\frac{i}{\hbar} H \psi \\
\PD{t}{\psi^\conj} &= \frac{i}{\hbar} H \psi^\conj \\
\end{align*}

So we have
\begin{align*}
\PD{t}{<\hat{x}>}
&= \frac{i}{\hbar} \int \psi x H {\psi^\conj} - \frac{i}{\hbar}\int \psi^\conj x H {\psi} 
\end{align*}

For the Schr\"{o}dinger Hamiltonian we have

\begin{align*}
H \psi &= - \frac{\hbar^2}{2m} \PDSq{x}{\psi} + V\psi \\
H \psi^\conj &= - \frac{\hbar^2}{2m} \PDSq{x}{\psi^\conj} + V\psi^\conj \\
\end{align*}

Combining these we have
\begin{align*}
\PD{t}{<\hat{x}>}
&= 
\frac{i}{\hbar} \int \psi x \left( - \frac{\hbar^2}{2m} \PDSq{x}{\psi^\conj} + V\psi^\conj \right) 
-\frac{i}{\hbar} \int \psi^\conj x \left(- \frac{\hbar^2}{2m} \PDSq{x}{\psi} + V\psi \right) \\
&= 
\frac{i\hbar}{2m} \int \left(\psi^\conj x \PDSq{x}{\psi} -\psi x \PDSq{x}{\psi^\conj} \right)
+\frac{i}{\hbar} \int \left(\psi x V\psi^\conj -\psi^\conj x V\psi \right) 
\\
\end{align*}

The second term is zero, and by integrating the first term by parts twice we have

\begin{align*}
\PD{t}{<\hat{x}>}
&= \frac{i\hbar}{2m} \int \psi^\conj \left(x \PDSq{x}{\psi} - \PDSq{x}{(\psi x)} \right) \\
&= \frac{i\hbar}{2m} \int \psi^\conj \left(x \PDSq{x}{\psi} - \PD{x}{}\left(x \PD{x}{\psi} + \psi\right) \right) \\
&= \frac{-i\hbar}{2m} (2) \int \psi^\conj \PD{x}{\psi} \\
&= \frac{1}{m} \int \psi^\conj \left(-i\hbar \PD{x}{} \right) {\psi} \\
\end{align*}

So we now have the QM equivalent of $p = mv$, directly from the Sch\"{o}dinger equation and the definition of expectation values
of operators.

\begin{align}
\PD{t}{<\hat{x}>} &= \frac{<p>}{m} 
\end{align}

This is the first inkling that it makes sense to assign the names position and momentum to the corresponding operators
of QM!  Now the QMD derivation is way shorter and tidier, but this needed only integration by parts.  We really don't
need the more advanced operator concepts to get this important result.  

\subsection{ Force from the derivative of the momentum operator expectation. }

Now lets calculate the momentum expectation change with time.

\begin{align*}
\PD{t}{<p>} 
&= \PD{t}{} \int \psi^\conj \left(-i \hbar \PD{x}{}\right) \psi \\
&= -i \hbar \int \PD{t}{\psi^\conj} \PD{x}{\psi} +{\psi^\conj} \PD{t}{}\PD{x}{\psi} \\
&= -i \hbar \int \PD{t}{\psi^\conj} \PD{x}{\psi} +{\psi^\conj} \PD{x}{}\PD{t}{\psi} \\
&= \int \PD{x}{\psi} H \psi^\conj - {\psi^\conj} \PD{x}{} H\psi \\
&= \int \PD{x}{\psi} H \psi^\conj + \PD{x}{\psi^\conj} H\psi \\
&= 
\int \PD{x}{\psi} \left(- \frac{\hbar^2}{2m} \PDSq{x}{\psi^\conj} + V\psi^\conj \right)
+ \PD{x}{\psi^\conj} \left(- \frac{\hbar^2}{2m} \PDSq{x}{\psi} + V\psi \right) 
\\
&= 
-
\frac{\hbar^2}{2m} 
\int \PD{x}{\psi} \PDSq{x}{\psi^\conj} + \PD{x}{\psi^\conj} \PDSq{x}{\psi} 
+\int \PD{x}{\psi} V\psi^\conj + \PD{x}{\psi^\conj} V\psi 
\\
&= 
- \frac{\hbar^2}{2m} \int \PD{x}{}\left(\PD{x}{\psi} \PD{x}{\psi^\conj}\right)
+\int \PD{x}{} \left( \psi V\psi^\conj \right) -\int \psi \PD{x}{V} \psi^\conj 
\\
\end{align*}

Now, again with the assumption that $\psi$ and its derivatives are sufficiently small to vanish at the boundaries of the integration (this was also done in the integration by parts above), the first two terms are zero, and the last is an expectation value.  Specifically, we then have

\begin{align}
\PD{t}{<p>} &= - \left<\PD{x}{V}\right>
\end{align}

... which appears to be the QM equivalent to the one dimensional version of $F = -\grad V$, instead all defined in terms of expectation values.

Very cool!  Now, before learning the Lagrangian formalism, I would have been satisfied with this.  We can replace Newton's law with 
Schr\"{o}dinger's equation, and logically everything else will follow from that.  Can we apply a procedure like this to 
the Lagrangian for the wave equation, and find an expectation equivalent to the classical $\LL = m\Bv^2/2 - V$?

An additional obvious question is how to express the expectation value in the three dimensional case instead of the one dimensional case?

\bibliographystyle{plainnat}
\bibliography{myrefs}

\end{document}
           % jan 22/09
%\documentclass{article}

%\usepackage{amsmath}
\usepackage{mathpazo}

%
% shorthand for bold symbols, convenient for vectors and matrices
%
\newcommand{\Ba}[0]{\mathbf{a}}
\newcommand{\Bb}[0]{\mathbf{b}}
\newcommand{\Bc}[0]{\mathbf{c}}
\newcommand{\Bd}[0]{\mathbf{d}}
\newcommand{\Be}[0]{\mathbf{e}}
\newcommand{\Bf}[0]{\mathbf{f}}
\newcommand{\Bg}[0]{\mathbf{g}}
\newcommand{\Bh}[0]{\mathbf{h}}
\newcommand{\Bi}[0]{\mathbf{i}}
\newcommand{\Bj}[0]{\mathbf{j}}
\newcommand{\Bk}[0]{\mathbf{k}}
\newcommand{\Bl}[0]{\mathbf{l}}
\newcommand{\Bm}[0]{\mathbf{m}}
\newcommand{\Bn}[0]{\mathbf{n}}
\newcommand{\Bo}[0]{\mathbf{o}}
\newcommand{\Bp}[0]{\mathbf{p}}
\newcommand{\Bq}[0]{\mathbf{q}}
\newcommand{\Br}[0]{\mathbf{r}}
\newcommand{\Bs}[0]{\mathbf{s}}
\newcommand{\Bt}[0]{\mathbf{t}}
\newcommand{\Bu}[0]{\mathbf{u}}
\newcommand{\Bv}[0]{\mathbf{v}}
\newcommand{\Bw}[0]{\mathbf{w}}
\newcommand{\Bx}[0]{\mathbf{x}}
\newcommand{\By}[0]{\mathbf{y}}
\newcommand{\Bz}[0]{\mathbf{z}}
\newcommand{\BA}[0]{\mathbf{A}}
\newcommand{\BB}[0]{\mathbf{B}}
\newcommand{\BC}[0]{\mathbf{C}}
\newcommand{\BD}[0]{\mathbf{D}}
\newcommand{\BE}[0]{\mathbf{E}}
\newcommand{\BF}[0]{\mathbf{F}}
\newcommand{\BG}[0]{\mathbf{G}}
\newcommand{\BH}[0]{\mathbf{H}}
\newcommand{\BI}[0]{\mathbf{I}}
\newcommand{\BJ}[0]{\mathbf{J}}
\newcommand{\BK}[0]{\mathbf{K}}
\newcommand{\BL}[0]{\mathbf{L}}
\newcommand{\BM}[0]{\mathbf{M}}
\newcommand{\BN}[0]{\mathbf{N}}
\newcommand{\BO}[0]{\mathbf{O}}
\newcommand{\BP}[0]{\mathbf{P}}
\newcommand{\BQ}[0]{\mathbf{Q}}
\newcommand{\BR}[0]{\mathbf{R}}
\newcommand{\BS}[0]{\mathbf{S}}
\newcommand{\BT}[0]{\mathbf{T}}
\newcommand{\BU}[0]{\mathbf{U}}
\newcommand{\BV}[0]{\mathbf{V}}
\newcommand{\BW}[0]{\mathbf{W}}
\newcommand{\BX}[0]{\mathbf{X}}
\newcommand{\BY}[0]{\mathbf{Y}}
\newcommand{\BZ}[0]{\mathbf{Z}}

\newcommand{\Bzero}[0]{\mathbf{0}}
\newcommand{\Btheta}[0]{\boldsymbol{\theta}}
\newcommand{\Btau}[0]{\boldsymbol{\tau}}
\newcommand{\Bomega}[0]{\boldsymbol{\omega}}

%
% shorthand for unit vectors
%
\newcommand{\acap}[0]{\hat{\Ba}}
\newcommand{\bcap}[0]{\hat{\Bb}}
\newcommand{\ccap}[0]{\hat{\Bc}}
\newcommand{\dcap}[0]{\hat{\Bd}}
\newcommand{\ecap}[0]{\hat{\Be}}
\newcommand{\fcap}[0]{\hat{\Bf}}
\newcommand{\gcap}[0]{\hat{\Bg}}
\newcommand{\hcap}[0]{\hat{\Bh}}
\newcommand{\icap}[0]{\hat{\Bi}}
\newcommand{\jcap}[0]{\hat{\Bj}}
\newcommand{\kcap}[0]{\hat{\Bk}}
\newcommand{\lcap}[0]{\hat{\Bl}}
\newcommand{\mcap}[0]{\hat{\Bm}}
\newcommand{\ncap}[0]{\hat{\Bn}}
\newcommand{\ocap}[0]{\hat{\Bo}}
\newcommand{\pcap}[0]{\hat{\Bp}}
\newcommand{\qcap}[0]{\hat{\Bq}}
\newcommand{\rcap}[0]{\hat{\Br}}
\newcommand{\scap}[0]{\hat{\Bs}}
\newcommand{\tcap}[0]{\hat{\Bt}}
\newcommand{\ucap}[0]{\hat{\Bu}}
\newcommand{\vcap}[0]{\hat{\Bv}}
\newcommand{\wcap}[0]{\hat{\Bw}}
\newcommand{\xcap}[0]{\hat{\Bx}}
\newcommand{\ycap}[0]{\hat{\By}}
\newcommand{\zcap}[0]{\hat{\Bz}}
\newcommand{\thetacap}[0]{\hat{\Btheta}}

%
% to write R^n and C^n in a distinguishable fashion.  Perhaps change this
% to the double lined characters upon figuring out how to do so.
%
\newcommand{\C}[1]{$\mathbb{C}^{#1}$}
\newcommand{\R}[1]{$\mathbb{R}^{#1}$}

%
% various generally useful helpers
%

% derivative of #1 wrt. #2:
\newcommand{\D}[2] {\frac {d#2} {d#1}}

\newcommand{\inv}[1]{\frac{1}{#1}}
\newcommand{\cross}[0]{\times}

\newcommand{\abs}[1]{\lvert{#1}\rvert}
\newcommand{\norm}[1]{\lVert{#1}\rVert}
\newcommand{\innerprod}[2]{\langle{#1}, {#2}\rangle}
\newcommand{\dotprod}[2]{{#1} \cdot {#2}}
\newcommand{\bdotprod}[2]{\left({#1} \cdot {#2}\right)}
\newcommand{\crossprod}[2]{{#1} \cross {#2}}
\newcommand{\tripleprod}[3]{\dotprod{\left(\crossprod{#1}{#2}\right)}{#3}}

\DeclareMathOperator{\Proj}{Proj}
\DeclareMathOperator{\Span}{span}
\DeclareMathOperator{\Sgn}{sgn}
\DeclareMathOperator{\Area}{Area}
\DeclareMathOperator{\Volume}{Volume}

%
% A few miscellaneous things specific to this document
%
\newcommand{\crossop}[1]{\crossprod{#1}{}}

% R2 vector.
\newcommand{\VectorTwo}[2]{
\begin{bmatrix}
 {#1} \\
 {#2}
\end{bmatrix}
}

\newcommand{\VectorN}[1]{
\begin{bmatrix}
{#1}_1 \\
{#1}_2 \\
\vdots \\
{#1}_N \\
\end{bmatrix}
}

\newcommand{\DETuvij}[4]{
\begin{vmatrix}
 {#1}_{#3} & {#1}_{#4} \\
 {#2}_{#3} & {#2}_{#4}
\end{vmatrix}
}

\newcommand{\DETuvwijk}[6]{
\begin{vmatrix}
 {#1}_{#4} & {#1}_{#5} & {#1}_{#6} \\
 {#2}_{#4} & {#2}_{#5} & {#2}_{#6} \\
 {#3}_{#4} & {#3}_{#5} & {#3}_{#6}
\end{vmatrix}
}

\newcommand{\DETuvwxijkl}[8]{
\begin{vmatrix}
 {#1}_{#5} & {#1}_{#6} & {#1}_{#7} & {#1}_{#8} \\
 {#2}_{#5} & {#2}_{#6} & {#2}_{#7} & {#2}_{#8} \\
 {#3}_{#5} & {#3}_{#6} & {#3}_{#7} & {#3}_{#8} \\
 {#4}_{#5} & {#4}_{#6} & {#4}_{#7} & {#4}_{#8} \\
\end{vmatrix}
}

%\newcommand{\DETuvwxyijklm}[10]{
%\begin{vmatrix}
% {#1}_{#6} & {#1}_{#7} & {#1}_{#8} & {#1}_{#9} & {#1}_{#10} \\
% {#2}_{#6} & {#2}_{#7} & {#2}_{#8} & {#2}_{#9} & {#2}_{#10} \\
% {#3}_{#6} & {#3}_{#7} & {#3}_{#8} & {#3}_{#9} & {#3}_{#10} \\
% {#4}_{#6} & {#4}_{#7} & {#4}_{#8} & {#4}_{#9} & {#4}_{#10} \\
% {#5}_{#6} & {#5}_{#7} & {#5}_{#8} & {#5}_{#9} & {#5}_{#10}
%\end{vmatrix}
%}

% R3 vector.
\newcommand{\VectorThree}[3]{
\begin{bmatrix}
 {#1} \\
 {#2} \\
 {#3}
\end{bmatrix}
}


%%<misc>
%
\newcommand{\Abs}[1]{{\left\lvert{#1}\right\rvert}}
\newcommand{\spacegrad}[0]{\boldsymbol{\nabla}}
\newcommand{\grad}[0]{\nabla}
\newcommand{\LL}[0]{\mathcal{L}}

% == \partial_{#1} {#2}
\newcommand{\PD}[2]{\frac{\partial {#2}}{\partial {#1}}}
% inline variant
\newcommand{\PDi}[2]{{\partial {#2}}/{\partial {#1}}}

\newcommand{\PDD}[3]{\frac{\partial^2 {#3}}{\partial {#1}\partial {#2}}}
%\newcommand{\PDd}[2]{\frac{\partial^2 {#2}}{{\partial{#1}}^2}}
\newcommand{\PDsq}[2]{\frac{\partial^2 {#2}}{(\partial {#1})^2}}

\newcommand{\Partial}[2]{\frac{\partial {#1}}{\partial {#2}}}
\DeclareMathOperator{\RejName}{Rej}
\newcommand{\Rej}[2]{\RejName_{#1}\left( {#2} \right)}
\newcommand{\Rm}[1]{\mathbb{R}^{#1}}
\newcommand{\Cm}[1]{\mathbb{C}^{#1}}
\newcommand{\conj}[0]{{*}}

%</misc>

% <grade selection>
%
\newcommand{\gpgrade}[2] {{\left\langle{{#1}}\right\rangle}_{#2}}

\newcommand{\gpgradezero}[1] {\gpgrade{#1}{}}
%\newcommand{\gpscalargrade}[1] {{\left\langle{{#1}}\right\rangle}}
%\newcommand{\gpgradezero}[1] {\gpgrade{#1}{0}}

%\newcommand{\gpgradeone}[1] {{\left\langle{{#1}}\right\rangle}_{1}}
\newcommand{\gpgradeone}[1] {\gpgrade{#1}{1}}

\newcommand{\gpgradetwo}[1] {\gpgrade{#1}{2}}
\newcommand{\gpgradethree}[1] {\gpgrade{#1}{3}}
\newcommand{\gpgradefour}[1] {\gpgrade{#1}{4}}
%
% </grade selection>



\newcommand{\adot}[0]{{\dot{a}}}
\newcommand{\bdot}[0]{{\dot{b}}}
% taken for centered dot:
%\newcommand{\cdot}[0]{{\dot{c}}}
%\newcommand{\ddot}[0]{{\dot{d}}}
\newcommand{\edot}[0]{{\dot{e}}}
\newcommand{\fdot}[0]{{\dot{f}}}
\newcommand{\gdot}[0]{{\dot{g}}}
\newcommand{\hdot}[0]{{\dot{h}}}
\newcommand{\idot}[0]{{\dot{i}}}
\newcommand{\jdot}[0]{{\dot{j}}}
\newcommand{\kdot}[0]{{\dot{k}}}
\newcommand{\ldot}[0]{{\dot{l}}}
\newcommand{\mdot}[0]{{\dot{m}}}
\newcommand{\ndot}[0]{{\dot{n}}}
%\newcommand{\odot}[0]{{\dot{o}}}
\newcommand{\pdot}[0]{{\dot{p}}}
\newcommand{\qdot}[0]{{\dot{q}}}
\newcommand{\rdot}[0]{{\dot{r}}}
\newcommand{\sdot}[0]{{\dot{s}}}
\newcommand{\tdot}[0]{{\dot{t}}}
\newcommand{\udot}[0]{{\dot{u}}}
\newcommand{\vdot}[0]{{\dot{v}}}
\newcommand{\wdot}[0]{{\dot{w}}}
\newcommand{\xdot}[0]{{\dot{x}}}
\newcommand{\ydot}[0]{{\dot{y}}}
\newcommand{\zdot}[0]{{\dot{z}}}
\newcommand{\addot}[0]{{\ddot{a}}}
\newcommand{\bddot}[0]{{\ddot{b}}}
\newcommand{\cddot}[0]{{\ddot{c}}}
%\newcommand{\dddot}[0]{{\ddot{d}}}
\newcommand{\eddot}[0]{{\ddot{e}}}
\newcommand{\fddot}[0]{{\ddot{f}}}
\newcommand{\gddot}[0]{{\ddot{g}}}
\newcommand{\hddot}[0]{{\ddot{h}}}
\newcommand{\iddot}[0]{{\ddot{i}}}
\newcommand{\jddot}[0]{{\ddot{j}}}
\newcommand{\kddot}[0]{{\ddot{k}}}
\newcommand{\lddot}[0]{{\ddot{l}}}
\newcommand{\mddot}[0]{{\ddot{m}}}
\newcommand{\nddot}[0]{{\ddot{n}}}
\newcommand{\oddot}[0]{{\ddot{o}}}
\newcommand{\pddot}[0]{{\ddot{p}}}
\newcommand{\qddot}[0]{{\ddot{q}}}
\newcommand{\rddot}[0]{{\ddot{r}}}
\newcommand{\sddot}[0]{{\ddot{s}}}
\newcommand{\tddot}[0]{{\ddot{t}}}
\newcommand{\uddot}[0]{{\ddot{u}}}
\newcommand{\vddot}[0]{{\ddot{v}}}
\newcommand{\wddot}[0]{{\ddot{w}}}
\newcommand{\xddot}[0]{{\ddot{x}}}
\newcommand{\yddot}[0]{{\ddot{y}}}
\newcommand{\zddot}[0]{{\ddot{z}}}

%<bold and dot greek symbols>
%

\newcommand{\Deltadot}[0]{{\dot{\Delta}}}
\newcommand{\Gammadot}[0]{{\dot{\Gamma}}}
\newcommand{\Lambdadot}[0]{{\dot{\Lambda}}}
\newcommand{\Omegadot}[0]{{\dot{\Omega}}}
\newcommand{\Phidot}[0]{{\dot{\Phi}}}
\newcommand{\Pidot}[0]{{\dot{\Pi}}}
\newcommand{\Psidot}[0]{{\dot{\Psi}}}
\newcommand{\Sigmadot}[0]{{\dot{\Sigma}}}
\newcommand{\Thetadot}[0]{{\dot{\Theta}}}
\newcommand{\Upsilondot}[0]{{\dot{\Upsilon}}}
\newcommand{\Xidot}[0]{{\dot{\Xi}}}
\newcommand{\alphadot}[0]{{\dot{\alpha}}}
\newcommand{\betadot}[0]{{\dot{\beta}}}
\newcommand{\chidot}[0]{{\dot{\chi}}}
\newcommand{\deltadot}[0]{{\dot{\delta}}}
\newcommand{\epsilondot}[0]{{\dot{\epsilon}}}
\newcommand{\etadot}[0]{{\dot{\eta}}}
\newcommand{\gammadot}[0]{{\dot{\gamma}}}
\newcommand{\kappadot}[0]{{\dot{\kappa}}}
\newcommand{\lambdadot}[0]{{\dot{\lambda}}}
\newcommand{\mudot}[0]{{\dot{\mu}}}
\newcommand{\nudot}[0]{{\dot{\nu}}}
\newcommand{\omegadot}[0]{{\dot{\omega}}}
\newcommand{\phidot}[0]{{\dot{\phi}}}
\newcommand{\pidot}[0]{{\dot{\pi}}}
\newcommand{\psidot}[0]{{\dot{\psi}}}
\newcommand{\rhodot}[0]{{\dot{\rho}}}
\newcommand{\sigmadot}[0]{{\dot{\sigma}}}
\newcommand{\taudot}[0]{{\dot{\tau}}}
\newcommand{\thetadot}[0]{{\dot{\theta}}}
\newcommand{\upsilondot}[0]{{\dot{\upsilon}}}
\newcommand{\varepsilondot}[0]{{\dot{\varepsilon}}}
\newcommand{\varphidot}[0]{{\dot{\varphi}}}
\newcommand{\varpidot}[0]{{\dot{\varpi}}}
\newcommand{\varrhodot}[0]{{\dot{\varrho}}}
\newcommand{\varsigmadot}[0]{{\dot{\varsigma}}}
\newcommand{\varthetadot}[0]{{\dot{\vartheta}}}
\newcommand{\xidot}[0]{{\dot{\xi}}}
\newcommand{\zetadot}[0]{{\dot{\zeta}}}

\newcommand{\Deltaddot}[0]{{\ddot{\Delta}}}
\newcommand{\Gammaddot}[0]{{\ddot{\Gamma}}}
\newcommand{\Lambdaddot}[0]{{\ddot{\Lambda}}}
\newcommand{\Omegaddot}[0]{{\ddot{\Omega}}}
\newcommand{\Phiddot}[0]{{\ddot{\Phi}}}
\newcommand{\Piddot}[0]{{\ddot{\Pi}}}
\newcommand{\Psiddot}[0]{{\ddot{\Psi}}}
\newcommand{\Sigmaddot}[0]{{\ddot{\Sigma}}}
\newcommand{\Thetaddot}[0]{{\ddot{\Theta}}}
\newcommand{\Upsilonddot}[0]{{\ddot{\Upsilon}}}
\newcommand{\Xiddot}[0]{{\ddot{\Xi}}}
\newcommand{\alphaddot}[0]{{\ddot{\alpha}}}
\newcommand{\betaddot}[0]{{\ddot{\beta}}}
\newcommand{\chiddot}[0]{{\ddot{\chi}}}
\newcommand{\deltaddot}[0]{{\ddot{\delta}}}
\newcommand{\epsilonddot}[0]{{\ddot{\epsilon}}}
\newcommand{\etaddot}[0]{{\ddot{\eta}}}
\newcommand{\gammaddot}[0]{{\ddot{\gamma}}}
\newcommand{\kappaddot}[0]{{\ddot{\kappa}}}
\newcommand{\lambdaddot}[0]{{\ddot{\lambda}}}
\newcommand{\muddot}[0]{{\ddot{\mu}}}
\newcommand{\nuddot}[0]{{\ddot{\nu}}}
\newcommand{\omegaddot}[0]{{\ddot{\omega}}}
\newcommand{\phiddot}[0]{{\ddot{\phi}}}
\newcommand{\piddot}[0]{{\ddot{\pi}}}
\newcommand{\psiddot}[0]{{\ddot{\psi}}}
\newcommand{\rhoddot}[0]{{\ddot{\rho}}}
\newcommand{\sigmaddot}[0]{{\ddot{\sigma}}}
\newcommand{\tauddot}[0]{{\ddot{\tau}}}
\newcommand{\thetaddot}[0]{{\ddot{\theta}}}
\newcommand{\upsilonddot}[0]{{\ddot{\upsilon}}}
\newcommand{\varepsilonddot}[0]{{\ddot{\varepsilon}}}
\newcommand{\varphiddot}[0]{{\ddot{\varphi}}}
\newcommand{\varpiddot}[0]{{\ddot{\varpi}}}
\newcommand{\varrhoddot}[0]{{\ddot{\varrho}}}
\newcommand{\varsigmaddot}[0]{{\ddot{\varsigma}}}
\newcommand{\varthetaddot}[0]{{\ddot{\vartheta}}}
\newcommand{\xiddot}[0]{{\ddot{\xi}}}
\newcommand{\zetaddot}[0]{{\ddot{\zeta}}}

\newcommand{\BDelta}[0]{\boldsymbol{\Delta}}
\newcommand{\BGamma}[0]{\boldsymbol{\Gamma}}
\newcommand{\BLambda}[0]{\boldsymbol{\Lambda}}
\newcommand{\BOmega}[0]{\boldsymbol{\Omega}}
\newcommand{\BPhi}[0]{\boldsymbol{\Phi}}
\newcommand{\BPi}[0]{\boldsymbol{\Pi}}
\newcommand{\BPsi}[0]{\boldsymbol{\Psi}}
\newcommand{\BSigma}[0]{\boldsymbol{\Sigma}}
\newcommand{\BTheta}[0]{\boldsymbol{\Theta}}
\newcommand{\BUpsilon}[0]{\boldsymbol{\Upsilon}}
\newcommand{\BXi}[0]{\boldsymbol{\Xi}}
\newcommand{\Balpha}[0]{\boldsymbol{\alpha}}
\newcommand{\Bbeta}[0]{\boldsymbol{\beta}}
\newcommand{\Bchi}[0]{\boldsymbol{\chi}}
\newcommand{\Bdelta}[0]{\boldsymbol{\delta}}
\newcommand{\Bepsilon}[0]{\boldsymbol{\epsilon}}
\newcommand{\Beta}[0]{\boldsymbol{\eta}}
\newcommand{\Bgamma}[0]{\boldsymbol{\gamma}}
\newcommand{\Bkappa}[0]{\boldsymbol{\kappa}}
\newcommand{\Blambda}[0]{\boldsymbol{\lambda}}
\newcommand{\Bmu}[0]{\boldsymbol{\mu}}
\newcommand{\Bnu}[0]{\boldsymbol{\nu}}
%\newcommand{\Bomega}[0]{\boldsymbol{\omega}}
\newcommand{\Bphi}[0]{\boldsymbol{\phi}}
\newcommand{\Bpi}[0]{\boldsymbol{\pi}}
\newcommand{\Bpsi}[0]{\boldsymbol{\psi}}
\newcommand{\Brho}[0]{\boldsymbol{\rho}}
\newcommand{\Bsigma}[0]{\boldsymbol{\sigma}}
%\newcommand{\Btau}[0]{\boldsymbol{\tau}}
%\newcommand{\Btheta}[0]{\boldsymbol{\theta}}
\newcommand{\Bupsilon}[0]{\boldsymbol{\upsilon}}
\newcommand{\Bvarepsilon}[0]{\boldsymbol{\varepsilon}}
\newcommand{\Bvarphi}[0]{\boldsymbol{\varphi}}
\newcommand{\Bvarpi}[0]{\boldsymbol{\varpi}}
\newcommand{\Bvarrho}[0]{\boldsymbol{\varrho}}
\newcommand{\Bvarsigma}[0]{\boldsymbol{\varsigma}}
\newcommand{\Bvartheta}[0]{\boldsymbol{\vartheta}}
\newcommand{\Bxi}[0]{\boldsymbol{\xi}}
\newcommand{\Bzeta}[0]{\boldsymbol{\zeta}}
%
%</bold and dot greek symbols>
%<infrequent>
%
%\newcommand{\AreaOp}[1]{\AName_{#1}}
%\newcommand{\Babs}[0]{\abs{\BB}}
%\newcommand{\Bcap}[0]{\hat{\BB}}
%\newcommand{\BrPrimeRej}[0]{\rcap(\rcap \wedge \Br')}
%\newcommand{\CA}[0]{\mathcal{A}}
%\newcommand{\Cos}[1]{\cos{\left({#1}\right)}}
%\newcommand{\Det}[1] {\abs{#1}}
%\newcommand{\Dsq}[2] {\frac {\partial^2 {#1}} {\partial {#2}^2}}
%\newcommand{\Exp}[1]{\exp{\left({#1}\right)}}
%\newcommand{\Norm}[1]{\left\lVert{#1}\right\rVert}
%\newcommand{\Sin}[1]{\sin{\left({#1}\right)}}
%\newcommand{\T}[0]{\text{T}}
%\newcommand{\VolumeOp}[1]{\VName_{#1}}
%\newcommand{\agrad}[0]{\Ba \cdot \nabla}
%\newcommand{\alphacap}[0]{\hat{\boldsymbol{\alpha}}}
%\newcommand{\Fcap}[0]{\hat{\BF}}
%\newcommand{\bithree}[0]{{\Bi}_3}
%\newcommand{\bxa}[0]{\Bx\Ba}
%\newcommand{\coordvec}[2]{
%\newcommand{\costheta}[0]{\acap \cdot \xcap}
%\newcommand{\ddt}[1]{\ddot{#1}}
%\newcommand{\ddu}[1] {\frac {d{#1}} {du}}
%\newcommand{\dsqxj}[2] {\frac {\partial^2 {#1}} {\partial {x_{#2}}^2}}
%\newcommand{\dtheta}[1]{\frac{d {#1}}{d \theta}}
%\newcommand{\dt}[1]{\dot{#1}}
%\newcommand{\dt}[1]{\frac{d {#1}}{dt}}
%\newcommand{\dxj}[2] {\frac {\partial {#1}} {\partial {x_{#2}}}}
%\newcommand{\halfPhi}[0]{\frac{\phi}{2}}
%\newcommand{\half}[0]{\inv{2}}
%\newcommand{\inv}[1]{\frac{1}{#1}}
%\newcommand{\laplacian}[0]{\nabla^2}
%\newcommand{\matrixoftx}[3]{
%\newcommand{\nrrp}[0]{\norm{\rcap \wedge \Br'}}
%\newcommand{\oiint}{\bigcirc \hspace{-1.4em} \int \hspace{-.8em} \int}
%\newcommand{\transpose}[1]{{#1}^{\text{T}}}
%\newcommand{\transpose}[1]{{{#1}^{\TextTranspose}}}
%\newcommand{\transpose}[1]{{{#1}^{\text{T}}}}
%\newcommand{\barA}[0]{\bar{A}}
%\newcommand{\qbar}[0]{\bar{q}}
%\newcommand{\qdotbar}[0]{\dot{\bar{q}}}
%
%</infrequent>





%\usepackage[bookmarks=true]{hyperref}

%\usepackage{color,cite,graphicx}
   % use colour in the document, put your citations as [1-4]
   % rather than [1,2,3,4] (it looks nicer, and the extended LaTeX2e
   % graphics package. 
%\usepackage{latexsym,amssymb,epsf} % don't remember if these are
   % needed, but their inclusion can't do any damage


\chapter{Bohm Chapter 9 problems. }
%\author{Peeter Joot \quad peeter.joot@gmail.com }
%\date{ March 6, 2009.  Last Revision: $Date: 2009/06/03 22:13:06 $ }

%\begin{document}

%\maketitle{}

%\tableofcontents

\section{Bohm Chapter 9 problems. }

Problems and additional details from reading of \cite{bohm1989qt}, chapter 9.

\subsection{P1. Momentum wave function normalization. }

Given a normalized wave function

\begin{align*}
\IIinf \psi^\conj(x) \psi(x) dx = 1
\end{align*}

Show that the wave function $\phi(k)$ is also normalized, and find the normalization factor for $\Phi(p)$.

\begin{align*}
\IIinf \phi^\conj(k) \phi(k) dk 
&= 
\IIinf \phi^\conj(k) \left( \inv{\sqrt{2\pi}} \IIinf \psi(x) e^{-i k x} dx \right) dk  \\
&= 
\IIinf \left( \inv{\sqrt{2\pi}} \IIinf \phi^\conj(k) e^{-i k x} dk \right) \psi(x) dx  \\
&= 
\IIinf {\left( \inv{\sqrt{2\pi}} \IIinf \phi(k) e^{i k x} dk \right)}^\conj \psi(x) dx  \\
&= 
\IIinf \psi^\conj(x) \psi(x) dx  \\
&= 1 \quad\quad\quad \square
\end{align*}

Bohm defines $\Phi(p) \propto \phi(k)$ with the normalization constant determined by $\int \Abs{\Phi(p)} dp = 1$.  Suppose we 
let $\Phi(p) = \alpha \phi(k)$, then we have

\begin{align*}
1 
&= \int \Phi^\conj(p) \Phi(p) dp \\
&= \int \alpha^2 \phi^\conj(k) \phi(k) \hbar d k
\end{align*}

So we want $\alpha^2 \hbar = 1$, and therefore $\Phi(p) = \inv{\sqrt{\hbar}} \phi(k)$.

In \cite{mcmahon2005qmd}, with followup in \cite{PJqmFourier} we've seen that an alternate Fourier transform pair can be used in terms of
momentum variables.  That is

\begin{align*}
\Phi(p) &= \FM \IIinf \psi(x) e^{-ipx/\hbar} dx \\
\psi(x) &= \FM \IIinf \Phi(p) e^{ipx/\hbar} dp \\
\end{align*}

Observe that this is consistent with Bohm's notation, since one can read off 
$\Phi(p)$ in terms of $\phi(k)$.
by inspection

\begin{align*}
\Phi(p) &= \FM \IIinf \psi(x) e^{-ipx/\hbar} dx = \inv{\sqrt{\hbar}} \phi(k)
\end{align*}

\subsection{P2. Expectation of polynomial momentum function. }

Given a function of momentum 

\begin{align*}
f(p) &= \sum C_n p^n
\end{align*}

Express the average, or expectation value of $f(p)$.  It is sufficient to consider one of the monomial terms, say $p^n$.  A translation 
to position basis via Fourier transformation produces the desired result

\begin{align*}
\expectation{p^n} 
&= \int \Phi^\conj(p) p^n \Phi(p) dp \\
&= \inv{2\pi \hbar} \iiint \left( \psi^\conj(x') e^{ipx'/\hbar} dx' \right) (\hbar k)^n \left( \psi(x) e^{-ipx/\hbar} dx \right) (\hbar dk) \\
&= \frac{\hbar^n}{2\pi } \iiint \psi^\conj(x') e^{ikx'} dx' k^n e^{-i k x} \psi(x) dx dk \\
\end{align*}

The $k^n$ can be reduced to differential form as Bohm did for the $\expectation{p}$ case

\begin{align*}
k^n e^{-i k x} 
&= k^{n-1} k e^{-i k x} \\
&= k^{n-1} i \PD{x}{}e^{-i k x} \\
&= k^{n-m} i^m \PDN{x}{}{m}e^{-i k x} \\
&= i^n \PDN{x}{}{n}e^{-i k x} \\
\end{align*}

This leaves something that's in shape for integration by parts

\begin{align*}
\expectation{p^n} 
&= \frac{(i\hbar)^n}{2\pi} \iiint \psi^\conj(x') e^{ikx'} dx' \left( \PDN{x}{}{n}e^{-i k x} \right) \psi(x) dx dk \\
&= \frac{(-i\hbar)^n}{2\pi} \iiint \psi^\conj(x') e^{ikx'} dx' \PDN{x}{\psi(x)}{n} e^{-i k x} dx dk \\
&= \frac{(-i\hbar)^n}{2\pi} \iiint \psi^\conj(x') e^{ik(x'-x)} \PDN{x}{\psi(x)}{n} dx' dx dk \\
&= {(-i\hbar)^n}{} \iint \psi^\conj(x') \PDN{x}{\psi(x)}{n} dx' dx \inv{2\pi}\int e^{ik(x'-x)} dk \\
\end{align*}

This last integral is really a distribution, and can be identified with the delta function $\delta(x'-x)$ operating on, in this case, the preceding integral.
%, and operates on a test function.  Suppose we designate such a test function as $a(u)$, then we have
%
%\int \inv{2\pi}\int e^{iku} dk a(u) du
%&= \int \inv{2\pi}\int e^{iku} a(u) du dk \\

%Now, we can apply the distribution theory as covered in \cite{osgoodFourier} to do a delta function reduction of the exponentials in 
%this integral.  Specifically pick a function $a(k)$

So we have
\begin{align*}
\expectation{p^n} 
&= {(-i\hbar)^n}{} \iint \psi^\conj(x') \PDN{x}{\psi(x)}{n} dx' dx \delta(x'-x) \\
&= {(-i\hbar)^n}{} \int \psi^\conj(x) \PDN{x}{\psi(x)}{n} dx \\
\end{align*}

We can put this into explicit operator form, nicely motivating the identification of $-i\hbar \PDi{x}{}$ with the momentum by virtue 
of the definition of the average or expectation value.

\begin{align*}
\expectation{p^n} 
&= \int \psi^\conj(x) {\left( -i \hbar \PD{x}{} \right)}^n {\psi(x)} dx \\
\end{align*}

\subsection{P3.  Expectation of position in momentum space. }

\begin{align*}
\expectation{x} 
&= \int \psi^\conj(x) x \psi(x) dx \\
&= 
\inv{2\pi\hbar} \iiint \Phi^\conj(p) e^{-ipx/\hbar} dp x \Phi(p') e^{ip'x/\hbar} dp' dx \\
&= 
\inv{2\pi\hbar} \iiint \Phi^\conj(p) e^{-ipx/\hbar} dp \left( -i \PD{p'}{} e^{ip'x/\hbar} \right) \Phi(p') dp' dx \\
&= 
\inv{2\pi\hbar} \iiint \Phi^\conj(p) e^{-ipx/\hbar} dp \left( i \PD{p'}{\Phi(p')} \right) e^{ip'x/\hbar} dp' dx \\
&= 
\iint \Phi^\conj(p) \left( i \PD{p'}{} \right) {\Phi(p')} dp dp' \inv{2\pi\hbar} \int e^{i(p'-p)x/\hbar} dx  \\
&= 
\iint \Phi^\conj(p) \left( i \PD{p'}{} \right) {\Phi(p')} dp dp' \delta(p'-p)  \\
\end{align*}

This is

\begin{align*}
\expectation{x} &= \int \Phi^\conj(p) \left( i \PD{p}{} \right) {\Phi(p)} dp 
\end{align*}

We see that expressing momentum in position space and position in momentum space both result in differential
operator forms in calculations of expected values

\begin{align}\label{eqn:bohm_ch9:operatorCorrespondance}
p &\sim -i \hbar \PD{x}{} \\
x &\sim i \hbar \PD{p}{}
\end{align}

Observe the Hamiltonian and Poisson equation structure in these two sets of operators.

\subsection{P4. Expectation of polynomial position function. }

This problem follows just as P2, and I'm not going to bother typing it up for myself.  For validity, we require
$x^n \phi(x) \rightarrow 0$ as $x \rightarrow \pm \infty$, or equivalently that $\PDN{p}{\Phi}{n} \rightarrow 0$.

\subsection{P5. Some commutator calculations. }

\subsubsection{P5. Position momentum moment commutators. }

Evaluate

\begin{align*}
f(x,p) = x^n p^m - p^m x^n
\end{align*}

Up to this point we've only seen operators in expectation values.  Let's look the simplest case with $n = m = 1$ in that
context

\begin{align*}
\expectation{f} 
&= \frac{\hbar}{i} \int \psi^\conj(x) \left(x \PD{x}{} - \PD{x}{} x \right) \psi(x) dx \\
&= \frac{\hbar}{i} \int \psi^\conj(x) \left(x \PD{x}{\psi(x)} - \psi(x) - x \PD{x}{\psi(x)} \right) dx \\
&= -\frac{\hbar}{i} \int \psi^\conj(x) \psi(x) dx \\
&= {i\hbar}
\end{align*}

So in the same way that the operator correspondence between momentum and the derivative as summarized in 
\ref{eqn:bohm_ch9:operatorCorrespondance}, one can associate the commutator operator with its action in the expectation value and
say

\begin{align}\label{eqn:bohm_ch9:commutator}
x p - p x \sim  i\hbar
\end{align}

The higher order commutator expansions could also be evaluated this way, but exploiting the operator nature directly 
makes this easier.  For the first order moment commutator above one can write

\begin{align*}
f(x,p) \psi(x) 
&= (x p - p x) \psi(x) \\
&= -i \hbar \left(x \PD{x}{} - \PD{x}{} x\right) \psi(x) \\
&= -i \hbar \left(x \PD{x}{\psi}(x) - \PD{x}{x \psi(x)} \right) \\
&= -i \hbar \left(x \PD{x}{\psi}(x) - \PD{x}{\psi(x)} -\psi(x) \right) \\
&= i \hbar \psi(x) \\
\end{align*}

So again we see that as a right acting operator the net effect on any wave function is the following action

\begin{align*}
(x p - p x) \psi = i \hbar \psi \\
\end{align*}

If one starts from this point and then calculates the expectation value the result will still be $i \hbar$, but working
with the probability integrals from the get go is just additional complication.

Building on this result we can then calculate the higher order moment differences of the problem by using the commutator
to change the order of operations

\begin{align*}
p x \sim -i \hbar + x p
\end{align*}

Let's use this for a couple simple examples to start
\begin{align*}
x^2 p - p x^2
&=
x^2 p - ( -i \hbar + x p) x \\
&=
x^2 p + i \hbar x - x ( -i \hbar + x p) \\
&=
x^2 p + 2 i \hbar x - x^2 p \\
&=
2 i \hbar x \\
\end{align*}

\begin{align*}
x p^2 - p^2 x
&=
x p^2 - p ( -i \hbar + x p) \\
&=
x p^2 + i \hbar p - p x p \\
&=
x p^2 + i \hbar p + ( +i \hbar - x p) p \\
&=
2 i \hbar p \\
\end{align*}

Calculation of third powers shows a pattern, and one can guess at an induction hypothesis

\begin{align*}
x p^n &= p^n x + n i \hbar p^{n-1} \\
-p x^n &= -x^n p + n i \hbar x^{n-1} \\
\end{align*}

The $n=1$ cases follow from $xp - px = i\hbar$, leaving only the induction on $n$.  For the momentum powers we have

\begin{align*}
x p^n p 
&= p^n x p + n i \hbar p^{n} \\
&= p^n (p x + i \hbar) + n i \hbar p^{n} \\
&= p^{n+1} x + (n+1) i \hbar p^{n} \quad\quad\quad\square \\
\end{align*}

For the position powers we have
\begin{align*}
-p x^n x 
&= -x^n p x + n i \hbar x^{n} \\
&= x^n (-x p + i \hbar) + n i \hbar x^{n} \\
&= -x^{n+1} p + (n+1) i \hbar x^{n} \quad\quad\quad\square \\
\end{align*}

This completes the proof for a first order version of the problem

\begin{align*}
x p^n - p^n x &= n i \hbar p^{n-1} \\
x^n p -p x^n &=  n i \hbar x^{n-1} \\
\end{align*}

Observe that working with the operator form changes the calculation of derivatives problem in the original
commutator evaluation to nothing more than an algebraic exercise.
% (but one where there's a requirement to not accidentally invert the product order).

The general case still remains.  Building up to that let's do a couple examples

%x p^n = 
%p^n x +  n i \hbar p^{n-1} \\
%
%x^n p = 
%p x^n  +  n i \hbar x^{n-1} \\
%
%x^{n-1} p = 
%(p x^{n-1}  +  (n-1) i \hbar x^{n-2})
%x^{n-2} p = 
%(p x^{n-2}  +  (n-2) i \hbar x^{n-3})

\begin{align*}
x^n p^2
&= (x^n p) p \\
&= (p x^n  +  n i \hbar x^{n-1} ) p \\
&= p (x^n p) +  n i \hbar (x^{n-1} p ) \\
&= p^2 x^n  +  2 n i \hbar p x^{n-1} +  n (n-1) (i \hbar)^2 x^{n-2} \\
\end{align*}

\begin{align*}
x^n p^3
&=
(x^n p^2) p \\
&=
(p^2 x^n  +  2 n i \hbar p x^{n-1} +  n (n-1) (i \hbar)^2 x^{n-2} ) p \\
&=
  p^2 ( p x^n  +  n i \hbar x^{n-1} ) 
+ 2 n i \hbar p ( p x^{n-1}  +  (n-1) i \hbar x^{n-2} ) 
+ n (n-1) (i \hbar)^2 ( p x^{n-2}  +  (n-2) i \hbar x^{n-3}) 
\\
&=
  p^3 x^n
+ 3 n (i \hbar) p^2 x^{n-1}  
+ 3 n(n-1) (i \hbar)^2 p x^{n-2} 
+ n (n-1)(n-2) (i \hbar)^3 x^{n-3}
\\
\end{align*}

We see what looks like binomial coefficients, so a reasonable inductive hypothesis, for $m \le n$

\begin{align}\label{eqn:bohm_ch9:commutatorMomentMlessThanN}
x^n p^m
&= \sum_{j=0}^m \binom{m}{j} (i \hbar)^j p^{m-j} x^{n-j} (n)(n-1)\cdots(n-j+1)
\end{align}

And in particular, for $m \le n$
\begin{align}
x^n p^m - p^m x^n
&= \sum_{j=1}^m \binom{m}{j} (i \hbar)^j p^{m-j} x^{n-j} (n)(n-1)\cdots(n-j+1) 
\end{align}

For $m \ge n$, let's start with

\begin{align*}
p^m x = x p^m -  m i \hbar p^{m-1} 
\end{align*}
%p^m x = x p^m -  m (i \hbar) p^{m-1} 
%p^{m-1} x = x p^{m-1} -  (m-1) (i \hbar) p^{m-2} 
%p^{m-2} x = x p^{m-2} -  (m-2) (i \hbar) p^{m-3} 

First do the $x^2$
\begin{align*}
p^m x^2 
&= x p^m x -  m (i \hbar) p^{m-1} x \\
&= x (p^m x) -  m (i \hbar) (p^{m-1} x) \\
&= x^2 p^m - 2 m (i \hbar) x p^{m-1} +  m (m-1)(i \hbar)^2 p^{m-2} \\
\end{align*}

And for the cube $x^3$
\begin{align*}
p^m x^3  
&= 
( p^m x^2 ) x \\
&= 
( x^2 p^m - 2 m (i \hbar) x p^{m-1} +  m (m-1)(i \hbar)^2 p^{m-2} ) x \\
&= 
x^2 (p^m x )
- 2 m (i \hbar) x (p^{m-1} x )
+ m (m-1)(i \hbar)^2 (p^{m-2} x) \\
&= 
x^2 ( x p^m -  m (i \hbar) p^{m-1} ) \\
&\quad- 2 m (i \hbar) x ( x p^{m-1} -  (m-1) (i \hbar) p^{m-2} ) \\
&\quad+ m (m-1)(i \hbar)^2 ( x p^{m-2} -  (m-2) (i \hbar) p^{m-3} ) \\
&= 
x^3 p^m 
- 3 m (i \hbar) x^2 p^{m-1} 
+ 3 m (m-1) (i \hbar)^2 x p^{m-2} 
- m (m-1)(i \hbar)^2 (m-2) (i \hbar) p^{m-3} \\
\end{align*}

It appears that in this case with $m \ge n$, like \ref{eqn:bohm_ch9:commutatorMomentMlessThanN}, we want as the induction statement 

\begin{align}
p^m x^n &= \sum_{j=0}^n \binom{n}{j} (-i \hbar)^j x^{n-j} p^{m-j} (m)(m-1)\cdots(m-j+1) 
\end{align}

And for the commutator moment the expected result, pending induction on the above, is
\begin{align}
x^n p^m - p^m x^n &= -\sum_{j=1}^n \binom{n}{j} (-i \hbar)^j x^{n-j} p^{m-j} (m)(m-1)\cdots(m-j+1) 
\end{align}

Summarizing, this is
\begin{align*}
&x^n p^m - p^m x^n
= \\
&\left\{
\begin{array}{l l}
\sum_{j=1}^m \binom{m}{j} (i \hbar)^j p^{m-j} x^{n-j} (n)(n-1)\cdots(n-j+1) & \quad \mbox{if $m \le n$} \\
-\sum_{j=1}^n \binom{n}{j} (-i \hbar)^j x^{n-j} p^{m-j} (m)(m-1)\cdots(m-j+1) & \quad \mbox{if $m \ge n$}
\end{array}
\right.
\end{align*}

\subsubsection{P5.b }

\begin{align*}
e^{i k x} p - 
p e^{ i k x}
\end{align*}

Reversing the second term via power series expansion we have

\begin{align*}
p e^{ i k x}
&=
p \sum_{n=0}^\infty \frac{( i k x )^n}{n!} \\
&=
\sum_{n=0}^\infty \frac{( i k )^n}{n!} (x^n p - n i \hbar x^{n-1} )
\\
&=
e^{ i k x} p
-\sum_{n=1}^\infty \frac{( i k )^n}{n!} (n i \hbar x^{n-1} )
\\
&=
e^{ i k x} p
-(i k)(i\hbar) \sum_{n=1}^\infty \frac{( i k x)^{n-1}}{(n-1)!} 
\\
&=
e^{ i k x} p
+ (k\hbar) e^{i k x}
\\
\end{align*}

So we have

\begin{align*}
e^{ i k x} p - p e^{ i k x} &= - (k\hbar) e^{i k x}
\end{align*}

\subsection{P6. Hermitian operators. Powers of momentum operators. }

Show that $p^n$ is Hermitian

\begin{align*}
\expectation{p^n}^\conj 
&=
\left( \int \psi^\conj (-i \hbar)^n \frac{d^n}{dx^n} \psi \right)^\conj \\
&=
\int \psi (i \hbar)^n \frac{d^n}{dx^n} \psi^\conj \\
&=
\int \left( (-1)^n \frac{d^n}{dx^n} \psi \right) (i \hbar)^n \psi^\conj \\
&=
\int \left( p^n \psi \right) \psi^\conj \\
&=
\expectation{p^n}
\end{align*}

Thus, by the definition of equation (13) in the text, this operator is
Hermitian.

Next is to show that 

\begin{align*}
f(p) = \sum_k A_k p^k
\end{align*}

is Hermitian, provided $A_k$ are all real.  This part is clear by inspection.

\subsection{P7. Hermitian operators. Powers of position operators. }

Want to show that the following is Hermitian

\begin{align*}
f(x) &= \sum A_k x^k
\end{align*}

If the conjugate of the expectation equals itself we required only $A_k = A_k^\conj$, so $A_k$ must be strictly real, and we are done.

\subsection{P8. Non Hermitian momentum power operators if derivative doesn't vanish. }

Show that if $\partial^n \psi/\partial x^n$ doesn't vanish then $(-i\hbar)^{n+1} \partial^{n+1}/\partial x^{n+1}$ is not Hermitian.

We want to evaluate the following and compare it to its conjugate

\begin{align*}
\expectation{p^{n+1}} 
&= (-i\hbar)^{n+1} \IIinf \psi^\conj \PDN{x}{\psi}{n+1} dx \\
&= 
(-i\hbar)^{n+1} \left. \psi^\conj \PDN{x}{\psi}{n} \right\vert_{-\infty}^{\infty}
+(-1)^{1}(-i\hbar)^{n+1} \IIinf \PDN{x}{\psi^\conj}{1} \PDN{x}{\psi}{n} dx 
\\
&= 
(-i\hbar)^{n+1} \left. \psi^\conj \PDN{x}{\psi}{n} \right\vert_{-\infty}^{\infty}
+(-1)^{1}(-i\hbar)^{n+1} \left. \PDN{x}{\psi^\conj}{1} \PDN{x}{\psi}{n-1} \right\vert_{-\infty}^{\infty}
+(-1)^{2}(-i\hbar)^{n+1} \IIinf \PDN{x}{\psi^\conj}{2} \PDN{x}{\psi}{n-1} dx 
\\
&= 
(-i\hbar)^{n+1} \sum_{k=0}^1
(-1)^{k}
\left. \PDN{x}{\psi^\conj}{k} \PDN{x}{\psi}{n-k} \right\vert_{-\infty}^{\infty}
+(-1)^{2}(-i\hbar)^{n+1} \IIinf \PDN{x}{\psi^\conj}{2} \PDN{x}{\psi}{n-1} dx  
\\
&= 
(-i\hbar)^{n+1} \sum_{k=0}^m
(-1)^{k}
\left. \PDN{x}{\psi^\conj}{k} \PDN{x}{\psi}{n-k} \right\vert_{-\infty}^{\infty}
+(-1)^{m+1}(-i\hbar)^{n+1} \IIinf \PDN{x}{\psi^\conj}{m+1} \PDN{x}{\psi}{n-m)} dx 
\\
&= 
(-i\hbar)^{n+1} \sum_{k=0}^n
(-1)^{k}
\left. \PDN{x}{\psi^\conj}{k} \PDN{x}{\psi}{n-k} \right\vert_{-\infty}^{\infty}
+(i\hbar)^{n+1} \IIinf \PDN{x}{\psi^\conj}{n+1} \psi dx 
\\
&= 
(-i\hbar)^{n+1} \sum_{k=0}^n
(-1)^{k}
\left. \PDN{x}{\psi^\conj}{k} \PDN{x}{\psi}{n-k} \right\vert_{-\infty}^{\infty}
+
\expectation{p^{n+1}}^\conj
\\
\end{align*}
%\newcommand{\PDN}[3]{\frac{\partial^{#3} {#2}}{\partial {#1}^{#3}}}

So we have
\begin{align*}
\expectation{p^{n+1}} - \expectation{p^{n+1}}^\conj
&= 
(-i\hbar)^{n+1} \sum_{k=0}^n
(-1)^{k}
\left. \PDN{x}{\psi^\conj}{k} \PDN{x}{\psi}{n-k} \right\vert_{-\infty}^{\infty}
\\
\end{align*}

If $p^{n+1}$ is Hermitian, then this difference should be zero, but if the indicated partial doesn't vanish this
remainder bit can be non-zero.

\subsection{P9. m, n'th moment is Hermitian. }

Consider the operator 

\begin{align*}
\sum A_{nm} \left(\frac{p^n x^m + x^m p^n}{2}\right) 
\end{align*}

Show that this is Hermitian if all $A_{nm}$ are real.

Consider first one specific term with $A_{nm}$, calculate the conjugate of the expectation value, and integrate
by parts

\begin{align*}
\left(
\int \psi^\conj \inv{2}(p^n x^m + x^m p^n) \psi
\right)^\conj
&=
\inv{2} (-1)^n
\int \psi (p^n x^m + x^m p^n) \psi^\conj \\
&=
\inv{2} (i\hbar)^n 
\int \psi \left(\frac{d^n (x^m \psi^\conj)}{dx^n} + x^m \frac{d^n \psi^\conj}{dx^n} \right) \\
&=
\inv{2} (-i\hbar)^n 
\int x^m \psi^\conj \left(\frac{d^n \psi }{dx^n} + \psi^\conj \frac{d^n (x^m \psi)}{dx^n} \right) \\
&=
\inv{2}
\int \psi^\conj ( x^m p^n + p^n x^m ) \psi
\end{align*}

This shows that $(p^n x^m + x^m p^n)/2$ is Hermitian, and the conjugation requires $A_{nm}$ to be real for the
product of the two to be Hermitian.

\subsection{P10. Hermitizing Classical operator $(px)^2$. }

Show that 
\begin{align*}
\inv{2}\left(x^2 p^2 + p^2 x^2 \right)
\end{align*}

and 
\begin{align*}
\inv{4}\left(x p + p x \right)^2
\end{align*}

lead to results that differ by a factor of $\hbar^2$.

To do so consider the difference of the expectation of this operator, first calculating this difference.  We will
want to use the commutator relation, in a few equivalent forms

\begin{align*}
xp - p x &= i \hbar \\
xp &= p x + i \hbar \\
px &= x p - i \hbar \\
x p + p x &= 2 p x + i \hbar \\
          &= 2 x p - i \hbar \\
\end{align*}

This gives us

\begin{align*}
\inv{4}\left(x p + p x \right)^2 
&= \inv{4}(2 p x + i \hbar)( 2 x p - i \hbar ) \\
&= p x^2 p + i\hbar \inv{2}(x p - p x) + \inv{4} \hbar^2 \\
&= p x^2 p - \hbar^2 \inv{2} + \inv{4} \hbar^2 \\
&= p x^2 p - \hbar^2 \inv{4} \\
\end{align*}

For the other operator, reduction to a form that also contains $p x^2 p$, we have

\begin{align*}
\inv{2}\left(x^2 p^2 + p^2 x^2 \right)
&=
\inv{2}\left(x (x p) p + p (p x) x \right) \\
&=
\inv{2}\left(x (p x + i \hbar) p + p (x p -i\hbar) x \right) \\
&=
\inv{2}\left((x p) x p + p x (p x) + i \hbar ( x p - p x ) \right) \\
&=
\inv{2}\left((p x + i\hbar) x p + p x ( x p - i\hbar) + (i \hbar)^2 \right) \\
&=
\inv{2}\left( 2 p x^2 p + i \hbar ( x p - p x) + -\hbar^2 \right) \\
&=
p x^2 p + -\hbar^2 
\end{align*}

So now, if we take the difference 
\begin{align*}
\inv{2}\left(x^2 p^2 + p^2 x^2 \right) - \inv{4}\left(x p + p x \right)^2 
&= (p x^2 p + -\hbar^2 ) - (p x^2 p - \hbar^2 \inv{4}) 
 \\
&= -\frac{3}{4}\hbar^2 
 \\
\end{align*}

The difference of the expectation values of these operators is thus of the order $\hbar^2$ as was to be calculated.

\subsection{P11.  An explicit calculation of a Hermitian operator. }

Show by integration by parts that $(xp)^\dagger = p x$.

The defining relation for the Hermitian conjugation operation is equation 16 in the text.

\begin{align*}
\int \psi^\conj (O^\dagger \psi) dx &= \int \psi (O^\conj \psi^\conj) dx
\end{align*}

For the operator $xp$, we have

\begin{align*}
\int \psi^\conj (xp)^\dagger \psi dx 
&= \int \psi (xp)^\conj \psi^\conj dx \\
&= (-i\hbar)^\conj \int \psi x \PD{x}{\psi^\conj} dx \\
&= - i\hbar \int \PD{x}{x \psi} \psi^\conj dx \\
&= \int \psi^\conj (p x) \psi dx \\
\end{align*}

So we have, as desired

\begin{align*}
(xp)^\dagger = p x
\end{align*}

\subsection{P12. Hermitian operator from antisymmetric difference. }

Show that $H = i (O - O^\dagger)$ is a Hermitian operator.

This follows directly from the definition, calculating the expectation

\begin{align*}
\expectation{H} 
&=\int \psi^\conj (i (O - O^\dagger) \psi ) \\
&=
i \int \psi^\conj (O \psi )
-i \int \psi^\conj (O^\dagger \psi ) \\
&=
i \int \psi^\conj (O \psi )
-i \int \psi (O^\conj \psi^\conj) \\
\end{align*}

Taking conjugates we have

\begin{align*}
\expectation{H}^\conj
&=
-i \int \psi (O^\conj \psi^\conj )
+i \int \psi^\conj (O \psi ) \\
&=
\expectation{H} \\
\quad\quad\quad \square
\end{align*}


\subsection{P13. When product of operators is Hermitian. }

What relation must exist between Hermitian $B$ and $A$ must exist for $AB$ to be Hermitian.

TODO:
Am guessing that this has something to do with the commutator of the operators.  This one I don't have a check mark
besides in my text, so did I ever figure it out?

\subsection{P14. Show directly that $i(p^2 x - xp^2)$ is Hermitian }

This follows from $(AB)^\dagger = BA$ in the text above.  We have

\begin{align*}
(i (p^2 x - x p^2))^\dagger 
&=
(x^\dagger p^\dagger p^\dagger - p^\dagger p^\dagger x^\dagger)(-i) \\
&=
-i (x p^2 - p^2 x) \\
&=
i (p^2 x - x p^2 ) \quad\quad\quad \square
\end{align*}

%\bibliographystyle{plainnat}
%\bibliography{myrefs}

%\end{document}
             % mar 6/09
%
% Copyright � 2012 Peeter Joot.  All Rights Reserved.
% Licenced as described in the file LICENSE under the root directory of this GIT repository.
%

%
%
%\documentclass{article}

%\usepackage{amsmath}
\usepackage{mathpazo}

%
% shorthand for bold symbols, convenient for vectors and matrices
%
\newcommand{\Ba}[0]{\mathbf{a}}
\newcommand{\Bb}[0]{\mathbf{b}}
\newcommand{\Bc}[0]{\mathbf{c}}
\newcommand{\Bd}[0]{\mathbf{d}}
\newcommand{\Be}[0]{\mathbf{e}}
\newcommand{\Bf}[0]{\mathbf{f}}
\newcommand{\Bg}[0]{\mathbf{g}}
\newcommand{\Bh}[0]{\mathbf{h}}
\newcommand{\Bi}[0]{\mathbf{i}}
\newcommand{\Bj}[0]{\mathbf{j}}
\newcommand{\Bk}[0]{\mathbf{k}}
\newcommand{\Bl}[0]{\mathbf{l}}
\newcommand{\Bm}[0]{\mathbf{m}}
\newcommand{\Bn}[0]{\mathbf{n}}
\newcommand{\Bo}[0]{\mathbf{o}}
\newcommand{\Bp}[0]{\mathbf{p}}
\newcommand{\Bq}[0]{\mathbf{q}}
\newcommand{\Br}[0]{\mathbf{r}}
\newcommand{\Bs}[0]{\mathbf{s}}
\newcommand{\Bt}[0]{\mathbf{t}}
\newcommand{\Bu}[0]{\mathbf{u}}
\newcommand{\Bv}[0]{\mathbf{v}}
\newcommand{\Bw}[0]{\mathbf{w}}
\newcommand{\Bx}[0]{\mathbf{x}}
\newcommand{\By}[0]{\mathbf{y}}
\newcommand{\Bz}[0]{\mathbf{z}}
\newcommand{\BA}[0]{\mathbf{A}}
\newcommand{\BB}[0]{\mathbf{B}}
\newcommand{\BC}[0]{\mathbf{C}}
\newcommand{\BD}[0]{\mathbf{D}}
\newcommand{\BE}[0]{\mathbf{E}}
\newcommand{\BF}[0]{\mathbf{F}}
\newcommand{\BG}[0]{\mathbf{G}}
\newcommand{\BH}[0]{\mathbf{H}}
\newcommand{\BI}[0]{\mathbf{I}}
\newcommand{\BJ}[0]{\mathbf{J}}
\newcommand{\BK}[0]{\mathbf{K}}
\newcommand{\BL}[0]{\mathbf{L}}
\newcommand{\BM}[0]{\mathbf{M}}
\newcommand{\BN}[0]{\mathbf{N}}
\newcommand{\BO}[0]{\mathbf{O}}
\newcommand{\BP}[0]{\mathbf{P}}
\newcommand{\BQ}[0]{\mathbf{Q}}
\newcommand{\BR}[0]{\mathbf{R}}
\newcommand{\BS}[0]{\mathbf{S}}
\newcommand{\BT}[0]{\mathbf{T}}
\newcommand{\BU}[0]{\mathbf{U}}
\newcommand{\BV}[0]{\mathbf{V}}
\newcommand{\BW}[0]{\mathbf{W}}
\newcommand{\BX}[0]{\mathbf{X}}
\newcommand{\BY}[0]{\mathbf{Y}}
\newcommand{\BZ}[0]{\mathbf{Z}}

\newcommand{\Bzero}[0]{\mathbf{0}}
\newcommand{\Btheta}[0]{\boldsymbol{\theta}}
\newcommand{\Btau}[0]{\boldsymbol{\tau}}
\newcommand{\Bomega}[0]{\boldsymbol{\omega}}

%
% shorthand for unit vectors
%
\newcommand{\acap}[0]{\hat{\Ba}}
\newcommand{\bcap}[0]{\hat{\Bb}}
\newcommand{\ccap}[0]{\hat{\Bc}}
\newcommand{\dcap}[0]{\hat{\Bd}}
\newcommand{\ecap}[0]{\hat{\Be}}
\newcommand{\fcap}[0]{\hat{\Bf}}
\newcommand{\gcap}[0]{\hat{\Bg}}
\newcommand{\hcap}[0]{\hat{\Bh}}
\newcommand{\icap}[0]{\hat{\Bi}}
\newcommand{\jcap}[0]{\hat{\Bj}}
\newcommand{\kcap}[0]{\hat{\Bk}}
\newcommand{\lcap}[0]{\hat{\Bl}}
\newcommand{\mcap}[0]{\hat{\Bm}}
\newcommand{\ncap}[0]{\hat{\Bn}}
\newcommand{\ocap}[0]{\hat{\Bo}}
\newcommand{\pcap}[0]{\hat{\Bp}}
\newcommand{\qcap}[0]{\hat{\Bq}}
\newcommand{\rcap}[0]{\hat{\Br}}
\newcommand{\scap}[0]{\hat{\Bs}}
\newcommand{\tcap}[0]{\hat{\Bt}}
\newcommand{\ucap}[0]{\hat{\Bu}}
\newcommand{\vcap}[0]{\hat{\Bv}}
\newcommand{\wcap}[0]{\hat{\Bw}}
\newcommand{\xcap}[0]{\hat{\Bx}}
\newcommand{\ycap}[0]{\hat{\By}}
\newcommand{\zcap}[0]{\hat{\Bz}}
\newcommand{\thetacap}[0]{\hat{\Btheta}}

%
% to write R^n and C^n in a distinguishable fashion.  Perhaps change this
% to the double lined characters upon figuring out how to do so.
%
\newcommand{\C}[1]{$\mathbb{C}^{#1}$}
\newcommand{\R}[1]{$\mathbb{R}^{#1}$}

%
% various generally useful helpers
%

% derivative of #1 wrt. #2:
\newcommand{\D}[2] {\frac {d#2} {d#1}}

\newcommand{\inv}[1]{\frac{1}{#1}}
\newcommand{\cross}[0]{\times}

\newcommand{\abs}[1]{\lvert{#1}\rvert}
\newcommand{\norm}[1]{\lVert{#1}\rVert}
\newcommand{\innerprod}[2]{\langle{#1}, {#2}\rangle}
\newcommand{\dotprod}[2]{{#1} \cdot {#2}}
\newcommand{\bdotprod}[2]{\left({#1} \cdot {#2}\right)}
\newcommand{\crossprod}[2]{{#1} \cross {#2}}
\newcommand{\tripleprod}[3]{\dotprod{\left(\crossprod{#1}{#2}\right)}{#3}}

\DeclareMathOperator{\Proj}{Proj}
\DeclareMathOperator{\Span}{span}
\DeclareMathOperator{\Sgn}{sgn}
\DeclareMathOperator{\Area}{Area}
\DeclareMathOperator{\Volume}{Volume}

%
% A few miscellaneous things specific to this document
%
\newcommand{\crossop}[1]{\crossprod{#1}{}}

% R2 vector.
\newcommand{\VectorTwo}[2]{
\begin{bmatrix}
 {#1} \\
 {#2}
\end{bmatrix}
}

\newcommand{\VectorN}[1]{
\begin{bmatrix}
{#1}_1 \\
{#1}_2 \\
\vdots \\
{#1}_N \\
\end{bmatrix}
}

\newcommand{\DETuvij}[4]{
\begin{vmatrix}
 {#1}_{#3} & {#1}_{#4} \\
 {#2}_{#3} & {#2}_{#4}
\end{vmatrix}
}

\newcommand{\DETuvwijk}[6]{
\begin{vmatrix}
 {#1}_{#4} & {#1}_{#5} & {#1}_{#6} \\
 {#2}_{#4} & {#2}_{#5} & {#2}_{#6} \\
 {#3}_{#4} & {#3}_{#5} & {#3}_{#6}
\end{vmatrix}
}

\newcommand{\DETuvwxijkl}[8]{
\begin{vmatrix}
 {#1}_{#5} & {#1}_{#6} & {#1}_{#7} & {#1}_{#8} \\
 {#2}_{#5} & {#2}_{#6} & {#2}_{#7} & {#2}_{#8} \\
 {#3}_{#5} & {#3}_{#6} & {#3}_{#7} & {#3}_{#8} \\
 {#4}_{#5} & {#4}_{#6} & {#4}_{#7} & {#4}_{#8} \\
\end{vmatrix}
}

%\newcommand{\DETuvwxyijklm}[10]{
%\begin{vmatrix}
% {#1}_{#6} & {#1}_{#7} & {#1}_{#8} & {#1}_{#9} & {#1}_{#10} \\
% {#2}_{#6} & {#2}_{#7} & {#2}_{#8} & {#2}_{#9} & {#2}_{#10} \\
% {#3}_{#6} & {#3}_{#7} & {#3}_{#8} & {#3}_{#9} & {#3}_{#10} \\
% {#4}_{#6} & {#4}_{#7} & {#4}_{#8} & {#4}_{#9} & {#4}_{#10} \\
% {#5}_{#6} & {#5}_{#7} & {#5}_{#8} & {#5}_{#9} & {#5}_{#10}
%\end{vmatrix}
%}

% R3 vector.
\newcommand{\VectorThree}[3]{
\begin{bmatrix}
 {#1} \\
 {#2} \\
 {#3}
\end{bmatrix}
}


%%<misc>
%
\newcommand{\Abs}[1]{{\left\lvert{#1}\right\rvert}}
\newcommand{\spacegrad}[0]{\boldsymbol{\nabla}}
\newcommand{\grad}[0]{\nabla}
\newcommand{\LL}[0]{\mathcal{L}}

% == \partial_{#1} {#2}
\newcommand{\PD}[2]{\frac{\partial {#2}}{\partial {#1}}}
% inline variant
\newcommand{\PDi}[2]{{\partial {#2}}/{\partial {#1}}}

\newcommand{\PDD}[3]{\frac{\partial^2 {#3}}{\partial {#1}\partial {#2}}}
%\newcommand{\PDd}[2]{\frac{\partial^2 {#2}}{{\partial{#1}}^2}}
\newcommand{\PDsq}[2]{\frac{\partial^2 {#2}}{(\partial {#1})^2}}

\newcommand{\Partial}[2]{\frac{\partial {#1}}{\partial {#2}}}
\DeclareMathOperator{\RejName}{Rej}
\newcommand{\Rej}[2]{\RejName_{#1}\left( {#2} \right)}
\newcommand{\Rm}[1]{\mathbb{R}^{#1}}
\newcommand{\Cm}[1]{\mathbb{C}^{#1}}
\newcommand{\conj}[0]{{*}}

%</misc>

% <grade selection>
%
\newcommand{\gpgrade}[2] {{\left\langle{{#1}}\right\rangle}_{#2}}

\newcommand{\gpgradezero}[1] {\gpgrade{#1}{}}
%\newcommand{\gpscalargrade}[1] {{\left\langle{{#1}}\right\rangle}}
%\newcommand{\gpgradezero}[1] {\gpgrade{#1}{0}}

%\newcommand{\gpgradeone}[1] {{\left\langle{{#1}}\right\rangle}_{1}}
\newcommand{\gpgradeone}[1] {\gpgrade{#1}{1}}

\newcommand{\gpgradetwo}[1] {\gpgrade{#1}{2}}
\newcommand{\gpgradethree}[1] {\gpgrade{#1}{3}}
\newcommand{\gpgradefour}[1] {\gpgrade{#1}{4}}
%
% </grade selection>



\newcommand{\adot}[0]{{\dot{a}}}
\newcommand{\bdot}[0]{{\dot{b}}}
% taken for centered dot:
%\newcommand{\cdot}[0]{{\dot{c}}}
%\newcommand{\ddot}[0]{{\dot{d}}}
\newcommand{\edot}[0]{{\dot{e}}}
\newcommand{\fdot}[0]{{\dot{f}}}
\newcommand{\gdot}[0]{{\dot{g}}}
\newcommand{\hdot}[0]{{\dot{h}}}
\newcommand{\idot}[0]{{\dot{i}}}
\newcommand{\jdot}[0]{{\dot{j}}}
\newcommand{\kdot}[0]{{\dot{k}}}
\newcommand{\ldot}[0]{{\dot{l}}}
\newcommand{\mdot}[0]{{\dot{m}}}
\newcommand{\ndot}[0]{{\dot{n}}}
%\newcommand{\odot}[0]{{\dot{o}}}
\newcommand{\pdot}[0]{{\dot{p}}}
\newcommand{\qdot}[0]{{\dot{q}}}
\newcommand{\rdot}[0]{{\dot{r}}}
\newcommand{\sdot}[0]{{\dot{s}}}
\newcommand{\tdot}[0]{{\dot{t}}}
\newcommand{\udot}[0]{{\dot{u}}}
\newcommand{\vdot}[0]{{\dot{v}}}
\newcommand{\wdot}[0]{{\dot{w}}}
\newcommand{\xdot}[0]{{\dot{x}}}
\newcommand{\ydot}[0]{{\dot{y}}}
\newcommand{\zdot}[0]{{\dot{z}}}
\newcommand{\addot}[0]{{\ddot{a}}}
\newcommand{\bddot}[0]{{\ddot{b}}}
\newcommand{\cddot}[0]{{\ddot{c}}}
%\newcommand{\dddot}[0]{{\ddot{d}}}
\newcommand{\eddot}[0]{{\ddot{e}}}
\newcommand{\fddot}[0]{{\ddot{f}}}
\newcommand{\gddot}[0]{{\ddot{g}}}
\newcommand{\hddot}[0]{{\ddot{h}}}
\newcommand{\iddot}[0]{{\ddot{i}}}
\newcommand{\jddot}[0]{{\ddot{j}}}
\newcommand{\kddot}[0]{{\ddot{k}}}
\newcommand{\lddot}[0]{{\ddot{l}}}
\newcommand{\mddot}[0]{{\ddot{m}}}
\newcommand{\nddot}[0]{{\ddot{n}}}
\newcommand{\oddot}[0]{{\ddot{o}}}
\newcommand{\pddot}[0]{{\ddot{p}}}
\newcommand{\qddot}[0]{{\ddot{q}}}
\newcommand{\rddot}[0]{{\ddot{r}}}
\newcommand{\sddot}[0]{{\ddot{s}}}
\newcommand{\tddot}[0]{{\ddot{t}}}
\newcommand{\uddot}[0]{{\ddot{u}}}
\newcommand{\vddot}[0]{{\ddot{v}}}
\newcommand{\wddot}[0]{{\ddot{w}}}
\newcommand{\xddot}[0]{{\ddot{x}}}
\newcommand{\yddot}[0]{{\ddot{y}}}
\newcommand{\zddot}[0]{{\ddot{z}}}

%<bold and dot greek symbols>
%

\newcommand{\Deltadot}[0]{{\dot{\Delta}}}
\newcommand{\Gammadot}[0]{{\dot{\Gamma}}}
\newcommand{\Lambdadot}[0]{{\dot{\Lambda}}}
\newcommand{\Omegadot}[0]{{\dot{\Omega}}}
\newcommand{\Phidot}[0]{{\dot{\Phi}}}
\newcommand{\Pidot}[0]{{\dot{\Pi}}}
\newcommand{\Psidot}[0]{{\dot{\Psi}}}
\newcommand{\Sigmadot}[0]{{\dot{\Sigma}}}
\newcommand{\Thetadot}[0]{{\dot{\Theta}}}
\newcommand{\Upsilondot}[0]{{\dot{\Upsilon}}}
\newcommand{\Xidot}[0]{{\dot{\Xi}}}
\newcommand{\alphadot}[0]{{\dot{\alpha}}}
\newcommand{\betadot}[0]{{\dot{\beta}}}
\newcommand{\chidot}[0]{{\dot{\chi}}}
\newcommand{\deltadot}[0]{{\dot{\delta}}}
\newcommand{\epsilondot}[0]{{\dot{\epsilon}}}
\newcommand{\etadot}[0]{{\dot{\eta}}}
\newcommand{\gammadot}[0]{{\dot{\gamma}}}
\newcommand{\kappadot}[0]{{\dot{\kappa}}}
\newcommand{\lambdadot}[0]{{\dot{\lambda}}}
\newcommand{\mudot}[0]{{\dot{\mu}}}
\newcommand{\nudot}[0]{{\dot{\nu}}}
\newcommand{\omegadot}[0]{{\dot{\omega}}}
\newcommand{\phidot}[0]{{\dot{\phi}}}
\newcommand{\pidot}[0]{{\dot{\pi}}}
\newcommand{\psidot}[0]{{\dot{\psi}}}
\newcommand{\rhodot}[0]{{\dot{\rho}}}
\newcommand{\sigmadot}[0]{{\dot{\sigma}}}
\newcommand{\taudot}[0]{{\dot{\tau}}}
\newcommand{\thetadot}[0]{{\dot{\theta}}}
\newcommand{\upsilondot}[0]{{\dot{\upsilon}}}
\newcommand{\varepsilondot}[0]{{\dot{\varepsilon}}}
\newcommand{\varphidot}[0]{{\dot{\varphi}}}
\newcommand{\varpidot}[0]{{\dot{\varpi}}}
\newcommand{\varrhodot}[0]{{\dot{\varrho}}}
\newcommand{\varsigmadot}[0]{{\dot{\varsigma}}}
\newcommand{\varthetadot}[0]{{\dot{\vartheta}}}
\newcommand{\xidot}[0]{{\dot{\xi}}}
\newcommand{\zetadot}[0]{{\dot{\zeta}}}

\newcommand{\Deltaddot}[0]{{\ddot{\Delta}}}
\newcommand{\Gammaddot}[0]{{\ddot{\Gamma}}}
\newcommand{\Lambdaddot}[0]{{\ddot{\Lambda}}}
\newcommand{\Omegaddot}[0]{{\ddot{\Omega}}}
\newcommand{\Phiddot}[0]{{\ddot{\Phi}}}
\newcommand{\Piddot}[0]{{\ddot{\Pi}}}
\newcommand{\Psiddot}[0]{{\ddot{\Psi}}}
\newcommand{\Sigmaddot}[0]{{\ddot{\Sigma}}}
\newcommand{\Thetaddot}[0]{{\ddot{\Theta}}}
\newcommand{\Upsilonddot}[0]{{\ddot{\Upsilon}}}
\newcommand{\Xiddot}[0]{{\ddot{\Xi}}}
\newcommand{\alphaddot}[0]{{\ddot{\alpha}}}
\newcommand{\betaddot}[0]{{\ddot{\beta}}}
\newcommand{\chiddot}[0]{{\ddot{\chi}}}
\newcommand{\deltaddot}[0]{{\ddot{\delta}}}
\newcommand{\epsilonddot}[0]{{\ddot{\epsilon}}}
\newcommand{\etaddot}[0]{{\ddot{\eta}}}
\newcommand{\gammaddot}[0]{{\ddot{\gamma}}}
\newcommand{\kappaddot}[0]{{\ddot{\kappa}}}
\newcommand{\lambdaddot}[0]{{\ddot{\lambda}}}
\newcommand{\muddot}[0]{{\ddot{\mu}}}
\newcommand{\nuddot}[0]{{\ddot{\nu}}}
\newcommand{\omegaddot}[0]{{\ddot{\omega}}}
\newcommand{\phiddot}[0]{{\ddot{\phi}}}
\newcommand{\piddot}[0]{{\ddot{\pi}}}
\newcommand{\psiddot}[0]{{\ddot{\psi}}}
\newcommand{\rhoddot}[0]{{\ddot{\rho}}}
\newcommand{\sigmaddot}[0]{{\ddot{\sigma}}}
\newcommand{\tauddot}[0]{{\ddot{\tau}}}
\newcommand{\thetaddot}[0]{{\ddot{\theta}}}
\newcommand{\upsilonddot}[0]{{\ddot{\upsilon}}}
\newcommand{\varepsilonddot}[0]{{\ddot{\varepsilon}}}
\newcommand{\varphiddot}[0]{{\ddot{\varphi}}}
\newcommand{\varpiddot}[0]{{\ddot{\varpi}}}
\newcommand{\varrhoddot}[0]{{\ddot{\varrho}}}
\newcommand{\varsigmaddot}[0]{{\ddot{\varsigma}}}
\newcommand{\varthetaddot}[0]{{\ddot{\vartheta}}}
\newcommand{\xiddot}[0]{{\ddot{\xi}}}
\newcommand{\zetaddot}[0]{{\ddot{\zeta}}}

\newcommand{\BDelta}[0]{\boldsymbol{\Delta}}
\newcommand{\BGamma}[0]{\boldsymbol{\Gamma}}
\newcommand{\BLambda}[0]{\boldsymbol{\Lambda}}
\newcommand{\BOmega}[0]{\boldsymbol{\Omega}}
\newcommand{\BPhi}[0]{\boldsymbol{\Phi}}
\newcommand{\BPi}[0]{\boldsymbol{\Pi}}
\newcommand{\BPsi}[0]{\boldsymbol{\Psi}}
\newcommand{\BSigma}[0]{\boldsymbol{\Sigma}}
\newcommand{\BTheta}[0]{\boldsymbol{\Theta}}
\newcommand{\BUpsilon}[0]{\boldsymbol{\Upsilon}}
\newcommand{\BXi}[0]{\boldsymbol{\Xi}}
\newcommand{\Balpha}[0]{\boldsymbol{\alpha}}
\newcommand{\Bbeta}[0]{\boldsymbol{\beta}}
\newcommand{\Bchi}[0]{\boldsymbol{\chi}}
\newcommand{\Bdelta}[0]{\boldsymbol{\delta}}
\newcommand{\Bepsilon}[0]{\boldsymbol{\epsilon}}
\newcommand{\Beta}[0]{\boldsymbol{\eta}}
\newcommand{\Bgamma}[0]{\boldsymbol{\gamma}}
\newcommand{\Bkappa}[0]{\boldsymbol{\kappa}}
\newcommand{\Blambda}[0]{\boldsymbol{\lambda}}
\newcommand{\Bmu}[0]{\boldsymbol{\mu}}
\newcommand{\Bnu}[0]{\boldsymbol{\nu}}
%\newcommand{\Bomega}[0]{\boldsymbol{\omega}}
\newcommand{\Bphi}[0]{\boldsymbol{\phi}}
\newcommand{\Bpi}[0]{\boldsymbol{\pi}}
\newcommand{\Bpsi}[0]{\boldsymbol{\psi}}
\newcommand{\Brho}[0]{\boldsymbol{\rho}}
\newcommand{\Bsigma}[0]{\boldsymbol{\sigma}}
%\newcommand{\Btau}[0]{\boldsymbol{\tau}}
%\newcommand{\Btheta}[0]{\boldsymbol{\theta}}
\newcommand{\Bupsilon}[0]{\boldsymbol{\upsilon}}
\newcommand{\Bvarepsilon}[0]{\boldsymbol{\varepsilon}}
\newcommand{\Bvarphi}[0]{\boldsymbol{\varphi}}
\newcommand{\Bvarpi}[0]{\boldsymbol{\varpi}}
\newcommand{\Bvarrho}[0]{\boldsymbol{\varrho}}
\newcommand{\Bvarsigma}[0]{\boldsymbol{\varsigma}}
\newcommand{\Bvartheta}[0]{\boldsymbol{\vartheta}}
\newcommand{\Bxi}[0]{\boldsymbol{\xi}}
\newcommand{\Bzeta}[0]{\boldsymbol{\zeta}}
%
%</bold and dot greek symbols>
%<infrequent>
%
%\newcommand{\AreaOp}[1]{\AName_{#1}}
%\newcommand{\Babs}[0]{\abs{\BB}}
%\newcommand{\Bcap}[0]{\hat{\BB}}
%\newcommand{\BrPrimeRej}[0]{\rcap(\rcap \wedge \Br')}
%\newcommand{\CA}[0]{\mathcal{A}}
%\newcommand{\Cos}[1]{\cos{\left({#1}\right)}}
%\newcommand{\Det}[1] {\abs{#1}}
%\newcommand{\Dsq}[2] {\frac {\partial^2 {#1}} {\partial {#2}^2}}
%\newcommand{\Exp}[1]{\exp{\left({#1}\right)}}
%\newcommand{\Norm}[1]{\left\lVert{#1}\right\rVert}
%\newcommand{\Sin}[1]{\sin{\left({#1}\right)}}
%\newcommand{\T}[0]{\text{T}}
%\newcommand{\VolumeOp}[1]{\VName_{#1}}
%\newcommand{\agrad}[0]{\Ba \cdot \nabla}
%\newcommand{\alphacap}[0]{\hat{\boldsymbol{\alpha}}}
%\newcommand{\Fcap}[0]{\hat{\BF}}
%\newcommand{\bithree}[0]{{\Bi}_3}
%\newcommand{\bxa}[0]{\Bx\Ba}
%\newcommand{\coordvec}[2]{
%\newcommand{\costheta}[0]{\acap \cdot \xcap}
%\newcommand{\ddt}[1]{\ddot{#1}}
%\newcommand{\ddu}[1] {\frac {d{#1}} {du}}
%\newcommand{\dsqxj}[2] {\frac {\partial^2 {#1}} {\partial {x_{#2}}^2}}
%\newcommand{\dtheta}[1]{\frac{d {#1}}{d \theta}}
%\newcommand{\dt}[1]{\dot{#1}}
%\newcommand{\dt}[1]{\frac{d {#1}}{dt}}
%\newcommand{\dxj}[2] {\frac {\partial {#1}} {\partial {x_{#2}}}}
%\newcommand{\halfPhi}[0]{\frac{\phi}{2}}
%\newcommand{\half}[0]{\inv{2}}
%\newcommand{\inv}[1]{\frac{1}{#1}}
%\newcommand{\laplacian}[0]{\nabla^2}
%\newcommand{\matrixoftx}[3]{
%\newcommand{\nrrp}[0]{\norm{\rcap \wedge \Br'}}
%\newcommand{\oiint}{\bigcirc \hspace{-1.4em} \int \hspace{-.8em} \int}
%\newcommand{\transpose}[1]{{#1}^{\text{T}}}
%\newcommand{\transpose}[1]{{{#1}^{\TextTranspose}}}
%\newcommand{\transpose}[1]{{{#1}^{\text{T}}}}
%\newcommand{\barA}[0]{\bar{A}}
%\newcommand{\qbar}[0]{\bar{q}}
%\newcommand{\qdotbar}[0]{\dot{\bar{q}}}
%
%</infrequent>





%\usepackage{txfonts} % for ointctr... (also appears to make "prettier" \int and \sum's)

%\usepackage[bookmarks=true]{hyperref}

%\usepackage{color,cite,graphicx}
   % use colour in the document, put your citations as [1-4]
   % rather than [1,2,3,4] (it looks nicer, and the extended LaTeX2e
   % graphics package.
%\usepackage{latexsym,amssymb,epsf} % do not remember if these are
   % needed, but their inclusion can not do any damage


\chapter{Dirac delta function in terms of orthogonal functions}
\label{chap:deltaOrthoSeries}
%\author{Peeter Joot \quad peeterjoot@protonmail.com }
\date{ March 8, 2009.  deltaOrthoSeries.tex }

%\begin{document}

%\maketitle{}
%\tableofcontents

\section{Motivation}

Chapter II of \citep{pauli2000wm} expresses the delta function in terms of
orthonormal basis functions, but the treatment is slightly
hard to follow.
Re-express some of this in my own words the slow and dumb way to get an
understanding of the ideas.  Also explore the summation representation of
the delta function and use it to relate Fourier series and transforms.

\section{Fourier coefficients}

Given an orthonormal basis

\begin{equation}\label{eqn:deltaOrthoSeries:20}
\begin{aligned}
\int u_m^\conj(x) u_n(x) = \delta_{mn}
\end{aligned}
\end{equation}

For a function that can be expressed entirely in this basis, such as

\begin{equation}\label{eqn:deltaOrthoSeries:40}
\begin{aligned}
f(x) = \sum_k a_k u_k(x)
\end{aligned}
\end{equation}

We can then compute the Fourier coefficients \(a_k\) in the normal fashion

\begin{equation}\label{eqn:deltaOrthoSeries:60}
\begin{aligned}
\int u_k^\conj(x) f(x) dx
&= \sum_n a_n \int u_k^\conj(x) u_n(x) dx \\
&= \sum_n a_n \delta_{kn} \\
&= a_k \\
\end{aligned}
\end{equation}

So we have
\begin{equation}\label{eqn:deltaOrthoSeries:80}
\begin{aligned}
f(x) = \sum_k a_k u_k(x)  = \sum_k u_k(x) \int u_k^\conj(x') f(x') dx'
\end{aligned}
\end{equation}

\subsection{Mean square convergence}

How good of a match is a subset of such a sum?  Pauli considers a mean convergence.

\begin{equation}\label{eqn:deltaOrthoSeries:100}
\begin{aligned}
0 &= \lim_{N \rightarrow \infty}\int
{\Abs{f(x') -\sum_{k=1}^N a_k u_k(x') }}^2 dx'  \\
&=
\int \left(f^\conj(x') -\sum_{k=1}^N a_k^\conj u_k^\conj(x') \right) \left(f(x') - \sum_{m=1}^N a_m u_m(x') \right)
dx' \\
&=
\int
\left( f^\conj(x') f(x')
-f^\conj(x') \sum_{m=1}^N a_m u_m(x')
- \sum_{k=1}^N a_k^\conj u_k^\conj(x') f(x')
+ \sum_{m=1}^N a_m u_m(x') \sum_{k=1}^N a_k^\conj u_k^\conj(x')  \right)
dx' \\
&=
\int f^\conj(x') f(x') dx'
- \sum_{m=1}^N a_m a_m^\conj
- \sum_{k=1}^N a_k^\conj a_k
+ \sum_{m=1}^N \sum_{k=1}^N a_m a_k^\conj \delta_{km} \\
%\int u_m(x') u_k^\conj(x') dx' \\
&= \int \Abs{f(x')}^2 dx' - \sum_{m=1}^N \Abs{a_m}^2 \\
\end{aligned}
\end{equation}

So if we have mean square equality in the limit as \(N \rightarrow \infty\), then it must also be true that

\begin{equation}\label{eqn:deltaOrthoSeries:120}
\begin{aligned}
\int \Abs{f(x')}^2 dx' = \sum_{m=1}^\infty \Abs{a_m}^2 \\
\end{aligned}
\end{equation}

He calls this the completeness relation.  If the orthonormal basis is sufficient to express the set of desired functions, then
the squared absolute value of such functions can be expressed entirely in terms of the Fourier coefficients.  The mean square
equality is weaker in the sense that a function can be mismatched to its Fourier representation at a set (of ``measure zero'') points,
and still meet the mean square equality statement.

\subsection{Generalizing the inner product}

Pauli next introduces the an inner product on functions (without calling it that)
in a somewhat indirect
fashion (ie: in terms of Fourier components instead of by definition).

Supposing that one has two functions built up by Fourier components

\begin{equation}\label{eqn:deltaOrthoSeries:140}
\begin{aligned}
f(x) &= \sum_k a_k u_k(x) \\
g(x) &= \sum_k b_k u_k(x) \\
\end{aligned}
\end{equation}

Then we have
\begin{equation}\label{eqn:deltaOrthoSeries:160}
\begin{aligned}
\int f^\conj(x) g(x) &= \sum_{k,m} a_k^\conj b_m \int u_k^\conj(x) u_m(x) = \sum_k a_k^\conj b_k \\
\int g^\conj(x) f(x) &= \sum_{k,m} a_k b_m^\conj \int u_m^\conj(x) u_k(x) = \sum_k b_k^\conj a_k \\
\end{aligned}
\end{equation}

This is something that is familiar to anybody who has taken a linear
algebra course, but perhaps had to be motivated when he wrote the book?

\subsection{Delta function as a sum}

Perhaps Pauli wrote this general function inner product that way to show a natural way that a sum of the
form

\begin{equation}\label{eqn:deltaOrthoSeries:180}
\begin{aligned}
\sum u_m^\conj(x) u_k(x)
\end{aligned}
\end{equation}

arises in use, because he now writes the completeness relation using a sum similar to that above

\begin{equation}\label{eqn:delta_ortho_series:deltaSum}
\begin{aligned}
\sum_{k} u_k^\conj(x') u_k(x) \equiv \delta(x-x')
\end{aligned}
\end{equation}

I had seen this in bra ket notation, in Susskind's lectures as noted in \chapcite{PJQmSusskind}, and also in \citep{mcmahon2005qmd} as the
identity operator

\begin{equation}\label{eqn:deltaOrthoSeries:200}
\begin{aligned}
\sum_{k} \ketbra{k}{k} \equiv \delta(x-x')
\end{aligned}
\end{equation}

From neither of those two sources did I understand where it came from (in Susskind's lectures it appeared to be
related to Fourier transforms).
As Pauli did, let us verify that this works, and try to relate this to a few specific choices of inner products (covering at
least classical Fourier series and the Fourier transform).

The relation of \eqnref{eqn:delta_ortho_series:deltaSum} can be shown to have delta function behavior by integration

\begin{equation}\label{eqn:deltaOrthoSeries:220}
\begin{aligned}
\int \sum_{k} u_k^\conj(x') u_k(x) f(x') dx'
&=
\sum_{k,m} u_k(x) a_m \int u_k^\conj(x') u_m(x') dx' \\
&=
\sum_{k,m} u_k(x) a_m \delta_{km} \\
&=
\sum_{k} u_k(x) a_k \\
&=
f(x)
\end{aligned}
\end{equation}

Strictly speaking this ought to be formulated in terms of mean square convergence since an arbitrary function f(x)
may differ from its Fourier sum at specific points (for example at points of discontinuity).

\subsubsection{Fourier series example}

Suppose the inner product is defined for the range \(I = [a, a+T]\).

\begin{equation}\label{eqn:deltaOrthoSeries:240}
\begin{aligned}
\Innerprod{f}{g} &= \int_{\partial I} f^\conj(x) g(x) dx
\end{aligned}
\end{equation}

What is the identity operator representation in the Fourier series basis \({u'}_k(x) = e^{ 2 \pi i k x / T}\)?  First the
normalization is required.

\begin{equation}\label{eqn:deltaOrthoSeries:260}
\begin{aligned}
\Innerprod{{u'}_k}{{u'}_m}
&= \int_{\partial I} e^{ 2 \pi i (m-k) x /T } dx  \\
&= \delta_{km} T
\end{aligned}
\end{equation}

So our orthonormalized basis is

\begin{equation}\label{eqn:deltaOrthoSeries:280}
\begin{aligned}
u_k(x) = \inv{\sqrt{T}} e^{ 2 \pi i k x / T}
\end{aligned}
\end{equation}

Given this orthonormal basis we can write

\begin{equation}\label{eqn:deltaOrthoSeries:300}
\begin{aligned}
f(x)
&= \sum_k a_k u_k(x) \\
a_k &= \int_{\partial I} u_k^\conj(x) f(x) dx = \Innerprod{u_k(x)}{f(x)} \\
\end{aligned}
\end{equation}

Or in a vector like notation

\begin{equation}\label{eqn:deltaOrthoSeries:320}
\begin{aligned}
f(x) &= \sum_k u_k(x) \Innerprod{u_k(x)}{f(x)}
\end{aligned}
\end{equation}

In this basis the
delta function (identity operator) form of \eqnref{eqn:delta_ortho_series:deltaSum}
becomes

\begin{equation}\label{eqn:deltaOrthoSeries:340}
\begin{aligned}
\delta(x- x') = \inv{{T}} \sum_k e^{ 2 \pi i k (x-x') / T}
\end{aligned}
\end{equation}

\subsubsection{Fourier transform inner-product}

For the Fourier transform we have an infinite range inner product

\begin{equation}\label{eqn:deltaOrthoSeries:360}
\begin{aligned}
\Innerprod{f}{g} &= \IIinf f^\conj(x) g(x) dx
\end{aligned}
\end{equation}

With a Fourier transform pair

\begin{equation}\label{eqn:deltaOrthoSeries:380}
\begin{aligned}
\hat{f}(k) &= \frac{1}{\sqrt{2\pi}} \int f(x) e^{-i k x} dx \\
{f}(x) &= \frac{1}{\sqrt{2\pi}} \int \hat{f}(k) e^{i k x } dk \\
\end{aligned}
\end{equation}

It appears that a natural choice of basis functions is actually \(u_k\) from the
Fourier series above with \(T=2\pi\).  That is

\begin{equation}\label{eqn:deltaOrthoSeries:400}
\begin{aligned}
u_k = \inv{\sqrt{2\pi}} e^{i k x}
\end{aligned}
\end{equation}

Our Fourier coefficients are now continuous and we have a form that
is very close to the discrete Fourier series

\begin{equation}\label{eqn:deltaOrthoSeries:420}
\begin{aligned}
f(x)
&= \int dk a_k u_k \\
a_k &= \int u_k^\conj(x) f(x) dx = \Innerprod{u_k(x)}{f(x)} \\
\end{aligned}
\end{equation}

Besides the inner product range difference from the discrete frequency case
the only other difference in this formulation is that we have a
\(\sum_k \rightarrow \int dk\) replacement.

What is the delta function representation in this inner product space?

A continuous variation of the summation delta function representation
in the Fourier series basis is

\begin{equation}\label{eqn:deltaOrthoSeries:440}
\begin{aligned}
\int dk u_k^\conj(x) u_k(x')
&=
\int dk \inv{2\pi} e^{ i k (x' - x)}
\end{aligned}
\end{equation}

Okay, cool.  The principle value of this integral is the sinc function
that is the familiar limiting form of the delta function.

This is an interesting and unifying way of expressing these Fourier
relationships.  The inner product is seen here to provide a more general
structure that is common to both the Fourier series and Fourier transform.
It is not surprising that the physicists rightly pick the algebraic
orthonormal function representation as fundamental ... too bad they do it
all with the braket notation that automatically obfuscates the subject.
%(somebody like me familiar with inner product spaces still can not look at
%a QM book and without wondering what sort of drug needs to be splook
%at a only they know anything about.

This also clarifies for me what Susskind did in his QM lectures.  There
he used the identity operator representation to express the Fourier transform
without ever touching on the tricky aspects of Fourier inversion.  That is a
tricky but interesting approach.

\subsubsection{Legendre polynomials}

Let us see how one non-Fourier like inner product function space
representation works out this way.

Using the Legendre inner product

\begin{equation}\label{eqn:deltaOrthoSeries:460}
\begin{aligned}
\Innerprod{f}{g} &= \int_{-1}^1 f(x) g(x) dx
\end{aligned}
\end{equation}

An orthonormal basis can be had by normalizing the
Legendre polynomials.
\href{ http://mathworld.wolfram.com/LegendrePolynomial.html }{Wolfram's Legendre Polynomial page} lists these in a number of closed forms

\begin{equation}\label{eqn:deltaOrthoSeries:480}
\begin{aligned}
P_n(x)
&= \inv{2 \pi i} \ointctrclockwise \frac{dt}{t^{n+1}\sqrt{1 - 2t x +t^2}} \\
&= \inv{2^n}\sum_{k=0}^n {\binom{n}{k}}^2 (x-1)^{n-k}(x+1)^k
\end{aligned}
\end{equation}

The first of these uses a closed contour around the origin.

These polynomials are not orthonormal, having
\begin{equation}\label{eqn:deltaOrthoSeries:500}
\begin{aligned}
\Innerprod{P_n}{P_m} &= \frac{2}{2 n + 1}\delta_{mn}
\end{aligned}
\end{equation}

So we have an orthonormal basis if we pick
\begin{equation}\label{eqn:deltaOrthoSeries:520}
\begin{aligned}
u_n(x) &= P_n(x) \sqrt{n + 1/2}
\end{aligned}
\end{equation}

Our delta function representation in this basis becomes

\begin{equation}\label{eqn:deltaOrthoSeries:540}
\begin{aligned}
\delta(x-x')
&\sim \sum_{n=0}^\infty \left(n+ \inv{2}\right) P_n(x') P_n(x) \\
&= -\inv{4\pi^2} \sum_{n=0}^\infty \left(n+ \inv{2}\right) \ointctrclockwise \frac{du}{u^{n+1}\sqrt{1 - 2u x' + u^2}} \ointctrclockwise \frac{dt}{t^{n+1}\sqrt{1 - 2t x + t^2}} \\
&= \sum_{n=0}^\infty \sum_{m=0}^n \sum_{k=0}^n \frac{n+ \inv{2}}{2^{2n}} {\binom{n}{m}}^2 (x'-1)^{n-m}(x'+1)^m {\binom{n}{k}}^2 (x-1)^{n-k}(x+1)^k
\end{aligned}
\end{equation}

Neither of these are familiar looking to me, but I was mostly curious to see one of these delta representations for a non-Fourier-ish basis.  A number of other
orthogonal polynomials can be found detailed in \href{http://mathworld.wolfram.com/OrthogonalPolynomials.html}{Wolfram's orthogonal polynomial article.}

%\bibliographystyle{plainnat}
%\bibliography{myrefs}

%\end{document}
    % mar 8/09
\documentclass{article}

\usepackage{amsmath}
\usepackage{mathpazo}

%
% shorthand for bold symbols, convenient for vectors and matrices
%
\newcommand{\Ba}[0]{\mathbf{a}}
\newcommand{\Bb}[0]{\mathbf{b}}
\newcommand{\Bc}[0]{\mathbf{c}}
\newcommand{\Bd}[0]{\mathbf{d}}
\newcommand{\Be}[0]{\mathbf{e}}
\newcommand{\Bf}[0]{\mathbf{f}}
\newcommand{\Bg}[0]{\mathbf{g}}
\newcommand{\Bh}[0]{\mathbf{h}}
\newcommand{\Bi}[0]{\mathbf{i}}
\newcommand{\Bj}[0]{\mathbf{j}}
\newcommand{\Bk}[0]{\mathbf{k}}
\newcommand{\Bl}[0]{\mathbf{l}}
\newcommand{\Bm}[0]{\mathbf{m}}
\newcommand{\Bn}[0]{\mathbf{n}}
\newcommand{\Bo}[0]{\mathbf{o}}
\newcommand{\Bp}[0]{\mathbf{p}}
\newcommand{\Bq}[0]{\mathbf{q}}
\newcommand{\Br}[0]{\mathbf{r}}
\newcommand{\Bs}[0]{\mathbf{s}}
\newcommand{\Bt}[0]{\mathbf{t}}
\newcommand{\Bu}[0]{\mathbf{u}}
\newcommand{\Bv}[0]{\mathbf{v}}
\newcommand{\Bw}[0]{\mathbf{w}}
\newcommand{\Bx}[0]{\mathbf{x}}
\newcommand{\By}[0]{\mathbf{y}}
\newcommand{\Bz}[0]{\mathbf{z}}
\newcommand{\BA}[0]{\mathbf{A}}
\newcommand{\BB}[0]{\mathbf{B}}
\newcommand{\BC}[0]{\mathbf{C}}
\newcommand{\BD}[0]{\mathbf{D}}
\newcommand{\BE}[0]{\mathbf{E}}
\newcommand{\BF}[0]{\mathbf{F}}
\newcommand{\BG}[0]{\mathbf{G}}
\newcommand{\BH}[0]{\mathbf{H}}
\newcommand{\BI}[0]{\mathbf{I}}
\newcommand{\BJ}[0]{\mathbf{J}}
\newcommand{\BK}[0]{\mathbf{K}}
\newcommand{\BL}[0]{\mathbf{L}}
\newcommand{\BM}[0]{\mathbf{M}}
\newcommand{\BN}[0]{\mathbf{N}}
\newcommand{\BO}[0]{\mathbf{O}}
\newcommand{\BP}[0]{\mathbf{P}}
\newcommand{\BQ}[0]{\mathbf{Q}}
\newcommand{\BR}[0]{\mathbf{R}}
\newcommand{\BS}[0]{\mathbf{S}}
\newcommand{\BT}[0]{\mathbf{T}}
\newcommand{\BU}[0]{\mathbf{U}}
\newcommand{\BV}[0]{\mathbf{V}}
\newcommand{\BW}[0]{\mathbf{W}}
\newcommand{\BX}[0]{\mathbf{X}}
\newcommand{\BY}[0]{\mathbf{Y}}
\newcommand{\BZ}[0]{\mathbf{Z}}

\newcommand{\Bzero}[0]{\mathbf{0}}
\newcommand{\Btheta}[0]{\boldsymbol{\theta}}
\newcommand{\Btau}[0]{\boldsymbol{\tau}}
\newcommand{\Bomega}[0]{\boldsymbol{\omega}}

%
% shorthand for unit vectors
%
\newcommand{\acap}[0]{\hat{\Ba}}
\newcommand{\bcap}[0]{\hat{\Bb}}
\newcommand{\ccap}[0]{\hat{\Bc}}
\newcommand{\dcap}[0]{\hat{\Bd}}
\newcommand{\ecap}[0]{\hat{\Be}}
\newcommand{\fcap}[0]{\hat{\Bf}}
\newcommand{\gcap}[0]{\hat{\Bg}}
\newcommand{\hcap}[0]{\hat{\Bh}}
\newcommand{\icap}[0]{\hat{\Bi}}
\newcommand{\jcap}[0]{\hat{\Bj}}
\newcommand{\kcap}[0]{\hat{\Bk}}
\newcommand{\lcap}[0]{\hat{\Bl}}
\newcommand{\mcap}[0]{\hat{\Bm}}
\newcommand{\ncap}[0]{\hat{\Bn}}
\newcommand{\ocap}[0]{\hat{\Bo}}
\newcommand{\pcap}[0]{\hat{\Bp}}
\newcommand{\qcap}[0]{\hat{\Bq}}
\newcommand{\rcap}[0]{\hat{\Br}}
\newcommand{\scap}[0]{\hat{\Bs}}
\newcommand{\tcap}[0]{\hat{\Bt}}
\newcommand{\ucap}[0]{\hat{\Bu}}
\newcommand{\vcap}[0]{\hat{\Bv}}
\newcommand{\wcap}[0]{\hat{\Bw}}
\newcommand{\xcap}[0]{\hat{\Bx}}
\newcommand{\ycap}[0]{\hat{\By}}
\newcommand{\zcap}[0]{\hat{\Bz}}
\newcommand{\thetacap}[0]{\hat{\Btheta}}

%
% to write R^n and C^n in a distinguishable fashion.  Perhaps change this
% to the double lined characters upon figuring out how to do so.
%
\newcommand{\C}[1]{$\mathbb{C}^{#1}$}
\newcommand{\R}[1]{$\mathbb{R}^{#1}$}

%
% various generally useful helpers
%

% derivative of #1 wrt. #2:
\newcommand{\D}[2] {\frac {d#2} {d#1}}

\newcommand{\inv}[1]{\frac{1}{#1}}
\newcommand{\cross}[0]{\times}

\newcommand{\abs}[1]{\lvert{#1}\rvert}
\newcommand{\norm}[1]{\lVert{#1}\rVert}
\newcommand{\innerprod}[2]{\langle{#1}, {#2}\rangle}
\newcommand{\dotprod}[2]{{#1} \cdot {#2}}
\newcommand{\bdotprod}[2]{\left({#1} \cdot {#2}\right)}
\newcommand{\crossprod}[2]{{#1} \cross {#2}}
\newcommand{\tripleprod}[3]{\dotprod{\left(\crossprod{#1}{#2}\right)}{#3}}

\DeclareMathOperator{\Proj}{Proj}
\DeclareMathOperator{\Span}{span}
\DeclareMathOperator{\Sgn}{sgn}
\DeclareMathOperator{\Area}{Area}
\DeclareMathOperator{\Volume}{Volume}

%
% A few miscellaneous things specific to this document
%
\newcommand{\crossop}[1]{\crossprod{#1}{}}

% R2 vector.
\newcommand{\VectorTwo}[2]{
\begin{bmatrix}
 {#1} \\
 {#2}
\end{bmatrix}
}

\newcommand{\VectorN}[1]{
\begin{bmatrix}
{#1}_1 \\
{#1}_2 \\
\vdots \\
{#1}_N \\
\end{bmatrix}
}

\newcommand{\DETuvij}[4]{
\begin{vmatrix}
 {#1}_{#3} & {#1}_{#4} \\
 {#2}_{#3} & {#2}_{#4}
\end{vmatrix}
}

\newcommand{\DETuvwijk}[6]{
\begin{vmatrix}
 {#1}_{#4} & {#1}_{#5} & {#1}_{#6} \\
 {#2}_{#4} & {#2}_{#5} & {#2}_{#6} \\
 {#3}_{#4} & {#3}_{#5} & {#3}_{#6}
\end{vmatrix}
}

\newcommand{\DETuvwxijkl}[8]{
\begin{vmatrix}
 {#1}_{#5} & {#1}_{#6} & {#1}_{#7} & {#1}_{#8} \\
 {#2}_{#5} & {#2}_{#6} & {#2}_{#7} & {#2}_{#8} \\
 {#3}_{#5} & {#3}_{#6} & {#3}_{#7} & {#3}_{#8} \\
 {#4}_{#5} & {#4}_{#6} & {#4}_{#7} & {#4}_{#8} \\
\end{vmatrix}
}

%\newcommand{\DETuvwxyijklm}[10]{
%\begin{vmatrix}
% {#1}_{#6} & {#1}_{#7} & {#1}_{#8} & {#1}_{#9} & {#1}_{#10} \\
% {#2}_{#6} & {#2}_{#7} & {#2}_{#8} & {#2}_{#9} & {#2}_{#10} \\
% {#3}_{#6} & {#3}_{#7} & {#3}_{#8} & {#3}_{#9} & {#3}_{#10} \\
% {#4}_{#6} & {#4}_{#7} & {#4}_{#8} & {#4}_{#9} & {#4}_{#10} \\
% {#5}_{#6} & {#5}_{#7} & {#5}_{#8} & {#5}_{#9} & {#5}_{#10}
%\end{vmatrix}
%}

% R3 vector.
\newcommand{\VectorThree}[3]{
\begin{bmatrix}
 {#1} \\
 {#2} \\
 {#3}
\end{bmatrix}
}


%<misc>
%
\newcommand{\Abs}[1]{{\left\lvert{#1}\right\rvert}}
\newcommand{\spacegrad}[0]{\boldsymbol{\nabla}}
\newcommand{\grad}[0]{\nabla}
\newcommand{\LL}[0]{\mathcal{L}}

% == \partial_{#1} {#2}
\newcommand{\PD}[2]{\frac{\partial {#2}}{\partial {#1}}}
% inline variant
\newcommand{\PDi}[2]{{\partial {#2}}/{\partial {#1}}}

\newcommand{\PDD}[3]{\frac{\partial^2 {#3}}{\partial {#1}\partial {#2}}}
%\newcommand{\PDd}[2]{\frac{\partial^2 {#2}}{{\partial{#1}}^2}}
\newcommand{\PDsq}[2]{\frac{\partial^2 {#2}}{(\partial {#1})^2}}

\newcommand{\Partial}[2]{\frac{\partial {#1}}{\partial {#2}}}
\DeclareMathOperator{\RejName}{Rej}
\newcommand{\Rej}[2]{\RejName_{#1}\left( {#2} \right)}
\newcommand{\Rm}[1]{\mathbb{R}^{#1}}
\newcommand{\Cm}[1]{\mathbb{C}^{#1}}
\newcommand{\conj}[0]{{*}}

%</misc>

% <grade selection>
%
\newcommand{\gpgrade}[2] {{\left\langle{{#1}}\right\rangle}_{#2}}

\newcommand{\gpgradezero}[1] {\gpgrade{#1}{}}
%\newcommand{\gpscalargrade}[1] {{\left\langle{{#1}}\right\rangle}}
%\newcommand{\gpgradezero}[1] {\gpgrade{#1}{0}}

%\newcommand{\gpgradeone}[1] {{\left\langle{{#1}}\right\rangle}_{1}}
\newcommand{\gpgradeone}[1] {\gpgrade{#1}{1}}

\newcommand{\gpgradetwo}[1] {\gpgrade{#1}{2}}
\newcommand{\gpgradethree}[1] {\gpgrade{#1}{3}}
\newcommand{\gpgradefour}[1] {\gpgrade{#1}{4}}
%
% </grade selection>



\newcommand{\adot}[0]{{\dot{a}}}
\newcommand{\bdot}[0]{{\dot{b}}}
% taken for centered dot:
%\newcommand{\cdot}[0]{{\dot{c}}}
%\newcommand{\ddot}[0]{{\dot{d}}}
\newcommand{\edot}[0]{{\dot{e}}}
\newcommand{\fdot}[0]{{\dot{f}}}
\newcommand{\gdot}[0]{{\dot{g}}}
\newcommand{\hdot}[0]{{\dot{h}}}
\newcommand{\idot}[0]{{\dot{i}}}
\newcommand{\jdot}[0]{{\dot{j}}}
\newcommand{\kdot}[0]{{\dot{k}}}
\newcommand{\ldot}[0]{{\dot{l}}}
\newcommand{\mdot}[0]{{\dot{m}}}
\newcommand{\ndot}[0]{{\dot{n}}}
%\newcommand{\odot}[0]{{\dot{o}}}
\newcommand{\pdot}[0]{{\dot{p}}}
\newcommand{\qdot}[0]{{\dot{q}}}
\newcommand{\rdot}[0]{{\dot{r}}}
\newcommand{\sdot}[0]{{\dot{s}}}
\newcommand{\tdot}[0]{{\dot{t}}}
\newcommand{\udot}[0]{{\dot{u}}}
\newcommand{\vdot}[0]{{\dot{v}}}
\newcommand{\wdot}[0]{{\dot{w}}}
\newcommand{\xdot}[0]{{\dot{x}}}
\newcommand{\ydot}[0]{{\dot{y}}}
\newcommand{\zdot}[0]{{\dot{z}}}
\newcommand{\addot}[0]{{\ddot{a}}}
\newcommand{\bddot}[0]{{\ddot{b}}}
\newcommand{\cddot}[0]{{\ddot{c}}}
%\newcommand{\dddot}[0]{{\ddot{d}}}
\newcommand{\eddot}[0]{{\ddot{e}}}
\newcommand{\fddot}[0]{{\ddot{f}}}
\newcommand{\gddot}[0]{{\ddot{g}}}
\newcommand{\hddot}[0]{{\ddot{h}}}
\newcommand{\iddot}[0]{{\ddot{i}}}
\newcommand{\jddot}[0]{{\ddot{j}}}
\newcommand{\kddot}[0]{{\ddot{k}}}
\newcommand{\lddot}[0]{{\ddot{l}}}
\newcommand{\mddot}[0]{{\ddot{m}}}
\newcommand{\nddot}[0]{{\ddot{n}}}
\newcommand{\oddot}[0]{{\ddot{o}}}
\newcommand{\pddot}[0]{{\ddot{p}}}
\newcommand{\qddot}[0]{{\ddot{q}}}
\newcommand{\rddot}[0]{{\ddot{r}}}
\newcommand{\sddot}[0]{{\ddot{s}}}
\newcommand{\tddot}[0]{{\ddot{t}}}
\newcommand{\uddot}[0]{{\ddot{u}}}
\newcommand{\vddot}[0]{{\ddot{v}}}
\newcommand{\wddot}[0]{{\ddot{w}}}
\newcommand{\xddot}[0]{{\ddot{x}}}
\newcommand{\yddot}[0]{{\ddot{y}}}
\newcommand{\zddot}[0]{{\ddot{z}}}

%<bold and dot greek symbols>
%

\newcommand{\Deltadot}[0]{{\dot{\Delta}}}
\newcommand{\Gammadot}[0]{{\dot{\Gamma}}}
\newcommand{\Lambdadot}[0]{{\dot{\Lambda}}}
\newcommand{\Omegadot}[0]{{\dot{\Omega}}}
\newcommand{\Phidot}[0]{{\dot{\Phi}}}
\newcommand{\Pidot}[0]{{\dot{\Pi}}}
\newcommand{\Psidot}[0]{{\dot{\Psi}}}
\newcommand{\Sigmadot}[0]{{\dot{\Sigma}}}
\newcommand{\Thetadot}[0]{{\dot{\Theta}}}
\newcommand{\Upsilondot}[0]{{\dot{\Upsilon}}}
\newcommand{\Xidot}[0]{{\dot{\Xi}}}
\newcommand{\alphadot}[0]{{\dot{\alpha}}}
\newcommand{\betadot}[0]{{\dot{\beta}}}
\newcommand{\chidot}[0]{{\dot{\chi}}}
\newcommand{\deltadot}[0]{{\dot{\delta}}}
\newcommand{\epsilondot}[0]{{\dot{\epsilon}}}
\newcommand{\etadot}[0]{{\dot{\eta}}}
\newcommand{\gammadot}[0]{{\dot{\gamma}}}
\newcommand{\kappadot}[0]{{\dot{\kappa}}}
\newcommand{\lambdadot}[0]{{\dot{\lambda}}}
\newcommand{\mudot}[0]{{\dot{\mu}}}
\newcommand{\nudot}[0]{{\dot{\nu}}}
\newcommand{\omegadot}[0]{{\dot{\omega}}}
\newcommand{\phidot}[0]{{\dot{\phi}}}
\newcommand{\pidot}[0]{{\dot{\pi}}}
\newcommand{\psidot}[0]{{\dot{\psi}}}
\newcommand{\rhodot}[0]{{\dot{\rho}}}
\newcommand{\sigmadot}[0]{{\dot{\sigma}}}
\newcommand{\taudot}[0]{{\dot{\tau}}}
\newcommand{\thetadot}[0]{{\dot{\theta}}}
\newcommand{\upsilondot}[0]{{\dot{\upsilon}}}
\newcommand{\varepsilondot}[0]{{\dot{\varepsilon}}}
\newcommand{\varphidot}[0]{{\dot{\varphi}}}
\newcommand{\varpidot}[0]{{\dot{\varpi}}}
\newcommand{\varrhodot}[0]{{\dot{\varrho}}}
\newcommand{\varsigmadot}[0]{{\dot{\varsigma}}}
\newcommand{\varthetadot}[0]{{\dot{\vartheta}}}
\newcommand{\xidot}[0]{{\dot{\xi}}}
\newcommand{\zetadot}[0]{{\dot{\zeta}}}

\newcommand{\Deltaddot}[0]{{\ddot{\Delta}}}
\newcommand{\Gammaddot}[0]{{\ddot{\Gamma}}}
\newcommand{\Lambdaddot}[0]{{\ddot{\Lambda}}}
\newcommand{\Omegaddot}[0]{{\ddot{\Omega}}}
\newcommand{\Phiddot}[0]{{\ddot{\Phi}}}
\newcommand{\Piddot}[0]{{\ddot{\Pi}}}
\newcommand{\Psiddot}[0]{{\ddot{\Psi}}}
\newcommand{\Sigmaddot}[0]{{\ddot{\Sigma}}}
\newcommand{\Thetaddot}[0]{{\ddot{\Theta}}}
\newcommand{\Upsilonddot}[0]{{\ddot{\Upsilon}}}
\newcommand{\Xiddot}[0]{{\ddot{\Xi}}}
\newcommand{\alphaddot}[0]{{\ddot{\alpha}}}
\newcommand{\betaddot}[0]{{\ddot{\beta}}}
\newcommand{\chiddot}[0]{{\ddot{\chi}}}
\newcommand{\deltaddot}[0]{{\ddot{\delta}}}
\newcommand{\epsilonddot}[0]{{\ddot{\epsilon}}}
\newcommand{\etaddot}[0]{{\ddot{\eta}}}
\newcommand{\gammaddot}[0]{{\ddot{\gamma}}}
\newcommand{\kappaddot}[0]{{\ddot{\kappa}}}
\newcommand{\lambdaddot}[0]{{\ddot{\lambda}}}
\newcommand{\muddot}[0]{{\ddot{\mu}}}
\newcommand{\nuddot}[0]{{\ddot{\nu}}}
\newcommand{\omegaddot}[0]{{\ddot{\omega}}}
\newcommand{\phiddot}[0]{{\ddot{\phi}}}
\newcommand{\piddot}[0]{{\ddot{\pi}}}
\newcommand{\psiddot}[0]{{\ddot{\psi}}}
\newcommand{\rhoddot}[0]{{\ddot{\rho}}}
\newcommand{\sigmaddot}[0]{{\ddot{\sigma}}}
\newcommand{\tauddot}[0]{{\ddot{\tau}}}
\newcommand{\thetaddot}[0]{{\ddot{\theta}}}
\newcommand{\upsilonddot}[0]{{\ddot{\upsilon}}}
\newcommand{\varepsilonddot}[0]{{\ddot{\varepsilon}}}
\newcommand{\varphiddot}[0]{{\ddot{\varphi}}}
\newcommand{\varpiddot}[0]{{\ddot{\varpi}}}
\newcommand{\varrhoddot}[0]{{\ddot{\varrho}}}
\newcommand{\varsigmaddot}[0]{{\ddot{\varsigma}}}
\newcommand{\varthetaddot}[0]{{\ddot{\vartheta}}}
\newcommand{\xiddot}[0]{{\ddot{\xi}}}
\newcommand{\zetaddot}[0]{{\ddot{\zeta}}}

\newcommand{\BDelta}[0]{\boldsymbol{\Delta}}
\newcommand{\BGamma}[0]{\boldsymbol{\Gamma}}
\newcommand{\BLambda}[0]{\boldsymbol{\Lambda}}
\newcommand{\BOmega}[0]{\boldsymbol{\Omega}}
\newcommand{\BPhi}[0]{\boldsymbol{\Phi}}
\newcommand{\BPi}[0]{\boldsymbol{\Pi}}
\newcommand{\BPsi}[0]{\boldsymbol{\Psi}}
\newcommand{\BSigma}[0]{\boldsymbol{\Sigma}}
\newcommand{\BTheta}[0]{\boldsymbol{\Theta}}
\newcommand{\BUpsilon}[0]{\boldsymbol{\Upsilon}}
\newcommand{\BXi}[0]{\boldsymbol{\Xi}}
\newcommand{\Balpha}[0]{\boldsymbol{\alpha}}
\newcommand{\Bbeta}[0]{\boldsymbol{\beta}}
\newcommand{\Bchi}[0]{\boldsymbol{\chi}}
\newcommand{\Bdelta}[0]{\boldsymbol{\delta}}
\newcommand{\Bepsilon}[0]{\boldsymbol{\epsilon}}
\newcommand{\Beta}[0]{\boldsymbol{\eta}}
\newcommand{\Bgamma}[0]{\boldsymbol{\gamma}}
\newcommand{\Bkappa}[0]{\boldsymbol{\kappa}}
\newcommand{\Blambda}[0]{\boldsymbol{\lambda}}
\newcommand{\Bmu}[0]{\boldsymbol{\mu}}
\newcommand{\Bnu}[0]{\boldsymbol{\nu}}
%\newcommand{\Bomega}[0]{\boldsymbol{\omega}}
\newcommand{\Bphi}[0]{\boldsymbol{\phi}}
\newcommand{\Bpi}[0]{\boldsymbol{\pi}}
\newcommand{\Bpsi}[0]{\boldsymbol{\psi}}
\newcommand{\Brho}[0]{\boldsymbol{\rho}}
\newcommand{\Bsigma}[0]{\boldsymbol{\sigma}}
%\newcommand{\Btau}[0]{\boldsymbol{\tau}}
%\newcommand{\Btheta}[0]{\boldsymbol{\theta}}
\newcommand{\Bupsilon}[0]{\boldsymbol{\upsilon}}
\newcommand{\Bvarepsilon}[0]{\boldsymbol{\varepsilon}}
\newcommand{\Bvarphi}[0]{\boldsymbol{\varphi}}
\newcommand{\Bvarpi}[0]{\boldsymbol{\varpi}}
\newcommand{\Bvarrho}[0]{\boldsymbol{\varrho}}
\newcommand{\Bvarsigma}[0]{\boldsymbol{\varsigma}}
\newcommand{\Bvartheta}[0]{\boldsymbol{\vartheta}}
\newcommand{\Bxi}[0]{\boldsymbol{\xi}}
\newcommand{\Bzeta}[0]{\boldsymbol{\zeta}}
%
%</bold and dot greek symbols>
%<infrequent>
%
%\newcommand{\AreaOp}[1]{\AName_{#1}}
%\newcommand{\Babs}[0]{\abs{\BB}}
%\newcommand{\Bcap}[0]{\hat{\BB}}
%\newcommand{\BrPrimeRej}[0]{\rcap(\rcap \wedge \Br')}
%\newcommand{\CA}[0]{\mathcal{A}}
%\newcommand{\Cos}[1]{\cos{\left({#1}\right)}}
%\newcommand{\Det}[1] {\abs{#1}}
%\newcommand{\Dsq}[2] {\frac {\partial^2 {#1}} {\partial {#2}^2}}
%\newcommand{\Exp}[1]{\exp{\left({#1}\right)}}
%\newcommand{\Norm}[1]{\left\lVert{#1}\right\rVert}
%\newcommand{\Sin}[1]{\sin{\left({#1}\right)}}
%\newcommand{\T}[0]{\text{T}}
%\newcommand{\VolumeOp}[1]{\VName_{#1}}
%\newcommand{\agrad}[0]{\Ba \cdot \nabla}
%\newcommand{\alphacap}[0]{\hat{\boldsymbol{\alpha}}}
%\newcommand{\Fcap}[0]{\hat{\BF}}
%\newcommand{\bithree}[0]{{\Bi}_3}
%\newcommand{\bxa}[0]{\Bx\Ba}
%\newcommand{\coordvec}[2]{
%\newcommand{\costheta}[0]{\acap \cdot \xcap}
%\newcommand{\ddt}[1]{\ddot{#1}}
%\newcommand{\ddu}[1] {\frac {d{#1}} {du}}
%\newcommand{\dsqxj}[2] {\frac {\partial^2 {#1}} {\partial {x_{#2}}^2}}
%\newcommand{\dtheta}[1]{\frac{d {#1}}{d \theta}}
%\newcommand{\dt}[1]{\dot{#1}}
%\newcommand{\dt}[1]{\frac{d {#1}}{dt}}
%\newcommand{\dxj}[2] {\frac {\partial {#1}} {\partial {x_{#2}}}}
%\newcommand{\halfPhi}[0]{\frac{\phi}{2}}
%\newcommand{\half}[0]{\inv{2}}
%\newcommand{\inv}[1]{\frac{1}{#1}}
%\newcommand{\laplacian}[0]{\nabla^2}
%\newcommand{\matrixoftx}[3]{
%\newcommand{\nrrp}[0]{\norm{\rcap \wedge \Br'}}
%\newcommand{\oiint}{\bigcirc \hspace{-1.4em} \int \hspace{-.8em} \int}
%\newcommand{\transpose}[1]{{#1}^{\text{T}}}
%\newcommand{\transpose}[1]{{{#1}^{\TextTranspose}}}
%\newcommand{\transpose}[1]{{{#1}^{\text{T}}}}
%\newcommand{\barA}[0]{\bar{A}}
%\newcommand{\qbar}[0]{\bar{q}}
%\newcommand{\qdotbar}[0]{\dot{\bar{q}}}
%
%</infrequent>





%\usepackage{listings}
%\usepackage{txfonts} % for ointctr... (also appears to make "prettier" \int and \sum's)
\usepackage[bookmarks=true]{hyperref}

\usepackage{color,cite,graphicx}
   % use colour in the document, put your citations as [1-4]
   % rather than [1,2,3,4] (it looks nicer, and the extended LaTeX2e
   % graphics package. 
\usepackage{latexsym,amssymb,epsf} % don't remember if these are
   % needed, but their inclusion can't do any damage


\title{ Some Klien-Gordon equation notes. }
\author{Peeter Joot \quad peeter.joot@gmail.com }
\date{ March 27, 2009.  Last Revision: $Date: 2009/03/27 22:53:49 $ }

\begin{document}

\maketitle{}
\tableofcontents
\section{ Motivation }

Want to explore the ideas of global and local gauge invariance.  I seem to recall that Susskind
used the Klien-Gordon Lagrangian, which had a form something like

\begin{align}\label{eqn:densityUndeterminedCoeff}
\LL = \partial^\mu \psi \partial_\mu \psi^\conj + \alpha m^2 \psi \psi^\conj
\end{align}

Since this was one of the simplest forms to apply the 
a relativisitic gauge transformation concept to.

\subsection{ Determine that constant. }

We want

\begin{align*}
\grad^2 psi = \frac{m^2 c^2}{\hbar^2} \psi
\end{align*}

So, to start things off let's do the variation on the Langrangian density of \ref{eqn:densityUndeterminedCoeff}.  For fun, let's try it with Feynman's method (first order Taylor expansion, and no memorization
of the field form of the Euler-Lagrange equations).

Write $\psi = \phi + \epsilon$, where $\phi$ is the desired solution and $\epsilon$ is a field that
vanishes on the boundaries of the action integral

\begin{align*}
S 
&= \int \LL d^4 x \\
%&= 
%\int \left( \partial^\mu (\phi + \epsilon) \partial_\mu (\phi + \epsilon)^\conj + \alpha m^2 (\phi + \epsilon) (\phi + \epsilon)^\conj \right) d^4 x \\
%&= 
%\int d^4 x \partial^\mu \phi \partial_\mu \phi^\conj 
%+ \alpha m^2 \int \phi \phi^\conj \right) d^4 x \\
%&+ \int d^4 x \partial^\mu \epsilon \partial_\mu \phi^\conj 
%+ \int d^4 x \partial^\mu \phi \partial_\mu \epsilon^\conj  \\
%&+ \alpha m^2 \int \phi \epsilon^\conj \right) d^4 x
%+ \alpha m^2 \int \epsilon \phi^\conj \right) d^4 x \\
%&+ \alpha m^2 \int \epsilon \epsilon^\conj \right) d^4 x
%+ \int d^4 x \partial^\mu \epsilon \partial_\mu \epsilon^\conj \\
\end{align*}

\bibliographystyle{plainnat}
\bibliography{myrefs}

\end{document}
         % mar 27/09
\documentclass{article}

\usepackage{amsmath}
\usepackage{mathpazo}

%
% shorthand for bold symbols, convenient for vectors and matrices
%
\newcommand{\Ba}[0]{\mathbf{a}}
\newcommand{\Bb}[0]{\mathbf{b}}
\newcommand{\Bc}[0]{\mathbf{c}}
\newcommand{\Bd}[0]{\mathbf{d}}
\newcommand{\Be}[0]{\mathbf{e}}
\newcommand{\Bf}[0]{\mathbf{f}}
\newcommand{\Bg}[0]{\mathbf{g}}
\newcommand{\Bh}[0]{\mathbf{h}}
\newcommand{\Bi}[0]{\mathbf{i}}
\newcommand{\Bj}[0]{\mathbf{j}}
\newcommand{\Bk}[0]{\mathbf{k}}
\newcommand{\Bl}[0]{\mathbf{l}}
\newcommand{\Bm}[0]{\mathbf{m}}
\newcommand{\Bn}[0]{\mathbf{n}}
\newcommand{\Bo}[0]{\mathbf{o}}
\newcommand{\Bp}[0]{\mathbf{p}}
\newcommand{\Bq}[0]{\mathbf{q}}
\newcommand{\Br}[0]{\mathbf{r}}
\newcommand{\Bs}[0]{\mathbf{s}}
\newcommand{\Bt}[0]{\mathbf{t}}
\newcommand{\Bu}[0]{\mathbf{u}}
\newcommand{\Bv}[0]{\mathbf{v}}
\newcommand{\Bw}[0]{\mathbf{w}}
\newcommand{\Bx}[0]{\mathbf{x}}
\newcommand{\By}[0]{\mathbf{y}}
\newcommand{\Bz}[0]{\mathbf{z}}
\newcommand{\BA}[0]{\mathbf{A}}
\newcommand{\BB}[0]{\mathbf{B}}
\newcommand{\BC}[0]{\mathbf{C}}
\newcommand{\BD}[0]{\mathbf{D}}
\newcommand{\BE}[0]{\mathbf{E}}
\newcommand{\BF}[0]{\mathbf{F}}
\newcommand{\BG}[0]{\mathbf{G}}
\newcommand{\BH}[0]{\mathbf{H}}
\newcommand{\BI}[0]{\mathbf{I}}
\newcommand{\BJ}[0]{\mathbf{J}}
\newcommand{\BK}[0]{\mathbf{K}}
\newcommand{\BL}[0]{\mathbf{L}}
\newcommand{\BM}[0]{\mathbf{M}}
\newcommand{\BN}[0]{\mathbf{N}}
\newcommand{\BO}[0]{\mathbf{O}}
\newcommand{\BP}[0]{\mathbf{P}}
\newcommand{\BQ}[0]{\mathbf{Q}}
\newcommand{\BR}[0]{\mathbf{R}}
\newcommand{\BS}[0]{\mathbf{S}}
\newcommand{\BT}[0]{\mathbf{T}}
\newcommand{\BU}[0]{\mathbf{U}}
\newcommand{\BV}[0]{\mathbf{V}}
\newcommand{\BW}[0]{\mathbf{W}}
\newcommand{\BX}[0]{\mathbf{X}}
\newcommand{\BY}[0]{\mathbf{Y}}
\newcommand{\BZ}[0]{\mathbf{Z}}

\newcommand{\Bzero}[0]{\mathbf{0}}
\newcommand{\Btheta}[0]{\boldsymbol{\theta}}
\newcommand{\Btau}[0]{\boldsymbol{\tau}}
\newcommand{\Bomega}[0]{\boldsymbol{\omega}}

%
% shorthand for unit vectors
%
\newcommand{\acap}[0]{\hat{\Ba}}
\newcommand{\bcap}[0]{\hat{\Bb}}
\newcommand{\ccap}[0]{\hat{\Bc}}
\newcommand{\dcap}[0]{\hat{\Bd}}
\newcommand{\ecap}[0]{\hat{\Be}}
\newcommand{\fcap}[0]{\hat{\Bf}}
\newcommand{\gcap}[0]{\hat{\Bg}}
\newcommand{\hcap}[0]{\hat{\Bh}}
\newcommand{\icap}[0]{\hat{\Bi}}
\newcommand{\jcap}[0]{\hat{\Bj}}
\newcommand{\kcap}[0]{\hat{\Bk}}
\newcommand{\lcap}[0]{\hat{\Bl}}
\newcommand{\mcap}[0]{\hat{\Bm}}
\newcommand{\ncap}[0]{\hat{\Bn}}
\newcommand{\ocap}[0]{\hat{\Bo}}
\newcommand{\pcap}[0]{\hat{\Bp}}
\newcommand{\qcap}[0]{\hat{\Bq}}
\newcommand{\rcap}[0]{\hat{\Br}}
\newcommand{\scap}[0]{\hat{\Bs}}
\newcommand{\tcap}[0]{\hat{\Bt}}
\newcommand{\ucap}[0]{\hat{\Bu}}
\newcommand{\vcap}[0]{\hat{\Bv}}
\newcommand{\wcap}[0]{\hat{\Bw}}
\newcommand{\xcap}[0]{\hat{\Bx}}
\newcommand{\ycap}[0]{\hat{\By}}
\newcommand{\zcap}[0]{\hat{\Bz}}
\newcommand{\thetacap}[0]{\hat{\Btheta}}

%
% to write R^n and C^n in a distinguishable fashion.  Perhaps change this
% to the double lined characters upon figuring out how to do so.
%
\newcommand{\C}[1]{$\mathbb{C}^{#1}$}
\newcommand{\R}[1]{$\mathbb{R}^{#1}$}

%
% various generally useful helpers
%

% derivative of #1 wrt. #2:
\newcommand{\D}[2] {\frac {d#2} {d#1}}

\newcommand{\inv}[1]{\frac{1}{#1}}
\newcommand{\cross}[0]{\times}

\newcommand{\abs}[1]{\lvert{#1}\rvert}
\newcommand{\norm}[1]{\lVert{#1}\rVert}
\newcommand{\innerprod}[2]{\langle{#1}, {#2}\rangle}
\newcommand{\dotprod}[2]{{#1} \cdot {#2}}
\newcommand{\bdotprod}[2]{\left({#1} \cdot {#2}\right)}
\newcommand{\crossprod}[2]{{#1} \cross {#2}}
\newcommand{\tripleprod}[3]{\dotprod{\left(\crossprod{#1}{#2}\right)}{#3}}

\DeclareMathOperator{\Proj}{Proj}
\DeclareMathOperator{\Span}{span}
\DeclareMathOperator{\Sgn}{sgn}
\DeclareMathOperator{\Area}{Area}
\DeclareMathOperator{\Volume}{Volume}

%
% A few miscellaneous things specific to this document
%
\newcommand{\crossop}[1]{\crossprod{#1}{}}

% R2 vector.
\newcommand{\VectorTwo}[2]{
\begin{bmatrix}
 {#1} \\
 {#2}
\end{bmatrix}
}

\newcommand{\VectorN}[1]{
\begin{bmatrix}
{#1}_1 \\
{#1}_2 \\
\vdots \\
{#1}_N \\
\end{bmatrix}
}

\newcommand{\DETuvij}[4]{
\begin{vmatrix}
 {#1}_{#3} & {#1}_{#4} \\
 {#2}_{#3} & {#2}_{#4}
\end{vmatrix}
}

\newcommand{\DETuvwijk}[6]{
\begin{vmatrix}
 {#1}_{#4} & {#1}_{#5} & {#1}_{#6} \\
 {#2}_{#4} & {#2}_{#5} & {#2}_{#6} \\
 {#3}_{#4} & {#3}_{#5} & {#3}_{#6}
\end{vmatrix}
}

\newcommand{\DETuvwxijkl}[8]{
\begin{vmatrix}
 {#1}_{#5} & {#1}_{#6} & {#1}_{#7} & {#1}_{#8} \\
 {#2}_{#5} & {#2}_{#6} & {#2}_{#7} & {#2}_{#8} \\
 {#3}_{#5} & {#3}_{#6} & {#3}_{#7} & {#3}_{#8} \\
 {#4}_{#5} & {#4}_{#6} & {#4}_{#7} & {#4}_{#8} \\
\end{vmatrix}
}

%\newcommand{\DETuvwxyijklm}[10]{
%\begin{vmatrix}
% {#1}_{#6} & {#1}_{#7} & {#1}_{#8} & {#1}_{#9} & {#1}_{#10} \\
% {#2}_{#6} & {#2}_{#7} & {#2}_{#8} & {#2}_{#9} & {#2}_{#10} \\
% {#3}_{#6} & {#3}_{#7} & {#3}_{#8} & {#3}_{#9} & {#3}_{#10} \\
% {#4}_{#6} & {#4}_{#7} & {#4}_{#8} & {#4}_{#9} & {#4}_{#10} \\
% {#5}_{#6} & {#5}_{#7} & {#5}_{#8} & {#5}_{#9} & {#5}_{#10}
%\end{vmatrix}
%}

% R3 vector.
\newcommand{\VectorThree}[3]{
\begin{bmatrix}
 {#1} \\
 {#2} \\
 {#3}
\end{bmatrix}
}


%<misc>
%
\newcommand{\Abs}[1]{{\left\lvert{#1}\right\rvert}}
\newcommand{\spacegrad}[0]{\boldsymbol{\nabla}}
\newcommand{\grad}[0]{\nabla}
\newcommand{\LL}[0]{\mathcal{L}}

% == \partial_{#1} {#2}
\newcommand{\PD}[2]{\frac{\partial {#2}}{\partial {#1}}}
% inline variant
\newcommand{\PDi}[2]{{\partial {#2}}/{\partial {#1}}}

\newcommand{\PDD}[3]{\frac{\partial^2 {#3}}{\partial {#1}\partial {#2}}}
%\newcommand{\PDd}[2]{\frac{\partial^2 {#2}}{{\partial{#1}}^2}}
\newcommand{\PDsq}[2]{\frac{\partial^2 {#2}}{(\partial {#1})^2}}

\newcommand{\Partial}[2]{\frac{\partial {#1}}{\partial {#2}}}
\DeclareMathOperator{\RejName}{Rej}
\newcommand{\Rej}[2]{\RejName_{#1}\left( {#2} \right)}
\newcommand{\Rm}[1]{\mathbb{R}^{#1}}
\newcommand{\Cm}[1]{\mathbb{C}^{#1}}
\newcommand{\conj}[0]{{*}}

%</misc>

% <grade selection>
%
\newcommand{\gpgrade}[2] {{\left\langle{{#1}}\right\rangle}_{#2}}

\newcommand{\gpgradezero}[1] {\gpgrade{#1}{}}
%\newcommand{\gpscalargrade}[1] {{\left\langle{{#1}}\right\rangle}}
%\newcommand{\gpgradezero}[1] {\gpgrade{#1}{0}}

%\newcommand{\gpgradeone}[1] {{\left\langle{{#1}}\right\rangle}_{1}}
\newcommand{\gpgradeone}[1] {\gpgrade{#1}{1}}

\newcommand{\gpgradetwo}[1] {\gpgrade{#1}{2}}
\newcommand{\gpgradethree}[1] {\gpgrade{#1}{3}}
\newcommand{\gpgradefour}[1] {\gpgrade{#1}{4}}
%
% </grade selection>



\newcommand{\adot}[0]{{\dot{a}}}
\newcommand{\bdot}[0]{{\dot{b}}}
% taken for centered dot:
%\newcommand{\cdot}[0]{{\dot{c}}}
%\newcommand{\ddot}[0]{{\dot{d}}}
\newcommand{\edot}[0]{{\dot{e}}}
\newcommand{\fdot}[0]{{\dot{f}}}
\newcommand{\gdot}[0]{{\dot{g}}}
\newcommand{\hdot}[0]{{\dot{h}}}
\newcommand{\idot}[0]{{\dot{i}}}
\newcommand{\jdot}[0]{{\dot{j}}}
\newcommand{\kdot}[0]{{\dot{k}}}
\newcommand{\ldot}[0]{{\dot{l}}}
\newcommand{\mdot}[0]{{\dot{m}}}
\newcommand{\ndot}[0]{{\dot{n}}}
%\newcommand{\odot}[0]{{\dot{o}}}
\newcommand{\pdot}[0]{{\dot{p}}}
\newcommand{\qdot}[0]{{\dot{q}}}
\newcommand{\rdot}[0]{{\dot{r}}}
\newcommand{\sdot}[0]{{\dot{s}}}
\newcommand{\tdot}[0]{{\dot{t}}}
\newcommand{\udot}[0]{{\dot{u}}}
\newcommand{\vdot}[0]{{\dot{v}}}
\newcommand{\wdot}[0]{{\dot{w}}}
\newcommand{\xdot}[0]{{\dot{x}}}
\newcommand{\ydot}[0]{{\dot{y}}}
\newcommand{\zdot}[0]{{\dot{z}}}
\newcommand{\addot}[0]{{\ddot{a}}}
\newcommand{\bddot}[0]{{\ddot{b}}}
\newcommand{\cddot}[0]{{\ddot{c}}}
%\newcommand{\dddot}[0]{{\ddot{d}}}
\newcommand{\eddot}[0]{{\ddot{e}}}
\newcommand{\fddot}[0]{{\ddot{f}}}
\newcommand{\gddot}[0]{{\ddot{g}}}
\newcommand{\hddot}[0]{{\ddot{h}}}
\newcommand{\iddot}[0]{{\ddot{i}}}
\newcommand{\jddot}[0]{{\ddot{j}}}
\newcommand{\kddot}[0]{{\ddot{k}}}
\newcommand{\lddot}[0]{{\ddot{l}}}
\newcommand{\mddot}[0]{{\ddot{m}}}
\newcommand{\nddot}[0]{{\ddot{n}}}
\newcommand{\oddot}[0]{{\ddot{o}}}
\newcommand{\pddot}[0]{{\ddot{p}}}
\newcommand{\qddot}[0]{{\ddot{q}}}
\newcommand{\rddot}[0]{{\ddot{r}}}
\newcommand{\sddot}[0]{{\ddot{s}}}
\newcommand{\tddot}[0]{{\ddot{t}}}
\newcommand{\uddot}[0]{{\ddot{u}}}
\newcommand{\vddot}[0]{{\ddot{v}}}
\newcommand{\wddot}[0]{{\ddot{w}}}
\newcommand{\xddot}[0]{{\ddot{x}}}
\newcommand{\yddot}[0]{{\ddot{y}}}
\newcommand{\zddot}[0]{{\ddot{z}}}

%<bold and dot greek symbols>
%

\newcommand{\Deltadot}[0]{{\dot{\Delta}}}
\newcommand{\Gammadot}[0]{{\dot{\Gamma}}}
\newcommand{\Lambdadot}[0]{{\dot{\Lambda}}}
\newcommand{\Omegadot}[0]{{\dot{\Omega}}}
\newcommand{\Phidot}[0]{{\dot{\Phi}}}
\newcommand{\Pidot}[0]{{\dot{\Pi}}}
\newcommand{\Psidot}[0]{{\dot{\Psi}}}
\newcommand{\Sigmadot}[0]{{\dot{\Sigma}}}
\newcommand{\Thetadot}[0]{{\dot{\Theta}}}
\newcommand{\Upsilondot}[0]{{\dot{\Upsilon}}}
\newcommand{\Xidot}[0]{{\dot{\Xi}}}
\newcommand{\alphadot}[0]{{\dot{\alpha}}}
\newcommand{\betadot}[0]{{\dot{\beta}}}
\newcommand{\chidot}[0]{{\dot{\chi}}}
\newcommand{\deltadot}[0]{{\dot{\delta}}}
\newcommand{\epsilondot}[0]{{\dot{\epsilon}}}
\newcommand{\etadot}[0]{{\dot{\eta}}}
\newcommand{\gammadot}[0]{{\dot{\gamma}}}
\newcommand{\kappadot}[0]{{\dot{\kappa}}}
\newcommand{\lambdadot}[0]{{\dot{\lambda}}}
\newcommand{\mudot}[0]{{\dot{\mu}}}
\newcommand{\nudot}[0]{{\dot{\nu}}}
\newcommand{\omegadot}[0]{{\dot{\omega}}}
\newcommand{\phidot}[0]{{\dot{\phi}}}
\newcommand{\pidot}[0]{{\dot{\pi}}}
\newcommand{\psidot}[0]{{\dot{\psi}}}
\newcommand{\rhodot}[0]{{\dot{\rho}}}
\newcommand{\sigmadot}[0]{{\dot{\sigma}}}
\newcommand{\taudot}[0]{{\dot{\tau}}}
\newcommand{\thetadot}[0]{{\dot{\theta}}}
\newcommand{\upsilondot}[0]{{\dot{\upsilon}}}
\newcommand{\varepsilondot}[0]{{\dot{\varepsilon}}}
\newcommand{\varphidot}[0]{{\dot{\varphi}}}
\newcommand{\varpidot}[0]{{\dot{\varpi}}}
\newcommand{\varrhodot}[0]{{\dot{\varrho}}}
\newcommand{\varsigmadot}[0]{{\dot{\varsigma}}}
\newcommand{\varthetadot}[0]{{\dot{\vartheta}}}
\newcommand{\xidot}[0]{{\dot{\xi}}}
\newcommand{\zetadot}[0]{{\dot{\zeta}}}

\newcommand{\Deltaddot}[0]{{\ddot{\Delta}}}
\newcommand{\Gammaddot}[0]{{\ddot{\Gamma}}}
\newcommand{\Lambdaddot}[0]{{\ddot{\Lambda}}}
\newcommand{\Omegaddot}[0]{{\ddot{\Omega}}}
\newcommand{\Phiddot}[0]{{\ddot{\Phi}}}
\newcommand{\Piddot}[0]{{\ddot{\Pi}}}
\newcommand{\Psiddot}[0]{{\ddot{\Psi}}}
\newcommand{\Sigmaddot}[0]{{\ddot{\Sigma}}}
\newcommand{\Thetaddot}[0]{{\ddot{\Theta}}}
\newcommand{\Upsilonddot}[0]{{\ddot{\Upsilon}}}
\newcommand{\Xiddot}[0]{{\ddot{\Xi}}}
\newcommand{\alphaddot}[0]{{\ddot{\alpha}}}
\newcommand{\betaddot}[0]{{\ddot{\beta}}}
\newcommand{\chiddot}[0]{{\ddot{\chi}}}
\newcommand{\deltaddot}[0]{{\ddot{\delta}}}
\newcommand{\epsilonddot}[0]{{\ddot{\epsilon}}}
\newcommand{\etaddot}[0]{{\ddot{\eta}}}
\newcommand{\gammaddot}[0]{{\ddot{\gamma}}}
\newcommand{\kappaddot}[0]{{\ddot{\kappa}}}
\newcommand{\lambdaddot}[0]{{\ddot{\lambda}}}
\newcommand{\muddot}[0]{{\ddot{\mu}}}
\newcommand{\nuddot}[0]{{\ddot{\nu}}}
\newcommand{\omegaddot}[0]{{\ddot{\omega}}}
\newcommand{\phiddot}[0]{{\ddot{\phi}}}
\newcommand{\piddot}[0]{{\ddot{\pi}}}
\newcommand{\psiddot}[0]{{\ddot{\psi}}}
\newcommand{\rhoddot}[0]{{\ddot{\rho}}}
\newcommand{\sigmaddot}[0]{{\ddot{\sigma}}}
\newcommand{\tauddot}[0]{{\ddot{\tau}}}
\newcommand{\thetaddot}[0]{{\ddot{\theta}}}
\newcommand{\upsilonddot}[0]{{\ddot{\upsilon}}}
\newcommand{\varepsilonddot}[0]{{\ddot{\varepsilon}}}
\newcommand{\varphiddot}[0]{{\ddot{\varphi}}}
\newcommand{\varpiddot}[0]{{\ddot{\varpi}}}
\newcommand{\varrhoddot}[0]{{\ddot{\varrho}}}
\newcommand{\varsigmaddot}[0]{{\ddot{\varsigma}}}
\newcommand{\varthetaddot}[0]{{\ddot{\vartheta}}}
\newcommand{\xiddot}[0]{{\ddot{\xi}}}
\newcommand{\zetaddot}[0]{{\ddot{\zeta}}}

\newcommand{\BDelta}[0]{\boldsymbol{\Delta}}
\newcommand{\BGamma}[0]{\boldsymbol{\Gamma}}
\newcommand{\BLambda}[0]{\boldsymbol{\Lambda}}
\newcommand{\BOmega}[0]{\boldsymbol{\Omega}}
\newcommand{\BPhi}[0]{\boldsymbol{\Phi}}
\newcommand{\BPi}[0]{\boldsymbol{\Pi}}
\newcommand{\BPsi}[0]{\boldsymbol{\Psi}}
\newcommand{\BSigma}[0]{\boldsymbol{\Sigma}}
\newcommand{\BTheta}[0]{\boldsymbol{\Theta}}
\newcommand{\BUpsilon}[0]{\boldsymbol{\Upsilon}}
\newcommand{\BXi}[0]{\boldsymbol{\Xi}}
\newcommand{\Balpha}[0]{\boldsymbol{\alpha}}
\newcommand{\Bbeta}[0]{\boldsymbol{\beta}}
\newcommand{\Bchi}[0]{\boldsymbol{\chi}}
\newcommand{\Bdelta}[0]{\boldsymbol{\delta}}
\newcommand{\Bepsilon}[0]{\boldsymbol{\epsilon}}
\newcommand{\Beta}[0]{\boldsymbol{\eta}}
\newcommand{\Bgamma}[0]{\boldsymbol{\gamma}}
\newcommand{\Bkappa}[0]{\boldsymbol{\kappa}}
\newcommand{\Blambda}[0]{\boldsymbol{\lambda}}
\newcommand{\Bmu}[0]{\boldsymbol{\mu}}
\newcommand{\Bnu}[0]{\boldsymbol{\nu}}
%\newcommand{\Bomega}[0]{\boldsymbol{\omega}}
\newcommand{\Bphi}[0]{\boldsymbol{\phi}}
\newcommand{\Bpi}[0]{\boldsymbol{\pi}}
\newcommand{\Bpsi}[0]{\boldsymbol{\psi}}
\newcommand{\Brho}[0]{\boldsymbol{\rho}}
\newcommand{\Bsigma}[0]{\boldsymbol{\sigma}}
%\newcommand{\Btau}[0]{\boldsymbol{\tau}}
%\newcommand{\Btheta}[0]{\boldsymbol{\theta}}
\newcommand{\Bupsilon}[0]{\boldsymbol{\upsilon}}
\newcommand{\Bvarepsilon}[0]{\boldsymbol{\varepsilon}}
\newcommand{\Bvarphi}[0]{\boldsymbol{\varphi}}
\newcommand{\Bvarpi}[0]{\boldsymbol{\varpi}}
\newcommand{\Bvarrho}[0]{\boldsymbol{\varrho}}
\newcommand{\Bvarsigma}[0]{\boldsymbol{\varsigma}}
\newcommand{\Bvartheta}[0]{\boldsymbol{\vartheta}}
\newcommand{\Bxi}[0]{\boldsymbol{\xi}}
\newcommand{\Bzeta}[0]{\boldsymbol{\zeta}}
%
%</bold and dot greek symbols>
%<infrequent>
%
%\newcommand{\AreaOp}[1]{\AName_{#1}}
%\newcommand{\Babs}[0]{\abs{\BB}}
%\newcommand{\Bcap}[0]{\hat{\BB}}
%\newcommand{\BrPrimeRej}[0]{\rcap(\rcap \wedge \Br')}
%\newcommand{\CA}[0]{\mathcal{A}}
%\newcommand{\Cos}[1]{\cos{\left({#1}\right)}}
%\newcommand{\Det}[1] {\abs{#1}}
%\newcommand{\Dsq}[2] {\frac {\partial^2 {#1}} {\partial {#2}^2}}
%\newcommand{\Exp}[1]{\exp{\left({#1}\right)}}
%\newcommand{\Norm}[1]{\left\lVert{#1}\right\rVert}
%\newcommand{\Sin}[1]{\sin{\left({#1}\right)}}
%\newcommand{\T}[0]{\text{T}}
%\newcommand{\VolumeOp}[1]{\VName_{#1}}
%\newcommand{\agrad}[0]{\Ba \cdot \nabla}
%\newcommand{\alphacap}[0]{\hat{\boldsymbol{\alpha}}}
%\newcommand{\Fcap}[0]{\hat{\BF}}
%\newcommand{\bithree}[0]{{\Bi}_3}
%\newcommand{\bxa}[0]{\Bx\Ba}
%\newcommand{\coordvec}[2]{
%\newcommand{\costheta}[0]{\acap \cdot \xcap}
%\newcommand{\ddt}[1]{\ddot{#1}}
%\newcommand{\ddu}[1] {\frac {d{#1}} {du}}
%\newcommand{\dsqxj}[2] {\frac {\partial^2 {#1}} {\partial {x_{#2}}^2}}
%\newcommand{\dtheta}[1]{\frac{d {#1}}{d \theta}}
%\newcommand{\dt}[1]{\dot{#1}}
%\newcommand{\dt}[1]{\frac{d {#1}}{dt}}
%\newcommand{\dxj}[2] {\frac {\partial {#1}} {\partial {x_{#2}}}}
%\newcommand{\halfPhi}[0]{\frac{\phi}{2}}
%\newcommand{\half}[0]{\inv{2}}
%\newcommand{\inv}[1]{\frac{1}{#1}}
%\newcommand{\laplacian}[0]{\nabla^2}
%\newcommand{\matrixoftx}[3]{
%\newcommand{\nrrp}[0]{\norm{\rcap \wedge \Br'}}
%\newcommand{\oiint}{\bigcirc \hspace{-1.4em} \int \hspace{-.8em} \int}
%\newcommand{\transpose}[1]{{#1}^{\text{T}}}
%\newcommand{\transpose}[1]{{{#1}^{\TextTranspose}}}
%\newcommand{\transpose}[1]{{{#1}^{\text{T}}}}
%\newcommand{\barA}[0]{\bar{A}}
%\newcommand{\qbar}[0]{\bar{q}}
%\newcommand{\qdotbar}[0]{\dot{\bar{q}}}
%
%</infrequent>





%\usepackage{listings}
\usepackage{txfonts} % for ointctr... (also appears to make "prettier" \int and \sum's)
\usepackage[bookmarks=true]{hyperref}

\usepackage{color,cite,graphicx}
   % use colour in the document, put your citations as [1-4]
   % rather than [1,2,3,4] (it looks nicer, and the extended LaTeX2e
   % graphics package.
\usepackage{latexsym,amssymb,epsf} % don't remember if these are
   % needed, but their inclusion can't do any damage


\title{ Quantum Harmonic Oscillator. }
\author{Peeter Joot \quad peeter.joot@gmail.com }
\date{ April 19, 2009.  Last Revision: $Date: 2009/04/20 04:09:27 $ }

\begin{document}

\maketitle{}
\tableofcontents
\section{ Motivation. }

In \cite{byron1992mca} (chapter II), an operator solution to the
(one dimensional) quantum
harmonic oscillator problem is presented.  Try this in a more old fashioned way,
as a comparison.

We want to solve the Schr\"{o}dinger equation for a quadratic potential

\begin{align}\label{eqn:toSolve}
-\frac{\hbar^2}{2m} \psi + \inv{2} m \omega^2 x^2 \psi = i \hbar \PD{t}{\psi}
\end{align}

\section{ Setup. }

\subsection{ Separation of variables. }

Equation \ref{eqn:toSolve} is separable, and to do so we can write

\begin{align*}
\psi(x,t) = \phi(x) T(t)
\end{align*}

and proceed to form the separated equation

\begin{align*}
-\frac{\hbar^2}{2m} \frac{\phi''}{\phi} + \inv{2} m \omega^2 x^2 = i \hbar \frac{T'}{T} = \text{constant}
\end{align*}

Writing $E$ for the constant, the and solving for the time function we have

\begin{align*}
(\ln(T))' &= -i \frac{E}{\hbar} \\
\implies \\
\ln(T) &= -i \frac{Et}{\hbar} + \ln(A)
\end{align*}

where $A$ is some constant.  This yields

\begin{align*}
T(t) = A e^{ -i E t/\hbar }
\end{align*}

What remains is now to solve the spatial wave equation

\begin{align}\label{eqn:spatial}
\phi'' + \left( \frac{m^2 \omega^2}{\hbar^2} x^2 - \frac{2m E}{\hbar^2}\right) \phi = 0
\end{align}

This doesn't really look too much like the harmonic oscillator problem of classical physics.  Let's remind ourself
what that was like before continuing.

\subsection{ Mass on a spring. }

The harmonic oscillator problem from classical physics
shows up many times, and is usually first seen when examining motion
of a mass on a spring.  There we have a restoring force that accelerates the
mass in the opposite direction from its equilibrium position

\begin{align}
m \xddot = -k x
\end{align}

This has two complex exponential solutions.  With a substitution of the test function

\begin{align*}
x = e^{i \omega t}
\end{align*}

we have

\begin{align*}
(-m \omega^2 + k) e^{i \omega t} = 0
\end{align*}

So the test function is a solution provided

\begin{align*}
\omega^2 = \frac{k}{m}
\end{align*}

That doesn't really help understanding why \ref{eqn:spatial} is labeled the Harmonic oscillator.  Let's instead put the equation into an energy form.  The work done against the spring
(potential energy to be returned when the mass is released) is

\begin{align*}
W 
&= - \int F \cdot dx \\
&= - \int -kx dx \\
&= \inv{2} k (x^2 - x_0^2)
\end{align*}

So, our Lagrangian is

\begin{align*}
\LL = \inv{2}m \xdot^2 - \inv{2} k (x^2 - x_0^2)
\end{align*}

The constant term in the potential can be dropped, since it won't contribute to the equations of motion.  Our conjugate momentum $\PDi{\xdot}{\LL}$ is just $m \xdot$, and the Hamiltonian
is therefore

\begin{align*}
H 
&= \xdot \PD{\xdot}{\LL} - \LL \\
&= m \xdot^2  - \left( \inv{2}m \xdot^2 - \inv{2} k x^2 \right) \\
&= \inv{2} m \xdot^2  + \inv{2} k x^2 \\
\end{align*}

Or

\begin{align*}
H &= \frac{p^2}{2 m } + \inv{2} m \omega^2 x^2 \\
\end{align*}

Ah.  In this form we see the structure of the QM Harmonic oscillator equation.  With the position space representation of the momentum operator $p \sim -i \hbar \PDi{x}{}$ we have
something now similar to \ref{eqn:toSolve}.

\section{ Series solution. }

\subsection{ Assuming Gaussian solutions. }

Having seen the operator solution of the QM harmonic oscillator problem, we will cheat, and use that as a starting point.  Assume that
the solution can be expressed as a scaled Gaussian as in

\begin{align}
\phi(x) &= f(x) e^{ - \alpha x^2/2 }
\end{align}

\begin{align*}
\phi'(x) &= \left( f'(x) - \alpha x f(x) \right) e^{ - \alpha x^2/2 } \\
\phi''(x)
&=
\left( f''(x) - \alpha f(x) -\alpha x f'(x) \right) e^{ - \alpha x^2/2 }
\left( f'(x) - \alpha x f(x) \right) (-\alpha x) e^{ - \alpha x^2/2 } \\
&=
\left( f''(x) - \alpha f(x) - 2 \alpha x f'(x) + \alpha^2 x^2 f(x) \right) e^{ - \alpha x^2/2 }
\end{align*}

Substitution of the scaled Gaussian test solution, and its derivatives, gives us

\begin{align*}
\left( f'' - \alpha f - 2 \alpha x f' + \alpha^2 x^2 f + \left( \frac{m^2 \omega^2}{\hbar^2} x^2 - \frac{2 m E}{\hbar^2} \right) f \right) e^{ -\alpha x^2 /2} = 0
\end{align*}

Since the exponential is never zero, this requires a zero for the differential equation

\begin{align}
f'' - 2 \alpha x f' + \left( \left(\alpha^2 + \frac{m^2 \omega^2}{\hbar^2} \right) x^2 - \left(\frac{2 m E}{\hbar^2} + \alpha \right) \right) f = 0
\end{align}

Compared to \ref{eqn:spatial}, this doesn't really appear to be much of an improvement, but let's work with it, looking for a term by term power series solution.

Before doing so, a couple helper variable substitutions appear to be in order.  Let

\begin{align*}
\beta^2 &= \alpha^2 + \frac{m^2 \omega^2}{\hbar^2}  \\
\sigma &= \frac{2 m E}{\hbar^2} + \alpha 
\end{align*}

So the new differential equation to solve is

\begin{align}
f'' - 2 \alpha x f' + \beta^2 x^2 f - \sigma f = 0
\end{align}

Assuming various power series solutions of the form

\begin{align*}
f_n(x) &= \sum_{r=0}^n a_r x^r
\end{align*}

derivatives are
\begin{align*}
f_n' &= \sum_{r=0}^n r a_r x^{r-1} \\
f_n'' &= \sum_{r=0}^n r(r-1) a_r x^{r-2}
\end{align*}

We have 
\begin{align*}
0 &= \sum_{r=0}^{n-2} (r+2)(r+1) a_{r+2} x^{r}
 - 2 \alpha \sum_{r=1}^n r a_r x^{r}
+ \beta^2
\sum_{r=0}^n a_r x^{r+2} 
- \sigma \sum_{r=0}^n a_r x^r
\end{align*}

This should be enough to figure out recurrence relations for the various constants in the polynomials.

However, before trying to acquire the recurrence relations in their most general form, an attempt at a few simple cases, looking for the lowest order
polynomial solutions explicitly, gets into trouble.

Specifically, if I try $n=0$, $n=1$, $n=2$ equating each of the polynomial coefficients to zero keeps killing all the coefficients in sequence.  What's 
gone wrong?

\subsection{ Try zeroth order polynomial scaled Gaussian explicitly. }

Somewhere above things went wrong.  How about a plain old Gaussian?  Let's substitute 

\begin{align*}
\psi = A e^{\alpha x^2/2}
\end{align*}

into 

\begin{align*}
-\frac{\hbar^2}{2m} \psi'' + \left( \inv{2}m \omega^2 x^2 - E \right) \psi = 0
\end{align*}

Dropping the constant $A$ temporarily the derivatives are
\begin{align*}
\psi' &= \alpha x e^{\alpha x^2/2} \\
\psi'' &= ( \alpha^2 x^2 + \alpha ) e^{\alpha x^2/2} \\
\end{align*}

This gives
\begin{align*}
\left( -( \alpha^2 x^2 + \alpha )\frac{\hbar^2}{2 m} + \left( \inv{2}m \omega^2 x^2 - E \right) \right) e^{\alpha x^2/2} = 0
\end{align*}

Equating $x^2$ and $x^0$ terms we have

\begin{align*}
\frac{\alpha^2 \hbar^2}{2 m} &= \inv{2}m \omega^2  \\
\frac{\alpha \hbar^2}{2 m} &= -E 
\end{align*}

Or
\begin{align*}
\alpha &= \pm \frac{ m \omega }{ \hbar }  \\
{\alpha } &= - \frac{2 m E }{\hbar^2}
\end{align*}

Since $\omega = \sqrt{k/m}$ is a given, we want $E$ in terms of $\omega$.  Picking $\alpha$ negative for positive energy, this is

\begin{align*}
E = \frac{ \omega \hbar }{ 2 } 
\end{align*}

The solution to the $T'/T$ equation is thus

\begin{align*}
T \propto e^{-i \omega t/2}
\end{align*}

So, except for the undetermined constant normalization factor, we have one full solution of the wave equation, 

\begin{align*}
\psi(x,t) = A \exp\left( - \frac{m \omega x^2 }{2 \hbar} - i \frac{\omega t}{2} \right)
\end{align*}

The normalization

\begin{align*}
1 
&= \int \psi \psi^\conj \\
&= A^2 \int \exp\left( - \frac{m \omega x^2 }{\hbar} \right) \\
&= A^2 \sqrt{ \frac{\hbar \pi}{ m \omega } }
\end{align*}

For a normalized solution
\begin{align}
\psi(x,t) = \left( 
\frac{ m \omega }{\hbar \pi}
\right)^{1/4} \exp\left( - \frac{m \omega x^2 }{2 \hbar} - i \frac{\omega t}{2} \right)
\end{align}

\subsection{ Try first order polynomial scaled Gaussian. }

Next, let's try 

\begin{align*}
\psi = (x + A) e^{-\alpha x^2/2}
\end{align*}

No coefficient for the first order nomial has been used since we will need a scale factor in the end for normalization anyways.  Taking derivatives

\begin{align*}
\psi' 
&= (1 + (x + A)(-\alpha x)) e^{-\alpha x^2/2} \\
&= (1 - \alpha A x - \alpha x^2 ) e^{-\alpha x^2/2} \\
\end{align*}

\begin{align*}
\psi'' 
&= 
\left( - 3 \alpha x - \alpha A + \alpha^2 A x^2 + \alpha^2 x^3 \right) e^{-\alpha x^2/2} \\
\end{align*}

So we have

\begin{align*}
-\frac{\hbar^2}{2m}\left( - 3 \alpha x - \alpha A + \alpha^2 A x^2 + \alpha^2 x^3 \right) 
+ \left( \inv{2}m \omega^2 x^2 - E \right) (x + A) = 0
\end{align*}

Equating either cubic or squared terms provides $\alpha$

\begin{align*}
\alpha = \pm \frac{m \omega}{\hbar}
\end{align*}

This is what we had for the plain old Gaussian as well.

Equating either the first and scalar terms gives us the energy 

\begin{align*}
E = \frac{3 \alpha \hbar^2}{2m} = \frac{3 \omega \hbar}{2}
\end{align*}

which differs from the zeroth order case by a factor of three.

It's curious that the coefficient $A$ cannot be determined.  I didn't expect it to be a free parameter.  Is the normalization enough to
fix this and any other leading factor?

With
\begin{align*}
\psi = (B x + A) \exp( - m \omega x^2/2 \hbar - 3 i \omega t / 2 )
\end{align*}

\begin{align*}
1 
&= \int \psi\psi^\conj \\
&= \int (B^2 x^2 + 2 A B x + A^2 ) e^{- m \omega x^2/ \hbar } \\
&= \int (B^2 x^2 + A^2 ) e^{- m \omega x^2/ \hbar } \\
&= A^2 \sqrt{\frac{\pi \hbar}{m \omega}} + B^2 \inv{2 \pi} \left( \frac{\pi \hbar}{ m \omega} \right)^{3/2} \\
&= \sqrt{\frac{\pi \hbar}{m \omega}} \left( A^2 + B^2 \frac{ \hbar}{ 2 m \omega} \right) \\
\end{align*}

In terms of an arbitrary constant $B$, this gives

\begin{align*}
\end{align*}

It is perhaps reasonable to pick $B=1$.  Regardless, the general solution for this first order polynomial scaled Gaussian is

\begin{align*}
\psi(x,t) = 
\left(B x \pm \sqrt{\sqrt{\frac{m \omega}{\pi \hbar}} - B^2 \frac{ \hbar}{ 2 m \omega} }\right) \exp\left( - \frac{m \omega x^2}{2 \hbar} - \frac{3 i \omega t }{ 2} \right)
\end{align*}

I expected something a bit more simple, without this extra degree of freedom.  That probably has to come from the orthonormality conditions on the
Hermite polynomials.

Now, there isn't anything here that is particularly special in these two cases, so I'd expect the error has to be a plain old algebra problem 
hiding in there somewhere.

\bibliographystyle{plainnat}
\bibliography{myrefs}

\end{document}
         % apr 19/09
%
% Copyright � 2012 Peeter Joot.  All Rights Reserved.
% Licenced as described in the file LICENSE under the root directory of this GIT repository.
%

% 
% 
%\documentclass{article}

%\usepackage{amsmath}
\usepackage{mathpazo}

%
% shorthand for bold symbols, convenient for vectors and matrices
%
\newcommand{\Ba}[0]{\mathbf{a}}
\newcommand{\Bb}[0]{\mathbf{b}}
\newcommand{\Bc}[0]{\mathbf{c}}
\newcommand{\Bd}[0]{\mathbf{d}}
\newcommand{\Be}[0]{\mathbf{e}}
\newcommand{\Bf}[0]{\mathbf{f}}
\newcommand{\Bg}[0]{\mathbf{g}}
\newcommand{\Bh}[0]{\mathbf{h}}
\newcommand{\Bi}[0]{\mathbf{i}}
\newcommand{\Bj}[0]{\mathbf{j}}
\newcommand{\Bk}[0]{\mathbf{k}}
\newcommand{\Bl}[0]{\mathbf{l}}
\newcommand{\Bm}[0]{\mathbf{m}}
\newcommand{\Bn}[0]{\mathbf{n}}
\newcommand{\Bo}[0]{\mathbf{o}}
\newcommand{\Bp}[0]{\mathbf{p}}
\newcommand{\Bq}[0]{\mathbf{q}}
\newcommand{\Br}[0]{\mathbf{r}}
\newcommand{\Bs}[0]{\mathbf{s}}
\newcommand{\Bt}[0]{\mathbf{t}}
\newcommand{\Bu}[0]{\mathbf{u}}
\newcommand{\Bv}[0]{\mathbf{v}}
\newcommand{\Bw}[0]{\mathbf{w}}
\newcommand{\Bx}[0]{\mathbf{x}}
\newcommand{\By}[0]{\mathbf{y}}
\newcommand{\Bz}[0]{\mathbf{z}}
\newcommand{\BA}[0]{\mathbf{A}}
\newcommand{\BB}[0]{\mathbf{B}}
\newcommand{\BC}[0]{\mathbf{C}}
\newcommand{\BD}[0]{\mathbf{D}}
\newcommand{\BE}[0]{\mathbf{E}}
\newcommand{\BF}[0]{\mathbf{F}}
\newcommand{\BG}[0]{\mathbf{G}}
\newcommand{\BH}[0]{\mathbf{H}}
\newcommand{\BI}[0]{\mathbf{I}}
\newcommand{\BJ}[0]{\mathbf{J}}
\newcommand{\BK}[0]{\mathbf{K}}
\newcommand{\BL}[0]{\mathbf{L}}
\newcommand{\BM}[0]{\mathbf{M}}
\newcommand{\BN}[0]{\mathbf{N}}
\newcommand{\BO}[0]{\mathbf{O}}
\newcommand{\BP}[0]{\mathbf{P}}
\newcommand{\BQ}[0]{\mathbf{Q}}
\newcommand{\BR}[0]{\mathbf{R}}
\newcommand{\BS}[0]{\mathbf{S}}
\newcommand{\BT}[0]{\mathbf{T}}
\newcommand{\BU}[0]{\mathbf{U}}
\newcommand{\BV}[0]{\mathbf{V}}
\newcommand{\BW}[0]{\mathbf{W}}
\newcommand{\BX}[0]{\mathbf{X}}
\newcommand{\BY}[0]{\mathbf{Y}}
\newcommand{\BZ}[0]{\mathbf{Z}}

\newcommand{\Bzero}[0]{\mathbf{0}}
\newcommand{\Btheta}[0]{\boldsymbol{\theta}}
\newcommand{\Btau}[0]{\boldsymbol{\tau}}
\newcommand{\Bomega}[0]{\boldsymbol{\omega}}

%
% shorthand for unit vectors
%
\newcommand{\acap}[0]{\hat{\Ba}}
\newcommand{\bcap}[0]{\hat{\Bb}}
\newcommand{\ccap}[0]{\hat{\Bc}}
\newcommand{\dcap}[0]{\hat{\Bd}}
\newcommand{\ecap}[0]{\hat{\Be}}
\newcommand{\fcap}[0]{\hat{\Bf}}
\newcommand{\gcap}[0]{\hat{\Bg}}
\newcommand{\hcap}[0]{\hat{\Bh}}
\newcommand{\icap}[0]{\hat{\Bi}}
\newcommand{\jcap}[0]{\hat{\Bj}}
\newcommand{\kcap}[0]{\hat{\Bk}}
\newcommand{\lcap}[0]{\hat{\Bl}}
\newcommand{\mcap}[0]{\hat{\Bm}}
\newcommand{\ncap}[0]{\hat{\Bn}}
\newcommand{\ocap}[0]{\hat{\Bo}}
\newcommand{\pcap}[0]{\hat{\Bp}}
\newcommand{\qcap}[0]{\hat{\Bq}}
\newcommand{\rcap}[0]{\hat{\Br}}
\newcommand{\scap}[0]{\hat{\Bs}}
\newcommand{\tcap}[0]{\hat{\Bt}}
\newcommand{\ucap}[0]{\hat{\Bu}}
\newcommand{\vcap}[0]{\hat{\Bv}}
\newcommand{\wcap}[0]{\hat{\Bw}}
\newcommand{\xcap}[0]{\hat{\Bx}}
\newcommand{\ycap}[0]{\hat{\By}}
\newcommand{\zcap}[0]{\hat{\Bz}}
\newcommand{\thetacap}[0]{\hat{\Btheta}}

%
% to write R^n and C^n in a distinguishable fashion.  Perhaps change this
% to the double lined characters upon figuring out how to do so.
%
\newcommand{\C}[1]{$\mathbb{C}^{#1}$}
\newcommand{\R}[1]{$\mathbb{R}^{#1}$}

%
% various generally useful helpers
%

% derivative of #1 wrt. #2:
\newcommand{\D}[2] {\frac {d#2} {d#1}}

\newcommand{\inv}[1]{\frac{1}{#1}}
\newcommand{\cross}[0]{\times}

\newcommand{\abs}[1]{\lvert{#1}\rvert}
\newcommand{\norm}[1]{\lVert{#1}\rVert}
\newcommand{\innerprod}[2]{\langle{#1}, {#2}\rangle}
\newcommand{\dotprod}[2]{{#1} \cdot {#2}}
\newcommand{\bdotprod}[2]{\left({#1} \cdot {#2}\right)}
\newcommand{\crossprod}[2]{{#1} \cross {#2}}
\newcommand{\tripleprod}[3]{\dotprod{\left(\crossprod{#1}{#2}\right)}{#3}}

\DeclareMathOperator{\Proj}{Proj}
\DeclareMathOperator{\Span}{span}
\DeclareMathOperator{\Sgn}{sgn}
\DeclareMathOperator{\Area}{Area}
\DeclareMathOperator{\Volume}{Volume}

%
% A few miscellaneous things specific to this document
%
\newcommand{\crossop}[1]{\crossprod{#1}{}}

% R2 vector.
\newcommand{\VectorTwo}[2]{
\begin{bmatrix}
 {#1} \\
 {#2}
\end{bmatrix}
}

\newcommand{\VectorN}[1]{
\begin{bmatrix}
{#1}_1 \\
{#1}_2 \\
\vdots \\
{#1}_N \\
\end{bmatrix}
}

\newcommand{\DETuvij}[4]{
\begin{vmatrix}
 {#1}_{#3} & {#1}_{#4} \\
 {#2}_{#3} & {#2}_{#4}
\end{vmatrix}
}

\newcommand{\DETuvwijk}[6]{
\begin{vmatrix}
 {#1}_{#4} & {#1}_{#5} & {#1}_{#6} \\
 {#2}_{#4} & {#2}_{#5} & {#2}_{#6} \\
 {#3}_{#4} & {#3}_{#5} & {#3}_{#6}
\end{vmatrix}
}

\newcommand{\DETuvwxijkl}[8]{
\begin{vmatrix}
 {#1}_{#5} & {#1}_{#6} & {#1}_{#7} & {#1}_{#8} \\
 {#2}_{#5} & {#2}_{#6} & {#2}_{#7} & {#2}_{#8} \\
 {#3}_{#5} & {#3}_{#6} & {#3}_{#7} & {#3}_{#8} \\
 {#4}_{#5} & {#4}_{#6} & {#4}_{#7} & {#4}_{#8} \\
\end{vmatrix}
}

%\newcommand{\DETuvwxyijklm}[10]{
%\begin{vmatrix}
% {#1}_{#6} & {#1}_{#7} & {#1}_{#8} & {#1}_{#9} & {#1}_{#10} \\
% {#2}_{#6} & {#2}_{#7} & {#2}_{#8} & {#2}_{#9} & {#2}_{#10} \\
% {#3}_{#6} & {#3}_{#7} & {#3}_{#8} & {#3}_{#9} & {#3}_{#10} \\
% {#4}_{#6} & {#4}_{#7} & {#4}_{#8} & {#4}_{#9} & {#4}_{#10} \\
% {#5}_{#6} & {#5}_{#7} & {#5}_{#8} & {#5}_{#9} & {#5}_{#10}
%\end{vmatrix}
%}

% R3 vector.
\newcommand{\VectorThree}[3]{
\begin{bmatrix}
 {#1} \\
 {#2} \\
 {#3}
\end{bmatrix}
}


%%<misc>
%
\newcommand{\Abs}[1]{{\left\lvert{#1}\right\rvert}}
\newcommand{\spacegrad}[0]{\boldsymbol{\nabla}}
\newcommand{\grad}[0]{\nabla}
\newcommand{\LL}[0]{\mathcal{L}}

% == \partial_{#1} {#2}
\newcommand{\PD}[2]{\frac{\partial {#2}}{\partial {#1}}}
% inline variant
\newcommand{\PDi}[2]{{\partial {#2}}/{\partial {#1}}}

\newcommand{\PDD}[3]{\frac{\partial^2 {#3}}{\partial {#1}\partial {#2}}}
%\newcommand{\PDd}[2]{\frac{\partial^2 {#2}}{{\partial{#1}}^2}}
\newcommand{\PDsq}[2]{\frac{\partial^2 {#2}}{(\partial {#1})^2}}

\newcommand{\Partial}[2]{\frac{\partial {#1}}{\partial {#2}}}
\DeclareMathOperator{\RejName}{Rej}
\newcommand{\Rej}[2]{\RejName_{#1}\left( {#2} \right)}
\newcommand{\Rm}[1]{\mathbb{R}^{#1}}
\newcommand{\Cm}[1]{\mathbb{C}^{#1}}
\newcommand{\conj}[0]{{*}}

%</misc>

% <grade selection>
%
\newcommand{\gpgrade}[2] {{\left\langle{{#1}}\right\rangle}_{#2}}

\newcommand{\gpgradezero}[1] {\gpgrade{#1}{}}
%\newcommand{\gpscalargrade}[1] {{\left\langle{{#1}}\right\rangle}}
%\newcommand{\gpgradezero}[1] {\gpgrade{#1}{0}}

%\newcommand{\gpgradeone}[1] {{\left\langle{{#1}}\right\rangle}_{1}}
\newcommand{\gpgradeone}[1] {\gpgrade{#1}{1}}

\newcommand{\gpgradetwo}[1] {\gpgrade{#1}{2}}
\newcommand{\gpgradethree}[1] {\gpgrade{#1}{3}}
\newcommand{\gpgradefour}[1] {\gpgrade{#1}{4}}
%
% </grade selection>



\newcommand{\adot}[0]{{\dot{a}}}
\newcommand{\bdot}[0]{{\dot{b}}}
% taken for centered dot:
%\newcommand{\cdot}[0]{{\dot{c}}}
%\newcommand{\ddot}[0]{{\dot{d}}}
\newcommand{\edot}[0]{{\dot{e}}}
\newcommand{\fdot}[0]{{\dot{f}}}
\newcommand{\gdot}[0]{{\dot{g}}}
\newcommand{\hdot}[0]{{\dot{h}}}
\newcommand{\idot}[0]{{\dot{i}}}
\newcommand{\jdot}[0]{{\dot{j}}}
\newcommand{\kdot}[0]{{\dot{k}}}
\newcommand{\ldot}[0]{{\dot{l}}}
\newcommand{\mdot}[0]{{\dot{m}}}
\newcommand{\ndot}[0]{{\dot{n}}}
%\newcommand{\odot}[0]{{\dot{o}}}
\newcommand{\pdot}[0]{{\dot{p}}}
\newcommand{\qdot}[0]{{\dot{q}}}
\newcommand{\rdot}[0]{{\dot{r}}}
\newcommand{\sdot}[0]{{\dot{s}}}
\newcommand{\tdot}[0]{{\dot{t}}}
\newcommand{\udot}[0]{{\dot{u}}}
\newcommand{\vdot}[0]{{\dot{v}}}
\newcommand{\wdot}[0]{{\dot{w}}}
\newcommand{\xdot}[0]{{\dot{x}}}
\newcommand{\ydot}[0]{{\dot{y}}}
\newcommand{\zdot}[0]{{\dot{z}}}
\newcommand{\addot}[0]{{\ddot{a}}}
\newcommand{\bddot}[0]{{\ddot{b}}}
\newcommand{\cddot}[0]{{\ddot{c}}}
%\newcommand{\dddot}[0]{{\ddot{d}}}
\newcommand{\eddot}[0]{{\ddot{e}}}
\newcommand{\fddot}[0]{{\ddot{f}}}
\newcommand{\gddot}[0]{{\ddot{g}}}
\newcommand{\hddot}[0]{{\ddot{h}}}
\newcommand{\iddot}[0]{{\ddot{i}}}
\newcommand{\jddot}[0]{{\ddot{j}}}
\newcommand{\kddot}[0]{{\ddot{k}}}
\newcommand{\lddot}[0]{{\ddot{l}}}
\newcommand{\mddot}[0]{{\ddot{m}}}
\newcommand{\nddot}[0]{{\ddot{n}}}
\newcommand{\oddot}[0]{{\ddot{o}}}
\newcommand{\pddot}[0]{{\ddot{p}}}
\newcommand{\qddot}[0]{{\ddot{q}}}
\newcommand{\rddot}[0]{{\ddot{r}}}
\newcommand{\sddot}[0]{{\ddot{s}}}
\newcommand{\tddot}[0]{{\ddot{t}}}
\newcommand{\uddot}[0]{{\ddot{u}}}
\newcommand{\vddot}[0]{{\ddot{v}}}
\newcommand{\wddot}[0]{{\ddot{w}}}
\newcommand{\xddot}[0]{{\ddot{x}}}
\newcommand{\yddot}[0]{{\ddot{y}}}
\newcommand{\zddot}[0]{{\ddot{z}}}

%<bold and dot greek symbols>
%

\newcommand{\Deltadot}[0]{{\dot{\Delta}}}
\newcommand{\Gammadot}[0]{{\dot{\Gamma}}}
\newcommand{\Lambdadot}[0]{{\dot{\Lambda}}}
\newcommand{\Omegadot}[0]{{\dot{\Omega}}}
\newcommand{\Phidot}[0]{{\dot{\Phi}}}
\newcommand{\Pidot}[0]{{\dot{\Pi}}}
\newcommand{\Psidot}[0]{{\dot{\Psi}}}
\newcommand{\Sigmadot}[0]{{\dot{\Sigma}}}
\newcommand{\Thetadot}[0]{{\dot{\Theta}}}
\newcommand{\Upsilondot}[0]{{\dot{\Upsilon}}}
\newcommand{\Xidot}[0]{{\dot{\Xi}}}
\newcommand{\alphadot}[0]{{\dot{\alpha}}}
\newcommand{\betadot}[0]{{\dot{\beta}}}
\newcommand{\chidot}[0]{{\dot{\chi}}}
\newcommand{\deltadot}[0]{{\dot{\delta}}}
\newcommand{\epsilondot}[0]{{\dot{\epsilon}}}
\newcommand{\etadot}[0]{{\dot{\eta}}}
\newcommand{\gammadot}[0]{{\dot{\gamma}}}
\newcommand{\kappadot}[0]{{\dot{\kappa}}}
\newcommand{\lambdadot}[0]{{\dot{\lambda}}}
\newcommand{\mudot}[0]{{\dot{\mu}}}
\newcommand{\nudot}[0]{{\dot{\nu}}}
\newcommand{\omegadot}[0]{{\dot{\omega}}}
\newcommand{\phidot}[0]{{\dot{\phi}}}
\newcommand{\pidot}[0]{{\dot{\pi}}}
\newcommand{\psidot}[0]{{\dot{\psi}}}
\newcommand{\rhodot}[0]{{\dot{\rho}}}
\newcommand{\sigmadot}[0]{{\dot{\sigma}}}
\newcommand{\taudot}[0]{{\dot{\tau}}}
\newcommand{\thetadot}[0]{{\dot{\theta}}}
\newcommand{\upsilondot}[0]{{\dot{\upsilon}}}
\newcommand{\varepsilondot}[0]{{\dot{\varepsilon}}}
\newcommand{\varphidot}[0]{{\dot{\varphi}}}
\newcommand{\varpidot}[0]{{\dot{\varpi}}}
\newcommand{\varrhodot}[0]{{\dot{\varrho}}}
\newcommand{\varsigmadot}[0]{{\dot{\varsigma}}}
\newcommand{\varthetadot}[0]{{\dot{\vartheta}}}
\newcommand{\xidot}[0]{{\dot{\xi}}}
\newcommand{\zetadot}[0]{{\dot{\zeta}}}

\newcommand{\Deltaddot}[0]{{\ddot{\Delta}}}
\newcommand{\Gammaddot}[0]{{\ddot{\Gamma}}}
\newcommand{\Lambdaddot}[0]{{\ddot{\Lambda}}}
\newcommand{\Omegaddot}[0]{{\ddot{\Omega}}}
\newcommand{\Phiddot}[0]{{\ddot{\Phi}}}
\newcommand{\Piddot}[0]{{\ddot{\Pi}}}
\newcommand{\Psiddot}[0]{{\ddot{\Psi}}}
\newcommand{\Sigmaddot}[0]{{\ddot{\Sigma}}}
\newcommand{\Thetaddot}[0]{{\ddot{\Theta}}}
\newcommand{\Upsilonddot}[0]{{\ddot{\Upsilon}}}
\newcommand{\Xiddot}[0]{{\ddot{\Xi}}}
\newcommand{\alphaddot}[0]{{\ddot{\alpha}}}
\newcommand{\betaddot}[0]{{\ddot{\beta}}}
\newcommand{\chiddot}[0]{{\ddot{\chi}}}
\newcommand{\deltaddot}[0]{{\ddot{\delta}}}
\newcommand{\epsilonddot}[0]{{\ddot{\epsilon}}}
\newcommand{\etaddot}[0]{{\ddot{\eta}}}
\newcommand{\gammaddot}[0]{{\ddot{\gamma}}}
\newcommand{\kappaddot}[0]{{\ddot{\kappa}}}
\newcommand{\lambdaddot}[0]{{\ddot{\lambda}}}
\newcommand{\muddot}[0]{{\ddot{\mu}}}
\newcommand{\nuddot}[0]{{\ddot{\nu}}}
\newcommand{\omegaddot}[0]{{\ddot{\omega}}}
\newcommand{\phiddot}[0]{{\ddot{\phi}}}
\newcommand{\piddot}[0]{{\ddot{\pi}}}
\newcommand{\psiddot}[0]{{\ddot{\psi}}}
\newcommand{\rhoddot}[0]{{\ddot{\rho}}}
\newcommand{\sigmaddot}[0]{{\ddot{\sigma}}}
\newcommand{\tauddot}[0]{{\ddot{\tau}}}
\newcommand{\thetaddot}[0]{{\ddot{\theta}}}
\newcommand{\upsilonddot}[0]{{\ddot{\upsilon}}}
\newcommand{\varepsilonddot}[0]{{\ddot{\varepsilon}}}
\newcommand{\varphiddot}[0]{{\ddot{\varphi}}}
\newcommand{\varpiddot}[0]{{\ddot{\varpi}}}
\newcommand{\varrhoddot}[0]{{\ddot{\varrho}}}
\newcommand{\varsigmaddot}[0]{{\ddot{\varsigma}}}
\newcommand{\varthetaddot}[0]{{\ddot{\vartheta}}}
\newcommand{\xiddot}[0]{{\ddot{\xi}}}
\newcommand{\zetaddot}[0]{{\ddot{\zeta}}}

\newcommand{\BDelta}[0]{\boldsymbol{\Delta}}
\newcommand{\BGamma}[0]{\boldsymbol{\Gamma}}
\newcommand{\BLambda}[0]{\boldsymbol{\Lambda}}
\newcommand{\BOmega}[0]{\boldsymbol{\Omega}}
\newcommand{\BPhi}[0]{\boldsymbol{\Phi}}
\newcommand{\BPi}[0]{\boldsymbol{\Pi}}
\newcommand{\BPsi}[0]{\boldsymbol{\Psi}}
\newcommand{\BSigma}[0]{\boldsymbol{\Sigma}}
\newcommand{\BTheta}[0]{\boldsymbol{\Theta}}
\newcommand{\BUpsilon}[0]{\boldsymbol{\Upsilon}}
\newcommand{\BXi}[0]{\boldsymbol{\Xi}}
\newcommand{\Balpha}[0]{\boldsymbol{\alpha}}
\newcommand{\Bbeta}[0]{\boldsymbol{\beta}}
\newcommand{\Bchi}[0]{\boldsymbol{\chi}}
\newcommand{\Bdelta}[0]{\boldsymbol{\delta}}
\newcommand{\Bepsilon}[0]{\boldsymbol{\epsilon}}
\newcommand{\Beta}[0]{\boldsymbol{\eta}}
\newcommand{\Bgamma}[0]{\boldsymbol{\gamma}}
\newcommand{\Bkappa}[0]{\boldsymbol{\kappa}}
\newcommand{\Blambda}[0]{\boldsymbol{\lambda}}
\newcommand{\Bmu}[0]{\boldsymbol{\mu}}
\newcommand{\Bnu}[0]{\boldsymbol{\nu}}
%\newcommand{\Bomega}[0]{\boldsymbol{\omega}}
\newcommand{\Bphi}[0]{\boldsymbol{\phi}}
\newcommand{\Bpi}[0]{\boldsymbol{\pi}}
\newcommand{\Bpsi}[0]{\boldsymbol{\psi}}
\newcommand{\Brho}[0]{\boldsymbol{\rho}}
\newcommand{\Bsigma}[0]{\boldsymbol{\sigma}}
%\newcommand{\Btau}[0]{\boldsymbol{\tau}}
%\newcommand{\Btheta}[0]{\boldsymbol{\theta}}
\newcommand{\Bupsilon}[0]{\boldsymbol{\upsilon}}
\newcommand{\Bvarepsilon}[0]{\boldsymbol{\varepsilon}}
\newcommand{\Bvarphi}[0]{\boldsymbol{\varphi}}
\newcommand{\Bvarpi}[0]{\boldsymbol{\varpi}}
\newcommand{\Bvarrho}[0]{\boldsymbol{\varrho}}
\newcommand{\Bvarsigma}[0]{\boldsymbol{\varsigma}}
\newcommand{\Bvartheta}[0]{\boldsymbol{\vartheta}}
\newcommand{\Bxi}[0]{\boldsymbol{\xi}}
\newcommand{\Bzeta}[0]{\boldsymbol{\zeta}}
%
%</bold and dot greek symbols>
%<infrequent>
%
%\newcommand{\AreaOp}[1]{\AName_{#1}}
%\newcommand{\Babs}[0]{\abs{\BB}}
%\newcommand{\Bcap}[0]{\hat{\BB}}
%\newcommand{\BrPrimeRej}[0]{\rcap(\rcap \wedge \Br')}
%\newcommand{\CA}[0]{\mathcal{A}}
%\newcommand{\Cos}[1]{\cos{\left({#1}\right)}}
%\newcommand{\Det}[1] {\abs{#1}}
%\newcommand{\Dsq}[2] {\frac {\partial^2 {#1}} {\partial {#2}^2}}
%\newcommand{\Exp}[1]{\exp{\left({#1}\right)}}
%\newcommand{\Norm}[1]{\left\lVert{#1}\right\rVert}
%\newcommand{\Sin}[1]{\sin{\left({#1}\right)}}
%\newcommand{\T}[0]{\text{T}}
%\newcommand{\VolumeOp}[1]{\VName_{#1}}
%\newcommand{\agrad}[0]{\Ba \cdot \nabla}
%\newcommand{\alphacap}[0]{\hat{\boldsymbol{\alpha}}}
%\newcommand{\Fcap}[0]{\hat{\BF}}
%\newcommand{\bithree}[0]{{\Bi}_3}
%\newcommand{\bxa}[0]{\Bx\Ba}
%\newcommand{\coordvec}[2]{
%\newcommand{\costheta}[0]{\acap \cdot \xcap}
%\newcommand{\ddt}[1]{\ddot{#1}}
%\newcommand{\ddu}[1] {\frac {d{#1}} {du}}
%\newcommand{\dsqxj}[2] {\frac {\partial^2 {#1}} {\partial {x_{#2}}^2}}
%\newcommand{\dtheta}[1]{\frac{d {#1}}{d \theta}}
%\newcommand{\dt}[1]{\dot{#1}}
%\newcommand{\dt}[1]{\frac{d {#1}}{dt}}
%\newcommand{\dxj}[2] {\frac {\partial {#1}} {\partial {x_{#2}}}}
%\newcommand{\halfPhi}[0]{\frac{\phi}{2}}
%\newcommand{\half}[0]{\inv{2}}
%\newcommand{\inv}[1]{\frac{1}{#1}}
%\newcommand{\laplacian}[0]{\nabla^2}
%\newcommand{\matrixoftx}[3]{
%\newcommand{\nrrp}[0]{\norm{\rcap \wedge \Br'}}
%\newcommand{\oiint}{\bigcirc \hspace{-1.4em} \int \hspace{-.8em} \int}
%\newcommand{\transpose}[1]{{#1}^{\text{T}}}
%\newcommand{\transpose}[1]{{{#1}^{\TextTranspose}}}
%\newcommand{\transpose}[1]{{{#1}^{\text{T}}}}
%\newcommand{\barA}[0]{\bar{A}}
%\newcommand{\qbar}[0]{\bar{q}}
%\newcommand{\qdotbar}[0]{\dot{\bar{q}}}
%
%</infrequent>





%\usepackage{listings}
%\usepackage{txfonts} % for ointctr... (also appears to make "prettier" \int and \sum's)
%\usepackage[bookmarks=true]{hyperref}

%\usepackage{color,cite,graphicx}
   % use colour in the document, put your citations as [1-4]
   % rather than [1,2,3,4] (it looks nicer, and the extended LaTeX2e
   % graphics package. 
%\usepackage{latexsym,amssymb,epsf} % do not remember if these are
   % needed, but their inclusion can not do any damage


\chapter{Bohm Chapter 10 problems}
\label{chap:bohmCh10}
%\author{Peeter Joot \quad peeter.joot@gmail.com }
\date{ April 23, 2009.  bohmCh10.tex }

%\begin{document}

%\maketitle{}
%\tableofcontents
\section{Bohm Chapter 10 problems}

Problems and additional details from reading of \citep{bohm1989qt}, chapter 10.

Differing from the text, the notation \(\expectation{O}\) has been used instead \(\overbar{O}\) mostly due to not knowing how to format the a wide overbar, and getting peculiar looking results.

\subsection{P1. Uncertainty calculations}

Calculate \(\Delta x \Delta p\) for a few wave functions

\subsubsection{Gaussian wave function}

\begin{equation}\label{eqn:bohmCh10:20}
\begin{aligned}
\psi = \alpha_1 e^{-\alpha x^2/2}
\end{aligned}
\end{equation}

Normalization

\begin{equation}\label{eqn:bohmCh10:40}
\begin{aligned}
1 
&= \Abs{\alpha_1}^2 \int e^{-\alpha x^2} dx \\
%&= \Abs{\alpha_1}^2 \sqrt{\pi/\alpha}
\end{aligned}
\end{equation}

Position expectation is zero, since it is odd:

\begin{equation}\label{eqn:bohmCh10:60}
\begin{aligned}
\expectation{x} \propto \int x e^{-\alpha x^2} dx = 0
\end{aligned}
\end{equation}

And the second power

\begin{equation}\label{eqn:bohmCh10:80}
\begin{aligned}
\expectation{x^2} 
&= \Abs{\alpha_1}^2 \int x^2 e^{-\alpha x^2} dx \\
&= \Abs{\alpha_1}^2 \int x (e^{-\alpha x^2}/-2\alpha)' dx \\
&= \Abs{\alpha_1}^2 \int (e^{-\alpha x^2}/2\alpha) dx \\
&= \inv{2\alpha} \mathLabelBox{\Abs{\alpha_1}^2 \int e^{-\alpha x^2} dx}{\(=1\)} \\
&= \inv{2\alpha}
\end{aligned}
\end{equation}

For the first momentum expectation, we have zero again since we end up with an odd integral:

\begin{equation}\label{eqn:bohmCh10:100}
\begin{aligned}
\expectation{p} 
&= -i \Hbar \Abs{\alpha_1}^2 \int e^{-\alpha x^2 /2} \frac{d}{dx} e^{-\alpha x^2 /2} dx \\
&= -i \Hbar \Abs{\alpha_1}^2 \int e^{-\alpha x^2 /2} (-\alpha x) e^{-\alpha x^2 /2} dx \\
&= 0
\end{aligned}
\end{equation}

And for the second power

\begin{equation}\label{eqn:bohmCh10:120}
\begin{aligned}
\expectation{p^2} 
&= - \Hbar^2 \Abs{\alpha_1}^2 \int e^{-\alpha x^2 /2} \frac{d}{dx} ((-\alpha x) e^{-\alpha x^2 /2}) dx \\
&= \Hbar^2 \Abs{\alpha_1}^2 \alpha \int e^{-\alpha x^2 /2} \frac{d}{dx} (x e^{-\alpha x^2 /2}) dx \\
&= -\Hbar^2 \Abs{\alpha_1}^2 \alpha \int \left(\frac{d}{dx}e^{-\alpha x^2 /2}\right) (x e^{-\alpha x^2 /2}) dx \\
&= \Hbar^2 \alpha^2 \mathLabelBox{\Abs{\alpha_1}^2 \int x^2 e^{-\alpha x^2} dx}{\(\expectation{x^2} = 1/2\alpha\)} \\
&= \Hbar^2 \alpha \inv{2}
\end{aligned}
\end{equation}

Assembling results

\begin{equation}\label{eqn:bohmCh10:140}
\begin{aligned}
\Delta x \Delta p 
&= \sqrt{\Hbar^2 \alpha \inv{2} \inv{2\alpha}} \\
&= \sqrt{\Hbar^2 \inv{4}} \\
&= \frac{\Hbar}{2}
\end{aligned}
\end{equation}

This is the expected result since equality with \(\Hbar/2\) occurs only with the Gaussian.

\subsubsection{Absolute valued exponential wave function}

\begin{equation}\label{eqn:bohmCh10:160}
\begin{aligned}
\psi &= \alpha_2 e^{-\alpha\Abs{x}}
\end{aligned}
\end{equation}

Normalization

\begin{equation}\label{eqn:bohmCh10:180}
\begin{aligned}
1 
&= 2 \Abs{\alpha_2}^2 \int_0^\infty e^{-2 \alpha{x}} dx \\
&= 2 \Abs{\alpha_2}^2 \left. \frac{e^{-2 \alpha{x}}}{-2\alpha} \right\vert_0^\infty \\
&= 2 \Abs{\alpha_2}^2 \inv{2\alpha}
\end{aligned}
\end{equation}

\begin{equation}\label{eqn:bohmCh10:200}
\begin{aligned}
\expectation{x} = 0
\end{aligned}
\end{equation}

(odd).

\begin{equation}\label{eqn:bohmCh10:220}
\begin{aligned}
\expectation{x^2} 
&= 2 \Abs{\alpha_2}^2 \int_0^\infty x^2 e^{-2\alpha x} dx \\
&= \Abs{\alpha_2}^2 \inv{2 \alpha^3}
\end{aligned}
\end{equation}

For the momentum we need derivatives

\begin{equation}\label{eqn:bohmCh10:240}
\begin{aligned}
\frac{d}{dx} e^{-\alpha \Abs{x}} 
&=
\left\{
\begin{array}{l l}
\frac{d}{dx}(e^{-\alpha x} & \quad \mbox{\(x>0\)} \\
\frac{d}{dx}(e^{\alpha x} & \quad \mbox{\(x<0\)} \\
\end{array}
\right. \\
&=
\left\{
\begin{array}{l l}
-\alpha (e^{-\alpha x} & \quad \mbox{\(x>0\)} \\
\alpha (e^{\alpha x} & \quad \mbox{\(x<0\)} \\
\end{array}
\right. \\
&=
-\alpha \sgn(x) e^{-\alpha \Abs{x}}
\end{aligned}
\end{equation}

\begin{equation}\label{eqn:bohmCh10:260}
\begin{aligned}
\expectation{p} 
&= i \Hbar \Abs{\alpha_2}^2 \alpha \int \sgn(x) e^{-2\alpha\Abs{x}} dx  \\
&= 0
\end{aligned}
\end{equation}

(odd)

\begin{equation}\label{eqn:bohmCh10:280}
\begin{aligned}
\expectation{p^2} 
&= (-i \Hbar)^2 \Abs{\alpha_2}^2 \alpha^2 \int (\sgn(x))^2 e^{-2\alpha\Abs{x}} dx  \\
&= - 2(\Hbar)^2 \Abs{\alpha_2}^2 \alpha^2 \int_0^\infty e^{-2\alpha\Abs{x}} dx  \\
&= 2(\Hbar)^2 \Abs{\alpha_2}^2 \alpha^2 \inv{2\alpha} \\
&= (\Hbar)^2 \Abs{\alpha_2}^2 \alpha \\
\end{aligned}
\end{equation}

Putting results together we have

\begin{equation}\label{eqn:bohmCh10:300}
\begin{aligned}
\expectation{p^2} \expectation{x^2} 
&= (\Hbar)^2 \alpha_2^4 \inv{2 \alpha^2} \\
&= (\Hbar)^2/2
\end{aligned}
\end{equation}

Same as in the Gaussian.

\subsubsection{Squared polynomial}

\begin{equation}\label{eqn:bohmCh10:320}
\begin{aligned}
\psi = \frac{\alpha_3}{(\alpha^2 + x^2)^2}
\end{aligned}
\end{equation}

For this one, the integrals were evaluated with Mathematica online integrator, where the contributions at \(\infty\) were scaled \(\arctan(x/\alpha)\) values.

\begin{equation}\label{eqn:bohmCh10:340}
\begin{aligned}
1 
&= \Abs{\alpha_3}^2 \int \frac{dx}{(\alpha^2 + x^2)^4} \\
&= \Abs{\alpha_3}^2 \frac{15 \pi}{48 \alpha^7}
\end{aligned}
\end{equation}

\begin{equation}\label{eqn:bohmCh10:360}
\begin{aligned}
\expectation{x} &= 0
\end{aligned}
\end{equation}

\begin{equation}\label{eqn:bohmCh10:380}
\begin{aligned}
\expectation{x^2} 
&= \Abs{\alpha_3}^2 \int \frac{x^2 dx}{(\alpha^2 + x^2)^4} \\
&= \Abs{\alpha_3}^2 \frac{3 \pi}{48 \alpha^5}
\end{aligned}
\end{equation}

\begin{equation}\label{eqn:bohmCh10:400}
\begin{aligned}
\expectation{p} 
&= -i \Hbar \Abs{\alpha_3}^2 \int \frac{dx}{(\alpha^2 + x^2)^2} \frac{d}{dx} \frac{1}{(\alpha^2 + x^2)^2}  \\
&= -i \Hbar \Abs{\alpha_3}^2 \int \frac{dx}{(\alpha^2 + x^2)^2} \frac{-4x}{(\alpha^2 + x^2)^2} \\
&= 0
\end{aligned}
\end{equation}

\begin{equation}\label{eqn:bohmCh10:420}
\begin{aligned}
\expectation{p^2} 
&= (-i \Hbar)^2 \Abs{\alpha_3}^2 \int \frac{dx}{(\alpha^2 + x^2)^2} \frac{d^2}{dx^2} \frac{1}{(\alpha^2 + x^2)^2}  \\
&= 4 \Hbar^2 \Abs{\alpha_3}^2 \int \frac{dx}{(\alpha^2 + x^2)^2} \frac{d}{dx} \frac{x}{(\alpha^2 + x^2)^3} \\
&= 4 \Hbar^2 \Abs{\alpha_3}^2 \int dx
\left( \frac{1}{(\alpha^2 + x^2)^5} \frac{x(-3)(2x)}{(\alpha^2 + x^2)^6} \right)
\\
&= 4 \Hbar^2 \Abs{\alpha_3}^2 \int dx \frac{\alpha^2 - 5x^2}{(\alpha^2 + x^2)^6} 
\\
&= 4 \Hbar^2 \Abs{\alpha_3}^2 \frac{105 \pi}{960 \alpha^9}
\\
\end{aligned}
\end{equation}

Assembling

\begin{equation}\label{eqn:bohmCh10:440}
\begin{aligned}
\expectation{p^2} \expectation{x^2} 
&= 4 \Hbar^2 \alpha_3^4 \frac{105 \pi}{960 \alpha^9} \frac{3 \pi}{48 \alpha^5} \\
&= \Hbar^2 \frac{7}{25}
\end{aligned}
\end{equation}

For
\begin{equation}\label{eqn:bohmCh10:460}
\begin{aligned}
\Delta{p} \Delta{x} 
&= \Hbar \frac{\sqrt{7}}{5} \\
&\approx 0.52 \Hbar  \\
&> \Hbar/2
\end{aligned}
\end{equation}

\subsection{P2. Correlation coefficients}

It is noted that a classical correlation coefficient for random variables \(x\), and \(p\) has the form

\begin{equation}\label{eqn:bohmCh10:480}
\begin{aligned}
C_{n,m}
&= \expectation{x^n p^m} - \expectation{x^n}\expectation{p^m}
\end{aligned}
\end{equation}

However, for operator expectation values and average of both orderings is more reasonable

\begin{equation}\label{eqn:bohmCh10:500}
\begin{aligned}
C_{n,m}
&= \inv{2} \left( 
\expectation{x^n p^m} - \expectation{x^n}\expectation{p^m} 
+ \expectation{p^m x^n} - \expectation{p^m}\expectation{x^n} 
\right) \\
&= \inv{2} \left( \expectation{x^n p^m} + \expectation{p^m x^n} \right) - \expectation{x^n}\expectation{p^m} 
\end{aligned}
\end{equation}

With the operator substitution \(p \rightarrow -i \Hbar d/dx\) this provides equation (7) in the text.

\subsubsection{first correlations}

Calculate \(C_{1,1}\), and \(C_{2,2}\) for 

\begin{equation}\label{eqn:bohmCh10:520}
\begin{aligned}
\psi = \alpha e^{-\alpha x^2/2}
\end{aligned}
\end{equation}

First the normalization and first and second order expectations.

\begin{equation}\label{eqn:bohmCh10:540}
\begin{aligned}
1 
&= \alpha^2 \int e^{-\alpha x^2} dx \\
&= \alpha^2 \sqrt{\pi/\alpha} \\
\implies \\
\alpha &= \pi^{1/3}
\end{aligned}
\end{equation}

\begin{equation}\label{eqn:bohmCh10:560}
\begin{aligned}
\expectation{x} &= 0
\end{aligned}
\end{equation}

\begin{equation}\label{eqn:bohmCh10:580}
\begin{aligned}
\expectation{p} &= 0
\end{aligned}
\end{equation}

And in particular

\begin{equation}\label{eqn:bohmCh10:600}
\begin{aligned}
\expectation{p} \expectation{x} &= 0 \\
\end{aligned}
\end{equation}

\subsubsection{second correlations}

\begin{equation}\label{eqn:bohmCh10:620}
\begin{aligned}
\expectation{x^2} 
&= \alpha^2 \int x^2 e^{-\alpha x^2} dx \\
&= \alpha^2 \int x x e^{-\alpha x^2} dx \\
&= \alpha^2 \int x (e^{-\alpha x^2}/-\alpha)' dx \\
&= \alpha^2 \int e^{-\alpha x^2}/\alpha dx \\
&= \alpha \int e^{-\alpha x^2} dx \\
&= \alpha \sqrt{\pi/\alpha} \\
&= \sqrt{\alpha \pi} \\
&= \pi^{2/3} \\
\end{aligned}
\end{equation}

\begin{equation}\label{eqn:bohmCh10:640}
\begin{aligned}
\expectation{p^2} 
&= -\alpha^2 \Hbar^2 \int e^{-\alpha x^2/2} \frac{d^2}{dx^2} e^{-\alpha x^2/2} dx \\
&= \alpha^2 \Hbar^2 \int \frac{d}{dx} e^{-\alpha x^2/2} \frac{d}{dx} e^{-\alpha x^2/2} dx \\
&= \alpha^2 \Hbar^2 \int (-\alpha x)^2 e^{-\alpha x^2} dx \\
&= \alpha^4 \Hbar^2 \int x^2 e^{-\alpha x^2} dx \\
&= \alpha^4 \Hbar^2 \int x x e^{-\alpha x^2} dx \\
&= \alpha^4 \Hbar^2 \int x (e^{-\alpha x^2}/-\alpha)' dx \\
&= \alpha^3 \Hbar^2 \int e^{-\alpha x^2} dx \\
&= \alpha^3 \Hbar^2 \inv{2\alpha} \\
&= \alpha^2 \Hbar^2  \\
&= \pi^{2/3} \Hbar^2 
\end{aligned}
\end{equation}

\begin{equation}\label{eqn:bohmCh10:660}
\begin{aligned}
\expectation{p^2} \expectation{x^2} &= \Hbar^2 \pi^{4/3}
\end{aligned}
\end{equation}

For the first terms we want

\begin{equation}\label{eqn:bohmCh10:680}
\begin{aligned}
\inv{2} \left( \expectation{x^2 p^2} + \expectation{p^2 x^2} \right)
&=
\frac{-\Hbar^2 \alpha^2}{2} \int \left( 
e^{-\alpha x^2/2} x^2 \frac{d^2}{dx^2} (e^{-\alpha x^2/2} )
+ e^{-\alpha x^2/2} \frac{d^2}{dx^2} ( x^2 e^{-\alpha x^2/2} ) \right) dx \\
&=
\frac{-\Hbar^2 \alpha^2}{2} \int \left( 
 \frac{d^2}{dx^2} ( e^{-\alpha x^2/2} x^2 ) e^{-\alpha x^2/2} 
+ e^{-\alpha x^2/2} \frac{d^2}{dx^2} ( x^2 e^{-\alpha x^2/2} ) \right) dx \\
&=
{-\Hbar^2 \alpha^2} \int e^{-\alpha x^2/2} \frac{d^2}{dx^2} ( e^{-\alpha x^2/2} x^2 ) dx \\
&=
{\Hbar^2 \alpha^2} \int \left( \frac{d}{dx} e^{-\alpha x^2/2} \right) \left( \frac{d}{dx} ( e^{-\alpha x^2/2} x^2 ) \right) dx \\
&=
{\Hbar^2 \alpha^2} \int (-\alpha x) e^{-\alpha x^2} (2x + x^2(-\alpha x)) dx \\
&=
{\Hbar^2 \alpha^3} \int x^2 e^{-\alpha x^2} (-2 + x^2 \alpha ) dx \\
&=
{-\Hbar^2 \alpha^2} \sqrt{\pi/\alpha}  \\
&=
-\Hbar^2 \pi
\end{aligned}
\end{equation}

This leaves 

\begin{equation}\label{eqn:bohmCh10:700}
\begin{aligned}
C_{2,2} &=
-\Hbar^2 \left( \pi + \pi^{2/3} \right)
\end{aligned}
\end{equation}

\subsection{P3. First correlations zero for real wave function}

Show that the first correlation coefficient is zero for any real wave function.

\begin{equation}\label{eqn:bohmCh10:720}
\begin{aligned}
C_{1,1} &= \inv{2}\left( \expectation{x p} + \expectation{p x} \right) - \expectation{x}\expectation{p}
\end{aligned}
\end{equation}

Calculate instead the equivalent problem

\begin{equation}\label{eqn:bohmCh10:740}
\begin{aligned}
2 C_{1,1}/(-i\Hbar) &= \left( \expectation{x \frac{d}{dx}} + \expectation{\frac{d}{dx} x} \right) - 2 \expectation{x}\expectation{\frac{d}{dx}}
\end{aligned}
\end{equation}

For the anti-commutator part we have

\begin{equation}\label{eqn:bohmCh10:760}
\begin{aligned}
\expectation{x \frac{d}{dx}} + \expectation{\frac{d}{dx} x} 
&=
\int \psi x \psi' + \psi (x \psi)' \\
&=
\int \psi x \psi' - \psi' x \psi \\
&= 0
\end{aligned}
\end{equation}

and for the remainder if one is zero then the sum is.  In particular

\begin{equation}\label{eqn:bohmCh10:780}
\begin{aligned}
\expectation{\frac{d}{dx}} 
&= \int_{-\infty}^\infty dx \psi \psi' \\
&= \inv{2} \int_{-\infty}^\infty dx (\psi^2)' \\
&= \inv{2} \left. \psi^2 \right\vert_{-\infty}^\infty
\end{aligned}
\end{equation}

Provided the wave function vanishes in the square at \(\pm \infty\), then we are done.

\subsection{P4}
\subsection{P5}
\subsection{P6}

The phase space text on this page is not clear to me.  Revisit after study
of phase space, Poisson brackets, and Liouville's theorem in a classical
context.

\subsection{P7. wave function for the position and momentum operators for the equality uncertainty case}

Note that in the definitions of \(\alpha\) and \(\beta\) right before equation (25) in the text, the symbols are reversed.  For consistency with condition (1)
this should be

\begin{equation}\label{eqn:bohmCh10:800}
\begin{aligned}
\beta &= (x - \overbar{x}) \\
\alpha &= (p - \overbar{p})
\end{aligned}
\end{equation}

Where condition (1) for equality in the Schwartz inequality for this 
generalized uncertainty principle is

\begin{equation}\label{eqn:bohmCh10:820}
\begin{aligned}
\alpha \psi &= C \beta \psi
\end{aligned}
\end{equation}

Putting the two together for this problem one has

\begin{equation}\label{eqn:bohmCh10:840}
\begin{aligned}
(p - \overbar{p}) \psi &= C (x - \overbar{x}) \psi \\
\implies \\
p \psi &= \overbar{p} \psi + C (x - \overbar{x}) \psi \\
\end{aligned}
\end{equation}

or
\begin{equation}\label{eqn:bohmCh10:860}
\begin{aligned}
\frac{\Hbar}{i} \frac{d\psi}{dx} &= \overbar{p} \psi + C (x - \overbar{x}) \psi \\
\end{aligned}
\end{equation}

Integrating, as was done in the \(\overbar{x} = \overbar{p} = 0\) case in the text, 
one has

\begin{equation}\label{eqn:bohmCh10:880}
\begin{aligned}
\ln \psi &= \frac{i}{\Hbar}( \overbar{p} x + C (x - \overbar{x})^2/2 ) + \ln D \\
\end{aligned}
\end{equation}

or
\begin{equation}\label{eqn:bohmCh10:900}
\begin{aligned}
\psi &= D e^{i\overbar{p} x/\Hbar} e^{ i C (x - \overbar{x})^2/2\Hbar } \\
\end{aligned}
\end{equation}

Note that this shows there is a typo in equation (26) in the text too (\(i\overbar{p} x\) needs the \(1/\Hbar\) factor).  References to \(C\) in the text preceding this should be \(C/\Hbar\) in a few cases too.

It was shown above that real wave functions have the vanishing \(C_{1,1}\) coefficient required for uncorrelated operators, and if that is the case \(i C/\Hbar\), must be a real negative constant.  Denoting that as \(-a\) as in the text one has

\begin{equation}\label{eqn:bohmCh10:920}
\begin{aligned}
\psi \propto e^{i\overbar{p} x/\Hbar} e^{ -a(x - \overbar{x})^2/2} \\
\end{aligned}
\end{equation}

\subsection{P8. Calculate uncertainty for the initially Gaussian wave function}

Wave function for the problem is

\begin{equation}\label{eqn:bohmCh10:940}
\begin{aligned}
\psi &= \alpha \exp\left( -(A - iB) \frac{x^2}{2} \right) \\
A &= \frac{(\Delta k)^2}{ 1 + \frac{\Hbar^2 t^2}{m^2}(\Delta k)^4 } \\
B &= (\Delta k)^4 \frac{\Hbar t}{m} \frac{1}{ 1 + \frac{\Hbar^2 t^2}{m^2}(\Delta k)^4 } \\
\end{aligned}
\end{equation}

The normalization is

\begin{equation}\label{eqn:bohmCh10:960}
\begin{aligned}
1 
&= \Abs{\alpha}^2 \int \exp\left( -A {x^2} \right) \\
&= \Abs{\alpha}^2 \sqrt{\frac{\pi}{A}}
\end{aligned}
\end{equation}

First moment
\begin{equation}\label{eqn:bohmCh10:980}
\begin{aligned}
\expectation{x} = 0
\end{aligned}
\end{equation}

Second moment
\begin{equation}\label{eqn:bohmCh10:1000}
\begin{aligned}
\expectation{x^2} 
&= 
\Abs{\alpha}^2 \int x^2 \exp\left( -A x^2 \right) \\
&= 
\Abs{\alpha}^2 \int x (\exp\left( -A x^2 \right)/(-2A))' \\
&= 
\Abs{\alpha}^2 \int \exp\left( -A x^2 \right)/2A \\
&= 
\Abs{\alpha}^2 \inv{2A} \sqrt{ \frac{\pi}{A}} \\
&= 
\inv{2A} \\
&=
\inv{2} \frac{ 1 + \frac{\Hbar^2 t^2}{m^2}(\Delta k)^4 }{(\Delta k)^2}
 \\
&=
\inv{2}\left(\inv{(\Delta k)^2} + \frac{\Hbar^2 t^2}{m^2}(\Delta k)^2 \right) \\
\end{aligned}
\end{equation}

For the momentum expectation

\begin{equation}\label{eqn:bohmCh10:1020}
\begin{aligned}
\expectation{p}/(-i\Hbar\Abs{\alpha}^2) 
&= \int dx
\exp\left( -(A + iB) \frac{x^2}{2} \right) 
\frac{d}{dx}
\exp\left( -(A - iB) \frac{x^2}{2} \right)  \\
&= \int dx
\exp\left( -(A + iB) \frac{x^2}{2} \right) 
(-(A -iB)x)
\exp\left( -(A - iB) \frac{x^2}{2} \right)  \\
&= 
-(A -iB)
\int x \exp\left( -A x^2 \right) dx
\\
&= 0
\end{aligned}
\end{equation}

Second moment
\begin{equation}\label{eqn:bohmCh10:1040}
\begin{aligned}
\expectation{p^2}/((-i\Hbar)^2\Abs{\alpha}^2) 
&= \int dx
\exp\left( -(A + iB) \frac{x^2}{2} \right) 
\frac{d}{dx}
(-(A -iB)x) \exp\left( -(A - iB) \frac{x^2}{2} \right)  \\
&= -\int dx
(-(A -iB)x) \exp\left( -(A - iB) \frac{x^2}{2} \right)  
\frac{d}{dx}
\exp\left( -(A + iB) \frac{x^2}{2} \right) 
\\
&= -\int dx
(-(A -iB)x) \exp\left( -(A - iB) \frac{x^2}{2} \right) 
(-(A +iB)x) \exp\left( -(A - iB) \frac{x^2}{2} \right)  \\
&= -\int x^2 dx
(A^2 + B^2) \exp\left( -(A - iB) \frac{x^2}{2} \right) 
\exp\left( -(A - iB) \frac{x^2}{2} \right)  \\
&= -\int x^2 dx (A^2 + B^2) \exp\left( -A x^2 \right) \\
&= - (A^2 + B^2) \int x (\exp\left( -A x^2 \right)/(-2A))' \\
&= -\frac{A^2 + B^2}{2A} \int \exp\left( -A x^2 \right) \\
\end{aligned}
\end{equation}

So we have
\begin{equation}\label{eqn:bohmCh10:1060}
\begin{aligned}
\expectation{p^2} &= \Hbar^2 \frac{A^2 + B^2}{2A} 
\end{aligned}
\end{equation}

But 
\begin{equation}\label{eqn:bohmCh10:1080}
\begin{aligned}
A^2 + B^2 
&= 
\inv{(1 + \frac{\Hbar^2 t^2}{m^2}(\Delta k)^4)^2} ((\Delta k)^4 + (\Delta k)^8 \left(\frac{\Hbar t}{m}\right)^2 ) \\
&= 
\inv{(1 + \frac{\Hbar^2 t^2}{m^2}(\Delta k)^4)^2} (\Delta k)^4(1 + (\Delta k)^4 \left(\frac{\Hbar t}{m}\right)^2 ) \\
&= 
\frac{(\Delta k)^4}{1 + \frac{\Hbar^2 t^2}{m^2}(\Delta k)^4} \\
&= (\Delta k)^2 A
%A &= \frac{(\Delta k)^2}{ 1 + \frac{\Hbar^2 t^2}{m^2}(\Delta k)^4 } \\
\end{aligned}
\end{equation}

So we have the constant second moment as desired

\begin{equation}\label{eqn:bohmCh10:1100}
\begin{aligned}
\expectation{p^2} &= \Hbar^2 (\Delta k)^2/2
\end{aligned}
\end{equation}

\begin{equation}\label{eqn:bohmCh10:1120}
\begin{aligned}
(\Delta x \Delta p)^2
&= \Hbar^2 \frac{(\Delta k)^2}{4} \left( \inv{(\Delta k)^2} + \frac{\Hbar^2 t^2}{m^2}(\Delta k)^2 \right) \\
&= \left(\frac{\Hbar}{2}\right)^2 \left( 1 + \frac{\Hbar^2 t^2}{m^2}(\Delta k)^4 \right) \\
\end{aligned}
\end{equation}

This matches all the expectations.  At \(t=0\) we have equality for the minimum uncertainty, and it grows as
time increases.

\subsection{P9. (first P9 of the chapter.)}

if \(\psi_A(x_1)\) and \(\psi_B(x_2)\) are independent, then the probability integral over all space is

\begin{equation}\label{eqn:bohmCh10:1140}
\begin{aligned}
\int P(x_1, x_2) dx_1 dx_2
&=
\iint \psi_A(x_1) \psi_B(x_2) dx_1 dx_2 \\
&=
\int \psi_A(x_1) dx_1 \int \psi_B(x_2) dx_2 \\
\end{aligned}
\end{equation}

So, if they are also separately normalized then so is this one.

\subsection{P9. (second P9 of the chapter.)}

Prove, ``It may be shown that if \(\psi\) is not an eigenfunction of the operator \(O\) must show some fluctuation.''

By fluctuation, I assume he means deviation of the moments, 
as in the variational differences

\begin{equation}\label{eqn:bohmCh10:1160}
\begin{aligned}
\expectation{O^n} - (\expectation{O})^n 
\end{aligned}
\end{equation}

How to prove this?  Suppose that the wave function is almost an eigenfunction
differing by a bit

\begin{equation}\label{eqn:bohmCh10:1180}
\begin{aligned}
O \psi = \lambda \psi + \epsilon
\end{aligned}
\end{equation}

Then we have for the moments

\begin{equation}\label{eqn:bohmCh10:1200}
\begin{aligned}
\expectation{O}
&= \int \psi^\conj O \psi \\
&= \int \psi^\conj ( \lambda \psi + \epsilon ) \\
&= \lambda + \int \psi^\conj \epsilon \\
\end{aligned}
\end{equation}

\begin{equation}\label{eqn:bohmCh10:1220}
\begin{aligned}
\expectation{O^2}
&= \int \psi^\conj O^2 \psi \\
&= \int \psi^\conj O \left( \lambda \psi + \epsilon \right) \\
&= \lambda \expectation{O} + \int \psi^\conj O \epsilon \\
&= \lambda \expectation{O} + \int \epsilon^\conj O \psi \\
&= \lambda \expectation{O} + \int \epsilon^\conj \left(\lambda \psi + \epsilon\right) \\
&= \left(\expectation{O} - \int \psi^\conj \epsilon \right)\expectation{O} + 
\lambda \int \epsilon^\conj \psi 
+ \int \epsilon^\conj \epsilon 
\\
&= \expectation{O}^2 - \int \psi^\conj \epsilon \expectation{O} + \left( \int \psi^\conj \epsilon - \expectation{O}\right) \int \epsilon^\conj \psi + \int \epsilon^\conj \epsilon 
\\
&= \expectation{O}^2 
- \expectation{O} \left( \int \epsilon^\conj \psi + \int \psi^\conj \epsilon \right)
+ \Abs{\int \psi^\conj \epsilon}^2
+ \int \epsilon^\conj \epsilon 
\\
\end{aligned}
\end{equation}

So we have

\begin{equation}\label{eqn:bohmCh10:1240}
\begin{aligned}
\expectation{O^2} - \expectation{O}^2 
&= 
\Abs{\int \psi^\conj \epsilon}^2 + \int \epsilon^\conj \epsilon - 2 \expectation{O} \Re \int \epsilon^\conj \psi \\
\end{aligned}
\end{equation}

There is no good reason to assume that this RHS should be zero in general, so at least for the second order moment this shows that we have the fluctuation when the wave
function is not an eigenfunction.

\subsection{P10}

\begin{equation}\label{eqn:bohmCh10:1260}
\begin{aligned}
\psi = \frac{A}{(x-x_0)^2 + (\Delta x)^2}
\end{aligned}
\end{equation}

Normalizing, picking the upper plane contour around \(i \Delta x\), we have

\begin{equation}\label{eqn:bohmCh10:1280}
\begin{aligned}
1 &= \int \psi^\conj \psi \\
&= A^2 \int_{-\infty}^\infty \frac{dx}{(x-x_0)^2 + (\Delta x)^2} \\
&= A^2 \int_{-\infty}^\infty \frac{dx}{x^2 + (\Delta x)^2} \\
&= A^2 \int_{-\infty}^\infty \frac{dx}{(x + i\Delta x)(x - i \Delta x)} \\
&= A^2 \frac{2 \pi i }{2 i\Delta x} \\
&= A^2 \frac{\pi}{\Delta x} \\
\end{aligned}
\end{equation}

So we have the desired normalization

\begin{equation}\label{eqn:bohmCh10:1300}
\begin{aligned}
A &= \sqrt{\frac{\Delta x}{\pi}} \\
\end{aligned}
\end{equation}

So how do you show that the now normalized wave function is an eigenfunction of \(x\) 

\begin{equation}\label{eqn:bohmCh10:1320}
\begin{aligned}
\psi &= \sqrt{\frac{\Delta x}{\pi}} \frac{1}{(x-x_0)^2 + (\Delta x)^2} \\
\end{aligned}
\end{equation}

Would a calculation of the expectation value for the position operator be sufficient?  That is

\begin{equation}\label{eqn:bohmCh10:1340}
\begin{aligned}
\expectation{x} 
&= A^2 \IIinf \frac{x dx }{(x-x_0)^2 + (\Delta x)^2} \\
&= A^2 \IIinf \frac{(u + x_0) du }{u^2 + (\Delta x)^2} \\
&= x_0 + A^2 \IIinf \frac{u du }{u^2 + (\Delta x)^2} \\
\end{aligned}
\end{equation}

Now \(u/(u^2 + \alpha^2)\) has antiderivative \(\ln(x^2 + \alpha^2)/2\), so the PV value of this integral for \(\Delta x \ne 0\)
is

\begin{equation}\label{eqn:bohmCh10:1360}
\begin{aligned}
PV A^2 \IIinf \frac{u du }{u^2 + (\Delta x)^2} 
PV \frac{\Delta x}{\pi} \IIinf \frac{u du }{u^2 + (\Delta x)^2} 
&= \inv{2} \lim_{R\rightarrow \infty} \frac{\Delta x}{\pi} \ln\left( \frac{R^2 + (\Delta x)^2}{(-R)^2 + (\Delta x)^2} \right) \\
&= 0
\end{aligned}
\end{equation}

So, for all \(\Delta x \ne 0\) (where that \(\PV\) integral goes messy), we have

\begin{equation}\label{eqn:bohmCh10:1380}
\begin{aligned}
\expectation{x} &= x_0 
\end{aligned}
\end{equation}

Is this sufficient to show that this wave function approaches an eigenfunction as \(\Delta x \rightarrow 0\).
I suppose that one could loosely argue that the \(A^2\) term kills off the log term at the limit of \(\Delta x = 0\) (if you are careful in the argument about how exactly \(\Delta x \rightarrow 0\) with \(R \rightarrow \infty\)).

\subsection{P11. Delta function example as limit}

\subsubsection{The problem}

Show that for

\begin{equation}\label{eqn:bohmCh10:1400}
\begin{aligned}
\delta_\epsilon(x - x_0) &= \frac{A}{(x-x_0)^2 + \epsilon^2}
\end{aligned}
\end{equation}

The limit has a delta function action

\begin{equation}\label{eqn:bohmCh10:1420}
\begin{aligned}
\delta(x - x_0) &= \lim_{\epsilon \rightarrow 0} \delta_\epsilon(x - x_0)
\end{aligned}
\end{equation}

Calculate \(A\), and explain why it is different than \(A\) in problem 10.

\subsubsection{Normalization}

First for the constant

\begin{equation}\label{eqn:bohmCh10:1440}
\begin{aligned}
1 &= \IIinf dx \delta_\epsilon(x - x_0) \\
&= A \IIinf \frac{dx}{x^2 + \epsilon^2} \\
&= A \IIinf \frac{dx}{(x + i\epsilon)(x - i \epsilon)} \\
&= A \frac{2 \pi i}{ 2 i \epsilon } \\
&= A \frac{\pi}{\epsilon } \\
\end{aligned}
\end{equation}

So we have

\begin{equation}\label{eqn:bohmCh10:1460}
\begin{aligned}
A = \frac{\epsilon }{\pi} \\
\end{aligned}
\end{equation}

This constant is necessarily different from the eigenfunction normalization, since that involved normalization in the square.  The resulting 
nascent delta function is

\begin{equation}\label{eqn:bohmCh10:1480}
\begin{aligned}
\delta_\epsilon(x - x_0) &= \frac{\epsilon }{\pi} \frac{1}{(x-x_0)^2 + \epsilon^2}
\end{aligned}
\end{equation}

\subsubsection{Delta function action}

How do we show that \(\delta_\epsilon\) behaves as a delta function in the limit?  The delta function is really defined by how it
acts on a test function in an integration operation, so let us calculate 

\begin{equation}\label{eqn:bohmCh10:1500}
\begin{aligned}
\IIinf \delta_\epsilon(x - x_0) f(x) dx 
&=
\frac{\epsilon }{\pi} \IIinf \frac{f(x) dx}{(x-x_0)^2 + \epsilon^2} \\
&=
\frac{\epsilon }{\pi} \IIinf \frac{f(u + x_0) du}{u^2 + \epsilon^2} \\
&=
\frac{\epsilon }{\pi} \IIinf \frac{f(u + x_0) du}{(u + i \epsilon)(u - i \epsilon)} \\
\end{aligned}
\end{equation}

An upper half plane contour, assuming that \(f(x_0 + i\epsilon)\) is a regular point (and that f(z) has no poles in the upper half plane) gives us

\begin{equation}\label{eqn:bohmCh10:1520}
\begin{aligned}
\IIinf \delta_\epsilon(x - x_0) f(x) dx 
&=
\frac{\epsilon }{\pi} 2 \pi i \frac{f(i \epsilon + x_0) }{2 i \epsilon} \\
&=
f(i \epsilon + x_0)
\end{aligned}
\end{equation}

So in the limit if \(f(x)\) is regular all the way down the \(x_0 + i\epsilon\) trajectory, we have

\begin{equation}\label{eqn:bohmCh10:1540}
\begin{aligned}
\lim_{\epsilon \rightarrow 0} \IIinf \delta_\epsilon(x - x_0) f(x) dx &= f(x_0)
\end{aligned}
\end{equation}

which is precisely the operational definition of the delta function.

\subsection{P12. Delta function differentiation}

Prove by successive differentiation that

\begin{equation}\label{eqn:bohmCh10:1560}
\begin{aligned}
\frac{d^n f(x)}{dx^n} 
&= \int_{-\infty}^{\infty} \delta(x - x_0) \frac{d^n f(x_0)}{dx_0} dx_0 \\
\end{aligned}
\end{equation}

Doing the integration by parts, and change of variables for the delta function derivatives:

\begin{equation}\label{eqn:bohmCh10:1580}
\begin{aligned}
\frac{d^n f(x)}{dx^n} 
&= \IIinf \frac{d^n \delta(x - x_0)}{dx^n} f(x_0) dx_0 \\
&= \IIinf \frac{d^n \delta(x-x_0)}{{dx_0}^n} (-1)^n f(x_0) dx_0 \\
%&= -(-1)^n \int_\infty^{-\infty} \frac{d^n \delta(x - x_0)}{dx_0^n} f(x_0) dx_0 \\
%&= (-1)^n \int_{-\infty}^{\infty} \frac{d^n \delta(x - x_0)}{dx_0^n} f(x_0) dx_0 \\
&= -(-1)^n \int_{-\infty}^{\infty} \frac{d^{n-1} \delta(x - x_0)}{dx_0^{n-1}} \frac{df(x_0)}{dx_0} dx_0 \\
&= (-1)^2 (-1)^n \int_{-\infty}^{\infty} \frac{d^{n-2} \delta(x - x_0)}{dx_0^{n-2}} \frac{d^2 f(x_0)}{dx_0^2} dx_0 \\
&= \cdots \\
&= (-1)^{n-1} (-1)^n \int_{-\infty}^{\infty} \frac{d^{n-(n-1)} \delta(x - x_0)}{dx_0^{n-(n-1)}} \frac{d^{n-1} f(x_0)}{dx_0^{n-1}} dx_0 \\
&= (-1)^{n-1} (-1)^n \int_{-\infty}^{\infty} \frac{d \delta(x - x_0)}{dx_0} \frac{d^{n-1} f(x_0)}{dx_0^{n-1}} dx_0 \\
&= (-1)^n (-1)^n \int_{-\infty}^{\infty} \delta(x - x_0) \frac{d^n f(x_0)}{dx_0} dx_0 \\
&= \int_{-\infty}^{\infty} \delta(x - x_0) \frac{d^n f(x_0)}{dx_0} dx_0 \\
\end{aligned}
\end{equation}

\subsection{P13. Eigenfunctions for particle in one dimensional box}

Box of side \(L\).  Eigenfunction of \(p\) are given as exponentials in the problem.  Stepping back slightly to see where these come from 
consider the operator eigenvalue statement itself.  This will in fact indirectly solve the problem

\begin{equation}\label{eqn:bohmCh10:1600}
\begin{aligned}
\lambda \psi 
&= p \psi_\lambda \\
&= \left(-i \Hbar \PD{x}{} \right) \psi_\lambda \\
\end{aligned}
\end{equation}

This can be integrated

\begin{equation}\label{eqn:bohmCh10:1620}
\begin{aligned}
(\ln \psi_\lambda)' &= i \lambda/\Hbar \\
\implies \\
\ln \psi_\lambda &= \frac{i \lambda x }{\Hbar} + \ln A \\
\implies \\
\psi_\lambda &= A \exp\left( \frac{i \lambda x }{\Hbar} \right) \\
\end{aligned}
\end{equation}

Considering two such eigenfunctions (normalization omitted) with an orthogonality requirement in the \([0,L]\) interval we have for \(\lambda \ne \mu\)

\begin{equation}\label{eqn:bohmCh10:1640}
\begin{aligned}
\int_0^L (\psi_\mu)^\conj \psi_\lambda dx 
&=
\int_0^L 
\exp\left( \frac{-i \mu x }{\Hbar} \right) \exp\left( \frac{i \lambda x }{\Hbar} \right) \\
&=
\int_0^L \exp\left( \frac{i (\lambda -\mu) x }{\Hbar} \right) \\
&=
\left. \inv{ i (\lambda -\mu)/\Hbar} \exp\left( \frac{i (\lambda -\mu) x }{\Hbar} \right) \right\vert_{0}^L
 \\
&=
\inv{ i (\lambda -\mu)/\Hbar} \left( \exp\left( \frac{i (\lambda -\mu) L }{\Hbar} \right) - 1 \right)
 \\
\end{aligned}
\end{equation}

So, for orthogonality we need \(\lambda L/\Hbar = 2\pi n_\lambda\), or more simply

\begin{equation}\label{eqn:bohmCh10:1660}
\begin{aligned}
\psi_n = \inv{\sqrt{L}} \exp\left( \frac{ 2 \pi i n x }{L } \right) \\
\end{aligned}
\end{equation}

It is interesting to see that the one dimensional particle in a box can be reduced to a first order differential equation or 
eigenvalue problem, instead of looking for solutions to the energy operator equation \((p^2/2m) \psi = E \psi\).

\subsection{P14. Fourier series representation of delta function}

TODO.

\subsection{P15}

This one has a prereq on the ch3 problems, which I did not do.  Revisit.

%\bibliographystyle{plainnat}
%\bibliography{myrefs}

%\end{document}
            % apr 23/09
\documentclass{article}

\usepackage{amsmath}
\usepackage{mathpazo}

%
% shorthand for bold symbols, convenient for vectors and matrices
%
\newcommand{\Ba}[0]{\mathbf{a}}
\newcommand{\Bb}[0]{\mathbf{b}}
\newcommand{\Bc}[0]{\mathbf{c}}
\newcommand{\Bd}[0]{\mathbf{d}}
\newcommand{\Be}[0]{\mathbf{e}}
\newcommand{\Bf}[0]{\mathbf{f}}
\newcommand{\Bg}[0]{\mathbf{g}}
\newcommand{\Bh}[0]{\mathbf{h}}
\newcommand{\Bi}[0]{\mathbf{i}}
\newcommand{\Bj}[0]{\mathbf{j}}
\newcommand{\Bk}[0]{\mathbf{k}}
\newcommand{\Bl}[0]{\mathbf{l}}
\newcommand{\Bm}[0]{\mathbf{m}}
\newcommand{\Bn}[0]{\mathbf{n}}
\newcommand{\Bo}[0]{\mathbf{o}}
\newcommand{\Bp}[0]{\mathbf{p}}
\newcommand{\Bq}[0]{\mathbf{q}}
\newcommand{\Br}[0]{\mathbf{r}}
\newcommand{\Bs}[0]{\mathbf{s}}
\newcommand{\Bt}[0]{\mathbf{t}}
\newcommand{\Bu}[0]{\mathbf{u}}
\newcommand{\Bv}[0]{\mathbf{v}}
\newcommand{\Bw}[0]{\mathbf{w}}
\newcommand{\Bx}[0]{\mathbf{x}}
\newcommand{\By}[0]{\mathbf{y}}
\newcommand{\Bz}[0]{\mathbf{z}}
\newcommand{\BA}[0]{\mathbf{A}}
\newcommand{\BB}[0]{\mathbf{B}}
\newcommand{\BC}[0]{\mathbf{C}}
\newcommand{\BD}[0]{\mathbf{D}}
\newcommand{\BE}[0]{\mathbf{E}}
\newcommand{\BF}[0]{\mathbf{F}}
\newcommand{\BG}[0]{\mathbf{G}}
\newcommand{\BH}[0]{\mathbf{H}}
\newcommand{\BI}[0]{\mathbf{I}}
\newcommand{\BJ}[0]{\mathbf{J}}
\newcommand{\BK}[0]{\mathbf{K}}
\newcommand{\BL}[0]{\mathbf{L}}
\newcommand{\BM}[0]{\mathbf{M}}
\newcommand{\BN}[0]{\mathbf{N}}
\newcommand{\BO}[0]{\mathbf{O}}
\newcommand{\BP}[0]{\mathbf{P}}
\newcommand{\BQ}[0]{\mathbf{Q}}
\newcommand{\BR}[0]{\mathbf{R}}
\newcommand{\BS}[0]{\mathbf{S}}
\newcommand{\BT}[0]{\mathbf{T}}
\newcommand{\BU}[0]{\mathbf{U}}
\newcommand{\BV}[0]{\mathbf{V}}
\newcommand{\BW}[0]{\mathbf{W}}
\newcommand{\BX}[0]{\mathbf{X}}
\newcommand{\BY}[0]{\mathbf{Y}}
\newcommand{\BZ}[0]{\mathbf{Z}}

\newcommand{\Bzero}[0]{\mathbf{0}}
\newcommand{\Btheta}[0]{\boldsymbol{\theta}}
\newcommand{\Btau}[0]{\boldsymbol{\tau}}
\newcommand{\Bomega}[0]{\boldsymbol{\omega}}

%
% shorthand for unit vectors
%
\newcommand{\acap}[0]{\hat{\Ba}}
\newcommand{\bcap}[0]{\hat{\Bb}}
\newcommand{\ccap}[0]{\hat{\Bc}}
\newcommand{\dcap}[0]{\hat{\Bd}}
\newcommand{\ecap}[0]{\hat{\Be}}
\newcommand{\fcap}[0]{\hat{\Bf}}
\newcommand{\gcap}[0]{\hat{\Bg}}
\newcommand{\hcap}[0]{\hat{\Bh}}
\newcommand{\icap}[0]{\hat{\Bi}}
\newcommand{\jcap}[0]{\hat{\Bj}}
\newcommand{\kcap}[0]{\hat{\Bk}}
\newcommand{\lcap}[0]{\hat{\Bl}}
\newcommand{\mcap}[0]{\hat{\Bm}}
\newcommand{\ncap}[0]{\hat{\Bn}}
\newcommand{\ocap}[0]{\hat{\Bo}}
\newcommand{\pcap}[0]{\hat{\Bp}}
\newcommand{\qcap}[0]{\hat{\Bq}}
\newcommand{\rcap}[0]{\hat{\Br}}
\newcommand{\scap}[0]{\hat{\Bs}}
\newcommand{\tcap}[0]{\hat{\Bt}}
\newcommand{\ucap}[0]{\hat{\Bu}}
\newcommand{\vcap}[0]{\hat{\Bv}}
\newcommand{\wcap}[0]{\hat{\Bw}}
\newcommand{\xcap}[0]{\hat{\Bx}}
\newcommand{\ycap}[0]{\hat{\By}}
\newcommand{\zcap}[0]{\hat{\Bz}}
\newcommand{\thetacap}[0]{\hat{\Btheta}}

%
% to write R^n and C^n in a distinguishable fashion.  Perhaps change this
% to the double lined characters upon figuring out how to do so.
%
\newcommand{\C}[1]{$\mathbb{C}^{#1}$}
\newcommand{\R}[1]{$\mathbb{R}^{#1}$}

%
% various generally useful helpers
%

% derivative of #1 wrt. #2:
\newcommand{\D}[2] {\frac {d#2} {d#1}}

\newcommand{\inv}[1]{\frac{1}{#1}}
\newcommand{\cross}[0]{\times}

\newcommand{\abs}[1]{\lvert{#1}\rvert}
\newcommand{\norm}[1]{\lVert{#1}\rVert}
\newcommand{\innerprod}[2]{\langle{#1}, {#2}\rangle}
\newcommand{\dotprod}[2]{{#1} \cdot {#2}}
\newcommand{\bdotprod}[2]{\left({#1} \cdot {#2}\right)}
\newcommand{\crossprod}[2]{{#1} \cross {#2}}
\newcommand{\tripleprod}[3]{\dotprod{\left(\crossprod{#1}{#2}\right)}{#3}}

\DeclareMathOperator{\Proj}{Proj}
\DeclareMathOperator{\Span}{span}
\DeclareMathOperator{\Sgn}{sgn}
\DeclareMathOperator{\Area}{Area}
\DeclareMathOperator{\Volume}{Volume}

%
% A few miscellaneous things specific to this document
%
\newcommand{\crossop}[1]{\crossprod{#1}{}}

% R2 vector.
\newcommand{\VectorTwo}[2]{
\begin{bmatrix}
 {#1} \\
 {#2}
\end{bmatrix}
}

\newcommand{\VectorN}[1]{
\begin{bmatrix}
{#1}_1 \\
{#1}_2 \\
\vdots \\
{#1}_N \\
\end{bmatrix}
}

\newcommand{\DETuvij}[4]{
\begin{vmatrix}
 {#1}_{#3} & {#1}_{#4} \\
 {#2}_{#3} & {#2}_{#4}
\end{vmatrix}
}

\newcommand{\DETuvwijk}[6]{
\begin{vmatrix}
 {#1}_{#4} & {#1}_{#5} & {#1}_{#6} \\
 {#2}_{#4} & {#2}_{#5} & {#2}_{#6} \\
 {#3}_{#4} & {#3}_{#5} & {#3}_{#6}
\end{vmatrix}
}

\newcommand{\DETuvwxijkl}[8]{
\begin{vmatrix}
 {#1}_{#5} & {#1}_{#6} & {#1}_{#7} & {#1}_{#8} \\
 {#2}_{#5} & {#2}_{#6} & {#2}_{#7} & {#2}_{#8} \\
 {#3}_{#5} & {#3}_{#6} & {#3}_{#7} & {#3}_{#8} \\
 {#4}_{#5} & {#4}_{#6} & {#4}_{#7} & {#4}_{#8} \\
\end{vmatrix}
}

%\newcommand{\DETuvwxyijklm}[10]{
%\begin{vmatrix}
% {#1}_{#6} & {#1}_{#7} & {#1}_{#8} & {#1}_{#9} & {#1}_{#10} \\
% {#2}_{#6} & {#2}_{#7} & {#2}_{#8} & {#2}_{#9} & {#2}_{#10} \\
% {#3}_{#6} & {#3}_{#7} & {#3}_{#8} & {#3}_{#9} & {#3}_{#10} \\
% {#4}_{#6} & {#4}_{#7} & {#4}_{#8} & {#4}_{#9} & {#4}_{#10} \\
% {#5}_{#6} & {#5}_{#7} & {#5}_{#8} & {#5}_{#9} & {#5}_{#10}
%\end{vmatrix}
%}

% R3 vector.
\newcommand{\VectorThree}[3]{
\begin{bmatrix}
 {#1} \\
 {#2} \\
 {#3}
\end{bmatrix}
}


%<misc>
%
\newcommand{\Abs}[1]{{\left\lvert{#1}\right\rvert}}
\newcommand{\spacegrad}[0]{\boldsymbol{\nabla}}
\newcommand{\grad}[0]{\nabla}
\newcommand{\LL}[0]{\mathcal{L}}

% == \partial_{#1} {#2}
\newcommand{\PD}[2]{\frac{\partial {#2}}{\partial {#1}}}
% inline variant
\newcommand{\PDi}[2]{{\partial {#2}}/{\partial {#1}}}

\newcommand{\PDD}[3]{\frac{\partial^2 {#3}}{\partial {#1}\partial {#2}}}
%\newcommand{\PDd}[2]{\frac{\partial^2 {#2}}{{\partial{#1}}^2}}
\newcommand{\PDsq}[2]{\frac{\partial^2 {#2}}{(\partial {#1})^2}}

\newcommand{\Partial}[2]{\frac{\partial {#1}}{\partial {#2}}}
\DeclareMathOperator{\RejName}{Rej}
\newcommand{\Rej}[2]{\RejName_{#1}\left( {#2} \right)}
\newcommand{\Rm}[1]{\mathbb{R}^{#1}}
\newcommand{\Cm}[1]{\mathbb{C}^{#1}}
\newcommand{\conj}[0]{{*}}

%</misc>

% <grade selection>
%
\newcommand{\gpgrade}[2] {{\left\langle{{#1}}\right\rangle}_{#2}}

\newcommand{\gpgradezero}[1] {\gpgrade{#1}{}}
%\newcommand{\gpscalargrade}[1] {{\left\langle{{#1}}\right\rangle}}
%\newcommand{\gpgradezero}[1] {\gpgrade{#1}{0}}

%\newcommand{\gpgradeone}[1] {{\left\langle{{#1}}\right\rangle}_{1}}
\newcommand{\gpgradeone}[1] {\gpgrade{#1}{1}}

\newcommand{\gpgradetwo}[1] {\gpgrade{#1}{2}}
\newcommand{\gpgradethree}[1] {\gpgrade{#1}{3}}
\newcommand{\gpgradefour}[1] {\gpgrade{#1}{4}}
%
% </grade selection>



\newcommand{\adot}[0]{{\dot{a}}}
\newcommand{\bdot}[0]{{\dot{b}}}
% taken for centered dot:
%\newcommand{\cdot}[0]{{\dot{c}}}
%\newcommand{\ddot}[0]{{\dot{d}}}
\newcommand{\edot}[0]{{\dot{e}}}
\newcommand{\fdot}[0]{{\dot{f}}}
\newcommand{\gdot}[0]{{\dot{g}}}
\newcommand{\hdot}[0]{{\dot{h}}}
\newcommand{\idot}[0]{{\dot{i}}}
\newcommand{\jdot}[0]{{\dot{j}}}
\newcommand{\kdot}[0]{{\dot{k}}}
\newcommand{\ldot}[0]{{\dot{l}}}
\newcommand{\mdot}[0]{{\dot{m}}}
\newcommand{\ndot}[0]{{\dot{n}}}
%\newcommand{\odot}[0]{{\dot{o}}}
\newcommand{\pdot}[0]{{\dot{p}}}
\newcommand{\qdot}[0]{{\dot{q}}}
\newcommand{\rdot}[0]{{\dot{r}}}
\newcommand{\sdot}[0]{{\dot{s}}}
\newcommand{\tdot}[0]{{\dot{t}}}
\newcommand{\udot}[0]{{\dot{u}}}
\newcommand{\vdot}[0]{{\dot{v}}}
\newcommand{\wdot}[0]{{\dot{w}}}
\newcommand{\xdot}[0]{{\dot{x}}}
\newcommand{\ydot}[0]{{\dot{y}}}
\newcommand{\zdot}[0]{{\dot{z}}}
\newcommand{\addot}[0]{{\ddot{a}}}
\newcommand{\bddot}[0]{{\ddot{b}}}
\newcommand{\cddot}[0]{{\ddot{c}}}
%\newcommand{\dddot}[0]{{\ddot{d}}}
\newcommand{\eddot}[0]{{\ddot{e}}}
\newcommand{\fddot}[0]{{\ddot{f}}}
\newcommand{\gddot}[0]{{\ddot{g}}}
\newcommand{\hddot}[0]{{\ddot{h}}}
\newcommand{\iddot}[0]{{\ddot{i}}}
\newcommand{\jddot}[0]{{\ddot{j}}}
\newcommand{\kddot}[0]{{\ddot{k}}}
\newcommand{\lddot}[0]{{\ddot{l}}}
\newcommand{\mddot}[0]{{\ddot{m}}}
\newcommand{\nddot}[0]{{\ddot{n}}}
\newcommand{\oddot}[0]{{\ddot{o}}}
\newcommand{\pddot}[0]{{\ddot{p}}}
\newcommand{\qddot}[0]{{\ddot{q}}}
\newcommand{\rddot}[0]{{\ddot{r}}}
\newcommand{\sddot}[0]{{\ddot{s}}}
\newcommand{\tddot}[0]{{\ddot{t}}}
\newcommand{\uddot}[0]{{\ddot{u}}}
\newcommand{\vddot}[0]{{\ddot{v}}}
\newcommand{\wddot}[0]{{\ddot{w}}}
\newcommand{\xddot}[0]{{\ddot{x}}}
\newcommand{\yddot}[0]{{\ddot{y}}}
\newcommand{\zddot}[0]{{\ddot{z}}}

%<bold and dot greek symbols>
%

\newcommand{\Deltadot}[0]{{\dot{\Delta}}}
\newcommand{\Gammadot}[0]{{\dot{\Gamma}}}
\newcommand{\Lambdadot}[0]{{\dot{\Lambda}}}
\newcommand{\Omegadot}[0]{{\dot{\Omega}}}
\newcommand{\Phidot}[0]{{\dot{\Phi}}}
\newcommand{\Pidot}[0]{{\dot{\Pi}}}
\newcommand{\Psidot}[0]{{\dot{\Psi}}}
\newcommand{\Sigmadot}[0]{{\dot{\Sigma}}}
\newcommand{\Thetadot}[0]{{\dot{\Theta}}}
\newcommand{\Upsilondot}[0]{{\dot{\Upsilon}}}
\newcommand{\Xidot}[0]{{\dot{\Xi}}}
\newcommand{\alphadot}[0]{{\dot{\alpha}}}
\newcommand{\betadot}[0]{{\dot{\beta}}}
\newcommand{\chidot}[0]{{\dot{\chi}}}
\newcommand{\deltadot}[0]{{\dot{\delta}}}
\newcommand{\epsilondot}[0]{{\dot{\epsilon}}}
\newcommand{\etadot}[0]{{\dot{\eta}}}
\newcommand{\gammadot}[0]{{\dot{\gamma}}}
\newcommand{\kappadot}[0]{{\dot{\kappa}}}
\newcommand{\lambdadot}[0]{{\dot{\lambda}}}
\newcommand{\mudot}[0]{{\dot{\mu}}}
\newcommand{\nudot}[0]{{\dot{\nu}}}
\newcommand{\omegadot}[0]{{\dot{\omega}}}
\newcommand{\phidot}[0]{{\dot{\phi}}}
\newcommand{\pidot}[0]{{\dot{\pi}}}
\newcommand{\psidot}[0]{{\dot{\psi}}}
\newcommand{\rhodot}[0]{{\dot{\rho}}}
\newcommand{\sigmadot}[0]{{\dot{\sigma}}}
\newcommand{\taudot}[0]{{\dot{\tau}}}
\newcommand{\thetadot}[0]{{\dot{\theta}}}
\newcommand{\upsilondot}[0]{{\dot{\upsilon}}}
\newcommand{\varepsilondot}[0]{{\dot{\varepsilon}}}
\newcommand{\varphidot}[0]{{\dot{\varphi}}}
\newcommand{\varpidot}[0]{{\dot{\varpi}}}
\newcommand{\varrhodot}[0]{{\dot{\varrho}}}
\newcommand{\varsigmadot}[0]{{\dot{\varsigma}}}
\newcommand{\varthetadot}[0]{{\dot{\vartheta}}}
\newcommand{\xidot}[0]{{\dot{\xi}}}
\newcommand{\zetadot}[0]{{\dot{\zeta}}}

\newcommand{\Deltaddot}[0]{{\ddot{\Delta}}}
\newcommand{\Gammaddot}[0]{{\ddot{\Gamma}}}
\newcommand{\Lambdaddot}[0]{{\ddot{\Lambda}}}
\newcommand{\Omegaddot}[0]{{\ddot{\Omega}}}
\newcommand{\Phiddot}[0]{{\ddot{\Phi}}}
\newcommand{\Piddot}[0]{{\ddot{\Pi}}}
\newcommand{\Psiddot}[0]{{\ddot{\Psi}}}
\newcommand{\Sigmaddot}[0]{{\ddot{\Sigma}}}
\newcommand{\Thetaddot}[0]{{\ddot{\Theta}}}
\newcommand{\Upsilonddot}[0]{{\ddot{\Upsilon}}}
\newcommand{\Xiddot}[0]{{\ddot{\Xi}}}
\newcommand{\alphaddot}[0]{{\ddot{\alpha}}}
\newcommand{\betaddot}[0]{{\ddot{\beta}}}
\newcommand{\chiddot}[0]{{\ddot{\chi}}}
\newcommand{\deltaddot}[0]{{\ddot{\delta}}}
\newcommand{\epsilonddot}[0]{{\ddot{\epsilon}}}
\newcommand{\etaddot}[0]{{\ddot{\eta}}}
\newcommand{\gammaddot}[0]{{\ddot{\gamma}}}
\newcommand{\kappaddot}[0]{{\ddot{\kappa}}}
\newcommand{\lambdaddot}[0]{{\ddot{\lambda}}}
\newcommand{\muddot}[0]{{\ddot{\mu}}}
\newcommand{\nuddot}[0]{{\ddot{\nu}}}
\newcommand{\omegaddot}[0]{{\ddot{\omega}}}
\newcommand{\phiddot}[0]{{\ddot{\phi}}}
\newcommand{\piddot}[0]{{\ddot{\pi}}}
\newcommand{\psiddot}[0]{{\ddot{\psi}}}
\newcommand{\rhoddot}[0]{{\ddot{\rho}}}
\newcommand{\sigmaddot}[0]{{\ddot{\sigma}}}
\newcommand{\tauddot}[0]{{\ddot{\tau}}}
\newcommand{\thetaddot}[0]{{\ddot{\theta}}}
\newcommand{\upsilonddot}[0]{{\ddot{\upsilon}}}
\newcommand{\varepsilonddot}[0]{{\ddot{\varepsilon}}}
\newcommand{\varphiddot}[0]{{\ddot{\varphi}}}
\newcommand{\varpiddot}[0]{{\ddot{\varpi}}}
\newcommand{\varrhoddot}[0]{{\ddot{\varrho}}}
\newcommand{\varsigmaddot}[0]{{\ddot{\varsigma}}}
\newcommand{\varthetaddot}[0]{{\ddot{\vartheta}}}
\newcommand{\xiddot}[0]{{\ddot{\xi}}}
\newcommand{\zetaddot}[0]{{\ddot{\zeta}}}

\newcommand{\BDelta}[0]{\boldsymbol{\Delta}}
\newcommand{\BGamma}[0]{\boldsymbol{\Gamma}}
\newcommand{\BLambda}[0]{\boldsymbol{\Lambda}}
\newcommand{\BOmega}[0]{\boldsymbol{\Omega}}
\newcommand{\BPhi}[0]{\boldsymbol{\Phi}}
\newcommand{\BPi}[0]{\boldsymbol{\Pi}}
\newcommand{\BPsi}[0]{\boldsymbol{\Psi}}
\newcommand{\BSigma}[0]{\boldsymbol{\Sigma}}
\newcommand{\BTheta}[0]{\boldsymbol{\Theta}}
\newcommand{\BUpsilon}[0]{\boldsymbol{\Upsilon}}
\newcommand{\BXi}[0]{\boldsymbol{\Xi}}
\newcommand{\Balpha}[0]{\boldsymbol{\alpha}}
\newcommand{\Bbeta}[0]{\boldsymbol{\beta}}
\newcommand{\Bchi}[0]{\boldsymbol{\chi}}
\newcommand{\Bdelta}[0]{\boldsymbol{\delta}}
\newcommand{\Bepsilon}[0]{\boldsymbol{\epsilon}}
\newcommand{\Beta}[0]{\boldsymbol{\eta}}
\newcommand{\Bgamma}[0]{\boldsymbol{\gamma}}
\newcommand{\Bkappa}[0]{\boldsymbol{\kappa}}
\newcommand{\Blambda}[0]{\boldsymbol{\lambda}}
\newcommand{\Bmu}[0]{\boldsymbol{\mu}}
\newcommand{\Bnu}[0]{\boldsymbol{\nu}}
%\newcommand{\Bomega}[0]{\boldsymbol{\omega}}
\newcommand{\Bphi}[0]{\boldsymbol{\phi}}
\newcommand{\Bpi}[0]{\boldsymbol{\pi}}
\newcommand{\Bpsi}[0]{\boldsymbol{\psi}}
\newcommand{\Brho}[0]{\boldsymbol{\rho}}
\newcommand{\Bsigma}[0]{\boldsymbol{\sigma}}
%\newcommand{\Btau}[0]{\boldsymbol{\tau}}
%\newcommand{\Btheta}[0]{\boldsymbol{\theta}}
\newcommand{\Bupsilon}[0]{\boldsymbol{\upsilon}}
\newcommand{\Bvarepsilon}[0]{\boldsymbol{\varepsilon}}
\newcommand{\Bvarphi}[0]{\boldsymbol{\varphi}}
\newcommand{\Bvarpi}[0]{\boldsymbol{\varpi}}
\newcommand{\Bvarrho}[0]{\boldsymbol{\varrho}}
\newcommand{\Bvarsigma}[0]{\boldsymbol{\varsigma}}
\newcommand{\Bvartheta}[0]{\boldsymbol{\vartheta}}
\newcommand{\Bxi}[0]{\boldsymbol{\xi}}
\newcommand{\Bzeta}[0]{\boldsymbol{\zeta}}
%
%</bold and dot greek symbols>
%<infrequent>
%
%\newcommand{\AreaOp}[1]{\AName_{#1}}
%\newcommand{\Babs}[0]{\abs{\BB}}
%\newcommand{\Bcap}[0]{\hat{\BB}}
%\newcommand{\BrPrimeRej}[0]{\rcap(\rcap \wedge \Br')}
%\newcommand{\CA}[0]{\mathcal{A}}
%\newcommand{\Cos}[1]{\cos{\left({#1}\right)}}
%\newcommand{\Det}[1] {\abs{#1}}
%\newcommand{\Dsq}[2] {\frac {\partial^2 {#1}} {\partial {#2}^2}}
%\newcommand{\Exp}[1]{\exp{\left({#1}\right)}}
%\newcommand{\Norm}[1]{\left\lVert{#1}\right\rVert}
%\newcommand{\Sin}[1]{\sin{\left({#1}\right)}}
%\newcommand{\T}[0]{\text{T}}
%\newcommand{\VolumeOp}[1]{\VName_{#1}}
%\newcommand{\agrad}[0]{\Ba \cdot \nabla}
%\newcommand{\alphacap}[0]{\hat{\boldsymbol{\alpha}}}
%\newcommand{\Fcap}[0]{\hat{\BF}}
%\newcommand{\bithree}[0]{{\Bi}_3}
%\newcommand{\bxa}[0]{\Bx\Ba}
%\newcommand{\coordvec}[2]{
%\newcommand{\costheta}[0]{\acap \cdot \xcap}
%\newcommand{\ddt}[1]{\ddot{#1}}
%\newcommand{\ddu}[1] {\frac {d{#1}} {du}}
%\newcommand{\dsqxj}[2] {\frac {\partial^2 {#1}} {\partial {x_{#2}}^2}}
%\newcommand{\dtheta}[1]{\frac{d {#1}}{d \theta}}
%\newcommand{\dt}[1]{\dot{#1}}
%\newcommand{\dt}[1]{\frac{d {#1}}{dt}}
%\newcommand{\dxj}[2] {\frac {\partial {#1}} {\partial {x_{#2}}}}
%\newcommand{\halfPhi}[0]{\frac{\phi}{2}}
%\newcommand{\half}[0]{\inv{2}}
%\newcommand{\inv}[1]{\frac{1}{#1}}
%\newcommand{\laplacian}[0]{\nabla^2}
%\newcommand{\matrixoftx}[3]{
%\newcommand{\nrrp}[0]{\norm{\rcap \wedge \Br'}}
%\newcommand{\oiint}{\bigcirc \hspace{-1.4em} \int \hspace{-.8em} \int}
%\newcommand{\transpose}[1]{{#1}^{\text{T}}}
%\newcommand{\transpose}[1]{{{#1}^{\TextTranspose}}}
%\newcommand{\transpose}[1]{{{#1}^{\text{T}}}}
%\newcommand{\barA}[0]{\bar{A}}
%\newcommand{\qbar}[0]{\bar{q}}
%\newcommand{\qdotbar}[0]{\dot{\bar{q}}}
%
%</infrequent>





\DeclareMathOperator{\sgn}{sgn}
\newcommand{\PDSq}[2]{\frac{\partial^2 {#2}}{\partial {#1}^2}}
\newcommand{\PDN}[3]{\frac{\partial^{#3} {#2}}{\partial {#1}^{#3}}}
\DeclareMathOperator{\sinc}{sinc}
\DeclareMathOperator{\PV}{PV}
\newcommand{\FF}[0]{\mathcal{F}}
\newcommand{\Sw}[0]{\mathcal{S}}
\newcommand{\IIinf}[0]{ \int_{-\infty}^\infty }
\newcommand{\FM}[0]{\inv{\sqrt{2\pi\hbar}}}
\newcommand{\expectation}[1]{\langle{#1}\rangle}

%\usepackage{listings}
%\usepackage{txfonts} % for ointctr... (also appears to make "prettier" \int and \sum's)
% makes \grad look funny though (almost like spacegrad, but narrower)
\usepackage[bookmarks=true]{hyperref}

\usepackage{color,cite,graphicx}
   % use colour in the document, put your citations as [1-4]
   % rather than [1,2,3,4] (it looks nicer, and the extended LaTeX2e
   % graphics package. 
\usepackage{latexsym,amssymb,epsf} % don't remember if these are
   % needed, but their inclusion can't do any damage


\title{ QM notes and problems for Bohm, chapter 11. }
\author{Peeter Joot \quad peeter.joot@gmail.com }
\date{ May 8, 2009.  Last Revision: $Date: 2009/05/09 04:33:54 $ }

\begin{document}

\maketitle{}
\tableofcontents
\section{ Bohm Chapter 11 problems. }

Problems and additional details from reading of \cite{bohm1989qt}, chapter 11.

\subsection{ Problem 1.  Probability currents for step potential. }

This problem and the associated text has a step potential $V$ for $x>0$.  The 
idea is that we have a left to right stream of particles with an associated
wave function, with reflected and transmitted coefficients.

Solutions to the stationary equation are sought in each of the intervals

\begin{align*}
-\frac{\hbar^2}{2m}\psi'' + (V-E)\psi = 0
\end{align*}

That is
\begin{align*}
\psi'' = - \frac{2m}{\hbar^2} (E-V)\psi 
\end{align*}

With an assumption of exponential solutions on the left of the barrier
(ie: no decay and associated hyperbolic solutions in the $V=0$ interval),
we must have $E>0$.

So, the solution can be written as the sum

\begin{align*}
\psi_1 = \sum A_{\pm} \exp\left( \pm i \sqrt{2mE} x / \hbar \right)
\end{align*}

Similarily for $x>0$, non-hyperbolic solutions are

\begin{align*}
\psi_2 = \sum B_{\pm} \exp\left( \pm i \sqrt{2m(E-V)} x / \hbar \right)
\end{align*}

Mathematically, there isn't anything that prevents picking $E-V <0$ solutions

\begin{align*}
\psi_2 = \sum B_{\pm}' \exp\left( \pm \sqrt{-2m(E-V)} x / \hbar \right)
\end{align*}

TODO: Come back to this and think it through.  Where will follow through using
wave function continuity and derivative continuity with these solutions
in the $x>0$ interval?

Continuity at $x=0$ ($\psi_1(0) = \psi_2(0)$) requires

\begin{align*}
A_{+} + A_{-} = B_{+} + B_{-}
\end{align*}

Whereas derivative continuity, with $p_1 = \sqrt{2mE}$, and $p_2 = \sqrt{2m (E-V)}$ requires

\begin{align*}
\frac{i p_1}{\hbar} (A_{+} - A_{-}) = \frac{i p_2}{\hbar} (B_{+} - B_{-}) 
\end{align*}

This is

\begin{align*}
\begin{bmatrix}
1 & 1 \\
p_1 & -p_1
\end{bmatrix}
\begin{bmatrix}
A_{+} \\
A_{-} 
\end{bmatrix}
=
\begin{bmatrix}
1 & 1 \\
p_2 & -p_2
\end{bmatrix}
\begin{bmatrix}
B_{+} \\
B_{-} 
\end{bmatrix}
\end{align*}

Matrix inversion gives us the $B$ coefficients in terms of $A$

\begin{align*}
\begin{bmatrix}
B_{+} \\
B_{-} 
\end{bmatrix}
&=
\inv{-2p_2}
\begin{bmatrix}
-p_2 & -1 \\
-p_2 & 1
\end{bmatrix}
\begin{bmatrix}
1 & 1 \\
p_1 & -p_1
\end{bmatrix}
\begin{bmatrix}
A_{+} \\
A_{-} 
\end{bmatrix} \\
&=
\inv{2p_2}
\begin{bmatrix}
p_2 & 1 \\
p_2 & -1
\end{bmatrix}
\begin{bmatrix}
1 & 1 \\
p_1 & -p_1
\end{bmatrix}
\begin{bmatrix}
A_{+} \\
A_{-} 
\end{bmatrix} \\
&=
\inv{2p_2}
\begin{bmatrix}
(p_1 + p_2) & (p_2 - p_1) \\
(p_2 - p_1) & (p_1 + p_2)
\end{bmatrix}
\begin{bmatrix}
A_{+} \\
A_{-} 
\end{bmatrix} \\
\end{align*}

Or equivalently

\begin{align*}
\begin{bmatrix}
A_{+} \\
A_{-} 
\end{bmatrix}
&=
\inv{2p_1}
\begin{bmatrix}
(p_1 + p_2) & (p_1 - p_2) \\
(p_1 - p_2) & (p_1 + p_2)
\end{bmatrix}
\begin{bmatrix}
B_{+} \\
B_{-} 
\end{bmatrix} \\
\end{align*}

Using this last, the 
assumption of no further barriers in the $x>0$ interval (ie: no reflection at 
$x = \infty$), allows the physical situation to dictate $B_{-} = 0$.  Then we
have

\begin{align*}
\begin{bmatrix}
A_{+} \\
A_{-} 
\end{bmatrix}
&=
\frac{B_{+} }{2p_1}
\begin{bmatrix}
p_1 + p_2 \\
p_1 - p_2 
\end{bmatrix}
\end{align*}

The first is 
\begin{align*}
A_{+} 
&=
\frac{B_{+} }{2p_1} (p_1 + p_2)
\end{align*}

Or
\begin{align*}
B_{+} 
&=
\frac{2 p_1 A_{+} }{p_1 + p_2}
\end{align*}

and the second is
\begin{align*}
A_{-} 
&=
\frac{B_{+} }{2p_1} (p_1 - p_2) \\
&=
\frac{2 p_1 A_{+} }{p_1 + p_2}
\frac{1}{2p_1} (p_1 - p_2) \\
&=
A_{+} \left(
\frac{ p_1 - p_2 }{p_1 + p_2} \right)
\end{align*}

This reduces the free parameters in the wave functions to the single 
amplitude

\begin{align*}
\psi_1 &= 
A_{+} e^{ i p_1 x/\hbar }
+A_{+} \left(
\frac{ p_1 - p_2 }{p_1 + p_2} \right)
e^{ -i p_1 x/\hbar } \\
\psi_2 &= A_{+} \frac{2 p_1 }{p_1 + p_2} e^{ i p_2 x/\hbar }
\end{align*}

Inspection provides a check that these do in fact satisfy the desired
continuity requirements.


\bibliographystyle{plainnat}
\bibliography{myrefs}

\end{document}
              % may 8/09
\documentclass{article}

\usepackage{amsmath}
\usepackage{mathpazo}

%
% shorthand for bold symbols, convenient for vectors and matrices
%
\newcommand{\Ba}[0]{\mathbf{a}}
\newcommand{\Bb}[0]{\mathbf{b}}
\newcommand{\Bc}[0]{\mathbf{c}}
\newcommand{\Bd}[0]{\mathbf{d}}
\newcommand{\Be}[0]{\mathbf{e}}
\newcommand{\Bf}[0]{\mathbf{f}}
\newcommand{\Bg}[0]{\mathbf{g}}
\newcommand{\Bh}[0]{\mathbf{h}}
\newcommand{\Bi}[0]{\mathbf{i}}
\newcommand{\Bj}[0]{\mathbf{j}}
\newcommand{\Bk}[0]{\mathbf{k}}
\newcommand{\Bl}[0]{\mathbf{l}}
\newcommand{\Bm}[0]{\mathbf{m}}
\newcommand{\Bn}[0]{\mathbf{n}}
\newcommand{\Bo}[0]{\mathbf{o}}
\newcommand{\Bp}[0]{\mathbf{p}}
\newcommand{\Bq}[0]{\mathbf{q}}
\newcommand{\Br}[0]{\mathbf{r}}
\newcommand{\Bs}[0]{\mathbf{s}}
\newcommand{\Bt}[0]{\mathbf{t}}
\newcommand{\Bu}[0]{\mathbf{u}}
\newcommand{\Bv}[0]{\mathbf{v}}
\newcommand{\Bw}[0]{\mathbf{w}}
\newcommand{\Bx}[0]{\mathbf{x}}
\newcommand{\By}[0]{\mathbf{y}}
\newcommand{\Bz}[0]{\mathbf{z}}
\newcommand{\BA}[0]{\mathbf{A}}
\newcommand{\BB}[0]{\mathbf{B}}
\newcommand{\BC}[0]{\mathbf{C}}
\newcommand{\BD}[0]{\mathbf{D}}
\newcommand{\BE}[0]{\mathbf{E}}
\newcommand{\BF}[0]{\mathbf{F}}
\newcommand{\BG}[0]{\mathbf{G}}
\newcommand{\BH}[0]{\mathbf{H}}
\newcommand{\BI}[0]{\mathbf{I}}
\newcommand{\BJ}[0]{\mathbf{J}}
\newcommand{\BK}[0]{\mathbf{K}}
\newcommand{\BL}[0]{\mathbf{L}}
\newcommand{\BM}[0]{\mathbf{M}}
\newcommand{\BN}[0]{\mathbf{N}}
\newcommand{\BO}[0]{\mathbf{O}}
\newcommand{\BP}[0]{\mathbf{P}}
\newcommand{\BQ}[0]{\mathbf{Q}}
\newcommand{\BR}[0]{\mathbf{R}}
\newcommand{\BS}[0]{\mathbf{S}}
\newcommand{\BT}[0]{\mathbf{T}}
\newcommand{\BU}[0]{\mathbf{U}}
\newcommand{\BV}[0]{\mathbf{V}}
\newcommand{\BW}[0]{\mathbf{W}}
\newcommand{\BX}[0]{\mathbf{X}}
\newcommand{\BY}[0]{\mathbf{Y}}
\newcommand{\BZ}[0]{\mathbf{Z}}

\newcommand{\Bzero}[0]{\mathbf{0}}
\newcommand{\Btheta}[0]{\boldsymbol{\theta}}
\newcommand{\Btau}[0]{\boldsymbol{\tau}}
\newcommand{\Bomega}[0]{\boldsymbol{\omega}}

%
% shorthand for unit vectors
%
\newcommand{\acap}[0]{\hat{\Ba}}
\newcommand{\bcap}[0]{\hat{\Bb}}
\newcommand{\ccap}[0]{\hat{\Bc}}
\newcommand{\dcap}[0]{\hat{\Bd}}
\newcommand{\ecap}[0]{\hat{\Be}}
\newcommand{\fcap}[0]{\hat{\Bf}}
\newcommand{\gcap}[0]{\hat{\Bg}}
\newcommand{\hcap}[0]{\hat{\Bh}}
\newcommand{\icap}[0]{\hat{\Bi}}
\newcommand{\jcap}[0]{\hat{\Bj}}
\newcommand{\kcap}[0]{\hat{\Bk}}
\newcommand{\lcap}[0]{\hat{\Bl}}
\newcommand{\mcap}[0]{\hat{\Bm}}
\newcommand{\ncap}[0]{\hat{\Bn}}
\newcommand{\ocap}[0]{\hat{\Bo}}
\newcommand{\pcap}[0]{\hat{\Bp}}
\newcommand{\qcap}[0]{\hat{\Bq}}
\newcommand{\rcap}[0]{\hat{\Br}}
\newcommand{\scap}[0]{\hat{\Bs}}
\newcommand{\tcap}[0]{\hat{\Bt}}
\newcommand{\ucap}[0]{\hat{\Bu}}
\newcommand{\vcap}[0]{\hat{\Bv}}
\newcommand{\wcap}[0]{\hat{\Bw}}
\newcommand{\xcap}[0]{\hat{\Bx}}
\newcommand{\ycap}[0]{\hat{\By}}
\newcommand{\zcap}[0]{\hat{\Bz}}
\newcommand{\thetacap}[0]{\hat{\Btheta}}

%
% to write R^n and C^n in a distinguishable fashion.  Perhaps change this
% to the double lined characters upon figuring out how to do so.
%
\newcommand{\C}[1]{$\mathbb{C}^{#1}$}
\newcommand{\R}[1]{$\mathbb{R}^{#1}$}

%
% various generally useful helpers
%

% derivative of #1 wrt. #2:
\newcommand{\D}[2] {\frac {d#2} {d#1}}

\newcommand{\inv}[1]{\frac{1}{#1}}
\newcommand{\cross}[0]{\times}

\newcommand{\abs}[1]{\lvert{#1}\rvert}
\newcommand{\norm}[1]{\lVert{#1}\rVert}
\newcommand{\innerprod}[2]{\langle{#1}, {#2}\rangle}
\newcommand{\dotprod}[2]{{#1} \cdot {#2}}
\newcommand{\bdotprod}[2]{\left({#1} \cdot {#2}\right)}
\newcommand{\crossprod}[2]{{#1} \cross {#2}}
\newcommand{\tripleprod}[3]{\dotprod{\left(\crossprod{#1}{#2}\right)}{#3}}

\DeclareMathOperator{\Proj}{Proj}
\DeclareMathOperator{\Span}{span}
\DeclareMathOperator{\Sgn}{sgn}
\DeclareMathOperator{\Area}{Area}
\DeclareMathOperator{\Volume}{Volume}

%
% A few miscellaneous things specific to this document
%
\newcommand{\crossop}[1]{\crossprod{#1}{}}

% R2 vector.
\newcommand{\VectorTwo}[2]{
\begin{bmatrix}
 {#1} \\
 {#2}
\end{bmatrix}
}

\newcommand{\VectorN}[1]{
\begin{bmatrix}
{#1}_1 \\
{#1}_2 \\
\vdots \\
{#1}_N \\
\end{bmatrix}
}

\newcommand{\DETuvij}[4]{
\begin{vmatrix}
 {#1}_{#3} & {#1}_{#4} \\
 {#2}_{#3} & {#2}_{#4}
\end{vmatrix}
}

\newcommand{\DETuvwijk}[6]{
\begin{vmatrix}
 {#1}_{#4} & {#1}_{#5} & {#1}_{#6} \\
 {#2}_{#4} & {#2}_{#5} & {#2}_{#6} \\
 {#3}_{#4} & {#3}_{#5} & {#3}_{#6}
\end{vmatrix}
}

\newcommand{\DETuvwxijkl}[8]{
\begin{vmatrix}
 {#1}_{#5} & {#1}_{#6} & {#1}_{#7} & {#1}_{#8} \\
 {#2}_{#5} & {#2}_{#6} & {#2}_{#7} & {#2}_{#8} \\
 {#3}_{#5} & {#3}_{#6} & {#3}_{#7} & {#3}_{#8} \\
 {#4}_{#5} & {#4}_{#6} & {#4}_{#7} & {#4}_{#8} \\
\end{vmatrix}
}

%\newcommand{\DETuvwxyijklm}[10]{
%\begin{vmatrix}
% {#1}_{#6} & {#1}_{#7} & {#1}_{#8} & {#1}_{#9} & {#1}_{#10} \\
% {#2}_{#6} & {#2}_{#7} & {#2}_{#8} & {#2}_{#9} & {#2}_{#10} \\
% {#3}_{#6} & {#3}_{#7} & {#3}_{#8} & {#3}_{#9} & {#3}_{#10} \\
% {#4}_{#6} & {#4}_{#7} & {#4}_{#8} & {#4}_{#9} & {#4}_{#10} \\
% {#5}_{#6} & {#5}_{#7} & {#5}_{#8} & {#5}_{#9} & {#5}_{#10}
%\end{vmatrix}
%}

% R3 vector.
\newcommand{\VectorThree}[3]{
\begin{bmatrix}
 {#1} \\
 {#2} \\
 {#3}
\end{bmatrix}
}


%<misc>
%
\newcommand{\Abs}[1]{{\left\lvert{#1}\right\rvert}}
\newcommand{\spacegrad}[0]{\boldsymbol{\nabla}}
\newcommand{\grad}[0]{\nabla}
\newcommand{\LL}[0]{\mathcal{L}}

% == \partial_{#1} {#2}
\newcommand{\PD}[2]{\frac{\partial {#2}}{\partial {#1}}}
% inline variant
\newcommand{\PDi}[2]{{\partial {#2}}/{\partial {#1}}}

\newcommand{\PDD}[3]{\frac{\partial^2 {#3}}{\partial {#1}\partial {#2}}}
%\newcommand{\PDd}[2]{\frac{\partial^2 {#2}}{{\partial{#1}}^2}}
\newcommand{\PDsq}[2]{\frac{\partial^2 {#2}}{(\partial {#1})^2}}

\newcommand{\Partial}[2]{\frac{\partial {#1}}{\partial {#2}}}
\DeclareMathOperator{\RejName}{Rej}
\newcommand{\Rej}[2]{\RejName_{#1}\left( {#2} \right)}
\newcommand{\Rm}[1]{\mathbb{R}^{#1}}
\newcommand{\Cm}[1]{\mathbb{C}^{#1}}
\newcommand{\conj}[0]{{*}}

%</misc>

% <grade selection>
%
\newcommand{\gpgrade}[2] {{\left\langle{{#1}}\right\rangle}_{#2}}

\newcommand{\gpgradezero}[1] {\gpgrade{#1}{}}
%\newcommand{\gpscalargrade}[1] {{\left\langle{{#1}}\right\rangle}}
%\newcommand{\gpgradezero}[1] {\gpgrade{#1}{0}}

%\newcommand{\gpgradeone}[1] {{\left\langle{{#1}}\right\rangle}_{1}}
\newcommand{\gpgradeone}[1] {\gpgrade{#1}{1}}

\newcommand{\gpgradetwo}[1] {\gpgrade{#1}{2}}
\newcommand{\gpgradethree}[1] {\gpgrade{#1}{3}}
\newcommand{\gpgradefour}[1] {\gpgrade{#1}{4}}
%
% </grade selection>



\newcommand{\adot}[0]{{\dot{a}}}
\newcommand{\bdot}[0]{{\dot{b}}}
% taken for centered dot:
%\newcommand{\cdot}[0]{{\dot{c}}}
%\newcommand{\ddot}[0]{{\dot{d}}}
\newcommand{\edot}[0]{{\dot{e}}}
\newcommand{\fdot}[0]{{\dot{f}}}
\newcommand{\gdot}[0]{{\dot{g}}}
\newcommand{\hdot}[0]{{\dot{h}}}
\newcommand{\idot}[0]{{\dot{i}}}
\newcommand{\jdot}[0]{{\dot{j}}}
\newcommand{\kdot}[0]{{\dot{k}}}
\newcommand{\ldot}[0]{{\dot{l}}}
\newcommand{\mdot}[0]{{\dot{m}}}
\newcommand{\ndot}[0]{{\dot{n}}}
%\newcommand{\odot}[0]{{\dot{o}}}
\newcommand{\pdot}[0]{{\dot{p}}}
\newcommand{\qdot}[0]{{\dot{q}}}
\newcommand{\rdot}[0]{{\dot{r}}}
\newcommand{\sdot}[0]{{\dot{s}}}
\newcommand{\tdot}[0]{{\dot{t}}}
\newcommand{\udot}[0]{{\dot{u}}}
\newcommand{\vdot}[0]{{\dot{v}}}
\newcommand{\wdot}[0]{{\dot{w}}}
\newcommand{\xdot}[0]{{\dot{x}}}
\newcommand{\ydot}[0]{{\dot{y}}}
\newcommand{\zdot}[0]{{\dot{z}}}
\newcommand{\addot}[0]{{\ddot{a}}}
\newcommand{\bddot}[0]{{\ddot{b}}}
\newcommand{\cddot}[0]{{\ddot{c}}}
%\newcommand{\dddot}[0]{{\ddot{d}}}
\newcommand{\eddot}[0]{{\ddot{e}}}
\newcommand{\fddot}[0]{{\ddot{f}}}
\newcommand{\gddot}[0]{{\ddot{g}}}
\newcommand{\hddot}[0]{{\ddot{h}}}
\newcommand{\iddot}[0]{{\ddot{i}}}
\newcommand{\jddot}[0]{{\ddot{j}}}
\newcommand{\kddot}[0]{{\ddot{k}}}
\newcommand{\lddot}[0]{{\ddot{l}}}
\newcommand{\mddot}[0]{{\ddot{m}}}
\newcommand{\nddot}[0]{{\ddot{n}}}
\newcommand{\oddot}[0]{{\ddot{o}}}
\newcommand{\pddot}[0]{{\ddot{p}}}
\newcommand{\qddot}[0]{{\ddot{q}}}
\newcommand{\rddot}[0]{{\ddot{r}}}
\newcommand{\sddot}[0]{{\ddot{s}}}
\newcommand{\tddot}[0]{{\ddot{t}}}
\newcommand{\uddot}[0]{{\ddot{u}}}
\newcommand{\vddot}[0]{{\ddot{v}}}
\newcommand{\wddot}[0]{{\ddot{w}}}
\newcommand{\xddot}[0]{{\ddot{x}}}
\newcommand{\yddot}[0]{{\ddot{y}}}
\newcommand{\zddot}[0]{{\ddot{z}}}

%<bold and dot greek symbols>
%

\newcommand{\Deltadot}[0]{{\dot{\Delta}}}
\newcommand{\Gammadot}[0]{{\dot{\Gamma}}}
\newcommand{\Lambdadot}[0]{{\dot{\Lambda}}}
\newcommand{\Omegadot}[0]{{\dot{\Omega}}}
\newcommand{\Phidot}[0]{{\dot{\Phi}}}
\newcommand{\Pidot}[0]{{\dot{\Pi}}}
\newcommand{\Psidot}[0]{{\dot{\Psi}}}
\newcommand{\Sigmadot}[0]{{\dot{\Sigma}}}
\newcommand{\Thetadot}[0]{{\dot{\Theta}}}
\newcommand{\Upsilondot}[0]{{\dot{\Upsilon}}}
\newcommand{\Xidot}[0]{{\dot{\Xi}}}
\newcommand{\alphadot}[0]{{\dot{\alpha}}}
\newcommand{\betadot}[0]{{\dot{\beta}}}
\newcommand{\chidot}[0]{{\dot{\chi}}}
\newcommand{\deltadot}[0]{{\dot{\delta}}}
\newcommand{\epsilondot}[0]{{\dot{\epsilon}}}
\newcommand{\etadot}[0]{{\dot{\eta}}}
\newcommand{\gammadot}[0]{{\dot{\gamma}}}
\newcommand{\kappadot}[0]{{\dot{\kappa}}}
\newcommand{\lambdadot}[0]{{\dot{\lambda}}}
\newcommand{\mudot}[0]{{\dot{\mu}}}
\newcommand{\nudot}[0]{{\dot{\nu}}}
\newcommand{\omegadot}[0]{{\dot{\omega}}}
\newcommand{\phidot}[0]{{\dot{\phi}}}
\newcommand{\pidot}[0]{{\dot{\pi}}}
\newcommand{\psidot}[0]{{\dot{\psi}}}
\newcommand{\rhodot}[0]{{\dot{\rho}}}
\newcommand{\sigmadot}[0]{{\dot{\sigma}}}
\newcommand{\taudot}[0]{{\dot{\tau}}}
\newcommand{\thetadot}[0]{{\dot{\theta}}}
\newcommand{\upsilondot}[0]{{\dot{\upsilon}}}
\newcommand{\varepsilondot}[0]{{\dot{\varepsilon}}}
\newcommand{\varphidot}[0]{{\dot{\varphi}}}
\newcommand{\varpidot}[0]{{\dot{\varpi}}}
\newcommand{\varrhodot}[0]{{\dot{\varrho}}}
\newcommand{\varsigmadot}[0]{{\dot{\varsigma}}}
\newcommand{\varthetadot}[0]{{\dot{\vartheta}}}
\newcommand{\xidot}[0]{{\dot{\xi}}}
\newcommand{\zetadot}[0]{{\dot{\zeta}}}

\newcommand{\Deltaddot}[0]{{\ddot{\Delta}}}
\newcommand{\Gammaddot}[0]{{\ddot{\Gamma}}}
\newcommand{\Lambdaddot}[0]{{\ddot{\Lambda}}}
\newcommand{\Omegaddot}[0]{{\ddot{\Omega}}}
\newcommand{\Phiddot}[0]{{\ddot{\Phi}}}
\newcommand{\Piddot}[0]{{\ddot{\Pi}}}
\newcommand{\Psiddot}[0]{{\ddot{\Psi}}}
\newcommand{\Sigmaddot}[0]{{\ddot{\Sigma}}}
\newcommand{\Thetaddot}[0]{{\ddot{\Theta}}}
\newcommand{\Upsilonddot}[0]{{\ddot{\Upsilon}}}
\newcommand{\Xiddot}[0]{{\ddot{\Xi}}}
\newcommand{\alphaddot}[0]{{\ddot{\alpha}}}
\newcommand{\betaddot}[0]{{\ddot{\beta}}}
\newcommand{\chiddot}[0]{{\ddot{\chi}}}
\newcommand{\deltaddot}[0]{{\ddot{\delta}}}
\newcommand{\epsilonddot}[0]{{\ddot{\epsilon}}}
\newcommand{\etaddot}[0]{{\ddot{\eta}}}
\newcommand{\gammaddot}[0]{{\ddot{\gamma}}}
\newcommand{\kappaddot}[0]{{\ddot{\kappa}}}
\newcommand{\lambdaddot}[0]{{\ddot{\lambda}}}
\newcommand{\muddot}[0]{{\ddot{\mu}}}
\newcommand{\nuddot}[0]{{\ddot{\nu}}}
\newcommand{\omegaddot}[0]{{\ddot{\omega}}}
\newcommand{\phiddot}[0]{{\ddot{\phi}}}
\newcommand{\piddot}[0]{{\ddot{\pi}}}
\newcommand{\psiddot}[0]{{\ddot{\psi}}}
\newcommand{\rhoddot}[0]{{\ddot{\rho}}}
\newcommand{\sigmaddot}[0]{{\ddot{\sigma}}}
\newcommand{\tauddot}[0]{{\ddot{\tau}}}
\newcommand{\thetaddot}[0]{{\ddot{\theta}}}
\newcommand{\upsilonddot}[0]{{\ddot{\upsilon}}}
\newcommand{\varepsilonddot}[0]{{\ddot{\varepsilon}}}
\newcommand{\varphiddot}[0]{{\ddot{\varphi}}}
\newcommand{\varpiddot}[0]{{\ddot{\varpi}}}
\newcommand{\varrhoddot}[0]{{\ddot{\varrho}}}
\newcommand{\varsigmaddot}[0]{{\ddot{\varsigma}}}
\newcommand{\varthetaddot}[0]{{\ddot{\vartheta}}}
\newcommand{\xiddot}[0]{{\ddot{\xi}}}
\newcommand{\zetaddot}[0]{{\ddot{\zeta}}}

\newcommand{\BDelta}[0]{\boldsymbol{\Delta}}
\newcommand{\BGamma}[0]{\boldsymbol{\Gamma}}
\newcommand{\BLambda}[0]{\boldsymbol{\Lambda}}
\newcommand{\BOmega}[0]{\boldsymbol{\Omega}}
\newcommand{\BPhi}[0]{\boldsymbol{\Phi}}
\newcommand{\BPi}[0]{\boldsymbol{\Pi}}
\newcommand{\BPsi}[0]{\boldsymbol{\Psi}}
\newcommand{\BSigma}[0]{\boldsymbol{\Sigma}}
\newcommand{\BTheta}[0]{\boldsymbol{\Theta}}
\newcommand{\BUpsilon}[0]{\boldsymbol{\Upsilon}}
\newcommand{\BXi}[0]{\boldsymbol{\Xi}}
\newcommand{\Balpha}[0]{\boldsymbol{\alpha}}
\newcommand{\Bbeta}[0]{\boldsymbol{\beta}}
\newcommand{\Bchi}[0]{\boldsymbol{\chi}}
\newcommand{\Bdelta}[0]{\boldsymbol{\delta}}
\newcommand{\Bepsilon}[0]{\boldsymbol{\epsilon}}
\newcommand{\Beta}[0]{\boldsymbol{\eta}}
\newcommand{\Bgamma}[0]{\boldsymbol{\gamma}}
\newcommand{\Bkappa}[0]{\boldsymbol{\kappa}}
\newcommand{\Blambda}[0]{\boldsymbol{\lambda}}
\newcommand{\Bmu}[0]{\boldsymbol{\mu}}
\newcommand{\Bnu}[0]{\boldsymbol{\nu}}
%\newcommand{\Bomega}[0]{\boldsymbol{\omega}}
\newcommand{\Bphi}[0]{\boldsymbol{\phi}}
\newcommand{\Bpi}[0]{\boldsymbol{\pi}}
\newcommand{\Bpsi}[0]{\boldsymbol{\psi}}
\newcommand{\Brho}[0]{\boldsymbol{\rho}}
\newcommand{\Bsigma}[0]{\boldsymbol{\sigma}}
%\newcommand{\Btau}[0]{\boldsymbol{\tau}}
%\newcommand{\Btheta}[0]{\boldsymbol{\theta}}
\newcommand{\Bupsilon}[0]{\boldsymbol{\upsilon}}
\newcommand{\Bvarepsilon}[0]{\boldsymbol{\varepsilon}}
\newcommand{\Bvarphi}[0]{\boldsymbol{\varphi}}
\newcommand{\Bvarpi}[0]{\boldsymbol{\varpi}}
\newcommand{\Bvarrho}[0]{\boldsymbol{\varrho}}
\newcommand{\Bvarsigma}[0]{\boldsymbol{\varsigma}}
\newcommand{\Bvartheta}[0]{\boldsymbol{\vartheta}}
\newcommand{\Bxi}[0]{\boldsymbol{\xi}}
\newcommand{\Bzeta}[0]{\boldsymbol{\zeta}}
%
%</bold and dot greek symbols>
%<infrequent>
%
%\newcommand{\AreaOp}[1]{\AName_{#1}}
%\newcommand{\Babs}[0]{\abs{\BB}}
%\newcommand{\Bcap}[0]{\hat{\BB}}
%\newcommand{\BrPrimeRej}[0]{\rcap(\rcap \wedge \Br')}
%\newcommand{\CA}[0]{\mathcal{A}}
%\newcommand{\Cos}[1]{\cos{\left({#1}\right)}}
%\newcommand{\Det}[1] {\abs{#1}}
%\newcommand{\Dsq}[2] {\frac {\partial^2 {#1}} {\partial {#2}^2}}
%\newcommand{\Exp}[1]{\exp{\left({#1}\right)}}
%\newcommand{\Norm}[1]{\left\lVert{#1}\right\rVert}
%\newcommand{\Sin}[1]{\sin{\left({#1}\right)}}
%\newcommand{\T}[0]{\text{T}}
%\newcommand{\VolumeOp}[1]{\VName_{#1}}
%\newcommand{\agrad}[0]{\Ba \cdot \nabla}
%\newcommand{\alphacap}[0]{\hat{\boldsymbol{\alpha}}}
%\newcommand{\Fcap}[0]{\hat{\BF}}
%\newcommand{\bithree}[0]{{\Bi}_3}
%\newcommand{\bxa}[0]{\Bx\Ba}
%\newcommand{\coordvec}[2]{
%\newcommand{\costheta}[0]{\acap \cdot \xcap}
%\newcommand{\ddt}[1]{\ddot{#1}}
%\newcommand{\ddu}[1] {\frac {d{#1}} {du}}
%\newcommand{\dsqxj}[2] {\frac {\partial^2 {#1}} {\partial {x_{#2}}^2}}
%\newcommand{\dtheta}[1]{\frac{d {#1}}{d \theta}}
%\newcommand{\dt}[1]{\dot{#1}}
%\newcommand{\dt}[1]{\frac{d {#1}}{dt}}
%\newcommand{\dxj}[2] {\frac {\partial {#1}} {\partial {x_{#2}}}}
%\newcommand{\halfPhi}[0]{\frac{\phi}{2}}
%\newcommand{\half}[0]{\inv{2}}
%\newcommand{\inv}[1]{\frac{1}{#1}}
%\newcommand{\laplacian}[0]{\nabla^2}
%\newcommand{\matrixoftx}[3]{
%\newcommand{\nrrp}[0]{\norm{\rcap \wedge \Br'}}
%\newcommand{\oiint}{\bigcirc \hspace{-1.4em} \int \hspace{-.8em} \int}
%\newcommand{\transpose}[1]{{#1}^{\text{T}}}
%\newcommand{\transpose}[1]{{{#1}^{\TextTranspose}}}
%\newcommand{\transpose}[1]{{{#1}^{\text{T}}}}
%\newcommand{\barA}[0]{\bar{A}}
%\newcommand{\qbar}[0]{\bar{q}}
%\newcommand{\qdotbar}[0]{\dot{\bar{q}}}
%
%</infrequent>





%\DeclareMathOperator{\Atan2}{atan2}
\DeclareMathOperator{\atan}{atan}

%\usepackage{listings}
%\usepackage{txfonts} % for ointctr... (also appears to make "prettier" \int and \sum's)
% makes \grad look funny though (almost like spacegrad, but narrower)
\usepackage[bookmarks=true]{hyperref}

\usepackage{color,cite,graphicx}
   % use colour in the document, put your citations as [1-4]
   % rather than [1,2,3,4] (it looks nicer, and the extended LaTeX2e
   % graphics package. 
\usepackage{latexsym,amssymb,epsf} % don't remember if these are
   % needed, but their inclusion can't do any damage


\title{ One dimensional rectanglular Quantum barrier problem. }
\author{Peeter Joot \quad peeter.joot@gmail.com }
\date{ May 11, 2009.  Last Revision: $Date: 2009/05/12 03:30:25 $ }

\begin{document}

\maketitle{}
\tableofcontents
\section{ Motivation. }

My first attempt at the probability current calculation for this
while doing the chapter 11 problems of \cite{bohm1989qt} led to trouble.
An attempt to look at this worked example in \cite{mcmahon2005qmd} didn't 
help much due to innumerable typos (at least a lot more than in my first attempt).
Here's a second bash at it, redoing the calculations from scratch.

\section{ Setup. }

The potential is taken to be zero everywhere except $x \in [0,a]$, where it is
$V$.  This divides the problem into three regions, I, II, and III, for 
before ($x<0$), in and after the barrier, and the wave functions for the 
$E >V$ case are respectively

\begin{equation}
\psi =
\left\{
\begin{array}{l l}
A e^{i k x} + B e^{-i k x} & \quad \mbox{if $x <0$} \\
C e^{ \beta(x-a)} + D e^{ -\beta(x-a)} & \quad \mbox{if $x \in [0,a]$} \\
E e^{i k(x-a)} & \quad \mbox{if $x >0$} \\
\end{array}
\right.
\end{equation}

%Where $p = \hbar \sqrt{2 m E} = \hbar k$
Where $k = \sqrt{2 m E}/\hbar$, and $\beta = \sqrt{2 m (E-V)}/\hbar$.
FIXME: check $\beta$ ... doesn't look right.

\subsection{ Equality at $x=a$ }

Equality of the wave functions and derivatives at $x=a$ gives

\begin{align*}
C + D &= E \\
C - D &= \frac{i k}{\beta}E
\end{align*}

which has solutions

\begin{align*}
C &= \frac{E}{2}( 1 + i k/\beta ) \\
D &= \frac{E}{2}( 1 - i k/\beta )
\end{align*}

Two of the free variables of the wave equation are now eliminated, and the wave function in the barrier region can now be written as
\begin{align}\label{eqn:psiInBarrier}
\psi =
\frac{E}{2}\left( \left( 1 + \frac{i k}{\beta} \right) e^{ \beta(x-a)} + \left( 1 - \frac{i k}{\beta} \right) e^{ -\beta(x-a)} \right)
\end{align}

\subsection{ Equality at $x=0$ }

Equating values and first derivatives at $x=0$ we have

\begin{align*}
A + B &=
\frac{E}{2}\left( \left( 1 + \frac{i k}{\beta} \right) e^{ -\beta a} + \left( 1 - \frac{i k}{\beta} \right) e^{ \beta a } \right) \\
A - B &=
\frac{\beta E}{2 i k}\left( \left( 1 + \frac{i k}{\beta} \right) e^{ -\beta a} - \left( 1 - \frac{i k}{\beta} \right) e^{ \beta a } \right)
\end{align*}

Taking sums and differences we have

\begin{align*}
A &= \frac{E}{4}\left( \left(1 + \frac{\beta}{ik}\right)\left( 1 + \frac{i k}{\beta} \right) e^{ -\beta a} + \left(1 - \frac{\beta}{ik}\right)\left( 1 - \frac{i k}{\beta} \right) e^{ \beta a } \right) \\
B &= \frac{E}{4}\left( \left(1 - \frac{\beta}{ik}\right)\left( 1 + \frac{i k}{\beta} \right) e^{ -\beta a} + \left(1 + \frac{\beta}{ik}\right)\left( 1 - \frac{i k}{\beta} \right) e^{ \beta a } \right) \\
\end{align*}

Expanding the products first for $B$

\begin{align*}
B 
&= \frac{E}{4}\left( \left(\frac{-\beta}{ik} + \frac{ik}{\beta} \right) e^{ -\beta a} + \left(\frac{\beta}{ik} - \frac{ik}{\beta} \right) e^{ \beta a } \right) \\
&= \frac{iE}{4}\left( \left(\frac{\beta}{k} + \frac{k}{\beta} \right) e^{ -\beta a} - \left(\frac{\beta}{k} + \frac{k}{\beta} \right) e^{ \beta a } \right) \\
&= \frac{iE}{4}\left(\frac{\beta}{k} + \frac{k}{\beta} \right) \left( e^{ -\beta a} - e^{ \beta a } \right) \\
&= \frac{-iE}{2}\left(\frac{\beta}{k} + \frac{k}{\beta} \right) \sinh\left( \beta a \right) \\
\end{align*}

Now for $A$ we have
\begin{align*}
A &= \frac{E}{4}\left( \left(2 + \frac{\beta}{ik} + \frac{ik}{\beta} \right) e^{ -\beta a} + \left(2 - \frac{\beta}{ik} - \frac{ik}{\beta} \right) e^{ \beta a } \right) \\
\end{align*}

The factor of exponentials are complex conjugates and can be put into polar form to simplify.  Writing

\begin{align*}
\gamma 
&= 2 + \frac{\beta}{ik} + \frac{ik}{\beta} \\
&= 2 + i \left( \frac{k}{\beta} -\frac{\beta}{k} \right) \\
&= \mu e^{i \theta} 
\end{align*}

Where 

\begin{align*}
\mu^2 &= 4 + \left( \frac{k}{\beta} -\frac{\beta}{k} \right)^2 \\
\theta &= \atan\left( \inv{2} \left(\frac{k}{\beta} - \frac{\beta}{k} \right) \right)
\end{align*}

We have

\begin{align*}
A 
&= \frac{\mu E}{4}\left( e^{ i\theta -\beta a} + e^{ -(i \theta - \beta a) } \right) \\
&= \frac{\mu E}{2}\cosh\left( i\theta -\beta a \right) \\
\end{align*}

For notational consistency it looks desirable to write something like

\begin{align*}
\nu &= -i \left(\frac{\beta}{k} + \frac{k}{\beta} \right) 
\end{align*}

Which leaves us with the wave function in the $x<0$ region as
\begin{align}
\psi &=
\frac{\mu E}{2}\cosh\left( i\theta -\beta a \right) e^{ i k x }
+\frac{\nu E}{2}\sinh\left( \beta a \right) e^{ -i k x }
\end{align}

\subsection{ Barrier wave function in polar form. }

Having seen how the polar form simplifies the final expression of the first region wave function, it can be seen that 
something similar can be done in the barrier region.  

In equation \ref{eqn:psiInBarrier}, we can utilize polar form for the $1+ik/\beta$ constant and its conjugate.  Writing

\begin{align*}
\tan\phi &= \frac{k}{\beta}
\end{align*}

we then have
\begin{align*}
\psi 
&=
\frac{E}{2}\sqrt{1 + \frac{k^2}{\beta^2}} \left( e^{i\phi} e^{ \beta(x-a)} + e^{-i\phi} e^{ -\beta(x-a)} \right) \\
\end{align*}

So we have in the $x \in [0,a]$ region
\begin{align}
\psi &= E \sqrt{1 + \frac{k^2}{\beta^2}} \cosh\left( i\phi + \beta(x-a) \right) 
\end{align}

Utilizing this should produce the same result.  Equality at $x=0$ then gives

\begin{align*}
A + B &= E \sqrt{1 + \frac{k^2}{\beta^2}} \cosh\left( i\phi -\beta a \right) \\
A - B &= E \frac{i\beta}{k} \sqrt{1 + \frac{k^2}{\beta^2}} \sinh\left( i\phi -\beta a \right) \\
\end{align*}

Or

\begin{align*}
A &= \frac{E}{2} \sqrt{1 + \frac{k^2}{\beta^2}} \left(
\cosh\left( i\phi -\beta a \right) + \frac{i\beta}{k} \sinh\left( i\phi -\beta a \right) 
\right) \\
B &= \frac{E}{2} \sqrt{1 + \frac{k^2}{\beta^2}} \left(
\cosh\left( i\phi -\beta a \right) - \frac{i\beta}{k} \sinh\left( i\phi -\beta a \right) 
\right) \\
\end{align*}

\bibliographystyle{plainnat}
\bibliography{myrefs}

\end{document}
           % may 11/09
\documentclass[]{eliblog}

\usepackage{amsmath}
\usepackage{mathpazo}

%
% shorthand for bold symbols, convenient for vectors and matrices
%
\newcommand{\Ba}[0]{\mathbf{a}}
\newcommand{\Bb}[0]{\mathbf{b}}
\newcommand{\Bc}[0]{\mathbf{c}}
\newcommand{\Bd}[0]{\mathbf{d}}
\newcommand{\Be}[0]{\mathbf{e}}
\newcommand{\Bf}[0]{\mathbf{f}}
\newcommand{\Bg}[0]{\mathbf{g}}
\newcommand{\Bh}[0]{\mathbf{h}}
\newcommand{\Bi}[0]{\mathbf{i}}
\newcommand{\Bj}[0]{\mathbf{j}}
\newcommand{\Bk}[0]{\mathbf{k}}
\newcommand{\Bl}[0]{\mathbf{l}}
\newcommand{\Bm}[0]{\mathbf{m}}
\newcommand{\Bn}[0]{\mathbf{n}}
\newcommand{\Bo}[0]{\mathbf{o}}
\newcommand{\Bp}[0]{\mathbf{p}}
\newcommand{\Bq}[0]{\mathbf{q}}
\newcommand{\Br}[0]{\mathbf{r}}
\newcommand{\Bs}[0]{\mathbf{s}}
\newcommand{\Bt}[0]{\mathbf{t}}
\newcommand{\Bu}[0]{\mathbf{u}}
\newcommand{\Bv}[0]{\mathbf{v}}
\newcommand{\Bw}[0]{\mathbf{w}}
\newcommand{\Bx}[0]{\mathbf{x}}
\newcommand{\By}[0]{\mathbf{y}}
\newcommand{\Bz}[0]{\mathbf{z}}
\newcommand{\BA}[0]{\mathbf{A}}
\newcommand{\BB}[0]{\mathbf{B}}
\newcommand{\BC}[0]{\mathbf{C}}
\newcommand{\BD}[0]{\mathbf{D}}
\newcommand{\BE}[0]{\mathbf{E}}
\newcommand{\BF}[0]{\mathbf{F}}
\newcommand{\BG}[0]{\mathbf{G}}
\newcommand{\BH}[0]{\mathbf{H}}
\newcommand{\BI}[0]{\mathbf{I}}
\newcommand{\BJ}[0]{\mathbf{J}}
\newcommand{\BK}[0]{\mathbf{K}}
\newcommand{\BL}[0]{\mathbf{L}}
\newcommand{\BM}[0]{\mathbf{M}}
\newcommand{\BN}[0]{\mathbf{N}}
\newcommand{\BO}[0]{\mathbf{O}}
\newcommand{\BP}[0]{\mathbf{P}}
\newcommand{\BQ}[0]{\mathbf{Q}}
\newcommand{\BR}[0]{\mathbf{R}}
\newcommand{\BS}[0]{\mathbf{S}}
\newcommand{\BT}[0]{\mathbf{T}}
\newcommand{\BU}[0]{\mathbf{U}}
\newcommand{\BV}[0]{\mathbf{V}}
\newcommand{\BW}[0]{\mathbf{W}}
\newcommand{\BX}[0]{\mathbf{X}}
\newcommand{\BY}[0]{\mathbf{Y}}
\newcommand{\BZ}[0]{\mathbf{Z}}

\newcommand{\Bzero}[0]{\mathbf{0}}
\newcommand{\Btheta}[0]{\boldsymbol{\theta}}
\newcommand{\Btau}[0]{\boldsymbol{\tau}}
\newcommand{\Bomega}[0]{\boldsymbol{\omega}}

%
% shorthand for unit vectors
%
\newcommand{\acap}[0]{\hat{\Ba}}
\newcommand{\bcap}[0]{\hat{\Bb}}
\newcommand{\ccap}[0]{\hat{\Bc}}
\newcommand{\dcap}[0]{\hat{\Bd}}
\newcommand{\ecap}[0]{\hat{\Be}}
\newcommand{\fcap}[0]{\hat{\Bf}}
\newcommand{\gcap}[0]{\hat{\Bg}}
\newcommand{\hcap}[0]{\hat{\Bh}}
\newcommand{\icap}[0]{\hat{\Bi}}
\newcommand{\jcap}[0]{\hat{\Bj}}
\newcommand{\kcap}[0]{\hat{\Bk}}
\newcommand{\lcap}[0]{\hat{\Bl}}
\newcommand{\mcap}[0]{\hat{\Bm}}
\newcommand{\ncap}[0]{\hat{\Bn}}
\newcommand{\ocap}[0]{\hat{\Bo}}
\newcommand{\pcap}[0]{\hat{\Bp}}
\newcommand{\qcap}[0]{\hat{\Bq}}
\newcommand{\rcap}[0]{\hat{\Br}}
\newcommand{\scap}[0]{\hat{\Bs}}
\newcommand{\tcap}[0]{\hat{\Bt}}
\newcommand{\ucap}[0]{\hat{\Bu}}
\newcommand{\vcap}[0]{\hat{\Bv}}
\newcommand{\wcap}[0]{\hat{\Bw}}
\newcommand{\xcap}[0]{\hat{\Bx}}
\newcommand{\ycap}[0]{\hat{\By}}
\newcommand{\zcap}[0]{\hat{\Bz}}
\newcommand{\thetacap}[0]{\hat{\Btheta}}

%
% to write R^n and C^n in a distinguishable fashion.  Perhaps change this
% to the double lined characters upon figuring out how to do so.
%
\newcommand{\C}[1]{$\mathbb{C}^{#1}$}
\newcommand{\R}[1]{$\mathbb{R}^{#1}$}

%
% various generally useful helpers
%

% derivative of #1 wrt. #2:
\newcommand{\D}[2] {\frac {d#2} {d#1}}

\newcommand{\inv}[1]{\frac{1}{#1}}
\newcommand{\cross}[0]{\times}

\newcommand{\abs}[1]{\lvert{#1}\rvert}
\newcommand{\norm}[1]{\lVert{#1}\rVert}
\newcommand{\innerprod}[2]{\langle{#1}, {#2}\rangle}
\newcommand{\dotprod}[2]{{#1} \cdot {#2}}
\newcommand{\bdotprod}[2]{\left({#1} \cdot {#2}\right)}
\newcommand{\crossprod}[2]{{#1} \cross {#2}}
\newcommand{\tripleprod}[3]{\dotprod{\left(\crossprod{#1}{#2}\right)}{#3}}

\DeclareMathOperator{\Proj}{Proj}
\DeclareMathOperator{\Span}{span}
\DeclareMathOperator{\Sgn}{sgn}
\DeclareMathOperator{\Area}{Area}
\DeclareMathOperator{\Volume}{Volume}

%
% A few miscellaneous things specific to this document
%
\newcommand{\crossop}[1]{\crossprod{#1}{}}

% R2 vector.
\newcommand{\VectorTwo}[2]{
\begin{bmatrix}
 {#1} \\
 {#2}
\end{bmatrix}
}

\newcommand{\VectorN}[1]{
\begin{bmatrix}
{#1}_1 \\
{#1}_2 \\
\vdots \\
{#1}_N \\
\end{bmatrix}
}

\newcommand{\DETuvij}[4]{
\begin{vmatrix}
 {#1}_{#3} & {#1}_{#4} \\
 {#2}_{#3} & {#2}_{#4}
\end{vmatrix}
}

\newcommand{\DETuvwijk}[6]{
\begin{vmatrix}
 {#1}_{#4} & {#1}_{#5} & {#1}_{#6} \\
 {#2}_{#4} & {#2}_{#5} & {#2}_{#6} \\
 {#3}_{#4} & {#3}_{#5} & {#3}_{#6}
\end{vmatrix}
}

\newcommand{\DETuvwxijkl}[8]{
\begin{vmatrix}
 {#1}_{#5} & {#1}_{#6} & {#1}_{#7} & {#1}_{#8} \\
 {#2}_{#5} & {#2}_{#6} & {#2}_{#7} & {#2}_{#8} \\
 {#3}_{#5} & {#3}_{#6} & {#3}_{#7} & {#3}_{#8} \\
 {#4}_{#5} & {#4}_{#6} & {#4}_{#7} & {#4}_{#8} \\
\end{vmatrix}
}

%\newcommand{\DETuvwxyijklm}[10]{
%\begin{vmatrix}
% {#1}_{#6} & {#1}_{#7} & {#1}_{#8} & {#1}_{#9} & {#1}_{#10} \\
% {#2}_{#6} & {#2}_{#7} & {#2}_{#8} & {#2}_{#9} & {#2}_{#10} \\
% {#3}_{#6} & {#3}_{#7} & {#3}_{#8} & {#3}_{#9} & {#3}_{#10} \\
% {#4}_{#6} & {#4}_{#7} & {#4}_{#8} & {#4}_{#9} & {#4}_{#10} \\
% {#5}_{#6} & {#5}_{#7} & {#5}_{#8} & {#5}_{#9} & {#5}_{#10}
%\end{vmatrix}
%}

% R3 vector.
\newcommand{\VectorThree}[3]{
\begin{bmatrix}
 {#1} \\
 {#2} \\
 {#3}
\end{bmatrix}
}



\author{Peeter Joot}
\email{peeter.joot@gmail.com}


\chapter{Eigenfunctions by application of the raising operator.}
\label{chap:sphericalHarmonicRaising}
%\useCCL
\blogpage{http://sites.google.com/site/peeterjoot/math2009/sphericalHarmonicRaising.pdf}
\date{Aug 18, 2009}
\revisionInfo{$RCSfile: sphericalHarmonicRaising.tex,v $ Last $Revision: 1.1 $ $Date: 2009/08/18 04:06:10 $}

%\beginArtWithToc
\beginArtNoToc

\section{Motivation}

In Bohm's QT (\cite{bohm1989qt}, the following spherical harmonic eigenfunctions of the raising operator are found

\begin{align}\label{eqn:foo0}
\psi_l^{l-s} = \frac{e^{i(l-s)\phi}}{(1-\zeta^2)^{(l-s)/2}} \frac{\partial^s}{\partial \zeta^s} (1-\zeta^2)^l
\end{align}

This (unnormalized) result (with $\zeta = \cos\theta$) is valid for $s \in [0,l]$.  As an exersize do this by applying the raising operator to $\psi_l^{-l}$.  This should help verify the result (unproven or unclear if proven) that the $\psi_l^m$ and $\psi_l^{-m}$ eigenfunctions differ only by a sign in the $\psi$ phase term.

\section{Guts}

The staring point, with $C$ for $\cos$ and $S$ for $\sin$, will be equations (15) from the text

\begin{align*}
L_z/\hbar &= -i \partial_\phi \\
L_x/\hbar &= i (S_\phi \partial_\theta + \cot\theta C_\phi \partial_\phi) \\
L_y/\hbar &= -i (C_\phi \partial_\theta - \cot\theta S_\phi \partial_\phi)
\end{align*}

From these the raising and lowering operators (setting $\hbar=1$) are respectively

\begin{align*}
L_x \pm iL_y &= 
i (S_\phi \partial_\theta + \cot\theta C_\phi \partial_\phi)
\pm (C_\phi \partial_\theta - \cot\theta S_\phi \partial_\phi)
\end{align*}

\EndArticle
%\EndNoBibArticle

%\documentclass[]{eliblog}

\usepackage{amsmath}
\usepackage{mathpazo}

%
% shorthand for bold symbols, convenient for vectors and matrices
%
\newcommand{\Ba}[0]{\mathbf{a}}
\newcommand{\Bb}[0]{\mathbf{b}}
\newcommand{\Bc}[0]{\mathbf{c}}
\newcommand{\Bd}[0]{\mathbf{d}}
\newcommand{\Be}[0]{\mathbf{e}}
\newcommand{\Bf}[0]{\mathbf{f}}
\newcommand{\Bg}[0]{\mathbf{g}}
\newcommand{\Bh}[0]{\mathbf{h}}
\newcommand{\Bi}[0]{\mathbf{i}}
\newcommand{\Bj}[0]{\mathbf{j}}
\newcommand{\Bk}[0]{\mathbf{k}}
\newcommand{\Bl}[0]{\mathbf{l}}
\newcommand{\Bm}[0]{\mathbf{m}}
\newcommand{\Bn}[0]{\mathbf{n}}
\newcommand{\Bo}[0]{\mathbf{o}}
\newcommand{\Bp}[0]{\mathbf{p}}
\newcommand{\Bq}[0]{\mathbf{q}}
\newcommand{\Br}[0]{\mathbf{r}}
\newcommand{\Bs}[0]{\mathbf{s}}
\newcommand{\Bt}[0]{\mathbf{t}}
\newcommand{\Bu}[0]{\mathbf{u}}
\newcommand{\Bv}[0]{\mathbf{v}}
\newcommand{\Bw}[0]{\mathbf{w}}
\newcommand{\Bx}[0]{\mathbf{x}}
\newcommand{\By}[0]{\mathbf{y}}
\newcommand{\Bz}[0]{\mathbf{z}}
\newcommand{\BA}[0]{\mathbf{A}}
\newcommand{\BB}[0]{\mathbf{B}}
\newcommand{\BC}[0]{\mathbf{C}}
\newcommand{\BD}[0]{\mathbf{D}}
\newcommand{\BE}[0]{\mathbf{E}}
\newcommand{\BF}[0]{\mathbf{F}}
\newcommand{\BG}[0]{\mathbf{G}}
\newcommand{\BH}[0]{\mathbf{H}}
\newcommand{\BI}[0]{\mathbf{I}}
\newcommand{\BJ}[0]{\mathbf{J}}
\newcommand{\BK}[0]{\mathbf{K}}
\newcommand{\BL}[0]{\mathbf{L}}
\newcommand{\BM}[0]{\mathbf{M}}
\newcommand{\BN}[0]{\mathbf{N}}
\newcommand{\BO}[0]{\mathbf{O}}
\newcommand{\BP}[0]{\mathbf{P}}
\newcommand{\BQ}[0]{\mathbf{Q}}
\newcommand{\BR}[0]{\mathbf{R}}
\newcommand{\BS}[0]{\mathbf{S}}
\newcommand{\BT}[0]{\mathbf{T}}
\newcommand{\BU}[0]{\mathbf{U}}
\newcommand{\BV}[0]{\mathbf{V}}
\newcommand{\BW}[0]{\mathbf{W}}
\newcommand{\BX}[0]{\mathbf{X}}
\newcommand{\BY}[0]{\mathbf{Y}}
\newcommand{\BZ}[0]{\mathbf{Z}}

\newcommand{\Bzero}[0]{\mathbf{0}}
\newcommand{\Btheta}[0]{\boldsymbol{\theta}}
\newcommand{\Btau}[0]{\boldsymbol{\tau}}
\newcommand{\Bomega}[0]{\boldsymbol{\omega}}

%
% shorthand for unit vectors
%
\newcommand{\acap}[0]{\hat{\Ba}}
\newcommand{\bcap}[0]{\hat{\Bb}}
\newcommand{\ccap}[0]{\hat{\Bc}}
\newcommand{\dcap}[0]{\hat{\Bd}}
\newcommand{\ecap}[0]{\hat{\Be}}
\newcommand{\fcap}[0]{\hat{\Bf}}
\newcommand{\gcap}[0]{\hat{\Bg}}
\newcommand{\hcap}[0]{\hat{\Bh}}
\newcommand{\icap}[0]{\hat{\Bi}}
\newcommand{\jcap}[0]{\hat{\Bj}}
\newcommand{\kcap}[0]{\hat{\Bk}}
\newcommand{\lcap}[0]{\hat{\Bl}}
\newcommand{\mcap}[0]{\hat{\Bm}}
\newcommand{\ncap}[0]{\hat{\Bn}}
\newcommand{\ocap}[0]{\hat{\Bo}}
\newcommand{\pcap}[0]{\hat{\Bp}}
\newcommand{\qcap}[0]{\hat{\Bq}}
\newcommand{\rcap}[0]{\hat{\Br}}
\newcommand{\scap}[0]{\hat{\Bs}}
\newcommand{\tcap}[0]{\hat{\Bt}}
\newcommand{\ucap}[0]{\hat{\Bu}}
\newcommand{\vcap}[0]{\hat{\Bv}}
\newcommand{\wcap}[0]{\hat{\Bw}}
\newcommand{\xcap}[0]{\hat{\Bx}}
\newcommand{\ycap}[0]{\hat{\By}}
\newcommand{\zcap}[0]{\hat{\Bz}}
\newcommand{\thetacap}[0]{\hat{\Btheta}}

%
% to write R^n and C^n in a distinguishable fashion.  Perhaps change this
% to the double lined characters upon figuring out how to do so.
%
\newcommand{\C}[1]{$\mathbb{C}^{#1}$}
\newcommand{\R}[1]{$\mathbb{R}^{#1}$}

%
% various generally useful helpers
%

% derivative of #1 wrt. #2:
\newcommand{\D}[2] {\frac {d#2} {d#1}}

\newcommand{\inv}[1]{\frac{1}{#1}}
\newcommand{\cross}[0]{\times}

\newcommand{\abs}[1]{\lvert{#1}\rvert}
\newcommand{\norm}[1]{\lVert{#1}\rVert}
\newcommand{\innerprod}[2]{\langle{#1}, {#2}\rangle}
\newcommand{\dotprod}[2]{{#1} \cdot {#2}}
\newcommand{\bdotprod}[2]{\left({#1} \cdot {#2}\right)}
\newcommand{\crossprod}[2]{{#1} \cross {#2}}
\newcommand{\tripleprod}[3]{\dotprod{\left(\crossprod{#1}{#2}\right)}{#3}}

\DeclareMathOperator{\Proj}{Proj}
\DeclareMathOperator{\Span}{span}
\DeclareMathOperator{\Sgn}{sgn}
\DeclareMathOperator{\Area}{Area}
\DeclareMathOperator{\Volume}{Volume}

%
% A few miscellaneous things specific to this document
%
\newcommand{\crossop}[1]{\crossprod{#1}{}}

% R2 vector.
\newcommand{\VectorTwo}[2]{
\begin{bmatrix}
 {#1} \\
 {#2}
\end{bmatrix}
}

\newcommand{\VectorN}[1]{
\begin{bmatrix}
{#1}_1 \\
{#1}_2 \\
\vdots \\
{#1}_N \\
\end{bmatrix}
}

\newcommand{\DETuvij}[4]{
\begin{vmatrix}
 {#1}_{#3} & {#1}_{#4} \\
 {#2}_{#3} & {#2}_{#4}
\end{vmatrix}
}

\newcommand{\DETuvwijk}[6]{
\begin{vmatrix}
 {#1}_{#4} & {#1}_{#5} & {#1}_{#6} \\
 {#2}_{#4} & {#2}_{#5} & {#2}_{#6} \\
 {#3}_{#4} & {#3}_{#5} & {#3}_{#6}
\end{vmatrix}
}

\newcommand{\DETuvwxijkl}[8]{
\begin{vmatrix}
 {#1}_{#5} & {#1}_{#6} & {#1}_{#7} & {#1}_{#8} \\
 {#2}_{#5} & {#2}_{#6} & {#2}_{#7} & {#2}_{#8} \\
 {#3}_{#5} & {#3}_{#6} & {#3}_{#7} & {#3}_{#8} \\
 {#4}_{#5} & {#4}_{#6} & {#4}_{#7} & {#4}_{#8} \\
\end{vmatrix}
}

%\newcommand{\DETuvwxyijklm}[10]{
%\begin{vmatrix}
% {#1}_{#6} & {#1}_{#7} & {#1}_{#8} & {#1}_{#9} & {#1}_{#10} \\
% {#2}_{#6} & {#2}_{#7} & {#2}_{#8} & {#2}_{#9} & {#2}_{#10} \\
% {#3}_{#6} & {#3}_{#7} & {#3}_{#8} & {#3}_{#9} & {#3}_{#10} \\
% {#4}_{#6} & {#4}_{#7} & {#4}_{#8} & {#4}_{#9} & {#4}_{#10} \\
% {#5}_{#6} & {#5}_{#7} & {#5}_{#8} & {#5}_{#9} & {#5}_{#10}
%\end{vmatrix}
%}

% R3 vector.
\newcommand{\VectorThree}[3]{
\begin{bmatrix}
 {#1} \\
 {#2} \\
 {#3}
\end{bmatrix}
}



\author{Peeter Joot}
\email{peeter.joot@gmail.com}


\chapter{Two particle center of mass Laplacian change of variables.}
\label{chap:twoParticleCMLaplacian}
%\useCCL
\blogpage{http://sites.google.com/site/peeterjoot/math2009/twoParticleCMLaplacian.pdf}
\date{Nov 30, 2009}
\revisionInfo{$RCSfile: twoParticleCMLaplacian.tex,v $ Last $Revision: 1.4 $ $Date: 2009/12/03 03:24:40 $}

\beginArtWithToc
%\beginArtNoToc

Exercise 15.2 in \citep{bohm1989qt} is to do a center of mass change of variables for the two particle Hamiltonian

\begin{align}\label{eqn:bohm15:pr2:1}
H = 
- \frac{\hbar^2}{2 m_1} {\spacegrad_1}^2
- \frac{\hbar^2}{2 m_2} {\spacegrad_2}^2
+ V(\Br_1 -\Br_2).
\end{align}

Before trying this, I was surprised that this would result in a diagonal form for the transformed Hamiltonian, so it is well worth doing the problem to see why this is the case.  He uses

\begin{align}\label{eqn:bohm15:pr2:2}
\BXI &= \Br_1 - \Br_2 \\
\BEta &= \inv{M}( m_1 \Br_1 + m_2 \Br_2 ).
\end{align}

Lets use coordinates ${x_k}^{(1)}$ for $\Br_1$, and ${x_k}^{(2)}$ for $\Br_2$.  Expanding the first order partial operator for $\PDi{{x_1}^{(1)}}{}$ by chain rule in terms of $\BEta$, and $\BXI$ coordinates we have

\begin{align*}
\PD{x_1^{(1)}}{}
&=
\PD{x_1^{(1)}}{\eta_k} \PD{\eta_k}{}
+\PD{x_1^{(1)}}{\xi_k} \PD{\xi_k}{} \\
&=
\frac{m_1}{M} \PD{\eta_1}{}
+ \PD{\xi_1}{}.
\end{align*}

We also have

\begin{align*}
\PD{x_1^{(2)}}{}
&=
\PD{x_1^{(2)}}{\eta_k} \PD{\eta_k}{}
+\PD{x_1^{(2)}}{\xi_k} \PD{\xi_k}{} \\
&=
\frac{m_2}{M} \PD{\eta_1}{}
- \PD{\xi_1}{}.
\end{align*}

The second partials for these $x$ coordinates are not a diagonal quadratic second partial operator, but are instead

\begin{align}\label{eqn:bohm15:pr2:3}
\PD{x_1^{(1)}}{} \PD{x_1^{(1)}}{}
&=
\frac{(m_1)^2}{M^2} \frac{\partial^2}{\partial \eta_1 \partial \eta_1}{}
+\frac{\partial^2}{\partial \xi_1 \partial \xi_1}{}
+2 \frac{m_1}{M} \frac{\partial^2}{\partial \xi_1 \partial \eta_1}{} \\
\PD{x_1^{(2)}}{} \PD{x_1^{(2)}}{}
&=
\frac{(m_2)^2}{M^2} \frac{\partial^2}{\partial \eta_1 \partial \eta_1}{}
+\frac{\partial^2}{\partial \xi_1 \partial \xi_1}{}
-2 \frac{m_2}{M} \frac{\partial^2}{\partial \xi_1 \partial \eta_1}{}.
\end{align}

The desired result follows directly, since the mixed partial terms conveniently cancel when we sum $(1/m_1) \PDi{x_1^{(1)}}{} \PDi{x_1^{(1)}}{} +(1/m_2) \PDi{x_1^{(2)}}{} \PDi{x_1^{(2)}}{}$.  This leaves us with

\begin{align}\label{eqn:bohm15:pr2:4}
H = 
\frac{-\hbar^2}{2} \sum_{k=1}^3 \left( 
\inv{M} \frac{\partial^2}{\partial \eta_k \partial \eta_k}{}
+ \left( \inv{m_1} + \inv{m_2} \right) \frac{\partial^2}{\partial \xi_k \partial \xi_k}{}
\right)
+ V(\BXI),
\end{align}

With the shorthand of the text

\begin{align}\label{eqn:bohm15:pr2:5}
\spacegrad_{\BEta} &= \sum_k \frac{\partial^2}{\partial \eta_k \partial \eta_k}{} \\
\spacegrad_{\BXI} &= \sum_k \frac{\partial^2}{\partial \xi_k \partial \xi_k}{},
\end{align}

this is the result to be proven.

\EndArticle
%\EndNoBibArticle

%
% Copyright � 2012 Peeter Joot.  All Rights Reserved.
% Licenced as described in the file LICENSE under the root directory of this GIT repository.
%

% 
% 
%%
% Copyright � 2015 Peeter Joot.  All Rights Reserved.
% Licenced as described in the file LICENSE under the root directory of this GIT repository.
%
\documentclass[]{eliblog}

\usepackage{amsmath}
\usepackage{mathpazo}

%
% shorthand for bold symbols, convenient for vectors and matrices
%
\newcommand{\Ba}[0]{\mathbf{a}}
\newcommand{\Bb}[0]{\mathbf{b}}
\newcommand{\Bc}[0]{\mathbf{c}}
\newcommand{\Bd}[0]{\mathbf{d}}
\newcommand{\Be}[0]{\mathbf{e}}
\newcommand{\Bf}[0]{\mathbf{f}}
\newcommand{\Bg}[0]{\mathbf{g}}
\newcommand{\Bh}[0]{\mathbf{h}}
\newcommand{\Bi}[0]{\mathbf{i}}
\newcommand{\Bj}[0]{\mathbf{j}}
\newcommand{\Bk}[0]{\mathbf{k}}
\newcommand{\Bl}[0]{\mathbf{l}}
\newcommand{\Bm}[0]{\mathbf{m}}
\newcommand{\Bn}[0]{\mathbf{n}}
\newcommand{\Bo}[0]{\mathbf{o}}
\newcommand{\Bp}[0]{\mathbf{p}}
\newcommand{\Bq}[0]{\mathbf{q}}
\newcommand{\Br}[0]{\mathbf{r}}
\newcommand{\Bs}[0]{\mathbf{s}}
\newcommand{\Bt}[0]{\mathbf{t}}
\newcommand{\Bu}[0]{\mathbf{u}}
\newcommand{\Bv}[0]{\mathbf{v}}
\newcommand{\Bw}[0]{\mathbf{w}}
\newcommand{\Bx}[0]{\mathbf{x}}
\newcommand{\By}[0]{\mathbf{y}}
\newcommand{\Bz}[0]{\mathbf{z}}
\newcommand{\BA}[0]{\mathbf{A}}
\newcommand{\BB}[0]{\mathbf{B}}
\newcommand{\BC}[0]{\mathbf{C}}
\newcommand{\BD}[0]{\mathbf{D}}
\newcommand{\BE}[0]{\mathbf{E}}
\newcommand{\BF}[0]{\mathbf{F}}
\newcommand{\BG}[0]{\mathbf{G}}
\newcommand{\BH}[0]{\mathbf{H}}
\newcommand{\BI}[0]{\mathbf{I}}
\newcommand{\BJ}[0]{\mathbf{J}}
\newcommand{\BK}[0]{\mathbf{K}}
\newcommand{\BL}[0]{\mathbf{L}}
\newcommand{\BM}[0]{\mathbf{M}}
\newcommand{\BN}[0]{\mathbf{N}}
\newcommand{\BO}[0]{\mathbf{O}}
\newcommand{\BP}[0]{\mathbf{P}}
\newcommand{\BQ}[0]{\mathbf{Q}}
\newcommand{\BR}[0]{\mathbf{R}}
\newcommand{\BS}[0]{\mathbf{S}}
\newcommand{\BT}[0]{\mathbf{T}}
\newcommand{\BU}[0]{\mathbf{U}}
\newcommand{\BV}[0]{\mathbf{V}}
\newcommand{\BW}[0]{\mathbf{W}}
\newcommand{\BX}[0]{\mathbf{X}}
\newcommand{\BY}[0]{\mathbf{Y}}
\newcommand{\BZ}[0]{\mathbf{Z}}

\newcommand{\Bzero}[0]{\mathbf{0}}
\newcommand{\Btheta}[0]{\boldsymbol{\theta}}
\newcommand{\Btau}[0]{\boldsymbol{\tau}}
\newcommand{\Bomega}[0]{\boldsymbol{\omega}}

%
% shorthand for unit vectors
%
\newcommand{\acap}[0]{\hat{\Ba}}
\newcommand{\bcap}[0]{\hat{\Bb}}
\newcommand{\ccap}[0]{\hat{\Bc}}
\newcommand{\dcap}[0]{\hat{\Bd}}
\newcommand{\ecap}[0]{\hat{\Be}}
\newcommand{\fcap}[0]{\hat{\Bf}}
\newcommand{\gcap}[0]{\hat{\Bg}}
\newcommand{\hcap}[0]{\hat{\Bh}}
\newcommand{\icap}[0]{\hat{\Bi}}
\newcommand{\jcap}[0]{\hat{\Bj}}
\newcommand{\kcap}[0]{\hat{\Bk}}
\newcommand{\lcap}[0]{\hat{\Bl}}
\newcommand{\mcap}[0]{\hat{\Bm}}
\newcommand{\ncap}[0]{\hat{\Bn}}
\newcommand{\ocap}[0]{\hat{\Bo}}
\newcommand{\pcap}[0]{\hat{\Bp}}
\newcommand{\qcap}[0]{\hat{\Bq}}
\newcommand{\rcap}[0]{\hat{\Br}}
\newcommand{\scap}[0]{\hat{\Bs}}
\newcommand{\tcap}[0]{\hat{\Bt}}
\newcommand{\ucap}[0]{\hat{\Bu}}
\newcommand{\vcap}[0]{\hat{\Bv}}
\newcommand{\wcap}[0]{\hat{\Bw}}
\newcommand{\xcap}[0]{\hat{\Bx}}
\newcommand{\ycap}[0]{\hat{\By}}
\newcommand{\zcap}[0]{\hat{\Bz}}
\newcommand{\thetacap}[0]{\hat{\Btheta}}

%
% to write R^n and C^n in a distinguishable fashion.  Perhaps change this
% to the double lined characters upon figuring out how to do so.
%
\newcommand{\C}[1]{$\mathbb{C}^{#1}$}
\newcommand{\R}[1]{$\mathbb{R}^{#1}$}

%
% various generally useful helpers
%

% derivative of #1 wrt. #2:
\newcommand{\D}[2] {\frac {d#2} {d#1}}

\newcommand{\inv}[1]{\frac{1}{#1}}
\newcommand{\cross}[0]{\times}

\newcommand{\abs}[1]{\lvert{#1}\rvert}
\newcommand{\norm}[1]{\lVert{#1}\rVert}
\newcommand{\innerprod}[2]{\langle{#1}, {#2}\rangle}
\newcommand{\dotprod}[2]{{#1} \cdot {#2}}
\newcommand{\bdotprod}[2]{\left({#1} \cdot {#2}\right)}
\newcommand{\crossprod}[2]{{#1} \cross {#2}}
\newcommand{\tripleprod}[3]{\dotprod{\left(\crossprod{#1}{#2}\right)}{#3}}

\DeclareMathOperator{\Proj}{Proj}
\DeclareMathOperator{\Span}{span}
\DeclareMathOperator{\Sgn}{sgn}
\DeclareMathOperator{\Area}{Area}
\DeclareMathOperator{\Volume}{Volume}

%
% A few miscellaneous things specific to this document
%
\newcommand{\crossop}[1]{\crossprod{#1}{}}

% R2 vector.
\newcommand{\VectorTwo}[2]{
\begin{bmatrix}
 {#1} \\
 {#2}
\end{bmatrix}
}

\newcommand{\VectorN}[1]{
\begin{bmatrix}
{#1}_1 \\
{#1}_2 \\
\vdots \\
{#1}_N \\
\end{bmatrix}
}

\newcommand{\DETuvij}[4]{
\begin{vmatrix}
 {#1}_{#3} & {#1}_{#4} \\
 {#2}_{#3} & {#2}_{#4}
\end{vmatrix}
}

\newcommand{\DETuvwijk}[6]{
\begin{vmatrix}
 {#1}_{#4} & {#1}_{#5} & {#1}_{#6} \\
 {#2}_{#4} & {#2}_{#5} & {#2}_{#6} \\
 {#3}_{#4} & {#3}_{#5} & {#3}_{#6}
\end{vmatrix}
}

\newcommand{\DETuvwxijkl}[8]{
\begin{vmatrix}
 {#1}_{#5} & {#1}_{#6} & {#1}_{#7} & {#1}_{#8} \\
 {#2}_{#5} & {#2}_{#6} & {#2}_{#7} & {#2}_{#8} \\
 {#3}_{#5} & {#3}_{#6} & {#3}_{#7} & {#3}_{#8} \\
 {#4}_{#5} & {#4}_{#6} & {#4}_{#7} & {#4}_{#8} \\
\end{vmatrix}
}

%\newcommand{\DETuvwxyijklm}[10]{
%\begin{vmatrix}
% {#1}_{#6} & {#1}_{#7} & {#1}_{#8} & {#1}_{#9} & {#1}_{#10} \\
% {#2}_{#6} & {#2}_{#7} & {#2}_{#8} & {#2}_{#9} & {#2}_{#10} \\
% {#3}_{#6} & {#3}_{#7} & {#3}_{#8} & {#3}_{#9} & {#3}_{#10} \\
% {#4}_{#6} & {#4}_{#7} & {#4}_{#8} & {#4}_{#9} & {#4}_{#10} \\
% {#5}_{#6} & {#5}_{#7} & {#5}_{#8} & {#5}_{#9} & {#5}_{#10}
%\end{vmatrix}
%}

% R3 vector.
\newcommand{\VectorThree}[3]{
\begin{bmatrix}
 {#1} \\
 {#2} \\
 {#3}
\end{bmatrix}
}



\author{Peeter Joot}
\email{peeter.joot@gmail.com}

%\documentclass[]{eliblogwidescreen}

\usepackage{amsmath}
\usepackage{mathpazo}

%
% shorthand for bold symbols, convenient for vectors and matrices
%
\newcommand{\Ba}[0]{\mathbf{a}}
\newcommand{\Bb}[0]{\mathbf{b}}
\newcommand{\Bc}[0]{\mathbf{c}}
\newcommand{\Bd}[0]{\mathbf{d}}
\newcommand{\Be}[0]{\mathbf{e}}
\newcommand{\Bf}[0]{\mathbf{f}}
\newcommand{\Bg}[0]{\mathbf{g}}
\newcommand{\Bh}[0]{\mathbf{h}}
\newcommand{\Bi}[0]{\mathbf{i}}
\newcommand{\Bj}[0]{\mathbf{j}}
\newcommand{\Bk}[0]{\mathbf{k}}
\newcommand{\Bl}[0]{\mathbf{l}}
\newcommand{\Bm}[0]{\mathbf{m}}
\newcommand{\Bn}[0]{\mathbf{n}}
\newcommand{\Bo}[0]{\mathbf{o}}
\newcommand{\Bp}[0]{\mathbf{p}}
\newcommand{\Bq}[0]{\mathbf{q}}
\newcommand{\Br}[0]{\mathbf{r}}
\newcommand{\Bs}[0]{\mathbf{s}}
\newcommand{\Bt}[0]{\mathbf{t}}
\newcommand{\Bu}[0]{\mathbf{u}}
\newcommand{\Bv}[0]{\mathbf{v}}
\newcommand{\Bw}[0]{\mathbf{w}}
\newcommand{\Bx}[0]{\mathbf{x}}
\newcommand{\By}[0]{\mathbf{y}}
\newcommand{\Bz}[0]{\mathbf{z}}
\newcommand{\BA}[0]{\mathbf{A}}
\newcommand{\BB}[0]{\mathbf{B}}
\newcommand{\BC}[0]{\mathbf{C}}
\newcommand{\BD}[0]{\mathbf{D}}
\newcommand{\BE}[0]{\mathbf{E}}
\newcommand{\BF}[0]{\mathbf{F}}
\newcommand{\BG}[0]{\mathbf{G}}
\newcommand{\BH}[0]{\mathbf{H}}
\newcommand{\BI}[0]{\mathbf{I}}
\newcommand{\BJ}[0]{\mathbf{J}}
\newcommand{\BK}[0]{\mathbf{K}}
\newcommand{\BL}[0]{\mathbf{L}}
\newcommand{\BM}[0]{\mathbf{M}}
\newcommand{\BN}[0]{\mathbf{N}}
\newcommand{\BO}[0]{\mathbf{O}}
\newcommand{\BP}[0]{\mathbf{P}}
\newcommand{\BQ}[0]{\mathbf{Q}}
\newcommand{\BR}[0]{\mathbf{R}}
\newcommand{\BS}[0]{\mathbf{S}}
\newcommand{\BT}[0]{\mathbf{T}}
\newcommand{\BU}[0]{\mathbf{U}}
\newcommand{\BV}[0]{\mathbf{V}}
\newcommand{\BW}[0]{\mathbf{W}}
\newcommand{\BX}[0]{\mathbf{X}}
\newcommand{\BY}[0]{\mathbf{Y}}
\newcommand{\BZ}[0]{\mathbf{Z}}

\newcommand{\Bzero}[0]{\mathbf{0}}
\newcommand{\Btheta}[0]{\boldsymbol{\theta}}
\newcommand{\Btau}[0]{\boldsymbol{\tau}}
\newcommand{\Bomega}[0]{\boldsymbol{\omega}}

%
% shorthand for unit vectors
%
\newcommand{\acap}[0]{\hat{\Ba}}
\newcommand{\bcap}[0]{\hat{\Bb}}
\newcommand{\ccap}[0]{\hat{\Bc}}
\newcommand{\dcap}[0]{\hat{\Bd}}
\newcommand{\ecap}[0]{\hat{\Be}}
\newcommand{\fcap}[0]{\hat{\Bf}}
\newcommand{\gcap}[0]{\hat{\Bg}}
\newcommand{\hcap}[0]{\hat{\Bh}}
\newcommand{\icap}[0]{\hat{\Bi}}
\newcommand{\jcap}[0]{\hat{\Bj}}
\newcommand{\kcap}[0]{\hat{\Bk}}
\newcommand{\lcap}[0]{\hat{\Bl}}
\newcommand{\mcap}[0]{\hat{\Bm}}
\newcommand{\ncap}[0]{\hat{\Bn}}
\newcommand{\ocap}[0]{\hat{\Bo}}
\newcommand{\pcap}[0]{\hat{\Bp}}
\newcommand{\qcap}[0]{\hat{\Bq}}
\newcommand{\rcap}[0]{\hat{\Br}}
\newcommand{\scap}[0]{\hat{\Bs}}
\newcommand{\tcap}[0]{\hat{\Bt}}
\newcommand{\ucap}[0]{\hat{\Bu}}
\newcommand{\vcap}[0]{\hat{\Bv}}
\newcommand{\wcap}[0]{\hat{\Bw}}
\newcommand{\xcap}[0]{\hat{\Bx}}
\newcommand{\ycap}[0]{\hat{\By}}
\newcommand{\zcap}[0]{\hat{\Bz}}
\newcommand{\thetacap}[0]{\hat{\Btheta}}

%
% to write R^n and C^n in a distinguishable fashion.  Perhaps change this
% to the double lined characters upon figuring out how to do so.
%
\newcommand{\C}[1]{$\mathbb{C}^{#1}$}
\newcommand{\R}[1]{$\mathbb{R}^{#1}$}

%
% various generally useful helpers
%

% derivative of #1 wrt. #2:
\newcommand{\D}[2] {\frac {d#2} {d#1}}

\newcommand{\inv}[1]{\frac{1}{#1}}
\newcommand{\cross}[0]{\times}

\newcommand{\abs}[1]{\lvert{#1}\rvert}
\newcommand{\norm}[1]{\lVert{#1}\rVert}
\newcommand{\innerprod}[2]{\langle{#1}, {#2}\rangle}
\newcommand{\dotprod}[2]{{#1} \cdot {#2}}
\newcommand{\bdotprod}[2]{\left({#1} \cdot {#2}\right)}
\newcommand{\crossprod}[2]{{#1} \cross {#2}}
\newcommand{\tripleprod}[3]{\dotprod{\left(\crossprod{#1}{#2}\right)}{#3}}

\DeclareMathOperator{\Proj}{Proj}
\DeclareMathOperator{\Span}{span}
\DeclareMathOperator{\Sgn}{sgn}
\DeclareMathOperator{\Area}{Area}
\DeclareMathOperator{\Volume}{Volume}

%
% A few miscellaneous things specific to this document
%
\newcommand{\crossop}[1]{\crossprod{#1}{}}

% R2 vector.
\newcommand{\VectorTwo}[2]{
\begin{bmatrix}
 {#1} \\
 {#2}
\end{bmatrix}
}

\newcommand{\VectorN}[1]{
\begin{bmatrix}
{#1}_1 \\
{#1}_2 \\
\vdots \\
{#1}_N \\
\end{bmatrix}
}

\newcommand{\DETuvij}[4]{
\begin{vmatrix}
 {#1}_{#3} & {#1}_{#4} \\
 {#2}_{#3} & {#2}_{#4}
\end{vmatrix}
}

\newcommand{\DETuvwijk}[6]{
\begin{vmatrix}
 {#1}_{#4} & {#1}_{#5} & {#1}_{#6} \\
 {#2}_{#4} & {#2}_{#5} & {#2}_{#6} \\
 {#3}_{#4} & {#3}_{#5} & {#3}_{#6}
\end{vmatrix}
}

\newcommand{\DETuvwxijkl}[8]{
\begin{vmatrix}
 {#1}_{#5} & {#1}_{#6} & {#1}_{#7} & {#1}_{#8} \\
 {#2}_{#5} & {#2}_{#6} & {#2}_{#7} & {#2}_{#8} \\
 {#3}_{#5} & {#3}_{#6} & {#3}_{#7} & {#3}_{#8} \\
 {#4}_{#5} & {#4}_{#6} & {#4}_{#7} & {#4}_{#8} \\
\end{vmatrix}
}

%\newcommand{\DETuvwxyijklm}[10]{
%\begin{vmatrix}
% {#1}_{#6} & {#1}_{#7} & {#1}_{#8} & {#1}_{#9} & {#1}_{#10} \\
% {#2}_{#6} & {#2}_{#7} & {#2}_{#8} & {#2}_{#9} & {#2}_{#10} \\
% {#3}_{#6} & {#3}_{#7} & {#3}_{#8} & {#3}_{#9} & {#3}_{#10} \\
% {#4}_{#6} & {#4}_{#7} & {#4}_{#8} & {#4}_{#9} & {#4}_{#10} \\
% {#5}_{#6} & {#5}_{#7} & {#5}_{#8} & {#5}_{#9} & {#5}_{#10}
%\end{vmatrix}
%}

% R3 vector.
\newcommand{\VectorThree}[3]{
\begin{bmatrix}
 {#1} \\
 {#2} \\
 {#3}
\end{bmatrix}
}



\author{Peeter Joot}
\email{peeter.joot@gmail.com}


\chapter{Time evolution of some wave functions}
\label{chap:liboff314}
%\useCCL
\blogpage{http://sites.google.com/site/peeterjoot/math2010/liboff314.pdf}
\date{May 23, 2010}
\revisionInfo{liboff314.tex}

%\beginArtWithToc
\beginArtNoToc

\section{Motivation}

In \citep{liboff2003iqm} is problem 3.14, Describe the time evolution of the following wavefunctions

\begin{equation}\label{eqn:liboff314:21}
\begin{aligned}%\label{eqn:liboff314:1}
\psi_1 &= A \sin \omega t \cos k (x + c t) \\
\psi_2 &= A \sin 10^{-5} k x \cos k (x - c t) \\
\psi_3 &= A \cos k ( x - c t ) \sin 10^{-5} k (x - c t)
\end{aligned}
\end{equation}

This is not really a QM problem, but seems worthwhile anyways, because it is not obvious looking at the functions what this is.

\section{\texorpdfstring{\(\psi_3\)}{psi 3}}

Let us start in reverse order with \(\psi_3\), but in a slightly more general form that is less error prone to manipulate.  These wavefunctions can be viewed as superpositions, and expanding out as exponentials temporarily gets us to a form that makes this more obvious.

\begin{equation}\label{eqn:liboff314:41}
\begin{aligned}
\psi 
&= A \sin k_1 ( x + v_1 t) \cos k_2 ( x + v_2 t) \\
&= \frac{A}{4i} \left( e^{ i k_1 ( x + v_1 t)} - e^{ -i k_1 ( x + v_1 t)} \right) \left( e^{ i k_2 ( x + v_2 t)} + e^{ -i k_2 ( x + v_2 t)} \right) \\
&= \frac{A}{2} \left( 
\sin ((k_1 + k_2) x + (k_1 v_1 + k_2 v_2 ) t) 
+ \sin ((k_1 - k_2) x + (k_1 v_1 - k_2 v_2 ) t) \right) \\
&= \frac{A}{2} \left( 
\sin \left( (k_1 + k_2) \left(x + \frac{k_1 v_1 + k_2 v_2 }{k_1 + k_2} t\right) \right)
+\sin \left( (k_1 - k_2) \left(x + \frac{k_1 v_1 - k_2 v_2 }{k_1 - k_2} t\right) \right)
\right)
\end{aligned}
\end{equation}

Now the problem is simplified to observing how a wave of the form \(\phi = \sin \kappa (x + v t)\) propagates, or really the interaction of two such waves moving together or against each other, depending on the signs of the constants.  Let us now put in the constants for \(\psi_3\) to get a better feel for it

\begin{equation}\label{eqn:liboff314:61}
\begin{aligned}
\psi_3
&= \frac{A}{2} \left( 
\sin \left( 1.00001 k \left(x - c t\right) \right)
-\sin \left( 0.99999 k \left(x - c t\right) \right)
\right)
\end{aligned}
\end{equation}

It was not obvious from the original product of sinusoids form that the question asked about that the resulting wave form stays in phase for its time propagation, but we see that to be the case above.  This really just leaves some thought about the standing wave itself to understand what is happening.  For that, at time 0, we have a destructive interference superposition of two almost identical period standing waves.  That near perfect cancellation will likely leave an envelope, and \href{http://www.wolframalpha.com/input/?i=Plot[Cos[x]+Sin[0.00001+x],+{x,+-317000,+317000}]}{a Mathematica plot} gives a better feel for this waveform.  This in turn will propagate at light speed down the x-axis.  Because \(k_1\) is so small we have a nearly linear, and nearly flat, envelope for the \(\cos k x\), as can be expected near the origin since we have there

\begin{equation}\label{eqn:liboff314:81}
\begin{aligned}
\psi_3(x, 0) \approx A 10^{-5} k x \cos k x
\end{aligned}
\end{equation}

Comparing to \href{http://www.wolframalpha.com/input/?i=Plot[Cos[x]+Sin[0.00001+x],+{x,+-317,+317}]}{a smaller range plot}, one sees that it is necessary to increase the plot range significantly before seeing the oscillatory nature of the envelope.

\section{\texorpdfstring{\(\psi_2\)}{psi 2}}

For \(\psi_2\) we have almost the same wave function, but out sine term has no time variation.  What does this do to the waveform?  Let us see if a sum and difference of angles form sheds some light on that.  We have

\begin{equation}\label{eqn:liboff314:101}
\begin{aligned}
\psi 
&= A \sin k_1 x \cos k_2 ( x + v_2 t) \\
&= \frac{A}{4i} \left( e^{ i k_1 x } - e^{ -i k_1 x } \right) \left( e^{ i k_2 ( x + v_2 t)} + e^{ -i k_2 ( x + v_2 t)} \right) \\
&= \frac{A}{2} \left( 
\sin ((k_1 + k_2) x + k_2 v_2 t) 
+ \sin ((k_1 - k_2) x - k_2 v_2 t) \right) \\
&= \frac{A}{2} \left( 
\sin \left( (k_1 + k_2) \left(x + \frac{k_2 v_2 }{k_1 + k_2} t\right) \right)
+\sin \left( (k_1 - k_2) \left(x - \frac{k_2 v_2 }{k_1 - k_2} t\right) \right)
\right)
\end{aligned}
\end{equation}

Specifically for \(k_1 = 10^{-5} k\), and \(k_2 = k\), we have

\begin{equation}\label{eqn:liboff314:121}
\begin{aligned}
\psi_2
&= \frac{A}{2} \left( 
\sin \left( 1.00001 k \left(x - \frac{1}{1.00001} c t\right) \right)
-\sin \left( 0.99999 k \left(x + \frac{1}{0.99999 } c t\right) \right)
\right)
\end{aligned}
\end{equation}

Again at \(t=0\) we have a very widely spread envelope with rapid oscillations within it.  There is a very small difference in the rate that the two components waveforms will go out of phase, and each of the component waveforms is moving in the opposite directions.  What does that phase change do to the evolution?  Looking back to the original product of sinusoids form for \(\psi_2\) I believe this will just mean we have the phase shifting with time within the envelope.

\section{\texorpdfstring{\(\psi_1\)}{psi 1}}

Finally for the first wave function we have both of the sinusoid factors with time variation.  What does that expand out to in terms of superposition of fundamental frequencies?

\begin{equation}\label{eqn:liboff314:141}
\begin{aligned}
\psi_1
&= A \sin \omega t \cos k ( x + c t) \\
&= \frac{A}{4i} \left( e^{ i \omega t} - e^{ -i \omega t } \right) \left( e^{ i k ( x + c t)} + e^{ -i k ( x + c t)} \right) \\
&= \frac{A}{2} \left( 
\sin ( k (x + c t) + \omega t )
-\sin ( k (x + c t) - \omega t )
\right),
\end{aligned}
\end{equation}

or
\begin{equation}\label{eqn:liboff314:161}
\begin{aligned}
\psi_1
&= \frac{A}{2} \left( 
\sin ( k x + ( k c + \omega ) t )
-\sin ( k x + ( k c - \omega) t )
\right)
\end{aligned}
\end{equation}

We have the superposition of two \(\sin k x\) wave forms, destructively interfering with each other, one with phase changing at the angular rate \(k c + \omega\), and the other at the rate \(k c - \omega\).  What is the overall waveform?  It is still not obvious what this is.  I actually have the inclination to try to not treat these analytically, but pull out some graphing software.  Something like the real Mathematica software would be nice since it would allow for the use of sliders to vary parameters and then animate the graphs as time varied.

\EndArticle

% 
% 
% 
% Copyright � 2012 Peeter Joot
% All Rights Reserved
% 
% This file may be reproduced and distributed in whole or in part, without fee, subject to the following conditions:
% 
% o The copyright notice above and this permission notice must be preserved complete on all complete or partial copies.
% 
% o Any translation or derived work must be approved by the author in writing before distribution.
% 
% o If you distribute this work in part, instructions for obtaining the complete version of this file must be included, and a means for obtaining a complete version provided.
% 
% 
% Exceptions to these rules may be granted for academic purposes: Write to the author and ask.
% 
% 
% 
%%
% Copyright � 2015 Peeter Joot.  All Rights Reserved.
% Licenced as described in the file LICENSE under the root directory of this GIT repository.
%
\documentclass[]{eliblog}

\usepackage{amsmath}
\usepackage{mathpazo}

%
% shorthand for bold symbols, convenient for vectors and matrices
%
\newcommand{\Ba}[0]{\mathbf{a}}
\newcommand{\Bb}[0]{\mathbf{b}}
\newcommand{\Bc}[0]{\mathbf{c}}
\newcommand{\Bd}[0]{\mathbf{d}}
\newcommand{\Be}[0]{\mathbf{e}}
\newcommand{\Bf}[0]{\mathbf{f}}
\newcommand{\Bg}[0]{\mathbf{g}}
\newcommand{\Bh}[0]{\mathbf{h}}
\newcommand{\Bi}[0]{\mathbf{i}}
\newcommand{\Bj}[0]{\mathbf{j}}
\newcommand{\Bk}[0]{\mathbf{k}}
\newcommand{\Bl}[0]{\mathbf{l}}
\newcommand{\Bm}[0]{\mathbf{m}}
\newcommand{\Bn}[0]{\mathbf{n}}
\newcommand{\Bo}[0]{\mathbf{o}}
\newcommand{\Bp}[0]{\mathbf{p}}
\newcommand{\Bq}[0]{\mathbf{q}}
\newcommand{\Br}[0]{\mathbf{r}}
\newcommand{\Bs}[0]{\mathbf{s}}
\newcommand{\Bt}[0]{\mathbf{t}}
\newcommand{\Bu}[0]{\mathbf{u}}
\newcommand{\Bv}[0]{\mathbf{v}}
\newcommand{\Bw}[0]{\mathbf{w}}
\newcommand{\Bx}[0]{\mathbf{x}}
\newcommand{\By}[0]{\mathbf{y}}
\newcommand{\Bz}[0]{\mathbf{z}}
\newcommand{\BA}[0]{\mathbf{A}}
\newcommand{\BB}[0]{\mathbf{B}}
\newcommand{\BC}[0]{\mathbf{C}}
\newcommand{\BD}[0]{\mathbf{D}}
\newcommand{\BE}[0]{\mathbf{E}}
\newcommand{\BF}[0]{\mathbf{F}}
\newcommand{\BG}[0]{\mathbf{G}}
\newcommand{\BH}[0]{\mathbf{H}}
\newcommand{\BI}[0]{\mathbf{I}}
\newcommand{\BJ}[0]{\mathbf{J}}
\newcommand{\BK}[0]{\mathbf{K}}
\newcommand{\BL}[0]{\mathbf{L}}
\newcommand{\BM}[0]{\mathbf{M}}
\newcommand{\BN}[0]{\mathbf{N}}
\newcommand{\BO}[0]{\mathbf{O}}
\newcommand{\BP}[0]{\mathbf{P}}
\newcommand{\BQ}[0]{\mathbf{Q}}
\newcommand{\BR}[0]{\mathbf{R}}
\newcommand{\BS}[0]{\mathbf{S}}
\newcommand{\BT}[0]{\mathbf{T}}
\newcommand{\BU}[0]{\mathbf{U}}
\newcommand{\BV}[0]{\mathbf{V}}
\newcommand{\BW}[0]{\mathbf{W}}
\newcommand{\BX}[0]{\mathbf{X}}
\newcommand{\BY}[0]{\mathbf{Y}}
\newcommand{\BZ}[0]{\mathbf{Z}}

\newcommand{\Bzero}[0]{\mathbf{0}}
\newcommand{\Btheta}[0]{\boldsymbol{\theta}}
\newcommand{\Btau}[0]{\boldsymbol{\tau}}
\newcommand{\Bomega}[0]{\boldsymbol{\omega}}

%
% shorthand for unit vectors
%
\newcommand{\acap}[0]{\hat{\Ba}}
\newcommand{\bcap}[0]{\hat{\Bb}}
\newcommand{\ccap}[0]{\hat{\Bc}}
\newcommand{\dcap}[0]{\hat{\Bd}}
\newcommand{\ecap}[0]{\hat{\Be}}
\newcommand{\fcap}[0]{\hat{\Bf}}
\newcommand{\gcap}[0]{\hat{\Bg}}
\newcommand{\hcap}[0]{\hat{\Bh}}
\newcommand{\icap}[0]{\hat{\Bi}}
\newcommand{\jcap}[0]{\hat{\Bj}}
\newcommand{\kcap}[0]{\hat{\Bk}}
\newcommand{\lcap}[0]{\hat{\Bl}}
\newcommand{\mcap}[0]{\hat{\Bm}}
\newcommand{\ncap}[0]{\hat{\Bn}}
\newcommand{\ocap}[0]{\hat{\Bo}}
\newcommand{\pcap}[0]{\hat{\Bp}}
\newcommand{\qcap}[0]{\hat{\Bq}}
\newcommand{\rcap}[0]{\hat{\Br}}
\newcommand{\scap}[0]{\hat{\Bs}}
\newcommand{\tcap}[0]{\hat{\Bt}}
\newcommand{\ucap}[0]{\hat{\Bu}}
\newcommand{\vcap}[0]{\hat{\Bv}}
\newcommand{\wcap}[0]{\hat{\Bw}}
\newcommand{\xcap}[0]{\hat{\Bx}}
\newcommand{\ycap}[0]{\hat{\By}}
\newcommand{\zcap}[0]{\hat{\Bz}}
\newcommand{\thetacap}[0]{\hat{\Btheta}}

%
% to write R^n and C^n in a distinguishable fashion.  Perhaps change this
% to the double lined characters upon figuring out how to do so.
%
\newcommand{\C}[1]{$\mathbb{C}^{#1}$}
\newcommand{\R}[1]{$\mathbb{R}^{#1}$}

%
% various generally useful helpers
%

% derivative of #1 wrt. #2:
\newcommand{\D}[2] {\frac {d#2} {d#1}}

\newcommand{\inv}[1]{\frac{1}{#1}}
\newcommand{\cross}[0]{\times}

\newcommand{\abs}[1]{\lvert{#1}\rvert}
\newcommand{\norm}[1]{\lVert{#1}\rVert}
\newcommand{\innerprod}[2]{\langle{#1}, {#2}\rangle}
\newcommand{\dotprod}[2]{{#1} \cdot {#2}}
\newcommand{\bdotprod}[2]{\left({#1} \cdot {#2}\right)}
\newcommand{\crossprod}[2]{{#1} \cross {#2}}
\newcommand{\tripleprod}[3]{\dotprod{\left(\crossprod{#1}{#2}\right)}{#3}}

\DeclareMathOperator{\Proj}{Proj}
\DeclareMathOperator{\Span}{span}
\DeclareMathOperator{\Sgn}{sgn}
\DeclareMathOperator{\Area}{Area}
\DeclareMathOperator{\Volume}{Volume}

%
% A few miscellaneous things specific to this document
%
\newcommand{\crossop}[1]{\crossprod{#1}{}}

% R2 vector.
\newcommand{\VectorTwo}[2]{
\begin{bmatrix}
 {#1} \\
 {#2}
\end{bmatrix}
}

\newcommand{\VectorN}[1]{
\begin{bmatrix}
{#1}_1 \\
{#1}_2 \\
\vdots \\
{#1}_N \\
\end{bmatrix}
}

\newcommand{\DETuvij}[4]{
\begin{vmatrix}
 {#1}_{#3} & {#1}_{#4} \\
 {#2}_{#3} & {#2}_{#4}
\end{vmatrix}
}

\newcommand{\DETuvwijk}[6]{
\begin{vmatrix}
 {#1}_{#4} & {#1}_{#5} & {#1}_{#6} \\
 {#2}_{#4} & {#2}_{#5} & {#2}_{#6} \\
 {#3}_{#4} & {#3}_{#5} & {#3}_{#6}
\end{vmatrix}
}

\newcommand{\DETuvwxijkl}[8]{
\begin{vmatrix}
 {#1}_{#5} & {#1}_{#6} & {#1}_{#7} & {#1}_{#8} \\
 {#2}_{#5} & {#2}_{#6} & {#2}_{#7} & {#2}_{#8} \\
 {#3}_{#5} & {#3}_{#6} & {#3}_{#7} & {#3}_{#8} \\
 {#4}_{#5} & {#4}_{#6} & {#4}_{#7} & {#4}_{#8} \\
\end{vmatrix}
}

%\newcommand{\DETuvwxyijklm}[10]{
%\begin{vmatrix}
% {#1}_{#6} & {#1}_{#7} & {#1}_{#8} & {#1}_{#9} & {#1}_{#10} \\
% {#2}_{#6} & {#2}_{#7} & {#2}_{#8} & {#2}_{#9} & {#2}_{#10} \\
% {#3}_{#6} & {#3}_{#7} & {#3}_{#8} & {#3}_{#9} & {#3}_{#10} \\
% {#4}_{#6} & {#4}_{#7} & {#4}_{#8} & {#4}_{#9} & {#4}_{#10} \\
% {#5}_{#6} & {#5}_{#7} & {#5}_{#8} & {#5}_{#9} & {#5}_{#10}
%\end{vmatrix}
%}

% R3 vector.
\newcommand{\VectorThree}[3]{
\begin{bmatrix}
 {#1} \\
 {#2} \\
 {#3}
\end{bmatrix}
}



\author{Peeter Joot}
\email{peeter.joot@gmail.com}

%\documentclass[]{eliblogwidescreen}

\usepackage{amsmath}
\usepackage{mathpazo}

%
% shorthand for bold symbols, convenient for vectors and matrices
%
\newcommand{\Ba}[0]{\mathbf{a}}
\newcommand{\Bb}[0]{\mathbf{b}}
\newcommand{\Bc}[0]{\mathbf{c}}
\newcommand{\Bd}[0]{\mathbf{d}}
\newcommand{\Be}[0]{\mathbf{e}}
\newcommand{\Bf}[0]{\mathbf{f}}
\newcommand{\Bg}[0]{\mathbf{g}}
\newcommand{\Bh}[0]{\mathbf{h}}
\newcommand{\Bi}[0]{\mathbf{i}}
\newcommand{\Bj}[0]{\mathbf{j}}
\newcommand{\Bk}[0]{\mathbf{k}}
\newcommand{\Bl}[0]{\mathbf{l}}
\newcommand{\Bm}[0]{\mathbf{m}}
\newcommand{\Bn}[0]{\mathbf{n}}
\newcommand{\Bo}[0]{\mathbf{o}}
\newcommand{\Bp}[0]{\mathbf{p}}
\newcommand{\Bq}[0]{\mathbf{q}}
\newcommand{\Br}[0]{\mathbf{r}}
\newcommand{\Bs}[0]{\mathbf{s}}
\newcommand{\Bt}[0]{\mathbf{t}}
\newcommand{\Bu}[0]{\mathbf{u}}
\newcommand{\Bv}[0]{\mathbf{v}}
\newcommand{\Bw}[0]{\mathbf{w}}
\newcommand{\Bx}[0]{\mathbf{x}}
\newcommand{\By}[0]{\mathbf{y}}
\newcommand{\Bz}[0]{\mathbf{z}}
\newcommand{\BA}[0]{\mathbf{A}}
\newcommand{\BB}[0]{\mathbf{B}}
\newcommand{\BC}[0]{\mathbf{C}}
\newcommand{\BD}[0]{\mathbf{D}}
\newcommand{\BE}[0]{\mathbf{E}}
\newcommand{\BF}[0]{\mathbf{F}}
\newcommand{\BG}[0]{\mathbf{G}}
\newcommand{\BH}[0]{\mathbf{H}}
\newcommand{\BI}[0]{\mathbf{I}}
\newcommand{\BJ}[0]{\mathbf{J}}
\newcommand{\BK}[0]{\mathbf{K}}
\newcommand{\BL}[0]{\mathbf{L}}
\newcommand{\BM}[0]{\mathbf{M}}
\newcommand{\BN}[0]{\mathbf{N}}
\newcommand{\BO}[0]{\mathbf{O}}
\newcommand{\BP}[0]{\mathbf{P}}
\newcommand{\BQ}[0]{\mathbf{Q}}
\newcommand{\BR}[0]{\mathbf{R}}
\newcommand{\BS}[0]{\mathbf{S}}
\newcommand{\BT}[0]{\mathbf{T}}
\newcommand{\BU}[0]{\mathbf{U}}
\newcommand{\BV}[0]{\mathbf{V}}
\newcommand{\BW}[0]{\mathbf{W}}
\newcommand{\BX}[0]{\mathbf{X}}
\newcommand{\BY}[0]{\mathbf{Y}}
\newcommand{\BZ}[0]{\mathbf{Z}}

\newcommand{\Bzero}[0]{\mathbf{0}}
\newcommand{\Btheta}[0]{\boldsymbol{\theta}}
\newcommand{\Btau}[0]{\boldsymbol{\tau}}
\newcommand{\Bomega}[0]{\boldsymbol{\omega}}

%
% shorthand for unit vectors
%
\newcommand{\acap}[0]{\hat{\Ba}}
\newcommand{\bcap}[0]{\hat{\Bb}}
\newcommand{\ccap}[0]{\hat{\Bc}}
\newcommand{\dcap}[0]{\hat{\Bd}}
\newcommand{\ecap}[0]{\hat{\Be}}
\newcommand{\fcap}[0]{\hat{\Bf}}
\newcommand{\gcap}[0]{\hat{\Bg}}
\newcommand{\hcap}[0]{\hat{\Bh}}
\newcommand{\icap}[0]{\hat{\Bi}}
\newcommand{\jcap}[0]{\hat{\Bj}}
\newcommand{\kcap}[0]{\hat{\Bk}}
\newcommand{\lcap}[0]{\hat{\Bl}}
\newcommand{\mcap}[0]{\hat{\Bm}}
\newcommand{\ncap}[0]{\hat{\Bn}}
\newcommand{\ocap}[0]{\hat{\Bo}}
\newcommand{\pcap}[0]{\hat{\Bp}}
\newcommand{\qcap}[0]{\hat{\Bq}}
\newcommand{\rcap}[0]{\hat{\Br}}
\newcommand{\scap}[0]{\hat{\Bs}}
\newcommand{\tcap}[0]{\hat{\Bt}}
\newcommand{\ucap}[0]{\hat{\Bu}}
\newcommand{\vcap}[0]{\hat{\Bv}}
\newcommand{\wcap}[0]{\hat{\Bw}}
\newcommand{\xcap}[0]{\hat{\Bx}}
\newcommand{\ycap}[0]{\hat{\By}}
\newcommand{\zcap}[0]{\hat{\Bz}}
\newcommand{\thetacap}[0]{\hat{\Btheta}}

%
% to write R^n and C^n in a distinguishable fashion.  Perhaps change this
% to the double lined characters upon figuring out how to do so.
%
\newcommand{\C}[1]{$\mathbb{C}^{#1}$}
\newcommand{\R}[1]{$\mathbb{R}^{#1}$}

%
% various generally useful helpers
%

% derivative of #1 wrt. #2:
\newcommand{\D}[2] {\frac {d#2} {d#1}}

\newcommand{\inv}[1]{\frac{1}{#1}}
\newcommand{\cross}[0]{\times}

\newcommand{\abs}[1]{\lvert{#1}\rvert}
\newcommand{\norm}[1]{\lVert{#1}\rVert}
\newcommand{\innerprod}[2]{\langle{#1}, {#2}\rangle}
\newcommand{\dotprod}[2]{{#1} \cdot {#2}}
\newcommand{\bdotprod}[2]{\left({#1} \cdot {#2}\right)}
\newcommand{\crossprod}[2]{{#1} \cross {#2}}
\newcommand{\tripleprod}[3]{\dotprod{\left(\crossprod{#1}{#2}\right)}{#3}}

\DeclareMathOperator{\Proj}{Proj}
\DeclareMathOperator{\Span}{span}
\DeclareMathOperator{\Sgn}{sgn}
\DeclareMathOperator{\Area}{Area}
\DeclareMathOperator{\Volume}{Volume}

%
% A few miscellaneous things specific to this document
%
\newcommand{\crossop}[1]{\crossprod{#1}{}}

% R2 vector.
\newcommand{\VectorTwo}[2]{
\begin{bmatrix}
 {#1} \\
 {#2}
\end{bmatrix}
}

\newcommand{\VectorN}[1]{
\begin{bmatrix}
{#1}_1 \\
{#1}_2 \\
\vdots \\
{#1}_N \\
\end{bmatrix}
}

\newcommand{\DETuvij}[4]{
\begin{vmatrix}
 {#1}_{#3} & {#1}_{#4} \\
 {#2}_{#3} & {#2}_{#4}
\end{vmatrix}
}

\newcommand{\DETuvwijk}[6]{
\begin{vmatrix}
 {#1}_{#4} & {#1}_{#5} & {#1}_{#6} \\
 {#2}_{#4} & {#2}_{#5} & {#2}_{#6} \\
 {#3}_{#4} & {#3}_{#5} & {#3}_{#6}
\end{vmatrix}
}

\newcommand{\DETuvwxijkl}[8]{
\begin{vmatrix}
 {#1}_{#5} & {#1}_{#6} & {#1}_{#7} & {#1}_{#8} \\
 {#2}_{#5} & {#2}_{#6} & {#2}_{#7} & {#2}_{#8} \\
 {#3}_{#5} & {#3}_{#6} & {#3}_{#7} & {#3}_{#8} \\
 {#4}_{#5} & {#4}_{#6} & {#4}_{#7} & {#4}_{#8} \\
\end{vmatrix}
}

%\newcommand{\DETuvwxyijklm}[10]{
%\begin{vmatrix}
% {#1}_{#6} & {#1}_{#7} & {#1}_{#8} & {#1}_{#9} & {#1}_{#10} \\
% {#2}_{#6} & {#2}_{#7} & {#2}_{#8} & {#2}_{#9} & {#2}_{#10} \\
% {#3}_{#6} & {#3}_{#7} & {#3}_{#8} & {#3}_{#9} & {#3}_{#10} \\
% {#4}_{#6} & {#4}_{#7} & {#4}_{#8} & {#4}_{#9} & {#4}_{#10} \\
% {#5}_{#6} & {#5}_{#7} & {#5}_{#8} & {#5}_{#9} & {#5}_{#10}
%\end{vmatrix}
%}

% R3 vector.
\newcommand{\VectorThree}[3]{
\begin{bmatrix}
 {#1} \\
 {#2} \\
 {#3}
\end{bmatrix}
}



\author{Peeter Joot}
\email{peeter.joot@gmail.com}


\chapter{Effect of sinusoid operators}
\label{chap:liboff319}
%\useCCL
\blogpage{http://sites.google.com/site/peeterjoot/math2010/liboff319.pdf}
\date{May 23, 2010}
\revisionInfo{liboff319.tex}

%\beginArtWithToc
\beginArtNoToc

\section{Problem 3.19.}

\citep{liboff2003iqm} problem 3.19 is

What is the effect of operating on an arbitrary function $f(x)$ with the following two operators

\begin{subequations}
\label{eqn:liboff319:1}
\begin{align}
\hat{O}_1 &\equiv \partial^2/\partial x^2 - 1 
+ sin^2 (\partial^3/\partial x^3)
+ cos^2 (\partial^3/\partial x^3) \\
\hat{O}_2 &\equiv 
+ cos (2 \partial/\partial x) 
+ sin^2 (\partial/\partial x)
+ \int_a^b dx
\end{align}
\end{subequations}

On the surface with $\sin^2 y + \cos^2 y = 1$ and $\cos 2y + 2 \sin^2 y = 1$ it appears that we have just
\begin{subequations}
\label{eqn:liboff319:2}
\begin{align}
\hat{O}_1 &\equiv \partial^2/\partial x^2  \\
\hat{O}_2 &\equiv 1 + \int_a^b dx
\end{align}
\end{subequations}

but it this justified when the sinusoids are functions of operators?  Let's look at the first case.  For some operator $\hat{f}$ we have

\begin{align*}
\sin^2 \hat{f} + \cos^2 \hat{f}
&=
-\inv{4} \left( 
e^{i\hat{f}} -e^{-i\hat{f}}
\right)
\left( 
e^{i\hat{f}} -e^{-i\hat{f}}
\right)
+\inv{4} \left( 
e^{i\hat{f}} +e^{-i\hat{f}}
\right)
\left( 
e^{i\hat{f}} +e^{-i\hat{f}}
\right) \\
&=
\inv{2} \left(
e^{i\hat{f}} e^{-i\hat{f}} +e^{-i\hat{f}} e^{i\hat{f}}
\right)
\end{align*}

Can we assume that these cancel for general operators?  How about for our specific differential operator $\hat{f} = \partial^3/\partial x^3$?  For that one we have

\begin{align*}
e^{i \partial^3/\partial x^3} e^{-i \partial^3/\partial x^3} g(x)
&=
\sum_{k=0}^\infty 
\inv{k!} 
\left(\frac{\partial^3}{\partial x^3}\right)^k
\sum_{m=0}^\infty 
\inv{m!} 
\left(\frac{\partial^3}{\partial x^3}\right)^m g(x)
\end{align*}

Since the differentials commute, so do the exponentials and we can write the slightly simpler

\begin{align*}
\sin^2 \hat{f} + \cos^2 \hat{f} = e^{i\hat{f}} e^{-i\hat{f}} 
\end{align*}

I'm pretty sure the commutative property of this differential operator would also allow us to say (in this case at least)

\begin{align*}
\sin^2 \hat{f} + \cos^2 \hat{f} = 1
\end{align*}

Will have to look up the combinatoric argument that allows one to write, for numbers,

\begin{align*}
e^x e^y = 
\sum_{k=0}^\infty 
\inv{k!} x^k 
\sum_{m=0}^\infty 
\inv{m!} y^m 
=
\sum_{j=0}^\infty 
\inv{j!} (x+y)^j 
= e^{x+y}
\end{align*}

If this only assumes that $x$ and $y$ commute, and not any other numeric properties then we have the supposed result \ref{eqn:liboff319:2}.  We also know of algebraic objects where this does not hold.  One example is exponentials of non-commuting square matrices, and other is non-commuting bivector exponentials.

\EndArticle

%
% Copyright � 2012 Peeter Joot.  All Rights Reserved.
% Licenced as described in the file LICENSE under the root directory of this GIT repository.
%

% 
% 
%%
% Copyright � 2015 Peeter Joot.  All Rights Reserved.
% Licenced as described in the file LICENSE under the root directory of this GIT repository.
%
\documentclass[]{eliblog}

\usepackage{amsmath}
\usepackage{mathpazo}

%
% shorthand for bold symbols, convenient for vectors and matrices
%
\newcommand{\Ba}[0]{\mathbf{a}}
\newcommand{\Bb}[0]{\mathbf{b}}
\newcommand{\Bc}[0]{\mathbf{c}}
\newcommand{\Bd}[0]{\mathbf{d}}
\newcommand{\Be}[0]{\mathbf{e}}
\newcommand{\Bf}[0]{\mathbf{f}}
\newcommand{\Bg}[0]{\mathbf{g}}
\newcommand{\Bh}[0]{\mathbf{h}}
\newcommand{\Bi}[0]{\mathbf{i}}
\newcommand{\Bj}[0]{\mathbf{j}}
\newcommand{\Bk}[0]{\mathbf{k}}
\newcommand{\Bl}[0]{\mathbf{l}}
\newcommand{\Bm}[0]{\mathbf{m}}
\newcommand{\Bn}[0]{\mathbf{n}}
\newcommand{\Bo}[0]{\mathbf{o}}
\newcommand{\Bp}[0]{\mathbf{p}}
\newcommand{\Bq}[0]{\mathbf{q}}
\newcommand{\Br}[0]{\mathbf{r}}
\newcommand{\Bs}[0]{\mathbf{s}}
\newcommand{\Bt}[0]{\mathbf{t}}
\newcommand{\Bu}[0]{\mathbf{u}}
\newcommand{\Bv}[0]{\mathbf{v}}
\newcommand{\Bw}[0]{\mathbf{w}}
\newcommand{\Bx}[0]{\mathbf{x}}
\newcommand{\By}[0]{\mathbf{y}}
\newcommand{\Bz}[0]{\mathbf{z}}
\newcommand{\BA}[0]{\mathbf{A}}
\newcommand{\BB}[0]{\mathbf{B}}
\newcommand{\BC}[0]{\mathbf{C}}
\newcommand{\BD}[0]{\mathbf{D}}
\newcommand{\BE}[0]{\mathbf{E}}
\newcommand{\BF}[0]{\mathbf{F}}
\newcommand{\BG}[0]{\mathbf{G}}
\newcommand{\BH}[0]{\mathbf{H}}
\newcommand{\BI}[0]{\mathbf{I}}
\newcommand{\BJ}[0]{\mathbf{J}}
\newcommand{\BK}[0]{\mathbf{K}}
\newcommand{\BL}[0]{\mathbf{L}}
\newcommand{\BM}[0]{\mathbf{M}}
\newcommand{\BN}[0]{\mathbf{N}}
\newcommand{\BO}[0]{\mathbf{O}}
\newcommand{\BP}[0]{\mathbf{P}}
\newcommand{\BQ}[0]{\mathbf{Q}}
\newcommand{\BR}[0]{\mathbf{R}}
\newcommand{\BS}[0]{\mathbf{S}}
\newcommand{\BT}[0]{\mathbf{T}}
\newcommand{\BU}[0]{\mathbf{U}}
\newcommand{\BV}[0]{\mathbf{V}}
\newcommand{\BW}[0]{\mathbf{W}}
\newcommand{\BX}[0]{\mathbf{X}}
\newcommand{\BY}[0]{\mathbf{Y}}
\newcommand{\BZ}[0]{\mathbf{Z}}

\newcommand{\Bzero}[0]{\mathbf{0}}
\newcommand{\Btheta}[0]{\boldsymbol{\theta}}
\newcommand{\Btau}[0]{\boldsymbol{\tau}}
\newcommand{\Bomega}[0]{\boldsymbol{\omega}}

%
% shorthand for unit vectors
%
\newcommand{\acap}[0]{\hat{\Ba}}
\newcommand{\bcap}[0]{\hat{\Bb}}
\newcommand{\ccap}[0]{\hat{\Bc}}
\newcommand{\dcap}[0]{\hat{\Bd}}
\newcommand{\ecap}[0]{\hat{\Be}}
\newcommand{\fcap}[0]{\hat{\Bf}}
\newcommand{\gcap}[0]{\hat{\Bg}}
\newcommand{\hcap}[0]{\hat{\Bh}}
\newcommand{\icap}[0]{\hat{\Bi}}
\newcommand{\jcap}[0]{\hat{\Bj}}
\newcommand{\kcap}[0]{\hat{\Bk}}
\newcommand{\lcap}[0]{\hat{\Bl}}
\newcommand{\mcap}[0]{\hat{\Bm}}
\newcommand{\ncap}[0]{\hat{\Bn}}
\newcommand{\ocap}[0]{\hat{\Bo}}
\newcommand{\pcap}[0]{\hat{\Bp}}
\newcommand{\qcap}[0]{\hat{\Bq}}
\newcommand{\rcap}[0]{\hat{\Br}}
\newcommand{\scap}[0]{\hat{\Bs}}
\newcommand{\tcap}[0]{\hat{\Bt}}
\newcommand{\ucap}[0]{\hat{\Bu}}
\newcommand{\vcap}[0]{\hat{\Bv}}
\newcommand{\wcap}[0]{\hat{\Bw}}
\newcommand{\xcap}[0]{\hat{\Bx}}
\newcommand{\ycap}[0]{\hat{\By}}
\newcommand{\zcap}[0]{\hat{\Bz}}
\newcommand{\thetacap}[0]{\hat{\Btheta}}

%
% to write R^n and C^n in a distinguishable fashion.  Perhaps change this
% to the double lined characters upon figuring out how to do so.
%
\newcommand{\C}[1]{$\mathbb{C}^{#1}$}
\newcommand{\R}[1]{$\mathbb{R}^{#1}$}

%
% various generally useful helpers
%

% derivative of #1 wrt. #2:
\newcommand{\D}[2] {\frac {d#2} {d#1}}

\newcommand{\inv}[1]{\frac{1}{#1}}
\newcommand{\cross}[0]{\times}

\newcommand{\abs}[1]{\lvert{#1}\rvert}
\newcommand{\norm}[1]{\lVert{#1}\rVert}
\newcommand{\innerprod}[2]{\langle{#1}, {#2}\rangle}
\newcommand{\dotprod}[2]{{#1} \cdot {#2}}
\newcommand{\bdotprod}[2]{\left({#1} \cdot {#2}\right)}
\newcommand{\crossprod}[2]{{#1} \cross {#2}}
\newcommand{\tripleprod}[3]{\dotprod{\left(\crossprod{#1}{#2}\right)}{#3}}

\DeclareMathOperator{\Proj}{Proj}
\DeclareMathOperator{\Span}{span}
\DeclareMathOperator{\Sgn}{sgn}
\DeclareMathOperator{\Area}{Area}
\DeclareMathOperator{\Volume}{Volume}

%
% A few miscellaneous things specific to this document
%
\newcommand{\crossop}[1]{\crossprod{#1}{}}

% R2 vector.
\newcommand{\VectorTwo}[2]{
\begin{bmatrix}
 {#1} \\
 {#2}
\end{bmatrix}
}

\newcommand{\VectorN}[1]{
\begin{bmatrix}
{#1}_1 \\
{#1}_2 \\
\vdots \\
{#1}_N \\
\end{bmatrix}
}

\newcommand{\DETuvij}[4]{
\begin{vmatrix}
 {#1}_{#3} & {#1}_{#4} \\
 {#2}_{#3} & {#2}_{#4}
\end{vmatrix}
}

\newcommand{\DETuvwijk}[6]{
\begin{vmatrix}
 {#1}_{#4} & {#1}_{#5} & {#1}_{#6} \\
 {#2}_{#4} & {#2}_{#5} & {#2}_{#6} \\
 {#3}_{#4} & {#3}_{#5} & {#3}_{#6}
\end{vmatrix}
}

\newcommand{\DETuvwxijkl}[8]{
\begin{vmatrix}
 {#1}_{#5} & {#1}_{#6} & {#1}_{#7} & {#1}_{#8} \\
 {#2}_{#5} & {#2}_{#6} & {#2}_{#7} & {#2}_{#8} \\
 {#3}_{#5} & {#3}_{#6} & {#3}_{#7} & {#3}_{#8} \\
 {#4}_{#5} & {#4}_{#6} & {#4}_{#7} & {#4}_{#8} \\
\end{vmatrix}
}

%\newcommand{\DETuvwxyijklm}[10]{
%\begin{vmatrix}
% {#1}_{#6} & {#1}_{#7} & {#1}_{#8} & {#1}_{#9} & {#1}_{#10} \\
% {#2}_{#6} & {#2}_{#7} & {#2}_{#8} & {#2}_{#9} & {#2}_{#10} \\
% {#3}_{#6} & {#3}_{#7} & {#3}_{#8} & {#3}_{#9} & {#3}_{#10} \\
% {#4}_{#6} & {#4}_{#7} & {#4}_{#8} & {#4}_{#9} & {#4}_{#10} \\
% {#5}_{#6} & {#5}_{#7} & {#5}_{#8} & {#5}_{#9} & {#5}_{#10}
%\end{vmatrix}
%}

% R3 vector.
\newcommand{\VectorThree}[3]{
\begin{bmatrix}
 {#1} \\
 {#2} \\
 {#3}
\end{bmatrix}
}



\author{Peeter Joot}
\email{peeter.joot@gmail.com}

%\documentclass[]{eliblogwidescreen}

\usepackage{amsmath}
\usepackage{mathpazo}

%
% shorthand for bold symbols, convenient for vectors and matrices
%
\newcommand{\Ba}[0]{\mathbf{a}}
\newcommand{\Bb}[0]{\mathbf{b}}
\newcommand{\Bc}[0]{\mathbf{c}}
\newcommand{\Bd}[0]{\mathbf{d}}
\newcommand{\Be}[0]{\mathbf{e}}
\newcommand{\Bf}[0]{\mathbf{f}}
\newcommand{\Bg}[0]{\mathbf{g}}
\newcommand{\Bh}[0]{\mathbf{h}}
\newcommand{\Bi}[0]{\mathbf{i}}
\newcommand{\Bj}[0]{\mathbf{j}}
\newcommand{\Bk}[0]{\mathbf{k}}
\newcommand{\Bl}[0]{\mathbf{l}}
\newcommand{\Bm}[0]{\mathbf{m}}
\newcommand{\Bn}[0]{\mathbf{n}}
\newcommand{\Bo}[0]{\mathbf{o}}
\newcommand{\Bp}[0]{\mathbf{p}}
\newcommand{\Bq}[0]{\mathbf{q}}
\newcommand{\Br}[0]{\mathbf{r}}
\newcommand{\Bs}[0]{\mathbf{s}}
\newcommand{\Bt}[0]{\mathbf{t}}
\newcommand{\Bu}[0]{\mathbf{u}}
\newcommand{\Bv}[0]{\mathbf{v}}
\newcommand{\Bw}[0]{\mathbf{w}}
\newcommand{\Bx}[0]{\mathbf{x}}
\newcommand{\By}[0]{\mathbf{y}}
\newcommand{\Bz}[0]{\mathbf{z}}
\newcommand{\BA}[0]{\mathbf{A}}
\newcommand{\BB}[0]{\mathbf{B}}
\newcommand{\BC}[0]{\mathbf{C}}
\newcommand{\BD}[0]{\mathbf{D}}
\newcommand{\BE}[0]{\mathbf{E}}
\newcommand{\BF}[0]{\mathbf{F}}
\newcommand{\BG}[0]{\mathbf{G}}
\newcommand{\BH}[0]{\mathbf{H}}
\newcommand{\BI}[0]{\mathbf{I}}
\newcommand{\BJ}[0]{\mathbf{J}}
\newcommand{\BK}[0]{\mathbf{K}}
\newcommand{\BL}[0]{\mathbf{L}}
\newcommand{\BM}[0]{\mathbf{M}}
\newcommand{\BN}[0]{\mathbf{N}}
\newcommand{\BO}[0]{\mathbf{O}}
\newcommand{\BP}[0]{\mathbf{P}}
\newcommand{\BQ}[0]{\mathbf{Q}}
\newcommand{\BR}[0]{\mathbf{R}}
\newcommand{\BS}[0]{\mathbf{S}}
\newcommand{\BT}[0]{\mathbf{T}}
\newcommand{\BU}[0]{\mathbf{U}}
\newcommand{\BV}[0]{\mathbf{V}}
\newcommand{\BW}[0]{\mathbf{W}}
\newcommand{\BX}[0]{\mathbf{X}}
\newcommand{\BY}[0]{\mathbf{Y}}
\newcommand{\BZ}[0]{\mathbf{Z}}

\newcommand{\Bzero}[0]{\mathbf{0}}
\newcommand{\Btheta}[0]{\boldsymbol{\theta}}
\newcommand{\Btau}[0]{\boldsymbol{\tau}}
\newcommand{\Bomega}[0]{\boldsymbol{\omega}}

%
% shorthand for unit vectors
%
\newcommand{\acap}[0]{\hat{\Ba}}
\newcommand{\bcap}[0]{\hat{\Bb}}
\newcommand{\ccap}[0]{\hat{\Bc}}
\newcommand{\dcap}[0]{\hat{\Bd}}
\newcommand{\ecap}[0]{\hat{\Be}}
\newcommand{\fcap}[0]{\hat{\Bf}}
\newcommand{\gcap}[0]{\hat{\Bg}}
\newcommand{\hcap}[0]{\hat{\Bh}}
\newcommand{\icap}[0]{\hat{\Bi}}
\newcommand{\jcap}[0]{\hat{\Bj}}
\newcommand{\kcap}[0]{\hat{\Bk}}
\newcommand{\lcap}[0]{\hat{\Bl}}
\newcommand{\mcap}[0]{\hat{\Bm}}
\newcommand{\ncap}[0]{\hat{\Bn}}
\newcommand{\ocap}[0]{\hat{\Bo}}
\newcommand{\pcap}[0]{\hat{\Bp}}
\newcommand{\qcap}[0]{\hat{\Bq}}
\newcommand{\rcap}[0]{\hat{\Br}}
\newcommand{\scap}[0]{\hat{\Bs}}
\newcommand{\tcap}[0]{\hat{\Bt}}
\newcommand{\ucap}[0]{\hat{\Bu}}
\newcommand{\vcap}[0]{\hat{\Bv}}
\newcommand{\wcap}[0]{\hat{\Bw}}
\newcommand{\xcap}[0]{\hat{\Bx}}
\newcommand{\ycap}[0]{\hat{\By}}
\newcommand{\zcap}[0]{\hat{\Bz}}
\newcommand{\thetacap}[0]{\hat{\Btheta}}

%
% to write R^n and C^n in a distinguishable fashion.  Perhaps change this
% to the double lined characters upon figuring out how to do so.
%
\newcommand{\C}[1]{$\mathbb{C}^{#1}$}
\newcommand{\R}[1]{$\mathbb{R}^{#1}$}

%
% various generally useful helpers
%

% derivative of #1 wrt. #2:
\newcommand{\D}[2] {\frac {d#2} {d#1}}

\newcommand{\inv}[1]{\frac{1}{#1}}
\newcommand{\cross}[0]{\times}

\newcommand{\abs}[1]{\lvert{#1}\rvert}
\newcommand{\norm}[1]{\lVert{#1}\rVert}
\newcommand{\innerprod}[2]{\langle{#1}, {#2}\rangle}
\newcommand{\dotprod}[2]{{#1} \cdot {#2}}
\newcommand{\bdotprod}[2]{\left({#1} \cdot {#2}\right)}
\newcommand{\crossprod}[2]{{#1} \cross {#2}}
\newcommand{\tripleprod}[3]{\dotprod{\left(\crossprod{#1}{#2}\right)}{#3}}

\DeclareMathOperator{\Proj}{Proj}
\DeclareMathOperator{\Span}{span}
\DeclareMathOperator{\Sgn}{sgn}
\DeclareMathOperator{\Area}{Area}
\DeclareMathOperator{\Volume}{Volume}

%
% A few miscellaneous things specific to this document
%
\newcommand{\crossop}[1]{\crossprod{#1}{}}

% R2 vector.
\newcommand{\VectorTwo}[2]{
\begin{bmatrix}
 {#1} \\
 {#2}
\end{bmatrix}
}

\newcommand{\VectorN}[1]{
\begin{bmatrix}
{#1}_1 \\
{#1}_2 \\
\vdots \\
{#1}_N \\
\end{bmatrix}
}

\newcommand{\DETuvij}[4]{
\begin{vmatrix}
 {#1}_{#3} & {#1}_{#4} \\
 {#2}_{#3} & {#2}_{#4}
\end{vmatrix}
}

\newcommand{\DETuvwijk}[6]{
\begin{vmatrix}
 {#1}_{#4} & {#1}_{#5} & {#1}_{#6} \\
 {#2}_{#4} & {#2}_{#5} & {#2}_{#6} \\
 {#3}_{#4} & {#3}_{#5} & {#3}_{#6}
\end{vmatrix}
}

\newcommand{\DETuvwxijkl}[8]{
\begin{vmatrix}
 {#1}_{#5} & {#1}_{#6} & {#1}_{#7} & {#1}_{#8} \\
 {#2}_{#5} & {#2}_{#6} & {#2}_{#7} & {#2}_{#8} \\
 {#3}_{#5} & {#3}_{#6} & {#3}_{#7} & {#3}_{#8} \\
 {#4}_{#5} & {#4}_{#6} & {#4}_{#7} & {#4}_{#8} \\
\end{vmatrix}
}

%\newcommand{\DETuvwxyijklm}[10]{
%\begin{vmatrix}
% {#1}_{#6} & {#1}_{#7} & {#1}_{#8} & {#1}_{#9} & {#1}_{#10} \\
% {#2}_{#6} & {#2}_{#7} & {#2}_{#8} & {#2}_{#9} & {#2}_{#10} \\
% {#3}_{#6} & {#3}_{#7} & {#3}_{#8} & {#3}_{#9} & {#3}_{#10} \\
% {#4}_{#6} & {#4}_{#7} & {#4}_{#8} & {#4}_{#9} & {#4}_{#10} \\
% {#5}_{#6} & {#5}_{#7} & {#5}_{#8} & {#5}_{#9} & {#5}_{#10}
%\end{vmatrix}
%}

% R3 vector.
\newcommand{\VectorThree}[3]{
\begin{bmatrix}
 {#1} \\
 {#2} \\
 {#3}
\end{bmatrix}
}



\author{Peeter Joot}
\email{peeter.joot@gmail.com}


\chapter{Infinite square well wavefunction}
\label{chap:liboff41}
%\useCCL
\blogpage{http://sites.google.com/site/peeterjoot/math2010/liboff41.pdf}
\date{May 31, 2010}
\revisionInfo{liboff41.tex}

%\beginArtWithToc
\beginArtNoToc

\section{Motivation}

Work problem 4.1 from \citep{liboff2003iqm}, calculation of the eigensolution for an infinite square well, with boundaries $[-a/2, a/2]$.  It is actually a bit tidier seeming to generalize this slightly to boundaries $[a,b]$, which also implicitly solves the problem.  This is surely a problem that is done in 700 other QM texts, but I liked the way I did it this time so am writing it down.

\section{Guts}

Our equation to solve is $i \Hbar \Psi_t = -(\Hbar^2/2m) \Psi_{xx}$.  Separation of variables $\Psi = T \phi$ gives us

\begin{align}\label{eqn:liboff41:1}
T &\propto e^{-i E t/\Hbar } \\
\phi'' &= -\frac{2 m E }{\Hbar^2} \phi
\end{align}

With $k^2 = 2 m E/\Hbar^2$, we have

\begin{align}\label{eqn:liboff41:2}
\phi = A e^{i k x } + B e^{-i k x},
\end{align}

and the usual $\phi(a) = \phi(b) = 0$ boundary conditions give us

\begin{align}\label{eqn:liboff41:3}
0 = 
\begin{bmatrix}
e^{i k a } & e^{-i k a} \\
e^{i k b } & e^{-i k b}
\end{bmatrix}
\begin{bmatrix}
A \\
B
\end{bmatrix}.
\end{align}

We must have a zero determinant, which gives us the constraints on $k$ immediately
\begin{align*}
0 &= e^{i k (a - b)} - e^{i k (b-a)} \\
&= 2 i \sin( k (a - b) ).
\end{align*}

So our constraint on $k$ in terms of integers $n$, and the corresponding integration constant $E$

\begin{align}\label{eqn:liboff41:4}
k &= \frac{n \pi}{b - a} \\
E &= \frac{\Hbar^2 n^2 \pi^2 }{2 m (b-a)^2}.
\end{align}

One of the constants $A,B$ can be eliminated directly by picking any one of the two zeros from \eqnref{eqn:liboff41:3}

\begin{align*}
&A e ^{i k a } + B e^{-i k a} = 0 \\
&\implies \\
&B = -A e ^{2 i k a } 
\end{align*}

So we have

\begin{align}\label{eqn:liboff41:5}
\phi = A \left( e^{i k x } - e^{ ik (2a - x) } \right).
\end{align}

Or,
\begin{align}\label{eqn:liboff41:5b}
\phi = 2 A i e^{i k a} \sin( k (x-a )) 
\end{align}

Because probability densities, currents and the expectations of any operators will always have paired $\phi$ and $\phi^\conj$ factors, any constant phase factors like $i e^{i k a}$ above can be dropped, or absorbed into the constant $A$, and we can write

\begin{align}\label{eqn:liboff41:5d}
\phi = 2 A \sin( k (x-a )) 
\end{align}

The only thing left is to fix $A$ by integrating $\Abs{\phi}^2$, for which we have

\begin{align*}
1 &= \int_a^b \phi \phi^\conj dx \\
&= A^2 \int_a^b dx \left( e^{i k x } - e^{ ik (2a - x) } \right) \left( e^{-i k x } - e^{ -ik (2a - x) } \right) \\
&= A^2 \int_a^b dx \left( 2 - e^{ik(2a - 2x)} - e^{ik(-2a + 2x)} \right) \\
&= 2 A^2 \int_a^b dx \left( 1 - \cos (2 k (a - x)) \right)
\end{align*}

This last trig term vanishes over the integration region and we are left with 

\begin{align}\label{eqn:liboff41:6}
A = \inv{ \sqrt{2 (b-a)}},
\end{align}

which essentially completes the problem.  A final substitution back into \eqnref{eqn:liboff41:5b} allows for a final tidy up

\begin{align}\label{eqn:liboff41:5c}
\phi = \sqrt{\frac{2}{b-a}} \sin( k (x-a )).
\end{align}

\EndArticle

%
% Copyright � 2012 Peeter Joot.  All Rights Reserved.
% Licenced as described in the file LICENSE under the root directory of this GIT repository.
%

%
%
%%
% Copyright � 2015 Peeter Joot.  All Rights Reserved.
% Licenced as described in the file LICENSE under the root directory of this GIT repository.
%
\documentclass[]{eliblog}

\usepackage{amsmath}
\usepackage{mathpazo}

%
% shorthand for bold symbols, convenient for vectors and matrices
%
\newcommand{\Ba}[0]{\mathbf{a}}
\newcommand{\Bb}[0]{\mathbf{b}}
\newcommand{\Bc}[0]{\mathbf{c}}
\newcommand{\Bd}[0]{\mathbf{d}}
\newcommand{\Be}[0]{\mathbf{e}}
\newcommand{\Bf}[0]{\mathbf{f}}
\newcommand{\Bg}[0]{\mathbf{g}}
\newcommand{\Bh}[0]{\mathbf{h}}
\newcommand{\Bi}[0]{\mathbf{i}}
\newcommand{\Bj}[0]{\mathbf{j}}
\newcommand{\Bk}[0]{\mathbf{k}}
\newcommand{\Bl}[0]{\mathbf{l}}
\newcommand{\Bm}[0]{\mathbf{m}}
\newcommand{\Bn}[0]{\mathbf{n}}
\newcommand{\Bo}[0]{\mathbf{o}}
\newcommand{\Bp}[0]{\mathbf{p}}
\newcommand{\Bq}[0]{\mathbf{q}}
\newcommand{\Br}[0]{\mathbf{r}}
\newcommand{\Bs}[0]{\mathbf{s}}
\newcommand{\Bt}[0]{\mathbf{t}}
\newcommand{\Bu}[0]{\mathbf{u}}
\newcommand{\Bv}[0]{\mathbf{v}}
\newcommand{\Bw}[0]{\mathbf{w}}
\newcommand{\Bx}[0]{\mathbf{x}}
\newcommand{\By}[0]{\mathbf{y}}
\newcommand{\Bz}[0]{\mathbf{z}}
\newcommand{\BA}[0]{\mathbf{A}}
\newcommand{\BB}[0]{\mathbf{B}}
\newcommand{\BC}[0]{\mathbf{C}}
\newcommand{\BD}[0]{\mathbf{D}}
\newcommand{\BE}[0]{\mathbf{E}}
\newcommand{\BF}[0]{\mathbf{F}}
\newcommand{\BG}[0]{\mathbf{G}}
\newcommand{\BH}[0]{\mathbf{H}}
\newcommand{\BI}[0]{\mathbf{I}}
\newcommand{\BJ}[0]{\mathbf{J}}
\newcommand{\BK}[0]{\mathbf{K}}
\newcommand{\BL}[0]{\mathbf{L}}
\newcommand{\BM}[0]{\mathbf{M}}
\newcommand{\BN}[0]{\mathbf{N}}
\newcommand{\BO}[0]{\mathbf{O}}
\newcommand{\BP}[0]{\mathbf{P}}
\newcommand{\BQ}[0]{\mathbf{Q}}
\newcommand{\BR}[0]{\mathbf{R}}
\newcommand{\BS}[0]{\mathbf{S}}
\newcommand{\BT}[0]{\mathbf{T}}
\newcommand{\BU}[0]{\mathbf{U}}
\newcommand{\BV}[0]{\mathbf{V}}
\newcommand{\BW}[0]{\mathbf{W}}
\newcommand{\BX}[0]{\mathbf{X}}
\newcommand{\BY}[0]{\mathbf{Y}}
\newcommand{\BZ}[0]{\mathbf{Z}}

\newcommand{\Bzero}[0]{\mathbf{0}}
\newcommand{\Btheta}[0]{\boldsymbol{\theta}}
\newcommand{\Btau}[0]{\boldsymbol{\tau}}
\newcommand{\Bomega}[0]{\boldsymbol{\omega}}

%
% shorthand for unit vectors
%
\newcommand{\acap}[0]{\hat{\Ba}}
\newcommand{\bcap}[0]{\hat{\Bb}}
\newcommand{\ccap}[0]{\hat{\Bc}}
\newcommand{\dcap}[0]{\hat{\Bd}}
\newcommand{\ecap}[0]{\hat{\Be}}
\newcommand{\fcap}[0]{\hat{\Bf}}
\newcommand{\gcap}[0]{\hat{\Bg}}
\newcommand{\hcap}[0]{\hat{\Bh}}
\newcommand{\icap}[0]{\hat{\Bi}}
\newcommand{\jcap}[0]{\hat{\Bj}}
\newcommand{\kcap}[0]{\hat{\Bk}}
\newcommand{\lcap}[0]{\hat{\Bl}}
\newcommand{\mcap}[0]{\hat{\Bm}}
\newcommand{\ncap}[0]{\hat{\Bn}}
\newcommand{\ocap}[0]{\hat{\Bo}}
\newcommand{\pcap}[0]{\hat{\Bp}}
\newcommand{\qcap}[0]{\hat{\Bq}}
\newcommand{\rcap}[0]{\hat{\Br}}
\newcommand{\scap}[0]{\hat{\Bs}}
\newcommand{\tcap}[0]{\hat{\Bt}}
\newcommand{\ucap}[0]{\hat{\Bu}}
\newcommand{\vcap}[0]{\hat{\Bv}}
\newcommand{\wcap}[0]{\hat{\Bw}}
\newcommand{\xcap}[0]{\hat{\Bx}}
\newcommand{\ycap}[0]{\hat{\By}}
\newcommand{\zcap}[0]{\hat{\Bz}}
\newcommand{\thetacap}[0]{\hat{\Btheta}}

%
% to write R^n and C^n in a distinguishable fashion.  Perhaps change this
% to the double lined characters upon figuring out how to do so.
%
\newcommand{\C}[1]{$\mathbb{C}^{#1}$}
\newcommand{\R}[1]{$\mathbb{R}^{#1}$}

%
% various generally useful helpers
%

% derivative of #1 wrt. #2:
\newcommand{\D}[2] {\frac {d#2} {d#1}}

\newcommand{\inv}[1]{\frac{1}{#1}}
\newcommand{\cross}[0]{\times}

\newcommand{\abs}[1]{\lvert{#1}\rvert}
\newcommand{\norm}[1]{\lVert{#1}\rVert}
\newcommand{\innerprod}[2]{\langle{#1}, {#2}\rangle}
\newcommand{\dotprod}[2]{{#1} \cdot {#2}}
\newcommand{\bdotprod}[2]{\left({#1} \cdot {#2}\right)}
\newcommand{\crossprod}[2]{{#1} \cross {#2}}
\newcommand{\tripleprod}[3]{\dotprod{\left(\crossprod{#1}{#2}\right)}{#3}}

\DeclareMathOperator{\Proj}{Proj}
\DeclareMathOperator{\Span}{span}
\DeclareMathOperator{\Sgn}{sgn}
\DeclareMathOperator{\Area}{Area}
\DeclareMathOperator{\Volume}{Volume}

%
% A few miscellaneous things specific to this document
%
\newcommand{\crossop}[1]{\crossprod{#1}{}}

% R2 vector.
\newcommand{\VectorTwo}[2]{
\begin{bmatrix}
 {#1} \\
 {#2}
\end{bmatrix}
}

\newcommand{\VectorN}[1]{
\begin{bmatrix}
{#1}_1 \\
{#1}_2 \\
\vdots \\
{#1}_N \\
\end{bmatrix}
}

\newcommand{\DETuvij}[4]{
\begin{vmatrix}
 {#1}_{#3} & {#1}_{#4} \\
 {#2}_{#3} & {#2}_{#4}
\end{vmatrix}
}

\newcommand{\DETuvwijk}[6]{
\begin{vmatrix}
 {#1}_{#4} & {#1}_{#5} & {#1}_{#6} \\
 {#2}_{#4} & {#2}_{#5} & {#2}_{#6} \\
 {#3}_{#4} & {#3}_{#5} & {#3}_{#6}
\end{vmatrix}
}

\newcommand{\DETuvwxijkl}[8]{
\begin{vmatrix}
 {#1}_{#5} & {#1}_{#6} & {#1}_{#7} & {#1}_{#8} \\
 {#2}_{#5} & {#2}_{#6} & {#2}_{#7} & {#2}_{#8} \\
 {#3}_{#5} & {#3}_{#6} & {#3}_{#7} & {#3}_{#8} \\
 {#4}_{#5} & {#4}_{#6} & {#4}_{#7} & {#4}_{#8} \\
\end{vmatrix}
}

%\newcommand{\DETuvwxyijklm}[10]{
%\begin{vmatrix}
% {#1}_{#6} & {#1}_{#7} & {#1}_{#8} & {#1}_{#9} & {#1}_{#10} \\
% {#2}_{#6} & {#2}_{#7} & {#2}_{#8} & {#2}_{#9} & {#2}_{#10} \\
% {#3}_{#6} & {#3}_{#7} & {#3}_{#8} & {#3}_{#9} & {#3}_{#10} \\
% {#4}_{#6} & {#4}_{#7} & {#4}_{#8} & {#4}_{#9} & {#4}_{#10} \\
% {#5}_{#6} & {#5}_{#7} & {#5}_{#8} & {#5}_{#9} & {#5}_{#10}
%\end{vmatrix}
%}

% R3 vector.
\newcommand{\VectorThree}[3]{
\begin{bmatrix}
 {#1} \\
 {#2} \\
 {#3}
\end{bmatrix}
}



\author{Peeter Joot}
\email{peeter.joot@gmail.com}

%\documentclass[]{eliblogwidescreen}

\usepackage{amsmath}
\usepackage{mathpazo}

%
% shorthand for bold symbols, convenient for vectors and matrices
%
\newcommand{\Ba}[0]{\mathbf{a}}
\newcommand{\Bb}[0]{\mathbf{b}}
\newcommand{\Bc}[0]{\mathbf{c}}
\newcommand{\Bd}[0]{\mathbf{d}}
\newcommand{\Be}[0]{\mathbf{e}}
\newcommand{\Bf}[0]{\mathbf{f}}
\newcommand{\Bg}[0]{\mathbf{g}}
\newcommand{\Bh}[0]{\mathbf{h}}
\newcommand{\Bi}[0]{\mathbf{i}}
\newcommand{\Bj}[0]{\mathbf{j}}
\newcommand{\Bk}[0]{\mathbf{k}}
\newcommand{\Bl}[0]{\mathbf{l}}
\newcommand{\Bm}[0]{\mathbf{m}}
\newcommand{\Bn}[0]{\mathbf{n}}
\newcommand{\Bo}[0]{\mathbf{o}}
\newcommand{\Bp}[0]{\mathbf{p}}
\newcommand{\Bq}[0]{\mathbf{q}}
\newcommand{\Br}[0]{\mathbf{r}}
\newcommand{\Bs}[0]{\mathbf{s}}
\newcommand{\Bt}[0]{\mathbf{t}}
\newcommand{\Bu}[0]{\mathbf{u}}
\newcommand{\Bv}[0]{\mathbf{v}}
\newcommand{\Bw}[0]{\mathbf{w}}
\newcommand{\Bx}[0]{\mathbf{x}}
\newcommand{\By}[0]{\mathbf{y}}
\newcommand{\Bz}[0]{\mathbf{z}}
\newcommand{\BA}[0]{\mathbf{A}}
\newcommand{\BB}[0]{\mathbf{B}}
\newcommand{\BC}[0]{\mathbf{C}}
\newcommand{\BD}[0]{\mathbf{D}}
\newcommand{\BE}[0]{\mathbf{E}}
\newcommand{\BF}[0]{\mathbf{F}}
\newcommand{\BG}[0]{\mathbf{G}}
\newcommand{\BH}[0]{\mathbf{H}}
\newcommand{\BI}[0]{\mathbf{I}}
\newcommand{\BJ}[0]{\mathbf{J}}
\newcommand{\BK}[0]{\mathbf{K}}
\newcommand{\BL}[0]{\mathbf{L}}
\newcommand{\BM}[0]{\mathbf{M}}
\newcommand{\BN}[0]{\mathbf{N}}
\newcommand{\BO}[0]{\mathbf{O}}
\newcommand{\BP}[0]{\mathbf{P}}
\newcommand{\BQ}[0]{\mathbf{Q}}
\newcommand{\BR}[0]{\mathbf{R}}
\newcommand{\BS}[0]{\mathbf{S}}
\newcommand{\BT}[0]{\mathbf{T}}
\newcommand{\BU}[0]{\mathbf{U}}
\newcommand{\BV}[0]{\mathbf{V}}
\newcommand{\BW}[0]{\mathbf{W}}
\newcommand{\BX}[0]{\mathbf{X}}
\newcommand{\BY}[0]{\mathbf{Y}}
\newcommand{\BZ}[0]{\mathbf{Z}}

\newcommand{\Bzero}[0]{\mathbf{0}}
\newcommand{\Btheta}[0]{\boldsymbol{\theta}}
\newcommand{\Btau}[0]{\boldsymbol{\tau}}
\newcommand{\Bomega}[0]{\boldsymbol{\omega}}

%
% shorthand for unit vectors
%
\newcommand{\acap}[0]{\hat{\Ba}}
\newcommand{\bcap}[0]{\hat{\Bb}}
\newcommand{\ccap}[0]{\hat{\Bc}}
\newcommand{\dcap}[0]{\hat{\Bd}}
\newcommand{\ecap}[0]{\hat{\Be}}
\newcommand{\fcap}[0]{\hat{\Bf}}
\newcommand{\gcap}[0]{\hat{\Bg}}
\newcommand{\hcap}[0]{\hat{\Bh}}
\newcommand{\icap}[0]{\hat{\Bi}}
\newcommand{\jcap}[0]{\hat{\Bj}}
\newcommand{\kcap}[0]{\hat{\Bk}}
\newcommand{\lcap}[0]{\hat{\Bl}}
\newcommand{\mcap}[0]{\hat{\Bm}}
\newcommand{\ncap}[0]{\hat{\Bn}}
\newcommand{\ocap}[0]{\hat{\Bo}}
\newcommand{\pcap}[0]{\hat{\Bp}}
\newcommand{\qcap}[0]{\hat{\Bq}}
\newcommand{\rcap}[0]{\hat{\Br}}
\newcommand{\scap}[0]{\hat{\Bs}}
\newcommand{\tcap}[0]{\hat{\Bt}}
\newcommand{\ucap}[0]{\hat{\Bu}}
\newcommand{\vcap}[0]{\hat{\Bv}}
\newcommand{\wcap}[0]{\hat{\Bw}}
\newcommand{\xcap}[0]{\hat{\Bx}}
\newcommand{\ycap}[0]{\hat{\By}}
\newcommand{\zcap}[0]{\hat{\Bz}}
\newcommand{\thetacap}[0]{\hat{\Btheta}}

%
% to write R^n and C^n in a distinguishable fashion.  Perhaps change this
% to the double lined characters upon figuring out how to do so.
%
\newcommand{\C}[1]{$\mathbb{C}^{#1}$}
\newcommand{\R}[1]{$\mathbb{R}^{#1}$}

%
% various generally useful helpers
%

% derivative of #1 wrt. #2:
\newcommand{\D}[2] {\frac {d#2} {d#1}}

\newcommand{\inv}[1]{\frac{1}{#1}}
\newcommand{\cross}[0]{\times}

\newcommand{\abs}[1]{\lvert{#1}\rvert}
\newcommand{\norm}[1]{\lVert{#1}\rVert}
\newcommand{\innerprod}[2]{\langle{#1}, {#2}\rangle}
\newcommand{\dotprod}[2]{{#1} \cdot {#2}}
\newcommand{\bdotprod}[2]{\left({#1} \cdot {#2}\right)}
\newcommand{\crossprod}[2]{{#1} \cross {#2}}
\newcommand{\tripleprod}[3]{\dotprod{\left(\crossprod{#1}{#2}\right)}{#3}}

\DeclareMathOperator{\Proj}{Proj}
\DeclareMathOperator{\Span}{span}
\DeclareMathOperator{\Sgn}{sgn}
\DeclareMathOperator{\Area}{Area}
\DeclareMathOperator{\Volume}{Volume}

%
% A few miscellaneous things specific to this document
%
\newcommand{\crossop}[1]{\crossprod{#1}{}}

% R2 vector.
\newcommand{\VectorTwo}[2]{
\begin{bmatrix}
 {#1} \\
 {#2}
\end{bmatrix}
}

\newcommand{\VectorN}[1]{
\begin{bmatrix}
{#1}_1 \\
{#1}_2 \\
\vdots \\
{#1}_N \\
\end{bmatrix}
}

\newcommand{\DETuvij}[4]{
\begin{vmatrix}
 {#1}_{#3} & {#1}_{#4} \\
 {#2}_{#3} & {#2}_{#4}
\end{vmatrix}
}

\newcommand{\DETuvwijk}[6]{
\begin{vmatrix}
 {#1}_{#4} & {#1}_{#5} & {#1}_{#6} \\
 {#2}_{#4} & {#2}_{#5} & {#2}_{#6} \\
 {#3}_{#4} & {#3}_{#5} & {#3}_{#6}
\end{vmatrix}
}

\newcommand{\DETuvwxijkl}[8]{
\begin{vmatrix}
 {#1}_{#5} & {#1}_{#6} & {#1}_{#7} & {#1}_{#8} \\
 {#2}_{#5} & {#2}_{#6} & {#2}_{#7} & {#2}_{#8} \\
 {#3}_{#5} & {#3}_{#6} & {#3}_{#7} & {#3}_{#8} \\
 {#4}_{#5} & {#4}_{#6} & {#4}_{#7} & {#4}_{#8} \\
\end{vmatrix}
}

%\newcommand{\DETuvwxyijklm}[10]{
%\begin{vmatrix}
% {#1}_{#6} & {#1}_{#7} & {#1}_{#8} & {#1}_{#9} & {#1}_{#10} \\
% {#2}_{#6} & {#2}_{#7} & {#2}_{#8} & {#2}_{#9} & {#2}_{#10} \\
% {#3}_{#6} & {#3}_{#7} & {#3}_{#8} & {#3}_{#9} & {#3}_{#10} \\
% {#4}_{#6} & {#4}_{#7} & {#4}_{#8} & {#4}_{#9} & {#4}_{#10} \\
% {#5}_{#6} & {#5}_{#7} & {#5}_{#8} & {#5}_{#9} & {#5}_{#10}
%\end{vmatrix}
%}

% R3 vector.
\newcommand{\VectorThree}[3]{
\begin{bmatrix}
 {#1} \\
 {#2} \\
 {#3}
\end{bmatrix}
}



\author{Peeter Joot}
\email{peeter.joot@gmail.com}


\chapter{More problems from Liboff chapter 4}
\label{chap:liboff43}
%\useCCL
\blogpage{http://sites.google.com/site/peeterjoot/math2010/liboff43.pdf}
\date{June 25, 2010}
\revisionInfo{liboff43.tex}

%\beginArtWithToc
\beginArtNoToc

\section{Motivation}

Some more problems from \citep{liboff2003iqm}.

\section{Problem 4.11}

Some problems on Hermitian adjoints.  The starting point is the definition of the adjoint \(A^\dagger\) of \(A\) in terms of the inner product

\begin{equation}\label{eqn:liboff43:20}
\begin{aligned}
\braket{\hatA^\dagger \phi}{\psi} = \braket{\phi}{\hatA \psi}
\end{aligned}
\end{equation}

\subsection{4.11 a}

\begin{equation}\label{eqn:liboff43:40}
\begin{aligned}
\braket{ \phi }{ (a \hatA + b \hatB) \psi }
&=
a \braket{ \phi }{ \hatA \psi } + b \braket{ \phi }{ \hatB \psi }  \\
&=
a \braket{ \hatA^\dagger \phi }{ \psi } + b \braket{ \hatB^\dagger \phi }{ \psi }  \\
&=
\braket{ a^\conj \hatA^\dagger \phi }{ \psi } + \braket{ b^\conj \hatB^\dagger \phi }{ \psi }  \\
&=
\braket{ (a^\conj \hatA^\dagger + b^\conj \hatB^\dagger ) \phi }{ \psi }  \\
&\implies \\
(a \hatA + b \hatB)^\dagger = (a^\conj \hatA^\dagger + b^\conj \hatB^\dagger)
\end{aligned}
\end{equation}

\subsection{4.11 b}
\begin{equation}\label{eqn:liboff43:60}
\begin{aligned}
\braket{ \phi }{ \hatA \hatB \psi }
&=
\braket{ \hatA^\dagger \phi }{ \hatB \psi }  \\
&=
\braket{ \hatB^\dagger \hatA^\dagger \phi }{ \psi }  \\
&\implies \\
(\hatA \hatB )^\dagger &=
\hatB^\dagger \hatA^\dagger
\end{aligned}
\end{equation}

%\subsection{4.11 c}
\subsection{4.11 d}

Hermitian adjoint of \(D^2\), where \(D = \PDi{x}{}\).  Here we need the integral form of the inner product

\begin{equation}\label{eqn:liboff43:80}
\begin{aligned}
\braket{\phi}{D^2 \psi}
&=
\int \phi^\conj \PD{x}{}\PD{x}{\psi} \\
&=
-\int \PD{x}{\phi^\conj} \PD{x}{\psi} \\
&=
\int \psi \PD{x}{}\PD{x}{\phi^\conj} \\
&\implies \\
(D^2)^\dagger &= D^2
\end{aligned}
\end{equation}

Since the text shows that the square of a Hermitian operator is Hermitian, one perhaps wonders if \(D\) is (but we expect not since \(\hatp = -i \Hbar D\) is Hermitian).

Suppose \(\hatA = aD\), we have

\begin{equation}\label{eqn:liboff43:100}
\begin{aligned}
\hatA^\dagger = -a^\conj D,
\end{aligned}
\end{equation}

so for this to be Hermitian (\(\hatA = \hatA^\dagger\)) we must have \(- a^\conj = a\).  If \(a = r e^{i\theta}\), we have

\begin{equation}\label{eqn:liboff43:120}
\begin{aligned}
-1 = e^{2 i\theta}
\end{aligned}
\end{equation}

So \(\theta = \pi (1/2 + n)\), and \(a = \pm i r\).  This fixes the scalar multiples of \(D\) that are required to form a Hermitian operator

\begin{equation}\label{eqn:liboff43:140}
\begin{aligned}
\hatA &= \pm i r D
\end{aligned}
\end{equation}

where \(r\) is any real positive constant.

\subsection{4.11 e}

\begin{equation}\label{eqn:liboff43:160}
\begin{aligned}
(\hatA \hatB - \hatB \hatA)^\dagger &= - (\hatA^\dagger \hatB^\dagger - \hatB^\dagger \hatA^\dagger)
\end{aligned}
\end{equation}

\subsection{4.11 f}

\begin{equation}\label{eqn:liboff43:180}
\begin{aligned}
(\hatA \hatB + \hatB \hatA)^\dagger &= \hatA^\dagger \hatB^\dagger + \hatB^\dagger \hatA^\dagger
\end{aligned}
\end{equation}

\subsection{4.11 g}

\begin{equation}\label{eqn:liboff43:200}
\begin{aligned}
i (\hatA \hatB - \hatB \hatA)^\dagger &= i ( \hatA^\dagger \hatB^\dagger - \hatB^\dagger \hatA^\dagger)
\end{aligned}
\end{equation}

\subsection{4.11 h}

This one was to calculate \((\hatA^\dagger)^\dagger\).  Intuitively I had expect that \((\hatA^\dagger)^\dagger = \hatA\).  How could one show this?

Trying to show this with Dirac notation, I got all mixed up initially.

Using the more straightforward and old fashioned integral notation (as in \citep{bohm1989qt}), this is more straightforward.  We have the Hermitian conjugate defined by

\begin{equation}\label{eqn:liboff43:220}
\begin{aligned}
\int \psi_2^\conj (\hatA \psi_1) = \int (\hatA^\dagger \psi_2^\conj) \psi_1,
\end{aligned}
\end{equation}

Or, more symmetrically, using braces to indicate operator direction

\begin{equation}\label{eqn:liboff43:240}
\begin{aligned}
\int \psi_2^\conj (\hatA \psi_1) = \int (\psi_2^\conj \hatA^\dagger) \psi_1.
\end{aligned}
\end{equation}

Introduce a couple of variable substitutions for clarity

\begin{equation}\label{eqn:liboff43:260}
\begin{aligned}
\phi_1 &= \psi_1^\conj \\
\phi_2 &= \psi_2^\conj \\
\hatB &= \hatA^\dagger.
\end{aligned}
\end{equation}

We then have

\begin{equation}\label{eqn:liboff43:280}
\begin{aligned}
\int \psi_2^\conj (\hatA \psi_1)
&=
\int (\psi_2^\conj \hatA^\dagger) \psi_1 \\
&=
\int (\phi_2 \hatB) \phi_1^\conj \\
&=
\int \phi_1^\conj (\hatB \phi_2) \\
&=
\int (\phi_1^\conj \hatB^\dagger) \phi_2 \\
&=
\int \phi_2 (\hatB^\dagger \phi_1^\conj) \\
&=
\int \psi_2^\conj (\hatA^{\dagger \dagger} \psi_1) \\
\end{aligned}
\end{equation}

Since this is true for all \(\psi_k\), we have \(\hatA = \hatA^{\dagger \dagger}\) as expected.

Having figured out the problem in the simpleton way, it is now simple to go back and translate this into the Dirac inner product notation without getting muddled.  We have

\begin{equation}\label{eqn:liboff43:300}
\begin{aligned}
\braket{ \psi_2 }{ \hatA \psi_1 }
&=
\braket{ \hatA^\dagger \psi_2 }{ \psi_1 }  \\
&=
\braket{ \hatB \phi_2^\conj }{ \phi_1^\conj }  \\
&=
{\braket{ \phi_1 }{ \hatB^\conj \phi_2}}^\conj  \\
&=
{\braket{ (\hatB^\conj)^\dagger \phi_1 }{ \phi_2}}^\conj  \\
&=
\braket{\phi_2^\conj }{ \hatB^\dagger \phi_1^\conj } \\
&=
\braket{\psi_2 }{ \hatA^{\dagger \dagger} \psi_1 } \\
\end{aligned}
\end{equation}

\subsection{4.11 i}

\begin{equation}\label{eqn:liboff43:320}
\begin{aligned}
(\hatA \hatA^\dagger)^\dagger &= (\hatA^\dagger)^\dagger \hatA^\dagger
\end{aligned}
\end{equation}

since \((\hatA^\dagger) ^\dagger = \hatA\)

\begin{equation}\label{eqn:liboff43:340}
\begin{aligned}
(\hatA \hatA^\dagger)^\dagger &= \hatA \hatA^\dagger.
\end{aligned}
\end{equation}

\section{Problem 4.12 d}

If \(\hatA\) is not Hermitian, is the product \(\hatA^\dagger \hatA\) Hermitian?  To start we need to verify that \(\braket{\psi}{\hatA^\dagger \phi} = \braket{\hatA \psi}{\phi}\).

\begin{equation}\label{eqn:liboff43:360}
\begin{aligned}
\braket{ \psi }{ \hatA^\dagger \phi }
&=
{\braket{ (\hatA^\dagger)^\conj \phi^\conj }{ \psi^\conj }}^\conj \\
&=
{\braket{ \phi^\conj }{ \hatA^\conj \psi^\conj }}^\conj \\
&=
\braket{ \psi }{ \hatA \psi }.
\end{aligned}
\end{equation}

With that verified we have

\begin{equation}\label{eqn:liboff43:380}
\begin{aligned}
\braket{ \psi }{ \hatA^\dagger \hatA \phi }
&=
\braket{ \hatA \psi }{ \hatA \phi }  \\
&=
\braket{ \hatA^\dagger \hatA \psi }{ \phi },
\end{aligned}
\end{equation}

so, the answer is yes.  Provided the adjoint exists, that product will be Hermitian.

\section{Problem 4.14}

Show that \(\Expectation{\hatA} = \Expectation{\hatA}^\conj\) (that it is real), if \(\hatA\) is Hermitian.  This follows by expansion of that conjugate

\begin{equation}\label{eqn:liboff43:400}
\begin{aligned}
\Expectation{\hatA}^\conj
&= \left(\int \psi^\conj \hatA \psi \right)^\conj \\
&= \int \psi \hatA^\conj \psi^\conj \\
&= \int (\hatA \psi)^\conj \psi \\
&= \braket{ \hatA \psi }{ \psi } \\
&= \braket{ \psi }{ \hatA^\dagger \psi } \\
&= \braket{ \psi }{ \hatA \psi } \\
&= \Expectation{\hatA}
\end{aligned}
\end{equation}

\EndArticle

%
% Copyright � 2015 Peeter Joot.  All Rights Reserved.
% Licenced as described in the file LICENSE under the root directory of this GIT repository.
%
\documentclass[]{eliblog}

\usepackage{amsmath}
\usepackage{mathpazo}

%
% shorthand for bold symbols, convenient for vectors and matrices
%
\newcommand{\Ba}[0]{\mathbf{a}}
\newcommand{\Bb}[0]{\mathbf{b}}
\newcommand{\Bc}[0]{\mathbf{c}}
\newcommand{\Bd}[0]{\mathbf{d}}
\newcommand{\Be}[0]{\mathbf{e}}
\newcommand{\Bf}[0]{\mathbf{f}}
\newcommand{\Bg}[0]{\mathbf{g}}
\newcommand{\Bh}[0]{\mathbf{h}}
\newcommand{\Bi}[0]{\mathbf{i}}
\newcommand{\Bj}[0]{\mathbf{j}}
\newcommand{\Bk}[0]{\mathbf{k}}
\newcommand{\Bl}[0]{\mathbf{l}}
\newcommand{\Bm}[0]{\mathbf{m}}
\newcommand{\Bn}[0]{\mathbf{n}}
\newcommand{\Bo}[0]{\mathbf{o}}
\newcommand{\Bp}[0]{\mathbf{p}}
\newcommand{\Bq}[0]{\mathbf{q}}
\newcommand{\Br}[0]{\mathbf{r}}
\newcommand{\Bs}[0]{\mathbf{s}}
\newcommand{\Bt}[0]{\mathbf{t}}
\newcommand{\Bu}[0]{\mathbf{u}}
\newcommand{\Bv}[0]{\mathbf{v}}
\newcommand{\Bw}[0]{\mathbf{w}}
\newcommand{\Bx}[0]{\mathbf{x}}
\newcommand{\By}[0]{\mathbf{y}}
\newcommand{\Bz}[0]{\mathbf{z}}
\newcommand{\BA}[0]{\mathbf{A}}
\newcommand{\BB}[0]{\mathbf{B}}
\newcommand{\BC}[0]{\mathbf{C}}
\newcommand{\BD}[0]{\mathbf{D}}
\newcommand{\BE}[0]{\mathbf{E}}
\newcommand{\BF}[0]{\mathbf{F}}
\newcommand{\BG}[0]{\mathbf{G}}
\newcommand{\BH}[0]{\mathbf{H}}
\newcommand{\BI}[0]{\mathbf{I}}
\newcommand{\BJ}[0]{\mathbf{J}}
\newcommand{\BK}[0]{\mathbf{K}}
\newcommand{\BL}[0]{\mathbf{L}}
\newcommand{\BM}[0]{\mathbf{M}}
\newcommand{\BN}[0]{\mathbf{N}}
\newcommand{\BO}[0]{\mathbf{O}}
\newcommand{\BP}[0]{\mathbf{P}}
\newcommand{\BQ}[0]{\mathbf{Q}}
\newcommand{\BR}[0]{\mathbf{R}}
\newcommand{\BS}[0]{\mathbf{S}}
\newcommand{\BT}[0]{\mathbf{T}}
\newcommand{\BU}[0]{\mathbf{U}}
\newcommand{\BV}[0]{\mathbf{V}}
\newcommand{\BW}[0]{\mathbf{W}}
\newcommand{\BX}[0]{\mathbf{X}}
\newcommand{\BY}[0]{\mathbf{Y}}
\newcommand{\BZ}[0]{\mathbf{Z}}

\newcommand{\Bzero}[0]{\mathbf{0}}
\newcommand{\Btheta}[0]{\boldsymbol{\theta}}
\newcommand{\Btau}[0]{\boldsymbol{\tau}}
\newcommand{\Bomega}[0]{\boldsymbol{\omega}}

%
% shorthand for unit vectors
%
\newcommand{\acap}[0]{\hat{\Ba}}
\newcommand{\bcap}[0]{\hat{\Bb}}
\newcommand{\ccap}[0]{\hat{\Bc}}
\newcommand{\dcap}[0]{\hat{\Bd}}
\newcommand{\ecap}[0]{\hat{\Be}}
\newcommand{\fcap}[0]{\hat{\Bf}}
\newcommand{\gcap}[0]{\hat{\Bg}}
\newcommand{\hcap}[0]{\hat{\Bh}}
\newcommand{\icap}[0]{\hat{\Bi}}
\newcommand{\jcap}[0]{\hat{\Bj}}
\newcommand{\kcap}[0]{\hat{\Bk}}
\newcommand{\lcap}[0]{\hat{\Bl}}
\newcommand{\mcap}[0]{\hat{\Bm}}
\newcommand{\ncap}[0]{\hat{\Bn}}
\newcommand{\ocap}[0]{\hat{\Bo}}
\newcommand{\pcap}[0]{\hat{\Bp}}
\newcommand{\qcap}[0]{\hat{\Bq}}
\newcommand{\rcap}[0]{\hat{\Br}}
\newcommand{\scap}[0]{\hat{\Bs}}
\newcommand{\tcap}[0]{\hat{\Bt}}
\newcommand{\ucap}[0]{\hat{\Bu}}
\newcommand{\vcap}[0]{\hat{\Bv}}
\newcommand{\wcap}[0]{\hat{\Bw}}
\newcommand{\xcap}[0]{\hat{\Bx}}
\newcommand{\ycap}[0]{\hat{\By}}
\newcommand{\zcap}[0]{\hat{\Bz}}
\newcommand{\thetacap}[0]{\hat{\Btheta}}

%
% to write R^n and C^n in a distinguishable fashion.  Perhaps change this
% to the double lined characters upon figuring out how to do so.
%
\newcommand{\C}[1]{$\mathbb{C}^{#1}$}
\newcommand{\R}[1]{$\mathbb{R}^{#1}$}

%
% various generally useful helpers
%

% derivative of #1 wrt. #2:
\newcommand{\D}[2] {\frac {d#2} {d#1}}

\newcommand{\inv}[1]{\frac{1}{#1}}
\newcommand{\cross}[0]{\times}

\newcommand{\abs}[1]{\lvert{#1}\rvert}
\newcommand{\norm}[1]{\lVert{#1}\rVert}
\newcommand{\innerprod}[2]{\langle{#1}, {#2}\rangle}
\newcommand{\dotprod}[2]{{#1} \cdot {#2}}
\newcommand{\bdotprod}[2]{\left({#1} \cdot {#2}\right)}
\newcommand{\crossprod}[2]{{#1} \cross {#2}}
\newcommand{\tripleprod}[3]{\dotprod{\left(\crossprod{#1}{#2}\right)}{#3}}

\DeclareMathOperator{\Proj}{Proj}
\DeclareMathOperator{\Span}{span}
\DeclareMathOperator{\Sgn}{sgn}
\DeclareMathOperator{\Area}{Area}
\DeclareMathOperator{\Volume}{Volume}

%
% A few miscellaneous things specific to this document
%
\newcommand{\crossop}[1]{\crossprod{#1}{}}

% R2 vector.
\newcommand{\VectorTwo}[2]{
\begin{bmatrix}
 {#1} \\
 {#2}
\end{bmatrix}
}

\newcommand{\VectorN}[1]{
\begin{bmatrix}
{#1}_1 \\
{#1}_2 \\
\vdots \\
{#1}_N \\
\end{bmatrix}
}

\newcommand{\DETuvij}[4]{
\begin{vmatrix}
 {#1}_{#3} & {#1}_{#4} \\
 {#2}_{#3} & {#2}_{#4}
\end{vmatrix}
}

\newcommand{\DETuvwijk}[6]{
\begin{vmatrix}
 {#1}_{#4} & {#1}_{#5} & {#1}_{#6} \\
 {#2}_{#4} & {#2}_{#5} & {#2}_{#6} \\
 {#3}_{#4} & {#3}_{#5} & {#3}_{#6}
\end{vmatrix}
}

\newcommand{\DETuvwxijkl}[8]{
\begin{vmatrix}
 {#1}_{#5} & {#1}_{#6} & {#1}_{#7} & {#1}_{#8} \\
 {#2}_{#5} & {#2}_{#6} & {#2}_{#7} & {#2}_{#8} \\
 {#3}_{#5} & {#3}_{#6} & {#3}_{#7} & {#3}_{#8} \\
 {#4}_{#5} & {#4}_{#6} & {#4}_{#7} & {#4}_{#8} \\
\end{vmatrix}
}

%\newcommand{\DETuvwxyijklm}[10]{
%\begin{vmatrix}
% {#1}_{#6} & {#1}_{#7} & {#1}_{#8} & {#1}_{#9} & {#1}_{#10} \\
% {#2}_{#6} & {#2}_{#7} & {#2}_{#8} & {#2}_{#9} & {#2}_{#10} \\
% {#3}_{#6} & {#3}_{#7} & {#3}_{#8} & {#3}_{#9} & {#3}_{#10} \\
% {#4}_{#6} & {#4}_{#7} & {#4}_{#8} & {#4}_{#9} & {#4}_{#10} \\
% {#5}_{#6} & {#5}_{#7} & {#5}_{#8} & {#5}_{#9} & {#5}_{#10}
%\end{vmatrix}
%}

% R3 vector.
\newcommand{\VectorThree}[3]{
\begin{bmatrix}
 {#1} \\
 {#2} \\
 {#3}
\end{bmatrix}
}



\author{Peeter Joot}
\email{peeter.joot@gmail.com}

%\documentclass[]{eliblogwidescreen}

\usepackage{amsmath}
\usepackage{mathpazo}

%
% shorthand for bold symbols, convenient for vectors and matrices
%
\newcommand{\Ba}[0]{\mathbf{a}}
\newcommand{\Bb}[0]{\mathbf{b}}
\newcommand{\Bc}[0]{\mathbf{c}}
\newcommand{\Bd}[0]{\mathbf{d}}
\newcommand{\Be}[0]{\mathbf{e}}
\newcommand{\Bf}[0]{\mathbf{f}}
\newcommand{\Bg}[0]{\mathbf{g}}
\newcommand{\Bh}[0]{\mathbf{h}}
\newcommand{\Bi}[0]{\mathbf{i}}
\newcommand{\Bj}[0]{\mathbf{j}}
\newcommand{\Bk}[0]{\mathbf{k}}
\newcommand{\Bl}[0]{\mathbf{l}}
\newcommand{\Bm}[0]{\mathbf{m}}
\newcommand{\Bn}[0]{\mathbf{n}}
\newcommand{\Bo}[0]{\mathbf{o}}
\newcommand{\Bp}[0]{\mathbf{p}}
\newcommand{\Bq}[0]{\mathbf{q}}
\newcommand{\Br}[0]{\mathbf{r}}
\newcommand{\Bs}[0]{\mathbf{s}}
\newcommand{\Bt}[0]{\mathbf{t}}
\newcommand{\Bu}[0]{\mathbf{u}}
\newcommand{\Bv}[0]{\mathbf{v}}
\newcommand{\Bw}[0]{\mathbf{w}}
\newcommand{\Bx}[0]{\mathbf{x}}
\newcommand{\By}[0]{\mathbf{y}}
\newcommand{\Bz}[0]{\mathbf{z}}
\newcommand{\BA}[0]{\mathbf{A}}
\newcommand{\BB}[0]{\mathbf{B}}
\newcommand{\BC}[0]{\mathbf{C}}
\newcommand{\BD}[0]{\mathbf{D}}
\newcommand{\BE}[0]{\mathbf{E}}
\newcommand{\BF}[0]{\mathbf{F}}
\newcommand{\BG}[0]{\mathbf{G}}
\newcommand{\BH}[0]{\mathbf{H}}
\newcommand{\BI}[0]{\mathbf{I}}
\newcommand{\BJ}[0]{\mathbf{J}}
\newcommand{\BK}[0]{\mathbf{K}}
\newcommand{\BL}[0]{\mathbf{L}}
\newcommand{\BM}[0]{\mathbf{M}}
\newcommand{\BN}[0]{\mathbf{N}}
\newcommand{\BO}[0]{\mathbf{O}}
\newcommand{\BP}[0]{\mathbf{P}}
\newcommand{\BQ}[0]{\mathbf{Q}}
\newcommand{\BR}[0]{\mathbf{R}}
\newcommand{\BS}[0]{\mathbf{S}}
\newcommand{\BT}[0]{\mathbf{T}}
\newcommand{\BU}[0]{\mathbf{U}}
\newcommand{\BV}[0]{\mathbf{V}}
\newcommand{\BW}[0]{\mathbf{W}}
\newcommand{\BX}[0]{\mathbf{X}}
\newcommand{\BY}[0]{\mathbf{Y}}
\newcommand{\BZ}[0]{\mathbf{Z}}

\newcommand{\Bzero}[0]{\mathbf{0}}
\newcommand{\Btheta}[0]{\boldsymbol{\theta}}
\newcommand{\Btau}[0]{\boldsymbol{\tau}}
\newcommand{\Bomega}[0]{\boldsymbol{\omega}}

%
% shorthand for unit vectors
%
\newcommand{\acap}[0]{\hat{\Ba}}
\newcommand{\bcap}[0]{\hat{\Bb}}
\newcommand{\ccap}[0]{\hat{\Bc}}
\newcommand{\dcap}[0]{\hat{\Bd}}
\newcommand{\ecap}[0]{\hat{\Be}}
\newcommand{\fcap}[0]{\hat{\Bf}}
\newcommand{\gcap}[0]{\hat{\Bg}}
\newcommand{\hcap}[0]{\hat{\Bh}}
\newcommand{\icap}[0]{\hat{\Bi}}
\newcommand{\jcap}[0]{\hat{\Bj}}
\newcommand{\kcap}[0]{\hat{\Bk}}
\newcommand{\lcap}[0]{\hat{\Bl}}
\newcommand{\mcap}[0]{\hat{\Bm}}
\newcommand{\ncap}[0]{\hat{\Bn}}
\newcommand{\ocap}[0]{\hat{\Bo}}
\newcommand{\pcap}[0]{\hat{\Bp}}
\newcommand{\qcap}[0]{\hat{\Bq}}
\newcommand{\rcap}[0]{\hat{\Br}}
\newcommand{\scap}[0]{\hat{\Bs}}
\newcommand{\tcap}[0]{\hat{\Bt}}
\newcommand{\ucap}[0]{\hat{\Bu}}
\newcommand{\vcap}[0]{\hat{\Bv}}
\newcommand{\wcap}[0]{\hat{\Bw}}
\newcommand{\xcap}[0]{\hat{\Bx}}
\newcommand{\ycap}[0]{\hat{\By}}
\newcommand{\zcap}[0]{\hat{\Bz}}
\newcommand{\thetacap}[0]{\hat{\Btheta}}

%
% to write R^n and C^n in a distinguishable fashion.  Perhaps change this
% to the double lined characters upon figuring out how to do so.
%
\newcommand{\C}[1]{$\mathbb{C}^{#1}$}
\newcommand{\R}[1]{$\mathbb{R}^{#1}$}

%
% various generally useful helpers
%

% derivative of #1 wrt. #2:
\newcommand{\D}[2] {\frac {d#2} {d#1}}

\newcommand{\inv}[1]{\frac{1}{#1}}
\newcommand{\cross}[0]{\times}

\newcommand{\abs}[1]{\lvert{#1}\rvert}
\newcommand{\norm}[1]{\lVert{#1}\rVert}
\newcommand{\innerprod}[2]{\langle{#1}, {#2}\rangle}
\newcommand{\dotprod}[2]{{#1} \cdot {#2}}
\newcommand{\bdotprod}[2]{\left({#1} \cdot {#2}\right)}
\newcommand{\crossprod}[2]{{#1} \cross {#2}}
\newcommand{\tripleprod}[3]{\dotprod{\left(\crossprod{#1}{#2}\right)}{#3}}

\DeclareMathOperator{\Proj}{Proj}
\DeclareMathOperator{\Span}{span}
\DeclareMathOperator{\Sgn}{sgn}
\DeclareMathOperator{\Area}{Area}
\DeclareMathOperator{\Volume}{Volume}

%
% A few miscellaneous things specific to this document
%
\newcommand{\crossop}[1]{\crossprod{#1}{}}

% R2 vector.
\newcommand{\VectorTwo}[2]{
\begin{bmatrix}
 {#1} \\
 {#2}
\end{bmatrix}
}

\newcommand{\VectorN}[1]{
\begin{bmatrix}
{#1}_1 \\
{#1}_2 \\
\vdots \\
{#1}_N \\
\end{bmatrix}
}

\newcommand{\DETuvij}[4]{
\begin{vmatrix}
 {#1}_{#3} & {#1}_{#4} \\
 {#2}_{#3} & {#2}_{#4}
\end{vmatrix}
}

\newcommand{\DETuvwijk}[6]{
\begin{vmatrix}
 {#1}_{#4} & {#1}_{#5} & {#1}_{#6} \\
 {#2}_{#4} & {#2}_{#5} & {#2}_{#6} \\
 {#3}_{#4} & {#3}_{#5} & {#3}_{#6}
\end{vmatrix}
}

\newcommand{\DETuvwxijkl}[8]{
\begin{vmatrix}
 {#1}_{#5} & {#1}_{#6} & {#1}_{#7} & {#1}_{#8} \\
 {#2}_{#5} & {#2}_{#6} & {#2}_{#7} & {#2}_{#8} \\
 {#3}_{#5} & {#3}_{#6} & {#3}_{#7} & {#3}_{#8} \\
 {#4}_{#5} & {#4}_{#6} & {#4}_{#7} & {#4}_{#8} \\
\end{vmatrix}
}

%\newcommand{\DETuvwxyijklm}[10]{
%\begin{vmatrix}
% {#1}_{#6} & {#1}_{#7} & {#1}_{#8} & {#1}_{#9} & {#1}_{#10} \\
% {#2}_{#6} & {#2}_{#7} & {#2}_{#8} & {#2}_{#9} & {#2}_{#10} \\
% {#3}_{#6} & {#3}_{#7} & {#3}_{#8} & {#3}_{#9} & {#3}_{#10} \\
% {#4}_{#6} & {#4}_{#7} & {#4}_{#8} & {#4}_{#9} & {#4}_{#10} \\
% {#5}_{#6} & {#5}_{#7} & {#5}_{#8} & {#5}_{#9} & {#5}_{#10}
%\end{vmatrix}
%}

% R3 vector.
\newcommand{\VectorThree}[3]{
\begin{bmatrix}
 {#1} \\
 {#2} \\
 {#3}
\end{bmatrix}
}



\author{Peeter Joot}
\email{peeter.joot@gmail.com}


\chapter{Fourier transformation of the Pauli QED wave equation.}
\label{chap:pauliFourier}
%\useCCL
\blogpage{http://sites.google.com/site/peeterjoot/math2010/pauliFourier.pdf}
\date{May 29, 2010}
\revisionInfo{pauliFourier.tex}

%\beginArtWithToc
\beginArtNoToc

\section{Motivation.}

In \cite{feynman1961qed}, Feynman writes the Pauli wave equation for a non-relativistic treatment of a mass in a scalar and vector potential electrodynamic field.  That is

\begin{align}\label{eqn:pauliFourier:1}
i \hbar \PD{t}{\Psi} = \inv{2m} \left( \Bp - \frac{e}{c} \BA \right)^2 \Psi + e \phi \Psi
\end{align}

Is this amenable to Fourier transform solution like so many other PDEs?  Let's give it a try.

\section{Fourier Notation.}

Our transform pair will be written

\begin{subequations}
\begin{align}
\Psi(\Bx, t) &= \inv{(\sqrt{2 \pi})^3} \int \hat{\Psi}(\Bk, t) e^{i \Bk \cdot \Bx} d^3 \Bk \label{eqn:pauliFourier:2a} \\
\hat{\Psi}(\Bk, t) &= \inv{(\sqrt{2 \pi})^3} \int \Psi(\Bx, t) e^{-i \Bk \cdot \Bx} d^3 \Bx \label{eqn:pauliFourier:2b}
\end{align}
\end{subequations}

\section{Guts}

If we expand out the Hamiltonian terms we have for the Pauli equation

\begin{align}\label{eqn:pauliFourier:3}
i \hbar \PD{t}{\Psi} = \inv{2m} \left( - \hbar^2 \spacegrad^2 - 2 i \hbar \frac{e}{c} \BA \cdot \spacegrad + \frac{e^2}{c^2} \BA^2 \right) \Psi + e \phi \Psi.
\end{align}

Let's now apply each of these derivative operations to our assumed Fourier solution $\Psi(\Bx, t)$ from \ref{eqn:pauliFourier:2a}.  Starting with the Laplacian we have

\begin{align}\label{eqn:pauliFourier:xx1}
\spacegrad^2 \Psi(\Bx, t) &=
\inv{(\sqrt{2 \pi})^3} \int \hat{\Psi}(\Bk, t) (i\Bk)^2 e^{i \Bk \cdot \Bx} d^3 \Bk.
\end{align}

For the $\BA \cdot \spacegrad$ operator application we have

\begin{align}\label{eqn:pauliFourier:xx2}
\BA \cdot \spacegrad \Psi(\Bx, t) &=
\inv{(\sqrt{2 \pi})^3} \int \hat{\Psi}(\Bk, t) (i \BA \cdot \Bk) e^{i \Bk \cdot \Bx} d^3 \Bk.
\end{align}

Putting both together we have

\begin{align}\label{eqn:pauliFourier:5}
0 &= 
\inv{(\sqrt{2 \pi})^3} \int 
\left(
-i \hbar \PD{t}{\hat{\Psi}} + \inv{2m} \left( \hbar^2 \Bk^2 + 2 \hbar \frac{e}{c} \BA \cdot \Bk + \frac{e^2}{c^2} \BA^2 \right) \hat{\Psi} + e \phi \hat{\Psi} \right)
e^{i \Bk \cdot \Bx} d^3 \Bk.
\end{align}

Switching to the wave number domain, we reduce the equation to a first order time domain problem, namely

\begin{align}\label{eqn:pauliFourier:6}
\PD{t}{\hat{\Psi}}(\Bk, t) = \inv{i \hbar} \left( \inv{2m} \left( \hbar \Bk + \frac{e}{c} \BA \right)^2 + e \phi \right) \Psi(\Bk, t)
\end{align}

Since $\BA = \BA(\Bx, t)$, and $\phi = \phi(\Bx, t)$, is this really meaningful in any way?  Let's suppose that it is, and further simplify the system by imposing a constraint of constant time potentials ($\PDi{t}{\BA} = \PDi{t}{\phi} = 0$).  That allows for direct integration of the wave function for

\begin{align}\label{eqn:pauliFourier:7}
\hat{\Psi}(\Bk, t) = \hat{\Psi}(\Bk, 0) \exp\left(
\inv{i \hbar} \left( \inv{2m} \left( \hbar \Bk + \frac{e}{c} \BA \right)^2 + e \phi \right) t
\right)
\end{align}

It it probably really in the inverse transform context that our equation \ref{eqn:pauliFourier:6} takes meaning, giving

\begin{align}\label{eqn:pauliFourier:8}
\Psi(\Bx, t) &= \inv{(\sqrt{2 \pi})^3} \int 
\hat{\Psi}(\Bk, 0) \exp\left(
\inv{i \hbar} \left( \inv{2m} \left( \hbar \Bk + \frac{e}{c} \BA \right)^2 + e \phi \right) t
+ i \Bk \cdot \Bx
\right)
d^3 \Bk.
\end{align}

Furthermore, inserting the inverse Fourier transform of $\hat{\Psi}(\Bk, 0)$, we have the time evolution of the wave function as a convolution integral

\begin{align}\label{eqn:pauliFourier:9}
\Psi(\Bx, t) &= \inv{(2 \pi)^3} \int 
\Psi(\Bx', 0) \exp\left(
\inv{i \hbar} \left( \inv{2m} \left( \hbar \Bk + \frac{e}{c} \BA \right)^2 + e \phi \right) t
+ i \Bk \cdot (\Bx - \Bx')
\right)
d^3 \Bk d^3 \Bx'.
\end{align}

Splitting out the convolution kernel, this takes a slightly tidier form

\begin{subequations}
\begin{align}\label{eqn:pauliFourier:10}
\Psi(\Bx, t) &= 
\int \hat{U}(\Bx - \Bx', t) \Psi(\Bx', 0) d^3 \Bx' \\
\hat{U}(\Bx, t) &= 
\inv{(2 \pi)^3} 
\int
\exp\left(
\inv{i \hbar} \left( \inv{2m} \left( \hbar \Bk + \frac{e}{c} \BA \right)^2 + e \phi \right) t
+ i \Bk \cdot \Bx 
\right)
d^3 \Bk.
\end{align}
\end{subequations}

\EndArticle
%\EndNoBibArticle


\part{University of Toronto PHY356F.  Quantum Mechanics I.}
%
% Copyright � 2015 Peeter Joot.  All Rights Reserved.
% Licenced as described in the file LICENSE under the root directory of this GIT repository.
%
\documentclass[]{eliblog}

\usepackage{amsmath}
\usepackage{mathpazo}

%
% shorthand for bold symbols, convenient for vectors and matrices
%
\newcommand{\Ba}[0]{\mathbf{a}}
\newcommand{\Bb}[0]{\mathbf{b}}
\newcommand{\Bc}[0]{\mathbf{c}}
\newcommand{\Bd}[0]{\mathbf{d}}
\newcommand{\Be}[0]{\mathbf{e}}
\newcommand{\Bf}[0]{\mathbf{f}}
\newcommand{\Bg}[0]{\mathbf{g}}
\newcommand{\Bh}[0]{\mathbf{h}}
\newcommand{\Bi}[0]{\mathbf{i}}
\newcommand{\Bj}[0]{\mathbf{j}}
\newcommand{\Bk}[0]{\mathbf{k}}
\newcommand{\Bl}[0]{\mathbf{l}}
\newcommand{\Bm}[0]{\mathbf{m}}
\newcommand{\Bn}[0]{\mathbf{n}}
\newcommand{\Bo}[0]{\mathbf{o}}
\newcommand{\Bp}[0]{\mathbf{p}}
\newcommand{\Bq}[0]{\mathbf{q}}
\newcommand{\Br}[0]{\mathbf{r}}
\newcommand{\Bs}[0]{\mathbf{s}}
\newcommand{\Bt}[0]{\mathbf{t}}
\newcommand{\Bu}[0]{\mathbf{u}}
\newcommand{\Bv}[0]{\mathbf{v}}
\newcommand{\Bw}[0]{\mathbf{w}}
\newcommand{\Bx}[0]{\mathbf{x}}
\newcommand{\By}[0]{\mathbf{y}}
\newcommand{\Bz}[0]{\mathbf{z}}
\newcommand{\BA}[0]{\mathbf{A}}
\newcommand{\BB}[0]{\mathbf{B}}
\newcommand{\BC}[0]{\mathbf{C}}
\newcommand{\BD}[0]{\mathbf{D}}
\newcommand{\BE}[0]{\mathbf{E}}
\newcommand{\BF}[0]{\mathbf{F}}
\newcommand{\BG}[0]{\mathbf{G}}
\newcommand{\BH}[0]{\mathbf{H}}
\newcommand{\BI}[0]{\mathbf{I}}
\newcommand{\BJ}[0]{\mathbf{J}}
\newcommand{\BK}[0]{\mathbf{K}}
\newcommand{\BL}[0]{\mathbf{L}}
\newcommand{\BM}[0]{\mathbf{M}}
\newcommand{\BN}[0]{\mathbf{N}}
\newcommand{\BO}[0]{\mathbf{O}}
\newcommand{\BP}[0]{\mathbf{P}}
\newcommand{\BQ}[0]{\mathbf{Q}}
\newcommand{\BR}[0]{\mathbf{R}}
\newcommand{\BS}[0]{\mathbf{S}}
\newcommand{\BT}[0]{\mathbf{T}}
\newcommand{\BU}[0]{\mathbf{U}}
\newcommand{\BV}[0]{\mathbf{V}}
\newcommand{\BW}[0]{\mathbf{W}}
\newcommand{\BX}[0]{\mathbf{X}}
\newcommand{\BY}[0]{\mathbf{Y}}
\newcommand{\BZ}[0]{\mathbf{Z}}

\newcommand{\Bzero}[0]{\mathbf{0}}
\newcommand{\Btheta}[0]{\boldsymbol{\theta}}
\newcommand{\Btau}[0]{\boldsymbol{\tau}}
\newcommand{\Bomega}[0]{\boldsymbol{\omega}}

%
% shorthand for unit vectors
%
\newcommand{\acap}[0]{\hat{\Ba}}
\newcommand{\bcap}[0]{\hat{\Bb}}
\newcommand{\ccap}[0]{\hat{\Bc}}
\newcommand{\dcap}[0]{\hat{\Bd}}
\newcommand{\ecap}[0]{\hat{\Be}}
\newcommand{\fcap}[0]{\hat{\Bf}}
\newcommand{\gcap}[0]{\hat{\Bg}}
\newcommand{\hcap}[0]{\hat{\Bh}}
\newcommand{\icap}[0]{\hat{\Bi}}
\newcommand{\jcap}[0]{\hat{\Bj}}
\newcommand{\kcap}[0]{\hat{\Bk}}
\newcommand{\lcap}[0]{\hat{\Bl}}
\newcommand{\mcap}[0]{\hat{\Bm}}
\newcommand{\ncap}[0]{\hat{\Bn}}
\newcommand{\ocap}[0]{\hat{\Bo}}
\newcommand{\pcap}[0]{\hat{\Bp}}
\newcommand{\qcap}[0]{\hat{\Bq}}
\newcommand{\rcap}[0]{\hat{\Br}}
\newcommand{\scap}[0]{\hat{\Bs}}
\newcommand{\tcap}[0]{\hat{\Bt}}
\newcommand{\ucap}[0]{\hat{\Bu}}
\newcommand{\vcap}[0]{\hat{\Bv}}
\newcommand{\wcap}[0]{\hat{\Bw}}
\newcommand{\xcap}[0]{\hat{\Bx}}
\newcommand{\ycap}[0]{\hat{\By}}
\newcommand{\zcap}[0]{\hat{\Bz}}
\newcommand{\thetacap}[0]{\hat{\Btheta}}

%
% to write R^n and C^n in a distinguishable fashion.  Perhaps change this
% to the double lined characters upon figuring out how to do so.
%
\newcommand{\C}[1]{$\mathbb{C}^{#1}$}
\newcommand{\R}[1]{$\mathbb{R}^{#1}$}

%
% various generally useful helpers
%

% derivative of #1 wrt. #2:
\newcommand{\D}[2] {\frac {d#2} {d#1}}

\newcommand{\inv}[1]{\frac{1}{#1}}
\newcommand{\cross}[0]{\times}

\newcommand{\abs}[1]{\lvert{#1}\rvert}
\newcommand{\norm}[1]{\lVert{#1}\rVert}
\newcommand{\innerprod}[2]{\langle{#1}, {#2}\rangle}
\newcommand{\dotprod}[2]{{#1} \cdot {#2}}
\newcommand{\bdotprod}[2]{\left({#1} \cdot {#2}\right)}
\newcommand{\crossprod}[2]{{#1} \cross {#2}}
\newcommand{\tripleprod}[3]{\dotprod{\left(\crossprod{#1}{#2}\right)}{#3}}

\DeclareMathOperator{\Proj}{Proj}
\DeclareMathOperator{\Span}{span}
\DeclareMathOperator{\Sgn}{sgn}
\DeclareMathOperator{\Area}{Area}
\DeclareMathOperator{\Volume}{Volume}

%
% A few miscellaneous things specific to this document
%
\newcommand{\crossop}[1]{\crossprod{#1}{}}

% R2 vector.
\newcommand{\VectorTwo}[2]{
\begin{bmatrix}
 {#1} \\
 {#2}
\end{bmatrix}
}

\newcommand{\VectorN}[1]{
\begin{bmatrix}
{#1}_1 \\
{#1}_2 \\
\vdots \\
{#1}_N \\
\end{bmatrix}
}

\newcommand{\DETuvij}[4]{
\begin{vmatrix}
 {#1}_{#3} & {#1}_{#4} \\
 {#2}_{#3} & {#2}_{#4}
\end{vmatrix}
}

\newcommand{\DETuvwijk}[6]{
\begin{vmatrix}
 {#1}_{#4} & {#1}_{#5} & {#1}_{#6} \\
 {#2}_{#4} & {#2}_{#5} & {#2}_{#6} \\
 {#3}_{#4} & {#3}_{#5} & {#3}_{#6}
\end{vmatrix}
}

\newcommand{\DETuvwxijkl}[8]{
\begin{vmatrix}
 {#1}_{#5} & {#1}_{#6} & {#1}_{#7} & {#1}_{#8} \\
 {#2}_{#5} & {#2}_{#6} & {#2}_{#7} & {#2}_{#8} \\
 {#3}_{#5} & {#3}_{#6} & {#3}_{#7} & {#3}_{#8} \\
 {#4}_{#5} & {#4}_{#6} & {#4}_{#7} & {#4}_{#8} \\
\end{vmatrix}
}

%\newcommand{\DETuvwxyijklm}[10]{
%\begin{vmatrix}
% {#1}_{#6} & {#1}_{#7} & {#1}_{#8} & {#1}_{#9} & {#1}_{#10} \\
% {#2}_{#6} & {#2}_{#7} & {#2}_{#8} & {#2}_{#9} & {#2}_{#10} \\
% {#3}_{#6} & {#3}_{#7} & {#3}_{#8} & {#3}_{#9} & {#3}_{#10} \\
% {#4}_{#6} & {#4}_{#7} & {#4}_{#8} & {#4}_{#9} & {#4}_{#10} \\
% {#5}_{#6} & {#5}_{#7} & {#5}_{#8} & {#5}_{#9} & {#5}_{#10}
%\end{vmatrix}
%}

% R3 vector.
\newcommand{\VectorThree}[3]{
\begin{bmatrix}
 {#1} \\
 {#2} \\
 {#3}
\end{bmatrix}
}



\author{Peeter Joot}
\email{peeter.joot@gmail.com}

%\documentclass[]{eliblogwidescreen}

\usepackage{amsmath}
\usepackage{mathpazo}

%
% shorthand for bold symbols, convenient for vectors and matrices
%
\newcommand{\Ba}[0]{\mathbf{a}}
\newcommand{\Bb}[0]{\mathbf{b}}
\newcommand{\Bc}[0]{\mathbf{c}}
\newcommand{\Bd}[0]{\mathbf{d}}
\newcommand{\Be}[0]{\mathbf{e}}
\newcommand{\Bf}[0]{\mathbf{f}}
\newcommand{\Bg}[0]{\mathbf{g}}
\newcommand{\Bh}[0]{\mathbf{h}}
\newcommand{\Bi}[0]{\mathbf{i}}
\newcommand{\Bj}[0]{\mathbf{j}}
\newcommand{\Bk}[0]{\mathbf{k}}
\newcommand{\Bl}[0]{\mathbf{l}}
\newcommand{\Bm}[0]{\mathbf{m}}
\newcommand{\Bn}[0]{\mathbf{n}}
\newcommand{\Bo}[0]{\mathbf{o}}
\newcommand{\Bp}[0]{\mathbf{p}}
\newcommand{\Bq}[0]{\mathbf{q}}
\newcommand{\Br}[0]{\mathbf{r}}
\newcommand{\Bs}[0]{\mathbf{s}}
\newcommand{\Bt}[0]{\mathbf{t}}
\newcommand{\Bu}[0]{\mathbf{u}}
\newcommand{\Bv}[0]{\mathbf{v}}
\newcommand{\Bw}[0]{\mathbf{w}}
\newcommand{\Bx}[0]{\mathbf{x}}
\newcommand{\By}[0]{\mathbf{y}}
\newcommand{\Bz}[0]{\mathbf{z}}
\newcommand{\BA}[0]{\mathbf{A}}
\newcommand{\BB}[0]{\mathbf{B}}
\newcommand{\BC}[0]{\mathbf{C}}
\newcommand{\BD}[0]{\mathbf{D}}
\newcommand{\BE}[0]{\mathbf{E}}
\newcommand{\BF}[0]{\mathbf{F}}
\newcommand{\BG}[0]{\mathbf{G}}
\newcommand{\BH}[0]{\mathbf{H}}
\newcommand{\BI}[0]{\mathbf{I}}
\newcommand{\BJ}[0]{\mathbf{J}}
\newcommand{\BK}[0]{\mathbf{K}}
\newcommand{\BL}[0]{\mathbf{L}}
\newcommand{\BM}[0]{\mathbf{M}}
\newcommand{\BN}[0]{\mathbf{N}}
\newcommand{\BO}[0]{\mathbf{O}}
\newcommand{\BP}[0]{\mathbf{P}}
\newcommand{\BQ}[0]{\mathbf{Q}}
\newcommand{\BR}[0]{\mathbf{R}}
\newcommand{\BS}[0]{\mathbf{S}}
\newcommand{\BT}[0]{\mathbf{T}}
\newcommand{\BU}[0]{\mathbf{U}}
\newcommand{\BV}[0]{\mathbf{V}}
\newcommand{\BW}[0]{\mathbf{W}}
\newcommand{\BX}[0]{\mathbf{X}}
\newcommand{\BY}[0]{\mathbf{Y}}
\newcommand{\BZ}[0]{\mathbf{Z}}

\newcommand{\Bzero}[0]{\mathbf{0}}
\newcommand{\Btheta}[0]{\boldsymbol{\theta}}
\newcommand{\Btau}[0]{\boldsymbol{\tau}}
\newcommand{\Bomega}[0]{\boldsymbol{\omega}}

%
% shorthand for unit vectors
%
\newcommand{\acap}[0]{\hat{\Ba}}
\newcommand{\bcap}[0]{\hat{\Bb}}
\newcommand{\ccap}[0]{\hat{\Bc}}
\newcommand{\dcap}[0]{\hat{\Bd}}
\newcommand{\ecap}[0]{\hat{\Be}}
\newcommand{\fcap}[0]{\hat{\Bf}}
\newcommand{\gcap}[0]{\hat{\Bg}}
\newcommand{\hcap}[0]{\hat{\Bh}}
\newcommand{\icap}[0]{\hat{\Bi}}
\newcommand{\jcap}[0]{\hat{\Bj}}
\newcommand{\kcap}[0]{\hat{\Bk}}
\newcommand{\lcap}[0]{\hat{\Bl}}
\newcommand{\mcap}[0]{\hat{\Bm}}
\newcommand{\ncap}[0]{\hat{\Bn}}
\newcommand{\ocap}[0]{\hat{\Bo}}
\newcommand{\pcap}[0]{\hat{\Bp}}
\newcommand{\qcap}[0]{\hat{\Bq}}
\newcommand{\rcap}[0]{\hat{\Br}}
\newcommand{\scap}[0]{\hat{\Bs}}
\newcommand{\tcap}[0]{\hat{\Bt}}
\newcommand{\ucap}[0]{\hat{\Bu}}
\newcommand{\vcap}[0]{\hat{\Bv}}
\newcommand{\wcap}[0]{\hat{\Bw}}
\newcommand{\xcap}[0]{\hat{\Bx}}
\newcommand{\ycap}[0]{\hat{\By}}
\newcommand{\zcap}[0]{\hat{\Bz}}
\newcommand{\thetacap}[0]{\hat{\Btheta}}

%
% to write R^n and C^n in a distinguishable fashion.  Perhaps change this
% to the double lined characters upon figuring out how to do so.
%
\newcommand{\C}[1]{$\mathbb{C}^{#1}$}
\newcommand{\R}[1]{$\mathbb{R}^{#1}$}

%
% various generally useful helpers
%

% derivative of #1 wrt. #2:
\newcommand{\D}[2] {\frac {d#2} {d#1}}

\newcommand{\inv}[1]{\frac{1}{#1}}
\newcommand{\cross}[0]{\times}

\newcommand{\abs}[1]{\lvert{#1}\rvert}
\newcommand{\norm}[1]{\lVert{#1}\rVert}
\newcommand{\innerprod}[2]{\langle{#1}, {#2}\rangle}
\newcommand{\dotprod}[2]{{#1} \cdot {#2}}
\newcommand{\bdotprod}[2]{\left({#1} \cdot {#2}\right)}
\newcommand{\crossprod}[2]{{#1} \cross {#2}}
\newcommand{\tripleprod}[3]{\dotprod{\left(\crossprod{#1}{#2}\right)}{#3}}

\DeclareMathOperator{\Proj}{Proj}
\DeclareMathOperator{\Span}{span}
\DeclareMathOperator{\Sgn}{sgn}
\DeclareMathOperator{\Area}{Area}
\DeclareMathOperator{\Volume}{Volume}

%
% A few miscellaneous things specific to this document
%
\newcommand{\crossop}[1]{\crossprod{#1}{}}

% R2 vector.
\newcommand{\VectorTwo}[2]{
\begin{bmatrix}
 {#1} \\
 {#2}
\end{bmatrix}
}

\newcommand{\VectorN}[1]{
\begin{bmatrix}
{#1}_1 \\
{#1}_2 \\
\vdots \\
{#1}_N \\
\end{bmatrix}
}

\newcommand{\DETuvij}[4]{
\begin{vmatrix}
 {#1}_{#3} & {#1}_{#4} \\
 {#2}_{#3} & {#2}_{#4}
\end{vmatrix}
}

\newcommand{\DETuvwijk}[6]{
\begin{vmatrix}
 {#1}_{#4} & {#1}_{#5} & {#1}_{#6} \\
 {#2}_{#4} & {#2}_{#5} & {#2}_{#6} \\
 {#3}_{#4} & {#3}_{#5} & {#3}_{#6}
\end{vmatrix}
}

\newcommand{\DETuvwxijkl}[8]{
\begin{vmatrix}
 {#1}_{#5} & {#1}_{#6} & {#1}_{#7} & {#1}_{#8} \\
 {#2}_{#5} & {#2}_{#6} & {#2}_{#7} & {#2}_{#8} \\
 {#3}_{#5} & {#3}_{#6} & {#3}_{#7} & {#3}_{#8} \\
 {#4}_{#5} & {#4}_{#6} & {#4}_{#7} & {#4}_{#8} \\
\end{vmatrix}
}

%\newcommand{\DETuvwxyijklm}[10]{
%\begin{vmatrix}
% {#1}_{#6} & {#1}_{#7} & {#1}_{#8} & {#1}_{#9} & {#1}_{#10} \\
% {#2}_{#6} & {#2}_{#7} & {#2}_{#8} & {#2}_{#9} & {#2}_{#10} \\
% {#3}_{#6} & {#3}_{#7} & {#3}_{#8} & {#3}_{#9} & {#3}_{#10} \\
% {#4}_{#6} & {#4}_{#7} & {#4}_{#8} & {#4}_{#9} & {#4}_{#10} \\
% {#5}_{#6} & {#5}_{#7} & {#5}_{#8} & {#5}_{#9} & {#5}_{#10}
%\end{vmatrix}
%}

% R3 vector.
\newcommand{\VectorThree}[3]{
\begin{bmatrix}
 {#1} \\
 {#2} \\
 {#3}
\end{bmatrix}
}



\author{Peeter Joot}
\email{peeter.joot@gmail.com}


\chapter{Dirac Notation Ponderings.}
\label{chap:desaiDiracNotes}
%\useCCL
\blogpage{http://sites.google.com/site/peeterjoot/math2010/desaiDiracNotes.pdf}
\date{June 23, 2010}
\revisionInfo{desaiDiracNotes.tex}

%\beginArtWithToc
\beginArtNoToc

\section{Motivation.}

I've got the textbook \cite{desai2009quantum} now for the QMI course I'll be taking in the fall, and have started some light perusing.  Starting things off is the Dirac bra ket notation.  Some aspects of that notation, or the explanation in the text, are not quite obvious to me so here I try to make sense of things.

\section{Dirac Adjoint notes.}

There are a pair of relations given to define the Dirac Adjoint.  These are 1.26 and 1.27 respectively:

\begin{align*}
\left( A \ket{\alpha} \right)^\conj &= \bra{\alpha} A^\dagger \\
{\bra{\beta} A \ket{\alpha}}^\conj &= \bra{\alpha} A^\dagger \ket{\beta}
\end{align*}

Is there some redundancy to these definitions.  Namely is 1.27 a consequence of 1.26?

Since the ket was defined as the conjugate of the bra, we can probably rewrite 1.26 as

\begin{align*}
\bra{\alpha} A^\conj &= \bra{\alpha} A^\dagger 
\end{align*}

Left ``multiplication'', by the ket $\ket{\beta}$ gives

\begin{align*}
(\bra{\alpha} A^\conj) \ket{\beta} &= (\bra{\alpha} A^\dagger) \ket{\beta} \\
{\bra{\beta} (A \ket{\alpha})}^\conj &= (\bra{\alpha} A^\dagger) \ket{\beta} \\
\end{align*}

I've added and retained parenthesis to retain the operational direction.  Is that operational direction not important?  For example, given an operator like $p = -i \hbar \partial_x$, it makes a big difference whether the operator operates to the left or to the right.  In the text, this last relation is equation 1.27 once the parens are dropped, so it does appear that 1.27 is a consequence of 1.26.  This also then seems to imply that in a bra operator ket sandwich, the operator implicitly operates on the ket (to the right), while an adjoint operator implicitly operates on the bra (to the left).

Let's compare this to the simpler and more pedestrian notation found in an old fashioned book like Bohm's \cite{bohm1989qt}.  His expectation values explicitly use an integral definition, and his adjoint definition is very explicit about order of operations.  Namely

\begin{align}\label{eqn:desaiDiracNotes:1}
\int \phi^\conj (A \psi) 
&\equiv \int \psi (A^\dagger \phi^\conj) 
\end{align}

Starting with a concrete definition like this seems a bit easier.  Suppose we also define the bra ket sandwich based on the integral as follows

\begin{align*}
\bra{\phi} A \ket{\psi} 
&\equiv \bra{\phi} (A \ket{\psi}) \\
&\equiv \int \phi^\conj (A \psi) \\
\end{align*}

Now, we can rewrite \ref{eqn:desaiDiracNotes:1}, as 

\begin{align*}
\int \phi^\conj (A \psi)   &\equiv \int \psi (A^\dagger \phi^\conj) \\
&\implies \\
\bra{\phi} (A \ket{\psi})  &= \bra{\psi^\conj} ( A^\dagger \ket{\phi^\conj} ) \\
&\implies \\
\left(\bra{\phi} (A \ket{\psi}) \right)^\conj  &= ( \bra{\phi} A^\dagger ) \ket{\psi}
\end{align*}

When starting off with the integral we see the notational requirement for non-adjoint operators to operate implicitly to the right, and the adjoint operators to operate implicitly to the left.  With that notation requirement we can drop the parens and recover 1.29.

A couple clarification goals are now complete.  The first is seeing how equation 1.28 in the text implies 1.29.  We also have reconciled the Dirac notation with the familiar integral inner product notation, and seen two different ways that clarify the implicit operator directionality in the bra operator ket sandwiches.

\EndArticle

%
% Copyright � 2012 Peeter Joot.  All Rights Reserved.
% Licenced as described in the file LICENSE under the root directory of this GIT repository.
%

%\chapter{Rotations using matrix exponentials}
\label{chap:rotationUnitary}
%\blogpage{http://sites.google.com/site/peeterjoot/math2010/rotationUnitary.pdf}
%\date{July 27, 2010}

\subsection{Motivation}

In \citep{desai2009quantum} it is noted in problem 1.3 that any Unitary operator can be expressed in exponential form

\begin{equation}\label{eqn:rotationUnitary:1}
U = e^{iC},
\end{equation}

where \(C\) is Hermitian.  This is a powerful result hiding away in this problem.  I have not actually managed to prove this yet to my satisfaction, but working through some examples is highly worthwhile.  In particular it is interesting to compute the matrix \(C\) for a rotation matrix.  One finds that the matrix for such a rotation operator is in fact one of the Pauli spin matrices, and I found it interesting that this falls out so naturally.  Additionally, it is rather slick that one is able to so concisely express the rotation in exponential form, something that is natural and powerful in complex variable algebra, and also possible using Geometric Algebra using exponentials of bivectors.  Here we can do it after all with nothing more than the plain old matrix algebra that everybody is already comfortable with.

\subsection{The logarithm of the Unitary matrix}

By inspection we can invert \eqnref{eqn:rotationUnitary:1} for \(C\), by taking the logarithm

\begin{equation}\label{eqn:rotationUnitary:2}
C = -i \ln U.
\end{equation}

The problem becomes one of evaluating the logarithm, or even giving meaning to it.  I will assume that the functions of matrices that we are interested in are all polynomial in powers of the matrix, as in

\begin{equation}\label{eqn:rotationUnitary:3}
f(U) = \sum_k \alpha_k U^k,
\end{equation}

and that such series are convergent.  Then using a spectral decomposition, possible since Unitary matrices are normal, we can write for diagonal \(\Sigma = {\begin{bmatrix} \lambda_i \end{bmatrix}}_i\)

\begin{equation}\label{eqn:rotationUnitary:4}
U = V \Sigma V^\dagger,
\end{equation}

and

\begin{equation}\label{eqn:rotationUnitary:3b}
f(U) = V \left( \sum_k \alpha_k \Sigma^k \right) V^\dagger = V {\begin{bmatrix} f(\lambda_i) \end{bmatrix}}_i V^\dagger.
\end{equation}

Provided the logarithm has a convergent power series representation for \(U\), we then have for our Hermitian matrix \(C\)

\begin{equation}\label{eqn:rotationUnitary:5}
C = -i V (\ln \Sigma) V^\dagger
\end{equation}

\subsubsection{Evaluate this logarithm for an \texorpdfstring{\(x,y\)}{x,y} plane rotation}

Given the rotation matrix

\begin{equation}\label{eqn:rotationUnitary:6}
U =
\begin{bmatrix}
\cos\theta & \sin\theta \\
-\sin\theta & \cos\theta
\end{bmatrix},
\end{equation}

We find that the eigenvalues are \(e^{\pm i\theta}\), with eigenvectors proportional to \((1, \pm i)\) respectively.  Our decomposition for \(U\) is then given by
\eqnref{eqn:rotationUnitary:4}, and

\begin{equation}\label{eqn:rotationUnitary:7}
\begin{aligned}
V &= \inv{\sqrt{2}}
\begin{bmatrix}
1 & 1 \\
i & -i
\end{bmatrix} \\
\Sigma &=
\begin{bmatrix}
e^{i\theta} & 0 \\
0 & e^{-i\theta}
\end{bmatrix}.
\end{aligned}
\end{equation}

Taking logs we have

\begin{equation}\label{eqn:rotationUnitary:63}
\begin{aligned}
C
&=
\frac{1}{2}
\begin{bmatrix}
1 & 1 \\
i & -i
\end{bmatrix}
\begin{bmatrix}
\theta & 0 \\
0 & -\theta
\end{bmatrix}
\begin{bmatrix}
1 & -i \\
1 & i
\end{bmatrix} \\
&=
\frac{1}{2}
\begin{bmatrix}
1 & 1 \\
i & -i
\end{bmatrix}
\begin{bmatrix}
\theta  & -i\theta \\
-\theta & -i\theta
\end{bmatrix}  \\
&=
\begin{bmatrix}
0 & -i\theta \\
i\theta & 0
\end{bmatrix}.
\end{aligned}
\end{equation}

With the Pauli matrix

\begin{equation}\label{eqn:rotationUnitary:8a}
\sigma_2 =
\begin{bmatrix}
0 & -i \\
i & 0
\end{bmatrix},
\end{equation}

we then have for an \(x,y\) plane rotation matrix just:

\begin{equation}\label{eqn:rotationUnitary:8}
C = \theta \sigma_2
\end{equation}

and
\begin{equation}\label{eqn:rotationUnitary:9}
U = e^{i \theta \sigma_2}.
\end{equation}

Immediately, since \(\sigma_2^2 = I\), this also provides us with a trigonometric expansion
\begin{equation}\label{eqn:rotationUnitary:10}
U = I \cos\theta + i \sigma_2 \sin\theta.
\end{equation}

By inspection one can see that this takes us full circle back to the original matrix form \eqnref{eqn:rotationUnitary:6} of the rotation.  The exponential form of
\eqnref{eqn:rotationUnitary:9} has a beauty that is however far superior to the plain old trigonometric matrix that we are comfortable with.  All without any geometric algebra or bivector exponentials.

\subsubsection{Three dimensional exponential rotation matrices}

By inspection, we can augment our matrix \(C\) for a three dimensional rotation in the \(x,y\) plane, or a \(y,z\) rotation, or a \(x,z\) rotation.  Those are, respectively

\begin{equation}\label{eqn:rotationUnitary:30}
\begin{aligned}
U_{x,y}
&=
\exp
\begin{bmatrix}
0 & \theta & 0 \\
-\theta & 0 & 0 \\
0 & 0 & i
\end{bmatrix} \\
U_{y,z}
&=
\exp
\begin{bmatrix}
i & 0 & 0 \\
0 & 0 & \theta \\
0 & -\theta & 0 \\
\end{bmatrix} \\
U_{x,z}
&=
\exp
\begin{bmatrix}
0 & 0 & \theta \\
0 & i & 0 \\
-\theta & 0 & 0 \\
\end{bmatrix}
\end{aligned}
\end{equation}

Each of these matrices can be related to each other by similarity transformation using the permutation matrices
\begin{equation}\label{eqn:rotationUnitary:83}
\begin{bmatrix}
0 & 0 & 1 \\
0 & 1 & 0 \\
1 & 0 & 0 \\
\end{bmatrix},
\end{equation}

and
\begin{equation}\label{eqn:rotationUnitary:103}
\begin{bmatrix}
1 & 0 & 0 \\
0 & 0 & 1 \\
0 & 1 & 0 \\
\end{bmatrix}.
\end{equation}

\subsubsection{Exponential matrix form for a Lorentz boost}

The next obvious thing to try with this matrix representation is a Lorentz boost.

\begin{equation}\label{eqn:rotationUnitary:40}
L =
\begin{bmatrix}
\cosh\alpha & -\sinh\alpha \\
-\sinh\alpha & \cosh\alpha
\end{bmatrix},
\end{equation}

where \(\cosh\alpha = \gamma\), and \(\tanh\alpha = \beta\).

This matrix has a spectral decomposition given by
\begin{equation}\label{eqn:rotationUnitary:41}
\begin{aligned}
V &= \inv{\sqrt{2}}
\begin{bmatrix}
1 & 1 \\
-1 & 1
\end{bmatrix} \\
\Sigma &=
\begin{bmatrix}
e^\alpha & 0 \\
0 & e^{-\alpha}
\end{bmatrix}.
\end{aligned}
\end{equation}

Taking logs and computing \(C\) we have

\begin{equation}\label{eqn:rotationUnitary:123}
\begin{aligned}
C
&=
-\frac{i}{2}
\begin{bmatrix}
1 & 1 \\
-1 & 1
\end{bmatrix}
\begin{bmatrix}
\alpha & 0 \\
0 & -\alpha
\end{bmatrix}
\begin{bmatrix}
1 & -1 \\
1 & 1
\end{bmatrix} \\
&=
-\frac{i}{2}
\begin{bmatrix}
1 & 1 \\
-1 & 1
\end{bmatrix}
\begin{bmatrix}
\alpha & -\alpha \\
-\alpha & -\alpha
\end{bmatrix} \\
&=
i \alpha
\begin{bmatrix}
0 & 1 \\
1 & 0
\end{bmatrix}.
\end{aligned}
\end{equation}

Again we have one of the Pauli spin matrices.  This time it is
\begin{equation}\label{eqn:rotationUnitary:42}
\sigma_1 =
\begin{bmatrix}
0 & 1 \\
1 & 0
\end{bmatrix}.
\end{equation}

So we can write our Lorentz boost \eqnref{eqn:rotationUnitary:40} as just

\begin{equation}\label{eqn:rotationUnitary:43}
L = e^{-\alpha \sigma_1} = I \cosh\alpha - \sigma_1 \sinh\alpha.
\end{equation}

By inspection again, we can come full circle by inspection from this last hyperbolic representation back to the original explicit matrix representation.  Quite nifty!

It occurred to me after the fact that the Lorentz boost is not Unitary.  The fact that the eigenvalues are not a purely complex phase term, like those of the rotation is actually a good hint that looking at how to characterize the eigenvalues of a unitary matrix can be used to show that the matrix \(C = -i V \ln \Sigma V^\dagger\) is Hermitian.

%
% Copyright � 2012 Peeter Joot.  All Rights Reserved.
% Licenced as described in the file LICENSE under the root directory of this GIT repository.
%

%\chapter{On commutation of exponentials}
\label{chap:exponentialCommutation}
%\blogpage{http://sites.google.com/site/peeterjoot/math2010/exponentialCommutation.pdf}
%\date{May 30, 2010}

\subsection{Motivation}

Previously while working 
\href{http://peeterjoot.wordpress.com/2010/05/23/effect-of-sinusoid-operators/}{a Liboff problem}, I wondered about what the conditions were required for exponentials to commute.  In those problems the exponential arguments were operators.  Exponentials of bivectors as in quaternion like spatial or Lorentz boosts are also good examples of (sometimes) non-commutative exponentials.  It appears likely that the key requirement is that the exponential arguments commute, but how does one show this?  Here this is explored a bit.

\subsection{Guts}

If one could show that it was true that

\begin{align}\label{eqn:exponentialCommutation:1}
e^{x} e^{y} = e^{x + y}.
\end{align}

Then it would also imply that 

\begin{align}\label{eqn:exponentialCommutation:2}
e^{x} e^{y} = e^{y} e^{x}.
\end{align}

Let us perform the school boy exercise to prove \eqnref{eqn:exponentialCommutation:1} and explore the restrictions for such a proof.  We assume a power series definition of the exponential operator, and do not assume the values \(x,y\) are numeric, instead just that they can be multiplied.  A commutative multiplication will not be assumed.

By virtue of the power series exponential definition we have

\begin{align}\label{eqn:exponentialCommutation:3}
e^{x} e^{y} = 
\sum_{k=0}^\infty \inv{k!} x^k
\sum_{m=0}^\infty \inv{m!} y^m.
\end{align}

To attempt to put this into \(e^{x + y}\) form we will need to change the order that we evaluate the double sum, and here a picture \cref{fig:gridSummation} is helpful.

\imageFigure{../../figures/phy356/gridSummation}{Double sum diagonal ordering}{fig:gridSummation}{0.4}

For somebody who has seen this summation trick before the picture probably says it all.  We want to iterate over all pairs \((k, m)\), and could do so in \(\{(k, 0), (k, 1), \cdots (k, \infty), k \in [0, \infty] \}\) order as in our sum.  This is all the pairs of points in the upper right hand side of the grid.  We can also cover these grid coordinates in a different order.  In particular, these can be iterated over the diagonals.  The first diagonal having the point \((0,0)\), the second with the points \(\{(0, 1), (1, 0)\}\), the third with the points \(\{(0, 2), (1, 1), (2, 0)\}\).

Observe that along each diagonal the sum of the coordinates is constant, and increases by one.  Also observe that the number of points in each diagonal is this sum.  These observations provide a natural way to index the new grid traversal.  Labeling each of these diagonals with index \(j\), and points on that subset with \(n=0,1,\cdots, j\), we can express the original loop indices \(k\) and \(m\) in terms of these new (coupled) loop indices \(j\) and \(n\) as follows

\begin{align}\label{eqn:exponentialCommutation:4}
k &= j - n \\
m &= n.
\end{align}

Our sum becomes

\begin{align}\label{eqn:exponentialCommutation:5}
e^{x} e^{y} = 
\sum_{j=0}^\infty \sum_{n=0}^j
\inv{(j-n)!} x^{j-n}
\inv{n!} y^n.
\end{align}

With one small rearrangement, by introducing a \(j!\) in both the numerator and the denominator, the goal is almost reached.

\begin{align}\label{eqn:exponentialCommutation:6}
e^{x} e^{y} 
= 
\sum_{j=0}^\infty \inv{j!} \sum_{n=0}^j \frac{j!}{(j-n)! n!} x^{j-n} y^n
= 
\sum_{j=0}^\infty \inv{j!} \sum_{n=0}^j \binom{n}{j} x^{j-n} y^n.
\end{align}

This shows where we have a requirement that \(x\) and \(y\) commute, because only in that case do we have a binomial expansion

\begin{align}\label{eqn:exponentialCommutation:6b}
(x + y)^j = \sum_{n=0}^j \binom{n}{j} x^{j-n} y^n,
\end{align}

in the interior sum.  This reduced the problem to a consideration of the implication of possible non-commutation have on the binomial expansion.  Consider the simple special case of \((x + y)^2\).  If \(x\) and \(y\) do not necessarily commute, then we have

\begin{align}\label{eqn:exponentialCommutation:7}
(x + y)^2 = x^2 + x y + y x + y^2
\end{align}

whereas the binomial expansion formula has no such allowance for non-commutative multiplication and just counts the number of times a product can occur in any ordering as in

\begin{align}\label{eqn:exponentialCommutation:8}
(x + y)^2 = x^2 + 2 x y + y^2 = x^2 + 2 y x + y^2.
\end{align}

One sees the built in requirement for commutative multiplication here.  Now this does not prove that \(e^{x} e^{y} != e^{y} e^{x}\) unconditionally if \(x\) and \(y\) do not commute, but we do see that a requirement for commutative multiplication is sufficient if we want equality of such commuted exponentials.  In particular, the end result of the Liboff calculation where we had

\begin{align}\label{eqn:exponentialCommutation:9}
e^{i \hat{f}} e^{-i \hat{f}},
\end{align}

and was assuming this to be unity even for the differential operators \(\hat{f}\) under consideration is now completely answered (since we have \((i \hat{f}) (-i \hat{f}) \psi = (-i \hat{f}) (i \hat{f}) \psi\)).

%
% Copyright � 2015 Peeter Joot.  All Rights Reserved.
% Licenced as described in the file LICENSE under the root directory of this GIT repository.
%
\documentclass[]{eliblog}

\usepackage{amsmath}
\usepackage{mathpazo}

%
% shorthand for bold symbols, convenient for vectors and matrices
%
\newcommand{\Ba}[0]{\mathbf{a}}
\newcommand{\Bb}[0]{\mathbf{b}}
\newcommand{\Bc}[0]{\mathbf{c}}
\newcommand{\Bd}[0]{\mathbf{d}}
\newcommand{\Be}[0]{\mathbf{e}}
\newcommand{\Bf}[0]{\mathbf{f}}
\newcommand{\Bg}[0]{\mathbf{g}}
\newcommand{\Bh}[0]{\mathbf{h}}
\newcommand{\Bi}[0]{\mathbf{i}}
\newcommand{\Bj}[0]{\mathbf{j}}
\newcommand{\Bk}[0]{\mathbf{k}}
\newcommand{\Bl}[0]{\mathbf{l}}
\newcommand{\Bm}[0]{\mathbf{m}}
\newcommand{\Bn}[0]{\mathbf{n}}
\newcommand{\Bo}[0]{\mathbf{o}}
\newcommand{\Bp}[0]{\mathbf{p}}
\newcommand{\Bq}[0]{\mathbf{q}}
\newcommand{\Br}[0]{\mathbf{r}}
\newcommand{\Bs}[0]{\mathbf{s}}
\newcommand{\Bt}[0]{\mathbf{t}}
\newcommand{\Bu}[0]{\mathbf{u}}
\newcommand{\Bv}[0]{\mathbf{v}}
\newcommand{\Bw}[0]{\mathbf{w}}
\newcommand{\Bx}[0]{\mathbf{x}}
\newcommand{\By}[0]{\mathbf{y}}
\newcommand{\Bz}[0]{\mathbf{z}}
\newcommand{\BA}[0]{\mathbf{A}}
\newcommand{\BB}[0]{\mathbf{B}}
\newcommand{\BC}[0]{\mathbf{C}}
\newcommand{\BD}[0]{\mathbf{D}}
\newcommand{\BE}[0]{\mathbf{E}}
\newcommand{\BF}[0]{\mathbf{F}}
\newcommand{\BG}[0]{\mathbf{G}}
\newcommand{\BH}[0]{\mathbf{H}}
\newcommand{\BI}[0]{\mathbf{I}}
\newcommand{\BJ}[0]{\mathbf{J}}
\newcommand{\BK}[0]{\mathbf{K}}
\newcommand{\BL}[0]{\mathbf{L}}
\newcommand{\BM}[0]{\mathbf{M}}
\newcommand{\BN}[0]{\mathbf{N}}
\newcommand{\BO}[0]{\mathbf{O}}
\newcommand{\BP}[0]{\mathbf{P}}
\newcommand{\BQ}[0]{\mathbf{Q}}
\newcommand{\BR}[0]{\mathbf{R}}
\newcommand{\BS}[0]{\mathbf{S}}
\newcommand{\BT}[0]{\mathbf{T}}
\newcommand{\BU}[0]{\mathbf{U}}
\newcommand{\BV}[0]{\mathbf{V}}
\newcommand{\BW}[0]{\mathbf{W}}
\newcommand{\BX}[0]{\mathbf{X}}
\newcommand{\BY}[0]{\mathbf{Y}}
\newcommand{\BZ}[0]{\mathbf{Z}}

\newcommand{\Bzero}[0]{\mathbf{0}}
\newcommand{\Btheta}[0]{\boldsymbol{\theta}}
\newcommand{\Btau}[0]{\boldsymbol{\tau}}
\newcommand{\Bomega}[0]{\boldsymbol{\omega}}

%
% shorthand for unit vectors
%
\newcommand{\acap}[0]{\hat{\Ba}}
\newcommand{\bcap}[0]{\hat{\Bb}}
\newcommand{\ccap}[0]{\hat{\Bc}}
\newcommand{\dcap}[0]{\hat{\Bd}}
\newcommand{\ecap}[0]{\hat{\Be}}
\newcommand{\fcap}[0]{\hat{\Bf}}
\newcommand{\gcap}[0]{\hat{\Bg}}
\newcommand{\hcap}[0]{\hat{\Bh}}
\newcommand{\icap}[0]{\hat{\Bi}}
\newcommand{\jcap}[0]{\hat{\Bj}}
\newcommand{\kcap}[0]{\hat{\Bk}}
\newcommand{\lcap}[0]{\hat{\Bl}}
\newcommand{\mcap}[0]{\hat{\Bm}}
\newcommand{\ncap}[0]{\hat{\Bn}}
\newcommand{\ocap}[0]{\hat{\Bo}}
\newcommand{\pcap}[0]{\hat{\Bp}}
\newcommand{\qcap}[0]{\hat{\Bq}}
\newcommand{\rcap}[0]{\hat{\Br}}
\newcommand{\scap}[0]{\hat{\Bs}}
\newcommand{\tcap}[0]{\hat{\Bt}}
\newcommand{\ucap}[0]{\hat{\Bu}}
\newcommand{\vcap}[0]{\hat{\Bv}}
\newcommand{\wcap}[0]{\hat{\Bw}}
\newcommand{\xcap}[0]{\hat{\Bx}}
\newcommand{\ycap}[0]{\hat{\By}}
\newcommand{\zcap}[0]{\hat{\Bz}}
\newcommand{\thetacap}[0]{\hat{\Btheta}}

%
% to write R^n and C^n in a distinguishable fashion.  Perhaps change this
% to the double lined characters upon figuring out how to do so.
%
\newcommand{\C}[1]{$\mathbb{C}^{#1}$}
\newcommand{\R}[1]{$\mathbb{R}^{#1}$}

%
% various generally useful helpers
%

% derivative of #1 wrt. #2:
\newcommand{\D}[2] {\frac {d#2} {d#1}}

\newcommand{\inv}[1]{\frac{1}{#1}}
\newcommand{\cross}[0]{\times}

\newcommand{\abs}[1]{\lvert{#1}\rvert}
\newcommand{\norm}[1]{\lVert{#1}\rVert}
\newcommand{\innerprod}[2]{\langle{#1}, {#2}\rangle}
\newcommand{\dotprod}[2]{{#1} \cdot {#2}}
\newcommand{\bdotprod}[2]{\left({#1} \cdot {#2}\right)}
\newcommand{\crossprod}[2]{{#1} \cross {#2}}
\newcommand{\tripleprod}[3]{\dotprod{\left(\crossprod{#1}{#2}\right)}{#3}}

\DeclareMathOperator{\Proj}{Proj}
\DeclareMathOperator{\Span}{span}
\DeclareMathOperator{\Sgn}{sgn}
\DeclareMathOperator{\Area}{Area}
\DeclareMathOperator{\Volume}{Volume}

%
% A few miscellaneous things specific to this document
%
\newcommand{\crossop}[1]{\crossprod{#1}{}}

% R2 vector.
\newcommand{\VectorTwo}[2]{
\begin{bmatrix}
 {#1} \\
 {#2}
\end{bmatrix}
}

\newcommand{\VectorN}[1]{
\begin{bmatrix}
{#1}_1 \\
{#1}_2 \\
\vdots \\
{#1}_N \\
\end{bmatrix}
}

\newcommand{\DETuvij}[4]{
\begin{vmatrix}
 {#1}_{#3} & {#1}_{#4} \\
 {#2}_{#3} & {#2}_{#4}
\end{vmatrix}
}

\newcommand{\DETuvwijk}[6]{
\begin{vmatrix}
 {#1}_{#4} & {#1}_{#5} & {#1}_{#6} \\
 {#2}_{#4} & {#2}_{#5} & {#2}_{#6} \\
 {#3}_{#4} & {#3}_{#5} & {#3}_{#6}
\end{vmatrix}
}

\newcommand{\DETuvwxijkl}[8]{
\begin{vmatrix}
 {#1}_{#5} & {#1}_{#6} & {#1}_{#7} & {#1}_{#8} \\
 {#2}_{#5} & {#2}_{#6} & {#2}_{#7} & {#2}_{#8} \\
 {#3}_{#5} & {#3}_{#6} & {#3}_{#7} & {#3}_{#8} \\
 {#4}_{#5} & {#4}_{#6} & {#4}_{#7} & {#4}_{#8} \\
\end{vmatrix}
}

%\newcommand{\DETuvwxyijklm}[10]{
%\begin{vmatrix}
% {#1}_{#6} & {#1}_{#7} & {#1}_{#8} & {#1}_{#9} & {#1}_{#10} \\
% {#2}_{#6} & {#2}_{#7} & {#2}_{#8} & {#2}_{#9} & {#2}_{#10} \\
% {#3}_{#6} & {#3}_{#7} & {#3}_{#8} & {#3}_{#9} & {#3}_{#10} \\
% {#4}_{#6} & {#4}_{#7} & {#4}_{#8} & {#4}_{#9} & {#4}_{#10} \\
% {#5}_{#6} & {#5}_{#7} & {#5}_{#8} & {#5}_{#9} & {#5}_{#10}
%\end{vmatrix}
%}

% R3 vector.
\newcommand{\VectorThree}[3]{
\begin{bmatrix}
 {#1} \\
 {#2} \\
 {#3}
\end{bmatrix}
}



\author{Peeter Joot}
\email{peeter.joot@gmail.com}

%\documentclass[]{eliblogwidescreen}

\usepackage{amsmath}
\usepackage{mathpazo}

%
% shorthand for bold symbols, convenient for vectors and matrices
%
\newcommand{\Ba}[0]{\mathbf{a}}
\newcommand{\Bb}[0]{\mathbf{b}}
\newcommand{\Bc}[0]{\mathbf{c}}
\newcommand{\Bd}[0]{\mathbf{d}}
\newcommand{\Be}[0]{\mathbf{e}}
\newcommand{\Bf}[0]{\mathbf{f}}
\newcommand{\Bg}[0]{\mathbf{g}}
\newcommand{\Bh}[0]{\mathbf{h}}
\newcommand{\Bi}[0]{\mathbf{i}}
\newcommand{\Bj}[0]{\mathbf{j}}
\newcommand{\Bk}[0]{\mathbf{k}}
\newcommand{\Bl}[0]{\mathbf{l}}
\newcommand{\Bm}[0]{\mathbf{m}}
\newcommand{\Bn}[0]{\mathbf{n}}
\newcommand{\Bo}[0]{\mathbf{o}}
\newcommand{\Bp}[0]{\mathbf{p}}
\newcommand{\Bq}[0]{\mathbf{q}}
\newcommand{\Br}[0]{\mathbf{r}}
\newcommand{\Bs}[0]{\mathbf{s}}
\newcommand{\Bt}[0]{\mathbf{t}}
\newcommand{\Bu}[0]{\mathbf{u}}
\newcommand{\Bv}[0]{\mathbf{v}}
\newcommand{\Bw}[0]{\mathbf{w}}
\newcommand{\Bx}[0]{\mathbf{x}}
\newcommand{\By}[0]{\mathbf{y}}
\newcommand{\Bz}[0]{\mathbf{z}}
\newcommand{\BA}[0]{\mathbf{A}}
\newcommand{\BB}[0]{\mathbf{B}}
\newcommand{\BC}[0]{\mathbf{C}}
\newcommand{\BD}[0]{\mathbf{D}}
\newcommand{\BE}[0]{\mathbf{E}}
\newcommand{\BF}[0]{\mathbf{F}}
\newcommand{\BG}[0]{\mathbf{G}}
\newcommand{\BH}[0]{\mathbf{H}}
\newcommand{\BI}[0]{\mathbf{I}}
\newcommand{\BJ}[0]{\mathbf{J}}
\newcommand{\BK}[0]{\mathbf{K}}
\newcommand{\BL}[0]{\mathbf{L}}
\newcommand{\BM}[0]{\mathbf{M}}
\newcommand{\BN}[0]{\mathbf{N}}
\newcommand{\BO}[0]{\mathbf{O}}
\newcommand{\BP}[0]{\mathbf{P}}
\newcommand{\BQ}[0]{\mathbf{Q}}
\newcommand{\BR}[0]{\mathbf{R}}
\newcommand{\BS}[0]{\mathbf{S}}
\newcommand{\BT}[0]{\mathbf{T}}
\newcommand{\BU}[0]{\mathbf{U}}
\newcommand{\BV}[0]{\mathbf{V}}
\newcommand{\BW}[0]{\mathbf{W}}
\newcommand{\BX}[0]{\mathbf{X}}
\newcommand{\BY}[0]{\mathbf{Y}}
\newcommand{\BZ}[0]{\mathbf{Z}}

\newcommand{\Bzero}[0]{\mathbf{0}}
\newcommand{\Btheta}[0]{\boldsymbol{\theta}}
\newcommand{\Btau}[0]{\boldsymbol{\tau}}
\newcommand{\Bomega}[0]{\boldsymbol{\omega}}

%
% shorthand for unit vectors
%
\newcommand{\acap}[0]{\hat{\Ba}}
\newcommand{\bcap}[0]{\hat{\Bb}}
\newcommand{\ccap}[0]{\hat{\Bc}}
\newcommand{\dcap}[0]{\hat{\Bd}}
\newcommand{\ecap}[0]{\hat{\Be}}
\newcommand{\fcap}[0]{\hat{\Bf}}
\newcommand{\gcap}[0]{\hat{\Bg}}
\newcommand{\hcap}[0]{\hat{\Bh}}
\newcommand{\icap}[0]{\hat{\Bi}}
\newcommand{\jcap}[0]{\hat{\Bj}}
\newcommand{\kcap}[0]{\hat{\Bk}}
\newcommand{\lcap}[0]{\hat{\Bl}}
\newcommand{\mcap}[0]{\hat{\Bm}}
\newcommand{\ncap}[0]{\hat{\Bn}}
\newcommand{\ocap}[0]{\hat{\Bo}}
\newcommand{\pcap}[0]{\hat{\Bp}}
\newcommand{\qcap}[0]{\hat{\Bq}}
\newcommand{\rcap}[0]{\hat{\Br}}
\newcommand{\scap}[0]{\hat{\Bs}}
\newcommand{\tcap}[0]{\hat{\Bt}}
\newcommand{\ucap}[0]{\hat{\Bu}}
\newcommand{\vcap}[0]{\hat{\Bv}}
\newcommand{\wcap}[0]{\hat{\Bw}}
\newcommand{\xcap}[0]{\hat{\Bx}}
\newcommand{\ycap}[0]{\hat{\By}}
\newcommand{\zcap}[0]{\hat{\Bz}}
\newcommand{\thetacap}[0]{\hat{\Btheta}}

%
% to write R^n and C^n in a distinguishable fashion.  Perhaps change this
% to the double lined characters upon figuring out how to do so.
%
\newcommand{\C}[1]{$\mathbb{C}^{#1}$}
\newcommand{\R}[1]{$\mathbb{R}^{#1}$}

%
% various generally useful helpers
%

% derivative of #1 wrt. #2:
\newcommand{\D}[2] {\frac {d#2} {d#1}}

\newcommand{\inv}[1]{\frac{1}{#1}}
\newcommand{\cross}[0]{\times}

\newcommand{\abs}[1]{\lvert{#1}\rvert}
\newcommand{\norm}[1]{\lVert{#1}\rVert}
\newcommand{\innerprod}[2]{\langle{#1}, {#2}\rangle}
\newcommand{\dotprod}[2]{{#1} \cdot {#2}}
\newcommand{\bdotprod}[2]{\left({#1} \cdot {#2}\right)}
\newcommand{\crossprod}[2]{{#1} \cross {#2}}
\newcommand{\tripleprod}[3]{\dotprod{\left(\crossprod{#1}{#2}\right)}{#3}}

\DeclareMathOperator{\Proj}{Proj}
\DeclareMathOperator{\Span}{span}
\DeclareMathOperator{\Sgn}{sgn}
\DeclareMathOperator{\Area}{Area}
\DeclareMathOperator{\Volume}{Volume}

%
% A few miscellaneous things specific to this document
%
\newcommand{\crossop}[1]{\crossprod{#1}{}}

% R2 vector.
\newcommand{\VectorTwo}[2]{
\begin{bmatrix}
 {#1} \\
 {#2}
\end{bmatrix}
}

\newcommand{\VectorN}[1]{
\begin{bmatrix}
{#1}_1 \\
{#1}_2 \\
\vdots \\
{#1}_N \\
\end{bmatrix}
}

\newcommand{\DETuvij}[4]{
\begin{vmatrix}
 {#1}_{#3} & {#1}_{#4} \\
 {#2}_{#3} & {#2}_{#4}
\end{vmatrix}
}

\newcommand{\DETuvwijk}[6]{
\begin{vmatrix}
 {#1}_{#4} & {#1}_{#5} & {#1}_{#6} \\
 {#2}_{#4} & {#2}_{#5} & {#2}_{#6} \\
 {#3}_{#4} & {#3}_{#5} & {#3}_{#6}
\end{vmatrix}
}

\newcommand{\DETuvwxijkl}[8]{
\begin{vmatrix}
 {#1}_{#5} & {#1}_{#6} & {#1}_{#7} & {#1}_{#8} \\
 {#2}_{#5} & {#2}_{#6} & {#2}_{#7} & {#2}_{#8} \\
 {#3}_{#5} & {#3}_{#6} & {#3}_{#7} & {#3}_{#8} \\
 {#4}_{#5} & {#4}_{#6} & {#4}_{#7} & {#4}_{#8} \\
\end{vmatrix}
}

%\newcommand{\DETuvwxyijklm}[10]{
%\begin{vmatrix}
% {#1}_{#6} & {#1}_{#7} & {#1}_{#8} & {#1}_{#9} & {#1}_{#10} \\
% {#2}_{#6} & {#2}_{#7} & {#2}_{#8} & {#2}_{#9} & {#2}_{#10} \\
% {#3}_{#6} & {#3}_{#7} & {#3}_{#8} & {#3}_{#9} & {#3}_{#10} \\
% {#4}_{#6} & {#4}_{#7} & {#4}_{#8} & {#4}_{#9} & {#4}_{#10} \\
% {#5}_{#6} & {#5}_{#7} & {#5}_{#8} & {#5}_{#9} & {#5}_{#10}
%\end{vmatrix}
%}

% R3 vector.
\newcommand{\VectorThree}[3]{
\begin{bmatrix}
 {#1} \\
 {#2} \\
 {#3}
\end{bmatrix}
}



\author{Peeter Joot}
\email{peeter.joot@gmail.com}


\chapter{Desai Chapter II notes and problems.}
\label{chap:desaiCh2}
%\useCCL
\blogpage{http://sites.google.com/site/peeterjoot/math2010/desaiCh2.pdf}
\date{Sept 19, 2010}
\revisionInfo{desaiCh2.tex}

\beginArtWithToc
%\beginArtNoToc

\section{Motivation.}

Chapter II notes for \cite{desai2009quantum}.

\section{Notes}
\subsection{Canonical Commutator}

Based on the canonical relationship $[X,P] = i\hbar$, and $\braket{x'}{x} = \delta(x'-x)$, Desai determines the form of the $P$ operator in continuous space.  A consequence of this is that the matrix element of the momentum operator is found to have a delta function specification

\begin{align*}
\bra{x'} P \ket{x} = \delta(x - x') \left( -i \hbar \frac{d}{dx} \right).
\end{align*}

In particular the matrix element associated with the state $\ket{\phi}$ is found to be

\begin{align*}
\bra{x'} P \ket{\phi} = -i \hbar \frac{d}{dx'} \phi(x').
\end{align*}

Compare this to \cite{liboff2003iqm}, where this last is taken as the definition of the momentum operator, and the relationship to the delta function is not spelled out explicitly.  This canonical commutator approach, while more abstract, seems to have less black magic involved in the setup.  We do require the commutator relationship $[X,P] = i\hbar$ to be pulled out of a magic hat, but at least the magic show is a structured one based on a small set of core assumptions.

It will likely be good to come back to this later when trying to reconcile this new (for me) Dirac notation with the more basic notation I'm already comfortable with.  When trying to compare the two, it will be good to note that there is a matrix element that is implied in the more old fashioned treatment in a book such as \cite{bohm1989qt}.

There is one fundamental assumption that appears to be made in this section that isn't justified by anything except the end result.  That is the assumption that $P$ is a derivative like operator, acting with a product rule action.  That's used to obtain (2.28) and is a fairly black magic operation.  This same assumption, is also hiding, somewhat sneakily, in the manipulation for (2.44).

If one has to make that assumption that $P$ is a derivative like operator, I don't feel this method of introducing it is any less arbitrary seeming.  It is still pulled out of a magic hat, only because the answer is known ahead of time.  The approach of \cite{bohm1989qt}, where the derivative nature is presented as consequence of transforming (via Fourier transforms) from the position to the momentum representation, seems much more intuitive and less arbitrary.

\subsection{Generalized momentum commutator.}

It is stated that

\begin{align*}
[P,X^n] = - n i \hbar X^{n-1}.
\end{align*}

Let's prove this.  The $n=1$ case is the canonical commutator, which is assumed.  Is there any good way to justify that from first principles, as presented in the text?  We have to prove this for $n$, given the relationship for $n-1$.  Expanding the $n$th power commutator we have

\begin{align*}
[P,X^n] 
&= P X^n - X^n P \\
&= P X^{n-1} X - X^{n } P \\
\end{align*}

Rearranging the $n-1$ result we have

\begin{align*}
P X^{n-1} = X^{n-1} P - (n-1) i \hbar X^{n-2},
\end{align*}

and can insert that in our $[P,X^n]$ expansion for

\begin{align*}
[P,X^n] 
&= \left( X^{n-1} P - (n-1) i \hbar X^{n-2} \right)X - X^{n } P \\
&= X^{n-1} (PX) - (n-1) i \hbar X^{n-1} - X^{n } P \\
&= X^{n-1} ( X P - i\hbar) - (n-1) i \hbar X^{n-1} - X^{n } P \\
&= -X^{n-1} i\hbar - (n-1) i \hbar X^{n-1} \\
&= -n i \hbar X^{n-1} 
\qquad\square
\end{align*}

\subsection{Uncertainty principle.}

The origin of the statement $[\Delta A, \Delta B] = [A, B]$ is not something that seemed obvious.  Expanding this out however is straightforward, and clarifies things.  That is

\begin{align*}
[\Delta A, \Delta B] 
&= (A - \expectation{A}) (B - \expectation{B}) - (B - \expectation{B}) (A - \expectation{A}) \\
&= 
\left( A B - \expectation{A} B - \expectation{B} A +\expectation{A} \expectation{B} \right)
-\left( B A - \expectation{B} A - \expectation{A} B +\expectation{B} \expectation{A} \right) \\
&= 
A B - B A \\
&= 
[A, B]
\qquad\square
\end{align*}

\subsection{Size of a particle}

I found it curious that using $\Delta x \Delta p \approx \hbar$ instead of $\Delta x \Delta p \ge \hbar/2$, was sufficient to obtain the hydrogen ground state energy $E_{\text{min}} = -e^2/2 a_0$, without also having to do any factor of two fudging.

\subsection{Space displacement operator.}

I'd be curious to know if others find the loose use of equality for approximation after approximation slightly disturbing too?

I also find it curious that (2.140) is written

\begin{align*}
D(x) = \exp\left( -i \frac{P}{\hbar} x \right),
\end{align*}

and not
\begin{align*}
D(x) = \exp\left( -i x \frac{P}{\hbar} \right).
\end{align*}

Is this intentional?  It doesn't seem like $P$ ought to be acting on $x$ in this case, so why order the terms that way?

Expanding the application of this operator, or at least its first order Taylor series, is helpful to get an idea about this.  Doing so, with the original $\Delta x'$ value used in the derivation of the text we have to start

\begin{align*}
D(\Delta x') \ket{\phi} 
&\approx \left(1 - i \frac{P}{\hbar} \Delta x' \right) \ket{\phi} \\
&= \left(1 - i \left( -i \hbar \delta(x -x') \frac{\partial}{\partial x} \right) \inv{\hbar} \Delta x'\right) \ket{\phi} \\
\end{align*}

This shows that the $\Delta x$ factor can be commuted with the momentum operator, as it is not a function of $x'$, so the question of $P x$, vs $x P$ above appears to be a non-issue.

Regardless of that conclusion, it seems worthy to continue an attempt at expanding this shift operator action on the state vector.  Let's do so, but do so by computing the matrix element $\bra{x'} D(\Delta x') \ket{\phi}$.  That is

\begin{align*}
\bra{x'} D(\Delta x') \ket{\phi} 
&\approx
\braket{x'}{\phi} - \bra{x'} \delta(x -x') \frac{\partial}{\partial x} \Delta x' \ket{\phi} \\
&=
\phi(x') - \int \bra{x'} \delta(x -x') \frac{\partial}{\partial x} \Delta x' \ket{x'} \braket{x'}{\phi} dx' \\
&=
\phi(x') - \Delta x' \int \delta(x -x') \frac{\partial}{\partial x} \braket{x'}{\phi} dx' \\
&=
\phi(x') - \Delta x' \frac{\partial}{\partial x'} \braket{x'}{\phi} \\
&=
\phi(x') - \Delta x' \frac{\partial}{\partial x'} \phi(x') \\
\end{align*}

This is consistent with the text.  It is interesting, and initially surprising that the space displacement operator when applied to a state vector introduces a negative shift in the wave function associated with that state vector.  In the derivation of the text, this was associated with the use of integration by parts (ie: due to the sign change in that integration).  Here we see it sneak back in, due to the $i^2$ once the momentum operator is expanded completely.

As last note and question.  The first order Taylor approximation of the momentum operator was used.  If the higher order terms are retained, as in

\begin{align*}
\exp\left( -i \Delta x' \frac{P}{\hbar} \right) = 
1 - \Delta x' \delta(x -x') \frac{\partial}{\partial x} + 
\inv{2} \left( - \Delta x' \delta(x -x') \frac{\partial}{\partial x} \right)^2 + \cdots,
\end{align*}

then how does one evaluate a squared delta function (or Nth power)?

Talked to Vatche about this after class.  The key to this is sequential evaluation.  Considering the simple case for $P^2$, we evaluate one operator at a time, and never actually square the delta function

\begin{align*}
\bra{x'} P^2 \ket{\phi} 
%&= \bra{x'} P (P \ket{\phi}) \\
%&= -i \hbar \int dx' \bra{x'} P (\delta(x-x') \PD{x}{} \ket{x'} \braket{x'}{\phi}) \\
%&= -i \hbar \bra{x'} P \PD{x'}{} \ket{x'} \braket{x'}{\phi}) \\
\end{align*}

I was also questioned why I was including the delta function at this point.  Why would I do that.  Thinking further on this, I see that isn't a reasonable thing to do.  That delta function only comes into the mix when one takes the matrix element of the momentum operator as in

\begin{align*}
\bra{x'} P \ket{x} = -i \hbar \delta(x-x') \frac{d}{dx'}. 
\end{align*}

This is very much like the fact that the delta function only shows up in the continuous representation in other context where one has matrix elements.  The most simple example of which is just

\begin{align*}
\braket{x'}{x} = \delta(x-x').
\end{align*}

I also see now that the momentum operator is directly identified with the derivative (no delta function) in two other places in the text.  These are equations (2.32) and (2.46) respectively:

\begin{align*}
P(x) &= -i \hbar \frac{d}{dx} \\
P &= -i \hbar \frac{d}{dX}.
\end{align*}

In the first, (2.32), I thought the $P(x)$ was somehow different, just a helpful expression found along the way, but now it occurs to me that this was intended to be an unambiguous representation of the momentum operator itself.

FIXME: rework the expansion of $D(\Delta x') \phi(x)$ above in light of this.  Examine the connection between that and 
$D(\Delta x') \ket{\phi}$ to get a feel for all the notational magic.

\subsection{Time evolution operator}

The phrase ``we identify time evolution with the Hamiltonian''.  What a magic hat maneuver!  Is there a way that this would be logical without already knowing the answer?

\subsection{Dispersion delta function representation.}

The Principle part notation here I found a bit unclear.  He writes

\begin{align*}
\lim_{\epsilon \rightarrow 0} 
\frac{(x'-x)}{(x'-x)^2 + \epsilon^2}
= 
P\left( \inv{x' - x} \right).
\end{align*}

In complex variables the principle part is the negative power series terms.  For example for $f(z) = \sum a_k z^k$, the principle part is

\begin{align*}
\sum_{k = -\infty}^{-1} a_k z^k
\end{align*}

This doesn't vanish at $z = 0$ as the principle part in this section is stated to.  In (2.202) he pulls the $P$ out of the integral, but I think the intention is really to keep this associated with the $1/(x'-x)$, as in

\begin{align*}
\lim_{\epsilon \rightarrow 0} 
\inv{\pi} \int_0^\infty dx' \frac{f(x')}{x'-x - i \epsilon}
= 
\inv{\pi} \int_0^\infty dx' f(x') P\left( \inv{x' - x} \right) + i f(x)
\end{align*}

Will this even have any relevance in this text?

\section{Problems.}
\subsection{1. Cauchy-Shwartz identity.}
\subsubsection{1. Blundering on the standard trick.}

This proof is a standard one from a linear algebra book, one for which has a tricky trick that I never remember.  For $\ket{c} = \ket{a} + \lambda \ket{b}$, we wish to take the inner product and find the magic value of $\lambda$ that leaves only real terms in the resulting expression.

That is

\begin{align*}
\Abs{\ket{a} + \lambda \ket{b}}^2 
&=
\braket{a + \lambda b}{a + \lambda b} \\
&= \Abs{\ket{a}}^2 + \Abs{\lambda}^2 \Abs{\ket{b}}^2 + \lambda \braket{a}{b} + \lambda^\conj \braket{b}{a} \\
\end{align*}

Let's try $\lambda = \braket{b}{a}/\Abs{\ket{b}}^n$.  This gives

\begin{align*}
\Abs{\ket{a} + \lambda \ket{b}}^2 
&= \Abs{\ket{a}}^2 + \Abs{\braket{b}{a}}^2 \Abs{\ket{b}}^{2-2n} + \Abs{\braket{a}{b}}^2/\Abs{\ket{b}}^n + \Abs{\braket{b}{a}}^2/\Abs{\ket{b}}^n \\
\end{align*}

We want terms to drop out so we pick $2 -2n = -n$, or $n = 2$.  This gives

\begin{align*}
\Abs{\ket{a} + \lambda \ket{b}}^2 
&= \Abs{\ket{a}}^2 + 3 \Abs{\braket{b}{a}}^2/\Abs{\ket{b}}^2 \\
\end{align*}

It appears that this wasn't the magic value desired.  If we pick $\lambda = -\braket{b}{a}/\Abs{\ket{b}}^n$ instead, we have what we want:

\begin{align*}
\Abs{\ket{a} + \lambda \ket{b}}^2 
&= \Abs{\ket{a}}^2 - \Abs{\braket{b}{a}}^2/\Abs{\ket{b}}^2 \ge 0
\end{align*}

And rearranging, and dropping the $\Abs{}$ notation in favour of inner products, we have the desired result:

\begin{align*}
\braket{a}{a} \braket{b}{b} \ge \braket{b}{a}\braket{a}{b}.
\end{align*}

\subsubsection{1. As a min/max problem.}

The trial and error approach above, where we trying to blunder upon the well known but obscure trick, kind of sucks.  We can also do this as a min/max problem, although this is made slightly messier by the complex vector space.  Define

\begin{align*}
f(\lambda) =
\braket{a}{a} + \lambda \lambda^\conj \braket{b}{b} + \lambda \braket{a}{b} + \lambda^\conj \braket{b}{a} \\
\end{align*}

Now, set $df/d\lambda = 0$

\begin{align*}
\frac{df}{d\lambda} 
&=
\left(\lambda^\conj + \lambda \frac{d\lambda^\conj}{d\lambda}\right) \braket{b}{b} + \braket{a}{b} + \frac{d\lambda^\conj}{d\lambda} \braket{b}{a} \\
&=
\lambda^\conj \braket{b}{b} + \braket{a}{b} 
+
\frac{d\lambda^\conj}{d\lambda} \Bigl( 
\lambda \braket{b}{b} + \braket{b}{a} \Bigr)
\end{align*}

Now, we have a bit of a problem with $d\lambda^\conj/d\lambda$, since that doesn't actually exist.  However, if we insist that what multiplies it is zero, we have $df/d\lambda = 0$ as desired.  This yields 

\begin{align*}
\lambda = - \frac{\braket{b}{a} }{ \braket{b}{b}  }
\end{align*}

as desired, and the remainder of the proof follows as before.  This is nicer, as slightly less black magic is required.

\subsection{2.}
\subsection{3.}
\subsection{4.}
\subsection{5. Hermitian radial differential operator.}

Show that the operator 

\begin{align*}
R = -i \hbar \PD{r}{},
\end{align*}

is not Hermitian, and find the constant $a$ so that 

\begin{align*}
T = -i \hbar \left( \PD{r}{} + \frac{a}{r} \right),
\end{align*}

is Hermitian.

For the first part of the problem we can show that

\begin{align*}
\left( \bra{\psicap} R \ket{\phicap} \right)^\conj \ne \bra{\phicap} R \ket{\psicap}.
\end{align*}

For the RHS we have

\begin{align*}
\bra{\phicap} R \ket{\psicap} 
= -i \hbar \iiint dr d\theta d\phi r^2 \sin\theta \phicap^\conj \PD{r}{\psicap}
\end{align*}

and for the LHS we have

\begin{align*}
\left( \bra{\psicap} R \ket{\phicap} \right)^\conj
&= i \hbar \iiint dr d\theta d\phi r^2 \sin\theta \psicap \PD{r}{\phicap^\conj} \\
&= -i \hbar \iiint dr d\theta d\phi \sin\theta 
\left( 2 r \psicap 
+ r^2 \PD{\psicap}{r} 
\right)
\phicap^\conj 
\\
\end{align*}

So, unless $r\psicap = 0$, the operator $R$ is not Hermitian.

Moving on to finding the constant $a$ such that $T$ is Hermitian we calculate

\begin{align*}
\left( \bra{\psicap} T \ket{\phicap} \right)^\conj
&= i \hbar \iiint dr d\theta d\phi r^2 \sin\theta \psicap \left( \PD{r}{} + \frac{a}{r} \right) \phicap^\conj \\
&= i \hbar \iiint dr d\theta d\phi \sin\theta \psicap \left( r^2 \PD{r}{} + a r \right) \phicap^\conj \\
&= -i \hbar \iiint dr d\theta d\phi \sin\theta \left( r^2 \PD{r}{\psicap} + 2 r \psicap - a r \psicap \right) \phicap^\conj \\
\end{align*}

and

\begin{align*}
\bra{\phicap} T \ket{\psicap} 
= -i \hbar \iiint dr d\theta d\phi r^2 \sin\theta \phicap^\conj \left( r^2 \PD{r}{\psicap} + a r \psicap \right)
\end{align*}

So, for $T$ to be Hermitian, we require

\begin{align*}
2 r - a r = a r.
\end{align*}

So $a = 1$, and our Hermitian operator is
\begin{align*}
T = -i \hbar \left( \PD{r}{} + \frac{1}{r} \right).
\end{align*}

\subsection{6. Radial directional derivative operator.}

\subsubsection{Problem.}
Show that 

\begin{align*}
D = \Bp \cdot \rcap + \rcap \cdot \Bp,
\end{align*}

is Hermitian.  Expand this operator in spherical coordinates.  Compare result to problem 5.

\subsubsection{Solution.}

Tackling the spherical coordinates expression of of the operator $D$, we have

\begin{align*}
\inv{-i\hbar} D \Psi 
&= \left( \spacegrad \cdot \rcap + \rcap \cdot \spacegrad \right) \Psi \\
&= 
\left( \spacegrad \cdot \rcap \right) \Psi 
+ \left( \spacegrad \Psi \right) \cdot \rcap 
+ \rcap \cdot \left(\spacegrad \Psi\right) \\
&=
\left( \spacegrad \cdot \rcap \right) \Psi 
+ 2 \rcap \cdot \left( \spacegrad \Psi \right).
\end{align*}

Here braces have been used to denote the extend of the operation of the gradient.  In spherical polar coordinates, our gradient is

\begin{align*}
\spacegrad \equiv 
\rcap \PD{r}{}
+\thetacap \inv{r} \PD{\theta}{}
+\phicap \inv{r \sin\theta} \PD{\phi}{}.
\end{align*}

This gets us most of the way there, and we have

\begin{align*}
\inv{-i\hbar} D \Psi 
&=
2 \PD{r}{\Psi} 
+ 
\left( 
\rcap \cdot \PD{r}{\rcap}
+\inv{r} \thetacap \cdot \PD{\theta}{\rcap}
+\inv{r \sin\theta} \phicap \cdot \PD{\phi}{\rcap}
\right) \Psi.
\end{align*}

Since $\PDi{r}{\rcap} = 0$, we are left with evaluating $\thetacap \cdot \PDi{\theta}{\rcap}$, and $\phicap \cdot \PDi{\phi}{\rcap}$.  To do so I chose to employ the (Geometric Algebra) exponential form of the spherical unit vectors \cite{sphericalPolarUnit}

\begin{align*}
I &= \Be_1 \Be_2 \Be_3 \\
\phicap &= \Be_{2} \exp( I \Be_3 \phi ) \\
\rcap &= \Be_3 \exp( I \phicap \theta ) \\
\thetacap &= \Be_1 \Be_2 \phicap \exp( I \phicap \theta ).
\end{align*}

The partials of interest are then

\begin{align*}
\PD{\theta}{\rcap} &= \Be_3 I \phicap \exp( I \phicap \theta ) = \thetacap,
\end{align*}

and

\begin{align*}
\PD{\phi}{\rcap} 
&= \PD{\phi}{} \Be_3 \left( \cos\theta + I \phicap \sin\theta \right) \\
&= \Be_1 \Be_2 \sin\theta \PD{\phi}{\phicap} \\
&= \Be_1 \Be_2 \sin\theta \Be_2 \Be_1 \Be_2 \exp( I \Be_3 \phi ) \\
&= \sin\theta \phicap.
\end{align*}

Only after computing these, did I find exactly these results for the partials of interest, in \href{http://mathworld.wolfram.com/SphericalCoordinates.html}{mathworld's Spherical Coordinates page}, which confirms these calculations.  Note that a different angle convention is used there, so one has to exchange $\phi$, and $\theta$ and the corresponding unit vector labels.

Substitution back into our expression for the operator we have
\begin{align*}
D &= - 2 i \hbar \left( \PD{r}{} + \inv{r} \right),
\end{align*}

an operator that is exactly twice the operator of problem 5, already shown to be Hermitian.  Since the constant numerical scaling of a Hermitian operator leaves it Hermitian, this shows that $D$ is Hermitian as expected.

\subsubsection{$\thetacap$ directional momentum operator}

Let's try this for the other unit vector directions too.  We also want

\begin{align*}
\left( \spacegrad \cdot \thetacap + \thetacap \cdot \spacegrad \right) \Psi
&=
2 \thetacap \cdot (\spacegrad \Psi) + \left( \spacegrad \cdot \thetacap \right) \Psi.
\end{align*}

The work consists of evaluating

\begin{align*}
\spacegrad \cdot \thetacap 
&= \rcap \cdot \PD{r}{\thetacap}
+ \inv{r} \thetacap \cdot \PD{\theta}{\thetacap}
+ \inv{r \sin\theta} \phicap \cdot \PD{\phi}{\thetacap}.
\end{align*}

This time we need the $\PDi{\theta}{\thetacap}$, $\PDi{\phi}{\thetacap}$ partials, which are

\begin{align*}
\PD{\theta}{\thetacap} 
&=
\Be_1 \Be_2 \phicap I \phicap \exp( I \phicap \theta) \\
&=
-\Be_3 \exp( I \phicap \theta) \\
&=
- \rcap.
\end{align*}

This has no $\thetacap$ component, so does not contribute to $\spacegrad \cdot \thetacap$.  Noting that

\begin{align*}
\PD{\phi}{\phicap} &= -\Be_1 \exp( I \Be_3 \phi ) = \Be_2 \Be_1 \phicap,
\end{align*}

the $\phi$ partial is

\begin{align*}
\PD{\phi}{\thetacap} &=
\Be_1 \Be_2 \left( 
\PD{\phi}{\phicap} \exp( I \phicap \theta )
+\phicap I \sin\theta \PD{\phi}{\phicap} 
\right) \\
&=
\phicap 
\left( 
\exp( I \phicap \theta )
+I \sin\theta \Be_2 \Be_1 \phicap
\right),
\end{align*}

with $\phicap$ component
\begin{align*}
\phicap \cdot \PD{\phi}{\thetacap} &=
\gpgradezero{
\exp( I \phicap \theta )
+I \sin\theta \Be_2 \Be_1 \phicap } \\
&=
\cos\theta + \Be_3 \cdot \phicap \sin\theta \\
&=
\cos\theta.
\end{align*}

Assembling the results, and labeling this operator $\Theta$ we have

\begin{align*}
\Theta &\equiv \inv{2} \left( \Bp \cdot \thetacap + \thetacap \cdot \Bp \right)  \\
&=
-i \hbar \inv{r} \left( \PD{\theta}{} + \inv{2} \cot\theta \right).
\end{align*}

It would be reasonable to expect this operator to also be Hermitian, and checking this explicitly by comparing
$\bra{\Phi} \Theta \ket{\Psi}^\conj$ and $\bra{\Psi} \Theta \ket{\Phi}$, shows that this is in fact the case.

\subsubsection{$\phicap$ directional momentum operator}

Let's try this for the other unit vector directions too.  We also want

\begin{align*}
\left( \spacegrad \cdot \phicap + \phicap \cdot \spacegrad \right) \Psi
&=
2 \phicap \cdot (\spacegrad \Psi) + \left( \spacegrad \cdot \phicap \right) \Psi.
\end{align*}

The work consists of evaluating

\begin{align*}
\spacegrad \cdot \phicap 
&= \rcap \cdot \PD{r}{\phicap}
+ \inv{r} \thetacap \cdot \PD{\theta}{\phicap}
+ \inv{r \sin\theta} \phicap \cdot \PD{\phi}{\phicap}.
\end{align*}

This time we need the $\PDi{\theta}{\phicap}$, $\PDi{\phi}{\phicap} = \Be_2 \Be_1 \phicap$ partials.  The $\theta$ partial is

\begin{align*}
\PD{\theta}{\phicap} 
&=
\PD{\theta}{} \Be_2 \exp( I \Be_3 \phi ) \\
&= 0.
\end{align*}

We conclude that $\spacegrad \cdot \phicap = 0$, and expect that we have one more Hermitian operator

\begin{align*}
\Phi &\equiv \inv{2} \left( \Bp \cdot \phicap + \phicap \cdot \Bp \right)  \\
&=
-i \hbar \inv{r \sin\theta} \PD{\phi}{}.
\end{align*}

It is simple to confirm that this is Hermitian since the integration by parts does not involve any of the volume element.  In fact, any operator $-i\hbar f(r,\theta) \PDi{\phi}{}$ would also be Hermitian, including the simplest case $-i\hbar \PDi{\phi}{}$.  Have to dig out my Bohm text again, since I seem to recall that one used in the spherical Harmonics chapter.

\subsubsection{A note on the Hermitian test and Dirac notation.}

I've been a bit loose with my notation.  I've stated that my demonstrations of the Hermitian nature have been done by showing

\begin{align*}
\bra{\phi} A \ket{\psi}^\conj - \bra{\psi} A \ket{\phi} = 0.
\end{align*}

However, what I've actually done is show that 

\begin{align*}
\left( \int d^3 \Bx \phi^\conj (\Bx) A(\Bx) \psi(\Bx) \right)^\conj - \int d^3 \Bx \psi^\conj (\Bx) A(\Bx) \phi(\Bx) = 0.
\end{align*}

To justify this note that 

\begin{align*}
\bra{\phi} A \ket{\psi}^\conj 
&=
\left( \iint d^3 \Br d^3 \Bs \braket{\phi}{\Br} \bra{\Br} A \ket{\Bs} \braket{\Bs}{\psi} \right)^\conj \\
&=
\iint d^3 \Br d^3 \Bs \phi(\Br) \delta^3(\Br - \Bs) A^\conj(\Bs) \psi(\Bs) \\
&=
\int d^3 \Br \phi(\Br) A^\conj(\Br) \psi(\Br),
\end{align*}

and
\begin{align*}
\bra{\phi} A \ket{\psi}^\conj 
&=
\iint d^3 \Br d^3 \Bs \braket{\psi}{\Br} \bra{\Br} A \ket{\Bs} \braket{\Bs}{\phi} \\
&=
\iint d^3 \Br d^3 \Bs \bra{\Br} \psi(\Br) \delta^3(\Br - \Bs) A(\Bs) \phi(\Bs) \\
&=
\int d^3 \Br \psi(\Br) A(\Br) \phi(\Br).
\end{align*}

Working backwards one sees that the comparison of the wave function integrals in explicit inner product notation is sufficient to demonstrate the Hermitian property.

\subsection{7. Some commutators.}
\subsubsection{7. Problem.}

For $D$ in problem 6, obtain

\begin{itemize}
\item i) $[D, x_i]$
\item ii) $[D, p_i]$
\item iii) $[D, L_i]$, where $L_i = \Be_i \cdot (\Br \cross \Bp)$.
\item iv) Show that $e^{i\alpha D/\hbar} x_i e^{-i\alpha D/\hbar} = e^\alpha x_i$
\end{itemize}

\subsubsection{7. Solution.}

To start observe that we can write the operator in a form more convient for use with cartesian coordinates:

\begin{align*}
D = -2 i\hbar \inv{r}( \Br \cdot \spacegrad + 1)
\end{align*}

\begin{itemize}
\item i) 

\begin{align*}
[D, x_i] \Psi
&=
D x_i \Psi - x_i D \Psi \\
&=
-2 i \hbar \inv{r} \left( \Br \cdot \spacegrad + 1 \right) x_i \Psi
+2 i \hbar x_i \inv{r} \left( \Br \cdot \spacegrad + 1 \right) \Psi \\
&=
-2 i \hbar \inv{r} \Br \cdot \spacegrad x_i \Psi
+2 i \hbar x_i \inv{r} \Br \cdot \spacegrad \Psi \\
&=
-2 i \hbar \inv{r} \Br \cdot (\spacegrad x_i) \Psi
-2 i \hbar x_i \inv{r} \Br \cdot \spacegrad \Psi
+2 i \hbar x_i \inv{r} \Br \cdot \spacegrad \Psi \\
&=
-2 i \hbar \inv{r} \Br \cdot \Be_i \Psi.
\end{align*}

So this first commutator is:

\begin{align*}
[D, x_i] = -2 i \hbar \frac{x_i}{r}.
\end{align*}

\item ii) $[D, p_i]$
\item iii) $[D, L_i]$

\item iv) Show that $e^{i\alpha D/\hbar} x_i e^{-i\alpha D/\hbar} = e^\alpha x_i$
\end{itemize}

\subsection{8.}
\subsection{9.}
\subsection{10.}
\subsection{11.}

\EndArticle
%\EndNoBibArticle

%
% Copyright � 2015 Peeter Joot.  All Rights Reserved.
% Licenced as described in the file LICENSE under the root directory of this GIT repository.
%
\documentclass[]{eliblog}

\usepackage{amsmath}
\usepackage{mathpazo}

%
% shorthand for bold symbols, convenient for vectors and matrices
%
\newcommand{\Ba}[0]{\mathbf{a}}
\newcommand{\Bb}[0]{\mathbf{b}}
\newcommand{\Bc}[0]{\mathbf{c}}
\newcommand{\Bd}[0]{\mathbf{d}}
\newcommand{\Be}[0]{\mathbf{e}}
\newcommand{\Bf}[0]{\mathbf{f}}
\newcommand{\Bg}[0]{\mathbf{g}}
\newcommand{\Bh}[0]{\mathbf{h}}
\newcommand{\Bi}[0]{\mathbf{i}}
\newcommand{\Bj}[0]{\mathbf{j}}
\newcommand{\Bk}[0]{\mathbf{k}}
\newcommand{\Bl}[0]{\mathbf{l}}
\newcommand{\Bm}[0]{\mathbf{m}}
\newcommand{\Bn}[0]{\mathbf{n}}
\newcommand{\Bo}[0]{\mathbf{o}}
\newcommand{\Bp}[0]{\mathbf{p}}
\newcommand{\Bq}[0]{\mathbf{q}}
\newcommand{\Br}[0]{\mathbf{r}}
\newcommand{\Bs}[0]{\mathbf{s}}
\newcommand{\Bt}[0]{\mathbf{t}}
\newcommand{\Bu}[0]{\mathbf{u}}
\newcommand{\Bv}[0]{\mathbf{v}}
\newcommand{\Bw}[0]{\mathbf{w}}
\newcommand{\Bx}[0]{\mathbf{x}}
\newcommand{\By}[0]{\mathbf{y}}
\newcommand{\Bz}[0]{\mathbf{z}}
\newcommand{\BA}[0]{\mathbf{A}}
\newcommand{\BB}[0]{\mathbf{B}}
\newcommand{\BC}[0]{\mathbf{C}}
\newcommand{\BD}[0]{\mathbf{D}}
\newcommand{\BE}[0]{\mathbf{E}}
\newcommand{\BF}[0]{\mathbf{F}}
\newcommand{\BG}[0]{\mathbf{G}}
\newcommand{\BH}[0]{\mathbf{H}}
\newcommand{\BI}[0]{\mathbf{I}}
\newcommand{\BJ}[0]{\mathbf{J}}
\newcommand{\BK}[0]{\mathbf{K}}
\newcommand{\BL}[0]{\mathbf{L}}
\newcommand{\BM}[0]{\mathbf{M}}
\newcommand{\BN}[0]{\mathbf{N}}
\newcommand{\BO}[0]{\mathbf{O}}
\newcommand{\BP}[0]{\mathbf{P}}
\newcommand{\BQ}[0]{\mathbf{Q}}
\newcommand{\BR}[0]{\mathbf{R}}
\newcommand{\BS}[0]{\mathbf{S}}
\newcommand{\BT}[0]{\mathbf{T}}
\newcommand{\BU}[0]{\mathbf{U}}
\newcommand{\BV}[0]{\mathbf{V}}
\newcommand{\BW}[0]{\mathbf{W}}
\newcommand{\BX}[0]{\mathbf{X}}
\newcommand{\BY}[0]{\mathbf{Y}}
\newcommand{\BZ}[0]{\mathbf{Z}}

\newcommand{\Bzero}[0]{\mathbf{0}}
\newcommand{\Btheta}[0]{\boldsymbol{\theta}}
\newcommand{\Btau}[0]{\boldsymbol{\tau}}
\newcommand{\Bomega}[0]{\boldsymbol{\omega}}

%
% shorthand for unit vectors
%
\newcommand{\acap}[0]{\hat{\Ba}}
\newcommand{\bcap}[0]{\hat{\Bb}}
\newcommand{\ccap}[0]{\hat{\Bc}}
\newcommand{\dcap}[0]{\hat{\Bd}}
\newcommand{\ecap}[0]{\hat{\Be}}
\newcommand{\fcap}[0]{\hat{\Bf}}
\newcommand{\gcap}[0]{\hat{\Bg}}
\newcommand{\hcap}[0]{\hat{\Bh}}
\newcommand{\icap}[0]{\hat{\Bi}}
\newcommand{\jcap}[0]{\hat{\Bj}}
\newcommand{\kcap}[0]{\hat{\Bk}}
\newcommand{\lcap}[0]{\hat{\Bl}}
\newcommand{\mcap}[0]{\hat{\Bm}}
\newcommand{\ncap}[0]{\hat{\Bn}}
\newcommand{\ocap}[0]{\hat{\Bo}}
\newcommand{\pcap}[0]{\hat{\Bp}}
\newcommand{\qcap}[0]{\hat{\Bq}}
\newcommand{\rcap}[0]{\hat{\Br}}
\newcommand{\scap}[0]{\hat{\Bs}}
\newcommand{\tcap}[0]{\hat{\Bt}}
\newcommand{\ucap}[0]{\hat{\Bu}}
\newcommand{\vcap}[0]{\hat{\Bv}}
\newcommand{\wcap}[0]{\hat{\Bw}}
\newcommand{\xcap}[0]{\hat{\Bx}}
\newcommand{\ycap}[0]{\hat{\By}}
\newcommand{\zcap}[0]{\hat{\Bz}}
\newcommand{\thetacap}[0]{\hat{\Btheta}}

%
% to write R^n and C^n in a distinguishable fashion.  Perhaps change this
% to the double lined characters upon figuring out how to do so.
%
\newcommand{\C}[1]{$\mathbb{C}^{#1}$}
\newcommand{\R}[1]{$\mathbb{R}^{#1}$}

%
% various generally useful helpers
%

% derivative of #1 wrt. #2:
\newcommand{\D}[2] {\frac {d#2} {d#1}}

\newcommand{\inv}[1]{\frac{1}{#1}}
\newcommand{\cross}[0]{\times}

\newcommand{\abs}[1]{\lvert{#1}\rvert}
\newcommand{\norm}[1]{\lVert{#1}\rVert}
\newcommand{\innerprod}[2]{\langle{#1}, {#2}\rangle}
\newcommand{\dotprod}[2]{{#1} \cdot {#2}}
\newcommand{\bdotprod}[2]{\left({#1} \cdot {#2}\right)}
\newcommand{\crossprod}[2]{{#1} \cross {#2}}
\newcommand{\tripleprod}[3]{\dotprod{\left(\crossprod{#1}{#2}\right)}{#3}}

\DeclareMathOperator{\Proj}{Proj}
\DeclareMathOperator{\Span}{span}
\DeclareMathOperator{\Sgn}{sgn}
\DeclareMathOperator{\Area}{Area}
\DeclareMathOperator{\Volume}{Volume}

%
% A few miscellaneous things specific to this document
%
\newcommand{\crossop}[1]{\crossprod{#1}{}}

% R2 vector.
\newcommand{\VectorTwo}[2]{
\begin{bmatrix}
 {#1} \\
 {#2}
\end{bmatrix}
}

\newcommand{\VectorN}[1]{
\begin{bmatrix}
{#1}_1 \\
{#1}_2 \\
\vdots \\
{#1}_N \\
\end{bmatrix}
}

\newcommand{\DETuvij}[4]{
\begin{vmatrix}
 {#1}_{#3} & {#1}_{#4} \\
 {#2}_{#3} & {#2}_{#4}
\end{vmatrix}
}

\newcommand{\DETuvwijk}[6]{
\begin{vmatrix}
 {#1}_{#4} & {#1}_{#5} & {#1}_{#6} \\
 {#2}_{#4} & {#2}_{#5} & {#2}_{#6} \\
 {#3}_{#4} & {#3}_{#5} & {#3}_{#6}
\end{vmatrix}
}

\newcommand{\DETuvwxijkl}[8]{
\begin{vmatrix}
 {#1}_{#5} & {#1}_{#6} & {#1}_{#7} & {#1}_{#8} \\
 {#2}_{#5} & {#2}_{#6} & {#2}_{#7} & {#2}_{#8} \\
 {#3}_{#5} & {#3}_{#6} & {#3}_{#7} & {#3}_{#8} \\
 {#4}_{#5} & {#4}_{#6} & {#4}_{#7} & {#4}_{#8} \\
\end{vmatrix}
}

%\newcommand{\DETuvwxyijklm}[10]{
%\begin{vmatrix}
% {#1}_{#6} & {#1}_{#7} & {#1}_{#8} & {#1}_{#9} & {#1}_{#10} \\
% {#2}_{#6} & {#2}_{#7} & {#2}_{#8} & {#2}_{#9} & {#2}_{#10} \\
% {#3}_{#6} & {#3}_{#7} & {#3}_{#8} & {#3}_{#9} & {#3}_{#10} \\
% {#4}_{#6} & {#4}_{#7} & {#4}_{#8} & {#4}_{#9} & {#4}_{#10} \\
% {#5}_{#6} & {#5}_{#7} & {#5}_{#8} & {#5}_{#9} & {#5}_{#10}
%\end{vmatrix}
%}

% R3 vector.
\newcommand{\VectorThree}[3]{
\begin{bmatrix}
 {#1} \\
 {#2} \\
 {#3}
\end{bmatrix}
}



\author{Peeter Joot}
\email{peeter.joot@gmail.com}

%\documentclass[]{eliblogwidescreen}

\usepackage{amsmath}
\usepackage{mathpazo}

%
% shorthand for bold symbols, convenient for vectors and matrices
%
\newcommand{\Ba}[0]{\mathbf{a}}
\newcommand{\Bb}[0]{\mathbf{b}}
\newcommand{\Bc}[0]{\mathbf{c}}
\newcommand{\Bd}[0]{\mathbf{d}}
\newcommand{\Be}[0]{\mathbf{e}}
\newcommand{\Bf}[0]{\mathbf{f}}
\newcommand{\Bg}[0]{\mathbf{g}}
\newcommand{\Bh}[0]{\mathbf{h}}
\newcommand{\Bi}[0]{\mathbf{i}}
\newcommand{\Bj}[0]{\mathbf{j}}
\newcommand{\Bk}[0]{\mathbf{k}}
\newcommand{\Bl}[0]{\mathbf{l}}
\newcommand{\Bm}[0]{\mathbf{m}}
\newcommand{\Bn}[0]{\mathbf{n}}
\newcommand{\Bo}[0]{\mathbf{o}}
\newcommand{\Bp}[0]{\mathbf{p}}
\newcommand{\Bq}[0]{\mathbf{q}}
\newcommand{\Br}[0]{\mathbf{r}}
\newcommand{\Bs}[0]{\mathbf{s}}
\newcommand{\Bt}[0]{\mathbf{t}}
\newcommand{\Bu}[0]{\mathbf{u}}
\newcommand{\Bv}[0]{\mathbf{v}}
\newcommand{\Bw}[0]{\mathbf{w}}
\newcommand{\Bx}[0]{\mathbf{x}}
\newcommand{\By}[0]{\mathbf{y}}
\newcommand{\Bz}[0]{\mathbf{z}}
\newcommand{\BA}[0]{\mathbf{A}}
\newcommand{\BB}[0]{\mathbf{B}}
\newcommand{\BC}[0]{\mathbf{C}}
\newcommand{\BD}[0]{\mathbf{D}}
\newcommand{\BE}[0]{\mathbf{E}}
\newcommand{\BF}[0]{\mathbf{F}}
\newcommand{\BG}[0]{\mathbf{G}}
\newcommand{\BH}[0]{\mathbf{H}}
\newcommand{\BI}[0]{\mathbf{I}}
\newcommand{\BJ}[0]{\mathbf{J}}
\newcommand{\BK}[0]{\mathbf{K}}
\newcommand{\BL}[0]{\mathbf{L}}
\newcommand{\BM}[0]{\mathbf{M}}
\newcommand{\BN}[0]{\mathbf{N}}
\newcommand{\BO}[0]{\mathbf{O}}
\newcommand{\BP}[0]{\mathbf{P}}
\newcommand{\BQ}[0]{\mathbf{Q}}
\newcommand{\BR}[0]{\mathbf{R}}
\newcommand{\BS}[0]{\mathbf{S}}
\newcommand{\BT}[0]{\mathbf{T}}
\newcommand{\BU}[0]{\mathbf{U}}
\newcommand{\BV}[0]{\mathbf{V}}
\newcommand{\BW}[0]{\mathbf{W}}
\newcommand{\BX}[0]{\mathbf{X}}
\newcommand{\BY}[0]{\mathbf{Y}}
\newcommand{\BZ}[0]{\mathbf{Z}}

\newcommand{\Bzero}[0]{\mathbf{0}}
\newcommand{\Btheta}[0]{\boldsymbol{\theta}}
\newcommand{\Btau}[0]{\boldsymbol{\tau}}
\newcommand{\Bomega}[0]{\boldsymbol{\omega}}

%
% shorthand for unit vectors
%
\newcommand{\acap}[0]{\hat{\Ba}}
\newcommand{\bcap}[0]{\hat{\Bb}}
\newcommand{\ccap}[0]{\hat{\Bc}}
\newcommand{\dcap}[0]{\hat{\Bd}}
\newcommand{\ecap}[0]{\hat{\Be}}
\newcommand{\fcap}[0]{\hat{\Bf}}
\newcommand{\gcap}[0]{\hat{\Bg}}
\newcommand{\hcap}[0]{\hat{\Bh}}
\newcommand{\icap}[0]{\hat{\Bi}}
\newcommand{\jcap}[0]{\hat{\Bj}}
\newcommand{\kcap}[0]{\hat{\Bk}}
\newcommand{\lcap}[0]{\hat{\Bl}}
\newcommand{\mcap}[0]{\hat{\Bm}}
\newcommand{\ncap}[0]{\hat{\Bn}}
\newcommand{\ocap}[0]{\hat{\Bo}}
\newcommand{\pcap}[0]{\hat{\Bp}}
\newcommand{\qcap}[0]{\hat{\Bq}}
\newcommand{\rcap}[0]{\hat{\Br}}
\newcommand{\scap}[0]{\hat{\Bs}}
\newcommand{\tcap}[0]{\hat{\Bt}}
\newcommand{\ucap}[0]{\hat{\Bu}}
\newcommand{\vcap}[0]{\hat{\Bv}}
\newcommand{\wcap}[0]{\hat{\Bw}}
\newcommand{\xcap}[0]{\hat{\Bx}}
\newcommand{\ycap}[0]{\hat{\By}}
\newcommand{\zcap}[0]{\hat{\Bz}}
\newcommand{\thetacap}[0]{\hat{\Btheta}}

%
% to write R^n and C^n in a distinguishable fashion.  Perhaps change this
% to the double lined characters upon figuring out how to do so.
%
\newcommand{\C}[1]{$\mathbb{C}^{#1}$}
\newcommand{\R}[1]{$\mathbb{R}^{#1}$}

%
% various generally useful helpers
%

% derivative of #1 wrt. #2:
\newcommand{\D}[2] {\frac {d#2} {d#1}}

\newcommand{\inv}[1]{\frac{1}{#1}}
\newcommand{\cross}[0]{\times}

\newcommand{\abs}[1]{\lvert{#1}\rvert}
\newcommand{\norm}[1]{\lVert{#1}\rVert}
\newcommand{\innerprod}[2]{\langle{#1}, {#2}\rangle}
\newcommand{\dotprod}[2]{{#1} \cdot {#2}}
\newcommand{\bdotprod}[2]{\left({#1} \cdot {#2}\right)}
\newcommand{\crossprod}[2]{{#1} \cross {#2}}
\newcommand{\tripleprod}[3]{\dotprod{\left(\crossprod{#1}{#2}\right)}{#3}}

\DeclareMathOperator{\Proj}{Proj}
\DeclareMathOperator{\Span}{span}
\DeclareMathOperator{\Sgn}{sgn}
\DeclareMathOperator{\Area}{Area}
\DeclareMathOperator{\Volume}{Volume}

%
% A few miscellaneous things specific to this document
%
\newcommand{\crossop}[1]{\crossprod{#1}{}}

% R2 vector.
\newcommand{\VectorTwo}[2]{
\begin{bmatrix}
 {#1} \\
 {#2}
\end{bmatrix}
}

\newcommand{\VectorN}[1]{
\begin{bmatrix}
{#1}_1 \\
{#1}_2 \\
\vdots \\
{#1}_N \\
\end{bmatrix}
}

\newcommand{\DETuvij}[4]{
\begin{vmatrix}
 {#1}_{#3} & {#1}_{#4} \\
 {#2}_{#3} & {#2}_{#4}
\end{vmatrix}
}

\newcommand{\DETuvwijk}[6]{
\begin{vmatrix}
 {#1}_{#4} & {#1}_{#5} & {#1}_{#6} \\
 {#2}_{#4} & {#2}_{#5} & {#2}_{#6} \\
 {#3}_{#4} & {#3}_{#5} & {#3}_{#6}
\end{vmatrix}
}

\newcommand{\DETuvwxijkl}[8]{
\begin{vmatrix}
 {#1}_{#5} & {#1}_{#6} & {#1}_{#7} & {#1}_{#8} \\
 {#2}_{#5} & {#2}_{#6} & {#2}_{#7} & {#2}_{#8} \\
 {#3}_{#5} & {#3}_{#6} & {#3}_{#7} & {#3}_{#8} \\
 {#4}_{#5} & {#4}_{#6} & {#4}_{#7} & {#4}_{#8} \\
\end{vmatrix}
}

%\newcommand{\DETuvwxyijklm}[10]{
%\begin{vmatrix}
% {#1}_{#6} & {#1}_{#7} & {#1}_{#8} & {#1}_{#9} & {#1}_{#10} \\
% {#2}_{#6} & {#2}_{#7} & {#2}_{#8} & {#2}_{#9} & {#2}_{#10} \\
% {#3}_{#6} & {#3}_{#7} & {#3}_{#8} & {#3}_{#9} & {#3}_{#10} \\
% {#4}_{#6} & {#4}_{#7} & {#4}_{#8} & {#4}_{#9} & {#4}_{#10} \\
% {#5}_{#6} & {#5}_{#7} & {#5}_{#8} & {#5}_{#9} & {#5}_{#10}
%\end{vmatrix}
%}

% R3 vector.
\newcommand{\VectorThree}[3]{
\begin{bmatrix}
 {#1} \\
 {#2} \\
 {#3}
\end{bmatrix}
}



\author{Peeter Joot}
\email{peeter.joot@gmail.com}


\chapter{Notes and problems for Desai chapter III.}
\label{chap:desaiCh3}
%\useCCL
\blogpage{http://sites.google.com/site/peeterjoot/math2010/desaiCh3.pdf}
\date{Oct 1, 2010}
\revisionInfo{desaiCh3.tex}

\beginArtWithToc
%\beginArtNoToc

\section{Notes.}

Chapter III notes and problems for \cite{desai2009quantum}.

FIXME:
Some puzzling stuff in the interaction section and superposition of time-dependent states sections.  Work through those here.

\section{Problems}

\subsection{Problem 1. Virial Theorem.}
\subsubsection{Statement.}

With the assumption that $\expectation{\Br \cdot \Bp}$ is independent of time, and 

\begin{align}\label{eqn:desaiCh3:100}
H = \frac{\Bp^2}{2m} + V(\Br) = T + V
\end{align}

show that 

\begin{align}\label{eqn:desaiCh3:102}
2 \expectation{T} = \expectation{ \Br \cdot \spacegrad V}.
\end{align}

\subsubsection{Solution.}

I floundered with this a bit, but found the required hint in \href{http://www.physicsforums.com/showthread.php?t=164682}{physicsforums}.  We can start with the Hamiltonian time derivative relation

\begin{align}\label{eqn:desaiCh3:103}
i\hbar \frac{d A_H}{dt} = \antisymmetric{A_H}{H}
\end{align}

So, with the assumption that $\expectation{\Br \cdot \Bp}$ is independent of time, and the use of a stationary state $\ket{\psi}$ for the expectation calculation we have

\begin{align*}
0 &=
\frac{d}{dt} \expectation{\Br \cdot \Bp}  \\
&=
\frac{d}{dt} \bra{\psi} \Br \cdot \Bp \ket{\psi} \\
&=
\bra{\psi} 
\frac{d}{dt} ( \Br \cdot \Bp ) \ket{\psi} \\
&= 
\inv{i\hbar} \expectation{ \antisymmetric{ \Br \cdot \Bp }{H} } \\
&= 
-\expectation{ \antisymmetric{ \Br \cdot \spacegrad }{\frac{\Bp^2}{2m}} } 
-\expectation{ \antisymmetric{ \Br \cdot \spacegrad }{V(\Br)} }.
\end{align*}

The exercise now becomes one of evaluating the remaining commutators.  For the Laplacian commutator we have

\begin{align*}
\antisymmetric{ \Br \cdot \spacegrad }{\spacegrad^2} \psi
&=
x_m \partial_m \partial_n \partial_n \psi 
- \partial_n \partial_n x_m \partial_m \psi \\
&=
x_m \partial_m \partial_n \partial_n \psi 
- \partial_n \partial_n \psi 
- \partial_n x_m \partial_n \partial_m \psi \\
&=
x_m \partial_m \partial_n \partial_n \psi 
- \partial_n \partial_n \psi 
- \partial_n \partial_n \psi 
- x_m \partial_n \partial_n \partial_m \psi \\
&=
- 2 \spacegrad^2 \psi
\end{align*}

For the potential commutator we have

\begin{align*}
\antisymmetric{ \Br \cdot \spacegrad }{V(\Br)} \psi
&=
x_m \partial_m V \psi 
-V x_m \partial_m \psi  \\
&=
x_m (\partial_m V) \psi 
x_m V \partial_m \psi 
-V x_m \partial_m \psi  \\
&=
\Bigl( \Br \cdot (\spacegrad V) \Bigr) \psi
\end{align*}

Putting all the $\hbar$ factors back in, we get

\begin{align}\label{eqn:desaiCh3:104}
2 \expectation{ \frac{\Bp^2}{2m} } = \expectation{ \Br \cdot (\spacegrad V) },
\end{align}

which is the desired result.

Followup: why assume $\expectation{\Br \cdot \Bp}$ is independent of time?

%\href{http://www.caelestis.de/dateien/UEA05_2.pdf}{http://www.caelestis.de/dateien/UEA05_2.pdf} (ie: we need the Hamiltonian commutator).


\subsection{Problem 2. Application of virial theorem.}

Calculate $\expectation{T}$ with $V = \lambda \ln(r/a)$.

\begin{align*}
\Br \cdot \spacegrad V 
&= r \rcap \cdot \rcap \lambda \PD{r}{\ln(r/a)} \\
&= \lambda r \inv{a} \frac{a}{r} \\
&= \lambda  \\
\implies \\
\expectation{T} &= \lambda/2
\end{align*}

\subsection{Problem 3. Heisenberg Position operator representation.}

\subsubsection{Part I.}
Express $x$ as an operator $x_H$ for $H = \Bp^2/2m$.

With 

\begin{align*}
\bra{\psi} x \ket{\psi} = \bra{\psi_0} U^\dagger x U \ket{\psi_0}
\end{align*}

We want to expand 
\begin{align*}
x_H 
&= U^\dagger x U \\
&= e^{i H t/\hbar} x e^{-iH t/\hbar} \\
&= \sum_{k,l = 0}^\infty \inv{k!} \inv{l!} 
\left(\frac{i H t}{\hbar}\right)^k 
x 
\left(\frac{-i H t}{\hbar}\right)^l .
\end{align*}

We to evaluate $H^k x H^l$ to proceed.  Using $p^n x = -i \hbar n p^{n-1} + x p^n$, we have

\begin{align*}
H^k x 
&= \inv{(2m)^k} p^2k x \\
&= \inv{(2m)^k} \left( -i \hbar (2k) p^{2k -1} + x p^2k \right) \\
&= x H^k + \inv{2m} (-i \hbar) (2k) p p^{2(k-1)}/(2m)^{k-1} \\
&= x H^k - \frac{i \hbar k}{m} p H^{k-1}.
\end{align*}

This gives us
\begin{align*}
x_H 
&= x - \frac{i \hbar p }{m} \sum_{k,l=0}^\infty \frac{k}{k!} \inv{l!}
\left(\frac{i t}{\hbar}\right)^k H^{k-1 + l}
\left(\frac{-i t}{\hbar}\right)^l  \\
&= x - \frac{i \hbar p i t }{m \hbar} 
\end{align*}

Or
\begin{align}\label{eqn:desaiCh3:303}
x_H 
&= x + \frac{p t }{m} 
\end{align}


\subsubsection{Part II.}
Express $x$ as an operator $x_H$ for $H = \Bp^2/2m + V$ with $V = \lambda x^m$.

In retrospect, for the first part of this problem, it would have been better to use the series expansion for this exponential sandwich

Or, in explicit form
\begin{align}\label{eqn:desaiCh3:304}
e^A B e^{-A}
&=
B 
+ \inv{1!} \antisymmetric{A}{B}
+ \inv{2!} 
\antisymmetric{A}{\antisymmetric{A}{B}}
+ \cdots
\end{align}

Doing so, we'd find for the first commutator

\begin{align}\label{eqn:desaiCh3:305}
\frac{i t}{2m \hbar} \antisymmetric{\Bp^2}{x} = \frac{t p}{m},
\end{align}

so that the series has only the first two terms, and we'd obtain the same result.  That seems like a logical approach to try here too.  For the first commutator, we get the same $tp/m$ result since $\antisymmetric{V,x} = 0$.

Employing 
\begin{align}\label{eqn:desaiCh3:306}
x^n p = i \hbar n x^{n-1} + p x^n,
\end{align}

I find 

\begin{align*}
\left( \frac{i t}{\hbar} \right)^2 \antisymmetric{H}{\antisymmetric{H}{x}} 
&= \frac{i \lambda t^2}{\hbar m } \antisymmetric{x^n}{p}  \\
&= - \frac{n t^2 \lambda}{m} x^{n-1} \\
&= - \frac{n t^2 V}{m x} \\
\end{align*}

The triple commutator gets no prettier, and I get

\begin{align*}
\left( \frac{i t}{\hbar} \right)^3 \antisymmetric{H}{\antisymmetric{H}{\antisymmetric{H}{x}}}
&= 
\frac{it}{\hbar} \antisymmetric{ \frac{\Bp^2}{2m} + \lambda x^n}{ - \frac{n t^2 V}{m x} } \\
&= 
-\frac{it}{\hbar} \frac{n t^2 }{m } \frac{\lambda}{2m} \antisymmetric{\Bp^2}{ x^{n-1}} \\
&= \cdots \\
&= \frac{n(n-1)t^3 V}{ 2 m^2 x^3 } (i \hbar n + 2 p x).
\end{align*}

Putting all the pieces together this gives

\begin{align}\label{eqn:desaiCh3:307}
x_H =
e^{iH t/\hbar} x e^{-iH t/\hbar}  &= 
x + \frac{tp}{m} - \frac{n t^2 V}{ 2 m x} 
+ \frac{n(n-1)t^3 V}{ 12 m^2 x^3 } (i \hbar n + 2 p x) + \cdots
\end{align}

If there is a closed form for this it isn't obvious to me.  Would a fixed lower degree potential function shed any more light on this.  How about the Harmonic oscillator Hamiltonian

\begin{align}\label{eqn:desaiCh3:308}
H = \frac{p^2}{2m} + \frac{m \omega^2 }{2} x^2
\end{align}

... this one works out nicely since there's an even-odd alternation.

Get

\begin{align}\label{eqn:desaiCh3:309}
x_H = x \cos (\omega^2 t^2 /2) + 
\frac{ p t }{m} \frac{\sin( \omega^2 t^2/2)}{ \omega^2 t^2/2 }
\end{align}

I'd not expect such a tidy result for an arbitrary $V(x) = \lambda x^n$ potential.

\subsection{Problem 4. Feynman-Hellman relation.}

For continuously parametrized eigenstate, eigenvalue and Hamiltonian $\ket{\psi(\lambda)}$, $E(\lambda)$ and $H(\lambda)$ respectively, we can relate the derivatives

\begin{align*}
\PD{\lambda}{} ( H \ket{\psi} ) &= \PD{\lambda}{} ( E \ket{\psi} ) \\
\implies \\
\PD{\lambda}{H} \ket{\psi} +H \PD{\lambda}{\ket{\psi}} &= \PD{\lambda}{E} \ket{\psi} + E \PD{\lambda}{\ket{\psi} } 
\end{align*}

Left multiplication by $\bra{\psi}$ gives

\begin{align*}
\bra{\psi}\PD{\lambda}{H} \ket{\psi} +\bra{\psi}H \PD{\lambda}{\ket{\psi}} &= \bra{\psi}\PD{\lambda}{E} \ket{\psi} +  E \bra{\psi}\PD{\lambda}{\ket{\psi} } \\
\implies \\
\bra{\psi}\PD{\lambda}{H} \ket{\psi} +(\bra{\psi}E) \PD{\lambda}{\ket{\psi}} &= \bra{\psi}\PD{\lambda}{E} \ket{\psi} +  E \bra{\psi}\PD{\lambda}{\ket{\psi} } \\
\implies \\
\bra{\psi}\PD{\lambda}{H} \ket{\psi} &= \PD{\lambda}{E} \braket{\psi}{\psi},
\end{align*}

which provides the desired identity
\begin{align}\label{eqn:desaiCh3:4}
\PD{\lambda}{E} 
= \bra{\psi(\lambda)}\PD{\lambda}{H} \ket{\psi(\lambda)}
%= \expectation{ \psi(\lambda) \PD{\lambda}{H} \psi(\lambda) }
\end{align}

\subsection{Problem 5. }
\subsection{Problem 6. }
\subsection{Problem 7. }
\subsection{Problem 8. }
\subsection{Problem 9. }
\subsection{Problem 10. }
\subsection{Problem 11. commutator of angular momentum with Hamiltonian.}

Show that $\antisymmetric{\BL}{H} = 0$, where $H = \Bp^2/2m + V(r)$.

This follows by considering $\antisymmetric{\BL}{\Bp^2}$, and $\antisymmetric{\BL}{V(r)}$.  Let

\begin{align}\label{eqn:desaiCh3:1100}
L_{jk} = x_j p_k - x_k p_j,
\end{align}

so that 

\begin{align}\label{eqn:desaiCh3:1101}
\BL = \Be_i \epsilon_{ijk} L_{jk}.
\end{align}

We now need to consider the commutators of the operators $L_{jk}$ with $\Bp^2$ and $V(r)$.

Let's start with $p^2$.  In particular

\begin{align*}
\Bp^2 x_m p_n
&=
p_k p_k x_m p_n \\
&=
p_k (p_k x_m) p_n \\
&=
p_k (-i\hbar \delta_{km} + x_m p_k) p_n \\
&=
-i\hbar p_m p_n + (p_k x_m) p_k p_n \\
&=
-i\hbar p_m p_n + (-i \hbar \delta_{km} + x_m p_k ) p_k p_n \\
&=
-2 i\hbar p_m p_n + x_m p_n \Bp^2.
\end{align*}

So our commutator with $\Bp^2$ is

\begin{align*}
\antisymmetric{L_{jk}}{\Bp^2}
&=
(x_j p_k - x_j p_k) \Bp^2 
-( -2 i\hbar p_j p_k + x_j p_k \Bp^2 +2 i\hbar p_k p_j - x_k p_j \Bp^2 ).
\end{align*}

Since $p_j p_k = p_k p_j$, all terms cancel out, and the problem is reduced to showing that 

\begin{align*}
\antisymmetric{\BL}{H} &= \antisymmetric{\BL}{V(r)} = 0.
\end{align*}

Now assume that $V(r)$ has a series representation

\begin{align*}
V(r) &= \sum_j a_j r^j = \sum_j a_j (x_k x_k)^{j/2}
\end{align*}

We'd like to consider the action of $x_m p_n$ on this function

\begin{align*}
x_m p_n V(r) \Psi
&= -i \hbar x_m \sum_j a_j \partial_n (x_k x_k)^{j/2} \Psi \\
&= -i \hbar x_m \sum_j a_j (j x_n (x_k x_k)^{j/2-1} + r^j \partial_n \Psi) \\
&= -\frac{i \hbar x_m x_n}{r^2} \sum_j a_j j r^j + 
x_m V(r) p_n \Psi
\end{align*}

\begin{align*}
L_{mn} V(r) 
&=
(x_m p_n - x_n p_m) V(r) \\
&= 
-\frac{i \hbar x_m x_n}{r^2} \sum_j a_j j r^j
+\frac{i \hbar x_n x_m}{r^2} \sum_j a_j j r^j 
+ 
V(r) (x_m p_n - x_n p_m )
\\
&= 
V(r) L_{mn}
\end{align*}

Thus $\antisymmetric{L_{mn}}{V(r)} = 0$ as expected, implying $\antisymmetric{\BL}{H} = 0$.

\EndArticle

%
% Copyright � 2015 Peeter Joot.  All Rights Reserved.
% Licenced as described in the file LICENSE under the root directory of this GIT repository.
%
\documentclass[]{eliblog}

\usepackage{amsmath}
\usepackage{mathpazo}

%
% shorthand for bold symbols, convenient for vectors and matrices
%
\newcommand{\Ba}[0]{\mathbf{a}}
\newcommand{\Bb}[0]{\mathbf{b}}
\newcommand{\Bc}[0]{\mathbf{c}}
\newcommand{\Bd}[0]{\mathbf{d}}
\newcommand{\Be}[0]{\mathbf{e}}
\newcommand{\Bf}[0]{\mathbf{f}}
\newcommand{\Bg}[0]{\mathbf{g}}
\newcommand{\Bh}[0]{\mathbf{h}}
\newcommand{\Bi}[0]{\mathbf{i}}
\newcommand{\Bj}[0]{\mathbf{j}}
\newcommand{\Bk}[0]{\mathbf{k}}
\newcommand{\Bl}[0]{\mathbf{l}}
\newcommand{\Bm}[0]{\mathbf{m}}
\newcommand{\Bn}[0]{\mathbf{n}}
\newcommand{\Bo}[0]{\mathbf{o}}
\newcommand{\Bp}[0]{\mathbf{p}}
\newcommand{\Bq}[0]{\mathbf{q}}
\newcommand{\Br}[0]{\mathbf{r}}
\newcommand{\Bs}[0]{\mathbf{s}}
\newcommand{\Bt}[0]{\mathbf{t}}
\newcommand{\Bu}[0]{\mathbf{u}}
\newcommand{\Bv}[0]{\mathbf{v}}
\newcommand{\Bw}[0]{\mathbf{w}}
\newcommand{\Bx}[0]{\mathbf{x}}
\newcommand{\By}[0]{\mathbf{y}}
\newcommand{\Bz}[0]{\mathbf{z}}
\newcommand{\BA}[0]{\mathbf{A}}
\newcommand{\BB}[0]{\mathbf{B}}
\newcommand{\BC}[0]{\mathbf{C}}
\newcommand{\BD}[0]{\mathbf{D}}
\newcommand{\BE}[0]{\mathbf{E}}
\newcommand{\BF}[0]{\mathbf{F}}
\newcommand{\BG}[0]{\mathbf{G}}
\newcommand{\BH}[0]{\mathbf{H}}
\newcommand{\BI}[0]{\mathbf{I}}
\newcommand{\BJ}[0]{\mathbf{J}}
\newcommand{\BK}[0]{\mathbf{K}}
\newcommand{\BL}[0]{\mathbf{L}}
\newcommand{\BM}[0]{\mathbf{M}}
\newcommand{\BN}[0]{\mathbf{N}}
\newcommand{\BO}[0]{\mathbf{O}}
\newcommand{\BP}[0]{\mathbf{P}}
\newcommand{\BQ}[0]{\mathbf{Q}}
\newcommand{\BR}[0]{\mathbf{R}}
\newcommand{\BS}[0]{\mathbf{S}}
\newcommand{\BT}[0]{\mathbf{T}}
\newcommand{\BU}[0]{\mathbf{U}}
\newcommand{\BV}[0]{\mathbf{V}}
\newcommand{\BW}[0]{\mathbf{W}}
\newcommand{\BX}[0]{\mathbf{X}}
\newcommand{\BY}[0]{\mathbf{Y}}
\newcommand{\BZ}[0]{\mathbf{Z}}

\newcommand{\Bzero}[0]{\mathbf{0}}
\newcommand{\Btheta}[0]{\boldsymbol{\theta}}
\newcommand{\Btau}[0]{\boldsymbol{\tau}}
\newcommand{\Bomega}[0]{\boldsymbol{\omega}}

%
% shorthand for unit vectors
%
\newcommand{\acap}[0]{\hat{\Ba}}
\newcommand{\bcap}[0]{\hat{\Bb}}
\newcommand{\ccap}[0]{\hat{\Bc}}
\newcommand{\dcap}[0]{\hat{\Bd}}
\newcommand{\ecap}[0]{\hat{\Be}}
\newcommand{\fcap}[0]{\hat{\Bf}}
\newcommand{\gcap}[0]{\hat{\Bg}}
\newcommand{\hcap}[0]{\hat{\Bh}}
\newcommand{\icap}[0]{\hat{\Bi}}
\newcommand{\jcap}[0]{\hat{\Bj}}
\newcommand{\kcap}[0]{\hat{\Bk}}
\newcommand{\lcap}[0]{\hat{\Bl}}
\newcommand{\mcap}[0]{\hat{\Bm}}
\newcommand{\ncap}[0]{\hat{\Bn}}
\newcommand{\ocap}[0]{\hat{\Bo}}
\newcommand{\pcap}[0]{\hat{\Bp}}
\newcommand{\qcap}[0]{\hat{\Bq}}
\newcommand{\rcap}[0]{\hat{\Br}}
\newcommand{\scap}[0]{\hat{\Bs}}
\newcommand{\tcap}[0]{\hat{\Bt}}
\newcommand{\ucap}[0]{\hat{\Bu}}
\newcommand{\vcap}[0]{\hat{\Bv}}
\newcommand{\wcap}[0]{\hat{\Bw}}
\newcommand{\xcap}[0]{\hat{\Bx}}
\newcommand{\ycap}[0]{\hat{\By}}
\newcommand{\zcap}[0]{\hat{\Bz}}
\newcommand{\thetacap}[0]{\hat{\Btheta}}

%
% to write R^n and C^n in a distinguishable fashion.  Perhaps change this
% to the double lined characters upon figuring out how to do so.
%
\newcommand{\C}[1]{$\mathbb{C}^{#1}$}
\newcommand{\R}[1]{$\mathbb{R}^{#1}$}

%
% various generally useful helpers
%

% derivative of #1 wrt. #2:
\newcommand{\D}[2] {\frac {d#2} {d#1}}

\newcommand{\inv}[1]{\frac{1}{#1}}
\newcommand{\cross}[0]{\times}

\newcommand{\abs}[1]{\lvert{#1}\rvert}
\newcommand{\norm}[1]{\lVert{#1}\rVert}
\newcommand{\innerprod}[2]{\langle{#1}, {#2}\rangle}
\newcommand{\dotprod}[2]{{#1} \cdot {#2}}
\newcommand{\bdotprod}[2]{\left({#1} \cdot {#2}\right)}
\newcommand{\crossprod}[2]{{#1} \cross {#2}}
\newcommand{\tripleprod}[3]{\dotprod{\left(\crossprod{#1}{#2}\right)}{#3}}

\DeclareMathOperator{\Proj}{Proj}
\DeclareMathOperator{\Span}{span}
\DeclareMathOperator{\Sgn}{sgn}
\DeclareMathOperator{\Area}{Area}
\DeclareMathOperator{\Volume}{Volume}

%
% A few miscellaneous things specific to this document
%
\newcommand{\crossop}[1]{\crossprod{#1}{}}

% R2 vector.
\newcommand{\VectorTwo}[2]{
\begin{bmatrix}
 {#1} \\
 {#2}
\end{bmatrix}
}

\newcommand{\VectorN}[1]{
\begin{bmatrix}
{#1}_1 \\
{#1}_2 \\
\vdots \\
{#1}_N \\
\end{bmatrix}
}

\newcommand{\DETuvij}[4]{
\begin{vmatrix}
 {#1}_{#3} & {#1}_{#4} \\
 {#2}_{#3} & {#2}_{#4}
\end{vmatrix}
}

\newcommand{\DETuvwijk}[6]{
\begin{vmatrix}
 {#1}_{#4} & {#1}_{#5} & {#1}_{#6} \\
 {#2}_{#4} & {#2}_{#5} & {#2}_{#6} \\
 {#3}_{#4} & {#3}_{#5} & {#3}_{#6}
\end{vmatrix}
}

\newcommand{\DETuvwxijkl}[8]{
\begin{vmatrix}
 {#1}_{#5} & {#1}_{#6} & {#1}_{#7} & {#1}_{#8} \\
 {#2}_{#5} & {#2}_{#6} & {#2}_{#7} & {#2}_{#8} \\
 {#3}_{#5} & {#3}_{#6} & {#3}_{#7} & {#3}_{#8} \\
 {#4}_{#5} & {#4}_{#6} & {#4}_{#7} & {#4}_{#8} \\
\end{vmatrix}
}

%\newcommand{\DETuvwxyijklm}[10]{
%\begin{vmatrix}
% {#1}_{#6} & {#1}_{#7} & {#1}_{#8} & {#1}_{#9} & {#1}_{#10} \\
% {#2}_{#6} & {#2}_{#7} & {#2}_{#8} & {#2}_{#9} & {#2}_{#10} \\
% {#3}_{#6} & {#3}_{#7} & {#3}_{#8} & {#3}_{#9} & {#3}_{#10} \\
% {#4}_{#6} & {#4}_{#7} & {#4}_{#8} & {#4}_{#9} & {#4}_{#10} \\
% {#5}_{#6} & {#5}_{#7} & {#5}_{#8} & {#5}_{#9} & {#5}_{#10}
%\end{vmatrix}
%}

% R3 vector.
\newcommand{\VectorThree}[3]{
\begin{bmatrix}
 {#1} \\
 {#2} \\
 {#3}
\end{bmatrix}
}



\author{Peeter Joot}
\email{peeter.joot@gmail.com}

%\documentclass[]{eliblogwidescreen}

\usepackage{amsmath}
\usepackage{mathpazo}

%
% shorthand for bold symbols, convenient for vectors and matrices
%
\newcommand{\Ba}[0]{\mathbf{a}}
\newcommand{\Bb}[0]{\mathbf{b}}
\newcommand{\Bc}[0]{\mathbf{c}}
\newcommand{\Bd}[0]{\mathbf{d}}
\newcommand{\Be}[0]{\mathbf{e}}
\newcommand{\Bf}[0]{\mathbf{f}}
\newcommand{\Bg}[0]{\mathbf{g}}
\newcommand{\Bh}[0]{\mathbf{h}}
\newcommand{\Bi}[0]{\mathbf{i}}
\newcommand{\Bj}[0]{\mathbf{j}}
\newcommand{\Bk}[0]{\mathbf{k}}
\newcommand{\Bl}[0]{\mathbf{l}}
\newcommand{\Bm}[0]{\mathbf{m}}
\newcommand{\Bn}[0]{\mathbf{n}}
\newcommand{\Bo}[0]{\mathbf{o}}
\newcommand{\Bp}[0]{\mathbf{p}}
\newcommand{\Bq}[0]{\mathbf{q}}
\newcommand{\Br}[0]{\mathbf{r}}
\newcommand{\Bs}[0]{\mathbf{s}}
\newcommand{\Bt}[0]{\mathbf{t}}
\newcommand{\Bu}[0]{\mathbf{u}}
\newcommand{\Bv}[0]{\mathbf{v}}
\newcommand{\Bw}[0]{\mathbf{w}}
\newcommand{\Bx}[0]{\mathbf{x}}
\newcommand{\By}[0]{\mathbf{y}}
\newcommand{\Bz}[0]{\mathbf{z}}
\newcommand{\BA}[0]{\mathbf{A}}
\newcommand{\BB}[0]{\mathbf{B}}
\newcommand{\BC}[0]{\mathbf{C}}
\newcommand{\BD}[0]{\mathbf{D}}
\newcommand{\BE}[0]{\mathbf{E}}
\newcommand{\BF}[0]{\mathbf{F}}
\newcommand{\BG}[0]{\mathbf{G}}
\newcommand{\BH}[0]{\mathbf{H}}
\newcommand{\BI}[0]{\mathbf{I}}
\newcommand{\BJ}[0]{\mathbf{J}}
\newcommand{\BK}[0]{\mathbf{K}}
\newcommand{\BL}[0]{\mathbf{L}}
\newcommand{\BM}[0]{\mathbf{M}}
\newcommand{\BN}[0]{\mathbf{N}}
\newcommand{\BO}[0]{\mathbf{O}}
\newcommand{\BP}[0]{\mathbf{P}}
\newcommand{\BQ}[0]{\mathbf{Q}}
\newcommand{\BR}[0]{\mathbf{R}}
\newcommand{\BS}[0]{\mathbf{S}}
\newcommand{\BT}[0]{\mathbf{T}}
\newcommand{\BU}[0]{\mathbf{U}}
\newcommand{\BV}[0]{\mathbf{V}}
\newcommand{\BW}[0]{\mathbf{W}}
\newcommand{\BX}[0]{\mathbf{X}}
\newcommand{\BY}[0]{\mathbf{Y}}
\newcommand{\BZ}[0]{\mathbf{Z}}

\newcommand{\Bzero}[0]{\mathbf{0}}
\newcommand{\Btheta}[0]{\boldsymbol{\theta}}
\newcommand{\Btau}[0]{\boldsymbol{\tau}}
\newcommand{\Bomega}[0]{\boldsymbol{\omega}}

%
% shorthand for unit vectors
%
\newcommand{\acap}[0]{\hat{\Ba}}
\newcommand{\bcap}[0]{\hat{\Bb}}
\newcommand{\ccap}[0]{\hat{\Bc}}
\newcommand{\dcap}[0]{\hat{\Bd}}
\newcommand{\ecap}[0]{\hat{\Be}}
\newcommand{\fcap}[0]{\hat{\Bf}}
\newcommand{\gcap}[0]{\hat{\Bg}}
\newcommand{\hcap}[0]{\hat{\Bh}}
\newcommand{\icap}[0]{\hat{\Bi}}
\newcommand{\jcap}[0]{\hat{\Bj}}
\newcommand{\kcap}[0]{\hat{\Bk}}
\newcommand{\lcap}[0]{\hat{\Bl}}
\newcommand{\mcap}[0]{\hat{\Bm}}
\newcommand{\ncap}[0]{\hat{\Bn}}
\newcommand{\ocap}[0]{\hat{\Bo}}
\newcommand{\pcap}[0]{\hat{\Bp}}
\newcommand{\qcap}[0]{\hat{\Bq}}
\newcommand{\rcap}[0]{\hat{\Br}}
\newcommand{\scap}[0]{\hat{\Bs}}
\newcommand{\tcap}[0]{\hat{\Bt}}
\newcommand{\ucap}[0]{\hat{\Bu}}
\newcommand{\vcap}[0]{\hat{\Bv}}
\newcommand{\wcap}[0]{\hat{\Bw}}
\newcommand{\xcap}[0]{\hat{\Bx}}
\newcommand{\ycap}[0]{\hat{\By}}
\newcommand{\zcap}[0]{\hat{\Bz}}
\newcommand{\thetacap}[0]{\hat{\Btheta}}

%
% to write R^n and C^n in a distinguishable fashion.  Perhaps change this
% to the double lined characters upon figuring out how to do so.
%
\newcommand{\C}[1]{$\mathbb{C}^{#1}$}
\newcommand{\R}[1]{$\mathbb{R}^{#1}$}

%
% various generally useful helpers
%

% derivative of #1 wrt. #2:
\newcommand{\D}[2] {\frac {d#2} {d#1}}

\newcommand{\inv}[1]{\frac{1}{#1}}
\newcommand{\cross}[0]{\times}

\newcommand{\abs}[1]{\lvert{#1}\rvert}
\newcommand{\norm}[1]{\lVert{#1}\rVert}
\newcommand{\innerprod}[2]{\langle{#1}, {#2}\rangle}
\newcommand{\dotprod}[2]{{#1} \cdot {#2}}
\newcommand{\bdotprod}[2]{\left({#1} \cdot {#2}\right)}
\newcommand{\crossprod}[2]{{#1} \cross {#2}}
\newcommand{\tripleprod}[3]{\dotprod{\left(\crossprod{#1}{#2}\right)}{#3}}

\DeclareMathOperator{\Proj}{Proj}
\DeclareMathOperator{\Span}{span}
\DeclareMathOperator{\Sgn}{sgn}
\DeclareMathOperator{\Area}{Area}
\DeclareMathOperator{\Volume}{Volume}

%
% A few miscellaneous things specific to this document
%
\newcommand{\crossop}[1]{\crossprod{#1}{}}

% R2 vector.
\newcommand{\VectorTwo}[2]{
\begin{bmatrix}
 {#1} \\
 {#2}
\end{bmatrix}
}

\newcommand{\VectorN}[1]{
\begin{bmatrix}
{#1}_1 \\
{#1}_2 \\
\vdots \\
{#1}_N \\
\end{bmatrix}
}

\newcommand{\DETuvij}[4]{
\begin{vmatrix}
 {#1}_{#3} & {#1}_{#4} \\
 {#2}_{#3} & {#2}_{#4}
\end{vmatrix}
}

\newcommand{\DETuvwijk}[6]{
\begin{vmatrix}
 {#1}_{#4} & {#1}_{#5} & {#1}_{#6} \\
 {#2}_{#4} & {#2}_{#5} & {#2}_{#6} \\
 {#3}_{#4} & {#3}_{#5} & {#3}_{#6}
\end{vmatrix}
}

\newcommand{\DETuvwxijkl}[8]{
\begin{vmatrix}
 {#1}_{#5} & {#1}_{#6} & {#1}_{#7} & {#1}_{#8} \\
 {#2}_{#5} & {#2}_{#6} & {#2}_{#7} & {#2}_{#8} \\
 {#3}_{#5} & {#3}_{#6} & {#3}_{#7} & {#3}_{#8} \\
 {#4}_{#5} & {#4}_{#6} & {#4}_{#7} & {#4}_{#8} \\
\end{vmatrix}
}

%\newcommand{\DETuvwxyijklm}[10]{
%\begin{vmatrix}
% {#1}_{#6} & {#1}_{#7} & {#1}_{#8} & {#1}_{#9} & {#1}_{#10} \\
% {#2}_{#6} & {#2}_{#7} & {#2}_{#8} & {#2}_{#9} & {#2}_{#10} \\
% {#3}_{#6} & {#3}_{#7} & {#3}_{#8} & {#3}_{#9} & {#3}_{#10} \\
% {#4}_{#6} & {#4}_{#7} & {#4}_{#8} & {#4}_{#9} & {#4}_{#10} \\
% {#5}_{#6} & {#5}_{#7} & {#5}_{#8} & {#5}_{#9} & {#5}_{#10}
%\end{vmatrix}
%}

% R3 vector.
\newcommand{\VectorThree}[3]{
\begin{bmatrix}
 {#1} \\
 {#2} \\
 {#3}
\end{bmatrix}
}



\author{Peeter Joot}
\email{peeter.joot@gmail.com}


\chapter{Notes and problems for Desai chapter IV.}
\label{chap:desaiCh4}
%\useCCL
\blogpage{http://sites.google.com/site/peeterjoot/math2010/desaiCh4.pdf}
\date{Oct 10, 2010}
\revisionInfo{desaiCh4.tex}

\beginArtWithToc
%\beginArtNoToc

\section{Notes.}

Chapter IV notes and problems for \cite{desai2009quantum}.

There's a lot of magic related to the spherical Harmonics in this chapter, with identities pulled out of the Author's butt.  It would be nice to work through that, but need a better reference to work from (or skip ahead to chapter 26 where some of this is apparently derived).

Other stuff pending background derivation and verification are

\begin{itemize}
\item Antisymmetric tensor summation identity.

\begin{align}\label{eqn:desaiCh4:1}
\sum_i \epsilon_{ijk} \epsilon_{iab} = \delta_{ja} \delta_{kb} - \delta_{jb}\delta_{ka}
\end{align}

This is obviously the coordinate equivalent of the dot product of two bivectors

\begin{align}\label{eqn:desaiCh4:2}
(\Be_j \wedge \Be_k) \cdot (\Be_a \wedge \Be_b) &=
( (\Be_j \wedge \Be_k) \cdot \Be_a ) \cdot \Be_b) =
\delta_{ka}\delta_{jb} - \delta_{ja}\delta_{kb}
\end{align}

We can prove \ref{eqn:desaiCh4:1} by expanding the LHS of \ref{eqn:desaiCh4:2} in coordinates

\begin{align*}
(\Be_j \wedge \Be_k) \cdot (\Be_a \wedge \Be_b)
&= \sum_{ie} \gpgradezero{
\epsilon_{ijk} \Be_j \Be_k \epsilon_{eab} \Be_a \Be_b
} \\
&=
\sum_{ie}
\epsilon_{ijk} \epsilon_{eab}
\gpgradezero{
(\Be_i \Be_i) \Be_j \Be_k (\Be_e \Be_e) \Be_a \Be_b
} \\
&=
\sum_{ie}
\epsilon_{ijk} \epsilon_{eab}
\gpgradezero{
\Be_i \Be_e I^2
} \\
&=
-\sum_{ie} \epsilon_{ijk} \epsilon_{eab} \delta_{ie} \\
&=
-
\sum_i
\epsilon_{ijk} \epsilon_{iab}
\qquad\square
\end{align*}

\item Question on raising and lowering arguments.

How equation (4.240) was arrived at is not clear.  In (4.239) he writes

\begin{align*}
\int_0^{2\pi} \int_0^{\pi} d\theta d\phi
(L_{-} Y_{lm})^\dagger
L_{-} Y_{lm} \sin\theta
\end{align*}

Shouldn't that Hermitian conjugation be just complex conjugation? if so one would have

\begin{align*}
\int_0^{2\pi} \int_0^{\pi} d\theta d\phi
L_{-}^\conj Y_{lm}^\conj
L_{-} Y_{lm} \sin\theta
\end{align*}

How does he end up with the $L_{-}$ and the $Y_{lm}^\conj$ interchanged.  What justifies this commutation?

A much clearer discussion of this can be found in \href{http://quantummechanics.ucsd.edu/ph130a/130_notes/node217.html}{The operators $L_{\pm}$}, where Dirac notation is used for the normalization discussion.

\item Another question on raising and lowering arguments.

The reasoning leading to (4.238) isn't clear to me.  I fail to see how the $L_{-}$ commutation with $\BL^2$ implies this?

\end{itemize}



\section{Problems}

\subsection{Problem 1.}
\subsubsection{Statement.}

Write down the free particle Sch equation for two dimensions in (i) Cartesian and (ii) polar coordinates.  Obtain the corresponding wavefunction.

\subsubsection{Solution.}

TODO.
\subsection{Problem 2.}
\subsubsection{Statement.}
\subsubsection{Solution.}

TODO.
\subsection{Problem 3.}
\subsubsection{Statement.}

Obtain the commutation relations $\antisymmetric{L_i}{L_j}$ by calculating the vector $\BL \cross \BL$ using the definition $\BL = \Br \cross \Bp$ directly instead of introducing a differential operator.

\subsubsection{Solution.}

Expressing the product $\BL \cross \BL$ in determinant form sheds some light on this question.  That is

\begin{align}\label{eqn:desaiCh4:300}
\begin{vmatrix}
 \Be_1 & \Be_2 & \Be_3 \\
 L_1 & L_2 & L_3 \\
 L_1 & L_2 & L_3
\end{vmatrix}
&=
 \Be_1 \antisymmetric{L_2}{L_3}
 +\Be_2 \antisymmetric{L_3}{L_1}
 +\Be_3 \antisymmetric{L_1}{L_2}
= \Be_i \epsilon_{ijk} \antisymmetric{L_j}{L_k}
\end{align}

We see that evaluating this cross product in turn requires evaluation of the set of commutators.  We can do that with the canonical commutator relationships directly using $L_i = \epsilon_{ijk} r_j p_k$ like so

\begin{align*}
\antisymmetric{L_i}{L_j}
&=
%\sum_{mnab}
\epsilon_{imn} r_m p_n \epsilon_{jab} r_a p_b
- \epsilon_{jab} r_a p_b \epsilon_{imn} r_m p_n \\
&=
%\sum_{mnab}
\epsilon_{imn} \epsilon_{jab} r_m (p_n r_a) p_b
- \epsilon_{jab} \epsilon_{imn} r_a (p_b r_m) p_n \\
&=
%\sum_{mnab}
\epsilon_{imn} \epsilon_{jab} r_m (r_a p_n -i \hbar \delta_{an}) p_b
- \epsilon_{jab} \epsilon_{imn} r_a (r_m p_b - i \hbar \delta{mb}) p_n \\
&=
%\sum_{mnab}
\epsilon_{imn} \epsilon_{jab} (r_m r_a p_n p_b - r_a r_m p_b p_n )
- i \hbar ( \epsilon_{imn} \epsilon_{jnb} r_m p_b - \epsilon_{jam} \epsilon_{imn} r_a p_n ).
\end{align*}

The first two terms cancel, and we can employ (4.179) to eliminate the antisymmetric tensors from the last two terms

\begin{align*}
\antisymmetric{L_i}{L_j}
&=
i \hbar ( \epsilon_{nim} \epsilon_{njb} r_m p_b - \epsilon_{mja} \epsilon_{min} r_a p_n ) \\
&=
i \hbar ( (\delta_{ij} \delta_{mb} -\delta_{ib} \delta_{mj}) r_m p_b - (\delta_{ji} \delta_{an} -\delta_{jn} \delta_{ai}) r_a p_n ) \\
&=
i \hbar (\delta_{ij} \delta_{mb} r_m p_b - \delta_{ji} \delta_{an} r_a p_n - \delta_{ib} \delta_{mj} r_m p_b + \delta_{jn} \delta_{ai} r_a p_n ) \\
&=
i \hbar (
\delta_{ij} r_m p_m
- \delta_{ji} r_a p_a
- r_j p_i
+ r_i p_j ) \\
\end{align*}

For $k \ne i,j$, this is $i\hbar (\Br \cross \Bp)_k$, so we can write

\begin{align}\label{eqn:desaiCh4:301}
\BL \cross \BL &= i\hbar \Be_k \epsilon_{kij} ( r_i p_j - r_j p_i ) = i\hbar \BL = i\hbar \Be_k L_k = i\hbar \BL.
\end{align}

In \cite{liboff2003iqm}, the commutator relationships are summarized this way, instead of using the antisymmetric tensor (4.224)

\begin{align}\label{eqn:desaiCh4:302}
\antisymmetric{L_i}{L_j} &= i \hbar \epsilon_{ijk} L_k
\end{align}

as here in Desai.  Both say the same thing.

\subsection{Problem 4.}
\subsubsection{Statement.}
\subsubsection{Solution.}

TODO.
\subsection{Problem 5.}
\subsubsection{Statement.}

A free particle is moving along a path of radius $R$.  Express the Hamiltonian in terms of the derivatives involving the polar angle of the particle and write down the Sch equation.  Determine the wavfunction and the energy eigenvalues of the particle.

\subsubsection{Solution.}

In classical mechanics our Lagrangian for this system is

\begin{align}\label{eqn:desaiCh4:500}
\LL = \inv{2} m R^2 \thetadot^2,
\end{align}

with the canonical momentum
\begin{align}\label{eqn:desaiCh4:501}
p_\theta = \PD{\thetadot}{\LL} = m R^2 \thetadot.
\end{align}

Thus the classical Hamiltonian is

\begin{align}\label{eqn:desaiCh4:502}
H = \inv{2m R^2} {p_\theta}^2.
\end{align}

By analogy the QM Hamiltonian operator will therefore be
\begin{align}\label{eqn:desaiCh4:503}
H = -\frac{\hbar^2}{2m R^2} \partial_{\theta\theta}.
\end{align}

For $\Psi = \Theta(\theta) T(t)$, separation of variables gives us

\begin{align}\label{eqn:desaiCh4:n}
-\frac{\hbar^2}{2m R^2} \frac{\Theta''}{\Theta} = i \hbar \frac{T'}{T} = E,
\end{align}

from which we have
\begin{align}\label{eqn:desaiCh4:504}
T &\propto e^{-i E t/\hbar} \\
\Theta &\propto e^{ \pm i \sqrt{2m E} R \theta/\hbar }.
\end{align}

Requiring single valued $\Theta$, equal at any multiples of $2\pi$, we have

\begin{align*}
e^{ \pm i \sqrt{2m E} R (\theta + 2\pi)/\hbar } = e^{ \pm i \sqrt{2m E} R \theta/\hbar },
\end{align*}

or
\begin{align*}
\pm \sqrt{2m E} \frac{R}{\hbar} 2\pi = 2 \pi n,
\end{align*}

Suffixing the energy values with this index we have

\begin{align}\label{eqn:desaiCh4:505}
E_n = \frac{n^2 \hbar^2}{2 m R^2}.
\end{align}

Allowing both positive and negative integer values for $n$ we have

\begin{align}\label{eqn:desaiCh4:506}
\Psi = \inv{\sqrt{2\pi}} e^{i n \theta} e^{-i E_n t/\hbar},
\end{align}

where the normalization was a result of the use of a $[0,2\pi]$ inner product over the angles

\begin{align}\label{eqn:desaiCh4:507}
\braket{\psi}{\phi} \equiv \int_0^{2\pi} \psi^\conj(\theta) \phi(\theta) d\theta.
\end{align}

\subsection{Problem 6.}
\subsubsection{Statement.}

Determine $\antisymmetric{L_i}{r}$ and $\antisymmetric{L_i}{\Br}$.

\subsubsection{Solution.}

Since $L_i$ contain only $\theta$ and $\phi$ partials, $\antisymmetric{L_i}{r} = 0$.  For the position vector, however, we have an angular dependence, and are left to evaluate $\antisymmetric{L_i}{\Br} = r \antisymmetric{L_i}{\rcap}$.  We'll need the partials for $\rcap$.  We have

\begin{align}\label{eqn:desaiCh4:600}
\rcap &= \Be_3 e^{I \phicap \theta} \\
\phicap &= \Be_2 e^{\Be_1 \Be_2 \phi} \\
I &= \Be_1 \Be_2 \Be_3
\end{align}

Evaluating the partials we have
\begin{align*}
\partial_\theta \rcap = \rcap I \phicap
\end{align*}

With
\begin{align}\label{eqn:desaiCh4:602}
\thetacap &= \tilde{R} \Be_1 R \\
\phicap &= \tilde{R} \Be_2 R \\
\rcap &= \tilde{R} \Be_3 R
\end{align}

where $\tilde{R} R = 1$, and $\thetacap \phicap \rcap = \Be_1 \Be_2 \Be_3$, we have

\begin{align}\label{eqn:desaiCh4:601}
\partial_\theta \rcap &= \tilde{R} \Be_3 \Be_1 \Be_2 \Be_3 \Be_2 R = \tilde{R} \Be_1 R = \thetacap
\end{align}

For the $\phi$ partial we have
\begin{align*}
\partial_\phi \rcap
&= \Be_3 \sin\theta I \phicap \Be_1 \Be_2 \\
%&= \sin\theta \Be_1 \Be_2 \Be_1 \Be_2 \phicap
&= \sin\theta \phicap
\end{align*}

We are now prepared to evaluate the commutators.  Starting with the easiest we have

\begin{align*}
\antisymmetric{L_z}{\rcap} \Psi
&=
-i \hbar (\partial_\phi \rcap \Psi - \rcap \partial_\phi \Psi ) \\
&=
-i \hbar (\partial_\phi \rcap) \Psi  \\
\end{align*}

So we have
\begin{align}\label{eqn:desaiCh4:610}
\antisymmetric{L_z}{\rcap}
&=
-i \hbar \sin\theta \phicap
\end{align}

Observe that by virtue of chain rule, only the action of the partials on $\rcap$ itself contributes, and all the partials applied to $\Psi$ cancel out due to the commutator differences.  That simplifies the remaining commutator evaluations.  For reference the polar form of $L_x$, and $L_y$ are

\begin{align}\label{eqn:desaiCh4:611}
L_x &= -i \hbar (-S_\phi \partial_\theta - C_\phi \cot\theta \partial_\phi) \\
L_y &= -i \hbar (C_\phi \partial_\theta - S_\phi \cot\theta \partial_\phi),
\end{align}

where the sines and cosines are written with $S$, and $C$ respectively for short.

We therefore have
\begin{align*}
\antisymmetric{L_x}{\rcap}
&= -i \hbar (-S_\phi (\partial_\theta \rcap) - C_\phi \cot\theta (\partial_\phi \rcap) ) \\
&= -i \hbar (-S_\phi \thetacap - C_\phi \cot\theta S_\theta \phicap ) \\
&= -i \hbar (-S_\phi \thetacap - C_\phi C_\theta \phicap ) \\
\end{align*}

and
\begin{align*}
\antisymmetric{L_y}{\rcap}
&= -i \hbar (C_\phi (\partial_\theta \rcap) - S_\phi \cot\theta (\partial_\phi \rcap)) \\
&= -i \hbar (C_\phi \thetacap - S_\phi C_\theta \phicap ).
\end{align*}

Adding back in the factor of $r$, and summarizing we have

\begin{align}\label{eqn:desaiCh4:620}
\antisymmetric{L_i}{r} &= 0 \\
\antisymmetric{L_x}{\Br} &= -i \hbar r (-\sin\phi \thetacap - \cos\phi \cos\theta \phicap ) \\
\antisymmetric{L_y}{\Br} &= -i \hbar r (\cos\phi \thetacap - \sin\phi \cos\theta \phicap ) \\
\antisymmetric{L_z}{\Br} &= -i \hbar r \sin\theta \phicap
\end{align}

\subsection{Problem 7.}
\subsubsection{Statement.}

Show that

\begin{align}\label{eqn:desaiCh4:700}
e^{-i\pi L_x /\hbar } \ket{l,m} = \ket{l,m-1}
\end{align}

\subsubsection{Solution.}

TODO.

\EndArticle

%
% Copyright � 2015 Peeter Joot.  All Rights Reserved.
% Licenced as described in the file LICENSE under the root directory of this GIT repository.
%
\documentclass[]{eliblog}

\usepackage{amsmath}
\usepackage{mathpazo}

%
% shorthand for bold symbols, convenient for vectors and matrices
%
\newcommand{\Ba}[0]{\mathbf{a}}
\newcommand{\Bb}[0]{\mathbf{b}}
\newcommand{\Bc}[0]{\mathbf{c}}
\newcommand{\Bd}[0]{\mathbf{d}}
\newcommand{\Be}[0]{\mathbf{e}}
\newcommand{\Bf}[0]{\mathbf{f}}
\newcommand{\Bg}[0]{\mathbf{g}}
\newcommand{\Bh}[0]{\mathbf{h}}
\newcommand{\Bi}[0]{\mathbf{i}}
\newcommand{\Bj}[0]{\mathbf{j}}
\newcommand{\Bk}[0]{\mathbf{k}}
\newcommand{\Bl}[0]{\mathbf{l}}
\newcommand{\Bm}[0]{\mathbf{m}}
\newcommand{\Bn}[0]{\mathbf{n}}
\newcommand{\Bo}[0]{\mathbf{o}}
\newcommand{\Bp}[0]{\mathbf{p}}
\newcommand{\Bq}[0]{\mathbf{q}}
\newcommand{\Br}[0]{\mathbf{r}}
\newcommand{\Bs}[0]{\mathbf{s}}
\newcommand{\Bt}[0]{\mathbf{t}}
\newcommand{\Bu}[0]{\mathbf{u}}
\newcommand{\Bv}[0]{\mathbf{v}}
\newcommand{\Bw}[0]{\mathbf{w}}
\newcommand{\Bx}[0]{\mathbf{x}}
\newcommand{\By}[0]{\mathbf{y}}
\newcommand{\Bz}[0]{\mathbf{z}}
\newcommand{\BA}[0]{\mathbf{A}}
\newcommand{\BB}[0]{\mathbf{B}}
\newcommand{\BC}[0]{\mathbf{C}}
\newcommand{\BD}[0]{\mathbf{D}}
\newcommand{\BE}[0]{\mathbf{E}}
\newcommand{\BF}[0]{\mathbf{F}}
\newcommand{\BG}[0]{\mathbf{G}}
\newcommand{\BH}[0]{\mathbf{H}}
\newcommand{\BI}[0]{\mathbf{I}}
\newcommand{\BJ}[0]{\mathbf{J}}
\newcommand{\BK}[0]{\mathbf{K}}
\newcommand{\BL}[0]{\mathbf{L}}
\newcommand{\BM}[0]{\mathbf{M}}
\newcommand{\BN}[0]{\mathbf{N}}
\newcommand{\BO}[0]{\mathbf{O}}
\newcommand{\BP}[0]{\mathbf{P}}
\newcommand{\BQ}[0]{\mathbf{Q}}
\newcommand{\BR}[0]{\mathbf{R}}
\newcommand{\BS}[0]{\mathbf{S}}
\newcommand{\BT}[0]{\mathbf{T}}
\newcommand{\BU}[0]{\mathbf{U}}
\newcommand{\BV}[0]{\mathbf{V}}
\newcommand{\BW}[0]{\mathbf{W}}
\newcommand{\BX}[0]{\mathbf{X}}
\newcommand{\BY}[0]{\mathbf{Y}}
\newcommand{\BZ}[0]{\mathbf{Z}}

\newcommand{\Bzero}[0]{\mathbf{0}}
\newcommand{\Btheta}[0]{\boldsymbol{\theta}}
\newcommand{\Btau}[0]{\boldsymbol{\tau}}
\newcommand{\Bomega}[0]{\boldsymbol{\omega}}

%
% shorthand for unit vectors
%
\newcommand{\acap}[0]{\hat{\Ba}}
\newcommand{\bcap}[0]{\hat{\Bb}}
\newcommand{\ccap}[0]{\hat{\Bc}}
\newcommand{\dcap}[0]{\hat{\Bd}}
\newcommand{\ecap}[0]{\hat{\Be}}
\newcommand{\fcap}[0]{\hat{\Bf}}
\newcommand{\gcap}[0]{\hat{\Bg}}
\newcommand{\hcap}[0]{\hat{\Bh}}
\newcommand{\icap}[0]{\hat{\Bi}}
\newcommand{\jcap}[0]{\hat{\Bj}}
\newcommand{\kcap}[0]{\hat{\Bk}}
\newcommand{\lcap}[0]{\hat{\Bl}}
\newcommand{\mcap}[0]{\hat{\Bm}}
\newcommand{\ncap}[0]{\hat{\Bn}}
\newcommand{\ocap}[0]{\hat{\Bo}}
\newcommand{\pcap}[0]{\hat{\Bp}}
\newcommand{\qcap}[0]{\hat{\Bq}}
\newcommand{\rcap}[0]{\hat{\Br}}
\newcommand{\scap}[0]{\hat{\Bs}}
\newcommand{\tcap}[0]{\hat{\Bt}}
\newcommand{\ucap}[0]{\hat{\Bu}}
\newcommand{\vcap}[0]{\hat{\Bv}}
\newcommand{\wcap}[0]{\hat{\Bw}}
\newcommand{\xcap}[0]{\hat{\Bx}}
\newcommand{\ycap}[0]{\hat{\By}}
\newcommand{\zcap}[0]{\hat{\Bz}}
\newcommand{\thetacap}[0]{\hat{\Btheta}}

%
% to write R^n and C^n in a distinguishable fashion.  Perhaps change this
% to the double lined characters upon figuring out how to do so.
%
\newcommand{\C}[1]{$\mathbb{C}^{#1}$}
\newcommand{\R}[1]{$\mathbb{R}^{#1}$}

%
% various generally useful helpers
%

% derivative of #1 wrt. #2:
\newcommand{\D}[2] {\frac {d#2} {d#1}}

\newcommand{\inv}[1]{\frac{1}{#1}}
\newcommand{\cross}[0]{\times}

\newcommand{\abs}[1]{\lvert{#1}\rvert}
\newcommand{\norm}[1]{\lVert{#1}\rVert}
\newcommand{\innerprod}[2]{\langle{#1}, {#2}\rangle}
\newcommand{\dotprod}[2]{{#1} \cdot {#2}}
\newcommand{\bdotprod}[2]{\left({#1} \cdot {#2}\right)}
\newcommand{\crossprod}[2]{{#1} \cross {#2}}
\newcommand{\tripleprod}[3]{\dotprod{\left(\crossprod{#1}{#2}\right)}{#3}}

\DeclareMathOperator{\Proj}{Proj}
\DeclareMathOperator{\Span}{span}
\DeclareMathOperator{\Sgn}{sgn}
\DeclareMathOperator{\Area}{Area}
\DeclareMathOperator{\Volume}{Volume}

%
% A few miscellaneous things specific to this document
%
\newcommand{\crossop}[1]{\crossprod{#1}{}}

% R2 vector.
\newcommand{\VectorTwo}[2]{
\begin{bmatrix}
 {#1} \\
 {#2}
\end{bmatrix}
}

\newcommand{\VectorN}[1]{
\begin{bmatrix}
{#1}_1 \\
{#1}_2 \\
\vdots \\
{#1}_N \\
\end{bmatrix}
}

\newcommand{\DETuvij}[4]{
\begin{vmatrix}
 {#1}_{#3} & {#1}_{#4} \\
 {#2}_{#3} & {#2}_{#4}
\end{vmatrix}
}

\newcommand{\DETuvwijk}[6]{
\begin{vmatrix}
 {#1}_{#4} & {#1}_{#5} & {#1}_{#6} \\
 {#2}_{#4} & {#2}_{#5} & {#2}_{#6} \\
 {#3}_{#4} & {#3}_{#5} & {#3}_{#6}
\end{vmatrix}
}

\newcommand{\DETuvwxijkl}[8]{
\begin{vmatrix}
 {#1}_{#5} & {#1}_{#6} & {#1}_{#7} & {#1}_{#8} \\
 {#2}_{#5} & {#2}_{#6} & {#2}_{#7} & {#2}_{#8} \\
 {#3}_{#5} & {#3}_{#6} & {#3}_{#7} & {#3}_{#8} \\
 {#4}_{#5} & {#4}_{#6} & {#4}_{#7} & {#4}_{#8} \\
\end{vmatrix}
}

%\newcommand{\DETuvwxyijklm}[10]{
%\begin{vmatrix}
% {#1}_{#6} & {#1}_{#7} & {#1}_{#8} & {#1}_{#9} & {#1}_{#10} \\
% {#2}_{#6} & {#2}_{#7} & {#2}_{#8} & {#2}_{#9} & {#2}_{#10} \\
% {#3}_{#6} & {#3}_{#7} & {#3}_{#8} & {#3}_{#9} & {#3}_{#10} \\
% {#4}_{#6} & {#4}_{#7} & {#4}_{#8} & {#4}_{#9} & {#4}_{#10} \\
% {#5}_{#6} & {#5}_{#7} & {#5}_{#8} & {#5}_{#9} & {#5}_{#10}
%\end{vmatrix}
%}

% R3 vector.
\newcommand{\VectorThree}[3]{
\begin{bmatrix}
 {#1} \\
 {#2} \\
 {#3}
\end{bmatrix}
}



\author{Peeter Joot}
\email{peeter.joot@gmail.com}

%\documentclass[]{eliblogwidescreen}

\usepackage{amsmath}
\usepackage{mathpazo}

%
% shorthand for bold symbols, convenient for vectors and matrices
%
\newcommand{\Ba}[0]{\mathbf{a}}
\newcommand{\Bb}[0]{\mathbf{b}}
\newcommand{\Bc}[0]{\mathbf{c}}
\newcommand{\Bd}[0]{\mathbf{d}}
\newcommand{\Be}[0]{\mathbf{e}}
\newcommand{\Bf}[0]{\mathbf{f}}
\newcommand{\Bg}[0]{\mathbf{g}}
\newcommand{\Bh}[0]{\mathbf{h}}
\newcommand{\Bi}[0]{\mathbf{i}}
\newcommand{\Bj}[0]{\mathbf{j}}
\newcommand{\Bk}[0]{\mathbf{k}}
\newcommand{\Bl}[0]{\mathbf{l}}
\newcommand{\Bm}[0]{\mathbf{m}}
\newcommand{\Bn}[0]{\mathbf{n}}
\newcommand{\Bo}[0]{\mathbf{o}}
\newcommand{\Bp}[0]{\mathbf{p}}
\newcommand{\Bq}[0]{\mathbf{q}}
\newcommand{\Br}[0]{\mathbf{r}}
\newcommand{\Bs}[0]{\mathbf{s}}
\newcommand{\Bt}[0]{\mathbf{t}}
\newcommand{\Bu}[0]{\mathbf{u}}
\newcommand{\Bv}[0]{\mathbf{v}}
\newcommand{\Bw}[0]{\mathbf{w}}
\newcommand{\Bx}[0]{\mathbf{x}}
\newcommand{\By}[0]{\mathbf{y}}
\newcommand{\Bz}[0]{\mathbf{z}}
\newcommand{\BA}[0]{\mathbf{A}}
\newcommand{\BB}[0]{\mathbf{B}}
\newcommand{\BC}[0]{\mathbf{C}}
\newcommand{\BD}[0]{\mathbf{D}}
\newcommand{\BE}[0]{\mathbf{E}}
\newcommand{\BF}[0]{\mathbf{F}}
\newcommand{\BG}[0]{\mathbf{G}}
\newcommand{\BH}[0]{\mathbf{H}}
\newcommand{\BI}[0]{\mathbf{I}}
\newcommand{\BJ}[0]{\mathbf{J}}
\newcommand{\BK}[0]{\mathbf{K}}
\newcommand{\BL}[0]{\mathbf{L}}
\newcommand{\BM}[0]{\mathbf{M}}
\newcommand{\BN}[0]{\mathbf{N}}
\newcommand{\BO}[0]{\mathbf{O}}
\newcommand{\BP}[0]{\mathbf{P}}
\newcommand{\BQ}[0]{\mathbf{Q}}
\newcommand{\BR}[0]{\mathbf{R}}
\newcommand{\BS}[0]{\mathbf{S}}
\newcommand{\BT}[0]{\mathbf{T}}
\newcommand{\BU}[0]{\mathbf{U}}
\newcommand{\BV}[0]{\mathbf{V}}
\newcommand{\BW}[0]{\mathbf{W}}
\newcommand{\BX}[0]{\mathbf{X}}
\newcommand{\BY}[0]{\mathbf{Y}}
\newcommand{\BZ}[0]{\mathbf{Z}}

\newcommand{\Bzero}[0]{\mathbf{0}}
\newcommand{\Btheta}[0]{\boldsymbol{\theta}}
\newcommand{\Btau}[0]{\boldsymbol{\tau}}
\newcommand{\Bomega}[0]{\boldsymbol{\omega}}

%
% shorthand for unit vectors
%
\newcommand{\acap}[0]{\hat{\Ba}}
\newcommand{\bcap}[0]{\hat{\Bb}}
\newcommand{\ccap}[0]{\hat{\Bc}}
\newcommand{\dcap}[0]{\hat{\Bd}}
\newcommand{\ecap}[0]{\hat{\Be}}
\newcommand{\fcap}[0]{\hat{\Bf}}
\newcommand{\gcap}[0]{\hat{\Bg}}
\newcommand{\hcap}[0]{\hat{\Bh}}
\newcommand{\icap}[0]{\hat{\Bi}}
\newcommand{\jcap}[0]{\hat{\Bj}}
\newcommand{\kcap}[0]{\hat{\Bk}}
\newcommand{\lcap}[0]{\hat{\Bl}}
\newcommand{\mcap}[0]{\hat{\Bm}}
\newcommand{\ncap}[0]{\hat{\Bn}}
\newcommand{\ocap}[0]{\hat{\Bo}}
\newcommand{\pcap}[0]{\hat{\Bp}}
\newcommand{\qcap}[0]{\hat{\Bq}}
\newcommand{\rcap}[0]{\hat{\Br}}
\newcommand{\scap}[0]{\hat{\Bs}}
\newcommand{\tcap}[0]{\hat{\Bt}}
\newcommand{\ucap}[0]{\hat{\Bu}}
\newcommand{\vcap}[0]{\hat{\Bv}}
\newcommand{\wcap}[0]{\hat{\Bw}}
\newcommand{\xcap}[0]{\hat{\Bx}}
\newcommand{\ycap}[0]{\hat{\By}}
\newcommand{\zcap}[0]{\hat{\Bz}}
\newcommand{\thetacap}[0]{\hat{\Btheta}}

%
% to write R^n and C^n in a distinguishable fashion.  Perhaps change this
% to the double lined characters upon figuring out how to do so.
%
\newcommand{\C}[1]{$\mathbb{C}^{#1}$}
\newcommand{\R}[1]{$\mathbb{R}^{#1}$}

%
% various generally useful helpers
%

% derivative of #1 wrt. #2:
\newcommand{\D}[2] {\frac {d#2} {d#1}}

\newcommand{\inv}[1]{\frac{1}{#1}}
\newcommand{\cross}[0]{\times}

\newcommand{\abs}[1]{\lvert{#1}\rvert}
\newcommand{\norm}[1]{\lVert{#1}\rVert}
\newcommand{\innerprod}[2]{\langle{#1}, {#2}\rangle}
\newcommand{\dotprod}[2]{{#1} \cdot {#2}}
\newcommand{\bdotprod}[2]{\left({#1} \cdot {#2}\right)}
\newcommand{\crossprod}[2]{{#1} \cross {#2}}
\newcommand{\tripleprod}[3]{\dotprod{\left(\crossprod{#1}{#2}\right)}{#3}}

\DeclareMathOperator{\Proj}{Proj}
\DeclareMathOperator{\Span}{span}
\DeclareMathOperator{\Sgn}{sgn}
\DeclareMathOperator{\Area}{Area}
\DeclareMathOperator{\Volume}{Volume}

%
% A few miscellaneous things specific to this document
%
\newcommand{\crossop}[1]{\crossprod{#1}{}}

% R2 vector.
\newcommand{\VectorTwo}[2]{
\begin{bmatrix}
 {#1} \\
 {#2}
\end{bmatrix}
}

\newcommand{\VectorN}[1]{
\begin{bmatrix}
{#1}_1 \\
{#1}_2 \\
\vdots \\
{#1}_N \\
\end{bmatrix}
}

\newcommand{\DETuvij}[4]{
\begin{vmatrix}
 {#1}_{#3} & {#1}_{#4} \\
 {#2}_{#3} & {#2}_{#4}
\end{vmatrix}
}

\newcommand{\DETuvwijk}[6]{
\begin{vmatrix}
 {#1}_{#4} & {#1}_{#5} & {#1}_{#6} \\
 {#2}_{#4} & {#2}_{#5} & {#2}_{#6} \\
 {#3}_{#4} & {#3}_{#5} & {#3}_{#6}
\end{vmatrix}
}

\newcommand{\DETuvwxijkl}[8]{
\begin{vmatrix}
 {#1}_{#5} & {#1}_{#6} & {#1}_{#7} & {#1}_{#8} \\
 {#2}_{#5} & {#2}_{#6} & {#2}_{#7} & {#2}_{#8} \\
 {#3}_{#5} & {#3}_{#6} & {#3}_{#7} & {#3}_{#8} \\
 {#4}_{#5} & {#4}_{#6} & {#4}_{#7} & {#4}_{#8} \\
\end{vmatrix}
}

%\newcommand{\DETuvwxyijklm}[10]{
%\begin{vmatrix}
% {#1}_{#6} & {#1}_{#7} & {#1}_{#8} & {#1}_{#9} & {#1}_{#10} \\
% {#2}_{#6} & {#2}_{#7} & {#2}_{#8} & {#2}_{#9} & {#2}_{#10} \\
% {#3}_{#6} & {#3}_{#7} & {#3}_{#8} & {#3}_{#9} & {#3}_{#10} \\
% {#4}_{#6} & {#4}_{#7} & {#4}_{#8} & {#4}_{#9} & {#4}_{#10} \\
% {#5}_{#6} & {#5}_{#7} & {#5}_{#8} & {#5}_{#9} & {#5}_{#10}
%\end{vmatrix}
%}

% R3 vector.
\newcommand{\VectorThree}[3]{
\begin{bmatrix}
 {#1} \\
 {#2} \\
 {#3}
\end{bmatrix}
}



\author{Peeter Joot}
\email{peeter.joot@gmail.com}


\chapter{Notes and problems for Desai Chapter V.}
\label{chap:desaiCh5}
%\useCCL
\blogpage{http://sites.google.com/site/peeterjoot/math2010/desaiCh5.pdf}
\date{Oct 18, 2010}
\revisionInfo{desaiCh5.tex}

\beginArtWithToc
%\beginArtNoToc

\section{Motivation.}

Chapter V notes for \cite{desai2009quantum}.

\section{Notes}
\section{Problems}

\subsection{Problem 1.}

\subsubsection{Statement.}
Obtain $S_x, S_y, S_z$ for spin 1 in the representation in which $S_z$ and $S^2$ are diagonal.

\subsubsection{Solution.}

For spin 1, we have

\begin{align}\label{eqn:desaiCh5:100}
S^2 = 1 (1+1) \hbar^2 \BOne
\end{align}

and are interested in the states $\ket{1,-1}, \ket{1, 0}, and \ket{1,1}$.  If, like angular momentum, we assume that we have for $m_s = -1,0,1$

\begin{align}\label{eqn:desaiCh5:101}
S_z \ket{1,m_s} = m_s \hbar \ket{1, m_s}
\end{align}

and introduce a column matrix representations for the kets as follows

\begin{align}\label{eqn:desaiCh5:102}
\ket{1,1} &=
\begin{bmatrix}
1 \\
0 \\
0
\end{bmatrix} \\
\ket{1,0} &=
\begin{bmatrix}
0 \\
1 \\
0
\end{bmatrix} \\
\ket{1,-1} &=
\begin{bmatrix}
0 \\
0 \\
-1
\end{bmatrix},
\end{align}

then we have, by inspection
\begin{align}\label{eqn:desaiCh5:103}
S_z &= \hbar
\begin{bmatrix}
1 & 0 & 0 \\
0 & 0 & 0 \\
0 & 0 & -1
\end{bmatrix}.
\end{align}

Note that, like the Pauli matrices, and unlike angular momentum, the spin states $\ket{-1, m_s}, \ket{0, m_s}$ have not been considered.  Do those have any physical interpretation?

That question aside, we can procede as in the text, utilizing the ladder operator commutators

\begin{align}\label{eqn:desaiCh5:104}
S_{\pm} &= S_x \pm i S_y,
\end{align}

to determine the values of $S_x$ and $S_y$ indirectly.  We find

\begin{align}\label{eqn:desaiCh5:105}
\antisymmetric{S_{+}}{S_{-}} &= 2 \hbar S_z \\
\antisymmetric{S_{+}}{S_{z}} &= -\hbar S_{+} \\
\antisymmetric{S_{-}}{S_{z}} &= \hbar S_{-}.
\end{align}

Let
\begin{align}\label{eqn:desaiCh5:106}
S_{+} &=
\begin{bmatrix}
a & b & c \\
d & e & f \\
g & h & i
\end{bmatrix}.
\end{align}

Looking for equality between $\antisymmetric{S_{z}}{S_{+}}/\hbar = S_{+}$, we find
\begin{align}\label{eqn:desaiCh5:107}
\begin{bmatrix}
0 & b & 2 c \\
-d & 0 & f \\
-2g & -h & 0
\end{bmatrix}
&=
\begin{bmatrix}
a & b & c \\
d & e & f \\
g & h & i
\end{bmatrix},
\end{align}

so we must have
\begin{align}\label{eqn:desaiCh5:108}
S_{+} &=
\begin{bmatrix}
0 & b & 0 \\
0 & 0 & f \\
0 & 0 & 0
\end{bmatrix}.
\end{align}

Furthermore, from $\antisymmetric{S_{+}}{S_{-}} = 2 \hbar S_z$, we find
\begin{align}\label{eqn:desaiCh5:109}
\begin{bmatrix}
\Abs{b}^2 & 0 & 0 \\
0 & \Abs{f}^2 - \Abs{b}^2 & 0 \\
0 & 0 & -\Abs{f}^2
\end{bmatrix}
&=
2 \hbar^2
\begin{bmatrix}
1 & 0 & 0 \\
0 & 0 & 0 \\
0 & 0 & -1
\end{bmatrix}.
\end{align}

We must have $\Abs{b}^2 = \Abs{f}^2 = 2 \hbar^2$.  We could probably pick any
$b = \sqrt{2} \hbar e^{i\phi}$, and $f = \sqrt{2} \hbar e^{i\theta}$, but assuming we have no reason for a non-zero phase we try

\begin{align}\label{eqn:desaiCh5:110}
S_{+}
&=
\sqrt{2} \hbar
\begin{bmatrix}
0 & 1 & 0 \\
0 & 0 & 1 \\
0 & 0 & 0
\end{bmatrix}.
\end{align}

Putting all the pieces back together, with $S_x = (S_{+} + S_{-})/2$, and $S_y = (S_{+} - S_{-})/2i$, we finally have
\begin{align}\label{eqn:desaiCh5:111}
S_x &=
\frac{\hbar}{\sqrt{2}}
\begin{bmatrix}
0 & 1 & 0 \\
1 & 0 & 1 \\
0 & 1 & 0
\end{bmatrix} \\
S_y &=
\frac{\hbar}{\sqrt{2} i}
\begin{bmatrix}
0 & 1 & 0 \\
-1 & 0 & 1 \\
0 & -1 & 0
\end{bmatrix} \\
S_z &=
\hbar
\begin{bmatrix}
1 & 0 & 0 \\
0 & 0 & 0 \\
0 & 0 & -1
\end{bmatrix}.
\end{align}

A quick calculation verifies that we have $S_x^2 + S_y^2 + S_z^2 = 2 \hbar \BOne$, as expected.

\subsection{Problem 2.}
\subsubsection{Statement.}

Obtain eigensolution for operator $A = a \sigma_y + b \sigma_z$.  Call the eigenstates $\ket{1}$ and $\ket{2}$, and determine the probabilities that they will correspond to $\sigma_x = +1$.

\subsubsection{Solution.}

The first part is straight forward, and we have
\begin{align*}
A &= a \PauliY + b \PauliZ \\
&=
\begin{bmatrix}
b & -i a \\
ia & -b
\end{bmatrix}.
\end{align*}

Taking $\Abs{A - \lambda I} = 0$ we get

\begin{align}\label{eqn:desaiCh5:202}
\lambda &= \pm \sqrt{a^2 + b^2},
\end{align}

with eigenvectors proportional to
\begin{align}\label{eqn:desaiCh5:203}
\ket{\pm} &=
\begin{bmatrix}
i a \\
b \mp \sqrt{a^2 + b^2}
\end{bmatrix}
\end{align}

The normalization constant is $1/\sqrt{2 (a^2 + b^2) \mp 2 b \sqrt{a^2 + b^2}}$.  Now we can call these $\ket{1}$, and $\ket{2}$ but what does the last part of the question mean?  What's meant by $\sigma_x = +1$?

Asking the prof about this, he says:

``I think it means that the result of a measurement of the x component of spin is $+1$. This corresponds to the eigenvalue of $\sigma_x$ being $+1$. The spin operator $S_x$ has eigenvalue $+\hbar/2$''.

Aside: Question to consider later.  Is is significant that $\bra{1} \sigma_x \ket{1} = \bra{2} \sigma_x \ket{2} = 0$?

So, how do we translate this into a mathematical statement?

First let's recall a couple of details.  Recall that the x spin operator has the matrix representation

\begin{align}\label{eqn:desaiCh5:204}
\sigma_x = \PauliX.
\end{align}

This has eigenvalues $\pm 1$, with eigenstates $(1,\pm 1)/\sqrt{2}$.  At the point when the x component spin is observed to be $+1$, the state of the system was then

\begin{align}\label{eqn:desaiCh5:205}
\ket{x+} =
\inv{\sqrt{2}}
\begin{bmatrix}
1 \\
1
\end{bmatrix}
\end{align}

Let's look at the ways that this state can be formed as linear combinations of our states $\ket{1}$, and $\ket{2}$.  That is

\begin{align}\label{eqn:desaiCh5:206}
\inv{\sqrt{2}}
\begin{bmatrix}
1 \\
1
\end{bmatrix}
&=
\alpha \ket{1}
+ \beta \ket{2},
\end{align}

or

\begin{align}\label{eqn:desaiCh5:207}
\begin{bmatrix}
1 \\
1
\end{bmatrix}
&=
\frac{\alpha}{\sqrt{(a^2 + b^2) - b \sqrt{a^2 + b^2}}}
\begin{bmatrix}
i a \\
b - \sqrt{a^2 + b^2}
\end{bmatrix}
+\frac{\beta}{\sqrt{(a^2 + b^2) + b \sqrt{a^2 + b^2}}}
\begin{bmatrix}
i a \\
b + \sqrt{a^2 + b^2}
\end{bmatrix}
\end{align}

Letting $c = \sqrt{a^2 + b^2}$, this is

\begin{align}\label{eqn:desaiCh5:208}
\begin{bmatrix}
1 \\
1
\end{bmatrix}
&=
\frac{\alpha}{\sqrt{c^2 - b c}}
\begin{bmatrix}
i a \\
b - c
\end{bmatrix}
+\frac{\beta}{\sqrt{c^2 + b c}}
\begin{bmatrix}
i a \\
b + c
\end{bmatrix}.
\end{align}

We can solve the $\alpha$ and $\beta$ with Cramer's rule, yielding
\begin{align*}
\begin{vmatrix}
1 & i a \\
1 & b - c
\end{vmatrix}
&=
\frac{\beta}{\sqrt{c^2 + b c}}
\begin{vmatrix}
i a  & i a \\
b + c & b - c
\end{vmatrix} \\
\begin{vmatrix}
1 & i a \\
1 & b + c
\end{vmatrix}
&=
\frac{\alpha}{\sqrt{c^2 - b c}}
\begin{vmatrix}
i a  & i a \\
b - c & b + c
\end{vmatrix},
\end{align*}

or
\begin{align}\label{eqn:desaiCh5:209}
\alpha &= \frac{(b + c - ia)\sqrt{c^2 - b c}}{2 i a c} \\ %= \frac{(a -i(b + c))\sqrt{1 - b/c}}{2 a} \\
\beta &= \frac{(b - c - ia)\sqrt{c^2 + b c}}{-2 i a c} %= \frac{(a + i(b - c))\sqrt{1 + b/c}}{2 a}.
\end{align}

It is $\Abs{\alpha}^2$ and $\Abs{\beta}^2$ that are probabilities, and after a bit of algebra we find that those are

\begin{align}\label{eqn:desaiCh5:210}
\Abs{\alpha}^2 = \Abs{\beta}^2 = \inv{2},
\end{align}

so if the x spin of the system is measured as $+1$, we have a $50\%$ chance that the measured eigenvalue for the operator $A$ would be $\sqrt{a^2 + b^2}$ (ie: with state $\ket{1}$.

Is that what the question was asking?  I think that I've actually got it backwards.  I think that the question was asking for the probability of finding state $\ket{x+}$ (measuring a spin 1 value for $\sigma_x$) given the state $\ket{1}$ or $\ket{2}$.

So, suppose that we have

\begin{align}\label{eqn:desaiCh5:211}
\mu_{+} \ket{x+} + \nu_{+} \ket{x-} &= \ket{1} \\
\mu_{-} \ket{x+} + \nu_{-} \ket{x-} &= \ket{2},
\end{align}

or (considering both cases simulaneously), 
\begin{align*}
\mu_{\pm}
\begin{bmatrix}
1 \\
1
\end{bmatrix}
+ \nu_{\pm}
\begin{bmatrix}
1 \\
-1
\end{bmatrix}
&= 
\inv{\sqrt{ c^2 \mp b c }} 
\begin{bmatrix}
i a \\
b \mp c
\end{bmatrix} \\
\implies \\
\mu_{\pm}
\begin{vmatrix}
1 & 1 \\
1 & -1
\end{vmatrix}
&= 
\inv{\sqrt{ c^2 \mp b c }} 
\begin{vmatrix}
i a & 1 \\
b \mp c & -1
\end{vmatrix},
\end{align*}

or
\begin{align}\label{eqn:desaiCh5:212}
\mu_{\pm} &= 
\frac{ia + b \mp c}{2 \sqrt{c^2 \mp bc}} .
\end{align}

Unsuprisingly, this mirrors the previous scenerio and we find that we have a probability $\Abs{\mu}^2 = 1/2$ of measuring a spin 1 value for $\sigma_x$ when the state of the operator $A$ has been measured as $\pm \sqrt{a^2 + b^2}$ (ie: in the states $\ket{1}$, or $\ket{2}$ respectively).

No measurement of the operator $A = a \sigma_y + b\sigma_z$ gives a biased prediction of the state of the state $\sigma_x$.  Loosely, this seems to justify calling these operators orthogonal.  This is consistent with the geometrical antisymetric nature of the spin components where we have $\sigma_y \sigma_x = -\sigma_x \sigma_y$, just like two orthogonal vectors under the Clifford product.

\subsection{Problem 3.}
\subsubsection{Statement.}

Obtain the expectation values of $S_x, S_y, S_z$ for the case of a spin $1/2$ particle with the spin pointed in the direction of a vector with azimuthal angle $\beta$ and polar angle $\alpha$.

\subsubsection{Solution.}

TODO.

\subsection{Problem 4.}
\subsubsection{Statement.}
\subsubsection{Solution.}

TODO.

\subsection{Problem 5.}
\subsubsection{Statement.}
\subsubsection{Solution.}

TODO.

\subsection{Problem 6.}
\subsubsection{Statement.}

If a Hamiltonian is given by $\Bsigma \cdot \Bn$ where $\Bn = (\sin\alpha\cos\beta, \sin\alpha\sin\beta, \cos\alpha)$, determine the time evolution operator as a 2 x 2 matrix.  If a state at $t = 0$ is given by 

\begin{align}\label{eqn:desaiCh5:600}
\ket{\phi(0)} = 
\begin{bmatrix}
a \\
b
\end{bmatrix},
\end{align}

then obtain $\ket{\phi(t)}$.

\subsubsection{Solution.}

Before diving into the meat of the problem, observe that a tidy factorization of the Hamiltonian is possible as a composition of rotations.  That is

\begin{align*}
H 
&= \Bsigma \cdot \Bn \\
&= \sin\alpha \sigma_1 ( \cos\beta + \sigma_1 \sigma_2 \sin\beta ) + \cos\alpha \sigma_3 \\
&= \sigma_3 \left(
\cos\alpha 
+ \sin\alpha \sigma_3 \sigma_1 e^{ i \sigma_3 \beta }
\right) \\
&= 
\sigma_3 \exp\left( \alpha i \sigma_2 
\exp\left( \beta i \sigma_3 
\right)
\right)
\end{align*}

So we have for the time evolution operator

\begin{align}\label{eqn:desaiCh5:610}
U(\Delta t) 
&=
\exp( -i \Delta t H /\hbar )
= 
\exp \left(
- \frac{\Delta t}{\hbar} i \sigma_3 \exp\Bigl( \alpha i \sigma_2 
\exp\left( \beta i \sigma_3 
\right)
\Bigr)
\right).
\end{align}

Does this really help?  I guess not, but it is nice and tidy.

Returning to the specifics of the problem, we note that squaring the Hamiltonian produces the identity matrix

\begin{align}\label{eqn:desaiCh5:615}
(\Bsigma \cdot \Bn)^2 &= I \Bn^2 = I.
\end{align}

This allows us to exponentiate $H$ by inspection utilizing

\begin{align}\label{eqn:desaiCh5:620}
e^{i \mu (\Bsigma \cdot \Bn) } = I \cos\mu + i (\Bsigma \cdot \Bn) \sin\mu
\end{align}

Writing $\sin\mu = S_\mu$, and $\cos\mu = C_\mu$, we have
\begin{align}\label{eqn:desaiCh5:625}
\Bsigma \cdot \Bn &=
\begin{bmatrix}
C_\alpha & S_\alpha e^{-i\beta} \\
S_\alpha e^{i\beta} & -C_\alpha
\end{bmatrix},
\end{align}

and thus
\begin{align}\label{eqn:desaiCh5:630}
U(\Delta t) = \exp( -i \Delta t H /\hbar )
=
\begin{bmatrix}
C_{\Delta t/\hbar} -i S_{\Delta t/\hbar} C_\alpha & -i S_{\Delta t/\hbar} S_\alpha e^{-i\beta} \\
-i S_{\Delta t/\hbar} S_\alpha e^{i\beta} & C_{\Delta t/\hbar} + i S_{\Delta t/\hbar} C_\alpha
\end{bmatrix}.
\end{align}

Note that as a sanity check we can calculate that $ U(\Delta t) U(\Delta t)^\dagger = 1$ as expected.

Now for $\Delta t = t$, we have 
\begin{align}\label{eqn:desaiCh5:640}
U(t,0) 
\begin{bmatrix}
a \\
b
\end{bmatrix}
&=
\begin{bmatrix}
a C_{t/\hbar} -a i S_{t/\hbar} C_\alpha  - b i S_{t/\hbar} S_\alpha e^{-i\beta} \\
-a i S_{t/\hbar} S_\alpha e^{i\beta} + b C_{t/\hbar} + b i S_{t/\hbar} C_\alpha
\end{bmatrix}.
\end{align}

It doesn't seem terribly illuminating to multiply this all out, but we can factor the results slightly to tidy it up.  That gives us

\begin{align}\label{eqn:desaiCh5:650}
U(t,0) 
\begin{bmatrix}
a \\
b
\end{bmatrix}
&=
\cos(t/\hbar)
\begin{bmatrix}
a \\
b
\end{bmatrix}
+ 
\sin(t/\hbar) \cos\alpha
\begin{bmatrix}
-a \\
b
\end{bmatrix}
+ i
\sin(t/\hbar) \sin\alpha
\begin{bmatrix}
b e^{-i\beta} \\
-a e^{i \beta}
\end{bmatrix}
\end{align}

\subsection{Problem 7.}
\subsubsection{Statement.}
\subsubsection{Solution.}

TODO.

\subsection{Problem 8.}
\subsubsection{Statement.}
\subsubsection{Solution.}

TODO.

\subsection{Problem 9.}
\subsubsection{Statement.}
\subsubsection{Solution.}

TODO.

\EndArticle

%
% Copyright � 2012 Peeter Joot.  All Rights Reserved.
% Licenced as described in the file LICENSE under the root directory of this GIT repository.
%

%\chapter{Notes and problems for Desai Chapter VI}
\label{chap:desaiCh6}
%\blogpage{http://sites.google.com/site/peeterjoot/math2010/desaiCh6.pdf}
%\date{Oct 18, 2010}

%\section{Motivation}
%
%Chapter VI notes for \citep{desai2009quantum}.
%
%\section{Notes}
%
%\subsection{section 6.5, interaction with orbital angular momentum}

\index{gauge invariance}
In \S 6.5 it is stated that we take
\begin{equation}\label{eqn:desaiCh6:1}
\begin{aligned}
\BA = \inv{2} (\BB \cross \Br)
\end{aligned}
\end{equation}

and that this reproduces the gauge condition \(\spacegrad \cdot \BA = 0\), and the requirement \(\spacegrad \cross \BA = \BB\).

These seem to imply that \(\BB\) is constant, which also accounts for the fact that he writes \(\Bmu \cdot \BL = \BL \cdot \Bmu\).

Consider the gauge condition first, by expanding the divergence of a cross product

\begin{equation}\label{eqn:desaiCh6:27}
\begin{aligned}
\spacegrad \cdot (\BF \cross \BG)
&=
\gpgradezero{ \spacegrad -I \frac{ \BF \BG - \BG \BF }{2} } \\
&=
-\inv{2} \gpgradezero{ I \spacegrad \BF \BG - I \spacegrad \BG \BF } \\
&=
-\inv{2} \gpgradezero{
I \BG(\rspacegrad \BF)  - I \BF (\rspacegrad \BG)
+I (\BG \lspacegrad) \BF - I (\BF \lspacegrad) \BG
} \\
&=
-\inv{2} \gpgradezero{
I \BG(\rspacegrad \wedge \BF)  - I \BF (\rspacegrad \wedge \BG)
+I (\BG \wedge \lspacegrad) \BF - I (\BF \wedge \lspacegrad) \BG
} \\
&=
\inv{2} \gpgradezero{
\BG (\rspacegrad \cross \BF)  - \BF (\rspacegrad \cross \BG)
+(\BG \cross \lspacegrad) \BF - (\BF \cross \lspacegrad) \BG
} \\
&=
\inv{2} \left(
\BG \cdot (\spacegrad \cross \BF)  - \BF \cdot (\spacegrad \cross \BG)
-\BF \cdot (\spacegrad \cross \BG)  + \BG \cdot (\spacegrad \cross \BF )
\right) \\
\end{aligned}
\end{equation}

This gives us

\begin{equation}\label{eqn:desaiCh6:2}
\begin{aligned}
\spacegrad \cdot (\BF \cross \BG)
&=
\BG \cdot (\spacegrad \cross \BF)  - \BF \cdot (\spacegrad \cross \BG)
\end{aligned}
\end{equation}

With \(\BA = (\BB \cross \Br)/2\) we then have

\begin{equation}\label{eqn:desaiCh6:3}
\begin{aligned}
\spacegrad \cdot \BA =
\inv{2} \Br \cdot (\spacegrad \cross \BB)  - \inv{2} \BB \cdot (\spacegrad \cross \Br)
=
\inv{2} \Br \cdot (\spacegrad \cross \BB)
\end{aligned}
\end{equation}

Unless \(\spacegrad \cross \BB\) is always perpendicular to \(\Br\) we can only have a zero divergence when \(\BB\) is constant.

Now, let us look at \(\spacegrad \cross \BA\).  We need another auxiliary identity

\begin{equation}\label{eqn:desaiCh6:47}
\begin{aligned}
\spacegrad \cross (\BF \cross \BG)
&=
-I \spacegrad \wedge (\BF \cross \BG) \\
&=
-\inv{2} \gpgradeone{
I \rspacegrad (\BF \cross \BG)
- I (\BF \cross \BG) \lspacegrad
} \\
&=
\inv{2} \left(
-\rspacegrad \cdot (\BF \wedge \BG)
+ (\BF \wedge \BG) \cdot \lspacegrad
\right) \\
&=
\inv{2} \left(
-(\rspacegrad \cdot \BF) \BG
+(\rspacegrad \cdot \BG) \BF
+ \BF (\BG \cdot \lspacegrad )
- \BG (\BF \cdot \lspacegrad )
\right)
\\
&=
\inv{2} \left(
-(\spacegrad \cdot \BF) \BG
+(\spacegrad \cdot \BG) \BF
+ (\spacegrad \cdot \BG ) \BF
- (\spacegrad \cdot \BF ) \BG
\right)
\end{aligned}
\end{equation}

Here the gradients are all still acting on both \(\BF\) and \(\BG\).  Expanding this out by chain rule we have

\begin{equation}\label{eqn:desaiCh6:67}
\begin{aligned}
2 \spacegrad \cross (\BF \cross \BG)
=
&-(\BF \cdot \spacegrad) \BG
-\BG (\spacegrad \cdot \BF)
+\BF (\spacegrad \cdot \BG)
+(\BG \cdot \spacegrad ) \BF  \\
\quad&+\BF (\spacegrad \cdot \BG )
+ (\BG \cdot \spacegrad ) \BF
- (\BF \cdot \spacegrad ) \BG
- \BG (\spacegrad \cdot \BF )
\end{aligned}
\end{equation}

or
\begin{equation}\label{eqn:desaiCh6:4}
\begin{aligned}
\spacegrad \cross (\BF \cross \BG)
&=
\BF (\spacegrad \cdot \BG) -(\BF \cdot \spacegrad) \BG
+(\BG \cdot \spacegrad ) \BF  -\BG (\spacegrad \cdot \BF)
\end{aligned}
\end{equation}

With \(\BF = \BB/2\), and \(\BG = \Br\), we have

\begin{equation}\label{eqn:desaiCh6:87}
\begin{aligned}
\spacegrad \cross \BA
&=
\inv{2}
\BB (\spacegrad \cdot \Br) -\inv{2}(\BB \cdot \spacegrad) \Br
+\inv{2}(\Br \cdot \spacegrad ) \BB  -\inv{2}\Br (\spacegrad \cdot \BB)
\end{aligned}
\end{equation}

We note that \(\spacegrad \cdot \Br = 3\), and

\begin{equation}\label{eqn:desaiCh6:107}
\begin{aligned}
(\BB \cdot \spacegrad ) \Br
&=
B_k \partial_k x_m \Be_m \\
&=
B_k \delta_{km} \Be_m \\
%&=
%B_m \Be_m
&=
\BB
\end{aligned}
\end{equation}

If \(\BB\) is constant, we have

\begin{equation}\label{eqn:desaiCh6:7}
\begin{aligned}
\spacegrad \cross \BA = \frac{3\BB}{2} - \frac{\BB}{2} = \BB,
\end{aligned}
\end{equation}

as desired.  Now this would all likely be a lot more intuitive if one started with constant \(\BB\) and derived from that what the vector potential was.  That is probably worth also thinking about.


%
% Copyright � 2015 Peeter Joot.  All Rights Reserved.
% Licenced as described in the file LICENSE under the root directory of this GIT repository.
%
\documentclass[]{eliblog}

\usepackage{amsmath}
\usepackage{mathpazo}

%
% shorthand for bold symbols, convenient for vectors and matrices
%
\newcommand{\Ba}[0]{\mathbf{a}}
\newcommand{\Bb}[0]{\mathbf{b}}
\newcommand{\Bc}[0]{\mathbf{c}}
\newcommand{\Bd}[0]{\mathbf{d}}
\newcommand{\Be}[0]{\mathbf{e}}
\newcommand{\Bf}[0]{\mathbf{f}}
\newcommand{\Bg}[0]{\mathbf{g}}
\newcommand{\Bh}[0]{\mathbf{h}}
\newcommand{\Bi}[0]{\mathbf{i}}
\newcommand{\Bj}[0]{\mathbf{j}}
\newcommand{\Bk}[0]{\mathbf{k}}
\newcommand{\Bl}[0]{\mathbf{l}}
\newcommand{\Bm}[0]{\mathbf{m}}
\newcommand{\Bn}[0]{\mathbf{n}}
\newcommand{\Bo}[0]{\mathbf{o}}
\newcommand{\Bp}[0]{\mathbf{p}}
\newcommand{\Bq}[0]{\mathbf{q}}
\newcommand{\Br}[0]{\mathbf{r}}
\newcommand{\Bs}[0]{\mathbf{s}}
\newcommand{\Bt}[0]{\mathbf{t}}
\newcommand{\Bu}[0]{\mathbf{u}}
\newcommand{\Bv}[0]{\mathbf{v}}
\newcommand{\Bw}[0]{\mathbf{w}}
\newcommand{\Bx}[0]{\mathbf{x}}
\newcommand{\By}[0]{\mathbf{y}}
\newcommand{\Bz}[0]{\mathbf{z}}
\newcommand{\BA}[0]{\mathbf{A}}
\newcommand{\BB}[0]{\mathbf{B}}
\newcommand{\BC}[0]{\mathbf{C}}
\newcommand{\BD}[0]{\mathbf{D}}
\newcommand{\BE}[0]{\mathbf{E}}
\newcommand{\BF}[0]{\mathbf{F}}
\newcommand{\BG}[0]{\mathbf{G}}
\newcommand{\BH}[0]{\mathbf{H}}
\newcommand{\BI}[0]{\mathbf{I}}
\newcommand{\BJ}[0]{\mathbf{J}}
\newcommand{\BK}[0]{\mathbf{K}}
\newcommand{\BL}[0]{\mathbf{L}}
\newcommand{\BM}[0]{\mathbf{M}}
\newcommand{\BN}[0]{\mathbf{N}}
\newcommand{\BO}[0]{\mathbf{O}}
\newcommand{\BP}[0]{\mathbf{P}}
\newcommand{\BQ}[0]{\mathbf{Q}}
\newcommand{\BR}[0]{\mathbf{R}}
\newcommand{\BS}[0]{\mathbf{S}}
\newcommand{\BT}[0]{\mathbf{T}}
\newcommand{\BU}[0]{\mathbf{U}}
\newcommand{\BV}[0]{\mathbf{V}}
\newcommand{\BW}[0]{\mathbf{W}}
\newcommand{\BX}[0]{\mathbf{X}}
\newcommand{\BY}[0]{\mathbf{Y}}
\newcommand{\BZ}[0]{\mathbf{Z}}

\newcommand{\Bzero}[0]{\mathbf{0}}
\newcommand{\Btheta}[0]{\boldsymbol{\theta}}
\newcommand{\Btau}[0]{\boldsymbol{\tau}}
\newcommand{\Bomega}[0]{\boldsymbol{\omega}}

%
% shorthand for unit vectors
%
\newcommand{\acap}[0]{\hat{\Ba}}
\newcommand{\bcap}[0]{\hat{\Bb}}
\newcommand{\ccap}[0]{\hat{\Bc}}
\newcommand{\dcap}[0]{\hat{\Bd}}
\newcommand{\ecap}[0]{\hat{\Be}}
\newcommand{\fcap}[0]{\hat{\Bf}}
\newcommand{\gcap}[0]{\hat{\Bg}}
\newcommand{\hcap}[0]{\hat{\Bh}}
\newcommand{\icap}[0]{\hat{\Bi}}
\newcommand{\jcap}[0]{\hat{\Bj}}
\newcommand{\kcap}[0]{\hat{\Bk}}
\newcommand{\lcap}[0]{\hat{\Bl}}
\newcommand{\mcap}[0]{\hat{\Bm}}
\newcommand{\ncap}[0]{\hat{\Bn}}
\newcommand{\ocap}[0]{\hat{\Bo}}
\newcommand{\pcap}[0]{\hat{\Bp}}
\newcommand{\qcap}[0]{\hat{\Bq}}
\newcommand{\rcap}[0]{\hat{\Br}}
\newcommand{\scap}[0]{\hat{\Bs}}
\newcommand{\tcap}[0]{\hat{\Bt}}
\newcommand{\ucap}[0]{\hat{\Bu}}
\newcommand{\vcap}[0]{\hat{\Bv}}
\newcommand{\wcap}[0]{\hat{\Bw}}
\newcommand{\xcap}[0]{\hat{\Bx}}
\newcommand{\ycap}[0]{\hat{\By}}
\newcommand{\zcap}[0]{\hat{\Bz}}
\newcommand{\thetacap}[0]{\hat{\Btheta}}

%
% to write R^n and C^n in a distinguishable fashion.  Perhaps change this
% to the double lined characters upon figuring out how to do so.
%
\newcommand{\C}[1]{$\mathbb{C}^{#1}$}
\newcommand{\R}[1]{$\mathbb{R}^{#1}$}

%
% various generally useful helpers
%

% derivative of #1 wrt. #2:
\newcommand{\D}[2] {\frac {d#2} {d#1}}

\newcommand{\inv}[1]{\frac{1}{#1}}
\newcommand{\cross}[0]{\times}

\newcommand{\abs}[1]{\lvert{#1}\rvert}
\newcommand{\norm}[1]{\lVert{#1}\rVert}
\newcommand{\innerprod}[2]{\langle{#1}, {#2}\rangle}
\newcommand{\dotprod}[2]{{#1} \cdot {#2}}
\newcommand{\bdotprod}[2]{\left({#1} \cdot {#2}\right)}
\newcommand{\crossprod}[2]{{#1} \cross {#2}}
\newcommand{\tripleprod}[3]{\dotprod{\left(\crossprod{#1}{#2}\right)}{#3}}

\DeclareMathOperator{\Proj}{Proj}
\DeclareMathOperator{\Span}{span}
\DeclareMathOperator{\Sgn}{sgn}
\DeclareMathOperator{\Area}{Area}
\DeclareMathOperator{\Volume}{Volume}

%
% A few miscellaneous things specific to this document
%
\newcommand{\crossop}[1]{\crossprod{#1}{}}

% R2 vector.
\newcommand{\VectorTwo}[2]{
\begin{bmatrix}
 {#1} \\
 {#2}
\end{bmatrix}
}

\newcommand{\VectorN}[1]{
\begin{bmatrix}
{#1}_1 \\
{#1}_2 \\
\vdots \\
{#1}_N \\
\end{bmatrix}
}

\newcommand{\DETuvij}[4]{
\begin{vmatrix}
 {#1}_{#3} & {#1}_{#4} \\
 {#2}_{#3} & {#2}_{#4}
\end{vmatrix}
}

\newcommand{\DETuvwijk}[6]{
\begin{vmatrix}
 {#1}_{#4} & {#1}_{#5} & {#1}_{#6} \\
 {#2}_{#4} & {#2}_{#5} & {#2}_{#6} \\
 {#3}_{#4} & {#3}_{#5} & {#3}_{#6}
\end{vmatrix}
}

\newcommand{\DETuvwxijkl}[8]{
\begin{vmatrix}
 {#1}_{#5} & {#1}_{#6} & {#1}_{#7} & {#1}_{#8} \\
 {#2}_{#5} & {#2}_{#6} & {#2}_{#7} & {#2}_{#8} \\
 {#3}_{#5} & {#3}_{#6} & {#3}_{#7} & {#3}_{#8} \\
 {#4}_{#5} & {#4}_{#6} & {#4}_{#7} & {#4}_{#8} \\
\end{vmatrix}
}

%\newcommand{\DETuvwxyijklm}[10]{
%\begin{vmatrix}
% {#1}_{#6} & {#1}_{#7} & {#1}_{#8} & {#1}_{#9} & {#1}_{#10} \\
% {#2}_{#6} & {#2}_{#7} & {#2}_{#8} & {#2}_{#9} & {#2}_{#10} \\
% {#3}_{#6} & {#3}_{#7} & {#3}_{#8} & {#3}_{#9} & {#3}_{#10} \\
% {#4}_{#6} & {#4}_{#7} & {#4}_{#8} & {#4}_{#9} & {#4}_{#10} \\
% {#5}_{#6} & {#5}_{#7} & {#5}_{#8} & {#5}_{#9} & {#5}_{#10}
%\end{vmatrix}
%}

% R3 vector.
\newcommand{\VectorThree}[3]{
\begin{bmatrix}
 {#1} \\
 {#2} \\
 {#3}
\end{bmatrix}
}



\author{Peeter Joot}
\email{peeter.joot@gmail.com}

%\documentclass[]{eliblogwidescreen}

\usepackage{amsmath}
\usepackage{mathpazo}

%
% shorthand for bold symbols, convenient for vectors and matrices
%
\newcommand{\Ba}[0]{\mathbf{a}}
\newcommand{\Bb}[0]{\mathbf{b}}
\newcommand{\Bc}[0]{\mathbf{c}}
\newcommand{\Bd}[0]{\mathbf{d}}
\newcommand{\Be}[0]{\mathbf{e}}
\newcommand{\Bf}[0]{\mathbf{f}}
\newcommand{\Bg}[0]{\mathbf{g}}
\newcommand{\Bh}[0]{\mathbf{h}}
\newcommand{\Bi}[0]{\mathbf{i}}
\newcommand{\Bj}[0]{\mathbf{j}}
\newcommand{\Bk}[0]{\mathbf{k}}
\newcommand{\Bl}[0]{\mathbf{l}}
\newcommand{\Bm}[0]{\mathbf{m}}
\newcommand{\Bn}[0]{\mathbf{n}}
\newcommand{\Bo}[0]{\mathbf{o}}
\newcommand{\Bp}[0]{\mathbf{p}}
\newcommand{\Bq}[0]{\mathbf{q}}
\newcommand{\Br}[0]{\mathbf{r}}
\newcommand{\Bs}[0]{\mathbf{s}}
\newcommand{\Bt}[0]{\mathbf{t}}
\newcommand{\Bu}[0]{\mathbf{u}}
\newcommand{\Bv}[0]{\mathbf{v}}
\newcommand{\Bw}[0]{\mathbf{w}}
\newcommand{\Bx}[0]{\mathbf{x}}
\newcommand{\By}[0]{\mathbf{y}}
\newcommand{\Bz}[0]{\mathbf{z}}
\newcommand{\BA}[0]{\mathbf{A}}
\newcommand{\BB}[0]{\mathbf{B}}
\newcommand{\BC}[0]{\mathbf{C}}
\newcommand{\BD}[0]{\mathbf{D}}
\newcommand{\BE}[0]{\mathbf{E}}
\newcommand{\BF}[0]{\mathbf{F}}
\newcommand{\BG}[0]{\mathbf{G}}
\newcommand{\BH}[0]{\mathbf{H}}
\newcommand{\BI}[0]{\mathbf{I}}
\newcommand{\BJ}[0]{\mathbf{J}}
\newcommand{\BK}[0]{\mathbf{K}}
\newcommand{\BL}[0]{\mathbf{L}}
\newcommand{\BM}[0]{\mathbf{M}}
\newcommand{\BN}[0]{\mathbf{N}}
\newcommand{\BO}[0]{\mathbf{O}}
\newcommand{\BP}[0]{\mathbf{P}}
\newcommand{\BQ}[0]{\mathbf{Q}}
\newcommand{\BR}[0]{\mathbf{R}}
\newcommand{\BS}[0]{\mathbf{S}}
\newcommand{\BT}[0]{\mathbf{T}}
\newcommand{\BU}[0]{\mathbf{U}}
\newcommand{\BV}[0]{\mathbf{V}}
\newcommand{\BW}[0]{\mathbf{W}}
\newcommand{\BX}[0]{\mathbf{X}}
\newcommand{\BY}[0]{\mathbf{Y}}
\newcommand{\BZ}[0]{\mathbf{Z}}

\newcommand{\Bzero}[0]{\mathbf{0}}
\newcommand{\Btheta}[0]{\boldsymbol{\theta}}
\newcommand{\Btau}[0]{\boldsymbol{\tau}}
\newcommand{\Bomega}[0]{\boldsymbol{\omega}}

%
% shorthand for unit vectors
%
\newcommand{\acap}[0]{\hat{\Ba}}
\newcommand{\bcap}[0]{\hat{\Bb}}
\newcommand{\ccap}[0]{\hat{\Bc}}
\newcommand{\dcap}[0]{\hat{\Bd}}
\newcommand{\ecap}[0]{\hat{\Be}}
\newcommand{\fcap}[0]{\hat{\Bf}}
\newcommand{\gcap}[0]{\hat{\Bg}}
\newcommand{\hcap}[0]{\hat{\Bh}}
\newcommand{\icap}[0]{\hat{\Bi}}
\newcommand{\jcap}[0]{\hat{\Bj}}
\newcommand{\kcap}[0]{\hat{\Bk}}
\newcommand{\lcap}[0]{\hat{\Bl}}
\newcommand{\mcap}[0]{\hat{\Bm}}
\newcommand{\ncap}[0]{\hat{\Bn}}
\newcommand{\ocap}[0]{\hat{\Bo}}
\newcommand{\pcap}[0]{\hat{\Bp}}
\newcommand{\qcap}[0]{\hat{\Bq}}
\newcommand{\rcap}[0]{\hat{\Br}}
\newcommand{\scap}[0]{\hat{\Bs}}
\newcommand{\tcap}[0]{\hat{\Bt}}
\newcommand{\ucap}[0]{\hat{\Bu}}
\newcommand{\vcap}[0]{\hat{\Bv}}
\newcommand{\wcap}[0]{\hat{\Bw}}
\newcommand{\xcap}[0]{\hat{\Bx}}
\newcommand{\ycap}[0]{\hat{\By}}
\newcommand{\zcap}[0]{\hat{\Bz}}
\newcommand{\thetacap}[0]{\hat{\Btheta}}

%
% to write R^n and C^n in a distinguishable fashion.  Perhaps change this
% to the double lined characters upon figuring out how to do so.
%
\newcommand{\C}[1]{$\mathbb{C}^{#1}$}
\newcommand{\R}[1]{$\mathbb{R}^{#1}$}

%
% various generally useful helpers
%

% derivative of #1 wrt. #2:
\newcommand{\D}[2] {\frac {d#2} {d#1}}

\newcommand{\inv}[1]{\frac{1}{#1}}
\newcommand{\cross}[0]{\times}

\newcommand{\abs}[1]{\lvert{#1}\rvert}
\newcommand{\norm}[1]{\lVert{#1}\rVert}
\newcommand{\innerprod}[2]{\langle{#1}, {#2}\rangle}
\newcommand{\dotprod}[2]{{#1} \cdot {#2}}
\newcommand{\bdotprod}[2]{\left({#1} \cdot {#2}\right)}
\newcommand{\crossprod}[2]{{#1} \cross {#2}}
\newcommand{\tripleprod}[3]{\dotprod{\left(\crossprod{#1}{#2}\right)}{#3}}

\DeclareMathOperator{\Proj}{Proj}
\DeclareMathOperator{\Span}{span}
\DeclareMathOperator{\Sgn}{sgn}
\DeclareMathOperator{\Area}{Area}
\DeclareMathOperator{\Volume}{Volume}

%
% A few miscellaneous things specific to this document
%
\newcommand{\crossop}[1]{\crossprod{#1}{}}

% R2 vector.
\newcommand{\VectorTwo}[2]{
\begin{bmatrix}
 {#1} \\
 {#2}
\end{bmatrix}
}

\newcommand{\VectorN}[1]{
\begin{bmatrix}
{#1}_1 \\
{#1}_2 \\
\vdots \\
{#1}_N \\
\end{bmatrix}
}

\newcommand{\DETuvij}[4]{
\begin{vmatrix}
 {#1}_{#3} & {#1}_{#4} \\
 {#2}_{#3} & {#2}_{#4}
\end{vmatrix}
}

\newcommand{\DETuvwijk}[6]{
\begin{vmatrix}
 {#1}_{#4} & {#1}_{#5} & {#1}_{#6} \\
 {#2}_{#4} & {#2}_{#5} & {#2}_{#6} \\
 {#3}_{#4} & {#3}_{#5} & {#3}_{#6}
\end{vmatrix}
}

\newcommand{\DETuvwxijkl}[8]{
\begin{vmatrix}
 {#1}_{#5} & {#1}_{#6} & {#1}_{#7} & {#1}_{#8} \\
 {#2}_{#5} & {#2}_{#6} & {#2}_{#7} & {#2}_{#8} \\
 {#3}_{#5} & {#3}_{#6} & {#3}_{#7} & {#3}_{#8} \\
 {#4}_{#5} & {#4}_{#6} & {#4}_{#7} & {#4}_{#8} \\
\end{vmatrix}
}

%\newcommand{\DETuvwxyijklm}[10]{
%\begin{vmatrix}
% {#1}_{#6} & {#1}_{#7} & {#1}_{#8} & {#1}_{#9} & {#1}_{#10} \\
% {#2}_{#6} & {#2}_{#7} & {#2}_{#8} & {#2}_{#9} & {#2}_{#10} \\
% {#3}_{#6} & {#3}_{#7} & {#3}_{#8} & {#3}_{#9} & {#3}_{#10} \\
% {#4}_{#6} & {#4}_{#7} & {#4}_{#8} & {#4}_{#9} & {#4}_{#10} \\
% {#5}_{#6} & {#5}_{#7} & {#5}_{#8} & {#5}_{#9} & {#5}_{#10}
%\end{vmatrix}
%}

% R3 vector.
\newcommand{\VectorThree}[3]{
\begin{bmatrix}
 {#1} \\
 {#2} \\
 {#3}
\end{bmatrix}
}



\author{Peeter Joot}
\email{peeter.joot@gmail.com}


\chapter{Desai Chapter 9 notes and problems.}
\label{chap:desaiCh9}
%\useCCL
\blogpage{http://sites.google.com/site/peeterjoot/math2010/desaiCh9.pdf}
\date{Nov 19, 2010}
\revisionInfo{desaiCh9.tex}

\beginArtWithToc
%\beginArtNoToc

\section{Motivation.}

Chapter 9 notes for \cite{desai2009quantum}.

\section{Notes}
\section{Problems}

\subsection{Problem 4.}
\subsubsection{Statement.}

Consider the following two-dimensional harmonic oscilator problem:

\begin{align}\label{eqn:desaiCh9:400}
-\frac{\hbar^2}{2m} \frac{\partial^2 u}{\partial x^2}
-\frac{\hbar^2}{2m} \frac{\partial^2 u}{\partial y^2}
+ \inv{2} K_1 x^2 u
+ \inv{2} K_2 y^2 u
= E u
\end{align}

where $(x,y)$ are the coordinates of the particle.  Use the separation of variables technique to obtain the energy eigenvalues.  Discuss the degeneracy in the eigenvalues if $K_1 = K_2$.

\subsubsection{Solution.}

Write $u = A(x) B(y)$.  Substitute and dividing throughout by $u$ we have

\begin{align}\label{eqn:desaiCh9:401}
\left( -\frac{\hbar^2}{2m} \frac{A''}{A} + \inv{2} K_1 x^2 \right)
+\left( -\frac{\hbar^2}{2m} \frac{B''}{B} + \inv{2} K_2 y^2 \right)
= E 
\end{align}

Introduction of a pair of constants $E_1, E_2$ for each of the independent terms we have

\begin{align}\label{eqn:desaiCh9:403}
H_1 A &= -\frac{\hbar^2}{2m} A'' + \inv{2} K_1 x^2 A = E_1 A \\
H_2 B &= -\frac{\hbar^2}{2m} B'' + \inv{2} K_1 y^2 B = E_2 B \\
H &= H_1 + H_2 \\
E  &= E_1 + E_2
\end{align}

For each of these equations we have a set of quantized eigenvalues and can write

\begin{align}\label{eqn:desaiCh9:404}
E_{1m} &= \left(m + \inv{2}\right) \hbar \sqrt{\frac{K_1}{m}} \\
E_{2n} &= \left(n + \inv{2}\right) \hbar \sqrt{\frac{K_2}{m}} \\
H_1 A_m(x) &= E_{1m} A_m(x) \\
H_2 A_n(y) &= E_{2n} B_n(y)
\end{align}

The complete eigenstates are then 

\begin{align}\label{eqn:desaiCh9:405}
u_{mn}(x,y) &= A_m(x) B_n(y)
\end{align}

with total energy satisfying 
\begin{align}\label{eqn:desaiCh9:406}
H u_{mn}(x,y) &= 
\frac{\hbar}{\sqrt{m}} \left( \left(m + \inv{2}\right) \sqrt{K_1} + \left(n + \inv{2}\right) \sqrt{K_2} \right) u_{mn}(x,y)
\end{align}

A general state requires a double sum over the possible combinations of states $\Psi = \sum_{mn} c_{mn} u_{mn}$, however if $K_1 = K_2 = K$, we cannot distinguish between $u_{mn}$ and $u_{nm}$ based on the energy eigenvalues

\begin{align}\label{eqn:desaiCh9:407}
H u_{mn}(x,y) &= \hbar\sqrt{\frac{K}{m}} \left( m + n + 1 \right) u_{mn}(x,y) = H u_{nm}(x,y)
\end{align}

In this case, we can write the wave function corresponding to a general state for the system as just $\Psi = \sum_{m+ n = \text{constant}} c_{mn} u_{mn}$.  This reduction in the cardinality of this set of basis eigenstates is the degeneracy to be discussed.

\subsection{Problem 5,6.}
\subsubsection{Statement.}

Consider now a variation on Problem 4 in which we have a coupled oscillator with the potential given by

\begin{align}\label{eqn:desaiCh9:500}
V(x,y) = \inv{2} K \Bigl( x^2 + y^2 + 2 \lambda x y \Bigr)
\end{align}

Obtain the energy eigenvalues by changing variables $(x,y)$ to $(x', y')$ such that the new potential is quadratic in $(x', y')$, without the coupling term.

\subsubsection{Solution.}

This has the look of a diagonalization problem so we write the potential in matrix form

\begin{align}\label{eqn:desaiCh9:501}
V(x,y) 
= \inv{2} K 
\begin{bmatrix}
x & y
\end{bmatrix}
\begin{bmatrix}
1 & \lambda \\
\lambda & 1 
\end{bmatrix}
\begin{bmatrix}
x \\ y
\end{bmatrix} = \inv{2} K \tilde{X} M X
\end{align}

The similarity transformation required is
\begin{align}\label{eqn:desaiCh9:502}
M = \inv{\sqrt{2}}
\begin{bmatrix}
1 & 1 \\
1 & -1
\end{bmatrix}
\begin{bmatrix}
1+ \lambda & 0 \\
0 & 1 - \lambda
\end{bmatrix}
\inv{\sqrt{2}}
\begin{bmatrix}
1 & 1 \\
1 & -1
\end{bmatrix}
\end{align}

Our change of variables is therefore
\begin{align}\label{eqn:desaiCh9:503}
X' = 
\inv{\sqrt{2}}
\begin{bmatrix}
1 & 1 \\
1 & -1
\end{bmatrix}
X
= 
\inv{\sqrt{2}}
\begin{bmatrix}
x + y \\
x - y
\end{bmatrix}
\end{align}

Our Laplacian should also remain diagonal under this orthonormal transformation, but we can verify this by expanding out the partials explicitly
\begin{align}\label{eqn:desaiCh9:504}
\PD{x}{} &= 
\PD{x}{x'}\PD{x'}{}
+\PD{x}{y'}\PD{y'}{} = \inv{\sqrt{2}} \left( \PD{x'}{} + \PD{y'}{} \right) \\
\PD{y}{} &= 
\PD{y}{x'}\PD{x'}{} +\PD{y}{y'}\PD{y'}{}
= \inv{\sqrt{2}} 
\left( \PD{x'}{} - \PD{y'}{} \right)
\end{align}

Squaring and summing we have
\begin{align}\label{eqn:desaiCh9:505}
\frac{\partial^2}{\partial x^2} +
\frac{\partial^2}{\partial y^2}
&=
\inv{2} \left( \PD{x'}{} + \PD{y'}{} \right)^2
+\inv{2} \left( \PD{x'}{} - \PD{y'}{} \right)^2
=
\frac{\partial^2}{\partial {x'}^2} +
\frac{\partial^2}{\partial {y'}^2}
\end{align}

Our transformed Hamiltonian operator is thus

\begin{align}\label{eqn:desaiCh9:506}
-\frac{\hbar^2}{2m} \frac{\partial^2 u}{\partial {x'}^2}
-\frac{\hbar^2}{2m} \frac{\partial^2 u}{\partial {y'}^2}
+ \inv{2} K(1+\lambda) {x'}^2 u
+ \inv{2} K(1-\lambda) {y'}^2 u
= E u
\end{align}

So, provided $\Abs{\lambda} < 1$, the energy eigenvalue equation is given by \ref{eqn:desaiCh9:406} with $K_1 = K(1+ \lambda)$, and $K_2 = K(1 -\lambda)$.

\EndArticle

%
% Copyright � 2015 Peeter Joot.  All Rights Reserved.
% Licenced as described in the file LICENSE under the root directory of this GIT repository.
%
\documentclass[]{eliblog}

\usepackage{amsmath}
\usepackage{mathpazo}

%
% shorthand for bold symbols, convenient for vectors and matrices
%
\newcommand{\Ba}[0]{\mathbf{a}}
\newcommand{\Bb}[0]{\mathbf{b}}
\newcommand{\Bc}[0]{\mathbf{c}}
\newcommand{\Bd}[0]{\mathbf{d}}
\newcommand{\Be}[0]{\mathbf{e}}
\newcommand{\Bf}[0]{\mathbf{f}}
\newcommand{\Bg}[0]{\mathbf{g}}
\newcommand{\Bh}[0]{\mathbf{h}}
\newcommand{\Bi}[0]{\mathbf{i}}
\newcommand{\Bj}[0]{\mathbf{j}}
\newcommand{\Bk}[0]{\mathbf{k}}
\newcommand{\Bl}[0]{\mathbf{l}}
\newcommand{\Bm}[0]{\mathbf{m}}
\newcommand{\Bn}[0]{\mathbf{n}}
\newcommand{\Bo}[0]{\mathbf{o}}
\newcommand{\Bp}[0]{\mathbf{p}}
\newcommand{\Bq}[0]{\mathbf{q}}
\newcommand{\Br}[0]{\mathbf{r}}
\newcommand{\Bs}[0]{\mathbf{s}}
\newcommand{\Bt}[0]{\mathbf{t}}
\newcommand{\Bu}[0]{\mathbf{u}}
\newcommand{\Bv}[0]{\mathbf{v}}
\newcommand{\Bw}[0]{\mathbf{w}}
\newcommand{\Bx}[0]{\mathbf{x}}
\newcommand{\By}[0]{\mathbf{y}}
\newcommand{\Bz}[0]{\mathbf{z}}
\newcommand{\BA}[0]{\mathbf{A}}
\newcommand{\BB}[0]{\mathbf{B}}
\newcommand{\BC}[0]{\mathbf{C}}
\newcommand{\BD}[0]{\mathbf{D}}
\newcommand{\BE}[0]{\mathbf{E}}
\newcommand{\BF}[0]{\mathbf{F}}
\newcommand{\BG}[0]{\mathbf{G}}
\newcommand{\BH}[0]{\mathbf{H}}
\newcommand{\BI}[0]{\mathbf{I}}
\newcommand{\BJ}[0]{\mathbf{J}}
\newcommand{\BK}[0]{\mathbf{K}}
\newcommand{\BL}[0]{\mathbf{L}}
\newcommand{\BM}[0]{\mathbf{M}}
\newcommand{\BN}[0]{\mathbf{N}}
\newcommand{\BO}[0]{\mathbf{O}}
\newcommand{\BP}[0]{\mathbf{P}}
\newcommand{\BQ}[0]{\mathbf{Q}}
\newcommand{\BR}[0]{\mathbf{R}}
\newcommand{\BS}[0]{\mathbf{S}}
\newcommand{\BT}[0]{\mathbf{T}}
\newcommand{\BU}[0]{\mathbf{U}}
\newcommand{\BV}[0]{\mathbf{V}}
\newcommand{\BW}[0]{\mathbf{W}}
\newcommand{\BX}[0]{\mathbf{X}}
\newcommand{\BY}[0]{\mathbf{Y}}
\newcommand{\BZ}[0]{\mathbf{Z}}

\newcommand{\Bzero}[0]{\mathbf{0}}
\newcommand{\Btheta}[0]{\boldsymbol{\theta}}
\newcommand{\Btau}[0]{\boldsymbol{\tau}}
\newcommand{\Bomega}[0]{\boldsymbol{\omega}}

%
% shorthand for unit vectors
%
\newcommand{\acap}[0]{\hat{\Ba}}
\newcommand{\bcap}[0]{\hat{\Bb}}
\newcommand{\ccap}[0]{\hat{\Bc}}
\newcommand{\dcap}[0]{\hat{\Bd}}
\newcommand{\ecap}[0]{\hat{\Be}}
\newcommand{\fcap}[0]{\hat{\Bf}}
\newcommand{\gcap}[0]{\hat{\Bg}}
\newcommand{\hcap}[0]{\hat{\Bh}}
\newcommand{\icap}[0]{\hat{\Bi}}
\newcommand{\jcap}[0]{\hat{\Bj}}
\newcommand{\kcap}[0]{\hat{\Bk}}
\newcommand{\lcap}[0]{\hat{\Bl}}
\newcommand{\mcap}[0]{\hat{\Bm}}
\newcommand{\ncap}[0]{\hat{\Bn}}
\newcommand{\ocap}[0]{\hat{\Bo}}
\newcommand{\pcap}[0]{\hat{\Bp}}
\newcommand{\qcap}[0]{\hat{\Bq}}
\newcommand{\rcap}[0]{\hat{\Br}}
\newcommand{\scap}[0]{\hat{\Bs}}
\newcommand{\tcap}[0]{\hat{\Bt}}
\newcommand{\ucap}[0]{\hat{\Bu}}
\newcommand{\vcap}[0]{\hat{\Bv}}
\newcommand{\wcap}[0]{\hat{\Bw}}
\newcommand{\xcap}[0]{\hat{\Bx}}
\newcommand{\ycap}[0]{\hat{\By}}
\newcommand{\zcap}[0]{\hat{\Bz}}
\newcommand{\thetacap}[0]{\hat{\Btheta}}

%
% to write R^n and C^n in a distinguishable fashion.  Perhaps change this
% to the double lined characters upon figuring out how to do so.
%
\newcommand{\C}[1]{$\mathbb{C}^{#1}$}
\newcommand{\R}[1]{$\mathbb{R}^{#1}$}

%
% various generally useful helpers
%

% derivative of #1 wrt. #2:
\newcommand{\D}[2] {\frac {d#2} {d#1}}

\newcommand{\inv}[1]{\frac{1}{#1}}
\newcommand{\cross}[0]{\times}

\newcommand{\abs}[1]{\lvert{#1}\rvert}
\newcommand{\norm}[1]{\lVert{#1}\rVert}
\newcommand{\innerprod}[2]{\langle{#1}, {#2}\rangle}
\newcommand{\dotprod}[2]{{#1} \cdot {#2}}
\newcommand{\bdotprod}[2]{\left({#1} \cdot {#2}\right)}
\newcommand{\crossprod}[2]{{#1} \cross {#2}}
\newcommand{\tripleprod}[3]{\dotprod{\left(\crossprod{#1}{#2}\right)}{#3}}

\DeclareMathOperator{\Proj}{Proj}
\DeclareMathOperator{\Span}{span}
\DeclareMathOperator{\Sgn}{sgn}
\DeclareMathOperator{\Area}{Area}
\DeclareMathOperator{\Volume}{Volume}

%
% A few miscellaneous things specific to this document
%
\newcommand{\crossop}[1]{\crossprod{#1}{}}

% R2 vector.
\newcommand{\VectorTwo}[2]{
\begin{bmatrix}
 {#1} \\
 {#2}
\end{bmatrix}
}

\newcommand{\VectorN}[1]{
\begin{bmatrix}
{#1}_1 \\
{#1}_2 \\
\vdots \\
{#1}_N \\
\end{bmatrix}
}

\newcommand{\DETuvij}[4]{
\begin{vmatrix}
 {#1}_{#3} & {#1}_{#4} \\
 {#2}_{#3} & {#2}_{#4}
\end{vmatrix}
}

\newcommand{\DETuvwijk}[6]{
\begin{vmatrix}
 {#1}_{#4} & {#1}_{#5} & {#1}_{#6} \\
 {#2}_{#4} & {#2}_{#5} & {#2}_{#6} \\
 {#3}_{#4} & {#3}_{#5} & {#3}_{#6}
\end{vmatrix}
}

\newcommand{\DETuvwxijkl}[8]{
\begin{vmatrix}
 {#1}_{#5} & {#1}_{#6} & {#1}_{#7} & {#1}_{#8} \\
 {#2}_{#5} & {#2}_{#6} & {#2}_{#7} & {#2}_{#8} \\
 {#3}_{#5} & {#3}_{#6} & {#3}_{#7} & {#3}_{#8} \\
 {#4}_{#5} & {#4}_{#6} & {#4}_{#7} & {#4}_{#8} \\
\end{vmatrix}
}

%\newcommand{\DETuvwxyijklm}[10]{
%\begin{vmatrix}
% {#1}_{#6} & {#1}_{#7} & {#1}_{#8} & {#1}_{#9} & {#1}_{#10} \\
% {#2}_{#6} & {#2}_{#7} & {#2}_{#8} & {#2}_{#9} & {#2}_{#10} \\
% {#3}_{#6} & {#3}_{#7} & {#3}_{#8} & {#3}_{#9} & {#3}_{#10} \\
% {#4}_{#6} & {#4}_{#7} & {#4}_{#8} & {#4}_{#9} & {#4}_{#10} \\
% {#5}_{#6} & {#5}_{#7} & {#5}_{#8} & {#5}_{#9} & {#5}_{#10}
%\end{vmatrix}
%}

% R3 vector.
\newcommand{\VectorThree}[3]{
\begin{bmatrix}
 {#1} \\
 {#2} \\
 {#3}
\end{bmatrix}
}



\author{Peeter Joot}
\email{peeter.joot@gmail.com}

%\documentclass[]{eliblogwidescreen}

\usepackage{amsmath}
\usepackage{mathpazo}

%
% shorthand for bold symbols, convenient for vectors and matrices
%
\newcommand{\Ba}[0]{\mathbf{a}}
\newcommand{\Bb}[0]{\mathbf{b}}
\newcommand{\Bc}[0]{\mathbf{c}}
\newcommand{\Bd}[0]{\mathbf{d}}
\newcommand{\Be}[0]{\mathbf{e}}
\newcommand{\Bf}[0]{\mathbf{f}}
\newcommand{\Bg}[0]{\mathbf{g}}
\newcommand{\Bh}[0]{\mathbf{h}}
\newcommand{\Bi}[0]{\mathbf{i}}
\newcommand{\Bj}[0]{\mathbf{j}}
\newcommand{\Bk}[0]{\mathbf{k}}
\newcommand{\Bl}[0]{\mathbf{l}}
\newcommand{\Bm}[0]{\mathbf{m}}
\newcommand{\Bn}[0]{\mathbf{n}}
\newcommand{\Bo}[0]{\mathbf{o}}
\newcommand{\Bp}[0]{\mathbf{p}}
\newcommand{\Bq}[0]{\mathbf{q}}
\newcommand{\Br}[0]{\mathbf{r}}
\newcommand{\Bs}[0]{\mathbf{s}}
\newcommand{\Bt}[0]{\mathbf{t}}
\newcommand{\Bu}[0]{\mathbf{u}}
\newcommand{\Bv}[0]{\mathbf{v}}
\newcommand{\Bw}[0]{\mathbf{w}}
\newcommand{\Bx}[0]{\mathbf{x}}
\newcommand{\By}[0]{\mathbf{y}}
\newcommand{\Bz}[0]{\mathbf{z}}
\newcommand{\BA}[0]{\mathbf{A}}
\newcommand{\BB}[0]{\mathbf{B}}
\newcommand{\BC}[0]{\mathbf{C}}
\newcommand{\BD}[0]{\mathbf{D}}
\newcommand{\BE}[0]{\mathbf{E}}
\newcommand{\BF}[0]{\mathbf{F}}
\newcommand{\BG}[0]{\mathbf{G}}
\newcommand{\BH}[0]{\mathbf{H}}
\newcommand{\BI}[0]{\mathbf{I}}
\newcommand{\BJ}[0]{\mathbf{J}}
\newcommand{\BK}[0]{\mathbf{K}}
\newcommand{\BL}[0]{\mathbf{L}}
\newcommand{\BM}[0]{\mathbf{M}}
\newcommand{\BN}[0]{\mathbf{N}}
\newcommand{\BO}[0]{\mathbf{O}}
\newcommand{\BP}[0]{\mathbf{P}}
\newcommand{\BQ}[0]{\mathbf{Q}}
\newcommand{\BR}[0]{\mathbf{R}}
\newcommand{\BS}[0]{\mathbf{S}}
\newcommand{\BT}[0]{\mathbf{T}}
\newcommand{\BU}[0]{\mathbf{U}}
\newcommand{\BV}[0]{\mathbf{V}}
\newcommand{\BW}[0]{\mathbf{W}}
\newcommand{\BX}[0]{\mathbf{X}}
\newcommand{\BY}[0]{\mathbf{Y}}
\newcommand{\BZ}[0]{\mathbf{Z}}

\newcommand{\Bzero}[0]{\mathbf{0}}
\newcommand{\Btheta}[0]{\boldsymbol{\theta}}
\newcommand{\Btau}[0]{\boldsymbol{\tau}}
\newcommand{\Bomega}[0]{\boldsymbol{\omega}}

%
% shorthand for unit vectors
%
\newcommand{\acap}[0]{\hat{\Ba}}
\newcommand{\bcap}[0]{\hat{\Bb}}
\newcommand{\ccap}[0]{\hat{\Bc}}
\newcommand{\dcap}[0]{\hat{\Bd}}
\newcommand{\ecap}[0]{\hat{\Be}}
\newcommand{\fcap}[0]{\hat{\Bf}}
\newcommand{\gcap}[0]{\hat{\Bg}}
\newcommand{\hcap}[0]{\hat{\Bh}}
\newcommand{\icap}[0]{\hat{\Bi}}
\newcommand{\jcap}[0]{\hat{\Bj}}
\newcommand{\kcap}[0]{\hat{\Bk}}
\newcommand{\lcap}[0]{\hat{\Bl}}
\newcommand{\mcap}[0]{\hat{\Bm}}
\newcommand{\ncap}[0]{\hat{\Bn}}
\newcommand{\ocap}[0]{\hat{\Bo}}
\newcommand{\pcap}[0]{\hat{\Bp}}
\newcommand{\qcap}[0]{\hat{\Bq}}
\newcommand{\rcap}[0]{\hat{\Br}}
\newcommand{\scap}[0]{\hat{\Bs}}
\newcommand{\tcap}[0]{\hat{\Bt}}
\newcommand{\ucap}[0]{\hat{\Bu}}
\newcommand{\vcap}[0]{\hat{\Bv}}
\newcommand{\wcap}[0]{\hat{\Bw}}
\newcommand{\xcap}[0]{\hat{\Bx}}
\newcommand{\ycap}[0]{\hat{\By}}
\newcommand{\zcap}[0]{\hat{\Bz}}
\newcommand{\thetacap}[0]{\hat{\Btheta}}

%
% to write R^n and C^n in a distinguishable fashion.  Perhaps change this
% to the double lined characters upon figuring out how to do so.
%
\newcommand{\C}[1]{$\mathbb{C}^{#1}$}
\newcommand{\R}[1]{$\mathbb{R}^{#1}$}

%
% various generally useful helpers
%

% derivative of #1 wrt. #2:
\newcommand{\D}[2] {\frac {d#2} {d#1}}

\newcommand{\inv}[1]{\frac{1}{#1}}
\newcommand{\cross}[0]{\times}

\newcommand{\abs}[1]{\lvert{#1}\rvert}
\newcommand{\norm}[1]{\lVert{#1}\rVert}
\newcommand{\innerprod}[2]{\langle{#1}, {#2}\rangle}
\newcommand{\dotprod}[2]{{#1} \cdot {#2}}
\newcommand{\bdotprod}[2]{\left({#1} \cdot {#2}\right)}
\newcommand{\crossprod}[2]{{#1} \cross {#2}}
\newcommand{\tripleprod}[3]{\dotprod{\left(\crossprod{#1}{#2}\right)}{#3}}

\DeclareMathOperator{\Proj}{Proj}
\DeclareMathOperator{\Span}{span}
\DeclareMathOperator{\Sgn}{sgn}
\DeclareMathOperator{\Area}{Area}
\DeclareMathOperator{\Volume}{Volume}

%
% A few miscellaneous things specific to this document
%
\newcommand{\crossop}[1]{\crossprod{#1}{}}

% R2 vector.
\newcommand{\VectorTwo}[2]{
\begin{bmatrix}
 {#1} \\
 {#2}
\end{bmatrix}
}

\newcommand{\VectorN}[1]{
\begin{bmatrix}
{#1}_1 \\
{#1}_2 \\
\vdots \\
{#1}_N \\
\end{bmatrix}
}

\newcommand{\DETuvij}[4]{
\begin{vmatrix}
 {#1}_{#3} & {#1}_{#4} \\
 {#2}_{#3} & {#2}_{#4}
\end{vmatrix}
}

\newcommand{\DETuvwijk}[6]{
\begin{vmatrix}
 {#1}_{#4} & {#1}_{#5} & {#1}_{#6} \\
 {#2}_{#4} & {#2}_{#5} & {#2}_{#6} \\
 {#3}_{#4} & {#3}_{#5} & {#3}_{#6}
\end{vmatrix}
}

\newcommand{\DETuvwxijkl}[8]{
\begin{vmatrix}
 {#1}_{#5} & {#1}_{#6} & {#1}_{#7} & {#1}_{#8} \\
 {#2}_{#5} & {#2}_{#6} & {#2}_{#7} & {#2}_{#8} \\
 {#3}_{#5} & {#3}_{#6} & {#3}_{#7} & {#3}_{#8} \\
 {#4}_{#5} & {#4}_{#6} & {#4}_{#7} & {#4}_{#8} \\
\end{vmatrix}
}

%\newcommand{\DETuvwxyijklm}[10]{
%\begin{vmatrix}
% {#1}_{#6} & {#1}_{#7} & {#1}_{#8} & {#1}_{#9} & {#1}_{#10} \\
% {#2}_{#6} & {#2}_{#7} & {#2}_{#8} & {#2}_{#9} & {#2}_{#10} \\
% {#3}_{#6} & {#3}_{#7} & {#3}_{#8} & {#3}_{#9} & {#3}_{#10} \\
% {#4}_{#6} & {#4}_{#7} & {#4}_{#8} & {#4}_{#9} & {#4}_{#10} \\
% {#5}_{#6} & {#5}_{#7} & {#5}_{#8} & {#5}_{#9} & {#5}_{#10}
%\end{vmatrix}
%}

% R3 vector.
\newcommand{\VectorThree}[3]{
\begin{bmatrix}
 {#1} \\
 {#2} \\
 {#3}
\end{bmatrix}
}



\author{Peeter Joot}
\email{peeter.joot@gmail.com}


\chapter{Some details for the radial solution of the hydrogen atom.}
\label{chap:hyrdogenLaguerre}
%\useCCL
\blogpage{http://sites.google.com/site/peeterjoot/math2010/hyrdogenLaguerre.pdf}
\date{Nov 24, 2010}
\revisionInfo{hyrdogenLaguerre.tex}

\beginArtWithToc
%\beginArtNoToc

\section{Motivation.}

For the hydrogen atom, after some variable substitutions the radial part of the Schr\"{o}dinger equation takes the form

\begin{equation}\label{eqn:hyrdogenLaguerre:10}
\frac{d^2 R_l}{d\rho^2} + \frac{2}{\rho} \frac{d R_l}{d\rho} + \left( \frac{\lambda}{\rho} - \frac{l(l+1)}{\rho^2} - \inv{4} \right) R_l = 0
\end{equation}

In \cite{desai2009quantum} it is argued that the functions $R_l$ are of the form

\begin{equation}\label{eqn:hyrdogenLaguerre:20}
R_l = \rho^s L(\rho) e^{-\rho/2}
\end{equation}

where $L$ is a polynomial in $\rho$, specifically Laguerre polynomials.  Let's look at some of those details a bit more closely.

\section{Guts}

The first part of the argument comes from considering the $\rho \rightarrow \infty$ case, where Schr\"{o}dinger's equation is approximately

\begin{equation}\label{eqn:hyrdogenLaguerre:10a}
\frac{d^2 R_l}{d\rho^2} - \inv{4} R_l \approx 0.
\end{equation}

This large $\rho$ approximation has solutions $e^{\pm \rho/2}$, and we take the negative sign case as physically meaningful in order for the wave function to be normalizable.

Next it is argued that polynomial multiples of this will also be approximate solutions.  Utilizing monomial multiple of the decreasing exponential as a trial solution, let's compute how this fits into the radial Schr\"{o}dinger's equation \ref{eqn:hyrdogenLaguerre:10} above.  Write

\begin{equation}\label{eqn:hyrdogenLaguerre:15}
R_l = \rho^s e^{-\rho/2}
\end{equation}

The derivatives are

\begin{align*}
R_l' &= \rho^{s-1} \left( s -\frac{\rho}{2}\right) e^{-\rho/2} \\
R_l'' &= 
\rho^{s-2} 
\left( s (s-1) -s \rho +\frac{1}{4} \rho^2 
\right)
e^{-\rho/2}
\end{align*}

and substitution yields
%\rho^{s-2} 
%e^{-\rho/2}
%\left( 
%s (s-1) 
%-s \rho 
%+\frac{1}{4} \rho^2 
%+2s 
%-\rho 
%+\lambda \rho 
%- l(l+1) 
%- \frac{\rho^2}{4} 
%\right)
%&=
\begin{equation}\label{eqn:hyrdogenLaguerre:16}
\rho^{s-2} 
e^{-\rho/2}
\left( 
(s - \rho) (s+1) 
+\lambda \rho 
- l(l+1) 
\right)
\end{equation}

There are two things that this can show.  The first is that for $\rho \rightarrow \infty$ this produces a polynomial with degree $s-2$ and $s-1$ terms multiplied by the exponential, and we have approximately

\begin{equation}\label{eqn:hyrdogenLaguerre:17}
\rho^{s-1} 
e^{-\rho/2}
(\lambda - s - 1)
\end{equation}

The $s-1$ terms will dominate the polynomial, but the exponential dominate all, approaching zero for $\rho \rightarrow \infty$, just as the non-polynomial multiplied $e^{-\rho/2}$ approximate solution will.  This confirms that in the limit this polynomial multiplied exponential still has the desired behavior in the large $\rho$ limit.  Also observe that in the limit of small $\rho$ we have approximately

\begin{equation}\label{eqn:hyrdogenLaguerre:18}
\rho^{s-2} 
e^{-\rho/2}
\left( 
s (s+1) - l(l+1) 
\right)
\end{equation}

Since $\rho^{s-2} \rightarrow \infty$ as $\rho \rightarrow 0$, we require either a different trial solution, or $s=l$ to have a normalizable wavefunction.

Before settling on $s=l$ let's compute the derivatives for a more general trial function, of the form \ref{eqn:hyrdogenLaguerre:20}, and substitute those.  After a bit of computation we find

\begin{equation}\label{eqn:hyrdogenLaguerre:19}
R_l' = \rho^{s-1} e^{-\rho/2} \left( \left( s - \frac{\rho}{2} \right) L + \rho L'
\right)
\end{equation}
\begin{equation}\label{eqn:hyrdogenLaguerre:19b}
R_l'' = \rho^{s-2} e^{-\rho/2} \left( 
\left( s(s-1) - s \rho + \frac{\rho^2}{4} \right) L 
+\left( 2 s \rho -\rho^2 \right) L'
+ \rho^2 L''
\right)
\end{equation}

Putting these together and substitution back into \ref{eqn:hyrdogenLaguerre:10} yields

\begin{equation}\label{eqn:hyrdogenLaguerre:19c}
0 = \rho^{s-2} e^{-\rho/2} \left( 
L \left( (s-\rho)(s+1) + \rho \lambda -l (l+1)
\right) 
+\rho L' \left( 2 (s+1) -\rho \right) 
+ \rho^2 L''
\right)
\end{equation}

In the $\rho \rightarrow 0$ limit where the $\rho^{s-2}$ terms dominate \ref{eqn:hyrdogenLaguerre:50} becomes
\begin{equation}\label{eqn:hyrdogenLaguerre:50}
0 \approx
\rho^{s-2} L \left(
s(s+1) - l(l+1)
\right)
\end{equation}

Again, this provides the $s=l$ or $s = -(l+1)$ possibilities from the text, and we discard $s=-(l+1)$ due to non-normalizability.  A side question.  How does one solve integer equations like this?

\subsection{What remains?}

With $s=l$ killing off the $\rho^{s-2}$ terms, what is our differential equation for $L$?

\begin{equation}\label{eqn:hyrdogenLaguerre:100}
0 = 
\rho L''
+L' \left( 2 (l+1) -\rho \right) 
+L \left( \lambda - (l+1) \right) 
\end{equation}

Comparing this to \cite{wiki:laguerre} we have something pretty close to the stated differential equation for the Laguerre polynomial.  Ours is of the form

\begin{equation}\label{eqn:hyrdogenLaguerre:100}
0 = 
\rho L''
+L' \left( m -\rho \right) 
+L n,
\end{equation}

whereas the wikipedia article has $m=1$ explicitly.  No change of variables involving a scalar multiplicative factor for $\rho$ appears to be able to get it into that form, and I am guessing this is the differential equation for the associated Laguerre polynomial (something not stated in the wikipedia article).

Let's derive the recurrence relations for the coefficients, and work out the first few such polynomials to compare.

\EndArticle
%\EndNoBibArticle

%
% Copyright � 2012 Peeter Joot.  All Rights Reserved.
% Licenced as described in the file LICENSE under the root directory of this GIT repository.
%

%\chapter{Desai Chapter 10 notes and problems}
\label{chap:desaiCh10}
%\blogpage{http://sites.google.com/site/peeterjoot/math2010/desaiCh10.pdf}
%\date{Nov 20, 2010}

%\section{Motivation}
%
%Chapter 10 notes for \citep{desai2009quantum}.
%
%\section{Notes}
%
In \S 10.3 (interaction with a electric field), Green's functions are introduced to solve the first order differential equation

\begin{equation}\label{eqn:desaiCh10:1}
\begin{aligned}
\frac{da}{dt} + i \omega_0 a = - i \omega_0 \lambda(t)
\end{aligned}
\end{equation}

A simpler way is to use the usual trick of assuming that we can take the constant term in the homogeneous solution and allow it to vary with time.

Since our homogeneous solution is of the form

\begin{equation}\label{eqn:desaiCh10:2}
\begin{aligned}
a_H(t) = a_H(0) e^{-i\omega_0 t},
\end{aligned}
\end{equation}

we can look for a specific solution to the forcing term equation of the form

\begin{equation}\label{eqn:desaiCh10:3}
\begin{aligned}
a_S(t) = f(t) e^{-i\omega_0 t}
\end{aligned}
\end{equation}

We get
\begin{equation}\label{eqn:desaiCh10:4}
\begin{aligned}
f' = -i \omega_0 \lambda(t) e^{i \omega_0 t}
\end{aligned}
\end{equation}

which can be integrate directly to find the non-homogeneous solution

\begin{equation}\label{eqn:desaiCh10:5}
\begin{aligned}
a_S(t) = a_S(t_0) e^{-i \omega_0 (t - t_0)} - i \omega_0 \int_{t_0}^t \lambda(t') e^{-i \omega_0 (t-t')} dt'
\end{aligned}
\end{equation}

Setting \(t_0 = -\infty\), with a requirement that \(a_S(-\infty) = 0\) and adding in a general homogeneous solution one then has 10.92 without the complications of Green's functions or the associated contour integrals.  I suppose the author wanted to introduce this as a general purpose tool and this was a simple way to do so.

His introduction of Green's functions this way I did not personally find very clear.  Specifically, he does not actually define what a Green's function is, and the Appendix 20.13 he refers to only discusses the subtleties of the associated Contour integration.  I did not understand where equation 10.83 came from in the first place.

Something like the following would have been helpful (the type of argument found in \citep{wiki:greens})
\index{Green's function}

Given a linear operator \(L\), such that \(L u(x) = f(x)\), we search for the Green's function \(G(x,s)\) such that \(L G(x,s) = \delta(x-s)\).  For such a function we have

\begin{equation}\label{eqn:desaiCh10:25}
\begin{aligned}
\int L G(x,s) f(s) ds 
&= \int \delta(x-s) f(s) ds \\
&= f(x)
\end{aligned}
\end{equation}

and by linearity we also have
\begin{equation}\label{eqn:desaiCh10:45}
\begin{aligned}
f(x) 
&=
\int L G(x,s) f(s) ds \\
&= L \int G(x,s) f(s) ds \\
\end{aligned}
\end{equation}

and can therefore identify \(u(x) = \int G(x,s) f(s) ds\) as the desired solution to \(L u(x) = f(x)\) once the Green's function \(G(x,s)\) associated with operator \(L\) has been determined.


%
% Copyright � 2015 Peeter Joot.  All Rights Reserved.
% Licenced as described in the file LICENSE under the root directory of this GIT repository.
%
\documentclass[]{eliblog}

\usepackage{amsmath}
\usepackage{mathpazo}

%
% shorthand for bold symbols, convenient for vectors and matrices
%
\newcommand{\Ba}[0]{\mathbf{a}}
\newcommand{\Bb}[0]{\mathbf{b}}
\newcommand{\Bc}[0]{\mathbf{c}}
\newcommand{\Bd}[0]{\mathbf{d}}
\newcommand{\Be}[0]{\mathbf{e}}
\newcommand{\Bf}[0]{\mathbf{f}}
\newcommand{\Bg}[0]{\mathbf{g}}
\newcommand{\Bh}[0]{\mathbf{h}}
\newcommand{\Bi}[0]{\mathbf{i}}
\newcommand{\Bj}[0]{\mathbf{j}}
\newcommand{\Bk}[0]{\mathbf{k}}
\newcommand{\Bl}[0]{\mathbf{l}}
\newcommand{\Bm}[0]{\mathbf{m}}
\newcommand{\Bn}[0]{\mathbf{n}}
\newcommand{\Bo}[0]{\mathbf{o}}
\newcommand{\Bp}[0]{\mathbf{p}}
\newcommand{\Bq}[0]{\mathbf{q}}
\newcommand{\Br}[0]{\mathbf{r}}
\newcommand{\Bs}[0]{\mathbf{s}}
\newcommand{\Bt}[0]{\mathbf{t}}
\newcommand{\Bu}[0]{\mathbf{u}}
\newcommand{\Bv}[0]{\mathbf{v}}
\newcommand{\Bw}[0]{\mathbf{w}}
\newcommand{\Bx}[0]{\mathbf{x}}
\newcommand{\By}[0]{\mathbf{y}}
\newcommand{\Bz}[0]{\mathbf{z}}
\newcommand{\BA}[0]{\mathbf{A}}
\newcommand{\BB}[0]{\mathbf{B}}
\newcommand{\BC}[0]{\mathbf{C}}
\newcommand{\BD}[0]{\mathbf{D}}
\newcommand{\BE}[0]{\mathbf{E}}
\newcommand{\BF}[0]{\mathbf{F}}
\newcommand{\BG}[0]{\mathbf{G}}
\newcommand{\BH}[0]{\mathbf{H}}
\newcommand{\BI}[0]{\mathbf{I}}
\newcommand{\BJ}[0]{\mathbf{J}}
\newcommand{\BK}[0]{\mathbf{K}}
\newcommand{\BL}[0]{\mathbf{L}}
\newcommand{\BM}[0]{\mathbf{M}}
\newcommand{\BN}[0]{\mathbf{N}}
\newcommand{\BO}[0]{\mathbf{O}}
\newcommand{\BP}[0]{\mathbf{P}}
\newcommand{\BQ}[0]{\mathbf{Q}}
\newcommand{\BR}[0]{\mathbf{R}}
\newcommand{\BS}[0]{\mathbf{S}}
\newcommand{\BT}[0]{\mathbf{T}}
\newcommand{\BU}[0]{\mathbf{U}}
\newcommand{\BV}[0]{\mathbf{V}}
\newcommand{\BW}[0]{\mathbf{W}}
\newcommand{\BX}[0]{\mathbf{X}}
\newcommand{\BY}[0]{\mathbf{Y}}
\newcommand{\BZ}[0]{\mathbf{Z}}

\newcommand{\Bzero}[0]{\mathbf{0}}
\newcommand{\Btheta}[0]{\boldsymbol{\theta}}
\newcommand{\Btau}[0]{\boldsymbol{\tau}}
\newcommand{\Bomega}[0]{\boldsymbol{\omega}}

%
% shorthand for unit vectors
%
\newcommand{\acap}[0]{\hat{\Ba}}
\newcommand{\bcap}[0]{\hat{\Bb}}
\newcommand{\ccap}[0]{\hat{\Bc}}
\newcommand{\dcap}[0]{\hat{\Bd}}
\newcommand{\ecap}[0]{\hat{\Be}}
\newcommand{\fcap}[0]{\hat{\Bf}}
\newcommand{\gcap}[0]{\hat{\Bg}}
\newcommand{\hcap}[0]{\hat{\Bh}}
\newcommand{\icap}[0]{\hat{\Bi}}
\newcommand{\jcap}[0]{\hat{\Bj}}
\newcommand{\kcap}[0]{\hat{\Bk}}
\newcommand{\lcap}[0]{\hat{\Bl}}
\newcommand{\mcap}[0]{\hat{\Bm}}
\newcommand{\ncap}[0]{\hat{\Bn}}
\newcommand{\ocap}[0]{\hat{\Bo}}
\newcommand{\pcap}[0]{\hat{\Bp}}
\newcommand{\qcap}[0]{\hat{\Bq}}
\newcommand{\rcap}[0]{\hat{\Br}}
\newcommand{\scap}[0]{\hat{\Bs}}
\newcommand{\tcap}[0]{\hat{\Bt}}
\newcommand{\ucap}[0]{\hat{\Bu}}
\newcommand{\vcap}[0]{\hat{\Bv}}
\newcommand{\wcap}[0]{\hat{\Bw}}
\newcommand{\xcap}[0]{\hat{\Bx}}
\newcommand{\ycap}[0]{\hat{\By}}
\newcommand{\zcap}[0]{\hat{\Bz}}
\newcommand{\thetacap}[0]{\hat{\Btheta}}

%
% to write R^n and C^n in a distinguishable fashion.  Perhaps change this
% to the double lined characters upon figuring out how to do so.
%
\newcommand{\C}[1]{$\mathbb{C}^{#1}$}
\newcommand{\R}[1]{$\mathbb{R}^{#1}$}

%
% various generally useful helpers
%

% derivative of #1 wrt. #2:
\newcommand{\D}[2] {\frac {d#2} {d#1}}

\newcommand{\inv}[1]{\frac{1}{#1}}
\newcommand{\cross}[0]{\times}

\newcommand{\abs}[1]{\lvert{#1}\rvert}
\newcommand{\norm}[1]{\lVert{#1}\rVert}
\newcommand{\innerprod}[2]{\langle{#1}, {#2}\rangle}
\newcommand{\dotprod}[2]{{#1} \cdot {#2}}
\newcommand{\bdotprod}[2]{\left({#1} \cdot {#2}\right)}
\newcommand{\crossprod}[2]{{#1} \cross {#2}}
\newcommand{\tripleprod}[3]{\dotprod{\left(\crossprod{#1}{#2}\right)}{#3}}

\DeclareMathOperator{\Proj}{Proj}
\DeclareMathOperator{\Span}{span}
\DeclareMathOperator{\Sgn}{sgn}
\DeclareMathOperator{\Area}{Area}
\DeclareMathOperator{\Volume}{Volume}

%
% A few miscellaneous things specific to this document
%
\newcommand{\crossop}[1]{\crossprod{#1}{}}

% R2 vector.
\newcommand{\VectorTwo}[2]{
\begin{bmatrix}
 {#1} \\
 {#2}
\end{bmatrix}
}

\newcommand{\VectorN}[1]{
\begin{bmatrix}
{#1}_1 \\
{#1}_2 \\
\vdots \\
{#1}_N \\
\end{bmatrix}
}

\newcommand{\DETuvij}[4]{
\begin{vmatrix}
 {#1}_{#3} & {#1}_{#4} \\
 {#2}_{#3} & {#2}_{#4}
\end{vmatrix}
}

\newcommand{\DETuvwijk}[6]{
\begin{vmatrix}
 {#1}_{#4} & {#1}_{#5} & {#1}_{#6} \\
 {#2}_{#4} & {#2}_{#5} & {#2}_{#6} \\
 {#3}_{#4} & {#3}_{#5} & {#3}_{#6}
\end{vmatrix}
}

\newcommand{\DETuvwxijkl}[8]{
\begin{vmatrix}
 {#1}_{#5} & {#1}_{#6} & {#1}_{#7} & {#1}_{#8} \\
 {#2}_{#5} & {#2}_{#6} & {#2}_{#7} & {#2}_{#8} \\
 {#3}_{#5} & {#3}_{#6} & {#3}_{#7} & {#3}_{#8} \\
 {#4}_{#5} & {#4}_{#6} & {#4}_{#7} & {#4}_{#8} \\
\end{vmatrix}
}

%\newcommand{\DETuvwxyijklm}[10]{
%\begin{vmatrix}
% {#1}_{#6} & {#1}_{#7} & {#1}_{#8} & {#1}_{#9} & {#1}_{#10} \\
% {#2}_{#6} & {#2}_{#7} & {#2}_{#8} & {#2}_{#9} & {#2}_{#10} \\
% {#3}_{#6} & {#3}_{#7} & {#3}_{#8} & {#3}_{#9} & {#3}_{#10} \\
% {#4}_{#6} & {#4}_{#7} & {#4}_{#8} & {#4}_{#9} & {#4}_{#10} \\
% {#5}_{#6} & {#5}_{#7} & {#5}_{#8} & {#5}_{#9} & {#5}_{#10}
%\end{vmatrix}
%}

% R3 vector.
\newcommand{\VectorThree}[3]{
\begin{bmatrix}
 {#1} \\
 {#2} \\
 {#3}
\end{bmatrix}
}



\author{Peeter Joot}
\email{peeter.joot@gmail.com}

%\documentclass[]{eliblogwidescreen}

\usepackage{amsmath}
\usepackage{mathpazo}

%
% shorthand for bold symbols, convenient for vectors and matrices
%
\newcommand{\Ba}[0]{\mathbf{a}}
\newcommand{\Bb}[0]{\mathbf{b}}
\newcommand{\Bc}[0]{\mathbf{c}}
\newcommand{\Bd}[0]{\mathbf{d}}
\newcommand{\Be}[0]{\mathbf{e}}
\newcommand{\Bf}[0]{\mathbf{f}}
\newcommand{\Bg}[0]{\mathbf{g}}
\newcommand{\Bh}[0]{\mathbf{h}}
\newcommand{\Bi}[0]{\mathbf{i}}
\newcommand{\Bj}[0]{\mathbf{j}}
\newcommand{\Bk}[0]{\mathbf{k}}
\newcommand{\Bl}[0]{\mathbf{l}}
\newcommand{\Bm}[0]{\mathbf{m}}
\newcommand{\Bn}[0]{\mathbf{n}}
\newcommand{\Bo}[0]{\mathbf{o}}
\newcommand{\Bp}[0]{\mathbf{p}}
\newcommand{\Bq}[0]{\mathbf{q}}
\newcommand{\Br}[0]{\mathbf{r}}
\newcommand{\Bs}[0]{\mathbf{s}}
\newcommand{\Bt}[0]{\mathbf{t}}
\newcommand{\Bu}[0]{\mathbf{u}}
\newcommand{\Bv}[0]{\mathbf{v}}
\newcommand{\Bw}[0]{\mathbf{w}}
\newcommand{\Bx}[0]{\mathbf{x}}
\newcommand{\By}[0]{\mathbf{y}}
\newcommand{\Bz}[0]{\mathbf{z}}
\newcommand{\BA}[0]{\mathbf{A}}
\newcommand{\BB}[0]{\mathbf{B}}
\newcommand{\BC}[0]{\mathbf{C}}
\newcommand{\BD}[0]{\mathbf{D}}
\newcommand{\BE}[0]{\mathbf{E}}
\newcommand{\BF}[0]{\mathbf{F}}
\newcommand{\BG}[0]{\mathbf{G}}
\newcommand{\BH}[0]{\mathbf{H}}
\newcommand{\BI}[0]{\mathbf{I}}
\newcommand{\BJ}[0]{\mathbf{J}}
\newcommand{\BK}[0]{\mathbf{K}}
\newcommand{\BL}[0]{\mathbf{L}}
\newcommand{\BM}[0]{\mathbf{M}}
\newcommand{\BN}[0]{\mathbf{N}}
\newcommand{\BO}[0]{\mathbf{O}}
\newcommand{\BP}[0]{\mathbf{P}}
\newcommand{\BQ}[0]{\mathbf{Q}}
\newcommand{\BR}[0]{\mathbf{R}}
\newcommand{\BS}[0]{\mathbf{S}}
\newcommand{\BT}[0]{\mathbf{T}}
\newcommand{\BU}[0]{\mathbf{U}}
\newcommand{\BV}[0]{\mathbf{V}}
\newcommand{\BW}[0]{\mathbf{W}}
\newcommand{\BX}[0]{\mathbf{X}}
\newcommand{\BY}[0]{\mathbf{Y}}
\newcommand{\BZ}[0]{\mathbf{Z}}

\newcommand{\Bzero}[0]{\mathbf{0}}
\newcommand{\Btheta}[0]{\boldsymbol{\theta}}
\newcommand{\Btau}[0]{\boldsymbol{\tau}}
\newcommand{\Bomega}[0]{\boldsymbol{\omega}}

%
% shorthand for unit vectors
%
\newcommand{\acap}[0]{\hat{\Ba}}
\newcommand{\bcap}[0]{\hat{\Bb}}
\newcommand{\ccap}[0]{\hat{\Bc}}
\newcommand{\dcap}[0]{\hat{\Bd}}
\newcommand{\ecap}[0]{\hat{\Be}}
\newcommand{\fcap}[0]{\hat{\Bf}}
\newcommand{\gcap}[0]{\hat{\Bg}}
\newcommand{\hcap}[0]{\hat{\Bh}}
\newcommand{\icap}[0]{\hat{\Bi}}
\newcommand{\jcap}[0]{\hat{\Bj}}
\newcommand{\kcap}[0]{\hat{\Bk}}
\newcommand{\lcap}[0]{\hat{\Bl}}
\newcommand{\mcap}[0]{\hat{\Bm}}
\newcommand{\ncap}[0]{\hat{\Bn}}
\newcommand{\ocap}[0]{\hat{\Bo}}
\newcommand{\pcap}[0]{\hat{\Bp}}
\newcommand{\qcap}[0]{\hat{\Bq}}
\newcommand{\rcap}[0]{\hat{\Br}}
\newcommand{\scap}[0]{\hat{\Bs}}
\newcommand{\tcap}[0]{\hat{\Bt}}
\newcommand{\ucap}[0]{\hat{\Bu}}
\newcommand{\vcap}[0]{\hat{\Bv}}
\newcommand{\wcap}[0]{\hat{\Bw}}
\newcommand{\xcap}[0]{\hat{\Bx}}
\newcommand{\ycap}[0]{\hat{\By}}
\newcommand{\zcap}[0]{\hat{\Bz}}
\newcommand{\thetacap}[0]{\hat{\Btheta}}

%
% to write R^n and C^n in a distinguishable fashion.  Perhaps change this
% to the double lined characters upon figuring out how to do so.
%
\newcommand{\C}[1]{$\mathbb{C}^{#1}$}
\newcommand{\R}[1]{$\mathbb{R}^{#1}$}

%
% various generally useful helpers
%

% derivative of #1 wrt. #2:
\newcommand{\D}[2] {\frac {d#2} {d#1}}

\newcommand{\inv}[1]{\frac{1}{#1}}
\newcommand{\cross}[0]{\times}

\newcommand{\abs}[1]{\lvert{#1}\rvert}
\newcommand{\norm}[1]{\lVert{#1}\rVert}
\newcommand{\innerprod}[2]{\langle{#1}, {#2}\rangle}
\newcommand{\dotprod}[2]{{#1} \cdot {#2}}
\newcommand{\bdotprod}[2]{\left({#1} \cdot {#2}\right)}
\newcommand{\crossprod}[2]{{#1} \cross {#2}}
\newcommand{\tripleprod}[3]{\dotprod{\left(\crossprod{#1}{#2}\right)}{#3}}

\DeclareMathOperator{\Proj}{Proj}
\DeclareMathOperator{\Span}{span}
\DeclareMathOperator{\Sgn}{sgn}
\DeclareMathOperator{\Area}{Area}
\DeclareMathOperator{\Volume}{Volume}

%
% A few miscellaneous things specific to this document
%
\newcommand{\crossop}[1]{\crossprod{#1}{}}

% R2 vector.
\newcommand{\VectorTwo}[2]{
\begin{bmatrix}
 {#1} \\
 {#2}
\end{bmatrix}
}

\newcommand{\VectorN}[1]{
\begin{bmatrix}
{#1}_1 \\
{#1}_2 \\
\vdots \\
{#1}_N \\
\end{bmatrix}
}

\newcommand{\DETuvij}[4]{
\begin{vmatrix}
 {#1}_{#3} & {#1}_{#4} \\
 {#2}_{#3} & {#2}_{#4}
\end{vmatrix}
}

\newcommand{\DETuvwijk}[6]{
\begin{vmatrix}
 {#1}_{#4} & {#1}_{#5} & {#1}_{#6} \\
 {#2}_{#4} & {#2}_{#5} & {#2}_{#6} \\
 {#3}_{#4} & {#3}_{#5} & {#3}_{#6}
\end{vmatrix}
}

\newcommand{\DETuvwxijkl}[8]{
\begin{vmatrix}
 {#1}_{#5} & {#1}_{#6} & {#1}_{#7} & {#1}_{#8} \\
 {#2}_{#5} & {#2}_{#6} & {#2}_{#7} & {#2}_{#8} \\
 {#3}_{#5} & {#3}_{#6} & {#3}_{#7} & {#3}_{#8} \\
 {#4}_{#5} & {#4}_{#6} & {#4}_{#7} & {#4}_{#8} \\
\end{vmatrix}
}

%\newcommand{\DETuvwxyijklm}[10]{
%\begin{vmatrix}
% {#1}_{#6} & {#1}_{#7} & {#1}_{#8} & {#1}_{#9} & {#1}_{#10} \\
% {#2}_{#6} & {#2}_{#7} & {#2}_{#8} & {#2}_{#9} & {#2}_{#10} \\
% {#3}_{#6} & {#3}_{#7} & {#3}_{#8} & {#3}_{#9} & {#3}_{#10} \\
% {#4}_{#6} & {#4}_{#7} & {#4}_{#8} & {#4}_{#9} & {#4}_{#10} \\
% {#5}_{#6} & {#5}_{#7} & {#5}_{#8} & {#5}_{#9} & {#5}_{#10}
%\end{vmatrix}
%}

% R3 vector.
\newcommand{\VectorThree}[3]{
\begin{bmatrix}
 {#1} \\
 {#2} \\
 {#3}
\end{bmatrix}
}



\author{Peeter Joot}
\email{peeter.joot@gmail.com}


\chapter{Notes and problems for Desai Chapter 26.}
\label{chap:desaiCh26}
%\useCCL
\blogpage{http://sites.google.com/site/peeterjoot/math2010/desaiCh26.pdf}
\date{Dec 9, 2010}
\revisionInfo{desaiCh26.tex}

\beginArtWithToc
%\beginArtNoToc

\section{Motivation.}

Chapter 26 notes for \cite{desai2009quantum}.

\section{Guts}

\subsection{Trig relations.}

To verify equations 26.3-5 in the text it's worth noting that 

\begin{align*}
\cos(a + b) 
&= \Re( e^{ia} e^{ib} ) \\
&= \Re( (\cos a + i \sin a)( \cos b + i \sin b) ) \\
&= \cos a \cos b - \sin a \sin b
\end{align*}

and
\begin{align*}
\sin(a + b) 
&= \Im( e^{ia} e^{ib} ) \\
&= \Im( (\cos a + i \sin a)( \cos b + i \sin b) ) \\
&= \cos a \sin b + \sin a \cos b
\end{align*}

So, for 
\begin{align}\label{eqn:desaiCh26:10}
x &= \rho \cos\alpha \\
y &= \rho \sin\alpha 
\end{align}

the transformed coordinates are
\begin{align*}
x' 
&= \rho \cos(\alpha + \phi) \\
&= \rho (\cos \alpha \cos \phi - \sin \alpha \sin \phi) \\
&= x \cos \phi - y \sin \phi
\end{align*}

and
\begin{align*}
y' 
&= \rho \sin(\alpha + \phi) \\
&= \rho (\cos \alpha \sin \phi + \sin \alpha \cos \phi) \\
&= x \sin \phi + y \cos \phi \\
\end{align*}

This allows us to read off the rotation matrix.  Without all the messy trig, we can also derive this matrix with geometric algebra.

\begin{align*}
\Bv' 
&= e^{- \Be_1 \Be_2 \phi/2 } \Bv e^{ \Be_1 \Be_2 \phi/2 } \\
&= v_3 \Be_3 + (v_1 \Be_1 + v_2 \Be_2) e^{ \Be_1 \Be_2 \phi } \\
&= v_3 \Be_3 + (v_1 \Be_1 + v_2 \Be_2) (\cos \phi + \Be_1 \Be_2 \sin\phi) \\
&= v_3 \Be_3 
+ \Be_1 (v_1 \cos\phi - v_2 \sin\phi)
+ \Be_2 (v_2 \cos\phi + v_1 \sin\phi)
\end{align*}

Here we use the Pauli-matrix like identities

\begin{align}\label{eqn:desaiCh26:20}
\Be_k^2 &= 1 \\
\Be_i \Be_j &= -\Be_j \Be_i,\quad i\ne j
\end{align}

and also note that $\Be_3$ commutes with the bivector for the $x,y$ plane $\Be_1 \Be_2$.  We can also read off the rotation matrix from this.

\subsection{Infinitesimal transformations.}

Recall that in the problems of Chapter 5, one representation of spin one matrices were calculated \chapcite{desaiCh5}.  Since the choice of the basis vectors was arbitrary in that exersize, we ended up with a different representation.  For $S_x, S_y, S_z$ as found in (26.20) and (26.23) we can also verify easily that we have eigenvalues $0, \pm \hbar$.  We can also show that our spin kets in this non-diagonal representation have the following column matrix representations:

\begin{align}\label{eqn:desaiCh26:30}
\ket{1,\pm 1}_x 
&=
\inv{\sqrt{2}} \begin{bmatrix}
0 \\
1 \\
\pm i
\end{bmatrix} \\
\ket{1,0}_x 
&=
\begin{bmatrix}
1 \\
0 \\
0 
\end{bmatrix} \\
\ket{1,\pm 1}_y 
&=
\inv{\sqrt{2}} \begin{bmatrix}
\pm i \\
0 \\
1 
\end{bmatrix} \\
\ket{1,0}_y 
&=
\begin{bmatrix}
0 \\
1 \\
0 
\end{bmatrix} \\
\ket{1,\pm 1}_z 
&=
\inv{\sqrt{2}} \begin{bmatrix}
1 \\
\pm i \\
0
\end{bmatrix} \\
\ket{1,0}_z 
&=
\begin{bmatrix}
0 \\
0 \\
1
\end{bmatrix} 
\end{align}

\subsection{Verifying the commutator relations.}

Given the (summation convention) matrix representation for the spin one operators

\begin{equation}\label{eqn:desaiCh26:40}
(S_i)_{jk} = - i \hbar \epsilon_{ijk},
\end{equation}

let's demonstrate the commutator relation of (26.25).

\begin{align*}
{\antisymmetric{S_i}{S_j}}_{rs} 
&=
(S_i S_j - S_j S_i)_{rs} \\
&=
\sum_t (S_i)_{rt} (S_j)_{ts} - (S_j)_{rt} (S_i)_{ts} \\
&=
(-i\hbar)^2 \sum_t \epsilon_{irt} \epsilon_{jts} - \epsilon_{jrt} \epsilon_{its} \\
&=
-(-i\hbar)^2 \sum_t \epsilon_{tir} \epsilon_{tjs} - \epsilon_{tjr} \epsilon_{tis} \\
\end{align*}

Now we can employ the summation rule for sums products of antisymmetic tensors over one free index (4.179) 

\begin{equation}\label{eqn:desaiCh26:50}
\sum_i 
\epsilon_{ijk} \epsilon_{iab}
= 
\delta_{ja}
\delta_{kb}
-\delta_{jb}
\delta_{ka}.
\end{equation}

Continuing we get
\begin{align*}
{\antisymmetric{S_i}{S_j}}_{rs} 
&=
-(-i\hbar)^2 \left(
\delta_{ij}
\delta_{rs}
-\delta_{is}
\delta_{rj}
-\delta_{ji}
\delta_{rs}
+\delta_{js}
\delta_{ri} \right) \\
&=
(-i\hbar)^2 \left( 
\delta_{is}
\delta_{jr}
-
\delta_{ir} 
\delta_{js}
\right)
\\
&=
(-i\hbar)^2 \sum_t \epsilon_{tij} \epsilon_{tsr}
\\
&=
i\hbar \sum_t \epsilon_{tij} (S_t)_{rs}
\qquad\square
\end{align*}

\subsection{General infinitesimal rotation.}

Equation (26.26) has for an infinitesimal rotation counterclockwise around the unit axis of rotation vector $\Bn$

\begin{equation}\label{eqn:desaiCh26:60}
\BV' = \BV + \epsilon \Bn \cross \BV.
\end{equation}

Let's derive this using the geometric algebra rotation expression for the same

\begin{align*}
\BV' 
&=
e^{-I\Bn \alpha/2}
\BV 
e^{I\Bn \alpha/2} \\
&=
e^{-I\Bn \alpha/2}
\left(
(\BV \cdot \Bn)\Bn
+(\BV \wedge \Bn)\Bn
\right)
e^{I\Bn \alpha/2} \\
&=
(\BV \cdot \Bn)\Bn
+(\BV \wedge \Bn)\Bn
e^{I\Bn \alpha}
\end{align*}

We note that $I\Bn$ and thus the exponential commutes with $\Bn$, and the projection component in the normal direction.  Similarily $I\Bn$ anticommutes with $(\BV \wedge \Bn) \Bn$.  This leaves us with

\begin{align*}
\BV' 
&=
(\BV \cdot \Bn)\Bn
\left(
+(\BV \wedge \Bn)\Bn
\right)
( \cos \alpha + I \Bn \sin\alpha)
\end{align*}

For $\alpha = \epsilon \rightarrow 0$, this is

\begin{align*}
\BV' 
&=
(\BV \cdot \Bn)\Bn
+(\BV \wedge \Bn)\Bn
( 1 + I \Bn \epsilon) \\
&=
(\BV \cdot \Bn)\Bn 
+(\BV \wedge \Bn)\Bn
+\epsilon I^2(\BV \cross \Bn)\Bn^2 \\
&=
\BV
+ \epsilon (\Bn \cross \BV) \qquad\square
\end{align*}

\EndArticle

%
% Copyright � 2015 Peeter Joot.  All Rights Reserved.
% Licenced as described in the file LICENSE under the root directory of this GIT repository.
%
\documentclass[]{eliblog}

\usepackage{amsmath}
\usepackage{mathpazo}

%
% shorthand for bold symbols, convenient for vectors and matrices
%
\newcommand{\Ba}[0]{\mathbf{a}}
\newcommand{\Bb}[0]{\mathbf{b}}
\newcommand{\Bc}[0]{\mathbf{c}}
\newcommand{\Bd}[0]{\mathbf{d}}
\newcommand{\Be}[0]{\mathbf{e}}
\newcommand{\Bf}[0]{\mathbf{f}}
\newcommand{\Bg}[0]{\mathbf{g}}
\newcommand{\Bh}[0]{\mathbf{h}}
\newcommand{\Bi}[0]{\mathbf{i}}
\newcommand{\Bj}[0]{\mathbf{j}}
\newcommand{\Bk}[0]{\mathbf{k}}
\newcommand{\Bl}[0]{\mathbf{l}}
\newcommand{\Bm}[0]{\mathbf{m}}
\newcommand{\Bn}[0]{\mathbf{n}}
\newcommand{\Bo}[0]{\mathbf{o}}
\newcommand{\Bp}[0]{\mathbf{p}}
\newcommand{\Bq}[0]{\mathbf{q}}
\newcommand{\Br}[0]{\mathbf{r}}
\newcommand{\Bs}[0]{\mathbf{s}}
\newcommand{\Bt}[0]{\mathbf{t}}
\newcommand{\Bu}[0]{\mathbf{u}}
\newcommand{\Bv}[0]{\mathbf{v}}
\newcommand{\Bw}[0]{\mathbf{w}}
\newcommand{\Bx}[0]{\mathbf{x}}
\newcommand{\By}[0]{\mathbf{y}}
\newcommand{\Bz}[0]{\mathbf{z}}
\newcommand{\BA}[0]{\mathbf{A}}
\newcommand{\BB}[0]{\mathbf{B}}
\newcommand{\BC}[0]{\mathbf{C}}
\newcommand{\BD}[0]{\mathbf{D}}
\newcommand{\BE}[0]{\mathbf{E}}
\newcommand{\BF}[0]{\mathbf{F}}
\newcommand{\BG}[0]{\mathbf{G}}
\newcommand{\BH}[0]{\mathbf{H}}
\newcommand{\BI}[0]{\mathbf{I}}
\newcommand{\BJ}[0]{\mathbf{J}}
\newcommand{\BK}[0]{\mathbf{K}}
\newcommand{\BL}[0]{\mathbf{L}}
\newcommand{\BM}[0]{\mathbf{M}}
\newcommand{\BN}[0]{\mathbf{N}}
\newcommand{\BO}[0]{\mathbf{O}}
\newcommand{\BP}[0]{\mathbf{P}}
\newcommand{\BQ}[0]{\mathbf{Q}}
\newcommand{\BR}[0]{\mathbf{R}}
\newcommand{\BS}[0]{\mathbf{S}}
\newcommand{\BT}[0]{\mathbf{T}}
\newcommand{\BU}[0]{\mathbf{U}}
\newcommand{\BV}[0]{\mathbf{V}}
\newcommand{\BW}[0]{\mathbf{W}}
\newcommand{\BX}[0]{\mathbf{X}}
\newcommand{\BY}[0]{\mathbf{Y}}
\newcommand{\BZ}[0]{\mathbf{Z}}

\newcommand{\Bzero}[0]{\mathbf{0}}
\newcommand{\Btheta}[0]{\boldsymbol{\theta}}
\newcommand{\Btau}[0]{\boldsymbol{\tau}}
\newcommand{\Bomega}[0]{\boldsymbol{\omega}}

%
% shorthand for unit vectors
%
\newcommand{\acap}[0]{\hat{\Ba}}
\newcommand{\bcap}[0]{\hat{\Bb}}
\newcommand{\ccap}[0]{\hat{\Bc}}
\newcommand{\dcap}[0]{\hat{\Bd}}
\newcommand{\ecap}[0]{\hat{\Be}}
\newcommand{\fcap}[0]{\hat{\Bf}}
\newcommand{\gcap}[0]{\hat{\Bg}}
\newcommand{\hcap}[0]{\hat{\Bh}}
\newcommand{\icap}[0]{\hat{\Bi}}
\newcommand{\jcap}[0]{\hat{\Bj}}
\newcommand{\kcap}[0]{\hat{\Bk}}
\newcommand{\lcap}[0]{\hat{\Bl}}
\newcommand{\mcap}[0]{\hat{\Bm}}
\newcommand{\ncap}[0]{\hat{\Bn}}
\newcommand{\ocap}[0]{\hat{\Bo}}
\newcommand{\pcap}[0]{\hat{\Bp}}
\newcommand{\qcap}[0]{\hat{\Bq}}
\newcommand{\rcap}[0]{\hat{\Br}}
\newcommand{\scap}[0]{\hat{\Bs}}
\newcommand{\tcap}[0]{\hat{\Bt}}
\newcommand{\ucap}[0]{\hat{\Bu}}
\newcommand{\vcap}[0]{\hat{\Bv}}
\newcommand{\wcap}[0]{\hat{\Bw}}
\newcommand{\xcap}[0]{\hat{\Bx}}
\newcommand{\ycap}[0]{\hat{\By}}
\newcommand{\zcap}[0]{\hat{\Bz}}
\newcommand{\thetacap}[0]{\hat{\Btheta}}

%
% to write R^n and C^n in a distinguishable fashion.  Perhaps change this
% to the double lined characters upon figuring out how to do so.
%
\newcommand{\C}[1]{$\mathbb{C}^{#1}$}
\newcommand{\R}[1]{$\mathbb{R}^{#1}$}

%
% various generally useful helpers
%

% derivative of #1 wrt. #2:
\newcommand{\D}[2] {\frac {d#2} {d#1}}

\newcommand{\inv}[1]{\frac{1}{#1}}
\newcommand{\cross}[0]{\times}

\newcommand{\abs}[1]{\lvert{#1}\rvert}
\newcommand{\norm}[1]{\lVert{#1}\rVert}
\newcommand{\innerprod}[2]{\langle{#1}, {#2}\rangle}
\newcommand{\dotprod}[2]{{#1} \cdot {#2}}
\newcommand{\bdotprod}[2]{\left({#1} \cdot {#2}\right)}
\newcommand{\crossprod}[2]{{#1} \cross {#2}}
\newcommand{\tripleprod}[3]{\dotprod{\left(\crossprod{#1}{#2}\right)}{#3}}

\DeclareMathOperator{\Proj}{Proj}
\DeclareMathOperator{\Span}{span}
\DeclareMathOperator{\Sgn}{sgn}
\DeclareMathOperator{\Area}{Area}
\DeclareMathOperator{\Volume}{Volume}

%
% A few miscellaneous things specific to this document
%
\newcommand{\crossop}[1]{\crossprod{#1}{}}

% R2 vector.
\newcommand{\VectorTwo}[2]{
\begin{bmatrix}
 {#1} \\
 {#2}
\end{bmatrix}
}

\newcommand{\VectorN}[1]{
\begin{bmatrix}
{#1}_1 \\
{#1}_2 \\
\vdots \\
{#1}_N \\
\end{bmatrix}
}

\newcommand{\DETuvij}[4]{
\begin{vmatrix}
 {#1}_{#3} & {#1}_{#4} \\
 {#2}_{#3} & {#2}_{#4}
\end{vmatrix}
}

\newcommand{\DETuvwijk}[6]{
\begin{vmatrix}
 {#1}_{#4} & {#1}_{#5} & {#1}_{#6} \\
 {#2}_{#4} & {#2}_{#5} & {#2}_{#6} \\
 {#3}_{#4} & {#3}_{#5} & {#3}_{#6}
\end{vmatrix}
}

\newcommand{\DETuvwxijkl}[8]{
\begin{vmatrix}
 {#1}_{#5} & {#1}_{#6} & {#1}_{#7} & {#1}_{#8} \\
 {#2}_{#5} & {#2}_{#6} & {#2}_{#7} & {#2}_{#8} \\
 {#3}_{#5} & {#3}_{#6} & {#3}_{#7} & {#3}_{#8} \\
 {#4}_{#5} & {#4}_{#6} & {#4}_{#7} & {#4}_{#8} \\
\end{vmatrix}
}

%\newcommand{\DETuvwxyijklm}[10]{
%\begin{vmatrix}
% {#1}_{#6} & {#1}_{#7} & {#1}_{#8} & {#1}_{#9} & {#1}_{#10} \\
% {#2}_{#6} & {#2}_{#7} & {#2}_{#8} & {#2}_{#9} & {#2}_{#10} \\
% {#3}_{#6} & {#3}_{#7} & {#3}_{#8} & {#3}_{#9} & {#3}_{#10} \\
% {#4}_{#6} & {#4}_{#7} & {#4}_{#8} & {#4}_{#9} & {#4}_{#10} \\
% {#5}_{#6} & {#5}_{#7} & {#5}_{#8} & {#5}_{#9} & {#5}_{#10}
%\end{vmatrix}
%}

% R3 vector.
\newcommand{\VectorThree}[3]{
\begin{bmatrix}
 {#1} \\
 {#2} \\
 {#3}
\end{bmatrix}
}



\author{Peeter Joot}
\email{peeter.joot@gmail.com}


%\label{chap:desaiTypos}
%\useCCL
%\blogpage{http://sites.google.com/site/peeterjoot/math2010/desaiTypos.pdf}
%\date{Oct 31, 2010}
%\revisionInfo{desaiTypos.tex}

\chapter{Believed to be typos in Desai's QM Text}

\beginArtNoToc

Vatche says that the root cause of what I've identified as a typo is in some cases incorrect, and that he's going through the text carefully himself too.  Go through his comments and see if there's stuff that I need to followup on to understand why we differed.

\section{Chapter I.}

\begin{itemize}
\item Page 1.  Vatche: The Hermitian, not complex conjugate, of $\bra{}$ is $\ket{}$.
\item Page 7.  Text before (1.43).  $\alpha$ instead of $a$ used.
\item Page 19.  Equation (1.122).  $\dagger$s omitted after first equality.
\end{itemize}

\section{Chapter II.}
\begin{itemize}
\item Page 40.  Text before (2.137).  Reference to equation (2.133) should be (2.135)
\item Page 53.  Is the "Also show that" here correct?  I get a different answer.
\end{itemize}

\section{Chapter III.}
\begin{itemize}
\item Page 61.  Equation (3.51).  $1/\hbar$ missing.
\item Page 62.  Equation (3.58).  Vatche: Remove the $U_I$ operators from Eq. (3.58)
\item Page 66.  Equation (3.92).  $-(d/dt \bra{\alpha}) \ket{\alpha}$ should be $+\ket{\alpha} d/dt \bra{\alpha}$.
\item Page 66.  Equation (3.93).  $H$ on wrong side of $\bra{\alpha}$ 
\item Page 74,76.  Vatche: remove the extra brackets from Eq (4.9) and (4.21).
\item Page 79.  Vatche: "The probability of finding this particle" should read "The probability density for this state at point x is"
\end{itemize}

\section{Chapter IV.}
\begin{itemize}
\item Page 81.  Equation (4.52).  Should be $-2\alpha$ in the exponent.  Vatche says instead: Replease the $k$ with $\alpha$ in the exponent.
\item Page 82.  Equation (4.67).  Vatche says: the negative sign should appear inside the large square brackets.
\item Page 82.  Equation (4.67).  $2\alpha$ in the denominator of the normalization should be $\alpha'/\pi$.
\item Page 83.  Equation (4.74).  A normalized wave function isn't required for the discussion, but if that was intended, a $1/\sqrt{2\pi}$ factor is missing.
\item Page 83-84.  Vatche says: In (4.67) and (4.77), the derivative should be evaluated at $k=k_0$.
\item Page 86.  Equation (4.99).  Extra brace in the exponent.
\item Page 87.  Equation (4.106).  Extra brace in the exponent.
\item Page 89.  Equation (4.124).  Vatche says: Change to $Q(\phi) = C exp(i \sqrt{\mu} \phi)$ and in (4.126) to $\sqrt{\mu} = m$.
\item Page 89.  Equation (4.129), (4.130).  $\lambda - m^2/...$ should be $\lambda + ...$
\item Page 92.  Equation (4.158).  Vatche says: should read $P_l(1) = 1$.
\item Page 93.  Equation (4.169).  conjugation missing for $Y_{lm}$.  $Y_{l'm'}$ is missing prime on the $l$ index.
\item Page 95.  Second line of text.  Language choice?  "We now implement".  perhaps utilize would be better?
\item Page 95.  Text before (4.193).  $i$ is in bold.
\item Page 96.  Text before (4.196).  $i$ is in bold.
\item Page 97.  (4.205).  $i$ is in bold.
\item Page 97.  (4.207-209).  $\Bi$, and $\Bj$s aren't in bold like $\Bk$
\item Page 101.  (4.245).  The right side should read $Y_{l,m+1}$
\item Page 101.  (4.239-240).  The approach here is unclear.  FIXME: incorporate lecture notes from class that did this using braket notation.
\item Page 102.  (4.248-249).  Commas missing to separate $l$, and $m\pm 1$ in the kets.
\end{itemize}

\section{Chapter V.}
\begin{itemize}
\item Page 113.  (5.86). One $\sigma$ isn't in bold.
\item Page 114.  (5.100). $\chi$ is in bold.
\item Page 115.  Text before (5.106). $\alpha$ in bold.
\item Page 118.  Switch of notation in problem 5 for ensemble averages.  $[S_i]$ used instead of $\expectation{S_i}_{\text{av}}$.
\end{itemize}

\section{Chapter VI.}
\begin{itemize}
\item ... (as mentioned previously)
\item Page 131.  Problem 1.  bold missing on $\BE$.
\end{itemize}

\section{Chapter 8.}
\begin{itemize}
\item Page 143.  (8.58).  $\beta$ should be negated.
\item Page 159.  (8.6.3).  Two references to Chapter 2 should be Chapter 4.
\item Page 160.  (8.199).  Want $\hbar^2$ not $\hbar$ in expression for $k$.
\item Page 162.  (Fig 8.9).  Figure is backwards compared to text (a bump instead of a well).
\item Page 165.  (8.235).  Extra $R_l$ factor inside parens.
\end{itemize}

\section{Chapter 9.}
\begin{itemize}
\item Page 174.  (9.5).  Have $\hbar/2m\omega$ instead of $\hbar m \omega/2$ in expression for $P$.
\item Page 181.  (9.57).  Factor of two missing.  Want $\frac{\alpha}{2 \sqrt{\pi}}$.
\item Page 186.  (Problem 10).  Sequencing the text and problems is off.  The green's function technique isn't introduced until chapter 10.
\end{itemize}

\section{Chapter 10.}
\begin{itemize}
\item Page 189.  (10.22).  It would be nice to have a reference to the appendix (ie: 10.100) for the chapter so that this identity isn't pulled out of a magic hat.
\item Page 192.  (10.44, 10.45).  $2 \alpha {\alpha^\conj}'$ should be $\alpha {\alpha^\conj}' + \alpha' \alpha^\conj$
\item Page 193.  (10.51).  Application (slowly, step by step explicitly) of 10.100 to expand the $e^{\frac{i}{\hbar}(p_0 X - x_0 P)}$ in the braket gives

\begin{align*}
\bra{x} e^{\frac{i}{\hbar}(p_0 X - x_0 P)} \ket{0}
&=
\bra{x} e^{\frac{i}{\hbar}p_0 X }
e^{-\frac{i}{\hbar}x_0 P}
e^{-\frac{i}{2\hbar}x_0 p_0 \antisymmetric{X}{P}}
\ket{0} \\
&=
\bra{x} e^{\frac{i}{\hbar}p_0 X }
e^{-\frac{i}{\hbar}x_0 P}
e^{\frac{x_0 p_0}{2} }
\ket{0} \\
&=
e^{\frac{x_0 p_0}{2} }
\bra{x} e^{\frac{i}{\hbar}p_0 X} 
e^{-\frac{i}{\hbar}x_0 P}
\ket{0} \\
&=
e^{\frac{x_0 p_0}{2} }
\left(
\bra{0} 
e^{\frac{i}{\hbar}x_0 P}
e^{-\frac{i}{\hbar}p_0 X} 
\ket{x}\right)^\conj \\
&=
e^{\frac{x_0 p_0}{2} }
\left(
\bra{0} 
e^{\frac{i}{\hbar}x_0 P}
\ket{x}
e^{-\frac{i}{\hbar}p_0 x} 
\right)^\conj \\
&=
e^{\frac{x_0 p_0}{2} } e^{\frac{i}{\hbar}p_0 x} 
\left(
\bra{0} 
e^{\frac{i}{\hbar}x_0 P}
\ket{x}
\right)^\conj \\
&=
e^{\frac{x_0 p_0}{2} } e^{\frac{i}{\hbar}p_0 x} 
\left(
\braket{0}{x - x_0}
\right)^\conj \\
&=
e^{\frac{x_0 p_0}{2} } e^{\frac{i}{\hbar}p_0 x} 
\braket{x - x_0}{0} \\
&=
e^{\frac{x_0 p_0}{2} } e^{\frac{i}{\hbar}p_0 x} 
\psi_0(x - x_0, 0)
\end{align*}

This is the same as (10.51) with the exception of a real scalar constant $e^{ x_0 p_0/2}$ multiplying the wave function.  Because of this I think that (10.51) should be a proportionality statement, and not an equality as in

\begin{align*}
\bra{x} e^{\frac{i}{\hbar}(p_0 X - x_0 P)} \ket{0} \propto
e^{\frac{i}{\hbar}p_0 x} \psi_0(x - x_0, 0)
\end{align*}

(ie: building this additional factor into the wave function normalization instead).
\item Page 196.  (text after 10.76).  Looks like reference to Chapter 9, should be Chapter 9 problem 5.

\item Page 197.  (text after 10.85).  Reference to Chapter 1 should be Chapter 2.

\end{itemize}

\EndArticle

%
% Copyright � 2012 Peeter Joot.  All Rights Reserved.
% Licenced as described in the file LICENSE under the root directory of this GIT repository.
%

%\chapter{Unitary exponential sandwich}
\label{chap:exponentialSandwichCommutator}
%\blogpage{http://sites.google.com/site/peeterjoot/math2010/exponentialSandwichCommutator.pdf}
%\date{Sept 27, 2010}

%\section{Motivation}

One of the chapter II exercises in \citep{desai2009quantum} involves a commutator exponential sandwich of the form

\begin{equation}\label{eqn:exponentialSandwichCommutator:1}
e^{i F} B e^{-iF}
\end{equation}

where \(F\) is Hermitian.  Asking about commutators on physicsforums I was told that such sandwiches (my term) preserve expectation values, and also have a Taylor series like expansion involving the repeated commutators.  Let us derive the commutator relationship.

%\section{Guts}

Let us expand a sandwich of this form in series, and shuffle the summation order so that we sum over all the index plane diagonals \(k + m = \text{constant}\).  That is

\begin{equation}\label{eqn:exponentialSandwichCommutator:24}
\begin{aligned}
e^{A} B e^{-A}
&=
\sum_{k,m=0}^\infty \inv{k!m!} A^k B (-A)^m \\
&=
\sum_{r=0}^\infty \sum_{m=0}^r \inv{(r-m)!m!} A^{r-m} B (-A)^m \\
&=
\sum_{r=0}^\infty \inv{r!} \sum_{m=0}^r \frac{r!}{(r-m)!m!} A^{r-m} B (-A)^m \\
&=
\sum_{r=0}^\infty \inv{r!} \sum_{m=0}^r \binom{r}{m} A^{r-m} B (-A)^m.
\end{aligned}
\end{equation}

Assuming that these interior sums can be written as commutators, we will shortly have an induction exercise.  Let us write these out for a couple values of \(r\) to get a feel for things.

\begin{itemize}
\item \(r=1\)

\begin{equation}\label{eqn:exponentialSandwichCommutator:44}
\binom{1}{0} A B + \binom{1}{1} B (-A) = \antisymmetric{A}{B}
\end{equation}

\item \(r=2\)

\begin{equation}\label{eqn:exponentialSandwichCommutator:64}
\binom{2}{0} A^2 B + \binom{2}{1} A B (-A) + \binom{2}{2} B (-A)^2 = A^2 B - 2 A B A + B A
\end{equation}

This compares exactly to the double commutator:
\begin{equation}\label{eqn:exponentialSandwichCommutator:84}
\begin{aligned}
\antisymmetric{A}{\antisymmetric{A}{B}}
&= 
A(A B - B A) -(A B - B A)A \\
&= 
A^2 B - A B A - A B A + B A^2 \\
&= 
A^2 B - 2 A B A + B A^2 \\
\end{aligned}
\end{equation}

\item \(r=3\)

\begin{equation}\label{eqn:exponentialSandwichCommutator:104}
\begin{aligned}
\binom{3}{0} A^3 B + \binom{3}{1} A^2 B (-A) + \binom{3}{2} A B (-A)^2 + \binom{3}{3} B (-A)^3 
&= 
A^3 B - 3 A^2 B A + 3 A B A^2 - B A^3.
\end{aligned}
\end{equation}

And this compares exactly to the triple commutator

\begin{equation}\label{eqn:exponentialSandwichCommutator:124}
\begin{aligned}
\antisymmetric{A}{\antisymmetric{A}{\antisymmetric{A}{B}}}
&=
A^3 B - 2 A^2 B A + A B A^2 -(A^2 B A - 2 A B A^2 + B A^3) \\
&=
A^3 B - 3 A^2 B A + 3 A B A^2 -B A^3 \\
\end{aligned}
\end{equation}
\end{itemize}

The induction pattern is clear.  Let us write the \(r\) fold commutator as

\begin{equation}\label{eqn:exponentialSandwichCommutator:2}
C_r(A,B) \equiv 
\mathLabelBox{[A, [A, \cdots, [A,}{\(r\) times}
B]] \cdots ] 
= \sum_{m=0}^r \binom{r}{m} A^{r-m} B (-A)^m,
\end{equation}

and calculate this for the \(r+1\) case to verify the induction hypothesis.  We have

\begin{equation}\label{eqn:exponentialSandwichCommutator:144}
\begin{aligned}
C_{r+1}(A,B) 
&= \sum_{m=0}^r \binom{r}{m} 
\left( A^{r-m+1} B (-A)^m
-A^{r-m} B (-A)^{m} A \right) \\
&= \sum_{m=0}^r \binom{r}{m} 
\left( A^{r-m+1} B (-A)^m
+A^{r-m} B (-A)^{m+1} \right) \\
&= 
A^{r+1} B
+ \sum_{m=1}^r \binom{r}{m} 
A^{r-m+1} B (-A)^m
+ \sum_{m=0}^{r-1} \binom{r}{m} 
A^{r-m} B (-A)^{m+1} 
+ B (-A)^{r+1} \\
&= 
A^{r+1} B
% k=m-1
% m = k+1
+ \sum_{k=0}^{r-1} \binom{r}{k+1} 
A^{r-k} B (-A)^{k+1}
+ \sum_{m=0}^{r-1} \binom{r}{m} 
A^{r-m} B (-A)^{m+1} 
+ B (-A)^{r+1} \\
&= 
A^{r+1} B
+ \sum_{k=0}^{r-1} \left( \binom{r}{k+1} + \binom{r}{k} \right) A^{r-k} B (-A)^{k+1}
+ B (-A)^{r+1} \\
\end{aligned}
\end{equation}

We now have to sum those binomial coefficients.  I like the search and replace technique for this, picking two visibly distinct numbers for \(r\), and \(k\) that are easy to manipulate without abstract confusion.  How about \(r=7\), and \(k=3\).  Using those we have

\begin{equation}\label{eqn:exponentialSandwichCommutator:164}
\begin{aligned}
\binom{7}{3+1} + \binom{7}{3} 
&=
\frac{7!}{(3+1)!(7-3-1)!}
+\frac{7!}{3!(7-3)!} \\
&=
\frac{7!(7-3)}{(3+1)!(7-3)!}
+\frac{7!(3+1)}{(3+1)!(7-3)!} \\
&=
\frac{7! \left( 7-3 + 3 + 1 \right) }{(3+1)!(7-3)!} \\
&=
\frac{(7+1)! }{(3+1)!((7+1)-(3+1))!}.
\end{aligned}
\end{equation}

Straight text replacement of \(7\) and \(3\) with \(r\) and \(k\) respectively now gives the harder to follow, but more general identity

\begin{equation}\label{eqn:exponentialSandwichCommutator:184}
\begin{aligned}
\binom{r}{k+1} + \binom{r}{k} 
&=
\frac{r!}{(k+1)!(r-k-1)!}
+\frac{r!}{k!(r-k)!} \\
&=
\frac{r!(r-k)}{(k+1)!(r-k)!}
+\frac{r!(k+1)}{(k+1)!(r-k)!} \\
&=
\frac{r! \left( r-k + k + 1 \right) }{(k+1)!(r-k)!} \\
&=
\frac{(r+1)! }{(k+1)!((r+1)-(k+1))!} \\
&=
\binom{r+1}{k+1}
\end{aligned}
\end{equation}

For our commutator we now have

\begin{equation}\label{eqn:exponentialSandwichCommutator:204}
\begin{aligned}
C_{r+1}(A,B) 
&= 
A^{r+1} B
+ \sum_{k=0}^{r-1} \binom{r+1}{k+1} A^{r-k} B (-A)^{k+1} 
+ B (-A)^{r+1} \\
&= 
A^{r+1} B
% k+1=s
% k=s-1
+ \sum_{s=1}^{r} \binom{r+1}{s} A^{r+1-s} B (-A)^{s} 
+ B (-A)^{r+1} \\
&= \sum_{s=0}^{r+1} \binom{r+1}{s} A^{r+1-s} B (-A)^{s} 
\qedmarker
\end{aligned}
\end{equation}

That completes the inductive proof and allows us to write 

\begin{equation}\label{eqn:exponentialSandwichCommutator:3}
\begin{aligned}
e^A B e^{-A}
&=
\sum_{r=0}^\infty \inv{r!} C_{r}(A,B),
\end{aligned}
\end{equation}

Or, in explicit form
\begin{equation}\label{eqn:exponentialSandwichCommutator:4}
\begin{aligned}
e^A B e^{-A}
&=
B 
+ \inv{1!} \antisymmetric{A}{B}
+ \inv{2!} 
\antisymmetric{A}{\antisymmetric{A}{B}}
+ \cdots
\end{aligned}
\end{equation}

\documentclass[]{eliblog}

\usepackage{amsmath}
\usepackage{mathpazo}

%
% shorthand for bold symbols, convenient for vectors and matrices
%
\newcommand{\Ba}[0]{\mathbf{a}}
\newcommand{\Bb}[0]{\mathbf{b}}
\newcommand{\Bc}[0]{\mathbf{c}}
\newcommand{\Bd}[0]{\mathbf{d}}
\newcommand{\Be}[0]{\mathbf{e}}
\newcommand{\Bf}[0]{\mathbf{f}}
\newcommand{\Bg}[0]{\mathbf{g}}
\newcommand{\Bh}[0]{\mathbf{h}}
\newcommand{\Bi}[0]{\mathbf{i}}
\newcommand{\Bj}[0]{\mathbf{j}}
\newcommand{\Bk}[0]{\mathbf{k}}
\newcommand{\Bl}[0]{\mathbf{l}}
\newcommand{\Bm}[0]{\mathbf{m}}
\newcommand{\Bn}[0]{\mathbf{n}}
\newcommand{\Bo}[0]{\mathbf{o}}
\newcommand{\Bp}[0]{\mathbf{p}}
\newcommand{\Bq}[0]{\mathbf{q}}
\newcommand{\Br}[0]{\mathbf{r}}
\newcommand{\Bs}[0]{\mathbf{s}}
\newcommand{\Bt}[0]{\mathbf{t}}
\newcommand{\Bu}[0]{\mathbf{u}}
\newcommand{\Bv}[0]{\mathbf{v}}
\newcommand{\Bw}[0]{\mathbf{w}}
\newcommand{\Bx}[0]{\mathbf{x}}
\newcommand{\By}[0]{\mathbf{y}}
\newcommand{\Bz}[0]{\mathbf{z}}
\newcommand{\BA}[0]{\mathbf{A}}
\newcommand{\BB}[0]{\mathbf{B}}
\newcommand{\BC}[0]{\mathbf{C}}
\newcommand{\BD}[0]{\mathbf{D}}
\newcommand{\BE}[0]{\mathbf{E}}
\newcommand{\BF}[0]{\mathbf{F}}
\newcommand{\BG}[0]{\mathbf{G}}
\newcommand{\BH}[0]{\mathbf{H}}
\newcommand{\BI}[0]{\mathbf{I}}
\newcommand{\BJ}[0]{\mathbf{J}}
\newcommand{\BK}[0]{\mathbf{K}}
\newcommand{\BL}[0]{\mathbf{L}}
\newcommand{\BM}[0]{\mathbf{M}}
\newcommand{\BN}[0]{\mathbf{N}}
\newcommand{\BO}[0]{\mathbf{O}}
\newcommand{\BP}[0]{\mathbf{P}}
\newcommand{\BQ}[0]{\mathbf{Q}}
\newcommand{\BR}[0]{\mathbf{R}}
\newcommand{\BS}[0]{\mathbf{S}}
\newcommand{\BT}[0]{\mathbf{T}}
\newcommand{\BU}[0]{\mathbf{U}}
\newcommand{\BV}[0]{\mathbf{V}}
\newcommand{\BW}[0]{\mathbf{W}}
\newcommand{\BX}[0]{\mathbf{X}}
\newcommand{\BY}[0]{\mathbf{Y}}
\newcommand{\BZ}[0]{\mathbf{Z}}

\newcommand{\Bzero}[0]{\mathbf{0}}
\newcommand{\Btheta}[0]{\boldsymbol{\theta}}
\newcommand{\Btau}[0]{\boldsymbol{\tau}}
\newcommand{\Bomega}[0]{\boldsymbol{\omega}}

%
% shorthand for unit vectors
%
\newcommand{\acap}[0]{\hat{\Ba}}
\newcommand{\bcap}[0]{\hat{\Bb}}
\newcommand{\ccap}[0]{\hat{\Bc}}
\newcommand{\dcap}[0]{\hat{\Bd}}
\newcommand{\ecap}[0]{\hat{\Be}}
\newcommand{\fcap}[0]{\hat{\Bf}}
\newcommand{\gcap}[0]{\hat{\Bg}}
\newcommand{\hcap}[0]{\hat{\Bh}}
\newcommand{\icap}[0]{\hat{\Bi}}
\newcommand{\jcap}[0]{\hat{\Bj}}
\newcommand{\kcap}[0]{\hat{\Bk}}
\newcommand{\lcap}[0]{\hat{\Bl}}
\newcommand{\mcap}[0]{\hat{\Bm}}
\newcommand{\ncap}[0]{\hat{\Bn}}
\newcommand{\ocap}[0]{\hat{\Bo}}
\newcommand{\pcap}[0]{\hat{\Bp}}
\newcommand{\qcap}[0]{\hat{\Bq}}
\newcommand{\rcap}[0]{\hat{\Br}}
\newcommand{\scap}[0]{\hat{\Bs}}
\newcommand{\tcap}[0]{\hat{\Bt}}
\newcommand{\ucap}[0]{\hat{\Bu}}
\newcommand{\vcap}[0]{\hat{\Bv}}
\newcommand{\wcap}[0]{\hat{\Bw}}
\newcommand{\xcap}[0]{\hat{\Bx}}
\newcommand{\ycap}[0]{\hat{\By}}
\newcommand{\zcap}[0]{\hat{\Bz}}
\newcommand{\thetacap}[0]{\hat{\Btheta}}

%
% to write R^n and C^n in a distinguishable fashion.  Perhaps change this
% to the double lined characters upon figuring out how to do so.
%
\newcommand{\C}[1]{$\mathbb{C}^{#1}$}
\newcommand{\R}[1]{$\mathbb{R}^{#1}$}

%
% various generally useful helpers
%

% derivative of #1 wrt. #2:
\newcommand{\D}[2] {\frac {d#2} {d#1}}

\newcommand{\inv}[1]{\frac{1}{#1}}
\newcommand{\cross}[0]{\times}

\newcommand{\abs}[1]{\lvert{#1}\rvert}
\newcommand{\norm}[1]{\lVert{#1}\rVert}
\newcommand{\innerprod}[2]{\langle{#1}, {#2}\rangle}
\newcommand{\dotprod}[2]{{#1} \cdot {#2}}
\newcommand{\bdotprod}[2]{\left({#1} \cdot {#2}\right)}
\newcommand{\crossprod}[2]{{#1} \cross {#2}}
\newcommand{\tripleprod}[3]{\dotprod{\left(\crossprod{#1}{#2}\right)}{#3}}

\DeclareMathOperator{\Proj}{Proj}
\DeclareMathOperator{\Span}{span}
\DeclareMathOperator{\Sgn}{sgn}
\DeclareMathOperator{\Area}{Area}
\DeclareMathOperator{\Volume}{Volume}

%
% A few miscellaneous things specific to this document
%
\newcommand{\crossop}[1]{\crossprod{#1}{}}

% R2 vector.
\newcommand{\VectorTwo}[2]{
\begin{bmatrix}
 {#1} \\
 {#2}
\end{bmatrix}
}

\newcommand{\VectorN}[1]{
\begin{bmatrix}
{#1}_1 \\
{#1}_2 \\
\vdots \\
{#1}_N \\
\end{bmatrix}
}

\newcommand{\DETuvij}[4]{
\begin{vmatrix}
 {#1}_{#3} & {#1}_{#4} \\
 {#2}_{#3} & {#2}_{#4}
\end{vmatrix}
}

\newcommand{\DETuvwijk}[6]{
\begin{vmatrix}
 {#1}_{#4} & {#1}_{#5} & {#1}_{#6} \\
 {#2}_{#4} & {#2}_{#5} & {#2}_{#6} \\
 {#3}_{#4} & {#3}_{#5} & {#3}_{#6}
\end{vmatrix}
}

\newcommand{\DETuvwxijkl}[8]{
\begin{vmatrix}
 {#1}_{#5} & {#1}_{#6} & {#1}_{#7} & {#1}_{#8} \\
 {#2}_{#5} & {#2}_{#6} & {#2}_{#7} & {#2}_{#8} \\
 {#3}_{#5} & {#3}_{#6} & {#3}_{#7} & {#3}_{#8} \\
 {#4}_{#5} & {#4}_{#6} & {#4}_{#7} & {#4}_{#8} \\
\end{vmatrix}
}

%\newcommand{\DETuvwxyijklm}[10]{
%\begin{vmatrix}
% {#1}_{#6} & {#1}_{#7} & {#1}_{#8} & {#1}_{#9} & {#1}_{#10} \\
% {#2}_{#6} & {#2}_{#7} & {#2}_{#8} & {#2}_{#9} & {#2}_{#10} \\
% {#3}_{#6} & {#3}_{#7} & {#3}_{#8} & {#3}_{#9} & {#3}_{#10} \\
% {#4}_{#6} & {#4}_{#7} & {#4}_{#8} & {#4}_{#9} & {#4}_{#10} \\
% {#5}_{#6} & {#5}_{#7} & {#5}_{#8} & {#5}_{#9} & {#5}_{#10}
%\end{vmatrix}
%}

% R3 vector.
\newcommand{\VectorThree}[3]{
\begin{bmatrix}
 {#1} \\
 {#2} \\
 {#3}
\end{bmatrix}
}



\author{Peeter Joot}
\email{peeter.joot@utoronto.ca}
%%
% Copyright � 2015 Peeter Joot.  All Rights Reserved.
% Licenced as described in the file LICENSE under the root directory of this GIT repository.
%
\documentclass[]{eliblog}

\usepackage{amsmath}
\usepackage{mathpazo}

%
% shorthand for bold symbols, convenient for vectors and matrices
%
\newcommand{\Ba}[0]{\mathbf{a}}
\newcommand{\Bb}[0]{\mathbf{b}}
\newcommand{\Bc}[0]{\mathbf{c}}
\newcommand{\Bd}[0]{\mathbf{d}}
\newcommand{\Be}[0]{\mathbf{e}}
\newcommand{\Bf}[0]{\mathbf{f}}
\newcommand{\Bg}[0]{\mathbf{g}}
\newcommand{\Bh}[0]{\mathbf{h}}
\newcommand{\Bi}[0]{\mathbf{i}}
\newcommand{\Bj}[0]{\mathbf{j}}
\newcommand{\Bk}[0]{\mathbf{k}}
\newcommand{\Bl}[0]{\mathbf{l}}
\newcommand{\Bm}[0]{\mathbf{m}}
\newcommand{\Bn}[0]{\mathbf{n}}
\newcommand{\Bo}[0]{\mathbf{o}}
\newcommand{\Bp}[0]{\mathbf{p}}
\newcommand{\Bq}[0]{\mathbf{q}}
\newcommand{\Br}[0]{\mathbf{r}}
\newcommand{\Bs}[0]{\mathbf{s}}
\newcommand{\Bt}[0]{\mathbf{t}}
\newcommand{\Bu}[0]{\mathbf{u}}
\newcommand{\Bv}[0]{\mathbf{v}}
\newcommand{\Bw}[0]{\mathbf{w}}
\newcommand{\Bx}[0]{\mathbf{x}}
\newcommand{\By}[0]{\mathbf{y}}
\newcommand{\Bz}[0]{\mathbf{z}}
\newcommand{\BA}[0]{\mathbf{A}}
\newcommand{\BB}[0]{\mathbf{B}}
\newcommand{\BC}[0]{\mathbf{C}}
\newcommand{\BD}[0]{\mathbf{D}}
\newcommand{\BE}[0]{\mathbf{E}}
\newcommand{\BF}[0]{\mathbf{F}}
\newcommand{\BG}[0]{\mathbf{G}}
\newcommand{\BH}[0]{\mathbf{H}}
\newcommand{\BI}[0]{\mathbf{I}}
\newcommand{\BJ}[0]{\mathbf{J}}
\newcommand{\BK}[0]{\mathbf{K}}
\newcommand{\BL}[0]{\mathbf{L}}
\newcommand{\BM}[0]{\mathbf{M}}
\newcommand{\BN}[0]{\mathbf{N}}
\newcommand{\BO}[0]{\mathbf{O}}
\newcommand{\BP}[0]{\mathbf{P}}
\newcommand{\BQ}[0]{\mathbf{Q}}
\newcommand{\BR}[0]{\mathbf{R}}
\newcommand{\BS}[0]{\mathbf{S}}
\newcommand{\BT}[0]{\mathbf{T}}
\newcommand{\BU}[0]{\mathbf{U}}
\newcommand{\BV}[0]{\mathbf{V}}
\newcommand{\BW}[0]{\mathbf{W}}
\newcommand{\BX}[0]{\mathbf{X}}
\newcommand{\BY}[0]{\mathbf{Y}}
\newcommand{\BZ}[0]{\mathbf{Z}}

\newcommand{\Bzero}[0]{\mathbf{0}}
\newcommand{\Btheta}[0]{\boldsymbol{\theta}}
\newcommand{\Btau}[0]{\boldsymbol{\tau}}
\newcommand{\Bomega}[0]{\boldsymbol{\omega}}

%
% shorthand for unit vectors
%
\newcommand{\acap}[0]{\hat{\Ba}}
\newcommand{\bcap}[0]{\hat{\Bb}}
\newcommand{\ccap}[0]{\hat{\Bc}}
\newcommand{\dcap}[0]{\hat{\Bd}}
\newcommand{\ecap}[0]{\hat{\Be}}
\newcommand{\fcap}[0]{\hat{\Bf}}
\newcommand{\gcap}[0]{\hat{\Bg}}
\newcommand{\hcap}[0]{\hat{\Bh}}
\newcommand{\icap}[0]{\hat{\Bi}}
\newcommand{\jcap}[0]{\hat{\Bj}}
\newcommand{\kcap}[0]{\hat{\Bk}}
\newcommand{\lcap}[0]{\hat{\Bl}}
\newcommand{\mcap}[0]{\hat{\Bm}}
\newcommand{\ncap}[0]{\hat{\Bn}}
\newcommand{\ocap}[0]{\hat{\Bo}}
\newcommand{\pcap}[0]{\hat{\Bp}}
\newcommand{\qcap}[0]{\hat{\Bq}}
\newcommand{\rcap}[0]{\hat{\Br}}
\newcommand{\scap}[0]{\hat{\Bs}}
\newcommand{\tcap}[0]{\hat{\Bt}}
\newcommand{\ucap}[0]{\hat{\Bu}}
\newcommand{\vcap}[0]{\hat{\Bv}}
\newcommand{\wcap}[0]{\hat{\Bw}}
\newcommand{\xcap}[0]{\hat{\Bx}}
\newcommand{\ycap}[0]{\hat{\By}}
\newcommand{\zcap}[0]{\hat{\Bz}}
\newcommand{\thetacap}[0]{\hat{\Btheta}}

%
% to write R^n and C^n in a distinguishable fashion.  Perhaps change this
% to the double lined characters upon figuring out how to do so.
%
\newcommand{\C}[1]{$\mathbb{C}^{#1}$}
\newcommand{\R}[1]{$\mathbb{R}^{#1}$}

%
% various generally useful helpers
%

% derivative of #1 wrt. #2:
\newcommand{\D}[2] {\frac {d#2} {d#1}}

\newcommand{\inv}[1]{\frac{1}{#1}}
\newcommand{\cross}[0]{\times}

\newcommand{\abs}[1]{\lvert{#1}\rvert}
\newcommand{\norm}[1]{\lVert{#1}\rVert}
\newcommand{\innerprod}[2]{\langle{#1}, {#2}\rangle}
\newcommand{\dotprod}[2]{{#1} \cdot {#2}}
\newcommand{\bdotprod}[2]{\left({#1} \cdot {#2}\right)}
\newcommand{\crossprod}[2]{{#1} \cross {#2}}
\newcommand{\tripleprod}[3]{\dotprod{\left(\crossprod{#1}{#2}\right)}{#3}}

\DeclareMathOperator{\Proj}{Proj}
\DeclareMathOperator{\Span}{span}
\DeclareMathOperator{\Sgn}{sgn}
\DeclareMathOperator{\Area}{Area}
\DeclareMathOperator{\Volume}{Volume}

%
% A few miscellaneous things specific to this document
%
\newcommand{\crossop}[1]{\crossprod{#1}{}}

% R2 vector.
\newcommand{\VectorTwo}[2]{
\begin{bmatrix}
 {#1} \\
 {#2}
\end{bmatrix}
}

\newcommand{\VectorN}[1]{
\begin{bmatrix}
{#1}_1 \\
{#1}_2 \\
\vdots \\
{#1}_N \\
\end{bmatrix}
}

\newcommand{\DETuvij}[4]{
\begin{vmatrix}
 {#1}_{#3} & {#1}_{#4} \\
 {#2}_{#3} & {#2}_{#4}
\end{vmatrix}
}

\newcommand{\DETuvwijk}[6]{
\begin{vmatrix}
 {#1}_{#4} & {#1}_{#5} & {#1}_{#6} \\
 {#2}_{#4} & {#2}_{#5} & {#2}_{#6} \\
 {#3}_{#4} & {#3}_{#5} & {#3}_{#6}
\end{vmatrix}
}

\newcommand{\DETuvwxijkl}[8]{
\begin{vmatrix}
 {#1}_{#5} & {#1}_{#6} & {#1}_{#7} & {#1}_{#8} \\
 {#2}_{#5} & {#2}_{#6} & {#2}_{#7} & {#2}_{#8} \\
 {#3}_{#5} & {#3}_{#6} & {#3}_{#7} & {#3}_{#8} \\
 {#4}_{#5} & {#4}_{#6} & {#4}_{#7} & {#4}_{#8} \\
\end{vmatrix}
}

%\newcommand{\DETuvwxyijklm}[10]{
%\begin{vmatrix}
% {#1}_{#6} & {#1}_{#7} & {#1}_{#8} & {#1}_{#9} & {#1}_{#10} \\
% {#2}_{#6} & {#2}_{#7} & {#2}_{#8} & {#2}_{#9} & {#2}_{#10} \\
% {#3}_{#6} & {#3}_{#7} & {#3}_{#8} & {#3}_{#9} & {#3}_{#10} \\
% {#4}_{#6} & {#4}_{#7} & {#4}_{#8} & {#4}_{#9} & {#4}_{#10} \\
% {#5}_{#6} & {#5}_{#7} & {#5}_{#8} & {#5}_{#9} & {#5}_{#10}
%\end{vmatrix}
%}

% R3 vector.
\newcommand{\VectorThree}[3]{
\begin{bmatrix}
 {#1} \\
 {#2} \\
 {#3}
\end{bmatrix}
}



\author{Peeter Joot}
\email{peeter.joot@gmail.com}

%\documentclass[]{eliblogwidescreen}

\usepackage{amsmath}
\usepackage{mathpazo}

%
% shorthand for bold symbols, convenient for vectors and matrices
%
\newcommand{\Ba}[0]{\mathbf{a}}
\newcommand{\Bb}[0]{\mathbf{b}}
\newcommand{\Bc}[0]{\mathbf{c}}
\newcommand{\Bd}[0]{\mathbf{d}}
\newcommand{\Be}[0]{\mathbf{e}}
\newcommand{\Bf}[0]{\mathbf{f}}
\newcommand{\Bg}[0]{\mathbf{g}}
\newcommand{\Bh}[0]{\mathbf{h}}
\newcommand{\Bi}[0]{\mathbf{i}}
\newcommand{\Bj}[0]{\mathbf{j}}
\newcommand{\Bk}[0]{\mathbf{k}}
\newcommand{\Bl}[0]{\mathbf{l}}
\newcommand{\Bm}[0]{\mathbf{m}}
\newcommand{\Bn}[0]{\mathbf{n}}
\newcommand{\Bo}[0]{\mathbf{o}}
\newcommand{\Bp}[0]{\mathbf{p}}
\newcommand{\Bq}[0]{\mathbf{q}}
\newcommand{\Br}[0]{\mathbf{r}}
\newcommand{\Bs}[0]{\mathbf{s}}
\newcommand{\Bt}[0]{\mathbf{t}}
\newcommand{\Bu}[0]{\mathbf{u}}
\newcommand{\Bv}[0]{\mathbf{v}}
\newcommand{\Bw}[0]{\mathbf{w}}
\newcommand{\Bx}[0]{\mathbf{x}}
\newcommand{\By}[0]{\mathbf{y}}
\newcommand{\Bz}[0]{\mathbf{z}}
\newcommand{\BA}[0]{\mathbf{A}}
\newcommand{\BB}[0]{\mathbf{B}}
\newcommand{\BC}[0]{\mathbf{C}}
\newcommand{\BD}[0]{\mathbf{D}}
\newcommand{\BE}[0]{\mathbf{E}}
\newcommand{\BF}[0]{\mathbf{F}}
\newcommand{\BG}[0]{\mathbf{G}}
\newcommand{\BH}[0]{\mathbf{H}}
\newcommand{\BI}[0]{\mathbf{I}}
\newcommand{\BJ}[0]{\mathbf{J}}
\newcommand{\BK}[0]{\mathbf{K}}
\newcommand{\BL}[0]{\mathbf{L}}
\newcommand{\BM}[0]{\mathbf{M}}
\newcommand{\BN}[0]{\mathbf{N}}
\newcommand{\BO}[0]{\mathbf{O}}
\newcommand{\BP}[0]{\mathbf{P}}
\newcommand{\BQ}[0]{\mathbf{Q}}
\newcommand{\BR}[0]{\mathbf{R}}
\newcommand{\BS}[0]{\mathbf{S}}
\newcommand{\BT}[0]{\mathbf{T}}
\newcommand{\BU}[0]{\mathbf{U}}
\newcommand{\BV}[0]{\mathbf{V}}
\newcommand{\BW}[0]{\mathbf{W}}
\newcommand{\BX}[0]{\mathbf{X}}
\newcommand{\BY}[0]{\mathbf{Y}}
\newcommand{\BZ}[0]{\mathbf{Z}}

\newcommand{\Bzero}[0]{\mathbf{0}}
\newcommand{\Btheta}[0]{\boldsymbol{\theta}}
\newcommand{\Btau}[0]{\boldsymbol{\tau}}
\newcommand{\Bomega}[0]{\boldsymbol{\omega}}

%
% shorthand for unit vectors
%
\newcommand{\acap}[0]{\hat{\Ba}}
\newcommand{\bcap}[0]{\hat{\Bb}}
\newcommand{\ccap}[0]{\hat{\Bc}}
\newcommand{\dcap}[0]{\hat{\Bd}}
\newcommand{\ecap}[0]{\hat{\Be}}
\newcommand{\fcap}[0]{\hat{\Bf}}
\newcommand{\gcap}[0]{\hat{\Bg}}
\newcommand{\hcap}[0]{\hat{\Bh}}
\newcommand{\icap}[0]{\hat{\Bi}}
\newcommand{\jcap}[0]{\hat{\Bj}}
\newcommand{\kcap}[0]{\hat{\Bk}}
\newcommand{\lcap}[0]{\hat{\Bl}}
\newcommand{\mcap}[0]{\hat{\Bm}}
\newcommand{\ncap}[0]{\hat{\Bn}}
\newcommand{\ocap}[0]{\hat{\Bo}}
\newcommand{\pcap}[0]{\hat{\Bp}}
\newcommand{\qcap}[0]{\hat{\Bq}}
\newcommand{\rcap}[0]{\hat{\Br}}
\newcommand{\scap}[0]{\hat{\Bs}}
\newcommand{\tcap}[0]{\hat{\Bt}}
\newcommand{\ucap}[0]{\hat{\Bu}}
\newcommand{\vcap}[0]{\hat{\Bv}}
\newcommand{\wcap}[0]{\hat{\Bw}}
\newcommand{\xcap}[0]{\hat{\Bx}}
\newcommand{\ycap}[0]{\hat{\By}}
\newcommand{\zcap}[0]{\hat{\Bz}}
\newcommand{\thetacap}[0]{\hat{\Btheta}}

%
% to write R^n and C^n in a distinguishable fashion.  Perhaps change this
% to the double lined characters upon figuring out how to do so.
%
\newcommand{\C}[1]{$\mathbb{C}^{#1}$}
\newcommand{\R}[1]{$\mathbb{R}^{#1}$}

%
% various generally useful helpers
%

% derivative of #1 wrt. #2:
\newcommand{\D}[2] {\frac {d#2} {d#1}}

\newcommand{\inv}[1]{\frac{1}{#1}}
\newcommand{\cross}[0]{\times}

\newcommand{\abs}[1]{\lvert{#1}\rvert}
\newcommand{\norm}[1]{\lVert{#1}\rVert}
\newcommand{\innerprod}[2]{\langle{#1}, {#2}\rangle}
\newcommand{\dotprod}[2]{{#1} \cdot {#2}}
\newcommand{\bdotprod}[2]{\left({#1} \cdot {#2}\right)}
\newcommand{\crossprod}[2]{{#1} \cross {#2}}
\newcommand{\tripleprod}[3]{\dotprod{\left(\crossprod{#1}{#2}\right)}{#3}}

\DeclareMathOperator{\Proj}{Proj}
\DeclareMathOperator{\Span}{span}
\DeclareMathOperator{\Sgn}{sgn}
\DeclareMathOperator{\Area}{Area}
\DeclareMathOperator{\Volume}{Volume}

%
% A few miscellaneous things specific to this document
%
\newcommand{\crossop}[1]{\crossprod{#1}{}}

% R2 vector.
\newcommand{\VectorTwo}[2]{
\begin{bmatrix}
 {#1} \\
 {#2}
\end{bmatrix}
}

\newcommand{\VectorN}[1]{
\begin{bmatrix}
{#1}_1 \\
{#1}_2 \\
\vdots \\
{#1}_N \\
\end{bmatrix}
}

\newcommand{\DETuvij}[4]{
\begin{vmatrix}
 {#1}_{#3} & {#1}_{#4} \\
 {#2}_{#3} & {#2}_{#4}
\end{vmatrix}
}

\newcommand{\DETuvwijk}[6]{
\begin{vmatrix}
 {#1}_{#4} & {#1}_{#5} & {#1}_{#6} \\
 {#2}_{#4} & {#2}_{#5} & {#2}_{#6} \\
 {#3}_{#4} & {#3}_{#5} & {#3}_{#6}
\end{vmatrix}
}

\newcommand{\DETuvwxijkl}[8]{
\begin{vmatrix}
 {#1}_{#5} & {#1}_{#6} & {#1}_{#7} & {#1}_{#8} \\
 {#2}_{#5} & {#2}_{#6} & {#2}_{#7} & {#2}_{#8} \\
 {#3}_{#5} & {#3}_{#6} & {#3}_{#7} & {#3}_{#8} \\
 {#4}_{#5} & {#4}_{#6} & {#4}_{#7} & {#4}_{#8} \\
\end{vmatrix}
}

%\newcommand{\DETuvwxyijklm}[10]{
%\begin{vmatrix}
% {#1}_{#6} & {#1}_{#7} & {#1}_{#8} & {#1}_{#9} & {#1}_{#10} \\
% {#2}_{#6} & {#2}_{#7} & {#2}_{#8} & {#2}_{#9} & {#2}_{#10} \\
% {#3}_{#6} & {#3}_{#7} & {#3}_{#8} & {#3}_{#9} & {#3}_{#10} \\
% {#4}_{#6} & {#4}_{#7} & {#4}_{#8} & {#4}_{#9} & {#4}_{#10} \\
% {#5}_{#6} & {#5}_{#7} & {#5}_{#8} & {#5}_{#9} & {#5}_{#10}
%\end{vmatrix}
%}

% R3 vector.
\newcommand{\VectorThree}[3]{
\begin{bmatrix}
 {#1} \\
 {#2} \\
 {#3}
\end{bmatrix}
}



\author{Peeter Joot}
\email{peeter.joot@gmail.com}


\chapter{PHY356 Problem Set I.}
\label{chap:qmIproblemSet1}
%\useCCL
%\blogpage{http://sites.google.com/site/peeterjoot/math2010/qmIproblemSet1.pdf}
%\date{Oct X, 2010}
%\revisionInfo{qmIproblemSet1.tex}

%\beginArtWithToc
\beginArtNoToc

\section{Problem 1.}

Assume that $X$ and $P = -i \hbar \PDi{x}{}$ are the x-direction position and momentum operators. Show that $\antisymmetric{X}{P}=i\hbar \BOne$. Find $\bra{x}(XP-PX)\ket{x'}$ using the above definitions. What is the physical meaning of this expression?

\subsection{Commutator Part.}

We can get a rough idea of the approach required by temporarily avoiding the Dirac notation that complicates things.  To do so, consider the commutator action on an arbitrary wave function $\psi(x)$

\begin{align*}
(x P - P x)\psi
&=
x P \psi + i\hbar \PD{x}{} (x \psi) \\
&=
x P \psi + i\hbar \left(\psi + \PD{x}{\psi} \right) \\
&=
x P \psi + i \hbar \psi - x P \psi \\
&=
i \hbar \psi
\end{align*}

Since this is true for all $\psi(x)$ we can make the identification
\begin{align*}
x P - P x &= i\hbar \BOne
\end{align*}

Since use of the Dirac notation is central to the lecture notes and course text, it seems reasonable to follow this up with the same procedure utilizing the Dirac notation.  We do so by considering the action of the commutator within a matrix element of the form

\begin{align*}
\bra{x} XP - PX \ket{\psi}.
\end{align*}

Considering the $XP$ part first we have

\begin{align*}
\bra{x} XP \ket{\psi}
&=
\int dx'
\bra{x} X \ket{x'}\bra{x'} P \ket{\psi} \\
&=
\int dx'
\braket{x}{x'} x' \bra{x'} P \ket{\psi} \\
&=
\int dx'
\delta(x-x')\bra{x'}x' P \ket{\psi}  \\
&=
\bra{x} x P \ket{\psi}.
\end{align*}

Now consider the $PX$ part

\begin{align*}
\bra{x} PX \ket{\psi}
&=
\bra{x} \BI PX \ket{\psi} \\
%&=
%\int dx'
%\bra{x} P \ket{x'}\bra{x'} X \ket{\psi}  \\
%&=
%\int dx'
%\bra{x} P \ket{x'}\bra{x'}x' \ket{\psi}  \\
%&=
%\int dx'
%\bra{x} (-i\hbar\PD{x}{}) \ket{x'}\bra{x'}x' \ket{\psi}  \\
%&=
%\int dx'
%\braket{x}{x'} (-i\hbar)\PD{x}{} \bra{x'}x' \ket{\psi}  \\
&=
\int dx'
\braket{x}{x'} \bra{x'} P X \ket{\psi}  \\
&=
\int dx'
\braket{x}{x'} \bra{x'} (-i\hbar)\PD{x}{} x' \ket{\psi}  \\
&=
\int dx'
\delta(x-x') \bra{x'} (-i\hbar)\PD{x}{} x' \ket{\psi}  \\
&=
\bra{x} (-i\hbar)\PD{x}{} x \ket{\psi}  \\
&=
\bra{x} (-i\hbar) \BOne \ket{\psi} + 
\bra{x} x (-i\hbar)\PD{x}{} \ket{\psi}  \\
&=
\bra{x} -i\hbar \BOne + x P \ket{\psi}  
\end{align*}

Taking the differences, we have for arbitrary states $\bra{x}$, and $\ket{\psi}$

\begin{align*}
\bra{x} XP - PX \ket{\psi} 
&= 
\bra{x} x P \ket{\psi} -
\bra{x} -i\hbar \BOne + x P \ket{\psi} \\
&= 
\bra{x} i \hbar \BOne \ket{\psi} 
\end{align*}

or

\begin{align*}
\bra{x} XP - PX - i \hbar \BOne \ket{\psi} = 0
\end{align*}

Since this is true for all $\bra{x}$ and $\ket{\psi}$, we can make the required identification $XP - PX = i \hbar \BOne $ as in the wave function approach.

%as required.  This seems like overkill, but does at least produce the expected result.  It also provides another worked example of the fairly tricky seeming Dirac notation.

\subsection{Matrix element.}

Having evaluated the commutator, the matrix element is simple to compute.  It is

\begin{align*}
\bra{x} XP - PX \ket{x'}
&=
\bra{x} i \hbar \BOne \ket{x'} \\
&=
i \hbar \braket{x}{x'} \\
&=
i \hbar \delta(x - x')
\end{align*}

\subsection{Physical meaning of matrix element.}

Now, what is the physical meaning of this matrix element?  When $x = x'$ we have an expectation value

\begin{align*}
\bra{x} XP - PX \ket{x} &= i \hbar,
\end{align*}

but this is not real valued, meaning that the commutator is not Hermitian, and therefore not an observable.  Since the question did not ask for the value of the matrix element of the Hermitian operator, $\antisymmetric{X}{P} = \hbar$, an $\hbar$ scaled identity operator (and an observable), it must be assumed that some other physical meaning is being asked for.

%I am at a loss to assign any further physical meaning to an operator that is not an observable.  The text doesn't provide any help that I can find.  In fact, there is almost no reference to anything physical so far in the text ... just an awful lot of math!

Is there supposed to be a physical meaning to the matrix element itself?  Again the text is no obvious help.  We have a physical meaning of a matrix element of an operator, only in the context of other questions.  One such question is an operator expectation value, for example for the position operator we have

\begin{align*}
\expectation{X}
&=
\bra{\psi} X \ket{\psi} \\
&=
\int dx' dx \braket{\psi}{x'} \bra{x'} X \ket{x} \braket{x}{\psi} \\
&=
\int dx \psi^\conj(x) x \psi(x).
\end{align*}

Here the matrix element $\bra{x'} X \ket{x}$ shows up as an impulse response like weighting factor in the expectation integral, altering the probability density in the region around $x = x'$.  It only seems to be when there is additional context do we have a physical meaning to the matrix element itself.  This example also requires the operator in question to be Hermitian, which isn't the case for the position-momentum commutator.


\section{Problem 2.}

The state of a one-dimensional system is given by $\ket{x_0}$. Does this system obey the position-momentum uncertainty relation? Explain your answer.

\subsection{Solution.}

%\EndArticle
\EndNoBibArticle

%
% Copyright � 2012 Peeter Joot.  All Rights Reserved.
% Licenced as described in the file LICENSE under the root directory of this GIT repository.
%

%\chapter{PHY356 Problem Set II}
\label{chap:qmIproblemSet2}
%\blogpage{http://sites.google.com/site/peeterjoot/math2010/qmIproblemSet2.pdf}
%\date{Oct 23, 2010}

\makeproblem{ps II}{problem:qmIproblemSet2:1}{

A particle of mass \(m\) is free to move along the x-direction such that \(V(X)=0\). Express the time evolution operator \(U(t,t_0)\) defined by Eq. (2.166) using the momentum eigenstates \(\ket{p}\) with delta-function normalization. Find \(\bra{x} U(t,t0) \ket{x'}\),  where \(\ket{x}\) and \(\ket{x'}\) are position eigenstates.  What is the physical meaning of this expression?

} % problem

\makeanswer{problem:qmIproblemSet2:1}{
\paragraph{Momentum matrix element}

We can expand the time evolution operator in series

\begin{equation}\label{eqn:qmIproblemSet2:5020}
\begin{aligned}
U(t,t_0) 
&= e^{-i H(t-t_0)/\Hbar} \\
&= e^{ -i P^2 (t-t_0)/ 2m \Hbar } \\
&= 1 + \sum_{k=1}^\infty \inv{k!} \left( -i \frac{P^2 (t-t_0)}{2m \Hbar} \right)^k.
\end{aligned}
\end{equation}

We can now evaluate the momentum matrix element \(\bra{p} U(t,t_0) \ket{p'}\), which will essentially require the value of \(\bra{p} P^{2k} \ket{p'}\).  That is

\begin{equation}\label{eqn:qmIproblemSet2:5040}
\begin{aligned}
\bra{p} P^{2k} \ket{p'}
&= \bra{p} P^{2k-1} P \ket{p'} \\
&= \bra{p} P^{2k-1} \ket{p'} p' \\
&= \cdots \\
&= \braket{p}{p'} (p')^{2k}.
\end{aligned}
\end{equation}

The momentum matrix element is therefore reduced to

\begin{equation}\label{eqn:qmIproblemSet2:1}
\bra{p} U(t,t_0) \ket{p'}
=
\braket{p}{p'} \exp\left( -i \frac{p^2 (t-t_0)}{2m \Hbar} \right)
= \delta(p-p') \exp\left( -i \frac{p^2 (t-t_0)}{2m \Hbar} \right)
\end{equation}

\paragraph{Position matrix element}
For the position matrix element we have a similar sum
\begin{equation}\label{eqn:qmIproblemSet2:5060}
\bra{x} U(t,t_0) \ket{x'} 
= 
\braket{x}{x'} 
+ \sum_{k=1}^\infty \inv{k!} \bra{x} \left( -i \frac{P^2 (t-t_0)}{2m \Hbar} \right)^k \ket{x'},
\end{equation}

and require \(\bra{x} P^{2k} \ket{x'}\) to continue.  That is

\begin{equation}\label{eqn:qmIproblemSet2:5080}
\begin{aligned}
\bra{x} P^{2k} \ket{x'}
&=
\int dx''\bra{x} P^{2k-1} \ket{x''}\bra{x''} P \ket{x'} \\
&=
\int dx''\bra{x} P^{2k-1} \ket{x''} \delta(x''-x') (-i\Hbar) \frac{d}{dx'} \\
&=
\bra{x} P^{2k-1} \ket{x'} (-i\Hbar) \frac{d}{dx'} \\
&= \cdots \\
&= \braket{x}{x'} \left( (-i\Hbar) \frac{d}{dx'} \right)^{2k}
\end{aligned}
\end{equation}

Our position matrix element is therefore the differential operator

\begin{equation}\label{eqn:qmIproblemSet2:10}
\bra{x} U(t,t_0) \ket{x'} 
=
\braket{x}{x'} \exp\left( \frac{i (t-t_0)\Hbar}{2m} \frac{d^2}{d{x'}^2} \right)
=\delta(x-x') \exp\left( \frac{i (t-t_0)\Hbar}{2m} \frac{d^2}{d{x'}^2} \right)
\end{equation}

\paragraph{Physical interpretation of the position matrix element operator}

Finally, we need to determine the physical meaning of such a matrix element operator.  

With the delta function that this matrix element operator includes it really only takes on a meaning with a convolution integral.  The simplest such integral would be

\begin{equation}\label{eqn:qmIproblemSet2:5100}
\begin{aligned}
\int dx' \bra{x} U \ket{x'} \braket{x'}{\phi_0} 
&=
\bra{x} U \ket{\phi_0} \\
&=
\braket{x}{\phi(t)} \\
&=
\phi(x,t),
\end{aligned}
\end{equation}

or
\begin{equation}\label{eqn:qmIproblemSet2:5120}
\phi(x,t) = \int dx' \bra{x} U \ket{x'} \phi(x',0)
\end{equation}

The LHS has a physical meaning, and in the absolute square

\begin{equation}\label{eqn:qmIproblemSet2:5000}
\int_{x_0}^{x_0+ \Delta x} \Abs{\phi(x,t)}^2 dx,
\end{equation}

provides the probability that the particle will be found in the region \([x_0, x_0+ \Delta x]\).  

If we ignore the absolute square requirement and think of the (presumed normalized) wave function \(\phi(x,t)\) more loosely as representing a probability directly, then we can in turn give a meaning to the matrix element \(\bra{x} U \ket{x'}\) for the time evolution operator.  This provides an operator valued weighting function that provides us with the probability that a particle initially at position \(x'\) will be at position \(x\) at time \(t\).  This probability is indirect since we need to absolute square and sum over a finite interval to obtain the probability of finding the particle in that interval.

Observe that the integral on the RHS of \eqnref{eqn:qmIproblemSet2:5000} is a summation over all \(x'\), so we can think of this as adding the probabilities that the particle was at each point to arrive at the total probability for finding it at the new location \(x\).  The time evolution operator matrix element provides the weighting in this conditional probability.

In \eqnref{eqn:qmIproblemSet2:10} we found that the time evolution operators matrix element is differential operator in the position representation.  In the general case this means that this probability weighting is not just numeric since the operation of the matrix element initial time wave function can produce wave functions for additional states.  In some special cases, we may find that this weighting is strictly numeric, and one such example would be the Gaussian wave packet \(\phi(x',0) = e^{-a{x'}^2}\).  Application of the differential operations would then produce polynomial weighted multiples of the original Gaussian.  In this special case we would be able to write

\begin{equation}\label{eqn:qmIproblemSet2:5140}
\phi(x,t) = \int dx' \bra{x} U \ket{x'} \phi(x',0) = \int dx' K(x,x',t) \phi(x',0) 
\end{equation}

Where \(K(x,x',t)\) is a polynomial valued function (and is in fact another exponential), and now just provides a numerical weighting for the conditional probability for the particle to move from \(x'\) to \(x\) in time \(t\).  In \citep{liboff2003iqm}, this \(K(x,x',t)\) is called the Propagator function.  It is perhaps justifiable to also call our similar operator valued matrix element a Propagator.

%Based on this, I would be inclined to state that the position matrix element of the time evolution operator \(\bra{x} U \ket{x'}\) represents something akin to an operator form of probability amplitude for a particle to travel between two points.
%Consider two additional contexts where this matrix element arises.  We can find this matrix element by expanding a normalized inner product for the time evolved state
%\begin{align*}
%1 
%&= \braket{\phi(t)}{\phi(t)} \\
%&= \bra{\phi(t)} U \ket{\phi_0} \\
%&= \iint dx dx' \bra{\phi(t)} \ket{x} \bra{x} U \ket{x'}\bra{x'} \ket{\phi_0} \\
%&= \iint dx dx' \phi^\conj(x,t) \bra{x} U \ket{x'} \phi(x,0).
%\end{align*}
%
%This matrix element also arises in the expectation value of the time evolution operator with respect to the initial time state.  That is
%
%\begin{align*}
%\expectation{U} 
%&= \bra{\phi_0} U \ket{\phi_0} \\
%&= \braket{\phi_0}{\phi} = \int dx \phi^\conj(x,0) \phi(x,t) \\
%&= \iint dx dx' \bra{\phi_0} \ket{x} \bra{x} U \ket{x'}\bra{x'} \ket{\phi_0} \\
%&= \iint dx dx' \phi^\conj(x,0) \bra{x} U \ket{x'} \phi(x,0)
%\end{align*}

%\subsection{My grade}
%
%I got full marks on this assignment.  There is apparently another way to do part of the first question on the position representation, and I was instructed by the TA to see the posted solution, which is not yet available.
} % answer

\documentclass[]{eliblog}

\usepackage{color}

\usepackage{amsmath}
\usepackage{mathpazo}

%
% shorthand for bold symbols, convenient for vectors and matrices
%
\newcommand{\Ba}[0]{\mathbf{a}}
\newcommand{\Bb}[0]{\mathbf{b}}
\newcommand{\Bc}[0]{\mathbf{c}}
\newcommand{\Bd}[0]{\mathbf{d}}
\newcommand{\Be}[0]{\mathbf{e}}
\newcommand{\Bf}[0]{\mathbf{f}}
\newcommand{\Bg}[0]{\mathbf{g}}
\newcommand{\Bh}[0]{\mathbf{h}}
\newcommand{\Bi}[0]{\mathbf{i}}
\newcommand{\Bj}[0]{\mathbf{j}}
\newcommand{\Bk}[0]{\mathbf{k}}
\newcommand{\Bl}[0]{\mathbf{l}}
\newcommand{\Bm}[0]{\mathbf{m}}
\newcommand{\Bn}[0]{\mathbf{n}}
\newcommand{\Bo}[0]{\mathbf{o}}
\newcommand{\Bp}[0]{\mathbf{p}}
\newcommand{\Bq}[0]{\mathbf{q}}
\newcommand{\Br}[0]{\mathbf{r}}
\newcommand{\Bs}[0]{\mathbf{s}}
\newcommand{\Bt}[0]{\mathbf{t}}
\newcommand{\Bu}[0]{\mathbf{u}}
\newcommand{\Bv}[0]{\mathbf{v}}
\newcommand{\Bw}[0]{\mathbf{w}}
\newcommand{\Bx}[0]{\mathbf{x}}
\newcommand{\By}[0]{\mathbf{y}}
\newcommand{\Bz}[0]{\mathbf{z}}
\newcommand{\BA}[0]{\mathbf{A}}
\newcommand{\BB}[0]{\mathbf{B}}
\newcommand{\BC}[0]{\mathbf{C}}
\newcommand{\BD}[0]{\mathbf{D}}
\newcommand{\BE}[0]{\mathbf{E}}
\newcommand{\BF}[0]{\mathbf{F}}
\newcommand{\BG}[0]{\mathbf{G}}
\newcommand{\BH}[0]{\mathbf{H}}
\newcommand{\BI}[0]{\mathbf{I}}
\newcommand{\BJ}[0]{\mathbf{J}}
\newcommand{\BK}[0]{\mathbf{K}}
\newcommand{\BL}[0]{\mathbf{L}}
\newcommand{\BM}[0]{\mathbf{M}}
\newcommand{\BN}[0]{\mathbf{N}}
\newcommand{\BO}[0]{\mathbf{O}}
\newcommand{\BP}[0]{\mathbf{P}}
\newcommand{\BQ}[0]{\mathbf{Q}}
\newcommand{\BR}[0]{\mathbf{R}}
\newcommand{\BS}[0]{\mathbf{S}}
\newcommand{\BT}[0]{\mathbf{T}}
\newcommand{\BU}[0]{\mathbf{U}}
\newcommand{\BV}[0]{\mathbf{V}}
\newcommand{\BW}[0]{\mathbf{W}}
\newcommand{\BX}[0]{\mathbf{X}}
\newcommand{\BY}[0]{\mathbf{Y}}
\newcommand{\BZ}[0]{\mathbf{Z}}

\newcommand{\Bzero}[0]{\mathbf{0}}
\newcommand{\Btheta}[0]{\boldsymbol{\theta}}
\newcommand{\Btau}[0]{\boldsymbol{\tau}}
\newcommand{\Bomega}[0]{\boldsymbol{\omega}}

%
% shorthand for unit vectors
%
\newcommand{\acap}[0]{\hat{\Ba}}
\newcommand{\bcap}[0]{\hat{\Bb}}
\newcommand{\ccap}[0]{\hat{\Bc}}
\newcommand{\dcap}[0]{\hat{\Bd}}
\newcommand{\ecap}[0]{\hat{\Be}}
\newcommand{\fcap}[0]{\hat{\Bf}}
\newcommand{\gcap}[0]{\hat{\Bg}}
\newcommand{\hcap}[0]{\hat{\Bh}}
\newcommand{\icap}[0]{\hat{\Bi}}
\newcommand{\jcap}[0]{\hat{\Bj}}
\newcommand{\kcap}[0]{\hat{\Bk}}
\newcommand{\lcap}[0]{\hat{\Bl}}
\newcommand{\mcap}[0]{\hat{\Bm}}
\newcommand{\ncap}[0]{\hat{\Bn}}
\newcommand{\ocap}[0]{\hat{\Bo}}
\newcommand{\pcap}[0]{\hat{\Bp}}
\newcommand{\qcap}[0]{\hat{\Bq}}
\newcommand{\rcap}[0]{\hat{\Br}}
\newcommand{\scap}[0]{\hat{\Bs}}
\newcommand{\tcap}[0]{\hat{\Bt}}
\newcommand{\ucap}[0]{\hat{\Bu}}
\newcommand{\vcap}[0]{\hat{\Bv}}
\newcommand{\wcap}[0]{\hat{\Bw}}
\newcommand{\xcap}[0]{\hat{\Bx}}
\newcommand{\ycap}[0]{\hat{\By}}
\newcommand{\zcap}[0]{\hat{\Bz}}
\newcommand{\thetacap}[0]{\hat{\Btheta}}

%
% to write R^n and C^n in a distinguishable fashion.  Perhaps change this
% to the double lined characters upon figuring out how to do so.
%
\newcommand{\C}[1]{$\mathbb{C}^{#1}$}
\newcommand{\R}[1]{$\mathbb{R}^{#1}$}

%
% various generally useful helpers
%

% derivative of #1 wrt. #2:
\newcommand{\D}[2] {\frac {d#2} {d#1}}

\newcommand{\inv}[1]{\frac{1}{#1}}
\newcommand{\cross}[0]{\times}

\newcommand{\abs}[1]{\lvert{#1}\rvert}
\newcommand{\norm}[1]{\lVert{#1}\rVert}
\newcommand{\innerprod}[2]{\langle{#1}, {#2}\rangle}
\newcommand{\dotprod}[2]{{#1} \cdot {#2}}
\newcommand{\bdotprod}[2]{\left({#1} \cdot {#2}\right)}
\newcommand{\crossprod}[2]{{#1} \cross {#2}}
\newcommand{\tripleprod}[3]{\dotprod{\left(\crossprod{#1}{#2}\right)}{#3}}

\DeclareMathOperator{\Proj}{Proj}
\DeclareMathOperator{\Span}{span}
\DeclareMathOperator{\Sgn}{sgn}
\DeclareMathOperator{\Area}{Area}
\DeclareMathOperator{\Volume}{Volume}

%
% A few miscellaneous things specific to this document
%
\newcommand{\crossop}[1]{\crossprod{#1}{}}

% R2 vector.
\newcommand{\VectorTwo}[2]{
\begin{bmatrix}
 {#1} \\
 {#2}
\end{bmatrix}
}

\newcommand{\VectorN}[1]{
\begin{bmatrix}
{#1}_1 \\
{#1}_2 \\
\vdots \\
{#1}_N \\
\end{bmatrix}
}

\newcommand{\DETuvij}[4]{
\begin{vmatrix}
 {#1}_{#3} & {#1}_{#4} \\
 {#2}_{#3} & {#2}_{#4}
\end{vmatrix}
}

\newcommand{\DETuvwijk}[6]{
\begin{vmatrix}
 {#1}_{#4} & {#1}_{#5} & {#1}_{#6} \\
 {#2}_{#4} & {#2}_{#5} & {#2}_{#6} \\
 {#3}_{#4} & {#3}_{#5} & {#3}_{#6}
\end{vmatrix}
}

\newcommand{\DETuvwxijkl}[8]{
\begin{vmatrix}
 {#1}_{#5} & {#1}_{#6} & {#1}_{#7} & {#1}_{#8} \\
 {#2}_{#5} & {#2}_{#6} & {#2}_{#7} & {#2}_{#8} \\
 {#3}_{#5} & {#3}_{#6} & {#3}_{#7} & {#3}_{#8} \\
 {#4}_{#5} & {#4}_{#6} & {#4}_{#7} & {#4}_{#8} \\
\end{vmatrix}
}

%\newcommand{\DETuvwxyijklm}[10]{
%\begin{vmatrix}
% {#1}_{#6} & {#1}_{#7} & {#1}_{#8} & {#1}_{#9} & {#1}_{#10} \\
% {#2}_{#6} & {#2}_{#7} & {#2}_{#8} & {#2}_{#9} & {#2}_{#10} \\
% {#3}_{#6} & {#3}_{#7} & {#3}_{#8} & {#3}_{#9} & {#3}_{#10} \\
% {#4}_{#6} & {#4}_{#7} & {#4}_{#8} & {#4}_{#9} & {#4}_{#10} \\
% {#5}_{#6} & {#5}_{#7} & {#5}_{#8} & {#5}_{#9} & {#5}_{#10}
%\end{vmatrix}
%}

% R3 vector.
\newcommand{\VectorThree}[3]{
\begin{bmatrix}
 {#1} \\
 {#2} \\
 {#3}
\end{bmatrix}
}



\author{Peeter Joot}
\email{peeter.joot@utoronto.ca, 920798560}
%%
% Copyright � 2015 Peeter Joot.  All Rights Reserved.
% Licenced as described in the file LICENSE under the root directory of this GIT repository.
%
\documentclass[]{eliblog}

\usepackage{amsmath}
\usepackage{mathpazo}

%
% shorthand for bold symbols, convenient for vectors and matrices
%
\newcommand{\Ba}[0]{\mathbf{a}}
\newcommand{\Bb}[0]{\mathbf{b}}
\newcommand{\Bc}[0]{\mathbf{c}}
\newcommand{\Bd}[0]{\mathbf{d}}
\newcommand{\Be}[0]{\mathbf{e}}
\newcommand{\Bf}[0]{\mathbf{f}}
\newcommand{\Bg}[0]{\mathbf{g}}
\newcommand{\Bh}[0]{\mathbf{h}}
\newcommand{\Bi}[0]{\mathbf{i}}
\newcommand{\Bj}[0]{\mathbf{j}}
\newcommand{\Bk}[0]{\mathbf{k}}
\newcommand{\Bl}[0]{\mathbf{l}}
\newcommand{\Bm}[0]{\mathbf{m}}
\newcommand{\Bn}[0]{\mathbf{n}}
\newcommand{\Bo}[0]{\mathbf{o}}
\newcommand{\Bp}[0]{\mathbf{p}}
\newcommand{\Bq}[0]{\mathbf{q}}
\newcommand{\Br}[0]{\mathbf{r}}
\newcommand{\Bs}[0]{\mathbf{s}}
\newcommand{\Bt}[0]{\mathbf{t}}
\newcommand{\Bu}[0]{\mathbf{u}}
\newcommand{\Bv}[0]{\mathbf{v}}
\newcommand{\Bw}[0]{\mathbf{w}}
\newcommand{\Bx}[0]{\mathbf{x}}
\newcommand{\By}[0]{\mathbf{y}}
\newcommand{\Bz}[0]{\mathbf{z}}
\newcommand{\BA}[0]{\mathbf{A}}
\newcommand{\BB}[0]{\mathbf{B}}
\newcommand{\BC}[0]{\mathbf{C}}
\newcommand{\BD}[0]{\mathbf{D}}
\newcommand{\BE}[0]{\mathbf{E}}
\newcommand{\BF}[0]{\mathbf{F}}
\newcommand{\BG}[0]{\mathbf{G}}
\newcommand{\BH}[0]{\mathbf{H}}
\newcommand{\BI}[0]{\mathbf{I}}
\newcommand{\BJ}[0]{\mathbf{J}}
\newcommand{\BK}[0]{\mathbf{K}}
\newcommand{\BL}[0]{\mathbf{L}}
\newcommand{\BM}[0]{\mathbf{M}}
\newcommand{\BN}[0]{\mathbf{N}}
\newcommand{\BO}[0]{\mathbf{O}}
\newcommand{\BP}[0]{\mathbf{P}}
\newcommand{\BQ}[0]{\mathbf{Q}}
\newcommand{\BR}[0]{\mathbf{R}}
\newcommand{\BS}[0]{\mathbf{S}}
\newcommand{\BT}[0]{\mathbf{T}}
\newcommand{\BU}[0]{\mathbf{U}}
\newcommand{\BV}[0]{\mathbf{V}}
\newcommand{\BW}[0]{\mathbf{W}}
\newcommand{\BX}[0]{\mathbf{X}}
\newcommand{\BY}[0]{\mathbf{Y}}
\newcommand{\BZ}[0]{\mathbf{Z}}

\newcommand{\Bzero}[0]{\mathbf{0}}
\newcommand{\Btheta}[0]{\boldsymbol{\theta}}
\newcommand{\Btau}[0]{\boldsymbol{\tau}}
\newcommand{\Bomega}[0]{\boldsymbol{\omega}}

%
% shorthand for unit vectors
%
\newcommand{\acap}[0]{\hat{\Ba}}
\newcommand{\bcap}[0]{\hat{\Bb}}
\newcommand{\ccap}[0]{\hat{\Bc}}
\newcommand{\dcap}[0]{\hat{\Bd}}
\newcommand{\ecap}[0]{\hat{\Be}}
\newcommand{\fcap}[0]{\hat{\Bf}}
\newcommand{\gcap}[0]{\hat{\Bg}}
\newcommand{\hcap}[0]{\hat{\Bh}}
\newcommand{\icap}[0]{\hat{\Bi}}
\newcommand{\jcap}[0]{\hat{\Bj}}
\newcommand{\kcap}[0]{\hat{\Bk}}
\newcommand{\lcap}[0]{\hat{\Bl}}
\newcommand{\mcap}[0]{\hat{\Bm}}
\newcommand{\ncap}[0]{\hat{\Bn}}
\newcommand{\ocap}[0]{\hat{\Bo}}
\newcommand{\pcap}[0]{\hat{\Bp}}
\newcommand{\qcap}[0]{\hat{\Bq}}
\newcommand{\rcap}[0]{\hat{\Br}}
\newcommand{\scap}[0]{\hat{\Bs}}
\newcommand{\tcap}[0]{\hat{\Bt}}
\newcommand{\ucap}[0]{\hat{\Bu}}
\newcommand{\vcap}[0]{\hat{\Bv}}
\newcommand{\wcap}[0]{\hat{\Bw}}
\newcommand{\xcap}[0]{\hat{\Bx}}
\newcommand{\ycap}[0]{\hat{\By}}
\newcommand{\zcap}[0]{\hat{\Bz}}
\newcommand{\thetacap}[0]{\hat{\Btheta}}

%
% to write R^n and C^n in a distinguishable fashion.  Perhaps change this
% to the double lined characters upon figuring out how to do so.
%
\newcommand{\C}[1]{$\mathbb{C}^{#1}$}
\newcommand{\R}[1]{$\mathbb{R}^{#1}$}

%
% various generally useful helpers
%

% derivative of #1 wrt. #2:
\newcommand{\D}[2] {\frac {d#2} {d#1}}

\newcommand{\inv}[1]{\frac{1}{#1}}
\newcommand{\cross}[0]{\times}

\newcommand{\abs}[1]{\lvert{#1}\rvert}
\newcommand{\norm}[1]{\lVert{#1}\rVert}
\newcommand{\innerprod}[2]{\langle{#1}, {#2}\rangle}
\newcommand{\dotprod}[2]{{#1} \cdot {#2}}
\newcommand{\bdotprod}[2]{\left({#1} \cdot {#2}\right)}
\newcommand{\crossprod}[2]{{#1} \cross {#2}}
\newcommand{\tripleprod}[3]{\dotprod{\left(\crossprod{#1}{#2}\right)}{#3}}

\DeclareMathOperator{\Proj}{Proj}
\DeclareMathOperator{\Span}{span}
\DeclareMathOperator{\Sgn}{sgn}
\DeclareMathOperator{\Area}{Area}
\DeclareMathOperator{\Volume}{Volume}

%
% A few miscellaneous things specific to this document
%
\newcommand{\crossop}[1]{\crossprod{#1}{}}

% R2 vector.
\newcommand{\VectorTwo}[2]{
\begin{bmatrix}
 {#1} \\
 {#2}
\end{bmatrix}
}

\newcommand{\VectorN}[1]{
\begin{bmatrix}
{#1}_1 \\
{#1}_2 \\
\vdots \\
{#1}_N \\
\end{bmatrix}
}

\newcommand{\DETuvij}[4]{
\begin{vmatrix}
 {#1}_{#3} & {#1}_{#4} \\
 {#2}_{#3} & {#2}_{#4}
\end{vmatrix}
}

\newcommand{\DETuvwijk}[6]{
\begin{vmatrix}
 {#1}_{#4} & {#1}_{#5} & {#1}_{#6} \\
 {#2}_{#4} & {#2}_{#5} & {#2}_{#6} \\
 {#3}_{#4} & {#3}_{#5} & {#3}_{#6}
\end{vmatrix}
}

\newcommand{\DETuvwxijkl}[8]{
\begin{vmatrix}
 {#1}_{#5} & {#1}_{#6} & {#1}_{#7} & {#1}_{#8} \\
 {#2}_{#5} & {#2}_{#6} & {#2}_{#7} & {#2}_{#8} \\
 {#3}_{#5} & {#3}_{#6} & {#3}_{#7} & {#3}_{#8} \\
 {#4}_{#5} & {#4}_{#6} & {#4}_{#7} & {#4}_{#8} \\
\end{vmatrix}
}

%\newcommand{\DETuvwxyijklm}[10]{
%\begin{vmatrix}
% {#1}_{#6} & {#1}_{#7} & {#1}_{#8} & {#1}_{#9} & {#1}_{#10} \\
% {#2}_{#6} & {#2}_{#7} & {#2}_{#8} & {#2}_{#9} & {#2}_{#10} \\
% {#3}_{#6} & {#3}_{#7} & {#3}_{#8} & {#3}_{#9} & {#3}_{#10} \\
% {#4}_{#6} & {#4}_{#7} & {#4}_{#8} & {#4}_{#9} & {#4}_{#10} \\
% {#5}_{#6} & {#5}_{#7} & {#5}_{#8} & {#5}_{#9} & {#5}_{#10}
%\end{vmatrix}
%}

% R3 vector.
\newcommand{\VectorThree}[3]{
\begin{bmatrix}
 {#1} \\
 {#2} \\
 {#3}
\end{bmatrix}
}



\author{Peeter Joot}
\email{peeter.joot@gmail.com}

%\documentclass[]{eliblogwidescreen}

\usepackage{amsmath}
\usepackage{mathpazo}

%
% shorthand for bold symbols, convenient for vectors and matrices
%
\newcommand{\Ba}[0]{\mathbf{a}}
\newcommand{\Bb}[0]{\mathbf{b}}
\newcommand{\Bc}[0]{\mathbf{c}}
\newcommand{\Bd}[0]{\mathbf{d}}
\newcommand{\Be}[0]{\mathbf{e}}
\newcommand{\Bf}[0]{\mathbf{f}}
\newcommand{\Bg}[0]{\mathbf{g}}
\newcommand{\Bh}[0]{\mathbf{h}}
\newcommand{\Bi}[0]{\mathbf{i}}
\newcommand{\Bj}[0]{\mathbf{j}}
\newcommand{\Bk}[0]{\mathbf{k}}
\newcommand{\Bl}[0]{\mathbf{l}}
\newcommand{\Bm}[0]{\mathbf{m}}
\newcommand{\Bn}[0]{\mathbf{n}}
\newcommand{\Bo}[0]{\mathbf{o}}
\newcommand{\Bp}[0]{\mathbf{p}}
\newcommand{\Bq}[0]{\mathbf{q}}
\newcommand{\Br}[0]{\mathbf{r}}
\newcommand{\Bs}[0]{\mathbf{s}}
\newcommand{\Bt}[0]{\mathbf{t}}
\newcommand{\Bu}[0]{\mathbf{u}}
\newcommand{\Bv}[0]{\mathbf{v}}
\newcommand{\Bw}[0]{\mathbf{w}}
\newcommand{\Bx}[0]{\mathbf{x}}
\newcommand{\By}[0]{\mathbf{y}}
\newcommand{\Bz}[0]{\mathbf{z}}
\newcommand{\BA}[0]{\mathbf{A}}
\newcommand{\BB}[0]{\mathbf{B}}
\newcommand{\BC}[0]{\mathbf{C}}
\newcommand{\BD}[0]{\mathbf{D}}
\newcommand{\BE}[0]{\mathbf{E}}
\newcommand{\BF}[0]{\mathbf{F}}
\newcommand{\BG}[0]{\mathbf{G}}
\newcommand{\BH}[0]{\mathbf{H}}
\newcommand{\BI}[0]{\mathbf{I}}
\newcommand{\BJ}[0]{\mathbf{J}}
\newcommand{\BK}[0]{\mathbf{K}}
\newcommand{\BL}[0]{\mathbf{L}}
\newcommand{\BM}[0]{\mathbf{M}}
\newcommand{\BN}[0]{\mathbf{N}}
\newcommand{\BO}[0]{\mathbf{O}}
\newcommand{\BP}[0]{\mathbf{P}}
\newcommand{\BQ}[0]{\mathbf{Q}}
\newcommand{\BR}[0]{\mathbf{R}}
\newcommand{\BS}[0]{\mathbf{S}}
\newcommand{\BT}[0]{\mathbf{T}}
\newcommand{\BU}[0]{\mathbf{U}}
\newcommand{\BV}[0]{\mathbf{V}}
\newcommand{\BW}[0]{\mathbf{W}}
\newcommand{\BX}[0]{\mathbf{X}}
\newcommand{\BY}[0]{\mathbf{Y}}
\newcommand{\BZ}[0]{\mathbf{Z}}

\newcommand{\Bzero}[0]{\mathbf{0}}
\newcommand{\Btheta}[0]{\boldsymbol{\theta}}
\newcommand{\Btau}[0]{\boldsymbol{\tau}}
\newcommand{\Bomega}[0]{\boldsymbol{\omega}}

%
% shorthand for unit vectors
%
\newcommand{\acap}[0]{\hat{\Ba}}
\newcommand{\bcap}[0]{\hat{\Bb}}
\newcommand{\ccap}[0]{\hat{\Bc}}
\newcommand{\dcap}[0]{\hat{\Bd}}
\newcommand{\ecap}[0]{\hat{\Be}}
\newcommand{\fcap}[0]{\hat{\Bf}}
\newcommand{\gcap}[0]{\hat{\Bg}}
\newcommand{\hcap}[0]{\hat{\Bh}}
\newcommand{\icap}[0]{\hat{\Bi}}
\newcommand{\jcap}[0]{\hat{\Bj}}
\newcommand{\kcap}[0]{\hat{\Bk}}
\newcommand{\lcap}[0]{\hat{\Bl}}
\newcommand{\mcap}[0]{\hat{\Bm}}
\newcommand{\ncap}[0]{\hat{\Bn}}
\newcommand{\ocap}[0]{\hat{\Bo}}
\newcommand{\pcap}[0]{\hat{\Bp}}
\newcommand{\qcap}[0]{\hat{\Bq}}
\newcommand{\rcap}[0]{\hat{\Br}}
\newcommand{\scap}[0]{\hat{\Bs}}
\newcommand{\tcap}[0]{\hat{\Bt}}
\newcommand{\ucap}[0]{\hat{\Bu}}
\newcommand{\vcap}[0]{\hat{\Bv}}
\newcommand{\wcap}[0]{\hat{\Bw}}
\newcommand{\xcap}[0]{\hat{\Bx}}
\newcommand{\ycap}[0]{\hat{\By}}
\newcommand{\zcap}[0]{\hat{\Bz}}
\newcommand{\thetacap}[0]{\hat{\Btheta}}

%
% to write R^n and C^n in a distinguishable fashion.  Perhaps change this
% to the double lined characters upon figuring out how to do so.
%
\newcommand{\C}[1]{$\mathbb{C}^{#1}$}
\newcommand{\R}[1]{$\mathbb{R}^{#1}$}

%
% various generally useful helpers
%

% derivative of #1 wrt. #2:
\newcommand{\D}[2] {\frac {d#2} {d#1}}

\newcommand{\inv}[1]{\frac{1}{#1}}
\newcommand{\cross}[0]{\times}

\newcommand{\abs}[1]{\lvert{#1}\rvert}
\newcommand{\norm}[1]{\lVert{#1}\rVert}
\newcommand{\innerprod}[2]{\langle{#1}, {#2}\rangle}
\newcommand{\dotprod}[2]{{#1} \cdot {#2}}
\newcommand{\bdotprod}[2]{\left({#1} \cdot {#2}\right)}
\newcommand{\crossprod}[2]{{#1} \cross {#2}}
\newcommand{\tripleprod}[3]{\dotprod{\left(\crossprod{#1}{#2}\right)}{#3}}

\DeclareMathOperator{\Proj}{Proj}
\DeclareMathOperator{\Span}{span}
\DeclareMathOperator{\Sgn}{sgn}
\DeclareMathOperator{\Area}{Area}
\DeclareMathOperator{\Volume}{Volume}

%
% A few miscellaneous things specific to this document
%
\newcommand{\crossop}[1]{\crossprod{#1}{}}

% R2 vector.
\newcommand{\VectorTwo}[2]{
\begin{bmatrix}
 {#1} \\
 {#2}
\end{bmatrix}
}

\newcommand{\VectorN}[1]{
\begin{bmatrix}
{#1}_1 \\
{#1}_2 \\
\vdots \\
{#1}_N \\
\end{bmatrix}
}

\newcommand{\DETuvij}[4]{
\begin{vmatrix}
 {#1}_{#3} & {#1}_{#4} \\
 {#2}_{#3} & {#2}_{#4}
\end{vmatrix}
}

\newcommand{\DETuvwijk}[6]{
\begin{vmatrix}
 {#1}_{#4} & {#1}_{#5} & {#1}_{#6} \\
 {#2}_{#4} & {#2}_{#5} & {#2}_{#6} \\
 {#3}_{#4} & {#3}_{#5} & {#3}_{#6}
\end{vmatrix}
}

\newcommand{\DETuvwxijkl}[8]{
\begin{vmatrix}
 {#1}_{#5} & {#1}_{#6} & {#1}_{#7} & {#1}_{#8} \\
 {#2}_{#5} & {#2}_{#6} & {#2}_{#7} & {#2}_{#8} \\
 {#3}_{#5} & {#3}_{#6} & {#3}_{#7} & {#3}_{#8} \\
 {#4}_{#5} & {#4}_{#6} & {#4}_{#7} & {#4}_{#8} \\
\end{vmatrix}
}

%\newcommand{\DETuvwxyijklm}[10]{
%\begin{vmatrix}
% {#1}_{#6} & {#1}_{#7} & {#1}_{#8} & {#1}_{#9} & {#1}_{#10} \\
% {#2}_{#6} & {#2}_{#7} & {#2}_{#8} & {#2}_{#9} & {#2}_{#10} \\
% {#3}_{#6} & {#3}_{#7} & {#3}_{#8} & {#3}_{#9} & {#3}_{#10} \\
% {#4}_{#6} & {#4}_{#7} & {#4}_{#8} & {#4}_{#9} & {#4}_{#10} \\
% {#5}_{#6} & {#5}_{#7} & {#5}_{#8} & {#5}_{#9} & {#5}_{#10}
%\end{vmatrix}
%}

% R3 vector.
\newcommand{\VectorThree}[3]{
\begin{bmatrix}
 {#1} \\
 {#2} \\
 {#3}
\end{bmatrix}
}



\author{Peeter Joot}
\email{peeter.joot@gmail.com}


\chapter{PHY356 Problem Set III.}
\label{chap:qmIproblemSet3}
%\useCCL
%\blogpage{http://sites.google.com/site/peeterjoot/math2010/qmIproblemSet3.pdf}
\date{Oct 23, 2010}
\revisionInfo{qmIproblemSet3.tex}

\beginArtNoToc
\section{Problem 1.}
\subsection{Statement}

A particle of mass $m$ is free to move along the x-direction such that $V(X)=0$. The state of the system is represented by the wavefunction Eq. (4.74)

\begin{align}\label{eqn:qmIproblemSet3:1}
\psi(x,t) = \int_{-\infty}^\infty dk e^{i k x} e^{- i \omega t} f(k)
\end{align}

with $f(k)$ given by Eq. (4.59). 

\begin{align}\label{eqn:qmIproblemSet3:2}
f(k) &= N e^{-\alpha k^2}
\end{align}

\begin{itemize}
\item 
(a) What is the group velocity associated with this state? 
\item 
(b) What is the probability for measuring the particle at position $x=x_0>0$ at time $t=t_0>0$? 
\item 
(c) What is the probability per unit length for measuring the particle at position $x=x_0>0$ at time $t=t_0>0$? 
\item 
(d) Explain the physical meaning of the above results.
\end{itemize}

\subsection{Solution}
\subsubsection{(a).  group velocity.}

To calculate the group velocity we need to know the dependence of $\omega$ on $k$.

Let's step back and consider the time evolution action on $\psi(x,0)$.  For the free particle case we have

\begin{align}\label{eqn:qmIproblemSet3:101}
H = \frac{\Bp^2}{2m} = -\frac{\hbar^2}{2m} \partial_{xx}
\end{align}

so

\begin{align*}
-\frac{i t}{\hbar} H \psi(x,0) 
&= 
\frac{i t \hbar }{2m} 
N \int_{-\infty}^\infty dk (i k)^2 e^{i k x - \alpha k^2} \\
&= 
N \int_{-\infty}^\infty dk \frac{-i t \hbar k^2}{2m} e^{i k x - \alpha k^2}
\end{align*}

Each successive application of $-iHt/\hbar$ will introduce another power of $-it\hbar k^2/2 m$, so once we sum all the terms of the exponential series $U(t) = e^{-iHt/\hbar}$ we have

\begin{align}\label{eqn:qmIproblemSet3:102}
\psi(x,t) =
N \int_{-\infty}^\infty dk \exp\left( 
\frac{-i t \hbar k^2}{2m} + i k x - \alpha k^2 \right).
\end{align}

Comparing with \ref{eqn:qmIproblemSet3:1} we find
\begin{align}\label{eqn:qmIproblemSet3:103}
\omega(k) = \frac{\hbar k^2}{2m}.
\end{align}

This completes this section of the problem since we are now able to calculate the group velocity 
\begin{align}\label{eqn:qmIproblemSet3:104}
v_g = \PD{k}{\omega(k)} = \frac{\hbar k}{m}.
\end{align}

\subsection{(b). What is the probability for measuring the particle at position $x=x_0>0$ at time $t=t_0>0$? }

\subsection{(c). What is the probability per unit length for measuring the particle at position $x=x_0>0$ at time $t=t_0>0$? }
\subsection{(d). Explain the physical meaning of the above results.}
\subsection{scratch notes. }
We can observe that the normalization does not provide this dependence since we have for time $t=0$

\begin{align}\label{eqn:qmIproblemSet3:100}
\psi(x,0) = N \int_{-\infty}^\infty dk e^{i k x - \alpha k^2},
\end{align}

and from this find $N = (\pi/2 \alpha)^{1/4}$.

\section{Problem 2.}

\subsection{Statement}
A particle with intrinsic angular momentum or spin $s=1/2$ is prepared in the spin-up with respect to the z-direction state $\ket{f}=\ket{z+}$. Determine

\begin{align}\label{eqn:qmIproblemSet3:3}
\left(\bra{f} \left( S_z - \bra{f} S_z \ket{f} \BOne \right)^2 \ket{f} \right)^{1/2}
\end{align}

and 

\begin{align}\label{eqn:qmIproblemSet3:4}
\left(\bra{f} \left( S_x - \bra{f} S_x \ket{f} \BOne \right)^2 \ket{f} \right)^{1/2}
\end{align}

and explain what these relations say about the system.

\subsection{Solution}

%\EndArticle
\EndNoBibArticle

%
% Copyright � 2012 Peeter Joot.  All Rights Reserved.
% Licenced as described in the file LICENSE under the root directory of this GIT repository.
%

%\chapter{PHY356 Problem Set 4}
\label{chap:qmIproblemSet4}
%\blogpage{http://sites.google.com/site/peeterjoot/math2010/qmIproblemSet4.pdf}
%\date{Nov 16, 2010}

\makeproblem{ps 4, p1.}{problem:qmIproblemSet4:1}{
Is it possible to derive the eigenvalues and eigenvectors presented in Section 8.2 from those in Section 8.1.2?  What does this say about the potential energy operator in these two situations?

For reference 8.1.2 was a finite potential barrier, \(V(x) = V_0, \Abs{x} > a\), and zero in the interior of the well.  This had trigonometric solutions in the interior, and died off exponentially past the boundary of the well.

On the other hand, 8.2 was a delta function potential \(V(x) = -g \delta(x)\), which had the solution \(u(x) = \sqrt{\beta} e^{-\beta \Abs{x}}\), where \(\beta = m g/\Hbar^2\).

} % problem

\makeanswer{problem:qmIproblemSet4:1}{

The pair of figures in the text \citep{desai2009quantum} for these potentials does not make it clear that there are possibly any similarities.  The attractive delta function potential is not illustrated (although the delta function is, but with opposite sign), and the scaling and the reference energy levels are different.  Let us illustrate these using the same reference energy level and sign conventions to make the similarities more obvious.

\imageFigure{../../figures/phy356/FiniteWellPotential}{8.1.2 Finite Well potential (with energy shifted downwards by \(V_0\))}{fig:FiniteWellPotential}{0.4}

\imageFigure{../../figures/phy356/deltaFunctionPotential}{8.2 Delta function potential}{fig:deltaFunctionPotential}{0.4}

The physics is not changed by picking a different point for the reference energy level, so let us compare the two potentials, and their solutions using \(V(x) = 0\) outside of the well for both cases.  The method used to solve the finite well problem in the text is hard to follow, so re-doing this from scratch in a slightly tidier way does not hurt.

Schr\"{o}dinger's equation for the finite well, in the \(\Abs{x} > a\) region is

\begin{equation}\label{eqn:qmIproblemSet4:110}
\begin{aligned}
-\frac{\Hbar^2}{2m} u'' = E u = - E_B u,
\end{aligned}
\end{equation}

where a positive bound state energy \(E_B = -E > 0\) has been introduced.

Writing
\begin{equation}\label{eqn:qmIproblemSet4:115}
\begin{aligned}
\beta = \sqrt{\frac{2 m E_B}{\Hbar^2}},
\end{aligned}
\end{equation}

the wave functions outside of the well are
\begin{equation}\label{eqn:qmIproblemSet4:120}
\begin{aligned}
u(x) =
\left\{
\begin{array}{l l}
u(-a) e^{\beta(x+a)} &\quad \mbox{\(x < -a\)} \\
u(a) e^{-\beta(x-a)} &\quad \mbox{\(x > a\)} \\
\end{array}
\right.
\end{aligned}
\end{equation}

Within the well Schr\"{o}dinger's equation is
\begin{equation}\label{eqn:qmIproblemSet4:125}
\begin{aligned}
-\frac{\Hbar^2}{2m} u'' - V_0 u = E u = - E_B u,
\end{aligned}
\end{equation}

or
\begin{equation}\label{eqn:qmIproblemSet4:126}
\begin{aligned}
\frac{\Hbar^2}{2m} u'' = - \frac{2m}{\Hbar^2} (V_0 - E_B) u,
\end{aligned}
\end{equation}

Noting that the bound state energies are the \(E_B < V_0\) values, let \(\alpha^2 = 2m (V_0 - E_B)/\Hbar^2\), so that the solutions are of the form
\begin{equation}\label{eqn:qmIproblemSet4:130}
\begin{aligned}
u(x) = A e^{i\alpha x} + B e^{-i\alpha x}.
\end{aligned}
\end{equation}

As was done for the wave functions outside of the well, the normalization constants can be expressed in terms of the values of the wave functions on the boundary.  That provides a pair of equations to solve

\begin{equation}\label{eqn:qmIproblemSet4:135}
\begin{aligned}
\begin{bmatrix}
u(a) \\
u(-a)
\end{bmatrix}
=
\begin{bmatrix}
e^{i \alpha a} & e^{-i \alpha a} \\
e^{-i \alpha a} & e^{i \alpha a}
\end{bmatrix}
\begin{bmatrix}
A \\
B
\end{bmatrix}.
\end{aligned}
\end{equation}

Inverting this and substitution back into \eqnref{eqn:qmIproblemSet4:130} yields
\begin{equation}\label{eqn:qmIproblemSet4:175}
\begin{aligned}
u(x)
&=
\begin{bmatrix}
e^{i\alpha x} & e^{-i\alpha x}
\end{bmatrix}
\begin{bmatrix}
A \\
B
\end{bmatrix} \\
&=
\begin{bmatrix}
e^{i\alpha x} & e^{-i\alpha x}
\end{bmatrix}
\inv{e^{2 i \alpha a} - e^{-2 i \alpha a}}
\begin{bmatrix}
e^{i \alpha a} & -e^{-i \alpha a} \\
-e^{-i \alpha a} & e^{i \alpha a}
\end{bmatrix}
\begin{bmatrix}
u(a) \\
u(-a)
\end{bmatrix} \\
&=
\begin{bmatrix}
\frac{\sin(\alpha (a + x))}{\sin(2 \alpha a)} &
\frac{\sin(\alpha (a - x))}{\sin(2 \alpha a)}
\end{bmatrix}
\begin{bmatrix}
u(a) \\
u(-a)
\end{bmatrix}.
\end{aligned}
\end{equation}

Expanding the last of these matrix products the wave function is close to completely specified.

\begin{equation}\label{eqn:qmIproblemSet4:140}
u(x) =
\left\{
\begin{array}{l l}
u(-a) e^{\beta(x+a)}
 & \quad \mbox{\(x < -a\)} \\
u(a) \frac{\sin(\alpha (a + x))}{\sin(2 \alpha a)} +
u(-a) \frac{\sin(\alpha (a - x))}{\sin(2 \alpha a)}
 & \quad \mbox{\(\Abs{x} < a\)} \\
u(a) e^{-\beta(x-a)}
 & \quad \mbox{\(x > a\)} \\
\end{array}
\right.
\end{equation}

There are still two unspecified constants \(u(\pm a)\) and the constraints on \(E_B\) have not been determined (both \(\alpha\) and \(\beta\) are functions of that energy level).  It should be possible to eliminate at least one of the \(u(\pm a)\) by computing the wavefunction normalization, and since the well is being narrowed the \(\alpha\) term will not be relevant.  Since only the vanishingly narrow case where \(a \rightarrow 0, x \in [-a,a]\) is of interest, the wave function in that interval approaches

\begin{equation}\label{eqn:qmIproblemSet4:145}
u(x) \rightarrow \inv{2} (u(a) + u(-a)) + \frac{x}{2} ( u(a) - u(-a) ) \rightarrow \inv{2} (u(a) + u(-a)).
\end{equation}

Since no discontinuity is expected this is just \(u(a) = u(-a)\).  Let us write \(\lim_{a\rightarrow 0} u(a) = A\) for short, and the limited width well wave function becomes

\begin{equation}\label{eqn:qmIproblemSet4:150}
u(x) =
\left\{
\begin{array}{l l}
A e^{\beta x}
 & \quad \mbox{\(x < 0\)} \\
A e^{-\beta x}
 & \quad \mbox{\(x > 0\)} \\
\end{array}
\right.
\end{equation}

This is now the same form as the delta function potential, and normalization also gives \(A = \sqrt{\beta}\).

One task remains before the attractive delta function potential can be considered a limiting case for the finite well, since the relation between \(a, V_0\), and \(g\) has not been established.  To do so integrate the Schr\"{o}dinger equation over the infinitesimal range \([-a,a]\).  This was done in the text for the delta function potential, and that provided the relation

\begin{equation}\label{eqn:qmIproblemSet4:155a}
\beta = \frac{mg}{\Hbar^2}
\end{equation}

For the finite well this is

\begin{equation}\label{eqn:qmIproblemSet4:151}
\int_{-a}^a -\frac{\Hbar^2}{2m} u'' - V_0 \int_{-a}^a u = -E_B \int_{-a}^a u \\
\end{equation}

In the limit as \(a \rightarrow 0\) this is
\begin{equation}\label{eqn:qmIproblemSet4:152}
\frac{\Hbar^2}{2m} (u'(a) - u'(-a)) + V_0 2 a u(0) = 2 E_B a u(0).
\end{equation}

Some care is required with the \(V_0 a\) term since \(a \rightarrow 0\) as \(V_0 \rightarrow \infty\), but the \(E_B\) term is unambiguously killed, leaving
\begin{equation}\label{eqn:qmIproblemSet4:153}
\frac{\Hbar^2}{2m} u(0) (-2\beta e^{-\beta a}) = -V_0 2 a u(0).
\end{equation}

The exponential vanishes in the limit and leaves

\begin{equation}\label{eqn:qmIproblemSet4:155}
\beta = \frac{m (2 a) V_0}{\Hbar^2}
\end{equation}

Comparing to \eqnref{eqn:qmIproblemSet4:155a} from the attractive delta function completes the problem.  The conclusion is that when the finite well is narrowed with \(a \rightarrow 0\), also letting \(V_0 \rightarrow \infty\) such that the absolute area of the well \(g = (2 a) V_0\) is maintained, the finite potential well produces exactly the attractive delta function wave function and associated bound state energy.

\paragraph{Grading notes}

Lost \(3/20\) marks, all in the first question.

I did not show that \(u(a) = u(-a)\).

I did not explain why the odd terms disappear in \eqnref{eqn:qmIproblemSet4:145}.

I also did not get agreement with my statement that ``but the \(E_B\) term is unambiguously killed'', where I have assumed that it remains finite.  Since \(V_0 \rightarrow \infty\), \(E_B\) could tend to infinity too.

\paragraph{Some references}

Some references that I found helpful to provide some of the context for WHY to consider the delta function potential in the first place are \citep{wiki:DeltaPotential}, \citep{deltaFunctionModelOfACrystal}, \citep{deltaFunctionPotentials}, \citep{theDeltaFunctionPotential}.

% can no longer access:
%\href{http://www.phys.ufl.edu/~rfield/PHY4604/images/Chapter2_20.pdf}{ufl}.

} % answer

\documentclass[]{eliblog}

\usepackage{color}
%\usepackage{txfonts} % for xi

\usepackage{amsmath}
\usepackage{mathpazo}

%
% shorthand for bold symbols, convenient for vectors and matrices
%
\newcommand{\Ba}[0]{\mathbf{a}}
\newcommand{\Bb}[0]{\mathbf{b}}
\newcommand{\Bc}[0]{\mathbf{c}}
\newcommand{\Bd}[0]{\mathbf{d}}
\newcommand{\Be}[0]{\mathbf{e}}
\newcommand{\Bf}[0]{\mathbf{f}}
\newcommand{\Bg}[0]{\mathbf{g}}
\newcommand{\Bh}[0]{\mathbf{h}}
\newcommand{\Bi}[0]{\mathbf{i}}
\newcommand{\Bj}[0]{\mathbf{j}}
\newcommand{\Bk}[0]{\mathbf{k}}
\newcommand{\Bl}[0]{\mathbf{l}}
\newcommand{\Bm}[0]{\mathbf{m}}
\newcommand{\Bn}[0]{\mathbf{n}}
\newcommand{\Bo}[0]{\mathbf{o}}
\newcommand{\Bp}[0]{\mathbf{p}}
\newcommand{\Bq}[0]{\mathbf{q}}
\newcommand{\Br}[0]{\mathbf{r}}
\newcommand{\Bs}[0]{\mathbf{s}}
\newcommand{\Bt}[0]{\mathbf{t}}
\newcommand{\Bu}[0]{\mathbf{u}}
\newcommand{\Bv}[0]{\mathbf{v}}
\newcommand{\Bw}[0]{\mathbf{w}}
\newcommand{\Bx}[0]{\mathbf{x}}
\newcommand{\By}[0]{\mathbf{y}}
\newcommand{\Bz}[0]{\mathbf{z}}
\newcommand{\BA}[0]{\mathbf{A}}
\newcommand{\BB}[0]{\mathbf{B}}
\newcommand{\BC}[0]{\mathbf{C}}
\newcommand{\BD}[0]{\mathbf{D}}
\newcommand{\BE}[0]{\mathbf{E}}
\newcommand{\BF}[0]{\mathbf{F}}
\newcommand{\BG}[0]{\mathbf{G}}
\newcommand{\BH}[0]{\mathbf{H}}
\newcommand{\BI}[0]{\mathbf{I}}
\newcommand{\BJ}[0]{\mathbf{J}}
\newcommand{\BK}[0]{\mathbf{K}}
\newcommand{\BL}[0]{\mathbf{L}}
\newcommand{\BM}[0]{\mathbf{M}}
\newcommand{\BN}[0]{\mathbf{N}}
\newcommand{\BO}[0]{\mathbf{O}}
\newcommand{\BP}[0]{\mathbf{P}}
\newcommand{\BQ}[0]{\mathbf{Q}}
\newcommand{\BR}[0]{\mathbf{R}}
\newcommand{\BS}[0]{\mathbf{S}}
\newcommand{\BT}[0]{\mathbf{T}}
\newcommand{\BU}[0]{\mathbf{U}}
\newcommand{\BV}[0]{\mathbf{V}}
\newcommand{\BW}[0]{\mathbf{W}}
\newcommand{\BX}[0]{\mathbf{X}}
\newcommand{\BY}[0]{\mathbf{Y}}
\newcommand{\BZ}[0]{\mathbf{Z}}

\newcommand{\Bzero}[0]{\mathbf{0}}
\newcommand{\Btheta}[0]{\boldsymbol{\theta}}
\newcommand{\Btau}[0]{\boldsymbol{\tau}}
\newcommand{\Bomega}[0]{\boldsymbol{\omega}}

%
% shorthand for unit vectors
%
\newcommand{\acap}[0]{\hat{\Ba}}
\newcommand{\bcap}[0]{\hat{\Bb}}
\newcommand{\ccap}[0]{\hat{\Bc}}
\newcommand{\dcap}[0]{\hat{\Bd}}
\newcommand{\ecap}[0]{\hat{\Be}}
\newcommand{\fcap}[0]{\hat{\Bf}}
\newcommand{\gcap}[0]{\hat{\Bg}}
\newcommand{\hcap}[0]{\hat{\Bh}}
\newcommand{\icap}[0]{\hat{\Bi}}
\newcommand{\jcap}[0]{\hat{\Bj}}
\newcommand{\kcap}[0]{\hat{\Bk}}
\newcommand{\lcap}[0]{\hat{\Bl}}
\newcommand{\mcap}[0]{\hat{\Bm}}
\newcommand{\ncap}[0]{\hat{\Bn}}
\newcommand{\ocap}[0]{\hat{\Bo}}
\newcommand{\pcap}[0]{\hat{\Bp}}
\newcommand{\qcap}[0]{\hat{\Bq}}
\newcommand{\rcap}[0]{\hat{\Br}}
\newcommand{\scap}[0]{\hat{\Bs}}
\newcommand{\tcap}[0]{\hat{\Bt}}
\newcommand{\ucap}[0]{\hat{\Bu}}
\newcommand{\vcap}[0]{\hat{\Bv}}
\newcommand{\wcap}[0]{\hat{\Bw}}
\newcommand{\xcap}[0]{\hat{\Bx}}
\newcommand{\ycap}[0]{\hat{\By}}
\newcommand{\zcap}[0]{\hat{\Bz}}
\newcommand{\thetacap}[0]{\hat{\Btheta}}

%
% to write R^n and C^n in a distinguishable fashion.  Perhaps change this
% to the double lined characters upon figuring out how to do so.
%
\newcommand{\C}[1]{$\mathbb{C}^{#1}$}
\newcommand{\R}[1]{$\mathbb{R}^{#1}$}

%
% various generally useful helpers
%

% derivative of #1 wrt. #2:
\newcommand{\D}[2] {\frac {d#2} {d#1}}

\newcommand{\inv}[1]{\frac{1}{#1}}
\newcommand{\cross}[0]{\times}

\newcommand{\abs}[1]{\lvert{#1}\rvert}
\newcommand{\norm}[1]{\lVert{#1}\rVert}
\newcommand{\innerprod}[2]{\langle{#1}, {#2}\rangle}
\newcommand{\dotprod}[2]{{#1} \cdot {#2}}
\newcommand{\bdotprod}[2]{\left({#1} \cdot {#2}\right)}
\newcommand{\crossprod}[2]{{#1} \cross {#2}}
\newcommand{\tripleprod}[3]{\dotprod{\left(\crossprod{#1}{#2}\right)}{#3}}

\DeclareMathOperator{\Proj}{Proj}
\DeclareMathOperator{\Span}{span}
\DeclareMathOperator{\Sgn}{sgn}
\DeclareMathOperator{\Area}{Area}
\DeclareMathOperator{\Volume}{Volume}

%
% A few miscellaneous things specific to this document
%
\newcommand{\crossop}[1]{\crossprod{#1}{}}

% R2 vector.
\newcommand{\VectorTwo}[2]{
\begin{bmatrix}
 {#1} \\
 {#2}
\end{bmatrix}
}

\newcommand{\VectorN}[1]{
\begin{bmatrix}
{#1}_1 \\
{#1}_2 \\
\vdots \\
{#1}_N \\
\end{bmatrix}
}

\newcommand{\DETuvij}[4]{
\begin{vmatrix}
 {#1}_{#3} & {#1}_{#4} \\
 {#2}_{#3} & {#2}_{#4}
\end{vmatrix}
}

\newcommand{\DETuvwijk}[6]{
\begin{vmatrix}
 {#1}_{#4} & {#1}_{#5} & {#1}_{#6} \\
 {#2}_{#4} & {#2}_{#5} & {#2}_{#6} \\
 {#3}_{#4} & {#3}_{#5} & {#3}_{#6}
\end{vmatrix}
}

\newcommand{\DETuvwxijkl}[8]{
\begin{vmatrix}
 {#1}_{#5} & {#1}_{#6} & {#1}_{#7} & {#1}_{#8} \\
 {#2}_{#5} & {#2}_{#6} & {#2}_{#7} & {#2}_{#8} \\
 {#3}_{#5} & {#3}_{#6} & {#3}_{#7} & {#3}_{#8} \\
 {#4}_{#5} & {#4}_{#6} & {#4}_{#7} & {#4}_{#8} \\
\end{vmatrix}
}

%\newcommand{\DETuvwxyijklm}[10]{
%\begin{vmatrix}
% {#1}_{#6} & {#1}_{#7} & {#1}_{#8} & {#1}_{#9} & {#1}_{#10} \\
% {#2}_{#6} & {#2}_{#7} & {#2}_{#8} & {#2}_{#9} & {#2}_{#10} \\
% {#3}_{#6} & {#3}_{#7} & {#3}_{#8} & {#3}_{#9} & {#3}_{#10} \\
% {#4}_{#6} & {#4}_{#7} & {#4}_{#8} & {#4}_{#9} & {#4}_{#10} \\
% {#5}_{#6} & {#5}_{#7} & {#5}_{#8} & {#5}_{#9} & {#5}_{#10}
%\end{vmatrix}
%}

% R3 vector.
\newcommand{\VectorThree}[3]{
\begin{bmatrix}
 {#1} \\
 {#2} \\
 {#3}
\end{bmatrix}
}



\author{Peeter Joot}
\email{peeter.joot@utoronto.ca, 920798560}
%%
% Copyright � 2015 Peeter Joot.  All Rights Reserved.
% Licenced as described in the file LICENSE under the root directory of this GIT repository.
%
\documentclass[]{eliblog}

\usepackage{amsmath}
\usepackage{mathpazo}

%
% shorthand for bold symbols, convenient for vectors and matrices
%
\newcommand{\Ba}[0]{\mathbf{a}}
\newcommand{\Bb}[0]{\mathbf{b}}
\newcommand{\Bc}[0]{\mathbf{c}}
\newcommand{\Bd}[0]{\mathbf{d}}
\newcommand{\Be}[0]{\mathbf{e}}
\newcommand{\Bf}[0]{\mathbf{f}}
\newcommand{\Bg}[0]{\mathbf{g}}
\newcommand{\Bh}[0]{\mathbf{h}}
\newcommand{\Bi}[0]{\mathbf{i}}
\newcommand{\Bj}[0]{\mathbf{j}}
\newcommand{\Bk}[0]{\mathbf{k}}
\newcommand{\Bl}[0]{\mathbf{l}}
\newcommand{\Bm}[0]{\mathbf{m}}
\newcommand{\Bn}[0]{\mathbf{n}}
\newcommand{\Bo}[0]{\mathbf{o}}
\newcommand{\Bp}[0]{\mathbf{p}}
\newcommand{\Bq}[0]{\mathbf{q}}
\newcommand{\Br}[0]{\mathbf{r}}
\newcommand{\Bs}[0]{\mathbf{s}}
\newcommand{\Bt}[0]{\mathbf{t}}
\newcommand{\Bu}[0]{\mathbf{u}}
\newcommand{\Bv}[0]{\mathbf{v}}
\newcommand{\Bw}[0]{\mathbf{w}}
\newcommand{\Bx}[0]{\mathbf{x}}
\newcommand{\By}[0]{\mathbf{y}}
\newcommand{\Bz}[0]{\mathbf{z}}
\newcommand{\BA}[0]{\mathbf{A}}
\newcommand{\BB}[0]{\mathbf{B}}
\newcommand{\BC}[0]{\mathbf{C}}
\newcommand{\BD}[0]{\mathbf{D}}
\newcommand{\BE}[0]{\mathbf{E}}
\newcommand{\BF}[0]{\mathbf{F}}
\newcommand{\BG}[0]{\mathbf{G}}
\newcommand{\BH}[0]{\mathbf{H}}
\newcommand{\BI}[0]{\mathbf{I}}
\newcommand{\BJ}[0]{\mathbf{J}}
\newcommand{\BK}[0]{\mathbf{K}}
\newcommand{\BL}[0]{\mathbf{L}}
\newcommand{\BM}[0]{\mathbf{M}}
\newcommand{\BN}[0]{\mathbf{N}}
\newcommand{\BO}[0]{\mathbf{O}}
\newcommand{\BP}[0]{\mathbf{P}}
\newcommand{\BQ}[0]{\mathbf{Q}}
\newcommand{\BR}[0]{\mathbf{R}}
\newcommand{\BS}[0]{\mathbf{S}}
\newcommand{\BT}[0]{\mathbf{T}}
\newcommand{\BU}[0]{\mathbf{U}}
\newcommand{\BV}[0]{\mathbf{V}}
\newcommand{\BW}[0]{\mathbf{W}}
\newcommand{\BX}[0]{\mathbf{X}}
\newcommand{\BY}[0]{\mathbf{Y}}
\newcommand{\BZ}[0]{\mathbf{Z}}

\newcommand{\Bzero}[0]{\mathbf{0}}
\newcommand{\Btheta}[0]{\boldsymbol{\theta}}
\newcommand{\Btau}[0]{\boldsymbol{\tau}}
\newcommand{\Bomega}[0]{\boldsymbol{\omega}}

%
% shorthand for unit vectors
%
\newcommand{\acap}[0]{\hat{\Ba}}
\newcommand{\bcap}[0]{\hat{\Bb}}
\newcommand{\ccap}[0]{\hat{\Bc}}
\newcommand{\dcap}[0]{\hat{\Bd}}
\newcommand{\ecap}[0]{\hat{\Be}}
\newcommand{\fcap}[0]{\hat{\Bf}}
\newcommand{\gcap}[0]{\hat{\Bg}}
\newcommand{\hcap}[0]{\hat{\Bh}}
\newcommand{\icap}[0]{\hat{\Bi}}
\newcommand{\jcap}[0]{\hat{\Bj}}
\newcommand{\kcap}[0]{\hat{\Bk}}
\newcommand{\lcap}[0]{\hat{\Bl}}
\newcommand{\mcap}[0]{\hat{\Bm}}
\newcommand{\ncap}[0]{\hat{\Bn}}
\newcommand{\ocap}[0]{\hat{\Bo}}
\newcommand{\pcap}[0]{\hat{\Bp}}
\newcommand{\qcap}[0]{\hat{\Bq}}
\newcommand{\rcap}[0]{\hat{\Br}}
\newcommand{\scap}[0]{\hat{\Bs}}
\newcommand{\tcap}[0]{\hat{\Bt}}
\newcommand{\ucap}[0]{\hat{\Bu}}
\newcommand{\vcap}[0]{\hat{\Bv}}
\newcommand{\wcap}[0]{\hat{\Bw}}
\newcommand{\xcap}[0]{\hat{\Bx}}
\newcommand{\ycap}[0]{\hat{\By}}
\newcommand{\zcap}[0]{\hat{\Bz}}
\newcommand{\thetacap}[0]{\hat{\Btheta}}

%
% to write R^n and C^n in a distinguishable fashion.  Perhaps change this
% to the double lined characters upon figuring out how to do so.
%
\newcommand{\C}[1]{$\mathbb{C}^{#1}$}
\newcommand{\R}[1]{$\mathbb{R}^{#1}$}

%
% various generally useful helpers
%

% derivative of #1 wrt. #2:
\newcommand{\D}[2] {\frac {d#2} {d#1}}

\newcommand{\inv}[1]{\frac{1}{#1}}
\newcommand{\cross}[0]{\times}

\newcommand{\abs}[1]{\lvert{#1}\rvert}
\newcommand{\norm}[1]{\lVert{#1}\rVert}
\newcommand{\innerprod}[2]{\langle{#1}, {#2}\rangle}
\newcommand{\dotprod}[2]{{#1} \cdot {#2}}
\newcommand{\bdotprod}[2]{\left({#1} \cdot {#2}\right)}
\newcommand{\crossprod}[2]{{#1} \cross {#2}}
\newcommand{\tripleprod}[3]{\dotprod{\left(\crossprod{#1}{#2}\right)}{#3}}

\DeclareMathOperator{\Proj}{Proj}
\DeclareMathOperator{\Span}{span}
\DeclareMathOperator{\Sgn}{sgn}
\DeclareMathOperator{\Area}{Area}
\DeclareMathOperator{\Volume}{Volume}

%
% A few miscellaneous things specific to this document
%
\newcommand{\crossop}[1]{\crossprod{#1}{}}

% R2 vector.
\newcommand{\VectorTwo}[2]{
\begin{bmatrix}
 {#1} \\
 {#2}
\end{bmatrix}
}

\newcommand{\VectorN}[1]{
\begin{bmatrix}
{#1}_1 \\
{#1}_2 \\
\vdots \\
{#1}_N \\
\end{bmatrix}
}

\newcommand{\DETuvij}[4]{
\begin{vmatrix}
 {#1}_{#3} & {#1}_{#4} \\
 {#2}_{#3} & {#2}_{#4}
\end{vmatrix}
}

\newcommand{\DETuvwijk}[6]{
\begin{vmatrix}
 {#1}_{#4} & {#1}_{#5} & {#1}_{#6} \\
 {#2}_{#4} & {#2}_{#5} & {#2}_{#6} \\
 {#3}_{#4} & {#3}_{#5} & {#3}_{#6}
\end{vmatrix}
}

\newcommand{\DETuvwxijkl}[8]{
\begin{vmatrix}
 {#1}_{#5} & {#1}_{#6} & {#1}_{#7} & {#1}_{#8} \\
 {#2}_{#5} & {#2}_{#6} & {#2}_{#7} & {#2}_{#8} \\
 {#3}_{#5} & {#3}_{#6} & {#3}_{#7} & {#3}_{#8} \\
 {#4}_{#5} & {#4}_{#6} & {#4}_{#7} & {#4}_{#8} \\
\end{vmatrix}
}

%\newcommand{\DETuvwxyijklm}[10]{
%\begin{vmatrix}
% {#1}_{#6} & {#1}_{#7} & {#1}_{#8} & {#1}_{#9} & {#1}_{#10} \\
% {#2}_{#6} & {#2}_{#7} & {#2}_{#8} & {#2}_{#9} & {#2}_{#10} \\
% {#3}_{#6} & {#3}_{#7} & {#3}_{#8} & {#3}_{#9} & {#3}_{#10} \\
% {#4}_{#6} & {#4}_{#7} & {#4}_{#8} & {#4}_{#9} & {#4}_{#10} \\
% {#5}_{#6} & {#5}_{#7} & {#5}_{#8} & {#5}_{#9} & {#5}_{#10}
%\end{vmatrix}
%}

% R3 vector.
\newcommand{\VectorThree}[3]{
\begin{bmatrix}
 {#1} \\
 {#2} \\
 {#3}
\end{bmatrix}
}



\author{Peeter Joot}
\email{peeter.joot@gmail.com}

%\documentclass[]{eliblogwidescreen}

\usepackage{amsmath}
\usepackage{mathpazo}

%
% shorthand for bold symbols, convenient for vectors and matrices
%
\newcommand{\Ba}[0]{\mathbf{a}}
\newcommand{\Bb}[0]{\mathbf{b}}
\newcommand{\Bc}[0]{\mathbf{c}}
\newcommand{\Bd}[0]{\mathbf{d}}
\newcommand{\Be}[0]{\mathbf{e}}
\newcommand{\Bf}[0]{\mathbf{f}}
\newcommand{\Bg}[0]{\mathbf{g}}
\newcommand{\Bh}[0]{\mathbf{h}}
\newcommand{\Bi}[0]{\mathbf{i}}
\newcommand{\Bj}[0]{\mathbf{j}}
\newcommand{\Bk}[0]{\mathbf{k}}
\newcommand{\Bl}[0]{\mathbf{l}}
\newcommand{\Bm}[0]{\mathbf{m}}
\newcommand{\Bn}[0]{\mathbf{n}}
\newcommand{\Bo}[0]{\mathbf{o}}
\newcommand{\Bp}[0]{\mathbf{p}}
\newcommand{\Bq}[0]{\mathbf{q}}
\newcommand{\Br}[0]{\mathbf{r}}
\newcommand{\Bs}[0]{\mathbf{s}}
\newcommand{\Bt}[0]{\mathbf{t}}
\newcommand{\Bu}[0]{\mathbf{u}}
\newcommand{\Bv}[0]{\mathbf{v}}
\newcommand{\Bw}[0]{\mathbf{w}}
\newcommand{\Bx}[0]{\mathbf{x}}
\newcommand{\By}[0]{\mathbf{y}}
\newcommand{\Bz}[0]{\mathbf{z}}
\newcommand{\BA}[0]{\mathbf{A}}
\newcommand{\BB}[0]{\mathbf{B}}
\newcommand{\BC}[0]{\mathbf{C}}
\newcommand{\BD}[0]{\mathbf{D}}
\newcommand{\BE}[0]{\mathbf{E}}
\newcommand{\BF}[0]{\mathbf{F}}
\newcommand{\BG}[0]{\mathbf{G}}
\newcommand{\BH}[0]{\mathbf{H}}
\newcommand{\BI}[0]{\mathbf{I}}
\newcommand{\BJ}[0]{\mathbf{J}}
\newcommand{\BK}[0]{\mathbf{K}}
\newcommand{\BL}[0]{\mathbf{L}}
\newcommand{\BM}[0]{\mathbf{M}}
\newcommand{\BN}[0]{\mathbf{N}}
\newcommand{\BO}[0]{\mathbf{O}}
\newcommand{\BP}[0]{\mathbf{P}}
\newcommand{\BQ}[0]{\mathbf{Q}}
\newcommand{\BR}[0]{\mathbf{R}}
\newcommand{\BS}[0]{\mathbf{S}}
\newcommand{\BT}[0]{\mathbf{T}}
\newcommand{\BU}[0]{\mathbf{U}}
\newcommand{\BV}[0]{\mathbf{V}}
\newcommand{\BW}[0]{\mathbf{W}}
\newcommand{\BX}[0]{\mathbf{X}}
\newcommand{\BY}[0]{\mathbf{Y}}
\newcommand{\BZ}[0]{\mathbf{Z}}

\newcommand{\Bzero}[0]{\mathbf{0}}
\newcommand{\Btheta}[0]{\boldsymbol{\theta}}
\newcommand{\Btau}[0]{\boldsymbol{\tau}}
\newcommand{\Bomega}[0]{\boldsymbol{\omega}}

%
% shorthand for unit vectors
%
\newcommand{\acap}[0]{\hat{\Ba}}
\newcommand{\bcap}[0]{\hat{\Bb}}
\newcommand{\ccap}[0]{\hat{\Bc}}
\newcommand{\dcap}[0]{\hat{\Bd}}
\newcommand{\ecap}[0]{\hat{\Be}}
\newcommand{\fcap}[0]{\hat{\Bf}}
\newcommand{\gcap}[0]{\hat{\Bg}}
\newcommand{\hcap}[0]{\hat{\Bh}}
\newcommand{\icap}[0]{\hat{\Bi}}
\newcommand{\jcap}[0]{\hat{\Bj}}
\newcommand{\kcap}[0]{\hat{\Bk}}
\newcommand{\lcap}[0]{\hat{\Bl}}
\newcommand{\mcap}[0]{\hat{\Bm}}
\newcommand{\ncap}[0]{\hat{\Bn}}
\newcommand{\ocap}[0]{\hat{\Bo}}
\newcommand{\pcap}[0]{\hat{\Bp}}
\newcommand{\qcap}[0]{\hat{\Bq}}
\newcommand{\rcap}[0]{\hat{\Br}}
\newcommand{\scap}[0]{\hat{\Bs}}
\newcommand{\tcap}[0]{\hat{\Bt}}
\newcommand{\ucap}[0]{\hat{\Bu}}
\newcommand{\vcap}[0]{\hat{\Bv}}
\newcommand{\wcap}[0]{\hat{\Bw}}
\newcommand{\xcap}[0]{\hat{\Bx}}
\newcommand{\ycap}[0]{\hat{\By}}
\newcommand{\zcap}[0]{\hat{\Bz}}
\newcommand{\thetacap}[0]{\hat{\Btheta}}

%
% to write R^n and C^n in a distinguishable fashion.  Perhaps change this
% to the double lined characters upon figuring out how to do so.
%
\newcommand{\C}[1]{$\mathbb{C}^{#1}$}
\newcommand{\R}[1]{$\mathbb{R}^{#1}$}

%
% various generally useful helpers
%

% derivative of #1 wrt. #2:
\newcommand{\D}[2] {\frac {d#2} {d#1}}

\newcommand{\inv}[1]{\frac{1}{#1}}
\newcommand{\cross}[0]{\times}

\newcommand{\abs}[1]{\lvert{#1}\rvert}
\newcommand{\norm}[1]{\lVert{#1}\rVert}
\newcommand{\innerprod}[2]{\langle{#1}, {#2}\rangle}
\newcommand{\dotprod}[2]{{#1} \cdot {#2}}
\newcommand{\bdotprod}[2]{\left({#1} \cdot {#2}\right)}
\newcommand{\crossprod}[2]{{#1} \cross {#2}}
\newcommand{\tripleprod}[3]{\dotprod{\left(\crossprod{#1}{#2}\right)}{#3}}

\DeclareMathOperator{\Proj}{Proj}
\DeclareMathOperator{\Span}{span}
\DeclareMathOperator{\Sgn}{sgn}
\DeclareMathOperator{\Area}{Area}
\DeclareMathOperator{\Volume}{Volume}

%
% A few miscellaneous things specific to this document
%
\newcommand{\crossop}[1]{\crossprod{#1}{}}

% R2 vector.
\newcommand{\VectorTwo}[2]{
\begin{bmatrix}
 {#1} \\
 {#2}
\end{bmatrix}
}

\newcommand{\VectorN}[1]{
\begin{bmatrix}
{#1}_1 \\
{#1}_2 \\
\vdots \\
{#1}_N \\
\end{bmatrix}
}

\newcommand{\DETuvij}[4]{
\begin{vmatrix}
 {#1}_{#3} & {#1}_{#4} \\
 {#2}_{#3} & {#2}_{#4}
\end{vmatrix}
}

\newcommand{\DETuvwijk}[6]{
\begin{vmatrix}
 {#1}_{#4} & {#1}_{#5} & {#1}_{#6} \\
 {#2}_{#4} & {#2}_{#5} & {#2}_{#6} \\
 {#3}_{#4} & {#3}_{#5} & {#3}_{#6}
\end{vmatrix}
}

\newcommand{\DETuvwxijkl}[8]{
\begin{vmatrix}
 {#1}_{#5} & {#1}_{#6} & {#1}_{#7} & {#1}_{#8} \\
 {#2}_{#5} & {#2}_{#6} & {#2}_{#7} & {#2}_{#8} \\
 {#3}_{#5} & {#3}_{#6} & {#3}_{#7} & {#3}_{#8} \\
 {#4}_{#5} & {#4}_{#6} & {#4}_{#7} & {#4}_{#8} \\
\end{vmatrix}
}

%\newcommand{\DETuvwxyijklm}[10]{
%\begin{vmatrix}
% {#1}_{#6} & {#1}_{#7} & {#1}_{#8} & {#1}_{#9} & {#1}_{#10} \\
% {#2}_{#6} & {#2}_{#7} & {#2}_{#8} & {#2}_{#9} & {#2}_{#10} \\
% {#3}_{#6} & {#3}_{#7} & {#3}_{#8} & {#3}_{#9} & {#3}_{#10} \\
% {#4}_{#6} & {#4}_{#7} & {#4}_{#8} & {#4}_{#9} & {#4}_{#10} \\
% {#5}_{#6} & {#5}_{#7} & {#5}_{#8} & {#5}_{#9} & {#5}_{#10}
%\end{vmatrix}
%}

% R3 vector.
\newcommand{\VectorThree}[3]{
\begin{bmatrix}
 {#1} \\
 {#2} \\
 {#3}
\end{bmatrix}
}



\author{Peeter Joot}
\email{peeter.joot@gmail.com}


\chapter{PHY356 Problem Set 5.}
\label{chap:qmIproblemSet5}
%\useCCL
%\blogpage{http://sites.google.com/site/peeterjoot/math2010/qmIproblemSet5.pdf}
\date{Nov 25, 2010}
\revisionInfo{qmIproblemSet5.tex}

\beginArtNoToc
\section{Problem.}
\subsection{Statement}

A particle of mass m moves along the x-direction such that $V(X)=\inv{2}KX^2$. Is the state $u(\xi) = B \xi e^{+\xi^2/2}$, where $\xi$ is given by Eq. (9.60), $B$ is a constant, and time $t=0$, an energy eigenstate of the system?  What is probability per unit length for measuring the particle at position $x=0$ at $t=t_0>0$?  Explain the physical meaning of the above results.

\subsection{Solution}

Recall that $\xi = \alpha x$, $\alpha = \sqrt{m\omega/\hbar}$, and $K = m \omega^2$.  With this variable substitution Schr\"{o}dinger's equation for this harmonic oscillator potential takes the form

\begin{equation}\label{eqn:qmIproblemSet5:10}
\frac{d^2 u}{d\xi^2} - \xi^2 u = \frac{2 E }{\hbar\omega} u
\end{equation}

While we can blindly substitute a function of the form $\xi e^{\xi^2/2}$ into this to get

\begin{align*}
\frac{d^2 u}{d\xi^2} - \xi^2 u 
&= 
\frac{d u}{d\xi} \left( 1 + \xi^2 \right) e^{\xi^2/2} - \xi^3 e^{\xi^2/2} \\
&= 
\left( 2 \xi + \xi + \xi^3 \right) e^{\xi^2/2} - \xi^3 e^{\xi^2/2} \\
&= 
3 \xi e^{\xi^2/2} 
\end{align*}

and formally make the identification $E = 3 \omega \hbar/2 = (1 + 1/2) \omega \hbar$, this isn't a normalizable wavefunction, and has no physical relavence.

%\EndArticle
\EndNoBibArticle

%%
% Copyright � 2015 Peeter Joot.  All Rights Reserved.
% Licenced as described in the file LICENSE under the root directory of this GIT repository.
%
\documentclass[]{eliblog}

\usepackage{amsmath}
\usepackage{mathpazo}

%
% shorthand for bold symbols, convenient for vectors and matrices
%
\newcommand{\Ba}[0]{\mathbf{a}}
\newcommand{\Bb}[0]{\mathbf{b}}
\newcommand{\Bc}[0]{\mathbf{c}}
\newcommand{\Bd}[0]{\mathbf{d}}
\newcommand{\Be}[0]{\mathbf{e}}
\newcommand{\Bf}[0]{\mathbf{f}}
\newcommand{\Bg}[0]{\mathbf{g}}
\newcommand{\Bh}[0]{\mathbf{h}}
\newcommand{\Bi}[0]{\mathbf{i}}
\newcommand{\Bj}[0]{\mathbf{j}}
\newcommand{\Bk}[0]{\mathbf{k}}
\newcommand{\Bl}[0]{\mathbf{l}}
\newcommand{\Bm}[0]{\mathbf{m}}
\newcommand{\Bn}[0]{\mathbf{n}}
\newcommand{\Bo}[0]{\mathbf{o}}
\newcommand{\Bp}[0]{\mathbf{p}}
\newcommand{\Bq}[0]{\mathbf{q}}
\newcommand{\Br}[0]{\mathbf{r}}
\newcommand{\Bs}[0]{\mathbf{s}}
\newcommand{\Bt}[0]{\mathbf{t}}
\newcommand{\Bu}[0]{\mathbf{u}}
\newcommand{\Bv}[0]{\mathbf{v}}
\newcommand{\Bw}[0]{\mathbf{w}}
\newcommand{\Bx}[0]{\mathbf{x}}
\newcommand{\By}[0]{\mathbf{y}}
\newcommand{\Bz}[0]{\mathbf{z}}
\newcommand{\BA}[0]{\mathbf{A}}
\newcommand{\BB}[0]{\mathbf{B}}
\newcommand{\BC}[0]{\mathbf{C}}
\newcommand{\BD}[0]{\mathbf{D}}
\newcommand{\BE}[0]{\mathbf{E}}
\newcommand{\BF}[0]{\mathbf{F}}
\newcommand{\BG}[0]{\mathbf{G}}
\newcommand{\BH}[0]{\mathbf{H}}
\newcommand{\BI}[0]{\mathbf{I}}
\newcommand{\BJ}[0]{\mathbf{J}}
\newcommand{\BK}[0]{\mathbf{K}}
\newcommand{\BL}[0]{\mathbf{L}}
\newcommand{\BM}[0]{\mathbf{M}}
\newcommand{\BN}[0]{\mathbf{N}}
\newcommand{\BO}[0]{\mathbf{O}}
\newcommand{\BP}[0]{\mathbf{P}}
\newcommand{\BQ}[0]{\mathbf{Q}}
\newcommand{\BR}[0]{\mathbf{R}}
\newcommand{\BS}[0]{\mathbf{S}}
\newcommand{\BT}[0]{\mathbf{T}}
\newcommand{\BU}[0]{\mathbf{U}}
\newcommand{\BV}[0]{\mathbf{V}}
\newcommand{\BW}[0]{\mathbf{W}}
\newcommand{\BX}[0]{\mathbf{X}}
\newcommand{\BY}[0]{\mathbf{Y}}
\newcommand{\BZ}[0]{\mathbf{Z}}

\newcommand{\Bzero}[0]{\mathbf{0}}
\newcommand{\Btheta}[0]{\boldsymbol{\theta}}
\newcommand{\Btau}[0]{\boldsymbol{\tau}}
\newcommand{\Bomega}[0]{\boldsymbol{\omega}}

%
% shorthand for unit vectors
%
\newcommand{\acap}[0]{\hat{\Ba}}
\newcommand{\bcap}[0]{\hat{\Bb}}
\newcommand{\ccap}[0]{\hat{\Bc}}
\newcommand{\dcap}[0]{\hat{\Bd}}
\newcommand{\ecap}[0]{\hat{\Be}}
\newcommand{\fcap}[0]{\hat{\Bf}}
\newcommand{\gcap}[0]{\hat{\Bg}}
\newcommand{\hcap}[0]{\hat{\Bh}}
\newcommand{\icap}[0]{\hat{\Bi}}
\newcommand{\jcap}[0]{\hat{\Bj}}
\newcommand{\kcap}[0]{\hat{\Bk}}
\newcommand{\lcap}[0]{\hat{\Bl}}
\newcommand{\mcap}[0]{\hat{\Bm}}
\newcommand{\ncap}[0]{\hat{\Bn}}
\newcommand{\ocap}[0]{\hat{\Bo}}
\newcommand{\pcap}[0]{\hat{\Bp}}
\newcommand{\qcap}[0]{\hat{\Bq}}
\newcommand{\rcap}[0]{\hat{\Br}}
\newcommand{\scap}[0]{\hat{\Bs}}
\newcommand{\tcap}[0]{\hat{\Bt}}
\newcommand{\ucap}[0]{\hat{\Bu}}
\newcommand{\vcap}[0]{\hat{\Bv}}
\newcommand{\wcap}[0]{\hat{\Bw}}
\newcommand{\xcap}[0]{\hat{\Bx}}
\newcommand{\ycap}[0]{\hat{\By}}
\newcommand{\zcap}[0]{\hat{\Bz}}
\newcommand{\thetacap}[0]{\hat{\Btheta}}

%
% to write R^n and C^n in a distinguishable fashion.  Perhaps change this
% to the double lined characters upon figuring out how to do so.
%
\newcommand{\C}[1]{$\mathbb{C}^{#1}$}
\newcommand{\R}[1]{$\mathbb{R}^{#1}$}

%
% various generally useful helpers
%

% derivative of #1 wrt. #2:
\newcommand{\D}[2] {\frac {d#2} {d#1}}

\newcommand{\inv}[1]{\frac{1}{#1}}
\newcommand{\cross}[0]{\times}

\newcommand{\abs}[1]{\lvert{#1}\rvert}
\newcommand{\norm}[1]{\lVert{#1}\rVert}
\newcommand{\innerprod}[2]{\langle{#1}, {#2}\rangle}
\newcommand{\dotprod}[2]{{#1} \cdot {#2}}
\newcommand{\bdotprod}[2]{\left({#1} \cdot {#2}\right)}
\newcommand{\crossprod}[2]{{#1} \cross {#2}}
\newcommand{\tripleprod}[3]{\dotprod{\left(\crossprod{#1}{#2}\right)}{#3}}

\DeclareMathOperator{\Proj}{Proj}
\DeclareMathOperator{\Span}{span}
\DeclareMathOperator{\Sgn}{sgn}
\DeclareMathOperator{\Area}{Area}
\DeclareMathOperator{\Volume}{Volume}

%
% A few miscellaneous things specific to this document
%
\newcommand{\crossop}[1]{\crossprod{#1}{}}

% R2 vector.
\newcommand{\VectorTwo}[2]{
\begin{bmatrix}
 {#1} \\
 {#2}
\end{bmatrix}
}

\newcommand{\VectorN}[1]{
\begin{bmatrix}
{#1}_1 \\
{#1}_2 \\
\vdots \\
{#1}_N \\
\end{bmatrix}
}

\newcommand{\DETuvij}[4]{
\begin{vmatrix}
 {#1}_{#3} & {#1}_{#4} \\
 {#2}_{#3} & {#2}_{#4}
\end{vmatrix}
}

\newcommand{\DETuvwijk}[6]{
\begin{vmatrix}
 {#1}_{#4} & {#1}_{#5} & {#1}_{#6} \\
 {#2}_{#4} & {#2}_{#5} & {#2}_{#6} \\
 {#3}_{#4} & {#3}_{#5} & {#3}_{#6}
\end{vmatrix}
}

\newcommand{\DETuvwxijkl}[8]{
\begin{vmatrix}
 {#1}_{#5} & {#1}_{#6} & {#1}_{#7} & {#1}_{#8} \\
 {#2}_{#5} & {#2}_{#6} & {#2}_{#7} & {#2}_{#8} \\
 {#3}_{#5} & {#3}_{#6} & {#3}_{#7} & {#3}_{#8} \\
 {#4}_{#5} & {#4}_{#6} & {#4}_{#7} & {#4}_{#8} \\
\end{vmatrix}
}

%\newcommand{\DETuvwxyijklm}[10]{
%\begin{vmatrix}
% {#1}_{#6} & {#1}_{#7} & {#1}_{#8} & {#1}_{#9} & {#1}_{#10} \\
% {#2}_{#6} & {#2}_{#7} & {#2}_{#8} & {#2}_{#9} & {#2}_{#10} \\
% {#3}_{#6} & {#3}_{#7} & {#3}_{#8} & {#3}_{#9} & {#3}_{#10} \\
% {#4}_{#6} & {#4}_{#7} & {#4}_{#8} & {#4}_{#9} & {#4}_{#10} \\
% {#5}_{#6} & {#5}_{#7} & {#5}_{#8} & {#5}_{#9} & {#5}_{#10}
%\end{vmatrix}
%}

% R3 vector.
\newcommand{\VectorThree}[3]{
\begin{bmatrix}
 {#1} \\
 {#2} \\
 {#3}
\end{bmatrix}
}



\author{Peeter Joot}
\email{peeter.joot@gmail.com}

%\documentclass[]{eliblogwidescreen}

\usepackage{amsmath}
\usepackage{mathpazo}

%
% shorthand for bold symbols, convenient for vectors and matrices
%
\newcommand{\Ba}[0]{\mathbf{a}}
\newcommand{\Bb}[0]{\mathbf{b}}
\newcommand{\Bc}[0]{\mathbf{c}}
\newcommand{\Bd}[0]{\mathbf{d}}
\newcommand{\Be}[0]{\mathbf{e}}
\newcommand{\Bf}[0]{\mathbf{f}}
\newcommand{\Bg}[0]{\mathbf{g}}
\newcommand{\Bh}[0]{\mathbf{h}}
\newcommand{\Bi}[0]{\mathbf{i}}
\newcommand{\Bj}[0]{\mathbf{j}}
\newcommand{\Bk}[0]{\mathbf{k}}
\newcommand{\Bl}[0]{\mathbf{l}}
\newcommand{\Bm}[0]{\mathbf{m}}
\newcommand{\Bn}[0]{\mathbf{n}}
\newcommand{\Bo}[0]{\mathbf{o}}
\newcommand{\Bp}[0]{\mathbf{p}}
\newcommand{\Bq}[0]{\mathbf{q}}
\newcommand{\Br}[0]{\mathbf{r}}
\newcommand{\Bs}[0]{\mathbf{s}}
\newcommand{\Bt}[0]{\mathbf{t}}
\newcommand{\Bu}[0]{\mathbf{u}}
\newcommand{\Bv}[0]{\mathbf{v}}
\newcommand{\Bw}[0]{\mathbf{w}}
\newcommand{\Bx}[0]{\mathbf{x}}
\newcommand{\By}[0]{\mathbf{y}}
\newcommand{\Bz}[0]{\mathbf{z}}
\newcommand{\BA}[0]{\mathbf{A}}
\newcommand{\BB}[0]{\mathbf{B}}
\newcommand{\BC}[0]{\mathbf{C}}
\newcommand{\BD}[0]{\mathbf{D}}
\newcommand{\BE}[0]{\mathbf{E}}
\newcommand{\BF}[0]{\mathbf{F}}
\newcommand{\BG}[0]{\mathbf{G}}
\newcommand{\BH}[0]{\mathbf{H}}
\newcommand{\BI}[0]{\mathbf{I}}
\newcommand{\BJ}[0]{\mathbf{J}}
\newcommand{\BK}[0]{\mathbf{K}}
\newcommand{\BL}[0]{\mathbf{L}}
\newcommand{\BM}[0]{\mathbf{M}}
\newcommand{\BN}[0]{\mathbf{N}}
\newcommand{\BO}[0]{\mathbf{O}}
\newcommand{\BP}[0]{\mathbf{P}}
\newcommand{\BQ}[0]{\mathbf{Q}}
\newcommand{\BR}[0]{\mathbf{R}}
\newcommand{\BS}[0]{\mathbf{S}}
\newcommand{\BT}[0]{\mathbf{T}}
\newcommand{\BU}[0]{\mathbf{U}}
\newcommand{\BV}[0]{\mathbf{V}}
\newcommand{\BW}[0]{\mathbf{W}}
\newcommand{\BX}[0]{\mathbf{X}}
\newcommand{\BY}[0]{\mathbf{Y}}
\newcommand{\BZ}[0]{\mathbf{Z}}

\newcommand{\Bzero}[0]{\mathbf{0}}
\newcommand{\Btheta}[0]{\boldsymbol{\theta}}
\newcommand{\Btau}[0]{\boldsymbol{\tau}}
\newcommand{\Bomega}[0]{\boldsymbol{\omega}}

%
% shorthand for unit vectors
%
\newcommand{\acap}[0]{\hat{\Ba}}
\newcommand{\bcap}[0]{\hat{\Bb}}
\newcommand{\ccap}[0]{\hat{\Bc}}
\newcommand{\dcap}[0]{\hat{\Bd}}
\newcommand{\ecap}[0]{\hat{\Be}}
\newcommand{\fcap}[0]{\hat{\Bf}}
\newcommand{\gcap}[0]{\hat{\Bg}}
\newcommand{\hcap}[0]{\hat{\Bh}}
\newcommand{\icap}[0]{\hat{\Bi}}
\newcommand{\jcap}[0]{\hat{\Bj}}
\newcommand{\kcap}[0]{\hat{\Bk}}
\newcommand{\lcap}[0]{\hat{\Bl}}
\newcommand{\mcap}[0]{\hat{\Bm}}
\newcommand{\ncap}[0]{\hat{\Bn}}
\newcommand{\ocap}[0]{\hat{\Bo}}
\newcommand{\pcap}[0]{\hat{\Bp}}
\newcommand{\qcap}[0]{\hat{\Bq}}
\newcommand{\rcap}[0]{\hat{\Br}}
\newcommand{\scap}[0]{\hat{\Bs}}
\newcommand{\tcap}[0]{\hat{\Bt}}
\newcommand{\ucap}[0]{\hat{\Bu}}
\newcommand{\vcap}[0]{\hat{\Bv}}
\newcommand{\wcap}[0]{\hat{\Bw}}
\newcommand{\xcap}[0]{\hat{\Bx}}
\newcommand{\ycap}[0]{\hat{\By}}
\newcommand{\zcap}[0]{\hat{\Bz}}
\newcommand{\thetacap}[0]{\hat{\Btheta}}

%
% to write R^n and C^n in a distinguishable fashion.  Perhaps change this
% to the double lined characters upon figuring out how to do so.
%
\newcommand{\C}[1]{$\mathbb{C}^{#1}$}
\newcommand{\R}[1]{$\mathbb{R}^{#1}$}

%
% various generally useful helpers
%

% derivative of #1 wrt. #2:
\newcommand{\D}[2] {\frac {d#2} {d#1}}

\newcommand{\inv}[1]{\frac{1}{#1}}
\newcommand{\cross}[0]{\times}

\newcommand{\abs}[1]{\lvert{#1}\rvert}
\newcommand{\norm}[1]{\lVert{#1}\rVert}
\newcommand{\innerprod}[2]{\langle{#1}, {#2}\rangle}
\newcommand{\dotprod}[2]{{#1} \cdot {#2}}
\newcommand{\bdotprod}[2]{\left({#1} \cdot {#2}\right)}
\newcommand{\crossprod}[2]{{#1} \cross {#2}}
\newcommand{\tripleprod}[3]{\dotprod{\left(\crossprod{#1}{#2}\right)}{#3}}

\DeclareMathOperator{\Proj}{Proj}
\DeclareMathOperator{\Span}{span}
\DeclareMathOperator{\Sgn}{sgn}
\DeclareMathOperator{\Area}{Area}
\DeclareMathOperator{\Volume}{Volume}

%
% A few miscellaneous things specific to this document
%
\newcommand{\crossop}[1]{\crossprod{#1}{}}

% R2 vector.
\newcommand{\VectorTwo}[2]{
\begin{bmatrix}
 {#1} \\
 {#2}
\end{bmatrix}
}

\newcommand{\VectorN}[1]{
\begin{bmatrix}
{#1}_1 \\
{#1}_2 \\
\vdots \\
{#1}_N \\
\end{bmatrix}
}

\newcommand{\DETuvij}[4]{
\begin{vmatrix}
 {#1}_{#3} & {#1}_{#4} \\
 {#2}_{#3} & {#2}_{#4}
\end{vmatrix}
}

\newcommand{\DETuvwijk}[6]{
\begin{vmatrix}
 {#1}_{#4} & {#1}_{#5} & {#1}_{#6} \\
 {#2}_{#4} & {#2}_{#5} & {#2}_{#6} \\
 {#3}_{#4} & {#3}_{#5} & {#3}_{#6}
\end{vmatrix}
}

\newcommand{\DETuvwxijkl}[8]{
\begin{vmatrix}
 {#1}_{#5} & {#1}_{#6} & {#1}_{#7} & {#1}_{#8} \\
 {#2}_{#5} & {#2}_{#6} & {#2}_{#7} & {#2}_{#8} \\
 {#3}_{#5} & {#3}_{#6} & {#3}_{#7} & {#3}_{#8} \\
 {#4}_{#5} & {#4}_{#6} & {#4}_{#7} & {#4}_{#8} \\
\end{vmatrix}
}

%\newcommand{\DETuvwxyijklm}[10]{
%\begin{vmatrix}
% {#1}_{#6} & {#1}_{#7} & {#1}_{#8} & {#1}_{#9} & {#1}_{#10} \\
% {#2}_{#6} & {#2}_{#7} & {#2}_{#8} & {#2}_{#9} & {#2}_{#10} \\
% {#3}_{#6} & {#3}_{#7} & {#3}_{#8} & {#3}_{#9} & {#3}_{#10} \\
% {#4}_{#6} & {#4}_{#7} & {#4}_{#8} & {#4}_{#9} & {#4}_{#10} \\
% {#5}_{#6} & {#5}_{#7} & {#5}_{#8} & {#5}_{#9} & {#5}_{#10}
%\end{vmatrix}
%}

% R3 vector.
\newcommand{\VectorThree}[3]{
\begin{bmatrix}
 {#1} \\
 {#2} \\
 {#3}
\end{bmatrix}
}



\author{Peeter Joot}
\email{peeter.joot@gmail.com}


\chapter{Some worked problems from old PHY356 exams.}
\label{chap:qmIexamPractice}
%\useCCL
\blogpage{http://sites.google.com/site/peeterjoot/math2010/qmIexamPractice.pdf}
\date{Dec X, 2010}
\revisionInfo{qmIexamPractice.tex}

\beginArtWithToc
%\beginArtNoToc

\section{Motivation.}

Some of the old exam questions that I did for preparation for the exam I liked, and thought I'd write up some of them for potential future reference.

\section{Questions from the Dec 2007 PHY355H1F exam.}
\subsection{1b.}

\paragraph{Q:} If $\Pi$ is the parity operator, defined by $\Pi \ket{x} = \ket{-x}$, where $\ket{x}$ is the eigenket of the position operator $X$ with eigenvalue $x$), and $P$ is the momentum operator conjugate to $X$, show (carefully) that $\Pi P \Pi = -P$.

\paragraph{A:}

Consider the matrix element $\bra{-x'} \antisymmetric{\Pi}{P} \ket{x}$.  This is

\begin{align*}
\bra{-x'} \antisymmetric{\Pi}{P} \ket{x}
&=
\bra{-x'} \Pi P - P \Pi \ket{x} \\
&=
\bra{-x'} \Pi P \ket{x} - \bra{-x} P \Pi \ket{x} \\
&=
\bra{x'} P \ket{x} - \bra{-x} P \ket{-x} \\
&=
- i \hbar \left(
\delta(x'-x) \PD{}{x}
-\underbrace{\delta(-x -(-x'))}_{= \delta(x'-x) = \delta(x-x')} \PD{}{-x}
\right) \\
&=
- 2 i \hbar 
\delta(x'-x) \PD{}{x} \\
&=
2 \bra{x'} P \ket{x} \\
&=
2 \bra{-x'} \Pi P \ket{x} \\
\end{align*}

We've taken advantage of the Hermitian property of $P$ and $\Pi$ here, and can rearrange for

\begin{equation}\label{eqn:qmIexamPractice2007Dec:1b:10}
\bra{-x'} \Pi P - P \Pi - 2 \Pi P \ket{x} = 0
\end{equation}

Since this is true for all $\bra{-x}$ and $\ket{x}$ we have

\begin{equation}\label{eqn:qmIexamPractice2007Dec:1b:20}
\Pi P + P \Pi = 0.
\end{equation}

Right multiplication by $\Pi$ and rearranging we have
\begin{equation}\label{eqn:qmIexamPractice2007Dec:1b:30}
\Pi P \Pi = - P \Pi \Pi = - P.
\end{equation}

\subsection{1f.}

\paragraph{Q:} For a free particle moving in one-dimention, the propagator (i.e. the coordinate representation of the evolution opeator),

\begin{equation}\label{eqn:qmIexamPractice2007Dec:1f:10}
G(x,x';t) = \bra{x} U(t) \ket{x'}
\end{equation}

is given by

\begin{equation}\label{eqn:qmIexamPractice2007Dec:1f:20}
G(x,x';t) = \sqrt{\frac{m}{2 \pi i \hbar t}} e^{i m (x-x')^2/ (2 \hbar t)}.
\end{equation}

\paragraph{A:}

This problem is actually fairly straightforward, but it is nice to work it having had a similar problem set question where we were asked about this time evolution operator matrix element (ie: what it's physical meaning is).  Here we have a concrete example of the form of this matrix operator.

Proceeding directly, we have

\begin{align*}
\bra{x} U \ket{x'}
&=
\int \braket{x}{p'} \bra{p'} U \ket{p} \braket{p}{x'} dp dp' \\
&=
\int u_{p'}(x) \bra{p'} e^{-i P^2 t/(2 m \hbar)} \ket{p} u_p^\conj(x') dp dp' \\
&=
\int u_{p'}(x) e^{-i p^2 t/(2 m \hbar)} \delta(p-p') u_p^\conj(x') dp dp' \\
&=
\int u_{p}(x) e^{-i p^2 t/(2 m \hbar)} u_p^\conj(x') dp \\
&=
\inv{(\sqrt{2 \pi \hbar})^2} \int e^{i p (x-x')/\hbar} e^{-i p^2 t/(2 m \hbar)} dp \\
&=
\inv{2 \pi \hbar} \int e^{i p (x-x')/\hbar} e^{-i p^2 t/(2 m \hbar)} dp \\
&=
\inv{2 \pi} 
\int e^{i k (x-x')} e^{-i \hbar k^2 t/(2 m)} dk \\
&=
\inv{2 \pi} 
\int dk e^{- \left(k^2 \frac{ i \hbar t}{2m} - i k (x-x')
\right)
} \\
&=
\inv{2 \pi} 
\int dk e^{- \frac{ i \hbar t}{2m}\left(k - i \frac{2m}{i \hbar t}\frac{(x-x')}{2} \right)^2
- \frac{i^2 2 m (x-x')^2}{4 i \hbar t} 
} \\
&=
\inv{2 \pi}  \sqrt{\pi} \sqrt{\frac{2m}{i \hbar t}}
e^{\frac{ i m (x-x')^2}{2 \hbar t}},
\end{align*}

which is the desired result.  Now, let's look at how this would be used.  We can express our time evolved state using this matrix element by introducing an identity

\begin{align*}
\braket{x}{\psi(t)} 
&=
\bra{x} U \ket{\psi(0)} \\
&=
\int dx' \bra{x} U \ket{x'} \braket{x'}{\psi(0)} \\
&=
\sqrt{\frac{m}{2 \pi i \hbar t}} 
\int dx' 
e^{i m (x-x')^2/ (2 \hbar t)}
\braket{x'}{\psi(0)} \\
\end{align*}

This gives us
\begin{equation}\label{eqn:qmIexamPractice2007Dec:1f:40}
\psi(x, t)
=
\sqrt{\frac{m}{2 \pi i \hbar t}} 
\int dx' 
e^{i m (x-x')^2/ (2 \hbar t)} \psi(x', 0)
\end{equation}

However, note that our free particle wave function at time zero is

\begin{equation}\label{eqn:qmIexamPractice2007Dec:1f:50}
\psi(x, 0) = \frac{e^{i p x/\hbar}}{\sqrt{2 \pi \hbar}}
\end{equation}

So the convolution integral \ref{eqn:qmIexamPractice2007Dec:1f:40} does not exist.  We likely have to require that the solution be not a pure state, but instead a superposition of a set of continuous states (a wave packet in position or momentum space related by Fourier transforms).  That is

\begin{align}\label{eqn:qmIexamPractice2007Dec:1f:60}
\psi(x, 0) &= 
\inv{\sqrt{2 \pi \hbar}} \int \hat{\psi}(p, 0) e^{i p x/\hbar} dp \\
\hat{\psi}(p, 0) &= 
\inv{\sqrt{2 \pi \hbar}} \int \psi(x'', 0) e^{-i p x''/\hbar} dx''
\end{align}

The time evolution of this wave packet is then determined by the propagator, and is

\begin{equation}\label{eqn:qmIexamPractice2007Dec:1f:70}
\psi(x,t) =
\sqrt{\frac{m}{2 \pi i \hbar t}} 
\inv{\sqrt{2 \pi \hbar}} 
\int dx' dp
e^{i m (x-x')^2/ (2 \hbar t)}
\hat{\psi}(p, 0) e^{i p x'/\hbar} ,
\end{equation}

or in terms of the position space wave packet evaluated at time zero

\begin{equation}\label{eqn:qmIexamPractice2007Dec:1f:80}
\psi(x,t) =
\sqrt{\frac{m}{2 \pi i \hbar t}}
\inv{2 \pi}
\int dx' dx'' dk
e^{i m (x-x')^2/ (2 \hbar t)}
e^{i k (x' - x'')} \psi(x'', 0)
\end{equation}

We see that the propagator also ends up with a Fourier transform structure, and we have

\begin{align}\label{eqn:qmIexamPractice2007Dec:1f:90}
\psi(x,t) &= \int dx' U(x, x' ; t) \psi(x', 0) \\
U(x, x' ; t) &=
\sqrt{\frac{m}{2 \pi i \hbar t}}
\inv{2 \pi}
\int du dk
e^{i m (x - x' - u)^2/ (2 \hbar t)}
e^{i k u }
\end{align}

Does that Fourier transform exist?  I'd not be suprised if it ended up with a delta function representation.  I'll hold off attempting to evaluate and reduce it until another day.

\subsection{2.}

%\EndArticle
\EndNoBibArticle

%
% Copyright � 2015 Peeter Joot.  All Rights Reserved.
% Licenced as described in the file LICENSE under the root directory of this GIT repository.
%
\documentclass[]{eliblog}

\usepackage{amsmath}
\usepackage{mathpazo}

%
% shorthand for bold symbols, convenient for vectors and matrices
%
\newcommand{\Ba}[0]{\mathbf{a}}
\newcommand{\Bb}[0]{\mathbf{b}}
\newcommand{\Bc}[0]{\mathbf{c}}
\newcommand{\Bd}[0]{\mathbf{d}}
\newcommand{\Be}[0]{\mathbf{e}}
\newcommand{\Bf}[0]{\mathbf{f}}
\newcommand{\Bg}[0]{\mathbf{g}}
\newcommand{\Bh}[0]{\mathbf{h}}
\newcommand{\Bi}[0]{\mathbf{i}}
\newcommand{\Bj}[0]{\mathbf{j}}
\newcommand{\Bk}[0]{\mathbf{k}}
\newcommand{\Bl}[0]{\mathbf{l}}
\newcommand{\Bm}[0]{\mathbf{m}}
\newcommand{\Bn}[0]{\mathbf{n}}
\newcommand{\Bo}[0]{\mathbf{o}}
\newcommand{\Bp}[0]{\mathbf{p}}
\newcommand{\Bq}[0]{\mathbf{q}}
\newcommand{\Br}[0]{\mathbf{r}}
\newcommand{\Bs}[0]{\mathbf{s}}
\newcommand{\Bt}[0]{\mathbf{t}}
\newcommand{\Bu}[0]{\mathbf{u}}
\newcommand{\Bv}[0]{\mathbf{v}}
\newcommand{\Bw}[0]{\mathbf{w}}
\newcommand{\Bx}[0]{\mathbf{x}}
\newcommand{\By}[0]{\mathbf{y}}
\newcommand{\Bz}[0]{\mathbf{z}}
\newcommand{\BA}[0]{\mathbf{A}}
\newcommand{\BB}[0]{\mathbf{B}}
\newcommand{\BC}[0]{\mathbf{C}}
\newcommand{\BD}[0]{\mathbf{D}}
\newcommand{\BE}[0]{\mathbf{E}}
\newcommand{\BF}[0]{\mathbf{F}}
\newcommand{\BG}[0]{\mathbf{G}}
\newcommand{\BH}[0]{\mathbf{H}}
\newcommand{\BI}[0]{\mathbf{I}}
\newcommand{\BJ}[0]{\mathbf{J}}
\newcommand{\BK}[0]{\mathbf{K}}
\newcommand{\BL}[0]{\mathbf{L}}
\newcommand{\BM}[0]{\mathbf{M}}
\newcommand{\BN}[0]{\mathbf{N}}
\newcommand{\BO}[0]{\mathbf{O}}
\newcommand{\BP}[0]{\mathbf{P}}
\newcommand{\BQ}[0]{\mathbf{Q}}
\newcommand{\BR}[0]{\mathbf{R}}
\newcommand{\BS}[0]{\mathbf{S}}
\newcommand{\BT}[0]{\mathbf{T}}
\newcommand{\BU}[0]{\mathbf{U}}
\newcommand{\BV}[0]{\mathbf{V}}
\newcommand{\BW}[0]{\mathbf{W}}
\newcommand{\BX}[0]{\mathbf{X}}
\newcommand{\BY}[0]{\mathbf{Y}}
\newcommand{\BZ}[0]{\mathbf{Z}}

\newcommand{\Bzero}[0]{\mathbf{0}}
\newcommand{\Btheta}[0]{\boldsymbol{\theta}}
\newcommand{\Btau}[0]{\boldsymbol{\tau}}
\newcommand{\Bomega}[0]{\boldsymbol{\omega}}

%
% shorthand for unit vectors
%
\newcommand{\acap}[0]{\hat{\Ba}}
\newcommand{\bcap}[0]{\hat{\Bb}}
\newcommand{\ccap}[0]{\hat{\Bc}}
\newcommand{\dcap}[0]{\hat{\Bd}}
\newcommand{\ecap}[0]{\hat{\Be}}
\newcommand{\fcap}[0]{\hat{\Bf}}
\newcommand{\gcap}[0]{\hat{\Bg}}
\newcommand{\hcap}[0]{\hat{\Bh}}
\newcommand{\icap}[0]{\hat{\Bi}}
\newcommand{\jcap}[0]{\hat{\Bj}}
\newcommand{\kcap}[0]{\hat{\Bk}}
\newcommand{\lcap}[0]{\hat{\Bl}}
\newcommand{\mcap}[0]{\hat{\Bm}}
\newcommand{\ncap}[0]{\hat{\Bn}}
\newcommand{\ocap}[0]{\hat{\Bo}}
\newcommand{\pcap}[0]{\hat{\Bp}}
\newcommand{\qcap}[0]{\hat{\Bq}}
\newcommand{\rcap}[0]{\hat{\Br}}
\newcommand{\scap}[0]{\hat{\Bs}}
\newcommand{\tcap}[0]{\hat{\Bt}}
\newcommand{\ucap}[0]{\hat{\Bu}}
\newcommand{\vcap}[0]{\hat{\Bv}}
\newcommand{\wcap}[0]{\hat{\Bw}}
\newcommand{\xcap}[0]{\hat{\Bx}}
\newcommand{\ycap}[0]{\hat{\By}}
\newcommand{\zcap}[0]{\hat{\Bz}}
\newcommand{\thetacap}[0]{\hat{\Btheta}}

%
% to write R^n and C^n in a distinguishable fashion.  Perhaps change this
% to the double lined characters upon figuring out how to do so.
%
\newcommand{\C}[1]{$\mathbb{C}^{#1}$}
\newcommand{\R}[1]{$\mathbb{R}^{#1}$}

%
% various generally useful helpers
%

% derivative of #1 wrt. #2:
\newcommand{\D}[2] {\frac {d#2} {d#1}}

\newcommand{\inv}[1]{\frac{1}{#1}}
\newcommand{\cross}[0]{\times}

\newcommand{\abs}[1]{\lvert{#1}\rvert}
\newcommand{\norm}[1]{\lVert{#1}\rVert}
\newcommand{\innerprod}[2]{\langle{#1}, {#2}\rangle}
\newcommand{\dotprod}[2]{{#1} \cdot {#2}}
\newcommand{\bdotprod}[2]{\left({#1} \cdot {#2}\right)}
\newcommand{\crossprod}[2]{{#1} \cross {#2}}
\newcommand{\tripleprod}[3]{\dotprod{\left(\crossprod{#1}{#2}\right)}{#3}}

\DeclareMathOperator{\Proj}{Proj}
\DeclareMathOperator{\Span}{span}
\DeclareMathOperator{\Sgn}{sgn}
\DeclareMathOperator{\Area}{Area}
\DeclareMathOperator{\Volume}{Volume}

%
% A few miscellaneous things specific to this document
%
\newcommand{\crossop}[1]{\crossprod{#1}{}}

% R2 vector.
\newcommand{\VectorTwo}[2]{
\begin{bmatrix}
 {#1} \\
 {#2}
\end{bmatrix}
}

\newcommand{\VectorN}[1]{
\begin{bmatrix}
{#1}_1 \\
{#1}_2 \\
\vdots \\
{#1}_N \\
\end{bmatrix}
}

\newcommand{\DETuvij}[4]{
\begin{vmatrix}
 {#1}_{#3} & {#1}_{#4} \\
 {#2}_{#3} & {#2}_{#4}
\end{vmatrix}
}

\newcommand{\DETuvwijk}[6]{
\begin{vmatrix}
 {#1}_{#4} & {#1}_{#5} & {#1}_{#6} \\
 {#2}_{#4} & {#2}_{#5} & {#2}_{#6} \\
 {#3}_{#4} & {#3}_{#5} & {#3}_{#6}
\end{vmatrix}
}

\newcommand{\DETuvwxijkl}[8]{
\begin{vmatrix}
 {#1}_{#5} & {#1}_{#6} & {#1}_{#7} & {#1}_{#8} \\
 {#2}_{#5} & {#2}_{#6} & {#2}_{#7} & {#2}_{#8} \\
 {#3}_{#5} & {#3}_{#6} & {#3}_{#7} & {#3}_{#8} \\
 {#4}_{#5} & {#4}_{#6} & {#4}_{#7} & {#4}_{#8} \\
\end{vmatrix}
}

%\newcommand{\DETuvwxyijklm}[10]{
%\begin{vmatrix}
% {#1}_{#6} & {#1}_{#7} & {#1}_{#8} & {#1}_{#9} & {#1}_{#10} \\
% {#2}_{#6} & {#2}_{#7} & {#2}_{#8} & {#2}_{#9} & {#2}_{#10} \\
% {#3}_{#6} & {#3}_{#7} & {#3}_{#8} & {#3}_{#9} & {#3}_{#10} \\
% {#4}_{#6} & {#4}_{#7} & {#4}_{#8} & {#4}_{#9} & {#4}_{#10} \\
% {#5}_{#6} & {#5}_{#7} & {#5}_{#8} & {#5}_{#9} & {#5}_{#10}
%\end{vmatrix}
%}

% R3 vector.
\newcommand{\VectorThree}[3]{
\begin{bmatrix}
 {#1} \\
 {#2} \\
 {#3}
\end{bmatrix}
}



\author{Peeter Joot}
\email{peeter.joot@gmail.com}

%\documentclass[]{eliblogwidescreen}

\usepackage{amsmath}
\usepackage{mathpazo}

%
% shorthand for bold symbols, convenient for vectors and matrices
%
\newcommand{\Ba}[0]{\mathbf{a}}
\newcommand{\Bb}[0]{\mathbf{b}}
\newcommand{\Bc}[0]{\mathbf{c}}
\newcommand{\Bd}[0]{\mathbf{d}}
\newcommand{\Be}[0]{\mathbf{e}}
\newcommand{\Bf}[0]{\mathbf{f}}
\newcommand{\Bg}[0]{\mathbf{g}}
\newcommand{\Bh}[0]{\mathbf{h}}
\newcommand{\Bi}[0]{\mathbf{i}}
\newcommand{\Bj}[0]{\mathbf{j}}
\newcommand{\Bk}[0]{\mathbf{k}}
\newcommand{\Bl}[0]{\mathbf{l}}
\newcommand{\Bm}[0]{\mathbf{m}}
\newcommand{\Bn}[0]{\mathbf{n}}
\newcommand{\Bo}[0]{\mathbf{o}}
\newcommand{\Bp}[0]{\mathbf{p}}
\newcommand{\Bq}[0]{\mathbf{q}}
\newcommand{\Br}[0]{\mathbf{r}}
\newcommand{\Bs}[0]{\mathbf{s}}
\newcommand{\Bt}[0]{\mathbf{t}}
\newcommand{\Bu}[0]{\mathbf{u}}
\newcommand{\Bv}[0]{\mathbf{v}}
\newcommand{\Bw}[0]{\mathbf{w}}
\newcommand{\Bx}[0]{\mathbf{x}}
\newcommand{\By}[0]{\mathbf{y}}
\newcommand{\Bz}[0]{\mathbf{z}}
\newcommand{\BA}[0]{\mathbf{A}}
\newcommand{\BB}[0]{\mathbf{B}}
\newcommand{\BC}[0]{\mathbf{C}}
\newcommand{\BD}[0]{\mathbf{D}}
\newcommand{\BE}[0]{\mathbf{E}}
\newcommand{\BF}[0]{\mathbf{F}}
\newcommand{\BG}[0]{\mathbf{G}}
\newcommand{\BH}[0]{\mathbf{H}}
\newcommand{\BI}[0]{\mathbf{I}}
\newcommand{\BJ}[0]{\mathbf{J}}
\newcommand{\BK}[0]{\mathbf{K}}
\newcommand{\BL}[0]{\mathbf{L}}
\newcommand{\BM}[0]{\mathbf{M}}
\newcommand{\BN}[0]{\mathbf{N}}
\newcommand{\BO}[0]{\mathbf{O}}
\newcommand{\BP}[0]{\mathbf{P}}
\newcommand{\BQ}[0]{\mathbf{Q}}
\newcommand{\BR}[0]{\mathbf{R}}
\newcommand{\BS}[0]{\mathbf{S}}
\newcommand{\BT}[0]{\mathbf{T}}
\newcommand{\BU}[0]{\mathbf{U}}
\newcommand{\BV}[0]{\mathbf{V}}
\newcommand{\BW}[0]{\mathbf{W}}
\newcommand{\BX}[0]{\mathbf{X}}
\newcommand{\BY}[0]{\mathbf{Y}}
\newcommand{\BZ}[0]{\mathbf{Z}}

\newcommand{\Bzero}[0]{\mathbf{0}}
\newcommand{\Btheta}[0]{\boldsymbol{\theta}}
\newcommand{\Btau}[0]{\boldsymbol{\tau}}
\newcommand{\Bomega}[0]{\boldsymbol{\omega}}

%
% shorthand for unit vectors
%
\newcommand{\acap}[0]{\hat{\Ba}}
\newcommand{\bcap}[0]{\hat{\Bb}}
\newcommand{\ccap}[0]{\hat{\Bc}}
\newcommand{\dcap}[0]{\hat{\Bd}}
\newcommand{\ecap}[0]{\hat{\Be}}
\newcommand{\fcap}[0]{\hat{\Bf}}
\newcommand{\gcap}[0]{\hat{\Bg}}
\newcommand{\hcap}[0]{\hat{\Bh}}
\newcommand{\icap}[0]{\hat{\Bi}}
\newcommand{\jcap}[0]{\hat{\Bj}}
\newcommand{\kcap}[0]{\hat{\Bk}}
\newcommand{\lcap}[0]{\hat{\Bl}}
\newcommand{\mcap}[0]{\hat{\Bm}}
\newcommand{\ncap}[0]{\hat{\Bn}}
\newcommand{\ocap}[0]{\hat{\Bo}}
\newcommand{\pcap}[0]{\hat{\Bp}}
\newcommand{\qcap}[0]{\hat{\Bq}}
\newcommand{\rcap}[0]{\hat{\Br}}
\newcommand{\scap}[0]{\hat{\Bs}}
\newcommand{\tcap}[0]{\hat{\Bt}}
\newcommand{\ucap}[0]{\hat{\Bu}}
\newcommand{\vcap}[0]{\hat{\Bv}}
\newcommand{\wcap}[0]{\hat{\Bw}}
\newcommand{\xcap}[0]{\hat{\Bx}}
\newcommand{\ycap}[0]{\hat{\By}}
\newcommand{\zcap}[0]{\hat{\Bz}}
\newcommand{\thetacap}[0]{\hat{\Btheta}}

%
% to write R^n and C^n in a distinguishable fashion.  Perhaps change this
% to the double lined characters upon figuring out how to do so.
%
\newcommand{\C}[1]{$\mathbb{C}^{#1}$}
\newcommand{\R}[1]{$\mathbb{R}^{#1}$}

%
% various generally useful helpers
%

% derivative of #1 wrt. #2:
\newcommand{\D}[2] {\frac {d#2} {d#1}}

\newcommand{\inv}[1]{\frac{1}{#1}}
\newcommand{\cross}[0]{\times}

\newcommand{\abs}[1]{\lvert{#1}\rvert}
\newcommand{\norm}[1]{\lVert{#1}\rVert}
\newcommand{\innerprod}[2]{\langle{#1}, {#2}\rangle}
\newcommand{\dotprod}[2]{{#1} \cdot {#2}}
\newcommand{\bdotprod}[2]{\left({#1} \cdot {#2}\right)}
\newcommand{\crossprod}[2]{{#1} \cross {#2}}
\newcommand{\tripleprod}[3]{\dotprod{\left(\crossprod{#1}{#2}\right)}{#3}}

\DeclareMathOperator{\Proj}{Proj}
\DeclareMathOperator{\Span}{span}
\DeclareMathOperator{\Sgn}{sgn}
\DeclareMathOperator{\Area}{Area}
\DeclareMathOperator{\Volume}{Volume}

%
% A few miscellaneous things specific to this document
%
\newcommand{\crossop}[1]{\crossprod{#1}{}}

% R2 vector.
\newcommand{\VectorTwo}[2]{
\begin{bmatrix}
 {#1} \\
 {#2}
\end{bmatrix}
}

\newcommand{\VectorN}[1]{
\begin{bmatrix}
{#1}_1 \\
{#1}_2 \\
\vdots \\
{#1}_N \\
\end{bmatrix}
}

\newcommand{\DETuvij}[4]{
\begin{vmatrix}
 {#1}_{#3} & {#1}_{#4} \\
 {#2}_{#3} & {#2}_{#4}
\end{vmatrix}
}

\newcommand{\DETuvwijk}[6]{
\begin{vmatrix}
 {#1}_{#4} & {#1}_{#5} & {#1}_{#6} \\
 {#2}_{#4} & {#2}_{#5} & {#2}_{#6} \\
 {#3}_{#4} & {#3}_{#5} & {#3}_{#6}
\end{vmatrix}
}

\newcommand{\DETuvwxijkl}[8]{
\begin{vmatrix}
 {#1}_{#5} & {#1}_{#6} & {#1}_{#7} & {#1}_{#8} \\
 {#2}_{#5} & {#2}_{#6} & {#2}_{#7} & {#2}_{#8} \\
 {#3}_{#5} & {#3}_{#6} & {#3}_{#7} & {#3}_{#8} \\
 {#4}_{#5} & {#4}_{#6} & {#4}_{#7} & {#4}_{#8} \\
\end{vmatrix}
}

%\newcommand{\DETuvwxyijklm}[10]{
%\begin{vmatrix}
% {#1}_{#6} & {#1}_{#7} & {#1}_{#8} & {#1}_{#9} & {#1}_{#10} \\
% {#2}_{#6} & {#2}_{#7} & {#2}_{#8} & {#2}_{#9} & {#2}_{#10} \\
% {#3}_{#6} & {#3}_{#7} & {#3}_{#8} & {#3}_{#9} & {#3}_{#10} \\
% {#4}_{#6} & {#4}_{#7} & {#4}_{#8} & {#4}_{#9} & {#4}_{#10} \\
% {#5}_{#6} & {#5}_{#7} & {#5}_{#8} & {#5}_{#9} & {#5}_{#10}
%\end{vmatrix}
%}

% R3 vector.
\newcommand{\VectorThree}[3]{
\begin{bmatrix}
 {#1} \\
 {#2} \\
 {#3}
\end{bmatrix}
}



\author{Peeter Joot}
\email{peeter.joot@gmail.com}


\chapter{PHY356F lecture notes.}
\label{chap:PHY356F}
%\useCCL
\blogpage{http://sites.google.com/site/peeterjoot/math2010/PHY356F.pdf}
%\date{Oct X, 2010}
\revisionInfo{PHY356F.tex}

%\beginArtWithToc
\beginArtNoToc

\section{Oct 12.}

Review.  What have we learned?

\subsection{Chapter 1.}
Information about systems comes from vectors and operators.  Express the vector $\ket0$ describing the system in terms of eigenvectors $\ket{a_n}$.  $n \in 1,2,3,\cdots$.

of some operator $A$.  What are the coefficients $c_n$?  Act on both sides by $\bra{a_m}$ to find

\begin{align*}
\braket{a_m}{\phi} 
&= \sum_n c_n \underbrace{\braket{a_m}{a_n}}_{\text{Kronicker delta}}  \\
&= \sum c_n \delta_{mn} \\
&= c_m
\end{align*}

\begin{align*}
c_m = \braket{a_m}{\phi}
\end{align*}

Analogy

\begin{align*}
\Bv = \sum_i v_i \Be_i 
\end{align*}

\begin{align*}
\Be_1 \cdot \Bv = \sum_i v_i \Be_1 \cdot \Be_i = v_1
\end{align*}

Physical information comes from the probability for obtaining a measurement of the physical entity associated with operator $A$.  The probability of obtaining outcome $a_m$, an eigenvalue of $A$, is $\Abs{c_n}^2$

\subsection{Chapter 2.}

Deal with operators that have continuous eigenvalues and eigenvectors.

We now express 

\begin{align*}
\ket{\phi} = \int dk \underbrace{f(k)}_{\text{coefficients analogous to $c_n$}} \ket{k}
\end{align*}

Now if we project onto $k'$

\begin{align*}
\braket{k'}{\phi} 
&= \int dk f(k) \underbrace{\braket{k'}{k}}_{\text{Dirac delta}} \\
&= \int dk f(k) \delta(k' -k) \\
&= f(k') 
\end{align*}

Unlike the discrete case, this is not a probability.  Probability density for obtaining outcome $k'$ is $\Abs{f(k')}^2$.

Example 2.
\begin{align*}
\ket{\phi} = \int dk f(k) \ket{k}
\end{align*}

Now if we project x onto both sides

\begin{align*}
\braket{x}{\phi} 
&= \int dk f(k) \braket{x}{k} \\
\end{align*}

With $\braket{x}{k} = u_k(x)$

\begin{align*}
\phi(x) 
&\equiv \braket{x}{\phi} \\
&= \int dk f(k) u_k(x)  \\
&= \int dk f(k) \inv{\sqrt{L}} e^{ikx}
\end{align*}

This is with periodic boundary value conditions for the normalization.  The infinite normalization is also possible.

\begin{align*}
\phi(x) 
&= \inv{\sqrt{L}} \int dk f(k) e^{ikx}
\end{align*}

Multiply both sides by $e^{-ik'x}/\sqrt{L}$ and integrate.  This is analogous to multiplying $\ket{\phi} = \int f(k) \ket{k} dk$ by $\bra{k'}$.  We get

\begin{align*}
\int \phi(x) \inv{\sqrt{L}} e^{-ik'x} dx
&= \inv{L} \iint dk f(k) e^{i(k-k')x} dx \\
&= \int dk f(k) \Bigl( \inv{L} \int e^{i(k-k')x} \Bigr) \\
&= \int dk f(k) \delta(k-k') \\
&= f(k')
\end{align*}

\begin{align*}
f(k') &=
\int \phi(x) \inv{\sqrt{L}} e^{-ik'x} dx
\end{align*}

We can talk about the state vector in terms of its position basis $\phi(x)$ or in the momentum space via Fourier transformation.  This is the equivalent thing, but just expressed different.  The question of interpretation in terms of probabilities works out the same.  Either way we look at the probability density.

The quantity

\begin{align*}
\ket{\phi} = \int dk f(k) \ket{k}
\end{align*}

is also called a wave packet state since it involves a superposition of many stats $\ket{k}$.  Example: See Fig 4.1 (Gaussian wave packet, with $\Abs{\phi}^2$ as the height).  This wave packet is a snapshot of the wave function amplitude at one specific time instant.  The evolution of this wave packet is governed by the Hamiltonian, which brings us to chapter 3.

\subsection{Chapter 3.}

For 
\begin{align*}
\ket{\phi} = \int dk f(k) \ket{k}
\end{align*}

How do we find $\ket{\phi(t)}$, the time evolved state?  Here we have the option of choosing which of the pictures (Schr\"{o}dinger, Heisenberg, interaction) we deal with.  Since the Heisenberg picture deals with time evolved operators, and the interaction picture with evolving Hamiltonian's, neither of these is required to answer this question.  Consider the Schr\"{o}dinger picture which gives 

\begin{align*}
\ket{\phi(t)} = \int dk f(k) \ket{k} e^{-i E_k t/\hbar}
\end{align*}

where $E_k$ is the eigenvalue of the Hamiltonian operator $H$.

STRONG SEEMING HINT: If looking for additional problems and homework, consider in detail the time evolution of the Gaussian wave packet state.

\subsection{Chapter 4.}

For three dimensions with $V(x,y,z) = 0$

\begin{align*}
H &= \inv{2m} \Bp^2 \\
\Bp &= \sum_i p_i \Be_i \\
\end{align*}

In the position representation, where

\begin{align*}
p_i &= -i \hbar \frac{d}{dx_i}
\end{align*}

the Sch equation is
\begin{align*}
H u(x,y,z) &= E u(x,y,z) \\
H &= -\frac{\hbar^2}{2m} \spacegrad^2 \\
= -\frac{\hbar^2}{2m} \left( 
\frac{\partial^2}{\partial {x}^2}
+\frac{\partial^2}{\partial {y}^2}
+\frac{\partial^2}{\partial {z}^2}
\right) 
\end{align*}

Separation of variables assumes it is possible to let

\begin{align*}
u(x,y,z) = X(x) Y(y) Z(z)
\end{align*}

(these capital letters are functions, not operators).

\begin{align*}
-\frac{\hbar^2}{2m} \left( 
YZ \frac{\partial^2 X}{\partial {x}^2}
+ XZ \frac{\partial^2 Y}{\partial {y}^2}
+ YZ \frac{\partial^2 Z}{\partial {z}^2}\right)
&= E X Y Z
\end{align*}

Dividing as usual by $XYZ$ we have

\begin{align*}
-\frac{\hbar^2}{2m} \left( 
\inv{X} \frac{\partial^2 X}{\partial {x}^2}
+ \inv{Y} \frac{\partial^2 Y}{\partial {y}^2}
+ \inv{Z} \frac{\partial^2 Z}{\partial {z}^2} \right)
&= E 
\end{align*}

The curious thing is that we have these three derivatives, which is supposed to be related to an Energy, which is independent of any $x,y,z$, so it must be that each of these is separately constant.  We can separate these into three individual equations

\begin{align*}
-\frac{\hbar^2}{2m} \inv{X} \frac{\partial^2 X}{\partial {x}^2} &= E_1 \\
-\frac{\hbar^2}{2m} \inv{Y} \frac{\partial^2 Y}{\partial {x}^2} &= E_2 \\
-\frac{\hbar^2}{2m} \inv{Z} \frac{\partial^2 Z}{\partial {x}^2} &= E_3
\end{align*}

or
\begin{align*}
\frac{\partial^2 X}{\partial {x}^2} &= \left( - \frac{2m E_1}{\hbar^2} \right) X  \\
\frac{\partial^2 Y}{\partial {x}^2} &= \left( - \frac{2m E_2}{\hbar^2} \right) Y  \\
\frac{\partial^2 Z}{\partial {x}^2} &= \left( - \frac{2m E_3}{\hbar^2} \right) Z 
\end{align*}

We have then

\begin{align*}
X(x) = C_1 e^{i k x}
\end{align*}

with
\begin{align*}
E_1 &= \frac{\hbar^2 k_1^2 }{2m} = \frac{p_1^2}{2m} \\
E_2 &= \frac{\hbar^2 k_2^2 }{2m} = \frac{p_2^2}{2m} \\
E_3 &= \frac{\hbar^2 k_3^2 }{2m} = \frac{p_3^2}{2m} 
\end{align*}

We are free to use any sort of normalization procedure we wish (periodic boundary conditions, infinite Dirac, ...)

\subsection{Angular momentum.}

HOMEWORK: go through the steps to understand how to formulate $\spacegrad^2$ in spherical polar coordinates.  This is a lot of work, but is good practice and background for dealing with the Hydrogen atom, something with spherical symmetry that is most naturally analyzed in the spherical polar coordinates.

In spherical coordinates (We won't go through this here, but it is good practice) with

\begin{align*}
x &= r \sin\theta \cos\phi \\
y &= r \sin\theta \sin\phi \\
z &= r \cos\theta
\end{align*}

we have with $u = u(r,\theta, \phi)$

\begin{align*}
-\frac{\hbar^2}{2m} \left( 
\inv{r} \partial_{rr} (r u) +  \inv{r^2 \sin\theta} \partial_\theta (\sin\theta \partial_\theta u) 
+ \inv{r^2 \sin^2\theta} \partial_{\phi\phi} u
 \right)
&= E u
\end{align*}

We see the start of a separation of variables attack with $u = R(r) Y(\theta, \phi)$.  We end up with

\begin{align*}
-\frac{\hbar^2}{2m} &\left( 
\frac{r}{R} (r R')' +  \inv{Y \sin\theta} \partial_\theta (\sin\theta \partial_\theta Y) 
+ \inv{Y \sin^2\theta} \partial_{\phi\phi} Y
 \right) \\
\end{align*}

\begin{align*}
r (r R')' + \left( \frac{2m E}{\hbar^2} r^2 - \lambda \right) R &= 0
\end{align*}
\begin{align*}
\inv{Y \sin\theta} \partial_\theta (\sin\theta \partial_\theta Y) + \inv{Y \sin^2\theta} \partial_{\phi\phi} Y &= -\lambda
\end{align*}

Application of separation of variables again, with $Y = P(\theta) Q(\phi)$ gives us

\begin{align*}
\inv{P \sin\theta} \partial_\theta (\sin\theta \partial_\theta P) + \inv{Q \sin^2\theta} \partial_{\phi\phi} Q &= -\lambda 
\end{align*}

\begin{align*}
\frac{\sin\theta}{P } \partial_\theta (\sin\theta \partial_\theta P) 
+\lambda  \sin^2\theta
+ \inv{Q } \partial_{\phi\phi} Q &= 0
\end{align*}

\begin{align*}
\frac{\sin\theta}{P } \partial_\theta (\sin\theta \partial_\theta P) + \lambda \sin^2\theta - \mu = 0
\inv{Q } \partial_{\phi\phi} Q &= -\mu
\end{align*}

or
\begin{align}\label{eqn:PHY356F:1000}
\frac{1}{P \sin\theta} \partial_\theta (\sin\theta \partial_\theta P) +\lambda -\frac{\mu}{\sin^2\theta} &= 0
\end{align}
\begin{align}\label{eqn:PHY356F:2000}
\partial_{\phi\phi} Q &= -\mu Q
\end{align}

The equation for $P$ can be solved using the Legendre function $P_l^m(\cos\theta)$ where $\lambda = l(l+1)$ and $l$ is an integer

Replacing $\mu$ with $m^2$, where $m$ is an integer

\begin{align*}
\frac{d^2 Q}{d\phi^2} &= -m^2 Q
\end{align*}

Imposing a periodic boundary condition $Q(\phi) = Q(\phi + 2\pi)$, where ($m = 0, \pm 1, \pm 2, \cdots$) we have

\begin{align*}
Q &= \inv{\sqrt{2\pi}} e^{im\phi}
\end{align*}

There is the overall solution $r(r,\theta,\phi) = R(r) Y(\theta, \phi)$ for a free particle.  The functions $Y(\theta, \phi)$ are

\begin{align*}
Y_{lm}(\theta, \phi) 
&= N \left( \inv{\sqrt{2\pi}} e^{im\phi} \right) \underbrace{ P_l^m(\cos\theta) }_{ -l \le m \le l }
\end{align*}

where $N$ is a normalization constant, and $m = 0, \pm 1, \pm 2, \cdots$.  $Y_{lm}$ is an eigenstate of the $\BL^2$ operator and $L_z$ (two for the price of one).  There's no specific reason for the direction $z$, but it is the direction picked out of convention.

Angular momentum is given by 

\begin{align*}
\BL = \Br \cross \Bp
\end{align*}

where 

\begin{align*}
\BR = x \xcap + y\ycap + z\zcap
\end{align*}

and 
\begin{align*}
\Bp = p_x \xcap + p_y\ycap + p_z\zcap
\end{align*}

The important thing to remember is that the aim of following all the math is to show that

\begin{align*}
\BL^2 Y_{lm} = \hbar^2 l (l+1) Y_{lm}
\end{align*}

and simultaneously 

\begin{align*}
\BL_z Y_{lm} = \hbar m Y_{lm}
\end{align*}

Part of the solution involves working with $\antisymmetric{L_z}{L_{+}}$, and $\antisymmetric{L_z}{L_{-}}$, where

\begin{align*}
L_{+} &= L_x + i L_y \\
L_{-} &= L_x - i L_y
\end{align*}

An exercise (not in the book) is to evaluate
\begin{align}\label{eqn:PHY356F:4000}
\antisymmetric{L_z}{L_{+}} 
&= L_z L_x + i L_z L_y - L_x L_z - i L_y L_z 
\end{align}

where
\begin{align}\label{eqn:PHY356F:5000}
\antisymmetric{L_x}{L_y}  &= i \hbar L_z \\
\antisymmetric{L_y}{L_z}  &= i \hbar L_x \\
\antisymmetric{L_z}{L_x}  &= i \hbar L_y
\end{align}

Substitution back in \ref{eqn:PHY356F:4000} we have

\begin{align*}
\antisymmetric{L_z}{L_{+}} 
&=
\antisymmetric{L_z}{L_x} 
+ i \antisymmetric{L_z}{L_y}  \\
&=
i \hbar ( L_y - i L_x ) \\
&=
\hbar ( i L_y +  L_x ) \\
&=
\hbar L_{+}
\end{align*}

%\EndArticle
\EndNoBibArticle


\part{Relativity.}
\documentclass{article}

\usepackage{amsmath}
\usepackage{mathpazo}

%
% shorthand for bold symbols, convenient for vectors and matrices
%
\newcommand{\Ba}[0]{\mathbf{a}}
\newcommand{\Bb}[0]{\mathbf{b}}
\newcommand{\Bc}[0]{\mathbf{c}}
\newcommand{\Bd}[0]{\mathbf{d}}
\newcommand{\Be}[0]{\mathbf{e}}
\newcommand{\Bf}[0]{\mathbf{f}}
\newcommand{\Bg}[0]{\mathbf{g}}
\newcommand{\Bh}[0]{\mathbf{h}}
\newcommand{\Bi}[0]{\mathbf{i}}
\newcommand{\Bj}[0]{\mathbf{j}}
\newcommand{\Bk}[0]{\mathbf{k}}
\newcommand{\Bl}[0]{\mathbf{l}}
\newcommand{\Bm}[0]{\mathbf{m}}
\newcommand{\Bn}[0]{\mathbf{n}}
\newcommand{\Bo}[0]{\mathbf{o}}
\newcommand{\Bp}[0]{\mathbf{p}}
\newcommand{\Bq}[0]{\mathbf{q}}
\newcommand{\Br}[0]{\mathbf{r}}
\newcommand{\Bs}[0]{\mathbf{s}}
\newcommand{\Bt}[0]{\mathbf{t}}
\newcommand{\Bu}[0]{\mathbf{u}}
\newcommand{\Bv}[0]{\mathbf{v}}
\newcommand{\Bw}[0]{\mathbf{w}}
\newcommand{\Bx}[0]{\mathbf{x}}
\newcommand{\By}[0]{\mathbf{y}}
\newcommand{\Bz}[0]{\mathbf{z}}
\newcommand{\BA}[0]{\mathbf{A}}
\newcommand{\BB}[0]{\mathbf{B}}
\newcommand{\BC}[0]{\mathbf{C}}
\newcommand{\BD}[0]{\mathbf{D}}
\newcommand{\BE}[0]{\mathbf{E}}
\newcommand{\BF}[0]{\mathbf{F}}
\newcommand{\BG}[0]{\mathbf{G}}
\newcommand{\BH}[0]{\mathbf{H}}
\newcommand{\BI}[0]{\mathbf{I}}
\newcommand{\BJ}[0]{\mathbf{J}}
\newcommand{\BK}[0]{\mathbf{K}}
\newcommand{\BL}[0]{\mathbf{L}}
\newcommand{\BM}[0]{\mathbf{M}}
\newcommand{\BN}[0]{\mathbf{N}}
\newcommand{\BO}[0]{\mathbf{O}}
\newcommand{\BP}[0]{\mathbf{P}}
\newcommand{\BQ}[0]{\mathbf{Q}}
\newcommand{\BR}[0]{\mathbf{R}}
\newcommand{\BS}[0]{\mathbf{S}}
\newcommand{\BT}[0]{\mathbf{T}}
\newcommand{\BU}[0]{\mathbf{U}}
\newcommand{\BV}[0]{\mathbf{V}}
\newcommand{\BW}[0]{\mathbf{W}}
\newcommand{\BX}[0]{\mathbf{X}}
\newcommand{\BY}[0]{\mathbf{Y}}
\newcommand{\BZ}[0]{\mathbf{Z}}

\newcommand{\Bzero}[0]{\mathbf{0}}
\newcommand{\Btheta}[0]{\boldsymbol{\theta}}
\newcommand{\Btau}[0]{\boldsymbol{\tau}}
\newcommand{\Bomega}[0]{\boldsymbol{\omega}}

%
% shorthand for unit vectors
%
\newcommand{\acap}[0]{\hat{\Ba}}
\newcommand{\bcap}[0]{\hat{\Bb}}
\newcommand{\ccap}[0]{\hat{\Bc}}
\newcommand{\dcap}[0]{\hat{\Bd}}
\newcommand{\ecap}[0]{\hat{\Be}}
\newcommand{\fcap}[0]{\hat{\Bf}}
\newcommand{\gcap}[0]{\hat{\Bg}}
\newcommand{\hcap}[0]{\hat{\Bh}}
\newcommand{\icap}[0]{\hat{\Bi}}
\newcommand{\jcap}[0]{\hat{\Bj}}
\newcommand{\kcap}[0]{\hat{\Bk}}
\newcommand{\lcap}[0]{\hat{\Bl}}
\newcommand{\mcap}[0]{\hat{\Bm}}
\newcommand{\ncap}[0]{\hat{\Bn}}
\newcommand{\ocap}[0]{\hat{\Bo}}
\newcommand{\pcap}[0]{\hat{\Bp}}
\newcommand{\qcap}[0]{\hat{\Bq}}
\newcommand{\rcap}[0]{\hat{\Br}}
\newcommand{\scap}[0]{\hat{\Bs}}
\newcommand{\tcap}[0]{\hat{\Bt}}
\newcommand{\ucap}[0]{\hat{\Bu}}
\newcommand{\vcap}[0]{\hat{\Bv}}
\newcommand{\wcap}[0]{\hat{\Bw}}
\newcommand{\xcap}[0]{\hat{\Bx}}
\newcommand{\ycap}[0]{\hat{\By}}
\newcommand{\zcap}[0]{\hat{\Bz}}
\newcommand{\thetacap}[0]{\hat{\Btheta}}

%
% to write R^n and C^n in a distinguishable fashion.  Perhaps change this
% to the double lined characters upon figuring out how to do so.
%
\newcommand{\C}[1]{$\mathbb{C}^{#1}$}
\newcommand{\R}[1]{$\mathbb{R}^{#1}$}

%
% various generally useful helpers
%

% derivative of #1 wrt. #2:
\newcommand{\D}[2] {\frac {d#2} {d#1}}

\newcommand{\inv}[1]{\frac{1}{#1}}
\newcommand{\cross}[0]{\times}

\newcommand{\abs}[1]{\lvert{#1}\rvert}
\newcommand{\norm}[1]{\lVert{#1}\rVert}
\newcommand{\innerprod}[2]{\langle{#1}, {#2}\rangle}
\newcommand{\dotprod}[2]{{#1} \cdot {#2}}
\newcommand{\bdotprod}[2]{\left({#1} \cdot {#2}\right)}
\newcommand{\crossprod}[2]{{#1} \cross {#2}}
\newcommand{\tripleprod}[3]{\dotprod{\left(\crossprod{#1}{#2}\right)}{#3}}

\DeclareMathOperator{\Proj}{Proj}
\DeclareMathOperator{\Span}{span}
\DeclareMathOperator{\Sgn}{sgn}
\DeclareMathOperator{\Area}{Area}
\DeclareMathOperator{\Volume}{Volume}

%
% A few miscellaneous things specific to this document
%
\newcommand{\crossop}[1]{\crossprod{#1}{}}

% R2 vector.
\newcommand{\VectorTwo}[2]{
\begin{bmatrix}
 {#1} \\
 {#2}
\end{bmatrix}
}

\newcommand{\VectorN}[1]{
\begin{bmatrix}
{#1}_1 \\
{#1}_2 \\
\vdots \\
{#1}_N \\
\end{bmatrix}
}

\newcommand{\DETuvij}[4]{
\begin{vmatrix}
 {#1}_{#3} & {#1}_{#4} \\
 {#2}_{#3} & {#2}_{#4}
\end{vmatrix}
}

\newcommand{\DETuvwijk}[6]{
\begin{vmatrix}
 {#1}_{#4} & {#1}_{#5} & {#1}_{#6} \\
 {#2}_{#4} & {#2}_{#5} & {#2}_{#6} \\
 {#3}_{#4} & {#3}_{#5} & {#3}_{#6}
\end{vmatrix}
}

\newcommand{\DETuvwxijkl}[8]{
\begin{vmatrix}
 {#1}_{#5} & {#1}_{#6} & {#1}_{#7} & {#1}_{#8} \\
 {#2}_{#5} & {#2}_{#6} & {#2}_{#7} & {#2}_{#8} \\
 {#3}_{#5} & {#3}_{#6} & {#3}_{#7} & {#3}_{#8} \\
 {#4}_{#5} & {#4}_{#6} & {#4}_{#7} & {#4}_{#8} \\
\end{vmatrix}
}

%\newcommand{\DETuvwxyijklm}[10]{
%\begin{vmatrix}
% {#1}_{#6} & {#1}_{#7} & {#1}_{#8} & {#1}_{#9} & {#1}_{#10} \\
% {#2}_{#6} & {#2}_{#7} & {#2}_{#8} & {#2}_{#9} & {#2}_{#10} \\
% {#3}_{#6} & {#3}_{#7} & {#3}_{#8} & {#3}_{#9} & {#3}_{#10} \\
% {#4}_{#6} & {#4}_{#7} & {#4}_{#8} & {#4}_{#9} & {#4}_{#10} \\
% {#5}_{#6} & {#5}_{#7} & {#5}_{#8} & {#5}_{#9} & {#5}_{#10}
%\end{vmatrix}
%}

% R3 vector.
\newcommand{\VectorThree}[3]{
\begin{bmatrix}
 {#1} \\
 {#2} \\
 {#3}
\end{bmatrix}
}


%<misc>
%
\newcommand{\Abs}[1]{{\left\lvert{#1}\right\rvert}}
\newcommand{\spacegrad}[0]{\boldsymbol{\nabla}}
\newcommand{\grad}[0]{\nabla}
\newcommand{\LL}[0]{\mathcal{L}}

% == \partial_{#1} {#2}
\newcommand{\PD}[2]{\frac{\partial {#2}}{\partial {#1}}}
% inline variant
\newcommand{\PDi}[2]{{\partial {#2}}/{\partial {#1}}}

\newcommand{\PDD}[3]{\frac{\partial^2 {#3}}{\partial {#1}\partial {#2}}}
%\newcommand{\PDd}[2]{\frac{\partial^2 {#2}}{{\partial{#1}}^2}}
\newcommand{\PDsq}[2]{\frac{\partial^2 {#2}}{(\partial {#1})^2}}

\newcommand{\Partial}[2]{\frac{\partial {#1}}{\partial {#2}}}
\DeclareMathOperator{\RejName}{Rej}
\newcommand{\Rej}[2]{\RejName_{#1}\left( {#2} \right)}
\newcommand{\Rm}[1]{\mathbb{R}^{#1}}
\newcommand{\Cm}[1]{\mathbb{C}^{#1}}
\newcommand{\conj}[0]{{*}}

%</misc>

% <grade selection>
%
\newcommand{\gpgrade}[2] {{\left\langle{{#1}}\right\rangle}_{#2}}

\newcommand{\gpgradezero}[1] {\gpgrade{#1}{}}
%\newcommand{\gpscalargrade}[1] {{\left\langle{{#1}}\right\rangle}}
%\newcommand{\gpgradezero}[1] {\gpgrade{#1}{0}}

%\newcommand{\gpgradeone}[1] {{\left\langle{{#1}}\right\rangle}_{1}}
\newcommand{\gpgradeone}[1] {\gpgrade{#1}{1}}

\newcommand{\gpgradetwo}[1] {\gpgrade{#1}{2}}
\newcommand{\gpgradethree}[1] {\gpgrade{#1}{3}}
\newcommand{\gpgradefour}[1] {\gpgrade{#1}{4}}
%
% </grade selection>



\newcommand{\adot}[0]{{\dot{a}}}
\newcommand{\bdot}[0]{{\dot{b}}}
% taken for centered dot:
%\newcommand{\cdot}[0]{{\dot{c}}}
%\newcommand{\ddot}[0]{{\dot{d}}}
\newcommand{\edot}[0]{{\dot{e}}}
\newcommand{\fdot}[0]{{\dot{f}}}
\newcommand{\gdot}[0]{{\dot{g}}}
\newcommand{\hdot}[0]{{\dot{h}}}
\newcommand{\idot}[0]{{\dot{i}}}
\newcommand{\jdot}[0]{{\dot{j}}}
\newcommand{\kdot}[0]{{\dot{k}}}
\newcommand{\ldot}[0]{{\dot{l}}}
\newcommand{\mdot}[0]{{\dot{m}}}
\newcommand{\ndot}[0]{{\dot{n}}}
%\newcommand{\odot}[0]{{\dot{o}}}
\newcommand{\pdot}[0]{{\dot{p}}}
\newcommand{\qdot}[0]{{\dot{q}}}
\newcommand{\rdot}[0]{{\dot{r}}}
\newcommand{\sdot}[0]{{\dot{s}}}
\newcommand{\tdot}[0]{{\dot{t}}}
\newcommand{\udot}[0]{{\dot{u}}}
\newcommand{\vdot}[0]{{\dot{v}}}
\newcommand{\wdot}[0]{{\dot{w}}}
\newcommand{\xdot}[0]{{\dot{x}}}
\newcommand{\ydot}[0]{{\dot{y}}}
\newcommand{\zdot}[0]{{\dot{z}}}
\newcommand{\addot}[0]{{\ddot{a}}}
\newcommand{\bddot}[0]{{\ddot{b}}}
\newcommand{\cddot}[0]{{\ddot{c}}}
%\newcommand{\dddot}[0]{{\ddot{d}}}
\newcommand{\eddot}[0]{{\ddot{e}}}
\newcommand{\fddot}[0]{{\ddot{f}}}
\newcommand{\gddot}[0]{{\ddot{g}}}
\newcommand{\hddot}[0]{{\ddot{h}}}
\newcommand{\iddot}[0]{{\ddot{i}}}
\newcommand{\jddot}[0]{{\ddot{j}}}
\newcommand{\kddot}[0]{{\ddot{k}}}
\newcommand{\lddot}[0]{{\ddot{l}}}
\newcommand{\mddot}[0]{{\ddot{m}}}
\newcommand{\nddot}[0]{{\ddot{n}}}
\newcommand{\oddot}[0]{{\ddot{o}}}
\newcommand{\pddot}[0]{{\ddot{p}}}
\newcommand{\qddot}[0]{{\ddot{q}}}
\newcommand{\rddot}[0]{{\ddot{r}}}
\newcommand{\sddot}[0]{{\ddot{s}}}
\newcommand{\tddot}[0]{{\ddot{t}}}
\newcommand{\uddot}[0]{{\ddot{u}}}
\newcommand{\vddot}[0]{{\ddot{v}}}
\newcommand{\wddot}[0]{{\ddot{w}}}
\newcommand{\xddot}[0]{{\ddot{x}}}
\newcommand{\yddot}[0]{{\ddot{y}}}
\newcommand{\zddot}[0]{{\ddot{z}}}

%<bold and dot greek symbols>
%

\newcommand{\Deltadot}[0]{{\dot{\Delta}}}
\newcommand{\Gammadot}[0]{{\dot{\Gamma}}}
\newcommand{\Lambdadot}[0]{{\dot{\Lambda}}}
\newcommand{\Omegadot}[0]{{\dot{\Omega}}}
\newcommand{\Phidot}[0]{{\dot{\Phi}}}
\newcommand{\Pidot}[0]{{\dot{\Pi}}}
\newcommand{\Psidot}[0]{{\dot{\Psi}}}
\newcommand{\Sigmadot}[0]{{\dot{\Sigma}}}
\newcommand{\Thetadot}[0]{{\dot{\Theta}}}
\newcommand{\Upsilondot}[0]{{\dot{\Upsilon}}}
\newcommand{\Xidot}[0]{{\dot{\Xi}}}
\newcommand{\alphadot}[0]{{\dot{\alpha}}}
\newcommand{\betadot}[0]{{\dot{\beta}}}
\newcommand{\chidot}[0]{{\dot{\chi}}}
\newcommand{\deltadot}[0]{{\dot{\delta}}}
\newcommand{\epsilondot}[0]{{\dot{\epsilon}}}
\newcommand{\etadot}[0]{{\dot{\eta}}}
\newcommand{\gammadot}[0]{{\dot{\gamma}}}
\newcommand{\kappadot}[0]{{\dot{\kappa}}}
\newcommand{\lambdadot}[0]{{\dot{\lambda}}}
\newcommand{\mudot}[0]{{\dot{\mu}}}
\newcommand{\nudot}[0]{{\dot{\nu}}}
\newcommand{\omegadot}[0]{{\dot{\omega}}}
\newcommand{\phidot}[0]{{\dot{\phi}}}
\newcommand{\pidot}[0]{{\dot{\pi}}}
\newcommand{\psidot}[0]{{\dot{\psi}}}
\newcommand{\rhodot}[0]{{\dot{\rho}}}
\newcommand{\sigmadot}[0]{{\dot{\sigma}}}
\newcommand{\taudot}[0]{{\dot{\tau}}}
\newcommand{\thetadot}[0]{{\dot{\theta}}}
\newcommand{\upsilondot}[0]{{\dot{\upsilon}}}
\newcommand{\varepsilondot}[0]{{\dot{\varepsilon}}}
\newcommand{\varphidot}[0]{{\dot{\varphi}}}
\newcommand{\varpidot}[0]{{\dot{\varpi}}}
\newcommand{\varrhodot}[0]{{\dot{\varrho}}}
\newcommand{\varsigmadot}[0]{{\dot{\varsigma}}}
\newcommand{\varthetadot}[0]{{\dot{\vartheta}}}
\newcommand{\xidot}[0]{{\dot{\xi}}}
\newcommand{\zetadot}[0]{{\dot{\zeta}}}

\newcommand{\Deltaddot}[0]{{\ddot{\Delta}}}
\newcommand{\Gammaddot}[0]{{\ddot{\Gamma}}}
\newcommand{\Lambdaddot}[0]{{\ddot{\Lambda}}}
\newcommand{\Omegaddot}[0]{{\ddot{\Omega}}}
\newcommand{\Phiddot}[0]{{\ddot{\Phi}}}
\newcommand{\Piddot}[0]{{\ddot{\Pi}}}
\newcommand{\Psiddot}[0]{{\ddot{\Psi}}}
\newcommand{\Sigmaddot}[0]{{\ddot{\Sigma}}}
\newcommand{\Thetaddot}[0]{{\ddot{\Theta}}}
\newcommand{\Upsilonddot}[0]{{\ddot{\Upsilon}}}
\newcommand{\Xiddot}[0]{{\ddot{\Xi}}}
\newcommand{\alphaddot}[0]{{\ddot{\alpha}}}
\newcommand{\betaddot}[0]{{\ddot{\beta}}}
\newcommand{\chiddot}[0]{{\ddot{\chi}}}
\newcommand{\deltaddot}[0]{{\ddot{\delta}}}
\newcommand{\epsilonddot}[0]{{\ddot{\epsilon}}}
\newcommand{\etaddot}[0]{{\ddot{\eta}}}
\newcommand{\gammaddot}[0]{{\ddot{\gamma}}}
\newcommand{\kappaddot}[0]{{\ddot{\kappa}}}
\newcommand{\lambdaddot}[0]{{\ddot{\lambda}}}
\newcommand{\muddot}[0]{{\ddot{\mu}}}
\newcommand{\nuddot}[0]{{\ddot{\nu}}}
\newcommand{\omegaddot}[0]{{\ddot{\omega}}}
\newcommand{\phiddot}[0]{{\ddot{\phi}}}
\newcommand{\piddot}[0]{{\ddot{\pi}}}
\newcommand{\psiddot}[0]{{\ddot{\psi}}}
\newcommand{\rhoddot}[0]{{\ddot{\rho}}}
\newcommand{\sigmaddot}[0]{{\ddot{\sigma}}}
\newcommand{\tauddot}[0]{{\ddot{\tau}}}
\newcommand{\thetaddot}[0]{{\ddot{\theta}}}
\newcommand{\upsilonddot}[0]{{\ddot{\upsilon}}}
\newcommand{\varepsilonddot}[0]{{\ddot{\varepsilon}}}
\newcommand{\varphiddot}[0]{{\ddot{\varphi}}}
\newcommand{\varpiddot}[0]{{\ddot{\varpi}}}
\newcommand{\varrhoddot}[0]{{\ddot{\varrho}}}
\newcommand{\varsigmaddot}[0]{{\ddot{\varsigma}}}
\newcommand{\varthetaddot}[0]{{\ddot{\vartheta}}}
\newcommand{\xiddot}[0]{{\ddot{\xi}}}
\newcommand{\zetaddot}[0]{{\ddot{\zeta}}}

\newcommand{\BDelta}[0]{\boldsymbol{\Delta}}
\newcommand{\BGamma}[0]{\boldsymbol{\Gamma}}
\newcommand{\BLambda}[0]{\boldsymbol{\Lambda}}
\newcommand{\BOmega}[0]{\boldsymbol{\Omega}}
\newcommand{\BPhi}[0]{\boldsymbol{\Phi}}
\newcommand{\BPi}[0]{\boldsymbol{\Pi}}
\newcommand{\BPsi}[0]{\boldsymbol{\Psi}}
\newcommand{\BSigma}[0]{\boldsymbol{\Sigma}}
\newcommand{\BTheta}[0]{\boldsymbol{\Theta}}
\newcommand{\BUpsilon}[0]{\boldsymbol{\Upsilon}}
\newcommand{\BXi}[0]{\boldsymbol{\Xi}}
\newcommand{\Balpha}[0]{\boldsymbol{\alpha}}
\newcommand{\Bbeta}[0]{\boldsymbol{\beta}}
\newcommand{\Bchi}[0]{\boldsymbol{\chi}}
\newcommand{\Bdelta}[0]{\boldsymbol{\delta}}
\newcommand{\Bepsilon}[0]{\boldsymbol{\epsilon}}
\newcommand{\Beta}[0]{\boldsymbol{\eta}}
\newcommand{\Bgamma}[0]{\boldsymbol{\gamma}}
\newcommand{\Bkappa}[0]{\boldsymbol{\kappa}}
\newcommand{\Blambda}[0]{\boldsymbol{\lambda}}
\newcommand{\Bmu}[0]{\boldsymbol{\mu}}
\newcommand{\Bnu}[0]{\boldsymbol{\nu}}
%\newcommand{\Bomega}[0]{\boldsymbol{\omega}}
\newcommand{\Bphi}[0]{\boldsymbol{\phi}}
\newcommand{\Bpi}[0]{\boldsymbol{\pi}}
\newcommand{\Bpsi}[0]{\boldsymbol{\psi}}
\newcommand{\Brho}[0]{\boldsymbol{\rho}}
\newcommand{\Bsigma}[0]{\boldsymbol{\sigma}}
%\newcommand{\Btau}[0]{\boldsymbol{\tau}}
%\newcommand{\Btheta}[0]{\boldsymbol{\theta}}
\newcommand{\Bupsilon}[0]{\boldsymbol{\upsilon}}
\newcommand{\Bvarepsilon}[0]{\boldsymbol{\varepsilon}}
\newcommand{\Bvarphi}[0]{\boldsymbol{\varphi}}
\newcommand{\Bvarpi}[0]{\boldsymbol{\varpi}}
\newcommand{\Bvarrho}[0]{\boldsymbol{\varrho}}
\newcommand{\Bvarsigma}[0]{\boldsymbol{\varsigma}}
\newcommand{\Bvartheta}[0]{\boldsymbol{\vartheta}}
\newcommand{\Bxi}[0]{\boldsymbol{\xi}}
\newcommand{\Bzeta}[0]{\boldsymbol{\zeta}}
%
%</bold and dot greek symbols>
%<infrequent>
%
%\newcommand{\AreaOp}[1]{\AName_{#1}}
%\newcommand{\Babs}[0]{\abs{\BB}}
%\newcommand{\Bcap}[0]{\hat{\BB}}
%\newcommand{\BrPrimeRej}[0]{\rcap(\rcap \wedge \Br')}
%\newcommand{\CA}[0]{\mathcal{A}}
%\newcommand{\Cos}[1]{\cos{\left({#1}\right)}}
%\newcommand{\Det}[1] {\abs{#1}}
%\newcommand{\Dsq}[2] {\frac {\partial^2 {#1}} {\partial {#2}^2}}
%\newcommand{\Exp}[1]{\exp{\left({#1}\right)}}
%\newcommand{\Norm}[1]{\left\lVert{#1}\right\rVert}
%\newcommand{\Sin}[1]{\sin{\left({#1}\right)}}
%\newcommand{\T}[0]{\text{T}}
%\newcommand{\VolumeOp}[1]{\VName_{#1}}
%\newcommand{\agrad}[0]{\Ba \cdot \nabla}
%\newcommand{\alphacap}[0]{\hat{\boldsymbol{\alpha}}}
%\newcommand{\Fcap}[0]{\hat{\BF}}
%\newcommand{\bithree}[0]{{\Bi}_3}
%\newcommand{\bxa}[0]{\Bx\Ba}
%\newcommand{\coordvec}[2]{
%\newcommand{\costheta}[0]{\acap \cdot \xcap}
%\newcommand{\ddt}[1]{\ddot{#1}}
%\newcommand{\ddu}[1] {\frac {d{#1}} {du}}
%\newcommand{\dsqxj}[2] {\frac {\partial^2 {#1}} {\partial {x_{#2}}^2}}
%\newcommand{\dtheta}[1]{\frac{d {#1}}{d \theta}}
%\newcommand{\dt}[1]{\dot{#1}}
%\newcommand{\dt}[1]{\frac{d {#1}}{dt}}
%\newcommand{\dxj}[2] {\frac {\partial {#1}} {\partial {x_{#2}}}}
%\newcommand{\halfPhi}[0]{\frac{\phi}{2}}
%\newcommand{\half}[0]{\inv{2}}
%\newcommand{\inv}[1]{\frac{1}{#1}}
%\newcommand{\laplacian}[0]{\nabla^2}
%\newcommand{\matrixoftx}[3]{
%\newcommand{\nrrp}[0]{\norm{\rcap \wedge \Br'}}
%\newcommand{\oiint}{\bigcirc \hspace{-1.4em} \int \hspace{-.8em} \int}
%\newcommand{\transpose}[1]{{#1}^{\text{T}}}
%\newcommand{\transpose}[1]{{{#1}^{\TextTranspose}}}
%\newcommand{\transpose}[1]{{{#1}^{\text{T}}}}
%\newcommand{\barA}[0]{\bar{A}}
%\newcommand{\qbar}[0]{\bar{q}}
%\newcommand{\qdotbar}[0]{\dot{\bar{q}}}
%
%</infrequent>





%\usepackage{listings}
%\usepackage{txfonts} % for ointctr... (also appears to make "prettier" \int and \sum's)
\usepackage[bookmarks=true]{hyperref}

\usepackage{color,cite,graphicx}
   % use colour in the document, put your citations as [1-4]
   % rather than [1,2,3,4] (it looks nicer, and the extended LaTeX2e
   % graphics package. 
\usepackage{latexsym,amssymb,epsf} % don't remember if these are
   % needed, but their inclusion can't do any damage


\title{ Relativistic acceleration. }
\author{Peeter Joot \quad peeter.joot@gmail.com }
\date{ April 10, 2009.  Last Revision: $Date: 2009/04/11 01:00:47 $ }

\begin{document}

\maketitle{}
\tableofcontents
\section{ Motivation. }

Continuing on with reading of \cite{pauli1981tr}, having 
clarified aspects of the four vector velocity in \cite{PJrelativityFourVectorVelocity}, it is now
time to move on to acceleration.

Do the chain rule calculations for the acceleration four vector equation given in equation (193).

\section{ Compute it. }

Compute the spatial and timelike components of the acceleration

\begin{align*}
B^\mu 
&= \frac{d^2 x^\mu}{d\tau^2} \\
&= \frac{d }{d\tau} \left( \frac{d x^\mu }{d\tau} \right) \\
&= \frac{d }{d\tau} \left( \frac{d x^\mu }{dt} \frac{dt}{d\tau} \right) \\
&= \left( \frac{d }{d\tau} \frac{d x^\mu }{dt} \right) \frac{dt}{d\tau} + \frac{d x^\mu }{dt} \frac{d^2t}{d\tau^2} \\
&= \frac{d^2 x^\mu }{dt^2} \left( \frac{dt}{d\tau} \right)^2 + \frac{d x^\mu }{dt} \frac{d^2t}{d\tau^2} \\
\end{align*}

For $\mu \in \{1,2,3\}$, the ${d^2 x^\mu }/{dt^2}$ terms are the regular old spatial acceleration components.
, and $dx^4/dt = c$.  Writing $\Bu^2 = \sum_{k=1}^3 (dx^k/dt)^2$, and $\beta^2 = \Bu^2/c^2$, we have

\begin{align*}
B^k &= \frac{d^2 x^k }{dt^2} \inv{1-\beta^2} + \frac{d x^k }{dt} \frac{d^2t}{d\tau^2} \\
B^4 &= 0 + c \frac{d^2t}{d\tau^2} \\
\end{align*}

In both of these is the $d^2t/d\tau^2$ term.  Let's expand that.

\begin{align*}
\frac{d^2t}{d\tau^2} 
&= \frac{d}{d\tau} \left( \inv{\sqrt{ 1 - \Bu^2/c^2 }} \right) \\
&= \frac{-1}{c^2} \frac{(-1/2) }{({ 1 - \Bu^2/c^2 })^{3/2}} \frac{d\Bu^2}{d\tau} \\
&= \frac{1}{c^2} \frac{(1/2) }{({ 1 - \Bu^2/c^2 })^{3/2}} 2 \Bu \cdot \frac{d\Bu}{d\tau} \\
&= \frac{1}{c^2} \frac{1}{({ 1 - \Bu^2/c^2 })^{3/2}} \Bu \cdot \frac{d\Bu}{dt} \frac{dt}{d\tau} \\
&= \frac{1}{c^2} \frac{1}{({ 1 - \Bu^2/c^2 })^{2}} \Bu \cdot \frac{d\Bu}{dt} \\
\end{align*}

In vector form, with $\Ba = d\Bu/dt$, we now have the following 

\begin{align*}
\BB &= \Ba \inv{1-\beta^2} + \Bu (\Bu \cdot \Ba) \inv{c^2} \inv{ (1-\beta^2)^2} \\
B^4 &= \inv{c} (\Bu \cdot \Ba) \inv{ (1-\beta^2)^2}
\end{align*}

This reproduces the equation from the Pauli text (except for the imaginary factor $i$ due to the Minkowski notation).  Except for $\gamma$ factors this calculation has a similar final form to that of the decomposition of acceleration in terms of radial components.

\bibliographystyle{plainnat}
\bibliography{myrefs}

\end{document}

%\documentclass{article}

%\usepackage{amsmath}
\usepackage{mathpazo}

%
% shorthand for bold symbols, convenient for vectors and matrices
%
\newcommand{\Ba}[0]{\mathbf{a}}
\newcommand{\Bb}[0]{\mathbf{b}}
\newcommand{\Bc}[0]{\mathbf{c}}
\newcommand{\Bd}[0]{\mathbf{d}}
\newcommand{\Be}[0]{\mathbf{e}}
\newcommand{\Bf}[0]{\mathbf{f}}
\newcommand{\Bg}[0]{\mathbf{g}}
\newcommand{\Bh}[0]{\mathbf{h}}
\newcommand{\Bi}[0]{\mathbf{i}}
\newcommand{\Bj}[0]{\mathbf{j}}
\newcommand{\Bk}[0]{\mathbf{k}}
\newcommand{\Bl}[0]{\mathbf{l}}
\newcommand{\Bm}[0]{\mathbf{m}}
\newcommand{\Bn}[0]{\mathbf{n}}
\newcommand{\Bo}[0]{\mathbf{o}}
\newcommand{\Bp}[0]{\mathbf{p}}
\newcommand{\Bq}[0]{\mathbf{q}}
\newcommand{\Br}[0]{\mathbf{r}}
\newcommand{\Bs}[0]{\mathbf{s}}
\newcommand{\Bt}[0]{\mathbf{t}}
\newcommand{\Bu}[0]{\mathbf{u}}
\newcommand{\Bv}[0]{\mathbf{v}}
\newcommand{\Bw}[0]{\mathbf{w}}
\newcommand{\Bx}[0]{\mathbf{x}}
\newcommand{\By}[0]{\mathbf{y}}
\newcommand{\Bz}[0]{\mathbf{z}}
\newcommand{\BA}[0]{\mathbf{A}}
\newcommand{\BB}[0]{\mathbf{B}}
\newcommand{\BC}[0]{\mathbf{C}}
\newcommand{\BD}[0]{\mathbf{D}}
\newcommand{\BE}[0]{\mathbf{E}}
\newcommand{\BF}[0]{\mathbf{F}}
\newcommand{\BG}[0]{\mathbf{G}}
\newcommand{\BH}[0]{\mathbf{H}}
\newcommand{\BI}[0]{\mathbf{I}}
\newcommand{\BJ}[0]{\mathbf{J}}
\newcommand{\BK}[0]{\mathbf{K}}
\newcommand{\BL}[0]{\mathbf{L}}
\newcommand{\BM}[0]{\mathbf{M}}
\newcommand{\BN}[0]{\mathbf{N}}
\newcommand{\BO}[0]{\mathbf{O}}
\newcommand{\BP}[0]{\mathbf{P}}
\newcommand{\BQ}[0]{\mathbf{Q}}
\newcommand{\BR}[0]{\mathbf{R}}
\newcommand{\BS}[0]{\mathbf{S}}
\newcommand{\BT}[0]{\mathbf{T}}
\newcommand{\BU}[0]{\mathbf{U}}
\newcommand{\BV}[0]{\mathbf{V}}
\newcommand{\BW}[0]{\mathbf{W}}
\newcommand{\BX}[0]{\mathbf{X}}
\newcommand{\BY}[0]{\mathbf{Y}}
\newcommand{\BZ}[0]{\mathbf{Z}}

\newcommand{\Bzero}[0]{\mathbf{0}}
\newcommand{\Btheta}[0]{\boldsymbol{\theta}}
\newcommand{\Btau}[0]{\boldsymbol{\tau}}
\newcommand{\Bomega}[0]{\boldsymbol{\omega}}

%
% shorthand for unit vectors
%
\newcommand{\acap}[0]{\hat{\Ba}}
\newcommand{\bcap}[0]{\hat{\Bb}}
\newcommand{\ccap}[0]{\hat{\Bc}}
\newcommand{\dcap}[0]{\hat{\Bd}}
\newcommand{\ecap}[0]{\hat{\Be}}
\newcommand{\fcap}[0]{\hat{\Bf}}
\newcommand{\gcap}[0]{\hat{\Bg}}
\newcommand{\hcap}[0]{\hat{\Bh}}
\newcommand{\icap}[0]{\hat{\Bi}}
\newcommand{\jcap}[0]{\hat{\Bj}}
\newcommand{\kcap}[0]{\hat{\Bk}}
\newcommand{\lcap}[0]{\hat{\Bl}}
\newcommand{\mcap}[0]{\hat{\Bm}}
\newcommand{\ncap}[0]{\hat{\Bn}}
\newcommand{\ocap}[0]{\hat{\Bo}}
\newcommand{\pcap}[0]{\hat{\Bp}}
\newcommand{\qcap}[0]{\hat{\Bq}}
\newcommand{\rcap}[0]{\hat{\Br}}
\newcommand{\scap}[0]{\hat{\Bs}}
\newcommand{\tcap}[0]{\hat{\Bt}}
\newcommand{\ucap}[0]{\hat{\Bu}}
\newcommand{\vcap}[0]{\hat{\Bv}}
\newcommand{\wcap}[0]{\hat{\Bw}}
\newcommand{\xcap}[0]{\hat{\Bx}}
\newcommand{\ycap}[0]{\hat{\By}}
\newcommand{\zcap}[0]{\hat{\Bz}}
\newcommand{\thetacap}[0]{\hat{\Btheta}}

%
% to write R^n and C^n in a distinguishable fashion.  Perhaps change this
% to the double lined characters upon figuring out how to do so.
%
\newcommand{\C}[1]{$\mathbb{C}^{#1}$}
\newcommand{\R}[1]{$\mathbb{R}^{#1}$}

%
% various generally useful helpers
%

% derivative of #1 wrt. #2:
\newcommand{\D}[2] {\frac {d#2} {d#1}}

\newcommand{\inv}[1]{\frac{1}{#1}}
\newcommand{\cross}[0]{\times}

\newcommand{\abs}[1]{\lvert{#1}\rvert}
\newcommand{\norm}[1]{\lVert{#1}\rVert}
\newcommand{\innerprod}[2]{\langle{#1}, {#2}\rangle}
\newcommand{\dotprod}[2]{{#1} \cdot {#2}}
\newcommand{\bdotprod}[2]{\left({#1} \cdot {#2}\right)}
\newcommand{\crossprod}[2]{{#1} \cross {#2}}
\newcommand{\tripleprod}[3]{\dotprod{\left(\crossprod{#1}{#2}\right)}{#3}}

\DeclareMathOperator{\Proj}{Proj}
\DeclareMathOperator{\Span}{span}
\DeclareMathOperator{\Sgn}{sgn}
\DeclareMathOperator{\Area}{Area}
\DeclareMathOperator{\Volume}{Volume}

%
% A few miscellaneous things specific to this document
%
\newcommand{\crossop}[1]{\crossprod{#1}{}}

% R2 vector.
\newcommand{\VectorTwo}[2]{
\begin{bmatrix}
 {#1} \\
 {#2}
\end{bmatrix}
}

\newcommand{\VectorN}[1]{
\begin{bmatrix}
{#1}_1 \\
{#1}_2 \\
\vdots \\
{#1}_N \\
\end{bmatrix}
}

\newcommand{\DETuvij}[4]{
\begin{vmatrix}
 {#1}_{#3} & {#1}_{#4} \\
 {#2}_{#3} & {#2}_{#4}
\end{vmatrix}
}

\newcommand{\DETuvwijk}[6]{
\begin{vmatrix}
 {#1}_{#4} & {#1}_{#5} & {#1}_{#6} \\
 {#2}_{#4} & {#2}_{#5} & {#2}_{#6} \\
 {#3}_{#4} & {#3}_{#5} & {#3}_{#6}
\end{vmatrix}
}

\newcommand{\DETuvwxijkl}[8]{
\begin{vmatrix}
 {#1}_{#5} & {#1}_{#6} & {#1}_{#7} & {#1}_{#8} \\
 {#2}_{#5} & {#2}_{#6} & {#2}_{#7} & {#2}_{#8} \\
 {#3}_{#5} & {#3}_{#6} & {#3}_{#7} & {#3}_{#8} \\
 {#4}_{#5} & {#4}_{#6} & {#4}_{#7} & {#4}_{#8} \\
\end{vmatrix}
}

%\newcommand{\DETuvwxyijklm}[10]{
%\begin{vmatrix}
% {#1}_{#6} & {#1}_{#7} & {#1}_{#8} & {#1}_{#9} & {#1}_{#10} \\
% {#2}_{#6} & {#2}_{#7} & {#2}_{#8} & {#2}_{#9} & {#2}_{#10} \\
% {#3}_{#6} & {#3}_{#7} & {#3}_{#8} & {#3}_{#9} & {#3}_{#10} \\
% {#4}_{#6} & {#4}_{#7} & {#4}_{#8} & {#4}_{#9} & {#4}_{#10} \\
% {#5}_{#6} & {#5}_{#7} & {#5}_{#8} & {#5}_{#9} & {#5}_{#10}
%\end{vmatrix}
%}

% R3 vector.
\newcommand{\VectorThree}[3]{
\begin{bmatrix}
 {#1} \\
 {#2} \\
 {#3}
\end{bmatrix}
}


%%<misc>
%
\newcommand{\Abs}[1]{{\left\lvert{#1}\right\rvert}}
\newcommand{\spacegrad}[0]{\boldsymbol{\nabla}}
\newcommand{\grad}[0]{\nabla}
\newcommand{\LL}[0]{\mathcal{L}}

% == \partial_{#1} {#2}
\newcommand{\PD}[2]{\frac{\partial {#2}}{\partial {#1}}}
% inline variant
\newcommand{\PDi}[2]{{\partial {#2}}/{\partial {#1}}}

\newcommand{\PDD}[3]{\frac{\partial^2 {#3}}{\partial {#1}\partial {#2}}}
%\newcommand{\PDd}[2]{\frac{\partial^2 {#2}}{{\partial{#1}}^2}}
\newcommand{\PDsq}[2]{\frac{\partial^2 {#2}}{(\partial {#1})^2}}

\newcommand{\Partial}[2]{\frac{\partial {#1}}{\partial {#2}}}
\DeclareMathOperator{\RejName}{Rej}
\newcommand{\Rej}[2]{\RejName_{#1}\left( {#2} \right)}
\newcommand{\Rm}[1]{\mathbb{R}^{#1}}
\newcommand{\Cm}[1]{\mathbb{C}^{#1}}
\newcommand{\conj}[0]{{*}}

%</misc>

% <grade selection>
%
\newcommand{\gpgrade}[2] {{\left\langle{{#1}}\right\rangle}_{#2}}

\newcommand{\gpgradezero}[1] {\gpgrade{#1}{}}
%\newcommand{\gpscalargrade}[1] {{\left\langle{{#1}}\right\rangle}}
%\newcommand{\gpgradezero}[1] {\gpgrade{#1}{0}}

%\newcommand{\gpgradeone}[1] {{\left\langle{{#1}}\right\rangle}_{1}}
\newcommand{\gpgradeone}[1] {\gpgrade{#1}{1}}

\newcommand{\gpgradetwo}[1] {\gpgrade{#1}{2}}
\newcommand{\gpgradethree}[1] {\gpgrade{#1}{3}}
\newcommand{\gpgradefour}[1] {\gpgrade{#1}{4}}
%
% </grade selection>



\newcommand{\adot}[0]{{\dot{a}}}
\newcommand{\bdot}[0]{{\dot{b}}}
% taken for centered dot:
%\newcommand{\cdot}[0]{{\dot{c}}}
%\newcommand{\ddot}[0]{{\dot{d}}}
\newcommand{\edot}[0]{{\dot{e}}}
\newcommand{\fdot}[0]{{\dot{f}}}
\newcommand{\gdot}[0]{{\dot{g}}}
\newcommand{\hdot}[0]{{\dot{h}}}
\newcommand{\idot}[0]{{\dot{i}}}
\newcommand{\jdot}[0]{{\dot{j}}}
\newcommand{\kdot}[0]{{\dot{k}}}
\newcommand{\ldot}[0]{{\dot{l}}}
\newcommand{\mdot}[0]{{\dot{m}}}
\newcommand{\ndot}[0]{{\dot{n}}}
%\newcommand{\odot}[0]{{\dot{o}}}
\newcommand{\pdot}[0]{{\dot{p}}}
\newcommand{\qdot}[0]{{\dot{q}}}
\newcommand{\rdot}[0]{{\dot{r}}}
\newcommand{\sdot}[0]{{\dot{s}}}
\newcommand{\tdot}[0]{{\dot{t}}}
\newcommand{\udot}[0]{{\dot{u}}}
\newcommand{\vdot}[0]{{\dot{v}}}
\newcommand{\wdot}[0]{{\dot{w}}}
\newcommand{\xdot}[0]{{\dot{x}}}
\newcommand{\ydot}[0]{{\dot{y}}}
\newcommand{\zdot}[0]{{\dot{z}}}
\newcommand{\addot}[0]{{\ddot{a}}}
\newcommand{\bddot}[0]{{\ddot{b}}}
\newcommand{\cddot}[0]{{\ddot{c}}}
%\newcommand{\dddot}[0]{{\ddot{d}}}
\newcommand{\eddot}[0]{{\ddot{e}}}
\newcommand{\fddot}[0]{{\ddot{f}}}
\newcommand{\gddot}[0]{{\ddot{g}}}
\newcommand{\hddot}[0]{{\ddot{h}}}
\newcommand{\iddot}[0]{{\ddot{i}}}
\newcommand{\jddot}[0]{{\ddot{j}}}
\newcommand{\kddot}[0]{{\ddot{k}}}
\newcommand{\lddot}[0]{{\ddot{l}}}
\newcommand{\mddot}[0]{{\ddot{m}}}
\newcommand{\nddot}[0]{{\ddot{n}}}
\newcommand{\oddot}[0]{{\ddot{o}}}
\newcommand{\pddot}[0]{{\ddot{p}}}
\newcommand{\qddot}[0]{{\ddot{q}}}
\newcommand{\rddot}[0]{{\ddot{r}}}
\newcommand{\sddot}[0]{{\ddot{s}}}
\newcommand{\tddot}[0]{{\ddot{t}}}
\newcommand{\uddot}[0]{{\ddot{u}}}
\newcommand{\vddot}[0]{{\ddot{v}}}
\newcommand{\wddot}[0]{{\ddot{w}}}
\newcommand{\xddot}[0]{{\ddot{x}}}
\newcommand{\yddot}[0]{{\ddot{y}}}
\newcommand{\zddot}[0]{{\ddot{z}}}

%<bold and dot greek symbols>
%

\newcommand{\Deltadot}[0]{{\dot{\Delta}}}
\newcommand{\Gammadot}[0]{{\dot{\Gamma}}}
\newcommand{\Lambdadot}[0]{{\dot{\Lambda}}}
\newcommand{\Omegadot}[0]{{\dot{\Omega}}}
\newcommand{\Phidot}[0]{{\dot{\Phi}}}
\newcommand{\Pidot}[0]{{\dot{\Pi}}}
\newcommand{\Psidot}[0]{{\dot{\Psi}}}
\newcommand{\Sigmadot}[0]{{\dot{\Sigma}}}
\newcommand{\Thetadot}[0]{{\dot{\Theta}}}
\newcommand{\Upsilondot}[0]{{\dot{\Upsilon}}}
\newcommand{\Xidot}[0]{{\dot{\Xi}}}
\newcommand{\alphadot}[0]{{\dot{\alpha}}}
\newcommand{\betadot}[0]{{\dot{\beta}}}
\newcommand{\chidot}[0]{{\dot{\chi}}}
\newcommand{\deltadot}[0]{{\dot{\delta}}}
\newcommand{\epsilondot}[0]{{\dot{\epsilon}}}
\newcommand{\etadot}[0]{{\dot{\eta}}}
\newcommand{\gammadot}[0]{{\dot{\gamma}}}
\newcommand{\kappadot}[0]{{\dot{\kappa}}}
\newcommand{\lambdadot}[0]{{\dot{\lambda}}}
\newcommand{\mudot}[0]{{\dot{\mu}}}
\newcommand{\nudot}[0]{{\dot{\nu}}}
\newcommand{\omegadot}[0]{{\dot{\omega}}}
\newcommand{\phidot}[0]{{\dot{\phi}}}
\newcommand{\pidot}[0]{{\dot{\pi}}}
\newcommand{\psidot}[0]{{\dot{\psi}}}
\newcommand{\rhodot}[0]{{\dot{\rho}}}
\newcommand{\sigmadot}[0]{{\dot{\sigma}}}
\newcommand{\taudot}[0]{{\dot{\tau}}}
\newcommand{\thetadot}[0]{{\dot{\theta}}}
\newcommand{\upsilondot}[0]{{\dot{\upsilon}}}
\newcommand{\varepsilondot}[0]{{\dot{\varepsilon}}}
\newcommand{\varphidot}[0]{{\dot{\varphi}}}
\newcommand{\varpidot}[0]{{\dot{\varpi}}}
\newcommand{\varrhodot}[0]{{\dot{\varrho}}}
\newcommand{\varsigmadot}[0]{{\dot{\varsigma}}}
\newcommand{\varthetadot}[0]{{\dot{\vartheta}}}
\newcommand{\xidot}[0]{{\dot{\xi}}}
\newcommand{\zetadot}[0]{{\dot{\zeta}}}

\newcommand{\Deltaddot}[0]{{\ddot{\Delta}}}
\newcommand{\Gammaddot}[0]{{\ddot{\Gamma}}}
\newcommand{\Lambdaddot}[0]{{\ddot{\Lambda}}}
\newcommand{\Omegaddot}[0]{{\ddot{\Omega}}}
\newcommand{\Phiddot}[0]{{\ddot{\Phi}}}
\newcommand{\Piddot}[0]{{\ddot{\Pi}}}
\newcommand{\Psiddot}[0]{{\ddot{\Psi}}}
\newcommand{\Sigmaddot}[0]{{\ddot{\Sigma}}}
\newcommand{\Thetaddot}[0]{{\ddot{\Theta}}}
\newcommand{\Upsilonddot}[0]{{\ddot{\Upsilon}}}
\newcommand{\Xiddot}[0]{{\ddot{\Xi}}}
\newcommand{\alphaddot}[0]{{\ddot{\alpha}}}
\newcommand{\betaddot}[0]{{\ddot{\beta}}}
\newcommand{\chiddot}[0]{{\ddot{\chi}}}
\newcommand{\deltaddot}[0]{{\ddot{\delta}}}
\newcommand{\epsilonddot}[0]{{\ddot{\epsilon}}}
\newcommand{\etaddot}[0]{{\ddot{\eta}}}
\newcommand{\gammaddot}[0]{{\ddot{\gamma}}}
\newcommand{\kappaddot}[0]{{\ddot{\kappa}}}
\newcommand{\lambdaddot}[0]{{\ddot{\lambda}}}
\newcommand{\muddot}[0]{{\ddot{\mu}}}
\newcommand{\nuddot}[0]{{\ddot{\nu}}}
\newcommand{\omegaddot}[0]{{\ddot{\omega}}}
\newcommand{\phiddot}[0]{{\ddot{\phi}}}
\newcommand{\piddot}[0]{{\ddot{\pi}}}
\newcommand{\psiddot}[0]{{\ddot{\psi}}}
\newcommand{\rhoddot}[0]{{\ddot{\rho}}}
\newcommand{\sigmaddot}[0]{{\ddot{\sigma}}}
\newcommand{\tauddot}[0]{{\ddot{\tau}}}
\newcommand{\thetaddot}[0]{{\ddot{\theta}}}
\newcommand{\upsilonddot}[0]{{\ddot{\upsilon}}}
\newcommand{\varepsilonddot}[0]{{\ddot{\varepsilon}}}
\newcommand{\varphiddot}[0]{{\ddot{\varphi}}}
\newcommand{\varpiddot}[0]{{\ddot{\varpi}}}
\newcommand{\varrhoddot}[0]{{\ddot{\varrho}}}
\newcommand{\varsigmaddot}[0]{{\ddot{\varsigma}}}
\newcommand{\varthetaddot}[0]{{\ddot{\vartheta}}}
\newcommand{\xiddot}[0]{{\ddot{\xi}}}
\newcommand{\zetaddot}[0]{{\ddot{\zeta}}}

\newcommand{\BDelta}[0]{\boldsymbol{\Delta}}
\newcommand{\BGamma}[0]{\boldsymbol{\Gamma}}
\newcommand{\BLambda}[0]{\boldsymbol{\Lambda}}
\newcommand{\BOmega}[0]{\boldsymbol{\Omega}}
\newcommand{\BPhi}[0]{\boldsymbol{\Phi}}
\newcommand{\BPi}[0]{\boldsymbol{\Pi}}
\newcommand{\BPsi}[0]{\boldsymbol{\Psi}}
\newcommand{\BSigma}[0]{\boldsymbol{\Sigma}}
\newcommand{\BTheta}[0]{\boldsymbol{\Theta}}
\newcommand{\BUpsilon}[0]{\boldsymbol{\Upsilon}}
\newcommand{\BXi}[0]{\boldsymbol{\Xi}}
\newcommand{\Balpha}[0]{\boldsymbol{\alpha}}
\newcommand{\Bbeta}[0]{\boldsymbol{\beta}}
\newcommand{\Bchi}[0]{\boldsymbol{\chi}}
\newcommand{\Bdelta}[0]{\boldsymbol{\delta}}
\newcommand{\Bepsilon}[0]{\boldsymbol{\epsilon}}
\newcommand{\Beta}[0]{\boldsymbol{\eta}}
\newcommand{\Bgamma}[0]{\boldsymbol{\gamma}}
\newcommand{\Bkappa}[0]{\boldsymbol{\kappa}}
\newcommand{\Blambda}[0]{\boldsymbol{\lambda}}
\newcommand{\Bmu}[0]{\boldsymbol{\mu}}
\newcommand{\Bnu}[0]{\boldsymbol{\nu}}
%\newcommand{\Bomega}[0]{\boldsymbol{\omega}}
\newcommand{\Bphi}[0]{\boldsymbol{\phi}}
\newcommand{\Bpi}[0]{\boldsymbol{\pi}}
\newcommand{\Bpsi}[0]{\boldsymbol{\psi}}
\newcommand{\Brho}[0]{\boldsymbol{\rho}}
\newcommand{\Bsigma}[0]{\boldsymbol{\sigma}}
%\newcommand{\Btau}[0]{\boldsymbol{\tau}}
%\newcommand{\Btheta}[0]{\boldsymbol{\theta}}
\newcommand{\Bupsilon}[0]{\boldsymbol{\upsilon}}
\newcommand{\Bvarepsilon}[0]{\boldsymbol{\varepsilon}}
\newcommand{\Bvarphi}[0]{\boldsymbol{\varphi}}
\newcommand{\Bvarpi}[0]{\boldsymbol{\varpi}}
\newcommand{\Bvarrho}[0]{\boldsymbol{\varrho}}
\newcommand{\Bvarsigma}[0]{\boldsymbol{\varsigma}}
\newcommand{\Bvartheta}[0]{\boldsymbol{\vartheta}}
\newcommand{\Bxi}[0]{\boldsymbol{\xi}}
\newcommand{\Bzeta}[0]{\boldsymbol{\zeta}}
%
%</bold and dot greek symbols>
%<infrequent>
%
%\newcommand{\AreaOp}[1]{\AName_{#1}}
%\newcommand{\Babs}[0]{\abs{\BB}}
%\newcommand{\Bcap}[0]{\hat{\BB}}
%\newcommand{\BrPrimeRej}[0]{\rcap(\rcap \wedge \Br')}
%\newcommand{\CA}[0]{\mathcal{A}}
%\newcommand{\Cos}[1]{\cos{\left({#1}\right)}}
%\newcommand{\Det}[1] {\abs{#1}}
%\newcommand{\Dsq}[2] {\frac {\partial^2 {#1}} {\partial {#2}^2}}
%\newcommand{\Exp}[1]{\exp{\left({#1}\right)}}
%\newcommand{\Norm}[1]{\left\lVert{#1}\right\rVert}
%\newcommand{\Sin}[1]{\sin{\left({#1}\right)}}
%\newcommand{\T}[0]{\text{T}}
%\newcommand{\VolumeOp}[1]{\VName_{#1}}
%\newcommand{\agrad}[0]{\Ba \cdot \nabla}
%\newcommand{\alphacap}[0]{\hat{\boldsymbol{\alpha}}}
%\newcommand{\Fcap}[0]{\hat{\BF}}
%\newcommand{\bithree}[0]{{\Bi}_3}
%\newcommand{\bxa}[0]{\Bx\Ba}
%\newcommand{\coordvec}[2]{
%\newcommand{\costheta}[0]{\acap \cdot \xcap}
%\newcommand{\ddt}[1]{\ddot{#1}}
%\newcommand{\ddu}[1] {\frac {d{#1}} {du}}
%\newcommand{\dsqxj}[2] {\frac {\partial^2 {#1}} {\partial {x_{#2}}^2}}
%\newcommand{\dtheta}[1]{\frac{d {#1}}{d \theta}}
%\newcommand{\dt}[1]{\dot{#1}}
%\newcommand{\dt}[1]{\frac{d {#1}}{dt}}
%\newcommand{\dxj}[2] {\frac {\partial {#1}} {\partial {x_{#2}}}}
%\newcommand{\halfPhi}[0]{\frac{\phi}{2}}
%\newcommand{\half}[0]{\inv{2}}
%\newcommand{\inv}[1]{\frac{1}{#1}}
%\newcommand{\laplacian}[0]{\nabla^2}
%\newcommand{\matrixoftx}[3]{
%\newcommand{\nrrp}[0]{\norm{\rcap \wedge \Br'}}
%\newcommand{\oiint}{\bigcirc \hspace{-1.4em} \int \hspace{-.8em} \int}
%\newcommand{\transpose}[1]{{#1}^{\text{T}}}
%\newcommand{\transpose}[1]{{{#1}^{\TextTranspose}}}
%\newcommand{\transpose}[1]{{{#1}^{\text{T}}}}
%\newcommand{\barA}[0]{\bar{A}}
%\newcommand{\qbar}[0]{\bar{q}}
%\newcommand{\qdotbar}[0]{\dot{\bar{q}}}
%
%</infrequent>





%\usepackage{listings}
%\usepackage{txfonts} % for ointctr... (also appears to make "prettier" \int and \sum's)
%\usepackage[bookmarks=true]{hyperref}

%\usepackage{color,cite,graphicx}
   % use colour in the document, put your citations as [1-4]
   % rather than [1,2,3,4] (it looks nicer, and the extended LaTeX2e
   % graphics package. 
%\usepackage{latexsym,amssymb,epsf} % don't remember if these are
   % needed, but their inclusion can't do any damage


\chapter{Four vector velocity addition notes. }
\label{chap:pauliFourVectorV}
%\author{Peeter Joot \quad peeter.joot@gmail.com }
\date{ April 8, 2009.  $RCSfile: pauliFourVectorV.tex,v $ Last $Revision: 1.14 $ $Date: 2009/06/14 23:51:45 $ }

%\begin{document}

%\maketitle{}
%\tableofcontents
\section{Motivation. }

Reconcile four vector transformed velocity coordinates with non-covariant form.
Specifically, equations (10) and (191) in \citep{pauli1981tr} look considerably
different on the surface, but must have the same content.

Equations (10) were also derived in a bit more detail than in Pauli's book in
\citep{PJpauliVelocityAddition} and are

\begin{align}\label{eqn:pauli_four_vector_v:eqn10}
u_x &= \frac{ {u_x}' + v  }{ 1 + v {u_x}'/c^2} \\
u_y &= \frac{{u_y}'}{\gamma (1 + v {u_x}'/c^2)} \\
u_z &= \frac{{u_z}'}{\gamma (1 + v {u_x}'/c^2)} \\
\gamma^{-1} &= \sqrt{ 1 - v^2/c^2}
\end{align}

whereas equations (191) are given as

\begin{align}\label{eqn:pauli_four_vector_v:eqn191}
{u^1}' &= \gamma ( u^1 + i (v/c) u^4) \\
{u^2}' &= {u^2} \\
{u^3}' &= {u^3} \\
{u^4}' &= \gamma ( u^4 - i (v/c) u^1)
\end{align}

\section{Derive the transformed velocity equations. }

Pauli uses a $(+,+,+,-)$ metric, with $ct = x^4 = - x_4$.  For much of his SR treatment he also uses the Minkowski representation $x^4 = x_4 = ict$. In the first representation we have

\begin{align*}
-c^2 
&= \frac{dx^\mu}{d\tau} \frac{dx_\mu}{d\tau} \\
&= \frac{dx^k}{d\tau} \frac{dx_k}{d\tau} + \frac{dx^4}{d\tau} \frac{dx_4}{d\tau} \\
&= \left(\frac{dt}{d\tau}\right)^2 \left( \sum_{k=1}^3 \left(\frac{dx^k}{dt}\right)^2 - \left(\frac{dx^4}{dt}\right)^2 \right) \\
&= \left(\frac{dt}{d\tau}\right)^2 \left( \Bu^2 - c^2 \right) \\
\end{align*}

Shuffling and taking roots produces a $\gamma$ factor by virtue of the invariant

\begin{align*}
%\gamma 
%&= 
\frac{dt}{d\tau} \\
&= \inv{\sqrt{ 1 - \Bu^2/c^2 }}
\end{align*}

This is enough to write the proper velocity in terms of a space time split 

\begin{align*}
\dot{X} 
&= \left(\frac{dx^\mu}{d\tau}\right) \\
&= \inv{\sqrt{ 1 - \Bu^2/c^2 }} (\Bu, c)
\end{align*}

As a four vector this can be Lorentz boosted.  For an 
x-axis boost we have

\begin{align*}
{\begin{bmatrix}
u^1 \\
u^2 \\
u^3 \\
u^4 \\
\end{bmatrix}}'
&=
\begin{bmatrix}
\gamma & 0 & 0 & - \gamma \beta \\
0 & 1 & 0 & 0 \\
0 & 0 & 1 & 0 \\
- \gamma \beta & 0 & 0 & \gamma \\
\end{bmatrix}
{\begin{bmatrix}
u^1 \\
u^2 \\
u^3 \\
u^4 \\
\end{bmatrix}} 
\\
\gamma &= \inv{\sqrt{ 1 - \beta^2 }} \\
\end{align*}

Expanding this we have

\begin{align}\label{eqn:pauli_four_vector_v:realRepresentationFourVector}
{u^1}' &= \gamma ( u^1 - \beta u^4) \\
{u^2}' &= {u^2} \\
{u^3}' &= {u^3} \\
{u^4}' &= \gamma ( u^4 - \beta u^1)
\end{align}

In the imaginary representation the Lorentz transform takes the form

\begin{align*}
{\begin{bmatrix}
u^1 \\
u^2 \\
u^3 \\
u^4 \\
\end{bmatrix}}'
&=
\begin{bmatrix}
\gamma & 0 & 0 & i \gamma \beta \\
0 & 1 & 0 & 0 \\
0 & 0 & 1 & 0 \\
- i \gamma \beta & 0 & 0 & \gamma \\
\end{bmatrix}
{\begin{bmatrix}
u^1 \\
u^2 \\
u^3 \\
u^4 \\
\end{bmatrix}} 
\end{align*}

Let's verify that this produces the same result by expansion

\begin{align*}
{u^1}' &= \gamma ( u^1 + i \beta u^4) \\
{u^2}' &= {u^2} \\
{u^3}' &= {u^3} \\
{u^4}' &= \gamma ( u^4 - \beta i u^1)
\end{align*}

with $u^4 \rightarrow i u^4$ to switch to a real representation this is

\begin{align*}
{u^1}' &= \gamma ( u^1 - \beta u^4) \\
{u^2}' &= {u^2} \\
{u^3}' &= {u^3} \\
{u^4}' &= \gamma ( u^4 - \beta u^1)
\end{align*}

Good.  This matches equations \ref{eqn:pauli_four_vector_v:realRepresentationFourVector}.  Now, we want to put these in an explicit space time representation
to compare against \ref{eqn:pauli_four_vector_v:eqn10}.  Since those are in real form, work with the real representation instead of the imaginary Minkowski
representation for such a comparison.

\subsection{WRONG: Non-covariant representation of the transformed velocity. }

Expanding out the proper time derivatives (assuming that $dx'/dt' = v$ is a correct interpretation of the math), we have

\begin{align*}
\inv{\sqrt{1 - v^2/c^2}} \frac{{dx'}^1}{dt'} &= \inv{\sqrt{1 - v^2/c^2}} \inv{\sqrt{1 - \Bu^2/c^2}} \left( \frac{dx^1}{dt} - \beta c \right) \\
\inv{\sqrt{1 - v^2/c^2}} \frac{{dx'}^2}{dt'} &= \inv{\sqrt{1 - \Bu^2/c^2}} \frac{dx^2}{dt} \\
\inv{\sqrt{1 - v^2/c^2}} \frac{{dx'}^3}{dt'} &= \inv{\sqrt{1 - \Bu^2/c^2}} \frac{dx^3}{dt} \\
\inv{\sqrt{1 - v^2/c^2}} \frac{{dx'}^4}{dt'} &= \inv{\sqrt{1 - v^2/c^2}} \inv{\sqrt{1 - \Bu^2/c^2}} \left( c - \beta \frac{dx^1}{dt} \right)
\end{align*}

Hmm.  That doesn't appear to match.

\section{Try again from scratch. }

\subsection{Boost a stationary particle. }

Instead of starting with a proper velocity with a spatial component, let's cut the complexity and consider the simplest case, a particle at rest.  The worldline (in two dimensions) for a particle in its rest frame is

\begin{align*}
X = (0, ct) 
\end{align*}

The proper velocity for this particle is 

\begin{align*}
u = \frac{dX}{d\tau} = \left(0, c\frac{dt}{d\tau} \right) 
\end{align*}

But since this is a particle in its rest frame $dt/d\tau = 1$, this proper velocity is

\begin{align*}
u = \left(0, c \right) 
\end{align*}

Observe that the norm of this vector (still using the time negative metric signature) is

\begin{align*}
u \cdot u = 0^2 - c^2 = -c^2
\end{align*}

Now, what happens when we apply a Lorentz boost to this?

\begin{align*}
u' &= 
\begin{bmatrix}
\gamma & - \gamma \beta \\
- \gamma \beta & \gamma \\
\end{bmatrix}
\begin{bmatrix}
0 \\
c
\end{bmatrix} \\
\end{align*}

This is
\begin{align}\label{eqn:pauli_four_vector_v:uPrime}
u' &= 
\gamma
\begin{bmatrix}
- \beta \\
1 \\
\end{bmatrix}
c
\end{align}

What's the norm of this vector.  It should be unchanged, so let's verify.

\begin{align*}
u' \cdot u' 
&= \gamma^2 \left( (- \beta)^2 - 1^2 \right) c^2 \\
&= - \gamma^2 \left( 1 - \beta^2 \right) c^2 \\
&= - c^2 \\
\end{align*}

Good, still have the expected $-c^2$ value.  For this boosted vector, what is $dt'/d\tau'$?

Note that in general for the components of $u'$ we have

\begin{align*}
\frac{{dx'}^\mu}{d\tau'}
&=
\frac{{dx'}^\mu}{dt'} \frac{{dt'}}{d\tau'}
\end{align*}

and in particular we have ${u'}^4 = c dt'/d\tau$ %, since the proper time $\tau'$ in the primed frame measures the time for the particle at rest
%in that frame.  This gives

\begin{align*}
{u'}^4
&=
\frac{{dx'}^4}{dt'} \frac{{dt'}}{d\tau'} \\
&= c \frac{{dt'}}{d\tau'} \\
\end{align*}

Comparing to \ref{eqn:pauli_four_vector_v:uPrime} we have

\begin{align*}
{u'}^4 
&= \gamma c \\
&= c \frac{{dt'}}{d\tau'} \\
\end{align*}

and therefore can write

\begin{align*}
\frac{{dt'}}{d\tau'} 
&= \gamma 
\end{align*}

Similarly the spatial velocity of the particle in the boosted frame is

\begin{align*}
{u'}^1
&=
\frac{{dx'}^1}{dt'} \frac{{dt'}}{d\tau'} \\
&= u_x' \frac{{dt'}}{d\tau'} \\
&= - \gamma v
\end{align*}

So we have 

\begin{align*}
u_x' = -v 
\end{align*}

This seems to make sense.  We move the frame along the positive x-axis, so a particle at rest at the origin of the stationary frame has a velocity $v$ in the opposite direction from the viewpoint of something at rest in the moving frame.

\subsection{Apply a second boost transformation. }

Okay, treating the almost too simple case in detail was helpful to see where to go next.  Now that we have a view of a particle at rest
from a moving frame, let's apply another boost so we have a second frame moving with relative velocity $\beta'$ with respect to the moving
frame.  Our transformation is

\begin{align*}
L' =
\begin{bmatrix}
\gamma' & - \gamma' \beta' \\
- \gamma' \beta' & \gamma' \\
\end{bmatrix}
\end{align*}

this second transformation takes the original proper velocity to
\begin{align*}
u'' &=
\gamma \gamma'
\begin{bmatrix}
1 & - \beta' \\
- \beta' & 1 \\
\end{bmatrix}
\begin{bmatrix}
- \beta \\
1 \\
\end{bmatrix}
c \\
\end{align*}

This is
\begin{align}\label{eqn:pauli_four_vector_v:boost2}
u'' &=
\gamma \gamma'
\begin{bmatrix}
-(\beta + \beta') \\
1 + \beta\beta'
\end{bmatrix}
c
\end{align}

Let's verify that we still have our invariant norm.

\begin{align*}
u'' \cdot u'' 
&=
\gamma^2 {\gamma'}^2
\left(
(\beta + \beta')^2 
-(1 + \beta\beta')^2
\right)
c^2 \\
&=
\gamma^2 {\gamma'}^2
\left(
\beta^2
+{\beta'}^2
+2 \beta\beta'
-1
-2 \beta\beta'
-\beta^2 {\beta'}^2
\right)
c^2 \\
&=
\gamma^2 {\gamma'}^2
\left(
\beta^2 (1 - {\beta'}^2)
-(1 -{\beta'}^2)
\right)
c^2 \\
&=
-\gamma^2 {\gamma'}^2 (1 -{\beta'}^2)(1 -\beta^2) c^2 \\
&=
- c^2 \\
\end{align*}

Now, we have ${u''}^4 = c dt''/d\tau''$ as before, so from equation
\ref{eqn:pauli_four_vector_v:boost2} the new compound $\gamma$ factor can be picked off

\begin{align*}
\frac{dt''}{d\tau''} &=
\gamma \gamma'( 1 + \beta\beta' )
\end{align*}

Using this and chain rule again we have the spatial velocity in the second moving frame for the particle at rest in the original frame.  This is

\begin{align*}
u_x'' 
&=
\frac{\frac{dx''}{dt''}}{ \frac{dt''}{d\tau''} } \\
&=
\frac{-\gamma \gamma' (\beta + \beta') c}{ \gamma \gamma'( 1 + \beta\beta' ) } \\
&=
\frac{-(\beta + \beta') c}{ 1 + \beta\beta' } \\
&=
\frac{-(v + v') }{ 1 + v v'/c^2 } \\
\end{align*}

Okay, good.  From consideration of proper velocities and their transformations we have something that is of
the form of Pauli's equation 10 (here equation \ref{eqn:pauli_four_vector_v:eqn10}), which is the standard form for colinear 
relativistic velocity addition.

There is a difference though, namely that Pauli's equation 10 expresses the reverse transformation.  Shuffling
equation \ref{eqn:pauli_four_vector_v:eqn10} to solve for $u_x'$, we have

\begin{align*}
u_x ( 1 + v {u_x}') &= { {u_x}' + v  } \\
\end{align*}

which gives

\begin{align*}
u_x' &= \frac{ {u_x} + (-v)  }{ 1 + (-v) {u_x}} \\
\end{align*}

An algebraic inversion of the equation has exactly the same form, but with the velocity negated in sign.

Now with $u_x = -v'$ we have an identification between this twice boosted frame observing the particle at
rest in the original frame.

\subsection{Perpendicular directions. }

Now, the only thing left to understand is the spatial representation of the boosted velocity 
for the perpendicular to the boost direction components.

To do so, let's treat a more general case for the proper velocity of a particle as seen in some observers ``rest frame''.  Given the particle worldline

\begin{align*}
X = (x^\mu)
\end{align*}

The proper velocity is

\begin{align*}
\frac{dX}{d\tau} = \left(\frac{x^k}{dt}, c \right) \frac{dt}{d\tau}
\end{align*}

Writing

\begin{align*}
u_x &= \frac{x^1}{dt} \\
u_y &= \frac{x^2}{dt} \\
u_z &= \frac{x^3}{dt} \\
\gamma_0 &= \frac{dt}{d\tau}
\end{align*}

Application of a boost produces

\begin{align*}
u'
&=
\begin{bmatrix}
\gamma & 0 & 0 & - \gamma \beta \\
0 & 1 & 0 & 0 \\
0 & 0 & 1 & 0 \\
- \gamma \beta & 0 & 0 & \gamma \\
\end{bmatrix}
\begin{bmatrix}
u_x \\
u_y \\
u_z \\
c
\end{bmatrix}
\gamma_0 \\
&=
\begin{bmatrix}
\gamma_0 \gamma (u_x - \beta c) \\
\gamma_0 u_y \\
\gamma_0 u_z \\
\gamma_0 \gamma ( -\beta u_x + c ) \\
\end{bmatrix} \\
\end{align*}

In particular we have

\begin{align*}
\frac{dx'}{d\tau'} &= \gamma_0 \gamma ( 1 -\beta u_x/c ) \\
\end{align*}

So can write

\begin{align*}
u_x' &=
\frac{\gamma_0 \gamma (u_x - \beta c) }
{\gamma_0 \gamma ( 1 -\beta u_x/c )} \\
u_y' &=
\frac{\gamma_0 u_y }
{\gamma_0 \gamma ( 1 -\beta u_x/c )} \\
u_z' &=
\frac{\gamma_0 u_z }
{\gamma_0 \gamma ( 1 -\beta u_x/c )} \\
\end{align*}

Reversing signs in $\beta$ to invert and canceling common factors this is

\begin{align*}
u_x &=
\frac{u_x' + v }
{ 1 + v u_x'/c^2 } \\
u_y &=
\frac{u_y' }
{\gamma ( 1 + v u_x'/c^2 )} \\
u_z &=
\frac{u_z' }
{\gamma ( 1 + v u_x'/c^2 )} \\
\end{align*}

A final substitution of $\gamma^{-1} = \sqrt{1 - v^2/c^2}$ and we have
equation \ref{eqn:pauli_four_vector_v:eqn10} as desired.  Pauli says this step is easy, and that's
true enough once the simpler cases are first understood.

%\bibliographystyle{plainnat}
%\bibliography{myrefs}

%\end{document}

%\documentclass{article}

%\usepackage{amsmath}
\usepackage{mathpazo}

%
% shorthand for bold symbols, convenient for vectors and matrices
%
\newcommand{\Ba}[0]{\mathbf{a}}
\newcommand{\Bb}[0]{\mathbf{b}}
\newcommand{\Bc}[0]{\mathbf{c}}
\newcommand{\Bd}[0]{\mathbf{d}}
\newcommand{\Be}[0]{\mathbf{e}}
\newcommand{\Bf}[0]{\mathbf{f}}
\newcommand{\Bg}[0]{\mathbf{g}}
\newcommand{\Bh}[0]{\mathbf{h}}
\newcommand{\Bi}[0]{\mathbf{i}}
\newcommand{\Bj}[0]{\mathbf{j}}
\newcommand{\Bk}[0]{\mathbf{k}}
\newcommand{\Bl}[0]{\mathbf{l}}
\newcommand{\Bm}[0]{\mathbf{m}}
\newcommand{\Bn}[0]{\mathbf{n}}
\newcommand{\Bo}[0]{\mathbf{o}}
\newcommand{\Bp}[0]{\mathbf{p}}
\newcommand{\Bq}[0]{\mathbf{q}}
\newcommand{\Br}[0]{\mathbf{r}}
\newcommand{\Bs}[0]{\mathbf{s}}
\newcommand{\Bt}[0]{\mathbf{t}}
\newcommand{\Bu}[0]{\mathbf{u}}
\newcommand{\Bv}[0]{\mathbf{v}}
\newcommand{\Bw}[0]{\mathbf{w}}
\newcommand{\Bx}[0]{\mathbf{x}}
\newcommand{\By}[0]{\mathbf{y}}
\newcommand{\Bz}[0]{\mathbf{z}}
\newcommand{\BA}[0]{\mathbf{A}}
\newcommand{\BB}[0]{\mathbf{B}}
\newcommand{\BC}[0]{\mathbf{C}}
\newcommand{\BD}[0]{\mathbf{D}}
\newcommand{\BE}[0]{\mathbf{E}}
\newcommand{\BF}[0]{\mathbf{F}}
\newcommand{\BG}[0]{\mathbf{G}}
\newcommand{\BH}[0]{\mathbf{H}}
\newcommand{\BI}[0]{\mathbf{I}}
\newcommand{\BJ}[0]{\mathbf{J}}
\newcommand{\BK}[0]{\mathbf{K}}
\newcommand{\BL}[0]{\mathbf{L}}
\newcommand{\BM}[0]{\mathbf{M}}
\newcommand{\BN}[0]{\mathbf{N}}
\newcommand{\BO}[0]{\mathbf{O}}
\newcommand{\BP}[0]{\mathbf{P}}
\newcommand{\BQ}[0]{\mathbf{Q}}
\newcommand{\BR}[0]{\mathbf{R}}
\newcommand{\BS}[0]{\mathbf{S}}
\newcommand{\BT}[0]{\mathbf{T}}
\newcommand{\BU}[0]{\mathbf{U}}
\newcommand{\BV}[0]{\mathbf{V}}
\newcommand{\BW}[0]{\mathbf{W}}
\newcommand{\BX}[0]{\mathbf{X}}
\newcommand{\BY}[0]{\mathbf{Y}}
\newcommand{\BZ}[0]{\mathbf{Z}}

\newcommand{\Bzero}[0]{\mathbf{0}}
\newcommand{\Btheta}[0]{\boldsymbol{\theta}}
\newcommand{\Btau}[0]{\boldsymbol{\tau}}
\newcommand{\Bomega}[0]{\boldsymbol{\omega}}

%
% shorthand for unit vectors
%
\newcommand{\acap}[0]{\hat{\Ba}}
\newcommand{\bcap}[0]{\hat{\Bb}}
\newcommand{\ccap}[0]{\hat{\Bc}}
\newcommand{\dcap}[0]{\hat{\Bd}}
\newcommand{\ecap}[0]{\hat{\Be}}
\newcommand{\fcap}[0]{\hat{\Bf}}
\newcommand{\gcap}[0]{\hat{\Bg}}
\newcommand{\hcap}[0]{\hat{\Bh}}
\newcommand{\icap}[0]{\hat{\Bi}}
\newcommand{\jcap}[0]{\hat{\Bj}}
\newcommand{\kcap}[0]{\hat{\Bk}}
\newcommand{\lcap}[0]{\hat{\Bl}}
\newcommand{\mcap}[0]{\hat{\Bm}}
\newcommand{\ncap}[0]{\hat{\Bn}}
\newcommand{\ocap}[0]{\hat{\Bo}}
\newcommand{\pcap}[0]{\hat{\Bp}}
\newcommand{\qcap}[0]{\hat{\Bq}}
\newcommand{\rcap}[0]{\hat{\Br}}
\newcommand{\scap}[0]{\hat{\Bs}}
\newcommand{\tcap}[0]{\hat{\Bt}}
\newcommand{\ucap}[0]{\hat{\Bu}}
\newcommand{\vcap}[0]{\hat{\Bv}}
\newcommand{\wcap}[0]{\hat{\Bw}}
\newcommand{\xcap}[0]{\hat{\Bx}}
\newcommand{\ycap}[0]{\hat{\By}}
\newcommand{\zcap}[0]{\hat{\Bz}}
\newcommand{\thetacap}[0]{\hat{\Btheta}}

%
% to write R^n and C^n in a distinguishable fashion.  Perhaps change this
% to the double lined characters upon figuring out how to do so.
%
\newcommand{\C}[1]{$\mathbb{C}^{#1}$}
\newcommand{\R}[1]{$\mathbb{R}^{#1}$}

%
% various generally useful helpers
%

% derivative of #1 wrt. #2:
\newcommand{\D}[2] {\frac {d#2} {d#1}}

\newcommand{\inv}[1]{\frac{1}{#1}}
\newcommand{\cross}[0]{\times}

\newcommand{\abs}[1]{\lvert{#1}\rvert}
\newcommand{\norm}[1]{\lVert{#1}\rVert}
\newcommand{\innerprod}[2]{\langle{#1}, {#2}\rangle}
\newcommand{\dotprod}[2]{{#1} \cdot {#2}}
\newcommand{\bdotprod}[2]{\left({#1} \cdot {#2}\right)}
\newcommand{\crossprod}[2]{{#1} \cross {#2}}
\newcommand{\tripleprod}[3]{\dotprod{\left(\crossprod{#1}{#2}\right)}{#3}}

\DeclareMathOperator{\Proj}{Proj}
\DeclareMathOperator{\Span}{span}
\DeclareMathOperator{\Sgn}{sgn}
\DeclareMathOperator{\Area}{Area}
\DeclareMathOperator{\Volume}{Volume}

%
% A few miscellaneous things specific to this document
%
\newcommand{\crossop}[1]{\crossprod{#1}{}}

% R2 vector.
\newcommand{\VectorTwo}[2]{
\begin{bmatrix}
 {#1} \\
 {#2}
\end{bmatrix}
}

\newcommand{\VectorN}[1]{
\begin{bmatrix}
{#1}_1 \\
{#1}_2 \\
\vdots \\
{#1}_N \\
\end{bmatrix}
}

\newcommand{\DETuvij}[4]{
\begin{vmatrix}
 {#1}_{#3} & {#1}_{#4} \\
 {#2}_{#3} & {#2}_{#4}
\end{vmatrix}
}

\newcommand{\DETuvwijk}[6]{
\begin{vmatrix}
 {#1}_{#4} & {#1}_{#5} & {#1}_{#6} \\
 {#2}_{#4} & {#2}_{#5} & {#2}_{#6} \\
 {#3}_{#4} & {#3}_{#5} & {#3}_{#6}
\end{vmatrix}
}

\newcommand{\DETuvwxijkl}[8]{
\begin{vmatrix}
 {#1}_{#5} & {#1}_{#6} & {#1}_{#7} & {#1}_{#8} \\
 {#2}_{#5} & {#2}_{#6} & {#2}_{#7} & {#2}_{#8} \\
 {#3}_{#5} & {#3}_{#6} & {#3}_{#7} & {#3}_{#8} \\
 {#4}_{#5} & {#4}_{#6} & {#4}_{#7} & {#4}_{#8} \\
\end{vmatrix}
}

%\newcommand{\DETuvwxyijklm}[10]{
%\begin{vmatrix}
% {#1}_{#6} & {#1}_{#7} & {#1}_{#8} & {#1}_{#9} & {#1}_{#10} \\
% {#2}_{#6} & {#2}_{#7} & {#2}_{#8} & {#2}_{#9} & {#2}_{#10} \\
% {#3}_{#6} & {#3}_{#7} & {#3}_{#8} & {#3}_{#9} & {#3}_{#10} \\
% {#4}_{#6} & {#4}_{#7} & {#4}_{#8} & {#4}_{#9} & {#4}_{#10} \\
% {#5}_{#6} & {#5}_{#7} & {#5}_{#8} & {#5}_{#9} & {#5}_{#10}
%\end{vmatrix}
%}

% R3 vector.
\newcommand{\VectorThree}[3]{
\begin{bmatrix}
 {#1} \\
 {#2} \\
 {#3}
\end{bmatrix}
}


%%<misc>
%
\newcommand{\Abs}[1]{{\left\lvert{#1}\right\rvert}}
\newcommand{\spacegrad}[0]{\boldsymbol{\nabla}}
\newcommand{\grad}[0]{\nabla}
\newcommand{\LL}[0]{\mathcal{L}}

% == \partial_{#1} {#2}
\newcommand{\PD}[2]{\frac{\partial {#2}}{\partial {#1}}}
% inline variant
\newcommand{\PDi}[2]{{\partial {#2}}/{\partial {#1}}}

\newcommand{\PDD}[3]{\frac{\partial^2 {#3}}{\partial {#1}\partial {#2}}}
%\newcommand{\PDd}[2]{\frac{\partial^2 {#2}}{{\partial{#1}}^2}}
\newcommand{\PDsq}[2]{\frac{\partial^2 {#2}}{(\partial {#1})^2}}

\newcommand{\Partial}[2]{\frac{\partial {#1}}{\partial {#2}}}
\DeclareMathOperator{\RejName}{Rej}
\newcommand{\Rej}[2]{\RejName_{#1}\left( {#2} \right)}
\newcommand{\Rm}[1]{\mathbb{R}^{#1}}
\newcommand{\Cm}[1]{\mathbb{C}^{#1}}
\newcommand{\conj}[0]{{*}}

%</misc>

% <grade selection>
%
\newcommand{\gpgrade}[2] {{\left\langle{{#1}}\right\rangle}_{#2}}

\newcommand{\gpgradezero}[1] {\gpgrade{#1}{}}
%\newcommand{\gpscalargrade}[1] {{\left\langle{{#1}}\right\rangle}}
%\newcommand{\gpgradezero}[1] {\gpgrade{#1}{0}}

%\newcommand{\gpgradeone}[1] {{\left\langle{{#1}}\right\rangle}_{1}}
\newcommand{\gpgradeone}[1] {\gpgrade{#1}{1}}

\newcommand{\gpgradetwo}[1] {\gpgrade{#1}{2}}
\newcommand{\gpgradethree}[1] {\gpgrade{#1}{3}}
\newcommand{\gpgradefour}[1] {\gpgrade{#1}{4}}
%
% </grade selection>



\newcommand{\adot}[0]{{\dot{a}}}
\newcommand{\bdot}[0]{{\dot{b}}}
% taken for centered dot:
%\newcommand{\cdot}[0]{{\dot{c}}}
%\newcommand{\ddot}[0]{{\dot{d}}}
\newcommand{\edot}[0]{{\dot{e}}}
\newcommand{\fdot}[0]{{\dot{f}}}
\newcommand{\gdot}[0]{{\dot{g}}}
\newcommand{\hdot}[0]{{\dot{h}}}
\newcommand{\idot}[0]{{\dot{i}}}
\newcommand{\jdot}[0]{{\dot{j}}}
\newcommand{\kdot}[0]{{\dot{k}}}
\newcommand{\ldot}[0]{{\dot{l}}}
\newcommand{\mdot}[0]{{\dot{m}}}
\newcommand{\ndot}[0]{{\dot{n}}}
%\newcommand{\odot}[0]{{\dot{o}}}
\newcommand{\pdot}[0]{{\dot{p}}}
\newcommand{\qdot}[0]{{\dot{q}}}
\newcommand{\rdot}[0]{{\dot{r}}}
\newcommand{\sdot}[0]{{\dot{s}}}
\newcommand{\tdot}[0]{{\dot{t}}}
\newcommand{\udot}[0]{{\dot{u}}}
\newcommand{\vdot}[0]{{\dot{v}}}
\newcommand{\wdot}[0]{{\dot{w}}}
\newcommand{\xdot}[0]{{\dot{x}}}
\newcommand{\ydot}[0]{{\dot{y}}}
\newcommand{\zdot}[0]{{\dot{z}}}
\newcommand{\addot}[0]{{\ddot{a}}}
\newcommand{\bddot}[0]{{\ddot{b}}}
\newcommand{\cddot}[0]{{\ddot{c}}}
%\newcommand{\dddot}[0]{{\ddot{d}}}
\newcommand{\eddot}[0]{{\ddot{e}}}
\newcommand{\fddot}[0]{{\ddot{f}}}
\newcommand{\gddot}[0]{{\ddot{g}}}
\newcommand{\hddot}[0]{{\ddot{h}}}
\newcommand{\iddot}[0]{{\ddot{i}}}
\newcommand{\jddot}[0]{{\ddot{j}}}
\newcommand{\kddot}[0]{{\ddot{k}}}
\newcommand{\lddot}[0]{{\ddot{l}}}
\newcommand{\mddot}[0]{{\ddot{m}}}
\newcommand{\nddot}[0]{{\ddot{n}}}
\newcommand{\oddot}[0]{{\ddot{o}}}
\newcommand{\pddot}[0]{{\ddot{p}}}
\newcommand{\qddot}[0]{{\ddot{q}}}
\newcommand{\rddot}[0]{{\ddot{r}}}
\newcommand{\sddot}[0]{{\ddot{s}}}
\newcommand{\tddot}[0]{{\ddot{t}}}
\newcommand{\uddot}[0]{{\ddot{u}}}
\newcommand{\vddot}[0]{{\ddot{v}}}
\newcommand{\wddot}[0]{{\ddot{w}}}
\newcommand{\xddot}[0]{{\ddot{x}}}
\newcommand{\yddot}[0]{{\ddot{y}}}
\newcommand{\zddot}[0]{{\ddot{z}}}

%<bold and dot greek symbols>
%

\newcommand{\Deltadot}[0]{{\dot{\Delta}}}
\newcommand{\Gammadot}[0]{{\dot{\Gamma}}}
\newcommand{\Lambdadot}[0]{{\dot{\Lambda}}}
\newcommand{\Omegadot}[0]{{\dot{\Omega}}}
\newcommand{\Phidot}[0]{{\dot{\Phi}}}
\newcommand{\Pidot}[0]{{\dot{\Pi}}}
\newcommand{\Psidot}[0]{{\dot{\Psi}}}
\newcommand{\Sigmadot}[0]{{\dot{\Sigma}}}
\newcommand{\Thetadot}[0]{{\dot{\Theta}}}
\newcommand{\Upsilondot}[0]{{\dot{\Upsilon}}}
\newcommand{\Xidot}[0]{{\dot{\Xi}}}
\newcommand{\alphadot}[0]{{\dot{\alpha}}}
\newcommand{\betadot}[0]{{\dot{\beta}}}
\newcommand{\chidot}[0]{{\dot{\chi}}}
\newcommand{\deltadot}[0]{{\dot{\delta}}}
\newcommand{\epsilondot}[0]{{\dot{\epsilon}}}
\newcommand{\etadot}[0]{{\dot{\eta}}}
\newcommand{\gammadot}[0]{{\dot{\gamma}}}
\newcommand{\kappadot}[0]{{\dot{\kappa}}}
\newcommand{\lambdadot}[0]{{\dot{\lambda}}}
\newcommand{\mudot}[0]{{\dot{\mu}}}
\newcommand{\nudot}[0]{{\dot{\nu}}}
\newcommand{\omegadot}[0]{{\dot{\omega}}}
\newcommand{\phidot}[0]{{\dot{\phi}}}
\newcommand{\pidot}[0]{{\dot{\pi}}}
\newcommand{\psidot}[0]{{\dot{\psi}}}
\newcommand{\rhodot}[0]{{\dot{\rho}}}
\newcommand{\sigmadot}[0]{{\dot{\sigma}}}
\newcommand{\taudot}[0]{{\dot{\tau}}}
\newcommand{\thetadot}[0]{{\dot{\theta}}}
\newcommand{\upsilondot}[0]{{\dot{\upsilon}}}
\newcommand{\varepsilondot}[0]{{\dot{\varepsilon}}}
\newcommand{\varphidot}[0]{{\dot{\varphi}}}
\newcommand{\varpidot}[0]{{\dot{\varpi}}}
\newcommand{\varrhodot}[0]{{\dot{\varrho}}}
\newcommand{\varsigmadot}[0]{{\dot{\varsigma}}}
\newcommand{\varthetadot}[0]{{\dot{\vartheta}}}
\newcommand{\xidot}[0]{{\dot{\xi}}}
\newcommand{\zetadot}[0]{{\dot{\zeta}}}

\newcommand{\Deltaddot}[0]{{\ddot{\Delta}}}
\newcommand{\Gammaddot}[0]{{\ddot{\Gamma}}}
\newcommand{\Lambdaddot}[0]{{\ddot{\Lambda}}}
\newcommand{\Omegaddot}[0]{{\ddot{\Omega}}}
\newcommand{\Phiddot}[0]{{\ddot{\Phi}}}
\newcommand{\Piddot}[0]{{\ddot{\Pi}}}
\newcommand{\Psiddot}[0]{{\ddot{\Psi}}}
\newcommand{\Sigmaddot}[0]{{\ddot{\Sigma}}}
\newcommand{\Thetaddot}[0]{{\ddot{\Theta}}}
\newcommand{\Upsilonddot}[0]{{\ddot{\Upsilon}}}
\newcommand{\Xiddot}[0]{{\ddot{\Xi}}}
\newcommand{\alphaddot}[0]{{\ddot{\alpha}}}
\newcommand{\betaddot}[0]{{\ddot{\beta}}}
\newcommand{\chiddot}[0]{{\ddot{\chi}}}
\newcommand{\deltaddot}[0]{{\ddot{\delta}}}
\newcommand{\epsilonddot}[0]{{\ddot{\epsilon}}}
\newcommand{\etaddot}[0]{{\ddot{\eta}}}
\newcommand{\gammaddot}[0]{{\ddot{\gamma}}}
\newcommand{\kappaddot}[0]{{\ddot{\kappa}}}
\newcommand{\lambdaddot}[0]{{\ddot{\lambda}}}
\newcommand{\muddot}[0]{{\ddot{\mu}}}
\newcommand{\nuddot}[0]{{\ddot{\nu}}}
\newcommand{\omegaddot}[0]{{\ddot{\omega}}}
\newcommand{\phiddot}[0]{{\ddot{\phi}}}
\newcommand{\piddot}[0]{{\ddot{\pi}}}
\newcommand{\psiddot}[0]{{\ddot{\psi}}}
\newcommand{\rhoddot}[0]{{\ddot{\rho}}}
\newcommand{\sigmaddot}[0]{{\ddot{\sigma}}}
\newcommand{\tauddot}[0]{{\ddot{\tau}}}
\newcommand{\thetaddot}[0]{{\ddot{\theta}}}
\newcommand{\upsilonddot}[0]{{\ddot{\upsilon}}}
\newcommand{\varepsilonddot}[0]{{\ddot{\varepsilon}}}
\newcommand{\varphiddot}[0]{{\ddot{\varphi}}}
\newcommand{\varpiddot}[0]{{\ddot{\varpi}}}
\newcommand{\varrhoddot}[0]{{\ddot{\varrho}}}
\newcommand{\varsigmaddot}[0]{{\ddot{\varsigma}}}
\newcommand{\varthetaddot}[0]{{\ddot{\vartheta}}}
\newcommand{\xiddot}[0]{{\ddot{\xi}}}
\newcommand{\zetaddot}[0]{{\ddot{\zeta}}}

\newcommand{\BDelta}[0]{\boldsymbol{\Delta}}
\newcommand{\BGamma}[0]{\boldsymbol{\Gamma}}
\newcommand{\BLambda}[0]{\boldsymbol{\Lambda}}
\newcommand{\BOmega}[0]{\boldsymbol{\Omega}}
\newcommand{\BPhi}[0]{\boldsymbol{\Phi}}
\newcommand{\BPi}[0]{\boldsymbol{\Pi}}
\newcommand{\BPsi}[0]{\boldsymbol{\Psi}}
\newcommand{\BSigma}[0]{\boldsymbol{\Sigma}}
\newcommand{\BTheta}[0]{\boldsymbol{\Theta}}
\newcommand{\BUpsilon}[0]{\boldsymbol{\Upsilon}}
\newcommand{\BXi}[0]{\boldsymbol{\Xi}}
\newcommand{\Balpha}[0]{\boldsymbol{\alpha}}
\newcommand{\Bbeta}[0]{\boldsymbol{\beta}}
\newcommand{\Bchi}[0]{\boldsymbol{\chi}}
\newcommand{\Bdelta}[0]{\boldsymbol{\delta}}
\newcommand{\Bepsilon}[0]{\boldsymbol{\epsilon}}
\newcommand{\Beta}[0]{\boldsymbol{\eta}}
\newcommand{\Bgamma}[0]{\boldsymbol{\gamma}}
\newcommand{\Bkappa}[0]{\boldsymbol{\kappa}}
\newcommand{\Blambda}[0]{\boldsymbol{\lambda}}
\newcommand{\Bmu}[0]{\boldsymbol{\mu}}
\newcommand{\Bnu}[0]{\boldsymbol{\nu}}
%\newcommand{\Bomega}[0]{\boldsymbol{\omega}}
\newcommand{\Bphi}[0]{\boldsymbol{\phi}}
\newcommand{\Bpi}[0]{\boldsymbol{\pi}}
\newcommand{\Bpsi}[0]{\boldsymbol{\psi}}
\newcommand{\Brho}[0]{\boldsymbol{\rho}}
\newcommand{\Bsigma}[0]{\boldsymbol{\sigma}}
%\newcommand{\Btau}[0]{\boldsymbol{\tau}}
%\newcommand{\Btheta}[0]{\boldsymbol{\theta}}
\newcommand{\Bupsilon}[0]{\boldsymbol{\upsilon}}
\newcommand{\Bvarepsilon}[0]{\boldsymbol{\varepsilon}}
\newcommand{\Bvarphi}[0]{\boldsymbol{\varphi}}
\newcommand{\Bvarpi}[0]{\boldsymbol{\varpi}}
\newcommand{\Bvarrho}[0]{\boldsymbol{\varrho}}
\newcommand{\Bvarsigma}[0]{\boldsymbol{\varsigma}}
\newcommand{\Bvartheta}[0]{\boldsymbol{\vartheta}}
\newcommand{\Bxi}[0]{\boldsymbol{\xi}}
\newcommand{\Bzeta}[0]{\boldsymbol{\zeta}}
%
%</bold and dot greek symbols>
%<infrequent>
%
%\newcommand{\AreaOp}[1]{\AName_{#1}}
%\newcommand{\Babs}[0]{\abs{\BB}}
%\newcommand{\Bcap}[0]{\hat{\BB}}
%\newcommand{\BrPrimeRej}[0]{\rcap(\rcap \wedge \Br')}
%\newcommand{\CA}[0]{\mathcal{A}}
%\newcommand{\Cos}[1]{\cos{\left({#1}\right)}}
%\newcommand{\Det}[1] {\abs{#1}}
%\newcommand{\Dsq}[2] {\frac {\partial^2 {#1}} {\partial {#2}^2}}
%\newcommand{\Exp}[1]{\exp{\left({#1}\right)}}
%\newcommand{\Norm}[1]{\left\lVert{#1}\right\rVert}
%\newcommand{\Sin}[1]{\sin{\left({#1}\right)}}
%\newcommand{\T}[0]{\text{T}}
%\newcommand{\VolumeOp}[1]{\VName_{#1}}
%\newcommand{\agrad}[0]{\Ba \cdot \nabla}
%\newcommand{\alphacap}[0]{\hat{\boldsymbol{\alpha}}}
%\newcommand{\Fcap}[0]{\hat{\BF}}
%\newcommand{\bithree}[0]{{\Bi}_3}
%\newcommand{\bxa}[0]{\Bx\Ba}
%\newcommand{\coordvec}[2]{
%\newcommand{\costheta}[0]{\acap \cdot \xcap}
%\newcommand{\ddt}[1]{\ddot{#1}}
%\newcommand{\ddu}[1] {\frac {d{#1}} {du}}
%\newcommand{\dsqxj}[2] {\frac {\partial^2 {#1}} {\partial {x_{#2}}^2}}
%\newcommand{\dtheta}[1]{\frac{d {#1}}{d \theta}}
%\newcommand{\dt}[1]{\dot{#1}}
%\newcommand{\dt}[1]{\frac{d {#1}}{dt}}
%\newcommand{\dxj}[2] {\frac {\partial {#1}} {\partial {x_{#2}}}}
%\newcommand{\halfPhi}[0]{\frac{\phi}{2}}
%\newcommand{\half}[0]{\inv{2}}
%\newcommand{\inv}[1]{\frac{1}{#1}}
%\newcommand{\laplacian}[0]{\nabla^2}
%\newcommand{\matrixoftx}[3]{
%\newcommand{\nrrp}[0]{\norm{\rcap \wedge \Br'}}
%\newcommand{\oiint}{\bigcirc \hspace{-1.4em} \int \hspace{-.8em} \int}
%\newcommand{\transpose}[1]{{#1}^{\text{T}}}
%\newcommand{\transpose}[1]{{{#1}^{\TextTranspose}}}
%\newcommand{\transpose}[1]{{{#1}^{\text{T}}}}
%\newcommand{\barA}[0]{\bar{A}}
%\newcommand{\qbar}[0]{\bar{q}}
%\newcommand{\qdotbar}[0]{\dot{\bar{q}}}
%
%</infrequent>




%\usepackage[bookmarks=true]{hyperref}

%\usepackage{color,cite,graphicx}
   % use colour in the document, put your citations as [1-4]
   % rather than [1,2,3,4] (it looks nicer, and the extended LaTeX2e
   % graphics package. 
%\usepackage{latexsym,amssymb,epsf} % don't remember if these are
   % needed, but their inclusion can't do any damage


\chapter{Pauli's relativity background in QM intro from "Wave Mechanics". }
%\author{Peeter Joot \quad peeter.joot@gmail.com}
\date{ Jan 24, 2009.  $RCSfile: pauliQmRelativityIntro.tex,v $ Last $Revision: 1.12 $ $Date: 2009/06/11 17:00:37 $ }

%\begin{document}

%\maketitle{}

%\tableofcontents

\section{Motivation. }

In \cite{pauli2000wm} a few relativity notes are made to build up to 
a relativistic wave equation (ie: the Klein-Gordon equation),
and show one can introduce a non-relativistic approximation of this
that has close to the form of a the free particle \Sch equation.  It 
is interesting to see things use relativity as a base.  This is exactly
opposite to the Klein-Gordon treatment in a text such as
\cite{srednicki2007qft} where a way to find a 
relativistically correct form starting from the \Sch equation is searched for.

Pauli's treatment is a bit too terse for me, but has a number of
interesting and illuminating features.  Here I walk through his treatment
at my own pace.

\section{Relativistic mechanics. }

\subsection{Energy in terms of momentum. }

Equation $1.4$ is the famous energy and momentum equations

\begin{align}\label{eqn:pauli_qm_relativity_intro:p1_4}
E &= \frac{ m c^2 }{\sqrt{1 - \Bv^2/c^2}} \\
\Bp &= \frac{ m \Bv }{\sqrt{1 - \Bv^2/c^2}} \\
\end{align}

These pair of these quantities is often now expressed as a four vector in various ways

\begin{align*}
p &= \left(\frac{E}{c}, \Bp \right) \\
p &= \frac{E}{c} \gamma_0 + \Bp \gamma_0 \\
\cdots
\end{align*}

These two quantities are observably interdependent, and this dependency can be made explicit by forming the sum

\begin{align*}
\Bp^2 + m^2 c^2 
&= \frac{ m^2 \Bv^2 }{{1 - \Bv^2/c^2}} +  \frac{ m^2 c^2( 1 - \Bv^2/c^2) }{{1 - \Bv^2/c^2}}  \\
&= \inv{1 - \Bv^2/c^2} m^2 \left( \Bv^2 + c^2( 1 - \Bv^2/c^2) \right) \\
&= \inv{1 - \Bv^2/c^2} m^2 \left( \Bv^2 + c^2 - \Bv^2 \right) \\
&= \inv{1 - \Bv^2/c^2} m^2 c^2 \\
\end{align*}

This recovers Pauli's equation $1.3$ (in the square).

\begin{align}\label{eqn:pauli_qm_relativity_intro:p1_3}
\frac{E^2}{c^2} = \Bp^2 + m^2 c^2 
\end{align}

This is slightly different from how I'm used to seeing this expressed, since Energy is singled out.
Rearranging slightly recovers the scalar invariant for the energy momentum four vector:

\begin{align*}
m^2 c^2 &= \frac{E^2}{c^2} -\Bp^2 
\end{align*}

\subsection{Energy-momentum four vector from Energy }

Now, interestingly, Pauli also points out that his equation \ref{eqn:pauli_qm_relativity_intro:p1_3} can be used to derive the four vector
equations for energy and momentum, only requiring one express the relationship between Kinetic energy and momentum
as one would do in plain old non-relativistic physics.  That is, starting with

\begin{align*}
E &= \inv{2} m \Bv^2 \\
\end{align*}

differentiation with respect to some parameter we can write

\begin{align*}
\frac{dE}{d\alpha} 
&= m \Bv \cdot \frac{d\Bv}{d\alpha} \\
&= \Bv \cdot \frac{d\Bp}{d\alpha} \\
\end{align*}

If the specific parametrization of the path is implied we have

\begin{align}\label{eqn:pauli_qm_relativity_intro:EvP}
{dE} &= \Bv \cdot {d\Bp}.
\end{align}

In coordinates this gives

\begin{align*}
{dE} &= \sum_k v_k dp_k \\
\end{align*}

Pauli uses this to express the velocity coordinates in terms of energy and momentum, and writes

\begin{align}\label{eqn:pauli_qm_relativity_intro:vEp}
v_k &= \PD{p_k}{E}
\end{align}

My way of getting this seems a bit fishy, dropping the explicit parametrization to get the one form, and then switching magically to partials, but once one gets to the end result it does not appear unreasonable.

Perhaps better is to skip the one form business completely, writing

\begin{align*}
E = \inv{2} \Bv \cdot \Bp = \inv{2} \sum_k v_k p_k
\end{align*}

But taking partials from this to get \ref{eqn:pauli_qm_relativity_intro:vEp} requires care since $p_k$ and $v_k$ are dependent.
%Perhaps notable is that Pauli goes from \ref{eqn:pauli_qm_relativity_intro:EvP} to \ref{eqn:pauli_qm_relativity_intro:vEp} directly.

Assuming \ref{eqn:pauli_qm_relativity_intro:vEp} is valid and applying this to \ref{eqn:pauli_qm_relativity_intro:p1_3}, it is relatively straightforward to
recover the four-vector energy-momentum equations.  

From
\begin{align*}
E &= \sqrt{\Bp^2 c^2 + m^2 c^4 } \\
\end{align*}

we calculate
\begin{align*}
v_k 
&= \PD{p_k}{E} \\
&= (2 p_k c^2) \inv{2} \frac{1}{\sqrt{\Bp^2 c^2 + m^2 c^4 }} \\
&= c^2 \frac{p_k}{E}
\end{align*}

Summing over all components
\begin{align*}
\frac{\Bv^2}{c^2} 
&= \sum_k \frac{(v_k)^2}{c^2} \\
&= c^2 \sum_k \frac{{p_k}^2}{E^2} \\
&= c^2 \frac{\Bp^2}{E^2} \\
\end{align*}

Subtracting this from one, gives us our gamma factor (squared), which is

\begin{align*}
1 - \frac{\Bv^2}{c^2} 
&= 1 - c^2 \frac{\Bp^2}{E^2} \\
&= \inv{E^2} \left( \Bp^2 c^2 + m^2 c^4  - c^2 {\Bp^2} \right) \\
&= \frac{m^2 c^4}{E^2} \\
\end{align*}

So, we have the energy half of \ref{eqn:pauli_qm_relativity_intro:p1_4}

\begin{align*}
E^2 &= \frac{m^2 c^4}{1 - \frac{\Bv^2}{c^2} }
\end{align*}

For the momentum we then have
\begin{align*}
\Bp^2 c^2 + m^2 c^4 &= \frac{m^2 c^4}{1 - \frac{\Bv^2}{c^2} }
\end{align*}
\begin{align*}
\Bp^2 
&= \frac{m^2 c^2}{1 - \frac{\Bv^2}{c^2} } - m^2 c^2 \frac{(1 - \frac{\Bv^2}{c^2} )}{1 - \frac{\Bv^2}{c^2} } \\
&= \frac{m^2 \Bv^2}{1 - \frac{\Bv^2}{c^2} } 
\end{align*}

the second half of \ref{eqn:pauli_qm_relativity_intro:p1_4}.

Pretty cool.  Given the energy momentum invariant, $m^2 c^2 = E^2/c^2 - \Bp^2$, and a requirement that the velocity, momentum, Kinetic energy combination is related precisely as in classical mechanics, with $v_k = \PDi{p_k}{E}$, we
recover the relativistic energy momentum four vector.

This is probably not surprising to somebody who knows relativity better than I, but it was
interesting to me to see this worked ``backwards'' this way.

\subsection{Afternote. }

A timely listening to Susskind's classical mechanics lecture 6, shows that 
this surprising method used by Pauli to work backwards from the energy 
is in fact a use of the Hamiltonian formalism to relate energy, velocity
and position.  We see here that one logically just has to pick the ``right''
energy construct, then the familiar relativistic energy and momentum relations
follow directly.  This requires nothing more than using the Hamiltonian
relationships in the same way that we would get the Newtonian equations
of motion from a classical energy relationship.

My failure to study the Hamiltonian formalism now stands out.  I planned to 
get to it eventually in a QM context, but Pauli shows here that an understanding
of that tool set is well justified in a classical mechanics context as well.

%\bibliographystyle{plainnat}
%\bibliography{myrefs}

%\end{document}

%
% Copyright � 2012 Peeter Joot.  All Rights Reserved.
% Licenced as described in the file LICENSE under the root directory of this GIT repository.
%

% 
% 
%\documentclass{article}

%\usepackage{amsmath}
\usepackage{mathpazo}

%
% shorthand for bold symbols, convenient for vectors and matrices
%
\newcommand{\Ba}[0]{\mathbf{a}}
\newcommand{\Bb}[0]{\mathbf{b}}
\newcommand{\Bc}[0]{\mathbf{c}}
\newcommand{\Bd}[0]{\mathbf{d}}
\newcommand{\Be}[0]{\mathbf{e}}
\newcommand{\Bf}[0]{\mathbf{f}}
\newcommand{\Bg}[0]{\mathbf{g}}
\newcommand{\Bh}[0]{\mathbf{h}}
\newcommand{\Bi}[0]{\mathbf{i}}
\newcommand{\Bj}[0]{\mathbf{j}}
\newcommand{\Bk}[0]{\mathbf{k}}
\newcommand{\Bl}[0]{\mathbf{l}}
\newcommand{\Bm}[0]{\mathbf{m}}
\newcommand{\Bn}[0]{\mathbf{n}}
\newcommand{\Bo}[0]{\mathbf{o}}
\newcommand{\Bp}[0]{\mathbf{p}}
\newcommand{\Bq}[0]{\mathbf{q}}
\newcommand{\Br}[0]{\mathbf{r}}
\newcommand{\Bs}[0]{\mathbf{s}}
\newcommand{\Bt}[0]{\mathbf{t}}
\newcommand{\Bu}[0]{\mathbf{u}}
\newcommand{\Bv}[0]{\mathbf{v}}
\newcommand{\Bw}[0]{\mathbf{w}}
\newcommand{\Bx}[0]{\mathbf{x}}
\newcommand{\By}[0]{\mathbf{y}}
\newcommand{\Bz}[0]{\mathbf{z}}
\newcommand{\BA}[0]{\mathbf{A}}
\newcommand{\BB}[0]{\mathbf{B}}
\newcommand{\BC}[0]{\mathbf{C}}
\newcommand{\BD}[0]{\mathbf{D}}
\newcommand{\BE}[0]{\mathbf{E}}
\newcommand{\BF}[0]{\mathbf{F}}
\newcommand{\BG}[0]{\mathbf{G}}
\newcommand{\BH}[0]{\mathbf{H}}
\newcommand{\BI}[0]{\mathbf{I}}
\newcommand{\BJ}[0]{\mathbf{J}}
\newcommand{\BK}[0]{\mathbf{K}}
\newcommand{\BL}[0]{\mathbf{L}}
\newcommand{\BM}[0]{\mathbf{M}}
\newcommand{\BN}[0]{\mathbf{N}}
\newcommand{\BO}[0]{\mathbf{O}}
\newcommand{\BP}[0]{\mathbf{P}}
\newcommand{\BQ}[0]{\mathbf{Q}}
\newcommand{\BR}[0]{\mathbf{R}}
\newcommand{\BS}[0]{\mathbf{S}}
\newcommand{\BT}[0]{\mathbf{T}}
\newcommand{\BU}[0]{\mathbf{U}}
\newcommand{\BV}[0]{\mathbf{V}}
\newcommand{\BW}[0]{\mathbf{W}}
\newcommand{\BX}[0]{\mathbf{X}}
\newcommand{\BY}[0]{\mathbf{Y}}
\newcommand{\BZ}[0]{\mathbf{Z}}

\newcommand{\Bzero}[0]{\mathbf{0}}
\newcommand{\Btheta}[0]{\boldsymbol{\theta}}
\newcommand{\Btau}[0]{\boldsymbol{\tau}}
\newcommand{\Bomega}[0]{\boldsymbol{\omega}}

%
% shorthand for unit vectors
%
\newcommand{\acap}[0]{\hat{\Ba}}
\newcommand{\bcap}[0]{\hat{\Bb}}
\newcommand{\ccap}[0]{\hat{\Bc}}
\newcommand{\dcap}[0]{\hat{\Bd}}
\newcommand{\ecap}[0]{\hat{\Be}}
\newcommand{\fcap}[0]{\hat{\Bf}}
\newcommand{\gcap}[0]{\hat{\Bg}}
\newcommand{\hcap}[0]{\hat{\Bh}}
\newcommand{\icap}[0]{\hat{\Bi}}
\newcommand{\jcap}[0]{\hat{\Bj}}
\newcommand{\kcap}[0]{\hat{\Bk}}
\newcommand{\lcap}[0]{\hat{\Bl}}
\newcommand{\mcap}[0]{\hat{\Bm}}
\newcommand{\ncap}[0]{\hat{\Bn}}
\newcommand{\ocap}[0]{\hat{\Bo}}
\newcommand{\pcap}[0]{\hat{\Bp}}
\newcommand{\qcap}[0]{\hat{\Bq}}
\newcommand{\rcap}[0]{\hat{\Br}}
\newcommand{\scap}[0]{\hat{\Bs}}
\newcommand{\tcap}[0]{\hat{\Bt}}
\newcommand{\ucap}[0]{\hat{\Bu}}
\newcommand{\vcap}[0]{\hat{\Bv}}
\newcommand{\wcap}[0]{\hat{\Bw}}
\newcommand{\xcap}[0]{\hat{\Bx}}
\newcommand{\ycap}[0]{\hat{\By}}
\newcommand{\zcap}[0]{\hat{\Bz}}
\newcommand{\thetacap}[0]{\hat{\Btheta}}

%
% to write R^n and C^n in a distinguishable fashion.  Perhaps change this
% to the double lined characters upon figuring out how to do so.
%
\newcommand{\C}[1]{$\mathbb{C}^{#1}$}
\newcommand{\R}[1]{$\mathbb{R}^{#1}$}

%
% various generally useful helpers
%

% derivative of #1 wrt. #2:
\newcommand{\D}[2] {\frac {d#2} {d#1}}

\newcommand{\inv}[1]{\frac{1}{#1}}
\newcommand{\cross}[0]{\times}

\newcommand{\abs}[1]{\lvert{#1}\rvert}
\newcommand{\norm}[1]{\lVert{#1}\rVert}
\newcommand{\innerprod}[2]{\langle{#1}, {#2}\rangle}
\newcommand{\dotprod}[2]{{#1} \cdot {#2}}
\newcommand{\bdotprod}[2]{\left({#1} \cdot {#2}\right)}
\newcommand{\crossprod}[2]{{#1} \cross {#2}}
\newcommand{\tripleprod}[3]{\dotprod{\left(\crossprod{#1}{#2}\right)}{#3}}

\DeclareMathOperator{\Proj}{Proj}
\DeclareMathOperator{\Span}{span}
\DeclareMathOperator{\Sgn}{sgn}
\DeclareMathOperator{\Area}{Area}
\DeclareMathOperator{\Volume}{Volume}

%
% A few miscellaneous things specific to this document
%
\newcommand{\crossop}[1]{\crossprod{#1}{}}

% R2 vector.
\newcommand{\VectorTwo}[2]{
\begin{bmatrix}
 {#1} \\
 {#2}
\end{bmatrix}
}

\newcommand{\VectorN}[1]{
\begin{bmatrix}
{#1}_1 \\
{#1}_2 \\
\vdots \\
{#1}_N \\
\end{bmatrix}
}

\newcommand{\DETuvij}[4]{
\begin{vmatrix}
 {#1}_{#3} & {#1}_{#4} \\
 {#2}_{#3} & {#2}_{#4}
\end{vmatrix}
}

\newcommand{\DETuvwijk}[6]{
\begin{vmatrix}
 {#1}_{#4} & {#1}_{#5} & {#1}_{#6} \\
 {#2}_{#4} & {#2}_{#5} & {#2}_{#6} \\
 {#3}_{#4} & {#3}_{#5} & {#3}_{#6}
\end{vmatrix}
}

\newcommand{\DETuvwxijkl}[8]{
\begin{vmatrix}
 {#1}_{#5} & {#1}_{#6} & {#1}_{#7} & {#1}_{#8} \\
 {#2}_{#5} & {#2}_{#6} & {#2}_{#7} & {#2}_{#8} \\
 {#3}_{#5} & {#3}_{#6} & {#3}_{#7} & {#3}_{#8} \\
 {#4}_{#5} & {#4}_{#6} & {#4}_{#7} & {#4}_{#8} \\
\end{vmatrix}
}

%\newcommand{\DETuvwxyijklm}[10]{
%\begin{vmatrix}
% {#1}_{#6} & {#1}_{#7} & {#1}_{#8} & {#1}_{#9} & {#1}_{#10} \\
% {#2}_{#6} & {#2}_{#7} & {#2}_{#8} & {#2}_{#9} & {#2}_{#10} \\
% {#3}_{#6} & {#3}_{#7} & {#3}_{#8} & {#3}_{#9} & {#3}_{#10} \\
% {#4}_{#6} & {#4}_{#7} & {#4}_{#8} & {#4}_{#9} & {#4}_{#10} \\
% {#5}_{#6} & {#5}_{#7} & {#5}_{#8} & {#5}_{#9} & {#5}_{#10}
%\end{vmatrix}
%}

% R3 vector.
\newcommand{\VectorThree}[3]{
\begin{bmatrix}
 {#1} \\
 {#2} \\
 {#3}
\end{bmatrix}
}


%%<misc>
%
\newcommand{\Abs}[1]{{\left\lvert{#1}\right\rvert}}
\newcommand{\spacegrad}[0]{\boldsymbol{\nabla}}
\newcommand{\grad}[0]{\nabla}
\newcommand{\LL}[0]{\mathcal{L}}

% == \partial_{#1} {#2}
\newcommand{\PD}[2]{\frac{\partial {#2}}{\partial {#1}}}
% inline variant
\newcommand{\PDi}[2]{{\partial {#2}}/{\partial {#1}}}

\newcommand{\PDD}[3]{\frac{\partial^2 {#3}}{\partial {#1}\partial {#2}}}
%\newcommand{\PDd}[2]{\frac{\partial^2 {#2}}{{\partial{#1}}^2}}
\newcommand{\PDsq}[2]{\frac{\partial^2 {#2}}{(\partial {#1})^2}}

\newcommand{\Partial}[2]{\frac{\partial {#1}}{\partial {#2}}}
\DeclareMathOperator{\RejName}{Rej}
\newcommand{\Rej}[2]{\RejName_{#1}\left( {#2} \right)}
\newcommand{\Rm}[1]{\mathbb{R}^{#1}}
\newcommand{\Cm}[1]{\mathbb{C}^{#1}}
\newcommand{\conj}[0]{{*}}

%</misc>

% <grade selection>
%
\newcommand{\gpgrade}[2] {{\left\langle{{#1}}\right\rangle}_{#2}}

\newcommand{\gpgradezero}[1] {\gpgrade{#1}{}}
%\newcommand{\gpscalargrade}[1] {{\left\langle{{#1}}\right\rangle}}
%\newcommand{\gpgradezero}[1] {\gpgrade{#1}{0}}

%\newcommand{\gpgradeone}[1] {{\left\langle{{#1}}\right\rangle}_{1}}
\newcommand{\gpgradeone}[1] {\gpgrade{#1}{1}}

\newcommand{\gpgradetwo}[1] {\gpgrade{#1}{2}}
\newcommand{\gpgradethree}[1] {\gpgrade{#1}{3}}
\newcommand{\gpgradefour}[1] {\gpgrade{#1}{4}}
%
% </grade selection>



\newcommand{\adot}[0]{{\dot{a}}}
\newcommand{\bdot}[0]{{\dot{b}}}
% taken for centered dot:
%\newcommand{\cdot}[0]{{\dot{c}}}
%\newcommand{\ddot}[0]{{\dot{d}}}
\newcommand{\edot}[0]{{\dot{e}}}
\newcommand{\fdot}[0]{{\dot{f}}}
\newcommand{\gdot}[0]{{\dot{g}}}
\newcommand{\hdot}[0]{{\dot{h}}}
\newcommand{\idot}[0]{{\dot{i}}}
\newcommand{\jdot}[0]{{\dot{j}}}
\newcommand{\kdot}[0]{{\dot{k}}}
\newcommand{\ldot}[0]{{\dot{l}}}
\newcommand{\mdot}[0]{{\dot{m}}}
\newcommand{\ndot}[0]{{\dot{n}}}
%\newcommand{\odot}[0]{{\dot{o}}}
\newcommand{\pdot}[0]{{\dot{p}}}
\newcommand{\qdot}[0]{{\dot{q}}}
\newcommand{\rdot}[0]{{\dot{r}}}
\newcommand{\sdot}[0]{{\dot{s}}}
\newcommand{\tdot}[0]{{\dot{t}}}
\newcommand{\udot}[0]{{\dot{u}}}
\newcommand{\vdot}[0]{{\dot{v}}}
\newcommand{\wdot}[0]{{\dot{w}}}
\newcommand{\xdot}[0]{{\dot{x}}}
\newcommand{\ydot}[0]{{\dot{y}}}
\newcommand{\zdot}[0]{{\dot{z}}}
\newcommand{\addot}[0]{{\ddot{a}}}
\newcommand{\bddot}[0]{{\ddot{b}}}
\newcommand{\cddot}[0]{{\ddot{c}}}
%\newcommand{\dddot}[0]{{\ddot{d}}}
\newcommand{\eddot}[0]{{\ddot{e}}}
\newcommand{\fddot}[0]{{\ddot{f}}}
\newcommand{\gddot}[0]{{\ddot{g}}}
\newcommand{\hddot}[0]{{\ddot{h}}}
\newcommand{\iddot}[0]{{\ddot{i}}}
\newcommand{\jddot}[0]{{\ddot{j}}}
\newcommand{\kddot}[0]{{\ddot{k}}}
\newcommand{\lddot}[0]{{\ddot{l}}}
\newcommand{\mddot}[0]{{\ddot{m}}}
\newcommand{\nddot}[0]{{\ddot{n}}}
\newcommand{\oddot}[0]{{\ddot{o}}}
\newcommand{\pddot}[0]{{\ddot{p}}}
\newcommand{\qddot}[0]{{\ddot{q}}}
\newcommand{\rddot}[0]{{\ddot{r}}}
\newcommand{\sddot}[0]{{\ddot{s}}}
\newcommand{\tddot}[0]{{\ddot{t}}}
\newcommand{\uddot}[0]{{\ddot{u}}}
\newcommand{\vddot}[0]{{\ddot{v}}}
\newcommand{\wddot}[0]{{\ddot{w}}}
\newcommand{\xddot}[0]{{\ddot{x}}}
\newcommand{\yddot}[0]{{\ddot{y}}}
\newcommand{\zddot}[0]{{\ddot{z}}}

%<bold and dot greek symbols>
%

\newcommand{\Deltadot}[0]{{\dot{\Delta}}}
\newcommand{\Gammadot}[0]{{\dot{\Gamma}}}
\newcommand{\Lambdadot}[0]{{\dot{\Lambda}}}
\newcommand{\Omegadot}[0]{{\dot{\Omega}}}
\newcommand{\Phidot}[0]{{\dot{\Phi}}}
\newcommand{\Pidot}[0]{{\dot{\Pi}}}
\newcommand{\Psidot}[0]{{\dot{\Psi}}}
\newcommand{\Sigmadot}[0]{{\dot{\Sigma}}}
\newcommand{\Thetadot}[0]{{\dot{\Theta}}}
\newcommand{\Upsilondot}[0]{{\dot{\Upsilon}}}
\newcommand{\Xidot}[0]{{\dot{\Xi}}}
\newcommand{\alphadot}[0]{{\dot{\alpha}}}
\newcommand{\betadot}[0]{{\dot{\beta}}}
\newcommand{\chidot}[0]{{\dot{\chi}}}
\newcommand{\deltadot}[0]{{\dot{\delta}}}
\newcommand{\epsilondot}[0]{{\dot{\epsilon}}}
\newcommand{\etadot}[0]{{\dot{\eta}}}
\newcommand{\gammadot}[0]{{\dot{\gamma}}}
\newcommand{\kappadot}[0]{{\dot{\kappa}}}
\newcommand{\lambdadot}[0]{{\dot{\lambda}}}
\newcommand{\mudot}[0]{{\dot{\mu}}}
\newcommand{\nudot}[0]{{\dot{\nu}}}
\newcommand{\omegadot}[0]{{\dot{\omega}}}
\newcommand{\phidot}[0]{{\dot{\phi}}}
\newcommand{\pidot}[0]{{\dot{\pi}}}
\newcommand{\psidot}[0]{{\dot{\psi}}}
\newcommand{\rhodot}[0]{{\dot{\rho}}}
\newcommand{\sigmadot}[0]{{\dot{\sigma}}}
\newcommand{\taudot}[0]{{\dot{\tau}}}
\newcommand{\thetadot}[0]{{\dot{\theta}}}
\newcommand{\upsilondot}[0]{{\dot{\upsilon}}}
\newcommand{\varepsilondot}[0]{{\dot{\varepsilon}}}
\newcommand{\varphidot}[0]{{\dot{\varphi}}}
\newcommand{\varpidot}[0]{{\dot{\varpi}}}
\newcommand{\varrhodot}[0]{{\dot{\varrho}}}
\newcommand{\varsigmadot}[0]{{\dot{\varsigma}}}
\newcommand{\varthetadot}[0]{{\dot{\vartheta}}}
\newcommand{\xidot}[0]{{\dot{\xi}}}
\newcommand{\zetadot}[0]{{\dot{\zeta}}}

\newcommand{\Deltaddot}[0]{{\ddot{\Delta}}}
\newcommand{\Gammaddot}[0]{{\ddot{\Gamma}}}
\newcommand{\Lambdaddot}[0]{{\ddot{\Lambda}}}
\newcommand{\Omegaddot}[0]{{\ddot{\Omega}}}
\newcommand{\Phiddot}[0]{{\ddot{\Phi}}}
\newcommand{\Piddot}[0]{{\ddot{\Pi}}}
\newcommand{\Psiddot}[0]{{\ddot{\Psi}}}
\newcommand{\Sigmaddot}[0]{{\ddot{\Sigma}}}
\newcommand{\Thetaddot}[0]{{\ddot{\Theta}}}
\newcommand{\Upsilonddot}[0]{{\ddot{\Upsilon}}}
\newcommand{\Xiddot}[0]{{\ddot{\Xi}}}
\newcommand{\alphaddot}[0]{{\ddot{\alpha}}}
\newcommand{\betaddot}[0]{{\ddot{\beta}}}
\newcommand{\chiddot}[0]{{\ddot{\chi}}}
\newcommand{\deltaddot}[0]{{\ddot{\delta}}}
\newcommand{\epsilonddot}[0]{{\ddot{\epsilon}}}
\newcommand{\etaddot}[0]{{\ddot{\eta}}}
\newcommand{\gammaddot}[0]{{\ddot{\gamma}}}
\newcommand{\kappaddot}[0]{{\ddot{\kappa}}}
\newcommand{\lambdaddot}[0]{{\ddot{\lambda}}}
\newcommand{\muddot}[0]{{\ddot{\mu}}}
\newcommand{\nuddot}[0]{{\ddot{\nu}}}
\newcommand{\omegaddot}[0]{{\ddot{\omega}}}
\newcommand{\phiddot}[0]{{\ddot{\phi}}}
\newcommand{\piddot}[0]{{\ddot{\pi}}}
\newcommand{\psiddot}[0]{{\ddot{\psi}}}
\newcommand{\rhoddot}[0]{{\ddot{\rho}}}
\newcommand{\sigmaddot}[0]{{\ddot{\sigma}}}
\newcommand{\tauddot}[0]{{\ddot{\tau}}}
\newcommand{\thetaddot}[0]{{\ddot{\theta}}}
\newcommand{\upsilonddot}[0]{{\ddot{\upsilon}}}
\newcommand{\varepsilonddot}[0]{{\ddot{\varepsilon}}}
\newcommand{\varphiddot}[0]{{\ddot{\varphi}}}
\newcommand{\varpiddot}[0]{{\ddot{\varpi}}}
\newcommand{\varrhoddot}[0]{{\ddot{\varrho}}}
\newcommand{\varsigmaddot}[0]{{\ddot{\varsigma}}}
\newcommand{\varthetaddot}[0]{{\ddot{\vartheta}}}
\newcommand{\xiddot}[0]{{\ddot{\xi}}}
\newcommand{\zetaddot}[0]{{\ddot{\zeta}}}

\newcommand{\BDelta}[0]{\boldsymbol{\Delta}}
\newcommand{\BGamma}[0]{\boldsymbol{\Gamma}}
\newcommand{\BLambda}[0]{\boldsymbol{\Lambda}}
\newcommand{\BOmega}[0]{\boldsymbol{\Omega}}
\newcommand{\BPhi}[0]{\boldsymbol{\Phi}}
\newcommand{\BPi}[0]{\boldsymbol{\Pi}}
\newcommand{\BPsi}[0]{\boldsymbol{\Psi}}
\newcommand{\BSigma}[0]{\boldsymbol{\Sigma}}
\newcommand{\BTheta}[0]{\boldsymbol{\Theta}}
\newcommand{\BUpsilon}[0]{\boldsymbol{\Upsilon}}
\newcommand{\BXi}[0]{\boldsymbol{\Xi}}
\newcommand{\Balpha}[0]{\boldsymbol{\alpha}}
\newcommand{\Bbeta}[0]{\boldsymbol{\beta}}
\newcommand{\Bchi}[0]{\boldsymbol{\chi}}
\newcommand{\Bdelta}[0]{\boldsymbol{\delta}}
\newcommand{\Bepsilon}[0]{\boldsymbol{\epsilon}}
\newcommand{\Beta}[0]{\boldsymbol{\eta}}
\newcommand{\Bgamma}[0]{\boldsymbol{\gamma}}
\newcommand{\Bkappa}[0]{\boldsymbol{\kappa}}
\newcommand{\Blambda}[0]{\boldsymbol{\lambda}}
\newcommand{\Bmu}[0]{\boldsymbol{\mu}}
\newcommand{\Bnu}[0]{\boldsymbol{\nu}}
%\newcommand{\Bomega}[0]{\boldsymbol{\omega}}
\newcommand{\Bphi}[0]{\boldsymbol{\phi}}
\newcommand{\Bpi}[0]{\boldsymbol{\pi}}
\newcommand{\Bpsi}[0]{\boldsymbol{\psi}}
\newcommand{\Brho}[0]{\boldsymbol{\rho}}
\newcommand{\Bsigma}[0]{\boldsymbol{\sigma}}
%\newcommand{\Btau}[0]{\boldsymbol{\tau}}
%\newcommand{\Btheta}[0]{\boldsymbol{\theta}}
\newcommand{\Bupsilon}[0]{\boldsymbol{\upsilon}}
\newcommand{\Bvarepsilon}[0]{\boldsymbol{\varepsilon}}
\newcommand{\Bvarphi}[0]{\boldsymbol{\varphi}}
\newcommand{\Bvarpi}[0]{\boldsymbol{\varpi}}
\newcommand{\Bvarrho}[0]{\boldsymbol{\varrho}}
\newcommand{\Bvarsigma}[0]{\boldsymbol{\varsigma}}
\newcommand{\Bvartheta}[0]{\boldsymbol{\vartheta}}
\newcommand{\Bxi}[0]{\boldsymbol{\xi}}
\newcommand{\Bzeta}[0]{\boldsymbol{\zeta}}
%
%</bold and dot greek symbols>
%<infrequent>
%
%\newcommand{\AreaOp}[1]{\AName_{#1}}
%\newcommand{\Babs}[0]{\abs{\BB}}
%\newcommand{\Bcap}[0]{\hat{\BB}}
%\newcommand{\BrPrimeRej}[0]{\rcap(\rcap \wedge \Br')}
%\newcommand{\CA}[0]{\mathcal{A}}
%\newcommand{\Cos}[1]{\cos{\left({#1}\right)}}
%\newcommand{\Det}[1] {\abs{#1}}
%\newcommand{\Dsq}[2] {\frac {\partial^2 {#1}} {\partial {#2}^2}}
%\newcommand{\Exp}[1]{\exp{\left({#1}\right)}}
%\newcommand{\Norm}[1]{\left\lVert{#1}\right\rVert}
%\newcommand{\Sin}[1]{\sin{\left({#1}\right)}}
%\newcommand{\T}[0]{\text{T}}
%\newcommand{\VolumeOp}[1]{\VName_{#1}}
%\newcommand{\agrad}[0]{\Ba \cdot \nabla}
%\newcommand{\alphacap}[0]{\hat{\boldsymbol{\alpha}}}
%\newcommand{\Fcap}[0]{\hat{\BF}}
%\newcommand{\bithree}[0]{{\Bi}_3}
%\newcommand{\bxa}[0]{\Bx\Ba}
%\newcommand{\coordvec}[2]{
%\newcommand{\costheta}[0]{\acap \cdot \xcap}
%\newcommand{\ddt}[1]{\ddot{#1}}
%\newcommand{\ddu}[1] {\frac {d{#1}} {du}}
%\newcommand{\dsqxj}[2] {\frac {\partial^2 {#1}} {\partial {x_{#2}}^2}}
%\newcommand{\dtheta}[1]{\frac{d {#1}}{d \theta}}
%\newcommand{\dt}[1]{\dot{#1}}
%\newcommand{\dt}[1]{\frac{d {#1}}{dt}}
%\newcommand{\dxj}[2] {\frac {\partial {#1}} {\partial {x_{#2}}}}
%\newcommand{\halfPhi}[0]{\frac{\phi}{2}}
%\newcommand{\half}[0]{\inv{2}}
%\newcommand{\inv}[1]{\frac{1}{#1}}
%\newcommand{\laplacian}[0]{\nabla^2}
%\newcommand{\matrixoftx}[3]{
%\newcommand{\nrrp}[0]{\norm{\rcap \wedge \Br'}}
%\newcommand{\oiint}{\bigcirc \hspace{-1.4em} \int \hspace{-.8em} \int}
%\newcommand{\transpose}[1]{{#1}^{\text{T}}}
%\newcommand{\transpose}[1]{{{#1}^{\TextTranspose}}}
%\newcommand{\transpose}[1]{{{#1}^{\text{T}}}}
%\newcommand{\barA}[0]{\bar{A}}
%\newcommand{\qbar}[0]{\bar{q}}
%\newcommand{\qdotbar}[0]{\dot{\bar{q}}}
%
%</infrequent>





%\usepackage[bookmarks=true]{hyperref}

%\usepackage{color,cite,graphicx}
   % use colour in the document, put your citations as [1-4]
   % rather than [1,2,3,4] (it looks nicer, and the extended LaTeX2e
   % graphics package. 
%\usepackage{latexsym,amssymb,epsf} % do not remember if these are
   % needed, but their inclusion can not do any damage


\chapter{Some notes on Pauli Relativity Velocity addition}
\label{chap:PJpauliVelocityAddition}
%\author{Peeter Joot \quad peeter.joot@gmail.com}
\date{ Dec 25, 2008.  velocityAddition.tex }

%\begin{document}
%\maketitle{}
%
%\tableofcontents

\section{Motivation}

Fill out some details from part 1.6, velocity addition of \citep{pauli1981tr}.

\section{}

Given a path \(x^i(t')\) i n the moving (primed) frame \(S'\), the aim is to 
express the observed velocities for this path from the rest frame \(S\).

From the Lorentz transformation of the coordinates we have (working with \(c=1\)) for a point in the moving frame

\begin{equation}\label{eqn:velocityAddition:20}
\begin{aligned}
x' &= \gamma ( x - vt) \\
t' &= \gamma ( t - vx)
\end{aligned}
\end{equation}

Or reversed (inverting velocities)

\begin{equation}\label{eqn:velocityAddition:40}
\begin{aligned}
x &= \gamma ( x' + vt') \\
t &= \gamma ( t' + vx')
\end{aligned}
\end{equation}

taking differentials we have

\begin{equation}\label{eqn:velocityAddition:60}
\begin{aligned}
dx &= \gamma ( dx' + v dt') \\
dy &= dy' \\
dz &= dz' \\
dt &= \gamma ( dt' + v dx')
\end{aligned}
\end{equation}

Dividing by \(dt'\), this is

\begin{equation}\label{eqn:velocityAddition:80}
\begin{aligned}
\frac{dx}{dt} &= \frac{ dx' + v dt' }{ dt' + v dx'} \\
\frac{dy}{dt} &= \frac{dy'}{\gamma (dt' + v dx')} \\
\frac{dz}{dt} &= \frac{dz'}{\gamma (dt' + v dx')} \\
\end{aligned}
\end{equation}

FIXME: do not like this dividing by differentials.  Try to re-express this 
using the chain rule.

Or in terms of velocity coordinates we have Pauli's equation 10.

\begin{equation}\label{eqn:velocityAddition:100}
\begin{aligned}
u_x &= \frac{ {u_x}' + v  }{ 1 + v {u_x}'} \\
u_y &= \frac{{u_y}'}{\gamma (1 + v {u_x}')} \\
u_z &= \frac{{u_z}'}{\gamma (1 + v {u_x}')} \\
\end{aligned}
\end{equation}

Next he writes \(u^2 = \sum_i {u_i}^2\), in terms of the primed velocities

\begin{equation}\label{eqn:velocityAddition:120}
\begin{aligned}
u^2 &= {u_x}^2 +{u_y}^2 +{u_z}^2  \\
&=
\inv{( 1 + v {u_x}' )^2 } \left(
( {u_x}' + v  )^2
+(1-v^2){{u_y}'}^2
+(1-v^2){{u_z}'}^2
\right) \\
&=
\inv{( 1 + v {u_x}' )^2 } \left(
{{u_x}'}^2
+ 2 {u_x}' v
+ v^2
+(1-v^2){{u_y}'}^2
+(1-v^2){{u_z}'}^2
+v^2 {{u_x}'}^2
-v^2 {{u_x}'}^2
\right) \\
&=
\inv{( 1 + v {u_x}' )^2 } \left(
{{u}'}^2
+ 2 {u_x}' v
+ v^2
+v^2 {{u_x}'}^2
-v^2 {{u}'}^2
\right) \\
\end{aligned}
\end{equation}

This leaves the squared velocity of the path as viewed from the rest frame as
\begin{equation}\label{eqn:velocity_addition:Usquare}
\begin{aligned}
u^2 &=
\inv{( 1 + v {u_x}' )^2 } \left(
{{u}'}^2(1-v^2)
+ 2 {u_x}' v
+v^2 (1 + {{u_x}'}^2)
\right)
\end{aligned}
\end{equation}

Now direction cosines for the velocity direction vector are introduced

\begin{equation}\label{eqn:velocityAddition:140}
\begin{aligned}
\ucap' &= (u_x', u_y', u_z')/u = (\cos\alpha', \cos\sigma', \cos\delta') \\
\ucap &= (u_x, u_y, u_z)/u = (\cos\alpha, \cos\sigma, \cos\delta)
\end{aligned}
\end{equation}

and in terms of the direction cosines one has equation 11:

\begin{equation}\label{eqn:velocityAddition:160}
\begin{aligned}
u^2 &= 
\inv{( 1 + v {u_x}' )^2 } \left(
{{u}'}^2
+v^2 
+ 2 u' \cos\alpha' v
+v^2 {u'}^2 ({\cos\alpha'}^2 - 1 )
\right) \\
&= \inv{( 1 + v u' \cos\alpha' )^2 } \left(
{{u}'}^2
+v^2 
+ 2 u' \cos\alpha' v
-v^2 {u'}^2 \sin^2\alpha'
\right) \\
\end{aligned}
\end{equation}

Pauli's equation 11a is \(1-u^2\), which is then factored into a tidy form
\begin{equation}\label{eqn:velocityAddition:180}
\begin{aligned}
1- u^2
&= \inv{( 1 + v u' \cos\alpha' )^2 } \left(
1 
+ 2 v u' \cos\alpha' 
+v^2 {u'}^2 \cos^2\alpha' 
-{{u}'}^2
-v^2 
- 2 u' v \cos\alpha' 
+v^2 {u'}^2 \sin^2\alpha'
\right) \\
&= \inv{( 1 + v u' \cos\alpha' )^2 } \left(
1 
-{{u}'}^2
-v^2 
+v^2 {u'}^2 
\right) \\
&= \frac{(1 -v^2 ) (1 -{{u}'}^2)}
{( 1 + v u' \cos\alpha' )^2 }
\end{aligned}
\end{equation}

This provides the gamma factor for the effective velocity as observed from the rest frame

\begin{equation}\label{eqn:velocityAddition:200}
\begin{aligned}
\inv{\sqrt{1-u^2}} 
&= \frac{ 1 + v u' \cos\alpha' }{\sqrt{1 -v^2 }\sqrt{1 -{u'}^2}}
\end{aligned}
\end{equation}

With the factors of c's retained, and for the special case where the velocity is colinear with the frame motion, this is part of the velocity addition equation found in many intro relativity treatments.  The generalization required is that instead of the second velocity itself we have the projection of that velocity
in the direction of the frame motion.
For the special
case of when the velocity \(u'\) is directed with the path of the moving frame,
the cosine will be unity, and the projection of that velocity in the frame motion direction is exactly the velocity to be compounded:

\begin{equation}\label{eqn:velocityAddition:220}
\begin{aligned}
\inv{\sqrt{1-u^2}} 
&= \frac{ 1 + v u' }{\sqrt{1 -v^2 }\sqrt{1 -{u'}^2}}
\end{aligned}
\end{equation}

We have another special case, considering perpendicular motion, for which we have \(\cos\alpha' = 0\), and thus

\begin{equation}\label{eqn:velocityAddition:240}
\begin{aligned}
\inv{\sqrt{1-u^2}} 
&= \frac{ 1 }{\sqrt{1 -v^2 }\sqrt{1 -{u'}^2}}
\end{aligned}
\end{equation}

Next he calculates \(tan \alpha\).  That is

\begin{equation}\label{eqn:velocityAddition:260}
\begin{aligned}
\tan\alpha 
&= \frac{\sin\alpha}{\cos\alpha} \\
&= \frac{\sqrt{1-\cos^2\alpha}}{\cos\alpha} \\
&= \frac{\sqrt{1-(u_x/u)^2}}{u_x/u} \\
&= \frac{\sqrt{u^2- {u_x}^2}}{u_x} \\
\end{aligned}
\end{equation}

From \eqnref{eqn:velocity_addition:Usquare} we have
\begin{equation}\label{eqn:velocityAddition:280}
\begin{aligned}
u^2 - {u_x}^2
&= \inv{( 1 + v {u_x}' )^2 } \left(
{{u}'}^2(1-v^2)
+ 2 {u_x}' v
+v^2 (1 + {{u_x}'}^2)
-(u_x' + v)^2
\right) \\
&= \inv{( 1 + v {u_x}' )^2 } \left(
{{u}'}^2(1-v^2)
%+ 2 {u_x}' v
+v^2 (1 + {{u_x}'}^2)
- {u_x'}^2 
- v^2 
%- 2 u_x' v
\right) \\
&= \inv{( 1 + v {u_x}' )^2 } \left(
({{u}'}^2 -{u_x'}^2) (1-v^2)
\right) \\
&= \inv{( 1 + v {u_x}' )^2 } \left(
{{u}'}^2(1 -\cos^2\alpha') (1-v^2)
\right) \\
&= \inv{( 1 + v {u_x}' )^2 } \left(
{{u}'}^2 \sin^2\alpha' (1-v^2)
\right) \\
\end{aligned}
\end{equation}

So, the tangent is
\begin{equation}\label{eqn:velocityAddition:300}
\begin{aligned}
\tan\alpha 
&= \pm \frac{ u' \sin\alpha' \sqrt{1-v^2}}{ {u_x}' + v  } \\
&= \pm \frac{ u' \sin\alpha' \sqrt{1-v^2}}{ u'\cos\alpha' + v  } \\
\end{aligned}
\end{equation}

which except for the \(\pm 1\) factor is Pauli's equation twelve.

Note that in this form we see some of the relative vector structure

\begin{equation}\label{eqn:velocityAddition:320}
\begin{aligned}
\frac{v \wedge \gamma_0}{v \cdot \gamma_0}
\end{aligned}
\end{equation}

of the STA four vector formulation (ie: sine and cosine mapping to rejection and projection terms respectively onto the timelike direction).

\subsection{Perpendicular direction cosines}

What are the equivalent relations for the \(y\) and \(z\) direction cosines for the velocity between the two frames?  For the \(y\) direction, our \(\sin^2\sigma\) is

\begin{equation}\label{eqn:velocityAddition:340}
\begin{aligned}
u^2 - u_y^2
&=
\inv{( 1 + v {u_x}' )^2 } \left(
{{u}'}^2(1-v^2)
+ 2 {u_x}' v
+v^2 (1 + {{u_x}'}^2)
-{{u_y}'}^2/\gamma^2
\right) \\
&=
\inv{( 1 + v {u_x}' )^2 } \left(
({{u}'}^2 - {{u_y}'}^2 -{{u_x}'}^2)(1-v^2)
+ 2 {u_x}' v
+v^2 
+ {{u_x}'}^2
\right) \\
&=
\inv{( 1 + v {u_x}' )^2 } \left(
({{u}'}^2 - {{u_y}'}^2 -{{u_x}'}^2)(1-v^2)
+ ({u_x}' + v)^2
\right) \\
&=
\inv{( 1 + v {u_x}' )^2 } \left(
{{u_z}'}^2(1-v^2)
+ ({u_x}' + v)^2
\right) \\
\end{aligned}
\end{equation}

This leaves us with a tangent of 

\begin{equation}\label{eqn:velocityAddition:360}
\begin{aligned}
\tan^2\sigma
&=
\inv{( 1 + v {u_x}' )^2 } \left(
{{u_z}'}^2(1-v^2)
+ ({u_x}' + v)^2
\right)
/\left(\frac{{u_y}'}{\gamma (1 + v {u_x}')}\right)^2 \\
&=
\frac{ \left(
{{u_z}'}^2(1-v^2)
+ ({u_x}' + v)^2
\right)
}{(1-v^2){{u_y}'}^2}
\end{aligned}
\end{equation}

which leaves
\begin{equation}\label{eqn:velocityAddition:380}
\begin{aligned}
\tan\sigma 
&= \pm \frac{ \sqrt{ \cos^2\delta' + \inv{1-v^2}(\cos\alpha' + v)^2 } }{\cos\sigma'}
\end{aligned}
\end{equation}

This is now enough to completely assemble the 
velocity vector as observed from the rest frame

\begin{equation}\label{eqn:velocityAddition:400}
\begin{aligned}
\Bu &=
\sqrt{1 -\frac{(1 -v^2 ) (1 -{{u}'}^2)}{( 1 + v u' \cos\alpha' )^2 }}
\begin{bmatrix}
\cos\left(\tan^{-1}\left(\frac{ u' \sin\alpha' \sqrt{1-v^2}}{ u'\cos\alpha' + v  }\right)\right) \\
\cos\left(\tan^{-1}\left(\frac{ \sqrt{ \cos^2\delta' + \inv{1-v^2}(\cos\alpha' + v)^2 } }{\cos\sigma'}\right)\right) \\
\cos\left(\tan^{-1}\left(\frac{ \sqrt{ \cos^2\sigma' + \inv{1-v^2}(\cos\alpha' + v)^2 } }{\cos\delta'}\right)\right)
\end{bmatrix}
\end{aligned}
\end{equation}

Wow.  What a mess (assuming I even got the algebra right)!

%\bibliographystyle{plainnat}
%\bibliography{myrefs}

%\end{document}

\documentclass{article}      % Specifies the document class

\usepackage{amsmath}
\usepackage{mathpazo}

%
% shorthand for bold symbols, convenient for vectors and matrices
%
\newcommand{\Ba}[0]{\mathbf{a}}
\newcommand{\Bb}[0]{\mathbf{b}}
\newcommand{\Bc}[0]{\mathbf{c}}
\newcommand{\Bd}[0]{\mathbf{d}}
\newcommand{\Be}[0]{\mathbf{e}}
\newcommand{\Bf}[0]{\mathbf{f}}
\newcommand{\Bg}[0]{\mathbf{g}}
\newcommand{\Bh}[0]{\mathbf{h}}
\newcommand{\Bi}[0]{\mathbf{i}}
\newcommand{\Bj}[0]{\mathbf{j}}
\newcommand{\Bk}[0]{\mathbf{k}}
\newcommand{\Bl}[0]{\mathbf{l}}
\newcommand{\Bm}[0]{\mathbf{m}}
\newcommand{\Bn}[0]{\mathbf{n}}
\newcommand{\Bo}[0]{\mathbf{o}}
\newcommand{\Bp}[0]{\mathbf{p}}
\newcommand{\Bq}[0]{\mathbf{q}}
\newcommand{\Br}[0]{\mathbf{r}}
\newcommand{\Bs}[0]{\mathbf{s}}
\newcommand{\Bt}[0]{\mathbf{t}}
\newcommand{\Bu}[0]{\mathbf{u}}
\newcommand{\Bv}[0]{\mathbf{v}}
\newcommand{\Bw}[0]{\mathbf{w}}
\newcommand{\Bx}[0]{\mathbf{x}}
\newcommand{\By}[0]{\mathbf{y}}
\newcommand{\Bz}[0]{\mathbf{z}}
\newcommand{\BA}[0]{\mathbf{A}}
\newcommand{\BB}[0]{\mathbf{B}}
\newcommand{\BC}[0]{\mathbf{C}}
\newcommand{\BD}[0]{\mathbf{D}}
\newcommand{\BE}[0]{\mathbf{E}}
\newcommand{\BF}[0]{\mathbf{F}}
\newcommand{\BG}[0]{\mathbf{G}}
\newcommand{\BH}[0]{\mathbf{H}}
\newcommand{\BI}[0]{\mathbf{I}}
\newcommand{\BJ}[0]{\mathbf{J}}
\newcommand{\BK}[0]{\mathbf{K}}
\newcommand{\BL}[0]{\mathbf{L}}
\newcommand{\BM}[0]{\mathbf{M}}
\newcommand{\BN}[0]{\mathbf{N}}
\newcommand{\BO}[0]{\mathbf{O}}
\newcommand{\BP}[0]{\mathbf{P}}
\newcommand{\BQ}[0]{\mathbf{Q}}
\newcommand{\BR}[0]{\mathbf{R}}
\newcommand{\BS}[0]{\mathbf{S}}
\newcommand{\BT}[0]{\mathbf{T}}
\newcommand{\BU}[0]{\mathbf{U}}
\newcommand{\BV}[0]{\mathbf{V}}
\newcommand{\BW}[0]{\mathbf{W}}
\newcommand{\BX}[0]{\mathbf{X}}
\newcommand{\BY}[0]{\mathbf{Y}}
\newcommand{\BZ}[0]{\mathbf{Z}}

\newcommand{\Bzero}[0]{\mathbf{0}}
\newcommand{\Btheta}[0]{\boldsymbol{\theta}}
\newcommand{\Btau}[0]{\boldsymbol{\tau}}
\newcommand{\Bomega}[0]{\boldsymbol{\omega}}

%
% shorthand for unit vectors
%
\newcommand{\acap}[0]{\hat{\Ba}}
\newcommand{\bcap}[0]{\hat{\Bb}}
\newcommand{\ccap}[0]{\hat{\Bc}}
\newcommand{\dcap}[0]{\hat{\Bd}}
\newcommand{\ecap}[0]{\hat{\Be}}
\newcommand{\fcap}[0]{\hat{\Bf}}
\newcommand{\gcap}[0]{\hat{\Bg}}
\newcommand{\hcap}[0]{\hat{\Bh}}
\newcommand{\icap}[0]{\hat{\Bi}}
\newcommand{\jcap}[0]{\hat{\Bj}}
\newcommand{\kcap}[0]{\hat{\Bk}}
\newcommand{\lcap}[0]{\hat{\Bl}}
\newcommand{\mcap}[0]{\hat{\Bm}}
\newcommand{\ncap}[0]{\hat{\Bn}}
\newcommand{\ocap}[0]{\hat{\Bo}}
\newcommand{\pcap}[0]{\hat{\Bp}}
\newcommand{\qcap}[0]{\hat{\Bq}}
\newcommand{\rcap}[0]{\hat{\Br}}
\newcommand{\scap}[0]{\hat{\Bs}}
\newcommand{\tcap}[0]{\hat{\Bt}}
\newcommand{\ucap}[0]{\hat{\Bu}}
\newcommand{\vcap}[0]{\hat{\Bv}}
\newcommand{\wcap}[0]{\hat{\Bw}}
\newcommand{\xcap}[0]{\hat{\Bx}}
\newcommand{\ycap}[0]{\hat{\By}}
\newcommand{\zcap}[0]{\hat{\Bz}}
\newcommand{\thetacap}[0]{\hat{\Btheta}}

%
% to write R^n and C^n in a distinguishable fashion.  Perhaps change this
% to the double lined characters upon figuring out how to do so.
%
\newcommand{\C}[1]{$\mathbb{C}^{#1}$}
\newcommand{\R}[1]{$\mathbb{R}^{#1}$}

%
% various generally useful helpers
%

% derivative of #1 wrt. #2:
\newcommand{\D}[2] {\frac {d#2} {d#1}}

\newcommand{\inv}[1]{\frac{1}{#1}}
\newcommand{\cross}[0]{\times}

\newcommand{\abs}[1]{\lvert{#1}\rvert}
\newcommand{\norm}[1]{\lVert{#1}\rVert}
\newcommand{\innerprod}[2]{\langle{#1}, {#2}\rangle}
\newcommand{\dotprod}[2]{{#1} \cdot {#2}}
\newcommand{\bdotprod}[2]{\left({#1} \cdot {#2}\right)}
\newcommand{\crossprod}[2]{{#1} \cross {#2}}
\newcommand{\tripleprod}[3]{\dotprod{\left(\crossprod{#1}{#2}\right)}{#3}}

\DeclareMathOperator{\Proj}{Proj}
\DeclareMathOperator{\Span}{span}
\DeclareMathOperator{\Sgn}{sgn}
\DeclareMathOperator{\Area}{Area}
\DeclareMathOperator{\Volume}{Volume}

%
% A few miscellaneous things specific to this document
%
\newcommand{\crossop}[1]{\crossprod{#1}{}}

% R2 vector.
\newcommand{\VectorTwo}[2]{
\begin{bmatrix}
 {#1} \\
 {#2}
\end{bmatrix}
}

\newcommand{\VectorN}[1]{
\begin{bmatrix}
{#1}_1 \\
{#1}_2 \\
\vdots \\
{#1}_N \\
\end{bmatrix}
}

\newcommand{\DETuvij}[4]{
\begin{vmatrix}
 {#1}_{#3} & {#1}_{#4} \\
 {#2}_{#3} & {#2}_{#4}
\end{vmatrix}
}

\newcommand{\DETuvwijk}[6]{
\begin{vmatrix}
 {#1}_{#4} & {#1}_{#5} & {#1}_{#6} \\
 {#2}_{#4} & {#2}_{#5} & {#2}_{#6} \\
 {#3}_{#4} & {#3}_{#5} & {#3}_{#6}
\end{vmatrix}
}

\newcommand{\DETuvwxijkl}[8]{
\begin{vmatrix}
 {#1}_{#5} & {#1}_{#6} & {#1}_{#7} & {#1}_{#8} \\
 {#2}_{#5} & {#2}_{#6} & {#2}_{#7} & {#2}_{#8} \\
 {#3}_{#5} & {#3}_{#6} & {#3}_{#7} & {#3}_{#8} \\
 {#4}_{#5} & {#4}_{#6} & {#4}_{#7} & {#4}_{#8} \\
\end{vmatrix}
}

%\newcommand{\DETuvwxyijklm}[10]{
%\begin{vmatrix}
% {#1}_{#6} & {#1}_{#7} & {#1}_{#8} & {#1}_{#9} & {#1}_{#10} \\
% {#2}_{#6} & {#2}_{#7} & {#2}_{#8} & {#2}_{#9} & {#2}_{#10} \\
% {#3}_{#6} & {#3}_{#7} & {#3}_{#8} & {#3}_{#9} & {#3}_{#10} \\
% {#4}_{#6} & {#4}_{#7} & {#4}_{#8} & {#4}_{#9} & {#4}_{#10} \\
% {#5}_{#6} & {#5}_{#7} & {#5}_{#8} & {#5}_{#9} & {#5}_{#10}
%\end{vmatrix}
%}

% R3 vector.
\newcommand{\VectorThree}[3]{
\begin{bmatrix}
 {#1} \\
 {#2} \\
 {#3}
\end{bmatrix}
}



%
% The real thing:
%

                             % The preamble begins here.
\title{} % Declares the document's title.
\author{Peeter Joot}         % Declares the author's name.
%\date{}        % Deleting this command produces today's date.

\begin{document}             % End of preamble and beginning of text.

\maketitle{}

\section{}

I was summarizing for myself the various four-vectors of mechanics:

\begin{align*}
x &= ct + \mathbf{x} \\
V &= \frac{dx}{d\tau} = \gamma(c + \mathbf{v}) \\
P &= m V = E/c + \gamma\mathbf{p} \\
f &= m\frac{d^2 x}{d\tau^2} = m\frac{d V}{d\tau} \\
\end{align*}

where:

\begin{align*}
\gamma^{-2} &= 1 - {\lvert \mathbf{v}/c \rvert}^2 \\
d\tau &= {\left(\frac{dx}{d\lambda} \cdot \frac{dx}{d\lambda}\right)}^{1/2} d\lambda \\
x \cdot x = {\lvert x \rvert}^2 &= c^2t^2 - {\lvert \mathbf{x} \rvert}^2 \\
E &= \int f \cdot (c d\tau) \\
\mathbf{v} &= \frac{d\mathbf{x}}{dt} \\
\mathbf{p} &= m\mathbf{v} \\
\end{align*}

Invarients for the first three four vectors are:

\begin{align*}
{\lvert x \rvert}^2 &= c^2 t^2 - {\lvert \mathbf{x} \rvert}^2 = c^2 \tau^2 \\
{\lvert V \rvert}^2 &= \gamma^2 (c^2 - {\lvert \mathbf{v} \rvert}^2) = c^2 \\
{\lvert P \rvert}^2 &= m^2 {\lvert V \rvert}^2 = m^2 c^2 \\
\end{align*}

Is the minkowski norm of the four vector force:

\[
f = m\frac{d^2 x}{d\tau^2} 
\]

also an invarient?  I think it has to be.  Assuming that is the case, what would the value (and significance if any) of this be?

\end{document}               % End of document.

%\documentclass{article}

%\usepackage{amsmath}
\usepackage{mathpazo}

%
% shorthand for bold symbols, convenient for vectors and matrices
%
\newcommand{\Ba}[0]{\mathbf{a}}
\newcommand{\Bb}[0]{\mathbf{b}}
\newcommand{\Bc}[0]{\mathbf{c}}
\newcommand{\Bd}[0]{\mathbf{d}}
\newcommand{\Be}[0]{\mathbf{e}}
\newcommand{\Bf}[0]{\mathbf{f}}
\newcommand{\Bg}[0]{\mathbf{g}}
\newcommand{\Bh}[0]{\mathbf{h}}
\newcommand{\Bi}[0]{\mathbf{i}}
\newcommand{\Bj}[0]{\mathbf{j}}
\newcommand{\Bk}[0]{\mathbf{k}}
\newcommand{\Bl}[0]{\mathbf{l}}
\newcommand{\Bm}[0]{\mathbf{m}}
\newcommand{\Bn}[0]{\mathbf{n}}
\newcommand{\Bo}[0]{\mathbf{o}}
\newcommand{\Bp}[0]{\mathbf{p}}
\newcommand{\Bq}[0]{\mathbf{q}}
\newcommand{\Br}[0]{\mathbf{r}}
\newcommand{\Bs}[0]{\mathbf{s}}
\newcommand{\Bt}[0]{\mathbf{t}}
\newcommand{\Bu}[0]{\mathbf{u}}
\newcommand{\Bv}[0]{\mathbf{v}}
\newcommand{\Bw}[0]{\mathbf{w}}
\newcommand{\Bx}[0]{\mathbf{x}}
\newcommand{\By}[0]{\mathbf{y}}
\newcommand{\Bz}[0]{\mathbf{z}}
\newcommand{\BA}[0]{\mathbf{A}}
\newcommand{\BB}[0]{\mathbf{B}}
\newcommand{\BC}[0]{\mathbf{C}}
\newcommand{\BD}[0]{\mathbf{D}}
\newcommand{\BE}[0]{\mathbf{E}}
\newcommand{\BF}[0]{\mathbf{F}}
\newcommand{\BG}[0]{\mathbf{G}}
\newcommand{\BH}[0]{\mathbf{H}}
\newcommand{\BI}[0]{\mathbf{I}}
\newcommand{\BJ}[0]{\mathbf{J}}
\newcommand{\BK}[0]{\mathbf{K}}
\newcommand{\BL}[0]{\mathbf{L}}
\newcommand{\BM}[0]{\mathbf{M}}
\newcommand{\BN}[0]{\mathbf{N}}
\newcommand{\BO}[0]{\mathbf{O}}
\newcommand{\BP}[0]{\mathbf{P}}
\newcommand{\BQ}[0]{\mathbf{Q}}
\newcommand{\BR}[0]{\mathbf{R}}
\newcommand{\BS}[0]{\mathbf{S}}
\newcommand{\BT}[0]{\mathbf{T}}
\newcommand{\BU}[0]{\mathbf{U}}
\newcommand{\BV}[0]{\mathbf{V}}
\newcommand{\BW}[0]{\mathbf{W}}
\newcommand{\BX}[0]{\mathbf{X}}
\newcommand{\BY}[0]{\mathbf{Y}}
\newcommand{\BZ}[0]{\mathbf{Z}}

\newcommand{\Bzero}[0]{\mathbf{0}}
\newcommand{\Btheta}[0]{\boldsymbol{\theta}}
\newcommand{\Btau}[0]{\boldsymbol{\tau}}
\newcommand{\Bomega}[0]{\boldsymbol{\omega}}

%
% shorthand for unit vectors
%
\newcommand{\acap}[0]{\hat{\Ba}}
\newcommand{\bcap}[0]{\hat{\Bb}}
\newcommand{\ccap}[0]{\hat{\Bc}}
\newcommand{\dcap}[0]{\hat{\Bd}}
\newcommand{\ecap}[0]{\hat{\Be}}
\newcommand{\fcap}[0]{\hat{\Bf}}
\newcommand{\gcap}[0]{\hat{\Bg}}
\newcommand{\hcap}[0]{\hat{\Bh}}
\newcommand{\icap}[0]{\hat{\Bi}}
\newcommand{\jcap}[0]{\hat{\Bj}}
\newcommand{\kcap}[0]{\hat{\Bk}}
\newcommand{\lcap}[0]{\hat{\Bl}}
\newcommand{\mcap}[0]{\hat{\Bm}}
\newcommand{\ncap}[0]{\hat{\Bn}}
\newcommand{\ocap}[0]{\hat{\Bo}}
\newcommand{\pcap}[0]{\hat{\Bp}}
\newcommand{\qcap}[0]{\hat{\Bq}}
\newcommand{\rcap}[0]{\hat{\Br}}
\newcommand{\scap}[0]{\hat{\Bs}}
\newcommand{\tcap}[0]{\hat{\Bt}}
\newcommand{\ucap}[0]{\hat{\Bu}}
\newcommand{\vcap}[0]{\hat{\Bv}}
\newcommand{\wcap}[0]{\hat{\Bw}}
\newcommand{\xcap}[0]{\hat{\Bx}}
\newcommand{\ycap}[0]{\hat{\By}}
\newcommand{\zcap}[0]{\hat{\Bz}}
\newcommand{\thetacap}[0]{\hat{\Btheta}}

%
% to write R^n and C^n in a distinguishable fashion.  Perhaps change this
% to the double lined characters upon figuring out how to do so.
%
\newcommand{\C}[1]{$\mathbb{C}^{#1}$}
\newcommand{\R}[1]{$\mathbb{R}^{#1}$}

%
% various generally useful helpers
%

% derivative of #1 wrt. #2:
\newcommand{\D}[2] {\frac {d#2} {d#1}}

\newcommand{\inv}[1]{\frac{1}{#1}}
\newcommand{\cross}[0]{\times}

\newcommand{\abs}[1]{\lvert{#1}\rvert}
\newcommand{\norm}[1]{\lVert{#1}\rVert}
\newcommand{\innerprod}[2]{\langle{#1}, {#2}\rangle}
\newcommand{\dotprod}[2]{{#1} \cdot {#2}}
\newcommand{\bdotprod}[2]{\left({#1} \cdot {#2}\right)}
\newcommand{\crossprod}[2]{{#1} \cross {#2}}
\newcommand{\tripleprod}[3]{\dotprod{\left(\crossprod{#1}{#2}\right)}{#3}}

\DeclareMathOperator{\Proj}{Proj}
\DeclareMathOperator{\Span}{span}
\DeclareMathOperator{\Sgn}{sgn}
\DeclareMathOperator{\Area}{Area}
\DeclareMathOperator{\Volume}{Volume}

%
% A few miscellaneous things specific to this document
%
\newcommand{\crossop}[1]{\crossprod{#1}{}}

% R2 vector.
\newcommand{\VectorTwo}[2]{
\begin{bmatrix}
 {#1} \\
 {#2}
\end{bmatrix}
}

\newcommand{\VectorN}[1]{
\begin{bmatrix}
{#1}_1 \\
{#1}_2 \\
\vdots \\
{#1}_N \\
\end{bmatrix}
}

\newcommand{\DETuvij}[4]{
\begin{vmatrix}
 {#1}_{#3} & {#1}_{#4} \\
 {#2}_{#3} & {#2}_{#4}
\end{vmatrix}
}

\newcommand{\DETuvwijk}[6]{
\begin{vmatrix}
 {#1}_{#4} & {#1}_{#5} & {#1}_{#6} \\
 {#2}_{#4} & {#2}_{#5} & {#2}_{#6} \\
 {#3}_{#4} & {#3}_{#5} & {#3}_{#6}
\end{vmatrix}
}

\newcommand{\DETuvwxijkl}[8]{
\begin{vmatrix}
 {#1}_{#5} & {#1}_{#6} & {#1}_{#7} & {#1}_{#8} \\
 {#2}_{#5} & {#2}_{#6} & {#2}_{#7} & {#2}_{#8} \\
 {#3}_{#5} & {#3}_{#6} & {#3}_{#7} & {#3}_{#8} \\
 {#4}_{#5} & {#4}_{#6} & {#4}_{#7} & {#4}_{#8} \\
\end{vmatrix}
}

%\newcommand{\DETuvwxyijklm}[10]{
%\begin{vmatrix}
% {#1}_{#6} & {#1}_{#7} & {#1}_{#8} & {#1}_{#9} & {#1}_{#10} \\
% {#2}_{#6} & {#2}_{#7} & {#2}_{#8} & {#2}_{#9} & {#2}_{#10} \\
% {#3}_{#6} & {#3}_{#7} & {#3}_{#8} & {#3}_{#9} & {#3}_{#10} \\
% {#4}_{#6} & {#4}_{#7} & {#4}_{#8} & {#4}_{#9} & {#4}_{#10} \\
% {#5}_{#6} & {#5}_{#7} & {#5}_{#8} & {#5}_{#9} & {#5}_{#10}
%\end{vmatrix}
%}

% R3 vector.
\newcommand{\VectorThree}[3]{
\begin{bmatrix}
 {#1} \\
 {#2} \\
 {#3}
\end{bmatrix}
}


%%<misc>
%
\newcommand{\Abs}[1]{{\left\lvert{#1}\right\rvert}}
\newcommand{\spacegrad}[0]{\boldsymbol{\nabla}}
\newcommand{\grad}[0]{\nabla}
\newcommand{\LL}[0]{\mathcal{L}}

% == \partial_{#1} {#2}
\newcommand{\PD}[2]{\frac{\partial {#2}}{\partial {#1}}}
% inline variant
\newcommand{\PDi}[2]{{\partial {#2}}/{\partial {#1}}}

\newcommand{\PDD}[3]{\frac{\partial^2 {#3}}{\partial {#1}\partial {#2}}}
%\newcommand{\PDd}[2]{\frac{\partial^2 {#2}}{{\partial{#1}}^2}}
\newcommand{\PDsq}[2]{\frac{\partial^2 {#2}}{(\partial {#1})^2}}

\newcommand{\Partial}[2]{\frac{\partial {#1}}{\partial {#2}}}
\DeclareMathOperator{\RejName}{Rej}
\newcommand{\Rej}[2]{\RejName_{#1}\left( {#2} \right)}
\newcommand{\Rm}[1]{\mathbb{R}^{#1}}
\newcommand{\Cm}[1]{\mathbb{C}^{#1}}
\newcommand{\conj}[0]{{*}}

%</misc>

% <grade selection>
%
\newcommand{\gpgrade}[2] {{\left\langle{{#1}}\right\rangle}_{#2}}

\newcommand{\gpgradezero}[1] {\gpgrade{#1}{}}
%\newcommand{\gpscalargrade}[1] {{\left\langle{{#1}}\right\rangle}}
%\newcommand{\gpgradezero}[1] {\gpgrade{#1}{0}}

%\newcommand{\gpgradeone}[1] {{\left\langle{{#1}}\right\rangle}_{1}}
\newcommand{\gpgradeone}[1] {\gpgrade{#1}{1}}

\newcommand{\gpgradetwo}[1] {\gpgrade{#1}{2}}
\newcommand{\gpgradethree}[1] {\gpgrade{#1}{3}}
\newcommand{\gpgradefour}[1] {\gpgrade{#1}{4}}
%
% </grade selection>



\newcommand{\adot}[0]{{\dot{a}}}
\newcommand{\bdot}[0]{{\dot{b}}}
% taken for centered dot:
%\newcommand{\cdot}[0]{{\dot{c}}}
%\newcommand{\ddot}[0]{{\dot{d}}}
\newcommand{\edot}[0]{{\dot{e}}}
\newcommand{\fdot}[0]{{\dot{f}}}
\newcommand{\gdot}[0]{{\dot{g}}}
\newcommand{\hdot}[0]{{\dot{h}}}
\newcommand{\idot}[0]{{\dot{i}}}
\newcommand{\jdot}[0]{{\dot{j}}}
\newcommand{\kdot}[0]{{\dot{k}}}
\newcommand{\ldot}[0]{{\dot{l}}}
\newcommand{\mdot}[0]{{\dot{m}}}
\newcommand{\ndot}[0]{{\dot{n}}}
%\newcommand{\odot}[0]{{\dot{o}}}
\newcommand{\pdot}[0]{{\dot{p}}}
\newcommand{\qdot}[0]{{\dot{q}}}
\newcommand{\rdot}[0]{{\dot{r}}}
\newcommand{\sdot}[0]{{\dot{s}}}
\newcommand{\tdot}[0]{{\dot{t}}}
\newcommand{\udot}[0]{{\dot{u}}}
\newcommand{\vdot}[0]{{\dot{v}}}
\newcommand{\wdot}[0]{{\dot{w}}}
\newcommand{\xdot}[0]{{\dot{x}}}
\newcommand{\ydot}[0]{{\dot{y}}}
\newcommand{\zdot}[0]{{\dot{z}}}
\newcommand{\addot}[0]{{\ddot{a}}}
\newcommand{\bddot}[0]{{\ddot{b}}}
\newcommand{\cddot}[0]{{\ddot{c}}}
%\newcommand{\dddot}[0]{{\ddot{d}}}
\newcommand{\eddot}[0]{{\ddot{e}}}
\newcommand{\fddot}[0]{{\ddot{f}}}
\newcommand{\gddot}[0]{{\ddot{g}}}
\newcommand{\hddot}[0]{{\ddot{h}}}
\newcommand{\iddot}[0]{{\ddot{i}}}
\newcommand{\jddot}[0]{{\ddot{j}}}
\newcommand{\kddot}[0]{{\ddot{k}}}
\newcommand{\lddot}[0]{{\ddot{l}}}
\newcommand{\mddot}[0]{{\ddot{m}}}
\newcommand{\nddot}[0]{{\ddot{n}}}
\newcommand{\oddot}[0]{{\ddot{o}}}
\newcommand{\pddot}[0]{{\ddot{p}}}
\newcommand{\qddot}[0]{{\ddot{q}}}
\newcommand{\rddot}[0]{{\ddot{r}}}
\newcommand{\sddot}[0]{{\ddot{s}}}
\newcommand{\tddot}[0]{{\ddot{t}}}
\newcommand{\uddot}[0]{{\ddot{u}}}
\newcommand{\vddot}[0]{{\ddot{v}}}
\newcommand{\wddot}[0]{{\ddot{w}}}
\newcommand{\xddot}[0]{{\ddot{x}}}
\newcommand{\yddot}[0]{{\ddot{y}}}
\newcommand{\zddot}[0]{{\ddot{z}}}

%<bold and dot greek symbols>
%

\newcommand{\Deltadot}[0]{{\dot{\Delta}}}
\newcommand{\Gammadot}[0]{{\dot{\Gamma}}}
\newcommand{\Lambdadot}[0]{{\dot{\Lambda}}}
\newcommand{\Omegadot}[0]{{\dot{\Omega}}}
\newcommand{\Phidot}[0]{{\dot{\Phi}}}
\newcommand{\Pidot}[0]{{\dot{\Pi}}}
\newcommand{\Psidot}[0]{{\dot{\Psi}}}
\newcommand{\Sigmadot}[0]{{\dot{\Sigma}}}
\newcommand{\Thetadot}[0]{{\dot{\Theta}}}
\newcommand{\Upsilondot}[0]{{\dot{\Upsilon}}}
\newcommand{\Xidot}[0]{{\dot{\Xi}}}
\newcommand{\alphadot}[0]{{\dot{\alpha}}}
\newcommand{\betadot}[0]{{\dot{\beta}}}
\newcommand{\chidot}[0]{{\dot{\chi}}}
\newcommand{\deltadot}[0]{{\dot{\delta}}}
\newcommand{\epsilondot}[0]{{\dot{\epsilon}}}
\newcommand{\etadot}[0]{{\dot{\eta}}}
\newcommand{\gammadot}[0]{{\dot{\gamma}}}
\newcommand{\kappadot}[0]{{\dot{\kappa}}}
\newcommand{\lambdadot}[0]{{\dot{\lambda}}}
\newcommand{\mudot}[0]{{\dot{\mu}}}
\newcommand{\nudot}[0]{{\dot{\nu}}}
\newcommand{\omegadot}[0]{{\dot{\omega}}}
\newcommand{\phidot}[0]{{\dot{\phi}}}
\newcommand{\pidot}[0]{{\dot{\pi}}}
\newcommand{\psidot}[0]{{\dot{\psi}}}
\newcommand{\rhodot}[0]{{\dot{\rho}}}
\newcommand{\sigmadot}[0]{{\dot{\sigma}}}
\newcommand{\taudot}[0]{{\dot{\tau}}}
\newcommand{\thetadot}[0]{{\dot{\theta}}}
\newcommand{\upsilondot}[0]{{\dot{\upsilon}}}
\newcommand{\varepsilondot}[0]{{\dot{\varepsilon}}}
\newcommand{\varphidot}[0]{{\dot{\varphi}}}
\newcommand{\varpidot}[0]{{\dot{\varpi}}}
\newcommand{\varrhodot}[0]{{\dot{\varrho}}}
\newcommand{\varsigmadot}[0]{{\dot{\varsigma}}}
\newcommand{\varthetadot}[0]{{\dot{\vartheta}}}
\newcommand{\xidot}[0]{{\dot{\xi}}}
\newcommand{\zetadot}[0]{{\dot{\zeta}}}

\newcommand{\Deltaddot}[0]{{\ddot{\Delta}}}
\newcommand{\Gammaddot}[0]{{\ddot{\Gamma}}}
\newcommand{\Lambdaddot}[0]{{\ddot{\Lambda}}}
\newcommand{\Omegaddot}[0]{{\ddot{\Omega}}}
\newcommand{\Phiddot}[0]{{\ddot{\Phi}}}
\newcommand{\Piddot}[0]{{\ddot{\Pi}}}
\newcommand{\Psiddot}[0]{{\ddot{\Psi}}}
\newcommand{\Sigmaddot}[0]{{\ddot{\Sigma}}}
\newcommand{\Thetaddot}[0]{{\ddot{\Theta}}}
\newcommand{\Upsilonddot}[0]{{\ddot{\Upsilon}}}
\newcommand{\Xiddot}[0]{{\ddot{\Xi}}}
\newcommand{\alphaddot}[0]{{\ddot{\alpha}}}
\newcommand{\betaddot}[0]{{\ddot{\beta}}}
\newcommand{\chiddot}[0]{{\ddot{\chi}}}
\newcommand{\deltaddot}[0]{{\ddot{\delta}}}
\newcommand{\epsilonddot}[0]{{\ddot{\epsilon}}}
\newcommand{\etaddot}[0]{{\ddot{\eta}}}
\newcommand{\gammaddot}[0]{{\ddot{\gamma}}}
\newcommand{\kappaddot}[0]{{\ddot{\kappa}}}
\newcommand{\lambdaddot}[0]{{\ddot{\lambda}}}
\newcommand{\muddot}[0]{{\ddot{\mu}}}
\newcommand{\nuddot}[0]{{\ddot{\nu}}}
\newcommand{\omegaddot}[0]{{\ddot{\omega}}}
\newcommand{\phiddot}[0]{{\ddot{\phi}}}
\newcommand{\piddot}[0]{{\ddot{\pi}}}
\newcommand{\psiddot}[0]{{\ddot{\psi}}}
\newcommand{\rhoddot}[0]{{\ddot{\rho}}}
\newcommand{\sigmaddot}[0]{{\ddot{\sigma}}}
\newcommand{\tauddot}[0]{{\ddot{\tau}}}
\newcommand{\thetaddot}[0]{{\ddot{\theta}}}
\newcommand{\upsilonddot}[0]{{\ddot{\upsilon}}}
\newcommand{\varepsilonddot}[0]{{\ddot{\varepsilon}}}
\newcommand{\varphiddot}[0]{{\ddot{\varphi}}}
\newcommand{\varpiddot}[0]{{\ddot{\varpi}}}
\newcommand{\varrhoddot}[0]{{\ddot{\varrho}}}
\newcommand{\varsigmaddot}[0]{{\ddot{\varsigma}}}
\newcommand{\varthetaddot}[0]{{\ddot{\vartheta}}}
\newcommand{\xiddot}[0]{{\ddot{\xi}}}
\newcommand{\zetaddot}[0]{{\ddot{\zeta}}}

\newcommand{\BDelta}[0]{\boldsymbol{\Delta}}
\newcommand{\BGamma}[0]{\boldsymbol{\Gamma}}
\newcommand{\BLambda}[0]{\boldsymbol{\Lambda}}
\newcommand{\BOmega}[0]{\boldsymbol{\Omega}}
\newcommand{\BPhi}[0]{\boldsymbol{\Phi}}
\newcommand{\BPi}[0]{\boldsymbol{\Pi}}
\newcommand{\BPsi}[0]{\boldsymbol{\Psi}}
\newcommand{\BSigma}[0]{\boldsymbol{\Sigma}}
\newcommand{\BTheta}[0]{\boldsymbol{\Theta}}
\newcommand{\BUpsilon}[0]{\boldsymbol{\Upsilon}}
\newcommand{\BXi}[0]{\boldsymbol{\Xi}}
\newcommand{\Balpha}[0]{\boldsymbol{\alpha}}
\newcommand{\Bbeta}[0]{\boldsymbol{\beta}}
\newcommand{\Bchi}[0]{\boldsymbol{\chi}}
\newcommand{\Bdelta}[0]{\boldsymbol{\delta}}
\newcommand{\Bepsilon}[0]{\boldsymbol{\epsilon}}
\newcommand{\Beta}[0]{\boldsymbol{\eta}}
\newcommand{\Bgamma}[0]{\boldsymbol{\gamma}}
\newcommand{\Bkappa}[0]{\boldsymbol{\kappa}}
\newcommand{\Blambda}[0]{\boldsymbol{\lambda}}
\newcommand{\Bmu}[0]{\boldsymbol{\mu}}
\newcommand{\Bnu}[0]{\boldsymbol{\nu}}
%\newcommand{\Bomega}[0]{\boldsymbol{\omega}}
\newcommand{\Bphi}[0]{\boldsymbol{\phi}}
\newcommand{\Bpi}[0]{\boldsymbol{\pi}}
\newcommand{\Bpsi}[0]{\boldsymbol{\psi}}
\newcommand{\Brho}[0]{\boldsymbol{\rho}}
\newcommand{\Bsigma}[0]{\boldsymbol{\sigma}}
%\newcommand{\Btau}[0]{\boldsymbol{\tau}}
%\newcommand{\Btheta}[0]{\boldsymbol{\theta}}
\newcommand{\Bupsilon}[0]{\boldsymbol{\upsilon}}
\newcommand{\Bvarepsilon}[0]{\boldsymbol{\varepsilon}}
\newcommand{\Bvarphi}[0]{\boldsymbol{\varphi}}
\newcommand{\Bvarpi}[0]{\boldsymbol{\varpi}}
\newcommand{\Bvarrho}[0]{\boldsymbol{\varrho}}
\newcommand{\Bvarsigma}[0]{\boldsymbol{\varsigma}}
\newcommand{\Bvartheta}[0]{\boldsymbol{\vartheta}}
\newcommand{\Bxi}[0]{\boldsymbol{\xi}}
\newcommand{\Bzeta}[0]{\boldsymbol{\zeta}}
%
%</bold and dot greek symbols>
%<infrequent>
%
%\newcommand{\AreaOp}[1]{\AName_{#1}}
%\newcommand{\Babs}[0]{\abs{\BB}}
%\newcommand{\Bcap}[0]{\hat{\BB}}
%\newcommand{\BrPrimeRej}[0]{\rcap(\rcap \wedge \Br')}
%\newcommand{\CA}[0]{\mathcal{A}}
%\newcommand{\Cos}[1]{\cos{\left({#1}\right)}}
%\newcommand{\Det}[1] {\abs{#1}}
%\newcommand{\Dsq}[2] {\frac {\partial^2 {#1}} {\partial {#2}^2}}
%\newcommand{\Exp}[1]{\exp{\left({#1}\right)}}
%\newcommand{\Norm}[1]{\left\lVert{#1}\right\rVert}
%\newcommand{\Sin}[1]{\sin{\left({#1}\right)}}
%\newcommand{\T}[0]{\text{T}}
%\newcommand{\VolumeOp}[1]{\VName_{#1}}
%\newcommand{\agrad}[0]{\Ba \cdot \nabla}
%\newcommand{\alphacap}[0]{\hat{\boldsymbol{\alpha}}}
%\newcommand{\Fcap}[0]{\hat{\BF}}
%\newcommand{\bithree}[0]{{\Bi}_3}
%\newcommand{\bxa}[0]{\Bx\Ba}
%\newcommand{\coordvec}[2]{
%\newcommand{\costheta}[0]{\acap \cdot \xcap}
%\newcommand{\ddt}[1]{\ddot{#1}}
%\newcommand{\ddu}[1] {\frac {d{#1}} {du}}
%\newcommand{\dsqxj}[2] {\frac {\partial^2 {#1}} {\partial {x_{#2}}^2}}
%\newcommand{\dtheta}[1]{\frac{d {#1}}{d \theta}}
%\newcommand{\dt}[1]{\dot{#1}}
%\newcommand{\dt}[1]{\frac{d {#1}}{dt}}
%\newcommand{\dxj}[2] {\frac {\partial {#1}} {\partial {x_{#2}}}}
%\newcommand{\halfPhi}[0]{\frac{\phi}{2}}
%\newcommand{\half}[0]{\inv{2}}
%\newcommand{\inv}[1]{\frac{1}{#1}}
%\newcommand{\laplacian}[0]{\nabla^2}
%\newcommand{\matrixoftx}[3]{
%\newcommand{\nrrp}[0]{\norm{\rcap \wedge \Br'}}
%\newcommand{\oiint}{\bigcirc \hspace{-1.4em} \int \hspace{-.8em} \int}
%\newcommand{\transpose}[1]{{#1}^{\text{T}}}
%\newcommand{\transpose}[1]{{{#1}^{\TextTranspose}}}
%\newcommand{\transpose}[1]{{{#1}^{\text{T}}}}
%\newcommand{\barA}[0]{\bar{A}}
%\newcommand{\qbar}[0]{\bar{q}}
%\newcommand{\qdotbar}[0]{\dot{\bar{q}}}
%
%</infrequent>





%\usepackage[bookmarks=true]{hyperref}

%\usepackage{color,cite,graphicx}
   % use colour in the document, put your citations as [1-4]
   % rather than [1,2,3,4] (it looks nicer, and the extended LaTeX2e
   % graphics package. 
%\usepackage{latexsym,amssymb,epsf} % don't remember if these are
   % needed, but their inclusion can't do any damage


\chapter{Some rapidity angle notes. }
\label{chap:rapidity}
%\author{Peeter Joot \quad peeter.joot@gmail.com}
\date{ Dec 18, 2008.  $RCSfile: rapidity.tex,v $ Last $Revision: 1.8 $ $Date: 2009/06/14 23:51:45 $ }

%\begin{document}

%\maketitle{}
%\tableofcontents

\section{Motivation. }

Lut writes, "setting up a little calculation, I'm writing a 4-velocity as"
 
\begin{align*}
( \gamma, \gamma \beta_x, \gamma \beta_y, \gamma \beta_z ),
\end{align*}
 
which has length $-1$ if $\gamma^{-2} = 1 - (\beta_x)^2+(\beta_y)^2+(\beta_z)^2$.
 
Can you write this in terms of the 3 rapidities $a_1, a_2, a_3$ ?

I wasn't able to answer this right away so it is worth an examination
of rapidity angles to ensure that I understand the ideas.

\section{Stuff. }

Putting back in the $c$ factors, and switching to the $+---$ signature I'm used to, the position
vector is

\begin{align*}
x &= x^\mu \gamma_\mu = ct \gamma_0 + x^i \gamma_i,
\end{align*}

for which the corresponding proper velocity is
\begin{align*}
v &= \frac{dx}{d\tau} = c \frac{dt}{d\tau} \gamma_0 + \frac{dx^i}{dt} \frac{dt}{d\tau} \gamma_i
\end{align*}

Writing $\gamma = dt/d\tau$, and squaring the proper velocity we have

\begin{align*}
\frac{v^2}{c^2}
&= 1 \\
&= \gamma^2 \left(\gamma_0 + \inv{c}\frac{dx^i}{dt} \gamma_i\right)^2 \\
&= \gamma^2 \left(1 - \sum_i \inv{c^2} \left(\frac{dx^i}{dt}\right)^2 \right) \\
\end{align*}

So we have 

\begin{align*}
\gamma 
&= \inv{\sqrt{1 - \sum_i \inv{c^2} \left(\frac{dx^i}{dt}\right)^2 }} \\
\end{align*}

Observe that $\gamma$ ranges from $1$ to infinity, and can thus be described by the $[0,\infty]$ range of the hyperbolic cosine function.  With
the relative velocity $\Bv = \sum_i (dx^i/dt) \sigma_i$, this is

\begin{align*}
\frac{dt}{d\tau} &= \gamma  \\
&= \cosh\alpha \\
&= \inv{\sqrt{1 - (\Bv/c)^2}}
\end{align*}

In terms of the hyperbolic cosine for $\gamma$ our proper velocity then becomes

\begin{align*}
v/c &= \cosh\alpha \left(1 + \frac{\Bv}{c} \right) \gamma_0
\end{align*}

Taking the hint from the Lorentz transform where we have both $\sinh$ and $\cosh$ factors can one write

\begin{align*}
\gamma \frac{\Bv}{c} = \sinh\alpha
\end{align*}

This gives

\begin{align*}
\frac{\Bv}{c} = \tanh\alpha
\end{align*}

so we need $\alpha$ to be a spacetime relative vector.  With $\cosh$ being an even function $\cosh{\alpha} = \cosh{\Abs{\alpha}}$, so this
is still a 
scalar as desired.  Inverting the relationship for $\alpha$ we have

\begin{align*}
\alpha = \tanh^{-1} (\Bv/c) = \vcap \tanh^{-1} (\Abs{\Bv/c})
\end{align*}

The unit vector $\vcap$ can be factored out of the inverse hyperbolic tangent function since it is odd (consider the Taylor series expansion of $\tanh^{-1}$ to see why one can do this).

Finally, we have by 
dotting with the spatial basis vectors $\sigma_i$ three quantities in terms of spacetime vector rapidity angle

\begin{align*}
\alpha_i = (\vcap \cdot \sigma_i) \tanh^{-1} (\Abs{\Bv/c}).
\end{align*}

The $\vcap \cdot \sigma_i$ parts are direction cosines, so the three rapidities Lut was asking about all appear to be weighted direction cosines.

%\bibliographystyle{plainnat}
%\bibliography{myrefs}

%\end{document}

%
% Copyright � 2012 Peeter Joot.  All Rights Reserved.
% Licenced as described in the file LICENSE under the root directory of this GIT repository.
%

%
%
\chapter{Lorentz invariance of energy momentum four vector}
\label{chap:invarianceEnMom}
%\blogpage{http://sites.google.com/site/peeterjoot/math2009/frequencyTx.pdf}
%\author{Peeter Joot \quad peeterjoot@protonmail.com }
%\date{ June 21, 2009.  invarianceEnMom.tex }

\section{Motivation}

A blurb on Lorentz invariance eventually removed from other notes.
Probably want to merge this with my treatment of application of the
chain rule to the wave equation as a method of
finding the Lorentz boost matrix.

\section{Prerequisite concepts.  Wave equation, and Lorentz invariance}

An unforced mechanical wave described by a function \(\psi(t,\Bx)\), propagating undamped and unforced with
velocity \(v\) is described by the familiar equation

\begin{align}\label{eqn:invarianceEnMom:waveEquationV}
\inv{v^2} \frac{\partial^2 \psi}{\partial t^2} - \spacegrad^2 \psi = 0
\end{align}

For the purposes of this discussion, a relativistic wave is described by (\(\eqnref{eqn:invarianceEnMom:waveEquationV}\)) with two
additional conditions.  The first is that the wave speed is
\(v = c\), the speed of light.  The second condition required for the label relativistic
is a restriction on the allowed coordinate transformations.  These are the linear transformations
of space time coordinates
\((t,x,y,z) \rightarrow (t', x', y', z')\) for which the wave equation retains precisely this form

\begin{align}\label{eqn:invarianceEnMom:waveEquationInv}
\inv{c^2} \frac{\partial^2 \psi}{\partial {t}^2}
- \frac{\partial^2 \psi}{\partial {x}^2}
- \frac{\partial^2 \psi}{\partial {y}^2}
- \frac{\partial^2 \psi}{\partial {z}^2}
=
\inv{c^2} \frac{\partial^2 \psi}{\partial {t'}^2}
- \frac{\partial^2 \psi}{\partial {x'}^2}
- \frac{\partial^2 \psi}{\partial {y'}^2}
- \frac{\partial^2 \psi}{\partial {z'}^2}
\end{align}

Such transformations, the Lorentz transformations,
are those that introduce no cross term such as \(\partial^2 \psi/\partial x' \partial y'\),
and do not change the wave velocity.  One can show that spatial rotations such as

\begin{align}
\begin{bmatrix}
ct' \\
x' \\
y' \\
z' \\
\end{bmatrix}
=
\begin{bmatrix}
1 & 0 & 0 & 0 \\
0 & 1 & 0 & 0 \\
0 & 0 & \cos\theta & \sin\theta \\
0 & 0 & -\sin\theta & \cos\theta \\
\end{bmatrix}
\begin{bmatrix}
ct \\
x \\
y \\
z \\
\end{bmatrix}
\end{align}

Or Lorentz boosts such as
\begin{align}
\begin{bmatrix}
ct' \\
x' \\
y' \\
z' \\
\end{bmatrix}
=
\begin{bmatrix}
\cosh\alpha & -\sinh\alpha & 0 & 0 \\
-\sinh\alpha & \cosh\alpha & 0 & 0 \\
0 & 0 & 1 & 0 \\
0 & 0 & 0 & 1 \\
\end{bmatrix}
\begin{bmatrix}
ct \\
x \\
y \\
z \\
\end{bmatrix}
\end{align}

Or any composition of such transformations meet this requirement.

In the linear transformations above the space and time coordinates were merged into a single vector representation,
the particle worldline vector, often written with shorthand such as

\begin{align}
X \equiv (ct, \Bx)
\end{align}

For such a vector, a Lorentz length can be defined

\begin{align}
X^2 \equiv c^2 t^2 - \Bx^2 \equiv c^2 t^2 - \Bx \cdot \Bx
\end{align}

Without specific discussion of the wave equation,
a more usual but equivalent definition of Lorentz transformations, are those that leave this
Lorentz length unchanged, as in

\begin{align}
c^2 t^2 - \Bx^2 = c^2 {t'}^2 - {\Bx'}^2
\end{align}

If we introduce a vector space time derivative operator

\begin{align}
\grad \equiv \left(\inv{c}\frac{\partial}{\partial t}, \spacegrad \right)
\end{align}

The Lorentz invariant length of this vector operator is in fact our wave equation operator

\begin{align}
\square \equiv \grad^2 = \inv{c^2}\frac{\partial^2}{\partial t^2} - \spacegrad \cdot \spacegrad
\end{align}

It is clear that the original requirement for wave equation invariance \eqnref{eqn:invarianceEnMom:waveEquationInv} is also contained within
this definition of Lorentz invariant length.

Unit vectors with respect to Lorentz length are necessarily Lorentz invariant.  Considering for example the time rate of change of
a particle worldline we have

\begin{align}
\left(\frac{d}{dt}(ct, \Bx) \right)^2 = c^2 - \Bv^2
\end{align}

which implies that the Lorentz length of
\begin{align}
\frac{1}{\sqrt{1 - \Bv^2/c^2}}(1, \Bv/c)
\end{align}

is just one.  For the purposes of this
Schr\"{o}dinger equation
discussion a scaling of this four vector so that it has dimensions of energy is required

\begin{align}
\frac{1}{\sqrt{1 - \Bv^2/c^2}}(m c^2, m\Bv c)
\end{align}

Algebraically, the Lorentz length of this four vector can easily be confirmed to be \((m c^2)^2\).  This quantity
we will identify as an energy-momentum four vector as follows

\begin{align}\label{eqn:invarianceEnMom:energyMomentumFourVec}
P = (E/c, \Bp)
\end{align}

With energy defined as
\begin{align}
E \equiv \frac{m c^2}{\sqrt{1 - \Bv^2/c^2}}
\end{align}

and spatial momentum defined as
\begin{align}
\Bp \equiv \frac{m \Bv}{\sqrt{1 - \Bv^2/c^2}}
\end{align}

It is beyond the scope of these notes to provide a good justification for this identification.
\footnote{This is a dodge, and having to make a statement like this shows that it is beyond the scope of the author's understanding to coherently justify this identification.  In the spirit of my engineering education I can at least work with it.}

From \eqnref{eqn:invarianceEnMom:energyMomentumFourVec} that Lorentz length of the
energy momentum four vector is

\begin{align}
P^2 = E^2/c^2 - \Bp^2 = m^2 c^2
\end{align}

%\documentclass[]{eliblog}

\usepackage{amsmath}
\usepackage{mathpazo}

%
% shorthand for bold symbols, convenient for vectors and matrices
%
\newcommand{\Ba}[0]{\mathbf{a}}
\newcommand{\Bb}[0]{\mathbf{b}}
\newcommand{\Bc}[0]{\mathbf{c}}
\newcommand{\Bd}[0]{\mathbf{d}}
\newcommand{\Be}[0]{\mathbf{e}}
\newcommand{\Bf}[0]{\mathbf{f}}
\newcommand{\Bg}[0]{\mathbf{g}}
\newcommand{\Bh}[0]{\mathbf{h}}
\newcommand{\Bi}[0]{\mathbf{i}}
\newcommand{\Bj}[0]{\mathbf{j}}
\newcommand{\Bk}[0]{\mathbf{k}}
\newcommand{\Bl}[0]{\mathbf{l}}
\newcommand{\Bm}[0]{\mathbf{m}}
\newcommand{\Bn}[0]{\mathbf{n}}
\newcommand{\Bo}[0]{\mathbf{o}}
\newcommand{\Bp}[0]{\mathbf{p}}
\newcommand{\Bq}[0]{\mathbf{q}}
\newcommand{\Br}[0]{\mathbf{r}}
\newcommand{\Bs}[0]{\mathbf{s}}
\newcommand{\Bt}[0]{\mathbf{t}}
\newcommand{\Bu}[0]{\mathbf{u}}
\newcommand{\Bv}[0]{\mathbf{v}}
\newcommand{\Bw}[0]{\mathbf{w}}
\newcommand{\Bx}[0]{\mathbf{x}}
\newcommand{\By}[0]{\mathbf{y}}
\newcommand{\Bz}[0]{\mathbf{z}}
\newcommand{\BA}[0]{\mathbf{A}}
\newcommand{\BB}[0]{\mathbf{B}}
\newcommand{\BC}[0]{\mathbf{C}}
\newcommand{\BD}[0]{\mathbf{D}}
\newcommand{\BE}[0]{\mathbf{E}}
\newcommand{\BF}[0]{\mathbf{F}}
\newcommand{\BG}[0]{\mathbf{G}}
\newcommand{\BH}[0]{\mathbf{H}}
\newcommand{\BI}[0]{\mathbf{I}}
\newcommand{\BJ}[0]{\mathbf{J}}
\newcommand{\BK}[0]{\mathbf{K}}
\newcommand{\BL}[0]{\mathbf{L}}
\newcommand{\BM}[0]{\mathbf{M}}
\newcommand{\BN}[0]{\mathbf{N}}
\newcommand{\BO}[0]{\mathbf{O}}
\newcommand{\BP}[0]{\mathbf{P}}
\newcommand{\BQ}[0]{\mathbf{Q}}
\newcommand{\BR}[0]{\mathbf{R}}
\newcommand{\BS}[0]{\mathbf{S}}
\newcommand{\BT}[0]{\mathbf{T}}
\newcommand{\BU}[0]{\mathbf{U}}
\newcommand{\BV}[0]{\mathbf{V}}
\newcommand{\BW}[0]{\mathbf{W}}
\newcommand{\BX}[0]{\mathbf{X}}
\newcommand{\BY}[0]{\mathbf{Y}}
\newcommand{\BZ}[0]{\mathbf{Z}}

\newcommand{\Bzero}[0]{\mathbf{0}}
\newcommand{\Btheta}[0]{\boldsymbol{\theta}}
\newcommand{\Btau}[0]{\boldsymbol{\tau}}
\newcommand{\Bomega}[0]{\boldsymbol{\omega}}

%
% shorthand for unit vectors
%
\newcommand{\acap}[0]{\hat{\Ba}}
\newcommand{\bcap}[0]{\hat{\Bb}}
\newcommand{\ccap}[0]{\hat{\Bc}}
\newcommand{\dcap}[0]{\hat{\Bd}}
\newcommand{\ecap}[0]{\hat{\Be}}
\newcommand{\fcap}[0]{\hat{\Bf}}
\newcommand{\gcap}[0]{\hat{\Bg}}
\newcommand{\hcap}[0]{\hat{\Bh}}
\newcommand{\icap}[0]{\hat{\Bi}}
\newcommand{\jcap}[0]{\hat{\Bj}}
\newcommand{\kcap}[0]{\hat{\Bk}}
\newcommand{\lcap}[0]{\hat{\Bl}}
\newcommand{\mcap}[0]{\hat{\Bm}}
\newcommand{\ncap}[0]{\hat{\Bn}}
\newcommand{\ocap}[0]{\hat{\Bo}}
\newcommand{\pcap}[0]{\hat{\Bp}}
\newcommand{\qcap}[0]{\hat{\Bq}}
\newcommand{\rcap}[0]{\hat{\Br}}
\newcommand{\scap}[0]{\hat{\Bs}}
\newcommand{\tcap}[0]{\hat{\Bt}}
\newcommand{\ucap}[0]{\hat{\Bu}}
\newcommand{\vcap}[0]{\hat{\Bv}}
\newcommand{\wcap}[0]{\hat{\Bw}}
\newcommand{\xcap}[0]{\hat{\Bx}}
\newcommand{\ycap}[0]{\hat{\By}}
\newcommand{\zcap}[0]{\hat{\Bz}}
\newcommand{\thetacap}[0]{\hat{\Btheta}}

%
% to write R^n and C^n in a distinguishable fashion.  Perhaps change this
% to the double lined characters upon figuring out how to do so.
%
\newcommand{\C}[1]{$\mathbb{C}^{#1}$}
\newcommand{\R}[1]{$\mathbb{R}^{#1}$}

%
% various generally useful helpers
%

% derivative of #1 wrt. #2:
\newcommand{\D}[2] {\frac {d#2} {d#1}}

\newcommand{\inv}[1]{\frac{1}{#1}}
\newcommand{\cross}[0]{\times}

\newcommand{\abs}[1]{\lvert{#1}\rvert}
\newcommand{\norm}[1]{\lVert{#1}\rVert}
\newcommand{\innerprod}[2]{\langle{#1}, {#2}\rangle}
\newcommand{\dotprod}[2]{{#1} \cdot {#2}}
\newcommand{\bdotprod}[2]{\left({#1} \cdot {#2}\right)}
\newcommand{\crossprod}[2]{{#1} \cross {#2}}
\newcommand{\tripleprod}[3]{\dotprod{\left(\crossprod{#1}{#2}\right)}{#3}}

\DeclareMathOperator{\Proj}{Proj}
\DeclareMathOperator{\Span}{span}
\DeclareMathOperator{\Sgn}{sgn}
\DeclareMathOperator{\Area}{Area}
\DeclareMathOperator{\Volume}{Volume}

%
% A few miscellaneous things specific to this document
%
\newcommand{\crossop}[1]{\crossprod{#1}{}}

% R2 vector.
\newcommand{\VectorTwo}[2]{
\begin{bmatrix}
 {#1} \\
 {#2}
\end{bmatrix}
}

\newcommand{\VectorN}[1]{
\begin{bmatrix}
{#1}_1 \\
{#1}_2 \\
\vdots \\
{#1}_N \\
\end{bmatrix}
}

\newcommand{\DETuvij}[4]{
\begin{vmatrix}
 {#1}_{#3} & {#1}_{#4} \\
 {#2}_{#3} & {#2}_{#4}
\end{vmatrix}
}

\newcommand{\DETuvwijk}[6]{
\begin{vmatrix}
 {#1}_{#4} & {#1}_{#5} & {#1}_{#6} \\
 {#2}_{#4} & {#2}_{#5} & {#2}_{#6} \\
 {#3}_{#4} & {#3}_{#5} & {#3}_{#6}
\end{vmatrix}
}

\newcommand{\DETuvwxijkl}[8]{
\begin{vmatrix}
 {#1}_{#5} & {#1}_{#6} & {#1}_{#7} & {#1}_{#8} \\
 {#2}_{#5} & {#2}_{#6} & {#2}_{#7} & {#2}_{#8} \\
 {#3}_{#5} & {#3}_{#6} & {#3}_{#7} & {#3}_{#8} \\
 {#4}_{#5} & {#4}_{#6} & {#4}_{#7} & {#4}_{#8} \\
\end{vmatrix}
}

%\newcommand{\DETuvwxyijklm}[10]{
%\begin{vmatrix}
% {#1}_{#6} & {#1}_{#7} & {#1}_{#8} & {#1}_{#9} & {#1}_{#10} \\
% {#2}_{#6} & {#2}_{#7} & {#2}_{#8} & {#2}_{#9} & {#2}_{#10} \\
% {#3}_{#6} & {#3}_{#7} & {#3}_{#8} & {#3}_{#9} & {#3}_{#10} \\
% {#4}_{#6} & {#4}_{#7} & {#4}_{#8} & {#4}_{#9} & {#4}_{#10} \\
% {#5}_{#6} & {#5}_{#7} & {#5}_{#8} & {#5}_{#9} & {#5}_{#10}
%\end{vmatrix}
%}

% R3 vector.
\newcommand{\VectorThree}[3]{
\begin{bmatrix}
 {#1} \\
 {#2} \\
 {#3}
\end{bmatrix}
}



\author{Peeter Joot}
\email{peeter.joot@gmail.com}


\chapter{Force free relativistic motion.}
\label{chap:constFourMomentum}
%\useCCL
\blogpage{http://sites.google.com/site/peeterjoot/math2009/constFourMomentum.pdf}
\date{Nov 15, 2009}
\revisionInfo{$RCSfile: constFourMomentum.tex,v $ Last $Revision: 1.3 $ $Date: 2009/11/15 16:23:27 $}

%\beginArtWithToc
\beginArtNoToc

\section{Motivation}

Considering the Euler-Lagrange solutions for the relativistic force free covariant Lagrangian

\begin{align}\label{eqn:constFourMomentum:qqq1}
\LL &= \inv{2} m \xdot^\mu \xdot_\mu \\
\xdot^\mu &= \frac{d x^\mu}{d \tau},
\end{align}

we get a set of four constant momentum equations

\begin{align}\label{eqn:constFourMomentum:qqq5}
m \xdot_\mu = m v_\mu(0).
\end{align}

Can we make some sense of this?  While this seems natural enough in comparison to Newtonian physics, we ``just'' add a component when switching to a four vector representation, the $\gamma$ factors that one may expect are nowhere obvious to be seen.  

\section{Guts}

A decomposition into an explicit spacetime split looks like it is the first step along the path to resolves this

\begin{align}\label{eqn:constFourMomentum:qqq6}
X \equiv (c t, \Bx) = (ct, x^1, x^2, x^3).
\end{align}

Considering first the time component of our equations of motion we have

\begin{align}\label{eqn:constFourMomentum:qqq7}
m c \frac{dt}{d\tau} = m v_0(0).
\end{align}

Or 
\begin{align}\label{eqn:constFourMomentum:qqq8}
\frac{dt}{d\tau} = \frac{v_0(0)}{c}.
\end{align}

For the spatial components we have

\begin{align*}
m \frac{d x_k}{d\tau} 
&=
m \frac{d x_k}{dt}  \frac{dt}{d\tau} \\
&=
m \frac{d x_k}{dt}  \frac{v_0(0)}{c}.
\end{align*}

With a switch to upper indexes, the remaining three equations of motion are then just

\begin{align}\label{eqn:constFourMomentum:qqq10}
\frac{d x^k}{dt} = c \frac{ v^k(0) }{ v_0(0) }.
\end{align}

Or
\begin{align}\label{eqn:constFourMomentum:qqq10a}
\frac{d \Bv}{dt} = c \frac{ \Bv(0) }{ v_0(0) }.
\end{align}

The math seems to be saying that relativistically, in the absence of forces, we have constant velocity in our rest frame.  This constant velocity is relative to the initial time component of the four velocity.  This is not what I would have expected from the relativistically corrected Newtons laws in three vector form

\begin{align}\label{eqn:constFourMomentum:qqq11}
\BF = \frac{d}{dt}\left( \frac{m \Bv}{\sqrt{1 - (\Bv/c)^2}} \right).
\end{align}

In this equation it appears that we should only expect constant velocity in the small speed limit where $\Bv/c$ can be neglected.  If we, however, take this equation and run with it, where does it lead?  Introducing a vector constant for the spatial momentum $\Bp(0)$ we have

\begin{align}\label{eqn:constFourMomentum:qqq12}
\frac{m \Bv}{\sqrt{1 - (\Bv/c)^2}} = \Bp_0.
\end{align}

We can now square and rearrange, yielding

\begin{align}\label{eqn:constFourMomentum:qqq13}
\frac{\Bv^2}{c^2} = \frac{ {\Bp_0}^2 } { m^2 c^2 + {\Bp_0}^2 }.
\end{align}

With the additional assumption that $\Bv$ and $\Bp_0$ are colinear we can take roots (the two could differ by an arbitrary spatial rotation), yielding

\begin{align}\label{eqn:constFourMomentum:qqq14}
\frac{\Bv}{c} = \frac{ \Bp_0} { \sqrt{m^2 c^2 + {\Bp_0}^2} }.
\end{align}

Just as seen starting from the covariant Lagrangian, we have constant spatial velocity in the absence of external forces.  There was no fundamental inconsistency between the covariant result and the relativistically corrected Newtonian force law.  It was just not initially obvious to me that this was the case.

%\EndArticle
\EndNoBibArticle

%%
% Copyright � 2015 Peeter Joot.  All Rights Reserved.
% Licenced as described in the file LICENSE under the root directory of this GIT repository.
%
\documentclass[]{eliblog}

\usepackage{amsmath}
\usepackage{mathpazo}

%
% shorthand for bold symbols, convenient for vectors and matrices
%
\newcommand{\Ba}[0]{\mathbf{a}}
\newcommand{\Bb}[0]{\mathbf{b}}
\newcommand{\Bc}[0]{\mathbf{c}}
\newcommand{\Bd}[0]{\mathbf{d}}
\newcommand{\Be}[0]{\mathbf{e}}
\newcommand{\Bf}[0]{\mathbf{f}}
\newcommand{\Bg}[0]{\mathbf{g}}
\newcommand{\Bh}[0]{\mathbf{h}}
\newcommand{\Bi}[0]{\mathbf{i}}
\newcommand{\Bj}[0]{\mathbf{j}}
\newcommand{\Bk}[0]{\mathbf{k}}
\newcommand{\Bl}[0]{\mathbf{l}}
\newcommand{\Bm}[0]{\mathbf{m}}
\newcommand{\Bn}[0]{\mathbf{n}}
\newcommand{\Bo}[0]{\mathbf{o}}
\newcommand{\Bp}[0]{\mathbf{p}}
\newcommand{\Bq}[0]{\mathbf{q}}
\newcommand{\Br}[0]{\mathbf{r}}
\newcommand{\Bs}[0]{\mathbf{s}}
\newcommand{\Bt}[0]{\mathbf{t}}
\newcommand{\Bu}[0]{\mathbf{u}}
\newcommand{\Bv}[0]{\mathbf{v}}
\newcommand{\Bw}[0]{\mathbf{w}}
\newcommand{\Bx}[0]{\mathbf{x}}
\newcommand{\By}[0]{\mathbf{y}}
\newcommand{\Bz}[0]{\mathbf{z}}
\newcommand{\BA}[0]{\mathbf{A}}
\newcommand{\BB}[0]{\mathbf{B}}
\newcommand{\BC}[0]{\mathbf{C}}
\newcommand{\BD}[0]{\mathbf{D}}
\newcommand{\BE}[0]{\mathbf{E}}
\newcommand{\BF}[0]{\mathbf{F}}
\newcommand{\BG}[0]{\mathbf{G}}
\newcommand{\BH}[0]{\mathbf{H}}
\newcommand{\BI}[0]{\mathbf{I}}
\newcommand{\BJ}[0]{\mathbf{J}}
\newcommand{\BK}[0]{\mathbf{K}}
\newcommand{\BL}[0]{\mathbf{L}}
\newcommand{\BM}[0]{\mathbf{M}}
\newcommand{\BN}[0]{\mathbf{N}}
\newcommand{\BO}[0]{\mathbf{O}}
\newcommand{\BP}[0]{\mathbf{P}}
\newcommand{\BQ}[0]{\mathbf{Q}}
\newcommand{\BR}[0]{\mathbf{R}}
\newcommand{\BS}[0]{\mathbf{S}}
\newcommand{\BT}[0]{\mathbf{T}}
\newcommand{\BU}[0]{\mathbf{U}}
\newcommand{\BV}[0]{\mathbf{V}}
\newcommand{\BW}[0]{\mathbf{W}}
\newcommand{\BX}[0]{\mathbf{X}}
\newcommand{\BY}[0]{\mathbf{Y}}
\newcommand{\BZ}[0]{\mathbf{Z}}

\newcommand{\Bzero}[0]{\mathbf{0}}
\newcommand{\Btheta}[0]{\boldsymbol{\theta}}
\newcommand{\Btau}[0]{\boldsymbol{\tau}}
\newcommand{\Bomega}[0]{\boldsymbol{\omega}}

%
% shorthand for unit vectors
%
\newcommand{\acap}[0]{\hat{\Ba}}
\newcommand{\bcap}[0]{\hat{\Bb}}
\newcommand{\ccap}[0]{\hat{\Bc}}
\newcommand{\dcap}[0]{\hat{\Bd}}
\newcommand{\ecap}[0]{\hat{\Be}}
\newcommand{\fcap}[0]{\hat{\Bf}}
\newcommand{\gcap}[0]{\hat{\Bg}}
\newcommand{\hcap}[0]{\hat{\Bh}}
\newcommand{\icap}[0]{\hat{\Bi}}
\newcommand{\jcap}[0]{\hat{\Bj}}
\newcommand{\kcap}[0]{\hat{\Bk}}
\newcommand{\lcap}[0]{\hat{\Bl}}
\newcommand{\mcap}[0]{\hat{\Bm}}
\newcommand{\ncap}[0]{\hat{\Bn}}
\newcommand{\ocap}[0]{\hat{\Bo}}
\newcommand{\pcap}[0]{\hat{\Bp}}
\newcommand{\qcap}[0]{\hat{\Bq}}
\newcommand{\rcap}[0]{\hat{\Br}}
\newcommand{\scap}[0]{\hat{\Bs}}
\newcommand{\tcap}[0]{\hat{\Bt}}
\newcommand{\ucap}[0]{\hat{\Bu}}
\newcommand{\vcap}[0]{\hat{\Bv}}
\newcommand{\wcap}[0]{\hat{\Bw}}
\newcommand{\xcap}[0]{\hat{\Bx}}
\newcommand{\ycap}[0]{\hat{\By}}
\newcommand{\zcap}[0]{\hat{\Bz}}
\newcommand{\thetacap}[0]{\hat{\Btheta}}

%
% to write R^n and C^n in a distinguishable fashion.  Perhaps change this
% to the double lined characters upon figuring out how to do so.
%
\newcommand{\C}[1]{$\mathbb{C}^{#1}$}
\newcommand{\R}[1]{$\mathbb{R}^{#1}$}

%
% various generally useful helpers
%

% derivative of #1 wrt. #2:
\newcommand{\D}[2] {\frac {d#2} {d#1}}

\newcommand{\inv}[1]{\frac{1}{#1}}
\newcommand{\cross}[0]{\times}

\newcommand{\abs}[1]{\lvert{#1}\rvert}
\newcommand{\norm}[1]{\lVert{#1}\rVert}
\newcommand{\innerprod}[2]{\langle{#1}, {#2}\rangle}
\newcommand{\dotprod}[2]{{#1} \cdot {#2}}
\newcommand{\bdotprod}[2]{\left({#1} \cdot {#2}\right)}
\newcommand{\crossprod}[2]{{#1} \cross {#2}}
\newcommand{\tripleprod}[3]{\dotprod{\left(\crossprod{#1}{#2}\right)}{#3}}

\DeclareMathOperator{\Proj}{Proj}
\DeclareMathOperator{\Span}{span}
\DeclareMathOperator{\Sgn}{sgn}
\DeclareMathOperator{\Area}{Area}
\DeclareMathOperator{\Volume}{Volume}

%
% A few miscellaneous things specific to this document
%
\newcommand{\crossop}[1]{\crossprod{#1}{}}

% R2 vector.
\newcommand{\VectorTwo}[2]{
\begin{bmatrix}
 {#1} \\
 {#2}
\end{bmatrix}
}

\newcommand{\VectorN}[1]{
\begin{bmatrix}
{#1}_1 \\
{#1}_2 \\
\vdots \\
{#1}_N \\
\end{bmatrix}
}

\newcommand{\DETuvij}[4]{
\begin{vmatrix}
 {#1}_{#3} & {#1}_{#4} \\
 {#2}_{#3} & {#2}_{#4}
\end{vmatrix}
}

\newcommand{\DETuvwijk}[6]{
\begin{vmatrix}
 {#1}_{#4} & {#1}_{#5} & {#1}_{#6} \\
 {#2}_{#4} & {#2}_{#5} & {#2}_{#6} \\
 {#3}_{#4} & {#3}_{#5} & {#3}_{#6}
\end{vmatrix}
}

\newcommand{\DETuvwxijkl}[8]{
\begin{vmatrix}
 {#1}_{#5} & {#1}_{#6} & {#1}_{#7} & {#1}_{#8} \\
 {#2}_{#5} & {#2}_{#6} & {#2}_{#7} & {#2}_{#8} \\
 {#3}_{#5} & {#3}_{#6} & {#3}_{#7} & {#3}_{#8} \\
 {#4}_{#5} & {#4}_{#6} & {#4}_{#7} & {#4}_{#8} \\
\end{vmatrix}
}

%\newcommand{\DETuvwxyijklm}[10]{
%\begin{vmatrix}
% {#1}_{#6} & {#1}_{#7} & {#1}_{#8} & {#1}_{#9} & {#1}_{#10} \\
% {#2}_{#6} & {#2}_{#7} & {#2}_{#8} & {#2}_{#9} & {#2}_{#10} \\
% {#3}_{#6} & {#3}_{#7} & {#3}_{#8} & {#3}_{#9} & {#3}_{#10} \\
% {#4}_{#6} & {#4}_{#7} & {#4}_{#8} & {#4}_{#9} & {#4}_{#10} \\
% {#5}_{#6} & {#5}_{#7} & {#5}_{#8} & {#5}_{#9} & {#5}_{#10}
%\end{vmatrix}
%}

% R3 vector.
\newcommand{\VectorThree}[3]{
\begin{bmatrix}
 {#1} \\
 {#2} \\
 {#3}
\end{bmatrix}
}



\author{Peeter Joot}
\email{peeter.joot@gmail.com}

%\documentclass[]{eliblogwidescreen}

\usepackage{amsmath}
\usepackage{mathpazo}

%
% shorthand for bold symbols, convenient for vectors and matrices
%
\newcommand{\Ba}[0]{\mathbf{a}}
\newcommand{\Bb}[0]{\mathbf{b}}
\newcommand{\Bc}[0]{\mathbf{c}}
\newcommand{\Bd}[0]{\mathbf{d}}
\newcommand{\Be}[0]{\mathbf{e}}
\newcommand{\Bf}[0]{\mathbf{f}}
\newcommand{\Bg}[0]{\mathbf{g}}
\newcommand{\Bh}[0]{\mathbf{h}}
\newcommand{\Bi}[0]{\mathbf{i}}
\newcommand{\Bj}[0]{\mathbf{j}}
\newcommand{\Bk}[0]{\mathbf{k}}
\newcommand{\Bl}[0]{\mathbf{l}}
\newcommand{\Bm}[0]{\mathbf{m}}
\newcommand{\Bn}[0]{\mathbf{n}}
\newcommand{\Bo}[0]{\mathbf{o}}
\newcommand{\Bp}[0]{\mathbf{p}}
\newcommand{\Bq}[0]{\mathbf{q}}
\newcommand{\Br}[0]{\mathbf{r}}
\newcommand{\Bs}[0]{\mathbf{s}}
\newcommand{\Bt}[0]{\mathbf{t}}
\newcommand{\Bu}[0]{\mathbf{u}}
\newcommand{\Bv}[0]{\mathbf{v}}
\newcommand{\Bw}[0]{\mathbf{w}}
\newcommand{\Bx}[0]{\mathbf{x}}
\newcommand{\By}[0]{\mathbf{y}}
\newcommand{\Bz}[0]{\mathbf{z}}
\newcommand{\BA}[0]{\mathbf{A}}
\newcommand{\BB}[0]{\mathbf{B}}
\newcommand{\BC}[0]{\mathbf{C}}
\newcommand{\BD}[0]{\mathbf{D}}
\newcommand{\BE}[0]{\mathbf{E}}
\newcommand{\BF}[0]{\mathbf{F}}
\newcommand{\BG}[0]{\mathbf{G}}
\newcommand{\BH}[0]{\mathbf{H}}
\newcommand{\BI}[0]{\mathbf{I}}
\newcommand{\BJ}[0]{\mathbf{J}}
\newcommand{\BK}[0]{\mathbf{K}}
\newcommand{\BL}[0]{\mathbf{L}}
\newcommand{\BM}[0]{\mathbf{M}}
\newcommand{\BN}[0]{\mathbf{N}}
\newcommand{\BO}[0]{\mathbf{O}}
\newcommand{\BP}[0]{\mathbf{P}}
\newcommand{\BQ}[0]{\mathbf{Q}}
\newcommand{\BR}[0]{\mathbf{R}}
\newcommand{\BS}[0]{\mathbf{S}}
\newcommand{\BT}[0]{\mathbf{T}}
\newcommand{\BU}[0]{\mathbf{U}}
\newcommand{\BV}[0]{\mathbf{V}}
\newcommand{\BW}[0]{\mathbf{W}}
\newcommand{\BX}[0]{\mathbf{X}}
\newcommand{\BY}[0]{\mathbf{Y}}
\newcommand{\BZ}[0]{\mathbf{Z}}

\newcommand{\Bzero}[0]{\mathbf{0}}
\newcommand{\Btheta}[0]{\boldsymbol{\theta}}
\newcommand{\Btau}[0]{\boldsymbol{\tau}}
\newcommand{\Bomega}[0]{\boldsymbol{\omega}}

%
% shorthand for unit vectors
%
\newcommand{\acap}[0]{\hat{\Ba}}
\newcommand{\bcap}[0]{\hat{\Bb}}
\newcommand{\ccap}[0]{\hat{\Bc}}
\newcommand{\dcap}[0]{\hat{\Bd}}
\newcommand{\ecap}[0]{\hat{\Be}}
\newcommand{\fcap}[0]{\hat{\Bf}}
\newcommand{\gcap}[0]{\hat{\Bg}}
\newcommand{\hcap}[0]{\hat{\Bh}}
\newcommand{\icap}[0]{\hat{\Bi}}
\newcommand{\jcap}[0]{\hat{\Bj}}
\newcommand{\kcap}[0]{\hat{\Bk}}
\newcommand{\lcap}[0]{\hat{\Bl}}
\newcommand{\mcap}[0]{\hat{\Bm}}
\newcommand{\ncap}[0]{\hat{\Bn}}
\newcommand{\ocap}[0]{\hat{\Bo}}
\newcommand{\pcap}[0]{\hat{\Bp}}
\newcommand{\qcap}[0]{\hat{\Bq}}
\newcommand{\rcap}[0]{\hat{\Br}}
\newcommand{\scap}[0]{\hat{\Bs}}
\newcommand{\tcap}[0]{\hat{\Bt}}
\newcommand{\ucap}[0]{\hat{\Bu}}
\newcommand{\vcap}[0]{\hat{\Bv}}
\newcommand{\wcap}[0]{\hat{\Bw}}
\newcommand{\xcap}[0]{\hat{\Bx}}
\newcommand{\ycap}[0]{\hat{\By}}
\newcommand{\zcap}[0]{\hat{\Bz}}
\newcommand{\thetacap}[0]{\hat{\Btheta}}

%
% to write R^n and C^n in a distinguishable fashion.  Perhaps change this
% to the double lined characters upon figuring out how to do so.
%
\newcommand{\C}[1]{$\mathbb{C}^{#1}$}
\newcommand{\R}[1]{$\mathbb{R}^{#1}$}

%
% various generally useful helpers
%

% derivative of #1 wrt. #2:
\newcommand{\D}[2] {\frac {d#2} {d#1}}

\newcommand{\inv}[1]{\frac{1}{#1}}
\newcommand{\cross}[0]{\times}

\newcommand{\abs}[1]{\lvert{#1}\rvert}
\newcommand{\norm}[1]{\lVert{#1}\rVert}
\newcommand{\innerprod}[2]{\langle{#1}, {#2}\rangle}
\newcommand{\dotprod}[2]{{#1} \cdot {#2}}
\newcommand{\bdotprod}[2]{\left({#1} \cdot {#2}\right)}
\newcommand{\crossprod}[2]{{#1} \cross {#2}}
\newcommand{\tripleprod}[3]{\dotprod{\left(\crossprod{#1}{#2}\right)}{#3}}

\DeclareMathOperator{\Proj}{Proj}
\DeclareMathOperator{\Span}{span}
\DeclareMathOperator{\Sgn}{sgn}
\DeclareMathOperator{\Area}{Area}
\DeclareMathOperator{\Volume}{Volume}

%
% A few miscellaneous things specific to this document
%
\newcommand{\crossop}[1]{\crossprod{#1}{}}

% R2 vector.
\newcommand{\VectorTwo}[2]{
\begin{bmatrix}
 {#1} \\
 {#2}
\end{bmatrix}
}

\newcommand{\VectorN}[1]{
\begin{bmatrix}
{#1}_1 \\
{#1}_2 \\
\vdots \\
{#1}_N \\
\end{bmatrix}
}

\newcommand{\DETuvij}[4]{
\begin{vmatrix}
 {#1}_{#3} & {#1}_{#4} \\
 {#2}_{#3} & {#2}_{#4}
\end{vmatrix}
}

\newcommand{\DETuvwijk}[6]{
\begin{vmatrix}
 {#1}_{#4} & {#1}_{#5} & {#1}_{#6} \\
 {#2}_{#4} & {#2}_{#5} & {#2}_{#6} \\
 {#3}_{#4} & {#3}_{#5} & {#3}_{#6}
\end{vmatrix}
}

\newcommand{\DETuvwxijkl}[8]{
\begin{vmatrix}
 {#1}_{#5} & {#1}_{#6} & {#1}_{#7} & {#1}_{#8} \\
 {#2}_{#5} & {#2}_{#6} & {#2}_{#7} & {#2}_{#8} \\
 {#3}_{#5} & {#3}_{#6} & {#3}_{#7} & {#3}_{#8} \\
 {#4}_{#5} & {#4}_{#6} & {#4}_{#7} & {#4}_{#8} \\
\end{vmatrix}
}

%\newcommand{\DETuvwxyijklm}[10]{
%\begin{vmatrix}
% {#1}_{#6} & {#1}_{#7} & {#1}_{#8} & {#1}_{#9} & {#1}_{#10} \\
% {#2}_{#6} & {#2}_{#7} & {#2}_{#8} & {#2}_{#9} & {#2}_{#10} \\
% {#3}_{#6} & {#3}_{#7} & {#3}_{#8} & {#3}_{#9} & {#3}_{#10} \\
% {#4}_{#6} & {#4}_{#7} & {#4}_{#8} & {#4}_{#9} & {#4}_{#10} \\
% {#5}_{#6} & {#5}_{#7} & {#5}_{#8} & {#5}_{#9} & {#5}_{#10}
%\end{vmatrix}
%}

% R3 vector.
\newcommand{\VectorThree}[3]{
\begin{bmatrix}
 {#1} \\
 {#2} \\
 {#3}
\end{bmatrix}
}



\author{Peeter Joot}
\email{peeter.joot@gmail.com}


\chapter{Errata for Feynman's Quantum Electrodynamics (Addison-Wesley)?}
\label{chap:feynmanQEDerrata}
%\useCCL
\blogpage{http://sites.google.com/site/peeterjoot/math2010/feynmanQEDerrata.pdf}
\date{May 28, 2010}
\revisionInfo{feynmanQEDerrata.tex}

\beginArtWithToc
%\beginArtNoToc

\section{Motivation.}

I got a nice present today which included one of \href{http://www.amazon.com/Quantum-Electrodynamics-Advanced-Book-Classics/dp/0201360756/ref=sr_1_1?ie=UTF8&s=books&qid=1275092228&sr=8-1}{Feynman's QED books} (Addison-Wesley Feb 98 first printing).  I noticed some early mistakes, and since I can't find an errata page anywhere, I'll collect them here, along with some other notes.

Eventually, if I get through the book, I'll see about sending this into the publisher.
% The original editor (a prof emeritus): David Pines <david.pines@gmail.com>
% knows of no official errata, and suggested a facebook site to allow people to 
% collaboratively collect notes on errors.

\section{On what I believe should be in the errata if it existed.}
\subsection{Third Lecture}
\subsubsection{Page 6.}

The electric field is given in terms of only the scalar potential
\begin{align*}
\BE = -\spacegrad \phi + \partial \phi/ \partial t,
\end{align*}

and should be
\begin{align*}
\BE = -\spacegrad \phi - \inv{c} \partial \BA/ \partial t.
\end{align*}

The invariant gauge transformation for the vector and scalar potentials are given as

\begin{align*}
\BA' &= \BA + \spacegrad \chi \\
\phi' &= \phi + \partial \chi / \partial t
\end{align*}

But these should be
\begin{align*}
\BA' &= \BA + \spacegrad \chi \\
\phi' &= \phi - \inv{c} \partial \chi / \partial t
\end{align*}

The sign was crossed on the scalar potential transformation.  Perhaps Feynman used $c=1$ in his lectures, and whoever made the notes wasn't consistent about including these in all the right places (but did so in some).

\subsubsection{Page 7.}

With the signs and constant terms of the gauge transformation for the potentials being off, so is the end result for the final set of transformations that leave the Pauli equation invariant.  That should be:

\begin{align*}
\BA' &= \BA + \spacegrad \chi \\
\phi' &= \phi - \inv{c} \PD{t}{ \chi } \\
\Psi' &= \exp\left( i \frac{e}{ \hbar c } \chi \right) \Psi,
\end{align*}

(with the intermediate steps corrected accordingly).

\subsubsection{Page 8.}

It's written

\begin{align*}
H = \inv{2m} \left( \Bp - \frac{e}{c}\BA \right)^2 - \frac{e \hbar}{2 m c} (\Bsigma \cdot \spacegrad \cross \BA) + e V
\end{align*}

It appears that the minus should be a positive here.

\begin{align*}
H = \inv{2m} \left( \Bp - \frac{e}{c}\BA \right)^2 + \frac{e \hbar}{2 m c} (\Bsigma \cdot \spacegrad \cross \BA) + e V
\end{align*}

It also appears that $\Bx^2 \equiv (\Bx \cdot \Bx) I$, where the identity matrix $I$ is implied.

Then, equation (1) which reads 

\begin{align*}
\spacegrad \cross \BA = \BK \cross \Be e^{i \BK \cdot \Bx} e^{i \omega t}
\end{align*}

should be
\begin{align*}
\spacegrad \cross \BA 
&= a \BK \cross \Be e^{i \BK \cdot \Bx} e^{-i \omega t} 
- a e^{i \BK \cdot \Bx} e^{-i \omega t} \Be \cross \spacegrad \\
&= \BA \cross (i \BK - \spacegrad )
\end{align*}

And in equation two the sign is wrong.  It reads

\begin{align*}
\Bp e^{i \BK \cdot \Bx} = e^{i \BK \cdot \Bx} (\Bp - \hbar \BK)
\end{align*}

but should be
\begin{align*}
\Bp e^{i \BK \cdot \Bx} = e^{i \BK \cdot \Bx} (\Bp + \hbar \BK)
\end{align*}

similarly the following $-\hbar \BK \cdot \Be$ should be positive, $\hbar \BK \cdot \Be$.  (this last has no effect since $\BK \cdot \Be$ is assumed zero since $\Be$ was picked as the transverse propagation direction for the electrodynamic wave).

\subsection{Seventh Lecture}
\subsubsection{Page 25.}

Last equation reads

\begin{align*}
E t - p_x x - p_y y - p_z z = p_\mu p_\mu
\end{align*}

should be

\begin{align*}
E t - p_x x - p_y y - p_z z = p_\mu x_\mu
\end{align*}

\subsubsection{Page 26.}

After ``but'' we have
\begin{align*}
p_0^2 = E^2 - m
\end{align*}

which should be

\begin{align*}
p_0^2 = E^2 - m^2
\end{align*}

\subsubsection{Page 29.}

The gauge transformation once again has the sign messed up.  It was written (from ${A_\mu}' = A_\mu + \nabla_\mu \chi$)

\begin{align*}
\BA' &= \BA + \spacegrad \chi \\
\phi' &= \phi + {\partial \chi}/{\partial t}
\end{align*}

but it should be

\begin{align*}
\BA' &= \BA - \spacegrad \chi \\
\phi' &= \phi + {\partial \chi}/{\partial t}
\end{align*}

(ie: $\nabla_m = -\partial_m$)

Then a bit later

\begin{align*}
\grad \cdot A' = \grad \cdot A + \grad \cdot \chi
\end{align*}

should be

\begin{align*}
\grad \cdot A' = \grad \cdot A + \grad \cdot \grad \chi
\end{align*}

\subsubsection{Page 29.}

\begin{align*}
dx/ds = (dx/dt)(dt/ds) = v_x/(1-y^2)^{1/2}
\end{align*}

should be
\begin{align*}
dx/ds = (dx/dt)(dt/ds) = v_x/(1-v^2)^{1/2}
\end{align*}

\section{Extended notes.}

\subsection{Second Lecture}

This isn't errata, but I found the following required slight exploration.  He gives (implicitly)

\begin{align*}
\overline{\sin^2(\omega t - \BK \cdot \Bx)} = \inv{2}
\end{align*}

Is this an average over space and time?  How would one do that?  What do we get just integrating this over the volume?  That dot product is $\BK \cdot \Bx = 2 \pi \left(\frac{m}{\lambda_1} x + \frac{n}{\lambda_2} y + \frac{o}{\lambda_3} z \right)$.  Our average over the volume, for $m \ne 0$, using \href{http://www.wolframalpha.com/input/?i=\int+sin^2(a+x+%2B+b)+dx}{wolfram alpha to do the dirty work}, is then

\begin{align*}
&\inv{\lambda_1 \lambda_2 \lambda_3} 
\int_{z=0}^{\lambda_3} dz
\int_{y=0}^{\lambda_2} dy
\int_{x=0}^{\lambda_1}
dx \sin^2 \left( 
-\frac{2 \pi m x}{\lambda_1} 
-\frac{2 \pi n y}{\lambda_2} 
-\frac{2 \pi o z}{\lambda_3} 
+ \omega t \right) \\
&=
\inv{\lambda_1 \lambda_2 \lambda_3} 
\int_{z=0}^{\lambda_3} dz
\int_{y=0}^{\lambda_2} dy
{\left.
\frac{-\lambda_1}{4 \pi m} \left( 
-\frac{2 \pi m }{\lambda_1} x 
-\frac{2 \pi n y}{\lambda_2} 
-\frac{2 \pi o z}{\lambda_3} 
+ \omega t \right)
\right\vert}_{x=0}^{\lambda_1} \\
&-
\inv{\lambda_1 \lambda_2 \lambda_3} 
\int_{z=0}^{\lambda_3} dz
\int_{y=0}^{\lambda_2} dy
{\left.
\frac{-\lambda_1}{8 \pi m} 
\sin \left( 2 \left(
-\frac{2 \pi m }{\lambda_1} x 
-\frac{2 \pi n y}{\lambda_2} 
-\frac{2 \pi o z}{\lambda_3} 
+ \omega t \right) \right)
\right\vert}_{x=0}^{\lambda_1}
\end{align*}

Since the sine integral vanishes, we have just $1/2$ as expected regardless of the angular frequency $\omega$.  Okay, that makes sense now.  Looks like $\omega$ is only relevant for the single $\BK = 0$ Fourier component, but that likely doesn't matter since I seem to recall that the $\BK = 0$ Fourier component of this oscillators in a box problem was entirely constant (and perhaps zero?).

\subsection{Third Lecture.  Page 7 notes.}

The units in the transformation for the wave function don't look right.  We want to transform the Pauli equation

\begin{align*}
i \hbar \PD{t}{\Psi} = \inv{2 m} \left( \Bp - \frac{e}{c} \BA \right)^2 \Psi + e \phi \Psi,
\end{align*}

with a transformation of the form
\begin{align*}
\BA' &= \BA + \spacegrad \chi \\
\phi' &= \phi - \inv{c} \PD{t}{\chi} \\
\Psi' &= e^{-i \mu} \Psi,
\end{align*}

Where $\mu \propto \chi$ is presumed, and we want to find the proportionality constant required for invariance.  With $\Bp = - i \hbar \spacegrad$ we have

\begin{align*}
\Bp \Psi' 
&=
-i \hbar \spacegrad e^{-i \mu} \Psi \\
&=
-i \hbar \left( 
-i (\spacegrad \mu) e^{-i \mu} \Psi 
+ e^{-i \mu} \spacegrad \Psi  
\right) \\
&=
+ e^{-i \mu} \left( \Bp + \hbar \spacegrad \mu \right) \Psi,
\end{align*}

so
\begin{align*}
(\Bp -\frac{e}{c} \BA' )\Psi' 
&=
e^{-i \mu} \left( \Bp - \frac{e}{c} \BA - \spacegrad (\hbar \mu + \frac{e}{c} \chi) \right) \Psi.
\end{align*}

For the time partial we have

\begin{align*}
\PD{t}{\Psi'} &= e^{-i \mu} \PD{t}{\Psi} -i \PD{t}{\mu} e^{-i \mu} \Psi,
\end{align*}

and the scalar potential term transforms as
\begin{align*}
e \phi' \Psi'
&=
e \left( \phi - \inv{c} \PD{t}{\chi} \right) e^{-i \mu } \Psi
\end{align*}

Putting the pieces together we have

\begin{align*}
i \hbar e^{-i \mu}
\left( \PD{t}{} -i \PD{t}{\mu} \right) \Psi 
&=
\inv{2m}
\left(\Bp -\frac{e}{c} \BA -\frac{e}{c} \spacegrad \chi \right)
e^{-i \mu} \left( \Bp - \frac{e}{c} \BA - \spacegrad (\hbar \mu + \frac{e}{c} \chi) \right) \Psi 
+ e \left( \phi - \inv{c} \PD{t}{\chi} \right) e^{-i \mu } \Psi
\end{align*}

We need one more intermediate result, that of

\begin{align*}
\Bp e^{-i \mu } \BD
&= 
- i \hbar e^{-i \mu} \left( -i (\spacegrad \mu) + \spacegrad \right) \BD \\
&= 
e^{-i\mu} (\Bp - \hbar \spacegrad \mu) \BD.
\end{align*}

So we have
\begin{align*}
i \hbar \PD{t}{\Psi}
+\hbar \PD{t}{\mu} \Psi 
&=
\inv{2m}
\left(\Bp - \hbar \spacegrad \mu -\frac{e}{c} \BA -\frac{e}{c} \spacegrad \chi \right)
\left( \Bp - \frac{e}{c} \BA - \spacegrad (\hbar \mu + \frac{e}{c} \chi) \right) \Psi 
+ e \left( \phi - \inv{c} \PD{t}{\chi} \right) \Psi.
\end{align*}

To get rid of the $\mu$, and $\chi$ time partials we need

\begin{align*}
\hbar \PD{t}{\mu} = - \frac{e}{c} \PD{t}{\chi}
\end{align*}

Or 
\begin{align*}
\mu = - \frac{e}{c\hbar} \chi
\end{align*}

This also kills off all the additional undesirable terms in the transformed $\BP^2$ operator (with $\BP = \Bp - e \BA/c$), leaving the invariant transformation completely specified

\begin{align*}
\BA' &= \BA + \spacegrad \chi \\
\phi' &= \phi - \inv{c} \PD{t}{ \chi } \\
\Psi' &= \exp\left( i \frac{e}{ \hbar c } \chi \right) \Psi,
\end{align*}

This is a fair bit different than the final result as noted in the text, but since that starts with the wrong electrodynamic gauge transformation, this is not too unexpected.

\subsection{Third Lecture.  Page 8 notes.}

Here we have

\begin{align*}
H = \inv{2m} \left( \Bp - \frac{e}{c}\BA \right)^2 - \frac{e \hbar}{2 m c} (\Bsigma \cdot \spacegrad \cross \BA) + e V
\end{align*}

whereas previously it was

\begin{align*}
i \hbar \PD{t}{\Psi} = \inv{2m} 
\left[\sigma \cdot \left( \Bp - \frac{e}{c} \BA \right)\right]
\left[\sigma \cdot \left( \Bp - \frac{e}{c} \BA \right)\right]
 \Psi + e \phi \Psi.
\end{align*}

What is this $[\sigma \cdot \Bx]$ notation?  In \citep{wiki:pauli} we have

\begin{align*}
\Ba \cdot \Bsigma &= a_i \sigma_i,
\end{align*}

Within these square braces it appears that this product is intended to be a tensor product, like so

\begin{align*}
\left[\sigma \cdot \Ba\right]
\left[\sigma \cdot \Bb\right] 
&\questionEquals \sum_{i,j} a_i \sigma_i b_j \sigma_j \\
&= (\Ba \cdot \Bb) I + i \sigma \cdot (\Ba \cross \Bb).
\end{align*}

For $H$ this would be

\begin{align*}
i \hbar \PD{t}{\Psi} 
&= \inv{2m} 
\left( \Bp - \frac{e}{c} \BA \right) \cdot \left( \Bp - \frac{e}{c} \BA \right)
- \frac{i^2 \hbar e}{2m c} \Bsigma \cdot (\spacegrad \cross \BA) \Psi + e \phi \Psi \\
&= \inv{2m} 
\left( \Bp - \frac{e}{c} \BA \right) \cdot \left( \Bp - \frac{e}{c} \BA \right)
+ \frac{\hbar e}{2m c} \Bsigma \cdot (\spacegrad \cross \BA) \Psi + e \phi \Psi.
\end{align*}

Ah.  The $i$ in $\Bp = -i \hbar \spacegrad$ is what does away with the $i$ in the Pauli matrix product.  However, there does appear to be a sign error.

Instead of guessing what Feynman means when he writes Pauli's equation, it would be better to just check what Pauli says.  In 

Now, how does one reconcile this with Pauli's text \citep{pauli2000wm} he writes

\begin{align*}
H = \inv{2m} \sum_{k=1}^3 \left( p_k - \frac{e}{c}A_k \right)^2 + e \phi V.
\end{align*}

There is no $\spacegrad \cross \BA$ operator term in Pauli's own text, just the scalar operator?

%\section{Followup.}
%
%On \href{http://www.feynmanlectures.info/}{In the errata section}, they say ``We invite you to contact us with contributions of errata, including: ... links to
%other lists of errata for works by Richard Feynman''.
%
%Above are what I believe to be typo notes for Feynman's "Quantum Electrodynamics" (Addison-Wesley), the Feb 98 first printing.
%
%However, some of these notes were elaboration for myself and aren't entirely restricted to the observed typos.
%
%Since I was unable to find an errata text for this book, I sent a link to these notes as indicated.  Was given the following advice:
%
%``Unfortunately I don't own a copy of Quantum Electrodynamics, so I am unable to check your errata. However, I do know that the book has about 200 pages, from which I can guess that your list is far from complete. I therefore suggest that you continue to keep notes as you read. When you have a (more or less) complete set of errata for the entire book (with errors listed in lexical order, and all "elaborations" removed) you should submit it to your peers for review, and then to the publisher \href{http://www.perseusbooksgroup.com/westview/book_detail.jsp?isbn=0201360756}{Perseus Westview Press}) and/or to the editor of the book. After your list of errata has been completed, reviewed and edited as needed, we would be glad to post a link to it on The Feynman Lectures website.''
%
%I'll continue to update these latex typo notes, and eventually, presuming I finish working my way through this text resubmit them.

\EndArticle
%\EndNoBibArticle


%\part{Sort}

\part{Cronology}
\chapter{Cronological Index}
\begin{itemize}

\item October 13, 2007 \ref{chap:gaWiki} Comparison of many traditional vector and GA identities

\item October 13, 2007 \ref{chap:gaWikiTorque} Torque

\item October 16, 2007 \ref{chap:PJUnitDer} Derivatives of a unit vector

\item October 16, 2007 \ref{chap:gaWikiCramers} Cramer's rule

\item October 22, 2007 \ref{chap:PJRadialDer} Radial components of vector derivatives

\item January 1, 2008 \ref{chap:plane} More details on NFCM plane formulation

\item January 29, 2008 \ref{chap:PJAngVel} Rotational dynamics

\item January 29, 2008 \ref{chap:maxwellsGa} Maxwell's equations expressed with Geometric Algebra

\item February 2, 2008 \ref{chap:quaternion} Quaternions

\item February 4, 2008 \ref{chap:legendre} Legendre Polynomials

\item February 15, 2008 \ref{chap:inertialTensor} Inertia Tensor

\item February 19, 2008 \ref{chap:rotor} Rotor Notes

\item February 28, 2008 \ref{chap:laplace} Exponential Solutions to Laplace Equation in \R{N}

\item March 9, 2008 \ref{chap:bivector} Bivector Geometry

\item March 9, 2008 \ref{chap:trivector} Trivector geometry

\item March 12, 2008 \ref{chap:kvectorExponential} Exponential of a blade

\item March 16, 2008 \ref{chap:scalarCommutes} Multivector product grade zero terms

\item March 17, 2008 \ref{chap:angleBetweenLineAndPlane} Angle between geometric elements

\item March 17, 2008 \ref{chap:gaGradeDotWedge} An earlier attempt to intuitively introduce the dot, wedge, cross, and geometric products

\item March 25, 2008 \ref{chap:bladegradereduction} Blade grade reduction

\item March 29, 2008 \ref{chap:reciprocalFrame} Reciprocal Frame Vectors

\item March 31, 2008 \ref{chap:gradientAndForms} Exterior derivative and chain rule components of the gradient

\item April 1, 2008 \ref{chap:orthodecomp} Orthogonal decomposition take II

\item April 11, 2008 \ref{chap:matrixReview} Matrix review

\item April 13, 2008 \ref{chap:locateSatellite} Satellite triangulation over sphere

\item April 30, 2008 \ref{chap:PJKeRot} Kinetic Energy in rotational frame

\item May 7, 2008 \ref{chap:lorentzRotation} Lorentz Force Trajectory

\item May 16, 2008 \ref{chap:obliqueProj} Oblique projection and reciprocal frame vectors

\item May 16, 2008 \ref{chap:matrixOfLinearTx} Matrix of grade k multivector linear transformations

\item May 16, 2008 \ref{chap:projectionAndMoorePenroseVectorInverse} Projection and Moore-Penrose vector inverse

\item May 17, 2008 \ref{chap:PJprojGen} Projection with generalized dot product

\item June 6, 2008 \ref{chap:tensor} Gradient and tensor notes

\item June 10, 2008 \ref{chap:PJAngAcc} Angular Velocity and Acceleration.  Again

\item June 25, 2008 \ref{chap:lorentz} Wave equation based Lorentz transformation derivation

\item July 8, 2008 \ref{chap:PJAngAccCross} Cross product Radial decomposition

\item July 12, 2008 \ref{chap:PJMaxwell2} Back to Maxwell's equations

\item July 16, 2008 \ref{chap:spacetimegrad} Lorentz transformation of spacetime gradient

\item July 20, 2008 \ref{chap:sgMx41} Magnetic field between two parallel wires

\item August 1, 2008 \ref{chap:fourvecDotinvariance} Four vector dot product invariance and Lorentz rotors

\item August 9, 2008 \ref{chap:newtonianLagrangianAndGradient} Newton's Law from Lagrangian

\item August 13, 2008 \ref{chap:cauchyGradient} Cauchy Equations expressed as a gradient

\item August 13, 2008 \ref{chap:velocityTx} Understanding four velocity transform from rest frame

\item August 15, 2008 \ref{chap:emPotential} Four vector potential

\item August 16, 2008 \ref{chap:PJSrGAFPLorentzForce} Lorentz force Law

\item August 21, 2008 \ref{chap:PJSrLagrangian} Covariant Lagrangian, and electrodynamic potential

\item August 25, 2008 \ref{chap:PJTongMf1} Solutions to David Tong's mf1 Lagrangian problems

\item August 28, 2008 \ref{chap:massVaryLagrangian} Equations of motion given mass variation with spacetime position

\item September 1, 2008 \ref{chap:PJCanMomentum} Vector canonical momentum

\item September 2, 2008 \ref{chap:outermorphismDet} OuterMorphism Question 

\item September 5, 2008 \ref{chap:emBivectorMetricDependencies} Metric signature dependencies

\item September 7, 2008 \ref{chap:PJMaxwellTensor} Tensor relations from bivector field equation

\item September 8, 2008 \ref{chap:PJMaxwellLagrangian} Direct variation of Maxwell equations

\item September 9, 2008 \ref{chap:PJMaxwellProj} Vector forms of Maxwell's equations as projection and rejection operations

\item September 18, 2008 \ref{chap:PJStokes1} Stokes law in wedge product form

\item September 26, 2008 \ref{chap:stokesMaxwellApplication} Application of Stokes Integrals to Maxwell's Equation

\item September 27, 2008 \ref{chap:PJStokes2} Stokes Law revisited with algebraic enumeration of boundary

\item October 8, 2008 \ref{chap:PJSrLorentzForce} Revisit Lorentz force from Lagrangian

\item October 10, 2008 \ref{chap:PJFieldLagrangian} Derivation of Euler-Lagrange field equations

\item October 12, 2008 \ref{chap:maxwellTensorLagrangian} Tensor Derivation of Covariant Lorentz Force from Lagrangian

\item October 13, 2008 \ref{chap:PJEulerLagrange} Euler Lagrange Equations

\item October 19, 2008 \ref{chap:PJBoostMaxwell} Lorentz Invariance of Maxwell Lagrangian

\item October 22, 2008 \ref{chap:PJLorentzTxInteraction} Lorentz transform Noether current for interaction Lagrangian

\item October 26, 2008 \ref{chap:gem} GravitoElectroMagnetism

\item October 29, 2008 \ref{chap:PJNoethersField} Field form of Noether's Law

\item November 1, 2008 \ref{chap:eulerangle} Euler Angle Notes

\item November 8, 2008 \ref{chap:complex} Hyper complex numbers and symplectic structure

\item November 13, 2008 \ref{chap:sphericalPolar} Spherical polar coordinates

\item November 22, 2008 \ref{chap:gaussianSurface} Gaussian Surface invariance for radial field

\item November 23, 2008 \ref{chap:chargeArcElement} Field due to line charge in arc

\item November 23, 2008 \ref{chap:chargeLineElement} Charge line element

\item November 27, 2008 \ref{chap:nfcmCh2} Some NFCM exercise solutions and notes

\item November 30, 2008 \ref{chap:PJwaveFourVector} Expressing wave equation exponential solutions using four vectors

\item November 30, 2008 \ref{chap:slerp} Rotor interpolation calculation

\item December 6, 2008 \ref{chap:pauliMatrix} Pauli Matrixes in Clifford Algebra

\item December 11, 2008 \ref{chap:bohr} Bohr Model

\item December 13, 2008 \ref{chap:PJDiracGamma} Gamma Matrices

\item December 21, 2008 \ref{chap:diracLagrangian} Dirac Lagrangian

\item December 27, 2008 \ref{chap:PJrayleighJeans} Rayleigh-Jeans Law Notes

\item December 29, 2008 \ref{chap:PJpoynting} Poynting vector and Electromagnetic Energy conservation

\item January 1, 2009 \ref{chap:PJemstresstensor} Energy momentum tensor

\item January 3, 2009 \ref{chap:PJelectricFieldEnergy} Field and wave energy and momentum

\item January 5, 2009 \ref{chap:vectorDifferentialIdentities} Vector Differential Identities

\item January 6, 2009 \ref{chap:dcPower} DC Power consumption formula for resistive load

\item January 9, 2009 \ref{chap:PJqmFourier} Some Fourier transform notes

\item January 11, 2009 \ref{chap:schCurrent} Schr\"{o}dinger equation probability conservation

\item January 13, 2009 \ref{chap:radial} Polar velocity and acceleration

\item January 18, 2009 \ref{chap:PJpoyntingRate} Time rate of change of the Poynting vector, and its conservation law

\item January 19, 2009 \ref{chap:PJheatFourier} Fourier Solutions to Heat and Wave equations

\item January 21, 2009 \ref{chap:fourierNotation} A cheatsheet for Fourier transform conventions

\item January 25, 2009 \ref{chap:PJemWave} Electrodynamic wave equation solutions

\item January 26, 2009 \ref{chap:PJwaveFourier} Fourier transform solutions to the wave equation

\item January 29, 2009 \ref{chap:PJfourierMaxwellSecondOrder} Fourier transform solutions to Maxwell's equation

\item January 31, 2009 \ref{chap:PJfirstOrderMaxwell} First order Fourier transform solution of Maxwell's equation

\item February 1, 2009 \ref{chap:PJ4dFourier} 4D Fourier transforms applied to Maxwell's equation

\item February 3, 2009 \ref{chap:PJFourierVacuum} Fourier series Vacuum Maxwell's equations

\item February 7, 2009 \ref{chap:potentialFourier} Lorentz Gauge Fourier Vacuum potential solutions

\item February 8, 2009 \ref{chap:PJplaneWave} Plane wave Fourier series solutions to the Maxwell vacuum equation

\item February 13, 2009 \ref{chap:PJstressEnergyLorentz} Lorentz force relation to the energy momentum tensor

\item February 17, 2009 \ref{chap:en_m_tensor} Energy momentum tensor relation to Lorentz force

\item February 18, 2009 \ref{chap:PJpoisson} Poisson and retarded Potential Green's functions from Fourier kernels

\item February 26, 2009 \ref{chap:nvolume} Spherical and hyperspherical parametrization

\item March 13, 2009 \ref{chap:levi} Levi-Civitica summation identity

\item March 18, 2009 \ref{chap:electronRotor} Lorentz force rotor formulation

\item April 15, 2009 \ref{chap:lorentzForcePQA} Lorentz force Lagrangian with conjugate momentum

\item April 18, 2009 \ref{chap:biotSavart} Biot Savart Derivation

\item April 20, 2009 \ref{chap:maxwellTensorFromLagrangian} Tensor derivation of non-dual Maxwell equation from Lagrangian

\item April 28, 2009 \ref{chap:PJmultiTaylors} Developing some intuition for Multivariable and Multivector Taylor Series

\item May 23, 2009 \ref{chap:lorentzForceTx} Lorentz boost of Lorentz force equations

\item May 28, 2009 \ref{chap:macroscopicMaxwell} Macroscopic Maxwell's equation

\item June 1, 2009 \ref{chap:poincareTx} Poincare transformations

\item June 5, 2009 \ref{chap:stressEnergyNoethers} Canonical energy momentum tensor and Lagrangian translation

\item June 17, 2009 \ref{chap:lForceLag2} Comparison of two covariant Lorentz force Lagrangians

\item June 21, 2009 \ref{chap:emVacWave} Wave equation form of Maxwell's equations

\item June 27, 2009 \ref{chap:frequencyTx} Relativistic Doppler formula

\end{itemize}


% END INCLUDES.
%-------------------------------------------------------

\bibliography{myrefs}
\bibliographystyle{unsrturl}
  \addcontentsline{toc}{chapter}{Bibliography}

\end{document}
