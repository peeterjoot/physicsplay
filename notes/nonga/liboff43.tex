%%
% Copyright � 2015 Peeter Joot.  All Rights Reserved.
% Licenced as described in the file LICENSE under the root directory of this GIT repository.
%
\documentclass[]{eliblog}

\usepackage{amsmath}
\usepackage{mathpazo}

%
% shorthand for bold symbols, convenient for vectors and matrices
%
\newcommand{\Ba}[0]{\mathbf{a}}
\newcommand{\Bb}[0]{\mathbf{b}}
\newcommand{\Bc}[0]{\mathbf{c}}
\newcommand{\Bd}[0]{\mathbf{d}}
\newcommand{\Be}[0]{\mathbf{e}}
\newcommand{\Bf}[0]{\mathbf{f}}
\newcommand{\Bg}[0]{\mathbf{g}}
\newcommand{\Bh}[0]{\mathbf{h}}
\newcommand{\Bi}[0]{\mathbf{i}}
\newcommand{\Bj}[0]{\mathbf{j}}
\newcommand{\Bk}[0]{\mathbf{k}}
\newcommand{\Bl}[0]{\mathbf{l}}
\newcommand{\Bm}[0]{\mathbf{m}}
\newcommand{\Bn}[0]{\mathbf{n}}
\newcommand{\Bo}[0]{\mathbf{o}}
\newcommand{\Bp}[0]{\mathbf{p}}
\newcommand{\Bq}[0]{\mathbf{q}}
\newcommand{\Br}[0]{\mathbf{r}}
\newcommand{\Bs}[0]{\mathbf{s}}
\newcommand{\Bt}[0]{\mathbf{t}}
\newcommand{\Bu}[0]{\mathbf{u}}
\newcommand{\Bv}[0]{\mathbf{v}}
\newcommand{\Bw}[0]{\mathbf{w}}
\newcommand{\Bx}[0]{\mathbf{x}}
\newcommand{\By}[0]{\mathbf{y}}
\newcommand{\Bz}[0]{\mathbf{z}}
\newcommand{\BA}[0]{\mathbf{A}}
\newcommand{\BB}[0]{\mathbf{B}}
\newcommand{\BC}[0]{\mathbf{C}}
\newcommand{\BD}[0]{\mathbf{D}}
\newcommand{\BE}[0]{\mathbf{E}}
\newcommand{\BF}[0]{\mathbf{F}}
\newcommand{\BG}[0]{\mathbf{G}}
\newcommand{\BH}[0]{\mathbf{H}}
\newcommand{\BI}[0]{\mathbf{I}}
\newcommand{\BJ}[0]{\mathbf{J}}
\newcommand{\BK}[0]{\mathbf{K}}
\newcommand{\BL}[0]{\mathbf{L}}
\newcommand{\BM}[0]{\mathbf{M}}
\newcommand{\BN}[0]{\mathbf{N}}
\newcommand{\BO}[0]{\mathbf{O}}
\newcommand{\BP}[0]{\mathbf{P}}
\newcommand{\BQ}[0]{\mathbf{Q}}
\newcommand{\BR}[0]{\mathbf{R}}
\newcommand{\BS}[0]{\mathbf{S}}
\newcommand{\BT}[0]{\mathbf{T}}
\newcommand{\BU}[0]{\mathbf{U}}
\newcommand{\BV}[0]{\mathbf{V}}
\newcommand{\BW}[0]{\mathbf{W}}
\newcommand{\BX}[0]{\mathbf{X}}
\newcommand{\BY}[0]{\mathbf{Y}}
\newcommand{\BZ}[0]{\mathbf{Z}}

\newcommand{\Bzero}[0]{\mathbf{0}}
\newcommand{\Btheta}[0]{\boldsymbol{\theta}}
\newcommand{\Btau}[0]{\boldsymbol{\tau}}
\newcommand{\Bomega}[0]{\boldsymbol{\omega}}

%
% shorthand for unit vectors
%
\newcommand{\acap}[0]{\hat{\Ba}}
\newcommand{\bcap}[0]{\hat{\Bb}}
\newcommand{\ccap}[0]{\hat{\Bc}}
\newcommand{\dcap}[0]{\hat{\Bd}}
\newcommand{\ecap}[0]{\hat{\Be}}
\newcommand{\fcap}[0]{\hat{\Bf}}
\newcommand{\gcap}[0]{\hat{\Bg}}
\newcommand{\hcap}[0]{\hat{\Bh}}
\newcommand{\icap}[0]{\hat{\Bi}}
\newcommand{\jcap}[0]{\hat{\Bj}}
\newcommand{\kcap}[0]{\hat{\Bk}}
\newcommand{\lcap}[0]{\hat{\Bl}}
\newcommand{\mcap}[0]{\hat{\Bm}}
\newcommand{\ncap}[0]{\hat{\Bn}}
\newcommand{\ocap}[0]{\hat{\Bo}}
\newcommand{\pcap}[0]{\hat{\Bp}}
\newcommand{\qcap}[0]{\hat{\Bq}}
\newcommand{\rcap}[0]{\hat{\Br}}
\newcommand{\scap}[0]{\hat{\Bs}}
\newcommand{\tcap}[0]{\hat{\Bt}}
\newcommand{\ucap}[0]{\hat{\Bu}}
\newcommand{\vcap}[0]{\hat{\Bv}}
\newcommand{\wcap}[0]{\hat{\Bw}}
\newcommand{\xcap}[0]{\hat{\Bx}}
\newcommand{\ycap}[0]{\hat{\By}}
\newcommand{\zcap}[0]{\hat{\Bz}}
\newcommand{\thetacap}[0]{\hat{\Btheta}}

%
% to write R^n and C^n in a distinguishable fashion.  Perhaps change this
% to the double lined characters upon figuring out how to do so.
%
\newcommand{\C}[1]{$\mathbb{C}^{#1}$}
\newcommand{\R}[1]{$\mathbb{R}^{#1}$}

%
% various generally useful helpers
%

% derivative of #1 wrt. #2:
\newcommand{\D}[2] {\frac {d#2} {d#1}}

\newcommand{\inv}[1]{\frac{1}{#1}}
\newcommand{\cross}[0]{\times}

\newcommand{\abs}[1]{\lvert{#1}\rvert}
\newcommand{\norm}[1]{\lVert{#1}\rVert}
\newcommand{\innerprod}[2]{\langle{#1}, {#2}\rangle}
\newcommand{\dotprod}[2]{{#1} \cdot {#2}}
\newcommand{\bdotprod}[2]{\left({#1} \cdot {#2}\right)}
\newcommand{\crossprod}[2]{{#1} \cross {#2}}
\newcommand{\tripleprod}[3]{\dotprod{\left(\crossprod{#1}{#2}\right)}{#3}}

\DeclareMathOperator{\Proj}{Proj}
\DeclareMathOperator{\Span}{span}
\DeclareMathOperator{\Sgn}{sgn}
\DeclareMathOperator{\Area}{Area}
\DeclareMathOperator{\Volume}{Volume}

%
% A few miscellaneous things specific to this document
%
\newcommand{\crossop}[1]{\crossprod{#1}{}}

% R2 vector.
\newcommand{\VectorTwo}[2]{
\begin{bmatrix}
 {#1} \\
 {#2}
\end{bmatrix}
}

\newcommand{\VectorN}[1]{
\begin{bmatrix}
{#1}_1 \\
{#1}_2 \\
\vdots \\
{#1}_N \\
\end{bmatrix}
}

\newcommand{\DETuvij}[4]{
\begin{vmatrix}
 {#1}_{#3} & {#1}_{#4} \\
 {#2}_{#3} & {#2}_{#4}
\end{vmatrix}
}

\newcommand{\DETuvwijk}[6]{
\begin{vmatrix}
 {#1}_{#4} & {#1}_{#5} & {#1}_{#6} \\
 {#2}_{#4} & {#2}_{#5} & {#2}_{#6} \\
 {#3}_{#4} & {#3}_{#5} & {#3}_{#6}
\end{vmatrix}
}

\newcommand{\DETuvwxijkl}[8]{
\begin{vmatrix}
 {#1}_{#5} & {#1}_{#6} & {#1}_{#7} & {#1}_{#8} \\
 {#2}_{#5} & {#2}_{#6} & {#2}_{#7} & {#2}_{#8} \\
 {#3}_{#5} & {#3}_{#6} & {#3}_{#7} & {#3}_{#8} \\
 {#4}_{#5} & {#4}_{#6} & {#4}_{#7} & {#4}_{#8} \\
\end{vmatrix}
}

%\newcommand{\DETuvwxyijklm}[10]{
%\begin{vmatrix}
% {#1}_{#6} & {#1}_{#7} & {#1}_{#8} & {#1}_{#9} & {#1}_{#10} \\
% {#2}_{#6} & {#2}_{#7} & {#2}_{#8} & {#2}_{#9} & {#2}_{#10} \\
% {#3}_{#6} & {#3}_{#7} & {#3}_{#8} & {#3}_{#9} & {#3}_{#10} \\
% {#4}_{#6} & {#4}_{#7} & {#4}_{#8} & {#4}_{#9} & {#4}_{#10} \\
% {#5}_{#6} & {#5}_{#7} & {#5}_{#8} & {#5}_{#9} & {#5}_{#10}
%\end{vmatrix}
%}

% R3 vector.
\newcommand{\VectorThree}[3]{
\begin{bmatrix}
 {#1} \\
 {#2} \\
 {#3}
\end{bmatrix}
}



\author{Peeter Joot}
\email{peeter.joot@gmail.com}

%\documentclass[]{eliblogwidescreen}

\usepackage{amsmath}
\usepackage{mathpazo}

%
% shorthand for bold symbols, convenient for vectors and matrices
%
\newcommand{\Ba}[0]{\mathbf{a}}
\newcommand{\Bb}[0]{\mathbf{b}}
\newcommand{\Bc}[0]{\mathbf{c}}
\newcommand{\Bd}[0]{\mathbf{d}}
\newcommand{\Be}[0]{\mathbf{e}}
\newcommand{\Bf}[0]{\mathbf{f}}
\newcommand{\Bg}[0]{\mathbf{g}}
\newcommand{\Bh}[0]{\mathbf{h}}
\newcommand{\Bi}[0]{\mathbf{i}}
\newcommand{\Bj}[0]{\mathbf{j}}
\newcommand{\Bk}[0]{\mathbf{k}}
\newcommand{\Bl}[0]{\mathbf{l}}
\newcommand{\Bm}[0]{\mathbf{m}}
\newcommand{\Bn}[0]{\mathbf{n}}
\newcommand{\Bo}[0]{\mathbf{o}}
\newcommand{\Bp}[0]{\mathbf{p}}
\newcommand{\Bq}[0]{\mathbf{q}}
\newcommand{\Br}[0]{\mathbf{r}}
\newcommand{\Bs}[0]{\mathbf{s}}
\newcommand{\Bt}[0]{\mathbf{t}}
\newcommand{\Bu}[0]{\mathbf{u}}
\newcommand{\Bv}[0]{\mathbf{v}}
\newcommand{\Bw}[0]{\mathbf{w}}
\newcommand{\Bx}[0]{\mathbf{x}}
\newcommand{\By}[0]{\mathbf{y}}
\newcommand{\Bz}[0]{\mathbf{z}}
\newcommand{\BA}[0]{\mathbf{A}}
\newcommand{\BB}[0]{\mathbf{B}}
\newcommand{\BC}[0]{\mathbf{C}}
\newcommand{\BD}[0]{\mathbf{D}}
\newcommand{\BE}[0]{\mathbf{E}}
\newcommand{\BF}[0]{\mathbf{F}}
\newcommand{\BG}[0]{\mathbf{G}}
\newcommand{\BH}[0]{\mathbf{H}}
\newcommand{\BI}[0]{\mathbf{I}}
\newcommand{\BJ}[0]{\mathbf{J}}
\newcommand{\BK}[0]{\mathbf{K}}
\newcommand{\BL}[0]{\mathbf{L}}
\newcommand{\BM}[0]{\mathbf{M}}
\newcommand{\BN}[0]{\mathbf{N}}
\newcommand{\BO}[0]{\mathbf{O}}
\newcommand{\BP}[0]{\mathbf{P}}
\newcommand{\BQ}[0]{\mathbf{Q}}
\newcommand{\BR}[0]{\mathbf{R}}
\newcommand{\BS}[0]{\mathbf{S}}
\newcommand{\BT}[0]{\mathbf{T}}
\newcommand{\BU}[0]{\mathbf{U}}
\newcommand{\BV}[0]{\mathbf{V}}
\newcommand{\BW}[0]{\mathbf{W}}
\newcommand{\BX}[0]{\mathbf{X}}
\newcommand{\BY}[0]{\mathbf{Y}}
\newcommand{\BZ}[0]{\mathbf{Z}}

\newcommand{\Bzero}[0]{\mathbf{0}}
\newcommand{\Btheta}[0]{\boldsymbol{\theta}}
\newcommand{\Btau}[0]{\boldsymbol{\tau}}
\newcommand{\Bomega}[0]{\boldsymbol{\omega}}

%
% shorthand for unit vectors
%
\newcommand{\acap}[0]{\hat{\Ba}}
\newcommand{\bcap}[0]{\hat{\Bb}}
\newcommand{\ccap}[0]{\hat{\Bc}}
\newcommand{\dcap}[0]{\hat{\Bd}}
\newcommand{\ecap}[0]{\hat{\Be}}
\newcommand{\fcap}[0]{\hat{\Bf}}
\newcommand{\gcap}[0]{\hat{\Bg}}
\newcommand{\hcap}[0]{\hat{\Bh}}
\newcommand{\icap}[0]{\hat{\Bi}}
\newcommand{\jcap}[0]{\hat{\Bj}}
\newcommand{\kcap}[0]{\hat{\Bk}}
\newcommand{\lcap}[0]{\hat{\Bl}}
\newcommand{\mcap}[0]{\hat{\Bm}}
\newcommand{\ncap}[0]{\hat{\Bn}}
\newcommand{\ocap}[0]{\hat{\Bo}}
\newcommand{\pcap}[0]{\hat{\Bp}}
\newcommand{\qcap}[0]{\hat{\Bq}}
\newcommand{\rcap}[0]{\hat{\Br}}
\newcommand{\scap}[0]{\hat{\Bs}}
\newcommand{\tcap}[0]{\hat{\Bt}}
\newcommand{\ucap}[0]{\hat{\Bu}}
\newcommand{\vcap}[0]{\hat{\Bv}}
\newcommand{\wcap}[0]{\hat{\Bw}}
\newcommand{\xcap}[0]{\hat{\Bx}}
\newcommand{\ycap}[0]{\hat{\By}}
\newcommand{\zcap}[0]{\hat{\Bz}}
\newcommand{\thetacap}[0]{\hat{\Btheta}}

%
% to write R^n and C^n in a distinguishable fashion.  Perhaps change this
% to the double lined characters upon figuring out how to do so.
%
\newcommand{\C}[1]{$\mathbb{C}^{#1}$}
\newcommand{\R}[1]{$\mathbb{R}^{#1}$}

%
% various generally useful helpers
%

% derivative of #1 wrt. #2:
\newcommand{\D}[2] {\frac {d#2} {d#1}}

\newcommand{\inv}[1]{\frac{1}{#1}}
\newcommand{\cross}[0]{\times}

\newcommand{\abs}[1]{\lvert{#1}\rvert}
\newcommand{\norm}[1]{\lVert{#1}\rVert}
\newcommand{\innerprod}[2]{\langle{#1}, {#2}\rangle}
\newcommand{\dotprod}[2]{{#1} \cdot {#2}}
\newcommand{\bdotprod}[2]{\left({#1} \cdot {#2}\right)}
\newcommand{\crossprod}[2]{{#1} \cross {#2}}
\newcommand{\tripleprod}[3]{\dotprod{\left(\crossprod{#1}{#2}\right)}{#3}}

\DeclareMathOperator{\Proj}{Proj}
\DeclareMathOperator{\Span}{span}
\DeclareMathOperator{\Sgn}{sgn}
\DeclareMathOperator{\Area}{Area}
\DeclareMathOperator{\Volume}{Volume}

%
% A few miscellaneous things specific to this document
%
\newcommand{\crossop}[1]{\crossprod{#1}{}}

% R2 vector.
\newcommand{\VectorTwo}[2]{
\begin{bmatrix}
 {#1} \\
 {#2}
\end{bmatrix}
}

\newcommand{\VectorN}[1]{
\begin{bmatrix}
{#1}_1 \\
{#1}_2 \\
\vdots \\
{#1}_N \\
\end{bmatrix}
}

\newcommand{\DETuvij}[4]{
\begin{vmatrix}
 {#1}_{#3} & {#1}_{#4} \\
 {#2}_{#3} & {#2}_{#4}
\end{vmatrix}
}

\newcommand{\DETuvwijk}[6]{
\begin{vmatrix}
 {#1}_{#4} & {#1}_{#5} & {#1}_{#6} \\
 {#2}_{#4} & {#2}_{#5} & {#2}_{#6} \\
 {#3}_{#4} & {#3}_{#5} & {#3}_{#6}
\end{vmatrix}
}

\newcommand{\DETuvwxijkl}[8]{
\begin{vmatrix}
 {#1}_{#5} & {#1}_{#6} & {#1}_{#7} & {#1}_{#8} \\
 {#2}_{#5} & {#2}_{#6} & {#2}_{#7} & {#2}_{#8} \\
 {#3}_{#5} & {#3}_{#6} & {#3}_{#7} & {#3}_{#8} \\
 {#4}_{#5} & {#4}_{#6} & {#4}_{#7} & {#4}_{#8} \\
\end{vmatrix}
}

%\newcommand{\DETuvwxyijklm}[10]{
%\begin{vmatrix}
% {#1}_{#6} & {#1}_{#7} & {#1}_{#8} & {#1}_{#9} & {#1}_{#10} \\
% {#2}_{#6} & {#2}_{#7} & {#2}_{#8} & {#2}_{#9} & {#2}_{#10} \\
% {#3}_{#6} & {#3}_{#7} & {#3}_{#8} & {#3}_{#9} & {#3}_{#10} \\
% {#4}_{#6} & {#4}_{#7} & {#4}_{#8} & {#4}_{#9} & {#4}_{#10} \\
% {#5}_{#6} & {#5}_{#7} & {#5}_{#8} & {#5}_{#9} & {#5}_{#10}
%\end{vmatrix}
%}

% R3 vector.
\newcommand{\VectorThree}[3]{
\begin{bmatrix}
 {#1} \\
 {#2} \\
 {#3}
\end{bmatrix}
}



\author{Peeter Joot}
\email{peeter.joot@gmail.com}


\chapter{More problems from Liboff chapter 4}
\label{chap:liboff43}
%\useCCL
\blogpage{http://sites.google.com/site/peeterjoot/math2010/liboff43.pdf}
\date{June 25, 2010}
\revisionInfo{liboff43.tex}

%\beginArtWithToc
\beginArtNoToc

\section{Motivation.}

Some more problems from \citep{liboff2003iqm}.

\section{Problem 4.11}

Some problems on Hermitian adjoints.  The starting point is the definition of the adjoint $A^\dagger$ of $A$ in terms of the inner product

\begin{align*}
\braket{\hatA^\dagger \phi}{\psi} = \braket{\phi}{\hatA \psi}
\end{align*}

\subsection{4.11 a}

\begin{align*}
\braket{ \phi }{ (a \hatA + b \hatB) \psi } 
&=
a \braket{ \phi }{ \hatA \psi } + b \braket{ \phi }{ \hatB \psi }  \\
&=
a \braket{ \hatA^\dagger \phi }{ \psi } + b \braket{ \hatB^\dagger \phi }{ \psi }  \\
&=
\braket{ a^\conj \hatA^\dagger \phi }{ \psi } + \braket{ b^\conj \hatB^\dagger \phi }{ \psi }  \\
&=
\braket{ (a^\conj \hatA^\dagger + b^\conj \hatB^\dagger ) \phi }{ \psi }  \\
&\implies \\
(a \hatA + b \hatB)^\dagger = (a^\conj \hatA^\dagger + b^\conj \hatB^\dagger)
\end{align*}

\subsection{4.11 b}
\begin{align*}
\braket{ \phi }{ \hatA \hatB \psi } 
&=
\braket{ \hatA^\dagger \phi }{ \hatB \psi }  \\
&=
\braket{ \hatB^\dagger \hatA^\dagger \phi }{ \psi }  \\
&\implies \\
(\hatA \hatB )^\dagger &=
\hatB^\dagger \hatA^\dagger 
\end{align*}

%\subsection{4.11 c}
\subsection{4.11 d}

Hermitian adjoint of $D^2$, where $D = \PDi{x}{}$.  Here we need the integral form of the inner product

\begin{align*}
\braket{\phi}{D^2 \psi} 
&=
\int \phi^\conj \PD{x}{}\PD{x}{\psi} \\
&=
-\int \PD{x}{\phi^\conj} \PD{x}{\psi} \\
&=
\int \psi \PD{x}{}\PD{x}{\phi^\conj} \\
&\implies \\
(D^2)^\dagger &= D^2
\end{align*}

Since the text shows that the square of a Hermitian operator is Hermitian, one perhaps wonders if $D$ is (but we expect not since $\hatp = -i \hbar D$ is Hermitian).

Suppose $\hatA = aD$, we have 

\begin{align*}
\hatA^\dagger = -a^\conj D,
\end{align*}

so for this to be Hermitian ($\hatA = \hatA^\dagger$) we must have $- a^\conj = a$.  If $a = r e^{i\theta}$, we have

\begin{align*}
-1 = e^{2 i\theta}
\end{align*}

So $\theta = \pi (1/2 + n)$, and $a = \pm i r$.  This fixes the scalar multiples of $D$ that are required to form a Hermitian operator

\begin{align*}
\hatA &= \pm i r D
\end{align*}

where $r$ is any real positive constant.

\subsection{4.11 e}

\begin{align*}
(\hatA \hatB - \hatB \hatA)^\dagger &= - (\hatA^\dagger \hatB^\dagger - \hatB^\dagger \hatA^\dagger)
\end{align*}

\subsection{4.11 f}

\begin{align*}
(\hatA \hatB + \hatB \hatA)^\dagger &= \hatA^\dagger \hatB^\dagger + \hatB^\dagger \hatA^\dagger
\end{align*}

\subsection{4.11 g}

\begin{align*}
i (\hatA \hatB - \hatB \hatA)^\dagger &= i ( \hatA^\dagger \hatB^\dagger - \hatB^\dagger \hatA^\dagger)
\end{align*}

\subsection{4.11 h}

This one was to calculate $(\hatA^\dagger)^\dagger$.  Intuitively I'd expect that $(\hatA^\dagger)^\dagger = \hatA$.  How could one show this?

Trying to show this with Dirac notation, I got all mixed up initially.

Using the more straightforward and old fashioned integral notation (as in \citep{bohm1989qt}), this is more straightforward.  We have the Hermitian conjugate defined by

\begin{align*}
\int \psi_2^\conj (\hatA \psi_1) = \int (\hatA^\dagger \psi_2^\conj) \psi_1,
\end{align*}

Or, more symmetrically, using braces to indicate operator direction

\begin{align*}
\int \psi_2^\conj (\hatA \psi_1) = \int (\psi_2^\conj \hatA^\dagger) \psi_1.
\end{align*}

Introduce a couple of variable substitutions for clarity

\begin{align*}
\phi_1 &= \psi_1^\conj \\
\phi_2 &= \psi_2^\conj \\
\hatB &= \hatA^\dagger.
\end{align*}

We then have

\begin{align*}
\int \psi_2^\conj (\hatA \psi_1)
&=
\int (\psi_2^\conj \hatA^\dagger) \psi_1 \\
&=
\int (\phi_2 \hatB) \phi_1^\conj \\
&=
\int \phi_1^\conj (\hatB \phi_2) \\
&=
\int (\phi_1^\conj \hatB^\dagger) \phi_2 \\
&=
\int \phi_2 (\hatB^\dagger \phi_1^\conj) \\
&=
\int \psi_2^\conj (\hatA^{\dagger \dagger} \psi_1) \\
\end{align*}

Since this is true for all $\psi_k$, we have $\hatA = \hatA^{\dagger \dagger}$ as expected.

Having figured out the problem in the simpleton way, it's now simple to go back and translate this into the Dirac inner product notation without getting muddled.  We have

\begin{align*}
\braket{ \psi_2 }{ \hatA \psi_1 } 
&=
\braket{ \hatA^\dagger \psi_2 }{ \psi_1 }  \\
&=
\braket{ \hatB \phi_2^\conj }{ \phi_1^\conj }  \\
&=
{\braket{ \phi_1 }{ \hatB^\conj \phi_2}}^\conj  \\
&=
{\braket{ (\hatB^\conj)^\dagger \phi_1 }{ \phi_2}}^\conj  \\
&=
\braket{\phi_2^\conj }{ \hatB^\dagger \phi_1^\conj } \\
&=
\braket{\psi_2 }{ \hatA^{\dagger \dagger} \psi_1 } \\
\end{align*}

\subsection{4.11 i}

\begin{align*}
(\hatA \hatA^\dagger)^\dagger &= (\hatA^\dagger)^\dagger \hatA^\dagger 
\end{align*}

since $(\hatA^\dagger) ^\dagger = \hatA$

\begin{align*}
(\hatA \hatA^\dagger)^\dagger &= \hatA \hatA^\dagger.
\end{align*}

\section{Problem 4.12 d}

If $\hatA$ is not Hermitian, is the product $\hatA^\dagger \hatA$ Hermitian?  To start we need to verify that $\braket{\psi}{\hatA^\dagger \phi} = \braket{\hatA \psi}{\phi}$.

\begin{align*}
\braket{ \psi }{ \hatA^\dagger \phi } 
&=
{\braket{ (\hatA^\dagger)^\conj \phi^\conj }{ \psi^\conj }}^\conj \\
&=
{\braket{ \phi^\conj }{ \hatA^\conj \psi^\conj }}^\conj \\
&=
\braket{ \psi }{ \hatA \psi }.
\end{align*}

With that verified we have

\begin{align*}
\braket{ \psi }{ \hatA^\dagger \hatA \phi } 
&=
\braket{ \hatA \psi }{ \hatA \phi }  \\
&=
\braket{ \hatA^\dagger \hatA \psi }{ \phi },
\end{align*}

so, the answer is yes.  Provided the adjoint exists, that product will be Hermitian.

\section{Problem 4.14}

Show that $\Expectation{\hatA} = \Expectation{\hatA}^\conj$ (that it is real), if $\hatA$ is Hermitian.  This follows by expansion of that conjugate

\begin{align*}
\Expectation{\hatA}^\conj 
&= \left(\int \psi^\conj \hatA \psi \right)^\conj \\
&= \int \psi \hatA^\conj \psi^\conj \\
&= \int (\hatA \psi)^\conj \psi \\
&= \braket{ \hatA \psi }{ \psi } \\
&= \braket{ \psi }{ \hatA^\dagger \psi } \\
&= \braket{ \psi }{ \hatA \psi } \\
&= \Expectation{\hatA}
\end{align*}

\EndArticle
