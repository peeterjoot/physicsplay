\documentclass{article}      % Specifies the document class

\usepackage{amsmath}
\usepackage{mathpazo}

%
% shorthand for bold symbols, convenient for vectors and matrices
%
\newcommand{\Ba}[0]{\mathbf{a}}
\newcommand{\Bb}[0]{\mathbf{b}}
\newcommand{\Bc}[0]{\mathbf{c}}
\newcommand{\Bd}[0]{\mathbf{d}}
\newcommand{\Be}[0]{\mathbf{e}}
\newcommand{\Bf}[0]{\mathbf{f}}
\newcommand{\Bg}[0]{\mathbf{g}}
\newcommand{\Bh}[0]{\mathbf{h}}
\newcommand{\Bi}[0]{\mathbf{i}}
\newcommand{\Bj}[0]{\mathbf{j}}
\newcommand{\Bk}[0]{\mathbf{k}}
\newcommand{\Bl}[0]{\mathbf{l}}
\newcommand{\Bm}[0]{\mathbf{m}}
\newcommand{\Bn}[0]{\mathbf{n}}
\newcommand{\Bo}[0]{\mathbf{o}}
\newcommand{\Bp}[0]{\mathbf{p}}
\newcommand{\Bq}[0]{\mathbf{q}}
\newcommand{\Br}[0]{\mathbf{r}}
\newcommand{\Bs}[0]{\mathbf{s}}
\newcommand{\Bt}[0]{\mathbf{t}}
\newcommand{\Bu}[0]{\mathbf{u}}
\newcommand{\Bv}[0]{\mathbf{v}}
\newcommand{\Bw}[0]{\mathbf{w}}
\newcommand{\Bx}[0]{\mathbf{x}}
\newcommand{\By}[0]{\mathbf{y}}
\newcommand{\Bz}[0]{\mathbf{z}}
\newcommand{\BA}[0]{\mathbf{A}}
\newcommand{\BB}[0]{\mathbf{B}}
\newcommand{\BC}[0]{\mathbf{C}}
\newcommand{\BD}[0]{\mathbf{D}}
\newcommand{\BE}[0]{\mathbf{E}}
\newcommand{\BF}[0]{\mathbf{F}}
\newcommand{\BG}[0]{\mathbf{G}}
\newcommand{\BH}[0]{\mathbf{H}}
\newcommand{\BI}[0]{\mathbf{I}}
\newcommand{\BJ}[0]{\mathbf{J}}
\newcommand{\BK}[0]{\mathbf{K}}
\newcommand{\BL}[0]{\mathbf{L}}
\newcommand{\BM}[0]{\mathbf{M}}
\newcommand{\BN}[0]{\mathbf{N}}
\newcommand{\BO}[0]{\mathbf{O}}
\newcommand{\BP}[0]{\mathbf{P}}
\newcommand{\BQ}[0]{\mathbf{Q}}
\newcommand{\BR}[0]{\mathbf{R}}
\newcommand{\BS}[0]{\mathbf{S}}
\newcommand{\BT}[0]{\mathbf{T}}
\newcommand{\BU}[0]{\mathbf{U}}
\newcommand{\BV}[0]{\mathbf{V}}
\newcommand{\BW}[0]{\mathbf{W}}
\newcommand{\BX}[0]{\mathbf{X}}
\newcommand{\BY}[0]{\mathbf{Y}}
\newcommand{\BZ}[0]{\mathbf{Z}}

\newcommand{\Bzero}[0]{\mathbf{0}}
\newcommand{\Btheta}[0]{\boldsymbol{\theta}}
\newcommand{\Btau}[0]{\boldsymbol{\tau}}
\newcommand{\Bomega}[0]{\boldsymbol{\omega}}

%
% shorthand for unit vectors
%
\newcommand{\acap}[0]{\hat{\Ba}}
\newcommand{\bcap}[0]{\hat{\Bb}}
\newcommand{\ccap}[0]{\hat{\Bc}}
\newcommand{\dcap}[0]{\hat{\Bd}}
\newcommand{\ecap}[0]{\hat{\Be}}
\newcommand{\fcap}[0]{\hat{\Bf}}
\newcommand{\gcap}[0]{\hat{\Bg}}
\newcommand{\hcap}[0]{\hat{\Bh}}
\newcommand{\icap}[0]{\hat{\Bi}}
\newcommand{\jcap}[0]{\hat{\Bj}}
\newcommand{\kcap}[0]{\hat{\Bk}}
\newcommand{\lcap}[0]{\hat{\Bl}}
\newcommand{\mcap}[0]{\hat{\Bm}}
\newcommand{\ncap}[0]{\hat{\Bn}}
\newcommand{\ocap}[0]{\hat{\Bo}}
\newcommand{\pcap}[0]{\hat{\Bp}}
\newcommand{\qcap}[0]{\hat{\Bq}}
\newcommand{\rcap}[0]{\hat{\Br}}
\newcommand{\scap}[0]{\hat{\Bs}}
\newcommand{\tcap}[0]{\hat{\Bt}}
\newcommand{\ucap}[0]{\hat{\Bu}}
\newcommand{\vcap}[0]{\hat{\Bv}}
\newcommand{\wcap}[0]{\hat{\Bw}}
\newcommand{\xcap}[0]{\hat{\Bx}}
\newcommand{\ycap}[0]{\hat{\By}}
\newcommand{\zcap}[0]{\hat{\Bz}}
\newcommand{\thetacap}[0]{\hat{\Btheta}}

%
% to write R^n and C^n in a distinguishable fashion.  Perhaps change this
% to the double lined characters upon figuring out how to do so.
%
\newcommand{\C}[1]{$\mathbb{C}^{#1}$}
\newcommand{\R}[1]{$\mathbb{R}^{#1}$}

%
% various generally useful helpers
%

% derivative of #1 wrt. #2:
\newcommand{\D}[2] {\frac {d#2} {d#1}}

\newcommand{\inv}[1]{\frac{1}{#1}}
\newcommand{\cross}[0]{\times}

\newcommand{\abs}[1]{\lvert{#1}\rvert}
\newcommand{\norm}[1]{\lVert{#1}\rVert}
\newcommand{\innerprod}[2]{\langle{#1}, {#2}\rangle}
\newcommand{\dotprod}[2]{{#1} \cdot {#2}}
\newcommand{\bdotprod}[2]{\left({#1} \cdot {#2}\right)}
\newcommand{\crossprod}[2]{{#1} \cross {#2}}
\newcommand{\tripleprod}[3]{\dotprod{\left(\crossprod{#1}{#2}\right)}{#3}}

\DeclareMathOperator{\Proj}{Proj}
\DeclareMathOperator{\Span}{span}
\DeclareMathOperator{\Sgn}{sgn}
\DeclareMathOperator{\Area}{Area}
\DeclareMathOperator{\Volume}{Volume}

%
% A few miscellaneous things specific to this document
%
\newcommand{\crossop}[1]{\crossprod{#1}{}}

% R2 vector.
\newcommand{\VectorTwo}[2]{
\begin{bmatrix}
 {#1} \\
 {#2}
\end{bmatrix}
}

\newcommand{\VectorN}[1]{
\begin{bmatrix}
{#1}_1 \\
{#1}_2 \\
\vdots \\
{#1}_N \\
\end{bmatrix}
}

\newcommand{\DETuvij}[4]{
\begin{vmatrix}
 {#1}_{#3} & {#1}_{#4} \\
 {#2}_{#3} & {#2}_{#4}
\end{vmatrix}
}

\newcommand{\DETuvwijk}[6]{
\begin{vmatrix}
 {#1}_{#4} & {#1}_{#5} & {#1}_{#6} \\
 {#2}_{#4} & {#2}_{#5} & {#2}_{#6} \\
 {#3}_{#4} & {#3}_{#5} & {#3}_{#6}
\end{vmatrix}
}

\newcommand{\DETuvwxijkl}[8]{
\begin{vmatrix}
 {#1}_{#5} & {#1}_{#6} & {#1}_{#7} & {#1}_{#8} \\
 {#2}_{#5} & {#2}_{#6} & {#2}_{#7} & {#2}_{#8} \\
 {#3}_{#5} & {#3}_{#6} & {#3}_{#7} & {#3}_{#8} \\
 {#4}_{#5} & {#4}_{#6} & {#4}_{#7} & {#4}_{#8} \\
\end{vmatrix}
}

%\newcommand{\DETuvwxyijklm}[10]{
%\begin{vmatrix}
% {#1}_{#6} & {#1}_{#7} & {#1}_{#8} & {#1}_{#9} & {#1}_{#10} \\
% {#2}_{#6} & {#2}_{#7} & {#2}_{#8} & {#2}_{#9} & {#2}_{#10} \\
% {#3}_{#6} & {#3}_{#7} & {#3}_{#8} & {#3}_{#9} & {#3}_{#10} \\
% {#4}_{#6} & {#4}_{#7} & {#4}_{#8} & {#4}_{#9} & {#4}_{#10} \\
% {#5}_{#6} & {#5}_{#7} & {#5}_{#8} & {#5}_{#9} & {#5}_{#10}
%\end{vmatrix}
%}

% R3 vector.
\newcommand{\VectorThree}[3]{
\begin{bmatrix}
 {#1} \\
 {#2} \\
 {#3}
\end{bmatrix}
}


\newcommand{\Brho}[0]{\boldsymbol{\rho}}
\newcommand{\LL}[0]{\mathcal{L}}
\newcommand{\Abs}[1]{\left\lvert{#1}\right\rvert}
\newcommand{\qdot}[0]{\dot{q}}
\newcommand{\qddot}[0]{\ddot{q}}
\newcommand{\xdot}[0]{\dot{x}}
\newcommand{\xddot}[0]{\ddot{x}}
\newcommand{\dotalpha}[0]{\dot{\alpha}}
\newcommand{\ddotalpha}[0]{\ddot{\alpha}}
\newcommand{\dottheta}[0]{\dot{\theta}}
\newcommand{\ddottheta}[0]{\ddot{\theta}}
% == \partial_{#1} {#2}
\newcommand{\PD}[2]{\frac{\partial {#2}}{\partial {#1}}}
\newcommand{\PDD}[3]{\frac{\partial^2 {#3}}{\partial {#1}\partial {#2}}}

%
% The real thing:
%

                             % The preamble begins here.
\title{Attempts at solutions for some Goldstein Mechanics problems.} % Declares the document's title.
\author{Peeter Joot}         % Declares the author's name.
\date{ }        % Deleting this command produces today's date.

\begin{document}             % End of preamble and beginning of text.

\maketitle{}

\section{ Problem 1.7 }

Barbell shape, equal masses.  center of rod between masses constrained to circular motion.

Assuming motion in a plane, the equation for the center of the rod is:

\begin{equation*}
c = a e^{i\theta}
\end{equation*}

and the two mass points positions are:
\begin{align*}
q_1 &= c + (l/2) e^{i\alpha} \\
q_2 &= c - (l/2) e^{i\alpha}
\end{align*}

taking derivatives:
\begin{align*}
\qdot_1 &= a i \dottheta e^{i\theta} + (l/2) i \dotalpha e^{i\alpha} \\
\qdot_2 &= a i \dottheta e^{i\theta} - (l/2) i \dotalpha e^{i\alpha} \\
\end{align*}

and squared magnitudes:

\begin{align*}
\qdot_{\pm}
&= \Abs{a \dottheta \pm (l/2) \dotalpha e^{i(\alpha - \theta)}}^2 \\
&= \left(a \dottheta   \pm   \inv{2} l \dotalpha \cos(\alpha - \theta)\right)^2 + \left(\inv{2} l \dotalpha \sin(\alpha - \theta)\right)^2
\end{align*}

Summing the kinetic terms yeilds

\begin{equation*}
K = m \left(a \dottheta \right)^2 + m \left(\inv{2} l \dotalpha\right)^2
\end{equation*}

Summing the potential energies, presuming that the motion is verticle, we have:

\begin{equation*}
V = m g (l/2) \cos\theta - m g (l/2) \cos \theta
\end{equation*}

So, the Lagrangian is just the Kinetic energy.

Taking derivatives to get the OEMs we have:

\begin{align*}
(m a^2 \dottheta)' &= 0 \\
\left(\inv{4} m l^2 \dotalpha \right)' &= 0
\end{align*}

This is suprising seeming.  Is this correct?

\section{ Problem 1.8 }

Hopefully, not a copyright violation, but here is the problem verbatim:

A system is composed of three particles of equal mass m.  Between any two of them there are forces derivable from a potential

\begin{equation*}
V = -g e^{-\mu r}
\end{equation*}

where r is the disance between the two particles.  In addition, two of the particles each exert a force on the third which can be obtained from a generalized potential of the form

\begin{equation*}
U = -f \Bv \cdot \Br
\end{equation*}

$\Bv$ being the relative velocity of the interacting particles and f a constant.  Set up the Lagragian for the system, using as coordinates the radius vector $\BR$ of the center of mass and the two vectors

\begin{align*}
\Brho_1 &= \Br_1 - \Br_3 \\
\Brho_2 &= \Br_2 - \Br_3
\end{align*}

Is the total angular momentum of the system conserved?

\subsection{ Solution attempt. }

The center of mass vector is:

\begin{equation*}
\BR = \inv{3}(\Br_1 + \Br_2 + \Br_3)
\end{equation*}

This can be used to express each of the position vectors in terms of the $\Brho_i$ vectors:

\begin{align*}
3 m \BR &= m (\Brho_1 + \Br_3) + m(\Brho_2 + \Br_3) + m \Br_3 \\
        &= 2 m (\Brho_1 + \Brho_2) + 3 m \Br_3 \\
  \Br_3 &= \BR - \inv{3}(\Brho_1 + \Brho_2) \\
\Br_2 = \Brho_2 + \Br_3 &= \Brho_2 + \Br_3 = \frac{2}{3} \Brho_2 - \inv{2} \Brho_1 + \BR \\
\Br_1 = \Brho_1 + \Br_3 &= \frac{2}{3} \Brho_1 - \inv{2} \Brho_2 + \BR \\
\end{align*}

Now, that is enough to specify the part of the Lagrangian from the potentials that act between all the particles

\begin{equation*}
\LL_U = \sum -U_{ij} = g \left( e^{-\mu \Abs{\Brho_1}} + e^{-\mu \Abs{\Brho_2}} + e^{-\mu \Abs{ \Brho_1 - \Brho_2 }} \right)
\end{equation*}

Now, we need to calculate the two $V$ potentials in terms.  If we consider with positions $\Br_1$, and $\Br_2$ to be the ones
that can exert a force on the third, the velocities of those masses relative to $\Br_3$ are:

\begin{equation*}
(\Br_3 - \Br_i)' = \dot{\Brho_i}
\end{equation*}

Adding this to the first half of the Lagrangian we have:
So, for the second half of the Lagrangian we have:

\begin{equation*}
\LL =
g \left( e^{-\mu \Abs{\Brho_1}} + e^{-\mu \Abs{\Brho_2}} + e^{-\mu \Abs{ \Brho_1 - \Brho_2 }} \right)
+ f \left(\BR - \inv{3}(\Brho_1 + \Brho_2) \right) \cdot \left( \dot{\Brho_1} + \dot{\Brho_2} \right)
\end{equation*}

So, there's the Lagrangian.

How about the angular momentum conservation question?  How to answer that?  One way would be to compute the forces from the Lagrangian, and take cross products but is that really the best way?  Perhaps the answer is as simple as observing that there are no external torque's on the system, thus $d\BL/dt = 0$, or angular momentum for the system is constant (conserved).

FIXME: FOLLOWUP: it has been suggested to me on PF that I should look at how this Lagrangian transforms under rotation.  The relative vectors (both speed and velocity) between the different points will be rotation invarient.  Think that is the case for the CM too.

\section{ Problem 2.1 }

Prove that the shortest length curve between two points in space is a straight line.

A first attempt of this I used:

\begin{equation*}
ds = \sqrt{ 1 + (dy/dx)^2 + (dz/dx)^2 } dx
\end{equation*}

Application of the Euler-Lagrange equations does show that one ends up with a linear relation between the y and z coordinates, but no mention of x.  Rather than write that up, consider instead a parameterization of the coordinates:

\begin{align*}
x &= x_1(\lambda) \\
y &= x_2(\lambda) \\
z &= x_3(\lambda)
\end{align*}

in terms of this arbitrary parameterization we have a segment length of:

\begin{equation*}
ds = \sqrt{ \sum \left(\frac{d x_i}{d\lambda}\right)^2 } d \lambda = f\left(x_i\right) d\lambda
\end{equation*}

Application of the Euler-Lagrange equation to $f$ we have:

\begin{align*}
\PD{x_i}{f} 
&= 0 \\
&= \frac{d}{d\lambda} \PD{\xdot_i}{} \sqrt{ \sum {\xdot_j}^2 } \\
&= \frac{d}{d\lambda} \frac{ \xdot_i }{\sqrt{ \sum {\xdot_j}^2 }}
\end{align*}

Therefore each of these quotients can be equated to a constant:

\begin{align*}
\frac{ \xdot_i }{\sqrt{ \sum {\xdot_j}^2 }} &= {c_i}^{-2} \\
{c_i}^2 \xdot_i^2 &= \sum {\xdot_j}^2 \\
({c_i}^2 -1)\xdot_i^2 &= \sum_{j \ne i} {\xdot_j}^2 \\
(1 - {c_i}^2)\xdot_i^2 + \sum_{j \ne i} {\xdot_j}^2 &= 0 
\end{align*}

This last form shows explicitly that not all of these squared derivative terms can be linearly independent.  In particular, we have a
zero determinant:

\begin{equation*}
0 =
\begin{vmatrix}
1 - c_1^2   & 1            & 1         & 1 & \hdots \\
1           & 1 - c_2^2    & 1         & 1 & \vdots \\
1           & 1            & 1 - c_3^2 & 1 & \\
            &              &           & \ddots & \\
            &              &           &        & 1 - {c_n}^2
\end{vmatrix}
\end{equation*}

Now, expanding this for a couple specific cases isn't too hard.  For $n=2$ we have:

\begin{align*}
0 &= (1 - c_1^2)(1-c_2^2) - 1 \\
c_1^2 + c_2^2 &= c_1^2 c_2^2 \\
c_1^2 &= \frac{c_2^2}{ c_2^2 - 1 } \\
c_2^2 - 1 &= \frac{c_2^2}{ c_1^2 }
\end{align*}

This can be substuited back into one our $c_2^2$ equation:

\begin{align*}
({c_2}^2 -1)\xdot_2^2 &= {\xdot_1}^2 \\
\frac{c_2^2}{ c_1^2 } \xdot_2^2 &= {\xdot_1}^2 \\
\pm \frac{c_2}{ c_1 } \xdot_2 &= {\xdot_1} \\
\pm \frac{c_2}{ c_1 } x_2 &= x_1 + \kappa \\
\end{align*}

This is precisely the straight line that was desired, but we have setup for proving that consideration of all path variations from two points 
in \R{N} space has the shortest distance when that path is a straight line.

Despite the general setup, I'm going to chicken out and show this only for the \R{3} case.  In that case our determinant expands to:

\begin{equation*}
c_1^2 + c_2^2 + c_3^2 = c_1^2 c_2^2 c_3^2
\end{equation*}

Since not all of the $\xdot_i^2$ can be linearly independent, one can be eliminated:

\begin{align*}
(1 - c_1^2) \xdot_1^2 + \xdot_2^2 + \xdot_3^2 &= 0 \\
(1 - c_2^2) \xdot_2^2 + \xdot_3^2 + \xdot_1^2 &= 0 \\
(1 - c_3^2) \xdot_3^2 + \xdot_1^2 + \xdot_2^2 &= 0
\end{align*}

Let's pick $\xdot_1^2$ to eliminate, and subst 2 into 3:

\begin{align*}
%(1 - c_1^2) (-(1 - c_2^2) \xdot_2^2 - \xdot_3^2) + \xdot_2^2 + \xdot_3^2 &= 0 \\
(1 - c_3^2) \xdot_3^2 + (-(1 - c_2^2) \xdot_2^2 - \xdot_3^2) + \xdot_2^2 &= 0
\implies \\
%\xdot_2^2 ( 1 - (1 - c_1^2)(1 - c_2^2) ) + \xdot_3^2 ( 1 - (1 - c_1^2) ) &= 0 \\
- c_3^2 \xdot_3^2 + c_2^2 \xdot_2 &= 0 \\
\pm c_3 \xdot_3 &= c_2 \xdot_2 \\
\end{align*}

%Which is, once again a straight line:
%
%\begin{equation*}
%\pm c_3 x_3 = c_2 x_2 + \kappa
%\end{equation*}

Since these equations are symmetric, we can do this for all, with the result:
\begin{align*}
\pm c_3 \xdot_3 &= c_2 \xdot_2 \\
\pm c_3 \xdot_3 &= c_1 \xdot_1 \\
\pm c_2 \xdot_2 &= c_1 \xdot_1 \\
\end{align*}

Since the $c_i$ constants are arbitrary, then we can for example pick the negative sign for $\pm c_2$, and the positive for the rest, then add all of these, and scale by two:

\begin{equation*}
c_3 \xdot_3 - c_2 \xdot_2 = c_1 \xdot_1
\end{equation*}

and integrating:

\begin{equation*}
c_3 x_3 - c_2 x_2 = c_1 x_1 + \kappa
\end{equation*}

Again, we have the general equation of a line, subject to the desired constraints on the end points.  In the end we didn't need to 
evaluate the determinant after all, as done in the 
\R{2} case.

\end{document}               % End of document.
