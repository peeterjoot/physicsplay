% 
% 
% 
% Copyright � 2012 Peeter Joot
% All Rights Reserved
% 
% This file may be reproduced and distributed in whole or in part, without fee, subject to the following conditions:
% 
% o The copyright notice above and this permission notice must be preserved complete on all complete or partial copies.
% 
% o Any translation or derived work must be approved by the author in writing before distribution.
% 
% o If you distribute this work in part, instructions for obtaining the complete version of this file must be included, and a means for obtaining a complete version provided.
% 
% 
% Exceptions to these rules may be granted for academic purposes: Write to the author and ask.
% 
% 
% 
%\documentclass[]{eliblog}

\usepackage{amsmath}
\usepackage{mathpazo}

%
% shorthand for bold symbols, convenient for vectors and matrices
%
\newcommand{\Ba}[0]{\mathbf{a}}
\newcommand{\Bb}[0]{\mathbf{b}}
\newcommand{\Bc}[0]{\mathbf{c}}
\newcommand{\Bd}[0]{\mathbf{d}}
\newcommand{\Be}[0]{\mathbf{e}}
\newcommand{\Bf}[0]{\mathbf{f}}
\newcommand{\Bg}[0]{\mathbf{g}}
\newcommand{\Bh}[0]{\mathbf{h}}
\newcommand{\Bi}[0]{\mathbf{i}}
\newcommand{\Bj}[0]{\mathbf{j}}
\newcommand{\Bk}[0]{\mathbf{k}}
\newcommand{\Bl}[0]{\mathbf{l}}
\newcommand{\Bm}[0]{\mathbf{m}}
\newcommand{\Bn}[0]{\mathbf{n}}
\newcommand{\Bo}[0]{\mathbf{o}}
\newcommand{\Bp}[0]{\mathbf{p}}
\newcommand{\Bq}[0]{\mathbf{q}}
\newcommand{\Br}[0]{\mathbf{r}}
\newcommand{\Bs}[0]{\mathbf{s}}
\newcommand{\Bt}[0]{\mathbf{t}}
\newcommand{\Bu}[0]{\mathbf{u}}
\newcommand{\Bv}[0]{\mathbf{v}}
\newcommand{\Bw}[0]{\mathbf{w}}
\newcommand{\Bx}[0]{\mathbf{x}}
\newcommand{\By}[0]{\mathbf{y}}
\newcommand{\Bz}[0]{\mathbf{z}}
\newcommand{\BA}[0]{\mathbf{A}}
\newcommand{\BB}[0]{\mathbf{B}}
\newcommand{\BC}[0]{\mathbf{C}}
\newcommand{\BD}[0]{\mathbf{D}}
\newcommand{\BE}[0]{\mathbf{E}}
\newcommand{\BF}[0]{\mathbf{F}}
\newcommand{\BG}[0]{\mathbf{G}}
\newcommand{\BH}[0]{\mathbf{H}}
\newcommand{\BI}[0]{\mathbf{I}}
\newcommand{\BJ}[0]{\mathbf{J}}
\newcommand{\BK}[0]{\mathbf{K}}
\newcommand{\BL}[0]{\mathbf{L}}
\newcommand{\BM}[0]{\mathbf{M}}
\newcommand{\BN}[0]{\mathbf{N}}
\newcommand{\BO}[0]{\mathbf{O}}
\newcommand{\BP}[0]{\mathbf{P}}
\newcommand{\BQ}[0]{\mathbf{Q}}
\newcommand{\BR}[0]{\mathbf{R}}
\newcommand{\BS}[0]{\mathbf{S}}
\newcommand{\BT}[0]{\mathbf{T}}
\newcommand{\BU}[0]{\mathbf{U}}
\newcommand{\BV}[0]{\mathbf{V}}
\newcommand{\BW}[0]{\mathbf{W}}
\newcommand{\BX}[0]{\mathbf{X}}
\newcommand{\BY}[0]{\mathbf{Y}}
\newcommand{\BZ}[0]{\mathbf{Z}}

\newcommand{\Bzero}[0]{\mathbf{0}}
\newcommand{\Btheta}[0]{\boldsymbol{\theta}}
\newcommand{\Btau}[0]{\boldsymbol{\tau}}
\newcommand{\Bomega}[0]{\boldsymbol{\omega}}

%
% shorthand for unit vectors
%
\newcommand{\acap}[0]{\hat{\Ba}}
\newcommand{\bcap}[0]{\hat{\Bb}}
\newcommand{\ccap}[0]{\hat{\Bc}}
\newcommand{\dcap}[0]{\hat{\Bd}}
\newcommand{\ecap}[0]{\hat{\Be}}
\newcommand{\fcap}[0]{\hat{\Bf}}
\newcommand{\gcap}[0]{\hat{\Bg}}
\newcommand{\hcap}[0]{\hat{\Bh}}
\newcommand{\icap}[0]{\hat{\Bi}}
\newcommand{\jcap}[0]{\hat{\Bj}}
\newcommand{\kcap}[0]{\hat{\Bk}}
\newcommand{\lcap}[0]{\hat{\Bl}}
\newcommand{\mcap}[0]{\hat{\Bm}}
\newcommand{\ncap}[0]{\hat{\Bn}}
\newcommand{\ocap}[0]{\hat{\Bo}}
\newcommand{\pcap}[0]{\hat{\Bp}}
\newcommand{\qcap}[0]{\hat{\Bq}}
\newcommand{\rcap}[0]{\hat{\Br}}
\newcommand{\scap}[0]{\hat{\Bs}}
\newcommand{\tcap}[0]{\hat{\Bt}}
\newcommand{\ucap}[0]{\hat{\Bu}}
\newcommand{\vcap}[0]{\hat{\Bv}}
\newcommand{\wcap}[0]{\hat{\Bw}}
\newcommand{\xcap}[0]{\hat{\Bx}}
\newcommand{\ycap}[0]{\hat{\By}}
\newcommand{\zcap}[0]{\hat{\Bz}}
\newcommand{\thetacap}[0]{\hat{\Btheta}}

%
% to write R^n and C^n in a distinguishable fashion.  Perhaps change this
% to the double lined characters upon figuring out how to do so.
%
\newcommand{\C}[1]{$\mathbb{C}^{#1}$}
\newcommand{\R}[1]{$\mathbb{R}^{#1}$}

%
% various generally useful helpers
%

% derivative of #1 wrt. #2:
\newcommand{\D}[2] {\frac {d#2} {d#1}}

\newcommand{\inv}[1]{\frac{1}{#1}}
\newcommand{\cross}[0]{\times}

\newcommand{\abs}[1]{\lvert{#1}\rvert}
\newcommand{\norm}[1]{\lVert{#1}\rVert}
\newcommand{\innerprod}[2]{\langle{#1}, {#2}\rangle}
\newcommand{\dotprod}[2]{{#1} \cdot {#2}}
\newcommand{\bdotprod}[2]{\left({#1} \cdot {#2}\right)}
\newcommand{\crossprod}[2]{{#1} \cross {#2}}
\newcommand{\tripleprod}[3]{\dotprod{\left(\crossprod{#1}{#2}\right)}{#3}}

\DeclareMathOperator{\Proj}{Proj}
\DeclareMathOperator{\Span}{span}
\DeclareMathOperator{\Sgn}{sgn}
\DeclareMathOperator{\Area}{Area}
\DeclareMathOperator{\Volume}{Volume}

%
% A few miscellaneous things specific to this document
%
\newcommand{\crossop}[1]{\crossprod{#1}{}}

% R2 vector.
\newcommand{\VectorTwo}[2]{
\begin{bmatrix}
 {#1} \\
 {#2}
\end{bmatrix}
}

\newcommand{\VectorN}[1]{
\begin{bmatrix}
{#1}_1 \\
{#1}_2 \\
\vdots \\
{#1}_N \\
\end{bmatrix}
}

\newcommand{\DETuvij}[4]{
\begin{vmatrix}
 {#1}_{#3} & {#1}_{#4} \\
 {#2}_{#3} & {#2}_{#4}
\end{vmatrix}
}

\newcommand{\DETuvwijk}[6]{
\begin{vmatrix}
 {#1}_{#4} & {#1}_{#5} & {#1}_{#6} \\
 {#2}_{#4} & {#2}_{#5} & {#2}_{#6} \\
 {#3}_{#4} & {#3}_{#5} & {#3}_{#6}
\end{vmatrix}
}

\newcommand{\DETuvwxijkl}[8]{
\begin{vmatrix}
 {#1}_{#5} & {#1}_{#6} & {#1}_{#7} & {#1}_{#8} \\
 {#2}_{#5} & {#2}_{#6} & {#2}_{#7} & {#2}_{#8} \\
 {#3}_{#5} & {#3}_{#6} & {#3}_{#7} & {#3}_{#8} \\
 {#4}_{#5} & {#4}_{#6} & {#4}_{#7} & {#4}_{#8} \\
\end{vmatrix}
}

%\newcommand{\DETuvwxyijklm}[10]{
%\begin{vmatrix}
% {#1}_{#6} & {#1}_{#7} & {#1}_{#8} & {#1}_{#9} & {#1}_{#10} \\
% {#2}_{#6} & {#2}_{#7} & {#2}_{#8} & {#2}_{#9} & {#2}_{#10} \\
% {#3}_{#6} & {#3}_{#7} & {#3}_{#8} & {#3}_{#9} & {#3}_{#10} \\
% {#4}_{#6} & {#4}_{#7} & {#4}_{#8} & {#4}_{#9} & {#4}_{#10} \\
% {#5}_{#6} & {#5}_{#7} & {#5}_{#8} & {#5}_{#9} & {#5}_{#10}
%\end{vmatrix}
%}

% R3 vector.
\newcommand{\VectorThree}[3]{
\begin{bmatrix}
 {#1} \\
 {#2} \\
 {#3}
\end{bmatrix}
}



\author{Peeter Joot}
\email{peeter.joot@gmail.com}


\chapter{Reader notes for Jackson 12.11, Retarded time solution to the wave equation.}
\label{chap:jacksonRetarded}
%\useCCL
\blogpage{http://sites.google.com/site/peeterjoot/math2009/jacksonRetarded.pdf}
\date{Sept 19, 2009}
\revisionInfo{jacksonRetarded.tex }

%\beginArtWithToc
\beginArtNoToc

\section{Motivation}

In \citep{gabook:PJpoisson} I blundered my way towards the retarded time Green's function solution to the 3D wave equation.  Jackson's \citep{jackson1975cew} (section 12.11) covers this in a much more coherent fashion.  It is however somewhat terse, and some details that were not immediately obvious to me were omitted.

Here are my notes for this section in case I want to refer to it again later.

\section{Guts}

The starting point is the electrodynamic wave equation

\begin{align}\label{eqn:jacksonRet:boo1}
\partial_\alpha F^{\alpha\beta} = \frac{4 \pi}{c} J^\beta
\end{align}

A substitution of $F^{\alpha \beta} = \partial^\alpha A^\beta - \partial^\beta A^\alpha$ gives us

\begin{align}\label{eqn:jacksonRet:boo2}
\partial_\alpha F^{\alpha\beta} = \partial_\alpha \partial^\alpha A^\beta - \partial_\alpha \partial^\beta A^\alpha
= \square A^\beta - \partial^\beta  (\partial_\alpha A^\alpha)
\end{align}

Thus with the Lorentz condition $\partial_\alpha A^\alpha = 0$ we have 

\begin{align}\label{eqn:jacksonRet:boo3}
\square A^\beta = \frac{4 \pi}{c} J^\beta
\end{align}

A set of four non-homogeneous wave equations to solve.  It is assumed that a Green's function of the form

\begin{align}\label{eqn:jacksonRet:boo4}
\square_x D(x - x') = \delta^4(x - x')
\end{align}

can be found.  Jackson states that this is possible in the absence of boundary surfaces, which seems to imply that the more general case would require $\square_x D(x, x') = \delta^4(x - x')$, where $D$ is not necessarily a function of the four vector difference $x - x'$.

What is really meant by this Green's function?  It only takes meaning in the context of the convolution integral.  Namely

\begin{align}\label{eqn:jacksonRet:boo5}
A^\beta = \int d^4 x' D(x, x') \frac{4 \pi}{c} J^\beta(x') 
\end{align}

So that
\begin{align*}
\square_x A^\beta 
&= \int d^4 x' \square_x D(x, x') \frac{4 \pi}{c} J^\beta(x') \\
&= \frac{4 \pi}{c} \int d^4 x' \delta^4(x - x') J^\beta(x') \\
&= \frac{4 \pi}{c} J^\beta(x) \\
\end{align*}

So if a function with this delta filtering property under the DeLambertian can be found we can find the non-homogeneous solutions directly by four-volume convolution.

It is implied in the text (probably stated explicitly somewhere earlier) that the asymmetric convention for the Fourier transform pairs is being used

\begin{align}\label{eqn:jacksonRet:boo6}
\tilde{f}(k) &= \int d^4 z f(z) e^{i k \cdot z} \\
f(z) &= \inv{(2\pi)^4} \int d^4 k \tilde{f}(k) e^{-i k \cdot z} 
\end{align}

where $d^4 k = dk_0 dk_1 dk_2 dk_3$, and $d^4 z = dz^0 dz^1 dz^2 dz^3$, and $k \cdot z = k_\mu z^\mu = k^\mu z_\mu$.

Assuming the validity of this transform pair, even for the delta distribution, we can find an integral representation of the delta using the transform pairs.  For the Fourier transform of delta we have

\begin{align*}
\tilde{\delta^4}(k) 
&= \int d^4 z \delta^4(z) e^{i k \cdot z} \\
&= e^{i k \cdot 0} \\
&= 1
\end{align*}

Performing the inverse transformation provides the delta function exponential integral representation 

\begin{align*}
\delta^4(z) 
&= \inv{(2\pi)^4} \int d^4 k \tilde{\delta^4}(k) e^{-i k \cdot z} \\
&= \inv{(2\pi)^4} \int d^4 k e^{-i k \cdot z} \\
\end{align*}

Just as a Fourier representation of the delta can be found, we can integrate by parts to find an integral representation of the Green's function that we seek.  Taking Fourier transforms

\begin{align*}
\FF(\square_x D(z))(k) 
&= \int d^4 z \partial_\alpha \partial^\alpha D(z) e^{i k \cdot z} \\
&= -\int d^4 z \partial^\alpha D(z) \partial_\alpha e^{i k_\beta z^\beta} \\
&= -\int d^4 z \partial^\alpha D(z) i k_\alpha e^{i k_\beta z^\beta} \\
&= \int d^4 z D(z) i k_\alpha \partial^\alpha e^{i k^\beta z_\beta} \\
&= -\int d^4 z D(z) k_\alpha k^\alpha e^{i k \cdot z } \\
&= - k^2 \tilde{D}(k)
\end{align*}

Using the assumed delta function property of this Green's function we also have

\begin{align*}
\FF(\square_x D(z))(k) 
&= \int d^4 z \delta^4(z) e^{i k \cdot z} \\
&= 1
\end{align*}

This completely specifies the Fourier transform of the Green's function
\begin{align}\label{eqn:jacksonRet:boo7}
\tilde{D}(k) &= - \inv{k^2}
\end{align}

and we can inverse transform to complete the task of finding an initial representation of the Green's function itself.  That is

\begin{align}\label{eqn:jacksonRet:boo8}
D(z) = -\inv{(2\pi)^4} \int d^4 k \inv{k^2} e^{-i k \cdot z} 
\end{align}

With an explicit spacetime split we have our integral prepped for the contour integration

\begin{align}\label{eqn:jacksonRet:boo9}
D(z) = -\inv{(2\pi)^4} \int d^3 k e^{i \Bk \cdot \Bz} \int_{-\infty}^\infty dk_0 \inv{k_0^2 - \Bk^2} e^{-i k_0 z_0} 
\end{align}

Here $\kappa = \Abs{\Bk}$ is used as in the text.  If we let $k_0 = R e^{i\theta}$ take on complex values, integrating over a semicircular arc, we have for the exponential 

\begin{align*}
\Abs{e^{-i k_0 z_0}}
&= \Abs{e^{-i R (\cos\theta + i \sin\theta) z_0} } \\
&= \Abs{e^{ z_0 R \sin\theta} e^{ -i z_0 R \cos\theta } } \\
&= \Abs{e^{ z_0 R \sin\theta} }
\end{align*}

In the upper half plane $\theta \in [0,\pi]$, so $\sin\theta$ is never negative, and the integral on an upper half plane semi-circular contour can only vanish as desired for $z_0 < 0$.  Similarly the infinite arc integral can only be zero for $z_0 > 0$ for a lower half plane contour.  This is mentioned in the text but I felt it more clear just writing out the exponential as above.

\begin{figure}[htp]
\centering
\includegraphics[totalheight=0.4\textheight]{retardedContourBoth}
\caption{Contours strictly above the $k_0 = 0$ axis}\label{fig:jacksonRet:retardedContourBoth}
\end{figure}

\begin{figure}[htp]
\centering
\includegraphics[totalheight=0.4\textheight]{retardedContourAroundPole}
\caption{Contour around pole}\label{fig:jacksonRet:retardedContourAroundPole}
\end{figure}

Having established the value on the loop at infinity we can now integrate over the contour $r_1$ as depicted in figure (\ref{fig:jacksonRet:retardedContourBoth}).  The problem is mainly reduced to an integral of the form figure (\ref{fig:jacksonRet:retardedContourAroundPole}) around the simple poles at $\alpha = \pm \kappa$

\begin{align}\label{eqn:jacksonRet:boo10}
I_\alpha = \oint \frac{f(z)}{z - \alpha} dz
\end{align}

With $z = \alpha + R e^{i\theta}$, and $\theta \in [\pi/2, 5\pi/2]$, we have

\begin{align}\label{eqn:jacksonRet:boo11}
I_\alpha = \int \frac{f(z)}{R e^{i\theta}} R i e^{i\theta} d\theta
\end{align}

with $R \rightarrow 0$, we are left with 

\begin{align}\label{eqn:jacksonRet:boo12}
I_\alpha = 2 \pi i f(\alpha)
\end{align}

There are six arcs on the contour of interest.  For the first two around the poles lets label the integral contributions $I_\kappa$ and $I_{-\kappa}$.  Along the infinite semicircular contour the integral vanishes with the right sign choice for $z_0$.  For the remainder lets write the integral contributions $I$.

Summing over the complete contour, specially chosen to enclose no poles, we have

\begin{align}\label{eqn:jacksonRet:boo13}
I + I_\kappa + I_{-\kappa} + 0 = 0 
\end{align}

For this $z_0 > 0$ integral we are left with the residue sum

\begin{align*}
\int_{-\infty}^\infty dk_0 \inv{k_0^2 - \Bk^2} e^{-i k_0 z_0} 
&= 
- 2 \pi i \left( 
{\left. \inv{k_0 - \kappa} e^{-i k_0 z_0} \right\vert}_{k_0 = -\kappa}
+{\left. \inv{k_0 + \kappa} e^{-i k_0 z_0} \right\vert}_{k_0 = \kappa}
\right) \\
&= 
\frac{2 \pi i^2}{\kappa} \sin(\kappa z_0)
\end{align*}

Since I can never remember the signs and integral orientations for the residue formula so I've always done it ``manually'' as above picking a zero valued contour.

\begin{figure}[htp]
\centering
\includegraphics[totalheight=0.4\textheight]{retardedContourOnAxis}
\caption{Contour exactly on the $k_0 = 0$ axis?}\label{fig:jacksonRet:retardedContourOnAxis}
\end{figure}

Now, the issue of where to place the contour wasn't really discussed mathematically.  Physically this makes the difference between causal and acausal behavior, but why put the contour strictly above or below the axis and not right on it.  If we put the contour exactly on the $k_0 = 0$ axis as in (\ref{fig:jacksonRet:retardedContourOnAxis}), then our integrals around the two half circular poles gives us a result off by a factor of two?  There is also an (implied) limiting procedure required to place the contour strictly above the axis, and the details of this aren't mentioned (and I also haven't thought them through).  Some of these would be worth thinking through in more detail, but for now lets ignore these.  We are left with

\begin{align}\label{eqn:jacksonRet:boo14}
D(z) = \frac{\theta(z_0)}{(2\pi)^3} \int d^3 k e^{i \Bk \cdot \Bz} \inv{\kappa} \sin(\kappa z_0)
\end{align}

How to reduce this to the single variable integral in $\kappa$ was not immediately clear to me.  Aligning $\Bz$ with the $\Be_3$ axis, and using a spherical polar representation for $\Bk$ we can write $\Bz \cdot \Bk = R \kappa \cos\theta$.  With this and the volume element $d^3 k = \kappa^2 \sin\theta d\theta d\phi d\kappa$, we have

\begin{align}\label{eqn:jacksonRet:boo15}
D(z) = \frac{\theta(z_0)}{(2\pi)^3} \int_0^\infty d\kappa \sin(\kappa z_0) \int_0^{2\pi} d\phi \int_0^\pi d\theta \kappa e^{i R \kappa \cos\theta} \sin\theta
\end{align}

This now happily submits to a nice variable substitution, unlike an integral like $\int e^{i \mu \cos\theta} d\theta = J_0(\Abs{\mu})$ which can be evaluated, but only in terms of Bessel functions or messy series expansion.  Writing $\tau = \kappa \cos\theta$, and $-d\tau = \kappa \sin\theta d\theta$ we have

\begin{align*}
\int_0^\pi d\theta \kappa e^{i R \kappa \cos\theta} \sin\theta
&=
-\int_{\kappa}^{-\kappa} d\tau e^{i R \tau} \\
&=
\frac{e^{i R \kappa}}{i R} -\frac{e^{-i R \kappa}}{i R} \\
&=
2 \inv{R} \sin(R \kappa)
\end{align*}

Our Green's function is now reduced to

\begin{align}\label{eqn:jacksonRet:boo16}
D(z) = \frac{\theta(z_0)}{2 \pi^2 R} \int_0^\infty d\kappa \sin(\kappa z_0) \sin(\kappa R)
\end{align}

Expanding out these sines in terms of exponentials we have

\begin{align*}
D(z) 
&= -\frac{\theta(z_0)}{8 \pi^2 R} \int_0^\infty d\kappa ( e^{i\kappa(z_0+R)} + e^{-i\kappa(z_0+R)} -e^{i\kappa(R-z_0)} - e^{i\kappa(z_0-R)} ) \\
&= -\frac{\theta(z_0)}{8 \pi^2 R} \left(
\int_0^\infty d\kappa \left( e^{i\kappa(z_0+R)} -e^{i\kappa(R-z_0)} \right) 
+\int_0^{-\infty} -d\kappa \left( e^{i\kappa(z_0+R)} - e^{-i\kappa(z_0-R)} \right) 
\right)
\\
&= \frac{\theta(z_0)}{8 \pi^2 R} \int_{-\infty}^\infty d\kappa \left( e^{i\kappa(R-z_0)} -e^{i\kappa(z_0+R)} \right) 
\\
\end{align*}

the sign in this first exponential differs from what Jackson obtained but it won't change the end result.  Did I make a mistake or did he?  Wonder what the third edition shows?  Using $\delta(x) = \int e^{-ikx} dk/2\pi$ we have

\begin{align}\label{eqn:jacksonRet:boo17}
D(z) = \frac{\theta(z_0)}{4 \pi R} \left( \delta(z_0 -R) - \delta(-(z_0 + R)) \right)
\end{align}

With $R = \Abs{\Bx - \Bx'} \ge 0$, and $z_0 = c(t - t') > 0$, this second delta cannot contribute, and we are left with the retarded Green's function

\begin{align}\label{eqn:jacksonRet:boo18}
D_r(z) = \frac{\theta(z_0)}{4 \pi R} \delta(c(t -t') - \Abs{\Bx - \Bx'}) 
\end{align}

Very slick.  I like the procedure, despite a few magic steps (like the choice to offset the contour).

\EndArticle
%\EndNoBibArticle
