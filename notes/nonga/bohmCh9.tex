\documentclass{article}

\usepackage{amsmath}
\usepackage{mathpazo}

%
% shorthand for bold symbols, convenient for vectors and matrices
%
\newcommand{\Ba}[0]{\mathbf{a}}
\newcommand{\Bb}[0]{\mathbf{b}}
\newcommand{\Bc}[0]{\mathbf{c}}
\newcommand{\Bd}[0]{\mathbf{d}}
\newcommand{\Be}[0]{\mathbf{e}}
\newcommand{\Bf}[0]{\mathbf{f}}
\newcommand{\Bg}[0]{\mathbf{g}}
\newcommand{\Bh}[0]{\mathbf{h}}
\newcommand{\Bi}[0]{\mathbf{i}}
\newcommand{\Bj}[0]{\mathbf{j}}
\newcommand{\Bk}[0]{\mathbf{k}}
\newcommand{\Bl}[0]{\mathbf{l}}
\newcommand{\Bm}[0]{\mathbf{m}}
\newcommand{\Bn}[0]{\mathbf{n}}
\newcommand{\Bo}[0]{\mathbf{o}}
\newcommand{\Bp}[0]{\mathbf{p}}
\newcommand{\Bq}[0]{\mathbf{q}}
\newcommand{\Br}[0]{\mathbf{r}}
\newcommand{\Bs}[0]{\mathbf{s}}
\newcommand{\Bt}[0]{\mathbf{t}}
\newcommand{\Bu}[0]{\mathbf{u}}
\newcommand{\Bv}[0]{\mathbf{v}}
\newcommand{\Bw}[0]{\mathbf{w}}
\newcommand{\Bx}[0]{\mathbf{x}}
\newcommand{\By}[0]{\mathbf{y}}
\newcommand{\Bz}[0]{\mathbf{z}}
\newcommand{\BA}[0]{\mathbf{A}}
\newcommand{\BB}[0]{\mathbf{B}}
\newcommand{\BC}[0]{\mathbf{C}}
\newcommand{\BD}[0]{\mathbf{D}}
\newcommand{\BE}[0]{\mathbf{E}}
\newcommand{\BF}[0]{\mathbf{F}}
\newcommand{\BG}[0]{\mathbf{G}}
\newcommand{\BH}[0]{\mathbf{H}}
\newcommand{\BI}[0]{\mathbf{I}}
\newcommand{\BJ}[0]{\mathbf{J}}
\newcommand{\BK}[0]{\mathbf{K}}
\newcommand{\BL}[0]{\mathbf{L}}
\newcommand{\BM}[0]{\mathbf{M}}
\newcommand{\BN}[0]{\mathbf{N}}
\newcommand{\BO}[0]{\mathbf{O}}
\newcommand{\BP}[0]{\mathbf{P}}
\newcommand{\BQ}[0]{\mathbf{Q}}
\newcommand{\BR}[0]{\mathbf{R}}
\newcommand{\BS}[0]{\mathbf{S}}
\newcommand{\BT}[0]{\mathbf{T}}
\newcommand{\BU}[0]{\mathbf{U}}
\newcommand{\BV}[0]{\mathbf{V}}
\newcommand{\BW}[0]{\mathbf{W}}
\newcommand{\BX}[0]{\mathbf{X}}
\newcommand{\BY}[0]{\mathbf{Y}}
\newcommand{\BZ}[0]{\mathbf{Z}}

\newcommand{\Bzero}[0]{\mathbf{0}}
\newcommand{\Btheta}[0]{\boldsymbol{\theta}}
\newcommand{\Btau}[0]{\boldsymbol{\tau}}
\newcommand{\Bomega}[0]{\boldsymbol{\omega}}

%
% shorthand for unit vectors
%
\newcommand{\acap}[0]{\hat{\Ba}}
\newcommand{\bcap}[0]{\hat{\Bb}}
\newcommand{\ccap}[0]{\hat{\Bc}}
\newcommand{\dcap}[0]{\hat{\Bd}}
\newcommand{\ecap}[0]{\hat{\Be}}
\newcommand{\fcap}[0]{\hat{\Bf}}
\newcommand{\gcap}[0]{\hat{\Bg}}
\newcommand{\hcap}[0]{\hat{\Bh}}
\newcommand{\icap}[0]{\hat{\Bi}}
\newcommand{\jcap}[0]{\hat{\Bj}}
\newcommand{\kcap}[0]{\hat{\Bk}}
\newcommand{\lcap}[0]{\hat{\Bl}}
\newcommand{\mcap}[0]{\hat{\Bm}}
\newcommand{\ncap}[0]{\hat{\Bn}}
\newcommand{\ocap}[0]{\hat{\Bo}}
\newcommand{\pcap}[0]{\hat{\Bp}}
\newcommand{\qcap}[0]{\hat{\Bq}}
\newcommand{\rcap}[0]{\hat{\Br}}
\newcommand{\scap}[0]{\hat{\Bs}}
\newcommand{\tcap}[0]{\hat{\Bt}}
\newcommand{\ucap}[0]{\hat{\Bu}}
\newcommand{\vcap}[0]{\hat{\Bv}}
\newcommand{\wcap}[0]{\hat{\Bw}}
\newcommand{\xcap}[0]{\hat{\Bx}}
\newcommand{\ycap}[0]{\hat{\By}}
\newcommand{\zcap}[0]{\hat{\Bz}}
\newcommand{\thetacap}[0]{\hat{\Btheta}}

%
% to write R^n and C^n in a distinguishable fashion.  Perhaps change this
% to the double lined characters upon figuring out how to do so.
%
\newcommand{\C}[1]{$\mathbb{C}^{#1}$}
\newcommand{\R}[1]{$\mathbb{R}^{#1}$}

%
% various generally useful helpers
%

% derivative of #1 wrt. #2:
\newcommand{\D}[2] {\frac {d#2} {d#1}}

\newcommand{\inv}[1]{\frac{1}{#1}}
\newcommand{\cross}[0]{\times}

\newcommand{\abs}[1]{\lvert{#1}\rvert}
\newcommand{\norm}[1]{\lVert{#1}\rVert}
\newcommand{\innerprod}[2]{\langle{#1}, {#2}\rangle}
\newcommand{\dotprod}[2]{{#1} \cdot {#2}}
\newcommand{\bdotprod}[2]{\left({#1} \cdot {#2}\right)}
\newcommand{\crossprod}[2]{{#1} \cross {#2}}
\newcommand{\tripleprod}[3]{\dotprod{\left(\crossprod{#1}{#2}\right)}{#3}}

\DeclareMathOperator{\Proj}{Proj}
\DeclareMathOperator{\Span}{span}
\DeclareMathOperator{\Sgn}{sgn}
\DeclareMathOperator{\Area}{Area}
\DeclareMathOperator{\Volume}{Volume}

%
% A few miscellaneous things specific to this document
%
\newcommand{\crossop}[1]{\crossprod{#1}{}}

% R2 vector.
\newcommand{\VectorTwo}[2]{
\begin{bmatrix}
 {#1} \\
 {#2}
\end{bmatrix}
}

\newcommand{\VectorN}[1]{
\begin{bmatrix}
{#1}_1 \\
{#1}_2 \\
\vdots \\
{#1}_N \\
\end{bmatrix}
}

\newcommand{\DETuvij}[4]{
\begin{vmatrix}
 {#1}_{#3} & {#1}_{#4} \\
 {#2}_{#3} & {#2}_{#4}
\end{vmatrix}
}

\newcommand{\DETuvwijk}[6]{
\begin{vmatrix}
 {#1}_{#4} & {#1}_{#5} & {#1}_{#6} \\
 {#2}_{#4} & {#2}_{#5} & {#2}_{#6} \\
 {#3}_{#4} & {#3}_{#5} & {#3}_{#6}
\end{vmatrix}
}

\newcommand{\DETuvwxijkl}[8]{
\begin{vmatrix}
 {#1}_{#5} & {#1}_{#6} & {#1}_{#7} & {#1}_{#8} \\
 {#2}_{#5} & {#2}_{#6} & {#2}_{#7} & {#2}_{#8} \\
 {#3}_{#5} & {#3}_{#6} & {#3}_{#7} & {#3}_{#8} \\
 {#4}_{#5} & {#4}_{#6} & {#4}_{#7} & {#4}_{#8} \\
\end{vmatrix}
}

%\newcommand{\DETuvwxyijklm}[10]{
%\begin{vmatrix}
% {#1}_{#6} & {#1}_{#7} & {#1}_{#8} & {#1}_{#9} & {#1}_{#10} \\
% {#2}_{#6} & {#2}_{#7} & {#2}_{#8} & {#2}_{#9} & {#2}_{#10} \\
% {#3}_{#6} & {#3}_{#7} & {#3}_{#8} & {#3}_{#9} & {#3}_{#10} \\
% {#4}_{#6} & {#4}_{#7} & {#4}_{#8} & {#4}_{#9} & {#4}_{#10} \\
% {#5}_{#6} & {#5}_{#7} & {#5}_{#8} & {#5}_{#9} & {#5}_{#10}
%\end{vmatrix}
%}

% R3 vector.
\newcommand{\VectorThree}[3]{
\begin{bmatrix}
 {#1} \\
 {#2} \\
 {#3}
\end{bmatrix}
}


%<misc>
%
\newcommand{\Abs}[1]{{\left\lvert{#1}\right\rvert}}
\newcommand{\spacegrad}[0]{\boldsymbol{\nabla}}
\newcommand{\grad}[0]{\nabla}
\newcommand{\LL}[0]{\mathcal{L}}

% == \partial_{#1} {#2}
\newcommand{\PD}[2]{\frac{\partial {#2}}{\partial {#1}}}
% inline variant
\newcommand{\PDi}[2]{{\partial {#2}}/{\partial {#1}}}

\newcommand{\PDD}[3]{\frac{\partial^2 {#3}}{\partial {#1}\partial {#2}}}
%\newcommand{\PDd}[2]{\frac{\partial^2 {#2}}{{\partial{#1}}^2}}
\newcommand{\PDsq}[2]{\frac{\partial^2 {#2}}{(\partial {#1})^2}}

\newcommand{\Partial}[2]{\frac{\partial {#1}}{\partial {#2}}}
\DeclareMathOperator{\RejName}{Rej}
\newcommand{\Rej}[2]{\RejName_{#1}\left( {#2} \right)}
\newcommand{\Rm}[1]{\mathbb{R}^{#1}}
\newcommand{\Cm}[1]{\mathbb{C}^{#1}}
\newcommand{\conj}[0]{{*}}

%</misc>

% <grade selection>
%
\newcommand{\gpgrade}[2] {{\left\langle{{#1}}\right\rangle}_{#2}}

\newcommand{\gpgradezero}[1] {\gpgrade{#1}{}}
%\newcommand{\gpscalargrade}[1] {{\left\langle{{#1}}\right\rangle}}
%\newcommand{\gpgradezero}[1] {\gpgrade{#1}{0}}

%\newcommand{\gpgradeone}[1] {{\left\langle{{#1}}\right\rangle}_{1}}
\newcommand{\gpgradeone}[1] {\gpgrade{#1}{1}}

\newcommand{\gpgradetwo}[1] {\gpgrade{#1}{2}}
\newcommand{\gpgradethree}[1] {\gpgrade{#1}{3}}
\newcommand{\gpgradefour}[1] {\gpgrade{#1}{4}}
%
% </grade selection>



\newcommand{\adot}[0]{{\dot{a}}}
\newcommand{\bdot}[0]{{\dot{b}}}
% taken for centered dot:
%\newcommand{\cdot}[0]{{\dot{c}}}
%\newcommand{\ddot}[0]{{\dot{d}}}
\newcommand{\edot}[0]{{\dot{e}}}
\newcommand{\fdot}[0]{{\dot{f}}}
\newcommand{\gdot}[0]{{\dot{g}}}
\newcommand{\hdot}[0]{{\dot{h}}}
\newcommand{\idot}[0]{{\dot{i}}}
\newcommand{\jdot}[0]{{\dot{j}}}
\newcommand{\kdot}[0]{{\dot{k}}}
\newcommand{\ldot}[0]{{\dot{l}}}
\newcommand{\mdot}[0]{{\dot{m}}}
\newcommand{\ndot}[0]{{\dot{n}}}
%\newcommand{\odot}[0]{{\dot{o}}}
\newcommand{\pdot}[0]{{\dot{p}}}
\newcommand{\qdot}[0]{{\dot{q}}}
\newcommand{\rdot}[0]{{\dot{r}}}
\newcommand{\sdot}[0]{{\dot{s}}}
\newcommand{\tdot}[0]{{\dot{t}}}
\newcommand{\udot}[0]{{\dot{u}}}
\newcommand{\vdot}[0]{{\dot{v}}}
\newcommand{\wdot}[0]{{\dot{w}}}
\newcommand{\xdot}[0]{{\dot{x}}}
\newcommand{\ydot}[0]{{\dot{y}}}
\newcommand{\zdot}[0]{{\dot{z}}}
\newcommand{\addot}[0]{{\ddot{a}}}
\newcommand{\bddot}[0]{{\ddot{b}}}
\newcommand{\cddot}[0]{{\ddot{c}}}
%\newcommand{\dddot}[0]{{\ddot{d}}}
\newcommand{\eddot}[0]{{\ddot{e}}}
\newcommand{\fddot}[0]{{\ddot{f}}}
\newcommand{\gddot}[0]{{\ddot{g}}}
\newcommand{\hddot}[0]{{\ddot{h}}}
\newcommand{\iddot}[0]{{\ddot{i}}}
\newcommand{\jddot}[0]{{\ddot{j}}}
\newcommand{\kddot}[0]{{\ddot{k}}}
\newcommand{\lddot}[0]{{\ddot{l}}}
\newcommand{\mddot}[0]{{\ddot{m}}}
\newcommand{\nddot}[0]{{\ddot{n}}}
\newcommand{\oddot}[0]{{\ddot{o}}}
\newcommand{\pddot}[0]{{\ddot{p}}}
\newcommand{\qddot}[0]{{\ddot{q}}}
\newcommand{\rddot}[0]{{\ddot{r}}}
\newcommand{\sddot}[0]{{\ddot{s}}}
\newcommand{\tddot}[0]{{\ddot{t}}}
\newcommand{\uddot}[0]{{\ddot{u}}}
\newcommand{\vddot}[0]{{\ddot{v}}}
\newcommand{\wddot}[0]{{\ddot{w}}}
\newcommand{\xddot}[0]{{\ddot{x}}}
\newcommand{\yddot}[0]{{\ddot{y}}}
\newcommand{\zddot}[0]{{\ddot{z}}}

%<bold and dot greek symbols>
%

\newcommand{\Deltadot}[0]{{\dot{\Delta}}}
\newcommand{\Gammadot}[0]{{\dot{\Gamma}}}
\newcommand{\Lambdadot}[0]{{\dot{\Lambda}}}
\newcommand{\Omegadot}[0]{{\dot{\Omega}}}
\newcommand{\Phidot}[0]{{\dot{\Phi}}}
\newcommand{\Pidot}[0]{{\dot{\Pi}}}
\newcommand{\Psidot}[0]{{\dot{\Psi}}}
\newcommand{\Sigmadot}[0]{{\dot{\Sigma}}}
\newcommand{\Thetadot}[0]{{\dot{\Theta}}}
\newcommand{\Upsilondot}[0]{{\dot{\Upsilon}}}
\newcommand{\Xidot}[0]{{\dot{\Xi}}}
\newcommand{\alphadot}[0]{{\dot{\alpha}}}
\newcommand{\betadot}[0]{{\dot{\beta}}}
\newcommand{\chidot}[0]{{\dot{\chi}}}
\newcommand{\deltadot}[0]{{\dot{\delta}}}
\newcommand{\epsilondot}[0]{{\dot{\epsilon}}}
\newcommand{\etadot}[0]{{\dot{\eta}}}
\newcommand{\gammadot}[0]{{\dot{\gamma}}}
\newcommand{\kappadot}[0]{{\dot{\kappa}}}
\newcommand{\lambdadot}[0]{{\dot{\lambda}}}
\newcommand{\mudot}[0]{{\dot{\mu}}}
\newcommand{\nudot}[0]{{\dot{\nu}}}
\newcommand{\omegadot}[0]{{\dot{\omega}}}
\newcommand{\phidot}[0]{{\dot{\phi}}}
\newcommand{\pidot}[0]{{\dot{\pi}}}
\newcommand{\psidot}[0]{{\dot{\psi}}}
\newcommand{\rhodot}[0]{{\dot{\rho}}}
\newcommand{\sigmadot}[0]{{\dot{\sigma}}}
\newcommand{\taudot}[0]{{\dot{\tau}}}
\newcommand{\thetadot}[0]{{\dot{\theta}}}
\newcommand{\upsilondot}[0]{{\dot{\upsilon}}}
\newcommand{\varepsilondot}[0]{{\dot{\varepsilon}}}
\newcommand{\varphidot}[0]{{\dot{\varphi}}}
\newcommand{\varpidot}[0]{{\dot{\varpi}}}
\newcommand{\varrhodot}[0]{{\dot{\varrho}}}
\newcommand{\varsigmadot}[0]{{\dot{\varsigma}}}
\newcommand{\varthetadot}[0]{{\dot{\vartheta}}}
\newcommand{\xidot}[0]{{\dot{\xi}}}
\newcommand{\zetadot}[0]{{\dot{\zeta}}}

\newcommand{\Deltaddot}[0]{{\ddot{\Delta}}}
\newcommand{\Gammaddot}[0]{{\ddot{\Gamma}}}
\newcommand{\Lambdaddot}[0]{{\ddot{\Lambda}}}
\newcommand{\Omegaddot}[0]{{\ddot{\Omega}}}
\newcommand{\Phiddot}[0]{{\ddot{\Phi}}}
\newcommand{\Piddot}[0]{{\ddot{\Pi}}}
\newcommand{\Psiddot}[0]{{\ddot{\Psi}}}
\newcommand{\Sigmaddot}[0]{{\ddot{\Sigma}}}
\newcommand{\Thetaddot}[0]{{\ddot{\Theta}}}
\newcommand{\Upsilonddot}[0]{{\ddot{\Upsilon}}}
\newcommand{\Xiddot}[0]{{\ddot{\Xi}}}
\newcommand{\alphaddot}[0]{{\ddot{\alpha}}}
\newcommand{\betaddot}[0]{{\ddot{\beta}}}
\newcommand{\chiddot}[0]{{\ddot{\chi}}}
\newcommand{\deltaddot}[0]{{\ddot{\delta}}}
\newcommand{\epsilonddot}[0]{{\ddot{\epsilon}}}
\newcommand{\etaddot}[0]{{\ddot{\eta}}}
\newcommand{\gammaddot}[0]{{\ddot{\gamma}}}
\newcommand{\kappaddot}[0]{{\ddot{\kappa}}}
\newcommand{\lambdaddot}[0]{{\ddot{\lambda}}}
\newcommand{\muddot}[0]{{\ddot{\mu}}}
\newcommand{\nuddot}[0]{{\ddot{\nu}}}
\newcommand{\omegaddot}[0]{{\ddot{\omega}}}
\newcommand{\phiddot}[0]{{\ddot{\phi}}}
\newcommand{\piddot}[0]{{\ddot{\pi}}}
\newcommand{\psiddot}[0]{{\ddot{\psi}}}
\newcommand{\rhoddot}[0]{{\ddot{\rho}}}
\newcommand{\sigmaddot}[0]{{\ddot{\sigma}}}
\newcommand{\tauddot}[0]{{\ddot{\tau}}}
\newcommand{\thetaddot}[0]{{\ddot{\theta}}}
\newcommand{\upsilonddot}[0]{{\ddot{\upsilon}}}
\newcommand{\varepsilonddot}[0]{{\ddot{\varepsilon}}}
\newcommand{\varphiddot}[0]{{\ddot{\varphi}}}
\newcommand{\varpiddot}[0]{{\ddot{\varpi}}}
\newcommand{\varrhoddot}[0]{{\ddot{\varrho}}}
\newcommand{\varsigmaddot}[0]{{\ddot{\varsigma}}}
\newcommand{\varthetaddot}[0]{{\ddot{\vartheta}}}
\newcommand{\xiddot}[0]{{\ddot{\xi}}}
\newcommand{\zetaddot}[0]{{\ddot{\zeta}}}

\newcommand{\BDelta}[0]{\boldsymbol{\Delta}}
\newcommand{\BGamma}[0]{\boldsymbol{\Gamma}}
\newcommand{\BLambda}[0]{\boldsymbol{\Lambda}}
\newcommand{\BOmega}[0]{\boldsymbol{\Omega}}
\newcommand{\BPhi}[0]{\boldsymbol{\Phi}}
\newcommand{\BPi}[0]{\boldsymbol{\Pi}}
\newcommand{\BPsi}[0]{\boldsymbol{\Psi}}
\newcommand{\BSigma}[0]{\boldsymbol{\Sigma}}
\newcommand{\BTheta}[0]{\boldsymbol{\Theta}}
\newcommand{\BUpsilon}[0]{\boldsymbol{\Upsilon}}
\newcommand{\BXi}[0]{\boldsymbol{\Xi}}
\newcommand{\Balpha}[0]{\boldsymbol{\alpha}}
\newcommand{\Bbeta}[0]{\boldsymbol{\beta}}
\newcommand{\Bchi}[0]{\boldsymbol{\chi}}
\newcommand{\Bdelta}[0]{\boldsymbol{\delta}}
\newcommand{\Bepsilon}[0]{\boldsymbol{\epsilon}}
\newcommand{\Beta}[0]{\boldsymbol{\eta}}
\newcommand{\Bgamma}[0]{\boldsymbol{\gamma}}
\newcommand{\Bkappa}[0]{\boldsymbol{\kappa}}
\newcommand{\Blambda}[0]{\boldsymbol{\lambda}}
\newcommand{\Bmu}[0]{\boldsymbol{\mu}}
\newcommand{\Bnu}[0]{\boldsymbol{\nu}}
%\newcommand{\Bomega}[0]{\boldsymbol{\omega}}
\newcommand{\Bphi}[0]{\boldsymbol{\phi}}
\newcommand{\Bpi}[0]{\boldsymbol{\pi}}
\newcommand{\Bpsi}[0]{\boldsymbol{\psi}}
\newcommand{\Brho}[0]{\boldsymbol{\rho}}
\newcommand{\Bsigma}[0]{\boldsymbol{\sigma}}
%\newcommand{\Btau}[0]{\boldsymbol{\tau}}
%\newcommand{\Btheta}[0]{\boldsymbol{\theta}}
\newcommand{\Bupsilon}[0]{\boldsymbol{\upsilon}}
\newcommand{\Bvarepsilon}[0]{\boldsymbol{\varepsilon}}
\newcommand{\Bvarphi}[0]{\boldsymbol{\varphi}}
\newcommand{\Bvarpi}[0]{\boldsymbol{\varpi}}
\newcommand{\Bvarrho}[0]{\boldsymbol{\varrho}}
\newcommand{\Bvarsigma}[0]{\boldsymbol{\varsigma}}
\newcommand{\Bvartheta}[0]{\boldsymbol{\vartheta}}
\newcommand{\Bxi}[0]{\boldsymbol{\xi}}
\newcommand{\Bzeta}[0]{\boldsymbol{\zeta}}
%
%</bold and dot greek symbols>
%<infrequent>
%
%\newcommand{\AreaOp}[1]{\AName_{#1}}
%\newcommand{\Babs}[0]{\abs{\BB}}
%\newcommand{\Bcap}[0]{\hat{\BB}}
%\newcommand{\BrPrimeRej}[0]{\rcap(\rcap \wedge \Br')}
%\newcommand{\CA}[0]{\mathcal{A}}
%\newcommand{\Cos}[1]{\cos{\left({#1}\right)}}
%\newcommand{\Det}[1] {\abs{#1}}
%\newcommand{\Dsq}[2] {\frac {\partial^2 {#1}} {\partial {#2}^2}}
%\newcommand{\Exp}[1]{\exp{\left({#1}\right)}}
%\newcommand{\Norm}[1]{\left\lVert{#1}\right\rVert}
%\newcommand{\Sin}[1]{\sin{\left({#1}\right)}}
%\newcommand{\T}[0]{\text{T}}
%\newcommand{\VolumeOp}[1]{\VName_{#1}}
%\newcommand{\agrad}[0]{\Ba \cdot \nabla}
%\newcommand{\alphacap}[0]{\hat{\boldsymbol{\alpha}}}
%\newcommand{\Fcap}[0]{\hat{\BF}}
%\newcommand{\bithree}[0]{{\Bi}_3}
%\newcommand{\bxa}[0]{\Bx\Ba}
%\newcommand{\coordvec}[2]{
%\newcommand{\costheta}[0]{\acap \cdot \xcap}
%\newcommand{\ddt}[1]{\ddot{#1}}
%\newcommand{\ddu}[1] {\frac {d{#1}} {du}}
%\newcommand{\dsqxj}[2] {\frac {\partial^2 {#1}} {\partial {x_{#2}}^2}}
%\newcommand{\dtheta}[1]{\frac{d {#1}}{d \theta}}
%\newcommand{\dt}[1]{\dot{#1}}
%\newcommand{\dt}[1]{\frac{d {#1}}{dt}}
%\newcommand{\dxj}[2] {\frac {\partial {#1}} {\partial {x_{#2}}}}
%\newcommand{\halfPhi}[0]{\frac{\phi}{2}}
%\newcommand{\half}[0]{\inv{2}}
%\newcommand{\inv}[1]{\frac{1}{#1}}
%\newcommand{\laplacian}[0]{\nabla^2}
%\newcommand{\matrixoftx}[3]{
%\newcommand{\nrrp}[0]{\norm{\rcap \wedge \Br'}}
%\newcommand{\oiint}{\bigcirc \hspace{-1.4em} \int \hspace{-.8em} \int}
%\newcommand{\transpose}[1]{{#1}^{\text{T}}}
%\newcommand{\transpose}[1]{{{#1}^{\TextTranspose}}}
%\newcommand{\transpose}[1]{{{#1}^{\text{T}}}}
%\newcommand{\barA}[0]{\bar{A}}
%\newcommand{\qbar}[0]{\bar{q}}
%\newcommand{\qdotbar}[0]{\dot{\bar{q}}}
%
%</infrequent>





\newcommand{\PDSq}[2]{\frac{\partial^2 {#2}}{\partial {#1}^2}}
\newcommand{\PDN}[3]{\frac{\partial^{#3} {#2}}{\partial {#1}^{#3}}}
\DeclareMathOperator{\sinc}{sinc}
\DeclareMathOperator{\PV}{PV}
\newcommand{\FF}[0]{\mathcal{F}}
\newcommand{\Sw}[0]{\mathcal{S}}
\newcommand{\IIinf}[0]{ \int_{-\infty}^\infty }
\newcommand{\FM}[0]{\inv{\sqrt{2\pi\hbar}}}

\usepackage[bookmarks=true]{hyperref}

\usepackage{color,cite,graphicx}
   % use colour in the document, put your citations as [1-4]
   % rather than [1,2,3,4] (it looks nicer, and the extended LaTeX2e
   % graphics package. 
\usepackage{latexsym,amssymb,epsf} % don't remember if these are
   % needed, but their inclusion can't do any damage


\title{ Bohm Chapter 9 problems. }
\author{Peeter Joot \quad peeter.joot@gmail.com }
\date{ March 6, 2009.  Last Revision: $Date: 2009/03/09 01:32:52 $ }

\begin{document}

\maketitle{}

\tableofcontents

\section{ Bohm Chapter 9 problems. }

Problems and additional details from reading of \cite{bohm1989qt}, chapter 9.

\subsection{ P1. Momentum wave function normalization. }

Given a normalized wave function

\begin{align*}
\IIinf \psi^\conj(x) \psi(x) dx = 1
\end{align*}

Show that the wave function $\phi(k)$ is also normalized, and find the normalization factor for $\Phi(p)$.

\begin{align*}
\IIinf \phi^\conj(k) \phi(k) dk 
&= 
\IIinf \phi^\conj(k) \left( \inv{\sqrt{2\pi}} \IIinf \psi(x) e^{-i k x} dx \right) dk  \\
&= 
\IIinf \left( \inv{\sqrt{2\pi}} \IIinf \phi^\conj(k) e^{-i k x} dk \right) \psi(x) dx  \\
&= 
\IIinf {\left( \inv{\sqrt{2\pi}} \IIinf \phi(k) e^{i k x} dk \right)}^\conj \psi(x) dx  \\
&= 
\IIinf \psi^\conj(x) \psi(x) dx  \\
&= 1 \quad\quad\quad \square
\end{align*}

Bohm defines $\Phi(p) \propto \phi(k)$ with the normalization constant determined by $\int \Abs{\Phi(p)} dp = 1$.  Suppose we 
let $\Phi(p) = \alpha \phi(k)$, then we have

\begin{align*}
1 
&= \int \Phi^\conj(p) \Phi(p) dp \\
&= \int \alpha^2 \phi^\conj(k) \phi(k) \hbar d k
\end{align*}

So we want $\alpha^2 \hbar = 1$, and therefore $\Phi(p) = \inv{\sqrt{\hbar}} \phi(k)$.

In \cite{mcmahon2005qmd}, with followup in \cite{PJqmFourier} we've seen that an alternate Fourier transform pair can be used in terms of
momentum variables.  That is

\begin{align*}
\Phi(p) &= \FM \IIinf \psi(x) e^{-ipx/\hbar} dx \\
\psi(x) &= \FM \IIinf \Phi(p) e^{ipx/\hbar} dp \\
\end{align*}

Observe that this is consistent with Bohm's notation, since one can read off 
$\Phi(p)$ in terms of $\phi(k)$.
by inspection

\begin{align*}
\Phi(p) &= \FM \IIinf \psi(x) e^{-ipx/\hbar} dx = \inv{\sqrt{\hbar}} \phi(k)
\end{align*}

\subsection{ P2. Expectation of polynomial momentum function. }

Given a function of momentum 

\begin{align*}
f(p) &= \sum C_n p^n
\end{align*}

Express the average, or expectation value of $f(p)$.  It is sufficient to consider one of the monomial terms, say $p^n$.  A translation 
to position basis via fourier transformation produces the desired result

\begin{align*}
\bar{p^n} 
&= \int \Phi^\conj(p) p^n \Phi(p) dp \\
&= \inv{2\pi \hbar} \iiint \left( \psi^\conj(x') e^{ipx'/\hbar} dx' \right) (\hbar k)^n \left( \psi(x) e^{-ipx/\hbar} dx \right) (\hbar dk) \\
&= \frac{\hbar^n}{2\pi } \iiint \psi^\conj(x') e^{ikx'} dx' k^n e^{-i k x} \psi(x) dx dk \\
\end{align*}

The $k^n$ can be reduced to differential form as Bohm did for the $\bar{p}$ case

\begin{align*}
k^n e^{-i k x} 
&= k^{n-1} k e^{-i k x} \\
&= k^{n-1} i \PD{x}{}e^{-i k x} \\
&= k^{n-m} i^m \PDN{x}{}{m}e^{-i k x} \\
&= i^n \PDN{x}{}{n}e^{-i k x} \\
\end{align*}

This leaves something that's in shape for integration by parts

\begin{align*}
\bar{p^n} 
&= \frac{(i\hbar)^n}{2\pi} \iiint \psi^\conj(x') e^{ikx'} dx' \left( \PDN{x}{}{n}e^{-i k x} \right) \psi(x) dx dk \\
&= \frac{(-i\hbar)^n}{2\pi} \iiint \psi^\conj(x') e^{ikx'} dx' \PDN{x}{\psi(x)}{n} e^{-i k x} dx dk \\
&= \frac{(-i\hbar)^n}{2\pi} \iiint \psi^\conj(x') e^{ik(x'-x)} \PDN{x}{\psi(x)}{n} dx' dx dk \\
&= {(-i\hbar)^n}{} \iint \psi^\conj(x') \PDN{x}{\psi(x)}{n} dx' dx \inv{2\pi}\int e^{ik(x'-x)} dk \\
\end{align*}

This last integral is really a distribution, and can be identified with the delta function $\delta(x'-x)$ operating on, in this case, the preceding integral.
%, and operates on a test function.  Suppose we designate such a test function as $a(u)$, then we have
%
%\int \inv{2\pi}\int e^{iku} dk a(u) du
%&= \int \inv{2\pi}\int e^{iku} a(u) du dk \\

%Now, we can apply the distribution theory as covered in \cite{osgoodFourier} to do a delta function reduction of the exponentials in 
%this integral.  Specifically pick a function $a(k)$

So we have
\begin{align*}
\bar{p^n} 
&= {(-i\hbar)^n}{} \iint \psi^\conj(x') \PDN{x}{\psi(x)}{n} dx' dx \delta(x'-x) \\
&= {(-i\hbar)^n}{} \int \psi^\conj(x) \PDN{x}{\psi(x)}{n} dx \\
\end{align*}

We can put this into explicit operator form, nicely motivating the identification of $-i\hbar \PDi{x}{}$ with the momentum by virtue 
of the definition of the average or expectation value.

\begin{align*}
\bar{p^n} 
&= \int \psi^\conj(x) {\left( -i \hbar \PD{x}{} \right)}^n {\psi(x)} dx \\
\end{align*}

\subsection{ P3.  Expection of position in momentum space. }

\begin{align*}
\bar{x} 
&= \int \psi^\conj(x) x \psi(x) dx \\
&= 
\inv{2\pi\hbar} \iiint \Phi^\conj(p) e^{-ipx/\hbar} dp x \Phi(p') e^{ip'x/\hbar} dp' dx \\
&= 
\inv{2\pi\hbar} \iiint \Phi^\conj(p) e^{-ipx/\hbar} dp \left( -i \PD{p'}{} e^{ip'x/\hbar} \right) \Phi(p') dp' dx \\
&= 
\inv{2\pi\hbar} \iiint \Phi^\conj(p) e^{-ipx/\hbar} dp \left( i \PD{p'}{\Phi(p')} \right) e^{ip'x/\hbar} dp' dx \\
&= 
\iint \Phi^\conj(p) \left( i \PD{p'}{} \right) {\Phi(p')} dp dp' \inv{2\pi\hbar} \int e^{i(p'-p)x/\hbar} dx  \\
&= 
\iint \Phi^\conj(p) \left( i \PD{p'}{} \right) {\Phi(p')} dp dp' \delta(p'-p)  \\
\end{align*}

This is

\begin{align*}
\bar{x} &= \int \Phi^\conj(p) \left( i \PD{p}{} \right) {\Phi(p)} dp 
\end{align*}

We see that expressing momentum in position space and position in momentum space both result in differential
operator forms in calculations of expected values

\begin{align}\label{eqn:operatorCorrespondance}
p &\sim -i \hbar \PD{x}{} \\
x &\sim i \hbar \PD{p}{}
\end{align}

Observe the Hamiltonian and Poisson equation structure in these two sets of operators.

\subsection{ P4. Expectation of polynomial position function. }

This problem follows just as P2, and I'm not going to bother typing it up for myself.  For validity, we require
$x^n \phi(x) \rightarrow 0$ as $x \rightarrow \pm \infty$, or equivalently that $\PDN{p}{\Phi}{n} \rightarrow 0$.

\subsection{ P5. Some commutator calculations. }

\subsubsection{ P5. Position momentum moment commutators. }

Evaluate

\begin{align*}
f(x,p) = x^n p^m - p^m x^n
\end{align*}

Up to this point we've only seen operators in expectation values.  Let's look the simplest case with $n = m = 1$ in that
context

\begin{align*}
\bar{f} 
&= \frac{\hbar}{i} \int \psi^\conj(x) \left(x \PD{x}{} - \PD{x}{} x \right) \psi(x) dx \\
&= \frac{\hbar}{i} \int \psi^\conj(x) \left(x \PD{x}{\psi(x)} - \psi(x) - x \PD{x}{\psi(x)} \right) dx \\
&= -\frac{\hbar}{i} \int \psi^\conj(x) \psi(x) dx \\
&= {i\hbar}
\end{align*}

So in the same way that the operator correspondance between momentum and the derivative as summarized in 
\ref{eqn:operatorCorrespondance}, one can associate the commutator operator with its action in the expectation value and
say

\begin{align}\label{eqn:commutator}
x p - p x \sim  i\hbar
\end{align}

The higher order commutator expansions could also be evaluated this way, but exploiting the operator nature directly 
makes this easier.  For the first order moment commutator above one can write

\begin{align*}
f(x,p) \psi(x) 
&= (x p - p x) \psi(x) \\
&= -i \hbar \left(x \PD{x}{} - \PD{x}{} x\right) \psi(x) \\
&= -i \hbar \left(x \PD{x}{\psi}(x) - \PD{x}{x \psi(x)} \right) \\
&= -i \hbar \left(x \PD{x}{\psi}(x) - \PD{x}{\psi(x)} -\psi(x) \right) \\
&= i \hbar \psi(x) \\
\end{align*}

So again we see that as a right acting operator the net effect on any wave function is the following action

\begin{align*}
(x p - p x) \psi = i \hbar \psi \\
\end{align*}

If one starts from this point and then calculates the expectation value the result will still be $i \hbar$, but working
with the probability integrals from the get go is just additional complication.

Building on this result we can then calculate the higher order moment differences of the problem by using the commutator
to change the order of operations

\begin{align*}
p x \sim -i \hbar + x p
\end{align*}

Let's use this for a couple simple examples to start
\begin{align*}
x^2 p - p x^2
&=
x^2 p - ( -i \hbar + x p) x \\
&=
x^2 p + i \hbar x - x ( -i \hbar + x p) \\
&=
x^2 p + 2 i \hbar x - x^2 p \\
&=
2 i \hbar x \\
\end{align*}

\begin{align*}
x p^2 - p^2 x
&=
x p^2 - p ( -i \hbar + x p) \\
&=
x p^2 + i \hbar p - p x p \\
&=
x p^2 + i \hbar p + ( +i \hbar - x p) p \\
&=
2 i \hbar p \\
\end{align*}

Calculation of third powers shows a pattern, and one can guess at an induction hypothesis

\begin{align*}
x p^n &= p^n x + n i \hbar p^{n-1} \\
-p x^n &= -x^n p + n i \hbar x^{n-1} \\
\end{align*}

The $n=1$ cases follow from $xp - px = i\hbar$, leaving only the induction on $n$.  For the momentum powers we have

\begin{align*}
x p^n p 
&= p^n x p + n i \hbar p^{n} \\
&= p^n (p x + i \hbar) + n i \hbar p^{n} \\
&= p^{n+1} x + (n+1) i \hbar p^{n} \quad\quad\quad\square \\
\end{align*}

For the position powers we have
\begin{align*}
-p x^n x 
&= -x^n p x + n i \hbar x^{n} \\
&= x^n (-x p + i \hbar) + n i \hbar x^{n} \\
&= -x^{n+1} p + (n+1) i \hbar x^{n} \quad\quad\quad\square \\
\end{align*}

This completes the proof for a first order version of the problem

\begin{align*}
x p^n - p^n x &= n i \hbar p^{n-1} \\
x^n p -p x^n &=  n i \hbar x^{n-1} \\
\end{align*}

Observe that working with the operator form changes the calculation of derivatives problem in the original
commutator evaluation to nothing more than an algebraic exersize.
% (but one where there's a requirement to not accidentally invert the product order).

The general case still remains.  Building up to that let's do a couple examples

%x p^n = 
%p^n x +  n i \hbar p^{n-1} \\
%
%x^n p = 
%p x^n  +  n i \hbar x^{n-1} \\
%
%x^{n-1} p = 
%(p x^{n-1}  +  (n-1) i \hbar x^{n-2})
%x^{n-2} p = 
%(p x^{n-2}  +  (n-2) i \hbar x^{n-3})

\begin{align*}
x^n p^2
&= (x^n p) p \\
&= (p x^n  +  n i \hbar x^{n-1} ) p \\
&= p (x^n p) +  n i \hbar (x^{n-1} p ) \\
&= p^2 x^n  +  2 n i \hbar p x^{n-1} +  n (n-1) (i \hbar)^2 x^{n-2} \\
\end{align*}

\begin{align*}
x^n p^3
&=
(x^n p^2) p \\
&=
(p^2 x^n  +  2 n i \hbar p x^{n-1} +  n (n-1) (i \hbar)^2 x^{n-2} ) p \\
&=
  p^2 ( p x^n  +  n i \hbar x^{n-1} ) 
+ 2 n i \hbar p ( p x^{n-1}  +  (n-1) i \hbar x^{n-2} ) 
+ n (n-1) (i \hbar)^2 ( p x^{n-2}  +  (n-2) i \hbar x^{n-3}) 
\\
&=
  p^3 x^n
+ 3 n (i \hbar) p^2 x^{n-1}  
+ 3 n(n-1) (i \hbar)^2 p x^{n-2} 
+ n (n-1)(n-2) (i \hbar)^3 x^{n-3}
\\
\end{align*}

We see what looks like binomial coefficients, so a reasonable inductive hypothesis, for $m \le n$

\begin{align}\label{eqn:commutatorMomentMlessThanN}
x^n p^m
&= \sum_{j=0}^m \binom{m}{j} (i \hbar)^j p^{m-j} x^{n-j} (n)(n-1)\cdots(n-j+1) \\
\end{align}

And in particular, for $m \le n$
\begin{align}
x^n p^m - p^m x^n
&= \sum_{j=1}^m \binom{m}{j} (i \hbar)^j p^{m-j} x^{n-j} (n)(n-1)\cdots(n-j+1) 
\end{align}

For $m \ge n$, let's start with

\begin{align*}
p^m x = x p^m -  m i \hbar p^{m-1} 
\end{align*}
%p^m x = x p^m -  m (i \hbar) p^{m-1} 
%p^{m-1} x = x p^{m-1} -  (m-1) (i \hbar) p^{m-2} 
%p^{m-2} x = x p^{m-2} -  (m-2) (i \hbar) p^{m-3} 

First do the $x^2$
\begin{align*}
p^m x^2 
&= x p^m x -  m (i \hbar) p^{m-1} x \\
&= x (p^m x) -  m (i \hbar) (p^{m-1} x) \\
&= x^2 p^m - 2 m (i \hbar) x p^{m-1} +  m (m-1)(i \hbar)^2 p^{m-2} \\
\end{align*}

And for the cube $x^3$
\begin{align*}
p^m x^3  
&= 
( p^m x^2 ) x \\
&= 
( x^2 p^m - 2 m (i \hbar) x p^{m-1} +  m (m-1)(i \hbar)^2 p^{m-2} ) x \\
&= 
x^2 (p^m x )
- 2 m (i \hbar) x (p^{m-1} x )
+ m (m-1)(i \hbar)^2 (p^{m-2} x) \\
&= 
x^2 ( x p^m -  m (i \hbar) p^{m-1} ) \\
&\quad- 2 m (i \hbar) x ( x p^{m-1} -  (m-1) (i \hbar) p^{m-2} ) \\
&\quad+ m (m-1)(i \hbar)^2 ( x p^{m-2} -  (m-2) (i \hbar) p^{m-3} ) \\
&= 
x^3 p^m 
- 3 m (i \hbar) x^2 p^{m-1} 
+ 3 m (m-1) (i \hbar)^2 x p^{m-2} 
- m (m-1)(i \hbar)^2 (m-2) (i \hbar) p^{m-3} \\
\end{align*}

It appears that in this case with $m \ge n$, like \ref{eqn:commutatorMomentMlessThanN}, we want as the induction statement 

\begin{align}
p^m x^n &= \sum_{j=0}^n \binom{n}{j} (-i \hbar)^j x^{n-j} p^{m-j} (m)(m-1)\cdots(m-j+1) 
\end{align}

And for the commutator moment the expected result, pending induction on the above, is
\begin{align}
x^n p^m - p^m x^n &= -\sum_{j=1}^n \binom{n}{j} (-i \hbar)^j x^{n-j} p^{m-j} (m)(m-1)\cdots(m-j+1) 
\end{align}

Summarizing, this is
\begin{align*}
&x^n p^m - p^m x^n
= \\
&\left\{
\begin{array}{l l}
\sum_{j=1}^m \binom{m}{j} (i \hbar)^j p^{m-j} x^{n-j} (n)(n-1)\cdots(n-j+1) & \quad \mbox{if $m \le n$} \\
-\sum_{j=1}^n \binom{n}{j} (-i \hbar)^j x^{n-j} p^{m-j} (m)(m-1)\cdots(m-j+1) & \quad \mbox{if $m \ge n$}
\end{array}
\right.
\end{align*}

\subsubsection{ P5.b }

\begin{align*}
e^{i k x} p - 
p e^{ i k x}
\end{align*}

Reversing the second term via power series expansion we have

\begin{align*}
p e^{ i k x}
&=
p \sum_{n=0}^\infty \frac{( i k x )^n}{n!} \\
&=
\sum_{n=0}^\infty \frac{( i k )^n}{n!} (x^n p - n i \hbar x^{n-1} )
\\
&=
e^{ i k x} p
-\sum_{n=1}^\infty \frac{( i k )^n}{n!} (n i \hbar x^{n-1} )
\\
&=
e^{ i k x} p
-(i k)(i\hbar) \sum_{n=1}^\infty \frac{( i k x)^{n-1}}{(n-1)!} 
\\
&=
e^{ i k x} p
+ (k\hbar) e^{i k x}
\\
\end{align*}

So we have

\begin{align*}
e^{ i k x} p - p e^{ i k x} &= - (k\hbar) e^{i k x}
\end{align*}

\subsection{ P6. Hermitian operators. Powers of momentum operators. }

TODO.

\subsection{ P7. Hermitian operators. Powers of position operators. }

TODO.

\subsection{ P8. Non Hermitian momentum power operators if derivative doesn't vanish. }

TODO.

\bibliographystyle{plainnat}
\bibliography{myrefs}

\end{document}
