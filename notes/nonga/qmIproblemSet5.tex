%\documentclass[]{eliblog}
%
%\usepackage{color}
%%\usepackage{txfonts} % for xi
%
%\usepackage{amsmath}
\usepackage{mathpazo}

%
% shorthand for bold symbols, convenient for vectors and matrices
%
\newcommand{\Ba}[0]{\mathbf{a}}
\newcommand{\Bb}[0]{\mathbf{b}}
\newcommand{\Bc}[0]{\mathbf{c}}
\newcommand{\Bd}[0]{\mathbf{d}}
\newcommand{\Be}[0]{\mathbf{e}}
\newcommand{\Bf}[0]{\mathbf{f}}
\newcommand{\Bg}[0]{\mathbf{g}}
\newcommand{\Bh}[0]{\mathbf{h}}
\newcommand{\Bi}[0]{\mathbf{i}}
\newcommand{\Bj}[0]{\mathbf{j}}
\newcommand{\Bk}[0]{\mathbf{k}}
\newcommand{\Bl}[0]{\mathbf{l}}
\newcommand{\Bm}[0]{\mathbf{m}}
\newcommand{\Bn}[0]{\mathbf{n}}
\newcommand{\Bo}[0]{\mathbf{o}}
\newcommand{\Bp}[0]{\mathbf{p}}
\newcommand{\Bq}[0]{\mathbf{q}}
\newcommand{\Br}[0]{\mathbf{r}}
\newcommand{\Bs}[0]{\mathbf{s}}
\newcommand{\Bt}[0]{\mathbf{t}}
\newcommand{\Bu}[0]{\mathbf{u}}
\newcommand{\Bv}[0]{\mathbf{v}}
\newcommand{\Bw}[0]{\mathbf{w}}
\newcommand{\Bx}[0]{\mathbf{x}}
\newcommand{\By}[0]{\mathbf{y}}
\newcommand{\Bz}[0]{\mathbf{z}}
\newcommand{\BA}[0]{\mathbf{A}}
\newcommand{\BB}[0]{\mathbf{B}}
\newcommand{\BC}[0]{\mathbf{C}}
\newcommand{\BD}[0]{\mathbf{D}}
\newcommand{\BE}[0]{\mathbf{E}}
\newcommand{\BF}[0]{\mathbf{F}}
\newcommand{\BG}[0]{\mathbf{G}}
\newcommand{\BH}[0]{\mathbf{H}}
\newcommand{\BI}[0]{\mathbf{I}}
\newcommand{\BJ}[0]{\mathbf{J}}
\newcommand{\BK}[0]{\mathbf{K}}
\newcommand{\BL}[0]{\mathbf{L}}
\newcommand{\BM}[0]{\mathbf{M}}
\newcommand{\BN}[0]{\mathbf{N}}
\newcommand{\BO}[0]{\mathbf{O}}
\newcommand{\BP}[0]{\mathbf{P}}
\newcommand{\BQ}[0]{\mathbf{Q}}
\newcommand{\BR}[0]{\mathbf{R}}
\newcommand{\BS}[0]{\mathbf{S}}
\newcommand{\BT}[0]{\mathbf{T}}
\newcommand{\BU}[0]{\mathbf{U}}
\newcommand{\BV}[0]{\mathbf{V}}
\newcommand{\BW}[0]{\mathbf{W}}
\newcommand{\BX}[0]{\mathbf{X}}
\newcommand{\BY}[0]{\mathbf{Y}}
\newcommand{\BZ}[0]{\mathbf{Z}}

\newcommand{\Bzero}[0]{\mathbf{0}}
\newcommand{\Btheta}[0]{\boldsymbol{\theta}}
\newcommand{\Btau}[0]{\boldsymbol{\tau}}
\newcommand{\Bomega}[0]{\boldsymbol{\omega}}

%
% shorthand for unit vectors
%
\newcommand{\acap}[0]{\hat{\Ba}}
\newcommand{\bcap}[0]{\hat{\Bb}}
\newcommand{\ccap}[0]{\hat{\Bc}}
\newcommand{\dcap}[0]{\hat{\Bd}}
\newcommand{\ecap}[0]{\hat{\Be}}
\newcommand{\fcap}[0]{\hat{\Bf}}
\newcommand{\gcap}[0]{\hat{\Bg}}
\newcommand{\hcap}[0]{\hat{\Bh}}
\newcommand{\icap}[0]{\hat{\Bi}}
\newcommand{\jcap}[0]{\hat{\Bj}}
\newcommand{\kcap}[0]{\hat{\Bk}}
\newcommand{\lcap}[0]{\hat{\Bl}}
\newcommand{\mcap}[0]{\hat{\Bm}}
\newcommand{\ncap}[0]{\hat{\Bn}}
\newcommand{\ocap}[0]{\hat{\Bo}}
\newcommand{\pcap}[0]{\hat{\Bp}}
\newcommand{\qcap}[0]{\hat{\Bq}}
\newcommand{\rcap}[0]{\hat{\Br}}
\newcommand{\scap}[0]{\hat{\Bs}}
\newcommand{\tcap}[0]{\hat{\Bt}}
\newcommand{\ucap}[0]{\hat{\Bu}}
\newcommand{\vcap}[0]{\hat{\Bv}}
\newcommand{\wcap}[0]{\hat{\Bw}}
\newcommand{\xcap}[0]{\hat{\Bx}}
\newcommand{\ycap}[0]{\hat{\By}}
\newcommand{\zcap}[0]{\hat{\Bz}}
\newcommand{\thetacap}[0]{\hat{\Btheta}}

%
% to write R^n and C^n in a distinguishable fashion.  Perhaps change this
% to the double lined characters upon figuring out how to do so.
%
\newcommand{\C}[1]{$\mathbb{C}^{#1}$}
\newcommand{\R}[1]{$\mathbb{R}^{#1}$}

%
% various generally useful helpers
%

% derivative of #1 wrt. #2:
\newcommand{\D}[2] {\frac {d#2} {d#1}}

\newcommand{\inv}[1]{\frac{1}{#1}}
\newcommand{\cross}[0]{\times}

\newcommand{\abs}[1]{\lvert{#1}\rvert}
\newcommand{\norm}[1]{\lVert{#1}\rVert}
\newcommand{\innerprod}[2]{\langle{#1}, {#2}\rangle}
\newcommand{\dotprod}[2]{{#1} \cdot {#2}}
\newcommand{\bdotprod}[2]{\left({#1} \cdot {#2}\right)}
\newcommand{\crossprod}[2]{{#1} \cross {#2}}
\newcommand{\tripleprod}[3]{\dotprod{\left(\crossprod{#1}{#2}\right)}{#3}}

\DeclareMathOperator{\Proj}{Proj}
\DeclareMathOperator{\Span}{span}
\DeclareMathOperator{\Sgn}{sgn}
\DeclareMathOperator{\Area}{Area}
\DeclareMathOperator{\Volume}{Volume}

%
% A few miscellaneous things specific to this document
%
\newcommand{\crossop}[1]{\crossprod{#1}{}}

% R2 vector.
\newcommand{\VectorTwo}[2]{
\begin{bmatrix}
 {#1} \\
 {#2}
\end{bmatrix}
}

\newcommand{\VectorN}[1]{
\begin{bmatrix}
{#1}_1 \\
{#1}_2 \\
\vdots \\
{#1}_N \\
\end{bmatrix}
}

\newcommand{\DETuvij}[4]{
\begin{vmatrix}
 {#1}_{#3} & {#1}_{#4} \\
 {#2}_{#3} & {#2}_{#4}
\end{vmatrix}
}

\newcommand{\DETuvwijk}[6]{
\begin{vmatrix}
 {#1}_{#4} & {#1}_{#5} & {#1}_{#6} \\
 {#2}_{#4} & {#2}_{#5} & {#2}_{#6} \\
 {#3}_{#4} & {#3}_{#5} & {#3}_{#6}
\end{vmatrix}
}

\newcommand{\DETuvwxijkl}[8]{
\begin{vmatrix}
 {#1}_{#5} & {#1}_{#6} & {#1}_{#7} & {#1}_{#8} \\
 {#2}_{#5} & {#2}_{#6} & {#2}_{#7} & {#2}_{#8} \\
 {#3}_{#5} & {#3}_{#6} & {#3}_{#7} & {#3}_{#8} \\
 {#4}_{#5} & {#4}_{#6} & {#4}_{#7} & {#4}_{#8} \\
\end{vmatrix}
}

%\newcommand{\DETuvwxyijklm}[10]{
%\begin{vmatrix}
% {#1}_{#6} & {#1}_{#7} & {#1}_{#8} & {#1}_{#9} & {#1}_{#10} \\
% {#2}_{#6} & {#2}_{#7} & {#2}_{#8} & {#2}_{#9} & {#2}_{#10} \\
% {#3}_{#6} & {#3}_{#7} & {#3}_{#8} & {#3}_{#9} & {#3}_{#10} \\
% {#4}_{#6} & {#4}_{#7} & {#4}_{#8} & {#4}_{#9} & {#4}_{#10} \\
% {#5}_{#6} & {#5}_{#7} & {#5}_{#8} & {#5}_{#9} & {#5}_{#10}
%\end{vmatrix}
%}

% R3 vector.
\newcommand{\VectorThree}[3]{
\begin{bmatrix}
 {#1} \\
 {#2} \\
 {#3}
\end{bmatrix}
}


%
%\author{Peeter Joot}
%\email{peeter.joot@utoronto.ca, 920798560}
%%
% Copyright � 2015 Peeter Joot.  All Rights Reserved.
% Licenced as described in the file LICENSE under the root directory of this GIT repository.
%
\documentclass[]{eliblog}

\usepackage{amsmath}
\usepackage{mathpazo}

%
% shorthand for bold symbols, convenient for vectors and matrices
%
\newcommand{\Ba}[0]{\mathbf{a}}
\newcommand{\Bb}[0]{\mathbf{b}}
\newcommand{\Bc}[0]{\mathbf{c}}
\newcommand{\Bd}[0]{\mathbf{d}}
\newcommand{\Be}[0]{\mathbf{e}}
\newcommand{\Bf}[0]{\mathbf{f}}
\newcommand{\Bg}[0]{\mathbf{g}}
\newcommand{\Bh}[0]{\mathbf{h}}
\newcommand{\Bi}[0]{\mathbf{i}}
\newcommand{\Bj}[0]{\mathbf{j}}
\newcommand{\Bk}[0]{\mathbf{k}}
\newcommand{\Bl}[0]{\mathbf{l}}
\newcommand{\Bm}[0]{\mathbf{m}}
\newcommand{\Bn}[0]{\mathbf{n}}
\newcommand{\Bo}[0]{\mathbf{o}}
\newcommand{\Bp}[0]{\mathbf{p}}
\newcommand{\Bq}[0]{\mathbf{q}}
\newcommand{\Br}[0]{\mathbf{r}}
\newcommand{\Bs}[0]{\mathbf{s}}
\newcommand{\Bt}[0]{\mathbf{t}}
\newcommand{\Bu}[0]{\mathbf{u}}
\newcommand{\Bv}[0]{\mathbf{v}}
\newcommand{\Bw}[0]{\mathbf{w}}
\newcommand{\Bx}[0]{\mathbf{x}}
\newcommand{\By}[0]{\mathbf{y}}
\newcommand{\Bz}[0]{\mathbf{z}}
\newcommand{\BA}[0]{\mathbf{A}}
\newcommand{\BB}[0]{\mathbf{B}}
\newcommand{\BC}[0]{\mathbf{C}}
\newcommand{\BD}[0]{\mathbf{D}}
\newcommand{\BE}[0]{\mathbf{E}}
\newcommand{\BF}[0]{\mathbf{F}}
\newcommand{\BG}[0]{\mathbf{G}}
\newcommand{\BH}[0]{\mathbf{H}}
\newcommand{\BI}[0]{\mathbf{I}}
\newcommand{\BJ}[0]{\mathbf{J}}
\newcommand{\BK}[0]{\mathbf{K}}
\newcommand{\BL}[0]{\mathbf{L}}
\newcommand{\BM}[0]{\mathbf{M}}
\newcommand{\BN}[0]{\mathbf{N}}
\newcommand{\BO}[0]{\mathbf{O}}
\newcommand{\BP}[0]{\mathbf{P}}
\newcommand{\BQ}[0]{\mathbf{Q}}
\newcommand{\BR}[0]{\mathbf{R}}
\newcommand{\BS}[0]{\mathbf{S}}
\newcommand{\BT}[0]{\mathbf{T}}
\newcommand{\BU}[0]{\mathbf{U}}
\newcommand{\BV}[0]{\mathbf{V}}
\newcommand{\BW}[0]{\mathbf{W}}
\newcommand{\BX}[0]{\mathbf{X}}
\newcommand{\BY}[0]{\mathbf{Y}}
\newcommand{\BZ}[0]{\mathbf{Z}}

\newcommand{\Bzero}[0]{\mathbf{0}}
\newcommand{\Btheta}[0]{\boldsymbol{\theta}}
\newcommand{\Btau}[0]{\boldsymbol{\tau}}
\newcommand{\Bomega}[0]{\boldsymbol{\omega}}

%
% shorthand for unit vectors
%
\newcommand{\acap}[0]{\hat{\Ba}}
\newcommand{\bcap}[0]{\hat{\Bb}}
\newcommand{\ccap}[0]{\hat{\Bc}}
\newcommand{\dcap}[0]{\hat{\Bd}}
\newcommand{\ecap}[0]{\hat{\Be}}
\newcommand{\fcap}[0]{\hat{\Bf}}
\newcommand{\gcap}[0]{\hat{\Bg}}
\newcommand{\hcap}[0]{\hat{\Bh}}
\newcommand{\icap}[0]{\hat{\Bi}}
\newcommand{\jcap}[0]{\hat{\Bj}}
\newcommand{\kcap}[0]{\hat{\Bk}}
\newcommand{\lcap}[0]{\hat{\Bl}}
\newcommand{\mcap}[0]{\hat{\Bm}}
\newcommand{\ncap}[0]{\hat{\Bn}}
\newcommand{\ocap}[0]{\hat{\Bo}}
\newcommand{\pcap}[0]{\hat{\Bp}}
\newcommand{\qcap}[0]{\hat{\Bq}}
\newcommand{\rcap}[0]{\hat{\Br}}
\newcommand{\scap}[0]{\hat{\Bs}}
\newcommand{\tcap}[0]{\hat{\Bt}}
\newcommand{\ucap}[0]{\hat{\Bu}}
\newcommand{\vcap}[0]{\hat{\Bv}}
\newcommand{\wcap}[0]{\hat{\Bw}}
\newcommand{\xcap}[0]{\hat{\Bx}}
\newcommand{\ycap}[0]{\hat{\By}}
\newcommand{\zcap}[0]{\hat{\Bz}}
\newcommand{\thetacap}[0]{\hat{\Btheta}}

%
% to write R^n and C^n in a distinguishable fashion.  Perhaps change this
% to the double lined characters upon figuring out how to do so.
%
\newcommand{\C}[1]{$\mathbb{C}^{#1}$}
\newcommand{\R}[1]{$\mathbb{R}^{#1}$}

%
% various generally useful helpers
%

% derivative of #1 wrt. #2:
\newcommand{\D}[2] {\frac {d#2} {d#1}}

\newcommand{\inv}[1]{\frac{1}{#1}}
\newcommand{\cross}[0]{\times}

\newcommand{\abs}[1]{\lvert{#1}\rvert}
\newcommand{\norm}[1]{\lVert{#1}\rVert}
\newcommand{\innerprod}[2]{\langle{#1}, {#2}\rangle}
\newcommand{\dotprod}[2]{{#1} \cdot {#2}}
\newcommand{\bdotprod}[2]{\left({#1} \cdot {#2}\right)}
\newcommand{\crossprod}[2]{{#1} \cross {#2}}
\newcommand{\tripleprod}[3]{\dotprod{\left(\crossprod{#1}{#2}\right)}{#3}}

\DeclareMathOperator{\Proj}{Proj}
\DeclareMathOperator{\Span}{span}
\DeclareMathOperator{\Sgn}{sgn}
\DeclareMathOperator{\Area}{Area}
\DeclareMathOperator{\Volume}{Volume}

%
% A few miscellaneous things specific to this document
%
\newcommand{\crossop}[1]{\crossprod{#1}{}}

% R2 vector.
\newcommand{\VectorTwo}[2]{
\begin{bmatrix}
 {#1} \\
 {#2}
\end{bmatrix}
}

\newcommand{\VectorN}[1]{
\begin{bmatrix}
{#1}_1 \\
{#1}_2 \\
\vdots \\
{#1}_N \\
\end{bmatrix}
}

\newcommand{\DETuvij}[4]{
\begin{vmatrix}
 {#1}_{#3} & {#1}_{#4} \\
 {#2}_{#3} & {#2}_{#4}
\end{vmatrix}
}

\newcommand{\DETuvwijk}[6]{
\begin{vmatrix}
 {#1}_{#4} & {#1}_{#5} & {#1}_{#6} \\
 {#2}_{#4} & {#2}_{#5} & {#2}_{#6} \\
 {#3}_{#4} & {#3}_{#5} & {#3}_{#6}
\end{vmatrix}
}

\newcommand{\DETuvwxijkl}[8]{
\begin{vmatrix}
 {#1}_{#5} & {#1}_{#6} & {#1}_{#7} & {#1}_{#8} \\
 {#2}_{#5} & {#2}_{#6} & {#2}_{#7} & {#2}_{#8} \\
 {#3}_{#5} & {#3}_{#6} & {#3}_{#7} & {#3}_{#8} \\
 {#4}_{#5} & {#4}_{#6} & {#4}_{#7} & {#4}_{#8} \\
\end{vmatrix}
}

%\newcommand{\DETuvwxyijklm}[10]{
%\begin{vmatrix}
% {#1}_{#6} & {#1}_{#7} & {#1}_{#8} & {#1}_{#9} & {#1}_{#10} \\
% {#2}_{#6} & {#2}_{#7} & {#2}_{#8} & {#2}_{#9} & {#2}_{#10} \\
% {#3}_{#6} & {#3}_{#7} & {#3}_{#8} & {#3}_{#9} & {#3}_{#10} \\
% {#4}_{#6} & {#4}_{#7} & {#4}_{#8} & {#4}_{#9} & {#4}_{#10} \\
% {#5}_{#6} & {#5}_{#7} & {#5}_{#8} & {#5}_{#9} & {#5}_{#10}
%\end{vmatrix}
%}

% R3 vector.
\newcommand{\VectorThree}[3]{
\begin{bmatrix}
 {#1} \\
 {#2} \\
 {#3}
\end{bmatrix}
}



\author{Peeter Joot}
\email{peeter.joot@gmail.com}


\chapter{PHY356 Problem Set 5.}
\label{chap:qmIproblemSet5}
\blogpage{http://sites.google.com/site/peeterjoot/math2010/qmIproblemSet5.pdf}
\date{Nov 25, 2010}
\revisionInfo{qmIproblemSet5.tex}

\beginArtNoToc
\section{Disclaimer.}

This problem set is as yet ungraded.

\section{Problem.}
\subsection{Statement}

A particle of mass m moves along the x-direction such that $V(X)=\inv{2}KX^2$. Is the state 

\begin{equation}\label{eqn:qmIproblemSet5:5}
u(\xi) = B \xi e^{+\xi^2/2},
\end{equation}

where $\xi$ is given by Eq. (9.60), $B$ is a constant, and time $t=0$, an energy eigenstate of the system?  What is probability per unit length for measuring the particle at position $x=0$ at $t=t_0>0$?  Explain the physical meaning of the above results.

\subsection{Solution}
\subsubsection{Is this state an energy eigenstate?}

Recall that $\xi = \alpha x$, $\alpha = \sqrt{m\omega/\hbar}$, and $K = m \omega^2$.  With this variable substitution Schr\"{o}dinger's equation for this harmonic oscillator potential takes the form

\begin{equation}\label{eqn:qmIproblemSet5:10}
\frac{d^2 u}{d\xi^2} - \xi^2 u = \frac{2 E }{\hbar\omega} u
\end{equation}

While we can blindly substitute a function of the form $\xi e^{\xi^2/2}$ into this to get

\begin{align*}
\inv{B} \left(\frac{d^2 u}{d\xi^2} - \xi^2 u\right)
&=
\frac{d}{d\xi} \left( 1 + \xi^2 \right) e^{\xi^2/2} - \xi^3 e^{\xi^2/2} \\
&=
\left( 2 \xi + \xi + \xi^3 \right) e^{\xi^2/2} - \xi^3 e^{\xi^2/2} \\
&=
3 \xi e^{\xi^2/2}
\end{align*}

and formally make the identification $E = 3 \omega \hbar/2 = (1 + 1/2) \omega \hbar$, this isn't a normalizable wavefunction, and has no physical relevance, unless we set $B = 0$.

By changing the problem, this state could be physically relevant.  We'd require a potential of the form

\begin{equation}\label{eqn:qmIproblemSet5:11}
V(x) =
\left\{
\begin{array}{l l}
f(x) & \quad \mbox{if $x < a$} \\
\inv{2} K x^2 & \quad \mbox{if $a < x < b$} \\
g(x) & \quad \mbox{if $x > b$} \\
\end{array}
\right.
\end{equation}

For example, $f(x) = V_1, g(x) = V_2$, for constant $V_1, V_2$.  For such a potential, within the harmonic well, a general solution of the form

\begin{equation}\label{eqn:qmIproblemSet5:19}
u(x,t) = \sum_n H_n(\xi) \Bigl(A_n e^{-\xi^2/2} + B_n e^{\xi^2/2} \Bigr) e^{-i E_n t/\hbar},
\end{equation}

is possible since normalization would not prohibit non-zero $B_n$ values in that situation.  For the wave function to be a physically relevant, we require it to be (absolute) square integrable, and must also integrate to unity over the entire interval.

\subsubsection{Probability per unit length at $x=0$.}

We cannot answer the question for the probability that the particle is found at the specific $x=0$ position at $t=t_0$ (that probability is zero in a continuous space), but we can answer the question for the probability that a particle is found in an interval surrounding a specific point at this time.  By calculating the average of the probability to find the particle in an interval, and dividing by that interval's length, we arrive at plausible definition of probability per unit length for an interval surrounding $x = x_0$

\begin{equation}\label{eqn:qmIproblemSet5:17}
P = \text{Probability per unit length near $x = x_0$} =
\lim_{\epsilon \rightarrow 0} \inv{\epsilon} \int_{x_0 - \epsilon/2}^{x_0 + \epsilon/2} \Abs{ \Psi(x, t_0) }^2 dx = \Abs{\Psi(x_0, t_0)}^2
\end{equation}

By this definition, the probability per unit length is just the probability density itself, evaluated at the point of interest.

Physically, for an interval small enough that the probability density is constant in magnitude over that interval, this probability per unit length times the length of this small interval, represents the probability that we will find the particle in that interval.

\subsubsection{Probability per unit length for the non-normalizable state given.}

It seems possible, albeit odd, that this question is asking for the probability per unit length for the non-normalizable $E_1$ wavefunction \ref{eqn:qmIproblemSet5:5}.  Since normalization requires $B=0$, that probability density is simply zero (or undefined, depending on one's point of view).

\subsubsection{Probability per unit length for some more interesting harmonic oscillator states.}

Suppose we form the wavefunction for a superposition of all the normalizable states

\begin{equation}\label{eqn:qmIproblemSet5:20}
u(x,t) = \sum_n A_n H_n(\xi) e^{-\xi^2/2} e^{-i E_n t/\hbar}
\end{equation}

Here it is assumed that the $A_n$ coefficients yield unit probability

\begin{equation}\label{eqn:qmIproblemSet5:30}
\int \Abs{u(x,0)}^2 dx = \sum_n \Abs{A_n}^2 = 1
\end{equation}

For the impure state of \ref{eqn:qmIproblemSet5:20} we have for the probability density

\begin{align*}
\Abs{u}^2
&=
\sum_{m,n}
A_n A_m^\conj H_n(\xi) H_m(\xi) e^{-\xi^2} e^{-i (E_n - E_m)t_0/\hbar} \\
&=
\sum_n
\Abs{A_n}^2 (H_n(\xi))^2 e^{-\xi^2}
+\sum_{m \ne n}
A_n A_m^\conj H_n(\xi) H_m(\xi) e^{-\xi^2} e^{-i (E_n - E_m)t_0/\hbar} \\
&=
\sum_n
\Abs{A_n}^2 (H_n(\xi))^2 e^{-\xi^2}
+\sum_{m \ne n}
A_n A_m^\conj H_n(\xi) H_m(\xi) e^{-\xi^2} e^{-i (E_n - E_m)t_0/\hbar} \\
&=
\sum_n
\Abs{A_n}^2 (H_n(\xi))^2 e^{-\xi^2}
+\sum_{m < n}
H_n(\xi) H_m(\xi)
\left(
A_n A_m^\conj
e^{-\xi^2} e^{-i (E_n - E_m)t_0/\hbar}
+A_m A_n^\conj
e^{-\xi^2} e^{-i (E_m - E_n)t_0/\hbar}
\right) \\
&=
\sum_n
\Abs{A_n}^2 (H_n(\xi))^2 e^{-\xi^2}
+2 \sum_{m < n}
H_n(\xi) H_m(\xi)
e^{-\xi^2}
\Re \left(
A_n A_m^\conj
e^{-i (E_n - E_m)t_0/\hbar}
\right) \\
&=
\sum_n
\Abs{A_n}^2 (H_n(\xi))^2 e^{-\xi^2}  \\
&\quad+2 \sum_{m < n}
H_n(\xi) H_m(\xi)
e^{-\xi^2}
\left(
\Re ( A_n A_m^\conj ) \cos( (n - m)\omega t_0)
+\Im ( A_n A_m^\conj ) \sin( (n - m)\omega t_0)
\right) \\
\end{align*}

Evaluation at the point $x = 0$, we have

\begin{equation}\label{eqn:qmIproblemSet5:500}
\Abs{u(0,t_0)}^2
=
\sum_n
\Abs{A_n}^2 (H_n(0))^2 +2 \sum_{m < n} H_n(0) H_m(0) \left( \Re ( A_n A_m^\conj ) \cos( (n - m)\omega t_0) +\Im ( A_n A_m^\conj ) \sin( (n - m)\omega t_0)
\right)
\end{equation}

It is interesting that the probability per unit length only has time dependence for a mixed state.

For a pure state and its wavefunction $u(x,t) = N_n H_n(\xi) e^{-\xi^2/2} e^{-i E_n t/\hbar}$ we have just
\begin{equation}\label{eqn:qmIproblemSet5:510}
\Abs{u(0,t_0)}^2
=
N_n^2 (H_n(0))^2 = \frac{\alpha}{\sqrt{\pi} 2^n n!} H_n(0)^2
\end{equation}

This is zero for odd $n$.  For even $n$ is appears that $(H_n(0))^2$ may equal $2^n$ (this is true at least up to n=4).  If that's the case, we have for non-mixed states, with even numbered energy quantum numbers, at $x=0$ a probability per unit length value of $\Abs{u(0,t_0)}^2 = \frac{\alpha}{\sqrt{\pi} n!}$.

%\EndArticle
\EndNoBibArticle
