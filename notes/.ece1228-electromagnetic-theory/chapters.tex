%
% Copyright � 2016 Peeter Joot.  All Rights Reserved.
% Licenced as described in the file LICENSE under the root directory of this GIT repository.
%
%----------------------------------------------------------------------------------------
%
% Prof. decoder ring:
%
% antigreat      == integrate
% ambeeguis      == ambiguous
% eikenvector    == eigenvector
% esmal          == small
% sine "of a" x  == sin(x)
% beta suba x    == beta_x
% constrain      == constraint
% havea expr     == have expr (example: havea e^x), plug this expression for beta -> plug this expression for abeta.
% isa            == is : f_{c1} isa bigger than f_{c2}
% den            == then
% ispectra       == spectrum
% smood          == smooth
% togeder        == together
%
%----------------------------------------------------------------------------------------
\part{Lecture notes}
   %\chapter{Electromagnetic fields}
   %
% Copyright � 2016 Peeter Joot.  All Rights Reserved.
% Licenced as described in the file LICENSE under the root directory of this GIT repository.
%
%\newcommand{\authorname}{Peeter Joot}
\newcommand{\email}{peeterjoot@protonmail.com}
\newcommand{\basename}{FIXMEbasenameUndefined}
\newcommand{\dirname}{notes/FIXMEdirnameUndefined/}

%\renewcommand{\basename}{emt1}
%\renewcommand{\dirname}{notes/ece1228/}
%\newcommand{\keywords}{ECE1228H}
%\newcommand{\authorname}{Peeter Joot}
\newcommand{\onlineurl}{http://sites.google.com/site/peeterjoot2/math2013/\basename.pdf}
\newcommand{\sourcepath}{\dirname\basename.tex}
\newcommand{\generatetitle}[1]{\chapter{#1}}

\newcommand{\vcsinfo}{%
\section*{}
\noindent{\color{DarkOliveGreen}{\rule{\linewidth}{0.1mm}}}
\paragraph{Document version}
%\paragraph{\color{Maroon}{Document version}}
{
\small
\begin{itemize}
\item Available online at:\\ 
\href{\onlineurl}{\onlineurl}
\item Git Repository: \input{./.revinfo/gitRepo.tex}
\item Source: \sourcepath
\item last commit: \input{./.revinfo/gitCommitString.tex}
\item commit date: \input{./.revinfo/gitCommitDate.tex}
\end{itemize}
}
}

%\PassOptionsToPackage{dvipsnames,svgnames}{xcolor}
\PassOptionsToPackage{square,numbers}{natbib}
\documentclass{scrreprt}

\usepackage[left=2cm,right=2cm]{geometry}
\usepackage[svgnames]{xcolor}
\usepackage{peeters_layout}

\usepackage{natbib}

\usepackage[
colorlinks=true,
bookmarks=false,
pdfauthor={\authorname, \email},
backref 
]{hyperref}

% http://tex.stackexchange.com/questions/75773/how-to-reference-problems-by-the-text-label-in-an-exercise-envioronment
\usepackage[english]{cleveref}
\crefname{Exercise}{exercise}{exercises}
\Crefname{Exercise}{Exercise}{Exercises}

\RequirePackage{titlesec}
\RequirePackage{ifthen}

% http://stackoverflow.com/questions/4932910/date-in-the-tabular-environment
\makeatletter
\let\insertdate\@date
\makeatother

\titleformat{\chapter}[display]
{\bfseries\Large}
{\color{DarkSlateGrey}\filleft \authorname
\ifthenelse{\isundefined{\studentnumber}}{}{\\ \studentnumber}
\ifthenelse{\isundefined{\email}}{}{\\ \email}
\ifthenelse{\isundefined{\dateintitle}}{}{\\ \insertdate}
%\ifthenelse{\isundefined{\coursename}}{}{\\ \coursename} % put in title instead.
}
{4ex}
{\color{DarkOliveGreen}{\titlerule}\color{Maroon}
\vspace{2ex}%
\filright}
[\vspace{2ex}%
\color{DarkOliveGreen}\titlerule
]

\newcommand{\beginArtWithToc}[0]{\begin{document}\tableofcontents}
\newcommand{\beginArtNoToc}[0]{\begin{document}}
\newcommand{\EndNoBibArticle}[0]{\end{document}}
\newcommand{\EndArticle}[0]{\bibliography{Bibliography}\bibliographystyle{plainnat}\end{document}}

% 
%\newcommand{\citep}[1]{\cite{#1}}

\colorSectionsForArticle


%
%%\usepackage{ece1228}
%\usepackage{peeters_braket}
%%\usepackage{peeters_layout_exercise}
%\usepackage{peeters_figures}
%\usepackage{mathtools}
%\usepackage{siunitx}
%
%\beginArtNoToc
%\generatetitle{ECE1228H Electromagnetic Theory.  Lecture 1: Introduction.  Taught by Prof.\ M. Mojahedi}
\chapter{Introduction}
%\label{chap:emt1}
%
%\paragraph{Disclaimer}
%
%Peeter's lecture notes from class.  These may be incoherent and rough.
%
%These are notes for the UofT course ECE1228H, Electromagnetic Theory, taught by Prof. M. Mojahedi, covering \textchapref{{1}} \%citep{balanis1989advanced} content.

\paragraph{Maxwell's equations}
\index{Maxwell's equations!time domain}

\begin{itemize}
\item Faraday's Law
\begin{dmath}\label{eqn:emtLecture1:20}
\spacegrad \cross \BE( \Br, t ) = - \PD{t}{\BB}(\Br, t) - \BM_i
\end{dmath}
\item Ampere-Maxwell equation
\begin{dmath}\label{eqn:emtLecture1:40}
\spacegrad \cross \BH( \Br, t ) = \BJ_\txtc(\Br, t) + \PD{t}{\BD}(\Br, t)
\end{dmath}
\item Gauss's law
\begin{dmath}\label{eqn:emtLecture1:80}
\spacegrad \cdot \BD(\Br, t) = \rho_{\txte\txtv}(\Br, t)
\end{dmath}
\item Gauss's law for magnetism
\begin{dmath}\label{eqn:emtLecture1:100}
\spacegrad \cdot \BB(\Br, t) = \rho_{\txtm\txtv}(\Br, t)
\end{dmath}
\end{itemize}

After unpacking, we have a total of eight equations, with four vectoral field variables, and 8 sources, all interrelated by partial derivatives in space and time coordinates.  

It will be left to homework to show that without the displacement current \( \PDi{t}{\BD} \), these equations will not satisfy conservation relations.

The fields are and sources are
\index{units}
\begin{itemize}
\item \( \BE \) Electric field intensity \si{V/m}.
\item \( \BB \) Magnetic flux density \si{V s/m^2} (or Tesla).
\item \( \BH \) Magnetic field intensity \si{A/m}.
\item \( \BD \) Electric flux density \si{C/m^2}.
\item \( \rho_{\txte\txtv} \) Electric charge volume density
\item \( \rho_{\txtm\txtv} \) Magnetic charge volume density
\item \( \BJ_{\txtc} \) Impressed (source) electric current density \si{A/m^2}.  This is the charge passing through a plane in a unit time.  Here \( \txtc \) is for ``conduction''.
\item \( \BM_{\txti} \) Impressed (source) magnetic current density \si{V/m^2}
\end{itemize}

In an undergrad context we'll have seen the electric and magnetic fields in the Lorentz force law

\begin{dmath}\label{eqn:emtLecture1:120}
\BF = q \Bv \cross \BB + q\BE.
\end{dmath}

In SI there are 7 basic units.  These include

\begin{itemize}
\item Length \si{m}.
\item mass \si{kg}.
\item Time \si{s}.
\item Ampere \si{A}.
\index{unit!ampere}
\item Kelvin \si{K} (temperature)
\index{unit!Kelvin}
\item Candela (luminous intensity)
\index{unit!candela}
\item Mole (amount of substance)
\index{unit!mole}
\end{itemize}

\index{unit!Coulomb}
Note that the Coulomb is not a fundamental unit, but the Ampere is.  This is because it is easier to measure.

For homework: show that magnetic field lines must close on themselves when there are no magnetic sources (zero divergence).  This is opposed to electric fields that spread out from the charge.

%\EndNoBibArticle

   %
% Copyright � 2016 Peeter Joot.  All Rights Reserved.
% Licenced as described in the file LICENSE under the root directory of this GIT repository.
%
%\newcommand{\authorname}{Peeter Joot}
\newcommand{\email}{peeterjoot@protonmail.com}
\newcommand{\basename}{FIXMEbasenameUndefined}
\newcommand{\dirname}{notes/FIXMEdirnameUndefined/}

%\renewcommand{\basename}{emt2}
%\renewcommand{\dirname}{notes/ece1228/}
%\newcommand{\keywords}{ECE1228H}
%\newcommand{\authorname}{Peeter Joot}
\newcommand{\onlineurl}{http://sites.google.com/site/peeterjoot2/math2013/\basename.pdf}
\newcommand{\sourcepath}{\dirname\basename.tex}
\newcommand{\generatetitle}[1]{\chapter{#1}}

\newcommand{\vcsinfo}{%
\section*{}
\noindent{\color{DarkOliveGreen}{\rule{\linewidth}{0.1mm}}}
\paragraph{Document version}
%\paragraph{\color{Maroon}{Document version}}
{
\small
\begin{itemize}
\item Available online at:\\ 
\href{\onlineurl}{\onlineurl}
\item Git Repository: \input{./.revinfo/gitRepo.tex}
\item Source: \sourcepath
\item last commit: \input{./.revinfo/gitCommitString.tex}
\item commit date: \input{./.revinfo/gitCommitDate.tex}
\end{itemize}
}
}

%\PassOptionsToPackage{dvipsnames,svgnames}{xcolor}
\PassOptionsToPackage{square,numbers}{natbib}
\documentclass{scrreprt}

\usepackage[left=2cm,right=2cm]{geometry}
\usepackage[svgnames]{xcolor}
\usepackage{peeters_layout}

\usepackage{natbib}

\usepackage[
colorlinks=true,
bookmarks=false,
pdfauthor={\authorname, \email},
backref 
]{hyperref}

% http://tex.stackexchange.com/questions/75773/how-to-reference-problems-by-the-text-label-in-an-exercise-envioronment
\usepackage[english]{cleveref}
\crefname{Exercise}{exercise}{exercises}
\Crefname{Exercise}{Exercise}{Exercises}

\RequirePackage{titlesec}
\RequirePackage{ifthen}

% http://stackoverflow.com/questions/4932910/date-in-the-tabular-environment
\makeatletter
\let\insertdate\@date
\makeatother

\titleformat{\chapter}[display]
{\bfseries\Large}
{\color{DarkSlateGrey}\filleft \authorname
\ifthenelse{\isundefined{\studentnumber}}{}{\\ \studentnumber}
\ifthenelse{\isundefined{\email}}{}{\\ \email}
\ifthenelse{\isundefined{\dateintitle}}{}{\\ \insertdate}
%\ifthenelse{\isundefined{\coursename}}{}{\\ \coursename} % put in title instead.
}
{4ex}
{\color{DarkOliveGreen}{\titlerule}\color{Maroon}
\vspace{2ex}%
\filright}
[\vspace{2ex}%
\color{DarkOliveGreen}\titlerule
]

\newcommand{\beginArtWithToc}[0]{\begin{document}\tableofcontents}
\newcommand{\beginArtNoToc}[0]{\begin{document}}
\newcommand{\EndNoBibArticle}[0]{\end{document}}
\newcommand{\EndArticle}[0]{\bibliography{Bibliography}\bibliographystyle{plainnat}\end{document}}

% 
%\newcommand{\citep}[1]{\cite{#1}}

\colorSectionsForArticle


%
%%\usepackage{ece1228}
%\usepackage{peeters_braket}
%%\usepackage{peeters_layout_exercise}
%\usepackage{peeters_figures}
%\usepackage{mathtools}
%\usepackage{siunitx}
%
%\beginArtNoToc
%\generatetitle{ECE1228H Electromagnetic Theory.  Lecture 2: Boundaries.  Taught by Prof.\ M. Mojahedi}
\chapter{Boundaries}
\label{chap:emt2}

%\paragraph{Disclaimer}
%
%Peeter's lecture notes from class.  These may be incoherent and rough.
%
%These are notes for the UofT course ECE1228H, Electromagnetic Theory, taught by Prof. M. Mojahedi, covering \textchapref{{1}} \citep{balanis1989advanced} content.
%
\paragraph{Integral forms}

Given Maxwell's equations at a point

\begin{dmath}\label{eqn:emtLecture2:20}
\begin{aligned}
\spacegrad \cross \BE &= -\PD{t}{\BB} \\
\spacegrad \cross \BH &= \BJ + \PD{t}{\BD} \\
\spacegrad \cdot \BD &= \rho_\txtv \\
\spacegrad \cdot \BB &= 0
\end{aligned}
\end{dmath}

what happens when we have different fields and currents on two sides of a boundary?  To answer these questions, we want to use the integral forms of Maxwell's equations, over the geometries illustrated in \cref{fig:loopAndPillbox:loopAndPillboxFig1}.
\index{boundary}

\imageFigure{../../figures/ece1228-emt/loopAndPillboxFig1}{Loop and pillbox configurations.}{fig:loopAndPillbox:loopAndPillboxFig1}{0.3}

To do so, we use Stokes' and the divergence theorems relating the area and volume integrals to the surfaces of those geometries.

These are 

\index{Stokes' Theorem}
\index{Divergence Theorem}
\begin{dmath}\label{eqn:emtLecture2:40}
\begin{aligned}
\iint_S \lr{ \spacegrad \cross \BA } \cdot d\Bs &= \oint_C \BA \cdot d\Bl \\
\iint_V \lr{ \spacegrad \cdot \BA } d\Bs &= \oint_A \BA \cdot d\Bs \\
\end{aligned}
\end{dmath}

\index{Faraday's law}
Application of the Stokes' to Faraday's law we get

\begin{dmath}\label{eqn:emtLecture2:60}
\oint_C \BE \cdot d\Bl = -\PD{t}{} \iint \BB \cdot d\Bs
\end{dmath}

UNITS: \( V/m \times m \)

The quantity 
\begin{dmath}\label{eqn:emtLecture2:80}
\iint \BB \cdot d\Bs,
\end{dmath}

is called the magnetic flux of \( \BB \), and changing of this flux is responsible for the generation of electromotive force.
\index{magnetic flux}

%F2: 
Similarly

\begin{dmath}\label{eqn:emtLecture2:100}
\begin{aligned}
\oint \BH \cdot d\Bl &= \iint \BJ \cdot d\Bs + \PD{t}{} \iint \BD \cdot d\Bs \\
\oint \BD \cdot d\Bs &= \iiint \rho_\txtv dV = Q_\txte \\
\oint \BB \cdot d\Bs &= 0.
\end{aligned}
\end{dmath}

\index{constitutive relations}
\paragraph{Constitutive relations}

With 12 unknowns in \( \BE, \BB, \BD, \BH \) and 8 equations in Maxwell's equations (or 6 if the divergence equations are considered redundant), things don't look too good for solutions.  In simple media, in the frequency domain, relations of the form

\begin{dmath}\label{eqn:emtLecture2:120}
\begin{aligned}
\BD( \Br, \omega ) &= \epsilon \BE( \Br, \omega ) \\
\BB( \Br, \omega ) &= \mu \BH( \Br, \omega ).
\end{aligned}
\end{dmath}

\index{permeability}
\index{macroscopic}
The permeabilities \( \epsilon \) and \( \mu \) are macroscopic beasts, determined either experimentally, or theoretically using an averaging process involving many (millions, or billions, or more) particles.  However, the theoretical determinations that have been attempted do not work well in practise and usually end up considerably different than the measured values.  We are referred to \citep{jackson1975cew} for one attempt to model the statistical microscopic effects non-quantum mechanically to justify the traditional macroscopic form of Maxwell's equations.

These can be position dependent, as in the grating sketched in \cref{fig:gratingL2:gratingL2Fig3}.

\imageFigure{../../figures/ece1228-emt/gratingL2Fig3}{Grating.}{fig:gratingL2:gratingL2Fig3}{0.3}

\index{capacitor}
\index{breakdown voltage}
The permeabilities can also depend on the strength of the fields.  An example, application of an electric field to gallium arsenide or glass can change the behaviour in the material.  We can also have non-linear effects, such as the effect on a capacitor when the voltage is increased.  The response near the breakdown point where the capacitor blows up demonstrates this spectacularly.  We can also have materials for which the permeabilities depend on the direction of the field, or the temperature, or the pressure in the environment, the tensile or compression forces on the material, or many other factors.  There are many other possible complicating factors, for example, the electric response \( \epsilon \) can depend on the magnetic field strength \( \Abs{\BB} \).  We could then write

\begin{dmath}\label{eqn:emtLecture2:140}
\epsilon = \epsilon( \Br, \Abs{\BE}, \BE/\Abs{\BE}, T, P, \Abs{\Beta}, \omega, k ).
\end{dmath}

Further complicating things is that \( \epsilon \) is a complex number (for fields specified in the frequency domain).

\index{anisotropic}
We can also have anisotropic situations where the electric and displacement fields are no longer colinear as sketched in \cref{fig:constituativeRelationsL2:constituativeRelationsL2Fig4}.

\imageFigure{../../figures/ece1228-emt/constituativeRelationsL2Fig4}{Anisotropic field relations.}{fig:constituativeRelationsL2:constituativeRelationsL2Fig4}{0.3}

which indicates that the permittivity \( \epsilon \) in the relation

\begin{dmath}\label{eqn:emtLecture2:160}
\BD = \epsilon \BE,
\end{dmath}

can be modelled as a matrix or as a second rank tensor.  When the off diagonal entries are zero, and the diagonal values are all equal, we have the special case where \( \epsilon \) is reduced to a function.  That function may still be complex-valued, and dependent on many factors, but it least it is scalar valued in this situation.

\index{polarization}
\index{magnetization}
\paragraph{Polarization and magnetization}

If we have a material (such as glass), we can generally assume that the induced field can be related to the vacuum field according to

\begin{dmath}\label{eqn:emtLecture2:180}
\BE = \BP + \epsilon_0 \BE,
\end{dmath}
and
\begin{dmath}\label{eqn:emtLecture2:200}
\BB = \mu_0 \BM + \mu_0 \BH = \mu_0 \lr{ \BM + \BH }.
\end{dmath}

\index{permittivity!vacuum}
Here the vacuum permittivity \( \epsilon_0 \) has the value \( 8.85 \times 10^{-12} \si{F/m} \).  When we are ignoring (fictional) magnetic sources, we have a constant relation between the magnetic fields \( \BB = \mu_0 \BH \).

Assuming \( \BP = \epsilon_0 \chi_\txte \BE \), then

\begin{dmath}\label{eqn:emtLecture2:220}
\BD 
= \epsilon_0 \BE + \epsilon_0 \chi_\txte \BE 
= \epsilon_0 ( 1 + \chi_\txte ) \BE ,
\end{dmath}

so with \( \epsilon_r = 1 + \chi_\txte \), and \( \epsilon = \epsilon_0 \epsilon_r \) we have

\begin{dmath}\label{eqn:emtLecture2:240}
\BD = \epsilon \BE.
\end{dmath}

\index{permittivity!relative}
Note that the relative permittivity \( \epsilon_r \) is dimensionless, whereas the vacuum permittivity has units of \si{F/m}.  We call \(\epsilon\) the (unqualified) permittivity.  The relative permittivity \( \epsilon_r\) is sometimes called the relative permittivity.

\index{index of refraction}
Another useful quantity is the index of refraction

\begin{dmath}\label{eqn:emtLecture2:260}
\eta = \sqrt{ \epsilon_r \mu_r } \approx \sqrt{\epsilon_r}.
\end{dmath}

Similar to the above we can write \( \BM = \chi_\txtm \BH \) then

\begin{dmath}\label{eqn:emtLecture2:280}
\BM = \mu_0 \BH + \mu_0 \BM = \mu_0 \lr{ 1 + \chi_\txtm } \BH
= \mu_0 \mu_r \BH
\end{dmath}

so with \( \mu_r = 1 + \chi_\txtm \), and \( \mu = \mu_0 \mu_r \) we have

\begin{dmath}\label{eqn:emtLecture2:300}
\BB = \mu \BH.
\end{dmath}

\paragraph{Linear and angular momentum in light}

\index{photon!momentum}
\index{photon!angular momentum}
It was pointed out that we have two relations in mechanics that relate momentum and forces

\begin{dmath}\label{eqn:emtLecture2:320}
\begin{aligned}
\BF &= \ddt{\BP} \\
\Btau &= \ddt{\BL},
\end{aligned}
\end{dmath}

where \( \BP = m \Bv \) is the linear momentum, and \( \BL = \Br \cross \Bp \) is the angular momentum.  In quantum electrodynamics, the photon can be described using a relationship between wave-vector and momentum

\begin{dmath}\label{eqn:emtLecture2:340}
\Bp 
= \Hbar \Bk
= \Hbar \frac{ 2\pi}{\lambda}
= \frac{h}{2\pi} \frac{ 2\pi}{\lambda}
= \frac{h}{\lambda},
\end{dmath}

where \( \hbar = 6.522 \times 10^{-16} \si{ev.s} \).

Photons are also governed by

\begin{dmath}\label{eqn:emtLecture2:360}
E = \Hbar \omega = h \nu.
\end{dmath}

\index{De-Broglie relation}
(De-Broglie's relations).

ASIDE: optical fibre at 1550 has the lowest amount of optical attenuation.  

Since photons have linear momentum, we can move things around using light.  With photons having both linear momentum and energy relationships, and there is a relation between between torque and linear momentum, it seems that there must be the possibility of light having angular momentum.

Is it possible to utilize the angular momentum to impose patterns on beams (such as laser beams).  For example, what if a beam could have a geometrical pattern along its line of propagation, being off in some regions, on in others.  This is in fact possible, generating beams that are ``self healing''.

The question was posed ``Is it possible to solve electromagnetic problems utilizing the force concepts?'', using the Lorentz
force equation

\begin{dmath}\label{eqn:emtLecture2:380}
\BF = q \Bv \cross \BB + q\BE.
\end{dmath}

This was not thought to be a productive approach due to the complexity.

FIXME: It appeared that this animated talk (probably not captured well) about momentum in light was linked to the idea of the Helmholtz theorem.  Exactly how was not clear to me.

\index{Helmholtz's theorem}
\paragraph{Helmholtz's theorem}

Suppose that we have a linear material where

\begin{dmath}\label{eqn:emtLecture2:400}
\begin{aligned}
\spacegrad \cross \BE &= -\PD{t}{\BB} \\
\spacegrad \cross \BH &= \BJ + \PD{t}{\BD} \\
\spacegrad \cdot \BE &= \frac{\rho_\txtv}{\epsilon_0} \\
\spacegrad \cdot \BH &= 0
\end{aligned}
\end{dmath}

We have relations between the divergence and curl of \( \BE \) given the sources.  Is that sufficient to determine \( \BE \) itself?  The answer is yes, which is due to the Helmholtz theorem.

Extra homework question (bonus) : can knowledge of the tangential components of the fields also be used to uniquely determine \( \BE \)?

Also homework: read notes about irrotational fields and solenoidal fields.

%\EndArticle

   \section{Problems}
      \input{continuityDisplacement.tex}
      %
% Copyright � 2016 Peeter Joot.  All Rights Reserved.
% Licenced as described in the file LICENSE under the root directory of this GIT repository.
%
%{
\newcommand{\authorname}{Peeter Joot}
\newcommand{\email}{peeterjoot@protonmail.com}
\newcommand{\basename}{FIXMEbasenameUndefined}
\newcommand{\dirname}{notes/FIXMEdirnameUndefined/}

\renewcommand{\basename}{griffithsEM2_7}
\renewcommand{\dirname}{notes/phy1520/}
%\newcommand{\dateintitle}{}
%\newcommand{\keywords}{}

\newcommand{\authorname}{Peeter Joot}
\newcommand{\onlineurl}{http://sites.google.com/site/peeterjoot2/math2013/\basename.pdf}
\newcommand{\sourcepath}{\dirname\basename.tex}
\newcommand{\generatetitle}[1]{\chapter{#1}}

\newcommand{\vcsinfo}{%
\section*{}
\noindent{\color{DarkOliveGreen}{\rule{\linewidth}{0.1mm}}}
\paragraph{Document version}
%\paragraph{\color{Maroon}{Document version}}
{
\small
\begin{itemize}
\item Available online at:\\ 
\href{\onlineurl}{\onlineurl}
\item Git Repository: \input{./.revinfo/gitRepo.tex}
\item Source: \sourcepath
\item last commit: \input{./.revinfo/gitCommitString.tex}
\item commit date: \input{./.revinfo/gitCommitDate.tex}
\end{itemize}
}
}

%\PassOptionsToPackage{dvipsnames,svgnames}{xcolor}
\PassOptionsToPackage{square,numbers}{natbib}
\documentclass{scrreprt}

\usepackage[left=2cm,right=2cm]{geometry}
\usepackage[svgnames]{xcolor}
\usepackage{peeters_layout}

\usepackage{natbib}

\usepackage[
colorlinks=true,
bookmarks=false,
pdfauthor={\authorname, \email},
backref 
]{hyperref}

% http://tex.stackexchange.com/questions/75773/how-to-reference-problems-by-the-text-label-in-an-exercise-envioronment
\usepackage[english]{cleveref}
\crefname{Exercise}{exercise}{exercises}
\Crefname{Exercise}{Exercise}{Exercises}

\RequirePackage{titlesec}
\RequirePackage{ifthen}

% http://stackoverflow.com/questions/4932910/date-in-the-tabular-environment
\makeatletter
\let\insertdate\@date
\makeatother

\titleformat{\chapter}[display]
{\bfseries\Large}
{\color{DarkSlateGrey}\filleft \authorname
\ifthenelse{\isundefined{\studentnumber}}{}{\\ \studentnumber}
\ifthenelse{\isundefined{\email}}{}{\\ \email}
\ifthenelse{\isundefined{\dateintitle}}{}{\\ \insertdate}
%\ifthenelse{\isundefined{\coursename}}{}{\\ \coursename} % put in title instead.
}
{4ex}
{\color{DarkOliveGreen}{\titlerule}\color{Maroon}
\vspace{2ex}%
\filright}
[\vspace{2ex}%
\color{DarkOliveGreen}\titlerule
]

\newcommand{\beginArtWithToc}[0]{\begin{document}\tableofcontents}
\newcommand{\beginArtNoToc}[0]{\begin{document}}
\newcommand{\EndNoBibArticle}[0]{\end{document}}
\newcommand{\EndArticle}[0]{\bibliography{Bibliography}\bibliographystyle{plainnat}\end{document}}

% 
%\newcommand{\citep}[1]{\cite{#1}}

\colorSectionsForArticle



\usepackage{peeters_layout_exercise}
\usepackage{peeters_braket}
\usepackage{peeters_figures}
\usepackage{siunitx}

\beginArtNoToc

\generatetitle{Electric field due to spherical shell}
%\chapter{electric field due to spherical shell}
%\label{chap:griffithsEM2_7}
% \citep{sakurai2014modern} pr X.Y
% \citep{pozar2009microwave}
% \citep{qftLectureNotes}

\makeoproblem{Electric field due to spherical shell}{problem:griffithsEM2_7:1}{\citep{griffiths1999introduction} pr. 2.7}{
Calculate the field due to a spherical shell.  The field is

\begin{dmath}\label{eqn:griffithsEM2_7:20}
\BE = \frac{\sigma}{4 \pi \epsilon_0} \int \frac{(\Br - \Br')}{\Abs{\Br - \Br'}^3} da',
\end{dmath}

where \( \Br' \) is the position to the area element on the shell.  For the test position, let \( \Br = z \Be_3 \).  
} % problem

\makeanswer{problem:griffithsEM2_7:1}{
We need to parameterize the area integral.  A complex-number like geometric algebra representation works nicely.

\begin{dmath}\label{eqn:griffithsEM2_7:40}
\Br' 
= R \lr{ \sin\theta \cos\phi, \sin\theta \sin\phi, \cos\theta }
= R \lr{ \Be_1 \sin\theta \lr{ \cos\phi + \Be_1 \Be_2 \sin\phi } + \Be_3 \cos\theta }
= R \lr{ \Be_1 \sin\theta e^{i\phi} + \Be_3 \cos\theta }.
\end{dmath}

Here \( i = \Be_1 \Be_2 \) has been used to represent to horizontal rotation plane.

The difference in position between the test vector and area-element is

\begin{dmath}\label{eqn:griffithsEM2_7:60}
\Br - \Br' 
= \Be_3 \lr{ z - R \cos\theta } - R \Be_1 \sin\theta e^{i \phi},
\end{dmath}

with an absolute squared length of

\begin{dmath}\label{eqn:griffithsEM2_7:80}
\Abs{\Br - \Br' }^2
= \lr{ z - R \cos\theta }^2 + R^2 \sin^2\theta 
= z^2 + R^2 - 2 z R \cos\theta.
\end{dmath}

As a side note, this is a kind of fun way to prove the old ``cosine-law'' identity.  With that done, the field integral can now be expressed explicitly

\begin{dmath}\label{eqn:griffithsEM2_7:100}
\BE 
= \frac{\sigma}{4 \pi \epsilon_0} \int_{\phi = 0}^{2\pi} \int_{\theta = 0}^\pi R^2 \sin\theta d\theta d\phi
\frac{\Be_3 \lr{ z - R \cos\theta } - R \Be_1 \sin\theta e^{i \phi}}
{
\lr{z^2 + R^2 - 2 z R \cos\theta}^{3/2}
}
= \frac{2 \pi R^2 \sigma \Be_3}{4 \pi \epsilon_0} \int_{\theta = 0}^\pi \sin\theta d\theta 
\frac{z - R \cos\theta}
{
\lr{z^2 + R^2 - 2 z R \cos\theta}^{3/2}
}
= \frac{2 \pi R^2 \sigma \Be_3}{4 \pi \epsilon_0} \int_{\theta = 0}^\pi \sin\theta d\theta 
\frac{ R( z/R - \cos\theta) }
{
(R^2)^{3/2} \lr{ (z/R)^2 + 1 - 2 (z/R) \cos\theta}^{3/2}
}
= \frac{\sigma \Be_3}{2 \epsilon_0} \int_{u = -1}^{1} du
\frac{ z/R - u}
{
\lr{1 + (z/R)^2 - 2 (z/R) u}^{3/2}
}.
\end{dmath}

Observe that all the azimuthal contributions get killed.  We expect that due to the symmetry of the problem.  We are left with an integral that submits to Mathematica, but doesn't look fun to attempt manually.  Specifically

\begin{dmath}\label{eqn:griffithsEM2_7:120}
\int_{-1}^1 \frac{a-u}{\lr{1 + a^2 - 2 a u}^{3/2}} du
=
\left\{
\begin{array}{l l}
\frac{2}{a^2} & \quad \mbox{if \( a > 1 \) } \\
0 & \quad \mbox{if \( a < 1 \) } 
\end{array}
\right.,
\end{dmath}

so

%\begin{dmath}\label{eqn:griffithsEM2_7:140}
\boxedEquation{eqn:griffithsEM2_7:160}{
\BE 
= 
\left\{
\begin{array}{l l}
\frac{\sigma (R/z)^2 \Be_3}{\epsilon_0} 
& \quad \mbox{if \( z > R \) } \\
0 & \quad \mbox{if \( z < R \) } 
\end{array}
\right.
}
%\end{dmath}

In the problem, it is pointed out to be careful of the sign when evaluating \( \sqrt{ R^2 + z^2 - 2 R z } \), however, I don't see where that is even useful?
} % answer

%}
\EndArticle
%\EndNoBibArticle

      \input{Set1Problem1.tex}
      \input{Set1Problem2.tex}
      \input{Set1Problem3.tex}
      \input{Set1Problem4.tex}
      \input{Set1Problem5.tex}
      \input{Set1Problem6.tex}
      %\input{Set1Appendix.tex}
      \input{Set2Problem2.tex}
      \input{Set2Problem3.tex}
      \input{Set2Problem5.tex}

   %
% Copyright � 2016 Peeter Joot.  All Rights Reserved.
% Licenced as described in the file LICENSE under the root directory of this GIT repository.
%
\newcommand{\authorname}{Peeter Joot}
\newcommand{\email}{peeterjoot@protonmail.com}
\newcommand{\basename}{FIXMEbasenameUndefined}
\newcommand{\dirname}{notes/FIXMEdirnameUndefined/}

\renewcommand{\basename}{emt3}
\renewcommand{\dirname}{notes/ece1228/}
\newcommand{\keywords}{ECE1228H}
\newcommand{\authorname}{Peeter Joot}
\newcommand{\onlineurl}{http://sites.google.com/site/peeterjoot2/math2013/\basename.pdf}
\newcommand{\sourcepath}{\dirname\basename.tex}
\newcommand{\generatetitle}[1]{\chapter{#1}}

\newcommand{\vcsinfo}{%
\section*{}
\noindent{\color{DarkOliveGreen}{\rule{\linewidth}{0.1mm}}}
\paragraph{Document version}
%\paragraph{\color{Maroon}{Document version}}
{
\small
\begin{itemize}
\item Available online at:\\ 
\href{\onlineurl}{\onlineurl}
\item Git Repository: \input{./.revinfo/gitRepo.tex}
\item Source: \sourcepath
\item last commit: \input{./.revinfo/gitCommitString.tex}
\item commit date: \input{./.revinfo/gitCommitDate.tex}
\end{itemize}
}
}

%\PassOptionsToPackage{dvipsnames,svgnames}{xcolor}
\PassOptionsToPackage{square,numbers}{natbib}
\documentclass{scrreprt}

\usepackage[left=2cm,right=2cm]{geometry}
\usepackage[svgnames]{xcolor}
\usepackage{peeters_layout}

\usepackage{natbib}

\usepackage[
colorlinks=true,
bookmarks=false,
pdfauthor={\authorname, \email},
backref 
]{hyperref}

% http://tex.stackexchange.com/questions/75773/how-to-reference-problems-by-the-text-label-in-an-exercise-envioronment
\usepackage[english]{cleveref}
\crefname{Exercise}{exercise}{exercises}
\Crefname{Exercise}{Exercise}{Exercises}

\RequirePackage{titlesec}
\RequirePackage{ifthen}

% http://stackoverflow.com/questions/4932910/date-in-the-tabular-environment
\makeatletter
\let\insertdate\@date
\makeatother

\titleformat{\chapter}[display]
{\bfseries\Large}
{\color{DarkSlateGrey}\filleft \authorname
\ifthenelse{\isundefined{\studentnumber}}{}{\\ \studentnumber}
\ifthenelse{\isundefined{\email}}{}{\\ \email}
\ifthenelse{\isundefined{\dateintitle}}{}{\\ \insertdate}
%\ifthenelse{\isundefined{\coursename}}{}{\\ \coursename} % put in title instead.
}
{4ex}
{\color{DarkOliveGreen}{\titlerule}\color{Maroon}
\vspace{2ex}%
\filright}
[\vspace{2ex}%
\color{DarkOliveGreen}\titlerule
]

\newcommand{\beginArtWithToc}[0]{\begin{document}\tableofcontents}
\newcommand{\beginArtNoToc}[0]{\begin{document}}
\newcommand{\EndNoBibArticle}[0]{\end{document}}
\newcommand{\EndArticle}[0]{\bibliography{Bibliography}\bibliographystyle{plainnat}\end{document}}

% 
%\newcommand{\citep}[1]{\cite{#1}}

\colorSectionsForArticle



%\usepackage{ece1228}
\usepackage{peeters_braket}
%\usepackage{peeters_layout_exercise}
\usepackage{peeters_figures}
\usepackage{mathtools}
\usepackage{siunitx}

\beginArtNoToc
\generatetitle{ECE1228H Electromagnetic Theory.  Lecture 3: Electrostatics.  Taught by Prof.\ M. Mojahedi}
%\chapter{Electrostatics}
\label{chap:emt3}

\paragraph{Disclaimer}

Peeter's lecture notes from class.  These may be incoherent and rough.

These are notes for the UofT course ECE1228H, Electromagnetic Theory, taught by Prof. M. Mojahedi, covering \textchapref{{1}} \citep{balanis1989advanced} content.

\paragraph{Polarization and Magnetization}

The importance of the polarization and magnetization given by

\begin{dmath}\label{eqn:emtLecture3:20}
\begin{aligned}
\BD &= \epsilon_0 \BE + \BP \\
\BP &= \epsilon_0 \chi_\txte \BE,
\end{aligned}
\end{dmath}

where 
\begin{dmath}\label{eqn:emtLecture3:40}
\begin{aligned}
\BD &= \epsilon \BE \\
\epsilon &= \epsilon_0 \epsilon_r \\
\epsilon_r = 1 + \chi_\txte.
\end{aligned}
\end{dmath}

\paragraph{Point charge.}

\begin{dmath}\label{eqn:emtLecture3:60}
\BE 
= \frac{q}{4 \pi \epsilon_0} \frac{\rcap}{\Br^2}
= \frac{q}{4 \pi \epsilon_0} \frac{\Br}{\Abs{\Br}^3}
= \frac{q}{4 \pi \epsilon_0} \frac{\Br}{r^3}.
\end{dmath}

In more complex media the \( \epsilon_0 \) here can be replaced by \( \epsilon \).
Here the vector \( \Br \) points from the charge to the observation point.

Note that the class notes use \( \cap{a}_R \) instead of \( \rcap \). 

When the charge isn't located at the origin, we must modify this accordingly

\begin{dmath}\label{eqn:emtLecture3:80}
\BE 
= \frac{q}{4 \pi \epsilon_0} \frac{\BR}{\Abs{\BR}^3}
= \frac{q}{4 \pi \epsilon_0} \frac{\BR}{R^3},
\end{dmath}

where \( \BR = \Br - \Br' \) still points from the location of the charge to the point of observation, as sketched in

F1.

This can be further generalized to collections of point charges by superposition

\begin{dmath}\label{eqn:emtLecture3:100}
\BE 
= \frac{1}{4 \pi \epsilon_0} \sum_i q_i \frac{\Br - \Br_i'}{\Abs{\Br - \Br_i'}^3}.
\end{dmath}

Observe that a potential that satisfies \( \BE = - \spacegrad V \) can be defined as

\begin{dmath}\label{eqn:emtLecture3:120}
V  
= \frac{1}{4 \pi \epsilon_0} \sum_i \frac{q_i}{\Abs{\Br - \Br_i'}}.
\end{dmath}

When we are considering real world scenerios (like touching your hair, and then the table), how do we deal with the billions of charges involved.  This can be done by considering the charges so small that they can be approximated as a continuous distribution of charges.

This can be done by introducing the concept of a continuous charge distribution \( \rho_\txtv(\Br') \).
The charge that is in a small differential volume element \( dV' \) is \( \rho(\Br') dV' \), and 
the superposition has the form

\begin{dmath}\label{eqn:emtLecture3:140}
\BE 
= \frac{1}{4 \pi \epsilon_0} \iiint dV' \rho_\txtv(\Br') \frac{\Br - \Br'}{\Abs{\Br - \Br'}^3},
\end{dmath}

with potential

\begin{dmath}\label{eqn:emtLecture3:160}
V  
= \frac{1}{4 \pi \epsilon_0} \iiint dV' \frac{\rho(\Br')}{\Abs{\Br - \Br'}}.
\end{dmath}

The surface charge density analogue of this is

\begin{dmath}\label{eqn:emtLecture3:180}
\BE 
= \frac{1}{4 \pi \epsilon_0} \iint dA' \rho_\txts(\Br') \frac{\Br - \Br'}{\Abs{\Br - \Br'}^3},
\end{dmath}

with potential

\begin{dmath}\label{eqn:emtLecture3:200}
V  
= \frac{1}{4 \pi \epsilon_0} \iint dA' \frac{\rho_\txts(\Br')}{\Abs{\Br - \Br'}}.
\end{dmath}

The line charge density analogue of this is

\begin{dmath}\label{eqn:emtLecture3:220}
\BE 
= \frac{1}{4 \pi \epsilon_0} \int dl' \rho_\txtl(\Br') \frac{\Br - \Br'}{\Abs{\Br - \Br'}^3},
\end{dmath}

with potential

\begin{dmath}\label{eqn:emtLecture3:240}
V  
= \frac{1}{4 \pi \epsilon_0} \int dl' \frac{\rho_\txtl(\Br')}{\Abs{\Br - \Br'}}.
\end{dmath}

The difficulty with any of these approaches is the charge density is hardly ever known.  When the charge density is known, this sorts of integrals may not be analytically calculable, but they do yield to numeric calculation.

We may often prefer the potential calculations of the field calculations because they are much easier, having just one component to deal with.

\paragraph{Homework question:}

Starting from Maxwell's equations, in particular \( \oint \BE \cdot d\Bs = Q/\epsilon_0 \) that \( \BE = E_r \rcap \), and explicitly demonstrate why there is no \( \theta \) or \( \phi \) dependencies in this field.  Also calculate the potential \( V \propto 1/r \) associated with an electric field, and show that \( \BE = -\spacegrad V \), and show that this implies that \( -\int_b^a \BE \cdot d\Bl = V_a - V_b \).

\EndArticle
%\EndNoBibArticle

   \section{Problems}
      \input{Set2Problem1.tex}
      \input{Set2Problem4.tex}
      %
% Copyright � 2016 Peeter Joot.  All Rights Reserved.
% Licenced as described in the file LICENSE under the root directory of this GIT repository.
%
%{
%\newcommand{\authorname}{Peeter Joot}
\newcommand{\email}{peeterjoot@protonmail.com}
\newcommand{\basename}{FIXMEbasenameUndefined}
\newcommand{\dirname}{notes/FIXMEdirnameUndefined/}

%\renewcommand{\basename}{dipoleMoment}
%%\renewcommand{\dirname}{notes/phy1520/}
%\renewcommand{\dirname}{notes/ece1228-electromagnetic-theory/}
%%\newcommand{\dateintitle}{}
%%\newcommand{\keywords}{}
%
%\newcommand{\authorname}{Peeter Joot}
\newcommand{\onlineurl}{http://sites.google.com/site/peeterjoot2/math2013/\basename.pdf}
\newcommand{\sourcepath}{\dirname\basename.tex}
\newcommand{\generatetitle}[1]{\chapter{#1}}

\newcommand{\vcsinfo}{%
\section*{}
\noindent{\color{DarkOliveGreen}{\rule{\linewidth}{0.1mm}}}
\paragraph{Document version}
%\paragraph{\color{Maroon}{Document version}}
{
\small
\begin{itemize}
\item Available online at:\\ 
\href{\onlineurl}{\onlineurl}
\item Git Repository: \input{./.revinfo/gitRepo.tex}
\item Source: \sourcepath
\item last commit: \input{./.revinfo/gitCommitString.tex}
\item commit date: \input{./.revinfo/gitCommitDate.tex}
\end{itemize}
}
}

%\PassOptionsToPackage{dvipsnames,svgnames}{xcolor}
\PassOptionsToPackage{square,numbers}{natbib}
\documentclass{scrreprt}

\usepackage[left=2cm,right=2cm]{geometry}
\usepackage[svgnames]{xcolor}
\usepackage{peeters_layout}

\usepackage{natbib}

\usepackage[
colorlinks=true,
bookmarks=false,
pdfauthor={\authorname, \email},
backref 
]{hyperref}

% http://tex.stackexchange.com/questions/75773/how-to-reference-problems-by-the-text-label-in-an-exercise-envioronment
\usepackage[english]{cleveref}
\crefname{Exercise}{exercise}{exercises}
\Crefname{Exercise}{Exercise}{Exercises}

\RequirePackage{titlesec}
\RequirePackage{ifthen}

% http://stackoverflow.com/questions/4932910/date-in-the-tabular-environment
\makeatletter
\let\insertdate\@date
\makeatother

\titleformat{\chapter}[display]
{\bfseries\Large}
{\color{DarkSlateGrey}\filleft \authorname
\ifthenelse{\isundefined{\studentnumber}}{}{\\ \studentnumber}
\ifthenelse{\isundefined{\email}}{}{\\ \email}
\ifthenelse{\isundefined{\dateintitle}}{}{\\ \insertdate}
%\ifthenelse{\isundefined{\coursename}}{}{\\ \coursename} % put in title instead.
}
{4ex}
{\color{DarkOliveGreen}{\titlerule}\color{Maroon}
\vspace{2ex}%
\filright}
[\vspace{2ex}%
\color{DarkOliveGreen}\titlerule
]

\newcommand{\beginArtWithToc}[0]{\begin{document}\tableofcontents}
\newcommand{\beginArtNoToc}[0]{\begin{document}}
\newcommand{\EndNoBibArticle}[0]{\end{document}}
\newcommand{\EndArticle}[0]{\bibliography{Bibliography}\bibliographystyle{plainnat}\end{document}}

% 
%\newcommand{\citep}[1]{\cite{#1}}

\colorSectionsForArticle


%
%\usepackage{peeters_layout_exercise}
%\usepackage{peeters_braket}
%\usepackage{peeters_figures}
%\usepackage{siunitx}
%
%\beginArtNoToc
%
%\generatetitle{Dipole moment}
%\chapter{Dipole moment}
%\label{chap:dipoleMoment}

\makeproblem{Field for an electric dipole.}{problem:dipoleMoment:dipole}{
\index{dipole!electric}

An equal charge dipole configuration is sketched in \cref{fig:dipoleSignConventionL3:dipoleSignConventionL3Fig3}.  Compute the electric field.

% L3:
%\imageFigure{../../figures/ece1228-emt/dipoleSignConventionL3Fig3}{Dipole sign convention.}{fig:dipoleSignConventionL3:dipoleSignConventionL3Fig3}{0.3}
} % problem

\makeanswer{problem:dipoleMoment:dipole}{
The vector from the origin to the observation point is

\begin{equation}\label{eqn:dipoleMoment:20}
\Br = \BR_1 + \Bd/2
= \BR_2 - \Bd/2,
\end{equation}

or

\begin{equation}\label{eqn:dipoleMoment:40}
\begin{aligned}
\BR_1 &= \Br - \Bd/2 \equiv \BR_{+} \\
\BR_2 &= \Br + \Bd/2 \equiv \BR_{-}.
\end{aligned}
\end{equation}

The electric field for this superposition is
\begin{dmath}\label{eqn:dipoleMoment:60}
\BE
=
\inv{4 \pi \epsilon_0} \lr{
\frac{q \BR_{+}}{\Abs{\BR_{+}}^3} -
\frac{q \BR_{-}}{\Abs{\BR_{-}}^3}
}
=
\frac{q}{4 \pi \epsilon_0} \lr{
\frac{\Br - \Bd/2}{\Abs{\BR_{+}}^3} -
\frac{\Br + \Bd/2}{\Abs{\BR_{-}}^3}
}
=
\frac{q}{4 \pi \epsilon_0} \lr{
\Br \lr{
\inv{\Abs{\BR_{+}}^3}
 -
\inv{\Abs{\BR_{-}}^3}
}
-
\frac{\Bd}{2} \lr{
\inv{\Abs{\BR_{+}}^3}
+
\inv{\Abs{\BR_{-}}^3}
}
}.
\end{dmath}

The magnitudes can be expanded in Taylor series

\begin{dmath}\label{eqn:dipoleMoment:80}
\Abs{\BR_{\pm}}^{3}
=
\lr{
\lr{ \Br \mp \Bd/2 } \cdot \lr{ \Br \mp \Bd/2 }
}^{-3/2}
=
\lr{
\lr{ \Br^2 + (\Bd/2)^2 \mp 2 \Br \cdot \Bd/2 }
}^{-3/2}
=
\lr{
\lr{ \Br^2 + (\Bd/2)^2 \mp \Br \cdot \Bd }
}^{-3/2}
=
(\Br^2)^{-3/2}
\lr{
\lr{ 1 + \lr{\frac{\Bd}{2 r}}^2 \mp \rcap \cdot \frac{\Bd}{r} }
}^{-1/2}
=
r^{-3}
\lr{
1
-\frac{3}{2}
\lr{ \lr{\frac{\Bd}{2 r}}^2 \mp \rcap \cdot \frac{\Bd}{r} }
+\lr{\frac{-3}{2}}
\lr{\frac{-5}{2}} \inv{2!}
\lr{ \lr{\frac{\Bd}{2 r}}^2 \mp \rcap \cdot \frac{\Bd}{r} }^2
+ \cdots
}.
\end{dmath}

Here \( r = \Abs{\Br} \), and the Taylor series was taken in the \( \Bd/r \ll 1 \) limit.  The sums and differences of these magnitudes, are to first order

\begin{dmath}\label{eqn:dipoleMoment:100}
\inv{\Abs{\BR_{+}}^3}
-
\inv{\Abs{\BR_{-}}^3}
=
2 \frac{1}{r^3} \lr{\frac{-3}{2}} \lr{-\rcap \cdot \frac{\Bd}{r}}
\approx
\frac{3}{r^4} \rcap \cdot \Bd,
\end{dmath}

and

\begin{dmath}\label{eqn:dipoleMoment:120}
\inv{\Abs{\BR_{+}}^3}
+
\inv{\Abs{\BR_{-}}^3}
\approx
\frac{2}{r^3}.
\end{dmath}

The \( \Br \gg \Bd \) limiting expression for the electric field is

\begin{dmath}\label{eqn:dipoleMoment:140}
\BE
\approx
\frac{q}{4 \pi \epsilon_0 r^3} \lr{
3 \rcap \lr{ \rcap \cdot \Bd }
-
2 \frac{\Bd}{2}
},
\end{dmath}

or, with \( \Bp = q \Bd \)

%\begin{dmath}\label{eqn:dipoleMoment:180}
\boxedEquation{eqn:dipoleMoment:180}{
\BE =
\frac{1}{4 \pi \epsilon_0 r^3} \lr{
3 \rcap \lr{ \rcap \cdot \Bp }
-\Bp
}.
}
%\end{dmath}

} % answer

%}
%\EndNoBibArticle

      %
% Copyright � 2016 Peeter Joot.  All Rights Reserved.
% Licenced as described in the file LICENSE under the root directory of this GIT repository.
%
%{
\newcommand{\authorname}{Peeter Joot}
\newcommand{\email}{peeterjoot@protonmail.com}
\newcommand{\basename}{FIXMEbasenameUndefined}
\newcommand{\dirname}{notes/FIXMEdirnameUndefined/}

\renewcommand{\basename}{dipolePotential}
%\renewcommand{\dirname}{notes/phy1520/}
\renewcommand{\dirname}{notes/ece1228-electromagnetic-theory/}
%\newcommand{\dateintitle}{}
%\newcommand{\keywords}{}

\newcommand{\authorname}{Peeter Joot}
\newcommand{\onlineurl}{http://sites.google.com/site/peeterjoot2/math2013/\basename.pdf}
\newcommand{\sourcepath}{\dirname\basename.tex}
\newcommand{\generatetitle}[1]{\chapter{#1}}

\newcommand{\vcsinfo}{%
\section*{}
\noindent{\color{DarkOliveGreen}{\rule{\linewidth}{0.1mm}}}
\paragraph{Document version}
%\paragraph{\color{Maroon}{Document version}}
{
\small
\begin{itemize}
\item Available online at:\\ 
\href{\onlineurl}{\onlineurl}
\item Git Repository: \input{./.revinfo/gitRepo.tex}
\item Source: \sourcepath
\item last commit: \input{./.revinfo/gitCommitString.tex}
\item commit date: \input{./.revinfo/gitCommitDate.tex}
\end{itemize}
}
}

%\PassOptionsToPackage{dvipsnames,svgnames}{xcolor}
\PassOptionsToPackage{square,numbers}{natbib}
\documentclass{scrreprt}

\usepackage[left=2cm,right=2cm]{geometry}
\usepackage[svgnames]{xcolor}
\usepackage{peeters_layout}

\usepackage{natbib}

\usepackage[
colorlinks=true,
bookmarks=false,
pdfauthor={\authorname, \email},
backref 
]{hyperref}

% http://tex.stackexchange.com/questions/75773/how-to-reference-problems-by-the-text-label-in-an-exercise-envioronment
\usepackage[english]{cleveref}
\crefname{Exercise}{exercise}{exercises}
\Crefname{Exercise}{Exercise}{Exercises}

\RequirePackage{titlesec}
\RequirePackage{ifthen}

% http://stackoverflow.com/questions/4932910/date-in-the-tabular-environment
\makeatletter
\let\insertdate\@date
\makeatother

\titleformat{\chapter}[display]
{\bfseries\Large}
{\color{DarkSlateGrey}\filleft \authorname
\ifthenelse{\isundefined{\studentnumber}}{}{\\ \studentnumber}
\ifthenelse{\isundefined{\email}}{}{\\ \email}
\ifthenelse{\isundefined{\dateintitle}}{}{\\ \insertdate}
%\ifthenelse{\isundefined{\coursename}}{}{\\ \coursename} % put in title instead.
}
{4ex}
{\color{DarkOliveGreen}{\titlerule}\color{Maroon}
\vspace{2ex}%
\filright}
[\vspace{2ex}%
\color{DarkOliveGreen}\titlerule
]

\newcommand{\beginArtWithToc}[0]{\begin{document}\tableofcontents}
\newcommand{\beginArtNoToc}[0]{\begin{document}}
\newcommand{\EndNoBibArticle}[0]{\end{document}}
\newcommand{\EndArticle}[0]{\bibliography{Bibliography}\bibliographystyle{plainnat}\end{document}}

% 
%\newcommand{\citep}[1]{\cite{#1}}

\colorSectionsForArticle



\usepackage{peeters_layout_exercise}
\usepackage{peeters_braket}
\usepackage{peeters_figures}
\usepackage{siunitx}

\beginArtNoToc

\generatetitle{Electric dipole potential}
%\chapter{Electric dipole potential}
%\label{chap:dipolePotential}
% \citep{sakurai2014modern} pr X.Y
% \citep{pozar2009microwave}
% \citep{qftLectureNotes}
% \citep{doran2003gap}
% \citep{jackson1975cew}
% \citep{griffiths1999introduction}

\makeproblem{Electric dipole potential}{problem:dipolePotential:1}{

Having shown that 

\begin{dmath}\label{eqn:dipolePotential:20}
\BE =
\frac{1}{4 \pi \epsilon_0 r^3} \lr{ 
3 \rcap \lr{ \rcap \cdot \Bp }
-\Bp 
},
\end{dmath}

find the expression for the electric potential for this field.
} % problem

\makeanswer{problem:dipolePotential:1}{

With the electric potential defined indirectly by
\begin{dmath}\label{eqn:dipolePotential:40}
\BE = -\spacegrad V,
\end{dmath}

we can integrate to find the difference in potential between two points
\begin{dmath}\label{eqn:dipolePotential:60}
\int_\Ba^\Bb \BE \cdot d\Bl = 
- \int
\int_\Ba^\Bb \spacegrad V \cdot d\Bl
=
- \lr{ V(\Bb) - V(\Ba) },
\end{dmath}

or
\begin{dmath}\label{eqn:dipolePotential:80}
V(\Bb) - V(\Ba) = -
\int_\Ba^\Bb \BE \cdot d\Bl.
\end{dmath}

Since the dipole potential is zero at \( \Br = \infty \), we have

\begin{dmath}\label{eqn:dipolePotential:100}
V(\Br) 
= -\int_\infty^\Br \BE \cdot d\Bl.
\end{dmath}

Let's integrate this on the radial path \( \Br(r') = r'\rcap \), for \( r' \in [\infty, r] \)

\begin{dmath}\label{eqn:dipolePotential:120}
V(\Br) 
= -\int_\infty^\Br \BE \cdot d\Bl
= -\int_\infty^\Br \BE \cdot \rcap dr'
= 
-
\frac{1}{4 \pi \epsilon_0 } 
\int_\infty^r \frac{dr'}{{r'}^3}
\rcap
\cdot
\lr{ 
3 \rcap \lr{ \rcap \cdot \Bp }
-\Bp 
}
=
-\frac{2}{4 \pi \epsilon_0 } 
\int_\infty^r dr' \frac{\rcap\cdot \Bp}{{r'}^3}
=
\frac{\rcap \cdot \Bp}{4 \pi \epsilon_0 } \evalrange{ \inv{{r'}^2} }{\infty}{r},
\end{dmath}

so
%\begin{dmath}\label{eqn:dipolePotential:160}
\boxedEquation{eqn:dipolePotential:140}{
V(\Br) =
\frac{ \rcap \cdot \Bp}{4 \pi \epsilon_0 }.
}
%\end{dmath}
} % answer

%}
%\EndArticle
\EndNoBibArticle


   %
% Copyright � 2016 Peeter Joot.  All Rights Reserved.
% Licenced as described in the file LICENSE under the root directory of this GIT repository.
%
\newcommand{\authorname}{Peeter Joot}
\newcommand{\email}{peeterjoot@protonmail.com}
\newcommand{\basename}{FIXMEbasenameUndefined}
\newcommand{\dirname}{notes/FIXMEdirnameUndefined/}

\renewcommand{\basename}{emt4}
\renewcommand{\dirname}{notes/ece1228/}
\newcommand{\keywords}{ECE1228H}
\newcommand{\authorname}{Peeter Joot}
\newcommand{\onlineurl}{http://sites.google.com/site/peeterjoot2/math2013/\basename.pdf}
\newcommand{\sourcepath}{\dirname\basename.tex}
\newcommand{\generatetitle}[1]{\chapter{#1}}

\newcommand{\vcsinfo}{%
\section*{}
\noindent{\color{DarkOliveGreen}{\rule{\linewidth}{0.1mm}}}
\paragraph{Document version}
%\paragraph{\color{Maroon}{Document version}}
{
\small
\begin{itemize}
\item Available online at:\\ 
\href{\onlineurl}{\onlineurl}
\item Git Repository: \input{./.revinfo/gitRepo.tex}
\item Source: \sourcepath
\item last commit: \input{./.revinfo/gitCommitString.tex}
\item commit date: \input{./.revinfo/gitCommitDate.tex}
\end{itemize}
}
}

%\PassOptionsToPackage{dvipsnames,svgnames}{xcolor}
\PassOptionsToPackage{square,numbers}{natbib}
\documentclass{scrreprt}

\usepackage[left=2cm,right=2cm]{geometry}
\usepackage[svgnames]{xcolor}
\usepackage{peeters_layout}

\usepackage{natbib}

\usepackage[
colorlinks=true,
bookmarks=false,
pdfauthor={\authorname, \email},
backref 
]{hyperref}

% http://tex.stackexchange.com/questions/75773/how-to-reference-problems-by-the-text-label-in-an-exercise-envioronment
\usepackage[english]{cleveref}
\crefname{Exercise}{exercise}{exercises}
\Crefname{Exercise}{Exercise}{Exercises}

\RequirePackage{titlesec}
\RequirePackage{ifthen}

% http://stackoverflow.com/questions/4932910/date-in-the-tabular-environment
\makeatletter
\let\insertdate\@date
\makeatother

\titleformat{\chapter}[display]
{\bfseries\Large}
{\color{DarkSlateGrey}\filleft \authorname
\ifthenelse{\isundefined{\studentnumber}}{}{\\ \studentnumber}
\ifthenelse{\isundefined{\email}}{}{\\ \email}
\ifthenelse{\isundefined{\dateintitle}}{}{\\ \insertdate}
%\ifthenelse{\isundefined{\coursename}}{}{\\ \coursename} % put in title instead.
}
{4ex}
{\color{DarkOliveGreen}{\titlerule}\color{Maroon}
\vspace{2ex}%
\filright}
[\vspace{2ex}%
\color{DarkOliveGreen}\titlerule
]

\newcommand{\beginArtWithToc}[0]{\begin{document}\tableofcontents}
\newcommand{\beginArtNoToc}[0]{\begin{document}}
\newcommand{\EndNoBibArticle}[0]{\end{document}}
\newcommand{\EndArticle}[0]{\bibliography{Bibliography}\bibliographystyle{plainnat}\end{document}}

% 
%\newcommand{\citep}[1]{\cite{#1}}

\colorSectionsForArticle



%\usepackage{ece1228}
\usepackage{peeters_braket}
%\usepackage{peeters_layout_exercise}
\usepackage{peeters_figures}
\usepackage{mathtools}
\usepackage{siunitx}

\beginArtNoToc
\generatetitle{ECE1228H Electromagnetic Theory.  Lecture 4: Magnetic moment, and Boundary value conditions.  Taught by Prof.\ M. Mojahedi}
%\chapter{Magnetic moment, and Boundary value conditions}
\label{chap:emt4}

%\paragraph{Disclaimer}
%
%Peeter's lecture notes from class.  These may be incoherent and rough.
%
%These are notes for the UofT course ECE1228H, Electromagnetic Theory, taught by Prof. M. Mojahedi, covering \textchapref{{1}} \citep{balanis1989advanced} content.

\paragraph{Magnetic moment.}

Using a semi-classical model of an electron, assuming that the electron circles the nuclei.  This is a completely wrong model, but useful.  In reality, electrons are random and probabilistic and do not follow defined paths.  We do however have a magnetic moment associated with the electron, and one associated with the spin of the electron, and a moment associated with the spin of the nuclei.  All of these concepts can be used to describe a more accurate model and such a model is discussed in \citep{jackson1975cew} chapters 11,12,13.

Ignoring the details of how the moments really occur physically, we can take it as a given that they exist, and model them as elemenetal magnetic dipole moments of the form

\begin{dmath}\label{eqn:emtLecture4:20}
d\Bm_i = \ncap_i I_i ds_i \qquad [\si{A m^2}].
\end{dmath}

Here the normal is defined in terms of the right hand rule with respect to the direction of the current as sketched in 

F1

Such dipole moments are actually what an MRI measures.  The noises that people describe from MRI machines are actually when the very powerful magnets are being rotated, allowing for the magnetic moments in the atoms of the body to be measured in different directions.

The magnetic polarization, or magnetization \( \BM \), in [\si{A/m}]] is given by

\begin{dmath}\label{eqn:emtLecture4:40}
\BM 
= \lim_{\Delta v \rightarrow 0} \lr{ \inv{\Delta v} \Bm_i }
= \lim_{\Delta v \rightarrow 0} \lr{ \inv{\Delta v} \sum_{i = 1}^{N \delta v} d\Bm_i }
= \lim_{\Delta v \rightarrow 0} \lr{ \inv{\Delta v} \sum_{i = 1}^{N \delta v} \ncap_i I_i ds_i } .
\end{dmath}

In materials the magnetization within the atoms are usually random, however, application of a magnetic field can force these to line up, as sketched in

F2

This is accomplished because an applied magnetic field acting on the magnetic moment introduces a torque, as also occured with dipole moments under applied electric fields

\begin{dmath}\label{eqn:emtLecture4:60}
\begin{aligned}
\Btau_B &= d\Bm \cross \BB_a \\
\Btau_E &= d\Bp \cross \BE_a.
\end{aligned}
\end{dmath}

There is an energy associated with this torque

\begin{dmath}\label{eqn:emtLecture4:80}
\begin{aligned}
\Delta U_B &= -d\Bm \cdot \BB_a \\
\Delta U_E &= -d\Bp \cdot \BE_a.
\end{aligned}
\end{dmath}

In analogy with the electric dipole moment analysis, it can be assumed that there is a linear relationship between the magnetic polarization and the applied magnetic field

\begin{dmath}\label{eqn:emtLecture4:100}
\BB = \mu_0 \BH_a + \mu_0 \BM = \mu_0\lr{ \BH_a + \BM },
\end{dmath}

where
\begin{dmath}\label{eqn:emtLecture4:120}
\BM = \chi_m \BH_a,
\end{dmath}

so
\begin{equation}\label{eqn:emtLecture4:140}
\BB 
= \mu_0\lr{ 1 + \chi_m } \BH_a
\equiv \mu \BH_a.
\end{equation}

Like electric dipoles, in a volume, we can have bound currents on the surface [\si{A/m}], as well as bound volume currents [\si{A/m^2}] as sketched in

F3

It can be shown, as with the electric dipoles related bound charge densities of \crefeqn:emtLecture3:620}, that magnetic currents can be defined

\begin{dmath}\label{eqn:emtLecture4:n}
\begin{aligned}
\BJ_{sm} &= \BM \cross \ncap \\
\BJ_{vm} &= \spacegrad \cross \BM,
\end{aligned}
\end{dmath}

(showing this may be given as a homework problem ... if not do it).

\paragraph{Conductivity}

We have two constitutive relationships so far
\begin{dmath}\label{eqn:emtLecture4:n}
\begin{aligned}
\BD &= \epsilon \BE \\
\BB &= \mu \BH
\end{aligned}
\end{dmath}

but this needs to be augmented by

\begin{dmath}\label{eqn:emtLecture4:n}
\BJ_c = \epsilon \BE.
\end{dmath}

There are a couple ways to discuss this.  One is to model \( \epsilon \) as a complex number.  Such a model is not entirely unconstrained.  Like with the Cauchy-Riemann conditions that relate derivatives of the real and imaginary parts of a complex number, there is a relationship (Kramers-Kronig \citep{wiki:kramersKronig}), an integral relationship that relates the real and imaginary parts of the permittivity \( \epsilon \).

\paragraph{Boundary conditions.}

\EndArticle

      \section{Problems}
      %
% Copyright � 2016 Peeter Joot.  All Rights Reserved.
% Licenced as described in the file LICENSE under the root directory of this GIT repository.
%
%{
\newcommand{\authorname}{Peeter Joot}
\newcommand{\email}{peeterjoot@protonmail.com}
\newcommand{\basename}{FIXMEbasenameUndefined}
\newcommand{\dirname}{notes/FIXMEdirnameUndefined/}

\renewcommand{\basename}{magneticMomentJackson}
%\renewcommand{\dirname}{notes/phy1520/}
\renewcommand{\dirname}{notes/ece1228-electromagnetic-theory/}
%\newcommand{\dateintitle}{}
%\newcommand{\keywords}{}

\newcommand{\authorname}{Peeter Joot}
\newcommand{\onlineurl}{http://sites.google.com/site/peeterjoot2/math2013/\basename.pdf}
\newcommand{\sourcepath}{\dirname\basename.tex}
\newcommand{\generatetitle}[1]{\chapter{#1}}

\newcommand{\vcsinfo}{%
\section*{}
\noindent{\color{DarkOliveGreen}{\rule{\linewidth}{0.1mm}}}
\paragraph{Document version}
%\paragraph{\color{Maroon}{Document version}}
{
\small
\begin{itemize}
\item Available online at:\\ 
\href{\onlineurl}{\onlineurl}
\item Git Repository: \input{./.revinfo/gitRepo.tex}
\item Source: \sourcepath
\item last commit: \input{./.revinfo/gitCommitString.tex}
\item commit date: \input{./.revinfo/gitCommitDate.tex}
\end{itemize}
}
}

%\PassOptionsToPackage{dvipsnames,svgnames}{xcolor}
\PassOptionsToPackage{square,numbers}{natbib}
\documentclass{scrreprt}

\usepackage[left=2cm,right=2cm]{geometry}
\usepackage[svgnames]{xcolor}
\usepackage{peeters_layout}

\usepackage{natbib}

\usepackage[
colorlinks=true,
bookmarks=false,
pdfauthor={\authorname, \email},
backref 
]{hyperref}

% http://tex.stackexchange.com/questions/75773/how-to-reference-problems-by-the-text-label-in-an-exercise-envioronment
\usepackage[english]{cleveref}
\crefname{Exercise}{exercise}{exercises}
\Crefname{Exercise}{Exercise}{Exercises}

\RequirePackage{titlesec}
\RequirePackage{ifthen}

% http://stackoverflow.com/questions/4932910/date-in-the-tabular-environment
\makeatletter
\let\insertdate\@date
\makeatother

\titleformat{\chapter}[display]
{\bfseries\Large}
{\color{DarkSlateGrey}\filleft \authorname
\ifthenelse{\isundefined{\studentnumber}}{}{\\ \studentnumber}
\ifthenelse{\isundefined{\email}}{}{\\ \email}
\ifthenelse{\isundefined{\dateintitle}}{}{\\ \insertdate}
%\ifthenelse{\isundefined{\coursename}}{}{\\ \coursename} % put in title instead.
}
{4ex}
{\color{DarkOliveGreen}{\titlerule}\color{Maroon}
\vspace{2ex}%
\filright}
[\vspace{2ex}%
\color{DarkOliveGreen}\titlerule
]

\newcommand{\beginArtWithToc}[0]{\begin{document}\tableofcontents}
\newcommand{\beginArtNoToc}[0]{\begin{document}}
\newcommand{\EndNoBibArticle}[0]{\end{document}}
\newcommand{\EndArticle}[0]{\bibliography{Bibliography}\bibliographystyle{plainnat}\end{document}}

% 
%\newcommand{\citep}[1]{\cite{#1}}

\colorSectionsForArticle



\usepackage{peeters_layout_exercise}
\usepackage{peeters_braket}
\usepackage{peeters_figures}
\usepackage{siunitx}
%\usepackage{txfonts} % \ointclockwise

\beginArtNoToc

\generatetitle{Magnetic moment for a localized magnetostatic current}
%\chapter{Magnetic moment for a localized magnetostatic current}
%\label{chap:magneticMomentJackson}
% \citep{sakurai2014modern} pr X.Y
% \citep{pozar2009microwave}
% \citep{qftLectureNotes}
% \citep{doran2003gap}
\paragraph{Motivation.}

I was once again reading my Jackson \citep{jackson1975cew}.  This time I found that his 
presentation of magnetic moment didn't really make sense to me.  Here's my own pass through it, filling in a number of details.  As I did last time, I'll also translate into SI units as I go.

\paragraph{Vector potential.}

The Biot-Savart expression for the magnetic field can be factored into a curl expression using the usual tricks

\begin{dmath}\label{eqn:magneticMomentJackson:20}
\BB 
= \frac{\mu_0}{4\pi} \int \frac{\BJ(\Bx') \cross (\Bx - \Bx')}{\Abs{\Bx - \Bx'}^3} d^3 x'
= -\frac{\mu_0}{4\pi} \int \BJ(\Bx') \cross \spacegrad \inv{\Abs{\Bx - \Bx'}} d^3 x'
= \frac{\mu_0}{4\pi} \spacegrad \cross \int \frac{\BJ(\Bx')}{\Abs{\Bx - \Bx'}} d^3 x',
\end{dmath}

so the vector potential, through its curl, defines the magnetic field \( \BB = \spacegrad \cross \BA \) is given by

\begin{dmath}\label{eqn:magneticMomentJackson:40}
\BA(\Bx) = \frac{\mu_0}{4 \pi} \int \frac{J(\Bx')}{\Abs{\Bx - \Bx'}} d^3 x'.
\end{dmath}

If the current source is localized (zero outside of some finite region), then there will always be a region for which \( \Abs{\Bx} \gg \Abs{\Bx'} \), so the denominator yields to Taylor expansion

\begin{dmath}\label{eqn:magneticMomentJackson:60}
\inv{\Abs{\Bx - \Bx'}}
=
\inv{\Abs{\Bx}} \lr{1 + \frac{\Abs{\Bx'}^2}{\Abs{\Bx}^2} - 2 \frac{\Bx \cdot \Bx'}{\Abs{\Bx}^2} }^{-1/2}
\approx
\inv{\Abs{\Bx}} \lr{ 1 + \frac{\Bx \cdot \Bx'}{\Abs{\Bx}^2} }
=
\inv{\Abs{\Bx}} + \frac{\Bx \cdot \Bx'}{\Abs{\Bx}^3}.
\end{dmath}

so the vector potential, far enough away from the current source is
\begin{dmath}\label{eqn:magneticMomentJackson:80}
\BB(\Bx) 
=
\frac{\mu_0}{4 \pi} \int \frac{J(\Bx')}{\Abs{\Bx}} d^3 x'
+\frac{\mu_0}{4 \pi} \int \frac{(\Bx \cdot \Bx')J(\Bx')}{\Abs{\Bx}^3} d^3 x'.
\end{dmath}

Jackson uses a sneaky trick to show that the first integral is killed for a localized source.  That trick appears to be based on evaluating the following divergence

\begin{dmath}\label{eqn:magneticMomentJackson:100}
\spacegrad \cdot (\BJ(\Bx) x_i)
=
(\spacegrad \cdot \BJ) x_i
+
(\spacegrad x_i) \cdot \BJ
=
(\Be_k \partial_k x_i) \cdot\BJ
=
\delta_{ki} J_k
=
J_i.
\end{dmath}

Note that this made use of the fact that \( \spacegrad \cdot \BJ = 0 \) for magnetostatics.  This provides a way to rewrite the current density as a divergence

\begin{dmath}\label{eqn:magneticMomentJackson:120}
\int \frac{J(\Bx')}{\Abs{\Bx}} d^3 x'
=
\Be_i \int \frac{\spacegrad' \cdot (x_i' \BJ(\Bx'))}{\Abs{\Bx}} d^3 x'
=
\frac{\Be_i}{\Abs{\Bx}} \int \spacegrad' \cdot (x_i' \BJ(\Bx')) d^3 x'
=
\frac{1}{\Abs{\Bx}} \oint \Bx' (d\Ba \cdot \BJ(\Bx')).
\end{dmath}

When \( \BJ \) is localized, this is zero provided we pick the integration surface for the volume outside of that localization region.

It is now desired to rewrite \( \int \Bx \cdot \Bx' \BJ \) as a triple cross product since the dot product of such a triple cross product has exactly this term in it

\begin{dmath}\label{eqn:magneticMomentJackson:140}
- \Bx \cross \int \Bx' \cross \BJ
=
\int (\Bx \cdot \Bx') \BJ
-
\int (\Bx \cdot \BJ) \Bx'
=
\int (\Bx \cdot \Bx') \BJ
-
\Be_k x_i \int J_i x_k',
\end{dmath}

so
\begin{dmath}\label{eqn:magneticMomentJackson:160}
\int (\Bx \cdot \Bx') \BJ
=
- \Bx \cross \int \Bx' \cross \BJ
+
\Be_k x_i \int J_i x_k'.
\end{dmath}

To get of this second term, the next sneaky trick is to consider the following divergence

\begin{dmath}\label{eqn:magneticMomentJackson:180}
\oint d\Ba' \cdot (\BJ(\Bx') x_i' x_j')
=
\int dV' \spacegrad' \cdot (\BJ(\Bx') x_i' x_j')
=
\int dV' (\spacegrad' \cdot \BJ)
+
\int dV' \BJ \cdot \spacegrad' (x_i' x_j')
=
\int dV' J_k \cdot \lr{ x_i' \partial_k x_j' + x_j' \partial_k x_i' }
=
\int dV' \lr{ J_k x_i' \delta_{kj} + J_k x_j' \delta_{ki} }
=
\int dV' \lr{ J_j x_i' + J_i x_j'}.
\end{dmath}

The surface integral is once again zero, which means that we have an antisymmetric relationship in integrals of the form

\begin{dmath}\label{eqn:magneticMomentJackson:200}
\int J_j x_i' = -\int J_i x_j'.
\end{dmath}

Now we can use the tensor algebra trick of writing \( y = (y + y)/2 \),

\begin{dmath}\label{eqn:magneticMomentJackson:220}
\int (\Bx \cdot \Bx') \BJ
=
- \Bx \cross \int \Bx' \cross \BJ
+
\Be_k x_i \int J_i x_k'
=
- \Bx \cross \int \Bx' \cross \BJ
+
\inv{2} \Be_k x_i \int \lr{ J_i x_k' + J_i x_k' }
=
- \Bx \cross \int \Bx' \cross \BJ
+
\inv{2} \Be_k x_i \int \lr{ J_i x_k' - J_k x_i' }
=
- \Bx \cross \int \Bx' \cross \BJ
+
\inv{2} \Be_k x_i \int (\BJ \cross \Bx')_j \epsilon_{ikj}
=
- \Bx \cross \int \Bx' \cross \BJ
-
\inv{2} \epsilon_{kij} \Be_k x_i \int (\BJ \cross \Bx')_j 
=
- \Bx \cross \int \Bx' \cross \BJ
-
\inv{2} \Bx \cross \int \BJ \cross \Bx'
=
- \Bx \cross \int \Bx' \cross \BJ
+
\inv{2} \Bx \cross \int \Bx' \cross \BJ
=
-\inv{2} \Bx \cross \int \Bx' \cross \BJ,
\end{dmath}

so

\begin{dmath}\label{eqn:magneticMomentJackson:240}
\BA(\Bx) \approx \frac{\mu_0}{4 \pi \Abs{\Bx}^3} \lr{ -\frac{\Bx}{2} } \int \Bx' \cross \BJ(\Bx') d^3 x'.
\end{dmath}

Letting 

%\begin{dmath}\label{eqn:magneticMomentJackson:260}
\boxedEquation{eqn:magneticMomentJackson:260}{
\Bm = \inv{2} \int \Bx' \cross \BJ(\Bx') d^3 x',
}
%\end{dmath}

the far field approximation of the vector potential is
%\begin{dmath}\label{eqn:magneticMomentJackson:280}
\boxedEquation{eqn:magneticMomentJackson:280}{
\BA(\Bx) = \frac{\mu_0}{4 \pi} \frac{\Bm \cross \Bx}{\Abs{\Bx}^3}.
}
%\end{dmath}

Note that when the current is restricted to an infintisimally thin loop, the magnetic moment reduces to

\begin{dmath}\label{eqn:magneticMomentJackson:300}
\Bm(\Bx) = \frac{I}{2} \int \Bx \cross d\Bl'.
\end{dmath}

Refering to \citep{griffiths1999introduction} (pr. 1.60), this can be seen to be \( I \) times the ``vector-area'' integral.

%}
\EndArticle

      %
% Copyright � 2016 Peeter Joot.  All Rights Reserved.
% Licenced as described in the file LICENSE under the root directory of this GIT repository.
%
%{
\newcommand{\authorname}{Peeter Joot}
\newcommand{\email}{peeterjoot@protonmail.com}
\newcommand{\basename}{FIXMEbasenameUndefined}
\newcommand{\dirname}{notes/FIXMEdirnameUndefined/}

\renewcommand{\basename}{vectorAreaGriffiths}
%\renewcommand{\dirname}{notes/phy1520/}
\renewcommand{\dirname}{notes/ece1228-electromagnetic-theory/}
%\newcommand{\dateintitle}{}
%\newcommand{\keywords}{}

\newcommand{\authorname}{Peeter Joot}
\newcommand{\onlineurl}{http://sites.google.com/site/peeterjoot2/math2013/\basename.pdf}
\newcommand{\sourcepath}{\dirname\basename.tex}
\newcommand{\generatetitle}[1]{\chapter{#1}}

\newcommand{\vcsinfo}{%
\section*{}
\noindent{\color{DarkOliveGreen}{\rule{\linewidth}{0.1mm}}}
\paragraph{Document version}
%\paragraph{\color{Maroon}{Document version}}
{
\small
\begin{itemize}
\item Available online at:\\ 
\href{\onlineurl}{\onlineurl}
\item Git Repository: \input{./.revinfo/gitRepo.tex}
\item Source: \sourcepath
\item last commit: \input{./.revinfo/gitCommitString.tex}
\item commit date: \input{./.revinfo/gitCommitDate.tex}
\end{itemize}
}
}

%\PassOptionsToPackage{dvipsnames,svgnames}{xcolor}
\PassOptionsToPackage{square,numbers}{natbib}
\documentclass{scrreprt}

\usepackage[left=2cm,right=2cm]{geometry}
\usepackage[svgnames]{xcolor}
\usepackage{peeters_layout}

\usepackage{natbib}

\usepackage[
colorlinks=true,
bookmarks=false,
pdfauthor={\authorname, \email},
backref 
]{hyperref}

% http://tex.stackexchange.com/questions/75773/how-to-reference-problems-by-the-text-label-in-an-exercise-envioronment
\usepackage[english]{cleveref}
\crefname{Exercise}{exercise}{exercises}
\Crefname{Exercise}{Exercise}{Exercises}

\RequirePackage{titlesec}
\RequirePackage{ifthen}

% http://stackoverflow.com/questions/4932910/date-in-the-tabular-environment
\makeatletter
\let\insertdate\@date
\makeatother

\titleformat{\chapter}[display]
{\bfseries\Large}
{\color{DarkSlateGrey}\filleft \authorname
\ifthenelse{\isundefined{\studentnumber}}{}{\\ \studentnumber}
\ifthenelse{\isundefined{\email}}{}{\\ \email}
\ifthenelse{\isundefined{\dateintitle}}{}{\\ \insertdate}
%\ifthenelse{\isundefined{\coursename}}{}{\\ \coursename} % put in title instead.
}
{4ex}
{\color{DarkOliveGreen}{\titlerule}\color{Maroon}
\vspace{2ex}%
\filright}
[\vspace{2ex}%
\color{DarkOliveGreen}\titlerule
]

\newcommand{\beginArtWithToc}[0]{\begin{document}\tableofcontents}
\newcommand{\beginArtNoToc}[0]{\begin{document}}
\newcommand{\EndNoBibArticle}[0]{\end{document}}
\newcommand{\EndArticle}[0]{\bibliography{Bibliography}\bibliographystyle{plainnat}\end{document}}

% 
%\newcommand{\citep}[1]{\cite{#1}}

\colorSectionsForArticle



\usepackage{peeters_layout_exercise}
\usepackage{peeters_braket}
\usepackage{peeters_figures}
\usepackage{siunitx}
\usepackage{txfonts} % \ointclockwise

\beginArtNoToc

\generatetitle{Vector Area}
%\chapter{Vector Area}

One of the results of this problem is required for a later one on magnetic moments that I'd like to do.

\makeoproblem{Vector Area.}{problem:vectorAreaGriffiths:1}{\citep{griffiths1999introduction} pr. 1.61}{

The integral 

\begin{dmath}\label{eqn:vectorAreaGriffiths:20}
\Ba = \int_S d\Ba,
\end{dmath}

is sometimes called the vector area of the surface \( S \).

\makesubproblem{}{problem:vectorAreaGriffiths:1:a}

Find the vector area of a hemispherical bowl of radius \( R \).
\makesubproblem{}{problem:vectorAreaGriffiths:1:b}

Show that \( \Ba = 0 \) for any closed surface.
\makesubproblem{}{problem:vectorAreaGriffiths:1:c}
Show that \( \Ba \) is the same for all surfaces sharing the same boundary.

\makesubproblem{}{problem:vectorAreaGriffiths:1:d}

Show that 
\begin{dmath}\label{eqn:vectorAreaGriffiths:40}
\Ba = \inv{2} \ointctrclockwise \Br \cross d\Bl,
\end{dmath}

where the integral is around the boundary line.

\makesubproblem{}{problem:vectorAreaGriffiths:1:e}

Show that 
\begin{dmath}\label{eqn:vectorAreaGriffiths:60}
\ointctrclockwise \lr{ \Bc \cdot \Br } d\Bl = \Ba \cross \Bc.
\end{dmath}
} % problem

\makeanswer{problem:vectorAreaGriffiths:1}{
\makeSubAnswer{}{problem:vectorAreaGriffiths:1:a}

\begin{dmath}\label{eqn:vectorAreaGriffiths:80}
\Ba
=
\int_{0}^{\pi/2} R^2 \sin\theta d\theta \int_0^{2\pi} d\phi
\lr{ \sin\theta \cos\phi, \sin\theta \sin\phi, \cos\theta }
=
R^2 \int_{0}^{\pi/2} d\theta \int_0^{2\pi} d\phi
\lr{ \sin^2\theta \cos\phi, \sin^2\theta \sin\phi, \sin\theta\cos\theta }
=
2 \pi R^2 \int_{0}^{\pi/2} d\theta \Be_3 
\sin\theta\cos\theta 
=
\pi R^2 
\Be_3 
\int_{0}^{\pi/2} d\theta 
\sin(2 \theta)
=
\pi R^2 
\Be_3 
\evalrange{\lr{\frac{-\cos(2 \theta)}{2}}}{0}{\pi/2}
=
\pi R^2 
\Be_3 
\lr{ 1 - (-1) }/2
=
\pi R^2 
\Be_3.
\end{dmath}

\makeSubAnswer{}{problem:vectorAreaGriffiths:1:b}

As hinted in the original problem description, this follows from

\begin{dmath}\label{eqn:vectorAreaGriffiths:n}
\int dV \spacegrad T = \oint T d\Ba,
\end{dmath}

simply by setting \( T = 1 \).

\makeSubAnswer{}{problem:vectorAreaGriffiths:1:c}

The 

\makeSubAnswer{}{problem:vectorAreaGriffiths:1:d}
\makeSubAnswer{}{problem:vectorAreaGriffiths:1:e}
} % answer

%}
\EndArticle

      %
% Copyright � 2016 Peeter Joot.  All Rights Reserved.
% Licenced as described in the file LICENSE under the root directory of this GIT repository.
%
\makeproblem{Tangential magnetic field boundary conditions.}{emt:problemSet3:1}{

\index{boundary conditions!tangential magnetic field}
In the class notes we showed that when there were no sources at the interface between two
media and neither of the two media was a perfect conductor \( \sigma_1, \sigma_2 \ne \infty \) the boundary condition
on the tangential magnetic field was given by

\begin{dmath}\label{eqn:emtProblemSet3Problem1:20}
\ncap \cross \lr{ \BH_2 - \BH_1 } = 0.
\end{dmath}

Here, show that when \( \BJ_i + \BJ_c = \BJ_{ic} \ne 0 \), the boundary condition is given by

\begin{dmath}\label{eqn:emtProblemSet3Problem1:40}
\ncap \cross \lr{ \BH_2 - \BH_1 } = \BJ_s,
\end{dmath}

where
\begin{dmath}\label{eqn:emtProblemSet3Problem1:60}
\BJ_s = \lim_{\Delta y \rightarrow 0} \BJ_{ic} \Delta y.
\end{dmath}

Note: Use the geometry provided in
\cref{fig:boundaryPs3:boundaryPs3Fig1}
for your proof.
\imageFigure{../../figures/ece1228-emt/boundaryPs3Fig1}{Boundary geometry.}{fig:boundaryPs3:boundaryPs3Fig1}{0.3}
} % makeproblem

\makeanswer{emt:problemSet3:1}{

Instead of integrating over a loop as done in class, a better way to tackle this problem is to integrate the curl over the same sort of pillbox that we use for deriving the boundary conditions from the divergence Maxwell's equations.

\index{Stokes' theorem}
The form of Stokes' theorem that we want, following the notation of \citep{aMacdonaldVAGC}, is

\begin{dmath}\label{eqn:emtProblemSet3Problem1:80}
\int_V d^3 \Bx \cdot \lr{ \boldpartial \wedge \BA } = \oint_{\partial V} d^2 \Bx \cdot \BA.
\end{dmath}

The \R{3} translation of this relation into traditional vector algebra, after applying some duality relations, is

\begin{dmath}\label{eqn:emtProblemSet3Problem1:100}
\int_V dV \spacegrad \cross \BA = \oint_{\partial V} dA \ncap \cross \BA,
\end{dmath}

where \( \ncap \) is the outwards normal.  Proving the general multivector Stokes relationship is beyond the scope of this problem, but we can validate
\cref{eqn:emtProblemSet3Problem1:100} by integrating the LHS over the infinitesimal rectangular prism sketched in \cref{fig:ps3Problem1ElementalVolume:ps3Problem1ElementalVolumeFig1}.

\imageFigure{../../figures/ece1228-emt/ps3Problem1ElementalVolumeFig1}{Elemental volume.}{fig:ps3Problem1ElementalVolume:ps3Problem1ElementalVolumeFig1}{0.2}

\begin{dmath}\label{eqn:emtProblemSet3Problem1:120}
\begin{aligned}
\oint_{\partial V} dA \ncap \cross \BA
&=
\oint_{\partial V} dx dy \Be_3 \cross \lr{ \BA(z_0 + \Delta z) - \BA(z_0) } \\
&+\oint_{\partial V} dy dz \Be_1 \cross \lr{ \BA(x_0 + \Delta x) - \BA(x_0) } \\
&+\oint_{\partial V} dz dx \Be_2 \cross \lr{ \BA(y_0 + \Delta y) - \BA(y_0) } \\
&=
\int_{V} dx dy \Be_3 \cross \lr{ dz \PD{z}{\BA} }
+\int_{V} dy dz \Be_1 \cross \lr{ dx \PD{x}{\BA} }
+\int_{V} dz dx \Be_2 \cross \lr{ dy \PD{y}{\BA} } \\
&=
\int_{V} dx dy dz \spacegrad \cross \BA.
\end{aligned}
\end{dmath}

Now, let's apply this to Ampere-Maxwell equation

\begin{dmath}\label{eqn:emtProblemSet3Problem1:140}
\spacegrad \cross \BH = \BJ_{ic} + \PD{t}{\BD},
\end{dmath}

where \( \BJ_{ic} = \BJ_s \delta(y) \).  We have

\begin{dmath}\label{eqn:emtProblemSet3Problem1:160}
\oint dA \ncap \cross \BH = \int dV \lr{ \BJ_s \delta(y) + \PD{t}{\BD} }.
\end{dmath}

This integral will be evaluated using the pillbox configuration of \cref{fig:ps3Problem1Pillbox:ps3Problem1PillboxFig1}.

\imageFigure{../../figures/ece1228-emt/ps3Problem1PillboxFig1}{Pillbox integration volume.}{fig:ps3Problem1Pillbox:ps3Problem1PillboxFig1}{0.2}

The delta function picks up only the contribution of \( \int dA \BJ_s(y=0) \), but \( \BJ_s \) only has a value on that surface anyways.  Taking the pillbox volume to zero in the \( \Delta y \rightarrow 0 \) limit, the LHS integral has only contributions from the top and bottom faces of the pillbox, and the \( \BD \) term, which is assumed finite, will get killed.  That leaves

\begin{dmath}\label{eqn:emtProblemSet3Problem1:180}
\int dA \ncap \cross \lr{ \BH_2 - \BH_1 } = \int dA \BJ_s
\end{dmath}

Both sets of integrands can now be brought under one integral
\begin{dmath}\label{eqn:emtProblemSet3Problem1:200}
\int dA \lr{ \ncap \cross \lr{ \BH_2 - \BH_1 } - \BJ_s } = 0.
\end{dmath}

This is valid for any pillbox surface, so the integrand must be zero,
which proves the desired boundary relation

%\begin{dmath}\label{eqn:emtProblemSet3Problem1:220}
\boxedEquation{eqn:emtProblemSet3Problem1:240}{
\ncap \cross \lr{ \BH_2 - \BH_1 } - \BJ_s = 0.
}
%\end{dmath}

Except for having arbitrarily picked the y-axis as the normal direction in the delta function representation of \( \BJ_{ic} \),
this derivation has the advantage of being coordinate free.  This is in contrast to
the procedure of \citep{balanis1989advanced} followed in class where multiple loop orientations across the boundary are required to prove the general result.
}

      \input{Set3Problem2.tex}
      \input{Set3Problem3.tex}
      \input{Set3Problem4.tex}
      %\section{Appendix I.  Current loop integral off axis.}

Initially I was curious what the current loop magnetic field integral would look like in general, allowing for an off axis observation point.

I found it natural to do that compuation using Geometric Algebra to express vector rotation in a plane and the other geometrical constructs of this problem.  The basic rules in that Algebra are that unit vectors square to unity (\(\Be_k^2 = 1 \)), and that orthogonal vectors anticommute (\( \Be_1 \Be_2 = -\Be_2 \Be_1 \)).  For example, letting \( i = \Be_1 \Be_2 \) the radial unit vector can be expessed as

\begin{dmath}\label{eqn:emtProblemSet3Problem2:160}
\rhocap(\theta)
=
\Be_1 e^{i \theta}
= \Be_1 \lr{ \cos\theta + \Be_1 \Be_2 \sin\theta } 
= \Be_1 \cos\theta + (\Be_1^2) \Be_2 \sin\theta
= \Be_1 \cos\theta + \Be_2 \sin\theta,
\end{dmath}

and the \( \thetacap \) direction vector is
\begin{dmath}\label{eqn:emtProblemSet3Appendix:n}
\thetacap(\theta)
=
\Be_2 e^{i \theta}
= \Be_2 \lr{ \cos\theta + \Be_1 \Be_2 \sin\theta } 
= \Be_2 \cos\theta + \Be_2 \Be_1 \Be_2 \sin\theta
= \Be_2 \cos\theta + \Be_2 (-\Be_2 \Be_1) \sin\theta
= \Be_2 \cos\theta - \Be_1 \sin\theta.
\end{dmath}

This allows for a compact expression of an off-axis observation point

\begin{dmath}\label{eqn:emtProblemSet3Problem2:60}
\Br = z \Be_3 + \rho \Be_1 e^{i\theta}.
\end{dmath}

Similarly, the charge point is
\begin{dmath}\label{eqn:emtProblemSet3Problem2:80}
\Br' = R_l \Be_1 e^{i \theta'},
\end{dmath}

and the element of the loop path is
\begin{dmath}\label{eqn:emtProblemSet3Problem2:100}
d\Bl' = R_l \Be_2 e^{i\theta'} d\theta'.
\end{dmath}

The difference vector from the charge position to the observation point is

\begin{dmath}\label{eqn:emtProblemSet3Problem2:120}
\BR 
= \Br - \Br'
=
z \Be_3 + \rho \Be_1 e^{i\theta}
-
R_l \Be_1 e^{i \theta'},
\end{dmath}

with squared length

\begin{dmath}\label{eqn:emtProblemSet3Problem2:140}
\BR^2 
=
z^2 + 
\lr{ \rho \Be_1 e^{i\theta}
-
R_l \Be_1 e^{i \theta'}
}
\cdot
\lr{ \rho \Be_1 e^{i\theta}
-
R_l \Be_1 e^{i \theta'}
}
=
z^2 + \rho^2 + R_l^2 - 2 \rho R_l \cos\lr{ \theta - \theta' }.
\end{dmath}

For the cross product, using a bivector duality transformation \( \Ba \cross \Bb = -\Be_1 \Be_2 \Be_3 (\Ba \wedge \Bb) \), and expressing the wedge product as a grade two selection, we have

\begin{dmath}\label{eqn:emtProblemSet3Problem2:180}
d\Bl' \cross \BR 
=
-\Be_1 \Be_2 \Be_3 R_l d\theta' \gpgradetwo{ 
\Be_2 e^{i \theta'} 
\lr{
z \Be_3 + \rho \Be_1 e^{i\theta}
-
R_l \Be_1 e^{i \theta'}
}
}
=
R_l d\theta' \lr{ 
z \Be_1 e^{i\theta'}
-
\Be_3 \rho \cos\lr{ \theta - \theta' }
+ \Be_3 R_l
}.
\end{dmath}

The final integral can now be assembled

\boxedEquation{eqn:emtProblemSet3Appendix:220}{
%\begin{dmath}\label{eqn:emtProblemSet3Problem2:200}
\BB = \frac{I \mu_0 R_l}{4\pi} \int_0^{2\pi} d\theta' 
\frac
{ z \Be_1 e^{i\theta'} - \Be_3 \rho \cos\lr{ \theta - \theta' } + \Be_3 R_l }
{ \lr{z^2 + \rho^2 + R_l^2 - 2 \rho R_l \cos\lr{ \theta - \theta' }}^{3/2} }.
%\end{dmath}
}

This is consistent with the traditional vector algebra derivation that led to \cref{eqn:emtProblemSet3Problem2:201} where \( \rho = 0 \) was assumed.
It is clear now, why the problem statement asked only to consider the z-axis observation points where \( \rho = 0 \).  With \( \theta' \) dependencies in the denominator, performing the integral above for \( \rho \ne 0 \) looks spectacularly unpleasant.

\section{Appendix II.  Normal and tangential decomposition.}

The decomposition of \cref{eqn:emtProblemSet3Problem3:60} can be derived easily using Geometric Algebra

\begin{dmath}\label{eqn:emtProblemSet3Problem3:80}
\BA 
= 
\ncap^2 \BA
=
\ncap (\ncap \cdot \BA)
+\ncap (\ncap \wedge \BA)
%=
%\ncap (\ncap \cdot \BA)
%+
%\ncap \cdot (\ncap \wedge \BA)
\end{dmath}

The last dot product can be expanded as a grade one (vector) selection

\begin{dmath}\label{eqn:emtProblemSet3Problem3:100}
\ncap (\ncap \wedge \BA)
=
\gpgradeone{
\ncap (\ncap \wedge \BA)
}
=
\gpgradeone{
\ncap I (\ncap \cross \BA)
}
=
I^2 \ncap \cross (\ncap \cross \BA)
=
- \ncap \cross (\ncap \cross \BA),
\end{dmath}

so the decomposition of a vector \( \BA \) in terms of its normal and tangential projections is
\begin{dmath}\label{eqn:emtProblemSet3Problem3:120}
\BA
=
\ncap (\ncap \cdot \BA)
-
\ncap \cross (\ncap \cross \BA).
\end{dmath}

I'm not sure how to derive this easily using traditional vector algebra, but it can be verified by expanding the triple cross product in coordinates using tensor contraction formalism

\begin{dmath}\label{eqn:emtProblemSet3Problem3:140}
-\ncap \cross (\ncap \cross \BA)
=
-\epsilon_{xyz} \Be_x n_y \lr{\ncap \cross \BA}_z
=
-\epsilon_{xyz} \Be_x n_y \epsilon_{zrs} n_r A_s
=
-\delta_{xy}^{[rs]}
\Be_x n_y n_r A_s
=
-\Be_x n_y \lr{ n_x A_y -n_y A_x }
= -\ncap (\ncap \cdot \BA) + (\ncap \cdot \ncap) \BA
= \BA - \ncap (\ncap \cdot \BA).
\end{dmath}

This last statement illustrates the geometry of this decomposition, showing that the tangential projection (or normal rejection) of a vector is really just the vector minus its normal projection.

%This can be rearranged to show that the 
%\begin{dmath}\label{eqn:emtProblemSet3Problem3:100}

%The tangential projection, can also be expanded in dot products
%
%\begin{dmath}\label{eqn:emtProblemSet3Problem3:200}
%\ncap (\ncap \wedge \BA)
%=
%\ncap \cdot (\ncap \wedge \BA)
%=
%\BA - \ncap (\ncap \cdot \BA)
%\end{dmath}

      \input{Set4Problem4.tex}
      \input{magneticFieldFromMoment.tex}

   %
% Copyright � 2016 Peeter Joot.  All Rights Reserved.
% Licenced as described in the file LICENSE under the root directory of this GIT repository.
%
%\newcommand{\authorname}{Peeter Joot}
\newcommand{\email}{peeterjoot@protonmail.com}
\newcommand{\basename}{FIXMEbasenameUndefined}
\newcommand{\dirname}{notes/FIXMEdirnameUndefined/}

%\renewcommand{\basename}{emt5}
%\renewcommand{\dirname}{notes/ece1228/}
%\newcommand{\keywords}{ECE1228H}
%\newcommand{\authorname}{Peeter Joot}
\newcommand{\onlineurl}{http://sites.google.com/site/peeterjoot2/math2013/\basename.pdf}
\newcommand{\sourcepath}{\dirname\basename.tex}
\newcommand{\generatetitle}[1]{\chapter{#1}}

\newcommand{\vcsinfo}{%
\section*{}
\noindent{\color{DarkOliveGreen}{\rule{\linewidth}{0.1mm}}}
\paragraph{Document version}
%\paragraph{\color{Maroon}{Document version}}
{
\small
\begin{itemize}
\item Available online at:\\ 
\href{\onlineurl}{\onlineurl}
\item Git Repository: \input{./.revinfo/gitRepo.tex}
\item Source: \sourcepath
\item last commit: \input{./.revinfo/gitCommitString.tex}
\item commit date: \input{./.revinfo/gitCommitDate.tex}
\end{itemize}
}
}

%\PassOptionsToPackage{dvipsnames,svgnames}{xcolor}
\PassOptionsToPackage{square,numbers}{natbib}
\documentclass{scrreprt}

\usepackage[left=2cm,right=2cm]{geometry}
\usepackage[svgnames]{xcolor}
\usepackage{peeters_layout}

\usepackage{natbib}

\usepackage[
colorlinks=true,
bookmarks=false,
pdfauthor={\authorname, \email},
backref 
]{hyperref}

% http://tex.stackexchange.com/questions/75773/how-to-reference-problems-by-the-text-label-in-an-exercise-envioronment
\usepackage[english]{cleveref}
\crefname{Exercise}{exercise}{exercises}
\Crefname{Exercise}{Exercise}{Exercises}

\RequirePackage{titlesec}
\RequirePackage{ifthen}

% http://stackoverflow.com/questions/4932910/date-in-the-tabular-environment
\makeatletter
\let\insertdate\@date
\makeatother

\titleformat{\chapter}[display]
{\bfseries\Large}
{\color{DarkSlateGrey}\filleft \authorname
\ifthenelse{\isundefined{\studentnumber}}{}{\\ \studentnumber}
\ifthenelse{\isundefined{\email}}{}{\\ \email}
\ifthenelse{\isundefined{\dateintitle}}{}{\\ \insertdate}
%\ifthenelse{\isundefined{\coursename}}{}{\\ \coursename} % put in title instead.
}
{4ex}
{\color{DarkOliveGreen}{\titlerule}\color{Maroon}
\vspace{2ex}%
\filright}
[\vspace{2ex}%
\color{DarkOliveGreen}\titlerule
]

\newcommand{\beginArtWithToc}[0]{\begin{document}\tableofcontents}
\newcommand{\beginArtNoToc}[0]{\begin{document}}
\newcommand{\EndNoBibArticle}[0]{\end{document}}
\newcommand{\EndArticle}[0]{\bibliography{Bibliography}\bibliographystyle{plainnat}\end{document}}

% 
%\newcommand{\citep}[1]{\cite{#1}}

\colorSectionsForArticle


%
%%\usepackage{ece1228}
%\usepackage{peeters_braket}
%%\usepackage{peeters_layout_exercise}
%\usepackage{peeters_figures}
%\usepackage{macros_cal}
%\usepackage{macros_bm}
%\usepackage{mathtools}
%\usepackage{siunitx}
%
%\beginArtNoToc
%\generatetitle{ECE1228H Electromagnetic Theory.  Lecture 5: Poynting vector.  Taught by Prof.\ M. Mojahedi}
\chapter{Poynting vector, and time harmonic (phasor) fields.}
\label{chap:emt5}

\paragraph{Poynting}

The cross product terms of Maxwell's equation are
\begin{equation}\label{eqn:emtLecture5:120}
\spacegrad \cross \BE 
= -\BM_i - \PD{t}{\BB}
= -\BM_i - \BM_d,
\end{equation}

where \(\BM_d\) is called the magnetic displacement current here.  For the magnetic curl we have

\begin{equation}\label{eqn:emtLecture5:140}
\spacegrad \cross \BH 
= \BJ_i + \BJ_c + \PD{t}{\BD}
= \BJ_i + \BJ_c + \BJ_d.
\end{equation}

From this (HW) we will show that 
\begin{dmath}\label{eqn:emtLecture5:160}
\spacegrad \cdot \lr{ \BE \cross \BH } + \BH \cdot \lr{ \BM_i + \BM_d }  + \BE \cdot \lr{ \BJ_i + \BJ_c + \BJ_d } = 0,
\end{dmath}

or
\begin{dmath}\label{eqn:emtLecture5:180}
\oint d\Ba \cdot \lr{ \BE \cross \BH } + \int dV \lr{ \BH \cdot \lr{ \BM_i + \BM_d }  + \BE \cdot \lr{ \BJ_i + \BJ_c + \BJ_d }} = 0,
\end{dmath}

or
\begin{dmath}\label{eqn:emtLecture5:200}
\oint d\Ba \cdot \lr{ \BE \cross \BH } 
+ \int dV \BH \cdot \BM_i
+ \int dV \BE \cdot \BJ_i
+ \int dV \BE \cdot \BJ_c
+ \int dV \lr{ \BH \cdot \PD{t}{\BB} + \BE \cdot \PD{t}{\BD} } = 0.
\end{dmath}

Define a supplied power density \( \rho_{\textrm{supp}} \)

\begin{dmath}\label{eqn:emtLecture5:220}
-\rho_{\textrm{supp}}
=
 \int dV \BH \cdot \BM_i
+ \int dV \BE \cdot \BJ_i.
\end{dmath}

When the medium is not dispersive or lossy, we have

\begin{dmath}\label{eqn:emtLecture5:240}
\int dV \BH \cdot \PD{t}{\BB} 
=
\mu \int dV \BH \cdot \PD{t}{\BH} 
=
\PD{t}{} \int dV \mu \Abs{\BH}^2.
\end{dmath}

The units of \( [\mu \Abs{\BH}^2] \) are \si{W}, so one can defined a magnetic energy density \( \mu \Abs{\BH}^2 \), and

\begin{dmath}\label{eqn:emtLecture5:260}
W_m = 
\int dV \mu \Abs{\BH}^2,
\end{dmath}

for

\begin{dmath}\label{eqn:emtLecture5:280}
\int dV \BH \cdot \PD{t}{\BB} 
=
\PD{t}{W_m}.
\end{dmath}

This is the rate of change of stored magnetic energy [\si{J/s} = \si{W}].

Similarly
\begin{dmath}\label{eqn:emtLecture5:300}
\int dV \BE \cdot \PD{t}{\BD} 
=
\epsilon
\int dV \BE \cdot \PD{t}{\BE} 
=
\PD{t}{} \int dV \epsilon \Abs{\BE}^2.
\end{dmath}

The electric energy density is \( \epsilon \Abs{\BE}^2 \).  Let

\begin{dmath}\label{eqn:emtLecture5:320}
W_e = 
\int dV \epsilon \Abs{\BE}^2,
\end{dmath}

and
\begin{dmath}\label{eqn:emtLecture5:340}
\int dV \BE \cdot \PD{t}{\BD} 
=
\PD{t}{W_e}.
\end{dmath}

We also have a term

\begin{dmath}\label{eqn:emtLecture5:360}
\int dV \BE \cdot \BJ_c 
=
\int dV \BE \cdot (\sigma \BE)
=
\int dV \sigma \Abs{\BE}^2
%\equiv ...
\end{dmath}

This is the rate of change of stored electric energy.

The remaining term is
\begin{dmath}\label{eqn:emtLecture5:380}
\oint d\Ba \cdot \lr{ \BE \cross \BH }
\end{dmath}

This is a density of the power that is leaving the volume.  The vector \( \BE \cross \BH \) is special, called the Poynting vector, and coincidentally points in the direction that the energy leaves the bounding surface per unit time.  We write

\begin{dmath}\label{eqn:emtLecture5:400}
\BS = \BE \cross \BH.
\end{dmath}

In vacuum the phase velocity \( \Bv_p \), group velocity \( \Bv_g \) and packet(?) velocity \( \Bv_p \) all line up.  This isn't the case in the media.

It turns out that without dissipation 

\begin{dmath}\label{eqn:emtLecture5:420}
\int \BH \cdot \PD{t}{\BB} = \int \BE \cdot \PD{t}{\BD}.
\end{dmath}

For example in an LC circuit \cref{fig:lecture4LCCircuit:lecture4LCCircuitFig1}
half the cycle the energy is stored in the inductor, and in the other half of the cycle the energy is stored in the capacitor.

\imageFigure{../../figures/ece1228-emt/lecture4LCCircuitFig1}{LC circuit.}{fig:lecture4LCCircuit:lecture4LCCircuitFig1}{0.2}

Summarizing

\begin{dmath}\label{eqn:emtLecture5:440}
\oint \lr{ \BE \cross \BH } \cdot d\Ba = P_{\textrm{exit}}.
\end{dmath}

\paragraph{Time harmonics}

Recall that we have differential equations to solve for each type of circuit element in the time domain.  For example in \cref{fig:lecture4inductor:lecture4inductorFig2a}, we have

\begin{dmath}\label{eqn:emtLecture5:980}
V_i(t) = L \ddt{i},
\end{dmath}

\imageFigure{../../figures/ece1228-emt/lecture4inductorFig2a}{Inductor.}{fig:lecture4inductor:lecture4inductorFig2a}{0.2}

and for the capacitor sketched in \cref{fig:lecture4cap:lecture4capFig2b}, we have
\begin{dmath}\label{eqn:emtLecture5:1000}
i_c(t) = C \ddt{V_c}.
\end{dmath}

\imageFigure{../../figures/ece1228-emt/lecture4capFig2b}{Capacitor.}{fig:lecture4cap:lecture4capFig2b}{0.2}

When we use Laplace or Fourier techniques to solve circuits with such differential equation elements.  The price that we paid for that was that we have to start dealing with complex-valued (phasor) quantities.  We can do this for field equations as well.  The goal is to remove the time domain coupling in Maxwell equations like

\begin{dmath}\label{eqn:emtLecture5:460}
\spacegrad \cross \BE(\Br, t) = -\PD{t}{\BB}(\Br, t)
\end{dmath}
\begin{dmath}\label{eqn:emtLecture5:480}
\spacegrad \cross \BH(\Br, t) = \sigma \BE + \PD{t}{\BD}(\Br, t).
\end{dmath}

For a single frequency, assume that the time dependency can be written as

\begin{dmath}\label{eqn:emtLecture5:500}
\BE(\Br, t) = \Real \lr{ \BE^\conj(\Br) e^{j \omega t} }.
\end{dmath}

We may now have to require \( \BE(\Br) \) to be complex valued.
We also have to be really careful about which convention of the time domain solution we are going to use, since we could just as easily use

\begin{dmath}\label{eqn:emtLecture5:720}
\BE(\Br, t) = \Real \lr{ \BE(\Br) e^{-j \omega t} }.
\end{dmath}

For example 
\begin{dmath}\label{eqn:emtLecture5:840}
\Real( e^{i k z} e^{-i\omega t} ) = \cos( k z - \omega t ),
\end{dmath}

is identical with
\begin{dmath}\label{eqn:emtLecture5:860}
\Real( e^{-j k z} e^{j\omega t} ) = \cos( \omega t -k z),
\end{dmath}

showing that a solution or its complex conjugate is equally valid.

Engineering books use \( e^{j \omega t} \) whereas most physicists use \( e^{-i \omega t } \).

What if we have more complex time dependencies, such as that sketched in \cref{fig:lecture4NonSine:lecture4NonSineFig3}?

\imageFigure{../../figures/ece1228-emt/lecture4NonSineFig3}{Non-sinusoidal time dependence.}{fig:lecture4NonSine:lecture4NonSineFig3}{0.2}

We can do this using Fourier superposition, adding a finite or infinite set of single frequency solutions.  The first order of business is to solve the system for a single frequency.

Let's write our Fourier transform pairs as
\begin{subequations}
\label{eqn:emtLecture5:520}
\begin{equation}\label{eqn:emtLecture5:540}
\calF(\BA(\Br, t)) = 
\BA(\Br, \omega)
=
\int_{-\infty}^\infty \BA(\Br, t) e^{-j \omega t} dt
\end{equation}
\begin{equation}\label{eqn:emtLecture5:560}
\BA(\Br, t) = \calF^{-1}(\BA(\Br, \omega))
=
\inv{2\pi} 
\int_{-\infty}^\infty \BA(\Br, \omega) e^{j \omega t} d\omega.
\end{equation}
\end{subequations}

In particular

\begin{equation}\label{eqn:emtLecture5:580}
\calF\lr{ \ddt{f(t)} } = j \omega F(\omega),
\end{equation}

so the Fourier transform of the Maxwell equation
\begin{dmath}\label{eqn:emtLecture5:600}
\calF\lr{ \spacegrad \cross \BE(\Br, t) }
=
\calF\lr{ -\PD{t}{\BB}(\Br, t) },
\end{dmath}

is

\begin{dmath}\label{eqn:emtLecture5:620}
\spacegrad \cross \BE(\Br, \omega) = - j\omega \BB(\Br, \omega).
\end{dmath}

The four Maxwell's equations can be written as

\begin{itemize}
\item Faraday's Law
\begin{dmath}\label{eqn:emtLecture5:640}
\spacegrad \cross \BE( \Br, \omega ) = - j \omega \BB(\Br, \omega) - \BM_i
\end{dmath}
\item Ampere-Maxwell equation
\begin{dmath}\label{eqn:emtLecture5:660}
\spacegrad \cross \BH( \Br, \omega ) = \BJ_\txtc(\Br, \omega) + \BD(\Br, \omega)
\end{dmath}
\item Gauss's law
\begin{dmath}\label{eqn:emtLecture5:680}
\spacegrad \cdot \BD(\Br, \omega) = \rho_{\txte\txtv}(\Br, \omega)
\end{dmath}
\item Gauss's law for magnetism
\begin{dmath}\label{eqn:emtLecture5:700}
\spacegrad \cdot \BB(\Br, \omega) = \rho_{\txtm\txtv}(\Br, \omega).
\end{dmath}
\end{itemize}

Now we can more easily model non-simple media with

\begin{dmath}\label{eqn:emtLecture5:740}
\begin{aligned}
\BB(\Br, \omega) &= \mu(\omega) \BH(\Br, \omega) \\
\BD(\Br, \omega) &= \epsilon(\omega) \BE(\Br, \omega).
\end{aligned}
\end{dmath}

so Maxwell's equations are

\begin{dmath}\label{eqn:emtLecture5:760}
\spacegrad \cross \BE( \Br, \omega ) = - j \omega \mu(\omega) \BH(\Br, \omega) - \BM_i
\end{dmath}
\begin{dmath}\label{eqn:emtLecture5:780}
\spacegrad \cross \BH( \Br, \omega ) = \BJ_\txtc(\Br, \omega) + \epsilon(\omega) \BE(\Br, \omega)
\end{dmath}
\begin{dmath}\label{eqn:emtLecture5:800}
\epsilon(\omega) \spacegrad \cdot \BE(\Br, \omega) = \rho_{\txte\txtv}(\Br, \omega)
\end{dmath}
\begin{dmath}\label{eqn:emtLecture5:820}
\mu(\omega) \spacegrad \cdot \BH(\Br, \omega) = \rho_{\txtm\txtv}(\Br, \omega).
\end{dmath}

\paragraph{Frequency domain Poynting}

The frequency domain (time harmonic) equivalent of the instantaneous Poynting theorem is

\begin{dmath}\label{eqn:emtLecture5:880}
\inv{2} \oint d\Ba \cdot \lr{ \BE \cross \BH^\conj } 
- \inv{2} \int dV \lr{ \BH^\conj \cdot \BM_i + \BE \cdot \BJ_i^\conj }
+ \inv{2} \int dV \sigma \Abs{\BE}^2
+ j \omega \inv{2} \int dV \lr{ \mu \Abs{\BH}^2 - \epsilon \Abs{\BE}^2 } = 0.
\end{dmath}

Showing this will probably be given as homework.

Since

\begin{dmath}\label{eqn:emtLecture5:900}
\Real(\BA) \cross \Real(\BB) \ne \Real( \BA \cross \BB ).
\end{dmath}

We want to find the instantaneous Poynting vector in terms of the phasor fields.  Following
\citep{balanis1989advanced}, where script is used for the instantaneous quantities and non-script for the phasors, we find

\begin{dmath}\label{eqn:emtLecture5:920}
\bcS(\Br, t) 
= \bcE(\Br, t) \cross \bcH(\Br, t)
= \Real(\bcE(\Br, t)) \cross \Real(\bcH(\Br, t))
= 
\frac{ \BE e^{j\omega t} + \BE^\conj e^{-j \omega t}}{2}
\cross
\frac{ \BH e^{j\omega t} + \BH^\conj e^{-j \omega t}}{2}
=
\inv{4}
\lr{
\BE \cross \BH^\conj + \BE^\conj \cross \BH
+ 
\BE \cross \BH e^{2 j\omega t} 
+ 
\BH \cross \BE e^{-2 j\omega t} 
}
=
\inv{2} \Real(\BE \cross \BH^\conj) + \inv{2} \Real( \BE \cross \BH  e^{2 j\omega t} ).
\end{dmath}

Should we time average over a period \( \expectation{.} = (1/T) \int_0^T (.) \) the second term is killed, so that

\begin{dmath}\label{eqn:emtLecture5:940}
\expectation{ \bcS }
=
\inv{2} \Real(\BE \cross \BH^\conj) + \inv{2} \Real( \BE \cross \BH  e^{2 j\omega t} ).
\end{dmath}

The instantaneous Poynting vector is thus
\begin{dmath}\label{eqn:emtLecture5:960}
\bcS(\Br, t) = \expectation{\BS} + \inv{2} \Real\lr{ \BE \cross \BH e^{j \omega t} }.
\end{dmath}

%\EndArticle

      \section{Problems}

      \input{Set4Problem1.tex}
      \input{Set4Problem2.tex}
      %
% Copyright � 2016 Peeter Joot.  All Rights Reserved.
% Licenced as described in the file LICENSE under the root directory of this GIT repository.
%
\makeproblem{Duality theorem.}{emt:problemSet4:3}{
\index{duality theorem}
Prove that if the time-harmonic fields \( \BE(\Br) \) and \( \BH(\Br) \)
are solutions to Maxwell's
equations in a simple, source free medium ( \( \BM_i = \BJ_i = \BJ_c = 0, \rho_{mv} = \rho_{ev} = 0 \) ),
characterized by \( \epsilon, \mu \) ; then
\( \BE'(\Br) = \eta \BH(\Br) \) and
\( \BH'(\Br) = -\frac{\BE(\Br)}{\eta} \)
are also solutions of
the Maxwell equations.
\( \eta \)
is the intrinsic impedance of the medium.
\paragraph{Remark}: By showing the above you have proved the validity of the so called duality
theorem.
} % makeproblem

\makeanswer{emt:problemSet4:3}{

In source free simple media, Maxwell's time-harmonic equations are

\begin{dmath}\label{eqn:emtProblemSet4Problem3:20}
\begin{aligned}
\spacegrad \cross \BH &= j \omega \epsilon \BE \\
\spacegrad \cross \BE &= -j \omega \mu \BH \\
\spacegrad \cdot \BH &= 0 \\
\spacegrad \cdot \BE &= 0,
\end{aligned}
\end{dmath}

Inserting \( \BH = \BE'/\eta, \BE = -\eta \BH' \), these are

\begin{dmath}\label{eqn:emtProblemSet4Problem3:40}
\begin{aligned}
\spacegrad \cross \BE' &= -j \omega \epsilon \eta^2 \BH' \\
\spacegrad \cross \BH' &= j \omega \frac{\mu}{\eta^2} \BE' \\
\spacegrad \cdot \BE' &= 0 \\
\spacegrad \cdot \BH' &= 0,
\end{aligned}
\end{dmath}

We see that \( \BE', \BH' \) are solutions provided

\begin{equation}\label{eqn:emtProblemSet4Problem3:60}
\begin{aligned}
\epsilon &= \frac{\mu}{\eta^2} \\
\mu &= \epsilon \eta^2,
\end{aligned}
\end{equation}

or

\begin{dmath}\label{eqn:emtProblemSet4Problem3:80}
\eta^2 = \frac{\mu}{\epsilon}.
\end{dmath}
}

      %\documentclass{article}

\usepackage{amsmath}
\usepackage{mathpazo}

%
% shorthand for bold symbols, convenient for vectors and matrices
%
\newcommand{\Ba}[0]{\mathbf{a}}
\newcommand{\Bb}[0]{\mathbf{b}}
\newcommand{\Bc}[0]{\mathbf{c}}
\newcommand{\Bd}[0]{\mathbf{d}}
\newcommand{\Be}[0]{\mathbf{e}}
\newcommand{\Bf}[0]{\mathbf{f}}
\newcommand{\Bg}[0]{\mathbf{g}}
\newcommand{\Bh}[0]{\mathbf{h}}
\newcommand{\Bi}[0]{\mathbf{i}}
\newcommand{\Bj}[0]{\mathbf{j}}
\newcommand{\Bk}[0]{\mathbf{k}}
\newcommand{\Bl}[0]{\mathbf{l}}
\newcommand{\Bm}[0]{\mathbf{m}}
\newcommand{\Bn}[0]{\mathbf{n}}
\newcommand{\Bo}[0]{\mathbf{o}}
\newcommand{\Bp}[0]{\mathbf{p}}
\newcommand{\Bq}[0]{\mathbf{q}}
\newcommand{\Br}[0]{\mathbf{r}}
\newcommand{\Bs}[0]{\mathbf{s}}
\newcommand{\Bt}[0]{\mathbf{t}}
\newcommand{\Bu}[0]{\mathbf{u}}
\newcommand{\Bv}[0]{\mathbf{v}}
\newcommand{\Bw}[0]{\mathbf{w}}
\newcommand{\Bx}[0]{\mathbf{x}}
\newcommand{\By}[0]{\mathbf{y}}
\newcommand{\Bz}[0]{\mathbf{z}}
\newcommand{\BA}[0]{\mathbf{A}}
\newcommand{\BB}[0]{\mathbf{B}}
\newcommand{\BC}[0]{\mathbf{C}}
\newcommand{\BD}[0]{\mathbf{D}}
\newcommand{\BE}[0]{\mathbf{E}}
\newcommand{\BF}[0]{\mathbf{F}}
\newcommand{\BG}[0]{\mathbf{G}}
\newcommand{\BH}[0]{\mathbf{H}}
\newcommand{\BI}[0]{\mathbf{I}}
\newcommand{\BJ}[0]{\mathbf{J}}
\newcommand{\BK}[0]{\mathbf{K}}
\newcommand{\BL}[0]{\mathbf{L}}
\newcommand{\BM}[0]{\mathbf{M}}
\newcommand{\BN}[0]{\mathbf{N}}
\newcommand{\BO}[0]{\mathbf{O}}
\newcommand{\BP}[0]{\mathbf{P}}
\newcommand{\BQ}[0]{\mathbf{Q}}
\newcommand{\BR}[0]{\mathbf{R}}
\newcommand{\BS}[0]{\mathbf{S}}
\newcommand{\BT}[0]{\mathbf{T}}
\newcommand{\BU}[0]{\mathbf{U}}
\newcommand{\BV}[0]{\mathbf{V}}
\newcommand{\BW}[0]{\mathbf{W}}
\newcommand{\BX}[0]{\mathbf{X}}
\newcommand{\BY}[0]{\mathbf{Y}}
\newcommand{\BZ}[0]{\mathbf{Z}}

\newcommand{\Bzero}[0]{\mathbf{0}}
\newcommand{\Btheta}[0]{\boldsymbol{\theta}}
\newcommand{\Btau}[0]{\boldsymbol{\tau}}
\newcommand{\Bomega}[0]{\boldsymbol{\omega}}

%
% shorthand for unit vectors
%
\newcommand{\acap}[0]{\hat{\Ba}}
\newcommand{\bcap}[0]{\hat{\Bb}}
\newcommand{\ccap}[0]{\hat{\Bc}}
\newcommand{\dcap}[0]{\hat{\Bd}}
\newcommand{\ecap}[0]{\hat{\Be}}
\newcommand{\fcap}[0]{\hat{\Bf}}
\newcommand{\gcap}[0]{\hat{\Bg}}
\newcommand{\hcap}[0]{\hat{\Bh}}
\newcommand{\icap}[0]{\hat{\Bi}}
\newcommand{\jcap}[0]{\hat{\Bj}}
\newcommand{\kcap}[0]{\hat{\Bk}}
\newcommand{\lcap}[0]{\hat{\Bl}}
\newcommand{\mcap}[0]{\hat{\Bm}}
\newcommand{\ncap}[0]{\hat{\Bn}}
\newcommand{\ocap}[0]{\hat{\Bo}}
\newcommand{\pcap}[0]{\hat{\Bp}}
\newcommand{\qcap}[0]{\hat{\Bq}}
\newcommand{\rcap}[0]{\hat{\Br}}
\newcommand{\scap}[0]{\hat{\Bs}}
\newcommand{\tcap}[0]{\hat{\Bt}}
\newcommand{\ucap}[0]{\hat{\Bu}}
\newcommand{\vcap}[0]{\hat{\Bv}}
\newcommand{\wcap}[0]{\hat{\Bw}}
\newcommand{\xcap}[0]{\hat{\Bx}}
\newcommand{\ycap}[0]{\hat{\By}}
\newcommand{\zcap}[0]{\hat{\Bz}}
\newcommand{\thetacap}[0]{\hat{\Btheta}}

%
% to write R^n and C^n in a distinguishable fashion.  Perhaps change this
% to the double lined characters upon figuring out how to do so.
%
\newcommand{\C}[1]{$\mathbb{C}^{#1}$}
\newcommand{\R}[1]{$\mathbb{R}^{#1}$}

%
% various generally useful helpers
%

% derivative of #1 wrt. #2:
\newcommand{\D}[2] {\frac {d#2} {d#1}}

\newcommand{\inv}[1]{\frac{1}{#1}}
\newcommand{\cross}[0]{\times}

\newcommand{\abs}[1]{\lvert{#1}\rvert}
\newcommand{\norm}[1]{\lVert{#1}\rVert}
\newcommand{\innerprod}[2]{\langle{#1}, {#2}\rangle}
\newcommand{\dotprod}[2]{{#1} \cdot {#2}}
\newcommand{\bdotprod}[2]{\left({#1} \cdot {#2}\right)}
\newcommand{\crossprod}[2]{{#1} \cross {#2}}
\newcommand{\tripleprod}[3]{\dotprod{\left(\crossprod{#1}{#2}\right)}{#3}}

\DeclareMathOperator{\Proj}{Proj}
\DeclareMathOperator{\Span}{span}
\DeclareMathOperator{\Sgn}{sgn}
\DeclareMathOperator{\Area}{Area}
\DeclareMathOperator{\Volume}{Volume}

%
% A few miscellaneous things specific to this document
%
\newcommand{\crossop}[1]{\crossprod{#1}{}}

% R2 vector.
\newcommand{\VectorTwo}[2]{
\begin{bmatrix}
 {#1} \\
 {#2}
\end{bmatrix}
}

\newcommand{\VectorN}[1]{
\begin{bmatrix}
{#1}_1 \\
{#1}_2 \\
\vdots \\
{#1}_N \\
\end{bmatrix}
}

\newcommand{\DETuvij}[4]{
\begin{vmatrix}
 {#1}_{#3} & {#1}_{#4} \\
 {#2}_{#3} & {#2}_{#4}
\end{vmatrix}
}

\newcommand{\DETuvwijk}[6]{
\begin{vmatrix}
 {#1}_{#4} & {#1}_{#5} & {#1}_{#6} \\
 {#2}_{#4} & {#2}_{#5} & {#2}_{#6} \\
 {#3}_{#4} & {#3}_{#5} & {#3}_{#6}
\end{vmatrix}
}

\newcommand{\DETuvwxijkl}[8]{
\begin{vmatrix}
 {#1}_{#5} & {#1}_{#6} & {#1}_{#7} & {#1}_{#8} \\
 {#2}_{#5} & {#2}_{#6} & {#2}_{#7} & {#2}_{#8} \\
 {#3}_{#5} & {#3}_{#6} & {#3}_{#7} & {#3}_{#8} \\
 {#4}_{#5} & {#4}_{#6} & {#4}_{#7} & {#4}_{#8} \\
\end{vmatrix}
}

%\newcommand{\DETuvwxyijklm}[10]{
%\begin{vmatrix}
% {#1}_{#6} & {#1}_{#7} & {#1}_{#8} & {#1}_{#9} & {#1}_{#10} \\
% {#2}_{#6} & {#2}_{#7} & {#2}_{#8} & {#2}_{#9} & {#2}_{#10} \\
% {#3}_{#6} & {#3}_{#7} & {#3}_{#8} & {#3}_{#9} & {#3}_{#10} \\
% {#4}_{#6} & {#4}_{#7} & {#4}_{#8} & {#4}_{#9} & {#4}_{#10} \\
% {#5}_{#6} & {#5}_{#7} & {#5}_{#8} & {#5}_{#9} & {#5}_{#10}
%\end{vmatrix}
%}

% R3 vector.
\newcommand{\VectorThree}[3]{
\begin{bmatrix}
 {#1} \\
 {#2} \\
 {#3}
\end{bmatrix}
}


%<misc>
%
\newcommand{\Abs}[1]{{\left\lvert{#1}\right\rvert}}
\newcommand{\spacegrad}[0]{\boldsymbol{\nabla}}
\newcommand{\grad}[0]{\nabla}
\newcommand{\LL}[0]{\mathcal{L}}

% == \partial_{#1} {#2}
\newcommand{\PD}[2]{\frac{\partial {#2}}{\partial {#1}}}
% inline variant
\newcommand{\PDi}[2]{{\partial {#2}}/{\partial {#1}}}

\newcommand{\PDD}[3]{\frac{\partial^2 {#3}}{\partial {#1}\partial {#2}}}
%\newcommand{\PDd}[2]{\frac{\partial^2 {#2}}{{\partial{#1}}^2}}
\newcommand{\PDsq}[2]{\frac{\partial^2 {#2}}{(\partial {#1})^2}}

\newcommand{\Partial}[2]{\frac{\partial {#1}}{\partial {#2}}}
\DeclareMathOperator{\RejName}{Rej}
\newcommand{\Rej}[2]{\RejName_{#1}\left( {#2} \right)}
\newcommand{\Rm}[1]{\mathbb{R}^{#1}}
\newcommand{\Cm}[1]{\mathbb{C}^{#1}}
\newcommand{\conj}[0]{{*}}

%</misc>

% <grade selection>
%
\newcommand{\gpgrade}[2] {{\left\langle{{#1}}\right\rangle}_{#2}}

\newcommand{\gpgradezero}[1] {\gpgrade{#1}{}}
%\newcommand{\gpscalargrade}[1] {{\left\langle{{#1}}\right\rangle}}
%\newcommand{\gpgradezero}[1] {\gpgrade{#1}{0}}

%\newcommand{\gpgradeone}[1] {{\left\langle{{#1}}\right\rangle}_{1}}
\newcommand{\gpgradeone}[1] {\gpgrade{#1}{1}}

\newcommand{\gpgradetwo}[1] {\gpgrade{#1}{2}}
\newcommand{\gpgradethree}[1] {\gpgrade{#1}{3}}
\newcommand{\gpgradefour}[1] {\gpgrade{#1}{4}}
%
% </grade selection>



\newcommand{\adot}[0]{{\dot{a}}}
\newcommand{\bdot}[0]{{\dot{b}}}
% taken for centered dot:
%\newcommand{\cdot}[0]{{\dot{c}}}
%\newcommand{\ddot}[0]{{\dot{d}}}
\newcommand{\edot}[0]{{\dot{e}}}
\newcommand{\fdot}[0]{{\dot{f}}}
\newcommand{\gdot}[0]{{\dot{g}}}
\newcommand{\hdot}[0]{{\dot{h}}}
\newcommand{\idot}[0]{{\dot{i}}}
\newcommand{\jdot}[0]{{\dot{j}}}
\newcommand{\kdot}[0]{{\dot{k}}}
\newcommand{\ldot}[0]{{\dot{l}}}
\newcommand{\mdot}[0]{{\dot{m}}}
\newcommand{\ndot}[0]{{\dot{n}}}
%\newcommand{\odot}[0]{{\dot{o}}}
\newcommand{\pdot}[0]{{\dot{p}}}
\newcommand{\qdot}[0]{{\dot{q}}}
\newcommand{\rdot}[0]{{\dot{r}}}
\newcommand{\sdot}[0]{{\dot{s}}}
\newcommand{\tdot}[0]{{\dot{t}}}
\newcommand{\udot}[0]{{\dot{u}}}
\newcommand{\vdot}[0]{{\dot{v}}}
\newcommand{\wdot}[0]{{\dot{w}}}
\newcommand{\xdot}[0]{{\dot{x}}}
\newcommand{\ydot}[0]{{\dot{y}}}
\newcommand{\zdot}[0]{{\dot{z}}}
\newcommand{\addot}[0]{{\ddot{a}}}
\newcommand{\bddot}[0]{{\ddot{b}}}
\newcommand{\cddot}[0]{{\ddot{c}}}
%\newcommand{\dddot}[0]{{\ddot{d}}}
\newcommand{\eddot}[0]{{\ddot{e}}}
\newcommand{\fddot}[0]{{\ddot{f}}}
\newcommand{\gddot}[0]{{\ddot{g}}}
\newcommand{\hddot}[0]{{\ddot{h}}}
\newcommand{\iddot}[0]{{\ddot{i}}}
\newcommand{\jddot}[0]{{\ddot{j}}}
\newcommand{\kddot}[0]{{\ddot{k}}}
\newcommand{\lddot}[0]{{\ddot{l}}}
\newcommand{\mddot}[0]{{\ddot{m}}}
\newcommand{\nddot}[0]{{\ddot{n}}}
\newcommand{\oddot}[0]{{\ddot{o}}}
\newcommand{\pddot}[0]{{\ddot{p}}}
\newcommand{\qddot}[0]{{\ddot{q}}}
\newcommand{\rddot}[0]{{\ddot{r}}}
\newcommand{\sddot}[0]{{\ddot{s}}}
\newcommand{\tddot}[0]{{\ddot{t}}}
\newcommand{\uddot}[0]{{\ddot{u}}}
\newcommand{\vddot}[0]{{\ddot{v}}}
\newcommand{\wddot}[0]{{\ddot{w}}}
\newcommand{\xddot}[0]{{\ddot{x}}}
\newcommand{\yddot}[0]{{\ddot{y}}}
\newcommand{\zddot}[0]{{\ddot{z}}}

%<bold and dot greek symbols>
%

\newcommand{\Deltadot}[0]{{\dot{\Delta}}}
\newcommand{\Gammadot}[0]{{\dot{\Gamma}}}
\newcommand{\Lambdadot}[0]{{\dot{\Lambda}}}
\newcommand{\Omegadot}[0]{{\dot{\Omega}}}
\newcommand{\Phidot}[0]{{\dot{\Phi}}}
\newcommand{\Pidot}[0]{{\dot{\Pi}}}
\newcommand{\Psidot}[0]{{\dot{\Psi}}}
\newcommand{\Sigmadot}[0]{{\dot{\Sigma}}}
\newcommand{\Thetadot}[0]{{\dot{\Theta}}}
\newcommand{\Upsilondot}[0]{{\dot{\Upsilon}}}
\newcommand{\Xidot}[0]{{\dot{\Xi}}}
\newcommand{\alphadot}[0]{{\dot{\alpha}}}
\newcommand{\betadot}[0]{{\dot{\beta}}}
\newcommand{\chidot}[0]{{\dot{\chi}}}
\newcommand{\deltadot}[0]{{\dot{\delta}}}
\newcommand{\epsilondot}[0]{{\dot{\epsilon}}}
\newcommand{\etadot}[0]{{\dot{\eta}}}
\newcommand{\gammadot}[0]{{\dot{\gamma}}}
\newcommand{\kappadot}[0]{{\dot{\kappa}}}
\newcommand{\lambdadot}[0]{{\dot{\lambda}}}
\newcommand{\mudot}[0]{{\dot{\mu}}}
\newcommand{\nudot}[0]{{\dot{\nu}}}
\newcommand{\omegadot}[0]{{\dot{\omega}}}
\newcommand{\phidot}[0]{{\dot{\phi}}}
\newcommand{\pidot}[0]{{\dot{\pi}}}
\newcommand{\psidot}[0]{{\dot{\psi}}}
\newcommand{\rhodot}[0]{{\dot{\rho}}}
\newcommand{\sigmadot}[0]{{\dot{\sigma}}}
\newcommand{\taudot}[0]{{\dot{\tau}}}
\newcommand{\thetadot}[0]{{\dot{\theta}}}
\newcommand{\upsilondot}[0]{{\dot{\upsilon}}}
\newcommand{\varepsilondot}[0]{{\dot{\varepsilon}}}
\newcommand{\varphidot}[0]{{\dot{\varphi}}}
\newcommand{\varpidot}[0]{{\dot{\varpi}}}
\newcommand{\varrhodot}[0]{{\dot{\varrho}}}
\newcommand{\varsigmadot}[0]{{\dot{\varsigma}}}
\newcommand{\varthetadot}[0]{{\dot{\vartheta}}}
\newcommand{\xidot}[0]{{\dot{\xi}}}
\newcommand{\zetadot}[0]{{\dot{\zeta}}}

\newcommand{\Deltaddot}[0]{{\ddot{\Delta}}}
\newcommand{\Gammaddot}[0]{{\ddot{\Gamma}}}
\newcommand{\Lambdaddot}[0]{{\ddot{\Lambda}}}
\newcommand{\Omegaddot}[0]{{\ddot{\Omega}}}
\newcommand{\Phiddot}[0]{{\ddot{\Phi}}}
\newcommand{\Piddot}[0]{{\ddot{\Pi}}}
\newcommand{\Psiddot}[0]{{\ddot{\Psi}}}
\newcommand{\Sigmaddot}[0]{{\ddot{\Sigma}}}
\newcommand{\Thetaddot}[0]{{\ddot{\Theta}}}
\newcommand{\Upsilonddot}[0]{{\ddot{\Upsilon}}}
\newcommand{\Xiddot}[0]{{\ddot{\Xi}}}
\newcommand{\alphaddot}[0]{{\ddot{\alpha}}}
\newcommand{\betaddot}[0]{{\ddot{\beta}}}
\newcommand{\chiddot}[0]{{\ddot{\chi}}}
\newcommand{\deltaddot}[0]{{\ddot{\delta}}}
\newcommand{\epsilonddot}[0]{{\ddot{\epsilon}}}
\newcommand{\etaddot}[0]{{\ddot{\eta}}}
\newcommand{\gammaddot}[0]{{\ddot{\gamma}}}
\newcommand{\kappaddot}[0]{{\ddot{\kappa}}}
\newcommand{\lambdaddot}[0]{{\ddot{\lambda}}}
\newcommand{\muddot}[0]{{\ddot{\mu}}}
\newcommand{\nuddot}[0]{{\ddot{\nu}}}
\newcommand{\omegaddot}[0]{{\ddot{\omega}}}
\newcommand{\phiddot}[0]{{\ddot{\phi}}}
\newcommand{\piddot}[0]{{\ddot{\pi}}}
\newcommand{\psiddot}[0]{{\ddot{\psi}}}
\newcommand{\rhoddot}[0]{{\ddot{\rho}}}
\newcommand{\sigmaddot}[0]{{\ddot{\sigma}}}
\newcommand{\tauddot}[0]{{\ddot{\tau}}}
\newcommand{\thetaddot}[0]{{\ddot{\theta}}}
\newcommand{\upsilonddot}[0]{{\ddot{\upsilon}}}
\newcommand{\varepsilonddot}[0]{{\ddot{\varepsilon}}}
\newcommand{\varphiddot}[0]{{\ddot{\varphi}}}
\newcommand{\varpiddot}[0]{{\ddot{\varpi}}}
\newcommand{\varrhoddot}[0]{{\ddot{\varrho}}}
\newcommand{\varsigmaddot}[0]{{\ddot{\varsigma}}}
\newcommand{\varthetaddot}[0]{{\ddot{\vartheta}}}
\newcommand{\xiddot}[0]{{\ddot{\xi}}}
\newcommand{\zetaddot}[0]{{\ddot{\zeta}}}

\newcommand{\BDelta}[0]{\boldsymbol{\Delta}}
\newcommand{\BGamma}[0]{\boldsymbol{\Gamma}}
\newcommand{\BLambda}[0]{\boldsymbol{\Lambda}}
\newcommand{\BOmega}[0]{\boldsymbol{\Omega}}
\newcommand{\BPhi}[0]{\boldsymbol{\Phi}}
\newcommand{\BPi}[0]{\boldsymbol{\Pi}}
\newcommand{\BPsi}[0]{\boldsymbol{\Psi}}
\newcommand{\BSigma}[0]{\boldsymbol{\Sigma}}
\newcommand{\BTheta}[0]{\boldsymbol{\Theta}}
\newcommand{\BUpsilon}[0]{\boldsymbol{\Upsilon}}
\newcommand{\BXi}[0]{\boldsymbol{\Xi}}
\newcommand{\Balpha}[0]{\boldsymbol{\alpha}}
\newcommand{\Bbeta}[0]{\boldsymbol{\beta}}
\newcommand{\Bchi}[0]{\boldsymbol{\chi}}
\newcommand{\Bdelta}[0]{\boldsymbol{\delta}}
\newcommand{\Bepsilon}[0]{\boldsymbol{\epsilon}}
\newcommand{\Beta}[0]{\boldsymbol{\eta}}
\newcommand{\Bgamma}[0]{\boldsymbol{\gamma}}
\newcommand{\Bkappa}[0]{\boldsymbol{\kappa}}
\newcommand{\Blambda}[0]{\boldsymbol{\lambda}}
\newcommand{\Bmu}[0]{\boldsymbol{\mu}}
\newcommand{\Bnu}[0]{\boldsymbol{\nu}}
%\newcommand{\Bomega}[0]{\boldsymbol{\omega}}
\newcommand{\Bphi}[0]{\boldsymbol{\phi}}
\newcommand{\Bpi}[0]{\boldsymbol{\pi}}
\newcommand{\Bpsi}[0]{\boldsymbol{\psi}}
\newcommand{\Brho}[0]{\boldsymbol{\rho}}
\newcommand{\Bsigma}[0]{\boldsymbol{\sigma}}
%\newcommand{\Btau}[0]{\boldsymbol{\tau}}
%\newcommand{\Btheta}[0]{\boldsymbol{\theta}}
\newcommand{\Bupsilon}[0]{\boldsymbol{\upsilon}}
\newcommand{\Bvarepsilon}[0]{\boldsymbol{\varepsilon}}
\newcommand{\Bvarphi}[0]{\boldsymbol{\varphi}}
\newcommand{\Bvarpi}[0]{\boldsymbol{\varpi}}
\newcommand{\Bvarrho}[0]{\boldsymbol{\varrho}}
\newcommand{\Bvarsigma}[0]{\boldsymbol{\varsigma}}
\newcommand{\Bvartheta}[0]{\boldsymbol{\vartheta}}
\newcommand{\Bxi}[0]{\boldsymbol{\xi}}
\newcommand{\Bzeta}[0]{\boldsymbol{\zeta}}
%
%</bold and dot greek symbols>
%<infrequent>
%
%\newcommand{\AreaOp}[1]{\AName_{#1}}
%\newcommand{\Babs}[0]{\abs{\BB}}
%\newcommand{\Bcap}[0]{\hat{\BB}}
%\newcommand{\BrPrimeRej}[0]{\rcap(\rcap \wedge \Br')}
%\newcommand{\CA}[0]{\mathcal{A}}
%\newcommand{\Cos}[1]{\cos{\left({#1}\right)}}
%\newcommand{\Det}[1] {\abs{#1}}
%\newcommand{\Dsq}[2] {\frac {\partial^2 {#1}} {\partial {#2}^2}}
%\newcommand{\Exp}[1]{\exp{\left({#1}\right)}}
%\newcommand{\Norm}[1]{\left\lVert{#1}\right\rVert}
%\newcommand{\Sin}[1]{\sin{\left({#1}\right)}}
%\newcommand{\T}[0]{\text{T}}
%\newcommand{\VolumeOp}[1]{\VName_{#1}}
%\newcommand{\agrad}[0]{\Ba \cdot \nabla}
%\newcommand{\alphacap}[0]{\hat{\boldsymbol{\alpha}}}
%\newcommand{\Fcap}[0]{\hat{\BF}}
%\newcommand{\bithree}[0]{{\Bi}_3}
%\newcommand{\bxa}[0]{\Bx\Ba}
%\newcommand{\coordvec}[2]{
%\newcommand{\costheta}[0]{\acap \cdot \xcap}
%\newcommand{\ddt}[1]{\ddot{#1}}
%\newcommand{\ddu}[1] {\frac {d{#1}} {du}}
%\newcommand{\dsqxj}[2] {\frac {\partial^2 {#1}} {\partial {x_{#2}}^2}}
%\newcommand{\dtheta}[1]{\frac{d {#1}}{d \theta}}
%\newcommand{\dt}[1]{\dot{#1}}
%\newcommand{\dt}[1]{\frac{d {#1}}{dt}}
%\newcommand{\dxj}[2] {\frac {\partial {#1}} {\partial {x_{#2}}}}
%\newcommand{\halfPhi}[0]{\frac{\phi}{2}}
%\newcommand{\half}[0]{\inv{2}}
%\newcommand{\inv}[1]{\frac{1}{#1}}
%\newcommand{\laplacian}[0]{\nabla^2}
%\newcommand{\matrixoftx}[3]{
%\newcommand{\nrrp}[0]{\norm{\rcap \wedge \Br'}}
%\newcommand{\oiint}{\bigcirc \hspace{-1.4em} \int \hspace{-.8em} \int}
%\newcommand{\transpose}[1]{{#1}^{\text{T}}}
%\newcommand{\transpose}[1]{{{#1}^{\TextTranspose}}}
%\newcommand{\transpose}[1]{{{#1}^{\text{T}}}}
%\newcommand{\barA}[0]{\bar{A}}
%\newcommand{\qbar}[0]{\bar{q}}
%\newcommand{\qdotbar}[0]{\dot{\bar{q}}}
%
%</infrequent>





\usepackage[bookmarks=true]{hyperref}

\usepackage{color,cite,graphicx}
   % use colour in the document, put your citations as [1-4]
   % rather than [1,2,3,4] (it looks nicer, and the extended LaTeX2e
   % graphics package. 
\usepackage{latexsym,amssymb,epsf} % don't remember if these are
   % needed, but their inclusion can't do any damage


\title{ Poynting vector and Electromagnetic Energy conservation. }
\author{Peeter Joot}
\date{ Dec 29, 2008.  Last Revision: $Date: 2008/12/31 01:06:10 $ }

\begin{document}

\maketitle{}

\tableofcontents

\section{ Motivation. }

Clarify Poynting discussion from \cite{doran2003gap}.

Equation 7.59 and 7.60 derives a $\BE \cross \BB$ quantity, the Poynting vector, as a sort of energy flux through the surface of the containing volume.

There are a couple of magic steps here that were not at all obvious to me.  Go through this in enough detail that it makes sense to me.

\section{ Charge free case. }

In SI units the Energy density is given as

\begin{align*}
U = \frac{\epsilon_0}{2}\left( \BE^2 + c^2 \BB^2 \right)
\end{align*}

FIXME: Don't truely understand where this part comes from.  The article \href{http://farside.ph.utexas.edu/teaching/em/lectures/node89.html}{Energy Conservation} looks promising to study this.

Given this energy density the rate of change of energy in a volume is then

\begin{align*}
\frac{dU}{dt} 
&= 
\frac{d}{dt} 
\frac{\epsilon_0}{2} \int dV \left( \BE^2 + c^2 \BB^2 \right) \\
&= 
\epsilon_0 \int dV \left( \BE \cdot \PD{t}{\BE} + c^2 \BB \cdot \PD{t}{\BB} \right) \\
\end{align*}

The next (omitted in the text) step is to utilize Maxwell's equation to eliminate the time derivatives.  Since this is the
charge and current free case, we can write Maxwell's as

\begin{align*}
0
&= \gamma_0 \grad F \\
&= \gamma_0 (\gamma^0 \partial_0 + \gamma^k \partial_k) F \\
&= (\partial_0 + \gamma_k\gamma_0 \partial_k) F \\
&= (\partial_0 + \sigma_k \partial_k) F \\
&= (\partial_0 + \spacegrad)F \\
&= (\partial_0 + \spacegrad)(\BE + ic \BB) \\
&= \partial_0 \BE + ic \partial_0 \BB + \spacegrad \BE + ic \spacegrad \BB \\
\end{align*}

In the spatial ($\sigma$) basis we can separate this into even and odd grades, which are separately equal to zero

\begin{align*}
0 &= \partial_0 \BE + ic \spacegrad \BB \\
%   1                  3,1 
0 &= ic \partial_0 \BB + \spacegrad \BE 
%  2                    0,2
\end{align*}

A selection of just the vector parts is

\begin{align*}
\partial_t \BE &= - ic^2 \spacegrad \wedge \BB \\
\partial_t \BB &= i\spacegrad \wedge \BE 
\end{align*}

Which can be back substituited into the energy flux
\begin{align*}
\frac{dU}{dt} 
&= \epsilon_0 \int dV \left( \BE \cdot (-i c^2 \spacegrad \wedge \BB) + c^2 \BB \cdot (i \spacegrad \wedge \BE) \right) \\
&= \epsilon_0 c^2 \int dV \gpgradezero{ \BB i \spacegrad \wedge \BE -\BE i \spacegrad \wedge \BB } \\
\end{align*}

Since the two divergence terms are zero we can drop the wedges here for

\begin{align*}
\frac{dU}{dt} 
&= \epsilon_0 c^2 \int dV \gpgradezero{ \BB i \spacegrad \BE -\BE i \spacegrad \BB } \\
&= \epsilon_0 c^2 \int dV \gpgradezero{ (i \BB) \spacegrad \BE -\BE \spacegrad (i\BB) } \\
&= \epsilon_0 c^2 \int dV \spacegrad \cdot ( (i \BB) \cdot \BE ) \\
\end{align*}

Justification for this last step can be found below in the derivation of equation \ref{eqn:poyntingDivergence}.

We can now use Stokes theorem to change this into a surface integral for a final energy flux 

\begin{align*}
\frac{dU}{dt} 
&= \epsilon_0 c^2 \int d\BA \cdot ( (i \BB) \cdot \BE ) \\
\end{align*}

This last bivector/vector dot product is the Poynting vector

\begin{align*}
(i \BB) \cdot \BE 
&= \gpgradeone{ (i \BB) \cdot \BE } \\
&= \gpgradeone{ i \BB \BE } \\
&= \gpgradeone{ i (\BB \wedge \BE) } \\
&= i (\BB \wedge \BE) \\
&= i^2(\BB \cross \BE) \\
&= \BE \cross \BB \\
\end{align*}

So, we can identity the quantity 

\begin{align}\label{eqn:poynting}
\epsilon_0 c^2 \BE \cross \BB = \epsilon_0 c (i c \BB) \cdot \BE 
\end{align}

As a directed energy density flux through the surface of a containing volume.


\section{ With charges and currents }
 
To calculate time derivatives we want to take Maxwell's equation and put into a form with explicit time derivatives, as was done before, but this time be more careful with the handling of the four vector current term.  Starting with left factoring out of a $\gamma_0$ from the spacetime gradient. 
 
\begin{align*}
\grad &= \gamma^0 \partial_0 + \gamma^k \partial_k \\
&= \gamma^0 (\partial_0 - \gamma^k \gamma_0 \partial_k) \\
&= \gamma^0 (\partial_0 + \sigma_k \partial_k) \\
\end{align*}

Similarily, the $\gamma_0$ can be factored from the current density

\begin{align*}
J 
&= \gamma_0 c \rho + \gamma_k J^k \\
&= \gamma_0 (c \rho - \gamma_k \gamma_0 J^k) \\
&= \gamma_0 (c \rho - \sigma_k J^k) \\
&= \gamma_0 (c \rho - \Bj )
\end{align*}

With this Maxwell's equation becomes
 
\begin{align*}
\gamma_0 \grad F &= \gamma_0 J / \epsilon_0 c \\
(\partial_0 + \spacegrad) ( \BE + i c \BB ) &= \rho/\epsilon_0 - \Bj/\epsilon_0 c \\
\end{align*}
 
A split into even and odd grades including current and charge density is thus
 
\begin{align*}
\spacegrad \BE + \partial_t (i \BB) &= \rho/\epsilon_0 \\
\spacegrad (i \BB) c^2 + \partial_t \BE &= -\Bj/\epsilon_0
\end{align*}
 
Now, taking time derivatives of the energy density gives

\begin{align*}
\PD{t}{U} 
&= \PD{t}{}\inv{2} \epsilon_0 \left( \BE^2 - (ic \BB)^2 \right) \\
&= \epsilon_0 \left( \BE \cdot \partial_t \BE - c^2 (i\BB) \cdot \partial_t (i\BB) \right) \\
&= \epsilon_0 \gpgradezero{ \BE ( -\Bj/\epsilon_0 -\spacegrad (i \BB) c^2 ) - c^2 (i\BB) ( -\spacegrad \BE + \rho/\epsilon_0 ) } \\
&= -\BE \cdot \Bj + c^2 \epsilon_0 \gpgradezero{ i\BB \spacegrad \BE -\BE \spacegrad (i \BB) } \\
&= -\BE \cdot \Bj + c^2 \epsilon_0 \left( (i\BB) \cdot (\spacegrad \wedge \BE) - \BE \cdot (\spacegrad \cdot (i \BB)) \right) \\
\end{align*}

Using equation \ref{eqn:poyntingDivergence}, we now have the rate of change of
field energy for the general case including currents.  That is

\begin{align}
\PD{t}{U} &= -\BE \cdot \Bj + c^2 \epsilon_0 \spacegrad \cdot (\BE \cdot (i\BB)) 
\end{align}

Written out in full, and in terms of the Poynting vector this is

\begin{align}
\PD{t}{}\frac{\epsilon_0}{2} \left(\BE^2 + c^2 \BB^2\right) + c^2 \epsilon_0 \spacegrad \cdot (\BE \cross \BB) &= -\BE \cdot \Bj 
\end{align}

\section{ Poynting vector in terms of complete field. }

In equation \ref{eqn:poynting} the individual parts of the complete Faraday
bivector $F = \BE + i c \BB$ stand out.  How would the Poynting vector be
expressed in terms of $F$ or in tensor form?

Since
\begin{align*}
F \gamma_0 = - \gamma_0(\BE - i c \BB)
\end{align*}

we have
\begin{align*}
\gamma^0 F \gamma_0 = - (\BE - i c \BB)
\end{align*}

and
\begin{align*}
i c \BB &= \inv{2}(F + \gamma^0 F \gamma_0) \\
\BE &= \inv{2}(F - \gamma^0 F \gamma_0) \\
\end{align*}

FIXME: tried using these but messed up.

%Without justifying all the steps I think that the following is valid
%
%\begin{align*}
%(i c \BB) \cdot \BE 
%&= \gpgradeone{(i c \BB) \cdot \BE } \\
%&= \gpgradeone{i c \BB \BE } \\
%&= \inv{4} (F + \gamma_0 F \gamma_0) \cdot (F - \gamma_0 F \gamma_0) \\
%&= \inv{4} (F^2 - \gamma_0 F \gamma_0 \gamma_0 F \gamma_0 + (\gamma_0 F \gamma_0) \cdot F - F \cdot (\gamma_0 F \gamma_0) ) \\
%&= \inv{4} ( (\gamma_0 F \gamma_0) \cdot F - F \cdot (\gamma_0 F \gamma_0) ) \\
%&= \inv{2} (\gamma_0 F \gamma_0) \cdot F 
%\end{align*}
%
%  The above is wrong.  This is - <F^\dagger F>/2, which is c^2 B^2 - E^2, which isn't even vector.

\section{ Energy Density from Lagrangian. }

I didn't get too far trying to calculate the electrodynamic Hamiltonian density for the general case, so I tried it for a very 
simple special case, with just an electric field component in one direction:

\begin{align*}
\mathcal{L}
&= \frac{1}{2}(E_x)^2 \\
&= \frac{1}{2}(F_{01})^2 \\
&= \frac{1}{2}(\partial_0 A_1 - \partial_1 A_0)^2 \\
\end{align*}

Goldstein gives the Hamiltonian density as

\begin{align*}
\pi &= \frac{\partial \mathcal{L}}{\partial \dot{n}} \\
\mathcal{H} &= \dot{n} \pi - \mathcal{L}
\end{align*}

If I try calculating this I get

\begin{align*}
\pi 
&= \frac{\partial}{\partial (\partial_0 A_1)} \left( \frac{1}{2}(\partial_0 A_1 - \partial_1 A_0)^2 \right) \\
&= \partial_0 A_1 - \partial_1 A_0 \\
&= F_{01} \\
\end{align*}

So this gives a Hamiltonian of
\begin{align*}
\mathcal{H}
&= \partial_0 A_1 F_{01} - \frac{1}{2}(\partial_0 A_1 - \partial_1 A_0)F_{01} \\
&= \frac{1}{2} (\partial_0 A_1 + \partial_1 A_0 )F_{01} 
&= \frac{1}{2} ((\partial_0 A_1)^2 - (\partial_1 A_0)^2 )
\end{align*}

For a Lagrangian density of $E^2 - B^2$ we have an energy density of $E^2 + B^2$, so I'd have expected the Hamiltonian density here to stay equal to $E_x^2/2$, but it 
doesn't look like that's what I get (what I calculated isn't at all familiar seeming).

If I haven't made a mistake here, perhaps I'm incorrect in assuming that the Hamiltonian density of the electrodynamic Lagrangian should be the energy density?

\section{ Appendix.  Messy details. }

For both the charge and the charge free case, we need a proof of 

\begin{align*}
(i\BB) \cdot (\spacegrad \wedge \BE) - \BE \cdot (\spacegrad \cdot (i \BB)) 
&= \spacegrad \cdot (\BE \cdot (i\BB)) 
\end{align*}

This is relativity straightforward, albeit tedious, to do backwards.

\begin{align*}
\spacegrad \cdot ((i \BB) \cdot \BE)
&= \gpgradezero{ \spacegrad ((i \BB) \cdot \BE)} \\
&= \inv{2} \gpgradezero{ \spacegrad ( i \BB \BE - \BE i \BB ) } \\
&= \inv{2} \gpgradezero{ 
  \dot{\spacegrad} i \dot{\BB} \BE 
+ \dot{\spacegrad} i \BB \dot{\BE}
- \dot{\spacegrad} \dot{\BE} i \BB 
- \dot{\spacegrad} \BE i \dot{\BB}
} \\
&= \inv{2} \gpgradezero{ 
  \BE \spacegrad (i \BB) - (i\dot{\BB}) \dot{\spacegrad} \BE
+ \dot{\BE} \dot{\spacegrad} i \BB - i \BB \spacegrad \BE
} \\
&= \inv{2} \left(
  \BE \cdot (\spacegrad \cdot (i \BB)) - ((i\dot{\BB}) \cdot \dot{\spacegrad}) \cdot \BE
+ (\dot{\BE} \wedge \dot{\spacegrad}) \cdot (i \BB) - (i \BB) \cdot (\spacegrad \wedge \BE) 
\right)
\\
\end{align*}

Grouping the two sets of repeated terms after reordering and the associated sign adjustments we have

\begin{align}\label{eqn:poyntingDivergence}
\spacegrad \cdot ((i \BB) \cdot \BE) &= \BE \cdot (\spacegrad \cdot (i \BB)) - (i \BB) \cdot (\spacegrad \wedge \BE)
\end{align}

which is the desired identity (in negated form) that was to be proved.

There is likely some theorem that could be used to avoid some of this algebra.

\bibliographystyle{plainnat}
\bibliography{myrefs}

\end{document}

      \input{Set7Problem1.tex}
      \input{poyntingTimeHarmonic.tex}

   %
% Copyright � 2016 Peeter Joot.  All Rights Reserved.
% Licenced as described in the file LICENSE under the root directory of this GIT repository.
%
\newcommand{\authorname}{Peeter Joot}
\newcommand{\email}{peeterjoot@protonmail.com}
\newcommand{\basename}{FIXMEbasenameUndefined}
\newcommand{\dirname}{notes/FIXMEdirnameUndefined/}

\renewcommand{\basename}{emt6}
\renewcommand{\dirname}{notes/ece1228/}
\newcommand{\keywords}{ECE1228H}
\newcommand{\authorname}{Peeter Joot}
\newcommand{\onlineurl}{http://sites.google.com/site/peeterjoot2/math2013/\basename.pdf}
\newcommand{\sourcepath}{\dirname\basename.tex}
\newcommand{\generatetitle}[1]{\chapter{#1}}

\newcommand{\vcsinfo}{%
\section*{}
\noindent{\color{DarkOliveGreen}{\rule{\linewidth}{0.1mm}}}
\paragraph{Document version}
%\paragraph{\color{Maroon}{Document version}}
{
\small
\begin{itemize}
\item Available online at:\\ 
\href{\onlineurl}{\onlineurl}
\item Git Repository: \input{./.revinfo/gitRepo.tex}
\item Source: \sourcepath
\item last commit: \input{./.revinfo/gitCommitString.tex}
\item commit date: \input{./.revinfo/gitCommitDate.tex}
\end{itemize}
}
}

%\PassOptionsToPackage{dvipsnames,svgnames}{xcolor}
\PassOptionsToPackage{square,numbers}{natbib}
\documentclass{scrreprt}

\usepackage[left=2cm,right=2cm]{geometry}
\usepackage[svgnames]{xcolor}
\usepackage{peeters_layout}

\usepackage{natbib}

\usepackage[
colorlinks=true,
bookmarks=false,
pdfauthor={\authorname, \email},
backref 
]{hyperref}

% http://tex.stackexchange.com/questions/75773/how-to-reference-problems-by-the-text-label-in-an-exercise-envioronment
\usepackage[english]{cleveref}
\crefname{Exercise}{exercise}{exercises}
\Crefname{Exercise}{Exercise}{Exercises}

\RequirePackage{titlesec}
\RequirePackage{ifthen}

% http://stackoverflow.com/questions/4932910/date-in-the-tabular-environment
\makeatletter
\let\insertdate\@date
\makeatother

\titleformat{\chapter}[display]
{\bfseries\Large}
{\color{DarkSlateGrey}\filleft \authorname
\ifthenelse{\isundefined{\studentnumber}}{}{\\ \studentnumber}
\ifthenelse{\isundefined{\email}}{}{\\ \email}
\ifthenelse{\isundefined{\dateintitle}}{}{\\ \insertdate}
%\ifthenelse{\isundefined{\coursename}}{}{\\ \coursename} % put in title instead.
}
{4ex}
{\color{DarkOliveGreen}{\titlerule}\color{Maroon}
\vspace{2ex}%
\filright}
[\vspace{2ex}%
\color{DarkOliveGreen}\titlerule
]

\newcommand{\beginArtWithToc}[0]{\begin{document}\tableofcontents}
\newcommand{\beginArtNoToc}[0]{\begin{document}}
\newcommand{\EndNoBibArticle}[0]{\end{document}}
\newcommand{\EndArticle}[0]{\bibliography{Bibliography}\bibliographystyle{plainnat}\end{document}}

% 
%\newcommand{\citep}[1]{\cite{#1}}

\colorSectionsForArticle



%\usepackage{ece1228}
\usepackage{peeters_braket}
%\usepackage{peeters_layout_exercise}
\usepackage{peeters_figures}
\usepackage{mathtools}
\usepackage{siunitx}
\usepackage{enumerate}

\beginArtNoToc
\generatetitle{ECE1228H Electromagnetic Theory.  Lecture 6: XXX.  Taught by Prof.\ M. Mojahedi}
%\chapter{XXX}
\label{chap:emt6}

%\paragraph{Disclaimer}
%
%Peeter's lecture notes from class.  These may be incoherent and rough.
%
%These are notes for the UofT course ECE1228H, Electromagnetic Theory, taught by Prof. M. Mojahedi, covering \textchapref{{1}} \citep{balanis1989advanced} content.

\paragraph{Lorentz-Lorenz Dispersion}

We will model the medium using a frequency representation of the permittivity

\begin{dmath}\label{eqn:emtLecture6:20}
\begin{aligned}
\epsilon(\omega) &= \epsilon'(\omega) - j \epsilon''(\omega) \\
\mu(\omega) &= \mu'(\omega) - j \mu''(\omega)
\end{aligned}
\end{dmath}

The real part is the phase, whereas the imaginary part is the loss.

\begin{dmath}\label{eqn:emtLecture6:40}
n = \frac{c}{v} 
= \frac{\sqrt{\epsilon \mu}}{\sqrt{\epsilon_0 \mu_0}}  
= \sqrt{\epsilon_r \mu_r}
\end{dmath}

We can also write

\begin{dmath}\label{eqn:emtLecture6:60}
n(\omega) = n'(\omega) - j n''(\omega)
\end{dmath}

If we are considering an electric dipole

\begin{dmath}\label{eqn:emtLecture6:80}
\BP_i = Q_i \Bx_i
\end{dmath}

With 

\begin{dmath}\label{eqn:emtLecture6:100}
\BP = \epsilon_0 \chi_e \BE,
\end{dmath}

and a time harmonic representation for the electric field

\begin{dmath}\label{eqn:emtLecture6:120}
\BE = \BE_0 e^{j \omega t}.
\end{dmath}

The dipole moment is assumed to be

\begin{dmath}\label{eqn:emtLecture6:140}
\BP = \lim_{\Delta v \rightarrow 0} \frac{ \sum_{i = 1}^{N \Delta v} \BP_i }{\Delta v} 
= \frac{ N \Delta v \Bp}{\Delta v}
= N \Bp
= N Q \Bx.
\end{dmath}

F1: 

We model the oscillating electron and nucleus as a mass and spring.
This electron oscillator model is often called the Lorentz model.  It is not really a model for atoms as such, but the way that an atom responds to pertubation.  At the time when Lorentz formulated the model it was not known that the nuclei havr massive mass as compared to the electrons.
The Lorentz assumption was that in the absence of applied eletric fields the centroids of positive and neagivve charges coincide, but when a field is applied, the electrons will experience a Lorentz force and will be displaced from their equilibrium position. 
The wrote ``the displacement immediately gives rise to a new force by which the particle is pulled back towards its original position, and which we may therefore appropriately distinguish by the name of elastic force.''

The forces of interest are

\begin{dmath}\label{eqn:emtLecture6:160}
\begin{aligned}
F_{\textrm{friction}} &= -D \frac{dx}{dt} = -D v \\
F_{\textrm{elastic}} &= -S x \\
F_{\textrm{external}} &= Q E = Q E_0 e^{j \omega t}
\end{aligned}
\end{dmath}

Adding all the forces, the electrical system, in one dimension, can be assumed to have the form

\begin{equation}\label{eqn:emtLecture6:180}
F = m \frac{d^2 x}{dt^2}
=
-D \frac{dx}{dt} 
-D v \\
-S x \\
+ Q E_0 e^{j \omega t},
\end{equation}

or
\begin{dmath}\label{eqn:emtLecture6:200}
\frac{d^2 x}{dt^2} + \frac{D}{m} \ddt{x} + \frac{S}{m} x = \frac{Q E_0}{m} e^{j \omega t}
\end{dmath}

Let's define 

\begin{dmath}\label{eqn:emtLecture6:220}
\begin{aligned}
\gamma &= \frac{D}{m} \\
\omega_0^2 &= \frac{S}{m},
\end{aligned}
\end{dmath}

so that

\begin{dmath}\label{eqn:emtLecture6:240}
\frac{d^2 x}{dt^2} + \gamma \ddt{x} + \omega_0^2 x = \frac{Q E_0}{m} e^{j \omega t}.
\end{dmath}

\paragraph{Calculating the permittivity and susceptibility}

With \( x = x_0 e^{j \omega t} \) we have

\begin{dmath}\label{eqn:emtLecture6:260}
x_0 \lr{ -\omega^2 + j\gamma \omega + \omega_0^2 } = \frac{Q E_0}{m},
\end{dmath}

or (with \( E = E_0 e^{j \omega t} \)), just

\begin{equation}\label{eqn:emtLecture6:280}
x = x_0 e^{j\omega t} 
= \frac{Q E}{m \lr{ -\omega^2 + j\gamma \omega + \omega_0^2 } }.
\end{equation}

\begin{enumerate}[I]
\item Assume that dipoles are identical
\item Assume no coupling between dipoles
\item There are N dipoles per unit volume.  In other words, N is the number of dipoles per unit volume.
\end{enumerate}

The polarization \( P(t) \) is given by

\begin{dmath}\label{eqn:emtLecture6:300}
P(t) = N Q x,
\end{dmath}

where \( Q \) is the charge associate with the unit dipole.  This has dimensions of [\si{\frac{1}{m^3} \times C \times m}], or [\si{C/m^2}].  This polarization is

\begin{dmath}\label{eqn:emtLecture6:440}
P(t)
= \frac{Q^2 N E/m}{\omega_0^2 -\omega^2 + j\gamma \omega }.
\end{dmath}

In particular, the ratio of the polarization to the electric field magnitude is

\begin{dmath}\label{eqn:emtLecture6:320}
\frac{P}{E}
= \frac{Q^2 N/ m}{\omega_0^2 -\omega^2 + j\gamma \omega }.
\end{dmath}

With \( P = \epsilon_0 \chi_e E \), we have

\begin{dmath}\label{eqn:emtLecture6:340}
\chi_e = \frac{Q^2 N/ m \epsilon_0}{\omega_0^2 -\omega^2 + j\gamma \omega }.
\end{dmath}

Define 

\begin{dmath}\label{eqn:emtLecture6:360}
\omega_p^2 = \frac{ Q^2 N}{m \epsilon_0},
\end{dmath}

which has dimensions [\si{1/s^2}].  Then

\begin{dmath}\label{eqn:emtLecture6:380}
\chi_e = \frac{\omega_p^2}{\omega_0^2 -\omega^2 + j\gamma \omega }.
\end{dmath}

With \( \epsilon_r = 1 + \chi_e \) we have

\begin{equation}\label{eqn:emtLecture6:400}
\epsilon_r 
= \frac{\epsilon}{\epsilon_0} 
= 1 + \frac{\omega_p^2}{\omega_0^2 -\omega^2 + j\gamma \omega }.
\end{equation}

%or
%\begin{dmath}\label{eqn:emtLecture6:420}
%\epsilon_r 
%= \frac{ \omega_0^2 -\omega^2 + j\gamma \omega + \omega_p^2}{\omega_0^2 -\omega^2 + j\gamma \omega }
%\end{dmath}

One can show that \( \epsilon_r = \epsilon_r' -j \epsilon_r'' \) are given bby

\begin{dmath}\label{eqn:emtLecture6:460}
\epsilon_r' = \frac{\omega_p^2 \lr{ \omega_0^2 - \omega^2 } }{ (\omega_0^2 - \omega^2)^2 + (\omega \gamma)^2 } + 1,
\end{dmath}
\begin{dmath}\label{eqn:emtLecture6:480}
\epsilon_r'' = \frac{\omega_p^2 \omega \gamma}{ (\omega_0^2 - \omega^2)^2 + (\omega \gamma)^2 }.
\end{dmath}

FIXME: calculate this.

\paragraph{No damping}

With \( D = 0 \), or \( \gamma = 0 \) then \( \epsilon_r'' = 0 \),

\begin{dmath}\label{eqn:emtLecture6:500}
x = \frac{Q E_0/m}{\omega^2 - \omega^2} e^{j \omega t},
\end{dmath}

and
\begin{equation}\label{eqn:emtLecture6:520}
\epsilon_r 
=
\epsilon_r'
= \frac{\epsilon}{\epsilon_0} 
= 
1 + \frac{\omega_p^2}{\omega_0^2 - \omega^2}.
\end{equation}

This has a curve like

F5

instead of the normal damped resonance curve

F5b

As \( \omega \rightarrow \omega_0 \), then the displacement \( x \rightarrow \infty \).  The frequency \( \omega_0 \) is called the resonance frequency of the system.

If the resonance frequency is zero (free charges), then

\begin{dmath}\label{eqn:emtLecture6:540}
\epsilon_r = \epsilon_r' = 1 - \frac{\omega_p^2}{\omega^2},
\end{dmath}

which is negative for \( \omega_p > \omega \).  

When damping is present, the resonance frequency is the root of the characteristic equation of the homogeneous part of \cref{eqn:emtLecture6:200}.

\paragraph{Multiple resonances}

When there are \( N \) molecules per unit volume, and each molecule has
Z electrons per molecule that have a binding frequency \( \omega_i \) and damping constant \( \gamma_i \), then it can be shown that 

\begin{dmath}\label{eqn:emtLecture6:560}
\epsilon_r = 1 + \frac{Q N^2}{m \epsilon_0} \sum \frac{ f_i }{\omega_i^2 - \omega^2 + j \gamma \omega }
\end{dmath}

A quantum mechanical derivation of the transition frequencies is used in this derivation.

%\EndArticle
\EndNoBibArticle

      \section{Problems}

      \input{Set5Problem1.tex}
      \input{Set5Problem2.tex}
      \input{Set5Problem3.tex}

   \chapter{Druid model}
      
usepackage: \ce{}. chem.

\paragraph{Druid model}

In this section we will investigate the optical properties of free electrons, or what is commonly called free electron gas.

By free electron gas we mean electrons that do not experience the restoring force which we considered for bound garges in the case of Lorentz model.  In particular, the resonance frequency \( \omega_0 \) for free electrons is zero.

There are two typical cases of free electron systems 

\begin{itemize}[a]
\item Metals.
\item Doped (n or p type) semiconductors.
\end{itemize}

For the moment we consider the case of metals.

Free electrons are responsible for high reflectivity and good thermal conductivity of metals up to optical frequencies.  A model that can be used to describe the high reflectivity of metals is the Drude model.

\paragraph{Plasma:} A neutral gas of free eletrons and heavy ions is called plasma.  Examples of plasma are metals and doped semiconduction, since these materials are a compination of free electrons and heavy ions which are, in sum, electrically neutral.

\paragraph{Drude-Lorentz model}, (or Drude model for short): similar to the case of bound charges we already studied for free electron plasma, we can start with a harmonic oscillator model.  However, in this case, since electrons are free, there is no restoring force (i.e. \(\omega_0 = 0 \).  Recall that in the spring mass model \( \omega_0^2 = S/m \) where \( S \) was the spring tension coefficient.

With such a model the Lorentz model equation

\begin{dmath}\label{eqn:druid:20}
\frac{d^2 x}{dt^2} + \gamma \ddt{x} + \omega_0^2 x = \frac{Q E_0}{m} e^{j \omega t},
\end{dmath}

is reduced to

\begin{dmath}\label{eqn:druid:40}
\frac{d^2 x}{dt^2} + \gamma \ddt{x} = \frac{Q E_0}{m} e^{j \omega t},
\end{dmath}

Again, assuming a solution of the form \( x_p = x_0 e^{j \omega t} \) for the particular solution and substituting in \cref{eqn:druid:40}, we have

\begin{dmath}\label{eqn:druid:80}
x_0 \lr{ (j\omega)^2 + \gamma (j \omega)} = \frac{Q E_0}{m},
\end{dmath}

or
\begin{dmath}\label{eqn:druid:60}
x 
= 
\frac{Q E/m}{-\omega^2 + j \gamma \omega },
\end{dmath}

Once more assuming identical particles that are not coupled and a linear isotropic medium and using the fact that \( \BP = N \Bp = N Q \Bx \), and

\begin{dmath}\label{eqn:druid:100}
\chi_e = \frac{\Abs{\BP}}{\epsilon_0 \Abs{\BE} }, 
\end{dmath}

we have

\begin{dmath}\label{eqn:druid:120}
\chi_e 
=
\frac{Q^2 N/m \epsilon_0}{-\omega^2 + j \gamma \omega },
\end{dmath}

or with \( \omega_p^2 = Q^2 N/m\epsilon_0\),

\begin{dmath}\label{eqn:druid:140}
\epsilon_r 
= 1 + \chi_e 
= 
1+
\frac{\omega_p^2}{-\omega^2 + j \gamma \omega }.
\end{dmath}

Plasma frequency, \( \omega_p \), can be understood as the natural resonance frequency by which the free electron gas (plasma) collectively (not individulal electrons ) oscillates.

Note that if we neglect the last term, i.e., let \( \gamma = 0 \) then

\begin{dmath}\label{eqn:druid:160}
\epilson_r = 1 - \frac{\omega_p^2}{\omega^2}.
\end{dmath}

From this it is clear that when \( \omega < \omega_p \), we have \( \epsilon_r < 1 \) and \( n = \sqrt{\epsilon_r} \) is purely imaginary, and the wave attenuates inside the electron plasma.

This means that for \( \omega < \omega_p \) electromagnetic waves do not propagate a large distance inside of metal.  However, for \( \omega > \omega_p \) the electron plasma (e.g. metal) is transparent.  The latter is called ultraviolate transparency of metal, because for most metals \( \omega_p \) is in the ultraviolate part of the spectrum.  For example, 

\begin{itemize}
\item For \ce{Al}
\begin{dmath}\label{eqn:druid:180}
\frac{\omega_p}{2 \pi} = 3.82 \times 10^{15} \si{Hz} \implies \lambda_p = 79 [nm],
\end{dmath}
\item For \ce{Au}
\begin{dmath}\label{eqn:druid:200}
\frac{\omega_p}{2 \pi} = 5.9 \times 10^{15} \si{Hz} \implies \lambda_p = 138 [nm],
\end{dmath}
\end{itemize}

Using \cref{eqn:druid:160} one can calculate 

\begin{dmath}\label{eqn:druid:220}
\tilde{n} = \sqrt{\epsilon_r},
\end{dmath}

and plot the reflectivity \( R \) at normal incidence

\begin{dmath}\label{eqn:druid:240}
R = \Abs{ \frac{\tilde{n} - 1 }{\tilde{n} + 1} },0jA
\end{dmath}

which will have a shape similar to

F3

\paragraph{Conductivity}

%\ddt{} \lr{ m \ddt{x} } + \gamma \ddt{x} = Q E_0 e^{i \omega t}

\begin{dmath}\label{eqn:druid:260}
\spacegrad \cross \BE(\Br, \omega) 
= \sigma \BE(\Br, \omega) + j \omega \epsilon_0 \BE(\Br, \omega)
= j \omega \epsilon_0 \lr{ 1 + \frac{\sigma}{j \omega \epsilon_0} } \BE(\Br, \omega)
= j \omega \epsilon_0 \lr{ 1 - \frac{j \sigma}{\omega \epsilon_0} } \BE(\Br, \omega)
\end{dmath}

This complex factor is the relative permittivity

\begin{dmath}\label{eqn:druid:280}
\epsilon_r  
\end{dmath}
= 1 - \frac{j \sigma}{\omega \epsilon_0},
\end{dmath}

and is why we write

\begin{dmath}\label{eqn:druid:300}
\epsilon(\omega) = \epsilon'(\omega) - j \epsilon''(\omega) 
\end{dmath}

%      FIXME: transcribe handwritten notes that were mostly skipped over in class?
      \section{Problems}

%\chapter{conductivity} % transcribe?  This was part of L7
   %
% Copyright � 2016 Peeter Joot.  All Rights Reserved.
% Licenced as described in the file LICENSE under the root directory of this GIT repository.
%
%\newcommand{\authorname}{Peeter Joot}
\newcommand{\email}{peeterjoot@protonmail.com}
\newcommand{\basename}{FIXMEbasenameUndefined}
\newcommand{\dirname}{notes/FIXMEdirnameUndefined/}

%\renewcommand{\basename}{emt7}
%\renewcommand{\dirname}{notes/ece1228/}
%\newcommand{\keywords}{ECE1228H}
%\newcommand{\authorname}{Peeter Joot}
\newcommand{\onlineurl}{http://sites.google.com/site/peeterjoot2/math2013/\basename.pdf}
\newcommand{\sourcepath}{\dirname\basename.tex}
\newcommand{\generatetitle}[1]{\chapter{#1}}

\newcommand{\vcsinfo}{%
\section*{}
\noindent{\color{DarkOliveGreen}{\rule{\linewidth}{0.1mm}}}
\paragraph{Document version}
%\paragraph{\color{Maroon}{Document version}}
{
\small
\begin{itemize}
\item Available online at:\\ 
\href{\onlineurl}{\onlineurl}
\item Git Repository: \input{./.revinfo/gitRepo.tex}
\item Source: \sourcepath
\item last commit: \input{./.revinfo/gitCommitString.tex}
\item commit date: \input{./.revinfo/gitCommitDate.tex}
\end{itemize}
}
}

%\PassOptionsToPackage{dvipsnames,svgnames}{xcolor}
\PassOptionsToPackage{square,numbers}{natbib}
\documentclass{scrreprt}

\usepackage[left=2cm,right=2cm]{geometry}
\usepackage[svgnames]{xcolor}
\usepackage{peeters_layout}

\usepackage{natbib}

\usepackage[
colorlinks=true,
bookmarks=false,
pdfauthor={\authorname, \email},
backref 
]{hyperref}

% http://tex.stackexchange.com/questions/75773/how-to-reference-problems-by-the-text-label-in-an-exercise-envioronment
\usepackage[english]{cleveref}
\crefname{Exercise}{exercise}{exercises}
\Crefname{Exercise}{Exercise}{Exercises}

\RequirePackage{titlesec}
\RequirePackage{ifthen}

% http://stackoverflow.com/questions/4932910/date-in-the-tabular-environment
\makeatletter
\let\insertdate\@date
\makeatother

\titleformat{\chapter}[display]
{\bfseries\Large}
{\color{DarkSlateGrey}\filleft \authorname
\ifthenelse{\isundefined{\studentnumber}}{}{\\ \studentnumber}
\ifthenelse{\isundefined{\email}}{}{\\ \email}
\ifthenelse{\isundefined{\dateintitle}}{}{\\ \insertdate}
%\ifthenelse{\isundefined{\coursename}}{}{\\ \coursename} % put in title instead.
}
{4ex}
{\color{DarkOliveGreen}{\titlerule}\color{Maroon}
\vspace{2ex}%
\filright}
[\vspace{2ex}%
\color{DarkOliveGreen}\titlerule
]

\newcommand{\beginArtWithToc}[0]{\begin{document}\tableofcontents}
\newcommand{\beginArtNoToc}[0]{\begin{document}}
\newcommand{\EndNoBibArticle}[0]{\end{document}}
\newcommand{\EndArticle}[0]{\bibliography{Bibliography}\bibliographystyle{plainnat}\end{document}}

% 
%\newcommand{\citep}[1]{\cite{#1}}

\colorSectionsForArticle


%
%%\usepackage{ece1228}
%\usepackage{peeters_braket}
%%\usepackage{peeters_layout_exercise}
%\usepackage{peeters_figures}
%\usepackage{mathtools}
%\usepackage{siunitx}
%\usepackage{macros_bm}
%
%\beginArtNoToc
%\generatetitle{ECE1228H Electromagnetic Theory.  Lecture 8: Wave equation.  Taught by Prof.\ M. Mojahedi}
\chapter{Wave equation}
%\label{chap:emt7}

%\paragraph{Disclaimer}
%
%Peeter's lecture notes from class.  These may be incoherent and rough.
%
%These are notes for the UofT course ECE1228H, Electromagnetic Theory, taught by Prof. M. Mojahedi, covering \textchapref{{1}} \citep{balanis1989advanced} content.
%
\paragraph{Wave equation}

\begin{dmath}\label{eqn:emtLecture7:20}
\begin{aligned}
\spacegrad \cross \bcE &= -\PD{t}{\bcB} - \bcM \\
\spacegrad \cross \bcH &= \PD{t}{\bcD} + \bcJ \\
\spacegrad \cross \bcB &= \rho_{mv} \\
\spacegrad \cross \bcD &= \rho_{ev} \\
\end{aligned}
\end{dmath}

Using an expansion of the triple cross product in terms of the Laplacian
\begin{dmath}\label{eqn:emtLecture7:40}
\spacegrad \cross \lr{ \spacegrad \cross \Bf }
=
-\spacegrad \cdot \lr{ \spacegrad \wedge \Bf }
=
-\spacegrad^2 \Bf
+ \spacegrad \lr{ \spacegrad \cdot \Bf },
\end{dmath}

we can evaluate the cross products

\begin{dmath}\label{eqn:emtLecture7:60}
\begin{aligned}
\spacegrad \cross \lr{ \spacegrad \cross \bcE } &= \spacegrad \cross \lr{ -\PD{t}{\bcB} - \bcM } \\
\spacegrad \cross \lr{ \spacegrad \cross \bcH } &= \spacegrad \cross \lr{ \PD{t}{\bcD} + \bcJ },
\end{aligned}
\end{dmath}

or
\begin{dmath}\label{eqn:emtLecture7:80}
\begin{aligned}
-\spacegrad^2 \bcE + \spacegrad \lr{ \spacegrad \cdot \bcE } &= -\mu \PD{t}{} \spacegrad \cross \bcH - \spacegrad \cross \bcM \\
-\spacegrad^2 \bcH + \spacegrad \lr{ \spacegrad \cdot \bcH } &= \epsilon \PD{t}{} \lr{ \spacegrad \cross \bcE } + \spacegrad \cross \bcJ,
\end{aligned}
\end{dmath}

or

\begin{dmath}\label{eqn:emtLecture7:100}
\begin{aligned}
-\spacegrad^2 \bcE + \inv{\epsilon} \spacegrad \rho_{ev} &= -\mu \PD{t}{} \lr{ \PD{t}{\bcD} + \bcJ } - \spacegrad \cross \bcM \\
-\spacegrad^2 \bcH + \inv{\mu} \spacegrad \rho_{mv} &= \epsilon \PD{t}{} \lr{ -\PD{t}{\bcB} - \bcM } + \spacegrad \cross \bcJ,
\end{aligned}
\end{dmath}

This decouples the equations for the electric and the magnetic fields

\begin{dmath}\label{eqn:emtLecture7:120}
\begin{aligned}
\spacegrad^2 \bcE &=
   \mu \epsilon \PDSq{t}{\bcE} +
   \inv{\epsilon} \spacegrad \rho_{ev} +
   \mu \PD{t}{\bcJ } +
   \spacegrad \cross \bcM \\
\spacegrad^2 \bcH &=
   \epsilon \mu \PDSq{t}{\bcH} +
   \inv{\mu} \spacegrad \rho_{mv} +
   \epsilon \PD{t}{\bcM } -
   \spacegrad \cross \bcJ,
\end{aligned}
\end{dmath}

Splitting the current between induced and bound (?) currents

\begin{equation}\label{eqn:emtLecture7:260}
\bcJ = \bcJ_i + \bcJ_c = \bcJ_i + \sigma \bcE,
\end{equation}

these become

\begin{dmath}\label{eqn:emtLecture7:160}
\begin{aligned}
\spacegrad^2 \bcE &=
   \mu \epsilon \PDSq{t}{\bcE} +
   \inv{\epsilon} \spacegrad \rho_{ev} +
   \mu \sigma \PD{t}{\bcE} +
   \spacegrad \cross \bcM +
   \mu \PD{t}{\bcJ_i} \\
\spacegrad^2 \bcH &=
   \epsilon \mu \PDSq{t}{\bcH} +
   \inv{\mu} \spacegrad \rho_{mv} +
   \epsilon \PD{t}{\bcM } +
   \sigma \mu \PD{t}{\bcH} +
   \sigma \bcM
-
   \spacegrad \cross \bcJ_i
.
\end{aligned}
\end{dmath}

\paragraph{Time harmonic form}

Assuming time harmonic dependence \( \bcX = \BX e^{j\omega t} \), we find

\begin{dmath}\label{eqn:emtLecture7:140}
\begin{aligned}
\spacegrad^2 \BE &=
   \lr{ - \omega^2 \mu \epsilon +
   j \omega \mu \sigma } \BE +
   \inv{\epsilon} \spacegrad \rho_{ev} +
   \spacegrad \cross \BM +
   j \omega \mu \BJ_i \\
\spacegrad^2 \BH &=
   \lr{ -\omega^2 \epsilon \mu +
   j \omega \sigma \mu } \BH +
   \inv{\mu} \spacegrad \rho_{mv} +
   (j \omega \epsilon + \sigma) \BM
-
   \spacegrad \cross \BJ_i.
\end{aligned}
\end{dmath}

For a lossy medium where \( \epsilon = \epsilon' -j \omega \epsilon'' \), the leading term factor is

\begin{dmath}\label{eqn:emtLecture7:180}
- \omega^2 \mu \epsilon + j \omega \mu \sigma
=
- \omega^2 \mu \epsilon' + j \omega \mu \lr{ \sigma + \omega \epsilon'' }.
\end{dmath}

With the definition
\begin{equation}\label{eqn:emtLecture7:200}
\gamma^2 = \lr{ \alpha + j \beta }^2 = - \omega^2 \mu \epsilon' + j \omega \mu \lr{ \sigma + \omega \epsilon'' },
\end{equation}

the wave equations have the form

\begin{dmath}\label{eqn:emtLecture7:220}
\begin{aligned}
\spacegrad^2 \BE &=
\gamma^2 \BE +
   \inv{\epsilon} \spacegrad \rho_{ev} +
   \spacegrad \cross \BM +
   j \omega \mu \BJ_i \\
\spacegrad^2 \BH &=
\gamma^2 \BH +
   \inv{\mu} \spacegrad \rho_{mv} +
   (j \omega \epsilon + \sigma) \BM
-
   \spacegrad \cross \BJ_i.
\end{aligned}
\end{dmath}

Here

\begin{itemize}
\item \( \alpha \) is the attenuation constant [\si{Np/m}]
\item \( \beta \) is the phase velocity [\si{rad/m}]
\item \( \gamma \) is the propagation constant [\si{1/m}]
\end{itemize}

We are usually interested in solutions in regions free of magnetic currents, induced electric currents, and free of any charge densities, in which case the wave equations are just

\begin{dmath}\label{eqn:emtLecture7:240}
\begin{aligned}
\spacegrad^2 \BE &= \gamma^2 \BE  \\
\spacegrad^2 \BH &= \gamma^2 \BH.
\end{aligned}
\end{dmath}

%\EndArticle
%\EndNoBibArticle

      \section{Problems}
         %
% Copyright � 2016 Peeter Joot.  All Rights Reserved.
% Licenced as described in the file LICENSE under the root directory of this GIT repository.
%
\makeproblem{Meissner effect.}{emt:problemSet6:1}{
The constitutive relation for superconductors in weak magnetic fields can be macroscopically
characterized by the first London equation

\begin{dmath}\label{eqn:emtproblemSet6Problem1:20}
\PD{t}{\BJ_{\mathrm{sup}}} = \alpha \BE,
\end{dmath}

and the second London equation
\begin{dmath}\label{eqn:emtproblemSet6Problem1:40}
\spacegrad \cross \BJ_{\mathrm{sup}} = -\alpha_1 \BB,
\end{dmath}
where \( \BJ_{\mathrm{sup}} \)
stands for the superconducting current,
\( \alpha = n_s q^2 /m \) and \( \alpha_1 \approx \alpha \), with
\( n_s \), \(m\), and \( q\)
denoting, respectively, the number density, the effective mass, and the charge of the Cooper pairs
responsible for the superconductivity in a charged Boson fluid model.

\makesubproblem{}{emt:problemSet6:1a}
From the first London equation, derive and equation for \( \dot{\BB} = \PDi{t}{\BB} \)
by using the static
Maxwell equation \( \spacegrad \cross \BH = \BJ_{\mathrm{sup}} \)
without the displacement current. Show that
\begin{dmath}\label{eqn:emtproblemSet6Problem1:60}
\spacegrad^2 \dot{\BB} = \mu_0 \alpha \dot{\BB}
\end{dmath}
\makesubproblem{}{emt:problemSet6:1b}
From the second London equation and the Ampere's law stated above derive an equation
for \( \BB \).
\makesubproblem{}{emt:problemSet6:1c}
What are the penetration depths in the
\partref{emt:problemSet6:1a}
and
\partref{emt:problemSet6:1b}
cases? Justify your answer.

\paragraph{Remark:} from above analysis we see that both the current and magnetic field are confined to a
thin layer of the order of the penetration depth which is very small. The exclusion of static
magnetic field in a superconductor is known as the Meissner effect experimentally discovered in
1933.
} % makeproblem

\makeanswer{emt:problemSet6:1}{
\makeSubAnswer{}{emt:problemSet6:1a}

Taking the curl of the first London equation \cref{eqn:emtproblemSet6Problem1:20} gives

\begin{dmath}\label{eqn:emtproblemSet6Problem1:80}
\spacegrad \cross \dot{\BJ}_{\mathrm{sup}}
= \alpha \spacegrad \cross \BE
= \alpha \lr{ -\dot{\BB} },
\end{dmath}

or
\begin{dmath}\label{eqn:emtproblemSet6Problem1:100}
\spacegrad \cross \dot{\BJ}_{\mathrm{sup}}
= -\alpha \dot{\BB},
\end{dmath}

which has the same structure as the time derivative of the second London equation, but with \( \alpha \) instead of \( \alpha_1 \).  Taking the curl once more gives

\begin{dmath}\label{eqn:emtproblemSet6Problem1:120}
0
=
\PD{t}{} \lr{
\spacegrad \cross \lr{ \spacegrad \cross \BJ_{\mathrm{sup}} } + \alpha \spacegrad \cross \BB
}
=
\PD{t}{} \lr{
-\spacegrad^2 \BJ_{\mathrm{sup}} + \spacegrad \lr{ \spacegrad \cdot \BJ_{\mathrm{sup}} }
+ \alpha \mu_0 \spacegrad \cross \BH
}
=
\PD{t}{} \lr{
-\spacegrad^2 \BJ_{\mathrm{sup}} + \spacegrad \lr{ \spacegrad \cdot \BJ_{\mathrm{sup}} }
+ \alpha \mu_0 \lr{ \BJ_{\mathrm{sup}} + \cancel{ \PD{t}{\BD} } }
},
\end{dmath}

or
\begin{dmath}\label{eqn:emtproblemSet6Problem1:140}
\alpha \mu_0 \dot{\BJ}_{\mathrm{sup}} = \spacegrad^2 \dot{\BJ}_{\mathrm{sup}} + \spacegrad \lr{ \spacegrad \cdot \dot{\BJ}_{\mathrm{sup}} }.
\end{dmath}

One final application of the curl operator gives
\begin{dmath}\label{eqn:emtproblemSet6Problem1:160}
0
=
-\alpha \mu_0 \spacegrad \cross \dot{\BJ}_{\mathrm{sup}} + \spacegrad \cross \lr{ \spacegrad^2 \dot{\BJ}_{\mathrm{sup}}} + \cancel{\spacegrad \cross \lr{ \spacegrad \lr{ \spacegrad \cdot \dot{\BJ}_{\mathrm{sup}} }}}.
=
-\alpha \mu_0 \lr{ -\alpha \dot{\BB} } + \spacegrad^2 \lr{ \spacegrad \cross \dot{\BJ}_{\mathrm{sup}} }
=
-\alpha \mu_0 \lr{ -\alpha \dot{\BB} } + \spacegrad^2 \lr{ -\alpha \dot{\BB} }
=
-\alpha \lr{ -\mu_0 \alpha \dot{\BB} + \spacegrad^2 \dot{\BB} }.
\end{dmath}

Note that this used
the fact that the curl of a gradient is zero, the fact that the curl and the Laplacian commute (\( \spacegrad^2 \epsilon_{rst} \Be_r \partial_s A_t = \epsilon_{rst} \Be_r \partial_s \spacegrad^2 A_t = \spacegrad \cross ( \spacegrad^2 \BA ) \)), and made two substitutions of
\cref{eqn:emtproblemSet6Problem1:100}
.  This gives the desired result

\begin{dmath}\label{eqn:emtproblemSet6Problem1:180}
\mu_0 \alpha \dot{\BB} = \spacegrad^2 \dot{\BB}.      \qedmarker
\end{dmath}

\makeSubAnswer{}{emt:problemSet6:1b}

Taking the double curl of the second London equation, we have
\begin{dmath}\label{eqn:emtproblemSet6Problem1:200}
\spacegrad \cross \lr{ \spacegrad \cross \lr{ -\alpha_1 \BB } }
=
-\alpha_1 \spacegrad \cross \lr{ \spacegrad \cross \BB }
=
-\alpha_1 \mu_0 \spacegrad \cross \lr{ \spacegrad \cross \BH }
=
-\alpha_1 \mu_0 \spacegrad \cross \lr{ \BJ + \cancel{\PD{t}{\BD}} }
=
-\alpha_1 \mu_0 \lr{ -\alpha_1 \BB }
=
-\alpha_1 \lr{ -\spacegrad^2 \BB + \spacegrad \cancel{ \spacegrad \cdot \BB } },
\end{dmath}

or
\begin{dmath}\label{eqn:emtproblemSet6Problem1:220}
\spacegrad^2 \BB = \alpha_1 \mu_0 \BB.
\end{dmath}

This has the structure of a homogeneous Helmholtz equation \( (\spacegrad^2 + k^2) \BB = 0 \) with an imaginary \( k \).

\makeSubAnswer{}{emt:problemSet6:1c}

The solution of \cref{eqn:emtproblemSet6Problem1:60} is

\begin{dmath}\label{eqn:emtproblemSet6Problem1:240}
\dot{\BB} = \dot{\BB}_0 \exp( \pm \sqrt{\alpha \mu_0} \kcap \cdot \Br ),
\end{dmath}

and the solution of \cref{eqn:emtproblemSet6Problem1:220} is
\begin{dmath}\label{eqn:emtproblemSet6Problem1:260}
\BB = \BB_0 \exp( \pm \sqrt{\alpha_1 \mu_0} \kcap \cdot \Br ),
\end{dmath}

when \( \alpha_1 = \alpha \), a magnetic field solution that satisfies both is

\begin{dmath}\label{eqn:emtproblemSet6Problem1:280}
\BB(\Br, t) = \BB_0 \cos( \omega t ) \exp( \pm \sqrt{\alpha_1 \mu_0} \kcap \cdot \Br ).
\end{dmath}

Anchoring the coordinate system at the boundary of the material, and picking an unbounded solution, requires that at depth \( \delta \) from that surface the exponential goes as

\begin{dmath}\label{eqn:emtproblemSet6Problem1:300}
e^{-\sqrt{\alpha \mu_0} \delta}.
\end{dmath}

The \( e^{-1} \) point is when

\begin{dmath}\label{eqn:emtproblemSet6Problem1:320}
\sqrt{\alpha \mu_0} \delta = 1,
\end{dmath}

or
\begin{dmath}\label{eqn:emtproblemSet6Problem1:340}
\delta
= \inv{\sqrt{\alpha \mu_0} }
= \inv{\sqrt{n_s q^2 \mu_0/m} }.
\end{dmath}

That is
\boxedEquation{eqn:emtproblemSet6Problem1:360}{
\delta
= \sqrt{ \frac{m}{n_s q^2 \mu_0} }.
}

Assuming an electrostatic configuration (where \( \spacegrad \cdot \BJ = 0 \)), the time derivative of the current
in \cref{eqn:emtproblemSet6Problem1:140}
is also seen to be governed by an equation of the form
\cref{eqn:emtproblemSet6Problem1:220}.  This means that both the magnetic field and current are restricted to the same thin layer, with continuing exponential tailoff past the skin depth.
}


   \chapter{Wave equation solutions}

In class, we walked through splitting up the wave equation into components, and separation of variables.  I didn't take notes on that.

Winding down that discussion, however, was a mention of phase and group velocity, and a phenomina called superluminal velocity.  This latter is analogous to quantum electron tunneling where a wave can make it through an aperature with a damped solution \( e^{-\alpha x} \) in the aperature interval, and sinuoidal solutions in the incident and transmitted regions as sketched in \cref{fig:L7:L7Fig1}.  The time \( \tau \) to get through the aperature is called the tunnelling time.

\imageFigure{../../figures/ece1228-emt/L7Fig1}{Superluminal tunneling.}{fig:L7:L7Fig1}{0.3}

      \section{Problems}
          \input{Set6Problem2.tex}
          \input{Set6Problem3.tex}

   \chapter{Wave equation solutions}
      %
% Copyright � 2016 Peeter Joot.  All Rights Reserved.
% Licenced as described in the file LICENSE under the root directory of this GIT repository.
%
\newcommand{\authorname}{Peeter Joot}
\newcommand{\email}{peeterjoot@protonmail.com}
\newcommand{\basename}{FIXMEbasenameUndefined}
\newcommand{\dirname}{notes/FIXMEdirnameUndefined/}

\renewcommand{\basename}{emt8}
\renewcommand{\dirname}{notes/ece1228/}
\newcommand{\keywords}{ECE1228H}
\newcommand{\authorname}{Peeter Joot}
\newcommand{\onlineurl}{http://sites.google.com/site/peeterjoot2/math2013/\basename.pdf}
\newcommand{\sourcepath}{\dirname\basename.tex}
\newcommand{\generatetitle}[1]{\chapter{#1}}

\newcommand{\vcsinfo}{%
\section*{}
\noindent{\color{DarkOliveGreen}{\rule{\linewidth}{0.1mm}}}
\paragraph{Document version}
%\paragraph{\color{Maroon}{Document version}}
{
\small
\begin{itemize}
\item Available online at:\\ 
\href{\onlineurl}{\onlineurl}
\item Git Repository: \input{./.revinfo/gitRepo.tex}
\item Source: \sourcepath
\item last commit: \input{./.revinfo/gitCommitString.tex}
\item commit date: \input{./.revinfo/gitCommitDate.tex}
\end{itemize}
}
}

%\PassOptionsToPackage{dvipsnames,svgnames}{xcolor}
\PassOptionsToPackage{square,numbers}{natbib}
\documentclass{scrreprt}

\usepackage[left=2cm,right=2cm]{geometry}
\usepackage[svgnames]{xcolor}
\usepackage{peeters_layout}

\usepackage{natbib}

\usepackage[
colorlinks=true,
bookmarks=false,
pdfauthor={\authorname, \email},
backref 
]{hyperref}

% http://tex.stackexchange.com/questions/75773/how-to-reference-problems-by-the-text-label-in-an-exercise-envioronment
\usepackage[english]{cleveref}
\crefname{Exercise}{exercise}{exercises}
\Crefname{Exercise}{Exercise}{Exercises}

\RequirePackage{titlesec}
\RequirePackage{ifthen}

% http://stackoverflow.com/questions/4932910/date-in-the-tabular-environment
\makeatletter
\let\insertdate\@date
\makeatother

\titleformat{\chapter}[display]
{\bfseries\Large}
{\color{DarkSlateGrey}\filleft \authorname
\ifthenelse{\isundefined{\studentnumber}}{}{\\ \studentnumber}
\ifthenelse{\isundefined{\email}}{}{\\ \email}
\ifthenelse{\isundefined{\dateintitle}}{}{\\ \insertdate}
%\ifthenelse{\isundefined{\coursename}}{}{\\ \coursename} % put in title instead.
}
{4ex}
{\color{DarkOliveGreen}{\titlerule}\color{Maroon}
\vspace{2ex}%
\filright}
[\vspace{2ex}%
\color{DarkOliveGreen}\titlerule
]

\newcommand{\beginArtWithToc}[0]{\begin{document}\tableofcontents}
\newcommand{\beginArtNoToc}[0]{\begin{document}}
\newcommand{\EndNoBibArticle}[0]{\end{document}}
\newcommand{\EndArticle}[0]{\bibliography{Bibliography}\bibliographystyle{plainnat}\end{document}}

% 
%\newcommand{\citep}[1]{\cite{#1}}

\colorSectionsForArticle



%\usepackage{ece1228}
\usepackage{peeters_braket}
%\usepackage{peeters_layout_exercise}
\usepackage{peeters_figures}
\usepackage{mathtools}
\usepackage{siunitx}

\beginArtNoToc
\generatetitle{ECE1228H Electromagnetic Theory.  Lecture 8: Waves.  Taught by Prof.\ M. Mojahedi}
%\chapter{Waves}
\label{chap:emt8}

%\paragraph{Disclaimer}
%
%Peeter's lecture notes from class.  These may be incoherent and rough.
%
%These are notes for the UofT course ECE1228H, Electromagnetic Theory, taught by Prof. M. Mojahedi, covering \textchapref{{1}} \citep{balanis1989advanced} content.

\paragraph{Cylindrical coorindate wave equation solutions}

Seek a function

\begin{dmath}\label{eqn:emtLecture8:20}
\BE = E_\rho \rhocap + E_\phi \phicap + E_z \zcap
\end{dmath}

solving 

\begin{dmath}\label{eqn:emtLecture8:40}
\spacegrad^2 \BE = -\beta^2 \BE.
\end{dmath}

One way to find the Laplacian in cylindrical coordinates is to use

\begin{dmath}\label{eqn:emtLecture8:60}
\spacegrad^2 \BE = 
\spacegrad \lr{ \spacegrad \cdot \BE }
-\spacegrad \cross \lr{ \spacegrad \cross \BE },
\end{dmath}

where 

\begin{dmath}\label{eqn:emtLecture8:80}
\spacegrad = \rhocap \PD{\rho}{} + \frac{\phicap}{\rho} \PD{\phi}{} + \zcap \PD{z}{}
\end{dmath}

Can be shown that:
\begin{dmath}\label{eqn:emtLecture8:100}
\spacegrad \cdot \BE = \inv{\rho} \PD{\rho}{} \lr{ \rho E_\rho } + \inv{\rho}\PD{\phi}{E_\phi} + \PD{z}{E_z}
\end{dmath}

and
\begin{dmath}\label{eqn:emtLecture8:120}
\spacegrad \cross \BE 
%= 
%\begin{vmatrix}
%\rhocap & \phicap & \zcap \\
%\partial_\rho & \inv{\rho}\partial_\phi & \partial_z \\
%E_\rho & \rho E_\phi & E_z
%\end{vmatrix}
=
\rhocap  \lr{ \inv{\rho} \partial_\phi E_z - \partial_z E_\phi }
+\phicap \lr{ \partial_z E_\rho - \partial_\rho E_z }
+\zcap   \lr{ \inv{\rho} \partial_\rho (\rho E_\phi) - \inv{\rho} \partial_\phi E_\rho }
\end{dmath}

This gives
\begin{dmath}\label{eqn:emtLecture8:200}
\spacegrad^2 \psi = 
\PDSq{\rho}{\psi}
+\inv{\rho} \PD{\rho}{\psi}
+\inv{\rho^2} \PDSq{\phi}{\psi}
+\PDSq{z}{\psi}.
\end{dmath}

and
\begin{dmath}\label{eqn:emtLecture8:220}
\begin{aligned}
\spacegrad^2 E_\rho &= \lr{ -\frac{E_\rho}{\rho^2} - \frac{2}{\rho^2} \PD{\phi}{E_\phi} } \\
\spacegrad^2 E_\phi &= \lr{ -\frac{E_\phi}{\rho^2} + \frac{2}{\rho^2} \PD{\phi}{E_\rho} } \\
\spacegrad^2 E_z    &= -\beta^2 E_\phi.
\end{aligned}
\end{dmath}

%Note that with \( i = \Be_1 \Be_2 \),
%
%\begin{dmath}\label{eqn:emtLecture8:140}
%\rhocap = \Be_1 e^{i \phi}
%\end{dmath}
%
%so
%\begin{equation}\label{eqn:emtLecture8:160}
%\PD{\phi}{\rhocap} = \Be_2 e^{i \phi} = \thetacap
%\end{equation}
%
%... the end result is
%
%\begin{dmath}\label{eqn:emtLecture8:180}
%\end{dmath}

\paragraph{TEM:} If we want to have a TEM mode it can be shown that we need an axial distribution mechanism, such as the core of a co-axial cable.

These are messy to solve in general, but we can solve the z-component without too much pain

\begin{dmath}\label{eqn:emtLecture8:240}
\PDSq{\rho}{E_z}
+\inv{\rho} \PD{\rho}{E_z}
+\inv{\rho^2} \PDSq{\phi}{E_z}
+\PDSq{z}{E_z}
=
-\beta^2 E_z
\end{dmath}

Solving this using separation of variables with

\begin{dmath}\label{eqn:emtLecture8:260}
E_z = R(\rho) P(\phi) Z(z)
\end{dmath}

\begin{dmath}\label{eqn:emtLecture8:280}
\inv{R}\lr{R'' + \inv{\rho} R'} + \inv{\rho^2 P} P'' + \frac{Z''}{Z} = -\beta^2
\end{dmath}

Assuming for some constant \( \beta_z \) that we have
\begin{dmath}\label{eqn:emtLecture8:300}
\frac{Z''}{Z} = -\beta_z^2,
\end{dmath}

then

\begin{dmath}\label{eqn:emtLecture8:320}
\inv{R}\lr{\rho^2 R'' + \rho R'} + \inv{P} P'' + \rho^2 \lr{\beta^2 - \beta_z^2} = 0
\end{dmath}

Now assume that
\begin{dmath}\label{eqn:emtLecture8:340}
\inv{P} P'' = -m^2,
\end{dmath}

and let \( \beta^2 - \beta_z^2 = \beta_\rho^2 \), which leaves

\begin{dmath}\label{eqn:emtLecture8:360}
\rho^2 R'' + \rho R' + \lr{ \rho^2 \beta_\rho^2 -m^2 } R = 0.
\end{dmath}

This is the Bessel differential equation, with travelling wave solution

\begin{dmath}\label{eqn:emtLecture8:380}
R(\rho) = 
A H_m^{(1)}(\beta_\rho \rho) 
+B H_m^{(2)}(\beta_\rho \rho),
\end{dmath}

and standing wave solutions
\begin{dmath}\label{eqn:emtLecture8:400}
R(\rho) = 
A J_m(\beta_\rho \rho)
+B Y_m(\beta_\rho \rho).
\end{dmath}

Here \( H_m^{(1)}, H_m^{(2)} \) are Hankel functions of the first and second kinds, and
\( J_m, Y_m \) are the Bessel functions of the first and second kinds.

For \( P(\phi) \) 
\begin{dmath}\label{eqn:emtLecture8:460}
P'' = -m^2 P
\end{dmath}

\paragraph{Quadropole potential}

In Jackson
\citep{jackson1975cew}
, the prove of which is scattered through chapter 3, is the following

\begin{dmath}\label{eqn:emtLecture8:420}
\inv{\Abs{\Bx - \Bx'}}
= 
4 \pi \sum_{l= 0}^\infty \sum_{m = -l}^l \inv{2 l + 1} \frac{(r')^l}{r^{l+1}} 
Y^\conj_{l,m}(\theta', \phi')
Y_{l,m}(\theta, \phi),
\end{dmath}

where \( Y_{l,m} \) are the spherical harmonics.  Plugging this into the potential we have

\begin{dmath}\label{eqn:emtLecture8:440}
\phi(\Bx) 
= \inv{4 \pi \epsilon_0} \int \frac{\rho(\Bx') d^3 x'}{\Abs{\Bx - \Bx'}}
= 
\inv{4 \pi \epsilon_0} \int \rho(\Bx') d^3 x' \lr{
4 \pi \sum_{l= 0}^\infty \sum_{m = -l}^l \inv{2 l + 1} \frac{(r')^l}{r^{l+1}} 
Y^\conj_{l,m}(\theta', \phi')
Y_{l,m}(\theta, \phi)
}
= 
\inv{\epsilon_0} 
\sum_{l= 0}^\infty \sum_{m = -l}^l \inv{2 l + 1} 
\int \rho(\Bx') d^3 x' \lr{
\frac{(r')^l}{r^{l+1}} 
Y^\conj_{l,m}(\theta', \phi')
Y_{l,m}(\theta, \phi)
}
= 
\inv{\epsilon_0} 
\sum_{l= 0}^\infty \sum_{m = -l}^l \inv{2 l + 1} 
\lr{ 
\int (r')^l \rho(\Bx') 
Y^\conj_{l,m}(\theta', \phi')
d^3 x' 
}
\frac{
Y_{l,m}(\theta, \phi)
}
{
r^{l+1}
}
\end{dmath}

The integral terms are called the coefficients of the multipole moments, denoted
\begin{dmath}\label{eqn:emtLecture8:480}
q_{l,m} = 
\int (r')^l \rho(\Bx') 
Y^\conj_{l,m}(\theta', \phi')
d^3 x',
\end{dmath}

The \( l = 0,1,2\) terms are, respectively, called the monopole, dipole, and quadropole terms of the potential
\begin{dmath}\label{eqn:emtLecture8:500}
\rho(\Bx) =
\inv{4 \pi \epsilon_0} 
\sum_{l= 0}^\infty \sum_{m = -l}^l \frac{4\pi} {2 l + 1} 
q_{l,m}
\frac{
Y_{l,m}(\theta, \phi)
}
{
r^{l+1}
}.
\end{dmath}

Note the power of this expansion.  Should we wish to compute the electric field, we have only to compute the qradient of  the last (\(Y_{l,m} r^{-l-1} \)) portion (since \( q_{l,m} \) is a constant).
 
\begin{dmath}\label{eqn:emtLecture8:520}
q_{1,1}
= 
-\int \sqrt{\frac{3}{8 \pi}} \sin\theta' e^{-i\phi'} r' \rho(\Bx') dV'
=
-\sqrt{\frac{3}{8 \pi}} \int \sin\theta' \lr{ \cos\phi' - i\sin\phi'} r' \rho(\Bx') dV'
=
-\sqrt{\frac{3}{8 \pi}} \lr{ 
\int x' \rho(\Bx') dV'
-i \int y' \rho(\Bx') dV'
}
=
-\sqrt{\frac{3}{8 \pi}} \lr{ 
p_x - i p_y
}.
\end{dmath}

Here we've used
\begin{dmath}\label{eqn:emtLecture8:540}
\begin{aligned}
x' &= r' \sin\theta' \cos\phi' \\
y' &= r' \sin\theta' \sin\phi' \\
z' &= r' \cos\theta'
\end{aligned}
\end{dmath}

and the \( Y_{11} \) representation

\begin{dmath}\label{eqn:emtLecture8:560}
\begin{aligned}
Y_{00} &= -\sqrt{\frac{1}{4 \pi}} \\
Y_{11} &= -\sqrt{\frac{3}{8 \pi}} \sin\theta e^{i\phi} \\
Y_{10} &=  \sqrt{\frac{3}{4 \pi}} \cos\theta  \\
Y_{22} &= -\inv{4} \sqrt{\frac{15}{2 \pi}} \sin^2\theta e^{2 i\phi} \\
Y_{21} &=  \inv{2} \sqrt{\frac{15}{2 \pi}} \sin\theta \cos\theta e^{i\phi} \\
Y_{20} &=  \inv{4} \sqrt{\frac{5}{\pi}} \lr{ 3 \cos^2\theta - 1 } \\
\end{aligned}
\end{dmath}

%HW: compute a few of the more tedious moment coeffients.  These have been exam questions in the past.

With the usual dipole moment expression

\begin{dmath}\label{eqn:emtLecture8:580}
\Bp = \int \Bx' \rho(\Bx') d^3 x',
\end{dmath}

and a quadropole moment defined as
\begin{dmath}\label{eqn:emtLecture8:600}
Q_{i,j} = \int \lr{ 3 x_i' x_j' - \delta_{ij} (r')^2 } \rho(\Bx') d^3 x',
\end{dmath}

the first order terms of the potential are now fully specified
\begin{dmath}\label{eqn:emtLecture8:620}
\phi(\Bx)
=
\inv{4 \pi \epsilon_0}
\lr{ 
q + \frac{\Bp \cdot \Bx}{r^3} + 
\inv{2} \sum_{ij} Q_{ij} \frac{x_i x_j}{r^5}
}.
\end{dmath}

\paragraph{Waves}

\begin{itemize}
\item The field is a modification of space-time
\item Mode is a particular field configuration for a given boundary value problem.  Many field configurations can satisfy Maxwell equations (wave equation).  These usually are referrred to as modes.  A mode is a self-consistent field distribution.
\item In a TEM mode, \( \BE \) and \( \BH \) are every point in space are constrained in a local plane, independent of time.  This plane is called the equiphase plane.  In general equiphase planes are not parallel at two different points along the trajectory of the wave.
\item If equiphase planes are parallel (i.e. the space orientation of the planes for TEM mode...
... next time.
\end{itemize}

%}
\EndArticle
%\EndNoBibArticle

      \section{Problems}
         \input{Set7Problem2.tex}
         \input{Set7Problem3.tex}

   \chapter{Quadrupole expansion}
      %
% Copyright © 2016 Peeter Joot.  All Rights Reserved.
% Licenced as described in the file LICENSE under the root directory of this GIT repository.
%

\paragraph{Quadropole potential}

In Jackson
\citep{jackson1975cew}
,
is the following

\begin{dmath}\label{eqn:emtLecture8:420}
\inv{\Abs{\Bx - \Bx'}}
=
4 \pi \sum_{l= 0}^\infty \sum_{m = -l}^l \inv{2 l + 1} \frac{(r')^l}{r^{l+1}}
Y^\conj_{l,m}(\theta', \phi')
Y_{l,m}(\theta, \phi),
\end{dmath}

where \( Y_{l,m} \) are the spherical harmonics.  It appears that this is actually just an orthogonal function expansion of the inverse distance (for a region outside of the charge density).  The proof of this in is scattered through chapter 3, dependent on a similar expansion in Legendre polynomials, for an the azimuthally symmetric configuration.

It looks like quite a project to get comfortable enough with these special functions to fully reproduce the proof of this identity.  We are forced to play engineer, and assume the mathematics works out.  If we do that and plug this inverse distance formula into
the potential we have

\begin{dmath}\label{eqn:emtLecture8:440}
\phi(\Bx)
= \inv{4 \pi \epsilon_0} \int \frac{\rho(\Bx') d^3 x'}{\Abs{\Bx - \Bx'}}
=
\inv{4 \pi \epsilon_0} \int \rho(\Bx') d^3 x' \lr{
4 \pi \sum_{l= 0}^\infty \sum_{m = -l}^l \inv{2 l + 1} \frac{(r')^l}{r^{l+1}}
Y^\conj_{l,m}(\theta', \phi')
Y_{l,m}(\theta, \phi)
}
=
\inv{\epsilon_0}
\sum_{l= 0}^\infty \sum_{m = -l}^l \inv{2 l + 1}
\int \rho(\Bx') d^3 x' \lr{
\frac{(r')^l}{r^{l+1}}
Y^\conj_{l,m}(\theta', \phi')
Y_{l,m}(\theta, \phi)
}
=
\inv{\epsilon_0}
\sum_{l= 0}^\infty \sum_{m = -l}^l \inv{2 l + 1}
\lr{
\int (r')^l \rho(\Bx')
Y^\conj_{l,m}(\theta', \phi')
d^3 x'
}
\frac{
Y_{l,m}(\theta, \phi)
}
{
r^{l+1}
}
\end{dmath}

The integral terms are called the coefficients of the multipole moments, denoted
\begin{dmath}\label{eqn:emtLecture8:480}
q_{l,m} =
\int (r')^l \rho(\Bx')
Y^\conj_{l,m}(\theta', \phi')
d^3 x',
\end{dmath}

The \( l = 0,1,2\) terms are, respectively, called the monopole, dipole, and quadropole terms of the potential
\begin{dmath}\label{eqn:emtLecture8:500}
\rho(\Bx) =
\inv{4 \pi \epsilon_0}
\sum_{l= 0}^\infty \sum_{m = -l}^l \frac{4\pi} {2 l + 1}
q_{l,m}
\frac{
Y_{l,m}(\theta, \phi)
}
{
r^{l+1}
}.
\end{dmath}

Note the power of this expansion.  Should we wish to compute the electric field, we have only to compute the qradient of  the last (\(Y_{l,m} r^{-l-1} \)) portion (since \( q_{l,m} \) is a constant).

\begin{dmath}\label{eqn:emtLecture8:520}
q_{1,1}
=
-\int \sqrt{\frac{3}{8 \pi}} \sin\theta' e^{-i\phi'} r' \rho(\Bx') dV'
=
-\sqrt{\frac{3}{8 \pi}} \int \sin\theta' \lr{ \cos\phi' - i\sin\phi'} r' \rho(\Bx') dV'
=
-\sqrt{\frac{3}{8 \pi}} \lr{
\int x' \rho(\Bx') dV'
-i \int y' \rho(\Bx') dV'
}
=
-\sqrt{\frac{3}{8 \pi}} \lr{
p_x - i p_y
}.
\end{dmath}

Here we've used
\begin{dmath}\label{eqn:emtLecture8:540}
\begin{aligned}
x' &= r' \sin\theta' \cos\phi' \\
y' &= r' \sin\theta' \sin\phi' \\
z' &= r' \cos\theta'
\end{aligned}
\end{dmath}

and the \( Y_{11} \) representation

\begin{dmath}\label{eqn:emtLecture8:560}
\begin{aligned}
Y_{00} &= -\sqrt{\frac{1}{4 \pi}} \\
Y_{11} &= -\sqrt{\frac{3}{8 \pi}} \sin\theta e^{i\phi} \\
Y_{10} &=  \sqrt{\frac{3}{4 \pi}} \cos\theta  \\
Y_{22} &= -\inv{4} \sqrt{\frac{15}{2 \pi}} \sin^2\theta e^{2 i\phi} \\
Y_{21} &=  \inv{2} \sqrt{\frac{15}{2 \pi}} \sin\theta \cos\theta e^{i\phi} \\
Y_{20} &=  \inv{4} \sqrt{\frac{5}{\pi}} \lr{ 3 \cos^2\theta - 1 } \\
\end{aligned}
\end{dmath}

%NOTE: compute a few of the more tedious moment coeffients.  These have been exam questions in the past.

With the usual dipole moment expression

\begin{dmath}\label{eqn:emtLecture8:580}
\Bp = \int \Bx' \rho(\Bx') d^3 x',
\end{dmath}

and a quadropole moment defined as
\begin{dmath}\label{eqn:emtLecture8:600}
Q_{i,j} = \int \lr{ 3 x_i' x_j' - \delta_{ij} (r')^2 } \rho(\Bx') d^3 x',
\end{dmath}

the first order terms of the potential are now fully specified
\begin{dmath}\label{eqn:emtLecture8:620}
\phi(\Bx)
=
\inv{4 \pi \epsilon_0}
\lr{
q + \frac{\Bp \cdot \Bx}{r^3} +
\inv{2} \sum_{ij} Q_{ij} \frac{x_i x_j}{r^5}
}.
\end{dmath}

      %
% Copyright � 2016 Peeter Joot.  All Rights Reserved.
% Licenced as described in the file LICENSE under the root directory of this GIT repository.
%
%{
%\newcommand{\authorname}{Peeter Joot}
\newcommand{\email}{peeterjoot@protonmail.com}
\newcommand{\basename}{FIXMEbasenameUndefined}
\newcommand{\dirname}{notes/FIXMEdirnameUndefined/}

%\renewcommand{\basename}{momentCoeffiecients}
%%\renewcommand{\dirname}{notes/phy1520/}
%\renewcommand{\dirname}{notes/ece1228-electromagnetic-theory/}
%%\newcommand{\dateintitle}{}
%%\newcommand{\keywords}{}
%
%\newcommand{\authorname}{Peeter Joot}
\newcommand{\onlineurl}{http://sites.google.com/site/peeterjoot2/math2013/\basename.pdf}
\newcommand{\sourcepath}{\dirname\basename.tex}
\newcommand{\generatetitle}[1]{\chapter{#1}}

\newcommand{\vcsinfo}{%
\section*{}
\noindent{\color{DarkOliveGreen}{\rule{\linewidth}{0.1mm}}}
\paragraph{Document version}
%\paragraph{\color{Maroon}{Document version}}
{
\small
\begin{itemize}
\item Available online at:\\ 
\href{\onlineurl}{\onlineurl}
\item Git Repository: \input{./.revinfo/gitRepo.tex}
\item Source: \sourcepath
\item last commit: \input{./.revinfo/gitCommitString.tex}
\item commit date: \input{./.revinfo/gitCommitDate.tex}
\end{itemize}
}
}

%\PassOptionsToPackage{dvipsnames,svgnames}{xcolor}
\PassOptionsToPackage{square,numbers}{natbib}
\documentclass{scrreprt}

\usepackage[left=2cm,right=2cm]{geometry}
\usepackage[svgnames]{xcolor}
\usepackage{peeters_layout}

\usepackage{natbib}

\usepackage[
colorlinks=true,
bookmarks=false,
pdfauthor={\authorname, \email},
backref 
]{hyperref}

% http://tex.stackexchange.com/questions/75773/how-to-reference-problems-by-the-text-label-in-an-exercise-envioronment
\usepackage[english]{cleveref}
\crefname{Exercise}{exercise}{exercises}
\Crefname{Exercise}{Exercise}{Exercises}

\RequirePackage{titlesec}
\RequirePackage{ifthen}

% http://stackoverflow.com/questions/4932910/date-in-the-tabular-environment
\makeatletter
\let\insertdate\@date
\makeatother

\titleformat{\chapter}[display]
{\bfseries\Large}
{\color{DarkSlateGrey}\filleft \authorname
\ifthenelse{\isundefined{\studentnumber}}{}{\\ \studentnumber}
\ifthenelse{\isundefined{\email}}{}{\\ \email}
\ifthenelse{\isundefined{\dateintitle}}{}{\\ \insertdate}
%\ifthenelse{\isundefined{\coursename}}{}{\\ \coursename} % put in title instead.
}
{4ex}
{\color{DarkOliveGreen}{\titlerule}\color{Maroon}
\vspace{2ex}%
\filright}
[\vspace{2ex}%
\color{DarkOliveGreen}\titlerule
]

\newcommand{\beginArtWithToc}[0]{\begin{document}\tableofcontents}
\newcommand{\beginArtNoToc}[0]{\begin{document}}
\newcommand{\EndNoBibArticle}[0]{\end{document}}
\newcommand{\EndArticle}[0]{\bibliography{Bibliography}\bibliographystyle{plainnat}\end{document}}

% 
%\newcommand{\citep}[1]{\cite{#1}}

\colorSectionsForArticle


%
%\usepackage{peeters_layout_exercise}
%\usepackage{peeters_braket}
%\usepackage{peeters_figures}
%\usepackage{siunitx}
%%\usepackage{txfonts} % \ointclockwise
%
%\beginArtNoToc
%
%\generatetitle{Dipole and Quadrupole electrostatic potential moments and coefficients}
%\chapter{Dipole and Quadropole electrostatic potential moments and coefficents}
%\label{chap:momentCoeffiecients}

\paragraph{Explicit moment and quadrupole expansion}

We calculated the \( q_{1,1} \) coefficient of the electrostatic moment, as covered in \citep{jackson1975cew} chapter 4.  Let's verify the rest, as well as the tensor sum formula for the quadrupole moment, and the spherical harmonic sum that yields the dipole moment potential.

%%%XX
%%%\begin{dmath}\label{eqn:momentCoeffiecients:20}
%%%q_{l,m} =
%%%\int (r')^l \rho(\Bx')
%%%Y^\conj_{l,m}(\theta', \phi')
%%%d^3 x',
%%%\end{dmath}
%%%
%%%The class notes also give the results for \( q_{0,0}, q_{1,0}, q_{2,2}, q_{2,1}, q_{2,0} \).  Let's verify those
%%%
%%%\paragraph{\(q_{0,0}\)}
%%%
%%%\begin{dmath}\label{eqn:momentCoeffiecients:40}
%%%q_{0,0}
%%%=
%%%\int (r')^0 \rho(\Bx')
%%%Y^\conj_{0,0}(\theta', \phi')
%%%d^3 x'
%%%=
%%%\inv{4\pi}
%%%\int \rho(\Bx') d^3 x'
%%%=
%%%\frac{q}{4\pi}
%%%\end{dmath}
%%%
%%%\paragraph{\(q_{1,0}\)}
%%%
%%%\begin{dmath}\label{eqn:momentCoeffiecients:60}
%%%q_{1,0}
%%%=
%%%\int r' \rho(\Bx')
%%%Y^\conj_{1,0}(\theta', \phi')
%%%d^3 x'
%%%=
%%%\sqrt{\frac{3}{4\pi}}
%%%\int r' \rho(\Bx')
%%%\cos\theta'
%%%d^3 x'
%%%=
%%%\sqrt{\frac{3}{4\pi}}
%%%\int r' \rho(\Bx') \cos\theta' d^3 x'
%%%=
%%%\sqrt{\frac{3}{4\pi}}
%%%\int z' \rho(\Bx') d^3 x'
%%%=
%%%\sqrt{\frac{3}{4\pi}} p_z
%%%\end{dmath}
%%%
%%%\paragraph{\(q_{2,2}\)}
%%%
%%%\begin{dmath}\label{eqn:momentCoeffiecients:80}
%%%q_{2,2}
%%%=
%%%\int (r')^2 \rho(\Bx')
%%%Y^\conj_{2,2}(\theta', \phi')
%%%d^3 x'
%%%=
%%%\sqrt{\frac{15}{32 \pi}}
%%%\int (r')^2 \rho(\Bx')
%%%\sin^2 \theta e^{-2 i\phi}
%%%d^3 x'
%%%=
%%%\sqrt{\frac{15}{32 \pi}}
%%%\int (r')^2 \rho(\Bx')
%%%\sin^2 \theta \lr{ \cos \phi - i \sin\phi }^2
%%%d^3 x'
%%%=
%%%\sqrt{\frac{15}{32 \pi}}
%%%\int (r')^2 \rho(\Bx')
%%%\sin^2 \theta \lr{ \cos^2\phi - \sin^2\phi - 2 i \cos\phi \sin\phi }
%%%d^3 x'
%%%=
%%%\sqrt{\frac{15}{32 \pi}}
%%%\int \rho(\Bx') \lr{
%%%(x')^2
%%%- (y')^2
%%%- 2 i x' y' } d^3 x'
%%%=
%%%\sqrt{\frac{15}{32 \pi}}
%%%\int \rho(\Bx') \lr{
%%%x' - i y'
%%%}^2 d^3 x'
%%%%=
%%%%\sqrt{\frac{15}{32 \pi}}
%%%%\lr{ p_x - i p_y }^2
%%%\end{dmath}
%%%
%%%\paragraph{\(q_{2,1}\)}
%%%
%%%\begin{dmath}\label{eqn:momentCoeffiecients:100}
%%%q_{2,1}
%%%=
%%%\int (r')^2 \rho(\Bx')
%%%Y^\conj_{2,1}(\theta', \phi')
%%%d^3 x'
%%%=
%%%-\sqrt{\frac{15}{8 \pi}}
%%%\int (r')^2 \rho(\Bx')
%%%\sin\theta' \cos\theta' e^{-i \phi}
%%%d^3 x'
%%%=
%%%-\sqrt{\frac{15}{8 \pi}}
%%%\int (r')^2 \rho(\Bx')
%%%\sin\theta' \cos\theta' \lr{ \cos\phi - i \sin\phi }
%%%d^3 x'
%%%=
%%%-\sqrt{\frac{15}{8 \pi}}
%%%\int \rho(\Bx')
%%%\lr{ x' z' - i y' z' }
%%%d^3 x'
%%%%=
%%%%-\sqrt{\frac{15}{8 \pi}} p_z \lr{ p_x - i p_y }.
%%%\end{dmath}
%%%
%%%\paragraph{\(q_{2,0}\)}
%%%
%%%\begin{dmath}\label{eqn:momentCoeffiecients:260}
%%%q_{2,0}
%%%=
%%%\int (r')^2 \rho(\Bx')
%%%Y^\conj_{2,0}(\theta', \phi')
%%%d^3 x'
%%%=
%%%\int (r')^2 \rho(\Bx') \sqrt{\frac{5}{4\pi}} \lr{ \frac{3}{2} \cos^2\theta - \inv{2} }
%%%d^3 x'
%%%=
%%%\inv{2} \sqrt{\frac{5}{4\pi}}
%%%\int \rho(\Bx')
%%%\lr{ 3 (z')^2 - (r')^2 }
%%%d^3 x'.
%%%\end{dmath}
%%%
%%%\paragraph{\(Q_{ij}\)}
%%%XX
The quadrupole term of the potential was stated to be

\begin{dmath}\label{eqn:momentCoeffiecients:120}
\inv{4 \pi \epsilon_0} \frac{4 \pi}{5 r^3} \sum_{m=-2}^2 \int (r')^2 \rho(\Bx') Y_{lm}^\conj(\theta', \phi') Y_{lm}(\theta, \phi)
=
\inv{2} \sum_{ij} Q_{ij} \frac{x_i x_j}{r^5},
\end{dmath}

where

\begin{dmath}\label{eqn:momentCoeffiecients:140}
Q_{i,j} = \int \lr{ 3 x_i' x_j' - \delta_{ij} (r')^2 } \rho(\Bx') d^3 x'.
\end{dmath}

Let's verify this.  First note that

\begin{dmath}\label{eqn:momentCoeffiecients:160}
Y_{l,m} = \sqrt{\frac{2 l + 1}{4 \pi} \frac{(l-m)!}{(l+m)!}} P_l^m(\cos\theta) e^{i m \phi},
\end{dmath}

and
\begin{dmath}\label{eqn:momentCoeffiecients:180}
P_l^{-m}(x) =
(-1)^m \frac{(l-m)!}{(l+m)!} P_l^m(x),
\end{dmath}

so
\begin{dmath}\label{eqn:momentCoeffiecients:200}
Y_{l,-m}
= \sqrt{\frac{2 l + 1}{4 \pi} \frac{(l+m)!}{(l-m)!} }
P_l^{-m}(\cos\theta)
e^{-i m \phi}
=
(-1)^m
\sqrt{\frac{2 l + 1}{4 \pi} \frac{(l-m)!}{(l+m)!} }
P_l^m(x)
e^{-i m \phi}
=
(-1)^m Y_{l,m}^\conj.
\end{dmath}

That means

\begin{dmath}\label{eqn:momentCoeffiecients:220}
q_{l,-m}
=
\int (r')^l \rho(\Bx')
Y^\conj_{l,-m}(\theta', \phi')
d^3 x'
=
(-1)^m
\int (r')^l \rho(\Bx')
Y_{l,m}(\theta', \phi')
d^3 x'
=
(-1)^m q_{lm}^\conj.
\end{dmath}

In particular, for \( m \ne 0 \)

\begin{dmath}\label{eqn:momentCoeffiecients:320}
(r')^l Y_{l, m}^\conj (\theta', \phi') r^l Y_{l, m}(\theta, \phi)
+ (r')^l Y_{l, -m}^\conj (\theta', \phi') r^l Y_{l, -m}(\theta, \phi)
=
(r')^l Y_{l, m}^\conj (\theta', \phi') r^l Y_{l, m}(\theta, \phi)
+ (r')^l Y_{l, m} (\theta', \phi') r^l Y_{l, m}^\conj(\theta, \phi) ,
\end{dmath}

or
\begin{dmath}\label{eqn:momentCoeffiecients:340}
(r')^l Y_{l, m}^\conj (\theta', \phi') r^l Y_{l, m}(\theta, \phi)
+ (r')^l Y_{l, -m}^\conj (\theta', \phi') r^l Y_{l, -m}(\theta, \phi)
=
2 \Real \lr{ (r')^l Y_{l, m}^\conj (\theta', \phi') r^l Y_{l, m}(\theta, \phi) }.
\end{dmath}

To verify the quadrupole expansion formula in a compact way it is helpful to compute some intermediate results.

\begin{dmath}\label{eqn:momentCoeffiecients:360}
r Y_{1, 1}
= -r \sqrt{\frac{3}{8 \pi}} \sin\theta e^{i\phi}
= -\sqrt{\frac{3}{8 \pi}} (x + i y),
\end{dmath}

\begin{dmath}\label{eqn:momentCoeffiecients:380}
r Y_{1, 0}
= r \sqrt{\frac{3}{4 \pi}} \cos\theta
= \sqrt{\frac{3}{4 \pi}} z,
\end{dmath}

\begin{dmath}\label{eqn:momentCoeffiecients:400}
r^2 Y_{2, 2}
= -r^2 \sqrt{\frac{15}{32 \pi}} \sin^2\theta e^{2 i\phi}
= - \sqrt{\frac{15}{32 \pi}} (x + i y)^2,
\end{dmath}

\begin{dmath}\label{eqn:momentCoeffiecients:420}
r^2 Y_{2, 1}
= r^2 \sqrt{\frac{15}{8 \pi}} \sin\theta \cos\theta e^{i\phi}
= \sqrt{\frac{15}{8 \pi}} z ( x + i y ),
\end{dmath}

\begin{dmath}\label{eqn:momentCoeffiecients:440}
r^2 Y_{2, 0}
= r^2 \sqrt{\frac{5}{16 \pi}} \lr{ 3 \cos^2\theta - 1 }
= \sqrt{\frac{5}{16 \pi}} \lr{ 3 z^2 - r^2 }.
\end{dmath}

Given primed coordinates and integrating the conjugate of each of these with \( \rho(\Bx') dV' \), we obtain the \( q_{lm} \) moment coefficients.  Those are

\begin{dmath}\label{eqn:momentCoeffiecients:460}
q_{11}
= -\sqrt{\frac{3}{8 \pi}} \int d^3 x' \rho(\Bx') (x - i y),
\end{dmath}

\begin{dmath}\label{eqn:momentCoeffiecients:480}
q_{1, 0}
= \sqrt{\frac{3}{4 \pi}} \int d^3 x' \rho(\Bx') z',
\end{dmath}

\begin{dmath}\label{eqn:momentCoeffiecients:500}
q_{2, 2}
= - \sqrt{\frac{15}{32 \pi}} \int d^3 x' \rho(\Bx') (x' - i y')^2,
\end{dmath}

\begin{dmath}\label{eqn:momentCoeffiecients:520}
q_{2, 1}
= \sqrt{\frac{15}{8 \pi}} \int d^3 x' \rho(\Bx') z' ( x' - i y' ),
\end{dmath}

\begin{dmath}\label{eqn:momentCoeffiecients:540}
q_{2, 0}
= \sqrt{\frac{5}{16 \pi}} \int d^3 x' \rho(\Bx') \lr{ 3 (z')^2 - (r')^2 }.
\end{dmath}

For the potential we are interested in

\begin{dmath}\label{eqn:momentCoeffiecients:560}
2 \Real q_{11} r^2 Y_{11}(\theta, \phi)
= 2 \frac{3}{8 \pi} \int d^3 x' \rho(\Bx') \Real \lr{ (x' - i y')( x + i y) }
= \frac{3}{4 \pi} \int d^3 x' \rho(\Bx') \lr{ x x' + y y' },
\end{dmath}

\begin{dmath}\label{eqn:momentCoeffiecients:580}
q_{1, 0} r Y_{1,0}(\theta, \phi)
= \frac{3}{4 \pi} \int d^3 x' \rho(\Bx') z' z,
\end{dmath}

\begin{dmath}\label{eqn:momentCoeffiecients:600}
2 \Real q_{22} r^2 Y_{22}(\theta, \phi)
= 2 \frac{15}{32 \pi} \int d^3 x' \rho(\Bx') \Real \lr{
(x' - i y')^2
(x + i y)^2
}
= \frac{15}{16 \pi} \int d^3 x' \rho(\Bx') \Real \lr{
((x')^2 - 2 i x' y' -(y')^2)
(x^2 + 2 i x y -y^2)
}
= \frac{15}{16 \pi} \int d^3 x' \rho(\Bx') \lr{
((x')^2 -(y')^2) (x^2 -y^2)
+ 4 x x' y y'
},
\end{dmath}

\begin{dmath}\label{eqn:momentCoeffiecients:620}
2 \Real q_{21} r^2 Y_{21}(\theta, \phi)
= 2 \frac{15}{8 \pi} \int d^3 x' \rho(\Bx') z \Real \lr{ ( x' - i y' ) (x + i y) }
= \frac{15}{4 \pi} \int d^3 x' \rho(\Bx') z \lr{ x x' + y y' },
\end{dmath}

and
\begin{dmath}\label{eqn:momentCoeffiecients:640}
q_{2, 0} r^2 Y_{20}(\theta, \phi)
= \frac{5}{16 \pi} \int d^3 x' \rho(\Bx') \lr{ 3 (z')^2 - (r')^2 } \lr{ 3 z^2 - r^2 }.
\end{dmath}

The dipole term of the potential is

\begin{dmath}\label{eqn:momentCoeffiecients:660}
\inv{ 4 \pi \epsilon_0 } \frac{4 \pi}{3 r^3}
\lr{
\frac{3}{4 \pi} \int d^3 x' \rho(\Bx') \lr{ x x' + y y' }
+
\frac{3}{4 \pi} \int d^3 x' \rho(\Bx') z' z
}
=
\inv{ 4 \pi \epsilon_0 r^3}
\Bx \cdot \int d^3 x' \rho(\Bx') \Bx'
=
\frac{\Bx \cdot \Bp}{ 4 \pi \epsilon_0 r^3},
\end{dmath}

as obtained directly when a strict dipole approximation was used.

Summing all the terms for the quadrupole gives

\begin{dmath}\label{eqn:momentCoeffiecients:680}
\begin{aligned}
\inv{ 4 \pi \epsilon r^5 } \frac{ 4 \pi }{5}
\biglr{
&\frac{15}{16 \pi} \int d^3 x' \rho(\Bx') \lr{
((x')^2 -(y')^2) (x^2 -y^2)
+ 4 x x' y y'
} \\
&+
\frac{15}{4 \pi} \int d^3 x' \rho(\Bx') z z' \lr{ x x' + y y' } \\
&+
\frac{5}{16 \pi} \int d^3 x' \rho(\Bx') \lr{ 3 (z')^2 - (r')^2 } \lr{ 3 z^2 - r^2 }
} \\
=
\inv{ 4 \pi \epsilon r^5 }
\int d^3 x' \rho(\Bx')
\inv{4}
\biglr{
   &3
   \lr{
   ((x')^2 -(y')^2) (x^2 -y^2)
   + 4 x x' y y'
   } \\
   &+
   12
   z z' \lr{ x x' + y y' } \\
   &+
   \lr{ 3 (z')^2 - (r')^2 } \lr{ 3 z^2 - r^2 }
}.
\end{aligned}
\end{dmath}

The portion in brackets is

\begin{dmath}\label{eqn:momentCoeffiecients:700}
\begin{aligned}
   3
   &\lr{
      ((x')^2 -(y')^2) (x^2 -y^2)
      + 4 x x' y y'
   } \\
   +
   12
   & z z' \lr{ x x' + y y' }  \\
   +
   &\lr{ 2 (z')^2 - (x')^2 - (y')^2} \lr{ 2 z^2 - x^2 -y^2 } \\
=
x^2 &\lr{
     3 (x')^2 - 3(y')^2
-
   \lr{ 2 (z')^2 - (x')^2 - (y')^2}
} \\
+
y^2 &\lr{
      -3 (x')^2 + 3 (y')^2
-
   \lr{ 2 (z')^2 - (x')^2 - (y')^2}
} \\
+
2 z^2 &\lr{
   2 (z')^2 - (x')^2 - (y')^2
} \\
+
&12{ x x' y y' + x x' z z' + y y' z z' } \\
=
2 x^2 &\lr{
     2 (x')^2 - (y')^2 - (z')^2
} \\
+
2 y^2 &\lr{
     2 (y')^2 - (x')^2 - (z')^2
} \\
+
2 z^2 &\lr{
   2 (z')^2 - (x')^2 - (y')^2
} \\
+
&12{ x x' y y' + x x' z z' + y y' z z' }.
\end{aligned}
\end{dmath}

The quadrupole sum can now be written as
\begin{dmath}\label{eqn:momentCoeffiecients:720}
\inv{2}
\inv{ 4 \pi \epsilon r^5 }
\int d^3 x' \rho(\Bx')
\biglr{
x^2 \lr{ 3 (x')^2 - (r')^2 }
+y^2 \lr{ 3 (y')^2 - (r')^2 }
+z^2 \lr{ 3 (z')^2 - (r')^2 }
+
3 \lr{
x y x' y'
+y x y' x'
+x z x' z'
+z x z' x'
+y z y' z'
+z y z' y'
}
},
\end{dmath}

which is precisely \cref{eqn:momentCoeffiecients:120}, the quadrupole potential stated in the text and class notes.

%}
%\EndArticle

      \section{Problems}
         %
% Copyright � 2016 Peeter Joot.  All Rights Reserved.
% Licenced as described in the file LICENSE under the root directory of this GIT repository.
%
%{
\newcommand{\authorname}{Peeter Joot}
\newcommand{\email}{peeterjoot@protonmail.com}
\newcommand{\basename}{FIXMEbasenameUndefined}
\newcommand{\dirname}{notes/FIXMEdirnameUndefined/}

\renewcommand{\basename}{dipoleFromSphericalMoments}
%\renewcommand{\dirname}{notes/phy1520/}
\renewcommand{\dirname}{notes/ece1228-electromagnetic-theory/}
%\newcommand{\dateintitle}{}
%\newcommand{\keywords}{}

\newcommand{\authorname}{Peeter Joot}
\newcommand{\onlineurl}{http://sites.google.com/site/peeterjoot2/math2013/\basename.pdf}
\newcommand{\sourcepath}{\dirname\basename.tex}
\newcommand{\generatetitle}[1]{\chapter{#1}}

\newcommand{\vcsinfo}{%
\section*{}
\noindent{\color{DarkOliveGreen}{\rule{\linewidth}{0.1mm}}}
\paragraph{Document version}
%\paragraph{\color{Maroon}{Document version}}
{
\small
\begin{itemize}
\item Available online at:\\ 
\href{\onlineurl}{\onlineurl}
\item Git Repository: \input{./.revinfo/gitRepo.tex}
\item Source: \sourcepath
\item last commit: \input{./.revinfo/gitCommitString.tex}
\item commit date: \input{./.revinfo/gitCommitDate.tex}
\end{itemize}
}
}

%\PassOptionsToPackage{dvipsnames,svgnames}{xcolor}
\PassOptionsToPackage{square,numbers}{natbib}
\documentclass{scrreprt}

\usepackage[left=2cm,right=2cm]{geometry}
\usepackage[svgnames]{xcolor}
\usepackage{peeters_layout}

\usepackage{natbib}

\usepackage[
colorlinks=true,
bookmarks=false,
pdfauthor={\authorname, \email},
backref 
]{hyperref}

% http://tex.stackexchange.com/questions/75773/how-to-reference-problems-by-the-text-label-in-an-exercise-envioronment
\usepackage[english]{cleveref}
\crefname{Exercise}{exercise}{exercises}
\Crefname{Exercise}{Exercise}{Exercises}

\RequirePackage{titlesec}
\RequirePackage{ifthen}

% http://stackoverflow.com/questions/4932910/date-in-the-tabular-environment
\makeatletter
\let\insertdate\@date
\makeatother

\titleformat{\chapter}[display]
{\bfseries\Large}
{\color{DarkSlateGrey}\filleft \authorname
\ifthenelse{\isundefined{\studentnumber}}{}{\\ \studentnumber}
\ifthenelse{\isundefined{\email}}{}{\\ \email}
\ifthenelse{\isundefined{\dateintitle}}{}{\\ \insertdate}
%\ifthenelse{\isundefined{\coursename}}{}{\\ \coursename} % put in title instead.
}
{4ex}
{\color{DarkOliveGreen}{\titlerule}\color{Maroon}
\vspace{2ex}%
\filright}
[\vspace{2ex}%
\color{DarkOliveGreen}\titlerule
]

\newcommand{\beginArtWithToc}[0]{\begin{document}\tableofcontents}
\newcommand{\beginArtNoToc}[0]{\begin{document}}
\newcommand{\EndNoBibArticle}[0]{\end{document}}
\newcommand{\EndArticle}[0]{\bibliography{Bibliography}\bibliographystyle{plainnat}\end{document}}

% 
%\newcommand{\citep}[1]{\cite{#1}}

\colorSectionsForArticle



\usepackage{peeters_layout_exercise}
\usepackage{peeters_braket}
\usepackage{peeters_figures}
\usepackage{siunitx}
%\usepackage{mhchem} % \ce{}
%\usepackage{macros_bm} % \bcM
%\usepackage{txfonts} % \ointclockwise

\beginArtNoToc

\generatetitle{Dipole field from spherical harmonics}
%\chapter{Dipole field from spherical harmonics}
%\label{chap:dipoleFromSphericalMoments}

As indicated in Jackson \citep{jackson1975cew}, the components of the electric field can be obtained directly from the multipole moments

\begin{dmath}\label{eqn:dipoleFromSphericalMoments:20}
\Phi(\Bx) 
= \inv{4 \pi \epsilon_0} \sum \frac{4 \pi}{ (2 l + 1) r^{l + 1} } q_{l m} Y_{l m},
\end{dmath}

so for the \( l,m \) contribution to this sum the components of the electric field are

\begin{dmath}\label{eqn:dipoleFromSphericalMoments:40}
E_r 
=
\inv{\epsilon_0} \sum \frac{l+1}{ (2 l + 1) r^{l + 2} } q_{l m} Y_{l m},
\end{dmath}

\begin{dmath}\label{eqn:dipoleFromSphericalMoments:60}
E_\theta 
= -\inv{\epsilon_0} \sum \frac{1}{ (2 l + 1) r^{l + 2} } q_{l m} \partial_\theta Y_{l m}
\end{dmath}

\begin{dmath}\label{eqn:dipoleFromSphericalMoments:80}
E_\phi 
= -\inv{\epsilon_0} \sum \frac{1}{ (2 l + 1) r^{l + 2} \sin\theta } q_{l m} \partial_\phi Y_{l m}
= -\inv{\epsilon_0} \sum \frac{j m}{ (2 l + 1) r^{l + 2} \sin\theta } q_{l m} Y_{l m}.
\end{dmath}

Here I've translated from CGS to SI.  Let's calculate the \( l = 1 \) electric field components directly from these expressions and check against the previously calculated results.

\begin{dmath}\label{eqn:dipoleFromSphericalMoments:100}
E_r 
=
\inv{\epsilon_0} \frac{2}{ 3 r^{3} } 
\lr{
   2 \lr{ -\sqrt{\frac{3}{8\pi}} }^2 \Real \lr{ 
      (p_x - j p_y) \sin\theta e^{j\phi}
   }
   +
   \lr{ \sqrt{\frac{3}{4\pi}} }^2 p_z \cos\theta
}
=
\frac{2}{4 \pi \epsilon_0 r^3} 
\lr{
   p_x \sin\theta \cos\phi + p_y \sin\theta \sin\phi + p_z \cos\theta
}
= 
\frac{1}{4 \pi \epsilon_0 r^3} 2 \Bp \cdot \rcap.
\end{dmath}

Note that 

\begin{dmath}\label{eqn:dipoleFromSphericalMoments:120}
\partial_\theta Y_{11} = -\sqrt{\frac{3}{8\pi}} \cos\theta e^{j \phi},
\end{dmath}

and

\begin{dmath}\label{eqn:dipoleFromSphericalMoments:140}
\partial_\theta Y_{1,-1} = \sqrt{\frac{3}{8\pi}} \cos\theta e^{-j \phi},
\end{dmath}

so

\begin{dmath}\label{eqn:dipoleFromSphericalMoments:160}
E_\theta 
=
-\inv{\epsilon_0} \frac{1}{ 3 r^{3} } 
\lr{
   2 \lr{ -\sqrt{\frac{3}{8\pi}} }^2 \Real \lr{ 
      (p_x - j p_y) \cos\theta e^{j\phi}
   }
   -
   \lr{ \sqrt{\frac{3}{4\pi}} }^2 p_z \sin\theta
}
=
-\frac{1}{4 \pi \epsilon_0 r^3} 
\lr{
   p_x \cos\theta \cos\phi + p_y \cos\theta \sin\phi - p_z \sin\theta
}
=
-\frac{1}{4 \pi \epsilon_0 r^3} \Bp \cdot \thetacap.
\end{dmath}

For the \(\phicap\) component, the \( m = 0 \) term is killed.  This leaves

\begin{dmath}\label{eqn:dipoleFromSphericalMoments:180}
E_\phi
=
-\frac{1}{\epsilon_0} \frac{1}{ 3 r^{3} \sin\theta } 
\lr{
j q_{11} Y_{11} - j q_{1,-1} Y_{1,-1}
}
=
-\frac{1}{3 \epsilon_0 r^{3} \sin\theta } 
\lr{
j q_{11} Y_{11} - j (-1)^{2m} q_{11}^\conj Y_{11}^\conj
}
=
\frac{2}{\epsilon_0} \frac{1}{ 3 r^{3} \sin\theta } 
\Imag q_{11} Y_{11}
=
\frac{2}{3 \epsilon_0 r^{3} \sin\theta } 
\Imag \lr{
   \lr{ -\sqrt{\frac{3}{8\pi}} }^2 (p_x - j p_y) \sin\theta e^{j \phi}
}
=
\frac{1}{ 4 \pi \epsilon_0 r^{3} } 
\Imag \lr{
   (p_x - j p_y) e^{j \phi}
}
=
\frac{1}{ 4 \pi \epsilon_0 r^{3} } 
\lr{
   p_x \sin\phi - p_y \cos\phi
}
=
-\frac{\Bp \cdot \phicap}{ 4 \pi \epsilon_0 r^3}.
\end{dmath}

That is
%\begin{dmath}\label{eqn:dipoleFromSphericalMoments:200}
\boxedEquation{eqn:dipoleFromSphericalMoments:200}{
\begin{aligned}
E_r &= 
\frac{2}{4 \pi \epsilon_0 r^3} 
\Bp \cdot \rcap \\
E_\theta &= -
\frac{1}{4 \pi \epsilon_0 r^3} 
\Bp \cdot \phicap \\
E_\phi &= -
\frac{1}{4 \pi \epsilon_0 r^3} 
\Bp \cdot \phicap.
\end{aligned}
}
%\end{dmath}

These are consistent with equations (4.12) from the text for when \( \Bp \) is aligned with the z-axis.

Observe that we can sum each of the projections of \( \BE \) to construct the total electric field due to this \( l = 1 \) term of the multipole moment sum

\begin{dmath}\label{eqn:dipoleFromSphericalMoments:n}
\BE 
=
\frac{1}{4 \pi \epsilon_0 r^3} 
\lr{
2 \rcap (\Bp \cdot \rcap) 
-
\phicap ( \Bp \cdot \phicap) 
-
\thetacap ( \Bp \cdot \thetacap) 
}
=
\frac{1}{4 \pi \epsilon_0 r^3} 
\lr{
3 \rcap (\Bp \cdot \rcap) 
-
\Bp
},
\end{dmath}

which recovers the expected dipole moment approximation.

%}
\EndArticle


   \chapter{Fresnel relations}
      %
% Copyright � 2016 Peeter Joot.  All Rights Reserved.
% Licenced as described in the file LICENSE under the root directory of this GIT repository.
%
\newcommand{\authorname}{Peeter Joot}
\newcommand{\email}{peeterjoot@protonmail.com}
\newcommand{\basename}{FIXMEbasenameUndefined}
\newcommand{\dirname}{notes/FIXMEdirnameUndefined/}

\renewcommand{\basename}{emt10}
\renewcommand{\dirname}{notes/ece1228/}
\newcommand{\keywords}{ECE1228H}
\newcommand{\authorname}{Peeter Joot}
\newcommand{\onlineurl}{http://sites.google.com/site/peeterjoot2/math2013/\basename.pdf}
\newcommand{\sourcepath}{\dirname\basename.tex}
\newcommand{\generatetitle}[1]{\chapter{#1}}

\newcommand{\vcsinfo}{%
\section*{}
\noindent{\color{DarkOliveGreen}{\rule{\linewidth}{0.1mm}}}
\paragraph{Document version}
%\paragraph{\color{Maroon}{Document version}}
{
\small
\begin{itemize}
\item Available online at:\\ 
\href{\onlineurl}{\onlineurl}
\item Git Repository: \input{./.revinfo/gitRepo.tex}
\item Source: \sourcepath
\item last commit: \input{./.revinfo/gitCommitString.tex}
\item commit date: \input{./.revinfo/gitCommitDate.tex}
\end{itemize}
}
}

%\PassOptionsToPackage{dvipsnames,svgnames}{xcolor}
\PassOptionsToPackage{square,numbers}{natbib}
\documentclass{scrreprt}

\usepackage[left=2cm,right=2cm]{geometry}
\usepackage[svgnames]{xcolor}
\usepackage{peeters_layout}

\usepackage{natbib}

\usepackage[
colorlinks=true,
bookmarks=false,
pdfauthor={\authorname, \email},
backref 
]{hyperref}

% http://tex.stackexchange.com/questions/75773/how-to-reference-problems-by-the-text-label-in-an-exercise-envioronment
\usepackage[english]{cleveref}
\crefname{Exercise}{exercise}{exercises}
\Crefname{Exercise}{Exercise}{Exercises}

\RequirePackage{titlesec}
\RequirePackage{ifthen}

% http://stackoverflow.com/questions/4932910/date-in-the-tabular-environment
\makeatletter
\let\insertdate\@date
\makeatother

\titleformat{\chapter}[display]
{\bfseries\Large}
{\color{DarkSlateGrey}\filleft \authorname
\ifthenelse{\isundefined{\studentnumber}}{}{\\ \studentnumber}
\ifthenelse{\isundefined{\email}}{}{\\ \email}
\ifthenelse{\isundefined{\dateintitle}}{}{\\ \insertdate}
%\ifthenelse{\isundefined{\coursename}}{}{\\ \coursename} % put in title instead.
}
{4ex}
{\color{DarkOliveGreen}{\titlerule}\color{Maroon}
\vspace{2ex}%
\filright}
[\vspace{2ex}%
\color{DarkOliveGreen}\titlerule
]

\newcommand{\beginArtWithToc}[0]{\begin{document}\tableofcontents}
\newcommand{\beginArtNoToc}[0]{\begin{document}}
\newcommand{\EndNoBibArticle}[0]{\end{document}}
\newcommand{\EndArticle}[0]{\bibliography{Bibliography}\bibliographystyle{plainnat}\end{document}}

% 
%\newcommand{\citep}[1]{\cite{#1}}

\colorSectionsForArticle



%\usepackage{ece1228}
\usepackage{peeters_braket}
%\usepackage{peeters_layout_exercise}
\usepackage{peeters_figures}
\usepackage{mathtools}
\usepackage{siunitx}
\usepackage{macros_bm}

\beginArtNoToc
\generatetitle{ECE1228H Electromagnetic Theory.  Lecture 10: Fresnel relations.  Taught by Prof.\ M. Mojahedi}
%\chapter{Fresnel relations}
\label{chap:emt10}

%\paragraph{Disclaimer}
%
%Peeter's lecture notes from class.  These may be incoherent and rough.
%
%These are notes for the UofT course ECE1228H, Electromagnetic Theory, taught by Prof. M. Mojahedi, covering \textchapref{{1}} \citep{balanis1989advanced} content.
%
\paragraph{Single interface TE mode.}

The Fresnel reflection geometry for an electric field \( \BE \) parallel to the interface (TE mode) is sketched in \cref{fig:l10TwoInterfaces:l10TwoInterfacesFig1}.

\imageFigure{../../figures/ece1228-emt/l10TwoInterfacesFig1}{Electric field TE mode Fresnel geometry.}{fig:l10TwoInterfaces:l10TwoInterfacesFig1}{0.2}

\begin{dmath}\label{eqn:emtLecture10:20}
   \bcE_i = \Be_2 E_i e^{j \omega t - j \Bk_{i} \cdot \Bx },
\end{dmath}

with an assumption that this field maintains it's polarization in both its reflected and transmitted components, so that

\begin{dmath}\label{eqn:emtLecture10:40}
   \bcE_r = \Be_2 r E_i e^{j \omega t - j \Bk_{r} \cdot \Bx },
\end{dmath}

and
\begin{dmath}\label{eqn:emtLecture10:60}
   \bcE_t = \Be_2 t E_i e^{j \omega t - j \Bk_{t} \cdot \Bx },
\end{dmath}

Measuring the angles \( \theta_i, \theta_r, \theta_t \) from the normal, with \( i = \Be_3 \Be_1 \) the wave vectors are

\begin{dmath}\label{eqn:emtLecture10:620}
\begin{aligned}
\Bk_{i} &= \Be_3 k_1 e^{i\theta_i} = k_1\lr{ \Be_3 \cos\theta_i + \Be_1\sin\theta_i } \\
\Bk_{r} &= -\Be_3 k_1 e^{-i\theta_r} = k_1 \lr{ -\Be_3 \cos\theta_r + \Be_1 \sin\theta_r } \\
\Bk_{t} &= \Be_3 k_2 e^{i\theta_t} = k_2 \lr{ \Be_3 \cos\theta_t + \Be_1 \sin\theta_t }
\end{aligned}
\end{dmath}

So the time harmonic electric fields are

\begin{dmath}\label{eqn:emtLecture10:640}
\begin{aligned}
   \BE_i &= \Be_2 E_i \exp\lr{ - j k_1 \lr{ z\cos\theta_i + x \sin\theta_i} } \\
   \BE_r &= \Be_2 r E_i \exp\lr{ - j k_1 \lr{ -z \cos\theta_r + x \sin\theta_r}} \\
   \BE_t &= \Be_2 t E_i \exp\lr{ - j k_2 \lr{ z \cos\theta_t + x \sin\theta_t}}.
\end{aligned}
\end{dmath}

The magnetic fields follow from Faraday's law

\begin{dmath}\label{eqn:emtLecture10:900}
\BH 
= \inv{-j \omega \mu } \spacegrad \cross \BE 
= \inv{-j \omega \mu } \spacegrad \cross \Be_2 e^{-j \Bk \cdot \Bx} 
= \inv{j \omega \mu } \Be_2 \cross \spacegrad e^{-j \Bk \cdot \Bx} 
= -\inv{\omega \mu } \Be_2 \cross \Bk e^{-j \Bk \cdot \Bx}
= \inv{\omega \mu } \Bk \cross \BE
\end{dmath}

We have 

\begin{dmath}\label{eqn:emtLecture10:920}
\begin{aligned}
\kcap_{i} \cross \Be_2 &= -\Be_1 \cos\theta_i + \Be_3\sin\theta_i  \\
\kcap_{r} \cross \Be_2 &= \Be_1 \cos\theta_r + \Be_3 \sin\theta_r  \\
\kcap_{t} \cross \Be_2 &= -\Be_1 \cos\theta_t + \Be_3 \sin\theta_t,
\end{aligned}
\end{dmath}

so
\begin{dmath}\label{eqn:emtLecture10:940}
\begin{aligned}
\BH_{i} &= \frac{k_1 E_i}{\omega\mu_1} \lr{ -\Be_1 \cos\theta_i + \Be_3\sin\theta_i } \exp\lr{ - j k_1 \lr{ z\cos\theta_i + x \sin\theta_i} } \\
\BH_{r} &= \frac{k_1 r E_i}{\omega\mu_1} \lr{ \Be_1 \cos\theta_r + \Be_3 \sin\theta_r } \exp\lr{ - j k_1 \lr{ -z \cos\theta_r + x \sin\theta_r}} \\
\BH_{t} &= \frac{k_2 t E_i}{\omega\mu_2} \lr{ -\Be_1 \cos\theta_t + \Be_3 \sin\theta_t } \exp\lr{ - j k_2 \lr{ z \cos\theta_t + x \sin\theta_t}}.
\end{aligned}
\end{dmath}

The boundary conditions at \( z = 0 \) with \( \ncap = \Be_3 \) are

\begin{dmath}\label{eqn:emtLecture10:960}
\begin{aligned}
\ncap \cross \BH_1 &= \ncap \cross \BH_2  \\
\ncap \cdot \BB_1 &= \ncap \cdot \BB_2  \\
\ncap \cross \BE_1 &= \ncap \cross \BE_2  \\
\ncap \cdot \BD_1 &= \ncap \cdot \BD_2,
\end{aligned}
\end{dmath}

which gives

\begin{subequations}
\label{eqn:emtLecture10:980}
\begin{dmath}\label{eqn:emtLecture10:1000}
-\frac{k_1 }{\mu_1} \cos\theta_i \exp\lr{ - j k_1 x \sin\theta_i } 
+
\frac{k_1 r }{\mu_1} \cos\theta_r \exp\lr{ - j k_1 x \sin\theta_r }
=
-\frac{k_2 t }{\mu_2} \cos\theta_t \exp\lr{ - j k_2 x \sin\theta_t },
\end{dmath}
\begin{dmath}\label{eqn:emtLecture10:1020}
k_1 \sin\theta_i \exp\lr{ - j k_1 x \sin\theta_i }
+
k_1 r \sin\theta_r \exp\lr{ + j k_1 x \sin\theta_r }
=
k_2 t \sin\theta_t \exp\lr{ - j k_2 x \sin\theta_t }
\end{dmath}
\begin{dmath}\label{eqn:emtLecture10:1040}
\exp\lr{ - j k_1 \lr{ x \sin\theta_i} } 
+
r \exp\lr{ - j k_1 \lr{ x \sin\theta_r}} 
=
t \exp\lr{ - j k_2 \lr{ x \sin\theta_t}}.
\end{dmath}
\end{subequations}

Since these must also hold at \( x = 0 \)

\begin{dmath}\label{eqn:emtLecture10:1060}
\begin{aligned}
-\frac{k_1 }{\mu_1} \cos\theta_i + \frac{k_1 r }{\mu_1} \cos\theta_r &= -\frac{k_2 t }{\mu_2} \cos\theta_t  \\
k_1 \sin\theta_i + k_1 r \sin\theta_r &= k_2 t \sin\theta_t  \\
1 + r &= t
\end{aligned}
\end{dmath}

When \( t = 0 \) the latter two equations give Shell's first law

%\begin{dmath}\label{eqn:emtLecture10:1080}
\boxedEquation{eqn:emtLecture10:1080}{
\sin\theta_i = \sin\theta_r.
}
%\end{dmath}

Assuming this holds for all \( r, t \) we have

\begin{dmath}\label{eqn:emtLecture10:1120}
k_1 \sin\theta_i (1 + r ) = k_2 t \sin\theta_t,
\end{dmath}

which is Snell's second law
%\begin{dmath}\label{eqn:emtLecture10:1140}
\boxedEquation{eqn:emtLecture10:1160}{
k_1 \sin\theta_i = k_2 \sin\theta_t.
}
%\end{dmath}

With
\begin{dmath}\label{eqn:emtLecture10:1200}
\begin{aligned}
k_{1z} &= k_1 \cos\theta_i \\
k_{2z} &= k_2 \cos\theta_t,
\end{aligned}
\end{dmath}

we can solve for \( r, t \) by inverting

\begin{dmath}\label{eqn:emtLecture10:1180}
\begin{bmatrix}
\mu_2 k_{1z} & \mu_1 k_{2z} \\
-1 & 1 \\
\end{bmatrix}
\begin{bmatrix}
r \\
t
\end{bmatrix}
=
\begin{bmatrix}
\mu_2 k_{1z} \\
1
\end{bmatrix},
\end{dmath}

which gives

\begin{dmath}\label{eqn:emtLecture10:1220}
\begin{bmatrix}
r \\
t
\end{bmatrix}
=
\begin{bmatrix}
1 & -\mu_1 k_{2z} \\
1 &  \mu_2 k_{1z}
\end{bmatrix}
\begin{bmatrix}
\mu_2 k_{1z} \\
1
\end{bmatrix},
\end{dmath}

or
%\begin{dmath}\label{eqn:emtLecture10:1240}
\boxedEquation{eqn:emtLecture10:1260}{
\begin{aligned}
r &= \frac{\mu_2 k_{1z} - \mu_1 k_{2z}}{\mu_2 k_{1z} + \mu_1 k_{2z}} \\
t &= \frac{2 \mu_2 k_{1z}}{\mu_2 k_{1z} + \mu_1 k_{2z}}
\end{aligned}
}
%\end{dmath}

\paragraph{Single interface TM mode.}

For completeness, now consider the TM mode.  In this case, the magnetic fields are

\begin{dmath}\label{eqn:emtLecture10:1320}
\begin{aligned}
   \BH_i &= \Be_2 H_i \exp\lr{ - j k_1 \lr{ z\cos\theta_i + x \sin\theta_i} } \\
   \BH_r &= \Be_2 r H_i \exp\lr{ - j k_1 \lr{ -z \cos\theta_r + x \sin\theta_r}} \\
   \BH_t &= \Be_2 t H_i \exp\lr{ - j k_2 \lr{ z \cos\theta_t + x \sin\theta_t}}.
\end{aligned}
\end{dmath}

Faraday's law also can provide the electric field from the magnetic

\begin{dmath}\label{eqn:emtLecture10:1280}
\kcap \cross \BH 
= \frac{k}{\omega \mu } \kcap \cross \lr{ \kcap \cross \BE }
= -\frac{k}{\omega \mu } \kcap \cdot \lr{ \kcap \wedge \BE } 
= -\frac{k}{\omega \mu } \lr{ \BE - \kcap \lr{ \kcap \cdot \BE } },
\end{dmath}

so

\begin{dmath}\label{eqn:emtLecture10:1300}
\BE = \frac{\omega \mu}{k} \BH \cross \kcap.
\end{dmath}

This gives
\begin{dmath}\label{eqn:emtLecture10:1340}
\begin{aligned}
   \BE_{i} &= -\frac{\omega \mu_1 H_i}{k_1} \lr{ -\Be_1 \cos\theta_i + \Be_3\sin\theta_i } \exp\lr{ - j k_1 \lr{ z\cos\theta_i + x \sin\theta_i} } \\
   \BE_{r} &= -\frac{\omega \mu_1 r H_i}{k_1} \lr{ \Be_1 \cos\theta_r + \Be_3 \sin\theta_r } \exp\lr{ - j k_1 \lr{ -z \cos\theta_r + x \sin\theta_r}} \\
   \BE_{t} &= -\frac{\omega \mu_2 t H_i}{k_2} \lr{ -\Be_1 \cos\theta_t + \Be_3 \sin\theta_t } \exp\lr{ - j k_2 \lr{ z \cos\theta_t + x \sin\theta_t}}.
\end{aligned}
\end{dmath}

Imposing the same constraints \cref{eqn:emtLecture10:960} we have

\begin{subequations}
\label{eqn:emtLecture10:1360}
%\ncap \cross \BH_1 &= \ncap \cross \BH_2  \\
%\ncap \cdot \BB_1 &= \ncap \cdot \BB_2  % 0 = 0
\begin{dmath}\label{eqn:emtLecture10:1380}
   \exp\lr{ - j k_1 x \sin\theta_i } + r \exp\lr{ - j k_1 x \sin\theta_r } =
   t \exp\lr{ - j k_2 x \sin\theta_t }
\end{dmath}
%\ncap \cross \BE_1 &= \ncap \cross \BE_2  \\
\begin{dmath}\label{eqn:emtLecture10:1400}
   -\frac{\mu_1 }{k_1} \cos\theta_i \exp\lr{ - j k_1 x \sin\theta_i } 
   +\frac{\mu_1 r }{k_1} \cos\theta_r \exp\lr{ - j k_1 x \sin\theta_r } 
=
   -\frac{\mu_2 t }{k_2} \cos\theta_t \exp\lr{ - j k_2 x \sin\theta_t }
\end{dmath}
%\ncap \cdot \BD_1 &= \ncap \cdot \BD_2,
\begin{dmath}\label{eqn:emtLecture10:1420}
   \frac{\epsilon_1 \mu_1 }{k_1} \sin\theta_i  \exp\lr{ - j k_1 x \sin\theta_i } +
   \frac{\epsilon_1 \mu_1 r }{k_1} \sin\theta_r  \exp\lr{ - j k_1 x \sin\theta_r } =
   \frac{\epsilon_2 \mu_2 t }{k_2} \sin\theta_t  \exp\lr{ - j k_2 x \sin\theta_t }.
\end{dmath}
\end{subequations}

Requiring these to be identical at \( x = 0 \) too, we have

\begin{dmath}\label{eqn:emtLecture10:1440}
\begin{aligned}
1 + r &= t \\
-\frac{\mu_1 }{k_1} \cos\theta_i +\frac{\mu_1 r }{k_1} \cos\theta_r &= -\frac{\mu_2 t }{k_2} \cos\theta_t  \\
\frac{\epsilon_1 \mu_1 }{k_1} \sin\theta_i  + \frac{\epsilon_1 \mu_1 r }{k_1} \sin\theta_r  &= \frac{\epsilon_2 \mu_2 t }{k_2} \sin\theta_t  
\end{aligned}
\end{dmath}

Once again, two of these equations can be simultaneously satisfied by \cref{eqn:emtLecture10:1080}, and \cref{eqn:emtLecture10:1120}.

That leaves
\begin{dmath}\label{eqn:emtLecture10:1460}
\begin{bmatrix}
\mu_1 k_2 \cos\theta_i & \mu_2 k_1 \cos\theta_t \\
-1 & 1
\end{bmatrix}
\begin{bmatrix}
r \\
t
\end{bmatrix}
= 
\begin{bmatrix}
\mu_1 k_2 \cos\theta_i \\
1
\end{bmatrix},
\end{dmath}

the inverse of which is
\begin{dmath}\label{eqn:emtLecture10:1480}
\begin{bmatrix}
r \\
t
\end{bmatrix}
= 
\inv{ \mu_1 k_2 \cos\theta_i + \mu_2 k_1 \cos\theta_t }
\begin{bmatrix}
1 & -\mu_2 k_1 \cos\theta_t \\
1 & \mu_1 k_2 \cos\theta_i
\end{bmatrix}
\begin{bmatrix}
\mu_1 k_2 \cos\theta_i \\
1
\end{bmatrix}
=
\inv{ \mu_1 k_2 \cos\theta_i + \mu_2 k_1 \cos\theta_t }
\begin{bmatrix}
\mu_1 k_2 \cos\theta_i - \mu_2 k_1 \cos\theta_t \\
2 \mu_1 k_2 \cos\theta_i 
\end{bmatrix}.
\end{dmath}

\paragraph{Two interface problem.}

It turns out that light passing through two interfaces ends up with reflected and transmitted components, but no mode that goes through the slab in a wave guide fashion.  Such a mode is possible when the light is incident in an end-fire configuration (i.e. a waveguide).  It's interesting that such a wave guide like mode isn't possible.

Consider a geometry with interfaces at \( z = 0, d \).  

The incident, and reflected wave components in the first medium are
\begin{dmath}\label{eqn:emtLecture10:140}
   A e^{-j k_{1z} z},
\end{dmath}
\begin{dmath}\label{eqn:emtLecture10:160}
   r A e^{j k_{1z} z},
\end{dmath}

in the second medium 

\begin{dmath}\label{eqn:emtLecture10:180}
   C e^{-j k_{1z} z},
\end{dmath}
\begin{dmath}\label{eqn:emtLecture10:200}
   D e^{j k_{1z} z},
\end{dmath}

and in the final medium
\begin{dmath}\label{eqn:emtLecture10:220}
   A t e^{-j k_{1z} (z-d)},
\end{dmath}

\begin{itemize}
   \item \( z = 0 \).
      % FIXME
   %A r_{12} = C + D
   \item \( z = d \).
      % FIXME
\end{itemize}
These relationships can be assembled into matrix form. The end result is

FIXME.

Something important to consider is that the index of refraction can be complex valued

\begin{dmath}\label{eqn:emtLecture10:240}
n(\omega) = n'(\omega) + j n''(\omega),
\end{dmath}

but all of these derivations are independent of this.

\paragraph{special case.}

If medium 1 and 3 are identical then

\begin{dmath}\label{eqn:emtLecture10:260}
r_{21} = r_{23} = -r_{12},
\end{dmath}

which gives

\begin{dmath}\label{eqn:emtLecture10:280}
   t^{\textrm{TE}} = \frac{t_{12} t_{21} e^{j \phi}}{1 - r_{21}^2 e^{2 j \phi}}
\end{dmath}
\begin{dmath}\label{eqn:emtLecture10:300}
   r^{\textrm{TE}} = r_{12} + \frac{t_{12} t_{21} r_{21} e^{2 j \phi}}{1 - r_{21}^2 e^{2 j \phi}}
\end{dmath}

where
\begin{dmath}\label{eqn:emtLecture10:320}
   \phi = -k_{2z} d = - \frac{\omega}{c} n_2 \cos\theta_2 d
   % \cos(\theta_2 d)?
\end{dmath}

It's possible to design the material so that there is no reflection from the slab, called the matched condition.

This requires

\begin{dmath}\label{eqn:emtLecture10:340}
r_{21} = \frac{-\mu_2 k_{1z} + \mu_1 k_{2z}}{\mu_2 k_{1z} + \mu_1 k_{2z}} = -r_{12}.
\end{dmath}

To force this to be zero, one can set
\begin{dmath}\label{eqn:emtLecture10:360}
\mu_2 k_{1z} = \mu_1 k_{2z},
\end{dmath}

or
\begin{dmath}\label{eqn:emtLecture10:380}
   \mu_2 \frac{\omega}{c} n_1 \cos \theta_1 =
   \mu_1 \frac{\omega}{c} n_2 \cos \theta_2
\end{dmath}

or
\begin{dmath}\label{eqn:emtLecture10:400}
   \mu_2 \sqrt{\epsilon_1 \mu_1} \cos \theta_1 = \mu_1 \sqrt{\epsilon_2 \mu_2} \cos \theta_2,
\end{dmath}

which is
\begin{dmath}\label{eqn:emtLecture10:420}
   \sqrt{\epsilon_1/\mu_1} \cos \theta_1 = \sqrt{\epsilon_2/\mu_2} \cos \theta_2,
\end{dmath}

or
\begin{dmath}\label{eqn:emtLecture10:440}
   \eta_1 \cos \theta_1 = \eta_2 \cos \theta_2.
\end{dmath}

At normal incidence, this reduces to

\begin{dmath}\label{eqn:emtLecture10:460}
   \eta_1 = \eta_2.
\end{dmath}

What happens to the transmission coefficient for such a slab.

One can find 

\begin{equation}\label{eqn:emtLecture10:480}
t_{12} = t_{21} = 1,
\end{equation}

under a matched condition, so

\begin{equation}\label{eqn:emtLecture10:500}
   t = t_{12} t_{21} e^{j \phi}.
\end{equation}

The matched slab only introduces a phase delay (at the specific frequency for which the slab is matched).

\paragraph{Group delay}

Under matched conditions where \( t^{\textrm{TE}} = e^{j \phi} = e^{j k_{2z} d} \), we can write

\begin{dmath}\label{eqn:emtLecture10:520}
- \PD{\phi}{\omega} 
= \PD{\omega}{} \lr{ k_{2z} d }
= d \PD{\omega}{k_{2z}}
= d /\PD{k_{2z}}{\omega}
= d / v_g,
\end{dmath}

so 

\begin{dmath}\label{eqn:emtLecture10:540}
v_g = -\frac{d}{\PD{\phi}{\omega}}.
\end{dmath}
%FIXME: last relation?

This is called the \textAndIndex{group delay}, and is essentially the velocity of the peak of the wave form, as sketched in

F5.

This is different than the phase velocity \( v_p = \omega/k \), as illustrated in the sketch of 

F2

If (FIXME: why?) 

\begin{dmath}\label{eqn:emtLecture10:660}
v_g
= \PD{k}{\omega}
= 1/\PD{\omega}{k},
\end{dmath}

and in an unbounded medium
\begin{dmath}\label{eqn:emtLecture10:560}
   k = \frac{\omega}{c} n(\omega),
\end{dmath}

so
\begin{dmath}\label{eqn:emtLecture10:580}
\PD{\omega}{k} 
= 
\inv{c} \lr{ n(\omega) + \omega \PD{\omega}{n} },
\end{dmath}

so
\begin{equation}\label{eqn:emtLecture10:600}
v_g 
= \frac{c}{n(\omega) + \omega \PD{\omega}{n} }
= \frac{c}{n_g}
\end{equation}

Since the phase is

\begin{dmath}\label{eqn:emtLecture10:680}
\phi = -\frac{\omega}{c} n(\omega) d
\end{dmath}

we can show that
\begin{dmath}\label{eqn:emtLecture10:700}
v_g 
= \frac{d}{-\PD{\omega}{\phi}}
\equiv
\frac{d}{\tau_g}.
\end{dmath}

By bounded medium, it is meant that the medium is not matched.

\paragraph{Brewster's angle}

Self study (examinable!) Brewster's angle for a single interface: The angle for which there is no reflection.

Does it exist for a TE mode, and also for a TM mode, and if not, for which polarization?

Should find that there is no Brewster angle for one of the polarizations when \( \mu_1 = \mu_2 \).  It may be that it's TE that always has some reflection.

\paragraph{Critical angle}

Self study (examinable!) critical angle, the angle for which \( \Abs{r} = 1 \)

Does it exist for a TE mode, and also for a TM mode, and if not, for which polarization?

We should find that this only occurs when we go from a medium with a low index of refraction to one where with a higher index of refraction (i.e. \( \theta_i \ge \theta_c \).

For such an angle, what is the value of the phase?

We should find that there is an electric and magnetic field on the other side of the medium, but that there is no transmitted power on that side of the interface (i.e. Poynting vector is zero (no power transfer) when the angle of incidance is greater than \( \theta_c \)).

Can have photon tunnelling through the space between two prisms separated by an air gap

F4

where, despite the critical angle, the time for the photon to ``tunnel'' through the space between the prisms is less than the time for the photon to go through just the first prism.

\section{Vector and electrostatic (scalar) potentials}

In electrostatics where
\begin{dmath}\label{eqn:emtLecture10:720}
\spacegrad \cross \BE  = 0,
\end{dmath}

the electric field must be the gradient of some function
\begin{dmath}\label{eqn:emtLecture10:740}
\BE = \pm \spacegrad V,
\end{dmath}

We pick negative to have things consistent with the notion of force.

In electrodynamics we have
\begin{dmath}\label{eqn:emtLecture10:760}
\spacegrad \cross \BE = - \PD{t}{\BB},
\end{dmath}

however, we also have

\begin{dmath}\label{eqn:emtLecture10:780}
\spacegrad \cdot \BB = 0,
\end{dmath}

so
\begin{dmath}\label{eqn:emtLecture10:800}
   \BB = \spacegrad \cross \BA.
\end{dmath}

This gives
\begin{dmath}\label{eqn:emtLecture10:820}
   \spacegrad \cross \BE = - \PD{t}{} \lr{ \spacegrad \cross \BA },
\end{dmath}

or
\begin{dmath}\label{eqn:emtLecture10:840}
0 = 
\spacegrad \cross \lr{ \BE + \PD{t}{\BA} },
\end{dmath}

so this curled quantity can be the gradient of something

\begin{dmath}\label{eqn:emtLecture10:860}
\BE + \PD{t}{\BA} = \pm \spacegrad V.
\end{dmath}

We pick negative again, so

\begin{dmath}\label{eqn:emtLecture10:880}
\BE = -\spacegrad V -\PD{t}{\BA}.
\end{dmath}

Observe that when the fields are electrostatic with no time dependence, we would have \( \BE = -\spacegrad V \).

%}
%\EndArticle
\EndNoBibArticle

      %
% Copyright � 2016 Peeter Joot.  All Rights Reserved.
% Licenced as described in the file LICENSE under the root directory of this GIT repository.
%
%{
%\newcommand{\authorname}{Peeter Joot}
\newcommand{\email}{peeterjoot@protonmail.com}
\newcommand{\basename}{FIXMEbasenameUndefined}
\newcommand{\dirname}{notes/FIXMEdirnameUndefined/}

%\renewcommand{\basename}{twoInterfaceNormal}
%%\renewcommand{\dirname}{notes/phy1520/}
%\renewcommand{\dirname}{notes/ece1228-electromagnetic-theory/}
%%\newcommand{\dateintitle}{}
%%\newcommand{\keywords}{}
%
%\newcommand{\authorname}{Peeter Joot}
\newcommand{\onlineurl}{http://sites.google.com/site/peeterjoot2/math2013/\basename.pdf}
\newcommand{\sourcepath}{\dirname\basename.tex}
\newcommand{\generatetitle}[1]{\chapter{#1}}

\newcommand{\vcsinfo}{%
\section*{}
\noindent{\color{DarkOliveGreen}{\rule{\linewidth}{0.1mm}}}
\paragraph{Document version}
%\paragraph{\color{Maroon}{Document version}}
{
\small
\begin{itemize}
\item Available online at:\\ 
\href{\onlineurl}{\onlineurl}
\item Git Repository: \input{./.revinfo/gitRepo.tex}
\item Source: \sourcepath
\item last commit: \input{./.revinfo/gitCommitString.tex}
\item commit date: \input{./.revinfo/gitCommitDate.tex}
\end{itemize}
}
}

%\PassOptionsToPackage{dvipsnames,svgnames}{xcolor}
\PassOptionsToPackage{square,numbers}{natbib}
\documentclass{scrreprt}

\usepackage[left=2cm,right=2cm]{geometry}
\usepackage[svgnames]{xcolor}
\usepackage{peeters_layout}

\usepackage{natbib}

\usepackage[
colorlinks=true,
bookmarks=false,
pdfauthor={\authorname, \email},
backref 
]{hyperref}

% http://tex.stackexchange.com/questions/75773/how-to-reference-problems-by-the-text-label-in-an-exercise-envioronment
\usepackage[english]{cleveref}
\crefname{Exercise}{exercise}{exercises}
\Crefname{Exercise}{Exercise}{Exercises}

\RequirePackage{titlesec}
\RequirePackage{ifthen}

% http://stackoverflow.com/questions/4932910/date-in-the-tabular-environment
\makeatletter
\let\insertdate\@date
\makeatother

\titleformat{\chapter}[display]
{\bfseries\Large}
{\color{DarkSlateGrey}\filleft \authorname
\ifthenelse{\isundefined{\studentnumber}}{}{\\ \studentnumber}
\ifthenelse{\isundefined{\email}}{}{\\ \email}
\ifthenelse{\isundefined{\dateintitle}}{}{\\ \insertdate}
%\ifthenelse{\isundefined{\coursename}}{}{\\ \coursename} % put in title instead.
}
{4ex}
{\color{DarkOliveGreen}{\titlerule}\color{Maroon}
\vspace{2ex}%
\filright}
[\vspace{2ex}%
\color{DarkOliveGreen}\titlerule
]

\newcommand{\beginArtWithToc}[0]{\begin{document}\tableofcontents}
\newcommand{\beginArtNoToc}[0]{\begin{document}}
\newcommand{\EndNoBibArticle}[0]{\end{document}}
\newcommand{\EndArticle}[0]{\bibliography{Bibliography}\bibliographystyle{plainnat}\end{document}}

% 
%\newcommand{\citep}[1]{\cite{#1}}

\colorSectionsForArticle


%
%\usepackage{peeters_layout_exercise}
%\usepackage{peeters_braket}
%\usepackage{peeters_figures}
%\usepackage{siunitx}
%\usepackage{enumerate}
%%\usepackage{mhchem} % \ce{}
%%\usepackage{macros_bm} % \bcM
%%\usepackage{txfonts} % \ointclockwise
%
%\beginArtNoToc
%
\section{Normal transmission and reflection through two interfaces.}
%\chapter{Normal transmission and reflection through two interfaces}
%\label{chap:twoInterfaceNormal}

%\paragraph{Motivation}
%
%In class an outline of normal transmission through a slab was presented.  Let's go through the details.

%\section{Two interfaces, normal incidence.}

The geometry of a two interface configuration is sketched in \cref{fig:l10TwoInterfaces:l10TwoInterfacesFig1}.

\imageFigure{../../figures/ece1228-emt/l10TwoInterfacesFig1}{Two interface transmission.}{fig:l10TwoInterfaces:l10TwoInterfacesFig1}{0.2}

Given a normal incident ray with magnitude \( A \), the respective forward and backwards rays in each the mediums can be written as

\begin{enumerate}[I]
\item
\begin{dmath}\label{eqn:twoInterfaceNormal:20}
\begin{aligned}
\rightarrow &\qquad A e^{-j k_{1z} z} \\
\leftarrow &\qquad A r e^{j k_{1z} z} \\
\end{aligned}
\end{dmath}
\item
\begin{dmath}\label{eqn:twoInterfaceNormal:40}
\begin{aligned}
\rightarrow &\qquad C e^{-j k_{2z} z} \\
\leftarrow &\qquad D e^{j k_{2z} z} \\
\end{aligned}
\end{dmath}
\item
\begin{dmath}\label{eqn:twoInterfaceNormal:60}
\begin{aligned}
\rightarrow &\qquad A t e^{-j k_{3z} (z-d)}
\end{aligned}
\end{dmath}
\end{enumerate}

Matching at \( z = 0 \) gives
\begin{dmath}\label{eqn:twoInterfaceNormal:80}
\begin{aligned}
A t_{12} + r_{21} D &= C \\
A r      &= A r_{12} + D t_{21},
\end{aligned}
\end{dmath}

whereas matching at \( z = d \) gives

\begin{dmath}\label{eqn:twoInterfaceNormal:100}
\begin{aligned}
A t &= C e^{-j k_{2z} d} t_{23} \\
D e^{j k_{2z} d} &= C e^{-j k_{2z} d} r_{23}
\end{aligned}
\end{dmath}

We have four linear equations in four unknowns \( r, t, C, D \), but only care about solving for \( r, t \).  Let's write \(
\gamma = e^{ j k_{2z} d }, C' = C/A, D' = D/A \), for

\begin{dmath}\label{eqn:twoInterfaceNormal:120}
\begin{aligned}
t_{12} + r_{21} D' &= C' \\
r      &= r_{12} + D' t_{21} \\
t \gamma &= C' t_{23} \\
D' \gamma^2 &= C' r_{23}
\end{aligned}
\end{dmath}

Solving for \( C', D' \) we get

\begin{dmath}\label{eqn:twoInterfaceNormal:140}
\begin{aligned}
D' \lr{ \gamma^2 - r_{21} r_{23} } &= t_{12} r_{23} \\
C' \lr{ \gamma^2 - r_{21} r_{23} } &= t_{12} \gamma^2,
\end{aligned}
\end{dmath}

so

\begin{dmath}\label{eqn:twoInterfaceNormal:160}
\begin{aligned}
r &= r_{12} + \frac{t_{12} t_{21} r_{23} }{\gamma^2 - r_{21} r_{23} } \\
t &= t_{23} \frac{ t_{12} \gamma }{\gamma^2 - r_{21} r_{23} }.
\end{aligned}
\end{dmath}

With \( \phi = -j k_{2z} d \), or \( \gamma = e^{-j\phi} \), we have

%\begin{dmath}\label{eqn:twoInterfaceNormal:180}
\boxedEquation{eqn:twoInterfaceNormal:180}{
\begin{aligned}
r &= r_{12} + \frac{t_{12} t_{21} r_{23} e^{2 j \phi} }{1 - r_{21} r_{23} e^{2 j \phi}} \\
t &= \frac{ t_{12} t_{23} e^{j\phi}}{1 - r_{21} r_{23} e^{2 j \phi}}.
\end{aligned}
}
%\end{dmath}

\paragraph{A slab}

When the materials in region I, and III are equal, then \( r_{12} = r_{32} \).  For a TE mode, we have

\begin{equation}\label{eqn:twoInterfaceNormal:200}
r_{12} = 
\frac{\mu_2 k_{1z} - \mu_1 k_{2z}}{\mu_2 k_{1z} + \mu_1 k_{2z}} 
= -r_{21}.
\end{equation}

so the reflection and transmission coefficients are

\begin{dmath}\label{eqn:twoInterfaceNormal:220}
\begin{aligned}
r^{\textrm{TE}} &= r_{12} \lr{ 1 - \frac{t_{12} t_{21} e^{2 j \phi} }{1 - r_{21}^2 e^{2 j \phi}} } \\
t^{\textrm{TE}} &= \frac{ t_{12} t_{21} e^{j\phi}}{1 - r_{21}^2 e^{2 j \phi}}.
\end{aligned}
\end{dmath}

It's possible to produce a matched condition for which \( r_{12} = r_{21} = 0 \), by selecting

\begin{dmath}\label{eqn:twoInterfaceNormal:240}
0 
= \mu_2 k_{1z} - \mu_1 k_{2z}
= \mu_1 \mu_2 \lr{ \inv{\mu_1} k_{1z} - \inv{\mu_2} k_{2z} }
= \mu_1 \mu_2 \omega \lr{ \frac{1}{v_1 \mu_1} \theta_1 - \frac{1}{v_2 \mu_2} \theta_2 },
\end{dmath}

or

\begin{dmath}\label{eqn:twoInterfaceNormal:260}
\inv{\eta_1} \cos\theta_1 = \inv{\eta_2} \cos\theta_2,
\end{dmath}

so the matching condition for normal incidence is just

\begin{dmath}\label{eqn:twoInterfaceNormal:280}
\eta_1 = \eta_2.
\end{dmath}

Given this matched condition, the transmission coefficient for the 1,2 interface is

\begin{dmath}\label{eqn:twoInterfaceNormal:300}
t_{12} 
= \frac{2 \mu_2 k_{1z}}{\mu_2 k_{1z} + \mu_1 k_{2z}}
= \frac{2 \mu_2 k_{1z}}{2 \mu_2 k_{1z} }
= 1,
\end{dmath}

so the matching condition yields
\begin{dmath}\label{eqn:twoInterfaceNormal:320}
t 
= 
t_{12} t_{21} e^{j\phi}
= 
e^{j\phi}
= 
e^{-j k_{2z} d}.
\end{dmath}

Normal transmission through a matched slab only introduces a phase delay.

%}
%\EndNoBibArticle

      %
% Copyright � 2016 Peeter Joot.  All Rights Reserved.
% Licenced as described in the file LICENSE under the root directory of this GIT repository.
%
%{
\newcommand{\authorname}{Peeter Joot}
\newcommand{\email}{peeterjoot@protonmail.com}
\newcommand{\basename}{FIXMEbasenameUndefined}
\newcommand{\dirname}{notes/FIXMEdirnameUndefined/}

\renewcommand{\basename}{brewsters}
%\renewcommand{\dirname}{notes/phy1520/}
\renewcommand{\dirname}{notes/ece1228-electromagnetic-theory/}
%\newcommand{\dateintitle}{}
%\newcommand{\keywords}{}

\newcommand{\authorname}{Peeter Joot}
\newcommand{\onlineurl}{http://sites.google.com/site/peeterjoot2/math2013/\basename.pdf}
\newcommand{\sourcepath}{\dirname\basename.tex}
\newcommand{\generatetitle}[1]{\chapter{#1}}

\newcommand{\vcsinfo}{%
\section*{}
\noindent{\color{DarkOliveGreen}{\rule{\linewidth}{0.1mm}}}
\paragraph{Document version}
%\paragraph{\color{Maroon}{Document version}}
{
\small
\begin{itemize}
\item Available online at:\\ 
\href{\onlineurl}{\onlineurl}
\item Git Repository: \input{./.revinfo/gitRepo.tex}
\item Source: \sourcepath
\item last commit: \input{./.revinfo/gitCommitString.tex}
\item commit date: \input{./.revinfo/gitCommitDate.tex}
\end{itemize}
}
}

%\PassOptionsToPackage{dvipsnames,svgnames}{xcolor}
\PassOptionsToPackage{square,numbers}{natbib}
\documentclass{scrreprt}

\usepackage[left=2cm,right=2cm]{geometry}
\usepackage[svgnames]{xcolor}
\usepackage{peeters_layout}

\usepackage{natbib}

\usepackage[
colorlinks=true,
bookmarks=false,
pdfauthor={\authorname, \email},
backref 
]{hyperref}

% http://tex.stackexchange.com/questions/75773/how-to-reference-problems-by-the-text-label-in-an-exercise-envioronment
\usepackage[english]{cleveref}
\crefname{Exercise}{exercise}{exercises}
\Crefname{Exercise}{Exercise}{Exercises}

\RequirePackage{titlesec}
\RequirePackage{ifthen}

% http://stackoverflow.com/questions/4932910/date-in-the-tabular-environment
\makeatletter
\let\insertdate\@date
\makeatother

\titleformat{\chapter}[display]
{\bfseries\Large}
{\color{DarkSlateGrey}\filleft \authorname
\ifthenelse{\isundefined{\studentnumber}}{}{\\ \studentnumber}
\ifthenelse{\isundefined{\email}}{}{\\ \email}
\ifthenelse{\isundefined{\dateintitle}}{}{\\ \insertdate}
%\ifthenelse{\isundefined{\coursename}}{}{\\ \coursename} % put in title instead.
}
{4ex}
{\color{DarkOliveGreen}{\titlerule}\color{Maroon}
\vspace{2ex}%
\filright}
[\vspace{2ex}%
\color{DarkOliveGreen}\titlerule
]

\newcommand{\beginArtWithToc}[0]{\begin{document}\tableofcontents}
\newcommand{\beginArtNoToc}[0]{\begin{document}}
\newcommand{\EndNoBibArticle}[0]{\end{document}}
\newcommand{\EndArticle}[0]{\bibliography{Bibliography}\bibliographystyle{plainnat}\end{document}}

% 
%\newcommand{\citep}[1]{\cite{#1}}

\colorSectionsForArticle



\usepackage{peeters_layout_exercise}
\usepackage{peeters_braket}
\usepackage{peeters_figures}
\usepackage{siunitx}
%\usepackage{mhchem} % \ce{}
%\usepackage{macros_bm} % \bcM
%\usepackage{macros_qed} % \qedmarker
%\usepackage{txfonts} % \ointclockwise

\beginArtNoToc

\generatetitle{Total internal reflection and Brewster's angles}
%\chapter{Total internal reflection and Brewster's angles}
%\label{chap:brewsters}
% \citep{griffiths1999introduction}

\section{Total internal reflection}

From Snell's second law we have

\begin{dmath}\label{eqn:brewsters:20}
\theta_t = \arcsin\lr{ \frac{n_i}{n_t} \sin\theta_i }.
\end{dmath}

This is plotted in \cref{fig:reflectionForBoth:reflectionForBothFig3}.

\imageFigure{../../figures/ece1228-emt/reflectionForBothFig3}{Transmission angle vs incident angle.}{fig:reflectionForBoth:reflectionForBothFig3}{0.3}

For the \( n_i > n_t \) case, for example, like shining from glass into air, there is a critical incident angle beyond which there is no real value of \( \theta_t \).  That critical incident angle occurs when \( \theta_t = \pi/2 \), which is

\begin{dmath}\label{eqn:brewsters:40}
\sin\theta_{ic} = \frac{n_t}{n_i} \sin(\pi/2).
\end{dmath}

With
\begin{dmath}\label{eqn:brewsters:340}
n = n_t/n_i
\end{dmath}

the critical angle is
\begin{dmath}\label{eqn:brewsters:60}
\theta_{ic} = \arcsin n.
\end{dmath}

Note that Snell's law can also be expressed in terms of this critical angle, allowing for the solution of the transmission angle in a convenient way
\begin{dmath}\label{eqn:brewsters:360}
\sin\theta_i 
= \frac{n_t}{n_i} \sin\theta_t
= n \sin\theta_t
= \sin\theta_{ic} \sin\theta_t,
\end{dmath}

or

\begin{dmath}\label{eqn:brewsters:380}
\sin\theta_t = \frac{\sin\theta_i}{\sin\theta_{ic}}.
\end{dmath}

Still for \( n_i > n_t \), at angles past \( \theta_{ic} \), the transmitted wave angle becomes complex as outlined in 
\citep{jackson1975cew}
, namely

\begin{dmath}\label{eqn:brewsters:400}
\cos^2\theta_t 
= 
1 - \sin^2 \theta_t
= 
1 - 
\frac{\sin^2\theta_i}{\sin^2\theta_{ic}}
= 
-\lr{ 
\frac{\sin^2\theta_i}{\sin^2\theta_{ic}}
-1 
},
\end{dmath}

or
\begin{dmath}\label{eqn:brewsters:420}
\cos\theta_t = 
j \sqrt{ 
\frac{\sin^2\theta_i}{\sin^2\theta_{ic}}
-1 
}.
\end{dmath}

Following the convention that puts the normal propagation direction along z, and the interface along x, the wave vector direction is
\begin{dmath}\label{eqn:brewsters:440}
\kcap_t 
= \Be_3 e^{ \Be_{31} \theta_t }
= \Be_3 \cos\theta_t + \Be_1 \sin\theta_t.
\end{dmath}

The phase factor for the transmitted field is

\begin{dmath}\label{eqn:brewsters:460}
\exp\lr{ j \omega t \pm j \Bk_t \cdot \Bx }
=
\exp\lr{ j \omega t \pm j k \kcap_t \cdot \Bx }
=
\exp\lr{ j \omega t \pm j k \lr{ z \cos\theta_t + x \sin\theta_t } }
=
\exp\lr{ 
   j \omega t 
   \pm j k \lr{ z j \sqrt{ \frac{\sin^2\theta_i}{\sin^2\theta_{ic}} -1 } + x \frac{\sin\theta_i}{\sin\theta_{ic}} } 
}
=
\exp\lr{ 
   j \omega t \pm k
\lr{
    j x \frac{\sin\theta_i}{\sin\theta_{ic}} 
   - z \sqrt{ \frac{\sin^2\theta_i}{\sin^2\theta_{ic}} -1 } 
}
}.
\end{dmath}

The propagation is channelled along the x axis, but the propagation into the second medium decays exponentially (or unphysically grows exponentially), only getting into the surface a small amount.

What is the average power transmission into the medium?  We are interested in the time average of the normal component of the Poynting vector \( \BS \cdot \ncap \).

\begin{dmath}\label{eqn:brewsters:480}
\BS 
= \inv{2} \BE \cross \BH^\conj
= \inv{2} \BE \cross \lr{ \inv{\eta} \kcap_t \cross \BE^\conj }
= -\inv{2 \eta} \BE \cdot \lr{ \kcap_t \wedge \BE^\conj }
= -\inv{2 \eta} \lr{
(\BE \cdot \kcap_t) \BE^\conj
-
\kcap_t \BE \cdot \BE^\conj
}
= 
\inv{2 \eta} 
\kcap_t \Abs{\BE}^2.
\end{dmath}

\begin{dmath}\label{eqn:brewsters:500}
\kcap_t \cdot \ncap
= \lr{ \Be_3 \cos\theta_t + \Be_1 \sin\theta_t } \cdot \Be_3
= \cos\theta_t
= 
j \sqrt{ 
\frac{\sin^2\theta_i}{\sin^2\theta_{ic}}
-1 
}.
\end{dmath}

Note that this is purely imaginary.  The time average real power transmission is

\begin{dmath}\label{eqn:brewsters:520}
\expectation{\BS \cdot \ncap}
=
\Real \lr{ 
j \sqrt{ 
\frac{\sin^2\theta_i}{\sin^2\theta_{ic}}
-1 
}
\frac{1}{2 \eta} \Abs{\BE}^2
}
= 0.
\end{dmath}

There is no power transmission into the second medium at or past the critical angle for total internal reflection.

\section{Brewster's angle}

Brewster's angle is the angle for which there the amplitude of the reflected component of the field is zero.  Recall that when the electric field is parallel(perpendicular) to the plane of incidence, the reflection amplitude (\citep{hecht1998hecht} eq. 4.38)

\begin{dmath}\label{eqn:brewsters:80}
r_\parallel 
=
\frac
{
\frac{ n_t }{\mu_t} \cos \theta_i
-\frac{ n_i }{\mu_i} \cos \theta_t
}
{
\frac{ n_t }{\mu_t} \cos \theta_i
+\frac{ n_i }{\mu_i} \cos \theta_t
}
\end{dmath}
\begin{dmath}\label{eqn:brewsters:100}
r_\perp 
=
\frac
{
\frac{ n_i }{\mu_i} \cos \theta_i
-\frac{ n_t }{\mu_t} \cos \theta_t
}
{
\frac{ n_i }{\mu_i} \cos \theta_i
+\frac{ n_t }{\mu_t} \cos \theta_t
}
\end{dmath}

There are limited conditions for which \( r_\perp \) is zero, at least for \( \mu_i = \mu_t \).  Using Snell's second law \( n_i \sin\theta_i = n_t \sin\theta_t \), that zero is found at

\begin{dmath}\label{eqn:brewsters:120}
n_i \cos \theta_i 
= n_t \cos \theta_t
= n_t \sqrt{ 1 - \sin^2 \theta_t }
= n_t \sqrt{ 1 - \frac{n_i^2}{n_t^2} \sin^2 \theta_i },
\end{dmath}

or

\begin{dmath}\label{eqn:brewsters:140}
\frac{n_i^2}{n_t^2} \cos^2 \theta_i = 1 - \frac{n_i^2}{n_t^2} \sin^2 \theta_i,
\end{dmath}

or
\begin{dmath}\label{eqn:brewsters:160}
\frac{n_i^2}{n_t^2} \lr{ \cos^2 \theta_i + \sin^2 \theta_i } = 1.
\end{dmath}

This has solutions only when \( n_i = \pm n_t \).  The \( n_i = n_t \) case is of no interest, since that is just propagation, so naturally there is no reflection.  The \( n_i = -n_t \) case is possible with the transmission into a negative index of refraction material that is matched in absolute magnitude with the index of refraction in the incident medium.

There are richer solutions for the \( r_\parallel \) zero.  Again considering \( \mu_1 = \mu_2 \) those occur when

\begin{dmath}\label{eqn:brewsters:180}
n_t \cos \theta_i
= n_i \cos \theta_t
= n_i \sqrt{ 1 - \frac{n_i^2}{n_t^2} \sin^2 \theta_i }
= n_i \sqrt{ 1 - \frac{n_i^2}{n_t^2} \sin^2 \theta_i }
\end{dmath}

Let \( n = n_t/n_i \), and square both sides.  This gives

\begin{dmath}\label{eqn:brewsters:200}
n^2 \cos^2 \theta_i 
= 1 - \inv{n^2} \sin^2 \theta_i
= 1 - \inv{n^2} (1 - \cos^2 \theta_i),
\end{dmath}

or

\begin{dmath}\label{eqn:brewsters:220}
\cos^2 \theta_i \lr{ n^2 + \inv{n^2}} = 1 - \inv{n^2},
\end{dmath}

or
\begin{dmath}\label{eqn:brewsters:240}
\cos^2 \theta_i 
= \frac{1 - \inv{n^2}}{ n^2 - \inv{n^2} }
= \frac{n^2 - 1}{ n^4 - 1 }
= \frac{n^2 - 1}{ (n^2 - 1)(n^2 + 1) }
= \frac{1}{ n^2 + 1 }.
\end{dmath}

We also have

\begin{dmath}\label{eqn:brewsters:260}
\sin^2 \theta_i 
=
1 - \frac{1}{ n^2 + 1 }
=
\frac{n^2}{ n^2 + 1 },
\end{dmath}

so
\begin{dmath}\label{eqn:brewsters:280}
\tan^2 \theta_i = n^2,
\end{dmath}

and
\begin{dmath}\label{eqn:brewsters:300}
\tan \theta_{iB} = \pm n,
\end{dmath}

For normal media where \( n_i > 0, n_t > 0 \), only the positive solution is physically relevant, which is

\boxedEquation{eqn:brewsters:320}{
\theta_{iB} = \arctan\lr{ \frac{n_t}{n_i} }.
}

%}
\EndArticle

      \section{Problems}
         %
% Copyright � 2016 Peeter Joot.  All Rights Reserved.
% Licenced as described in the file LICENSE under the root directory of this GIT repository.
%
%{
%\newcommand{\authorname}{Peeter Joot}
\newcommand{\email}{peeterjoot@protonmail.com}
\newcommand{\basename}{FIXMEbasenameUndefined}
\newcommand{\dirname}{notes/FIXMEdirnameUndefined/}

%\renewcommand{\basename}{fresnelSumAndDifferenceAngleFormulas}
%%\renewcommand{\dirname}{notes/phy1520/}
%\renewcommand{\dirname}{notes/ece1228-electromagnetic-theory/}
%%\newcommand{\dateintitle}{}
%%\newcommand{\keywords}{}
%
%\newcommand{\authorname}{Peeter Joot}
\newcommand{\onlineurl}{http://sites.google.com/site/peeterjoot2/math2013/\basename.pdf}
\newcommand{\sourcepath}{\dirname\basename.tex}
\newcommand{\generatetitle}[1]{\chapter{#1}}

\newcommand{\vcsinfo}{%
\section*{}
\noindent{\color{DarkOliveGreen}{\rule{\linewidth}{0.1mm}}}
\paragraph{Document version}
%\paragraph{\color{Maroon}{Document version}}
{
\small
\begin{itemize}
\item Available online at:\\ 
\href{\onlineurl}{\onlineurl}
\item Git Repository: \input{./.revinfo/gitRepo.tex}
\item Source: \sourcepath
\item last commit: \input{./.revinfo/gitCommitString.tex}
\item commit date: \input{./.revinfo/gitCommitDate.tex}
\end{itemize}
}
}

%\PassOptionsToPackage{dvipsnames,svgnames}{xcolor}
\PassOptionsToPackage{square,numbers}{natbib}
\documentclass{scrreprt}

\usepackage[left=2cm,right=2cm]{geometry}
\usepackage[svgnames]{xcolor}
\usepackage{peeters_layout}

\usepackage{natbib}

\usepackage[
colorlinks=true,
bookmarks=false,
pdfauthor={\authorname, \email},
backref 
]{hyperref}

% http://tex.stackexchange.com/questions/75773/how-to-reference-problems-by-the-text-label-in-an-exercise-envioronment
\usepackage[english]{cleveref}
\crefname{Exercise}{exercise}{exercises}
\Crefname{Exercise}{Exercise}{Exercises}

\RequirePackage{titlesec}
\RequirePackage{ifthen}

% http://stackoverflow.com/questions/4932910/date-in-the-tabular-environment
\makeatletter
\let\insertdate\@date
\makeatother

\titleformat{\chapter}[display]
{\bfseries\Large}
{\color{DarkSlateGrey}\filleft \authorname
\ifthenelse{\isundefined{\studentnumber}}{}{\\ \studentnumber}
\ifthenelse{\isundefined{\email}}{}{\\ \email}
\ifthenelse{\isundefined{\dateintitle}}{}{\\ \insertdate}
%\ifthenelse{\isundefined{\coursename}}{}{\\ \coursename} % put in title instead.
}
{4ex}
{\color{DarkOliveGreen}{\titlerule}\color{Maroon}
\vspace{2ex}%
\filright}
[\vspace{2ex}%
\color{DarkOliveGreen}\titlerule
]

\newcommand{\beginArtWithToc}[0]{\begin{document}\tableofcontents}
\newcommand{\beginArtNoToc}[0]{\begin{document}}
\newcommand{\EndNoBibArticle}[0]{\end{document}}
\newcommand{\EndArticle}[0]{\bibliography{Bibliography}\bibliographystyle{plainnat}\end{document}}

% 
%\newcommand{\citep}[1]{\cite{#1}}

\colorSectionsForArticle


%
%\usepackage{peeters_layout_exercise}
%\usepackage{peeters_braket}
%\usepackage{peeters_figures}
%\usepackage{siunitx}
%%\usepackage{mhchem} % \ce{}
%%\usepackage{macros_bm} % \bcM
%%\usepackage{txfonts} % \ointclockwise
%
%\beginArtNoToc
%
%\generatetitle{Fresnel angular sum and difference formulas}
%\chapter{Fresnel angular sum and difference formulas}
%\label{chap:fresnelSumAndDifferenceAngleFormulas}

\makeoproblem{Fresnel sum and difference formulas.}{problem:fresnelSumAndDifferenceAngleFormulas:1}{\citep{hecht1998hecht} pr. 4.39}{

Given a \( \mu_1 = \mu_2 \) constraint, show that the Fresnel equations have the form

\begin{subequations}
\label{eqn:fresnelSumAndDifferenceAngleFormulas:260}
\begin{dmath}\label{eqn:fresnelSumAndDifferenceAngleFormulas:280}
r^{\textrm{TE}}
=
\frac {
\sin( \theta_t - \theta_i )
} {
\sin( \theta_t + \theta_i )
}
\end{dmath}
\begin{dmath}\label{eqn:fresnelSumAndDifferenceAngleFormulas:300}
r^{\textrm{TM}}
=
\frac
{\tan(\theta_i -\theta_t)}
{\tan(\theta_i +\theta_t)}
\end{dmath}
\begin{dmath}\label{eqn:fresnelSumAndDifferenceAngleFormulas:320}
t^{\textrm{TE}}
= \frac{ 2  \sin\theta_t \cos\theta_i }
{ \sin(\theta_i + \theta_t) }
\end{dmath}
\begin{dmath}\label{eqn:fresnelSumAndDifferenceAngleFormulas:340}
t^{\textrm{TM}}
=
{ \sin(\theta_i + \theta_t) \cos(\theta_i - \theta_t) }.
\end{dmath}
\end{subequations}
} % problem

\makeanswer{problem:fresnelSumAndDifferenceAngleFormulas:1}{

We need a couple trig identities to start with.

\begin{dmath}\label{eqn:fresnelSumAndDifferenceAngleFormulas:20}
\sin(a + b)
=
\Imag\lr{ e^{j(a + b)} }
=
\Imag\lr{
e^{ja} e^{+ jb}
}
=
\Imag\lr{
(\cos a + j \sin a) (\cos b + j \sin b)
}
=
\sin a \cos b + \cos a \sin b.
\end{dmath}

Allowing for both signs we have

\begin{dmath}\label{eqn:fresnelSumAndDifferenceAngleFormulas:240}
\begin{aligned}
\sin(a + b) &= \sin a \cos b + \cos a \sin b \\
\sin(a - b) &= \sin a \cos b - \cos a \sin b.
\end{aligned}
\end{dmath}

The mixed sine and cosine product can be expressed as a sum of sines

\begin{dmath}\label{eqn:fresnelSumAndDifferenceAngleFormulas:40}
2 \sin a \cos b = \sin(a + b) + \sin(a - b).
\end{dmath}

With \( 2 x = a + b, 2 y = a - b \), or \( a = x + y, b = x - y \), we find

\begin{dmath}\label{eqn:fresnelSumAndDifferenceAngleFormulas:60}
\begin{aligned}
2 \sin(x + y) \cos (x - y) &= \sin( 2 x ) + \sin( 2 y ) \\
2 \sin(x - y) \cos (x + y) &= \sin( 2 x ) - \sin( 2 y ).
\end{aligned}
\end{dmath}

Returning to the problem.  When \( \mu_1 = \mu_2 \) the Fresnel equations were found to be

\begin{dmath}\label{eqn:fresnelSumAndDifferenceAngleFormulas:100}
\begin{aligned}
r^{\textrm{TE}} &= \frac { n_1 \cos\theta_i - n_2 \cos\theta_t } { n_1 \cos\theta_i + n_2 \cos\theta_t } \\
r^{\textrm{TM}} &= \frac{n_2 \cos\theta_i - n_1 \cos\theta_t }{ n_2 \cos\theta_i + n_1 \cos\theta_t } \\
t^{\textrm{TE}} &= \frac{ 2 n_1 \cos\theta_i } { n_1 \cos\theta_i + n_2 \cos\theta_t } \\
t^{\textrm{TM}} &= \frac{2 n_1 \cos\theta_i }{ n_2 \cos\theta_i + n_1 \cos\theta_t }.
\end{aligned}
\end{dmath}

Using Snell's law, one of \( n_1, n_2 \) can be eliminated, for example

\begin{dmath}\label{eqn:fresnelSumAndDifferenceAngleFormulas:120}
n_1 = n_2 \frac{\sin \theta_t}{\sin\theta_i}.
\end{dmath}

Inserting this and proceeding with the application of the trig identities above, we have

\begin{subequations}
\label{eqn:fresnelSumAndDifferenceAngleFormulas:140}
\begin{dmath}\label{eqn:fresnelSumAndDifferenceAngleFormulas:160}
r^{\textrm{TE}}
= \frac { n_2 \frac{\sin\theta_t}{\sin\theta_i} \cos\theta_i - n_2 \cos\theta_t } { n_2 \frac{\sin\theta_t}{\sin\theta_i} \cos\theta_i + n_2 \cos\theta_t }
=
\frac {
\sin\theta_t \cos\theta_i - \cos\theta_t \sin\theta_i
} {
\sin\theta_t \cos\theta_i + \cos\theta_t \sin\theta_i
}
=
\frac {
\sin( \theta_t - \theta_i )
} {
\sin( \theta_t + \theta_i )
}
\end{dmath}
\begin{dmath}\label{eqn:fresnelSumAndDifferenceAngleFormulas:180}
r^{\textrm{TM}}
= \frac{n_2 \cos\theta_i - n_2 \frac{\sin\theta_t}{\sin\theta_i} \cos\theta_t }{ n_2 \cos\theta_i + n_2 \frac{\sin\theta_t}{\sin\theta_i} \cos\theta_t }
= \frac{
\sin\theta_i \cos\theta_i - \sin\theta_t \cos\theta_t
}{
\sin\theta_i \cos\theta_i + \sin\theta_t \cos\theta_t
}
= \frac{\inv{2} \sin(2 \theta_i) -  \inv{2} \sin(2 \theta_t) }{ \inv{2} \sin(2 \theta_i) +  \inv{2} \sin(2 \theta_t) }
= \frac
{\sin(\theta_i - \theta_t)\cos(\theta_i + \theta_t) }
{\sin(\theta_i + \theta_t)\cos(\theta_i - \theta_t) }
=
\frac
{\tan(\theta_i -\theta_t)}
{\tan(\theta_i +\theta_t)}
\end{dmath}
\begin{dmath}\label{eqn:fresnelSumAndDifferenceAngleFormulas:200}
t^{\textrm{TE}}
= \frac{ 2 n_2 \frac{\sin\theta_t}{\sin\theta_i} \cos\theta_i } { n_2 \frac{\sin\theta_t}{\sin\theta_i} \cos\theta_i + n_2 \cos\theta_t }
= \frac{ 2  \sin\theta_t \cos\theta_i } { \sin\theta_t \cos\theta_i + \cos\theta_t \sin\theta_i }
= \frac{ 2  \sin\theta_t \cos\theta_i }
{ \sin(\theta_i + \theta_t) }
\end{dmath}
\begin{dmath}\label{eqn:fresnelSumAndDifferenceAngleFormulas:220}
t^{\textrm{TM}}
= \frac{2 n_2 \frac{\sin\theta_t}{\sin\theta_i} \cos\theta_i }{ n_2 \cos\theta_i + n_2 \frac{\sin\theta_t}{\sin\theta_i} \cos\theta_t }
= \frac{2  \sin\theta_t \cos\theta_i }{ \sin\theta_i \cos\theta_i +  \sin\theta_t \cos\theta_t }
= \frac{2  \sin\theta_t \cos\theta_i }
{ \inv{2} \sin(2 \theta_i) +  \inv{2} \sin(2 \theta_t) }
= \frac{2 \sin\theta_t \cos\theta_i }
{ \sin(\theta_i + \theta_t) \cos(\theta_i - \theta_t) }
\end{dmath}
\end{subequations}
} % answer

%}
%\EndArticle

         %
% Copyright � 2016 Peeter Joot.  All Rights Reserved.
% Licenced as described in the file LICENSE under the root directory of this GIT repository.
%
\makeproblem{Fresnel TM equations.}{emt:problemSet8:1}{
For the geometry shown in \cref{fig:ps8:ps8Fig1}, obtain the TM (E)
Fresnel reflection and transmission coefficients. Express your
results in terms of the propagation constant \( k_{1z} \) and
and \( k_{2z} \),
(i.e., the projection of
\( \Bk_1 \) and
\( \Bk_2 \)
along z-direction.) Note that the
interface is at \( z=0 \) plane.

\imageFigure{../../figures/ece1228-emt/ps8Fig1}{TM mode geometry.}{fig:ps8:ps8Fig1}{0.2}
} % makeproblem

\makeanswer{emt:problemSet8:1}{

From the figure, with \( i = \Be_3 \Be_1 \) the propagation unit vectors are
\begin{subequations}
\label{eqn:emtproblemSet8Problem1:20}
\begin{dmath}\label{eqn:emtproblemSet8Problem1:40}
\kcap_1
= \Be_3 e^{i \theta_1}
= \Be_3 \cos\theta_1 + \Be_1 \sin\theta_1
\end{dmath}
\begin{dmath}\label{eqn:emtproblemSet8Problem1:60}
\kcap_1'
= -\Be_3 e^{-i \theta_1'}
= -\Be_3 \cos\theta_1' + \Be_1 \sin\theta_1'
\end{dmath}
\begin{dmath}\label{eqn:emtproblemSet8Problem1:80}
\kcap_2
= \Be_3 e^{i \theta_2}
= \Be_3 \cos\theta_2 + \Be_1 \sin\theta_2
\end{dmath}
\end{subequations}

Recall that Faraday's law shows that \( \Bk, \BE, \BH \) is a right handed triple.  In particular

\begin{dmath}\label{eqn:emtproblemSet8Problem1:100}
-j \omega \mu \BH
=
\spacegrad \cross \BE
=
-\BE_0 \cross \spacegrad e^{j \omega t - j \Bk \cdot \Bx}
=
-\BE_0 \cross (-j \Bk) e^{j \omega t - j \Bk \cdot \Bx}
=
j \BE \cross \Bk,
\end{dmath}

or
\begin{dmath}\label{eqn:emtproblemSet8Problem1:120}
\BH
=
\inv{-j \omega \mu} j \BE \cross \Bk
=
\inv{\omega \mu} \Bk \cross \BE.
=
\inv{\eta} \kcap \cross \BE.
\end{dmath}

This means that \( \BH_i, \BH_t \) must be directed along the \( +\Be_2 \) direction, whereas \( \BH_r \) is directed in the \( -\Be_2 \) direction.  Note that the phase of this reflected magnetic field is opposite to what might be considered a natural choice, so we should that \( r \) is negative compared to a reference that picks the opposite phase convention.

The electric field directions from the figure are

\begin{subequations}
\label{eqn:emtproblemSet8Problem1:140}
\begin{dmath}\label{eqn:emtproblemSet8Problem1:160}
\Ecap_i
=
\kcap_1 i
= \Be_3 e^{i \theta_1} i
= \Be_1 e^{i \theta_1}
= \Be_1 \cos\theta_1 - \Be_3 \sin\theta_1
\end{dmath}
\begin{dmath}\label{eqn:emtproblemSet8Problem1:180}
\Ecap_r
=
\kcap_1' (-i)
= -\Be_3 e^{-i \theta_1'} (-i)
= \Be_1 e^{-i \theta_1'}
= \Be_1 \cos\theta_1' + \Be_3 \sin\theta_1'
\end{dmath}
\begin{dmath}\label{eqn:emtproblemSet8Problem1:200}
\Ecap_t
=
\kcap_2 i
= \Be_3 e^{i \theta_2} i
= \Be_1 e^{i \theta_2}
= \Be_1 \cos\theta_2 - \Be_3 \sin\theta_2
\end{dmath}
\end{subequations}

The boundary value conditions, with \( \ncap = \Be_3 \), are
\begin{dmath}\label{eqn:emtproblemSet8Problem1:220}
\begin{aligned}
\ncap \cross \lr{ \BH_1 - \BH_2 } &= 0 \\
\ncap \cdot \lr{ \BB_1 - \BB_2 } &= 0 \\
\ncap \cross \lr{ \BE_1 - \BE_2 } &= 0 \\
\ncap \cdot \lr{ \BD_1 - \BD_2 } &= 0,
\end{aligned}
\end{dmath}

where
\begin{subequations}
\label{eqn:emtproblemSet8Problem1:240}
\begin{dmath}\label{eqn:emtproblemSet8Problem1:260}
\BE_1
=
E_0 \lr{ \Be_1 \cos\theta_1 - \Be_3 \sin\theta_1 } e^{-j \Bk_1 \cdot \Bx }
+
E_0 r \lr{ \Be_1 \cos\theta_1' + \Be_3 \sin\theta_1' } e^{-j \Bk_1' \cdot \Bx }
\end{dmath}
\begin{dmath}\label{eqn:emtproblemSet8Problem1:280}
\BE_2
=
E_0 t \lr{ \Be_1 \cos\theta_2 - \Be_3 \sin\theta_2 } e^{-j \Bk_2 \cdot \Bx }
\end{dmath}
\begin{dmath}\label{eqn:emtproblemSet8Problem1:300}
\BH_1
=
\Be_2 \frac{E_0}{\eta_1} e^{-j \Bk_1 \cdot \Bx }
-
\Be_2 \frac{E_0 r}{\eta_1} e^{-j \Bk_1' \cdot \Bx }
\end{dmath}
\begin{dmath}\label{eqn:emtproblemSet8Problem1:320}
\BH_2
=
\Be_2 \frac{E_0 t}{\eta_2} e^{-j \Bk_2 \cdot \Bx }.
\end{dmath}
\end{subequations}

The boundary value constraints can be seen to resolve to the following set of scalar equations

\begin{subequations}
\label{eqn:emtproblemSet8Problem1:340}
\begin{dmath}\label{eqn:emtproblemSet8Problem1:360}
\frac{E_0}{\eta_1} e^{-j \Bk_1 \cdot \Bx}
-\frac{E_0 r}{\eta_1} e^{-j \Bk_1' \cdot \Bx}
=
\frac{t E_0}{\eta_2} e^{-j \Bk_2 \cdot \Bx}
\end{dmath}
\begin{dmath}\label{eqn:emtproblemSet8Problem1:380}
E_0 \cos\theta_1 e^{-j \Bk_1 \cdot \Bx } + E_0 r \cos\theta_1' e^{-j \Bk_1' \cdot \Bx }
=
E_0 t \cos\theta_2 e^{-j \Bk_2 \cdot \Bx }
\end{dmath}
\begin{dmath}\label{eqn:emtproblemSet8Problem1:400}
-\epsilon_1 E_0 \sin\theta_1 e^{-j \Bk_1 \cdot \Bx }
+
\epsilon_1 E_0 r \sin\theta_1' e^{-j \Bk_1' \cdot \Bx }
=
-\epsilon_2 E_0 t \sin\theta_2 e^{-j \Bk_2 \cdot \Bx }
\end{dmath}
\end{subequations}

where equality is required at all points \( \Bx = x \Be_1 \) along the \( z = 0 \) axis.
In order for the phase factors to cancel out, as they do at the origin, we require

\begin{equation}\label{eqn:emtproblemSet8Problem1:420}
\Bk_1 \cdot \Be_1 = \Bk_1' \cdot \Be_1 = \Bk_2 \cdot \Be_2.
\end{equation}

The \( \Bk_1, \Bk_1' \) equality is Snell's first law, a requirement that the incident angle equals the reflection angle

\begin{dmath}\label{eqn:emtproblemSet8Problem1:440}
k_1 \sin\theta_1 = k_1 \sin\theta_1'.
\end{dmath}

The remaining phase equality is Snell's second law in disguise
\begin{dmath}\label{eqn:emtproblemSet8Problem1:460}
0
= \Bk_1 \cdot \Be_1 - \Bk_2 \cdot \Be_2
= k_1 \sin\theta_1 - k_2 \sin\theta_2
= \frac{\omega}{v_1} \sin\theta_1 - \frac{\omega}{v_2} \sin\theta_2
= \frac{\omega}{c}\frac{c}{v_1} \sin\theta_1 - \frac{\omega}{c}\frac{c}{v_2} \sin\theta_2
= \frac{\omega}{c} \lr{ n_1 \sin\theta_1 - n_2 \sin\theta_2 },
\end{dmath}

or
\begin{dmath}\label{eqn:emtproblemSet8Problem1:480}
n_1 \sin\theta_1 = n_2 \sin\theta_2.
\end{dmath}

With equality of all the phase terms, we are left with

\begin{subequations}
\label{eqn:emtproblemSet8Problem1:500}
\begin{dmath}\label{eqn:emtproblemSet8Problem1:520}
\frac{1}{\eta_1}
-\frac{r}{\eta_1}
=
\frac{t}{\eta_2}
\end{dmath}
\begin{dmath}\label{eqn:emtproblemSet8Problem1:540}
\cos\theta_1(1 + r)
=
t \cos\theta_2
\end{dmath}
\begin{dmath}\label{eqn:emtproblemSet8Problem1:560}
-\epsilon_1 \sin\theta_1 (1 - r)
=
-\epsilon_2 t \sin\theta_2
\end{dmath}
\end{subequations}

Since \( \epsilon \eta = \sqrt{\epsilon\mu} = n/c \), one of these is redundant since the first and last just re-express Snell's law.  That leaves two equations in two unknowns (\(r,t\))

\begin{dmath}\label{eqn:emtproblemSet8Problem1:580}
\begin{bmatrix}
r \\
t
\end{bmatrix}
=
{
\begin{bmatrix}
1 & \eta_1/\eta_2 \\
-\cos\theta_1 & \cos\theta_2
\end{bmatrix}
}^{-1}
\begin{bmatrix}
1 \\
\cos\theta_1
\end{bmatrix}
=
\inv{ \eta_1 \cos\theta_1 + \eta_2 \cos\theta_2 }
\begin{bmatrix}
\eta_2 \cos\theta_2 & -\eta_1 \\
\eta_2 \cos\theta_1 & \eta_2
\end{bmatrix}
\begin{bmatrix}
1 \\
\cos\theta_1
\end{bmatrix},
\end{dmath}

or
\begin{dmath}\label{eqn:emtproblemSet8Problem1:600}
\begin{aligned}
r &=
\frac{ \eta_2 \cos\theta_2 -\eta_1 \cos\theta_1 }
{ \eta_1 \cos\theta_1 + \eta_2 \cos\theta_2 } \\
t &=
\frac{ 2 \eta_2 \cos\theta_1 }
{ \eta_1 \cos\theta_1 + \eta_2 \cos\theta_2 }.
\end{aligned}
\end{dmath}

As expected, this reflection coefficient has a different sign, than \citep{hecht1998hecht} (4.38), where the magnetic fields were all aligned along \( +\Be_2 \).

To express this in terms of \( k_{1z}, k_{2z} \) we have to rewrite expressions of the form

\begin{dmath}\label{eqn:emtproblemSet8Problem1:620}
\eta_1 \cos\theta_1
=
\frac{\eta_1}{k_1} k_{1z}
=
\frac{\eta_1 v_1}{\omega} k_{1z}
=
\frac{k_{1z}}{\omega} \sqrt{\frac{\mu_1}{\epsilon_1}} \inv{\sqrt{\epsilon_1 \mu_1}}
=
\frac{k_{1z}}{\epsilon_1 \omega}.
\end{dmath}

This gives
\begin{dmath}\label{eqn:emtproblemSet8Problem1:640}
\begin{aligned}
r &=
\frac{  \frac{k_{2z}}{\epsilon_2} - \frac{k_{1z}}{\epsilon_1} }
{  \frac{k_{1z}}{\epsilon_1} +  \frac{k_{2z}}{\epsilon_2} } \\
t &=
\frac{ 2 \frac{\eta_2}{\eta_1}  \frac{k_{1z}}{\epsilon_1} }
{  \frac{k_{1z}}{\epsilon_1} +  \frac{k_{2z}}{\epsilon_2} },
\end{aligned}
\end{dmath}

or
\boxedEquation{eqn:emtproblemSet8Problem1:660}{
\begin{aligned}
r &=
\frac{  \epsilon_1 k_{2z} - \epsilon_2 k_{1z} }
{  \epsilon_2 k_{1z} +  \epsilon_1 k_{2z} } \\
t &=
\frac{ 2 \frac{\eta_2}{\eta_1}  \epsilon_2 k_{1z} }
{  \epsilon_2 k_{1z} +  \epsilon_1 k_{2z} }.
\end{aligned}
}

}

         \input{Set8Problem2.tex}
         \input{Set8Problem3.tex}
         \input{Set9Problem1.tex}
         %\input{Set9Problem2.tex}
   \chapter{Gauge freedom.}
      \section{Problems}
         \input{Set9Problem3.tex}
