%
% Copyright � 2013 Peeter Joot.  All Rights Reserved.
% Licenced as described in the file LICENSE under the root directory of this GIT repository.
%
\makeoproblem{Drude conductivity formula}{condensedMatter:problemSet10:1}{2013 ps10 p1}{
The density of point-like impurities in a metal can be characterized by a `mean free path,' \(l_\nought\), of the conduction electrons, this being the distance between scattering centres that randomize the velocity.
\index{mean free path}
\index{Drude model}

\makesubproblem{}{condensedMatter:problemSet10:1a}

Assuming that electrons travel at the Fermi velocity between scattering centres in a
free electron metal, show that the Drude formula for the conductivity can be rewritten
as:

\begin{equation}\label{eqn:condensedMatterProblemSet10Problem1:20}
\sigma = \frac{\kF^2 e^2 l_\nought}{3 \Hbar \pi^2}
\end{equation}

\makesubproblem{}{condensedMatter:problemSet10:1b}
A sample of copper has a residual resistivity below \(1 \Unit{K}\) of \(10^{-8} \Omega \Unit{m}\). (Note: resistivity \(\rho = 1/\sigma\).) Treating copper as a free electron metal with a spherical Fermi surface accommodating one charge carrier per copper atom, estimate \(l_\nought\) below \(1 \Unit{K}\) for this sample of copper. (Free-electron parameters of copper are given in \citep{ibach2009solid} Table 6.1)

\makesubproblem{}{condensedMatter:problemSet10:1c}
Calculate the mean scattering time \(\tau\) for this sample of copper below \(1 \Unit{K}\).
\index{mean free time}

} % makeproblem

\makeanswer{condensedMatter:problemSet10:1}{
\makeSubAnswer{}{condensedMatter:problemSet10:1a}

The Fermi velocity is

\begin{equation}\label{eqn:condensedMatterProblemSet10Problem1:40}
\vF = \frac{\pF}{m} = \frac{ \Hbar \kF}{m^\conj},
\end{equation}

so that the `mean free path' is

\begin{equation}\label{eqn:condensedMatterProblemSet10Problem1:60}
l_\nought = \vF \tau = \frac{ \Hbar \kF \tau}{m^\conj}.
\end{equation}

Putting these all together, the conductivity as given by \eqnref{eqn:condensedMatterProblemSet10Problem1:20} is

\begin{dmath}\label{eqn:condensedMatterProblemSet10Problem1:80}
\sigma
= \frac{\kF^2 e^2}{3 \cancel{\Hbar} \pi^2} \frac{ \cancel{\Hbar} \kF \tau}{m^\conj}
= \frac{\kF^3}{3 \pi^2} \frac{ e^2 \tau}{m^\conj}
= n \frac{ e^2 \tau}{m^\conj},
\end{dmath}

which recovers the form we derived in class.

\makeSubAnswer{}{condensedMatter:problemSet10:1b}

Using the tabulated info

\begin{equation}\label{eqn:condensedMatterProblemSet10Problem1:100}
l_\nought
= \frac{ 3 \pi^2 \Hbar }{ \kF^2 e^2 \rho }
=
\frac
{
3 \pi^2 \lr{ 1.05 \times 10^{-34} \Unit{ J s } }
}
{
\lr{ 1.36 \times \frac{10^8}{10^{-2} \Unit{m}} }^2
\lr{ 1.6 \times 10^{-19} \Unit{ C } }^2
\lr{ 10^{-8} \Unitfrac{ J s}{C^2} \Unit{m} }
}.
%= 6.57 \times 10^{-8} \Unit{m},
\end{equation}

This is

\boxedEquation{eqn:condensedMatterProblemSet10Problem1:120}{
l_\nought
= 6.57 \times 10^{-6} \Unit{cm}.
}

With copper having an FCC lattice constant of \(\sim 3.6 \angstrom\), the number of atoms that an electron moves past before a collision is on the order of \(10^2\).

\makeSubAnswer{}{condensedMatter:problemSet10:1c}

The mean scattering time is

\begin{dmath}\label{eqn:condensedMatterProblemSet10Problem1:140}
\tau
= \frac{l_\nought}{\vF}
= \frac{ 6.57 \times 10^{-6} \Unit{cm}}
{
1.57 \times 10^8 \Unitfrac{cm}{s}
}
\end{dmath}

This is

\boxedEquation{eqn:condensedMatterProblemSet10Problem1:160}{
\tau
=
4.18 \times 10^{-14} \Unit{s},
}

a surprisingly fast seeming time until the magnitude of \(\vF \approx 0.005 c\) is considered.
}
