%
% Copyright � 2013 Peeter Joot.  All Rights Reserved.
% Licenced as described in the file LICENSE under the root directory of this GIT repository.
%
\makeoproblem{Laue vs.\ powder diffraction}{condensedMatter:problemSet4:1}{2013 ps4 p1}{

This is based on assigned reading, \S 3.7, pp.\ 69-70.

You are given a single crystal of an unknown material and asked to determine the
lattice constants.  Should you use Laue or powder x-ray diffraction to
do this?  Explain why.   (Note that in order to do powder diffraction you would
have to crush the single crystal.  Assume that you are allowed to do this.)

} % makeproblem

\makeanswer{condensedMatter:problemSet4:1}{

We should use powder diffraction to determine the lattice constants of this unknown material.

For a material of known structure we can use Laue diffraction to determine the orientation of the lattice, for example, to prepare a hexagonal prism sample of a crystal so that the planes of the sample are oriented in a fashion that matches the internal structure.  Laue diffraction employs a continuous spectrum of x-rays, allowing for observation of all the reflections with lattice points for which the Ewald sphere intersections lie within the range of \(\Bk_\nought\) values of this radiation.  That is sufficient to observe the symmetries associated with the orientation of the crystal.

\paragraph{Grading remark}: ``Why not lattice constants?''  Reviewing the grading remarks, it appears that an explicit reference to the Bragg condition was desired here.  The \(\BG = \BK\) condition required for Laue diffraction corresponds \(n \lambda = 2 d_{h k l} \sin\theta\).  Since Laue diffraction is using all wavelengths, when we get a spot we do not know the specific wavelength and thus cannot extract \(d\), since that requires knowing both \(\theta\) and \(\lambda\).  Conversely when we use powder diffraction where \(\lambda\) is fixed (and we measure \(\theta\)) we can determine \(d\) (or integer multiples of it?) from the Bragg condition.

When using a fixed wavelength source (and thus a source with fixed \(k_\nought\)), we obtain diffraction only when that source is suitably oriented so that the \(\Bk_\nought\) passes through lattice points on the Ewald sphere.  When that specific orientation is available it should allow for accurate determination of the lattice constants associated with that orientation, allowing a subset of the lattice structure to be determined accurately.  The powder diffraction method allows for that internal structure to be measured by using a sample of the crystal that has been broken into pieces small enough that all possible orientations of the crystal are present.  Given that distribution of orientations, a fixed wavelength source (and thus fixed \(k_\nought\)) can be used to obtain reflections for all the Ewald sphere orientations passing through the reciprocal lattice points.
}
