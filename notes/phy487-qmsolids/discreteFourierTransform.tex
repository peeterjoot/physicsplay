%
% Copyright � 2013 Peeter Joot.  All Rights Reserved.
% Licenced as described in the file LICENSE under the root directory of this GIT repository.
%
%\newcommand{\authorname}{Peeter Joot}
\newcommand{\email}{peeterjoot@protonmail.com}
\newcommand{\basename}{FIXMEbasenameUndefined}
\newcommand{\dirname}{notes/FIXMEdirnameUndefined/}

%\renewcommand{\basename}{discreteFourierTransform}
%\renewcommand{\dirname}{notes/phy487/}
%%\newcommand{\dateintitle}{}
%\newcommand{\keywords}{Discrete Fourier transform, PHY487, Kronecker delta}
%
%\newcommand{\authorname}{Peeter Joot}
\newcommand{\onlineurl}{http://sites.google.com/site/peeterjoot2/math2013/\basename.pdf}
\newcommand{\sourcepath}{\dirname\basename.tex}
\newcommand{\generatetitle}[1]{\chapter{#1}}

\newcommand{\vcsinfo}{%
\section*{}
\noindent{\color{DarkOliveGreen}{\rule{\linewidth}{0.1mm}}}
\paragraph{Document version}
%\paragraph{\color{Maroon}{Document version}}
{
\small
\begin{itemize}
\item Available online at:\\ 
\href{\onlineurl}{\onlineurl}
\item Git Repository: \input{./.revinfo/gitRepo.tex}
\item Source: \sourcepath
\item last commit: \input{./.revinfo/gitCommitString.tex}
\item commit date: \input{./.revinfo/gitCommitDate.tex}
\end{itemize}
}
}

%\PassOptionsToPackage{dvipsnames,svgnames}{xcolor}
\PassOptionsToPackage{square,numbers}{natbib}
\documentclass{scrreprt}

\usepackage[left=2cm,right=2cm]{geometry}
\usepackage[svgnames]{xcolor}
\usepackage{peeters_layout}

\usepackage{natbib}

\usepackage[
colorlinks=true,
bookmarks=false,
pdfauthor={\authorname, \email},
backref 
]{hyperref}

% http://tex.stackexchange.com/questions/75773/how-to-reference-problems-by-the-text-label-in-an-exercise-envioronment
\usepackage[english]{cleveref}
\crefname{Exercise}{exercise}{exercises}
\Crefname{Exercise}{Exercise}{Exercises}

\RequirePackage{titlesec}
\RequirePackage{ifthen}

% http://stackoverflow.com/questions/4932910/date-in-the-tabular-environment
\makeatletter
\let\insertdate\@date
\makeatother

\titleformat{\chapter}[display]
{\bfseries\Large}
{\color{DarkSlateGrey}\filleft \authorname
\ifthenelse{\isundefined{\studentnumber}}{}{\\ \studentnumber}
\ifthenelse{\isundefined{\email}}{}{\\ \email}
\ifthenelse{\isundefined{\dateintitle}}{}{\\ \insertdate}
%\ifthenelse{\isundefined{\coursename}}{}{\\ \coursename} % put in title instead.
}
{4ex}
{\color{DarkOliveGreen}{\titlerule}\color{Maroon}
\vspace{2ex}%
\filright}
[\vspace{2ex}%
\color{DarkOliveGreen}\titlerule
]

\newcommand{\beginArtWithToc}[0]{\begin{document}\tableofcontents}
\newcommand{\beginArtNoToc}[0]{\begin{document}}
\newcommand{\EndNoBibArticle}[0]{\end{document}}
\newcommand{\EndArticle}[0]{\bibliography{Bibliography}\bibliographystyle{plainnat}\end{document}}

% 
%\newcommand{\citep}[1]{\cite{#1}}

\colorSectionsForArticle


%
%\beginArtNoToc
%
%\generatetitle{Discrete Fourier transform}
\label{chap:discreteFourierTransform}
\index{discrete Fourier transform}

For decoupling trial solutions of lattice vibrations we have what appears to be a need for the use of a \textAndIndex{discrete Fourier transform}.  This is described briefly in \citep{haykin1994cs}, but there's no proof of the inversion formula.  Let's work through that detail.

Let's start given a periodic signal of the following form

\begin{dmath}\label{eqn:discreteFourierTransform:20}
a_n = \sum_{k = 0}^{N-1} A_k e^{-2 \pi i k n/N },
\end{dmath}

and assume that there's an inversion formula of the form

\begin{dmath}\label{eqn:discreteFourierTransform:40}
A_k' \propto \sum_{n = 0}^{N-1} a_n e^{2 \pi i k' n/N }.
\end{dmath}

Direct substitution should show if such a transform pair is valid, and determine the proportionality constant.  Let's try this

\begin{dmath}\label{eqn:discreteFourierTransform:60}
\sum_{n = 0}^{N-1} a_n e^{2 \pi i k' n/N }
=
\sum_{n = 0}^{N-1} e^{2 \pi i k' n/N }
\sum_{k = 0}^{N-1} A_k e^{-2 \pi i k n/N }
=
\sum_{k = 0}^{N-1}
A_k
\sum_{n = 0}^{N-1} e^{2 \pi i (k' - k)n/N }.
\end{dmath}

Observe that the interior sum is just \(N\) when \(k' = k\).  Let's sum this for the values \(k \ne k'\), writing

\begin{dmath}\label{eqn:discreteFourierTransform:80}
S_N(k'-k) = \sum_{n = 0}^{N-1} e^{2 \pi i (k' - k)n/N }.
\end{dmath}

With \(a = \exp( 2 \pi i (k' - k)/N)\) we have

\begin{dmath}\label{eqn:discreteFourierTransform:100}
S_N(k'-k)
= \sum_{n = 0}^{N-1} a^n
= \frac{a^N - 1}{a - 1}
= \frac{a^{N/2}}{a^{1/2}} \frac{a^{N/2} - a^{-N/2}}{a^{1/2} - a^{-1/2}}
= e^{ \pi i (k' - k)(1 - 1/N)}
\frac{ \sin \pi (k'-k) }{\sin \pi (k' - k)/N}.
\end{dmath}

Observe that \(k' - k \in [-N+1, N-1]\), and necessarily takes on just integer values.  We have terms of the form \(\sin \pi m\), for integer \(m\) in the numerator, always zero.  In the denominator, the sine argument is in the range

\begin{dmath}\label{eqn:discreteFourierTransform:120}
[ \pi \lr{ -1 + \inv{N} }, \pi \lr{ 1 - \inv{N} }],
\end{dmath}

We can visualize that range as all the points on a sine curve with the integer multiples of \(\pi\) omitted, as in \cref{fig:discreteFourierRange:discreteFourierRangeFig1}.

\mathImageFigure{../../figures/phy487/discreteFourierRangeFig1}{Sine with integer multiples of \(\pi\) omitted}{fig:discreteFourierRange:discreteFourierRangeFig1}{0.2}{discreteFourierRangeFig1.nb}

Clearly the denominator is always non-zero when \(k \ne k'\).   This provides us with the desired inverse transformation relationship

\begin{dmath}\label{eqn:discreteFourierTransform:140}
\sum_{n = 0}^{N-1} a_n e^{2 \pi i k' n/N }
= N \sum_{k' = 0}^{N-1} A_k' \delta_{k, k'}
= N A_k'.
\end{dmath}

\paragraph{Summary}

Now that we know the relationship between the discrete set of values \(a_n\) and this discrete transformation of those values, let's write the transform pair in a form that explicitly expresses this relationship.

\boxedEquation{eqn:discreteFourierTransform:160}{
\begin{aligned}
a_n &= \sum_{k = 0}^{N-1} \tilde{a}_k e^{-2 \pi i k n/N } \\
\tilde{a}_k &= \inv{N} \sum_{n = 0}^{N-1} a_n e^{2 \pi i k n/N }.
\end{aligned}
}

We have also shown that our discrete sum of exponentials has an delta function operator nature, a fact that will likely be handy in various manipulations.

\boxedEquation{eqn:discreteFourierTransform:200}{
\sum_{n = 0}^{N-1} e^{2 \pi i (k' - k)n/N } = N \delta_{k, k'}.
}

%\EndArticle
