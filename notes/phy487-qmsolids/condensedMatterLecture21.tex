%
% Copyright � 2013 Peeter Joot.  All Rights Reserved.
% Licenced as described in the file LICENSE under the root directory of this GIT repository.
%
%\newcommand{\authorname}{Peeter Joot}
\newcommand{\email}{peeterjoot@protonmail.com}
\newcommand{\basename}{FIXMEbasenameUndefined}
\newcommand{\dirname}{notes/FIXMEdirnameUndefined/}

%\renewcommand{\basename}{condensedMatterLecture21}
%\renewcommand{\dirname}{notes/phy487/}
%\newcommand{\keywords}{Condensed matter physics, PHY487H1F}
%\newcommand{\authorname}{Peeter Joot}
\newcommand{\onlineurl}{http://sites.google.com/site/peeterjoot2/math2013/\basename.pdf}
\newcommand{\sourcepath}{\dirname\basename.tex}
\newcommand{\generatetitle}[1]{\chapter{#1}}

\newcommand{\vcsinfo}{%
\section*{}
\noindent{\color{DarkOliveGreen}{\rule{\linewidth}{0.1mm}}}
\paragraph{Document version}
%\paragraph{\color{Maroon}{Document version}}
{
\small
\begin{itemize}
\item Available online at:\\ 
\href{\onlineurl}{\onlineurl}
\item Git Repository: \input{./.revinfo/gitRepo.tex}
\item Source: \sourcepath
\item last commit: \input{./.revinfo/gitCommitString.tex}
\item commit date: \input{./.revinfo/gitCommitDate.tex}
\end{itemize}
}
}

%\PassOptionsToPackage{dvipsnames,svgnames}{xcolor}
\PassOptionsToPackage{square,numbers}{natbib}
\documentclass{scrreprt}

\usepackage[left=2cm,right=2cm]{geometry}
\usepackage[svgnames]{xcolor}
\usepackage{peeters_layout}

\usepackage{natbib}

\usepackage[
colorlinks=true,
bookmarks=false,
pdfauthor={\authorname, \email},
backref 
]{hyperref}

% http://tex.stackexchange.com/questions/75773/how-to-reference-problems-by-the-text-label-in-an-exercise-envioronment
\usepackage[english]{cleveref}
\crefname{Exercise}{exercise}{exercises}
\Crefname{Exercise}{Exercise}{Exercises}

\RequirePackage{titlesec}
\RequirePackage{ifthen}

% http://stackoverflow.com/questions/4932910/date-in-the-tabular-environment
\makeatletter
\let\insertdate\@date
\makeatother

\titleformat{\chapter}[display]
{\bfseries\Large}
{\color{DarkSlateGrey}\filleft \authorname
\ifthenelse{\isundefined{\studentnumber}}{}{\\ \studentnumber}
\ifthenelse{\isundefined{\email}}{}{\\ \email}
\ifthenelse{\isundefined{\dateintitle}}{}{\\ \insertdate}
%\ifthenelse{\isundefined{\coursename}}{}{\\ \coursename} % put in title instead.
}
{4ex}
{\color{DarkOliveGreen}{\titlerule}\color{Maroon}
\vspace{2ex}%
\filright}
[\vspace{2ex}%
\color{DarkOliveGreen}\titlerule
]

\newcommand{\beginArtWithToc}[0]{\begin{document}\tableofcontents}
\newcommand{\beginArtNoToc}[0]{\begin{document}}
\newcommand{\EndNoBibArticle}[0]{\end{document}}
\newcommand{\EndArticle}[0]{\bibliography{Bibliography}\bibliographystyle{plainnat}\end{document}}

% 
%\newcommand{\citep}[1]{\cite{#1}}

\colorSectionsForArticle


%
%%\citep{harald2003solid} \S x.y
%
%%\usepackage{mhchem}
%\usepackage[version=3]{mhchem}
%\usepackage{units}
%\usepackage{bm} % \EE
%\newcommand{\nought}[0]{\circ}
%%\newcommand{\EF}[0]{\epsilon_{\txtF}}
%\newcommand{\EF}[0]{E_{\txtF}}
%\newcommand{\kF}[0]{k_{\txtF}}
%
%\beginArtNoToc
%\generatetitle{PHY487H1F Condensed Matter Physics.  Lecture 21: Electron-phonon scattering.  Taught by Prof.\ Stephen Julian}
\label{chap:condensedMatterLecture21}

%\section{Disclaimer}
%
%Peeter's lecture notes from class.  May not be entirely coherent.

\section{Electron-phonon scattering}
\index{scattering}

\reading \citep{ibach2009solid} \textchapref{9} (pp 258-259).

\paragraph{Last time}

\begin{subequations}
\begin{dmath}\label{eqn:condensedMatterLecture21:20}
\Bj = \sigma \bcE
\end{dmath}
\begin{dmath}\label{eqn:condensedMatterLecture21:40}
\sigma = \frac{n e^2 \tau}{m^\conj}
\end{dmath}
\end{subequations}

Here \(\tau\) is the \dquoteAndIndex{mean scattering time}, which is the time to randomize \(\Bv\).

We now continue to discuss scattering, a phenomena due to departure from periodicity.  For phonons, this is proportional to \(\expectation{n_q}_{th}\), where \(E_q = (n_q + 1/2) \Hbar \omega_q\).

%\cref{fig:qmSolidsL21:qmSolidsL21Fig1}.
\imageFigure{../../figures/phy487/qmSolidsL21Fig1}{k-space scattering}{fig:qmSolidsL21:qmSolidsL21Fig1}{0.2}

Small \(q\) (long \(\lambda\)) phonons are not very effective at randomizing \(\Bv\).  The effectiveness of the scattering is \(\propto q^2\).

%\cref{fig:qmSolidsL21:qmSolidsL21Fig2}.
%\imageFigure{../../figures/phy487/qmSolidsL21Fig2}{2:CAPTION}{fig:qmSolidsL21:qmSolidsL21Fig2}{0.2}
%\cref{fig:qmSolidsL21:qmSolidsL21Fig3}.
\imageFigure{../../figures/phy487/qmSolidsL21Fig3}{Scattering confinement to small range of k-space}{fig:qmSolidsL21:qmSolidsL21Fig3}{0.2}

We'd found 

\begin{subequations}
\begin{dmath}\label{eqn:condensedMatterLecture21:60}
\inv{\tau_{\mathrm{ph}}(q)} \propto q^2 \expectation{n_q}
\end{dmath}
\begin{dmath}\label{eqn:condensedMatterLecture21:80}
\expectation{n_q} = \inv{e^{\Hbar \omega_q/\kB T} - 1}
\end{dmath}
\end{subequations}

Combining these, we have

\begin{dmath}\label{eqn:condensedMatterLecture21:100}
\inv{\tau_{\mathrm{ph}}} \propto 
%\frac{2 V}{(2 \pi)^3}
\lr{ \cdots }
\int 4 \pi q^2 dq q^2 
\inv{e^{\Hbar \omega_q/\kB T} - 1}
\end{dmath}

In the high temperature limit, all modes have \(\expectation{n_q} \propto T\), for

\begin{subequations}
\begin{dmath}\label{eqn:condensedMatterLecture21:120}
\inv{\tau_{\mathrm{ph}}} \propto  T
\end{dmath}
\begin{dmath}\label{eqn:condensedMatterLecture21:140}
\sigma \propto \inv{T}
\end{dmath}
\begin{dmath}\label{eqn:condensedMatterLecture21:160}
\rho = \inv{\sigma} \alpha T
\end{dmath}
\end{subequations}

In the low temperature limit.  As in the Debye theory, let 

\begin{equation}\label{eqn:condensedMatterLecture21:180}
x = \frac{\Hbar \omega}{\kB T} = 
\frac{\Hbar c q}{\kB T},
\end{equation}

for

\begin{dmath}\label{eqn:condensedMatterLecture21:200}
\inv{\tau_{\mathrm{ph}}} \propto 
\lr{ \cdots } \lr{ \kB T }^5 
\mathLabelBox
{
\int_0^{\Theta/T} \frac{ x^5 dx}{e^x - 1}
}
{
some number
}.
\end{dmath}

This gives

\begin{equation}\label{eqn:condensedMatterLecture21:220}
\rho = \rho_0 + A T^5.
\end{equation}

At high \(T\), the \(T^1\) behavior is very general.  At low \(T\), \(T^5\) is less universal.

%This is really universal for almost all materials.  
The full range of resistivity is sketched in \cref{fig:qmSolidsL21:qmSolidsL21Fig4}.

\imageFigure{../../figures/phy487/qmSolidsL21Fig4}{Resistivity temperature dependence}{fig:qmSolidsL21:qmSolidsL21Fig4}{0.2}

\section{Electron-electron scattering}
\index{scattering}

Reading: May not be in the text?

Electrons can scatter from other electrons, but they must conserve energy and (crystal) momentum.

\paragraph{\(T = 0\)}

%\cref{fig:qmSolidsL21:qmSolidsL21Fig5}.
\imageFigure{../../figures/phy487/qmSolidsL21Fig5}{Filled Fermi sphere}{fig:qmSolidsL21:qmSolidsL21Fig5}{0.2}

Consider 1 electron outside a Fermi sphere.  The transition \(\ket{i} \rightarrow \ket{f}\) can only go to an empty state, within \(\delta E\) of \(\EF\).

%\cref{fig:qmSolidsL21:qmSolidsL21Fig6}.
\imageFigure{../../figures/phy487/qmSolidsL21Fig6}{With temperature dependence state transitions still effectively confined to range of energies}{fig:qmSolidsL21:qmSolidsL21Fig6}{0.2}

Must scatter off an electron that starts inside \(\EF\), ends outside \(\EF\), and it's \(\Delta E \simeq \delta E\).  eg. (a) to (b).

So both scattering events are restricted by \(\delta E\), or

\boxedEquation{eqn:condensedMatterLecture21:240}{
\inv{\tau} \propto (\delta E)^2.
}

A state at \(\EF\) has infinite lifetime.

At \(T > 0\), the argument is similar.

F6

\(\ket{i}\) has a choice of states from within \(\kB T\) of \(\EF\) to scatter to.  States within \(\kB T\) of \(\EF\) to scatter from.

\begin{equation}\label{eqn:condensedMatterLecture21:260}
\inv{\tau} \propto T^2, \qquad \mbox{max\((T^2, (\delta E)^2)\)},
\end{equation}

or

\boxedEquation{eqn:condensedMatterLecture21:280}{
\rho(T) = \rho_\nought + A T^2.
}

This is \textunderline{universal}.  This is a famous result, from \dquoteAndIndex{Fermi liquid theory}, developed by Landau and Fermi.

%\EndArticle
