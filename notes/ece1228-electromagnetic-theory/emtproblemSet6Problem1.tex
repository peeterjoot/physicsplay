%
% Copyright � 2016 Peeter Joot.  All Rights Reserved.
% Licenced as described in the file LICENSE under the root directory of this GIT repository.
%
\makeproblem{Meissner effect.}{emt:problemSet6:1}{ 
The constitutive relation for superconductors in weak magnetic fields can be macroscopically
characterized by the first London equation

\begin{dmath}\label{eqn:emtproblemSet6Problem1:20}
\PD{t}{\BJ_{\mathrm{sup}}} = \alpha \BE,
\end{dmath}

and the second London equation
\begin{dmath}\label{eqn:emtproblemSet6Problem1:40}
\spacegrad \cross \BJ_{\mathrm{sup}} = -\alpha_1 \BB,
\end{dmath}
where \( \BJ_{\mathrm{sup}} \)
stands for the superconducting current, 
\( \alpha = n_s q^2 /m \) and \( \alpha_1 \approx \alpha \), with 
\( n_s \), \(m\), and \( q\) 
denoting, respectively, the number density, the effective mass, and the charge of the Cooper pairs
responsible for the superconductivity in a charged Boson fluid model.

\makesubproblem{}{emt:problemSet6:1a}
From the first London equation, derive and equation for \( \dot{\BB} = \PDi{t}{\BB} \) 
by using the static
Maxwell equation \( \spacegrad \cross \BH = \BJ_{\mathrm{sup}} \)
without the displacement current. Show that
\begin{dmath}\label{eqn:emtproblemSet6Problem1:60}
\spacegrad^2 \dot{\BB} = \mu_0 \alpha \dot{\BB}
\end{dmath}
\makesubproblem{}{emt:problemSet6:1b}
From the second London equation and the Ampere's law stated above derive an equation
for \( \BB \).
\makesubproblem{}{emt:problemSet6:1c}
What are the penetration depths in the 
\partref{emt:problemSet6:1a}
and
\partref{emt:problemSet6:1b}
cases? Justify your answer.

\paragraph{Remark:} from above analysis we see that both the current and magnetic field are confined to a
thin layer of the order of the penetration depth which is very small. The exclusion of static
magnetic field in a superconductor is known as the Meissner effect experimentally discovered in
1933.
} % makeproblem

\makeanswer{emt:problemSet6:1}{ 
\makeSubAnswer{}{emt:problemSet6:1a}

Taking the curl of the first London equation \cref{eqn:emtproblemSet6Problem1:20} gives

\begin{dmath}\label{eqn:emtproblemSet6Problem1:80}
\spacegrad \cross \dot{\BJ}_{\mathrm{sup}} 
= \alpha \spacegrad \cross \BE
= \alpha \lr{ -\dot{\BB} },
\end{dmath}

or
\begin{dmath}\label{eqn:emtproblemSet6Problem1:100}
\spacegrad \cross \dot{\BJ}_{\mathrm{sup}} 
= -\alpha \dot{\BB},
\end{dmath}

which has the same structure as the time derivative of the second London equation, but with \( \alpha \) instead of \( \alpha_1 \).  Taking the curl once more gives

\begin{dmath}\label{eqn:emtproblemSet6Problem1:120}
0 
=
\PD{t}{} \lr{
\spacegrad \cross \lr{ \spacegrad \cross \BJ_{\mathrm{sup}} } + \alpha \spacegrad \cross \BB
}
=
\PD{t}{} \lr{
-\spacegrad^2 \BJ_{\mathrm{sup}} + \spacegrad \lr{ \spacegrad \cdot \BJ_{\mathrm{sup}} }
+ \alpha \mu \spacegrad \cross \BH
}
=
\PD{t}{} \lr{
-\spacegrad^2 \BJ_{\mathrm{sup}} + \spacegrad \lr{ \spacegrad \cdot \BJ_{\mathrm{sup}} }
+ \alpha \mu \lr{ \BJ_{\mathrm{sup}} + \cancel{ \PD{t}{\BD} } }
},
\end{dmath}

or
\begin{dmath}\label{eqn:emtproblemSet6Problem1:140}
\alpha \mu \dot{\BJ}_{\mathrm{sup}} = \spacegrad^2 \dot{\BJ}_{\mathrm{sup}} + \spacegrad \lr{ \spacegrad \cdot \dot{\BJ}_{\mathrm{sup}} }.
\end{dmath}

One final application of the curl operator gives
\begin{dmath}\label{eqn:emtproblemSet6Problem1:160}
0 
= 
-\alpha \mu \spacegrad \cross \dot{\BJ}_{\mathrm{sup}} + \spacegrad \cross \lr{ \spacegrad^2 \dot{\BJ}_{\mathrm{sup}}} + \cancel{\spacegrad \cross \lr{ \spacegrad \lr{ \spacegrad \cdot \dot{\BJ}_{\mathrm{sup}} }}}.
= 
-\alpha \mu \lr{ -\alpha \dot{\BB} } + \spacegrad^2 \lr{ \spacegrad \cross \dot{\BJ}_{\mathrm{sup}} }
= 
-\alpha \mu \lr{ -\alpha \dot{\BB} } + \spacegrad^2 \lr{ -\alpha \dot{\BB} }
= 
-\alpha \lr{ -\mu \alpha \dot{\BB} + \spacegrad^2 \dot{\BB} }.
\end{dmath}

Note that this used
the fact that the curl of a gradient is zero, the fact that the curl and the Laplacian commute (\( \spacegrad^2 \epsilon_{rst} \Be_r \partial_s A_t = \epsilon_{rst} \Be_r \partial_s \spacegrad^2 A_t = \spacegrad \cross ( \spacegrad^2 \BA ) \)), and made two substitutions of
\cref{eqn:emtproblemSet6Problem1:100}
.  This gives the desired result

\begin{dmath}\label{eqn:emtproblemSet6Problem1:180}
\mu \alpha \dot{\BB} = \spacegrad^2 \dot{\BB}.      \qedmarker
\end{dmath}

\makeSubAnswer{}{emt:problemSet6:1b}

TODO.
\makeSubAnswer{}{emt:problemSet6:1c}

TODO.
}
