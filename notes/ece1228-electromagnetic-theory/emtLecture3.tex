%
% Copyright � 2016 Peeter Joot.  All Rights Reserved.
% Licenced as described in the file LICENSE under the root directory of this GIT repository.
%
\newcommand{\authorname}{Peeter Joot}
\newcommand{\email}{peeterjoot@protonmail.com}
\newcommand{\basename}{FIXMEbasenameUndefined}
\newcommand{\dirname}{notes/FIXMEdirnameUndefined/}

\renewcommand{\basename}{emt3}
\renewcommand{\dirname}{notes/ece1228/}
\newcommand{\keywords}{ECE1228H}
\newcommand{\authorname}{Peeter Joot}
\newcommand{\onlineurl}{http://sites.google.com/site/peeterjoot2/math2013/\basename.pdf}
\newcommand{\sourcepath}{\dirname\basename.tex}
\newcommand{\generatetitle}[1]{\chapter{#1}}

\newcommand{\vcsinfo}{%
\section*{}
\noindent{\color{DarkOliveGreen}{\rule{\linewidth}{0.1mm}}}
\paragraph{Document version}
%\paragraph{\color{Maroon}{Document version}}
{
\small
\begin{itemize}
\item Available online at:\\ 
\href{\onlineurl}{\onlineurl}
\item Git Repository: \input{./.revinfo/gitRepo.tex}
\item Source: \sourcepath
\item last commit: \input{./.revinfo/gitCommitString.tex}
\item commit date: \input{./.revinfo/gitCommitDate.tex}
\end{itemize}
}
}

%\PassOptionsToPackage{dvipsnames,svgnames}{xcolor}
\PassOptionsToPackage{square,numbers}{natbib}
\documentclass{scrreprt}

\usepackage[left=2cm,right=2cm]{geometry}
\usepackage[svgnames]{xcolor}
\usepackage{peeters_layout}

\usepackage{natbib}

\usepackage[
colorlinks=true,
bookmarks=false,
pdfauthor={\authorname, \email},
backref 
]{hyperref}

% http://tex.stackexchange.com/questions/75773/how-to-reference-problems-by-the-text-label-in-an-exercise-envioronment
\usepackage[english]{cleveref}
\crefname{Exercise}{exercise}{exercises}
\Crefname{Exercise}{Exercise}{Exercises}

\RequirePackage{titlesec}
\RequirePackage{ifthen}

% http://stackoverflow.com/questions/4932910/date-in-the-tabular-environment
\makeatletter
\let\insertdate\@date
\makeatother

\titleformat{\chapter}[display]
{\bfseries\Large}
{\color{DarkSlateGrey}\filleft \authorname
\ifthenelse{\isundefined{\studentnumber}}{}{\\ \studentnumber}
\ifthenelse{\isundefined{\email}}{}{\\ \email}
\ifthenelse{\isundefined{\dateintitle}}{}{\\ \insertdate}
%\ifthenelse{\isundefined{\coursename}}{}{\\ \coursename} % put in title instead.
}
{4ex}
{\color{DarkOliveGreen}{\titlerule}\color{Maroon}
\vspace{2ex}%
\filright}
[\vspace{2ex}%
\color{DarkOliveGreen}\titlerule
]

\newcommand{\beginArtWithToc}[0]{\begin{document}\tableofcontents}
\newcommand{\beginArtNoToc}[0]{\begin{document}}
\newcommand{\EndNoBibArticle}[0]{\end{document}}
\newcommand{\EndArticle}[0]{\bibliography{Bibliography}\bibliographystyle{plainnat}\end{document}}

% 
%\newcommand{\citep}[1]{\cite{#1}}

\colorSectionsForArticle



%\usepackage{ece1228}
\usepackage{peeters_braket}
%\usepackage{peeters_layout_exercise}
\usepackage{peeters_figures}
\usepackage{mathtools}
\usepackage{siunitx}

\beginArtNoToc
\generatetitle{ECE1228H Electromagnetic Theory.  Lecture 3: Electrostatics.  Taught by Prof.\ M. Mojahedi}
%\chapter{Electrostatics}
\label{chap:emt3}

\paragraph{Disclaimer}

Peeter's lecture notes from class.  These may be incoherent and rough.

These are notes for the UofT course ECE1228H, Electromagnetic Theory, taught by Prof. M. Mojahedi, covering \textchapref{{1}} \citep{balanis1989advanced} content.

\paragraph{Polarization and Magnetization}

The importance of the polarization and magnetization given by

\begin{dmath}\label{eqn:emtLecture3:20}
\begin{aligned}
\BD &= \epsilon_0 \BE + \BP \\
\BP &= \epsilon_0 \chi_\txte \BE,
\end{aligned}
\end{dmath}

where 
\begin{dmath}\label{eqn:emtLecture3:40}
\begin{aligned}
\BD &= \epsilon \BE \\
\epsilon &= \epsilon_0 \epsilon_r \\
\epsilon_r = 1 + \chi_\txte.
\end{aligned}
\end{dmath}

\paragraph{Point charge.}

\begin{dmath}\label{eqn:emtLecture3:60}
\BE 
= \frac{q}{4 \pi \epsilon_0} \frac{\rcap}{\Br^2}
= \frac{q}{4 \pi \epsilon_0} \frac{\Br}{\Abs{\Br}^3}
= \frac{q}{4 \pi \epsilon_0} \frac{\Br}{r^3}.
\end{dmath}

In more complex media the \( \epsilon_0 \) here can be replaced by \( \epsilon \).
Here the vector \( \Br \) points from the charge to the observation point.

Note that the class notes use \( \cap{a}_R \) instead of \( \rcap \). 

When the charge isn't located at the origin, we must modify this accordingly

\begin{dmath}\label{eqn:emtLecture3:80}
\BE 
= \frac{q}{4 \pi \epsilon_0} \frac{\BR}{\Abs{\BR}^3}
= \frac{q}{4 \pi \epsilon_0} \frac{\BR}{R^3},
\end{dmath}

where \( \BR = \Br - \Br' \) still points from the location of the charge to the point of observation, as sketched in

F1.

This can be further generalized to collections of point charges by superposition

\begin{dmath}\label{eqn:emtLecture3:100}
\BE 
= \frac{1}{4 \pi \epsilon_0} \sum_i q_i \frac{\Br - \Br_i'}{\Abs{\Br - \Br_i'}^3}.
\end{dmath}

Observe that a potential that satisfies \( \BE = - \spacegrad V \) can be defined as

\begin{dmath}\label{eqn:emtLecture3:120}
V  
= \frac{1}{4 \pi \epsilon_0} \sum_i \frac{q_i}{\Abs{\Br - \Br_i'}}.
\end{dmath}

When we are considering real world scenerios (like touching your hair, and then the table), how do we deal with the billions of charges involved.  This can be done by considering the charges so small that they can be approximated as a continuous distribution of charges.

This can be done by introducing the concept of a continuous charge distribution \( \rho_\txtv(\Br') \).
The charge that is in a small differential volume element \( dV' \) is \( \rho(\Br') dV' \), and 
the superposition has the form

\begin{dmath}\label{eqn:emtLecture3:140}
\BE 
= \frac{1}{4 \pi \epsilon_0} \iiint dV' \rho_\txtv(\Br') \frac{\Br - \Br'}{\Abs{\Br - \Br'}^3},
\end{dmath}

with potential

\begin{dmath}\label{eqn:emtLecture3:160}
V  
= \frac{1}{4 \pi \epsilon_0} \iiint dV' \frac{\rho(\Br')}{\Abs{\Br - \Br'}}.
\end{dmath}

The surface charge density analogue of this is

\begin{dmath}\label{eqn:emtLecture3:180}
\BE 
= \frac{1}{4 \pi \epsilon_0} \iint dA' \rho_\txts(\Br') \frac{\Br - \Br'}{\Abs{\Br - \Br'}^3},
\end{dmath}

with potential

\begin{dmath}\label{eqn:emtLecture3:200}
V  
= \frac{1}{4 \pi \epsilon_0} \iint dA' \frac{\rho_\txts(\Br')}{\Abs{\Br - \Br'}}.
\end{dmath}

The line charge density analogue of this is

\begin{dmath}\label{eqn:emtLecture3:220}
\BE 
= \frac{1}{4 \pi \epsilon_0} \int dl' \rho_\txtl(\Br') \frac{\Br - \Br'}{\Abs{\Br - \Br'}^3},
\end{dmath}

with potential

\begin{dmath}\label{eqn:emtLecture3:240}
V  
= \frac{1}{4 \pi \epsilon_0} \int dl' \frac{\rho_\txtl(\Br')}{\Abs{\Br - \Br'}}.
\end{dmath}

The difficulty with any of these approaches is the charge density is hardly ever known.  When the charge density is known, this sorts of integrals may not be analytically calculable, but they do yield to numeric calculation.

We may often prefer the potential calculations of the field calculations because they are much easier, having just one component to deal with.

\paragraph{Homework question:}

Starting from Maxwell's equations, in particular \( \oint \BE \cdot d\Bs = Q/\epsilon_0 \) that \( \BE = E_r \rcap \), and explicitly demonstrate why there is no \( \theta \) or \( \phi \) dependencies in this field.  Also calculate the potential \( V \propto 1/r \) associated with an electric field, and show that \( \BE = -\spacegrad V \), and show that this implies that \( -\int_b^a \BE \cdot d\Bl = V_a - V_b \).

\EndArticle
%\EndNoBibArticle
