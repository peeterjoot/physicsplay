%
% Copyright � 2016 Peeter Joot.  All Rights Reserved.
% Licenced as described in the file LICENSE under the root directory of this GIT repository.
%
\makeproblem{Potentials under different gauges.}{emt:problemSet9:3}{ 

Using the non-existence of magnetic monopole and Faraday's law 

\makesubproblem{}{emt:problemSet9:3a}
Define the vector and scalar vector potentials 
\( \BA(\Br, t) \) and
\( V(\Br, t) \).

\makesubproblem{}{emt:problemSet9:3b}

Let \( \BJ = \BJ_i + \BJ_c \) be 
the current \,[\si{A/m}] and 
\( \rho \)
be the charge \,[\si{C/m}] densities. 
Assuming a simple medium and Lorentz gauge, derive the decoupled non-homogeneous wave equations for
\( \BA(\Br, t) \) and
\( V(\Br, t) \).

\makesubproblem{}{emt:problemSet9:3c}
Replace the Lorentz gauge of 
\partref{emt:problemSet9:3b}
%part (b) 
with the Coulomb gauge, and obtain the non-homogeneous differential equations for 
\( \BA(\Br, t) \) and
\( V(\Br, t) \).

\makesubproblem{}{emt:problemSet9:3d}
What fundamental theorem allows us to use different gauges in 
\partref{emt:problemSet9:3b}
and
\partref{emt:problemSet9:3b}
%parts (b) and (c)
? (Justify your answer.)

\paragraph{Note:} From the problem's statement, it should be clear that I want the results for the instantaneous fields and not in the form of time harmonic fields.
} % makeproblem

\makeanswer{emt:problemSet9:3}{ 
\makeSubAnswer{}{emt:problemSet9:3a}

The non-existence of magnetic monopoles means that we are working with the standard form of Maxwell's equations, with no volume magnetic charge density, nor any magnetic currents

\begin{subequations}
\label{eqn:emtproblemSet9Problem3:20}
\begin{dmath}\label{eqn:emtproblemSet9Problem3:40}
\spacegrad \cross \BE = -\partial_t \BB
\end{dmath}
\begin{dmath}\label{eqn:emtproblemSet9Problem3:60}
\spacegrad \cross \BH = \BJ + \partial_t \BD
\end{dmath}
\begin{dmath}\label{eqn:emtproblemSet9Problem3:80}
\spacegrad \cross \BD = \rho
\end{dmath}
\begin{dmath}\label{eqn:emtproblemSet9Problem3:100}
\spacegrad \cross \BB = 0.
\end{dmath}
\end{subequations}

Helmholtz's theorem applied to \cref{eqn:emtproblemSet9Problem3:100}, indicates that the magnetic field \( \BB \) must be a curl

\begin{dmath}\label{eqn:emtproblemSet9Problem3:120}
\BB = \spacegrad \cross \BA,
\end{dmath}

for some \( \BA \).  Plugging this into \cref{eqn:emtproblemSet9Problem3:40} we have
\begin{dmath}\label{eqn:emtproblemSet9Problem3:140}
\spacegrad \cross \BE = -\partial_t \lr{ \spacegrad \cross \BA },
\end{dmath}

or
\begin{dmath}\label{eqn:emtproblemSet9Problem3:160}
\spacegrad \cross \lr{ \BE + \partial_t \BA } = 0.
\end{dmath}

This is satisfied by setting

\begin{dmath}\label{eqn:emtproblemSet9Problem3:180}
\BE + \partial_t \BA = \grad \psi,
\end{dmath}

for some \( \psi \), say \( \psi = -V \), or

\begin{dmath}\label{eqn:emtproblemSet9Problem3:200}
\BE = -\partial_t \BA - \spacegrad V.
\end{dmath}

Ignoring gauge freedom temporarily, in simple media where \( \BB = \mu \BH \) and \( \BD = \epsilon \BE \), the Ampere-Maxwell equation is

\begin{dmath}\label{eqn:emtproblemSet9Problem3:220}
0 
= \spacegrad \cross \BH - \BJ - \partial_t \BD
= \spacegrad \cross \frac{ \spacegrad \cross \BA }{\mu} - \BJ - \epsilon \partial_t \lr{ -\partial_t \BA - \spacegrad V },
\end{dmath}

or

%\begin{dmath}\label{eqn:emtproblemSet9Problem3:240}
\boxedEquation{eqn:emtproblemSet9Problem3:240}{
-\spacegrad^2 \BA + \mu\epsilon \partial_{tt} \BA + \spacegrad \lr{ \spacegrad \cdot \BA + \epsilon\mu \partial_t V } = \mu \BJ.
}
%\end{dmath}

Gauss's law takes the form

\begin{dmath}\label{eqn:emtproblemSet9Problem3:300}
\frac{\rho}{\epsilon} 
=
\spacegrad \cdot \BE 
=
\spacegrad \cdot \lr{ 
-\partial_t \BA - \spacegrad V
}
\end{dmath}

or
%\begin{dmath}\label{eqn:emtproblemSet9Problem3:320}
\boxedEquation{eqn:emtproblemSet9Problem3:320}{
-\spacegrad^2 V 
-\partial_t \spacegrad \cdot \BA
=
\frac{\rho}{\epsilon}.
}
%\end{dmath}

\makeSubAnswer{}{emt:problemSet9:3b}

The Lorentz gauge is usually defined in vacuum where

\begin{dmath}\label{eqn:emtproblemSet9Problem3:260}
0
= \partial_\mu A^\mu 
= \partial_0 (V/c) + \partial_k A^k
= \inv{c^2} \partial_t V + \spacegrad \cdot \BA.
\end{dmath}

Generalizing this to simple media, we can define the Lorentz gauge condition as

%\begin{dmath}\label{eqn:emtproblemSet9Problem3:280}
\boxedEquation{eqn:emtproblemSet9Problem3:280}{
0 = \spacegrad \cdot \BA + \epsilon\mu \partial_t V,
}
%\end{dmath}

which reduces to \cref{eqn:emtproblemSet9Problem3:260} when \( \epsilon\mu = \epsilon_0 \mu_0 = 1/c^2 \).

Inserting this into the Ampere-Maxwell equation, we are left with a non-homogeneous wave equation for \( BA \)

\boxedEquation{eqn:emtproblemSet9Problem3:360}{
-\spacegrad^2 \BA + \mu\epsilon \partial_{tt} \BA = \mu \BJ.
}

Inserting into Gauss's law, we have

\begin{dmath}\label{eqn:emtproblemSet9Problem3:380}
\frac{\rho}{\epsilon}
=
-\spacegrad^2 V 
-\partial_t \lr{ \spacegrad \cdot \BA }
=
-\spacegrad^2 V 
-\partial_t \lr{ -\epsilon\mu \partial_t V },
\end{dmath}

a second non-homogeneous wave equation for the potential \( V \)

%\begin{dmath}\label{eqn:emtproblemSet9Problem3:400}
\boxedEquation{eqn:emtproblemSet9Problem3:420}{
-\spacegrad^2 V + \epsilon\mu \partial_{tt} V 
=
\frac{\rho}{\epsilon}.
}
%\end{dmath}

Note that if \( \BJ = \BJ_i + \BJ_c = \BJ_i + \sigma \BE \), then \cref{eqn:emtproblemSet9Problem3:360} is not fully decoupled.  Instead we have

\begin{dmath}\label{eqn:emtproblemSet9Problem3:440}
-\spacegrad^2 \BA + \mu\epsilon \partial_{tt} \BA 
= \mu \lr{ 
   \BJ_i + \sigma \lr{
      -\partial_t \BA - \spacegrad V
   }
}.
\end{dmath}

To decouple this, we can take the divergence of both sides
\begin{dmath}\label{eqn:emtproblemSet9Problem3:460}
-\spacegrad^2 \spacegrad \cdot \BA + \mu\epsilon \partial_{tt} \spacegrad \cdot \BA 
= \mu \lr{ 
   \spacegrad \cdot \BJ_i + \sigma \lr{
      -\partial_t \spacegrad \cdot \BA - \spacegrad^2 V
   }
}
= \mu \lr{ 
   \spacegrad \cdot \BJ_i + \sigma \lr{
      -\partial_t \spacegrad \cdot \BA + \frac{\rho}{\epsilon} + \partial_t \spacegrad \cdot \BA
   }
}
%= \mu \lr{ 
%   \spacegrad \cdot \BJ_i + \sigma \lr{
%\frac{\rho}{\epsilon} 
%   }
%}
= 
\mu \spacegrad \cdot \BJ_i
 + \frac{\mu \sigma}{\epsilon} \rho,
\end{dmath}

which yields a third order equation for \( \BA \)

\begin{dmath}\label{eqn:emtproblemSet9Problem3:480}
\spacegrad \cdot \lr{
-\spacegrad^2 \BA + \mu\epsilon \partial_{tt} \BA 
}
=
\mu \spacegrad \cdot \BJ_i
 + \frac{\mu \sigma}{\epsilon} \rho.
\end{dmath}

\makeSubAnswer{}{emt:problemSet9:3c}

In the Coulomb gauge we set \( \spacegrad \cdot \BA = 0 \), leaving

\boxedEquation{eqn:emtproblemSet9Problem3:500}{
-\spacegrad^2 V = \frac{\rho}{\epsilon},
}

and
\begin{dmath}\label{eqn:emtproblemSet9Problem3:520}
-\spacegrad^2 \BA + \mu\epsilon \partial_{tt} \BA + \epsilon\mu \partial_t \spacegrad V = \mu \BJ
\end{dmath}

This can clearly be separated by taking the divergence of both sides

\begin{dmath}\label{eqn:emtproblemSet9Problem3:540}
\spacegrad \cdot \lr{ -\spacegrad^2 \BA + \mu\epsilon \partial_{tt} \BA } + \epsilon\mu \partial_t \lr{ -\frac{\rho}{\epsilon}} = \mu \spacegrad \cdot \BJ.
\end{dmath}

Using the continuity equation, this is a wave equation for the divergence of the vector potential \( \BA \)

%\begin{dmath}\label{eqn:emtproblemSet9Problem3:740}
\boxedEquation{eqn:emtproblemSet9Problem3:760}{
-\spacegrad^2 \lr{ \spacegrad \cdot \BA } + \mu\epsilon \partial_{tt} \lr{ \spacegrad \cdot \BA } = 0.
}
%\end{dmath}

Another way to look at this, as outlined in \citep{jackson1975cew} is to note that 
the potential equation \cref{eqn:emtproblemSet9Problem3:540}
depends only on the divergence of \( \BJ \), and not on the curl of \( \BJ \).  This suggests that a Helmholtz decomposition of the current into its divergence and and divergence free components in the second order equation \cref{eqn:emtproblemSet9Problem3:520} may help eliminate the coupling, without resorting to a third order PDE.

That decomposition follows from expanding \( \spacegrad^2 \BJ/R \) in two ways using the delta function \( -4 \pi \delta(\Bx - \Bx') = \spacegrad^2 1/R \) representation, as well as directly

\begin{dmath}\label{eqn:emtproblemSet9Problem3:560}
- 4 \pi \BJ(\Bx) 
=
\int \spacegrad^2 \frac{\BJ(\Bx')}{\Abs{\Bx - \Bx'}} d^3 x'
=
\spacegrad 
\int \spacegrad \cdot \frac{\BJ(\Bx')}{\Abs{\Bx - \Bx'}} d^3 x'
+
\spacegrad \cdot
\int \spacegrad \wedge \frac{\BJ(\Bx')}{\Abs{\Bx - \Bx'}} d^3 x'
=
-\spacegrad 
\int \BJ(\Bx') \cdot \spacegrad' \inv{\Abs{\Bx - \Bx'}} d^3 x'
+
\spacegrad \cdot \lr{ \spacegrad \wedge
\int \frac{\BJ(\Bx')}{\Abs{\Bx - \Bx'}} d^3 x'
}
=
-\spacegrad 
\int \spacegrad' \cdot \frac{\BJ(\Bx')}{\Abs{\Bx - \Bx'}} d^3 x'
+\spacegrad 
\int \frac{\spacegrad' \cdot \BJ(\Bx')}{\Abs{\Bx - \Bx'}} d^3 x'
-
\spacegrad \cross \lr{
\spacegrad \cross 
\int \frac{\BJ(\Bx')}{\Abs{\Bx - \Bx'}} d^3 x'
}
\end{dmath}

The first term can be converted to a surface integral

\begin{dmath}\label{eqn:emtproblemSet9Problem3:580}
-\spacegrad 
\int \spacegrad' \cdot \frac{\BJ(\Bx')}{\Abs{\Bx - \Bx'}} d^3 x'
=
-\spacegrad 
\int d\BA' \cdot \frac{\BJ(\Bx')}{\Abs{\Bx - \Bx'}},
\end{dmath}

so provided the currents are either localized or \( \Abs{\BJ}/R \rightarrow 0 \) on an infinite sphere, we can make the identification

\begin{equation}\label{eqn:emtproblemSet9Problem3:600}
\BJ(\Bx) 
=
\spacegrad 
\inv{4\pi} \int \frac{\spacegrad' \cdot \BJ(\Bx')}{\Abs{\Bx - \Bx'}} d^3 x'
-
\spacegrad \cross \spacegrad \cross \inv{4 \pi} \int \frac{\BJ(\Bx')}{\Abs{\Bx - \Bx'}} d^3 x'
\equiv
\BJ_l + 
\BJ_t,
\end{equation}

where \( \spacegrad \cross \BJ_l = 0 \) (irrotational, or longitudinal), whereas \( \spacegrad \cdot \BJ_t = 0 \) (solenoidal or transverse).  The irrotational property is clear from inspection, and the transverse property can be verified readily

\begin{dmath}\label{eqn:emtproblemSet9Problem3:620}
\spacegrad \cdot \lr{ \spacegrad \cross \lr{ \spacegrad \cross \BX } }
=
-\spacegrad \cdot \lr{ \spacegrad \cdot \lr{ \spacegrad \wedge \BX } }
=
-\spacegrad \cdot \lr{ \spacegrad^2 \BX - \spacegrad \lr{ \spacegrad \cdot \BX } }
=
-\spacegrad \cdot \lr{\spacegrad^2 \BX} + \spacegrad^2 \lr{ \spacegrad \cdot \BX }
= 0.
\end{dmath}

Since

\begin{dmath}\label{eqn:emtproblemSet9Problem3:640}
V(\Bx, t)
=
\inv{4 \pi \epsilon} \int \frac{\rho(\Bx', t)}{\Abs{\Bx - \Bx'}} d^3 x',
\end{dmath}

we have

\begin{dmath}\label{eqn:emtproblemSet9Problem3:660}
\spacegrad \PD{t}{V}
=
\inv{4 \pi \epsilon} \spacegrad \int \frac{\partial_t \rho(\Bx', t)}{\Abs{\Bx - \Bx'}} d^3 x'
=
\inv{4 \pi \epsilon} \spacegrad \int \frac{-\spacegrad' \cdot \BJ}{\Abs{\Bx - \Bx'}} d^3 x'
=
\frac{\BJ_l}{\epsilon}.
\end{dmath}

The Ampere-Maxwell equation \cref{eqn:emtproblemSet9Problem3:520} is reduced to

\begin{dmath}\label{eqn:emtproblemSet9Problem3:680}
-\spacegrad^2 \BA + \mu\epsilon \partial_{tt} \BA + \epsilon\mu \lr{ \frac{\BJ_l}{\epsilon} } = \mu \BJ,
\end{dmath}

or
\boxedEquation{eqn:emtproblemSet9Problem3:700}{
-\spacegrad^2 \BA + \mu\epsilon \partial_{tt} \BA = \mu \BJ_t,
}

where
\begin{dmath}\label{eqn:emtproblemSet9Problem3:720}
\BJ_t =
-
\spacegrad \cross \spacegrad \cross \inv{4 \pi} \int \frac{\BJ(\Bx')}{\Abs{\Bx - \Bx'}} d^3 x'.
\end{dmath}

\makeSubAnswer{}{emt:problemSet9:3d}

Because the magnetic field is defined in terms of the curl of a potential, we can alter that potential by any gradient, and not change the magnetic field

\begin{dmath}\label{eqn:emtproblemSet9Problem3:780}
\BB = \spacegrad \lr{ \BA + \spacegrad \chi }.
\end{dmath}

What corresponding change to the scalar potential must be made after a mapping 

\begin{dmath}\label{eqn:emtproblemSet9Problem3:800}
\BA \rightarrow \BA + \spacegrad \chi = \BA'.
\end{dmath}

The electric field is

\begin{dmath}\label{eqn:emtproblemSet9Problem3:820}
\BE 
= -\spacegrad V - \partial_t \BA
= -\spacegrad V - \partial_t \lr{ \BA' - \spacegrad \chi }
= -\spacegrad \lr{ V - \partial_t \chi } - \partial_t \BA'.
\end{dmath}

We see that the electric and magnetic fields will be unchanged provided the vector and scalar potential are changed in pairs as
%\begin{dmath}\label{eqn:emtproblemSet9Problem3:840}
\boxedEquation{eqn:emtproblemSet9Problem3:860}{
\begin{aligned}
\BA' &= \BA + \spacegrad \chi \\
V' &= V - \partial_t \chi.
\end{aligned}
}
%\end{dmath}

We should also expect that the form of Maxwell's equations will be left unchanged.  Plugging these transformed potentials into Gauss's law \cref{eqn:emtproblemSet9Problem3:320}, we recover the original field equation in terms of the unaltered potentials

\begin{dmath}\label{eqn:emtproblemSet9Problem3:880}
\frac{\rho}{\epsilon}
=
-\spacegrad^2 V'
-\partial_t \spacegrad \cdot \BA'
=
-\spacegrad^2 \lr{ V - \partial_t \chi }
-\partial_t \spacegrad \cdot \lr{ \BA + \spacegrad \chi }
=
-\spacegrad^2 V 
-\partial_t \spacegrad \cdot \BA
- \spacegrad \partial_t \chi
-\partial_t \spacegrad^2 \chi 
=
-\spacegrad^2 V 
-\partial_t \spacegrad \cdot \BA.
\end{dmath}

Similarily, plugging in the transformed potentials into \cref{eqn:emtproblemSet9Problem3:240} gives

\begin{dmath}\label{eqn:emtproblemSet9Problem3:900}
\mu \BJ
=
-\spacegrad^2 \BA' + \mu\epsilon \partial_{tt} \BA' + \spacegrad \lr{ \spacegrad \cdot \BA' + \epsilon\mu \partial_t V' } 
=
-\spacegrad^2 \lr{ \BA + \spacegrad \chi }
+ \mu\epsilon \partial_{tt} \lr{ \BA + \spacegrad \chi }
+ \spacegrad \lr{ \spacegrad \cdot \lr{ \BA + \spacegrad \chi } + \epsilon\mu \partial_t \lr{ V - \partial_t \chi } } 
=
-\spacegrad^2 \BA 
+ \mu\epsilon \partial_{tt} \BA 
+ \spacegrad \lr{ \spacegrad \cdot \BA } + \epsilon\mu \partial_t V  
-\spacegrad^2 \spacegrad \chi
+ \mu\epsilon \partial_{tt} \spacegrad \chi
+ \spacegrad \spacegrad^2 \chi - \epsilon\mu \spacegrad \partial_{tt} \chi  
=
-\spacegrad^2 \BA 
+ \mu\epsilon \partial_{tt} \BA 
+ \spacegrad \lr{ \spacegrad \cdot \BA } + \epsilon\mu \partial_t V,
\end{dmath}

showing that both pairs of potential equations are unaltered by a transformation of the form \cref{eqn:emtproblemSet9Problem3:860}.
}
