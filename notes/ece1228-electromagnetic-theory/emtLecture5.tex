%
% Copyright � 2016 Peeter Joot.  All Rights Reserved.
% Licenced as described in the file LICENSE under the root directory of this GIT repository.
%
%\newcommand{\authorname}{Peeter Joot}
\newcommand{\email}{peeterjoot@protonmail.com}
\newcommand{\basename}{FIXMEbasenameUndefined}
\newcommand{\dirname}{notes/FIXMEdirnameUndefined/}

%\renewcommand{\basename}{emt5}
%\renewcommand{\dirname}{notes/ece1228/}
%\newcommand{\keywords}{ECE1228H}
%\newcommand{\authorname}{Peeter Joot}
\newcommand{\onlineurl}{http://sites.google.com/site/peeterjoot2/math2013/\basename.pdf}
\newcommand{\sourcepath}{\dirname\basename.tex}
\newcommand{\generatetitle}[1]{\chapter{#1}}

\newcommand{\vcsinfo}{%
\section*{}
\noindent{\color{DarkOliveGreen}{\rule{\linewidth}{0.1mm}}}
\paragraph{Document version}
%\paragraph{\color{Maroon}{Document version}}
{
\small
\begin{itemize}
\item Available online at:\\ 
\href{\onlineurl}{\onlineurl}
\item Git Repository: \input{./.revinfo/gitRepo.tex}
\item Source: \sourcepath
\item last commit: \input{./.revinfo/gitCommitString.tex}
\item commit date: \input{./.revinfo/gitCommitDate.tex}
\end{itemize}
}
}

%\PassOptionsToPackage{dvipsnames,svgnames}{xcolor}
\PassOptionsToPackage{square,numbers}{natbib}
\documentclass{scrreprt}

\usepackage[left=2cm,right=2cm]{geometry}
\usepackage[svgnames]{xcolor}
\usepackage{peeters_layout}

\usepackage{natbib}

\usepackage[
colorlinks=true,
bookmarks=false,
pdfauthor={\authorname, \email},
backref 
]{hyperref}

% http://tex.stackexchange.com/questions/75773/how-to-reference-problems-by-the-text-label-in-an-exercise-envioronment
\usepackage[english]{cleveref}
\crefname{Exercise}{exercise}{exercises}
\Crefname{Exercise}{Exercise}{Exercises}

\RequirePackage{titlesec}
\RequirePackage{ifthen}

% http://stackoverflow.com/questions/4932910/date-in-the-tabular-environment
\makeatletter
\let\insertdate\@date
\makeatother

\titleformat{\chapter}[display]
{\bfseries\Large}
{\color{DarkSlateGrey}\filleft \authorname
\ifthenelse{\isundefined{\studentnumber}}{}{\\ \studentnumber}
\ifthenelse{\isundefined{\email}}{}{\\ \email}
\ifthenelse{\isundefined{\dateintitle}}{}{\\ \insertdate}
%\ifthenelse{\isundefined{\coursename}}{}{\\ \coursename} % put in title instead.
}
{4ex}
{\color{DarkOliveGreen}{\titlerule}\color{Maroon}
\vspace{2ex}%
\filright}
[\vspace{2ex}%
\color{DarkOliveGreen}\titlerule
]

\newcommand{\beginArtWithToc}[0]{\begin{document}\tableofcontents}
\newcommand{\beginArtNoToc}[0]{\begin{document}}
\newcommand{\EndNoBibArticle}[0]{\end{document}}
\newcommand{\EndArticle}[0]{\bibliography{Bibliography}\bibliographystyle{plainnat}\end{document}}

% 
%\newcommand{\citep}[1]{\cite{#1}}

\colorSectionsForArticle


%
%%\usepackage{ece1228}
%\usepackage{peeters_braket}
%%\usepackage{peeters_layout_exercise}
%\usepackage{peeters_figures}
%\usepackage{macros_cal}
%\usepackage{macros_bm}
%\usepackage{mathtools}
%\usepackage{siunitx}
%
%\beginArtNoToc
%\generatetitle{ECE1228H Electromagnetic Theory.  Lecture 5: Poynting vector.  Taught by Prof.\ M. Mojahedi}
\chapter{Poynting vector, and time harmonic (phasor) fields.}
\label{chap:emt5}

\paragraph{Poynting}

The cross product terms of Maxwell's equation are
\begin{equation}\label{eqn:emtLecture5:120}
\spacegrad \cross \BE 
= -\BM_i - \PD{t}{\BB}
= -\BM_i - \BM_d,
\end{equation}

where \(\BM_d\) is called the magnetic displacement current here.  For the magnetic curl we have

\begin{equation}\label{eqn:emtLecture5:140}
\spacegrad \cross \BH 
= \BJ_i + \BJ_c + \PD{t}{\BD}
= \BJ_i + \BJ_c + \BJ_d.
\end{equation}

From this (HW) we will show that 
\begin{dmath}\label{eqn:emtLecture5:160}
\spacegrad \cdot \lr{ \BE \cross \BH } + \BH \cdot \lr{ \BM_i + \BM_d }  + \BE \cdot \lr{ \BJ_i + \BJ_c + \BJ_d } = 0,
\end{dmath}

or
\begin{dmath}\label{eqn:emtLecture5:180}
\oint d\Ba \cdot \lr{ \BE \cross \BH } + \int dV \lr{ \BH \cdot \lr{ \BM_i + \BM_d }  + \BE \cdot \lr{ \BJ_i + \BJ_c + \BJ_d }} = 0,
\end{dmath}

or
\begin{dmath}\label{eqn:emtLecture5:200}
\oint d\Ba \cdot \lr{ \BE \cross \BH } 
+ \int dV \BH \cdot \BM_i
+ \int dV \BE \cdot \BJ_i
+ \int dV \BE \cdot \BJ_c
+ \int dV \lr{ \BH \cdot \PD{t}{\BB} + \BE \cdot \PD{t}{\BD} } = 0.
\end{dmath}

Define a supplied power density \( \rho_{\textrm{supp}} \)

\begin{dmath}\label{eqn:emtLecture5:220}
-\rho_{\textrm{supp}}
=
 \int dV \BH \cdot \BM_i
+ \int dV \BE \cdot \BJ_i.
\end{dmath}

When the medium is not dispersive or lossy, we have

\begin{dmath}\label{eqn:emtLecture5:240}
\int dV \BH \cdot \PD{t}{\BB} 
=
\mu \int dV \BH \cdot \PD{t}{\BH} 
=
\PD{t}{} \int dV \mu \Abs{\BH}^2.
\end{dmath}

The units of \( [\mu \Abs{\BH}^2] \) are \si{W}, so one can defined a magnetic energy density \( \mu \Abs{\BH}^2 \), and

\begin{dmath}\label{eqn:emtLecture5:260}
W_m = 
\int dV \mu \Abs{\BH}^2,
\end{dmath}

for

\begin{dmath}\label{eqn:emtLecture5:280}
\int dV \BH \cdot \PD{t}{\BB} 
=
\PD{t}{W_m}.
\end{dmath}

This is the rate of change of stored magnetic energy [\si{J/s} = \si{W}].

Similarly
\begin{dmath}\label{eqn:emtLecture5:300}
\int dV \BE \cdot \PD{t}{\BD} 
=
\epsilon
\int dV \BE \cdot \PD{t}{\BE} 
=
\PD{t}{} \int dV \epsilon \Abs{\BE}^2.
\end{dmath}

The electric energy density is \( \epsilon \Abs{\BE}^2 \).  Let

\begin{dmath}\label{eqn:emtLecture5:320}
W_e = 
\int dV \epsilon \Abs{\BE}^2,
\end{dmath}

and
\begin{dmath}\label{eqn:emtLecture5:340}
\int dV \BE \cdot \PD{t}{\BD} 
=
\PD{t}{W_e}.
\end{dmath}

We also have a term

\begin{dmath}\label{eqn:emtLecture5:360}
\int dV \BE \cdot \BJ_c 
=
\int dV \BE \cdot (\sigma \BE)
=
\int dV \sigma \Abs{\BE}^2
%\equiv ...
\end{dmath}

This is the rate of change of stored electric energy.

The remaining term is
\begin{dmath}\label{eqn:emtLecture5:380}
\oint d\Ba \cdot \lr{ \BE \cross \BH }
\end{dmath}

This is a density of the power that is leaving the volume.  The vector \( \BE \cross \BH \) is special, called the Poynting vector, and coincidentally points in the direction that the energy leaves the bounding surface per unit time.  We write

\begin{dmath}\label{eqn:emtLecture5:400}
\BS = \BE \cross \BH.
\end{dmath}

In vacuum the phase velocity \( \Bv_p \), group velocity \( \Bv_g \) and packet(?) velocity \( \Bv_p \) all line up.  This isn't the case in the media.

It turns out that without dissipation 

\begin{dmath}\label{eqn:emtLecture5:420}
\int \BH \cdot \PD{t}{\BB} = \int \BE \cdot \PD{t}{\BD}.
\end{dmath}

For example in an LC circuit \cref{fig:lecture4LCCircuit:lecture4LCCircuitFig1}
half the cycle the energy is stored in the inductor, and in the other half of the cycle the energy is stored in the capacitor.

\imageFigure{../../figures/ece1228-emt/lecture4LCCircuitFig1}{LC circuit.}{fig:lecture4LCCircuit:lecture4LCCircuitFig1}{0.2}

Summarizing

\begin{dmath}\label{eqn:emtLecture5:440}
\oint \lr{ \BE \cross \BH } \cdot d\Ba = P_{\textrm{exit}}.
\end{dmath}

\paragraph{Time harmonics}

Recall that we have differential equations to solve for each type of circuit element in the time domain.  For example in \cref{fig:lecture4inductor:lecture4inductorFig2a}, we have

\begin{dmath}\label{eqn:emtLecture5:980}
V_i(t) = L \ddt{i},
\end{dmath}

\imageFigure{../../figures/ece1228-emt/lecture4inductorFig2a}{Inductor.}{fig:lecture4inductor:lecture4inductorFig2a}{0.2}

and for the capacitor sketched in \cref{fig:lecture4cap:lecture4capFig2b}, we have
\begin{dmath}\label{eqn:emtLecture5:1000}
i_c(t) = C \ddt{V_c}.
\end{dmath}

\imageFigure{../../figures/ece1228-emt/lecture4capFig2b}{Capacitor.}{fig:lecture4cap:lecture4capFig2b}{0.2}

When we use Laplace or Fourier techniques to solve circuits with such differential equation elements.  The price that we paid for that was that we have to start dealing with complex-valued (phasor) quantities.  We can do this for field equations as well.  The goal is to remove the time domain coupling in Maxwell equations like

\begin{dmath}\label{eqn:emtLecture5:460}
\spacegrad \cross \BE(\Br, t) = -\PD{t}{\BB}(\Br, t)
\end{dmath}
\begin{dmath}\label{eqn:emtLecture5:480}
\spacegrad \cross \BH(\Br, t) = \sigma \BE + \PD{t}{\BD}(\Br, t).
\end{dmath}

For a single frequency, assume that the time dependency can be written as

\begin{dmath}\label{eqn:emtLecture5:500}
\BE(\Br, t) = \Real \lr{ \BE^\conj(\Br) e^{j \omega t} }.
\end{dmath}

We may now have to require \( \BE(\Br) \) to be complex valued.
We also have to be really careful about which convention of the time domain solution we are going to use, since we could just as easily use

\begin{dmath}\label{eqn:emtLecture5:720}
\BE(\Br, t) = \Real \lr{ \BE(\Br) e^{-j \omega t} }.
\end{dmath}

For example 
\begin{dmath}\label{eqn:emtLecture5:840}
\Real( e^{i k z} e^{-i\omega t} ) = \cos( k z - \omega t ),
\end{dmath}

is identical with
\begin{dmath}\label{eqn:emtLecture5:860}
\Real( e^{-j k z} e^{j\omega t} ) = \cos( \omega t -k z),
\end{dmath}

showing that a solution or its complex conjugate is equally valid.

Engineering books use \( e^{j \omega t} \) whereas most physicists use \( e^{-i \omega t } \).

What if we have more complex time dependencies, such as that sketched in \cref{fig:lecture4NonSine:lecture4NonSineFig3}?

\imageFigure{../../figures/ece1228-emt/lecture4NonSineFig3}{Non-sinusoidal time dependence.}{fig:lecture4NonSine:lecture4NonSineFig3}{0.2}

We can do this using Fourier superposition, adding a finite or infinite set of single frequency solutions.  The first order of business is to solve the system for a single frequency.

Let's write our Fourier transform pairs as
\begin{subequations}
\label{eqn:emtLecture5:520}
\begin{equation}\label{eqn:emtLecture5:540}
\calF(\BA(\Br, t)) = 
\BA(\Br, \omega)
=
\int_{-\infty}^\infty \BA(\Br, t) e^{-j \omega t} dt
\end{equation}
\begin{equation}\label{eqn:emtLecture5:560}
\BA(\Br, t) = \calF^{-1}(\BA(\Br, \omega))
=
\inv{2\pi} 
\int_{-\infty}^\infty \BA(\Br, \omega) e^{j \omega t} d\omega.
\end{equation}
\end{subequations}

In particular

\begin{equation}\label{eqn:emtLecture5:580}
\calF\lr{ \ddt{f(t)} } = j \omega F(\omega),
\end{equation}

so the Fourier transform of the Maxwell equation
\begin{dmath}\label{eqn:emtLecture5:600}
\calF\lr{ \spacegrad \cross \BE(\Br, t) }
=
\calF\lr{ -\PD{t}{\BB}(\Br, t) },
\end{dmath}

is

\begin{dmath}\label{eqn:emtLecture5:620}
\spacegrad \cross \BE(\Br, \omega) = - j\omega \BB(\Br, \omega).
\end{dmath}

The four Maxwell's equations can be written as

\begin{itemize}
\item Faraday's Law
\begin{dmath}\label{eqn:emtLecture5:640}
\spacegrad \cross \BE( \Br, \omega ) = - j \omega \BB(\Br, \omega) - \BM_i
\end{dmath}
\item Ampere-Maxwell equation
\begin{dmath}\label{eqn:emtLecture5:660}
\spacegrad \cross \BH( \Br, \omega ) = \BJ_\txtc(\Br, \omega) + \BD(\Br, \omega)
\end{dmath}
\item Gauss's law
\begin{dmath}\label{eqn:emtLecture5:680}
\spacegrad \cdot \BD(\Br, \omega) = \rho_{\txte\txtv}(\Br, \omega)
\end{dmath}
\item Gauss's law for magnetism
\begin{dmath}\label{eqn:emtLecture5:700}
\spacegrad \cdot \BB(\Br, \omega) = \rho_{\txtm\txtv}(\Br, \omega).
\end{dmath}
\end{itemize}

Now we can more easily model non-simple media with

\begin{dmath}\label{eqn:emtLecture5:740}
\begin{aligned}
\BB(\Br, \omega) &= \mu(\omega) \BH(\Br, \omega) \\
\BD(\Br, \omega) &= \epsilon(\omega) \BE(\Br, \omega).
\end{aligned}
\end{dmath}

so Maxwell's equations are

\begin{dmath}\label{eqn:emtLecture5:760}
\spacegrad \cross \BE( \Br, \omega ) = - j \omega \mu(\omega) \BH(\Br, \omega) - \BM_i
\end{dmath}
\begin{dmath}\label{eqn:emtLecture5:780}
\spacegrad \cross \BH( \Br, \omega ) = \BJ_\txtc(\Br, \omega) + \epsilon(\omega) \BE(\Br, \omega)
\end{dmath}
\begin{dmath}\label{eqn:emtLecture5:800}
\epsilon(\omega) \spacegrad \cdot \BE(\Br, \omega) = \rho_{\txte\txtv}(\Br, \omega)
\end{dmath}
\begin{dmath}\label{eqn:emtLecture5:820}
\mu(\omega) \spacegrad \cdot \BH(\Br, \omega) = \rho_{\txtm\txtv}(\Br, \omega).
\end{dmath}

\paragraph{Frequency domain Poynting}

The frequency domain (time harmonic) equivalent of the instantaneous Poynting theorem is

\begin{dmath}\label{eqn:emtLecture5:880}
\inv{2} \oint d\Ba \cdot \lr{ \BE \cross \BH^\conj } 
- \inv{2} \int dV \lr{ \BH^\conj \cdot \BM_i + \BE \cdot \BJ_i^\conj }
+ \inv{2} \int dV \sigma \Abs{\BE}^2
+ j \omega \inv{2} \int dV \lr{ \mu \Abs{\BH}^2 - \epsilon \Abs{\BE}^2 } = 0.
\end{dmath}

Showing this will probably be given as homework.

Since

\begin{dmath}\label{eqn:emtLecture5:900}
\Real(\BA) \cross \Real(\BB) \ne \Real( \BA \cross \BB ).
\end{dmath}

We want to find the instantaneous Poynting vector in terms of the phasor fields.  Following
\citep{balanis1989advanced}, where script is used for the instantaneous quantities and non-script for the phasors, we find

\begin{dmath}\label{eqn:emtLecture5:920}
\bcS(\Br, t) 
= \bcE(\Br, t) \cross \bcH(\Br, t)
= \Real(\bcE(\Br, t)) \cross \Real(\bcH(\Br, t))
= 
\frac{ \BE e^{j\omega t} + \BE^\conj e^{-j \omega t}}{2}
\cross
\frac{ \BH e^{j\omega t} + \BH^\conj e^{-j \omega t}}{2}
=
\inv{4}
\lr{
\BE \cross \BH^\conj + \BE^\conj \cross \BH
+ 
\BE \cross \BH e^{2 j\omega t} 
+ 
\BH \cross \BE e^{-2 j\omega t} 
}
=
\inv{2} \Real(\BE \cross \BH^\conj) + \inv{2} \Real( \BE \cross \BH  e^{2 j\omega t} ).
\end{dmath}

Should we time average over a period \( \expectation{.} = (1/T) \int_0^T (.) \) the second term is killed, so that

\begin{dmath}\label{eqn:emtLecture5:940}
\expectation{ \bcS }
=
\inv{2} \Real(\BE \cross \BH^\conj) + \inv{2} \Real( \BE \cross \BH  e^{2 j\omega t} ).
\end{dmath}

The instantaneous Poynting vector is thus
\begin{dmath}\label{eqn:emtLecture5:960}
\bcS(\Br, t) = \expectation{\BS} + \inv{2} \Real\lr{ \BE \cross \BH e^{j \omega t} }.
\end{dmath}

%\EndArticle
