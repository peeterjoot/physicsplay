%
% Copyright � 2016 Peeter Joot.  All Rights Reserved.
% Licenced as described in the file LICENSE under the root directory of this GIT repository.
%
\makeproblem{Orthogonality conditions for the fields.}{emt:problemSet7:3}{ 
Consider plane waves

\begin{dmath}\label{eqn:emtproblemSet7Problem3:20}
\begin{aligned}
\BE &= \BE_0 e^{-j \Bk \cdot \Br + j \omega t } \\
\BH &= \BH_0 e^{-j \Bk \cdot \Br + j \omega t }
\end{aligned}
\end{dmath}

propagating in a homogeneous,
lossless, source free region for which \( \epsilon > 0 \), \( \mu > 0 \), and where \( \BE_0, \BH_0 \) are constant.

\makesubproblem{}{emt:problemSet7:3a}
Show that 
\( \Bk \perp \BE \) and \( \Bk \perp \BH \).
\makesubproblem{}{emt:problemSet7:3b}
Show that  \( \Bk, \BE, \BH \) form a right hand triplet as indicated in \cref{fig:kEHrightHandTriplet:kEHrightHandTripletFig1}.

\imageFigure{../../figures/ece1228-emt/kEHrightHandTripletFig1}{Right handed triplet.}{fig:kEHrightHandTriplet:kEHrightHandTripletFig1}{0.2}

\paragraph{Hint:} show that 
\( \Bk \cross \BE = \omega \mu \BH \) and 
\( \Bk \cross \BH = -\omega \epsilon \BE \).
\makesubproblem{}{emt:problemSet7:3c}
Now suppose \( \epsilon, \mu < 0 \), how does the figure change?  Redraw the figure.
} % makeproblem

\makeanswer{emt:problemSet7:3}{ 
\makeSubAnswer{}{emt:problemSet7:3a}

TODO.
\makeSubAnswer{}{emt:problemSet7:3b}

TODO.
\makeSubAnswer{}{emt:problemSet7:3c}

TODO.
}
