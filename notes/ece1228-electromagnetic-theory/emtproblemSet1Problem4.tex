%
% Copyright � 2016 Peeter Joot.  All Rights Reserved.
% Licenced as described in the file LICENSE under the root directory of this GIT repository.
%
\makeproblem{Conducting sheet with hole.}{emt:problemSet1:4}{ 
%Figure 
\Cref{fig:emtLect2:emtLect2Fig4}.
shows a flat, positive, non-conducting sheet of charge with uniform charge density \( \sigma \) [\si{C/m^2}]. A small circular hole of radius \(R \) is cut in the middle of the surface as shown.
 
\imageFigure{../../figures/ece1228-emt/emtLect2Fig4}{Conducting sheet with a hole.}{fig:emtLect2:emtLect2Fig4}{0.3}

Calculate the electric field intensity \(\BE\) at point \(P\), a distance \(z\) from the center of the hole along its axis.

Hint 1: Ignore the field fringe effects around all edges.
Hint 2: Calculate the field due to a disk of radius \(R\) and use superposition.

} % makeproblem

\makeanswer{emt:problemSet1:4}{ 

TODO.
}
