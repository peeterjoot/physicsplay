%
% Copyright � 2016 Peeter Joot.  All Rights Reserved.
% Licenced as described in the file LICENSE under the root directory of this GIT repository.
%
\makeproblem{Lossy waves.}{emt:problemSet6:2}{ 
In the case of lossy medium the wave equation was given by

\begin{dmath}\label{eqn:emtproblemSet6Problem2:20}
\spacegrad^2 \BE = \gamma^2 \BE,
\end{dmath}

where

\begin{dmath}\label{eqn:emtproblemSet6Problem2:40}
\gamma^2 = \lr{ \alpha + j \beta }^2.
\end{dmath}
Now consider a medium for which \( \epsilon(\omega) = \epsilon'(\omega) \) (i.e. \( \epsilon''(\omega) = 0 \)), \( \sigma = \sigma_0 \) (i.e. \(\omega \tau \sim 0 \) in the Drude model), and \( \mu \) is a constant and real.
For this case obtain the expression for \( \alpha \) and \(\beta\) in terms of \( \omega, \mu, \epsilon', \sigma_0 \).

The uniform plane wave
\begin{dmath}\label{eqn:emtproblemSet6Problem2:60}
\BE(\Br, t) = E_0
\lr{ \xcap \cos\theta - \zcap \sin\theta } \cos\lr{ \omega t -k \sin\theta x - k \cos\theta z }
\end{dmath}

is propagating in the \(x-z\) plane as sketched in \cref{fig:emtProblemSet6:emtProblemSet6Fig1} 
in a simple medium with \( \sigma = 0\).
\imageFigure{../../figures/ece1228-emt/emtProblemSet6Fig1}{Linear wave front.}{fig:emtProblemSet6:emtProblemSet6Fig1}{0.3}
Here, \( E_0 \) is a real constant and \( k \) is the propagation
constant. Answer the following questions and show all your
work.
\makesubproblem{}{emt:problemSet6:2a}
Determine the associated magnetic field \( \BH(\Br, t) \).
\makesubproblem{}{emt:problemSet6:2b}
Determine the time averaged Poynting vector, \( \expectation{\BS(\Br, t)} \).
\makesubproblem{}{emt:problemSet6:2c}
Determine the stored magnetic energy density, \( W_m(\Br, t) \).
\makesubproblem{}{emt:problemSet6:2d}
Determine the components of phase velocity vector \( \Bv_p \) along x and z.
} % makeproblem

\makeanswer{emt:problemSet6:2}{ 

First let's determine the requested \( \alpha \) and \( \beta \) values.  These are defined by

\begin{dmath}\label{eqn:emtproblemSet6Problem2:80}
\gamma^2 
= \lr{\alpha + j \beta}^2 
= -\mu \epsilon \omega^2 + j \omega \mu \sigma
= -\mu \lr{ \epsilon' - j \epsilon''} \omega^2 + j \omega \mu \sigma
= -\mu \epsilon' \omega^2 + j \omega \mu \lr{ \sigma + \epsilon'' \omega }.
\end{dmath}

When \( \epsilon'' = 0 \) and \( \sigma = \sigma_0 \), this is
\begin{dmath}\label{eqn:emtproblemSet6Problem2:100}
\lr{\alpha + j \beta}^2 
= -\mu \epsilon' \omega^2 + j \omega \mu \sigma_0
= \mu \epsilon' \omega^2 \lr{ -1 + j \frac{\sigma_0}{\epsilon' \omega} }
%= \sqrt{ \lr{ \mu \epsilon' \omega^2}^2 + \lr{ \omega \mu \sigma_0 }^2 }
%\exp\lr{ 
%-j \Atan\lr{\frac{ \sigma_0 }{\epsilon' \omega}}
%}
%= 
%\mu \epsilon' \omega^2
%\sqrt{ 1 + \lr{ \frac{\sigma_0}{\epsilon' \omega} }^2 }
%\exp\lr{ 
%-j \Atan\lr{\frac{ \sigma_0 }{\epsilon' \omega}}
%},
\end{dmath}

To attempt to take the square root of \( z = -1 + j t \), I wrote

\begin{dmath}\label{eqn:emtproblemSet6Problem2:120}
z = \sqrt{ 1 + t^2 } e^{-j\Atan t },
\end{dmath}

so
\begin{dmath}\label{eqn:emtproblemSet6Problem2:140}
\sqrt{z} 
= \lr{ 1 + t^2 }^{1/4} e^{-\frac{j}{2}\Atan t }
= 
\lr{ 1 + t^2 }^{1/4} 
\lr{ 
\cos\lr{ \inv{2} \Atan t }
- j \sin\lr{ \inv{2} \Atan t } }.
\end{dmath}

Using 
\begin{dmath}\label{eqn:emtproblemSet6Problem2:160}
\begin{aligned}
\cos\lr{ \inv{2} \Atan t } &= \inv{\sqrt{2}} \sqrt{ 1 + \inv{\sqrt{1 + t^2}}} \\
\sin\lr{ \inv{2} \Atan t } &= \frac{t}{\sqrt{2}} \inv{ \sqrt{ 1 + t^2} \sqrt{ 1 + \inv{\sqrt{1 + t^2}}}}.
\end{aligned}
\end{dmath}

Some simplification yields

\begin{dmath}\label{eqn:emtproblemSet6Problem2:180}
\sqrt{z} = \inv{\sqrt{2}} \sqrt{ 1 + \sqrt{ 1 + t^2} } - j \frac{t}{\sqrt{2} \sqrt{ 1 + \sqrt{ 1 + t^2} }},
\end{dmath}

however, squaring that yields \( 1 - j t \) and not \( -1 + j t \), indicating that a \( j^2 \)  phase factor was lost somewhere in the algebra (probably related to the quadrant ambiguity of the arctan).  Inserting that phase factor back in we find

\boxedEquation{eqn:emtproblemSet6Problem2:200}{
\begin{aligned}
\alpha &= \inv{\sqrt{2}} \frac{\sigma_0 \sqrt{\frac{\mu}{\epsilon'}}} { \sqrt{ 1 + \sqrt{ 1 + \lr{\frac{\sigma_0}{\epsilon' \omega} }^2 }} } \\
\beta &= \inv{\sqrt{2}} \omega \sqrt{\mu\epsilon'} \sqrt{ 1 + \sqrt{ 1 + \lr{\frac{\sigma_0}{\epsilon' \omega} }^2 }}.
\end{aligned}
}

A computation of \( (\alpha + j \beta)^2 \) shows that this yields \( -\mu \epsilon' \omega^2 + j \omega \mu \sigma_0 \) as expected.

\makeSubAnswer{}{emt:problemSet6:2a}
\makeSubAnswer{}{emt:problemSet6:2b}
\makeSubAnswer{}{emt:problemSet6:2c}
\makeSubAnswer{}{emt:problemSet6:2d}

TODO.
}
