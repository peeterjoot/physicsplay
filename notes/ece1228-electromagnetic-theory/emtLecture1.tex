%
% Copyright � 2016 Peeter Joot.  All Rights Reserved.
% Licenced as described in the file LICENSE under the root directory of this GIT repository.
%
%\newcommand{\authorname}{Peeter Joot}
\newcommand{\email}{peeterjoot@protonmail.com}
\newcommand{\basename}{FIXMEbasenameUndefined}
\newcommand{\dirname}{notes/FIXMEdirnameUndefined/}

%\renewcommand{\basename}{emt1}
%\renewcommand{\dirname}{notes/ece1228/}
%\newcommand{\keywords}{ECE1228H}
%\newcommand{\authorname}{Peeter Joot}
\newcommand{\onlineurl}{http://sites.google.com/site/peeterjoot2/math2013/\basename.pdf}
\newcommand{\sourcepath}{\dirname\basename.tex}
\newcommand{\generatetitle}[1]{\chapter{#1}}

\newcommand{\vcsinfo}{%
\section*{}
\noindent{\color{DarkOliveGreen}{\rule{\linewidth}{0.1mm}}}
\paragraph{Document version}
%\paragraph{\color{Maroon}{Document version}}
{
\small
\begin{itemize}
\item Available online at:\\ 
\href{\onlineurl}{\onlineurl}
\item Git Repository: \input{./.revinfo/gitRepo.tex}
\item Source: \sourcepath
\item last commit: \input{./.revinfo/gitCommitString.tex}
\item commit date: \input{./.revinfo/gitCommitDate.tex}
\end{itemize}
}
}

%\PassOptionsToPackage{dvipsnames,svgnames}{xcolor}
\PassOptionsToPackage{square,numbers}{natbib}
\documentclass{scrreprt}

\usepackage[left=2cm,right=2cm]{geometry}
\usepackage[svgnames]{xcolor}
\usepackage{peeters_layout}

\usepackage{natbib}

\usepackage[
colorlinks=true,
bookmarks=false,
pdfauthor={\authorname, \email},
backref 
]{hyperref}

% http://tex.stackexchange.com/questions/75773/how-to-reference-problems-by-the-text-label-in-an-exercise-envioronment
\usepackage[english]{cleveref}
\crefname{Exercise}{exercise}{exercises}
\Crefname{Exercise}{Exercise}{Exercises}

\RequirePackage{titlesec}
\RequirePackage{ifthen}

% http://stackoverflow.com/questions/4932910/date-in-the-tabular-environment
\makeatletter
\let\insertdate\@date
\makeatother

\titleformat{\chapter}[display]
{\bfseries\Large}
{\color{DarkSlateGrey}\filleft \authorname
\ifthenelse{\isundefined{\studentnumber}}{}{\\ \studentnumber}
\ifthenelse{\isundefined{\email}}{}{\\ \email}
\ifthenelse{\isundefined{\dateintitle}}{}{\\ \insertdate}
%\ifthenelse{\isundefined{\coursename}}{}{\\ \coursename} % put in title instead.
}
{4ex}
{\color{DarkOliveGreen}{\titlerule}\color{Maroon}
\vspace{2ex}%
\filright}
[\vspace{2ex}%
\color{DarkOliveGreen}\titlerule
]

\newcommand{\beginArtWithToc}[0]{\begin{document}\tableofcontents}
\newcommand{\beginArtNoToc}[0]{\begin{document}}
\newcommand{\EndNoBibArticle}[0]{\end{document}}
\newcommand{\EndArticle}[0]{\bibliography{Bibliography}\bibliographystyle{plainnat}\end{document}}

% 
%\newcommand{\citep}[1]{\cite{#1}}

\colorSectionsForArticle


%
%%\usepackage{ece1228}
%\usepackage{peeters_braket}
%%\usepackage{peeters_layout_exercise}
%\usepackage{peeters_figures}
%\usepackage{mathtools}
%\usepackage{siunitx}
%
%\beginArtNoToc
%\generatetitle{ECE1228H Electromagnetic Theory.  Lecture 1: Introduction.  Taught by Prof.\ M. Mojahedi}
\chapter{Introduction}
%\label{chap:emt1}
%
%\paragraph{Disclaimer}
%
%Peeter's lecture notes from class.  These may be incoherent and rough.
%
%These are notes for the UofT course ECE1228H, Electromagnetic Theory, taught by Prof. M. Mojahedi, covering \textchapref{{1}} \%citep{balanis1989advanced} content.

\paragraph{Maxwell's equations}
\index{Maxwell's equations!time domain}

\begin{itemize}
\item Faraday's Law
\begin{dmath}\label{eqn:emtLecture1:20}
\spacegrad \cross \BE( \Br, t ) = - \PD{t}{\BB}(\Br, t) - \BM_i
\end{dmath}
\item Ampere-Maxwell equation
\begin{dmath}\label{eqn:emtLecture1:40}
\spacegrad \cross \BH( \Br, t ) = \BJ_\txtc(\Br, t) + \PD{t}{\BD}(\Br, t)
\end{dmath}
\item Gauss's law
\begin{dmath}\label{eqn:emtLecture1:80}
\spacegrad \cdot \BD(\Br, t) = \rho_{\txte\txtv}(\Br, t)
\end{dmath}
\item Gauss's law for magnetism
\begin{dmath}\label{eqn:emtLecture1:100}
\spacegrad \cdot \BB(\Br, t) = \rho_{\txtm\txtv}(\Br, t)
\end{dmath}
\end{itemize}

After unpacking, we have a total of eight equations, with four vectoral field variables, and 8 sources, all interrelated by partial derivatives in space and time coordinates.  

It will be left to homework to show that without the displacement current \( \PDi{t}{\BD} \), these equations will not satisfy conservation relations.

The fields are and sources are
\index{units}
\begin{itemize}
\item \( \BE \) Electric field intensity \si{V/m}.
\item \( \BB \) Magnetic flux density \si{V s/m^2} (or Tesla).
\item \( \BH \) Magnetic field intensity \si{A/m}.
\item \( \BD \) Electric flux density \si{C/m^2}.
\item \( \rho_{\txte\txtv} \) Electric charge volume density
\item \( \rho_{\txtm\txtv} \) Magnetic charge volume density
\item \( \BJ_{\txtc} \) Impressed (source) electric current density \si{A/m^2}.  This is the charge passing through a plane in a unit time.  Here \( \txtc \) is for ``conduction''.
\item \( \BM_{\txti} \) Impressed (source) magnetic current density \si{V/m^2}
\end{itemize}

In an undergrad context we'll have seen the electric and magnetic fields in the Lorentz force law

\begin{dmath}\label{eqn:emtLecture1:120}
\BF = q \Bv \cross \BB + q\BE.
\end{dmath}

In SI there are 7 basic units.  These include

\begin{itemize}
\item Length \si{m}.
\item mass \si{kg}.
\item Time \si{s}.
\item Ampere \si{A}.
\index{unit!ampere}
\item Kelvin \si{K} (temperature)
\index{unit!Kelvin}
\item Candela (luminous intensity)
\index{unit!candela}
\item Mole (amount of substance)
\index{unit!mole}
\end{itemize}

\index{unit!Coulomb}
Note that the Coulomb is not a fundamental unit, but the Ampere is.  This is because it is easier to measure.

For homework: show that magnetic field lines must close on themselves when there are no magnetic sources (zero divergence).  This is opposed to electric fields that spread out from the charge.

%\EndNoBibArticle
