%
% Copyright � 2016 Peeter Joot.  All Rights Reserved.
% Licenced as described in the file LICENSE under the root directory of this GIT repository.
%
\newcommand{\authorname}{Peeter Joot}
\newcommand{\email}{peeterjoot@protonmail.com}
\newcommand{\basename}{FIXMEbasenameUndefined}
\newcommand{\dirname}{notes/FIXMEdirnameUndefined/}

\renewcommand{\basename}{emt9}
\renewcommand{\dirname}{notes/ece1228/}
\newcommand{\keywords}{ECE1228H}
\newcommand{\authorname}{Peeter Joot}
\newcommand{\onlineurl}{http://sites.google.com/site/peeterjoot2/math2013/\basename.pdf}
\newcommand{\sourcepath}{\dirname\basename.tex}
\newcommand{\generatetitle}[1]{\chapter{#1}}

\newcommand{\vcsinfo}{%
\section*{}
\noindent{\color{DarkOliveGreen}{\rule{\linewidth}{0.1mm}}}
\paragraph{Document version}
%\paragraph{\color{Maroon}{Document version}}
{
\small
\begin{itemize}
\item Available online at:\\ 
\href{\onlineurl}{\onlineurl}
\item Git Repository: \input{./.revinfo/gitRepo.tex}
\item Source: \sourcepath
\item last commit: \input{./.revinfo/gitCommitString.tex}
\item commit date: \input{./.revinfo/gitCommitDate.tex}
\end{itemize}
}
}

%\PassOptionsToPackage{dvipsnames,svgnames}{xcolor}
\PassOptionsToPackage{square,numbers}{natbib}
\documentclass{scrreprt}

\usepackage[left=2cm,right=2cm]{geometry}
\usepackage[svgnames]{xcolor}
\usepackage{peeters_layout}

\usepackage{natbib}

\usepackage[
colorlinks=true,
bookmarks=false,
pdfauthor={\authorname, \email},
backref 
]{hyperref}

% http://tex.stackexchange.com/questions/75773/how-to-reference-problems-by-the-text-label-in-an-exercise-envioronment
\usepackage[english]{cleveref}
\crefname{Exercise}{exercise}{exercises}
\Crefname{Exercise}{Exercise}{Exercises}

\RequirePackage{titlesec}
\RequirePackage{ifthen}

% http://stackoverflow.com/questions/4932910/date-in-the-tabular-environment
\makeatletter
\let\insertdate\@date
\makeatother

\titleformat{\chapter}[display]
{\bfseries\Large}
{\color{DarkSlateGrey}\filleft \authorname
\ifthenelse{\isundefined{\studentnumber}}{}{\\ \studentnumber}
\ifthenelse{\isundefined{\email}}{}{\\ \email}
\ifthenelse{\isundefined{\dateintitle}}{}{\\ \insertdate}
%\ifthenelse{\isundefined{\coursename}}{}{\\ \coursename} % put in title instead.
}
{4ex}
{\color{DarkOliveGreen}{\titlerule}\color{Maroon}
\vspace{2ex}%
\filright}
[\vspace{2ex}%
\color{DarkOliveGreen}\titlerule
]

\newcommand{\beginArtWithToc}[0]{\begin{document}\tableofcontents}
\newcommand{\beginArtNoToc}[0]{\begin{document}}
\newcommand{\EndNoBibArticle}[0]{\end{document}}
\newcommand{\EndArticle}[0]{\bibliography{Bibliography}\bibliographystyle{plainnat}\end{document}}

% 
%\newcommand{\citep}[1]{\cite{#1}}

\colorSectionsForArticle



%\usepackage{ece1228}
\usepackage{peeters_braket}
%\usepackage{peeters_layout_exercise}
\usepackage{peeters_figures}
\usepackage{mathtools}
\usepackage{siunitx}

\beginArtNoToc
\generatetitle{ECE1228H Electromagnetic Theory.  Lecture 9: XXX.  Taught by Prof.\ M. Mojahedi}
%\chapter{XXX}
\label{chap:emt9}

\paragraph{Disclaimer}

Peeter's lecture notes from class.  These may be incoherent and rough.

These are notes for the UofT course ECE1228H, Electromagnetic Theory, taught by Prof. M. Mojahedi, covering \textchapref{{1}} \citep{balanis1989advanced} content.

\paragraph{YYY}

When the magnitude and direction of the fields remain constant, we call this a plane wave.  This is a non-physical entity, but can be made physical with superposition.


On dimensions, note that

\begin{dmath}\label{eqn:emtLecture9:20}
k
= \frac{\omega}{c} n
= \omega \sqrt{ \mu_0 \epsilon_0 } \sqrt{\mu_r} \sqrt{\epsilon_r}
\end{dmath}

so the magnetic field for a plane wave is

\begin{dmath}\label{eqn:emtLecture9:40}
\BH
= \frac{k}{\omega \mu} \kcap \cross \BE
= \frac{ \omega \sqrt{ \mu_0 \epsilon_0 } \sqrt{\mu_r} \sqrt{\epsilon_r} }{\omega \mu} \kcap \cross \BE
= \frac{ \sqrt{ \mu_0 \epsilon_0 } \sqrt{\mu_r} \sqrt{\epsilon_r} }{\mu} \kcap \cross \BE
= \sqrt{ \frac{ \epsilon_0 \epsilon_r }{\mu_0\mu_r } } \kcap \cross \BE
= \inv{ \sqrt{\mu/\epsilon} } \kcap \cross \BE
= \inv{ \eta } \kcap \cross \BE,
\end{dmath}

where
\begin{dmath}\label{eqn:emtLecture9:60}
\eta = \sqrt{\mu/\epsilon}
\end{dmath}

is the medium intrinsic impedance and we can define
\begin{dmath}\label{eqn:emtLecture9:80}
\eta_0 = \sqrt{\mu_0/\epsilon_0} = 377 \,\Omega,
\end{dmath}

as the free space intrinsic impedance.  This was argued as a definitive superiority of the SI unit system, since this impedance reationship is made clear, something that would not be the case in CGS.

\paragraph{Rant on negative index of refraction, diffraction, uncertainty, and related concepts}

Note: poynting vector and \( \Bk \) only colinear in simple isotropic media!

Note: When \( \epsilon \) and \( \mu \) are both negative, we must pick the negative sign for \( n = -\sqrt{\epsilon\mu} \).  Recall that we saw negative \( \epsilon \) for low frequencies for metals in the Druid model where \( \epsilon = 1 - \omega_p^2/\omega^2 \).

Negative \( \mu \) can be obtained in the microwave domain in the Lorentzian model.  Pandree did this with split ring resonators.

Smith did both negative \( epsilon \) and \( \mu \) together and measured the transmission of such a wave.  It was shown that instead of attenuation and reflection there was transmission.

One theorized application of negative index of refraction is to construct a ``perfect lens'' to use a flat lens to image a source.  This has been done in very near field applications with such a setup.

Of interest is Abbe's diffraction limit for optical systems \citep{wiki:abbe}, defined by

\begin{dmath}\label{eqn:emtLecture9:100}
\Delta r \ge \frac{\lambda_0}{2 n}.
\end{dmath}

F4

which is an application of the constraint equation

\begin{dmath}\label{eqn:emtLecture9:120}
\Bk^2 = \lr{\frac{\omega n}{c}}^2
\end{dmath}

Also of interest:

Negative dielectric paper: Koboyashi

\paragraph{Transverse propagation}

The plane wave solution to Maxwell's equations, which are transverse, unlike sound (compression waves), begs us to wonder what the wave is travelling through.

Maxwell postulated a medium, the Ether, that would carry this wave.  The Michaelson-Morley experiment is considered the definitive proof that no such medium exists.



\EndArticle
%\EndNoBibArticle
