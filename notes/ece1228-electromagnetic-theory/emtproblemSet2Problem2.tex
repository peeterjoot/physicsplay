%
% Copyright � 2016 Peeter Joot.  All Rights Reserved.
% Licenced as described in the file LICENSE under the root directory of this GIT repository.
%
\makeproblem{Infinite line charge.}{emt:problemSet2:2}{ 

An infinitely long straight line charge has a constant charge density \( \rho_\txtl \) [\si{C/m}].

\makesubproblem{}{emt:problemSet2:2a}
Using the integral formulation for \( \BE \)
discussed in the class calculate the electric
field at an arbitrary point \( \BA(\rho, \phi, z) \).
\makesubproblem{}{emt:problemSet2:2b}

Using the Gauss law calculate the same as \partref{emt:problemSet2:2a}.

\makesubproblem{}{emt:problemSet2:2c}

Now suppose that our uniformly charged (\( \rho_\txtl \) constant) 
has a finite extension from \( z = a \) to \( z = b \), as sketched in \cref{fig:problemset2:problemset2Fig2}.
\imageFigure{../../figures/ece1228-emt/problemset2Fig2}{Line charge.}{fig:problemset2:problemset2Fig2}{0.3}
Find the electric field at the
arbitrary point \( \BA \).

Note: Express your results in cylindrical coordinate
system.
} % makeproblem

\makeanswer{emt:problemSet2:2}{ 
\makeSubAnswer{}{emt:problemSet2:2a}

Since the line charge is of infinite length and rotationally symmetric with respect to \( \phi \), the observation point can be positioned anywhere convienient, such as

\begin{dmath}\label{eqn:emtProblemSet2Problem2:20}
\BA = \rho \rhocap.
\end{dmath}

Let the point on the wire be
\begin{dmath}\label{eqn:emtProblemSet2Problem2:40}
\Br' = z \zcap.
\end{dmath}

The absolute distance between the observation point and the element of charge is
\begin{dmath}\label{eqn:emtProblemSet2Problem2:60}
\Abs{\BA - \Br'} = \sqrt{ (\rho \rhocap)^2 + (z \zcap)^2} = \sqrt{ \rho^2 + z^2 }.
\end{dmath}

The electric field can now be expressed in integral form
\begin{dmath}\label{eqn:emtProblemSet2Problem2:80}
\BE(\BA) 
=
\inv{ 4 \pi \epsilon_0 } \int_{-\infty}^\infty dz \rho_\txtl \frac{ \rho \rhocap - z \zcap }{ \lr{\rho^2 + z^2}^{3/2} }
=
\frac{\sigma_\txtl \BA }{ 4 \pi \epsilon_0 } \int_{-\infty}^\infty dz \inv{ \lr{\rho^2 + z^2}^{3/2} }
=
\frac{\sigma_\txtl \BA }{ 4 \pi \epsilon_0 \rho^3 } \int_{-\infty}^\infty dz \inv{ \lr{1 + (z/\rho)^2}^{3/2} }
=
\frac{\sigma_\txtl \BA }{ 4 \pi \epsilon_0 \rho^2 } \int_{-\infty}^\infty du \inv{ \lr{1 + u^2}^{3/2} }.
\end{dmath}

The \( \zcap \) term was killed since the integrand was an odd function in \( z \).  Note that

\begin{dmath}\label{eqn:emtProblemSet2Problem2:100}
   \int \frac{du}{ \lr{1 + u^2}^{3/2} }
= \frac{u}{\sqrt{1 + u^2}},
\end{dmath}

which has a PV limit over \([-\infty,\infty]\) of 2.  This gives

%\begin{dmath}\label{eqn:emtProblemSet2Problem2:120}
\boxedEquation{eqn:emtProblemSet2Problem2:120}{
\BE(\rho) 
=
\inv{ 4 \pi \epsilon_0 } 
\frac{2 \sigma_\txtl }{\rho} \rhocap.
}
%\end{dmath}

\makeSubAnswer{}{emt:problemSet2:2b}

Using Gauss's law integrating over a cylindrical segment surrounding the wire, we have

\begin{dmath}\label{eqn:emtProblemSet2Problem2:n}
\Delta z \frac{\rho_\txtl}{\epsilon_0} 
= \int \spacegrad \cdot \BE dV
= \oint \ncap \cdot \BE dA
= E_\rho(\rho) (2 \pi \rho \Delta z),
\end{dmath}

This assumes the end surfaces at infinity contribute nothing to the surface integral, and gives after rearrangement

\begin{dmath}\label{eqn:emtProblemSet2Problem2:n}
E_\rho(\rho) 
= \inv{2 \pi \rho } \frac{\rho_\txtl}{\epsilon_0}.
\end{dmath}

This matches \cref{eqn:emtProblemSet2Problem2:120} as expected.

\makeSubAnswer{}{emt:problemSet2:2c}

TODO.
}
