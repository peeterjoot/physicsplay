%
% Copyright � 2016 Peeter Joot.  All Rights Reserved.
% Licenced as described in the file LICENSE under the root directory of this GIT repository.
%
\makeproblem{Infinite line charge.}{emt:problemSet2:2}{ 

An infinitely long straight line charge has a constant charge density \( \rho_\txtl \) [\si{C/m}].

\makesubproblem{}{emt:problemSet2:2a}
Using the integral formulation for \( \BE \)
discussed in the class calculate the electric
field at an arbitrary point \( \BA(\rho, \phi, z) \).
\makesubproblem{}{emt:problemSet2:2b}

Using the Gauss law calculate the same as \partref{emt:problemSet2:2a}.

\makesubproblem{}{emt:problemSet2:2c}

Now suppose that our uniformly charged (\( \rho_\txtl \) constant) 
has a finite extension from \( z = a \) to \( z = b \), as sketched in \cref{fig:problemset2:problemset2Fig2}.
\imageFigure{../../figures/ece1228-emt/problemset2Fig2}{Line charge.}{fig:problemset2:problemset2Fig2}{0.3}
Find the electric field at the
arbitrary point \( \BA \).

Note: Express your results in cylindrical coordinate
system.
} % makeproblem

\makeanswer{emt:problemSet2:2}{ 
\makeSubAnswer{}{emt:problemSet2:2a}

TODO.
\makeSubAnswer{}{emt:problemSet2:2b}

TODO.
\makeSubAnswer{}{emt:problemSet2:2c}

TODO.
}
