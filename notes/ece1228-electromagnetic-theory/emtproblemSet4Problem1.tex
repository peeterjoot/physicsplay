%
% Copyright � 2016 Peeter Joot.  All Rights Reserved.
% Licenced as described in the file LICENSE under the root directory of this GIT repository.
%
\makeproblem{Index of refraction.}{emt:problemSet4:1}{ 
Transmitter \( T \) of a time-harmonic wave of frequency \( \nu \) moves with velocity \( \BU \)
at
an angle \( \theta \) relative to the direct line to a stationary receiver \( R \), as sketched in
\cref{fig:ps4:ps4Fig1}.
\imageFigure{../../figures/ece1228-emt/ps4Fig1}{Field refraction.}{fig:ps4:ps4Fig1}{0.3}
\makesubproblem{}{emt:problemSet4:1a}
Derive the expression for the frequency detected by the receiver \(R\), assuming that the
medium between \(T\) and \(R\) has a positive index of refraction \(n\). (Apply the appropriate
approximations.)

\makesubproblem{}{emt:problemSet4:1b}
How is the expression obtained in 
\partref{emt:problemSet4:1a}
is modified if the medium is a metamaterial
with negative index of refraction.
\makesubproblem{}{emt:problemSet4:1c}
From the physical point of view, how is the situation in 
\partref{emt:problemSet4:1b}
different from 
\partref{emt:problemSet4:1a}
?
} % makeproblem

\makeanswer{emt:problemSet4:1}{ 
\makeSubAnswer{}{emt:problemSet4:1a}

TODO.
\makeSubAnswer{}{emt:problemSet4:1b}

TODO.
\makeSubAnswer{}{emt:problemSet4:1c}

TODO.
}
