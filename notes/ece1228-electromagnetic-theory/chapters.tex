%
% Copyright � 2016 Peeter Joot.  All Rights Reserved.
% Licenced as described in the file LICENSE under the root directory of this GIT repository.
%
%----------------------------------------------------------------------------------------
%
% Prof. decoder ring:
%
% antigreat == integrate
% ambeeguis == ambiguous
% eikenvector == eigenvector
% 
%----------------------------------------------------------------------------------------
\part{Lecture notes}
   %\chapter{Electromagnetic fields}
   %
% Copyright � 2016 Peeter Joot.  All Rights Reserved.
% Licenced as described in the file LICENSE under the root directory of this GIT repository.
%
%\newcommand{\authorname}{Peeter Joot}
\newcommand{\email}{peeterjoot@protonmail.com}
\newcommand{\basename}{FIXMEbasenameUndefined}
\newcommand{\dirname}{notes/FIXMEdirnameUndefined/}

%\renewcommand{\basename}{emt1}
%\renewcommand{\dirname}{notes/ece1228/}
%\newcommand{\keywords}{ECE1228H}
%\newcommand{\authorname}{Peeter Joot}
\newcommand{\onlineurl}{http://sites.google.com/site/peeterjoot2/math2013/\basename.pdf}
\newcommand{\sourcepath}{\dirname\basename.tex}
\newcommand{\generatetitle}[1]{\chapter{#1}}

\newcommand{\vcsinfo}{%
\section*{}
\noindent{\color{DarkOliveGreen}{\rule{\linewidth}{0.1mm}}}
\paragraph{Document version}
%\paragraph{\color{Maroon}{Document version}}
{
\small
\begin{itemize}
\item Available online at:\\ 
\href{\onlineurl}{\onlineurl}
\item Git Repository: \input{./.revinfo/gitRepo.tex}
\item Source: \sourcepath
\item last commit: \input{./.revinfo/gitCommitString.tex}
\item commit date: \input{./.revinfo/gitCommitDate.tex}
\end{itemize}
}
}

%\PassOptionsToPackage{dvipsnames,svgnames}{xcolor}
\PassOptionsToPackage{square,numbers}{natbib}
\documentclass{scrreprt}

\usepackage[left=2cm,right=2cm]{geometry}
\usepackage[svgnames]{xcolor}
\usepackage{peeters_layout}

\usepackage{natbib}

\usepackage[
colorlinks=true,
bookmarks=false,
pdfauthor={\authorname, \email},
backref 
]{hyperref}

% http://tex.stackexchange.com/questions/75773/how-to-reference-problems-by-the-text-label-in-an-exercise-envioronment
\usepackage[english]{cleveref}
\crefname{Exercise}{exercise}{exercises}
\Crefname{Exercise}{Exercise}{Exercises}

\RequirePackage{titlesec}
\RequirePackage{ifthen}

% http://stackoverflow.com/questions/4932910/date-in-the-tabular-environment
\makeatletter
\let\insertdate\@date
\makeatother

\titleformat{\chapter}[display]
{\bfseries\Large}
{\color{DarkSlateGrey}\filleft \authorname
\ifthenelse{\isundefined{\studentnumber}}{}{\\ \studentnumber}
\ifthenelse{\isundefined{\email}}{}{\\ \email}
\ifthenelse{\isundefined{\dateintitle}}{}{\\ \insertdate}
%\ifthenelse{\isundefined{\coursename}}{}{\\ \coursename} % put in title instead.
}
{4ex}
{\color{DarkOliveGreen}{\titlerule}\color{Maroon}
\vspace{2ex}%
\filright}
[\vspace{2ex}%
\color{DarkOliveGreen}\titlerule
]

\newcommand{\beginArtWithToc}[0]{\begin{document}\tableofcontents}
\newcommand{\beginArtNoToc}[0]{\begin{document}}
\newcommand{\EndNoBibArticle}[0]{\end{document}}
\newcommand{\EndArticle}[0]{\bibliography{Bibliography}\bibliographystyle{plainnat}\end{document}}

% 
%\newcommand{\citep}[1]{\cite{#1}}

\colorSectionsForArticle


%
%%\usepackage{ece1228}
%\usepackage{peeters_braket}
%%\usepackage{peeters_layout_exercise}
%\usepackage{peeters_figures}
%\usepackage{mathtools}
%\usepackage{siunitx}
%
%\beginArtNoToc
%\generatetitle{ECE1228H Electromagnetic Theory.  Lecture 1: Introduction.  Taught by Prof.\ M. Mojahedi}
\chapter{Introduction}
%\label{chap:emt1}
%
%\paragraph{Disclaimer}
%
%Peeter's lecture notes from class.  These may be incoherent and rough.
%
%These are notes for the UofT course ECE1228H, Electromagnetic Theory, taught by Prof. M. Mojahedi, covering \textchapref{{1}} \%citep{balanis1989advanced} content.

\paragraph{Maxwell's equations}
\index{Maxwell's equations!time domain}

\begin{itemize}
\item Faraday's Law
\begin{dmath}\label{eqn:emtLecture1:20}
\spacegrad \cross \BE( \Br, t ) = - \PD{t}{\BB}(\Br, t) - \BM_i
\end{dmath}
\item Ampere-Maxwell equation
\begin{dmath}\label{eqn:emtLecture1:40}
\spacegrad \cross \BH( \Br, t ) = \BJ_\txtc(\Br, t) + \PD{t}{\BD}(\Br, t)
\end{dmath}
\item Gauss's law
\begin{dmath}\label{eqn:emtLecture1:80}
\spacegrad \cdot \BD(\Br, t) = \rho_{\txte\txtv}(\Br, t)
\end{dmath}
\item Gauss's law for magnetism
\begin{dmath}\label{eqn:emtLecture1:100}
\spacegrad \cdot \BB(\Br, t) = \rho_{\txtm\txtv}(\Br, t)
\end{dmath}
\end{itemize}

After unpacking, we have a total of eight equations, with four vectoral field variables, and 8 sources, all interrelated by partial derivatives in space and time coordinates.  

It will be left to homework to show that without the displacement current \( \PDi{t}{\BD} \), these equations will not satisfy conservation relations.

The fields are and sources are
\index{units}
\begin{itemize}
\item \( \BE \) Electric field intensity \si{V/m}.
\item \( \BB \) Magnetic flux density \si{V s/m^2} (or Tesla).
\item \( \BH \) Magnetic field intensity \si{A/m}.
\item \( \BD \) Electric flux density \si{C/m^2}.
\item \( \rho_{\txte\txtv} \) Electric charge volume density
\item \( \rho_{\txtm\txtv} \) Magnetic charge volume density
\item \( \BJ_{\txtc} \) Impressed (source) electric current density \si{A/m^2}.  This is the charge passing through a plane in a unit time.  Here \( \txtc \) is for ``conduction''.
\item \( \BM_{\txti} \) Impressed (source) magnetic current density \si{V/m^2}
\end{itemize}

In an undergrad context we'll have seen the electric and magnetic fields in the Lorentz force law

\begin{dmath}\label{eqn:emtLecture1:120}
\BF = q \Bv \cross \BB + q\BE.
\end{dmath}

In SI there are 7 basic units.  These include

\begin{itemize}
\item Length \si{m}.
\item mass \si{kg}.
\item Time \si{s}.
\item Ampere \si{A}.
\index{unit!ampere}
\item Kelvin \si{K} (temperature)
\index{unit!Kelvin}
\item Candela (luminous intensity)
\index{unit!candela}
\item Mole (amount of substance)
\index{unit!mole}
\end{itemize}

\index{unit!Coulomb}
Note that the Coulomb is not a fundamental unit, but the Ampere is.  This is because it is easier to measure.

For homework: show that magnetic field lines must close on themselves when there are no magnetic sources (zero divergence).  This is opposed to electric fields that spread out from the charge.

%\EndNoBibArticle

   %
% Copyright � 2016 Peeter Joot.  All Rights Reserved.
% Licenced as described in the file LICENSE under the root directory of this GIT repository.
%
%\newcommand{\authorname}{Peeter Joot}
\newcommand{\email}{peeterjoot@protonmail.com}
\newcommand{\basename}{FIXMEbasenameUndefined}
\newcommand{\dirname}{notes/FIXMEdirnameUndefined/}

%\renewcommand{\basename}{emt2}
%\renewcommand{\dirname}{notes/ece1228/}
%\newcommand{\keywords}{ECE1228H}
%\newcommand{\authorname}{Peeter Joot}
\newcommand{\onlineurl}{http://sites.google.com/site/peeterjoot2/math2013/\basename.pdf}
\newcommand{\sourcepath}{\dirname\basename.tex}
\newcommand{\generatetitle}[1]{\chapter{#1}}

\newcommand{\vcsinfo}{%
\section*{}
\noindent{\color{DarkOliveGreen}{\rule{\linewidth}{0.1mm}}}
\paragraph{Document version}
%\paragraph{\color{Maroon}{Document version}}
{
\small
\begin{itemize}
\item Available online at:\\ 
\href{\onlineurl}{\onlineurl}
\item Git Repository: \input{./.revinfo/gitRepo.tex}
\item Source: \sourcepath
\item last commit: \input{./.revinfo/gitCommitString.tex}
\item commit date: \input{./.revinfo/gitCommitDate.tex}
\end{itemize}
}
}

%\PassOptionsToPackage{dvipsnames,svgnames}{xcolor}
\PassOptionsToPackage{square,numbers}{natbib}
\documentclass{scrreprt}

\usepackage[left=2cm,right=2cm]{geometry}
\usepackage[svgnames]{xcolor}
\usepackage{peeters_layout}

\usepackage{natbib}

\usepackage[
colorlinks=true,
bookmarks=false,
pdfauthor={\authorname, \email},
backref 
]{hyperref}

% http://tex.stackexchange.com/questions/75773/how-to-reference-problems-by-the-text-label-in-an-exercise-envioronment
\usepackage[english]{cleveref}
\crefname{Exercise}{exercise}{exercises}
\Crefname{Exercise}{Exercise}{Exercises}

\RequirePackage{titlesec}
\RequirePackage{ifthen}

% http://stackoverflow.com/questions/4932910/date-in-the-tabular-environment
\makeatletter
\let\insertdate\@date
\makeatother

\titleformat{\chapter}[display]
{\bfseries\Large}
{\color{DarkSlateGrey}\filleft \authorname
\ifthenelse{\isundefined{\studentnumber}}{}{\\ \studentnumber}
\ifthenelse{\isundefined{\email}}{}{\\ \email}
\ifthenelse{\isundefined{\dateintitle}}{}{\\ \insertdate}
%\ifthenelse{\isundefined{\coursename}}{}{\\ \coursename} % put in title instead.
}
{4ex}
{\color{DarkOliveGreen}{\titlerule}\color{Maroon}
\vspace{2ex}%
\filright}
[\vspace{2ex}%
\color{DarkOliveGreen}\titlerule
]

\newcommand{\beginArtWithToc}[0]{\begin{document}\tableofcontents}
\newcommand{\beginArtNoToc}[0]{\begin{document}}
\newcommand{\EndNoBibArticle}[0]{\end{document}}
\newcommand{\EndArticle}[0]{\bibliography{Bibliography}\bibliographystyle{plainnat}\end{document}}

% 
%\newcommand{\citep}[1]{\cite{#1}}

\colorSectionsForArticle


%
%%\usepackage{ece1228}
%\usepackage{peeters_braket}
%%\usepackage{peeters_layout_exercise}
%\usepackage{peeters_figures}
%\usepackage{mathtools}
%\usepackage{siunitx}
%
%\beginArtNoToc
%\generatetitle{ECE1228H Electromagnetic Theory.  Lecture 2: Boundaries.  Taught by Prof.\ M. Mojahedi}
\chapter{Boundaries}
\label{chap:emt2}

%\paragraph{Disclaimer}
%
%Peeter's lecture notes from class.  These may be incoherent and rough.
%
%These are notes for the UofT course ECE1228H, Electromagnetic Theory, taught by Prof. M. Mojahedi, covering \textchapref{{1}} \citep{balanis1989advanced} content.
%
\paragraph{Integral forms}

Given Maxwell's equations at a point

\begin{dmath}\label{eqn:emtLecture2:20}
\begin{aligned}
\spacegrad \cross \BE &= -\PD{t}{\BB} \\
\spacegrad \cross \BH &= \BJ + \PD{t}{\BD} \\
\spacegrad \cdot \BD &= \rho_\txtv \\
\spacegrad \cdot \BB &= 0
\end{aligned}
\end{dmath}

what happens when we have different fields and currents on two sides of a boundary?  To answer these questions, we want to use the integral forms of Maxwell's equations, over the geometries illustrated in \cref{fig:loopAndPillbox:loopAndPillboxFig1}.
\index{boundary}

\imageFigure{../../figures/ece1228-emt/loopAndPillboxFig1}{Loop and pillbox configurations.}{fig:loopAndPillbox:loopAndPillboxFig1}{0.3}

To do so, we use Stokes' and the divergence theorems relating the area and volume integrals to the surfaces of those geometries.

These are 

\index{Stokes' Theorem}
\index{Divergence Theorem}
\begin{dmath}\label{eqn:emtLecture2:40}
\begin{aligned}
\iint_S \lr{ \spacegrad \cross \BA } \cdot d\Bs &= \oint_C \BA \cdot d\Bl \\
\iint_V \lr{ \spacegrad \cdot \BA } d\Bs &= \oint_A \BA \cdot d\Bs \\
\end{aligned}
\end{dmath}

\index{Faraday's law}
Application of the Stokes' to Faraday's law we get

\begin{dmath}\label{eqn:emtLecture2:60}
\oint_C \BE \cdot d\Bl = -\PD{t}{} \iint \BB \cdot d\Bs
\end{dmath}

UNITS: \( V/m \times m \)

The quantity 
\begin{dmath}\label{eqn:emtLecture2:80}
\iint \BB \cdot d\Bs,
\end{dmath}

is called the magnetic flux of \( \BB \), and changing of this flux is responsible for the generation of electromotive force.
\index{magnetic flux}

%F2: 
Similarly

\begin{dmath}\label{eqn:emtLecture2:100}
\begin{aligned}
\oint \BH \cdot d\Bl &= \iint \BJ \cdot d\Bs + \PD{t}{} \iint \BD \cdot d\Bs \\
\oint \BD \cdot d\Bs &= \iiint \rho_\txtv dV = Q_\txte \\
\oint \BB \cdot d\Bs &= 0.
\end{aligned}
\end{dmath}

\index{constitutive relations}
\paragraph{Constitutive relations}

With 12 unknowns in \( \BE, \BB, \BD, \BH \) and 8 equations in Maxwell's equations (or 6 if the divergence equations are considered redundant), things don't look too good for solutions.  In simple media, in the frequency domain, relations of the form

\begin{dmath}\label{eqn:emtLecture2:120}
\begin{aligned}
\BD( \Br, \omega ) &= \epsilon \BE( \Br, \omega ) \\
\BB( \Br, \omega ) &= \mu \BH( \Br, \omega ).
\end{aligned}
\end{dmath}

\index{permeability}
\index{macroscopic}
The permeabilities \( \epsilon \) and \( \mu \) are macroscopic beasts, determined either experimentally, or theoretically using an averaging process involving many (millions, or billions, or more) particles.  However, the theoretical determinations that have been attempted do not work well in practise and usually end up considerably different than the measured values.  We are referred to \citep{jackson1975cew} for one attempt to model the statistical microscopic effects non-quantum mechanically to justify the traditional macroscopic form of Maxwell's equations.

These can be position dependent, as in the grating sketched in \cref{fig:gratingL2:gratingL2Fig3}.

\imageFigure{../../figures/ece1228-emt/gratingL2Fig3}{Grating.}{fig:gratingL2:gratingL2Fig3}{0.3}

\index{capacitor}
\index{breakdown voltage}
The permeabilities can also depend on the strength of the fields.  An example, application of an electric field to gallium arsenide or glass can change the behaviour in the material.  We can also have non-linear effects, such as the effect on a capacitor when the voltage is increased.  The response near the breakdown point where the capacitor blows up demonstrates this spectacularly.  We can also have materials for which the permeabilities depend on the direction of the field, or the temperature, or the pressure in the environment, the tensile or compression forces on the material, or many other factors.  There are many other possible complicating factors, for example, the electric response \( \epsilon \) can depend on the magnetic field strength \( \Abs{\BB} \).  We could then write

\begin{dmath}\label{eqn:emtLecture2:140}
\epsilon = \epsilon( \Br, \Abs{\BE}, \BE/\Abs{\BE}, T, P, \Abs{\Beta}, \omega, k ).
\end{dmath}

Further complicating things is that \( \epsilon \) is a complex number (for fields specified in the frequency domain).

\index{anisotropic}
We can also have anisotropic situations where the electric and displacement fields are no longer colinear as sketched in \cref{fig:constituativeRelationsL2:constituativeRelationsL2Fig4}.

\imageFigure{../../figures/ece1228-emt/constituativeRelationsL2Fig4}{Anisotropic field relations.}{fig:constituativeRelationsL2:constituativeRelationsL2Fig4}{0.3}

which indicates that the permittivity \( \epsilon \) in the relation

\begin{dmath}\label{eqn:emtLecture2:160}
\BD = \epsilon \BE,
\end{dmath}

can be modelled as a matrix or as a second rank tensor.  When the off diagonal entries are zero, and the diagonal values are all equal, we have the special case where \( \epsilon \) is reduced to a function.  That function may still be complex-valued, and dependent on many factors, but it least it is scalar valued in this situation.

\index{polarization}
\index{magnetization}
\paragraph{Polarization and magnetization}

If we have a material (such as glass), we can generally assume that the induced field can be related to the vacuum field according to

\begin{dmath}\label{eqn:emtLecture2:180}
\BE = \BP + \epsilon_0 \BE,
\end{dmath}
and
\begin{dmath}\label{eqn:emtLecture2:200}
\BB = \mu_0 \BM + \mu_0 \BH = \mu_0 \lr{ \BM + \BH }.
\end{dmath}

\index{permittivity!vacuum}
Here the vacuum permittivity \( \epsilon_0 \) has the value \( 8.85 \times 10^{-12} \si{F/m} \).  When we are ignoring (fictional) magnetic sources, we have a constant relation between the magnetic fields \( \BB = \mu_0 \BH \).

Assuming \( \BP = \epsilon_0 \chi_\txte \BE \), then

\begin{dmath}\label{eqn:emtLecture2:220}
\BD 
= \epsilon_0 \BE + \epsilon_0 \chi_\txte \BE 
= \epsilon_0 ( 1 + \chi_\txte ) \BE ,
\end{dmath}

so with \( \epsilon_r = 1 + \chi_\txte \), and \( \epsilon = \epsilon_0 \epsilon_r \) we have

\begin{dmath}\label{eqn:emtLecture2:240}
\BD = \epsilon \BE.
\end{dmath}

\index{permittivity!relative}
Note that the relative permittivity \( \epsilon_r \) is dimensionless, whereas the vacuum permittivity has units of \si{F/m}.  We call \(\epsilon\) the (unqualified) permittivity.  The relative permittivity \( \epsilon_r\) is sometimes called the relative permittivity.

\index{index of refraction}
Another useful quantity is the index of refraction

\begin{dmath}\label{eqn:emtLecture2:260}
\eta = \sqrt{ \epsilon_r \mu_r } \approx \sqrt{\epsilon_r}.
\end{dmath}

Similar to the above we can write \( \BM = \chi_\txtm \BH \) then

\begin{dmath}\label{eqn:emtLecture2:280}
\BM = \mu_0 \BH + \mu_0 \BM = \mu_0 \lr{ 1 + \chi_\txtm } \BH
= \mu_0 \mu_r \BH
\end{dmath}

so with \( \mu_r = 1 + \chi_\txtm \), and \( \mu = \mu_0 \mu_r \) we have

\begin{dmath}\label{eqn:emtLecture2:300}
\BB = \mu \BH.
\end{dmath}

\paragraph{Linear and angular momentum in light}

\index{photon!momentum}
\index{photon!angular momentum}
It was pointed out that we have two relations in mechanics that relate momentum and forces

\begin{dmath}\label{eqn:emtLecture2:320}
\begin{aligned}
\BF &= \ddt{\BP} \\
\Btau &= \ddt{\BL},
\end{aligned}
\end{dmath}

where \( \BP = m \Bv \) is the linear momentum, and \( \BL = \Br \cross \Bp \) is the angular momentum.  In quantum electrodynamics, the photon can be described using a relationship between wave-vector and momentum

\begin{dmath}\label{eqn:emtLecture2:340}
\Bp 
= \Hbar \Bk
= \Hbar \frac{ 2\pi}{\lambda}
= \frac{h}{2\pi} \frac{ 2\pi}{\lambda}
= \frac{h}{\lambda},
\end{dmath}

where \( \hbar = 6.522 \times 10^{-16} \si{ev.s} \).

Photons are also governed by

\begin{dmath}\label{eqn:emtLecture2:360}
E = \Hbar \omega = h \nu.
\end{dmath}

\index{De-Broglie relation}
(De-Broglie's relations).

ASIDE: optical fibre at 1550 has the lowest amount of optical attenuation.  

Since photons have linear momentum, we can move things around using light.  With photons having both linear momentum and energy relationships, and there is a relation between between torque and linear momentum, it seems that there must be the possibility of light having angular momentum.

Is it possible to utilize the angular momentum to impose patterns on beams (such as laser beams).  For example, what if a beam could have a geometrical pattern along its line of propagation, being off in some regions, on in others.  This is in fact possible, generating beams that are ``self healing''.

The question was posed ``Is it possible to solve electromagnetic problems utilizing the force concepts?'', using the Lorentz
force equation

\begin{dmath}\label{eqn:emtLecture2:380}
\BF = q \Bv \cross \BB + q\BE.
\end{dmath}

This was not thought to be a productive approach due to the complexity.

FIXME: It appeared that this animated talk (probably not captured well) about momentum in light was linked to the idea of the Helmholtz theorem.  Exactly how was not clear to me.

\index{Helmholtz's theorem}
\paragraph{Helmholtz's theorem}

Suppose that we have a linear material where

\begin{dmath}\label{eqn:emtLecture2:400}
\begin{aligned}
\spacegrad \cross \BE &= -\PD{t}{\BB} \\
\spacegrad \cross \BH &= \BJ + \PD{t}{\BD} \\
\spacegrad \cdot \BE &= \frac{\rho_\txtv}{\epsilon_0} \\
\spacegrad \cdot \BH &= 0
\end{aligned}
\end{dmath}

We have relations between the divergence and curl of \( \BE \) given the sources.  Is that sufficient to determine \( \BE \) itself?  The answer is yes, which is due to the Helmholtz theorem.

Extra homework question (bonus) : can knowledge of the tangential components of the fields also be used to uniquely determine \( \BE \)?

Also homework: read notes about irrotational fields and solenoidal fields.

%\EndArticle

   \section{Problems}
      %
% Copyright � 2016 Peeter Joot.  All Rights Reserved.
% Licenced as described in the file LICENSE under the root directory of this GIT repository.
%
%{
\newcommand{\authorname}{Peeter Joot}
\newcommand{\email}{peeterjoot@protonmail.com}
\newcommand{\basename}{FIXMEbasenameUndefined}
\newcommand{\dirname}{notes/FIXMEdirnameUndefined/}

\renewcommand{\basename}{griffithsEM2_7}
\renewcommand{\dirname}{notes/phy1520/}
%\newcommand{\dateintitle}{}
%\newcommand{\keywords}{}

\newcommand{\authorname}{Peeter Joot}
\newcommand{\onlineurl}{http://sites.google.com/site/peeterjoot2/math2013/\basename.pdf}
\newcommand{\sourcepath}{\dirname\basename.tex}
\newcommand{\generatetitle}[1]{\chapter{#1}}

\newcommand{\vcsinfo}{%
\section*{}
\noindent{\color{DarkOliveGreen}{\rule{\linewidth}{0.1mm}}}
\paragraph{Document version}
%\paragraph{\color{Maroon}{Document version}}
{
\small
\begin{itemize}
\item Available online at:\\ 
\href{\onlineurl}{\onlineurl}
\item Git Repository: \input{./.revinfo/gitRepo.tex}
\item Source: \sourcepath
\item last commit: \input{./.revinfo/gitCommitString.tex}
\item commit date: \input{./.revinfo/gitCommitDate.tex}
\end{itemize}
}
}

%\PassOptionsToPackage{dvipsnames,svgnames}{xcolor}
\PassOptionsToPackage{square,numbers}{natbib}
\documentclass{scrreprt}

\usepackage[left=2cm,right=2cm]{geometry}
\usepackage[svgnames]{xcolor}
\usepackage{peeters_layout}

\usepackage{natbib}

\usepackage[
colorlinks=true,
bookmarks=false,
pdfauthor={\authorname, \email},
backref 
]{hyperref}

% http://tex.stackexchange.com/questions/75773/how-to-reference-problems-by-the-text-label-in-an-exercise-envioronment
\usepackage[english]{cleveref}
\crefname{Exercise}{exercise}{exercises}
\Crefname{Exercise}{Exercise}{Exercises}

\RequirePackage{titlesec}
\RequirePackage{ifthen}

% http://stackoverflow.com/questions/4932910/date-in-the-tabular-environment
\makeatletter
\let\insertdate\@date
\makeatother

\titleformat{\chapter}[display]
{\bfseries\Large}
{\color{DarkSlateGrey}\filleft \authorname
\ifthenelse{\isundefined{\studentnumber}}{}{\\ \studentnumber}
\ifthenelse{\isundefined{\email}}{}{\\ \email}
\ifthenelse{\isundefined{\dateintitle}}{}{\\ \insertdate}
%\ifthenelse{\isundefined{\coursename}}{}{\\ \coursename} % put in title instead.
}
{4ex}
{\color{DarkOliveGreen}{\titlerule}\color{Maroon}
\vspace{2ex}%
\filright}
[\vspace{2ex}%
\color{DarkOliveGreen}\titlerule
]

\newcommand{\beginArtWithToc}[0]{\begin{document}\tableofcontents}
\newcommand{\beginArtNoToc}[0]{\begin{document}}
\newcommand{\EndNoBibArticle}[0]{\end{document}}
\newcommand{\EndArticle}[0]{\bibliography{Bibliography}\bibliographystyle{plainnat}\end{document}}

% 
%\newcommand{\citep}[1]{\cite{#1}}

\colorSectionsForArticle



\usepackage{peeters_layout_exercise}
\usepackage{peeters_braket}
\usepackage{peeters_figures}
\usepackage{siunitx}

\beginArtNoToc

\generatetitle{Electric field due to spherical shell}
%\chapter{electric field due to spherical shell}
%\label{chap:griffithsEM2_7}
% \citep{sakurai2014modern} pr X.Y
% \citep{pozar2009microwave}
% \citep{qftLectureNotes}

\makeoproblem{Electric field due to spherical shell}{problem:griffithsEM2_7:1}{\citep{griffiths1999introduction} pr. 2.7}{
Calculate the field due to a spherical shell.  The field is

\begin{dmath}\label{eqn:griffithsEM2_7:20}
\BE = \frac{\sigma}{4 \pi \epsilon_0} \int \frac{(\Br - \Br')}{\Abs{\Br - \Br'}^3} da',
\end{dmath}

where \( \Br' \) is the position to the area element on the shell.  For the test position, let \( \Br = z \Be_3 \).  
} % problem

\makeanswer{problem:griffithsEM2_7:1}{
We need to parameterize the area integral.  A complex-number like geometric algebra representation works nicely.

\begin{dmath}\label{eqn:griffithsEM2_7:40}
\Br' 
= R \lr{ \sin\theta \cos\phi, \sin\theta \sin\phi, \cos\theta }
= R \lr{ \Be_1 \sin\theta \lr{ \cos\phi + \Be_1 \Be_2 \sin\phi } + \Be_3 \cos\theta }
= R \lr{ \Be_1 \sin\theta e^{i\phi} + \Be_3 \cos\theta }.
\end{dmath}

Here \( i = \Be_1 \Be_2 \) has been used to represent to horizontal rotation plane.

The difference in position between the test vector and area-element is

\begin{dmath}\label{eqn:griffithsEM2_7:60}
\Br - \Br' 
= \Be_3 \lr{ z - R \cos\theta } - R \Be_1 \sin\theta e^{i \phi},
\end{dmath}

with an absolute squared length of

\begin{dmath}\label{eqn:griffithsEM2_7:80}
\Abs{\Br - \Br' }^2
= \lr{ z - R \cos\theta }^2 + R^2 \sin^2\theta 
= z^2 + R^2 - 2 z R \cos\theta.
\end{dmath}

As a side note, this is a kind of fun way to prove the old ``cosine-law'' identity.  With that done, the field integral can now be expressed explicitly

\begin{dmath}\label{eqn:griffithsEM2_7:100}
\BE 
= \frac{\sigma}{4 \pi \epsilon_0} \int_{\phi = 0}^{2\pi} \int_{\theta = 0}^\pi R^2 \sin\theta d\theta d\phi
\frac{\Be_3 \lr{ z - R \cos\theta } - R \Be_1 \sin\theta e^{i \phi}}
{
\lr{z^2 + R^2 - 2 z R \cos\theta}^{3/2}
}
= \frac{2 \pi R^2 \sigma \Be_3}{4 \pi \epsilon_0} \int_{\theta = 0}^\pi \sin\theta d\theta 
\frac{z - R \cos\theta}
{
\lr{z^2 + R^2 - 2 z R \cos\theta}^{3/2}
}
= \frac{2 \pi R^2 \sigma \Be_3}{4 \pi \epsilon_0} \int_{\theta = 0}^\pi \sin\theta d\theta 
\frac{ R( z/R - \cos\theta) }
{
(R^2)^{3/2} \lr{ (z/R)^2 + 1 - 2 (z/R) \cos\theta}^{3/2}
}
= \frac{\sigma \Be_3}{2 \epsilon_0} \int_{u = -1}^{1} du
\frac{ z/R - u}
{
\lr{1 + (z/R)^2 - 2 (z/R) u}^{3/2}
}.
\end{dmath}

Observe that all the azimuthal contributions get killed.  We expect that due to the symmetry of the problem.  We are left with an integral that submits to Mathematica, but doesn't look fun to attempt manually.  Specifically

\begin{dmath}\label{eqn:griffithsEM2_7:120}
\int_{-1}^1 \frac{a-u}{\lr{1 + a^2 - 2 a u}^{3/2}} du
=
\left\{
\begin{array}{l l}
\frac{2}{a^2} & \quad \mbox{if \( a > 1 \) } \\
0 & \quad \mbox{if \( a < 1 \) } 
\end{array}
\right.,
\end{dmath}

so

%\begin{dmath}\label{eqn:griffithsEM2_7:140}
\boxedEquation{eqn:griffithsEM2_7:160}{
\BE 
= 
\left\{
\begin{array}{l l}
\frac{\sigma (R/z)^2 \Be_3}{\epsilon_0} 
& \quad \mbox{if \( z > R \) } \\
0 & \quad \mbox{if \( z < R \) } 
\end{array}
\right.
}
%\end{dmath}

In the problem, it is pointed out to be careful of the sign when evaluating \( \sqrt{ R^2 + z^2 - 2 R z } \), however, I don't see where that is even useful?
} % answer

%}
\EndArticle
%\EndNoBibArticle

%%%      %
% Copyright � 2016 Peeter Joot.  All Rights Reserved.
% Licenced as described in the file LICENSE under the root directory of this GIT repository.
%
\makeproblem{Solenoidal fields}{emt:problemSet1:1}{ 
For the electric fields graphically shown below indicate whether the fields are solenoidal (divergence free) or not. In the case of non-solenoidal fields indicate the charge generating the field is positive or negative. Justify your answer.

%\cref{fig:emtLect2:emtLect2Fig1}.
\imageFigure{../../figures/ece1228-emt/emtLect2Fig1}{Field lines}{fig:emtLect2:emtLect2Fig1}{0.3}
} % makeproblem

\makeanswer{emt:problemSet1:1}{ 

\begin{enumerate}[(a)]
\item
The first set of field lines has the appearance of non-solenoidal.  To demonstrate this a graphical-numeric approximation of \( \int \spacegrad \cdot \BE \propto \sum_i \ncap \cdot \BE_i \) is sketched in \cref{fig:nonSolinoidal:nonSolinoidalFig2}.

\imageFigure{../../figures/ece1228-emt/nonSolinoidalFig2}{Graphical divergence integration.}{fig:nonSolinoidal:nonSolinoidalFig2}{0.15}

For each field line \( \BE_i \), passing through this square integration volume, the length of the projection onto the \( x \) axis is shorter on the right side of the box than the left.  Suppose the left hand projections of \( \BE \) onto \( \xcap \) are \( 0.9 \), and \( 0.8 \) vs. \( 0.7\), and \(0.6\) on the right for the bottom and top red field lines respectively.  The flux of those field lines is proportional to

\begin{dmath}\label{eqn:emtProblemSet1Problem1:20}
\sum_i \ncap \cdot \BE \approx (0.7 - 0.9) + (0.6 - 0.8) = -0.4,
\end{dmath}

so this field appears to be non-solenoidal.  As for the charges generating the field, this field has the look of a small portion of a dipole field as sketched in \cref{fig:dipole:dipoleFig1}, with the lines in the supplied figure flowing out of a positive charge to a negative.

\imageFigure{../../figures/ece1228-emt/dipoleFig1}{Crude sketch of dipole field.}{fig:dipole:dipoleFig1}{0.15}

\item
This next figure has the appearance of the electric field lines coming out of a single positive charge

\begin{dmath}\label{eqn:emtProblemSet1Problem1:40}
\BE = \frac{q}{4 \pi \epsilon_0} \frac{\rcap}{r^2}.
\end{dmath}

Such a field is divergence free everywhere but the origin.  For \( \Br \ne 0 \)

\begin{dmath}\label{eqn:emtProblemSet1Problem1:60}
\spacegrad \cdot \BE 
= 
\frac{q}{4 \pi \epsilon_0} \spacegrad \cdot \frac{\Br}{r^3}
= 
\frac{q}{4 \pi \epsilon_0} \lr{ \frac{\spacegrad \cdot \Br }{r^3} + \lr{ \spacegrad \inv{r^3} } \cdot \Br }
= 
\frac{q}{4 \pi \epsilon_0} \lr{ \frac{ 3 }{r^3} + \lr{ -\frac{3}{2} 2 \frac{\Br}{r^5} } \cdot \Br }
= 
0.
\end{dmath}

Because of the singularity at the origin, this is still a solenoidal field, as shown by the divergence integral

\begin{dmath}\label{eqn:emtProblemSet1Problem1:80}
\int_V \spacegrad \cdot \BE dV 
=
\oint_{\partial V} \ncap \cdot \BE dA 
=
\frac{q}{4 \pi \epsilon_0} \iint \rcap \cdot \ncap r^2 \sin\theta d\theta d\phi
=
\frac{q}{4 \pi \epsilon_0} \iint \ncap \cdot \frac{\rcap}{r^2} r^2 \sin\theta d\theta d\phi
=
\frac{q}{4 \pi \epsilon_0} 4 \pi
=
\frac{q}{\epsilon_0}.
\end{dmath}

\item
This last field is solenoidal, since the field lines are all of equal magnitude and direction.  Suppose that field was

\begin{dmath}\label{eqn:emtProblemSet1Problem1:100}
\BE = \xcap E,
\end{dmath}

where \( E \) is constant.  The divergence is then

\begin{dmath}\label{eqn:emtProblemSet1Problem1:120}
\spacegrad \cdot \BE = \PD{x}{E} = 0.
\end{dmath}

\end{enumerate}
} % answer

%%%      %
% Copyright � 2016 Peeter Joot.  All Rights Reserved.
% Licenced as described in the file LICENSE under the root directory of this GIT repository.
%
\makeproblem{description}{emt:problemSet1:2}{ 
\makesubproblem{}{emt:problemSet1:2a}
} % makeproblem

\makeanswer{emt:problemSet1:2}{ 
\makeSubAnswer{}{emt:problemSet1:2a}

TODO.
}

%%%      %
% Copyright � 2016 Peeter Joot.  All Rights Reserved.
% Licenced as described in the file LICENSE under the root directory of this GIT repository.
%
\makeproblem{Solenoidal and irrotational fields.}{emt:problemSet1:3}{ 
In terms of \(\BE\) or \(\BH\) give an example for each of the following conditions:

\makesubproblem{}{emt:problemSet1:3a}
Field is solenoidal and irrotational.
\makesubproblem{}{emt:problemSet1:3b}
Field is solenoidal and rotational.
\makesubproblem{}{emt:problemSet1:3c}
Field is irrotational and non-solenoidal.
\makesubproblem{}{emt:problemSet1:3d}
Field is non-solenoidal and rotational.
} % makeproblem

\makeanswer{emt:problemSet1:3}{ 
\makeSubAnswer{}{emt:problemSet1:3a}

TODO.
\makeSubAnswer{}{emt:problemSet1:3b}

TODO.
\makeSubAnswer{}{emt:problemSet1:3c}

TODO.
\makeSubAnswer{}{emt:problemSet1:3d}

TODO.
}

%%%      %
% Copyright � 2016 Peeter Joot.  All Rights Reserved.
% Licenced as described in the file LICENSE under the root directory of this GIT repository.
%
\makeproblem{Conducting sheet with hole.}{emt:problemSet1:4}{ 
%Figure 
\Cref{fig:emtLect2:emtLect2Fig4}.
shows a flat, positive, non-conducting sheet of charge with uniform charge density \( \sigma \) [\si{C/m^2}]. A small circular hole of radius \(R \) is cut in the middle of the surface as shown.
 
\imageFigure{../../figures/ece1228-emt/emtLect2Fig4}{Conducting sheet with a hole.}{fig:emtLect2:emtLect2Fig4}{0.3}

Calculate the electric field intensity \(\BE\) at point \(P\), a distance \(z\) from the center of the hole along its axis.

Hint 1: Ignore the field fringe effects around all edges.
Hint 2: Calculate the field due to a disk of radius \(R\) and use superposition.

} % makeproblem

\makeanswer{emt:problemSet1:4}{ 

TODO.
}

%%%      %
% Copyright � 2016 Peeter Joot.  All Rights Reserved.
% Licenced as described in the file LICENSE under the root directory of this GIT repository.
%
\makeproblem{Helmholtz theorem}{emt:problemSet1:5}{ 
Prove the first Helmholtz's theorem, i.e. if vector \(\BM_1\) is defined by its divergence

\begin{dmath}\label{eqn:emtProblemSet1Problem5:20}
\spacegrad \cdot \BM_1 = s
\end{dmath}

and its curl
\begin{dmath}\label{eqn:emtProblemSet1Problem5:40}
\spacegrad \cross \BM_1 = \BC 
\end{dmath}

within a region and its normal component \( \BM_{1\txtn} \) over the boundary, then \( \BM_1 \) is 
uniquely specified.

Note: Assume there is a vector \( \BM_2 \) with its divergence and curl equal to \( s \) and \( \BC \)
respectively, then show that \( \BM_1 = \BM_2 \) .

%\makesubproblem{}{emt:problemSet1:5a}
} % makeproblem

\makeanswer{emt:problemSet1:5}{ 
%\makeSubAnswer{}{emt:problemSet1:5a}

TODO.
}

%%%      %
% Copyright � 2016 Peeter Joot.  All Rights Reserved.
% Licenced as described in the file LICENSE under the root directory of this GIT repository.
%
\makeproblem{description}{emt:problemSet1:6}{ 
\makesubproblem{}{emt:problemSet1:6a}
} % makeproblem

\makeanswer{emt:problemSet1:6}{ 
\makeSubAnswer{}{emt:problemSet1:6a}

TODO.
}

%%%      %\input{emtproblemSet1Appendix.tex}
%%%      %
% Copyright � 2016 Peeter Joot.  All Rights Reserved.
% Licenced as described in the file LICENSE under the root directory of this GIT repository.
%
\makeproblem{Infinite line charge.}{emt:problemSet2:2}{ 

An infinitely long straight line charge has a constant charge density \( \rho_\txtl \) [\si{C/m}].

\makesubproblem{}{emt:problemSet2:2a}
Using the integral formulation for \( \BE \)
discussed in the class calculate the electric
field at an arbitrary point \( \BA(\rho, \phi, z) \).
\makesubproblem{}{emt:problemSet2:2b}

Using the Gauss law calculate the same as \partref{emt:problemSet2:2a}.

\makesubproblem{}{emt:problemSet2:2c}

Now suppose that our uniformly charged (\( \rho_\txtl \) constant) 
has a finite extension from \( z = a \) to \( z = b \), as sketched in \cref{fig:problemset2:problemset2Fig2}.
\imageFigure{../../figures/ece1228-emt/problemset2Fig2}{Line charge.}{fig:problemset2:problemset2Fig2}{0.3}
Find the electric field at the
arbitrary point \( \BA \).

Note: Express your results in cylindrical coordinate
system.
} % makeproblem

\makeanswer{emt:problemSet2:2}{ 
\makeSubAnswer{}{emt:problemSet2:2a}

Since the line charge is of infinite length and rotationally symmetric with respect to \( \phi \), the observation point can be positioned anywhere convienient, such as

\begin{dmath}\label{eqn:emtProblemSet2Problem2:20}
\BA = \rho \rhocap.
\end{dmath}

Let the point on the wire be
\begin{dmath}\label{eqn:emtProblemSet2Problem2:40}
\Br' = z \zcap.
\end{dmath}

The absolute distance between the observation point and the element of charge is
\begin{dmath}\label{eqn:emtProblemSet2Problem2:60}
\Abs{\BA - \Br'} = \sqrt{ (\rho \rhocap)^2 + (z \zcap)^2} = \sqrt{ \rho^2 + z^2 }.
\end{dmath}

The electric field can now be expressed in integral form
\begin{dmath}\label{eqn:emtProblemSet2Problem2:80}
\BE(\BA) 
=
\inv{ 4 \pi \epsilon_0 } \int_{-\infty}^\infty dz \rho_\txtl \frac{ \rho \rhocap - z \zcap }{ \lr{\rho^2 + z^2}^{3/2} }
=
\frac{\sigma_\txtl \BA }{ 4 \pi \epsilon_0 } \int_{-\infty}^\infty dz \inv{ \lr{\rho^2 + z^2}^{3/2} }
=
\frac{\sigma_\txtl \BA }{ 4 \pi \epsilon_0 \rho^3 } \int_{-\infty}^\infty dz \inv{ \lr{1 + (z/\rho)^2}^{3/2} }
=
\frac{\sigma_\txtl \BA }{ 4 \pi \epsilon_0 \rho^2 } \int_{-\infty}^\infty du \inv{ \lr{1 + u^2}^{3/2} }.
\end{dmath}

The \( \zcap \) term was killed since the integrand was an odd function in \( z \).  Note that

\begin{dmath}\label{eqn:emtProblemSet2Problem2:100}
   \int \frac{du}{ \lr{1 + u^2}^{3/2} }
= \frac{u}{\sqrt{1 + u^2}},
\end{dmath}

which has a PV limit over \([-\infty,\infty]\) of 2.  This gives

%\begin{dmath}\label{eqn:emtProblemSet2Problem2:120}
\boxedEquation{eqn:emtProblemSet2Problem2:120}{
\BE(\rho) 
=
\inv{ 4 \pi \epsilon_0 } 
\frac{2 \sigma_\txtl }{\rho} \rhocap.
}
%\end{dmath}

\makeSubAnswer{}{emt:problemSet2:2b}

Using Gauss's law integrating over a cylindrical segment surrounding the wire, we have

\begin{dmath}\label{eqn:emtProblemSet2Problem2:n}
\Delta z \frac{\rho_\txtl}{\epsilon_0} 
= \int \spacegrad \cdot \BE dV
= \oint \ncap \cdot \BE dA
= E_\rho(\rho) (2 \pi \rho \Delta z),
\end{dmath}

This assumes the end surfaces at infinity contribute nothing to the surface integral, and gives after rearrangement

\begin{dmath}\label{eqn:emtProblemSet2Problem2:n}
E_\rho(\rho) 
= \inv{2 \pi \rho } \frac{\rho_\txtl}{\epsilon_0}.
\end{dmath}

This matches \cref{eqn:emtProblemSet2Problem2:120} as expected.

\makeSubAnswer{}{emt:problemSet2:2c}

TODO.
}

%%%      %
% Copyright � 2016 Peeter Joot.  All Rights Reserved.
% Licenced as described in the file LICENSE under the root directory of this GIT repository.
%
\makeproblem{Gradient in cylindrical coordinates.}{emt:problemSet2:3}{ 

If gradient of a scalar function \( \psi \) rectangular coordinate system is given by

\begin{dmath}\label{eqn:emtproblemSet2Problem3:20}
\spacegrad \psi = 
\Be_1 \PD{x}{\psi}
+\Be_2 \PD{y}{\psi}
+\Be_3 \PD{z}{\psi},
\end{dmath}

using coordinate transformation and chain rule show
that the gradient of \( \psi \) in cylindrical coordinates is given by

\begin{dmath}\label{eqn:emtproblemSet2Problem3:40}
\spacegrad \psi = 
\rhocap \PD{\rho}{\psi}
+\phicap \PD{\phi}{\psi}
+\zcap \PD{z}{\psi}.
\end{dmath}
} % makeproblem

\makeanswer{emt:problemSet2:3}{ 

TODO.
}

%%%      %
% Copyright � 2016 Peeter Joot.  All Rights Reserved.
% Licenced as described in the file LICENSE under the root directory of this GIT repository.
%
\makeproblem{Point charge.}{emt:problemSet2:5}{ 
\makesubproblem{}{emt:problemSet2:5a}
Consider a point charge \( q \). Using Maxwell equations, derive an expression for the
electric field \( \BE \) 
generated by \( q \) at the distance \( \Br \) from it.  Clearly express your
assumptions and justify them.
\makesubproblem{}{emt:problemSet2:5b}
Derive an expression for the force experience by the charge \( q' \) located at distance \( \Br \)
from the charge \( q \). (This is called Coulomb force)
\makesubproblem{}{emt:problemSet2:5c}
Derive an expression for the electrostatic potential \( V \) at the distance \( \Br \) from the
charge \( q \) with respect to the electrostatic potential at infinity. For convenience, set the
value of electrostatic potential at infinity to zero.
} % makeproblem

\makeanswer{emt:problemSet2:5}{ 
\makeSubAnswer{}{emt:problemSet2:5a}

TODO.
\makeSubAnswer{}{emt:problemSet2:5b}

TODO.
\makeSubAnswer{}{emt:problemSet2:5c}

TODO.
}


   %
% Copyright � 2016 Peeter Joot.  All Rights Reserved.
% Licenced as described in the file LICENSE under the root directory of this GIT repository.
%
\newcommand{\authorname}{Peeter Joot}
\newcommand{\email}{peeterjoot@protonmail.com}
\newcommand{\basename}{FIXMEbasenameUndefined}
\newcommand{\dirname}{notes/FIXMEdirnameUndefined/}

\renewcommand{\basename}{emt3}
\renewcommand{\dirname}{notes/ece1228/}
\newcommand{\keywords}{ECE1228H}
\newcommand{\authorname}{Peeter Joot}
\newcommand{\onlineurl}{http://sites.google.com/site/peeterjoot2/math2013/\basename.pdf}
\newcommand{\sourcepath}{\dirname\basename.tex}
\newcommand{\generatetitle}[1]{\chapter{#1}}

\newcommand{\vcsinfo}{%
\section*{}
\noindent{\color{DarkOliveGreen}{\rule{\linewidth}{0.1mm}}}
\paragraph{Document version}
%\paragraph{\color{Maroon}{Document version}}
{
\small
\begin{itemize}
\item Available online at:\\ 
\href{\onlineurl}{\onlineurl}
\item Git Repository: \input{./.revinfo/gitRepo.tex}
\item Source: \sourcepath
\item last commit: \input{./.revinfo/gitCommitString.tex}
\item commit date: \input{./.revinfo/gitCommitDate.tex}
\end{itemize}
}
}

%\PassOptionsToPackage{dvipsnames,svgnames}{xcolor}
\PassOptionsToPackage{square,numbers}{natbib}
\documentclass{scrreprt}

\usepackage[left=2cm,right=2cm]{geometry}
\usepackage[svgnames]{xcolor}
\usepackage{peeters_layout}

\usepackage{natbib}

\usepackage[
colorlinks=true,
bookmarks=false,
pdfauthor={\authorname, \email},
backref 
]{hyperref}

% http://tex.stackexchange.com/questions/75773/how-to-reference-problems-by-the-text-label-in-an-exercise-envioronment
\usepackage[english]{cleveref}
\crefname{Exercise}{exercise}{exercises}
\Crefname{Exercise}{Exercise}{Exercises}

\RequirePackage{titlesec}
\RequirePackage{ifthen}

% http://stackoverflow.com/questions/4932910/date-in-the-tabular-environment
\makeatletter
\let\insertdate\@date
\makeatother

\titleformat{\chapter}[display]
{\bfseries\Large}
{\color{DarkSlateGrey}\filleft \authorname
\ifthenelse{\isundefined{\studentnumber}}{}{\\ \studentnumber}
\ifthenelse{\isundefined{\email}}{}{\\ \email}
\ifthenelse{\isundefined{\dateintitle}}{}{\\ \insertdate}
%\ifthenelse{\isundefined{\coursename}}{}{\\ \coursename} % put in title instead.
}
{4ex}
{\color{DarkOliveGreen}{\titlerule}\color{Maroon}
\vspace{2ex}%
\filright}
[\vspace{2ex}%
\color{DarkOliveGreen}\titlerule
]

\newcommand{\beginArtWithToc}[0]{\begin{document}\tableofcontents}
\newcommand{\beginArtNoToc}[0]{\begin{document}}
\newcommand{\EndNoBibArticle}[0]{\end{document}}
\newcommand{\EndArticle}[0]{\bibliography{Bibliography}\bibliographystyle{plainnat}\end{document}}

% 
%\newcommand{\citep}[1]{\cite{#1}}

\colorSectionsForArticle



%\usepackage{ece1228}
\usepackage{peeters_braket}
%\usepackage{peeters_layout_exercise}
\usepackage{peeters_figures}
\usepackage{mathtools}
\usepackage{siunitx}

\beginArtNoToc
\generatetitle{ECE1228H Electromagnetic Theory.  Lecture 3: Electrostatics.  Taught by Prof.\ M. Mojahedi}
%\chapter{Electrostatics}
\label{chap:emt3}

\paragraph{Disclaimer}

Peeter's lecture notes from class.  These may be incoherent and rough.

These are notes for the UofT course ECE1228H, Electromagnetic Theory, taught by Prof. M. Mojahedi, covering \textchapref{{1}} \citep{balanis1989advanced} content.

\paragraph{Polarization and Magnetization}

The importance of the polarization and magnetization given by

\begin{dmath}\label{eqn:emtLecture3:20}
\begin{aligned}
\BD &= \epsilon_0 \BE + \BP \\
\BP &= \epsilon_0 \chi_\txte \BE,
\end{aligned}
\end{dmath}

where 
\begin{dmath}\label{eqn:emtLecture3:40}
\begin{aligned}
\BD &= \epsilon \BE \\
\epsilon &= \epsilon_0 \epsilon_r \\
\epsilon_r = 1 + \chi_\txte.
\end{aligned}
\end{dmath}

\paragraph{Point charge.}

\begin{dmath}\label{eqn:emtLecture3:60}
\BE 
= \frac{q}{4 \pi \epsilon_0} \frac{\rcap}{\Br^2}
= \frac{q}{4 \pi \epsilon_0} \frac{\Br}{\Abs{\Br}^3}
= \frac{q}{4 \pi \epsilon_0} \frac{\Br}{r^3}.
\end{dmath}

In more complex media the \( \epsilon_0 \) here can be replaced by \( \epsilon \).
Here the vector \( \Br \) points from the charge to the observation point.

Note that the class notes use \( \cap{a}_R \) instead of \( \rcap \). 

When the charge isn't located at the origin, we must modify this accordingly

\begin{dmath}\label{eqn:emtLecture3:80}
\BE 
= \frac{q}{4 \pi \epsilon_0} \frac{\BR}{\Abs{\BR}^3}
= \frac{q}{4 \pi \epsilon_0} \frac{\BR}{R^3},
\end{dmath}

where \( \BR = \Br - \Br' \) still points from the location of the charge to the point of observation, as sketched in

F1.

This can be further generalized to collections of point charges by superposition

\begin{dmath}\label{eqn:emtLecture3:100}
\BE 
= \frac{1}{4 \pi \epsilon_0} \sum_i q_i \frac{\Br - \Br_i'}{\Abs{\Br - \Br_i'}^3}.
\end{dmath}

Observe that a potential that satisfies \( \BE = - \spacegrad V \) can be defined as

\begin{dmath}\label{eqn:emtLecture3:120}
V  
= \frac{1}{4 \pi \epsilon_0} \sum_i \frac{q_i}{\Abs{\Br - \Br_i'}}.
\end{dmath}

When we are considering real world scenerios (like touching your hair, and then the table), how do we deal with the billions of charges involved.  This can be done by considering the charges so small that they can be approximated as a continuous distribution of charges.

This can be done by introducing the concept of a continuous charge distribution \( \rho_\txtv(\Br') \).
The charge that is in a small differential volume element \( dV' \) is \( \rho(\Br') dV' \), and 
the superposition has the form

\begin{dmath}\label{eqn:emtLecture3:140}
\BE 
= \frac{1}{4 \pi \epsilon_0} \iiint dV' \rho_\txtv(\Br') \frac{\Br - \Br'}{\Abs{\Br - \Br'}^3},
\end{dmath}

with potential

\begin{dmath}\label{eqn:emtLecture3:160}
V  
= \frac{1}{4 \pi \epsilon_0} \iiint dV' \frac{\rho(\Br')}{\Abs{\Br - \Br'}}.
\end{dmath}

The surface charge density analogue of this is

\begin{dmath}\label{eqn:emtLecture3:180}
\BE 
= \frac{1}{4 \pi \epsilon_0} \iint dA' \rho_\txts(\Br') \frac{\Br - \Br'}{\Abs{\Br - \Br'}^3},
\end{dmath}

with potential

\begin{dmath}\label{eqn:emtLecture3:200}
V  
= \frac{1}{4 \pi \epsilon_0} \iint dA' \frac{\rho_\txts(\Br')}{\Abs{\Br - \Br'}}.
\end{dmath}

The line charge density analogue of this is

\begin{dmath}\label{eqn:emtLecture3:220}
\BE 
= \frac{1}{4 \pi \epsilon_0} \int dl' \rho_\txtl(\Br') \frac{\Br - \Br'}{\Abs{\Br - \Br'}^3},
\end{dmath}

with potential

\begin{dmath}\label{eqn:emtLecture3:240}
V  
= \frac{1}{4 \pi \epsilon_0} \int dl' \frac{\rho_\txtl(\Br')}{\Abs{\Br - \Br'}}.
\end{dmath}

The difficulty with any of these approaches is the charge density is hardly ever known.  When the charge density is known, this sorts of integrals may not be analytically calculable, but they do yield to numeric calculation.

We may often prefer the potential calculations of the field calculations because they are much easier, having just one component to deal with.

\paragraph{Homework question:}

Starting from Maxwell's equations, in particular \( \oint \BE \cdot d\Bs = Q/\epsilon_0 \) that \( \BE = E_r \rcap \), and explicitly demonstrate why there is no \( \theta \) or \( \phi \) dependencies in this field.  Also calculate the potential \( V \propto 1/r \) associated with an electric field, and show that \( \BE = -\spacegrad V \), and show that this implies that \( -\int_b^a \BE \cdot d\Bl = V_a - V_b \).

\EndArticle
%\EndNoBibArticle

   \section{Problems}
%%%      %
% Copyright � 2016 Peeter Joot.  All Rights Reserved.
% Licenced as described in the file LICENSE under the root directory of this GIT repository.
%
\makeproblem{Electric Dipole.}{emt:problemSet2:1}{ 

An electric dipole is shown in \cref{fig:problemset2:problemset2Fig1}.

\imageFigure{../../figures/ece1228-emt/problemset2Fig1}{Electric dipole configuration.}{fig:problemset2:problemset2Fig1}{0.3}

\makesubproblem{}{emt:problemSet2:1b}
Find the Potential \( V \) at an arbitrary point \( \BA \).

\makesubproblem{}{emt:problemSet2:1a}
Calculate the field \( \BE \) from the above potential.

(show that it is the same result we obtained in the class).
} % makeproblem

\makeanswer{emt:problemSet2:1}{ 

\makeSubAnswer{}{emt:problemSet2:1a}

Following \cref{fig:dipoleSignConventionL3:dipoleSignConventionL3Fig3}, the vector from the origin to the observation point is

\imageFigure{../../figures/ece1228-emt/dipoleSignConventionL3Fig3}{Dipole sign convention.}{fig:dipoleSignConventionL3:dipoleSignConventionL3Fig3}{0.3}

\begin{equation}\label{eqn:emtProblemSet2Problem1:20}
\Br = \BR_1 + \Bd/2
= \BR_2 - \Bd/2,
\end{equation}

or

\begin{equation}\label{eqn:emtProblemSet2Problem1:40}
\begin{aligned}
\BR_1 &= \Br - \Bd/2 \equiv \BR_{+} \\
\BR_2 &= \Br + \Bd/2 \equiv \BR_{-}.
\end{aligned}
\end{equation}

The potential for this superposition is
\begin{dmath}\label{eqn:emtProblemSet2Problem1:60}
V
= 
\inv{4 \pi \epsilon_0} \lr{ 
\frac{q}{\Abs{\BR_{+}}} -
\frac{q}{\Abs{\BR_{-}}} 
}
= 
\frac{q}{4 \pi \epsilon_0} \lr{ 
\frac{1}{\Abs{\BR_{+}}} -
\frac{1}{\Abs{\BR_{-}}} 
}.
\end{dmath}

The magnitudes can be expanded in Taylor series

\begin{dmath}\label{eqn:emtProblemSet2Problem1:80}
\Abs{\BR_{\pm}}^{-1}
=
\lr{ 
\lr{ \Br \mp \Bd/2 } \cdot \lr{ \Br \mp \Bd/2 } 
}^{-1/2}
=
\lr{ 
\lr{ \Br^2 + (\Bd/2)^2 \mp 2 \Br \cdot \Bd/2 }
}^{-1/2}
=
\lr{ 
\lr{ \Br^2 + (\Bd/2)^2 \mp \Br \cdot \Bd }
}^{-1/2}
=
(\Br^2)^{-1/2}
\lr{ 
\lr{ 1 + \lr{\frac{\Bd}{2 r}}^2 \mp \rcap \cdot \frac{\Bd}{r} }
}^{-1/2}
=
r^{-1}
\lr{ 
1 
-\frac{1}{2}
\lr{ \lr{\frac{\Bd}{2 r}}^2 \mp \rcap \cdot \frac{\Bd}{r} }
+\lr{\frac{-1}{2}}
\lr{\frac{-3}{2}} \inv{2!}
\lr{ \lr{\frac{\Bd}{2 r}}^2 \mp \rcap \cdot \frac{\Bd}{r} }^2
+ \cdots
}.
\end{dmath}

Here \( r = \Abs{\Br} \), and the Taylor series was taken in the \( \Bd \ll r \) limit.  The sums and differences of these magnitudes, are to first order

\begin{dmath}\label{eqn:emtProblemSet2Problem1:100}
\inv{\Abs{\BR_{+}}}
-
\inv{\Abs{\BR_{-}}}
\approx
2 \frac{1}{r} \lr{\frac{-1}{2}} \lr{-\rcap \cdot \frac{\Bd}{r}}
=
\frac{1}{r^2} \rcap \cdot \Bd,
\end{dmath}

for
%\begin{dmath}\label{eqn:emtProblemSet2Problem1:120}
\boxedEquation{eqn:emtProblemSet2Problem1:120}{
   V = \frac{\rcap \cdot \Bd}{4 \pi \epsilon_0 r^2 }.
}
%\end{dmath}

\makeSubAnswer{}{emt:problemSet2:1b}

The electric field follows from \( \BE = -\spacegrad V \).  First note that

\begin{dmath}\label{eqn:emtProblemSet2Problem1:140}
\spacegrad \inv{r^n} 
= 
\Be_k \partial_k (x_m x_m)^{-n/2}
= 
-\frac{n}{2} \Be_k \frac{2 x_m \delta_{k m}}{r^{n+2}}
= 
-n \frac{\rcap}{r^{n+1}}.
\end{dmath}

Computing the gradient of the dot product, we find
\begin{dmath}\label{eqn:emtProblemSet2Problem1:160}
\spacegrad \frac{\rcap}{r^2} \cdot \Bd
=
\spacegrad \frac{\Br}{r^3} \cdot \Bd
=
\Be_k \partial_k \frac{x_m d_m}{r^3} 
=
\Be_k \frac{\delta_{k m} d_m}{r^3} 
+ \Br \cdot \Bd \spacegrad \inv{r^3}
=
\frac{\Bd}{r^3} 
-3 \Br \cdot \Bd \frac{\rcap}{r^4}
=
\frac{\Bd - 3 (\rcap \cdot \Bd) \rcap}{r^3},
\end{dmath}

so

\begin{dmath}\label{eqn:emtProblemSet2Problem1:180}
V(\Br) 
= \frac{q}{4 \pi \epsilon_0}
\frac{3 (\rcap \cdot \Bd) \rcap -\Bd}{r^3}.
\end{dmath}

With \( \Bp = q \Bd \), this is the result found in class

%\begin{dmath}\label{eqn:emtProblemSet2Problem1:200}
\boxedEquation{eqn:emtProblemSet2Problem1:200}{
V(\Br) 
= \frac{1}{4 \pi \epsilon_0}
\frac{3 (\rcap \cdot \Bp) \rcap -\Bp}{r^3}.
}
%\end{dmath}
}

%%%      %
% Copyright � 2016 Peeter Joot.  All Rights Reserved.
% Licenced as described in the file LICENSE under the root directory of this GIT repository.
%
%{
\makeproblem{Dipole moment density for disk.}{emt:problemSet2:4}{ 
A dielectric circular disk of radius \( a \) and thickness \( d \) is permanently polarized with a
dipole moment per unit volume \( \BP \) [\si{C/m^2}], where \( \Abs{\BP} \) 
is
constant and parallel to the disk axis (z-axis here) as
shown in \cref{fig:problemset2:problemset2Fig3}.

\imageFigure{../../figures/ece1228-emt/problemset2Fig3}{Circular disk geometry.}{fig:problemset2:problemset2Fig3}{0.3}

\makesubproblem{}{emt:problemSet2:4a}
Calculate the potential along the disk axis for \( z > 0 \).
\makesubproblem{}{emt:problemSet2:4b}
Approximate the result obtained in \partref{emt:problemSet2:4a} for the case
of \( Z \gg d \).
} % makeproblem

\makeanswer{emt:problemSet2:4}{ 
\makeSubAnswer{}{emt:problemSet2:4a}

In class the potential for a discrete dipole pair was found to be

\begin{dmath}\label{eqn:emtProblemSet2Problem4:20}
V(\Br) = \inv{4 \pi \epsilon_0} \frac{\rcap \cdot \Bp}{r^2},
\end{dmath}

so in the continuum, the element of dipole moment is \( d\Bp = \BP(\Br') dV' \), for a total potential of

\begin{dmath}\label{eqn:emtProblemSet2Problem4:40}
V(\Br)
= \inv{4 \pi \epsilon_0} \int dV' \frac{\lr{ \Br - \Br'} \cdot \BP(\Br')}{\lr{\Br - \Br'}^3}.
\end{dmath}

Restricting \( \Br \) to the z-axis, with \( \Br(z) = z \zcap \), and using cylindrical coordinates for the disk

\begin{dmath}\label{eqn:emtProblemSet2Problem4:60}
\Br' = z'\zcap + \rho'\rhocap,
\end{dmath}

where \( z' \in [-d,0] \) and \( \rho' \in [0, a] \).  The difference vector between the charge and the observation point is

\begin{dmath}\label{eqn:emtProblemSet2Problem4:80}
\Br - \Br' = (z - z')\zcap - \rho' \rhocap,
\end{dmath}

with magnitude
\begin{dmath}\label{eqn:emtProblemSet2Problem4:100}
\Abs{\Br - \Br'}^2 = (z - z')^2 + (\rho')^2.
\end{dmath}

With \( \BP = \Abs{\BP} \zcap \), the potential along the axis is

\begin{dmath}\label{eqn:emtProblemSet2Problem4:120}
V(z)
= 
\frac{2 \pi \Abs{\BP} }{4 \pi \epsilon_0} 
\int_0^a \rho' d\rho \int_{-d}^0 dz' \frac{\lr{ (z - z')\zcap - \rho' \rhocap} \cdot \zcap}{\lr{(z - z')^2 + (\rho')^2}^{3/2}}
= 
\frac{\Abs{\BP} }{2 \epsilon_0} 
\int_0^a \rho' d\rho \int_{-d}^0 dz' \frac{ (z - z') }{\lr{(z - z')^2 + (\rho')^2}^{3/2}}
= 
% u = z' - z
% v = \rho'
-\frac{\Abs{\BP} }{2 \epsilon_0} 
\int_0^a dv \int_{-d-z}^{-z} du \frac{ u v }{\lr{u^2 + v^2}^{3/2}}
= 
\frac{\Abs{\BP} }{2 \epsilon_0} 
\int_{-d-z}^{-z} du \evalrange{\lr{\frac{ u }{\lr{u^2 + v^2}^{1/2}}}}{v =0}{a}
= 
\frac{\Abs{\BP} }{2 \epsilon_0} 
\int_{-d-z}^{-z} du \lr{ \frac{ u }{\sqrt{u^2 + a^2}} - \frac{u}{\Abs{u}} }
= 
\frac{\Abs{\BP} }{2 \epsilon_0} 
\int_{-d-z}^{-z} du \lr{ \frac{ u }{\sqrt{u^2 + a^2}} - \sgn{u}}.
\end{dmath}

Looking at the values of the integration variable \( u \), note that
for \( z > 0 \), \( u < 0 \) and for \( z < -d \), \( u > 0 \), so the potential outside of the disk is

\begin{dmath}\label{eqn:emtProblemSet2Problem4:140}
V(z) 
= 
\frac{\Abs{\BP} }{2 \epsilon_0} \lr{ d \sgn(z) + \int_{-d-z}^{-z} du \frac{ u }{\sqrt{u^2 + a^2}} }
= 
\frac{\Abs{\BP} }{2 \epsilon_0} \lr{ d \sgn(z) + 
\evalrange{\lr{\sqrt{u^2 + a^2}}}{-d-z}{-z}
},
\end{dmath}

or

%\begin{dmath}\label{eqn:emtProblemSet2Problem4:160}
\boxedEquation{eqn:emtProblemSet2Problem4:180}{
V(z) 
= 
\frac{\Abs{\BP} }{2 \epsilon_0} \lr{ d \sgn(z) + 
\sqrt{ z^2 + a^2 } - \sqrt{ (d+z)^2 + a^2 }
}.
}
%\end{dmath}

\makeSubAnswer{}{emt:problemSet2:4b}

Let 

\begin{dmath}\label{eqn:emtProblemSet2Problem4:200}
f(z) = \sqrt{ z^2 + a^2 },
\end{dmath}

which has a derivative of
%This has a derivative that is zero at the origin

\begin{dmath}\label{eqn:emtProblemSet2Problem4:220}
f'(z) = \frac{z}{\sqrt{z^2 + a^2}}.
\end{dmath}

%The second derivative is
%\begin{dmath}\label{eqn:emtProblemSet2Problem4:240}
%f''(z) 
%= \frac{1}{\sqrt{z^2 + a^2}} + (-1/2) \frac{ z(2z) }{\lr{z^2 + a^2}^{3/2}}
%= \frac{ a^2 }{\lr{z^2 + a^2}^{3/2}}.
%\end{dmath}
%
The first order Taylor approximation is 

\begin{dmath}\label{eqn:emtProblemSet2Problem4:260}
f(z + d) - f(z)
\approx f'(z) d 
=
\frac{z d}{\sqrt{z^2 + a^2}},
\end{dmath}

so for \( \Abs{z} \gg d \) the potential is approximated by

\begin{dmath}\label{eqn:emtProblemSet2Problem4:280}
%\boxedEquation{eqn:emtProblemSet2Problem4:300}{
V(z) 
\approx
\frac{\Abs{\BP} }{2 \epsilon_0} \lr{ d \sgn(z) - \frac{z d}{\sqrt{z^2 + a^2}} },
%}
\end{dmath}

or
%\begin{dmath}\label{eqn:emtProblemSet2Problem4:360}
\boxedEquation{eqn:emtProblemSet2Problem4:380}{
V(z) 
=
\frac{\Abs{\BP} d \sgn(z)}{2 \epsilon_0} \lr{ 1 -
\frac{1}{\sqrt{1+ (a/z)^2}}
}.
}
%\end{dmath}

Note that if \( \Abs{z} \gg a \) too, this can be further approximated as

%\boxedEquation{eqn:emtProblemSet2Problem4:340}{
\begin{dmath}\label{eqn:emtProblemSet2Problem4:320}
V(z)
\approx
\frac{\Abs{\BP} d \sgn(z)}{2 \epsilon_0} \lr{ 1 - (1 + (-1/2) (a/z)^2) }
=
\inv{4 \pi \epsilon_0} \frac{\Abs{\BP} \sgn(z) (d \pi a^2)}{z^2}.
\end{dmath}
%}

This is a slightly tidier result that shows the asymptotic inverse-square \( z \) dependence of the potential more clearly than \cref{eqn:emtProblemSet2Problem4:380}, a result that assumed \( z \gg d \), but did not assume \( z \gg a \).  We also have the volume of the disk as an explicit factor in this approximation.
%, but requires \( z \\gg a,d \), a more strict limiting value than the \( z \gg d \) requested in the problem.
}
%}

      %
% Copyright � 2016 Peeter Joot.  All Rights Reserved.
% Licenced as described in the file LICENSE under the root directory of this GIT repository.
%
%{
%\newcommand{\authorname}{Peeter Joot}
\newcommand{\email}{peeterjoot@protonmail.com}
\newcommand{\basename}{FIXMEbasenameUndefined}
\newcommand{\dirname}{notes/FIXMEdirnameUndefined/}

%\renewcommand{\basename}{dipoleMoment}
%%\renewcommand{\dirname}{notes/phy1520/}
%\renewcommand{\dirname}{notes/ece1228-electromagnetic-theory/}
%%\newcommand{\dateintitle}{}
%%\newcommand{\keywords}{}
%
%\newcommand{\authorname}{Peeter Joot}
\newcommand{\onlineurl}{http://sites.google.com/site/peeterjoot2/math2013/\basename.pdf}
\newcommand{\sourcepath}{\dirname\basename.tex}
\newcommand{\generatetitle}[1]{\chapter{#1}}

\newcommand{\vcsinfo}{%
\section*{}
\noindent{\color{DarkOliveGreen}{\rule{\linewidth}{0.1mm}}}
\paragraph{Document version}
%\paragraph{\color{Maroon}{Document version}}
{
\small
\begin{itemize}
\item Available online at:\\ 
\href{\onlineurl}{\onlineurl}
\item Git Repository: \input{./.revinfo/gitRepo.tex}
\item Source: \sourcepath
\item last commit: \input{./.revinfo/gitCommitString.tex}
\item commit date: \input{./.revinfo/gitCommitDate.tex}
\end{itemize}
}
}

%\PassOptionsToPackage{dvipsnames,svgnames}{xcolor}
\PassOptionsToPackage{square,numbers}{natbib}
\documentclass{scrreprt}

\usepackage[left=2cm,right=2cm]{geometry}
\usepackage[svgnames]{xcolor}
\usepackage{peeters_layout}

\usepackage{natbib}

\usepackage[
colorlinks=true,
bookmarks=false,
pdfauthor={\authorname, \email},
backref 
]{hyperref}

% http://tex.stackexchange.com/questions/75773/how-to-reference-problems-by-the-text-label-in-an-exercise-envioronment
\usepackage[english]{cleveref}
\crefname{Exercise}{exercise}{exercises}
\Crefname{Exercise}{Exercise}{Exercises}

\RequirePackage{titlesec}
\RequirePackage{ifthen}

% http://stackoverflow.com/questions/4932910/date-in-the-tabular-environment
\makeatletter
\let\insertdate\@date
\makeatother

\titleformat{\chapter}[display]
{\bfseries\Large}
{\color{DarkSlateGrey}\filleft \authorname
\ifthenelse{\isundefined{\studentnumber}}{}{\\ \studentnumber}
\ifthenelse{\isundefined{\email}}{}{\\ \email}
\ifthenelse{\isundefined{\dateintitle}}{}{\\ \insertdate}
%\ifthenelse{\isundefined{\coursename}}{}{\\ \coursename} % put in title instead.
}
{4ex}
{\color{DarkOliveGreen}{\titlerule}\color{Maroon}
\vspace{2ex}%
\filright}
[\vspace{2ex}%
\color{DarkOliveGreen}\titlerule
]

\newcommand{\beginArtWithToc}[0]{\begin{document}\tableofcontents}
\newcommand{\beginArtNoToc}[0]{\begin{document}}
\newcommand{\EndNoBibArticle}[0]{\end{document}}
\newcommand{\EndArticle}[0]{\bibliography{Bibliography}\bibliographystyle{plainnat}\end{document}}

% 
%\newcommand{\citep}[1]{\cite{#1}}

\colorSectionsForArticle


%
%\usepackage{peeters_layout_exercise}
%\usepackage{peeters_braket}
%\usepackage{peeters_figures}
%\usepackage{siunitx}
%
%\beginArtNoToc
%
%\generatetitle{Dipole moment}
%\chapter{Dipole moment}
%\label{chap:dipoleMoment}

\makeproblem{Field for an electric dipole.}{problem:dipoleMoment:dipole}{
\index{dipole!electric}

An equal charge dipole configuration is sketched in \cref{fig:dipoleSignConventionL3:dipoleSignConventionL3Fig3}.  Compute the electric field.

% L3:
%\imageFigure{../../figures/ece1228-emt/dipoleSignConventionL3Fig3}{Dipole sign convention.}{fig:dipoleSignConventionL3:dipoleSignConventionL3Fig3}{0.3}
} % problem

\makeanswer{problem:dipoleMoment:dipole}{
The vector from the origin to the observation point is

\begin{equation}\label{eqn:dipoleMoment:20}
\Br = \BR_1 + \Bd/2
= \BR_2 - \Bd/2,
\end{equation}

or

\begin{equation}\label{eqn:dipoleMoment:40}
\begin{aligned}
\BR_1 &= \Br - \Bd/2 \equiv \BR_{+} \\
\BR_2 &= \Br + \Bd/2 \equiv \BR_{-}.
\end{aligned}
\end{equation}

The electric field for this superposition is
\begin{dmath}\label{eqn:dipoleMoment:60}
\BE
=
\inv{4 \pi \epsilon_0} \lr{
\frac{q \BR_{+}}{\Abs{\BR_{+}}^3} -
\frac{q \BR_{-}}{\Abs{\BR_{-}}^3}
}
=
\frac{q}{4 \pi \epsilon_0} \lr{
\frac{\Br - \Bd/2}{\Abs{\BR_{+}}^3} -
\frac{\Br + \Bd/2}{\Abs{\BR_{-}}^3}
}
=
\frac{q}{4 \pi \epsilon_0} \lr{
\Br \lr{
\inv{\Abs{\BR_{+}}^3}
 -
\inv{\Abs{\BR_{-}}^3}
}
-
\frac{\Bd}{2} \lr{
\inv{\Abs{\BR_{+}}^3}
+
\inv{\Abs{\BR_{-}}^3}
}
}.
\end{dmath}

The magnitudes can be expanded in Taylor series

\begin{dmath}\label{eqn:dipoleMoment:80}
\Abs{\BR_{\pm}}^{3}
=
\lr{
\lr{ \Br \mp \Bd/2 } \cdot \lr{ \Br \mp \Bd/2 }
}^{-3/2}
=
\lr{
\lr{ \Br^2 + (\Bd/2)^2 \mp 2 \Br \cdot \Bd/2 }
}^{-3/2}
=
\lr{
\lr{ \Br^2 + (\Bd/2)^2 \mp \Br \cdot \Bd }
}^{-3/2}
=
(\Br^2)^{-3/2}
\lr{
\lr{ 1 + \lr{\frac{\Bd}{2 r}}^2 \mp \rcap \cdot \frac{\Bd}{r} }
}^{-1/2}
=
r^{-3}
\lr{
1
-\frac{3}{2}
\lr{ \lr{\frac{\Bd}{2 r}}^2 \mp \rcap \cdot \frac{\Bd}{r} }
+\lr{\frac{-3}{2}}
\lr{\frac{-5}{2}} \inv{2!}
\lr{ \lr{\frac{\Bd}{2 r}}^2 \mp \rcap \cdot \frac{\Bd}{r} }^2
+ \cdots
}.
\end{dmath}

Here \( r = \Abs{\Br} \), and the Taylor series was taken in the \( \Bd/r \ll 1 \) limit.  The sums and differences of these magnitudes, are to first order

\begin{dmath}\label{eqn:dipoleMoment:100}
\inv{\Abs{\BR_{+}}^3}
-
\inv{\Abs{\BR_{-}}^3}
=
2 \frac{1}{r^3} \lr{\frac{-3}{2}} \lr{-\rcap \cdot \frac{\Bd}{r}}
\approx
\frac{3}{r^4} \rcap \cdot \Bd,
\end{dmath}

and

\begin{dmath}\label{eqn:dipoleMoment:120}
\inv{\Abs{\BR_{+}}^3}
+
\inv{\Abs{\BR_{-}}^3}
\approx
\frac{2}{r^3}.
\end{dmath}

The \( \Br \gg \Bd \) limiting expression for the electric field is

\begin{dmath}\label{eqn:dipoleMoment:140}
\BE
\approx
\frac{q}{4 \pi \epsilon_0 r^3} \lr{
3 \rcap \lr{ \rcap \cdot \Bd }
-
2 \frac{\Bd}{2}
},
\end{dmath}

or, with \( \Bp = q \Bd \)

%\begin{dmath}\label{eqn:dipoleMoment:180}
\boxedEquation{eqn:dipoleMoment:180}{
\BE =
\frac{1}{4 \pi \epsilon_0 r^3} \lr{
3 \rcap \lr{ \rcap \cdot \Bp }
-\Bp
}.
}
%\end{dmath}

} % answer

%}
%\EndNoBibArticle

      %
% Copyright � 2016 Peeter Joot.  All Rights Reserved.
% Licenced as described in the file LICENSE under the root directory of this GIT repository.
%
%{
\newcommand{\authorname}{Peeter Joot}
\newcommand{\email}{peeterjoot@protonmail.com}
\newcommand{\basename}{FIXMEbasenameUndefined}
\newcommand{\dirname}{notes/FIXMEdirnameUndefined/}

\renewcommand{\basename}{dipolePotential}
%\renewcommand{\dirname}{notes/phy1520/}
\renewcommand{\dirname}{notes/ece1228-electromagnetic-theory/}
%\newcommand{\dateintitle}{}
%\newcommand{\keywords}{}

\newcommand{\authorname}{Peeter Joot}
\newcommand{\onlineurl}{http://sites.google.com/site/peeterjoot2/math2013/\basename.pdf}
\newcommand{\sourcepath}{\dirname\basename.tex}
\newcommand{\generatetitle}[1]{\chapter{#1}}

\newcommand{\vcsinfo}{%
\section*{}
\noindent{\color{DarkOliveGreen}{\rule{\linewidth}{0.1mm}}}
\paragraph{Document version}
%\paragraph{\color{Maroon}{Document version}}
{
\small
\begin{itemize}
\item Available online at:\\ 
\href{\onlineurl}{\onlineurl}
\item Git Repository: \input{./.revinfo/gitRepo.tex}
\item Source: \sourcepath
\item last commit: \input{./.revinfo/gitCommitString.tex}
\item commit date: \input{./.revinfo/gitCommitDate.tex}
\end{itemize}
}
}

%\PassOptionsToPackage{dvipsnames,svgnames}{xcolor}
\PassOptionsToPackage{square,numbers}{natbib}
\documentclass{scrreprt}

\usepackage[left=2cm,right=2cm]{geometry}
\usepackage[svgnames]{xcolor}
\usepackage{peeters_layout}

\usepackage{natbib}

\usepackage[
colorlinks=true,
bookmarks=false,
pdfauthor={\authorname, \email},
backref 
]{hyperref}

% http://tex.stackexchange.com/questions/75773/how-to-reference-problems-by-the-text-label-in-an-exercise-envioronment
\usepackage[english]{cleveref}
\crefname{Exercise}{exercise}{exercises}
\Crefname{Exercise}{Exercise}{Exercises}

\RequirePackage{titlesec}
\RequirePackage{ifthen}

% http://stackoverflow.com/questions/4932910/date-in-the-tabular-environment
\makeatletter
\let\insertdate\@date
\makeatother

\titleformat{\chapter}[display]
{\bfseries\Large}
{\color{DarkSlateGrey}\filleft \authorname
\ifthenelse{\isundefined{\studentnumber}}{}{\\ \studentnumber}
\ifthenelse{\isundefined{\email}}{}{\\ \email}
\ifthenelse{\isundefined{\dateintitle}}{}{\\ \insertdate}
%\ifthenelse{\isundefined{\coursename}}{}{\\ \coursename} % put in title instead.
}
{4ex}
{\color{DarkOliveGreen}{\titlerule}\color{Maroon}
\vspace{2ex}%
\filright}
[\vspace{2ex}%
\color{DarkOliveGreen}\titlerule
]

\newcommand{\beginArtWithToc}[0]{\begin{document}\tableofcontents}
\newcommand{\beginArtNoToc}[0]{\begin{document}}
\newcommand{\EndNoBibArticle}[0]{\end{document}}
\newcommand{\EndArticle}[0]{\bibliography{Bibliography}\bibliographystyle{plainnat}\end{document}}

% 
%\newcommand{\citep}[1]{\cite{#1}}

\colorSectionsForArticle



\usepackage{peeters_layout_exercise}
\usepackage{peeters_braket}
\usepackage{peeters_figures}
\usepackage{siunitx}

\beginArtNoToc

\generatetitle{Electric dipole potential}
%\chapter{Electric dipole potential}
%\label{chap:dipolePotential}
% \citep{sakurai2014modern} pr X.Y
% \citep{pozar2009microwave}
% \citep{qftLectureNotes}
% \citep{doran2003gap}
% \citep{jackson1975cew}
% \citep{griffiths1999introduction}

\makeproblem{Electric dipole potential}{problem:dipolePotential:1}{

Having shown that 

\begin{dmath}\label{eqn:dipolePotential:20}
\BE =
\frac{1}{4 \pi \epsilon_0 r^3} \lr{ 
3 \rcap \lr{ \rcap \cdot \Bp }
-\Bp 
},
\end{dmath}

find the expression for the electric potential for this field.
} % problem

\makeanswer{problem:dipolePotential:1}{

With the electric potential defined indirectly by
\begin{dmath}\label{eqn:dipolePotential:40}
\BE = -\spacegrad V,
\end{dmath}

we can integrate to find the difference in potential between two points
\begin{dmath}\label{eqn:dipolePotential:60}
\int_\Ba^\Bb \BE \cdot d\Bl = 
- \int
\int_\Ba^\Bb \spacegrad V \cdot d\Bl
=
- \lr{ V(\Bb) - V(\Ba) },
\end{dmath}

or
\begin{dmath}\label{eqn:dipolePotential:80}
V(\Bb) - V(\Ba) = -
\int_\Ba^\Bb \BE \cdot d\Bl.
\end{dmath}

Since the dipole potential is zero at \( \Br = \infty \), we have

\begin{dmath}\label{eqn:dipolePotential:100}
V(\Br) 
= -\int_\infty^\Br \BE \cdot d\Bl.
\end{dmath}

Let's integrate this on the radial path \( \Br(r') = r'\rcap \), for \( r' \in [\infty, r] \)

\begin{dmath}\label{eqn:dipolePotential:120}
V(\Br) 
= -\int_\infty^\Br \BE \cdot d\Bl
= -\int_\infty^\Br \BE \cdot \rcap dr'
= 
-
\frac{1}{4 \pi \epsilon_0 } 
\int_\infty^r \frac{dr'}{{r'}^3}
\rcap
\cdot
\lr{ 
3 \rcap \lr{ \rcap \cdot \Bp }
-\Bp 
}
=
-\frac{2}{4 \pi \epsilon_0 } 
\int_\infty^r dr' \frac{\rcap\cdot \Bp}{{r'}^3}
=
\frac{\rcap \cdot \Bp}{4 \pi \epsilon_0 } \evalrange{ \inv{{r'}^2} }{\infty}{r},
\end{dmath}

so
%\begin{dmath}\label{eqn:dipolePotential:160}
\boxedEquation{eqn:dipolePotential:140}{
V(\Br) =
\frac{ \rcap \cdot \Bp}{4 \pi \epsilon_0 }.
}
%\end{dmath}
} % answer

%}
%\EndArticle
\EndNoBibArticle


   %
% Copyright � 2016 Peeter Joot.  All Rights Reserved.
% Licenced as described in the file LICENSE under the root directory of this GIT repository.
%
\newcommand{\authorname}{Peeter Joot}
\newcommand{\email}{peeterjoot@protonmail.com}
\newcommand{\basename}{FIXMEbasenameUndefined}
\newcommand{\dirname}{notes/FIXMEdirnameUndefined/}

\renewcommand{\basename}{emt4}
\renewcommand{\dirname}{notes/ece1228/}
\newcommand{\keywords}{ECE1228H}
\newcommand{\authorname}{Peeter Joot}
\newcommand{\onlineurl}{http://sites.google.com/site/peeterjoot2/math2013/\basename.pdf}
\newcommand{\sourcepath}{\dirname\basename.tex}
\newcommand{\generatetitle}[1]{\chapter{#1}}

\newcommand{\vcsinfo}{%
\section*{}
\noindent{\color{DarkOliveGreen}{\rule{\linewidth}{0.1mm}}}
\paragraph{Document version}
%\paragraph{\color{Maroon}{Document version}}
{
\small
\begin{itemize}
\item Available online at:\\ 
\href{\onlineurl}{\onlineurl}
\item Git Repository: \input{./.revinfo/gitRepo.tex}
\item Source: \sourcepath
\item last commit: \input{./.revinfo/gitCommitString.tex}
\item commit date: \input{./.revinfo/gitCommitDate.tex}
\end{itemize}
}
}

%\PassOptionsToPackage{dvipsnames,svgnames}{xcolor}
\PassOptionsToPackage{square,numbers}{natbib}
\documentclass{scrreprt}

\usepackage[left=2cm,right=2cm]{geometry}
\usepackage[svgnames]{xcolor}
\usepackage{peeters_layout}

\usepackage{natbib}

\usepackage[
colorlinks=true,
bookmarks=false,
pdfauthor={\authorname, \email},
backref 
]{hyperref}

% http://tex.stackexchange.com/questions/75773/how-to-reference-problems-by-the-text-label-in-an-exercise-envioronment
\usepackage[english]{cleveref}
\crefname{Exercise}{exercise}{exercises}
\Crefname{Exercise}{Exercise}{Exercises}

\RequirePackage{titlesec}
\RequirePackage{ifthen}

% http://stackoverflow.com/questions/4932910/date-in-the-tabular-environment
\makeatletter
\let\insertdate\@date
\makeatother

\titleformat{\chapter}[display]
{\bfseries\Large}
{\color{DarkSlateGrey}\filleft \authorname
\ifthenelse{\isundefined{\studentnumber}}{}{\\ \studentnumber}
\ifthenelse{\isundefined{\email}}{}{\\ \email}
\ifthenelse{\isundefined{\dateintitle}}{}{\\ \insertdate}
%\ifthenelse{\isundefined{\coursename}}{}{\\ \coursename} % put in title instead.
}
{4ex}
{\color{DarkOliveGreen}{\titlerule}\color{Maroon}
\vspace{2ex}%
\filright}
[\vspace{2ex}%
\color{DarkOliveGreen}\titlerule
]

\newcommand{\beginArtWithToc}[0]{\begin{document}\tableofcontents}
\newcommand{\beginArtNoToc}[0]{\begin{document}}
\newcommand{\EndNoBibArticle}[0]{\end{document}}
\newcommand{\EndArticle}[0]{\bibliography{Bibliography}\bibliographystyle{plainnat}\end{document}}

% 
%\newcommand{\citep}[1]{\cite{#1}}

\colorSectionsForArticle



%\usepackage{ece1228}
\usepackage{peeters_braket}
%\usepackage{peeters_layout_exercise}
\usepackage{peeters_figures}
\usepackage{mathtools}
\usepackage{siunitx}

\beginArtNoToc
\generatetitle{ECE1228H Electromagnetic Theory.  Lecture 4: Magnetic moment, and Boundary value conditions.  Taught by Prof.\ M. Mojahedi}
%\chapter{Magnetic moment, and Boundary value conditions}
\label{chap:emt4}

%\paragraph{Disclaimer}
%
%Peeter's lecture notes from class.  These may be incoherent and rough.
%
%These are notes for the UofT course ECE1228H, Electromagnetic Theory, taught by Prof. M. Mojahedi, covering \textchapref{{1}} \citep{balanis1989advanced} content.

\paragraph{Magnetic moment.}

Using a semi-classical model of an electron, assuming that the electron circles the nuclei.  This is a completely wrong model, but useful.  In reality, electrons are random and probabilistic and do not follow defined paths.  We do however have a magnetic moment associated with the electron, and one associated with the spin of the electron, and a moment associated with the spin of the nuclei.  All of these concepts can be used to describe a more accurate model and such a model is discussed in \citep{jackson1975cew} chapters 11,12,13.

Ignoring the details of how the moments really occur physically, we can take it as a given that they exist, and model them as elemenetal magnetic dipole moments of the form

\begin{dmath}\label{eqn:emtLecture4:20}
d\Bm_i = \ncap_i I_i ds_i \qquad [\si{A m^2}].
\end{dmath}

Here the normal is defined in terms of the right hand rule with respect to the direction of the current as sketched in 

F1

Such dipole moments are actually what an MRI measures.  The noises that people describe from MRI machines are actually when the very powerful magnets are being rotated, allowing for the magnetic moments in the atoms of the body to be measured in different directions.

The magnetic polarization, or magnetization \( \BM \), in [\si{A/m}]] is given by

\begin{dmath}\label{eqn:emtLecture4:40}
\BM 
= \lim_{\Delta v \rightarrow 0} \lr{ \inv{\Delta v} \Bm_i }
= \lim_{\Delta v \rightarrow 0} \lr{ \inv{\Delta v} \sum_{i = 1}^{N \delta v} d\Bm_i }
= \lim_{\Delta v \rightarrow 0} \lr{ \inv{\Delta v} \sum_{i = 1}^{N \delta v} \ncap_i I_i ds_i } .
\end{dmath}

In materials the magnetization within the atoms are usually random, however, application of a magnetic field can force these to line up, as sketched in

F2

This is accomplished because an applied magnetic field acting on the magnetic moment introduces a torque, as also occured with dipole moments under applied electric fields

\begin{dmath}\label{eqn:emtLecture4:60}
\begin{aligned}
\Btau_B &= d\Bm \cross \BB_a \\
\Btau_E &= d\Bp \cross \BE_a.
\end{aligned}
\end{dmath}

There is an energy associated with this torque

\begin{dmath}\label{eqn:emtLecture4:80}
\begin{aligned}
\Delta U_B &= -d\Bm \cdot \BB_a \\
\Delta U_E &= -d\Bp \cdot \BE_a.
\end{aligned}
\end{dmath}

In analogy with the electric dipole moment analysis, it can be assumed that there is a linear relationship between the magnetic polarization and the applied magnetic field

\begin{dmath}\label{eqn:emtLecture4:100}
\BB = \mu_0 \BH_a + \mu_0 \BM = \mu_0\lr{ \BH_a + \BM },
\end{dmath}

where
\begin{dmath}\label{eqn:emtLecture4:120}
\BM = \chi_m \BH_a,
\end{dmath}

so
\begin{equation}\label{eqn:emtLecture4:140}
\BB 
= \mu_0\lr{ 1 + \chi_m } \BH_a
\equiv \mu \BH_a.
\end{equation}

Like electric dipoles, in a volume, we can have bound currents on the surface [\si{A/m}], as well as bound volume currents [\si{A/m^2}] as sketched in

F3

It can be shown, as with the electric dipoles related bound charge densities of \crefeqn:emtLecture3:620}, that magnetic currents can be defined

\begin{dmath}\label{eqn:emtLecture4:n}
\begin{aligned}
\BJ_{sm} &= \BM \cross \ncap \\
\BJ_{vm} &= \spacegrad \cross \BM,
\end{aligned}
\end{dmath}

(showing this may be given as a homework problem ... if not do it).

\paragraph{Conductivity}

We have two constitutive relationships so far
\begin{dmath}\label{eqn:emtLecture4:n}
\begin{aligned}
\BD &= \epsilon \BE \\
\BB &= \mu \BH
\end{aligned}
\end{dmath}

but this needs to be augmented by

\begin{dmath}\label{eqn:emtLecture4:n}
\BJ_c = \epsilon \BE.
\end{dmath}

There are a couple ways to discuss this.  One is to model \( \epsilon \) as a complex number.  Such a model is not entirely unconstrained.  Like with the Cauchy-Riemann conditions that relate derivatives of the real and imaginary parts of a complex number, there is a relationship (Kramers-Kronig \citep{wiki:kramersKronig}), an integral relationship that relates the real and imaginary parts of the permittivity \( \epsilon \).

\paragraph{Boundary conditions.}

\EndArticle

      \section{Problems}
      %
% Copyright � 2016 Peeter Joot.  All Rights Reserved.
% Licenced as described in the file LICENSE under the root directory of this GIT repository.
%
%{
\newcommand{\authorname}{Peeter Joot}
\newcommand{\email}{peeterjoot@protonmail.com}
\newcommand{\basename}{FIXMEbasenameUndefined}
\newcommand{\dirname}{notes/FIXMEdirnameUndefined/}

\renewcommand{\basename}{magneticMomentJackson}
%\renewcommand{\dirname}{notes/phy1520/}
\renewcommand{\dirname}{notes/ece1228-electromagnetic-theory/}
%\newcommand{\dateintitle}{}
%\newcommand{\keywords}{}

\newcommand{\authorname}{Peeter Joot}
\newcommand{\onlineurl}{http://sites.google.com/site/peeterjoot2/math2013/\basename.pdf}
\newcommand{\sourcepath}{\dirname\basename.tex}
\newcommand{\generatetitle}[1]{\chapter{#1}}

\newcommand{\vcsinfo}{%
\section*{}
\noindent{\color{DarkOliveGreen}{\rule{\linewidth}{0.1mm}}}
\paragraph{Document version}
%\paragraph{\color{Maroon}{Document version}}
{
\small
\begin{itemize}
\item Available online at:\\ 
\href{\onlineurl}{\onlineurl}
\item Git Repository: \input{./.revinfo/gitRepo.tex}
\item Source: \sourcepath
\item last commit: \input{./.revinfo/gitCommitString.tex}
\item commit date: \input{./.revinfo/gitCommitDate.tex}
\end{itemize}
}
}

%\PassOptionsToPackage{dvipsnames,svgnames}{xcolor}
\PassOptionsToPackage{square,numbers}{natbib}
\documentclass{scrreprt}

\usepackage[left=2cm,right=2cm]{geometry}
\usepackage[svgnames]{xcolor}
\usepackage{peeters_layout}

\usepackage{natbib}

\usepackage[
colorlinks=true,
bookmarks=false,
pdfauthor={\authorname, \email},
backref 
]{hyperref}

% http://tex.stackexchange.com/questions/75773/how-to-reference-problems-by-the-text-label-in-an-exercise-envioronment
\usepackage[english]{cleveref}
\crefname{Exercise}{exercise}{exercises}
\Crefname{Exercise}{Exercise}{Exercises}

\RequirePackage{titlesec}
\RequirePackage{ifthen}

% http://stackoverflow.com/questions/4932910/date-in-the-tabular-environment
\makeatletter
\let\insertdate\@date
\makeatother

\titleformat{\chapter}[display]
{\bfseries\Large}
{\color{DarkSlateGrey}\filleft \authorname
\ifthenelse{\isundefined{\studentnumber}}{}{\\ \studentnumber}
\ifthenelse{\isundefined{\email}}{}{\\ \email}
\ifthenelse{\isundefined{\dateintitle}}{}{\\ \insertdate}
%\ifthenelse{\isundefined{\coursename}}{}{\\ \coursename} % put in title instead.
}
{4ex}
{\color{DarkOliveGreen}{\titlerule}\color{Maroon}
\vspace{2ex}%
\filright}
[\vspace{2ex}%
\color{DarkOliveGreen}\titlerule
]

\newcommand{\beginArtWithToc}[0]{\begin{document}\tableofcontents}
\newcommand{\beginArtNoToc}[0]{\begin{document}}
\newcommand{\EndNoBibArticle}[0]{\end{document}}
\newcommand{\EndArticle}[0]{\bibliography{Bibliography}\bibliographystyle{plainnat}\end{document}}

% 
%\newcommand{\citep}[1]{\cite{#1}}

\colorSectionsForArticle



\usepackage{peeters_layout_exercise}
\usepackage{peeters_braket}
\usepackage{peeters_figures}
\usepackage{siunitx}
%\usepackage{txfonts} % \ointclockwise

\beginArtNoToc

\generatetitle{Magnetic moment for a localized magnetostatic current}
%\chapter{Magnetic moment for a localized magnetostatic current}
%\label{chap:magneticMomentJackson}
% \citep{sakurai2014modern} pr X.Y
% \citep{pozar2009microwave}
% \citep{qftLectureNotes}
% \citep{doran2003gap}
\paragraph{Motivation.}

I was once again reading my Jackson \citep{jackson1975cew}.  This time I found that his 
presentation of magnetic moment didn't really make sense to me.  Here's my own pass through it, filling in a number of details.  As I did last time, I'll also translate into SI units as I go.

\paragraph{Vector potential.}

The Biot-Savart expression for the magnetic field can be factored into a curl expression using the usual tricks

\begin{dmath}\label{eqn:magneticMomentJackson:20}
\BB 
= \frac{\mu_0}{4\pi} \int \frac{\BJ(\Bx') \cross (\Bx - \Bx')}{\Abs{\Bx - \Bx'}^3} d^3 x'
= -\frac{\mu_0}{4\pi} \int \BJ(\Bx') \cross \spacegrad \inv{\Abs{\Bx - \Bx'}} d^3 x'
= \frac{\mu_0}{4\pi} \spacegrad \cross \int \frac{\BJ(\Bx')}{\Abs{\Bx - \Bx'}} d^3 x',
\end{dmath}

so the vector potential, through its curl, defines the magnetic field \( \BB = \spacegrad \cross \BA \) is given by

\begin{dmath}\label{eqn:magneticMomentJackson:40}
\BA(\Bx) = \frac{\mu_0}{4 \pi} \int \frac{J(\Bx')}{\Abs{\Bx - \Bx'}} d^3 x'.
\end{dmath}

If the current source is localized (zero outside of some finite region), then there will always be a region for which \( \Abs{\Bx} \gg \Abs{\Bx'} \), so the denominator yields to Taylor expansion

\begin{dmath}\label{eqn:magneticMomentJackson:60}
\inv{\Abs{\Bx - \Bx'}}
=
\inv{\Abs{\Bx}} \lr{1 + \frac{\Abs{\Bx'}^2}{\Abs{\Bx}^2} - 2 \frac{\Bx \cdot \Bx'}{\Abs{\Bx}^2} }^{-1/2}
\approx
\inv{\Abs{\Bx}} \lr{ 1 + \frac{\Bx \cdot \Bx'}{\Abs{\Bx}^2} }
=
\inv{\Abs{\Bx}} + \frac{\Bx \cdot \Bx'}{\Abs{\Bx}^3}.
\end{dmath}

so the vector potential, far enough away from the current source is
\begin{dmath}\label{eqn:magneticMomentJackson:80}
\BB(\Bx) 
=
\frac{\mu_0}{4 \pi} \int \frac{J(\Bx')}{\Abs{\Bx}} d^3 x'
+\frac{\mu_0}{4 \pi} \int \frac{(\Bx \cdot \Bx')J(\Bx')}{\Abs{\Bx}^3} d^3 x'.
\end{dmath}

Jackson uses a sneaky trick to show that the first integral is killed for a localized source.  That trick appears to be based on evaluating the following divergence

\begin{dmath}\label{eqn:magneticMomentJackson:100}
\spacegrad \cdot (\BJ(\Bx) x_i)
=
(\spacegrad \cdot \BJ) x_i
+
(\spacegrad x_i) \cdot \BJ
=
(\Be_k \partial_k x_i) \cdot\BJ
=
\delta_{ki} J_k
=
J_i.
\end{dmath}

Note that this made use of the fact that \( \spacegrad \cdot \BJ = 0 \) for magnetostatics.  This provides a way to rewrite the current density as a divergence

\begin{dmath}\label{eqn:magneticMomentJackson:120}
\int \frac{J(\Bx')}{\Abs{\Bx}} d^3 x'
=
\Be_i \int \frac{\spacegrad' \cdot (x_i' \BJ(\Bx'))}{\Abs{\Bx}} d^3 x'
=
\frac{\Be_i}{\Abs{\Bx}} \int \spacegrad' \cdot (x_i' \BJ(\Bx')) d^3 x'
=
\frac{1}{\Abs{\Bx}} \oint \Bx' (d\Ba \cdot \BJ(\Bx')).
\end{dmath}

When \( \BJ \) is localized, this is zero provided we pick the integration surface for the volume outside of that localization region.

It is now desired to rewrite \( \int \Bx \cdot \Bx' \BJ \) as a triple cross product since the dot product of such a triple cross product has exactly this term in it

\begin{dmath}\label{eqn:magneticMomentJackson:140}
- \Bx \cross \int \Bx' \cross \BJ
=
\int (\Bx \cdot \Bx') \BJ
-
\int (\Bx \cdot \BJ) \Bx'
=
\int (\Bx \cdot \Bx') \BJ
-
\Be_k x_i \int J_i x_k',
\end{dmath}

so
\begin{dmath}\label{eqn:magneticMomentJackson:160}
\int (\Bx \cdot \Bx') \BJ
=
- \Bx \cross \int \Bx' \cross \BJ
+
\Be_k x_i \int J_i x_k'.
\end{dmath}

To get of this second term, the next sneaky trick is to consider the following divergence

\begin{dmath}\label{eqn:magneticMomentJackson:180}
\oint d\Ba' \cdot (\BJ(\Bx') x_i' x_j')
=
\int dV' \spacegrad' \cdot (\BJ(\Bx') x_i' x_j')
=
\int dV' (\spacegrad' \cdot \BJ)
+
\int dV' \BJ \cdot \spacegrad' (x_i' x_j')
=
\int dV' J_k \cdot \lr{ x_i' \partial_k x_j' + x_j' \partial_k x_i' }
=
\int dV' \lr{ J_k x_i' \delta_{kj} + J_k x_j' \delta_{ki} }
=
\int dV' \lr{ J_j x_i' + J_i x_j'}.
\end{dmath}

The surface integral is once again zero, which means that we have an antisymmetric relationship in integrals of the form

\begin{dmath}\label{eqn:magneticMomentJackson:200}
\int J_j x_i' = -\int J_i x_j'.
\end{dmath}

Now we can use the tensor algebra trick of writing \( y = (y + y)/2 \),

\begin{dmath}\label{eqn:magneticMomentJackson:220}
\int (\Bx \cdot \Bx') \BJ
=
- \Bx \cross \int \Bx' \cross \BJ
+
\Be_k x_i \int J_i x_k'
=
- \Bx \cross \int \Bx' \cross \BJ
+
\inv{2} \Be_k x_i \int \lr{ J_i x_k' + J_i x_k' }
=
- \Bx \cross \int \Bx' \cross \BJ
+
\inv{2} \Be_k x_i \int \lr{ J_i x_k' - J_k x_i' }
=
- \Bx \cross \int \Bx' \cross \BJ
+
\inv{2} \Be_k x_i \int (\BJ \cross \Bx')_j \epsilon_{ikj}
=
- \Bx \cross \int \Bx' \cross \BJ
-
\inv{2} \epsilon_{kij} \Be_k x_i \int (\BJ \cross \Bx')_j 
=
- \Bx \cross \int \Bx' \cross \BJ
-
\inv{2} \Bx \cross \int \BJ \cross \Bx'
=
- \Bx \cross \int \Bx' \cross \BJ
+
\inv{2} \Bx \cross \int \Bx' \cross \BJ
=
-\inv{2} \Bx \cross \int \Bx' \cross \BJ,
\end{dmath}

so

\begin{dmath}\label{eqn:magneticMomentJackson:240}
\BA(\Bx) \approx \frac{\mu_0}{4 \pi \Abs{\Bx}^3} \lr{ -\frac{\Bx}{2} } \int \Bx' \cross \BJ(\Bx') d^3 x'.
\end{dmath}

Letting 

%\begin{dmath}\label{eqn:magneticMomentJackson:260}
\boxedEquation{eqn:magneticMomentJackson:260}{
\Bm = \inv{2} \int \Bx' \cross \BJ(\Bx') d^3 x',
}
%\end{dmath}

the far field approximation of the vector potential is
%\begin{dmath}\label{eqn:magneticMomentJackson:280}
\boxedEquation{eqn:magneticMomentJackson:280}{
\BA(\Bx) = \frac{\mu_0}{4 \pi} \frac{\Bm \cross \Bx}{\Abs{\Bx}^3}.
}
%\end{dmath}

Note that when the current is restricted to an infintisimally thin loop, the magnetic moment reduces to

\begin{dmath}\label{eqn:magneticMomentJackson:300}
\Bm(\Bx) = \frac{I}{2} \int \Bx \cross d\Bl'.
\end{dmath}

Refering to \citep{griffiths1999introduction} (pr. 1.60), this can be seen to be \( I \) times the ``vector-area'' integral.

%}
\EndArticle

      %
% Copyright � 2016 Peeter Joot.  All Rights Reserved.
% Licenced as described in the file LICENSE under the root directory of this GIT repository.
%
%{
\newcommand{\authorname}{Peeter Joot}
\newcommand{\email}{peeterjoot@protonmail.com}
\newcommand{\basename}{FIXMEbasenameUndefined}
\newcommand{\dirname}{notes/FIXMEdirnameUndefined/}

\renewcommand{\basename}{vectorAreaGriffiths}
%\renewcommand{\dirname}{notes/phy1520/}
\renewcommand{\dirname}{notes/ece1228-electromagnetic-theory/}
%\newcommand{\dateintitle}{}
%\newcommand{\keywords}{}

\newcommand{\authorname}{Peeter Joot}
\newcommand{\onlineurl}{http://sites.google.com/site/peeterjoot2/math2013/\basename.pdf}
\newcommand{\sourcepath}{\dirname\basename.tex}
\newcommand{\generatetitle}[1]{\chapter{#1}}

\newcommand{\vcsinfo}{%
\section*{}
\noindent{\color{DarkOliveGreen}{\rule{\linewidth}{0.1mm}}}
\paragraph{Document version}
%\paragraph{\color{Maroon}{Document version}}
{
\small
\begin{itemize}
\item Available online at:\\ 
\href{\onlineurl}{\onlineurl}
\item Git Repository: \input{./.revinfo/gitRepo.tex}
\item Source: \sourcepath
\item last commit: \input{./.revinfo/gitCommitString.tex}
\item commit date: \input{./.revinfo/gitCommitDate.tex}
\end{itemize}
}
}

%\PassOptionsToPackage{dvipsnames,svgnames}{xcolor}
\PassOptionsToPackage{square,numbers}{natbib}
\documentclass{scrreprt}

\usepackage[left=2cm,right=2cm]{geometry}
\usepackage[svgnames]{xcolor}
\usepackage{peeters_layout}

\usepackage{natbib}

\usepackage[
colorlinks=true,
bookmarks=false,
pdfauthor={\authorname, \email},
backref 
]{hyperref}

% http://tex.stackexchange.com/questions/75773/how-to-reference-problems-by-the-text-label-in-an-exercise-envioronment
\usepackage[english]{cleveref}
\crefname{Exercise}{exercise}{exercises}
\Crefname{Exercise}{Exercise}{Exercises}

\RequirePackage{titlesec}
\RequirePackage{ifthen}

% http://stackoverflow.com/questions/4932910/date-in-the-tabular-environment
\makeatletter
\let\insertdate\@date
\makeatother

\titleformat{\chapter}[display]
{\bfseries\Large}
{\color{DarkSlateGrey}\filleft \authorname
\ifthenelse{\isundefined{\studentnumber}}{}{\\ \studentnumber}
\ifthenelse{\isundefined{\email}}{}{\\ \email}
\ifthenelse{\isundefined{\dateintitle}}{}{\\ \insertdate}
%\ifthenelse{\isundefined{\coursename}}{}{\\ \coursename} % put in title instead.
}
{4ex}
{\color{DarkOliveGreen}{\titlerule}\color{Maroon}
\vspace{2ex}%
\filright}
[\vspace{2ex}%
\color{DarkOliveGreen}\titlerule
]

\newcommand{\beginArtWithToc}[0]{\begin{document}\tableofcontents}
\newcommand{\beginArtNoToc}[0]{\begin{document}}
\newcommand{\EndNoBibArticle}[0]{\end{document}}
\newcommand{\EndArticle}[0]{\bibliography{Bibliography}\bibliographystyle{plainnat}\end{document}}

% 
%\newcommand{\citep}[1]{\cite{#1}}

\colorSectionsForArticle



\usepackage{peeters_layout_exercise}
\usepackage{peeters_braket}
\usepackage{peeters_figures}
\usepackage{siunitx}
\usepackage{txfonts} % \ointclockwise

\beginArtNoToc

\generatetitle{Vector Area}
%\chapter{Vector Area}

One of the results of this problem is required for a later one on magnetic moments that I'd like to do.

\makeoproblem{Vector Area.}{problem:vectorAreaGriffiths:1}{\citep{griffiths1999introduction} pr. 1.61}{

The integral 

\begin{dmath}\label{eqn:vectorAreaGriffiths:20}
\Ba = \int_S d\Ba,
\end{dmath}

is sometimes called the vector area of the surface \( S \).

\makesubproblem{}{problem:vectorAreaGriffiths:1:a}

Find the vector area of a hemispherical bowl of radius \( R \).
\makesubproblem{}{problem:vectorAreaGriffiths:1:b}

Show that \( \Ba = 0 \) for any closed surface.
\makesubproblem{}{problem:vectorAreaGriffiths:1:c}
Show that \( \Ba \) is the same for all surfaces sharing the same boundary.

\makesubproblem{}{problem:vectorAreaGriffiths:1:d}

Show that 
\begin{dmath}\label{eqn:vectorAreaGriffiths:40}
\Ba = \inv{2} \ointctrclockwise \Br \cross d\Bl,
\end{dmath}

where the integral is around the boundary line.

\makesubproblem{}{problem:vectorAreaGriffiths:1:e}

Show that 
\begin{dmath}\label{eqn:vectorAreaGriffiths:60}
\ointctrclockwise \lr{ \Bc \cdot \Br } d\Bl = \Ba \cross \Bc.
\end{dmath}
} % problem

\makeanswer{problem:vectorAreaGriffiths:1}{
\makeSubAnswer{}{problem:vectorAreaGriffiths:1:a}

\begin{dmath}\label{eqn:vectorAreaGriffiths:80}
\Ba
=
\int_{0}^{\pi/2} R^2 \sin\theta d\theta \int_0^{2\pi} d\phi
\lr{ \sin\theta \cos\phi, \sin\theta \sin\phi, \cos\theta }
=
R^2 \int_{0}^{\pi/2} d\theta \int_0^{2\pi} d\phi
\lr{ \sin^2\theta \cos\phi, \sin^2\theta \sin\phi, \sin\theta\cos\theta }
=
2 \pi R^2 \int_{0}^{\pi/2} d\theta \Be_3 
\sin\theta\cos\theta 
=
\pi R^2 
\Be_3 
\int_{0}^{\pi/2} d\theta 
\sin(2 \theta)
=
\pi R^2 
\Be_3 
\evalrange{\lr{\frac{-\cos(2 \theta)}{2}}}{0}{\pi/2}
=
\pi R^2 
\Be_3 
\lr{ 1 - (-1) }/2
=
\pi R^2 
\Be_3.
\end{dmath}

\makeSubAnswer{}{problem:vectorAreaGriffiths:1:b}

As hinted in the original problem description, this follows from

\begin{dmath}\label{eqn:vectorAreaGriffiths:n}
\int dV \spacegrad T = \oint T d\Ba,
\end{dmath}

simply by setting \( T = 1 \).

\makeSubAnswer{}{problem:vectorAreaGriffiths:1:c}

The 

\makeSubAnswer{}{problem:vectorAreaGriffiths:1:d}
\makeSubAnswer{}{problem:vectorAreaGriffiths:1:e}
} % answer

%}
\EndArticle

%%%      %
% Copyright � 2016 Peeter Joot.  All Rights Reserved.
% Licenced as described in the file LICENSE under the root directory of this GIT repository.
%
\makeproblem{Tangential magnetic field boundary conditions.}{emt:problemSet3:1}{ 

In the class notes we showed that when there were no sources at the interface between two
media and neither of the two media was a perfect conductor \( \sigma_1, \sigma_2 \ne \infty \) the boundary condition
on the tangential magnetic field was given by

\begin{dmath}\label{eqn:emtProblemSet3Problem1:20}
\ncap \cross \lr{ \BH_2 - \BH_1 } = 0.
\end{dmath}

Here, show that when \( \BJ_i + \BJ_c = \BJ_{ic} \ne 0 \), the boundary condition is given by 

\begin{dmath}\label{eqn:emtProblemSet3Problem1:40}
\ncap \cross \lr{ \BH_2 - \BH_1 } = \BJ_s,
\end{dmath}

where
\begin{dmath}\label{eqn:emtProblemSet3Problem1:60}
\BJ_s = \lim_{\Delta y \rightarrow 0} \BJ_{ic} \Delta y.
\end{dmath}

Note: Use the geometry provided in 
\cref{fig:boundaryPs3:boundaryPs3Fig1}
for your proof.
\imageFigure{../../figures/ece1228-emt/boundaryPs3Fig1}{Boundary geometry.}{fig:boundaryPs3:boundaryPs3Fig1}{0.3}
} % makeproblem

\makeanswer{emt:problemSet3:1}{ 

Instead of integrating over a loop as done in class, a better way to tackle this problem is to integrate the curl over the same sort of pillbox that we use for deriving the boundary conditions from the divergence Maxwell's equations.

The form of Stokes' law that we want, following the notation of \citep{aMacdonaldVAGC}, is

\begin{dmath}\label{eqn:emtProblemSet3Problem1:80}
\int_V d^3 \Bx \cdot \lr{ \boldpartial \wedge \BA } = \oint_{\partial V} d^2 \Bx \cdot \BA.
\end{dmath}

The \R{3} translation of this relation into traditional vector algebra, after applying some duality relations, is

\begin{dmath}\label{eqn:emtProblemSet3Problem1:100}
\int_V dV \spacegrad \cross \BA = \oint_{\partial V} dA \ncap \cross \BA,
\end{dmath}

where \( \ncap \) is the outwards normal.  Proving the general multivector Stokes relationship is beyond the scope of this homework assignment, but we can validate
\cref{eqn:emtProblemSet3Problem1:100} by integrating the LHS over the infinitesimal rectangular prism sketched in

F1

\begin{dmath}\label{eqn:emtProblemSet3Problem1:120}
\begin{aligned}
\oint_{\partial V} dA \ncap \cross \BA
&=
\oint_{\partial V} dx dy \Be_3 \cross \lr{ \BA(z+) - \BA(z-) } \\
&+\oint_{\partial V} dy dz \Be_3 \cross \lr{ \BA(x+) - \BA(x-) } \\
&+\oint_{\partial V} dz dx \Be_3 \cross \lr{ \BA(Y+) - \BA(Y-) } \\
&=
\int_{V} dx dy \Be_3 \cross \lr{ dz \PD{z}{\BA} }
+\int_{V} dy dz \Be_3 \cross \lr{ dx \PD{x}{\BA} }
+\int_{V} dz dx \Be_3 \cross \lr{ dy \PD{y}{\BA} } \\
&=
\int_{V} dx dy dz \spacegrad \cross \BA.
\end{aligned}
\end{dmath}

Now, let's apply this to Ampere-Maxwell equation

\begin{dmath}\label{eqn:emtProblemSet3Problem1:140}
\spacegrad \cross \BH = \BJ_{ic} + \PD{t}{\BD},
\end{dmath}

where \( \BJ_{ic} = \BJ_s \delta(y) \).  We have

\begin{dmath}\label{eqn:emtProblemSet3Problem1:160}
\oint dA \ncap \cross \BH = \int dV \lr{ \BJ_s \delta(y) + \PD{t}{\BD} }.
\end{dmath}

Using the pillbox configuration of 

F2

and taking the pillbox volume to zero in the \( \Delta y \rightarrow 0 \) limit, the LHS integral has only contributions from the top and bottom faces of the pillbox, and the \( \BD \) term, which is assumed finite, will get killed.  That is

\begin{dmath}\label{eqn:emtProblemSet3Problem1:180}
\int dA \ncap \cross \lr{ \BH_2 - \BH_1 } = \int dA \BJ_s
\end{dmath}

Both sets of integrands can now be brought under one integral
\begin{dmath}\label{eqn:emtProblemSet3Problem1:200}
\int dA \lr{ \ncap \cross \lr{ \BH_2 - \BH_1 } - \int dA \BJ_s } = 0,
\end{dmath}

which proves the desired boundary relation

%\begin{dmath}\label{eqn:emtProblemSet3Problem1:220}
\boxedEquation{eqn:emtProblemSet3Problem1:240}{
\ncap \cross \lr{ \BH_2 - \BH_1 } - \BJ_s = 0.\qedmarker
}
%\end{dmath}

Unlike the procedure of \citep{balanis1989advanced} followed in class, this method of proof has the advantage of being coordinate free (except for having arbitrarily picked the y-axis as the normal direction).
}

%%%      %
% Copyright � 2016 Peeter Joot.  All Rights Reserved.
% Licenced as described in the file LICENSE under the root directory of this GIT repository.
%
\makeproblem{Magnetic field for a current loop.}{emt:problemSet3:2}{ 

A loop of wire located in x-y plane carrying current \(I\) is shown in \cref{fig:currentLoopPs3:currentLoopPs3Fig2}.
The loop's radius is \(R_l\).
\imageFigure{../../figures/ece1228-emt/currentLoopPs3Fig2}{Current loop.}{fig:currentLoopPs3:currentLoopPs3Fig2}{0.3}
\makesubproblem{}{emt:problemSet3:2a}
Calculate the magnetic field flux density, \( \BB \), along the loop axis at a distance \( z \) from its center.

\makesubproblem{}{emt:problemSet3:2b}
Simplify the results in 
\partref{emt:problemSet3:2a}
for large distances along the z-axis (\( z \gg R_l \)).

\makesubproblem{}{emt:problemSet3:2c}
Express the results in 
\partref{emt:problemSet3:2b}
in terms of magnetic dipole
moment. Make sure you write the expression in vector
form.
\makesubproblem{}{emt:problemSet3:2d}
In keeping with your understanding of magnetic bar's
north and south poles, designate the north and south poles
for the current carrying loop shown in the figure. 

\paragraph{Hint:} Use Biot-Savart law which states the following: A
differential current element, \( I d\Bl' \), produces a differential
magnetic field, \( d\BB \),
at a distance \( R \) from the current
element given by

\begin{dmath}\label{eqn:emtProblemSet3Problem2:20}
d\BB = \frac{\mu_0}{4 \pi} \frac{I d\Bl' \cross \BR }{R^3},
\end{dmath}

or
\begin{dmath}\label{eqn:emtProblemSet3Problem2:40}
\BB = \frac{\mu_0}{4 \pi} \int \frac{I d\Bl' \cross \BR }{R^3},
\end{dmath}

Note that integration is carried over the source (current) and \( R \) points from the current elements
to the point of observation. 
} % makeproblem

\makeanswer{emt:problemSet3:2}{ 
\makeSubAnswer{}{emt:problemSet3:2a}

TODO.
\makeSubAnswer{}{emt:problemSet3:2b}

TODO.
\makeSubAnswer{}{emt:problemSet3:2c}

TODO.
\makeSubAnswer{}{emt:problemSet3:2d}

TODO.
}

%%%      %
% Copyright � 2016 Peeter Joot.  All Rights Reserved.
% Licenced as described in the file LICENSE under the root directory of this GIT repository.
%
\makeproblem{Electric field across dielectric boundary.}{emt:problemSet3:3}{
\index{boundary conditions!electric field}
\index{dielectric!boundary conditions}
The plane \( 3x + 2y + z = 12 \) [\si{m}] describes the interface between a dielectric and free space.
The origin side of the interface has \( \epsilon_{r 1} = 3 \) and \( \BE_1 = 2 \xcap + 5 \zcap \) [\si{V/m}]. What is \(\BE_2\)
(the field on the other side of the interface)?
} % makeproblem

\makeanswer{emt:problemSet3:3}{

The geometry of the problem is sketched roughly in \cref{fig:ps3Problem3Plane:ps3Problem3PlaneFig1}.

\imageFigure{../../figures/ece1228-emt/ps3Problem3PlaneFig1}{Interfaces on sides of a plane.}{fig:ps3Problem3Plane:ps3Problem3PlaneFig1}{0.3}

Assuming that there are no sources, the relationships between the fields on each side of the interface are

\begin{dmath}\label{eqn:emtProblemSet3Problem3:20}
\begin{aligned}
%\ncap \cdot \lr{ \epsilon_0 \epsilon_{r2} \BE_2 - \epsilon_0 \epsilon_{r1} \BE_1 } &= 0
\ncap \cdot \lr{ \BD_2 - \BD_1 } &= 0 \\
\ncap \cross \lr{ \BE_2 - \BE_1 } &= 0
\end{aligned}
\end{dmath}

%Since \( \epsilon_{r2} = 1 \),
After cancelling common factors of \( \epsilon_0 \) the first relationship can be written

\begin{dmath}\label{eqn:emtProblemSet3Problem3:40}
\ncap \cdot \BE_2 = \frac{\epsilon_{r1}}{\epsilon_{r2}} \ncap \cdot \BE_1.
\end{dmath}

Adding the normal and the tangential components of \( \BE_2 \), we have

\begin{dmath}\label{eqn:emtProblemSet3Problem3:60}
\BE_2
=
\ncap \lr{ \ncap \cdot \BE_2 } -
\ncap \cross \lr{ \ncap \cross \BE_2 }
=
\frac{\epsilon_{r1}}{\epsilon_{r2}} \ncap \lr{ \ncap \cdot \BE_1 }
- \ncap \cross \lr{ \ncap \cross \BE_1 }.
\end{dmath}

By expanding the tangential projection (normal rejection) of a vector as

\begin{dmath}\label{eqn:emtProblemSet3Problem3:160}
\BA_t
=
- \ncap \cross \lr{ \ncap \cross \BA }
=
\BA - \ncap (\ncap \cdot \BA),
\end{dmath}

we find
\begin{dmath}\label{eqn:emtProblemSet3Problem3:180}
\BE_2
=
\frac{\epsilon_{r1}}{\epsilon_{r2}} \ncap \lr{ \ncap \cdot \BE_1 }
+ \lr{ \BE_1 - \ncap (\ncap \cdot \BE_1) }
=
\BE_1 + \lr{\frac{\epsilon_{r1}}{\epsilon_{r2}} -1} \ncap \lr{ \ncap \cdot \BE_1 },
\end{dmath}

or
\boxedEquation{eqn:emtProblemSet3Problem3:320}{
\BE_2
=
\BE_1 + \frac{\frac{\epsilon_{r1}}{\epsilon_{r2}} -1}{\Abs{\Bn}^2} \Bn \lr{ \Bn \cdot \BE_1 }.
}

The rest of the problem is routine algebra.

\begin{subequations}
\label{eqn:emtProblemSet3Problem3:220}
\begin{dmath}\label{eqn:emtProblemSet3Problem3:240}
\Bn^2 = (3,2,1) \cdot (3,2,1) = 9 + 4 + 1 = 14,
\end{dmath}
\begin{dmath}\label{eqn:emtProblemSet3Problem3:260}
\Bn \cdot \BE_1
=
(3,2,1) \cdot (2,0,5)
=
6+5
=11,
\end{dmath}
\end{subequations}

so
\begin{dmath}\label{eqn:emtProblemSet3Problem3:280}
\BE_2
=
(2,0,5) + \frac{2 \times 11}{14} (3,2,1)
=
\inv{7}( (14,0,35) + (33,22,11) ),
\end{dmath}

which is

\boxedEquation{eqn:emtProblemSet3Problem3:300}{
\BE_2
=
\frac{47}{7} \xcap + \frac{22}{7} \ycap + \frac{46}{7} \zcap
\qquad[\si{V/m}].
}
}

%%%      %
% Copyright � 2016 Peeter Joot.  All Rights Reserved.
% Licenced as described in the file LICENSE under the root directory of this GIT repository.
%
\makeproblem{Laplacian form of delta function.}{emt:problemSet3:4}{ 
Prove that

\begin{dmath}\label{eqn:emtProblemSet3Problem4:20}
-\spacegrad^2 \inv{r} = 4 \pi \delta^3(\Br),
\end{dmath}

where \( r = \Abs{\Br} \) is the position vector.
} % makeproblem

\makeanswer{emt:problemSet3:4}{ 

TODO.
}

      %\section{Appendix I.  Current loop integral off axis.}

Initially I was curious what the current loop magnetic field integral would look like in general, allowing for an off axis observation point.

I found it natural to do that compuation using Geometric Algebra to express vector rotation in a plane and the other geometrical constructs of this problem.  The basic rules in that Algebra are that unit vectors square to unity (\(\Be_k^2 = 1 \)), and that orthogonal vectors anticommute (\( \Be_1 \Be_2 = -\Be_2 \Be_1 \)).  For example, letting \( i = \Be_1 \Be_2 \) the radial unit vector can be expessed as

\begin{dmath}\label{eqn:emtProblemSet3Problem2:160}
\rhocap(\theta)
=
\Be_1 e^{i \theta}
= \Be_1 \lr{ \cos\theta + \Be_1 \Be_2 \sin\theta } 
= \Be_1 \cos\theta + (\Be_1^2) \Be_2 \sin\theta
= \Be_1 \cos\theta + \Be_2 \sin\theta,
\end{dmath}

and the \( \thetacap \) direction vector is
\begin{dmath}\label{eqn:emtProblemSet3Appendix:n}
\thetacap(\theta)
=
\Be_2 e^{i \theta}
= \Be_2 \lr{ \cos\theta + \Be_1 \Be_2 \sin\theta } 
= \Be_2 \cos\theta + \Be_2 \Be_1 \Be_2 \sin\theta
= \Be_2 \cos\theta + \Be_2 (-\Be_2 \Be_1) \sin\theta
= \Be_2 \cos\theta - \Be_1 \sin\theta.
\end{dmath}

This allows for a compact expression of an off-axis observation point

\begin{dmath}\label{eqn:emtProblemSet3Problem2:60}
\Br = z \Be_3 + \rho \Be_1 e^{i\theta}.
\end{dmath}

Similarly, the charge point is
\begin{dmath}\label{eqn:emtProblemSet3Problem2:80}
\Br' = R_l \Be_1 e^{i \theta'},
\end{dmath}

and the element of the loop path is
\begin{dmath}\label{eqn:emtProblemSet3Problem2:100}
d\Bl' = R_l \Be_2 e^{i\theta'} d\theta'.
\end{dmath}

The difference vector from the charge position to the observation point is

\begin{dmath}\label{eqn:emtProblemSet3Problem2:120}
\BR 
= \Br - \Br'
=
z \Be_3 + \rho \Be_1 e^{i\theta}
-
R_l \Be_1 e^{i \theta'},
\end{dmath}

with squared length

\begin{dmath}\label{eqn:emtProblemSet3Problem2:140}
\BR^2 
=
z^2 + 
\lr{ \rho \Be_1 e^{i\theta}
-
R_l \Be_1 e^{i \theta'}
}
\cdot
\lr{ \rho \Be_1 e^{i\theta}
-
R_l \Be_1 e^{i \theta'}
}
=
z^2 + \rho^2 + R_l^2 - 2 \rho R_l \cos\lr{ \theta - \theta' }.
\end{dmath}

For the cross product, using a bivector duality transformation \( \Ba \cross \Bb = -\Be_1 \Be_2 \Be_3 (\Ba \wedge \Bb) \), and expressing the wedge product as a grade two selection, we have

\begin{dmath}\label{eqn:emtProblemSet3Problem2:180}
d\Bl' \cross \BR 
=
-\Be_1 \Be_2 \Be_3 R_l d\theta' \gpgradetwo{ 
\Be_2 e^{i \theta'} 
\lr{
z \Be_3 + \rho \Be_1 e^{i\theta}
-
R_l \Be_1 e^{i \theta'}
}
}
=
R_l d\theta' \lr{ 
z \Be_1 e^{i\theta'}
-
\Be_3 \rho \cos\lr{ \theta - \theta' }
+ \Be_3 R_l
}.
\end{dmath}

The final integral can now be assembled

\boxedEquation{eqn:emtProblemSet3Appendix:220}{
%\begin{dmath}\label{eqn:emtProblemSet3Problem2:200}
\BB = \frac{I \mu_0 R_l}{4\pi} \int_0^{2\pi} d\theta' 
\frac
{ z \Be_1 e^{i\theta'} - \Be_3 \rho \cos\lr{ \theta - \theta' } + \Be_3 R_l }
{ \lr{z^2 + \rho^2 + R_l^2 - 2 \rho R_l \cos\lr{ \theta - \theta' }}^{3/2} }.
%\end{dmath}
}

This is consistent with the traditional vector algebra derivation that led to \cref{eqn:emtProblemSet3Problem2:201} where \( \rho = 0 \) was assumed.
It is clear now, why the problem statement asked only to consider the z-axis observation points where \( \rho = 0 \).  With \( \theta' \) dependencies in the denominator, performing the integral above for \( \rho \ne 0 \) looks spectacularly unpleasant.

\section{Appendix II.  Normal and tangential decomposition.}

The decomposition of \cref{eqn:emtProblemSet3Problem3:60} can be derived easily using Geometric Algebra

\begin{dmath}\label{eqn:emtProblemSet3Problem3:80}
\BA 
= 
\ncap^2 \BA
=
\ncap (\ncap \cdot \BA)
+\ncap (\ncap \wedge \BA)
%=
%\ncap (\ncap \cdot \BA)
%+
%\ncap \cdot (\ncap \wedge \BA)
\end{dmath}

The last dot product can be expanded as a grade one (vector) selection

\begin{dmath}\label{eqn:emtProblemSet3Problem3:100}
\ncap (\ncap \wedge \BA)
=
\gpgradeone{
\ncap (\ncap \wedge \BA)
}
=
\gpgradeone{
\ncap I (\ncap \cross \BA)
}
=
I^2 \ncap \cross (\ncap \cross \BA)
=
- \ncap \cross (\ncap \cross \BA),
\end{dmath}

so the decomposition of a vector \( \BA \) in terms of its normal and tangential projections is
\begin{dmath}\label{eqn:emtProblemSet3Problem3:120}
\BA
=
\ncap (\ncap \cdot \BA)
-
\ncap \cross (\ncap \cross \BA).
\end{dmath}

I'm not sure how to derive this easily using traditional vector algebra, but it can be verified by expanding the triple cross product in coordinates using tensor contraction formalism

\begin{dmath}\label{eqn:emtProblemSet3Problem3:140}
-\ncap \cross (\ncap \cross \BA)
=
-\epsilon_{xyz} \Be_x n_y \lr{\ncap \cross \BA}_z
=
-\epsilon_{xyz} \Be_x n_y \epsilon_{zrs} n_r A_s
=
-\delta_{xy}^{[rs]}
\Be_x n_y n_r A_s
=
-\Be_x n_y \lr{ n_x A_y -n_y A_x }
= -\ncap (\ncap \cdot \BA) + (\ncap \cdot \ncap) \BA
= \BA - \ncap (\ncap \cdot \BA).
\end{dmath}

This last statement illustrates the geometry of this decomposition, showing that the tangential projection (or normal rejection) of a vector is really just the vector minus its normal projection.

%This can be rearranged to show that the 
%\begin{dmath}\label{eqn:emtProblemSet3Problem3:100}

%The tangential projection, can also be expanded in dot products
%
%\begin{dmath}\label{eqn:emtProblemSet3Problem3:200}
%\ncap (\ncap \wedge \BA)
%=
%\ncap \cdot (\ncap \wedge \BA)
%=
%\BA - \ncap (\ncap \cdot \BA)
%\end{dmath}

%%%      %
% Copyright � 2016 Peeter Joot.  All Rights Reserved.
% Licenced as described in the file LICENSE under the root directory of this GIT repository.
%
\makeproblem{Conductor charge distribution on surface.}{emt:problemSet4:4}{ 
We have stated that the boundary condition for a perfect conductor is such that there is
no electric field or charge distribution inside of the conductor. Here we will study the
dynamics of this process. Start with continuity equation
\( \spacegrad \cdot \BJ = -\PDi{t}{\rho} \), where \( \BJ \) 
is the
current density [\si{A/m^2}] and \( \rho \) is the charge density [\si{C/m^3}]. Show that a charge (charge
density) placed inside a conductor will decay in an exponential manner.
} % makeproblem

\makeanswer{emt:problemSet4:4}{ 

TODO.
}


   %
% Copyright � 2016 Peeter Joot.  All Rights Reserved.
% Licenced as described in the file LICENSE under the root directory of this GIT repository.
%
%\newcommand{\authorname}{Peeter Joot}
\newcommand{\email}{peeterjoot@protonmail.com}
\newcommand{\basename}{FIXMEbasenameUndefined}
\newcommand{\dirname}{notes/FIXMEdirnameUndefined/}

%\renewcommand{\basename}{emt5}
%\renewcommand{\dirname}{notes/ece1228/}
%\newcommand{\keywords}{ECE1228H}
%\newcommand{\authorname}{Peeter Joot}
\newcommand{\onlineurl}{http://sites.google.com/site/peeterjoot2/math2013/\basename.pdf}
\newcommand{\sourcepath}{\dirname\basename.tex}
\newcommand{\generatetitle}[1]{\chapter{#1}}

\newcommand{\vcsinfo}{%
\section*{}
\noindent{\color{DarkOliveGreen}{\rule{\linewidth}{0.1mm}}}
\paragraph{Document version}
%\paragraph{\color{Maroon}{Document version}}
{
\small
\begin{itemize}
\item Available online at:\\ 
\href{\onlineurl}{\onlineurl}
\item Git Repository: \input{./.revinfo/gitRepo.tex}
\item Source: \sourcepath
\item last commit: \input{./.revinfo/gitCommitString.tex}
\item commit date: \input{./.revinfo/gitCommitDate.tex}
\end{itemize}
}
}

%\PassOptionsToPackage{dvipsnames,svgnames}{xcolor}
\PassOptionsToPackage{square,numbers}{natbib}
\documentclass{scrreprt}

\usepackage[left=2cm,right=2cm]{geometry}
\usepackage[svgnames]{xcolor}
\usepackage{peeters_layout}

\usepackage{natbib}

\usepackage[
colorlinks=true,
bookmarks=false,
pdfauthor={\authorname, \email},
backref 
]{hyperref}

% http://tex.stackexchange.com/questions/75773/how-to-reference-problems-by-the-text-label-in-an-exercise-envioronment
\usepackage[english]{cleveref}
\crefname{Exercise}{exercise}{exercises}
\Crefname{Exercise}{Exercise}{Exercises}

\RequirePackage{titlesec}
\RequirePackage{ifthen}

% http://stackoverflow.com/questions/4932910/date-in-the-tabular-environment
\makeatletter
\let\insertdate\@date
\makeatother

\titleformat{\chapter}[display]
{\bfseries\Large}
{\color{DarkSlateGrey}\filleft \authorname
\ifthenelse{\isundefined{\studentnumber}}{}{\\ \studentnumber}
\ifthenelse{\isundefined{\email}}{}{\\ \email}
\ifthenelse{\isundefined{\dateintitle}}{}{\\ \insertdate}
%\ifthenelse{\isundefined{\coursename}}{}{\\ \coursename} % put in title instead.
}
{4ex}
{\color{DarkOliveGreen}{\titlerule}\color{Maroon}
\vspace{2ex}%
\filright}
[\vspace{2ex}%
\color{DarkOliveGreen}\titlerule
]

\newcommand{\beginArtWithToc}[0]{\begin{document}\tableofcontents}
\newcommand{\beginArtNoToc}[0]{\begin{document}}
\newcommand{\EndNoBibArticle}[0]{\end{document}}
\newcommand{\EndArticle}[0]{\bibliography{Bibliography}\bibliographystyle{plainnat}\end{document}}

% 
%\newcommand{\citep}[1]{\cite{#1}}

\colorSectionsForArticle


%
%%\usepackage{ece1228}
%\usepackage{peeters_braket}
%%\usepackage{peeters_layout_exercise}
%\usepackage{peeters_figures}
%\usepackage{macros_cal}
%\usepackage{macros_bm}
%\usepackage{mathtools}
%\usepackage{siunitx}
%
%\beginArtNoToc
%\generatetitle{ECE1228H Electromagnetic Theory.  Lecture 5: Poynting vector.  Taught by Prof.\ M. Mojahedi}
\chapter{Poynting vector, and time harmonic (phasor) fields.}
\label{chap:emt5}

\paragraph{Poynting}

The cross product terms of Maxwell's equation are
\begin{equation}\label{eqn:emtLecture5:120}
\spacegrad \cross \BE 
= -\BM_i - \PD{t}{\BB}
= -\BM_i - \BM_d,
\end{equation}

where \(\BM_d\) is called the magnetic displacement current here.  For the magnetic curl we have

\begin{equation}\label{eqn:emtLecture5:140}
\spacegrad \cross \BH 
= \BJ_i + \BJ_c + \PD{t}{\BD}
= \BJ_i + \BJ_c + \BJ_d.
\end{equation}

From this (HW) we will show that 
\begin{dmath}\label{eqn:emtLecture5:160}
\spacegrad \cdot \lr{ \BE \cross \BH } + \BH \cdot \lr{ \BM_i + \BM_d }  + \BE \cdot \lr{ \BJ_i + \BJ_c + \BJ_d } = 0,
\end{dmath}

or
\begin{dmath}\label{eqn:emtLecture5:180}
\oint d\Ba \cdot \lr{ \BE \cross \BH } + \int dV \lr{ \BH \cdot \lr{ \BM_i + \BM_d }  + \BE \cdot \lr{ \BJ_i + \BJ_c + \BJ_d }} = 0,
\end{dmath}

or
\begin{dmath}\label{eqn:emtLecture5:200}
\oint d\Ba \cdot \lr{ \BE \cross \BH } 
+ \int dV \BH \cdot \BM_i
+ \int dV \BE \cdot \BJ_i
+ \int dV \BE \cdot \BJ_c
+ \int dV \lr{ \BH \cdot \PD{t}{\BB} + \BE \cdot \PD{t}{\BD} } = 0.
\end{dmath}

Define a supplied power density \( \rho_{\textrm{supp}} \)

\begin{dmath}\label{eqn:emtLecture5:220}
-\rho_{\textrm{supp}}
=
 \int dV \BH \cdot \BM_i
+ \int dV \BE \cdot \BJ_i.
\end{dmath}

When the medium is not dispersive or lossy, we have

\begin{dmath}\label{eqn:emtLecture5:240}
\int dV \BH \cdot \PD{t}{\BB} 
=
\mu \int dV \BH \cdot \PD{t}{\BH} 
=
\PD{t}{} \int dV \mu \Abs{\BH}^2.
\end{dmath}

The units of \( [\mu \Abs{\BH}^2] \) are \si{W}, so one can defined a magnetic energy density \( \mu \Abs{\BH}^2 \), and

\begin{dmath}\label{eqn:emtLecture5:260}
W_m = 
\int dV \mu \Abs{\BH}^2,
\end{dmath}

for

\begin{dmath}\label{eqn:emtLecture5:280}
\int dV \BH \cdot \PD{t}{\BB} 
=
\PD{t}{W_m}.
\end{dmath}

This is the rate of change of stored magnetic energy [\si{J/s} = \si{W}].

Similarly
\begin{dmath}\label{eqn:emtLecture5:300}
\int dV \BE \cdot \PD{t}{\BD} 
=
\epsilon
\int dV \BE \cdot \PD{t}{\BE} 
=
\PD{t}{} \int dV \epsilon \Abs{\BE}^2.
\end{dmath}

The electric energy density is \( \epsilon \Abs{\BE}^2 \).  Let

\begin{dmath}\label{eqn:emtLecture5:320}
W_e = 
\int dV \epsilon \Abs{\BE}^2,
\end{dmath}

and
\begin{dmath}\label{eqn:emtLecture5:340}
\int dV \BE \cdot \PD{t}{\BD} 
=
\PD{t}{W_e}.
\end{dmath}

We also have a term

\begin{dmath}\label{eqn:emtLecture5:360}
\int dV \BE \cdot \BJ_c 
=
\int dV \BE \cdot (\sigma \BE)
=
\int dV \sigma \Abs{\BE}^2
%\equiv ...
\end{dmath}

This is the rate of change of stored electric energy.

The remaining term is
\begin{dmath}\label{eqn:emtLecture5:380}
\oint d\Ba \cdot \lr{ \BE \cross \BH }
\end{dmath}

This is a density of the power that is leaving the volume.  The vector \( \BE \cross \BH \) is special, called the Poynting vector, and coincidentally points in the direction that the energy leaves the bounding surface per unit time.  We write

\begin{dmath}\label{eqn:emtLecture5:400}
\BS = \BE \cross \BH.
\end{dmath}

In vacuum the phase velocity \( \Bv_p \), group velocity \( \Bv_g \) and packet(?) velocity \( \Bv_p \) all line up.  This isn't the case in the media.

It turns out that without dissipation 

\begin{dmath}\label{eqn:emtLecture5:420}
\int \BH \cdot \PD{t}{\BB} = \int \BE \cdot \PD{t}{\BD}.
\end{dmath}

For example in an LC circuit \cref{fig:lecture4LCCircuit:lecture4LCCircuitFig1}
half the cycle the energy is stored in the inductor, and in the other half of the cycle the energy is stored in the capacitor.

\imageFigure{../../figures/ece1228-emt/lecture4LCCircuitFig1}{LC circuit.}{fig:lecture4LCCircuit:lecture4LCCircuitFig1}{0.2}

Summarizing

\begin{dmath}\label{eqn:emtLecture5:440}
\oint \lr{ \BE \cross \BH } \cdot d\Ba = P_{\textrm{exit}}.
\end{dmath}

\paragraph{Time harmonics}

Recall that we have differential equations to solve for each type of circuit element in the time domain.  For example in \cref{fig:lecture4inductor:lecture4inductorFig2a}, we have

\begin{dmath}\label{eqn:emtLecture5:980}
V_i(t) = L \ddt{i},
\end{dmath}

\imageFigure{../../figures/ece1228-emt/lecture4inductorFig2a}{Inductor.}{fig:lecture4inductor:lecture4inductorFig2a}{0.2}

and for the capacitor sketched in \cref{fig:lecture4cap:lecture4capFig2b}, we have
\begin{dmath}\label{eqn:emtLecture5:1000}
i_c(t) = C \ddt{V_c}.
\end{dmath}

\imageFigure{../../figures/ece1228-emt/lecture4capFig2b}{Capacitor.}{fig:lecture4cap:lecture4capFig2b}{0.2}

When we use Laplace or Fourier techniques to solve circuits with such differential equation elements.  The price that we paid for that was that we have to start dealing with complex-valued (phasor) quantities.  We can do this for field equations as well.  The goal is to remove the time domain coupling in Maxwell equations like

\begin{dmath}\label{eqn:emtLecture5:460}
\spacegrad \cross \BE(\Br, t) = -\PD{t}{\BB}(\Br, t)
\end{dmath}
\begin{dmath}\label{eqn:emtLecture5:480}
\spacegrad \cross \BH(\Br, t) = \sigma \BE + \PD{t}{\BD}(\Br, t).
\end{dmath}

For a single frequency, assume that the time dependency can be written as

\begin{dmath}\label{eqn:emtLecture5:500}
\BE(\Br, t) = \Real \lr{ \BE^\conj(\Br) e^{j \omega t} }.
\end{dmath}

We may now have to require \( \BE(\Br) \) to be complex valued.
We also have to be really careful about which convention of the time domain solution we are going to use, since we could just as easily use

\begin{dmath}\label{eqn:emtLecture5:720}
\BE(\Br, t) = \Real \lr{ \BE(\Br) e^{-j \omega t} }.
\end{dmath}

For example 
\begin{dmath}\label{eqn:emtLecture5:840}
\Real( e^{i k z} e^{-i\omega t} ) = \cos( k z - \omega t ),
\end{dmath}

is identical with
\begin{dmath}\label{eqn:emtLecture5:860}
\Real( e^{-j k z} e^{j\omega t} ) = \cos( \omega t -k z),
\end{dmath}

showing that a solution or its complex conjugate is equally valid.

Engineering books use \( e^{j \omega t} \) whereas most physicists use \( e^{-i \omega t } \).

What if we have more complex time dependencies, such as that sketched in \cref{fig:lecture4NonSine:lecture4NonSineFig3}?

\imageFigure{../../figures/ece1228-emt/lecture4NonSineFig3}{Non-sinusoidal time dependence.}{fig:lecture4NonSine:lecture4NonSineFig3}{0.2}

We can do this using Fourier superposition, adding a finite or infinite set of single frequency solutions.  The first order of business is to solve the system for a single frequency.

Let's write our Fourier transform pairs as
\begin{subequations}
\label{eqn:emtLecture5:520}
\begin{equation}\label{eqn:emtLecture5:540}
\calF(\BA(\Br, t)) = 
\BA(\Br, \omega)
=
\int_{-\infty}^\infty \BA(\Br, t) e^{-j \omega t} dt
\end{equation}
\begin{equation}\label{eqn:emtLecture5:560}
\BA(\Br, t) = \calF^{-1}(\BA(\Br, \omega))
=
\inv{2\pi} 
\int_{-\infty}^\infty \BA(\Br, \omega) e^{j \omega t} d\omega.
\end{equation}
\end{subequations}

In particular

\begin{equation}\label{eqn:emtLecture5:580}
\calF\lr{ \ddt{f(t)} } = j \omega F(\omega),
\end{equation}

so the Fourier transform of the Maxwell equation
\begin{dmath}\label{eqn:emtLecture5:600}
\calF\lr{ \spacegrad \cross \BE(\Br, t) }
=
\calF\lr{ -\PD{t}{\BB}(\Br, t) },
\end{dmath}

is

\begin{dmath}\label{eqn:emtLecture5:620}
\spacegrad \cross \BE(\Br, \omega) = - j\omega \BB(\Br, \omega).
\end{dmath}

The four Maxwell's equations can be written as

\begin{itemize}
\item Faraday's Law
\begin{dmath}\label{eqn:emtLecture5:640}
\spacegrad \cross \BE( \Br, \omega ) = - j \omega \BB(\Br, \omega) - \BM_i
\end{dmath}
\item Ampere-Maxwell equation
\begin{dmath}\label{eqn:emtLecture5:660}
\spacegrad \cross \BH( \Br, \omega ) = \BJ_\txtc(\Br, \omega) + \BD(\Br, \omega)
\end{dmath}
\item Gauss's law
\begin{dmath}\label{eqn:emtLecture5:680}
\spacegrad \cdot \BD(\Br, \omega) = \rho_{\txte\txtv}(\Br, \omega)
\end{dmath}
\item Gauss's law for magnetism
\begin{dmath}\label{eqn:emtLecture5:700}
\spacegrad \cdot \BB(\Br, \omega) = \rho_{\txtm\txtv}(\Br, \omega).
\end{dmath}
\end{itemize}

Now we can more easily model non-simple media with

\begin{dmath}\label{eqn:emtLecture5:740}
\begin{aligned}
\BB(\Br, \omega) &= \mu(\omega) \BH(\Br, \omega) \\
\BD(\Br, \omega) &= \epsilon(\omega) \BE(\Br, \omega).
\end{aligned}
\end{dmath}

so Maxwell's equations are

\begin{dmath}\label{eqn:emtLecture5:760}
\spacegrad \cross \BE( \Br, \omega ) = - j \omega \mu(\omega) \BH(\Br, \omega) - \BM_i
\end{dmath}
\begin{dmath}\label{eqn:emtLecture5:780}
\spacegrad \cross \BH( \Br, \omega ) = \BJ_\txtc(\Br, \omega) + \epsilon(\omega) \BE(\Br, \omega)
\end{dmath}
\begin{dmath}\label{eqn:emtLecture5:800}
\epsilon(\omega) \spacegrad \cdot \BE(\Br, \omega) = \rho_{\txte\txtv}(\Br, \omega)
\end{dmath}
\begin{dmath}\label{eqn:emtLecture5:820}
\mu(\omega) \spacegrad \cdot \BH(\Br, \omega) = \rho_{\txtm\txtv}(\Br, \omega).
\end{dmath}

\paragraph{Frequency domain Poynting}

The frequency domain (time harmonic) equivalent of the instantaneous Poynting theorem is

\begin{dmath}\label{eqn:emtLecture5:880}
\inv{2} \oint d\Ba \cdot \lr{ \BE \cross \BH^\conj } 
- \inv{2} \int dV \lr{ \BH^\conj \cdot \BM_i + \BE \cdot \BJ_i^\conj }
+ \inv{2} \int dV \sigma \Abs{\BE}^2
+ j \omega \inv{2} \int dV \lr{ \mu \Abs{\BH}^2 - \epsilon \Abs{\BE}^2 } = 0.
\end{dmath}

Showing this will probably be given as homework.

Since

\begin{dmath}\label{eqn:emtLecture5:900}
\Real(\BA) \cross \Real(\BB) \ne \Real( \BA \cross \BB ).
\end{dmath}

We want to find the instantaneous Poynting vector in terms of the phasor fields.  Following
\citep{balanis1989advanced}, where script is used for the instantaneous quantities and non-script for the phasors, we find

\begin{dmath}\label{eqn:emtLecture5:920}
\bcS(\Br, t) 
= \bcE(\Br, t) \cross \bcH(\Br, t)
= \Real(\bcE(\Br, t)) \cross \Real(\bcH(\Br, t))
= 
\frac{ \BE e^{j\omega t} + \BE^\conj e^{-j \omega t}}{2}
\cross
\frac{ \BH e^{j\omega t} + \BH^\conj e^{-j \omega t}}{2}
=
\inv{4}
\lr{
\BE \cross \BH^\conj + \BE^\conj \cross \BH
+ 
\BE \cross \BH e^{2 j\omega t} 
+ 
\BH \cross \BE e^{-2 j\omega t} 
}
=
\inv{2} \Real(\BE \cross \BH^\conj) + \inv{2} \Real( \BE \cross \BH  e^{2 j\omega t} ).
\end{dmath}

Should we time average over a period \( \expectation{.} = (1/T) \int_0^T (.) \) the second term is killed, so that

\begin{dmath}\label{eqn:emtLecture5:940}
\expectation{ \bcS }
=
\inv{2} \Real(\BE \cross \BH^\conj) + \inv{2} \Real( \BE \cross \BH  e^{2 j\omega t} ).
\end{dmath}

The instantaneous Poynting vector is thus
\begin{dmath}\label{eqn:emtLecture5:960}
\bcS(\Br, t) = \expectation{\BS} + \inv{2} \Real\lr{ \BE \cross \BH e^{j \omega t} }.
\end{dmath}

%\EndArticle

      \section{Problems}

%%%      %
% Copyright � 2016 Peeter Joot.  All Rights Reserved.
% Licenced as described in the file LICENSE under the root directory of this GIT repository.
%
\makeproblem{Index of refraction.}{emt:problemSet4:1}{ 
Transmitter \( T \) of a time-harmonic wave of frequency \( \nu \) moves with velocity \( \BU \)
at
an angle \( \theta \) relative to the direct line to a stationary receiver \( R \), as sketched in
\cref{fig:ps4:ps4Fig1}.
\imageFigure{../../figures/ece1228-emt/ps4Fig1}{Field refraction.}{fig:ps4:ps4Fig1}{0.3}
\makesubproblem{}{emt:problemSet4:1a}
Derive the expression for the frequency detected by the receiver \(R\), assuming that the
medium between \(T\) and \(R\) has a positive index of refraction \(n\). (Apply the appropriate
approximations.)

\makesubproblem{}{emt:problemSet4:1b}
How is the expression obtained in 
\partref{emt:problemSet4:1a}
is modified if the medium is a metamaterial
with negative index of refraction.
\makesubproblem{}{emt:problemSet4:1c}
From the physical point of view, how is the situation in 
\partref{emt:problemSet4:1b}
different from 
\partref{emt:problemSet4:1a}
?
} % makeproblem

\makeanswer{emt:problemSet4:1}{ 
\makeSubAnswer{}{emt:problemSet4:1a}

TODO.
\makeSubAnswer{}{emt:problemSet4:1b}

TODO.
\makeSubAnswer{}{emt:problemSet4:1c}

TODO.
}

%%%      %
% Copyright � 2016 Peeter Joot.  All Rights Reserved.
% Licenced as described in the file LICENSE under the root directory of this GIT repository.
%
\makeproblem{description}{emt:problemSet4:2}{ 
Prove that if \( \Real\lr{ \BA(z) e^{j \omega t}} = \Real\lr{ \BB(z) e^{j \omega t}} \), then 
\( \BA(z) = \BB(z) \).
This means that the \( \Real() \)
operator can be removed on phasors of the same frequency.
} % makeproblem

\makeanswer{emt:problemSet4:2}{ 

TODO.
}

%%%      %
% Copyright � 2016 Peeter Joot.  All Rights Reserved.
% Licenced as described in the file LICENSE under the root directory of this GIT repository.
%
\makeproblem{Duality theorem.}{emt:problemSet4:3}{ 
Prove that if the time-harmonic fields \( \BE(\Br) \) and \( \BH(\Br) \)
are solutions to Maxwell's
equations in a simple, source free medium ( \( \BM_i = \BJ_i = \BJ_c = 0, \rho_{mv} = \rho_{ev} = 0 \) ),
characterized by \( \epsilon, \mu \) ; then
\( \BE'(\Br) = \eta \BH(\Br) \) and
\( \BH'(\Br) = -\frac{\BE(\Br)}{\eta} \)
are also solutions of
the Maxwell equations. 
\( \eta \)
is the intrinsic impedance of the medium.
\paragraph{Remark}: By showing the above you have proved the validity of the so called duality
theorem.
} % makeproblem

\makeanswer{emt:problemSet4:3}{ 

TODO.
}

      %\documentclass{article}

\usepackage{amsmath}
\usepackage{mathpazo}

%
% shorthand for bold symbols, convenient for vectors and matrices
%
\newcommand{\Ba}[0]{\mathbf{a}}
\newcommand{\Bb}[0]{\mathbf{b}}
\newcommand{\Bc}[0]{\mathbf{c}}
\newcommand{\Bd}[0]{\mathbf{d}}
\newcommand{\Be}[0]{\mathbf{e}}
\newcommand{\Bf}[0]{\mathbf{f}}
\newcommand{\Bg}[0]{\mathbf{g}}
\newcommand{\Bh}[0]{\mathbf{h}}
\newcommand{\Bi}[0]{\mathbf{i}}
\newcommand{\Bj}[0]{\mathbf{j}}
\newcommand{\Bk}[0]{\mathbf{k}}
\newcommand{\Bl}[0]{\mathbf{l}}
\newcommand{\Bm}[0]{\mathbf{m}}
\newcommand{\Bn}[0]{\mathbf{n}}
\newcommand{\Bo}[0]{\mathbf{o}}
\newcommand{\Bp}[0]{\mathbf{p}}
\newcommand{\Bq}[0]{\mathbf{q}}
\newcommand{\Br}[0]{\mathbf{r}}
\newcommand{\Bs}[0]{\mathbf{s}}
\newcommand{\Bt}[0]{\mathbf{t}}
\newcommand{\Bu}[0]{\mathbf{u}}
\newcommand{\Bv}[0]{\mathbf{v}}
\newcommand{\Bw}[0]{\mathbf{w}}
\newcommand{\Bx}[0]{\mathbf{x}}
\newcommand{\By}[0]{\mathbf{y}}
\newcommand{\Bz}[0]{\mathbf{z}}
\newcommand{\BA}[0]{\mathbf{A}}
\newcommand{\BB}[0]{\mathbf{B}}
\newcommand{\BC}[0]{\mathbf{C}}
\newcommand{\BD}[0]{\mathbf{D}}
\newcommand{\BE}[0]{\mathbf{E}}
\newcommand{\BF}[0]{\mathbf{F}}
\newcommand{\BG}[0]{\mathbf{G}}
\newcommand{\BH}[0]{\mathbf{H}}
\newcommand{\BI}[0]{\mathbf{I}}
\newcommand{\BJ}[0]{\mathbf{J}}
\newcommand{\BK}[0]{\mathbf{K}}
\newcommand{\BL}[0]{\mathbf{L}}
\newcommand{\BM}[0]{\mathbf{M}}
\newcommand{\BN}[0]{\mathbf{N}}
\newcommand{\BO}[0]{\mathbf{O}}
\newcommand{\BP}[0]{\mathbf{P}}
\newcommand{\BQ}[0]{\mathbf{Q}}
\newcommand{\BR}[0]{\mathbf{R}}
\newcommand{\BS}[0]{\mathbf{S}}
\newcommand{\BT}[0]{\mathbf{T}}
\newcommand{\BU}[0]{\mathbf{U}}
\newcommand{\BV}[0]{\mathbf{V}}
\newcommand{\BW}[0]{\mathbf{W}}
\newcommand{\BX}[0]{\mathbf{X}}
\newcommand{\BY}[0]{\mathbf{Y}}
\newcommand{\BZ}[0]{\mathbf{Z}}

\newcommand{\Bzero}[0]{\mathbf{0}}
\newcommand{\Btheta}[0]{\boldsymbol{\theta}}
\newcommand{\Btau}[0]{\boldsymbol{\tau}}
\newcommand{\Bomega}[0]{\boldsymbol{\omega}}

%
% shorthand for unit vectors
%
\newcommand{\acap}[0]{\hat{\Ba}}
\newcommand{\bcap}[0]{\hat{\Bb}}
\newcommand{\ccap}[0]{\hat{\Bc}}
\newcommand{\dcap}[0]{\hat{\Bd}}
\newcommand{\ecap}[0]{\hat{\Be}}
\newcommand{\fcap}[0]{\hat{\Bf}}
\newcommand{\gcap}[0]{\hat{\Bg}}
\newcommand{\hcap}[0]{\hat{\Bh}}
\newcommand{\icap}[0]{\hat{\Bi}}
\newcommand{\jcap}[0]{\hat{\Bj}}
\newcommand{\kcap}[0]{\hat{\Bk}}
\newcommand{\lcap}[0]{\hat{\Bl}}
\newcommand{\mcap}[0]{\hat{\Bm}}
\newcommand{\ncap}[0]{\hat{\Bn}}
\newcommand{\ocap}[0]{\hat{\Bo}}
\newcommand{\pcap}[0]{\hat{\Bp}}
\newcommand{\qcap}[0]{\hat{\Bq}}
\newcommand{\rcap}[0]{\hat{\Br}}
\newcommand{\scap}[0]{\hat{\Bs}}
\newcommand{\tcap}[0]{\hat{\Bt}}
\newcommand{\ucap}[0]{\hat{\Bu}}
\newcommand{\vcap}[0]{\hat{\Bv}}
\newcommand{\wcap}[0]{\hat{\Bw}}
\newcommand{\xcap}[0]{\hat{\Bx}}
\newcommand{\ycap}[0]{\hat{\By}}
\newcommand{\zcap}[0]{\hat{\Bz}}
\newcommand{\thetacap}[0]{\hat{\Btheta}}

%
% to write R^n and C^n in a distinguishable fashion.  Perhaps change this
% to the double lined characters upon figuring out how to do so.
%
\newcommand{\C}[1]{$\mathbb{C}^{#1}$}
\newcommand{\R}[1]{$\mathbb{R}^{#1}$}

%
% various generally useful helpers
%

% derivative of #1 wrt. #2:
\newcommand{\D}[2] {\frac {d#2} {d#1}}

\newcommand{\inv}[1]{\frac{1}{#1}}
\newcommand{\cross}[0]{\times}

\newcommand{\abs}[1]{\lvert{#1}\rvert}
\newcommand{\norm}[1]{\lVert{#1}\rVert}
\newcommand{\innerprod}[2]{\langle{#1}, {#2}\rangle}
\newcommand{\dotprod}[2]{{#1} \cdot {#2}}
\newcommand{\bdotprod}[2]{\left({#1} \cdot {#2}\right)}
\newcommand{\crossprod}[2]{{#1} \cross {#2}}
\newcommand{\tripleprod}[3]{\dotprod{\left(\crossprod{#1}{#2}\right)}{#3}}

\DeclareMathOperator{\Proj}{Proj}
\DeclareMathOperator{\Span}{span}
\DeclareMathOperator{\Sgn}{sgn}
\DeclareMathOperator{\Area}{Area}
\DeclareMathOperator{\Volume}{Volume}

%
% A few miscellaneous things specific to this document
%
\newcommand{\crossop}[1]{\crossprod{#1}{}}

% R2 vector.
\newcommand{\VectorTwo}[2]{
\begin{bmatrix}
 {#1} \\
 {#2}
\end{bmatrix}
}

\newcommand{\VectorN}[1]{
\begin{bmatrix}
{#1}_1 \\
{#1}_2 \\
\vdots \\
{#1}_N \\
\end{bmatrix}
}

\newcommand{\DETuvij}[4]{
\begin{vmatrix}
 {#1}_{#3} & {#1}_{#4} \\
 {#2}_{#3} & {#2}_{#4}
\end{vmatrix}
}

\newcommand{\DETuvwijk}[6]{
\begin{vmatrix}
 {#1}_{#4} & {#1}_{#5} & {#1}_{#6} \\
 {#2}_{#4} & {#2}_{#5} & {#2}_{#6} \\
 {#3}_{#4} & {#3}_{#5} & {#3}_{#6}
\end{vmatrix}
}

\newcommand{\DETuvwxijkl}[8]{
\begin{vmatrix}
 {#1}_{#5} & {#1}_{#6} & {#1}_{#7} & {#1}_{#8} \\
 {#2}_{#5} & {#2}_{#6} & {#2}_{#7} & {#2}_{#8} \\
 {#3}_{#5} & {#3}_{#6} & {#3}_{#7} & {#3}_{#8} \\
 {#4}_{#5} & {#4}_{#6} & {#4}_{#7} & {#4}_{#8} \\
\end{vmatrix}
}

%\newcommand{\DETuvwxyijklm}[10]{
%\begin{vmatrix}
% {#1}_{#6} & {#1}_{#7} & {#1}_{#8} & {#1}_{#9} & {#1}_{#10} \\
% {#2}_{#6} & {#2}_{#7} & {#2}_{#8} & {#2}_{#9} & {#2}_{#10} \\
% {#3}_{#6} & {#3}_{#7} & {#3}_{#8} & {#3}_{#9} & {#3}_{#10} \\
% {#4}_{#6} & {#4}_{#7} & {#4}_{#8} & {#4}_{#9} & {#4}_{#10} \\
% {#5}_{#6} & {#5}_{#7} & {#5}_{#8} & {#5}_{#9} & {#5}_{#10}
%\end{vmatrix}
%}

% R3 vector.
\newcommand{\VectorThree}[3]{
\begin{bmatrix}
 {#1} \\
 {#2} \\
 {#3}
\end{bmatrix}
}


%<misc>
%
\newcommand{\Abs}[1]{{\left\lvert{#1}\right\rvert}}
\newcommand{\spacegrad}[0]{\boldsymbol{\nabla}}
\newcommand{\grad}[0]{\nabla}
\newcommand{\LL}[0]{\mathcal{L}}

% == \partial_{#1} {#2}
\newcommand{\PD}[2]{\frac{\partial {#2}}{\partial {#1}}}
% inline variant
\newcommand{\PDi}[2]{{\partial {#2}}/{\partial {#1}}}

\newcommand{\PDD}[3]{\frac{\partial^2 {#3}}{\partial {#1}\partial {#2}}}
%\newcommand{\PDd}[2]{\frac{\partial^2 {#2}}{{\partial{#1}}^2}}
\newcommand{\PDsq}[2]{\frac{\partial^2 {#2}}{(\partial {#1})^2}}

\newcommand{\Partial}[2]{\frac{\partial {#1}}{\partial {#2}}}
\DeclareMathOperator{\RejName}{Rej}
\newcommand{\Rej}[2]{\RejName_{#1}\left( {#2} \right)}
\newcommand{\Rm}[1]{\mathbb{R}^{#1}}
\newcommand{\Cm}[1]{\mathbb{C}^{#1}}
\newcommand{\conj}[0]{{*}}

%</misc>

% <grade selection>
%
\newcommand{\gpgrade}[2] {{\left\langle{{#1}}\right\rangle}_{#2}}

\newcommand{\gpgradezero}[1] {\gpgrade{#1}{}}
%\newcommand{\gpscalargrade}[1] {{\left\langle{{#1}}\right\rangle}}
%\newcommand{\gpgradezero}[1] {\gpgrade{#1}{0}}

%\newcommand{\gpgradeone}[1] {{\left\langle{{#1}}\right\rangle}_{1}}
\newcommand{\gpgradeone}[1] {\gpgrade{#1}{1}}

\newcommand{\gpgradetwo}[1] {\gpgrade{#1}{2}}
\newcommand{\gpgradethree}[1] {\gpgrade{#1}{3}}
\newcommand{\gpgradefour}[1] {\gpgrade{#1}{4}}
%
% </grade selection>



\newcommand{\adot}[0]{{\dot{a}}}
\newcommand{\bdot}[0]{{\dot{b}}}
% taken for centered dot:
%\newcommand{\cdot}[0]{{\dot{c}}}
%\newcommand{\ddot}[0]{{\dot{d}}}
\newcommand{\edot}[0]{{\dot{e}}}
\newcommand{\fdot}[0]{{\dot{f}}}
\newcommand{\gdot}[0]{{\dot{g}}}
\newcommand{\hdot}[0]{{\dot{h}}}
\newcommand{\idot}[0]{{\dot{i}}}
\newcommand{\jdot}[0]{{\dot{j}}}
\newcommand{\kdot}[0]{{\dot{k}}}
\newcommand{\ldot}[0]{{\dot{l}}}
\newcommand{\mdot}[0]{{\dot{m}}}
\newcommand{\ndot}[0]{{\dot{n}}}
%\newcommand{\odot}[0]{{\dot{o}}}
\newcommand{\pdot}[0]{{\dot{p}}}
\newcommand{\qdot}[0]{{\dot{q}}}
\newcommand{\rdot}[0]{{\dot{r}}}
\newcommand{\sdot}[0]{{\dot{s}}}
\newcommand{\tdot}[0]{{\dot{t}}}
\newcommand{\udot}[0]{{\dot{u}}}
\newcommand{\vdot}[0]{{\dot{v}}}
\newcommand{\wdot}[0]{{\dot{w}}}
\newcommand{\xdot}[0]{{\dot{x}}}
\newcommand{\ydot}[0]{{\dot{y}}}
\newcommand{\zdot}[0]{{\dot{z}}}
\newcommand{\addot}[0]{{\ddot{a}}}
\newcommand{\bddot}[0]{{\ddot{b}}}
\newcommand{\cddot}[0]{{\ddot{c}}}
%\newcommand{\dddot}[0]{{\ddot{d}}}
\newcommand{\eddot}[0]{{\ddot{e}}}
\newcommand{\fddot}[0]{{\ddot{f}}}
\newcommand{\gddot}[0]{{\ddot{g}}}
\newcommand{\hddot}[0]{{\ddot{h}}}
\newcommand{\iddot}[0]{{\ddot{i}}}
\newcommand{\jddot}[0]{{\ddot{j}}}
\newcommand{\kddot}[0]{{\ddot{k}}}
\newcommand{\lddot}[0]{{\ddot{l}}}
\newcommand{\mddot}[0]{{\ddot{m}}}
\newcommand{\nddot}[0]{{\ddot{n}}}
\newcommand{\oddot}[0]{{\ddot{o}}}
\newcommand{\pddot}[0]{{\ddot{p}}}
\newcommand{\qddot}[0]{{\ddot{q}}}
\newcommand{\rddot}[0]{{\ddot{r}}}
\newcommand{\sddot}[0]{{\ddot{s}}}
\newcommand{\tddot}[0]{{\ddot{t}}}
\newcommand{\uddot}[0]{{\ddot{u}}}
\newcommand{\vddot}[0]{{\ddot{v}}}
\newcommand{\wddot}[0]{{\ddot{w}}}
\newcommand{\xddot}[0]{{\ddot{x}}}
\newcommand{\yddot}[0]{{\ddot{y}}}
\newcommand{\zddot}[0]{{\ddot{z}}}

%<bold and dot greek symbols>
%

\newcommand{\Deltadot}[0]{{\dot{\Delta}}}
\newcommand{\Gammadot}[0]{{\dot{\Gamma}}}
\newcommand{\Lambdadot}[0]{{\dot{\Lambda}}}
\newcommand{\Omegadot}[0]{{\dot{\Omega}}}
\newcommand{\Phidot}[0]{{\dot{\Phi}}}
\newcommand{\Pidot}[0]{{\dot{\Pi}}}
\newcommand{\Psidot}[0]{{\dot{\Psi}}}
\newcommand{\Sigmadot}[0]{{\dot{\Sigma}}}
\newcommand{\Thetadot}[0]{{\dot{\Theta}}}
\newcommand{\Upsilondot}[0]{{\dot{\Upsilon}}}
\newcommand{\Xidot}[0]{{\dot{\Xi}}}
\newcommand{\alphadot}[0]{{\dot{\alpha}}}
\newcommand{\betadot}[0]{{\dot{\beta}}}
\newcommand{\chidot}[0]{{\dot{\chi}}}
\newcommand{\deltadot}[0]{{\dot{\delta}}}
\newcommand{\epsilondot}[0]{{\dot{\epsilon}}}
\newcommand{\etadot}[0]{{\dot{\eta}}}
\newcommand{\gammadot}[0]{{\dot{\gamma}}}
\newcommand{\kappadot}[0]{{\dot{\kappa}}}
\newcommand{\lambdadot}[0]{{\dot{\lambda}}}
\newcommand{\mudot}[0]{{\dot{\mu}}}
\newcommand{\nudot}[0]{{\dot{\nu}}}
\newcommand{\omegadot}[0]{{\dot{\omega}}}
\newcommand{\phidot}[0]{{\dot{\phi}}}
\newcommand{\pidot}[0]{{\dot{\pi}}}
\newcommand{\psidot}[0]{{\dot{\psi}}}
\newcommand{\rhodot}[0]{{\dot{\rho}}}
\newcommand{\sigmadot}[0]{{\dot{\sigma}}}
\newcommand{\taudot}[0]{{\dot{\tau}}}
\newcommand{\thetadot}[0]{{\dot{\theta}}}
\newcommand{\upsilondot}[0]{{\dot{\upsilon}}}
\newcommand{\varepsilondot}[0]{{\dot{\varepsilon}}}
\newcommand{\varphidot}[0]{{\dot{\varphi}}}
\newcommand{\varpidot}[0]{{\dot{\varpi}}}
\newcommand{\varrhodot}[0]{{\dot{\varrho}}}
\newcommand{\varsigmadot}[0]{{\dot{\varsigma}}}
\newcommand{\varthetadot}[0]{{\dot{\vartheta}}}
\newcommand{\xidot}[0]{{\dot{\xi}}}
\newcommand{\zetadot}[0]{{\dot{\zeta}}}

\newcommand{\Deltaddot}[0]{{\ddot{\Delta}}}
\newcommand{\Gammaddot}[0]{{\ddot{\Gamma}}}
\newcommand{\Lambdaddot}[0]{{\ddot{\Lambda}}}
\newcommand{\Omegaddot}[0]{{\ddot{\Omega}}}
\newcommand{\Phiddot}[0]{{\ddot{\Phi}}}
\newcommand{\Piddot}[0]{{\ddot{\Pi}}}
\newcommand{\Psiddot}[0]{{\ddot{\Psi}}}
\newcommand{\Sigmaddot}[0]{{\ddot{\Sigma}}}
\newcommand{\Thetaddot}[0]{{\ddot{\Theta}}}
\newcommand{\Upsilonddot}[0]{{\ddot{\Upsilon}}}
\newcommand{\Xiddot}[0]{{\ddot{\Xi}}}
\newcommand{\alphaddot}[0]{{\ddot{\alpha}}}
\newcommand{\betaddot}[0]{{\ddot{\beta}}}
\newcommand{\chiddot}[0]{{\ddot{\chi}}}
\newcommand{\deltaddot}[0]{{\ddot{\delta}}}
\newcommand{\epsilonddot}[0]{{\ddot{\epsilon}}}
\newcommand{\etaddot}[0]{{\ddot{\eta}}}
\newcommand{\gammaddot}[0]{{\ddot{\gamma}}}
\newcommand{\kappaddot}[0]{{\ddot{\kappa}}}
\newcommand{\lambdaddot}[0]{{\ddot{\lambda}}}
\newcommand{\muddot}[0]{{\ddot{\mu}}}
\newcommand{\nuddot}[0]{{\ddot{\nu}}}
\newcommand{\omegaddot}[0]{{\ddot{\omega}}}
\newcommand{\phiddot}[0]{{\ddot{\phi}}}
\newcommand{\piddot}[0]{{\ddot{\pi}}}
\newcommand{\psiddot}[0]{{\ddot{\psi}}}
\newcommand{\rhoddot}[0]{{\ddot{\rho}}}
\newcommand{\sigmaddot}[0]{{\ddot{\sigma}}}
\newcommand{\tauddot}[0]{{\ddot{\tau}}}
\newcommand{\thetaddot}[0]{{\ddot{\theta}}}
\newcommand{\upsilonddot}[0]{{\ddot{\upsilon}}}
\newcommand{\varepsilonddot}[0]{{\ddot{\varepsilon}}}
\newcommand{\varphiddot}[0]{{\ddot{\varphi}}}
\newcommand{\varpiddot}[0]{{\ddot{\varpi}}}
\newcommand{\varrhoddot}[0]{{\ddot{\varrho}}}
\newcommand{\varsigmaddot}[0]{{\ddot{\varsigma}}}
\newcommand{\varthetaddot}[0]{{\ddot{\vartheta}}}
\newcommand{\xiddot}[0]{{\ddot{\xi}}}
\newcommand{\zetaddot}[0]{{\ddot{\zeta}}}

\newcommand{\BDelta}[0]{\boldsymbol{\Delta}}
\newcommand{\BGamma}[0]{\boldsymbol{\Gamma}}
\newcommand{\BLambda}[0]{\boldsymbol{\Lambda}}
\newcommand{\BOmega}[0]{\boldsymbol{\Omega}}
\newcommand{\BPhi}[0]{\boldsymbol{\Phi}}
\newcommand{\BPi}[0]{\boldsymbol{\Pi}}
\newcommand{\BPsi}[0]{\boldsymbol{\Psi}}
\newcommand{\BSigma}[0]{\boldsymbol{\Sigma}}
\newcommand{\BTheta}[0]{\boldsymbol{\Theta}}
\newcommand{\BUpsilon}[0]{\boldsymbol{\Upsilon}}
\newcommand{\BXi}[0]{\boldsymbol{\Xi}}
\newcommand{\Balpha}[0]{\boldsymbol{\alpha}}
\newcommand{\Bbeta}[0]{\boldsymbol{\beta}}
\newcommand{\Bchi}[0]{\boldsymbol{\chi}}
\newcommand{\Bdelta}[0]{\boldsymbol{\delta}}
\newcommand{\Bepsilon}[0]{\boldsymbol{\epsilon}}
\newcommand{\Beta}[0]{\boldsymbol{\eta}}
\newcommand{\Bgamma}[0]{\boldsymbol{\gamma}}
\newcommand{\Bkappa}[0]{\boldsymbol{\kappa}}
\newcommand{\Blambda}[0]{\boldsymbol{\lambda}}
\newcommand{\Bmu}[0]{\boldsymbol{\mu}}
\newcommand{\Bnu}[0]{\boldsymbol{\nu}}
%\newcommand{\Bomega}[0]{\boldsymbol{\omega}}
\newcommand{\Bphi}[0]{\boldsymbol{\phi}}
\newcommand{\Bpi}[0]{\boldsymbol{\pi}}
\newcommand{\Bpsi}[0]{\boldsymbol{\psi}}
\newcommand{\Brho}[0]{\boldsymbol{\rho}}
\newcommand{\Bsigma}[0]{\boldsymbol{\sigma}}
%\newcommand{\Btau}[0]{\boldsymbol{\tau}}
%\newcommand{\Btheta}[0]{\boldsymbol{\theta}}
\newcommand{\Bupsilon}[0]{\boldsymbol{\upsilon}}
\newcommand{\Bvarepsilon}[0]{\boldsymbol{\varepsilon}}
\newcommand{\Bvarphi}[0]{\boldsymbol{\varphi}}
\newcommand{\Bvarpi}[0]{\boldsymbol{\varpi}}
\newcommand{\Bvarrho}[0]{\boldsymbol{\varrho}}
\newcommand{\Bvarsigma}[0]{\boldsymbol{\varsigma}}
\newcommand{\Bvartheta}[0]{\boldsymbol{\vartheta}}
\newcommand{\Bxi}[0]{\boldsymbol{\xi}}
\newcommand{\Bzeta}[0]{\boldsymbol{\zeta}}
%
%</bold and dot greek symbols>
%<infrequent>
%
%\newcommand{\AreaOp}[1]{\AName_{#1}}
%\newcommand{\Babs}[0]{\abs{\BB}}
%\newcommand{\Bcap}[0]{\hat{\BB}}
%\newcommand{\BrPrimeRej}[0]{\rcap(\rcap \wedge \Br')}
%\newcommand{\CA}[0]{\mathcal{A}}
%\newcommand{\Cos}[1]{\cos{\left({#1}\right)}}
%\newcommand{\Det}[1] {\abs{#1}}
%\newcommand{\Dsq}[2] {\frac {\partial^2 {#1}} {\partial {#2}^2}}
%\newcommand{\Exp}[1]{\exp{\left({#1}\right)}}
%\newcommand{\Norm}[1]{\left\lVert{#1}\right\rVert}
%\newcommand{\Sin}[1]{\sin{\left({#1}\right)}}
%\newcommand{\T}[0]{\text{T}}
%\newcommand{\VolumeOp}[1]{\VName_{#1}}
%\newcommand{\agrad}[0]{\Ba \cdot \nabla}
%\newcommand{\alphacap}[0]{\hat{\boldsymbol{\alpha}}}
%\newcommand{\Fcap}[0]{\hat{\BF}}
%\newcommand{\bithree}[0]{{\Bi}_3}
%\newcommand{\bxa}[0]{\Bx\Ba}
%\newcommand{\coordvec}[2]{
%\newcommand{\costheta}[0]{\acap \cdot \xcap}
%\newcommand{\ddt}[1]{\ddot{#1}}
%\newcommand{\ddu}[1] {\frac {d{#1}} {du}}
%\newcommand{\dsqxj}[2] {\frac {\partial^2 {#1}} {\partial {x_{#2}}^2}}
%\newcommand{\dtheta}[1]{\frac{d {#1}}{d \theta}}
%\newcommand{\dt}[1]{\dot{#1}}
%\newcommand{\dt}[1]{\frac{d {#1}}{dt}}
%\newcommand{\dxj}[2] {\frac {\partial {#1}} {\partial {x_{#2}}}}
%\newcommand{\halfPhi}[0]{\frac{\phi}{2}}
%\newcommand{\half}[0]{\inv{2}}
%\newcommand{\inv}[1]{\frac{1}{#1}}
%\newcommand{\laplacian}[0]{\nabla^2}
%\newcommand{\matrixoftx}[3]{
%\newcommand{\nrrp}[0]{\norm{\rcap \wedge \Br'}}
%\newcommand{\oiint}{\bigcirc \hspace{-1.4em} \int \hspace{-.8em} \int}
%\newcommand{\transpose}[1]{{#1}^{\text{T}}}
%\newcommand{\transpose}[1]{{{#1}^{\TextTranspose}}}
%\newcommand{\transpose}[1]{{{#1}^{\text{T}}}}
%\newcommand{\barA}[0]{\bar{A}}
%\newcommand{\qbar}[0]{\bar{q}}
%\newcommand{\qdotbar}[0]{\dot{\bar{q}}}
%
%</infrequent>





\usepackage[bookmarks=true]{hyperref}

\usepackage{color,cite,graphicx}
   % use colour in the document, put your citations as [1-4]
   % rather than [1,2,3,4] (it looks nicer, and the extended LaTeX2e
   % graphics package. 
\usepackage{latexsym,amssymb,epsf} % don't remember if these are
   % needed, but their inclusion can't do any damage


\title{ Poynting vector and Electromagnetic Energy conservation. }
\author{Peeter Joot}
\date{ Dec 29, 2008.  Last Revision: $Date: 2008/12/31 01:06:10 $ }

\begin{document}

\maketitle{}

\tableofcontents

\section{ Motivation. }

Clarify Poynting discussion from \cite{doran2003gap}.

Equation 7.59 and 7.60 derives a $\BE \cross \BB$ quantity, the Poynting vector, as a sort of energy flux through the surface of the containing volume.

There are a couple of magic steps here that were not at all obvious to me.  Go through this in enough detail that it makes sense to me.

\section{ Charge free case. }

In SI units the Energy density is given as

\begin{align*}
U = \frac{\epsilon_0}{2}\left( \BE^2 + c^2 \BB^2 \right)
\end{align*}

FIXME: Don't truely understand where this part comes from.  The article \href{http://farside.ph.utexas.edu/teaching/em/lectures/node89.html}{Energy Conservation} looks promising to study this.

Given this energy density the rate of change of energy in a volume is then

\begin{align*}
\frac{dU}{dt} 
&= 
\frac{d}{dt} 
\frac{\epsilon_0}{2} \int dV \left( \BE^2 + c^2 \BB^2 \right) \\
&= 
\epsilon_0 \int dV \left( \BE \cdot \PD{t}{\BE} + c^2 \BB \cdot \PD{t}{\BB} \right) \\
\end{align*}

The next (omitted in the text) step is to utilize Maxwell's equation to eliminate the time derivatives.  Since this is the
charge and current free case, we can write Maxwell's as

\begin{align*}
0
&= \gamma_0 \grad F \\
&= \gamma_0 (\gamma^0 \partial_0 + \gamma^k \partial_k) F \\
&= (\partial_0 + \gamma_k\gamma_0 \partial_k) F \\
&= (\partial_0 + \sigma_k \partial_k) F \\
&= (\partial_0 + \spacegrad)F \\
&= (\partial_0 + \spacegrad)(\BE + ic \BB) \\
&= \partial_0 \BE + ic \partial_0 \BB + \spacegrad \BE + ic \spacegrad \BB \\
\end{align*}

In the spatial ($\sigma$) basis we can separate this into even and odd grades, which are separately equal to zero

\begin{align*}
0 &= \partial_0 \BE + ic \spacegrad \BB \\
%   1                  3,1 
0 &= ic \partial_0 \BB + \spacegrad \BE 
%  2                    0,2
\end{align*}

A selection of just the vector parts is

\begin{align*}
\partial_t \BE &= - ic^2 \spacegrad \wedge \BB \\
\partial_t \BB &= i\spacegrad \wedge \BE 
\end{align*}

Which can be back substituited into the energy flux
\begin{align*}
\frac{dU}{dt} 
&= \epsilon_0 \int dV \left( \BE \cdot (-i c^2 \spacegrad \wedge \BB) + c^2 \BB \cdot (i \spacegrad \wedge \BE) \right) \\
&= \epsilon_0 c^2 \int dV \gpgradezero{ \BB i \spacegrad \wedge \BE -\BE i \spacegrad \wedge \BB } \\
\end{align*}

Since the two divergence terms are zero we can drop the wedges here for

\begin{align*}
\frac{dU}{dt} 
&= \epsilon_0 c^2 \int dV \gpgradezero{ \BB i \spacegrad \BE -\BE i \spacegrad \BB } \\
&= \epsilon_0 c^2 \int dV \gpgradezero{ (i \BB) \spacegrad \BE -\BE \spacegrad (i\BB) } \\
&= \epsilon_0 c^2 \int dV \spacegrad \cdot ( (i \BB) \cdot \BE ) \\
\end{align*}

Justification for this last step can be found below in the derivation of equation \ref{eqn:poyntingDivergence}.

We can now use Stokes theorem to change this into a surface integral for a final energy flux 

\begin{align*}
\frac{dU}{dt} 
&= \epsilon_0 c^2 \int d\BA \cdot ( (i \BB) \cdot \BE ) \\
\end{align*}

This last bivector/vector dot product is the Poynting vector

\begin{align*}
(i \BB) \cdot \BE 
&= \gpgradeone{ (i \BB) \cdot \BE } \\
&= \gpgradeone{ i \BB \BE } \\
&= \gpgradeone{ i (\BB \wedge \BE) } \\
&= i (\BB \wedge \BE) \\
&= i^2(\BB \cross \BE) \\
&= \BE \cross \BB \\
\end{align*}

So, we can identity the quantity 

\begin{align}\label{eqn:poynting}
\epsilon_0 c^2 \BE \cross \BB = \epsilon_0 c (i c \BB) \cdot \BE 
\end{align}

As a directed energy density flux through the surface of a containing volume.


\section{ With charges and currents }
 
To calculate time derivatives we want to take Maxwell's equation and put into a form with explicit time derivatives, as was done before, but this time be more careful with the handling of the four vector current term.  Starting with left factoring out of a $\gamma_0$ from the spacetime gradient. 
 
\begin{align*}
\grad &= \gamma^0 \partial_0 + \gamma^k \partial_k \\
&= \gamma^0 (\partial_0 - \gamma^k \gamma_0 \partial_k) \\
&= \gamma^0 (\partial_0 + \sigma_k \partial_k) \\
\end{align*}

Similarily, the $\gamma_0$ can be factored from the current density

\begin{align*}
J 
&= \gamma_0 c \rho + \gamma_k J^k \\
&= \gamma_0 (c \rho - \gamma_k \gamma_0 J^k) \\
&= \gamma_0 (c \rho - \sigma_k J^k) \\
&= \gamma_0 (c \rho - \Bj )
\end{align*}

With this Maxwell's equation becomes
 
\begin{align*}
\gamma_0 \grad F &= \gamma_0 J / \epsilon_0 c \\
(\partial_0 + \spacegrad) ( \BE + i c \BB ) &= \rho/\epsilon_0 - \Bj/\epsilon_0 c \\
\end{align*}
 
A split into even and odd grades including current and charge density is thus
 
\begin{align*}
\spacegrad \BE + \partial_t (i \BB) &= \rho/\epsilon_0 \\
\spacegrad (i \BB) c^2 + \partial_t \BE &= -\Bj/\epsilon_0
\end{align*}
 
Now, taking time derivatives of the energy density gives

\begin{align*}
\PD{t}{U} 
&= \PD{t}{}\inv{2} \epsilon_0 \left( \BE^2 - (ic \BB)^2 \right) \\
&= \epsilon_0 \left( \BE \cdot \partial_t \BE - c^2 (i\BB) \cdot \partial_t (i\BB) \right) \\
&= \epsilon_0 \gpgradezero{ \BE ( -\Bj/\epsilon_0 -\spacegrad (i \BB) c^2 ) - c^2 (i\BB) ( -\spacegrad \BE + \rho/\epsilon_0 ) } \\
&= -\BE \cdot \Bj + c^2 \epsilon_0 \gpgradezero{ i\BB \spacegrad \BE -\BE \spacegrad (i \BB) } \\
&= -\BE \cdot \Bj + c^2 \epsilon_0 \left( (i\BB) \cdot (\spacegrad \wedge \BE) - \BE \cdot (\spacegrad \cdot (i \BB)) \right) \\
\end{align*}

Using equation \ref{eqn:poyntingDivergence}, we now have the rate of change of
field energy for the general case including currents.  That is

\begin{align}
\PD{t}{U} &= -\BE \cdot \Bj + c^2 \epsilon_0 \spacegrad \cdot (\BE \cdot (i\BB)) 
\end{align}

Written out in full, and in terms of the Poynting vector this is

\begin{align}
\PD{t}{}\frac{\epsilon_0}{2} \left(\BE^2 + c^2 \BB^2\right) + c^2 \epsilon_0 \spacegrad \cdot (\BE \cross \BB) &= -\BE \cdot \Bj 
\end{align}

\section{ Poynting vector in terms of complete field. }

In equation \ref{eqn:poynting} the individual parts of the complete Faraday
bivector $F = \BE + i c \BB$ stand out.  How would the Poynting vector be
expressed in terms of $F$ or in tensor form?

Since
\begin{align*}
F \gamma_0 = - \gamma_0(\BE - i c \BB)
\end{align*}

we have
\begin{align*}
\gamma^0 F \gamma_0 = - (\BE - i c \BB)
\end{align*}

and
\begin{align*}
i c \BB &= \inv{2}(F + \gamma^0 F \gamma_0) \\
\BE &= \inv{2}(F - \gamma^0 F \gamma_0) \\
\end{align*}

FIXME: tried using these but messed up.

%Without justifying all the steps I think that the following is valid
%
%\begin{align*}
%(i c \BB) \cdot \BE 
%&= \gpgradeone{(i c \BB) \cdot \BE } \\
%&= \gpgradeone{i c \BB \BE } \\
%&= \inv{4} (F + \gamma_0 F \gamma_0) \cdot (F - \gamma_0 F \gamma_0) \\
%&= \inv{4} (F^2 - \gamma_0 F \gamma_0 \gamma_0 F \gamma_0 + (\gamma_0 F \gamma_0) \cdot F - F \cdot (\gamma_0 F \gamma_0) ) \\
%&= \inv{4} ( (\gamma_0 F \gamma_0) \cdot F - F \cdot (\gamma_0 F \gamma_0) ) \\
%&= \inv{2} (\gamma_0 F \gamma_0) \cdot F 
%\end{align*}
%
%  The above is wrong.  This is - <F^\dagger F>/2, which is c^2 B^2 - E^2, which isn't even vector.

\section{ Energy Density from Lagrangian. }

I didn't get too far trying to calculate the electrodynamic Hamiltonian density for the general case, so I tried it for a very 
simple special case, with just an electric field component in one direction:

\begin{align*}
\mathcal{L}
&= \frac{1}{2}(E_x)^2 \\
&= \frac{1}{2}(F_{01})^2 \\
&= \frac{1}{2}(\partial_0 A_1 - \partial_1 A_0)^2 \\
\end{align*}

Goldstein gives the Hamiltonian density as

\begin{align*}
\pi &= \frac{\partial \mathcal{L}}{\partial \dot{n}} \\
\mathcal{H} &= \dot{n} \pi - \mathcal{L}
\end{align*}

If I try calculating this I get

\begin{align*}
\pi 
&= \frac{\partial}{\partial (\partial_0 A_1)} \left( \frac{1}{2}(\partial_0 A_1 - \partial_1 A_0)^2 \right) \\
&= \partial_0 A_1 - \partial_1 A_0 \\
&= F_{01} \\
\end{align*}

So this gives a Hamiltonian of
\begin{align*}
\mathcal{H}
&= \partial_0 A_1 F_{01} - \frac{1}{2}(\partial_0 A_1 - \partial_1 A_0)F_{01} \\
&= \frac{1}{2} (\partial_0 A_1 + \partial_1 A_0 )F_{01} 
&= \frac{1}{2} ((\partial_0 A_1)^2 - (\partial_1 A_0)^2 )
\end{align*}

For a Lagrangian density of $E^2 - B^2$ we have an energy density of $E^2 + B^2$, so I'd have expected the Hamiltonian density here to stay equal to $E_x^2/2$, but it 
doesn't look like that's what I get (what I calculated isn't at all familiar seeming).

If I haven't made a mistake here, perhaps I'm incorrect in assuming that the Hamiltonian density of the electrodynamic Lagrangian should be the energy density?

\section{ Appendix.  Messy details. }

For both the charge and the charge free case, we need a proof of 

\begin{align*}
(i\BB) \cdot (\spacegrad \wedge \BE) - \BE \cdot (\spacegrad \cdot (i \BB)) 
&= \spacegrad \cdot (\BE \cdot (i\BB)) 
\end{align*}

This is relativity straightforward, albeit tedious, to do backwards.

\begin{align*}
\spacegrad \cdot ((i \BB) \cdot \BE)
&= \gpgradezero{ \spacegrad ((i \BB) \cdot \BE)} \\
&= \inv{2} \gpgradezero{ \spacegrad ( i \BB \BE - \BE i \BB ) } \\
&= \inv{2} \gpgradezero{ 
  \dot{\spacegrad} i \dot{\BB} \BE 
+ \dot{\spacegrad} i \BB \dot{\BE}
- \dot{\spacegrad} \dot{\BE} i \BB 
- \dot{\spacegrad} \BE i \dot{\BB}
} \\
&= \inv{2} \gpgradezero{ 
  \BE \spacegrad (i \BB) - (i\dot{\BB}) \dot{\spacegrad} \BE
+ \dot{\BE} \dot{\spacegrad} i \BB - i \BB \spacegrad \BE
} \\
&= \inv{2} \left(
  \BE \cdot (\spacegrad \cdot (i \BB)) - ((i\dot{\BB}) \cdot \dot{\spacegrad}) \cdot \BE
+ (\dot{\BE} \wedge \dot{\spacegrad}) \cdot (i \BB) - (i \BB) \cdot (\spacegrad \wedge \BE) 
\right)
\\
\end{align*}

Grouping the two sets of repeated terms after reordering and the associated sign adjustments we have

\begin{align}\label{eqn:poyntingDivergence}
\spacegrad \cdot ((i \BB) \cdot \BE) &= \BE \cdot (\spacegrad \cdot (i \BB)) - (i \BB) \cdot (\spacegrad \wedge \BE)
\end{align}

which is the desired identity (in negated form) that was to be proved.

There is likely some theorem that could be used to avoid some of this algebra.

\bibliographystyle{plainnat}
\bibliography{myrefs}

\end{document}

%%%      %%
% Copyright � 2016 Peeter Joot.  All Rights Reserved.
% Licenced as described in the file LICENSE under the root directory of this GIT repository.
%
\makeproblem{Poynting theorem}{emt:problemSet7:1}{
Using Maxwell's equations given in the class notes, derive the Poynting theorem in both
differential and integral form for instantaneous fields. Assume a linear, homogeneous medium
with no temporal dispersion.
} % makeproblem

\makeanswer{emt:problemSet7:1}{

Given
\begin{equation}\label{eqn:emtproblemSet7Problem1:20}
\spacegrad \cross \BE
= -\BM_i - \PD{t}{\BB},
\end{equation}

and
\begin{equation}\label{eqn:emtproblemSet7Problem1:40}
\spacegrad \cross \BH
= \BJ_i + \BJ_c + \PD{t}{\BD},
\end{equation}

we want to expand the divergence of \( \BE \cross \BH \) to find the form of the Poynting theorem.


First we need the chain rule for of this sort of divergence.  Using primes to indicate the scope of the gradient operation

\begin{dmath}\label{eqn:emtproblemSet7Problem1:60}
\spacegrad \cdot \lr{ \BE \cross \BH }
=
\spacegrad' \cdot \lr{ \BE' \cross \BH }
-
\spacegrad' \cdot \lr{ \BH' \cross \BE }
=
\BH \cdot \lr{ \spacegrad' \cross \BE' }
-
\BH \cdot \lr{ \spacegrad' \cross \BH' }
%= %\gpgradezero{ %\spacegrad \lr{ \BE \cross \BH }
%}
%=
%\gpgradezero{
%-I \spacegrad \lr{ \BE \wedge \BH }
%}
%=
%-I \spacegrad \wedge \lr{ \BE \wedge \BH }
%=
%-I \BH \wedge \lr{ \spacegrad \wedge \BE }
%+I \BE \wedge \lr{ \spacegrad \wedge \BH }
=
\BH \cdot \lr{ \spacegrad \cross \BE }
-
\BE \cdot \lr{ \spacegrad \cross \BH }.
\end{dmath}

In the second step, cyclic permutation of the triple product was used.
This checks against the inside front cover of Jackson \citep{jackson1975cew}.  Now we can plug in the Maxwell equation cross products.

\begin{dmath}\label{eqn:emtproblemSet7Problem1:80}
\spacegrad \cdot \lr{ \BE \cross \BH }
=
\BH \cdot \lr{ -\BM_i - \PD{t}{\BB} }
-
\BE \cdot \lr{ \BJ_i + \BJ_c + \PD{t}{\BD} }
=
-\BH \cdot \BM_i
-\mu \BH \cdot \PD{t}{\BH}
-
\BE \cdot \BJ_i
-
\BE \cdot \BJ_c
-
\epsilon \BE \cdot \PD{t}{\BE},
\end{dmath}

or

%\begin{dmath}\label{eqn:emtproblemSet7Problem1:100}
\boxedEquation{eqn:emtproblemSet7Problem1:120}{
0
=
\spacegrad \cdot \lr{ \BE \cross \BH }
+ \frac{\epsilon}{2} \PD{t}{} \Abs{ \BE }^2
+ \frac{\mu}{2} \PD{t}{} \Abs{ \BH }^2
+ \BH \cdot \BM_i
+ \BE \cdot \BJ_i
+ \sigma \Abs{\BE}^2.
}
%\end{dmath}

In integral form this is

\begin{dmath}\label{eqn:emtproblemSet7Problem1:140}
0
=
\int d\BA \cdot \lr{ \BE \cross \BH }
+ \inv{2} \PD{t}{} \int dV \lr{
\epsilon \Abs{ \BE }^2
+ \mu \Abs{ \BH }^2
}
+ \int dV \BH \cdot \BM_i
+ \int dV \BE \cdot \BJ_i
+ \sigma \int dV \Abs{\BE}^2.
\end{dmath}

}

      \input{poyntingTimeHarmonic.tex}

   %
% Copyright � 2016 Peeter Joot.  All Rights Reserved.
% Licenced as described in the file LICENSE under the root directory of this GIT repository.
%
\newcommand{\authorname}{Peeter Joot}
\newcommand{\email}{peeterjoot@protonmail.com}
\newcommand{\basename}{FIXMEbasenameUndefined}
\newcommand{\dirname}{notes/FIXMEdirnameUndefined/}

\renewcommand{\basename}{emt6}
\renewcommand{\dirname}{notes/ece1228/}
\newcommand{\keywords}{ECE1228H}
\newcommand{\authorname}{Peeter Joot}
\newcommand{\onlineurl}{http://sites.google.com/site/peeterjoot2/math2013/\basename.pdf}
\newcommand{\sourcepath}{\dirname\basename.tex}
\newcommand{\generatetitle}[1]{\chapter{#1}}

\newcommand{\vcsinfo}{%
\section*{}
\noindent{\color{DarkOliveGreen}{\rule{\linewidth}{0.1mm}}}
\paragraph{Document version}
%\paragraph{\color{Maroon}{Document version}}
{
\small
\begin{itemize}
\item Available online at:\\ 
\href{\onlineurl}{\onlineurl}
\item Git Repository: \input{./.revinfo/gitRepo.tex}
\item Source: \sourcepath
\item last commit: \input{./.revinfo/gitCommitString.tex}
\item commit date: \input{./.revinfo/gitCommitDate.tex}
\end{itemize}
}
}

%\PassOptionsToPackage{dvipsnames,svgnames}{xcolor}
\PassOptionsToPackage{square,numbers}{natbib}
\documentclass{scrreprt}

\usepackage[left=2cm,right=2cm]{geometry}
\usepackage[svgnames]{xcolor}
\usepackage{peeters_layout}

\usepackage{natbib}

\usepackage[
colorlinks=true,
bookmarks=false,
pdfauthor={\authorname, \email},
backref 
]{hyperref}

% http://tex.stackexchange.com/questions/75773/how-to-reference-problems-by-the-text-label-in-an-exercise-envioronment
\usepackage[english]{cleveref}
\crefname{Exercise}{exercise}{exercises}
\Crefname{Exercise}{Exercise}{Exercises}

\RequirePackage{titlesec}
\RequirePackage{ifthen}

% http://stackoverflow.com/questions/4932910/date-in-the-tabular-environment
\makeatletter
\let\insertdate\@date
\makeatother

\titleformat{\chapter}[display]
{\bfseries\Large}
{\color{DarkSlateGrey}\filleft \authorname
\ifthenelse{\isundefined{\studentnumber}}{}{\\ \studentnumber}
\ifthenelse{\isundefined{\email}}{}{\\ \email}
\ifthenelse{\isundefined{\dateintitle}}{}{\\ \insertdate}
%\ifthenelse{\isundefined{\coursename}}{}{\\ \coursename} % put in title instead.
}
{4ex}
{\color{DarkOliveGreen}{\titlerule}\color{Maroon}
\vspace{2ex}%
\filright}
[\vspace{2ex}%
\color{DarkOliveGreen}\titlerule
]

\newcommand{\beginArtWithToc}[0]{\begin{document}\tableofcontents}
\newcommand{\beginArtNoToc}[0]{\begin{document}}
\newcommand{\EndNoBibArticle}[0]{\end{document}}
\newcommand{\EndArticle}[0]{\bibliography{Bibliography}\bibliographystyle{plainnat}\end{document}}

% 
%\newcommand{\citep}[1]{\cite{#1}}

\colorSectionsForArticle



%\usepackage{ece1228}
\usepackage{peeters_braket}
%\usepackage{peeters_layout_exercise}
\usepackage{peeters_figures}
\usepackage{mathtools}
\usepackage{siunitx}
\usepackage{enumerate}

\beginArtNoToc
\generatetitle{ECE1228H Electromagnetic Theory.  Lecture 6: XXX.  Taught by Prof.\ M. Mojahedi}
%\chapter{XXX}
\label{chap:emt6}

%\paragraph{Disclaimer}
%
%Peeter's lecture notes from class.  These may be incoherent and rough.
%
%These are notes for the UofT course ECE1228H, Electromagnetic Theory, taught by Prof. M. Mojahedi, covering \textchapref{{1}} \citep{balanis1989advanced} content.

\paragraph{Lorentz-Lorenz Dispersion}

We will model the medium using a frequency representation of the permittivity

\begin{dmath}\label{eqn:emtLecture6:20}
\begin{aligned}
\epsilon(\omega) &= \epsilon'(\omega) - j \epsilon''(\omega) \\
\mu(\omega) &= \mu'(\omega) - j \mu''(\omega)
\end{aligned}
\end{dmath}

The real part is the phase, whereas the imaginary part is the loss.

\begin{dmath}\label{eqn:emtLecture6:40}
n = \frac{c}{v} 
= \frac{\sqrt{\epsilon \mu}}{\sqrt{\epsilon_0 \mu_0}}  
= \sqrt{\epsilon_r \mu_r}
\end{dmath}

We can also write

\begin{dmath}\label{eqn:emtLecture6:60}
n(\omega) = n'(\omega) - j n''(\omega)
\end{dmath}

If we are considering an electric dipole

\begin{dmath}\label{eqn:emtLecture6:80}
\BP_i = Q_i \Bx_i
\end{dmath}

With 

\begin{dmath}\label{eqn:emtLecture6:100}
\BP = \epsilon_0 \chi_e \BE,
\end{dmath}

and a time harmonic representation for the electric field

\begin{dmath}\label{eqn:emtLecture6:120}
\BE = \BE_0 e^{j \omega t}.
\end{dmath}

The dipole moment is assumed to be

\begin{dmath}\label{eqn:emtLecture6:140}
\BP = \lim_{\Delta v \rightarrow 0} \frac{ \sum_{i = 1}^{N \Delta v} \BP_i }{\Delta v} 
= \frac{ N \Delta v \Bp}{\Delta v}
= N \Bp
= N Q \Bx.
\end{dmath}

F1: 

We model the oscillating electron and nucleus as a mass and spring.
This electron oscillator model is often called the Lorentz model.  It is not really a model for atoms as such, but the way that an atom responds to pertubation.  At the time when Lorentz formulated the model it was not known that the nuclei havr massive mass as compared to the electrons.
The Lorentz assumption was that in the absence of applied eletric fields the centroids of positive and neagivve charges coincide, but when a field is applied, the electrons will experience a Lorentz force and will be displaced from their equilibrium position. 
The wrote ``the displacement immediately gives rise to a new force by which the particle is pulled back towards its original position, and which we may therefore appropriately distinguish by the name of elastic force.''

The forces of interest are

\begin{dmath}\label{eqn:emtLecture6:160}
\begin{aligned}
F_{\textrm{friction}} &= -D \frac{dx}{dt} = -D v \\
F_{\textrm{elastic}} &= -S x \\
F_{\textrm{external}} &= Q E = Q E_0 e^{j \omega t}
\end{aligned}
\end{dmath}

Adding all the forces, the electrical system, in one dimension, can be assumed to have the form

\begin{equation}\label{eqn:emtLecture6:180}
F = m \frac{d^2 x}{dt^2}
=
-D \frac{dx}{dt} 
-D v \\
-S x \\
+ Q E_0 e^{j \omega t},
\end{equation}

or
\begin{dmath}\label{eqn:emtLecture6:200}
\frac{d^2 x}{dt^2} + \frac{D}{m} \ddt{x} + \frac{S}{m} x = \frac{Q E_0}{m} e^{j \omega t}
\end{dmath}

Let's define 

\begin{dmath}\label{eqn:emtLecture6:220}
\begin{aligned}
\gamma &= \frac{D}{m} \\
\omega_0^2 &= \frac{S}{m},
\end{aligned}
\end{dmath}

so that

\begin{dmath}\label{eqn:emtLecture6:240}
\frac{d^2 x}{dt^2} + \gamma \ddt{x} + \omega_0^2 x = \frac{Q E_0}{m} e^{j \omega t}.
\end{dmath}

\paragraph{Calculating the permittivity and susceptibility}

With \( x = x_0 e^{j \omega t} \) we have

\begin{dmath}\label{eqn:emtLecture6:260}
x_0 \lr{ -\omega^2 + j\gamma \omega + \omega_0^2 } = \frac{Q E_0}{m},
\end{dmath}

or (with \( E = E_0 e^{j \omega t} \)), just

\begin{equation}\label{eqn:emtLecture6:280}
x = x_0 e^{j\omega t} 
= \frac{Q E}{m \lr{ -\omega^2 + j\gamma \omega + \omega_0^2 } }.
\end{equation}

\begin{enumerate}[I]
\item Assume that dipoles are identical
\item Assume no coupling between dipoles
\item There are N dipoles per unit volume.  In other words, N is the number of dipoles per unit volume.
\end{enumerate}

The polarization \( P(t) \) is given by

\begin{dmath}\label{eqn:emtLecture6:300}
P(t) = N Q x,
\end{dmath}

where \( Q \) is the charge associate with the unit dipole.  This has dimensions of [\si{\frac{1}{m^3} \times C \times m}], or [\si{C/m^2}].  This polarization is

\begin{dmath}\label{eqn:emtLecture6:440}
P(t)
= \frac{Q^2 N E/m}{\omega_0^2 -\omega^2 + j\gamma \omega }.
\end{dmath}

In particular, the ratio of the polarization to the electric field magnitude is

\begin{dmath}\label{eqn:emtLecture6:320}
\frac{P}{E}
= \frac{Q^2 N/ m}{\omega_0^2 -\omega^2 + j\gamma \omega }.
\end{dmath}

With \( P = \epsilon_0 \chi_e E \), we have

\begin{dmath}\label{eqn:emtLecture6:340}
\chi_e = \frac{Q^2 N/ m \epsilon_0}{\omega_0^2 -\omega^2 + j\gamma \omega }.
\end{dmath}

Define 

\begin{dmath}\label{eqn:emtLecture6:360}
\omega_p^2 = \frac{ Q^2 N}{m \epsilon_0},
\end{dmath}

which has dimensions [\si{1/s^2}].  Then

\begin{dmath}\label{eqn:emtLecture6:380}
\chi_e = \frac{\omega_p^2}{\omega_0^2 -\omega^2 + j\gamma \omega }.
\end{dmath}

With \( \epsilon_r = 1 + \chi_e \) we have

\begin{equation}\label{eqn:emtLecture6:400}
\epsilon_r 
= \frac{\epsilon}{\epsilon_0} 
= 1 + \frac{\omega_p^2}{\omega_0^2 -\omega^2 + j\gamma \omega }.
\end{equation}

%or
%\begin{dmath}\label{eqn:emtLecture6:420}
%\epsilon_r 
%= \frac{ \omega_0^2 -\omega^2 + j\gamma \omega + \omega_p^2}{\omega_0^2 -\omega^2 + j\gamma \omega }
%\end{dmath}

One can show that \( \epsilon_r = \epsilon_r' -j \epsilon_r'' \) are given bby

\begin{dmath}\label{eqn:emtLecture6:460}
\epsilon_r' = \frac{\omega_p^2 \lr{ \omega_0^2 - \omega^2 } }{ (\omega_0^2 - \omega^2)^2 + (\omega \gamma)^2 } + 1,
\end{dmath}
\begin{dmath}\label{eqn:emtLecture6:480}
\epsilon_r'' = \frac{\omega_p^2 \omega \gamma}{ (\omega_0^2 - \omega^2)^2 + (\omega \gamma)^2 }.
\end{dmath}

FIXME: calculate this.

\paragraph{No damping}

With \( D = 0 \), or \( \gamma = 0 \) then \( \epsilon_r'' = 0 \),

\begin{dmath}\label{eqn:emtLecture6:500}
x = \frac{Q E_0/m}{\omega^2 - \omega^2} e^{j \omega t},
\end{dmath}

and
\begin{equation}\label{eqn:emtLecture6:520}
\epsilon_r 
=
\epsilon_r'
= \frac{\epsilon}{\epsilon_0} 
= 
1 + \frac{\omega_p^2}{\omega_0^2 - \omega^2}.
\end{equation}

This has a curve like

F5

instead of the normal damped resonance curve

F5b

As \( \omega \rightarrow \omega_0 \), then the displacement \( x \rightarrow \infty \).  The frequency \( \omega_0 \) is called the resonance frequency of the system.

If the resonance frequency is zero (free charges), then

\begin{dmath}\label{eqn:emtLecture6:540}
\epsilon_r = \epsilon_r' = 1 - \frac{\omega_p^2}{\omega^2},
\end{dmath}

which is negative for \( \omega_p > \omega \).  

When damping is present, the resonance frequency is the root of the characteristic equation of the homogeneous part of \cref{eqn:emtLecture6:200}.

\paragraph{Multiple resonances}

When there are \( N \) molecules per unit volume, and each molecule has
Z electrons per molecule that have a binding frequency \( \omega_i \) and damping constant \( \gamma_i \), then it can be shown that 

\begin{dmath}\label{eqn:emtLecture6:560}
\epsilon_r = 1 + \frac{Q N^2}{m \epsilon_0} \sum \frac{ f_i }{\omega_i^2 - \omega^2 + j \gamma \omega }
\end{dmath}

A quantum mechanical derivation of the transition frequencies is used in this derivation.

%\EndArticle
\EndNoBibArticle

      \section{Problems}

%%%      %
% Copyright � 2016 Peeter Joot.  All Rights Reserved.
% Licenced as described in the file LICENSE under the root directory of this GIT repository.
%
\makeproblem{Passive medium.}{emt:problemSet5:1}{ 

Parameters for \ce{AlGaN} (a passive medium) are given as

\begin{dmath}\label{eqn:emtProblemSet5Problem1:20}
\begin{aligned}
\omega_0 &= 1.921 \times 10^{14} \si{rad/s} \\
\omega_p &= 3.328 \times 10^{14} \si{rad/s} \\
\gamma   &= 9.756 \times 10^{12} \si{rad/s} \\
\end{aligned}
\end{dmath}

Assuming Lorentz model:

\makesubproblem{}{emt:problemSet5:1a}
Plot the real and imaginary parts of the index of refraction for the range of \( \omega = 0 \) to \( \omega = 6 \times 10^{14} \).
On the figure identify the region of anomalous dispersion.

\makesubproblem{}{emt:problemSet5:1b}
Plot the real and imaginary parts of the relative permittivity for the same range as in \partref{emt:problemSet5:1a}.

On the figure identify the region of anomalous dispersion.
} % makeproblem

\makeanswer{emt:problemSet5:1}{ 
\makeSubAnswer{}{emt:problemSet5:1a}

TODO.
\makeSubAnswer{}{emt:problemSet5:1b}

TODO.
}

%%%      %
% Copyright � 2016 Peeter Joot.  All Rights Reserved.
% Licenced as described in the file LICENSE under the root directory of this GIT repository.
%
\makeproblem{Medium with multiple resonances.}{emt:problemSet5:2}{ 

Relative permittivity for a medium with multiple resonances is given by:

\begin{equation}\label{eqn:emtProblemSet5Problem2:20}
\epsilon_r = 1 + \chi_e =
1 + \sum_{k=1} \frac{ \omega_{p,k} }{\omega_{0,k}^2 - \omega^2 + j \gamma_k \omega }
\end{equation}

Moreover, the case of an \textit{active medium} (i.e. medium with gain) can be modeled by allowing \( \omega_{p,k} \)
in above to become purely imaginary. 
Under these conditions, plot 

\begin{dmath}\label{eqn:emtProblemSet5Problem2:40}
\Real\lr{ n(\omega) } -1,
\end{dmath}

and
\begin{dmath}\label{eqn:emtProblemSet5Problem2:60}
\Imag\lr{ n(\omega) }
\end{dmath}
as a function of detuning frequency, 

\begin{dmath}\label{eqn:emtProblemSet5Problem2:80}
\nu = \frac{\omega - \omega_c}{2 \pi},
\end{dmath}

for ammonia vapor (an active
medium) where

\begin{dmath}\label{eqn:emtProblemSet5Problem2:100}
\begin{aligned}
\omega_{0,1} &= 2.4165825 \times 10^{15} \si{rad/s} \\
\omega_{0,2} &= 2.4166175 \times 10^{15} \si{rad/s} \\
\omega_{p,k} = \omega_p &= 10^{10} \si{rad/s} \\
\gamma_{k} = \gamma &= 5 \times 10^9 \si{rad/s} \\
\ifrac{(\omega - \omega_c)}{2 \pi} & \in [-7,7] \si{GHz} \\
\omega_c &= 2.4166 \times 10^{15} \si{rad/s}
\end{aligned}
\end{dmath}
} % makeproblem

\makeanswer{emt:problemSet5:2}{ 

TODO.
}

%%%      %
% Copyright � 2016 Peeter Joot.  All Rights Reserved.
% Licenced as described in the file LICENSE under the root directory of this GIT repository.
%
\makeproblem{Susceptibility kernel.}{emt:problemSet5:3}{ 

The relation between the electric flux density \( \BD \) and the electric field \( \BE \) is given by,

\begin{dmath}\label{eqn:emtProblemSet5Problem3:20}
\BD(\Bx, t) = \epsilon_0
\lr{
\BE(\Bx, t) 
+ \int_{-\infty}^\infty G(\tau) \BE(\Bx, t - \tau) d\tau,
}
\end{dmath}

where \( G(\tau) \) is the
susceptibility kernel given by
\begin{dmath}\label{eqn:emtProblemSet5Problem3:40}
G(\tau) =
%\inv{\sqrt{2 \pi}}
\inv{2 \pi}
\int_{-\infty}^\infty 
\lr{\frac{\epsilon(\omega)}{\epsilon_0} - 1}
e^{-j \omega t} d\tau.
\end{dmath}

\makesubproblem{}{emt:problemSet5:3a}
Show that 
\begin{dmath}\label{eqn:emtProblemSet5Problem3:60}
\epsilon(-\omega) = \epsilon^\conj(\omega)
\end{dmath}

\makesubproblem{}{emt:problemSet5:3b}
Show that for \( \epsilon(\omega) = \epsilon'(\omega) + j \epsilon''(\omega) \), 
\( \epsilon'(\omega) \) is even
and \( \epsilon''(\omega) \) is odd.
} % makeproblem

\makeanswer{emt:problemSet5:3}{ 
\makeSubAnswer{}{emt:problemSet5:3a}

TODO.
\makeSubAnswer{}{emt:problemSet5:3b}

TODO.
}


   \chapter{Druid model}
      
usepackage: \ce{}. chem.

\paragraph{Druid model}

In this section we will investigate the optical properties of free electrons, or what is commonly called free electron gas.

By free electron gas we mean electrons that do not experience the restoring force which we considered for bound garges in the case of Lorentz model.  In particular, the resonance frequency \( \omega_0 \) for free electrons is zero.

There are two typical cases of free electron systems 

\begin{itemize}[a]
\item Metals.
\item Doped (n or p type) semiconductors.
\end{itemize}

For the moment we consider the case of metals.

Free electrons are responsible for high reflectivity and good thermal conductivity of metals up to optical frequencies.  A model that can be used to describe the high reflectivity of metals is the Drude model.

\paragraph{Plasma:} A neutral gas of free eletrons and heavy ions is called plasma.  Examples of plasma are metals and doped semiconduction, since these materials are a compination of free electrons and heavy ions which are, in sum, electrically neutral.

\paragraph{Drude-Lorentz model}, (or Drude model for short): similar to the case of bound charges we already studied for free electron plasma, we can start with a harmonic oscillator model.  However, in this case, since electrons are free, there is no restoring force (i.e. \(\omega_0 = 0 \).  Recall that in the spring mass model \( \omega_0^2 = S/m \) where \( S \) was the spring tension coefficient.

With such a model the Lorentz model equation

\begin{dmath}\label{eqn:druid:20}
\frac{d^2 x}{dt^2} + \gamma \ddt{x} + \omega_0^2 x = \frac{Q E_0}{m} e^{j \omega t},
\end{dmath}

is reduced to

\begin{dmath}\label{eqn:druid:40}
\frac{d^2 x}{dt^2} + \gamma \ddt{x} = \frac{Q E_0}{m} e^{j \omega t},
\end{dmath}

Again, assuming a solution of the form \( x_p = x_0 e^{j \omega t} \) for the particular solution and substituting in \cref{eqn:druid:40}, we have

\begin{dmath}\label{eqn:druid:80}
x_0 \lr{ (j\omega)^2 + \gamma (j \omega)} = \frac{Q E_0}{m},
\end{dmath}

or
\begin{dmath}\label{eqn:druid:60}
x 
= 
\frac{Q E/m}{-\omega^2 + j \gamma \omega },
\end{dmath}

Once more assuming identical particles that are not coupled and a linear isotropic medium and using the fact that \( \BP = N \Bp = N Q \Bx \), and

\begin{dmath}\label{eqn:druid:100}
\chi_e = \frac{\Abs{\BP}}{\epsilon_0 \Abs{\BE} }, 
\end{dmath}

we have

\begin{dmath}\label{eqn:druid:120}
\chi_e 
=
\frac{Q^2 N/m \epsilon_0}{-\omega^2 + j \gamma \omega },
\end{dmath}

or with \( \omega_p^2 = Q^2 N/m\epsilon_0\),

\begin{dmath}\label{eqn:druid:140}
\epsilon_r 
= 1 + \chi_e 
= 
1+
\frac{\omega_p^2}{-\omega^2 + j \gamma \omega }.
\end{dmath}

Plasma frequency, \( \omega_p \), can be understood as the natural resonance frequency by which the free electron gas (plasma) collectively (not individulal electrons ) oscillates.

Note that if we neglect the last term, i.e., let \( \gamma = 0 \) then

\begin{dmath}\label{eqn:druid:160}
\epilson_r = 1 - \frac{\omega_p^2}{\omega^2}.
\end{dmath}

From this it is clear that when \( \omega < \omega_p \), we have \( \epsilon_r < 1 \) and \( n = \sqrt{\epsilon_r} \) is purely imaginary, and the wave attenuates inside the electron plasma.

This means that for \( \omega < \omega_p \) electromagnetic waves do not propagate a large distance inside of metal.  However, for \( \omega > \omega_p \) the electron plasma (e.g. metal) is transparent.  The latter is called ultraviolate transparency of metal, because for most metals \( \omega_p \) is in the ultraviolate part of the spectrum.  For example, 

\begin{itemize}
\item For \ce{Al}
\begin{dmath}\label{eqn:druid:180}
\frac{\omega_p}{2 \pi} = 3.82 \times 10^{15} \si{Hz} \implies \lambda_p = 79 [nm],
\end{dmath}
\item For \ce{Au}
\begin{dmath}\label{eqn:druid:200}
\frac{\omega_p}{2 \pi} = 5.9 \times 10^{15} \si{Hz} \implies \lambda_p = 138 [nm],
\end{dmath}
\end{itemize}

Using \cref{eqn:druid:160} one can calculate 

\begin{dmath}\label{eqn:druid:220}
\tilde{n} = \sqrt{\epsilon_r},
\end{dmath}

and plot the reflectivity \( R \) at normal incidence

\begin{dmath}\label{eqn:druid:240}
R = \Abs{ \frac{\tilde{n} - 1 }{\tilde{n} + 1} },0jA
\end{dmath}

which will have a shape similar to

F3

\paragraph{Conductivity}

%\ddt{} \lr{ m \ddt{x} } + \gamma \ddt{x} = Q E_0 e^{i \omega t}

\begin{dmath}\label{eqn:druid:260}
\spacegrad \cross \BE(\Br, \omega) 
= \sigma \BE(\Br, \omega) + j \omega \epsilon_0 \BE(\Br, \omega)
= j \omega \epsilon_0 \lr{ 1 + \frac{\sigma}{j \omega \epsilon_0} } \BE(\Br, \omega)
= j \omega \epsilon_0 \lr{ 1 - \frac{j \sigma}{\omega \epsilon_0} } \BE(\Br, \omega)
\end{dmath}

This complex factor is the relative permittivity

\begin{dmath}\label{eqn:druid:280}
\epsilon_r  
\end{dmath}
= 1 - \frac{j \sigma}{\omega \epsilon_0},
\end{dmath}

and is why we write

\begin{dmath}\label{eqn:druid:300}
\epsilon(\omega) = \epsilon'(\omega) - j \epsilon''(\omega) 
\end{dmath}

%      FIXME: transcribe handwritten notes that were mostly skipped over in class?
      \section{Problems}

%\chapter{conductivity} % transcribe?  This was part of L7
   %
% Copyright � 2016 Peeter Joot.  All Rights Reserved.
% Licenced as described in the file LICENSE under the root directory of this GIT repository.
%
%\newcommand{\authorname}{Peeter Joot}
\newcommand{\email}{peeterjoot@protonmail.com}
\newcommand{\basename}{FIXMEbasenameUndefined}
\newcommand{\dirname}{notes/FIXMEdirnameUndefined/}

%\renewcommand{\basename}{emt7}
%\renewcommand{\dirname}{notes/ece1228/}
%\newcommand{\keywords}{ECE1228H}
%\newcommand{\authorname}{Peeter Joot}
\newcommand{\onlineurl}{http://sites.google.com/site/peeterjoot2/math2013/\basename.pdf}
\newcommand{\sourcepath}{\dirname\basename.tex}
\newcommand{\generatetitle}[1]{\chapter{#1}}

\newcommand{\vcsinfo}{%
\section*{}
\noindent{\color{DarkOliveGreen}{\rule{\linewidth}{0.1mm}}}
\paragraph{Document version}
%\paragraph{\color{Maroon}{Document version}}
{
\small
\begin{itemize}
\item Available online at:\\ 
\href{\onlineurl}{\onlineurl}
\item Git Repository: \input{./.revinfo/gitRepo.tex}
\item Source: \sourcepath
\item last commit: \input{./.revinfo/gitCommitString.tex}
\item commit date: \input{./.revinfo/gitCommitDate.tex}
\end{itemize}
}
}

%\PassOptionsToPackage{dvipsnames,svgnames}{xcolor}
\PassOptionsToPackage{square,numbers}{natbib}
\documentclass{scrreprt}

\usepackage[left=2cm,right=2cm]{geometry}
\usepackage[svgnames]{xcolor}
\usepackage{peeters_layout}

\usepackage{natbib}

\usepackage[
colorlinks=true,
bookmarks=false,
pdfauthor={\authorname, \email},
backref 
]{hyperref}

% http://tex.stackexchange.com/questions/75773/how-to-reference-problems-by-the-text-label-in-an-exercise-envioronment
\usepackage[english]{cleveref}
\crefname{Exercise}{exercise}{exercises}
\Crefname{Exercise}{Exercise}{Exercises}

\RequirePackage{titlesec}
\RequirePackage{ifthen}

% http://stackoverflow.com/questions/4932910/date-in-the-tabular-environment
\makeatletter
\let\insertdate\@date
\makeatother

\titleformat{\chapter}[display]
{\bfseries\Large}
{\color{DarkSlateGrey}\filleft \authorname
\ifthenelse{\isundefined{\studentnumber}}{}{\\ \studentnumber}
\ifthenelse{\isundefined{\email}}{}{\\ \email}
\ifthenelse{\isundefined{\dateintitle}}{}{\\ \insertdate}
%\ifthenelse{\isundefined{\coursename}}{}{\\ \coursename} % put in title instead.
}
{4ex}
{\color{DarkOliveGreen}{\titlerule}\color{Maroon}
\vspace{2ex}%
\filright}
[\vspace{2ex}%
\color{DarkOliveGreen}\titlerule
]

\newcommand{\beginArtWithToc}[0]{\begin{document}\tableofcontents}
\newcommand{\beginArtNoToc}[0]{\begin{document}}
\newcommand{\EndNoBibArticle}[0]{\end{document}}
\newcommand{\EndArticle}[0]{\bibliography{Bibliography}\bibliographystyle{plainnat}\end{document}}

% 
%\newcommand{\citep}[1]{\cite{#1}}

\colorSectionsForArticle


%
%%\usepackage{ece1228}
%\usepackage{peeters_braket}
%%\usepackage{peeters_layout_exercise}
%\usepackage{peeters_figures}
%\usepackage{mathtools}
%\usepackage{siunitx}
%\usepackage{macros_bm}
%
%\beginArtNoToc
%\generatetitle{ECE1228H Electromagnetic Theory.  Lecture 8: Wave equation.  Taught by Prof.\ M. Mojahedi}
\chapter{Wave equation}
%\label{chap:emt7}

%\paragraph{Disclaimer}
%
%Peeter's lecture notes from class.  These may be incoherent and rough.
%
%These are notes for the UofT course ECE1228H, Electromagnetic Theory, taught by Prof. M. Mojahedi, covering \textchapref{{1}} \citep{balanis1989advanced} content.
%
\paragraph{Wave equation}

\begin{dmath}\label{eqn:emtLecture7:20}
\begin{aligned}
\spacegrad \cross \bcE &= -\PD{t}{\bcB} - \bcM \\
\spacegrad \cross \bcH &= \PD{t}{\bcD} + \bcJ \\
\spacegrad \cross \bcB &= \rho_{mv} \\
\spacegrad \cross \bcD &= \rho_{ev} \\
\end{aligned}
\end{dmath}

Using an expansion of the triple cross product in terms of the Laplacian
\begin{dmath}\label{eqn:emtLecture7:40}
\spacegrad \cross \lr{ \spacegrad \cross \Bf }
=
-\spacegrad \cdot \lr{ \spacegrad \wedge \Bf }
=
-\spacegrad^2 \Bf
+ \spacegrad \lr{ \spacegrad \cdot \Bf },
\end{dmath}

we can evaluate the cross products

\begin{dmath}\label{eqn:emtLecture7:60}
\begin{aligned}
\spacegrad \cross \lr{ \spacegrad \cross \bcE } &= \spacegrad \cross \lr{ -\PD{t}{\bcB} - \bcM } \\
\spacegrad \cross \lr{ \spacegrad \cross \bcH } &= \spacegrad \cross \lr{ \PD{t}{\bcD} + \bcJ },
\end{aligned}
\end{dmath}

or
\begin{dmath}\label{eqn:emtLecture7:80}
\begin{aligned}
-\spacegrad^2 \bcE + \spacegrad \lr{ \spacegrad \cdot \bcE } &= -\mu \PD{t}{} \spacegrad \cross \bcH - \spacegrad \cross \bcM \\
-\spacegrad^2 \bcH + \spacegrad \lr{ \spacegrad \cdot \bcH } &= \epsilon \PD{t}{} \lr{ \spacegrad \cross \bcE } + \spacegrad \cross \bcJ,
\end{aligned}
\end{dmath}

or

\begin{dmath}\label{eqn:emtLecture7:100}
\begin{aligned}
-\spacegrad^2 \bcE + \inv{\epsilon} \spacegrad \rho_{ev} &= -\mu \PD{t}{} \lr{ \PD{t}{\bcD} + \bcJ } - \spacegrad \cross \bcM \\
-\spacegrad^2 \bcH + \inv{\mu} \spacegrad \rho_{mv} &= \epsilon \PD{t}{} \lr{ -\PD{t}{\bcB} - \bcM } + \spacegrad \cross \bcJ,
\end{aligned}
\end{dmath}

This decouples the equations for the electric and the magnetic fields

\begin{dmath}\label{eqn:emtLecture7:120}
\begin{aligned}
\spacegrad^2 \bcE &=
   \mu \epsilon \PDSq{t}{\bcE} +
   \inv{\epsilon} \spacegrad \rho_{ev} +
   \mu \PD{t}{\bcJ } +
   \spacegrad \cross \bcM \\
\spacegrad^2 \bcH &=
   \epsilon \mu \PDSq{t}{\bcH} +
   \inv{\mu} \spacegrad \rho_{mv} +
   \epsilon \PD{t}{\bcM } -
   \spacegrad \cross \bcJ,
\end{aligned}
\end{dmath}

Splitting the current between induced and bound (?) currents

\begin{equation}\label{eqn:emtLecture7:260}
\bcJ = \bcJ_i + \bcJ_c = \bcJ_i + \sigma \bcE,
\end{equation}

these become

\begin{dmath}\label{eqn:emtLecture7:160}
\begin{aligned}
\spacegrad^2 \bcE &=
   \mu \epsilon \PDSq{t}{\bcE} +
   \inv{\epsilon} \spacegrad \rho_{ev} +
   \mu \sigma \PD{t}{\bcE} +
   \spacegrad \cross \bcM +
   \mu \PD{t}{\bcJ_i} \\
\spacegrad^2 \bcH &=
   \epsilon \mu \PDSq{t}{\bcH} +
   \inv{\mu} \spacegrad \rho_{mv} +
   \epsilon \PD{t}{\bcM } +
   \sigma \mu \PD{t}{\bcH} +
   \sigma \bcM
-
   \spacegrad \cross \bcJ_i
.
\end{aligned}
\end{dmath}

\paragraph{Time harmonic form}

Assuming time harmonic dependence \( \bcX = \BX e^{j\omega t} \), we find

\begin{dmath}\label{eqn:emtLecture7:140}
\begin{aligned}
\spacegrad^2 \BE &=
   \lr{ - \omega^2 \mu \epsilon +
   j \omega \mu \sigma } \BE +
   \inv{\epsilon} \spacegrad \rho_{ev} +
   \spacegrad \cross \BM +
   j \omega \mu \BJ_i \\
\spacegrad^2 \BH &=
   \lr{ -\omega^2 \epsilon \mu +
   j \omega \sigma \mu } \BH +
   \inv{\mu} \spacegrad \rho_{mv} +
   (j \omega \epsilon + \sigma) \BM
-
   \spacegrad \cross \BJ_i.
\end{aligned}
\end{dmath}

For a lossy medium where \( \epsilon = \epsilon' -j \omega \epsilon'' \), the leading term factor is

\begin{dmath}\label{eqn:emtLecture7:180}
- \omega^2 \mu \epsilon + j \omega \mu \sigma
=
- \omega^2 \mu \epsilon' + j \omega \mu \lr{ \sigma + \omega \epsilon'' }.
\end{dmath}

With the definition
\begin{equation}\label{eqn:emtLecture7:200}
\gamma^2 = \lr{ \alpha + j \beta }^2 = - \omega^2 \mu \epsilon' + j \omega \mu \lr{ \sigma + \omega \epsilon'' },
\end{equation}

the wave equations have the form

\begin{dmath}\label{eqn:emtLecture7:220}
\begin{aligned}
\spacegrad^2 \BE &=
\gamma^2 \BE +
   \inv{\epsilon} \spacegrad \rho_{ev} +
   \spacegrad \cross \BM +
   j \omega \mu \BJ_i \\
\spacegrad^2 \BH &=
\gamma^2 \BH +
   \inv{\mu} \spacegrad \rho_{mv} +
   (j \omega \epsilon + \sigma) \BM
-
   \spacegrad \cross \BJ_i.
\end{aligned}
\end{dmath}

Here

\begin{itemize}
\item \( \alpha \) is the attenuation constant [\si{Np/m}]
\item \( \beta \) is the phase velocity [\si{rad/m}]
\item \( \gamma \) is the propagation constant [\si{1/m}]
\end{itemize}

We are usually interested in solutions in regions free of magnetic currents, induced electric currents, and free of any charge densities, in which case the wave equations are just

\begin{dmath}\label{eqn:emtLecture7:240}
\begin{aligned}
\spacegrad^2 \BE &= \gamma^2 \BE  \\
\spacegrad^2 \BH &= \gamma^2 \BH.
\end{aligned}
\end{dmath}

%\EndArticle
%\EndNoBibArticle

      \section{Problems}
%%%         %
% Copyright � 2016 Peeter Joot.  All Rights Reserved.
% Licenced as described in the file LICENSE under the root directory of this GIT repository.
%
\makeproblem{Meissner effect.}{emt:problemSet6:1}{ 
The constitutive relation for superconductors in weak magnetic fields can be macroscopically
characterized by the first London equation

\begin{dmath}\label{eqn:emtproblemSet6Problem1:20}
\PD{t}{\BJ_{\mathrm{sup}}} = \alpha \BE,
\end{dmath}

and the second London equation
\begin{dmath}\label{eqn:emtproblemSet6Problem1:40}
\spacegrad \cross \BJ_{\mathrm{sup}} = -\alpha_1 \BB,
\end{dmath}
where \( \BJ_{\mathrm{sup}} \)
stands for the superconducting current, 
\( \alpha = n_s q^2 /m \) and \( \alpha_1 \approx \alpha \), with 
\( n_s \), \(m\), and \( q\) 
denoting, respectively, the number density, the effective mass, and the charge of the Cooper pairs
responsible for the superconductivity in a charged Boson fluid model.

\makesubproblem{}{emt:problemSet6:1a}
From the first London equation, derive and equation for \( \dot{\BB} = \PDi{t}{\BB} \) 
by using the static
Maxwell equation \( \spacegrad \cross \BH = \BJ_{\mathrm{sup}} \)
without the displacement current. Show that
\begin{dmath}\label{eqn:emtproblemSet6Problem1:60}
\spacegrad^2 \dot{\BB} = \mu_0 \alpha \dot{\BB}
\end{dmath}
\makesubproblem{}{emt:problemSet6:1b}
From the second London equation and the Ampere's law stated above derive an equation
for \( \BB \).
\makesubproblem{}{emt:problemSet6:1c}
What are the penetration depths in the 
\partref{emt:problemSet6:1a}
and
\partref{emt:problemSet6:1b}
cases? Justify your answer.

\paragraph{Remark:} from above analysis we see that both the current and magnetic field are confined to a
thin layer of the order of the penetration depth which is very small. The exclusion of static
magnetic field in a superconductor is known as the Meissner effect experimentally discovered in
1933.
} % makeproblem

\makeanswer{emt:problemSet6:1}{ 
\makeSubAnswer{}{emt:problemSet6:1a}

Taking the curl of the first London equation \cref{eqn:emtproblemSet6Problem1:20}, we have

\begin{dmath}\label{eqn:emtproblemSet6Problem1:160}
0 
= \spacegrad \cross \PD{t}{\BJ_{\mathrm{sup}}} - \alpha \spacegrad \cross \BE
= \PD{t}{} \lr{ \spacegrad \cross \BJ_{\mathrm{sup} } }
- \alpha \spacegrad \cross \BE
= \PD{t}{} \lr{ \spacegrad \cross \BJ_{\mathrm{sup} } }
- \alpha \lr{ -\PD{t}{ \BB } }.
= \spacegrad \cross \dot{\BJ}_{\mathrm{sup} } 
+ \alpha \dot{\BB}.
\end{dmath}

Taking the curl one more time

\begin{dmath}\label{eqn:emtproblemSet6Problem1:180}
0 
= \PD{t}{} \lr{ \spacegrad \cross \lr{ \spacegrad \cross \BJ_{\mathrm{sup} } } }
+ \alpha \mu \PD{t}{ } \spacegrad \cross \BH
= \PD{t}{} \lr{
-\spacegrad^2 \BJ_{\mathrm{sup}} + \spacegrad \lr{ \spacegrad \cdot \BJ_{\mathrm{sup} } }
}
+ \alpha \mu \PD{t}{ } \BJ_{\mathrm{sup}},
\end{dmath}

This is

\begin{dmath}\label{eqn:emtproblemSet6Problem1:140}
\spacegrad^2 \dot{\BJ}_{\mathrm{sup}} =
\alpha \mu \dot{\BJ}_{\mathrm{sup}}
+ \spacegrad \lr{ \spacegrad \cdot \dot{\BJ}_{\mathrm{sup} } }.
\end{dmath}

%%%\begin{dmath}\label{eqn:emtproblemSet6Problem1:80}
%%%0
%%%=
%%%\spacegrad \cross 
%%%\PD{t}{} 
%%%\BJ_{\mathrm{sup}} 
%%%-
%%%\alpha \spacegrad \cross \BE 
%%%=
%%%\PD{t}{} 
%%%\lr{
%%%\spacegrad \cross \BJ_{\mathrm{sup}} 
%%%}
%%%-
%%%\alpha \lr{ -\PD{t}{\BB} }
%%%=
%%%\PD{t}{} 
%%%\lr{
%%%\spacegrad \cross \BJ_{\mathrm{sup}} 
%%%+\alpha \BB
%%%},
%%%\end{dmath}
%%%
%%%Taking the curl twice more, again commuting time and space derivatives
%%%\begin{dmath}\label{eqn:emtproblemSet6Problem1:100}
%%%0
%%%=
%%%\spacegrad \cross \lr{ \spacegrad \cross
%%%\PD{t}{} 
%%%\lr{
%%%\spacegrad \cross \BJ_{\mathrm{sup}} 
%%%+\alpha \BB
%%%}
%%%}
%%%=
%%%\spacegrad \cross \lr{
%%%\PD{t}{} 
%%%\lr{
%%%\spacegrad \cross \lr{ \spacegrad \cross \BJ_{\mathrm{sup}}  }
%%%+\alpha \spacegrad \cross \BB
%%%}
%%%}
%%%=
%%%\spacegrad \cross \lr{
%%%\PD{t}{} 
%%%\lr{
%%%\spacegrad \cross \lr{ -\alpha_1 \BB }
%%%+\alpha \mu_0 \spacegrad \cross \BH
%%%}
%%%}
%%%=
%%%\PD{t}{} 
%%%\lr{
%%%-\alpha_1 \spacegrad \cross \lr{ \spacegrad \cross \BB }
%%%+\alpha \mu_0 \spacegrad \cross \BJ_{\mathrm{sup}}
%%%}
%%%=
%%%\PD{t}{} 
%%%\lr{
%%%-\alpha_1 \lr{ -\spacegrad^2 \BB - \spacegrad \cancel{ \spacegrad \cdot \BB } }
%%%+\alpha \mu_0 \lr{ -\alpha_1 \BB }
%%%}
%%%=
%%%\alpha_1 \lr{ \spacegrad^2 \dot{\BB} - \mu_0 \alpha \dot{\BB} },
%%%\end{dmath}
%%%
%%%or
%%%\begin{dmath}\label{eqn:emtproblemSet6Problem1:120}
%%%\spacegrad^2 \dot{\BB} = \mu_0 \alpha \dot{\BB}. \qedmarker
%%%\end{dmath}

\makeSubAnswer{}{emt:problemSet6:1b}

TODO.
\makeSubAnswer{}{emt:problemSet6:1c}

TODO.
}


   \chapter{Wave equation solutions}

In class, we walked through splitting up the wave equation into components, and separation of variables.  I didn't take notes on that.

Winding down that discussion, however, was a mention of phase and group velocity, and a phenomina called superluminal velocity.  This latter is analogous to quantum electron tunneling where a wave can make it through an aperature with a damped solution \( e^{-\alpha x} \) in the aperature interval, and sinuoidal solutions in the incident and transmitted regions as sketched in \cref{fig:L7:L7Fig1}.  The time \( \tau \) to get through the aperature is called the tunnelling time.

\imageFigure{../../figures/ece1228-emt/L7Fig1}{Superluminal tunneling.}{fig:L7:L7Fig1}{0.3}

      \section{Problems}
%%%          %
% Copyright � 2016 Peeter Joot.  All Rights Reserved.
% Licenced as described in the file LICENSE under the root directory of this GIT repository.
%
\makeproblem{Lossy waves.}{emt:problemSet6:2}{ 
In the case of lossy medium the wave equation was given by

\begin{dmath}\label{eqn:emtproblemSet6Problem2:20}
\spacegrad^2 \BE = \gamma^2 \BE,
\end{dmath}

where

\begin{dmath}\label{eqn:emtproblemSet6Problem2:40}
\gamma^2 = \lr{ \alpha + j \beta }^2.
\end{dmath}
Now consider a medium for which \( \epsilon(\omega) = \epsilon'(\omega) \) (i.e. \( \epsilon''(\omega) = 0 \)), \( \sigma = \sigma_0 \) (i.e. \(\omega \tau \sim 0 \) in the Drude model), and \( \mu \) is a constant and real.
For this case obtain the expression for \( \alpha \) and \(\beta\) in terms of \( \omega, \mu, \epsilon', \sigma_0 \).

The uniform plane wave
\begin{dmath}\label{eqn:emtproblemSet6Problem2:60}
\BE(\Br, t) = E_0
\lr{ \xcap \cos\theta - \zcap \sin\theta } \cos\lr{ \omega t -k \sin\theta x - k \cos\theta z }
\end{dmath}

is propagating in the \(x-z\) plane as sketched in \cref{fig:emtProblemSet6:emtProblemSet6Fig1} 
in a simple medium with \( \sigma = 0\).
\imageFigure{../../figures/ece1228-emt/emtProblemSet6Fig1}{Linear wave front.}{fig:emtProblemSet6:emtProblemSet6Fig1}{0.3}
Here, \( E_0 \) is a real constant and \( k \) is the propagation
constant. Answer the following questions and show all your
work.
\makesubproblem{}{emt:problemSet6:2a}
Determine the associated magnetic field \( \BH(\Br, t) \).
\makesubproblem{}{emt:problemSet6:2b}
Determine the time averaged Poynting vector, \( \expectation{\BS(\Br, t)} \).
\makesubproblem{}{emt:problemSet6:2c}
Determine the stored magnetic energy density, \( W_m(\Br, t) \).
\makesubproblem{}{emt:problemSet6:2d}
Determine the components of phase velocity vector \( \Bv_p \) along x and z.
} % makeproblem

\makeanswer{emt:problemSet6:2}{ 
\makeSubAnswer{}{emt:problemSet6:2a}
\makeSubAnswer{}{emt:problemSet6:2b}
\makeSubAnswer{}{emt:problemSet6:2c}
\makeSubAnswer{}{emt:problemSet6:2d}

TODO.
}

%%%          %
% Copyright © 2016 Peeter Joot.  All Rights Reserved.
% Licenced as described in the file LICENSE under the root directory of this GIT repository.
%

\makeproblem{Uniform plane wave.}{emt:problemSet6:3}{
\paragraph{Note:} This seemed like a separate problem, and has been split out from the problem 2 as specified in the original problem set handout.

The uniform plane wave
\begin{dmath}\label{eqn:emtproblemSet6Problem3:60}
\bcE(\Br, t) = E_0
\lr{ \xcap \cos\theta - \zcap \sin\theta } \cos\lr{ \omega t -k \sin\theta x - k \cos\theta z }
\end{dmath}

is propagating in the \(x-z\) plane as sketched in \cref{fig:emtProblemSet6:emtProblemSet6Fig1}
in a simple medium with \( \sigma = 0\).
\imageFigure{../../figures/ece1228-emt/emtProblemSet6Fig1}{Linear wave front.}{fig:emtProblemSet6:emtProblemSet6Fig1}{0.3}
Here, \( E_0 \) is a real constant and \( k \) is the propagation
constant. Answer the following questions and show all your
work.
\makesubproblem{}{emt:problemSet6:3a}
Determine the associated magnetic field \( \BH(\Br, t) \).
\makesubproblem{}{emt:problemSet6:3b}
Determine the time averaged Poynting vector, \( \expectation{\BS(\Br, t)} \).
\makesubproblem{}{emt:problemSet6:3c}
Determine the stored magnetic energy density, \( W_m(\Br, t) \).
\makesubproblem{}{emt:problemSet6:3d}
Determine the components of phase velocity vector \( \Bv_p \) along x and z.
} % makeproblem

\makeanswer{emt:problemSet6:3}{

The wave equation for \( \bcE \) is

\begin{dmath}\label{eqn:emtproblemSet6Problem3:80}
\spacegrad^2 \bcE  = \mu \epsilon \PDSq{t}{\bcE} + \mu \sigma \PD{t}{\bcE}.
\end{dmath}

With \( \sigma = 0 \) and \( E_0 \) real, the permittivity \( \epsilon \) must also be real.  In the frequency domain, this means that the waves are governed by the equations

\begin{dmath}\label{eqn:emtproblemSet6Problem3:100}
\begin{aligned}
\spacegrad^2 \BE &= - \omega^2 \mu \epsilon \BE \\
\spacegrad^2 \BH &= - \omega^2 \mu \epsilon \BH \\
\spacegrad \cross \BE &= -j \omega \mu \BH \\
\spacegrad \cross \BH &= j \omega \epsilon \BE \\
\spacegrad \cdot \BH &= 0 \\
\spacegrad \cdot \BE &= 0,
\end{aligned}
\end{dmath}

where
\begin{dmath}\label{eqn:emtproblemSet6Problem3:120}
\begin{aligned}
\BE &= E_0 \lr{ \xcap \cos\theta - \zcap \sin\theta } e^{-j \Bk \cdot \Br} \\
\Bk &= k\lr{ \xcap \sin\theta + \zcap \cos\theta }.
\end{aligned}
\end{dmath}

We also require that

\begin{dmath}\label{eqn:emtproblemSet6Problem3:140}
(-j \Bk)^2 = -\omega^2 \mu \epsilon,
\end{dmath}

or

\begin{dmath}\label{eqn:emtproblemSet6Problem3:160}
\frac{\omega}{k} = \inv{\sqrt{\mu \epsilon}}.
\end{dmath}

\makeSubAnswer{}{emt:problemSet6:3a}

In the frequency domain, the magnetic field can be obtained directly from the electric field

\begin{dmath}\label{eqn:emtproblemSet6Problem3:180}
\BH
= \inv{-j \omega \mu} \spacegrad \cross \BE,
=
\frac{E_0}{-j \omega \mu}
\begin{vmatrix}
\xcap & \ycap & \zcap \\
\partial_x & \partial_y & \partial_z \\
\cos\theta
e^{-j \Bk \cdot \Br} & 0 &
- \sin\theta
e^{-j \Bk \cdot \Br}
\end{vmatrix}
=
-\frac{E_0 \ycap}{-j \omega \mu}
\lr{
-\sin\theta \partial_x
-\cos\theta \partial_z
}
e^{-j \Bk \cdot \Br}
=
-\frac{-j k E_0 \ycap}{j \omega \mu}
\lr{
\sin\theta \sin\theta +
\cos\theta \cos\theta
}
e^{-j \Bk \cdot \Br}
=
\frac{k E_0 \ycap}{\omega \mu}
e^{-j \Bk \cdot \Br}
=
\frac{\sqrt{\mu \epsilon}E_0 \ycap}{\mu}
e^{-j \Bk \cdot \Br},
\end{dmath}

or
\boxedEquation{eqn:emtproblemSet6Problem3:200}{
\bcH(\Br, t) = \sqrt{\frac{\epsilon}{\mu}} E_0 \ycap \cos\lr{ \omega t - \Bk \cdot \Br }.
}

\makeSubAnswer{}{emt:problemSet6:3b}

The instantaneous Poynting vector is

\begin{dmath}\label{eqn:emtproblemSet6Problem3:220}
\bcS
= \bcE \cross \bcH
= \inv{4}
\lr{ \BE e^{j \omega t} + \BE^\conj e^{-j\omega t} }
\cross
\lr{ \BH e^{j \omega t} + \BH^\conj e^{-j\omega t} }
=
\inv{4}
\lr{
\BE \cross \BH^\conj + \BH \cross \BE^\conj + \BE \cross \BH e^{2 j \omega t} + \BE^\conj \cross \BH^\conj e^{-2 j \omega t}
}
=
\inv{2}
\Real \lr{
\BE \cross \BH^\conj + \BE \cross \BH e^{2 j \omega t}
},
\end{dmath}

The average is
\begin{dmath}\label{eqn:emtproblemSet6Problem3:240}
\expectation{\bcS}
=
\inv{2}
\inv{T} \int_0^T
\Real \lr{
\BE \cross \BH^\conj + \BE \cross \BH e^{2 j \omega t}
}
=
\inv{2} \Real \BE \cross \BH^\conj
=
\inv{2} E_0^2 \sqrt{\frac{\epsilon}{\mu}}
\begin{vmatrix}
\xcap & \ycap & \zcap \\
\cos\theta & 0 & -\sin\theta \\
0 & 1 & 0
\end{vmatrix}
=
\inv{2} E_0^2 \sqrt{\frac{\epsilon}{\mu}}
\lr{ \xcap \sin\theta + \zcap \cos\theta },
\end{dmath}

or
\boxedEquation{eqn:emtproblemSet6Problem3:260}{
\expectation{\bcS}
=
\inv{2} E_0^2 \sqrt{\frac{\epsilon}{\mu}} \kcap.
}

\makeSubAnswer{}{emt:problemSet6:3c}

The stored magnetic energy density is

\begin{dmath}\label{eqn:emtproblemSet6Problem3:280}
W_m
= \inv{2} \mu \Abs{\BH}^2
= \inv{2} \mu \frac{\epsilon}{\mu} E_0^2
= \inv{2} \epsilon E_0^2.
\end{dmath}

This equals the stored electric energy density, as expected.

\makeSubAnswer{}{emt:problemSet6:3d}

The phase velocity are the velocities that satisfy

\begin{dmath}\label{eqn:emtproblemSet6Problem3:300}
0
=
\ddt{} \lr{ \omega t - \Bk \cdot \Br }
=
\omega - \Bk \cdot \frac{d\Br}{dt},
\end{dmath}

or
\begin{dmath}\label{eqn:emtproblemSet6Problem3:380}
\omega =
\Bk \cdot \Bv_p
=
k_x v_x
+
k_z v_z
=
k \sin\theta v_x
+
k \cos\theta v_z.
\end{dmath}

This has many solutions, including superluminal phase velocities such as:

\begin{dmath}\label{eqn:emtproblemSet6Problem3:420}
\begin{aligned}
\Bv_p &= (\omega/(k\sin\theta), 0, 0) \\
\Bv_p &= (0, 0, \omega/(k\cos\theta)) \\
\Bv_p &= \frac{\omega}{2 k}(1/\sin\theta, 0, 1/\cos\theta)
\end{aligned}
\end{dmath}

I'm unsure if those are physically relavent.  However,
it is reasonable to assume that a solution where \( \Bv_p \) is colinear with the energy propagation direction \( \Bk \) is the solution of interest.  In that case, the components of the phase velocity are

\begin{dmath}\label{eqn:emtproblemSet6Problem3:320}
\begin{aligned}
v_x &= \frac{\omega}{k}\sin\theta \\
v_z &= \frac{\omega}{k}\cos\theta,
\end{aligned}
\end{dmath}

or

\boxedEquation{eqn:emtproblemSet6Problem3:400}{
\begin{aligned}
v_x &= c \sin\theta \\
v_z &= c \cos\theta
\end{aligned}
}

where

\begin{dmath}\label{eqn:emtproblemSet6Problem3:360}
c = \inv{\sqrt{\mu \epsilon}}.
\end{dmath}

%Both of these phase velocity components are greater than the speed of the wave.
}


   \chapter{Cylindrical coordinate wave equation solutions}
      %
% Copyright � 2016 Peeter Joot.  All Rights Reserved.
% Licenced as described in the file LICENSE under the root directory of this GIT repository.
%
\newcommand{\authorname}{Peeter Joot}
\newcommand{\email}{peeterjoot@protonmail.com}
\newcommand{\basename}{FIXMEbasenameUndefined}
\newcommand{\dirname}{notes/FIXMEdirnameUndefined/}

\renewcommand{\basename}{emt8}
\renewcommand{\dirname}{notes/ece1228/}
\newcommand{\keywords}{ECE1228H}
\newcommand{\authorname}{Peeter Joot}
\newcommand{\onlineurl}{http://sites.google.com/site/peeterjoot2/math2013/\basename.pdf}
\newcommand{\sourcepath}{\dirname\basename.tex}
\newcommand{\generatetitle}[1]{\chapter{#1}}

\newcommand{\vcsinfo}{%
\section*{}
\noindent{\color{DarkOliveGreen}{\rule{\linewidth}{0.1mm}}}
\paragraph{Document version}
%\paragraph{\color{Maroon}{Document version}}
{
\small
\begin{itemize}
\item Available online at:\\ 
\href{\onlineurl}{\onlineurl}
\item Git Repository: \input{./.revinfo/gitRepo.tex}
\item Source: \sourcepath
\item last commit: \input{./.revinfo/gitCommitString.tex}
\item commit date: \input{./.revinfo/gitCommitDate.tex}
\end{itemize}
}
}

%\PassOptionsToPackage{dvipsnames,svgnames}{xcolor}
\PassOptionsToPackage{square,numbers}{natbib}
\documentclass{scrreprt}

\usepackage[left=2cm,right=2cm]{geometry}
\usepackage[svgnames]{xcolor}
\usepackage{peeters_layout}

\usepackage{natbib}

\usepackage[
colorlinks=true,
bookmarks=false,
pdfauthor={\authorname, \email},
backref 
]{hyperref}

% http://tex.stackexchange.com/questions/75773/how-to-reference-problems-by-the-text-label-in-an-exercise-envioronment
\usepackage[english]{cleveref}
\crefname{Exercise}{exercise}{exercises}
\Crefname{Exercise}{Exercise}{Exercises}

\RequirePackage{titlesec}
\RequirePackage{ifthen}

% http://stackoverflow.com/questions/4932910/date-in-the-tabular-environment
\makeatletter
\let\insertdate\@date
\makeatother

\titleformat{\chapter}[display]
{\bfseries\Large}
{\color{DarkSlateGrey}\filleft \authorname
\ifthenelse{\isundefined{\studentnumber}}{}{\\ \studentnumber}
\ifthenelse{\isundefined{\email}}{}{\\ \email}
\ifthenelse{\isundefined{\dateintitle}}{}{\\ \insertdate}
%\ifthenelse{\isundefined{\coursename}}{}{\\ \coursename} % put in title instead.
}
{4ex}
{\color{DarkOliveGreen}{\titlerule}\color{Maroon}
\vspace{2ex}%
\filright}
[\vspace{2ex}%
\color{DarkOliveGreen}\titlerule
]

\newcommand{\beginArtWithToc}[0]{\begin{document}\tableofcontents}
\newcommand{\beginArtNoToc}[0]{\begin{document}}
\newcommand{\EndNoBibArticle}[0]{\end{document}}
\newcommand{\EndArticle}[0]{\bibliography{Bibliography}\bibliographystyle{plainnat}\end{document}}

% 
%\newcommand{\citep}[1]{\cite{#1}}

\colorSectionsForArticle



%\usepackage{ece1228}
\usepackage{peeters_braket}
%\usepackage{peeters_layout_exercise}
\usepackage{peeters_figures}
\usepackage{mathtools}
\usepackage{siunitx}

\beginArtNoToc
\generatetitle{ECE1228H Electromagnetic Theory.  Lecture 8: Waves.  Taught by Prof.\ M. Mojahedi}
%\chapter{Waves}
\label{chap:emt8}

%\paragraph{Disclaimer}
%
%Peeter's lecture notes from class.  These may be incoherent and rough.
%
%These are notes for the UofT course ECE1228H, Electromagnetic Theory, taught by Prof. M. Mojahedi, covering \textchapref{{1}} \citep{balanis1989advanced} content.

\paragraph{Cylindrical coorindate wave equation solutions}

Seek a function

\begin{dmath}\label{eqn:emtLecture8:20}
\BE = E_\rho \rhocap + E_\phi \phicap + E_z \zcap
\end{dmath}

solving 

\begin{dmath}\label{eqn:emtLecture8:40}
\spacegrad^2 \BE = -\beta^2 \BE.
\end{dmath}

One way to find the Laplacian in cylindrical coordinates is to use

\begin{dmath}\label{eqn:emtLecture8:60}
\spacegrad^2 \BE = 
\spacegrad \lr{ \spacegrad \cdot \BE }
-\spacegrad \cross \lr{ \spacegrad \cross \BE },
\end{dmath}

where 

\begin{dmath}\label{eqn:emtLecture8:80}
\spacegrad = \rhocap \PD{\rho}{} + \frac{\phicap}{\rho} \PD{\phi}{} + \zcap \PD{z}{}
\end{dmath}

Can be shown that:
\begin{dmath}\label{eqn:emtLecture8:100}
\spacegrad \cdot \BE = \inv{\rho} \PD{\rho}{} \lr{ \rho E_\rho } + \inv{\rho}\PD{\phi}{E_\phi} + \PD{z}{E_z}
\end{dmath}

and
\begin{dmath}\label{eqn:emtLecture8:120}
\spacegrad \cross \BE 
%= 
%\begin{vmatrix}
%\rhocap & \phicap & \zcap \\
%\partial_\rho & \inv{\rho}\partial_\phi & \partial_z \\
%E_\rho & \rho E_\phi & E_z
%\end{vmatrix}
=
\rhocap  \lr{ \inv{\rho} \partial_\phi E_z - \partial_z E_\phi }
+\phicap \lr{ \partial_z E_\rho - \partial_\rho E_z }
+\zcap   \lr{ \inv{\rho} \partial_\rho (\rho E_\phi) - \inv{\rho} \partial_\phi E_\rho }
\end{dmath}

This gives
\begin{dmath}\label{eqn:emtLecture8:200}
\spacegrad^2 \psi = 
\PDSq{\rho}{\psi}
+\inv{\rho} \PD{\rho}{\psi}
+\inv{\rho^2} \PDSq{\phi}{\psi}
+\PDSq{z}{\psi}.
\end{dmath}

and
\begin{dmath}\label{eqn:emtLecture8:220}
\begin{aligned}
\spacegrad^2 E_\rho &= \lr{ -\frac{E_\rho}{\rho^2} - \frac{2}{\rho^2} \PD{\phi}{E_\phi} } \\
\spacegrad^2 E_\phi &= \lr{ -\frac{E_\phi}{\rho^2} + \frac{2}{\rho^2} \PD{\phi}{E_\rho} } \\
\spacegrad^2 E_z    &= -\beta^2 E_\phi.
\end{aligned}
\end{dmath}

%Note that with \( i = \Be_1 \Be_2 \),
%
%\begin{dmath}\label{eqn:emtLecture8:140}
%\rhocap = \Be_1 e^{i \phi}
%\end{dmath}
%
%so
%\begin{equation}\label{eqn:emtLecture8:160}
%\PD{\phi}{\rhocap} = \Be_2 e^{i \phi} = \thetacap
%\end{equation}
%
%... the end result is
%
%\begin{dmath}\label{eqn:emtLecture8:180}
%\end{dmath}

\paragraph{TEM:} If we want to have a TEM mode it can be shown that we need an axial distribution mechanism, such as the core of a co-axial cable.

These are messy to solve in general, but we can solve the z-component without too much pain

\begin{dmath}\label{eqn:emtLecture8:240}
\PDSq{\rho}{E_z}
+\inv{\rho} \PD{\rho}{E_z}
+\inv{\rho^2} \PDSq{\phi}{E_z}
+\PDSq{z}{E_z}
=
-\beta^2 E_z
\end{dmath}

Solving this using separation of variables with

\begin{dmath}\label{eqn:emtLecture8:260}
E_z = R(\rho) P(\phi) Z(z)
\end{dmath}

\begin{dmath}\label{eqn:emtLecture8:280}
\inv{R}\lr{R'' + \inv{\rho} R'} + \inv{\rho^2 P} P'' + \frac{Z''}{Z} = -\beta^2
\end{dmath}

Assuming for some constant \( \beta_z \) that we have
\begin{dmath}\label{eqn:emtLecture8:300}
\frac{Z''}{Z} = -\beta_z^2,
\end{dmath}

then

\begin{dmath}\label{eqn:emtLecture8:320}
\inv{R}\lr{\rho^2 R'' + \rho R'} + \inv{P} P'' + \rho^2 \lr{\beta^2 - \beta_z^2} = 0
\end{dmath}

Now assume that
\begin{dmath}\label{eqn:emtLecture8:340}
\inv{P} P'' = -m^2,
\end{dmath}

and let \( \beta^2 - \beta_z^2 = \beta_\rho^2 \), which leaves

\begin{dmath}\label{eqn:emtLecture8:360}
\rho^2 R'' + \rho R' + \lr{ \rho^2 \beta_\rho^2 -m^2 } R = 0.
\end{dmath}

This is the Bessel differential equation, with travelling wave solution

\begin{dmath}\label{eqn:emtLecture8:380}
R(\rho) = 
A H_m^{(1)}(\beta_\rho \rho) 
+B H_m^{(2)}(\beta_\rho \rho),
\end{dmath}

and standing wave solutions
\begin{dmath}\label{eqn:emtLecture8:400}
R(\rho) = 
A J_m(\beta_\rho \rho)
+B Y_m(\beta_\rho \rho).
\end{dmath}

Here \( H_m^{(1)}, H_m^{(2)} \) are Hankel functions of the first and second kinds, and
\( J_m, Y_m \) are the Bessel functions of the first and second kinds.

For \( P(\phi) \) 
\begin{dmath}\label{eqn:emtLecture8:460}
P'' = -m^2 P
\end{dmath}

\paragraph{Quadropole potential}

In Jackson
\citep{jackson1975cew}
, the prove of which is scattered through chapter 3, is the following

\begin{dmath}\label{eqn:emtLecture8:420}
\inv{\Abs{\Bx - \Bx'}}
= 
4 \pi \sum_{l= 0}^\infty \sum_{m = -l}^l \inv{2 l + 1} \frac{(r')^l}{r^{l+1}} 
Y^\conj_{l,m}(\theta', \phi')
Y_{l,m}(\theta, \phi),
\end{dmath}

where \( Y_{l,m} \) are the spherical harmonics.  Plugging this into the potential we have

\begin{dmath}\label{eqn:emtLecture8:440}
\phi(\Bx) 
= \inv{4 \pi \epsilon_0} \int \frac{\rho(\Bx') d^3 x'}{\Abs{\Bx - \Bx'}}
= 
\inv{4 \pi \epsilon_0} \int \rho(\Bx') d^3 x' \lr{
4 \pi \sum_{l= 0}^\infty \sum_{m = -l}^l \inv{2 l + 1} \frac{(r')^l}{r^{l+1}} 
Y^\conj_{l,m}(\theta', \phi')
Y_{l,m}(\theta, \phi)
}
= 
\inv{\epsilon_0} 
\sum_{l= 0}^\infty \sum_{m = -l}^l \inv{2 l + 1} 
\int \rho(\Bx') d^3 x' \lr{
\frac{(r')^l}{r^{l+1}} 
Y^\conj_{l,m}(\theta', \phi')
Y_{l,m}(\theta, \phi)
}
= 
\inv{\epsilon_0} 
\sum_{l= 0}^\infty \sum_{m = -l}^l \inv{2 l + 1} 
\lr{ 
\int (r')^l \rho(\Bx') 
Y^\conj_{l,m}(\theta', \phi')
d^3 x' 
}
\frac{
Y_{l,m}(\theta, \phi)
}
{
r^{l+1}
}
\end{dmath}

The integral terms are called the coefficients of the multipole moments, denoted
\begin{dmath}\label{eqn:emtLecture8:480}
q_{l,m} = 
\int (r')^l \rho(\Bx') 
Y^\conj_{l,m}(\theta', \phi')
d^3 x',
\end{dmath}

The \( l = 0,1,2\) terms are, respectively, called the monopole, dipole, and quadropole terms of the potential
\begin{dmath}\label{eqn:emtLecture8:500}
\rho(\Bx) =
\inv{4 \pi \epsilon_0} 
\sum_{l= 0}^\infty \sum_{m = -l}^l \frac{4\pi} {2 l + 1} 
q_{l,m}
\frac{
Y_{l,m}(\theta, \phi)
}
{
r^{l+1}
}.
\end{dmath}

Note the power of this expansion.  Should we wish to compute the electric field, we have only to compute the qradient of  the last (\(Y_{l,m} r^{-l-1} \)) portion (since \( q_{l,m} \) is a constant).
 
\begin{dmath}\label{eqn:emtLecture8:520}
q_{1,1}
= 
-\int \sqrt{\frac{3}{8 \pi}} \sin\theta' e^{-i\phi'} r' \rho(\Bx') dV'
=
-\sqrt{\frac{3}{8 \pi}} \int \sin\theta' \lr{ \cos\phi' - i\sin\phi'} r' \rho(\Bx') dV'
=
-\sqrt{\frac{3}{8 \pi}} \lr{ 
\int x' \rho(\Bx') dV'
-i \int y' \rho(\Bx') dV'
}
=
-\sqrt{\frac{3}{8 \pi}} \lr{ 
p_x - i p_y
}.
\end{dmath}

Here we've used
\begin{dmath}\label{eqn:emtLecture8:540}
\begin{aligned}
x' &= r' \sin\theta' \cos\phi' \\
y' &= r' \sin\theta' \sin\phi' \\
z' &= r' \cos\theta'
\end{aligned}
\end{dmath}

and the \( Y_{11} \) representation

\begin{dmath}\label{eqn:emtLecture8:560}
\begin{aligned}
Y_{00} &= -\sqrt{\frac{1}{4 \pi}} \\
Y_{11} &= -\sqrt{\frac{3}{8 \pi}} \sin\theta e^{i\phi} \\
Y_{10} &=  \sqrt{\frac{3}{4 \pi}} \cos\theta  \\
Y_{22} &= -\inv{4} \sqrt{\frac{15}{2 \pi}} \sin^2\theta e^{2 i\phi} \\
Y_{21} &=  \inv{2} \sqrt{\frac{15}{2 \pi}} \sin\theta \cos\theta e^{i\phi} \\
Y_{20} &=  \inv{4} \sqrt{\frac{5}{\pi}} \lr{ 3 \cos^2\theta - 1 } \\
\end{aligned}
\end{dmath}

%HW: compute a few of the more tedious moment coeffients.  These have been exam questions in the past.

With the usual dipole moment expression

\begin{dmath}\label{eqn:emtLecture8:580}
\Bp = \int \Bx' \rho(\Bx') d^3 x',
\end{dmath}

and a quadropole moment defined as
\begin{dmath}\label{eqn:emtLecture8:600}
Q_{i,j} = \int \lr{ 3 x_i' x_j' - \delta_{ij} (r')^2 } \rho(\Bx') d^3 x',
\end{dmath}

the first order terms of the potential are now fully specified
\begin{dmath}\label{eqn:emtLecture8:620}
\phi(\Bx)
=
\inv{4 \pi \epsilon_0}
\lr{ 
q + \frac{\Bp \cdot \Bx}{r^3} + 
\inv{2} \sum_{ij} Q_{ij} \frac{x_i x_j}{r^5}
}.
\end{dmath}

\paragraph{Waves}

\begin{itemize}
\item The field is a modification of space-time
\item Mode is a particular field configuration for a given boundary value problem.  Many field configurations can satisfy Maxwell equations (wave equation).  These usually are referrred to as modes.  A mode is a self-consistent field distribution.
\item In a TEM mode, \( \BE \) and \( \BH \) are every point in space are constrained in a local plane, independent of time.  This plane is called the equiphase plane.  In general equiphase planes are not parallel at two different points along the trajectory of the wave.
\item If equiphase planes are parallel (i.e. the space orientation of the planes for TEM mode...
... next time.
\end{itemize}

%}
\EndArticle
%\EndNoBibArticle


   \chapter{Quadrupole expansion}
      %
% Copyright © 2016 Peeter Joot.  All Rights Reserved.
% Licenced as described in the file LICENSE under the root directory of this GIT repository.
%

\paragraph{Quadropole potential}

In Jackson
\citep{jackson1975cew}
,
is the following

\begin{dmath}\label{eqn:emtLecture8:420}
\inv{\Abs{\Bx - \Bx'}}
=
4 \pi \sum_{l= 0}^\infty \sum_{m = -l}^l \inv{2 l + 1} \frac{(r')^l}{r^{l+1}}
Y^\conj_{l,m}(\theta', \phi')
Y_{l,m}(\theta, \phi),
\end{dmath}

where \( Y_{l,m} \) are the spherical harmonics.  It appears that this is actually just an orthogonal function expansion of the inverse distance (for a region outside of the charge density).  The proof of this in is scattered through chapter 3, dependent on a similar expansion in Legendre polynomials, for an the azimuthally symmetric configuration.

It looks like quite a project to get comfortable enough with these special functions to fully reproduce the proof of this identity.  We are forced to play engineer, and assume the mathematics works out.  If we do that and plug this inverse distance formula into
the potential we have

\begin{dmath}\label{eqn:emtLecture8:440}
\phi(\Bx)
= \inv{4 \pi \epsilon_0} \int \frac{\rho(\Bx') d^3 x'}{\Abs{\Bx - \Bx'}}
=
\inv{4 \pi \epsilon_0} \int \rho(\Bx') d^3 x' \lr{
4 \pi \sum_{l= 0}^\infty \sum_{m = -l}^l \inv{2 l + 1} \frac{(r')^l}{r^{l+1}}
Y^\conj_{l,m}(\theta', \phi')
Y_{l,m}(\theta, \phi)
}
=
\inv{\epsilon_0}
\sum_{l= 0}^\infty \sum_{m = -l}^l \inv{2 l + 1}
\int \rho(\Bx') d^3 x' \lr{
\frac{(r')^l}{r^{l+1}}
Y^\conj_{l,m}(\theta', \phi')
Y_{l,m}(\theta, \phi)
}
=
\inv{\epsilon_0}
\sum_{l= 0}^\infty \sum_{m = -l}^l \inv{2 l + 1}
\lr{
\int (r')^l \rho(\Bx')
Y^\conj_{l,m}(\theta', \phi')
d^3 x'
}
\frac{
Y_{l,m}(\theta, \phi)
}
{
r^{l+1}
}
\end{dmath}

The integral terms are called the coefficients of the multipole moments, denoted
\begin{dmath}\label{eqn:emtLecture8:480}
q_{l,m} =
\int (r')^l \rho(\Bx')
Y^\conj_{l,m}(\theta', \phi')
d^3 x',
\end{dmath}

The \( l = 0,1,2\) terms are, respectively, called the monopole, dipole, and quadropole terms of the potential
\begin{dmath}\label{eqn:emtLecture8:500}
\rho(\Bx) =
\inv{4 \pi \epsilon_0}
\sum_{l= 0}^\infty \sum_{m = -l}^l \frac{4\pi} {2 l + 1}
q_{l,m}
\frac{
Y_{l,m}(\theta, \phi)
}
{
r^{l+1}
}.
\end{dmath}

Note the power of this expansion.  Should we wish to compute the electric field, we have only to compute the qradient of  the last (\(Y_{l,m} r^{-l-1} \)) portion (since \( q_{l,m} \) is a constant).

\begin{dmath}\label{eqn:emtLecture8:520}
q_{1,1}
=
-\int \sqrt{\frac{3}{8 \pi}} \sin\theta' e^{-i\phi'} r' \rho(\Bx') dV'
=
-\sqrt{\frac{3}{8 \pi}} \int \sin\theta' \lr{ \cos\phi' - i\sin\phi'} r' \rho(\Bx') dV'
=
-\sqrt{\frac{3}{8 \pi}} \lr{
\int x' \rho(\Bx') dV'
-i \int y' \rho(\Bx') dV'
}
=
-\sqrt{\frac{3}{8 \pi}} \lr{
p_x - i p_y
}.
\end{dmath}

Here we've used
\begin{dmath}\label{eqn:emtLecture8:540}
\begin{aligned}
x' &= r' \sin\theta' \cos\phi' \\
y' &= r' \sin\theta' \sin\phi' \\
z' &= r' \cos\theta'
\end{aligned}
\end{dmath}

and the \( Y_{11} \) representation

\begin{dmath}\label{eqn:emtLecture8:560}
\begin{aligned}
Y_{00} &= -\sqrt{\frac{1}{4 \pi}} \\
Y_{11} &= -\sqrt{\frac{3}{8 \pi}} \sin\theta e^{i\phi} \\
Y_{10} &=  \sqrt{\frac{3}{4 \pi}} \cos\theta  \\
Y_{22} &= -\inv{4} \sqrt{\frac{15}{2 \pi}} \sin^2\theta e^{2 i\phi} \\
Y_{21} &=  \inv{2} \sqrt{\frac{15}{2 \pi}} \sin\theta \cos\theta e^{i\phi} \\
Y_{20} &=  \inv{4} \sqrt{\frac{5}{\pi}} \lr{ 3 \cos^2\theta - 1 } \\
\end{aligned}
\end{dmath}

%NOTE: compute a few of the more tedious moment coeffients.  These have been exam questions in the past.

With the usual dipole moment expression

\begin{dmath}\label{eqn:emtLecture8:580}
\Bp = \int \Bx' \rho(\Bx') d^3 x',
\end{dmath}

and a quadropole moment defined as
\begin{dmath}\label{eqn:emtLecture8:600}
Q_{i,j} = \int \lr{ 3 x_i' x_j' - \delta_{ij} (r')^2 } \rho(\Bx') d^3 x',
\end{dmath}

the first order terms of the potential are now fully specified
\begin{dmath}\label{eqn:emtLecture8:620}
\phi(\Bx)
=
\inv{4 \pi \epsilon_0}
\lr{
q + \frac{\Bp \cdot \Bx}{r^3} +
\inv{2} \sum_{ij} Q_{ij} \frac{x_i x_j}{r^5}
}.
\end{dmath}

      %
% Copyright � 2016 Peeter Joot.  All Rights Reserved.
% Licenced as described in the file LICENSE under the root directory of this GIT repository.
%
%{
%\newcommand{\authorname}{Peeter Joot}
\newcommand{\email}{peeterjoot@protonmail.com}
\newcommand{\basename}{FIXMEbasenameUndefined}
\newcommand{\dirname}{notes/FIXMEdirnameUndefined/}

%\renewcommand{\basename}{momentCoeffiecients}
%%\renewcommand{\dirname}{notes/phy1520/}
%\renewcommand{\dirname}{notes/ece1228-electromagnetic-theory/}
%%\newcommand{\dateintitle}{}
%%\newcommand{\keywords}{}
%
%\newcommand{\authorname}{Peeter Joot}
\newcommand{\onlineurl}{http://sites.google.com/site/peeterjoot2/math2013/\basename.pdf}
\newcommand{\sourcepath}{\dirname\basename.tex}
\newcommand{\generatetitle}[1]{\chapter{#1}}

\newcommand{\vcsinfo}{%
\section*{}
\noindent{\color{DarkOliveGreen}{\rule{\linewidth}{0.1mm}}}
\paragraph{Document version}
%\paragraph{\color{Maroon}{Document version}}
{
\small
\begin{itemize}
\item Available online at:\\ 
\href{\onlineurl}{\onlineurl}
\item Git Repository: \input{./.revinfo/gitRepo.tex}
\item Source: \sourcepath
\item last commit: \input{./.revinfo/gitCommitString.tex}
\item commit date: \input{./.revinfo/gitCommitDate.tex}
\end{itemize}
}
}

%\PassOptionsToPackage{dvipsnames,svgnames}{xcolor}
\PassOptionsToPackage{square,numbers}{natbib}
\documentclass{scrreprt}

\usepackage[left=2cm,right=2cm]{geometry}
\usepackage[svgnames]{xcolor}
\usepackage{peeters_layout}

\usepackage{natbib}

\usepackage[
colorlinks=true,
bookmarks=false,
pdfauthor={\authorname, \email},
backref 
]{hyperref}

% http://tex.stackexchange.com/questions/75773/how-to-reference-problems-by-the-text-label-in-an-exercise-envioronment
\usepackage[english]{cleveref}
\crefname{Exercise}{exercise}{exercises}
\Crefname{Exercise}{Exercise}{Exercises}

\RequirePackage{titlesec}
\RequirePackage{ifthen}

% http://stackoverflow.com/questions/4932910/date-in-the-tabular-environment
\makeatletter
\let\insertdate\@date
\makeatother

\titleformat{\chapter}[display]
{\bfseries\Large}
{\color{DarkSlateGrey}\filleft \authorname
\ifthenelse{\isundefined{\studentnumber}}{}{\\ \studentnumber}
\ifthenelse{\isundefined{\email}}{}{\\ \email}
\ifthenelse{\isundefined{\dateintitle}}{}{\\ \insertdate}
%\ifthenelse{\isundefined{\coursename}}{}{\\ \coursename} % put in title instead.
}
{4ex}
{\color{DarkOliveGreen}{\titlerule}\color{Maroon}
\vspace{2ex}%
\filright}
[\vspace{2ex}%
\color{DarkOliveGreen}\titlerule
]

\newcommand{\beginArtWithToc}[0]{\begin{document}\tableofcontents}
\newcommand{\beginArtNoToc}[0]{\begin{document}}
\newcommand{\EndNoBibArticle}[0]{\end{document}}
\newcommand{\EndArticle}[0]{\bibliography{Bibliography}\bibliographystyle{plainnat}\end{document}}

% 
%\newcommand{\citep}[1]{\cite{#1}}

\colorSectionsForArticle


%
%\usepackage{peeters_layout_exercise}
%\usepackage{peeters_braket}
%\usepackage{peeters_figures}
%\usepackage{siunitx}
%%\usepackage{txfonts} % \ointclockwise
%
%\beginArtNoToc
%
%\generatetitle{Dipole and Quadrupole electrostatic potential moments and coefficients}
%\chapter{Dipole and Quadropole electrostatic potential moments and coefficents}
%\label{chap:momentCoeffiecients}

\paragraph{Explicit moment and quadrupole expansion}

We calculated the \( q_{1,1} \) coefficient of the electrostatic moment, as covered in \citep{jackson1975cew} chapter 4.  Let's verify the rest, as well as the tensor sum formula for the quadrupole moment, and the spherical harmonic sum that yields the dipole moment potential.

%%%XX
%%%\begin{dmath}\label{eqn:momentCoeffiecients:20}
%%%q_{l,m} =
%%%\int (r')^l \rho(\Bx')
%%%Y^\conj_{l,m}(\theta', \phi')
%%%d^3 x',
%%%\end{dmath}
%%%
%%%The class notes also give the results for \( q_{0,0}, q_{1,0}, q_{2,2}, q_{2,1}, q_{2,0} \).  Let's verify those
%%%
%%%\paragraph{\(q_{0,0}\)}
%%%
%%%\begin{dmath}\label{eqn:momentCoeffiecients:40}
%%%q_{0,0}
%%%=
%%%\int (r')^0 \rho(\Bx')
%%%Y^\conj_{0,0}(\theta', \phi')
%%%d^3 x'
%%%=
%%%\inv{4\pi}
%%%\int \rho(\Bx') d^3 x'
%%%=
%%%\frac{q}{4\pi}
%%%\end{dmath}
%%%
%%%\paragraph{\(q_{1,0}\)}
%%%
%%%\begin{dmath}\label{eqn:momentCoeffiecients:60}
%%%q_{1,0}
%%%=
%%%\int r' \rho(\Bx')
%%%Y^\conj_{1,0}(\theta', \phi')
%%%d^3 x'
%%%=
%%%\sqrt{\frac{3}{4\pi}}
%%%\int r' \rho(\Bx')
%%%\cos\theta'
%%%d^3 x'
%%%=
%%%\sqrt{\frac{3}{4\pi}}
%%%\int r' \rho(\Bx') \cos\theta' d^3 x'
%%%=
%%%\sqrt{\frac{3}{4\pi}}
%%%\int z' \rho(\Bx') d^3 x'
%%%=
%%%\sqrt{\frac{3}{4\pi}} p_z
%%%\end{dmath}
%%%
%%%\paragraph{\(q_{2,2}\)}
%%%
%%%\begin{dmath}\label{eqn:momentCoeffiecients:80}
%%%q_{2,2}
%%%=
%%%\int (r')^2 \rho(\Bx')
%%%Y^\conj_{2,2}(\theta', \phi')
%%%d^3 x'
%%%=
%%%\sqrt{\frac{15}{32 \pi}}
%%%\int (r')^2 \rho(\Bx')
%%%\sin^2 \theta e^{-2 i\phi}
%%%d^3 x'
%%%=
%%%\sqrt{\frac{15}{32 \pi}}
%%%\int (r')^2 \rho(\Bx')
%%%\sin^2 \theta \lr{ \cos \phi - i \sin\phi }^2
%%%d^3 x'
%%%=
%%%\sqrt{\frac{15}{32 \pi}}
%%%\int (r')^2 \rho(\Bx')
%%%\sin^2 \theta \lr{ \cos^2\phi - \sin^2\phi - 2 i \cos\phi \sin\phi }
%%%d^3 x'
%%%=
%%%\sqrt{\frac{15}{32 \pi}}
%%%\int \rho(\Bx') \lr{
%%%(x')^2
%%%- (y')^2
%%%- 2 i x' y' } d^3 x'
%%%=
%%%\sqrt{\frac{15}{32 \pi}}
%%%\int \rho(\Bx') \lr{
%%%x' - i y'
%%%}^2 d^3 x'
%%%%=
%%%%\sqrt{\frac{15}{32 \pi}}
%%%%\lr{ p_x - i p_y }^2
%%%\end{dmath}
%%%
%%%\paragraph{\(q_{2,1}\)}
%%%
%%%\begin{dmath}\label{eqn:momentCoeffiecients:100}
%%%q_{2,1}
%%%=
%%%\int (r')^2 \rho(\Bx')
%%%Y^\conj_{2,1}(\theta', \phi')
%%%d^3 x'
%%%=
%%%-\sqrt{\frac{15}{8 \pi}}
%%%\int (r')^2 \rho(\Bx')
%%%\sin\theta' \cos\theta' e^{-i \phi}
%%%d^3 x'
%%%=
%%%-\sqrt{\frac{15}{8 \pi}}
%%%\int (r')^2 \rho(\Bx')
%%%\sin\theta' \cos\theta' \lr{ \cos\phi - i \sin\phi }
%%%d^3 x'
%%%=
%%%-\sqrt{\frac{15}{8 \pi}}
%%%\int \rho(\Bx')
%%%\lr{ x' z' - i y' z' }
%%%d^3 x'
%%%%=
%%%%-\sqrt{\frac{15}{8 \pi}} p_z \lr{ p_x - i p_y }.
%%%\end{dmath}
%%%
%%%\paragraph{\(q_{2,0}\)}
%%%
%%%\begin{dmath}\label{eqn:momentCoeffiecients:260}
%%%q_{2,0}
%%%=
%%%\int (r')^2 \rho(\Bx')
%%%Y^\conj_{2,0}(\theta', \phi')
%%%d^3 x'
%%%=
%%%\int (r')^2 \rho(\Bx') \sqrt{\frac{5}{4\pi}} \lr{ \frac{3}{2} \cos^2\theta - \inv{2} }
%%%d^3 x'
%%%=
%%%\inv{2} \sqrt{\frac{5}{4\pi}}
%%%\int \rho(\Bx')
%%%\lr{ 3 (z')^2 - (r')^2 }
%%%d^3 x'.
%%%\end{dmath}
%%%
%%%\paragraph{\(Q_{ij}\)}
%%%XX
The quadrupole term of the potential was stated to be

\begin{dmath}\label{eqn:momentCoeffiecients:120}
\inv{4 \pi \epsilon_0} \frac{4 \pi}{5 r^3} \sum_{m=-2}^2 \int (r')^2 \rho(\Bx') Y_{lm}^\conj(\theta', \phi') Y_{lm}(\theta, \phi)
=
\inv{2} \sum_{ij} Q_{ij} \frac{x_i x_j}{r^5},
\end{dmath}

where

\begin{dmath}\label{eqn:momentCoeffiecients:140}
Q_{i,j} = \int \lr{ 3 x_i' x_j' - \delta_{ij} (r')^2 } \rho(\Bx') d^3 x'.
\end{dmath}

Let's verify this.  First note that

\begin{dmath}\label{eqn:momentCoeffiecients:160}
Y_{l,m} = \sqrt{\frac{2 l + 1}{4 \pi} \frac{(l-m)!}{(l+m)!}} P_l^m(\cos\theta) e^{i m \phi},
\end{dmath}

and
\begin{dmath}\label{eqn:momentCoeffiecients:180}
P_l^{-m}(x) =
(-1)^m \frac{(l-m)!}{(l+m)!} P_l^m(x),
\end{dmath}

so
\begin{dmath}\label{eqn:momentCoeffiecients:200}
Y_{l,-m}
= \sqrt{\frac{2 l + 1}{4 \pi} \frac{(l+m)!}{(l-m)!} }
P_l^{-m}(\cos\theta)
e^{-i m \phi}
=
(-1)^m
\sqrt{\frac{2 l + 1}{4 \pi} \frac{(l-m)!}{(l+m)!} }
P_l^m(x)
e^{-i m \phi}
=
(-1)^m Y_{l,m}^\conj.
\end{dmath}

That means

\begin{dmath}\label{eqn:momentCoeffiecients:220}
q_{l,-m}
=
\int (r')^l \rho(\Bx')
Y^\conj_{l,-m}(\theta', \phi')
d^3 x'
=
(-1)^m
\int (r')^l \rho(\Bx')
Y_{l,m}(\theta', \phi')
d^3 x'
=
(-1)^m q_{lm}^\conj.
\end{dmath}

In particular, for \( m \ne 0 \)

\begin{dmath}\label{eqn:momentCoeffiecients:320}
(r')^l Y_{l, m}^\conj (\theta', \phi') r^l Y_{l, m}(\theta, \phi)
+ (r')^l Y_{l, -m}^\conj (\theta', \phi') r^l Y_{l, -m}(\theta, \phi)
=
(r')^l Y_{l, m}^\conj (\theta', \phi') r^l Y_{l, m}(\theta, \phi)
+ (r')^l Y_{l, m} (\theta', \phi') r^l Y_{l, m}^\conj(\theta, \phi) ,
\end{dmath}

or
\begin{dmath}\label{eqn:momentCoeffiecients:340}
(r')^l Y_{l, m}^\conj (\theta', \phi') r^l Y_{l, m}(\theta, \phi)
+ (r')^l Y_{l, -m}^\conj (\theta', \phi') r^l Y_{l, -m}(\theta, \phi)
=
2 \Real \lr{ (r')^l Y_{l, m}^\conj (\theta', \phi') r^l Y_{l, m}(\theta, \phi) }.
\end{dmath}

To verify the quadrupole expansion formula in a compact way it is helpful to compute some intermediate results.

\begin{dmath}\label{eqn:momentCoeffiecients:360}
r Y_{1, 1}
= -r \sqrt{\frac{3}{8 \pi}} \sin\theta e^{i\phi}
= -\sqrt{\frac{3}{8 \pi}} (x + i y),
\end{dmath}

\begin{dmath}\label{eqn:momentCoeffiecients:380}
r Y_{1, 0}
= r \sqrt{\frac{3}{4 \pi}} \cos\theta
= \sqrt{\frac{3}{4 \pi}} z,
\end{dmath}

\begin{dmath}\label{eqn:momentCoeffiecients:400}
r^2 Y_{2, 2}
= -r^2 \sqrt{\frac{15}{32 \pi}} \sin^2\theta e^{2 i\phi}
= - \sqrt{\frac{15}{32 \pi}} (x + i y)^2,
\end{dmath}

\begin{dmath}\label{eqn:momentCoeffiecients:420}
r^2 Y_{2, 1}
= r^2 \sqrt{\frac{15}{8 \pi}} \sin\theta \cos\theta e^{i\phi}
= \sqrt{\frac{15}{8 \pi}} z ( x + i y ),
\end{dmath}

\begin{dmath}\label{eqn:momentCoeffiecients:440}
r^2 Y_{2, 0}
= r^2 \sqrt{\frac{5}{16 \pi}} \lr{ 3 \cos^2\theta - 1 }
= \sqrt{\frac{5}{16 \pi}} \lr{ 3 z^2 - r^2 }.
\end{dmath}

Given primed coordinates and integrating the conjugate of each of these with \( \rho(\Bx') dV' \), we obtain the \( q_{lm} \) moment coefficients.  Those are

\begin{dmath}\label{eqn:momentCoeffiecients:460}
q_{11}
= -\sqrt{\frac{3}{8 \pi}} \int d^3 x' \rho(\Bx') (x - i y),
\end{dmath}

\begin{dmath}\label{eqn:momentCoeffiecients:480}
q_{1, 0}
= \sqrt{\frac{3}{4 \pi}} \int d^3 x' \rho(\Bx') z',
\end{dmath}

\begin{dmath}\label{eqn:momentCoeffiecients:500}
q_{2, 2}
= - \sqrt{\frac{15}{32 \pi}} \int d^3 x' \rho(\Bx') (x' - i y')^2,
\end{dmath}

\begin{dmath}\label{eqn:momentCoeffiecients:520}
q_{2, 1}
= \sqrt{\frac{15}{8 \pi}} \int d^3 x' \rho(\Bx') z' ( x' - i y' ),
\end{dmath}

\begin{dmath}\label{eqn:momentCoeffiecients:540}
q_{2, 0}
= \sqrt{\frac{5}{16 \pi}} \int d^3 x' \rho(\Bx') \lr{ 3 (z')^2 - (r')^2 }.
\end{dmath}

For the potential we are interested in

\begin{dmath}\label{eqn:momentCoeffiecients:560}
2 \Real q_{11} r^2 Y_{11}(\theta, \phi)
= 2 \frac{3}{8 \pi} \int d^3 x' \rho(\Bx') \Real \lr{ (x' - i y')( x + i y) }
= \frac{3}{4 \pi} \int d^3 x' \rho(\Bx') \lr{ x x' + y y' },
\end{dmath}

\begin{dmath}\label{eqn:momentCoeffiecients:580}
q_{1, 0} r Y_{1,0}(\theta, \phi)
= \frac{3}{4 \pi} \int d^3 x' \rho(\Bx') z' z,
\end{dmath}

\begin{dmath}\label{eqn:momentCoeffiecients:600}
2 \Real q_{22} r^2 Y_{22}(\theta, \phi)
= 2 \frac{15}{32 \pi} \int d^3 x' \rho(\Bx') \Real \lr{
(x' - i y')^2
(x + i y)^2
}
= \frac{15}{16 \pi} \int d^3 x' \rho(\Bx') \Real \lr{
((x')^2 - 2 i x' y' -(y')^2)
(x^2 + 2 i x y -y^2)
}
= \frac{15}{16 \pi} \int d^3 x' \rho(\Bx') \lr{
((x')^2 -(y')^2) (x^2 -y^2)
+ 4 x x' y y'
},
\end{dmath}

\begin{dmath}\label{eqn:momentCoeffiecients:620}
2 \Real q_{21} r^2 Y_{21}(\theta, \phi)
= 2 \frac{15}{8 \pi} \int d^3 x' \rho(\Bx') z \Real \lr{ ( x' - i y' ) (x + i y) }
= \frac{15}{4 \pi} \int d^3 x' \rho(\Bx') z \lr{ x x' + y y' },
\end{dmath}

and
\begin{dmath}\label{eqn:momentCoeffiecients:640}
q_{2, 0} r^2 Y_{20}(\theta, \phi)
= \frac{5}{16 \pi} \int d^3 x' \rho(\Bx') \lr{ 3 (z')^2 - (r')^2 } \lr{ 3 z^2 - r^2 }.
\end{dmath}

The dipole term of the potential is

\begin{dmath}\label{eqn:momentCoeffiecients:660}
\inv{ 4 \pi \epsilon_0 } \frac{4 \pi}{3 r^3}
\lr{
\frac{3}{4 \pi} \int d^3 x' \rho(\Bx') \lr{ x x' + y y' }
+
\frac{3}{4 \pi} \int d^3 x' \rho(\Bx') z' z
}
=
\inv{ 4 \pi \epsilon_0 r^3}
\Bx \cdot \int d^3 x' \rho(\Bx') \Bx'
=
\frac{\Bx \cdot \Bp}{ 4 \pi \epsilon_0 r^3},
\end{dmath}

as obtained directly when a strict dipole approximation was used.

Summing all the terms for the quadrupole gives

\begin{dmath}\label{eqn:momentCoeffiecients:680}
\begin{aligned}
\inv{ 4 \pi \epsilon r^5 } \frac{ 4 \pi }{5}
\biglr{
&\frac{15}{16 \pi} \int d^3 x' \rho(\Bx') \lr{
((x')^2 -(y')^2) (x^2 -y^2)
+ 4 x x' y y'
} \\
&+
\frac{15}{4 \pi} \int d^3 x' \rho(\Bx') z z' \lr{ x x' + y y' } \\
&+
\frac{5}{16 \pi} \int d^3 x' \rho(\Bx') \lr{ 3 (z')^2 - (r')^2 } \lr{ 3 z^2 - r^2 }
} \\
=
\inv{ 4 \pi \epsilon r^5 }
\int d^3 x' \rho(\Bx')
\inv{4}
\biglr{
   &3
   \lr{
   ((x')^2 -(y')^2) (x^2 -y^2)
   + 4 x x' y y'
   } \\
   &+
   12
   z z' \lr{ x x' + y y' } \\
   &+
   \lr{ 3 (z')^2 - (r')^2 } \lr{ 3 z^2 - r^2 }
}.
\end{aligned}
\end{dmath}

The portion in brackets is

\begin{dmath}\label{eqn:momentCoeffiecients:700}
\begin{aligned}
   3
   &\lr{
      ((x')^2 -(y')^2) (x^2 -y^2)
      + 4 x x' y y'
   } \\
   +
   12
   & z z' \lr{ x x' + y y' }  \\
   +
   &\lr{ 2 (z')^2 - (x')^2 - (y')^2} \lr{ 2 z^2 - x^2 -y^2 } \\
=
x^2 &\lr{
     3 (x')^2 - 3(y')^2
-
   \lr{ 2 (z')^2 - (x')^2 - (y')^2}
} \\
+
y^2 &\lr{
      -3 (x')^2 + 3 (y')^2
-
   \lr{ 2 (z')^2 - (x')^2 - (y')^2}
} \\
+
2 z^2 &\lr{
   2 (z')^2 - (x')^2 - (y')^2
} \\
+
&12{ x x' y y' + x x' z z' + y y' z z' } \\
=
2 x^2 &\lr{
     2 (x')^2 - (y')^2 - (z')^2
} \\
+
2 y^2 &\lr{
     2 (y')^2 - (x')^2 - (z')^2
} \\
+
2 z^2 &\lr{
   2 (z')^2 - (x')^2 - (y')^2
} \\
+
&12{ x x' y y' + x x' z z' + y y' z z' }.
\end{aligned}
\end{dmath}

The quadrupole sum can now be written as
\begin{dmath}\label{eqn:momentCoeffiecients:720}
\inv{2}
\inv{ 4 \pi \epsilon r^5 }
\int d^3 x' \rho(\Bx')
\biglr{
x^2 \lr{ 3 (x')^2 - (r')^2 }
+y^2 \lr{ 3 (y')^2 - (r')^2 }
+z^2 \lr{ 3 (z')^2 - (r')^2 }
+
3 \lr{
x y x' y'
+y x y' x'
+x z x' z'
+z x z' x'
+y z y' z'
+z y z' y'
}
},
\end{dmath}

which is precisely \cref{eqn:momentCoeffiecients:120}, the quadrupole potential stated in the text and class notes.

%}
%\EndArticle

