\section{Appendix I.  Current loop integral off axis.}

Initially I was curious what the current loop magnetic field integral would look like in general, allowing for an off axis observation point.

I found it natural to do that compuation using Geometric Algebra to express vector rotation in a plane and the other geometrical constructs of this problem.  The basic rules in that Algebra are that unit vectors square to unity (\(\Be_k^2 = 1 \)), and that orthogonal vectors anticommute (\( \Be_1 \Be_2 = -\Be_2 \Be_1 \)).  For example, letting \( i = \Be_1 \Be_2 \) the radial unit vector can be expessed as

\begin{dmath}\label{eqn:emtProblemSet3Problem2:160}
\rhocap(\theta)
=
\Be_1 e^{i \theta}
= \Be_1 \lr{ \cos\theta + \Be_1 \Be_2 \sin\theta } 
= \Be_1 \cos\theta + (\Be_1^2) \Be_2 \sin\theta
= \Be_1 \cos\theta + \Be_2 \sin\theta,
\end{dmath}

and the \( \thetacap \) direction vector is
\begin{dmath}\label{eqn:emtProblemSet3Appendix:n}
\thetacap(\theta)
=
\Be_2 e^{i \theta}
= \Be_2 \lr{ \cos\theta + \Be_1 \Be_2 \sin\theta } 
= \Be_2 \cos\theta + \Be_2 \Be_1 \Be_2 \sin\theta
= \Be_2 \cos\theta + \Be_2 (-\Be_2 \Be_1) \sin\theta
= \Be_2 \cos\theta - \Be_1 \sin\theta.
\end{dmath}

This allows for a compact expression of an off-axis observation point

\begin{dmath}\label{eqn:emtProblemSet3Problem2:60}
\Br = z \Be_3 + \rho \Be_1 e^{i\theta}.
\end{dmath}

Similarly, the charge point is
\begin{dmath}\label{eqn:emtProblemSet3Problem2:80}
\Br' = R_l \Be_1 e^{i \theta'},
\end{dmath}

and the element of the loop path is
\begin{dmath}\label{eqn:emtProblemSet3Problem2:100}
d\Bl' = R_l \Be_2 e^{i\theta'} d\theta'.
\end{dmath}

The difference vector from the charge position to the observation point is

\begin{dmath}\label{eqn:emtProblemSet3Problem2:120}
\BR 
= \Br - \Br'
=
z \Be_3 + \rho \Be_1 e^{i\theta}
-
R_l \Be_1 e^{i \theta'},
\end{dmath}

with squared length

\begin{dmath}\label{eqn:emtProblemSet3Problem2:140}
\BR^2 
=
z^2 + 
\lr{ \rho \Be_1 e^{i\theta}
-
R_l \Be_1 e^{i \theta'}
}
\cdot
\lr{ \rho \Be_1 e^{i\theta}
-
R_l \Be_1 e^{i \theta'}
}
=
z^2 + \rho^2 + R_l^2 - 2 \rho R_l \cos\lr{ \theta - \theta' }.
\end{dmath}

For the cross product, using a bivector duality transformation \( \Ba \cross \Bb = -\Be_1 \Be_2 \Be_3 (\Ba \wedge \Bb) \), and expressing the wedge product as a grade two selection, we have

\begin{dmath}\label{eqn:emtProblemSet3Problem2:180}
d\Bl' \cross \BR 
=
-\Be_1 \Be_2 \Be_3 R_l d\theta' \gpgradetwo{ 
\Be_2 e^{i \theta'} 
\lr{
z \Be_3 + \rho \Be_1 e^{i\theta}
-
R_l \Be_1 e^{i \theta'}
}
}
=
R_l d\theta' \lr{ 
z \Be_1 e^{i\theta'}
-
\Be_3 \rho \cos\lr{ \theta - \theta' }
+ \Be_3 R_l
}.
\end{dmath}

The final integral can now be assembled

\boxedEquation{eqn:emtProblemSet3Appendix:220}{
%\begin{dmath}\label{eqn:emtProblemSet3Problem2:200}
\BB = \frac{I \mu_0 R_l}{4\pi} \int_0^{2\pi} d\theta' 
\frac
{ z \Be_1 e^{i\theta'} - \Be_3 \rho \cos\lr{ \theta - \theta' } + \Be_3 R_l }
{ \lr{z^2 + \rho^2 + R_l^2 - 2 \rho R_l \cos\lr{ \theta - \theta' }}^{3/2} }.
%\end{dmath}
}

This is consistent with the traditional vector algebra derivation that led to \cref{eqn:emtProblemSet3Problem2:201} where \( \rho = 0 \) was assumed.
It is clear now, why the problem statement asked only to consider the z-axis observation points where \( \rho = 0 \).  With \( \theta' \) dependencies in the denominator, performing the integral above for \( \rho \ne 0 \) looks spectacularly unpleasant.

\section{Appendix II.  Normal and tangential decomposition.}

The decomposition of \cref{eqn:emtProblemSet3Problem3:60} can be derived easily using Geometric Algebra

\begin{dmath}\label{eqn:emtProblemSet3Problem3:80}
\BA 
= 
\ncap^2 \BA
=
\ncap (\ncap \cdot \BA)
+\ncap (\ncap \wedge \BA)
%=
%\ncap (\ncap \cdot \BA)
%+
%\ncap \cdot (\ncap \wedge \BA)
\end{dmath}

The last dot product can be expanded as a grade one (vector) selection

\begin{dmath}\label{eqn:emtProblemSet3Problem3:100}
\ncap (\ncap \wedge \BA)
=
\gpgradeone{
\ncap (\ncap \wedge \BA)
}
=
\gpgradeone{
\ncap I (\ncap \cross \BA)
}
=
I^2 \ncap \cross (\ncap \cross \BA)
=
- \ncap \cross (\ncap \cross \BA),
\end{dmath}

so the decomposition of a vector \( \BA \) in terms of its normal and tangential projections is
\begin{dmath}\label{eqn:emtProblemSet3Problem3:120}
\BA
=
\ncap (\ncap \cdot \BA)
-
\ncap \cross (\ncap \cross \BA).
\end{dmath}

I'm not sure how to derive this easily using traditional vector algebra, but it can be verified by expanding the triple cross product in coordinates using tensor contraction formalism

\begin{dmath}\label{eqn:emtProblemSet3Problem3:140}
-\ncap \cross (\ncap \cross \BA)
=
-\epsilon_{xyz} \Be_x n_y \lr{\ncap \cross \BA}_z
=
-\epsilon_{xyz} \Be_x n_y \epsilon_{zrs} n_r A_s
=
-\delta_{xy}^{[rs]}
\Be_x n_y n_r A_s
=
-\Be_x n_y \lr{ n_x A_y -n_y A_x }
= -\ncap (\ncap \cdot \BA) + (\ncap \cdot \ncap) \BA
= \BA - \ncap (\ncap \cdot \BA).
\end{dmath}

This last statement illustrates the geometry of this decomposition, showing that the tangential projection (or normal rejection) of a vector is really just the vector minus its normal projection.

%This can be rearranged to show that the 
%\begin{dmath}\label{eqn:emtProblemSet3Problem3:100}

%The tangential projection, can also be expanded in dot products
%
%\begin{dmath}\label{eqn:emtProblemSet3Problem3:200}
%\ncap (\ncap \wedge \BA)
%=
%\ncap \cdot (\ncap \wedge \BA)
%=
%\BA - \ncap (\ncap \cdot \BA)
%\end{dmath}
