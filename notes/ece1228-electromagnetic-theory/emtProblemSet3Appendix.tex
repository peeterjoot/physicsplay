%
% Copyright © 2016 Peeter Joot.  All Rights Reserved.
% Licenced as described in the file LICENSE under the root directory of this GIT repository.
%
%\section{Appendix II.  Problem 3.  Normal and tangential decomposition.}

The decomposition of \cref{eqn:emtProblemSet3Problem3:60} can be derived easily using Geometric Algebra

\begin{dmath}\label{eqn:emtProblemSet3Problem3:80}
\BA 
= 
\ncap^2 \BA
=
\ncap (\ncap \cdot \BA)
+\ncap (\ncap \wedge \BA)
%=
%\ncap (\ncap \cdot \BA)
%+
%\ncap \cdot (\ncap \wedge \BA)
\end{dmath}

The last dot product can be expanded as a grade one (vector) selection

\begin{dmath}\label{eqn:emtProblemSet3Problem3:100}
\ncap (\ncap \wedge \BA)
=
\gpgradeone{
\ncap (\ncap \wedge \BA)
}
=
\gpgradeone{
\ncap I (\ncap \cross \BA)
}
=
I^2 \ncap \cross (\ncap \cross \BA)
=
- \ncap \cross (\ncap \cross \BA),
\end{dmath}

so the decomposition of a vector \( \BA \) in terms of its normal and tangential projections is
\begin{dmath}\label{eqn:emtProblemSet3Problem3:120}
\BA
=
\ncap (\ncap \cdot \BA)
-
\ncap \cross (\ncap \cross \BA).
\end{dmath}

I'm not sure how to naturally determine this relationship using traditional vector algebra.  However, it can be verified by expanding the triple cross product in coordinates using tensor contraction formalism

\begin{dmath}\label{eqn:emtProblemSet3Problem3:140}
-\ncap \cross (\ncap \cross \BA)
=
-\epsilon_{xyz} \Be_x n_y \lr{\ncap \cross \BA}_z
=
-\epsilon_{xyz} \Be_x n_y \epsilon_{zrs} n_r A_s
=
-\delta_{xy}^{[rs]}
\Be_x n_y n_r A_s
=
-\Be_x n_y \lr{ n_x A_y -n_y A_x }
= -\ncap (\ncap \cdot \BA) + (\ncap \cdot \ncap) \BA
= \BA - \ncap (\ncap \cdot \BA).
\end{dmath}

This last statement illustrates the geometry of this decomposition, showing that the tangential projection (or normal rejection) of a vector is really just the vector minus its normal projection.

%This can be rearranged to show that the 
%\begin{dmath}\label{eqn:emtProblemSet3Problem3:100}

%The tangential projection, can also be expanded in dot products
%
%\begin{dmath}\label{eqn:emtProblemSet3Problem3:200}
%\ncap (\ncap \wedge \BA)
%=
%\ncap \cdot (\ncap \wedge \BA)
%=
%\BA - \ncap (\ncap \cdot \BA)
%\end{dmath}
