%
% Copyright � 2016 Peeter Joot.  All Rights Reserved.
% Licenced as described in the file LICENSE under the root directory of this GIT repository.
%
\makeproblem{Helmholtz theorem}{emt:problemSet1:5}{ 
Prove the first Helmholtz's theorem, i.e. if vector \(\BM_1\) is defined by its divergence

\begin{dmath}\label{eqn:emtProblemSet1Problem5:20}
\spacegrad \cdot \BM_1 = s
\end{dmath}

and its curl
\begin{dmath}\label{eqn:emtProblemSet1Problem5:40}
\spacegrad \cross \BM_1 = \BC 
\end{dmath}

within a region and its normal component \( \BM_{1\txtn} \) over the boundary, then \( \BM_1 \) is 
uniquely specified.

Note: Assume there is a vector \( \BM_2 \) with its divergence and curl equal to \( s \) and \( \BC \)
respectively, then show that \( \BM_1 = \BM_2 \) .

%\makesubproblem{}{emt:problemSet1:5a}
} % makeproblem

%Here's a homework question from ece1228 that screams for solution using Geometric Algebra techniques.  If I submitted such a solution, then my Prof probably wouldn't know what I was doing, so I'll probably also have to try to solve it another way too.

\makeanswer{emt:problemSet1:5}{ 

This can be demonstrated by inversion, which screams for an attempt using Geometric Algebra techniques.
Dropping the suffix from \( \BM_1 \), the gradient of the vector can be written as a single even grade multivector

\begin{dmath}\label{eqn:emtProblemSet1Problem5:60}
\spacegrad \BM
= \spacegrad \cdot \BM + I \spacegrad \cross \BM
= s + I \BC.
\end{dmath}

Unlike the divergence or curl, the gradient is invertible, with the \R{3} Green's function

\begin{dmath}\label{eqn:emtProblemSet1Problem5:80}
\begin{aligned}
G(\Bx ; \By) &= \inv{4 \pi} \frac{ \Bx - \By }{\Abs{\Bx - \By}^3} \\
\spacegrad \BG &= \spacegrad \cdot \BG = \delta(\Bx - \By).
\end{aligned}
\end{dmath}

This Green's function result is taken from \citep{doran2003gap}, where it is used to generalize the Cauchy integral equations to higher dimensions.

The inversion equation is an application of the Fundamental Theorem of (Geometric) Calculus

\begin{dmath}\label{eqn:emtProblemSet1Problem5:100}
\oint_{\partial V} G d^2 \Bx \BM
=
\int_V G d^3 \Bx \lrspacegrad \BM 
=
\int_V d^3 \Bx (G \spacegrad) \BM 
+
\int_V d^3 \Bx G (\spacegrad \BM)
=
\int_V d^3 \Bx \delta(\Bx - \By) \BM 
+
\int_V d^3 \Bx G \lr{ s + I \BC }.
\end{dmath}

In the first volume integral the gradient operates bidirectionally on both \( G \) and \( \BM \).  Because \( d^3 \Bx \propto I \) in \R{3} we can commute it with any grades.  With \( d^3 \Bx = I dV \), and \( d^2 \Bx \ncap = I dA \), we have

\begin{dmath}\label{eqn:emtProblemSet1Problem5:120}
\BM(\By)
=
\inv{4\pi} \oint_{\partial V} dA \frac{ \Bx - \By }{\Abs{\Bx - \By}^3} \ncap \BM
-
\inv{4\pi} \int_V dV \frac{ \Bx - \By }{\Abs{\Bx - \By}^3} \lr{ s + I \BC }.
\end{dmath}

\paragraph{Trivector components.}
We expect the trivectors grades on the RHS to sum to zero.  For that to occur, after setting
\( \Br = \Bx - \By \)
we must have

\begin{dmath}\label{eqn:emtProblemSet1Problem5:160}
\int_V dV \frac{ \rcap \cdot \BC }{ r^2 }
=
\oint_{\partial V} dA \inv{ r^2 } I ( \rcap \wedge \ncap \wedge \BM )
=
\oint_{\partial V} dA \inv{ r^2 } (\ncap \cross \rcap) \cdot \BM.
\end{dmath}

If we require that \( \lim_{\Bx' \rightarrow \infty} \ifrac{\Abs{\BM(\Bx')}}{\Abs{\Bx - \Bx'}^2 } \rightarrow 0 \)
, the area integral is killed in the limit.  Somehow the LHS integral must be zero.  It isn't clear that \( \BC \cdot \rcap/r^2 \) should be zero, however, this expression is vaguely reminicent of an integrand related to \( \spacegrad \cdot \BC \).
Looking back to the definition
\cref{eqn:emtProblemSet1Problem5:40}, it can be observed that 
\( \BC \) cannot be arbitrary.  A divergence constraint must be imposted on \( \BC \).  In particular, because \( \BC = \spacegrad \cross \BM_1 \), and \( \spacegrad \cdot (\spacegrad \cross \BM_1) = 0 \), the vector \( \BC \) is required to have zero divergence.  That fact can be used to show that the 
remaining integral in the trivector grade selection is zero.  To do so first observe that

\begin{equation}\label{eqn:emtProblemSet1Problem5:260}
\spacegrad \inv{\Abs{\Br - \Br'}} 
=
-\spacegrad' \inv{\Abs{\Br - \Br'}} 
=
-\frac{\Bx - \Bx'}{\Abs{\Bx - \Bx'}^3}.
\end{equation}

With this result, the zero divergence requirement for \( \BC \), and the chain rule can be used to rewrite the LHS of \cref{eqn:emtProblemSet1Problem5:160}, we have

\begin{dmath}\label{eqn:emtProblemSet1Problem5:280}
\int_V dV \frac{ \rcap \cdot \BC(\Bx') }{ r^2 }
=
\int_V dV \spacegrad' \inv{\Abs{\Bx - \Bx'} } \cdot \BC(\Bx')
=
\int_V dV \spacegrad' \cdot \frac{\BC(\Bx')}{\Abs{\Bx - \Bx'} }
-\int_V dV \frac{\spacegrad' \cdot \BC(\Bx')}{ \Abs{\Bx - \Bx'} }
=
\int_{\partial V} d\BA \cdot \frac{\BC(\Bx')}{\Abs{\Bx - \Bx'} }
-\int_V dV \frac{\spacegrad' \cdot \BC(\Bx')}{ \Abs{\Bx - \Bx'} }.
\end{dmath}

Requiring that \( \lim_{\Bx' \rightarrow \infty} \ifrac{\BC(\Bx')}{\Abs{\Bx - \Bx'} } \rightarrow 0 \) and imposing the zero divergence requirement on \( \BC \), we see that all the trivector grade contributions in \cref{eqn:emtProblemSet1Problem5:120} are zero.

\paragraph{Vector components.}
Now that we know any trivector grades of the multivector \cref{eqn:emtProblemSet1Problem5:120} are zero, the only remaining contributions are the vector components

\begin{dmath}\label{eqn:emtProblemSet1Problem5:121}
\BM(\By)
=
\inv{4\pi} \oint_{\partial V} dA \inv{ r^2 } \gpgradeone{ \rcap \ncap \BM }
-
\inv{4\pi} \int_V dV \frac{ s \rcap }{ r^2 } 
-
\inv{4\pi} \int_V dV \frac{ \rcap \cdot (I\BC) }{ r^2 }
=
\inv{4\pi} \oint_{\partial V} dA \inv{ r^2 } \gpgradeone{ \rcap \ncap \BM }
+
\inv{4\pi} \int_V dV \frac{ \rcap \cross \BC - s \rcap }{ r^2 },
\end{dmath}

where the cross product comes from the expansion

\begin{dmath}\label{eqn:emtProblemSet1Problem5:140}
\rcap \cdot (I\BC)
=
\gpgradeone{ \rcap I\BC }
=
I (\rcap \wedge \BC)
=
-\rcap \cross \BC.
\end{dmath}

The vector grade selection can be expanded as 

\begin{dmath}\label{eqn:emtProblemSet1Problem5:220}
\gpgradeone{ \rcap \ncap \BM }
=
\lr{ \rcap (\ncap \cdot \BM) + \rcap \cdot (\ncap \wedge \BM) }
=
\rcap (\ncap \cdot \BM) + (\rcap \cdot \ncap) \BM - \ncap ( \rcap \cdot \BM ).
\end{dmath}

This looks like it is probably largest when all of \( \rcap, \ncap, \BM \) are colinear, so will likely have magnitude \( \gpgradeone{ \rcap \ncap \BM } < \Abs{\BM} \).  What we can say for sure is 

\begin{dmath}\label{eqn:emtProblemSet1Problem5:240}
\Abs{\lr{ \inv{4\pi} \oint_{\partial V} dA \inv{ r^2 } \gpgradeone{ \rcap \ncap \BM } }}
\ge
\inv{4\pi} \oint_{\partial V} dA \inv{ r^2 } \Abs{ \gpgradeone{ \rcap \ncap \BM } }
\ge
\inv{4\pi} \oint_{\partial V} dA \inv{ r^2 } 3 \Abs{ \BM },
\end{dmath}

so if \( \lim_{\Bx' \rightarrow \infty} \ifrac{\Abs{\BM(\Bx')}}{\Abs{\Bx - \Bx'}^2 } \rightarrow 0 \), this whole surface integral will be killed,
leaving \( \BM \) uniquely determined by a weighted integral of the curl and divergence values \( \BC \) and \( s \) respectively

%\begin{boxed}\label{eqn:emtProblemSet1Problem5:200}
\boxedEquation{eqn:emtProblemSet1Problem5:200}{
\BM(\Bx)
=
\inv{4\pi} \int_V dV' \frac{ \BC(\Bx') \cross (\Bx - \Bx') + (\Bx - \Bx') s(\Bx')}{ \Abs{\Bx - \Bx'}^3 }.
}
%\end{boxed}
}
