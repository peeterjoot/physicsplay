%
% Copyright � 2016 Peeter Joot.  All Rights Reserved.
% Licenced as described in the file LICENSE under the root directory of this GIT repository.
%
\makeproblem{Helmholtz theorem}{emt:problemSet1:5}{
\index{Helmholtz's theorem}
Prove the first Helmholtz's theorem, i.e. if vector \(\BM\) is defined by its divergence

\begin{dmath}\label{eqn:emtProblemSet1Problem5:20}
\spacegrad \cdot \BM = s
\end{dmath}

and its curl
\begin{dmath}\label{eqn:emtProblemSet1Problem5:40}
\spacegrad \cross \BM = \BC
\end{dmath}

within a region and its normal component \( \BM_{\txtn} \) over the boundary, then \( \BM \) is
uniquely specified.

%Note: Assume there is a vector \( \BN \) with its divergence and curl equal to \( s \) and \( \BC \) respectively, then show that \( \BM = \BN \) .
} % makeproblem

\makeanswer{emt:problemSet1:5}{

%I attempted this problem in two different ways.  My first approach assembled the divergence and curl relations above into a single (Geometric Algebra) multivector gradient equation and applied the vector valued Green's function for the gradient to invert that equation.  That approach logically led from the differential equation for \( \BM \) to the solution for \( \BM \) in terms of \( s \) and \( \BC \).  However, this strategy introduced some complexities
%that make me doubt the correctness of the associated boundary analysis.
%
%Even if the details of the boundary handling in my multivector approach is not correct, I thought that approach was interesting enough to share, and have placed it in an appendix to this problem set.  It is accompanied with a primer on Geometric Algebra that is hopefully enough to allow the reader to grasp the basic ideas of the approach, but is probably not sufficient to understand all the details without further study.
%
%The answer obtained in that first attempt at this problem, when \( \Abs{\BM}/r^2 \), and \( \Abs{\BC}/r \) both vanish on an infinite sphere, is that the field has a unique value determined by \( s \) and \( \BC \), namely
%
%\begin{dmath}\label{eqn:emtProblemSet1Problem5:60}
%\BM =
%-\spacegrad \int_V dV' \frac{ s(\Bx')}{ 4 \pi \Abs{\Bx - \Bx'} }
%+\spacegrad \cross \int_V dV' \frac{ \BC(\Bx') }{ 4 \pi \Abs{\Bx - \Bx'} }.
%\end{dmath}
%
%It's possible to work backwards from this result to obtain second order gradient terms applied to \( \BM(\Bx')/\Abs{\Bx - \Bx'} \) .  This suggests that a Laplacian (i.e. scalar) representation of the delta function may be a superior way to tackle this problem, perhaps also yielding a simpler result for the boundary term.  This is in fact the case, and the logical starting point 

We can compute the relationship between the vector \( \BM \) using a convolution representation of the vector function \( \BM \) and a Laplacian representations of the delta function

\begin{dmath}\label{eqn:emtProblemSet1Problem5:80}
\BM(\Bx) = \int_V dV' \delta(\Bx - \Bx') \BM(\Bx').
\end{dmath}

\index{Laplacian}
\index{delta function!Laplacian}
The Laplacian representation of the delta function in \R{3} is

\begin{dmath}\label{eqn:emtProblemSet1Problem5:100}
\delta(\Bx - \Bx') = -\inv{4\pi} \spacegrad^2 \inv{\Abs{\Bx - \Bx'}},
\end{dmath}

\index{convolution}
so \( \BM \) can be represented as the following convolution

\begin{dmath}\label{eqn:emtProblemSet1Problem5:120}
\BM(\Bx) = -\inv{4\pi} \int_V dV' \spacegrad^2 \inv{\Abs{\Bx - \Bx'}} \BM(\Bx').
\end{dmath}

As noted in \cref{eqn:usefulFormulas:700}
%\cref{eqn:emtProblemSet1Appendix:460}
the Laplacian of a vector can be factored as

\begin{dmath}\label{eqn:emtProblemSet1Problem5:140}
\spacegrad^2 \Ba
=
\spacegrad (\spacegrad \cdot \Ba)
-
\spacegrad \cross (\spacegrad \cross \Ba).
\end{dmath}

%Note that the last term can be written in cross product notation using \( \Bc \cdot (\Ba \wedge \Bb) = -\Bc \cross (\Ba \cross \Bb) \) if desired.

Using this relation and proceeding with a few applications of the chain rule, plus the fact that \(
\spacegrad 1/\Abs{\Bx - \Bx'} = -\spacegrad' 1/\Abs{\Bx - \Bx'} \), we find

\begin{dmath}\label{eqn:emtProblemSet1Problem5:160}
-4 \pi \BM(\Bx)
= \int_V dV' \spacegrad^2 \inv{\Abs{\Bx - \Bx'}} \BM(\Bx')
=
\spacegrad \int_V dV' \spacegrad \cdot \inv{\Abs{\Bx - \Bx'}} \BM(\Bx')
- \spacegrad \cross \int_V dV' \spacegrad \cross \inv{\Abs{\Bx - \Bx'}} \BM(\Bx')
=
-\spacegrad \int_V dV' \lr{ \spacegrad' \cdot \inv{\Abs{\Bx - \Bx'}}} \BM(\Bx')
+\spacegrad \cross \int_V dV' \lr{ \spacegrad' \inv{\Abs{\Bx - \Bx'}} } \cross \BM(\Bx')
=
-\spacegrad \int_V dV' \spacegrad' \cdot \frac{ \BM(\Bx') }{\Abs{\Bx - \Bx'}}
+\spacegrad \int_V dV' \frac{ \spacegrad' \cdot \BM(\Bx')}{\Abs{\Bx - \Bx'}}
+\spacegrad \cross \int_V dV' \spacegrad' \cross \frac{\BM(\Bx')}{\Abs{\Bx - \Bx'}}
-\spacegrad \cross \int_V dV' \frac{ \spacegrad' \cross \BM(\Bx')}{\Abs{\Bx - \Bx'}}
%=
%-\spacegrad \int_{\partial V} dA' \ncap \cdot \frac{ \BM(\Bx') }{\Abs{\Bx - \Bx'}}
%+\spacegrad \int_V dV' \frac{ s(\Bx')}{\Abs{\Bx - \Bx'}}
%-\spacegrad \cdot \int_{\partial V} dV' \ncap \wedge \frac{\BM(\Bx')}{\Abs{\Bx - \Bx'}}
%+\spacegrad \cdot \int_V dV' \frac{ I \BC(\Bx')}{\Abs{\Bx - \Bx'}}
,
\end{dmath}

or

%\begin{dmath}\label{eqn:emtProblemSet1Problem5:180}
\boxedEquation{eqn:emtProblemSet1Problem5:200}{
\BM(\Bx)
=
-\spacegrad \int_V dV' \frac{ s(\Bx')}{\Abs{\Bx - \Bx'}}
+\spacegrad \cross \int_V dV' \frac{ \BC(\Bx')}{\Abs{\Bx - \Bx'}}
+\spacegrad \int_{\partial V} dA' \ncap \cdot \frac{ \BM(\Bx') }{\Abs{\Bx - \Bx'}}
-\spacegrad \cross \int_{\partial V} dV' \ncap \cross \frac{\BM(\Bx')}{\Abs{\Bx - \Bx'}}
.
%\end{dmath}
}

The vector \( \BM \) is unique if \( \Abs{\BM(\Bx')}/\Abs{\Bx - \Bx'} \) vanishes on the infinite sphere.  For solutions valid in and on a specific surface, the general solution is found by a superposition of the unique infinite-sphere solution, with a specific integral equation solution dependent on the normal and tangential components of \( \BM \) on the surface of interest.
}
