%
% Copyright � 2016 Peeter Joot.  All Rights Reserved.
% Licenced as described in the file LICENSE under the root directory of this GIT repository.
%
\makeproblem{Helmholtz theorem}{emt:problemSet1:5}{ 
Prove the first Helmholtz's theorem, i.e. if vector \(\BM_1\) is defined by its divergence

\begin{dmath}\label{eqn:emtProblemSet1Problem5:20}
\spacegrad \cdot \BM_1 = s
\end{dmath}

and its curl
\begin{dmath}\label{eqn:emtProblemSet1Problem5:40}
\spacegrad \cross \BM_1 = \BC 
\end{dmath}

within a region and its normal component \( \BM_{1\txtn} \) over the boundary, then \( \BM_1 \) is 
uniquely specified.

Note: Assume there is a vector \( \BM_2 \) with its divergence and curl equal to \( s \) and \( \BC \)
respectively, then show that \( \BM_1 = \BM_2 \) .

%\makesubproblem{}{emt:problemSet1:5a}
} % makeproblem

%Here's a homework question from ece1228 that screams for solution using Geometric Algebra techniques.  If I submitted such a solution, then my Prof probably wouldn't know what I was doing, so I'll probably also have to try to solve it another way too.

\makeanswer{emt:problemSet1:5}{ 

This can be demonstrated by inversion, which screams for an attempt using Geometric Algebra techniques.
Dropping the suffix from \( \BM_1 \), the gradient of the vector can be written as a single even grade multivector

\begin{dmath}\label{eqn:emtProblemSet1Problem5:60}
\spacegrad \BM
= \spacegrad \cdot \BM + I \spacegrad \cross \BM
= s + I \BC.
\end{dmath}

Unlike the divergence or curl, the gradient is invertible, with the \R{3} Green's function

\begin{dmath}\label{eqn:emtProblemSet1Problem5:80}
\begin{aligned}
G(\Bx ; \By) &= \inv{4 \pi} \frac{ \Bx - \By }{\Abs{\Bx - \By}^3} \\
\spacegrad \BG &= \spacegrad \cdot \BG = \delta(\Bx - \By).
\end{aligned}
\end{dmath}

This Green's function result is taken from \citep{doran2003gap}, where it is used to generalize the Cauchy integral equations to higher dimensions.

The inversion equation is an application of the Fundamental Theorem of (Geometric) Calculus

\begin{dmath}\label{eqn:emtProblemSet1Problem5:100}
\oint_{\partial V} G d^2 \Bx \BM
=
\int_V G d^3 \Bx \lrspacegrad \BM 
=
\int_V d^3 \Bx (G \spacegrad) \BM 
+
\int_V d^3 \Bx G (\spacegrad \BM)
=
\int_V d^3 \Bx \delta(\Bx - \By) \BM 
+
\int_V d^3 \Bx G \lr{ s + I \BC }.
\end{dmath}

In the first volume integral the gradient operates bidirectionally on both \( G \) and \( \BM \).  Because \( d^3 \Bx \propto I \) in \R{3} we can commute it with any grades.  With \( d^3 \Bx = I dV \), and \( d^2 \Bx \ncap = I dA \), we have

\begin{dmath}\label{eqn:emtProblemSet1Problem5:120}
\BM(\By)
=
\inv{4\pi} \oint_{\partial V} dA \frac{ \Bx - \By }{\Abs{\Bx - \By}^3} \ncap \BM
-
\inv{4\pi} \int_V dV \frac{ \Bx - \By }{\Abs{\Bx - \By}^3} \lr{ s + I \BC }.
\end{dmath}

Integrating the surface integral over a sphere centered on \( \By \), after setting \( \Br = \Bx - \By = r \ncap \), the integrand can be written as

\begin{dmath}\label{eqn:emtProblemSet1Problem5:320}
\inv{4\pi} \frac{ \Bx - \By }{\Abs{\Bx - \By}^3} \ncap \BM
=
\inv{4\pi} \frac{ r \ncap }{ r^3 } \ncap \BM
=
\inv{4\pi} \frac{ \BM(\By) }{ r^2 }.
\end{dmath}

Provided the following constraint on the absolute value of the vector function \( \BM \) is imposed on the infinite sphere 

\begin{dmath}\label{eqn:emtProblemSet1Problem5:340}
\lim_{\By \rightarrow \infty} \frac{\Abs{\BM(\By)}}{\Abs{\Bx - \By}^2 } \rightarrow 0,
\end{dmath}

then this surface integral is killed in the limit, leaving

\begin{dmath}\label{eqn:emtProblemSet1Problem5:360}
\BM(\By)
=
-
\inv{4\pi} \int_V dV \frac{ \Bx - \By }{\Abs{\Bx - \By}^3} \lr{ s + I \BC }.
\end{dmath}

\paragraph{Trivector components.}

Any trivector grades in the remaining integral must be zero.  That is 

\begin{dmath}\label{eqn:emtProblemSet1Problem5:160}
0 
=
-\inv{4\pi} \int_V dV \frac{ \gpgradethree{\rcap I \BC} }{ r^2 }
=
-\inv{4\pi} \int_V dV \frac{ I (\rcap \cdot \BC) }{ r^2 }
\end{dmath}

It isn't clear that \( \BC \cdot \rcap/r^2 \) should be zero, however, this expression is vaguely reminicent of an integrand related to \( \spacegrad \cdot \BC \).
Looking back to the definition
\cref{eqn:emtProblemSet1Problem5:40}, it can be observed that 
\( \BC \) cannot be arbitrary.  A divergence constraint must be imposted on \( \BC \).  In particular, because \( \BC = \spacegrad \cross \BM_1 \), and \( \spacegrad \cdot (\spacegrad \cross \BM_1) = 0 \), the vector \( \BC \) is required to have zero divergence.  That fact can be exploited by noting

\begin{equation}\label{eqn:emtProblemSet1Problem5:260}
\spacegrad \inv{\Abs{\Bx - \By}} 
=
-\spacegrad' \inv{\Abs{\Bx - \By}} 
=
-\frac{\Bx - \By}{\Abs{\Bx - \By}^3}.
\end{equation}

With this result, the zero divergence requirement for \( \BC \), and the chain rule can be used to rewrite \cref{eqn:emtProblemSet1Problem5:160}

\begin{dmath}\label{eqn:emtProblemSet1Problem5:280}
\int_V dV \frac{ \rcap \cdot \BC(\By) }{ r^2 }
=
\int_V dV \spacegrad' \inv{\Abs{\Bx - \By} } \cdot \BC(\By)
=
\int_V dV \spacegrad' \cdot \frac{\BC(\By)}{\Abs{\Bx - \By} }
-\int_V dV \frac{\spacegrad' \cdot \BC(\By)}{ \Abs{\Bx - \By} }
=
\int_{\partial V} d\BA \cdot \frac{\BC(\By)}{\Abs{\Bx - \By} }
-\int_V dV \frac{\spacegrad' \cdot \BC(\By)}{ \Abs{\Bx - \By} }.
\end{dmath}

The zero divergence requirement for \( \BC \) kills the second integral.  The first is killed if an additional constraint is imposed on the system

\begin{dmath}\label{eqn:emtProblemSet1Problem5:380}
\lim_{\By \rightarrow \infty} \frac{\BC(\By)}{\Abs{\Bx - \By} } \rightarrow 0.
\end{dmath}

That constraint ensures that all the trivector grade contributions in \cref{eqn:emtProblemSet1Problem5:360} are zero.

\paragraph{Vector components.}
Now that we know any trivector grades of the multivector \cref{eqn:emtProblemSet1Problem5:120} are zero, the only remaining contributions are the vector components

\begin{dmath}\label{eqn:emtProblemSet1Problem5:121}
\BM(\By)
=
%\inv{4\pi} \oint_{\partial V} dA \inv{ r^2 } \gpgradeone{ \rcap \ncap \BM }
-
\inv{4\pi} \int_V dV \frac{ s \rcap }{ r^2 } 
-
\inv{4\pi} \int_V dV \frac{ \rcap \cdot (I\BC) }{ r^2 }
=
%\inv{4\pi} \oint_{\partial V} dA \inv{ r^2 } \gpgradeone{ \rcap \ncap \BM }
%+
\inv{4\pi} \int_V dV \frac{ \rcap \cross \BC - s \rcap }{ r^2 },
\end{dmath}

where the cross product comes from the expansion

\begin{dmath}\label{eqn:emtProblemSet1Problem5:140}
\rcap \cdot (I\BC)
=
\gpgradeone{ \rcap I\BC }
=
I (\rcap \wedge \BC)
=
-\rcap \cross \BC.
\end{dmath}

%XX
%The vector grade selection can be simplified by noting that when the integral is evaluated over the sphere centered at \( \Bx \) expanded as 
%
%\begin{dmath}\label{eqn:emtProblemSet1Problem5:220}
%\gpgradeone{ \rcap \ncap \BM }
%=
%\lr{ \rcap (\ncap \cdot \BM) + \rcap \cdot (\ncap \wedge \BM) }
%=
%\rcap (\ncap \cdot \BM) + (\rcap \cdot \ncap) \BM - \ncap ( \rcap \cdot \BM ).
%\end{dmath}
%
%This looks like it is probably largest when all of \( \rcap, \ncap, \BM \) are colinear, so will likely have magnitude \( \gpgradeone{ \rcap \ncap \BM } < \Abs{\BM} \).  What we can say for sure is 
%
%\begin{dmath}\label{eqn:emtProblemSet1Problem5:240}
%\Abs{\lr{ \inv{4\pi} \oint_{\partial V} dA \inv{ r^2 } \gpgradeone{ \rcap \ncap \BM } }}
%\ge
%\inv{4\pi} \oint_{\partial V} dA \inv{ r^2 } \Abs{ \gpgradeone{ \rcap \ncap \BM } }
%\ge
%\inv{4\pi} \oint_{\partial V} dA \inv{ r^2 } 3 \Abs{ \BM },
%\end{dmath}
%
%so if \( \lim_{\By \rightarrow \infty} \ifrac{\Abs{\BM(\By)}}{\Abs{\Bx - \By}^2 } \rightarrow 0 \), this whole surface integral will be killed,
We are left with \( \BM \) uniquely determined by a weighted integral of the curl and divergence values \( \BC \) and \( s \) respectively.  With a change of variables for tidiness, that is

%\begin{boxed}\label{eqn:emtProblemSet1Problem5:200}
\boxedEquation{eqn:emtProblemSet1Problem5:200}{
\BM(\Bx)
=
\inv{4\pi} \int_V dV' \frac{ \BC(\Bx') \cross (\Bx - \Bx') + (\Bx - \Bx') s(\Bx')}{ \Abs{\Bx - \Bx'}^3 }.
}

This result can also be written in terms of potential functions

\begin{dmath}\label{eqn:emtProblemSet1Problem5:300}
\BM(\Bx)
=
-\spacegrad \int_V dV' \frac{ s(\Bx')}{ 4 \pi \Abs{\Bx - \Bx'} }
+\spacegrad \cross \int_V dV' \frac{ \BC(\Bx') }{ 4 \pi \Abs{\Bx - \Bx'} }.
\end{dmath}

This vector is unique provided the limiting constraints \cref{eqn:emtProblemSet1Problem5:340} \cref{eqn:emtProblemSet1Problem5:380} are imposed on the field specification.

%\end{boxed}
}
