%
% Copyright � 2016 Peeter Joot.  All Rights Reserved.
% Licenced as described in the file LICENSE under the root directory of this GIT repository.
%
\newcommand{\authorname}{Peeter Joot}
\newcommand{\email}{peeterjoot@protonmail.com}
\newcommand{\basename}{FIXMEbasenameUndefined}
\newcommand{\dirname}{notes/FIXMEdirnameUndefined/}

\renewcommand{\basename}{emt4}
\renewcommand{\dirname}{notes/ece1228/}
\newcommand{\keywords}{ECE1228H}
\newcommand{\authorname}{Peeter Joot}
\newcommand{\onlineurl}{http://sites.google.com/site/peeterjoot2/math2013/\basename.pdf}
\newcommand{\sourcepath}{\dirname\basename.tex}
\newcommand{\generatetitle}[1]{\chapter{#1}}

\newcommand{\vcsinfo}{%
\section*{}
\noindent{\color{DarkOliveGreen}{\rule{\linewidth}{0.1mm}}}
\paragraph{Document version}
%\paragraph{\color{Maroon}{Document version}}
{
\small
\begin{itemize}
\item Available online at:\\ 
\href{\onlineurl}{\onlineurl}
\item Git Repository: \input{./.revinfo/gitRepo.tex}
\item Source: \sourcepath
\item last commit: \input{./.revinfo/gitCommitString.tex}
\item commit date: \input{./.revinfo/gitCommitDate.tex}
\end{itemize}
}
}

%\PassOptionsToPackage{dvipsnames,svgnames}{xcolor}
\PassOptionsToPackage{square,numbers}{natbib}
\documentclass{scrreprt}

\usepackage[left=2cm,right=2cm]{geometry}
\usepackage[svgnames]{xcolor}
\usepackage{peeters_layout}

\usepackage{natbib}

\usepackage[
colorlinks=true,
bookmarks=false,
pdfauthor={\authorname, \email},
backref 
]{hyperref}

% http://tex.stackexchange.com/questions/75773/how-to-reference-problems-by-the-text-label-in-an-exercise-envioronment
\usepackage[english]{cleveref}
\crefname{Exercise}{exercise}{exercises}
\Crefname{Exercise}{Exercise}{Exercises}

\RequirePackage{titlesec}
\RequirePackage{ifthen}

% http://stackoverflow.com/questions/4932910/date-in-the-tabular-environment
\makeatletter
\let\insertdate\@date
\makeatother

\titleformat{\chapter}[display]
{\bfseries\Large}
{\color{DarkSlateGrey}\filleft \authorname
\ifthenelse{\isundefined{\studentnumber}}{}{\\ \studentnumber}
\ifthenelse{\isundefined{\email}}{}{\\ \email}
\ifthenelse{\isundefined{\dateintitle}}{}{\\ \insertdate}
%\ifthenelse{\isundefined{\coursename}}{}{\\ \coursename} % put in title instead.
}
{4ex}
{\color{DarkOliveGreen}{\titlerule}\color{Maroon}
\vspace{2ex}%
\filright}
[\vspace{2ex}%
\color{DarkOliveGreen}\titlerule
]

\newcommand{\beginArtWithToc}[0]{\begin{document}\tableofcontents}
\newcommand{\beginArtNoToc}[0]{\begin{document}}
\newcommand{\EndNoBibArticle}[0]{\end{document}}
\newcommand{\EndArticle}[0]{\bibliography{Bibliography}\bibliographystyle{plainnat}\end{document}}

% 
%\newcommand{\citep}[1]{\cite{#1}}

\colorSectionsForArticle



%\usepackage{ece1228}
\usepackage{peeters_braket}
%\usepackage{peeters_layout_exercise}
\usepackage{peeters_figures}
\usepackage{mathtools}
\usepackage{siunitx}

\beginArtNoToc
\generatetitle{ECE1228H Electromagnetic Theory.  Lecture 4: Magnetic moment, and Boundary value conditions.  Taught by Prof.\ M. Mojahedi}
%\chapter{Magnetic moment, and Boundary value conditions}
\label{chap:emt4}

%\paragraph{Disclaimer}
%
%Peeter's lecture notes from class.  These may be incoherent and rough.
%
%These are notes for the UofT course ECE1228H, Electromagnetic Theory, taught by Prof. M. Mojahedi, covering \textchapref{{1}} \citep{balanis1989advanced} content.

\paragraph{Magnetic moment.}

Using a semi-classical model of an electron, assuming that the electron circles the nuclei.  This is a completely wrong model, but useful.  In reality, electrons are random and probabilistic and do not follow defined paths.  We do however have a magnetic moment associated with the electron, and one associated with the spin of the electron, and a moment associated with the spin of the nuclei.  All of these concepts can be used to describe a more accurate model and such a model is discussed in \citep{jackson1975cew} chapters 11,12,13.

Ignoring the details of how the moments really occur physically, we can take it as a given that they exist, and model them as elemenetal magnetic dipole moments of the form

\begin{dmath}\label{eqn:emtLecture4:20}
d\Bm_i = \ncap_i I_i ds_i \qquad [\si{A m^2}].
\end{dmath}

Here the normal is defined in terms of the right hand rule with respect to the direction of the current as sketched in 

F1

Such dipole moments are actually what an MRI measures.  The noises that people describe from MRI machines are actually when the very powerful magnets are being rotated, allowing for the magnetic moments in the atoms of the body to be measured in different directions.

The magnetic polarization, or magnetization \( \BM \), in [\si{A/m}]] is given by

\begin{dmath}\label{eqn:emtLecture4:40}
\BM 
= \lim_{\Delta v \rightarrow 0} \lr{ \inv{\Delta v} \Bm_i }
= \lim_{\Delta v \rightarrow 0} \lr{ \inv{\Delta v} \sum_{i = 1}^{N \delta v} d\Bm_i }
= \lim_{\Delta v \rightarrow 0} \lr{ \inv{\Delta v} \sum_{i = 1}^{N \delta v} \ncap_i I_i ds_i } .
\end{dmath}

In materials the magnetization within the atoms are usually random, however, application of a magnetic field can force these to line up, as sketched in

F2

This is accomplished because an applied magnetic field acting on the magnetic moment introduces a torque, as also occured with dipole moments under applied electric fields

\begin{dmath}\label{eqn:emtLecture4:60}
\begin{aligned}
\Btau_B &= d\Bm \cross \BB_a \\
\Btau_E &= d\Bp \cross \BE_a.
\end{aligned}
\end{dmath}

There is an energy associated with this torque

\begin{dmath}\label{eqn:emtLecture4:80}
\begin{aligned}
\Delta U_B &= -d\Bm \cdot \BB_a \\
\Delta U_E &= -d\Bp \cdot \BE_a.
\end{aligned}
\end{dmath}

In analogy with the electric dipole moment analysis, it can be assumed that there is a linear relationship between the magnetic polarization and the applied magnetic field

\begin{dmath}\label{eqn:emtLecture4:100}
\BB = \mu_0 \BH_a + \mu_0 \BM = \mu_0\lr{ \BH_a + \BM },
\end{dmath}

where
\begin{dmath}\label{eqn:emtLecture4:120}
\BM = \chi_m \BH_a,
\end{dmath}

so
\begin{equation}\label{eqn:emtLecture4:140}
\BB 
= \mu_0\lr{ 1 + \chi_m } \BH_a
\equiv \mu \BH_a.
\end{equation}

Like electric dipoles, in a volume, we can have bound currents on the surface [\si{A/m}], as well as bound volume currents [\si{A/m^2}] as sketched in

F3

It can be shown, as with the electric dipoles related bound charge densities of \crefeqn:emtLecture3:620}, that magnetic currents can be defined

\begin{dmath}\label{eqn:emtLecture4:n}
\begin{aligned}
\BJ_{sm} &= \BM \cross \ncap \\
\BJ_{vm} &= \spacegrad \cross \BM,
\end{aligned}
\end{dmath}

(showing this may be given as a homework problem ... if not do it).

\paragraph{Conductivity}

We have two constitutive relationships so far
\begin{dmath}\label{eqn:emtLecture4:n}
\begin{aligned}
\BD &= \epsilon \BE \\
\BB &= \mu \BH
\end{aligned}
\end{dmath}

but this needs to be augmented by

\begin{dmath}\label{eqn:emtLecture4:n}
\BJ_c = \epsilon \BE.
\end{dmath}

There are a couple ways to discuss this.  One is to model \( \epsilon \) as a complex number.  Such a model is not entirely unconstrained.  Like with the Cauchy-Riemann conditions that relate derivatives of the real and imaginary parts of a complex number, there is a relationship (Kramers-Kronig \citep{wiki:kramersKronig}), an integral relationship that relates the real and imaginary parts of the permittivity \( \epsilon \).

\paragraph{Boundary conditions.}

\EndArticle
