\section{Appendix I.  Helpful identities.}

This result was used in problem 3 and 6.

\begin{dmath}\label{eqn:emtProblemSet1Appendix:460}
\spacegrad \cross \lr{ \spacegrad \cross \BA }
=
\epsilon_{a b c} \Be_a \partial_b \lr{ \epsilon_{r s t} \Be_r \partial_s A_t }_c
=
\epsilon_{a b c} \Be_a \partial_b \epsilon_{c s t} \partial_s A_t
=
\delta_{ab}^{[st]}
\Be_a \partial_b \partial_s A_t
=
\Be_a \partial_b \lr{ \partial_a A_b - \partial_b A_a }
=
\spacegrad \lr{ \spacegrad \cdot \BA } - \spacegrad^2 \BA.
\end{dmath}

\section{Appendix II.  Geometric Algebra.}

I've used Geometric Algebra in my solution to problem 5, some of which was somewhat advanced.  I've provided an overview of the algebra here for reference.  Further details can be found in references such as \citep{doran2003gap}, 
\citep{hestenes1999nfc}, \citep{dorst2007gac} and \citep{aMacdonaldVAGC}.

Geometric Algebra defines a non-commutative, associative vector product

\begin{equation}\label{eqn:emtProblemSet1Appendix:20}
\Ba \Bb \Bc
=
(\Ba \Bb) \Bc
=
\Ba (\Bb \Bc),
\end{equation}

where the square of a vector equals the squared vector magnitude

\begin{dmath}\label{eqn:emtProblemSet1Appendix:40}
\Ba^2 = \Abs{\Ba}^2,
\end{dmath}

In Euclidean spaces such a squared vector is always positive, but that is not necessarily the case in the mixed signature spaces used in special relativity.

There are a number of consequences of these two simple vector multiplication rules.

\begin{itemize}
\item Squared unit vectors have a unit magnitude (up to a sign).  In a Euclidean space such a product is always positive

\begin{dmath}\label{eqn:emtProblemSet1Appendix:60}
(\Be_1)^2 = 1.
\end{dmath}

\item Products of perpendicular vectors anticommute.

\begin{dmath}\label{eqn:emtProblemSet1Appendix:80}
2 
=
(\Be_1 + \Be_2)^2 
= (\Be_1 + \Be_2)(\Be_1 + \Be_2)
= \Be_1^2 + \Be_2 \Be_1 + \Be_1 \Be_2 + \Be_2^2
= 2 + \Be_2 \Be_1 + \Be_1 \Be_2.
\end{dmath}

A product of two perpendicular vectors is called a bivector, and can be used to represent an oriented plane.  The last line above shows an example of a scalar and bivector sum, called a multivector.  In general Geometric Algebra allows sums of scalars, vectors, bivectors, and higher degree analogues (grades) be summed.

Comparison of the RHS and LHS of \cref{eqn:emtProblemSet1Appendix:80} shows that we must have

\begin{dmath}\label{eqn:emtProblemSet1Appendix:100}
\Be_2 \Be_1 = -\Be_1 \Be_2.
\end{dmath}

It is true in general that the product of two perpendicular vectors anticommutes.  When, as above, such a product is a product of
two orthonormal vectors, it behaves like a non-commutative imaginary quantity, as it has an imaginary square in Euclidean spaces

\begin{dmath}\label{eqn:emtProblemSet1Appendix:120}
(\Be_1 \Be_2)^2
=
(\Be_1 \Be_2)
(\Be_1 \Be_2)
=
\Be_1 (\Be_2
\Be_1) \Be_2
=
-\Be_1 (\Be_1
\Be_2) \Be_2
=
-(\Be_1 \Be_1)
(\Be_2 \Be_2)
=-1.
\end{dmath}

Such ``imaginary'' (unit bivectors) have important applications describing rotations in Euclidean spaces, and boosts in Minkowski spaces.

\item
The product of three perpendicular vectors, such as

\begin{dmath}\label{eqn:emtProblemSet1Appendix:140}
I = \Be_1 \Be_2 \Be_3,
\end{dmath}

is called a trivector.  In \R{3}, the product of three orthonormal vectors is called a pseudoscalar for the space, and can represent an oriented volume element.  The quantity \( I \) above is the typical orientation picked for the \R{3} unit pseudoscalar.  This quantity also has characteristics of an imaginary number

\begin{dmath}\label{eqn:emtProblemSet1Appendix:160}
I^2 
= 
(\Be_1 \Be_2 \Be_3)
(\Be_1 \Be_2 \Be_3)
= 
\Be_1 \Be_2 (\Be_3
\Be_1) \Be_2 \Be_3
=
-\Be_1 \Be_2 \Be_1
\Be_3 \Be_2 \Be_3
=
-\Be_1 (\Be_2 \Be_1)
(\Be_3 \Be_2) \Be_3
=
-\Be_1 (\Be_1 \Be_2)
(\Be_2 \Be_3) \Be_3
=
-
\Be_1^2
\Be_2^2
\Be_3^2
=
-1.
\end{dmath}

\item The product of two vectors in \R{3} can be expressed as the sum of a symmetric scalar product and antisymmetric bivector product

\begin{dmath}\label{eqn:emtProblemSet1Appendix:480}
\Ba \Bb 
=
\sum_{i,j = 1}^n \Be_i \Be_j a_i b_j
=
\sum_{i = 1}^n \Be_i^2 a_i b_i
+
\sum_{0 < i \ne j \le n} \Be_i \Be_j a_i b_j
=
\sum_{i = 1}^n a_i b_i
+
\sum_{0 < i < j \le n} \Be_i \Be_j (a_i b_j - a_j b_i).
\end{dmath}

The first (symmetric) term is clearly the dot product.  The antisymmetric term is designated the wedge product.  In general these are written

\begin{dmath}\label{eqn:emtProblemSet1Appendix:500}
\Ba \Bb = \Ba \cdot \Bb + \Ba \wedge \Bb,
\end{dmath}

where
\begin{dmath}\label{eqn:emtProblemSet1Appendix:520}
\begin{aligned}
\Ba \cdot \Bb &\equiv \inv{2} \lr{ \Ba \Bb + \Bb \Ba } \\
\Ba \wedge \Bb &\equiv \inv{2} \lr{ \Ba \Bb - \Bb \Ba },
\end{aligned}
\end{dmath}

The coordinate expansion of both can be seen above, but in \R{3} the wedge can also be written 

\begin{equation}\label{eqn:emtProblemSet1Appendix:540}
\Ba \wedge \Bb 
=
\Be_1 \Be_2 \Be_3
(\Ba \cross \Bb)
=
I
(\Ba \cross \Bb).
\end{equation}

This allows for an handy dot plus cross product expansion of the vector product

\begin{dmath}\label{eqn:emtProblemSet1Appendix:180}
\Ba \Bb = \Ba \cdot \Bb + I (\Ba \cross \Bb).
\end{dmath}

This result should be familiar to the student of quantum spin states where one writes

\begin{dmath}\label{eqn:emtProblemSet1Appendix:200}
(\Bsigma \cdot \Ba) (\Bsigma \cdot \Bb) = (\Ba \cdot \Bb) + i (\Ba \cross \Bb) \cdot \Bsigma.
\end{dmath}

This correspondence is because the Pauli spin basis is a specific matrix representation of a Geometric Algebra, satisfying the same commutator and anticommutator relationships.  A number of other algebra structures, such as complex numbers, and quaterions can also be modelled as Geometric Algebra elements.
\item It is often useful to utilize the grade selection operator
\( \gpgrade{M}{n} \) and scalar grade selection operator \( \gpgradezero{M} = \gpgrade{M}{0} \)
to select the scalar, vector, bivector, trivector, or higher grade algebraic elements.  For example, operating on vectors \( \Ba, \Bb, \Bc \), we have

\begin{dmath}\label{eqn:emtProblemSet1Appendix:580}
\begin{aligned}
\gpgradezero{ \Ba \Bb }
&= \Ba \cdot \Bb \\
\gpgradeone{ \Ba \Bb \Bc }
&= 
\Ba (\Bb \cdot \Bc)
+ 
\Ba \cdot (\Bb \wedge \Bc) \\
&=
\Ba (\Bb \cdot \Bc)
+ 
(\Ba \cdot \Bb) \Bc
- 
(\Ba \cdot \Bc) \Bb \\
\gpgradetwo{\Ba \Bb} &=
\Ba \wedge \Bb \\
\gpgradethree{\Ba \Bb \Bc} &=
\Ba \wedge \Bb \wedge \Bc.
\end{aligned}
\end{dmath}

Note that the wedge product of any number of vectors such as \( \Ba \wedge \Bb \wedge \Bc \) is associative and can be expressed in terms of the complete antisymmetrization of the product of those vectors.  A consequence of that is the fact a wedge product that includes any colinear vectors in the product is zero.
\end{itemize}

\makeexample{Helmholz equations.}{example:emtProblemSet1Appendix:1}{

As an example of the power of \cref{eqn:emtProblemSet1Appendix:180}, consider the following alternate Alternate Helmholtz equation derivation.
Application of \cref{eqn:emtProblemSet1Appendix:180} to 
Maxwell equations in the frequency domain for source free simple media gives

\begin{subequations}
\label{eqn:emtProblemSet1Problem6:340}
\begin{dmath}\label{eqn:emtProblemSet1Problem6:360}
\spacegrad \BE = -j \omega I \BB
\end{dmath}
\begin{dmath}\label{eqn:emtProblemSet1Problem6:380}
\spacegrad I \BB = -j \omega \mu \epsilon \BE.
\end{dmath}
\end{subequations}

Operation with the gradient from the left produces the Helmholtz equation for each of the fields using nothing more than multiplication and simple substitution

\begin{subequations}
\label{eqn:emtProblemSet1Problem6:400}
\begin{dmath}\label{eqn:emtProblemSet1Problem6:420}
\spacegrad^2 \BE = - \mu \epsilon \omega^2 \BE
\end{dmath}
\begin{dmath}\label{eqn:emtProblemSet1Problem6:440}
\spacegrad^2 I \BB = - \mu \epsilon \omega^2 I \BB.
\end{dmath}
\end{subequations}

There was no reason to go through the headache of looking up or deriving the expansion of \( \spacegrad \cross (\spacegrad \cross \BA ) \) as was required in problem 6 to find the wave equation constraint on \( \beta_\txtz \).

Observe that the usual Helmholtz equation for \( \BB \) doesn't have a pseudoscalar factor.  That result can be obtained by just cancelling the factors \( I \) since the \R{3} Euclidean pseudoscalar commutes with all grades (this isn't the case in \R{2} nor in Minkowski spaces.)
} % example

\makeexample{Factoring the Laplacian.}{example:emtProblemSet1Appendix:2}{

In \cref{eqn:emtProblemSet1Appendix:460} the identity 

\begin{dmath}\label{eqn:emtProblemSet1Appendix:660}
\spacegrad \cross \lr{ \spacegrad \cross \BA } = \spacegrad \lr{ \spacegrad \cdot \BA } - \spacegrad^2 \BA.
\end{dmath}

was demonstrated using tensor contraction techniques.  We can also do this with Geometric Algebra by factoring the Laplacian action on a vector

\begin{dmath}\label{eqn:emtProblemSet1Appendix:700}
\spacegrad^2 \BA
=
\spacegrad (\spacegrad \BA)
=
\spacegrad (\spacegrad \cdot \BA + \spacegrad \wedge \BA)
=
\spacegrad (\spacegrad \cdot \BA)
+
\spacegrad \cdot (\spacegrad \wedge \BA)
+
\cancel{\spacegrad \wedge \spacegrad \wedge \BA}
=
\spacegrad (\spacegrad \cdot \BA)
+
\spacegrad \cdot (\spacegrad \wedge \BA).
\end{dmath}

Should we wish to express the last term using cross products, a grade one selection operation can be used
\begin{dmath}\label{eqn:emtProblemSet1Appendix:680}
\spacegrad \cdot (\spacegrad \wedge \BA)
=
\gpgradeone{ \spacegrad (\spacegrad \wedge \BA) }
=
\gpgradeone{ \spacegrad I (\spacegrad \cross \BA) }
=
\gpgradeone{ I \spacegrad \wedge (\spacegrad \cross \BA) }
=
\gpgradeone{ I^2 \spacegrad \cross (\spacegrad \cross \BA) }
=
-\spacegrad \cross (\spacegrad \cross \BA).
\end{dmath}

Here coordinate expansion was not required in any step.
} % example

\section{Appendix III.  Geometric Algebra solution to problem 5.}

\paragraph{An equivalent multivector problem.}

This problem screams for an attempt using Geometric Algebra techniques, since
the gradient of this vector can be written as a single even grade multivector

\begin{equation}\label{eqn:emtProblemSet1Problem5AppendixGA:60}
\spacegrad \BM
= \spacegrad \cdot \BM + I \spacegrad \cross \BM
= s + I \BC.
\end{equation}

Observe that the Laplacian of \( \BM \) is vector valued

\begin{dmath}\label{eqn:emtProblemSet1Problem5AppendixGA:400}
\spacegrad^2 \BM
= \spacegrad s + I \spacegrad \BC.
\end{dmath}

This means that \( \spacegrad \BC \) must be a bivector \( \spacegrad \BC = \spacegrad \wedge \BC \), or that \( \BC \) has zero divergence

\begin{dmath}\label{eqn:emtProblemSet1Problem5AppendixGA:420}
\spacegrad \cdot \BC = 0.
\end{dmath}

This required constraint on \( \BC \) will show up in subsequent analysis.  An equivalent problem to the one posed 
is to show that
the even grade multivector equation \( \spacegrad \BM = s + I \BC \) has an inverse given the constraint
specified by \cref{eqn:emtProblemSet1Problem5AppendixGA:420}.

\paragraph{Inverting the gradient equation.}

The Green's function for the gradient can be found in \citep{doran2003gap}, where it is used to generalize the Cauchy integral equations to higher dimensions.
%Unlike the divergence or curl, the gradient is invertible, with the \R{3} Green's function

\begin{dmath}\label{eqn:emtProblemSet1Problem5AppendixGA:80}
\begin{aligned}
G(\Bx ; \Bx') &= \inv{4 \pi} \frac{ \Bx - \Bx' }{\Abs{\Bx - \Bx'}^3} \\
\spacegrad \BG(\Bx, \Bx') &= \spacegrad \cdot \BG(\Bx, \Bx') = \delta(\Bx - \Bx') = -\spacegrad' \BG(\Bx, \Bx').
\end{aligned}
\end{dmath}

The inversion equation is an application of the Fundamental Theorem of (Geometric) Calculus, with the gradient operating bidirectionally on the Green's function and the vector function

\begin{dmath}\label{eqn:emtProblemSet1Problem5AppendixGA:100}
\oint_{\partial V} G(\Bx, \Bx') d^2 \Bx' \BM(\Bx')
=
\int_V G(\Bx, \Bx') d^3 \Bx \lrspacegrad' \BM(\Bx')
=
\int_V d^3 \Bx (G(\Bx, \Bx') \lspacegrad') \BM(\Bx')
+
\int_V d^3 \Bx G(\Bx, \Bx') (\spacegrad' \BM(\Bx'))
=
-\int_V d^3 \Bx \delta(\Bx - \By) \BM(\Bx')
+
\int_V d^3 \Bx G(\Bx, \Bx') \lr{ s(\Bx') + I \BC(\Bx') }
=
-I \BM(\Bx)
+
\inv{4 \pi} \int_V d^3 \Bx \frac{ \Bx - \Bx'}{ \Abs{\Bx - \Bx'}^3 } \lr{ s(\Bx') + I \BC(\Bx') }.
\end{dmath}

The integrals are in terms of the primed coordinates so that the end result is a function of \( \Bx \).
To rearrange for \( \BM \),
let \( d^3 \Bx' = I dV' \), and \( d^2 \Bx' \ncap(\Bx') = I dA' \), then 
%to find
% in terms of \( \BM \), 
right multiply with the pseudoscalar \( I \), noting that in \R{3} the pseudoscalar commutes with any grades

\begin{dmath}\label{eqn:emtProblemSet1Problem5AppendixGA:440}
\BM(\Bx)
=
I \oint_{\partial V} G(\Bx, \Bx') I dA' \ncap \BM(\Bx')
-
I \inv{4 \pi} \int_V I dV' \frac{ \Bx - \Bx'}{ \Abs{\Bx - \Bx'}^3 } \lr{ s(\Bx') + I \BC(\Bx') }
=
-\oint_{\partial V} dA' G(\Bx, \Bx') \ncap \BM(\Bx')
+
\inv{4 \pi} \int_V dV' \frac{ \Bx - \Bx'}{ \Abs{\Bx - \Bx'}^3 } \lr{ s(\Bx') + I \BC(\Bx') }.
\end{dmath}

This can be decomposed into a vector and a trivector equation.  Let \( \Br = \Bx - \Bx' = r \rcap \), and note that

\begin{dmath}\label{eqn:emtProblemSet1Problem5AppendixGA:500}
\gpgradeone{ \rcap I \BC }
=
\gpgradeone{ I \rcap \BC }
=
I \rcap \wedge \BC
=
-\rcap \cross \BC,
\end{dmath}

so this pair of equations can be written as

\begin{subequations}
\label{eqn:emtProblemSet1Problem5AppendixGA:460}
\begin{dmath}\label{eqn:emtProblemSet1Problem5AppendixGA:520}
\BM(\Bx)
=
-\inv{4 \pi} \oint_{\partial V} dA' \frac{\gpgradeone{ \rcap \ncap \BM(\Bx') }}{r^2}
+
\inv{4 \pi} \int_V dV' \lr{
\frac{\rcap}{r^2} s(\Bx') - 
\frac{\rcap}{r^2} \cross \BC(\Bx') }
\end{dmath}
\begin{dmath}\label{eqn:emtProblemSet1Problem5AppendixGA:480}
0
=
-\inv{4 \pi} \oint_{\partial V} dA' \frac{\rcap}{r^2} \wedge \ncap \wedge \BM(\Bx')
+
\frac{I}{4 \pi} \int_V dV' \frac{ \rcap \cdot \BC(\Bx') }{r^2}.
\end{dmath}
\end{subequations}

\paragraph{Trivector grades.}
Consider the last integral in the pseudoscalar equation above.  Since we expect no pseudoscalar components, this must be zero, or cancel perfectly.  It's not obvious that this is the case, but a transformation to a surface integral shows the constraints required for that to be the case.  To do so note

\begin{dmath}\label{eqn:emtProblemSet1Problem5AppendixGA:540}
\spacegrad \inv{\Bx - \Bx'}
= -\spacegrad' \inv{\Bx - \Bx'}
=
-\frac{\Bx - \Bx'}{\Abs{\Bx - \Bx'}^3} 
= -\frac{\rcap}{r^2}.
\end{dmath}

Using this and the chain rule we have

\begin{dmath}\label{eqn:emtProblemSet1Problem5AppendixGA:560}
\frac{I}{4 \pi} \int_V dV' \frac{ \rcap \cdot \BC(\Bx') }{r^2}
=
\frac{I}{4 \pi} \int_V dV' \lr{ \spacegrad' \inv{ r } } \cdot \BC(\Bx') 
=
\frac{I}{4 \pi} \int_V dV' \spacegrad' \cdot \frac{\BC(\Bx')}{r}
-
\frac{I}{4 \pi} \int_V dV' \frac{ \spacegrad' \cdot \BC(\Bx') }{r}
=
\frac{I}{4 \pi} \int_V dV' \spacegrad' \cdot \frac{\BC(\Bx')}{r}.
=
\frac{I}{4 \pi} \int_{\partial V} dA' \ncap(\Bx') \cdot \frac{\BC(\Bx')}{r}.
\end{dmath}

The divergence of \( \BC \) above was killed by recalling the constraint \cref{eqn:emtProblemSet1Problem5AppendixGA:420}.
This means that \cref{eqn:emtProblemSet1Problem5AppendixGA:480} can be rewritten as entirely as surface integral and eventually reduced to a single triple product

\begin{dmath}\label{eqn:emtProblemSet1Problem5AppendixGA:580}
0
=
-\frac{I}{4 \pi} \oint_{\partial V} dA' \lr{
\frac{\rcap}{r^2} \cdot (\ncap \cross \BM(\Bx'))
-\ncap \cdot \frac{\BC(\Bx')}{r}
}
=
\frac{I}{4 \pi} \oint_{\partial V} dA' \ncap \cdot \lr{
\frac{\rcap}{r^2} \cross \BM(\Bx')
+ \frac{\BC(\Bx')}{r}
}
=
\frac{I}{4 \pi} \oint_{\partial V} dA' \ncap \cdot \lr{
\lr{ \spacegrad' \inv{r}} \cross \BM(\Bx')
+ \frac{\BC(\Bx')}{r}
}
=
\frac{I}{4 \pi} \oint_{\partial V} dA' \ncap \cdot \lr{
\spacegrad' \cross \frac{\BM(\Bx')}{r}
+ \cancel{\frac{\BC(\Bx')}{r}
-\frac{\spacegrad' \cross \BM(\Bx')}{r}}
}
%=
%\frac{I}{4 \pi} \oint_{\partial V} dA' \ncap \cdot \lr{
%\spacegrad' \cross \frac{\BM(\Bx')}{r}
%}
%=
%\frac{I}{4 \pi} \oint_{\partial V} dA' 
%\ncap \cdot \lr{
%\lrspacegrad' \cross \frac{\BM(\Bx')}{r}
%}
%+
%\ncap \cdot \lr{
%\lspacegrad' \cross \frac{\BM(\Bx')}{r}
%}
=
\frac{I}{4 \pi} \oint_{\partial V} dA' 
\spacegrad' \cdot 
\frac{\BM(\Bx') \cross \ncap}{r}
-
\frac{\BM(\Bx')}{r} \cdot \lr{
\cancel{\spacegrad' \cross \ncap}
}
=
\frac{I}{4 \pi} \oint_{\partial V} dA' 
\spacegrad' \cdot 
\frac{\BM(\Bx') \cross \ncap}{r}.
\end{dmath}

\paragraph{Final results.}

Assembling things back into a single multivector equation, the complete inversion integral for \( \BM \) is

\begin{dmath}\label{eqn:emtProblemSet1Problem5AppendixGA:600}
\BM(\Bx)
=
\inv{4 \pi} \oint_{\partial V} dA' 
\lr{
\spacegrad' \wedge 
\frac{\BM(\Bx') \wedge \ncap}{r}
-\frac{\gpgradeone{ \rcap \ncap \BM(\Bx') }}{r^2}
}
+
\inv{4 \pi} \int_V dV' \lr{
\frac{\rcap}{r^2} s(\Bx') - 
\frac{\rcap}{r^2} \cross \BC(\Bx') }.
\end{dmath}

This shows that vector \( \BM \) can be recovered uniquely from \( s, \BC \) when \( \Abs{\BM}/r^2 \) vanishes on an infinite surface.  
If we restrict attention to a finite surface, we have to add to the fixed solution a specific solution that depends on the value of \( \BM \) on that surface.  The vector portion of that surface integrand contains

\begin{dmath}\label{eqn:emtProblemSet1Problem5AppendixGA:640}
\gpgradeone{ \rcap \ncap \BM }
=
\rcap (\ncap \cdot \BM )
+
\rcap \cdot (\ncap \wedge \BM )
=
\rcap (\ncap \cdot \BM )
+
(\rcap \cdot \ncap) \BM 
-
(\rcap \cdot \BM ) \ncap.
\end{dmath}

The constraints required by a zero triple product \( \spacegrad' \cdot (\BM(\Bx') \cross \ncap(\Bx')) \) are complicated on a such a general finite surface.  
Consider instead, for simplicity, the case of a spherical surface, which can be analyzed more easily.  
In that case the outward normal of the surface centred on the test charge point \( \Bx \) is \( \ncap = -\rcap \).  The pseudoscalar integrand is not generally killed unless the divergence of its tangential component on this surface is zero.  One way that this can occur is for \( \BM \cross \ncap = 0 \), so that \( -\gpgradeone{ \rcap \ncap \BM } = \BM = (\BM \cdot \ncap) \ncap = \BM_\txtn \).  

This gives

\begin{dmath}\label{eqn:emtProblemSet1Problem5AppendixGA:620}
%\boxedEquation{eqn:emtProblemSet1Problem5AppendixGA:620}{
\BM(\Bx)
=
\inv{4 \pi} \oint_{\Abs{\Bx - \Bx'} = r} dA' \frac{\BM_\txtn(\Bx')}{r^2}
+
\inv{4 \pi} \int_V dV' \lr{
\frac{\rcap}{r^2} s(\Bx') + 
\BC(\Bx') \cross \frac{\rcap}{r^2} },
%}
\end{dmath}

or, in terms of potential functions, which is arguably tidier

%\begin{dmath}\label{eqn:emtProblemSet1Problem5AppendixGA:300}
\boxedEquation{eqn:emtProblemSet1Problem5AppendixGA:300}{
\BM(\Bx)
=
\inv{4 \pi} \oint_{\Abs{\Bx - \Bx'} = r} dA' \frac{\BM_\txtn(\Bx')}{r^2}
-\spacegrad \int_V dV' \frac{ s(\Bx')}{ 4 \pi r }
+\spacegrad \cross \int_V dV' \frac{ \BC(\Bx') }{ 4 \pi r }.
}
%\end{dmath}

Such a potential form provides a hint how this problem might be attempted without the use of Geometric Algebra.

\section{Appendix IV.  2nd Geometric Algebra solution to problem 5.}

Here's a third way of deriving the Helmholtz theorem inversion relation.  This is a refinement of the traditional vector algebra solution that led to \cref{eqn:emtProblemSet1Problem5:200}, that uses a factorization of the Laplacian directly, deferring any expansion in terms of dot and cross (or wedge) products until the very end.

Starting from the first line of \cref{eqn:emtProblemSet1Problem5:160}, we have

\begin{dmath}\label{eqn:emtProblemSet1Appendix:720}
-4 \pi \BM(\Bx)
= \int_V dV' \spacegrad^2 \inv{\Abs{\Bx - \Bx'}} \BM(\Bx')
= \gpgradeone{\int_V dV' \spacegrad^2 \inv{\Abs{\Bx - \Bx'}} \BM(\Bx')}
= -\gpgradeone{\int_V dV' \spacegrad \lr{ \spacegrad' \inv{\Abs{\Bx - \Bx'}}} \BM(\Bx')}
= -\gpgradeone{\int_V dV' \spacegrad \lr{ 
\spacegrad' \frac{\BM(\Bx')}{\Abs{\Bx - \Bx'}}
-\frac{\spacegrad' \BM(\Bx')}{\Abs{\Bx - \Bx'}}
} }
= 
-\gpgradeone{\spacegrad \int_{\partial V} dA' 
\ncap \frac{\BM(\Bx')}{\Abs{\Bx - \Bx'}}
 }
+\gpgradeone{\spacegrad \int_V dV' 
\frac{s(\Bx') + I\BC(\Bx')}{\Abs{\Bx - \Bx'}}
 }
\end{dmath}

By inserting a no-op grade selection operation in the second step, the troublesome trivector term that shows up in my first
attempt is eliminated.  This leaves us with a boundary term, and term that recovers the unique solution on the infinite sphere when \( \Abs{\BM}/\Abs{\Bx - \Bx'} \rightarrow 0 \).  Noting that \( \gpgradeone{ \Ba I \Bb } = I^2 \Ba \cross \Bb = -\Ba \cross \Bb \), we recover the non-boundary integral of \cref{eqn:emtProblemSet1Problem5:200}.  The boundary term is seen to have a particularly simple form using this technique.  Note that the dot and double cross product expression obtained with the vector algebra approach can be recovered from this directly if desired using an expansion of the following form

\begin{dmath}\label{eqn:emtProblemSet1Appendix:740}
\gpgradeone{ \spacegrad \ncap \BX }
=
\gpgradeone{ \spacegrad (\ncap \cdot \BX) }
+
\gpgradeone{ \spacegrad (\ncap \wedge \BX) }
=
\spacegrad (\ncap \cdot \BX) 
+
\gpgradeone{ \spacegrad I (\ncap \cross \BX) }
=
\spacegrad (\ncap \cdot \BX) 
-
\spacegrad \cross (\ncap \cross \BX).
\end{dmath}

Using this expansion in \cref{eqn:emtProblemSet1Appendix:720} recovers \cref{eqn:emtProblemSet1Problem5:200}.  This is the most compact of all three solution attempts, but also requires full knowledge of the Geometric Algebra toolbox to understand.
