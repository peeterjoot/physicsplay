\section{Appendix}

I've used Geometric Algebra in my solution to problem 5.  It is not feasible to start from first principles in that problem, but I'll outline the basic ideas below.  More details can be found in \citep{doran2003gap}, 
\citep{hestenes1999nfc}, \citep{dorst2007gac} and \citep{aMacdonaldVAGC}.

Geometric Algebra defines a non-commutative, but associative vector product

\begin{equation}\label{eqn:emtProblemSet1Appendix:20}
\Ba \Bb \Bc
=
(\Ba \Bb) \Bc
=
\Ba (\Bb \Bc),
\end{equation}

where the square of a vector equals the squared vector magnitude

\begin{dmath}\label{eqn:emtProblemSet1Appendix:40}
\Ba^2 = \Abs{\Ba}^2,
\end{dmath}

In Euclidean spaces such a squared vector is always positive, but that is not neccessarily the case in the mixed signature spaces used in special relativity.

There are a number of consequences of these two simple vector multiplication rules.

\begin{itemize}
\item Squared unit vectors have a unit magnitude

\begin{dmath}\label{eqn:emtProblemSet1Appendix:60}
(\Be_1)^2 = 1.
\end{dmath}

\item Products of perpendicular vectors anticommute.

\begin{dmath}\label{eqn:emtProblemSet1Appendix:80}
2 
=
(\Be_1 + \Be_2)^2 
= (\Be_1 + \Be_2)(\Be_1 + \Be_2)
= \Be_1^2 + \Be_2 \Be_1 + \Be_1 \Be_2 + \Be_2^2
= 2 + \Be_2 \Be_1 + \Be_1 \Be_2.
\end{dmath}

A product of two perpendicular vectors is called a bivector, and can be used to represent an oriented plane.  The last line above shows an example of a scalar and bivector sum, called a multivector.  In general Geometric Algebra allows sums of scalars, vectors, bivectors, and higher degree analogues (grades) be summed.

Comparison of the RHS and LHS of \cref{eqn:emtProblemSet1Appendix:80} shows that we must have

\begin{dmath}\label{eqn:emtProblemSet1Appendix:100}
\Be_2 \Be_1 = -\Be_1 \Be_2.
\end{dmath}

It is true in general that the product of two perpendicular vectors anticommutes.  When, as above, such a product is a product of
two orthonormal vectors, it behaves like a non-commutative imaginary quantity, as it has an imaginary square in Euclidean spaces

\begin{dmath}\label{eqn:emtProblemSet1Appendix:120}
(\Be_1 \Be_2)^2
=
(\Be_1 \Be_2)
(\Be_1 \Be_2)
=
\Be_1 (\Be_2
\Be_1) \Be_2
=
-\Be_1 (\Be_1
\Be_2) \Be_2
=
-(\Be_1 \Be_1)
(\Be_2 \Be_2)
=-1.
\end{dmath}

Such ``imaginary'' (unit bivectors) have important applications describing rotations in Euclidean spaces, and boost in Minkowski (mixed signature) spaces.

\item
The product of three perpendicular vectors, such as

\begin{dmath}\label{eqn:emtProblemSet1Appendix:140}
I = \Be_1 \Be_2 \Be_3,
\end{dmath}

is called a trivector.  In \R{3}, the product of three orthonormal vectors is called a pseudoscalar for the space, and can represent an oriented volume element.  The quantity \( I \) above is the typical orientation picked for the \R{3} unit pseudoscalar.  This quantity also has characteristics of an imaginary number

\begin{dmath}\label{eqn:emtProblemSet1Appendix:160}
I^2 
= 
(\Be_1 \Be_2 \Be_3)
(\Be_1 \Be_2 \Be_3)
= 
\Be_1 \Be_2 (\Be_3
\Be_1) \Be_2 \Be_3
=
-\Be_1 \Be_2 \Be_1
\Be_3 \Be_2 \Be_3
=
-\Be_1 (\Be_2 \Be_1)
(\Be_3 \Be_2) \Be_3
=
-\Be_1 (\Be_1 \Be_2)
(\Be_2 \Be_3) \Be_3
=
-
\Be_1^2
\Be_2^2
\Be_3^2
=
-1.
\end{dmath}

\item The product of two vectors in \R{3} can be expressed as the sum of a dot and pseudoscalar weighted cross product.

\begin{dmath}\label{eqn:emtProblemSet1Appendix:n}
\Ba \Bb 
=
\sum_{i,j = 1}^3 \Be_i \Be_j a_i b_j
=
\sum_{i = 1}^3 \Be_i^2 a_i b_i
+
\sum_{i \ne j} \Be_i \Be_j a_i b_j
=
\Ba \cdot \Bb
+
\sum_{i < j} \Be_i \Be_j (a_i b_j - a_j b_i)
=
\Ba \cdot \Bb
+
\Be_1 \Be_2 \Be_3
(\Ba \cross \Bb),
\end{dmath}

or 

\begin{dmath}\label{eqn:emtProblemSet1Appendix:180}
\Ba \Bb = \Ba \cdot \Bb + I (\Ba \cross \Bb).
\end{dmath}

A similar result is familar to the student of quantum spin states where one writes

\begin{dmath}\label{eqn:emtProblemSet1Appendix:200}
(\Bsigma \cdot \Ba) (\Bsigma \cdot \Bb) = (\Ba \cdot \Bb) + i (\Ba \cross \Bb) \cdot \Bsigma.
\end{dmath}

This correspondence is because the Pauli spin algebra is a particular representation of a Geometric Algebra, as are complex numbers and quaterions.

As an example of the power of \cref{eqn:emtProblemSet1Appendix:180}, consider the following alternate Alternate Helmholtz equation derivation.
Application of \cref{eqn:emtProblemSet1Appendix:180} to 
Maxwell equations in source free simple media gives

\begin{subequations}
\label{eqn:emtProblemSet1Problem6:340}
\begin{dmath}\label{eqn:emtProblemSet1Problem6:360}
\spacegrad \BE = -j \omega I \BB
\end{dmath}
\begin{dmath}\label{eqn:emtProblemSet1Problem6:380}
\spacegrad I \BB = -j \omega \mu \epsilon \BE.
\end{dmath}
\end{subequations}

Operation with the gradient from the left produces the Helmholtz equation for each of the fields using nothing more than multiplication and simple substitution

\begin{subequations}
\label{eqn:emtProblemSet1Problem6:400}
\begin{dmath}\label{eqn:emtProblemSet1Problem6:420}
\spacegrad^2 \BE = - \mu \epsilon \omega^2 \BE
\end{dmath}
\begin{dmath}\label{eqn:emtProblemSet1Problem6:440}
\spacegrad^2 I \BB = - \mu \epsilon \omega^2 I \BB.
\end{dmath}
\end{subequations}

There was no reason to go through the headache of looking up or deriving the expansion of \( \spacegrad \cross (\spacegrad \cross \BA \) as was required in problem 6 to find the wave equation constraint on \( \beta_\txtz \).

Observe that the usual Helmholtz equation for \( \BB \) doesn't have a pseudoscalar factor.  That result can be obtained by just cancelling the factors \( I \) since the \R{3} Euclidean pseudoscalar commutes with all grades (this isn't the case in \R{2} nor in Minkowski spaces.)

\end{itemize}
