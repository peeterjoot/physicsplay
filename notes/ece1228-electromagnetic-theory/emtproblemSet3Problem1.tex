%
% Copyright � 2016 Peeter Joot.  All Rights Reserved.
% Licenced as described in the file LICENSE under the root directory of this GIT repository.
%
\makeproblem{Tangential magnetic field boundary conditions.}{emt:problemSet3:1}{ 

In the class note we showed that when there were no sources at the interface between two
media and neither of the two media was a perfect conductor \( \sigma_1, \sigma_2 \ne \infty \) the boundary condition
on the tangential magnetic field was given by

\begin{dmath}\label{eqn:emtProblemSet3Problem1:20}
\ncap \cross \lr{ \BH_2 - \BH_1 } = 0.
\end{dmath}

Here, show that when \( \BJ_i + \BJ_c = \BJ_{ic} \ne 0 \), the boundary condition is given by 

\begin{dmath}\label{eqn:emtProblemSet3Problem1:40}
\ncap \cross \lr{ \BH_2 - \BH_1 } = \BJ_s,
\end{dmath}

where
\begin{dmath}\label{eqn:emtProblemSet3Problem1:60}
\BJ_s = \lim_{\Delta y \rightarrow 0} \BJ_{ic} \Delta y.
\end{dmath}

Note: Use the geometry provided in 
\cref{fig:boundaryPs3:boundaryPs3Fig1}
for your proof.
\imageFigure{../../figures/ece1228-emt/boundaryPs3Fig1}{Boundary geometry.}{fig:boundaryPs3:boundaryPs3Fig1}{0.3}
} % makeproblem

\makeanswer{emt:problemSet3:1}{ 

TODO.
}
