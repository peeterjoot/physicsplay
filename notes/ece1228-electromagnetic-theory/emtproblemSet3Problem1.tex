%
% Copyright � 2016 Peeter Joot.  All Rights Reserved.
% Licenced as described in the file LICENSE under the root directory of this GIT repository.
%
\makeproblem{Tangential magnetic field boundary conditions.}{emt:problemSet3:1}{ 

In the class notes we showed that when there were no sources at the interface between two
media and neither of the two media was a perfect conductor \( \sigma_1, \sigma_2 \ne \infty \) the boundary condition
on the tangential magnetic field was given by

\begin{dmath}\label{eqn:emtProblemSet3Problem1:20}
\ncap \cross \lr{ \BH_2 - \BH_1 } = 0.
\end{dmath}

Here, show that when \( \BJ_i + \BJ_c = \BJ_{ic} \ne 0 \), the boundary condition is given by 

\begin{dmath}\label{eqn:emtProblemSet3Problem1:40}
\ncap \cross \lr{ \BH_2 - \BH_1 } = \BJ_s,
\end{dmath}

where
\begin{dmath}\label{eqn:emtProblemSet3Problem1:60}
\BJ_s = \lim_{\Delta y \rightarrow 0} \BJ_{ic} \Delta y.
\end{dmath}

Note: Use the geometry provided in 
\cref{fig:boundaryPs3:boundaryPs3Fig1}
for your proof.
\imageFigure{../../figures/ece1228-emt/boundaryPs3Fig1}{Boundary geometry.}{fig:boundaryPs3:boundaryPs3Fig1}{0.3}
} % makeproblem

\makeanswer{emt:problemSet3:1}{ 

Instead of integrating over a loop as done in class, a better way to tackle this problem is to integrate the curl over the same sort of pillbox that we use for deriving the boundary conditions from the divergence Maxwell's equations.

The form of Stokes' law that we want, following the notation of \citep{aMacdonaldVAGC}, is

\begin{dmath}\label{eqn:emtProblemSet3Problem1:80}
\int_V d^3 \Bx \cdot \lr{ \boldpartial \wedge \BA } = \oint_{\partial V} d^2 \Bx \cdot \BA.
\end{dmath}

The \R{3} translation of this relation into traditional vector algebra, after applying some duality relations, is

\begin{dmath}\label{eqn:emtProblemSet3Problem1:100}
\int_V dV \spacegrad \cross \BA = \oint_{\partial V} dA \ncap \cross \BA,
\end{dmath}

where \( \ncap \) is the outwards normal.  Proving the general multivector Stokes relationship is beyond the scope of this homework assignment, but we can validate
\cref{eqn:emtProblemSet3Problem1:100} by integrating the LHS over the infinitesimal rectangular prism sketched in

F1

\begin{dmath}\label{eqn:emtProblemSet3Problem1:120}
\begin{aligned}
\oint_{\partial V} dA \ncap \cross \BA
&=
\oint_{\partial V} dx dy \Be_3 \cross \lr{ \BA(z+) - \BA(z-) } \\
&+\oint_{\partial V} dy dz \Be_3 \cross \lr{ \BA(x+) - \BA(x-) } \\
&+\oint_{\partial V} dz dx \Be_3 \cross \lr{ \BA(Y+) - \BA(Y-) } \\
&=
\int_{V} dx dy \Be_3 \cross \lr{ dz \PD{z}{\BA} }
+\int_{V} dy dz \Be_3 \cross \lr{ dx \PD{x}{\BA} }
+\int_{V} dz dx \Be_3 \cross \lr{ dy \PD{y}{\BA} } \\
&=
\int_{V} dx dy dz \spacegrad \cross \BA.
\end{aligned}
\end{dmath}

Now, let's apply this to Ampere-Maxwell equation

\begin{dmath}\label{eqn:emtProblemSet3Problem1:140}
\spacegrad \cross \BH = \BJ_{ic} + \PD{t}{\BD},
\end{dmath}

where \( \BJ_{ic} = \BJ_s \delta(y) \).  We have

\begin{dmath}\label{eqn:emtProblemSet3Problem1:160}
\oint dA \ncap \cross \BH = \int dV \lr{ \BJ_s \delta(y) + \PD{t}{\BD} }.
\end{dmath}

Using the pillbox configuration of 

F2

and taking the pillbox volume to zero in the \( \Delta y \rightarrow 0 \) limit, the LHS integral has only contributions from the top and bottom faces of the pillbox, and the \( \BD \) term, which is assumed finite, will get killed.  That is

\begin{dmath}\label{eqn:emtProblemSet3Problem1:180}
\int dA \ncap \cross \lr{ \BH_2 - \BH_1 } = \int dA \BJ_s
\end{dmath}

Both sets of integrands can now be brought under one integral
\begin{dmath}\label{eqn:emtProblemSet3Problem1:200}
\int dA \lr{ \ncap \cross \lr{ \BH_2 - \BH_1 } - \int dA \BJ_s } = 0,
\end{dmath}

which proves the desired boundary relation

%\begin{dmath}\label{eqn:emtProblemSet3Problem1:220}
\boxedEquation{eqn:emtProblemSet3Problem1:240}{
\ncap \cross \lr{ \BH_2 - \BH_1 } - \BJ_s = 0.\qedmarker
}
%\end{dmath}

Unlike the procedure of \citep{balanis1989advanced} followed in class, this method of proof has the advantage of being coordinate free (except for having arbitrarily picked the y-axis as the normal direction).
}
