%
% Copyright � 2016 Peeter Joot.  All Rights Reserved.
% Licenced as described in the file LICENSE under the root directory of this GIT repository.
%
%{
%\newcommand{\authorname}{Peeter Joot}
\newcommand{\email}{peeterjoot@protonmail.com}
\newcommand{\basename}{FIXMEbasenameUndefined}
\newcommand{\dirname}{notes/FIXMEdirnameUndefined/}

%\renewcommand{\basename}{dipoleMoment}
%%\renewcommand{\dirname}{notes/phy1520/}
%\renewcommand{\dirname}{notes/ece1228-electromagnetic-theory/}
%%\newcommand{\dateintitle}{}
%%\newcommand{\keywords}{}
%
%\newcommand{\authorname}{Peeter Joot}
\newcommand{\onlineurl}{http://sites.google.com/site/peeterjoot2/math2013/\basename.pdf}
\newcommand{\sourcepath}{\dirname\basename.tex}
\newcommand{\generatetitle}[1]{\chapter{#1}}

\newcommand{\vcsinfo}{%
\section*{}
\noindent{\color{DarkOliveGreen}{\rule{\linewidth}{0.1mm}}}
\paragraph{Document version}
%\paragraph{\color{Maroon}{Document version}}
{
\small
\begin{itemize}
\item Available online at:\\ 
\href{\onlineurl}{\onlineurl}
\item Git Repository: \input{./.revinfo/gitRepo.tex}
\item Source: \sourcepath
\item last commit: \input{./.revinfo/gitCommitString.tex}
\item commit date: \input{./.revinfo/gitCommitDate.tex}
\end{itemize}
}
}

%\PassOptionsToPackage{dvipsnames,svgnames}{xcolor}
\PassOptionsToPackage{square,numbers}{natbib}
\documentclass{scrreprt}

\usepackage[left=2cm,right=2cm]{geometry}
\usepackage[svgnames]{xcolor}
\usepackage{peeters_layout}

\usepackage{natbib}

\usepackage[
colorlinks=true,
bookmarks=false,
pdfauthor={\authorname, \email},
backref 
]{hyperref}

% http://tex.stackexchange.com/questions/75773/how-to-reference-problems-by-the-text-label-in-an-exercise-envioronment
\usepackage[english]{cleveref}
\crefname{Exercise}{exercise}{exercises}
\Crefname{Exercise}{Exercise}{Exercises}

\RequirePackage{titlesec}
\RequirePackage{ifthen}

% http://stackoverflow.com/questions/4932910/date-in-the-tabular-environment
\makeatletter
\let\insertdate\@date
\makeatother

\titleformat{\chapter}[display]
{\bfseries\Large}
{\color{DarkSlateGrey}\filleft \authorname
\ifthenelse{\isundefined{\studentnumber}}{}{\\ \studentnumber}
\ifthenelse{\isundefined{\email}}{}{\\ \email}
\ifthenelse{\isundefined{\dateintitle}}{}{\\ \insertdate}
%\ifthenelse{\isundefined{\coursename}}{}{\\ \coursename} % put in title instead.
}
{4ex}
{\color{DarkOliveGreen}{\titlerule}\color{Maroon}
\vspace{2ex}%
\filright}
[\vspace{2ex}%
\color{DarkOliveGreen}\titlerule
]

\newcommand{\beginArtWithToc}[0]{\begin{document}\tableofcontents}
\newcommand{\beginArtNoToc}[0]{\begin{document}}
\newcommand{\EndNoBibArticle}[0]{\end{document}}
\newcommand{\EndArticle}[0]{\bibliography{Bibliography}\bibliographystyle{plainnat}\end{document}}

% 
%\newcommand{\citep}[1]{\cite{#1}}

\colorSectionsForArticle


%
%\usepackage{peeters_layout_exercise}
%\usepackage{peeters_braket}
%\usepackage{peeters_figures}
%\usepackage{siunitx}
%
%\beginArtNoToc
%
%\generatetitle{Dipole moment}
%\chapter{Dipole moment}
%\label{chap:dipoleMoment}

\makeproblem{Field for an electric dipole.}{problem:dipoleMoment:dipole}{
\index{dipole!electric}

An equal charge dipole configuration is sketched in \cref{fig:dipoleSignConventionL3:dipoleSignConventionL3Fig3}.  Compute the electric field.

% L3:
%\imageFigure{../../figures/ece1228-emt/dipoleSignConventionL3Fig3}{Dipole sign convention.}{fig:dipoleSignConventionL3:dipoleSignConventionL3Fig3}{0.3}
} % problem

\makeanswer{problem:dipoleMoment:dipole}{
The vector from the origin to the observation point is

\begin{equation}\label{eqn:dipoleMoment:20}
\Br = \BR_1 + \Bd/2
= \BR_2 - \Bd/2,
\end{equation}

or

\begin{equation}\label{eqn:dipoleMoment:40}
\begin{aligned}
\BR_1 &= \Br - \Bd/2 \equiv \BR_{+} \\
\BR_2 &= \Br + \Bd/2 \equiv \BR_{-}.
\end{aligned}
\end{equation}

The electric field for this superposition is
\begin{dmath}\label{eqn:dipoleMoment:60}
\BE
=
\inv{4 \pi \epsilon_0} \lr{
\frac{q \BR_{+}}{\Abs{\BR_{+}}^3} -
\frac{q \BR_{-}}{\Abs{\BR_{-}}^3}
}
=
\frac{q}{4 \pi \epsilon_0} \lr{
\frac{\Br - \Bd/2}{\Abs{\BR_{+}}^3} -
\frac{\Br + \Bd/2}{\Abs{\BR_{-}}^3}
}
=
\frac{q}{4 \pi \epsilon_0} \lr{
\Br \lr{
\inv{\Abs{\BR_{+}}^3}
 -
\inv{\Abs{\BR_{-}}^3}
}
-
\frac{\Bd}{2} \lr{
\inv{\Abs{\BR_{+}}^3}
+
\inv{\Abs{\BR_{-}}^3}
}
}.
\end{dmath}

The magnitudes can be expanded in Taylor series

\begin{dmath}\label{eqn:dipoleMoment:80}
\Abs{\BR_{\pm}}^{3}
=
\lr{
\lr{ \Br \mp \Bd/2 } \cdot \lr{ \Br \mp \Bd/2 }
}^{-3/2}
=
\lr{
\lr{ \Br^2 + (\Bd/2)^2 \mp 2 \Br \cdot \Bd/2 }
}^{-3/2}
=
\lr{
\lr{ \Br^2 + (\Bd/2)^2 \mp \Br \cdot \Bd }
}^{-3/2}
=
(\Br^2)^{-3/2}
\lr{
\lr{ 1 + \lr{\frac{\Bd}{2 r}}^2 \mp \rcap \cdot \frac{\Bd}{r} }
}^{-1/2}
=
r^{-3}
\lr{
1
-\frac{3}{2}
\lr{ \lr{\frac{\Bd}{2 r}}^2 \mp \rcap \cdot \frac{\Bd}{r} }
+\lr{\frac{-3}{2}}
\lr{\frac{-5}{2}} \inv{2!}
\lr{ \lr{\frac{\Bd}{2 r}}^2 \mp \rcap \cdot \frac{\Bd}{r} }^2
+ \cdots
}.
\end{dmath}

Here \( r = \Abs{\Br} \), and the Taylor series was taken in the \( \Bd/r \ll 1 \) limit.  The sums and differences of these magnitudes, are to first order

\begin{dmath}\label{eqn:dipoleMoment:100}
\inv{\Abs{\BR_{+}}^3}
-
\inv{\Abs{\BR_{-}}^3}
=
2 \frac{1}{r^3} \lr{\frac{-3}{2}} \lr{-\rcap \cdot \frac{\Bd}{r}}
\approx
\frac{3}{r^4} \rcap \cdot \Bd,
\end{dmath}

and

\begin{dmath}\label{eqn:dipoleMoment:120}
\inv{\Abs{\BR_{+}}^3}
+
\inv{\Abs{\BR_{-}}^3}
\approx
\frac{2}{r^3}.
\end{dmath}

The \( \Br \gg \Bd \) limiting expression for the electric field is

\begin{dmath}\label{eqn:dipoleMoment:140}
\BE
\approx
\frac{q}{4 \pi \epsilon_0 r^3} \lr{
3 \rcap \lr{ \rcap \cdot \Bd }
-
2 \frac{\Bd}{2}
},
\end{dmath}

or, with \( \Bp = q \Bd \)

%\begin{dmath}\label{eqn:dipoleMoment:180}
\boxedEquation{eqn:dipoleMoment:180}{
\BE =
\frac{1}{4 \pi \epsilon_0 r^3} \lr{
3 \rcap \lr{ \rcap \cdot \Bp }
-\Bp
}.
}
%\end{dmath}

} % answer

%}
%\EndNoBibArticle
