%
% Copyright � 2016 Peeter Joot.  All Rights Reserved.
% Licenced as described in the file LICENSE under the root directory of this GIT repository.
%
\makeproblem{Magnetic field for a current loop.}{emt:problemSet3:2}{ 

A loop of wire located in x-y plane carrying current \(I\) is shown in \cref{fig:currentLoopPs3:currentLoopPs3Fig2}.
The loop's radius is \(R_l\).
\imageFigure{../../figures/ece1228-emt/currentLoopPs3Fig2}{Current loop.}{fig:currentLoopPs3:currentLoopPs3Fig2}{0.3}
\makesubproblem{}{emt:problemSet3:2a}
Calculate the magnetic field flux density, \( \BB \), along the loop axis at a distance \( z \) from its center.

\makesubproblem{}{emt:problemSet3:2b}
Simplify the results in 
\partref{emt:problemSet3:2a}
for large distances along the z-axis (\( z \gg R_l \)).

\makesubproblem{}{emt:problemSet3:2c}
Express the results in 
\partref{emt:problemSet3:2b}
in terms of magnetic dipole
moment. Make sure you write the expression in vector
form.
\makesubproblem{}{emt:problemSet3:2d}
In keeping with your understanding of magnetic bar's
north and south poles, designate the north and south poles
for the current carrying loop shown in the figure. 

\paragraph{Hint:} Use Biot-Savart law which states the following: A
differential current element, \( I d\Bl' \), produces a differential
magnetic field, \( d\BB \),
at a distance \( R \) from the current
element given by

\begin{dmath}\label{eqn:emtProblemSet3Problem2:20}
d\BB = \frac{\mu_0}{4 \pi} \frac{I d\Bl' \cross \BR }{R^3},
\end{dmath}

or
\begin{dmath}\label{eqn:emtProblemSet3Problem2:40}
\BB = \frac{\mu_0}{4 \pi} \int \frac{I d\Bl' \cross \BR }{R^3},
\end{dmath}

Note that integration is carried over the source (current) and \( R \) points from the current elements
to the point of observation. 
} % makeproblem

\makeanswer{emt:problemSet3:2}{ 
\makeSubAnswer{}{emt:problemSet3:2a}

TODO.
\makeSubAnswer{}{emt:problemSet3:2b}

TODO.
\makeSubAnswer{}{emt:problemSet3:2c}

TODO.
\makeSubAnswer{}{emt:problemSet3:2d}

TODO.
}
