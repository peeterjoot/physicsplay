%
% Copyright � 2016 Peeter Joot.  All Rights Reserved.
% Licenced as described in the file LICENSE under the root directory of this GIT repository.
%
\makeproblem{Finite length photonic crystal.}{emt:problemSet9:1}{ 

Consider a truncated (finite length) one dimensional photonic crystal shown in \cref{fig:finitePhotonic:finitePhotonicFig1} below,
in which there are \( N \) dielectric slabs of index \( n_j \) 
and length \( d_j \). Find the transmission and reflection functions for this structure as a function of
\( \lambda_1 \),
\( \lambda_2 \),
\( a \),
\( b \),
\( g \)
and 
\( \beta_i \), where
\( \lambda_1 \) and
\( \lambda_2 \)
are the
eigenvalues of the one period matrix \( \BM \) given in problem 3 of last week and 
\( a \),
\( b \),
\( g \),
and
\( \beta_i \)
are also defined in the same problem.

\imageFigure{../../figures/ece1228-emt/finitePhotonicFig1}{Finite photonic crystal.}{fig:finitePhotonic:finitePhotonicFig1}{0.25}
} % makeproblem

\makeanswer{emt:problemSet9:1}{ 

In problem set 8, it was 
found that the one period fields were related by

\begin{dmath}\label{eqn:emtproblemSet9Problem1:20}
\begin{bmatrix}
E_{l,i}' \\
E_{r,i}'
\end{bmatrix}
=
\BM
\begin{bmatrix}
E_{l,i+1}' \\
E_{r,i+1}'
\end{bmatrix},
\end{dmath}

where \( \rho_{i,j} \) was the Fresnel coefficient for the \((i,j)\) interface, and

\begin{dmath}\label{eqn:emtproblemSet9Problem1:40}
\BM
=
\inv{1 - \rho_{i,j}^2}
\begin{bmatrix}
(e^{j \phi_j} - e^{-j \phi_j} \rho_{ij}^2)
e^{ j\phi_i }
& - \rho_{ij}( e^{j \phi_j} - e^{- j\phi_j} ) e^{-j\phi_i} \\
\rho_{ij}( e^{j\phi_j} - e^{- j \phi_j} ) e^{j\phi_i} &
(e^{-j \phi_j} - e^{j \phi_j} \rho_{ij}^2) e^{-j \phi_i} \\
\end{bmatrix}.
\end{dmath}

Before using this and potentially accumulating errors, let's check this against the slab reflection and transmission amplitudes calculated in class.  If the wave amplitude exiting the slab is \( E \), then \( E_{r,i+1}' = E e^{-j \beta_i d_i} = E e^{j \phi_i} \), so we have

\begin{dmath}\label{eqn:emtproblemSet9Problem1:60}
\begin{bmatrix}
E_{l,i}' \\
E_{r,i}'
\end{bmatrix}
=
\inv{1 - \rho_{i,j}^2}
\begin{bmatrix}
(e^{j \phi_j} - e^{-j \phi_j} \rho_{ij}^2)
e^{ j\phi_i }
& - \rho_{ij}( e^{j \phi_j} - e^{- j\phi_j} ) e^{-j\phi_i} \\
\rho_{ij}( e^{j\phi_j} - e^{- j \phi_j} ) e^{j\phi_i} &
(e^{-j \phi_j} - e^{j \phi_j} \rho_{ij}^2) e^{-j \phi_i} \\
\end{bmatrix}
\begin{bmatrix}
0 \\
E e^{j \phi_i} 
\end{bmatrix}
=
\frac{E}{1 - \rho_{i,j}^2}
\begin{bmatrix}
- \rho_{ij}( e^{j \phi_j} - e^{- j\phi_j} ) \\
e^{-j \phi_j} - e^{j \phi_j} \rho_{ij}^2 \\
\end{bmatrix}
\end{dmath}

The transmission coefficient is

\begin{dmath}\label{eqn:emtproblemSet9Problem1:80}
t 
= \frac{E}{E_{r,i}'}
= \frac{ 1 - \rho_{i,j}^2 }{ e^{-j \phi_j} - e^{j \phi_j} \rho_{ij}^2 }
= \frac{ (1 - \rho_{i,j}^2) e^{j \phi_j} }{ 1 - e^{2 j \phi_j} \rho_{ij}^2 }
= \frac{ \tau_{ij} \tau_{ji} e^{j \phi_j} }{ 1 - e^{2 j \phi_j} \rho_{ij}^2 }.
\end{dmath}

This matches what we found in class.  For the reflection amplitude we have

\begin{dmath}\label{eqn:emtproblemSet9Problem1:100}
r
= \frac{E_{l,i}'}{E_{r,i}'}
= 
\frac{
- \rho_{ij}( e^{j \phi_j} - e^{- j\phi_j} )
}
{
e^{-j \phi_j} - e^{j \phi_j} \rho_{ij}^2
}
= 
\frac{
\rho_{ij}( 1 - e^{2 j \phi_j}  )
}
{
1 - e^{2 j \phi_j} \rho_{ij}^2
}.
\end{dmath}

Compare this to the value found in class
\begin{dmath}\label{eqn:emtproblemSet9Problem1:120}
r 
=
\rho_{ij} + \frac{\tau_{ij} \tau_{ji} \rho_{ji} e^{ 2 j \phi_j} }{1 - \rho_{ij}^2 e^{2 j \phi_j}}
=
\rho_{ij} - \frac{ (1 - \rho_{ij}^2) \rho_{ij} e^{ 2 j \phi_j} }{1 - \rho_{ij}^2 e^{2 j \phi_j}}
=
\rho_{ij} 
\lr{ 
   1 - 
      \frac{ (1 - \rho_{ij}^2) e^{ 2 j \phi_j} }{1 - \rho_{ij}^2 e^{2 j \phi_j} } 
}
=
\rho_{ij} 
   \frac{ 1 - \rho_{ij}^2 e^{2 j \phi_j} - (1 - \rho_{ij}^2) e^{ 2 j \phi_j} }{1 - \rho_{ij}^2 e^{2 j \phi_j}} 
=
\rho_{ij} 
   \frac{ 1 - e^{ 2 j \phi_j} }{1 - \rho_{ij}^2 e^{2 j \phi_j}},
\end{dmath}

which matches \cref{eqn:emtproblemSet9Problem1:100} as desired, and provides at least partial confidence before continuing.  We can now state the relation between the inputs and outputs for the entire system

\begin{dmath}\label{eqn:emtproblemSet9Problem1:140}
\begin{bmatrix}
E_{l,i}' \\
E_{r,i}'
\end{bmatrix}
= 
\BM^N
\begin{bmatrix}
0 \\
E e^{j \phi_i} 
\end{bmatrix}.
\end{dmath}

If \( \BM \) satisfies the eigenvalue relations

\begin{dmath}\label{eqn:emtproblemSet9Problem1:160}
\begin{aligned}
\BM \Be_1 &= \lambda_1 \Be_1 \\
\BM \Be_2 &= \lambda_2 \Be_2 \\
\end{aligned},
\end{dmath}

then we can form
\begin{dmath}\label{eqn:emtproblemSet9Problem1:180}
\BM 
\begin{bmatrix}
\Be_1 & \Be_2 
\end{bmatrix}
=
\begin{bmatrix}
\Be_1 \lambda_1 & \Be_2 \lambda_2
\end{bmatrix}
=
\begin{bmatrix}
\Be_1 & \Be_2 
\end{bmatrix}
\begin{bmatrix}
\lambda_1 & 0 \\
0 & \lambda_2
\end{bmatrix}.
\end{dmath}

Let
\begin{dmath}\label{eqn:emtproblemSet9Problem1:200}
\begin{aligned}
\BE &=
\begin{bmatrix}
\Be_1 & \Be_2
\end{bmatrix} \\
\BD 
&= 
\begin{bmatrix}
\lambda_1 & 0 \\
0 & \lambda_2
\end{bmatrix}
\end{aligned},
\end{dmath}

so that

\begin{dmath}\label{eqn:emtproblemSet9Problem1:220}
\BM = \BE \BD \BE^{-1}.
\end{dmath}

Integer powers of \( \BM \) can now be computed easily

\begin{dmath}\label{eqn:emtproblemSet9Problem1:240}
\BM^N 
= 
(\BE \BD \BE^{-1}) (\BE \BD \BE^{-1}) \cdots (\BE \BD \BE^{-1}) 
= 
\BE \BD^N \BE^{-1}
= 
\BE 
\begin{bmatrix}
\lambda_1^N & 0 \\
0 & \lambda_2^N
\end{bmatrix}
\BE^{-1}
.
\end{dmath}

This gives

\begin{dmath}\label{eqn:emtproblemSet9Problem1:260}
\begin{bmatrix}
E_{l,i}' \\
E_{r,i}'
\end{bmatrix}
= 
E 
\BE \BD^N \BE^{-1}
\begin{bmatrix}
0 \\
e^{j \phi_i} 
\end{bmatrix}
.
\end{dmath}

The symbolic expansion of the eigenvalues and eigenvectors for this problem are not particularly illuminating

\begin{dmath}\label{eqn:emtproblemSet9Problem1:n}
\begin{aligned}
\lambda_1 =
 \frac{
e^{-j \left(\phi_i+\phi_j\right)}
}{2 \left(\rho_{ij}^2-1\right)} 
\biglr{
&
   e^{2 j \phi_i} \rho_{ij}^2
   +
   e^{2 j \phi_j} \left(\rho_{ij}^2
   -
   e^{2 j \phi_i}\right) \\
&-2 
\sqrt{
   e^{2 j \left(\phi_i+\phi_j\right)} 
   \left(\left(\cos  \left(\phi_i+\phi_j\right)-\left(\cos  \left(\phi_i-\phi_j\right)\right) \rho_{ij}^2\right){}^2-\left(\rho_{ij}^2-1\right){}^2\right)}-1
}
\\
\lambda_2 = \frac{
e^{-j \left(\phi_i+\phi_j\right)}
}{2 \left(\rho_{ij}^2-1\right)} 
\biglr{
&
   e^{2 j \phi_i} \rho_{ij}^2
   +
   e^{2 j \phi_j} \left(\rho_{ij}^2
   -
   e^{2 j \phi_i}
   \right) \\
&
+2 \sqrt{
   e^{2 j \left(\phi_i+\phi_j\right)}
    \left(\left(\cos  \left(\phi_i+\phi_j\right)-\left(\cos  \left(\phi_i-\phi_j\right)\right) \rho_{ij}^2\right){}^2-\left(\rho_{ij}^2-1\right){}^2\right)}-1
}
\\
\end{aligned}
\end{dmath}

The corresponding eigenvectors are even messier.  They can be found in 
\nbcite{ps9:ps9p1Eigenvalues.nb}
{ps9/ps9p1Eigenvalues.nb} if interested.

The transmission 
and reflection
amplitude coefficients are

\boxedEquation{eqn:emtproblemSet9Problem1:280}{
\begin{aligned}
t &= \frac{E} { E_{r,i}' } =
   \inv{ 
   \begin{bmatrix}
   0 & 1
   \end{bmatrix}
   \BE \BD^N \BE^{-1}
   \begin{bmatrix}
   0 \\
   e^{j \phi_i} 
   \end{bmatrix}
   } \\
r &= \frac{ E_{l,i}' } { E_{r,i}' } =
   \frac
   {
   \begin{bmatrix}
   1 & 0
   \end{bmatrix}
      \BE \BD^N \BE^{-1}
      \begin{bmatrix}
      0 \\
      e^{j \phi_i} 
      \end{bmatrix}
   }
   {
   \begin{bmatrix}
   0 & 1
   \end{bmatrix}
      \BE \BD^N \BE^{-1}
      \begin{bmatrix}
      0 \\
      e^{j \phi_i} 
      \end{bmatrix}
   }
\end{aligned}
}


}
