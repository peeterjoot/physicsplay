%
% Copyright � 2016 Peeter Joot.  All Rights Reserved.
% Licenced as described in the file LICENSE under the root directory of this GIT repository.
%
\newcommand{\authorname}{Peeter Joot}
\newcommand{\email}{peeterjoot@protonmail.com}
\newcommand{\basename}{FIXMEbasenameUndefined}
\newcommand{\dirname}{notes/FIXMEdirnameUndefined/}

\renewcommand{\basename}{emt2}
\renewcommand{\dirname}{notes/ece1228/}
\newcommand{\keywords}{ECE1228H}
\newcommand{\authorname}{Peeter Joot}
\newcommand{\onlineurl}{http://sites.google.com/site/peeterjoot2/math2013/\basename.pdf}
\newcommand{\sourcepath}{\dirname\basename.tex}
\newcommand{\generatetitle}[1]{\chapter{#1}}

\newcommand{\vcsinfo}{%
\section*{}
\noindent{\color{DarkOliveGreen}{\rule{\linewidth}{0.1mm}}}
\paragraph{Document version}
%\paragraph{\color{Maroon}{Document version}}
{
\small
\begin{itemize}
\item Available online at:\\ 
\href{\onlineurl}{\onlineurl}
\item Git Repository: \input{./.revinfo/gitRepo.tex}
\item Source: \sourcepath
\item last commit: \input{./.revinfo/gitCommitString.tex}
\item commit date: \input{./.revinfo/gitCommitDate.tex}
\end{itemize}
}
}

%\PassOptionsToPackage{dvipsnames,svgnames}{xcolor}
\PassOptionsToPackage{square,numbers}{natbib}
\documentclass{scrreprt}

\usepackage[left=2cm,right=2cm]{geometry}
\usepackage[svgnames]{xcolor}
\usepackage{peeters_layout}

\usepackage{natbib}

\usepackage[
colorlinks=true,
bookmarks=false,
pdfauthor={\authorname, \email},
backref 
]{hyperref}

% http://tex.stackexchange.com/questions/75773/how-to-reference-problems-by-the-text-label-in-an-exercise-envioronment
\usepackage[english]{cleveref}
\crefname{Exercise}{exercise}{exercises}
\Crefname{Exercise}{Exercise}{Exercises}

\RequirePackage{titlesec}
\RequirePackage{ifthen}

% http://stackoverflow.com/questions/4932910/date-in-the-tabular-environment
\makeatletter
\let\insertdate\@date
\makeatother

\titleformat{\chapter}[display]
{\bfseries\Large}
{\color{DarkSlateGrey}\filleft \authorname
\ifthenelse{\isundefined{\studentnumber}}{}{\\ \studentnumber}
\ifthenelse{\isundefined{\email}}{}{\\ \email}
\ifthenelse{\isundefined{\dateintitle}}{}{\\ \insertdate}
%\ifthenelse{\isundefined{\coursename}}{}{\\ \coursename} % put in title instead.
}
{4ex}
{\color{DarkOliveGreen}{\titlerule}\color{Maroon}
\vspace{2ex}%
\filright}
[\vspace{2ex}%
\color{DarkOliveGreen}\titlerule
]

\newcommand{\beginArtWithToc}[0]{\begin{document}\tableofcontents}
\newcommand{\beginArtNoToc}[0]{\begin{document}}
\newcommand{\EndNoBibArticle}[0]{\end{document}}
\newcommand{\EndArticle}[0]{\bibliography{Bibliography}\bibliographystyle{plainnat}\end{document}}

% 
%\newcommand{\citep}[1]{\cite{#1}}

\colorSectionsForArticle



%\usepackage{ece1228}
\usepackage{peeters_braket}
%\usepackage{peeters_layout_exercise}
\usepackage{peeters_figures}
\usepackage{mathtools}
\usepackage{siunitx}

\beginArtNoToc
\generatetitle{ECE1228H Electromagnetic Theory.  Lecture 2: Boundaries.  Taught by Prof.\ M. Mojahedi}
%\chapter{Boundaries}
\label{chap:emt2}

\paragraph{Disclaimer}

Peeter's lecture notes from class.  These may be incoherent and rough.

These are notes for the UofT course ECE1228H, Electromagnetic Theory, taught by Prof. M. Mojahedi, covering \textchapref{{1}} \citep{balanis1989advanced} content.

\paragraph{YYY}

Given Maxwell's equations at a point

\begin{dmath}\label{eqn:emtLecture2:20}
\begin{aligned}
\spacegrad \cross \BE &= -\PD{t}{\BB} \\
\spacegrad \cross \BH &= \BJ + \PD{t}{\BD} \\
\spacegrad \cdot \BD &= \rho_\txtv \\
\spacegrad \cdot \BB &= 0
\end{aligned}
\end{dmath}

what happens when we have different fields and currents on two sides of a boundary?  To answer these questions, we want to use the integral forms of Maxwell's equations.

F1: curved surface with loop and pillboxes.

To do so, we use Stokes' and the divergence theorems relating the area and volume integrals to the surfaces of those geometries.

These are 

\begin{dmath}\label{eqn:emtLecture2:40}
\begin{aligned}
\iint_S \lr{ \spacegrad \cross \BA } \cdot d\Bs &= \oint_C \BA \cdot d\Bl \\
\iint_V \lr{ \spacegrad \cdot \BA } d\Bs &= \oint_A \BA \cdot d\Bs \\
\end{aligned}
\end{dmath}

Application of the Stokes' to Faraday's law we get

\begin{dmath}\label{eqn:emtLecture2:60}
\oint_C \BE \cdot d\Bl = -\PD{t}{} \iint \BB \cdot d\Bs
\end{dmath}

UNITS: \( V/m \times m \)

The quantity 
\begin{dmath}\label{eqn:emtLecture2:80}
\iint \BB \cdot d\Bs,
\end{dmath}

is called the magnetic flux of \( \BB \), and changing of this flux is responsible for the generation of electromotive force.

F2: 

Similarly

\begin{dmath}\label{eqn:emtLecture2:100}
\begin{aligned}
\oint \BH \cdot d\Bl &= \iint \BJ \cdot d\Bs + \PD{t}{} \iint \BD \cdot d\Bs \\
\oint \BD \cdot d\Bs &= \iiint \rho_\txtv dV = Q_\txte \\
\oint \BB \cdot d\Bs &= 0.
\end{aligned}
\end{dmath}

\paragraph{Constitutive relations}

With 12 unknowns in \( \BE, \BB, \BD, \BH \) and 8 equations in Maxwell's equations (or 6 if the divergence equations are considered redundant), things don't look too good for solutions.  In simple media, in the frequency domain, relations of the form

\begin{dmath}\label{eqn:emtLecture2:n}
\begin{aligned}
\BD( \Br, \omega ) &= \epsilon \BE( \Br, \omega ) \\
\BB( \Br, \omega ) &= \mu \BH( \Br, \omega ).
\end{aligned}
\end{dmath}

The permeabilities \( \epsilon \) and \( \mu \) are macroscopic beasts, determined either experimentally, or theoretically using an averaging process involving many (millions, or billions, or more) particles.  These can be position dependent, as in a grating

F3: 

The permeabilities can also depend on the strength of the fields.  An example, application of an electric field to gallium arsinide or glass can change the behaviour in the material.  We can also have non-linear effects, such as the effect on a capacitor when the voltage is increased.  The response near the breakdown point where the capacitor blows up demonstrates this spectacularly.  We can also have materials for which the permeabilities depend on the direction of the field, or the temperature, or the pressure in the environment, the tensile or compression forces on the material, or many other factors.  There are many other possible complicating factors, for example, the electric response \( \epsilon \) can depend on the magnetic field strength \( \Abs{\BB} \).  We could then write

\begin{dmath}\label{eqn:emtLecture2:n}
\epsilon = \epsilon( \Br, \Abs{\BE}, \BE/\Abs{\BE}, T, P, \Abs{\Beta}, \omega, k ).
\end{dmath}

Further complicating things is that \( \epsilon \) is a complex number (for fields specified in the frequency domain).

We can also have anisotropic situations where the electric and displacement fields are no longer colinear

F4:

which indicates that \( \epsilon \) in

\begin{dmath}\label{eqn:emtLecture2:n}
\BD = \epsilon \BE,
\end{dmath}

can be modelled as a matrix or as a second rank tensor.  When the off diagonal entries are zero, and the diagonal values are all equal, we have the special case where \( \epsilon \) is reduced to a function.  That function may still be complex-valued, and dependent on many factors, but it least it is scalar valued in this situation.



\EndArticle
%\EndNoBibArticle
