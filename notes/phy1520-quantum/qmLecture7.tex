%
% Copyright � 2015 Peeter Joot.  All Rights Reserved.
% Licenced as described in the file LICENSE under the root directory of this GIT repository.
%
%\newcommand{\authorname}{Peeter Joot}
\newcommand{\email}{peeterjoot@protonmail.com}
\newcommand{\basename}{FIXMEbasenameUndefined}
\newcommand{\dirname}{notes/FIXMEdirnameUndefined/}

%\renewcommand{\basename}{qmLecture7}
%\renewcommand{\dirname}{notes/phy1520/}
%\newcommand{\keywords}{PHY1520H}
%\newcommand{\authorname}{Peeter Joot}
\newcommand{\onlineurl}{http://sites.google.com/site/peeterjoot2/math2013/\basename.pdf}
\newcommand{\sourcepath}{\dirname\basename.tex}
\newcommand{\generatetitle}[1]{\chapter{#1}}

\newcommand{\vcsinfo}{%
\section*{}
\noindent{\color{DarkOliveGreen}{\rule{\linewidth}{0.1mm}}}
\paragraph{Document version}
%\paragraph{\color{Maroon}{Document version}}
{
\small
\begin{itemize}
\item Available online at:\\ 
\href{\onlineurl}{\onlineurl}
\item Git Repository: \input{./.revinfo/gitRepo.tex}
\item Source: \sourcepath
\item last commit: \input{./.revinfo/gitCommitString.tex}
\item commit date: \input{./.revinfo/gitCommitDate.tex}
\end{itemize}
}
}

%\PassOptionsToPackage{dvipsnames,svgnames}{xcolor}
\PassOptionsToPackage{square,numbers}{natbib}
\documentclass{scrreprt}

\usepackage[left=2cm,right=2cm]{geometry}
\usepackage[svgnames]{xcolor}
\usepackage{peeters_layout}

\usepackage{natbib}

\usepackage[
colorlinks=true,
bookmarks=false,
pdfauthor={\authorname, \email},
backref 
]{hyperref}

% http://tex.stackexchange.com/questions/75773/how-to-reference-problems-by-the-text-label-in-an-exercise-envioronment
\usepackage[english]{cleveref}
\crefname{Exercise}{exercise}{exercises}
\Crefname{Exercise}{Exercise}{Exercises}

\RequirePackage{titlesec}
\RequirePackage{ifthen}

% http://stackoverflow.com/questions/4932910/date-in-the-tabular-environment
\makeatletter
\let\insertdate\@date
\makeatother

\titleformat{\chapter}[display]
{\bfseries\Large}
{\color{DarkSlateGrey}\filleft \authorname
\ifthenelse{\isundefined{\studentnumber}}{}{\\ \studentnumber}
\ifthenelse{\isundefined{\email}}{}{\\ \email}
\ifthenelse{\isundefined{\dateintitle}}{}{\\ \insertdate}
%\ifthenelse{\isundefined{\coursename}}{}{\\ \coursename} % put in title instead.
}
{4ex}
{\color{DarkOliveGreen}{\titlerule}\color{Maroon}
\vspace{2ex}%
\filright}
[\vspace{2ex}%
\color{DarkOliveGreen}\titlerule
]

\newcommand{\beginArtWithToc}[0]{\begin{document}\tableofcontents}
\newcommand{\beginArtNoToc}[0]{\begin{document}}
\newcommand{\EndNoBibArticle}[0]{\end{document}}
\newcommand{\EndArticle}[0]{\bibliography{Bibliography}\bibliographystyle{plainnat}\end{document}}

% 
%\newcommand{\citep}[1]{\cite{#1}}

\colorSectionsForArticle


%
%%\usepackage{phy1520}
%\usepackage{peeters_braket}
%%\usepackage{peeters_layout_exercise}
%\usepackage{peeters_figures}
%\usepackage{mathtools}
%\usepackage{mhchem}
%
%
%\beginArtNoToc
%\generatetitle{PHY1520H Graduate Quantum Mechanics.  Lecture 7: Aharonov-Bohm effect and Landau levels.  Taught by Prof.\ Arun Paramekanti}
%%\chapter{Aharonov-Bohm effect and Landau levels}
%\label{chap:qmLecture7}
%
%\paragraph{Disclaimer}
%
%Peeter's lecture notes from class.  These may be incoherent and rough.
%
%These are notes for the UofT course PHY1520, Graduate Quantum Mechanics, taught by Prof. Paramekanti, covering \textchapref{{1}} \citep{sakurai2014modern} content.
%
\paragraph{problem set note.}

In the problem set we'll look at interference patterns for two slit electron interference like that of \cref{fig:lecture7:lecture7Fig1}, where a magnetic whisker that introduces flux is added to the configuration.

\imageFigure{../../figures/phy1520/lecture7Fig1}{Two slit interference with magnetic whisker.}{fig:lecture7:lecture7Fig1}{0.2}

\paragraph{Aharonov-Bohm effect (cont.)}

Why do we have the zeros at integral multiples of \( h/q \)?  Consider a particle in a circular trajectory as sketched in \cref{fig:lecture7:lecture7Fig3}

\imageFigure{../../figures/phy1520/lecture7Fig3}{Circular trajectory.}{fig:lecture7:lecture7Fig3}{0.1}

FIXME: Prof mentioned:

\begin{dmath}\label{eqn:qmLecture7:20}
\phi_{\textrm{loop}} = q \frac{ h p/ q }{\Hbar} = 2 \pi p
\end{dmath}

... I'm not sure what that was about now.

In classical mechanics we have

\begin{dmath}\label{eqn:qmLecture7:40}
\oint p dq
\end{dmath}

The integral zero points are related to such a loop, but the \( q \BA \) portion of the momentum \( \Bp - q \BA \) needs to be considered.

\paragraph{Superconductors}
\index{superconductor}

After cooling some materials sufficiently, superconductivity, a complete lack of resistance to electrical flow can be observed.  A resistivity vs temperature plot of such a material is sketched in \cref{fig:lecture7:lecture7Fig4}.

\imageFigure{../../figures/phy1520/lecture7Fig4}{Superconductivity with comparison to superfluidity.}{fig:lecture7:lecture7Fig4}{0.2}

Just like \ce{He^4} can undergo Bose condensation, superconductivity can be explained by a hybrid Bosonic state where electrons are paired into one state containing integral spin.

The Little-Parks experiment puts a superconducting ring around a magnetic whisker as sketched in \cref{fig:lecture7:lecture7Fig6}.

\imageFigure{../../figures/phy1520/lecture7Fig6}{Little-Parks superconducting ring.}{fig:lecture7:lecture7Fig6}{0.1}

This experiment shows that the effective charge of the circulating charge was \( 2 e \), validating the concept of Cooper-pairing, the Bosonic combination (integral spin) of electrons in superconduction.

\paragraph{Motion around magnetic field}
\index{magnetic field}
\index{Little-Parks superconductor}

%F7
%\cref{fig:lecture7:lecture7Fig7}.
%\imageFigure{../../figures/phy1520/lecture7Fig7}{CAPTION: lecture7Fig7}{fig:lecture7:lecture7Fig7}{0.2}

\begin{dmath}\label{eqn:qmLecture7:140}
\omega_{\textrm{c}} = \frac{e B}{m}
\end{dmath}

We work with what is now called the Landau gauge

\begin{dmath}\label{eqn:qmLecture7:60}
\BA = \lr{ 0, B x, 0 }
\end{dmath}

This gives

\begin{dmath}\label{eqn:qmLecture7:80}
\BB
= \spacegrad \cross \BA
= \lr{ \partial_x A_y - \partial_y A_x } \zcap
= B \zcap.
\end{dmath}

An alternate gauge choice, the symmetric gauge, is

\begin{dmath}\label{eqn:qmLecture7:100}
\BA = \lr{ -\frac{B y}{2}, \frac{B x}{2}, 0 },
\end{dmath}

that also has the same magnetic field

\begin{dmath}\label{eqn:qmLecture7:120}
\BB
= \spacegrad \BA
= \lr{ \partial_x A_y - \partial_y A_x } \zcap
= \lr{ \frac{B}{2} - \lr{ - \frac{B}{2} } } \zcap
= B \zcap.
\end{dmath}

We expect the physics for each to have the same results, although the wave functions in one gauge may be more complicated than in the other.

Our Hamiltonian is

\begin{dmath}\label{eqn:qmLecture7:160}
H
= \inv{2 m} \lr{ \Bp - e \BA }^2
= \inv{2 m} \hatp_x^2 + \inv{2 m} \lr{ \hatp_y - e B \hatx }^2
\end{dmath}

We can solve after noting that

\begin{dmath}\label{eqn:qmLecture7:180}
\antisymmetric{\hatp_y}{H} = 0
\end{dmath}

means that

\begin{dmath}\label{eqn:qmLecture7:200}
\Psi(x,y) = e^{i k_y y} \phi(x)
\end{dmath}

The eigensystem

\begin{dmath}\label{eqn:qmLecture7:220}
H \psi(x, y) = E \phi(x, y) ,
\end{dmath}

becomes

\begin{dmath}\label{eqn:qmLecture7:240}
\lr{ \inv{2 m} \hatp_x^2 + \inv{2 m} \lr{ \Hbar k_y - e B \hatx}^2 } \phi(x)
= E \phi(x).
\end{dmath}

This reduced Hamiltonian can be rewritten as

\begin{dmath}\label{eqn:qmLecture7:320}
H_x
= \inv{2 m} p_x^2 + \inv{2 m} e^2 B^2 \lr{ \hatx - \frac{\Hbar k_y}{e B} }^2
\equiv \inv{2 m} p_x^2 + \inv{2} m \omega^2 \lr{ \hatx - x_0 }^2
\end{dmath}

where

\begin{dmath}\label{eqn:qmLecture7:260}
\inv{2 m} e^2 B^2 = \inv{2} m \omega^2,
\end{dmath}

or
\begin{dmath}\label{eqn:qmLecture7:280}
\omega = \frac{ e B}{m} \equiv \omega_\txtc.
\end{dmath}

and

\begin{dmath}\label{eqn:qmLecture7:300}
x_0 = \frac{\Hbar}{k_y}{e B}.
\end{dmath}

But what is this \( x_0 \)?  Because \( k_y \) is not really specified in this problem, we can consider that we have a zero point energy for every \( k_y \), but the oscillator position is shifted for every such value of \( k_y \).  For each set of energy levels \cref{fig:lecture7:lecture7Fig8} we can consider that there is a different zero point energy for each possible \( k_y \).

\imageFigure{../../figures/phy1520/lecture7Fig8}{Energy levels, and Energy vs flux.}{fig:lecture7:lecture7Fig8}{0.1}

\index{degeneracy}
This is an infinitely degenerate system with an infinite number of states for any given energy level.

This tells us that there is a problem, and have to reconsider the assumption that any \( k_y \) is acceptable.

To resolve this we can introduce periodic boundary conditions, imagining that a square is rotated in space forming a cylinder as sketched in \cref{fig:lecture7:lecture7Fig9}.

\imageFigure{../../figures/phy1520/lecture7Fig9}{Landau degeneracy region.}{fig:lecture7:lecture7Fig9}{0.1}

Requiring quantized momentum

\begin{dmath}\label{eqn:qmLecture7:340}
k_y L_y = 2 \pi n,
\end{dmath}

or

\begin{equation}\label{eqn:qmLecture7:360}
k_y = \frac{2 \pi n}{L_y}, \qquad n \in \bbZ,
\end{equation}

gives

\begin{dmath}\label{eqn:qmLecture7:380}
x_0(n) = \frac{\Hbar}{e B} \frac{ 2 \pi n}{L_y},
\end{dmath}

with \( x_0 \le L_x \).  The range is thus restricted to

\begin{dmath}\label{eqn:qmLecture7:400}
\frac{\Hbar}{e B} \frac{ 2 \pi n_{\textrm{max}}}{L_y} = L_x,
\end{dmath}

or

\begin{dmath}\label{eqn:qmLecture7:420}
n_{\textrm{max}} =
\mathLabelBox
[ labelstyle={below of=m\themathLableNode, below of=m\themathLableNode} ]
{L_x L_y}{area} \frac{ e B }{2 \pi \Hbar }
\end{dmath}

That is

\begin{dmath}\label{eqn:qmLecture7:440}
n_{\textrm{max}}
= \frac{\Phi_{\textrm{total}}}{h/e}
= \frac{\Phi_{\textrm{total}}}{\Phi_0}.
\end{dmath}

%F10
%\cref{fig:lecture7:lecture7Fig10}.
%\imageFigure{../../figures/phy1520/lecture7Fig10}{CAPTION: lecture7Fig10}{fig:lecture7:lecture7Fig10}{0.2}

Attempting to measure Hall-effect systems, it was found that the Hall conductivity was quantized like

\begin{dmath}\label{eqn:qmLecture7:460}
\sigma_{x y} = p \frac{e^2}{h}.
\end{dmath}

\index{Landau levels}
This quantization is explained by these Landau levels, and this experimental apparatus provides one of the more accurate ways to measure the fine structure constant.

%\cref{fig:lecture7:lecture7Fig11}.
%\imageFigure{../../figures/phy1520/lecture7Fig11}{CAPTION: lecture7Fig11}{fig:lecture7:lecture7Fig11}{0.2}

%\EndArticle
