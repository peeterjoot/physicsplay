%
% Copyright � 2015 Peeter Joot.  All Rights Reserved.
% Licenced as described in the file LICENSE under the root directory of this GIT repository.
%
%\newcommand{\authorname}{Peeter Joot}
\newcommand{\email}{peeterjoot@protonmail.com}
\newcommand{\basename}{FIXMEbasenameUndefined}
\newcommand{\dirname}{notes/FIXMEdirnameUndefined/}

%\renewcommand{\basename}{bakercambell}
%\renewcommand{\dirname}{notes/phy1520/}
%%\newcommand{\dateintitle}{}
%%\newcommand{\keywords}{}
%
%\newcommand{\authorname}{Peeter Joot}
\newcommand{\onlineurl}{http://sites.google.com/site/peeterjoot2/math2013/\basename.pdf}
\newcommand{\sourcepath}{\dirname\basename.tex}
\newcommand{\generatetitle}[1]{\chapter{#1}}

\newcommand{\vcsinfo}{%
\section*{}
\noindent{\color{DarkOliveGreen}{\rule{\linewidth}{0.1mm}}}
\paragraph{Document version}
%\paragraph{\color{Maroon}{Document version}}
{
\small
\begin{itemize}
\item Available online at:\\ 
\href{\onlineurl}{\onlineurl}
\item Git Repository: \input{./.revinfo/gitRepo.tex}
\item Source: \sourcepath
\item last commit: \input{./.revinfo/gitCommitString.tex}
\item commit date: \input{./.revinfo/gitCommitDate.tex}
\end{itemize}
}
}

%\PassOptionsToPackage{dvipsnames,svgnames}{xcolor}
\PassOptionsToPackage{square,numbers}{natbib}
\documentclass{scrreprt}

\usepackage[left=2cm,right=2cm]{geometry}
\usepackage[svgnames]{xcolor}
\usepackage{peeters_layout}

\usepackage{natbib}

\usepackage[
colorlinks=true,
bookmarks=false,
pdfauthor={\authorname, \email},
backref 
]{hyperref}

% http://tex.stackexchange.com/questions/75773/how-to-reference-problems-by-the-text-label-in-an-exercise-envioronment
\usepackage[english]{cleveref}
\crefname{Exercise}{exercise}{exercises}
\Crefname{Exercise}{Exercise}{Exercises}

\RequirePackage{titlesec}
\RequirePackage{ifthen}

% http://stackoverflow.com/questions/4932910/date-in-the-tabular-environment
\makeatletter
\let\insertdate\@date
\makeatother

\titleformat{\chapter}[display]
{\bfseries\Large}
{\color{DarkSlateGrey}\filleft \authorname
\ifthenelse{\isundefined{\studentnumber}}{}{\\ \studentnumber}
\ifthenelse{\isundefined{\email}}{}{\\ \email}
\ifthenelse{\isundefined{\dateintitle}}{}{\\ \insertdate}
%\ifthenelse{\isundefined{\coursename}}{}{\\ \coursename} % put in title instead.
}
{4ex}
{\color{DarkOliveGreen}{\titlerule}\color{Maroon}
\vspace{2ex}%
\filright}
[\vspace{2ex}%
\color{DarkOliveGreen}\titlerule
]

\newcommand{\beginArtWithToc}[0]{\begin{document}\tableofcontents}
\newcommand{\beginArtNoToc}[0]{\begin{document}}
\newcommand{\EndNoBibArticle}[0]{\end{document}}
\newcommand{\EndArticle}[0]{\bibliography{Bibliography}\bibliographystyle{plainnat}\end{document}}

% 
%\newcommand{\citep}[1]{\cite{#1}}

\colorSectionsForArticle


%
%\usepackage{peeters_layout_exercise}
%\usepackage{peeters_braket}
%\usepackage{peeters_figures}
%\usepackage{macros_qed}
%
%\beginArtNoToc
%
%\generatetitle{A curious proof of the Baker-Campbell-Hausdorff formula}
%\label{chap:bakercambell}

Equation (39) of \citep{glauber1951some} states the Baker-Campbell-Hausdorff formula for two operators \( a, b\) that commute with their commutator \( \antisymmetric{a}{b} \)

\begin{dmath}\label{eqn:bakercambell:20}
e^a e^b = e^{a + b + \antisymmetric{a}{b}/2},
\end{dmath}

and provides the outline of an interesting method of proof.  That method is to consider the derivative of

\begin{dmath}\label{eqn:bakercambell:40}
f(\lambda) = e^{\lambda a} e^{\lambda b} e^{-\lambda (a + b)},
\end{dmath}

That derivative is
\begin{dmath}\label{eqn:bakercambell:60}
\frac{df}{d\lambda}
=
e^{\lambda a} a e^{\lambda b} e^{-\lambda (a + b)}
+
e^{\lambda a} b e^{\lambda b} e^{-\lambda (a + b)}
-
e^{\lambda a} b e^{\lambda b} (a + b)e^{-\lambda (a + b)}
=
e^{\lambda a} \lr{
a e^{\lambda b}
+
b e^{\lambda b}
-
e^{\lambda b} (a+b)
}
e^{-\lambda (a + b)}
=
e^{\lambda a} \lr{
\antisymmetric{a}{e^{\lambda b}}
+
\cancel{\antisymmetric{b}{e^{\lambda b}}}
}
e^{-\lambda (a + b)}
=
e^{\lambda a}
\antisymmetric{a}{e^{\lambda b}}
e^{-\lambda (a + b)}
.
\end{dmath}

The commutator above is proportional to \( \antisymmetric{a}{b} \)

\begin{dmath}\label{eqn:bakercambell:80}
\antisymmetric{a}{e^{\lambda b}}
=
\sum_{k=0}^\infty \frac{\lambda^k}{k!} \antisymmetric{a}{ b^k }
=
\sum_{k=0}^\infty \frac{\lambda^k}{k!} k b^{k-1} \antisymmetric{a}{b}
=
\lambda \sum_{k=1}^\infty \frac{\lambda^{k-1}}{(k-1)!} b^{k-1} \antisymmetric{a}{b}
=
\lambda e^{\lambda b} \antisymmetric{a}{b},
\end{dmath}

so

\begin{dmath}\label{eqn:bakercambell:100}
\frac{df}{d\lambda} = \lambda \antisymmetric{a}{b} f.
\end{dmath}

To get the above, we should also do the induction demonstration for \( \antisymmetric{a}{ b^k } = k b^{k-1} \antisymmetric{a}{b} \).

This clearly holds for \( k = 0,1 \).  For any other \( k \) we have

\begin{dmath}\label{eqn:bakercambell:120}
\antisymmetric{a}{b^{k+1}}
=
a b^{k+1} - b^{k+1} a
=
\lr{ \antisymmetric{a}{b^{k}} + b^k a
} b - b^{k+1} a
=
k b^{k-1} \antisymmetric{a}{b} b
+ b^k \lr{ \antisymmetric{a}{b} + \cancel{b a} }
- \cancel{b^{k+1} a}
=
k b^{k} \antisymmetric{a}{b}
+ b^k \antisymmetric{a}{b}
=
(k+1) b^k \antisymmetric{a}{b} \qedmarker
\end{dmath}

Observe that \cref{eqn:bakercambell:100} is solved by

\begin{dmath}\label{eqn:bakercambell:140}
f = e^{\lambda^2\antisymmetric{a}{b}/2},
\end{dmath}

which gives

\begin{dmath}\label{eqn:bakercambell:160}
e^{\lambda^2 \antisymmetric{a}{b}/2} =
e^{\lambda a} e^{\lambda b} e^{-\lambda (a + b)}.
\end{dmath}

Right multiplication by \( e^{\lambda (a + b)} \) which commutes with \( e^{\lambda^2 \antisymmetric{a}{b}/2} \) and setting \( \lambda = 1 \) recovers \cref{eqn:bakercambell:20} as desired.

What I wonder looking at this, is what thought process led to trying this in the first place?  This is not what I would consider an obvious approach to demonstrating this identity.

%\EndArticle
